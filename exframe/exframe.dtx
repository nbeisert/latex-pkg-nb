% \iffalse
%
% exframe.dtx Copyright (C) 2011-2020 Niklas Beisert
%
% This work may be distributed and/or modified under the
% conditions of the LaTeX Project Public License, either version 1.3
% of this license or (at your option) any later version.
% The latest version of this license is in
%   http://www.latex-project.org/lppl.txt
% and version 1.3 or later is part of all distributions of LaTeX
% version 2005/12/01 or later.
%
% This work has the LPPL maintenance status `maintained'.
%
% The Current Maintainer of this work is Niklas Beisert.
%
% This work consists of the files exframe.dtx and exframe.ins
% and the derived files exframe.sty and exfsamp.tex, exfserm.tex,
% exfser01.tex, exfser02.tex, exfser03.tex, exfseraa.tex,
% exfserpe.tex, exfserpf.tex, exfsermk.sh, exfsermk.mak.
%
%<package|samplesingle|samplemultimain>\NeedsTeXFormat{LaTeX2e}[1996/12/01]
%<package>\ProvidesPackage{exframe}[2020/01/11 v3.31 Framework for Exercise Problems]
%<samplesingle>\ProvidesFile{exfsamp.tex}[2020/01/11 v3.31 standalone sample for exframe]
%<samplemultimain>\ProvidesFile{exfserm.tex}[2020/01/11 v3.31 multipart sample for exframe]
%<*driver>
\def\thedate#1{2020/01/11}\def\theversion#1{v3.31}
\ProvidesFile{exframe.dtx}[\thedate{} \theversion{} exframe reference manual file]
\PassOptionsToClass{10pt,a4paper}{article}
\documentclass{ltxdoc}

\usepackage[margin=35mm]{geometry}
\usepackage{hyperref}
\usepackage{hyperxmp}
\usepackage[usenames]{color}

\hypersetup{colorlinks=true}
\hypersetup{pdfstartview=FitH}
\hypersetup{pdfpagemode=UseNone}
\hypersetup{keeppdfinfo=true}
\hypersetup{pdfsource={}}
\hypersetup{pdflang={en-UK}}
\hypersetup{pdfcopyright={Copyright 2010-2020 Niklas Beisert.
  This work may be distributed and/or modified under the
  conditions of the LaTeX Project Public License, either version 1.3
  of this license or (at your option) any later version.}}
\hypersetup{pdflicenseurl={http://www.latex-project.org/lppl.txt}}
\hypersetup{pdfcontactaddress={ETH Zurich, ITP, HIT K,
  Wolfgang-Pauli-Strasse 27}}
\hypersetup{pdfcontactpostcode={8093}}
\hypersetup{pdfcontactcity={Zurich}}
\hypersetup{pdfcontactcountry={Switzerland}}
\hypersetup{pdfcontactemail={nbeisert@itp.phys.ethz.ch}}
\hypersetup{pdfcontacturl={http://people.phys.ethz.ch/\xmptilde nbeisert/}}

\newcommand{\secref}[1]{\hyperref[#1]{section \ref*{#1}}}

\parskip1ex
\parindent0pt
\let\olditemize\itemize
\def\itemize{\olditemize\parskip0pt}

\begin{document}

\title{The \textsf{exframe} Package}
\hypersetup{pdftitle={The exframe Package}}
\author{Niklas Beisert\\[2ex]
  Institut f\"ur Theoretische Physik\\
  Eidgen\"ossische Technische Hochschule Z\"urich\\
  Wolfgang-Pauli-Strasse 27, 8093 Z\"urich, Switzerland\\[1ex]
  \href{mailto:nbeisert@itp.phys.ethz.ch}
  {\texttt{nbeisert@itp.phys.ethz.ch}}}
\hypersetup{pdfauthor={Niklas Beisert}}
\hypersetup{pdfsubject={Manual for the LaTeX2e Package exframe}}
\date{\thedate{}, \theversion{}}
\maketitle

\begin{abstract}\noindent
\textsf{exframe} is a \LaTeXe{} package
which provides a general purpose framework to describe and typeset
exercises and exam questions along with their solutions.
The package features mechanisms
to hide or postpone solutions,
to assign and handle points,
to collect problems on exercise sheets,
to store and use metadata
and to implement a consistent numbering.
It also provides a very flexible interface for configuring and customising
the formatting, layout and representation of the exercise content.
\end{abstract}

\begingroup
\parskip0ex
\tableofcontents
\endgroup

%%%%%%%%%%%%%%%%%%%%%%%%%%%%%%%%%%%%%%%%%%%%%%%%%%%%%%%%%%%%%%%%%%%%%%%%%%%%%%%%
%%%%%%%%%%%%%%%%%%%%%%%%%%%%%%%%%%%%%%%%%%%%%%%%%%%%%%%%%%%%%%%%%%%%%%%%%%%%%%%%
\section{Introduction}

This package provides a framework to describe and typeset
exercises (homework problems, classroom exercises, quizzes, exam questions,
exercise questions in books and lecture notes, \ldots)
and their solutions or answers.
The aim of this package is to set up a few \LaTeX{} environments
into which questions and corresponding answers can be filled conveniently.
The main task of the package is to manage the text and data
that are provided in the source document,
perform some common operations on them,
and then output the content appropriately.
The package has the following goals, tasks and features:
%
\begin{itemize}
\item
The package is designed with generality in mind.
It is meant to be usable in many different situations.
The primary target is science and education, but it may well
be useful in other areas.

\item
The package defines a basic functional layout for the output
and provides many options to reshape the layout and formatting
according to the author's needs and wishes.

\item
The package can handle two layers of exercises:
main problems and subproblems.
The use of subproblems is optional.

\item
The display of solutions can be configured:
Solutions can be hidden for a hand-out version of exercise sheets.
When displayed, they may appear immediately,
collectively after the problem,
at the end of each sheet or at some manually defined location.

\item
The package can handle exercise sheets which combine several exercise problems:
A \LaTeX{} document can consist of an individual sheet
or of a collection of sheets (e.g.\ spanning a lecture course).
In the latter case, the document files can be set up
such that single sheets as well as a collection of all sheets
can be compiled; the package \textsf{childdoc} may be of assistance.

\item
The package can handle points to be credited:
Points will be displayed according to the layout.
Overall points for a problem or a sheet can be added automatically.
Points can also be stored and used elsewhere.

\item
The package provides an interface to specify exercise metadata
(author, source, \ldots):
Some basic types of metadata are predefined and more specific metadata
categories can be added.

\item
The package can use alternative counters for equations
within solutions (and problems). This is to ensure a consistent
numbering independently of whether solutions are output or not.

\end{itemize}

%%%%%%%%%%%%%%%%%%%%%%%%%%%%%%%%%%%%%%%%%%%%%%%%%%%%%%%%%%%%%%%%%%%%%%%%%%%%%%%%
%%%%%%%%%%%%%%%%%%%%%%%%%%%%%%%%%%%%%%%%%%%%%%%%%%%%%%%%%%%%%%%%%%%%%%%%%%%%%%%%
\section{Usage}
\label{sec:usage}

To use the package \textsf{exframe} add the command
%
\begin{center}
|\usepackage{exframe}|
\end{center}
%
to the preamble of the \LaTeX{} document.

%%%%%%%%%%%%%%%%%%%%%%%%%%%%%%%%%%%%%%%%%%%%%%%%%%%%%%%%%%%%%%%%%%%%%%%%%%%%%%%%
\subsection{Exercise Environments}
\label{sec:environments}

The package provides four environments
to describe the main entities of exercise problems.
Additional information on the exercises can be provided
in the optional arguments to these environments
which will be discussed in the following sections.
Furthermore, a limited set of commands
is provided for control and extra features,
see the sections below for details.

\DescribeEnv{problem}
The |problem| environment describes
an exercise problem:
%
\begin{center}
\begin{tabular}{l}
|\begin{problem}[|\textit{opts}|]|\\
|  |\textit{problem text and subproblems}\\
|\end{problem}|
\end{tabular}
\end{center}
%
As one of the many available options \textit{opts},
one can provide a title for the exercise by
specifying |title={|\textit{title}|}|.
If no title is given, the problem number will be displayed instead.
See \secref{sec:metadata} and \secref{sec:points}
for a description of the available options.

\DescribeEnv{subproblem}
The |subproblem| environment describes
a subproblem, part or an individual question of an exercise problem:
%
\begin{center}
\begin{tabular}{l}
|\begin{subproblem}[|\textit{opts}|]|\\
|  |\textit{subproblem text}\\
|\end{subproblem}|
\end{tabular}
\end{center}
%
A |subproblem| environment must be contained within a |problem| environment
(however, a |problem| block need not contain |subproblem| blocks).

\DescribeEnv{solution}
The |solution| environment describes
the solution to a problem or a subproblem:
\begin{center}
\begin{tabular}{l}
|\begin{solution}[|\textit{opts}|]|\\
|  |\textit{solution text}\\
|\end{solution}|
\end{tabular}
\end{center}
%
A |solution| environment should be
at the end of a |subproblem| or |problem| environment
(it is not mandatory to provide a |solution|).
It can be contained within the corresponding environment or it can follow it.
Depending on the choice of solution display, see \secref{sec:solutions},
the output may have a slightly different layout.
In terms of logic, it is preferred
to define a solution \emph{within} the corresponding environment;
this may also have some technical advantages and produce
a slightly better result in terms of layout.

\DescribeEnv{sheet}
The |sheet| environment describes an exercise sheet:
%
\begin{center}
\begin{tabular}{l}
|\begin{sheet}[|\textit{opts}|]|\\
|  |\textit{sheet text and problems}\\
|\end{sheet}|
\end{tabular}
\end{center}
%
A sheet typically contains one or several problems
(it is not mandatory to group problems into a |sheet|).
There may or may not be additional auxiliary text introducing the problems.
A header will be added to the sheet according to the specified layout.

%%%%%%%%%%%%%%%%%%%%%%%%%%%%%%%%%%%%%%%%%%%%%%%%%%%%%%%%%%%%%%%%%%%%%%%%%%%%%%%%
\subsection{Solution and Problem Display}
\label{sec:solutions}

There are several options to control the output of solutions
and of problems.

%%%%%%%%%%%%%%%%%%%%%%%%%%%%%%%%%%%%%%%%
\DescribeMacro{solutions}
Most importantly, the display of solutions
can be disabled or enabled altogether:
%
\begin{center}
|\exercisesetup{solutions|[|=true|\textbar|false|]|}|
\end{center}
%
Solutions are hidden by default,
and their display needs to be activated explicitly
(it suffices to specify the option |solutions| without the value |true|).
It is also possible to control the display
by an analogous package option |solutions|,
see \secref{sec:options} for further information.

%%%%%%%%%%%%%%%%%%%%%%%%%%%%%%%%%%%%%%%%
\DescribeMacro{\ifsolutions}
\DescribeMacro{onlysolutions}
The display of solutions is reflected by the conditional |\ifsolutions|.
As the hiding of solutions is performed automatically,
the conditional would typically be used to change some details,
e.g.\ for adjusting titles:
%
\begin{center}
|\ifsolutions Solutions\||else Exercises\||fi|
\end{center}
%
Alternatively, content to be processed only in solutions mode
can be enclosed in an |onlysolutions| block:
%
\begin{center}
\begin{tabular}{l}
|\begin{onlysolutions}|\\
|...|\\
|\end{onlysolutions}|
\end{tabular}
\end{center}
%
This structure can be useful to hide auxiliary text or material
if all solution content is to be stripped from a source file,
e.g.\ by an automated \textsf{sed} filter rule
(doubling of backslashes required for \textsf{sed}
as well as for shell script strings):
%
\begin{center}
\begin{tabular}{l}
|sed "/\\\\begin{solution}/,/\\\\end{solution}/d;"\|\\
|"/\\\\begin{onlysolutions}/,/\\\\end{onlysolutions}/d"|
\end{tabular}
\end{center}
%

%%%%%%%%%%%%%%%%%%%%%%%%%%%%%%%%%%%%%%%%
\DescribeMacro{solutionequation}
As solutions can contain numbered equations
while the display of solutions can be switched on and off,
it is important to assign a different counter for equations within solutions
in order for the equation numbers to be stable.
A separate counter for equations within solutions is enabled by default.
It can be disabled by:
%
\begin{center}
|\exercisestyle{solutionequation=false}|
\end{center}
%
This option prepends the letter `S' to equation
numbers within solutions which are counted separately;
the display can be configured differently, see \secref{sec:layout}.

%%%%%%%%%%%%%%%%%%%%%%%%%%%%%%%%%%%%%%%%
\DescribeMacro{solutionbelow}
\DescribeMacro{\insertsolutions}
The package allows to collect solutions
and defer their display to particular locations:
%
\begin{center}
|\exercisestyle{solutionbelow=|\textit{pos}|}|
\end{center}
%
The available choices for \textit{pos} are
to display solutions where they are defined (|here|),
defer them to the end of the current subproblem (|subproblem|),
problem (|problem|) or sheet (|sheet|)
or display them at a manually chosen location (|manual|).
Note that typically solutions are defined at the end of a (sub)problem
and therefore the choice |here| is similar to (|sub|)|problem|.
The latter form, however, makes sure that a solution
does not inherit the margin of the parent environment.
The alternate modes |problem*| and |subproblem*|
positions the solution \emph{after} the (sub)problem environment
such that it does not inherit any layout,
but also no definitions made in the parent environment.
In |manual| mode, all solutions are collected (with appropriate headers)
until they are output by the directive |\insertsolutions|.
If no solutions are stored in the buffer
(or if the mode is not |manual|), |\insertsolutions| has no effect.

%%%%%%%%%%%%%%%%%%%%%%%%%%%%%%%%%%%%%%%%
\DescribeMacro{\writesolutions}
Another option to handle solutions is to write them to a file
for later use.
Writing to a file is initiated by:
%
\begin{center}
|\writesolutions[|\textit{filename}|]|
\end{center}
%
The optional argument describes the filename as \textit{filename}|.sol|;
no argument defaults to the main tex filename as |\jobname.sol|;
the extension |.sol| can be customised by the configuration |extsolutions|.
This mode overrides the |solutionbelow| behaviour described above;
all subsequent solutions are written to the file.
The file is closed by |\closesolutions| and the display of solutions
returns to manual mode.
It is not necessary to close a file as it will be closed
automatically by reading from a file, writing to another file
or by the end of the document.

%%%%%%%%%%%%%%%%%%%%%%%%%%%%%%%%%%%%%%%%
\DescribeMacro{\readsolutions}
Solutions are read from a file by:
%
\begin{center}
|\readsolutions[|\textit{filename}|]|
\end{center}
%
This command outputs a sectional title
and reads the file via |\input{|\textit{filename}|.sol}|.

\medskip

%%%%%%%%%%%%%%%%%%%%%%%%%%%%%%%%%%%%%%%%
\DescribeMacro{solutionbuf}
\DescribeMacro{problembuf}
The package offers similar functionality to control the display of problems.
In order to have any control over the content of |problem| environments,
the latter needs to be read into an internal buffer.
Reading of solutions and problems to internal buffers is activated
or deactivated by:
%
\begin{center}
\begin{tabular}{l}
|\exercisesetup{solutionbuf|[|=true|\textbar|false|]|}|\\
|\exercisesetup{problembuf|[|=true|\textbar|false|]|}|
\end{tabular}
\end{center}
%
By default, |solution| environments are read to an internal buffer,
while the content of |problem| environments is processed directly
by the \TeX\ engine.
Therefore, the following options to control the display
of |problem| environments require the statement |\exercisesetup{problembuf}|.

%%%%%%%%%%%%%%%%%%%%%%%%%%%%%%%%%%%%%%%%
\DescribeMacro{problemmanual}
\DescribeMacro{\insertproblems}
The immediate display of |problem| environments is controlled by:
%
\begin{center}
|\exercisestyle{problemmanual|[|=true|\textbar|false|]|}|
\end{center}
%
In the default automatic mode, problems are displayed directly
where they are declared.
In manual mode, problems are collected to an internal buffer,
and only displayed by issuing |\insertproblems|.

Note that |solution| environments should be
declared within the corresponding |problem| environment in order to
preserve their appropriate association.
The |solution| environment is then processed at the place
where the |problem| environment is displayed,
and it may (or may not) be deferred further.

%%%%%%%%%%%%%%%%%%%%%%%%%%%%%%%%%%%%%%%%
\DescribeMacro{\writeproblems}
\DescribeMacro{\readproblems}
Problems can be written out to an external file for later usage.
The functionality is analogous to solutions
and uses the macros:
%
\begin{center}
\begin{tabular}{l}
|\writeproblems[|\textit{filename}|]|\\
|\readproblems[|\textit{filename}|]|
\end{tabular}
\end{center}
%
The optional argument describes the filename as \textit{filename}|.prb|;
no argument defaults to the main tex filename as |\jobname.prb|;
the extension |.prb| can be customised by the configuration |extproblems|.

%%%%%%%%%%%%%%%%%%%%%%%%%%%%%%%%%%%%%%%%
\DescribeMacro{disable}
\DescribeMacro{insertproblemselect}
The display of a particular |problem|
can be suppressed altogether by an optional argument:
%
\begin{center}
|\begin{problem}[disable]|
\end{center}
%

This option can be exploited to automatically
suppress certain classes of problems as follows:
A hook function |insertproblemselect| declared by:
%
\begin{center}
|\exerciseconfig{insertproblemselect}[1]{|\textit{code}|}|
\end{center}
%
can call
|\setproblemdata{disable}| whenever a problem is to be suppressed.
In order to decide, the optional argument of the |problem| environment
is passed on to the hook function as the single argument.
Note that the argument needs to be processed manually.

%%%%%%%%%%%%%%%%%%%%%%%%%%%%%%%%%%%%%%%%%%%%%%%%%%%%%%%%%%%%%%%%%%%%%%%%%%%%%%%%
\subsection{Metadata}
\label{sec:metadata}

In a collection of exercise problems
it makes sense to keep track of metadata
for the overall collection as well as for individual problems
and potentially display some of them.
The framework defines a standard set of metadata fields
and offers functionality to add more specialised metadata fields.

%%%%%%%%%%%%%%%%%%%%%%%%%%%%%%%%%%%%%%%%
\DescribeMacro{\exercisedata}
Global metadata is specified by the command:
%
\begin{center}
|\exercisedata{|\textit{data}|}|
\end{center}
%
The argument \textit{data}
is a comma-separated list of metadata specifications
in the form \textit{key}|={|\textit{value}|}|.
The standard set of global metadata keys consists of:
%
\begin{itemize}
\item |author|:
principal author(s) of the exercise collection;
also invokes the \LaTeX{} command |\author|;
will be written to pdf documents.
\item |title|:
title of the exercise collection;
also invokes the \LaTeX{} command |\title|;
will be written to pdf documents.
\item |date|:
date of the exercise collection;
also invokes the \LaTeX{} command |\date|;
will be written to pdf documents.
\item |subject|:
subject area of the exercise collection;
will be written to pdf documents.
\item |keyword|:
keyword(s) for the exercise collection;
will be written to pdf documents.
\item |course|:
title of the course (class, lecture, module, \ldots)
for the exercise collection.
\item |institution|:
institution (school, department, institute, university, \ldots)
offering the course or exercise collection.
\item |instructor|:
instructor(s) for the course or exercise;
this field refers to person(s) who organise
the corresponding course or exercises
whereas |author| refers to the principal creator of the material.
\item |period|:
period (year, season, date, term identifier, \ldots)
of the corresponding course.
\item |material|:
type of material
(exercises, homework assignments, exam, quizzes, solutions, \ldots).
\end{itemize}
%
%%%%%%%%%%%%%%%%%%%%%%%%%%%%%%%%%%%%%%%%
\DescribeMacro{\defexercisedata}
Additional custom fields for global metadata can be created with:
%
\begin{center}
|\defexercisedata{|\textit{key}|}|
\end{center}
%

%%%%%%%%%%%%%%%%%%%%%%%%%%%%%%%%%%%%%%%%
\DescribeMacro{\getexercisedata}
\DescribeMacro{\exercisedataempty}
Global metadata should typically be specified
somewhere at the top of the main document,
and it can be inserted wherever needed.
There are two commands to read and process metadata.
To insert the value of metadata field \textit{key} use:
%
\begin{center}
|\getexercisedata{|\textit{key}|}|
\end{center}
%
In some situations the output should depend on whether
a metadata has been filled
(e.g.\ to fill a default value or to display something else instead).
This can be checked with the conditional:
%
\begin{center}
|\exercisedataempty{|\textit{key}|}|%
  |{|\textit{empty code}|}{|\textit{filled code}|}|
\end{center}
%
The \textit{empty code} is executed if no value
or an empty value has been specified;
otherwise the \textit{filled code} is executed.

%%%%%%%%%%%%%%%%%%%%%%%%%%%%%%%%%%%%%%%%
\DescribeMacro{sheet}
\DescribeMacro{problem}
The package offers a similar mechanism to describe and use metadata
for sheets and problems:
%
\begin{center}
\begin{tabular}{l}
|\begin{sheet}[|\textit{opts}|]|\\
|\begin{problem}[|\textit{opts}|]|
\end{tabular}
\end{center}
%
The argument \textit{opt}
is a comma-separated list which can contain metadata specifications
in the form \textit{key}|={|\textit{value}|}|.
The standard set of metadata keys for sheets consists of:
%
\begin{itemize}
\item |due|:
indication of the due date for the exercise sheet.
\item |handout|:
indication of the handout date for the exercise sheet.
\item |title|:
specifies a title for the sheet;
when reading value (see below), returns composed title;
untitled sheets will be displayed by their number;
title will be written to pdf documents.
\item |rawtitle| (for reading only):
contains the raw title as specified by |title|.
\item |author|:
author(s) of the sheet;
will be written to pdf documents.
\item |editor|:
editor(s) of the sheet; this field refers to a person
who makes adjustments to the sheet
whereas |author| refers to the creator of the sheet.
\item |editdate|:
indication of the date when the sheet was last edited.
\end{itemize}
%
The standard set of metadata keys for problems consists of:
%
\begin{itemize}
\item |title|:
specifies a title for the problem;
when reading value (see below), returns composed title;
untitled problems will be displayed by their number.
\item |rawtitle| (for reading only):
contains the raw title as specified by |title|.
\end{itemize}
%

%%%%%%%%%%%%%%%%%%%%%%%%%%%%%%%%%%%%%%%%
\DescribeMacro{\defsheetdata}
\DescribeMacro{\setsheetdata}
\DescribeMacro{\getsheetdata}
\DescribeMacro{\sheetdataempty}
\DescribeMacro{\defproblemdata}
\DescribeMacro{\setproblemdata}
\DescribeMacro{\getproblemdata}
\DescribeMacro{\problemdataempty}
Metadata for sheets can be used in the same way as the global metadata.
The following directives are analogous to |\defexercisedata|,
|\exercisedata|, |\getexercisedata| and |\exercisedataempty|:
%
\begin{center}
\begin{tabular}{l}
|\defsheetdata{|\textit{key}|}|\\
|\setsheetdata{|\textit{data}|}|\\
|\getsheetdata{|\textit{key}|}|\\
|\sheetdataempty{|\textit{key}|}|%
  |{|\textit{empty code}|}{|\textit{filled code}|}|\\[1ex]
|\defproblemdata{|\textit{key}|}|\\
|\setproblemdata{|\textit{data}|}|\\
|\getproblemdata{|\textit{key}|}|\\
|\problemdataempty{|\textit{key}|}|%
  |{|\textit{empty code}|}{|\textit{filled code}|}|
\end{tabular}
\end{center}
%

%%%%%%%%%%%%%%%%%%%%%%%%%%%%%%%%%%%%%%%%
\DescribeMacro{pdfdata}
\DescribeMacro{\writeexercisedata}
The most relevant metadata can be written to
the metadata section of pdf files
(using pdf\LaTeX{} and the package \textsf{hyperref}
whenever loaded).
This feature is configured by:
%
\begin{center}
|\exercisesetup{pdfdata|[|=auto|\textbar|manual|%
  \textbar|sheet|\textbar|off|]|}|
\end{center}
%
The option |auto| writes the global metadata
|title|, |author|, |subject| and |keyword|
to the corresponding fields in the pdf file.
To make this work, these must be defined before the
|\begin{document}| directive.
The option |manual| allows to manually write these metadata
by the command |\writeexercisedata|.
It should be issued after the metadata have been set,
but before any content is written to the pdf file.
In other words, it can be anywhere in the document preamble
directly after |\begin{document}|,
or following a couple of content-free definitions at the beginning
of the document body (in case the metadata should be set
within the document body for some reason).
The option |sheet| writes out the metadata
at the beginning of the first |sheet| environment
(which should follow |\begin{document}| without any content in between).
This option is primarily for filling the
|author| and |title| fields with metadata
of a sheet rather than a collection of exercises.
Note that if no |author| is defined for the sheet,
the global metadata |author| is used.
The option |off| disables all writing of metadata.

%%%%%%%%%%%%%%%%%%%%%%%%%%%%%%%%%%%%%%%%
\DescribeMacro{problem}
\DescribeMacro{subproblem}
\DescribeMacro{solution}
There is an additional mechanism to keep track of metadata
for problems, subproblems and solutions
which can be displayed in the opening line of these entities.
Displayed metadata serve two purposes:
they are used to describe the quality of a problem
or they are intended for internal documentation purposes.
Their output can be controlled individually,
e.g.\ only in development versions of a document.
Note that specifying a key more than once
will display the content multiple times
in the order in which they are encountered.
Displayed metadata are specified
at the top of the corresponding environment:
%
\begin{center}
\begin{tabular}{l}
|\begin{problem}[|\textit{opts}|]|\\
|\begin{subproblem}[|\textit{opts}|]|\\
|\begin{solution}[|\textit{opts}|]|
\end{tabular}
\end{center}
%

The standard set of displayed metadata keys consists of:
%
\begin{itemize}
\item |author|:
author(s) of the problem (or subproblem, solution).
\item |editor|:
editor(s) of the problem; this field refers to a person
who has made adjustments to the problem
whereas |author| refers to the creator of the problem.
\item |source|:
source of the problem;
in case the problem has been taken from elsewhere
(conceptually or literally).
\item |difficulty|:
indication of the level of difficulty of the problem.
\item |keyword|:
keyword(s) for the problem;
\item |comment|:
some comment on the problem.
\item |optional|
(display enabled by default):
whether addressing the problem is mandatory or optional;
by default the text will be displayed after the title
in italic shape.
\end{itemize}
%
By default, only the |optional| items are displayed,
all other types of items are hidden;
controlling the display for each type of item is described below.

%%%%%%%%%%%%%%%%%%%%%%%%%%%%%%%%%%%%%%%%
\DescribeMacro{extdata}
Further displayed metadata keys
are defined by the package option |extdata|, see \secref{sec:options}:
%
\begin{itemize}
\item |review|:
field to review the aspects of the problem
(quality, length, appropriateness, difficulty, \ldots).
\item |recycle|:
indication of previous instances where this problem was used.
\item |timesolve|:
indication of the time needed to solve this problem (or subproblem).
\item |timepresent|:
indication of the time needed to present this problem
(or subproblem, solution).
\end{itemize}
%

%%%%%%%%%%%%%%%%%%%%%%%%%%%%%%%%%%%%%%%%
\DescribeMacro{\showprobleminfo}
The display of the above metadata fields for a problem
(or subproblem, solution) is controlled by:
%
\begin{center}
|\showprobleminfo{|\textit{keys}|}|
\end{center}
%
Here \textit{keys} is a comma-separated list of keys to
be activated (\textit{key} or \textit{key}|=true|)
or deactivated (\textit{key}|=false|).

%%%%%%%%%%%%%%%%%%%%%%%%%%%%%%%%%%%%%%%%
\DescribeMacro{\defprobleminfo}
Displayable metadata can be defined or adjusted by:
%
\begin{center}
|\defprobleminfo{|\textit{key}|}{|\textit{code}|}|
\end{center}
%
Here \textit{key} specifies the metadata field and
\textit{code} the code to display this type of metadata
where the argument |#1| represents the data to be displayed.

%%%%%%%%%%%%%%%%%%%%%%%%%%%%%%%%%%%%%%%%
\DescribeMacro{insertprobleminfo}
\DescribeMacro{insertsubprobleminfo}
\DescribeMacro{insertsolutioninfo}
\DescribeMacro{\addprobleminfo}
\DescribeMacro{\addprobleminfo*}
Additional information can be injected into the opening line
of problems and solutions by the definitions:
%
\begin{center}
\begin{tabular}{l}
|\exerciseconfig{insertprobleminfo}{|\textit{code}|}|\\
|\exerciseconfig{insertsubprobleminfo}{|\textit{code}|}|\\
|\exerciseconfig{insertsolutioninfo}{|\textit{code}|}|
\end{tabular}
\end{center}
%
The hook code \textit{code} will be called after processing
the environment arguments. Information can be added to the opening line by:
%
\begin{center}
\begin{tabular}{l}
|\addprobleminfo{|\textit{info}|}|\\
|\addprobleminfo*{|\textit{info}|}|
\end{tabular}
\end{center}
%
The unstarred command adds information at the end of the opening line,
the starred version at the beginning (but after the title or identifier).

%%%%%%%%%%%%%%%%%%%%%%%%%%%%%%%%%%%%%%%%%%%%%%%%%%%%%%%%%%%%%%%%%%%%%%%%%%%%%%%%
\subsection{Points}
\label{sec:points}

%%%%%%%%%%%%%%%%%%%%%%%%%%%%%%%%%%%%%%%%
\DescribeMacro{points}
Exercise problems or certain parts of them
can be credited with points (credits, awards, \ldots).
The package provides an interface to specify and manage such points.
Points are declared by the option |points=|\textit{points}
for the environments |sheet|, |problem| and |subproblem|.
These numbers will be printed to the opening line of
problems and subproblems.

Note that the points should normally be integer numbers.
Fractional points are permissible as well,
but the internal storage by the \TeX{} engine is somewhat limited,
so that only fractions with powers of two as denominators
(.5, multiples of .25, .125, .0625, \ldots) are reliable.
More general fractional decimal numbers such as multiples of 0.2
will be subject to rounding errors and will not display nicely.

Bonus points can be specified in the format
|points=|[\textit{regular}][|+|\textit{bonus}].
By default, such points will be printed as
[\textit{regular}][|+|\textit{bonus}] where 0 components are omitted.

%%%%%%%%%%%%%%%%%%%%%%%%%%%%%%%%%%%%%%%%
\DescribeMacro{problempointsat}
\DescribeMacro{subproblempointsat}
\DescribeMacro{solutionpointsat}
The location where points of problems and subproblems
shall be displayed can be adjusted individually by:
%
\begin{center}
\begin{tabular}{l}
|\exercisestyle{problempointsat=start|\textbar|start*|\textbar|margin|%
  \textbar|end|\textbar|manual|\textbar|off}|\\
|\exercisestyle{subproblempointsat=start|\textbar|start*|\textbar|margin|%
  \textbar|end|\textbar|manual|\textbar|off}|\\
|\exercisestyle{solutionpointsat=start|\textbar|start*|\textbar|margin|%
  \textbar|end|\textbar|manual|\textbar|off}|
\end{tabular}
\end{center}
%
The default values are |start| and |end| for problems and subproblems,
respectively.
The option |start| displays points at
the very end of the opening line;
the option |start*| displays them at the start of it.
The option |end| displays points at the end of the
problem or subproblem text.
The option |margin| displays points in the margin.
The option |manual| displays points at a manually chosen location
specified by the directive |\showpoints|.
Note that |\showpoints| can also be used for the option |end|
to display the points prematurely
(e.g.\ if the text ends with a displayed equation,
it may make sense to display the points just before the equation).
The option |off| disables the display of points.

%%%%%%%%%%%%%%%%%%%%%%%%%%%%%%%%%%%%%%%%
\DescribeMacro{\getsheetdata}
Points for sheets are only stored by the package;
they must displayed manually.
Within the corresponding |sheet| environment
the points can be accessed by:
%
\begin{center}
|\getsheetdata{points}|
\end{center}
%

%%%%%%%%%%%%%%%%%%%%%%%%%%%%%%%%%%%%%%%%
\DescribeMacro{\getsheetpoints}
\DescribeMacro{\getproblempoints}
\DescribeMacro{\getsubproblempoints}
\DescribeMacro{\getsolutionpoints}
\DescribeMacro{\extractpoints}
\DescribeMacro{\switchpoints}
The package allows to read the point totals
for other sheets and problems:
%
\begin{center}
\begin{tabular}{l}
|\getsheetpoints{|\textit{tag}|}|\\
|\getproblempoints{|\textit{tag}|}|\\
|\getsubproblempoints{}|\\
|\getsolutionpoints{}|
\end{tabular}
\end{center}
%
Here \textit{tag} is the tag assigned to the corresponding sheet or problem,
see \secref{sec:labels}.
An empty argument \textit{tag} refers to the current
sheet, (sub)problem or solution.
If bonus points are used, the points will be returned
in the format [\textit{regular}][|+|\textit{bonus}];
the components \textit{regular} and \textit{bonus}
can be extracted from the returned expression
by |\extractpoints| and |\extractpoints*|, respectively.
A convenient case switch of the returned value can be performed by:
%
\begin{center}
|\switchpoints{|\textit{reg}|}{|\textit{bonus}|}{|\textit{both}|}{|%
\textit{none}|}{|\textit{val}|}|
\end{center}
%
Here \textit{val} is the value returned from the points register,
\textit{reg} is displayed for purely regular points,
\textit{bonus} is displayed for purely bonus points,
\textit{both} is displayed for mixed points,
\textit{none} is displayed for no points.
In each of the four expressions,
|#1| will be replaced by the regular points
and |#2| by the bonus points.

%%%%%%%%%%%%%%%%%%%%%%%%%%%%%%%%%%%%%%%%
\DescribeMacro{\awardpoints}
Grading instructions with points to be awarded
can be specified in the solution text by:
%
\begin{center}
\begin{tabular}{l}
|\awardpoints[|\textit{details}|]{|\textit{points}|}|\\
|\awardpoints*[|\textit{details}|]{|\textit{points}|}|
\end{tabular}
\end{center}
%
Here \textit{details} is an optional text with further details,
e.g.\ to explain under which conditions
these points are to be awarded.
The starred form is used to specify optional points or alternative paths
with alternative grading instructions.
These points will be marked and not be used for the computation of a total.

%%%%%%%%%%%%%%%%%%%%%%%%%%%%%%%%%%%%%%%%
\DescribeMacro{warntext}
The package attempts to add up
the points of subproblems to the problem total
and likewise the points of problems to the sheet total.
The package also performs some sanity checks on the provided numbers:
If points are specified for both subproblems and problems
or for both problems and sheets, they will be compared.
Also the points within solutions (excluding optional or alternative points)
are added up and compared to the corresponding problem or subproblem.
Furthermore the package checks whether points
are defined for all subproblems within a problem
or all problems within a sheet.
Mismatches are reported as package warnings.
As point mismatches can be rather severe,
there is an option to write such warnings directly into
the output document (to be removed before distribution):
%
\begin{center}
|\exercisesetup{warntext|[|=true|\textbar|false|]|}|
\end{center}
%

%%%%%%%%%%%%%%%%%%%%%%%%%%%%%%%%%%%%%%%%
\DescribeMacro{fracpoints}
The package offers pretty display of fractional points
with denominators 2, 4 and 8 by writing the decimal part
as a fraction, e.g.\ 1.75 $\to$ 1$^3\mskip-4mu/\mskip-2mu_4$.
This feature is enabled by:
%
\begin{center}
|\exercisestyle{fracpoints}|
\end{center}
%

%%%%%%%%%%%%%%%%%%%%%%%%%%%%%%%%%%%%%%%%%%%%%%%%%%%%%%%%%%%%%%%%%%%%%%%%%%%%%%%%
\subsection{Labels and Tags}
\label{sec:labels}

%%%%%%%%%%%%%%%%%%%%%%%%%%%%%%%%%%%%%%%%
\DescribeMacro{label}
\LaTeX{} provides labels to make references to remote parts of the text.
Labels can be set as usual by |\label{|\textit{label}|}| within the
|problem|, |subproblems| and |sheet| environments.
Alternatively, they can be specified
as the environment option:
%
\begin{center}
|label={|\textit{label}|}|
\end{center}
%

%%%%%%%%%%%%%%%%%%%%%%%%%%%%%%%%%%%%%%%%
\DescribeMacro{tag}
\DescribeMacro{\sheettag}
\DescribeMacro{\problemtag}
The package provides an additional mechanism to tag sheets and problems.
Each |sheet| and each |problem| can be assigned a unique tag \textit{tag} by
the environment option:
%
\begin{center}
|tag={|\textit{tag}|}|
\end{center}
%
This tag is used for reading point totals
as described in \secref{sec:points}.
Furthermore, the macro |\sheettag| or |\problemtag|
is set to the tag \textit{tag} within the current environment.
If no tag is specified it matches the number of the sheet or problem;
note that this number can change by reordering sheets and problems
and therefore it should not be used to identify the entity from
other parts of the document.

A useful application for tags is to encapsulate labels
within individual sheets and problems
which are part of a collection of exercises.
Labels which are composed as
|\sheettag-|\textit{label} or
|\problemtag-|\textit{label}
can be considered local and will not clash with labels
defined within a different environment.
Within the same sheet or problem, local labels can be accessed
by the same construction.
They can also be accessed from remote parts of the document by fully
expanding |\sheettag| or |\problemtag| for the desired target environment.

%%%%%%%%%%%%%%%%%%%%%%%%%%%%%%%%%%%%%%%%
\DescribeMacro{autolabelsheet}
\DescribeMacro{autolabelproblem}
If unique tags are specified, the package can automatically create labels
for sheets (|sheet:|\textit{tag}) and problems (|prob:|\textit{tag}) by:
%
\begin{center}
\begin{tabular}{l}
|\exercisesetup{autolabelsheet|[|=true|\textbar|false|]|}|\\
|\exercisesetup{autolabelproblem|[|=true|\textbar|false|]|}|
\end{tabular}
\end{center}
%

%%%%%%%%%%%%%%%%%%%%%%%%%%%%%%%%%%%%%%%%%%%%%%%%%%%%%%%%%%%%%%%%%%%%%%%%%%%%%%%%
\subsection{Layout}
\label{sec:layout}

The package provides a large number of parameters to adjust the
display of exercises to a desired layout.

%%%%%%%%%%%%%%%%%%%%%%%%%%%%%%%%%%%%%%%%
\DescribeMacro{\exerciseconfig}
Configuration settings are declared and modified by the command:
%
\begin{center}
|\exerciseconfig{|\textit{key}|}[|\textit{narg}|]{|\textit{value}|}|
\end{center}
%
Here \textit{key} is a key and \textit{value} is its assigned value.
Configuration options can also be macros with arguments in which case
\textit{narg} is the number of arguments and \textit{value}
is the macro definition using arguments |#|\textit{n}.
The command |\exerciseconfig| therefore is analogous to
|\|(|re|)|newcommand| except that the definitions are
encapsulated by the package and any previous definition
is overwritten without checking.

\DescribeMacro{\exerciseconfigappend}
\DescribeMacro{\exerciseconfigprepend}
In some cases it may be useful to be able to
append or prepend to a (parameterless) definition by:
%
\begin{center}
\begin{tabular}{l}
|\exerciseconfigappend{|\textit{key}|}{|\textit{value}|}|\\
|\exerciseconfigprepend{|\textit{key}|}{|\textit{value}|}|
\end{tabular}
\end{center}

%%%%%%%%%%%%%%%%%%%%%%%%%%%%%%%%%%%%%%%%
\DescribeMacro{\getexerciseconfig}
\DescribeMacro{\exerciseconfigempty}
Configuration definitions can be read by:
%
\begin{center}
|\getexerciseconfig{|\textit{key}|}|[\textit{arguments}]
\end{center}
%
The number of arguments after |{|\textit{key}|}|
must match the optional argument \textit{nargs} of the definition.
Furthermore, it can be checked whether a configuration definition
is empty:
%
\begin{center}
|\exerciseconfigempty{|\textit{key}|}|%
  |{|\textit{empty code}|}{|\textit{filled code}|}|
\end{center}
%
The \textit{empty code} is executed if no value
or an empty value has been specified.
Otherwise the \textit{filled code} is executed.

The package defines numerous layout configuration options.
They are listed along with their original definition
and a brief description in \secref{sec:imp-config}.
They include options to:
%
\begin{itemize}
\item
adjust the language for the principal entities of this package
like `sheet(s)', `problem(s)', `solution(s)', `points(s)';
\item
adjust the fonts styles of various parts of the text;
\item
adjust the spacing above, below, between various elements;
\item
define code to process data and insert text at various locations;
\item
compose text to be used in various situations;
\item
adjust the appearance of counters;
\item
adjust some other behaviour of the package.
\end{itemize}
%
The following will highlight only few examples.

%%%%%%%%%%%%%%%%%%%%%%%%%%%%%%%%%%%%%%%%
\DescribeMacro{insertsheettitle}
An important setting is:
%
\begin{center}
|\exerciseconfig{insertsheettitle}{|\textit{code}|}|
\end{center}
%
The code \textit{code} is meant to print the title
or header of an exercise sheet.
The minimalistic default code |\centerline{\getsheetdata{title}}|
merely prints the sheet title ``Sheet \#'' at the centre of a line.
Commonly, one would replace this by a more elaborate header
(potentially with some more information,
appealing layout, logos, \ldots).
In order to design a header template,
it makes sense to retrieve data
via |\getexercisedata| and |\getsheetdata|
described in \secref{sec:metadata}.
Likewise |\exercisedataempty| and |\sheetdataempty|
can be used to display default values or alternative data
if some particular data is not provided.
An example is given by the |plainheader| extended style option
defined in \secref{sec:imp-styles}.

%%%%%%%%%%%%%%%%%%%%%%%%%%%%%%%%%%%%%%%%
\DescribeMacro{composetitleproblem}
Another noteworthy example is |composetitleproblem|
to compose the title for a problem. It takes two parameters,
the number and the title.
The (somewhat simplified) default declaration is:
%
\begin{center}
\begin{tabular}{l}
|\exerciseconfig{composetitleproblem}[2]{\exerciseifempty{#2}|\\
|  {\getexerciseconfig{termproblem}|\\
|   \getexerciseconfig{composeitemproblem}{#1}}|\\
|  {\getexerciseconfig{composeitemproblem}{#1} #2}}|
\end{tabular}
\end{center}
%
This checks whether the title is empty.
If no title is given use ``Problem \#.'',
otherwise use ``\#. \textit{title}''.
Here the term ``Problem'' is made abstract
by the configuration |termproblem|
(e.g.\ to support internationalisation)
and the problem number is further composed obtained by
the configuration |composeitemproblem|
which takes the bare number as argument
and returns it followed by a dot.

%%%%%%%%%%%%%%%%%%%%%%%%%%%%%%%%%%%%%%%%
\DescribeMacro{\exerciseifempty}
\DescribeMacro{\exerciseifnotempty}
Handy conditionals command to check
whether an expression \textit{expr} is empty are:
%
\begin{center}
\begin{tabular}{l}
|\exerciseifempty{|\textit{expr}|}|%
  |{|\textit{empty code}|}{|\textit{filled code}|}|\\
|\exerciseifnotempty{|\textit{expr}|}|%
  |{|\textit{filled code}|}|
\end{tabular}
\end{center}
%
Their main purpose is to test
whether some provided expression \textit{expt} is empty.
They expand to the common \TeX{} constructs
|\if&#1&#2\else#3\fi| and |\if&#1&\else#2\fi|
which work assuming that \textit{expr} is not too exotic
(e.g.\ it should not start with the character `|&|'
and other special \TeX{} characters or macros are potentially dangerous).

%%%%%%%%%%%%%%%%%%%%%%%%%%%%%%%%%%%%%%%%%%%%%%%%%%%%%%%%%%%%%%%%%%%%%%%%%%%%%%%%
\subsection{Exercise Styles}
\label{sec:styles}

The package provides a mechanism to define exercise styles
which customise the display of exercises in some coordinated fashion.

%%%%%%%%%%%%%%%%%%%%%%%%%%%%%%%%%%%%%%%%
\DescribeMacro{\exercisestyle}
Style(s) are activated by the command:
%
\begin{center}
|\exercisestyle{|\textit{styles}|}|
\end{center}
%
Here \textit{styles} is a comma-separated list of styles,
where each style is given by a pair \textit{style}[|={|\textit{argument}|}|].
The package defines a couple of standard styles:
%
\begin{itemize}
\item |solutionbelow=|\textit{pos}
(can take values |here|, |subproblem|, |subproblem*|,
|problem|, |problem*|, |sheet| and |manual|;
initially set to |subproblem|) --
positions the solutions below the indicated environments;
see \secref{sec:solutions} for details.

\item |problempointsat=|\textit{pos}
(can take values |start|, |start*|, |margin|, |end| and |manual|;
initially set to |start|) --
displays points in problems at the indicated location;
see \secref{sec:points} for details.

\item |subproblempointsat=|\textit{pos}
(can take values |start|, |start*|, |margin|, |end| and |manual|;
initially set to |end|) --
displays points in subproblems at the indicated location;
see \secref{sec:points} for details.

\item |solutionpointsat=|\textit{pos}
(can take values |start|, |start*|, |margin|, |end| and |manual|;
initially set to |end|) --
displays points in solutions at the indicated location;
see \secref{sec:points} for details.

\item |problemby={|\textit{counter}|}| --
number problems with the prefix \textit{counter},
i.e.\ reset the problem counter whenever \textit{counter} increases
and use a composite label \textit{counter}|.|\textit{problem}
to identify problems.

\item |equationby={|\textit{counter}|}| --
number the dedicated equation counters for sheets, problems and solutions
with the prefix \textit{counter}.

\item |problembysheet| --
number problems by sheet.

\item |equationbysheet| --
number dedicated equations for sheets, problems and solutions by sheet;
note that the main equation counter is unaffected by this setting,
it therefore makes sense to also activate the style |sheetequation|
or use |\counterwithin{equation}{sheet}|.

\item |pagebysheet| --
number pages by sheet and denote pages by \textit{sheet}|.|\textit{page};
this style is useful to generate stable page numbers for
a collection of sheets.

\item |sheetequation|[|=true|\textbar|false|]
(no value implies |true|, initially set to |false|) --
use a dedicated equation counter within sheets.

\item |problemequation|[|=true|\textbar|false|]
(no value implies |true|, initially set to |false|) --
use a dedicated equation counter within problems.

\item |solutionequation|[|=true|\textbar|false|]
(no value implies |true|, initially set to |true|) --
use a dedicated equation counter within solutions.

\item |fracpoints|[|=true|\textbar|false|]
(no value implies |true|, initially set to |false|) --
display fractional points for denominators 2, 4, 8;
see \secref{sec:points} for details.

\item |twoside|[|=true|\textbar|false|]
(no value implies |true|, initially set to |false|) --
enable/disable two-sided layout;
in two-sided layout, sheets will start on odd pages
and empty pages are added at the end of sheets
to produce an even number of pages.

\end{itemize}

%%%%%%%%%%%%%%%%%%%%%%%%%%%%%%%%%%%%%%%%
\DescribeMacro{extstyle}
Further exercise styles
are defined by the package option |extstyle|, see \secref{sec:options}:
%
\begin{itemize}
\item |plainheader| --
define a plain sheet header to display
some essential exercise and sheet data:
|course|, |institution|, |instructor|, |period| (optional), sheet |title|,
see \secref{sec:metadata};
the line below the header, font styles and spaces can be adjusted,
see the definition in \secref{sec:imp-styles}.

\item |contents| --
display sheets and problems in the table of contents
(as sections and subsections).

\item |solutionsf| --
display solutions in sans serif font family.

\item |solutiondimproblem| --
dim the problem text whenever solutions are displayed.

\item |solutionsep| --
separate the solutions from the remaining text by horizontal lines.

\end{itemize}

%%%%%%%%%%%%%%%%%%%%%%%%%%%%%%%%%%%%%%%%
\DescribeMacro{\defexercisestyle}
Custom styles can be defined by:
%
\begin{center}
\begin{tabular}{l}
|\defexercisestyle{|\textit{style}|}{|\textit{init}|}|\\
|\defexercisestylearg[|\textit{default}|]{|\textit{style}|}{|\textit{init}|}|
\end{tabular}
\end{center}
%
This feature can be used to predefine certain aspects of
the exercises layout.
For example, different default page layouts could be declared in this way.
The first version declares a style which is
initialised by the code \textit{item} upon activation by
|\exercisestyle{|\textit{style}[|=true|]|}|.
Note that |\exercisestyle{|\textit{style}|=false}| does nothing.
The second version declares a style which is activated by
|\exercisestyle{|\textit{style}[|={|\textit{arg}|}|]|}|
and which calls \textit{item} with the argument |#1| referring to \textit{arg}
(or \textit{default} if no argument is given).

%%%%%%%%%%%%%%%%%%%%%%%%%%%%%%%%%%%%%%%%%%%%%%%%%%%%%%%%%%%%%%%%%%%%%%%%%%%%%%%%
\subsection{Package Options}
\label{sec:options}

%%%%%%%%%%%%%%%%%%%%%%%%%%%%%%%%%%%%%%%%
\DescribeMacro{\exercisesetup}
Features and options of general nature can be selected by the commands:
%
\begin{center}
\begin{tabular}{rl}
&|\usepackage[|\textit{opts}|]{exframe}|
\\
or&|\PassOptionsToPackage{|\textit{opts}|}{exframe}|
\\
or&|\exercisesetup{|\textit{opts}|}|
\end{tabular}
\end{center}
%
|\PassOptionsToPackage| must be used before |\usepackage|;
|\exercisesetup| must be used afterwards.
\textit{opts} is a comma-separated list of options.

The following options are available only when
loading the package, i.e.\ they will not work
within |\exercisesetup|:
%
\begin{itemize}
\item |extdata|[|=true|\textbar|false|]
(no value implies |true|, initially set to |false|) --
define some more advanced metadata entries.

\item |extstyle|[|=true|\textbar|false|]
(no value implies |true|, initially set to |false|) --
define some more advanced styles.

\item |problemenv=|\textit{name} --
redefine environment name |problem|.
This and the following alike options may be useful in
quickly adjusting existing sources to
the \textsf{exframe} framework
if the original framework works similarly
and no special features are used.
Otherwise, it is highly advisable to leave the
names of environments and counters
defined by the package untouched.

\item |subproblemenv=|\textit{name} --
redefine environment name |subproblem|.

\item |solutionenv=|\textit{name} --
redefine environment name |solution|.

\item |sheetenv=|\textit{name} --
redefine environment name |sheet|.

\item |problemcounter=|\textit{name} --
redefine counter name |problem|.

\item |subproblemcounter=|\textit{name} --
redefine counter name |subproblem|.

\item |solutioncounter=|\textit{name} --
redefine counter name |solution|.

\item |sheetcounter=|\textit{name} --
redefine counter name |sheet|.

\end{itemize}
%

The following options can be specified by all three methods described above:
%
\begin{itemize}
\item |solutions|[|=true|\textbar|false|]
(no value implies |true|, initially set to |false|) --
Enable/disable display of solutions.
Sets the conditional |\ifsolutions| accordingly.

\item |pdfdata|[|=auto|\textbar|manual|\textbar|sheet|\textbar|off|]
(no value implies |auto|, initially set to |auto|) --
control writing most relevant metadata to pdf files;
has no effect without package \textsf{hyperref}.

\item |lineno|[|=true|\textbar|false|]
(no value implies |true|, initially set to |false|) --
enable/disable writing of line numbers as comments
into solution files.

\item |twoside|[|=true|\textbar|false|]
(no value implies |true|, initially set to |false|) --
enable/disable two-sided layout;
see \secref{sec:styles} for details.

\item |solutionhref|[|=true|\textbar|true|]
(no value implies |true|, initially set to |false|) --
enable/disable use of hyper-references from solutions
to the corresponding problems;
has no effect without package \textsf{hyperref}.

\item |warntext|[|=true|\textbar|false|]
(no value implies |true|, initially set to |false|) --
enable/disable writing of relevant warning messages
(points mismatch, point sums require update)
into the document output for easier detection.

\item |autolabelsheet|[|=true|\textbar|false|]
(no value implies |true|, initially set to |false|) --
enable/disable automatically assigning labels
(|sheet:\sheettag|; can be adjusted)
to sheets according to their tag |\sheettag|.

\item |autolabelproblem|[|=true|\textbar|false|]
(no value implies |true|, initially set to |false|) --
enable/disable automatically assigning labels
(|prob:\problemtag|; can be adjusted)
to problems according to their tag |\problemtag|.

\item |solutionbuf|[|=true|\textbar|false|]
(no value implies |true|, initially set to |true|) --
enable/disable buffering for |solution| environments
in order to control their display;
disabling buffering can be helpful
in debugging faulty |solution| environments;
it might also resolve some tokenisation issues in special circumstances;
note that the display of solutions cannot be suppressed
with |\exercisesetup{solutions=false}| when buffering if disabled.

\item |problembuf|[|=true|\textbar|false|]
(no value implies |true|, initially set to |false|) --
enable/disable buffering for |problem| environments
in order to control their display.

\end{itemize}

%%%%%%%%%%%%%%%%%%%%%%%%%%%%%%%%%%%%%%%%%%%%%%%%%%%%%%%%%%%%%%%%%%%%%%%%%%%%%%%%
%%%%%%%%%%%%%%%%%%%%%%%%%%%%%%%%%%%%%%%%%%%%%%%%%%%%%%%%%%%%%%%%%%%%%%%%%%%%%%%%
\section{Information}

%%%%%%%%%%%%%%%%%%%%%%%%%%%%%%%%%%%%%%%%%%%%%%%%%%%%%%%%%%%%%%%%%%%%%%%%%%%%%%%%
\subsection{Copyright}

Copyright \copyright{} 2011--2020 Niklas Beisert

This work may be distributed and/or modified under the
conditions of the \LaTeX{} Project Public License, either version 1.3
of this license or (at your option) any later version.
The latest version of this license is in
  \url{http://www.latex-project.org/lppl.txt}
and version 1.3 or later is part of all distributions of \LaTeX{}
version 2005/12/01 or later.

This work has the LPPL maintenance status `maintained'.

The Current Maintainer of this work is Niklas Beisert.

This work consists of the files |README.txt|, |exframe.ins| and |exframe.dtx|
as well as the derived files |exframe.sty|, |exfsamp.tex|, |exfserm.tex|,
|exfser|\textit{nn}|.tex| (\textit{nn}=|01|, |02|, |03|, |aa|),
|exfserpe.tex|, |exfserpf.tex|, |exfsermk.sh|, |exfsermk.mak|
and |exframe.pdf|.

%%%%%%%%%%%%%%%%%%%%%%%%%%%%%%%%%%%%%%%%%%%%%%%%%%%%%%%%%%%%%%%%%%%%%%%%%%%%%%%%
\subsection{Files and Installation}

The package consists of the files:
%
\begin{center}
\begin{tabular}{ll}
    |README.txt|   & readme file \\
    |exframe.ins|  & installation file \\
    |exframe.dtx|  & source file \\
    |exframe.sty|  & package file \\
    |exfsamp.tex|  & sample file \\
    |exfserm.tex|  & multipart sample main file \\
    |exfser01.tex| & multipart sample sheet 1 \\
    |exfser02.tex| & multipart sample sheet 2 \\
    |exfser03.tex| & multipart sample sheet 3 \\
    |exfseraa.tex| & multipart sample unused problems \\
    |exfserpe.tex| & multipart sample problem E \\
    |exfserpf.tex| & multipart sample problem F \\
    |exfsermk.sh|  & multipart sample compile script \\
    |exfsermk.mak| & multipart sample makefile \\
    |exframe.pdf|  & manual
\end{tabular}
\end{center}
%
The distribution consists of the files
|README.txt|, |exframe.ins| and |exframe.dtx|.
%
\begin{itemize}
\item
Run (pdf)\LaTeX{} on |exframe.dtx|
to compile the manual |exframe.pdf| (this file).
\item
Run \LaTeX{} on |exframe.ins| to create the package |exframe.sty|
and the samples consisting of |exfsamp.tex|, |exfserm.tex|,
|exfser01.tex|, |exfser02.tex|, |exfser03.tex|, |exfseraa.tex|,
|exfserpe.tex|, |exfserpf.tex|, |exfsermk.sh|, |exfsermk.mak|.
Copy the file |exframe.sty| to an appropriate directory of your \LaTeX{}
distribution, e.g.\ \textit{texmf-root}|/tex/latex/exframe|.
\end{itemize}

%%%%%%%%%%%%%%%%%%%%%%%%%%%%%%%%%%%%%%%%%%%%%%%%%%%%%%%%%%%%%%%%%%%%%%%%%%%%%%%%
\subsection{Related Packages}

The package makes use of other packages available at CTAN:
\begin{itemize}
\item
This package relies on some functionality of the package \textsf{verbatim}
to read verbatim code from the \LaTeX{} source without expansion of macros.
Compatibility with the \textsf{verbatim} package
has been tested with v1.5q (2014/10/28).
\item
This package uses the package
\href{http://ctan.org/pkg/xkeyval}{\textsf{xkeyval}}
to process the options for the package, environments and macros.
Compatibility with the \textsf{xkeyval} package
has been tested with v2.7a (2014/12/03).
\item
This package can use the package
\href{http://ctan.org/pkg/hyperref}{\textsf{hyperref}}
to include hyperlinks between problems and solutions.
Compatibility with the \textsf{hyperref} package
has been tested with v6.88e (2018/11/30).
\item
This package can use the package
\href{http://ctan.org/pkg/amstext}{\textsf{amstext}}
(which is automatically loaded by \textsf{amsmath})
to display text within equations.
Compatibility with the \textsf{amstext} package
has been tested with v2.01 (2000/06/29).
\item
This package uses the command |\currfilename|
provided by the package \textsf{currfile} (if available and loaded)
to indicate the \LaTeX{} source file in the generated metapost file.
Compatibility with the \textsf{currfile} package
has been tested with v0.7c (2015/04/23).
\end{itemize}

There are several other \LaTeX{} packages
which offer a similar functionality
varying largely in scope and sophistication:
%
\begin{itemize}
\item
The package \href{http://ctan.org/pkg/exsheets}{\textsf{exsheets}}
and its successor \href{http://ctan.org/pkg/xsim}{\textsf{xsim}}
provide a \LaTeX{} 3 style for typesetting exercises with solutions.
They offer options to
hide or delay solutions,
print only specific problems,
deal with points,
specify metadata,
handle exercise collections,
as well as some more specific options.
They allow to adjust the layout and choose among predefined ones.
\item
The package \href{http://ctan.org/pkg/exercise}{\textsf{exercise}}
provides a style for typesetting exercises with solutions.
It offers many options to
hide or delay solutions,
print only specific problems,
specify some metadata
as well as some more specific options.
It allows to customise the layout.
\item
The package \href{http://ctan.org/pkg/exercises}{\textsf{exercises}}
provides a style for typesetting exercises with solutions.
It offers options to hide solutions and deal with points.
It allows basic customisation of the layout.
\item
The package \href{http://ctan.org/pkg/exam}{\textsf{exam}}
provides a document class for typesetting exams conveniently.
It offers many options to hide solutions, deal with points
and deal with other exam-specific tasks.
It allows to adjust the layout and choose among predefined ones.
\item
The package \href{http://ctan.org/pkg/probsoln}{\textsf{probsoln}}
provides a style for typesetting exercises with solutions
which are stored in a collection.
It offers options to
hide solutions
and to assemble problems from an external collection.
\item
The packages
\href{http://ctan.org/pkg/uebungsblatt}{\textsf{uebungsblatt}},
\href{http://ctan.org/pkg/uassign}{\textsf{uassign}},
\href{http://ctan.org/pkg/mathexam}{\textsf{mathexam}},
\href{http://ctan.org/pkg/exsol}{\textsf{exsol}},
\href{https://github.com/mbauman/homework}{\textsf{homework}},
\href{https://gist.github.com/jhwilson/1278588}{\textsf{jhwhw}}
provide basic functionality for somewhat more particular situations.
\end{itemize}
%
See CTAN categories
\href{http://ctan.org/topic/exercise}{\textsf{exercise}}
and \href{http://ctan.org/topic/exam}{\textsf{exam}}
for further up-to-date packages.

The philosophy of the present package is to
define a low-level framework to describe exercises with solutions
to be used in various situations.
The aim is to provide the means to describe the content
(problems, solutions, sheets) in a simple fashion
and separate it from the various layout definitions and choices
which will define the appearance of the content.
The interface was designed to reduce potential conflict
with other packages and definitions.
The package itself does not define an elaborate layout,
but it provides means to adjust it in many ways and
to predefine custom layout schemes.
The package offers most of the functionality of the above packages,
but (presently) misses out on some more advanced features,
see \secref{sec:suggestions}.

\iffalse
%%%%%%%%%%%%%%%%%%%%%%%%%%%%%%%%%%%%%%%%%%%%%%%%%%%%%%%%%%%%%%%%%%%%%%%%%%%%%%%%
\subsection{Suggested Customisations}
\label{sec:customisations}

\textbf{philosophy:}
this is fundamental, relies on few other basic packages.
can improve by combining with other packages which provide solutions
for particular applications. some suggestions:

\textbf{Optional Starred}

\textbf{Till's Boxen}

%%%%%%%%%%%%%%%%%%%%%%%%%%%%%%%%%%%%%%%%%%%%%%%%%%%%%%%%%%%%%
%% Realize background color for solutions with tcolorbox:
%%%%%%%%%%%%%%%%%%%%%%%%%%%%%%%%%%%%%%%%%%%%%%%%%%%%%%%%%%%%%
%\RequirePackage{tcolorbox}
%\tcbuselibrary{breakable}
%\exercisestyle{solutionsep}
%\exerciseconfig{insertsolutionsbefore}{%
%  \begin{tcolorbox}[breakable,boxrule=0.2pt,sharp corners=all]}
%\exerciseconfig{insertsolutionsafter}{%
%  \end{tcolorbox}}
%% cannot use marginpar inside tcolorbox (nested floats), use marginnote instead:
%\exercisestyle{solutionpointsat=margin}
%\RequirePackage{marginnote}
%\exerciseconfig{insertpointsmargin}[1]{\marginnote{\footnotesize #1}}
%% Don't indent solutions:
%\exerciseconfig{skipsolutionitemsub}{0ex}
%% remove initial "Solution:" because the colorbox is highlight enough:
%\exerciseconfig{composetitlesolutionsingle}[2]{}

\textbf{Collections}

\textbf{keep list of tags per sheet. code to loop through list. Manuel Benz}

\fi

%%%%%%%%%%%%%%%%%%%%%%%%%%%%%%%%%%%%%%%%%%%%%%%%%%%%%%%%%%%%%%%%%%%%%%%%%%%%%%%%
\subsection{Feature Suggestions}
\label{sec:suggestions}

The following is a list of features which may be useful for future
versions of this package:
%
\begin{itemize}
\item
Add a section on useful combinations of customisation settings
to achieve specific goals. Please send suggestions.
\item
Option to hide problem text while maintaining access to embedded solutions
(for a version containing only solutions):
this is difficult to implement because the problem environment
cannot simply be discarded, but would have to be scanned very carefully
for the embedded solution;
instead process problems to some document and save solutions to file,
then read solutions from different document.
\item
Define structures for multiple-choice questions.
\end{itemize}

%%%%%%%%%%%%%%%%%%%%%%%%%%%%%%%%%%%%%%%%%%%%%%%%%%%%%%%%%%%%%%%%%%%%%%%%%%%%%%%%
\subsection{Revision History}

\iffalse
%%%%%%%%%%%%%%%%%%%%%%%%%%%%%%%%%%%%%%%%
\paragraph{v3.3+:} 2020/01/12

\begin{itemize}
\item
section on extensions
\item
\ldots
\end{itemize}
\fi

%%%%%%%%%%%%%%%%%%%%%%%%%%%%%%%%%%%%%%%%
\paragraph{v3.31:} 2020/01/11

\begin{itemize}
\item
|onlysolutions| environment for solution mode content
\item
sample multipart setup streamlined
\end{itemize}

%%%%%%%%%%%%%%%%%%%%%%%%%%%%%%%%%%%%%%%%
\paragraph{v3.3:} 2019/06/15

\begin{itemize}
\item
control display of |problem| environments via
package option |problembuf|:
manual display, write to file, disable individual problems
\item
|solutionbelow| mode |here*| superseded by
package option |solutionbuf|
\item
display total points within solution: |solutionpointsat|
(thanks to Till Bargheer for suggestion)
\item
read points for current sheet, (sub)problem and solution
(thanks to Johannes Hahn for suggestion)
\item
case switch for bonus points
(thanks to Johannes Hahn for suggestion)
\item
option to |disable| particular problems,
control by hook function
(thanks to Manuel Benz for suggestion)
\item
provided interface |\showfracpoints| and |\exerciseconfig{frac}|
for fractional points display
\item
filename extensions configurable
\end{itemize}

%%%%%%%%%%%%%%%%%%%%%%%%%%%%%%%%%%%%%%%%
\paragraph{v3.2:} 2019/05/01

\begin{itemize}
\item
bonus points can be specified as
|points=|[\textit{regular}][|+|\textit{bonus}]
\item
|solutionbelow| mode |here*| added
for direct processing of the solution environment
\item
multipart sample added
\end{itemize}

%%%%%%%%%%%%%%%%%%%%%%%%%%%%%%%%%%%%%%%%
\paragraph{v3.11:} 2019/04/15

\begin{itemize}
\item
fix interaction with
package \href{http://ctan.org/pkg/calc}{\textsf{calc}}
(thanks to Johannes Hahn for bug report)
\item
fix style |fracpoints| in combination
with some [|sub|]|problempointsat| choices
(thanks to Johannes Hahn for bug report)
\item
fix spacing for [|sub|]|problempointsat=margin|
(thanks to Johannes Hahn for bug report)
\end{itemize}

%%%%%%%%%%%%%%%%%%%%%%%%%%%%%%%%%%%%%%%%
\paragraph{v3.1:} 2019/01/21

\begin{itemize}
\item
alternate placement modes for solutions
\item
fixed expansion of problem title
\item
reset font size for problem text
\end{itemize}

%%%%%%%%%%%%%%%%%%%%%%%%%%%%%%%%%%%%%%%%
\paragraph{v3.0:} 2019/01/16

\begin{itemize}
\item
renamed to |exframe.sty|
\item
first version published on CTAN
\item
overhaul and streamline interface
\item
solution processing remodelled
\item
changed metadata handling
\item
changed and generalised points handling
\item
generalised sectioning layout
\item
changed layout specification model
\item
insert hyperlinks using \textsf{hyperref}
\item
manual, example and installation package added
\end{itemize}

%%%%%%%%%%%%%%%%%%%%%%%%%%%%%%%%%%%%%%%%
\paragraph{v2.0 -- v2.6:} 2014/10/03 -- 2018/11/05

\begin{itemize}
\item
changed metadata interface
\item
broadened scope
\item
added more layout options
\item
added more metadata
\item
added sheet and problem tags
\item
add and remember points
\end{itemize}

%%%%%%%%%%%%%%%%%%%%%%%%%%%%%%%%%%%%%%%%
\paragraph{v1.1 -- v1.6:} 2014/08/07 -- 2014/09/14

\begin{itemize}
\item
renamed to |nbprob.sty|
\item
added metadata
\item
added points
\item
added layout configuration
\item
removed specific macros
\end{itemize}

%%%%%%%%%%%%%%%%%%%%%%%%%%%%%%%%%%%%%%%%
\paragraph{v1.0 -- v1.02:} 2011/09/23 -- 2013/03/17

\begin{itemize}
\item
first version as |problems.cls|
\item
dedicated layout and macros for author's exercise sheets
\end{itemize}

%%%%%%%%%%%%%%%%%%%%%%%%%%%%%%%%%%%%%%%%%%%%%%%%%%%%%%%%%%%%%%%%%%%%%%%%%%%%%%%%
%%%%%%%%%%%%%%%%%%%%%%%%%%%%%%%%%%%%%%%%%%%%%%%%%%%%%%%%%%%%%%%%%%%%%%%%%%%%%%%%
%%%%%%%%%%%%%%%%%%%%%%%%%%%%%%%%%%%%%%%%%%%%%%%%%%%%%%%%%%%%%%%%%%%%%%%%%%%%%%%%
\appendix

\settowidth\MacroIndent{\rmfamily\scriptsize 0000\ }

 \DocInput{exframe.dtx}

\end{document}
%</driver>
% \fi
%
% %%%%%%%%%%%%%%%%%%%%%%%%%%%%%%%%%%%%%%%%%%%%%%%%%%%%%%%%%%%%%%%%%%%%%%%%%%%%%%
% %%%%%%%%%%%%%%%%%%%%%%%%%%%%%%%%%%%%%%%%%%%%%%%%%%%%%%%%%%%%%%%%%%%%%%%%%%%%%%
% \section{Standalone Sample}
% \label{sec:samplesingle}
%\iffalse
%<*samplesingle>
%\fi
%
% This section provides an example of how to use
% some of the \textsf{exframe} features.
% The resulting layout will be somewhat messy
% due to a random selection of features.
%
% This example file describes a single exercise sheet.
% The other sheet of the series would be declared
% analogously in independent documents.
%
% %%%%%%%%%%%%%%%%%%%%%%%%%%%%%%%%%%%%%%
% \paragraph{Preamble.}
%
% Standard document class:
%    \begin{macrocode}
\documentclass[12pt]{article}
%    \end{macrocode}

% Use package \textsf{geometry} to set the page layout;
% adjust the paragraph shape:
%    \begin{macrocode}
\usepackage{geometry}
\geometry{layout=a4paper}
\geometry{paper=a4paper}
\geometry{margin=2.5cm}
\parindent0pt
\parskip0.5ex
%    \end{macrocode}

% Include \textsf{amsmath}, \textsf{hyperref}
% and the \textsf{exframe} package:
%    \begin{macrocode}
\usepackage{amsmath}
\usepackage{hyperref}
\usepackage[extstyle]{exframe}
%    \end{macrocode}

% %%%%%%%%%%%%%%%%%%%%%%%%%%%%%%%%%%%%%%
% \paragraph{Solutions Switch.}
%
% It will be useful to have the switch to turn on/off
% the display of solutions near the top
% of the source file, potentially with the opposite setting commented out:
%    \begin{macrocode}
\exercisesetup{solutions=true}
%%\exercisesetup{solutions=false}
%    \end{macrocode}

% %%%%%%%%%%%%%%%%%%%%%%%%%%%%%%%%%%%%%%
% \paragraph{Layout Declarations.}
%
% The following layout declarations adjust the general layout
% of exercise sheets. They may as well be moved into an include file.
%
% Declare a header for exercise sheets
% to display several relevant pieces of data;
% display points total:
%    \begin{macrocode}
\exercisestyle{plainheader}
\exerciseconfig{composeheaderbelowright}{\getsheetdata{points}}%
%    \end{macrocode}

% Redefine the appearance of some counters;
% sheets should be labelled by capital roman numerals,
% subproblems by lowercase roman numerals;
% declare the widest subproblem item to be expected:
%    \begin{macrocode}
\exerciseconfig{countersheet}{\Roman{sheet}}
\exerciseconfig{countersubproblem}{\roman{subproblem})}
\exerciseconfig{countersubproblemmax}{vii)}
%    \end{macrocode}

% Automatically display an asterisk for all
% subproblems with bonus points only;
% remove space to separate items:
%    \begin{macrocode}
\exerciseconfig{insertsubprobleminfo}{%
 \switchpoints{}{\addprobleminfo*{%
   \hspace{-\getexerciseconfig{skipsubprobleminfo}}*}}%
  {}{}{\getsubproblempoints{}}}
%    \end{macrocode}

% Redefine the terms to be used for sheet(s);
% here, a German version:
%    \begin{macrocode}
\exerciseconfig{termsheet}{\"Ubungsblatt}
\exerciseconfig{termsheets}{\"Ubungsbl\"atter}
%    \end{macrocode}

% Display points for problems in the margin;
% change margin display to use the left margin;
% use the abbreviated form `$n$p.':
%    \begin{macrocode}
\exercisestyle{problempointsat=margin}
\reversemarginpar
\exerciseconfig{composepointsmargin}[1]{#1p.}
\exerciseconfig{composepointspairmargin}[2]{
  \ifdim#2pt=0pt#1p.%
  \else\ifdim#1pt=0pt+#2p.%
  \else#1+#2p.%
  \fi\fi}
%    \end{macrocode}

% Change the basic font style for all titles to be bold sans-serif:
%    \begin{macrocode}
\exerciseconfig{styletitle}{\sffamily\bfseries}
%    \end{macrocode}

% Add a significant amount of space below problems:
%    \begin{macrocode}
\exerciseconfig{skipproblembelow}{1.5cm}
%    \end{macrocode}

% Display half points as fractions:
%    \begin{macrocode}
\exercisestyle{fracpoints}
%    \end{macrocode}
% Show solutions below each problem
% (may try alternatives |subproblem| or |sheet|):
%    \begin{macrocode}
\exercisestyle{solutionbelow=problem}
%    \end{macrocode}
% Separate solutions by horizontal lines (extended style):
%    \begin{macrocode}
\exercisestyle{solutionsep}
%    \end{macrocode}

% Add course name to sheet title metadata:
%    \begin{macrocode}
\exerciseconfig{composemetasheet}[2]{\getexercisedata{course},
  \exerciseifempty{#2}{\getexerciseconfig{termsheet} #1}{#2}}
%    \end{macrocode}

% Set title and author for pdf metadata:
%    \begin{macrocode}
\exercisesetup{pdfdata=sheet}
\exercisedata{title=%
  {\getexercisedata{course}, \getexercisedata{material}}}
\exercisedata{author=%
  {\getexercisedata{instructor}, \getexercisedata{institution}}}
%    \end{macrocode}

% %%%%%%%%%%%%%%%%%%%%%%%%%%%%%%%%%%%%%%
% \paragraph{Exercise Series Data.}
%
% Set some data on the current series:
%    \begin{macrocode}
\exercisedata{institution={Katharinen-Volksschule}}
\exercisedata{course={Mathematik}}
\exercisedata{instructor={J.\ G.\ B\"uttner}}
\exercisedata{period={ca.\ 1786}}
\exercisedata{material={\"Ubungsaufgaben}}
%    \end{macrocode}

% %%%%%%%%%%%%%%%%%%%%%%%%%%%%%%%%%%%%%%
% \paragraph{Body.}
%
%    \begin{macrocode}
\begin{document}
%    \end{macrocode}

% Start sheet number 5:
%    \begin{macrocode}
\begin{sheet}[number=5]
%    \end{macrocode}

% Start a problem with a title:
%    \begin{macrocode}
\begin{problem}[title={Sums},points=99+4]
%    \end{macrocode}

% Some introduction to the problem:
%    \begin{macrocode}
This problem deals with sums and series.
%    \end{macrocode}

% A subproblem with a local label:
%    \begin{macrocode}
\begin{subproblem}[points=2,difficulty=simple,label={\problemtag-simplesum}]
Compute the sum
\showpoints
\begin{equation}
1+2+3.
\end{equation}
%    \end{macrocode}

% Provide a solution for the subproblem (within the subproblem environment):
%    \begin{macrocode}
\begin{solution}
The result is
\begin{equation}
1+2+3=6.
\end{equation}
\end{solution}
%    \end{macrocode}

% End subproblem:
%    \begin{macrocode}
\end{subproblem}
%    \end{macrocode}

% Another subproblem:
%    \begin{macrocode}
\begin{subproblem}[points=97+0.5,difficulty=lengthy]
Compute the sum
\begin{equation}
1+2+3+\ldots+98+99+100.
\end{equation}
Keep calm and calculate!
%%That ought to keep him occupied for a while
\end{subproblem}
%    \end{macrocode}

% Provide a solution for the previous subproblem
% (layout may differ slightly from declaration within);
% declare author:
%    \begin{macrocode}
\begin{solution}[author={C.\ F.\ Gau\ss}]
We use the result $1+2+3=6$ from part \ref{\problemtag-simplesum}
to jumpstart the calculation. The remaining sums yield
\awardpoints*[1 for each remaining sum]{97}
\begin{equation}
6+4+5+\ldots+99+100=5050.
\end{equation}
Alternatively the summands can be grouped into pairs as follows:
\begin{align}
1+100&=101,\\
2+99&=101,\\
3+98&=101,\\
\ldots &\nonumber\\
50+51&=101.
\end{align}
These amount to 50 times the same number 101.
Therefore the sum equals
\begin{equation}
1+2+\ldots+99+100=50\cdot 101=5050.
\end{equation}
\textit{Ligget se!} \awardpoints{97+0.5}
\end{solution}
%    \end{macrocode}

% Some text between subproblems:
%    \begin{macrocode}
You may give the final part a try:
%    \end{macrocode}

% Final subproblem; this one is optional:
%    \begin{macrocode}
\begin{subproblem}[optional={optional},
  difficulty={requires inspiration},points={+3.5}]
Compute the series
\showpoints
\begin{equation}
1+2+3+\ldots
\end{equation}
%    \end{macrocode}

% Provide a solution:
%    \begin{macrocode}
\begin{solution}
The series is divergent, so the result is $\infty$ \awardpoints{+1}.
\par
However, after subtracting the divergent part,
the result clearly is
\begin{equation}
\zeta(-1)=-\frac{1}{12}\,,
\end{equation}
where the zeta-function $\zeta(s)$ is defined by
\begin{equation}
\zeta(s):=\sum_{k=1}^\infty \frac{1}{k^s}\,.
\end{equation}
This definition holds only for $s>1$ where the sum is convergent,
but one can continue the complex analytic function to $s<0$
\awardpoints{+1.5}.
\par
Another way of understanding the result
is to use the indefinite summation formula
for arbitrary exponent $s$ in the summand
(which also follows from the Euler--MacLaurin formula)
\begin{equation}
\sum_n n^s
= \frac{n^{s+1}}{s+1}
 -\sum_{j=0}^s  \frac{\zeta(j-s)\,s!}{(s-j)!\,j!}\,n^j
= \ldots - \zeta(-s)\,n^0.
\end{equation}
Curiously, the constant term with $j=0$ is just the desired result
but with the wrong sign
(in fact, the constant term of an indefinite sum is ambiguous;
for the claim we merely set $j=0$
in the expression which holds for others values of $j$)
\awardpoints{+0.5}.
In order to understand the sign,
we propose that the above formula describes the regularised result
for the sum with limits $+\infty$ and $n$
\begin{equation}
\sum_{k=+\infty}^n k^s
\simeq \frac{n^{s+1}}{s+1}
 -\sum_{j=0}^s  \frac{\zeta(j-s)\,s!}{(s-j)!\,j!}\,n^j.
\end{equation}
Then we flip the summation limits of the desired sum
to bring it into the above form
\awardpoints{+0.5}
\begin{equation}
\sum_{k=1}^\infty k^s
= -\sum_{k=\infty}^0 k^s
\simeq \zeta(-s).
\end{equation}
\end{solution}
%    \end{macrocode}

% End subproblem:
%    \begin{macrocode}
\end{subproblem}
%    \end{macrocode}

% End problem:
%    \begin{macrocode}
\end{problem}
%    \end{macrocode}

% Another problem; this one is untitled:
%    \begin{macrocode}
\begin{problem}[points=1, difficulty=insane]
Show that the equation
\begin{equation}
a^3+b^3=c^3
\end{equation}
has no positive integer solutions.
\end{problem}
%    \end{macrocode}

% A solution can also follow a problem
% (but the layout may be slightly different,
% e.g.\ here the space below the problem will
% appear before the solution):
%    \begin{macrocode}
\begin{solution}
\normalmarginpar
This is beyond the scope of this example.
\marginpar{\footnotesize\raggedright does not fit here.\par}
\end{solution}
%    \end{macrocode}

% End sheet:
%    \begin{macrocode}
\end{sheet}
%    \end{macrocode}

% End of document body:
%    \begin{macrocode}
\end{document}
%    \end{macrocode}
%\iffalse
%</samplesingle>
%\fi
%
% %%%%%%%%%%%%%%%%%%%%%%%%%%%%%%%%%%%%%%%%%%%%%%%%%%%%%%%%%%%%%%%%%%%%%%%%%%%%%%
% %%%%%%%%%%%%%%%%%%%%%%%%%%%%%%%%%%%%%%%%%%%%%%%%%%%%%%%%%%%%%%%%%%%%%%%%%%%%%%
% \section{Multipart Sample}
% \label{sec:samplemulti}
%
% The second example describes a series of exercise sheets
% which can be compiled as a collection or as individual sheets.
% This example describes a versatile setup
% with several convenient features;
% most of these features can be adjusted or removed easily as they
% mostly enhance the setup and do not interact with each other strongly.
%
% %%%%%%%%%%%%%%%%%%%%%%%%%%%%%%%%%%%%%%%%%%%%%%%%%%%%%%%%%%%%%%%%%%%%%%%%%%%%%%
% \subsection{Main File}
% \label{sec:samplemultimain}
%\iffalse
%<*samplemultimain>
%\fi
%
% The main source file is called |exfserm.tex|.
% It is referenced at several places within the setup,
% and when changing the name they need to be adjusted accordingly.
%
% %%%%%%%%%%%%%%%%%%%%%%%%%%%%%%%%%%%%%%
% \paragraph{childdoc Mechanism.}
%
% The setup uses the package \textsf{childdoc} to allow compilation
% of the series as a whole or in parts and with various sets of options:
%    \begin{macrocode}
% \iffalse
%
% childdoc.dtx Copyright (C) 2017-2018 Niklas Beisert
%
% This work may be distributed and/or modified under the
% conditions of the LaTeX Project Public License, either version 1.3
% of this license or (at your option) any later version.
% The latest version of this license is in
%   http://www.latex-project.org/lppl.txt
% and version 1.3 or later is part of all distributions of LaTeX
% version 2005/12/01 or later.
%
% This work has the LPPL maintenance status `maintained'.
%
% The Current Maintainer of this work is Niklas Beisert.
%
% This work consists of the files childdoc.dtx and childdoc.ins
% and the derived files childdoc.def and cdocsamp.tex with
% cdocsch1.tex, cdocsch2.tex, cdocsdrf.tex, cdocsfn1.tex, cdocsfn2.tex.
%
%<package>\ifdefined\childdocmain\endinput\fi
%<package>\ProvidesFile{childdoc.def}[2018/12/30 v2.0 child document driver]
%<samplemain>\ProvidesFile{cdocsamp.tex}[2018/12/30 v2.0 sample for childdoc]
%<*driver>
%\ProvidesFile{childdoc.drv}[2018/12/30 v2.0 childdoc reference manual file]
\PassOptionsToClass{10pt,a4paper}{article}
\documentclass{ltxdoc}

\usepackage[margin=35mm]{geometry}
\usepackage{hyperref}
\usepackage{hyperxmp}
\usepackage[usenames]{color}

\hypersetup{colorlinks=true}
\hypersetup{pdfstartview=FitH}
\hypersetup{pdfpagemode=UseNone}
\hypersetup{pdfsource={}}
\hypersetup{pdflang={en-UK}}
\hypersetup{pdfcopyright={Copyright 2017-2018 Niklas Beisert.
  This work may be distributed and/or modified under the
  conditions of the LaTeX Project Public License, either version 1.3
  of this license or (at your option) any later version.}}
\hypersetup{pdflicenseurl={http://www.latex-project.org/lppl.txt}}
\hypersetup{pdfcontactaddress={ETH Zurich, ITP, HIT K,
  Wolfgang-Pauli-Strasse 27}}
\hypersetup{pdfcontactpostcode={8093}}
\hypersetup{pdfcontactcity={Zurich}}
\hypersetup{pdfcontactcountry={Switzerland}}
\hypersetup{pdfcontactemail={nbeisert@itp.phys.ethz.ch}}
\hypersetup{pdfcontacturl={http://people.phys.ethz.ch/\xmptilde nbeisert/}}

\newcommand{\secref}[1]{\hyperref[#1]{section \ref*{#1}}}

\parskip1ex
\parindent0pt
\let\olditemize\itemize
\def\itemize{\olditemize\parskip0pt}

\begin{document}

\title{The \textsf{childdoc} Package}
\hypersetup{pdftitle={The childdoc Package}}
\author{Niklas Beisert\\[2ex]
  Institut f\"ur Theoretische Physik\\
  Eidgen\"ossische Technische Hochschule Z\"urich\\
  Wolfgang-Pauli-Strasse 27, 8093 Z\"urich, Switzerland\\[1ex]
  \href{mailto:nbeisert@itp.phys.ethz.ch}
  {\texttt{nbeisert@itp.phys.ethz.ch}}}
\hypersetup{pdfauthor={Niklas Beisert}}
\hypersetup{pdfsubject={Manual for the LaTeX2e Package childdoc}}
\date{30 December 2018, \textsf{v2.0}}
\maketitle

\begin{abstract}\noindent
\textsf{childdoc} is a \LaTeXe{} package
that enables the direct compilation
of document sections included by |\include|
to individual files.
\end{abstract}

\begingroup
\parskip0ex
\tableofcontents
\endgroup

%%%%%%%%%%%%%%%%%%%%%%%%%%%%%%%%%%%%%%%%%%%%%%%%%%%%%%%%%%%%%%%%%%%%%%%%%%%%%%%%
%%%%%%%%%%%%%%%%%%%%%%%%%%%%%%%%%%%%%%%%%%%%%%%%%%%%%%%%%%%%%%%%%%%%%%%%%%%%%%%%
\section{Introduction}

\LaTeX{} provides a mechanism to structure a large document (such as a book)
into a main file and several child files (containing the chapters)
using the |\include| command.
This mechanism is beneficial for documents
which span hundreds of pages in order to
make the source file(s) more manageable.
Moreover, compilation can be restricted to
selected child files by means of the |\includeonly| command.
The latter feature can be used to reduce the compilation time while editing
(this was significantly more useful in the earlier days of \LaTeX{})
or to generate a smaller document which is easier to navigate.
Another application of |\includeonly| is to generate
documents consisting of selected parts of the complete document.

However, there are a few drawbacks of the plain |\include| mechanism:
\begin{itemize}
\item
The child files cannot be compiled on their own,
they can only be compiled via the main file.
A naive editing environment
(such as a text editor with an option
to have the current file processed by \LaTeX)
may require one to switch to the main file before compiling;
attempting to compile the child file produces errors.
\item
The main file must be modified (each time)
to adjust the |\includeonly| command
to the present needs. This easily leaves the main file in a messy state.
\item
The generated document will always carry the filename
of the main document. This is inconvenient if
several child files are to be compiled and
to be kept for distribution.
\end{itemize}

The present package provides a simple interface
to make child files individually compilable by \LaTeX{}.
Compiling a child file then has the same effect as compiling
the main file with an |\includeonly| command
to select the appropriate child.
Moreover the generated document will carry the name of the child
rather than the main file.
This resolves all three above issues.

This feature is meant to make the editing of books,
thesis documents and lecture notes somewhat more convenient.
However, the package can also be used efficiently for
composing a series of documents (such as exercise sheets)
which are typically distributed individually.
It then assists the author in generating the individual documents
(potentially in different versions)
as well as a document containing the collected series.
Another application is in developing style files
or other kinds of included material
where compilation of the style file could redirect
to a sample or test file.

%%%%%%%%%%%%%%%%%%%%%%%%%%%%%%%%%%%%%%%%%%%%%%%%%%%%%%%%%%%%%%%%%%%%%%%%%%%%%%%%
%%%%%%%%%%%%%%%%%%%%%%%%%%%%%%%%%%%%%%%%%%%%%%%%%%%%%%%%%%%%%%%%%%%%%%%%%%%%%%%%
\section{Usage}

First of all, the package \textsf{childdoc} is \emph{not} a standard
\LaTeXe{} |.sty| style file! Therefore it needs to be invoked in
a non-standard way.

%%%%%%%%%%%%%%%%%%%%%%%%%%%%%%%%%%%%%%%%%%%%%%%%%%%%%%%%%%%%%%%%%%%%%%%%%%%%%%%%
\subsection{Included Files}
\label{sec:include}

%%%%%%%%%%%%%%%%%%%%%%%%%%%%%%%%%%%%%%%%
\DescribeMacro{\childdocmain}
To use the package, add the commands
\begin{center}
\begin{tabular}{l}
|% \iffalse
%
% childdoc.dtx Copyright (C) 2017-2018 Niklas Beisert
%
% This work may be distributed and/or modified under the
% conditions of the LaTeX Project Public License, either version 1.3
% of this license or (at your option) any later version.
% The latest version of this license is in
%   http://www.latex-project.org/lppl.txt
% and version 1.3 or later is part of all distributions of LaTeX
% version 2005/12/01 or later.
%
% This work has the LPPL maintenance status `maintained'.
%
% The Current Maintainer of this work is Niklas Beisert.
%
% This work consists of the files childdoc.dtx and childdoc.ins
% and the derived files childdoc.def and cdocsamp.tex with
% cdocsch1.tex, cdocsch2.tex, cdocsdrf.tex, cdocsfn1.tex, cdocsfn2.tex.
%
%<package>\ifdefined\childdocmain\endinput\fi
%<package>\ProvidesFile{childdoc.def}[2018/12/30 v2.0 child document driver]
%<samplemain>\ProvidesFile{cdocsamp.tex}[2018/12/30 v2.0 sample for childdoc]
%<*driver>
%\ProvidesFile{childdoc.drv}[2018/12/30 v2.0 childdoc reference manual file]
\PassOptionsToClass{10pt,a4paper}{article}
\documentclass{ltxdoc}

\usepackage[margin=35mm]{geometry}
\usepackage{hyperref}
\usepackage{hyperxmp}
\usepackage[usenames]{color}

\hypersetup{colorlinks=true}
\hypersetup{pdfstartview=FitH}
\hypersetup{pdfpagemode=UseNone}
\hypersetup{pdfsource={}}
\hypersetup{pdflang={en-UK}}
\hypersetup{pdfcopyright={Copyright 2017-2018 Niklas Beisert.
  This work may be distributed and/or modified under the
  conditions of the LaTeX Project Public License, either version 1.3
  of this license or (at your option) any later version.}}
\hypersetup{pdflicenseurl={http://www.latex-project.org/lppl.txt}}
\hypersetup{pdfcontactaddress={ETH Zurich, ITP, HIT K,
  Wolfgang-Pauli-Strasse 27}}
\hypersetup{pdfcontactpostcode={8093}}
\hypersetup{pdfcontactcity={Zurich}}
\hypersetup{pdfcontactcountry={Switzerland}}
\hypersetup{pdfcontactemail={nbeisert@itp.phys.ethz.ch}}
\hypersetup{pdfcontacturl={http://people.phys.ethz.ch/\xmptilde nbeisert/}}

\newcommand{\secref}[1]{\hyperref[#1]{section \ref*{#1}}}

\parskip1ex
\parindent0pt
\let\olditemize\itemize
\def\itemize{\olditemize\parskip0pt}

\begin{document}

\title{The \textsf{childdoc} Package}
\hypersetup{pdftitle={The childdoc Package}}
\author{Niklas Beisert\\[2ex]
  Institut f\"ur Theoretische Physik\\
  Eidgen\"ossische Technische Hochschule Z\"urich\\
  Wolfgang-Pauli-Strasse 27, 8093 Z\"urich, Switzerland\\[1ex]
  \href{mailto:nbeisert@itp.phys.ethz.ch}
  {\texttt{nbeisert@itp.phys.ethz.ch}}}
\hypersetup{pdfauthor={Niklas Beisert}}
\hypersetup{pdfsubject={Manual for the LaTeX2e Package childdoc}}
\date{30 December 2018, \textsf{v2.0}}
\maketitle

\begin{abstract}\noindent
\textsf{childdoc} is a \LaTeXe{} package
that enables the direct compilation
of document sections included by |\include|
to individual files.
\end{abstract}

\begingroup
\parskip0ex
\tableofcontents
\endgroup

%%%%%%%%%%%%%%%%%%%%%%%%%%%%%%%%%%%%%%%%%%%%%%%%%%%%%%%%%%%%%%%%%%%%%%%%%%%%%%%%
%%%%%%%%%%%%%%%%%%%%%%%%%%%%%%%%%%%%%%%%%%%%%%%%%%%%%%%%%%%%%%%%%%%%%%%%%%%%%%%%
\section{Introduction}

\LaTeX{} provides a mechanism to structure a large document (such as a book)
into a main file and several child files (containing the chapters)
using the |\include| command.
This mechanism is beneficial for documents
which span hundreds of pages in order to
make the source file(s) more manageable.
Moreover, compilation can be restricted to
selected child files by means of the |\includeonly| command.
The latter feature can be used to reduce the compilation time while editing
(this was significantly more useful in the earlier days of \LaTeX{})
or to generate a smaller document which is easier to navigate.
Another application of |\includeonly| is to generate
documents consisting of selected parts of the complete document.

However, there are a few drawbacks of the plain |\include| mechanism:
\begin{itemize}
\item
The child files cannot be compiled on their own,
they can only be compiled via the main file.
A naive editing environment
(such as a text editor with an option
to have the current file processed by \LaTeX)
may require one to switch to the main file before compiling;
attempting to compile the child file produces errors.
\item
The main file must be modified (each time)
to adjust the |\includeonly| command
to the present needs. This easily leaves the main file in a messy state.
\item
The generated document will always carry the filename
of the main document. This is inconvenient if
several child files are to be compiled and
to be kept for distribution.
\end{itemize}

The present package provides a simple interface
to make child files individually compilable by \LaTeX{}.
Compiling a child file then has the same effect as compiling
the main file with an |\includeonly| command
to select the appropriate child.
Moreover the generated document will carry the name of the child
rather than the main file.
This resolves all three above issues.

This feature is meant to make the editing of books,
thesis documents and lecture notes somewhat more convenient.
However, the package can also be used efficiently for
composing a series of documents (such as exercise sheets)
which are typically distributed individually.
It then assists the author in generating the individual documents
(potentially in different versions)
as well as a document containing the collected series.
Another application is in developing style files
or other kinds of included material
where compilation of the style file could redirect
to a sample or test file.

%%%%%%%%%%%%%%%%%%%%%%%%%%%%%%%%%%%%%%%%%%%%%%%%%%%%%%%%%%%%%%%%%%%%%%%%%%%%%%%%
%%%%%%%%%%%%%%%%%%%%%%%%%%%%%%%%%%%%%%%%%%%%%%%%%%%%%%%%%%%%%%%%%%%%%%%%%%%%%%%%
\section{Usage}

First of all, the package \textsf{childdoc} is \emph{not} a standard
\LaTeXe{} |.sty| style file! Therefore it needs to be invoked in
a non-standard way.

%%%%%%%%%%%%%%%%%%%%%%%%%%%%%%%%%%%%%%%%%%%%%%%%%%%%%%%%%%%%%%%%%%%%%%%%%%%%%%%%
\subsection{Included Files}
\label{sec:include}

%%%%%%%%%%%%%%%%%%%%%%%%%%%%%%%%%%%%%%%%
\DescribeMacro{\childdocmain}
To use the package, add the commands
\begin{center}
\begin{tabular}{l}
|% \iffalse
%
% childdoc.dtx Copyright (C) 2017-2018 Niklas Beisert
%
% This work may be distributed and/or modified under the
% conditions of the LaTeX Project Public License, either version 1.3
% of this license or (at your option) any later version.
% The latest version of this license is in
%   http://www.latex-project.org/lppl.txt
% and version 1.3 or later is part of all distributions of LaTeX
% version 2005/12/01 or later.
%
% This work has the LPPL maintenance status `maintained'.
%
% The Current Maintainer of this work is Niklas Beisert.
%
% This work consists of the files childdoc.dtx and childdoc.ins
% and the derived files childdoc.def and cdocsamp.tex with
% cdocsch1.tex, cdocsch2.tex, cdocsdrf.tex, cdocsfn1.tex, cdocsfn2.tex.
%
%<package>\ifdefined\childdocmain\endinput\fi
%<package>\ProvidesFile{childdoc.def}[2018/12/30 v2.0 child document driver]
%<samplemain>\ProvidesFile{cdocsamp.tex}[2018/12/30 v2.0 sample for childdoc]
%<*driver>
%\ProvidesFile{childdoc.drv}[2018/12/30 v2.0 childdoc reference manual file]
\PassOptionsToClass{10pt,a4paper}{article}
\documentclass{ltxdoc}

\usepackage[margin=35mm]{geometry}
\usepackage{hyperref}
\usepackage{hyperxmp}
\usepackage[usenames]{color}

\hypersetup{colorlinks=true}
\hypersetup{pdfstartview=FitH}
\hypersetup{pdfpagemode=UseNone}
\hypersetup{pdfsource={}}
\hypersetup{pdflang={en-UK}}
\hypersetup{pdfcopyright={Copyright 2017-2018 Niklas Beisert.
  This work may be distributed and/or modified under the
  conditions of the LaTeX Project Public License, either version 1.3
  of this license or (at your option) any later version.}}
\hypersetup{pdflicenseurl={http://www.latex-project.org/lppl.txt}}
\hypersetup{pdfcontactaddress={ETH Zurich, ITP, HIT K,
  Wolfgang-Pauli-Strasse 27}}
\hypersetup{pdfcontactpostcode={8093}}
\hypersetup{pdfcontactcity={Zurich}}
\hypersetup{pdfcontactcountry={Switzerland}}
\hypersetup{pdfcontactemail={nbeisert@itp.phys.ethz.ch}}
\hypersetup{pdfcontacturl={http://people.phys.ethz.ch/\xmptilde nbeisert/}}

\newcommand{\secref}[1]{\hyperref[#1]{section \ref*{#1}}}

\parskip1ex
\parindent0pt
\let\olditemize\itemize
\def\itemize{\olditemize\parskip0pt}

\begin{document}

\title{The \textsf{childdoc} Package}
\hypersetup{pdftitle={The childdoc Package}}
\author{Niklas Beisert\\[2ex]
  Institut f\"ur Theoretische Physik\\
  Eidgen\"ossische Technische Hochschule Z\"urich\\
  Wolfgang-Pauli-Strasse 27, 8093 Z\"urich, Switzerland\\[1ex]
  \href{mailto:nbeisert@itp.phys.ethz.ch}
  {\texttt{nbeisert@itp.phys.ethz.ch}}}
\hypersetup{pdfauthor={Niklas Beisert}}
\hypersetup{pdfsubject={Manual for the LaTeX2e Package childdoc}}
\date{30 December 2018, \textsf{v2.0}}
\maketitle

\begin{abstract}\noindent
\textsf{childdoc} is a \LaTeXe{} package
that enables the direct compilation
of document sections included by |\include|
to individual files.
\end{abstract}

\begingroup
\parskip0ex
\tableofcontents
\endgroup

%%%%%%%%%%%%%%%%%%%%%%%%%%%%%%%%%%%%%%%%%%%%%%%%%%%%%%%%%%%%%%%%%%%%%%%%%%%%%%%%
%%%%%%%%%%%%%%%%%%%%%%%%%%%%%%%%%%%%%%%%%%%%%%%%%%%%%%%%%%%%%%%%%%%%%%%%%%%%%%%%
\section{Introduction}

\LaTeX{} provides a mechanism to structure a large document (such as a book)
into a main file and several child files (containing the chapters)
using the |\include| command.
This mechanism is beneficial for documents
which span hundreds of pages in order to
make the source file(s) more manageable.
Moreover, compilation can be restricted to
selected child files by means of the |\includeonly| command.
The latter feature can be used to reduce the compilation time while editing
(this was significantly more useful in the earlier days of \LaTeX{})
or to generate a smaller document which is easier to navigate.
Another application of |\includeonly| is to generate
documents consisting of selected parts of the complete document.

However, there are a few drawbacks of the plain |\include| mechanism:
\begin{itemize}
\item
The child files cannot be compiled on their own,
they can only be compiled via the main file.
A naive editing environment
(such as a text editor with an option
to have the current file processed by \LaTeX)
may require one to switch to the main file before compiling;
attempting to compile the child file produces errors.
\item
The main file must be modified (each time)
to adjust the |\includeonly| command
to the present needs. This easily leaves the main file in a messy state.
\item
The generated document will always carry the filename
of the main document. This is inconvenient if
several child files are to be compiled and
to be kept for distribution.
\end{itemize}

The present package provides a simple interface
to make child files individually compilable by \LaTeX{}.
Compiling a child file then has the same effect as compiling
the main file with an |\includeonly| command
to select the appropriate child.
Moreover the generated document will carry the name of the child
rather than the main file.
This resolves all three above issues.

This feature is meant to make the editing of books,
thesis documents and lecture notes somewhat more convenient.
However, the package can also be used efficiently for
composing a series of documents (such as exercise sheets)
which are typically distributed individually.
It then assists the author in generating the individual documents
(potentially in different versions)
as well as a document containing the collected series.
Another application is in developing style files
or other kinds of included material
where compilation of the style file could redirect
to a sample or test file.

%%%%%%%%%%%%%%%%%%%%%%%%%%%%%%%%%%%%%%%%%%%%%%%%%%%%%%%%%%%%%%%%%%%%%%%%%%%%%%%%
%%%%%%%%%%%%%%%%%%%%%%%%%%%%%%%%%%%%%%%%%%%%%%%%%%%%%%%%%%%%%%%%%%%%%%%%%%%%%%%%
\section{Usage}

First of all, the package \textsf{childdoc} is \emph{not} a standard
\LaTeXe{} |.sty| style file! Therefore it needs to be invoked in
a non-standard way.

%%%%%%%%%%%%%%%%%%%%%%%%%%%%%%%%%%%%%%%%%%%%%%%%%%%%%%%%%%%%%%%%%%%%%%%%%%%%%%%%
\subsection{Included Files}
\label{sec:include}

%%%%%%%%%%%%%%%%%%%%%%%%%%%%%%%%%%%%%%%%
\DescribeMacro{\childdocmain}
To use the package, add the commands
\begin{center}
\begin{tabular}{l}
|\input{childdoc.def}|\\
|\childdocmain{}|\\
\end{tabular}
\end{center}
at the very top of the main \LaTeX{} file,
in particular \emph{before} the |\documentclass| statement!
The argument of |\childdocmain| should be left empty
(but it must be present).

%%%%%%%%%%%%%%%%%%%%%%%%%%%%%%%%%%%%%%%%
\DescribeMacro{\childdocof}
Furthermore, add the commands
\begin{center}
\begin{tabular}{l}
|\input{childdoc.def}|\\
|\childdocof{|\textit{main}|}|\\
\end{tabular}
\end{center}
at the top of every child file \textit{child}
which is included by |\include{|\textit{child}|}|
from within the main file
(or at least for those files to be compiled individually).
The argument \textit{main} must be the filename of the main file.

There are a couple of
considerations in setting up the main and child documents:

%%%%%%%%%%%%%%%%%%%%%%%%%%%%%%%%%%%%%%%%
\paragraph{Restrictions.}

Please note the following restrictions:
\begin{itemize}
\item
|\childdocmain| must be called with one argument \textit{main}
to ensure compatibility with earlier version of the package.
It must either be empty (|\childdocmain{}|)
or precisely match the filename of the main file in which it is specified.
See \secref{sec:detection} for further information.
\item
The filename \textit{main} must be specified without the |.tex| extension.
\item
The filename \textit{main} is case sensitive
(even in case-insensitive file systems)
due to internal string comparison.
\item
The argument \textit{main} should be fully expanded, it cannot be a macro.
\item
Subdirectories and special characters should be avoided in filenames.
\item
The command |\childdocmain{|\textit{main}|}| must be followed by a whitespace.
It should not be followed immediately by another command
or by a comment mark `|%|'.
This is because the \TeX{} parser reads the token immediately following
the argument of |\childdocmain| and puts it
at the beginning of every child section;
however, a white\-space is ignored.
\end{itemize}

%%%%%%%%%%%%%%%%%%%%%%%%%%%%%%%%%%%%%%%%
\paragraph{Content of Main File.}

It is advisable to place all content in the child files included by |\include|.
Any output contained in the main file will appear in all child documents
unless suppressed manually;
it cannot be suppressed automatically by the |\includeonly| directive
and thus should normally be avoided.
A method to include some content in the main file
by means of conditional processing is described in \secref{sec:conditional}.

%%%%%%%%%%%%%%%%%%%%%%%%%%%%%%%%%%%%%%%%
\paragraph{Page Numbering.}

When only a part of the document is compiled,
the appropriate numbering of pages
(as well as other status parameters)
is determined from the |.aux| files.
The latter contain information from previous passes.
However this information needs to propagate through
all intermediate child documents.
Therefore the page numbering in child documents may well
be inconsistent until the complete document is compiled at least once.

A useful (if unconventional) way to always ensure a consistent
page numbering is to restart the numbering in each child document
and denote the pages by `\textit{child}|.|\textit{page}'
where \textit{child} represents the chapter/section number of the child file.
This can be achieved by the command
|\numberwithin{page}{|\textit{child}|}|
of the \textsf{amsmath} package
where \textit{child} can be |chapter| or |section|
depending on the chosen structuring.
Alternatively, one can modify the macro |\thepage| appropriately
and reset the counter |page| at the start of each child file.

%%%%%%%%%%%%%%%%%%%%%%%%%%%%%%%%%%%%%%%%%%%%%%%%%%%%%%%%%%%%%%%%%%%%%%%%%%%%%%%%
\subsection{Conditional Processing}
\label{sec:conditional}

The package provides a mechanism to compile different versions
of a document. To customise the versions further some conditional processing
can come in handy to distinguish which version is being compiled.
The package provides two macros to describe the compilation context:

%%%%%%%%%%%%%%%%%%%%%%%%%%%%%%%%%%%%%%%%
\DescribeMacro{\ifchilddoc}
The conditional |\ifchilddoc| distinguishes between the compilation of
child documents and the main document:
%
\begin{center}
|\ifchilddoc |\textit{child-code}| |[|\||else |\textit{main-code}]| \||fi|
\end{center}

%%%%%%%%%%%%%%%%%%%%%%%%%%%%%%%%%%%%%%%%
\DescribeMacro{\childdocname}
\DescribeMacro{\childdocjob}
The macro |\childdocname| contains the filename (without extension)
of the main or child file being processed.
Note that |\childdocjob| will always contain the name of the main file.

%%%%%%%%%%%%%%%%%%%%%%%%%%%%%%%%%%%%%%%%
\paragraph{Title Page.}

Conditional processing can be used to include a title or banner page
in the main document when proper precautions are taken.
Importantly, the code in the main file should ensure that the page counter
(as well as other status parameters which are stored in the |.aux| files)
takes the same value after the conditional processing.
Otherwise the page numbers may take divergent values
depending on which part is compiled.

For example, a title page could be declared by:
%
\begin{center}
\begin{tabular}{l}
|\ifchilddoc\||else|\\
|\addtocounter{page}{-1}|\\
\textit{code for title page}\\
|\newpage|\\
|\||fi|
\end{tabular}
\end{center}
%
A banner page for the child documents can be generated by:
%
\begin{center}
\begin{tabular}{l}
|\ifchilddoc|\\
|\addtocounter{page}{-1}|\\
\textit{code for banner page}\\
|\newpage|\\
|\||fi|
\end{tabular}
\end{center}
%
Here one could write a message such as:
\begin{center}
|This is the part \childdocname{} of \childdocjob{}.|
\end{center}

%%%%%%%%%%%%%%%%%%%%%%%%%%%%%%%%%%%%%%%%%%%%%%%%%%%%%%%%%%%%%%%%%%%%%%%%%%%%%%%%
\subsection{Flags}
\label{sec:flags}

The package makes it easy to generate different versions
of the main or child documents.
To this end compilation flags can be defined
and assigned different default values.
They will be particularly useful in conjunction
with the forwarding mechanism described in \secref{sec:forward}.

For example, it may be useful to have a flag |\version|
which can be set to |draft| or |final|.
The document source will contain some conditional code
depending on the value of |\version|.
Suppose further, the flag should default to |final| for the main file
and to |draft| for child files
which is a natural assignment for editing the document.
This is achieved by placing the following code
in the preamble of the main document
(below the |\childdocmain| directive):
%
\begin{center}
\begin{tabular}{l}
|\ifchilddoc|\\
|\providecommand{\version}{draft}|\\
|\||else|\\
|\providecommand{\version}{final}|\\
|\||fi|
\end{tabular}
\end{center}
%
The definition by |\providecommand| makes sure
that previous definitions are not overwritten.
Further statements |\providecommand{\version}{...}|
can thus be added before the above code to override it.

For the main file, one might add a line
(between |\childdocmain| and the above block)
%
\begin{center}
|%\ifchilddoc\||else\providecommand{\version}{draft}\||fi|
\end{center}
%
which can be uncommented to produce a draft version.
Likewise one can add a line to the very top of a child file
(above the |\childdocof{|\textit{main}|}| directive)
%
\begin{center}
|%\providecommand{\version}{final}|
\end{center}
%
which can be uncommented to produce the final version of this child document.

%%%%%%%%%%%%%%%%%%%%%%%%%%%%%%%%%%%%%%%%%%%%%%%%%%%%%%%%%%%%%%%%%%%%%%%%%%%%%%%%
\subsection{Forwarding}
\label{sec:forward}

Different versions of the main or child documents
using compilation flags as described in \secref{sec:flags}
can be (permanently) stored in different files
for convenient compilation, viewing and distribution.
To this end, the package defines a command
to pass on compilation to a different file:

%%%%%%%%%%%%%%%%%%%%%%%%%%%%%%%%%%%%%%%%
\DescribeMacro{\childdocforward}
The command |\childdocforward| redirects processing to
another source file:
%
\begin{center}
\begin{tabular}{l}
|\input{childdoc.def}|\\
|\childdocforward[|\textit{main}|]{|\textit{dest}|}|\\
\end{tabular}
\end{center}
%
The argument \textit{dest} is the destination file
(without extension).
It should be the main file or one of the child files.
Note that further \textsf{childdoc} directives
such as |\childdocof| and |\childdocforward|
in the indicated file will be processed in this form.
The optional argument \textit{main}
passes on directly to the main file \textit{main}
while pretending to compile the child \textit{dest}.
This form behaves as if \textit{dest}
issues |\childdocof{|\textit{main}|}| right away,
and no further \textsf{childdoc} directives will be processed.

%%%%%%%%%%%%%%%%%%%%%%%%%%%%%%%%%%%%%%%%
\DescribeMacro{\...prefix}
In the alternative form |\childdocforwardprefix|,
%
\begin{center}
\begin{tabular}{l}
|\input{childdoc.def}|\\
|\childdocforwardprefix[|\textit{main}|]{|\textit{prefix}|}{|\textit{dest}|}|
\end{tabular}
\end{center}
%
the destination file is determined by a pattern
depending on the current file:
To make this work, the current file must be called
`{\textit{prefix}\hspace{0.2em}\textit{suffix}}'
with \textit{prefix} matching precisely the argument.
Processing is then passed on to the file
`{\textit{dest}\hspace{0.2em}\textit{suffix}}'.
Surely, the same effect is achieved by
directly specifying the
argument `{\textit{dest}\hspace{0.2em}\textit{suffix}}'
in the first form.
However, that requires to set up a different file
for each child. With the alternative form of the command
all these files can have exactly the same content
which simplifies setting them up and maintaining them.

For example, the following file |draft.tex|
with a compilation flag |\version| as described in \secref{sec:flags}
compiles the main document as a draft:
%
\begin{center}
\begin{tabular}{l}
|\def\version{draft}|\\
|\input{childdoc.def}|\\
|\childdocforward{|\textit{main}|}|
\end{tabular}
\end{center}
%
Likewise, the following files |final|\textit{nn}|.tex|
compile the final version of the child document
|child|\textit{nn}|.tex|:
%
\begin{center}
\begin{tabular}{l}
|\def\version{final}|\\
|\input{childdoc.def}|\\
|\childdocforwardprefix{final}{child}|
\end{tabular}
\end{center}
%

Note that when several versions of a main file and/or of each child file
are to be generated, it may be convenient to set up a |Makefile| or
shell script to automatise the process.

%%%%%%%%%%%%%%%%%%%%%%%%%%%%%%%%%%%%%%%%%%%%%%%%%%%%%%%%%%%%%%%%%%%%%%%%%%%%%%%%
\subsection{Command Line Processing}
\label{sec:commandline}

The effect of redirection files can also be achieved by invoking
the \LaTeX{} compiler with a more elaborate command line.
Most conveniently this should be done as part
of a shell script or a |Makefile|.

When using \textsf{childdoc} in the main file, the following
command lines effectively perform a redirection
(note that depending on the shell being used,
backslashes may have to be doubled: `|\|' $\to$ `|\\|'):
%
\begin{center}
|... -jobname "|\textit{target}|" |\\|"|[\textit{flags}]%
|\input{childdoc.def}\childdocforward[|\textit{main}|]{|\textit{dest}|}"|
\end{center}
%
Here \textit{target} is the name of the output file,
\textit{main} is the name of the main file
and \textit{dest} is the name of the main or child file to be processed
(all filenames without extensions).
The optional argument \textit{main} can be omitted
if \textit{main} matches \textit{dest}.
Optionally, compilation \textit{flags} can be defined via |\def| commands.
This command line makes the \TeX{} engine believe
it is compiling the file \textit{target}
whose content is specified as the latter parameter.
The provided code then forwards the processing to
\textit{main} or \textit{dest} as described in \secref{sec:forward}.

%%%%%%%%%%%%%%%%%%%%%%%%%%%%%%%%%%%%%%%%%%%%%%%%%%%%%%%%%%%%%%%%%%%%%%%%%%%%%%%%
\subsection{Include by Input}
\label{sec:input}

Including child documents by |\include| has some restrictions by design.
Most notably, the content of a child document always occupies
its own set of pages; pages cannot be shared between child documents.
Usually, this behaviour makes perfect sense
because each child document contain an essential part of the document.
However, in some situations it may be desirable to compose
a document from a collection of parts
without having mandatory page breaks between then.
For this case, the package
provides a mechanism to include parts
by |\input| which can also be processed individually.
However, by construction this mechanism
requires manual handling of the content to be output.

%%%%%%%%%%%%%%%%%%%%%%%%%%%%%%%%%%%%%%%%
\DescribeMacro{\ifchilddocmanual}
The main file should be prepared as usual, see \secref{sec:include}.
However, the document body must make a distinction
between processing of an individual part and of the main document, e.g.:
%
\begin{center}
\begin{tabular}{l}
|\ifchilddocmanual|\\
|\input{\childdocname}|\\
|\||else|\\
\textit{document body with }|\input{|\textit{part}|}|\\
|\||fi|
\end{tabular}
\end{center}
%
The conditional |\ifchilddocmanual| is true whenever
a part to be included by |\input| is being compiled,
and the name of the part is stored in |\childdocname|.

%%%%%%%%%%%%%%%%%%%%%%%%%%%%%%%%%%%%%%%%
\DescribeMacro{\childdocby}
Each part to be included by |\input| should start with:
%
\begin{center}
\begin{tabular}{l}
|\input{childdoc.def}|\\
|\childdocby{|\textit{main}|}|\\
\end{tabular}
\end{center}
%
The directive |\childdocby| is similar to |\childdocof|
described in \secref{sec:include},
but the subsequent selection of content must be done manually.
To that end, both |\ifchilddoc| and |\ifchilddocmanual|
will be true upon processing of a part,
and the name of the part is stored in |\childdocname|.
Note that |\jobname| will be set to the filename of the current part
so that each part receives an individual |.aux| file
that does not interfere with the |.aux| file(s) of the main document.
This behaviour can be altered by the alternative form
|\childdocby[*]{|\textit{main}|}| (with a non-empty optional argument)
which uses the |.aux| file of the main document
by setting |\jobname| to \textit{main}.

%%%%%%%%%%%%%%%%%%%%%%%%%%%%%%%%%%%%%%%%%%%%%%%%%%%%%%%%%%%%%%%%%%%%%%%%%%%%%%%%
\subsection{Driver Development}
\label{sec:driver}

The \textsf{childdoc} mechanism can also be use for the development
of definition files such as \LaTeX{} styles or classes.
This case differs from the above setup with multiple parts
included by |\include| in that no |\includeonly| should be invoked.
This can be achieved by starting the include file
(before |\ProvidesPackage|) with:
%
\begin{center}
\begin{tabular}{l}
|\input{childdoc.def}|\\
|\childdocforward{|\textit{main}|}|\\
\end{tabular}
\end{center}
%
or alternatively with:
%
\begin{center}
\begin{tabular}{l}
|\input{childdoc.def}|\\
|\childdocby{|\textit{main}|}|\\
\end{tabular}
\end{center}
%
Both forms have slightly different effects as described above.
The main file is prepared as usual, see \secref{sec:include}.

%%%%%%%%%%%%%%%%%%%%%%%%%%%%%%%%%%%%%%%%%%%%%%%%%%%%%%%%%%%%%%%%%%%%%%%%%%%%%%%%
\subsection{Legacy Detection}
\label{sec:detection}

The directive |\childdocmain| in the main file can detect
whether the complete document or merely a child is to be compiled
even without using the directive |\childdocof|.
This method is deprecated because it is less robust
and there is no compelling reason to use it;
it is merely provided for backward compatibility
and it may be removed in future versions.

If the detection mechanism is to be used,
it is mandatory to correctly specify
the filename of the main file as the argument of |\childdocmain|:
%
\begin{center}
\begin{tabular}{l}
|\input{childdoc.def}|\\
|\childdocmain{|\textit{main}|}|\\
\end{tabular}
\end{center}
%
If |\jobname| does not match the argument \textit{main} of |\childdocmain|,
it is assumed that |\jobname| points to the child file to be compiled.
When using |\childdocmain| with the main file specified as argument,
it suffices to start a child file
with just |\input{|\textit{main}|}|
without loading of the package and using |\childdocof|.
If instead all processing is done
with the appropriate \textsf{childdoc} directives,
the argument of \textit{main} of |\childdocmain| can be empty.

An alternative version of the command line processing described
in \secref{sec:commandline} using the detection mechanism reads:
%
\begin{center}
|... -jobname "|\textit{target}|" "|[\textit{flags}]%
[|\def\jobname{|\textit{dest}|}|]|\input{|\textit{main}|}"|
\end{center}

%%%%%%%%%%%%%%%%%%%%%%%%%%%%%%%%%%%%%%%%%%%%%%%%%%%%%%%%%%%%%%%%%%%%%%%%%%%%%%%%
\subsection{Manual Code}
\label{sec:manual}

In case one cannot be certain whether the definitions file |childdoc.def|
is installed on the target \TeX{} distribution
and one prefers not to ship it,
it is conceivable to paste a few relevant commands into the sources.

To that end, drop all statements |\input{childdoc.def}|
and perform the replacements as outlined below.
Instead of |\childdocmain{|\textit{main}|}| add the following code
to the top of the main file:
%
\begin{center}
\begin{tabular}{l}
|\||ifdefined\childdocname\endinput\||fi\newif\ifchilddoc|\\
|\edef\childdocname{\scantokens\expandafter{\jobname\noexpand}}|\\
|\def\childdocmain{|\textit{main}|}\||ifx\childdocmain\childdocname\||else|\\
|\childdoctrue\includeonly{\childdocname}\let\jobname\childdocmain\||fi|\\
\end{tabular}
\end{center}
%
Instead of |\childdocof{|\textit{main}|}| just include the main file
at the top of each child file:
%
\begin{center}
|\input{|\textit{main}|}|
\end{center}
%
A simple redirection |\childdocforward{|\textit{dest}|}| is achieved by:
%
\begin{center}
|\def\jobname{|\textit{dest}|}\input{\jobname}|
\end{center}
%
The redirection with prefix
|\childdocforwardprefix[|\textit{prefix}|]{|\textit{dest}|}|
is accomplished by:
%
\begin{center}
\begin{tabular}{l}
|{\edef\jobname{\scantokens\expandafter{\jobname\noexpand}}|\\
|\def\redirectjob |\textit{prefix}|#1~~~{\gdef\jobname{|\textit{dest}|#1}}|\\
|\expandafter\redirectjob\jobname~~~}\input{\jobname}|
\end{tabular}
\end{center}

In an alternative approach,
child documents can be compiled by a specific command line
without additional code or specific definitions:
%
\begin{center}
|... -jobname "|\textit{target}|" "|[\textit{flags}]%
|\includeonly{|\textit{dest}|}\input{|\textit{main}|}"|
\end{center}
%

%%%%%%%%%%%%%%%%%%%%%%%%%%%%%%%%%%%%%%%%%%%%%%%%%%%%%%%%%%%%%%%%%%%%%%%%%%%%%%%%
%%%%%%%%%%%%%%%%%%%%%%%%%%%%%%%%%%%%%%%%%%%%%%%%%%%%%%%%%%%%%%%%%%%%%%%%%%%%%%%%
\section{Information}

%%%%%%%%%%%%%%%%%%%%%%%%%%%%%%%%%%%%%%%%%%%%%%%%%%%%%%%%%%%%%%%%%%%%%%%%%%%%%%%%
\subsection{Copyright}

Copyright \copyright{} 2017--2018 Niklas Beisert

This work may be distributed and/or modified under the
conditions of the \LaTeX{} Project Public License, either version 1.3
of this license or (at your option) any later version.
The latest version of this license is in
  \url{http://www.latex-project.org/lppl.txt}
and version 1.3 or later is part of all distributions of \LaTeX{}
version 2005/12/01 or later.

This work has the LPPL maintenance status `maintained'.

The Current Maintainer of this work is Niklas Beisert.

This work consists of the files |README.txt|, |childdoc.ins| and |childdoc.dtx|
as well as the derived files |childdoc.def|, |cdocsamp.tex|
with |cdocsch1.tex|, |cdocsch2.tex|, |cdocspt3.tex|, |cdocspt4.tex|,
|cdocsdrf.tex|, |cdocsfn1.tex|, |cdocsfn2.tex|
as well as |childdoc.pdf|.

%%%%%%%%%%%%%%%%%%%%%%%%%%%%%%%%%%%%%%%%%%%%%%%%%%%%%%%%%%%%%%%%%%%%%%%%%%%%%%%%
\subsection{Files and Installation}

The package consists of the files:
%
\begin{center}
\begin{tabular}{ll}
    |README.txt|   & readme file \\
    |childdoc.ins| & installation file \\
    |childdoc.dtx| & source file \\
    |childdoc.def| & definition file \\
    |cdocsamp.tex| & sample main file \\
    |cdocsch1.tex| & sample include file \\
    |cdocsch2.tex| & sample include file \\
    |cdocspt3.tex| & sample part file \\
    |cdocspt4.tex| & sample part file \\
    |cdocsdrf.tex| & sample redirection file \\
    |cdocsfn1.tex| & sample redirection file \\
    |cdocsfn2.tex| & sample redirection file \\
    |childdoc.pdf| & manual
\end{tabular}
\end{center}
%
The distribution consists of the files
|README.txt|, |childdoc.ins| and |childdoc.dtx|.
%
\begin{itemize}
\item
Run (pdf)\LaTeX{} on |childdoc.dtx|
to compile the manual |childdoc.pdf| (this file).
\item
Run \LaTeX{} on |childdoc.ins| to create the definitions file |childdoc.def|
and the sample |cdocsamp.tex| with include files
|cdocsch1.tex|, |cdocsch2.tex|, |cdocspt3.tex|, |cdocspt4.tex|,
|cdocsdrf.tex|, |cdocsfn1.tex|, |cdocsfn2.tex|.
Then copy the file |childdoc.def| to an appropriate directory of your \LaTeX{}
distribution, e.g.\ \textit{texmf-root}|/tex/latex/childdoc|.
\end{itemize}

%%%%%%%%%%%%%%%%%%%%%%%%%%%%%%%%%%%%%%%%%%%%%%%%%%%%%%%%%%%%%%%%%%%%%%%%%%%%%%%%
\subsection{Related CTAN Packages}

There are several other packages which offer a similar functionality:
%
\begin{itemize}
\item
The packages
\href{http://ctan.org/pkg/docmute}{\textsf{docmute}},
\href{http://ctan.org/pkg/includex}{\textsf{includex}} and
\href{http://ctan.org/pkg/standalone}{\textsf{standalone}}
provide commands to include only the document body of
a child file thus allowing both files to be compiled individually.
\item
The packages \href{http://ctan.org/pkg/subdocs}{\textsf{subdocs}}
and \href{http://ctan.org/pkg/subfiles}{\textsf{subfiles}}
provide structures in which the main and child documents can be
encapsulated and allowing them to be compiled individually.
The inclusion mechanism is different from the conventional |\include|.
\item
The package \href{http://ctan.org/pkg/combine}{\textsf{combine}}
is an elaborate solution to combine several documents into one.
\end{itemize}
%
See also the CTAN topic \href{http://ctan.org/topic/subdocs}{\textsf{subdocs}}
for further related packages.
The present package differs from the above solutions in that
a document structure constructed with the conventional |\include| mechanism
just needs two extra commands at the top of every file
such that all constituent files can be compiled individually.

%%%%%%%%%%%%%%%%%%%%%%%%%%%%%%%%%%%%%%%%%%%%%%%%%%%%%%%%%%%%%%%%%%%%%%%%%%%%%%%%
%\subsection{Feature Suggestions}
%
%The following is a list of features which may be useful for future
%versions of this package:
%%
%\begin{itemize}
%\item
%\ldots
%\end{itemize}

%%%%%%%%%%%%%%%%%%%%%%%%%%%%%%%%%%%%%%%%%%%%%%%%%%%%%%%%%%%%%%%%%%%%%%%%%%%%%%%%
\subsection{Revision History}

%%%%%%%%%%%%%%%%%%%%%%%%%%%%%%%%%%%%%%%%
\paragraph{v2.0:} 2018/12/30

\begin{itemize}
\item
immediate forward processing
\item
added |\childdocby| mechanism
\item
manual restructured
\end{itemize}

%%%%%%%%%%%%%%%%%%%%%%%%%%%%%%%%%%%%%%%%
\paragraph{v1.6:} 2018/01/17

\begin{itemize}
\item
application for development of include files
\item
corrections to manual
\end{itemize}

%%%%%%%%%%%%%%%%%%%%%%%%%%%%%%%%%%%%%%%%
\paragraph{v1.5:} 2017/05/21

\begin{itemize}
\item
more complete structuring introduced
\item
|\childdocof| introduced
\item
|\childdoc| renamed to |\childdocmain|
\item
|\childredirect| renamed to |\childdocforward| and |\childdocforwardprefix|
and functionality expanded
\end{itemize}

%%%%%%%%%%%%%%%%%%%%%%%%%%%%%%%%%%%%%%%%
\paragraph{v1.0:} 2017/04/27

\begin{itemize}
\item
manual and install package
\item
first version published on CTAN
\end{itemize}

%%%%%%%%%%%%%%%%%%%%%%%%%%%%%%%%%%%%%%%%
\paragraph{v0.6:} 2017/04/26

\begin{itemize}
\item
redirection mechanism added
\end{itemize}

%%%%%%%%%%%%%%%%%%%%%%%%%%%%%%%%%%%%%%%%
\paragraph{v0.5:} 2017/04/26

\begin{itemize}
\item
functionality in definition file
\end{itemize}


%%%%%%%%%%%%%%%%%%%%%%%%%%%%%%%%%%%%%%%%%%%%%%%%%%%%%%%%%%%%%%%%%%%%%%%%%%%%%%%%
%%%%%%%%%%%%%%%%%%%%%%%%%%%%%%%%%%%%%%%%%%%%%%%%%%%%%%%%%%%%%%%%%%%%%%%%%%%%%%%%
%%%%%%%%%%%%%%%%%%%%%%%%%%%%%%%%%%%%%%%%%%%%%%%%%%%%%%%%%%%%%%%%%%%%%%%%%%%%%%%%
\appendix

\settowidth\MacroIndent{\rmfamily\scriptsize 000\ }

 \DocInput{childdoc.dtx}

\end{document}
%</driver>
% \fi
%
% %%%%%%%%%%%%%%%%%%%%%%%%%%%%%%%%%%%%%%%%%%%%%%%%%%%%%%%%%%%%%%%%%%%%%%%%%%%%%%
% %%%%%%%%%%%%%%%%%%%%%%%%%%%%%%%%%%%%%%%%%%%%%%%%%%%%%%%%%%%%%%%%%%%%%%%%%%%%%%
% \section{Sample}
%\iffalse
%<*samplemain>
%\fi
%
% The following presents a sample document
% with two chapters, two parts, a title page,
% a compile flag as well as three forwarding files to set the flag.
% It consists of eight |.tex| files:
% \begin{center}
% \begin{tabular}{ll}
% |cdocsamp.tex|&main file\\
% |cdocsch1.tex|&include file for chapter 1\\
% |cdocsch2.tex|&include file for chapter 2\\
% |cdocspt3.tex|&include file for part 3\\
% |cdocspt4.tex|&include file for part 4\\
% |cdocsdrf.tex|&forwarding file for main file in draft mode\\
% |cdocsfi1.tex|&forwarding file for final version of chapter 1\\
% |cdocsfi2.tex|&forwarding file for final version of chapter 2\\
% \end{tabular}
% \end{center}
% Each of the eight files can be compiled directly by the \LaTeX{} compiler.
%
% %%%%%%%%%%%%%%%%%%%%%%%%%%%%%%%%%%%%%%
% \paragraph{Main File.}
%
% The main file is called |cdocsamp.tex|.
%
% Load the \textsf{childdoc} definitions and
% declare the filename for the main document:
%    \begin{macrocode}
\input{childdoc.def}
\childdocmain{}
%    \end{macrocode}

% Optional override for |\version| flag:
%    \begin{macrocode}
%%\ifchilddoc\else\providecommand{\version}{draft}\fi
%    \end{macrocode}

% Define the default values for the |\version| flag
% (|final| for the main file and |draft| for childs):
%    \begin{macrocode}
\ifchilddoc
\providecommand{\version}{draft}
\else
\providecommand{\version}{final}
\fi
%    \end{macrocode}

% Load the standard document class:
%    \begin{macrocode}
\documentclass[12pt]{article}
%    \end{macrocode}

% Start the document body:
%    \begin{macrocode}
\begin{document}
%    \end{macrocode}

% Declare a title page.
% Print title, part of document being processed and version flag:
%    \begin{macrocode}
\addtocounter{page}{-1}
\begin{center}
{\LARGE\bfseries{}childdoc example\par}
\vspace{1cm}
\ifchilddoc
\ifchilddocmanual part\else chapter\fi:
`\childdocname' of `\childdocjob'\par
\else
main document: `\childdocjob'\par
\fi
version: \version\par
\end{center}
\newpage
%    \end{macrocode}

% Manually include selected file,
% otherwise process as usual:
%    \begin{macrocode}
\ifchilddocmanual
\section*{part `\childdocname'}
\input{\childdocname}
\else
%    \end{macrocode}

% Include the two chapters:
%    \begin{macrocode}
\include{cdocsch1}
\include{cdocsch2}
%    \end{macrocode}

% Include the two parts unless only chapters should be displayed:
%    \begin{macrocode}
\ifchilddoc\else
\section{part three}
\input{cdocspt3}
\section{part four}
\input{cdocspt4}
\fi
%    \end{macrocode}

% Process as usual until here:
%    \begin{macrocode}
\fi
%    \end{macrocode}

% End of document body:
%    \begin{macrocode}
\end{document}
%    \end{macrocode}
%\iffalse
%</samplemain>
%\fi
%
% %%%%%%%%%%%%%%%%%%%%%%%%%%%%%%%%%%%%%%
% \paragraph{Chapter Include Files.}
%
% The include files are called |cdocsch1.tex| and |cdocsch2.tex|.
%
%\iffalse
%<*samplechap1|samplechap2>
%\fi

% Optional override for |\version| flag:
%    \begin{macrocode}
%%\providecommand{\version}{final}
%    \end{macrocode}

% Include the main document:
%    \begin{macrocode}
\input{childdoc.def}
\childdocof{cdocsamp}
%    \end{macrocode}

%\iffalse
%</samplechap1|samplechap2>
%\fi
%
%\iffalse
%<*samplechap1>
%\fi
% Some text for chapter 1:
%    \begin{macrocode}
\section{one}
some text in chapter one
%    \end{macrocode}

%\iffalse
%</samplechap1>
%\fi
% Some text for chapter 2:
%\iffalse
%<*samplechap2>
%\fi
%    \begin{macrocode}
\section{two}
more text in chapter two
%    \end{macrocode}

%\iffalse
%</samplechap2>
%\fi
%
% %%%%%%%%%%%%%%%%%%%%%%%%%%%%%%%%%%%%%%
% \paragraph{Part Include Files.}
%
% The include files are called |cdocspt3.tex| and |cdocspt4.tex|.
%
%\iffalse
%<*samplepart3|samplepart4>
%\fi

% Optional override for |\version| flag:
%    \begin{macrocode}
%%\providecommand{\version}{final}
%    \end{macrocode}

% Include the main document:
%    \begin{macrocode}
\input{childdoc.def}
\childdocby{cdocsamp}
%    \end{macrocode}

%\iffalse
%</samplepart3|samplepart4>
%\fi
%
%\iffalse
%<*samplepart3>
%\fi
% Some text for part 3:
%    \begin{macrocode}
some text in part three
%    \end{macrocode}

%\iffalse
%</samplepart3>
%\fi
% Some text for part 4:
%\iffalse
%<*samplepart4>
%\fi
%    \begin{macrocode}
more text in part four
%    \end{macrocode}

%\iffalse
%</samplepart4>
%\fi
%
% %%%%%%%%%%%%%%%%%%%%%%%%%%%%%%%%%%%%%%
% \paragraph{Forwarding for a Complete Draft.}
%
% The following forwarding file |cdocsdrf.tex|
% compiles the main document in draft mode:
%\iffalse
%<*sampledraft>
%\fi
%    \begin{macrocode}
\def\version{draft}
\input{childdoc.def}
\childdocforward{cdocsamp}
%    \end{macrocode}

%\iffalse
%</sampledraft>
%\fi
%
% %%%%%%%%%%%%%%%%%%%%%%%%%%%%%%%%%%%%%%
% \paragraph{Forwarding for Final Version of the Chapters.}
%
% The following forwarding files |cdocsfn1.tex| and |cdocsfn2.tex|
% (with identical content)
% compile the final versions of the child documents
% |cdocsch1.tex| and |cdocsch2.tex|, respectively:
%\iffalse
%<*samplefinal>
%\fi
%    \begin{macrocode}
\def\version{final}
\input{childdoc.def}
\childdocforwardprefix[cdocsamp]{cdocsfn}{cdocsch}
%    \end{macrocode}

%\iffalse
%</samplefinal>
%\fi
%
% %%%%%%%%%%%%%%%%%%%%%%%%%%%%%%%%%%%%%%
% \paragraph{Command Line Processing.}
%
% The following three command lines generate the output files
% |cdocscld|, |cdocscl1| and |cdocscl2|
% which should be identical to
% |cdocsdrf|, |cdocsch1| and |cdocsfn2|, respectively:
% \begin{center}
% \begin{tabular}{l}
% |latex -jobname cdocscld \|\\
% |  "\def\version{draft}\input{childdoc.def}\childdocforward{cdocsamp}"|\\
% |latex -jobname cdocscl1 \|\\
% |  "\input{childdoc.def}\childdocforward[cdocsamp]{cdocsch1}"|\\
% |latex -jobname cdocscl2 \|\\
% |  "\def\version{final}\input{childdoc.def}\childdocforward{cdocsch2}"|
% \end{tabular}
% \end{center}
% Note that the trailing backslash on each first line
% merely continues the input to the second line
% (for convenient cut ant paste).
% Furthermore, the command |latex| can be replaced by any
% of its alternative versions such as |pdflatex|.
%
% %%%%%%%%%%%%%%%%%%%%%%%%%%%%%%%%%%%%%%%%%%%%%%%%%%%%%%%%%%%%%%%%%%%%%%%%%%%%%%
% %%%%%%%%%%%%%%%%%%%%%%%%%%%%%%%%%%%%%%%%%%%%%%%%%%%%%%%%%%%%%%%%%%%%%%%%%%%%%%
% \section{Implementation}
%\iffalse
%<*package>
%\fi
%
% This section describes the definitions file |childdoc.def|.

% The definitions cannot be loaded using |\usepackage| or |\RequirePackage|
% which has a mechanism to prevent loading a style file more than once.
% When loading the definitions by means of |\input|
% multiple instances have to be prevented manually:
%\iffalse
%This code needs to be before the `\ProvidesFile' directive
%which is defined at the beginning of this file.
%Therefore it is also placed there and commented out here.
%</package>
%<*discard>
%\fi
%    \begin{macrocode}
\ifdefined\childdocmain\endinput\fi
%    \end{macrocode}
%\iffalse
%</discard>
%<*package>
%\fi
%
% \macro{\ifchilddoc}
% \macro{\ifchilddocmanual}
% The conditional |\ifchilddoc| tells whether a
% child (true) or main (false) document is being compiled.
% The conditional |\ifchilddocmanual| tells whether
% the |\includeonly| mechanism is used (false) or
% the selection of child files must be performed manually (true).
% The definitions initialise to false:
%    \begin{macrocode}
\newif\ifchilddoc
\newif\ifchilddocmanual
%    \end{macrocode}

% \macro{\childdocname}
% \macro{\childdocjob}
% The macro |\childdocname| stores the name of the main document
% to be compiled. The macro |\childdocjob| stores the name of
% the document on which the \LaTeX{} compiler was originally invoked.
% The content of |\jobname| cannot be compared
% to filenames specified in the source due to different catcodes.
% The following code rescans |\jobname|, stores the result
% in |\childdocname| and saves a copy in |\childdocjob|:
%    \begin{macrocode}
\edef\childdocname{\scantokens\expandafter{\jobname\noexpand}}
\let\childdocjob\childdocname
%    \end{macrocode}

% \macro{\childdocdisable}
% The macro |\childdocdisable| prevents the main file
% from being processed more than once.
% At this stage, the main document command |\childdocmain|
% is assumed to be called once again where it should do nothing.
% Any subsequent call to it should prevent
% a secondary processing of the main document
% It overwrites the forwarding commands
% |\childdocof| and |\childdocforward|
% with empty macros to prevent further inclusions of the main document:
%    \begin{macrocode}
\newcommand{\childdocdisable}
{
  \renewcommand{\childdocmain}[1]{\renewcommand{\childdocmain}[1]{\endinput}}
  \renewcommand{\childdocof}[1]{}
  \renewcommand{\childdocby}[2][]{}
  \renewcommand{\childdocforward}[2][]{}
  \renewcommand{\childdocdisable}{}
}
%    \end{macrocode}

% \macro{\childdocmain}
% The macro |\childdocmain| is to be called at the top of the main file
% with nothing or the main filename (without extension) as argument.
% First, it breaks loops.
% If the argument is not empty and does not match |\childdocname|
% (which is set by the first inclusion of |childdoc.def|),
% |\ifchilddoc| is set to true, |\includeonly| is applied to the child file
% and |\jobname| is set to the main file
% (for proper handling of |.aux| files):
%    \begin{macrocode}
\newcommand{\childdocmain}[1]
{
  \childdocdisable\childdocmain{}
  \if?#1?\else
    \begingroup
      \def\childdoctmp{#1}
      \ifx\childdoctmp\childdocname
        \def\childdoctmp{}
      \else
        \def\childdoctmp
        {
          \childdoctrue
          \includeonly{\childdocname}
          \def\childdocjob{#1}
          \def\jobname{#1}
        }
      \fi
      \expandafter
    \endgroup
    \childdoctmp
  \fi
}
%    \end{macrocode}

% \macro{\childdocof}
% The command |\childdocof| redirects
% compilation to the main file |#1|.
%    \begin{macrocode}
\newcommand{\childdocof}[1]
{
  \childdocdisable
  \childdoctrue
  \includeonly{\childdocname}
  \def\jobname{#1}
  \def\childdocjob{#1}
  \input{#1}
}
%    \end{macrocode}

% \macro{\childdocby}
% The command |\childdocby| ....
%    \begin{macrocode}
\newcommand{\childdocby}[2][]
{
  \childdocdisable
  \childdoctrue
  \childdocmanualtrue
  \if?#1?\else
    \def\jobname{#2}
  \fi
  \def\childdocjob{#2}
  \input{#2}
  \endinput
}
%    \end{macrocode}

% \macro{\childdocforward}
% The command |\childdocforward| redirects
% compilation to the main file or
% (if the optional argument is given) a child file.
% Parameters are set as if the main file
% or a child file starting with |\childdocof| was compiled.
% Then compilation is handed over to the main file:
%    \begin{macrocode}
\newcommand{\childdocforward}[2][]
{
  \begingroup
    \if?#1?
      \def\childdoctmp
      {
        \def\childdocname{#2}
        \def\childdocjob{#2}
        \def\jobname{#2}
        \input{#2}
        \endinput
      }
    \else
      \def\childdoctmp
      {
        \childdocdisable
        \def\childdocname{#2}
        \childdoctrue
        \includeonly{#2}
        \def\childdocjob{#1}
        \def\jobname{#1}
        \input{#1}
        \endinput
      }
    \fi
    \expandafter
  \endgroup
  \childdoctmp
}
%    \end{macrocode}

% \macro{\childdocforwardprefix}
% The command |\childdocforwardprefix| redirects
% compilation to the main or a child file by means of a pattern.
% The prefix |#1| in the current filename is replaced by |#2|
% and the suffix of the current filename is kept
% (it is assumed that the filename does not contain the substring `|~~~|'
% which is used as a delimiter).
% Compilation is handed over to the new file by |\childdocforward|:
%    \begin{macrocode}
\newcommand{\childdocforwardprefix}[3][]
{
  \begingroup
    \def\childdocextract #2##1~~~{\def\childdoctmp{\childdocforward[#1]{#3##1}}}
    \expandafter\childdocextract\childdocname~~~
    \expandafter
  \endgroup
  \childdoctmp
}
%    \end{macrocode}

% \macro{\childdoc}
% The deprecated macro |\childdoc| is a legacy version of |\childdocmain|:
%    \begin{macrocode}
\newcommand{\childdoc}{\childdocmain}
%    \end{macrocode}

% \macro{\childdocredirect}
% The deprecated macro |\childdocredirect| is a legacy version
% of |\childdocforward| and |\childdocforwardprefix|:
%    \begin{macrocode}
\newcommand{\childdocredirect}[2][]
{
  \begingroup
    \if?#1?
      \def\childdoctmp{\childdocforward{#2}}
    \else
      \def\childdoctmp{\childdocforwardprefix{#1}{#2}}
    \fi
    \expandafter
  \endgroup
  \childdoctmp
}
%    \end{macrocode}

%\iffalse
%</package>
%\fi
%
\endinput
|\\
|\childdocmain{}|\\
\end{tabular}
\end{center}
at the very top of the main \LaTeX{} file,
in particular \emph{before} the |\documentclass| statement!
The argument of |\childdocmain| should be left empty
(but it must be present).

%%%%%%%%%%%%%%%%%%%%%%%%%%%%%%%%%%%%%%%%
\DescribeMacro{\childdocof}
Furthermore, add the commands
\begin{center}
\begin{tabular}{l}
|% \iffalse
%
% childdoc.dtx Copyright (C) 2017-2018 Niklas Beisert
%
% This work may be distributed and/or modified under the
% conditions of the LaTeX Project Public License, either version 1.3
% of this license or (at your option) any later version.
% The latest version of this license is in
%   http://www.latex-project.org/lppl.txt
% and version 1.3 or later is part of all distributions of LaTeX
% version 2005/12/01 or later.
%
% This work has the LPPL maintenance status `maintained'.
%
% The Current Maintainer of this work is Niklas Beisert.
%
% This work consists of the files childdoc.dtx and childdoc.ins
% and the derived files childdoc.def and cdocsamp.tex with
% cdocsch1.tex, cdocsch2.tex, cdocsdrf.tex, cdocsfn1.tex, cdocsfn2.tex.
%
%<package>\ifdefined\childdocmain\endinput\fi
%<package>\ProvidesFile{childdoc.def}[2018/12/30 v2.0 child document driver]
%<samplemain>\ProvidesFile{cdocsamp.tex}[2018/12/30 v2.0 sample for childdoc]
%<*driver>
%\ProvidesFile{childdoc.drv}[2018/12/30 v2.0 childdoc reference manual file]
\PassOptionsToClass{10pt,a4paper}{article}
\documentclass{ltxdoc}

\usepackage[margin=35mm]{geometry}
\usepackage{hyperref}
\usepackage{hyperxmp}
\usepackage[usenames]{color}

\hypersetup{colorlinks=true}
\hypersetup{pdfstartview=FitH}
\hypersetup{pdfpagemode=UseNone}
\hypersetup{pdfsource={}}
\hypersetup{pdflang={en-UK}}
\hypersetup{pdfcopyright={Copyright 2017-2018 Niklas Beisert.
  This work may be distributed and/or modified under the
  conditions of the LaTeX Project Public License, either version 1.3
  of this license or (at your option) any later version.}}
\hypersetup{pdflicenseurl={http://www.latex-project.org/lppl.txt}}
\hypersetup{pdfcontactaddress={ETH Zurich, ITP, HIT K,
  Wolfgang-Pauli-Strasse 27}}
\hypersetup{pdfcontactpostcode={8093}}
\hypersetup{pdfcontactcity={Zurich}}
\hypersetup{pdfcontactcountry={Switzerland}}
\hypersetup{pdfcontactemail={nbeisert@itp.phys.ethz.ch}}
\hypersetup{pdfcontacturl={http://people.phys.ethz.ch/\xmptilde nbeisert/}}

\newcommand{\secref}[1]{\hyperref[#1]{section \ref*{#1}}}

\parskip1ex
\parindent0pt
\let\olditemize\itemize
\def\itemize{\olditemize\parskip0pt}

\begin{document}

\title{The \textsf{childdoc} Package}
\hypersetup{pdftitle={The childdoc Package}}
\author{Niklas Beisert\\[2ex]
  Institut f\"ur Theoretische Physik\\
  Eidgen\"ossische Technische Hochschule Z\"urich\\
  Wolfgang-Pauli-Strasse 27, 8093 Z\"urich, Switzerland\\[1ex]
  \href{mailto:nbeisert@itp.phys.ethz.ch}
  {\texttt{nbeisert@itp.phys.ethz.ch}}}
\hypersetup{pdfauthor={Niklas Beisert}}
\hypersetup{pdfsubject={Manual for the LaTeX2e Package childdoc}}
\date{30 December 2018, \textsf{v2.0}}
\maketitle

\begin{abstract}\noindent
\textsf{childdoc} is a \LaTeXe{} package
that enables the direct compilation
of document sections included by |\include|
to individual files.
\end{abstract}

\begingroup
\parskip0ex
\tableofcontents
\endgroup

%%%%%%%%%%%%%%%%%%%%%%%%%%%%%%%%%%%%%%%%%%%%%%%%%%%%%%%%%%%%%%%%%%%%%%%%%%%%%%%%
%%%%%%%%%%%%%%%%%%%%%%%%%%%%%%%%%%%%%%%%%%%%%%%%%%%%%%%%%%%%%%%%%%%%%%%%%%%%%%%%
\section{Introduction}

\LaTeX{} provides a mechanism to structure a large document (such as a book)
into a main file and several child files (containing the chapters)
using the |\include| command.
This mechanism is beneficial for documents
which span hundreds of pages in order to
make the source file(s) more manageable.
Moreover, compilation can be restricted to
selected child files by means of the |\includeonly| command.
The latter feature can be used to reduce the compilation time while editing
(this was significantly more useful in the earlier days of \LaTeX{})
or to generate a smaller document which is easier to navigate.
Another application of |\includeonly| is to generate
documents consisting of selected parts of the complete document.

However, there are a few drawbacks of the plain |\include| mechanism:
\begin{itemize}
\item
The child files cannot be compiled on their own,
they can only be compiled via the main file.
A naive editing environment
(such as a text editor with an option
to have the current file processed by \LaTeX)
may require one to switch to the main file before compiling;
attempting to compile the child file produces errors.
\item
The main file must be modified (each time)
to adjust the |\includeonly| command
to the present needs. This easily leaves the main file in a messy state.
\item
The generated document will always carry the filename
of the main document. This is inconvenient if
several child files are to be compiled and
to be kept for distribution.
\end{itemize}

The present package provides a simple interface
to make child files individually compilable by \LaTeX{}.
Compiling a child file then has the same effect as compiling
the main file with an |\includeonly| command
to select the appropriate child.
Moreover the generated document will carry the name of the child
rather than the main file.
This resolves all three above issues.

This feature is meant to make the editing of books,
thesis documents and lecture notes somewhat more convenient.
However, the package can also be used efficiently for
composing a series of documents (such as exercise sheets)
which are typically distributed individually.
It then assists the author in generating the individual documents
(potentially in different versions)
as well as a document containing the collected series.
Another application is in developing style files
or other kinds of included material
where compilation of the style file could redirect
to a sample or test file.

%%%%%%%%%%%%%%%%%%%%%%%%%%%%%%%%%%%%%%%%%%%%%%%%%%%%%%%%%%%%%%%%%%%%%%%%%%%%%%%%
%%%%%%%%%%%%%%%%%%%%%%%%%%%%%%%%%%%%%%%%%%%%%%%%%%%%%%%%%%%%%%%%%%%%%%%%%%%%%%%%
\section{Usage}

First of all, the package \textsf{childdoc} is \emph{not} a standard
\LaTeXe{} |.sty| style file! Therefore it needs to be invoked in
a non-standard way.

%%%%%%%%%%%%%%%%%%%%%%%%%%%%%%%%%%%%%%%%%%%%%%%%%%%%%%%%%%%%%%%%%%%%%%%%%%%%%%%%
\subsection{Included Files}
\label{sec:include}

%%%%%%%%%%%%%%%%%%%%%%%%%%%%%%%%%%%%%%%%
\DescribeMacro{\childdocmain}
To use the package, add the commands
\begin{center}
\begin{tabular}{l}
|\input{childdoc.def}|\\
|\childdocmain{}|\\
\end{tabular}
\end{center}
at the very top of the main \LaTeX{} file,
in particular \emph{before} the |\documentclass| statement!
The argument of |\childdocmain| should be left empty
(but it must be present).

%%%%%%%%%%%%%%%%%%%%%%%%%%%%%%%%%%%%%%%%
\DescribeMacro{\childdocof}
Furthermore, add the commands
\begin{center}
\begin{tabular}{l}
|\input{childdoc.def}|\\
|\childdocof{|\textit{main}|}|\\
\end{tabular}
\end{center}
at the top of every child file \textit{child}
which is included by |\include{|\textit{child}|}|
from within the main file
(or at least for those files to be compiled individually).
The argument \textit{main} must be the filename of the main file.

There are a couple of
considerations in setting up the main and child documents:

%%%%%%%%%%%%%%%%%%%%%%%%%%%%%%%%%%%%%%%%
\paragraph{Restrictions.}

Please note the following restrictions:
\begin{itemize}
\item
|\childdocmain| must be called with one argument \textit{main}
to ensure compatibility with earlier version of the package.
It must either be empty (|\childdocmain{}|)
or precisely match the filename of the main file in which it is specified.
See \secref{sec:detection} for further information.
\item
The filename \textit{main} must be specified without the |.tex| extension.
\item
The filename \textit{main} is case sensitive
(even in case-insensitive file systems)
due to internal string comparison.
\item
The argument \textit{main} should be fully expanded, it cannot be a macro.
\item
Subdirectories and special characters should be avoided in filenames.
\item
The command |\childdocmain{|\textit{main}|}| must be followed by a whitespace.
It should not be followed immediately by another command
or by a comment mark `|%|'.
This is because the \TeX{} parser reads the token immediately following
the argument of |\childdocmain| and puts it
at the beginning of every child section;
however, a white\-space is ignored.
\end{itemize}

%%%%%%%%%%%%%%%%%%%%%%%%%%%%%%%%%%%%%%%%
\paragraph{Content of Main File.}

It is advisable to place all content in the child files included by |\include|.
Any output contained in the main file will appear in all child documents
unless suppressed manually;
it cannot be suppressed automatically by the |\includeonly| directive
and thus should normally be avoided.
A method to include some content in the main file
by means of conditional processing is described in \secref{sec:conditional}.

%%%%%%%%%%%%%%%%%%%%%%%%%%%%%%%%%%%%%%%%
\paragraph{Page Numbering.}

When only a part of the document is compiled,
the appropriate numbering of pages
(as well as other status parameters)
is determined from the |.aux| files.
The latter contain information from previous passes.
However this information needs to propagate through
all intermediate child documents.
Therefore the page numbering in child documents may well
be inconsistent until the complete document is compiled at least once.

A useful (if unconventional) way to always ensure a consistent
page numbering is to restart the numbering in each child document
and denote the pages by `\textit{child}|.|\textit{page}'
where \textit{child} represents the chapter/section number of the child file.
This can be achieved by the command
|\numberwithin{page}{|\textit{child}|}|
of the \textsf{amsmath} package
where \textit{child} can be |chapter| or |section|
depending on the chosen structuring.
Alternatively, one can modify the macro |\thepage| appropriately
and reset the counter |page| at the start of each child file.

%%%%%%%%%%%%%%%%%%%%%%%%%%%%%%%%%%%%%%%%%%%%%%%%%%%%%%%%%%%%%%%%%%%%%%%%%%%%%%%%
\subsection{Conditional Processing}
\label{sec:conditional}

The package provides a mechanism to compile different versions
of a document. To customise the versions further some conditional processing
can come in handy to distinguish which version is being compiled.
The package provides two macros to describe the compilation context:

%%%%%%%%%%%%%%%%%%%%%%%%%%%%%%%%%%%%%%%%
\DescribeMacro{\ifchilddoc}
The conditional |\ifchilddoc| distinguishes between the compilation of
child documents and the main document:
%
\begin{center}
|\ifchilddoc |\textit{child-code}| |[|\||else |\textit{main-code}]| \||fi|
\end{center}

%%%%%%%%%%%%%%%%%%%%%%%%%%%%%%%%%%%%%%%%
\DescribeMacro{\childdocname}
\DescribeMacro{\childdocjob}
The macro |\childdocname| contains the filename (without extension)
of the main or child file being processed.
Note that |\childdocjob| will always contain the name of the main file.

%%%%%%%%%%%%%%%%%%%%%%%%%%%%%%%%%%%%%%%%
\paragraph{Title Page.}

Conditional processing can be used to include a title or banner page
in the main document when proper precautions are taken.
Importantly, the code in the main file should ensure that the page counter
(as well as other status parameters which are stored in the |.aux| files)
takes the same value after the conditional processing.
Otherwise the page numbers may take divergent values
depending on which part is compiled.

For example, a title page could be declared by:
%
\begin{center}
\begin{tabular}{l}
|\ifchilddoc\||else|\\
|\addtocounter{page}{-1}|\\
\textit{code for title page}\\
|\newpage|\\
|\||fi|
\end{tabular}
\end{center}
%
A banner page for the child documents can be generated by:
%
\begin{center}
\begin{tabular}{l}
|\ifchilddoc|\\
|\addtocounter{page}{-1}|\\
\textit{code for banner page}\\
|\newpage|\\
|\||fi|
\end{tabular}
\end{center}
%
Here one could write a message such as:
\begin{center}
|This is the part \childdocname{} of \childdocjob{}.|
\end{center}

%%%%%%%%%%%%%%%%%%%%%%%%%%%%%%%%%%%%%%%%%%%%%%%%%%%%%%%%%%%%%%%%%%%%%%%%%%%%%%%%
\subsection{Flags}
\label{sec:flags}

The package makes it easy to generate different versions
of the main or child documents.
To this end compilation flags can be defined
and assigned different default values.
They will be particularly useful in conjunction
with the forwarding mechanism described in \secref{sec:forward}.

For example, it may be useful to have a flag |\version|
which can be set to |draft| or |final|.
The document source will contain some conditional code
depending on the value of |\version|.
Suppose further, the flag should default to |final| for the main file
and to |draft| for child files
which is a natural assignment for editing the document.
This is achieved by placing the following code
in the preamble of the main document
(below the |\childdocmain| directive):
%
\begin{center}
\begin{tabular}{l}
|\ifchilddoc|\\
|\providecommand{\version}{draft}|\\
|\||else|\\
|\providecommand{\version}{final}|\\
|\||fi|
\end{tabular}
\end{center}
%
The definition by |\providecommand| makes sure
that previous definitions are not overwritten.
Further statements |\providecommand{\version}{...}|
can thus be added before the above code to override it.

For the main file, one might add a line
(between |\childdocmain| and the above block)
%
\begin{center}
|%\ifchilddoc\||else\providecommand{\version}{draft}\||fi|
\end{center}
%
which can be uncommented to produce a draft version.
Likewise one can add a line to the very top of a child file
(above the |\childdocof{|\textit{main}|}| directive)
%
\begin{center}
|%\providecommand{\version}{final}|
\end{center}
%
which can be uncommented to produce the final version of this child document.

%%%%%%%%%%%%%%%%%%%%%%%%%%%%%%%%%%%%%%%%%%%%%%%%%%%%%%%%%%%%%%%%%%%%%%%%%%%%%%%%
\subsection{Forwarding}
\label{sec:forward}

Different versions of the main or child documents
using compilation flags as described in \secref{sec:flags}
can be (permanently) stored in different files
for convenient compilation, viewing and distribution.
To this end, the package defines a command
to pass on compilation to a different file:

%%%%%%%%%%%%%%%%%%%%%%%%%%%%%%%%%%%%%%%%
\DescribeMacro{\childdocforward}
The command |\childdocforward| redirects processing to
another source file:
%
\begin{center}
\begin{tabular}{l}
|\input{childdoc.def}|\\
|\childdocforward[|\textit{main}|]{|\textit{dest}|}|\\
\end{tabular}
\end{center}
%
The argument \textit{dest} is the destination file
(without extension).
It should be the main file or one of the child files.
Note that further \textsf{childdoc} directives
such as |\childdocof| and |\childdocforward|
in the indicated file will be processed in this form.
The optional argument \textit{main}
passes on directly to the main file \textit{main}
while pretending to compile the child \textit{dest}.
This form behaves as if \textit{dest}
issues |\childdocof{|\textit{main}|}| right away,
and no further \textsf{childdoc} directives will be processed.

%%%%%%%%%%%%%%%%%%%%%%%%%%%%%%%%%%%%%%%%
\DescribeMacro{\...prefix}
In the alternative form |\childdocforwardprefix|,
%
\begin{center}
\begin{tabular}{l}
|\input{childdoc.def}|\\
|\childdocforwardprefix[|\textit{main}|]{|\textit{prefix}|}{|\textit{dest}|}|
\end{tabular}
\end{center}
%
the destination file is determined by a pattern
depending on the current file:
To make this work, the current file must be called
`{\textit{prefix}\hspace{0.2em}\textit{suffix}}'
with \textit{prefix} matching precisely the argument.
Processing is then passed on to the file
`{\textit{dest}\hspace{0.2em}\textit{suffix}}'.
Surely, the same effect is achieved by
directly specifying the
argument `{\textit{dest}\hspace{0.2em}\textit{suffix}}'
in the first form.
However, that requires to set up a different file
for each child. With the alternative form of the command
all these files can have exactly the same content
which simplifies setting them up and maintaining them.

For example, the following file |draft.tex|
with a compilation flag |\version| as described in \secref{sec:flags}
compiles the main document as a draft:
%
\begin{center}
\begin{tabular}{l}
|\def\version{draft}|\\
|\input{childdoc.def}|\\
|\childdocforward{|\textit{main}|}|
\end{tabular}
\end{center}
%
Likewise, the following files |final|\textit{nn}|.tex|
compile the final version of the child document
|child|\textit{nn}|.tex|:
%
\begin{center}
\begin{tabular}{l}
|\def\version{final}|\\
|\input{childdoc.def}|\\
|\childdocforwardprefix{final}{child}|
\end{tabular}
\end{center}
%

Note that when several versions of a main file and/or of each child file
are to be generated, it may be convenient to set up a |Makefile| or
shell script to automatise the process.

%%%%%%%%%%%%%%%%%%%%%%%%%%%%%%%%%%%%%%%%%%%%%%%%%%%%%%%%%%%%%%%%%%%%%%%%%%%%%%%%
\subsection{Command Line Processing}
\label{sec:commandline}

The effect of redirection files can also be achieved by invoking
the \LaTeX{} compiler with a more elaborate command line.
Most conveniently this should be done as part
of a shell script or a |Makefile|.

When using \textsf{childdoc} in the main file, the following
command lines effectively perform a redirection
(note that depending on the shell being used,
backslashes may have to be doubled: `|\|' $\to$ `|\\|'):
%
\begin{center}
|... -jobname "|\textit{target}|" |\\|"|[\textit{flags}]%
|\input{childdoc.def}\childdocforward[|\textit{main}|]{|\textit{dest}|}"|
\end{center}
%
Here \textit{target} is the name of the output file,
\textit{main} is the name of the main file
and \textit{dest} is the name of the main or child file to be processed
(all filenames without extensions).
The optional argument \textit{main} can be omitted
if \textit{main} matches \textit{dest}.
Optionally, compilation \textit{flags} can be defined via |\def| commands.
This command line makes the \TeX{} engine believe
it is compiling the file \textit{target}
whose content is specified as the latter parameter.
The provided code then forwards the processing to
\textit{main} or \textit{dest} as described in \secref{sec:forward}.

%%%%%%%%%%%%%%%%%%%%%%%%%%%%%%%%%%%%%%%%%%%%%%%%%%%%%%%%%%%%%%%%%%%%%%%%%%%%%%%%
\subsection{Include by Input}
\label{sec:input}

Including child documents by |\include| has some restrictions by design.
Most notably, the content of a child document always occupies
its own set of pages; pages cannot be shared between child documents.
Usually, this behaviour makes perfect sense
because each child document contain an essential part of the document.
However, in some situations it may be desirable to compose
a document from a collection of parts
without having mandatory page breaks between then.
For this case, the package
provides a mechanism to include parts
by |\input| which can also be processed individually.
However, by construction this mechanism
requires manual handling of the content to be output.

%%%%%%%%%%%%%%%%%%%%%%%%%%%%%%%%%%%%%%%%
\DescribeMacro{\ifchilddocmanual}
The main file should be prepared as usual, see \secref{sec:include}.
However, the document body must make a distinction
between processing of an individual part and of the main document, e.g.:
%
\begin{center}
\begin{tabular}{l}
|\ifchilddocmanual|\\
|\input{\childdocname}|\\
|\||else|\\
\textit{document body with }|\input{|\textit{part}|}|\\
|\||fi|
\end{tabular}
\end{center}
%
The conditional |\ifchilddocmanual| is true whenever
a part to be included by |\input| is being compiled,
and the name of the part is stored in |\childdocname|.

%%%%%%%%%%%%%%%%%%%%%%%%%%%%%%%%%%%%%%%%
\DescribeMacro{\childdocby}
Each part to be included by |\input| should start with:
%
\begin{center}
\begin{tabular}{l}
|\input{childdoc.def}|\\
|\childdocby{|\textit{main}|}|\\
\end{tabular}
\end{center}
%
The directive |\childdocby| is similar to |\childdocof|
described in \secref{sec:include},
but the subsequent selection of content must be done manually.
To that end, both |\ifchilddoc| and |\ifchilddocmanual|
will be true upon processing of a part,
and the name of the part is stored in |\childdocname|.
Note that |\jobname| will be set to the filename of the current part
so that each part receives an individual |.aux| file
that does not interfere with the |.aux| file(s) of the main document.
This behaviour can be altered by the alternative form
|\childdocby[*]{|\textit{main}|}| (with a non-empty optional argument)
which uses the |.aux| file of the main document
by setting |\jobname| to \textit{main}.

%%%%%%%%%%%%%%%%%%%%%%%%%%%%%%%%%%%%%%%%%%%%%%%%%%%%%%%%%%%%%%%%%%%%%%%%%%%%%%%%
\subsection{Driver Development}
\label{sec:driver}

The \textsf{childdoc} mechanism can also be use for the development
of definition files such as \LaTeX{} styles or classes.
This case differs from the above setup with multiple parts
included by |\include| in that no |\includeonly| should be invoked.
This can be achieved by starting the include file
(before |\ProvidesPackage|) with:
%
\begin{center}
\begin{tabular}{l}
|\input{childdoc.def}|\\
|\childdocforward{|\textit{main}|}|\\
\end{tabular}
\end{center}
%
or alternatively with:
%
\begin{center}
\begin{tabular}{l}
|\input{childdoc.def}|\\
|\childdocby{|\textit{main}|}|\\
\end{tabular}
\end{center}
%
Both forms have slightly different effects as described above.
The main file is prepared as usual, see \secref{sec:include}.

%%%%%%%%%%%%%%%%%%%%%%%%%%%%%%%%%%%%%%%%%%%%%%%%%%%%%%%%%%%%%%%%%%%%%%%%%%%%%%%%
\subsection{Legacy Detection}
\label{sec:detection}

The directive |\childdocmain| in the main file can detect
whether the complete document or merely a child is to be compiled
even without using the directive |\childdocof|.
This method is deprecated because it is less robust
and there is no compelling reason to use it;
it is merely provided for backward compatibility
and it may be removed in future versions.

If the detection mechanism is to be used,
it is mandatory to correctly specify
the filename of the main file as the argument of |\childdocmain|:
%
\begin{center}
\begin{tabular}{l}
|\input{childdoc.def}|\\
|\childdocmain{|\textit{main}|}|\\
\end{tabular}
\end{center}
%
If |\jobname| does not match the argument \textit{main} of |\childdocmain|,
it is assumed that |\jobname| points to the child file to be compiled.
When using |\childdocmain| with the main file specified as argument,
it suffices to start a child file
with just |\input{|\textit{main}|}|
without loading of the package and using |\childdocof|.
If instead all processing is done
with the appropriate \textsf{childdoc} directives,
the argument of \textit{main} of |\childdocmain| can be empty.

An alternative version of the command line processing described
in \secref{sec:commandline} using the detection mechanism reads:
%
\begin{center}
|... -jobname "|\textit{target}|" "|[\textit{flags}]%
[|\def\jobname{|\textit{dest}|}|]|\input{|\textit{main}|}"|
\end{center}

%%%%%%%%%%%%%%%%%%%%%%%%%%%%%%%%%%%%%%%%%%%%%%%%%%%%%%%%%%%%%%%%%%%%%%%%%%%%%%%%
\subsection{Manual Code}
\label{sec:manual}

In case one cannot be certain whether the definitions file |childdoc.def|
is installed on the target \TeX{} distribution
and one prefers not to ship it,
it is conceivable to paste a few relevant commands into the sources.

To that end, drop all statements |\input{childdoc.def}|
and perform the replacements as outlined below.
Instead of |\childdocmain{|\textit{main}|}| add the following code
to the top of the main file:
%
\begin{center}
\begin{tabular}{l}
|\||ifdefined\childdocname\endinput\||fi\newif\ifchilddoc|\\
|\edef\childdocname{\scantokens\expandafter{\jobname\noexpand}}|\\
|\def\childdocmain{|\textit{main}|}\||ifx\childdocmain\childdocname\||else|\\
|\childdoctrue\includeonly{\childdocname}\let\jobname\childdocmain\||fi|\\
\end{tabular}
\end{center}
%
Instead of |\childdocof{|\textit{main}|}| just include the main file
at the top of each child file:
%
\begin{center}
|\input{|\textit{main}|}|
\end{center}
%
A simple redirection |\childdocforward{|\textit{dest}|}| is achieved by:
%
\begin{center}
|\def\jobname{|\textit{dest}|}\input{\jobname}|
\end{center}
%
The redirection with prefix
|\childdocforwardprefix[|\textit{prefix}|]{|\textit{dest}|}|
is accomplished by:
%
\begin{center}
\begin{tabular}{l}
|{\edef\jobname{\scantokens\expandafter{\jobname\noexpand}}|\\
|\def\redirectjob |\textit{prefix}|#1~~~{\gdef\jobname{|\textit{dest}|#1}}|\\
|\expandafter\redirectjob\jobname~~~}\input{\jobname}|
\end{tabular}
\end{center}

In an alternative approach,
child documents can be compiled by a specific command line
without additional code or specific definitions:
%
\begin{center}
|... -jobname "|\textit{target}|" "|[\textit{flags}]%
|\includeonly{|\textit{dest}|}\input{|\textit{main}|}"|
\end{center}
%

%%%%%%%%%%%%%%%%%%%%%%%%%%%%%%%%%%%%%%%%%%%%%%%%%%%%%%%%%%%%%%%%%%%%%%%%%%%%%%%%
%%%%%%%%%%%%%%%%%%%%%%%%%%%%%%%%%%%%%%%%%%%%%%%%%%%%%%%%%%%%%%%%%%%%%%%%%%%%%%%%
\section{Information}

%%%%%%%%%%%%%%%%%%%%%%%%%%%%%%%%%%%%%%%%%%%%%%%%%%%%%%%%%%%%%%%%%%%%%%%%%%%%%%%%
\subsection{Copyright}

Copyright \copyright{} 2017--2018 Niklas Beisert

This work may be distributed and/or modified under the
conditions of the \LaTeX{} Project Public License, either version 1.3
of this license or (at your option) any later version.
The latest version of this license is in
  \url{http://www.latex-project.org/lppl.txt}
and version 1.3 or later is part of all distributions of \LaTeX{}
version 2005/12/01 or later.

This work has the LPPL maintenance status `maintained'.

The Current Maintainer of this work is Niklas Beisert.

This work consists of the files |README.txt|, |childdoc.ins| and |childdoc.dtx|
as well as the derived files |childdoc.def|, |cdocsamp.tex|
with |cdocsch1.tex|, |cdocsch2.tex|, |cdocspt3.tex|, |cdocspt4.tex|,
|cdocsdrf.tex|, |cdocsfn1.tex|, |cdocsfn2.tex|
as well as |childdoc.pdf|.

%%%%%%%%%%%%%%%%%%%%%%%%%%%%%%%%%%%%%%%%%%%%%%%%%%%%%%%%%%%%%%%%%%%%%%%%%%%%%%%%
\subsection{Files and Installation}

The package consists of the files:
%
\begin{center}
\begin{tabular}{ll}
    |README.txt|   & readme file \\
    |childdoc.ins| & installation file \\
    |childdoc.dtx| & source file \\
    |childdoc.def| & definition file \\
    |cdocsamp.tex| & sample main file \\
    |cdocsch1.tex| & sample include file \\
    |cdocsch2.tex| & sample include file \\
    |cdocspt3.tex| & sample part file \\
    |cdocspt4.tex| & sample part file \\
    |cdocsdrf.tex| & sample redirection file \\
    |cdocsfn1.tex| & sample redirection file \\
    |cdocsfn2.tex| & sample redirection file \\
    |childdoc.pdf| & manual
\end{tabular}
\end{center}
%
The distribution consists of the files
|README.txt|, |childdoc.ins| and |childdoc.dtx|.
%
\begin{itemize}
\item
Run (pdf)\LaTeX{} on |childdoc.dtx|
to compile the manual |childdoc.pdf| (this file).
\item
Run \LaTeX{} on |childdoc.ins| to create the definitions file |childdoc.def|
and the sample |cdocsamp.tex| with include files
|cdocsch1.tex|, |cdocsch2.tex|, |cdocspt3.tex|, |cdocspt4.tex|,
|cdocsdrf.tex|, |cdocsfn1.tex|, |cdocsfn2.tex|.
Then copy the file |childdoc.def| to an appropriate directory of your \LaTeX{}
distribution, e.g.\ \textit{texmf-root}|/tex/latex/childdoc|.
\end{itemize}

%%%%%%%%%%%%%%%%%%%%%%%%%%%%%%%%%%%%%%%%%%%%%%%%%%%%%%%%%%%%%%%%%%%%%%%%%%%%%%%%
\subsection{Related CTAN Packages}

There are several other packages which offer a similar functionality:
%
\begin{itemize}
\item
The packages
\href{http://ctan.org/pkg/docmute}{\textsf{docmute}},
\href{http://ctan.org/pkg/includex}{\textsf{includex}} and
\href{http://ctan.org/pkg/standalone}{\textsf{standalone}}
provide commands to include only the document body of
a child file thus allowing both files to be compiled individually.
\item
The packages \href{http://ctan.org/pkg/subdocs}{\textsf{subdocs}}
and \href{http://ctan.org/pkg/subfiles}{\textsf{subfiles}}
provide structures in which the main and child documents can be
encapsulated and allowing them to be compiled individually.
The inclusion mechanism is different from the conventional |\include|.
\item
The package \href{http://ctan.org/pkg/combine}{\textsf{combine}}
is an elaborate solution to combine several documents into one.
\end{itemize}
%
See also the CTAN topic \href{http://ctan.org/topic/subdocs}{\textsf{subdocs}}
for further related packages.
The present package differs from the above solutions in that
a document structure constructed with the conventional |\include| mechanism
just needs two extra commands at the top of every file
such that all constituent files can be compiled individually.

%%%%%%%%%%%%%%%%%%%%%%%%%%%%%%%%%%%%%%%%%%%%%%%%%%%%%%%%%%%%%%%%%%%%%%%%%%%%%%%%
%\subsection{Feature Suggestions}
%
%The following is a list of features which may be useful for future
%versions of this package:
%%
%\begin{itemize}
%\item
%\ldots
%\end{itemize}

%%%%%%%%%%%%%%%%%%%%%%%%%%%%%%%%%%%%%%%%%%%%%%%%%%%%%%%%%%%%%%%%%%%%%%%%%%%%%%%%
\subsection{Revision History}

%%%%%%%%%%%%%%%%%%%%%%%%%%%%%%%%%%%%%%%%
\paragraph{v2.0:} 2018/12/30

\begin{itemize}
\item
immediate forward processing
\item
added |\childdocby| mechanism
\item
manual restructured
\end{itemize}

%%%%%%%%%%%%%%%%%%%%%%%%%%%%%%%%%%%%%%%%
\paragraph{v1.6:} 2018/01/17

\begin{itemize}
\item
application for development of include files
\item
corrections to manual
\end{itemize}

%%%%%%%%%%%%%%%%%%%%%%%%%%%%%%%%%%%%%%%%
\paragraph{v1.5:} 2017/05/21

\begin{itemize}
\item
more complete structuring introduced
\item
|\childdocof| introduced
\item
|\childdoc| renamed to |\childdocmain|
\item
|\childredirect| renamed to |\childdocforward| and |\childdocforwardprefix|
and functionality expanded
\end{itemize}

%%%%%%%%%%%%%%%%%%%%%%%%%%%%%%%%%%%%%%%%
\paragraph{v1.0:} 2017/04/27

\begin{itemize}
\item
manual and install package
\item
first version published on CTAN
\end{itemize}

%%%%%%%%%%%%%%%%%%%%%%%%%%%%%%%%%%%%%%%%
\paragraph{v0.6:} 2017/04/26

\begin{itemize}
\item
redirection mechanism added
\end{itemize}

%%%%%%%%%%%%%%%%%%%%%%%%%%%%%%%%%%%%%%%%
\paragraph{v0.5:} 2017/04/26

\begin{itemize}
\item
functionality in definition file
\end{itemize}


%%%%%%%%%%%%%%%%%%%%%%%%%%%%%%%%%%%%%%%%%%%%%%%%%%%%%%%%%%%%%%%%%%%%%%%%%%%%%%%%
%%%%%%%%%%%%%%%%%%%%%%%%%%%%%%%%%%%%%%%%%%%%%%%%%%%%%%%%%%%%%%%%%%%%%%%%%%%%%%%%
%%%%%%%%%%%%%%%%%%%%%%%%%%%%%%%%%%%%%%%%%%%%%%%%%%%%%%%%%%%%%%%%%%%%%%%%%%%%%%%%
\appendix

\settowidth\MacroIndent{\rmfamily\scriptsize 000\ }

 \DocInput{childdoc.dtx}

\end{document}
%</driver>
% \fi
%
% %%%%%%%%%%%%%%%%%%%%%%%%%%%%%%%%%%%%%%%%%%%%%%%%%%%%%%%%%%%%%%%%%%%%%%%%%%%%%%
% %%%%%%%%%%%%%%%%%%%%%%%%%%%%%%%%%%%%%%%%%%%%%%%%%%%%%%%%%%%%%%%%%%%%%%%%%%%%%%
% \section{Sample}
%\iffalse
%<*samplemain>
%\fi
%
% The following presents a sample document
% with two chapters, two parts, a title page,
% a compile flag as well as three forwarding files to set the flag.
% It consists of eight |.tex| files:
% \begin{center}
% \begin{tabular}{ll}
% |cdocsamp.tex|&main file\\
% |cdocsch1.tex|&include file for chapter 1\\
% |cdocsch2.tex|&include file for chapter 2\\
% |cdocspt3.tex|&include file for part 3\\
% |cdocspt4.tex|&include file for part 4\\
% |cdocsdrf.tex|&forwarding file for main file in draft mode\\
% |cdocsfi1.tex|&forwarding file for final version of chapter 1\\
% |cdocsfi2.tex|&forwarding file for final version of chapter 2\\
% \end{tabular}
% \end{center}
% Each of the eight files can be compiled directly by the \LaTeX{} compiler.
%
% %%%%%%%%%%%%%%%%%%%%%%%%%%%%%%%%%%%%%%
% \paragraph{Main File.}
%
% The main file is called |cdocsamp.tex|.
%
% Load the \textsf{childdoc} definitions and
% declare the filename for the main document:
%    \begin{macrocode}
\input{childdoc.def}
\childdocmain{}
%    \end{macrocode}

% Optional override for |\version| flag:
%    \begin{macrocode}
%%\ifchilddoc\else\providecommand{\version}{draft}\fi
%    \end{macrocode}

% Define the default values for the |\version| flag
% (|final| for the main file and |draft| for childs):
%    \begin{macrocode}
\ifchilddoc
\providecommand{\version}{draft}
\else
\providecommand{\version}{final}
\fi
%    \end{macrocode}

% Load the standard document class:
%    \begin{macrocode}
\documentclass[12pt]{article}
%    \end{macrocode}

% Start the document body:
%    \begin{macrocode}
\begin{document}
%    \end{macrocode}

% Declare a title page.
% Print title, part of document being processed and version flag:
%    \begin{macrocode}
\addtocounter{page}{-1}
\begin{center}
{\LARGE\bfseries{}childdoc example\par}
\vspace{1cm}
\ifchilddoc
\ifchilddocmanual part\else chapter\fi:
`\childdocname' of `\childdocjob'\par
\else
main document: `\childdocjob'\par
\fi
version: \version\par
\end{center}
\newpage
%    \end{macrocode}

% Manually include selected file,
% otherwise process as usual:
%    \begin{macrocode}
\ifchilddocmanual
\section*{part `\childdocname'}
\input{\childdocname}
\else
%    \end{macrocode}

% Include the two chapters:
%    \begin{macrocode}
\include{cdocsch1}
\include{cdocsch2}
%    \end{macrocode}

% Include the two parts unless only chapters should be displayed:
%    \begin{macrocode}
\ifchilddoc\else
\section{part three}
\input{cdocspt3}
\section{part four}
\input{cdocspt4}
\fi
%    \end{macrocode}

% Process as usual until here:
%    \begin{macrocode}
\fi
%    \end{macrocode}

% End of document body:
%    \begin{macrocode}
\end{document}
%    \end{macrocode}
%\iffalse
%</samplemain>
%\fi
%
% %%%%%%%%%%%%%%%%%%%%%%%%%%%%%%%%%%%%%%
% \paragraph{Chapter Include Files.}
%
% The include files are called |cdocsch1.tex| and |cdocsch2.tex|.
%
%\iffalse
%<*samplechap1|samplechap2>
%\fi

% Optional override for |\version| flag:
%    \begin{macrocode}
%%\providecommand{\version}{final}
%    \end{macrocode}

% Include the main document:
%    \begin{macrocode}
\input{childdoc.def}
\childdocof{cdocsamp}
%    \end{macrocode}

%\iffalse
%</samplechap1|samplechap2>
%\fi
%
%\iffalse
%<*samplechap1>
%\fi
% Some text for chapter 1:
%    \begin{macrocode}
\section{one}
some text in chapter one
%    \end{macrocode}

%\iffalse
%</samplechap1>
%\fi
% Some text for chapter 2:
%\iffalse
%<*samplechap2>
%\fi
%    \begin{macrocode}
\section{two}
more text in chapter two
%    \end{macrocode}

%\iffalse
%</samplechap2>
%\fi
%
% %%%%%%%%%%%%%%%%%%%%%%%%%%%%%%%%%%%%%%
% \paragraph{Part Include Files.}
%
% The include files are called |cdocspt3.tex| and |cdocspt4.tex|.
%
%\iffalse
%<*samplepart3|samplepart4>
%\fi

% Optional override for |\version| flag:
%    \begin{macrocode}
%%\providecommand{\version}{final}
%    \end{macrocode}

% Include the main document:
%    \begin{macrocode}
\input{childdoc.def}
\childdocby{cdocsamp}
%    \end{macrocode}

%\iffalse
%</samplepart3|samplepart4>
%\fi
%
%\iffalse
%<*samplepart3>
%\fi
% Some text for part 3:
%    \begin{macrocode}
some text in part three
%    \end{macrocode}

%\iffalse
%</samplepart3>
%\fi
% Some text for part 4:
%\iffalse
%<*samplepart4>
%\fi
%    \begin{macrocode}
more text in part four
%    \end{macrocode}

%\iffalse
%</samplepart4>
%\fi
%
% %%%%%%%%%%%%%%%%%%%%%%%%%%%%%%%%%%%%%%
% \paragraph{Forwarding for a Complete Draft.}
%
% The following forwarding file |cdocsdrf.tex|
% compiles the main document in draft mode:
%\iffalse
%<*sampledraft>
%\fi
%    \begin{macrocode}
\def\version{draft}
\input{childdoc.def}
\childdocforward{cdocsamp}
%    \end{macrocode}

%\iffalse
%</sampledraft>
%\fi
%
% %%%%%%%%%%%%%%%%%%%%%%%%%%%%%%%%%%%%%%
% \paragraph{Forwarding for Final Version of the Chapters.}
%
% The following forwarding files |cdocsfn1.tex| and |cdocsfn2.tex|
% (with identical content)
% compile the final versions of the child documents
% |cdocsch1.tex| and |cdocsch2.tex|, respectively:
%\iffalse
%<*samplefinal>
%\fi
%    \begin{macrocode}
\def\version{final}
\input{childdoc.def}
\childdocforwardprefix[cdocsamp]{cdocsfn}{cdocsch}
%    \end{macrocode}

%\iffalse
%</samplefinal>
%\fi
%
% %%%%%%%%%%%%%%%%%%%%%%%%%%%%%%%%%%%%%%
% \paragraph{Command Line Processing.}
%
% The following three command lines generate the output files
% |cdocscld|, |cdocscl1| and |cdocscl2|
% which should be identical to
% |cdocsdrf|, |cdocsch1| and |cdocsfn2|, respectively:
% \begin{center}
% \begin{tabular}{l}
% |latex -jobname cdocscld \|\\
% |  "\def\version{draft}\input{childdoc.def}\childdocforward{cdocsamp}"|\\
% |latex -jobname cdocscl1 \|\\
% |  "\input{childdoc.def}\childdocforward[cdocsamp]{cdocsch1}"|\\
% |latex -jobname cdocscl2 \|\\
% |  "\def\version{final}\input{childdoc.def}\childdocforward{cdocsch2}"|
% \end{tabular}
% \end{center}
% Note that the trailing backslash on each first line
% merely continues the input to the second line
% (for convenient cut ant paste).
% Furthermore, the command |latex| can be replaced by any
% of its alternative versions such as |pdflatex|.
%
% %%%%%%%%%%%%%%%%%%%%%%%%%%%%%%%%%%%%%%%%%%%%%%%%%%%%%%%%%%%%%%%%%%%%%%%%%%%%%%
% %%%%%%%%%%%%%%%%%%%%%%%%%%%%%%%%%%%%%%%%%%%%%%%%%%%%%%%%%%%%%%%%%%%%%%%%%%%%%%
% \section{Implementation}
%\iffalse
%<*package>
%\fi
%
% This section describes the definitions file |childdoc.def|.

% The definitions cannot be loaded using |\usepackage| or |\RequirePackage|
% which has a mechanism to prevent loading a style file more than once.
% When loading the definitions by means of |\input|
% multiple instances have to be prevented manually:
%\iffalse
%This code needs to be before the `\ProvidesFile' directive
%which is defined at the beginning of this file.
%Therefore it is also placed there and commented out here.
%</package>
%<*discard>
%\fi
%    \begin{macrocode}
\ifdefined\childdocmain\endinput\fi
%    \end{macrocode}
%\iffalse
%</discard>
%<*package>
%\fi
%
% \macro{\ifchilddoc}
% \macro{\ifchilddocmanual}
% The conditional |\ifchilddoc| tells whether a
% child (true) or main (false) document is being compiled.
% The conditional |\ifchilddocmanual| tells whether
% the |\includeonly| mechanism is used (false) or
% the selection of child files must be performed manually (true).
% The definitions initialise to false:
%    \begin{macrocode}
\newif\ifchilddoc
\newif\ifchilddocmanual
%    \end{macrocode}

% \macro{\childdocname}
% \macro{\childdocjob}
% The macro |\childdocname| stores the name of the main document
% to be compiled. The macro |\childdocjob| stores the name of
% the document on which the \LaTeX{} compiler was originally invoked.
% The content of |\jobname| cannot be compared
% to filenames specified in the source due to different catcodes.
% The following code rescans |\jobname|, stores the result
% in |\childdocname| and saves a copy in |\childdocjob|:
%    \begin{macrocode}
\edef\childdocname{\scantokens\expandafter{\jobname\noexpand}}
\let\childdocjob\childdocname
%    \end{macrocode}

% \macro{\childdocdisable}
% The macro |\childdocdisable| prevents the main file
% from being processed more than once.
% At this stage, the main document command |\childdocmain|
% is assumed to be called once again where it should do nothing.
% Any subsequent call to it should prevent
% a secondary processing of the main document
% It overwrites the forwarding commands
% |\childdocof| and |\childdocforward|
% with empty macros to prevent further inclusions of the main document:
%    \begin{macrocode}
\newcommand{\childdocdisable}
{
  \renewcommand{\childdocmain}[1]{\renewcommand{\childdocmain}[1]{\endinput}}
  \renewcommand{\childdocof}[1]{}
  \renewcommand{\childdocby}[2][]{}
  \renewcommand{\childdocforward}[2][]{}
  \renewcommand{\childdocdisable}{}
}
%    \end{macrocode}

% \macro{\childdocmain}
% The macro |\childdocmain| is to be called at the top of the main file
% with nothing or the main filename (without extension) as argument.
% First, it breaks loops.
% If the argument is not empty and does not match |\childdocname|
% (which is set by the first inclusion of |childdoc.def|),
% |\ifchilddoc| is set to true, |\includeonly| is applied to the child file
% and |\jobname| is set to the main file
% (for proper handling of |.aux| files):
%    \begin{macrocode}
\newcommand{\childdocmain}[1]
{
  \childdocdisable\childdocmain{}
  \if?#1?\else
    \begingroup
      \def\childdoctmp{#1}
      \ifx\childdoctmp\childdocname
        \def\childdoctmp{}
      \else
        \def\childdoctmp
        {
          \childdoctrue
          \includeonly{\childdocname}
          \def\childdocjob{#1}
          \def\jobname{#1}
        }
      \fi
      \expandafter
    \endgroup
    \childdoctmp
  \fi
}
%    \end{macrocode}

% \macro{\childdocof}
% The command |\childdocof| redirects
% compilation to the main file |#1|.
%    \begin{macrocode}
\newcommand{\childdocof}[1]
{
  \childdocdisable
  \childdoctrue
  \includeonly{\childdocname}
  \def\jobname{#1}
  \def\childdocjob{#1}
  \input{#1}
}
%    \end{macrocode}

% \macro{\childdocby}
% The command |\childdocby| ....
%    \begin{macrocode}
\newcommand{\childdocby}[2][]
{
  \childdocdisable
  \childdoctrue
  \childdocmanualtrue
  \if?#1?\else
    \def\jobname{#2}
  \fi
  \def\childdocjob{#2}
  \input{#2}
  \endinput
}
%    \end{macrocode}

% \macro{\childdocforward}
% The command |\childdocforward| redirects
% compilation to the main file or
% (if the optional argument is given) a child file.
% Parameters are set as if the main file
% or a child file starting with |\childdocof| was compiled.
% Then compilation is handed over to the main file:
%    \begin{macrocode}
\newcommand{\childdocforward}[2][]
{
  \begingroup
    \if?#1?
      \def\childdoctmp
      {
        \def\childdocname{#2}
        \def\childdocjob{#2}
        \def\jobname{#2}
        \input{#2}
        \endinput
      }
    \else
      \def\childdoctmp
      {
        \childdocdisable
        \def\childdocname{#2}
        \childdoctrue
        \includeonly{#2}
        \def\childdocjob{#1}
        \def\jobname{#1}
        \input{#1}
        \endinput
      }
    \fi
    \expandafter
  \endgroup
  \childdoctmp
}
%    \end{macrocode}

% \macro{\childdocforwardprefix}
% The command |\childdocforwardprefix| redirects
% compilation to the main or a child file by means of a pattern.
% The prefix |#1| in the current filename is replaced by |#2|
% and the suffix of the current filename is kept
% (it is assumed that the filename does not contain the substring `|~~~|'
% which is used as a delimiter).
% Compilation is handed over to the new file by |\childdocforward|:
%    \begin{macrocode}
\newcommand{\childdocforwardprefix}[3][]
{
  \begingroup
    \def\childdocextract #2##1~~~{\def\childdoctmp{\childdocforward[#1]{#3##1}}}
    \expandafter\childdocextract\childdocname~~~
    \expandafter
  \endgroup
  \childdoctmp
}
%    \end{macrocode}

% \macro{\childdoc}
% The deprecated macro |\childdoc| is a legacy version of |\childdocmain|:
%    \begin{macrocode}
\newcommand{\childdoc}{\childdocmain}
%    \end{macrocode}

% \macro{\childdocredirect}
% The deprecated macro |\childdocredirect| is a legacy version
% of |\childdocforward| and |\childdocforwardprefix|:
%    \begin{macrocode}
\newcommand{\childdocredirect}[2][]
{
  \begingroup
    \if?#1?
      \def\childdoctmp{\childdocforward{#2}}
    \else
      \def\childdoctmp{\childdocforwardprefix{#1}{#2}}
    \fi
    \expandafter
  \endgroup
  \childdoctmp
}
%    \end{macrocode}

%\iffalse
%</package>
%\fi
%
\endinput
|\\
|\childdocof{|\textit{main}|}|\\
\end{tabular}
\end{center}
at the top of every child file \textit{child}
which is included by |\include{|\textit{child}|}|
from within the main file
(or at least for those files to be compiled individually).
The argument \textit{main} must be the filename of the main file.

There are a couple of
considerations in setting up the main and child documents:

%%%%%%%%%%%%%%%%%%%%%%%%%%%%%%%%%%%%%%%%
\paragraph{Restrictions.}

Please note the following restrictions:
\begin{itemize}
\item
|\childdocmain| must be called with one argument \textit{main}
to ensure compatibility with earlier version of the package.
It must either be empty (|\childdocmain{}|)
or precisely match the filename of the main file in which it is specified.
See \secref{sec:detection} for further information.
\item
The filename \textit{main} must be specified without the |.tex| extension.
\item
The filename \textit{main} is case sensitive
(even in case-insensitive file systems)
due to internal string comparison.
\item
The argument \textit{main} should be fully expanded, it cannot be a macro.
\item
Subdirectories and special characters should be avoided in filenames.
\item
The command |\childdocmain{|\textit{main}|}| must be followed by a whitespace.
It should not be followed immediately by another command
or by a comment mark `|%|'.
This is because the \TeX{} parser reads the token immediately following
the argument of |\childdocmain| and puts it
at the beginning of every child section;
however, a white\-space is ignored.
\end{itemize}

%%%%%%%%%%%%%%%%%%%%%%%%%%%%%%%%%%%%%%%%
\paragraph{Content of Main File.}

It is advisable to place all content in the child files included by |\include|.
Any output contained in the main file will appear in all child documents
unless suppressed manually;
it cannot be suppressed automatically by the |\includeonly| directive
and thus should normally be avoided.
A method to include some content in the main file
by means of conditional processing is described in \secref{sec:conditional}.

%%%%%%%%%%%%%%%%%%%%%%%%%%%%%%%%%%%%%%%%
\paragraph{Page Numbering.}

When only a part of the document is compiled,
the appropriate numbering of pages
(as well as other status parameters)
is determined from the |.aux| files.
The latter contain information from previous passes.
However this information needs to propagate through
all intermediate child documents.
Therefore the page numbering in child documents may well
be inconsistent until the complete document is compiled at least once.

A useful (if unconventional) way to always ensure a consistent
page numbering is to restart the numbering in each child document
and denote the pages by `\textit{child}|.|\textit{page}'
where \textit{child} represents the chapter/section number of the child file.
This can be achieved by the command
|\numberwithin{page}{|\textit{child}|}|
of the \textsf{amsmath} package
where \textit{child} can be |chapter| or |section|
depending on the chosen structuring.
Alternatively, one can modify the macro |\thepage| appropriately
and reset the counter |page| at the start of each child file.

%%%%%%%%%%%%%%%%%%%%%%%%%%%%%%%%%%%%%%%%%%%%%%%%%%%%%%%%%%%%%%%%%%%%%%%%%%%%%%%%
\subsection{Conditional Processing}
\label{sec:conditional}

The package provides a mechanism to compile different versions
of a document. To customise the versions further some conditional processing
can come in handy to distinguish which version is being compiled.
The package provides two macros to describe the compilation context:

%%%%%%%%%%%%%%%%%%%%%%%%%%%%%%%%%%%%%%%%
\DescribeMacro{\ifchilddoc}
The conditional |\ifchilddoc| distinguishes between the compilation of
child documents and the main document:
%
\begin{center}
|\ifchilddoc |\textit{child-code}| |[|\||else |\textit{main-code}]| \||fi|
\end{center}

%%%%%%%%%%%%%%%%%%%%%%%%%%%%%%%%%%%%%%%%
\DescribeMacro{\childdocname}
\DescribeMacro{\childdocjob}
The macro |\childdocname| contains the filename (without extension)
of the main or child file being processed.
Note that |\childdocjob| will always contain the name of the main file.

%%%%%%%%%%%%%%%%%%%%%%%%%%%%%%%%%%%%%%%%
\paragraph{Title Page.}

Conditional processing can be used to include a title or banner page
in the main document when proper precautions are taken.
Importantly, the code in the main file should ensure that the page counter
(as well as other status parameters which are stored in the |.aux| files)
takes the same value after the conditional processing.
Otherwise the page numbers may take divergent values
depending on which part is compiled.

For example, a title page could be declared by:
%
\begin{center}
\begin{tabular}{l}
|\ifchilddoc\||else|\\
|\addtocounter{page}{-1}|\\
\textit{code for title page}\\
|\newpage|\\
|\||fi|
\end{tabular}
\end{center}
%
A banner page for the child documents can be generated by:
%
\begin{center}
\begin{tabular}{l}
|\ifchilddoc|\\
|\addtocounter{page}{-1}|\\
\textit{code for banner page}\\
|\newpage|\\
|\||fi|
\end{tabular}
\end{center}
%
Here one could write a message such as:
\begin{center}
|This is the part \childdocname{} of \childdocjob{}.|
\end{center}

%%%%%%%%%%%%%%%%%%%%%%%%%%%%%%%%%%%%%%%%%%%%%%%%%%%%%%%%%%%%%%%%%%%%%%%%%%%%%%%%
\subsection{Flags}
\label{sec:flags}

The package makes it easy to generate different versions
of the main or child documents.
To this end compilation flags can be defined
and assigned different default values.
They will be particularly useful in conjunction
with the forwarding mechanism described in \secref{sec:forward}.

For example, it may be useful to have a flag |\version|
which can be set to |draft| or |final|.
The document source will contain some conditional code
depending on the value of |\version|.
Suppose further, the flag should default to |final| for the main file
and to |draft| for child files
which is a natural assignment for editing the document.
This is achieved by placing the following code
in the preamble of the main document
(below the |\childdocmain| directive):
%
\begin{center}
\begin{tabular}{l}
|\ifchilddoc|\\
|\providecommand{\version}{draft}|\\
|\||else|\\
|\providecommand{\version}{final}|\\
|\||fi|
\end{tabular}
\end{center}
%
The definition by |\providecommand| makes sure
that previous definitions are not overwritten.
Further statements |\providecommand{\version}{...}|
can thus be added before the above code to override it.

For the main file, one might add a line
(between |\childdocmain| and the above block)
%
\begin{center}
|%\ifchilddoc\||else\providecommand{\version}{draft}\||fi|
\end{center}
%
which can be uncommented to produce a draft version.
Likewise one can add a line to the very top of a child file
(above the |\childdocof{|\textit{main}|}| directive)
%
\begin{center}
|%\providecommand{\version}{final}|
\end{center}
%
which can be uncommented to produce the final version of this child document.

%%%%%%%%%%%%%%%%%%%%%%%%%%%%%%%%%%%%%%%%%%%%%%%%%%%%%%%%%%%%%%%%%%%%%%%%%%%%%%%%
\subsection{Forwarding}
\label{sec:forward}

Different versions of the main or child documents
using compilation flags as described in \secref{sec:flags}
can be (permanently) stored in different files
for convenient compilation, viewing and distribution.
To this end, the package defines a command
to pass on compilation to a different file:

%%%%%%%%%%%%%%%%%%%%%%%%%%%%%%%%%%%%%%%%
\DescribeMacro{\childdocforward}
The command |\childdocforward| redirects processing to
another source file:
%
\begin{center}
\begin{tabular}{l}
|% \iffalse
%
% childdoc.dtx Copyright (C) 2017-2018 Niklas Beisert
%
% This work may be distributed and/or modified under the
% conditions of the LaTeX Project Public License, either version 1.3
% of this license or (at your option) any later version.
% The latest version of this license is in
%   http://www.latex-project.org/lppl.txt
% and version 1.3 or later is part of all distributions of LaTeX
% version 2005/12/01 or later.
%
% This work has the LPPL maintenance status `maintained'.
%
% The Current Maintainer of this work is Niklas Beisert.
%
% This work consists of the files childdoc.dtx and childdoc.ins
% and the derived files childdoc.def and cdocsamp.tex with
% cdocsch1.tex, cdocsch2.tex, cdocsdrf.tex, cdocsfn1.tex, cdocsfn2.tex.
%
%<package>\ifdefined\childdocmain\endinput\fi
%<package>\ProvidesFile{childdoc.def}[2018/12/30 v2.0 child document driver]
%<samplemain>\ProvidesFile{cdocsamp.tex}[2018/12/30 v2.0 sample for childdoc]
%<*driver>
%\ProvidesFile{childdoc.drv}[2018/12/30 v2.0 childdoc reference manual file]
\PassOptionsToClass{10pt,a4paper}{article}
\documentclass{ltxdoc}

\usepackage[margin=35mm]{geometry}
\usepackage{hyperref}
\usepackage{hyperxmp}
\usepackage[usenames]{color}

\hypersetup{colorlinks=true}
\hypersetup{pdfstartview=FitH}
\hypersetup{pdfpagemode=UseNone}
\hypersetup{pdfsource={}}
\hypersetup{pdflang={en-UK}}
\hypersetup{pdfcopyright={Copyright 2017-2018 Niklas Beisert.
  This work may be distributed and/or modified under the
  conditions of the LaTeX Project Public License, either version 1.3
  of this license or (at your option) any later version.}}
\hypersetup{pdflicenseurl={http://www.latex-project.org/lppl.txt}}
\hypersetup{pdfcontactaddress={ETH Zurich, ITP, HIT K,
  Wolfgang-Pauli-Strasse 27}}
\hypersetup{pdfcontactpostcode={8093}}
\hypersetup{pdfcontactcity={Zurich}}
\hypersetup{pdfcontactcountry={Switzerland}}
\hypersetup{pdfcontactemail={nbeisert@itp.phys.ethz.ch}}
\hypersetup{pdfcontacturl={http://people.phys.ethz.ch/\xmptilde nbeisert/}}

\newcommand{\secref}[1]{\hyperref[#1]{section \ref*{#1}}}

\parskip1ex
\parindent0pt
\let\olditemize\itemize
\def\itemize{\olditemize\parskip0pt}

\begin{document}

\title{The \textsf{childdoc} Package}
\hypersetup{pdftitle={The childdoc Package}}
\author{Niklas Beisert\\[2ex]
  Institut f\"ur Theoretische Physik\\
  Eidgen\"ossische Technische Hochschule Z\"urich\\
  Wolfgang-Pauli-Strasse 27, 8093 Z\"urich, Switzerland\\[1ex]
  \href{mailto:nbeisert@itp.phys.ethz.ch}
  {\texttt{nbeisert@itp.phys.ethz.ch}}}
\hypersetup{pdfauthor={Niklas Beisert}}
\hypersetup{pdfsubject={Manual for the LaTeX2e Package childdoc}}
\date{30 December 2018, \textsf{v2.0}}
\maketitle

\begin{abstract}\noindent
\textsf{childdoc} is a \LaTeXe{} package
that enables the direct compilation
of document sections included by |\include|
to individual files.
\end{abstract}

\begingroup
\parskip0ex
\tableofcontents
\endgroup

%%%%%%%%%%%%%%%%%%%%%%%%%%%%%%%%%%%%%%%%%%%%%%%%%%%%%%%%%%%%%%%%%%%%%%%%%%%%%%%%
%%%%%%%%%%%%%%%%%%%%%%%%%%%%%%%%%%%%%%%%%%%%%%%%%%%%%%%%%%%%%%%%%%%%%%%%%%%%%%%%
\section{Introduction}

\LaTeX{} provides a mechanism to structure a large document (such as a book)
into a main file and several child files (containing the chapters)
using the |\include| command.
This mechanism is beneficial for documents
which span hundreds of pages in order to
make the source file(s) more manageable.
Moreover, compilation can be restricted to
selected child files by means of the |\includeonly| command.
The latter feature can be used to reduce the compilation time while editing
(this was significantly more useful in the earlier days of \LaTeX{})
or to generate a smaller document which is easier to navigate.
Another application of |\includeonly| is to generate
documents consisting of selected parts of the complete document.

However, there are a few drawbacks of the plain |\include| mechanism:
\begin{itemize}
\item
The child files cannot be compiled on their own,
they can only be compiled via the main file.
A naive editing environment
(such as a text editor with an option
to have the current file processed by \LaTeX)
may require one to switch to the main file before compiling;
attempting to compile the child file produces errors.
\item
The main file must be modified (each time)
to adjust the |\includeonly| command
to the present needs. This easily leaves the main file in a messy state.
\item
The generated document will always carry the filename
of the main document. This is inconvenient if
several child files are to be compiled and
to be kept for distribution.
\end{itemize}

The present package provides a simple interface
to make child files individually compilable by \LaTeX{}.
Compiling a child file then has the same effect as compiling
the main file with an |\includeonly| command
to select the appropriate child.
Moreover the generated document will carry the name of the child
rather than the main file.
This resolves all three above issues.

This feature is meant to make the editing of books,
thesis documents and lecture notes somewhat more convenient.
However, the package can also be used efficiently for
composing a series of documents (such as exercise sheets)
which are typically distributed individually.
It then assists the author in generating the individual documents
(potentially in different versions)
as well as a document containing the collected series.
Another application is in developing style files
or other kinds of included material
where compilation of the style file could redirect
to a sample or test file.

%%%%%%%%%%%%%%%%%%%%%%%%%%%%%%%%%%%%%%%%%%%%%%%%%%%%%%%%%%%%%%%%%%%%%%%%%%%%%%%%
%%%%%%%%%%%%%%%%%%%%%%%%%%%%%%%%%%%%%%%%%%%%%%%%%%%%%%%%%%%%%%%%%%%%%%%%%%%%%%%%
\section{Usage}

First of all, the package \textsf{childdoc} is \emph{not} a standard
\LaTeXe{} |.sty| style file! Therefore it needs to be invoked in
a non-standard way.

%%%%%%%%%%%%%%%%%%%%%%%%%%%%%%%%%%%%%%%%%%%%%%%%%%%%%%%%%%%%%%%%%%%%%%%%%%%%%%%%
\subsection{Included Files}
\label{sec:include}

%%%%%%%%%%%%%%%%%%%%%%%%%%%%%%%%%%%%%%%%
\DescribeMacro{\childdocmain}
To use the package, add the commands
\begin{center}
\begin{tabular}{l}
|\input{childdoc.def}|\\
|\childdocmain{}|\\
\end{tabular}
\end{center}
at the very top of the main \LaTeX{} file,
in particular \emph{before} the |\documentclass| statement!
The argument of |\childdocmain| should be left empty
(but it must be present).

%%%%%%%%%%%%%%%%%%%%%%%%%%%%%%%%%%%%%%%%
\DescribeMacro{\childdocof}
Furthermore, add the commands
\begin{center}
\begin{tabular}{l}
|\input{childdoc.def}|\\
|\childdocof{|\textit{main}|}|\\
\end{tabular}
\end{center}
at the top of every child file \textit{child}
which is included by |\include{|\textit{child}|}|
from within the main file
(or at least for those files to be compiled individually).
The argument \textit{main} must be the filename of the main file.

There are a couple of
considerations in setting up the main and child documents:

%%%%%%%%%%%%%%%%%%%%%%%%%%%%%%%%%%%%%%%%
\paragraph{Restrictions.}

Please note the following restrictions:
\begin{itemize}
\item
|\childdocmain| must be called with one argument \textit{main}
to ensure compatibility with earlier version of the package.
It must either be empty (|\childdocmain{}|)
or precisely match the filename of the main file in which it is specified.
See \secref{sec:detection} for further information.
\item
The filename \textit{main} must be specified without the |.tex| extension.
\item
The filename \textit{main} is case sensitive
(even in case-insensitive file systems)
due to internal string comparison.
\item
The argument \textit{main} should be fully expanded, it cannot be a macro.
\item
Subdirectories and special characters should be avoided in filenames.
\item
The command |\childdocmain{|\textit{main}|}| must be followed by a whitespace.
It should not be followed immediately by another command
or by a comment mark `|%|'.
This is because the \TeX{} parser reads the token immediately following
the argument of |\childdocmain| and puts it
at the beginning of every child section;
however, a white\-space is ignored.
\end{itemize}

%%%%%%%%%%%%%%%%%%%%%%%%%%%%%%%%%%%%%%%%
\paragraph{Content of Main File.}

It is advisable to place all content in the child files included by |\include|.
Any output contained in the main file will appear in all child documents
unless suppressed manually;
it cannot be suppressed automatically by the |\includeonly| directive
and thus should normally be avoided.
A method to include some content in the main file
by means of conditional processing is described in \secref{sec:conditional}.

%%%%%%%%%%%%%%%%%%%%%%%%%%%%%%%%%%%%%%%%
\paragraph{Page Numbering.}

When only a part of the document is compiled,
the appropriate numbering of pages
(as well as other status parameters)
is determined from the |.aux| files.
The latter contain information from previous passes.
However this information needs to propagate through
all intermediate child documents.
Therefore the page numbering in child documents may well
be inconsistent until the complete document is compiled at least once.

A useful (if unconventional) way to always ensure a consistent
page numbering is to restart the numbering in each child document
and denote the pages by `\textit{child}|.|\textit{page}'
where \textit{child} represents the chapter/section number of the child file.
This can be achieved by the command
|\numberwithin{page}{|\textit{child}|}|
of the \textsf{amsmath} package
where \textit{child} can be |chapter| or |section|
depending on the chosen structuring.
Alternatively, one can modify the macro |\thepage| appropriately
and reset the counter |page| at the start of each child file.

%%%%%%%%%%%%%%%%%%%%%%%%%%%%%%%%%%%%%%%%%%%%%%%%%%%%%%%%%%%%%%%%%%%%%%%%%%%%%%%%
\subsection{Conditional Processing}
\label{sec:conditional}

The package provides a mechanism to compile different versions
of a document. To customise the versions further some conditional processing
can come in handy to distinguish which version is being compiled.
The package provides two macros to describe the compilation context:

%%%%%%%%%%%%%%%%%%%%%%%%%%%%%%%%%%%%%%%%
\DescribeMacro{\ifchilddoc}
The conditional |\ifchilddoc| distinguishes between the compilation of
child documents and the main document:
%
\begin{center}
|\ifchilddoc |\textit{child-code}| |[|\||else |\textit{main-code}]| \||fi|
\end{center}

%%%%%%%%%%%%%%%%%%%%%%%%%%%%%%%%%%%%%%%%
\DescribeMacro{\childdocname}
\DescribeMacro{\childdocjob}
The macro |\childdocname| contains the filename (without extension)
of the main or child file being processed.
Note that |\childdocjob| will always contain the name of the main file.

%%%%%%%%%%%%%%%%%%%%%%%%%%%%%%%%%%%%%%%%
\paragraph{Title Page.}

Conditional processing can be used to include a title or banner page
in the main document when proper precautions are taken.
Importantly, the code in the main file should ensure that the page counter
(as well as other status parameters which are stored in the |.aux| files)
takes the same value after the conditional processing.
Otherwise the page numbers may take divergent values
depending on which part is compiled.

For example, a title page could be declared by:
%
\begin{center}
\begin{tabular}{l}
|\ifchilddoc\||else|\\
|\addtocounter{page}{-1}|\\
\textit{code for title page}\\
|\newpage|\\
|\||fi|
\end{tabular}
\end{center}
%
A banner page for the child documents can be generated by:
%
\begin{center}
\begin{tabular}{l}
|\ifchilddoc|\\
|\addtocounter{page}{-1}|\\
\textit{code for banner page}\\
|\newpage|\\
|\||fi|
\end{tabular}
\end{center}
%
Here one could write a message such as:
\begin{center}
|This is the part \childdocname{} of \childdocjob{}.|
\end{center}

%%%%%%%%%%%%%%%%%%%%%%%%%%%%%%%%%%%%%%%%%%%%%%%%%%%%%%%%%%%%%%%%%%%%%%%%%%%%%%%%
\subsection{Flags}
\label{sec:flags}

The package makes it easy to generate different versions
of the main or child documents.
To this end compilation flags can be defined
and assigned different default values.
They will be particularly useful in conjunction
with the forwarding mechanism described in \secref{sec:forward}.

For example, it may be useful to have a flag |\version|
which can be set to |draft| or |final|.
The document source will contain some conditional code
depending on the value of |\version|.
Suppose further, the flag should default to |final| for the main file
and to |draft| for child files
which is a natural assignment for editing the document.
This is achieved by placing the following code
in the preamble of the main document
(below the |\childdocmain| directive):
%
\begin{center}
\begin{tabular}{l}
|\ifchilddoc|\\
|\providecommand{\version}{draft}|\\
|\||else|\\
|\providecommand{\version}{final}|\\
|\||fi|
\end{tabular}
\end{center}
%
The definition by |\providecommand| makes sure
that previous definitions are not overwritten.
Further statements |\providecommand{\version}{...}|
can thus be added before the above code to override it.

For the main file, one might add a line
(between |\childdocmain| and the above block)
%
\begin{center}
|%\ifchilddoc\||else\providecommand{\version}{draft}\||fi|
\end{center}
%
which can be uncommented to produce a draft version.
Likewise one can add a line to the very top of a child file
(above the |\childdocof{|\textit{main}|}| directive)
%
\begin{center}
|%\providecommand{\version}{final}|
\end{center}
%
which can be uncommented to produce the final version of this child document.

%%%%%%%%%%%%%%%%%%%%%%%%%%%%%%%%%%%%%%%%%%%%%%%%%%%%%%%%%%%%%%%%%%%%%%%%%%%%%%%%
\subsection{Forwarding}
\label{sec:forward}

Different versions of the main or child documents
using compilation flags as described in \secref{sec:flags}
can be (permanently) stored in different files
for convenient compilation, viewing and distribution.
To this end, the package defines a command
to pass on compilation to a different file:

%%%%%%%%%%%%%%%%%%%%%%%%%%%%%%%%%%%%%%%%
\DescribeMacro{\childdocforward}
The command |\childdocforward| redirects processing to
another source file:
%
\begin{center}
\begin{tabular}{l}
|\input{childdoc.def}|\\
|\childdocforward[|\textit{main}|]{|\textit{dest}|}|\\
\end{tabular}
\end{center}
%
The argument \textit{dest} is the destination file
(without extension).
It should be the main file or one of the child files.
Note that further \textsf{childdoc} directives
such as |\childdocof| and |\childdocforward|
in the indicated file will be processed in this form.
The optional argument \textit{main}
passes on directly to the main file \textit{main}
while pretending to compile the child \textit{dest}.
This form behaves as if \textit{dest}
issues |\childdocof{|\textit{main}|}| right away,
and no further \textsf{childdoc} directives will be processed.

%%%%%%%%%%%%%%%%%%%%%%%%%%%%%%%%%%%%%%%%
\DescribeMacro{\...prefix}
In the alternative form |\childdocforwardprefix|,
%
\begin{center}
\begin{tabular}{l}
|\input{childdoc.def}|\\
|\childdocforwardprefix[|\textit{main}|]{|\textit{prefix}|}{|\textit{dest}|}|
\end{tabular}
\end{center}
%
the destination file is determined by a pattern
depending on the current file:
To make this work, the current file must be called
`{\textit{prefix}\hspace{0.2em}\textit{suffix}}'
with \textit{prefix} matching precisely the argument.
Processing is then passed on to the file
`{\textit{dest}\hspace{0.2em}\textit{suffix}}'.
Surely, the same effect is achieved by
directly specifying the
argument `{\textit{dest}\hspace{0.2em}\textit{suffix}}'
in the first form.
However, that requires to set up a different file
for each child. With the alternative form of the command
all these files can have exactly the same content
which simplifies setting them up and maintaining them.

For example, the following file |draft.tex|
with a compilation flag |\version| as described in \secref{sec:flags}
compiles the main document as a draft:
%
\begin{center}
\begin{tabular}{l}
|\def\version{draft}|\\
|\input{childdoc.def}|\\
|\childdocforward{|\textit{main}|}|
\end{tabular}
\end{center}
%
Likewise, the following files |final|\textit{nn}|.tex|
compile the final version of the child document
|child|\textit{nn}|.tex|:
%
\begin{center}
\begin{tabular}{l}
|\def\version{final}|\\
|\input{childdoc.def}|\\
|\childdocforwardprefix{final}{child}|
\end{tabular}
\end{center}
%

Note that when several versions of a main file and/or of each child file
are to be generated, it may be convenient to set up a |Makefile| or
shell script to automatise the process.

%%%%%%%%%%%%%%%%%%%%%%%%%%%%%%%%%%%%%%%%%%%%%%%%%%%%%%%%%%%%%%%%%%%%%%%%%%%%%%%%
\subsection{Command Line Processing}
\label{sec:commandline}

The effect of redirection files can also be achieved by invoking
the \LaTeX{} compiler with a more elaborate command line.
Most conveniently this should be done as part
of a shell script or a |Makefile|.

When using \textsf{childdoc} in the main file, the following
command lines effectively perform a redirection
(note that depending on the shell being used,
backslashes may have to be doubled: `|\|' $\to$ `|\\|'):
%
\begin{center}
|... -jobname "|\textit{target}|" |\\|"|[\textit{flags}]%
|\input{childdoc.def}\childdocforward[|\textit{main}|]{|\textit{dest}|}"|
\end{center}
%
Here \textit{target} is the name of the output file,
\textit{main} is the name of the main file
and \textit{dest} is the name of the main or child file to be processed
(all filenames without extensions).
The optional argument \textit{main} can be omitted
if \textit{main} matches \textit{dest}.
Optionally, compilation \textit{flags} can be defined via |\def| commands.
This command line makes the \TeX{} engine believe
it is compiling the file \textit{target}
whose content is specified as the latter parameter.
The provided code then forwards the processing to
\textit{main} or \textit{dest} as described in \secref{sec:forward}.

%%%%%%%%%%%%%%%%%%%%%%%%%%%%%%%%%%%%%%%%%%%%%%%%%%%%%%%%%%%%%%%%%%%%%%%%%%%%%%%%
\subsection{Include by Input}
\label{sec:input}

Including child documents by |\include| has some restrictions by design.
Most notably, the content of a child document always occupies
its own set of pages; pages cannot be shared between child documents.
Usually, this behaviour makes perfect sense
because each child document contain an essential part of the document.
However, in some situations it may be desirable to compose
a document from a collection of parts
without having mandatory page breaks between then.
For this case, the package
provides a mechanism to include parts
by |\input| which can also be processed individually.
However, by construction this mechanism
requires manual handling of the content to be output.

%%%%%%%%%%%%%%%%%%%%%%%%%%%%%%%%%%%%%%%%
\DescribeMacro{\ifchilddocmanual}
The main file should be prepared as usual, see \secref{sec:include}.
However, the document body must make a distinction
between processing of an individual part and of the main document, e.g.:
%
\begin{center}
\begin{tabular}{l}
|\ifchilddocmanual|\\
|\input{\childdocname}|\\
|\||else|\\
\textit{document body with }|\input{|\textit{part}|}|\\
|\||fi|
\end{tabular}
\end{center}
%
The conditional |\ifchilddocmanual| is true whenever
a part to be included by |\input| is being compiled,
and the name of the part is stored in |\childdocname|.

%%%%%%%%%%%%%%%%%%%%%%%%%%%%%%%%%%%%%%%%
\DescribeMacro{\childdocby}
Each part to be included by |\input| should start with:
%
\begin{center}
\begin{tabular}{l}
|\input{childdoc.def}|\\
|\childdocby{|\textit{main}|}|\\
\end{tabular}
\end{center}
%
The directive |\childdocby| is similar to |\childdocof|
described in \secref{sec:include},
but the subsequent selection of content must be done manually.
To that end, both |\ifchilddoc| and |\ifchilddocmanual|
will be true upon processing of a part,
and the name of the part is stored in |\childdocname|.
Note that |\jobname| will be set to the filename of the current part
so that each part receives an individual |.aux| file
that does not interfere with the |.aux| file(s) of the main document.
This behaviour can be altered by the alternative form
|\childdocby[*]{|\textit{main}|}| (with a non-empty optional argument)
which uses the |.aux| file of the main document
by setting |\jobname| to \textit{main}.

%%%%%%%%%%%%%%%%%%%%%%%%%%%%%%%%%%%%%%%%%%%%%%%%%%%%%%%%%%%%%%%%%%%%%%%%%%%%%%%%
\subsection{Driver Development}
\label{sec:driver}

The \textsf{childdoc} mechanism can also be use for the development
of definition files such as \LaTeX{} styles or classes.
This case differs from the above setup with multiple parts
included by |\include| in that no |\includeonly| should be invoked.
This can be achieved by starting the include file
(before |\ProvidesPackage|) with:
%
\begin{center}
\begin{tabular}{l}
|\input{childdoc.def}|\\
|\childdocforward{|\textit{main}|}|\\
\end{tabular}
\end{center}
%
or alternatively with:
%
\begin{center}
\begin{tabular}{l}
|\input{childdoc.def}|\\
|\childdocby{|\textit{main}|}|\\
\end{tabular}
\end{center}
%
Both forms have slightly different effects as described above.
The main file is prepared as usual, see \secref{sec:include}.

%%%%%%%%%%%%%%%%%%%%%%%%%%%%%%%%%%%%%%%%%%%%%%%%%%%%%%%%%%%%%%%%%%%%%%%%%%%%%%%%
\subsection{Legacy Detection}
\label{sec:detection}

The directive |\childdocmain| in the main file can detect
whether the complete document or merely a child is to be compiled
even without using the directive |\childdocof|.
This method is deprecated because it is less robust
and there is no compelling reason to use it;
it is merely provided for backward compatibility
and it may be removed in future versions.

If the detection mechanism is to be used,
it is mandatory to correctly specify
the filename of the main file as the argument of |\childdocmain|:
%
\begin{center}
\begin{tabular}{l}
|\input{childdoc.def}|\\
|\childdocmain{|\textit{main}|}|\\
\end{tabular}
\end{center}
%
If |\jobname| does not match the argument \textit{main} of |\childdocmain|,
it is assumed that |\jobname| points to the child file to be compiled.
When using |\childdocmain| with the main file specified as argument,
it suffices to start a child file
with just |\input{|\textit{main}|}|
without loading of the package and using |\childdocof|.
If instead all processing is done
with the appropriate \textsf{childdoc} directives,
the argument of \textit{main} of |\childdocmain| can be empty.

An alternative version of the command line processing described
in \secref{sec:commandline} using the detection mechanism reads:
%
\begin{center}
|... -jobname "|\textit{target}|" "|[\textit{flags}]%
[|\def\jobname{|\textit{dest}|}|]|\input{|\textit{main}|}"|
\end{center}

%%%%%%%%%%%%%%%%%%%%%%%%%%%%%%%%%%%%%%%%%%%%%%%%%%%%%%%%%%%%%%%%%%%%%%%%%%%%%%%%
\subsection{Manual Code}
\label{sec:manual}

In case one cannot be certain whether the definitions file |childdoc.def|
is installed on the target \TeX{} distribution
and one prefers not to ship it,
it is conceivable to paste a few relevant commands into the sources.

To that end, drop all statements |\input{childdoc.def}|
and perform the replacements as outlined below.
Instead of |\childdocmain{|\textit{main}|}| add the following code
to the top of the main file:
%
\begin{center}
\begin{tabular}{l}
|\||ifdefined\childdocname\endinput\||fi\newif\ifchilddoc|\\
|\edef\childdocname{\scantokens\expandafter{\jobname\noexpand}}|\\
|\def\childdocmain{|\textit{main}|}\||ifx\childdocmain\childdocname\||else|\\
|\childdoctrue\includeonly{\childdocname}\let\jobname\childdocmain\||fi|\\
\end{tabular}
\end{center}
%
Instead of |\childdocof{|\textit{main}|}| just include the main file
at the top of each child file:
%
\begin{center}
|\input{|\textit{main}|}|
\end{center}
%
A simple redirection |\childdocforward{|\textit{dest}|}| is achieved by:
%
\begin{center}
|\def\jobname{|\textit{dest}|}\input{\jobname}|
\end{center}
%
The redirection with prefix
|\childdocforwardprefix[|\textit{prefix}|]{|\textit{dest}|}|
is accomplished by:
%
\begin{center}
\begin{tabular}{l}
|{\edef\jobname{\scantokens\expandafter{\jobname\noexpand}}|\\
|\def\redirectjob |\textit{prefix}|#1~~~{\gdef\jobname{|\textit{dest}|#1}}|\\
|\expandafter\redirectjob\jobname~~~}\input{\jobname}|
\end{tabular}
\end{center}

In an alternative approach,
child documents can be compiled by a specific command line
without additional code or specific definitions:
%
\begin{center}
|... -jobname "|\textit{target}|" "|[\textit{flags}]%
|\includeonly{|\textit{dest}|}\input{|\textit{main}|}"|
\end{center}
%

%%%%%%%%%%%%%%%%%%%%%%%%%%%%%%%%%%%%%%%%%%%%%%%%%%%%%%%%%%%%%%%%%%%%%%%%%%%%%%%%
%%%%%%%%%%%%%%%%%%%%%%%%%%%%%%%%%%%%%%%%%%%%%%%%%%%%%%%%%%%%%%%%%%%%%%%%%%%%%%%%
\section{Information}

%%%%%%%%%%%%%%%%%%%%%%%%%%%%%%%%%%%%%%%%%%%%%%%%%%%%%%%%%%%%%%%%%%%%%%%%%%%%%%%%
\subsection{Copyright}

Copyright \copyright{} 2017--2018 Niklas Beisert

This work may be distributed and/or modified under the
conditions of the \LaTeX{} Project Public License, either version 1.3
of this license or (at your option) any later version.
The latest version of this license is in
  \url{http://www.latex-project.org/lppl.txt}
and version 1.3 or later is part of all distributions of \LaTeX{}
version 2005/12/01 or later.

This work has the LPPL maintenance status `maintained'.

The Current Maintainer of this work is Niklas Beisert.

This work consists of the files |README.txt|, |childdoc.ins| and |childdoc.dtx|
as well as the derived files |childdoc.def|, |cdocsamp.tex|
with |cdocsch1.tex|, |cdocsch2.tex|, |cdocspt3.tex|, |cdocspt4.tex|,
|cdocsdrf.tex|, |cdocsfn1.tex|, |cdocsfn2.tex|
as well as |childdoc.pdf|.

%%%%%%%%%%%%%%%%%%%%%%%%%%%%%%%%%%%%%%%%%%%%%%%%%%%%%%%%%%%%%%%%%%%%%%%%%%%%%%%%
\subsection{Files and Installation}

The package consists of the files:
%
\begin{center}
\begin{tabular}{ll}
    |README.txt|   & readme file \\
    |childdoc.ins| & installation file \\
    |childdoc.dtx| & source file \\
    |childdoc.def| & definition file \\
    |cdocsamp.tex| & sample main file \\
    |cdocsch1.tex| & sample include file \\
    |cdocsch2.tex| & sample include file \\
    |cdocspt3.tex| & sample part file \\
    |cdocspt4.tex| & sample part file \\
    |cdocsdrf.tex| & sample redirection file \\
    |cdocsfn1.tex| & sample redirection file \\
    |cdocsfn2.tex| & sample redirection file \\
    |childdoc.pdf| & manual
\end{tabular}
\end{center}
%
The distribution consists of the files
|README.txt|, |childdoc.ins| and |childdoc.dtx|.
%
\begin{itemize}
\item
Run (pdf)\LaTeX{} on |childdoc.dtx|
to compile the manual |childdoc.pdf| (this file).
\item
Run \LaTeX{} on |childdoc.ins| to create the definitions file |childdoc.def|
and the sample |cdocsamp.tex| with include files
|cdocsch1.tex|, |cdocsch2.tex|, |cdocspt3.tex|, |cdocspt4.tex|,
|cdocsdrf.tex|, |cdocsfn1.tex|, |cdocsfn2.tex|.
Then copy the file |childdoc.def| to an appropriate directory of your \LaTeX{}
distribution, e.g.\ \textit{texmf-root}|/tex/latex/childdoc|.
\end{itemize}

%%%%%%%%%%%%%%%%%%%%%%%%%%%%%%%%%%%%%%%%%%%%%%%%%%%%%%%%%%%%%%%%%%%%%%%%%%%%%%%%
\subsection{Related CTAN Packages}

There are several other packages which offer a similar functionality:
%
\begin{itemize}
\item
The packages
\href{http://ctan.org/pkg/docmute}{\textsf{docmute}},
\href{http://ctan.org/pkg/includex}{\textsf{includex}} and
\href{http://ctan.org/pkg/standalone}{\textsf{standalone}}
provide commands to include only the document body of
a child file thus allowing both files to be compiled individually.
\item
The packages \href{http://ctan.org/pkg/subdocs}{\textsf{subdocs}}
and \href{http://ctan.org/pkg/subfiles}{\textsf{subfiles}}
provide structures in which the main and child documents can be
encapsulated and allowing them to be compiled individually.
The inclusion mechanism is different from the conventional |\include|.
\item
The package \href{http://ctan.org/pkg/combine}{\textsf{combine}}
is an elaborate solution to combine several documents into one.
\end{itemize}
%
See also the CTAN topic \href{http://ctan.org/topic/subdocs}{\textsf{subdocs}}
for further related packages.
The present package differs from the above solutions in that
a document structure constructed with the conventional |\include| mechanism
just needs two extra commands at the top of every file
such that all constituent files can be compiled individually.

%%%%%%%%%%%%%%%%%%%%%%%%%%%%%%%%%%%%%%%%%%%%%%%%%%%%%%%%%%%%%%%%%%%%%%%%%%%%%%%%
%\subsection{Feature Suggestions}
%
%The following is a list of features which may be useful for future
%versions of this package:
%%
%\begin{itemize}
%\item
%\ldots
%\end{itemize}

%%%%%%%%%%%%%%%%%%%%%%%%%%%%%%%%%%%%%%%%%%%%%%%%%%%%%%%%%%%%%%%%%%%%%%%%%%%%%%%%
\subsection{Revision History}

%%%%%%%%%%%%%%%%%%%%%%%%%%%%%%%%%%%%%%%%
\paragraph{v2.0:} 2018/12/30

\begin{itemize}
\item
immediate forward processing
\item
added |\childdocby| mechanism
\item
manual restructured
\end{itemize}

%%%%%%%%%%%%%%%%%%%%%%%%%%%%%%%%%%%%%%%%
\paragraph{v1.6:} 2018/01/17

\begin{itemize}
\item
application for development of include files
\item
corrections to manual
\end{itemize}

%%%%%%%%%%%%%%%%%%%%%%%%%%%%%%%%%%%%%%%%
\paragraph{v1.5:} 2017/05/21

\begin{itemize}
\item
more complete structuring introduced
\item
|\childdocof| introduced
\item
|\childdoc| renamed to |\childdocmain|
\item
|\childredirect| renamed to |\childdocforward| and |\childdocforwardprefix|
and functionality expanded
\end{itemize}

%%%%%%%%%%%%%%%%%%%%%%%%%%%%%%%%%%%%%%%%
\paragraph{v1.0:} 2017/04/27

\begin{itemize}
\item
manual and install package
\item
first version published on CTAN
\end{itemize}

%%%%%%%%%%%%%%%%%%%%%%%%%%%%%%%%%%%%%%%%
\paragraph{v0.6:} 2017/04/26

\begin{itemize}
\item
redirection mechanism added
\end{itemize}

%%%%%%%%%%%%%%%%%%%%%%%%%%%%%%%%%%%%%%%%
\paragraph{v0.5:} 2017/04/26

\begin{itemize}
\item
functionality in definition file
\end{itemize}


%%%%%%%%%%%%%%%%%%%%%%%%%%%%%%%%%%%%%%%%%%%%%%%%%%%%%%%%%%%%%%%%%%%%%%%%%%%%%%%%
%%%%%%%%%%%%%%%%%%%%%%%%%%%%%%%%%%%%%%%%%%%%%%%%%%%%%%%%%%%%%%%%%%%%%%%%%%%%%%%%
%%%%%%%%%%%%%%%%%%%%%%%%%%%%%%%%%%%%%%%%%%%%%%%%%%%%%%%%%%%%%%%%%%%%%%%%%%%%%%%%
\appendix

\settowidth\MacroIndent{\rmfamily\scriptsize 000\ }

 \DocInput{childdoc.dtx}

\end{document}
%</driver>
% \fi
%
% %%%%%%%%%%%%%%%%%%%%%%%%%%%%%%%%%%%%%%%%%%%%%%%%%%%%%%%%%%%%%%%%%%%%%%%%%%%%%%
% %%%%%%%%%%%%%%%%%%%%%%%%%%%%%%%%%%%%%%%%%%%%%%%%%%%%%%%%%%%%%%%%%%%%%%%%%%%%%%
% \section{Sample}
%\iffalse
%<*samplemain>
%\fi
%
% The following presents a sample document
% with two chapters, two parts, a title page,
% a compile flag as well as three forwarding files to set the flag.
% It consists of eight |.tex| files:
% \begin{center}
% \begin{tabular}{ll}
% |cdocsamp.tex|&main file\\
% |cdocsch1.tex|&include file for chapter 1\\
% |cdocsch2.tex|&include file for chapter 2\\
% |cdocspt3.tex|&include file for part 3\\
% |cdocspt4.tex|&include file for part 4\\
% |cdocsdrf.tex|&forwarding file for main file in draft mode\\
% |cdocsfi1.tex|&forwarding file for final version of chapter 1\\
% |cdocsfi2.tex|&forwarding file for final version of chapter 2\\
% \end{tabular}
% \end{center}
% Each of the eight files can be compiled directly by the \LaTeX{} compiler.
%
% %%%%%%%%%%%%%%%%%%%%%%%%%%%%%%%%%%%%%%
% \paragraph{Main File.}
%
% The main file is called |cdocsamp.tex|.
%
% Load the \textsf{childdoc} definitions and
% declare the filename for the main document:
%    \begin{macrocode}
\input{childdoc.def}
\childdocmain{}
%    \end{macrocode}

% Optional override for |\version| flag:
%    \begin{macrocode}
%%\ifchilddoc\else\providecommand{\version}{draft}\fi
%    \end{macrocode}

% Define the default values for the |\version| flag
% (|final| for the main file and |draft| for childs):
%    \begin{macrocode}
\ifchilddoc
\providecommand{\version}{draft}
\else
\providecommand{\version}{final}
\fi
%    \end{macrocode}

% Load the standard document class:
%    \begin{macrocode}
\documentclass[12pt]{article}
%    \end{macrocode}

% Start the document body:
%    \begin{macrocode}
\begin{document}
%    \end{macrocode}

% Declare a title page.
% Print title, part of document being processed and version flag:
%    \begin{macrocode}
\addtocounter{page}{-1}
\begin{center}
{\LARGE\bfseries{}childdoc example\par}
\vspace{1cm}
\ifchilddoc
\ifchilddocmanual part\else chapter\fi:
`\childdocname' of `\childdocjob'\par
\else
main document: `\childdocjob'\par
\fi
version: \version\par
\end{center}
\newpage
%    \end{macrocode}

% Manually include selected file,
% otherwise process as usual:
%    \begin{macrocode}
\ifchilddocmanual
\section*{part `\childdocname'}
\input{\childdocname}
\else
%    \end{macrocode}

% Include the two chapters:
%    \begin{macrocode}
\include{cdocsch1}
\include{cdocsch2}
%    \end{macrocode}

% Include the two parts unless only chapters should be displayed:
%    \begin{macrocode}
\ifchilddoc\else
\section{part three}
\input{cdocspt3}
\section{part four}
\input{cdocspt4}
\fi
%    \end{macrocode}

% Process as usual until here:
%    \begin{macrocode}
\fi
%    \end{macrocode}

% End of document body:
%    \begin{macrocode}
\end{document}
%    \end{macrocode}
%\iffalse
%</samplemain>
%\fi
%
% %%%%%%%%%%%%%%%%%%%%%%%%%%%%%%%%%%%%%%
% \paragraph{Chapter Include Files.}
%
% The include files are called |cdocsch1.tex| and |cdocsch2.tex|.
%
%\iffalse
%<*samplechap1|samplechap2>
%\fi

% Optional override for |\version| flag:
%    \begin{macrocode}
%%\providecommand{\version}{final}
%    \end{macrocode}

% Include the main document:
%    \begin{macrocode}
\input{childdoc.def}
\childdocof{cdocsamp}
%    \end{macrocode}

%\iffalse
%</samplechap1|samplechap2>
%\fi
%
%\iffalse
%<*samplechap1>
%\fi
% Some text for chapter 1:
%    \begin{macrocode}
\section{one}
some text in chapter one
%    \end{macrocode}

%\iffalse
%</samplechap1>
%\fi
% Some text for chapter 2:
%\iffalse
%<*samplechap2>
%\fi
%    \begin{macrocode}
\section{two}
more text in chapter two
%    \end{macrocode}

%\iffalse
%</samplechap2>
%\fi
%
% %%%%%%%%%%%%%%%%%%%%%%%%%%%%%%%%%%%%%%
% \paragraph{Part Include Files.}
%
% The include files are called |cdocspt3.tex| and |cdocspt4.tex|.
%
%\iffalse
%<*samplepart3|samplepart4>
%\fi

% Optional override for |\version| flag:
%    \begin{macrocode}
%%\providecommand{\version}{final}
%    \end{macrocode}

% Include the main document:
%    \begin{macrocode}
\input{childdoc.def}
\childdocby{cdocsamp}
%    \end{macrocode}

%\iffalse
%</samplepart3|samplepart4>
%\fi
%
%\iffalse
%<*samplepart3>
%\fi
% Some text for part 3:
%    \begin{macrocode}
some text in part three
%    \end{macrocode}

%\iffalse
%</samplepart3>
%\fi
% Some text for part 4:
%\iffalse
%<*samplepart4>
%\fi
%    \begin{macrocode}
more text in part four
%    \end{macrocode}

%\iffalse
%</samplepart4>
%\fi
%
% %%%%%%%%%%%%%%%%%%%%%%%%%%%%%%%%%%%%%%
% \paragraph{Forwarding for a Complete Draft.}
%
% The following forwarding file |cdocsdrf.tex|
% compiles the main document in draft mode:
%\iffalse
%<*sampledraft>
%\fi
%    \begin{macrocode}
\def\version{draft}
\input{childdoc.def}
\childdocforward{cdocsamp}
%    \end{macrocode}

%\iffalse
%</sampledraft>
%\fi
%
% %%%%%%%%%%%%%%%%%%%%%%%%%%%%%%%%%%%%%%
% \paragraph{Forwarding for Final Version of the Chapters.}
%
% The following forwarding files |cdocsfn1.tex| and |cdocsfn2.tex|
% (with identical content)
% compile the final versions of the child documents
% |cdocsch1.tex| and |cdocsch2.tex|, respectively:
%\iffalse
%<*samplefinal>
%\fi
%    \begin{macrocode}
\def\version{final}
\input{childdoc.def}
\childdocforwardprefix[cdocsamp]{cdocsfn}{cdocsch}
%    \end{macrocode}

%\iffalse
%</samplefinal>
%\fi
%
% %%%%%%%%%%%%%%%%%%%%%%%%%%%%%%%%%%%%%%
% \paragraph{Command Line Processing.}
%
% The following three command lines generate the output files
% |cdocscld|, |cdocscl1| and |cdocscl2|
% which should be identical to
% |cdocsdrf|, |cdocsch1| and |cdocsfn2|, respectively:
% \begin{center}
% \begin{tabular}{l}
% |latex -jobname cdocscld \|\\
% |  "\def\version{draft}\input{childdoc.def}\childdocforward{cdocsamp}"|\\
% |latex -jobname cdocscl1 \|\\
% |  "\input{childdoc.def}\childdocforward[cdocsamp]{cdocsch1}"|\\
% |latex -jobname cdocscl2 \|\\
% |  "\def\version{final}\input{childdoc.def}\childdocforward{cdocsch2}"|
% \end{tabular}
% \end{center}
% Note that the trailing backslash on each first line
% merely continues the input to the second line
% (for convenient cut ant paste).
% Furthermore, the command |latex| can be replaced by any
% of its alternative versions such as |pdflatex|.
%
% %%%%%%%%%%%%%%%%%%%%%%%%%%%%%%%%%%%%%%%%%%%%%%%%%%%%%%%%%%%%%%%%%%%%%%%%%%%%%%
% %%%%%%%%%%%%%%%%%%%%%%%%%%%%%%%%%%%%%%%%%%%%%%%%%%%%%%%%%%%%%%%%%%%%%%%%%%%%%%
% \section{Implementation}
%\iffalse
%<*package>
%\fi
%
% This section describes the definitions file |childdoc.def|.

% The definitions cannot be loaded using |\usepackage| or |\RequirePackage|
% which has a mechanism to prevent loading a style file more than once.
% When loading the definitions by means of |\input|
% multiple instances have to be prevented manually:
%\iffalse
%This code needs to be before the `\ProvidesFile' directive
%which is defined at the beginning of this file.
%Therefore it is also placed there and commented out here.
%</package>
%<*discard>
%\fi
%    \begin{macrocode}
\ifdefined\childdocmain\endinput\fi
%    \end{macrocode}
%\iffalse
%</discard>
%<*package>
%\fi
%
% \macro{\ifchilddoc}
% \macro{\ifchilddocmanual}
% The conditional |\ifchilddoc| tells whether a
% child (true) or main (false) document is being compiled.
% The conditional |\ifchilddocmanual| tells whether
% the |\includeonly| mechanism is used (false) or
% the selection of child files must be performed manually (true).
% The definitions initialise to false:
%    \begin{macrocode}
\newif\ifchilddoc
\newif\ifchilddocmanual
%    \end{macrocode}

% \macro{\childdocname}
% \macro{\childdocjob}
% The macro |\childdocname| stores the name of the main document
% to be compiled. The macro |\childdocjob| stores the name of
% the document on which the \LaTeX{} compiler was originally invoked.
% The content of |\jobname| cannot be compared
% to filenames specified in the source due to different catcodes.
% The following code rescans |\jobname|, stores the result
% in |\childdocname| and saves a copy in |\childdocjob|:
%    \begin{macrocode}
\edef\childdocname{\scantokens\expandafter{\jobname\noexpand}}
\let\childdocjob\childdocname
%    \end{macrocode}

% \macro{\childdocdisable}
% The macro |\childdocdisable| prevents the main file
% from being processed more than once.
% At this stage, the main document command |\childdocmain|
% is assumed to be called once again where it should do nothing.
% Any subsequent call to it should prevent
% a secondary processing of the main document
% It overwrites the forwarding commands
% |\childdocof| and |\childdocforward|
% with empty macros to prevent further inclusions of the main document:
%    \begin{macrocode}
\newcommand{\childdocdisable}
{
  \renewcommand{\childdocmain}[1]{\renewcommand{\childdocmain}[1]{\endinput}}
  \renewcommand{\childdocof}[1]{}
  \renewcommand{\childdocby}[2][]{}
  \renewcommand{\childdocforward}[2][]{}
  \renewcommand{\childdocdisable}{}
}
%    \end{macrocode}

% \macro{\childdocmain}
% The macro |\childdocmain| is to be called at the top of the main file
% with nothing or the main filename (without extension) as argument.
% First, it breaks loops.
% If the argument is not empty and does not match |\childdocname|
% (which is set by the first inclusion of |childdoc.def|),
% |\ifchilddoc| is set to true, |\includeonly| is applied to the child file
% and |\jobname| is set to the main file
% (for proper handling of |.aux| files):
%    \begin{macrocode}
\newcommand{\childdocmain}[1]
{
  \childdocdisable\childdocmain{}
  \if?#1?\else
    \begingroup
      \def\childdoctmp{#1}
      \ifx\childdoctmp\childdocname
        \def\childdoctmp{}
      \else
        \def\childdoctmp
        {
          \childdoctrue
          \includeonly{\childdocname}
          \def\childdocjob{#1}
          \def\jobname{#1}
        }
      \fi
      \expandafter
    \endgroup
    \childdoctmp
  \fi
}
%    \end{macrocode}

% \macro{\childdocof}
% The command |\childdocof| redirects
% compilation to the main file |#1|.
%    \begin{macrocode}
\newcommand{\childdocof}[1]
{
  \childdocdisable
  \childdoctrue
  \includeonly{\childdocname}
  \def\jobname{#1}
  \def\childdocjob{#1}
  \input{#1}
}
%    \end{macrocode}

% \macro{\childdocby}
% The command |\childdocby| ....
%    \begin{macrocode}
\newcommand{\childdocby}[2][]
{
  \childdocdisable
  \childdoctrue
  \childdocmanualtrue
  \if?#1?\else
    \def\jobname{#2}
  \fi
  \def\childdocjob{#2}
  \input{#2}
  \endinput
}
%    \end{macrocode}

% \macro{\childdocforward}
% The command |\childdocforward| redirects
% compilation to the main file or
% (if the optional argument is given) a child file.
% Parameters are set as if the main file
% or a child file starting with |\childdocof| was compiled.
% Then compilation is handed over to the main file:
%    \begin{macrocode}
\newcommand{\childdocforward}[2][]
{
  \begingroup
    \if?#1?
      \def\childdoctmp
      {
        \def\childdocname{#2}
        \def\childdocjob{#2}
        \def\jobname{#2}
        \input{#2}
        \endinput
      }
    \else
      \def\childdoctmp
      {
        \childdocdisable
        \def\childdocname{#2}
        \childdoctrue
        \includeonly{#2}
        \def\childdocjob{#1}
        \def\jobname{#1}
        \input{#1}
        \endinput
      }
    \fi
    \expandafter
  \endgroup
  \childdoctmp
}
%    \end{macrocode}

% \macro{\childdocforwardprefix}
% The command |\childdocforwardprefix| redirects
% compilation to the main or a child file by means of a pattern.
% The prefix |#1| in the current filename is replaced by |#2|
% and the suffix of the current filename is kept
% (it is assumed that the filename does not contain the substring `|~~~|'
% which is used as a delimiter).
% Compilation is handed over to the new file by |\childdocforward|:
%    \begin{macrocode}
\newcommand{\childdocforwardprefix}[3][]
{
  \begingroup
    \def\childdocextract #2##1~~~{\def\childdoctmp{\childdocforward[#1]{#3##1}}}
    \expandafter\childdocextract\childdocname~~~
    \expandafter
  \endgroup
  \childdoctmp
}
%    \end{macrocode}

% \macro{\childdoc}
% The deprecated macro |\childdoc| is a legacy version of |\childdocmain|:
%    \begin{macrocode}
\newcommand{\childdoc}{\childdocmain}
%    \end{macrocode}

% \macro{\childdocredirect}
% The deprecated macro |\childdocredirect| is a legacy version
% of |\childdocforward| and |\childdocforwardprefix|:
%    \begin{macrocode}
\newcommand{\childdocredirect}[2][]
{
  \begingroup
    \if?#1?
      \def\childdoctmp{\childdocforward{#2}}
    \else
      \def\childdoctmp{\childdocforwardprefix{#1}{#2}}
    \fi
    \expandafter
  \endgroup
  \childdoctmp
}
%    \end{macrocode}

%\iffalse
%</package>
%\fi
%
\endinput
|\\
|\childdocforward[|\textit{main}|]{|\textit{dest}|}|\\
\end{tabular}
\end{center}
%
The argument \textit{dest} is the destination file
(without extension).
It should be the main file or one of the child files.
Note that further \textsf{childdoc} directives
such as |\childdocof| and |\childdocforward|
in the indicated file will be processed in this form.
The optional argument \textit{main}
passes on directly to the main file \textit{main}
while pretending to compile the child \textit{dest}.
This form behaves as if \textit{dest}
issues |\childdocof{|\textit{main}|}| right away,
and no further \textsf{childdoc} directives will be processed.

%%%%%%%%%%%%%%%%%%%%%%%%%%%%%%%%%%%%%%%%
\DescribeMacro{\...prefix}
In the alternative form |\childdocforwardprefix|,
%
\begin{center}
\begin{tabular}{l}
|% \iffalse
%
% childdoc.dtx Copyright (C) 2017-2018 Niklas Beisert
%
% This work may be distributed and/or modified under the
% conditions of the LaTeX Project Public License, either version 1.3
% of this license or (at your option) any later version.
% The latest version of this license is in
%   http://www.latex-project.org/lppl.txt
% and version 1.3 or later is part of all distributions of LaTeX
% version 2005/12/01 or later.
%
% This work has the LPPL maintenance status `maintained'.
%
% The Current Maintainer of this work is Niklas Beisert.
%
% This work consists of the files childdoc.dtx and childdoc.ins
% and the derived files childdoc.def and cdocsamp.tex with
% cdocsch1.tex, cdocsch2.tex, cdocsdrf.tex, cdocsfn1.tex, cdocsfn2.tex.
%
%<package>\ifdefined\childdocmain\endinput\fi
%<package>\ProvidesFile{childdoc.def}[2018/12/30 v2.0 child document driver]
%<samplemain>\ProvidesFile{cdocsamp.tex}[2018/12/30 v2.0 sample for childdoc]
%<*driver>
%\ProvidesFile{childdoc.drv}[2018/12/30 v2.0 childdoc reference manual file]
\PassOptionsToClass{10pt,a4paper}{article}
\documentclass{ltxdoc}

\usepackage[margin=35mm]{geometry}
\usepackage{hyperref}
\usepackage{hyperxmp}
\usepackage[usenames]{color}

\hypersetup{colorlinks=true}
\hypersetup{pdfstartview=FitH}
\hypersetup{pdfpagemode=UseNone}
\hypersetup{pdfsource={}}
\hypersetup{pdflang={en-UK}}
\hypersetup{pdfcopyright={Copyright 2017-2018 Niklas Beisert.
  This work may be distributed and/or modified under the
  conditions of the LaTeX Project Public License, either version 1.3
  of this license or (at your option) any later version.}}
\hypersetup{pdflicenseurl={http://www.latex-project.org/lppl.txt}}
\hypersetup{pdfcontactaddress={ETH Zurich, ITP, HIT K,
  Wolfgang-Pauli-Strasse 27}}
\hypersetup{pdfcontactpostcode={8093}}
\hypersetup{pdfcontactcity={Zurich}}
\hypersetup{pdfcontactcountry={Switzerland}}
\hypersetup{pdfcontactemail={nbeisert@itp.phys.ethz.ch}}
\hypersetup{pdfcontacturl={http://people.phys.ethz.ch/\xmptilde nbeisert/}}

\newcommand{\secref}[1]{\hyperref[#1]{section \ref*{#1}}}

\parskip1ex
\parindent0pt
\let\olditemize\itemize
\def\itemize{\olditemize\parskip0pt}

\begin{document}

\title{The \textsf{childdoc} Package}
\hypersetup{pdftitle={The childdoc Package}}
\author{Niklas Beisert\\[2ex]
  Institut f\"ur Theoretische Physik\\
  Eidgen\"ossische Technische Hochschule Z\"urich\\
  Wolfgang-Pauli-Strasse 27, 8093 Z\"urich, Switzerland\\[1ex]
  \href{mailto:nbeisert@itp.phys.ethz.ch}
  {\texttt{nbeisert@itp.phys.ethz.ch}}}
\hypersetup{pdfauthor={Niklas Beisert}}
\hypersetup{pdfsubject={Manual for the LaTeX2e Package childdoc}}
\date{30 December 2018, \textsf{v2.0}}
\maketitle

\begin{abstract}\noindent
\textsf{childdoc} is a \LaTeXe{} package
that enables the direct compilation
of document sections included by |\include|
to individual files.
\end{abstract}

\begingroup
\parskip0ex
\tableofcontents
\endgroup

%%%%%%%%%%%%%%%%%%%%%%%%%%%%%%%%%%%%%%%%%%%%%%%%%%%%%%%%%%%%%%%%%%%%%%%%%%%%%%%%
%%%%%%%%%%%%%%%%%%%%%%%%%%%%%%%%%%%%%%%%%%%%%%%%%%%%%%%%%%%%%%%%%%%%%%%%%%%%%%%%
\section{Introduction}

\LaTeX{} provides a mechanism to structure a large document (such as a book)
into a main file and several child files (containing the chapters)
using the |\include| command.
This mechanism is beneficial for documents
which span hundreds of pages in order to
make the source file(s) more manageable.
Moreover, compilation can be restricted to
selected child files by means of the |\includeonly| command.
The latter feature can be used to reduce the compilation time while editing
(this was significantly more useful in the earlier days of \LaTeX{})
or to generate a smaller document which is easier to navigate.
Another application of |\includeonly| is to generate
documents consisting of selected parts of the complete document.

However, there are a few drawbacks of the plain |\include| mechanism:
\begin{itemize}
\item
The child files cannot be compiled on their own,
they can only be compiled via the main file.
A naive editing environment
(such as a text editor with an option
to have the current file processed by \LaTeX)
may require one to switch to the main file before compiling;
attempting to compile the child file produces errors.
\item
The main file must be modified (each time)
to adjust the |\includeonly| command
to the present needs. This easily leaves the main file in a messy state.
\item
The generated document will always carry the filename
of the main document. This is inconvenient if
several child files are to be compiled and
to be kept for distribution.
\end{itemize}

The present package provides a simple interface
to make child files individually compilable by \LaTeX{}.
Compiling a child file then has the same effect as compiling
the main file with an |\includeonly| command
to select the appropriate child.
Moreover the generated document will carry the name of the child
rather than the main file.
This resolves all three above issues.

This feature is meant to make the editing of books,
thesis documents and lecture notes somewhat more convenient.
However, the package can also be used efficiently for
composing a series of documents (such as exercise sheets)
which are typically distributed individually.
It then assists the author in generating the individual documents
(potentially in different versions)
as well as a document containing the collected series.
Another application is in developing style files
or other kinds of included material
where compilation of the style file could redirect
to a sample or test file.

%%%%%%%%%%%%%%%%%%%%%%%%%%%%%%%%%%%%%%%%%%%%%%%%%%%%%%%%%%%%%%%%%%%%%%%%%%%%%%%%
%%%%%%%%%%%%%%%%%%%%%%%%%%%%%%%%%%%%%%%%%%%%%%%%%%%%%%%%%%%%%%%%%%%%%%%%%%%%%%%%
\section{Usage}

First of all, the package \textsf{childdoc} is \emph{not} a standard
\LaTeXe{} |.sty| style file! Therefore it needs to be invoked in
a non-standard way.

%%%%%%%%%%%%%%%%%%%%%%%%%%%%%%%%%%%%%%%%%%%%%%%%%%%%%%%%%%%%%%%%%%%%%%%%%%%%%%%%
\subsection{Included Files}
\label{sec:include}

%%%%%%%%%%%%%%%%%%%%%%%%%%%%%%%%%%%%%%%%
\DescribeMacro{\childdocmain}
To use the package, add the commands
\begin{center}
\begin{tabular}{l}
|\input{childdoc.def}|\\
|\childdocmain{}|\\
\end{tabular}
\end{center}
at the very top of the main \LaTeX{} file,
in particular \emph{before} the |\documentclass| statement!
The argument of |\childdocmain| should be left empty
(but it must be present).

%%%%%%%%%%%%%%%%%%%%%%%%%%%%%%%%%%%%%%%%
\DescribeMacro{\childdocof}
Furthermore, add the commands
\begin{center}
\begin{tabular}{l}
|\input{childdoc.def}|\\
|\childdocof{|\textit{main}|}|\\
\end{tabular}
\end{center}
at the top of every child file \textit{child}
which is included by |\include{|\textit{child}|}|
from within the main file
(or at least for those files to be compiled individually).
The argument \textit{main} must be the filename of the main file.

There are a couple of
considerations in setting up the main and child documents:

%%%%%%%%%%%%%%%%%%%%%%%%%%%%%%%%%%%%%%%%
\paragraph{Restrictions.}

Please note the following restrictions:
\begin{itemize}
\item
|\childdocmain| must be called with one argument \textit{main}
to ensure compatibility with earlier version of the package.
It must either be empty (|\childdocmain{}|)
or precisely match the filename of the main file in which it is specified.
See \secref{sec:detection} for further information.
\item
The filename \textit{main} must be specified without the |.tex| extension.
\item
The filename \textit{main} is case sensitive
(even in case-insensitive file systems)
due to internal string comparison.
\item
The argument \textit{main} should be fully expanded, it cannot be a macro.
\item
Subdirectories and special characters should be avoided in filenames.
\item
The command |\childdocmain{|\textit{main}|}| must be followed by a whitespace.
It should not be followed immediately by another command
or by a comment mark `|%|'.
This is because the \TeX{} parser reads the token immediately following
the argument of |\childdocmain| and puts it
at the beginning of every child section;
however, a white\-space is ignored.
\end{itemize}

%%%%%%%%%%%%%%%%%%%%%%%%%%%%%%%%%%%%%%%%
\paragraph{Content of Main File.}

It is advisable to place all content in the child files included by |\include|.
Any output contained in the main file will appear in all child documents
unless suppressed manually;
it cannot be suppressed automatically by the |\includeonly| directive
and thus should normally be avoided.
A method to include some content in the main file
by means of conditional processing is described in \secref{sec:conditional}.

%%%%%%%%%%%%%%%%%%%%%%%%%%%%%%%%%%%%%%%%
\paragraph{Page Numbering.}

When only a part of the document is compiled,
the appropriate numbering of pages
(as well as other status parameters)
is determined from the |.aux| files.
The latter contain information from previous passes.
However this information needs to propagate through
all intermediate child documents.
Therefore the page numbering in child documents may well
be inconsistent until the complete document is compiled at least once.

A useful (if unconventional) way to always ensure a consistent
page numbering is to restart the numbering in each child document
and denote the pages by `\textit{child}|.|\textit{page}'
where \textit{child} represents the chapter/section number of the child file.
This can be achieved by the command
|\numberwithin{page}{|\textit{child}|}|
of the \textsf{amsmath} package
where \textit{child} can be |chapter| or |section|
depending on the chosen structuring.
Alternatively, one can modify the macro |\thepage| appropriately
and reset the counter |page| at the start of each child file.

%%%%%%%%%%%%%%%%%%%%%%%%%%%%%%%%%%%%%%%%%%%%%%%%%%%%%%%%%%%%%%%%%%%%%%%%%%%%%%%%
\subsection{Conditional Processing}
\label{sec:conditional}

The package provides a mechanism to compile different versions
of a document. To customise the versions further some conditional processing
can come in handy to distinguish which version is being compiled.
The package provides two macros to describe the compilation context:

%%%%%%%%%%%%%%%%%%%%%%%%%%%%%%%%%%%%%%%%
\DescribeMacro{\ifchilddoc}
The conditional |\ifchilddoc| distinguishes between the compilation of
child documents and the main document:
%
\begin{center}
|\ifchilddoc |\textit{child-code}| |[|\||else |\textit{main-code}]| \||fi|
\end{center}

%%%%%%%%%%%%%%%%%%%%%%%%%%%%%%%%%%%%%%%%
\DescribeMacro{\childdocname}
\DescribeMacro{\childdocjob}
The macro |\childdocname| contains the filename (without extension)
of the main or child file being processed.
Note that |\childdocjob| will always contain the name of the main file.

%%%%%%%%%%%%%%%%%%%%%%%%%%%%%%%%%%%%%%%%
\paragraph{Title Page.}

Conditional processing can be used to include a title or banner page
in the main document when proper precautions are taken.
Importantly, the code in the main file should ensure that the page counter
(as well as other status parameters which are stored in the |.aux| files)
takes the same value after the conditional processing.
Otherwise the page numbers may take divergent values
depending on which part is compiled.

For example, a title page could be declared by:
%
\begin{center}
\begin{tabular}{l}
|\ifchilddoc\||else|\\
|\addtocounter{page}{-1}|\\
\textit{code for title page}\\
|\newpage|\\
|\||fi|
\end{tabular}
\end{center}
%
A banner page for the child documents can be generated by:
%
\begin{center}
\begin{tabular}{l}
|\ifchilddoc|\\
|\addtocounter{page}{-1}|\\
\textit{code for banner page}\\
|\newpage|\\
|\||fi|
\end{tabular}
\end{center}
%
Here one could write a message such as:
\begin{center}
|This is the part \childdocname{} of \childdocjob{}.|
\end{center}

%%%%%%%%%%%%%%%%%%%%%%%%%%%%%%%%%%%%%%%%%%%%%%%%%%%%%%%%%%%%%%%%%%%%%%%%%%%%%%%%
\subsection{Flags}
\label{sec:flags}

The package makes it easy to generate different versions
of the main or child documents.
To this end compilation flags can be defined
and assigned different default values.
They will be particularly useful in conjunction
with the forwarding mechanism described in \secref{sec:forward}.

For example, it may be useful to have a flag |\version|
which can be set to |draft| or |final|.
The document source will contain some conditional code
depending on the value of |\version|.
Suppose further, the flag should default to |final| for the main file
and to |draft| for child files
which is a natural assignment for editing the document.
This is achieved by placing the following code
in the preamble of the main document
(below the |\childdocmain| directive):
%
\begin{center}
\begin{tabular}{l}
|\ifchilddoc|\\
|\providecommand{\version}{draft}|\\
|\||else|\\
|\providecommand{\version}{final}|\\
|\||fi|
\end{tabular}
\end{center}
%
The definition by |\providecommand| makes sure
that previous definitions are not overwritten.
Further statements |\providecommand{\version}{...}|
can thus be added before the above code to override it.

For the main file, one might add a line
(between |\childdocmain| and the above block)
%
\begin{center}
|%\ifchilddoc\||else\providecommand{\version}{draft}\||fi|
\end{center}
%
which can be uncommented to produce a draft version.
Likewise one can add a line to the very top of a child file
(above the |\childdocof{|\textit{main}|}| directive)
%
\begin{center}
|%\providecommand{\version}{final}|
\end{center}
%
which can be uncommented to produce the final version of this child document.

%%%%%%%%%%%%%%%%%%%%%%%%%%%%%%%%%%%%%%%%%%%%%%%%%%%%%%%%%%%%%%%%%%%%%%%%%%%%%%%%
\subsection{Forwarding}
\label{sec:forward}

Different versions of the main or child documents
using compilation flags as described in \secref{sec:flags}
can be (permanently) stored in different files
for convenient compilation, viewing and distribution.
To this end, the package defines a command
to pass on compilation to a different file:

%%%%%%%%%%%%%%%%%%%%%%%%%%%%%%%%%%%%%%%%
\DescribeMacro{\childdocforward}
The command |\childdocforward| redirects processing to
another source file:
%
\begin{center}
\begin{tabular}{l}
|\input{childdoc.def}|\\
|\childdocforward[|\textit{main}|]{|\textit{dest}|}|\\
\end{tabular}
\end{center}
%
The argument \textit{dest} is the destination file
(without extension).
It should be the main file or one of the child files.
Note that further \textsf{childdoc} directives
such as |\childdocof| and |\childdocforward|
in the indicated file will be processed in this form.
The optional argument \textit{main}
passes on directly to the main file \textit{main}
while pretending to compile the child \textit{dest}.
This form behaves as if \textit{dest}
issues |\childdocof{|\textit{main}|}| right away,
and no further \textsf{childdoc} directives will be processed.

%%%%%%%%%%%%%%%%%%%%%%%%%%%%%%%%%%%%%%%%
\DescribeMacro{\...prefix}
In the alternative form |\childdocforwardprefix|,
%
\begin{center}
\begin{tabular}{l}
|\input{childdoc.def}|\\
|\childdocforwardprefix[|\textit{main}|]{|\textit{prefix}|}{|\textit{dest}|}|
\end{tabular}
\end{center}
%
the destination file is determined by a pattern
depending on the current file:
To make this work, the current file must be called
`{\textit{prefix}\hspace{0.2em}\textit{suffix}}'
with \textit{prefix} matching precisely the argument.
Processing is then passed on to the file
`{\textit{dest}\hspace{0.2em}\textit{suffix}}'.
Surely, the same effect is achieved by
directly specifying the
argument `{\textit{dest}\hspace{0.2em}\textit{suffix}}'
in the first form.
However, that requires to set up a different file
for each child. With the alternative form of the command
all these files can have exactly the same content
which simplifies setting them up and maintaining them.

For example, the following file |draft.tex|
with a compilation flag |\version| as described in \secref{sec:flags}
compiles the main document as a draft:
%
\begin{center}
\begin{tabular}{l}
|\def\version{draft}|\\
|\input{childdoc.def}|\\
|\childdocforward{|\textit{main}|}|
\end{tabular}
\end{center}
%
Likewise, the following files |final|\textit{nn}|.tex|
compile the final version of the child document
|child|\textit{nn}|.tex|:
%
\begin{center}
\begin{tabular}{l}
|\def\version{final}|\\
|\input{childdoc.def}|\\
|\childdocforwardprefix{final}{child}|
\end{tabular}
\end{center}
%

Note that when several versions of a main file and/or of each child file
are to be generated, it may be convenient to set up a |Makefile| or
shell script to automatise the process.

%%%%%%%%%%%%%%%%%%%%%%%%%%%%%%%%%%%%%%%%%%%%%%%%%%%%%%%%%%%%%%%%%%%%%%%%%%%%%%%%
\subsection{Command Line Processing}
\label{sec:commandline}

The effect of redirection files can also be achieved by invoking
the \LaTeX{} compiler with a more elaborate command line.
Most conveniently this should be done as part
of a shell script or a |Makefile|.

When using \textsf{childdoc} in the main file, the following
command lines effectively perform a redirection
(note that depending on the shell being used,
backslashes may have to be doubled: `|\|' $\to$ `|\\|'):
%
\begin{center}
|... -jobname "|\textit{target}|" |\\|"|[\textit{flags}]%
|\input{childdoc.def}\childdocforward[|\textit{main}|]{|\textit{dest}|}"|
\end{center}
%
Here \textit{target} is the name of the output file,
\textit{main} is the name of the main file
and \textit{dest} is the name of the main or child file to be processed
(all filenames without extensions).
The optional argument \textit{main} can be omitted
if \textit{main} matches \textit{dest}.
Optionally, compilation \textit{flags} can be defined via |\def| commands.
This command line makes the \TeX{} engine believe
it is compiling the file \textit{target}
whose content is specified as the latter parameter.
The provided code then forwards the processing to
\textit{main} or \textit{dest} as described in \secref{sec:forward}.

%%%%%%%%%%%%%%%%%%%%%%%%%%%%%%%%%%%%%%%%%%%%%%%%%%%%%%%%%%%%%%%%%%%%%%%%%%%%%%%%
\subsection{Include by Input}
\label{sec:input}

Including child documents by |\include| has some restrictions by design.
Most notably, the content of a child document always occupies
its own set of pages; pages cannot be shared between child documents.
Usually, this behaviour makes perfect sense
because each child document contain an essential part of the document.
However, in some situations it may be desirable to compose
a document from a collection of parts
without having mandatory page breaks between then.
For this case, the package
provides a mechanism to include parts
by |\input| which can also be processed individually.
However, by construction this mechanism
requires manual handling of the content to be output.

%%%%%%%%%%%%%%%%%%%%%%%%%%%%%%%%%%%%%%%%
\DescribeMacro{\ifchilddocmanual}
The main file should be prepared as usual, see \secref{sec:include}.
However, the document body must make a distinction
between processing of an individual part and of the main document, e.g.:
%
\begin{center}
\begin{tabular}{l}
|\ifchilddocmanual|\\
|\input{\childdocname}|\\
|\||else|\\
\textit{document body with }|\input{|\textit{part}|}|\\
|\||fi|
\end{tabular}
\end{center}
%
The conditional |\ifchilddocmanual| is true whenever
a part to be included by |\input| is being compiled,
and the name of the part is stored in |\childdocname|.

%%%%%%%%%%%%%%%%%%%%%%%%%%%%%%%%%%%%%%%%
\DescribeMacro{\childdocby}
Each part to be included by |\input| should start with:
%
\begin{center}
\begin{tabular}{l}
|\input{childdoc.def}|\\
|\childdocby{|\textit{main}|}|\\
\end{tabular}
\end{center}
%
The directive |\childdocby| is similar to |\childdocof|
described in \secref{sec:include},
but the subsequent selection of content must be done manually.
To that end, both |\ifchilddoc| and |\ifchilddocmanual|
will be true upon processing of a part,
and the name of the part is stored in |\childdocname|.
Note that |\jobname| will be set to the filename of the current part
so that each part receives an individual |.aux| file
that does not interfere with the |.aux| file(s) of the main document.
This behaviour can be altered by the alternative form
|\childdocby[*]{|\textit{main}|}| (with a non-empty optional argument)
which uses the |.aux| file of the main document
by setting |\jobname| to \textit{main}.

%%%%%%%%%%%%%%%%%%%%%%%%%%%%%%%%%%%%%%%%%%%%%%%%%%%%%%%%%%%%%%%%%%%%%%%%%%%%%%%%
\subsection{Driver Development}
\label{sec:driver}

The \textsf{childdoc} mechanism can also be use for the development
of definition files such as \LaTeX{} styles or classes.
This case differs from the above setup with multiple parts
included by |\include| in that no |\includeonly| should be invoked.
This can be achieved by starting the include file
(before |\ProvidesPackage|) with:
%
\begin{center}
\begin{tabular}{l}
|\input{childdoc.def}|\\
|\childdocforward{|\textit{main}|}|\\
\end{tabular}
\end{center}
%
or alternatively with:
%
\begin{center}
\begin{tabular}{l}
|\input{childdoc.def}|\\
|\childdocby{|\textit{main}|}|\\
\end{tabular}
\end{center}
%
Both forms have slightly different effects as described above.
The main file is prepared as usual, see \secref{sec:include}.

%%%%%%%%%%%%%%%%%%%%%%%%%%%%%%%%%%%%%%%%%%%%%%%%%%%%%%%%%%%%%%%%%%%%%%%%%%%%%%%%
\subsection{Legacy Detection}
\label{sec:detection}

The directive |\childdocmain| in the main file can detect
whether the complete document or merely a child is to be compiled
even without using the directive |\childdocof|.
This method is deprecated because it is less robust
and there is no compelling reason to use it;
it is merely provided for backward compatibility
and it may be removed in future versions.

If the detection mechanism is to be used,
it is mandatory to correctly specify
the filename of the main file as the argument of |\childdocmain|:
%
\begin{center}
\begin{tabular}{l}
|\input{childdoc.def}|\\
|\childdocmain{|\textit{main}|}|\\
\end{tabular}
\end{center}
%
If |\jobname| does not match the argument \textit{main} of |\childdocmain|,
it is assumed that |\jobname| points to the child file to be compiled.
When using |\childdocmain| with the main file specified as argument,
it suffices to start a child file
with just |\input{|\textit{main}|}|
without loading of the package and using |\childdocof|.
If instead all processing is done
with the appropriate \textsf{childdoc} directives,
the argument of \textit{main} of |\childdocmain| can be empty.

An alternative version of the command line processing described
in \secref{sec:commandline} using the detection mechanism reads:
%
\begin{center}
|... -jobname "|\textit{target}|" "|[\textit{flags}]%
[|\def\jobname{|\textit{dest}|}|]|\input{|\textit{main}|}"|
\end{center}

%%%%%%%%%%%%%%%%%%%%%%%%%%%%%%%%%%%%%%%%%%%%%%%%%%%%%%%%%%%%%%%%%%%%%%%%%%%%%%%%
\subsection{Manual Code}
\label{sec:manual}

In case one cannot be certain whether the definitions file |childdoc.def|
is installed on the target \TeX{} distribution
and one prefers not to ship it,
it is conceivable to paste a few relevant commands into the sources.

To that end, drop all statements |\input{childdoc.def}|
and perform the replacements as outlined below.
Instead of |\childdocmain{|\textit{main}|}| add the following code
to the top of the main file:
%
\begin{center}
\begin{tabular}{l}
|\||ifdefined\childdocname\endinput\||fi\newif\ifchilddoc|\\
|\edef\childdocname{\scantokens\expandafter{\jobname\noexpand}}|\\
|\def\childdocmain{|\textit{main}|}\||ifx\childdocmain\childdocname\||else|\\
|\childdoctrue\includeonly{\childdocname}\let\jobname\childdocmain\||fi|\\
\end{tabular}
\end{center}
%
Instead of |\childdocof{|\textit{main}|}| just include the main file
at the top of each child file:
%
\begin{center}
|\input{|\textit{main}|}|
\end{center}
%
A simple redirection |\childdocforward{|\textit{dest}|}| is achieved by:
%
\begin{center}
|\def\jobname{|\textit{dest}|}\input{\jobname}|
\end{center}
%
The redirection with prefix
|\childdocforwardprefix[|\textit{prefix}|]{|\textit{dest}|}|
is accomplished by:
%
\begin{center}
\begin{tabular}{l}
|{\edef\jobname{\scantokens\expandafter{\jobname\noexpand}}|\\
|\def\redirectjob |\textit{prefix}|#1~~~{\gdef\jobname{|\textit{dest}|#1}}|\\
|\expandafter\redirectjob\jobname~~~}\input{\jobname}|
\end{tabular}
\end{center}

In an alternative approach,
child documents can be compiled by a specific command line
without additional code or specific definitions:
%
\begin{center}
|... -jobname "|\textit{target}|" "|[\textit{flags}]%
|\includeonly{|\textit{dest}|}\input{|\textit{main}|}"|
\end{center}
%

%%%%%%%%%%%%%%%%%%%%%%%%%%%%%%%%%%%%%%%%%%%%%%%%%%%%%%%%%%%%%%%%%%%%%%%%%%%%%%%%
%%%%%%%%%%%%%%%%%%%%%%%%%%%%%%%%%%%%%%%%%%%%%%%%%%%%%%%%%%%%%%%%%%%%%%%%%%%%%%%%
\section{Information}

%%%%%%%%%%%%%%%%%%%%%%%%%%%%%%%%%%%%%%%%%%%%%%%%%%%%%%%%%%%%%%%%%%%%%%%%%%%%%%%%
\subsection{Copyright}

Copyright \copyright{} 2017--2018 Niklas Beisert

This work may be distributed and/or modified under the
conditions of the \LaTeX{} Project Public License, either version 1.3
of this license or (at your option) any later version.
The latest version of this license is in
  \url{http://www.latex-project.org/lppl.txt}
and version 1.3 or later is part of all distributions of \LaTeX{}
version 2005/12/01 or later.

This work has the LPPL maintenance status `maintained'.

The Current Maintainer of this work is Niklas Beisert.

This work consists of the files |README.txt|, |childdoc.ins| and |childdoc.dtx|
as well as the derived files |childdoc.def|, |cdocsamp.tex|
with |cdocsch1.tex|, |cdocsch2.tex|, |cdocspt3.tex|, |cdocspt4.tex|,
|cdocsdrf.tex|, |cdocsfn1.tex|, |cdocsfn2.tex|
as well as |childdoc.pdf|.

%%%%%%%%%%%%%%%%%%%%%%%%%%%%%%%%%%%%%%%%%%%%%%%%%%%%%%%%%%%%%%%%%%%%%%%%%%%%%%%%
\subsection{Files and Installation}

The package consists of the files:
%
\begin{center}
\begin{tabular}{ll}
    |README.txt|   & readme file \\
    |childdoc.ins| & installation file \\
    |childdoc.dtx| & source file \\
    |childdoc.def| & definition file \\
    |cdocsamp.tex| & sample main file \\
    |cdocsch1.tex| & sample include file \\
    |cdocsch2.tex| & sample include file \\
    |cdocspt3.tex| & sample part file \\
    |cdocspt4.tex| & sample part file \\
    |cdocsdrf.tex| & sample redirection file \\
    |cdocsfn1.tex| & sample redirection file \\
    |cdocsfn2.tex| & sample redirection file \\
    |childdoc.pdf| & manual
\end{tabular}
\end{center}
%
The distribution consists of the files
|README.txt|, |childdoc.ins| and |childdoc.dtx|.
%
\begin{itemize}
\item
Run (pdf)\LaTeX{} on |childdoc.dtx|
to compile the manual |childdoc.pdf| (this file).
\item
Run \LaTeX{} on |childdoc.ins| to create the definitions file |childdoc.def|
and the sample |cdocsamp.tex| with include files
|cdocsch1.tex|, |cdocsch2.tex|, |cdocspt3.tex|, |cdocspt4.tex|,
|cdocsdrf.tex|, |cdocsfn1.tex|, |cdocsfn2.tex|.
Then copy the file |childdoc.def| to an appropriate directory of your \LaTeX{}
distribution, e.g.\ \textit{texmf-root}|/tex/latex/childdoc|.
\end{itemize}

%%%%%%%%%%%%%%%%%%%%%%%%%%%%%%%%%%%%%%%%%%%%%%%%%%%%%%%%%%%%%%%%%%%%%%%%%%%%%%%%
\subsection{Related CTAN Packages}

There are several other packages which offer a similar functionality:
%
\begin{itemize}
\item
The packages
\href{http://ctan.org/pkg/docmute}{\textsf{docmute}},
\href{http://ctan.org/pkg/includex}{\textsf{includex}} and
\href{http://ctan.org/pkg/standalone}{\textsf{standalone}}
provide commands to include only the document body of
a child file thus allowing both files to be compiled individually.
\item
The packages \href{http://ctan.org/pkg/subdocs}{\textsf{subdocs}}
and \href{http://ctan.org/pkg/subfiles}{\textsf{subfiles}}
provide structures in which the main and child documents can be
encapsulated and allowing them to be compiled individually.
The inclusion mechanism is different from the conventional |\include|.
\item
The package \href{http://ctan.org/pkg/combine}{\textsf{combine}}
is an elaborate solution to combine several documents into one.
\end{itemize}
%
See also the CTAN topic \href{http://ctan.org/topic/subdocs}{\textsf{subdocs}}
for further related packages.
The present package differs from the above solutions in that
a document structure constructed with the conventional |\include| mechanism
just needs two extra commands at the top of every file
such that all constituent files can be compiled individually.

%%%%%%%%%%%%%%%%%%%%%%%%%%%%%%%%%%%%%%%%%%%%%%%%%%%%%%%%%%%%%%%%%%%%%%%%%%%%%%%%
%\subsection{Feature Suggestions}
%
%The following is a list of features which may be useful for future
%versions of this package:
%%
%\begin{itemize}
%\item
%\ldots
%\end{itemize}

%%%%%%%%%%%%%%%%%%%%%%%%%%%%%%%%%%%%%%%%%%%%%%%%%%%%%%%%%%%%%%%%%%%%%%%%%%%%%%%%
\subsection{Revision History}

%%%%%%%%%%%%%%%%%%%%%%%%%%%%%%%%%%%%%%%%
\paragraph{v2.0:} 2018/12/30

\begin{itemize}
\item
immediate forward processing
\item
added |\childdocby| mechanism
\item
manual restructured
\end{itemize}

%%%%%%%%%%%%%%%%%%%%%%%%%%%%%%%%%%%%%%%%
\paragraph{v1.6:} 2018/01/17

\begin{itemize}
\item
application for development of include files
\item
corrections to manual
\end{itemize}

%%%%%%%%%%%%%%%%%%%%%%%%%%%%%%%%%%%%%%%%
\paragraph{v1.5:} 2017/05/21

\begin{itemize}
\item
more complete structuring introduced
\item
|\childdocof| introduced
\item
|\childdoc| renamed to |\childdocmain|
\item
|\childredirect| renamed to |\childdocforward| and |\childdocforwardprefix|
and functionality expanded
\end{itemize}

%%%%%%%%%%%%%%%%%%%%%%%%%%%%%%%%%%%%%%%%
\paragraph{v1.0:} 2017/04/27

\begin{itemize}
\item
manual and install package
\item
first version published on CTAN
\end{itemize}

%%%%%%%%%%%%%%%%%%%%%%%%%%%%%%%%%%%%%%%%
\paragraph{v0.6:} 2017/04/26

\begin{itemize}
\item
redirection mechanism added
\end{itemize}

%%%%%%%%%%%%%%%%%%%%%%%%%%%%%%%%%%%%%%%%
\paragraph{v0.5:} 2017/04/26

\begin{itemize}
\item
functionality in definition file
\end{itemize}


%%%%%%%%%%%%%%%%%%%%%%%%%%%%%%%%%%%%%%%%%%%%%%%%%%%%%%%%%%%%%%%%%%%%%%%%%%%%%%%%
%%%%%%%%%%%%%%%%%%%%%%%%%%%%%%%%%%%%%%%%%%%%%%%%%%%%%%%%%%%%%%%%%%%%%%%%%%%%%%%%
%%%%%%%%%%%%%%%%%%%%%%%%%%%%%%%%%%%%%%%%%%%%%%%%%%%%%%%%%%%%%%%%%%%%%%%%%%%%%%%%
\appendix

\settowidth\MacroIndent{\rmfamily\scriptsize 000\ }

 \DocInput{childdoc.dtx}

\end{document}
%</driver>
% \fi
%
% %%%%%%%%%%%%%%%%%%%%%%%%%%%%%%%%%%%%%%%%%%%%%%%%%%%%%%%%%%%%%%%%%%%%%%%%%%%%%%
% %%%%%%%%%%%%%%%%%%%%%%%%%%%%%%%%%%%%%%%%%%%%%%%%%%%%%%%%%%%%%%%%%%%%%%%%%%%%%%
% \section{Sample}
%\iffalse
%<*samplemain>
%\fi
%
% The following presents a sample document
% with two chapters, two parts, a title page,
% a compile flag as well as three forwarding files to set the flag.
% It consists of eight |.tex| files:
% \begin{center}
% \begin{tabular}{ll}
% |cdocsamp.tex|&main file\\
% |cdocsch1.tex|&include file for chapter 1\\
% |cdocsch2.tex|&include file for chapter 2\\
% |cdocspt3.tex|&include file for part 3\\
% |cdocspt4.tex|&include file for part 4\\
% |cdocsdrf.tex|&forwarding file for main file in draft mode\\
% |cdocsfi1.tex|&forwarding file for final version of chapter 1\\
% |cdocsfi2.tex|&forwarding file for final version of chapter 2\\
% \end{tabular}
% \end{center}
% Each of the eight files can be compiled directly by the \LaTeX{} compiler.
%
% %%%%%%%%%%%%%%%%%%%%%%%%%%%%%%%%%%%%%%
% \paragraph{Main File.}
%
% The main file is called |cdocsamp.tex|.
%
% Load the \textsf{childdoc} definitions and
% declare the filename for the main document:
%    \begin{macrocode}
\input{childdoc.def}
\childdocmain{}
%    \end{macrocode}

% Optional override for |\version| flag:
%    \begin{macrocode}
%%\ifchilddoc\else\providecommand{\version}{draft}\fi
%    \end{macrocode}

% Define the default values for the |\version| flag
% (|final| for the main file and |draft| for childs):
%    \begin{macrocode}
\ifchilddoc
\providecommand{\version}{draft}
\else
\providecommand{\version}{final}
\fi
%    \end{macrocode}

% Load the standard document class:
%    \begin{macrocode}
\documentclass[12pt]{article}
%    \end{macrocode}

% Start the document body:
%    \begin{macrocode}
\begin{document}
%    \end{macrocode}

% Declare a title page.
% Print title, part of document being processed and version flag:
%    \begin{macrocode}
\addtocounter{page}{-1}
\begin{center}
{\LARGE\bfseries{}childdoc example\par}
\vspace{1cm}
\ifchilddoc
\ifchilddocmanual part\else chapter\fi:
`\childdocname' of `\childdocjob'\par
\else
main document: `\childdocjob'\par
\fi
version: \version\par
\end{center}
\newpage
%    \end{macrocode}

% Manually include selected file,
% otherwise process as usual:
%    \begin{macrocode}
\ifchilddocmanual
\section*{part `\childdocname'}
\input{\childdocname}
\else
%    \end{macrocode}

% Include the two chapters:
%    \begin{macrocode}
\include{cdocsch1}
\include{cdocsch2}
%    \end{macrocode}

% Include the two parts unless only chapters should be displayed:
%    \begin{macrocode}
\ifchilddoc\else
\section{part three}
\input{cdocspt3}
\section{part four}
\input{cdocspt4}
\fi
%    \end{macrocode}

% Process as usual until here:
%    \begin{macrocode}
\fi
%    \end{macrocode}

% End of document body:
%    \begin{macrocode}
\end{document}
%    \end{macrocode}
%\iffalse
%</samplemain>
%\fi
%
% %%%%%%%%%%%%%%%%%%%%%%%%%%%%%%%%%%%%%%
% \paragraph{Chapter Include Files.}
%
% The include files are called |cdocsch1.tex| and |cdocsch2.tex|.
%
%\iffalse
%<*samplechap1|samplechap2>
%\fi

% Optional override for |\version| flag:
%    \begin{macrocode}
%%\providecommand{\version}{final}
%    \end{macrocode}

% Include the main document:
%    \begin{macrocode}
\input{childdoc.def}
\childdocof{cdocsamp}
%    \end{macrocode}

%\iffalse
%</samplechap1|samplechap2>
%\fi
%
%\iffalse
%<*samplechap1>
%\fi
% Some text for chapter 1:
%    \begin{macrocode}
\section{one}
some text in chapter one
%    \end{macrocode}

%\iffalse
%</samplechap1>
%\fi
% Some text for chapter 2:
%\iffalse
%<*samplechap2>
%\fi
%    \begin{macrocode}
\section{two}
more text in chapter two
%    \end{macrocode}

%\iffalse
%</samplechap2>
%\fi
%
% %%%%%%%%%%%%%%%%%%%%%%%%%%%%%%%%%%%%%%
% \paragraph{Part Include Files.}
%
% The include files are called |cdocspt3.tex| and |cdocspt4.tex|.
%
%\iffalse
%<*samplepart3|samplepart4>
%\fi

% Optional override for |\version| flag:
%    \begin{macrocode}
%%\providecommand{\version}{final}
%    \end{macrocode}

% Include the main document:
%    \begin{macrocode}
\input{childdoc.def}
\childdocby{cdocsamp}
%    \end{macrocode}

%\iffalse
%</samplepart3|samplepart4>
%\fi
%
%\iffalse
%<*samplepart3>
%\fi
% Some text for part 3:
%    \begin{macrocode}
some text in part three
%    \end{macrocode}

%\iffalse
%</samplepart3>
%\fi
% Some text for part 4:
%\iffalse
%<*samplepart4>
%\fi
%    \begin{macrocode}
more text in part four
%    \end{macrocode}

%\iffalse
%</samplepart4>
%\fi
%
% %%%%%%%%%%%%%%%%%%%%%%%%%%%%%%%%%%%%%%
% \paragraph{Forwarding for a Complete Draft.}
%
% The following forwarding file |cdocsdrf.tex|
% compiles the main document in draft mode:
%\iffalse
%<*sampledraft>
%\fi
%    \begin{macrocode}
\def\version{draft}
\input{childdoc.def}
\childdocforward{cdocsamp}
%    \end{macrocode}

%\iffalse
%</sampledraft>
%\fi
%
% %%%%%%%%%%%%%%%%%%%%%%%%%%%%%%%%%%%%%%
% \paragraph{Forwarding for Final Version of the Chapters.}
%
% The following forwarding files |cdocsfn1.tex| and |cdocsfn2.tex|
% (with identical content)
% compile the final versions of the child documents
% |cdocsch1.tex| and |cdocsch2.tex|, respectively:
%\iffalse
%<*samplefinal>
%\fi
%    \begin{macrocode}
\def\version{final}
\input{childdoc.def}
\childdocforwardprefix[cdocsamp]{cdocsfn}{cdocsch}
%    \end{macrocode}

%\iffalse
%</samplefinal>
%\fi
%
% %%%%%%%%%%%%%%%%%%%%%%%%%%%%%%%%%%%%%%
% \paragraph{Command Line Processing.}
%
% The following three command lines generate the output files
% |cdocscld|, |cdocscl1| and |cdocscl2|
% which should be identical to
% |cdocsdrf|, |cdocsch1| and |cdocsfn2|, respectively:
% \begin{center}
% \begin{tabular}{l}
% |latex -jobname cdocscld \|\\
% |  "\def\version{draft}\input{childdoc.def}\childdocforward{cdocsamp}"|\\
% |latex -jobname cdocscl1 \|\\
% |  "\input{childdoc.def}\childdocforward[cdocsamp]{cdocsch1}"|\\
% |latex -jobname cdocscl2 \|\\
% |  "\def\version{final}\input{childdoc.def}\childdocforward{cdocsch2}"|
% \end{tabular}
% \end{center}
% Note that the trailing backslash on each first line
% merely continues the input to the second line
% (for convenient cut ant paste).
% Furthermore, the command |latex| can be replaced by any
% of its alternative versions such as |pdflatex|.
%
% %%%%%%%%%%%%%%%%%%%%%%%%%%%%%%%%%%%%%%%%%%%%%%%%%%%%%%%%%%%%%%%%%%%%%%%%%%%%%%
% %%%%%%%%%%%%%%%%%%%%%%%%%%%%%%%%%%%%%%%%%%%%%%%%%%%%%%%%%%%%%%%%%%%%%%%%%%%%%%
% \section{Implementation}
%\iffalse
%<*package>
%\fi
%
% This section describes the definitions file |childdoc.def|.

% The definitions cannot be loaded using |\usepackage| or |\RequirePackage|
% which has a mechanism to prevent loading a style file more than once.
% When loading the definitions by means of |\input|
% multiple instances have to be prevented manually:
%\iffalse
%This code needs to be before the `\ProvidesFile' directive
%which is defined at the beginning of this file.
%Therefore it is also placed there and commented out here.
%</package>
%<*discard>
%\fi
%    \begin{macrocode}
\ifdefined\childdocmain\endinput\fi
%    \end{macrocode}
%\iffalse
%</discard>
%<*package>
%\fi
%
% \macro{\ifchilddoc}
% \macro{\ifchilddocmanual}
% The conditional |\ifchilddoc| tells whether a
% child (true) or main (false) document is being compiled.
% The conditional |\ifchilddocmanual| tells whether
% the |\includeonly| mechanism is used (false) or
% the selection of child files must be performed manually (true).
% The definitions initialise to false:
%    \begin{macrocode}
\newif\ifchilddoc
\newif\ifchilddocmanual
%    \end{macrocode}

% \macro{\childdocname}
% \macro{\childdocjob}
% The macro |\childdocname| stores the name of the main document
% to be compiled. The macro |\childdocjob| stores the name of
% the document on which the \LaTeX{} compiler was originally invoked.
% The content of |\jobname| cannot be compared
% to filenames specified in the source due to different catcodes.
% The following code rescans |\jobname|, stores the result
% in |\childdocname| and saves a copy in |\childdocjob|:
%    \begin{macrocode}
\edef\childdocname{\scantokens\expandafter{\jobname\noexpand}}
\let\childdocjob\childdocname
%    \end{macrocode}

% \macro{\childdocdisable}
% The macro |\childdocdisable| prevents the main file
% from being processed more than once.
% At this stage, the main document command |\childdocmain|
% is assumed to be called once again where it should do nothing.
% Any subsequent call to it should prevent
% a secondary processing of the main document
% It overwrites the forwarding commands
% |\childdocof| and |\childdocforward|
% with empty macros to prevent further inclusions of the main document:
%    \begin{macrocode}
\newcommand{\childdocdisable}
{
  \renewcommand{\childdocmain}[1]{\renewcommand{\childdocmain}[1]{\endinput}}
  \renewcommand{\childdocof}[1]{}
  \renewcommand{\childdocby}[2][]{}
  \renewcommand{\childdocforward}[2][]{}
  \renewcommand{\childdocdisable}{}
}
%    \end{macrocode}

% \macro{\childdocmain}
% The macro |\childdocmain| is to be called at the top of the main file
% with nothing or the main filename (without extension) as argument.
% First, it breaks loops.
% If the argument is not empty and does not match |\childdocname|
% (which is set by the first inclusion of |childdoc.def|),
% |\ifchilddoc| is set to true, |\includeonly| is applied to the child file
% and |\jobname| is set to the main file
% (for proper handling of |.aux| files):
%    \begin{macrocode}
\newcommand{\childdocmain}[1]
{
  \childdocdisable\childdocmain{}
  \if?#1?\else
    \begingroup
      \def\childdoctmp{#1}
      \ifx\childdoctmp\childdocname
        \def\childdoctmp{}
      \else
        \def\childdoctmp
        {
          \childdoctrue
          \includeonly{\childdocname}
          \def\childdocjob{#1}
          \def\jobname{#1}
        }
      \fi
      \expandafter
    \endgroup
    \childdoctmp
  \fi
}
%    \end{macrocode}

% \macro{\childdocof}
% The command |\childdocof| redirects
% compilation to the main file |#1|.
%    \begin{macrocode}
\newcommand{\childdocof}[1]
{
  \childdocdisable
  \childdoctrue
  \includeonly{\childdocname}
  \def\jobname{#1}
  \def\childdocjob{#1}
  \input{#1}
}
%    \end{macrocode}

% \macro{\childdocby}
% The command |\childdocby| ....
%    \begin{macrocode}
\newcommand{\childdocby}[2][]
{
  \childdocdisable
  \childdoctrue
  \childdocmanualtrue
  \if?#1?\else
    \def\jobname{#2}
  \fi
  \def\childdocjob{#2}
  \input{#2}
  \endinput
}
%    \end{macrocode}

% \macro{\childdocforward}
% The command |\childdocforward| redirects
% compilation to the main file or
% (if the optional argument is given) a child file.
% Parameters are set as if the main file
% or a child file starting with |\childdocof| was compiled.
% Then compilation is handed over to the main file:
%    \begin{macrocode}
\newcommand{\childdocforward}[2][]
{
  \begingroup
    \if?#1?
      \def\childdoctmp
      {
        \def\childdocname{#2}
        \def\childdocjob{#2}
        \def\jobname{#2}
        \input{#2}
        \endinput
      }
    \else
      \def\childdoctmp
      {
        \childdocdisable
        \def\childdocname{#2}
        \childdoctrue
        \includeonly{#2}
        \def\childdocjob{#1}
        \def\jobname{#1}
        \input{#1}
        \endinput
      }
    \fi
    \expandafter
  \endgroup
  \childdoctmp
}
%    \end{macrocode}

% \macro{\childdocforwardprefix}
% The command |\childdocforwardprefix| redirects
% compilation to the main or a child file by means of a pattern.
% The prefix |#1| in the current filename is replaced by |#2|
% and the suffix of the current filename is kept
% (it is assumed that the filename does not contain the substring `|~~~|'
% which is used as a delimiter).
% Compilation is handed over to the new file by |\childdocforward|:
%    \begin{macrocode}
\newcommand{\childdocforwardprefix}[3][]
{
  \begingroup
    \def\childdocextract #2##1~~~{\def\childdoctmp{\childdocforward[#1]{#3##1}}}
    \expandafter\childdocextract\childdocname~~~
    \expandafter
  \endgroup
  \childdoctmp
}
%    \end{macrocode}

% \macro{\childdoc}
% The deprecated macro |\childdoc| is a legacy version of |\childdocmain|:
%    \begin{macrocode}
\newcommand{\childdoc}{\childdocmain}
%    \end{macrocode}

% \macro{\childdocredirect}
% The deprecated macro |\childdocredirect| is a legacy version
% of |\childdocforward| and |\childdocforwardprefix|:
%    \begin{macrocode}
\newcommand{\childdocredirect}[2][]
{
  \begingroup
    \if?#1?
      \def\childdoctmp{\childdocforward{#2}}
    \else
      \def\childdoctmp{\childdocforwardprefix{#1}{#2}}
    \fi
    \expandafter
  \endgroup
  \childdoctmp
}
%    \end{macrocode}

%\iffalse
%</package>
%\fi
%
\endinput
|\\
|\childdocforwardprefix[|\textit{main}|]{|\textit{prefix}|}{|\textit{dest}|}|
\end{tabular}
\end{center}
%
the destination file is determined by a pattern
depending on the current file:
To make this work, the current file must be called
`{\textit{prefix}\hspace{0.2em}\textit{suffix}}'
with \textit{prefix} matching precisely the argument.
Processing is then passed on to the file
`{\textit{dest}\hspace{0.2em}\textit{suffix}}'.
Surely, the same effect is achieved by
directly specifying the
argument `{\textit{dest}\hspace{0.2em}\textit{suffix}}'
in the first form.
However, that requires to set up a different file
for each child. With the alternative form of the command
all these files can have exactly the same content
which simplifies setting them up and maintaining them.

For example, the following file |draft.tex|
with a compilation flag |\version| as described in \secref{sec:flags}
compiles the main document as a draft:
%
\begin{center}
\begin{tabular}{l}
|\def\version{draft}|\\
|% \iffalse
%
% childdoc.dtx Copyright (C) 2017-2018 Niklas Beisert
%
% This work may be distributed and/or modified under the
% conditions of the LaTeX Project Public License, either version 1.3
% of this license or (at your option) any later version.
% The latest version of this license is in
%   http://www.latex-project.org/lppl.txt
% and version 1.3 or later is part of all distributions of LaTeX
% version 2005/12/01 or later.
%
% This work has the LPPL maintenance status `maintained'.
%
% The Current Maintainer of this work is Niklas Beisert.
%
% This work consists of the files childdoc.dtx and childdoc.ins
% and the derived files childdoc.def and cdocsamp.tex with
% cdocsch1.tex, cdocsch2.tex, cdocsdrf.tex, cdocsfn1.tex, cdocsfn2.tex.
%
%<package>\ifdefined\childdocmain\endinput\fi
%<package>\ProvidesFile{childdoc.def}[2018/12/30 v2.0 child document driver]
%<samplemain>\ProvidesFile{cdocsamp.tex}[2018/12/30 v2.0 sample for childdoc]
%<*driver>
%\ProvidesFile{childdoc.drv}[2018/12/30 v2.0 childdoc reference manual file]
\PassOptionsToClass{10pt,a4paper}{article}
\documentclass{ltxdoc}

\usepackage[margin=35mm]{geometry}
\usepackage{hyperref}
\usepackage{hyperxmp}
\usepackage[usenames]{color}

\hypersetup{colorlinks=true}
\hypersetup{pdfstartview=FitH}
\hypersetup{pdfpagemode=UseNone}
\hypersetup{pdfsource={}}
\hypersetup{pdflang={en-UK}}
\hypersetup{pdfcopyright={Copyright 2017-2018 Niklas Beisert.
  This work may be distributed and/or modified under the
  conditions of the LaTeX Project Public License, either version 1.3
  of this license or (at your option) any later version.}}
\hypersetup{pdflicenseurl={http://www.latex-project.org/lppl.txt}}
\hypersetup{pdfcontactaddress={ETH Zurich, ITP, HIT K,
  Wolfgang-Pauli-Strasse 27}}
\hypersetup{pdfcontactpostcode={8093}}
\hypersetup{pdfcontactcity={Zurich}}
\hypersetup{pdfcontactcountry={Switzerland}}
\hypersetup{pdfcontactemail={nbeisert@itp.phys.ethz.ch}}
\hypersetup{pdfcontacturl={http://people.phys.ethz.ch/\xmptilde nbeisert/}}

\newcommand{\secref}[1]{\hyperref[#1]{section \ref*{#1}}}

\parskip1ex
\parindent0pt
\let\olditemize\itemize
\def\itemize{\olditemize\parskip0pt}

\begin{document}

\title{The \textsf{childdoc} Package}
\hypersetup{pdftitle={The childdoc Package}}
\author{Niklas Beisert\\[2ex]
  Institut f\"ur Theoretische Physik\\
  Eidgen\"ossische Technische Hochschule Z\"urich\\
  Wolfgang-Pauli-Strasse 27, 8093 Z\"urich, Switzerland\\[1ex]
  \href{mailto:nbeisert@itp.phys.ethz.ch}
  {\texttt{nbeisert@itp.phys.ethz.ch}}}
\hypersetup{pdfauthor={Niklas Beisert}}
\hypersetup{pdfsubject={Manual for the LaTeX2e Package childdoc}}
\date{30 December 2018, \textsf{v2.0}}
\maketitle

\begin{abstract}\noindent
\textsf{childdoc} is a \LaTeXe{} package
that enables the direct compilation
of document sections included by |\include|
to individual files.
\end{abstract}

\begingroup
\parskip0ex
\tableofcontents
\endgroup

%%%%%%%%%%%%%%%%%%%%%%%%%%%%%%%%%%%%%%%%%%%%%%%%%%%%%%%%%%%%%%%%%%%%%%%%%%%%%%%%
%%%%%%%%%%%%%%%%%%%%%%%%%%%%%%%%%%%%%%%%%%%%%%%%%%%%%%%%%%%%%%%%%%%%%%%%%%%%%%%%
\section{Introduction}

\LaTeX{} provides a mechanism to structure a large document (such as a book)
into a main file and several child files (containing the chapters)
using the |\include| command.
This mechanism is beneficial for documents
which span hundreds of pages in order to
make the source file(s) more manageable.
Moreover, compilation can be restricted to
selected child files by means of the |\includeonly| command.
The latter feature can be used to reduce the compilation time while editing
(this was significantly more useful in the earlier days of \LaTeX{})
or to generate a smaller document which is easier to navigate.
Another application of |\includeonly| is to generate
documents consisting of selected parts of the complete document.

However, there are a few drawbacks of the plain |\include| mechanism:
\begin{itemize}
\item
The child files cannot be compiled on their own,
they can only be compiled via the main file.
A naive editing environment
(such as a text editor with an option
to have the current file processed by \LaTeX)
may require one to switch to the main file before compiling;
attempting to compile the child file produces errors.
\item
The main file must be modified (each time)
to adjust the |\includeonly| command
to the present needs. This easily leaves the main file in a messy state.
\item
The generated document will always carry the filename
of the main document. This is inconvenient if
several child files are to be compiled and
to be kept for distribution.
\end{itemize}

The present package provides a simple interface
to make child files individually compilable by \LaTeX{}.
Compiling a child file then has the same effect as compiling
the main file with an |\includeonly| command
to select the appropriate child.
Moreover the generated document will carry the name of the child
rather than the main file.
This resolves all three above issues.

This feature is meant to make the editing of books,
thesis documents and lecture notes somewhat more convenient.
However, the package can also be used efficiently for
composing a series of documents (such as exercise sheets)
which are typically distributed individually.
It then assists the author in generating the individual documents
(potentially in different versions)
as well as a document containing the collected series.
Another application is in developing style files
or other kinds of included material
where compilation of the style file could redirect
to a sample or test file.

%%%%%%%%%%%%%%%%%%%%%%%%%%%%%%%%%%%%%%%%%%%%%%%%%%%%%%%%%%%%%%%%%%%%%%%%%%%%%%%%
%%%%%%%%%%%%%%%%%%%%%%%%%%%%%%%%%%%%%%%%%%%%%%%%%%%%%%%%%%%%%%%%%%%%%%%%%%%%%%%%
\section{Usage}

First of all, the package \textsf{childdoc} is \emph{not} a standard
\LaTeXe{} |.sty| style file! Therefore it needs to be invoked in
a non-standard way.

%%%%%%%%%%%%%%%%%%%%%%%%%%%%%%%%%%%%%%%%%%%%%%%%%%%%%%%%%%%%%%%%%%%%%%%%%%%%%%%%
\subsection{Included Files}
\label{sec:include}

%%%%%%%%%%%%%%%%%%%%%%%%%%%%%%%%%%%%%%%%
\DescribeMacro{\childdocmain}
To use the package, add the commands
\begin{center}
\begin{tabular}{l}
|\input{childdoc.def}|\\
|\childdocmain{}|\\
\end{tabular}
\end{center}
at the very top of the main \LaTeX{} file,
in particular \emph{before} the |\documentclass| statement!
The argument of |\childdocmain| should be left empty
(but it must be present).

%%%%%%%%%%%%%%%%%%%%%%%%%%%%%%%%%%%%%%%%
\DescribeMacro{\childdocof}
Furthermore, add the commands
\begin{center}
\begin{tabular}{l}
|\input{childdoc.def}|\\
|\childdocof{|\textit{main}|}|\\
\end{tabular}
\end{center}
at the top of every child file \textit{child}
which is included by |\include{|\textit{child}|}|
from within the main file
(or at least for those files to be compiled individually).
The argument \textit{main} must be the filename of the main file.

There are a couple of
considerations in setting up the main and child documents:

%%%%%%%%%%%%%%%%%%%%%%%%%%%%%%%%%%%%%%%%
\paragraph{Restrictions.}

Please note the following restrictions:
\begin{itemize}
\item
|\childdocmain| must be called with one argument \textit{main}
to ensure compatibility with earlier version of the package.
It must either be empty (|\childdocmain{}|)
or precisely match the filename of the main file in which it is specified.
See \secref{sec:detection} for further information.
\item
The filename \textit{main} must be specified without the |.tex| extension.
\item
The filename \textit{main} is case sensitive
(even in case-insensitive file systems)
due to internal string comparison.
\item
The argument \textit{main} should be fully expanded, it cannot be a macro.
\item
Subdirectories and special characters should be avoided in filenames.
\item
The command |\childdocmain{|\textit{main}|}| must be followed by a whitespace.
It should not be followed immediately by another command
or by a comment mark `|%|'.
This is because the \TeX{} parser reads the token immediately following
the argument of |\childdocmain| and puts it
at the beginning of every child section;
however, a white\-space is ignored.
\end{itemize}

%%%%%%%%%%%%%%%%%%%%%%%%%%%%%%%%%%%%%%%%
\paragraph{Content of Main File.}

It is advisable to place all content in the child files included by |\include|.
Any output contained in the main file will appear in all child documents
unless suppressed manually;
it cannot be suppressed automatically by the |\includeonly| directive
and thus should normally be avoided.
A method to include some content in the main file
by means of conditional processing is described in \secref{sec:conditional}.

%%%%%%%%%%%%%%%%%%%%%%%%%%%%%%%%%%%%%%%%
\paragraph{Page Numbering.}

When only a part of the document is compiled,
the appropriate numbering of pages
(as well as other status parameters)
is determined from the |.aux| files.
The latter contain information from previous passes.
However this information needs to propagate through
all intermediate child documents.
Therefore the page numbering in child documents may well
be inconsistent until the complete document is compiled at least once.

A useful (if unconventional) way to always ensure a consistent
page numbering is to restart the numbering in each child document
and denote the pages by `\textit{child}|.|\textit{page}'
where \textit{child} represents the chapter/section number of the child file.
This can be achieved by the command
|\numberwithin{page}{|\textit{child}|}|
of the \textsf{amsmath} package
where \textit{child} can be |chapter| or |section|
depending on the chosen structuring.
Alternatively, one can modify the macro |\thepage| appropriately
and reset the counter |page| at the start of each child file.

%%%%%%%%%%%%%%%%%%%%%%%%%%%%%%%%%%%%%%%%%%%%%%%%%%%%%%%%%%%%%%%%%%%%%%%%%%%%%%%%
\subsection{Conditional Processing}
\label{sec:conditional}

The package provides a mechanism to compile different versions
of a document. To customise the versions further some conditional processing
can come in handy to distinguish which version is being compiled.
The package provides two macros to describe the compilation context:

%%%%%%%%%%%%%%%%%%%%%%%%%%%%%%%%%%%%%%%%
\DescribeMacro{\ifchilddoc}
The conditional |\ifchilddoc| distinguishes between the compilation of
child documents and the main document:
%
\begin{center}
|\ifchilddoc |\textit{child-code}| |[|\||else |\textit{main-code}]| \||fi|
\end{center}

%%%%%%%%%%%%%%%%%%%%%%%%%%%%%%%%%%%%%%%%
\DescribeMacro{\childdocname}
\DescribeMacro{\childdocjob}
The macro |\childdocname| contains the filename (without extension)
of the main or child file being processed.
Note that |\childdocjob| will always contain the name of the main file.

%%%%%%%%%%%%%%%%%%%%%%%%%%%%%%%%%%%%%%%%
\paragraph{Title Page.}

Conditional processing can be used to include a title or banner page
in the main document when proper precautions are taken.
Importantly, the code in the main file should ensure that the page counter
(as well as other status parameters which are stored in the |.aux| files)
takes the same value after the conditional processing.
Otherwise the page numbers may take divergent values
depending on which part is compiled.

For example, a title page could be declared by:
%
\begin{center}
\begin{tabular}{l}
|\ifchilddoc\||else|\\
|\addtocounter{page}{-1}|\\
\textit{code for title page}\\
|\newpage|\\
|\||fi|
\end{tabular}
\end{center}
%
A banner page for the child documents can be generated by:
%
\begin{center}
\begin{tabular}{l}
|\ifchilddoc|\\
|\addtocounter{page}{-1}|\\
\textit{code for banner page}\\
|\newpage|\\
|\||fi|
\end{tabular}
\end{center}
%
Here one could write a message such as:
\begin{center}
|This is the part \childdocname{} of \childdocjob{}.|
\end{center}

%%%%%%%%%%%%%%%%%%%%%%%%%%%%%%%%%%%%%%%%%%%%%%%%%%%%%%%%%%%%%%%%%%%%%%%%%%%%%%%%
\subsection{Flags}
\label{sec:flags}

The package makes it easy to generate different versions
of the main or child documents.
To this end compilation flags can be defined
and assigned different default values.
They will be particularly useful in conjunction
with the forwarding mechanism described in \secref{sec:forward}.

For example, it may be useful to have a flag |\version|
which can be set to |draft| or |final|.
The document source will contain some conditional code
depending on the value of |\version|.
Suppose further, the flag should default to |final| for the main file
and to |draft| for child files
which is a natural assignment for editing the document.
This is achieved by placing the following code
in the preamble of the main document
(below the |\childdocmain| directive):
%
\begin{center}
\begin{tabular}{l}
|\ifchilddoc|\\
|\providecommand{\version}{draft}|\\
|\||else|\\
|\providecommand{\version}{final}|\\
|\||fi|
\end{tabular}
\end{center}
%
The definition by |\providecommand| makes sure
that previous definitions are not overwritten.
Further statements |\providecommand{\version}{...}|
can thus be added before the above code to override it.

For the main file, one might add a line
(between |\childdocmain| and the above block)
%
\begin{center}
|%\ifchilddoc\||else\providecommand{\version}{draft}\||fi|
\end{center}
%
which can be uncommented to produce a draft version.
Likewise one can add a line to the very top of a child file
(above the |\childdocof{|\textit{main}|}| directive)
%
\begin{center}
|%\providecommand{\version}{final}|
\end{center}
%
which can be uncommented to produce the final version of this child document.

%%%%%%%%%%%%%%%%%%%%%%%%%%%%%%%%%%%%%%%%%%%%%%%%%%%%%%%%%%%%%%%%%%%%%%%%%%%%%%%%
\subsection{Forwarding}
\label{sec:forward}

Different versions of the main or child documents
using compilation flags as described in \secref{sec:flags}
can be (permanently) stored in different files
for convenient compilation, viewing and distribution.
To this end, the package defines a command
to pass on compilation to a different file:

%%%%%%%%%%%%%%%%%%%%%%%%%%%%%%%%%%%%%%%%
\DescribeMacro{\childdocforward}
The command |\childdocforward| redirects processing to
another source file:
%
\begin{center}
\begin{tabular}{l}
|\input{childdoc.def}|\\
|\childdocforward[|\textit{main}|]{|\textit{dest}|}|\\
\end{tabular}
\end{center}
%
The argument \textit{dest} is the destination file
(without extension).
It should be the main file or one of the child files.
Note that further \textsf{childdoc} directives
such as |\childdocof| and |\childdocforward|
in the indicated file will be processed in this form.
The optional argument \textit{main}
passes on directly to the main file \textit{main}
while pretending to compile the child \textit{dest}.
This form behaves as if \textit{dest}
issues |\childdocof{|\textit{main}|}| right away,
and no further \textsf{childdoc} directives will be processed.

%%%%%%%%%%%%%%%%%%%%%%%%%%%%%%%%%%%%%%%%
\DescribeMacro{\...prefix}
In the alternative form |\childdocforwardprefix|,
%
\begin{center}
\begin{tabular}{l}
|\input{childdoc.def}|\\
|\childdocforwardprefix[|\textit{main}|]{|\textit{prefix}|}{|\textit{dest}|}|
\end{tabular}
\end{center}
%
the destination file is determined by a pattern
depending on the current file:
To make this work, the current file must be called
`{\textit{prefix}\hspace{0.2em}\textit{suffix}}'
with \textit{prefix} matching precisely the argument.
Processing is then passed on to the file
`{\textit{dest}\hspace{0.2em}\textit{suffix}}'.
Surely, the same effect is achieved by
directly specifying the
argument `{\textit{dest}\hspace{0.2em}\textit{suffix}}'
in the first form.
However, that requires to set up a different file
for each child. With the alternative form of the command
all these files can have exactly the same content
which simplifies setting them up and maintaining them.

For example, the following file |draft.tex|
with a compilation flag |\version| as described in \secref{sec:flags}
compiles the main document as a draft:
%
\begin{center}
\begin{tabular}{l}
|\def\version{draft}|\\
|\input{childdoc.def}|\\
|\childdocforward{|\textit{main}|}|
\end{tabular}
\end{center}
%
Likewise, the following files |final|\textit{nn}|.tex|
compile the final version of the child document
|child|\textit{nn}|.tex|:
%
\begin{center}
\begin{tabular}{l}
|\def\version{final}|\\
|\input{childdoc.def}|\\
|\childdocforwardprefix{final}{child}|
\end{tabular}
\end{center}
%

Note that when several versions of a main file and/or of each child file
are to be generated, it may be convenient to set up a |Makefile| or
shell script to automatise the process.

%%%%%%%%%%%%%%%%%%%%%%%%%%%%%%%%%%%%%%%%%%%%%%%%%%%%%%%%%%%%%%%%%%%%%%%%%%%%%%%%
\subsection{Command Line Processing}
\label{sec:commandline}

The effect of redirection files can also be achieved by invoking
the \LaTeX{} compiler with a more elaborate command line.
Most conveniently this should be done as part
of a shell script or a |Makefile|.

When using \textsf{childdoc} in the main file, the following
command lines effectively perform a redirection
(note that depending on the shell being used,
backslashes may have to be doubled: `|\|' $\to$ `|\\|'):
%
\begin{center}
|... -jobname "|\textit{target}|" |\\|"|[\textit{flags}]%
|\input{childdoc.def}\childdocforward[|\textit{main}|]{|\textit{dest}|}"|
\end{center}
%
Here \textit{target} is the name of the output file,
\textit{main} is the name of the main file
and \textit{dest} is the name of the main or child file to be processed
(all filenames without extensions).
The optional argument \textit{main} can be omitted
if \textit{main} matches \textit{dest}.
Optionally, compilation \textit{flags} can be defined via |\def| commands.
This command line makes the \TeX{} engine believe
it is compiling the file \textit{target}
whose content is specified as the latter parameter.
The provided code then forwards the processing to
\textit{main} or \textit{dest} as described in \secref{sec:forward}.

%%%%%%%%%%%%%%%%%%%%%%%%%%%%%%%%%%%%%%%%%%%%%%%%%%%%%%%%%%%%%%%%%%%%%%%%%%%%%%%%
\subsection{Include by Input}
\label{sec:input}

Including child documents by |\include| has some restrictions by design.
Most notably, the content of a child document always occupies
its own set of pages; pages cannot be shared between child documents.
Usually, this behaviour makes perfect sense
because each child document contain an essential part of the document.
However, in some situations it may be desirable to compose
a document from a collection of parts
without having mandatory page breaks between then.
For this case, the package
provides a mechanism to include parts
by |\input| which can also be processed individually.
However, by construction this mechanism
requires manual handling of the content to be output.

%%%%%%%%%%%%%%%%%%%%%%%%%%%%%%%%%%%%%%%%
\DescribeMacro{\ifchilddocmanual}
The main file should be prepared as usual, see \secref{sec:include}.
However, the document body must make a distinction
between processing of an individual part and of the main document, e.g.:
%
\begin{center}
\begin{tabular}{l}
|\ifchilddocmanual|\\
|\input{\childdocname}|\\
|\||else|\\
\textit{document body with }|\input{|\textit{part}|}|\\
|\||fi|
\end{tabular}
\end{center}
%
The conditional |\ifchilddocmanual| is true whenever
a part to be included by |\input| is being compiled,
and the name of the part is stored in |\childdocname|.

%%%%%%%%%%%%%%%%%%%%%%%%%%%%%%%%%%%%%%%%
\DescribeMacro{\childdocby}
Each part to be included by |\input| should start with:
%
\begin{center}
\begin{tabular}{l}
|\input{childdoc.def}|\\
|\childdocby{|\textit{main}|}|\\
\end{tabular}
\end{center}
%
The directive |\childdocby| is similar to |\childdocof|
described in \secref{sec:include},
but the subsequent selection of content must be done manually.
To that end, both |\ifchilddoc| and |\ifchilddocmanual|
will be true upon processing of a part,
and the name of the part is stored in |\childdocname|.
Note that |\jobname| will be set to the filename of the current part
so that each part receives an individual |.aux| file
that does not interfere with the |.aux| file(s) of the main document.
This behaviour can be altered by the alternative form
|\childdocby[*]{|\textit{main}|}| (with a non-empty optional argument)
which uses the |.aux| file of the main document
by setting |\jobname| to \textit{main}.

%%%%%%%%%%%%%%%%%%%%%%%%%%%%%%%%%%%%%%%%%%%%%%%%%%%%%%%%%%%%%%%%%%%%%%%%%%%%%%%%
\subsection{Driver Development}
\label{sec:driver}

The \textsf{childdoc} mechanism can also be use for the development
of definition files such as \LaTeX{} styles or classes.
This case differs from the above setup with multiple parts
included by |\include| in that no |\includeonly| should be invoked.
This can be achieved by starting the include file
(before |\ProvidesPackage|) with:
%
\begin{center}
\begin{tabular}{l}
|\input{childdoc.def}|\\
|\childdocforward{|\textit{main}|}|\\
\end{tabular}
\end{center}
%
or alternatively with:
%
\begin{center}
\begin{tabular}{l}
|\input{childdoc.def}|\\
|\childdocby{|\textit{main}|}|\\
\end{tabular}
\end{center}
%
Both forms have slightly different effects as described above.
The main file is prepared as usual, see \secref{sec:include}.

%%%%%%%%%%%%%%%%%%%%%%%%%%%%%%%%%%%%%%%%%%%%%%%%%%%%%%%%%%%%%%%%%%%%%%%%%%%%%%%%
\subsection{Legacy Detection}
\label{sec:detection}

The directive |\childdocmain| in the main file can detect
whether the complete document or merely a child is to be compiled
even without using the directive |\childdocof|.
This method is deprecated because it is less robust
and there is no compelling reason to use it;
it is merely provided for backward compatibility
and it may be removed in future versions.

If the detection mechanism is to be used,
it is mandatory to correctly specify
the filename of the main file as the argument of |\childdocmain|:
%
\begin{center}
\begin{tabular}{l}
|\input{childdoc.def}|\\
|\childdocmain{|\textit{main}|}|\\
\end{tabular}
\end{center}
%
If |\jobname| does not match the argument \textit{main} of |\childdocmain|,
it is assumed that |\jobname| points to the child file to be compiled.
When using |\childdocmain| with the main file specified as argument,
it suffices to start a child file
with just |\input{|\textit{main}|}|
without loading of the package and using |\childdocof|.
If instead all processing is done
with the appropriate \textsf{childdoc} directives,
the argument of \textit{main} of |\childdocmain| can be empty.

An alternative version of the command line processing described
in \secref{sec:commandline} using the detection mechanism reads:
%
\begin{center}
|... -jobname "|\textit{target}|" "|[\textit{flags}]%
[|\def\jobname{|\textit{dest}|}|]|\input{|\textit{main}|}"|
\end{center}

%%%%%%%%%%%%%%%%%%%%%%%%%%%%%%%%%%%%%%%%%%%%%%%%%%%%%%%%%%%%%%%%%%%%%%%%%%%%%%%%
\subsection{Manual Code}
\label{sec:manual}

In case one cannot be certain whether the definitions file |childdoc.def|
is installed on the target \TeX{} distribution
and one prefers not to ship it,
it is conceivable to paste a few relevant commands into the sources.

To that end, drop all statements |\input{childdoc.def}|
and perform the replacements as outlined below.
Instead of |\childdocmain{|\textit{main}|}| add the following code
to the top of the main file:
%
\begin{center}
\begin{tabular}{l}
|\||ifdefined\childdocname\endinput\||fi\newif\ifchilddoc|\\
|\edef\childdocname{\scantokens\expandafter{\jobname\noexpand}}|\\
|\def\childdocmain{|\textit{main}|}\||ifx\childdocmain\childdocname\||else|\\
|\childdoctrue\includeonly{\childdocname}\let\jobname\childdocmain\||fi|\\
\end{tabular}
\end{center}
%
Instead of |\childdocof{|\textit{main}|}| just include the main file
at the top of each child file:
%
\begin{center}
|\input{|\textit{main}|}|
\end{center}
%
A simple redirection |\childdocforward{|\textit{dest}|}| is achieved by:
%
\begin{center}
|\def\jobname{|\textit{dest}|}\input{\jobname}|
\end{center}
%
The redirection with prefix
|\childdocforwardprefix[|\textit{prefix}|]{|\textit{dest}|}|
is accomplished by:
%
\begin{center}
\begin{tabular}{l}
|{\edef\jobname{\scantokens\expandafter{\jobname\noexpand}}|\\
|\def\redirectjob |\textit{prefix}|#1~~~{\gdef\jobname{|\textit{dest}|#1}}|\\
|\expandafter\redirectjob\jobname~~~}\input{\jobname}|
\end{tabular}
\end{center}

In an alternative approach,
child documents can be compiled by a specific command line
without additional code or specific definitions:
%
\begin{center}
|... -jobname "|\textit{target}|" "|[\textit{flags}]%
|\includeonly{|\textit{dest}|}\input{|\textit{main}|}"|
\end{center}
%

%%%%%%%%%%%%%%%%%%%%%%%%%%%%%%%%%%%%%%%%%%%%%%%%%%%%%%%%%%%%%%%%%%%%%%%%%%%%%%%%
%%%%%%%%%%%%%%%%%%%%%%%%%%%%%%%%%%%%%%%%%%%%%%%%%%%%%%%%%%%%%%%%%%%%%%%%%%%%%%%%
\section{Information}

%%%%%%%%%%%%%%%%%%%%%%%%%%%%%%%%%%%%%%%%%%%%%%%%%%%%%%%%%%%%%%%%%%%%%%%%%%%%%%%%
\subsection{Copyright}

Copyright \copyright{} 2017--2018 Niklas Beisert

This work may be distributed and/or modified under the
conditions of the \LaTeX{} Project Public License, either version 1.3
of this license or (at your option) any later version.
The latest version of this license is in
  \url{http://www.latex-project.org/lppl.txt}
and version 1.3 or later is part of all distributions of \LaTeX{}
version 2005/12/01 or later.

This work has the LPPL maintenance status `maintained'.

The Current Maintainer of this work is Niklas Beisert.

This work consists of the files |README.txt|, |childdoc.ins| and |childdoc.dtx|
as well as the derived files |childdoc.def|, |cdocsamp.tex|
with |cdocsch1.tex|, |cdocsch2.tex|, |cdocspt3.tex|, |cdocspt4.tex|,
|cdocsdrf.tex|, |cdocsfn1.tex|, |cdocsfn2.tex|
as well as |childdoc.pdf|.

%%%%%%%%%%%%%%%%%%%%%%%%%%%%%%%%%%%%%%%%%%%%%%%%%%%%%%%%%%%%%%%%%%%%%%%%%%%%%%%%
\subsection{Files and Installation}

The package consists of the files:
%
\begin{center}
\begin{tabular}{ll}
    |README.txt|   & readme file \\
    |childdoc.ins| & installation file \\
    |childdoc.dtx| & source file \\
    |childdoc.def| & definition file \\
    |cdocsamp.tex| & sample main file \\
    |cdocsch1.tex| & sample include file \\
    |cdocsch2.tex| & sample include file \\
    |cdocspt3.tex| & sample part file \\
    |cdocspt4.tex| & sample part file \\
    |cdocsdrf.tex| & sample redirection file \\
    |cdocsfn1.tex| & sample redirection file \\
    |cdocsfn2.tex| & sample redirection file \\
    |childdoc.pdf| & manual
\end{tabular}
\end{center}
%
The distribution consists of the files
|README.txt|, |childdoc.ins| and |childdoc.dtx|.
%
\begin{itemize}
\item
Run (pdf)\LaTeX{} on |childdoc.dtx|
to compile the manual |childdoc.pdf| (this file).
\item
Run \LaTeX{} on |childdoc.ins| to create the definitions file |childdoc.def|
and the sample |cdocsamp.tex| with include files
|cdocsch1.tex|, |cdocsch2.tex|, |cdocspt3.tex|, |cdocspt4.tex|,
|cdocsdrf.tex|, |cdocsfn1.tex|, |cdocsfn2.tex|.
Then copy the file |childdoc.def| to an appropriate directory of your \LaTeX{}
distribution, e.g.\ \textit{texmf-root}|/tex/latex/childdoc|.
\end{itemize}

%%%%%%%%%%%%%%%%%%%%%%%%%%%%%%%%%%%%%%%%%%%%%%%%%%%%%%%%%%%%%%%%%%%%%%%%%%%%%%%%
\subsection{Related CTAN Packages}

There are several other packages which offer a similar functionality:
%
\begin{itemize}
\item
The packages
\href{http://ctan.org/pkg/docmute}{\textsf{docmute}},
\href{http://ctan.org/pkg/includex}{\textsf{includex}} and
\href{http://ctan.org/pkg/standalone}{\textsf{standalone}}
provide commands to include only the document body of
a child file thus allowing both files to be compiled individually.
\item
The packages \href{http://ctan.org/pkg/subdocs}{\textsf{subdocs}}
and \href{http://ctan.org/pkg/subfiles}{\textsf{subfiles}}
provide structures in which the main and child documents can be
encapsulated and allowing them to be compiled individually.
The inclusion mechanism is different from the conventional |\include|.
\item
The package \href{http://ctan.org/pkg/combine}{\textsf{combine}}
is an elaborate solution to combine several documents into one.
\end{itemize}
%
See also the CTAN topic \href{http://ctan.org/topic/subdocs}{\textsf{subdocs}}
for further related packages.
The present package differs from the above solutions in that
a document structure constructed with the conventional |\include| mechanism
just needs two extra commands at the top of every file
such that all constituent files can be compiled individually.

%%%%%%%%%%%%%%%%%%%%%%%%%%%%%%%%%%%%%%%%%%%%%%%%%%%%%%%%%%%%%%%%%%%%%%%%%%%%%%%%
%\subsection{Feature Suggestions}
%
%The following is a list of features which may be useful for future
%versions of this package:
%%
%\begin{itemize}
%\item
%\ldots
%\end{itemize}

%%%%%%%%%%%%%%%%%%%%%%%%%%%%%%%%%%%%%%%%%%%%%%%%%%%%%%%%%%%%%%%%%%%%%%%%%%%%%%%%
\subsection{Revision History}

%%%%%%%%%%%%%%%%%%%%%%%%%%%%%%%%%%%%%%%%
\paragraph{v2.0:} 2018/12/30

\begin{itemize}
\item
immediate forward processing
\item
added |\childdocby| mechanism
\item
manual restructured
\end{itemize}

%%%%%%%%%%%%%%%%%%%%%%%%%%%%%%%%%%%%%%%%
\paragraph{v1.6:} 2018/01/17

\begin{itemize}
\item
application for development of include files
\item
corrections to manual
\end{itemize}

%%%%%%%%%%%%%%%%%%%%%%%%%%%%%%%%%%%%%%%%
\paragraph{v1.5:} 2017/05/21

\begin{itemize}
\item
more complete structuring introduced
\item
|\childdocof| introduced
\item
|\childdoc| renamed to |\childdocmain|
\item
|\childredirect| renamed to |\childdocforward| and |\childdocforwardprefix|
and functionality expanded
\end{itemize}

%%%%%%%%%%%%%%%%%%%%%%%%%%%%%%%%%%%%%%%%
\paragraph{v1.0:} 2017/04/27

\begin{itemize}
\item
manual and install package
\item
first version published on CTAN
\end{itemize}

%%%%%%%%%%%%%%%%%%%%%%%%%%%%%%%%%%%%%%%%
\paragraph{v0.6:} 2017/04/26

\begin{itemize}
\item
redirection mechanism added
\end{itemize}

%%%%%%%%%%%%%%%%%%%%%%%%%%%%%%%%%%%%%%%%
\paragraph{v0.5:} 2017/04/26

\begin{itemize}
\item
functionality in definition file
\end{itemize}


%%%%%%%%%%%%%%%%%%%%%%%%%%%%%%%%%%%%%%%%%%%%%%%%%%%%%%%%%%%%%%%%%%%%%%%%%%%%%%%%
%%%%%%%%%%%%%%%%%%%%%%%%%%%%%%%%%%%%%%%%%%%%%%%%%%%%%%%%%%%%%%%%%%%%%%%%%%%%%%%%
%%%%%%%%%%%%%%%%%%%%%%%%%%%%%%%%%%%%%%%%%%%%%%%%%%%%%%%%%%%%%%%%%%%%%%%%%%%%%%%%
\appendix

\settowidth\MacroIndent{\rmfamily\scriptsize 000\ }

 \DocInput{childdoc.dtx}

\end{document}
%</driver>
% \fi
%
% %%%%%%%%%%%%%%%%%%%%%%%%%%%%%%%%%%%%%%%%%%%%%%%%%%%%%%%%%%%%%%%%%%%%%%%%%%%%%%
% %%%%%%%%%%%%%%%%%%%%%%%%%%%%%%%%%%%%%%%%%%%%%%%%%%%%%%%%%%%%%%%%%%%%%%%%%%%%%%
% \section{Sample}
%\iffalse
%<*samplemain>
%\fi
%
% The following presents a sample document
% with two chapters, two parts, a title page,
% a compile flag as well as three forwarding files to set the flag.
% It consists of eight |.tex| files:
% \begin{center}
% \begin{tabular}{ll}
% |cdocsamp.tex|&main file\\
% |cdocsch1.tex|&include file for chapter 1\\
% |cdocsch2.tex|&include file for chapter 2\\
% |cdocspt3.tex|&include file for part 3\\
% |cdocspt4.tex|&include file for part 4\\
% |cdocsdrf.tex|&forwarding file for main file in draft mode\\
% |cdocsfi1.tex|&forwarding file for final version of chapter 1\\
% |cdocsfi2.tex|&forwarding file for final version of chapter 2\\
% \end{tabular}
% \end{center}
% Each of the eight files can be compiled directly by the \LaTeX{} compiler.
%
% %%%%%%%%%%%%%%%%%%%%%%%%%%%%%%%%%%%%%%
% \paragraph{Main File.}
%
% The main file is called |cdocsamp.tex|.
%
% Load the \textsf{childdoc} definitions and
% declare the filename for the main document:
%    \begin{macrocode}
\input{childdoc.def}
\childdocmain{}
%    \end{macrocode}

% Optional override for |\version| flag:
%    \begin{macrocode}
%%\ifchilddoc\else\providecommand{\version}{draft}\fi
%    \end{macrocode}

% Define the default values for the |\version| flag
% (|final| for the main file and |draft| for childs):
%    \begin{macrocode}
\ifchilddoc
\providecommand{\version}{draft}
\else
\providecommand{\version}{final}
\fi
%    \end{macrocode}

% Load the standard document class:
%    \begin{macrocode}
\documentclass[12pt]{article}
%    \end{macrocode}

% Start the document body:
%    \begin{macrocode}
\begin{document}
%    \end{macrocode}

% Declare a title page.
% Print title, part of document being processed and version flag:
%    \begin{macrocode}
\addtocounter{page}{-1}
\begin{center}
{\LARGE\bfseries{}childdoc example\par}
\vspace{1cm}
\ifchilddoc
\ifchilddocmanual part\else chapter\fi:
`\childdocname' of `\childdocjob'\par
\else
main document: `\childdocjob'\par
\fi
version: \version\par
\end{center}
\newpage
%    \end{macrocode}

% Manually include selected file,
% otherwise process as usual:
%    \begin{macrocode}
\ifchilddocmanual
\section*{part `\childdocname'}
\input{\childdocname}
\else
%    \end{macrocode}

% Include the two chapters:
%    \begin{macrocode}
\include{cdocsch1}
\include{cdocsch2}
%    \end{macrocode}

% Include the two parts unless only chapters should be displayed:
%    \begin{macrocode}
\ifchilddoc\else
\section{part three}
\input{cdocspt3}
\section{part four}
\input{cdocspt4}
\fi
%    \end{macrocode}

% Process as usual until here:
%    \begin{macrocode}
\fi
%    \end{macrocode}

% End of document body:
%    \begin{macrocode}
\end{document}
%    \end{macrocode}
%\iffalse
%</samplemain>
%\fi
%
% %%%%%%%%%%%%%%%%%%%%%%%%%%%%%%%%%%%%%%
% \paragraph{Chapter Include Files.}
%
% The include files are called |cdocsch1.tex| and |cdocsch2.tex|.
%
%\iffalse
%<*samplechap1|samplechap2>
%\fi

% Optional override for |\version| flag:
%    \begin{macrocode}
%%\providecommand{\version}{final}
%    \end{macrocode}

% Include the main document:
%    \begin{macrocode}
\input{childdoc.def}
\childdocof{cdocsamp}
%    \end{macrocode}

%\iffalse
%</samplechap1|samplechap2>
%\fi
%
%\iffalse
%<*samplechap1>
%\fi
% Some text for chapter 1:
%    \begin{macrocode}
\section{one}
some text in chapter one
%    \end{macrocode}

%\iffalse
%</samplechap1>
%\fi
% Some text for chapter 2:
%\iffalse
%<*samplechap2>
%\fi
%    \begin{macrocode}
\section{two}
more text in chapter two
%    \end{macrocode}

%\iffalse
%</samplechap2>
%\fi
%
% %%%%%%%%%%%%%%%%%%%%%%%%%%%%%%%%%%%%%%
% \paragraph{Part Include Files.}
%
% The include files are called |cdocspt3.tex| and |cdocspt4.tex|.
%
%\iffalse
%<*samplepart3|samplepart4>
%\fi

% Optional override for |\version| flag:
%    \begin{macrocode}
%%\providecommand{\version}{final}
%    \end{macrocode}

% Include the main document:
%    \begin{macrocode}
\input{childdoc.def}
\childdocby{cdocsamp}
%    \end{macrocode}

%\iffalse
%</samplepart3|samplepart4>
%\fi
%
%\iffalse
%<*samplepart3>
%\fi
% Some text for part 3:
%    \begin{macrocode}
some text in part three
%    \end{macrocode}

%\iffalse
%</samplepart3>
%\fi
% Some text for part 4:
%\iffalse
%<*samplepart4>
%\fi
%    \begin{macrocode}
more text in part four
%    \end{macrocode}

%\iffalse
%</samplepart4>
%\fi
%
% %%%%%%%%%%%%%%%%%%%%%%%%%%%%%%%%%%%%%%
% \paragraph{Forwarding for a Complete Draft.}
%
% The following forwarding file |cdocsdrf.tex|
% compiles the main document in draft mode:
%\iffalse
%<*sampledraft>
%\fi
%    \begin{macrocode}
\def\version{draft}
\input{childdoc.def}
\childdocforward{cdocsamp}
%    \end{macrocode}

%\iffalse
%</sampledraft>
%\fi
%
% %%%%%%%%%%%%%%%%%%%%%%%%%%%%%%%%%%%%%%
% \paragraph{Forwarding for Final Version of the Chapters.}
%
% The following forwarding files |cdocsfn1.tex| and |cdocsfn2.tex|
% (with identical content)
% compile the final versions of the child documents
% |cdocsch1.tex| and |cdocsch2.tex|, respectively:
%\iffalse
%<*samplefinal>
%\fi
%    \begin{macrocode}
\def\version{final}
\input{childdoc.def}
\childdocforwardprefix[cdocsamp]{cdocsfn}{cdocsch}
%    \end{macrocode}

%\iffalse
%</samplefinal>
%\fi
%
% %%%%%%%%%%%%%%%%%%%%%%%%%%%%%%%%%%%%%%
% \paragraph{Command Line Processing.}
%
% The following three command lines generate the output files
% |cdocscld|, |cdocscl1| and |cdocscl2|
% which should be identical to
% |cdocsdrf|, |cdocsch1| and |cdocsfn2|, respectively:
% \begin{center}
% \begin{tabular}{l}
% |latex -jobname cdocscld \|\\
% |  "\def\version{draft}\input{childdoc.def}\childdocforward{cdocsamp}"|\\
% |latex -jobname cdocscl1 \|\\
% |  "\input{childdoc.def}\childdocforward[cdocsamp]{cdocsch1}"|\\
% |latex -jobname cdocscl2 \|\\
% |  "\def\version{final}\input{childdoc.def}\childdocforward{cdocsch2}"|
% \end{tabular}
% \end{center}
% Note that the trailing backslash on each first line
% merely continues the input to the second line
% (for convenient cut ant paste).
% Furthermore, the command |latex| can be replaced by any
% of its alternative versions such as |pdflatex|.
%
% %%%%%%%%%%%%%%%%%%%%%%%%%%%%%%%%%%%%%%%%%%%%%%%%%%%%%%%%%%%%%%%%%%%%%%%%%%%%%%
% %%%%%%%%%%%%%%%%%%%%%%%%%%%%%%%%%%%%%%%%%%%%%%%%%%%%%%%%%%%%%%%%%%%%%%%%%%%%%%
% \section{Implementation}
%\iffalse
%<*package>
%\fi
%
% This section describes the definitions file |childdoc.def|.

% The definitions cannot be loaded using |\usepackage| or |\RequirePackage|
% which has a mechanism to prevent loading a style file more than once.
% When loading the definitions by means of |\input|
% multiple instances have to be prevented manually:
%\iffalse
%This code needs to be before the `\ProvidesFile' directive
%which is defined at the beginning of this file.
%Therefore it is also placed there and commented out here.
%</package>
%<*discard>
%\fi
%    \begin{macrocode}
\ifdefined\childdocmain\endinput\fi
%    \end{macrocode}
%\iffalse
%</discard>
%<*package>
%\fi
%
% \macro{\ifchilddoc}
% \macro{\ifchilddocmanual}
% The conditional |\ifchilddoc| tells whether a
% child (true) or main (false) document is being compiled.
% The conditional |\ifchilddocmanual| tells whether
% the |\includeonly| mechanism is used (false) or
% the selection of child files must be performed manually (true).
% The definitions initialise to false:
%    \begin{macrocode}
\newif\ifchilddoc
\newif\ifchilddocmanual
%    \end{macrocode}

% \macro{\childdocname}
% \macro{\childdocjob}
% The macro |\childdocname| stores the name of the main document
% to be compiled. The macro |\childdocjob| stores the name of
% the document on which the \LaTeX{} compiler was originally invoked.
% The content of |\jobname| cannot be compared
% to filenames specified in the source due to different catcodes.
% The following code rescans |\jobname|, stores the result
% in |\childdocname| and saves a copy in |\childdocjob|:
%    \begin{macrocode}
\edef\childdocname{\scantokens\expandafter{\jobname\noexpand}}
\let\childdocjob\childdocname
%    \end{macrocode}

% \macro{\childdocdisable}
% The macro |\childdocdisable| prevents the main file
% from being processed more than once.
% At this stage, the main document command |\childdocmain|
% is assumed to be called once again where it should do nothing.
% Any subsequent call to it should prevent
% a secondary processing of the main document
% It overwrites the forwarding commands
% |\childdocof| and |\childdocforward|
% with empty macros to prevent further inclusions of the main document:
%    \begin{macrocode}
\newcommand{\childdocdisable}
{
  \renewcommand{\childdocmain}[1]{\renewcommand{\childdocmain}[1]{\endinput}}
  \renewcommand{\childdocof}[1]{}
  \renewcommand{\childdocby}[2][]{}
  \renewcommand{\childdocforward}[2][]{}
  \renewcommand{\childdocdisable}{}
}
%    \end{macrocode}

% \macro{\childdocmain}
% The macro |\childdocmain| is to be called at the top of the main file
% with nothing or the main filename (without extension) as argument.
% First, it breaks loops.
% If the argument is not empty and does not match |\childdocname|
% (which is set by the first inclusion of |childdoc.def|),
% |\ifchilddoc| is set to true, |\includeonly| is applied to the child file
% and |\jobname| is set to the main file
% (for proper handling of |.aux| files):
%    \begin{macrocode}
\newcommand{\childdocmain}[1]
{
  \childdocdisable\childdocmain{}
  \if?#1?\else
    \begingroup
      \def\childdoctmp{#1}
      \ifx\childdoctmp\childdocname
        \def\childdoctmp{}
      \else
        \def\childdoctmp
        {
          \childdoctrue
          \includeonly{\childdocname}
          \def\childdocjob{#1}
          \def\jobname{#1}
        }
      \fi
      \expandafter
    \endgroup
    \childdoctmp
  \fi
}
%    \end{macrocode}

% \macro{\childdocof}
% The command |\childdocof| redirects
% compilation to the main file |#1|.
%    \begin{macrocode}
\newcommand{\childdocof}[1]
{
  \childdocdisable
  \childdoctrue
  \includeonly{\childdocname}
  \def\jobname{#1}
  \def\childdocjob{#1}
  \input{#1}
}
%    \end{macrocode}

% \macro{\childdocby}
% The command |\childdocby| ....
%    \begin{macrocode}
\newcommand{\childdocby}[2][]
{
  \childdocdisable
  \childdoctrue
  \childdocmanualtrue
  \if?#1?\else
    \def\jobname{#2}
  \fi
  \def\childdocjob{#2}
  \input{#2}
  \endinput
}
%    \end{macrocode}

% \macro{\childdocforward}
% The command |\childdocforward| redirects
% compilation to the main file or
% (if the optional argument is given) a child file.
% Parameters are set as if the main file
% or a child file starting with |\childdocof| was compiled.
% Then compilation is handed over to the main file:
%    \begin{macrocode}
\newcommand{\childdocforward}[2][]
{
  \begingroup
    \if?#1?
      \def\childdoctmp
      {
        \def\childdocname{#2}
        \def\childdocjob{#2}
        \def\jobname{#2}
        \input{#2}
        \endinput
      }
    \else
      \def\childdoctmp
      {
        \childdocdisable
        \def\childdocname{#2}
        \childdoctrue
        \includeonly{#2}
        \def\childdocjob{#1}
        \def\jobname{#1}
        \input{#1}
        \endinput
      }
    \fi
    \expandafter
  \endgroup
  \childdoctmp
}
%    \end{macrocode}

% \macro{\childdocforwardprefix}
% The command |\childdocforwardprefix| redirects
% compilation to the main or a child file by means of a pattern.
% The prefix |#1| in the current filename is replaced by |#2|
% and the suffix of the current filename is kept
% (it is assumed that the filename does not contain the substring `|~~~|'
% which is used as a delimiter).
% Compilation is handed over to the new file by |\childdocforward|:
%    \begin{macrocode}
\newcommand{\childdocforwardprefix}[3][]
{
  \begingroup
    \def\childdocextract #2##1~~~{\def\childdoctmp{\childdocforward[#1]{#3##1}}}
    \expandafter\childdocextract\childdocname~~~
    \expandafter
  \endgroup
  \childdoctmp
}
%    \end{macrocode}

% \macro{\childdoc}
% The deprecated macro |\childdoc| is a legacy version of |\childdocmain|:
%    \begin{macrocode}
\newcommand{\childdoc}{\childdocmain}
%    \end{macrocode}

% \macro{\childdocredirect}
% The deprecated macro |\childdocredirect| is a legacy version
% of |\childdocforward| and |\childdocforwardprefix|:
%    \begin{macrocode}
\newcommand{\childdocredirect}[2][]
{
  \begingroup
    \if?#1?
      \def\childdoctmp{\childdocforward{#2}}
    \else
      \def\childdoctmp{\childdocforwardprefix{#1}{#2}}
    \fi
    \expandafter
  \endgroup
  \childdoctmp
}
%    \end{macrocode}

%\iffalse
%</package>
%\fi
%
\endinput
|\\
|\childdocforward{|\textit{main}|}|
\end{tabular}
\end{center}
%
Likewise, the following files |final|\textit{nn}|.tex|
compile the final version of the child document
|child|\textit{nn}|.tex|:
%
\begin{center}
\begin{tabular}{l}
|\def\version{final}|\\
|% \iffalse
%
% childdoc.dtx Copyright (C) 2017-2018 Niklas Beisert
%
% This work may be distributed and/or modified under the
% conditions of the LaTeX Project Public License, either version 1.3
% of this license or (at your option) any later version.
% The latest version of this license is in
%   http://www.latex-project.org/lppl.txt
% and version 1.3 or later is part of all distributions of LaTeX
% version 2005/12/01 or later.
%
% This work has the LPPL maintenance status `maintained'.
%
% The Current Maintainer of this work is Niklas Beisert.
%
% This work consists of the files childdoc.dtx and childdoc.ins
% and the derived files childdoc.def and cdocsamp.tex with
% cdocsch1.tex, cdocsch2.tex, cdocsdrf.tex, cdocsfn1.tex, cdocsfn2.tex.
%
%<package>\ifdefined\childdocmain\endinput\fi
%<package>\ProvidesFile{childdoc.def}[2018/12/30 v2.0 child document driver]
%<samplemain>\ProvidesFile{cdocsamp.tex}[2018/12/30 v2.0 sample for childdoc]
%<*driver>
%\ProvidesFile{childdoc.drv}[2018/12/30 v2.0 childdoc reference manual file]
\PassOptionsToClass{10pt,a4paper}{article}
\documentclass{ltxdoc}

\usepackage[margin=35mm]{geometry}
\usepackage{hyperref}
\usepackage{hyperxmp}
\usepackage[usenames]{color}

\hypersetup{colorlinks=true}
\hypersetup{pdfstartview=FitH}
\hypersetup{pdfpagemode=UseNone}
\hypersetup{pdfsource={}}
\hypersetup{pdflang={en-UK}}
\hypersetup{pdfcopyright={Copyright 2017-2018 Niklas Beisert.
  This work may be distributed and/or modified under the
  conditions of the LaTeX Project Public License, either version 1.3
  of this license or (at your option) any later version.}}
\hypersetup{pdflicenseurl={http://www.latex-project.org/lppl.txt}}
\hypersetup{pdfcontactaddress={ETH Zurich, ITP, HIT K,
  Wolfgang-Pauli-Strasse 27}}
\hypersetup{pdfcontactpostcode={8093}}
\hypersetup{pdfcontactcity={Zurich}}
\hypersetup{pdfcontactcountry={Switzerland}}
\hypersetup{pdfcontactemail={nbeisert@itp.phys.ethz.ch}}
\hypersetup{pdfcontacturl={http://people.phys.ethz.ch/\xmptilde nbeisert/}}

\newcommand{\secref}[1]{\hyperref[#1]{section \ref*{#1}}}

\parskip1ex
\parindent0pt
\let\olditemize\itemize
\def\itemize{\olditemize\parskip0pt}

\begin{document}

\title{The \textsf{childdoc} Package}
\hypersetup{pdftitle={The childdoc Package}}
\author{Niklas Beisert\\[2ex]
  Institut f\"ur Theoretische Physik\\
  Eidgen\"ossische Technische Hochschule Z\"urich\\
  Wolfgang-Pauli-Strasse 27, 8093 Z\"urich, Switzerland\\[1ex]
  \href{mailto:nbeisert@itp.phys.ethz.ch}
  {\texttt{nbeisert@itp.phys.ethz.ch}}}
\hypersetup{pdfauthor={Niklas Beisert}}
\hypersetup{pdfsubject={Manual for the LaTeX2e Package childdoc}}
\date{30 December 2018, \textsf{v2.0}}
\maketitle

\begin{abstract}\noindent
\textsf{childdoc} is a \LaTeXe{} package
that enables the direct compilation
of document sections included by |\include|
to individual files.
\end{abstract}

\begingroup
\parskip0ex
\tableofcontents
\endgroup

%%%%%%%%%%%%%%%%%%%%%%%%%%%%%%%%%%%%%%%%%%%%%%%%%%%%%%%%%%%%%%%%%%%%%%%%%%%%%%%%
%%%%%%%%%%%%%%%%%%%%%%%%%%%%%%%%%%%%%%%%%%%%%%%%%%%%%%%%%%%%%%%%%%%%%%%%%%%%%%%%
\section{Introduction}

\LaTeX{} provides a mechanism to structure a large document (such as a book)
into a main file and several child files (containing the chapters)
using the |\include| command.
This mechanism is beneficial for documents
which span hundreds of pages in order to
make the source file(s) more manageable.
Moreover, compilation can be restricted to
selected child files by means of the |\includeonly| command.
The latter feature can be used to reduce the compilation time while editing
(this was significantly more useful in the earlier days of \LaTeX{})
or to generate a smaller document which is easier to navigate.
Another application of |\includeonly| is to generate
documents consisting of selected parts of the complete document.

However, there are a few drawbacks of the plain |\include| mechanism:
\begin{itemize}
\item
The child files cannot be compiled on their own,
they can only be compiled via the main file.
A naive editing environment
(such as a text editor with an option
to have the current file processed by \LaTeX)
may require one to switch to the main file before compiling;
attempting to compile the child file produces errors.
\item
The main file must be modified (each time)
to adjust the |\includeonly| command
to the present needs. This easily leaves the main file in a messy state.
\item
The generated document will always carry the filename
of the main document. This is inconvenient if
several child files are to be compiled and
to be kept for distribution.
\end{itemize}

The present package provides a simple interface
to make child files individually compilable by \LaTeX{}.
Compiling a child file then has the same effect as compiling
the main file with an |\includeonly| command
to select the appropriate child.
Moreover the generated document will carry the name of the child
rather than the main file.
This resolves all three above issues.

This feature is meant to make the editing of books,
thesis documents and lecture notes somewhat more convenient.
However, the package can also be used efficiently for
composing a series of documents (such as exercise sheets)
which are typically distributed individually.
It then assists the author in generating the individual documents
(potentially in different versions)
as well as a document containing the collected series.
Another application is in developing style files
or other kinds of included material
where compilation of the style file could redirect
to a sample or test file.

%%%%%%%%%%%%%%%%%%%%%%%%%%%%%%%%%%%%%%%%%%%%%%%%%%%%%%%%%%%%%%%%%%%%%%%%%%%%%%%%
%%%%%%%%%%%%%%%%%%%%%%%%%%%%%%%%%%%%%%%%%%%%%%%%%%%%%%%%%%%%%%%%%%%%%%%%%%%%%%%%
\section{Usage}

First of all, the package \textsf{childdoc} is \emph{not} a standard
\LaTeXe{} |.sty| style file! Therefore it needs to be invoked in
a non-standard way.

%%%%%%%%%%%%%%%%%%%%%%%%%%%%%%%%%%%%%%%%%%%%%%%%%%%%%%%%%%%%%%%%%%%%%%%%%%%%%%%%
\subsection{Included Files}
\label{sec:include}

%%%%%%%%%%%%%%%%%%%%%%%%%%%%%%%%%%%%%%%%
\DescribeMacro{\childdocmain}
To use the package, add the commands
\begin{center}
\begin{tabular}{l}
|\input{childdoc.def}|\\
|\childdocmain{}|\\
\end{tabular}
\end{center}
at the very top of the main \LaTeX{} file,
in particular \emph{before} the |\documentclass| statement!
The argument of |\childdocmain| should be left empty
(but it must be present).

%%%%%%%%%%%%%%%%%%%%%%%%%%%%%%%%%%%%%%%%
\DescribeMacro{\childdocof}
Furthermore, add the commands
\begin{center}
\begin{tabular}{l}
|\input{childdoc.def}|\\
|\childdocof{|\textit{main}|}|\\
\end{tabular}
\end{center}
at the top of every child file \textit{child}
which is included by |\include{|\textit{child}|}|
from within the main file
(or at least for those files to be compiled individually).
The argument \textit{main} must be the filename of the main file.

There are a couple of
considerations in setting up the main and child documents:

%%%%%%%%%%%%%%%%%%%%%%%%%%%%%%%%%%%%%%%%
\paragraph{Restrictions.}

Please note the following restrictions:
\begin{itemize}
\item
|\childdocmain| must be called with one argument \textit{main}
to ensure compatibility with earlier version of the package.
It must either be empty (|\childdocmain{}|)
or precisely match the filename of the main file in which it is specified.
See \secref{sec:detection} for further information.
\item
The filename \textit{main} must be specified without the |.tex| extension.
\item
The filename \textit{main} is case sensitive
(even in case-insensitive file systems)
due to internal string comparison.
\item
The argument \textit{main} should be fully expanded, it cannot be a macro.
\item
Subdirectories and special characters should be avoided in filenames.
\item
The command |\childdocmain{|\textit{main}|}| must be followed by a whitespace.
It should not be followed immediately by another command
or by a comment mark `|%|'.
This is because the \TeX{} parser reads the token immediately following
the argument of |\childdocmain| and puts it
at the beginning of every child section;
however, a white\-space is ignored.
\end{itemize}

%%%%%%%%%%%%%%%%%%%%%%%%%%%%%%%%%%%%%%%%
\paragraph{Content of Main File.}

It is advisable to place all content in the child files included by |\include|.
Any output contained in the main file will appear in all child documents
unless suppressed manually;
it cannot be suppressed automatically by the |\includeonly| directive
and thus should normally be avoided.
A method to include some content in the main file
by means of conditional processing is described in \secref{sec:conditional}.

%%%%%%%%%%%%%%%%%%%%%%%%%%%%%%%%%%%%%%%%
\paragraph{Page Numbering.}

When only a part of the document is compiled,
the appropriate numbering of pages
(as well as other status parameters)
is determined from the |.aux| files.
The latter contain information from previous passes.
However this information needs to propagate through
all intermediate child documents.
Therefore the page numbering in child documents may well
be inconsistent until the complete document is compiled at least once.

A useful (if unconventional) way to always ensure a consistent
page numbering is to restart the numbering in each child document
and denote the pages by `\textit{child}|.|\textit{page}'
where \textit{child} represents the chapter/section number of the child file.
This can be achieved by the command
|\numberwithin{page}{|\textit{child}|}|
of the \textsf{amsmath} package
where \textit{child} can be |chapter| or |section|
depending on the chosen structuring.
Alternatively, one can modify the macro |\thepage| appropriately
and reset the counter |page| at the start of each child file.

%%%%%%%%%%%%%%%%%%%%%%%%%%%%%%%%%%%%%%%%%%%%%%%%%%%%%%%%%%%%%%%%%%%%%%%%%%%%%%%%
\subsection{Conditional Processing}
\label{sec:conditional}

The package provides a mechanism to compile different versions
of a document. To customise the versions further some conditional processing
can come in handy to distinguish which version is being compiled.
The package provides two macros to describe the compilation context:

%%%%%%%%%%%%%%%%%%%%%%%%%%%%%%%%%%%%%%%%
\DescribeMacro{\ifchilddoc}
The conditional |\ifchilddoc| distinguishes between the compilation of
child documents and the main document:
%
\begin{center}
|\ifchilddoc |\textit{child-code}| |[|\||else |\textit{main-code}]| \||fi|
\end{center}

%%%%%%%%%%%%%%%%%%%%%%%%%%%%%%%%%%%%%%%%
\DescribeMacro{\childdocname}
\DescribeMacro{\childdocjob}
The macro |\childdocname| contains the filename (without extension)
of the main or child file being processed.
Note that |\childdocjob| will always contain the name of the main file.

%%%%%%%%%%%%%%%%%%%%%%%%%%%%%%%%%%%%%%%%
\paragraph{Title Page.}

Conditional processing can be used to include a title or banner page
in the main document when proper precautions are taken.
Importantly, the code in the main file should ensure that the page counter
(as well as other status parameters which are stored in the |.aux| files)
takes the same value after the conditional processing.
Otherwise the page numbers may take divergent values
depending on which part is compiled.

For example, a title page could be declared by:
%
\begin{center}
\begin{tabular}{l}
|\ifchilddoc\||else|\\
|\addtocounter{page}{-1}|\\
\textit{code for title page}\\
|\newpage|\\
|\||fi|
\end{tabular}
\end{center}
%
A banner page for the child documents can be generated by:
%
\begin{center}
\begin{tabular}{l}
|\ifchilddoc|\\
|\addtocounter{page}{-1}|\\
\textit{code for banner page}\\
|\newpage|\\
|\||fi|
\end{tabular}
\end{center}
%
Here one could write a message such as:
\begin{center}
|This is the part \childdocname{} of \childdocjob{}.|
\end{center}

%%%%%%%%%%%%%%%%%%%%%%%%%%%%%%%%%%%%%%%%%%%%%%%%%%%%%%%%%%%%%%%%%%%%%%%%%%%%%%%%
\subsection{Flags}
\label{sec:flags}

The package makes it easy to generate different versions
of the main or child documents.
To this end compilation flags can be defined
and assigned different default values.
They will be particularly useful in conjunction
with the forwarding mechanism described in \secref{sec:forward}.

For example, it may be useful to have a flag |\version|
which can be set to |draft| or |final|.
The document source will contain some conditional code
depending on the value of |\version|.
Suppose further, the flag should default to |final| for the main file
and to |draft| for child files
which is a natural assignment for editing the document.
This is achieved by placing the following code
in the preamble of the main document
(below the |\childdocmain| directive):
%
\begin{center}
\begin{tabular}{l}
|\ifchilddoc|\\
|\providecommand{\version}{draft}|\\
|\||else|\\
|\providecommand{\version}{final}|\\
|\||fi|
\end{tabular}
\end{center}
%
The definition by |\providecommand| makes sure
that previous definitions are not overwritten.
Further statements |\providecommand{\version}{...}|
can thus be added before the above code to override it.

For the main file, one might add a line
(between |\childdocmain| and the above block)
%
\begin{center}
|%\ifchilddoc\||else\providecommand{\version}{draft}\||fi|
\end{center}
%
which can be uncommented to produce a draft version.
Likewise one can add a line to the very top of a child file
(above the |\childdocof{|\textit{main}|}| directive)
%
\begin{center}
|%\providecommand{\version}{final}|
\end{center}
%
which can be uncommented to produce the final version of this child document.

%%%%%%%%%%%%%%%%%%%%%%%%%%%%%%%%%%%%%%%%%%%%%%%%%%%%%%%%%%%%%%%%%%%%%%%%%%%%%%%%
\subsection{Forwarding}
\label{sec:forward}

Different versions of the main or child documents
using compilation flags as described in \secref{sec:flags}
can be (permanently) stored in different files
for convenient compilation, viewing and distribution.
To this end, the package defines a command
to pass on compilation to a different file:

%%%%%%%%%%%%%%%%%%%%%%%%%%%%%%%%%%%%%%%%
\DescribeMacro{\childdocforward}
The command |\childdocforward| redirects processing to
another source file:
%
\begin{center}
\begin{tabular}{l}
|\input{childdoc.def}|\\
|\childdocforward[|\textit{main}|]{|\textit{dest}|}|\\
\end{tabular}
\end{center}
%
The argument \textit{dest} is the destination file
(without extension).
It should be the main file or one of the child files.
Note that further \textsf{childdoc} directives
such as |\childdocof| and |\childdocforward|
in the indicated file will be processed in this form.
The optional argument \textit{main}
passes on directly to the main file \textit{main}
while pretending to compile the child \textit{dest}.
This form behaves as if \textit{dest}
issues |\childdocof{|\textit{main}|}| right away,
and no further \textsf{childdoc} directives will be processed.

%%%%%%%%%%%%%%%%%%%%%%%%%%%%%%%%%%%%%%%%
\DescribeMacro{\...prefix}
In the alternative form |\childdocforwardprefix|,
%
\begin{center}
\begin{tabular}{l}
|\input{childdoc.def}|\\
|\childdocforwardprefix[|\textit{main}|]{|\textit{prefix}|}{|\textit{dest}|}|
\end{tabular}
\end{center}
%
the destination file is determined by a pattern
depending on the current file:
To make this work, the current file must be called
`{\textit{prefix}\hspace{0.2em}\textit{suffix}}'
with \textit{prefix} matching precisely the argument.
Processing is then passed on to the file
`{\textit{dest}\hspace{0.2em}\textit{suffix}}'.
Surely, the same effect is achieved by
directly specifying the
argument `{\textit{dest}\hspace{0.2em}\textit{suffix}}'
in the first form.
However, that requires to set up a different file
for each child. With the alternative form of the command
all these files can have exactly the same content
which simplifies setting them up and maintaining them.

For example, the following file |draft.tex|
with a compilation flag |\version| as described in \secref{sec:flags}
compiles the main document as a draft:
%
\begin{center}
\begin{tabular}{l}
|\def\version{draft}|\\
|\input{childdoc.def}|\\
|\childdocforward{|\textit{main}|}|
\end{tabular}
\end{center}
%
Likewise, the following files |final|\textit{nn}|.tex|
compile the final version of the child document
|child|\textit{nn}|.tex|:
%
\begin{center}
\begin{tabular}{l}
|\def\version{final}|\\
|\input{childdoc.def}|\\
|\childdocforwardprefix{final}{child}|
\end{tabular}
\end{center}
%

Note that when several versions of a main file and/or of each child file
are to be generated, it may be convenient to set up a |Makefile| or
shell script to automatise the process.

%%%%%%%%%%%%%%%%%%%%%%%%%%%%%%%%%%%%%%%%%%%%%%%%%%%%%%%%%%%%%%%%%%%%%%%%%%%%%%%%
\subsection{Command Line Processing}
\label{sec:commandline}

The effect of redirection files can also be achieved by invoking
the \LaTeX{} compiler with a more elaborate command line.
Most conveniently this should be done as part
of a shell script or a |Makefile|.

When using \textsf{childdoc} in the main file, the following
command lines effectively perform a redirection
(note that depending on the shell being used,
backslashes may have to be doubled: `|\|' $\to$ `|\\|'):
%
\begin{center}
|... -jobname "|\textit{target}|" |\\|"|[\textit{flags}]%
|\input{childdoc.def}\childdocforward[|\textit{main}|]{|\textit{dest}|}"|
\end{center}
%
Here \textit{target} is the name of the output file,
\textit{main} is the name of the main file
and \textit{dest} is the name of the main or child file to be processed
(all filenames without extensions).
The optional argument \textit{main} can be omitted
if \textit{main} matches \textit{dest}.
Optionally, compilation \textit{flags} can be defined via |\def| commands.
This command line makes the \TeX{} engine believe
it is compiling the file \textit{target}
whose content is specified as the latter parameter.
The provided code then forwards the processing to
\textit{main} or \textit{dest} as described in \secref{sec:forward}.

%%%%%%%%%%%%%%%%%%%%%%%%%%%%%%%%%%%%%%%%%%%%%%%%%%%%%%%%%%%%%%%%%%%%%%%%%%%%%%%%
\subsection{Include by Input}
\label{sec:input}

Including child documents by |\include| has some restrictions by design.
Most notably, the content of a child document always occupies
its own set of pages; pages cannot be shared between child documents.
Usually, this behaviour makes perfect sense
because each child document contain an essential part of the document.
However, in some situations it may be desirable to compose
a document from a collection of parts
without having mandatory page breaks between then.
For this case, the package
provides a mechanism to include parts
by |\input| which can also be processed individually.
However, by construction this mechanism
requires manual handling of the content to be output.

%%%%%%%%%%%%%%%%%%%%%%%%%%%%%%%%%%%%%%%%
\DescribeMacro{\ifchilddocmanual}
The main file should be prepared as usual, see \secref{sec:include}.
However, the document body must make a distinction
between processing of an individual part and of the main document, e.g.:
%
\begin{center}
\begin{tabular}{l}
|\ifchilddocmanual|\\
|\input{\childdocname}|\\
|\||else|\\
\textit{document body with }|\input{|\textit{part}|}|\\
|\||fi|
\end{tabular}
\end{center}
%
The conditional |\ifchilddocmanual| is true whenever
a part to be included by |\input| is being compiled,
and the name of the part is stored in |\childdocname|.

%%%%%%%%%%%%%%%%%%%%%%%%%%%%%%%%%%%%%%%%
\DescribeMacro{\childdocby}
Each part to be included by |\input| should start with:
%
\begin{center}
\begin{tabular}{l}
|\input{childdoc.def}|\\
|\childdocby{|\textit{main}|}|\\
\end{tabular}
\end{center}
%
The directive |\childdocby| is similar to |\childdocof|
described in \secref{sec:include},
but the subsequent selection of content must be done manually.
To that end, both |\ifchilddoc| and |\ifchilddocmanual|
will be true upon processing of a part,
and the name of the part is stored in |\childdocname|.
Note that |\jobname| will be set to the filename of the current part
so that each part receives an individual |.aux| file
that does not interfere with the |.aux| file(s) of the main document.
This behaviour can be altered by the alternative form
|\childdocby[*]{|\textit{main}|}| (with a non-empty optional argument)
which uses the |.aux| file of the main document
by setting |\jobname| to \textit{main}.

%%%%%%%%%%%%%%%%%%%%%%%%%%%%%%%%%%%%%%%%%%%%%%%%%%%%%%%%%%%%%%%%%%%%%%%%%%%%%%%%
\subsection{Driver Development}
\label{sec:driver}

The \textsf{childdoc} mechanism can also be use for the development
of definition files such as \LaTeX{} styles or classes.
This case differs from the above setup with multiple parts
included by |\include| in that no |\includeonly| should be invoked.
This can be achieved by starting the include file
(before |\ProvidesPackage|) with:
%
\begin{center}
\begin{tabular}{l}
|\input{childdoc.def}|\\
|\childdocforward{|\textit{main}|}|\\
\end{tabular}
\end{center}
%
or alternatively with:
%
\begin{center}
\begin{tabular}{l}
|\input{childdoc.def}|\\
|\childdocby{|\textit{main}|}|\\
\end{tabular}
\end{center}
%
Both forms have slightly different effects as described above.
The main file is prepared as usual, see \secref{sec:include}.

%%%%%%%%%%%%%%%%%%%%%%%%%%%%%%%%%%%%%%%%%%%%%%%%%%%%%%%%%%%%%%%%%%%%%%%%%%%%%%%%
\subsection{Legacy Detection}
\label{sec:detection}

The directive |\childdocmain| in the main file can detect
whether the complete document or merely a child is to be compiled
even without using the directive |\childdocof|.
This method is deprecated because it is less robust
and there is no compelling reason to use it;
it is merely provided for backward compatibility
and it may be removed in future versions.

If the detection mechanism is to be used,
it is mandatory to correctly specify
the filename of the main file as the argument of |\childdocmain|:
%
\begin{center}
\begin{tabular}{l}
|\input{childdoc.def}|\\
|\childdocmain{|\textit{main}|}|\\
\end{tabular}
\end{center}
%
If |\jobname| does not match the argument \textit{main} of |\childdocmain|,
it is assumed that |\jobname| points to the child file to be compiled.
When using |\childdocmain| with the main file specified as argument,
it suffices to start a child file
with just |\input{|\textit{main}|}|
without loading of the package and using |\childdocof|.
If instead all processing is done
with the appropriate \textsf{childdoc} directives,
the argument of \textit{main} of |\childdocmain| can be empty.

An alternative version of the command line processing described
in \secref{sec:commandline} using the detection mechanism reads:
%
\begin{center}
|... -jobname "|\textit{target}|" "|[\textit{flags}]%
[|\def\jobname{|\textit{dest}|}|]|\input{|\textit{main}|}"|
\end{center}

%%%%%%%%%%%%%%%%%%%%%%%%%%%%%%%%%%%%%%%%%%%%%%%%%%%%%%%%%%%%%%%%%%%%%%%%%%%%%%%%
\subsection{Manual Code}
\label{sec:manual}

In case one cannot be certain whether the definitions file |childdoc.def|
is installed on the target \TeX{} distribution
and one prefers not to ship it,
it is conceivable to paste a few relevant commands into the sources.

To that end, drop all statements |\input{childdoc.def}|
and perform the replacements as outlined below.
Instead of |\childdocmain{|\textit{main}|}| add the following code
to the top of the main file:
%
\begin{center}
\begin{tabular}{l}
|\||ifdefined\childdocname\endinput\||fi\newif\ifchilddoc|\\
|\edef\childdocname{\scantokens\expandafter{\jobname\noexpand}}|\\
|\def\childdocmain{|\textit{main}|}\||ifx\childdocmain\childdocname\||else|\\
|\childdoctrue\includeonly{\childdocname}\let\jobname\childdocmain\||fi|\\
\end{tabular}
\end{center}
%
Instead of |\childdocof{|\textit{main}|}| just include the main file
at the top of each child file:
%
\begin{center}
|\input{|\textit{main}|}|
\end{center}
%
A simple redirection |\childdocforward{|\textit{dest}|}| is achieved by:
%
\begin{center}
|\def\jobname{|\textit{dest}|}\input{\jobname}|
\end{center}
%
The redirection with prefix
|\childdocforwardprefix[|\textit{prefix}|]{|\textit{dest}|}|
is accomplished by:
%
\begin{center}
\begin{tabular}{l}
|{\edef\jobname{\scantokens\expandafter{\jobname\noexpand}}|\\
|\def\redirectjob |\textit{prefix}|#1~~~{\gdef\jobname{|\textit{dest}|#1}}|\\
|\expandafter\redirectjob\jobname~~~}\input{\jobname}|
\end{tabular}
\end{center}

In an alternative approach,
child documents can be compiled by a specific command line
without additional code or specific definitions:
%
\begin{center}
|... -jobname "|\textit{target}|" "|[\textit{flags}]%
|\includeonly{|\textit{dest}|}\input{|\textit{main}|}"|
\end{center}
%

%%%%%%%%%%%%%%%%%%%%%%%%%%%%%%%%%%%%%%%%%%%%%%%%%%%%%%%%%%%%%%%%%%%%%%%%%%%%%%%%
%%%%%%%%%%%%%%%%%%%%%%%%%%%%%%%%%%%%%%%%%%%%%%%%%%%%%%%%%%%%%%%%%%%%%%%%%%%%%%%%
\section{Information}

%%%%%%%%%%%%%%%%%%%%%%%%%%%%%%%%%%%%%%%%%%%%%%%%%%%%%%%%%%%%%%%%%%%%%%%%%%%%%%%%
\subsection{Copyright}

Copyright \copyright{} 2017--2018 Niklas Beisert

This work may be distributed and/or modified under the
conditions of the \LaTeX{} Project Public License, either version 1.3
of this license or (at your option) any later version.
The latest version of this license is in
  \url{http://www.latex-project.org/lppl.txt}
and version 1.3 or later is part of all distributions of \LaTeX{}
version 2005/12/01 or later.

This work has the LPPL maintenance status `maintained'.

The Current Maintainer of this work is Niklas Beisert.

This work consists of the files |README.txt|, |childdoc.ins| and |childdoc.dtx|
as well as the derived files |childdoc.def|, |cdocsamp.tex|
with |cdocsch1.tex|, |cdocsch2.tex|, |cdocspt3.tex|, |cdocspt4.tex|,
|cdocsdrf.tex|, |cdocsfn1.tex|, |cdocsfn2.tex|
as well as |childdoc.pdf|.

%%%%%%%%%%%%%%%%%%%%%%%%%%%%%%%%%%%%%%%%%%%%%%%%%%%%%%%%%%%%%%%%%%%%%%%%%%%%%%%%
\subsection{Files and Installation}

The package consists of the files:
%
\begin{center}
\begin{tabular}{ll}
    |README.txt|   & readme file \\
    |childdoc.ins| & installation file \\
    |childdoc.dtx| & source file \\
    |childdoc.def| & definition file \\
    |cdocsamp.tex| & sample main file \\
    |cdocsch1.tex| & sample include file \\
    |cdocsch2.tex| & sample include file \\
    |cdocspt3.tex| & sample part file \\
    |cdocspt4.tex| & sample part file \\
    |cdocsdrf.tex| & sample redirection file \\
    |cdocsfn1.tex| & sample redirection file \\
    |cdocsfn2.tex| & sample redirection file \\
    |childdoc.pdf| & manual
\end{tabular}
\end{center}
%
The distribution consists of the files
|README.txt|, |childdoc.ins| and |childdoc.dtx|.
%
\begin{itemize}
\item
Run (pdf)\LaTeX{} on |childdoc.dtx|
to compile the manual |childdoc.pdf| (this file).
\item
Run \LaTeX{} on |childdoc.ins| to create the definitions file |childdoc.def|
and the sample |cdocsamp.tex| with include files
|cdocsch1.tex|, |cdocsch2.tex|, |cdocspt3.tex|, |cdocspt4.tex|,
|cdocsdrf.tex|, |cdocsfn1.tex|, |cdocsfn2.tex|.
Then copy the file |childdoc.def| to an appropriate directory of your \LaTeX{}
distribution, e.g.\ \textit{texmf-root}|/tex/latex/childdoc|.
\end{itemize}

%%%%%%%%%%%%%%%%%%%%%%%%%%%%%%%%%%%%%%%%%%%%%%%%%%%%%%%%%%%%%%%%%%%%%%%%%%%%%%%%
\subsection{Related CTAN Packages}

There are several other packages which offer a similar functionality:
%
\begin{itemize}
\item
The packages
\href{http://ctan.org/pkg/docmute}{\textsf{docmute}},
\href{http://ctan.org/pkg/includex}{\textsf{includex}} and
\href{http://ctan.org/pkg/standalone}{\textsf{standalone}}
provide commands to include only the document body of
a child file thus allowing both files to be compiled individually.
\item
The packages \href{http://ctan.org/pkg/subdocs}{\textsf{subdocs}}
and \href{http://ctan.org/pkg/subfiles}{\textsf{subfiles}}
provide structures in which the main and child documents can be
encapsulated and allowing them to be compiled individually.
The inclusion mechanism is different from the conventional |\include|.
\item
The package \href{http://ctan.org/pkg/combine}{\textsf{combine}}
is an elaborate solution to combine several documents into one.
\end{itemize}
%
See also the CTAN topic \href{http://ctan.org/topic/subdocs}{\textsf{subdocs}}
for further related packages.
The present package differs from the above solutions in that
a document structure constructed with the conventional |\include| mechanism
just needs two extra commands at the top of every file
such that all constituent files can be compiled individually.

%%%%%%%%%%%%%%%%%%%%%%%%%%%%%%%%%%%%%%%%%%%%%%%%%%%%%%%%%%%%%%%%%%%%%%%%%%%%%%%%
%\subsection{Feature Suggestions}
%
%The following is a list of features which may be useful for future
%versions of this package:
%%
%\begin{itemize}
%\item
%\ldots
%\end{itemize}

%%%%%%%%%%%%%%%%%%%%%%%%%%%%%%%%%%%%%%%%%%%%%%%%%%%%%%%%%%%%%%%%%%%%%%%%%%%%%%%%
\subsection{Revision History}

%%%%%%%%%%%%%%%%%%%%%%%%%%%%%%%%%%%%%%%%
\paragraph{v2.0:} 2018/12/30

\begin{itemize}
\item
immediate forward processing
\item
added |\childdocby| mechanism
\item
manual restructured
\end{itemize}

%%%%%%%%%%%%%%%%%%%%%%%%%%%%%%%%%%%%%%%%
\paragraph{v1.6:} 2018/01/17

\begin{itemize}
\item
application for development of include files
\item
corrections to manual
\end{itemize}

%%%%%%%%%%%%%%%%%%%%%%%%%%%%%%%%%%%%%%%%
\paragraph{v1.5:} 2017/05/21

\begin{itemize}
\item
more complete structuring introduced
\item
|\childdocof| introduced
\item
|\childdoc| renamed to |\childdocmain|
\item
|\childredirect| renamed to |\childdocforward| and |\childdocforwardprefix|
and functionality expanded
\end{itemize}

%%%%%%%%%%%%%%%%%%%%%%%%%%%%%%%%%%%%%%%%
\paragraph{v1.0:} 2017/04/27

\begin{itemize}
\item
manual and install package
\item
first version published on CTAN
\end{itemize}

%%%%%%%%%%%%%%%%%%%%%%%%%%%%%%%%%%%%%%%%
\paragraph{v0.6:} 2017/04/26

\begin{itemize}
\item
redirection mechanism added
\end{itemize}

%%%%%%%%%%%%%%%%%%%%%%%%%%%%%%%%%%%%%%%%
\paragraph{v0.5:} 2017/04/26

\begin{itemize}
\item
functionality in definition file
\end{itemize}


%%%%%%%%%%%%%%%%%%%%%%%%%%%%%%%%%%%%%%%%%%%%%%%%%%%%%%%%%%%%%%%%%%%%%%%%%%%%%%%%
%%%%%%%%%%%%%%%%%%%%%%%%%%%%%%%%%%%%%%%%%%%%%%%%%%%%%%%%%%%%%%%%%%%%%%%%%%%%%%%%
%%%%%%%%%%%%%%%%%%%%%%%%%%%%%%%%%%%%%%%%%%%%%%%%%%%%%%%%%%%%%%%%%%%%%%%%%%%%%%%%
\appendix

\settowidth\MacroIndent{\rmfamily\scriptsize 000\ }

 \DocInput{childdoc.dtx}

\end{document}
%</driver>
% \fi
%
% %%%%%%%%%%%%%%%%%%%%%%%%%%%%%%%%%%%%%%%%%%%%%%%%%%%%%%%%%%%%%%%%%%%%%%%%%%%%%%
% %%%%%%%%%%%%%%%%%%%%%%%%%%%%%%%%%%%%%%%%%%%%%%%%%%%%%%%%%%%%%%%%%%%%%%%%%%%%%%
% \section{Sample}
%\iffalse
%<*samplemain>
%\fi
%
% The following presents a sample document
% with two chapters, two parts, a title page,
% a compile flag as well as three forwarding files to set the flag.
% It consists of eight |.tex| files:
% \begin{center}
% \begin{tabular}{ll}
% |cdocsamp.tex|&main file\\
% |cdocsch1.tex|&include file for chapter 1\\
% |cdocsch2.tex|&include file for chapter 2\\
% |cdocspt3.tex|&include file for part 3\\
% |cdocspt4.tex|&include file for part 4\\
% |cdocsdrf.tex|&forwarding file for main file in draft mode\\
% |cdocsfi1.tex|&forwarding file for final version of chapter 1\\
% |cdocsfi2.tex|&forwarding file for final version of chapter 2\\
% \end{tabular}
% \end{center}
% Each of the eight files can be compiled directly by the \LaTeX{} compiler.
%
% %%%%%%%%%%%%%%%%%%%%%%%%%%%%%%%%%%%%%%
% \paragraph{Main File.}
%
% The main file is called |cdocsamp.tex|.
%
% Load the \textsf{childdoc} definitions and
% declare the filename for the main document:
%    \begin{macrocode}
\input{childdoc.def}
\childdocmain{}
%    \end{macrocode}

% Optional override for |\version| flag:
%    \begin{macrocode}
%%\ifchilddoc\else\providecommand{\version}{draft}\fi
%    \end{macrocode}

% Define the default values for the |\version| flag
% (|final| for the main file and |draft| for childs):
%    \begin{macrocode}
\ifchilddoc
\providecommand{\version}{draft}
\else
\providecommand{\version}{final}
\fi
%    \end{macrocode}

% Load the standard document class:
%    \begin{macrocode}
\documentclass[12pt]{article}
%    \end{macrocode}

% Start the document body:
%    \begin{macrocode}
\begin{document}
%    \end{macrocode}

% Declare a title page.
% Print title, part of document being processed and version flag:
%    \begin{macrocode}
\addtocounter{page}{-1}
\begin{center}
{\LARGE\bfseries{}childdoc example\par}
\vspace{1cm}
\ifchilddoc
\ifchilddocmanual part\else chapter\fi:
`\childdocname' of `\childdocjob'\par
\else
main document: `\childdocjob'\par
\fi
version: \version\par
\end{center}
\newpage
%    \end{macrocode}

% Manually include selected file,
% otherwise process as usual:
%    \begin{macrocode}
\ifchilddocmanual
\section*{part `\childdocname'}
\input{\childdocname}
\else
%    \end{macrocode}

% Include the two chapters:
%    \begin{macrocode}
\include{cdocsch1}
\include{cdocsch2}
%    \end{macrocode}

% Include the two parts unless only chapters should be displayed:
%    \begin{macrocode}
\ifchilddoc\else
\section{part three}
\input{cdocspt3}
\section{part four}
\input{cdocspt4}
\fi
%    \end{macrocode}

% Process as usual until here:
%    \begin{macrocode}
\fi
%    \end{macrocode}

% End of document body:
%    \begin{macrocode}
\end{document}
%    \end{macrocode}
%\iffalse
%</samplemain>
%\fi
%
% %%%%%%%%%%%%%%%%%%%%%%%%%%%%%%%%%%%%%%
% \paragraph{Chapter Include Files.}
%
% The include files are called |cdocsch1.tex| and |cdocsch2.tex|.
%
%\iffalse
%<*samplechap1|samplechap2>
%\fi

% Optional override for |\version| flag:
%    \begin{macrocode}
%%\providecommand{\version}{final}
%    \end{macrocode}

% Include the main document:
%    \begin{macrocode}
\input{childdoc.def}
\childdocof{cdocsamp}
%    \end{macrocode}

%\iffalse
%</samplechap1|samplechap2>
%\fi
%
%\iffalse
%<*samplechap1>
%\fi
% Some text for chapter 1:
%    \begin{macrocode}
\section{one}
some text in chapter one
%    \end{macrocode}

%\iffalse
%</samplechap1>
%\fi
% Some text for chapter 2:
%\iffalse
%<*samplechap2>
%\fi
%    \begin{macrocode}
\section{two}
more text in chapter two
%    \end{macrocode}

%\iffalse
%</samplechap2>
%\fi
%
% %%%%%%%%%%%%%%%%%%%%%%%%%%%%%%%%%%%%%%
% \paragraph{Part Include Files.}
%
% The include files are called |cdocspt3.tex| and |cdocspt4.tex|.
%
%\iffalse
%<*samplepart3|samplepart4>
%\fi

% Optional override for |\version| flag:
%    \begin{macrocode}
%%\providecommand{\version}{final}
%    \end{macrocode}

% Include the main document:
%    \begin{macrocode}
\input{childdoc.def}
\childdocby{cdocsamp}
%    \end{macrocode}

%\iffalse
%</samplepart3|samplepart4>
%\fi
%
%\iffalse
%<*samplepart3>
%\fi
% Some text for part 3:
%    \begin{macrocode}
some text in part three
%    \end{macrocode}

%\iffalse
%</samplepart3>
%\fi
% Some text for part 4:
%\iffalse
%<*samplepart4>
%\fi
%    \begin{macrocode}
more text in part four
%    \end{macrocode}

%\iffalse
%</samplepart4>
%\fi
%
% %%%%%%%%%%%%%%%%%%%%%%%%%%%%%%%%%%%%%%
% \paragraph{Forwarding for a Complete Draft.}
%
% The following forwarding file |cdocsdrf.tex|
% compiles the main document in draft mode:
%\iffalse
%<*sampledraft>
%\fi
%    \begin{macrocode}
\def\version{draft}
\input{childdoc.def}
\childdocforward{cdocsamp}
%    \end{macrocode}

%\iffalse
%</sampledraft>
%\fi
%
% %%%%%%%%%%%%%%%%%%%%%%%%%%%%%%%%%%%%%%
% \paragraph{Forwarding for Final Version of the Chapters.}
%
% The following forwarding files |cdocsfn1.tex| and |cdocsfn2.tex|
% (with identical content)
% compile the final versions of the child documents
% |cdocsch1.tex| and |cdocsch2.tex|, respectively:
%\iffalse
%<*samplefinal>
%\fi
%    \begin{macrocode}
\def\version{final}
\input{childdoc.def}
\childdocforwardprefix[cdocsamp]{cdocsfn}{cdocsch}
%    \end{macrocode}

%\iffalse
%</samplefinal>
%\fi
%
% %%%%%%%%%%%%%%%%%%%%%%%%%%%%%%%%%%%%%%
% \paragraph{Command Line Processing.}
%
% The following three command lines generate the output files
% |cdocscld|, |cdocscl1| and |cdocscl2|
% which should be identical to
% |cdocsdrf|, |cdocsch1| and |cdocsfn2|, respectively:
% \begin{center}
% \begin{tabular}{l}
% |latex -jobname cdocscld \|\\
% |  "\def\version{draft}\input{childdoc.def}\childdocforward{cdocsamp}"|\\
% |latex -jobname cdocscl1 \|\\
% |  "\input{childdoc.def}\childdocforward[cdocsamp]{cdocsch1}"|\\
% |latex -jobname cdocscl2 \|\\
% |  "\def\version{final}\input{childdoc.def}\childdocforward{cdocsch2}"|
% \end{tabular}
% \end{center}
% Note that the trailing backslash on each first line
% merely continues the input to the second line
% (for convenient cut ant paste).
% Furthermore, the command |latex| can be replaced by any
% of its alternative versions such as |pdflatex|.
%
% %%%%%%%%%%%%%%%%%%%%%%%%%%%%%%%%%%%%%%%%%%%%%%%%%%%%%%%%%%%%%%%%%%%%%%%%%%%%%%
% %%%%%%%%%%%%%%%%%%%%%%%%%%%%%%%%%%%%%%%%%%%%%%%%%%%%%%%%%%%%%%%%%%%%%%%%%%%%%%
% \section{Implementation}
%\iffalse
%<*package>
%\fi
%
% This section describes the definitions file |childdoc.def|.

% The definitions cannot be loaded using |\usepackage| or |\RequirePackage|
% which has a mechanism to prevent loading a style file more than once.
% When loading the definitions by means of |\input|
% multiple instances have to be prevented manually:
%\iffalse
%This code needs to be before the `\ProvidesFile' directive
%which is defined at the beginning of this file.
%Therefore it is also placed there and commented out here.
%</package>
%<*discard>
%\fi
%    \begin{macrocode}
\ifdefined\childdocmain\endinput\fi
%    \end{macrocode}
%\iffalse
%</discard>
%<*package>
%\fi
%
% \macro{\ifchilddoc}
% \macro{\ifchilddocmanual}
% The conditional |\ifchilddoc| tells whether a
% child (true) or main (false) document is being compiled.
% The conditional |\ifchilddocmanual| tells whether
% the |\includeonly| mechanism is used (false) or
% the selection of child files must be performed manually (true).
% The definitions initialise to false:
%    \begin{macrocode}
\newif\ifchilddoc
\newif\ifchilddocmanual
%    \end{macrocode}

% \macro{\childdocname}
% \macro{\childdocjob}
% The macro |\childdocname| stores the name of the main document
% to be compiled. The macro |\childdocjob| stores the name of
% the document on which the \LaTeX{} compiler was originally invoked.
% The content of |\jobname| cannot be compared
% to filenames specified in the source due to different catcodes.
% The following code rescans |\jobname|, stores the result
% in |\childdocname| and saves a copy in |\childdocjob|:
%    \begin{macrocode}
\edef\childdocname{\scantokens\expandafter{\jobname\noexpand}}
\let\childdocjob\childdocname
%    \end{macrocode}

% \macro{\childdocdisable}
% The macro |\childdocdisable| prevents the main file
% from being processed more than once.
% At this stage, the main document command |\childdocmain|
% is assumed to be called once again where it should do nothing.
% Any subsequent call to it should prevent
% a secondary processing of the main document
% It overwrites the forwarding commands
% |\childdocof| and |\childdocforward|
% with empty macros to prevent further inclusions of the main document:
%    \begin{macrocode}
\newcommand{\childdocdisable}
{
  \renewcommand{\childdocmain}[1]{\renewcommand{\childdocmain}[1]{\endinput}}
  \renewcommand{\childdocof}[1]{}
  \renewcommand{\childdocby}[2][]{}
  \renewcommand{\childdocforward}[2][]{}
  \renewcommand{\childdocdisable}{}
}
%    \end{macrocode}

% \macro{\childdocmain}
% The macro |\childdocmain| is to be called at the top of the main file
% with nothing or the main filename (without extension) as argument.
% First, it breaks loops.
% If the argument is not empty and does not match |\childdocname|
% (which is set by the first inclusion of |childdoc.def|),
% |\ifchilddoc| is set to true, |\includeonly| is applied to the child file
% and |\jobname| is set to the main file
% (for proper handling of |.aux| files):
%    \begin{macrocode}
\newcommand{\childdocmain}[1]
{
  \childdocdisable\childdocmain{}
  \if?#1?\else
    \begingroup
      \def\childdoctmp{#1}
      \ifx\childdoctmp\childdocname
        \def\childdoctmp{}
      \else
        \def\childdoctmp
        {
          \childdoctrue
          \includeonly{\childdocname}
          \def\childdocjob{#1}
          \def\jobname{#1}
        }
      \fi
      \expandafter
    \endgroup
    \childdoctmp
  \fi
}
%    \end{macrocode}

% \macro{\childdocof}
% The command |\childdocof| redirects
% compilation to the main file |#1|.
%    \begin{macrocode}
\newcommand{\childdocof}[1]
{
  \childdocdisable
  \childdoctrue
  \includeonly{\childdocname}
  \def\jobname{#1}
  \def\childdocjob{#1}
  \input{#1}
}
%    \end{macrocode}

% \macro{\childdocby}
% The command |\childdocby| ....
%    \begin{macrocode}
\newcommand{\childdocby}[2][]
{
  \childdocdisable
  \childdoctrue
  \childdocmanualtrue
  \if?#1?\else
    \def\jobname{#2}
  \fi
  \def\childdocjob{#2}
  \input{#2}
  \endinput
}
%    \end{macrocode}

% \macro{\childdocforward}
% The command |\childdocforward| redirects
% compilation to the main file or
% (if the optional argument is given) a child file.
% Parameters are set as if the main file
% or a child file starting with |\childdocof| was compiled.
% Then compilation is handed over to the main file:
%    \begin{macrocode}
\newcommand{\childdocforward}[2][]
{
  \begingroup
    \if?#1?
      \def\childdoctmp
      {
        \def\childdocname{#2}
        \def\childdocjob{#2}
        \def\jobname{#2}
        \input{#2}
        \endinput
      }
    \else
      \def\childdoctmp
      {
        \childdocdisable
        \def\childdocname{#2}
        \childdoctrue
        \includeonly{#2}
        \def\childdocjob{#1}
        \def\jobname{#1}
        \input{#1}
        \endinput
      }
    \fi
    \expandafter
  \endgroup
  \childdoctmp
}
%    \end{macrocode}

% \macro{\childdocforwardprefix}
% The command |\childdocforwardprefix| redirects
% compilation to the main or a child file by means of a pattern.
% The prefix |#1| in the current filename is replaced by |#2|
% and the suffix of the current filename is kept
% (it is assumed that the filename does not contain the substring `|~~~|'
% which is used as a delimiter).
% Compilation is handed over to the new file by |\childdocforward|:
%    \begin{macrocode}
\newcommand{\childdocforwardprefix}[3][]
{
  \begingroup
    \def\childdocextract #2##1~~~{\def\childdoctmp{\childdocforward[#1]{#3##1}}}
    \expandafter\childdocextract\childdocname~~~
    \expandafter
  \endgroup
  \childdoctmp
}
%    \end{macrocode}

% \macro{\childdoc}
% The deprecated macro |\childdoc| is a legacy version of |\childdocmain|:
%    \begin{macrocode}
\newcommand{\childdoc}{\childdocmain}
%    \end{macrocode}

% \macro{\childdocredirect}
% The deprecated macro |\childdocredirect| is a legacy version
% of |\childdocforward| and |\childdocforwardprefix|:
%    \begin{macrocode}
\newcommand{\childdocredirect}[2][]
{
  \begingroup
    \if?#1?
      \def\childdoctmp{\childdocforward{#2}}
    \else
      \def\childdoctmp{\childdocforwardprefix{#1}{#2}}
    \fi
    \expandafter
  \endgroup
  \childdoctmp
}
%    \end{macrocode}

%\iffalse
%</package>
%\fi
%
\endinput
|\\
|\childdocforwardprefix{final}{child}|
\end{tabular}
\end{center}
%

Note that when several versions of a main file and/or of each child file
are to be generated, it may be convenient to set up a |Makefile| or
shell script to automatise the process.

%%%%%%%%%%%%%%%%%%%%%%%%%%%%%%%%%%%%%%%%%%%%%%%%%%%%%%%%%%%%%%%%%%%%%%%%%%%%%%%%
\subsection{Command Line Processing}
\label{sec:commandline}

The effect of redirection files can also be achieved by invoking
the \LaTeX{} compiler with a more elaborate command line.
Most conveniently this should be done as part
of a shell script or a |Makefile|.

When using \textsf{childdoc} in the main file, the following
command lines effectively perform a redirection
(note that depending on the shell being used,
backslashes may have to be doubled: `|\|' $\to$ `|\\|'):
%
\begin{center}
|... -jobname "|\textit{target}|" |\\|"|[\textit{flags}]%
|% \iffalse
%
% childdoc.dtx Copyright (C) 2017-2018 Niklas Beisert
%
% This work may be distributed and/or modified under the
% conditions of the LaTeX Project Public License, either version 1.3
% of this license or (at your option) any later version.
% The latest version of this license is in
%   http://www.latex-project.org/lppl.txt
% and version 1.3 or later is part of all distributions of LaTeX
% version 2005/12/01 or later.
%
% This work has the LPPL maintenance status `maintained'.
%
% The Current Maintainer of this work is Niklas Beisert.
%
% This work consists of the files childdoc.dtx and childdoc.ins
% and the derived files childdoc.def and cdocsamp.tex with
% cdocsch1.tex, cdocsch2.tex, cdocsdrf.tex, cdocsfn1.tex, cdocsfn2.tex.
%
%<package>\ifdefined\childdocmain\endinput\fi
%<package>\ProvidesFile{childdoc.def}[2018/12/30 v2.0 child document driver]
%<samplemain>\ProvidesFile{cdocsamp.tex}[2018/12/30 v2.0 sample for childdoc]
%<*driver>
%\ProvidesFile{childdoc.drv}[2018/12/30 v2.0 childdoc reference manual file]
\PassOptionsToClass{10pt,a4paper}{article}
\documentclass{ltxdoc}

\usepackage[margin=35mm]{geometry}
\usepackage{hyperref}
\usepackage{hyperxmp}
\usepackage[usenames]{color}

\hypersetup{colorlinks=true}
\hypersetup{pdfstartview=FitH}
\hypersetup{pdfpagemode=UseNone}
\hypersetup{pdfsource={}}
\hypersetup{pdflang={en-UK}}
\hypersetup{pdfcopyright={Copyright 2017-2018 Niklas Beisert.
  This work may be distributed and/or modified under the
  conditions of the LaTeX Project Public License, either version 1.3
  of this license or (at your option) any later version.}}
\hypersetup{pdflicenseurl={http://www.latex-project.org/lppl.txt}}
\hypersetup{pdfcontactaddress={ETH Zurich, ITP, HIT K,
  Wolfgang-Pauli-Strasse 27}}
\hypersetup{pdfcontactpostcode={8093}}
\hypersetup{pdfcontactcity={Zurich}}
\hypersetup{pdfcontactcountry={Switzerland}}
\hypersetup{pdfcontactemail={nbeisert@itp.phys.ethz.ch}}
\hypersetup{pdfcontacturl={http://people.phys.ethz.ch/\xmptilde nbeisert/}}

\newcommand{\secref}[1]{\hyperref[#1]{section \ref*{#1}}}

\parskip1ex
\parindent0pt
\let\olditemize\itemize
\def\itemize{\olditemize\parskip0pt}

\begin{document}

\title{The \textsf{childdoc} Package}
\hypersetup{pdftitle={The childdoc Package}}
\author{Niklas Beisert\\[2ex]
  Institut f\"ur Theoretische Physik\\
  Eidgen\"ossische Technische Hochschule Z\"urich\\
  Wolfgang-Pauli-Strasse 27, 8093 Z\"urich, Switzerland\\[1ex]
  \href{mailto:nbeisert@itp.phys.ethz.ch}
  {\texttt{nbeisert@itp.phys.ethz.ch}}}
\hypersetup{pdfauthor={Niklas Beisert}}
\hypersetup{pdfsubject={Manual for the LaTeX2e Package childdoc}}
\date{30 December 2018, \textsf{v2.0}}
\maketitle

\begin{abstract}\noindent
\textsf{childdoc} is a \LaTeXe{} package
that enables the direct compilation
of document sections included by |\include|
to individual files.
\end{abstract}

\begingroup
\parskip0ex
\tableofcontents
\endgroup

%%%%%%%%%%%%%%%%%%%%%%%%%%%%%%%%%%%%%%%%%%%%%%%%%%%%%%%%%%%%%%%%%%%%%%%%%%%%%%%%
%%%%%%%%%%%%%%%%%%%%%%%%%%%%%%%%%%%%%%%%%%%%%%%%%%%%%%%%%%%%%%%%%%%%%%%%%%%%%%%%
\section{Introduction}

\LaTeX{} provides a mechanism to structure a large document (such as a book)
into a main file and several child files (containing the chapters)
using the |\include| command.
This mechanism is beneficial for documents
which span hundreds of pages in order to
make the source file(s) more manageable.
Moreover, compilation can be restricted to
selected child files by means of the |\includeonly| command.
The latter feature can be used to reduce the compilation time while editing
(this was significantly more useful in the earlier days of \LaTeX{})
or to generate a smaller document which is easier to navigate.
Another application of |\includeonly| is to generate
documents consisting of selected parts of the complete document.

However, there are a few drawbacks of the plain |\include| mechanism:
\begin{itemize}
\item
The child files cannot be compiled on their own,
they can only be compiled via the main file.
A naive editing environment
(such as a text editor with an option
to have the current file processed by \LaTeX)
may require one to switch to the main file before compiling;
attempting to compile the child file produces errors.
\item
The main file must be modified (each time)
to adjust the |\includeonly| command
to the present needs. This easily leaves the main file in a messy state.
\item
The generated document will always carry the filename
of the main document. This is inconvenient if
several child files are to be compiled and
to be kept for distribution.
\end{itemize}

The present package provides a simple interface
to make child files individually compilable by \LaTeX{}.
Compiling a child file then has the same effect as compiling
the main file with an |\includeonly| command
to select the appropriate child.
Moreover the generated document will carry the name of the child
rather than the main file.
This resolves all three above issues.

This feature is meant to make the editing of books,
thesis documents and lecture notes somewhat more convenient.
However, the package can also be used efficiently for
composing a series of documents (such as exercise sheets)
which are typically distributed individually.
It then assists the author in generating the individual documents
(potentially in different versions)
as well as a document containing the collected series.
Another application is in developing style files
or other kinds of included material
where compilation of the style file could redirect
to a sample or test file.

%%%%%%%%%%%%%%%%%%%%%%%%%%%%%%%%%%%%%%%%%%%%%%%%%%%%%%%%%%%%%%%%%%%%%%%%%%%%%%%%
%%%%%%%%%%%%%%%%%%%%%%%%%%%%%%%%%%%%%%%%%%%%%%%%%%%%%%%%%%%%%%%%%%%%%%%%%%%%%%%%
\section{Usage}

First of all, the package \textsf{childdoc} is \emph{not} a standard
\LaTeXe{} |.sty| style file! Therefore it needs to be invoked in
a non-standard way.

%%%%%%%%%%%%%%%%%%%%%%%%%%%%%%%%%%%%%%%%%%%%%%%%%%%%%%%%%%%%%%%%%%%%%%%%%%%%%%%%
\subsection{Included Files}
\label{sec:include}

%%%%%%%%%%%%%%%%%%%%%%%%%%%%%%%%%%%%%%%%
\DescribeMacro{\childdocmain}
To use the package, add the commands
\begin{center}
\begin{tabular}{l}
|\input{childdoc.def}|\\
|\childdocmain{}|\\
\end{tabular}
\end{center}
at the very top of the main \LaTeX{} file,
in particular \emph{before} the |\documentclass| statement!
The argument of |\childdocmain| should be left empty
(but it must be present).

%%%%%%%%%%%%%%%%%%%%%%%%%%%%%%%%%%%%%%%%
\DescribeMacro{\childdocof}
Furthermore, add the commands
\begin{center}
\begin{tabular}{l}
|\input{childdoc.def}|\\
|\childdocof{|\textit{main}|}|\\
\end{tabular}
\end{center}
at the top of every child file \textit{child}
which is included by |\include{|\textit{child}|}|
from within the main file
(or at least for those files to be compiled individually).
The argument \textit{main} must be the filename of the main file.

There are a couple of
considerations in setting up the main and child documents:

%%%%%%%%%%%%%%%%%%%%%%%%%%%%%%%%%%%%%%%%
\paragraph{Restrictions.}

Please note the following restrictions:
\begin{itemize}
\item
|\childdocmain| must be called with one argument \textit{main}
to ensure compatibility with earlier version of the package.
It must either be empty (|\childdocmain{}|)
or precisely match the filename of the main file in which it is specified.
See \secref{sec:detection} for further information.
\item
The filename \textit{main} must be specified without the |.tex| extension.
\item
The filename \textit{main} is case sensitive
(even in case-insensitive file systems)
due to internal string comparison.
\item
The argument \textit{main} should be fully expanded, it cannot be a macro.
\item
Subdirectories and special characters should be avoided in filenames.
\item
The command |\childdocmain{|\textit{main}|}| must be followed by a whitespace.
It should not be followed immediately by another command
or by a comment mark `|%|'.
This is because the \TeX{} parser reads the token immediately following
the argument of |\childdocmain| and puts it
at the beginning of every child section;
however, a white\-space is ignored.
\end{itemize}

%%%%%%%%%%%%%%%%%%%%%%%%%%%%%%%%%%%%%%%%
\paragraph{Content of Main File.}

It is advisable to place all content in the child files included by |\include|.
Any output contained in the main file will appear in all child documents
unless suppressed manually;
it cannot be suppressed automatically by the |\includeonly| directive
and thus should normally be avoided.
A method to include some content in the main file
by means of conditional processing is described in \secref{sec:conditional}.

%%%%%%%%%%%%%%%%%%%%%%%%%%%%%%%%%%%%%%%%
\paragraph{Page Numbering.}

When only a part of the document is compiled,
the appropriate numbering of pages
(as well as other status parameters)
is determined from the |.aux| files.
The latter contain information from previous passes.
However this information needs to propagate through
all intermediate child documents.
Therefore the page numbering in child documents may well
be inconsistent until the complete document is compiled at least once.

A useful (if unconventional) way to always ensure a consistent
page numbering is to restart the numbering in each child document
and denote the pages by `\textit{child}|.|\textit{page}'
where \textit{child} represents the chapter/section number of the child file.
This can be achieved by the command
|\numberwithin{page}{|\textit{child}|}|
of the \textsf{amsmath} package
where \textit{child} can be |chapter| or |section|
depending on the chosen structuring.
Alternatively, one can modify the macro |\thepage| appropriately
and reset the counter |page| at the start of each child file.

%%%%%%%%%%%%%%%%%%%%%%%%%%%%%%%%%%%%%%%%%%%%%%%%%%%%%%%%%%%%%%%%%%%%%%%%%%%%%%%%
\subsection{Conditional Processing}
\label{sec:conditional}

The package provides a mechanism to compile different versions
of a document. To customise the versions further some conditional processing
can come in handy to distinguish which version is being compiled.
The package provides two macros to describe the compilation context:

%%%%%%%%%%%%%%%%%%%%%%%%%%%%%%%%%%%%%%%%
\DescribeMacro{\ifchilddoc}
The conditional |\ifchilddoc| distinguishes between the compilation of
child documents and the main document:
%
\begin{center}
|\ifchilddoc |\textit{child-code}| |[|\||else |\textit{main-code}]| \||fi|
\end{center}

%%%%%%%%%%%%%%%%%%%%%%%%%%%%%%%%%%%%%%%%
\DescribeMacro{\childdocname}
\DescribeMacro{\childdocjob}
The macro |\childdocname| contains the filename (without extension)
of the main or child file being processed.
Note that |\childdocjob| will always contain the name of the main file.

%%%%%%%%%%%%%%%%%%%%%%%%%%%%%%%%%%%%%%%%
\paragraph{Title Page.}

Conditional processing can be used to include a title or banner page
in the main document when proper precautions are taken.
Importantly, the code in the main file should ensure that the page counter
(as well as other status parameters which are stored in the |.aux| files)
takes the same value after the conditional processing.
Otherwise the page numbers may take divergent values
depending on which part is compiled.

For example, a title page could be declared by:
%
\begin{center}
\begin{tabular}{l}
|\ifchilddoc\||else|\\
|\addtocounter{page}{-1}|\\
\textit{code for title page}\\
|\newpage|\\
|\||fi|
\end{tabular}
\end{center}
%
A banner page for the child documents can be generated by:
%
\begin{center}
\begin{tabular}{l}
|\ifchilddoc|\\
|\addtocounter{page}{-1}|\\
\textit{code for banner page}\\
|\newpage|\\
|\||fi|
\end{tabular}
\end{center}
%
Here one could write a message such as:
\begin{center}
|This is the part \childdocname{} of \childdocjob{}.|
\end{center}

%%%%%%%%%%%%%%%%%%%%%%%%%%%%%%%%%%%%%%%%%%%%%%%%%%%%%%%%%%%%%%%%%%%%%%%%%%%%%%%%
\subsection{Flags}
\label{sec:flags}

The package makes it easy to generate different versions
of the main or child documents.
To this end compilation flags can be defined
and assigned different default values.
They will be particularly useful in conjunction
with the forwarding mechanism described in \secref{sec:forward}.

For example, it may be useful to have a flag |\version|
which can be set to |draft| or |final|.
The document source will contain some conditional code
depending on the value of |\version|.
Suppose further, the flag should default to |final| for the main file
and to |draft| for child files
which is a natural assignment for editing the document.
This is achieved by placing the following code
in the preamble of the main document
(below the |\childdocmain| directive):
%
\begin{center}
\begin{tabular}{l}
|\ifchilddoc|\\
|\providecommand{\version}{draft}|\\
|\||else|\\
|\providecommand{\version}{final}|\\
|\||fi|
\end{tabular}
\end{center}
%
The definition by |\providecommand| makes sure
that previous definitions are not overwritten.
Further statements |\providecommand{\version}{...}|
can thus be added before the above code to override it.

For the main file, one might add a line
(between |\childdocmain| and the above block)
%
\begin{center}
|%\ifchilddoc\||else\providecommand{\version}{draft}\||fi|
\end{center}
%
which can be uncommented to produce a draft version.
Likewise one can add a line to the very top of a child file
(above the |\childdocof{|\textit{main}|}| directive)
%
\begin{center}
|%\providecommand{\version}{final}|
\end{center}
%
which can be uncommented to produce the final version of this child document.

%%%%%%%%%%%%%%%%%%%%%%%%%%%%%%%%%%%%%%%%%%%%%%%%%%%%%%%%%%%%%%%%%%%%%%%%%%%%%%%%
\subsection{Forwarding}
\label{sec:forward}

Different versions of the main or child documents
using compilation flags as described in \secref{sec:flags}
can be (permanently) stored in different files
for convenient compilation, viewing and distribution.
To this end, the package defines a command
to pass on compilation to a different file:

%%%%%%%%%%%%%%%%%%%%%%%%%%%%%%%%%%%%%%%%
\DescribeMacro{\childdocforward}
The command |\childdocforward| redirects processing to
another source file:
%
\begin{center}
\begin{tabular}{l}
|\input{childdoc.def}|\\
|\childdocforward[|\textit{main}|]{|\textit{dest}|}|\\
\end{tabular}
\end{center}
%
The argument \textit{dest} is the destination file
(without extension).
It should be the main file or one of the child files.
Note that further \textsf{childdoc} directives
such as |\childdocof| and |\childdocforward|
in the indicated file will be processed in this form.
The optional argument \textit{main}
passes on directly to the main file \textit{main}
while pretending to compile the child \textit{dest}.
This form behaves as if \textit{dest}
issues |\childdocof{|\textit{main}|}| right away,
and no further \textsf{childdoc} directives will be processed.

%%%%%%%%%%%%%%%%%%%%%%%%%%%%%%%%%%%%%%%%
\DescribeMacro{\...prefix}
In the alternative form |\childdocforwardprefix|,
%
\begin{center}
\begin{tabular}{l}
|\input{childdoc.def}|\\
|\childdocforwardprefix[|\textit{main}|]{|\textit{prefix}|}{|\textit{dest}|}|
\end{tabular}
\end{center}
%
the destination file is determined by a pattern
depending on the current file:
To make this work, the current file must be called
`{\textit{prefix}\hspace{0.2em}\textit{suffix}}'
with \textit{prefix} matching precisely the argument.
Processing is then passed on to the file
`{\textit{dest}\hspace{0.2em}\textit{suffix}}'.
Surely, the same effect is achieved by
directly specifying the
argument `{\textit{dest}\hspace{0.2em}\textit{suffix}}'
in the first form.
However, that requires to set up a different file
for each child. With the alternative form of the command
all these files can have exactly the same content
which simplifies setting them up and maintaining them.

For example, the following file |draft.tex|
with a compilation flag |\version| as described in \secref{sec:flags}
compiles the main document as a draft:
%
\begin{center}
\begin{tabular}{l}
|\def\version{draft}|\\
|\input{childdoc.def}|\\
|\childdocforward{|\textit{main}|}|
\end{tabular}
\end{center}
%
Likewise, the following files |final|\textit{nn}|.tex|
compile the final version of the child document
|child|\textit{nn}|.tex|:
%
\begin{center}
\begin{tabular}{l}
|\def\version{final}|\\
|\input{childdoc.def}|\\
|\childdocforwardprefix{final}{child}|
\end{tabular}
\end{center}
%

Note that when several versions of a main file and/or of each child file
are to be generated, it may be convenient to set up a |Makefile| or
shell script to automatise the process.

%%%%%%%%%%%%%%%%%%%%%%%%%%%%%%%%%%%%%%%%%%%%%%%%%%%%%%%%%%%%%%%%%%%%%%%%%%%%%%%%
\subsection{Command Line Processing}
\label{sec:commandline}

The effect of redirection files can also be achieved by invoking
the \LaTeX{} compiler with a more elaborate command line.
Most conveniently this should be done as part
of a shell script or a |Makefile|.

When using \textsf{childdoc} in the main file, the following
command lines effectively perform a redirection
(note that depending on the shell being used,
backslashes may have to be doubled: `|\|' $\to$ `|\\|'):
%
\begin{center}
|... -jobname "|\textit{target}|" |\\|"|[\textit{flags}]%
|\input{childdoc.def}\childdocforward[|\textit{main}|]{|\textit{dest}|}"|
\end{center}
%
Here \textit{target} is the name of the output file,
\textit{main} is the name of the main file
and \textit{dest} is the name of the main or child file to be processed
(all filenames without extensions).
The optional argument \textit{main} can be omitted
if \textit{main} matches \textit{dest}.
Optionally, compilation \textit{flags} can be defined via |\def| commands.
This command line makes the \TeX{} engine believe
it is compiling the file \textit{target}
whose content is specified as the latter parameter.
The provided code then forwards the processing to
\textit{main} or \textit{dest} as described in \secref{sec:forward}.

%%%%%%%%%%%%%%%%%%%%%%%%%%%%%%%%%%%%%%%%%%%%%%%%%%%%%%%%%%%%%%%%%%%%%%%%%%%%%%%%
\subsection{Include by Input}
\label{sec:input}

Including child documents by |\include| has some restrictions by design.
Most notably, the content of a child document always occupies
its own set of pages; pages cannot be shared between child documents.
Usually, this behaviour makes perfect sense
because each child document contain an essential part of the document.
However, in some situations it may be desirable to compose
a document from a collection of parts
without having mandatory page breaks between then.
For this case, the package
provides a mechanism to include parts
by |\input| which can also be processed individually.
However, by construction this mechanism
requires manual handling of the content to be output.

%%%%%%%%%%%%%%%%%%%%%%%%%%%%%%%%%%%%%%%%
\DescribeMacro{\ifchilddocmanual}
The main file should be prepared as usual, see \secref{sec:include}.
However, the document body must make a distinction
between processing of an individual part and of the main document, e.g.:
%
\begin{center}
\begin{tabular}{l}
|\ifchilddocmanual|\\
|\input{\childdocname}|\\
|\||else|\\
\textit{document body with }|\input{|\textit{part}|}|\\
|\||fi|
\end{tabular}
\end{center}
%
The conditional |\ifchilddocmanual| is true whenever
a part to be included by |\input| is being compiled,
and the name of the part is stored in |\childdocname|.

%%%%%%%%%%%%%%%%%%%%%%%%%%%%%%%%%%%%%%%%
\DescribeMacro{\childdocby}
Each part to be included by |\input| should start with:
%
\begin{center}
\begin{tabular}{l}
|\input{childdoc.def}|\\
|\childdocby{|\textit{main}|}|\\
\end{tabular}
\end{center}
%
The directive |\childdocby| is similar to |\childdocof|
described in \secref{sec:include},
but the subsequent selection of content must be done manually.
To that end, both |\ifchilddoc| and |\ifchilddocmanual|
will be true upon processing of a part,
and the name of the part is stored in |\childdocname|.
Note that |\jobname| will be set to the filename of the current part
so that each part receives an individual |.aux| file
that does not interfere with the |.aux| file(s) of the main document.
This behaviour can be altered by the alternative form
|\childdocby[*]{|\textit{main}|}| (with a non-empty optional argument)
which uses the |.aux| file of the main document
by setting |\jobname| to \textit{main}.

%%%%%%%%%%%%%%%%%%%%%%%%%%%%%%%%%%%%%%%%%%%%%%%%%%%%%%%%%%%%%%%%%%%%%%%%%%%%%%%%
\subsection{Driver Development}
\label{sec:driver}

The \textsf{childdoc} mechanism can also be use for the development
of definition files such as \LaTeX{} styles or classes.
This case differs from the above setup with multiple parts
included by |\include| in that no |\includeonly| should be invoked.
This can be achieved by starting the include file
(before |\ProvidesPackage|) with:
%
\begin{center}
\begin{tabular}{l}
|\input{childdoc.def}|\\
|\childdocforward{|\textit{main}|}|\\
\end{tabular}
\end{center}
%
or alternatively with:
%
\begin{center}
\begin{tabular}{l}
|\input{childdoc.def}|\\
|\childdocby{|\textit{main}|}|\\
\end{tabular}
\end{center}
%
Both forms have slightly different effects as described above.
The main file is prepared as usual, see \secref{sec:include}.

%%%%%%%%%%%%%%%%%%%%%%%%%%%%%%%%%%%%%%%%%%%%%%%%%%%%%%%%%%%%%%%%%%%%%%%%%%%%%%%%
\subsection{Legacy Detection}
\label{sec:detection}

The directive |\childdocmain| in the main file can detect
whether the complete document or merely a child is to be compiled
even without using the directive |\childdocof|.
This method is deprecated because it is less robust
and there is no compelling reason to use it;
it is merely provided for backward compatibility
and it may be removed in future versions.

If the detection mechanism is to be used,
it is mandatory to correctly specify
the filename of the main file as the argument of |\childdocmain|:
%
\begin{center}
\begin{tabular}{l}
|\input{childdoc.def}|\\
|\childdocmain{|\textit{main}|}|\\
\end{tabular}
\end{center}
%
If |\jobname| does not match the argument \textit{main} of |\childdocmain|,
it is assumed that |\jobname| points to the child file to be compiled.
When using |\childdocmain| with the main file specified as argument,
it suffices to start a child file
with just |\input{|\textit{main}|}|
without loading of the package and using |\childdocof|.
If instead all processing is done
with the appropriate \textsf{childdoc} directives,
the argument of \textit{main} of |\childdocmain| can be empty.

An alternative version of the command line processing described
in \secref{sec:commandline} using the detection mechanism reads:
%
\begin{center}
|... -jobname "|\textit{target}|" "|[\textit{flags}]%
[|\def\jobname{|\textit{dest}|}|]|\input{|\textit{main}|}"|
\end{center}

%%%%%%%%%%%%%%%%%%%%%%%%%%%%%%%%%%%%%%%%%%%%%%%%%%%%%%%%%%%%%%%%%%%%%%%%%%%%%%%%
\subsection{Manual Code}
\label{sec:manual}

In case one cannot be certain whether the definitions file |childdoc.def|
is installed on the target \TeX{} distribution
and one prefers not to ship it,
it is conceivable to paste a few relevant commands into the sources.

To that end, drop all statements |\input{childdoc.def}|
and perform the replacements as outlined below.
Instead of |\childdocmain{|\textit{main}|}| add the following code
to the top of the main file:
%
\begin{center}
\begin{tabular}{l}
|\||ifdefined\childdocname\endinput\||fi\newif\ifchilddoc|\\
|\edef\childdocname{\scantokens\expandafter{\jobname\noexpand}}|\\
|\def\childdocmain{|\textit{main}|}\||ifx\childdocmain\childdocname\||else|\\
|\childdoctrue\includeonly{\childdocname}\let\jobname\childdocmain\||fi|\\
\end{tabular}
\end{center}
%
Instead of |\childdocof{|\textit{main}|}| just include the main file
at the top of each child file:
%
\begin{center}
|\input{|\textit{main}|}|
\end{center}
%
A simple redirection |\childdocforward{|\textit{dest}|}| is achieved by:
%
\begin{center}
|\def\jobname{|\textit{dest}|}\input{\jobname}|
\end{center}
%
The redirection with prefix
|\childdocforwardprefix[|\textit{prefix}|]{|\textit{dest}|}|
is accomplished by:
%
\begin{center}
\begin{tabular}{l}
|{\edef\jobname{\scantokens\expandafter{\jobname\noexpand}}|\\
|\def\redirectjob |\textit{prefix}|#1~~~{\gdef\jobname{|\textit{dest}|#1}}|\\
|\expandafter\redirectjob\jobname~~~}\input{\jobname}|
\end{tabular}
\end{center}

In an alternative approach,
child documents can be compiled by a specific command line
without additional code or specific definitions:
%
\begin{center}
|... -jobname "|\textit{target}|" "|[\textit{flags}]%
|\includeonly{|\textit{dest}|}\input{|\textit{main}|}"|
\end{center}
%

%%%%%%%%%%%%%%%%%%%%%%%%%%%%%%%%%%%%%%%%%%%%%%%%%%%%%%%%%%%%%%%%%%%%%%%%%%%%%%%%
%%%%%%%%%%%%%%%%%%%%%%%%%%%%%%%%%%%%%%%%%%%%%%%%%%%%%%%%%%%%%%%%%%%%%%%%%%%%%%%%
\section{Information}

%%%%%%%%%%%%%%%%%%%%%%%%%%%%%%%%%%%%%%%%%%%%%%%%%%%%%%%%%%%%%%%%%%%%%%%%%%%%%%%%
\subsection{Copyright}

Copyright \copyright{} 2017--2018 Niklas Beisert

This work may be distributed and/or modified under the
conditions of the \LaTeX{} Project Public License, either version 1.3
of this license or (at your option) any later version.
The latest version of this license is in
  \url{http://www.latex-project.org/lppl.txt}
and version 1.3 or later is part of all distributions of \LaTeX{}
version 2005/12/01 or later.

This work has the LPPL maintenance status `maintained'.

The Current Maintainer of this work is Niklas Beisert.

This work consists of the files |README.txt|, |childdoc.ins| and |childdoc.dtx|
as well as the derived files |childdoc.def|, |cdocsamp.tex|
with |cdocsch1.tex|, |cdocsch2.tex|, |cdocspt3.tex|, |cdocspt4.tex|,
|cdocsdrf.tex|, |cdocsfn1.tex|, |cdocsfn2.tex|
as well as |childdoc.pdf|.

%%%%%%%%%%%%%%%%%%%%%%%%%%%%%%%%%%%%%%%%%%%%%%%%%%%%%%%%%%%%%%%%%%%%%%%%%%%%%%%%
\subsection{Files and Installation}

The package consists of the files:
%
\begin{center}
\begin{tabular}{ll}
    |README.txt|   & readme file \\
    |childdoc.ins| & installation file \\
    |childdoc.dtx| & source file \\
    |childdoc.def| & definition file \\
    |cdocsamp.tex| & sample main file \\
    |cdocsch1.tex| & sample include file \\
    |cdocsch2.tex| & sample include file \\
    |cdocspt3.tex| & sample part file \\
    |cdocspt4.tex| & sample part file \\
    |cdocsdrf.tex| & sample redirection file \\
    |cdocsfn1.tex| & sample redirection file \\
    |cdocsfn2.tex| & sample redirection file \\
    |childdoc.pdf| & manual
\end{tabular}
\end{center}
%
The distribution consists of the files
|README.txt|, |childdoc.ins| and |childdoc.dtx|.
%
\begin{itemize}
\item
Run (pdf)\LaTeX{} on |childdoc.dtx|
to compile the manual |childdoc.pdf| (this file).
\item
Run \LaTeX{} on |childdoc.ins| to create the definitions file |childdoc.def|
and the sample |cdocsamp.tex| with include files
|cdocsch1.tex|, |cdocsch2.tex|, |cdocspt3.tex|, |cdocspt4.tex|,
|cdocsdrf.tex|, |cdocsfn1.tex|, |cdocsfn2.tex|.
Then copy the file |childdoc.def| to an appropriate directory of your \LaTeX{}
distribution, e.g.\ \textit{texmf-root}|/tex/latex/childdoc|.
\end{itemize}

%%%%%%%%%%%%%%%%%%%%%%%%%%%%%%%%%%%%%%%%%%%%%%%%%%%%%%%%%%%%%%%%%%%%%%%%%%%%%%%%
\subsection{Related CTAN Packages}

There are several other packages which offer a similar functionality:
%
\begin{itemize}
\item
The packages
\href{http://ctan.org/pkg/docmute}{\textsf{docmute}},
\href{http://ctan.org/pkg/includex}{\textsf{includex}} and
\href{http://ctan.org/pkg/standalone}{\textsf{standalone}}
provide commands to include only the document body of
a child file thus allowing both files to be compiled individually.
\item
The packages \href{http://ctan.org/pkg/subdocs}{\textsf{subdocs}}
and \href{http://ctan.org/pkg/subfiles}{\textsf{subfiles}}
provide structures in which the main and child documents can be
encapsulated and allowing them to be compiled individually.
The inclusion mechanism is different from the conventional |\include|.
\item
The package \href{http://ctan.org/pkg/combine}{\textsf{combine}}
is an elaborate solution to combine several documents into one.
\end{itemize}
%
See also the CTAN topic \href{http://ctan.org/topic/subdocs}{\textsf{subdocs}}
for further related packages.
The present package differs from the above solutions in that
a document structure constructed with the conventional |\include| mechanism
just needs two extra commands at the top of every file
such that all constituent files can be compiled individually.

%%%%%%%%%%%%%%%%%%%%%%%%%%%%%%%%%%%%%%%%%%%%%%%%%%%%%%%%%%%%%%%%%%%%%%%%%%%%%%%%
%\subsection{Feature Suggestions}
%
%The following is a list of features which may be useful for future
%versions of this package:
%%
%\begin{itemize}
%\item
%\ldots
%\end{itemize}

%%%%%%%%%%%%%%%%%%%%%%%%%%%%%%%%%%%%%%%%%%%%%%%%%%%%%%%%%%%%%%%%%%%%%%%%%%%%%%%%
\subsection{Revision History}

%%%%%%%%%%%%%%%%%%%%%%%%%%%%%%%%%%%%%%%%
\paragraph{v2.0:} 2018/12/30

\begin{itemize}
\item
immediate forward processing
\item
added |\childdocby| mechanism
\item
manual restructured
\end{itemize}

%%%%%%%%%%%%%%%%%%%%%%%%%%%%%%%%%%%%%%%%
\paragraph{v1.6:} 2018/01/17

\begin{itemize}
\item
application for development of include files
\item
corrections to manual
\end{itemize}

%%%%%%%%%%%%%%%%%%%%%%%%%%%%%%%%%%%%%%%%
\paragraph{v1.5:} 2017/05/21

\begin{itemize}
\item
more complete structuring introduced
\item
|\childdocof| introduced
\item
|\childdoc| renamed to |\childdocmain|
\item
|\childredirect| renamed to |\childdocforward| and |\childdocforwardprefix|
and functionality expanded
\end{itemize}

%%%%%%%%%%%%%%%%%%%%%%%%%%%%%%%%%%%%%%%%
\paragraph{v1.0:} 2017/04/27

\begin{itemize}
\item
manual and install package
\item
first version published on CTAN
\end{itemize}

%%%%%%%%%%%%%%%%%%%%%%%%%%%%%%%%%%%%%%%%
\paragraph{v0.6:} 2017/04/26

\begin{itemize}
\item
redirection mechanism added
\end{itemize}

%%%%%%%%%%%%%%%%%%%%%%%%%%%%%%%%%%%%%%%%
\paragraph{v0.5:} 2017/04/26

\begin{itemize}
\item
functionality in definition file
\end{itemize}


%%%%%%%%%%%%%%%%%%%%%%%%%%%%%%%%%%%%%%%%%%%%%%%%%%%%%%%%%%%%%%%%%%%%%%%%%%%%%%%%
%%%%%%%%%%%%%%%%%%%%%%%%%%%%%%%%%%%%%%%%%%%%%%%%%%%%%%%%%%%%%%%%%%%%%%%%%%%%%%%%
%%%%%%%%%%%%%%%%%%%%%%%%%%%%%%%%%%%%%%%%%%%%%%%%%%%%%%%%%%%%%%%%%%%%%%%%%%%%%%%%
\appendix

\settowidth\MacroIndent{\rmfamily\scriptsize 000\ }

 \DocInput{childdoc.dtx}

\end{document}
%</driver>
% \fi
%
% %%%%%%%%%%%%%%%%%%%%%%%%%%%%%%%%%%%%%%%%%%%%%%%%%%%%%%%%%%%%%%%%%%%%%%%%%%%%%%
% %%%%%%%%%%%%%%%%%%%%%%%%%%%%%%%%%%%%%%%%%%%%%%%%%%%%%%%%%%%%%%%%%%%%%%%%%%%%%%
% \section{Sample}
%\iffalse
%<*samplemain>
%\fi
%
% The following presents a sample document
% with two chapters, two parts, a title page,
% a compile flag as well as three forwarding files to set the flag.
% It consists of eight |.tex| files:
% \begin{center}
% \begin{tabular}{ll}
% |cdocsamp.tex|&main file\\
% |cdocsch1.tex|&include file for chapter 1\\
% |cdocsch2.tex|&include file for chapter 2\\
% |cdocspt3.tex|&include file for part 3\\
% |cdocspt4.tex|&include file for part 4\\
% |cdocsdrf.tex|&forwarding file for main file in draft mode\\
% |cdocsfi1.tex|&forwarding file for final version of chapter 1\\
% |cdocsfi2.tex|&forwarding file for final version of chapter 2\\
% \end{tabular}
% \end{center}
% Each of the eight files can be compiled directly by the \LaTeX{} compiler.
%
% %%%%%%%%%%%%%%%%%%%%%%%%%%%%%%%%%%%%%%
% \paragraph{Main File.}
%
% The main file is called |cdocsamp.tex|.
%
% Load the \textsf{childdoc} definitions and
% declare the filename for the main document:
%    \begin{macrocode}
\input{childdoc.def}
\childdocmain{}
%    \end{macrocode}

% Optional override for |\version| flag:
%    \begin{macrocode}
%%\ifchilddoc\else\providecommand{\version}{draft}\fi
%    \end{macrocode}

% Define the default values for the |\version| flag
% (|final| for the main file and |draft| for childs):
%    \begin{macrocode}
\ifchilddoc
\providecommand{\version}{draft}
\else
\providecommand{\version}{final}
\fi
%    \end{macrocode}

% Load the standard document class:
%    \begin{macrocode}
\documentclass[12pt]{article}
%    \end{macrocode}

% Start the document body:
%    \begin{macrocode}
\begin{document}
%    \end{macrocode}

% Declare a title page.
% Print title, part of document being processed and version flag:
%    \begin{macrocode}
\addtocounter{page}{-1}
\begin{center}
{\LARGE\bfseries{}childdoc example\par}
\vspace{1cm}
\ifchilddoc
\ifchilddocmanual part\else chapter\fi:
`\childdocname' of `\childdocjob'\par
\else
main document: `\childdocjob'\par
\fi
version: \version\par
\end{center}
\newpage
%    \end{macrocode}

% Manually include selected file,
% otherwise process as usual:
%    \begin{macrocode}
\ifchilddocmanual
\section*{part `\childdocname'}
\input{\childdocname}
\else
%    \end{macrocode}

% Include the two chapters:
%    \begin{macrocode}
\include{cdocsch1}
\include{cdocsch2}
%    \end{macrocode}

% Include the two parts unless only chapters should be displayed:
%    \begin{macrocode}
\ifchilddoc\else
\section{part three}
\input{cdocspt3}
\section{part four}
\input{cdocspt4}
\fi
%    \end{macrocode}

% Process as usual until here:
%    \begin{macrocode}
\fi
%    \end{macrocode}

% End of document body:
%    \begin{macrocode}
\end{document}
%    \end{macrocode}
%\iffalse
%</samplemain>
%\fi
%
% %%%%%%%%%%%%%%%%%%%%%%%%%%%%%%%%%%%%%%
% \paragraph{Chapter Include Files.}
%
% The include files are called |cdocsch1.tex| and |cdocsch2.tex|.
%
%\iffalse
%<*samplechap1|samplechap2>
%\fi

% Optional override for |\version| flag:
%    \begin{macrocode}
%%\providecommand{\version}{final}
%    \end{macrocode}

% Include the main document:
%    \begin{macrocode}
\input{childdoc.def}
\childdocof{cdocsamp}
%    \end{macrocode}

%\iffalse
%</samplechap1|samplechap2>
%\fi
%
%\iffalse
%<*samplechap1>
%\fi
% Some text for chapter 1:
%    \begin{macrocode}
\section{one}
some text in chapter one
%    \end{macrocode}

%\iffalse
%</samplechap1>
%\fi
% Some text for chapter 2:
%\iffalse
%<*samplechap2>
%\fi
%    \begin{macrocode}
\section{two}
more text in chapter two
%    \end{macrocode}

%\iffalse
%</samplechap2>
%\fi
%
% %%%%%%%%%%%%%%%%%%%%%%%%%%%%%%%%%%%%%%
% \paragraph{Part Include Files.}
%
% The include files are called |cdocspt3.tex| and |cdocspt4.tex|.
%
%\iffalse
%<*samplepart3|samplepart4>
%\fi

% Optional override for |\version| flag:
%    \begin{macrocode}
%%\providecommand{\version}{final}
%    \end{macrocode}

% Include the main document:
%    \begin{macrocode}
\input{childdoc.def}
\childdocby{cdocsamp}
%    \end{macrocode}

%\iffalse
%</samplepart3|samplepart4>
%\fi
%
%\iffalse
%<*samplepart3>
%\fi
% Some text for part 3:
%    \begin{macrocode}
some text in part three
%    \end{macrocode}

%\iffalse
%</samplepart3>
%\fi
% Some text for part 4:
%\iffalse
%<*samplepart4>
%\fi
%    \begin{macrocode}
more text in part four
%    \end{macrocode}

%\iffalse
%</samplepart4>
%\fi
%
% %%%%%%%%%%%%%%%%%%%%%%%%%%%%%%%%%%%%%%
% \paragraph{Forwarding for a Complete Draft.}
%
% The following forwarding file |cdocsdrf.tex|
% compiles the main document in draft mode:
%\iffalse
%<*sampledraft>
%\fi
%    \begin{macrocode}
\def\version{draft}
\input{childdoc.def}
\childdocforward{cdocsamp}
%    \end{macrocode}

%\iffalse
%</sampledraft>
%\fi
%
% %%%%%%%%%%%%%%%%%%%%%%%%%%%%%%%%%%%%%%
% \paragraph{Forwarding for Final Version of the Chapters.}
%
% The following forwarding files |cdocsfn1.tex| and |cdocsfn2.tex|
% (with identical content)
% compile the final versions of the child documents
% |cdocsch1.tex| and |cdocsch2.tex|, respectively:
%\iffalse
%<*samplefinal>
%\fi
%    \begin{macrocode}
\def\version{final}
\input{childdoc.def}
\childdocforwardprefix[cdocsamp]{cdocsfn}{cdocsch}
%    \end{macrocode}

%\iffalse
%</samplefinal>
%\fi
%
% %%%%%%%%%%%%%%%%%%%%%%%%%%%%%%%%%%%%%%
% \paragraph{Command Line Processing.}
%
% The following three command lines generate the output files
% |cdocscld|, |cdocscl1| and |cdocscl2|
% which should be identical to
% |cdocsdrf|, |cdocsch1| and |cdocsfn2|, respectively:
% \begin{center}
% \begin{tabular}{l}
% |latex -jobname cdocscld \|\\
% |  "\def\version{draft}\input{childdoc.def}\childdocforward{cdocsamp}"|\\
% |latex -jobname cdocscl1 \|\\
% |  "\input{childdoc.def}\childdocforward[cdocsamp]{cdocsch1}"|\\
% |latex -jobname cdocscl2 \|\\
% |  "\def\version{final}\input{childdoc.def}\childdocforward{cdocsch2}"|
% \end{tabular}
% \end{center}
% Note that the trailing backslash on each first line
% merely continues the input to the second line
% (for convenient cut ant paste).
% Furthermore, the command |latex| can be replaced by any
% of its alternative versions such as |pdflatex|.
%
% %%%%%%%%%%%%%%%%%%%%%%%%%%%%%%%%%%%%%%%%%%%%%%%%%%%%%%%%%%%%%%%%%%%%%%%%%%%%%%
% %%%%%%%%%%%%%%%%%%%%%%%%%%%%%%%%%%%%%%%%%%%%%%%%%%%%%%%%%%%%%%%%%%%%%%%%%%%%%%
% \section{Implementation}
%\iffalse
%<*package>
%\fi
%
% This section describes the definitions file |childdoc.def|.

% The definitions cannot be loaded using |\usepackage| or |\RequirePackage|
% which has a mechanism to prevent loading a style file more than once.
% When loading the definitions by means of |\input|
% multiple instances have to be prevented manually:
%\iffalse
%This code needs to be before the `\ProvidesFile' directive
%which is defined at the beginning of this file.
%Therefore it is also placed there and commented out here.
%</package>
%<*discard>
%\fi
%    \begin{macrocode}
\ifdefined\childdocmain\endinput\fi
%    \end{macrocode}
%\iffalse
%</discard>
%<*package>
%\fi
%
% \macro{\ifchilddoc}
% \macro{\ifchilddocmanual}
% The conditional |\ifchilddoc| tells whether a
% child (true) or main (false) document is being compiled.
% The conditional |\ifchilddocmanual| tells whether
% the |\includeonly| mechanism is used (false) or
% the selection of child files must be performed manually (true).
% The definitions initialise to false:
%    \begin{macrocode}
\newif\ifchilddoc
\newif\ifchilddocmanual
%    \end{macrocode}

% \macro{\childdocname}
% \macro{\childdocjob}
% The macro |\childdocname| stores the name of the main document
% to be compiled. The macro |\childdocjob| stores the name of
% the document on which the \LaTeX{} compiler was originally invoked.
% The content of |\jobname| cannot be compared
% to filenames specified in the source due to different catcodes.
% The following code rescans |\jobname|, stores the result
% in |\childdocname| and saves a copy in |\childdocjob|:
%    \begin{macrocode}
\edef\childdocname{\scantokens\expandafter{\jobname\noexpand}}
\let\childdocjob\childdocname
%    \end{macrocode}

% \macro{\childdocdisable}
% The macro |\childdocdisable| prevents the main file
% from being processed more than once.
% At this stage, the main document command |\childdocmain|
% is assumed to be called once again where it should do nothing.
% Any subsequent call to it should prevent
% a secondary processing of the main document
% It overwrites the forwarding commands
% |\childdocof| and |\childdocforward|
% with empty macros to prevent further inclusions of the main document:
%    \begin{macrocode}
\newcommand{\childdocdisable}
{
  \renewcommand{\childdocmain}[1]{\renewcommand{\childdocmain}[1]{\endinput}}
  \renewcommand{\childdocof}[1]{}
  \renewcommand{\childdocby}[2][]{}
  \renewcommand{\childdocforward}[2][]{}
  \renewcommand{\childdocdisable}{}
}
%    \end{macrocode}

% \macro{\childdocmain}
% The macro |\childdocmain| is to be called at the top of the main file
% with nothing or the main filename (without extension) as argument.
% First, it breaks loops.
% If the argument is not empty and does not match |\childdocname|
% (which is set by the first inclusion of |childdoc.def|),
% |\ifchilddoc| is set to true, |\includeonly| is applied to the child file
% and |\jobname| is set to the main file
% (for proper handling of |.aux| files):
%    \begin{macrocode}
\newcommand{\childdocmain}[1]
{
  \childdocdisable\childdocmain{}
  \if?#1?\else
    \begingroup
      \def\childdoctmp{#1}
      \ifx\childdoctmp\childdocname
        \def\childdoctmp{}
      \else
        \def\childdoctmp
        {
          \childdoctrue
          \includeonly{\childdocname}
          \def\childdocjob{#1}
          \def\jobname{#1}
        }
      \fi
      \expandafter
    \endgroup
    \childdoctmp
  \fi
}
%    \end{macrocode}

% \macro{\childdocof}
% The command |\childdocof| redirects
% compilation to the main file |#1|.
%    \begin{macrocode}
\newcommand{\childdocof}[1]
{
  \childdocdisable
  \childdoctrue
  \includeonly{\childdocname}
  \def\jobname{#1}
  \def\childdocjob{#1}
  \input{#1}
}
%    \end{macrocode}

% \macro{\childdocby}
% The command |\childdocby| ....
%    \begin{macrocode}
\newcommand{\childdocby}[2][]
{
  \childdocdisable
  \childdoctrue
  \childdocmanualtrue
  \if?#1?\else
    \def\jobname{#2}
  \fi
  \def\childdocjob{#2}
  \input{#2}
  \endinput
}
%    \end{macrocode}

% \macro{\childdocforward}
% The command |\childdocforward| redirects
% compilation to the main file or
% (if the optional argument is given) a child file.
% Parameters are set as if the main file
% or a child file starting with |\childdocof| was compiled.
% Then compilation is handed over to the main file:
%    \begin{macrocode}
\newcommand{\childdocforward}[2][]
{
  \begingroup
    \if?#1?
      \def\childdoctmp
      {
        \def\childdocname{#2}
        \def\childdocjob{#2}
        \def\jobname{#2}
        \input{#2}
        \endinput
      }
    \else
      \def\childdoctmp
      {
        \childdocdisable
        \def\childdocname{#2}
        \childdoctrue
        \includeonly{#2}
        \def\childdocjob{#1}
        \def\jobname{#1}
        \input{#1}
        \endinput
      }
    \fi
    \expandafter
  \endgroup
  \childdoctmp
}
%    \end{macrocode}

% \macro{\childdocforwardprefix}
% The command |\childdocforwardprefix| redirects
% compilation to the main or a child file by means of a pattern.
% The prefix |#1| in the current filename is replaced by |#2|
% and the suffix of the current filename is kept
% (it is assumed that the filename does not contain the substring `|~~~|'
% which is used as a delimiter).
% Compilation is handed over to the new file by |\childdocforward|:
%    \begin{macrocode}
\newcommand{\childdocforwardprefix}[3][]
{
  \begingroup
    \def\childdocextract #2##1~~~{\def\childdoctmp{\childdocforward[#1]{#3##1}}}
    \expandafter\childdocextract\childdocname~~~
    \expandafter
  \endgroup
  \childdoctmp
}
%    \end{macrocode}

% \macro{\childdoc}
% The deprecated macro |\childdoc| is a legacy version of |\childdocmain|:
%    \begin{macrocode}
\newcommand{\childdoc}{\childdocmain}
%    \end{macrocode}

% \macro{\childdocredirect}
% The deprecated macro |\childdocredirect| is a legacy version
% of |\childdocforward| and |\childdocforwardprefix|:
%    \begin{macrocode}
\newcommand{\childdocredirect}[2][]
{
  \begingroup
    \if?#1?
      \def\childdoctmp{\childdocforward{#2}}
    \else
      \def\childdoctmp{\childdocforwardprefix{#1}{#2}}
    \fi
    \expandafter
  \endgroup
  \childdoctmp
}
%    \end{macrocode}

%\iffalse
%</package>
%\fi
%
\endinput
\childdocforward[|\textit{main}|]{|\textit{dest}|}"|
\end{center}
%
Here \textit{target} is the name of the output file,
\textit{main} is the name of the main file
and \textit{dest} is the name of the main or child file to be processed
(all filenames without extensions).
The optional argument \textit{main} can be omitted
if \textit{main} matches \textit{dest}.
Optionally, compilation \textit{flags} can be defined via |\def| commands.
This command line makes the \TeX{} engine believe
it is compiling the file \textit{target}
whose content is specified as the latter parameter.
The provided code then forwards the processing to
\textit{main} or \textit{dest} as described in \secref{sec:forward}.

%%%%%%%%%%%%%%%%%%%%%%%%%%%%%%%%%%%%%%%%%%%%%%%%%%%%%%%%%%%%%%%%%%%%%%%%%%%%%%%%
\subsection{Include by Input}
\label{sec:input}

Including child documents by |\include| has some restrictions by design.
Most notably, the content of a child document always occupies
its own set of pages; pages cannot be shared between child documents.
Usually, this behaviour makes perfect sense
because each child document contain an essential part of the document.
However, in some situations it may be desirable to compose
a document from a collection of parts
without having mandatory page breaks between then.
For this case, the package
provides a mechanism to include parts
by |\input| which can also be processed individually.
However, by construction this mechanism
requires manual handling of the content to be output.

%%%%%%%%%%%%%%%%%%%%%%%%%%%%%%%%%%%%%%%%
\DescribeMacro{\ifchilddocmanual}
The main file should be prepared as usual, see \secref{sec:include}.
However, the document body must make a distinction
between processing of an individual part and of the main document, e.g.:
%
\begin{center}
\begin{tabular}{l}
|\ifchilddocmanual|\\
|\input{\childdocname}|\\
|\||else|\\
\textit{document body with }|\input{|\textit{part}|}|\\
|\||fi|
\end{tabular}
\end{center}
%
The conditional |\ifchilddocmanual| is true whenever
a part to be included by |\input| is being compiled,
and the name of the part is stored in |\childdocname|.

%%%%%%%%%%%%%%%%%%%%%%%%%%%%%%%%%%%%%%%%
\DescribeMacro{\childdocby}
Each part to be included by |\input| should start with:
%
\begin{center}
\begin{tabular}{l}
|% \iffalse
%
% childdoc.dtx Copyright (C) 2017-2018 Niklas Beisert
%
% This work may be distributed and/or modified under the
% conditions of the LaTeX Project Public License, either version 1.3
% of this license or (at your option) any later version.
% The latest version of this license is in
%   http://www.latex-project.org/lppl.txt
% and version 1.3 or later is part of all distributions of LaTeX
% version 2005/12/01 or later.
%
% This work has the LPPL maintenance status `maintained'.
%
% The Current Maintainer of this work is Niklas Beisert.
%
% This work consists of the files childdoc.dtx and childdoc.ins
% and the derived files childdoc.def and cdocsamp.tex with
% cdocsch1.tex, cdocsch2.tex, cdocsdrf.tex, cdocsfn1.tex, cdocsfn2.tex.
%
%<package>\ifdefined\childdocmain\endinput\fi
%<package>\ProvidesFile{childdoc.def}[2018/12/30 v2.0 child document driver]
%<samplemain>\ProvidesFile{cdocsamp.tex}[2018/12/30 v2.0 sample for childdoc]
%<*driver>
%\ProvidesFile{childdoc.drv}[2018/12/30 v2.0 childdoc reference manual file]
\PassOptionsToClass{10pt,a4paper}{article}
\documentclass{ltxdoc}

\usepackage[margin=35mm]{geometry}
\usepackage{hyperref}
\usepackage{hyperxmp}
\usepackage[usenames]{color}

\hypersetup{colorlinks=true}
\hypersetup{pdfstartview=FitH}
\hypersetup{pdfpagemode=UseNone}
\hypersetup{pdfsource={}}
\hypersetup{pdflang={en-UK}}
\hypersetup{pdfcopyright={Copyright 2017-2018 Niklas Beisert.
  This work may be distributed and/or modified under the
  conditions of the LaTeX Project Public License, either version 1.3
  of this license or (at your option) any later version.}}
\hypersetup{pdflicenseurl={http://www.latex-project.org/lppl.txt}}
\hypersetup{pdfcontactaddress={ETH Zurich, ITP, HIT K,
  Wolfgang-Pauli-Strasse 27}}
\hypersetup{pdfcontactpostcode={8093}}
\hypersetup{pdfcontactcity={Zurich}}
\hypersetup{pdfcontactcountry={Switzerland}}
\hypersetup{pdfcontactemail={nbeisert@itp.phys.ethz.ch}}
\hypersetup{pdfcontacturl={http://people.phys.ethz.ch/\xmptilde nbeisert/}}

\newcommand{\secref}[1]{\hyperref[#1]{section \ref*{#1}}}

\parskip1ex
\parindent0pt
\let\olditemize\itemize
\def\itemize{\olditemize\parskip0pt}

\begin{document}

\title{The \textsf{childdoc} Package}
\hypersetup{pdftitle={The childdoc Package}}
\author{Niklas Beisert\\[2ex]
  Institut f\"ur Theoretische Physik\\
  Eidgen\"ossische Technische Hochschule Z\"urich\\
  Wolfgang-Pauli-Strasse 27, 8093 Z\"urich, Switzerland\\[1ex]
  \href{mailto:nbeisert@itp.phys.ethz.ch}
  {\texttt{nbeisert@itp.phys.ethz.ch}}}
\hypersetup{pdfauthor={Niklas Beisert}}
\hypersetup{pdfsubject={Manual for the LaTeX2e Package childdoc}}
\date{30 December 2018, \textsf{v2.0}}
\maketitle

\begin{abstract}\noindent
\textsf{childdoc} is a \LaTeXe{} package
that enables the direct compilation
of document sections included by |\include|
to individual files.
\end{abstract}

\begingroup
\parskip0ex
\tableofcontents
\endgroup

%%%%%%%%%%%%%%%%%%%%%%%%%%%%%%%%%%%%%%%%%%%%%%%%%%%%%%%%%%%%%%%%%%%%%%%%%%%%%%%%
%%%%%%%%%%%%%%%%%%%%%%%%%%%%%%%%%%%%%%%%%%%%%%%%%%%%%%%%%%%%%%%%%%%%%%%%%%%%%%%%
\section{Introduction}

\LaTeX{} provides a mechanism to structure a large document (such as a book)
into a main file and several child files (containing the chapters)
using the |\include| command.
This mechanism is beneficial for documents
which span hundreds of pages in order to
make the source file(s) more manageable.
Moreover, compilation can be restricted to
selected child files by means of the |\includeonly| command.
The latter feature can be used to reduce the compilation time while editing
(this was significantly more useful in the earlier days of \LaTeX{})
or to generate a smaller document which is easier to navigate.
Another application of |\includeonly| is to generate
documents consisting of selected parts of the complete document.

However, there are a few drawbacks of the plain |\include| mechanism:
\begin{itemize}
\item
The child files cannot be compiled on their own,
they can only be compiled via the main file.
A naive editing environment
(such as a text editor with an option
to have the current file processed by \LaTeX)
may require one to switch to the main file before compiling;
attempting to compile the child file produces errors.
\item
The main file must be modified (each time)
to adjust the |\includeonly| command
to the present needs. This easily leaves the main file in a messy state.
\item
The generated document will always carry the filename
of the main document. This is inconvenient if
several child files are to be compiled and
to be kept for distribution.
\end{itemize}

The present package provides a simple interface
to make child files individually compilable by \LaTeX{}.
Compiling a child file then has the same effect as compiling
the main file with an |\includeonly| command
to select the appropriate child.
Moreover the generated document will carry the name of the child
rather than the main file.
This resolves all three above issues.

This feature is meant to make the editing of books,
thesis documents and lecture notes somewhat more convenient.
However, the package can also be used efficiently for
composing a series of documents (such as exercise sheets)
which are typically distributed individually.
It then assists the author in generating the individual documents
(potentially in different versions)
as well as a document containing the collected series.
Another application is in developing style files
or other kinds of included material
where compilation of the style file could redirect
to a sample or test file.

%%%%%%%%%%%%%%%%%%%%%%%%%%%%%%%%%%%%%%%%%%%%%%%%%%%%%%%%%%%%%%%%%%%%%%%%%%%%%%%%
%%%%%%%%%%%%%%%%%%%%%%%%%%%%%%%%%%%%%%%%%%%%%%%%%%%%%%%%%%%%%%%%%%%%%%%%%%%%%%%%
\section{Usage}

First of all, the package \textsf{childdoc} is \emph{not} a standard
\LaTeXe{} |.sty| style file! Therefore it needs to be invoked in
a non-standard way.

%%%%%%%%%%%%%%%%%%%%%%%%%%%%%%%%%%%%%%%%%%%%%%%%%%%%%%%%%%%%%%%%%%%%%%%%%%%%%%%%
\subsection{Included Files}
\label{sec:include}

%%%%%%%%%%%%%%%%%%%%%%%%%%%%%%%%%%%%%%%%
\DescribeMacro{\childdocmain}
To use the package, add the commands
\begin{center}
\begin{tabular}{l}
|\input{childdoc.def}|\\
|\childdocmain{}|\\
\end{tabular}
\end{center}
at the very top of the main \LaTeX{} file,
in particular \emph{before} the |\documentclass| statement!
The argument of |\childdocmain| should be left empty
(but it must be present).

%%%%%%%%%%%%%%%%%%%%%%%%%%%%%%%%%%%%%%%%
\DescribeMacro{\childdocof}
Furthermore, add the commands
\begin{center}
\begin{tabular}{l}
|\input{childdoc.def}|\\
|\childdocof{|\textit{main}|}|\\
\end{tabular}
\end{center}
at the top of every child file \textit{child}
which is included by |\include{|\textit{child}|}|
from within the main file
(or at least for those files to be compiled individually).
The argument \textit{main} must be the filename of the main file.

There are a couple of
considerations in setting up the main and child documents:

%%%%%%%%%%%%%%%%%%%%%%%%%%%%%%%%%%%%%%%%
\paragraph{Restrictions.}

Please note the following restrictions:
\begin{itemize}
\item
|\childdocmain| must be called with one argument \textit{main}
to ensure compatibility with earlier version of the package.
It must either be empty (|\childdocmain{}|)
or precisely match the filename of the main file in which it is specified.
See \secref{sec:detection} for further information.
\item
The filename \textit{main} must be specified without the |.tex| extension.
\item
The filename \textit{main} is case sensitive
(even in case-insensitive file systems)
due to internal string comparison.
\item
The argument \textit{main} should be fully expanded, it cannot be a macro.
\item
Subdirectories and special characters should be avoided in filenames.
\item
The command |\childdocmain{|\textit{main}|}| must be followed by a whitespace.
It should not be followed immediately by another command
or by a comment mark `|%|'.
This is because the \TeX{} parser reads the token immediately following
the argument of |\childdocmain| and puts it
at the beginning of every child section;
however, a white\-space is ignored.
\end{itemize}

%%%%%%%%%%%%%%%%%%%%%%%%%%%%%%%%%%%%%%%%
\paragraph{Content of Main File.}

It is advisable to place all content in the child files included by |\include|.
Any output contained in the main file will appear in all child documents
unless suppressed manually;
it cannot be suppressed automatically by the |\includeonly| directive
and thus should normally be avoided.
A method to include some content in the main file
by means of conditional processing is described in \secref{sec:conditional}.

%%%%%%%%%%%%%%%%%%%%%%%%%%%%%%%%%%%%%%%%
\paragraph{Page Numbering.}

When only a part of the document is compiled,
the appropriate numbering of pages
(as well as other status parameters)
is determined from the |.aux| files.
The latter contain information from previous passes.
However this information needs to propagate through
all intermediate child documents.
Therefore the page numbering in child documents may well
be inconsistent until the complete document is compiled at least once.

A useful (if unconventional) way to always ensure a consistent
page numbering is to restart the numbering in each child document
and denote the pages by `\textit{child}|.|\textit{page}'
where \textit{child} represents the chapter/section number of the child file.
This can be achieved by the command
|\numberwithin{page}{|\textit{child}|}|
of the \textsf{amsmath} package
where \textit{child} can be |chapter| or |section|
depending on the chosen structuring.
Alternatively, one can modify the macro |\thepage| appropriately
and reset the counter |page| at the start of each child file.

%%%%%%%%%%%%%%%%%%%%%%%%%%%%%%%%%%%%%%%%%%%%%%%%%%%%%%%%%%%%%%%%%%%%%%%%%%%%%%%%
\subsection{Conditional Processing}
\label{sec:conditional}

The package provides a mechanism to compile different versions
of a document. To customise the versions further some conditional processing
can come in handy to distinguish which version is being compiled.
The package provides two macros to describe the compilation context:

%%%%%%%%%%%%%%%%%%%%%%%%%%%%%%%%%%%%%%%%
\DescribeMacro{\ifchilddoc}
The conditional |\ifchilddoc| distinguishes between the compilation of
child documents and the main document:
%
\begin{center}
|\ifchilddoc |\textit{child-code}| |[|\||else |\textit{main-code}]| \||fi|
\end{center}

%%%%%%%%%%%%%%%%%%%%%%%%%%%%%%%%%%%%%%%%
\DescribeMacro{\childdocname}
\DescribeMacro{\childdocjob}
The macro |\childdocname| contains the filename (without extension)
of the main or child file being processed.
Note that |\childdocjob| will always contain the name of the main file.

%%%%%%%%%%%%%%%%%%%%%%%%%%%%%%%%%%%%%%%%
\paragraph{Title Page.}

Conditional processing can be used to include a title or banner page
in the main document when proper precautions are taken.
Importantly, the code in the main file should ensure that the page counter
(as well as other status parameters which are stored in the |.aux| files)
takes the same value after the conditional processing.
Otherwise the page numbers may take divergent values
depending on which part is compiled.

For example, a title page could be declared by:
%
\begin{center}
\begin{tabular}{l}
|\ifchilddoc\||else|\\
|\addtocounter{page}{-1}|\\
\textit{code for title page}\\
|\newpage|\\
|\||fi|
\end{tabular}
\end{center}
%
A banner page for the child documents can be generated by:
%
\begin{center}
\begin{tabular}{l}
|\ifchilddoc|\\
|\addtocounter{page}{-1}|\\
\textit{code for banner page}\\
|\newpage|\\
|\||fi|
\end{tabular}
\end{center}
%
Here one could write a message such as:
\begin{center}
|This is the part \childdocname{} of \childdocjob{}.|
\end{center}

%%%%%%%%%%%%%%%%%%%%%%%%%%%%%%%%%%%%%%%%%%%%%%%%%%%%%%%%%%%%%%%%%%%%%%%%%%%%%%%%
\subsection{Flags}
\label{sec:flags}

The package makes it easy to generate different versions
of the main or child documents.
To this end compilation flags can be defined
and assigned different default values.
They will be particularly useful in conjunction
with the forwarding mechanism described in \secref{sec:forward}.

For example, it may be useful to have a flag |\version|
which can be set to |draft| or |final|.
The document source will contain some conditional code
depending on the value of |\version|.
Suppose further, the flag should default to |final| for the main file
and to |draft| for child files
which is a natural assignment for editing the document.
This is achieved by placing the following code
in the preamble of the main document
(below the |\childdocmain| directive):
%
\begin{center}
\begin{tabular}{l}
|\ifchilddoc|\\
|\providecommand{\version}{draft}|\\
|\||else|\\
|\providecommand{\version}{final}|\\
|\||fi|
\end{tabular}
\end{center}
%
The definition by |\providecommand| makes sure
that previous definitions are not overwritten.
Further statements |\providecommand{\version}{...}|
can thus be added before the above code to override it.

For the main file, one might add a line
(between |\childdocmain| and the above block)
%
\begin{center}
|%\ifchilddoc\||else\providecommand{\version}{draft}\||fi|
\end{center}
%
which can be uncommented to produce a draft version.
Likewise one can add a line to the very top of a child file
(above the |\childdocof{|\textit{main}|}| directive)
%
\begin{center}
|%\providecommand{\version}{final}|
\end{center}
%
which can be uncommented to produce the final version of this child document.

%%%%%%%%%%%%%%%%%%%%%%%%%%%%%%%%%%%%%%%%%%%%%%%%%%%%%%%%%%%%%%%%%%%%%%%%%%%%%%%%
\subsection{Forwarding}
\label{sec:forward}

Different versions of the main or child documents
using compilation flags as described in \secref{sec:flags}
can be (permanently) stored in different files
for convenient compilation, viewing and distribution.
To this end, the package defines a command
to pass on compilation to a different file:

%%%%%%%%%%%%%%%%%%%%%%%%%%%%%%%%%%%%%%%%
\DescribeMacro{\childdocforward}
The command |\childdocforward| redirects processing to
another source file:
%
\begin{center}
\begin{tabular}{l}
|\input{childdoc.def}|\\
|\childdocforward[|\textit{main}|]{|\textit{dest}|}|\\
\end{tabular}
\end{center}
%
The argument \textit{dest} is the destination file
(without extension).
It should be the main file or one of the child files.
Note that further \textsf{childdoc} directives
such as |\childdocof| and |\childdocforward|
in the indicated file will be processed in this form.
The optional argument \textit{main}
passes on directly to the main file \textit{main}
while pretending to compile the child \textit{dest}.
This form behaves as if \textit{dest}
issues |\childdocof{|\textit{main}|}| right away,
and no further \textsf{childdoc} directives will be processed.

%%%%%%%%%%%%%%%%%%%%%%%%%%%%%%%%%%%%%%%%
\DescribeMacro{\...prefix}
In the alternative form |\childdocforwardprefix|,
%
\begin{center}
\begin{tabular}{l}
|\input{childdoc.def}|\\
|\childdocforwardprefix[|\textit{main}|]{|\textit{prefix}|}{|\textit{dest}|}|
\end{tabular}
\end{center}
%
the destination file is determined by a pattern
depending on the current file:
To make this work, the current file must be called
`{\textit{prefix}\hspace{0.2em}\textit{suffix}}'
with \textit{prefix} matching precisely the argument.
Processing is then passed on to the file
`{\textit{dest}\hspace{0.2em}\textit{suffix}}'.
Surely, the same effect is achieved by
directly specifying the
argument `{\textit{dest}\hspace{0.2em}\textit{suffix}}'
in the first form.
However, that requires to set up a different file
for each child. With the alternative form of the command
all these files can have exactly the same content
which simplifies setting them up and maintaining them.

For example, the following file |draft.tex|
with a compilation flag |\version| as described in \secref{sec:flags}
compiles the main document as a draft:
%
\begin{center}
\begin{tabular}{l}
|\def\version{draft}|\\
|\input{childdoc.def}|\\
|\childdocforward{|\textit{main}|}|
\end{tabular}
\end{center}
%
Likewise, the following files |final|\textit{nn}|.tex|
compile the final version of the child document
|child|\textit{nn}|.tex|:
%
\begin{center}
\begin{tabular}{l}
|\def\version{final}|\\
|\input{childdoc.def}|\\
|\childdocforwardprefix{final}{child}|
\end{tabular}
\end{center}
%

Note that when several versions of a main file and/or of each child file
are to be generated, it may be convenient to set up a |Makefile| or
shell script to automatise the process.

%%%%%%%%%%%%%%%%%%%%%%%%%%%%%%%%%%%%%%%%%%%%%%%%%%%%%%%%%%%%%%%%%%%%%%%%%%%%%%%%
\subsection{Command Line Processing}
\label{sec:commandline}

The effect of redirection files can also be achieved by invoking
the \LaTeX{} compiler with a more elaborate command line.
Most conveniently this should be done as part
of a shell script or a |Makefile|.

When using \textsf{childdoc} in the main file, the following
command lines effectively perform a redirection
(note that depending on the shell being used,
backslashes may have to be doubled: `|\|' $\to$ `|\\|'):
%
\begin{center}
|... -jobname "|\textit{target}|" |\\|"|[\textit{flags}]%
|\input{childdoc.def}\childdocforward[|\textit{main}|]{|\textit{dest}|}"|
\end{center}
%
Here \textit{target} is the name of the output file,
\textit{main} is the name of the main file
and \textit{dest} is the name of the main or child file to be processed
(all filenames without extensions).
The optional argument \textit{main} can be omitted
if \textit{main} matches \textit{dest}.
Optionally, compilation \textit{flags} can be defined via |\def| commands.
This command line makes the \TeX{} engine believe
it is compiling the file \textit{target}
whose content is specified as the latter parameter.
The provided code then forwards the processing to
\textit{main} or \textit{dest} as described in \secref{sec:forward}.

%%%%%%%%%%%%%%%%%%%%%%%%%%%%%%%%%%%%%%%%%%%%%%%%%%%%%%%%%%%%%%%%%%%%%%%%%%%%%%%%
\subsection{Include by Input}
\label{sec:input}

Including child documents by |\include| has some restrictions by design.
Most notably, the content of a child document always occupies
its own set of pages; pages cannot be shared between child documents.
Usually, this behaviour makes perfect sense
because each child document contain an essential part of the document.
However, in some situations it may be desirable to compose
a document from a collection of parts
without having mandatory page breaks between then.
For this case, the package
provides a mechanism to include parts
by |\input| which can also be processed individually.
However, by construction this mechanism
requires manual handling of the content to be output.

%%%%%%%%%%%%%%%%%%%%%%%%%%%%%%%%%%%%%%%%
\DescribeMacro{\ifchilddocmanual}
The main file should be prepared as usual, see \secref{sec:include}.
However, the document body must make a distinction
between processing of an individual part and of the main document, e.g.:
%
\begin{center}
\begin{tabular}{l}
|\ifchilddocmanual|\\
|\input{\childdocname}|\\
|\||else|\\
\textit{document body with }|\input{|\textit{part}|}|\\
|\||fi|
\end{tabular}
\end{center}
%
The conditional |\ifchilddocmanual| is true whenever
a part to be included by |\input| is being compiled,
and the name of the part is stored in |\childdocname|.

%%%%%%%%%%%%%%%%%%%%%%%%%%%%%%%%%%%%%%%%
\DescribeMacro{\childdocby}
Each part to be included by |\input| should start with:
%
\begin{center}
\begin{tabular}{l}
|\input{childdoc.def}|\\
|\childdocby{|\textit{main}|}|\\
\end{tabular}
\end{center}
%
The directive |\childdocby| is similar to |\childdocof|
described in \secref{sec:include},
but the subsequent selection of content must be done manually.
To that end, both |\ifchilddoc| and |\ifchilddocmanual|
will be true upon processing of a part,
and the name of the part is stored in |\childdocname|.
Note that |\jobname| will be set to the filename of the current part
so that each part receives an individual |.aux| file
that does not interfere with the |.aux| file(s) of the main document.
This behaviour can be altered by the alternative form
|\childdocby[*]{|\textit{main}|}| (with a non-empty optional argument)
which uses the |.aux| file of the main document
by setting |\jobname| to \textit{main}.

%%%%%%%%%%%%%%%%%%%%%%%%%%%%%%%%%%%%%%%%%%%%%%%%%%%%%%%%%%%%%%%%%%%%%%%%%%%%%%%%
\subsection{Driver Development}
\label{sec:driver}

The \textsf{childdoc} mechanism can also be use for the development
of definition files such as \LaTeX{} styles or classes.
This case differs from the above setup with multiple parts
included by |\include| in that no |\includeonly| should be invoked.
This can be achieved by starting the include file
(before |\ProvidesPackage|) with:
%
\begin{center}
\begin{tabular}{l}
|\input{childdoc.def}|\\
|\childdocforward{|\textit{main}|}|\\
\end{tabular}
\end{center}
%
or alternatively with:
%
\begin{center}
\begin{tabular}{l}
|\input{childdoc.def}|\\
|\childdocby{|\textit{main}|}|\\
\end{tabular}
\end{center}
%
Both forms have slightly different effects as described above.
The main file is prepared as usual, see \secref{sec:include}.

%%%%%%%%%%%%%%%%%%%%%%%%%%%%%%%%%%%%%%%%%%%%%%%%%%%%%%%%%%%%%%%%%%%%%%%%%%%%%%%%
\subsection{Legacy Detection}
\label{sec:detection}

The directive |\childdocmain| in the main file can detect
whether the complete document or merely a child is to be compiled
even without using the directive |\childdocof|.
This method is deprecated because it is less robust
and there is no compelling reason to use it;
it is merely provided for backward compatibility
and it may be removed in future versions.

If the detection mechanism is to be used,
it is mandatory to correctly specify
the filename of the main file as the argument of |\childdocmain|:
%
\begin{center}
\begin{tabular}{l}
|\input{childdoc.def}|\\
|\childdocmain{|\textit{main}|}|\\
\end{tabular}
\end{center}
%
If |\jobname| does not match the argument \textit{main} of |\childdocmain|,
it is assumed that |\jobname| points to the child file to be compiled.
When using |\childdocmain| with the main file specified as argument,
it suffices to start a child file
with just |\input{|\textit{main}|}|
without loading of the package and using |\childdocof|.
If instead all processing is done
with the appropriate \textsf{childdoc} directives,
the argument of \textit{main} of |\childdocmain| can be empty.

An alternative version of the command line processing described
in \secref{sec:commandline} using the detection mechanism reads:
%
\begin{center}
|... -jobname "|\textit{target}|" "|[\textit{flags}]%
[|\def\jobname{|\textit{dest}|}|]|\input{|\textit{main}|}"|
\end{center}

%%%%%%%%%%%%%%%%%%%%%%%%%%%%%%%%%%%%%%%%%%%%%%%%%%%%%%%%%%%%%%%%%%%%%%%%%%%%%%%%
\subsection{Manual Code}
\label{sec:manual}

In case one cannot be certain whether the definitions file |childdoc.def|
is installed on the target \TeX{} distribution
and one prefers not to ship it,
it is conceivable to paste a few relevant commands into the sources.

To that end, drop all statements |\input{childdoc.def}|
and perform the replacements as outlined below.
Instead of |\childdocmain{|\textit{main}|}| add the following code
to the top of the main file:
%
\begin{center}
\begin{tabular}{l}
|\||ifdefined\childdocname\endinput\||fi\newif\ifchilddoc|\\
|\edef\childdocname{\scantokens\expandafter{\jobname\noexpand}}|\\
|\def\childdocmain{|\textit{main}|}\||ifx\childdocmain\childdocname\||else|\\
|\childdoctrue\includeonly{\childdocname}\let\jobname\childdocmain\||fi|\\
\end{tabular}
\end{center}
%
Instead of |\childdocof{|\textit{main}|}| just include the main file
at the top of each child file:
%
\begin{center}
|\input{|\textit{main}|}|
\end{center}
%
A simple redirection |\childdocforward{|\textit{dest}|}| is achieved by:
%
\begin{center}
|\def\jobname{|\textit{dest}|}\input{\jobname}|
\end{center}
%
The redirection with prefix
|\childdocforwardprefix[|\textit{prefix}|]{|\textit{dest}|}|
is accomplished by:
%
\begin{center}
\begin{tabular}{l}
|{\edef\jobname{\scantokens\expandafter{\jobname\noexpand}}|\\
|\def\redirectjob |\textit{prefix}|#1~~~{\gdef\jobname{|\textit{dest}|#1}}|\\
|\expandafter\redirectjob\jobname~~~}\input{\jobname}|
\end{tabular}
\end{center}

In an alternative approach,
child documents can be compiled by a specific command line
without additional code or specific definitions:
%
\begin{center}
|... -jobname "|\textit{target}|" "|[\textit{flags}]%
|\includeonly{|\textit{dest}|}\input{|\textit{main}|}"|
\end{center}
%

%%%%%%%%%%%%%%%%%%%%%%%%%%%%%%%%%%%%%%%%%%%%%%%%%%%%%%%%%%%%%%%%%%%%%%%%%%%%%%%%
%%%%%%%%%%%%%%%%%%%%%%%%%%%%%%%%%%%%%%%%%%%%%%%%%%%%%%%%%%%%%%%%%%%%%%%%%%%%%%%%
\section{Information}

%%%%%%%%%%%%%%%%%%%%%%%%%%%%%%%%%%%%%%%%%%%%%%%%%%%%%%%%%%%%%%%%%%%%%%%%%%%%%%%%
\subsection{Copyright}

Copyright \copyright{} 2017--2018 Niklas Beisert

This work may be distributed and/or modified under the
conditions of the \LaTeX{} Project Public License, either version 1.3
of this license or (at your option) any later version.
The latest version of this license is in
  \url{http://www.latex-project.org/lppl.txt}
and version 1.3 or later is part of all distributions of \LaTeX{}
version 2005/12/01 or later.

This work has the LPPL maintenance status `maintained'.

The Current Maintainer of this work is Niklas Beisert.

This work consists of the files |README.txt|, |childdoc.ins| and |childdoc.dtx|
as well as the derived files |childdoc.def|, |cdocsamp.tex|
with |cdocsch1.tex|, |cdocsch2.tex|, |cdocspt3.tex|, |cdocspt4.tex|,
|cdocsdrf.tex|, |cdocsfn1.tex|, |cdocsfn2.tex|
as well as |childdoc.pdf|.

%%%%%%%%%%%%%%%%%%%%%%%%%%%%%%%%%%%%%%%%%%%%%%%%%%%%%%%%%%%%%%%%%%%%%%%%%%%%%%%%
\subsection{Files and Installation}

The package consists of the files:
%
\begin{center}
\begin{tabular}{ll}
    |README.txt|   & readme file \\
    |childdoc.ins| & installation file \\
    |childdoc.dtx| & source file \\
    |childdoc.def| & definition file \\
    |cdocsamp.tex| & sample main file \\
    |cdocsch1.tex| & sample include file \\
    |cdocsch2.tex| & sample include file \\
    |cdocspt3.tex| & sample part file \\
    |cdocspt4.tex| & sample part file \\
    |cdocsdrf.tex| & sample redirection file \\
    |cdocsfn1.tex| & sample redirection file \\
    |cdocsfn2.tex| & sample redirection file \\
    |childdoc.pdf| & manual
\end{tabular}
\end{center}
%
The distribution consists of the files
|README.txt|, |childdoc.ins| and |childdoc.dtx|.
%
\begin{itemize}
\item
Run (pdf)\LaTeX{} on |childdoc.dtx|
to compile the manual |childdoc.pdf| (this file).
\item
Run \LaTeX{} on |childdoc.ins| to create the definitions file |childdoc.def|
and the sample |cdocsamp.tex| with include files
|cdocsch1.tex|, |cdocsch2.tex|, |cdocspt3.tex|, |cdocspt4.tex|,
|cdocsdrf.tex|, |cdocsfn1.tex|, |cdocsfn2.tex|.
Then copy the file |childdoc.def| to an appropriate directory of your \LaTeX{}
distribution, e.g.\ \textit{texmf-root}|/tex/latex/childdoc|.
\end{itemize}

%%%%%%%%%%%%%%%%%%%%%%%%%%%%%%%%%%%%%%%%%%%%%%%%%%%%%%%%%%%%%%%%%%%%%%%%%%%%%%%%
\subsection{Related CTAN Packages}

There are several other packages which offer a similar functionality:
%
\begin{itemize}
\item
The packages
\href{http://ctan.org/pkg/docmute}{\textsf{docmute}},
\href{http://ctan.org/pkg/includex}{\textsf{includex}} and
\href{http://ctan.org/pkg/standalone}{\textsf{standalone}}
provide commands to include only the document body of
a child file thus allowing both files to be compiled individually.
\item
The packages \href{http://ctan.org/pkg/subdocs}{\textsf{subdocs}}
and \href{http://ctan.org/pkg/subfiles}{\textsf{subfiles}}
provide structures in which the main and child documents can be
encapsulated and allowing them to be compiled individually.
The inclusion mechanism is different from the conventional |\include|.
\item
The package \href{http://ctan.org/pkg/combine}{\textsf{combine}}
is an elaborate solution to combine several documents into one.
\end{itemize}
%
See also the CTAN topic \href{http://ctan.org/topic/subdocs}{\textsf{subdocs}}
for further related packages.
The present package differs from the above solutions in that
a document structure constructed with the conventional |\include| mechanism
just needs two extra commands at the top of every file
such that all constituent files can be compiled individually.

%%%%%%%%%%%%%%%%%%%%%%%%%%%%%%%%%%%%%%%%%%%%%%%%%%%%%%%%%%%%%%%%%%%%%%%%%%%%%%%%
%\subsection{Feature Suggestions}
%
%The following is a list of features which may be useful for future
%versions of this package:
%%
%\begin{itemize}
%\item
%\ldots
%\end{itemize}

%%%%%%%%%%%%%%%%%%%%%%%%%%%%%%%%%%%%%%%%%%%%%%%%%%%%%%%%%%%%%%%%%%%%%%%%%%%%%%%%
\subsection{Revision History}

%%%%%%%%%%%%%%%%%%%%%%%%%%%%%%%%%%%%%%%%
\paragraph{v2.0:} 2018/12/30

\begin{itemize}
\item
immediate forward processing
\item
added |\childdocby| mechanism
\item
manual restructured
\end{itemize}

%%%%%%%%%%%%%%%%%%%%%%%%%%%%%%%%%%%%%%%%
\paragraph{v1.6:} 2018/01/17

\begin{itemize}
\item
application for development of include files
\item
corrections to manual
\end{itemize}

%%%%%%%%%%%%%%%%%%%%%%%%%%%%%%%%%%%%%%%%
\paragraph{v1.5:} 2017/05/21

\begin{itemize}
\item
more complete structuring introduced
\item
|\childdocof| introduced
\item
|\childdoc| renamed to |\childdocmain|
\item
|\childredirect| renamed to |\childdocforward| and |\childdocforwardprefix|
and functionality expanded
\end{itemize}

%%%%%%%%%%%%%%%%%%%%%%%%%%%%%%%%%%%%%%%%
\paragraph{v1.0:} 2017/04/27

\begin{itemize}
\item
manual and install package
\item
first version published on CTAN
\end{itemize}

%%%%%%%%%%%%%%%%%%%%%%%%%%%%%%%%%%%%%%%%
\paragraph{v0.6:} 2017/04/26

\begin{itemize}
\item
redirection mechanism added
\end{itemize}

%%%%%%%%%%%%%%%%%%%%%%%%%%%%%%%%%%%%%%%%
\paragraph{v0.5:} 2017/04/26

\begin{itemize}
\item
functionality in definition file
\end{itemize}


%%%%%%%%%%%%%%%%%%%%%%%%%%%%%%%%%%%%%%%%%%%%%%%%%%%%%%%%%%%%%%%%%%%%%%%%%%%%%%%%
%%%%%%%%%%%%%%%%%%%%%%%%%%%%%%%%%%%%%%%%%%%%%%%%%%%%%%%%%%%%%%%%%%%%%%%%%%%%%%%%
%%%%%%%%%%%%%%%%%%%%%%%%%%%%%%%%%%%%%%%%%%%%%%%%%%%%%%%%%%%%%%%%%%%%%%%%%%%%%%%%
\appendix

\settowidth\MacroIndent{\rmfamily\scriptsize 000\ }

 \DocInput{childdoc.dtx}

\end{document}
%</driver>
% \fi
%
% %%%%%%%%%%%%%%%%%%%%%%%%%%%%%%%%%%%%%%%%%%%%%%%%%%%%%%%%%%%%%%%%%%%%%%%%%%%%%%
% %%%%%%%%%%%%%%%%%%%%%%%%%%%%%%%%%%%%%%%%%%%%%%%%%%%%%%%%%%%%%%%%%%%%%%%%%%%%%%
% \section{Sample}
%\iffalse
%<*samplemain>
%\fi
%
% The following presents a sample document
% with two chapters, two parts, a title page,
% a compile flag as well as three forwarding files to set the flag.
% It consists of eight |.tex| files:
% \begin{center}
% \begin{tabular}{ll}
% |cdocsamp.tex|&main file\\
% |cdocsch1.tex|&include file for chapter 1\\
% |cdocsch2.tex|&include file for chapter 2\\
% |cdocspt3.tex|&include file for part 3\\
% |cdocspt4.tex|&include file for part 4\\
% |cdocsdrf.tex|&forwarding file for main file in draft mode\\
% |cdocsfi1.tex|&forwarding file for final version of chapter 1\\
% |cdocsfi2.tex|&forwarding file for final version of chapter 2\\
% \end{tabular}
% \end{center}
% Each of the eight files can be compiled directly by the \LaTeX{} compiler.
%
% %%%%%%%%%%%%%%%%%%%%%%%%%%%%%%%%%%%%%%
% \paragraph{Main File.}
%
% The main file is called |cdocsamp.tex|.
%
% Load the \textsf{childdoc} definitions and
% declare the filename for the main document:
%    \begin{macrocode}
\input{childdoc.def}
\childdocmain{}
%    \end{macrocode}

% Optional override for |\version| flag:
%    \begin{macrocode}
%%\ifchilddoc\else\providecommand{\version}{draft}\fi
%    \end{macrocode}

% Define the default values for the |\version| flag
% (|final| for the main file and |draft| for childs):
%    \begin{macrocode}
\ifchilddoc
\providecommand{\version}{draft}
\else
\providecommand{\version}{final}
\fi
%    \end{macrocode}

% Load the standard document class:
%    \begin{macrocode}
\documentclass[12pt]{article}
%    \end{macrocode}

% Start the document body:
%    \begin{macrocode}
\begin{document}
%    \end{macrocode}

% Declare a title page.
% Print title, part of document being processed and version flag:
%    \begin{macrocode}
\addtocounter{page}{-1}
\begin{center}
{\LARGE\bfseries{}childdoc example\par}
\vspace{1cm}
\ifchilddoc
\ifchilddocmanual part\else chapter\fi:
`\childdocname' of `\childdocjob'\par
\else
main document: `\childdocjob'\par
\fi
version: \version\par
\end{center}
\newpage
%    \end{macrocode}

% Manually include selected file,
% otherwise process as usual:
%    \begin{macrocode}
\ifchilddocmanual
\section*{part `\childdocname'}
\input{\childdocname}
\else
%    \end{macrocode}

% Include the two chapters:
%    \begin{macrocode}
\include{cdocsch1}
\include{cdocsch2}
%    \end{macrocode}

% Include the two parts unless only chapters should be displayed:
%    \begin{macrocode}
\ifchilddoc\else
\section{part three}
\input{cdocspt3}
\section{part four}
\input{cdocspt4}
\fi
%    \end{macrocode}

% Process as usual until here:
%    \begin{macrocode}
\fi
%    \end{macrocode}

% End of document body:
%    \begin{macrocode}
\end{document}
%    \end{macrocode}
%\iffalse
%</samplemain>
%\fi
%
% %%%%%%%%%%%%%%%%%%%%%%%%%%%%%%%%%%%%%%
% \paragraph{Chapter Include Files.}
%
% The include files are called |cdocsch1.tex| and |cdocsch2.tex|.
%
%\iffalse
%<*samplechap1|samplechap2>
%\fi

% Optional override for |\version| flag:
%    \begin{macrocode}
%%\providecommand{\version}{final}
%    \end{macrocode}

% Include the main document:
%    \begin{macrocode}
\input{childdoc.def}
\childdocof{cdocsamp}
%    \end{macrocode}

%\iffalse
%</samplechap1|samplechap2>
%\fi
%
%\iffalse
%<*samplechap1>
%\fi
% Some text for chapter 1:
%    \begin{macrocode}
\section{one}
some text in chapter one
%    \end{macrocode}

%\iffalse
%</samplechap1>
%\fi
% Some text for chapter 2:
%\iffalse
%<*samplechap2>
%\fi
%    \begin{macrocode}
\section{two}
more text in chapter two
%    \end{macrocode}

%\iffalse
%</samplechap2>
%\fi
%
% %%%%%%%%%%%%%%%%%%%%%%%%%%%%%%%%%%%%%%
% \paragraph{Part Include Files.}
%
% The include files are called |cdocspt3.tex| and |cdocspt4.tex|.
%
%\iffalse
%<*samplepart3|samplepart4>
%\fi

% Optional override for |\version| flag:
%    \begin{macrocode}
%%\providecommand{\version}{final}
%    \end{macrocode}

% Include the main document:
%    \begin{macrocode}
\input{childdoc.def}
\childdocby{cdocsamp}
%    \end{macrocode}

%\iffalse
%</samplepart3|samplepart4>
%\fi
%
%\iffalse
%<*samplepart3>
%\fi
% Some text for part 3:
%    \begin{macrocode}
some text in part three
%    \end{macrocode}

%\iffalse
%</samplepart3>
%\fi
% Some text for part 4:
%\iffalse
%<*samplepart4>
%\fi
%    \begin{macrocode}
more text in part four
%    \end{macrocode}

%\iffalse
%</samplepart4>
%\fi
%
% %%%%%%%%%%%%%%%%%%%%%%%%%%%%%%%%%%%%%%
% \paragraph{Forwarding for a Complete Draft.}
%
% The following forwarding file |cdocsdrf.tex|
% compiles the main document in draft mode:
%\iffalse
%<*sampledraft>
%\fi
%    \begin{macrocode}
\def\version{draft}
\input{childdoc.def}
\childdocforward{cdocsamp}
%    \end{macrocode}

%\iffalse
%</sampledraft>
%\fi
%
% %%%%%%%%%%%%%%%%%%%%%%%%%%%%%%%%%%%%%%
% \paragraph{Forwarding for Final Version of the Chapters.}
%
% The following forwarding files |cdocsfn1.tex| and |cdocsfn2.tex|
% (with identical content)
% compile the final versions of the child documents
% |cdocsch1.tex| and |cdocsch2.tex|, respectively:
%\iffalse
%<*samplefinal>
%\fi
%    \begin{macrocode}
\def\version{final}
\input{childdoc.def}
\childdocforwardprefix[cdocsamp]{cdocsfn}{cdocsch}
%    \end{macrocode}

%\iffalse
%</samplefinal>
%\fi
%
% %%%%%%%%%%%%%%%%%%%%%%%%%%%%%%%%%%%%%%
% \paragraph{Command Line Processing.}
%
% The following three command lines generate the output files
% |cdocscld|, |cdocscl1| and |cdocscl2|
% which should be identical to
% |cdocsdrf|, |cdocsch1| and |cdocsfn2|, respectively:
% \begin{center}
% \begin{tabular}{l}
% |latex -jobname cdocscld \|\\
% |  "\def\version{draft}\input{childdoc.def}\childdocforward{cdocsamp}"|\\
% |latex -jobname cdocscl1 \|\\
% |  "\input{childdoc.def}\childdocforward[cdocsamp]{cdocsch1}"|\\
% |latex -jobname cdocscl2 \|\\
% |  "\def\version{final}\input{childdoc.def}\childdocforward{cdocsch2}"|
% \end{tabular}
% \end{center}
% Note that the trailing backslash on each first line
% merely continues the input to the second line
% (for convenient cut ant paste).
% Furthermore, the command |latex| can be replaced by any
% of its alternative versions such as |pdflatex|.
%
% %%%%%%%%%%%%%%%%%%%%%%%%%%%%%%%%%%%%%%%%%%%%%%%%%%%%%%%%%%%%%%%%%%%%%%%%%%%%%%
% %%%%%%%%%%%%%%%%%%%%%%%%%%%%%%%%%%%%%%%%%%%%%%%%%%%%%%%%%%%%%%%%%%%%%%%%%%%%%%
% \section{Implementation}
%\iffalse
%<*package>
%\fi
%
% This section describes the definitions file |childdoc.def|.

% The definitions cannot be loaded using |\usepackage| or |\RequirePackage|
% which has a mechanism to prevent loading a style file more than once.
% When loading the definitions by means of |\input|
% multiple instances have to be prevented manually:
%\iffalse
%This code needs to be before the `\ProvidesFile' directive
%which is defined at the beginning of this file.
%Therefore it is also placed there and commented out here.
%</package>
%<*discard>
%\fi
%    \begin{macrocode}
\ifdefined\childdocmain\endinput\fi
%    \end{macrocode}
%\iffalse
%</discard>
%<*package>
%\fi
%
% \macro{\ifchilddoc}
% \macro{\ifchilddocmanual}
% The conditional |\ifchilddoc| tells whether a
% child (true) or main (false) document is being compiled.
% The conditional |\ifchilddocmanual| tells whether
% the |\includeonly| mechanism is used (false) or
% the selection of child files must be performed manually (true).
% The definitions initialise to false:
%    \begin{macrocode}
\newif\ifchilddoc
\newif\ifchilddocmanual
%    \end{macrocode}

% \macro{\childdocname}
% \macro{\childdocjob}
% The macro |\childdocname| stores the name of the main document
% to be compiled. The macro |\childdocjob| stores the name of
% the document on which the \LaTeX{} compiler was originally invoked.
% The content of |\jobname| cannot be compared
% to filenames specified in the source due to different catcodes.
% The following code rescans |\jobname|, stores the result
% in |\childdocname| and saves a copy in |\childdocjob|:
%    \begin{macrocode}
\edef\childdocname{\scantokens\expandafter{\jobname\noexpand}}
\let\childdocjob\childdocname
%    \end{macrocode}

% \macro{\childdocdisable}
% The macro |\childdocdisable| prevents the main file
% from being processed more than once.
% At this stage, the main document command |\childdocmain|
% is assumed to be called once again where it should do nothing.
% Any subsequent call to it should prevent
% a secondary processing of the main document
% It overwrites the forwarding commands
% |\childdocof| and |\childdocforward|
% with empty macros to prevent further inclusions of the main document:
%    \begin{macrocode}
\newcommand{\childdocdisable}
{
  \renewcommand{\childdocmain}[1]{\renewcommand{\childdocmain}[1]{\endinput}}
  \renewcommand{\childdocof}[1]{}
  \renewcommand{\childdocby}[2][]{}
  \renewcommand{\childdocforward}[2][]{}
  \renewcommand{\childdocdisable}{}
}
%    \end{macrocode}

% \macro{\childdocmain}
% The macro |\childdocmain| is to be called at the top of the main file
% with nothing or the main filename (without extension) as argument.
% First, it breaks loops.
% If the argument is not empty and does not match |\childdocname|
% (which is set by the first inclusion of |childdoc.def|),
% |\ifchilddoc| is set to true, |\includeonly| is applied to the child file
% and |\jobname| is set to the main file
% (for proper handling of |.aux| files):
%    \begin{macrocode}
\newcommand{\childdocmain}[1]
{
  \childdocdisable\childdocmain{}
  \if?#1?\else
    \begingroup
      \def\childdoctmp{#1}
      \ifx\childdoctmp\childdocname
        \def\childdoctmp{}
      \else
        \def\childdoctmp
        {
          \childdoctrue
          \includeonly{\childdocname}
          \def\childdocjob{#1}
          \def\jobname{#1}
        }
      \fi
      \expandafter
    \endgroup
    \childdoctmp
  \fi
}
%    \end{macrocode}

% \macro{\childdocof}
% The command |\childdocof| redirects
% compilation to the main file |#1|.
%    \begin{macrocode}
\newcommand{\childdocof}[1]
{
  \childdocdisable
  \childdoctrue
  \includeonly{\childdocname}
  \def\jobname{#1}
  \def\childdocjob{#1}
  \input{#1}
}
%    \end{macrocode}

% \macro{\childdocby}
% The command |\childdocby| ....
%    \begin{macrocode}
\newcommand{\childdocby}[2][]
{
  \childdocdisable
  \childdoctrue
  \childdocmanualtrue
  \if?#1?\else
    \def\jobname{#2}
  \fi
  \def\childdocjob{#2}
  \input{#2}
  \endinput
}
%    \end{macrocode}

% \macro{\childdocforward}
% The command |\childdocforward| redirects
% compilation to the main file or
% (if the optional argument is given) a child file.
% Parameters are set as if the main file
% or a child file starting with |\childdocof| was compiled.
% Then compilation is handed over to the main file:
%    \begin{macrocode}
\newcommand{\childdocforward}[2][]
{
  \begingroup
    \if?#1?
      \def\childdoctmp
      {
        \def\childdocname{#2}
        \def\childdocjob{#2}
        \def\jobname{#2}
        \input{#2}
        \endinput
      }
    \else
      \def\childdoctmp
      {
        \childdocdisable
        \def\childdocname{#2}
        \childdoctrue
        \includeonly{#2}
        \def\childdocjob{#1}
        \def\jobname{#1}
        \input{#1}
        \endinput
      }
    \fi
    \expandafter
  \endgroup
  \childdoctmp
}
%    \end{macrocode}

% \macro{\childdocforwardprefix}
% The command |\childdocforwardprefix| redirects
% compilation to the main or a child file by means of a pattern.
% The prefix |#1| in the current filename is replaced by |#2|
% and the suffix of the current filename is kept
% (it is assumed that the filename does not contain the substring `|~~~|'
% which is used as a delimiter).
% Compilation is handed over to the new file by |\childdocforward|:
%    \begin{macrocode}
\newcommand{\childdocforwardprefix}[3][]
{
  \begingroup
    \def\childdocextract #2##1~~~{\def\childdoctmp{\childdocforward[#1]{#3##1}}}
    \expandafter\childdocextract\childdocname~~~
    \expandafter
  \endgroup
  \childdoctmp
}
%    \end{macrocode}

% \macro{\childdoc}
% The deprecated macro |\childdoc| is a legacy version of |\childdocmain|:
%    \begin{macrocode}
\newcommand{\childdoc}{\childdocmain}
%    \end{macrocode}

% \macro{\childdocredirect}
% The deprecated macro |\childdocredirect| is a legacy version
% of |\childdocforward| and |\childdocforwardprefix|:
%    \begin{macrocode}
\newcommand{\childdocredirect}[2][]
{
  \begingroup
    \if?#1?
      \def\childdoctmp{\childdocforward{#2}}
    \else
      \def\childdoctmp{\childdocforwardprefix{#1}{#2}}
    \fi
    \expandafter
  \endgroup
  \childdoctmp
}
%    \end{macrocode}

%\iffalse
%</package>
%\fi
%
\endinput
|\\
|\childdocby{|\textit{main}|}|\\
\end{tabular}
\end{center}
%
The directive |\childdocby| is similar to |\childdocof|
described in \secref{sec:include},
but the subsequent selection of content must be done manually.
To that end, both |\ifchilddoc| and |\ifchilddocmanual|
will be true upon processing of a part,
and the name of the part is stored in |\childdocname|.
Note that |\jobname| will be set to the filename of the current part
so that each part receives an individual |.aux| file
that does not interfere with the |.aux| file(s) of the main document.
This behaviour can be altered by the alternative form
|\childdocby[*]{|\textit{main}|}| (with a non-empty optional argument)
which uses the |.aux| file of the main document
by setting |\jobname| to \textit{main}.

%%%%%%%%%%%%%%%%%%%%%%%%%%%%%%%%%%%%%%%%%%%%%%%%%%%%%%%%%%%%%%%%%%%%%%%%%%%%%%%%
\subsection{Driver Development}
\label{sec:driver}

The \textsf{childdoc} mechanism can also be use for the development
of definition files such as \LaTeX{} styles or classes.
This case differs from the above setup with multiple parts
included by |\include| in that no |\includeonly| should be invoked.
This can be achieved by starting the include file
(before |\ProvidesPackage|) with:
%
\begin{center}
\begin{tabular}{l}
|% \iffalse
%
% childdoc.dtx Copyright (C) 2017-2018 Niklas Beisert
%
% This work may be distributed and/or modified under the
% conditions of the LaTeX Project Public License, either version 1.3
% of this license or (at your option) any later version.
% The latest version of this license is in
%   http://www.latex-project.org/lppl.txt
% and version 1.3 or later is part of all distributions of LaTeX
% version 2005/12/01 or later.
%
% This work has the LPPL maintenance status `maintained'.
%
% The Current Maintainer of this work is Niklas Beisert.
%
% This work consists of the files childdoc.dtx and childdoc.ins
% and the derived files childdoc.def and cdocsamp.tex with
% cdocsch1.tex, cdocsch2.tex, cdocsdrf.tex, cdocsfn1.tex, cdocsfn2.tex.
%
%<package>\ifdefined\childdocmain\endinput\fi
%<package>\ProvidesFile{childdoc.def}[2018/12/30 v2.0 child document driver]
%<samplemain>\ProvidesFile{cdocsamp.tex}[2018/12/30 v2.0 sample for childdoc]
%<*driver>
%\ProvidesFile{childdoc.drv}[2018/12/30 v2.0 childdoc reference manual file]
\PassOptionsToClass{10pt,a4paper}{article}
\documentclass{ltxdoc}

\usepackage[margin=35mm]{geometry}
\usepackage{hyperref}
\usepackage{hyperxmp}
\usepackage[usenames]{color}

\hypersetup{colorlinks=true}
\hypersetup{pdfstartview=FitH}
\hypersetup{pdfpagemode=UseNone}
\hypersetup{pdfsource={}}
\hypersetup{pdflang={en-UK}}
\hypersetup{pdfcopyright={Copyright 2017-2018 Niklas Beisert.
  This work may be distributed and/or modified under the
  conditions of the LaTeX Project Public License, either version 1.3
  of this license or (at your option) any later version.}}
\hypersetup{pdflicenseurl={http://www.latex-project.org/lppl.txt}}
\hypersetup{pdfcontactaddress={ETH Zurich, ITP, HIT K,
  Wolfgang-Pauli-Strasse 27}}
\hypersetup{pdfcontactpostcode={8093}}
\hypersetup{pdfcontactcity={Zurich}}
\hypersetup{pdfcontactcountry={Switzerland}}
\hypersetup{pdfcontactemail={nbeisert@itp.phys.ethz.ch}}
\hypersetup{pdfcontacturl={http://people.phys.ethz.ch/\xmptilde nbeisert/}}

\newcommand{\secref}[1]{\hyperref[#1]{section \ref*{#1}}}

\parskip1ex
\parindent0pt
\let\olditemize\itemize
\def\itemize{\olditemize\parskip0pt}

\begin{document}

\title{The \textsf{childdoc} Package}
\hypersetup{pdftitle={The childdoc Package}}
\author{Niklas Beisert\\[2ex]
  Institut f\"ur Theoretische Physik\\
  Eidgen\"ossische Technische Hochschule Z\"urich\\
  Wolfgang-Pauli-Strasse 27, 8093 Z\"urich, Switzerland\\[1ex]
  \href{mailto:nbeisert@itp.phys.ethz.ch}
  {\texttt{nbeisert@itp.phys.ethz.ch}}}
\hypersetup{pdfauthor={Niklas Beisert}}
\hypersetup{pdfsubject={Manual for the LaTeX2e Package childdoc}}
\date{30 December 2018, \textsf{v2.0}}
\maketitle

\begin{abstract}\noindent
\textsf{childdoc} is a \LaTeXe{} package
that enables the direct compilation
of document sections included by |\include|
to individual files.
\end{abstract}

\begingroup
\parskip0ex
\tableofcontents
\endgroup

%%%%%%%%%%%%%%%%%%%%%%%%%%%%%%%%%%%%%%%%%%%%%%%%%%%%%%%%%%%%%%%%%%%%%%%%%%%%%%%%
%%%%%%%%%%%%%%%%%%%%%%%%%%%%%%%%%%%%%%%%%%%%%%%%%%%%%%%%%%%%%%%%%%%%%%%%%%%%%%%%
\section{Introduction}

\LaTeX{} provides a mechanism to structure a large document (such as a book)
into a main file and several child files (containing the chapters)
using the |\include| command.
This mechanism is beneficial for documents
which span hundreds of pages in order to
make the source file(s) more manageable.
Moreover, compilation can be restricted to
selected child files by means of the |\includeonly| command.
The latter feature can be used to reduce the compilation time while editing
(this was significantly more useful in the earlier days of \LaTeX{})
or to generate a smaller document which is easier to navigate.
Another application of |\includeonly| is to generate
documents consisting of selected parts of the complete document.

However, there are a few drawbacks of the plain |\include| mechanism:
\begin{itemize}
\item
The child files cannot be compiled on their own,
they can only be compiled via the main file.
A naive editing environment
(such as a text editor with an option
to have the current file processed by \LaTeX)
may require one to switch to the main file before compiling;
attempting to compile the child file produces errors.
\item
The main file must be modified (each time)
to adjust the |\includeonly| command
to the present needs. This easily leaves the main file in a messy state.
\item
The generated document will always carry the filename
of the main document. This is inconvenient if
several child files are to be compiled and
to be kept for distribution.
\end{itemize}

The present package provides a simple interface
to make child files individually compilable by \LaTeX{}.
Compiling a child file then has the same effect as compiling
the main file with an |\includeonly| command
to select the appropriate child.
Moreover the generated document will carry the name of the child
rather than the main file.
This resolves all three above issues.

This feature is meant to make the editing of books,
thesis documents and lecture notes somewhat more convenient.
However, the package can also be used efficiently for
composing a series of documents (such as exercise sheets)
which are typically distributed individually.
It then assists the author in generating the individual documents
(potentially in different versions)
as well as a document containing the collected series.
Another application is in developing style files
or other kinds of included material
where compilation of the style file could redirect
to a sample or test file.

%%%%%%%%%%%%%%%%%%%%%%%%%%%%%%%%%%%%%%%%%%%%%%%%%%%%%%%%%%%%%%%%%%%%%%%%%%%%%%%%
%%%%%%%%%%%%%%%%%%%%%%%%%%%%%%%%%%%%%%%%%%%%%%%%%%%%%%%%%%%%%%%%%%%%%%%%%%%%%%%%
\section{Usage}

First of all, the package \textsf{childdoc} is \emph{not} a standard
\LaTeXe{} |.sty| style file! Therefore it needs to be invoked in
a non-standard way.

%%%%%%%%%%%%%%%%%%%%%%%%%%%%%%%%%%%%%%%%%%%%%%%%%%%%%%%%%%%%%%%%%%%%%%%%%%%%%%%%
\subsection{Included Files}
\label{sec:include}

%%%%%%%%%%%%%%%%%%%%%%%%%%%%%%%%%%%%%%%%
\DescribeMacro{\childdocmain}
To use the package, add the commands
\begin{center}
\begin{tabular}{l}
|\input{childdoc.def}|\\
|\childdocmain{}|\\
\end{tabular}
\end{center}
at the very top of the main \LaTeX{} file,
in particular \emph{before} the |\documentclass| statement!
The argument of |\childdocmain| should be left empty
(but it must be present).

%%%%%%%%%%%%%%%%%%%%%%%%%%%%%%%%%%%%%%%%
\DescribeMacro{\childdocof}
Furthermore, add the commands
\begin{center}
\begin{tabular}{l}
|\input{childdoc.def}|\\
|\childdocof{|\textit{main}|}|\\
\end{tabular}
\end{center}
at the top of every child file \textit{child}
which is included by |\include{|\textit{child}|}|
from within the main file
(or at least for those files to be compiled individually).
The argument \textit{main} must be the filename of the main file.

There are a couple of
considerations in setting up the main and child documents:

%%%%%%%%%%%%%%%%%%%%%%%%%%%%%%%%%%%%%%%%
\paragraph{Restrictions.}

Please note the following restrictions:
\begin{itemize}
\item
|\childdocmain| must be called with one argument \textit{main}
to ensure compatibility with earlier version of the package.
It must either be empty (|\childdocmain{}|)
or precisely match the filename of the main file in which it is specified.
See \secref{sec:detection} for further information.
\item
The filename \textit{main} must be specified without the |.tex| extension.
\item
The filename \textit{main} is case sensitive
(even in case-insensitive file systems)
due to internal string comparison.
\item
The argument \textit{main} should be fully expanded, it cannot be a macro.
\item
Subdirectories and special characters should be avoided in filenames.
\item
The command |\childdocmain{|\textit{main}|}| must be followed by a whitespace.
It should not be followed immediately by another command
or by a comment mark `|%|'.
This is because the \TeX{} parser reads the token immediately following
the argument of |\childdocmain| and puts it
at the beginning of every child section;
however, a white\-space is ignored.
\end{itemize}

%%%%%%%%%%%%%%%%%%%%%%%%%%%%%%%%%%%%%%%%
\paragraph{Content of Main File.}

It is advisable to place all content in the child files included by |\include|.
Any output contained in the main file will appear in all child documents
unless suppressed manually;
it cannot be suppressed automatically by the |\includeonly| directive
and thus should normally be avoided.
A method to include some content in the main file
by means of conditional processing is described in \secref{sec:conditional}.

%%%%%%%%%%%%%%%%%%%%%%%%%%%%%%%%%%%%%%%%
\paragraph{Page Numbering.}

When only a part of the document is compiled,
the appropriate numbering of pages
(as well as other status parameters)
is determined from the |.aux| files.
The latter contain information from previous passes.
However this information needs to propagate through
all intermediate child documents.
Therefore the page numbering in child documents may well
be inconsistent until the complete document is compiled at least once.

A useful (if unconventional) way to always ensure a consistent
page numbering is to restart the numbering in each child document
and denote the pages by `\textit{child}|.|\textit{page}'
where \textit{child} represents the chapter/section number of the child file.
This can be achieved by the command
|\numberwithin{page}{|\textit{child}|}|
of the \textsf{amsmath} package
where \textit{child} can be |chapter| or |section|
depending on the chosen structuring.
Alternatively, one can modify the macro |\thepage| appropriately
and reset the counter |page| at the start of each child file.

%%%%%%%%%%%%%%%%%%%%%%%%%%%%%%%%%%%%%%%%%%%%%%%%%%%%%%%%%%%%%%%%%%%%%%%%%%%%%%%%
\subsection{Conditional Processing}
\label{sec:conditional}

The package provides a mechanism to compile different versions
of a document. To customise the versions further some conditional processing
can come in handy to distinguish which version is being compiled.
The package provides two macros to describe the compilation context:

%%%%%%%%%%%%%%%%%%%%%%%%%%%%%%%%%%%%%%%%
\DescribeMacro{\ifchilddoc}
The conditional |\ifchilddoc| distinguishes between the compilation of
child documents and the main document:
%
\begin{center}
|\ifchilddoc |\textit{child-code}| |[|\||else |\textit{main-code}]| \||fi|
\end{center}

%%%%%%%%%%%%%%%%%%%%%%%%%%%%%%%%%%%%%%%%
\DescribeMacro{\childdocname}
\DescribeMacro{\childdocjob}
The macro |\childdocname| contains the filename (without extension)
of the main or child file being processed.
Note that |\childdocjob| will always contain the name of the main file.

%%%%%%%%%%%%%%%%%%%%%%%%%%%%%%%%%%%%%%%%
\paragraph{Title Page.}

Conditional processing can be used to include a title or banner page
in the main document when proper precautions are taken.
Importantly, the code in the main file should ensure that the page counter
(as well as other status parameters which are stored in the |.aux| files)
takes the same value after the conditional processing.
Otherwise the page numbers may take divergent values
depending on which part is compiled.

For example, a title page could be declared by:
%
\begin{center}
\begin{tabular}{l}
|\ifchilddoc\||else|\\
|\addtocounter{page}{-1}|\\
\textit{code for title page}\\
|\newpage|\\
|\||fi|
\end{tabular}
\end{center}
%
A banner page for the child documents can be generated by:
%
\begin{center}
\begin{tabular}{l}
|\ifchilddoc|\\
|\addtocounter{page}{-1}|\\
\textit{code for banner page}\\
|\newpage|\\
|\||fi|
\end{tabular}
\end{center}
%
Here one could write a message such as:
\begin{center}
|This is the part \childdocname{} of \childdocjob{}.|
\end{center}

%%%%%%%%%%%%%%%%%%%%%%%%%%%%%%%%%%%%%%%%%%%%%%%%%%%%%%%%%%%%%%%%%%%%%%%%%%%%%%%%
\subsection{Flags}
\label{sec:flags}

The package makes it easy to generate different versions
of the main or child documents.
To this end compilation flags can be defined
and assigned different default values.
They will be particularly useful in conjunction
with the forwarding mechanism described in \secref{sec:forward}.

For example, it may be useful to have a flag |\version|
which can be set to |draft| or |final|.
The document source will contain some conditional code
depending on the value of |\version|.
Suppose further, the flag should default to |final| for the main file
and to |draft| for child files
which is a natural assignment for editing the document.
This is achieved by placing the following code
in the preamble of the main document
(below the |\childdocmain| directive):
%
\begin{center}
\begin{tabular}{l}
|\ifchilddoc|\\
|\providecommand{\version}{draft}|\\
|\||else|\\
|\providecommand{\version}{final}|\\
|\||fi|
\end{tabular}
\end{center}
%
The definition by |\providecommand| makes sure
that previous definitions are not overwritten.
Further statements |\providecommand{\version}{...}|
can thus be added before the above code to override it.

For the main file, one might add a line
(between |\childdocmain| and the above block)
%
\begin{center}
|%\ifchilddoc\||else\providecommand{\version}{draft}\||fi|
\end{center}
%
which can be uncommented to produce a draft version.
Likewise one can add a line to the very top of a child file
(above the |\childdocof{|\textit{main}|}| directive)
%
\begin{center}
|%\providecommand{\version}{final}|
\end{center}
%
which can be uncommented to produce the final version of this child document.

%%%%%%%%%%%%%%%%%%%%%%%%%%%%%%%%%%%%%%%%%%%%%%%%%%%%%%%%%%%%%%%%%%%%%%%%%%%%%%%%
\subsection{Forwarding}
\label{sec:forward}

Different versions of the main or child documents
using compilation flags as described in \secref{sec:flags}
can be (permanently) stored in different files
for convenient compilation, viewing and distribution.
To this end, the package defines a command
to pass on compilation to a different file:

%%%%%%%%%%%%%%%%%%%%%%%%%%%%%%%%%%%%%%%%
\DescribeMacro{\childdocforward}
The command |\childdocforward| redirects processing to
another source file:
%
\begin{center}
\begin{tabular}{l}
|\input{childdoc.def}|\\
|\childdocforward[|\textit{main}|]{|\textit{dest}|}|\\
\end{tabular}
\end{center}
%
The argument \textit{dest} is the destination file
(without extension).
It should be the main file or one of the child files.
Note that further \textsf{childdoc} directives
such as |\childdocof| and |\childdocforward|
in the indicated file will be processed in this form.
The optional argument \textit{main}
passes on directly to the main file \textit{main}
while pretending to compile the child \textit{dest}.
This form behaves as if \textit{dest}
issues |\childdocof{|\textit{main}|}| right away,
and no further \textsf{childdoc} directives will be processed.

%%%%%%%%%%%%%%%%%%%%%%%%%%%%%%%%%%%%%%%%
\DescribeMacro{\...prefix}
In the alternative form |\childdocforwardprefix|,
%
\begin{center}
\begin{tabular}{l}
|\input{childdoc.def}|\\
|\childdocforwardprefix[|\textit{main}|]{|\textit{prefix}|}{|\textit{dest}|}|
\end{tabular}
\end{center}
%
the destination file is determined by a pattern
depending on the current file:
To make this work, the current file must be called
`{\textit{prefix}\hspace{0.2em}\textit{suffix}}'
with \textit{prefix} matching precisely the argument.
Processing is then passed on to the file
`{\textit{dest}\hspace{0.2em}\textit{suffix}}'.
Surely, the same effect is achieved by
directly specifying the
argument `{\textit{dest}\hspace{0.2em}\textit{suffix}}'
in the first form.
However, that requires to set up a different file
for each child. With the alternative form of the command
all these files can have exactly the same content
which simplifies setting them up and maintaining them.

For example, the following file |draft.tex|
with a compilation flag |\version| as described in \secref{sec:flags}
compiles the main document as a draft:
%
\begin{center}
\begin{tabular}{l}
|\def\version{draft}|\\
|\input{childdoc.def}|\\
|\childdocforward{|\textit{main}|}|
\end{tabular}
\end{center}
%
Likewise, the following files |final|\textit{nn}|.tex|
compile the final version of the child document
|child|\textit{nn}|.tex|:
%
\begin{center}
\begin{tabular}{l}
|\def\version{final}|\\
|\input{childdoc.def}|\\
|\childdocforwardprefix{final}{child}|
\end{tabular}
\end{center}
%

Note that when several versions of a main file and/or of each child file
are to be generated, it may be convenient to set up a |Makefile| or
shell script to automatise the process.

%%%%%%%%%%%%%%%%%%%%%%%%%%%%%%%%%%%%%%%%%%%%%%%%%%%%%%%%%%%%%%%%%%%%%%%%%%%%%%%%
\subsection{Command Line Processing}
\label{sec:commandline}

The effect of redirection files can also be achieved by invoking
the \LaTeX{} compiler with a more elaborate command line.
Most conveniently this should be done as part
of a shell script or a |Makefile|.

When using \textsf{childdoc} in the main file, the following
command lines effectively perform a redirection
(note that depending on the shell being used,
backslashes may have to be doubled: `|\|' $\to$ `|\\|'):
%
\begin{center}
|... -jobname "|\textit{target}|" |\\|"|[\textit{flags}]%
|\input{childdoc.def}\childdocforward[|\textit{main}|]{|\textit{dest}|}"|
\end{center}
%
Here \textit{target} is the name of the output file,
\textit{main} is the name of the main file
and \textit{dest} is the name of the main or child file to be processed
(all filenames without extensions).
The optional argument \textit{main} can be omitted
if \textit{main} matches \textit{dest}.
Optionally, compilation \textit{flags} can be defined via |\def| commands.
This command line makes the \TeX{} engine believe
it is compiling the file \textit{target}
whose content is specified as the latter parameter.
The provided code then forwards the processing to
\textit{main} or \textit{dest} as described in \secref{sec:forward}.

%%%%%%%%%%%%%%%%%%%%%%%%%%%%%%%%%%%%%%%%%%%%%%%%%%%%%%%%%%%%%%%%%%%%%%%%%%%%%%%%
\subsection{Include by Input}
\label{sec:input}

Including child documents by |\include| has some restrictions by design.
Most notably, the content of a child document always occupies
its own set of pages; pages cannot be shared between child documents.
Usually, this behaviour makes perfect sense
because each child document contain an essential part of the document.
However, in some situations it may be desirable to compose
a document from a collection of parts
without having mandatory page breaks between then.
For this case, the package
provides a mechanism to include parts
by |\input| which can also be processed individually.
However, by construction this mechanism
requires manual handling of the content to be output.

%%%%%%%%%%%%%%%%%%%%%%%%%%%%%%%%%%%%%%%%
\DescribeMacro{\ifchilddocmanual}
The main file should be prepared as usual, see \secref{sec:include}.
However, the document body must make a distinction
between processing of an individual part and of the main document, e.g.:
%
\begin{center}
\begin{tabular}{l}
|\ifchilddocmanual|\\
|\input{\childdocname}|\\
|\||else|\\
\textit{document body with }|\input{|\textit{part}|}|\\
|\||fi|
\end{tabular}
\end{center}
%
The conditional |\ifchilddocmanual| is true whenever
a part to be included by |\input| is being compiled,
and the name of the part is stored in |\childdocname|.

%%%%%%%%%%%%%%%%%%%%%%%%%%%%%%%%%%%%%%%%
\DescribeMacro{\childdocby}
Each part to be included by |\input| should start with:
%
\begin{center}
\begin{tabular}{l}
|\input{childdoc.def}|\\
|\childdocby{|\textit{main}|}|\\
\end{tabular}
\end{center}
%
The directive |\childdocby| is similar to |\childdocof|
described in \secref{sec:include},
but the subsequent selection of content must be done manually.
To that end, both |\ifchilddoc| and |\ifchilddocmanual|
will be true upon processing of a part,
and the name of the part is stored in |\childdocname|.
Note that |\jobname| will be set to the filename of the current part
so that each part receives an individual |.aux| file
that does not interfere with the |.aux| file(s) of the main document.
This behaviour can be altered by the alternative form
|\childdocby[*]{|\textit{main}|}| (with a non-empty optional argument)
which uses the |.aux| file of the main document
by setting |\jobname| to \textit{main}.

%%%%%%%%%%%%%%%%%%%%%%%%%%%%%%%%%%%%%%%%%%%%%%%%%%%%%%%%%%%%%%%%%%%%%%%%%%%%%%%%
\subsection{Driver Development}
\label{sec:driver}

The \textsf{childdoc} mechanism can also be use for the development
of definition files such as \LaTeX{} styles or classes.
This case differs from the above setup with multiple parts
included by |\include| in that no |\includeonly| should be invoked.
This can be achieved by starting the include file
(before |\ProvidesPackage|) with:
%
\begin{center}
\begin{tabular}{l}
|\input{childdoc.def}|\\
|\childdocforward{|\textit{main}|}|\\
\end{tabular}
\end{center}
%
or alternatively with:
%
\begin{center}
\begin{tabular}{l}
|\input{childdoc.def}|\\
|\childdocby{|\textit{main}|}|\\
\end{tabular}
\end{center}
%
Both forms have slightly different effects as described above.
The main file is prepared as usual, see \secref{sec:include}.

%%%%%%%%%%%%%%%%%%%%%%%%%%%%%%%%%%%%%%%%%%%%%%%%%%%%%%%%%%%%%%%%%%%%%%%%%%%%%%%%
\subsection{Legacy Detection}
\label{sec:detection}

The directive |\childdocmain| in the main file can detect
whether the complete document or merely a child is to be compiled
even without using the directive |\childdocof|.
This method is deprecated because it is less robust
and there is no compelling reason to use it;
it is merely provided for backward compatibility
and it may be removed in future versions.

If the detection mechanism is to be used,
it is mandatory to correctly specify
the filename of the main file as the argument of |\childdocmain|:
%
\begin{center}
\begin{tabular}{l}
|\input{childdoc.def}|\\
|\childdocmain{|\textit{main}|}|\\
\end{tabular}
\end{center}
%
If |\jobname| does not match the argument \textit{main} of |\childdocmain|,
it is assumed that |\jobname| points to the child file to be compiled.
When using |\childdocmain| with the main file specified as argument,
it suffices to start a child file
with just |\input{|\textit{main}|}|
without loading of the package and using |\childdocof|.
If instead all processing is done
with the appropriate \textsf{childdoc} directives,
the argument of \textit{main} of |\childdocmain| can be empty.

An alternative version of the command line processing described
in \secref{sec:commandline} using the detection mechanism reads:
%
\begin{center}
|... -jobname "|\textit{target}|" "|[\textit{flags}]%
[|\def\jobname{|\textit{dest}|}|]|\input{|\textit{main}|}"|
\end{center}

%%%%%%%%%%%%%%%%%%%%%%%%%%%%%%%%%%%%%%%%%%%%%%%%%%%%%%%%%%%%%%%%%%%%%%%%%%%%%%%%
\subsection{Manual Code}
\label{sec:manual}

In case one cannot be certain whether the definitions file |childdoc.def|
is installed on the target \TeX{} distribution
and one prefers not to ship it,
it is conceivable to paste a few relevant commands into the sources.

To that end, drop all statements |\input{childdoc.def}|
and perform the replacements as outlined below.
Instead of |\childdocmain{|\textit{main}|}| add the following code
to the top of the main file:
%
\begin{center}
\begin{tabular}{l}
|\||ifdefined\childdocname\endinput\||fi\newif\ifchilddoc|\\
|\edef\childdocname{\scantokens\expandafter{\jobname\noexpand}}|\\
|\def\childdocmain{|\textit{main}|}\||ifx\childdocmain\childdocname\||else|\\
|\childdoctrue\includeonly{\childdocname}\let\jobname\childdocmain\||fi|\\
\end{tabular}
\end{center}
%
Instead of |\childdocof{|\textit{main}|}| just include the main file
at the top of each child file:
%
\begin{center}
|\input{|\textit{main}|}|
\end{center}
%
A simple redirection |\childdocforward{|\textit{dest}|}| is achieved by:
%
\begin{center}
|\def\jobname{|\textit{dest}|}\input{\jobname}|
\end{center}
%
The redirection with prefix
|\childdocforwardprefix[|\textit{prefix}|]{|\textit{dest}|}|
is accomplished by:
%
\begin{center}
\begin{tabular}{l}
|{\edef\jobname{\scantokens\expandafter{\jobname\noexpand}}|\\
|\def\redirectjob |\textit{prefix}|#1~~~{\gdef\jobname{|\textit{dest}|#1}}|\\
|\expandafter\redirectjob\jobname~~~}\input{\jobname}|
\end{tabular}
\end{center}

In an alternative approach,
child documents can be compiled by a specific command line
without additional code or specific definitions:
%
\begin{center}
|... -jobname "|\textit{target}|" "|[\textit{flags}]%
|\includeonly{|\textit{dest}|}\input{|\textit{main}|}"|
\end{center}
%

%%%%%%%%%%%%%%%%%%%%%%%%%%%%%%%%%%%%%%%%%%%%%%%%%%%%%%%%%%%%%%%%%%%%%%%%%%%%%%%%
%%%%%%%%%%%%%%%%%%%%%%%%%%%%%%%%%%%%%%%%%%%%%%%%%%%%%%%%%%%%%%%%%%%%%%%%%%%%%%%%
\section{Information}

%%%%%%%%%%%%%%%%%%%%%%%%%%%%%%%%%%%%%%%%%%%%%%%%%%%%%%%%%%%%%%%%%%%%%%%%%%%%%%%%
\subsection{Copyright}

Copyright \copyright{} 2017--2018 Niklas Beisert

This work may be distributed and/or modified under the
conditions of the \LaTeX{} Project Public License, either version 1.3
of this license or (at your option) any later version.
The latest version of this license is in
  \url{http://www.latex-project.org/lppl.txt}
and version 1.3 or later is part of all distributions of \LaTeX{}
version 2005/12/01 or later.

This work has the LPPL maintenance status `maintained'.

The Current Maintainer of this work is Niklas Beisert.

This work consists of the files |README.txt|, |childdoc.ins| and |childdoc.dtx|
as well as the derived files |childdoc.def|, |cdocsamp.tex|
with |cdocsch1.tex|, |cdocsch2.tex|, |cdocspt3.tex|, |cdocspt4.tex|,
|cdocsdrf.tex|, |cdocsfn1.tex|, |cdocsfn2.tex|
as well as |childdoc.pdf|.

%%%%%%%%%%%%%%%%%%%%%%%%%%%%%%%%%%%%%%%%%%%%%%%%%%%%%%%%%%%%%%%%%%%%%%%%%%%%%%%%
\subsection{Files and Installation}

The package consists of the files:
%
\begin{center}
\begin{tabular}{ll}
    |README.txt|   & readme file \\
    |childdoc.ins| & installation file \\
    |childdoc.dtx| & source file \\
    |childdoc.def| & definition file \\
    |cdocsamp.tex| & sample main file \\
    |cdocsch1.tex| & sample include file \\
    |cdocsch2.tex| & sample include file \\
    |cdocspt3.tex| & sample part file \\
    |cdocspt4.tex| & sample part file \\
    |cdocsdrf.tex| & sample redirection file \\
    |cdocsfn1.tex| & sample redirection file \\
    |cdocsfn2.tex| & sample redirection file \\
    |childdoc.pdf| & manual
\end{tabular}
\end{center}
%
The distribution consists of the files
|README.txt|, |childdoc.ins| and |childdoc.dtx|.
%
\begin{itemize}
\item
Run (pdf)\LaTeX{} on |childdoc.dtx|
to compile the manual |childdoc.pdf| (this file).
\item
Run \LaTeX{} on |childdoc.ins| to create the definitions file |childdoc.def|
and the sample |cdocsamp.tex| with include files
|cdocsch1.tex|, |cdocsch2.tex|, |cdocspt3.tex|, |cdocspt4.tex|,
|cdocsdrf.tex|, |cdocsfn1.tex|, |cdocsfn2.tex|.
Then copy the file |childdoc.def| to an appropriate directory of your \LaTeX{}
distribution, e.g.\ \textit{texmf-root}|/tex/latex/childdoc|.
\end{itemize}

%%%%%%%%%%%%%%%%%%%%%%%%%%%%%%%%%%%%%%%%%%%%%%%%%%%%%%%%%%%%%%%%%%%%%%%%%%%%%%%%
\subsection{Related CTAN Packages}

There are several other packages which offer a similar functionality:
%
\begin{itemize}
\item
The packages
\href{http://ctan.org/pkg/docmute}{\textsf{docmute}},
\href{http://ctan.org/pkg/includex}{\textsf{includex}} and
\href{http://ctan.org/pkg/standalone}{\textsf{standalone}}
provide commands to include only the document body of
a child file thus allowing both files to be compiled individually.
\item
The packages \href{http://ctan.org/pkg/subdocs}{\textsf{subdocs}}
and \href{http://ctan.org/pkg/subfiles}{\textsf{subfiles}}
provide structures in which the main and child documents can be
encapsulated and allowing them to be compiled individually.
The inclusion mechanism is different from the conventional |\include|.
\item
The package \href{http://ctan.org/pkg/combine}{\textsf{combine}}
is an elaborate solution to combine several documents into one.
\end{itemize}
%
See also the CTAN topic \href{http://ctan.org/topic/subdocs}{\textsf{subdocs}}
for further related packages.
The present package differs from the above solutions in that
a document structure constructed with the conventional |\include| mechanism
just needs two extra commands at the top of every file
such that all constituent files can be compiled individually.

%%%%%%%%%%%%%%%%%%%%%%%%%%%%%%%%%%%%%%%%%%%%%%%%%%%%%%%%%%%%%%%%%%%%%%%%%%%%%%%%
%\subsection{Feature Suggestions}
%
%The following is a list of features which may be useful for future
%versions of this package:
%%
%\begin{itemize}
%\item
%\ldots
%\end{itemize}

%%%%%%%%%%%%%%%%%%%%%%%%%%%%%%%%%%%%%%%%%%%%%%%%%%%%%%%%%%%%%%%%%%%%%%%%%%%%%%%%
\subsection{Revision History}

%%%%%%%%%%%%%%%%%%%%%%%%%%%%%%%%%%%%%%%%
\paragraph{v2.0:} 2018/12/30

\begin{itemize}
\item
immediate forward processing
\item
added |\childdocby| mechanism
\item
manual restructured
\end{itemize}

%%%%%%%%%%%%%%%%%%%%%%%%%%%%%%%%%%%%%%%%
\paragraph{v1.6:} 2018/01/17

\begin{itemize}
\item
application for development of include files
\item
corrections to manual
\end{itemize}

%%%%%%%%%%%%%%%%%%%%%%%%%%%%%%%%%%%%%%%%
\paragraph{v1.5:} 2017/05/21

\begin{itemize}
\item
more complete structuring introduced
\item
|\childdocof| introduced
\item
|\childdoc| renamed to |\childdocmain|
\item
|\childredirect| renamed to |\childdocforward| and |\childdocforwardprefix|
and functionality expanded
\end{itemize}

%%%%%%%%%%%%%%%%%%%%%%%%%%%%%%%%%%%%%%%%
\paragraph{v1.0:} 2017/04/27

\begin{itemize}
\item
manual and install package
\item
first version published on CTAN
\end{itemize}

%%%%%%%%%%%%%%%%%%%%%%%%%%%%%%%%%%%%%%%%
\paragraph{v0.6:} 2017/04/26

\begin{itemize}
\item
redirection mechanism added
\end{itemize}

%%%%%%%%%%%%%%%%%%%%%%%%%%%%%%%%%%%%%%%%
\paragraph{v0.5:} 2017/04/26

\begin{itemize}
\item
functionality in definition file
\end{itemize}


%%%%%%%%%%%%%%%%%%%%%%%%%%%%%%%%%%%%%%%%%%%%%%%%%%%%%%%%%%%%%%%%%%%%%%%%%%%%%%%%
%%%%%%%%%%%%%%%%%%%%%%%%%%%%%%%%%%%%%%%%%%%%%%%%%%%%%%%%%%%%%%%%%%%%%%%%%%%%%%%%
%%%%%%%%%%%%%%%%%%%%%%%%%%%%%%%%%%%%%%%%%%%%%%%%%%%%%%%%%%%%%%%%%%%%%%%%%%%%%%%%
\appendix

\settowidth\MacroIndent{\rmfamily\scriptsize 000\ }

 \DocInput{childdoc.dtx}

\end{document}
%</driver>
% \fi
%
% %%%%%%%%%%%%%%%%%%%%%%%%%%%%%%%%%%%%%%%%%%%%%%%%%%%%%%%%%%%%%%%%%%%%%%%%%%%%%%
% %%%%%%%%%%%%%%%%%%%%%%%%%%%%%%%%%%%%%%%%%%%%%%%%%%%%%%%%%%%%%%%%%%%%%%%%%%%%%%
% \section{Sample}
%\iffalse
%<*samplemain>
%\fi
%
% The following presents a sample document
% with two chapters, two parts, a title page,
% a compile flag as well as three forwarding files to set the flag.
% It consists of eight |.tex| files:
% \begin{center}
% \begin{tabular}{ll}
% |cdocsamp.tex|&main file\\
% |cdocsch1.tex|&include file for chapter 1\\
% |cdocsch2.tex|&include file for chapter 2\\
% |cdocspt3.tex|&include file for part 3\\
% |cdocspt4.tex|&include file for part 4\\
% |cdocsdrf.tex|&forwarding file for main file in draft mode\\
% |cdocsfi1.tex|&forwarding file for final version of chapter 1\\
% |cdocsfi2.tex|&forwarding file for final version of chapter 2\\
% \end{tabular}
% \end{center}
% Each of the eight files can be compiled directly by the \LaTeX{} compiler.
%
% %%%%%%%%%%%%%%%%%%%%%%%%%%%%%%%%%%%%%%
% \paragraph{Main File.}
%
% The main file is called |cdocsamp.tex|.
%
% Load the \textsf{childdoc} definitions and
% declare the filename for the main document:
%    \begin{macrocode}
\input{childdoc.def}
\childdocmain{}
%    \end{macrocode}

% Optional override for |\version| flag:
%    \begin{macrocode}
%%\ifchilddoc\else\providecommand{\version}{draft}\fi
%    \end{macrocode}

% Define the default values for the |\version| flag
% (|final| for the main file and |draft| for childs):
%    \begin{macrocode}
\ifchilddoc
\providecommand{\version}{draft}
\else
\providecommand{\version}{final}
\fi
%    \end{macrocode}

% Load the standard document class:
%    \begin{macrocode}
\documentclass[12pt]{article}
%    \end{macrocode}

% Start the document body:
%    \begin{macrocode}
\begin{document}
%    \end{macrocode}

% Declare a title page.
% Print title, part of document being processed and version flag:
%    \begin{macrocode}
\addtocounter{page}{-1}
\begin{center}
{\LARGE\bfseries{}childdoc example\par}
\vspace{1cm}
\ifchilddoc
\ifchilddocmanual part\else chapter\fi:
`\childdocname' of `\childdocjob'\par
\else
main document: `\childdocjob'\par
\fi
version: \version\par
\end{center}
\newpage
%    \end{macrocode}

% Manually include selected file,
% otherwise process as usual:
%    \begin{macrocode}
\ifchilddocmanual
\section*{part `\childdocname'}
\input{\childdocname}
\else
%    \end{macrocode}

% Include the two chapters:
%    \begin{macrocode}
\include{cdocsch1}
\include{cdocsch2}
%    \end{macrocode}

% Include the two parts unless only chapters should be displayed:
%    \begin{macrocode}
\ifchilddoc\else
\section{part three}
\input{cdocspt3}
\section{part four}
\input{cdocspt4}
\fi
%    \end{macrocode}

% Process as usual until here:
%    \begin{macrocode}
\fi
%    \end{macrocode}

% End of document body:
%    \begin{macrocode}
\end{document}
%    \end{macrocode}
%\iffalse
%</samplemain>
%\fi
%
% %%%%%%%%%%%%%%%%%%%%%%%%%%%%%%%%%%%%%%
% \paragraph{Chapter Include Files.}
%
% The include files are called |cdocsch1.tex| and |cdocsch2.tex|.
%
%\iffalse
%<*samplechap1|samplechap2>
%\fi

% Optional override for |\version| flag:
%    \begin{macrocode}
%%\providecommand{\version}{final}
%    \end{macrocode}

% Include the main document:
%    \begin{macrocode}
\input{childdoc.def}
\childdocof{cdocsamp}
%    \end{macrocode}

%\iffalse
%</samplechap1|samplechap2>
%\fi
%
%\iffalse
%<*samplechap1>
%\fi
% Some text for chapter 1:
%    \begin{macrocode}
\section{one}
some text in chapter one
%    \end{macrocode}

%\iffalse
%</samplechap1>
%\fi
% Some text for chapter 2:
%\iffalse
%<*samplechap2>
%\fi
%    \begin{macrocode}
\section{two}
more text in chapter two
%    \end{macrocode}

%\iffalse
%</samplechap2>
%\fi
%
% %%%%%%%%%%%%%%%%%%%%%%%%%%%%%%%%%%%%%%
% \paragraph{Part Include Files.}
%
% The include files are called |cdocspt3.tex| and |cdocspt4.tex|.
%
%\iffalse
%<*samplepart3|samplepart4>
%\fi

% Optional override for |\version| flag:
%    \begin{macrocode}
%%\providecommand{\version}{final}
%    \end{macrocode}

% Include the main document:
%    \begin{macrocode}
\input{childdoc.def}
\childdocby{cdocsamp}
%    \end{macrocode}

%\iffalse
%</samplepart3|samplepart4>
%\fi
%
%\iffalse
%<*samplepart3>
%\fi
% Some text for part 3:
%    \begin{macrocode}
some text in part three
%    \end{macrocode}

%\iffalse
%</samplepart3>
%\fi
% Some text for part 4:
%\iffalse
%<*samplepart4>
%\fi
%    \begin{macrocode}
more text in part four
%    \end{macrocode}

%\iffalse
%</samplepart4>
%\fi
%
% %%%%%%%%%%%%%%%%%%%%%%%%%%%%%%%%%%%%%%
% \paragraph{Forwarding for a Complete Draft.}
%
% The following forwarding file |cdocsdrf.tex|
% compiles the main document in draft mode:
%\iffalse
%<*sampledraft>
%\fi
%    \begin{macrocode}
\def\version{draft}
\input{childdoc.def}
\childdocforward{cdocsamp}
%    \end{macrocode}

%\iffalse
%</sampledraft>
%\fi
%
% %%%%%%%%%%%%%%%%%%%%%%%%%%%%%%%%%%%%%%
% \paragraph{Forwarding for Final Version of the Chapters.}
%
% The following forwarding files |cdocsfn1.tex| and |cdocsfn2.tex|
% (with identical content)
% compile the final versions of the child documents
% |cdocsch1.tex| and |cdocsch2.tex|, respectively:
%\iffalse
%<*samplefinal>
%\fi
%    \begin{macrocode}
\def\version{final}
\input{childdoc.def}
\childdocforwardprefix[cdocsamp]{cdocsfn}{cdocsch}
%    \end{macrocode}

%\iffalse
%</samplefinal>
%\fi
%
% %%%%%%%%%%%%%%%%%%%%%%%%%%%%%%%%%%%%%%
% \paragraph{Command Line Processing.}
%
% The following three command lines generate the output files
% |cdocscld|, |cdocscl1| and |cdocscl2|
% which should be identical to
% |cdocsdrf|, |cdocsch1| and |cdocsfn2|, respectively:
% \begin{center}
% \begin{tabular}{l}
% |latex -jobname cdocscld \|\\
% |  "\def\version{draft}\input{childdoc.def}\childdocforward{cdocsamp}"|\\
% |latex -jobname cdocscl1 \|\\
% |  "\input{childdoc.def}\childdocforward[cdocsamp]{cdocsch1}"|\\
% |latex -jobname cdocscl2 \|\\
% |  "\def\version{final}\input{childdoc.def}\childdocforward{cdocsch2}"|
% \end{tabular}
% \end{center}
% Note that the trailing backslash on each first line
% merely continues the input to the second line
% (for convenient cut ant paste).
% Furthermore, the command |latex| can be replaced by any
% of its alternative versions such as |pdflatex|.
%
% %%%%%%%%%%%%%%%%%%%%%%%%%%%%%%%%%%%%%%%%%%%%%%%%%%%%%%%%%%%%%%%%%%%%%%%%%%%%%%
% %%%%%%%%%%%%%%%%%%%%%%%%%%%%%%%%%%%%%%%%%%%%%%%%%%%%%%%%%%%%%%%%%%%%%%%%%%%%%%
% \section{Implementation}
%\iffalse
%<*package>
%\fi
%
% This section describes the definitions file |childdoc.def|.

% The definitions cannot be loaded using |\usepackage| or |\RequirePackage|
% which has a mechanism to prevent loading a style file more than once.
% When loading the definitions by means of |\input|
% multiple instances have to be prevented manually:
%\iffalse
%This code needs to be before the `\ProvidesFile' directive
%which is defined at the beginning of this file.
%Therefore it is also placed there and commented out here.
%</package>
%<*discard>
%\fi
%    \begin{macrocode}
\ifdefined\childdocmain\endinput\fi
%    \end{macrocode}
%\iffalse
%</discard>
%<*package>
%\fi
%
% \macro{\ifchilddoc}
% \macro{\ifchilddocmanual}
% The conditional |\ifchilddoc| tells whether a
% child (true) or main (false) document is being compiled.
% The conditional |\ifchilddocmanual| tells whether
% the |\includeonly| mechanism is used (false) or
% the selection of child files must be performed manually (true).
% The definitions initialise to false:
%    \begin{macrocode}
\newif\ifchilddoc
\newif\ifchilddocmanual
%    \end{macrocode}

% \macro{\childdocname}
% \macro{\childdocjob}
% The macro |\childdocname| stores the name of the main document
% to be compiled. The macro |\childdocjob| stores the name of
% the document on which the \LaTeX{} compiler was originally invoked.
% The content of |\jobname| cannot be compared
% to filenames specified in the source due to different catcodes.
% The following code rescans |\jobname|, stores the result
% in |\childdocname| and saves a copy in |\childdocjob|:
%    \begin{macrocode}
\edef\childdocname{\scantokens\expandafter{\jobname\noexpand}}
\let\childdocjob\childdocname
%    \end{macrocode}

% \macro{\childdocdisable}
% The macro |\childdocdisable| prevents the main file
% from being processed more than once.
% At this stage, the main document command |\childdocmain|
% is assumed to be called once again where it should do nothing.
% Any subsequent call to it should prevent
% a secondary processing of the main document
% It overwrites the forwarding commands
% |\childdocof| and |\childdocforward|
% with empty macros to prevent further inclusions of the main document:
%    \begin{macrocode}
\newcommand{\childdocdisable}
{
  \renewcommand{\childdocmain}[1]{\renewcommand{\childdocmain}[1]{\endinput}}
  \renewcommand{\childdocof}[1]{}
  \renewcommand{\childdocby}[2][]{}
  \renewcommand{\childdocforward}[2][]{}
  \renewcommand{\childdocdisable}{}
}
%    \end{macrocode}

% \macro{\childdocmain}
% The macro |\childdocmain| is to be called at the top of the main file
% with nothing or the main filename (without extension) as argument.
% First, it breaks loops.
% If the argument is not empty and does not match |\childdocname|
% (which is set by the first inclusion of |childdoc.def|),
% |\ifchilddoc| is set to true, |\includeonly| is applied to the child file
% and |\jobname| is set to the main file
% (for proper handling of |.aux| files):
%    \begin{macrocode}
\newcommand{\childdocmain}[1]
{
  \childdocdisable\childdocmain{}
  \if?#1?\else
    \begingroup
      \def\childdoctmp{#1}
      \ifx\childdoctmp\childdocname
        \def\childdoctmp{}
      \else
        \def\childdoctmp
        {
          \childdoctrue
          \includeonly{\childdocname}
          \def\childdocjob{#1}
          \def\jobname{#1}
        }
      \fi
      \expandafter
    \endgroup
    \childdoctmp
  \fi
}
%    \end{macrocode}

% \macro{\childdocof}
% The command |\childdocof| redirects
% compilation to the main file |#1|.
%    \begin{macrocode}
\newcommand{\childdocof}[1]
{
  \childdocdisable
  \childdoctrue
  \includeonly{\childdocname}
  \def\jobname{#1}
  \def\childdocjob{#1}
  \input{#1}
}
%    \end{macrocode}

% \macro{\childdocby}
% The command |\childdocby| ....
%    \begin{macrocode}
\newcommand{\childdocby}[2][]
{
  \childdocdisable
  \childdoctrue
  \childdocmanualtrue
  \if?#1?\else
    \def\jobname{#2}
  \fi
  \def\childdocjob{#2}
  \input{#2}
  \endinput
}
%    \end{macrocode}

% \macro{\childdocforward}
% The command |\childdocforward| redirects
% compilation to the main file or
% (if the optional argument is given) a child file.
% Parameters are set as if the main file
% or a child file starting with |\childdocof| was compiled.
% Then compilation is handed over to the main file:
%    \begin{macrocode}
\newcommand{\childdocforward}[2][]
{
  \begingroup
    \if?#1?
      \def\childdoctmp
      {
        \def\childdocname{#2}
        \def\childdocjob{#2}
        \def\jobname{#2}
        \input{#2}
        \endinput
      }
    \else
      \def\childdoctmp
      {
        \childdocdisable
        \def\childdocname{#2}
        \childdoctrue
        \includeonly{#2}
        \def\childdocjob{#1}
        \def\jobname{#1}
        \input{#1}
        \endinput
      }
    \fi
    \expandafter
  \endgroup
  \childdoctmp
}
%    \end{macrocode}

% \macro{\childdocforwardprefix}
% The command |\childdocforwardprefix| redirects
% compilation to the main or a child file by means of a pattern.
% The prefix |#1| in the current filename is replaced by |#2|
% and the suffix of the current filename is kept
% (it is assumed that the filename does not contain the substring `|~~~|'
% which is used as a delimiter).
% Compilation is handed over to the new file by |\childdocforward|:
%    \begin{macrocode}
\newcommand{\childdocforwardprefix}[3][]
{
  \begingroup
    \def\childdocextract #2##1~~~{\def\childdoctmp{\childdocforward[#1]{#3##1}}}
    \expandafter\childdocextract\childdocname~~~
    \expandafter
  \endgroup
  \childdoctmp
}
%    \end{macrocode}

% \macro{\childdoc}
% The deprecated macro |\childdoc| is a legacy version of |\childdocmain|:
%    \begin{macrocode}
\newcommand{\childdoc}{\childdocmain}
%    \end{macrocode}

% \macro{\childdocredirect}
% The deprecated macro |\childdocredirect| is a legacy version
% of |\childdocforward| and |\childdocforwardprefix|:
%    \begin{macrocode}
\newcommand{\childdocredirect}[2][]
{
  \begingroup
    \if?#1?
      \def\childdoctmp{\childdocforward{#2}}
    \else
      \def\childdoctmp{\childdocforwardprefix{#1}{#2}}
    \fi
    \expandafter
  \endgroup
  \childdoctmp
}
%    \end{macrocode}

%\iffalse
%</package>
%\fi
%
\endinput
|\\
|\childdocforward{|\textit{main}|}|\\
\end{tabular}
\end{center}
%
or alternatively with:
%
\begin{center}
\begin{tabular}{l}
|% \iffalse
%
% childdoc.dtx Copyright (C) 2017-2018 Niklas Beisert
%
% This work may be distributed and/or modified under the
% conditions of the LaTeX Project Public License, either version 1.3
% of this license or (at your option) any later version.
% The latest version of this license is in
%   http://www.latex-project.org/lppl.txt
% and version 1.3 or later is part of all distributions of LaTeX
% version 2005/12/01 or later.
%
% This work has the LPPL maintenance status `maintained'.
%
% The Current Maintainer of this work is Niklas Beisert.
%
% This work consists of the files childdoc.dtx and childdoc.ins
% and the derived files childdoc.def and cdocsamp.tex with
% cdocsch1.tex, cdocsch2.tex, cdocsdrf.tex, cdocsfn1.tex, cdocsfn2.tex.
%
%<package>\ifdefined\childdocmain\endinput\fi
%<package>\ProvidesFile{childdoc.def}[2018/12/30 v2.0 child document driver]
%<samplemain>\ProvidesFile{cdocsamp.tex}[2018/12/30 v2.0 sample for childdoc]
%<*driver>
%\ProvidesFile{childdoc.drv}[2018/12/30 v2.0 childdoc reference manual file]
\PassOptionsToClass{10pt,a4paper}{article}
\documentclass{ltxdoc}

\usepackage[margin=35mm]{geometry}
\usepackage{hyperref}
\usepackage{hyperxmp}
\usepackage[usenames]{color}

\hypersetup{colorlinks=true}
\hypersetup{pdfstartview=FitH}
\hypersetup{pdfpagemode=UseNone}
\hypersetup{pdfsource={}}
\hypersetup{pdflang={en-UK}}
\hypersetup{pdfcopyright={Copyright 2017-2018 Niklas Beisert.
  This work may be distributed and/or modified under the
  conditions of the LaTeX Project Public License, either version 1.3
  of this license or (at your option) any later version.}}
\hypersetup{pdflicenseurl={http://www.latex-project.org/lppl.txt}}
\hypersetup{pdfcontactaddress={ETH Zurich, ITP, HIT K,
  Wolfgang-Pauli-Strasse 27}}
\hypersetup{pdfcontactpostcode={8093}}
\hypersetup{pdfcontactcity={Zurich}}
\hypersetup{pdfcontactcountry={Switzerland}}
\hypersetup{pdfcontactemail={nbeisert@itp.phys.ethz.ch}}
\hypersetup{pdfcontacturl={http://people.phys.ethz.ch/\xmptilde nbeisert/}}

\newcommand{\secref}[1]{\hyperref[#1]{section \ref*{#1}}}

\parskip1ex
\parindent0pt
\let\olditemize\itemize
\def\itemize{\olditemize\parskip0pt}

\begin{document}

\title{The \textsf{childdoc} Package}
\hypersetup{pdftitle={The childdoc Package}}
\author{Niklas Beisert\\[2ex]
  Institut f\"ur Theoretische Physik\\
  Eidgen\"ossische Technische Hochschule Z\"urich\\
  Wolfgang-Pauli-Strasse 27, 8093 Z\"urich, Switzerland\\[1ex]
  \href{mailto:nbeisert@itp.phys.ethz.ch}
  {\texttt{nbeisert@itp.phys.ethz.ch}}}
\hypersetup{pdfauthor={Niklas Beisert}}
\hypersetup{pdfsubject={Manual for the LaTeX2e Package childdoc}}
\date{30 December 2018, \textsf{v2.0}}
\maketitle

\begin{abstract}\noindent
\textsf{childdoc} is a \LaTeXe{} package
that enables the direct compilation
of document sections included by |\include|
to individual files.
\end{abstract}

\begingroup
\parskip0ex
\tableofcontents
\endgroup

%%%%%%%%%%%%%%%%%%%%%%%%%%%%%%%%%%%%%%%%%%%%%%%%%%%%%%%%%%%%%%%%%%%%%%%%%%%%%%%%
%%%%%%%%%%%%%%%%%%%%%%%%%%%%%%%%%%%%%%%%%%%%%%%%%%%%%%%%%%%%%%%%%%%%%%%%%%%%%%%%
\section{Introduction}

\LaTeX{} provides a mechanism to structure a large document (such as a book)
into a main file and several child files (containing the chapters)
using the |\include| command.
This mechanism is beneficial for documents
which span hundreds of pages in order to
make the source file(s) more manageable.
Moreover, compilation can be restricted to
selected child files by means of the |\includeonly| command.
The latter feature can be used to reduce the compilation time while editing
(this was significantly more useful in the earlier days of \LaTeX{})
or to generate a smaller document which is easier to navigate.
Another application of |\includeonly| is to generate
documents consisting of selected parts of the complete document.

However, there are a few drawbacks of the plain |\include| mechanism:
\begin{itemize}
\item
The child files cannot be compiled on their own,
they can only be compiled via the main file.
A naive editing environment
(such as a text editor with an option
to have the current file processed by \LaTeX)
may require one to switch to the main file before compiling;
attempting to compile the child file produces errors.
\item
The main file must be modified (each time)
to adjust the |\includeonly| command
to the present needs. This easily leaves the main file in a messy state.
\item
The generated document will always carry the filename
of the main document. This is inconvenient if
several child files are to be compiled and
to be kept for distribution.
\end{itemize}

The present package provides a simple interface
to make child files individually compilable by \LaTeX{}.
Compiling a child file then has the same effect as compiling
the main file with an |\includeonly| command
to select the appropriate child.
Moreover the generated document will carry the name of the child
rather than the main file.
This resolves all three above issues.

This feature is meant to make the editing of books,
thesis documents and lecture notes somewhat more convenient.
However, the package can also be used efficiently for
composing a series of documents (such as exercise sheets)
which are typically distributed individually.
It then assists the author in generating the individual documents
(potentially in different versions)
as well as a document containing the collected series.
Another application is in developing style files
or other kinds of included material
where compilation of the style file could redirect
to a sample or test file.

%%%%%%%%%%%%%%%%%%%%%%%%%%%%%%%%%%%%%%%%%%%%%%%%%%%%%%%%%%%%%%%%%%%%%%%%%%%%%%%%
%%%%%%%%%%%%%%%%%%%%%%%%%%%%%%%%%%%%%%%%%%%%%%%%%%%%%%%%%%%%%%%%%%%%%%%%%%%%%%%%
\section{Usage}

First of all, the package \textsf{childdoc} is \emph{not} a standard
\LaTeXe{} |.sty| style file! Therefore it needs to be invoked in
a non-standard way.

%%%%%%%%%%%%%%%%%%%%%%%%%%%%%%%%%%%%%%%%%%%%%%%%%%%%%%%%%%%%%%%%%%%%%%%%%%%%%%%%
\subsection{Included Files}
\label{sec:include}

%%%%%%%%%%%%%%%%%%%%%%%%%%%%%%%%%%%%%%%%
\DescribeMacro{\childdocmain}
To use the package, add the commands
\begin{center}
\begin{tabular}{l}
|\input{childdoc.def}|\\
|\childdocmain{}|\\
\end{tabular}
\end{center}
at the very top of the main \LaTeX{} file,
in particular \emph{before} the |\documentclass| statement!
The argument of |\childdocmain| should be left empty
(but it must be present).

%%%%%%%%%%%%%%%%%%%%%%%%%%%%%%%%%%%%%%%%
\DescribeMacro{\childdocof}
Furthermore, add the commands
\begin{center}
\begin{tabular}{l}
|\input{childdoc.def}|\\
|\childdocof{|\textit{main}|}|\\
\end{tabular}
\end{center}
at the top of every child file \textit{child}
which is included by |\include{|\textit{child}|}|
from within the main file
(or at least for those files to be compiled individually).
The argument \textit{main} must be the filename of the main file.

There are a couple of
considerations in setting up the main and child documents:

%%%%%%%%%%%%%%%%%%%%%%%%%%%%%%%%%%%%%%%%
\paragraph{Restrictions.}

Please note the following restrictions:
\begin{itemize}
\item
|\childdocmain| must be called with one argument \textit{main}
to ensure compatibility with earlier version of the package.
It must either be empty (|\childdocmain{}|)
or precisely match the filename of the main file in which it is specified.
See \secref{sec:detection} for further information.
\item
The filename \textit{main} must be specified without the |.tex| extension.
\item
The filename \textit{main} is case sensitive
(even in case-insensitive file systems)
due to internal string comparison.
\item
The argument \textit{main} should be fully expanded, it cannot be a macro.
\item
Subdirectories and special characters should be avoided in filenames.
\item
The command |\childdocmain{|\textit{main}|}| must be followed by a whitespace.
It should not be followed immediately by another command
or by a comment mark `|%|'.
This is because the \TeX{} parser reads the token immediately following
the argument of |\childdocmain| and puts it
at the beginning of every child section;
however, a white\-space is ignored.
\end{itemize}

%%%%%%%%%%%%%%%%%%%%%%%%%%%%%%%%%%%%%%%%
\paragraph{Content of Main File.}

It is advisable to place all content in the child files included by |\include|.
Any output contained in the main file will appear in all child documents
unless suppressed manually;
it cannot be suppressed automatically by the |\includeonly| directive
and thus should normally be avoided.
A method to include some content in the main file
by means of conditional processing is described in \secref{sec:conditional}.

%%%%%%%%%%%%%%%%%%%%%%%%%%%%%%%%%%%%%%%%
\paragraph{Page Numbering.}

When only a part of the document is compiled,
the appropriate numbering of pages
(as well as other status parameters)
is determined from the |.aux| files.
The latter contain information from previous passes.
However this information needs to propagate through
all intermediate child documents.
Therefore the page numbering in child documents may well
be inconsistent until the complete document is compiled at least once.

A useful (if unconventional) way to always ensure a consistent
page numbering is to restart the numbering in each child document
and denote the pages by `\textit{child}|.|\textit{page}'
where \textit{child} represents the chapter/section number of the child file.
This can be achieved by the command
|\numberwithin{page}{|\textit{child}|}|
of the \textsf{amsmath} package
where \textit{child} can be |chapter| or |section|
depending on the chosen structuring.
Alternatively, one can modify the macro |\thepage| appropriately
and reset the counter |page| at the start of each child file.

%%%%%%%%%%%%%%%%%%%%%%%%%%%%%%%%%%%%%%%%%%%%%%%%%%%%%%%%%%%%%%%%%%%%%%%%%%%%%%%%
\subsection{Conditional Processing}
\label{sec:conditional}

The package provides a mechanism to compile different versions
of a document. To customise the versions further some conditional processing
can come in handy to distinguish which version is being compiled.
The package provides two macros to describe the compilation context:

%%%%%%%%%%%%%%%%%%%%%%%%%%%%%%%%%%%%%%%%
\DescribeMacro{\ifchilddoc}
The conditional |\ifchilddoc| distinguishes between the compilation of
child documents and the main document:
%
\begin{center}
|\ifchilddoc |\textit{child-code}| |[|\||else |\textit{main-code}]| \||fi|
\end{center}

%%%%%%%%%%%%%%%%%%%%%%%%%%%%%%%%%%%%%%%%
\DescribeMacro{\childdocname}
\DescribeMacro{\childdocjob}
The macro |\childdocname| contains the filename (without extension)
of the main or child file being processed.
Note that |\childdocjob| will always contain the name of the main file.

%%%%%%%%%%%%%%%%%%%%%%%%%%%%%%%%%%%%%%%%
\paragraph{Title Page.}

Conditional processing can be used to include a title or banner page
in the main document when proper precautions are taken.
Importantly, the code in the main file should ensure that the page counter
(as well as other status parameters which are stored in the |.aux| files)
takes the same value after the conditional processing.
Otherwise the page numbers may take divergent values
depending on which part is compiled.

For example, a title page could be declared by:
%
\begin{center}
\begin{tabular}{l}
|\ifchilddoc\||else|\\
|\addtocounter{page}{-1}|\\
\textit{code for title page}\\
|\newpage|\\
|\||fi|
\end{tabular}
\end{center}
%
A banner page for the child documents can be generated by:
%
\begin{center}
\begin{tabular}{l}
|\ifchilddoc|\\
|\addtocounter{page}{-1}|\\
\textit{code for banner page}\\
|\newpage|\\
|\||fi|
\end{tabular}
\end{center}
%
Here one could write a message such as:
\begin{center}
|This is the part \childdocname{} of \childdocjob{}.|
\end{center}

%%%%%%%%%%%%%%%%%%%%%%%%%%%%%%%%%%%%%%%%%%%%%%%%%%%%%%%%%%%%%%%%%%%%%%%%%%%%%%%%
\subsection{Flags}
\label{sec:flags}

The package makes it easy to generate different versions
of the main or child documents.
To this end compilation flags can be defined
and assigned different default values.
They will be particularly useful in conjunction
with the forwarding mechanism described in \secref{sec:forward}.

For example, it may be useful to have a flag |\version|
which can be set to |draft| or |final|.
The document source will contain some conditional code
depending on the value of |\version|.
Suppose further, the flag should default to |final| for the main file
and to |draft| for child files
which is a natural assignment for editing the document.
This is achieved by placing the following code
in the preamble of the main document
(below the |\childdocmain| directive):
%
\begin{center}
\begin{tabular}{l}
|\ifchilddoc|\\
|\providecommand{\version}{draft}|\\
|\||else|\\
|\providecommand{\version}{final}|\\
|\||fi|
\end{tabular}
\end{center}
%
The definition by |\providecommand| makes sure
that previous definitions are not overwritten.
Further statements |\providecommand{\version}{...}|
can thus be added before the above code to override it.

For the main file, one might add a line
(between |\childdocmain| and the above block)
%
\begin{center}
|%\ifchilddoc\||else\providecommand{\version}{draft}\||fi|
\end{center}
%
which can be uncommented to produce a draft version.
Likewise one can add a line to the very top of a child file
(above the |\childdocof{|\textit{main}|}| directive)
%
\begin{center}
|%\providecommand{\version}{final}|
\end{center}
%
which can be uncommented to produce the final version of this child document.

%%%%%%%%%%%%%%%%%%%%%%%%%%%%%%%%%%%%%%%%%%%%%%%%%%%%%%%%%%%%%%%%%%%%%%%%%%%%%%%%
\subsection{Forwarding}
\label{sec:forward}

Different versions of the main or child documents
using compilation flags as described in \secref{sec:flags}
can be (permanently) stored in different files
for convenient compilation, viewing and distribution.
To this end, the package defines a command
to pass on compilation to a different file:

%%%%%%%%%%%%%%%%%%%%%%%%%%%%%%%%%%%%%%%%
\DescribeMacro{\childdocforward}
The command |\childdocforward| redirects processing to
another source file:
%
\begin{center}
\begin{tabular}{l}
|\input{childdoc.def}|\\
|\childdocforward[|\textit{main}|]{|\textit{dest}|}|\\
\end{tabular}
\end{center}
%
The argument \textit{dest} is the destination file
(without extension).
It should be the main file or one of the child files.
Note that further \textsf{childdoc} directives
such as |\childdocof| and |\childdocforward|
in the indicated file will be processed in this form.
The optional argument \textit{main}
passes on directly to the main file \textit{main}
while pretending to compile the child \textit{dest}.
This form behaves as if \textit{dest}
issues |\childdocof{|\textit{main}|}| right away,
and no further \textsf{childdoc} directives will be processed.

%%%%%%%%%%%%%%%%%%%%%%%%%%%%%%%%%%%%%%%%
\DescribeMacro{\...prefix}
In the alternative form |\childdocforwardprefix|,
%
\begin{center}
\begin{tabular}{l}
|\input{childdoc.def}|\\
|\childdocforwardprefix[|\textit{main}|]{|\textit{prefix}|}{|\textit{dest}|}|
\end{tabular}
\end{center}
%
the destination file is determined by a pattern
depending on the current file:
To make this work, the current file must be called
`{\textit{prefix}\hspace{0.2em}\textit{suffix}}'
with \textit{prefix} matching precisely the argument.
Processing is then passed on to the file
`{\textit{dest}\hspace{0.2em}\textit{suffix}}'.
Surely, the same effect is achieved by
directly specifying the
argument `{\textit{dest}\hspace{0.2em}\textit{suffix}}'
in the first form.
However, that requires to set up a different file
for each child. With the alternative form of the command
all these files can have exactly the same content
which simplifies setting them up and maintaining them.

For example, the following file |draft.tex|
with a compilation flag |\version| as described in \secref{sec:flags}
compiles the main document as a draft:
%
\begin{center}
\begin{tabular}{l}
|\def\version{draft}|\\
|\input{childdoc.def}|\\
|\childdocforward{|\textit{main}|}|
\end{tabular}
\end{center}
%
Likewise, the following files |final|\textit{nn}|.tex|
compile the final version of the child document
|child|\textit{nn}|.tex|:
%
\begin{center}
\begin{tabular}{l}
|\def\version{final}|\\
|\input{childdoc.def}|\\
|\childdocforwardprefix{final}{child}|
\end{tabular}
\end{center}
%

Note that when several versions of a main file and/or of each child file
are to be generated, it may be convenient to set up a |Makefile| or
shell script to automatise the process.

%%%%%%%%%%%%%%%%%%%%%%%%%%%%%%%%%%%%%%%%%%%%%%%%%%%%%%%%%%%%%%%%%%%%%%%%%%%%%%%%
\subsection{Command Line Processing}
\label{sec:commandline}

The effect of redirection files can also be achieved by invoking
the \LaTeX{} compiler with a more elaborate command line.
Most conveniently this should be done as part
of a shell script or a |Makefile|.

When using \textsf{childdoc} in the main file, the following
command lines effectively perform a redirection
(note that depending on the shell being used,
backslashes may have to be doubled: `|\|' $\to$ `|\\|'):
%
\begin{center}
|... -jobname "|\textit{target}|" |\\|"|[\textit{flags}]%
|\input{childdoc.def}\childdocforward[|\textit{main}|]{|\textit{dest}|}"|
\end{center}
%
Here \textit{target} is the name of the output file,
\textit{main} is the name of the main file
and \textit{dest} is the name of the main or child file to be processed
(all filenames without extensions).
The optional argument \textit{main} can be omitted
if \textit{main} matches \textit{dest}.
Optionally, compilation \textit{flags} can be defined via |\def| commands.
This command line makes the \TeX{} engine believe
it is compiling the file \textit{target}
whose content is specified as the latter parameter.
The provided code then forwards the processing to
\textit{main} or \textit{dest} as described in \secref{sec:forward}.

%%%%%%%%%%%%%%%%%%%%%%%%%%%%%%%%%%%%%%%%%%%%%%%%%%%%%%%%%%%%%%%%%%%%%%%%%%%%%%%%
\subsection{Include by Input}
\label{sec:input}

Including child documents by |\include| has some restrictions by design.
Most notably, the content of a child document always occupies
its own set of pages; pages cannot be shared between child documents.
Usually, this behaviour makes perfect sense
because each child document contain an essential part of the document.
However, in some situations it may be desirable to compose
a document from a collection of parts
without having mandatory page breaks between then.
For this case, the package
provides a mechanism to include parts
by |\input| which can also be processed individually.
However, by construction this mechanism
requires manual handling of the content to be output.

%%%%%%%%%%%%%%%%%%%%%%%%%%%%%%%%%%%%%%%%
\DescribeMacro{\ifchilddocmanual}
The main file should be prepared as usual, see \secref{sec:include}.
However, the document body must make a distinction
between processing of an individual part and of the main document, e.g.:
%
\begin{center}
\begin{tabular}{l}
|\ifchilddocmanual|\\
|\input{\childdocname}|\\
|\||else|\\
\textit{document body with }|\input{|\textit{part}|}|\\
|\||fi|
\end{tabular}
\end{center}
%
The conditional |\ifchilddocmanual| is true whenever
a part to be included by |\input| is being compiled,
and the name of the part is stored in |\childdocname|.

%%%%%%%%%%%%%%%%%%%%%%%%%%%%%%%%%%%%%%%%
\DescribeMacro{\childdocby}
Each part to be included by |\input| should start with:
%
\begin{center}
\begin{tabular}{l}
|\input{childdoc.def}|\\
|\childdocby{|\textit{main}|}|\\
\end{tabular}
\end{center}
%
The directive |\childdocby| is similar to |\childdocof|
described in \secref{sec:include},
but the subsequent selection of content must be done manually.
To that end, both |\ifchilddoc| and |\ifchilddocmanual|
will be true upon processing of a part,
and the name of the part is stored in |\childdocname|.
Note that |\jobname| will be set to the filename of the current part
so that each part receives an individual |.aux| file
that does not interfere with the |.aux| file(s) of the main document.
This behaviour can be altered by the alternative form
|\childdocby[*]{|\textit{main}|}| (with a non-empty optional argument)
which uses the |.aux| file of the main document
by setting |\jobname| to \textit{main}.

%%%%%%%%%%%%%%%%%%%%%%%%%%%%%%%%%%%%%%%%%%%%%%%%%%%%%%%%%%%%%%%%%%%%%%%%%%%%%%%%
\subsection{Driver Development}
\label{sec:driver}

The \textsf{childdoc} mechanism can also be use for the development
of definition files such as \LaTeX{} styles or classes.
This case differs from the above setup with multiple parts
included by |\include| in that no |\includeonly| should be invoked.
This can be achieved by starting the include file
(before |\ProvidesPackage|) with:
%
\begin{center}
\begin{tabular}{l}
|\input{childdoc.def}|\\
|\childdocforward{|\textit{main}|}|\\
\end{tabular}
\end{center}
%
or alternatively with:
%
\begin{center}
\begin{tabular}{l}
|\input{childdoc.def}|\\
|\childdocby{|\textit{main}|}|\\
\end{tabular}
\end{center}
%
Both forms have slightly different effects as described above.
The main file is prepared as usual, see \secref{sec:include}.

%%%%%%%%%%%%%%%%%%%%%%%%%%%%%%%%%%%%%%%%%%%%%%%%%%%%%%%%%%%%%%%%%%%%%%%%%%%%%%%%
\subsection{Legacy Detection}
\label{sec:detection}

The directive |\childdocmain| in the main file can detect
whether the complete document or merely a child is to be compiled
even without using the directive |\childdocof|.
This method is deprecated because it is less robust
and there is no compelling reason to use it;
it is merely provided for backward compatibility
and it may be removed in future versions.

If the detection mechanism is to be used,
it is mandatory to correctly specify
the filename of the main file as the argument of |\childdocmain|:
%
\begin{center}
\begin{tabular}{l}
|\input{childdoc.def}|\\
|\childdocmain{|\textit{main}|}|\\
\end{tabular}
\end{center}
%
If |\jobname| does not match the argument \textit{main} of |\childdocmain|,
it is assumed that |\jobname| points to the child file to be compiled.
When using |\childdocmain| with the main file specified as argument,
it suffices to start a child file
with just |\input{|\textit{main}|}|
without loading of the package and using |\childdocof|.
If instead all processing is done
with the appropriate \textsf{childdoc} directives,
the argument of \textit{main} of |\childdocmain| can be empty.

An alternative version of the command line processing described
in \secref{sec:commandline} using the detection mechanism reads:
%
\begin{center}
|... -jobname "|\textit{target}|" "|[\textit{flags}]%
[|\def\jobname{|\textit{dest}|}|]|\input{|\textit{main}|}"|
\end{center}

%%%%%%%%%%%%%%%%%%%%%%%%%%%%%%%%%%%%%%%%%%%%%%%%%%%%%%%%%%%%%%%%%%%%%%%%%%%%%%%%
\subsection{Manual Code}
\label{sec:manual}

In case one cannot be certain whether the definitions file |childdoc.def|
is installed on the target \TeX{} distribution
and one prefers not to ship it,
it is conceivable to paste a few relevant commands into the sources.

To that end, drop all statements |\input{childdoc.def}|
and perform the replacements as outlined below.
Instead of |\childdocmain{|\textit{main}|}| add the following code
to the top of the main file:
%
\begin{center}
\begin{tabular}{l}
|\||ifdefined\childdocname\endinput\||fi\newif\ifchilddoc|\\
|\edef\childdocname{\scantokens\expandafter{\jobname\noexpand}}|\\
|\def\childdocmain{|\textit{main}|}\||ifx\childdocmain\childdocname\||else|\\
|\childdoctrue\includeonly{\childdocname}\let\jobname\childdocmain\||fi|\\
\end{tabular}
\end{center}
%
Instead of |\childdocof{|\textit{main}|}| just include the main file
at the top of each child file:
%
\begin{center}
|\input{|\textit{main}|}|
\end{center}
%
A simple redirection |\childdocforward{|\textit{dest}|}| is achieved by:
%
\begin{center}
|\def\jobname{|\textit{dest}|}\input{\jobname}|
\end{center}
%
The redirection with prefix
|\childdocforwardprefix[|\textit{prefix}|]{|\textit{dest}|}|
is accomplished by:
%
\begin{center}
\begin{tabular}{l}
|{\edef\jobname{\scantokens\expandafter{\jobname\noexpand}}|\\
|\def\redirectjob |\textit{prefix}|#1~~~{\gdef\jobname{|\textit{dest}|#1}}|\\
|\expandafter\redirectjob\jobname~~~}\input{\jobname}|
\end{tabular}
\end{center}

In an alternative approach,
child documents can be compiled by a specific command line
without additional code or specific definitions:
%
\begin{center}
|... -jobname "|\textit{target}|" "|[\textit{flags}]%
|\includeonly{|\textit{dest}|}\input{|\textit{main}|}"|
\end{center}
%

%%%%%%%%%%%%%%%%%%%%%%%%%%%%%%%%%%%%%%%%%%%%%%%%%%%%%%%%%%%%%%%%%%%%%%%%%%%%%%%%
%%%%%%%%%%%%%%%%%%%%%%%%%%%%%%%%%%%%%%%%%%%%%%%%%%%%%%%%%%%%%%%%%%%%%%%%%%%%%%%%
\section{Information}

%%%%%%%%%%%%%%%%%%%%%%%%%%%%%%%%%%%%%%%%%%%%%%%%%%%%%%%%%%%%%%%%%%%%%%%%%%%%%%%%
\subsection{Copyright}

Copyright \copyright{} 2017--2018 Niklas Beisert

This work may be distributed and/or modified under the
conditions of the \LaTeX{} Project Public License, either version 1.3
of this license or (at your option) any later version.
The latest version of this license is in
  \url{http://www.latex-project.org/lppl.txt}
and version 1.3 or later is part of all distributions of \LaTeX{}
version 2005/12/01 or later.

This work has the LPPL maintenance status `maintained'.

The Current Maintainer of this work is Niklas Beisert.

This work consists of the files |README.txt|, |childdoc.ins| and |childdoc.dtx|
as well as the derived files |childdoc.def|, |cdocsamp.tex|
with |cdocsch1.tex|, |cdocsch2.tex|, |cdocspt3.tex|, |cdocspt4.tex|,
|cdocsdrf.tex|, |cdocsfn1.tex|, |cdocsfn2.tex|
as well as |childdoc.pdf|.

%%%%%%%%%%%%%%%%%%%%%%%%%%%%%%%%%%%%%%%%%%%%%%%%%%%%%%%%%%%%%%%%%%%%%%%%%%%%%%%%
\subsection{Files and Installation}

The package consists of the files:
%
\begin{center}
\begin{tabular}{ll}
    |README.txt|   & readme file \\
    |childdoc.ins| & installation file \\
    |childdoc.dtx| & source file \\
    |childdoc.def| & definition file \\
    |cdocsamp.tex| & sample main file \\
    |cdocsch1.tex| & sample include file \\
    |cdocsch2.tex| & sample include file \\
    |cdocspt3.tex| & sample part file \\
    |cdocspt4.tex| & sample part file \\
    |cdocsdrf.tex| & sample redirection file \\
    |cdocsfn1.tex| & sample redirection file \\
    |cdocsfn2.tex| & sample redirection file \\
    |childdoc.pdf| & manual
\end{tabular}
\end{center}
%
The distribution consists of the files
|README.txt|, |childdoc.ins| and |childdoc.dtx|.
%
\begin{itemize}
\item
Run (pdf)\LaTeX{} on |childdoc.dtx|
to compile the manual |childdoc.pdf| (this file).
\item
Run \LaTeX{} on |childdoc.ins| to create the definitions file |childdoc.def|
and the sample |cdocsamp.tex| with include files
|cdocsch1.tex|, |cdocsch2.tex|, |cdocspt3.tex|, |cdocspt4.tex|,
|cdocsdrf.tex|, |cdocsfn1.tex|, |cdocsfn2.tex|.
Then copy the file |childdoc.def| to an appropriate directory of your \LaTeX{}
distribution, e.g.\ \textit{texmf-root}|/tex/latex/childdoc|.
\end{itemize}

%%%%%%%%%%%%%%%%%%%%%%%%%%%%%%%%%%%%%%%%%%%%%%%%%%%%%%%%%%%%%%%%%%%%%%%%%%%%%%%%
\subsection{Related CTAN Packages}

There are several other packages which offer a similar functionality:
%
\begin{itemize}
\item
The packages
\href{http://ctan.org/pkg/docmute}{\textsf{docmute}},
\href{http://ctan.org/pkg/includex}{\textsf{includex}} and
\href{http://ctan.org/pkg/standalone}{\textsf{standalone}}
provide commands to include only the document body of
a child file thus allowing both files to be compiled individually.
\item
The packages \href{http://ctan.org/pkg/subdocs}{\textsf{subdocs}}
and \href{http://ctan.org/pkg/subfiles}{\textsf{subfiles}}
provide structures in which the main and child documents can be
encapsulated and allowing them to be compiled individually.
The inclusion mechanism is different from the conventional |\include|.
\item
The package \href{http://ctan.org/pkg/combine}{\textsf{combine}}
is an elaborate solution to combine several documents into one.
\end{itemize}
%
See also the CTAN topic \href{http://ctan.org/topic/subdocs}{\textsf{subdocs}}
for further related packages.
The present package differs from the above solutions in that
a document structure constructed with the conventional |\include| mechanism
just needs two extra commands at the top of every file
such that all constituent files can be compiled individually.

%%%%%%%%%%%%%%%%%%%%%%%%%%%%%%%%%%%%%%%%%%%%%%%%%%%%%%%%%%%%%%%%%%%%%%%%%%%%%%%%
%\subsection{Feature Suggestions}
%
%The following is a list of features which may be useful for future
%versions of this package:
%%
%\begin{itemize}
%\item
%\ldots
%\end{itemize}

%%%%%%%%%%%%%%%%%%%%%%%%%%%%%%%%%%%%%%%%%%%%%%%%%%%%%%%%%%%%%%%%%%%%%%%%%%%%%%%%
\subsection{Revision History}

%%%%%%%%%%%%%%%%%%%%%%%%%%%%%%%%%%%%%%%%
\paragraph{v2.0:} 2018/12/30

\begin{itemize}
\item
immediate forward processing
\item
added |\childdocby| mechanism
\item
manual restructured
\end{itemize}

%%%%%%%%%%%%%%%%%%%%%%%%%%%%%%%%%%%%%%%%
\paragraph{v1.6:} 2018/01/17

\begin{itemize}
\item
application for development of include files
\item
corrections to manual
\end{itemize}

%%%%%%%%%%%%%%%%%%%%%%%%%%%%%%%%%%%%%%%%
\paragraph{v1.5:} 2017/05/21

\begin{itemize}
\item
more complete structuring introduced
\item
|\childdocof| introduced
\item
|\childdoc| renamed to |\childdocmain|
\item
|\childredirect| renamed to |\childdocforward| and |\childdocforwardprefix|
and functionality expanded
\end{itemize}

%%%%%%%%%%%%%%%%%%%%%%%%%%%%%%%%%%%%%%%%
\paragraph{v1.0:} 2017/04/27

\begin{itemize}
\item
manual and install package
\item
first version published on CTAN
\end{itemize}

%%%%%%%%%%%%%%%%%%%%%%%%%%%%%%%%%%%%%%%%
\paragraph{v0.6:} 2017/04/26

\begin{itemize}
\item
redirection mechanism added
\end{itemize}

%%%%%%%%%%%%%%%%%%%%%%%%%%%%%%%%%%%%%%%%
\paragraph{v0.5:} 2017/04/26

\begin{itemize}
\item
functionality in definition file
\end{itemize}


%%%%%%%%%%%%%%%%%%%%%%%%%%%%%%%%%%%%%%%%%%%%%%%%%%%%%%%%%%%%%%%%%%%%%%%%%%%%%%%%
%%%%%%%%%%%%%%%%%%%%%%%%%%%%%%%%%%%%%%%%%%%%%%%%%%%%%%%%%%%%%%%%%%%%%%%%%%%%%%%%
%%%%%%%%%%%%%%%%%%%%%%%%%%%%%%%%%%%%%%%%%%%%%%%%%%%%%%%%%%%%%%%%%%%%%%%%%%%%%%%%
\appendix

\settowidth\MacroIndent{\rmfamily\scriptsize 000\ }

 \DocInput{childdoc.dtx}

\end{document}
%</driver>
% \fi
%
% %%%%%%%%%%%%%%%%%%%%%%%%%%%%%%%%%%%%%%%%%%%%%%%%%%%%%%%%%%%%%%%%%%%%%%%%%%%%%%
% %%%%%%%%%%%%%%%%%%%%%%%%%%%%%%%%%%%%%%%%%%%%%%%%%%%%%%%%%%%%%%%%%%%%%%%%%%%%%%
% \section{Sample}
%\iffalse
%<*samplemain>
%\fi
%
% The following presents a sample document
% with two chapters, two parts, a title page,
% a compile flag as well as three forwarding files to set the flag.
% It consists of eight |.tex| files:
% \begin{center}
% \begin{tabular}{ll}
% |cdocsamp.tex|&main file\\
% |cdocsch1.tex|&include file for chapter 1\\
% |cdocsch2.tex|&include file for chapter 2\\
% |cdocspt3.tex|&include file for part 3\\
% |cdocspt4.tex|&include file for part 4\\
% |cdocsdrf.tex|&forwarding file for main file in draft mode\\
% |cdocsfi1.tex|&forwarding file for final version of chapter 1\\
% |cdocsfi2.tex|&forwarding file for final version of chapter 2\\
% \end{tabular}
% \end{center}
% Each of the eight files can be compiled directly by the \LaTeX{} compiler.
%
% %%%%%%%%%%%%%%%%%%%%%%%%%%%%%%%%%%%%%%
% \paragraph{Main File.}
%
% The main file is called |cdocsamp.tex|.
%
% Load the \textsf{childdoc} definitions and
% declare the filename for the main document:
%    \begin{macrocode}
\input{childdoc.def}
\childdocmain{}
%    \end{macrocode}

% Optional override for |\version| flag:
%    \begin{macrocode}
%%\ifchilddoc\else\providecommand{\version}{draft}\fi
%    \end{macrocode}

% Define the default values for the |\version| flag
% (|final| for the main file and |draft| for childs):
%    \begin{macrocode}
\ifchilddoc
\providecommand{\version}{draft}
\else
\providecommand{\version}{final}
\fi
%    \end{macrocode}

% Load the standard document class:
%    \begin{macrocode}
\documentclass[12pt]{article}
%    \end{macrocode}

% Start the document body:
%    \begin{macrocode}
\begin{document}
%    \end{macrocode}

% Declare a title page.
% Print title, part of document being processed and version flag:
%    \begin{macrocode}
\addtocounter{page}{-1}
\begin{center}
{\LARGE\bfseries{}childdoc example\par}
\vspace{1cm}
\ifchilddoc
\ifchilddocmanual part\else chapter\fi:
`\childdocname' of `\childdocjob'\par
\else
main document: `\childdocjob'\par
\fi
version: \version\par
\end{center}
\newpage
%    \end{macrocode}

% Manually include selected file,
% otherwise process as usual:
%    \begin{macrocode}
\ifchilddocmanual
\section*{part `\childdocname'}
\input{\childdocname}
\else
%    \end{macrocode}

% Include the two chapters:
%    \begin{macrocode}
\include{cdocsch1}
\include{cdocsch2}
%    \end{macrocode}

% Include the two parts unless only chapters should be displayed:
%    \begin{macrocode}
\ifchilddoc\else
\section{part three}
\input{cdocspt3}
\section{part four}
\input{cdocspt4}
\fi
%    \end{macrocode}

% Process as usual until here:
%    \begin{macrocode}
\fi
%    \end{macrocode}

% End of document body:
%    \begin{macrocode}
\end{document}
%    \end{macrocode}
%\iffalse
%</samplemain>
%\fi
%
% %%%%%%%%%%%%%%%%%%%%%%%%%%%%%%%%%%%%%%
% \paragraph{Chapter Include Files.}
%
% The include files are called |cdocsch1.tex| and |cdocsch2.tex|.
%
%\iffalse
%<*samplechap1|samplechap2>
%\fi

% Optional override for |\version| flag:
%    \begin{macrocode}
%%\providecommand{\version}{final}
%    \end{macrocode}

% Include the main document:
%    \begin{macrocode}
\input{childdoc.def}
\childdocof{cdocsamp}
%    \end{macrocode}

%\iffalse
%</samplechap1|samplechap2>
%\fi
%
%\iffalse
%<*samplechap1>
%\fi
% Some text for chapter 1:
%    \begin{macrocode}
\section{one}
some text in chapter one
%    \end{macrocode}

%\iffalse
%</samplechap1>
%\fi
% Some text for chapter 2:
%\iffalse
%<*samplechap2>
%\fi
%    \begin{macrocode}
\section{two}
more text in chapter two
%    \end{macrocode}

%\iffalse
%</samplechap2>
%\fi
%
% %%%%%%%%%%%%%%%%%%%%%%%%%%%%%%%%%%%%%%
% \paragraph{Part Include Files.}
%
% The include files are called |cdocspt3.tex| and |cdocspt4.tex|.
%
%\iffalse
%<*samplepart3|samplepart4>
%\fi

% Optional override for |\version| flag:
%    \begin{macrocode}
%%\providecommand{\version}{final}
%    \end{macrocode}

% Include the main document:
%    \begin{macrocode}
\input{childdoc.def}
\childdocby{cdocsamp}
%    \end{macrocode}

%\iffalse
%</samplepart3|samplepart4>
%\fi
%
%\iffalse
%<*samplepart3>
%\fi
% Some text for part 3:
%    \begin{macrocode}
some text in part three
%    \end{macrocode}

%\iffalse
%</samplepart3>
%\fi
% Some text for part 4:
%\iffalse
%<*samplepart4>
%\fi
%    \begin{macrocode}
more text in part four
%    \end{macrocode}

%\iffalse
%</samplepart4>
%\fi
%
% %%%%%%%%%%%%%%%%%%%%%%%%%%%%%%%%%%%%%%
% \paragraph{Forwarding for a Complete Draft.}
%
% The following forwarding file |cdocsdrf.tex|
% compiles the main document in draft mode:
%\iffalse
%<*sampledraft>
%\fi
%    \begin{macrocode}
\def\version{draft}
\input{childdoc.def}
\childdocforward{cdocsamp}
%    \end{macrocode}

%\iffalse
%</sampledraft>
%\fi
%
% %%%%%%%%%%%%%%%%%%%%%%%%%%%%%%%%%%%%%%
% \paragraph{Forwarding for Final Version of the Chapters.}
%
% The following forwarding files |cdocsfn1.tex| and |cdocsfn2.tex|
% (with identical content)
% compile the final versions of the child documents
% |cdocsch1.tex| and |cdocsch2.tex|, respectively:
%\iffalse
%<*samplefinal>
%\fi
%    \begin{macrocode}
\def\version{final}
\input{childdoc.def}
\childdocforwardprefix[cdocsamp]{cdocsfn}{cdocsch}
%    \end{macrocode}

%\iffalse
%</samplefinal>
%\fi
%
% %%%%%%%%%%%%%%%%%%%%%%%%%%%%%%%%%%%%%%
% \paragraph{Command Line Processing.}
%
% The following three command lines generate the output files
% |cdocscld|, |cdocscl1| and |cdocscl2|
% which should be identical to
% |cdocsdrf|, |cdocsch1| and |cdocsfn2|, respectively:
% \begin{center}
% \begin{tabular}{l}
% |latex -jobname cdocscld \|\\
% |  "\def\version{draft}\input{childdoc.def}\childdocforward{cdocsamp}"|\\
% |latex -jobname cdocscl1 \|\\
% |  "\input{childdoc.def}\childdocforward[cdocsamp]{cdocsch1}"|\\
% |latex -jobname cdocscl2 \|\\
% |  "\def\version{final}\input{childdoc.def}\childdocforward{cdocsch2}"|
% \end{tabular}
% \end{center}
% Note that the trailing backslash on each first line
% merely continues the input to the second line
% (for convenient cut ant paste).
% Furthermore, the command |latex| can be replaced by any
% of its alternative versions such as |pdflatex|.
%
% %%%%%%%%%%%%%%%%%%%%%%%%%%%%%%%%%%%%%%%%%%%%%%%%%%%%%%%%%%%%%%%%%%%%%%%%%%%%%%
% %%%%%%%%%%%%%%%%%%%%%%%%%%%%%%%%%%%%%%%%%%%%%%%%%%%%%%%%%%%%%%%%%%%%%%%%%%%%%%
% \section{Implementation}
%\iffalse
%<*package>
%\fi
%
% This section describes the definitions file |childdoc.def|.

% The definitions cannot be loaded using |\usepackage| or |\RequirePackage|
% which has a mechanism to prevent loading a style file more than once.
% When loading the definitions by means of |\input|
% multiple instances have to be prevented manually:
%\iffalse
%This code needs to be before the `\ProvidesFile' directive
%which is defined at the beginning of this file.
%Therefore it is also placed there and commented out here.
%</package>
%<*discard>
%\fi
%    \begin{macrocode}
\ifdefined\childdocmain\endinput\fi
%    \end{macrocode}
%\iffalse
%</discard>
%<*package>
%\fi
%
% \macro{\ifchilddoc}
% \macro{\ifchilddocmanual}
% The conditional |\ifchilddoc| tells whether a
% child (true) or main (false) document is being compiled.
% The conditional |\ifchilddocmanual| tells whether
% the |\includeonly| mechanism is used (false) or
% the selection of child files must be performed manually (true).
% The definitions initialise to false:
%    \begin{macrocode}
\newif\ifchilddoc
\newif\ifchilddocmanual
%    \end{macrocode}

% \macro{\childdocname}
% \macro{\childdocjob}
% The macro |\childdocname| stores the name of the main document
% to be compiled. The macro |\childdocjob| stores the name of
% the document on which the \LaTeX{} compiler was originally invoked.
% The content of |\jobname| cannot be compared
% to filenames specified in the source due to different catcodes.
% The following code rescans |\jobname|, stores the result
% in |\childdocname| and saves a copy in |\childdocjob|:
%    \begin{macrocode}
\edef\childdocname{\scantokens\expandafter{\jobname\noexpand}}
\let\childdocjob\childdocname
%    \end{macrocode}

% \macro{\childdocdisable}
% The macro |\childdocdisable| prevents the main file
% from being processed more than once.
% At this stage, the main document command |\childdocmain|
% is assumed to be called once again where it should do nothing.
% Any subsequent call to it should prevent
% a secondary processing of the main document
% It overwrites the forwarding commands
% |\childdocof| and |\childdocforward|
% with empty macros to prevent further inclusions of the main document:
%    \begin{macrocode}
\newcommand{\childdocdisable}
{
  \renewcommand{\childdocmain}[1]{\renewcommand{\childdocmain}[1]{\endinput}}
  \renewcommand{\childdocof}[1]{}
  \renewcommand{\childdocby}[2][]{}
  \renewcommand{\childdocforward}[2][]{}
  \renewcommand{\childdocdisable}{}
}
%    \end{macrocode}

% \macro{\childdocmain}
% The macro |\childdocmain| is to be called at the top of the main file
% with nothing or the main filename (without extension) as argument.
% First, it breaks loops.
% If the argument is not empty and does not match |\childdocname|
% (which is set by the first inclusion of |childdoc.def|),
% |\ifchilddoc| is set to true, |\includeonly| is applied to the child file
% and |\jobname| is set to the main file
% (for proper handling of |.aux| files):
%    \begin{macrocode}
\newcommand{\childdocmain}[1]
{
  \childdocdisable\childdocmain{}
  \if?#1?\else
    \begingroup
      \def\childdoctmp{#1}
      \ifx\childdoctmp\childdocname
        \def\childdoctmp{}
      \else
        \def\childdoctmp
        {
          \childdoctrue
          \includeonly{\childdocname}
          \def\childdocjob{#1}
          \def\jobname{#1}
        }
      \fi
      \expandafter
    \endgroup
    \childdoctmp
  \fi
}
%    \end{macrocode}

% \macro{\childdocof}
% The command |\childdocof| redirects
% compilation to the main file |#1|.
%    \begin{macrocode}
\newcommand{\childdocof}[1]
{
  \childdocdisable
  \childdoctrue
  \includeonly{\childdocname}
  \def\jobname{#1}
  \def\childdocjob{#1}
  \input{#1}
}
%    \end{macrocode}

% \macro{\childdocby}
% The command |\childdocby| ....
%    \begin{macrocode}
\newcommand{\childdocby}[2][]
{
  \childdocdisable
  \childdoctrue
  \childdocmanualtrue
  \if?#1?\else
    \def\jobname{#2}
  \fi
  \def\childdocjob{#2}
  \input{#2}
  \endinput
}
%    \end{macrocode}

% \macro{\childdocforward}
% The command |\childdocforward| redirects
% compilation to the main file or
% (if the optional argument is given) a child file.
% Parameters are set as if the main file
% or a child file starting with |\childdocof| was compiled.
% Then compilation is handed over to the main file:
%    \begin{macrocode}
\newcommand{\childdocforward}[2][]
{
  \begingroup
    \if?#1?
      \def\childdoctmp
      {
        \def\childdocname{#2}
        \def\childdocjob{#2}
        \def\jobname{#2}
        \input{#2}
        \endinput
      }
    \else
      \def\childdoctmp
      {
        \childdocdisable
        \def\childdocname{#2}
        \childdoctrue
        \includeonly{#2}
        \def\childdocjob{#1}
        \def\jobname{#1}
        \input{#1}
        \endinput
      }
    \fi
    \expandafter
  \endgroup
  \childdoctmp
}
%    \end{macrocode}

% \macro{\childdocforwardprefix}
% The command |\childdocforwardprefix| redirects
% compilation to the main or a child file by means of a pattern.
% The prefix |#1| in the current filename is replaced by |#2|
% and the suffix of the current filename is kept
% (it is assumed that the filename does not contain the substring `|~~~|'
% which is used as a delimiter).
% Compilation is handed over to the new file by |\childdocforward|:
%    \begin{macrocode}
\newcommand{\childdocforwardprefix}[3][]
{
  \begingroup
    \def\childdocextract #2##1~~~{\def\childdoctmp{\childdocforward[#1]{#3##1}}}
    \expandafter\childdocextract\childdocname~~~
    \expandafter
  \endgroup
  \childdoctmp
}
%    \end{macrocode}

% \macro{\childdoc}
% The deprecated macro |\childdoc| is a legacy version of |\childdocmain|:
%    \begin{macrocode}
\newcommand{\childdoc}{\childdocmain}
%    \end{macrocode}

% \macro{\childdocredirect}
% The deprecated macro |\childdocredirect| is a legacy version
% of |\childdocforward| and |\childdocforwardprefix|:
%    \begin{macrocode}
\newcommand{\childdocredirect}[2][]
{
  \begingroup
    \if?#1?
      \def\childdoctmp{\childdocforward{#2}}
    \else
      \def\childdoctmp{\childdocforwardprefix{#1}{#2}}
    \fi
    \expandafter
  \endgroup
  \childdoctmp
}
%    \end{macrocode}

%\iffalse
%</package>
%\fi
%
\endinput
|\\
|\childdocby{|\textit{main}|}|\\
\end{tabular}
\end{center}
%
Both forms have slightly different effects as described above.
The main file is prepared as usual, see \secref{sec:include}.

%%%%%%%%%%%%%%%%%%%%%%%%%%%%%%%%%%%%%%%%%%%%%%%%%%%%%%%%%%%%%%%%%%%%%%%%%%%%%%%%
\subsection{Legacy Detection}
\label{sec:detection}

The directive |\childdocmain| in the main file can detect
whether the complete document or merely a child is to be compiled
even without using the directive |\childdocof|.
This method is deprecated because it is less robust
and there is no compelling reason to use it;
it is merely provided for backward compatibility
and it may be removed in future versions.

If the detection mechanism is to be used,
it is mandatory to correctly specify
the filename of the main file as the argument of |\childdocmain|:
%
\begin{center}
\begin{tabular}{l}
|% \iffalse
%
% childdoc.dtx Copyright (C) 2017-2018 Niklas Beisert
%
% This work may be distributed and/or modified under the
% conditions of the LaTeX Project Public License, either version 1.3
% of this license or (at your option) any later version.
% The latest version of this license is in
%   http://www.latex-project.org/lppl.txt
% and version 1.3 or later is part of all distributions of LaTeX
% version 2005/12/01 or later.
%
% This work has the LPPL maintenance status `maintained'.
%
% The Current Maintainer of this work is Niklas Beisert.
%
% This work consists of the files childdoc.dtx and childdoc.ins
% and the derived files childdoc.def and cdocsamp.tex with
% cdocsch1.tex, cdocsch2.tex, cdocsdrf.tex, cdocsfn1.tex, cdocsfn2.tex.
%
%<package>\ifdefined\childdocmain\endinput\fi
%<package>\ProvidesFile{childdoc.def}[2018/12/30 v2.0 child document driver]
%<samplemain>\ProvidesFile{cdocsamp.tex}[2018/12/30 v2.0 sample for childdoc]
%<*driver>
%\ProvidesFile{childdoc.drv}[2018/12/30 v2.0 childdoc reference manual file]
\PassOptionsToClass{10pt,a4paper}{article}
\documentclass{ltxdoc}

\usepackage[margin=35mm]{geometry}
\usepackage{hyperref}
\usepackage{hyperxmp}
\usepackage[usenames]{color}

\hypersetup{colorlinks=true}
\hypersetup{pdfstartview=FitH}
\hypersetup{pdfpagemode=UseNone}
\hypersetup{pdfsource={}}
\hypersetup{pdflang={en-UK}}
\hypersetup{pdfcopyright={Copyright 2017-2018 Niklas Beisert.
  This work may be distributed and/or modified under the
  conditions of the LaTeX Project Public License, either version 1.3
  of this license or (at your option) any later version.}}
\hypersetup{pdflicenseurl={http://www.latex-project.org/lppl.txt}}
\hypersetup{pdfcontactaddress={ETH Zurich, ITP, HIT K,
  Wolfgang-Pauli-Strasse 27}}
\hypersetup{pdfcontactpostcode={8093}}
\hypersetup{pdfcontactcity={Zurich}}
\hypersetup{pdfcontactcountry={Switzerland}}
\hypersetup{pdfcontactemail={nbeisert@itp.phys.ethz.ch}}
\hypersetup{pdfcontacturl={http://people.phys.ethz.ch/\xmptilde nbeisert/}}

\newcommand{\secref}[1]{\hyperref[#1]{section \ref*{#1}}}

\parskip1ex
\parindent0pt
\let\olditemize\itemize
\def\itemize{\olditemize\parskip0pt}

\begin{document}

\title{The \textsf{childdoc} Package}
\hypersetup{pdftitle={The childdoc Package}}
\author{Niklas Beisert\\[2ex]
  Institut f\"ur Theoretische Physik\\
  Eidgen\"ossische Technische Hochschule Z\"urich\\
  Wolfgang-Pauli-Strasse 27, 8093 Z\"urich, Switzerland\\[1ex]
  \href{mailto:nbeisert@itp.phys.ethz.ch}
  {\texttt{nbeisert@itp.phys.ethz.ch}}}
\hypersetup{pdfauthor={Niklas Beisert}}
\hypersetup{pdfsubject={Manual for the LaTeX2e Package childdoc}}
\date{30 December 2018, \textsf{v2.0}}
\maketitle

\begin{abstract}\noindent
\textsf{childdoc} is a \LaTeXe{} package
that enables the direct compilation
of document sections included by |\include|
to individual files.
\end{abstract}

\begingroup
\parskip0ex
\tableofcontents
\endgroup

%%%%%%%%%%%%%%%%%%%%%%%%%%%%%%%%%%%%%%%%%%%%%%%%%%%%%%%%%%%%%%%%%%%%%%%%%%%%%%%%
%%%%%%%%%%%%%%%%%%%%%%%%%%%%%%%%%%%%%%%%%%%%%%%%%%%%%%%%%%%%%%%%%%%%%%%%%%%%%%%%
\section{Introduction}

\LaTeX{} provides a mechanism to structure a large document (such as a book)
into a main file and several child files (containing the chapters)
using the |\include| command.
This mechanism is beneficial for documents
which span hundreds of pages in order to
make the source file(s) more manageable.
Moreover, compilation can be restricted to
selected child files by means of the |\includeonly| command.
The latter feature can be used to reduce the compilation time while editing
(this was significantly more useful in the earlier days of \LaTeX{})
or to generate a smaller document which is easier to navigate.
Another application of |\includeonly| is to generate
documents consisting of selected parts of the complete document.

However, there are a few drawbacks of the plain |\include| mechanism:
\begin{itemize}
\item
The child files cannot be compiled on their own,
they can only be compiled via the main file.
A naive editing environment
(such as a text editor with an option
to have the current file processed by \LaTeX)
may require one to switch to the main file before compiling;
attempting to compile the child file produces errors.
\item
The main file must be modified (each time)
to adjust the |\includeonly| command
to the present needs. This easily leaves the main file in a messy state.
\item
The generated document will always carry the filename
of the main document. This is inconvenient if
several child files are to be compiled and
to be kept for distribution.
\end{itemize}

The present package provides a simple interface
to make child files individually compilable by \LaTeX{}.
Compiling a child file then has the same effect as compiling
the main file with an |\includeonly| command
to select the appropriate child.
Moreover the generated document will carry the name of the child
rather than the main file.
This resolves all three above issues.

This feature is meant to make the editing of books,
thesis documents and lecture notes somewhat more convenient.
However, the package can also be used efficiently for
composing a series of documents (such as exercise sheets)
which are typically distributed individually.
It then assists the author in generating the individual documents
(potentially in different versions)
as well as a document containing the collected series.
Another application is in developing style files
or other kinds of included material
where compilation of the style file could redirect
to a sample or test file.

%%%%%%%%%%%%%%%%%%%%%%%%%%%%%%%%%%%%%%%%%%%%%%%%%%%%%%%%%%%%%%%%%%%%%%%%%%%%%%%%
%%%%%%%%%%%%%%%%%%%%%%%%%%%%%%%%%%%%%%%%%%%%%%%%%%%%%%%%%%%%%%%%%%%%%%%%%%%%%%%%
\section{Usage}

First of all, the package \textsf{childdoc} is \emph{not} a standard
\LaTeXe{} |.sty| style file! Therefore it needs to be invoked in
a non-standard way.

%%%%%%%%%%%%%%%%%%%%%%%%%%%%%%%%%%%%%%%%%%%%%%%%%%%%%%%%%%%%%%%%%%%%%%%%%%%%%%%%
\subsection{Included Files}
\label{sec:include}

%%%%%%%%%%%%%%%%%%%%%%%%%%%%%%%%%%%%%%%%
\DescribeMacro{\childdocmain}
To use the package, add the commands
\begin{center}
\begin{tabular}{l}
|\input{childdoc.def}|\\
|\childdocmain{}|\\
\end{tabular}
\end{center}
at the very top of the main \LaTeX{} file,
in particular \emph{before} the |\documentclass| statement!
The argument of |\childdocmain| should be left empty
(but it must be present).

%%%%%%%%%%%%%%%%%%%%%%%%%%%%%%%%%%%%%%%%
\DescribeMacro{\childdocof}
Furthermore, add the commands
\begin{center}
\begin{tabular}{l}
|\input{childdoc.def}|\\
|\childdocof{|\textit{main}|}|\\
\end{tabular}
\end{center}
at the top of every child file \textit{child}
which is included by |\include{|\textit{child}|}|
from within the main file
(or at least for those files to be compiled individually).
The argument \textit{main} must be the filename of the main file.

There are a couple of
considerations in setting up the main and child documents:

%%%%%%%%%%%%%%%%%%%%%%%%%%%%%%%%%%%%%%%%
\paragraph{Restrictions.}

Please note the following restrictions:
\begin{itemize}
\item
|\childdocmain| must be called with one argument \textit{main}
to ensure compatibility with earlier version of the package.
It must either be empty (|\childdocmain{}|)
or precisely match the filename of the main file in which it is specified.
See \secref{sec:detection} for further information.
\item
The filename \textit{main} must be specified without the |.tex| extension.
\item
The filename \textit{main} is case sensitive
(even in case-insensitive file systems)
due to internal string comparison.
\item
The argument \textit{main} should be fully expanded, it cannot be a macro.
\item
Subdirectories and special characters should be avoided in filenames.
\item
The command |\childdocmain{|\textit{main}|}| must be followed by a whitespace.
It should not be followed immediately by another command
or by a comment mark `|%|'.
This is because the \TeX{} parser reads the token immediately following
the argument of |\childdocmain| and puts it
at the beginning of every child section;
however, a white\-space is ignored.
\end{itemize}

%%%%%%%%%%%%%%%%%%%%%%%%%%%%%%%%%%%%%%%%
\paragraph{Content of Main File.}

It is advisable to place all content in the child files included by |\include|.
Any output contained in the main file will appear in all child documents
unless suppressed manually;
it cannot be suppressed automatically by the |\includeonly| directive
and thus should normally be avoided.
A method to include some content in the main file
by means of conditional processing is described in \secref{sec:conditional}.

%%%%%%%%%%%%%%%%%%%%%%%%%%%%%%%%%%%%%%%%
\paragraph{Page Numbering.}

When only a part of the document is compiled,
the appropriate numbering of pages
(as well as other status parameters)
is determined from the |.aux| files.
The latter contain information from previous passes.
However this information needs to propagate through
all intermediate child documents.
Therefore the page numbering in child documents may well
be inconsistent until the complete document is compiled at least once.

A useful (if unconventional) way to always ensure a consistent
page numbering is to restart the numbering in each child document
and denote the pages by `\textit{child}|.|\textit{page}'
where \textit{child} represents the chapter/section number of the child file.
This can be achieved by the command
|\numberwithin{page}{|\textit{child}|}|
of the \textsf{amsmath} package
where \textit{child} can be |chapter| or |section|
depending on the chosen structuring.
Alternatively, one can modify the macro |\thepage| appropriately
and reset the counter |page| at the start of each child file.

%%%%%%%%%%%%%%%%%%%%%%%%%%%%%%%%%%%%%%%%%%%%%%%%%%%%%%%%%%%%%%%%%%%%%%%%%%%%%%%%
\subsection{Conditional Processing}
\label{sec:conditional}

The package provides a mechanism to compile different versions
of a document. To customise the versions further some conditional processing
can come in handy to distinguish which version is being compiled.
The package provides two macros to describe the compilation context:

%%%%%%%%%%%%%%%%%%%%%%%%%%%%%%%%%%%%%%%%
\DescribeMacro{\ifchilddoc}
The conditional |\ifchilddoc| distinguishes between the compilation of
child documents and the main document:
%
\begin{center}
|\ifchilddoc |\textit{child-code}| |[|\||else |\textit{main-code}]| \||fi|
\end{center}

%%%%%%%%%%%%%%%%%%%%%%%%%%%%%%%%%%%%%%%%
\DescribeMacro{\childdocname}
\DescribeMacro{\childdocjob}
The macro |\childdocname| contains the filename (without extension)
of the main or child file being processed.
Note that |\childdocjob| will always contain the name of the main file.

%%%%%%%%%%%%%%%%%%%%%%%%%%%%%%%%%%%%%%%%
\paragraph{Title Page.}

Conditional processing can be used to include a title or banner page
in the main document when proper precautions are taken.
Importantly, the code in the main file should ensure that the page counter
(as well as other status parameters which are stored in the |.aux| files)
takes the same value after the conditional processing.
Otherwise the page numbers may take divergent values
depending on which part is compiled.

For example, a title page could be declared by:
%
\begin{center}
\begin{tabular}{l}
|\ifchilddoc\||else|\\
|\addtocounter{page}{-1}|\\
\textit{code for title page}\\
|\newpage|\\
|\||fi|
\end{tabular}
\end{center}
%
A banner page for the child documents can be generated by:
%
\begin{center}
\begin{tabular}{l}
|\ifchilddoc|\\
|\addtocounter{page}{-1}|\\
\textit{code for banner page}\\
|\newpage|\\
|\||fi|
\end{tabular}
\end{center}
%
Here one could write a message such as:
\begin{center}
|This is the part \childdocname{} of \childdocjob{}.|
\end{center}

%%%%%%%%%%%%%%%%%%%%%%%%%%%%%%%%%%%%%%%%%%%%%%%%%%%%%%%%%%%%%%%%%%%%%%%%%%%%%%%%
\subsection{Flags}
\label{sec:flags}

The package makes it easy to generate different versions
of the main or child documents.
To this end compilation flags can be defined
and assigned different default values.
They will be particularly useful in conjunction
with the forwarding mechanism described in \secref{sec:forward}.

For example, it may be useful to have a flag |\version|
which can be set to |draft| or |final|.
The document source will contain some conditional code
depending on the value of |\version|.
Suppose further, the flag should default to |final| for the main file
and to |draft| for child files
which is a natural assignment for editing the document.
This is achieved by placing the following code
in the preamble of the main document
(below the |\childdocmain| directive):
%
\begin{center}
\begin{tabular}{l}
|\ifchilddoc|\\
|\providecommand{\version}{draft}|\\
|\||else|\\
|\providecommand{\version}{final}|\\
|\||fi|
\end{tabular}
\end{center}
%
The definition by |\providecommand| makes sure
that previous definitions are not overwritten.
Further statements |\providecommand{\version}{...}|
can thus be added before the above code to override it.

For the main file, one might add a line
(between |\childdocmain| and the above block)
%
\begin{center}
|%\ifchilddoc\||else\providecommand{\version}{draft}\||fi|
\end{center}
%
which can be uncommented to produce a draft version.
Likewise one can add a line to the very top of a child file
(above the |\childdocof{|\textit{main}|}| directive)
%
\begin{center}
|%\providecommand{\version}{final}|
\end{center}
%
which can be uncommented to produce the final version of this child document.

%%%%%%%%%%%%%%%%%%%%%%%%%%%%%%%%%%%%%%%%%%%%%%%%%%%%%%%%%%%%%%%%%%%%%%%%%%%%%%%%
\subsection{Forwarding}
\label{sec:forward}

Different versions of the main or child documents
using compilation flags as described in \secref{sec:flags}
can be (permanently) stored in different files
for convenient compilation, viewing and distribution.
To this end, the package defines a command
to pass on compilation to a different file:

%%%%%%%%%%%%%%%%%%%%%%%%%%%%%%%%%%%%%%%%
\DescribeMacro{\childdocforward}
The command |\childdocforward| redirects processing to
another source file:
%
\begin{center}
\begin{tabular}{l}
|\input{childdoc.def}|\\
|\childdocforward[|\textit{main}|]{|\textit{dest}|}|\\
\end{tabular}
\end{center}
%
The argument \textit{dest} is the destination file
(without extension).
It should be the main file or one of the child files.
Note that further \textsf{childdoc} directives
such as |\childdocof| and |\childdocforward|
in the indicated file will be processed in this form.
The optional argument \textit{main}
passes on directly to the main file \textit{main}
while pretending to compile the child \textit{dest}.
This form behaves as if \textit{dest}
issues |\childdocof{|\textit{main}|}| right away,
and no further \textsf{childdoc} directives will be processed.

%%%%%%%%%%%%%%%%%%%%%%%%%%%%%%%%%%%%%%%%
\DescribeMacro{\...prefix}
In the alternative form |\childdocforwardprefix|,
%
\begin{center}
\begin{tabular}{l}
|\input{childdoc.def}|\\
|\childdocforwardprefix[|\textit{main}|]{|\textit{prefix}|}{|\textit{dest}|}|
\end{tabular}
\end{center}
%
the destination file is determined by a pattern
depending on the current file:
To make this work, the current file must be called
`{\textit{prefix}\hspace{0.2em}\textit{suffix}}'
with \textit{prefix} matching precisely the argument.
Processing is then passed on to the file
`{\textit{dest}\hspace{0.2em}\textit{suffix}}'.
Surely, the same effect is achieved by
directly specifying the
argument `{\textit{dest}\hspace{0.2em}\textit{suffix}}'
in the first form.
However, that requires to set up a different file
for each child. With the alternative form of the command
all these files can have exactly the same content
which simplifies setting them up and maintaining them.

For example, the following file |draft.tex|
with a compilation flag |\version| as described in \secref{sec:flags}
compiles the main document as a draft:
%
\begin{center}
\begin{tabular}{l}
|\def\version{draft}|\\
|\input{childdoc.def}|\\
|\childdocforward{|\textit{main}|}|
\end{tabular}
\end{center}
%
Likewise, the following files |final|\textit{nn}|.tex|
compile the final version of the child document
|child|\textit{nn}|.tex|:
%
\begin{center}
\begin{tabular}{l}
|\def\version{final}|\\
|\input{childdoc.def}|\\
|\childdocforwardprefix{final}{child}|
\end{tabular}
\end{center}
%

Note that when several versions of a main file and/or of each child file
are to be generated, it may be convenient to set up a |Makefile| or
shell script to automatise the process.

%%%%%%%%%%%%%%%%%%%%%%%%%%%%%%%%%%%%%%%%%%%%%%%%%%%%%%%%%%%%%%%%%%%%%%%%%%%%%%%%
\subsection{Command Line Processing}
\label{sec:commandline}

The effect of redirection files can also be achieved by invoking
the \LaTeX{} compiler with a more elaborate command line.
Most conveniently this should be done as part
of a shell script or a |Makefile|.

When using \textsf{childdoc} in the main file, the following
command lines effectively perform a redirection
(note that depending on the shell being used,
backslashes may have to be doubled: `|\|' $\to$ `|\\|'):
%
\begin{center}
|... -jobname "|\textit{target}|" |\\|"|[\textit{flags}]%
|\input{childdoc.def}\childdocforward[|\textit{main}|]{|\textit{dest}|}"|
\end{center}
%
Here \textit{target} is the name of the output file,
\textit{main} is the name of the main file
and \textit{dest} is the name of the main or child file to be processed
(all filenames without extensions).
The optional argument \textit{main} can be omitted
if \textit{main} matches \textit{dest}.
Optionally, compilation \textit{flags} can be defined via |\def| commands.
This command line makes the \TeX{} engine believe
it is compiling the file \textit{target}
whose content is specified as the latter parameter.
The provided code then forwards the processing to
\textit{main} or \textit{dest} as described in \secref{sec:forward}.

%%%%%%%%%%%%%%%%%%%%%%%%%%%%%%%%%%%%%%%%%%%%%%%%%%%%%%%%%%%%%%%%%%%%%%%%%%%%%%%%
\subsection{Include by Input}
\label{sec:input}

Including child documents by |\include| has some restrictions by design.
Most notably, the content of a child document always occupies
its own set of pages; pages cannot be shared between child documents.
Usually, this behaviour makes perfect sense
because each child document contain an essential part of the document.
However, in some situations it may be desirable to compose
a document from a collection of parts
without having mandatory page breaks between then.
For this case, the package
provides a mechanism to include parts
by |\input| which can also be processed individually.
However, by construction this mechanism
requires manual handling of the content to be output.

%%%%%%%%%%%%%%%%%%%%%%%%%%%%%%%%%%%%%%%%
\DescribeMacro{\ifchilddocmanual}
The main file should be prepared as usual, see \secref{sec:include}.
However, the document body must make a distinction
between processing of an individual part and of the main document, e.g.:
%
\begin{center}
\begin{tabular}{l}
|\ifchilddocmanual|\\
|\input{\childdocname}|\\
|\||else|\\
\textit{document body with }|\input{|\textit{part}|}|\\
|\||fi|
\end{tabular}
\end{center}
%
The conditional |\ifchilddocmanual| is true whenever
a part to be included by |\input| is being compiled,
and the name of the part is stored in |\childdocname|.

%%%%%%%%%%%%%%%%%%%%%%%%%%%%%%%%%%%%%%%%
\DescribeMacro{\childdocby}
Each part to be included by |\input| should start with:
%
\begin{center}
\begin{tabular}{l}
|\input{childdoc.def}|\\
|\childdocby{|\textit{main}|}|\\
\end{tabular}
\end{center}
%
The directive |\childdocby| is similar to |\childdocof|
described in \secref{sec:include},
but the subsequent selection of content must be done manually.
To that end, both |\ifchilddoc| and |\ifchilddocmanual|
will be true upon processing of a part,
and the name of the part is stored in |\childdocname|.
Note that |\jobname| will be set to the filename of the current part
so that each part receives an individual |.aux| file
that does not interfere with the |.aux| file(s) of the main document.
This behaviour can be altered by the alternative form
|\childdocby[*]{|\textit{main}|}| (with a non-empty optional argument)
which uses the |.aux| file of the main document
by setting |\jobname| to \textit{main}.

%%%%%%%%%%%%%%%%%%%%%%%%%%%%%%%%%%%%%%%%%%%%%%%%%%%%%%%%%%%%%%%%%%%%%%%%%%%%%%%%
\subsection{Driver Development}
\label{sec:driver}

The \textsf{childdoc} mechanism can also be use for the development
of definition files such as \LaTeX{} styles or classes.
This case differs from the above setup with multiple parts
included by |\include| in that no |\includeonly| should be invoked.
This can be achieved by starting the include file
(before |\ProvidesPackage|) with:
%
\begin{center}
\begin{tabular}{l}
|\input{childdoc.def}|\\
|\childdocforward{|\textit{main}|}|\\
\end{tabular}
\end{center}
%
or alternatively with:
%
\begin{center}
\begin{tabular}{l}
|\input{childdoc.def}|\\
|\childdocby{|\textit{main}|}|\\
\end{tabular}
\end{center}
%
Both forms have slightly different effects as described above.
The main file is prepared as usual, see \secref{sec:include}.

%%%%%%%%%%%%%%%%%%%%%%%%%%%%%%%%%%%%%%%%%%%%%%%%%%%%%%%%%%%%%%%%%%%%%%%%%%%%%%%%
\subsection{Legacy Detection}
\label{sec:detection}

The directive |\childdocmain| in the main file can detect
whether the complete document or merely a child is to be compiled
even without using the directive |\childdocof|.
This method is deprecated because it is less robust
and there is no compelling reason to use it;
it is merely provided for backward compatibility
and it may be removed in future versions.

If the detection mechanism is to be used,
it is mandatory to correctly specify
the filename of the main file as the argument of |\childdocmain|:
%
\begin{center}
\begin{tabular}{l}
|\input{childdoc.def}|\\
|\childdocmain{|\textit{main}|}|\\
\end{tabular}
\end{center}
%
If |\jobname| does not match the argument \textit{main} of |\childdocmain|,
it is assumed that |\jobname| points to the child file to be compiled.
When using |\childdocmain| with the main file specified as argument,
it suffices to start a child file
with just |\input{|\textit{main}|}|
without loading of the package and using |\childdocof|.
If instead all processing is done
with the appropriate \textsf{childdoc} directives,
the argument of \textit{main} of |\childdocmain| can be empty.

An alternative version of the command line processing described
in \secref{sec:commandline} using the detection mechanism reads:
%
\begin{center}
|... -jobname "|\textit{target}|" "|[\textit{flags}]%
[|\def\jobname{|\textit{dest}|}|]|\input{|\textit{main}|}"|
\end{center}

%%%%%%%%%%%%%%%%%%%%%%%%%%%%%%%%%%%%%%%%%%%%%%%%%%%%%%%%%%%%%%%%%%%%%%%%%%%%%%%%
\subsection{Manual Code}
\label{sec:manual}

In case one cannot be certain whether the definitions file |childdoc.def|
is installed on the target \TeX{} distribution
and one prefers not to ship it,
it is conceivable to paste a few relevant commands into the sources.

To that end, drop all statements |\input{childdoc.def}|
and perform the replacements as outlined below.
Instead of |\childdocmain{|\textit{main}|}| add the following code
to the top of the main file:
%
\begin{center}
\begin{tabular}{l}
|\||ifdefined\childdocname\endinput\||fi\newif\ifchilddoc|\\
|\edef\childdocname{\scantokens\expandafter{\jobname\noexpand}}|\\
|\def\childdocmain{|\textit{main}|}\||ifx\childdocmain\childdocname\||else|\\
|\childdoctrue\includeonly{\childdocname}\let\jobname\childdocmain\||fi|\\
\end{tabular}
\end{center}
%
Instead of |\childdocof{|\textit{main}|}| just include the main file
at the top of each child file:
%
\begin{center}
|\input{|\textit{main}|}|
\end{center}
%
A simple redirection |\childdocforward{|\textit{dest}|}| is achieved by:
%
\begin{center}
|\def\jobname{|\textit{dest}|}\input{\jobname}|
\end{center}
%
The redirection with prefix
|\childdocforwardprefix[|\textit{prefix}|]{|\textit{dest}|}|
is accomplished by:
%
\begin{center}
\begin{tabular}{l}
|{\edef\jobname{\scantokens\expandafter{\jobname\noexpand}}|\\
|\def\redirectjob |\textit{prefix}|#1~~~{\gdef\jobname{|\textit{dest}|#1}}|\\
|\expandafter\redirectjob\jobname~~~}\input{\jobname}|
\end{tabular}
\end{center}

In an alternative approach,
child documents can be compiled by a specific command line
without additional code or specific definitions:
%
\begin{center}
|... -jobname "|\textit{target}|" "|[\textit{flags}]%
|\includeonly{|\textit{dest}|}\input{|\textit{main}|}"|
\end{center}
%

%%%%%%%%%%%%%%%%%%%%%%%%%%%%%%%%%%%%%%%%%%%%%%%%%%%%%%%%%%%%%%%%%%%%%%%%%%%%%%%%
%%%%%%%%%%%%%%%%%%%%%%%%%%%%%%%%%%%%%%%%%%%%%%%%%%%%%%%%%%%%%%%%%%%%%%%%%%%%%%%%
\section{Information}

%%%%%%%%%%%%%%%%%%%%%%%%%%%%%%%%%%%%%%%%%%%%%%%%%%%%%%%%%%%%%%%%%%%%%%%%%%%%%%%%
\subsection{Copyright}

Copyright \copyright{} 2017--2018 Niklas Beisert

This work may be distributed and/or modified under the
conditions of the \LaTeX{} Project Public License, either version 1.3
of this license or (at your option) any later version.
The latest version of this license is in
  \url{http://www.latex-project.org/lppl.txt}
and version 1.3 or later is part of all distributions of \LaTeX{}
version 2005/12/01 or later.

This work has the LPPL maintenance status `maintained'.

The Current Maintainer of this work is Niklas Beisert.

This work consists of the files |README.txt|, |childdoc.ins| and |childdoc.dtx|
as well as the derived files |childdoc.def|, |cdocsamp.tex|
with |cdocsch1.tex|, |cdocsch2.tex|, |cdocspt3.tex|, |cdocspt4.tex|,
|cdocsdrf.tex|, |cdocsfn1.tex|, |cdocsfn2.tex|
as well as |childdoc.pdf|.

%%%%%%%%%%%%%%%%%%%%%%%%%%%%%%%%%%%%%%%%%%%%%%%%%%%%%%%%%%%%%%%%%%%%%%%%%%%%%%%%
\subsection{Files and Installation}

The package consists of the files:
%
\begin{center}
\begin{tabular}{ll}
    |README.txt|   & readme file \\
    |childdoc.ins| & installation file \\
    |childdoc.dtx| & source file \\
    |childdoc.def| & definition file \\
    |cdocsamp.tex| & sample main file \\
    |cdocsch1.tex| & sample include file \\
    |cdocsch2.tex| & sample include file \\
    |cdocspt3.tex| & sample part file \\
    |cdocspt4.tex| & sample part file \\
    |cdocsdrf.tex| & sample redirection file \\
    |cdocsfn1.tex| & sample redirection file \\
    |cdocsfn2.tex| & sample redirection file \\
    |childdoc.pdf| & manual
\end{tabular}
\end{center}
%
The distribution consists of the files
|README.txt|, |childdoc.ins| and |childdoc.dtx|.
%
\begin{itemize}
\item
Run (pdf)\LaTeX{} on |childdoc.dtx|
to compile the manual |childdoc.pdf| (this file).
\item
Run \LaTeX{} on |childdoc.ins| to create the definitions file |childdoc.def|
and the sample |cdocsamp.tex| with include files
|cdocsch1.tex|, |cdocsch2.tex|, |cdocspt3.tex|, |cdocspt4.tex|,
|cdocsdrf.tex|, |cdocsfn1.tex|, |cdocsfn2.tex|.
Then copy the file |childdoc.def| to an appropriate directory of your \LaTeX{}
distribution, e.g.\ \textit{texmf-root}|/tex/latex/childdoc|.
\end{itemize}

%%%%%%%%%%%%%%%%%%%%%%%%%%%%%%%%%%%%%%%%%%%%%%%%%%%%%%%%%%%%%%%%%%%%%%%%%%%%%%%%
\subsection{Related CTAN Packages}

There are several other packages which offer a similar functionality:
%
\begin{itemize}
\item
The packages
\href{http://ctan.org/pkg/docmute}{\textsf{docmute}},
\href{http://ctan.org/pkg/includex}{\textsf{includex}} and
\href{http://ctan.org/pkg/standalone}{\textsf{standalone}}
provide commands to include only the document body of
a child file thus allowing both files to be compiled individually.
\item
The packages \href{http://ctan.org/pkg/subdocs}{\textsf{subdocs}}
and \href{http://ctan.org/pkg/subfiles}{\textsf{subfiles}}
provide structures in which the main and child documents can be
encapsulated and allowing them to be compiled individually.
The inclusion mechanism is different from the conventional |\include|.
\item
The package \href{http://ctan.org/pkg/combine}{\textsf{combine}}
is an elaborate solution to combine several documents into one.
\end{itemize}
%
See also the CTAN topic \href{http://ctan.org/topic/subdocs}{\textsf{subdocs}}
for further related packages.
The present package differs from the above solutions in that
a document structure constructed with the conventional |\include| mechanism
just needs two extra commands at the top of every file
such that all constituent files can be compiled individually.

%%%%%%%%%%%%%%%%%%%%%%%%%%%%%%%%%%%%%%%%%%%%%%%%%%%%%%%%%%%%%%%%%%%%%%%%%%%%%%%%
%\subsection{Feature Suggestions}
%
%The following is a list of features which may be useful for future
%versions of this package:
%%
%\begin{itemize}
%\item
%\ldots
%\end{itemize}

%%%%%%%%%%%%%%%%%%%%%%%%%%%%%%%%%%%%%%%%%%%%%%%%%%%%%%%%%%%%%%%%%%%%%%%%%%%%%%%%
\subsection{Revision History}

%%%%%%%%%%%%%%%%%%%%%%%%%%%%%%%%%%%%%%%%
\paragraph{v2.0:} 2018/12/30

\begin{itemize}
\item
immediate forward processing
\item
added |\childdocby| mechanism
\item
manual restructured
\end{itemize}

%%%%%%%%%%%%%%%%%%%%%%%%%%%%%%%%%%%%%%%%
\paragraph{v1.6:} 2018/01/17

\begin{itemize}
\item
application for development of include files
\item
corrections to manual
\end{itemize}

%%%%%%%%%%%%%%%%%%%%%%%%%%%%%%%%%%%%%%%%
\paragraph{v1.5:} 2017/05/21

\begin{itemize}
\item
more complete structuring introduced
\item
|\childdocof| introduced
\item
|\childdoc| renamed to |\childdocmain|
\item
|\childredirect| renamed to |\childdocforward| and |\childdocforwardprefix|
and functionality expanded
\end{itemize}

%%%%%%%%%%%%%%%%%%%%%%%%%%%%%%%%%%%%%%%%
\paragraph{v1.0:} 2017/04/27

\begin{itemize}
\item
manual and install package
\item
first version published on CTAN
\end{itemize}

%%%%%%%%%%%%%%%%%%%%%%%%%%%%%%%%%%%%%%%%
\paragraph{v0.6:} 2017/04/26

\begin{itemize}
\item
redirection mechanism added
\end{itemize}

%%%%%%%%%%%%%%%%%%%%%%%%%%%%%%%%%%%%%%%%
\paragraph{v0.5:} 2017/04/26

\begin{itemize}
\item
functionality in definition file
\end{itemize}


%%%%%%%%%%%%%%%%%%%%%%%%%%%%%%%%%%%%%%%%%%%%%%%%%%%%%%%%%%%%%%%%%%%%%%%%%%%%%%%%
%%%%%%%%%%%%%%%%%%%%%%%%%%%%%%%%%%%%%%%%%%%%%%%%%%%%%%%%%%%%%%%%%%%%%%%%%%%%%%%%
%%%%%%%%%%%%%%%%%%%%%%%%%%%%%%%%%%%%%%%%%%%%%%%%%%%%%%%%%%%%%%%%%%%%%%%%%%%%%%%%
\appendix

\settowidth\MacroIndent{\rmfamily\scriptsize 000\ }

 \DocInput{childdoc.dtx}

\end{document}
%</driver>
% \fi
%
% %%%%%%%%%%%%%%%%%%%%%%%%%%%%%%%%%%%%%%%%%%%%%%%%%%%%%%%%%%%%%%%%%%%%%%%%%%%%%%
% %%%%%%%%%%%%%%%%%%%%%%%%%%%%%%%%%%%%%%%%%%%%%%%%%%%%%%%%%%%%%%%%%%%%%%%%%%%%%%
% \section{Sample}
%\iffalse
%<*samplemain>
%\fi
%
% The following presents a sample document
% with two chapters, two parts, a title page,
% a compile flag as well as three forwarding files to set the flag.
% It consists of eight |.tex| files:
% \begin{center}
% \begin{tabular}{ll}
% |cdocsamp.tex|&main file\\
% |cdocsch1.tex|&include file for chapter 1\\
% |cdocsch2.tex|&include file for chapter 2\\
% |cdocspt3.tex|&include file for part 3\\
% |cdocspt4.tex|&include file for part 4\\
% |cdocsdrf.tex|&forwarding file for main file in draft mode\\
% |cdocsfi1.tex|&forwarding file for final version of chapter 1\\
% |cdocsfi2.tex|&forwarding file for final version of chapter 2\\
% \end{tabular}
% \end{center}
% Each of the eight files can be compiled directly by the \LaTeX{} compiler.
%
% %%%%%%%%%%%%%%%%%%%%%%%%%%%%%%%%%%%%%%
% \paragraph{Main File.}
%
% The main file is called |cdocsamp.tex|.
%
% Load the \textsf{childdoc} definitions and
% declare the filename for the main document:
%    \begin{macrocode}
\input{childdoc.def}
\childdocmain{}
%    \end{macrocode}

% Optional override for |\version| flag:
%    \begin{macrocode}
%%\ifchilddoc\else\providecommand{\version}{draft}\fi
%    \end{macrocode}

% Define the default values for the |\version| flag
% (|final| for the main file and |draft| for childs):
%    \begin{macrocode}
\ifchilddoc
\providecommand{\version}{draft}
\else
\providecommand{\version}{final}
\fi
%    \end{macrocode}

% Load the standard document class:
%    \begin{macrocode}
\documentclass[12pt]{article}
%    \end{macrocode}

% Start the document body:
%    \begin{macrocode}
\begin{document}
%    \end{macrocode}

% Declare a title page.
% Print title, part of document being processed and version flag:
%    \begin{macrocode}
\addtocounter{page}{-1}
\begin{center}
{\LARGE\bfseries{}childdoc example\par}
\vspace{1cm}
\ifchilddoc
\ifchilddocmanual part\else chapter\fi:
`\childdocname' of `\childdocjob'\par
\else
main document: `\childdocjob'\par
\fi
version: \version\par
\end{center}
\newpage
%    \end{macrocode}

% Manually include selected file,
% otherwise process as usual:
%    \begin{macrocode}
\ifchilddocmanual
\section*{part `\childdocname'}
\input{\childdocname}
\else
%    \end{macrocode}

% Include the two chapters:
%    \begin{macrocode}
\include{cdocsch1}
\include{cdocsch2}
%    \end{macrocode}

% Include the two parts unless only chapters should be displayed:
%    \begin{macrocode}
\ifchilddoc\else
\section{part three}
\input{cdocspt3}
\section{part four}
\input{cdocspt4}
\fi
%    \end{macrocode}

% Process as usual until here:
%    \begin{macrocode}
\fi
%    \end{macrocode}

% End of document body:
%    \begin{macrocode}
\end{document}
%    \end{macrocode}
%\iffalse
%</samplemain>
%\fi
%
% %%%%%%%%%%%%%%%%%%%%%%%%%%%%%%%%%%%%%%
% \paragraph{Chapter Include Files.}
%
% The include files are called |cdocsch1.tex| and |cdocsch2.tex|.
%
%\iffalse
%<*samplechap1|samplechap2>
%\fi

% Optional override for |\version| flag:
%    \begin{macrocode}
%%\providecommand{\version}{final}
%    \end{macrocode}

% Include the main document:
%    \begin{macrocode}
\input{childdoc.def}
\childdocof{cdocsamp}
%    \end{macrocode}

%\iffalse
%</samplechap1|samplechap2>
%\fi
%
%\iffalse
%<*samplechap1>
%\fi
% Some text for chapter 1:
%    \begin{macrocode}
\section{one}
some text in chapter one
%    \end{macrocode}

%\iffalse
%</samplechap1>
%\fi
% Some text for chapter 2:
%\iffalse
%<*samplechap2>
%\fi
%    \begin{macrocode}
\section{two}
more text in chapter two
%    \end{macrocode}

%\iffalse
%</samplechap2>
%\fi
%
% %%%%%%%%%%%%%%%%%%%%%%%%%%%%%%%%%%%%%%
% \paragraph{Part Include Files.}
%
% The include files are called |cdocspt3.tex| and |cdocspt4.tex|.
%
%\iffalse
%<*samplepart3|samplepart4>
%\fi

% Optional override for |\version| flag:
%    \begin{macrocode}
%%\providecommand{\version}{final}
%    \end{macrocode}

% Include the main document:
%    \begin{macrocode}
\input{childdoc.def}
\childdocby{cdocsamp}
%    \end{macrocode}

%\iffalse
%</samplepart3|samplepart4>
%\fi
%
%\iffalse
%<*samplepart3>
%\fi
% Some text for part 3:
%    \begin{macrocode}
some text in part three
%    \end{macrocode}

%\iffalse
%</samplepart3>
%\fi
% Some text for part 4:
%\iffalse
%<*samplepart4>
%\fi
%    \begin{macrocode}
more text in part four
%    \end{macrocode}

%\iffalse
%</samplepart4>
%\fi
%
% %%%%%%%%%%%%%%%%%%%%%%%%%%%%%%%%%%%%%%
% \paragraph{Forwarding for a Complete Draft.}
%
% The following forwarding file |cdocsdrf.tex|
% compiles the main document in draft mode:
%\iffalse
%<*sampledraft>
%\fi
%    \begin{macrocode}
\def\version{draft}
\input{childdoc.def}
\childdocforward{cdocsamp}
%    \end{macrocode}

%\iffalse
%</sampledraft>
%\fi
%
% %%%%%%%%%%%%%%%%%%%%%%%%%%%%%%%%%%%%%%
% \paragraph{Forwarding for Final Version of the Chapters.}
%
% The following forwarding files |cdocsfn1.tex| and |cdocsfn2.tex|
% (with identical content)
% compile the final versions of the child documents
% |cdocsch1.tex| and |cdocsch2.tex|, respectively:
%\iffalse
%<*samplefinal>
%\fi
%    \begin{macrocode}
\def\version{final}
\input{childdoc.def}
\childdocforwardprefix[cdocsamp]{cdocsfn}{cdocsch}
%    \end{macrocode}

%\iffalse
%</samplefinal>
%\fi
%
% %%%%%%%%%%%%%%%%%%%%%%%%%%%%%%%%%%%%%%
% \paragraph{Command Line Processing.}
%
% The following three command lines generate the output files
% |cdocscld|, |cdocscl1| and |cdocscl2|
% which should be identical to
% |cdocsdrf|, |cdocsch1| and |cdocsfn2|, respectively:
% \begin{center}
% \begin{tabular}{l}
% |latex -jobname cdocscld \|\\
% |  "\def\version{draft}\input{childdoc.def}\childdocforward{cdocsamp}"|\\
% |latex -jobname cdocscl1 \|\\
% |  "\input{childdoc.def}\childdocforward[cdocsamp]{cdocsch1}"|\\
% |latex -jobname cdocscl2 \|\\
% |  "\def\version{final}\input{childdoc.def}\childdocforward{cdocsch2}"|
% \end{tabular}
% \end{center}
% Note that the trailing backslash on each first line
% merely continues the input to the second line
% (for convenient cut ant paste).
% Furthermore, the command |latex| can be replaced by any
% of its alternative versions such as |pdflatex|.
%
% %%%%%%%%%%%%%%%%%%%%%%%%%%%%%%%%%%%%%%%%%%%%%%%%%%%%%%%%%%%%%%%%%%%%%%%%%%%%%%
% %%%%%%%%%%%%%%%%%%%%%%%%%%%%%%%%%%%%%%%%%%%%%%%%%%%%%%%%%%%%%%%%%%%%%%%%%%%%%%
% \section{Implementation}
%\iffalse
%<*package>
%\fi
%
% This section describes the definitions file |childdoc.def|.

% The definitions cannot be loaded using |\usepackage| or |\RequirePackage|
% which has a mechanism to prevent loading a style file more than once.
% When loading the definitions by means of |\input|
% multiple instances have to be prevented manually:
%\iffalse
%This code needs to be before the `\ProvidesFile' directive
%which is defined at the beginning of this file.
%Therefore it is also placed there and commented out here.
%</package>
%<*discard>
%\fi
%    \begin{macrocode}
\ifdefined\childdocmain\endinput\fi
%    \end{macrocode}
%\iffalse
%</discard>
%<*package>
%\fi
%
% \macro{\ifchilddoc}
% \macro{\ifchilddocmanual}
% The conditional |\ifchilddoc| tells whether a
% child (true) or main (false) document is being compiled.
% The conditional |\ifchilddocmanual| tells whether
% the |\includeonly| mechanism is used (false) or
% the selection of child files must be performed manually (true).
% The definitions initialise to false:
%    \begin{macrocode}
\newif\ifchilddoc
\newif\ifchilddocmanual
%    \end{macrocode}

% \macro{\childdocname}
% \macro{\childdocjob}
% The macro |\childdocname| stores the name of the main document
% to be compiled. The macro |\childdocjob| stores the name of
% the document on which the \LaTeX{} compiler was originally invoked.
% The content of |\jobname| cannot be compared
% to filenames specified in the source due to different catcodes.
% The following code rescans |\jobname|, stores the result
% in |\childdocname| and saves a copy in |\childdocjob|:
%    \begin{macrocode}
\edef\childdocname{\scantokens\expandafter{\jobname\noexpand}}
\let\childdocjob\childdocname
%    \end{macrocode}

% \macro{\childdocdisable}
% The macro |\childdocdisable| prevents the main file
% from being processed more than once.
% At this stage, the main document command |\childdocmain|
% is assumed to be called once again where it should do nothing.
% Any subsequent call to it should prevent
% a secondary processing of the main document
% It overwrites the forwarding commands
% |\childdocof| and |\childdocforward|
% with empty macros to prevent further inclusions of the main document:
%    \begin{macrocode}
\newcommand{\childdocdisable}
{
  \renewcommand{\childdocmain}[1]{\renewcommand{\childdocmain}[1]{\endinput}}
  \renewcommand{\childdocof}[1]{}
  \renewcommand{\childdocby}[2][]{}
  \renewcommand{\childdocforward}[2][]{}
  \renewcommand{\childdocdisable}{}
}
%    \end{macrocode}

% \macro{\childdocmain}
% The macro |\childdocmain| is to be called at the top of the main file
% with nothing or the main filename (without extension) as argument.
% First, it breaks loops.
% If the argument is not empty and does not match |\childdocname|
% (which is set by the first inclusion of |childdoc.def|),
% |\ifchilddoc| is set to true, |\includeonly| is applied to the child file
% and |\jobname| is set to the main file
% (for proper handling of |.aux| files):
%    \begin{macrocode}
\newcommand{\childdocmain}[1]
{
  \childdocdisable\childdocmain{}
  \if?#1?\else
    \begingroup
      \def\childdoctmp{#1}
      \ifx\childdoctmp\childdocname
        \def\childdoctmp{}
      \else
        \def\childdoctmp
        {
          \childdoctrue
          \includeonly{\childdocname}
          \def\childdocjob{#1}
          \def\jobname{#1}
        }
      \fi
      \expandafter
    \endgroup
    \childdoctmp
  \fi
}
%    \end{macrocode}

% \macro{\childdocof}
% The command |\childdocof| redirects
% compilation to the main file |#1|.
%    \begin{macrocode}
\newcommand{\childdocof}[1]
{
  \childdocdisable
  \childdoctrue
  \includeonly{\childdocname}
  \def\jobname{#1}
  \def\childdocjob{#1}
  \input{#1}
}
%    \end{macrocode}

% \macro{\childdocby}
% The command |\childdocby| ....
%    \begin{macrocode}
\newcommand{\childdocby}[2][]
{
  \childdocdisable
  \childdoctrue
  \childdocmanualtrue
  \if?#1?\else
    \def\jobname{#2}
  \fi
  \def\childdocjob{#2}
  \input{#2}
  \endinput
}
%    \end{macrocode}

% \macro{\childdocforward}
% The command |\childdocforward| redirects
% compilation to the main file or
% (if the optional argument is given) a child file.
% Parameters are set as if the main file
% or a child file starting with |\childdocof| was compiled.
% Then compilation is handed over to the main file:
%    \begin{macrocode}
\newcommand{\childdocforward}[2][]
{
  \begingroup
    \if?#1?
      \def\childdoctmp
      {
        \def\childdocname{#2}
        \def\childdocjob{#2}
        \def\jobname{#2}
        \input{#2}
        \endinput
      }
    \else
      \def\childdoctmp
      {
        \childdocdisable
        \def\childdocname{#2}
        \childdoctrue
        \includeonly{#2}
        \def\childdocjob{#1}
        \def\jobname{#1}
        \input{#1}
        \endinput
      }
    \fi
    \expandafter
  \endgroup
  \childdoctmp
}
%    \end{macrocode}

% \macro{\childdocforwardprefix}
% The command |\childdocforwardprefix| redirects
% compilation to the main or a child file by means of a pattern.
% The prefix |#1| in the current filename is replaced by |#2|
% and the suffix of the current filename is kept
% (it is assumed that the filename does not contain the substring `|~~~|'
% which is used as a delimiter).
% Compilation is handed over to the new file by |\childdocforward|:
%    \begin{macrocode}
\newcommand{\childdocforwardprefix}[3][]
{
  \begingroup
    \def\childdocextract #2##1~~~{\def\childdoctmp{\childdocforward[#1]{#3##1}}}
    \expandafter\childdocextract\childdocname~~~
    \expandafter
  \endgroup
  \childdoctmp
}
%    \end{macrocode}

% \macro{\childdoc}
% The deprecated macro |\childdoc| is a legacy version of |\childdocmain|:
%    \begin{macrocode}
\newcommand{\childdoc}{\childdocmain}
%    \end{macrocode}

% \macro{\childdocredirect}
% The deprecated macro |\childdocredirect| is a legacy version
% of |\childdocforward| and |\childdocforwardprefix|:
%    \begin{macrocode}
\newcommand{\childdocredirect}[2][]
{
  \begingroup
    \if?#1?
      \def\childdoctmp{\childdocforward{#2}}
    \else
      \def\childdoctmp{\childdocforwardprefix{#1}{#2}}
    \fi
    \expandafter
  \endgroup
  \childdoctmp
}
%    \end{macrocode}

%\iffalse
%</package>
%\fi
%
\endinput
|\\
|\childdocmain{|\textit{main}|}|\\
\end{tabular}
\end{center}
%
If |\jobname| does not match the argument \textit{main} of |\childdocmain|,
it is assumed that |\jobname| points to the child file to be compiled.
When using |\childdocmain| with the main file specified as argument,
it suffices to start a child file
with just |\input{|\textit{main}|}|
without loading of the package and using |\childdocof|.
If instead all processing is done
with the appropriate \textsf{childdoc} directives,
the argument of \textit{main} of |\childdocmain| can be empty.

An alternative version of the command line processing described
in \secref{sec:commandline} using the detection mechanism reads:
%
\begin{center}
|... -jobname "|\textit{target}|" "|[\textit{flags}]%
[|\def\jobname{|\textit{dest}|}|]|\input{|\textit{main}|}"|
\end{center}

%%%%%%%%%%%%%%%%%%%%%%%%%%%%%%%%%%%%%%%%%%%%%%%%%%%%%%%%%%%%%%%%%%%%%%%%%%%%%%%%
\subsection{Manual Code}
\label{sec:manual}

In case one cannot be certain whether the definitions file |childdoc.def|
is installed on the target \TeX{} distribution
and one prefers not to ship it,
it is conceivable to paste a few relevant commands into the sources.

To that end, drop all statements |% \iffalse
%
% childdoc.dtx Copyright (C) 2017-2018 Niklas Beisert
%
% This work may be distributed and/or modified under the
% conditions of the LaTeX Project Public License, either version 1.3
% of this license or (at your option) any later version.
% The latest version of this license is in
%   http://www.latex-project.org/lppl.txt
% and version 1.3 or later is part of all distributions of LaTeX
% version 2005/12/01 or later.
%
% This work has the LPPL maintenance status `maintained'.
%
% The Current Maintainer of this work is Niklas Beisert.
%
% This work consists of the files childdoc.dtx and childdoc.ins
% and the derived files childdoc.def and cdocsamp.tex with
% cdocsch1.tex, cdocsch2.tex, cdocsdrf.tex, cdocsfn1.tex, cdocsfn2.tex.
%
%<package>\ifdefined\childdocmain\endinput\fi
%<package>\ProvidesFile{childdoc.def}[2018/12/30 v2.0 child document driver]
%<samplemain>\ProvidesFile{cdocsamp.tex}[2018/12/30 v2.0 sample for childdoc]
%<*driver>
%\ProvidesFile{childdoc.drv}[2018/12/30 v2.0 childdoc reference manual file]
\PassOptionsToClass{10pt,a4paper}{article}
\documentclass{ltxdoc}

\usepackage[margin=35mm]{geometry}
\usepackage{hyperref}
\usepackage{hyperxmp}
\usepackage[usenames]{color}

\hypersetup{colorlinks=true}
\hypersetup{pdfstartview=FitH}
\hypersetup{pdfpagemode=UseNone}
\hypersetup{pdfsource={}}
\hypersetup{pdflang={en-UK}}
\hypersetup{pdfcopyright={Copyright 2017-2018 Niklas Beisert.
  This work may be distributed and/or modified under the
  conditions of the LaTeX Project Public License, either version 1.3
  of this license or (at your option) any later version.}}
\hypersetup{pdflicenseurl={http://www.latex-project.org/lppl.txt}}
\hypersetup{pdfcontactaddress={ETH Zurich, ITP, HIT K,
  Wolfgang-Pauli-Strasse 27}}
\hypersetup{pdfcontactpostcode={8093}}
\hypersetup{pdfcontactcity={Zurich}}
\hypersetup{pdfcontactcountry={Switzerland}}
\hypersetup{pdfcontactemail={nbeisert@itp.phys.ethz.ch}}
\hypersetup{pdfcontacturl={http://people.phys.ethz.ch/\xmptilde nbeisert/}}

\newcommand{\secref}[1]{\hyperref[#1]{section \ref*{#1}}}

\parskip1ex
\parindent0pt
\let\olditemize\itemize
\def\itemize{\olditemize\parskip0pt}

\begin{document}

\title{The \textsf{childdoc} Package}
\hypersetup{pdftitle={The childdoc Package}}
\author{Niklas Beisert\\[2ex]
  Institut f\"ur Theoretische Physik\\
  Eidgen\"ossische Technische Hochschule Z\"urich\\
  Wolfgang-Pauli-Strasse 27, 8093 Z\"urich, Switzerland\\[1ex]
  \href{mailto:nbeisert@itp.phys.ethz.ch}
  {\texttt{nbeisert@itp.phys.ethz.ch}}}
\hypersetup{pdfauthor={Niklas Beisert}}
\hypersetup{pdfsubject={Manual for the LaTeX2e Package childdoc}}
\date{30 December 2018, \textsf{v2.0}}
\maketitle

\begin{abstract}\noindent
\textsf{childdoc} is a \LaTeXe{} package
that enables the direct compilation
of document sections included by |\include|
to individual files.
\end{abstract}

\begingroup
\parskip0ex
\tableofcontents
\endgroup

%%%%%%%%%%%%%%%%%%%%%%%%%%%%%%%%%%%%%%%%%%%%%%%%%%%%%%%%%%%%%%%%%%%%%%%%%%%%%%%%
%%%%%%%%%%%%%%%%%%%%%%%%%%%%%%%%%%%%%%%%%%%%%%%%%%%%%%%%%%%%%%%%%%%%%%%%%%%%%%%%
\section{Introduction}

\LaTeX{} provides a mechanism to structure a large document (such as a book)
into a main file and several child files (containing the chapters)
using the |\include| command.
This mechanism is beneficial for documents
which span hundreds of pages in order to
make the source file(s) more manageable.
Moreover, compilation can be restricted to
selected child files by means of the |\includeonly| command.
The latter feature can be used to reduce the compilation time while editing
(this was significantly more useful in the earlier days of \LaTeX{})
or to generate a smaller document which is easier to navigate.
Another application of |\includeonly| is to generate
documents consisting of selected parts of the complete document.

However, there are a few drawbacks of the plain |\include| mechanism:
\begin{itemize}
\item
The child files cannot be compiled on their own,
they can only be compiled via the main file.
A naive editing environment
(such as a text editor with an option
to have the current file processed by \LaTeX)
may require one to switch to the main file before compiling;
attempting to compile the child file produces errors.
\item
The main file must be modified (each time)
to adjust the |\includeonly| command
to the present needs. This easily leaves the main file in a messy state.
\item
The generated document will always carry the filename
of the main document. This is inconvenient if
several child files are to be compiled and
to be kept for distribution.
\end{itemize}

The present package provides a simple interface
to make child files individually compilable by \LaTeX{}.
Compiling a child file then has the same effect as compiling
the main file with an |\includeonly| command
to select the appropriate child.
Moreover the generated document will carry the name of the child
rather than the main file.
This resolves all three above issues.

This feature is meant to make the editing of books,
thesis documents and lecture notes somewhat more convenient.
However, the package can also be used efficiently for
composing a series of documents (such as exercise sheets)
which are typically distributed individually.
It then assists the author in generating the individual documents
(potentially in different versions)
as well as a document containing the collected series.
Another application is in developing style files
or other kinds of included material
where compilation of the style file could redirect
to a sample or test file.

%%%%%%%%%%%%%%%%%%%%%%%%%%%%%%%%%%%%%%%%%%%%%%%%%%%%%%%%%%%%%%%%%%%%%%%%%%%%%%%%
%%%%%%%%%%%%%%%%%%%%%%%%%%%%%%%%%%%%%%%%%%%%%%%%%%%%%%%%%%%%%%%%%%%%%%%%%%%%%%%%
\section{Usage}

First of all, the package \textsf{childdoc} is \emph{not} a standard
\LaTeXe{} |.sty| style file! Therefore it needs to be invoked in
a non-standard way.

%%%%%%%%%%%%%%%%%%%%%%%%%%%%%%%%%%%%%%%%%%%%%%%%%%%%%%%%%%%%%%%%%%%%%%%%%%%%%%%%
\subsection{Included Files}
\label{sec:include}

%%%%%%%%%%%%%%%%%%%%%%%%%%%%%%%%%%%%%%%%
\DescribeMacro{\childdocmain}
To use the package, add the commands
\begin{center}
\begin{tabular}{l}
|\input{childdoc.def}|\\
|\childdocmain{}|\\
\end{tabular}
\end{center}
at the very top of the main \LaTeX{} file,
in particular \emph{before} the |\documentclass| statement!
The argument of |\childdocmain| should be left empty
(but it must be present).

%%%%%%%%%%%%%%%%%%%%%%%%%%%%%%%%%%%%%%%%
\DescribeMacro{\childdocof}
Furthermore, add the commands
\begin{center}
\begin{tabular}{l}
|\input{childdoc.def}|\\
|\childdocof{|\textit{main}|}|\\
\end{tabular}
\end{center}
at the top of every child file \textit{child}
which is included by |\include{|\textit{child}|}|
from within the main file
(or at least for those files to be compiled individually).
The argument \textit{main} must be the filename of the main file.

There are a couple of
considerations in setting up the main and child documents:

%%%%%%%%%%%%%%%%%%%%%%%%%%%%%%%%%%%%%%%%
\paragraph{Restrictions.}

Please note the following restrictions:
\begin{itemize}
\item
|\childdocmain| must be called with one argument \textit{main}
to ensure compatibility with earlier version of the package.
It must either be empty (|\childdocmain{}|)
or precisely match the filename of the main file in which it is specified.
See \secref{sec:detection} for further information.
\item
The filename \textit{main} must be specified without the |.tex| extension.
\item
The filename \textit{main} is case sensitive
(even in case-insensitive file systems)
due to internal string comparison.
\item
The argument \textit{main} should be fully expanded, it cannot be a macro.
\item
Subdirectories and special characters should be avoided in filenames.
\item
The command |\childdocmain{|\textit{main}|}| must be followed by a whitespace.
It should not be followed immediately by another command
or by a comment mark `|%|'.
This is because the \TeX{} parser reads the token immediately following
the argument of |\childdocmain| and puts it
at the beginning of every child section;
however, a white\-space is ignored.
\end{itemize}

%%%%%%%%%%%%%%%%%%%%%%%%%%%%%%%%%%%%%%%%
\paragraph{Content of Main File.}

It is advisable to place all content in the child files included by |\include|.
Any output contained in the main file will appear in all child documents
unless suppressed manually;
it cannot be suppressed automatically by the |\includeonly| directive
and thus should normally be avoided.
A method to include some content in the main file
by means of conditional processing is described in \secref{sec:conditional}.

%%%%%%%%%%%%%%%%%%%%%%%%%%%%%%%%%%%%%%%%
\paragraph{Page Numbering.}

When only a part of the document is compiled,
the appropriate numbering of pages
(as well as other status parameters)
is determined from the |.aux| files.
The latter contain information from previous passes.
However this information needs to propagate through
all intermediate child documents.
Therefore the page numbering in child documents may well
be inconsistent until the complete document is compiled at least once.

A useful (if unconventional) way to always ensure a consistent
page numbering is to restart the numbering in each child document
and denote the pages by `\textit{child}|.|\textit{page}'
where \textit{child} represents the chapter/section number of the child file.
This can be achieved by the command
|\numberwithin{page}{|\textit{child}|}|
of the \textsf{amsmath} package
where \textit{child} can be |chapter| or |section|
depending on the chosen structuring.
Alternatively, one can modify the macro |\thepage| appropriately
and reset the counter |page| at the start of each child file.

%%%%%%%%%%%%%%%%%%%%%%%%%%%%%%%%%%%%%%%%%%%%%%%%%%%%%%%%%%%%%%%%%%%%%%%%%%%%%%%%
\subsection{Conditional Processing}
\label{sec:conditional}

The package provides a mechanism to compile different versions
of a document. To customise the versions further some conditional processing
can come in handy to distinguish which version is being compiled.
The package provides two macros to describe the compilation context:

%%%%%%%%%%%%%%%%%%%%%%%%%%%%%%%%%%%%%%%%
\DescribeMacro{\ifchilddoc}
The conditional |\ifchilddoc| distinguishes between the compilation of
child documents and the main document:
%
\begin{center}
|\ifchilddoc |\textit{child-code}| |[|\||else |\textit{main-code}]| \||fi|
\end{center}

%%%%%%%%%%%%%%%%%%%%%%%%%%%%%%%%%%%%%%%%
\DescribeMacro{\childdocname}
\DescribeMacro{\childdocjob}
The macro |\childdocname| contains the filename (without extension)
of the main or child file being processed.
Note that |\childdocjob| will always contain the name of the main file.

%%%%%%%%%%%%%%%%%%%%%%%%%%%%%%%%%%%%%%%%
\paragraph{Title Page.}

Conditional processing can be used to include a title or banner page
in the main document when proper precautions are taken.
Importantly, the code in the main file should ensure that the page counter
(as well as other status parameters which are stored in the |.aux| files)
takes the same value after the conditional processing.
Otherwise the page numbers may take divergent values
depending on which part is compiled.

For example, a title page could be declared by:
%
\begin{center}
\begin{tabular}{l}
|\ifchilddoc\||else|\\
|\addtocounter{page}{-1}|\\
\textit{code for title page}\\
|\newpage|\\
|\||fi|
\end{tabular}
\end{center}
%
A banner page for the child documents can be generated by:
%
\begin{center}
\begin{tabular}{l}
|\ifchilddoc|\\
|\addtocounter{page}{-1}|\\
\textit{code for banner page}\\
|\newpage|\\
|\||fi|
\end{tabular}
\end{center}
%
Here one could write a message such as:
\begin{center}
|This is the part \childdocname{} of \childdocjob{}.|
\end{center}

%%%%%%%%%%%%%%%%%%%%%%%%%%%%%%%%%%%%%%%%%%%%%%%%%%%%%%%%%%%%%%%%%%%%%%%%%%%%%%%%
\subsection{Flags}
\label{sec:flags}

The package makes it easy to generate different versions
of the main or child documents.
To this end compilation flags can be defined
and assigned different default values.
They will be particularly useful in conjunction
with the forwarding mechanism described in \secref{sec:forward}.

For example, it may be useful to have a flag |\version|
which can be set to |draft| or |final|.
The document source will contain some conditional code
depending on the value of |\version|.
Suppose further, the flag should default to |final| for the main file
and to |draft| for child files
which is a natural assignment for editing the document.
This is achieved by placing the following code
in the preamble of the main document
(below the |\childdocmain| directive):
%
\begin{center}
\begin{tabular}{l}
|\ifchilddoc|\\
|\providecommand{\version}{draft}|\\
|\||else|\\
|\providecommand{\version}{final}|\\
|\||fi|
\end{tabular}
\end{center}
%
The definition by |\providecommand| makes sure
that previous definitions are not overwritten.
Further statements |\providecommand{\version}{...}|
can thus be added before the above code to override it.

For the main file, one might add a line
(between |\childdocmain| and the above block)
%
\begin{center}
|%\ifchilddoc\||else\providecommand{\version}{draft}\||fi|
\end{center}
%
which can be uncommented to produce a draft version.
Likewise one can add a line to the very top of a child file
(above the |\childdocof{|\textit{main}|}| directive)
%
\begin{center}
|%\providecommand{\version}{final}|
\end{center}
%
which can be uncommented to produce the final version of this child document.

%%%%%%%%%%%%%%%%%%%%%%%%%%%%%%%%%%%%%%%%%%%%%%%%%%%%%%%%%%%%%%%%%%%%%%%%%%%%%%%%
\subsection{Forwarding}
\label{sec:forward}

Different versions of the main or child documents
using compilation flags as described in \secref{sec:flags}
can be (permanently) stored in different files
for convenient compilation, viewing and distribution.
To this end, the package defines a command
to pass on compilation to a different file:

%%%%%%%%%%%%%%%%%%%%%%%%%%%%%%%%%%%%%%%%
\DescribeMacro{\childdocforward}
The command |\childdocforward| redirects processing to
another source file:
%
\begin{center}
\begin{tabular}{l}
|\input{childdoc.def}|\\
|\childdocforward[|\textit{main}|]{|\textit{dest}|}|\\
\end{tabular}
\end{center}
%
The argument \textit{dest} is the destination file
(without extension).
It should be the main file or one of the child files.
Note that further \textsf{childdoc} directives
such as |\childdocof| and |\childdocforward|
in the indicated file will be processed in this form.
The optional argument \textit{main}
passes on directly to the main file \textit{main}
while pretending to compile the child \textit{dest}.
This form behaves as if \textit{dest}
issues |\childdocof{|\textit{main}|}| right away,
and no further \textsf{childdoc} directives will be processed.

%%%%%%%%%%%%%%%%%%%%%%%%%%%%%%%%%%%%%%%%
\DescribeMacro{\...prefix}
In the alternative form |\childdocforwardprefix|,
%
\begin{center}
\begin{tabular}{l}
|\input{childdoc.def}|\\
|\childdocforwardprefix[|\textit{main}|]{|\textit{prefix}|}{|\textit{dest}|}|
\end{tabular}
\end{center}
%
the destination file is determined by a pattern
depending on the current file:
To make this work, the current file must be called
`{\textit{prefix}\hspace{0.2em}\textit{suffix}}'
with \textit{prefix} matching precisely the argument.
Processing is then passed on to the file
`{\textit{dest}\hspace{0.2em}\textit{suffix}}'.
Surely, the same effect is achieved by
directly specifying the
argument `{\textit{dest}\hspace{0.2em}\textit{suffix}}'
in the first form.
However, that requires to set up a different file
for each child. With the alternative form of the command
all these files can have exactly the same content
which simplifies setting them up and maintaining them.

For example, the following file |draft.tex|
with a compilation flag |\version| as described in \secref{sec:flags}
compiles the main document as a draft:
%
\begin{center}
\begin{tabular}{l}
|\def\version{draft}|\\
|\input{childdoc.def}|\\
|\childdocforward{|\textit{main}|}|
\end{tabular}
\end{center}
%
Likewise, the following files |final|\textit{nn}|.tex|
compile the final version of the child document
|child|\textit{nn}|.tex|:
%
\begin{center}
\begin{tabular}{l}
|\def\version{final}|\\
|\input{childdoc.def}|\\
|\childdocforwardprefix{final}{child}|
\end{tabular}
\end{center}
%

Note that when several versions of a main file and/or of each child file
are to be generated, it may be convenient to set up a |Makefile| or
shell script to automatise the process.

%%%%%%%%%%%%%%%%%%%%%%%%%%%%%%%%%%%%%%%%%%%%%%%%%%%%%%%%%%%%%%%%%%%%%%%%%%%%%%%%
\subsection{Command Line Processing}
\label{sec:commandline}

The effect of redirection files can also be achieved by invoking
the \LaTeX{} compiler with a more elaborate command line.
Most conveniently this should be done as part
of a shell script or a |Makefile|.

When using \textsf{childdoc} in the main file, the following
command lines effectively perform a redirection
(note that depending on the shell being used,
backslashes may have to be doubled: `|\|' $\to$ `|\\|'):
%
\begin{center}
|... -jobname "|\textit{target}|" |\\|"|[\textit{flags}]%
|\input{childdoc.def}\childdocforward[|\textit{main}|]{|\textit{dest}|}"|
\end{center}
%
Here \textit{target} is the name of the output file,
\textit{main} is the name of the main file
and \textit{dest} is the name of the main or child file to be processed
(all filenames without extensions).
The optional argument \textit{main} can be omitted
if \textit{main} matches \textit{dest}.
Optionally, compilation \textit{flags} can be defined via |\def| commands.
This command line makes the \TeX{} engine believe
it is compiling the file \textit{target}
whose content is specified as the latter parameter.
The provided code then forwards the processing to
\textit{main} or \textit{dest} as described in \secref{sec:forward}.

%%%%%%%%%%%%%%%%%%%%%%%%%%%%%%%%%%%%%%%%%%%%%%%%%%%%%%%%%%%%%%%%%%%%%%%%%%%%%%%%
\subsection{Include by Input}
\label{sec:input}

Including child documents by |\include| has some restrictions by design.
Most notably, the content of a child document always occupies
its own set of pages; pages cannot be shared between child documents.
Usually, this behaviour makes perfect sense
because each child document contain an essential part of the document.
However, in some situations it may be desirable to compose
a document from a collection of parts
without having mandatory page breaks between then.
For this case, the package
provides a mechanism to include parts
by |\input| which can also be processed individually.
However, by construction this mechanism
requires manual handling of the content to be output.

%%%%%%%%%%%%%%%%%%%%%%%%%%%%%%%%%%%%%%%%
\DescribeMacro{\ifchilddocmanual}
The main file should be prepared as usual, see \secref{sec:include}.
However, the document body must make a distinction
between processing of an individual part and of the main document, e.g.:
%
\begin{center}
\begin{tabular}{l}
|\ifchilddocmanual|\\
|\input{\childdocname}|\\
|\||else|\\
\textit{document body with }|\input{|\textit{part}|}|\\
|\||fi|
\end{tabular}
\end{center}
%
The conditional |\ifchilddocmanual| is true whenever
a part to be included by |\input| is being compiled,
and the name of the part is stored in |\childdocname|.

%%%%%%%%%%%%%%%%%%%%%%%%%%%%%%%%%%%%%%%%
\DescribeMacro{\childdocby}
Each part to be included by |\input| should start with:
%
\begin{center}
\begin{tabular}{l}
|\input{childdoc.def}|\\
|\childdocby{|\textit{main}|}|\\
\end{tabular}
\end{center}
%
The directive |\childdocby| is similar to |\childdocof|
described in \secref{sec:include},
but the subsequent selection of content must be done manually.
To that end, both |\ifchilddoc| and |\ifchilddocmanual|
will be true upon processing of a part,
and the name of the part is stored in |\childdocname|.
Note that |\jobname| will be set to the filename of the current part
so that each part receives an individual |.aux| file
that does not interfere with the |.aux| file(s) of the main document.
This behaviour can be altered by the alternative form
|\childdocby[*]{|\textit{main}|}| (with a non-empty optional argument)
which uses the |.aux| file of the main document
by setting |\jobname| to \textit{main}.

%%%%%%%%%%%%%%%%%%%%%%%%%%%%%%%%%%%%%%%%%%%%%%%%%%%%%%%%%%%%%%%%%%%%%%%%%%%%%%%%
\subsection{Driver Development}
\label{sec:driver}

The \textsf{childdoc} mechanism can also be use for the development
of definition files such as \LaTeX{} styles or classes.
This case differs from the above setup with multiple parts
included by |\include| in that no |\includeonly| should be invoked.
This can be achieved by starting the include file
(before |\ProvidesPackage|) with:
%
\begin{center}
\begin{tabular}{l}
|\input{childdoc.def}|\\
|\childdocforward{|\textit{main}|}|\\
\end{tabular}
\end{center}
%
or alternatively with:
%
\begin{center}
\begin{tabular}{l}
|\input{childdoc.def}|\\
|\childdocby{|\textit{main}|}|\\
\end{tabular}
\end{center}
%
Both forms have slightly different effects as described above.
The main file is prepared as usual, see \secref{sec:include}.

%%%%%%%%%%%%%%%%%%%%%%%%%%%%%%%%%%%%%%%%%%%%%%%%%%%%%%%%%%%%%%%%%%%%%%%%%%%%%%%%
\subsection{Legacy Detection}
\label{sec:detection}

The directive |\childdocmain| in the main file can detect
whether the complete document or merely a child is to be compiled
even without using the directive |\childdocof|.
This method is deprecated because it is less robust
and there is no compelling reason to use it;
it is merely provided for backward compatibility
and it may be removed in future versions.

If the detection mechanism is to be used,
it is mandatory to correctly specify
the filename of the main file as the argument of |\childdocmain|:
%
\begin{center}
\begin{tabular}{l}
|\input{childdoc.def}|\\
|\childdocmain{|\textit{main}|}|\\
\end{tabular}
\end{center}
%
If |\jobname| does not match the argument \textit{main} of |\childdocmain|,
it is assumed that |\jobname| points to the child file to be compiled.
When using |\childdocmain| with the main file specified as argument,
it suffices to start a child file
with just |\input{|\textit{main}|}|
without loading of the package and using |\childdocof|.
If instead all processing is done
with the appropriate \textsf{childdoc} directives,
the argument of \textit{main} of |\childdocmain| can be empty.

An alternative version of the command line processing described
in \secref{sec:commandline} using the detection mechanism reads:
%
\begin{center}
|... -jobname "|\textit{target}|" "|[\textit{flags}]%
[|\def\jobname{|\textit{dest}|}|]|\input{|\textit{main}|}"|
\end{center}

%%%%%%%%%%%%%%%%%%%%%%%%%%%%%%%%%%%%%%%%%%%%%%%%%%%%%%%%%%%%%%%%%%%%%%%%%%%%%%%%
\subsection{Manual Code}
\label{sec:manual}

In case one cannot be certain whether the definitions file |childdoc.def|
is installed on the target \TeX{} distribution
and one prefers not to ship it,
it is conceivable to paste a few relevant commands into the sources.

To that end, drop all statements |\input{childdoc.def}|
and perform the replacements as outlined below.
Instead of |\childdocmain{|\textit{main}|}| add the following code
to the top of the main file:
%
\begin{center}
\begin{tabular}{l}
|\||ifdefined\childdocname\endinput\||fi\newif\ifchilddoc|\\
|\edef\childdocname{\scantokens\expandafter{\jobname\noexpand}}|\\
|\def\childdocmain{|\textit{main}|}\||ifx\childdocmain\childdocname\||else|\\
|\childdoctrue\includeonly{\childdocname}\let\jobname\childdocmain\||fi|\\
\end{tabular}
\end{center}
%
Instead of |\childdocof{|\textit{main}|}| just include the main file
at the top of each child file:
%
\begin{center}
|\input{|\textit{main}|}|
\end{center}
%
A simple redirection |\childdocforward{|\textit{dest}|}| is achieved by:
%
\begin{center}
|\def\jobname{|\textit{dest}|}\input{\jobname}|
\end{center}
%
The redirection with prefix
|\childdocforwardprefix[|\textit{prefix}|]{|\textit{dest}|}|
is accomplished by:
%
\begin{center}
\begin{tabular}{l}
|{\edef\jobname{\scantokens\expandafter{\jobname\noexpand}}|\\
|\def\redirectjob |\textit{prefix}|#1~~~{\gdef\jobname{|\textit{dest}|#1}}|\\
|\expandafter\redirectjob\jobname~~~}\input{\jobname}|
\end{tabular}
\end{center}

In an alternative approach,
child documents can be compiled by a specific command line
without additional code or specific definitions:
%
\begin{center}
|... -jobname "|\textit{target}|" "|[\textit{flags}]%
|\includeonly{|\textit{dest}|}\input{|\textit{main}|}"|
\end{center}
%

%%%%%%%%%%%%%%%%%%%%%%%%%%%%%%%%%%%%%%%%%%%%%%%%%%%%%%%%%%%%%%%%%%%%%%%%%%%%%%%%
%%%%%%%%%%%%%%%%%%%%%%%%%%%%%%%%%%%%%%%%%%%%%%%%%%%%%%%%%%%%%%%%%%%%%%%%%%%%%%%%
\section{Information}

%%%%%%%%%%%%%%%%%%%%%%%%%%%%%%%%%%%%%%%%%%%%%%%%%%%%%%%%%%%%%%%%%%%%%%%%%%%%%%%%
\subsection{Copyright}

Copyright \copyright{} 2017--2018 Niklas Beisert

This work may be distributed and/or modified under the
conditions of the \LaTeX{} Project Public License, either version 1.3
of this license or (at your option) any later version.
The latest version of this license is in
  \url{http://www.latex-project.org/lppl.txt}
and version 1.3 or later is part of all distributions of \LaTeX{}
version 2005/12/01 or later.

This work has the LPPL maintenance status `maintained'.

The Current Maintainer of this work is Niklas Beisert.

This work consists of the files |README.txt|, |childdoc.ins| and |childdoc.dtx|
as well as the derived files |childdoc.def|, |cdocsamp.tex|
with |cdocsch1.tex|, |cdocsch2.tex|, |cdocspt3.tex|, |cdocspt4.tex|,
|cdocsdrf.tex|, |cdocsfn1.tex|, |cdocsfn2.tex|
as well as |childdoc.pdf|.

%%%%%%%%%%%%%%%%%%%%%%%%%%%%%%%%%%%%%%%%%%%%%%%%%%%%%%%%%%%%%%%%%%%%%%%%%%%%%%%%
\subsection{Files and Installation}

The package consists of the files:
%
\begin{center}
\begin{tabular}{ll}
    |README.txt|   & readme file \\
    |childdoc.ins| & installation file \\
    |childdoc.dtx| & source file \\
    |childdoc.def| & definition file \\
    |cdocsamp.tex| & sample main file \\
    |cdocsch1.tex| & sample include file \\
    |cdocsch2.tex| & sample include file \\
    |cdocspt3.tex| & sample part file \\
    |cdocspt4.tex| & sample part file \\
    |cdocsdrf.tex| & sample redirection file \\
    |cdocsfn1.tex| & sample redirection file \\
    |cdocsfn2.tex| & sample redirection file \\
    |childdoc.pdf| & manual
\end{tabular}
\end{center}
%
The distribution consists of the files
|README.txt|, |childdoc.ins| and |childdoc.dtx|.
%
\begin{itemize}
\item
Run (pdf)\LaTeX{} on |childdoc.dtx|
to compile the manual |childdoc.pdf| (this file).
\item
Run \LaTeX{} on |childdoc.ins| to create the definitions file |childdoc.def|
and the sample |cdocsamp.tex| with include files
|cdocsch1.tex|, |cdocsch2.tex|, |cdocspt3.tex|, |cdocspt4.tex|,
|cdocsdrf.tex|, |cdocsfn1.tex|, |cdocsfn2.tex|.
Then copy the file |childdoc.def| to an appropriate directory of your \LaTeX{}
distribution, e.g.\ \textit{texmf-root}|/tex/latex/childdoc|.
\end{itemize}

%%%%%%%%%%%%%%%%%%%%%%%%%%%%%%%%%%%%%%%%%%%%%%%%%%%%%%%%%%%%%%%%%%%%%%%%%%%%%%%%
\subsection{Related CTAN Packages}

There are several other packages which offer a similar functionality:
%
\begin{itemize}
\item
The packages
\href{http://ctan.org/pkg/docmute}{\textsf{docmute}},
\href{http://ctan.org/pkg/includex}{\textsf{includex}} and
\href{http://ctan.org/pkg/standalone}{\textsf{standalone}}
provide commands to include only the document body of
a child file thus allowing both files to be compiled individually.
\item
The packages \href{http://ctan.org/pkg/subdocs}{\textsf{subdocs}}
and \href{http://ctan.org/pkg/subfiles}{\textsf{subfiles}}
provide structures in which the main and child documents can be
encapsulated and allowing them to be compiled individually.
The inclusion mechanism is different from the conventional |\include|.
\item
The package \href{http://ctan.org/pkg/combine}{\textsf{combine}}
is an elaborate solution to combine several documents into one.
\end{itemize}
%
See also the CTAN topic \href{http://ctan.org/topic/subdocs}{\textsf{subdocs}}
for further related packages.
The present package differs from the above solutions in that
a document structure constructed with the conventional |\include| mechanism
just needs two extra commands at the top of every file
such that all constituent files can be compiled individually.

%%%%%%%%%%%%%%%%%%%%%%%%%%%%%%%%%%%%%%%%%%%%%%%%%%%%%%%%%%%%%%%%%%%%%%%%%%%%%%%%
%\subsection{Feature Suggestions}
%
%The following is a list of features which may be useful for future
%versions of this package:
%%
%\begin{itemize}
%\item
%\ldots
%\end{itemize}

%%%%%%%%%%%%%%%%%%%%%%%%%%%%%%%%%%%%%%%%%%%%%%%%%%%%%%%%%%%%%%%%%%%%%%%%%%%%%%%%
\subsection{Revision History}

%%%%%%%%%%%%%%%%%%%%%%%%%%%%%%%%%%%%%%%%
\paragraph{v2.0:} 2018/12/30

\begin{itemize}
\item
immediate forward processing
\item
added |\childdocby| mechanism
\item
manual restructured
\end{itemize}

%%%%%%%%%%%%%%%%%%%%%%%%%%%%%%%%%%%%%%%%
\paragraph{v1.6:} 2018/01/17

\begin{itemize}
\item
application for development of include files
\item
corrections to manual
\end{itemize}

%%%%%%%%%%%%%%%%%%%%%%%%%%%%%%%%%%%%%%%%
\paragraph{v1.5:} 2017/05/21

\begin{itemize}
\item
more complete structuring introduced
\item
|\childdocof| introduced
\item
|\childdoc| renamed to |\childdocmain|
\item
|\childredirect| renamed to |\childdocforward| and |\childdocforwardprefix|
and functionality expanded
\end{itemize}

%%%%%%%%%%%%%%%%%%%%%%%%%%%%%%%%%%%%%%%%
\paragraph{v1.0:} 2017/04/27

\begin{itemize}
\item
manual and install package
\item
first version published on CTAN
\end{itemize}

%%%%%%%%%%%%%%%%%%%%%%%%%%%%%%%%%%%%%%%%
\paragraph{v0.6:} 2017/04/26

\begin{itemize}
\item
redirection mechanism added
\end{itemize}

%%%%%%%%%%%%%%%%%%%%%%%%%%%%%%%%%%%%%%%%
\paragraph{v0.5:} 2017/04/26

\begin{itemize}
\item
functionality in definition file
\end{itemize}


%%%%%%%%%%%%%%%%%%%%%%%%%%%%%%%%%%%%%%%%%%%%%%%%%%%%%%%%%%%%%%%%%%%%%%%%%%%%%%%%
%%%%%%%%%%%%%%%%%%%%%%%%%%%%%%%%%%%%%%%%%%%%%%%%%%%%%%%%%%%%%%%%%%%%%%%%%%%%%%%%
%%%%%%%%%%%%%%%%%%%%%%%%%%%%%%%%%%%%%%%%%%%%%%%%%%%%%%%%%%%%%%%%%%%%%%%%%%%%%%%%
\appendix

\settowidth\MacroIndent{\rmfamily\scriptsize 000\ }

 \DocInput{childdoc.dtx}

\end{document}
%</driver>
% \fi
%
% %%%%%%%%%%%%%%%%%%%%%%%%%%%%%%%%%%%%%%%%%%%%%%%%%%%%%%%%%%%%%%%%%%%%%%%%%%%%%%
% %%%%%%%%%%%%%%%%%%%%%%%%%%%%%%%%%%%%%%%%%%%%%%%%%%%%%%%%%%%%%%%%%%%%%%%%%%%%%%
% \section{Sample}
%\iffalse
%<*samplemain>
%\fi
%
% The following presents a sample document
% with two chapters, two parts, a title page,
% a compile flag as well as three forwarding files to set the flag.
% It consists of eight |.tex| files:
% \begin{center}
% \begin{tabular}{ll}
% |cdocsamp.tex|&main file\\
% |cdocsch1.tex|&include file for chapter 1\\
% |cdocsch2.tex|&include file for chapter 2\\
% |cdocspt3.tex|&include file for part 3\\
% |cdocspt4.tex|&include file for part 4\\
% |cdocsdrf.tex|&forwarding file for main file in draft mode\\
% |cdocsfi1.tex|&forwarding file for final version of chapter 1\\
% |cdocsfi2.tex|&forwarding file for final version of chapter 2\\
% \end{tabular}
% \end{center}
% Each of the eight files can be compiled directly by the \LaTeX{} compiler.
%
% %%%%%%%%%%%%%%%%%%%%%%%%%%%%%%%%%%%%%%
% \paragraph{Main File.}
%
% The main file is called |cdocsamp.tex|.
%
% Load the \textsf{childdoc} definitions and
% declare the filename for the main document:
%    \begin{macrocode}
\input{childdoc.def}
\childdocmain{}
%    \end{macrocode}

% Optional override for |\version| flag:
%    \begin{macrocode}
%%\ifchilddoc\else\providecommand{\version}{draft}\fi
%    \end{macrocode}

% Define the default values for the |\version| flag
% (|final| for the main file and |draft| for childs):
%    \begin{macrocode}
\ifchilddoc
\providecommand{\version}{draft}
\else
\providecommand{\version}{final}
\fi
%    \end{macrocode}

% Load the standard document class:
%    \begin{macrocode}
\documentclass[12pt]{article}
%    \end{macrocode}

% Start the document body:
%    \begin{macrocode}
\begin{document}
%    \end{macrocode}

% Declare a title page.
% Print title, part of document being processed and version flag:
%    \begin{macrocode}
\addtocounter{page}{-1}
\begin{center}
{\LARGE\bfseries{}childdoc example\par}
\vspace{1cm}
\ifchilddoc
\ifchilddocmanual part\else chapter\fi:
`\childdocname' of `\childdocjob'\par
\else
main document: `\childdocjob'\par
\fi
version: \version\par
\end{center}
\newpage
%    \end{macrocode}

% Manually include selected file,
% otherwise process as usual:
%    \begin{macrocode}
\ifchilddocmanual
\section*{part `\childdocname'}
\input{\childdocname}
\else
%    \end{macrocode}

% Include the two chapters:
%    \begin{macrocode}
\include{cdocsch1}
\include{cdocsch2}
%    \end{macrocode}

% Include the two parts unless only chapters should be displayed:
%    \begin{macrocode}
\ifchilddoc\else
\section{part three}
\input{cdocspt3}
\section{part four}
\input{cdocspt4}
\fi
%    \end{macrocode}

% Process as usual until here:
%    \begin{macrocode}
\fi
%    \end{macrocode}

% End of document body:
%    \begin{macrocode}
\end{document}
%    \end{macrocode}
%\iffalse
%</samplemain>
%\fi
%
% %%%%%%%%%%%%%%%%%%%%%%%%%%%%%%%%%%%%%%
% \paragraph{Chapter Include Files.}
%
% The include files are called |cdocsch1.tex| and |cdocsch2.tex|.
%
%\iffalse
%<*samplechap1|samplechap2>
%\fi

% Optional override for |\version| flag:
%    \begin{macrocode}
%%\providecommand{\version}{final}
%    \end{macrocode}

% Include the main document:
%    \begin{macrocode}
\input{childdoc.def}
\childdocof{cdocsamp}
%    \end{macrocode}

%\iffalse
%</samplechap1|samplechap2>
%\fi
%
%\iffalse
%<*samplechap1>
%\fi
% Some text for chapter 1:
%    \begin{macrocode}
\section{one}
some text in chapter one
%    \end{macrocode}

%\iffalse
%</samplechap1>
%\fi
% Some text for chapter 2:
%\iffalse
%<*samplechap2>
%\fi
%    \begin{macrocode}
\section{two}
more text in chapter two
%    \end{macrocode}

%\iffalse
%</samplechap2>
%\fi
%
% %%%%%%%%%%%%%%%%%%%%%%%%%%%%%%%%%%%%%%
% \paragraph{Part Include Files.}
%
% The include files are called |cdocspt3.tex| and |cdocspt4.tex|.
%
%\iffalse
%<*samplepart3|samplepart4>
%\fi

% Optional override for |\version| flag:
%    \begin{macrocode}
%%\providecommand{\version}{final}
%    \end{macrocode}

% Include the main document:
%    \begin{macrocode}
\input{childdoc.def}
\childdocby{cdocsamp}
%    \end{macrocode}

%\iffalse
%</samplepart3|samplepart4>
%\fi
%
%\iffalse
%<*samplepart3>
%\fi
% Some text for part 3:
%    \begin{macrocode}
some text in part three
%    \end{macrocode}

%\iffalse
%</samplepart3>
%\fi
% Some text for part 4:
%\iffalse
%<*samplepart4>
%\fi
%    \begin{macrocode}
more text in part four
%    \end{macrocode}

%\iffalse
%</samplepart4>
%\fi
%
% %%%%%%%%%%%%%%%%%%%%%%%%%%%%%%%%%%%%%%
% \paragraph{Forwarding for a Complete Draft.}
%
% The following forwarding file |cdocsdrf.tex|
% compiles the main document in draft mode:
%\iffalse
%<*sampledraft>
%\fi
%    \begin{macrocode}
\def\version{draft}
\input{childdoc.def}
\childdocforward{cdocsamp}
%    \end{macrocode}

%\iffalse
%</sampledraft>
%\fi
%
% %%%%%%%%%%%%%%%%%%%%%%%%%%%%%%%%%%%%%%
% \paragraph{Forwarding for Final Version of the Chapters.}
%
% The following forwarding files |cdocsfn1.tex| and |cdocsfn2.tex|
% (with identical content)
% compile the final versions of the child documents
% |cdocsch1.tex| and |cdocsch2.tex|, respectively:
%\iffalse
%<*samplefinal>
%\fi
%    \begin{macrocode}
\def\version{final}
\input{childdoc.def}
\childdocforwardprefix[cdocsamp]{cdocsfn}{cdocsch}
%    \end{macrocode}

%\iffalse
%</samplefinal>
%\fi
%
% %%%%%%%%%%%%%%%%%%%%%%%%%%%%%%%%%%%%%%
% \paragraph{Command Line Processing.}
%
% The following three command lines generate the output files
% |cdocscld|, |cdocscl1| and |cdocscl2|
% which should be identical to
% |cdocsdrf|, |cdocsch1| and |cdocsfn2|, respectively:
% \begin{center}
% \begin{tabular}{l}
% |latex -jobname cdocscld \|\\
% |  "\def\version{draft}\input{childdoc.def}\childdocforward{cdocsamp}"|\\
% |latex -jobname cdocscl1 \|\\
% |  "\input{childdoc.def}\childdocforward[cdocsamp]{cdocsch1}"|\\
% |latex -jobname cdocscl2 \|\\
% |  "\def\version{final}\input{childdoc.def}\childdocforward{cdocsch2}"|
% \end{tabular}
% \end{center}
% Note that the trailing backslash on each first line
% merely continues the input to the second line
% (for convenient cut ant paste).
% Furthermore, the command |latex| can be replaced by any
% of its alternative versions such as |pdflatex|.
%
% %%%%%%%%%%%%%%%%%%%%%%%%%%%%%%%%%%%%%%%%%%%%%%%%%%%%%%%%%%%%%%%%%%%%%%%%%%%%%%
% %%%%%%%%%%%%%%%%%%%%%%%%%%%%%%%%%%%%%%%%%%%%%%%%%%%%%%%%%%%%%%%%%%%%%%%%%%%%%%
% \section{Implementation}
%\iffalse
%<*package>
%\fi
%
% This section describes the definitions file |childdoc.def|.

% The definitions cannot be loaded using |\usepackage| or |\RequirePackage|
% which has a mechanism to prevent loading a style file more than once.
% When loading the definitions by means of |\input|
% multiple instances have to be prevented manually:
%\iffalse
%This code needs to be before the `\ProvidesFile' directive
%which is defined at the beginning of this file.
%Therefore it is also placed there and commented out here.
%</package>
%<*discard>
%\fi
%    \begin{macrocode}
\ifdefined\childdocmain\endinput\fi
%    \end{macrocode}
%\iffalse
%</discard>
%<*package>
%\fi
%
% \macro{\ifchilddoc}
% \macro{\ifchilddocmanual}
% The conditional |\ifchilddoc| tells whether a
% child (true) or main (false) document is being compiled.
% The conditional |\ifchilddocmanual| tells whether
% the |\includeonly| mechanism is used (false) or
% the selection of child files must be performed manually (true).
% The definitions initialise to false:
%    \begin{macrocode}
\newif\ifchilddoc
\newif\ifchilddocmanual
%    \end{macrocode}

% \macro{\childdocname}
% \macro{\childdocjob}
% The macro |\childdocname| stores the name of the main document
% to be compiled. The macro |\childdocjob| stores the name of
% the document on which the \LaTeX{} compiler was originally invoked.
% The content of |\jobname| cannot be compared
% to filenames specified in the source due to different catcodes.
% The following code rescans |\jobname|, stores the result
% in |\childdocname| and saves a copy in |\childdocjob|:
%    \begin{macrocode}
\edef\childdocname{\scantokens\expandafter{\jobname\noexpand}}
\let\childdocjob\childdocname
%    \end{macrocode}

% \macro{\childdocdisable}
% The macro |\childdocdisable| prevents the main file
% from being processed more than once.
% At this stage, the main document command |\childdocmain|
% is assumed to be called once again where it should do nothing.
% Any subsequent call to it should prevent
% a secondary processing of the main document
% It overwrites the forwarding commands
% |\childdocof| and |\childdocforward|
% with empty macros to prevent further inclusions of the main document:
%    \begin{macrocode}
\newcommand{\childdocdisable}
{
  \renewcommand{\childdocmain}[1]{\renewcommand{\childdocmain}[1]{\endinput}}
  \renewcommand{\childdocof}[1]{}
  \renewcommand{\childdocby}[2][]{}
  \renewcommand{\childdocforward}[2][]{}
  \renewcommand{\childdocdisable}{}
}
%    \end{macrocode}

% \macro{\childdocmain}
% The macro |\childdocmain| is to be called at the top of the main file
% with nothing or the main filename (without extension) as argument.
% First, it breaks loops.
% If the argument is not empty and does not match |\childdocname|
% (which is set by the first inclusion of |childdoc.def|),
% |\ifchilddoc| is set to true, |\includeonly| is applied to the child file
% and |\jobname| is set to the main file
% (for proper handling of |.aux| files):
%    \begin{macrocode}
\newcommand{\childdocmain}[1]
{
  \childdocdisable\childdocmain{}
  \if?#1?\else
    \begingroup
      \def\childdoctmp{#1}
      \ifx\childdoctmp\childdocname
        \def\childdoctmp{}
      \else
        \def\childdoctmp
        {
          \childdoctrue
          \includeonly{\childdocname}
          \def\childdocjob{#1}
          \def\jobname{#1}
        }
      \fi
      \expandafter
    \endgroup
    \childdoctmp
  \fi
}
%    \end{macrocode}

% \macro{\childdocof}
% The command |\childdocof| redirects
% compilation to the main file |#1|.
%    \begin{macrocode}
\newcommand{\childdocof}[1]
{
  \childdocdisable
  \childdoctrue
  \includeonly{\childdocname}
  \def\jobname{#1}
  \def\childdocjob{#1}
  \input{#1}
}
%    \end{macrocode}

% \macro{\childdocby}
% The command |\childdocby| ....
%    \begin{macrocode}
\newcommand{\childdocby}[2][]
{
  \childdocdisable
  \childdoctrue
  \childdocmanualtrue
  \if?#1?\else
    \def\jobname{#2}
  \fi
  \def\childdocjob{#2}
  \input{#2}
  \endinput
}
%    \end{macrocode}

% \macro{\childdocforward}
% The command |\childdocforward| redirects
% compilation to the main file or
% (if the optional argument is given) a child file.
% Parameters are set as if the main file
% or a child file starting with |\childdocof| was compiled.
% Then compilation is handed over to the main file:
%    \begin{macrocode}
\newcommand{\childdocforward}[2][]
{
  \begingroup
    \if?#1?
      \def\childdoctmp
      {
        \def\childdocname{#2}
        \def\childdocjob{#2}
        \def\jobname{#2}
        \input{#2}
        \endinput
      }
    \else
      \def\childdoctmp
      {
        \childdocdisable
        \def\childdocname{#2}
        \childdoctrue
        \includeonly{#2}
        \def\childdocjob{#1}
        \def\jobname{#1}
        \input{#1}
        \endinput
      }
    \fi
    \expandafter
  \endgroup
  \childdoctmp
}
%    \end{macrocode}

% \macro{\childdocforwardprefix}
% The command |\childdocforwardprefix| redirects
% compilation to the main or a child file by means of a pattern.
% The prefix |#1| in the current filename is replaced by |#2|
% and the suffix of the current filename is kept
% (it is assumed that the filename does not contain the substring `|~~~|'
% which is used as a delimiter).
% Compilation is handed over to the new file by |\childdocforward|:
%    \begin{macrocode}
\newcommand{\childdocforwardprefix}[3][]
{
  \begingroup
    \def\childdocextract #2##1~~~{\def\childdoctmp{\childdocforward[#1]{#3##1}}}
    \expandafter\childdocextract\childdocname~~~
    \expandafter
  \endgroup
  \childdoctmp
}
%    \end{macrocode}

% \macro{\childdoc}
% The deprecated macro |\childdoc| is a legacy version of |\childdocmain|:
%    \begin{macrocode}
\newcommand{\childdoc}{\childdocmain}
%    \end{macrocode}

% \macro{\childdocredirect}
% The deprecated macro |\childdocredirect| is a legacy version
% of |\childdocforward| and |\childdocforwardprefix|:
%    \begin{macrocode}
\newcommand{\childdocredirect}[2][]
{
  \begingroup
    \if?#1?
      \def\childdoctmp{\childdocforward{#2}}
    \else
      \def\childdoctmp{\childdocforwardprefix{#1}{#2}}
    \fi
    \expandafter
  \endgroup
  \childdoctmp
}
%    \end{macrocode}

%\iffalse
%</package>
%\fi
%
\endinput
|
and perform the replacements as outlined below.
Instead of |\childdocmain{|\textit{main}|}| add the following code
to the top of the main file:
%
\begin{center}
\begin{tabular}{l}
|\||ifdefined\childdocname\endinput\||fi\newif\ifchilddoc|\\
|\edef\childdocname{\scantokens\expandafter{\jobname\noexpand}}|\\
|\def\childdocmain{|\textit{main}|}\||ifx\childdocmain\childdocname\||else|\\
|\childdoctrue\includeonly{\childdocname}\let\jobname\childdocmain\||fi|\\
\end{tabular}
\end{center}
%
Instead of |\childdocof{|\textit{main}|}| just include the main file
at the top of each child file:
%
\begin{center}
|\input{|\textit{main}|}|
\end{center}
%
A simple redirection |\childdocforward{|\textit{dest}|}| is achieved by:
%
\begin{center}
|\def\jobname{|\textit{dest}|}\input{\jobname}|
\end{center}
%
The redirection with prefix
|\childdocforwardprefix[|\textit{prefix}|]{|\textit{dest}|}|
is accomplished by:
%
\begin{center}
\begin{tabular}{l}
|{\edef\jobname{\scantokens\expandafter{\jobname\noexpand}}|\\
|\def\redirectjob |\textit{prefix}|#1~~~{\gdef\jobname{|\textit{dest}|#1}}|\\
|\expandafter\redirectjob\jobname~~~}\input{\jobname}|
\end{tabular}
\end{center}

In an alternative approach,
child documents can be compiled by a specific command line
without additional code or specific definitions:
%
\begin{center}
|... -jobname "|\textit{target}|" "|[\textit{flags}]%
|\includeonly{|\textit{dest}|}\input{|\textit{main}|}"|
\end{center}
%

%%%%%%%%%%%%%%%%%%%%%%%%%%%%%%%%%%%%%%%%%%%%%%%%%%%%%%%%%%%%%%%%%%%%%%%%%%%%%%%%
%%%%%%%%%%%%%%%%%%%%%%%%%%%%%%%%%%%%%%%%%%%%%%%%%%%%%%%%%%%%%%%%%%%%%%%%%%%%%%%%
\section{Information}

%%%%%%%%%%%%%%%%%%%%%%%%%%%%%%%%%%%%%%%%%%%%%%%%%%%%%%%%%%%%%%%%%%%%%%%%%%%%%%%%
\subsection{Copyright}

Copyright \copyright{} 2017--2018 Niklas Beisert

This work may be distributed and/or modified under the
conditions of the \LaTeX{} Project Public License, either version 1.3
of this license or (at your option) any later version.
The latest version of this license is in
  \url{http://www.latex-project.org/lppl.txt}
and version 1.3 or later is part of all distributions of \LaTeX{}
version 2005/12/01 or later.

This work has the LPPL maintenance status `maintained'.

The Current Maintainer of this work is Niklas Beisert.

This work consists of the files |README.txt|, |childdoc.ins| and |childdoc.dtx|
as well as the derived files |childdoc.def|, |cdocsamp.tex|
with |cdocsch1.tex|, |cdocsch2.tex|, |cdocspt3.tex|, |cdocspt4.tex|,
|cdocsdrf.tex|, |cdocsfn1.tex|, |cdocsfn2.tex|
as well as |childdoc.pdf|.

%%%%%%%%%%%%%%%%%%%%%%%%%%%%%%%%%%%%%%%%%%%%%%%%%%%%%%%%%%%%%%%%%%%%%%%%%%%%%%%%
\subsection{Files and Installation}

The package consists of the files:
%
\begin{center}
\begin{tabular}{ll}
    |README.txt|   & readme file \\
    |childdoc.ins| & installation file \\
    |childdoc.dtx| & source file \\
    |childdoc.def| & definition file \\
    |cdocsamp.tex| & sample main file \\
    |cdocsch1.tex| & sample include file \\
    |cdocsch2.tex| & sample include file \\
    |cdocspt3.tex| & sample part file \\
    |cdocspt4.tex| & sample part file \\
    |cdocsdrf.tex| & sample redirection file \\
    |cdocsfn1.tex| & sample redirection file \\
    |cdocsfn2.tex| & sample redirection file \\
    |childdoc.pdf| & manual
\end{tabular}
\end{center}
%
The distribution consists of the files
|README.txt|, |childdoc.ins| and |childdoc.dtx|.
%
\begin{itemize}
\item
Run (pdf)\LaTeX{} on |childdoc.dtx|
to compile the manual |childdoc.pdf| (this file).
\item
Run \LaTeX{} on |childdoc.ins| to create the definitions file |childdoc.def|
and the sample |cdocsamp.tex| with include files
|cdocsch1.tex|, |cdocsch2.tex|, |cdocspt3.tex|, |cdocspt4.tex|,
|cdocsdrf.tex|, |cdocsfn1.tex|, |cdocsfn2.tex|.
Then copy the file |childdoc.def| to an appropriate directory of your \LaTeX{}
distribution, e.g.\ \textit{texmf-root}|/tex/latex/childdoc|.
\end{itemize}

%%%%%%%%%%%%%%%%%%%%%%%%%%%%%%%%%%%%%%%%%%%%%%%%%%%%%%%%%%%%%%%%%%%%%%%%%%%%%%%%
\subsection{Related CTAN Packages}

There are several other packages which offer a similar functionality:
%
\begin{itemize}
\item
The packages
\href{http://ctan.org/pkg/docmute}{\textsf{docmute}},
\href{http://ctan.org/pkg/includex}{\textsf{includex}} and
\href{http://ctan.org/pkg/standalone}{\textsf{standalone}}
provide commands to include only the document body of
a child file thus allowing both files to be compiled individually.
\item
The packages \href{http://ctan.org/pkg/subdocs}{\textsf{subdocs}}
and \href{http://ctan.org/pkg/subfiles}{\textsf{subfiles}}
provide structures in which the main and child documents can be
encapsulated and allowing them to be compiled individually.
The inclusion mechanism is different from the conventional |\include|.
\item
The package \href{http://ctan.org/pkg/combine}{\textsf{combine}}
is an elaborate solution to combine several documents into one.
\end{itemize}
%
See also the CTAN topic \href{http://ctan.org/topic/subdocs}{\textsf{subdocs}}
for further related packages.
The present package differs from the above solutions in that
a document structure constructed with the conventional |\include| mechanism
just needs two extra commands at the top of every file
such that all constituent files can be compiled individually.

%%%%%%%%%%%%%%%%%%%%%%%%%%%%%%%%%%%%%%%%%%%%%%%%%%%%%%%%%%%%%%%%%%%%%%%%%%%%%%%%
%\subsection{Feature Suggestions}
%
%The following is a list of features which may be useful for future
%versions of this package:
%%
%\begin{itemize}
%\item
%\ldots
%\end{itemize}

%%%%%%%%%%%%%%%%%%%%%%%%%%%%%%%%%%%%%%%%%%%%%%%%%%%%%%%%%%%%%%%%%%%%%%%%%%%%%%%%
\subsection{Revision History}

%%%%%%%%%%%%%%%%%%%%%%%%%%%%%%%%%%%%%%%%
\paragraph{v2.0:} 2018/12/30

\begin{itemize}
\item
immediate forward processing
\item
added |\childdocby| mechanism
\item
manual restructured
\end{itemize}

%%%%%%%%%%%%%%%%%%%%%%%%%%%%%%%%%%%%%%%%
\paragraph{v1.6:} 2018/01/17

\begin{itemize}
\item
application for development of include files
\item
corrections to manual
\end{itemize}

%%%%%%%%%%%%%%%%%%%%%%%%%%%%%%%%%%%%%%%%
\paragraph{v1.5:} 2017/05/21

\begin{itemize}
\item
more complete structuring introduced
\item
|\childdocof| introduced
\item
|\childdoc| renamed to |\childdocmain|
\item
|\childredirect| renamed to |\childdocforward| and |\childdocforwardprefix|
and functionality expanded
\end{itemize}

%%%%%%%%%%%%%%%%%%%%%%%%%%%%%%%%%%%%%%%%
\paragraph{v1.0:} 2017/04/27

\begin{itemize}
\item
manual and install package
\item
first version published on CTAN
\end{itemize}

%%%%%%%%%%%%%%%%%%%%%%%%%%%%%%%%%%%%%%%%
\paragraph{v0.6:} 2017/04/26

\begin{itemize}
\item
redirection mechanism added
\end{itemize}

%%%%%%%%%%%%%%%%%%%%%%%%%%%%%%%%%%%%%%%%
\paragraph{v0.5:} 2017/04/26

\begin{itemize}
\item
functionality in definition file
\end{itemize}


%%%%%%%%%%%%%%%%%%%%%%%%%%%%%%%%%%%%%%%%%%%%%%%%%%%%%%%%%%%%%%%%%%%%%%%%%%%%%%%%
%%%%%%%%%%%%%%%%%%%%%%%%%%%%%%%%%%%%%%%%%%%%%%%%%%%%%%%%%%%%%%%%%%%%%%%%%%%%%%%%
%%%%%%%%%%%%%%%%%%%%%%%%%%%%%%%%%%%%%%%%%%%%%%%%%%%%%%%%%%%%%%%%%%%%%%%%%%%%%%%%
\appendix

\settowidth\MacroIndent{\rmfamily\scriptsize 000\ }

 \DocInput{childdoc.dtx}

\end{document}
%</driver>
% \fi
%
% %%%%%%%%%%%%%%%%%%%%%%%%%%%%%%%%%%%%%%%%%%%%%%%%%%%%%%%%%%%%%%%%%%%%%%%%%%%%%%
% %%%%%%%%%%%%%%%%%%%%%%%%%%%%%%%%%%%%%%%%%%%%%%%%%%%%%%%%%%%%%%%%%%%%%%%%%%%%%%
% \section{Sample}
%\iffalse
%<*samplemain>
%\fi
%
% The following presents a sample document
% with two chapters, two parts, a title page,
% a compile flag as well as three forwarding files to set the flag.
% It consists of eight |.tex| files:
% \begin{center}
% \begin{tabular}{ll}
% |cdocsamp.tex|&main file\\
% |cdocsch1.tex|&include file for chapter 1\\
% |cdocsch2.tex|&include file for chapter 2\\
% |cdocspt3.tex|&include file for part 3\\
% |cdocspt4.tex|&include file for part 4\\
% |cdocsdrf.tex|&forwarding file for main file in draft mode\\
% |cdocsfi1.tex|&forwarding file for final version of chapter 1\\
% |cdocsfi2.tex|&forwarding file for final version of chapter 2\\
% \end{tabular}
% \end{center}
% Each of the eight files can be compiled directly by the \LaTeX{} compiler.
%
% %%%%%%%%%%%%%%%%%%%%%%%%%%%%%%%%%%%%%%
% \paragraph{Main File.}
%
% The main file is called |cdocsamp.tex|.
%
% Load the \textsf{childdoc} definitions and
% declare the filename for the main document:
%    \begin{macrocode}
% \iffalse
%
% childdoc.dtx Copyright (C) 2017-2018 Niklas Beisert
%
% This work may be distributed and/or modified under the
% conditions of the LaTeX Project Public License, either version 1.3
% of this license or (at your option) any later version.
% The latest version of this license is in
%   http://www.latex-project.org/lppl.txt
% and version 1.3 or later is part of all distributions of LaTeX
% version 2005/12/01 or later.
%
% This work has the LPPL maintenance status `maintained'.
%
% The Current Maintainer of this work is Niklas Beisert.
%
% This work consists of the files childdoc.dtx and childdoc.ins
% and the derived files childdoc.def and cdocsamp.tex with
% cdocsch1.tex, cdocsch2.tex, cdocsdrf.tex, cdocsfn1.tex, cdocsfn2.tex.
%
%<package>\ifdefined\childdocmain\endinput\fi
%<package>\ProvidesFile{childdoc.def}[2018/12/30 v2.0 child document driver]
%<samplemain>\ProvidesFile{cdocsamp.tex}[2018/12/30 v2.0 sample for childdoc]
%<*driver>
%\ProvidesFile{childdoc.drv}[2018/12/30 v2.0 childdoc reference manual file]
\PassOptionsToClass{10pt,a4paper}{article}
\documentclass{ltxdoc}

\usepackage[margin=35mm]{geometry}
\usepackage{hyperref}
\usepackage{hyperxmp}
\usepackage[usenames]{color}

\hypersetup{colorlinks=true}
\hypersetup{pdfstartview=FitH}
\hypersetup{pdfpagemode=UseNone}
\hypersetup{pdfsource={}}
\hypersetup{pdflang={en-UK}}
\hypersetup{pdfcopyright={Copyright 2017-2018 Niklas Beisert.
  This work may be distributed and/or modified under the
  conditions of the LaTeX Project Public License, either version 1.3
  of this license or (at your option) any later version.}}
\hypersetup{pdflicenseurl={http://www.latex-project.org/lppl.txt}}
\hypersetup{pdfcontactaddress={ETH Zurich, ITP, HIT K,
  Wolfgang-Pauli-Strasse 27}}
\hypersetup{pdfcontactpostcode={8093}}
\hypersetup{pdfcontactcity={Zurich}}
\hypersetup{pdfcontactcountry={Switzerland}}
\hypersetup{pdfcontactemail={nbeisert@itp.phys.ethz.ch}}
\hypersetup{pdfcontacturl={http://people.phys.ethz.ch/\xmptilde nbeisert/}}

\newcommand{\secref}[1]{\hyperref[#1]{section \ref*{#1}}}

\parskip1ex
\parindent0pt
\let\olditemize\itemize
\def\itemize{\olditemize\parskip0pt}

\begin{document}

\title{The \textsf{childdoc} Package}
\hypersetup{pdftitle={The childdoc Package}}
\author{Niklas Beisert\\[2ex]
  Institut f\"ur Theoretische Physik\\
  Eidgen\"ossische Technische Hochschule Z\"urich\\
  Wolfgang-Pauli-Strasse 27, 8093 Z\"urich, Switzerland\\[1ex]
  \href{mailto:nbeisert@itp.phys.ethz.ch}
  {\texttt{nbeisert@itp.phys.ethz.ch}}}
\hypersetup{pdfauthor={Niklas Beisert}}
\hypersetup{pdfsubject={Manual for the LaTeX2e Package childdoc}}
\date{30 December 2018, \textsf{v2.0}}
\maketitle

\begin{abstract}\noindent
\textsf{childdoc} is a \LaTeXe{} package
that enables the direct compilation
of document sections included by |\include|
to individual files.
\end{abstract}

\begingroup
\parskip0ex
\tableofcontents
\endgroup

%%%%%%%%%%%%%%%%%%%%%%%%%%%%%%%%%%%%%%%%%%%%%%%%%%%%%%%%%%%%%%%%%%%%%%%%%%%%%%%%
%%%%%%%%%%%%%%%%%%%%%%%%%%%%%%%%%%%%%%%%%%%%%%%%%%%%%%%%%%%%%%%%%%%%%%%%%%%%%%%%
\section{Introduction}

\LaTeX{} provides a mechanism to structure a large document (such as a book)
into a main file and several child files (containing the chapters)
using the |\include| command.
This mechanism is beneficial for documents
which span hundreds of pages in order to
make the source file(s) more manageable.
Moreover, compilation can be restricted to
selected child files by means of the |\includeonly| command.
The latter feature can be used to reduce the compilation time while editing
(this was significantly more useful in the earlier days of \LaTeX{})
or to generate a smaller document which is easier to navigate.
Another application of |\includeonly| is to generate
documents consisting of selected parts of the complete document.

However, there are a few drawbacks of the plain |\include| mechanism:
\begin{itemize}
\item
The child files cannot be compiled on their own,
they can only be compiled via the main file.
A naive editing environment
(such as a text editor with an option
to have the current file processed by \LaTeX)
may require one to switch to the main file before compiling;
attempting to compile the child file produces errors.
\item
The main file must be modified (each time)
to adjust the |\includeonly| command
to the present needs. This easily leaves the main file in a messy state.
\item
The generated document will always carry the filename
of the main document. This is inconvenient if
several child files are to be compiled and
to be kept for distribution.
\end{itemize}

The present package provides a simple interface
to make child files individually compilable by \LaTeX{}.
Compiling a child file then has the same effect as compiling
the main file with an |\includeonly| command
to select the appropriate child.
Moreover the generated document will carry the name of the child
rather than the main file.
This resolves all three above issues.

This feature is meant to make the editing of books,
thesis documents and lecture notes somewhat more convenient.
However, the package can also be used efficiently for
composing a series of documents (such as exercise sheets)
which are typically distributed individually.
It then assists the author in generating the individual documents
(potentially in different versions)
as well as a document containing the collected series.
Another application is in developing style files
or other kinds of included material
where compilation of the style file could redirect
to a sample or test file.

%%%%%%%%%%%%%%%%%%%%%%%%%%%%%%%%%%%%%%%%%%%%%%%%%%%%%%%%%%%%%%%%%%%%%%%%%%%%%%%%
%%%%%%%%%%%%%%%%%%%%%%%%%%%%%%%%%%%%%%%%%%%%%%%%%%%%%%%%%%%%%%%%%%%%%%%%%%%%%%%%
\section{Usage}

First of all, the package \textsf{childdoc} is \emph{not} a standard
\LaTeXe{} |.sty| style file! Therefore it needs to be invoked in
a non-standard way.

%%%%%%%%%%%%%%%%%%%%%%%%%%%%%%%%%%%%%%%%%%%%%%%%%%%%%%%%%%%%%%%%%%%%%%%%%%%%%%%%
\subsection{Included Files}
\label{sec:include}

%%%%%%%%%%%%%%%%%%%%%%%%%%%%%%%%%%%%%%%%
\DescribeMacro{\childdocmain}
To use the package, add the commands
\begin{center}
\begin{tabular}{l}
|\input{childdoc.def}|\\
|\childdocmain{}|\\
\end{tabular}
\end{center}
at the very top of the main \LaTeX{} file,
in particular \emph{before} the |\documentclass| statement!
The argument of |\childdocmain| should be left empty
(but it must be present).

%%%%%%%%%%%%%%%%%%%%%%%%%%%%%%%%%%%%%%%%
\DescribeMacro{\childdocof}
Furthermore, add the commands
\begin{center}
\begin{tabular}{l}
|\input{childdoc.def}|\\
|\childdocof{|\textit{main}|}|\\
\end{tabular}
\end{center}
at the top of every child file \textit{child}
which is included by |\include{|\textit{child}|}|
from within the main file
(or at least for those files to be compiled individually).
The argument \textit{main} must be the filename of the main file.

There are a couple of
considerations in setting up the main and child documents:

%%%%%%%%%%%%%%%%%%%%%%%%%%%%%%%%%%%%%%%%
\paragraph{Restrictions.}

Please note the following restrictions:
\begin{itemize}
\item
|\childdocmain| must be called with one argument \textit{main}
to ensure compatibility with earlier version of the package.
It must either be empty (|\childdocmain{}|)
or precisely match the filename of the main file in which it is specified.
See \secref{sec:detection} for further information.
\item
The filename \textit{main} must be specified without the |.tex| extension.
\item
The filename \textit{main} is case sensitive
(even in case-insensitive file systems)
due to internal string comparison.
\item
The argument \textit{main} should be fully expanded, it cannot be a macro.
\item
Subdirectories and special characters should be avoided in filenames.
\item
The command |\childdocmain{|\textit{main}|}| must be followed by a whitespace.
It should not be followed immediately by another command
or by a comment mark `|%|'.
This is because the \TeX{} parser reads the token immediately following
the argument of |\childdocmain| and puts it
at the beginning of every child section;
however, a white\-space is ignored.
\end{itemize}

%%%%%%%%%%%%%%%%%%%%%%%%%%%%%%%%%%%%%%%%
\paragraph{Content of Main File.}

It is advisable to place all content in the child files included by |\include|.
Any output contained in the main file will appear in all child documents
unless suppressed manually;
it cannot be suppressed automatically by the |\includeonly| directive
and thus should normally be avoided.
A method to include some content in the main file
by means of conditional processing is described in \secref{sec:conditional}.

%%%%%%%%%%%%%%%%%%%%%%%%%%%%%%%%%%%%%%%%
\paragraph{Page Numbering.}

When only a part of the document is compiled,
the appropriate numbering of pages
(as well as other status parameters)
is determined from the |.aux| files.
The latter contain information from previous passes.
However this information needs to propagate through
all intermediate child documents.
Therefore the page numbering in child documents may well
be inconsistent until the complete document is compiled at least once.

A useful (if unconventional) way to always ensure a consistent
page numbering is to restart the numbering in each child document
and denote the pages by `\textit{child}|.|\textit{page}'
where \textit{child} represents the chapter/section number of the child file.
This can be achieved by the command
|\numberwithin{page}{|\textit{child}|}|
of the \textsf{amsmath} package
where \textit{child} can be |chapter| or |section|
depending on the chosen structuring.
Alternatively, one can modify the macro |\thepage| appropriately
and reset the counter |page| at the start of each child file.

%%%%%%%%%%%%%%%%%%%%%%%%%%%%%%%%%%%%%%%%%%%%%%%%%%%%%%%%%%%%%%%%%%%%%%%%%%%%%%%%
\subsection{Conditional Processing}
\label{sec:conditional}

The package provides a mechanism to compile different versions
of a document. To customise the versions further some conditional processing
can come in handy to distinguish which version is being compiled.
The package provides two macros to describe the compilation context:

%%%%%%%%%%%%%%%%%%%%%%%%%%%%%%%%%%%%%%%%
\DescribeMacro{\ifchilddoc}
The conditional |\ifchilddoc| distinguishes between the compilation of
child documents and the main document:
%
\begin{center}
|\ifchilddoc |\textit{child-code}| |[|\||else |\textit{main-code}]| \||fi|
\end{center}

%%%%%%%%%%%%%%%%%%%%%%%%%%%%%%%%%%%%%%%%
\DescribeMacro{\childdocname}
\DescribeMacro{\childdocjob}
The macro |\childdocname| contains the filename (without extension)
of the main or child file being processed.
Note that |\childdocjob| will always contain the name of the main file.

%%%%%%%%%%%%%%%%%%%%%%%%%%%%%%%%%%%%%%%%
\paragraph{Title Page.}

Conditional processing can be used to include a title or banner page
in the main document when proper precautions are taken.
Importantly, the code in the main file should ensure that the page counter
(as well as other status parameters which are stored in the |.aux| files)
takes the same value after the conditional processing.
Otherwise the page numbers may take divergent values
depending on which part is compiled.

For example, a title page could be declared by:
%
\begin{center}
\begin{tabular}{l}
|\ifchilddoc\||else|\\
|\addtocounter{page}{-1}|\\
\textit{code for title page}\\
|\newpage|\\
|\||fi|
\end{tabular}
\end{center}
%
A banner page for the child documents can be generated by:
%
\begin{center}
\begin{tabular}{l}
|\ifchilddoc|\\
|\addtocounter{page}{-1}|\\
\textit{code for banner page}\\
|\newpage|\\
|\||fi|
\end{tabular}
\end{center}
%
Here one could write a message such as:
\begin{center}
|This is the part \childdocname{} of \childdocjob{}.|
\end{center}

%%%%%%%%%%%%%%%%%%%%%%%%%%%%%%%%%%%%%%%%%%%%%%%%%%%%%%%%%%%%%%%%%%%%%%%%%%%%%%%%
\subsection{Flags}
\label{sec:flags}

The package makes it easy to generate different versions
of the main or child documents.
To this end compilation flags can be defined
and assigned different default values.
They will be particularly useful in conjunction
with the forwarding mechanism described in \secref{sec:forward}.

For example, it may be useful to have a flag |\version|
which can be set to |draft| or |final|.
The document source will contain some conditional code
depending on the value of |\version|.
Suppose further, the flag should default to |final| for the main file
and to |draft| for child files
which is a natural assignment for editing the document.
This is achieved by placing the following code
in the preamble of the main document
(below the |\childdocmain| directive):
%
\begin{center}
\begin{tabular}{l}
|\ifchilddoc|\\
|\providecommand{\version}{draft}|\\
|\||else|\\
|\providecommand{\version}{final}|\\
|\||fi|
\end{tabular}
\end{center}
%
The definition by |\providecommand| makes sure
that previous definitions are not overwritten.
Further statements |\providecommand{\version}{...}|
can thus be added before the above code to override it.

For the main file, one might add a line
(between |\childdocmain| and the above block)
%
\begin{center}
|%\ifchilddoc\||else\providecommand{\version}{draft}\||fi|
\end{center}
%
which can be uncommented to produce a draft version.
Likewise one can add a line to the very top of a child file
(above the |\childdocof{|\textit{main}|}| directive)
%
\begin{center}
|%\providecommand{\version}{final}|
\end{center}
%
which can be uncommented to produce the final version of this child document.

%%%%%%%%%%%%%%%%%%%%%%%%%%%%%%%%%%%%%%%%%%%%%%%%%%%%%%%%%%%%%%%%%%%%%%%%%%%%%%%%
\subsection{Forwarding}
\label{sec:forward}

Different versions of the main or child documents
using compilation flags as described in \secref{sec:flags}
can be (permanently) stored in different files
for convenient compilation, viewing and distribution.
To this end, the package defines a command
to pass on compilation to a different file:

%%%%%%%%%%%%%%%%%%%%%%%%%%%%%%%%%%%%%%%%
\DescribeMacro{\childdocforward}
The command |\childdocforward| redirects processing to
another source file:
%
\begin{center}
\begin{tabular}{l}
|\input{childdoc.def}|\\
|\childdocforward[|\textit{main}|]{|\textit{dest}|}|\\
\end{tabular}
\end{center}
%
The argument \textit{dest} is the destination file
(without extension).
It should be the main file or one of the child files.
Note that further \textsf{childdoc} directives
such as |\childdocof| and |\childdocforward|
in the indicated file will be processed in this form.
The optional argument \textit{main}
passes on directly to the main file \textit{main}
while pretending to compile the child \textit{dest}.
This form behaves as if \textit{dest}
issues |\childdocof{|\textit{main}|}| right away,
and no further \textsf{childdoc} directives will be processed.

%%%%%%%%%%%%%%%%%%%%%%%%%%%%%%%%%%%%%%%%
\DescribeMacro{\...prefix}
In the alternative form |\childdocforwardprefix|,
%
\begin{center}
\begin{tabular}{l}
|\input{childdoc.def}|\\
|\childdocforwardprefix[|\textit{main}|]{|\textit{prefix}|}{|\textit{dest}|}|
\end{tabular}
\end{center}
%
the destination file is determined by a pattern
depending on the current file:
To make this work, the current file must be called
`{\textit{prefix}\hspace{0.2em}\textit{suffix}}'
with \textit{prefix} matching precisely the argument.
Processing is then passed on to the file
`{\textit{dest}\hspace{0.2em}\textit{suffix}}'.
Surely, the same effect is achieved by
directly specifying the
argument `{\textit{dest}\hspace{0.2em}\textit{suffix}}'
in the first form.
However, that requires to set up a different file
for each child. With the alternative form of the command
all these files can have exactly the same content
which simplifies setting them up and maintaining them.

For example, the following file |draft.tex|
with a compilation flag |\version| as described in \secref{sec:flags}
compiles the main document as a draft:
%
\begin{center}
\begin{tabular}{l}
|\def\version{draft}|\\
|\input{childdoc.def}|\\
|\childdocforward{|\textit{main}|}|
\end{tabular}
\end{center}
%
Likewise, the following files |final|\textit{nn}|.tex|
compile the final version of the child document
|child|\textit{nn}|.tex|:
%
\begin{center}
\begin{tabular}{l}
|\def\version{final}|\\
|\input{childdoc.def}|\\
|\childdocforwardprefix{final}{child}|
\end{tabular}
\end{center}
%

Note that when several versions of a main file and/or of each child file
are to be generated, it may be convenient to set up a |Makefile| or
shell script to automatise the process.

%%%%%%%%%%%%%%%%%%%%%%%%%%%%%%%%%%%%%%%%%%%%%%%%%%%%%%%%%%%%%%%%%%%%%%%%%%%%%%%%
\subsection{Command Line Processing}
\label{sec:commandline}

The effect of redirection files can also be achieved by invoking
the \LaTeX{} compiler with a more elaborate command line.
Most conveniently this should be done as part
of a shell script or a |Makefile|.

When using \textsf{childdoc} in the main file, the following
command lines effectively perform a redirection
(note that depending on the shell being used,
backslashes may have to be doubled: `|\|' $\to$ `|\\|'):
%
\begin{center}
|... -jobname "|\textit{target}|" |\\|"|[\textit{flags}]%
|\input{childdoc.def}\childdocforward[|\textit{main}|]{|\textit{dest}|}"|
\end{center}
%
Here \textit{target} is the name of the output file,
\textit{main} is the name of the main file
and \textit{dest} is the name of the main or child file to be processed
(all filenames without extensions).
The optional argument \textit{main} can be omitted
if \textit{main} matches \textit{dest}.
Optionally, compilation \textit{flags} can be defined via |\def| commands.
This command line makes the \TeX{} engine believe
it is compiling the file \textit{target}
whose content is specified as the latter parameter.
The provided code then forwards the processing to
\textit{main} or \textit{dest} as described in \secref{sec:forward}.

%%%%%%%%%%%%%%%%%%%%%%%%%%%%%%%%%%%%%%%%%%%%%%%%%%%%%%%%%%%%%%%%%%%%%%%%%%%%%%%%
\subsection{Include by Input}
\label{sec:input}

Including child documents by |\include| has some restrictions by design.
Most notably, the content of a child document always occupies
its own set of pages; pages cannot be shared between child documents.
Usually, this behaviour makes perfect sense
because each child document contain an essential part of the document.
However, in some situations it may be desirable to compose
a document from a collection of parts
without having mandatory page breaks between then.
For this case, the package
provides a mechanism to include parts
by |\input| which can also be processed individually.
However, by construction this mechanism
requires manual handling of the content to be output.

%%%%%%%%%%%%%%%%%%%%%%%%%%%%%%%%%%%%%%%%
\DescribeMacro{\ifchilddocmanual}
The main file should be prepared as usual, see \secref{sec:include}.
However, the document body must make a distinction
between processing of an individual part and of the main document, e.g.:
%
\begin{center}
\begin{tabular}{l}
|\ifchilddocmanual|\\
|\input{\childdocname}|\\
|\||else|\\
\textit{document body with }|\input{|\textit{part}|}|\\
|\||fi|
\end{tabular}
\end{center}
%
The conditional |\ifchilddocmanual| is true whenever
a part to be included by |\input| is being compiled,
and the name of the part is stored in |\childdocname|.

%%%%%%%%%%%%%%%%%%%%%%%%%%%%%%%%%%%%%%%%
\DescribeMacro{\childdocby}
Each part to be included by |\input| should start with:
%
\begin{center}
\begin{tabular}{l}
|\input{childdoc.def}|\\
|\childdocby{|\textit{main}|}|\\
\end{tabular}
\end{center}
%
The directive |\childdocby| is similar to |\childdocof|
described in \secref{sec:include},
but the subsequent selection of content must be done manually.
To that end, both |\ifchilddoc| and |\ifchilddocmanual|
will be true upon processing of a part,
and the name of the part is stored in |\childdocname|.
Note that |\jobname| will be set to the filename of the current part
so that each part receives an individual |.aux| file
that does not interfere with the |.aux| file(s) of the main document.
This behaviour can be altered by the alternative form
|\childdocby[*]{|\textit{main}|}| (with a non-empty optional argument)
which uses the |.aux| file of the main document
by setting |\jobname| to \textit{main}.

%%%%%%%%%%%%%%%%%%%%%%%%%%%%%%%%%%%%%%%%%%%%%%%%%%%%%%%%%%%%%%%%%%%%%%%%%%%%%%%%
\subsection{Driver Development}
\label{sec:driver}

The \textsf{childdoc} mechanism can also be use for the development
of definition files such as \LaTeX{} styles or classes.
This case differs from the above setup with multiple parts
included by |\include| in that no |\includeonly| should be invoked.
This can be achieved by starting the include file
(before |\ProvidesPackage|) with:
%
\begin{center}
\begin{tabular}{l}
|\input{childdoc.def}|\\
|\childdocforward{|\textit{main}|}|\\
\end{tabular}
\end{center}
%
or alternatively with:
%
\begin{center}
\begin{tabular}{l}
|\input{childdoc.def}|\\
|\childdocby{|\textit{main}|}|\\
\end{tabular}
\end{center}
%
Both forms have slightly different effects as described above.
The main file is prepared as usual, see \secref{sec:include}.

%%%%%%%%%%%%%%%%%%%%%%%%%%%%%%%%%%%%%%%%%%%%%%%%%%%%%%%%%%%%%%%%%%%%%%%%%%%%%%%%
\subsection{Legacy Detection}
\label{sec:detection}

The directive |\childdocmain| in the main file can detect
whether the complete document or merely a child is to be compiled
even without using the directive |\childdocof|.
This method is deprecated because it is less robust
and there is no compelling reason to use it;
it is merely provided for backward compatibility
and it may be removed in future versions.

If the detection mechanism is to be used,
it is mandatory to correctly specify
the filename of the main file as the argument of |\childdocmain|:
%
\begin{center}
\begin{tabular}{l}
|\input{childdoc.def}|\\
|\childdocmain{|\textit{main}|}|\\
\end{tabular}
\end{center}
%
If |\jobname| does not match the argument \textit{main} of |\childdocmain|,
it is assumed that |\jobname| points to the child file to be compiled.
When using |\childdocmain| with the main file specified as argument,
it suffices to start a child file
with just |\input{|\textit{main}|}|
without loading of the package and using |\childdocof|.
If instead all processing is done
with the appropriate \textsf{childdoc} directives,
the argument of \textit{main} of |\childdocmain| can be empty.

An alternative version of the command line processing described
in \secref{sec:commandline} using the detection mechanism reads:
%
\begin{center}
|... -jobname "|\textit{target}|" "|[\textit{flags}]%
[|\def\jobname{|\textit{dest}|}|]|\input{|\textit{main}|}"|
\end{center}

%%%%%%%%%%%%%%%%%%%%%%%%%%%%%%%%%%%%%%%%%%%%%%%%%%%%%%%%%%%%%%%%%%%%%%%%%%%%%%%%
\subsection{Manual Code}
\label{sec:manual}

In case one cannot be certain whether the definitions file |childdoc.def|
is installed on the target \TeX{} distribution
and one prefers not to ship it,
it is conceivable to paste a few relevant commands into the sources.

To that end, drop all statements |\input{childdoc.def}|
and perform the replacements as outlined below.
Instead of |\childdocmain{|\textit{main}|}| add the following code
to the top of the main file:
%
\begin{center}
\begin{tabular}{l}
|\||ifdefined\childdocname\endinput\||fi\newif\ifchilddoc|\\
|\edef\childdocname{\scantokens\expandafter{\jobname\noexpand}}|\\
|\def\childdocmain{|\textit{main}|}\||ifx\childdocmain\childdocname\||else|\\
|\childdoctrue\includeonly{\childdocname}\let\jobname\childdocmain\||fi|\\
\end{tabular}
\end{center}
%
Instead of |\childdocof{|\textit{main}|}| just include the main file
at the top of each child file:
%
\begin{center}
|\input{|\textit{main}|}|
\end{center}
%
A simple redirection |\childdocforward{|\textit{dest}|}| is achieved by:
%
\begin{center}
|\def\jobname{|\textit{dest}|}\input{\jobname}|
\end{center}
%
The redirection with prefix
|\childdocforwardprefix[|\textit{prefix}|]{|\textit{dest}|}|
is accomplished by:
%
\begin{center}
\begin{tabular}{l}
|{\edef\jobname{\scantokens\expandafter{\jobname\noexpand}}|\\
|\def\redirectjob |\textit{prefix}|#1~~~{\gdef\jobname{|\textit{dest}|#1}}|\\
|\expandafter\redirectjob\jobname~~~}\input{\jobname}|
\end{tabular}
\end{center}

In an alternative approach,
child documents can be compiled by a specific command line
without additional code or specific definitions:
%
\begin{center}
|... -jobname "|\textit{target}|" "|[\textit{flags}]%
|\includeonly{|\textit{dest}|}\input{|\textit{main}|}"|
\end{center}
%

%%%%%%%%%%%%%%%%%%%%%%%%%%%%%%%%%%%%%%%%%%%%%%%%%%%%%%%%%%%%%%%%%%%%%%%%%%%%%%%%
%%%%%%%%%%%%%%%%%%%%%%%%%%%%%%%%%%%%%%%%%%%%%%%%%%%%%%%%%%%%%%%%%%%%%%%%%%%%%%%%
\section{Information}

%%%%%%%%%%%%%%%%%%%%%%%%%%%%%%%%%%%%%%%%%%%%%%%%%%%%%%%%%%%%%%%%%%%%%%%%%%%%%%%%
\subsection{Copyright}

Copyright \copyright{} 2017--2018 Niklas Beisert

This work may be distributed and/or modified under the
conditions of the \LaTeX{} Project Public License, either version 1.3
of this license or (at your option) any later version.
The latest version of this license is in
  \url{http://www.latex-project.org/lppl.txt}
and version 1.3 or later is part of all distributions of \LaTeX{}
version 2005/12/01 or later.

This work has the LPPL maintenance status `maintained'.

The Current Maintainer of this work is Niklas Beisert.

This work consists of the files |README.txt|, |childdoc.ins| and |childdoc.dtx|
as well as the derived files |childdoc.def|, |cdocsamp.tex|
with |cdocsch1.tex|, |cdocsch2.tex|, |cdocspt3.tex|, |cdocspt4.tex|,
|cdocsdrf.tex|, |cdocsfn1.tex|, |cdocsfn2.tex|
as well as |childdoc.pdf|.

%%%%%%%%%%%%%%%%%%%%%%%%%%%%%%%%%%%%%%%%%%%%%%%%%%%%%%%%%%%%%%%%%%%%%%%%%%%%%%%%
\subsection{Files and Installation}

The package consists of the files:
%
\begin{center}
\begin{tabular}{ll}
    |README.txt|   & readme file \\
    |childdoc.ins| & installation file \\
    |childdoc.dtx| & source file \\
    |childdoc.def| & definition file \\
    |cdocsamp.tex| & sample main file \\
    |cdocsch1.tex| & sample include file \\
    |cdocsch2.tex| & sample include file \\
    |cdocspt3.tex| & sample part file \\
    |cdocspt4.tex| & sample part file \\
    |cdocsdrf.tex| & sample redirection file \\
    |cdocsfn1.tex| & sample redirection file \\
    |cdocsfn2.tex| & sample redirection file \\
    |childdoc.pdf| & manual
\end{tabular}
\end{center}
%
The distribution consists of the files
|README.txt|, |childdoc.ins| and |childdoc.dtx|.
%
\begin{itemize}
\item
Run (pdf)\LaTeX{} on |childdoc.dtx|
to compile the manual |childdoc.pdf| (this file).
\item
Run \LaTeX{} on |childdoc.ins| to create the definitions file |childdoc.def|
and the sample |cdocsamp.tex| with include files
|cdocsch1.tex|, |cdocsch2.tex|, |cdocspt3.tex|, |cdocspt4.tex|,
|cdocsdrf.tex|, |cdocsfn1.tex|, |cdocsfn2.tex|.
Then copy the file |childdoc.def| to an appropriate directory of your \LaTeX{}
distribution, e.g.\ \textit{texmf-root}|/tex/latex/childdoc|.
\end{itemize}

%%%%%%%%%%%%%%%%%%%%%%%%%%%%%%%%%%%%%%%%%%%%%%%%%%%%%%%%%%%%%%%%%%%%%%%%%%%%%%%%
\subsection{Related CTAN Packages}

There are several other packages which offer a similar functionality:
%
\begin{itemize}
\item
The packages
\href{http://ctan.org/pkg/docmute}{\textsf{docmute}},
\href{http://ctan.org/pkg/includex}{\textsf{includex}} and
\href{http://ctan.org/pkg/standalone}{\textsf{standalone}}
provide commands to include only the document body of
a child file thus allowing both files to be compiled individually.
\item
The packages \href{http://ctan.org/pkg/subdocs}{\textsf{subdocs}}
and \href{http://ctan.org/pkg/subfiles}{\textsf{subfiles}}
provide structures in which the main and child documents can be
encapsulated and allowing them to be compiled individually.
The inclusion mechanism is different from the conventional |\include|.
\item
The package \href{http://ctan.org/pkg/combine}{\textsf{combine}}
is an elaborate solution to combine several documents into one.
\end{itemize}
%
See also the CTAN topic \href{http://ctan.org/topic/subdocs}{\textsf{subdocs}}
for further related packages.
The present package differs from the above solutions in that
a document structure constructed with the conventional |\include| mechanism
just needs two extra commands at the top of every file
such that all constituent files can be compiled individually.

%%%%%%%%%%%%%%%%%%%%%%%%%%%%%%%%%%%%%%%%%%%%%%%%%%%%%%%%%%%%%%%%%%%%%%%%%%%%%%%%
%\subsection{Feature Suggestions}
%
%The following is a list of features which may be useful for future
%versions of this package:
%%
%\begin{itemize}
%\item
%\ldots
%\end{itemize}

%%%%%%%%%%%%%%%%%%%%%%%%%%%%%%%%%%%%%%%%%%%%%%%%%%%%%%%%%%%%%%%%%%%%%%%%%%%%%%%%
\subsection{Revision History}

%%%%%%%%%%%%%%%%%%%%%%%%%%%%%%%%%%%%%%%%
\paragraph{v2.0:} 2018/12/30

\begin{itemize}
\item
immediate forward processing
\item
added |\childdocby| mechanism
\item
manual restructured
\end{itemize}

%%%%%%%%%%%%%%%%%%%%%%%%%%%%%%%%%%%%%%%%
\paragraph{v1.6:} 2018/01/17

\begin{itemize}
\item
application for development of include files
\item
corrections to manual
\end{itemize}

%%%%%%%%%%%%%%%%%%%%%%%%%%%%%%%%%%%%%%%%
\paragraph{v1.5:} 2017/05/21

\begin{itemize}
\item
more complete structuring introduced
\item
|\childdocof| introduced
\item
|\childdoc| renamed to |\childdocmain|
\item
|\childredirect| renamed to |\childdocforward| and |\childdocforwardprefix|
and functionality expanded
\end{itemize}

%%%%%%%%%%%%%%%%%%%%%%%%%%%%%%%%%%%%%%%%
\paragraph{v1.0:} 2017/04/27

\begin{itemize}
\item
manual and install package
\item
first version published on CTAN
\end{itemize}

%%%%%%%%%%%%%%%%%%%%%%%%%%%%%%%%%%%%%%%%
\paragraph{v0.6:} 2017/04/26

\begin{itemize}
\item
redirection mechanism added
\end{itemize}

%%%%%%%%%%%%%%%%%%%%%%%%%%%%%%%%%%%%%%%%
\paragraph{v0.5:} 2017/04/26

\begin{itemize}
\item
functionality in definition file
\end{itemize}


%%%%%%%%%%%%%%%%%%%%%%%%%%%%%%%%%%%%%%%%%%%%%%%%%%%%%%%%%%%%%%%%%%%%%%%%%%%%%%%%
%%%%%%%%%%%%%%%%%%%%%%%%%%%%%%%%%%%%%%%%%%%%%%%%%%%%%%%%%%%%%%%%%%%%%%%%%%%%%%%%
%%%%%%%%%%%%%%%%%%%%%%%%%%%%%%%%%%%%%%%%%%%%%%%%%%%%%%%%%%%%%%%%%%%%%%%%%%%%%%%%
\appendix

\settowidth\MacroIndent{\rmfamily\scriptsize 000\ }

 \DocInput{childdoc.dtx}

\end{document}
%</driver>
% \fi
%
% %%%%%%%%%%%%%%%%%%%%%%%%%%%%%%%%%%%%%%%%%%%%%%%%%%%%%%%%%%%%%%%%%%%%%%%%%%%%%%
% %%%%%%%%%%%%%%%%%%%%%%%%%%%%%%%%%%%%%%%%%%%%%%%%%%%%%%%%%%%%%%%%%%%%%%%%%%%%%%
% \section{Sample}
%\iffalse
%<*samplemain>
%\fi
%
% The following presents a sample document
% with two chapters, two parts, a title page,
% a compile flag as well as three forwarding files to set the flag.
% It consists of eight |.tex| files:
% \begin{center}
% \begin{tabular}{ll}
% |cdocsamp.tex|&main file\\
% |cdocsch1.tex|&include file for chapter 1\\
% |cdocsch2.tex|&include file for chapter 2\\
% |cdocspt3.tex|&include file for part 3\\
% |cdocspt4.tex|&include file for part 4\\
% |cdocsdrf.tex|&forwarding file for main file in draft mode\\
% |cdocsfi1.tex|&forwarding file for final version of chapter 1\\
% |cdocsfi2.tex|&forwarding file for final version of chapter 2\\
% \end{tabular}
% \end{center}
% Each of the eight files can be compiled directly by the \LaTeX{} compiler.
%
% %%%%%%%%%%%%%%%%%%%%%%%%%%%%%%%%%%%%%%
% \paragraph{Main File.}
%
% The main file is called |cdocsamp.tex|.
%
% Load the \textsf{childdoc} definitions and
% declare the filename for the main document:
%    \begin{macrocode}
\input{childdoc.def}
\childdocmain{}
%    \end{macrocode}

% Optional override for |\version| flag:
%    \begin{macrocode}
%%\ifchilddoc\else\providecommand{\version}{draft}\fi
%    \end{macrocode}

% Define the default values for the |\version| flag
% (|final| for the main file and |draft| for childs):
%    \begin{macrocode}
\ifchilddoc
\providecommand{\version}{draft}
\else
\providecommand{\version}{final}
\fi
%    \end{macrocode}

% Load the standard document class:
%    \begin{macrocode}
\documentclass[12pt]{article}
%    \end{macrocode}

% Start the document body:
%    \begin{macrocode}
\begin{document}
%    \end{macrocode}

% Declare a title page.
% Print title, part of document being processed and version flag:
%    \begin{macrocode}
\addtocounter{page}{-1}
\begin{center}
{\LARGE\bfseries{}childdoc example\par}
\vspace{1cm}
\ifchilddoc
\ifchilddocmanual part\else chapter\fi:
`\childdocname' of `\childdocjob'\par
\else
main document: `\childdocjob'\par
\fi
version: \version\par
\end{center}
\newpage
%    \end{macrocode}

% Manually include selected file,
% otherwise process as usual:
%    \begin{macrocode}
\ifchilddocmanual
\section*{part `\childdocname'}
\input{\childdocname}
\else
%    \end{macrocode}

% Include the two chapters:
%    \begin{macrocode}
\include{cdocsch1}
\include{cdocsch2}
%    \end{macrocode}

% Include the two parts unless only chapters should be displayed:
%    \begin{macrocode}
\ifchilddoc\else
\section{part three}
\input{cdocspt3}
\section{part four}
\input{cdocspt4}
\fi
%    \end{macrocode}

% Process as usual until here:
%    \begin{macrocode}
\fi
%    \end{macrocode}

% End of document body:
%    \begin{macrocode}
\end{document}
%    \end{macrocode}
%\iffalse
%</samplemain>
%\fi
%
% %%%%%%%%%%%%%%%%%%%%%%%%%%%%%%%%%%%%%%
% \paragraph{Chapter Include Files.}
%
% The include files are called |cdocsch1.tex| and |cdocsch2.tex|.
%
%\iffalse
%<*samplechap1|samplechap2>
%\fi

% Optional override for |\version| flag:
%    \begin{macrocode}
%%\providecommand{\version}{final}
%    \end{macrocode}

% Include the main document:
%    \begin{macrocode}
\input{childdoc.def}
\childdocof{cdocsamp}
%    \end{macrocode}

%\iffalse
%</samplechap1|samplechap2>
%\fi
%
%\iffalse
%<*samplechap1>
%\fi
% Some text for chapter 1:
%    \begin{macrocode}
\section{one}
some text in chapter one
%    \end{macrocode}

%\iffalse
%</samplechap1>
%\fi
% Some text for chapter 2:
%\iffalse
%<*samplechap2>
%\fi
%    \begin{macrocode}
\section{two}
more text in chapter two
%    \end{macrocode}

%\iffalse
%</samplechap2>
%\fi
%
% %%%%%%%%%%%%%%%%%%%%%%%%%%%%%%%%%%%%%%
% \paragraph{Part Include Files.}
%
% The include files are called |cdocspt3.tex| and |cdocspt4.tex|.
%
%\iffalse
%<*samplepart3|samplepart4>
%\fi

% Optional override for |\version| flag:
%    \begin{macrocode}
%%\providecommand{\version}{final}
%    \end{macrocode}

% Include the main document:
%    \begin{macrocode}
\input{childdoc.def}
\childdocby{cdocsamp}
%    \end{macrocode}

%\iffalse
%</samplepart3|samplepart4>
%\fi
%
%\iffalse
%<*samplepart3>
%\fi
% Some text for part 3:
%    \begin{macrocode}
some text in part three
%    \end{macrocode}

%\iffalse
%</samplepart3>
%\fi
% Some text for part 4:
%\iffalse
%<*samplepart4>
%\fi
%    \begin{macrocode}
more text in part four
%    \end{macrocode}

%\iffalse
%</samplepart4>
%\fi
%
% %%%%%%%%%%%%%%%%%%%%%%%%%%%%%%%%%%%%%%
% \paragraph{Forwarding for a Complete Draft.}
%
% The following forwarding file |cdocsdrf.tex|
% compiles the main document in draft mode:
%\iffalse
%<*sampledraft>
%\fi
%    \begin{macrocode}
\def\version{draft}
\input{childdoc.def}
\childdocforward{cdocsamp}
%    \end{macrocode}

%\iffalse
%</sampledraft>
%\fi
%
% %%%%%%%%%%%%%%%%%%%%%%%%%%%%%%%%%%%%%%
% \paragraph{Forwarding for Final Version of the Chapters.}
%
% The following forwarding files |cdocsfn1.tex| and |cdocsfn2.tex|
% (with identical content)
% compile the final versions of the child documents
% |cdocsch1.tex| and |cdocsch2.tex|, respectively:
%\iffalse
%<*samplefinal>
%\fi
%    \begin{macrocode}
\def\version{final}
\input{childdoc.def}
\childdocforwardprefix[cdocsamp]{cdocsfn}{cdocsch}
%    \end{macrocode}

%\iffalse
%</samplefinal>
%\fi
%
% %%%%%%%%%%%%%%%%%%%%%%%%%%%%%%%%%%%%%%
% \paragraph{Command Line Processing.}
%
% The following three command lines generate the output files
% |cdocscld|, |cdocscl1| and |cdocscl2|
% which should be identical to
% |cdocsdrf|, |cdocsch1| and |cdocsfn2|, respectively:
% \begin{center}
% \begin{tabular}{l}
% |latex -jobname cdocscld \|\\
% |  "\def\version{draft}\input{childdoc.def}\childdocforward{cdocsamp}"|\\
% |latex -jobname cdocscl1 \|\\
% |  "\input{childdoc.def}\childdocforward[cdocsamp]{cdocsch1}"|\\
% |latex -jobname cdocscl2 \|\\
% |  "\def\version{final}\input{childdoc.def}\childdocforward{cdocsch2}"|
% \end{tabular}
% \end{center}
% Note that the trailing backslash on each first line
% merely continues the input to the second line
% (for convenient cut ant paste).
% Furthermore, the command |latex| can be replaced by any
% of its alternative versions such as |pdflatex|.
%
% %%%%%%%%%%%%%%%%%%%%%%%%%%%%%%%%%%%%%%%%%%%%%%%%%%%%%%%%%%%%%%%%%%%%%%%%%%%%%%
% %%%%%%%%%%%%%%%%%%%%%%%%%%%%%%%%%%%%%%%%%%%%%%%%%%%%%%%%%%%%%%%%%%%%%%%%%%%%%%
% \section{Implementation}
%\iffalse
%<*package>
%\fi
%
% This section describes the definitions file |childdoc.def|.

% The definitions cannot be loaded using |\usepackage| or |\RequirePackage|
% which has a mechanism to prevent loading a style file more than once.
% When loading the definitions by means of |\input|
% multiple instances have to be prevented manually:
%\iffalse
%This code needs to be before the `\ProvidesFile' directive
%which is defined at the beginning of this file.
%Therefore it is also placed there and commented out here.
%</package>
%<*discard>
%\fi
%    \begin{macrocode}
\ifdefined\childdocmain\endinput\fi
%    \end{macrocode}
%\iffalse
%</discard>
%<*package>
%\fi
%
% \macro{\ifchilddoc}
% \macro{\ifchilddocmanual}
% The conditional |\ifchilddoc| tells whether a
% child (true) or main (false) document is being compiled.
% The conditional |\ifchilddocmanual| tells whether
% the |\includeonly| mechanism is used (false) or
% the selection of child files must be performed manually (true).
% The definitions initialise to false:
%    \begin{macrocode}
\newif\ifchilddoc
\newif\ifchilddocmanual
%    \end{macrocode}

% \macro{\childdocname}
% \macro{\childdocjob}
% The macro |\childdocname| stores the name of the main document
% to be compiled. The macro |\childdocjob| stores the name of
% the document on which the \LaTeX{} compiler was originally invoked.
% The content of |\jobname| cannot be compared
% to filenames specified in the source due to different catcodes.
% The following code rescans |\jobname|, stores the result
% in |\childdocname| and saves a copy in |\childdocjob|:
%    \begin{macrocode}
\edef\childdocname{\scantokens\expandafter{\jobname\noexpand}}
\let\childdocjob\childdocname
%    \end{macrocode}

% \macro{\childdocdisable}
% The macro |\childdocdisable| prevents the main file
% from being processed more than once.
% At this stage, the main document command |\childdocmain|
% is assumed to be called once again where it should do nothing.
% Any subsequent call to it should prevent
% a secondary processing of the main document
% It overwrites the forwarding commands
% |\childdocof| and |\childdocforward|
% with empty macros to prevent further inclusions of the main document:
%    \begin{macrocode}
\newcommand{\childdocdisable}
{
  \renewcommand{\childdocmain}[1]{\renewcommand{\childdocmain}[1]{\endinput}}
  \renewcommand{\childdocof}[1]{}
  \renewcommand{\childdocby}[2][]{}
  \renewcommand{\childdocforward}[2][]{}
  \renewcommand{\childdocdisable}{}
}
%    \end{macrocode}

% \macro{\childdocmain}
% The macro |\childdocmain| is to be called at the top of the main file
% with nothing or the main filename (without extension) as argument.
% First, it breaks loops.
% If the argument is not empty and does not match |\childdocname|
% (which is set by the first inclusion of |childdoc.def|),
% |\ifchilddoc| is set to true, |\includeonly| is applied to the child file
% and |\jobname| is set to the main file
% (for proper handling of |.aux| files):
%    \begin{macrocode}
\newcommand{\childdocmain}[1]
{
  \childdocdisable\childdocmain{}
  \if?#1?\else
    \begingroup
      \def\childdoctmp{#1}
      \ifx\childdoctmp\childdocname
        \def\childdoctmp{}
      \else
        \def\childdoctmp
        {
          \childdoctrue
          \includeonly{\childdocname}
          \def\childdocjob{#1}
          \def\jobname{#1}
        }
      \fi
      \expandafter
    \endgroup
    \childdoctmp
  \fi
}
%    \end{macrocode}

% \macro{\childdocof}
% The command |\childdocof| redirects
% compilation to the main file |#1|.
%    \begin{macrocode}
\newcommand{\childdocof}[1]
{
  \childdocdisable
  \childdoctrue
  \includeonly{\childdocname}
  \def\jobname{#1}
  \def\childdocjob{#1}
  \input{#1}
}
%    \end{macrocode}

% \macro{\childdocby}
% The command |\childdocby| ....
%    \begin{macrocode}
\newcommand{\childdocby}[2][]
{
  \childdocdisable
  \childdoctrue
  \childdocmanualtrue
  \if?#1?\else
    \def\jobname{#2}
  \fi
  \def\childdocjob{#2}
  \input{#2}
  \endinput
}
%    \end{macrocode}

% \macro{\childdocforward}
% The command |\childdocforward| redirects
% compilation to the main file or
% (if the optional argument is given) a child file.
% Parameters are set as if the main file
% or a child file starting with |\childdocof| was compiled.
% Then compilation is handed over to the main file:
%    \begin{macrocode}
\newcommand{\childdocforward}[2][]
{
  \begingroup
    \if?#1?
      \def\childdoctmp
      {
        \def\childdocname{#2}
        \def\childdocjob{#2}
        \def\jobname{#2}
        \input{#2}
        \endinput
      }
    \else
      \def\childdoctmp
      {
        \childdocdisable
        \def\childdocname{#2}
        \childdoctrue
        \includeonly{#2}
        \def\childdocjob{#1}
        \def\jobname{#1}
        \input{#1}
        \endinput
      }
    \fi
    \expandafter
  \endgroup
  \childdoctmp
}
%    \end{macrocode}

% \macro{\childdocforwardprefix}
% The command |\childdocforwardprefix| redirects
% compilation to the main or a child file by means of a pattern.
% The prefix |#1| in the current filename is replaced by |#2|
% and the suffix of the current filename is kept
% (it is assumed that the filename does not contain the substring `|~~~|'
% which is used as a delimiter).
% Compilation is handed over to the new file by |\childdocforward|:
%    \begin{macrocode}
\newcommand{\childdocforwardprefix}[3][]
{
  \begingroup
    \def\childdocextract #2##1~~~{\def\childdoctmp{\childdocforward[#1]{#3##1}}}
    \expandafter\childdocextract\childdocname~~~
    \expandafter
  \endgroup
  \childdoctmp
}
%    \end{macrocode}

% \macro{\childdoc}
% The deprecated macro |\childdoc| is a legacy version of |\childdocmain|:
%    \begin{macrocode}
\newcommand{\childdoc}{\childdocmain}
%    \end{macrocode}

% \macro{\childdocredirect}
% The deprecated macro |\childdocredirect| is a legacy version
% of |\childdocforward| and |\childdocforwardprefix|:
%    \begin{macrocode}
\newcommand{\childdocredirect}[2][]
{
  \begingroup
    \if?#1?
      \def\childdoctmp{\childdocforward{#2}}
    \else
      \def\childdoctmp{\childdocforwardprefix{#1}{#2}}
    \fi
    \expandafter
  \endgroup
  \childdoctmp
}
%    \end{macrocode}

%\iffalse
%</package>
%\fi
%
\endinput

\childdocmain{}
%    \end{macrocode}

% Optional override for |\version| flag:
%    \begin{macrocode}
%%\ifchilddoc\else\providecommand{\version}{draft}\fi
%    \end{macrocode}

% Define the default values for the |\version| flag
% (|final| for the main file and |draft| for childs):
%    \begin{macrocode}
\ifchilddoc
\providecommand{\version}{draft}
\else
\providecommand{\version}{final}
\fi
%    \end{macrocode}

% Load the standard document class:
%    \begin{macrocode}
\documentclass[12pt]{article}
%    \end{macrocode}

% Start the document body:
%    \begin{macrocode}
\begin{document}
%    \end{macrocode}

% Declare a title page.
% Print title, part of document being processed and version flag:
%    \begin{macrocode}
\addtocounter{page}{-1}
\begin{center}
{\LARGE\bfseries{}childdoc example\par}
\vspace{1cm}
\ifchilddoc
\ifchilddocmanual part\else chapter\fi:
`\childdocname' of `\childdocjob'\par
\else
main document: `\childdocjob'\par
\fi
version: \version\par
\end{center}
\newpage
%    \end{macrocode}

% Manually include selected file,
% otherwise process as usual:
%    \begin{macrocode}
\ifchilddocmanual
\section*{part `\childdocname'}
\input{\childdocname}
\else
%    \end{macrocode}

% Include the two chapters:
%    \begin{macrocode}
\include{cdocsch1}
\include{cdocsch2}
%    \end{macrocode}

% Include the two parts unless only chapters should be displayed:
%    \begin{macrocode}
\ifchilddoc\else
\section{part three}
\input{cdocspt3}
\section{part four}
\input{cdocspt4}
\fi
%    \end{macrocode}

% Process as usual until here:
%    \begin{macrocode}
\fi
%    \end{macrocode}

% End of document body:
%    \begin{macrocode}
\end{document}
%    \end{macrocode}
%\iffalse
%</samplemain>
%\fi
%
% %%%%%%%%%%%%%%%%%%%%%%%%%%%%%%%%%%%%%%
% \paragraph{Chapter Include Files.}
%
% The include files are called |cdocsch1.tex| and |cdocsch2.tex|.
%
%\iffalse
%<*samplechap1|samplechap2>
%\fi

% Optional override for |\version| flag:
%    \begin{macrocode}
%%\providecommand{\version}{final}
%    \end{macrocode}

% Include the main document:
%    \begin{macrocode}
% \iffalse
%
% childdoc.dtx Copyright (C) 2017-2018 Niklas Beisert
%
% This work may be distributed and/or modified under the
% conditions of the LaTeX Project Public License, either version 1.3
% of this license or (at your option) any later version.
% The latest version of this license is in
%   http://www.latex-project.org/lppl.txt
% and version 1.3 or later is part of all distributions of LaTeX
% version 2005/12/01 or later.
%
% This work has the LPPL maintenance status `maintained'.
%
% The Current Maintainer of this work is Niklas Beisert.
%
% This work consists of the files childdoc.dtx and childdoc.ins
% and the derived files childdoc.def and cdocsamp.tex with
% cdocsch1.tex, cdocsch2.tex, cdocsdrf.tex, cdocsfn1.tex, cdocsfn2.tex.
%
%<package>\ifdefined\childdocmain\endinput\fi
%<package>\ProvidesFile{childdoc.def}[2018/12/30 v2.0 child document driver]
%<samplemain>\ProvidesFile{cdocsamp.tex}[2018/12/30 v2.0 sample for childdoc]
%<*driver>
%\ProvidesFile{childdoc.drv}[2018/12/30 v2.0 childdoc reference manual file]
\PassOptionsToClass{10pt,a4paper}{article}
\documentclass{ltxdoc}

\usepackage[margin=35mm]{geometry}
\usepackage{hyperref}
\usepackage{hyperxmp}
\usepackage[usenames]{color}

\hypersetup{colorlinks=true}
\hypersetup{pdfstartview=FitH}
\hypersetup{pdfpagemode=UseNone}
\hypersetup{pdfsource={}}
\hypersetup{pdflang={en-UK}}
\hypersetup{pdfcopyright={Copyright 2017-2018 Niklas Beisert.
  This work may be distributed and/or modified under the
  conditions of the LaTeX Project Public License, either version 1.3
  of this license or (at your option) any later version.}}
\hypersetup{pdflicenseurl={http://www.latex-project.org/lppl.txt}}
\hypersetup{pdfcontactaddress={ETH Zurich, ITP, HIT K,
  Wolfgang-Pauli-Strasse 27}}
\hypersetup{pdfcontactpostcode={8093}}
\hypersetup{pdfcontactcity={Zurich}}
\hypersetup{pdfcontactcountry={Switzerland}}
\hypersetup{pdfcontactemail={nbeisert@itp.phys.ethz.ch}}
\hypersetup{pdfcontacturl={http://people.phys.ethz.ch/\xmptilde nbeisert/}}

\newcommand{\secref}[1]{\hyperref[#1]{section \ref*{#1}}}

\parskip1ex
\parindent0pt
\let\olditemize\itemize
\def\itemize{\olditemize\parskip0pt}

\begin{document}

\title{The \textsf{childdoc} Package}
\hypersetup{pdftitle={The childdoc Package}}
\author{Niklas Beisert\\[2ex]
  Institut f\"ur Theoretische Physik\\
  Eidgen\"ossische Technische Hochschule Z\"urich\\
  Wolfgang-Pauli-Strasse 27, 8093 Z\"urich, Switzerland\\[1ex]
  \href{mailto:nbeisert@itp.phys.ethz.ch}
  {\texttt{nbeisert@itp.phys.ethz.ch}}}
\hypersetup{pdfauthor={Niklas Beisert}}
\hypersetup{pdfsubject={Manual for the LaTeX2e Package childdoc}}
\date{30 December 2018, \textsf{v2.0}}
\maketitle

\begin{abstract}\noindent
\textsf{childdoc} is a \LaTeXe{} package
that enables the direct compilation
of document sections included by |\include|
to individual files.
\end{abstract}

\begingroup
\parskip0ex
\tableofcontents
\endgroup

%%%%%%%%%%%%%%%%%%%%%%%%%%%%%%%%%%%%%%%%%%%%%%%%%%%%%%%%%%%%%%%%%%%%%%%%%%%%%%%%
%%%%%%%%%%%%%%%%%%%%%%%%%%%%%%%%%%%%%%%%%%%%%%%%%%%%%%%%%%%%%%%%%%%%%%%%%%%%%%%%
\section{Introduction}

\LaTeX{} provides a mechanism to structure a large document (such as a book)
into a main file and several child files (containing the chapters)
using the |\include| command.
This mechanism is beneficial for documents
which span hundreds of pages in order to
make the source file(s) more manageable.
Moreover, compilation can be restricted to
selected child files by means of the |\includeonly| command.
The latter feature can be used to reduce the compilation time while editing
(this was significantly more useful in the earlier days of \LaTeX{})
or to generate a smaller document which is easier to navigate.
Another application of |\includeonly| is to generate
documents consisting of selected parts of the complete document.

However, there are a few drawbacks of the plain |\include| mechanism:
\begin{itemize}
\item
The child files cannot be compiled on their own,
they can only be compiled via the main file.
A naive editing environment
(such as a text editor with an option
to have the current file processed by \LaTeX)
may require one to switch to the main file before compiling;
attempting to compile the child file produces errors.
\item
The main file must be modified (each time)
to adjust the |\includeonly| command
to the present needs. This easily leaves the main file in a messy state.
\item
The generated document will always carry the filename
of the main document. This is inconvenient if
several child files are to be compiled and
to be kept for distribution.
\end{itemize}

The present package provides a simple interface
to make child files individually compilable by \LaTeX{}.
Compiling a child file then has the same effect as compiling
the main file with an |\includeonly| command
to select the appropriate child.
Moreover the generated document will carry the name of the child
rather than the main file.
This resolves all three above issues.

This feature is meant to make the editing of books,
thesis documents and lecture notes somewhat more convenient.
However, the package can also be used efficiently for
composing a series of documents (such as exercise sheets)
which are typically distributed individually.
It then assists the author in generating the individual documents
(potentially in different versions)
as well as a document containing the collected series.
Another application is in developing style files
or other kinds of included material
where compilation of the style file could redirect
to a sample or test file.

%%%%%%%%%%%%%%%%%%%%%%%%%%%%%%%%%%%%%%%%%%%%%%%%%%%%%%%%%%%%%%%%%%%%%%%%%%%%%%%%
%%%%%%%%%%%%%%%%%%%%%%%%%%%%%%%%%%%%%%%%%%%%%%%%%%%%%%%%%%%%%%%%%%%%%%%%%%%%%%%%
\section{Usage}

First of all, the package \textsf{childdoc} is \emph{not} a standard
\LaTeXe{} |.sty| style file! Therefore it needs to be invoked in
a non-standard way.

%%%%%%%%%%%%%%%%%%%%%%%%%%%%%%%%%%%%%%%%%%%%%%%%%%%%%%%%%%%%%%%%%%%%%%%%%%%%%%%%
\subsection{Included Files}
\label{sec:include}

%%%%%%%%%%%%%%%%%%%%%%%%%%%%%%%%%%%%%%%%
\DescribeMacro{\childdocmain}
To use the package, add the commands
\begin{center}
\begin{tabular}{l}
|\input{childdoc.def}|\\
|\childdocmain{}|\\
\end{tabular}
\end{center}
at the very top of the main \LaTeX{} file,
in particular \emph{before} the |\documentclass| statement!
The argument of |\childdocmain| should be left empty
(but it must be present).

%%%%%%%%%%%%%%%%%%%%%%%%%%%%%%%%%%%%%%%%
\DescribeMacro{\childdocof}
Furthermore, add the commands
\begin{center}
\begin{tabular}{l}
|\input{childdoc.def}|\\
|\childdocof{|\textit{main}|}|\\
\end{tabular}
\end{center}
at the top of every child file \textit{child}
which is included by |\include{|\textit{child}|}|
from within the main file
(or at least for those files to be compiled individually).
The argument \textit{main} must be the filename of the main file.

There are a couple of
considerations in setting up the main and child documents:

%%%%%%%%%%%%%%%%%%%%%%%%%%%%%%%%%%%%%%%%
\paragraph{Restrictions.}

Please note the following restrictions:
\begin{itemize}
\item
|\childdocmain| must be called with one argument \textit{main}
to ensure compatibility with earlier version of the package.
It must either be empty (|\childdocmain{}|)
or precisely match the filename of the main file in which it is specified.
See \secref{sec:detection} for further information.
\item
The filename \textit{main} must be specified without the |.tex| extension.
\item
The filename \textit{main} is case sensitive
(even in case-insensitive file systems)
due to internal string comparison.
\item
The argument \textit{main} should be fully expanded, it cannot be a macro.
\item
Subdirectories and special characters should be avoided in filenames.
\item
The command |\childdocmain{|\textit{main}|}| must be followed by a whitespace.
It should not be followed immediately by another command
or by a comment mark `|%|'.
This is because the \TeX{} parser reads the token immediately following
the argument of |\childdocmain| and puts it
at the beginning of every child section;
however, a white\-space is ignored.
\end{itemize}

%%%%%%%%%%%%%%%%%%%%%%%%%%%%%%%%%%%%%%%%
\paragraph{Content of Main File.}

It is advisable to place all content in the child files included by |\include|.
Any output contained in the main file will appear in all child documents
unless suppressed manually;
it cannot be suppressed automatically by the |\includeonly| directive
and thus should normally be avoided.
A method to include some content in the main file
by means of conditional processing is described in \secref{sec:conditional}.

%%%%%%%%%%%%%%%%%%%%%%%%%%%%%%%%%%%%%%%%
\paragraph{Page Numbering.}

When only a part of the document is compiled,
the appropriate numbering of pages
(as well as other status parameters)
is determined from the |.aux| files.
The latter contain information from previous passes.
However this information needs to propagate through
all intermediate child documents.
Therefore the page numbering in child documents may well
be inconsistent until the complete document is compiled at least once.

A useful (if unconventional) way to always ensure a consistent
page numbering is to restart the numbering in each child document
and denote the pages by `\textit{child}|.|\textit{page}'
where \textit{child} represents the chapter/section number of the child file.
This can be achieved by the command
|\numberwithin{page}{|\textit{child}|}|
of the \textsf{amsmath} package
where \textit{child} can be |chapter| or |section|
depending on the chosen structuring.
Alternatively, one can modify the macro |\thepage| appropriately
and reset the counter |page| at the start of each child file.

%%%%%%%%%%%%%%%%%%%%%%%%%%%%%%%%%%%%%%%%%%%%%%%%%%%%%%%%%%%%%%%%%%%%%%%%%%%%%%%%
\subsection{Conditional Processing}
\label{sec:conditional}

The package provides a mechanism to compile different versions
of a document. To customise the versions further some conditional processing
can come in handy to distinguish which version is being compiled.
The package provides two macros to describe the compilation context:

%%%%%%%%%%%%%%%%%%%%%%%%%%%%%%%%%%%%%%%%
\DescribeMacro{\ifchilddoc}
The conditional |\ifchilddoc| distinguishes between the compilation of
child documents and the main document:
%
\begin{center}
|\ifchilddoc |\textit{child-code}| |[|\||else |\textit{main-code}]| \||fi|
\end{center}

%%%%%%%%%%%%%%%%%%%%%%%%%%%%%%%%%%%%%%%%
\DescribeMacro{\childdocname}
\DescribeMacro{\childdocjob}
The macro |\childdocname| contains the filename (without extension)
of the main or child file being processed.
Note that |\childdocjob| will always contain the name of the main file.

%%%%%%%%%%%%%%%%%%%%%%%%%%%%%%%%%%%%%%%%
\paragraph{Title Page.}

Conditional processing can be used to include a title or banner page
in the main document when proper precautions are taken.
Importantly, the code in the main file should ensure that the page counter
(as well as other status parameters which are stored in the |.aux| files)
takes the same value after the conditional processing.
Otherwise the page numbers may take divergent values
depending on which part is compiled.

For example, a title page could be declared by:
%
\begin{center}
\begin{tabular}{l}
|\ifchilddoc\||else|\\
|\addtocounter{page}{-1}|\\
\textit{code for title page}\\
|\newpage|\\
|\||fi|
\end{tabular}
\end{center}
%
A banner page for the child documents can be generated by:
%
\begin{center}
\begin{tabular}{l}
|\ifchilddoc|\\
|\addtocounter{page}{-1}|\\
\textit{code for banner page}\\
|\newpage|\\
|\||fi|
\end{tabular}
\end{center}
%
Here one could write a message such as:
\begin{center}
|This is the part \childdocname{} of \childdocjob{}.|
\end{center}

%%%%%%%%%%%%%%%%%%%%%%%%%%%%%%%%%%%%%%%%%%%%%%%%%%%%%%%%%%%%%%%%%%%%%%%%%%%%%%%%
\subsection{Flags}
\label{sec:flags}

The package makes it easy to generate different versions
of the main or child documents.
To this end compilation flags can be defined
and assigned different default values.
They will be particularly useful in conjunction
with the forwarding mechanism described in \secref{sec:forward}.

For example, it may be useful to have a flag |\version|
which can be set to |draft| or |final|.
The document source will contain some conditional code
depending on the value of |\version|.
Suppose further, the flag should default to |final| for the main file
and to |draft| for child files
which is a natural assignment for editing the document.
This is achieved by placing the following code
in the preamble of the main document
(below the |\childdocmain| directive):
%
\begin{center}
\begin{tabular}{l}
|\ifchilddoc|\\
|\providecommand{\version}{draft}|\\
|\||else|\\
|\providecommand{\version}{final}|\\
|\||fi|
\end{tabular}
\end{center}
%
The definition by |\providecommand| makes sure
that previous definitions are not overwritten.
Further statements |\providecommand{\version}{...}|
can thus be added before the above code to override it.

For the main file, one might add a line
(between |\childdocmain| and the above block)
%
\begin{center}
|%\ifchilddoc\||else\providecommand{\version}{draft}\||fi|
\end{center}
%
which can be uncommented to produce a draft version.
Likewise one can add a line to the very top of a child file
(above the |\childdocof{|\textit{main}|}| directive)
%
\begin{center}
|%\providecommand{\version}{final}|
\end{center}
%
which can be uncommented to produce the final version of this child document.

%%%%%%%%%%%%%%%%%%%%%%%%%%%%%%%%%%%%%%%%%%%%%%%%%%%%%%%%%%%%%%%%%%%%%%%%%%%%%%%%
\subsection{Forwarding}
\label{sec:forward}

Different versions of the main or child documents
using compilation flags as described in \secref{sec:flags}
can be (permanently) stored in different files
for convenient compilation, viewing and distribution.
To this end, the package defines a command
to pass on compilation to a different file:

%%%%%%%%%%%%%%%%%%%%%%%%%%%%%%%%%%%%%%%%
\DescribeMacro{\childdocforward}
The command |\childdocforward| redirects processing to
another source file:
%
\begin{center}
\begin{tabular}{l}
|\input{childdoc.def}|\\
|\childdocforward[|\textit{main}|]{|\textit{dest}|}|\\
\end{tabular}
\end{center}
%
The argument \textit{dest} is the destination file
(without extension).
It should be the main file or one of the child files.
Note that further \textsf{childdoc} directives
such as |\childdocof| and |\childdocforward|
in the indicated file will be processed in this form.
The optional argument \textit{main}
passes on directly to the main file \textit{main}
while pretending to compile the child \textit{dest}.
This form behaves as if \textit{dest}
issues |\childdocof{|\textit{main}|}| right away,
and no further \textsf{childdoc} directives will be processed.

%%%%%%%%%%%%%%%%%%%%%%%%%%%%%%%%%%%%%%%%
\DescribeMacro{\...prefix}
In the alternative form |\childdocforwardprefix|,
%
\begin{center}
\begin{tabular}{l}
|\input{childdoc.def}|\\
|\childdocforwardprefix[|\textit{main}|]{|\textit{prefix}|}{|\textit{dest}|}|
\end{tabular}
\end{center}
%
the destination file is determined by a pattern
depending on the current file:
To make this work, the current file must be called
`{\textit{prefix}\hspace{0.2em}\textit{suffix}}'
with \textit{prefix} matching precisely the argument.
Processing is then passed on to the file
`{\textit{dest}\hspace{0.2em}\textit{suffix}}'.
Surely, the same effect is achieved by
directly specifying the
argument `{\textit{dest}\hspace{0.2em}\textit{suffix}}'
in the first form.
However, that requires to set up a different file
for each child. With the alternative form of the command
all these files can have exactly the same content
which simplifies setting them up and maintaining them.

For example, the following file |draft.tex|
with a compilation flag |\version| as described in \secref{sec:flags}
compiles the main document as a draft:
%
\begin{center}
\begin{tabular}{l}
|\def\version{draft}|\\
|\input{childdoc.def}|\\
|\childdocforward{|\textit{main}|}|
\end{tabular}
\end{center}
%
Likewise, the following files |final|\textit{nn}|.tex|
compile the final version of the child document
|child|\textit{nn}|.tex|:
%
\begin{center}
\begin{tabular}{l}
|\def\version{final}|\\
|\input{childdoc.def}|\\
|\childdocforwardprefix{final}{child}|
\end{tabular}
\end{center}
%

Note that when several versions of a main file and/or of each child file
are to be generated, it may be convenient to set up a |Makefile| or
shell script to automatise the process.

%%%%%%%%%%%%%%%%%%%%%%%%%%%%%%%%%%%%%%%%%%%%%%%%%%%%%%%%%%%%%%%%%%%%%%%%%%%%%%%%
\subsection{Command Line Processing}
\label{sec:commandline}

The effect of redirection files can also be achieved by invoking
the \LaTeX{} compiler with a more elaborate command line.
Most conveniently this should be done as part
of a shell script or a |Makefile|.

When using \textsf{childdoc} in the main file, the following
command lines effectively perform a redirection
(note that depending on the shell being used,
backslashes may have to be doubled: `|\|' $\to$ `|\\|'):
%
\begin{center}
|... -jobname "|\textit{target}|" |\\|"|[\textit{flags}]%
|\input{childdoc.def}\childdocforward[|\textit{main}|]{|\textit{dest}|}"|
\end{center}
%
Here \textit{target} is the name of the output file,
\textit{main} is the name of the main file
and \textit{dest} is the name of the main or child file to be processed
(all filenames without extensions).
The optional argument \textit{main} can be omitted
if \textit{main} matches \textit{dest}.
Optionally, compilation \textit{flags} can be defined via |\def| commands.
This command line makes the \TeX{} engine believe
it is compiling the file \textit{target}
whose content is specified as the latter parameter.
The provided code then forwards the processing to
\textit{main} or \textit{dest} as described in \secref{sec:forward}.

%%%%%%%%%%%%%%%%%%%%%%%%%%%%%%%%%%%%%%%%%%%%%%%%%%%%%%%%%%%%%%%%%%%%%%%%%%%%%%%%
\subsection{Include by Input}
\label{sec:input}

Including child documents by |\include| has some restrictions by design.
Most notably, the content of a child document always occupies
its own set of pages; pages cannot be shared between child documents.
Usually, this behaviour makes perfect sense
because each child document contain an essential part of the document.
However, in some situations it may be desirable to compose
a document from a collection of parts
without having mandatory page breaks between then.
For this case, the package
provides a mechanism to include parts
by |\input| which can also be processed individually.
However, by construction this mechanism
requires manual handling of the content to be output.

%%%%%%%%%%%%%%%%%%%%%%%%%%%%%%%%%%%%%%%%
\DescribeMacro{\ifchilddocmanual}
The main file should be prepared as usual, see \secref{sec:include}.
However, the document body must make a distinction
between processing of an individual part and of the main document, e.g.:
%
\begin{center}
\begin{tabular}{l}
|\ifchilddocmanual|\\
|\input{\childdocname}|\\
|\||else|\\
\textit{document body with }|\input{|\textit{part}|}|\\
|\||fi|
\end{tabular}
\end{center}
%
The conditional |\ifchilddocmanual| is true whenever
a part to be included by |\input| is being compiled,
and the name of the part is stored in |\childdocname|.

%%%%%%%%%%%%%%%%%%%%%%%%%%%%%%%%%%%%%%%%
\DescribeMacro{\childdocby}
Each part to be included by |\input| should start with:
%
\begin{center}
\begin{tabular}{l}
|\input{childdoc.def}|\\
|\childdocby{|\textit{main}|}|\\
\end{tabular}
\end{center}
%
The directive |\childdocby| is similar to |\childdocof|
described in \secref{sec:include},
but the subsequent selection of content must be done manually.
To that end, both |\ifchilddoc| and |\ifchilddocmanual|
will be true upon processing of a part,
and the name of the part is stored in |\childdocname|.
Note that |\jobname| will be set to the filename of the current part
so that each part receives an individual |.aux| file
that does not interfere with the |.aux| file(s) of the main document.
This behaviour can be altered by the alternative form
|\childdocby[*]{|\textit{main}|}| (with a non-empty optional argument)
which uses the |.aux| file of the main document
by setting |\jobname| to \textit{main}.

%%%%%%%%%%%%%%%%%%%%%%%%%%%%%%%%%%%%%%%%%%%%%%%%%%%%%%%%%%%%%%%%%%%%%%%%%%%%%%%%
\subsection{Driver Development}
\label{sec:driver}

The \textsf{childdoc} mechanism can also be use for the development
of definition files such as \LaTeX{} styles or classes.
This case differs from the above setup with multiple parts
included by |\include| in that no |\includeonly| should be invoked.
This can be achieved by starting the include file
(before |\ProvidesPackage|) with:
%
\begin{center}
\begin{tabular}{l}
|\input{childdoc.def}|\\
|\childdocforward{|\textit{main}|}|\\
\end{tabular}
\end{center}
%
or alternatively with:
%
\begin{center}
\begin{tabular}{l}
|\input{childdoc.def}|\\
|\childdocby{|\textit{main}|}|\\
\end{tabular}
\end{center}
%
Both forms have slightly different effects as described above.
The main file is prepared as usual, see \secref{sec:include}.

%%%%%%%%%%%%%%%%%%%%%%%%%%%%%%%%%%%%%%%%%%%%%%%%%%%%%%%%%%%%%%%%%%%%%%%%%%%%%%%%
\subsection{Legacy Detection}
\label{sec:detection}

The directive |\childdocmain| in the main file can detect
whether the complete document or merely a child is to be compiled
even without using the directive |\childdocof|.
This method is deprecated because it is less robust
and there is no compelling reason to use it;
it is merely provided for backward compatibility
and it may be removed in future versions.

If the detection mechanism is to be used,
it is mandatory to correctly specify
the filename of the main file as the argument of |\childdocmain|:
%
\begin{center}
\begin{tabular}{l}
|\input{childdoc.def}|\\
|\childdocmain{|\textit{main}|}|\\
\end{tabular}
\end{center}
%
If |\jobname| does not match the argument \textit{main} of |\childdocmain|,
it is assumed that |\jobname| points to the child file to be compiled.
When using |\childdocmain| with the main file specified as argument,
it suffices to start a child file
with just |\input{|\textit{main}|}|
without loading of the package and using |\childdocof|.
If instead all processing is done
with the appropriate \textsf{childdoc} directives,
the argument of \textit{main} of |\childdocmain| can be empty.

An alternative version of the command line processing described
in \secref{sec:commandline} using the detection mechanism reads:
%
\begin{center}
|... -jobname "|\textit{target}|" "|[\textit{flags}]%
[|\def\jobname{|\textit{dest}|}|]|\input{|\textit{main}|}"|
\end{center}

%%%%%%%%%%%%%%%%%%%%%%%%%%%%%%%%%%%%%%%%%%%%%%%%%%%%%%%%%%%%%%%%%%%%%%%%%%%%%%%%
\subsection{Manual Code}
\label{sec:manual}

In case one cannot be certain whether the definitions file |childdoc.def|
is installed on the target \TeX{} distribution
and one prefers not to ship it,
it is conceivable to paste a few relevant commands into the sources.

To that end, drop all statements |\input{childdoc.def}|
and perform the replacements as outlined below.
Instead of |\childdocmain{|\textit{main}|}| add the following code
to the top of the main file:
%
\begin{center}
\begin{tabular}{l}
|\||ifdefined\childdocname\endinput\||fi\newif\ifchilddoc|\\
|\edef\childdocname{\scantokens\expandafter{\jobname\noexpand}}|\\
|\def\childdocmain{|\textit{main}|}\||ifx\childdocmain\childdocname\||else|\\
|\childdoctrue\includeonly{\childdocname}\let\jobname\childdocmain\||fi|\\
\end{tabular}
\end{center}
%
Instead of |\childdocof{|\textit{main}|}| just include the main file
at the top of each child file:
%
\begin{center}
|\input{|\textit{main}|}|
\end{center}
%
A simple redirection |\childdocforward{|\textit{dest}|}| is achieved by:
%
\begin{center}
|\def\jobname{|\textit{dest}|}\input{\jobname}|
\end{center}
%
The redirection with prefix
|\childdocforwardprefix[|\textit{prefix}|]{|\textit{dest}|}|
is accomplished by:
%
\begin{center}
\begin{tabular}{l}
|{\edef\jobname{\scantokens\expandafter{\jobname\noexpand}}|\\
|\def\redirectjob |\textit{prefix}|#1~~~{\gdef\jobname{|\textit{dest}|#1}}|\\
|\expandafter\redirectjob\jobname~~~}\input{\jobname}|
\end{tabular}
\end{center}

In an alternative approach,
child documents can be compiled by a specific command line
without additional code or specific definitions:
%
\begin{center}
|... -jobname "|\textit{target}|" "|[\textit{flags}]%
|\includeonly{|\textit{dest}|}\input{|\textit{main}|}"|
\end{center}
%

%%%%%%%%%%%%%%%%%%%%%%%%%%%%%%%%%%%%%%%%%%%%%%%%%%%%%%%%%%%%%%%%%%%%%%%%%%%%%%%%
%%%%%%%%%%%%%%%%%%%%%%%%%%%%%%%%%%%%%%%%%%%%%%%%%%%%%%%%%%%%%%%%%%%%%%%%%%%%%%%%
\section{Information}

%%%%%%%%%%%%%%%%%%%%%%%%%%%%%%%%%%%%%%%%%%%%%%%%%%%%%%%%%%%%%%%%%%%%%%%%%%%%%%%%
\subsection{Copyright}

Copyright \copyright{} 2017--2018 Niklas Beisert

This work may be distributed and/or modified under the
conditions of the \LaTeX{} Project Public License, either version 1.3
of this license or (at your option) any later version.
The latest version of this license is in
  \url{http://www.latex-project.org/lppl.txt}
and version 1.3 or later is part of all distributions of \LaTeX{}
version 2005/12/01 or later.

This work has the LPPL maintenance status `maintained'.

The Current Maintainer of this work is Niklas Beisert.

This work consists of the files |README.txt|, |childdoc.ins| and |childdoc.dtx|
as well as the derived files |childdoc.def|, |cdocsamp.tex|
with |cdocsch1.tex|, |cdocsch2.tex|, |cdocspt3.tex|, |cdocspt4.tex|,
|cdocsdrf.tex|, |cdocsfn1.tex|, |cdocsfn2.tex|
as well as |childdoc.pdf|.

%%%%%%%%%%%%%%%%%%%%%%%%%%%%%%%%%%%%%%%%%%%%%%%%%%%%%%%%%%%%%%%%%%%%%%%%%%%%%%%%
\subsection{Files and Installation}

The package consists of the files:
%
\begin{center}
\begin{tabular}{ll}
    |README.txt|   & readme file \\
    |childdoc.ins| & installation file \\
    |childdoc.dtx| & source file \\
    |childdoc.def| & definition file \\
    |cdocsamp.tex| & sample main file \\
    |cdocsch1.tex| & sample include file \\
    |cdocsch2.tex| & sample include file \\
    |cdocspt3.tex| & sample part file \\
    |cdocspt4.tex| & sample part file \\
    |cdocsdrf.tex| & sample redirection file \\
    |cdocsfn1.tex| & sample redirection file \\
    |cdocsfn2.tex| & sample redirection file \\
    |childdoc.pdf| & manual
\end{tabular}
\end{center}
%
The distribution consists of the files
|README.txt|, |childdoc.ins| and |childdoc.dtx|.
%
\begin{itemize}
\item
Run (pdf)\LaTeX{} on |childdoc.dtx|
to compile the manual |childdoc.pdf| (this file).
\item
Run \LaTeX{} on |childdoc.ins| to create the definitions file |childdoc.def|
and the sample |cdocsamp.tex| with include files
|cdocsch1.tex|, |cdocsch2.tex|, |cdocspt3.tex|, |cdocspt4.tex|,
|cdocsdrf.tex|, |cdocsfn1.tex|, |cdocsfn2.tex|.
Then copy the file |childdoc.def| to an appropriate directory of your \LaTeX{}
distribution, e.g.\ \textit{texmf-root}|/tex/latex/childdoc|.
\end{itemize}

%%%%%%%%%%%%%%%%%%%%%%%%%%%%%%%%%%%%%%%%%%%%%%%%%%%%%%%%%%%%%%%%%%%%%%%%%%%%%%%%
\subsection{Related CTAN Packages}

There are several other packages which offer a similar functionality:
%
\begin{itemize}
\item
The packages
\href{http://ctan.org/pkg/docmute}{\textsf{docmute}},
\href{http://ctan.org/pkg/includex}{\textsf{includex}} and
\href{http://ctan.org/pkg/standalone}{\textsf{standalone}}
provide commands to include only the document body of
a child file thus allowing both files to be compiled individually.
\item
The packages \href{http://ctan.org/pkg/subdocs}{\textsf{subdocs}}
and \href{http://ctan.org/pkg/subfiles}{\textsf{subfiles}}
provide structures in which the main and child documents can be
encapsulated and allowing them to be compiled individually.
The inclusion mechanism is different from the conventional |\include|.
\item
The package \href{http://ctan.org/pkg/combine}{\textsf{combine}}
is an elaborate solution to combine several documents into one.
\end{itemize}
%
See also the CTAN topic \href{http://ctan.org/topic/subdocs}{\textsf{subdocs}}
for further related packages.
The present package differs from the above solutions in that
a document structure constructed with the conventional |\include| mechanism
just needs two extra commands at the top of every file
such that all constituent files can be compiled individually.

%%%%%%%%%%%%%%%%%%%%%%%%%%%%%%%%%%%%%%%%%%%%%%%%%%%%%%%%%%%%%%%%%%%%%%%%%%%%%%%%
%\subsection{Feature Suggestions}
%
%The following is a list of features which may be useful for future
%versions of this package:
%%
%\begin{itemize}
%\item
%\ldots
%\end{itemize}

%%%%%%%%%%%%%%%%%%%%%%%%%%%%%%%%%%%%%%%%%%%%%%%%%%%%%%%%%%%%%%%%%%%%%%%%%%%%%%%%
\subsection{Revision History}

%%%%%%%%%%%%%%%%%%%%%%%%%%%%%%%%%%%%%%%%
\paragraph{v2.0:} 2018/12/30

\begin{itemize}
\item
immediate forward processing
\item
added |\childdocby| mechanism
\item
manual restructured
\end{itemize}

%%%%%%%%%%%%%%%%%%%%%%%%%%%%%%%%%%%%%%%%
\paragraph{v1.6:} 2018/01/17

\begin{itemize}
\item
application for development of include files
\item
corrections to manual
\end{itemize}

%%%%%%%%%%%%%%%%%%%%%%%%%%%%%%%%%%%%%%%%
\paragraph{v1.5:} 2017/05/21

\begin{itemize}
\item
more complete structuring introduced
\item
|\childdocof| introduced
\item
|\childdoc| renamed to |\childdocmain|
\item
|\childredirect| renamed to |\childdocforward| and |\childdocforwardprefix|
and functionality expanded
\end{itemize}

%%%%%%%%%%%%%%%%%%%%%%%%%%%%%%%%%%%%%%%%
\paragraph{v1.0:} 2017/04/27

\begin{itemize}
\item
manual and install package
\item
first version published on CTAN
\end{itemize}

%%%%%%%%%%%%%%%%%%%%%%%%%%%%%%%%%%%%%%%%
\paragraph{v0.6:} 2017/04/26

\begin{itemize}
\item
redirection mechanism added
\end{itemize}

%%%%%%%%%%%%%%%%%%%%%%%%%%%%%%%%%%%%%%%%
\paragraph{v0.5:} 2017/04/26

\begin{itemize}
\item
functionality in definition file
\end{itemize}


%%%%%%%%%%%%%%%%%%%%%%%%%%%%%%%%%%%%%%%%%%%%%%%%%%%%%%%%%%%%%%%%%%%%%%%%%%%%%%%%
%%%%%%%%%%%%%%%%%%%%%%%%%%%%%%%%%%%%%%%%%%%%%%%%%%%%%%%%%%%%%%%%%%%%%%%%%%%%%%%%
%%%%%%%%%%%%%%%%%%%%%%%%%%%%%%%%%%%%%%%%%%%%%%%%%%%%%%%%%%%%%%%%%%%%%%%%%%%%%%%%
\appendix

\settowidth\MacroIndent{\rmfamily\scriptsize 000\ }

 \DocInput{childdoc.dtx}

\end{document}
%</driver>
% \fi
%
% %%%%%%%%%%%%%%%%%%%%%%%%%%%%%%%%%%%%%%%%%%%%%%%%%%%%%%%%%%%%%%%%%%%%%%%%%%%%%%
% %%%%%%%%%%%%%%%%%%%%%%%%%%%%%%%%%%%%%%%%%%%%%%%%%%%%%%%%%%%%%%%%%%%%%%%%%%%%%%
% \section{Sample}
%\iffalse
%<*samplemain>
%\fi
%
% The following presents a sample document
% with two chapters, two parts, a title page,
% a compile flag as well as three forwarding files to set the flag.
% It consists of eight |.tex| files:
% \begin{center}
% \begin{tabular}{ll}
% |cdocsamp.tex|&main file\\
% |cdocsch1.tex|&include file for chapter 1\\
% |cdocsch2.tex|&include file for chapter 2\\
% |cdocspt3.tex|&include file for part 3\\
% |cdocspt4.tex|&include file for part 4\\
% |cdocsdrf.tex|&forwarding file for main file in draft mode\\
% |cdocsfi1.tex|&forwarding file for final version of chapter 1\\
% |cdocsfi2.tex|&forwarding file for final version of chapter 2\\
% \end{tabular}
% \end{center}
% Each of the eight files can be compiled directly by the \LaTeX{} compiler.
%
% %%%%%%%%%%%%%%%%%%%%%%%%%%%%%%%%%%%%%%
% \paragraph{Main File.}
%
% The main file is called |cdocsamp.tex|.
%
% Load the \textsf{childdoc} definitions and
% declare the filename for the main document:
%    \begin{macrocode}
\input{childdoc.def}
\childdocmain{}
%    \end{macrocode}

% Optional override for |\version| flag:
%    \begin{macrocode}
%%\ifchilddoc\else\providecommand{\version}{draft}\fi
%    \end{macrocode}

% Define the default values for the |\version| flag
% (|final| for the main file and |draft| for childs):
%    \begin{macrocode}
\ifchilddoc
\providecommand{\version}{draft}
\else
\providecommand{\version}{final}
\fi
%    \end{macrocode}

% Load the standard document class:
%    \begin{macrocode}
\documentclass[12pt]{article}
%    \end{macrocode}

% Start the document body:
%    \begin{macrocode}
\begin{document}
%    \end{macrocode}

% Declare a title page.
% Print title, part of document being processed and version flag:
%    \begin{macrocode}
\addtocounter{page}{-1}
\begin{center}
{\LARGE\bfseries{}childdoc example\par}
\vspace{1cm}
\ifchilddoc
\ifchilddocmanual part\else chapter\fi:
`\childdocname' of `\childdocjob'\par
\else
main document: `\childdocjob'\par
\fi
version: \version\par
\end{center}
\newpage
%    \end{macrocode}

% Manually include selected file,
% otherwise process as usual:
%    \begin{macrocode}
\ifchilddocmanual
\section*{part `\childdocname'}
\input{\childdocname}
\else
%    \end{macrocode}

% Include the two chapters:
%    \begin{macrocode}
\include{cdocsch1}
\include{cdocsch2}
%    \end{macrocode}

% Include the two parts unless only chapters should be displayed:
%    \begin{macrocode}
\ifchilddoc\else
\section{part three}
\input{cdocspt3}
\section{part four}
\input{cdocspt4}
\fi
%    \end{macrocode}

% Process as usual until here:
%    \begin{macrocode}
\fi
%    \end{macrocode}

% End of document body:
%    \begin{macrocode}
\end{document}
%    \end{macrocode}
%\iffalse
%</samplemain>
%\fi
%
% %%%%%%%%%%%%%%%%%%%%%%%%%%%%%%%%%%%%%%
% \paragraph{Chapter Include Files.}
%
% The include files are called |cdocsch1.tex| and |cdocsch2.tex|.
%
%\iffalse
%<*samplechap1|samplechap2>
%\fi

% Optional override for |\version| flag:
%    \begin{macrocode}
%%\providecommand{\version}{final}
%    \end{macrocode}

% Include the main document:
%    \begin{macrocode}
\input{childdoc.def}
\childdocof{cdocsamp}
%    \end{macrocode}

%\iffalse
%</samplechap1|samplechap2>
%\fi
%
%\iffalse
%<*samplechap1>
%\fi
% Some text for chapter 1:
%    \begin{macrocode}
\section{one}
some text in chapter one
%    \end{macrocode}

%\iffalse
%</samplechap1>
%\fi
% Some text for chapter 2:
%\iffalse
%<*samplechap2>
%\fi
%    \begin{macrocode}
\section{two}
more text in chapter two
%    \end{macrocode}

%\iffalse
%</samplechap2>
%\fi
%
% %%%%%%%%%%%%%%%%%%%%%%%%%%%%%%%%%%%%%%
% \paragraph{Part Include Files.}
%
% The include files are called |cdocspt3.tex| and |cdocspt4.tex|.
%
%\iffalse
%<*samplepart3|samplepart4>
%\fi

% Optional override for |\version| flag:
%    \begin{macrocode}
%%\providecommand{\version}{final}
%    \end{macrocode}

% Include the main document:
%    \begin{macrocode}
\input{childdoc.def}
\childdocby{cdocsamp}
%    \end{macrocode}

%\iffalse
%</samplepart3|samplepart4>
%\fi
%
%\iffalse
%<*samplepart3>
%\fi
% Some text for part 3:
%    \begin{macrocode}
some text in part three
%    \end{macrocode}

%\iffalse
%</samplepart3>
%\fi
% Some text for part 4:
%\iffalse
%<*samplepart4>
%\fi
%    \begin{macrocode}
more text in part four
%    \end{macrocode}

%\iffalse
%</samplepart4>
%\fi
%
% %%%%%%%%%%%%%%%%%%%%%%%%%%%%%%%%%%%%%%
% \paragraph{Forwarding for a Complete Draft.}
%
% The following forwarding file |cdocsdrf.tex|
% compiles the main document in draft mode:
%\iffalse
%<*sampledraft>
%\fi
%    \begin{macrocode}
\def\version{draft}
\input{childdoc.def}
\childdocforward{cdocsamp}
%    \end{macrocode}

%\iffalse
%</sampledraft>
%\fi
%
% %%%%%%%%%%%%%%%%%%%%%%%%%%%%%%%%%%%%%%
% \paragraph{Forwarding for Final Version of the Chapters.}
%
% The following forwarding files |cdocsfn1.tex| and |cdocsfn2.tex|
% (with identical content)
% compile the final versions of the child documents
% |cdocsch1.tex| and |cdocsch2.tex|, respectively:
%\iffalse
%<*samplefinal>
%\fi
%    \begin{macrocode}
\def\version{final}
\input{childdoc.def}
\childdocforwardprefix[cdocsamp]{cdocsfn}{cdocsch}
%    \end{macrocode}

%\iffalse
%</samplefinal>
%\fi
%
% %%%%%%%%%%%%%%%%%%%%%%%%%%%%%%%%%%%%%%
% \paragraph{Command Line Processing.}
%
% The following three command lines generate the output files
% |cdocscld|, |cdocscl1| and |cdocscl2|
% which should be identical to
% |cdocsdrf|, |cdocsch1| and |cdocsfn2|, respectively:
% \begin{center}
% \begin{tabular}{l}
% |latex -jobname cdocscld \|\\
% |  "\def\version{draft}\input{childdoc.def}\childdocforward{cdocsamp}"|\\
% |latex -jobname cdocscl1 \|\\
% |  "\input{childdoc.def}\childdocforward[cdocsamp]{cdocsch1}"|\\
% |latex -jobname cdocscl2 \|\\
% |  "\def\version{final}\input{childdoc.def}\childdocforward{cdocsch2}"|
% \end{tabular}
% \end{center}
% Note that the trailing backslash on each first line
% merely continues the input to the second line
% (for convenient cut ant paste).
% Furthermore, the command |latex| can be replaced by any
% of its alternative versions such as |pdflatex|.
%
% %%%%%%%%%%%%%%%%%%%%%%%%%%%%%%%%%%%%%%%%%%%%%%%%%%%%%%%%%%%%%%%%%%%%%%%%%%%%%%
% %%%%%%%%%%%%%%%%%%%%%%%%%%%%%%%%%%%%%%%%%%%%%%%%%%%%%%%%%%%%%%%%%%%%%%%%%%%%%%
% \section{Implementation}
%\iffalse
%<*package>
%\fi
%
% This section describes the definitions file |childdoc.def|.

% The definitions cannot be loaded using |\usepackage| or |\RequirePackage|
% which has a mechanism to prevent loading a style file more than once.
% When loading the definitions by means of |\input|
% multiple instances have to be prevented manually:
%\iffalse
%This code needs to be before the `\ProvidesFile' directive
%which is defined at the beginning of this file.
%Therefore it is also placed there and commented out here.
%</package>
%<*discard>
%\fi
%    \begin{macrocode}
\ifdefined\childdocmain\endinput\fi
%    \end{macrocode}
%\iffalse
%</discard>
%<*package>
%\fi
%
% \macro{\ifchilddoc}
% \macro{\ifchilddocmanual}
% The conditional |\ifchilddoc| tells whether a
% child (true) or main (false) document is being compiled.
% The conditional |\ifchilddocmanual| tells whether
% the |\includeonly| mechanism is used (false) or
% the selection of child files must be performed manually (true).
% The definitions initialise to false:
%    \begin{macrocode}
\newif\ifchilddoc
\newif\ifchilddocmanual
%    \end{macrocode}

% \macro{\childdocname}
% \macro{\childdocjob}
% The macro |\childdocname| stores the name of the main document
% to be compiled. The macro |\childdocjob| stores the name of
% the document on which the \LaTeX{} compiler was originally invoked.
% The content of |\jobname| cannot be compared
% to filenames specified in the source due to different catcodes.
% The following code rescans |\jobname|, stores the result
% in |\childdocname| and saves a copy in |\childdocjob|:
%    \begin{macrocode}
\edef\childdocname{\scantokens\expandafter{\jobname\noexpand}}
\let\childdocjob\childdocname
%    \end{macrocode}

% \macro{\childdocdisable}
% The macro |\childdocdisable| prevents the main file
% from being processed more than once.
% At this stage, the main document command |\childdocmain|
% is assumed to be called once again where it should do nothing.
% Any subsequent call to it should prevent
% a secondary processing of the main document
% It overwrites the forwarding commands
% |\childdocof| and |\childdocforward|
% with empty macros to prevent further inclusions of the main document:
%    \begin{macrocode}
\newcommand{\childdocdisable}
{
  \renewcommand{\childdocmain}[1]{\renewcommand{\childdocmain}[1]{\endinput}}
  \renewcommand{\childdocof}[1]{}
  \renewcommand{\childdocby}[2][]{}
  \renewcommand{\childdocforward}[2][]{}
  \renewcommand{\childdocdisable}{}
}
%    \end{macrocode}

% \macro{\childdocmain}
% The macro |\childdocmain| is to be called at the top of the main file
% with nothing or the main filename (without extension) as argument.
% First, it breaks loops.
% If the argument is not empty and does not match |\childdocname|
% (which is set by the first inclusion of |childdoc.def|),
% |\ifchilddoc| is set to true, |\includeonly| is applied to the child file
% and |\jobname| is set to the main file
% (for proper handling of |.aux| files):
%    \begin{macrocode}
\newcommand{\childdocmain}[1]
{
  \childdocdisable\childdocmain{}
  \if?#1?\else
    \begingroup
      \def\childdoctmp{#1}
      \ifx\childdoctmp\childdocname
        \def\childdoctmp{}
      \else
        \def\childdoctmp
        {
          \childdoctrue
          \includeonly{\childdocname}
          \def\childdocjob{#1}
          \def\jobname{#1}
        }
      \fi
      \expandafter
    \endgroup
    \childdoctmp
  \fi
}
%    \end{macrocode}

% \macro{\childdocof}
% The command |\childdocof| redirects
% compilation to the main file |#1|.
%    \begin{macrocode}
\newcommand{\childdocof}[1]
{
  \childdocdisable
  \childdoctrue
  \includeonly{\childdocname}
  \def\jobname{#1}
  \def\childdocjob{#1}
  \input{#1}
}
%    \end{macrocode}

% \macro{\childdocby}
% The command |\childdocby| ....
%    \begin{macrocode}
\newcommand{\childdocby}[2][]
{
  \childdocdisable
  \childdoctrue
  \childdocmanualtrue
  \if?#1?\else
    \def\jobname{#2}
  \fi
  \def\childdocjob{#2}
  \input{#2}
  \endinput
}
%    \end{macrocode}

% \macro{\childdocforward}
% The command |\childdocforward| redirects
% compilation to the main file or
% (if the optional argument is given) a child file.
% Parameters are set as if the main file
% or a child file starting with |\childdocof| was compiled.
% Then compilation is handed over to the main file:
%    \begin{macrocode}
\newcommand{\childdocforward}[2][]
{
  \begingroup
    \if?#1?
      \def\childdoctmp
      {
        \def\childdocname{#2}
        \def\childdocjob{#2}
        \def\jobname{#2}
        \input{#2}
        \endinput
      }
    \else
      \def\childdoctmp
      {
        \childdocdisable
        \def\childdocname{#2}
        \childdoctrue
        \includeonly{#2}
        \def\childdocjob{#1}
        \def\jobname{#1}
        \input{#1}
        \endinput
      }
    \fi
    \expandafter
  \endgroup
  \childdoctmp
}
%    \end{macrocode}

% \macro{\childdocforwardprefix}
% The command |\childdocforwardprefix| redirects
% compilation to the main or a child file by means of a pattern.
% The prefix |#1| in the current filename is replaced by |#2|
% and the suffix of the current filename is kept
% (it is assumed that the filename does not contain the substring `|~~~|'
% which is used as a delimiter).
% Compilation is handed over to the new file by |\childdocforward|:
%    \begin{macrocode}
\newcommand{\childdocforwardprefix}[3][]
{
  \begingroup
    \def\childdocextract #2##1~~~{\def\childdoctmp{\childdocforward[#1]{#3##1}}}
    \expandafter\childdocextract\childdocname~~~
    \expandafter
  \endgroup
  \childdoctmp
}
%    \end{macrocode}

% \macro{\childdoc}
% The deprecated macro |\childdoc| is a legacy version of |\childdocmain|:
%    \begin{macrocode}
\newcommand{\childdoc}{\childdocmain}
%    \end{macrocode}

% \macro{\childdocredirect}
% The deprecated macro |\childdocredirect| is a legacy version
% of |\childdocforward| and |\childdocforwardprefix|:
%    \begin{macrocode}
\newcommand{\childdocredirect}[2][]
{
  \begingroup
    \if?#1?
      \def\childdoctmp{\childdocforward{#2}}
    \else
      \def\childdoctmp{\childdocforwardprefix{#1}{#2}}
    \fi
    \expandafter
  \endgroup
  \childdoctmp
}
%    \end{macrocode}

%\iffalse
%</package>
%\fi
%
\endinput

\childdocof{cdocsamp}
%    \end{macrocode}

%\iffalse
%</samplechap1|samplechap2>
%\fi
%
%\iffalse
%<*samplechap1>
%\fi
% Some text for chapter 1:
%    \begin{macrocode}
\section{one}
some text in chapter one
%    \end{macrocode}

%\iffalse
%</samplechap1>
%\fi
% Some text for chapter 2:
%\iffalse
%<*samplechap2>
%\fi
%    \begin{macrocode}
\section{two}
more text in chapter two
%    \end{macrocode}

%\iffalse
%</samplechap2>
%\fi
%
% %%%%%%%%%%%%%%%%%%%%%%%%%%%%%%%%%%%%%%
% \paragraph{Part Include Files.}
%
% The include files are called |cdocspt3.tex| and |cdocspt4.tex|.
%
%\iffalse
%<*samplepart3|samplepart4>
%\fi

% Optional override for |\version| flag:
%    \begin{macrocode}
%%\providecommand{\version}{final}
%    \end{macrocode}

% Include the main document:
%    \begin{macrocode}
% \iffalse
%
% childdoc.dtx Copyright (C) 2017-2018 Niklas Beisert
%
% This work may be distributed and/or modified under the
% conditions of the LaTeX Project Public License, either version 1.3
% of this license or (at your option) any later version.
% The latest version of this license is in
%   http://www.latex-project.org/lppl.txt
% and version 1.3 or later is part of all distributions of LaTeX
% version 2005/12/01 or later.
%
% This work has the LPPL maintenance status `maintained'.
%
% The Current Maintainer of this work is Niklas Beisert.
%
% This work consists of the files childdoc.dtx and childdoc.ins
% and the derived files childdoc.def and cdocsamp.tex with
% cdocsch1.tex, cdocsch2.tex, cdocsdrf.tex, cdocsfn1.tex, cdocsfn2.tex.
%
%<package>\ifdefined\childdocmain\endinput\fi
%<package>\ProvidesFile{childdoc.def}[2018/12/30 v2.0 child document driver]
%<samplemain>\ProvidesFile{cdocsamp.tex}[2018/12/30 v2.0 sample for childdoc]
%<*driver>
%\ProvidesFile{childdoc.drv}[2018/12/30 v2.0 childdoc reference manual file]
\PassOptionsToClass{10pt,a4paper}{article}
\documentclass{ltxdoc}

\usepackage[margin=35mm]{geometry}
\usepackage{hyperref}
\usepackage{hyperxmp}
\usepackage[usenames]{color}

\hypersetup{colorlinks=true}
\hypersetup{pdfstartview=FitH}
\hypersetup{pdfpagemode=UseNone}
\hypersetup{pdfsource={}}
\hypersetup{pdflang={en-UK}}
\hypersetup{pdfcopyright={Copyright 2017-2018 Niklas Beisert.
  This work may be distributed and/or modified under the
  conditions of the LaTeX Project Public License, either version 1.3
  of this license or (at your option) any later version.}}
\hypersetup{pdflicenseurl={http://www.latex-project.org/lppl.txt}}
\hypersetup{pdfcontactaddress={ETH Zurich, ITP, HIT K,
  Wolfgang-Pauli-Strasse 27}}
\hypersetup{pdfcontactpostcode={8093}}
\hypersetup{pdfcontactcity={Zurich}}
\hypersetup{pdfcontactcountry={Switzerland}}
\hypersetup{pdfcontactemail={nbeisert@itp.phys.ethz.ch}}
\hypersetup{pdfcontacturl={http://people.phys.ethz.ch/\xmptilde nbeisert/}}

\newcommand{\secref}[1]{\hyperref[#1]{section \ref*{#1}}}

\parskip1ex
\parindent0pt
\let\olditemize\itemize
\def\itemize{\olditemize\parskip0pt}

\begin{document}

\title{The \textsf{childdoc} Package}
\hypersetup{pdftitle={The childdoc Package}}
\author{Niklas Beisert\\[2ex]
  Institut f\"ur Theoretische Physik\\
  Eidgen\"ossische Technische Hochschule Z\"urich\\
  Wolfgang-Pauli-Strasse 27, 8093 Z\"urich, Switzerland\\[1ex]
  \href{mailto:nbeisert@itp.phys.ethz.ch}
  {\texttt{nbeisert@itp.phys.ethz.ch}}}
\hypersetup{pdfauthor={Niklas Beisert}}
\hypersetup{pdfsubject={Manual for the LaTeX2e Package childdoc}}
\date{30 December 2018, \textsf{v2.0}}
\maketitle

\begin{abstract}\noindent
\textsf{childdoc} is a \LaTeXe{} package
that enables the direct compilation
of document sections included by |\include|
to individual files.
\end{abstract}

\begingroup
\parskip0ex
\tableofcontents
\endgroup

%%%%%%%%%%%%%%%%%%%%%%%%%%%%%%%%%%%%%%%%%%%%%%%%%%%%%%%%%%%%%%%%%%%%%%%%%%%%%%%%
%%%%%%%%%%%%%%%%%%%%%%%%%%%%%%%%%%%%%%%%%%%%%%%%%%%%%%%%%%%%%%%%%%%%%%%%%%%%%%%%
\section{Introduction}

\LaTeX{} provides a mechanism to structure a large document (such as a book)
into a main file and several child files (containing the chapters)
using the |\include| command.
This mechanism is beneficial for documents
which span hundreds of pages in order to
make the source file(s) more manageable.
Moreover, compilation can be restricted to
selected child files by means of the |\includeonly| command.
The latter feature can be used to reduce the compilation time while editing
(this was significantly more useful in the earlier days of \LaTeX{})
or to generate a smaller document which is easier to navigate.
Another application of |\includeonly| is to generate
documents consisting of selected parts of the complete document.

However, there are a few drawbacks of the plain |\include| mechanism:
\begin{itemize}
\item
The child files cannot be compiled on their own,
they can only be compiled via the main file.
A naive editing environment
(such as a text editor with an option
to have the current file processed by \LaTeX)
may require one to switch to the main file before compiling;
attempting to compile the child file produces errors.
\item
The main file must be modified (each time)
to adjust the |\includeonly| command
to the present needs. This easily leaves the main file in a messy state.
\item
The generated document will always carry the filename
of the main document. This is inconvenient if
several child files are to be compiled and
to be kept for distribution.
\end{itemize}

The present package provides a simple interface
to make child files individually compilable by \LaTeX{}.
Compiling a child file then has the same effect as compiling
the main file with an |\includeonly| command
to select the appropriate child.
Moreover the generated document will carry the name of the child
rather than the main file.
This resolves all three above issues.

This feature is meant to make the editing of books,
thesis documents and lecture notes somewhat more convenient.
However, the package can also be used efficiently for
composing a series of documents (such as exercise sheets)
which are typically distributed individually.
It then assists the author in generating the individual documents
(potentially in different versions)
as well as a document containing the collected series.
Another application is in developing style files
or other kinds of included material
where compilation of the style file could redirect
to a sample or test file.

%%%%%%%%%%%%%%%%%%%%%%%%%%%%%%%%%%%%%%%%%%%%%%%%%%%%%%%%%%%%%%%%%%%%%%%%%%%%%%%%
%%%%%%%%%%%%%%%%%%%%%%%%%%%%%%%%%%%%%%%%%%%%%%%%%%%%%%%%%%%%%%%%%%%%%%%%%%%%%%%%
\section{Usage}

First of all, the package \textsf{childdoc} is \emph{not} a standard
\LaTeXe{} |.sty| style file! Therefore it needs to be invoked in
a non-standard way.

%%%%%%%%%%%%%%%%%%%%%%%%%%%%%%%%%%%%%%%%%%%%%%%%%%%%%%%%%%%%%%%%%%%%%%%%%%%%%%%%
\subsection{Included Files}
\label{sec:include}

%%%%%%%%%%%%%%%%%%%%%%%%%%%%%%%%%%%%%%%%
\DescribeMacro{\childdocmain}
To use the package, add the commands
\begin{center}
\begin{tabular}{l}
|\input{childdoc.def}|\\
|\childdocmain{}|\\
\end{tabular}
\end{center}
at the very top of the main \LaTeX{} file,
in particular \emph{before} the |\documentclass| statement!
The argument of |\childdocmain| should be left empty
(but it must be present).

%%%%%%%%%%%%%%%%%%%%%%%%%%%%%%%%%%%%%%%%
\DescribeMacro{\childdocof}
Furthermore, add the commands
\begin{center}
\begin{tabular}{l}
|\input{childdoc.def}|\\
|\childdocof{|\textit{main}|}|\\
\end{tabular}
\end{center}
at the top of every child file \textit{child}
which is included by |\include{|\textit{child}|}|
from within the main file
(or at least for those files to be compiled individually).
The argument \textit{main} must be the filename of the main file.

There are a couple of
considerations in setting up the main and child documents:

%%%%%%%%%%%%%%%%%%%%%%%%%%%%%%%%%%%%%%%%
\paragraph{Restrictions.}

Please note the following restrictions:
\begin{itemize}
\item
|\childdocmain| must be called with one argument \textit{main}
to ensure compatibility with earlier version of the package.
It must either be empty (|\childdocmain{}|)
or precisely match the filename of the main file in which it is specified.
See \secref{sec:detection} for further information.
\item
The filename \textit{main} must be specified without the |.tex| extension.
\item
The filename \textit{main} is case sensitive
(even in case-insensitive file systems)
due to internal string comparison.
\item
The argument \textit{main} should be fully expanded, it cannot be a macro.
\item
Subdirectories and special characters should be avoided in filenames.
\item
The command |\childdocmain{|\textit{main}|}| must be followed by a whitespace.
It should not be followed immediately by another command
or by a comment mark `|%|'.
This is because the \TeX{} parser reads the token immediately following
the argument of |\childdocmain| and puts it
at the beginning of every child section;
however, a white\-space is ignored.
\end{itemize}

%%%%%%%%%%%%%%%%%%%%%%%%%%%%%%%%%%%%%%%%
\paragraph{Content of Main File.}

It is advisable to place all content in the child files included by |\include|.
Any output contained in the main file will appear in all child documents
unless suppressed manually;
it cannot be suppressed automatically by the |\includeonly| directive
and thus should normally be avoided.
A method to include some content in the main file
by means of conditional processing is described in \secref{sec:conditional}.

%%%%%%%%%%%%%%%%%%%%%%%%%%%%%%%%%%%%%%%%
\paragraph{Page Numbering.}

When only a part of the document is compiled,
the appropriate numbering of pages
(as well as other status parameters)
is determined from the |.aux| files.
The latter contain information from previous passes.
However this information needs to propagate through
all intermediate child documents.
Therefore the page numbering in child documents may well
be inconsistent until the complete document is compiled at least once.

A useful (if unconventional) way to always ensure a consistent
page numbering is to restart the numbering in each child document
and denote the pages by `\textit{child}|.|\textit{page}'
where \textit{child} represents the chapter/section number of the child file.
This can be achieved by the command
|\numberwithin{page}{|\textit{child}|}|
of the \textsf{amsmath} package
where \textit{child} can be |chapter| or |section|
depending on the chosen structuring.
Alternatively, one can modify the macro |\thepage| appropriately
and reset the counter |page| at the start of each child file.

%%%%%%%%%%%%%%%%%%%%%%%%%%%%%%%%%%%%%%%%%%%%%%%%%%%%%%%%%%%%%%%%%%%%%%%%%%%%%%%%
\subsection{Conditional Processing}
\label{sec:conditional}

The package provides a mechanism to compile different versions
of a document. To customise the versions further some conditional processing
can come in handy to distinguish which version is being compiled.
The package provides two macros to describe the compilation context:

%%%%%%%%%%%%%%%%%%%%%%%%%%%%%%%%%%%%%%%%
\DescribeMacro{\ifchilddoc}
The conditional |\ifchilddoc| distinguishes between the compilation of
child documents and the main document:
%
\begin{center}
|\ifchilddoc |\textit{child-code}| |[|\||else |\textit{main-code}]| \||fi|
\end{center}

%%%%%%%%%%%%%%%%%%%%%%%%%%%%%%%%%%%%%%%%
\DescribeMacro{\childdocname}
\DescribeMacro{\childdocjob}
The macro |\childdocname| contains the filename (without extension)
of the main or child file being processed.
Note that |\childdocjob| will always contain the name of the main file.

%%%%%%%%%%%%%%%%%%%%%%%%%%%%%%%%%%%%%%%%
\paragraph{Title Page.}

Conditional processing can be used to include a title or banner page
in the main document when proper precautions are taken.
Importantly, the code in the main file should ensure that the page counter
(as well as other status parameters which are stored in the |.aux| files)
takes the same value after the conditional processing.
Otherwise the page numbers may take divergent values
depending on which part is compiled.

For example, a title page could be declared by:
%
\begin{center}
\begin{tabular}{l}
|\ifchilddoc\||else|\\
|\addtocounter{page}{-1}|\\
\textit{code for title page}\\
|\newpage|\\
|\||fi|
\end{tabular}
\end{center}
%
A banner page for the child documents can be generated by:
%
\begin{center}
\begin{tabular}{l}
|\ifchilddoc|\\
|\addtocounter{page}{-1}|\\
\textit{code for banner page}\\
|\newpage|\\
|\||fi|
\end{tabular}
\end{center}
%
Here one could write a message such as:
\begin{center}
|This is the part \childdocname{} of \childdocjob{}.|
\end{center}

%%%%%%%%%%%%%%%%%%%%%%%%%%%%%%%%%%%%%%%%%%%%%%%%%%%%%%%%%%%%%%%%%%%%%%%%%%%%%%%%
\subsection{Flags}
\label{sec:flags}

The package makes it easy to generate different versions
of the main or child documents.
To this end compilation flags can be defined
and assigned different default values.
They will be particularly useful in conjunction
with the forwarding mechanism described in \secref{sec:forward}.

For example, it may be useful to have a flag |\version|
which can be set to |draft| or |final|.
The document source will contain some conditional code
depending on the value of |\version|.
Suppose further, the flag should default to |final| for the main file
and to |draft| for child files
which is a natural assignment for editing the document.
This is achieved by placing the following code
in the preamble of the main document
(below the |\childdocmain| directive):
%
\begin{center}
\begin{tabular}{l}
|\ifchilddoc|\\
|\providecommand{\version}{draft}|\\
|\||else|\\
|\providecommand{\version}{final}|\\
|\||fi|
\end{tabular}
\end{center}
%
The definition by |\providecommand| makes sure
that previous definitions are not overwritten.
Further statements |\providecommand{\version}{...}|
can thus be added before the above code to override it.

For the main file, one might add a line
(between |\childdocmain| and the above block)
%
\begin{center}
|%\ifchilddoc\||else\providecommand{\version}{draft}\||fi|
\end{center}
%
which can be uncommented to produce a draft version.
Likewise one can add a line to the very top of a child file
(above the |\childdocof{|\textit{main}|}| directive)
%
\begin{center}
|%\providecommand{\version}{final}|
\end{center}
%
which can be uncommented to produce the final version of this child document.

%%%%%%%%%%%%%%%%%%%%%%%%%%%%%%%%%%%%%%%%%%%%%%%%%%%%%%%%%%%%%%%%%%%%%%%%%%%%%%%%
\subsection{Forwarding}
\label{sec:forward}

Different versions of the main or child documents
using compilation flags as described in \secref{sec:flags}
can be (permanently) stored in different files
for convenient compilation, viewing and distribution.
To this end, the package defines a command
to pass on compilation to a different file:

%%%%%%%%%%%%%%%%%%%%%%%%%%%%%%%%%%%%%%%%
\DescribeMacro{\childdocforward}
The command |\childdocforward| redirects processing to
another source file:
%
\begin{center}
\begin{tabular}{l}
|\input{childdoc.def}|\\
|\childdocforward[|\textit{main}|]{|\textit{dest}|}|\\
\end{tabular}
\end{center}
%
The argument \textit{dest} is the destination file
(without extension).
It should be the main file or one of the child files.
Note that further \textsf{childdoc} directives
such as |\childdocof| and |\childdocforward|
in the indicated file will be processed in this form.
The optional argument \textit{main}
passes on directly to the main file \textit{main}
while pretending to compile the child \textit{dest}.
This form behaves as if \textit{dest}
issues |\childdocof{|\textit{main}|}| right away,
and no further \textsf{childdoc} directives will be processed.

%%%%%%%%%%%%%%%%%%%%%%%%%%%%%%%%%%%%%%%%
\DescribeMacro{\...prefix}
In the alternative form |\childdocforwardprefix|,
%
\begin{center}
\begin{tabular}{l}
|\input{childdoc.def}|\\
|\childdocforwardprefix[|\textit{main}|]{|\textit{prefix}|}{|\textit{dest}|}|
\end{tabular}
\end{center}
%
the destination file is determined by a pattern
depending on the current file:
To make this work, the current file must be called
`{\textit{prefix}\hspace{0.2em}\textit{suffix}}'
with \textit{prefix} matching precisely the argument.
Processing is then passed on to the file
`{\textit{dest}\hspace{0.2em}\textit{suffix}}'.
Surely, the same effect is achieved by
directly specifying the
argument `{\textit{dest}\hspace{0.2em}\textit{suffix}}'
in the first form.
However, that requires to set up a different file
for each child. With the alternative form of the command
all these files can have exactly the same content
which simplifies setting them up and maintaining them.

For example, the following file |draft.tex|
with a compilation flag |\version| as described in \secref{sec:flags}
compiles the main document as a draft:
%
\begin{center}
\begin{tabular}{l}
|\def\version{draft}|\\
|\input{childdoc.def}|\\
|\childdocforward{|\textit{main}|}|
\end{tabular}
\end{center}
%
Likewise, the following files |final|\textit{nn}|.tex|
compile the final version of the child document
|child|\textit{nn}|.tex|:
%
\begin{center}
\begin{tabular}{l}
|\def\version{final}|\\
|\input{childdoc.def}|\\
|\childdocforwardprefix{final}{child}|
\end{tabular}
\end{center}
%

Note that when several versions of a main file and/or of each child file
are to be generated, it may be convenient to set up a |Makefile| or
shell script to automatise the process.

%%%%%%%%%%%%%%%%%%%%%%%%%%%%%%%%%%%%%%%%%%%%%%%%%%%%%%%%%%%%%%%%%%%%%%%%%%%%%%%%
\subsection{Command Line Processing}
\label{sec:commandline}

The effect of redirection files can also be achieved by invoking
the \LaTeX{} compiler with a more elaborate command line.
Most conveniently this should be done as part
of a shell script or a |Makefile|.

When using \textsf{childdoc} in the main file, the following
command lines effectively perform a redirection
(note that depending on the shell being used,
backslashes may have to be doubled: `|\|' $\to$ `|\\|'):
%
\begin{center}
|... -jobname "|\textit{target}|" |\\|"|[\textit{flags}]%
|\input{childdoc.def}\childdocforward[|\textit{main}|]{|\textit{dest}|}"|
\end{center}
%
Here \textit{target} is the name of the output file,
\textit{main} is the name of the main file
and \textit{dest} is the name of the main or child file to be processed
(all filenames without extensions).
The optional argument \textit{main} can be omitted
if \textit{main} matches \textit{dest}.
Optionally, compilation \textit{flags} can be defined via |\def| commands.
This command line makes the \TeX{} engine believe
it is compiling the file \textit{target}
whose content is specified as the latter parameter.
The provided code then forwards the processing to
\textit{main} or \textit{dest} as described in \secref{sec:forward}.

%%%%%%%%%%%%%%%%%%%%%%%%%%%%%%%%%%%%%%%%%%%%%%%%%%%%%%%%%%%%%%%%%%%%%%%%%%%%%%%%
\subsection{Include by Input}
\label{sec:input}

Including child documents by |\include| has some restrictions by design.
Most notably, the content of a child document always occupies
its own set of pages; pages cannot be shared between child documents.
Usually, this behaviour makes perfect sense
because each child document contain an essential part of the document.
However, in some situations it may be desirable to compose
a document from a collection of parts
without having mandatory page breaks between then.
For this case, the package
provides a mechanism to include parts
by |\input| which can also be processed individually.
However, by construction this mechanism
requires manual handling of the content to be output.

%%%%%%%%%%%%%%%%%%%%%%%%%%%%%%%%%%%%%%%%
\DescribeMacro{\ifchilddocmanual}
The main file should be prepared as usual, see \secref{sec:include}.
However, the document body must make a distinction
between processing of an individual part and of the main document, e.g.:
%
\begin{center}
\begin{tabular}{l}
|\ifchilddocmanual|\\
|\input{\childdocname}|\\
|\||else|\\
\textit{document body with }|\input{|\textit{part}|}|\\
|\||fi|
\end{tabular}
\end{center}
%
The conditional |\ifchilddocmanual| is true whenever
a part to be included by |\input| is being compiled,
and the name of the part is stored in |\childdocname|.

%%%%%%%%%%%%%%%%%%%%%%%%%%%%%%%%%%%%%%%%
\DescribeMacro{\childdocby}
Each part to be included by |\input| should start with:
%
\begin{center}
\begin{tabular}{l}
|\input{childdoc.def}|\\
|\childdocby{|\textit{main}|}|\\
\end{tabular}
\end{center}
%
The directive |\childdocby| is similar to |\childdocof|
described in \secref{sec:include},
but the subsequent selection of content must be done manually.
To that end, both |\ifchilddoc| and |\ifchilddocmanual|
will be true upon processing of a part,
and the name of the part is stored in |\childdocname|.
Note that |\jobname| will be set to the filename of the current part
so that each part receives an individual |.aux| file
that does not interfere with the |.aux| file(s) of the main document.
This behaviour can be altered by the alternative form
|\childdocby[*]{|\textit{main}|}| (with a non-empty optional argument)
which uses the |.aux| file of the main document
by setting |\jobname| to \textit{main}.

%%%%%%%%%%%%%%%%%%%%%%%%%%%%%%%%%%%%%%%%%%%%%%%%%%%%%%%%%%%%%%%%%%%%%%%%%%%%%%%%
\subsection{Driver Development}
\label{sec:driver}

The \textsf{childdoc} mechanism can also be use for the development
of definition files such as \LaTeX{} styles or classes.
This case differs from the above setup with multiple parts
included by |\include| in that no |\includeonly| should be invoked.
This can be achieved by starting the include file
(before |\ProvidesPackage|) with:
%
\begin{center}
\begin{tabular}{l}
|\input{childdoc.def}|\\
|\childdocforward{|\textit{main}|}|\\
\end{tabular}
\end{center}
%
or alternatively with:
%
\begin{center}
\begin{tabular}{l}
|\input{childdoc.def}|\\
|\childdocby{|\textit{main}|}|\\
\end{tabular}
\end{center}
%
Both forms have slightly different effects as described above.
The main file is prepared as usual, see \secref{sec:include}.

%%%%%%%%%%%%%%%%%%%%%%%%%%%%%%%%%%%%%%%%%%%%%%%%%%%%%%%%%%%%%%%%%%%%%%%%%%%%%%%%
\subsection{Legacy Detection}
\label{sec:detection}

The directive |\childdocmain| in the main file can detect
whether the complete document or merely a child is to be compiled
even without using the directive |\childdocof|.
This method is deprecated because it is less robust
and there is no compelling reason to use it;
it is merely provided for backward compatibility
and it may be removed in future versions.

If the detection mechanism is to be used,
it is mandatory to correctly specify
the filename of the main file as the argument of |\childdocmain|:
%
\begin{center}
\begin{tabular}{l}
|\input{childdoc.def}|\\
|\childdocmain{|\textit{main}|}|\\
\end{tabular}
\end{center}
%
If |\jobname| does not match the argument \textit{main} of |\childdocmain|,
it is assumed that |\jobname| points to the child file to be compiled.
When using |\childdocmain| with the main file specified as argument,
it suffices to start a child file
with just |\input{|\textit{main}|}|
without loading of the package and using |\childdocof|.
If instead all processing is done
with the appropriate \textsf{childdoc} directives,
the argument of \textit{main} of |\childdocmain| can be empty.

An alternative version of the command line processing described
in \secref{sec:commandline} using the detection mechanism reads:
%
\begin{center}
|... -jobname "|\textit{target}|" "|[\textit{flags}]%
[|\def\jobname{|\textit{dest}|}|]|\input{|\textit{main}|}"|
\end{center}

%%%%%%%%%%%%%%%%%%%%%%%%%%%%%%%%%%%%%%%%%%%%%%%%%%%%%%%%%%%%%%%%%%%%%%%%%%%%%%%%
\subsection{Manual Code}
\label{sec:manual}

In case one cannot be certain whether the definitions file |childdoc.def|
is installed on the target \TeX{} distribution
and one prefers not to ship it,
it is conceivable to paste a few relevant commands into the sources.

To that end, drop all statements |\input{childdoc.def}|
and perform the replacements as outlined below.
Instead of |\childdocmain{|\textit{main}|}| add the following code
to the top of the main file:
%
\begin{center}
\begin{tabular}{l}
|\||ifdefined\childdocname\endinput\||fi\newif\ifchilddoc|\\
|\edef\childdocname{\scantokens\expandafter{\jobname\noexpand}}|\\
|\def\childdocmain{|\textit{main}|}\||ifx\childdocmain\childdocname\||else|\\
|\childdoctrue\includeonly{\childdocname}\let\jobname\childdocmain\||fi|\\
\end{tabular}
\end{center}
%
Instead of |\childdocof{|\textit{main}|}| just include the main file
at the top of each child file:
%
\begin{center}
|\input{|\textit{main}|}|
\end{center}
%
A simple redirection |\childdocforward{|\textit{dest}|}| is achieved by:
%
\begin{center}
|\def\jobname{|\textit{dest}|}\input{\jobname}|
\end{center}
%
The redirection with prefix
|\childdocforwardprefix[|\textit{prefix}|]{|\textit{dest}|}|
is accomplished by:
%
\begin{center}
\begin{tabular}{l}
|{\edef\jobname{\scantokens\expandafter{\jobname\noexpand}}|\\
|\def\redirectjob |\textit{prefix}|#1~~~{\gdef\jobname{|\textit{dest}|#1}}|\\
|\expandafter\redirectjob\jobname~~~}\input{\jobname}|
\end{tabular}
\end{center}

In an alternative approach,
child documents can be compiled by a specific command line
without additional code or specific definitions:
%
\begin{center}
|... -jobname "|\textit{target}|" "|[\textit{flags}]%
|\includeonly{|\textit{dest}|}\input{|\textit{main}|}"|
\end{center}
%

%%%%%%%%%%%%%%%%%%%%%%%%%%%%%%%%%%%%%%%%%%%%%%%%%%%%%%%%%%%%%%%%%%%%%%%%%%%%%%%%
%%%%%%%%%%%%%%%%%%%%%%%%%%%%%%%%%%%%%%%%%%%%%%%%%%%%%%%%%%%%%%%%%%%%%%%%%%%%%%%%
\section{Information}

%%%%%%%%%%%%%%%%%%%%%%%%%%%%%%%%%%%%%%%%%%%%%%%%%%%%%%%%%%%%%%%%%%%%%%%%%%%%%%%%
\subsection{Copyright}

Copyright \copyright{} 2017--2018 Niklas Beisert

This work may be distributed and/or modified under the
conditions of the \LaTeX{} Project Public License, either version 1.3
of this license or (at your option) any later version.
The latest version of this license is in
  \url{http://www.latex-project.org/lppl.txt}
and version 1.3 or later is part of all distributions of \LaTeX{}
version 2005/12/01 or later.

This work has the LPPL maintenance status `maintained'.

The Current Maintainer of this work is Niklas Beisert.

This work consists of the files |README.txt|, |childdoc.ins| and |childdoc.dtx|
as well as the derived files |childdoc.def|, |cdocsamp.tex|
with |cdocsch1.tex|, |cdocsch2.tex|, |cdocspt3.tex|, |cdocspt4.tex|,
|cdocsdrf.tex|, |cdocsfn1.tex|, |cdocsfn2.tex|
as well as |childdoc.pdf|.

%%%%%%%%%%%%%%%%%%%%%%%%%%%%%%%%%%%%%%%%%%%%%%%%%%%%%%%%%%%%%%%%%%%%%%%%%%%%%%%%
\subsection{Files and Installation}

The package consists of the files:
%
\begin{center}
\begin{tabular}{ll}
    |README.txt|   & readme file \\
    |childdoc.ins| & installation file \\
    |childdoc.dtx| & source file \\
    |childdoc.def| & definition file \\
    |cdocsamp.tex| & sample main file \\
    |cdocsch1.tex| & sample include file \\
    |cdocsch2.tex| & sample include file \\
    |cdocspt3.tex| & sample part file \\
    |cdocspt4.tex| & sample part file \\
    |cdocsdrf.tex| & sample redirection file \\
    |cdocsfn1.tex| & sample redirection file \\
    |cdocsfn2.tex| & sample redirection file \\
    |childdoc.pdf| & manual
\end{tabular}
\end{center}
%
The distribution consists of the files
|README.txt|, |childdoc.ins| and |childdoc.dtx|.
%
\begin{itemize}
\item
Run (pdf)\LaTeX{} on |childdoc.dtx|
to compile the manual |childdoc.pdf| (this file).
\item
Run \LaTeX{} on |childdoc.ins| to create the definitions file |childdoc.def|
and the sample |cdocsamp.tex| with include files
|cdocsch1.tex|, |cdocsch2.tex|, |cdocspt3.tex|, |cdocspt4.tex|,
|cdocsdrf.tex|, |cdocsfn1.tex|, |cdocsfn2.tex|.
Then copy the file |childdoc.def| to an appropriate directory of your \LaTeX{}
distribution, e.g.\ \textit{texmf-root}|/tex/latex/childdoc|.
\end{itemize}

%%%%%%%%%%%%%%%%%%%%%%%%%%%%%%%%%%%%%%%%%%%%%%%%%%%%%%%%%%%%%%%%%%%%%%%%%%%%%%%%
\subsection{Related CTAN Packages}

There are several other packages which offer a similar functionality:
%
\begin{itemize}
\item
The packages
\href{http://ctan.org/pkg/docmute}{\textsf{docmute}},
\href{http://ctan.org/pkg/includex}{\textsf{includex}} and
\href{http://ctan.org/pkg/standalone}{\textsf{standalone}}
provide commands to include only the document body of
a child file thus allowing both files to be compiled individually.
\item
The packages \href{http://ctan.org/pkg/subdocs}{\textsf{subdocs}}
and \href{http://ctan.org/pkg/subfiles}{\textsf{subfiles}}
provide structures in which the main and child documents can be
encapsulated and allowing them to be compiled individually.
The inclusion mechanism is different from the conventional |\include|.
\item
The package \href{http://ctan.org/pkg/combine}{\textsf{combine}}
is an elaborate solution to combine several documents into one.
\end{itemize}
%
See also the CTAN topic \href{http://ctan.org/topic/subdocs}{\textsf{subdocs}}
for further related packages.
The present package differs from the above solutions in that
a document structure constructed with the conventional |\include| mechanism
just needs two extra commands at the top of every file
such that all constituent files can be compiled individually.

%%%%%%%%%%%%%%%%%%%%%%%%%%%%%%%%%%%%%%%%%%%%%%%%%%%%%%%%%%%%%%%%%%%%%%%%%%%%%%%%
%\subsection{Feature Suggestions}
%
%The following is a list of features which may be useful for future
%versions of this package:
%%
%\begin{itemize}
%\item
%\ldots
%\end{itemize}

%%%%%%%%%%%%%%%%%%%%%%%%%%%%%%%%%%%%%%%%%%%%%%%%%%%%%%%%%%%%%%%%%%%%%%%%%%%%%%%%
\subsection{Revision History}

%%%%%%%%%%%%%%%%%%%%%%%%%%%%%%%%%%%%%%%%
\paragraph{v2.0:} 2018/12/30

\begin{itemize}
\item
immediate forward processing
\item
added |\childdocby| mechanism
\item
manual restructured
\end{itemize}

%%%%%%%%%%%%%%%%%%%%%%%%%%%%%%%%%%%%%%%%
\paragraph{v1.6:} 2018/01/17

\begin{itemize}
\item
application for development of include files
\item
corrections to manual
\end{itemize}

%%%%%%%%%%%%%%%%%%%%%%%%%%%%%%%%%%%%%%%%
\paragraph{v1.5:} 2017/05/21

\begin{itemize}
\item
more complete structuring introduced
\item
|\childdocof| introduced
\item
|\childdoc| renamed to |\childdocmain|
\item
|\childredirect| renamed to |\childdocforward| and |\childdocforwardprefix|
and functionality expanded
\end{itemize}

%%%%%%%%%%%%%%%%%%%%%%%%%%%%%%%%%%%%%%%%
\paragraph{v1.0:} 2017/04/27

\begin{itemize}
\item
manual and install package
\item
first version published on CTAN
\end{itemize}

%%%%%%%%%%%%%%%%%%%%%%%%%%%%%%%%%%%%%%%%
\paragraph{v0.6:} 2017/04/26

\begin{itemize}
\item
redirection mechanism added
\end{itemize}

%%%%%%%%%%%%%%%%%%%%%%%%%%%%%%%%%%%%%%%%
\paragraph{v0.5:} 2017/04/26

\begin{itemize}
\item
functionality in definition file
\end{itemize}


%%%%%%%%%%%%%%%%%%%%%%%%%%%%%%%%%%%%%%%%%%%%%%%%%%%%%%%%%%%%%%%%%%%%%%%%%%%%%%%%
%%%%%%%%%%%%%%%%%%%%%%%%%%%%%%%%%%%%%%%%%%%%%%%%%%%%%%%%%%%%%%%%%%%%%%%%%%%%%%%%
%%%%%%%%%%%%%%%%%%%%%%%%%%%%%%%%%%%%%%%%%%%%%%%%%%%%%%%%%%%%%%%%%%%%%%%%%%%%%%%%
\appendix

\settowidth\MacroIndent{\rmfamily\scriptsize 000\ }

 \DocInput{childdoc.dtx}

\end{document}
%</driver>
% \fi
%
% %%%%%%%%%%%%%%%%%%%%%%%%%%%%%%%%%%%%%%%%%%%%%%%%%%%%%%%%%%%%%%%%%%%%%%%%%%%%%%
% %%%%%%%%%%%%%%%%%%%%%%%%%%%%%%%%%%%%%%%%%%%%%%%%%%%%%%%%%%%%%%%%%%%%%%%%%%%%%%
% \section{Sample}
%\iffalse
%<*samplemain>
%\fi
%
% The following presents a sample document
% with two chapters, two parts, a title page,
% a compile flag as well as three forwarding files to set the flag.
% It consists of eight |.tex| files:
% \begin{center}
% \begin{tabular}{ll}
% |cdocsamp.tex|&main file\\
% |cdocsch1.tex|&include file for chapter 1\\
% |cdocsch2.tex|&include file for chapter 2\\
% |cdocspt3.tex|&include file for part 3\\
% |cdocspt4.tex|&include file for part 4\\
% |cdocsdrf.tex|&forwarding file for main file in draft mode\\
% |cdocsfi1.tex|&forwarding file for final version of chapter 1\\
% |cdocsfi2.tex|&forwarding file for final version of chapter 2\\
% \end{tabular}
% \end{center}
% Each of the eight files can be compiled directly by the \LaTeX{} compiler.
%
% %%%%%%%%%%%%%%%%%%%%%%%%%%%%%%%%%%%%%%
% \paragraph{Main File.}
%
% The main file is called |cdocsamp.tex|.
%
% Load the \textsf{childdoc} definitions and
% declare the filename for the main document:
%    \begin{macrocode}
\input{childdoc.def}
\childdocmain{}
%    \end{macrocode}

% Optional override for |\version| flag:
%    \begin{macrocode}
%%\ifchilddoc\else\providecommand{\version}{draft}\fi
%    \end{macrocode}

% Define the default values for the |\version| flag
% (|final| for the main file and |draft| for childs):
%    \begin{macrocode}
\ifchilddoc
\providecommand{\version}{draft}
\else
\providecommand{\version}{final}
\fi
%    \end{macrocode}

% Load the standard document class:
%    \begin{macrocode}
\documentclass[12pt]{article}
%    \end{macrocode}

% Start the document body:
%    \begin{macrocode}
\begin{document}
%    \end{macrocode}

% Declare a title page.
% Print title, part of document being processed and version flag:
%    \begin{macrocode}
\addtocounter{page}{-1}
\begin{center}
{\LARGE\bfseries{}childdoc example\par}
\vspace{1cm}
\ifchilddoc
\ifchilddocmanual part\else chapter\fi:
`\childdocname' of `\childdocjob'\par
\else
main document: `\childdocjob'\par
\fi
version: \version\par
\end{center}
\newpage
%    \end{macrocode}

% Manually include selected file,
% otherwise process as usual:
%    \begin{macrocode}
\ifchilddocmanual
\section*{part `\childdocname'}
\input{\childdocname}
\else
%    \end{macrocode}

% Include the two chapters:
%    \begin{macrocode}
\include{cdocsch1}
\include{cdocsch2}
%    \end{macrocode}

% Include the two parts unless only chapters should be displayed:
%    \begin{macrocode}
\ifchilddoc\else
\section{part three}
\input{cdocspt3}
\section{part four}
\input{cdocspt4}
\fi
%    \end{macrocode}

% Process as usual until here:
%    \begin{macrocode}
\fi
%    \end{macrocode}

% End of document body:
%    \begin{macrocode}
\end{document}
%    \end{macrocode}
%\iffalse
%</samplemain>
%\fi
%
% %%%%%%%%%%%%%%%%%%%%%%%%%%%%%%%%%%%%%%
% \paragraph{Chapter Include Files.}
%
% The include files are called |cdocsch1.tex| and |cdocsch2.tex|.
%
%\iffalse
%<*samplechap1|samplechap2>
%\fi

% Optional override for |\version| flag:
%    \begin{macrocode}
%%\providecommand{\version}{final}
%    \end{macrocode}

% Include the main document:
%    \begin{macrocode}
\input{childdoc.def}
\childdocof{cdocsamp}
%    \end{macrocode}

%\iffalse
%</samplechap1|samplechap2>
%\fi
%
%\iffalse
%<*samplechap1>
%\fi
% Some text for chapter 1:
%    \begin{macrocode}
\section{one}
some text in chapter one
%    \end{macrocode}

%\iffalse
%</samplechap1>
%\fi
% Some text for chapter 2:
%\iffalse
%<*samplechap2>
%\fi
%    \begin{macrocode}
\section{two}
more text in chapter two
%    \end{macrocode}

%\iffalse
%</samplechap2>
%\fi
%
% %%%%%%%%%%%%%%%%%%%%%%%%%%%%%%%%%%%%%%
% \paragraph{Part Include Files.}
%
% The include files are called |cdocspt3.tex| and |cdocspt4.tex|.
%
%\iffalse
%<*samplepart3|samplepart4>
%\fi

% Optional override for |\version| flag:
%    \begin{macrocode}
%%\providecommand{\version}{final}
%    \end{macrocode}

% Include the main document:
%    \begin{macrocode}
\input{childdoc.def}
\childdocby{cdocsamp}
%    \end{macrocode}

%\iffalse
%</samplepart3|samplepart4>
%\fi
%
%\iffalse
%<*samplepart3>
%\fi
% Some text for part 3:
%    \begin{macrocode}
some text in part three
%    \end{macrocode}

%\iffalse
%</samplepart3>
%\fi
% Some text for part 4:
%\iffalse
%<*samplepart4>
%\fi
%    \begin{macrocode}
more text in part four
%    \end{macrocode}

%\iffalse
%</samplepart4>
%\fi
%
% %%%%%%%%%%%%%%%%%%%%%%%%%%%%%%%%%%%%%%
% \paragraph{Forwarding for a Complete Draft.}
%
% The following forwarding file |cdocsdrf.tex|
% compiles the main document in draft mode:
%\iffalse
%<*sampledraft>
%\fi
%    \begin{macrocode}
\def\version{draft}
\input{childdoc.def}
\childdocforward{cdocsamp}
%    \end{macrocode}

%\iffalse
%</sampledraft>
%\fi
%
% %%%%%%%%%%%%%%%%%%%%%%%%%%%%%%%%%%%%%%
% \paragraph{Forwarding for Final Version of the Chapters.}
%
% The following forwarding files |cdocsfn1.tex| and |cdocsfn2.tex|
% (with identical content)
% compile the final versions of the child documents
% |cdocsch1.tex| and |cdocsch2.tex|, respectively:
%\iffalse
%<*samplefinal>
%\fi
%    \begin{macrocode}
\def\version{final}
\input{childdoc.def}
\childdocforwardprefix[cdocsamp]{cdocsfn}{cdocsch}
%    \end{macrocode}

%\iffalse
%</samplefinal>
%\fi
%
% %%%%%%%%%%%%%%%%%%%%%%%%%%%%%%%%%%%%%%
% \paragraph{Command Line Processing.}
%
% The following three command lines generate the output files
% |cdocscld|, |cdocscl1| and |cdocscl2|
% which should be identical to
% |cdocsdrf|, |cdocsch1| and |cdocsfn2|, respectively:
% \begin{center}
% \begin{tabular}{l}
% |latex -jobname cdocscld \|\\
% |  "\def\version{draft}\input{childdoc.def}\childdocforward{cdocsamp}"|\\
% |latex -jobname cdocscl1 \|\\
% |  "\input{childdoc.def}\childdocforward[cdocsamp]{cdocsch1}"|\\
% |latex -jobname cdocscl2 \|\\
% |  "\def\version{final}\input{childdoc.def}\childdocforward{cdocsch2}"|
% \end{tabular}
% \end{center}
% Note that the trailing backslash on each first line
% merely continues the input to the second line
% (for convenient cut ant paste).
% Furthermore, the command |latex| can be replaced by any
% of its alternative versions such as |pdflatex|.
%
% %%%%%%%%%%%%%%%%%%%%%%%%%%%%%%%%%%%%%%%%%%%%%%%%%%%%%%%%%%%%%%%%%%%%%%%%%%%%%%
% %%%%%%%%%%%%%%%%%%%%%%%%%%%%%%%%%%%%%%%%%%%%%%%%%%%%%%%%%%%%%%%%%%%%%%%%%%%%%%
% \section{Implementation}
%\iffalse
%<*package>
%\fi
%
% This section describes the definitions file |childdoc.def|.

% The definitions cannot be loaded using |\usepackage| or |\RequirePackage|
% which has a mechanism to prevent loading a style file more than once.
% When loading the definitions by means of |\input|
% multiple instances have to be prevented manually:
%\iffalse
%This code needs to be before the `\ProvidesFile' directive
%which is defined at the beginning of this file.
%Therefore it is also placed there and commented out here.
%</package>
%<*discard>
%\fi
%    \begin{macrocode}
\ifdefined\childdocmain\endinput\fi
%    \end{macrocode}
%\iffalse
%</discard>
%<*package>
%\fi
%
% \macro{\ifchilddoc}
% \macro{\ifchilddocmanual}
% The conditional |\ifchilddoc| tells whether a
% child (true) or main (false) document is being compiled.
% The conditional |\ifchilddocmanual| tells whether
% the |\includeonly| mechanism is used (false) or
% the selection of child files must be performed manually (true).
% The definitions initialise to false:
%    \begin{macrocode}
\newif\ifchilddoc
\newif\ifchilddocmanual
%    \end{macrocode}

% \macro{\childdocname}
% \macro{\childdocjob}
% The macro |\childdocname| stores the name of the main document
% to be compiled. The macro |\childdocjob| stores the name of
% the document on which the \LaTeX{} compiler was originally invoked.
% The content of |\jobname| cannot be compared
% to filenames specified in the source due to different catcodes.
% The following code rescans |\jobname|, stores the result
% in |\childdocname| and saves a copy in |\childdocjob|:
%    \begin{macrocode}
\edef\childdocname{\scantokens\expandafter{\jobname\noexpand}}
\let\childdocjob\childdocname
%    \end{macrocode}

% \macro{\childdocdisable}
% The macro |\childdocdisable| prevents the main file
% from being processed more than once.
% At this stage, the main document command |\childdocmain|
% is assumed to be called once again where it should do nothing.
% Any subsequent call to it should prevent
% a secondary processing of the main document
% It overwrites the forwarding commands
% |\childdocof| and |\childdocforward|
% with empty macros to prevent further inclusions of the main document:
%    \begin{macrocode}
\newcommand{\childdocdisable}
{
  \renewcommand{\childdocmain}[1]{\renewcommand{\childdocmain}[1]{\endinput}}
  \renewcommand{\childdocof}[1]{}
  \renewcommand{\childdocby}[2][]{}
  \renewcommand{\childdocforward}[2][]{}
  \renewcommand{\childdocdisable}{}
}
%    \end{macrocode}

% \macro{\childdocmain}
% The macro |\childdocmain| is to be called at the top of the main file
% with nothing or the main filename (without extension) as argument.
% First, it breaks loops.
% If the argument is not empty and does not match |\childdocname|
% (which is set by the first inclusion of |childdoc.def|),
% |\ifchilddoc| is set to true, |\includeonly| is applied to the child file
% and |\jobname| is set to the main file
% (for proper handling of |.aux| files):
%    \begin{macrocode}
\newcommand{\childdocmain}[1]
{
  \childdocdisable\childdocmain{}
  \if?#1?\else
    \begingroup
      \def\childdoctmp{#1}
      \ifx\childdoctmp\childdocname
        \def\childdoctmp{}
      \else
        \def\childdoctmp
        {
          \childdoctrue
          \includeonly{\childdocname}
          \def\childdocjob{#1}
          \def\jobname{#1}
        }
      \fi
      \expandafter
    \endgroup
    \childdoctmp
  \fi
}
%    \end{macrocode}

% \macro{\childdocof}
% The command |\childdocof| redirects
% compilation to the main file |#1|.
%    \begin{macrocode}
\newcommand{\childdocof}[1]
{
  \childdocdisable
  \childdoctrue
  \includeonly{\childdocname}
  \def\jobname{#1}
  \def\childdocjob{#1}
  \input{#1}
}
%    \end{macrocode}

% \macro{\childdocby}
% The command |\childdocby| ....
%    \begin{macrocode}
\newcommand{\childdocby}[2][]
{
  \childdocdisable
  \childdoctrue
  \childdocmanualtrue
  \if?#1?\else
    \def\jobname{#2}
  \fi
  \def\childdocjob{#2}
  \input{#2}
  \endinput
}
%    \end{macrocode}

% \macro{\childdocforward}
% The command |\childdocforward| redirects
% compilation to the main file or
% (if the optional argument is given) a child file.
% Parameters are set as if the main file
% or a child file starting with |\childdocof| was compiled.
% Then compilation is handed over to the main file:
%    \begin{macrocode}
\newcommand{\childdocforward}[2][]
{
  \begingroup
    \if?#1?
      \def\childdoctmp
      {
        \def\childdocname{#2}
        \def\childdocjob{#2}
        \def\jobname{#2}
        \input{#2}
        \endinput
      }
    \else
      \def\childdoctmp
      {
        \childdocdisable
        \def\childdocname{#2}
        \childdoctrue
        \includeonly{#2}
        \def\childdocjob{#1}
        \def\jobname{#1}
        \input{#1}
        \endinput
      }
    \fi
    \expandafter
  \endgroup
  \childdoctmp
}
%    \end{macrocode}

% \macro{\childdocforwardprefix}
% The command |\childdocforwardprefix| redirects
% compilation to the main or a child file by means of a pattern.
% The prefix |#1| in the current filename is replaced by |#2|
% and the suffix of the current filename is kept
% (it is assumed that the filename does not contain the substring `|~~~|'
% which is used as a delimiter).
% Compilation is handed over to the new file by |\childdocforward|:
%    \begin{macrocode}
\newcommand{\childdocforwardprefix}[3][]
{
  \begingroup
    \def\childdocextract #2##1~~~{\def\childdoctmp{\childdocforward[#1]{#3##1}}}
    \expandafter\childdocextract\childdocname~~~
    \expandafter
  \endgroup
  \childdoctmp
}
%    \end{macrocode}

% \macro{\childdoc}
% The deprecated macro |\childdoc| is a legacy version of |\childdocmain|:
%    \begin{macrocode}
\newcommand{\childdoc}{\childdocmain}
%    \end{macrocode}

% \macro{\childdocredirect}
% The deprecated macro |\childdocredirect| is a legacy version
% of |\childdocforward| and |\childdocforwardprefix|:
%    \begin{macrocode}
\newcommand{\childdocredirect}[2][]
{
  \begingroup
    \if?#1?
      \def\childdoctmp{\childdocforward{#2}}
    \else
      \def\childdoctmp{\childdocforwardprefix{#1}{#2}}
    \fi
    \expandafter
  \endgroup
  \childdoctmp
}
%    \end{macrocode}

%\iffalse
%</package>
%\fi
%
\endinput

\childdocby{cdocsamp}
%    \end{macrocode}

%\iffalse
%</samplepart3|samplepart4>
%\fi
%
%\iffalse
%<*samplepart3>
%\fi
% Some text for part 3:
%    \begin{macrocode}
some text in part three
%    \end{macrocode}

%\iffalse
%</samplepart3>
%\fi
% Some text for part 4:
%\iffalse
%<*samplepart4>
%\fi
%    \begin{macrocode}
more text in part four
%    \end{macrocode}

%\iffalse
%</samplepart4>
%\fi
%
% %%%%%%%%%%%%%%%%%%%%%%%%%%%%%%%%%%%%%%
% \paragraph{Forwarding for a Complete Draft.}
%
% The following forwarding file |cdocsdrf.tex|
% compiles the main document in draft mode:
%\iffalse
%<*sampledraft>
%\fi
%    \begin{macrocode}
\def\version{draft}
% \iffalse
%
% childdoc.dtx Copyright (C) 2017-2018 Niklas Beisert
%
% This work may be distributed and/or modified under the
% conditions of the LaTeX Project Public License, either version 1.3
% of this license or (at your option) any later version.
% The latest version of this license is in
%   http://www.latex-project.org/lppl.txt
% and version 1.3 or later is part of all distributions of LaTeX
% version 2005/12/01 or later.
%
% This work has the LPPL maintenance status `maintained'.
%
% The Current Maintainer of this work is Niklas Beisert.
%
% This work consists of the files childdoc.dtx and childdoc.ins
% and the derived files childdoc.def and cdocsamp.tex with
% cdocsch1.tex, cdocsch2.tex, cdocsdrf.tex, cdocsfn1.tex, cdocsfn2.tex.
%
%<package>\ifdefined\childdocmain\endinput\fi
%<package>\ProvidesFile{childdoc.def}[2018/12/30 v2.0 child document driver]
%<samplemain>\ProvidesFile{cdocsamp.tex}[2018/12/30 v2.0 sample for childdoc]
%<*driver>
%\ProvidesFile{childdoc.drv}[2018/12/30 v2.0 childdoc reference manual file]
\PassOptionsToClass{10pt,a4paper}{article}
\documentclass{ltxdoc}

\usepackage[margin=35mm]{geometry}
\usepackage{hyperref}
\usepackage{hyperxmp}
\usepackage[usenames]{color}

\hypersetup{colorlinks=true}
\hypersetup{pdfstartview=FitH}
\hypersetup{pdfpagemode=UseNone}
\hypersetup{pdfsource={}}
\hypersetup{pdflang={en-UK}}
\hypersetup{pdfcopyright={Copyright 2017-2018 Niklas Beisert.
  This work may be distributed and/or modified under the
  conditions of the LaTeX Project Public License, either version 1.3
  of this license or (at your option) any later version.}}
\hypersetup{pdflicenseurl={http://www.latex-project.org/lppl.txt}}
\hypersetup{pdfcontactaddress={ETH Zurich, ITP, HIT K,
  Wolfgang-Pauli-Strasse 27}}
\hypersetup{pdfcontactpostcode={8093}}
\hypersetup{pdfcontactcity={Zurich}}
\hypersetup{pdfcontactcountry={Switzerland}}
\hypersetup{pdfcontactemail={nbeisert@itp.phys.ethz.ch}}
\hypersetup{pdfcontacturl={http://people.phys.ethz.ch/\xmptilde nbeisert/}}

\newcommand{\secref}[1]{\hyperref[#1]{section \ref*{#1}}}

\parskip1ex
\parindent0pt
\let\olditemize\itemize
\def\itemize{\olditemize\parskip0pt}

\begin{document}

\title{The \textsf{childdoc} Package}
\hypersetup{pdftitle={The childdoc Package}}
\author{Niklas Beisert\\[2ex]
  Institut f\"ur Theoretische Physik\\
  Eidgen\"ossische Technische Hochschule Z\"urich\\
  Wolfgang-Pauli-Strasse 27, 8093 Z\"urich, Switzerland\\[1ex]
  \href{mailto:nbeisert@itp.phys.ethz.ch}
  {\texttt{nbeisert@itp.phys.ethz.ch}}}
\hypersetup{pdfauthor={Niklas Beisert}}
\hypersetup{pdfsubject={Manual for the LaTeX2e Package childdoc}}
\date{30 December 2018, \textsf{v2.0}}
\maketitle

\begin{abstract}\noindent
\textsf{childdoc} is a \LaTeXe{} package
that enables the direct compilation
of document sections included by |\include|
to individual files.
\end{abstract}

\begingroup
\parskip0ex
\tableofcontents
\endgroup

%%%%%%%%%%%%%%%%%%%%%%%%%%%%%%%%%%%%%%%%%%%%%%%%%%%%%%%%%%%%%%%%%%%%%%%%%%%%%%%%
%%%%%%%%%%%%%%%%%%%%%%%%%%%%%%%%%%%%%%%%%%%%%%%%%%%%%%%%%%%%%%%%%%%%%%%%%%%%%%%%
\section{Introduction}

\LaTeX{} provides a mechanism to structure a large document (such as a book)
into a main file and several child files (containing the chapters)
using the |\include| command.
This mechanism is beneficial for documents
which span hundreds of pages in order to
make the source file(s) more manageable.
Moreover, compilation can be restricted to
selected child files by means of the |\includeonly| command.
The latter feature can be used to reduce the compilation time while editing
(this was significantly more useful in the earlier days of \LaTeX{})
or to generate a smaller document which is easier to navigate.
Another application of |\includeonly| is to generate
documents consisting of selected parts of the complete document.

However, there are a few drawbacks of the plain |\include| mechanism:
\begin{itemize}
\item
The child files cannot be compiled on their own,
they can only be compiled via the main file.
A naive editing environment
(such as a text editor with an option
to have the current file processed by \LaTeX)
may require one to switch to the main file before compiling;
attempting to compile the child file produces errors.
\item
The main file must be modified (each time)
to adjust the |\includeonly| command
to the present needs. This easily leaves the main file in a messy state.
\item
The generated document will always carry the filename
of the main document. This is inconvenient if
several child files are to be compiled and
to be kept for distribution.
\end{itemize}

The present package provides a simple interface
to make child files individually compilable by \LaTeX{}.
Compiling a child file then has the same effect as compiling
the main file with an |\includeonly| command
to select the appropriate child.
Moreover the generated document will carry the name of the child
rather than the main file.
This resolves all three above issues.

This feature is meant to make the editing of books,
thesis documents and lecture notes somewhat more convenient.
However, the package can also be used efficiently for
composing a series of documents (such as exercise sheets)
which are typically distributed individually.
It then assists the author in generating the individual documents
(potentially in different versions)
as well as a document containing the collected series.
Another application is in developing style files
or other kinds of included material
where compilation of the style file could redirect
to a sample or test file.

%%%%%%%%%%%%%%%%%%%%%%%%%%%%%%%%%%%%%%%%%%%%%%%%%%%%%%%%%%%%%%%%%%%%%%%%%%%%%%%%
%%%%%%%%%%%%%%%%%%%%%%%%%%%%%%%%%%%%%%%%%%%%%%%%%%%%%%%%%%%%%%%%%%%%%%%%%%%%%%%%
\section{Usage}

First of all, the package \textsf{childdoc} is \emph{not} a standard
\LaTeXe{} |.sty| style file! Therefore it needs to be invoked in
a non-standard way.

%%%%%%%%%%%%%%%%%%%%%%%%%%%%%%%%%%%%%%%%%%%%%%%%%%%%%%%%%%%%%%%%%%%%%%%%%%%%%%%%
\subsection{Included Files}
\label{sec:include}

%%%%%%%%%%%%%%%%%%%%%%%%%%%%%%%%%%%%%%%%
\DescribeMacro{\childdocmain}
To use the package, add the commands
\begin{center}
\begin{tabular}{l}
|\input{childdoc.def}|\\
|\childdocmain{}|\\
\end{tabular}
\end{center}
at the very top of the main \LaTeX{} file,
in particular \emph{before} the |\documentclass| statement!
The argument of |\childdocmain| should be left empty
(but it must be present).

%%%%%%%%%%%%%%%%%%%%%%%%%%%%%%%%%%%%%%%%
\DescribeMacro{\childdocof}
Furthermore, add the commands
\begin{center}
\begin{tabular}{l}
|\input{childdoc.def}|\\
|\childdocof{|\textit{main}|}|\\
\end{tabular}
\end{center}
at the top of every child file \textit{child}
which is included by |\include{|\textit{child}|}|
from within the main file
(or at least for those files to be compiled individually).
The argument \textit{main} must be the filename of the main file.

There are a couple of
considerations in setting up the main and child documents:

%%%%%%%%%%%%%%%%%%%%%%%%%%%%%%%%%%%%%%%%
\paragraph{Restrictions.}

Please note the following restrictions:
\begin{itemize}
\item
|\childdocmain| must be called with one argument \textit{main}
to ensure compatibility with earlier version of the package.
It must either be empty (|\childdocmain{}|)
or precisely match the filename of the main file in which it is specified.
See \secref{sec:detection} for further information.
\item
The filename \textit{main} must be specified without the |.tex| extension.
\item
The filename \textit{main} is case sensitive
(even in case-insensitive file systems)
due to internal string comparison.
\item
The argument \textit{main} should be fully expanded, it cannot be a macro.
\item
Subdirectories and special characters should be avoided in filenames.
\item
The command |\childdocmain{|\textit{main}|}| must be followed by a whitespace.
It should not be followed immediately by another command
or by a comment mark `|%|'.
This is because the \TeX{} parser reads the token immediately following
the argument of |\childdocmain| and puts it
at the beginning of every child section;
however, a white\-space is ignored.
\end{itemize}

%%%%%%%%%%%%%%%%%%%%%%%%%%%%%%%%%%%%%%%%
\paragraph{Content of Main File.}

It is advisable to place all content in the child files included by |\include|.
Any output contained in the main file will appear in all child documents
unless suppressed manually;
it cannot be suppressed automatically by the |\includeonly| directive
and thus should normally be avoided.
A method to include some content in the main file
by means of conditional processing is described in \secref{sec:conditional}.

%%%%%%%%%%%%%%%%%%%%%%%%%%%%%%%%%%%%%%%%
\paragraph{Page Numbering.}

When only a part of the document is compiled,
the appropriate numbering of pages
(as well as other status parameters)
is determined from the |.aux| files.
The latter contain information from previous passes.
However this information needs to propagate through
all intermediate child documents.
Therefore the page numbering in child documents may well
be inconsistent until the complete document is compiled at least once.

A useful (if unconventional) way to always ensure a consistent
page numbering is to restart the numbering in each child document
and denote the pages by `\textit{child}|.|\textit{page}'
where \textit{child} represents the chapter/section number of the child file.
This can be achieved by the command
|\numberwithin{page}{|\textit{child}|}|
of the \textsf{amsmath} package
where \textit{child} can be |chapter| or |section|
depending on the chosen structuring.
Alternatively, one can modify the macro |\thepage| appropriately
and reset the counter |page| at the start of each child file.

%%%%%%%%%%%%%%%%%%%%%%%%%%%%%%%%%%%%%%%%%%%%%%%%%%%%%%%%%%%%%%%%%%%%%%%%%%%%%%%%
\subsection{Conditional Processing}
\label{sec:conditional}

The package provides a mechanism to compile different versions
of a document. To customise the versions further some conditional processing
can come in handy to distinguish which version is being compiled.
The package provides two macros to describe the compilation context:

%%%%%%%%%%%%%%%%%%%%%%%%%%%%%%%%%%%%%%%%
\DescribeMacro{\ifchilddoc}
The conditional |\ifchilddoc| distinguishes between the compilation of
child documents and the main document:
%
\begin{center}
|\ifchilddoc |\textit{child-code}| |[|\||else |\textit{main-code}]| \||fi|
\end{center}

%%%%%%%%%%%%%%%%%%%%%%%%%%%%%%%%%%%%%%%%
\DescribeMacro{\childdocname}
\DescribeMacro{\childdocjob}
The macro |\childdocname| contains the filename (without extension)
of the main or child file being processed.
Note that |\childdocjob| will always contain the name of the main file.

%%%%%%%%%%%%%%%%%%%%%%%%%%%%%%%%%%%%%%%%
\paragraph{Title Page.}

Conditional processing can be used to include a title or banner page
in the main document when proper precautions are taken.
Importantly, the code in the main file should ensure that the page counter
(as well as other status parameters which are stored in the |.aux| files)
takes the same value after the conditional processing.
Otherwise the page numbers may take divergent values
depending on which part is compiled.

For example, a title page could be declared by:
%
\begin{center}
\begin{tabular}{l}
|\ifchilddoc\||else|\\
|\addtocounter{page}{-1}|\\
\textit{code for title page}\\
|\newpage|\\
|\||fi|
\end{tabular}
\end{center}
%
A banner page for the child documents can be generated by:
%
\begin{center}
\begin{tabular}{l}
|\ifchilddoc|\\
|\addtocounter{page}{-1}|\\
\textit{code for banner page}\\
|\newpage|\\
|\||fi|
\end{tabular}
\end{center}
%
Here one could write a message such as:
\begin{center}
|This is the part \childdocname{} of \childdocjob{}.|
\end{center}

%%%%%%%%%%%%%%%%%%%%%%%%%%%%%%%%%%%%%%%%%%%%%%%%%%%%%%%%%%%%%%%%%%%%%%%%%%%%%%%%
\subsection{Flags}
\label{sec:flags}

The package makes it easy to generate different versions
of the main or child documents.
To this end compilation flags can be defined
and assigned different default values.
They will be particularly useful in conjunction
with the forwarding mechanism described in \secref{sec:forward}.

For example, it may be useful to have a flag |\version|
which can be set to |draft| or |final|.
The document source will contain some conditional code
depending on the value of |\version|.
Suppose further, the flag should default to |final| for the main file
and to |draft| for child files
which is a natural assignment for editing the document.
This is achieved by placing the following code
in the preamble of the main document
(below the |\childdocmain| directive):
%
\begin{center}
\begin{tabular}{l}
|\ifchilddoc|\\
|\providecommand{\version}{draft}|\\
|\||else|\\
|\providecommand{\version}{final}|\\
|\||fi|
\end{tabular}
\end{center}
%
The definition by |\providecommand| makes sure
that previous definitions are not overwritten.
Further statements |\providecommand{\version}{...}|
can thus be added before the above code to override it.

For the main file, one might add a line
(between |\childdocmain| and the above block)
%
\begin{center}
|%\ifchilddoc\||else\providecommand{\version}{draft}\||fi|
\end{center}
%
which can be uncommented to produce a draft version.
Likewise one can add a line to the very top of a child file
(above the |\childdocof{|\textit{main}|}| directive)
%
\begin{center}
|%\providecommand{\version}{final}|
\end{center}
%
which can be uncommented to produce the final version of this child document.

%%%%%%%%%%%%%%%%%%%%%%%%%%%%%%%%%%%%%%%%%%%%%%%%%%%%%%%%%%%%%%%%%%%%%%%%%%%%%%%%
\subsection{Forwarding}
\label{sec:forward}

Different versions of the main or child documents
using compilation flags as described in \secref{sec:flags}
can be (permanently) stored in different files
for convenient compilation, viewing and distribution.
To this end, the package defines a command
to pass on compilation to a different file:

%%%%%%%%%%%%%%%%%%%%%%%%%%%%%%%%%%%%%%%%
\DescribeMacro{\childdocforward}
The command |\childdocforward| redirects processing to
another source file:
%
\begin{center}
\begin{tabular}{l}
|\input{childdoc.def}|\\
|\childdocforward[|\textit{main}|]{|\textit{dest}|}|\\
\end{tabular}
\end{center}
%
The argument \textit{dest} is the destination file
(without extension).
It should be the main file or one of the child files.
Note that further \textsf{childdoc} directives
such as |\childdocof| and |\childdocforward|
in the indicated file will be processed in this form.
The optional argument \textit{main}
passes on directly to the main file \textit{main}
while pretending to compile the child \textit{dest}.
This form behaves as if \textit{dest}
issues |\childdocof{|\textit{main}|}| right away,
and no further \textsf{childdoc} directives will be processed.

%%%%%%%%%%%%%%%%%%%%%%%%%%%%%%%%%%%%%%%%
\DescribeMacro{\...prefix}
In the alternative form |\childdocforwardprefix|,
%
\begin{center}
\begin{tabular}{l}
|\input{childdoc.def}|\\
|\childdocforwardprefix[|\textit{main}|]{|\textit{prefix}|}{|\textit{dest}|}|
\end{tabular}
\end{center}
%
the destination file is determined by a pattern
depending on the current file:
To make this work, the current file must be called
`{\textit{prefix}\hspace{0.2em}\textit{suffix}}'
with \textit{prefix} matching precisely the argument.
Processing is then passed on to the file
`{\textit{dest}\hspace{0.2em}\textit{suffix}}'.
Surely, the same effect is achieved by
directly specifying the
argument `{\textit{dest}\hspace{0.2em}\textit{suffix}}'
in the first form.
However, that requires to set up a different file
for each child. With the alternative form of the command
all these files can have exactly the same content
which simplifies setting them up and maintaining them.

For example, the following file |draft.tex|
with a compilation flag |\version| as described in \secref{sec:flags}
compiles the main document as a draft:
%
\begin{center}
\begin{tabular}{l}
|\def\version{draft}|\\
|\input{childdoc.def}|\\
|\childdocforward{|\textit{main}|}|
\end{tabular}
\end{center}
%
Likewise, the following files |final|\textit{nn}|.tex|
compile the final version of the child document
|child|\textit{nn}|.tex|:
%
\begin{center}
\begin{tabular}{l}
|\def\version{final}|\\
|\input{childdoc.def}|\\
|\childdocforwardprefix{final}{child}|
\end{tabular}
\end{center}
%

Note that when several versions of a main file and/or of each child file
are to be generated, it may be convenient to set up a |Makefile| or
shell script to automatise the process.

%%%%%%%%%%%%%%%%%%%%%%%%%%%%%%%%%%%%%%%%%%%%%%%%%%%%%%%%%%%%%%%%%%%%%%%%%%%%%%%%
\subsection{Command Line Processing}
\label{sec:commandline}

The effect of redirection files can also be achieved by invoking
the \LaTeX{} compiler with a more elaborate command line.
Most conveniently this should be done as part
of a shell script or a |Makefile|.

When using \textsf{childdoc} in the main file, the following
command lines effectively perform a redirection
(note that depending on the shell being used,
backslashes may have to be doubled: `|\|' $\to$ `|\\|'):
%
\begin{center}
|... -jobname "|\textit{target}|" |\\|"|[\textit{flags}]%
|\input{childdoc.def}\childdocforward[|\textit{main}|]{|\textit{dest}|}"|
\end{center}
%
Here \textit{target} is the name of the output file,
\textit{main} is the name of the main file
and \textit{dest} is the name of the main or child file to be processed
(all filenames without extensions).
The optional argument \textit{main} can be omitted
if \textit{main} matches \textit{dest}.
Optionally, compilation \textit{flags} can be defined via |\def| commands.
This command line makes the \TeX{} engine believe
it is compiling the file \textit{target}
whose content is specified as the latter parameter.
The provided code then forwards the processing to
\textit{main} or \textit{dest} as described in \secref{sec:forward}.

%%%%%%%%%%%%%%%%%%%%%%%%%%%%%%%%%%%%%%%%%%%%%%%%%%%%%%%%%%%%%%%%%%%%%%%%%%%%%%%%
\subsection{Include by Input}
\label{sec:input}

Including child documents by |\include| has some restrictions by design.
Most notably, the content of a child document always occupies
its own set of pages; pages cannot be shared between child documents.
Usually, this behaviour makes perfect sense
because each child document contain an essential part of the document.
However, in some situations it may be desirable to compose
a document from a collection of parts
without having mandatory page breaks between then.
For this case, the package
provides a mechanism to include parts
by |\input| which can also be processed individually.
However, by construction this mechanism
requires manual handling of the content to be output.

%%%%%%%%%%%%%%%%%%%%%%%%%%%%%%%%%%%%%%%%
\DescribeMacro{\ifchilddocmanual}
The main file should be prepared as usual, see \secref{sec:include}.
However, the document body must make a distinction
between processing of an individual part and of the main document, e.g.:
%
\begin{center}
\begin{tabular}{l}
|\ifchilddocmanual|\\
|\input{\childdocname}|\\
|\||else|\\
\textit{document body with }|\input{|\textit{part}|}|\\
|\||fi|
\end{tabular}
\end{center}
%
The conditional |\ifchilddocmanual| is true whenever
a part to be included by |\input| is being compiled,
and the name of the part is stored in |\childdocname|.

%%%%%%%%%%%%%%%%%%%%%%%%%%%%%%%%%%%%%%%%
\DescribeMacro{\childdocby}
Each part to be included by |\input| should start with:
%
\begin{center}
\begin{tabular}{l}
|\input{childdoc.def}|\\
|\childdocby{|\textit{main}|}|\\
\end{tabular}
\end{center}
%
The directive |\childdocby| is similar to |\childdocof|
described in \secref{sec:include},
but the subsequent selection of content must be done manually.
To that end, both |\ifchilddoc| and |\ifchilddocmanual|
will be true upon processing of a part,
and the name of the part is stored in |\childdocname|.
Note that |\jobname| will be set to the filename of the current part
so that each part receives an individual |.aux| file
that does not interfere with the |.aux| file(s) of the main document.
This behaviour can be altered by the alternative form
|\childdocby[*]{|\textit{main}|}| (with a non-empty optional argument)
which uses the |.aux| file of the main document
by setting |\jobname| to \textit{main}.

%%%%%%%%%%%%%%%%%%%%%%%%%%%%%%%%%%%%%%%%%%%%%%%%%%%%%%%%%%%%%%%%%%%%%%%%%%%%%%%%
\subsection{Driver Development}
\label{sec:driver}

The \textsf{childdoc} mechanism can also be use for the development
of definition files such as \LaTeX{} styles or classes.
This case differs from the above setup with multiple parts
included by |\include| in that no |\includeonly| should be invoked.
This can be achieved by starting the include file
(before |\ProvidesPackage|) with:
%
\begin{center}
\begin{tabular}{l}
|\input{childdoc.def}|\\
|\childdocforward{|\textit{main}|}|\\
\end{tabular}
\end{center}
%
or alternatively with:
%
\begin{center}
\begin{tabular}{l}
|\input{childdoc.def}|\\
|\childdocby{|\textit{main}|}|\\
\end{tabular}
\end{center}
%
Both forms have slightly different effects as described above.
The main file is prepared as usual, see \secref{sec:include}.

%%%%%%%%%%%%%%%%%%%%%%%%%%%%%%%%%%%%%%%%%%%%%%%%%%%%%%%%%%%%%%%%%%%%%%%%%%%%%%%%
\subsection{Legacy Detection}
\label{sec:detection}

The directive |\childdocmain| in the main file can detect
whether the complete document or merely a child is to be compiled
even without using the directive |\childdocof|.
This method is deprecated because it is less robust
and there is no compelling reason to use it;
it is merely provided for backward compatibility
and it may be removed in future versions.

If the detection mechanism is to be used,
it is mandatory to correctly specify
the filename of the main file as the argument of |\childdocmain|:
%
\begin{center}
\begin{tabular}{l}
|\input{childdoc.def}|\\
|\childdocmain{|\textit{main}|}|\\
\end{tabular}
\end{center}
%
If |\jobname| does not match the argument \textit{main} of |\childdocmain|,
it is assumed that |\jobname| points to the child file to be compiled.
When using |\childdocmain| with the main file specified as argument,
it suffices to start a child file
with just |\input{|\textit{main}|}|
without loading of the package and using |\childdocof|.
If instead all processing is done
with the appropriate \textsf{childdoc} directives,
the argument of \textit{main} of |\childdocmain| can be empty.

An alternative version of the command line processing described
in \secref{sec:commandline} using the detection mechanism reads:
%
\begin{center}
|... -jobname "|\textit{target}|" "|[\textit{flags}]%
[|\def\jobname{|\textit{dest}|}|]|\input{|\textit{main}|}"|
\end{center}

%%%%%%%%%%%%%%%%%%%%%%%%%%%%%%%%%%%%%%%%%%%%%%%%%%%%%%%%%%%%%%%%%%%%%%%%%%%%%%%%
\subsection{Manual Code}
\label{sec:manual}

In case one cannot be certain whether the definitions file |childdoc.def|
is installed on the target \TeX{} distribution
and one prefers not to ship it,
it is conceivable to paste a few relevant commands into the sources.

To that end, drop all statements |\input{childdoc.def}|
and perform the replacements as outlined below.
Instead of |\childdocmain{|\textit{main}|}| add the following code
to the top of the main file:
%
\begin{center}
\begin{tabular}{l}
|\||ifdefined\childdocname\endinput\||fi\newif\ifchilddoc|\\
|\edef\childdocname{\scantokens\expandafter{\jobname\noexpand}}|\\
|\def\childdocmain{|\textit{main}|}\||ifx\childdocmain\childdocname\||else|\\
|\childdoctrue\includeonly{\childdocname}\let\jobname\childdocmain\||fi|\\
\end{tabular}
\end{center}
%
Instead of |\childdocof{|\textit{main}|}| just include the main file
at the top of each child file:
%
\begin{center}
|\input{|\textit{main}|}|
\end{center}
%
A simple redirection |\childdocforward{|\textit{dest}|}| is achieved by:
%
\begin{center}
|\def\jobname{|\textit{dest}|}\input{\jobname}|
\end{center}
%
The redirection with prefix
|\childdocforwardprefix[|\textit{prefix}|]{|\textit{dest}|}|
is accomplished by:
%
\begin{center}
\begin{tabular}{l}
|{\edef\jobname{\scantokens\expandafter{\jobname\noexpand}}|\\
|\def\redirectjob |\textit{prefix}|#1~~~{\gdef\jobname{|\textit{dest}|#1}}|\\
|\expandafter\redirectjob\jobname~~~}\input{\jobname}|
\end{tabular}
\end{center}

In an alternative approach,
child documents can be compiled by a specific command line
without additional code or specific definitions:
%
\begin{center}
|... -jobname "|\textit{target}|" "|[\textit{flags}]%
|\includeonly{|\textit{dest}|}\input{|\textit{main}|}"|
\end{center}
%

%%%%%%%%%%%%%%%%%%%%%%%%%%%%%%%%%%%%%%%%%%%%%%%%%%%%%%%%%%%%%%%%%%%%%%%%%%%%%%%%
%%%%%%%%%%%%%%%%%%%%%%%%%%%%%%%%%%%%%%%%%%%%%%%%%%%%%%%%%%%%%%%%%%%%%%%%%%%%%%%%
\section{Information}

%%%%%%%%%%%%%%%%%%%%%%%%%%%%%%%%%%%%%%%%%%%%%%%%%%%%%%%%%%%%%%%%%%%%%%%%%%%%%%%%
\subsection{Copyright}

Copyright \copyright{} 2017--2018 Niklas Beisert

This work may be distributed and/or modified under the
conditions of the \LaTeX{} Project Public License, either version 1.3
of this license or (at your option) any later version.
The latest version of this license is in
  \url{http://www.latex-project.org/lppl.txt}
and version 1.3 or later is part of all distributions of \LaTeX{}
version 2005/12/01 or later.

This work has the LPPL maintenance status `maintained'.

The Current Maintainer of this work is Niklas Beisert.

This work consists of the files |README.txt|, |childdoc.ins| and |childdoc.dtx|
as well as the derived files |childdoc.def|, |cdocsamp.tex|
with |cdocsch1.tex|, |cdocsch2.tex|, |cdocspt3.tex|, |cdocspt4.tex|,
|cdocsdrf.tex|, |cdocsfn1.tex|, |cdocsfn2.tex|
as well as |childdoc.pdf|.

%%%%%%%%%%%%%%%%%%%%%%%%%%%%%%%%%%%%%%%%%%%%%%%%%%%%%%%%%%%%%%%%%%%%%%%%%%%%%%%%
\subsection{Files and Installation}

The package consists of the files:
%
\begin{center}
\begin{tabular}{ll}
    |README.txt|   & readme file \\
    |childdoc.ins| & installation file \\
    |childdoc.dtx| & source file \\
    |childdoc.def| & definition file \\
    |cdocsamp.tex| & sample main file \\
    |cdocsch1.tex| & sample include file \\
    |cdocsch2.tex| & sample include file \\
    |cdocspt3.tex| & sample part file \\
    |cdocspt4.tex| & sample part file \\
    |cdocsdrf.tex| & sample redirection file \\
    |cdocsfn1.tex| & sample redirection file \\
    |cdocsfn2.tex| & sample redirection file \\
    |childdoc.pdf| & manual
\end{tabular}
\end{center}
%
The distribution consists of the files
|README.txt|, |childdoc.ins| and |childdoc.dtx|.
%
\begin{itemize}
\item
Run (pdf)\LaTeX{} on |childdoc.dtx|
to compile the manual |childdoc.pdf| (this file).
\item
Run \LaTeX{} on |childdoc.ins| to create the definitions file |childdoc.def|
and the sample |cdocsamp.tex| with include files
|cdocsch1.tex|, |cdocsch2.tex|, |cdocspt3.tex|, |cdocspt4.tex|,
|cdocsdrf.tex|, |cdocsfn1.tex|, |cdocsfn2.tex|.
Then copy the file |childdoc.def| to an appropriate directory of your \LaTeX{}
distribution, e.g.\ \textit{texmf-root}|/tex/latex/childdoc|.
\end{itemize}

%%%%%%%%%%%%%%%%%%%%%%%%%%%%%%%%%%%%%%%%%%%%%%%%%%%%%%%%%%%%%%%%%%%%%%%%%%%%%%%%
\subsection{Related CTAN Packages}

There are several other packages which offer a similar functionality:
%
\begin{itemize}
\item
The packages
\href{http://ctan.org/pkg/docmute}{\textsf{docmute}},
\href{http://ctan.org/pkg/includex}{\textsf{includex}} and
\href{http://ctan.org/pkg/standalone}{\textsf{standalone}}
provide commands to include only the document body of
a child file thus allowing both files to be compiled individually.
\item
The packages \href{http://ctan.org/pkg/subdocs}{\textsf{subdocs}}
and \href{http://ctan.org/pkg/subfiles}{\textsf{subfiles}}
provide structures in which the main and child documents can be
encapsulated and allowing them to be compiled individually.
The inclusion mechanism is different from the conventional |\include|.
\item
The package \href{http://ctan.org/pkg/combine}{\textsf{combine}}
is an elaborate solution to combine several documents into one.
\end{itemize}
%
See also the CTAN topic \href{http://ctan.org/topic/subdocs}{\textsf{subdocs}}
for further related packages.
The present package differs from the above solutions in that
a document structure constructed with the conventional |\include| mechanism
just needs two extra commands at the top of every file
such that all constituent files can be compiled individually.

%%%%%%%%%%%%%%%%%%%%%%%%%%%%%%%%%%%%%%%%%%%%%%%%%%%%%%%%%%%%%%%%%%%%%%%%%%%%%%%%
%\subsection{Feature Suggestions}
%
%The following is a list of features which may be useful for future
%versions of this package:
%%
%\begin{itemize}
%\item
%\ldots
%\end{itemize}

%%%%%%%%%%%%%%%%%%%%%%%%%%%%%%%%%%%%%%%%%%%%%%%%%%%%%%%%%%%%%%%%%%%%%%%%%%%%%%%%
\subsection{Revision History}

%%%%%%%%%%%%%%%%%%%%%%%%%%%%%%%%%%%%%%%%
\paragraph{v2.0:} 2018/12/30

\begin{itemize}
\item
immediate forward processing
\item
added |\childdocby| mechanism
\item
manual restructured
\end{itemize}

%%%%%%%%%%%%%%%%%%%%%%%%%%%%%%%%%%%%%%%%
\paragraph{v1.6:} 2018/01/17

\begin{itemize}
\item
application for development of include files
\item
corrections to manual
\end{itemize}

%%%%%%%%%%%%%%%%%%%%%%%%%%%%%%%%%%%%%%%%
\paragraph{v1.5:} 2017/05/21

\begin{itemize}
\item
more complete structuring introduced
\item
|\childdocof| introduced
\item
|\childdoc| renamed to |\childdocmain|
\item
|\childredirect| renamed to |\childdocforward| and |\childdocforwardprefix|
and functionality expanded
\end{itemize}

%%%%%%%%%%%%%%%%%%%%%%%%%%%%%%%%%%%%%%%%
\paragraph{v1.0:} 2017/04/27

\begin{itemize}
\item
manual and install package
\item
first version published on CTAN
\end{itemize}

%%%%%%%%%%%%%%%%%%%%%%%%%%%%%%%%%%%%%%%%
\paragraph{v0.6:} 2017/04/26

\begin{itemize}
\item
redirection mechanism added
\end{itemize}

%%%%%%%%%%%%%%%%%%%%%%%%%%%%%%%%%%%%%%%%
\paragraph{v0.5:} 2017/04/26

\begin{itemize}
\item
functionality in definition file
\end{itemize}


%%%%%%%%%%%%%%%%%%%%%%%%%%%%%%%%%%%%%%%%%%%%%%%%%%%%%%%%%%%%%%%%%%%%%%%%%%%%%%%%
%%%%%%%%%%%%%%%%%%%%%%%%%%%%%%%%%%%%%%%%%%%%%%%%%%%%%%%%%%%%%%%%%%%%%%%%%%%%%%%%
%%%%%%%%%%%%%%%%%%%%%%%%%%%%%%%%%%%%%%%%%%%%%%%%%%%%%%%%%%%%%%%%%%%%%%%%%%%%%%%%
\appendix

\settowidth\MacroIndent{\rmfamily\scriptsize 000\ }

 \DocInput{childdoc.dtx}

\end{document}
%</driver>
% \fi
%
% %%%%%%%%%%%%%%%%%%%%%%%%%%%%%%%%%%%%%%%%%%%%%%%%%%%%%%%%%%%%%%%%%%%%%%%%%%%%%%
% %%%%%%%%%%%%%%%%%%%%%%%%%%%%%%%%%%%%%%%%%%%%%%%%%%%%%%%%%%%%%%%%%%%%%%%%%%%%%%
% \section{Sample}
%\iffalse
%<*samplemain>
%\fi
%
% The following presents a sample document
% with two chapters, two parts, a title page,
% a compile flag as well as three forwarding files to set the flag.
% It consists of eight |.tex| files:
% \begin{center}
% \begin{tabular}{ll}
% |cdocsamp.tex|&main file\\
% |cdocsch1.tex|&include file for chapter 1\\
% |cdocsch2.tex|&include file for chapter 2\\
% |cdocspt3.tex|&include file for part 3\\
% |cdocspt4.tex|&include file for part 4\\
% |cdocsdrf.tex|&forwarding file for main file in draft mode\\
% |cdocsfi1.tex|&forwarding file for final version of chapter 1\\
% |cdocsfi2.tex|&forwarding file for final version of chapter 2\\
% \end{tabular}
% \end{center}
% Each of the eight files can be compiled directly by the \LaTeX{} compiler.
%
% %%%%%%%%%%%%%%%%%%%%%%%%%%%%%%%%%%%%%%
% \paragraph{Main File.}
%
% The main file is called |cdocsamp.tex|.
%
% Load the \textsf{childdoc} definitions and
% declare the filename for the main document:
%    \begin{macrocode}
\input{childdoc.def}
\childdocmain{}
%    \end{macrocode}

% Optional override for |\version| flag:
%    \begin{macrocode}
%%\ifchilddoc\else\providecommand{\version}{draft}\fi
%    \end{macrocode}

% Define the default values for the |\version| flag
% (|final| for the main file and |draft| for childs):
%    \begin{macrocode}
\ifchilddoc
\providecommand{\version}{draft}
\else
\providecommand{\version}{final}
\fi
%    \end{macrocode}

% Load the standard document class:
%    \begin{macrocode}
\documentclass[12pt]{article}
%    \end{macrocode}

% Start the document body:
%    \begin{macrocode}
\begin{document}
%    \end{macrocode}

% Declare a title page.
% Print title, part of document being processed and version flag:
%    \begin{macrocode}
\addtocounter{page}{-1}
\begin{center}
{\LARGE\bfseries{}childdoc example\par}
\vspace{1cm}
\ifchilddoc
\ifchilddocmanual part\else chapter\fi:
`\childdocname' of `\childdocjob'\par
\else
main document: `\childdocjob'\par
\fi
version: \version\par
\end{center}
\newpage
%    \end{macrocode}

% Manually include selected file,
% otherwise process as usual:
%    \begin{macrocode}
\ifchilddocmanual
\section*{part `\childdocname'}
\input{\childdocname}
\else
%    \end{macrocode}

% Include the two chapters:
%    \begin{macrocode}
\include{cdocsch1}
\include{cdocsch2}
%    \end{macrocode}

% Include the two parts unless only chapters should be displayed:
%    \begin{macrocode}
\ifchilddoc\else
\section{part three}
\input{cdocspt3}
\section{part four}
\input{cdocspt4}
\fi
%    \end{macrocode}

% Process as usual until here:
%    \begin{macrocode}
\fi
%    \end{macrocode}

% End of document body:
%    \begin{macrocode}
\end{document}
%    \end{macrocode}
%\iffalse
%</samplemain>
%\fi
%
% %%%%%%%%%%%%%%%%%%%%%%%%%%%%%%%%%%%%%%
% \paragraph{Chapter Include Files.}
%
% The include files are called |cdocsch1.tex| and |cdocsch2.tex|.
%
%\iffalse
%<*samplechap1|samplechap2>
%\fi

% Optional override for |\version| flag:
%    \begin{macrocode}
%%\providecommand{\version}{final}
%    \end{macrocode}

% Include the main document:
%    \begin{macrocode}
\input{childdoc.def}
\childdocof{cdocsamp}
%    \end{macrocode}

%\iffalse
%</samplechap1|samplechap2>
%\fi
%
%\iffalse
%<*samplechap1>
%\fi
% Some text for chapter 1:
%    \begin{macrocode}
\section{one}
some text in chapter one
%    \end{macrocode}

%\iffalse
%</samplechap1>
%\fi
% Some text for chapter 2:
%\iffalse
%<*samplechap2>
%\fi
%    \begin{macrocode}
\section{two}
more text in chapter two
%    \end{macrocode}

%\iffalse
%</samplechap2>
%\fi
%
% %%%%%%%%%%%%%%%%%%%%%%%%%%%%%%%%%%%%%%
% \paragraph{Part Include Files.}
%
% The include files are called |cdocspt3.tex| and |cdocspt4.tex|.
%
%\iffalse
%<*samplepart3|samplepart4>
%\fi

% Optional override for |\version| flag:
%    \begin{macrocode}
%%\providecommand{\version}{final}
%    \end{macrocode}

% Include the main document:
%    \begin{macrocode}
\input{childdoc.def}
\childdocby{cdocsamp}
%    \end{macrocode}

%\iffalse
%</samplepart3|samplepart4>
%\fi
%
%\iffalse
%<*samplepart3>
%\fi
% Some text for part 3:
%    \begin{macrocode}
some text in part three
%    \end{macrocode}

%\iffalse
%</samplepart3>
%\fi
% Some text for part 4:
%\iffalse
%<*samplepart4>
%\fi
%    \begin{macrocode}
more text in part four
%    \end{macrocode}

%\iffalse
%</samplepart4>
%\fi
%
% %%%%%%%%%%%%%%%%%%%%%%%%%%%%%%%%%%%%%%
% \paragraph{Forwarding for a Complete Draft.}
%
% The following forwarding file |cdocsdrf.tex|
% compiles the main document in draft mode:
%\iffalse
%<*sampledraft>
%\fi
%    \begin{macrocode}
\def\version{draft}
\input{childdoc.def}
\childdocforward{cdocsamp}
%    \end{macrocode}

%\iffalse
%</sampledraft>
%\fi
%
% %%%%%%%%%%%%%%%%%%%%%%%%%%%%%%%%%%%%%%
% \paragraph{Forwarding for Final Version of the Chapters.}
%
% The following forwarding files |cdocsfn1.tex| and |cdocsfn2.tex|
% (with identical content)
% compile the final versions of the child documents
% |cdocsch1.tex| and |cdocsch2.tex|, respectively:
%\iffalse
%<*samplefinal>
%\fi
%    \begin{macrocode}
\def\version{final}
\input{childdoc.def}
\childdocforwardprefix[cdocsamp]{cdocsfn}{cdocsch}
%    \end{macrocode}

%\iffalse
%</samplefinal>
%\fi
%
% %%%%%%%%%%%%%%%%%%%%%%%%%%%%%%%%%%%%%%
% \paragraph{Command Line Processing.}
%
% The following three command lines generate the output files
% |cdocscld|, |cdocscl1| and |cdocscl2|
% which should be identical to
% |cdocsdrf|, |cdocsch1| and |cdocsfn2|, respectively:
% \begin{center}
% \begin{tabular}{l}
% |latex -jobname cdocscld \|\\
% |  "\def\version{draft}\input{childdoc.def}\childdocforward{cdocsamp}"|\\
% |latex -jobname cdocscl1 \|\\
% |  "\input{childdoc.def}\childdocforward[cdocsamp]{cdocsch1}"|\\
% |latex -jobname cdocscl2 \|\\
% |  "\def\version{final}\input{childdoc.def}\childdocforward{cdocsch2}"|
% \end{tabular}
% \end{center}
% Note that the trailing backslash on each first line
% merely continues the input to the second line
% (for convenient cut ant paste).
% Furthermore, the command |latex| can be replaced by any
% of its alternative versions such as |pdflatex|.
%
% %%%%%%%%%%%%%%%%%%%%%%%%%%%%%%%%%%%%%%%%%%%%%%%%%%%%%%%%%%%%%%%%%%%%%%%%%%%%%%
% %%%%%%%%%%%%%%%%%%%%%%%%%%%%%%%%%%%%%%%%%%%%%%%%%%%%%%%%%%%%%%%%%%%%%%%%%%%%%%
% \section{Implementation}
%\iffalse
%<*package>
%\fi
%
% This section describes the definitions file |childdoc.def|.

% The definitions cannot be loaded using |\usepackage| or |\RequirePackage|
% which has a mechanism to prevent loading a style file more than once.
% When loading the definitions by means of |\input|
% multiple instances have to be prevented manually:
%\iffalse
%This code needs to be before the `\ProvidesFile' directive
%which is defined at the beginning of this file.
%Therefore it is also placed there and commented out here.
%</package>
%<*discard>
%\fi
%    \begin{macrocode}
\ifdefined\childdocmain\endinput\fi
%    \end{macrocode}
%\iffalse
%</discard>
%<*package>
%\fi
%
% \macro{\ifchilddoc}
% \macro{\ifchilddocmanual}
% The conditional |\ifchilddoc| tells whether a
% child (true) or main (false) document is being compiled.
% The conditional |\ifchilddocmanual| tells whether
% the |\includeonly| mechanism is used (false) or
% the selection of child files must be performed manually (true).
% The definitions initialise to false:
%    \begin{macrocode}
\newif\ifchilddoc
\newif\ifchilddocmanual
%    \end{macrocode}

% \macro{\childdocname}
% \macro{\childdocjob}
% The macro |\childdocname| stores the name of the main document
% to be compiled. The macro |\childdocjob| stores the name of
% the document on which the \LaTeX{} compiler was originally invoked.
% The content of |\jobname| cannot be compared
% to filenames specified in the source due to different catcodes.
% The following code rescans |\jobname|, stores the result
% in |\childdocname| and saves a copy in |\childdocjob|:
%    \begin{macrocode}
\edef\childdocname{\scantokens\expandafter{\jobname\noexpand}}
\let\childdocjob\childdocname
%    \end{macrocode}

% \macro{\childdocdisable}
% The macro |\childdocdisable| prevents the main file
% from being processed more than once.
% At this stage, the main document command |\childdocmain|
% is assumed to be called once again where it should do nothing.
% Any subsequent call to it should prevent
% a secondary processing of the main document
% It overwrites the forwarding commands
% |\childdocof| and |\childdocforward|
% with empty macros to prevent further inclusions of the main document:
%    \begin{macrocode}
\newcommand{\childdocdisable}
{
  \renewcommand{\childdocmain}[1]{\renewcommand{\childdocmain}[1]{\endinput}}
  \renewcommand{\childdocof}[1]{}
  \renewcommand{\childdocby}[2][]{}
  \renewcommand{\childdocforward}[2][]{}
  \renewcommand{\childdocdisable}{}
}
%    \end{macrocode}

% \macro{\childdocmain}
% The macro |\childdocmain| is to be called at the top of the main file
% with nothing or the main filename (without extension) as argument.
% First, it breaks loops.
% If the argument is not empty and does not match |\childdocname|
% (which is set by the first inclusion of |childdoc.def|),
% |\ifchilddoc| is set to true, |\includeonly| is applied to the child file
% and |\jobname| is set to the main file
% (for proper handling of |.aux| files):
%    \begin{macrocode}
\newcommand{\childdocmain}[1]
{
  \childdocdisable\childdocmain{}
  \if?#1?\else
    \begingroup
      \def\childdoctmp{#1}
      \ifx\childdoctmp\childdocname
        \def\childdoctmp{}
      \else
        \def\childdoctmp
        {
          \childdoctrue
          \includeonly{\childdocname}
          \def\childdocjob{#1}
          \def\jobname{#1}
        }
      \fi
      \expandafter
    \endgroup
    \childdoctmp
  \fi
}
%    \end{macrocode}

% \macro{\childdocof}
% The command |\childdocof| redirects
% compilation to the main file |#1|.
%    \begin{macrocode}
\newcommand{\childdocof}[1]
{
  \childdocdisable
  \childdoctrue
  \includeonly{\childdocname}
  \def\jobname{#1}
  \def\childdocjob{#1}
  \input{#1}
}
%    \end{macrocode}

% \macro{\childdocby}
% The command |\childdocby| ....
%    \begin{macrocode}
\newcommand{\childdocby}[2][]
{
  \childdocdisable
  \childdoctrue
  \childdocmanualtrue
  \if?#1?\else
    \def\jobname{#2}
  \fi
  \def\childdocjob{#2}
  \input{#2}
  \endinput
}
%    \end{macrocode}

% \macro{\childdocforward}
% The command |\childdocforward| redirects
% compilation to the main file or
% (if the optional argument is given) a child file.
% Parameters are set as if the main file
% or a child file starting with |\childdocof| was compiled.
% Then compilation is handed over to the main file:
%    \begin{macrocode}
\newcommand{\childdocforward}[2][]
{
  \begingroup
    \if?#1?
      \def\childdoctmp
      {
        \def\childdocname{#2}
        \def\childdocjob{#2}
        \def\jobname{#2}
        \input{#2}
        \endinput
      }
    \else
      \def\childdoctmp
      {
        \childdocdisable
        \def\childdocname{#2}
        \childdoctrue
        \includeonly{#2}
        \def\childdocjob{#1}
        \def\jobname{#1}
        \input{#1}
        \endinput
      }
    \fi
    \expandafter
  \endgroup
  \childdoctmp
}
%    \end{macrocode}

% \macro{\childdocforwardprefix}
% The command |\childdocforwardprefix| redirects
% compilation to the main or a child file by means of a pattern.
% The prefix |#1| in the current filename is replaced by |#2|
% and the suffix of the current filename is kept
% (it is assumed that the filename does not contain the substring `|~~~|'
% which is used as a delimiter).
% Compilation is handed over to the new file by |\childdocforward|:
%    \begin{macrocode}
\newcommand{\childdocforwardprefix}[3][]
{
  \begingroup
    \def\childdocextract #2##1~~~{\def\childdoctmp{\childdocforward[#1]{#3##1}}}
    \expandafter\childdocextract\childdocname~~~
    \expandafter
  \endgroup
  \childdoctmp
}
%    \end{macrocode}

% \macro{\childdoc}
% The deprecated macro |\childdoc| is a legacy version of |\childdocmain|:
%    \begin{macrocode}
\newcommand{\childdoc}{\childdocmain}
%    \end{macrocode}

% \macro{\childdocredirect}
% The deprecated macro |\childdocredirect| is a legacy version
% of |\childdocforward| and |\childdocforwardprefix|:
%    \begin{macrocode}
\newcommand{\childdocredirect}[2][]
{
  \begingroup
    \if?#1?
      \def\childdoctmp{\childdocforward{#2}}
    \else
      \def\childdoctmp{\childdocforwardprefix{#1}{#2}}
    \fi
    \expandafter
  \endgroup
  \childdoctmp
}
%    \end{macrocode}

%\iffalse
%</package>
%\fi
%
\endinput

\childdocforward{cdocsamp}
%    \end{macrocode}

%\iffalse
%</sampledraft>
%\fi
%
% %%%%%%%%%%%%%%%%%%%%%%%%%%%%%%%%%%%%%%
% \paragraph{Forwarding for Final Version of the Chapters.}
%
% The following forwarding files |cdocsfn1.tex| and |cdocsfn2.tex|
% (with identical content)
% compile the final versions of the child documents
% |cdocsch1.tex| and |cdocsch2.tex|, respectively:
%\iffalse
%<*samplefinal>
%\fi
%    \begin{macrocode}
\def\version{final}
% \iffalse
%
% childdoc.dtx Copyright (C) 2017-2018 Niklas Beisert
%
% This work may be distributed and/or modified under the
% conditions of the LaTeX Project Public License, either version 1.3
% of this license or (at your option) any later version.
% The latest version of this license is in
%   http://www.latex-project.org/lppl.txt
% and version 1.3 or later is part of all distributions of LaTeX
% version 2005/12/01 or later.
%
% This work has the LPPL maintenance status `maintained'.
%
% The Current Maintainer of this work is Niklas Beisert.
%
% This work consists of the files childdoc.dtx and childdoc.ins
% and the derived files childdoc.def and cdocsamp.tex with
% cdocsch1.tex, cdocsch2.tex, cdocsdrf.tex, cdocsfn1.tex, cdocsfn2.tex.
%
%<package>\ifdefined\childdocmain\endinput\fi
%<package>\ProvidesFile{childdoc.def}[2018/12/30 v2.0 child document driver]
%<samplemain>\ProvidesFile{cdocsamp.tex}[2018/12/30 v2.0 sample for childdoc]
%<*driver>
%\ProvidesFile{childdoc.drv}[2018/12/30 v2.0 childdoc reference manual file]
\PassOptionsToClass{10pt,a4paper}{article}
\documentclass{ltxdoc}

\usepackage[margin=35mm]{geometry}
\usepackage{hyperref}
\usepackage{hyperxmp}
\usepackage[usenames]{color}

\hypersetup{colorlinks=true}
\hypersetup{pdfstartview=FitH}
\hypersetup{pdfpagemode=UseNone}
\hypersetup{pdfsource={}}
\hypersetup{pdflang={en-UK}}
\hypersetup{pdfcopyright={Copyright 2017-2018 Niklas Beisert.
  This work may be distributed and/or modified under the
  conditions of the LaTeX Project Public License, either version 1.3
  of this license or (at your option) any later version.}}
\hypersetup{pdflicenseurl={http://www.latex-project.org/lppl.txt}}
\hypersetup{pdfcontactaddress={ETH Zurich, ITP, HIT K,
  Wolfgang-Pauli-Strasse 27}}
\hypersetup{pdfcontactpostcode={8093}}
\hypersetup{pdfcontactcity={Zurich}}
\hypersetup{pdfcontactcountry={Switzerland}}
\hypersetup{pdfcontactemail={nbeisert@itp.phys.ethz.ch}}
\hypersetup{pdfcontacturl={http://people.phys.ethz.ch/\xmptilde nbeisert/}}

\newcommand{\secref}[1]{\hyperref[#1]{section \ref*{#1}}}

\parskip1ex
\parindent0pt
\let\olditemize\itemize
\def\itemize{\olditemize\parskip0pt}

\begin{document}

\title{The \textsf{childdoc} Package}
\hypersetup{pdftitle={The childdoc Package}}
\author{Niklas Beisert\\[2ex]
  Institut f\"ur Theoretische Physik\\
  Eidgen\"ossische Technische Hochschule Z\"urich\\
  Wolfgang-Pauli-Strasse 27, 8093 Z\"urich, Switzerland\\[1ex]
  \href{mailto:nbeisert@itp.phys.ethz.ch}
  {\texttt{nbeisert@itp.phys.ethz.ch}}}
\hypersetup{pdfauthor={Niklas Beisert}}
\hypersetup{pdfsubject={Manual for the LaTeX2e Package childdoc}}
\date{30 December 2018, \textsf{v2.0}}
\maketitle

\begin{abstract}\noindent
\textsf{childdoc} is a \LaTeXe{} package
that enables the direct compilation
of document sections included by |\include|
to individual files.
\end{abstract}

\begingroup
\parskip0ex
\tableofcontents
\endgroup

%%%%%%%%%%%%%%%%%%%%%%%%%%%%%%%%%%%%%%%%%%%%%%%%%%%%%%%%%%%%%%%%%%%%%%%%%%%%%%%%
%%%%%%%%%%%%%%%%%%%%%%%%%%%%%%%%%%%%%%%%%%%%%%%%%%%%%%%%%%%%%%%%%%%%%%%%%%%%%%%%
\section{Introduction}

\LaTeX{} provides a mechanism to structure a large document (such as a book)
into a main file and several child files (containing the chapters)
using the |\include| command.
This mechanism is beneficial for documents
which span hundreds of pages in order to
make the source file(s) more manageable.
Moreover, compilation can be restricted to
selected child files by means of the |\includeonly| command.
The latter feature can be used to reduce the compilation time while editing
(this was significantly more useful in the earlier days of \LaTeX{})
or to generate a smaller document which is easier to navigate.
Another application of |\includeonly| is to generate
documents consisting of selected parts of the complete document.

However, there are a few drawbacks of the plain |\include| mechanism:
\begin{itemize}
\item
The child files cannot be compiled on their own,
they can only be compiled via the main file.
A naive editing environment
(such as a text editor with an option
to have the current file processed by \LaTeX)
may require one to switch to the main file before compiling;
attempting to compile the child file produces errors.
\item
The main file must be modified (each time)
to adjust the |\includeonly| command
to the present needs. This easily leaves the main file in a messy state.
\item
The generated document will always carry the filename
of the main document. This is inconvenient if
several child files are to be compiled and
to be kept for distribution.
\end{itemize}

The present package provides a simple interface
to make child files individually compilable by \LaTeX{}.
Compiling a child file then has the same effect as compiling
the main file with an |\includeonly| command
to select the appropriate child.
Moreover the generated document will carry the name of the child
rather than the main file.
This resolves all three above issues.

This feature is meant to make the editing of books,
thesis documents and lecture notes somewhat more convenient.
However, the package can also be used efficiently for
composing a series of documents (such as exercise sheets)
which are typically distributed individually.
It then assists the author in generating the individual documents
(potentially in different versions)
as well as a document containing the collected series.
Another application is in developing style files
or other kinds of included material
where compilation of the style file could redirect
to a sample or test file.

%%%%%%%%%%%%%%%%%%%%%%%%%%%%%%%%%%%%%%%%%%%%%%%%%%%%%%%%%%%%%%%%%%%%%%%%%%%%%%%%
%%%%%%%%%%%%%%%%%%%%%%%%%%%%%%%%%%%%%%%%%%%%%%%%%%%%%%%%%%%%%%%%%%%%%%%%%%%%%%%%
\section{Usage}

First of all, the package \textsf{childdoc} is \emph{not} a standard
\LaTeXe{} |.sty| style file! Therefore it needs to be invoked in
a non-standard way.

%%%%%%%%%%%%%%%%%%%%%%%%%%%%%%%%%%%%%%%%%%%%%%%%%%%%%%%%%%%%%%%%%%%%%%%%%%%%%%%%
\subsection{Included Files}
\label{sec:include}

%%%%%%%%%%%%%%%%%%%%%%%%%%%%%%%%%%%%%%%%
\DescribeMacro{\childdocmain}
To use the package, add the commands
\begin{center}
\begin{tabular}{l}
|\input{childdoc.def}|\\
|\childdocmain{}|\\
\end{tabular}
\end{center}
at the very top of the main \LaTeX{} file,
in particular \emph{before} the |\documentclass| statement!
The argument of |\childdocmain| should be left empty
(but it must be present).

%%%%%%%%%%%%%%%%%%%%%%%%%%%%%%%%%%%%%%%%
\DescribeMacro{\childdocof}
Furthermore, add the commands
\begin{center}
\begin{tabular}{l}
|\input{childdoc.def}|\\
|\childdocof{|\textit{main}|}|\\
\end{tabular}
\end{center}
at the top of every child file \textit{child}
which is included by |\include{|\textit{child}|}|
from within the main file
(or at least for those files to be compiled individually).
The argument \textit{main} must be the filename of the main file.

There are a couple of
considerations in setting up the main and child documents:

%%%%%%%%%%%%%%%%%%%%%%%%%%%%%%%%%%%%%%%%
\paragraph{Restrictions.}

Please note the following restrictions:
\begin{itemize}
\item
|\childdocmain| must be called with one argument \textit{main}
to ensure compatibility with earlier version of the package.
It must either be empty (|\childdocmain{}|)
or precisely match the filename of the main file in which it is specified.
See \secref{sec:detection} for further information.
\item
The filename \textit{main} must be specified without the |.tex| extension.
\item
The filename \textit{main} is case sensitive
(even in case-insensitive file systems)
due to internal string comparison.
\item
The argument \textit{main} should be fully expanded, it cannot be a macro.
\item
Subdirectories and special characters should be avoided in filenames.
\item
The command |\childdocmain{|\textit{main}|}| must be followed by a whitespace.
It should not be followed immediately by another command
or by a comment mark `|%|'.
This is because the \TeX{} parser reads the token immediately following
the argument of |\childdocmain| and puts it
at the beginning of every child section;
however, a white\-space is ignored.
\end{itemize}

%%%%%%%%%%%%%%%%%%%%%%%%%%%%%%%%%%%%%%%%
\paragraph{Content of Main File.}

It is advisable to place all content in the child files included by |\include|.
Any output contained in the main file will appear in all child documents
unless suppressed manually;
it cannot be suppressed automatically by the |\includeonly| directive
and thus should normally be avoided.
A method to include some content in the main file
by means of conditional processing is described in \secref{sec:conditional}.

%%%%%%%%%%%%%%%%%%%%%%%%%%%%%%%%%%%%%%%%
\paragraph{Page Numbering.}

When only a part of the document is compiled,
the appropriate numbering of pages
(as well as other status parameters)
is determined from the |.aux| files.
The latter contain information from previous passes.
However this information needs to propagate through
all intermediate child documents.
Therefore the page numbering in child documents may well
be inconsistent until the complete document is compiled at least once.

A useful (if unconventional) way to always ensure a consistent
page numbering is to restart the numbering in each child document
and denote the pages by `\textit{child}|.|\textit{page}'
where \textit{child} represents the chapter/section number of the child file.
This can be achieved by the command
|\numberwithin{page}{|\textit{child}|}|
of the \textsf{amsmath} package
where \textit{child} can be |chapter| or |section|
depending on the chosen structuring.
Alternatively, one can modify the macro |\thepage| appropriately
and reset the counter |page| at the start of each child file.

%%%%%%%%%%%%%%%%%%%%%%%%%%%%%%%%%%%%%%%%%%%%%%%%%%%%%%%%%%%%%%%%%%%%%%%%%%%%%%%%
\subsection{Conditional Processing}
\label{sec:conditional}

The package provides a mechanism to compile different versions
of a document. To customise the versions further some conditional processing
can come in handy to distinguish which version is being compiled.
The package provides two macros to describe the compilation context:

%%%%%%%%%%%%%%%%%%%%%%%%%%%%%%%%%%%%%%%%
\DescribeMacro{\ifchilddoc}
The conditional |\ifchilddoc| distinguishes between the compilation of
child documents and the main document:
%
\begin{center}
|\ifchilddoc |\textit{child-code}| |[|\||else |\textit{main-code}]| \||fi|
\end{center}

%%%%%%%%%%%%%%%%%%%%%%%%%%%%%%%%%%%%%%%%
\DescribeMacro{\childdocname}
\DescribeMacro{\childdocjob}
The macro |\childdocname| contains the filename (without extension)
of the main or child file being processed.
Note that |\childdocjob| will always contain the name of the main file.

%%%%%%%%%%%%%%%%%%%%%%%%%%%%%%%%%%%%%%%%
\paragraph{Title Page.}

Conditional processing can be used to include a title or banner page
in the main document when proper precautions are taken.
Importantly, the code in the main file should ensure that the page counter
(as well as other status parameters which are stored in the |.aux| files)
takes the same value after the conditional processing.
Otherwise the page numbers may take divergent values
depending on which part is compiled.

For example, a title page could be declared by:
%
\begin{center}
\begin{tabular}{l}
|\ifchilddoc\||else|\\
|\addtocounter{page}{-1}|\\
\textit{code for title page}\\
|\newpage|\\
|\||fi|
\end{tabular}
\end{center}
%
A banner page for the child documents can be generated by:
%
\begin{center}
\begin{tabular}{l}
|\ifchilddoc|\\
|\addtocounter{page}{-1}|\\
\textit{code for banner page}\\
|\newpage|\\
|\||fi|
\end{tabular}
\end{center}
%
Here one could write a message such as:
\begin{center}
|This is the part \childdocname{} of \childdocjob{}.|
\end{center}

%%%%%%%%%%%%%%%%%%%%%%%%%%%%%%%%%%%%%%%%%%%%%%%%%%%%%%%%%%%%%%%%%%%%%%%%%%%%%%%%
\subsection{Flags}
\label{sec:flags}

The package makes it easy to generate different versions
of the main or child documents.
To this end compilation flags can be defined
and assigned different default values.
They will be particularly useful in conjunction
with the forwarding mechanism described in \secref{sec:forward}.

For example, it may be useful to have a flag |\version|
which can be set to |draft| or |final|.
The document source will contain some conditional code
depending on the value of |\version|.
Suppose further, the flag should default to |final| for the main file
and to |draft| for child files
which is a natural assignment for editing the document.
This is achieved by placing the following code
in the preamble of the main document
(below the |\childdocmain| directive):
%
\begin{center}
\begin{tabular}{l}
|\ifchilddoc|\\
|\providecommand{\version}{draft}|\\
|\||else|\\
|\providecommand{\version}{final}|\\
|\||fi|
\end{tabular}
\end{center}
%
The definition by |\providecommand| makes sure
that previous definitions are not overwritten.
Further statements |\providecommand{\version}{...}|
can thus be added before the above code to override it.

For the main file, one might add a line
(between |\childdocmain| and the above block)
%
\begin{center}
|%\ifchilddoc\||else\providecommand{\version}{draft}\||fi|
\end{center}
%
which can be uncommented to produce a draft version.
Likewise one can add a line to the very top of a child file
(above the |\childdocof{|\textit{main}|}| directive)
%
\begin{center}
|%\providecommand{\version}{final}|
\end{center}
%
which can be uncommented to produce the final version of this child document.

%%%%%%%%%%%%%%%%%%%%%%%%%%%%%%%%%%%%%%%%%%%%%%%%%%%%%%%%%%%%%%%%%%%%%%%%%%%%%%%%
\subsection{Forwarding}
\label{sec:forward}

Different versions of the main or child documents
using compilation flags as described in \secref{sec:flags}
can be (permanently) stored in different files
for convenient compilation, viewing and distribution.
To this end, the package defines a command
to pass on compilation to a different file:

%%%%%%%%%%%%%%%%%%%%%%%%%%%%%%%%%%%%%%%%
\DescribeMacro{\childdocforward}
The command |\childdocforward| redirects processing to
another source file:
%
\begin{center}
\begin{tabular}{l}
|\input{childdoc.def}|\\
|\childdocforward[|\textit{main}|]{|\textit{dest}|}|\\
\end{tabular}
\end{center}
%
The argument \textit{dest} is the destination file
(without extension).
It should be the main file or one of the child files.
Note that further \textsf{childdoc} directives
such as |\childdocof| and |\childdocforward|
in the indicated file will be processed in this form.
The optional argument \textit{main}
passes on directly to the main file \textit{main}
while pretending to compile the child \textit{dest}.
This form behaves as if \textit{dest}
issues |\childdocof{|\textit{main}|}| right away,
and no further \textsf{childdoc} directives will be processed.

%%%%%%%%%%%%%%%%%%%%%%%%%%%%%%%%%%%%%%%%
\DescribeMacro{\...prefix}
In the alternative form |\childdocforwardprefix|,
%
\begin{center}
\begin{tabular}{l}
|\input{childdoc.def}|\\
|\childdocforwardprefix[|\textit{main}|]{|\textit{prefix}|}{|\textit{dest}|}|
\end{tabular}
\end{center}
%
the destination file is determined by a pattern
depending on the current file:
To make this work, the current file must be called
`{\textit{prefix}\hspace{0.2em}\textit{suffix}}'
with \textit{prefix} matching precisely the argument.
Processing is then passed on to the file
`{\textit{dest}\hspace{0.2em}\textit{suffix}}'.
Surely, the same effect is achieved by
directly specifying the
argument `{\textit{dest}\hspace{0.2em}\textit{suffix}}'
in the first form.
However, that requires to set up a different file
for each child. With the alternative form of the command
all these files can have exactly the same content
which simplifies setting them up and maintaining them.

For example, the following file |draft.tex|
with a compilation flag |\version| as described in \secref{sec:flags}
compiles the main document as a draft:
%
\begin{center}
\begin{tabular}{l}
|\def\version{draft}|\\
|\input{childdoc.def}|\\
|\childdocforward{|\textit{main}|}|
\end{tabular}
\end{center}
%
Likewise, the following files |final|\textit{nn}|.tex|
compile the final version of the child document
|child|\textit{nn}|.tex|:
%
\begin{center}
\begin{tabular}{l}
|\def\version{final}|\\
|\input{childdoc.def}|\\
|\childdocforwardprefix{final}{child}|
\end{tabular}
\end{center}
%

Note that when several versions of a main file and/or of each child file
are to be generated, it may be convenient to set up a |Makefile| or
shell script to automatise the process.

%%%%%%%%%%%%%%%%%%%%%%%%%%%%%%%%%%%%%%%%%%%%%%%%%%%%%%%%%%%%%%%%%%%%%%%%%%%%%%%%
\subsection{Command Line Processing}
\label{sec:commandline}

The effect of redirection files can also be achieved by invoking
the \LaTeX{} compiler with a more elaborate command line.
Most conveniently this should be done as part
of a shell script or a |Makefile|.

When using \textsf{childdoc} in the main file, the following
command lines effectively perform a redirection
(note that depending on the shell being used,
backslashes may have to be doubled: `|\|' $\to$ `|\\|'):
%
\begin{center}
|... -jobname "|\textit{target}|" |\\|"|[\textit{flags}]%
|\input{childdoc.def}\childdocforward[|\textit{main}|]{|\textit{dest}|}"|
\end{center}
%
Here \textit{target} is the name of the output file,
\textit{main} is the name of the main file
and \textit{dest} is the name of the main or child file to be processed
(all filenames without extensions).
The optional argument \textit{main} can be omitted
if \textit{main} matches \textit{dest}.
Optionally, compilation \textit{flags} can be defined via |\def| commands.
This command line makes the \TeX{} engine believe
it is compiling the file \textit{target}
whose content is specified as the latter parameter.
The provided code then forwards the processing to
\textit{main} or \textit{dest} as described in \secref{sec:forward}.

%%%%%%%%%%%%%%%%%%%%%%%%%%%%%%%%%%%%%%%%%%%%%%%%%%%%%%%%%%%%%%%%%%%%%%%%%%%%%%%%
\subsection{Include by Input}
\label{sec:input}

Including child documents by |\include| has some restrictions by design.
Most notably, the content of a child document always occupies
its own set of pages; pages cannot be shared between child documents.
Usually, this behaviour makes perfect sense
because each child document contain an essential part of the document.
However, in some situations it may be desirable to compose
a document from a collection of parts
without having mandatory page breaks between then.
For this case, the package
provides a mechanism to include parts
by |\input| which can also be processed individually.
However, by construction this mechanism
requires manual handling of the content to be output.

%%%%%%%%%%%%%%%%%%%%%%%%%%%%%%%%%%%%%%%%
\DescribeMacro{\ifchilddocmanual}
The main file should be prepared as usual, see \secref{sec:include}.
However, the document body must make a distinction
between processing of an individual part and of the main document, e.g.:
%
\begin{center}
\begin{tabular}{l}
|\ifchilddocmanual|\\
|\input{\childdocname}|\\
|\||else|\\
\textit{document body with }|\input{|\textit{part}|}|\\
|\||fi|
\end{tabular}
\end{center}
%
The conditional |\ifchilddocmanual| is true whenever
a part to be included by |\input| is being compiled,
and the name of the part is stored in |\childdocname|.

%%%%%%%%%%%%%%%%%%%%%%%%%%%%%%%%%%%%%%%%
\DescribeMacro{\childdocby}
Each part to be included by |\input| should start with:
%
\begin{center}
\begin{tabular}{l}
|\input{childdoc.def}|\\
|\childdocby{|\textit{main}|}|\\
\end{tabular}
\end{center}
%
The directive |\childdocby| is similar to |\childdocof|
described in \secref{sec:include},
but the subsequent selection of content must be done manually.
To that end, both |\ifchilddoc| and |\ifchilddocmanual|
will be true upon processing of a part,
and the name of the part is stored in |\childdocname|.
Note that |\jobname| will be set to the filename of the current part
so that each part receives an individual |.aux| file
that does not interfere with the |.aux| file(s) of the main document.
This behaviour can be altered by the alternative form
|\childdocby[*]{|\textit{main}|}| (with a non-empty optional argument)
which uses the |.aux| file of the main document
by setting |\jobname| to \textit{main}.

%%%%%%%%%%%%%%%%%%%%%%%%%%%%%%%%%%%%%%%%%%%%%%%%%%%%%%%%%%%%%%%%%%%%%%%%%%%%%%%%
\subsection{Driver Development}
\label{sec:driver}

The \textsf{childdoc} mechanism can also be use for the development
of definition files such as \LaTeX{} styles or classes.
This case differs from the above setup with multiple parts
included by |\include| in that no |\includeonly| should be invoked.
This can be achieved by starting the include file
(before |\ProvidesPackage|) with:
%
\begin{center}
\begin{tabular}{l}
|\input{childdoc.def}|\\
|\childdocforward{|\textit{main}|}|\\
\end{tabular}
\end{center}
%
or alternatively with:
%
\begin{center}
\begin{tabular}{l}
|\input{childdoc.def}|\\
|\childdocby{|\textit{main}|}|\\
\end{tabular}
\end{center}
%
Both forms have slightly different effects as described above.
The main file is prepared as usual, see \secref{sec:include}.

%%%%%%%%%%%%%%%%%%%%%%%%%%%%%%%%%%%%%%%%%%%%%%%%%%%%%%%%%%%%%%%%%%%%%%%%%%%%%%%%
\subsection{Legacy Detection}
\label{sec:detection}

The directive |\childdocmain| in the main file can detect
whether the complete document or merely a child is to be compiled
even without using the directive |\childdocof|.
This method is deprecated because it is less robust
and there is no compelling reason to use it;
it is merely provided for backward compatibility
and it may be removed in future versions.

If the detection mechanism is to be used,
it is mandatory to correctly specify
the filename of the main file as the argument of |\childdocmain|:
%
\begin{center}
\begin{tabular}{l}
|\input{childdoc.def}|\\
|\childdocmain{|\textit{main}|}|\\
\end{tabular}
\end{center}
%
If |\jobname| does not match the argument \textit{main} of |\childdocmain|,
it is assumed that |\jobname| points to the child file to be compiled.
When using |\childdocmain| with the main file specified as argument,
it suffices to start a child file
with just |\input{|\textit{main}|}|
without loading of the package and using |\childdocof|.
If instead all processing is done
with the appropriate \textsf{childdoc} directives,
the argument of \textit{main} of |\childdocmain| can be empty.

An alternative version of the command line processing described
in \secref{sec:commandline} using the detection mechanism reads:
%
\begin{center}
|... -jobname "|\textit{target}|" "|[\textit{flags}]%
[|\def\jobname{|\textit{dest}|}|]|\input{|\textit{main}|}"|
\end{center}

%%%%%%%%%%%%%%%%%%%%%%%%%%%%%%%%%%%%%%%%%%%%%%%%%%%%%%%%%%%%%%%%%%%%%%%%%%%%%%%%
\subsection{Manual Code}
\label{sec:manual}

In case one cannot be certain whether the definitions file |childdoc.def|
is installed on the target \TeX{} distribution
and one prefers not to ship it,
it is conceivable to paste a few relevant commands into the sources.

To that end, drop all statements |\input{childdoc.def}|
and perform the replacements as outlined below.
Instead of |\childdocmain{|\textit{main}|}| add the following code
to the top of the main file:
%
\begin{center}
\begin{tabular}{l}
|\||ifdefined\childdocname\endinput\||fi\newif\ifchilddoc|\\
|\edef\childdocname{\scantokens\expandafter{\jobname\noexpand}}|\\
|\def\childdocmain{|\textit{main}|}\||ifx\childdocmain\childdocname\||else|\\
|\childdoctrue\includeonly{\childdocname}\let\jobname\childdocmain\||fi|\\
\end{tabular}
\end{center}
%
Instead of |\childdocof{|\textit{main}|}| just include the main file
at the top of each child file:
%
\begin{center}
|\input{|\textit{main}|}|
\end{center}
%
A simple redirection |\childdocforward{|\textit{dest}|}| is achieved by:
%
\begin{center}
|\def\jobname{|\textit{dest}|}\input{\jobname}|
\end{center}
%
The redirection with prefix
|\childdocforwardprefix[|\textit{prefix}|]{|\textit{dest}|}|
is accomplished by:
%
\begin{center}
\begin{tabular}{l}
|{\edef\jobname{\scantokens\expandafter{\jobname\noexpand}}|\\
|\def\redirectjob |\textit{prefix}|#1~~~{\gdef\jobname{|\textit{dest}|#1}}|\\
|\expandafter\redirectjob\jobname~~~}\input{\jobname}|
\end{tabular}
\end{center}

In an alternative approach,
child documents can be compiled by a specific command line
without additional code or specific definitions:
%
\begin{center}
|... -jobname "|\textit{target}|" "|[\textit{flags}]%
|\includeonly{|\textit{dest}|}\input{|\textit{main}|}"|
\end{center}
%

%%%%%%%%%%%%%%%%%%%%%%%%%%%%%%%%%%%%%%%%%%%%%%%%%%%%%%%%%%%%%%%%%%%%%%%%%%%%%%%%
%%%%%%%%%%%%%%%%%%%%%%%%%%%%%%%%%%%%%%%%%%%%%%%%%%%%%%%%%%%%%%%%%%%%%%%%%%%%%%%%
\section{Information}

%%%%%%%%%%%%%%%%%%%%%%%%%%%%%%%%%%%%%%%%%%%%%%%%%%%%%%%%%%%%%%%%%%%%%%%%%%%%%%%%
\subsection{Copyright}

Copyright \copyright{} 2017--2018 Niklas Beisert

This work may be distributed and/or modified under the
conditions of the \LaTeX{} Project Public License, either version 1.3
of this license or (at your option) any later version.
The latest version of this license is in
  \url{http://www.latex-project.org/lppl.txt}
and version 1.3 or later is part of all distributions of \LaTeX{}
version 2005/12/01 or later.

This work has the LPPL maintenance status `maintained'.

The Current Maintainer of this work is Niklas Beisert.

This work consists of the files |README.txt|, |childdoc.ins| and |childdoc.dtx|
as well as the derived files |childdoc.def|, |cdocsamp.tex|
with |cdocsch1.tex|, |cdocsch2.tex|, |cdocspt3.tex|, |cdocspt4.tex|,
|cdocsdrf.tex|, |cdocsfn1.tex|, |cdocsfn2.tex|
as well as |childdoc.pdf|.

%%%%%%%%%%%%%%%%%%%%%%%%%%%%%%%%%%%%%%%%%%%%%%%%%%%%%%%%%%%%%%%%%%%%%%%%%%%%%%%%
\subsection{Files and Installation}

The package consists of the files:
%
\begin{center}
\begin{tabular}{ll}
    |README.txt|   & readme file \\
    |childdoc.ins| & installation file \\
    |childdoc.dtx| & source file \\
    |childdoc.def| & definition file \\
    |cdocsamp.tex| & sample main file \\
    |cdocsch1.tex| & sample include file \\
    |cdocsch2.tex| & sample include file \\
    |cdocspt3.tex| & sample part file \\
    |cdocspt4.tex| & sample part file \\
    |cdocsdrf.tex| & sample redirection file \\
    |cdocsfn1.tex| & sample redirection file \\
    |cdocsfn2.tex| & sample redirection file \\
    |childdoc.pdf| & manual
\end{tabular}
\end{center}
%
The distribution consists of the files
|README.txt|, |childdoc.ins| and |childdoc.dtx|.
%
\begin{itemize}
\item
Run (pdf)\LaTeX{} on |childdoc.dtx|
to compile the manual |childdoc.pdf| (this file).
\item
Run \LaTeX{} on |childdoc.ins| to create the definitions file |childdoc.def|
and the sample |cdocsamp.tex| with include files
|cdocsch1.tex|, |cdocsch2.tex|, |cdocspt3.tex|, |cdocspt4.tex|,
|cdocsdrf.tex|, |cdocsfn1.tex|, |cdocsfn2.tex|.
Then copy the file |childdoc.def| to an appropriate directory of your \LaTeX{}
distribution, e.g.\ \textit{texmf-root}|/tex/latex/childdoc|.
\end{itemize}

%%%%%%%%%%%%%%%%%%%%%%%%%%%%%%%%%%%%%%%%%%%%%%%%%%%%%%%%%%%%%%%%%%%%%%%%%%%%%%%%
\subsection{Related CTAN Packages}

There are several other packages which offer a similar functionality:
%
\begin{itemize}
\item
The packages
\href{http://ctan.org/pkg/docmute}{\textsf{docmute}},
\href{http://ctan.org/pkg/includex}{\textsf{includex}} and
\href{http://ctan.org/pkg/standalone}{\textsf{standalone}}
provide commands to include only the document body of
a child file thus allowing both files to be compiled individually.
\item
The packages \href{http://ctan.org/pkg/subdocs}{\textsf{subdocs}}
and \href{http://ctan.org/pkg/subfiles}{\textsf{subfiles}}
provide structures in which the main and child documents can be
encapsulated and allowing them to be compiled individually.
The inclusion mechanism is different from the conventional |\include|.
\item
The package \href{http://ctan.org/pkg/combine}{\textsf{combine}}
is an elaborate solution to combine several documents into one.
\end{itemize}
%
See also the CTAN topic \href{http://ctan.org/topic/subdocs}{\textsf{subdocs}}
for further related packages.
The present package differs from the above solutions in that
a document structure constructed with the conventional |\include| mechanism
just needs two extra commands at the top of every file
such that all constituent files can be compiled individually.

%%%%%%%%%%%%%%%%%%%%%%%%%%%%%%%%%%%%%%%%%%%%%%%%%%%%%%%%%%%%%%%%%%%%%%%%%%%%%%%%
%\subsection{Feature Suggestions}
%
%The following is a list of features which may be useful for future
%versions of this package:
%%
%\begin{itemize}
%\item
%\ldots
%\end{itemize}

%%%%%%%%%%%%%%%%%%%%%%%%%%%%%%%%%%%%%%%%%%%%%%%%%%%%%%%%%%%%%%%%%%%%%%%%%%%%%%%%
\subsection{Revision History}

%%%%%%%%%%%%%%%%%%%%%%%%%%%%%%%%%%%%%%%%
\paragraph{v2.0:} 2018/12/30

\begin{itemize}
\item
immediate forward processing
\item
added |\childdocby| mechanism
\item
manual restructured
\end{itemize}

%%%%%%%%%%%%%%%%%%%%%%%%%%%%%%%%%%%%%%%%
\paragraph{v1.6:} 2018/01/17

\begin{itemize}
\item
application for development of include files
\item
corrections to manual
\end{itemize}

%%%%%%%%%%%%%%%%%%%%%%%%%%%%%%%%%%%%%%%%
\paragraph{v1.5:} 2017/05/21

\begin{itemize}
\item
more complete structuring introduced
\item
|\childdocof| introduced
\item
|\childdoc| renamed to |\childdocmain|
\item
|\childredirect| renamed to |\childdocforward| and |\childdocforwardprefix|
and functionality expanded
\end{itemize}

%%%%%%%%%%%%%%%%%%%%%%%%%%%%%%%%%%%%%%%%
\paragraph{v1.0:} 2017/04/27

\begin{itemize}
\item
manual and install package
\item
first version published on CTAN
\end{itemize}

%%%%%%%%%%%%%%%%%%%%%%%%%%%%%%%%%%%%%%%%
\paragraph{v0.6:} 2017/04/26

\begin{itemize}
\item
redirection mechanism added
\end{itemize}

%%%%%%%%%%%%%%%%%%%%%%%%%%%%%%%%%%%%%%%%
\paragraph{v0.5:} 2017/04/26

\begin{itemize}
\item
functionality in definition file
\end{itemize}


%%%%%%%%%%%%%%%%%%%%%%%%%%%%%%%%%%%%%%%%%%%%%%%%%%%%%%%%%%%%%%%%%%%%%%%%%%%%%%%%
%%%%%%%%%%%%%%%%%%%%%%%%%%%%%%%%%%%%%%%%%%%%%%%%%%%%%%%%%%%%%%%%%%%%%%%%%%%%%%%%
%%%%%%%%%%%%%%%%%%%%%%%%%%%%%%%%%%%%%%%%%%%%%%%%%%%%%%%%%%%%%%%%%%%%%%%%%%%%%%%%
\appendix

\settowidth\MacroIndent{\rmfamily\scriptsize 000\ }

 \DocInput{childdoc.dtx}

\end{document}
%</driver>
% \fi
%
% %%%%%%%%%%%%%%%%%%%%%%%%%%%%%%%%%%%%%%%%%%%%%%%%%%%%%%%%%%%%%%%%%%%%%%%%%%%%%%
% %%%%%%%%%%%%%%%%%%%%%%%%%%%%%%%%%%%%%%%%%%%%%%%%%%%%%%%%%%%%%%%%%%%%%%%%%%%%%%
% \section{Sample}
%\iffalse
%<*samplemain>
%\fi
%
% The following presents a sample document
% with two chapters, two parts, a title page,
% a compile flag as well as three forwarding files to set the flag.
% It consists of eight |.tex| files:
% \begin{center}
% \begin{tabular}{ll}
% |cdocsamp.tex|&main file\\
% |cdocsch1.tex|&include file for chapter 1\\
% |cdocsch2.tex|&include file for chapter 2\\
% |cdocspt3.tex|&include file for part 3\\
% |cdocspt4.tex|&include file for part 4\\
% |cdocsdrf.tex|&forwarding file for main file in draft mode\\
% |cdocsfi1.tex|&forwarding file for final version of chapter 1\\
% |cdocsfi2.tex|&forwarding file for final version of chapter 2\\
% \end{tabular}
% \end{center}
% Each of the eight files can be compiled directly by the \LaTeX{} compiler.
%
% %%%%%%%%%%%%%%%%%%%%%%%%%%%%%%%%%%%%%%
% \paragraph{Main File.}
%
% The main file is called |cdocsamp.tex|.
%
% Load the \textsf{childdoc} definitions and
% declare the filename for the main document:
%    \begin{macrocode}
\input{childdoc.def}
\childdocmain{}
%    \end{macrocode}

% Optional override for |\version| flag:
%    \begin{macrocode}
%%\ifchilddoc\else\providecommand{\version}{draft}\fi
%    \end{macrocode}

% Define the default values for the |\version| flag
% (|final| for the main file and |draft| for childs):
%    \begin{macrocode}
\ifchilddoc
\providecommand{\version}{draft}
\else
\providecommand{\version}{final}
\fi
%    \end{macrocode}

% Load the standard document class:
%    \begin{macrocode}
\documentclass[12pt]{article}
%    \end{macrocode}

% Start the document body:
%    \begin{macrocode}
\begin{document}
%    \end{macrocode}

% Declare a title page.
% Print title, part of document being processed and version flag:
%    \begin{macrocode}
\addtocounter{page}{-1}
\begin{center}
{\LARGE\bfseries{}childdoc example\par}
\vspace{1cm}
\ifchilddoc
\ifchilddocmanual part\else chapter\fi:
`\childdocname' of `\childdocjob'\par
\else
main document: `\childdocjob'\par
\fi
version: \version\par
\end{center}
\newpage
%    \end{macrocode}

% Manually include selected file,
% otherwise process as usual:
%    \begin{macrocode}
\ifchilddocmanual
\section*{part `\childdocname'}
\input{\childdocname}
\else
%    \end{macrocode}

% Include the two chapters:
%    \begin{macrocode}
\include{cdocsch1}
\include{cdocsch2}
%    \end{macrocode}

% Include the two parts unless only chapters should be displayed:
%    \begin{macrocode}
\ifchilddoc\else
\section{part three}
\input{cdocspt3}
\section{part four}
\input{cdocspt4}
\fi
%    \end{macrocode}

% Process as usual until here:
%    \begin{macrocode}
\fi
%    \end{macrocode}

% End of document body:
%    \begin{macrocode}
\end{document}
%    \end{macrocode}
%\iffalse
%</samplemain>
%\fi
%
% %%%%%%%%%%%%%%%%%%%%%%%%%%%%%%%%%%%%%%
% \paragraph{Chapter Include Files.}
%
% The include files are called |cdocsch1.tex| and |cdocsch2.tex|.
%
%\iffalse
%<*samplechap1|samplechap2>
%\fi

% Optional override for |\version| flag:
%    \begin{macrocode}
%%\providecommand{\version}{final}
%    \end{macrocode}

% Include the main document:
%    \begin{macrocode}
\input{childdoc.def}
\childdocof{cdocsamp}
%    \end{macrocode}

%\iffalse
%</samplechap1|samplechap2>
%\fi
%
%\iffalse
%<*samplechap1>
%\fi
% Some text for chapter 1:
%    \begin{macrocode}
\section{one}
some text in chapter one
%    \end{macrocode}

%\iffalse
%</samplechap1>
%\fi
% Some text for chapter 2:
%\iffalse
%<*samplechap2>
%\fi
%    \begin{macrocode}
\section{two}
more text in chapter two
%    \end{macrocode}

%\iffalse
%</samplechap2>
%\fi
%
% %%%%%%%%%%%%%%%%%%%%%%%%%%%%%%%%%%%%%%
% \paragraph{Part Include Files.}
%
% The include files are called |cdocspt3.tex| and |cdocspt4.tex|.
%
%\iffalse
%<*samplepart3|samplepart4>
%\fi

% Optional override for |\version| flag:
%    \begin{macrocode}
%%\providecommand{\version}{final}
%    \end{macrocode}

% Include the main document:
%    \begin{macrocode}
\input{childdoc.def}
\childdocby{cdocsamp}
%    \end{macrocode}

%\iffalse
%</samplepart3|samplepart4>
%\fi
%
%\iffalse
%<*samplepart3>
%\fi
% Some text for part 3:
%    \begin{macrocode}
some text in part three
%    \end{macrocode}

%\iffalse
%</samplepart3>
%\fi
% Some text for part 4:
%\iffalse
%<*samplepart4>
%\fi
%    \begin{macrocode}
more text in part four
%    \end{macrocode}

%\iffalse
%</samplepart4>
%\fi
%
% %%%%%%%%%%%%%%%%%%%%%%%%%%%%%%%%%%%%%%
% \paragraph{Forwarding for a Complete Draft.}
%
% The following forwarding file |cdocsdrf.tex|
% compiles the main document in draft mode:
%\iffalse
%<*sampledraft>
%\fi
%    \begin{macrocode}
\def\version{draft}
\input{childdoc.def}
\childdocforward{cdocsamp}
%    \end{macrocode}

%\iffalse
%</sampledraft>
%\fi
%
% %%%%%%%%%%%%%%%%%%%%%%%%%%%%%%%%%%%%%%
% \paragraph{Forwarding for Final Version of the Chapters.}
%
% The following forwarding files |cdocsfn1.tex| and |cdocsfn2.tex|
% (with identical content)
% compile the final versions of the child documents
% |cdocsch1.tex| and |cdocsch2.tex|, respectively:
%\iffalse
%<*samplefinal>
%\fi
%    \begin{macrocode}
\def\version{final}
\input{childdoc.def}
\childdocforwardprefix[cdocsamp]{cdocsfn}{cdocsch}
%    \end{macrocode}

%\iffalse
%</samplefinal>
%\fi
%
% %%%%%%%%%%%%%%%%%%%%%%%%%%%%%%%%%%%%%%
% \paragraph{Command Line Processing.}
%
% The following three command lines generate the output files
% |cdocscld|, |cdocscl1| and |cdocscl2|
% which should be identical to
% |cdocsdrf|, |cdocsch1| and |cdocsfn2|, respectively:
% \begin{center}
% \begin{tabular}{l}
% |latex -jobname cdocscld \|\\
% |  "\def\version{draft}\input{childdoc.def}\childdocforward{cdocsamp}"|\\
% |latex -jobname cdocscl1 \|\\
% |  "\input{childdoc.def}\childdocforward[cdocsamp]{cdocsch1}"|\\
% |latex -jobname cdocscl2 \|\\
% |  "\def\version{final}\input{childdoc.def}\childdocforward{cdocsch2}"|
% \end{tabular}
% \end{center}
% Note that the trailing backslash on each first line
% merely continues the input to the second line
% (for convenient cut ant paste).
% Furthermore, the command |latex| can be replaced by any
% of its alternative versions such as |pdflatex|.
%
% %%%%%%%%%%%%%%%%%%%%%%%%%%%%%%%%%%%%%%%%%%%%%%%%%%%%%%%%%%%%%%%%%%%%%%%%%%%%%%
% %%%%%%%%%%%%%%%%%%%%%%%%%%%%%%%%%%%%%%%%%%%%%%%%%%%%%%%%%%%%%%%%%%%%%%%%%%%%%%
% \section{Implementation}
%\iffalse
%<*package>
%\fi
%
% This section describes the definitions file |childdoc.def|.

% The definitions cannot be loaded using |\usepackage| or |\RequirePackage|
% which has a mechanism to prevent loading a style file more than once.
% When loading the definitions by means of |\input|
% multiple instances have to be prevented manually:
%\iffalse
%This code needs to be before the `\ProvidesFile' directive
%which is defined at the beginning of this file.
%Therefore it is also placed there and commented out here.
%</package>
%<*discard>
%\fi
%    \begin{macrocode}
\ifdefined\childdocmain\endinput\fi
%    \end{macrocode}
%\iffalse
%</discard>
%<*package>
%\fi
%
% \macro{\ifchilddoc}
% \macro{\ifchilddocmanual}
% The conditional |\ifchilddoc| tells whether a
% child (true) or main (false) document is being compiled.
% The conditional |\ifchilddocmanual| tells whether
% the |\includeonly| mechanism is used (false) or
% the selection of child files must be performed manually (true).
% The definitions initialise to false:
%    \begin{macrocode}
\newif\ifchilddoc
\newif\ifchilddocmanual
%    \end{macrocode}

% \macro{\childdocname}
% \macro{\childdocjob}
% The macro |\childdocname| stores the name of the main document
% to be compiled. The macro |\childdocjob| stores the name of
% the document on which the \LaTeX{} compiler was originally invoked.
% The content of |\jobname| cannot be compared
% to filenames specified in the source due to different catcodes.
% The following code rescans |\jobname|, stores the result
% in |\childdocname| and saves a copy in |\childdocjob|:
%    \begin{macrocode}
\edef\childdocname{\scantokens\expandafter{\jobname\noexpand}}
\let\childdocjob\childdocname
%    \end{macrocode}

% \macro{\childdocdisable}
% The macro |\childdocdisable| prevents the main file
% from being processed more than once.
% At this stage, the main document command |\childdocmain|
% is assumed to be called once again where it should do nothing.
% Any subsequent call to it should prevent
% a secondary processing of the main document
% It overwrites the forwarding commands
% |\childdocof| and |\childdocforward|
% with empty macros to prevent further inclusions of the main document:
%    \begin{macrocode}
\newcommand{\childdocdisable}
{
  \renewcommand{\childdocmain}[1]{\renewcommand{\childdocmain}[1]{\endinput}}
  \renewcommand{\childdocof}[1]{}
  \renewcommand{\childdocby}[2][]{}
  \renewcommand{\childdocforward}[2][]{}
  \renewcommand{\childdocdisable}{}
}
%    \end{macrocode}

% \macro{\childdocmain}
% The macro |\childdocmain| is to be called at the top of the main file
% with nothing or the main filename (without extension) as argument.
% First, it breaks loops.
% If the argument is not empty and does not match |\childdocname|
% (which is set by the first inclusion of |childdoc.def|),
% |\ifchilddoc| is set to true, |\includeonly| is applied to the child file
% and |\jobname| is set to the main file
% (for proper handling of |.aux| files):
%    \begin{macrocode}
\newcommand{\childdocmain}[1]
{
  \childdocdisable\childdocmain{}
  \if?#1?\else
    \begingroup
      \def\childdoctmp{#1}
      \ifx\childdoctmp\childdocname
        \def\childdoctmp{}
      \else
        \def\childdoctmp
        {
          \childdoctrue
          \includeonly{\childdocname}
          \def\childdocjob{#1}
          \def\jobname{#1}
        }
      \fi
      \expandafter
    \endgroup
    \childdoctmp
  \fi
}
%    \end{macrocode}

% \macro{\childdocof}
% The command |\childdocof| redirects
% compilation to the main file |#1|.
%    \begin{macrocode}
\newcommand{\childdocof}[1]
{
  \childdocdisable
  \childdoctrue
  \includeonly{\childdocname}
  \def\jobname{#1}
  \def\childdocjob{#1}
  \input{#1}
}
%    \end{macrocode}

% \macro{\childdocby}
% The command |\childdocby| ....
%    \begin{macrocode}
\newcommand{\childdocby}[2][]
{
  \childdocdisable
  \childdoctrue
  \childdocmanualtrue
  \if?#1?\else
    \def\jobname{#2}
  \fi
  \def\childdocjob{#2}
  \input{#2}
  \endinput
}
%    \end{macrocode}

% \macro{\childdocforward}
% The command |\childdocforward| redirects
% compilation to the main file or
% (if the optional argument is given) a child file.
% Parameters are set as if the main file
% or a child file starting with |\childdocof| was compiled.
% Then compilation is handed over to the main file:
%    \begin{macrocode}
\newcommand{\childdocforward}[2][]
{
  \begingroup
    \if?#1?
      \def\childdoctmp
      {
        \def\childdocname{#2}
        \def\childdocjob{#2}
        \def\jobname{#2}
        \input{#2}
        \endinput
      }
    \else
      \def\childdoctmp
      {
        \childdocdisable
        \def\childdocname{#2}
        \childdoctrue
        \includeonly{#2}
        \def\childdocjob{#1}
        \def\jobname{#1}
        \input{#1}
        \endinput
      }
    \fi
    \expandafter
  \endgroup
  \childdoctmp
}
%    \end{macrocode}

% \macro{\childdocforwardprefix}
% The command |\childdocforwardprefix| redirects
% compilation to the main or a child file by means of a pattern.
% The prefix |#1| in the current filename is replaced by |#2|
% and the suffix of the current filename is kept
% (it is assumed that the filename does not contain the substring `|~~~|'
% which is used as a delimiter).
% Compilation is handed over to the new file by |\childdocforward|:
%    \begin{macrocode}
\newcommand{\childdocforwardprefix}[3][]
{
  \begingroup
    \def\childdocextract #2##1~~~{\def\childdoctmp{\childdocforward[#1]{#3##1}}}
    \expandafter\childdocextract\childdocname~~~
    \expandafter
  \endgroup
  \childdoctmp
}
%    \end{macrocode}

% \macro{\childdoc}
% The deprecated macro |\childdoc| is a legacy version of |\childdocmain|:
%    \begin{macrocode}
\newcommand{\childdoc}{\childdocmain}
%    \end{macrocode}

% \macro{\childdocredirect}
% The deprecated macro |\childdocredirect| is a legacy version
% of |\childdocforward| and |\childdocforwardprefix|:
%    \begin{macrocode}
\newcommand{\childdocredirect}[2][]
{
  \begingroup
    \if?#1?
      \def\childdoctmp{\childdocforward{#2}}
    \else
      \def\childdoctmp{\childdocforwardprefix{#1}{#2}}
    \fi
    \expandafter
  \endgroup
  \childdoctmp
}
%    \end{macrocode}

%\iffalse
%</package>
%\fi
%
\endinput

\childdocforwardprefix[cdocsamp]{cdocsfn}{cdocsch}
%    \end{macrocode}

%\iffalse
%</samplefinal>
%\fi
%
% %%%%%%%%%%%%%%%%%%%%%%%%%%%%%%%%%%%%%%
% \paragraph{Command Line Processing.}
%
% The following three command lines generate the output files
% |cdocscld|, |cdocscl1| and |cdocscl2|
% which should be identical to
% |cdocsdrf|, |cdocsch1| and |cdocsfn2|, respectively:
% \begin{center}
% \begin{tabular}{l}
% |latex -jobname cdocscld \|\\
% |  "\def\version{draft}% \iffalse
%
% childdoc.dtx Copyright (C) 2017-2018 Niklas Beisert
%
% This work may be distributed and/or modified under the
% conditions of the LaTeX Project Public License, either version 1.3
% of this license or (at your option) any later version.
% The latest version of this license is in
%   http://www.latex-project.org/lppl.txt
% and version 1.3 or later is part of all distributions of LaTeX
% version 2005/12/01 or later.
%
% This work has the LPPL maintenance status `maintained'.
%
% The Current Maintainer of this work is Niklas Beisert.
%
% This work consists of the files childdoc.dtx and childdoc.ins
% and the derived files childdoc.def and cdocsamp.tex with
% cdocsch1.tex, cdocsch2.tex, cdocsdrf.tex, cdocsfn1.tex, cdocsfn2.tex.
%
%<package>\ifdefined\childdocmain\endinput\fi
%<package>\ProvidesFile{childdoc.def}[2018/12/30 v2.0 child document driver]
%<samplemain>\ProvidesFile{cdocsamp.tex}[2018/12/30 v2.0 sample for childdoc]
%<*driver>
%\ProvidesFile{childdoc.drv}[2018/12/30 v2.0 childdoc reference manual file]
\PassOptionsToClass{10pt,a4paper}{article}
\documentclass{ltxdoc}

\usepackage[margin=35mm]{geometry}
\usepackage{hyperref}
\usepackage{hyperxmp}
\usepackage[usenames]{color}

\hypersetup{colorlinks=true}
\hypersetup{pdfstartview=FitH}
\hypersetup{pdfpagemode=UseNone}
\hypersetup{pdfsource={}}
\hypersetup{pdflang={en-UK}}
\hypersetup{pdfcopyright={Copyright 2017-2018 Niklas Beisert.
  This work may be distributed and/or modified under the
  conditions of the LaTeX Project Public License, either version 1.3
  of this license or (at your option) any later version.}}
\hypersetup{pdflicenseurl={http://www.latex-project.org/lppl.txt}}
\hypersetup{pdfcontactaddress={ETH Zurich, ITP, HIT K,
  Wolfgang-Pauli-Strasse 27}}
\hypersetup{pdfcontactpostcode={8093}}
\hypersetup{pdfcontactcity={Zurich}}
\hypersetup{pdfcontactcountry={Switzerland}}
\hypersetup{pdfcontactemail={nbeisert@itp.phys.ethz.ch}}
\hypersetup{pdfcontacturl={http://people.phys.ethz.ch/\xmptilde nbeisert/}}

\newcommand{\secref}[1]{\hyperref[#1]{section \ref*{#1}}}

\parskip1ex
\parindent0pt
\let\olditemize\itemize
\def\itemize{\olditemize\parskip0pt}

\begin{document}

\title{The \textsf{childdoc} Package}
\hypersetup{pdftitle={The childdoc Package}}
\author{Niklas Beisert\\[2ex]
  Institut f\"ur Theoretische Physik\\
  Eidgen\"ossische Technische Hochschule Z\"urich\\
  Wolfgang-Pauli-Strasse 27, 8093 Z\"urich, Switzerland\\[1ex]
  \href{mailto:nbeisert@itp.phys.ethz.ch}
  {\texttt{nbeisert@itp.phys.ethz.ch}}}
\hypersetup{pdfauthor={Niklas Beisert}}
\hypersetup{pdfsubject={Manual for the LaTeX2e Package childdoc}}
\date{30 December 2018, \textsf{v2.0}}
\maketitle

\begin{abstract}\noindent
\textsf{childdoc} is a \LaTeXe{} package
that enables the direct compilation
of document sections included by |\include|
to individual files.
\end{abstract}

\begingroup
\parskip0ex
\tableofcontents
\endgroup

%%%%%%%%%%%%%%%%%%%%%%%%%%%%%%%%%%%%%%%%%%%%%%%%%%%%%%%%%%%%%%%%%%%%%%%%%%%%%%%%
%%%%%%%%%%%%%%%%%%%%%%%%%%%%%%%%%%%%%%%%%%%%%%%%%%%%%%%%%%%%%%%%%%%%%%%%%%%%%%%%
\section{Introduction}

\LaTeX{} provides a mechanism to structure a large document (such as a book)
into a main file and several child files (containing the chapters)
using the |\include| command.
This mechanism is beneficial for documents
which span hundreds of pages in order to
make the source file(s) more manageable.
Moreover, compilation can be restricted to
selected child files by means of the |\includeonly| command.
The latter feature can be used to reduce the compilation time while editing
(this was significantly more useful in the earlier days of \LaTeX{})
or to generate a smaller document which is easier to navigate.
Another application of |\includeonly| is to generate
documents consisting of selected parts of the complete document.

However, there are a few drawbacks of the plain |\include| mechanism:
\begin{itemize}
\item
The child files cannot be compiled on their own,
they can only be compiled via the main file.
A naive editing environment
(such as a text editor with an option
to have the current file processed by \LaTeX)
may require one to switch to the main file before compiling;
attempting to compile the child file produces errors.
\item
The main file must be modified (each time)
to adjust the |\includeonly| command
to the present needs. This easily leaves the main file in a messy state.
\item
The generated document will always carry the filename
of the main document. This is inconvenient if
several child files are to be compiled and
to be kept for distribution.
\end{itemize}

The present package provides a simple interface
to make child files individually compilable by \LaTeX{}.
Compiling a child file then has the same effect as compiling
the main file with an |\includeonly| command
to select the appropriate child.
Moreover the generated document will carry the name of the child
rather than the main file.
This resolves all three above issues.

This feature is meant to make the editing of books,
thesis documents and lecture notes somewhat more convenient.
However, the package can also be used efficiently for
composing a series of documents (such as exercise sheets)
which are typically distributed individually.
It then assists the author in generating the individual documents
(potentially in different versions)
as well as a document containing the collected series.
Another application is in developing style files
or other kinds of included material
where compilation of the style file could redirect
to a sample or test file.

%%%%%%%%%%%%%%%%%%%%%%%%%%%%%%%%%%%%%%%%%%%%%%%%%%%%%%%%%%%%%%%%%%%%%%%%%%%%%%%%
%%%%%%%%%%%%%%%%%%%%%%%%%%%%%%%%%%%%%%%%%%%%%%%%%%%%%%%%%%%%%%%%%%%%%%%%%%%%%%%%
\section{Usage}

First of all, the package \textsf{childdoc} is \emph{not} a standard
\LaTeXe{} |.sty| style file! Therefore it needs to be invoked in
a non-standard way.

%%%%%%%%%%%%%%%%%%%%%%%%%%%%%%%%%%%%%%%%%%%%%%%%%%%%%%%%%%%%%%%%%%%%%%%%%%%%%%%%
\subsection{Included Files}
\label{sec:include}

%%%%%%%%%%%%%%%%%%%%%%%%%%%%%%%%%%%%%%%%
\DescribeMacro{\childdocmain}
To use the package, add the commands
\begin{center}
\begin{tabular}{l}
|\input{childdoc.def}|\\
|\childdocmain{}|\\
\end{tabular}
\end{center}
at the very top of the main \LaTeX{} file,
in particular \emph{before} the |\documentclass| statement!
The argument of |\childdocmain| should be left empty
(but it must be present).

%%%%%%%%%%%%%%%%%%%%%%%%%%%%%%%%%%%%%%%%
\DescribeMacro{\childdocof}
Furthermore, add the commands
\begin{center}
\begin{tabular}{l}
|\input{childdoc.def}|\\
|\childdocof{|\textit{main}|}|\\
\end{tabular}
\end{center}
at the top of every child file \textit{child}
which is included by |\include{|\textit{child}|}|
from within the main file
(or at least for those files to be compiled individually).
The argument \textit{main} must be the filename of the main file.

There are a couple of
considerations in setting up the main and child documents:

%%%%%%%%%%%%%%%%%%%%%%%%%%%%%%%%%%%%%%%%
\paragraph{Restrictions.}

Please note the following restrictions:
\begin{itemize}
\item
|\childdocmain| must be called with one argument \textit{main}
to ensure compatibility with earlier version of the package.
It must either be empty (|\childdocmain{}|)
or precisely match the filename of the main file in which it is specified.
See \secref{sec:detection} for further information.
\item
The filename \textit{main} must be specified without the |.tex| extension.
\item
The filename \textit{main} is case sensitive
(even in case-insensitive file systems)
due to internal string comparison.
\item
The argument \textit{main} should be fully expanded, it cannot be a macro.
\item
Subdirectories and special characters should be avoided in filenames.
\item
The command |\childdocmain{|\textit{main}|}| must be followed by a whitespace.
It should not be followed immediately by another command
or by a comment mark `|%|'.
This is because the \TeX{} parser reads the token immediately following
the argument of |\childdocmain| and puts it
at the beginning of every child section;
however, a white\-space is ignored.
\end{itemize}

%%%%%%%%%%%%%%%%%%%%%%%%%%%%%%%%%%%%%%%%
\paragraph{Content of Main File.}

It is advisable to place all content in the child files included by |\include|.
Any output contained in the main file will appear in all child documents
unless suppressed manually;
it cannot be suppressed automatically by the |\includeonly| directive
and thus should normally be avoided.
A method to include some content in the main file
by means of conditional processing is described in \secref{sec:conditional}.

%%%%%%%%%%%%%%%%%%%%%%%%%%%%%%%%%%%%%%%%
\paragraph{Page Numbering.}

When only a part of the document is compiled,
the appropriate numbering of pages
(as well as other status parameters)
is determined from the |.aux| files.
The latter contain information from previous passes.
However this information needs to propagate through
all intermediate child documents.
Therefore the page numbering in child documents may well
be inconsistent until the complete document is compiled at least once.

A useful (if unconventional) way to always ensure a consistent
page numbering is to restart the numbering in each child document
and denote the pages by `\textit{child}|.|\textit{page}'
where \textit{child} represents the chapter/section number of the child file.
This can be achieved by the command
|\numberwithin{page}{|\textit{child}|}|
of the \textsf{amsmath} package
where \textit{child} can be |chapter| or |section|
depending on the chosen structuring.
Alternatively, one can modify the macro |\thepage| appropriately
and reset the counter |page| at the start of each child file.

%%%%%%%%%%%%%%%%%%%%%%%%%%%%%%%%%%%%%%%%%%%%%%%%%%%%%%%%%%%%%%%%%%%%%%%%%%%%%%%%
\subsection{Conditional Processing}
\label{sec:conditional}

The package provides a mechanism to compile different versions
of a document. To customise the versions further some conditional processing
can come in handy to distinguish which version is being compiled.
The package provides two macros to describe the compilation context:

%%%%%%%%%%%%%%%%%%%%%%%%%%%%%%%%%%%%%%%%
\DescribeMacro{\ifchilddoc}
The conditional |\ifchilddoc| distinguishes between the compilation of
child documents and the main document:
%
\begin{center}
|\ifchilddoc |\textit{child-code}| |[|\||else |\textit{main-code}]| \||fi|
\end{center}

%%%%%%%%%%%%%%%%%%%%%%%%%%%%%%%%%%%%%%%%
\DescribeMacro{\childdocname}
\DescribeMacro{\childdocjob}
The macro |\childdocname| contains the filename (without extension)
of the main or child file being processed.
Note that |\childdocjob| will always contain the name of the main file.

%%%%%%%%%%%%%%%%%%%%%%%%%%%%%%%%%%%%%%%%
\paragraph{Title Page.}

Conditional processing can be used to include a title or banner page
in the main document when proper precautions are taken.
Importantly, the code in the main file should ensure that the page counter
(as well as other status parameters which are stored in the |.aux| files)
takes the same value after the conditional processing.
Otherwise the page numbers may take divergent values
depending on which part is compiled.

For example, a title page could be declared by:
%
\begin{center}
\begin{tabular}{l}
|\ifchilddoc\||else|\\
|\addtocounter{page}{-1}|\\
\textit{code for title page}\\
|\newpage|\\
|\||fi|
\end{tabular}
\end{center}
%
A banner page for the child documents can be generated by:
%
\begin{center}
\begin{tabular}{l}
|\ifchilddoc|\\
|\addtocounter{page}{-1}|\\
\textit{code for banner page}\\
|\newpage|\\
|\||fi|
\end{tabular}
\end{center}
%
Here one could write a message such as:
\begin{center}
|This is the part \childdocname{} of \childdocjob{}.|
\end{center}

%%%%%%%%%%%%%%%%%%%%%%%%%%%%%%%%%%%%%%%%%%%%%%%%%%%%%%%%%%%%%%%%%%%%%%%%%%%%%%%%
\subsection{Flags}
\label{sec:flags}

The package makes it easy to generate different versions
of the main or child documents.
To this end compilation flags can be defined
and assigned different default values.
They will be particularly useful in conjunction
with the forwarding mechanism described in \secref{sec:forward}.

For example, it may be useful to have a flag |\version|
which can be set to |draft| or |final|.
The document source will contain some conditional code
depending on the value of |\version|.
Suppose further, the flag should default to |final| for the main file
and to |draft| for child files
which is a natural assignment for editing the document.
This is achieved by placing the following code
in the preamble of the main document
(below the |\childdocmain| directive):
%
\begin{center}
\begin{tabular}{l}
|\ifchilddoc|\\
|\providecommand{\version}{draft}|\\
|\||else|\\
|\providecommand{\version}{final}|\\
|\||fi|
\end{tabular}
\end{center}
%
The definition by |\providecommand| makes sure
that previous definitions are not overwritten.
Further statements |\providecommand{\version}{...}|
can thus be added before the above code to override it.

For the main file, one might add a line
(between |\childdocmain| and the above block)
%
\begin{center}
|%\ifchilddoc\||else\providecommand{\version}{draft}\||fi|
\end{center}
%
which can be uncommented to produce a draft version.
Likewise one can add a line to the very top of a child file
(above the |\childdocof{|\textit{main}|}| directive)
%
\begin{center}
|%\providecommand{\version}{final}|
\end{center}
%
which can be uncommented to produce the final version of this child document.

%%%%%%%%%%%%%%%%%%%%%%%%%%%%%%%%%%%%%%%%%%%%%%%%%%%%%%%%%%%%%%%%%%%%%%%%%%%%%%%%
\subsection{Forwarding}
\label{sec:forward}

Different versions of the main or child documents
using compilation flags as described in \secref{sec:flags}
can be (permanently) stored in different files
for convenient compilation, viewing and distribution.
To this end, the package defines a command
to pass on compilation to a different file:

%%%%%%%%%%%%%%%%%%%%%%%%%%%%%%%%%%%%%%%%
\DescribeMacro{\childdocforward}
The command |\childdocforward| redirects processing to
another source file:
%
\begin{center}
\begin{tabular}{l}
|\input{childdoc.def}|\\
|\childdocforward[|\textit{main}|]{|\textit{dest}|}|\\
\end{tabular}
\end{center}
%
The argument \textit{dest} is the destination file
(without extension).
It should be the main file or one of the child files.
Note that further \textsf{childdoc} directives
such as |\childdocof| and |\childdocforward|
in the indicated file will be processed in this form.
The optional argument \textit{main}
passes on directly to the main file \textit{main}
while pretending to compile the child \textit{dest}.
This form behaves as if \textit{dest}
issues |\childdocof{|\textit{main}|}| right away,
and no further \textsf{childdoc} directives will be processed.

%%%%%%%%%%%%%%%%%%%%%%%%%%%%%%%%%%%%%%%%
\DescribeMacro{\...prefix}
In the alternative form |\childdocforwardprefix|,
%
\begin{center}
\begin{tabular}{l}
|\input{childdoc.def}|\\
|\childdocforwardprefix[|\textit{main}|]{|\textit{prefix}|}{|\textit{dest}|}|
\end{tabular}
\end{center}
%
the destination file is determined by a pattern
depending on the current file:
To make this work, the current file must be called
`{\textit{prefix}\hspace{0.2em}\textit{suffix}}'
with \textit{prefix} matching precisely the argument.
Processing is then passed on to the file
`{\textit{dest}\hspace{0.2em}\textit{suffix}}'.
Surely, the same effect is achieved by
directly specifying the
argument `{\textit{dest}\hspace{0.2em}\textit{suffix}}'
in the first form.
However, that requires to set up a different file
for each child. With the alternative form of the command
all these files can have exactly the same content
which simplifies setting them up and maintaining them.

For example, the following file |draft.tex|
with a compilation flag |\version| as described in \secref{sec:flags}
compiles the main document as a draft:
%
\begin{center}
\begin{tabular}{l}
|\def\version{draft}|\\
|\input{childdoc.def}|\\
|\childdocforward{|\textit{main}|}|
\end{tabular}
\end{center}
%
Likewise, the following files |final|\textit{nn}|.tex|
compile the final version of the child document
|child|\textit{nn}|.tex|:
%
\begin{center}
\begin{tabular}{l}
|\def\version{final}|\\
|\input{childdoc.def}|\\
|\childdocforwardprefix{final}{child}|
\end{tabular}
\end{center}
%

Note that when several versions of a main file and/or of each child file
are to be generated, it may be convenient to set up a |Makefile| or
shell script to automatise the process.

%%%%%%%%%%%%%%%%%%%%%%%%%%%%%%%%%%%%%%%%%%%%%%%%%%%%%%%%%%%%%%%%%%%%%%%%%%%%%%%%
\subsection{Command Line Processing}
\label{sec:commandline}

The effect of redirection files can also be achieved by invoking
the \LaTeX{} compiler with a more elaborate command line.
Most conveniently this should be done as part
of a shell script or a |Makefile|.

When using \textsf{childdoc} in the main file, the following
command lines effectively perform a redirection
(note that depending on the shell being used,
backslashes may have to be doubled: `|\|' $\to$ `|\\|'):
%
\begin{center}
|... -jobname "|\textit{target}|" |\\|"|[\textit{flags}]%
|\input{childdoc.def}\childdocforward[|\textit{main}|]{|\textit{dest}|}"|
\end{center}
%
Here \textit{target} is the name of the output file,
\textit{main} is the name of the main file
and \textit{dest} is the name of the main or child file to be processed
(all filenames without extensions).
The optional argument \textit{main} can be omitted
if \textit{main} matches \textit{dest}.
Optionally, compilation \textit{flags} can be defined via |\def| commands.
This command line makes the \TeX{} engine believe
it is compiling the file \textit{target}
whose content is specified as the latter parameter.
The provided code then forwards the processing to
\textit{main} or \textit{dest} as described in \secref{sec:forward}.

%%%%%%%%%%%%%%%%%%%%%%%%%%%%%%%%%%%%%%%%%%%%%%%%%%%%%%%%%%%%%%%%%%%%%%%%%%%%%%%%
\subsection{Include by Input}
\label{sec:input}

Including child documents by |\include| has some restrictions by design.
Most notably, the content of a child document always occupies
its own set of pages; pages cannot be shared between child documents.
Usually, this behaviour makes perfect sense
because each child document contain an essential part of the document.
However, in some situations it may be desirable to compose
a document from a collection of parts
without having mandatory page breaks between then.
For this case, the package
provides a mechanism to include parts
by |\input| which can also be processed individually.
However, by construction this mechanism
requires manual handling of the content to be output.

%%%%%%%%%%%%%%%%%%%%%%%%%%%%%%%%%%%%%%%%
\DescribeMacro{\ifchilddocmanual}
The main file should be prepared as usual, see \secref{sec:include}.
However, the document body must make a distinction
between processing of an individual part and of the main document, e.g.:
%
\begin{center}
\begin{tabular}{l}
|\ifchilddocmanual|\\
|\input{\childdocname}|\\
|\||else|\\
\textit{document body with }|\input{|\textit{part}|}|\\
|\||fi|
\end{tabular}
\end{center}
%
The conditional |\ifchilddocmanual| is true whenever
a part to be included by |\input| is being compiled,
and the name of the part is stored in |\childdocname|.

%%%%%%%%%%%%%%%%%%%%%%%%%%%%%%%%%%%%%%%%
\DescribeMacro{\childdocby}
Each part to be included by |\input| should start with:
%
\begin{center}
\begin{tabular}{l}
|\input{childdoc.def}|\\
|\childdocby{|\textit{main}|}|\\
\end{tabular}
\end{center}
%
The directive |\childdocby| is similar to |\childdocof|
described in \secref{sec:include},
but the subsequent selection of content must be done manually.
To that end, both |\ifchilddoc| and |\ifchilddocmanual|
will be true upon processing of a part,
and the name of the part is stored in |\childdocname|.
Note that |\jobname| will be set to the filename of the current part
so that each part receives an individual |.aux| file
that does not interfere with the |.aux| file(s) of the main document.
This behaviour can be altered by the alternative form
|\childdocby[*]{|\textit{main}|}| (with a non-empty optional argument)
which uses the |.aux| file of the main document
by setting |\jobname| to \textit{main}.

%%%%%%%%%%%%%%%%%%%%%%%%%%%%%%%%%%%%%%%%%%%%%%%%%%%%%%%%%%%%%%%%%%%%%%%%%%%%%%%%
\subsection{Driver Development}
\label{sec:driver}

The \textsf{childdoc} mechanism can also be use for the development
of definition files such as \LaTeX{} styles or classes.
This case differs from the above setup with multiple parts
included by |\include| in that no |\includeonly| should be invoked.
This can be achieved by starting the include file
(before |\ProvidesPackage|) with:
%
\begin{center}
\begin{tabular}{l}
|\input{childdoc.def}|\\
|\childdocforward{|\textit{main}|}|\\
\end{tabular}
\end{center}
%
or alternatively with:
%
\begin{center}
\begin{tabular}{l}
|\input{childdoc.def}|\\
|\childdocby{|\textit{main}|}|\\
\end{tabular}
\end{center}
%
Both forms have slightly different effects as described above.
The main file is prepared as usual, see \secref{sec:include}.

%%%%%%%%%%%%%%%%%%%%%%%%%%%%%%%%%%%%%%%%%%%%%%%%%%%%%%%%%%%%%%%%%%%%%%%%%%%%%%%%
\subsection{Legacy Detection}
\label{sec:detection}

The directive |\childdocmain| in the main file can detect
whether the complete document or merely a child is to be compiled
even without using the directive |\childdocof|.
This method is deprecated because it is less robust
and there is no compelling reason to use it;
it is merely provided for backward compatibility
and it may be removed in future versions.

If the detection mechanism is to be used,
it is mandatory to correctly specify
the filename of the main file as the argument of |\childdocmain|:
%
\begin{center}
\begin{tabular}{l}
|\input{childdoc.def}|\\
|\childdocmain{|\textit{main}|}|\\
\end{tabular}
\end{center}
%
If |\jobname| does not match the argument \textit{main} of |\childdocmain|,
it is assumed that |\jobname| points to the child file to be compiled.
When using |\childdocmain| with the main file specified as argument,
it suffices to start a child file
with just |\input{|\textit{main}|}|
without loading of the package and using |\childdocof|.
If instead all processing is done
with the appropriate \textsf{childdoc} directives,
the argument of \textit{main} of |\childdocmain| can be empty.

An alternative version of the command line processing described
in \secref{sec:commandline} using the detection mechanism reads:
%
\begin{center}
|... -jobname "|\textit{target}|" "|[\textit{flags}]%
[|\def\jobname{|\textit{dest}|}|]|\input{|\textit{main}|}"|
\end{center}

%%%%%%%%%%%%%%%%%%%%%%%%%%%%%%%%%%%%%%%%%%%%%%%%%%%%%%%%%%%%%%%%%%%%%%%%%%%%%%%%
\subsection{Manual Code}
\label{sec:manual}

In case one cannot be certain whether the definitions file |childdoc.def|
is installed on the target \TeX{} distribution
and one prefers not to ship it,
it is conceivable to paste a few relevant commands into the sources.

To that end, drop all statements |\input{childdoc.def}|
and perform the replacements as outlined below.
Instead of |\childdocmain{|\textit{main}|}| add the following code
to the top of the main file:
%
\begin{center}
\begin{tabular}{l}
|\||ifdefined\childdocname\endinput\||fi\newif\ifchilddoc|\\
|\edef\childdocname{\scantokens\expandafter{\jobname\noexpand}}|\\
|\def\childdocmain{|\textit{main}|}\||ifx\childdocmain\childdocname\||else|\\
|\childdoctrue\includeonly{\childdocname}\let\jobname\childdocmain\||fi|\\
\end{tabular}
\end{center}
%
Instead of |\childdocof{|\textit{main}|}| just include the main file
at the top of each child file:
%
\begin{center}
|\input{|\textit{main}|}|
\end{center}
%
A simple redirection |\childdocforward{|\textit{dest}|}| is achieved by:
%
\begin{center}
|\def\jobname{|\textit{dest}|}\input{\jobname}|
\end{center}
%
The redirection with prefix
|\childdocforwardprefix[|\textit{prefix}|]{|\textit{dest}|}|
is accomplished by:
%
\begin{center}
\begin{tabular}{l}
|{\edef\jobname{\scantokens\expandafter{\jobname\noexpand}}|\\
|\def\redirectjob |\textit{prefix}|#1~~~{\gdef\jobname{|\textit{dest}|#1}}|\\
|\expandafter\redirectjob\jobname~~~}\input{\jobname}|
\end{tabular}
\end{center}

In an alternative approach,
child documents can be compiled by a specific command line
without additional code or specific definitions:
%
\begin{center}
|... -jobname "|\textit{target}|" "|[\textit{flags}]%
|\includeonly{|\textit{dest}|}\input{|\textit{main}|}"|
\end{center}
%

%%%%%%%%%%%%%%%%%%%%%%%%%%%%%%%%%%%%%%%%%%%%%%%%%%%%%%%%%%%%%%%%%%%%%%%%%%%%%%%%
%%%%%%%%%%%%%%%%%%%%%%%%%%%%%%%%%%%%%%%%%%%%%%%%%%%%%%%%%%%%%%%%%%%%%%%%%%%%%%%%
\section{Information}

%%%%%%%%%%%%%%%%%%%%%%%%%%%%%%%%%%%%%%%%%%%%%%%%%%%%%%%%%%%%%%%%%%%%%%%%%%%%%%%%
\subsection{Copyright}

Copyright \copyright{} 2017--2018 Niklas Beisert

This work may be distributed and/or modified under the
conditions of the \LaTeX{} Project Public License, either version 1.3
of this license or (at your option) any later version.
The latest version of this license is in
  \url{http://www.latex-project.org/lppl.txt}
and version 1.3 or later is part of all distributions of \LaTeX{}
version 2005/12/01 or later.

This work has the LPPL maintenance status `maintained'.

The Current Maintainer of this work is Niklas Beisert.

This work consists of the files |README.txt|, |childdoc.ins| and |childdoc.dtx|
as well as the derived files |childdoc.def|, |cdocsamp.tex|
with |cdocsch1.tex|, |cdocsch2.tex|, |cdocspt3.tex|, |cdocspt4.tex|,
|cdocsdrf.tex|, |cdocsfn1.tex|, |cdocsfn2.tex|
as well as |childdoc.pdf|.

%%%%%%%%%%%%%%%%%%%%%%%%%%%%%%%%%%%%%%%%%%%%%%%%%%%%%%%%%%%%%%%%%%%%%%%%%%%%%%%%
\subsection{Files and Installation}

The package consists of the files:
%
\begin{center}
\begin{tabular}{ll}
    |README.txt|   & readme file \\
    |childdoc.ins| & installation file \\
    |childdoc.dtx| & source file \\
    |childdoc.def| & definition file \\
    |cdocsamp.tex| & sample main file \\
    |cdocsch1.tex| & sample include file \\
    |cdocsch2.tex| & sample include file \\
    |cdocspt3.tex| & sample part file \\
    |cdocspt4.tex| & sample part file \\
    |cdocsdrf.tex| & sample redirection file \\
    |cdocsfn1.tex| & sample redirection file \\
    |cdocsfn2.tex| & sample redirection file \\
    |childdoc.pdf| & manual
\end{tabular}
\end{center}
%
The distribution consists of the files
|README.txt|, |childdoc.ins| and |childdoc.dtx|.
%
\begin{itemize}
\item
Run (pdf)\LaTeX{} on |childdoc.dtx|
to compile the manual |childdoc.pdf| (this file).
\item
Run \LaTeX{} on |childdoc.ins| to create the definitions file |childdoc.def|
and the sample |cdocsamp.tex| with include files
|cdocsch1.tex|, |cdocsch2.tex|, |cdocspt3.tex|, |cdocspt4.tex|,
|cdocsdrf.tex|, |cdocsfn1.tex|, |cdocsfn2.tex|.
Then copy the file |childdoc.def| to an appropriate directory of your \LaTeX{}
distribution, e.g.\ \textit{texmf-root}|/tex/latex/childdoc|.
\end{itemize}

%%%%%%%%%%%%%%%%%%%%%%%%%%%%%%%%%%%%%%%%%%%%%%%%%%%%%%%%%%%%%%%%%%%%%%%%%%%%%%%%
\subsection{Related CTAN Packages}

There are several other packages which offer a similar functionality:
%
\begin{itemize}
\item
The packages
\href{http://ctan.org/pkg/docmute}{\textsf{docmute}},
\href{http://ctan.org/pkg/includex}{\textsf{includex}} and
\href{http://ctan.org/pkg/standalone}{\textsf{standalone}}
provide commands to include only the document body of
a child file thus allowing both files to be compiled individually.
\item
The packages \href{http://ctan.org/pkg/subdocs}{\textsf{subdocs}}
and \href{http://ctan.org/pkg/subfiles}{\textsf{subfiles}}
provide structures in which the main and child documents can be
encapsulated and allowing them to be compiled individually.
The inclusion mechanism is different from the conventional |\include|.
\item
The package \href{http://ctan.org/pkg/combine}{\textsf{combine}}
is an elaborate solution to combine several documents into one.
\end{itemize}
%
See also the CTAN topic \href{http://ctan.org/topic/subdocs}{\textsf{subdocs}}
for further related packages.
The present package differs from the above solutions in that
a document structure constructed with the conventional |\include| mechanism
just needs two extra commands at the top of every file
such that all constituent files can be compiled individually.

%%%%%%%%%%%%%%%%%%%%%%%%%%%%%%%%%%%%%%%%%%%%%%%%%%%%%%%%%%%%%%%%%%%%%%%%%%%%%%%%
%\subsection{Feature Suggestions}
%
%The following is a list of features which may be useful for future
%versions of this package:
%%
%\begin{itemize}
%\item
%\ldots
%\end{itemize}

%%%%%%%%%%%%%%%%%%%%%%%%%%%%%%%%%%%%%%%%%%%%%%%%%%%%%%%%%%%%%%%%%%%%%%%%%%%%%%%%
\subsection{Revision History}

%%%%%%%%%%%%%%%%%%%%%%%%%%%%%%%%%%%%%%%%
\paragraph{v2.0:} 2018/12/30

\begin{itemize}
\item
immediate forward processing
\item
added |\childdocby| mechanism
\item
manual restructured
\end{itemize}

%%%%%%%%%%%%%%%%%%%%%%%%%%%%%%%%%%%%%%%%
\paragraph{v1.6:} 2018/01/17

\begin{itemize}
\item
application for development of include files
\item
corrections to manual
\end{itemize}

%%%%%%%%%%%%%%%%%%%%%%%%%%%%%%%%%%%%%%%%
\paragraph{v1.5:} 2017/05/21

\begin{itemize}
\item
more complete structuring introduced
\item
|\childdocof| introduced
\item
|\childdoc| renamed to |\childdocmain|
\item
|\childredirect| renamed to |\childdocforward| and |\childdocforwardprefix|
and functionality expanded
\end{itemize}

%%%%%%%%%%%%%%%%%%%%%%%%%%%%%%%%%%%%%%%%
\paragraph{v1.0:} 2017/04/27

\begin{itemize}
\item
manual and install package
\item
first version published on CTAN
\end{itemize}

%%%%%%%%%%%%%%%%%%%%%%%%%%%%%%%%%%%%%%%%
\paragraph{v0.6:} 2017/04/26

\begin{itemize}
\item
redirection mechanism added
\end{itemize}

%%%%%%%%%%%%%%%%%%%%%%%%%%%%%%%%%%%%%%%%
\paragraph{v0.5:} 2017/04/26

\begin{itemize}
\item
functionality in definition file
\end{itemize}


%%%%%%%%%%%%%%%%%%%%%%%%%%%%%%%%%%%%%%%%%%%%%%%%%%%%%%%%%%%%%%%%%%%%%%%%%%%%%%%%
%%%%%%%%%%%%%%%%%%%%%%%%%%%%%%%%%%%%%%%%%%%%%%%%%%%%%%%%%%%%%%%%%%%%%%%%%%%%%%%%
%%%%%%%%%%%%%%%%%%%%%%%%%%%%%%%%%%%%%%%%%%%%%%%%%%%%%%%%%%%%%%%%%%%%%%%%%%%%%%%%
\appendix

\settowidth\MacroIndent{\rmfamily\scriptsize 000\ }

 \DocInput{childdoc.dtx}

\end{document}
%</driver>
% \fi
%
% %%%%%%%%%%%%%%%%%%%%%%%%%%%%%%%%%%%%%%%%%%%%%%%%%%%%%%%%%%%%%%%%%%%%%%%%%%%%%%
% %%%%%%%%%%%%%%%%%%%%%%%%%%%%%%%%%%%%%%%%%%%%%%%%%%%%%%%%%%%%%%%%%%%%%%%%%%%%%%
% \section{Sample}
%\iffalse
%<*samplemain>
%\fi
%
% The following presents a sample document
% with two chapters, two parts, a title page,
% a compile flag as well as three forwarding files to set the flag.
% It consists of eight |.tex| files:
% \begin{center}
% \begin{tabular}{ll}
% |cdocsamp.tex|&main file\\
% |cdocsch1.tex|&include file for chapter 1\\
% |cdocsch2.tex|&include file for chapter 2\\
% |cdocspt3.tex|&include file for part 3\\
% |cdocspt4.tex|&include file for part 4\\
% |cdocsdrf.tex|&forwarding file for main file in draft mode\\
% |cdocsfi1.tex|&forwarding file for final version of chapter 1\\
% |cdocsfi2.tex|&forwarding file for final version of chapter 2\\
% \end{tabular}
% \end{center}
% Each of the eight files can be compiled directly by the \LaTeX{} compiler.
%
% %%%%%%%%%%%%%%%%%%%%%%%%%%%%%%%%%%%%%%
% \paragraph{Main File.}
%
% The main file is called |cdocsamp.tex|.
%
% Load the \textsf{childdoc} definitions and
% declare the filename for the main document:
%    \begin{macrocode}
\input{childdoc.def}
\childdocmain{}
%    \end{macrocode}

% Optional override for |\version| flag:
%    \begin{macrocode}
%%\ifchilddoc\else\providecommand{\version}{draft}\fi
%    \end{macrocode}

% Define the default values for the |\version| flag
% (|final| for the main file and |draft| for childs):
%    \begin{macrocode}
\ifchilddoc
\providecommand{\version}{draft}
\else
\providecommand{\version}{final}
\fi
%    \end{macrocode}

% Load the standard document class:
%    \begin{macrocode}
\documentclass[12pt]{article}
%    \end{macrocode}

% Start the document body:
%    \begin{macrocode}
\begin{document}
%    \end{macrocode}

% Declare a title page.
% Print title, part of document being processed and version flag:
%    \begin{macrocode}
\addtocounter{page}{-1}
\begin{center}
{\LARGE\bfseries{}childdoc example\par}
\vspace{1cm}
\ifchilddoc
\ifchilddocmanual part\else chapter\fi:
`\childdocname' of `\childdocjob'\par
\else
main document: `\childdocjob'\par
\fi
version: \version\par
\end{center}
\newpage
%    \end{macrocode}

% Manually include selected file,
% otherwise process as usual:
%    \begin{macrocode}
\ifchilddocmanual
\section*{part `\childdocname'}
\input{\childdocname}
\else
%    \end{macrocode}

% Include the two chapters:
%    \begin{macrocode}
\include{cdocsch1}
\include{cdocsch2}
%    \end{macrocode}

% Include the two parts unless only chapters should be displayed:
%    \begin{macrocode}
\ifchilddoc\else
\section{part three}
\input{cdocspt3}
\section{part four}
\input{cdocspt4}
\fi
%    \end{macrocode}

% Process as usual until here:
%    \begin{macrocode}
\fi
%    \end{macrocode}

% End of document body:
%    \begin{macrocode}
\end{document}
%    \end{macrocode}
%\iffalse
%</samplemain>
%\fi
%
% %%%%%%%%%%%%%%%%%%%%%%%%%%%%%%%%%%%%%%
% \paragraph{Chapter Include Files.}
%
% The include files are called |cdocsch1.tex| and |cdocsch2.tex|.
%
%\iffalse
%<*samplechap1|samplechap2>
%\fi

% Optional override for |\version| flag:
%    \begin{macrocode}
%%\providecommand{\version}{final}
%    \end{macrocode}

% Include the main document:
%    \begin{macrocode}
\input{childdoc.def}
\childdocof{cdocsamp}
%    \end{macrocode}

%\iffalse
%</samplechap1|samplechap2>
%\fi
%
%\iffalse
%<*samplechap1>
%\fi
% Some text for chapter 1:
%    \begin{macrocode}
\section{one}
some text in chapter one
%    \end{macrocode}

%\iffalse
%</samplechap1>
%\fi
% Some text for chapter 2:
%\iffalse
%<*samplechap2>
%\fi
%    \begin{macrocode}
\section{two}
more text in chapter two
%    \end{macrocode}

%\iffalse
%</samplechap2>
%\fi
%
% %%%%%%%%%%%%%%%%%%%%%%%%%%%%%%%%%%%%%%
% \paragraph{Part Include Files.}
%
% The include files are called |cdocspt3.tex| and |cdocspt4.tex|.
%
%\iffalse
%<*samplepart3|samplepart4>
%\fi

% Optional override for |\version| flag:
%    \begin{macrocode}
%%\providecommand{\version}{final}
%    \end{macrocode}

% Include the main document:
%    \begin{macrocode}
\input{childdoc.def}
\childdocby{cdocsamp}
%    \end{macrocode}

%\iffalse
%</samplepart3|samplepart4>
%\fi
%
%\iffalse
%<*samplepart3>
%\fi
% Some text for part 3:
%    \begin{macrocode}
some text in part three
%    \end{macrocode}

%\iffalse
%</samplepart3>
%\fi
% Some text for part 4:
%\iffalse
%<*samplepart4>
%\fi
%    \begin{macrocode}
more text in part four
%    \end{macrocode}

%\iffalse
%</samplepart4>
%\fi
%
% %%%%%%%%%%%%%%%%%%%%%%%%%%%%%%%%%%%%%%
% \paragraph{Forwarding for a Complete Draft.}
%
% The following forwarding file |cdocsdrf.tex|
% compiles the main document in draft mode:
%\iffalse
%<*sampledraft>
%\fi
%    \begin{macrocode}
\def\version{draft}
\input{childdoc.def}
\childdocforward{cdocsamp}
%    \end{macrocode}

%\iffalse
%</sampledraft>
%\fi
%
% %%%%%%%%%%%%%%%%%%%%%%%%%%%%%%%%%%%%%%
% \paragraph{Forwarding for Final Version of the Chapters.}
%
% The following forwarding files |cdocsfn1.tex| and |cdocsfn2.tex|
% (with identical content)
% compile the final versions of the child documents
% |cdocsch1.tex| and |cdocsch2.tex|, respectively:
%\iffalse
%<*samplefinal>
%\fi
%    \begin{macrocode}
\def\version{final}
\input{childdoc.def}
\childdocforwardprefix[cdocsamp]{cdocsfn}{cdocsch}
%    \end{macrocode}

%\iffalse
%</samplefinal>
%\fi
%
% %%%%%%%%%%%%%%%%%%%%%%%%%%%%%%%%%%%%%%
% \paragraph{Command Line Processing.}
%
% The following three command lines generate the output files
% |cdocscld|, |cdocscl1| and |cdocscl2|
% which should be identical to
% |cdocsdrf|, |cdocsch1| and |cdocsfn2|, respectively:
% \begin{center}
% \begin{tabular}{l}
% |latex -jobname cdocscld \|\\
% |  "\def\version{draft}\input{childdoc.def}\childdocforward{cdocsamp}"|\\
% |latex -jobname cdocscl1 \|\\
% |  "\input{childdoc.def}\childdocforward[cdocsamp]{cdocsch1}"|\\
% |latex -jobname cdocscl2 \|\\
% |  "\def\version{final}\input{childdoc.def}\childdocforward{cdocsch2}"|
% \end{tabular}
% \end{center}
% Note that the trailing backslash on each first line
% merely continues the input to the second line
% (for convenient cut ant paste).
% Furthermore, the command |latex| can be replaced by any
% of its alternative versions such as |pdflatex|.
%
% %%%%%%%%%%%%%%%%%%%%%%%%%%%%%%%%%%%%%%%%%%%%%%%%%%%%%%%%%%%%%%%%%%%%%%%%%%%%%%
% %%%%%%%%%%%%%%%%%%%%%%%%%%%%%%%%%%%%%%%%%%%%%%%%%%%%%%%%%%%%%%%%%%%%%%%%%%%%%%
% \section{Implementation}
%\iffalse
%<*package>
%\fi
%
% This section describes the definitions file |childdoc.def|.

% The definitions cannot be loaded using |\usepackage| or |\RequirePackage|
% which has a mechanism to prevent loading a style file more than once.
% When loading the definitions by means of |\input|
% multiple instances have to be prevented manually:
%\iffalse
%This code needs to be before the `\ProvidesFile' directive
%which is defined at the beginning of this file.
%Therefore it is also placed there and commented out here.
%</package>
%<*discard>
%\fi
%    \begin{macrocode}
\ifdefined\childdocmain\endinput\fi
%    \end{macrocode}
%\iffalse
%</discard>
%<*package>
%\fi
%
% \macro{\ifchilddoc}
% \macro{\ifchilddocmanual}
% The conditional |\ifchilddoc| tells whether a
% child (true) or main (false) document is being compiled.
% The conditional |\ifchilddocmanual| tells whether
% the |\includeonly| mechanism is used (false) or
% the selection of child files must be performed manually (true).
% The definitions initialise to false:
%    \begin{macrocode}
\newif\ifchilddoc
\newif\ifchilddocmanual
%    \end{macrocode}

% \macro{\childdocname}
% \macro{\childdocjob}
% The macro |\childdocname| stores the name of the main document
% to be compiled. The macro |\childdocjob| stores the name of
% the document on which the \LaTeX{} compiler was originally invoked.
% The content of |\jobname| cannot be compared
% to filenames specified in the source due to different catcodes.
% The following code rescans |\jobname|, stores the result
% in |\childdocname| and saves a copy in |\childdocjob|:
%    \begin{macrocode}
\edef\childdocname{\scantokens\expandafter{\jobname\noexpand}}
\let\childdocjob\childdocname
%    \end{macrocode}

% \macro{\childdocdisable}
% The macro |\childdocdisable| prevents the main file
% from being processed more than once.
% At this stage, the main document command |\childdocmain|
% is assumed to be called once again where it should do nothing.
% Any subsequent call to it should prevent
% a secondary processing of the main document
% It overwrites the forwarding commands
% |\childdocof| and |\childdocforward|
% with empty macros to prevent further inclusions of the main document:
%    \begin{macrocode}
\newcommand{\childdocdisable}
{
  \renewcommand{\childdocmain}[1]{\renewcommand{\childdocmain}[1]{\endinput}}
  \renewcommand{\childdocof}[1]{}
  \renewcommand{\childdocby}[2][]{}
  \renewcommand{\childdocforward}[2][]{}
  \renewcommand{\childdocdisable}{}
}
%    \end{macrocode}

% \macro{\childdocmain}
% The macro |\childdocmain| is to be called at the top of the main file
% with nothing or the main filename (without extension) as argument.
% First, it breaks loops.
% If the argument is not empty and does not match |\childdocname|
% (which is set by the first inclusion of |childdoc.def|),
% |\ifchilddoc| is set to true, |\includeonly| is applied to the child file
% and |\jobname| is set to the main file
% (for proper handling of |.aux| files):
%    \begin{macrocode}
\newcommand{\childdocmain}[1]
{
  \childdocdisable\childdocmain{}
  \if?#1?\else
    \begingroup
      \def\childdoctmp{#1}
      \ifx\childdoctmp\childdocname
        \def\childdoctmp{}
      \else
        \def\childdoctmp
        {
          \childdoctrue
          \includeonly{\childdocname}
          \def\childdocjob{#1}
          \def\jobname{#1}
        }
      \fi
      \expandafter
    \endgroup
    \childdoctmp
  \fi
}
%    \end{macrocode}

% \macro{\childdocof}
% The command |\childdocof| redirects
% compilation to the main file |#1|.
%    \begin{macrocode}
\newcommand{\childdocof}[1]
{
  \childdocdisable
  \childdoctrue
  \includeonly{\childdocname}
  \def\jobname{#1}
  \def\childdocjob{#1}
  \input{#1}
}
%    \end{macrocode}

% \macro{\childdocby}
% The command |\childdocby| ....
%    \begin{macrocode}
\newcommand{\childdocby}[2][]
{
  \childdocdisable
  \childdoctrue
  \childdocmanualtrue
  \if?#1?\else
    \def\jobname{#2}
  \fi
  \def\childdocjob{#2}
  \input{#2}
  \endinput
}
%    \end{macrocode}

% \macro{\childdocforward}
% The command |\childdocforward| redirects
% compilation to the main file or
% (if the optional argument is given) a child file.
% Parameters are set as if the main file
% or a child file starting with |\childdocof| was compiled.
% Then compilation is handed over to the main file:
%    \begin{macrocode}
\newcommand{\childdocforward}[2][]
{
  \begingroup
    \if?#1?
      \def\childdoctmp
      {
        \def\childdocname{#2}
        \def\childdocjob{#2}
        \def\jobname{#2}
        \input{#2}
        \endinput
      }
    \else
      \def\childdoctmp
      {
        \childdocdisable
        \def\childdocname{#2}
        \childdoctrue
        \includeonly{#2}
        \def\childdocjob{#1}
        \def\jobname{#1}
        \input{#1}
        \endinput
      }
    \fi
    \expandafter
  \endgroup
  \childdoctmp
}
%    \end{macrocode}

% \macro{\childdocforwardprefix}
% The command |\childdocforwardprefix| redirects
% compilation to the main or a child file by means of a pattern.
% The prefix |#1| in the current filename is replaced by |#2|
% and the suffix of the current filename is kept
% (it is assumed that the filename does not contain the substring `|~~~|'
% which is used as a delimiter).
% Compilation is handed over to the new file by |\childdocforward|:
%    \begin{macrocode}
\newcommand{\childdocforwardprefix}[3][]
{
  \begingroup
    \def\childdocextract #2##1~~~{\def\childdoctmp{\childdocforward[#1]{#3##1}}}
    \expandafter\childdocextract\childdocname~~~
    \expandafter
  \endgroup
  \childdoctmp
}
%    \end{macrocode}

% \macro{\childdoc}
% The deprecated macro |\childdoc| is a legacy version of |\childdocmain|:
%    \begin{macrocode}
\newcommand{\childdoc}{\childdocmain}
%    \end{macrocode}

% \macro{\childdocredirect}
% The deprecated macro |\childdocredirect| is a legacy version
% of |\childdocforward| and |\childdocforwardprefix|:
%    \begin{macrocode}
\newcommand{\childdocredirect}[2][]
{
  \begingroup
    \if?#1?
      \def\childdoctmp{\childdocforward{#2}}
    \else
      \def\childdoctmp{\childdocforwardprefix{#1}{#2}}
    \fi
    \expandafter
  \endgroup
  \childdoctmp
}
%    \end{macrocode}

%\iffalse
%</package>
%\fi
%
\endinput
\childdocforward{cdocsamp}"|\\
% |latex -jobname cdocscl1 \|\\
% |  "% \iffalse
%
% childdoc.dtx Copyright (C) 2017-2018 Niklas Beisert
%
% This work may be distributed and/or modified under the
% conditions of the LaTeX Project Public License, either version 1.3
% of this license or (at your option) any later version.
% The latest version of this license is in
%   http://www.latex-project.org/lppl.txt
% and version 1.3 or later is part of all distributions of LaTeX
% version 2005/12/01 or later.
%
% This work has the LPPL maintenance status `maintained'.
%
% The Current Maintainer of this work is Niklas Beisert.
%
% This work consists of the files childdoc.dtx and childdoc.ins
% and the derived files childdoc.def and cdocsamp.tex with
% cdocsch1.tex, cdocsch2.tex, cdocsdrf.tex, cdocsfn1.tex, cdocsfn2.tex.
%
%<package>\ifdefined\childdocmain\endinput\fi
%<package>\ProvidesFile{childdoc.def}[2018/12/30 v2.0 child document driver]
%<samplemain>\ProvidesFile{cdocsamp.tex}[2018/12/30 v2.0 sample for childdoc]
%<*driver>
%\ProvidesFile{childdoc.drv}[2018/12/30 v2.0 childdoc reference manual file]
\PassOptionsToClass{10pt,a4paper}{article}
\documentclass{ltxdoc}

\usepackage[margin=35mm]{geometry}
\usepackage{hyperref}
\usepackage{hyperxmp}
\usepackage[usenames]{color}

\hypersetup{colorlinks=true}
\hypersetup{pdfstartview=FitH}
\hypersetup{pdfpagemode=UseNone}
\hypersetup{pdfsource={}}
\hypersetup{pdflang={en-UK}}
\hypersetup{pdfcopyright={Copyright 2017-2018 Niklas Beisert.
  This work may be distributed and/or modified under the
  conditions of the LaTeX Project Public License, either version 1.3
  of this license or (at your option) any later version.}}
\hypersetup{pdflicenseurl={http://www.latex-project.org/lppl.txt}}
\hypersetup{pdfcontactaddress={ETH Zurich, ITP, HIT K,
  Wolfgang-Pauli-Strasse 27}}
\hypersetup{pdfcontactpostcode={8093}}
\hypersetup{pdfcontactcity={Zurich}}
\hypersetup{pdfcontactcountry={Switzerland}}
\hypersetup{pdfcontactemail={nbeisert@itp.phys.ethz.ch}}
\hypersetup{pdfcontacturl={http://people.phys.ethz.ch/\xmptilde nbeisert/}}

\newcommand{\secref}[1]{\hyperref[#1]{section \ref*{#1}}}

\parskip1ex
\parindent0pt
\let\olditemize\itemize
\def\itemize{\olditemize\parskip0pt}

\begin{document}

\title{The \textsf{childdoc} Package}
\hypersetup{pdftitle={The childdoc Package}}
\author{Niklas Beisert\\[2ex]
  Institut f\"ur Theoretische Physik\\
  Eidgen\"ossische Technische Hochschule Z\"urich\\
  Wolfgang-Pauli-Strasse 27, 8093 Z\"urich, Switzerland\\[1ex]
  \href{mailto:nbeisert@itp.phys.ethz.ch}
  {\texttt{nbeisert@itp.phys.ethz.ch}}}
\hypersetup{pdfauthor={Niklas Beisert}}
\hypersetup{pdfsubject={Manual for the LaTeX2e Package childdoc}}
\date{30 December 2018, \textsf{v2.0}}
\maketitle

\begin{abstract}\noindent
\textsf{childdoc} is a \LaTeXe{} package
that enables the direct compilation
of document sections included by |\include|
to individual files.
\end{abstract}

\begingroup
\parskip0ex
\tableofcontents
\endgroup

%%%%%%%%%%%%%%%%%%%%%%%%%%%%%%%%%%%%%%%%%%%%%%%%%%%%%%%%%%%%%%%%%%%%%%%%%%%%%%%%
%%%%%%%%%%%%%%%%%%%%%%%%%%%%%%%%%%%%%%%%%%%%%%%%%%%%%%%%%%%%%%%%%%%%%%%%%%%%%%%%
\section{Introduction}

\LaTeX{} provides a mechanism to structure a large document (such as a book)
into a main file and several child files (containing the chapters)
using the |\include| command.
This mechanism is beneficial for documents
which span hundreds of pages in order to
make the source file(s) more manageable.
Moreover, compilation can be restricted to
selected child files by means of the |\includeonly| command.
The latter feature can be used to reduce the compilation time while editing
(this was significantly more useful in the earlier days of \LaTeX{})
or to generate a smaller document which is easier to navigate.
Another application of |\includeonly| is to generate
documents consisting of selected parts of the complete document.

However, there are a few drawbacks of the plain |\include| mechanism:
\begin{itemize}
\item
The child files cannot be compiled on their own,
they can only be compiled via the main file.
A naive editing environment
(such as a text editor with an option
to have the current file processed by \LaTeX)
may require one to switch to the main file before compiling;
attempting to compile the child file produces errors.
\item
The main file must be modified (each time)
to adjust the |\includeonly| command
to the present needs. This easily leaves the main file in a messy state.
\item
The generated document will always carry the filename
of the main document. This is inconvenient if
several child files are to be compiled and
to be kept for distribution.
\end{itemize}

The present package provides a simple interface
to make child files individually compilable by \LaTeX{}.
Compiling a child file then has the same effect as compiling
the main file with an |\includeonly| command
to select the appropriate child.
Moreover the generated document will carry the name of the child
rather than the main file.
This resolves all three above issues.

This feature is meant to make the editing of books,
thesis documents and lecture notes somewhat more convenient.
However, the package can also be used efficiently for
composing a series of documents (such as exercise sheets)
which are typically distributed individually.
It then assists the author in generating the individual documents
(potentially in different versions)
as well as a document containing the collected series.
Another application is in developing style files
or other kinds of included material
where compilation of the style file could redirect
to a sample or test file.

%%%%%%%%%%%%%%%%%%%%%%%%%%%%%%%%%%%%%%%%%%%%%%%%%%%%%%%%%%%%%%%%%%%%%%%%%%%%%%%%
%%%%%%%%%%%%%%%%%%%%%%%%%%%%%%%%%%%%%%%%%%%%%%%%%%%%%%%%%%%%%%%%%%%%%%%%%%%%%%%%
\section{Usage}

First of all, the package \textsf{childdoc} is \emph{not} a standard
\LaTeXe{} |.sty| style file! Therefore it needs to be invoked in
a non-standard way.

%%%%%%%%%%%%%%%%%%%%%%%%%%%%%%%%%%%%%%%%%%%%%%%%%%%%%%%%%%%%%%%%%%%%%%%%%%%%%%%%
\subsection{Included Files}
\label{sec:include}

%%%%%%%%%%%%%%%%%%%%%%%%%%%%%%%%%%%%%%%%
\DescribeMacro{\childdocmain}
To use the package, add the commands
\begin{center}
\begin{tabular}{l}
|\input{childdoc.def}|\\
|\childdocmain{}|\\
\end{tabular}
\end{center}
at the very top of the main \LaTeX{} file,
in particular \emph{before} the |\documentclass| statement!
The argument of |\childdocmain| should be left empty
(but it must be present).

%%%%%%%%%%%%%%%%%%%%%%%%%%%%%%%%%%%%%%%%
\DescribeMacro{\childdocof}
Furthermore, add the commands
\begin{center}
\begin{tabular}{l}
|\input{childdoc.def}|\\
|\childdocof{|\textit{main}|}|\\
\end{tabular}
\end{center}
at the top of every child file \textit{child}
which is included by |\include{|\textit{child}|}|
from within the main file
(or at least for those files to be compiled individually).
The argument \textit{main} must be the filename of the main file.

There are a couple of
considerations in setting up the main and child documents:

%%%%%%%%%%%%%%%%%%%%%%%%%%%%%%%%%%%%%%%%
\paragraph{Restrictions.}

Please note the following restrictions:
\begin{itemize}
\item
|\childdocmain| must be called with one argument \textit{main}
to ensure compatibility with earlier version of the package.
It must either be empty (|\childdocmain{}|)
or precisely match the filename of the main file in which it is specified.
See \secref{sec:detection} for further information.
\item
The filename \textit{main} must be specified without the |.tex| extension.
\item
The filename \textit{main} is case sensitive
(even in case-insensitive file systems)
due to internal string comparison.
\item
The argument \textit{main} should be fully expanded, it cannot be a macro.
\item
Subdirectories and special characters should be avoided in filenames.
\item
The command |\childdocmain{|\textit{main}|}| must be followed by a whitespace.
It should not be followed immediately by another command
or by a comment mark `|%|'.
This is because the \TeX{} parser reads the token immediately following
the argument of |\childdocmain| and puts it
at the beginning of every child section;
however, a white\-space is ignored.
\end{itemize}

%%%%%%%%%%%%%%%%%%%%%%%%%%%%%%%%%%%%%%%%
\paragraph{Content of Main File.}

It is advisable to place all content in the child files included by |\include|.
Any output contained in the main file will appear in all child documents
unless suppressed manually;
it cannot be suppressed automatically by the |\includeonly| directive
and thus should normally be avoided.
A method to include some content in the main file
by means of conditional processing is described in \secref{sec:conditional}.

%%%%%%%%%%%%%%%%%%%%%%%%%%%%%%%%%%%%%%%%
\paragraph{Page Numbering.}

When only a part of the document is compiled,
the appropriate numbering of pages
(as well as other status parameters)
is determined from the |.aux| files.
The latter contain information from previous passes.
However this information needs to propagate through
all intermediate child documents.
Therefore the page numbering in child documents may well
be inconsistent until the complete document is compiled at least once.

A useful (if unconventional) way to always ensure a consistent
page numbering is to restart the numbering in each child document
and denote the pages by `\textit{child}|.|\textit{page}'
where \textit{child} represents the chapter/section number of the child file.
This can be achieved by the command
|\numberwithin{page}{|\textit{child}|}|
of the \textsf{amsmath} package
where \textit{child} can be |chapter| or |section|
depending on the chosen structuring.
Alternatively, one can modify the macro |\thepage| appropriately
and reset the counter |page| at the start of each child file.

%%%%%%%%%%%%%%%%%%%%%%%%%%%%%%%%%%%%%%%%%%%%%%%%%%%%%%%%%%%%%%%%%%%%%%%%%%%%%%%%
\subsection{Conditional Processing}
\label{sec:conditional}

The package provides a mechanism to compile different versions
of a document. To customise the versions further some conditional processing
can come in handy to distinguish which version is being compiled.
The package provides two macros to describe the compilation context:

%%%%%%%%%%%%%%%%%%%%%%%%%%%%%%%%%%%%%%%%
\DescribeMacro{\ifchilddoc}
The conditional |\ifchilddoc| distinguishes between the compilation of
child documents and the main document:
%
\begin{center}
|\ifchilddoc |\textit{child-code}| |[|\||else |\textit{main-code}]| \||fi|
\end{center}

%%%%%%%%%%%%%%%%%%%%%%%%%%%%%%%%%%%%%%%%
\DescribeMacro{\childdocname}
\DescribeMacro{\childdocjob}
The macro |\childdocname| contains the filename (without extension)
of the main or child file being processed.
Note that |\childdocjob| will always contain the name of the main file.

%%%%%%%%%%%%%%%%%%%%%%%%%%%%%%%%%%%%%%%%
\paragraph{Title Page.}

Conditional processing can be used to include a title or banner page
in the main document when proper precautions are taken.
Importantly, the code in the main file should ensure that the page counter
(as well as other status parameters which are stored in the |.aux| files)
takes the same value after the conditional processing.
Otherwise the page numbers may take divergent values
depending on which part is compiled.

For example, a title page could be declared by:
%
\begin{center}
\begin{tabular}{l}
|\ifchilddoc\||else|\\
|\addtocounter{page}{-1}|\\
\textit{code for title page}\\
|\newpage|\\
|\||fi|
\end{tabular}
\end{center}
%
A banner page for the child documents can be generated by:
%
\begin{center}
\begin{tabular}{l}
|\ifchilddoc|\\
|\addtocounter{page}{-1}|\\
\textit{code for banner page}\\
|\newpage|\\
|\||fi|
\end{tabular}
\end{center}
%
Here one could write a message such as:
\begin{center}
|This is the part \childdocname{} of \childdocjob{}.|
\end{center}

%%%%%%%%%%%%%%%%%%%%%%%%%%%%%%%%%%%%%%%%%%%%%%%%%%%%%%%%%%%%%%%%%%%%%%%%%%%%%%%%
\subsection{Flags}
\label{sec:flags}

The package makes it easy to generate different versions
of the main or child documents.
To this end compilation flags can be defined
and assigned different default values.
They will be particularly useful in conjunction
with the forwarding mechanism described in \secref{sec:forward}.

For example, it may be useful to have a flag |\version|
which can be set to |draft| or |final|.
The document source will contain some conditional code
depending on the value of |\version|.
Suppose further, the flag should default to |final| for the main file
and to |draft| for child files
which is a natural assignment for editing the document.
This is achieved by placing the following code
in the preamble of the main document
(below the |\childdocmain| directive):
%
\begin{center}
\begin{tabular}{l}
|\ifchilddoc|\\
|\providecommand{\version}{draft}|\\
|\||else|\\
|\providecommand{\version}{final}|\\
|\||fi|
\end{tabular}
\end{center}
%
The definition by |\providecommand| makes sure
that previous definitions are not overwritten.
Further statements |\providecommand{\version}{...}|
can thus be added before the above code to override it.

For the main file, one might add a line
(between |\childdocmain| and the above block)
%
\begin{center}
|%\ifchilddoc\||else\providecommand{\version}{draft}\||fi|
\end{center}
%
which can be uncommented to produce a draft version.
Likewise one can add a line to the very top of a child file
(above the |\childdocof{|\textit{main}|}| directive)
%
\begin{center}
|%\providecommand{\version}{final}|
\end{center}
%
which can be uncommented to produce the final version of this child document.

%%%%%%%%%%%%%%%%%%%%%%%%%%%%%%%%%%%%%%%%%%%%%%%%%%%%%%%%%%%%%%%%%%%%%%%%%%%%%%%%
\subsection{Forwarding}
\label{sec:forward}

Different versions of the main or child documents
using compilation flags as described in \secref{sec:flags}
can be (permanently) stored in different files
for convenient compilation, viewing and distribution.
To this end, the package defines a command
to pass on compilation to a different file:

%%%%%%%%%%%%%%%%%%%%%%%%%%%%%%%%%%%%%%%%
\DescribeMacro{\childdocforward}
The command |\childdocforward| redirects processing to
another source file:
%
\begin{center}
\begin{tabular}{l}
|\input{childdoc.def}|\\
|\childdocforward[|\textit{main}|]{|\textit{dest}|}|\\
\end{tabular}
\end{center}
%
The argument \textit{dest} is the destination file
(without extension).
It should be the main file or one of the child files.
Note that further \textsf{childdoc} directives
such as |\childdocof| and |\childdocforward|
in the indicated file will be processed in this form.
The optional argument \textit{main}
passes on directly to the main file \textit{main}
while pretending to compile the child \textit{dest}.
This form behaves as if \textit{dest}
issues |\childdocof{|\textit{main}|}| right away,
and no further \textsf{childdoc} directives will be processed.

%%%%%%%%%%%%%%%%%%%%%%%%%%%%%%%%%%%%%%%%
\DescribeMacro{\...prefix}
In the alternative form |\childdocforwardprefix|,
%
\begin{center}
\begin{tabular}{l}
|\input{childdoc.def}|\\
|\childdocforwardprefix[|\textit{main}|]{|\textit{prefix}|}{|\textit{dest}|}|
\end{tabular}
\end{center}
%
the destination file is determined by a pattern
depending on the current file:
To make this work, the current file must be called
`{\textit{prefix}\hspace{0.2em}\textit{suffix}}'
with \textit{prefix} matching precisely the argument.
Processing is then passed on to the file
`{\textit{dest}\hspace{0.2em}\textit{suffix}}'.
Surely, the same effect is achieved by
directly specifying the
argument `{\textit{dest}\hspace{0.2em}\textit{suffix}}'
in the first form.
However, that requires to set up a different file
for each child. With the alternative form of the command
all these files can have exactly the same content
which simplifies setting them up and maintaining them.

For example, the following file |draft.tex|
with a compilation flag |\version| as described in \secref{sec:flags}
compiles the main document as a draft:
%
\begin{center}
\begin{tabular}{l}
|\def\version{draft}|\\
|\input{childdoc.def}|\\
|\childdocforward{|\textit{main}|}|
\end{tabular}
\end{center}
%
Likewise, the following files |final|\textit{nn}|.tex|
compile the final version of the child document
|child|\textit{nn}|.tex|:
%
\begin{center}
\begin{tabular}{l}
|\def\version{final}|\\
|\input{childdoc.def}|\\
|\childdocforwardprefix{final}{child}|
\end{tabular}
\end{center}
%

Note that when several versions of a main file and/or of each child file
are to be generated, it may be convenient to set up a |Makefile| or
shell script to automatise the process.

%%%%%%%%%%%%%%%%%%%%%%%%%%%%%%%%%%%%%%%%%%%%%%%%%%%%%%%%%%%%%%%%%%%%%%%%%%%%%%%%
\subsection{Command Line Processing}
\label{sec:commandline}

The effect of redirection files can also be achieved by invoking
the \LaTeX{} compiler with a more elaborate command line.
Most conveniently this should be done as part
of a shell script or a |Makefile|.

When using \textsf{childdoc} in the main file, the following
command lines effectively perform a redirection
(note that depending on the shell being used,
backslashes may have to be doubled: `|\|' $\to$ `|\\|'):
%
\begin{center}
|... -jobname "|\textit{target}|" |\\|"|[\textit{flags}]%
|\input{childdoc.def}\childdocforward[|\textit{main}|]{|\textit{dest}|}"|
\end{center}
%
Here \textit{target} is the name of the output file,
\textit{main} is the name of the main file
and \textit{dest} is the name of the main or child file to be processed
(all filenames without extensions).
The optional argument \textit{main} can be omitted
if \textit{main} matches \textit{dest}.
Optionally, compilation \textit{flags} can be defined via |\def| commands.
This command line makes the \TeX{} engine believe
it is compiling the file \textit{target}
whose content is specified as the latter parameter.
The provided code then forwards the processing to
\textit{main} or \textit{dest} as described in \secref{sec:forward}.

%%%%%%%%%%%%%%%%%%%%%%%%%%%%%%%%%%%%%%%%%%%%%%%%%%%%%%%%%%%%%%%%%%%%%%%%%%%%%%%%
\subsection{Include by Input}
\label{sec:input}

Including child documents by |\include| has some restrictions by design.
Most notably, the content of a child document always occupies
its own set of pages; pages cannot be shared between child documents.
Usually, this behaviour makes perfect sense
because each child document contain an essential part of the document.
However, in some situations it may be desirable to compose
a document from a collection of parts
without having mandatory page breaks between then.
For this case, the package
provides a mechanism to include parts
by |\input| which can also be processed individually.
However, by construction this mechanism
requires manual handling of the content to be output.

%%%%%%%%%%%%%%%%%%%%%%%%%%%%%%%%%%%%%%%%
\DescribeMacro{\ifchilddocmanual}
The main file should be prepared as usual, see \secref{sec:include}.
However, the document body must make a distinction
between processing of an individual part and of the main document, e.g.:
%
\begin{center}
\begin{tabular}{l}
|\ifchilddocmanual|\\
|\input{\childdocname}|\\
|\||else|\\
\textit{document body with }|\input{|\textit{part}|}|\\
|\||fi|
\end{tabular}
\end{center}
%
The conditional |\ifchilddocmanual| is true whenever
a part to be included by |\input| is being compiled,
and the name of the part is stored in |\childdocname|.

%%%%%%%%%%%%%%%%%%%%%%%%%%%%%%%%%%%%%%%%
\DescribeMacro{\childdocby}
Each part to be included by |\input| should start with:
%
\begin{center}
\begin{tabular}{l}
|\input{childdoc.def}|\\
|\childdocby{|\textit{main}|}|\\
\end{tabular}
\end{center}
%
The directive |\childdocby| is similar to |\childdocof|
described in \secref{sec:include},
but the subsequent selection of content must be done manually.
To that end, both |\ifchilddoc| and |\ifchilddocmanual|
will be true upon processing of a part,
and the name of the part is stored in |\childdocname|.
Note that |\jobname| will be set to the filename of the current part
so that each part receives an individual |.aux| file
that does not interfere with the |.aux| file(s) of the main document.
This behaviour can be altered by the alternative form
|\childdocby[*]{|\textit{main}|}| (with a non-empty optional argument)
which uses the |.aux| file of the main document
by setting |\jobname| to \textit{main}.

%%%%%%%%%%%%%%%%%%%%%%%%%%%%%%%%%%%%%%%%%%%%%%%%%%%%%%%%%%%%%%%%%%%%%%%%%%%%%%%%
\subsection{Driver Development}
\label{sec:driver}

The \textsf{childdoc} mechanism can also be use for the development
of definition files such as \LaTeX{} styles or classes.
This case differs from the above setup with multiple parts
included by |\include| in that no |\includeonly| should be invoked.
This can be achieved by starting the include file
(before |\ProvidesPackage|) with:
%
\begin{center}
\begin{tabular}{l}
|\input{childdoc.def}|\\
|\childdocforward{|\textit{main}|}|\\
\end{tabular}
\end{center}
%
or alternatively with:
%
\begin{center}
\begin{tabular}{l}
|\input{childdoc.def}|\\
|\childdocby{|\textit{main}|}|\\
\end{tabular}
\end{center}
%
Both forms have slightly different effects as described above.
The main file is prepared as usual, see \secref{sec:include}.

%%%%%%%%%%%%%%%%%%%%%%%%%%%%%%%%%%%%%%%%%%%%%%%%%%%%%%%%%%%%%%%%%%%%%%%%%%%%%%%%
\subsection{Legacy Detection}
\label{sec:detection}

The directive |\childdocmain| in the main file can detect
whether the complete document or merely a child is to be compiled
even without using the directive |\childdocof|.
This method is deprecated because it is less robust
and there is no compelling reason to use it;
it is merely provided for backward compatibility
and it may be removed in future versions.

If the detection mechanism is to be used,
it is mandatory to correctly specify
the filename of the main file as the argument of |\childdocmain|:
%
\begin{center}
\begin{tabular}{l}
|\input{childdoc.def}|\\
|\childdocmain{|\textit{main}|}|\\
\end{tabular}
\end{center}
%
If |\jobname| does not match the argument \textit{main} of |\childdocmain|,
it is assumed that |\jobname| points to the child file to be compiled.
When using |\childdocmain| with the main file specified as argument,
it suffices to start a child file
with just |\input{|\textit{main}|}|
without loading of the package and using |\childdocof|.
If instead all processing is done
with the appropriate \textsf{childdoc} directives,
the argument of \textit{main} of |\childdocmain| can be empty.

An alternative version of the command line processing described
in \secref{sec:commandline} using the detection mechanism reads:
%
\begin{center}
|... -jobname "|\textit{target}|" "|[\textit{flags}]%
[|\def\jobname{|\textit{dest}|}|]|\input{|\textit{main}|}"|
\end{center}

%%%%%%%%%%%%%%%%%%%%%%%%%%%%%%%%%%%%%%%%%%%%%%%%%%%%%%%%%%%%%%%%%%%%%%%%%%%%%%%%
\subsection{Manual Code}
\label{sec:manual}

In case one cannot be certain whether the definitions file |childdoc.def|
is installed on the target \TeX{} distribution
and one prefers not to ship it,
it is conceivable to paste a few relevant commands into the sources.

To that end, drop all statements |\input{childdoc.def}|
and perform the replacements as outlined below.
Instead of |\childdocmain{|\textit{main}|}| add the following code
to the top of the main file:
%
\begin{center}
\begin{tabular}{l}
|\||ifdefined\childdocname\endinput\||fi\newif\ifchilddoc|\\
|\edef\childdocname{\scantokens\expandafter{\jobname\noexpand}}|\\
|\def\childdocmain{|\textit{main}|}\||ifx\childdocmain\childdocname\||else|\\
|\childdoctrue\includeonly{\childdocname}\let\jobname\childdocmain\||fi|\\
\end{tabular}
\end{center}
%
Instead of |\childdocof{|\textit{main}|}| just include the main file
at the top of each child file:
%
\begin{center}
|\input{|\textit{main}|}|
\end{center}
%
A simple redirection |\childdocforward{|\textit{dest}|}| is achieved by:
%
\begin{center}
|\def\jobname{|\textit{dest}|}\input{\jobname}|
\end{center}
%
The redirection with prefix
|\childdocforwardprefix[|\textit{prefix}|]{|\textit{dest}|}|
is accomplished by:
%
\begin{center}
\begin{tabular}{l}
|{\edef\jobname{\scantokens\expandafter{\jobname\noexpand}}|\\
|\def\redirectjob |\textit{prefix}|#1~~~{\gdef\jobname{|\textit{dest}|#1}}|\\
|\expandafter\redirectjob\jobname~~~}\input{\jobname}|
\end{tabular}
\end{center}

In an alternative approach,
child documents can be compiled by a specific command line
without additional code or specific definitions:
%
\begin{center}
|... -jobname "|\textit{target}|" "|[\textit{flags}]%
|\includeonly{|\textit{dest}|}\input{|\textit{main}|}"|
\end{center}
%

%%%%%%%%%%%%%%%%%%%%%%%%%%%%%%%%%%%%%%%%%%%%%%%%%%%%%%%%%%%%%%%%%%%%%%%%%%%%%%%%
%%%%%%%%%%%%%%%%%%%%%%%%%%%%%%%%%%%%%%%%%%%%%%%%%%%%%%%%%%%%%%%%%%%%%%%%%%%%%%%%
\section{Information}

%%%%%%%%%%%%%%%%%%%%%%%%%%%%%%%%%%%%%%%%%%%%%%%%%%%%%%%%%%%%%%%%%%%%%%%%%%%%%%%%
\subsection{Copyright}

Copyright \copyright{} 2017--2018 Niklas Beisert

This work may be distributed and/or modified under the
conditions of the \LaTeX{} Project Public License, either version 1.3
of this license or (at your option) any later version.
The latest version of this license is in
  \url{http://www.latex-project.org/lppl.txt}
and version 1.3 or later is part of all distributions of \LaTeX{}
version 2005/12/01 or later.

This work has the LPPL maintenance status `maintained'.

The Current Maintainer of this work is Niklas Beisert.

This work consists of the files |README.txt|, |childdoc.ins| and |childdoc.dtx|
as well as the derived files |childdoc.def|, |cdocsamp.tex|
with |cdocsch1.tex|, |cdocsch2.tex|, |cdocspt3.tex|, |cdocspt4.tex|,
|cdocsdrf.tex|, |cdocsfn1.tex|, |cdocsfn2.tex|
as well as |childdoc.pdf|.

%%%%%%%%%%%%%%%%%%%%%%%%%%%%%%%%%%%%%%%%%%%%%%%%%%%%%%%%%%%%%%%%%%%%%%%%%%%%%%%%
\subsection{Files and Installation}

The package consists of the files:
%
\begin{center}
\begin{tabular}{ll}
    |README.txt|   & readme file \\
    |childdoc.ins| & installation file \\
    |childdoc.dtx| & source file \\
    |childdoc.def| & definition file \\
    |cdocsamp.tex| & sample main file \\
    |cdocsch1.tex| & sample include file \\
    |cdocsch2.tex| & sample include file \\
    |cdocspt3.tex| & sample part file \\
    |cdocspt4.tex| & sample part file \\
    |cdocsdrf.tex| & sample redirection file \\
    |cdocsfn1.tex| & sample redirection file \\
    |cdocsfn2.tex| & sample redirection file \\
    |childdoc.pdf| & manual
\end{tabular}
\end{center}
%
The distribution consists of the files
|README.txt|, |childdoc.ins| and |childdoc.dtx|.
%
\begin{itemize}
\item
Run (pdf)\LaTeX{} on |childdoc.dtx|
to compile the manual |childdoc.pdf| (this file).
\item
Run \LaTeX{} on |childdoc.ins| to create the definitions file |childdoc.def|
and the sample |cdocsamp.tex| with include files
|cdocsch1.tex|, |cdocsch2.tex|, |cdocspt3.tex|, |cdocspt4.tex|,
|cdocsdrf.tex|, |cdocsfn1.tex|, |cdocsfn2.tex|.
Then copy the file |childdoc.def| to an appropriate directory of your \LaTeX{}
distribution, e.g.\ \textit{texmf-root}|/tex/latex/childdoc|.
\end{itemize}

%%%%%%%%%%%%%%%%%%%%%%%%%%%%%%%%%%%%%%%%%%%%%%%%%%%%%%%%%%%%%%%%%%%%%%%%%%%%%%%%
\subsection{Related CTAN Packages}

There are several other packages which offer a similar functionality:
%
\begin{itemize}
\item
The packages
\href{http://ctan.org/pkg/docmute}{\textsf{docmute}},
\href{http://ctan.org/pkg/includex}{\textsf{includex}} and
\href{http://ctan.org/pkg/standalone}{\textsf{standalone}}
provide commands to include only the document body of
a child file thus allowing both files to be compiled individually.
\item
The packages \href{http://ctan.org/pkg/subdocs}{\textsf{subdocs}}
and \href{http://ctan.org/pkg/subfiles}{\textsf{subfiles}}
provide structures in which the main and child documents can be
encapsulated and allowing them to be compiled individually.
The inclusion mechanism is different from the conventional |\include|.
\item
The package \href{http://ctan.org/pkg/combine}{\textsf{combine}}
is an elaborate solution to combine several documents into one.
\end{itemize}
%
See also the CTAN topic \href{http://ctan.org/topic/subdocs}{\textsf{subdocs}}
for further related packages.
The present package differs from the above solutions in that
a document structure constructed with the conventional |\include| mechanism
just needs two extra commands at the top of every file
such that all constituent files can be compiled individually.

%%%%%%%%%%%%%%%%%%%%%%%%%%%%%%%%%%%%%%%%%%%%%%%%%%%%%%%%%%%%%%%%%%%%%%%%%%%%%%%%
%\subsection{Feature Suggestions}
%
%The following is a list of features which may be useful for future
%versions of this package:
%%
%\begin{itemize}
%\item
%\ldots
%\end{itemize}

%%%%%%%%%%%%%%%%%%%%%%%%%%%%%%%%%%%%%%%%%%%%%%%%%%%%%%%%%%%%%%%%%%%%%%%%%%%%%%%%
\subsection{Revision History}

%%%%%%%%%%%%%%%%%%%%%%%%%%%%%%%%%%%%%%%%
\paragraph{v2.0:} 2018/12/30

\begin{itemize}
\item
immediate forward processing
\item
added |\childdocby| mechanism
\item
manual restructured
\end{itemize}

%%%%%%%%%%%%%%%%%%%%%%%%%%%%%%%%%%%%%%%%
\paragraph{v1.6:} 2018/01/17

\begin{itemize}
\item
application for development of include files
\item
corrections to manual
\end{itemize}

%%%%%%%%%%%%%%%%%%%%%%%%%%%%%%%%%%%%%%%%
\paragraph{v1.5:} 2017/05/21

\begin{itemize}
\item
more complete structuring introduced
\item
|\childdocof| introduced
\item
|\childdoc| renamed to |\childdocmain|
\item
|\childredirect| renamed to |\childdocforward| and |\childdocforwardprefix|
and functionality expanded
\end{itemize}

%%%%%%%%%%%%%%%%%%%%%%%%%%%%%%%%%%%%%%%%
\paragraph{v1.0:} 2017/04/27

\begin{itemize}
\item
manual and install package
\item
first version published on CTAN
\end{itemize}

%%%%%%%%%%%%%%%%%%%%%%%%%%%%%%%%%%%%%%%%
\paragraph{v0.6:} 2017/04/26

\begin{itemize}
\item
redirection mechanism added
\end{itemize}

%%%%%%%%%%%%%%%%%%%%%%%%%%%%%%%%%%%%%%%%
\paragraph{v0.5:} 2017/04/26

\begin{itemize}
\item
functionality in definition file
\end{itemize}


%%%%%%%%%%%%%%%%%%%%%%%%%%%%%%%%%%%%%%%%%%%%%%%%%%%%%%%%%%%%%%%%%%%%%%%%%%%%%%%%
%%%%%%%%%%%%%%%%%%%%%%%%%%%%%%%%%%%%%%%%%%%%%%%%%%%%%%%%%%%%%%%%%%%%%%%%%%%%%%%%
%%%%%%%%%%%%%%%%%%%%%%%%%%%%%%%%%%%%%%%%%%%%%%%%%%%%%%%%%%%%%%%%%%%%%%%%%%%%%%%%
\appendix

\settowidth\MacroIndent{\rmfamily\scriptsize 000\ }

 \DocInput{childdoc.dtx}

\end{document}
%</driver>
% \fi
%
% %%%%%%%%%%%%%%%%%%%%%%%%%%%%%%%%%%%%%%%%%%%%%%%%%%%%%%%%%%%%%%%%%%%%%%%%%%%%%%
% %%%%%%%%%%%%%%%%%%%%%%%%%%%%%%%%%%%%%%%%%%%%%%%%%%%%%%%%%%%%%%%%%%%%%%%%%%%%%%
% \section{Sample}
%\iffalse
%<*samplemain>
%\fi
%
% The following presents a sample document
% with two chapters, two parts, a title page,
% a compile flag as well as three forwarding files to set the flag.
% It consists of eight |.tex| files:
% \begin{center}
% \begin{tabular}{ll}
% |cdocsamp.tex|&main file\\
% |cdocsch1.tex|&include file for chapter 1\\
% |cdocsch2.tex|&include file for chapter 2\\
% |cdocspt3.tex|&include file for part 3\\
% |cdocspt4.tex|&include file for part 4\\
% |cdocsdrf.tex|&forwarding file for main file in draft mode\\
% |cdocsfi1.tex|&forwarding file for final version of chapter 1\\
% |cdocsfi2.tex|&forwarding file for final version of chapter 2\\
% \end{tabular}
% \end{center}
% Each of the eight files can be compiled directly by the \LaTeX{} compiler.
%
% %%%%%%%%%%%%%%%%%%%%%%%%%%%%%%%%%%%%%%
% \paragraph{Main File.}
%
% The main file is called |cdocsamp.tex|.
%
% Load the \textsf{childdoc} definitions and
% declare the filename for the main document:
%    \begin{macrocode}
\input{childdoc.def}
\childdocmain{}
%    \end{macrocode}

% Optional override for |\version| flag:
%    \begin{macrocode}
%%\ifchilddoc\else\providecommand{\version}{draft}\fi
%    \end{macrocode}

% Define the default values for the |\version| flag
% (|final| for the main file and |draft| for childs):
%    \begin{macrocode}
\ifchilddoc
\providecommand{\version}{draft}
\else
\providecommand{\version}{final}
\fi
%    \end{macrocode}

% Load the standard document class:
%    \begin{macrocode}
\documentclass[12pt]{article}
%    \end{macrocode}

% Start the document body:
%    \begin{macrocode}
\begin{document}
%    \end{macrocode}

% Declare a title page.
% Print title, part of document being processed and version flag:
%    \begin{macrocode}
\addtocounter{page}{-1}
\begin{center}
{\LARGE\bfseries{}childdoc example\par}
\vspace{1cm}
\ifchilddoc
\ifchilddocmanual part\else chapter\fi:
`\childdocname' of `\childdocjob'\par
\else
main document: `\childdocjob'\par
\fi
version: \version\par
\end{center}
\newpage
%    \end{macrocode}

% Manually include selected file,
% otherwise process as usual:
%    \begin{macrocode}
\ifchilddocmanual
\section*{part `\childdocname'}
\input{\childdocname}
\else
%    \end{macrocode}

% Include the two chapters:
%    \begin{macrocode}
\include{cdocsch1}
\include{cdocsch2}
%    \end{macrocode}

% Include the two parts unless only chapters should be displayed:
%    \begin{macrocode}
\ifchilddoc\else
\section{part three}
\input{cdocspt3}
\section{part four}
\input{cdocspt4}
\fi
%    \end{macrocode}

% Process as usual until here:
%    \begin{macrocode}
\fi
%    \end{macrocode}

% End of document body:
%    \begin{macrocode}
\end{document}
%    \end{macrocode}
%\iffalse
%</samplemain>
%\fi
%
% %%%%%%%%%%%%%%%%%%%%%%%%%%%%%%%%%%%%%%
% \paragraph{Chapter Include Files.}
%
% The include files are called |cdocsch1.tex| and |cdocsch2.tex|.
%
%\iffalse
%<*samplechap1|samplechap2>
%\fi

% Optional override for |\version| flag:
%    \begin{macrocode}
%%\providecommand{\version}{final}
%    \end{macrocode}

% Include the main document:
%    \begin{macrocode}
\input{childdoc.def}
\childdocof{cdocsamp}
%    \end{macrocode}

%\iffalse
%</samplechap1|samplechap2>
%\fi
%
%\iffalse
%<*samplechap1>
%\fi
% Some text for chapter 1:
%    \begin{macrocode}
\section{one}
some text in chapter one
%    \end{macrocode}

%\iffalse
%</samplechap1>
%\fi
% Some text for chapter 2:
%\iffalse
%<*samplechap2>
%\fi
%    \begin{macrocode}
\section{two}
more text in chapter two
%    \end{macrocode}

%\iffalse
%</samplechap2>
%\fi
%
% %%%%%%%%%%%%%%%%%%%%%%%%%%%%%%%%%%%%%%
% \paragraph{Part Include Files.}
%
% The include files are called |cdocspt3.tex| and |cdocspt4.tex|.
%
%\iffalse
%<*samplepart3|samplepart4>
%\fi

% Optional override for |\version| flag:
%    \begin{macrocode}
%%\providecommand{\version}{final}
%    \end{macrocode}

% Include the main document:
%    \begin{macrocode}
\input{childdoc.def}
\childdocby{cdocsamp}
%    \end{macrocode}

%\iffalse
%</samplepart3|samplepart4>
%\fi
%
%\iffalse
%<*samplepart3>
%\fi
% Some text for part 3:
%    \begin{macrocode}
some text in part three
%    \end{macrocode}

%\iffalse
%</samplepart3>
%\fi
% Some text for part 4:
%\iffalse
%<*samplepart4>
%\fi
%    \begin{macrocode}
more text in part four
%    \end{macrocode}

%\iffalse
%</samplepart4>
%\fi
%
% %%%%%%%%%%%%%%%%%%%%%%%%%%%%%%%%%%%%%%
% \paragraph{Forwarding for a Complete Draft.}
%
% The following forwarding file |cdocsdrf.tex|
% compiles the main document in draft mode:
%\iffalse
%<*sampledraft>
%\fi
%    \begin{macrocode}
\def\version{draft}
\input{childdoc.def}
\childdocforward{cdocsamp}
%    \end{macrocode}

%\iffalse
%</sampledraft>
%\fi
%
% %%%%%%%%%%%%%%%%%%%%%%%%%%%%%%%%%%%%%%
% \paragraph{Forwarding for Final Version of the Chapters.}
%
% The following forwarding files |cdocsfn1.tex| and |cdocsfn2.tex|
% (with identical content)
% compile the final versions of the child documents
% |cdocsch1.tex| and |cdocsch2.tex|, respectively:
%\iffalse
%<*samplefinal>
%\fi
%    \begin{macrocode}
\def\version{final}
\input{childdoc.def}
\childdocforwardprefix[cdocsamp]{cdocsfn}{cdocsch}
%    \end{macrocode}

%\iffalse
%</samplefinal>
%\fi
%
% %%%%%%%%%%%%%%%%%%%%%%%%%%%%%%%%%%%%%%
% \paragraph{Command Line Processing.}
%
% The following three command lines generate the output files
% |cdocscld|, |cdocscl1| and |cdocscl2|
% which should be identical to
% |cdocsdrf|, |cdocsch1| and |cdocsfn2|, respectively:
% \begin{center}
% \begin{tabular}{l}
% |latex -jobname cdocscld \|\\
% |  "\def\version{draft}\input{childdoc.def}\childdocforward{cdocsamp}"|\\
% |latex -jobname cdocscl1 \|\\
% |  "\input{childdoc.def}\childdocforward[cdocsamp]{cdocsch1}"|\\
% |latex -jobname cdocscl2 \|\\
% |  "\def\version{final}\input{childdoc.def}\childdocforward{cdocsch2}"|
% \end{tabular}
% \end{center}
% Note that the trailing backslash on each first line
% merely continues the input to the second line
% (for convenient cut ant paste).
% Furthermore, the command |latex| can be replaced by any
% of its alternative versions such as |pdflatex|.
%
% %%%%%%%%%%%%%%%%%%%%%%%%%%%%%%%%%%%%%%%%%%%%%%%%%%%%%%%%%%%%%%%%%%%%%%%%%%%%%%
% %%%%%%%%%%%%%%%%%%%%%%%%%%%%%%%%%%%%%%%%%%%%%%%%%%%%%%%%%%%%%%%%%%%%%%%%%%%%%%
% \section{Implementation}
%\iffalse
%<*package>
%\fi
%
% This section describes the definitions file |childdoc.def|.

% The definitions cannot be loaded using |\usepackage| or |\RequirePackage|
% which has a mechanism to prevent loading a style file more than once.
% When loading the definitions by means of |\input|
% multiple instances have to be prevented manually:
%\iffalse
%This code needs to be before the `\ProvidesFile' directive
%which is defined at the beginning of this file.
%Therefore it is also placed there and commented out here.
%</package>
%<*discard>
%\fi
%    \begin{macrocode}
\ifdefined\childdocmain\endinput\fi
%    \end{macrocode}
%\iffalse
%</discard>
%<*package>
%\fi
%
% \macro{\ifchilddoc}
% \macro{\ifchilddocmanual}
% The conditional |\ifchilddoc| tells whether a
% child (true) or main (false) document is being compiled.
% The conditional |\ifchilddocmanual| tells whether
% the |\includeonly| mechanism is used (false) or
% the selection of child files must be performed manually (true).
% The definitions initialise to false:
%    \begin{macrocode}
\newif\ifchilddoc
\newif\ifchilddocmanual
%    \end{macrocode}

% \macro{\childdocname}
% \macro{\childdocjob}
% The macro |\childdocname| stores the name of the main document
% to be compiled. The macro |\childdocjob| stores the name of
% the document on which the \LaTeX{} compiler was originally invoked.
% The content of |\jobname| cannot be compared
% to filenames specified in the source due to different catcodes.
% The following code rescans |\jobname|, stores the result
% in |\childdocname| and saves a copy in |\childdocjob|:
%    \begin{macrocode}
\edef\childdocname{\scantokens\expandafter{\jobname\noexpand}}
\let\childdocjob\childdocname
%    \end{macrocode}

% \macro{\childdocdisable}
% The macro |\childdocdisable| prevents the main file
% from being processed more than once.
% At this stage, the main document command |\childdocmain|
% is assumed to be called once again where it should do nothing.
% Any subsequent call to it should prevent
% a secondary processing of the main document
% It overwrites the forwarding commands
% |\childdocof| and |\childdocforward|
% with empty macros to prevent further inclusions of the main document:
%    \begin{macrocode}
\newcommand{\childdocdisable}
{
  \renewcommand{\childdocmain}[1]{\renewcommand{\childdocmain}[1]{\endinput}}
  \renewcommand{\childdocof}[1]{}
  \renewcommand{\childdocby}[2][]{}
  \renewcommand{\childdocforward}[2][]{}
  \renewcommand{\childdocdisable}{}
}
%    \end{macrocode}

% \macro{\childdocmain}
% The macro |\childdocmain| is to be called at the top of the main file
% with nothing or the main filename (without extension) as argument.
% First, it breaks loops.
% If the argument is not empty and does not match |\childdocname|
% (which is set by the first inclusion of |childdoc.def|),
% |\ifchilddoc| is set to true, |\includeonly| is applied to the child file
% and |\jobname| is set to the main file
% (for proper handling of |.aux| files):
%    \begin{macrocode}
\newcommand{\childdocmain}[1]
{
  \childdocdisable\childdocmain{}
  \if?#1?\else
    \begingroup
      \def\childdoctmp{#1}
      \ifx\childdoctmp\childdocname
        \def\childdoctmp{}
      \else
        \def\childdoctmp
        {
          \childdoctrue
          \includeonly{\childdocname}
          \def\childdocjob{#1}
          \def\jobname{#1}
        }
      \fi
      \expandafter
    \endgroup
    \childdoctmp
  \fi
}
%    \end{macrocode}

% \macro{\childdocof}
% The command |\childdocof| redirects
% compilation to the main file |#1|.
%    \begin{macrocode}
\newcommand{\childdocof}[1]
{
  \childdocdisable
  \childdoctrue
  \includeonly{\childdocname}
  \def\jobname{#1}
  \def\childdocjob{#1}
  \input{#1}
}
%    \end{macrocode}

% \macro{\childdocby}
% The command |\childdocby| ....
%    \begin{macrocode}
\newcommand{\childdocby}[2][]
{
  \childdocdisable
  \childdoctrue
  \childdocmanualtrue
  \if?#1?\else
    \def\jobname{#2}
  \fi
  \def\childdocjob{#2}
  \input{#2}
  \endinput
}
%    \end{macrocode}

% \macro{\childdocforward}
% The command |\childdocforward| redirects
% compilation to the main file or
% (if the optional argument is given) a child file.
% Parameters are set as if the main file
% or a child file starting with |\childdocof| was compiled.
% Then compilation is handed over to the main file:
%    \begin{macrocode}
\newcommand{\childdocforward}[2][]
{
  \begingroup
    \if?#1?
      \def\childdoctmp
      {
        \def\childdocname{#2}
        \def\childdocjob{#2}
        \def\jobname{#2}
        \input{#2}
        \endinput
      }
    \else
      \def\childdoctmp
      {
        \childdocdisable
        \def\childdocname{#2}
        \childdoctrue
        \includeonly{#2}
        \def\childdocjob{#1}
        \def\jobname{#1}
        \input{#1}
        \endinput
      }
    \fi
    \expandafter
  \endgroup
  \childdoctmp
}
%    \end{macrocode}

% \macro{\childdocforwardprefix}
% The command |\childdocforwardprefix| redirects
% compilation to the main or a child file by means of a pattern.
% The prefix |#1| in the current filename is replaced by |#2|
% and the suffix of the current filename is kept
% (it is assumed that the filename does not contain the substring `|~~~|'
% which is used as a delimiter).
% Compilation is handed over to the new file by |\childdocforward|:
%    \begin{macrocode}
\newcommand{\childdocforwardprefix}[3][]
{
  \begingroup
    \def\childdocextract #2##1~~~{\def\childdoctmp{\childdocforward[#1]{#3##1}}}
    \expandafter\childdocextract\childdocname~~~
    \expandafter
  \endgroup
  \childdoctmp
}
%    \end{macrocode}

% \macro{\childdoc}
% The deprecated macro |\childdoc| is a legacy version of |\childdocmain|:
%    \begin{macrocode}
\newcommand{\childdoc}{\childdocmain}
%    \end{macrocode}

% \macro{\childdocredirect}
% The deprecated macro |\childdocredirect| is a legacy version
% of |\childdocforward| and |\childdocforwardprefix|:
%    \begin{macrocode}
\newcommand{\childdocredirect}[2][]
{
  \begingroup
    \if?#1?
      \def\childdoctmp{\childdocforward{#2}}
    \else
      \def\childdoctmp{\childdocforwardprefix{#1}{#2}}
    \fi
    \expandafter
  \endgroup
  \childdoctmp
}
%    \end{macrocode}

%\iffalse
%</package>
%\fi
%
\endinput
\childdocforward[cdocsamp]{cdocsch1}"|\\
% |latex -jobname cdocscl2 \|\\
% |  "\def\version{final}% \iffalse
%
% childdoc.dtx Copyright (C) 2017-2018 Niklas Beisert
%
% This work may be distributed and/or modified under the
% conditions of the LaTeX Project Public License, either version 1.3
% of this license or (at your option) any later version.
% The latest version of this license is in
%   http://www.latex-project.org/lppl.txt
% and version 1.3 or later is part of all distributions of LaTeX
% version 2005/12/01 or later.
%
% This work has the LPPL maintenance status `maintained'.
%
% The Current Maintainer of this work is Niklas Beisert.
%
% This work consists of the files childdoc.dtx and childdoc.ins
% and the derived files childdoc.def and cdocsamp.tex with
% cdocsch1.tex, cdocsch2.tex, cdocsdrf.tex, cdocsfn1.tex, cdocsfn2.tex.
%
%<package>\ifdefined\childdocmain\endinput\fi
%<package>\ProvidesFile{childdoc.def}[2018/12/30 v2.0 child document driver]
%<samplemain>\ProvidesFile{cdocsamp.tex}[2018/12/30 v2.0 sample for childdoc]
%<*driver>
%\ProvidesFile{childdoc.drv}[2018/12/30 v2.0 childdoc reference manual file]
\PassOptionsToClass{10pt,a4paper}{article}
\documentclass{ltxdoc}

\usepackage[margin=35mm]{geometry}
\usepackage{hyperref}
\usepackage{hyperxmp}
\usepackage[usenames]{color}

\hypersetup{colorlinks=true}
\hypersetup{pdfstartview=FitH}
\hypersetup{pdfpagemode=UseNone}
\hypersetup{pdfsource={}}
\hypersetup{pdflang={en-UK}}
\hypersetup{pdfcopyright={Copyright 2017-2018 Niklas Beisert.
  This work may be distributed and/or modified under the
  conditions of the LaTeX Project Public License, either version 1.3
  of this license or (at your option) any later version.}}
\hypersetup{pdflicenseurl={http://www.latex-project.org/lppl.txt}}
\hypersetup{pdfcontactaddress={ETH Zurich, ITP, HIT K,
  Wolfgang-Pauli-Strasse 27}}
\hypersetup{pdfcontactpostcode={8093}}
\hypersetup{pdfcontactcity={Zurich}}
\hypersetup{pdfcontactcountry={Switzerland}}
\hypersetup{pdfcontactemail={nbeisert@itp.phys.ethz.ch}}
\hypersetup{pdfcontacturl={http://people.phys.ethz.ch/\xmptilde nbeisert/}}

\newcommand{\secref}[1]{\hyperref[#1]{section \ref*{#1}}}

\parskip1ex
\parindent0pt
\let\olditemize\itemize
\def\itemize{\olditemize\parskip0pt}

\begin{document}

\title{The \textsf{childdoc} Package}
\hypersetup{pdftitle={The childdoc Package}}
\author{Niklas Beisert\\[2ex]
  Institut f\"ur Theoretische Physik\\
  Eidgen\"ossische Technische Hochschule Z\"urich\\
  Wolfgang-Pauli-Strasse 27, 8093 Z\"urich, Switzerland\\[1ex]
  \href{mailto:nbeisert@itp.phys.ethz.ch}
  {\texttt{nbeisert@itp.phys.ethz.ch}}}
\hypersetup{pdfauthor={Niklas Beisert}}
\hypersetup{pdfsubject={Manual for the LaTeX2e Package childdoc}}
\date{30 December 2018, \textsf{v2.0}}
\maketitle

\begin{abstract}\noindent
\textsf{childdoc} is a \LaTeXe{} package
that enables the direct compilation
of document sections included by |\include|
to individual files.
\end{abstract}

\begingroup
\parskip0ex
\tableofcontents
\endgroup

%%%%%%%%%%%%%%%%%%%%%%%%%%%%%%%%%%%%%%%%%%%%%%%%%%%%%%%%%%%%%%%%%%%%%%%%%%%%%%%%
%%%%%%%%%%%%%%%%%%%%%%%%%%%%%%%%%%%%%%%%%%%%%%%%%%%%%%%%%%%%%%%%%%%%%%%%%%%%%%%%
\section{Introduction}

\LaTeX{} provides a mechanism to structure a large document (such as a book)
into a main file and several child files (containing the chapters)
using the |\include| command.
This mechanism is beneficial for documents
which span hundreds of pages in order to
make the source file(s) more manageable.
Moreover, compilation can be restricted to
selected child files by means of the |\includeonly| command.
The latter feature can be used to reduce the compilation time while editing
(this was significantly more useful in the earlier days of \LaTeX{})
or to generate a smaller document which is easier to navigate.
Another application of |\includeonly| is to generate
documents consisting of selected parts of the complete document.

However, there are a few drawbacks of the plain |\include| mechanism:
\begin{itemize}
\item
The child files cannot be compiled on their own,
they can only be compiled via the main file.
A naive editing environment
(such as a text editor with an option
to have the current file processed by \LaTeX)
may require one to switch to the main file before compiling;
attempting to compile the child file produces errors.
\item
The main file must be modified (each time)
to adjust the |\includeonly| command
to the present needs. This easily leaves the main file in a messy state.
\item
The generated document will always carry the filename
of the main document. This is inconvenient if
several child files are to be compiled and
to be kept for distribution.
\end{itemize}

The present package provides a simple interface
to make child files individually compilable by \LaTeX{}.
Compiling a child file then has the same effect as compiling
the main file with an |\includeonly| command
to select the appropriate child.
Moreover the generated document will carry the name of the child
rather than the main file.
This resolves all three above issues.

This feature is meant to make the editing of books,
thesis documents and lecture notes somewhat more convenient.
However, the package can also be used efficiently for
composing a series of documents (such as exercise sheets)
which are typically distributed individually.
It then assists the author in generating the individual documents
(potentially in different versions)
as well as a document containing the collected series.
Another application is in developing style files
or other kinds of included material
where compilation of the style file could redirect
to a sample or test file.

%%%%%%%%%%%%%%%%%%%%%%%%%%%%%%%%%%%%%%%%%%%%%%%%%%%%%%%%%%%%%%%%%%%%%%%%%%%%%%%%
%%%%%%%%%%%%%%%%%%%%%%%%%%%%%%%%%%%%%%%%%%%%%%%%%%%%%%%%%%%%%%%%%%%%%%%%%%%%%%%%
\section{Usage}

First of all, the package \textsf{childdoc} is \emph{not} a standard
\LaTeXe{} |.sty| style file! Therefore it needs to be invoked in
a non-standard way.

%%%%%%%%%%%%%%%%%%%%%%%%%%%%%%%%%%%%%%%%%%%%%%%%%%%%%%%%%%%%%%%%%%%%%%%%%%%%%%%%
\subsection{Included Files}
\label{sec:include}

%%%%%%%%%%%%%%%%%%%%%%%%%%%%%%%%%%%%%%%%
\DescribeMacro{\childdocmain}
To use the package, add the commands
\begin{center}
\begin{tabular}{l}
|\input{childdoc.def}|\\
|\childdocmain{}|\\
\end{tabular}
\end{center}
at the very top of the main \LaTeX{} file,
in particular \emph{before} the |\documentclass| statement!
The argument of |\childdocmain| should be left empty
(but it must be present).

%%%%%%%%%%%%%%%%%%%%%%%%%%%%%%%%%%%%%%%%
\DescribeMacro{\childdocof}
Furthermore, add the commands
\begin{center}
\begin{tabular}{l}
|\input{childdoc.def}|\\
|\childdocof{|\textit{main}|}|\\
\end{tabular}
\end{center}
at the top of every child file \textit{child}
which is included by |\include{|\textit{child}|}|
from within the main file
(or at least for those files to be compiled individually).
The argument \textit{main} must be the filename of the main file.

There are a couple of
considerations in setting up the main and child documents:

%%%%%%%%%%%%%%%%%%%%%%%%%%%%%%%%%%%%%%%%
\paragraph{Restrictions.}

Please note the following restrictions:
\begin{itemize}
\item
|\childdocmain| must be called with one argument \textit{main}
to ensure compatibility with earlier version of the package.
It must either be empty (|\childdocmain{}|)
or precisely match the filename of the main file in which it is specified.
See \secref{sec:detection} for further information.
\item
The filename \textit{main} must be specified without the |.tex| extension.
\item
The filename \textit{main} is case sensitive
(even in case-insensitive file systems)
due to internal string comparison.
\item
The argument \textit{main} should be fully expanded, it cannot be a macro.
\item
Subdirectories and special characters should be avoided in filenames.
\item
The command |\childdocmain{|\textit{main}|}| must be followed by a whitespace.
It should not be followed immediately by another command
or by a comment mark `|%|'.
This is because the \TeX{} parser reads the token immediately following
the argument of |\childdocmain| and puts it
at the beginning of every child section;
however, a white\-space is ignored.
\end{itemize}

%%%%%%%%%%%%%%%%%%%%%%%%%%%%%%%%%%%%%%%%
\paragraph{Content of Main File.}

It is advisable to place all content in the child files included by |\include|.
Any output contained in the main file will appear in all child documents
unless suppressed manually;
it cannot be suppressed automatically by the |\includeonly| directive
and thus should normally be avoided.
A method to include some content in the main file
by means of conditional processing is described in \secref{sec:conditional}.

%%%%%%%%%%%%%%%%%%%%%%%%%%%%%%%%%%%%%%%%
\paragraph{Page Numbering.}

When only a part of the document is compiled,
the appropriate numbering of pages
(as well as other status parameters)
is determined from the |.aux| files.
The latter contain information from previous passes.
However this information needs to propagate through
all intermediate child documents.
Therefore the page numbering in child documents may well
be inconsistent until the complete document is compiled at least once.

A useful (if unconventional) way to always ensure a consistent
page numbering is to restart the numbering in each child document
and denote the pages by `\textit{child}|.|\textit{page}'
where \textit{child} represents the chapter/section number of the child file.
This can be achieved by the command
|\numberwithin{page}{|\textit{child}|}|
of the \textsf{amsmath} package
where \textit{child} can be |chapter| or |section|
depending on the chosen structuring.
Alternatively, one can modify the macro |\thepage| appropriately
and reset the counter |page| at the start of each child file.

%%%%%%%%%%%%%%%%%%%%%%%%%%%%%%%%%%%%%%%%%%%%%%%%%%%%%%%%%%%%%%%%%%%%%%%%%%%%%%%%
\subsection{Conditional Processing}
\label{sec:conditional}

The package provides a mechanism to compile different versions
of a document. To customise the versions further some conditional processing
can come in handy to distinguish which version is being compiled.
The package provides two macros to describe the compilation context:

%%%%%%%%%%%%%%%%%%%%%%%%%%%%%%%%%%%%%%%%
\DescribeMacro{\ifchilddoc}
The conditional |\ifchilddoc| distinguishes between the compilation of
child documents and the main document:
%
\begin{center}
|\ifchilddoc |\textit{child-code}| |[|\||else |\textit{main-code}]| \||fi|
\end{center}

%%%%%%%%%%%%%%%%%%%%%%%%%%%%%%%%%%%%%%%%
\DescribeMacro{\childdocname}
\DescribeMacro{\childdocjob}
The macro |\childdocname| contains the filename (without extension)
of the main or child file being processed.
Note that |\childdocjob| will always contain the name of the main file.

%%%%%%%%%%%%%%%%%%%%%%%%%%%%%%%%%%%%%%%%
\paragraph{Title Page.}

Conditional processing can be used to include a title or banner page
in the main document when proper precautions are taken.
Importantly, the code in the main file should ensure that the page counter
(as well as other status parameters which are stored in the |.aux| files)
takes the same value after the conditional processing.
Otherwise the page numbers may take divergent values
depending on which part is compiled.

For example, a title page could be declared by:
%
\begin{center}
\begin{tabular}{l}
|\ifchilddoc\||else|\\
|\addtocounter{page}{-1}|\\
\textit{code for title page}\\
|\newpage|\\
|\||fi|
\end{tabular}
\end{center}
%
A banner page for the child documents can be generated by:
%
\begin{center}
\begin{tabular}{l}
|\ifchilddoc|\\
|\addtocounter{page}{-1}|\\
\textit{code for banner page}\\
|\newpage|\\
|\||fi|
\end{tabular}
\end{center}
%
Here one could write a message such as:
\begin{center}
|This is the part \childdocname{} of \childdocjob{}.|
\end{center}

%%%%%%%%%%%%%%%%%%%%%%%%%%%%%%%%%%%%%%%%%%%%%%%%%%%%%%%%%%%%%%%%%%%%%%%%%%%%%%%%
\subsection{Flags}
\label{sec:flags}

The package makes it easy to generate different versions
of the main or child documents.
To this end compilation flags can be defined
and assigned different default values.
They will be particularly useful in conjunction
with the forwarding mechanism described in \secref{sec:forward}.

For example, it may be useful to have a flag |\version|
which can be set to |draft| or |final|.
The document source will contain some conditional code
depending on the value of |\version|.
Suppose further, the flag should default to |final| for the main file
and to |draft| for child files
which is a natural assignment for editing the document.
This is achieved by placing the following code
in the preamble of the main document
(below the |\childdocmain| directive):
%
\begin{center}
\begin{tabular}{l}
|\ifchilddoc|\\
|\providecommand{\version}{draft}|\\
|\||else|\\
|\providecommand{\version}{final}|\\
|\||fi|
\end{tabular}
\end{center}
%
The definition by |\providecommand| makes sure
that previous definitions are not overwritten.
Further statements |\providecommand{\version}{...}|
can thus be added before the above code to override it.

For the main file, one might add a line
(between |\childdocmain| and the above block)
%
\begin{center}
|%\ifchilddoc\||else\providecommand{\version}{draft}\||fi|
\end{center}
%
which can be uncommented to produce a draft version.
Likewise one can add a line to the very top of a child file
(above the |\childdocof{|\textit{main}|}| directive)
%
\begin{center}
|%\providecommand{\version}{final}|
\end{center}
%
which can be uncommented to produce the final version of this child document.

%%%%%%%%%%%%%%%%%%%%%%%%%%%%%%%%%%%%%%%%%%%%%%%%%%%%%%%%%%%%%%%%%%%%%%%%%%%%%%%%
\subsection{Forwarding}
\label{sec:forward}

Different versions of the main or child documents
using compilation flags as described in \secref{sec:flags}
can be (permanently) stored in different files
for convenient compilation, viewing and distribution.
To this end, the package defines a command
to pass on compilation to a different file:

%%%%%%%%%%%%%%%%%%%%%%%%%%%%%%%%%%%%%%%%
\DescribeMacro{\childdocforward}
The command |\childdocforward| redirects processing to
another source file:
%
\begin{center}
\begin{tabular}{l}
|\input{childdoc.def}|\\
|\childdocforward[|\textit{main}|]{|\textit{dest}|}|\\
\end{tabular}
\end{center}
%
The argument \textit{dest} is the destination file
(without extension).
It should be the main file or one of the child files.
Note that further \textsf{childdoc} directives
such as |\childdocof| and |\childdocforward|
in the indicated file will be processed in this form.
The optional argument \textit{main}
passes on directly to the main file \textit{main}
while pretending to compile the child \textit{dest}.
This form behaves as if \textit{dest}
issues |\childdocof{|\textit{main}|}| right away,
and no further \textsf{childdoc} directives will be processed.

%%%%%%%%%%%%%%%%%%%%%%%%%%%%%%%%%%%%%%%%
\DescribeMacro{\...prefix}
In the alternative form |\childdocforwardprefix|,
%
\begin{center}
\begin{tabular}{l}
|\input{childdoc.def}|\\
|\childdocforwardprefix[|\textit{main}|]{|\textit{prefix}|}{|\textit{dest}|}|
\end{tabular}
\end{center}
%
the destination file is determined by a pattern
depending on the current file:
To make this work, the current file must be called
`{\textit{prefix}\hspace{0.2em}\textit{suffix}}'
with \textit{prefix} matching precisely the argument.
Processing is then passed on to the file
`{\textit{dest}\hspace{0.2em}\textit{suffix}}'.
Surely, the same effect is achieved by
directly specifying the
argument `{\textit{dest}\hspace{0.2em}\textit{suffix}}'
in the first form.
However, that requires to set up a different file
for each child. With the alternative form of the command
all these files can have exactly the same content
which simplifies setting them up and maintaining them.

For example, the following file |draft.tex|
with a compilation flag |\version| as described in \secref{sec:flags}
compiles the main document as a draft:
%
\begin{center}
\begin{tabular}{l}
|\def\version{draft}|\\
|\input{childdoc.def}|\\
|\childdocforward{|\textit{main}|}|
\end{tabular}
\end{center}
%
Likewise, the following files |final|\textit{nn}|.tex|
compile the final version of the child document
|child|\textit{nn}|.tex|:
%
\begin{center}
\begin{tabular}{l}
|\def\version{final}|\\
|\input{childdoc.def}|\\
|\childdocforwardprefix{final}{child}|
\end{tabular}
\end{center}
%

Note that when several versions of a main file and/or of each child file
are to be generated, it may be convenient to set up a |Makefile| or
shell script to automatise the process.

%%%%%%%%%%%%%%%%%%%%%%%%%%%%%%%%%%%%%%%%%%%%%%%%%%%%%%%%%%%%%%%%%%%%%%%%%%%%%%%%
\subsection{Command Line Processing}
\label{sec:commandline}

The effect of redirection files can also be achieved by invoking
the \LaTeX{} compiler with a more elaborate command line.
Most conveniently this should be done as part
of a shell script or a |Makefile|.

When using \textsf{childdoc} in the main file, the following
command lines effectively perform a redirection
(note that depending on the shell being used,
backslashes may have to be doubled: `|\|' $\to$ `|\\|'):
%
\begin{center}
|... -jobname "|\textit{target}|" |\\|"|[\textit{flags}]%
|\input{childdoc.def}\childdocforward[|\textit{main}|]{|\textit{dest}|}"|
\end{center}
%
Here \textit{target} is the name of the output file,
\textit{main} is the name of the main file
and \textit{dest} is the name of the main or child file to be processed
(all filenames without extensions).
The optional argument \textit{main} can be omitted
if \textit{main} matches \textit{dest}.
Optionally, compilation \textit{flags} can be defined via |\def| commands.
This command line makes the \TeX{} engine believe
it is compiling the file \textit{target}
whose content is specified as the latter parameter.
The provided code then forwards the processing to
\textit{main} or \textit{dest} as described in \secref{sec:forward}.

%%%%%%%%%%%%%%%%%%%%%%%%%%%%%%%%%%%%%%%%%%%%%%%%%%%%%%%%%%%%%%%%%%%%%%%%%%%%%%%%
\subsection{Include by Input}
\label{sec:input}

Including child documents by |\include| has some restrictions by design.
Most notably, the content of a child document always occupies
its own set of pages; pages cannot be shared between child documents.
Usually, this behaviour makes perfect sense
because each child document contain an essential part of the document.
However, in some situations it may be desirable to compose
a document from a collection of parts
without having mandatory page breaks between then.
For this case, the package
provides a mechanism to include parts
by |\input| which can also be processed individually.
However, by construction this mechanism
requires manual handling of the content to be output.

%%%%%%%%%%%%%%%%%%%%%%%%%%%%%%%%%%%%%%%%
\DescribeMacro{\ifchilddocmanual}
The main file should be prepared as usual, see \secref{sec:include}.
However, the document body must make a distinction
between processing of an individual part and of the main document, e.g.:
%
\begin{center}
\begin{tabular}{l}
|\ifchilddocmanual|\\
|\input{\childdocname}|\\
|\||else|\\
\textit{document body with }|\input{|\textit{part}|}|\\
|\||fi|
\end{tabular}
\end{center}
%
The conditional |\ifchilddocmanual| is true whenever
a part to be included by |\input| is being compiled,
and the name of the part is stored in |\childdocname|.

%%%%%%%%%%%%%%%%%%%%%%%%%%%%%%%%%%%%%%%%
\DescribeMacro{\childdocby}
Each part to be included by |\input| should start with:
%
\begin{center}
\begin{tabular}{l}
|\input{childdoc.def}|\\
|\childdocby{|\textit{main}|}|\\
\end{tabular}
\end{center}
%
The directive |\childdocby| is similar to |\childdocof|
described in \secref{sec:include},
but the subsequent selection of content must be done manually.
To that end, both |\ifchilddoc| and |\ifchilddocmanual|
will be true upon processing of a part,
and the name of the part is stored in |\childdocname|.
Note that |\jobname| will be set to the filename of the current part
so that each part receives an individual |.aux| file
that does not interfere with the |.aux| file(s) of the main document.
This behaviour can be altered by the alternative form
|\childdocby[*]{|\textit{main}|}| (with a non-empty optional argument)
which uses the |.aux| file of the main document
by setting |\jobname| to \textit{main}.

%%%%%%%%%%%%%%%%%%%%%%%%%%%%%%%%%%%%%%%%%%%%%%%%%%%%%%%%%%%%%%%%%%%%%%%%%%%%%%%%
\subsection{Driver Development}
\label{sec:driver}

The \textsf{childdoc} mechanism can also be use for the development
of definition files such as \LaTeX{} styles or classes.
This case differs from the above setup with multiple parts
included by |\include| in that no |\includeonly| should be invoked.
This can be achieved by starting the include file
(before |\ProvidesPackage|) with:
%
\begin{center}
\begin{tabular}{l}
|\input{childdoc.def}|\\
|\childdocforward{|\textit{main}|}|\\
\end{tabular}
\end{center}
%
or alternatively with:
%
\begin{center}
\begin{tabular}{l}
|\input{childdoc.def}|\\
|\childdocby{|\textit{main}|}|\\
\end{tabular}
\end{center}
%
Both forms have slightly different effects as described above.
The main file is prepared as usual, see \secref{sec:include}.

%%%%%%%%%%%%%%%%%%%%%%%%%%%%%%%%%%%%%%%%%%%%%%%%%%%%%%%%%%%%%%%%%%%%%%%%%%%%%%%%
\subsection{Legacy Detection}
\label{sec:detection}

The directive |\childdocmain| in the main file can detect
whether the complete document or merely a child is to be compiled
even without using the directive |\childdocof|.
This method is deprecated because it is less robust
and there is no compelling reason to use it;
it is merely provided for backward compatibility
and it may be removed in future versions.

If the detection mechanism is to be used,
it is mandatory to correctly specify
the filename of the main file as the argument of |\childdocmain|:
%
\begin{center}
\begin{tabular}{l}
|\input{childdoc.def}|\\
|\childdocmain{|\textit{main}|}|\\
\end{tabular}
\end{center}
%
If |\jobname| does not match the argument \textit{main} of |\childdocmain|,
it is assumed that |\jobname| points to the child file to be compiled.
When using |\childdocmain| with the main file specified as argument,
it suffices to start a child file
with just |\input{|\textit{main}|}|
without loading of the package and using |\childdocof|.
If instead all processing is done
with the appropriate \textsf{childdoc} directives,
the argument of \textit{main} of |\childdocmain| can be empty.

An alternative version of the command line processing described
in \secref{sec:commandline} using the detection mechanism reads:
%
\begin{center}
|... -jobname "|\textit{target}|" "|[\textit{flags}]%
[|\def\jobname{|\textit{dest}|}|]|\input{|\textit{main}|}"|
\end{center}

%%%%%%%%%%%%%%%%%%%%%%%%%%%%%%%%%%%%%%%%%%%%%%%%%%%%%%%%%%%%%%%%%%%%%%%%%%%%%%%%
\subsection{Manual Code}
\label{sec:manual}

In case one cannot be certain whether the definitions file |childdoc.def|
is installed on the target \TeX{} distribution
and one prefers not to ship it,
it is conceivable to paste a few relevant commands into the sources.

To that end, drop all statements |\input{childdoc.def}|
and perform the replacements as outlined below.
Instead of |\childdocmain{|\textit{main}|}| add the following code
to the top of the main file:
%
\begin{center}
\begin{tabular}{l}
|\||ifdefined\childdocname\endinput\||fi\newif\ifchilddoc|\\
|\edef\childdocname{\scantokens\expandafter{\jobname\noexpand}}|\\
|\def\childdocmain{|\textit{main}|}\||ifx\childdocmain\childdocname\||else|\\
|\childdoctrue\includeonly{\childdocname}\let\jobname\childdocmain\||fi|\\
\end{tabular}
\end{center}
%
Instead of |\childdocof{|\textit{main}|}| just include the main file
at the top of each child file:
%
\begin{center}
|\input{|\textit{main}|}|
\end{center}
%
A simple redirection |\childdocforward{|\textit{dest}|}| is achieved by:
%
\begin{center}
|\def\jobname{|\textit{dest}|}\input{\jobname}|
\end{center}
%
The redirection with prefix
|\childdocforwardprefix[|\textit{prefix}|]{|\textit{dest}|}|
is accomplished by:
%
\begin{center}
\begin{tabular}{l}
|{\edef\jobname{\scantokens\expandafter{\jobname\noexpand}}|\\
|\def\redirectjob |\textit{prefix}|#1~~~{\gdef\jobname{|\textit{dest}|#1}}|\\
|\expandafter\redirectjob\jobname~~~}\input{\jobname}|
\end{tabular}
\end{center}

In an alternative approach,
child documents can be compiled by a specific command line
without additional code or specific definitions:
%
\begin{center}
|... -jobname "|\textit{target}|" "|[\textit{flags}]%
|\includeonly{|\textit{dest}|}\input{|\textit{main}|}"|
\end{center}
%

%%%%%%%%%%%%%%%%%%%%%%%%%%%%%%%%%%%%%%%%%%%%%%%%%%%%%%%%%%%%%%%%%%%%%%%%%%%%%%%%
%%%%%%%%%%%%%%%%%%%%%%%%%%%%%%%%%%%%%%%%%%%%%%%%%%%%%%%%%%%%%%%%%%%%%%%%%%%%%%%%
\section{Information}

%%%%%%%%%%%%%%%%%%%%%%%%%%%%%%%%%%%%%%%%%%%%%%%%%%%%%%%%%%%%%%%%%%%%%%%%%%%%%%%%
\subsection{Copyright}

Copyright \copyright{} 2017--2018 Niklas Beisert

This work may be distributed and/or modified under the
conditions of the \LaTeX{} Project Public License, either version 1.3
of this license or (at your option) any later version.
The latest version of this license is in
  \url{http://www.latex-project.org/lppl.txt}
and version 1.3 or later is part of all distributions of \LaTeX{}
version 2005/12/01 or later.

This work has the LPPL maintenance status `maintained'.

The Current Maintainer of this work is Niklas Beisert.

This work consists of the files |README.txt|, |childdoc.ins| and |childdoc.dtx|
as well as the derived files |childdoc.def|, |cdocsamp.tex|
with |cdocsch1.tex|, |cdocsch2.tex|, |cdocspt3.tex|, |cdocspt4.tex|,
|cdocsdrf.tex|, |cdocsfn1.tex|, |cdocsfn2.tex|
as well as |childdoc.pdf|.

%%%%%%%%%%%%%%%%%%%%%%%%%%%%%%%%%%%%%%%%%%%%%%%%%%%%%%%%%%%%%%%%%%%%%%%%%%%%%%%%
\subsection{Files and Installation}

The package consists of the files:
%
\begin{center}
\begin{tabular}{ll}
    |README.txt|   & readme file \\
    |childdoc.ins| & installation file \\
    |childdoc.dtx| & source file \\
    |childdoc.def| & definition file \\
    |cdocsamp.tex| & sample main file \\
    |cdocsch1.tex| & sample include file \\
    |cdocsch2.tex| & sample include file \\
    |cdocspt3.tex| & sample part file \\
    |cdocspt4.tex| & sample part file \\
    |cdocsdrf.tex| & sample redirection file \\
    |cdocsfn1.tex| & sample redirection file \\
    |cdocsfn2.tex| & sample redirection file \\
    |childdoc.pdf| & manual
\end{tabular}
\end{center}
%
The distribution consists of the files
|README.txt|, |childdoc.ins| and |childdoc.dtx|.
%
\begin{itemize}
\item
Run (pdf)\LaTeX{} on |childdoc.dtx|
to compile the manual |childdoc.pdf| (this file).
\item
Run \LaTeX{} on |childdoc.ins| to create the definitions file |childdoc.def|
and the sample |cdocsamp.tex| with include files
|cdocsch1.tex|, |cdocsch2.tex|, |cdocspt3.tex|, |cdocspt4.tex|,
|cdocsdrf.tex|, |cdocsfn1.tex|, |cdocsfn2.tex|.
Then copy the file |childdoc.def| to an appropriate directory of your \LaTeX{}
distribution, e.g.\ \textit{texmf-root}|/tex/latex/childdoc|.
\end{itemize}

%%%%%%%%%%%%%%%%%%%%%%%%%%%%%%%%%%%%%%%%%%%%%%%%%%%%%%%%%%%%%%%%%%%%%%%%%%%%%%%%
\subsection{Related CTAN Packages}

There are several other packages which offer a similar functionality:
%
\begin{itemize}
\item
The packages
\href{http://ctan.org/pkg/docmute}{\textsf{docmute}},
\href{http://ctan.org/pkg/includex}{\textsf{includex}} and
\href{http://ctan.org/pkg/standalone}{\textsf{standalone}}
provide commands to include only the document body of
a child file thus allowing both files to be compiled individually.
\item
The packages \href{http://ctan.org/pkg/subdocs}{\textsf{subdocs}}
and \href{http://ctan.org/pkg/subfiles}{\textsf{subfiles}}
provide structures in which the main and child documents can be
encapsulated and allowing them to be compiled individually.
The inclusion mechanism is different from the conventional |\include|.
\item
The package \href{http://ctan.org/pkg/combine}{\textsf{combine}}
is an elaborate solution to combine several documents into one.
\end{itemize}
%
See also the CTAN topic \href{http://ctan.org/topic/subdocs}{\textsf{subdocs}}
for further related packages.
The present package differs from the above solutions in that
a document structure constructed with the conventional |\include| mechanism
just needs two extra commands at the top of every file
such that all constituent files can be compiled individually.

%%%%%%%%%%%%%%%%%%%%%%%%%%%%%%%%%%%%%%%%%%%%%%%%%%%%%%%%%%%%%%%%%%%%%%%%%%%%%%%%
%\subsection{Feature Suggestions}
%
%The following is a list of features which may be useful for future
%versions of this package:
%%
%\begin{itemize}
%\item
%\ldots
%\end{itemize}

%%%%%%%%%%%%%%%%%%%%%%%%%%%%%%%%%%%%%%%%%%%%%%%%%%%%%%%%%%%%%%%%%%%%%%%%%%%%%%%%
\subsection{Revision History}

%%%%%%%%%%%%%%%%%%%%%%%%%%%%%%%%%%%%%%%%
\paragraph{v2.0:} 2018/12/30

\begin{itemize}
\item
immediate forward processing
\item
added |\childdocby| mechanism
\item
manual restructured
\end{itemize}

%%%%%%%%%%%%%%%%%%%%%%%%%%%%%%%%%%%%%%%%
\paragraph{v1.6:} 2018/01/17

\begin{itemize}
\item
application for development of include files
\item
corrections to manual
\end{itemize}

%%%%%%%%%%%%%%%%%%%%%%%%%%%%%%%%%%%%%%%%
\paragraph{v1.5:} 2017/05/21

\begin{itemize}
\item
more complete structuring introduced
\item
|\childdocof| introduced
\item
|\childdoc| renamed to |\childdocmain|
\item
|\childredirect| renamed to |\childdocforward| and |\childdocforwardprefix|
and functionality expanded
\end{itemize}

%%%%%%%%%%%%%%%%%%%%%%%%%%%%%%%%%%%%%%%%
\paragraph{v1.0:} 2017/04/27

\begin{itemize}
\item
manual and install package
\item
first version published on CTAN
\end{itemize}

%%%%%%%%%%%%%%%%%%%%%%%%%%%%%%%%%%%%%%%%
\paragraph{v0.6:} 2017/04/26

\begin{itemize}
\item
redirection mechanism added
\end{itemize}

%%%%%%%%%%%%%%%%%%%%%%%%%%%%%%%%%%%%%%%%
\paragraph{v0.5:} 2017/04/26

\begin{itemize}
\item
functionality in definition file
\end{itemize}


%%%%%%%%%%%%%%%%%%%%%%%%%%%%%%%%%%%%%%%%%%%%%%%%%%%%%%%%%%%%%%%%%%%%%%%%%%%%%%%%
%%%%%%%%%%%%%%%%%%%%%%%%%%%%%%%%%%%%%%%%%%%%%%%%%%%%%%%%%%%%%%%%%%%%%%%%%%%%%%%%
%%%%%%%%%%%%%%%%%%%%%%%%%%%%%%%%%%%%%%%%%%%%%%%%%%%%%%%%%%%%%%%%%%%%%%%%%%%%%%%%
\appendix

\settowidth\MacroIndent{\rmfamily\scriptsize 000\ }

 \DocInput{childdoc.dtx}

\end{document}
%</driver>
% \fi
%
% %%%%%%%%%%%%%%%%%%%%%%%%%%%%%%%%%%%%%%%%%%%%%%%%%%%%%%%%%%%%%%%%%%%%%%%%%%%%%%
% %%%%%%%%%%%%%%%%%%%%%%%%%%%%%%%%%%%%%%%%%%%%%%%%%%%%%%%%%%%%%%%%%%%%%%%%%%%%%%
% \section{Sample}
%\iffalse
%<*samplemain>
%\fi
%
% The following presents a sample document
% with two chapters, two parts, a title page,
% a compile flag as well as three forwarding files to set the flag.
% It consists of eight |.tex| files:
% \begin{center}
% \begin{tabular}{ll}
% |cdocsamp.tex|&main file\\
% |cdocsch1.tex|&include file for chapter 1\\
% |cdocsch2.tex|&include file for chapter 2\\
% |cdocspt3.tex|&include file for part 3\\
% |cdocspt4.tex|&include file for part 4\\
% |cdocsdrf.tex|&forwarding file for main file in draft mode\\
% |cdocsfi1.tex|&forwarding file for final version of chapter 1\\
% |cdocsfi2.tex|&forwarding file for final version of chapter 2\\
% \end{tabular}
% \end{center}
% Each of the eight files can be compiled directly by the \LaTeX{} compiler.
%
% %%%%%%%%%%%%%%%%%%%%%%%%%%%%%%%%%%%%%%
% \paragraph{Main File.}
%
% The main file is called |cdocsamp.tex|.
%
% Load the \textsf{childdoc} definitions and
% declare the filename for the main document:
%    \begin{macrocode}
\input{childdoc.def}
\childdocmain{}
%    \end{macrocode}

% Optional override for |\version| flag:
%    \begin{macrocode}
%%\ifchilddoc\else\providecommand{\version}{draft}\fi
%    \end{macrocode}

% Define the default values for the |\version| flag
% (|final| for the main file and |draft| for childs):
%    \begin{macrocode}
\ifchilddoc
\providecommand{\version}{draft}
\else
\providecommand{\version}{final}
\fi
%    \end{macrocode}

% Load the standard document class:
%    \begin{macrocode}
\documentclass[12pt]{article}
%    \end{macrocode}

% Start the document body:
%    \begin{macrocode}
\begin{document}
%    \end{macrocode}

% Declare a title page.
% Print title, part of document being processed and version flag:
%    \begin{macrocode}
\addtocounter{page}{-1}
\begin{center}
{\LARGE\bfseries{}childdoc example\par}
\vspace{1cm}
\ifchilddoc
\ifchilddocmanual part\else chapter\fi:
`\childdocname' of `\childdocjob'\par
\else
main document: `\childdocjob'\par
\fi
version: \version\par
\end{center}
\newpage
%    \end{macrocode}

% Manually include selected file,
% otherwise process as usual:
%    \begin{macrocode}
\ifchilddocmanual
\section*{part `\childdocname'}
\input{\childdocname}
\else
%    \end{macrocode}

% Include the two chapters:
%    \begin{macrocode}
\include{cdocsch1}
\include{cdocsch2}
%    \end{macrocode}

% Include the two parts unless only chapters should be displayed:
%    \begin{macrocode}
\ifchilddoc\else
\section{part three}
\input{cdocspt3}
\section{part four}
\input{cdocspt4}
\fi
%    \end{macrocode}

% Process as usual until here:
%    \begin{macrocode}
\fi
%    \end{macrocode}

% End of document body:
%    \begin{macrocode}
\end{document}
%    \end{macrocode}
%\iffalse
%</samplemain>
%\fi
%
% %%%%%%%%%%%%%%%%%%%%%%%%%%%%%%%%%%%%%%
% \paragraph{Chapter Include Files.}
%
% The include files are called |cdocsch1.tex| and |cdocsch2.tex|.
%
%\iffalse
%<*samplechap1|samplechap2>
%\fi

% Optional override for |\version| flag:
%    \begin{macrocode}
%%\providecommand{\version}{final}
%    \end{macrocode}

% Include the main document:
%    \begin{macrocode}
\input{childdoc.def}
\childdocof{cdocsamp}
%    \end{macrocode}

%\iffalse
%</samplechap1|samplechap2>
%\fi
%
%\iffalse
%<*samplechap1>
%\fi
% Some text for chapter 1:
%    \begin{macrocode}
\section{one}
some text in chapter one
%    \end{macrocode}

%\iffalse
%</samplechap1>
%\fi
% Some text for chapter 2:
%\iffalse
%<*samplechap2>
%\fi
%    \begin{macrocode}
\section{two}
more text in chapter two
%    \end{macrocode}

%\iffalse
%</samplechap2>
%\fi
%
% %%%%%%%%%%%%%%%%%%%%%%%%%%%%%%%%%%%%%%
% \paragraph{Part Include Files.}
%
% The include files are called |cdocspt3.tex| and |cdocspt4.tex|.
%
%\iffalse
%<*samplepart3|samplepart4>
%\fi

% Optional override for |\version| flag:
%    \begin{macrocode}
%%\providecommand{\version}{final}
%    \end{macrocode}

% Include the main document:
%    \begin{macrocode}
\input{childdoc.def}
\childdocby{cdocsamp}
%    \end{macrocode}

%\iffalse
%</samplepart3|samplepart4>
%\fi
%
%\iffalse
%<*samplepart3>
%\fi
% Some text for part 3:
%    \begin{macrocode}
some text in part three
%    \end{macrocode}

%\iffalse
%</samplepart3>
%\fi
% Some text for part 4:
%\iffalse
%<*samplepart4>
%\fi
%    \begin{macrocode}
more text in part four
%    \end{macrocode}

%\iffalse
%</samplepart4>
%\fi
%
% %%%%%%%%%%%%%%%%%%%%%%%%%%%%%%%%%%%%%%
% \paragraph{Forwarding for a Complete Draft.}
%
% The following forwarding file |cdocsdrf.tex|
% compiles the main document in draft mode:
%\iffalse
%<*sampledraft>
%\fi
%    \begin{macrocode}
\def\version{draft}
\input{childdoc.def}
\childdocforward{cdocsamp}
%    \end{macrocode}

%\iffalse
%</sampledraft>
%\fi
%
% %%%%%%%%%%%%%%%%%%%%%%%%%%%%%%%%%%%%%%
% \paragraph{Forwarding for Final Version of the Chapters.}
%
% The following forwarding files |cdocsfn1.tex| and |cdocsfn2.tex|
% (with identical content)
% compile the final versions of the child documents
% |cdocsch1.tex| and |cdocsch2.tex|, respectively:
%\iffalse
%<*samplefinal>
%\fi
%    \begin{macrocode}
\def\version{final}
\input{childdoc.def}
\childdocforwardprefix[cdocsamp]{cdocsfn}{cdocsch}
%    \end{macrocode}

%\iffalse
%</samplefinal>
%\fi
%
% %%%%%%%%%%%%%%%%%%%%%%%%%%%%%%%%%%%%%%
% \paragraph{Command Line Processing.}
%
% The following three command lines generate the output files
% |cdocscld|, |cdocscl1| and |cdocscl2|
% which should be identical to
% |cdocsdrf|, |cdocsch1| and |cdocsfn2|, respectively:
% \begin{center}
% \begin{tabular}{l}
% |latex -jobname cdocscld \|\\
% |  "\def\version{draft}\input{childdoc.def}\childdocforward{cdocsamp}"|\\
% |latex -jobname cdocscl1 \|\\
% |  "\input{childdoc.def}\childdocforward[cdocsamp]{cdocsch1}"|\\
% |latex -jobname cdocscl2 \|\\
% |  "\def\version{final}\input{childdoc.def}\childdocforward{cdocsch2}"|
% \end{tabular}
% \end{center}
% Note that the trailing backslash on each first line
% merely continues the input to the second line
% (for convenient cut ant paste).
% Furthermore, the command |latex| can be replaced by any
% of its alternative versions such as |pdflatex|.
%
% %%%%%%%%%%%%%%%%%%%%%%%%%%%%%%%%%%%%%%%%%%%%%%%%%%%%%%%%%%%%%%%%%%%%%%%%%%%%%%
% %%%%%%%%%%%%%%%%%%%%%%%%%%%%%%%%%%%%%%%%%%%%%%%%%%%%%%%%%%%%%%%%%%%%%%%%%%%%%%
% \section{Implementation}
%\iffalse
%<*package>
%\fi
%
% This section describes the definitions file |childdoc.def|.

% The definitions cannot be loaded using |\usepackage| or |\RequirePackage|
% which has a mechanism to prevent loading a style file more than once.
% When loading the definitions by means of |\input|
% multiple instances have to be prevented manually:
%\iffalse
%This code needs to be before the `\ProvidesFile' directive
%which is defined at the beginning of this file.
%Therefore it is also placed there and commented out here.
%</package>
%<*discard>
%\fi
%    \begin{macrocode}
\ifdefined\childdocmain\endinput\fi
%    \end{macrocode}
%\iffalse
%</discard>
%<*package>
%\fi
%
% \macro{\ifchilddoc}
% \macro{\ifchilddocmanual}
% The conditional |\ifchilddoc| tells whether a
% child (true) or main (false) document is being compiled.
% The conditional |\ifchilddocmanual| tells whether
% the |\includeonly| mechanism is used (false) or
% the selection of child files must be performed manually (true).
% The definitions initialise to false:
%    \begin{macrocode}
\newif\ifchilddoc
\newif\ifchilddocmanual
%    \end{macrocode}

% \macro{\childdocname}
% \macro{\childdocjob}
% The macro |\childdocname| stores the name of the main document
% to be compiled. The macro |\childdocjob| stores the name of
% the document on which the \LaTeX{} compiler was originally invoked.
% The content of |\jobname| cannot be compared
% to filenames specified in the source due to different catcodes.
% The following code rescans |\jobname|, stores the result
% in |\childdocname| and saves a copy in |\childdocjob|:
%    \begin{macrocode}
\edef\childdocname{\scantokens\expandafter{\jobname\noexpand}}
\let\childdocjob\childdocname
%    \end{macrocode}

% \macro{\childdocdisable}
% The macro |\childdocdisable| prevents the main file
% from being processed more than once.
% At this stage, the main document command |\childdocmain|
% is assumed to be called once again where it should do nothing.
% Any subsequent call to it should prevent
% a secondary processing of the main document
% It overwrites the forwarding commands
% |\childdocof| and |\childdocforward|
% with empty macros to prevent further inclusions of the main document:
%    \begin{macrocode}
\newcommand{\childdocdisable}
{
  \renewcommand{\childdocmain}[1]{\renewcommand{\childdocmain}[1]{\endinput}}
  \renewcommand{\childdocof}[1]{}
  \renewcommand{\childdocby}[2][]{}
  \renewcommand{\childdocforward}[2][]{}
  \renewcommand{\childdocdisable}{}
}
%    \end{macrocode}

% \macro{\childdocmain}
% The macro |\childdocmain| is to be called at the top of the main file
% with nothing or the main filename (without extension) as argument.
% First, it breaks loops.
% If the argument is not empty and does not match |\childdocname|
% (which is set by the first inclusion of |childdoc.def|),
% |\ifchilddoc| is set to true, |\includeonly| is applied to the child file
% and |\jobname| is set to the main file
% (for proper handling of |.aux| files):
%    \begin{macrocode}
\newcommand{\childdocmain}[1]
{
  \childdocdisable\childdocmain{}
  \if?#1?\else
    \begingroup
      \def\childdoctmp{#1}
      \ifx\childdoctmp\childdocname
        \def\childdoctmp{}
      \else
        \def\childdoctmp
        {
          \childdoctrue
          \includeonly{\childdocname}
          \def\childdocjob{#1}
          \def\jobname{#1}
        }
      \fi
      \expandafter
    \endgroup
    \childdoctmp
  \fi
}
%    \end{macrocode}

% \macro{\childdocof}
% The command |\childdocof| redirects
% compilation to the main file |#1|.
%    \begin{macrocode}
\newcommand{\childdocof}[1]
{
  \childdocdisable
  \childdoctrue
  \includeonly{\childdocname}
  \def\jobname{#1}
  \def\childdocjob{#1}
  \input{#1}
}
%    \end{macrocode}

% \macro{\childdocby}
% The command |\childdocby| ....
%    \begin{macrocode}
\newcommand{\childdocby}[2][]
{
  \childdocdisable
  \childdoctrue
  \childdocmanualtrue
  \if?#1?\else
    \def\jobname{#2}
  \fi
  \def\childdocjob{#2}
  \input{#2}
  \endinput
}
%    \end{macrocode}

% \macro{\childdocforward}
% The command |\childdocforward| redirects
% compilation to the main file or
% (if the optional argument is given) a child file.
% Parameters are set as if the main file
% or a child file starting with |\childdocof| was compiled.
% Then compilation is handed over to the main file:
%    \begin{macrocode}
\newcommand{\childdocforward}[2][]
{
  \begingroup
    \if?#1?
      \def\childdoctmp
      {
        \def\childdocname{#2}
        \def\childdocjob{#2}
        \def\jobname{#2}
        \input{#2}
        \endinput
      }
    \else
      \def\childdoctmp
      {
        \childdocdisable
        \def\childdocname{#2}
        \childdoctrue
        \includeonly{#2}
        \def\childdocjob{#1}
        \def\jobname{#1}
        \input{#1}
        \endinput
      }
    \fi
    \expandafter
  \endgroup
  \childdoctmp
}
%    \end{macrocode}

% \macro{\childdocforwardprefix}
% The command |\childdocforwardprefix| redirects
% compilation to the main or a child file by means of a pattern.
% The prefix |#1| in the current filename is replaced by |#2|
% and the suffix of the current filename is kept
% (it is assumed that the filename does not contain the substring `|~~~|'
% which is used as a delimiter).
% Compilation is handed over to the new file by |\childdocforward|:
%    \begin{macrocode}
\newcommand{\childdocforwardprefix}[3][]
{
  \begingroup
    \def\childdocextract #2##1~~~{\def\childdoctmp{\childdocforward[#1]{#3##1}}}
    \expandafter\childdocextract\childdocname~~~
    \expandafter
  \endgroup
  \childdoctmp
}
%    \end{macrocode}

% \macro{\childdoc}
% The deprecated macro |\childdoc| is a legacy version of |\childdocmain|:
%    \begin{macrocode}
\newcommand{\childdoc}{\childdocmain}
%    \end{macrocode}

% \macro{\childdocredirect}
% The deprecated macro |\childdocredirect| is a legacy version
% of |\childdocforward| and |\childdocforwardprefix|:
%    \begin{macrocode}
\newcommand{\childdocredirect}[2][]
{
  \begingroup
    \if?#1?
      \def\childdoctmp{\childdocforward{#2}}
    \else
      \def\childdoctmp{\childdocforwardprefix{#1}{#2}}
    \fi
    \expandafter
  \endgroup
  \childdoctmp
}
%    \end{macrocode}

%\iffalse
%</package>
%\fi
%
\endinput
\childdocforward{cdocsch2}"|
% \end{tabular}
% \end{center}
% Note that the trailing backslash on each first line
% merely continues the input to the second line
% (for convenient cut ant paste).
% Furthermore, the command |latex| can be replaced by any
% of its alternative versions such as |pdflatex|.
%
% %%%%%%%%%%%%%%%%%%%%%%%%%%%%%%%%%%%%%%%%%%%%%%%%%%%%%%%%%%%%%%%%%%%%%%%%%%%%%%
% %%%%%%%%%%%%%%%%%%%%%%%%%%%%%%%%%%%%%%%%%%%%%%%%%%%%%%%%%%%%%%%%%%%%%%%%%%%%%%
% \section{Implementation}
%\iffalse
%<*package>
%\fi
%
% This section describes the definitions file |childdoc.def|.

% The definitions cannot be loaded using |\usepackage| or |\RequirePackage|
% which has a mechanism to prevent loading a style file more than once.
% When loading the definitions by means of |\input|
% multiple instances have to be prevented manually:
%\iffalse
%This code needs to be before the `\ProvidesFile' directive
%which is defined at the beginning of this file.
%Therefore it is also placed there and commented out here.
%</package>
%<*discard>
%\fi
%    \begin{macrocode}
\ifdefined\childdocmain\endinput\fi
%    \end{macrocode}
%\iffalse
%</discard>
%<*package>
%\fi
%
% \macro{\ifchilddoc}
% \macro{\ifchilddocmanual}
% The conditional |\ifchilddoc| tells whether a
% child (true) or main (false) document is being compiled.
% The conditional |\ifchilddocmanual| tells whether
% the |\includeonly| mechanism is used (false) or
% the selection of child files must be performed manually (true).
% The definitions initialise to false:
%    \begin{macrocode}
\newif\ifchilddoc
\newif\ifchilddocmanual
%    \end{macrocode}

% \macro{\childdocname}
% \macro{\childdocjob}
% The macro |\childdocname| stores the name of the main document
% to be compiled. The macro |\childdocjob| stores the name of
% the document on which the \LaTeX{} compiler was originally invoked.
% The content of |\jobname| cannot be compared
% to filenames specified in the source due to different catcodes.
% The following code rescans |\jobname|, stores the result
% in |\childdocname| and saves a copy in |\childdocjob|:
%    \begin{macrocode}
\edef\childdocname{\scantokens\expandafter{\jobname\noexpand}}
\let\childdocjob\childdocname
%    \end{macrocode}

% \macro{\childdocdisable}
% The macro |\childdocdisable| prevents the main file
% from being processed more than once.
% At this stage, the main document command |\childdocmain|
% is assumed to be called once again where it should do nothing.
% Any subsequent call to it should prevent
% a secondary processing of the main document
% It overwrites the forwarding commands
% |\childdocof| and |\childdocforward|
% with empty macros to prevent further inclusions of the main document:
%    \begin{macrocode}
\newcommand{\childdocdisable}
{
  \renewcommand{\childdocmain}[1]{\renewcommand{\childdocmain}[1]{\endinput}}
  \renewcommand{\childdocof}[1]{}
  \renewcommand{\childdocby}[2][]{}
  \renewcommand{\childdocforward}[2][]{}
  \renewcommand{\childdocdisable}{}
}
%    \end{macrocode}

% \macro{\childdocmain}
% The macro |\childdocmain| is to be called at the top of the main file
% with nothing or the main filename (without extension) as argument.
% First, it breaks loops.
% If the argument is not empty and does not match |\childdocname|
% (which is set by the first inclusion of |childdoc.def|),
% |\ifchilddoc| is set to true, |\includeonly| is applied to the child file
% and |\jobname| is set to the main file
% (for proper handling of |.aux| files):
%    \begin{macrocode}
\newcommand{\childdocmain}[1]
{
  \childdocdisable\childdocmain{}
  \if?#1?\else
    \begingroup
      \def\childdoctmp{#1}
      \ifx\childdoctmp\childdocname
        \def\childdoctmp{}
      \else
        \def\childdoctmp
        {
          \childdoctrue
          \includeonly{\childdocname}
          \def\childdocjob{#1}
          \def\jobname{#1}
        }
      \fi
      \expandafter
    \endgroup
    \childdoctmp
  \fi
}
%    \end{macrocode}

% \macro{\childdocof}
% The command |\childdocof| redirects
% compilation to the main file |#1|.
%    \begin{macrocode}
\newcommand{\childdocof}[1]
{
  \childdocdisable
  \childdoctrue
  \includeonly{\childdocname}
  \def\jobname{#1}
  \def\childdocjob{#1}
  \input{#1}
}
%    \end{macrocode}

% \macro{\childdocby}
% The command |\childdocby| ....
%    \begin{macrocode}
\newcommand{\childdocby}[2][]
{
  \childdocdisable
  \childdoctrue
  \childdocmanualtrue
  \if?#1?\else
    \def\jobname{#2}
  \fi
  \def\childdocjob{#2}
  \input{#2}
  \endinput
}
%    \end{macrocode}

% \macro{\childdocforward}
% The command |\childdocforward| redirects
% compilation to the main file or
% (if the optional argument is given) a child file.
% Parameters are set as if the main file
% or a child file starting with |\childdocof| was compiled.
% Then compilation is handed over to the main file:
%    \begin{macrocode}
\newcommand{\childdocforward}[2][]
{
  \begingroup
    \if?#1?
      \def\childdoctmp
      {
        \def\childdocname{#2}
        \def\childdocjob{#2}
        \def\jobname{#2}
        \input{#2}
        \endinput
      }
    \else
      \def\childdoctmp
      {
        \childdocdisable
        \def\childdocname{#2}
        \childdoctrue
        \includeonly{#2}
        \def\childdocjob{#1}
        \def\jobname{#1}
        \input{#1}
        \endinput
      }
    \fi
    \expandafter
  \endgroup
  \childdoctmp
}
%    \end{macrocode}

% \macro{\childdocforwardprefix}
% The command |\childdocforwardprefix| redirects
% compilation to the main or a child file by means of a pattern.
% The prefix |#1| in the current filename is replaced by |#2|
% and the suffix of the current filename is kept
% (it is assumed that the filename does not contain the substring `|~~~|'
% which is used as a delimiter).
% Compilation is handed over to the new file by |\childdocforward|:
%    \begin{macrocode}
\newcommand{\childdocforwardprefix}[3][]
{
  \begingroup
    \def\childdocextract #2##1~~~{\def\childdoctmp{\childdocforward[#1]{#3##1}}}
    \expandafter\childdocextract\childdocname~~~
    \expandafter
  \endgroup
  \childdoctmp
}
%    \end{macrocode}

% \macro{\childdoc}
% The deprecated macro |\childdoc| is a legacy version of |\childdocmain|:
%    \begin{macrocode}
\newcommand{\childdoc}{\childdocmain}
%    \end{macrocode}

% \macro{\childdocredirect}
% The deprecated macro |\childdocredirect| is a legacy version
% of |\childdocforward| and |\childdocforwardprefix|:
%    \begin{macrocode}
\newcommand{\childdocredirect}[2][]
{
  \begingroup
    \if?#1?
      \def\childdoctmp{\childdocforward{#2}}
    \else
      \def\childdoctmp{\childdocforwardprefix{#1}{#2}}
    \fi
    \expandafter
  \endgroup
  \childdoctmp
}
%    \end{macrocode}

%\iffalse
%</package>
%\fi
%
\endinput
|\\
|\childdocmain{}|\\
\end{tabular}
\end{center}
at the very top of the main \LaTeX{} file,
in particular \emph{before} the |\documentclass| statement!
The argument of |\childdocmain| should be left empty
(but it must be present).

%%%%%%%%%%%%%%%%%%%%%%%%%%%%%%%%%%%%%%%%
\DescribeMacro{\childdocof}
Furthermore, add the commands
\begin{center}
\begin{tabular}{l}
|% \iffalse
%
% childdoc.dtx Copyright (C) 2017-2018 Niklas Beisert
%
% This work may be distributed and/or modified under the
% conditions of the LaTeX Project Public License, either version 1.3
% of this license or (at your option) any later version.
% The latest version of this license is in
%   http://www.latex-project.org/lppl.txt
% and version 1.3 or later is part of all distributions of LaTeX
% version 2005/12/01 or later.
%
% This work has the LPPL maintenance status `maintained'.
%
% The Current Maintainer of this work is Niklas Beisert.
%
% This work consists of the files childdoc.dtx and childdoc.ins
% and the derived files childdoc.def and cdocsamp.tex with
% cdocsch1.tex, cdocsch2.tex, cdocsdrf.tex, cdocsfn1.tex, cdocsfn2.tex.
%
%<package>\ifdefined\childdocmain\endinput\fi
%<package>\ProvidesFile{childdoc.def}[2018/12/30 v2.0 child document driver]
%<samplemain>\ProvidesFile{cdocsamp.tex}[2018/12/30 v2.0 sample for childdoc]
%<*driver>
%\ProvidesFile{childdoc.drv}[2018/12/30 v2.0 childdoc reference manual file]
\PassOptionsToClass{10pt,a4paper}{article}
\documentclass{ltxdoc}

\usepackage[margin=35mm]{geometry}
\usepackage{hyperref}
\usepackage{hyperxmp}
\usepackage[usenames]{color}

\hypersetup{colorlinks=true}
\hypersetup{pdfstartview=FitH}
\hypersetup{pdfpagemode=UseNone}
\hypersetup{pdfsource={}}
\hypersetup{pdflang={en-UK}}
\hypersetup{pdfcopyright={Copyright 2017-2018 Niklas Beisert.
  This work may be distributed and/or modified under the
  conditions of the LaTeX Project Public License, either version 1.3
  of this license or (at your option) any later version.}}
\hypersetup{pdflicenseurl={http://www.latex-project.org/lppl.txt}}
\hypersetup{pdfcontactaddress={ETH Zurich, ITP, HIT K,
  Wolfgang-Pauli-Strasse 27}}
\hypersetup{pdfcontactpostcode={8093}}
\hypersetup{pdfcontactcity={Zurich}}
\hypersetup{pdfcontactcountry={Switzerland}}
\hypersetup{pdfcontactemail={nbeisert@itp.phys.ethz.ch}}
\hypersetup{pdfcontacturl={http://people.phys.ethz.ch/\xmptilde nbeisert/}}

\newcommand{\secref}[1]{\hyperref[#1]{section \ref*{#1}}}

\parskip1ex
\parindent0pt
\let\olditemize\itemize
\def\itemize{\olditemize\parskip0pt}

\begin{document}

\title{The \textsf{childdoc} Package}
\hypersetup{pdftitle={The childdoc Package}}
\author{Niklas Beisert\\[2ex]
  Institut f\"ur Theoretische Physik\\
  Eidgen\"ossische Technische Hochschule Z\"urich\\
  Wolfgang-Pauli-Strasse 27, 8093 Z\"urich, Switzerland\\[1ex]
  \href{mailto:nbeisert@itp.phys.ethz.ch}
  {\texttt{nbeisert@itp.phys.ethz.ch}}}
\hypersetup{pdfauthor={Niklas Beisert}}
\hypersetup{pdfsubject={Manual for the LaTeX2e Package childdoc}}
\date{30 December 2018, \textsf{v2.0}}
\maketitle

\begin{abstract}\noindent
\textsf{childdoc} is a \LaTeXe{} package
that enables the direct compilation
of document sections included by |\include|
to individual files.
\end{abstract}

\begingroup
\parskip0ex
\tableofcontents
\endgroup

%%%%%%%%%%%%%%%%%%%%%%%%%%%%%%%%%%%%%%%%%%%%%%%%%%%%%%%%%%%%%%%%%%%%%%%%%%%%%%%%
%%%%%%%%%%%%%%%%%%%%%%%%%%%%%%%%%%%%%%%%%%%%%%%%%%%%%%%%%%%%%%%%%%%%%%%%%%%%%%%%
\section{Introduction}

\LaTeX{} provides a mechanism to structure a large document (such as a book)
into a main file and several child files (containing the chapters)
using the |\include| command.
This mechanism is beneficial for documents
which span hundreds of pages in order to
make the source file(s) more manageable.
Moreover, compilation can be restricted to
selected child files by means of the |\includeonly| command.
The latter feature can be used to reduce the compilation time while editing
(this was significantly more useful in the earlier days of \LaTeX{})
or to generate a smaller document which is easier to navigate.
Another application of |\includeonly| is to generate
documents consisting of selected parts of the complete document.

However, there are a few drawbacks of the plain |\include| mechanism:
\begin{itemize}
\item
The child files cannot be compiled on their own,
they can only be compiled via the main file.
A naive editing environment
(such as a text editor with an option
to have the current file processed by \LaTeX)
may require one to switch to the main file before compiling;
attempting to compile the child file produces errors.
\item
The main file must be modified (each time)
to adjust the |\includeonly| command
to the present needs. This easily leaves the main file in a messy state.
\item
The generated document will always carry the filename
of the main document. This is inconvenient if
several child files are to be compiled and
to be kept for distribution.
\end{itemize}

The present package provides a simple interface
to make child files individually compilable by \LaTeX{}.
Compiling a child file then has the same effect as compiling
the main file with an |\includeonly| command
to select the appropriate child.
Moreover the generated document will carry the name of the child
rather than the main file.
This resolves all three above issues.

This feature is meant to make the editing of books,
thesis documents and lecture notes somewhat more convenient.
However, the package can also be used efficiently for
composing a series of documents (such as exercise sheets)
which are typically distributed individually.
It then assists the author in generating the individual documents
(potentially in different versions)
as well as a document containing the collected series.
Another application is in developing style files
or other kinds of included material
where compilation of the style file could redirect
to a sample or test file.

%%%%%%%%%%%%%%%%%%%%%%%%%%%%%%%%%%%%%%%%%%%%%%%%%%%%%%%%%%%%%%%%%%%%%%%%%%%%%%%%
%%%%%%%%%%%%%%%%%%%%%%%%%%%%%%%%%%%%%%%%%%%%%%%%%%%%%%%%%%%%%%%%%%%%%%%%%%%%%%%%
\section{Usage}

First of all, the package \textsf{childdoc} is \emph{not} a standard
\LaTeXe{} |.sty| style file! Therefore it needs to be invoked in
a non-standard way.

%%%%%%%%%%%%%%%%%%%%%%%%%%%%%%%%%%%%%%%%%%%%%%%%%%%%%%%%%%%%%%%%%%%%%%%%%%%%%%%%
\subsection{Included Files}
\label{sec:include}

%%%%%%%%%%%%%%%%%%%%%%%%%%%%%%%%%%%%%%%%
\DescribeMacro{\childdocmain}
To use the package, add the commands
\begin{center}
\begin{tabular}{l}
|% \iffalse
%
% childdoc.dtx Copyright (C) 2017-2018 Niklas Beisert
%
% This work may be distributed and/or modified under the
% conditions of the LaTeX Project Public License, either version 1.3
% of this license or (at your option) any later version.
% The latest version of this license is in
%   http://www.latex-project.org/lppl.txt
% and version 1.3 or later is part of all distributions of LaTeX
% version 2005/12/01 or later.
%
% This work has the LPPL maintenance status `maintained'.
%
% The Current Maintainer of this work is Niklas Beisert.
%
% This work consists of the files childdoc.dtx and childdoc.ins
% and the derived files childdoc.def and cdocsamp.tex with
% cdocsch1.tex, cdocsch2.tex, cdocsdrf.tex, cdocsfn1.tex, cdocsfn2.tex.
%
%<package>\ifdefined\childdocmain\endinput\fi
%<package>\ProvidesFile{childdoc.def}[2018/12/30 v2.0 child document driver]
%<samplemain>\ProvidesFile{cdocsamp.tex}[2018/12/30 v2.0 sample for childdoc]
%<*driver>
%\ProvidesFile{childdoc.drv}[2018/12/30 v2.0 childdoc reference manual file]
\PassOptionsToClass{10pt,a4paper}{article}
\documentclass{ltxdoc}

\usepackage[margin=35mm]{geometry}
\usepackage{hyperref}
\usepackage{hyperxmp}
\usepackage[usenames]{color}

\hypersetup{colorlinks=true}
\hypersetup{pdfstartview=FitH}
\hypersetup{pdfpagemode=UseNone}
\hypersetup{pdfsource={}}
\hypersetup{pdflang={en-UK}}
\hypersetup{pdfcopyright={Copyright 2017-2018 Niklas Beisert.
  This work may be distributed and/or modified under the
  conditions of the LaTeX Project Public License, either version 1.3
  of this license or (at your option) any later version.}}
\hypersetup{pdflicenseurl={http://www.latex-project.org/lppl.txt}}
\hypersetup{pdfcontactaddress={ETH Zurich, ITP, HIT K,
  Wolfgang-Pauli-Strasse 27}}
\hypersetup{pdfcontactpostcode={8093}}
\hypersetup{pdfcontactcity={Zurich}}
\hypersetup{pdfcontactcountry={Switzerland}}
\hypersetup{pdfcontactemail={nbeisert@itp.phys.ethz.ch}}
\hypersetup{pdfcontacturl={http://people.phys.ethz.ch/\xmptilde nbeisert/}}

\newcommand{\secref}[1]{\hyperref[#1]{section \ref*{#1}}}

\parskip1ex
\parindent0pt
\let\olditemize\itemize
\def\itemize{\olditemize\parskip0pt}

\begin{document}

\title{The \textsf{childdoc} Package}
\hypersetup{pdftitle={The childdoc Package}}
\author{Niklas Beisert\\[2ex]
  Institut f\"ur Theoretische Physik\\
  Eidgen\"ossische Technische Hochschule Z\"urich\\
  Wolfgang-Pauli-Strasse 27, 8093 Z\"urich, Switzerland\\[1ex]
  \href{mailto:nbeisert@itp.phys.ethz.ch}
  {\texttt{nbeisert@itp.phys.ethz.ch}}}
\hypersetup{pdfauthor={Niklas Beisert}}
\hypersetup{pdfsubject={Manual for the LaTeX2e Package childdoc}}
\date{30 December 2018, \textsf{v2.0}}
\maketitle

\begin{abstract}\noindent
\textsf{childdoc} is a \LaTeXe{} package
that enables the direct compilation
of document sections included by |\include|
to individual files.
\end{abstract}

\begingroup
\parskip0ex
\tableofcontents
\endgroup

%%%%%%%%%%%%%%%%%%%%%%%%%%%%%%%%%%%%%%%%%%%%%%%%%%%%%%%%%%%%%%%%%%%%%%%%%%%%%%%%
%%%%%%%%%%%%%%%%%%%%%%%%%%%%%%%%%%%%%%%%%%%%%%%%%%%%%%%%%%%%%%%%%%%%%%%%%%%%%%%%
\section{Introduction}

\LaTeX{} provides a mechanism to structure a large document (such as a book)
into a main file and several child files (containing the chapters)
using the |\include| command.
This mechanism is beneficial for documents
which span hundreds of pages in order to
make the source file(s) more manageable.
Moreover, compilation can be restricted to
selected child files by means of the |\includeonly| command.
The latter feature can be used to reduce the compilation time while editing
(this was significantly more useful in the earlier days of \LaTeX{})
or to generate a smaller document which is easier to navigate.
Another application of |\includeonly| is to generate
documents consisting of selected parts of the complete document.

However, there are a few drawbacks of the plain |\include| mechanism:
\begin{itemize}
\item
The child files cannot be compiled on their own,
they can only be compiled via the main file.
A naive editing environment
(such as a text editor with an option
to have the current file processed by \LaTeX)
may require one to switch to the main file before compiling;
attempting to compile the child file produces errors.
\item
The main file must be modified (each time)
to adjust the |\includeonly| command
to the present needs. This easily leaves the main file in a messy state.
\item
The generated document will always carry the filename
of the main document. This is inconvenient if
several child files are to be compiled and
to be kept for distribution.
\end{itemize}

The present package provides a simple interface
to make child files individually compilable by \LaTeX{}.
Compiling a child file then has the same effect as compiling
the main file with an |\includeonly| command
to select the appropriate child.
Moreover the generated document will carry the name of the child
rather than the main file.
This resolves all three above issues.

This feature is meant to make the editing of books,
thesis documents and lecture notes somewhat more convenient.
However, the package can also be used efficiently for
composing a series of documents (such as exercise sheets)
which are typically distributed individually.
It then assists the author in generating the individual documents
(potentially in different versions)
as well as a document containing the collected series.
Another application is in developing style files
or other kinds of included material
where compilation of the style file could redirect
to a sample or test file.

%%%%%%%%%%%%%%%%%%%%%%%%%%%%%%%%%%%%%%%%%%%%%%%%%%%%%%%%%%%%%%%%%%%%%%%%%%%%%%%%
%%%%%%%%%%%%%%%%%%%%%%%%%%%%%%%%%%%%%%%%%%%%%%%%%%%%%%%%%%%%%%%%%%%%%%%%%%%%%%%%
\section{Usage}

First of all, the package \textsf{childdoc} is \emph{not} a standard
\LaTeXe{} |.sty| style file! Therefore it needs to be invoked in
a non-standard way.

%%%%%%%%%%%%%%%%%%%%%%%%%%%%%%%%%%%%%%%%%%%%%%%%%%%%%%%%%%%%%%%%%%%%%%%%%%%%%%%%
\subsection{Included Files}
\label{sec:include}

%%%%%%%%%%%%%%%%%%%%%%%%%%%%%%%%%%%%%%%%
\DescribeMacro{\childdocmain}
To use the package, add the commands
\begin{center}
\begin{tabular}{l}
|\input{childdoc.def}|\\
|\childdocmain{}|\\
\end{tabular}
\end{center}
at the very top of the main \LaTeX{} file,
in particular \emph{before} the |\documentclass| statement!
The argument of |\childdocmain| should be left empty
(but it must be present).

%%%%%%%%%%%%%%%%%%%%%%%%%%%%%%%%%%%%%%%%
\DescribeMacro{\childdocof}
Furthermore, add the commands
\begin{center}
\begin{tabular}{l}
|\input{childdoc.def}|\\
|\childdocof{|\textit{main}|}|\\
\end{tabular}
\end{center}
at the top of every child file \textit{child}
which is included by |\include{|\textit{child}|}|
from within the main file
(or at least for those files to be compiled individually).
The argument \textit{main} must be the filename of the main file.

There are a couple of
considerations in setting up the main and child documents:

%%%%%%%%%%%%%%%%%%%%%%%%%%%%%%%%%%%%%%%%
\paragraph{Restrictions.}

Please note the following restrictions:
\begin{itemize}
\item
|\childdocmain| must be called with one argument \textit{main}
to ensure compatibility with earlier version of the package.
It must either be empty (|\childdocmain{}|)
or precisely match the filename of the main file in which it is specified.
See \secref{sec:detection} for further information.
\item
The filename \textit{main} must be specified without the |.tex| extension.
\item
The filename \textit{main} is case sensitive
(even in case-insensitive file systems)
due to internal string comparison.
\item
The argument \textit{main} should be fully expanded, it cannot be a macro.
\item
Subdirectories and special characters should be avoided in filenames.
\item
The command |\childdocmain{|\textit{main}|}| must be followed by a whitespace.
It should not be followed immediately by another command
or by a comment mark `|%|'.
This is because the \TeX{} parser reads the token immediately following
the argument of |\childdocmain| and puts it
at the beginning of every child section;
however, a white\-space is ignored.
\end{itemize}

%%%%%%%%%%%%%%%%%%%%%%%%%%%%%%%%%%%%%%%%
\paragraph{Content of Main File.}

It is advisable to place all content in the child files included by |\include|.
Any output contained in the main file will appear in all child documents
unless suppressed manually;
it cannot be suppressed automatically by the |\includeonly| directive
and thus should normally be avoided.
A method to include some content in the main file
by means of conditional processing is described in \secref{sec:conditional}.

%%%%%%%%%%%%%%%%%%%%%%%%%%%%%%%%%%%%%%%%
\paragraph{Page Numbering.}

When only a part of the document is compiled,
the appropriate numbering of pages
(as well as other status parameters)
is determined from the |.aux| files.
The latter contain information from previous passes.
However this information needs to propagate through
all intermediate child documents.
Therefore the page numbering in child documents may well
be inconsistent until the complete document is compiled at least once.

A useful (if unconventional) way to always ensure a consistent
page numbering is to restart the numbering in each child document
and denote the pages by `\textit{child}|.|\textit{page}'
where \textit{child} represents the chapter/section number of the child file.
This can be achieved by the command
|\numberwithin{page}{|\textit{child}|}|
of the \textsf{amsmath} package
where \textit{child} can be |chapter| or |section|
depending on the chosen structuring.
Alternatively, one can modify the macro |\thepage| appropriately
and reset the counter |page| at the start of each child file.

%%%%%%%%%%%%%%%%%%%%%%%%%%%%%%%%%%%%%%%%%%%%%%%%%%%%%%%%%%%%%%%%%%%%%%%%%%%%%%%%
\subsection{Conditional Processing}
\label{sec:conditional}

The package provides a mechanism to compile different versions
of a document. To customise the versions further some conditional processing
can come in handy to distinguish which version is being compiled.
The package provides two macros to describe the compilation context:

%%%%%%%%%%%%%%%%%%%%%%%%%%%%%%%%%%%%%%%%
\DescribeMacro{\ifchilddoc}
The conditional |\ifchilddoc| distinguishes between the compilation of
child documents and the main document:
%
\begin{center}
|\ifchilddoc |\textit{child-code}| |[|\||else |\textit{main-code}]| \||fi|
\end{center}

%%%%%%%%%%%%%%%%%%%%%%%%%%%%%%%%%%%%%%%%
\DescribeMacro{\childdocname}
\DescribeMacro{\childdocjob}
The macro |\childdocname| contains the filename (without extension)
of the main or child file being processed.
Note that |\childdocjob| will always contain the name of the main file.

%%%%%%%%%%%%%%%%%%%%%%%%%%%%%%%%%%%%%%%%
\paragraph{Title Page.}

Conditional processing can be used to include a title or banner page
in the main document when proper precautions are taken.
Importantly, the code in the main file should ensure that the page counter
(as well as other status parameters which are stored in the |.aux| files)
takes the same value after the conditional processing.
Otherwise the page numbers may take divergent values
depending on which part is compiled.

For example, a title page could be declared by:
%
\begin{center}
\begin{tabular}{l}
|\ifchilddoc\||else|\\
|\addtocounter{page}{-1}|\\
\textit{code for title page}\\
|\newpage|\\
|\||fi|
\end{tabular}
\end{center}
%
A banner page for the child documents can be generated by:
%
\begin{center}
\begin{tabular}{l}
|\ifchilddoc|\\
|\addtocounter{page}{-1}|\\
\textit{code for banner page}\\
|\newpage|\\
|\||fi|
\end{tabular}
\end{center}
%
Here one could write a message such as:
\begin{center}
|This is the part \childdocname{} of \childdocjob{}.|
\end{center}

%%%%%%%%%%%%%%%%%%%%%%%%%%%%%%%%%%%%%%%%%%%%%%%%%%%%%%%%%%%%%%%%%%%%%%%%%%%%%%%%
\subsection{Flags}
\label{sec:flags}

The package makes it easy to generate different versions
of the main or child documents.
To this end compilation flags can be defined
and assigned different default values.
They will be particularly useful in conjunction
with the forwarding mechanism described in \secref{sec:forward}.

For example, it may be useful to have a flag |\version|
which can be set to |draft| or |final|.
The document source will contain some conditional code
depending on the value of |\version|.
Suppose further, the flag should default to |final| for the main file
and to |draft| for child files
which is a natural assignment for editing the document.
This is achieved by placing the following code
in the preamble of the main document
(below the |\childdocmain| directive):
%
\begin{center}
\begin{tabular}{l}
|\ifchilddoc|\\
|\providecommand{\version}{draft}|\\
|\||else|\\
|\providecommand{\version}{final}|\\
|\||fi|
\end{tabular}
\end{center}
%
The definition by |\providecommand| makes sure
that previous definitions are not overwritten.
Further statements |\providecommand{\version}{...}|
can thus be added before the above code to override it.

For the main file, one might add a line
(between |\childdocmain| and the above block)
%
\begin{center}
|%\ifchilddoc\||else\providecommand{\version}{draft}\||fi|
\end{center}
%
which can be uncommented to produce a draft version.
Likewise one can add a line to the very top of a child file
(above the |\childdocof{|\textit{main}|}| directive)
%
\begin{center}
|%\providecommand{\version}{final}|
\end{center}
%
which can be uncommented to produce the final version of this child document.

%%%%%%%%%%%%%%%%%%%%%%%%%%%%%%%%%%%%%%%%%%%%%%%%%%%%%%%%%%%%%%%%%%%%%%%%%%%%%%%%
\subsection{Forwarding}
\label{sec:forward}

Different versions of the main or child documents
using compilation flags as described in \secref{sec:flags}
can be (permanently) stored in different files
for convenient compilation, viewing and distribution.
To this end, the package defines a command
to pass on compilation to a different file:

%%%%%%%%%%%%%%%%%%%%%%%%%%%%%%%%%%%%%%%%
\DescribeMacro{\childdocforward}
The command |\childdocforward| redirects processing to
another source file:
%
\begin{center}
\begin{tabular}{l}
|\input{childdoc.def}|\\
|\childdocforward[|\textit{main}|]{|\textit{dest}|}|\\
\end{tabular}
\end{center}
%
The argument \textit{dest} is the destination file
(without extension).
It should be the main file or one of the child files.
Note that further \textsf{childdoc} directives
such as |\childdocof| and |\childdocforward|
in the indicated file will be processed in this form.
The optional argument \textit{main}
passes on directly to the main file \textit{main}
while pretending to compile the child \textit{dest}.
This form behaves as if \textit{dest}
issues |\childdocof{|\textit{main}|}| right away,
and no further \textsf{childdoc} directives will be processed.

%%%%%%%%%%%%%%%%%%%%%%%%%%%%%%%%%%%%%%%%
\DescribeMacro{\...prefix}
In the alternative form |\childdocforwardprefix|,
%
\begin{center}
\begin{tabular}{l}
|\input{childdoc.def}|\\
|\childdocforwardprefix[|\textit{main}|]{|\textit{prefix}|}{|\textit{dest}|}|
\end{tabular}
\end{center}
%
the destination file is determined by a pattern
depending on the current file:
To make this work, the current file must be called
`{\textit{prefix}\hspace{0.2em}\textit{suffix}}'
with \textit{prefix} matching precisely the argument.
Processing is then passed on to the file
`{\textit{dest}\hspace{0.2em}\textit{suffix}}'.
Surely, the same effect is achieved by
directly specifying the
argument `{\textit{dest}\hspace{0.2em}\textit{suffix}}'
in the first form.
However, that requires to set up a different file
for each child. With the alternative form of the command
all these files can have exactly the same content
which simplifies setting them up and maintaining them.

For example, the following file |draft.tex|
with a compilation flag |\version| as described in \secref{sec:flags}
compiles the main document as a draft:
%
\begin{center}
\begin{tabular}{l}
|\def\version{draft}|\\
|\input{childdoc.def}|\\
|\childdocforward{|\textit{main}|}|
\end{tabular}
\end{center}
%
Likewise, the following files |final|\textit{nn}|.tex|
compile the final version of the child document
|child|\textit{nn}|.tex|:
%
\begin{center}
\begin{tabular}{l}
|\def\version{final}|\\
|\input{childdoc.def}|\\
|\childdocforwardprefix{final}{child}|
\end{tabular}
\end{center}
%

Note that when several versions of a main file and/or of each child file
are to be generated, it may be convenient to set up a |Makefile| or
shell script to automatise the process.

%%%%%%%%%%%%%%%%%%%%%%%%%%%%%%%%%%%%%%%%%%%%%%%%%%%%%%%%%%%%%%%%%%%%%%%%%%%%%%%%
\subsection{Command Line Processing}
\label{sec:commandline}

The effect of redirection files can also be achieved by invoking
the \LaTeX{} compiler with a more elaborate command line.
Most conveniently this should be done as part
of a shell script or a |Makefile|.

When using \textsf{childdoc} in the main file, the following
command lines effectively perform a redirection
(note that depending on the shell being used,
backslashes may have to be doubled: `|\|' $\to$ `|\\|'):
%
\begin{center}
|... -jobname "|\textit{target}|" |\\|"|[\textit{flags}]%
|\input{childdoc.def}\childdocforward[|\textit{main}|]{|\textit{dest}|}"|
\end{center}
%
Here \textit{target} is the name of the output file,
\textit{main} is the name of the main file
and \textit{dest} is the name of the main or child file to be processed
(all filenames without extensions).
The optional argument \textit{main} can be omitted
if \textit{main} matches \textit{dest}.
Optionally, compilation \textit{flags} can be defined via |\def| commands.
This command line makes the \TeX{} engine believe
it is compiling the file \textit{target}
whose content is specified as the latter parameter.
The provided code then forwards the processing to
\textit{main} or \textit{dest} as described in \secref{sec:forward}.

%%%%%%%%%%%%%%%%%%%%%%%%%%%%%%%%%%%%%%%%%%%%%%%%%%%%%%%%%%%%%%%%%%%%%%%%%%%%%%%%
\subsection{Include by Input}
\label{sec:input}

Including child documents by |\include| has some restrictions by design.
Most notably, the content of a child document always occupies
its own set of pages; pages cannot be shared between child documents.
Usually, this behaviour makes perfect sense
because each child document contain an essential part of the document.
However, in some situations it may be desirable to compose
a document from a collection of parts
without having mandatory page breaks between then.
For this case, the package
provides a mechanism to include parts
by |\input| which can also be processed individually.
However, by construction this mechanism
requires manual handling of the content to be output.

%%%%%%%%%%%%%%%%%%%%%%%%%%%%%%%%%%%%%%%%
\DescribeMacro{\ifchilddocmanual}
The main file should be prepared as usual, see \secref{sec:include}.
However, the document body must make a distinction
between processing of an individual part and of the main document, e.g.:
%
\begin{center}
\begin{tabular}{l}
|\ifchilddocmanual|\\
|\input{\childdocname}|\\
|\||else|\\
\textit{document body with }|\input{|\textit{part}|}|\\
|\||fi|
\end{tabular}
\end{center}
%
The conditional |\ifchilddocmanual| is true whenever
a part to be included by |\input| is being compiled,
and the name of the part is stored in |\childdocname|.

%%%%%%%%%%%%%%%%%%%%%%%%%%%%%%%%%%%%%%%%
\DescribeMacro{\childdocby}
Each part to be included by |\input| should start with:
%
\begin{center}
\begin{tabular}{l}
|\input{childdoc.def}|\\
|\childdocby{|\textit{main}|}|\\
\end{tabular}
\end{center}
%
The directive |\childdocby| is similar to |\childdocof|
described in \secref{sec:include},
but the subsequent selection of content must be done manually.
To that end, both |\ifchilddoc| and |\ifchilddocmanual|
will be true upon processing of a part,
and the name of the part is stored in |\childdocname|.
Note that |\jobname| will be set to the filename of the current part
so that each part receives an individual |.aux| file
that does not interfere with the |.aux| file(s) of the main document.
This behaviour can be altered by the alternative form
|\childdocby[*]{|\textit{main}|}| (with a non-empty optional argument)
which uses the |.aux| file of the main document
by setting |\jobname| to \textit{main}.

%%%%%%%%%%%%%%%%%%%%%%%%%%%%%%%%%%%%%%%%%%%%%%%%%%%%%%%%%%%%%%%%%%%%%%%%%%%%%%%%
\subsection{Driver Development}
\label{sec:driver}

The \textsf{childdoc} mechanism can also be use for the development
of definition files such as \LaTeX{} styles or classes.
This case differs from the above setup with multiple parts
included by |\include| in that no |\includeonly| should be invoked.
This can be achieved by starting the include file
(before |\ProvidesPackage|) with:
%
\begin{center}
\begin{tabular}{l}
|\input{childdoc.def}|\\
|\childdocforward{|\textit{main}|}|\\
\end{tabular}
\end{center}
%
or alternatively with:
%
\begin{center}
\begin{tabular}{l}
|\input{childdoc.def}|\\
|\childdocby{|\textit{main}|}|\\
\end{tabular}
\end{center}
%
Both forms have slightly different effects as described above.
The main file is prepared as usual, see \secref{sec:include}.

%%%%%%%%%%%%%%%%%%%%%%%%%%%%%%%%%%%%%%%%%%%%%%%%%%%%%%%%%%%%%%%%%%%%%%%%%%%%%%%%
\subsection{Legacy Detection}
\label{sec:detection}

The directive |\childdocmain| in the main file can detect
whether the complete document or merely a child is to be compiled
even without using the directive |\childdocof|.
This method is deprecated because it is less robust
and there is no compelling reason to use it;
it is merely provided for backward compatibility
and it may be removed in future versions.

If the detection mechanism is to be used,
it is mandatory to correctly specify
the filename of the main file as the argument of |\childdocmain|:
%
\begin{center}
\begin{tabular}{l}
|\input{childdoc.def}|\\
|\childdocmain{|\textit{main}|}|\\
\end{tabular}
\end{center}
%
If |\jobname| does not match the argument \textit{main} of |\childdocmain|,
it is assumed that |\jobname| points to the child file to be compiled.
When using |\childdocmain| with the main file specified as argument,
it suffices to start a child file
with just |\input{|\textit{main}|}|
without loading of the package and using |\childdocof|.
If instead all processing is done
with the appropriate \textsf{childdoc} directives,
the argument of \textit{main} of |\childdocmain| can be empty.

An alternative version of the command line processing described
in \secref{sec:commandline} using the detection mechanism reads:
%
\begin{center}
|... -jobname "|\textit{target}|" "|[\textit{flags}]%
[|\def\jobname{|\textit{dest}|}|]|\input{|\textit{main}|}"|
\end{center}

%%%%%%%%%%%%%%%%%%%%%%%%%%%%%%%%%%%%%%%%%%%%%%%%%%%%%%%%%%%%%%%%%%%%%%%%%%%%%%%%
\subsection{Manual Code}
\label{sec:manual}

In case one cannot be certain whether the definitions file |childdoc.def|
is installed on the target \TeX{} distribution
and one prefers not to ship it,
it is conceivable to paste a few relevant commands into the sources.

To that end, drop all statements |\input{childdoc.def}|
and perform the replacements as outlined below.
Instead of |\childdocmain{|\textit{main}|}| add the following code
to the top of the main file:
%
\begin{center}
\begin{tabular}{l}
|\||ifdefined\childdocname\endinput\||fi\newif\ifchilddoc|\\
|\edef\childdocname{\scantokens\expandafter{\jobname\noexpand}}|\\
|\def\childdocmain{|\textit{main}|}\||ifx\childdocmain\childdocname\||else|\\
|\childdoctrue\includeonly{\childdocname}\let\jobname\childdocmain\||fi|\\
\end{tabular}
\end{center}
%
Instead of |\childdocof{|\textit{main}|}| just include the main file
at the top of each child file:
%
\begin{center}
|\input{|\textit{main}|}|
\end{center}
%
A simple redirection |\childdocforward{|\textit{dest}|}| is achieved by:
%
\begin{center}
|\def\jobname{|\textit{dest}|}\input{\jobname}|
\end{center}
%
The redirection with prefix
|\childdocforwardprefix[|\textit{prefix}|]{|\textit{dest}|}|
is accomplished by:
%
\begin{center}
\begin{tabular}{l}
|{\edef\jobname{\scantokens\expandafter{\jobname\noexpand}}|\\
|\def\redirectjob |\textit{prefix}|#1~~~{\gdef\jobname{|\textit{dest}|#1}}|\\
|\expandafter\redirectjob\jobname~~~}\input{\jobname}|
\end{tabular}
\end{center}

In an alternative approach,
child documents can be compiled by a specific command line
without additional code or specific definitions:
%
\begin{center}
|... -jobname "|\textit{target}|" "|[\textit{flags}]%
|\includeonly{|\textit{dest}|}\input{|\textit{main}|}"|
\end{center}
%

%%%%%%%%%%%%%%%%%%%%%%%%%%%%%%%%%%%%%%%%%%%%%%%%%%%%%%%%%%%%%%%%%%%%%%%%%%%%%%%%
%%%%%%%%%%%%%%%%%%%%%%%%%%%%%%%%%%%%%%%%%%%%%%%%%%%%%%%%%%%%%%%%%%%%%%%%%%%%%%%%
\section{Information}

%%%%%%%%%%%%%%%%%%%%%%%%%%%%%%%%%%%%%%%%%%%%%%%%%%%%%%%%%%%%%%%%%%%%%%%%%%%%%%%%
\subsection{Copyright}

Copyright \copyright{} 2017--2018 Niklas Beisert

This work may be distributed and/or modified under the
conditions of the \LaTeX{} Project Public License, either version 1.3
of this license or (at your option) any later version.
The latest version of this license is in
  \url{http://www.latex-project.org/lppl.txt}
and version 1.3 or later is part of all distributions of \LaTeX{}
version 2005/12/01 or later.

This work has the LPPL maintenance status `maintained'.

The Current Maintainer of this work is Niklas Beisert.

This work consists of the files |README.txt|, |childdoc.ins| and |childdoc.dtx|
as well as the derived files |childdoc.def|, |cdocsamp.tex|
with |cdocsch1.tex|, |cdocsch2.tex|, |cdocspt3.tex|, |cdocspt4.tex|,
|cdocsdrf.tex|, |cdocsfn1.tex|, |cdocsfn2.tex|
as well as |childdoc.pdf|.

%%%%%%%%%%%%%%%%%%%%%%%%%%%%%%%%%%%%%%%%%%%%%%%%%%%%%%%%%%%%%%%%%%%%%%%%%%%%%%%%
\subsection{Files and Installation}

The package consists of the files:
%
\begin{center}
\begin{tabular}{ll}
    |README.txt|   & readme file \\
    |childdoc.ins| & installation file \\
    |childdoc.dtx| & source file \\
    |childdoc.def| & definition file \\
    |cdocsamp.tex| & sample main file \\
    |cdocsch1.tex| & sample include file \\
    |cdocsch2.tex| & sample include file \\
    |cdocspt3.tex| & sample part file \\
    |cdocspt4.tex| & sample part file \\
    |cdocsdrf.tex| & sample redirection file \\
    |cdocsfn1.tex| & sample redirection file \\
    |cdocsfn2.tex| & sample redirection file \\
    |childdoc.pdf| & manual
\end{tabular}
\end{center}
%
The distribution consists of the files
|README.txt|, |childdoc.ins| and |childdoc.dtx|.
%
\begin{itemize}
\item
Run (pdf)\LaTeX{} on |childdoc.dtx|
to compile the manual |childdoc.pdf| (this file).
\item
Run \LaTeX{} on |childdoc.ins| to create the definitions file |childdoc.def|
and the sample |cdocsamp.tex| with include files
|cdocsch1.tex|, |cdocsch2.tex|, |cdocspt3.tex|, |cdocspt4.tex|,
|cdocsdrf.tex|, |cdocsfn1.tex|, |cdocsfn2.tex|.
Then copy the file |childdoc.def| to an appropriate directory of your \LaTeX{}
distribution, e.g.\ \textit{texmf-root}|/tex/latex/childdoc|.
\end{itemize}

%%%%%%%%%%%%%%%%%%%%%%%%%%%%%%%%%%%%%%%%%%%%%%%%%%%%%%%%%%%%%%%%%%%%%%%%%%%%%%%%
\subsection{Related CTAN Packages}

There are several other packages which offer a similar functionality:
%
\begin{itemize}
\item
The packages
\href{http://ctan.org/pkg/docmute}{\textsf{docmute}},
\href{http://ctan.org/pkg/includex}{\textsf{includex}} and
\href{http://ctan.org/pkg/standalone}{\textsf{standalone}}
provide commands to include only the document body of
a child file thus allowing both files to be compiled individually.
\item
The packages \href{http://ctan.org/pkg/subdocs}{\textsf{subdocs}}
and \href{http://ctan.org/pkg/subfiles}{\textsf{subfiles}}
provide structures in which the main and child documents can be
encapsulated and allowing them to be compiled individually.
The inclusion mechanism is different from the conventional |\include|.
\item
The package \href{http://ctan.org/pkg/combine}{\textsf{combine}}
is an elaborate solution to combine several documents into one.
\end{itemize}
%
See also the CTAN topic \href{http://ctan.org/topic/subdocs}{\textsf{subdocs}}
for further related packages.
The present package differs from the above solutions in that
a document structure constructed with the conventional |\include| mechanism
just needs two extra commands at the top of every file
such that all constituent files can be compiled individually.

%%%%%%%%%%%%%%%%%%%%%%%%%%%%%%%%%%%%%%%%%%%%%%%%%%%%%%%%%%%%%%%%%%%%%%%%%%%%%%%%
%\subsection{Feature Suggestions}
%
%The following is a list of features which may be useful for future
%versions of this package:
%%
%\begin{itemize}
%\item
%\ldots
%\end{itemize}

%%%%%%%%%%%%%%%%%%%%%%%%%%%%%%%%%%%%%%%%%%%%%%%%%%%%%%%%%%%%%%%%%%%%%%%%%%%%%%%%
\subsection{Revision History}

%%%%%%%%%%%%%%%%%%%%%%%%%%%%%%%%%%%%%%%%
\paragraph{v2.0:} 2018/12/30

\begin{itemize}
\item
immediate forward processing
\item
added |\childdocby| mechanism
\item
manual restructured
\end{itemize}

%%%%%%%%%%%%%%%%%%%%%%%%%%%%%%%%%%%%%%%%
\paragraph{v1.6:} 2018/01/17

\begin{itemize}
\item
application for development of include files
\item
corrections to manual
\end{itemize}

%%%%%%%%%%%%%%%%%%%%%%%%%%%%%%%%%%%%%%%%
\paragraph{v1.5:} 2017/05/21

\begin{itemize}
\item
more complete structuring introduced
\item
|\childdocof| introduced
\item
|\childdoc| renamed to |\childdocmain|
\item
|\childredirect| renamed to |\childdocforward| and |\childdocforwardprefix|
and functionality expanded
\end{itemize}

%%%%%%%%%%%%%%%%%%%%%%%%%%%%%%%%%%%%%%%%
\paragraph{v1.0:} 2017/04/27

\begin{itemize}
\item
manual and install package
\item
first version published on CTAN
\end{itemize}

%%%%%%%%%%%%%%%%%%%%%%%%%%%%%%%%%%%%%%%%
\paragraph{v0.6:} 2017/04/26

\begin{itemize}
\item
redirection mechanism added
\end{itemize}

%%%%%%%%%%%%%%%%%%%%%%%%%%%%%%%%%%%%%%%%
\paragraph{v0.5:} 2017/04/26

\begin{itemize}
\item
functionality in definition file
\end{itemize}


%%%%%%%%%%%%%%%%%%%%%%%%%%%%%%%%%%%%%%%%%%%%%%%%%%%%%%%%%%%%%%%%%%%%%%%%%%%%%%%%
%%%%%%%%%%%%%%%%%%%%%%%%%%%%%%%%%%%%%%%%%%%%%%%%%%%%%%%%%%%%%%%%%%%%%%%%%%%%%%%%
%%%%%%%%%%%%%%%%%%%%%%%%%%%%%%%%%%%%%%%%%%%%%%%%%%%%%%%%%%%%%%%%%%%%%%%%%%%%%%%%
\appendix

\settowidth\MacroIndent{\rmfamily\scriptsize 000\ }

 \DocInput{childdoc.dtx}

\end{document}
%</driver>
% \fi
%
% %%%%%%%%%%%%%%%%%%%%%%%%%%%%%%%%%%%%%%%%%%%%%%%%%%%%%%%%%%%%%%%%%%%%%%%%%%%%%%
% %%%%%%%%%%%%%%%%%%%%%%%%%%%%%%%%%%%%%%%%%%%%%%%%%%%%%%%%%%%%%%%%%%%%%%%%%%%%%%
% \section{Sample}
%\iffalse
%<*samplemain>
%\fi
%
% The following presents a sample document
% with two chapters, two parts, a title page,
% a compile flag as well as three forwarding files to set the flag.
% It consists of eight |.tex| files:
% \begin{center}
% \begin{tabular}{ll}
% |cdocsamp.tex|&main file\\
% |cdocsch1.tex|&include file for chapter 1\\
% |cdocsch2.tex|&include file for chapter 2\\
% |cdocspt3.tex|&include file for part 3\\
% |cdocspt4.tex|&include file for part 4\\
% |cdocsdrf.tex|&forwarding file for main file in draft mode\\
% |cdocsfi1.tex|&forwarding file for final version of chapter 1\\
% |cdocsfi2.tex|&forwarding file for final version of chapter 2\\
% \end{tabular}
% \end{center}
% Each of the eight files can be compiled directly by the \LaTeX{} compiler.
%
% %%%%%%%%%%%%%%%%%%%%%%%%%%%%%%%%%%%%%%
% \paragraph{Main File.}
%
% The main file is called |cdocsamp.tex|.
%
% Load the \textsf{childdoc} definitions and
% declare the filename for the main document:
%    \begin{macrocode}
\input{childdoc.def}
\childdocmain{}
%    \end{macrocode}

% Optional override for |\version| flag:
%    \begin{macrocode}
%%\ifchilddoc\else\providecommand{\version}{draft}\fi
%    \end{macrocode}

% Define the default values for the |\version| flag
% (|final| for the main file and |draft| for childs):
%    \begin{macrocode}
\ifchilddoc
\providecommand{\version}{draft}
\else
\providecommand{\version}{final}
\fi
%    \end{macrocode}

% Load the standard document class:
%    \begin{macrocode}
\documentclass[12pt]{article}
%    \end{macrocode}

% Start the document body:
%    \begin{macrocode}
\begin{document}
%    \end{macrocode}

% Declare a title page.
% Print title, part of document being processed and version flag:
%    \begin{macrocode}
\addtocounter{page}{-1}
\begin{center}
{\LARGE\bfseries{}childdoc example\par}
\vspace{1cm}
\ifchilddoc
\ifchilddocmanual part\else chapter\fi:
`\childdocname' of `\childdocjob'\par
\else
main document: `\childdocjob'\par
\fi
version: \version\par
\end{center}
\newpage
%    \end{macrocode}

% Manually include selected file,
% otherwise process as usual:
%    \begin{macrocode}
\ifchilddocmanual
\section*{part `\childdocname'}
\input{\childdocname}
\else
%    \end{macrocode}

% Include the two chapters:
%    \begin{macrocode}
\include{cdocsch1}
\include{cdocsch2}
%    \end{macrocode}

% Include the two parts unless only chapters should be displayed:
%    \begin{macrocode}
\ifchilddoc\else
\section{part three}
\input{cdocspt3}
\section{part four}
\input{cdocspt4}
\fi
%    \end{macrocode}

% Process as usual until here:
%    \begin{macrocode}
\fi
%    \end{macrocode}

% End of document body:
%    \begin{macrocode}
\end{document}
%    \end{macrocode}
%\iffalse
%</samplemain>
%\fi
%
% %%%%%%%%%%%%%%%%%%%%%%%%%%%%%%%%%%%%%%
% \paragraph{Chapter Include Files.}
%
% The include files are called |cdocsch1.tex| and |cdocsch2.tex|.
%
%\iffalse
%<*samplechap1|samplechap2>
%\fi

% Optional override for |\version| flag:
%    \begin{macrocode}
%%\providecommand{\version}{final}
%    \end{macrocode}

% Include the main document:
%    \begin{macrocode}
\input{childdoc.def}
\childdocof{cdocsamp}
%    \end{macrocode}

%\iffalse
%</samplechap1|samplechap2>
%\fi
%
%\iffalse
%<*samplechap1>
%\fi
% Some text for chapter 1:
%    \begin{macrocode}
\section{one}
some text in chapter one
%    \end{macrocode}

%\iffalse
%</samplechap1>
%\fi
% Some text for chapter 2:
%\iffalse
%<*samplechap2>
%\fi
%    \begin{macrocode}
\section{two}
more text in chapter two
%    \end{macrocode}

%\iffalse
%</samplechap2>
%\fi
%
% %%%%%%%%%%%%%%%%%%%%%%%%%%%%%%%%%%%%%%
% \paragraph{Part Include Files.}
%
% The include files are called |cdocspt3.tex| and |cdocspt4.tex|.
%
%\iffalse
%<*samplepart3|samplepart4>
%\fi

% Optional override for |\version| flag:
%    \begin{macrocode}
%%\providecommand{\version}{final}
%    \end{macrocode}

% Include the main document:
%    \begin{macrocode}
\input{childdoc.def}
\childdocby{cdocsamp}
%    \end{macrocode}

%\iffalse
%</samplepart3|samplepart4>
%\fi
%
%\iffalse
%<*samplepart3>
%\fi
% Some text for part 3:
%    \begin{macrocode}
some text in part three
%    \end{macrocode}

%\iffalse
%</samplepart3>
%\fi
% Some text for part 4:
%\iffalse
%<*samplepart4>
%\fi
%    \begin{macrocode}
more text in part four
%    \end{macrocode}

%\iffalse
%</samplepart4>
%\fi
%
% %%%%%%%%%%%%%%%%%%%%%%%%%%%%%%%%%%%%%%
% \paragraph{Forwarding for a Complete Draft.}
%
% The following forwarding file |cdocsdrf.tex|
% compiles the main document in draft mode:
%\iffalse
%<*sampledraft>
%\fi
%    \begin{macrocode}
\def\version{draft}
\input{childdoc.def}
\childdocforward{cdocsamp}
%    \end{macrocode}

%\iffalse
%</sampledraft>
%\fi
%
% %%%%%%%%%%%%%%%%%%%%%%%%%%%%%%%%%%%%%%
% \paragraph{Forwarding for Final Version of the Chapters.}
%
% The following forwarding files |cdocsfn1.tex| and |cdocsfn2.tex|
% (with identical content)
% compile the final versions of the child documents
% |cdocsch1.tex| and |cdocsch2.tex|, respectively:
%\iffalse
%<*samplefinal>
%\fi
%    \begin{macrocode}
\def\version{final}
\input{childdoc.def}
\childdocforwardprefix[cdocsamp]{cdocsfn}{cdocsch}
%    \end{macrocode}

%\iffalse
%</samplefinal>
%\fi
%
% %%%%%%%%%%%%%%%%%%%%%%%%%%%%%%%%%%%%%%
% \paragraph{Command Line Processing.}
%
% The following three command lines generate the output files
% |cdocscld|, |cdocscl1| and |cdocscl2|
% which should be identical to
% |cdocsdrf|, |cdocsch1| and |cdocsfn2|, respectively:
% \begin{center}
% \begin{tabular}{l}
% |latex -jobname cdocscld \|\\
% |  "\def\version{draft}\input{childdoc.def}\childdocforward{cdocsamp}"|\\
% |latex -jobname cdocscl1 \|\\
% |  "\input{childdoc.def}\childdocforward[cdocsamp]{cdocsch1}"|\\
% |latex -jobname cdocscl2 \|\\
% |  "\def\version{final}\input{childdoc.def}\childdocforward{cdocsch2}"|
% \end{tabular}
% \end{center}
% Note that the trailing backslash on each first line
% merely continues the input to the second line
% (for convenient cut ant paste).
% Furthermore, the command |latex| can be replaced by any
% of its alternative versions such as |pdflatex|.
%
% %%%%%%%%%%%%%%%%%%%%%%%%%%%%%%%%%%%%%%%%%%%%%%%%%%%%%%%%%%%%%%%%%%%%%%%%%%%%%%
% %%%%%%%%%%%%%%%%%%%%%%%%%%%%%%%%%%%%%%%%%%%%%%%%%%%%%%%%%%%%%%%%%%%%%%%%%%%%%%
% \section{Implementation}
%\iffalse
%<*package>
%\fi
%
% This section describes the definitions file |childdoc.def|.

% The definitions cannot be loaded using |\usepackage| or |\RequirePackage|
% which has a mechanism to prevent loading a style file more than once.
% When loading the definitions by means of |\input|
% multiple instances have to be prevented manually:
%\iffalse
%This code needs to be before the `\ProvidesFile' directive
%which is defined at the beginning of this file.
%Therefore it is also placed there and commented out here.
%</package>
%<*discard>
%\fi
%    \begin{macrocode}
\ifdefined\childdocmain\endinput\fi
%    \end{macrocode}
%\iffalse
%</discard>
%<*package>
%\fi
%
% \macro{\ifchilddoc}
% \macro{\ifchilddocmanual}
% The conditional |\ifchilddoc| tells whether a
% child (true) or main (false) document is being compiled.
% The conditional |\ifchilddocmanual| tells whether
% the |\includeonly| mechanism is used (false) or
% the selection of child files must be performed manually (true).
% The definitions initialise to false:
%    \begin{macrocode}
\newif\ifchilddoc
\newif\ifchilddocmanual
%    \end{macrocode}

% \macro{\childdocname}
% \macro{\childdocjob}
% The macro |\childdocname| stores the name of the main document
% to be compiled. The macro |\childdocjob| stores the name of
% the document on which the \LaTeX{} compiler was originally invoked.
% The content of |\jobname| cannot be compared
% to filenames specified in the source due to different catcodes.
% The following code rescans |\jobname|, stores the result
% in |\childdocname| and saves a copy in |\childdocjob|:
%    \begin{macrocode}
\edef\childdocname{\scantokens\expandafter{\jobname\noexpand}}
\let\childdocjob\childdocname
%    \end{macrocode}

% \macro{\childdocdisable}
% The macro |\childdocdisable| prevents the main file
% from being processed more than once.
% At this stage, the main document command |\childdocmain|
% is assumed to be called once again where it should do nothing.
% Any subsequent call to it should prevent
% a secondary processing of the main document
% It overwrites the forwarding commands
% |\childdocof| and |\childdocforward|
% with empty macros to prevent further inclusions of the main document:
%    \begin{macrocode}
\newcommand{\childdocdisable}
{
  \renewcommand{\childdocmain}[1]{\renewcommand{\childdocmain}[1]{\endinput}}
  \renewcommand{\childdocof}[1]{}
  \renewcommand{\childdocby}[2][]{}
  \renewcommand{\childdocforward}[2][]{}
  \renewcommand{\childdocdisable}{}
}
%    \end{macrocode}

% \macro{\childdocmain}
% The macro |\childdocmain| is to be called at the top of the main file
% with nothing or the main filename (without extension) as argument.
% First, it breaks loops.
% If the argument is not empty and does not match |\childdocname|
% (which is set by the first inclusion of |childdoc.def|),
% |\ifchilddoc| is set to true, |\includeonly| is applied to the child file
% and |\jobname| is set to the main file
% (for proper handling of |.aux| files):
%    \begin{macrocode}
\newcommand{\childdocmain}[1]
{
  \childdocdisable\childdocmain{}
  \if?#1?\else
    \begingroup
      \def\childdoctmp{#1}
      \ifx\childdoctmp\childdocname
        \def\childdoctmp{}
      \else
        \def\childdoctmp
        {
          \childdoctrue
          \includeonly{\childdocname}
          \def\childdocjob{#1}
          \def\jobname{#1}
        }
      \fi
      \expandafter
    \endgroup
    \childdoctmp
  \fi
}
%    \end{macrocode}

% \macro{\childdocof}
% The command |\childdocof| redirects
% compilation to the main file |#1|.
%    \begin{macrocode}
\newcommand{\childdocof}[1]
{
  \childdocdisable
  \childdoctrue
  \includeonly{\childdocname}
  \def\jobname{#1}
  \def\childdocjob{#1}
  \input{#1}
}
%    \end{macrocode}

% \macro{\childdocby}
% The command |\childdocby| ....
%    \begin{macrocode}
\newcommand{\childdocby}[2][]
{
  \childdocdisable
  \childdoctrue
  \childdocmanualtrue
  \if?#1?\else
    \def\jobname{#2}
  \fi
  \def\childdocjob{#2}
  \input{#2}
  \endinput
}
%    \end{macrocode}

% \macro{\childdocforward}
% The command |\childdocforward| redirects
% compilation to the main file or
% (if the optional argument is given) a child file.
% Parameters are set as if the main file
% or a child file starting with |\childdocof| was compiled.
% Then compilation is handed over to the main file:
%    \begin{macrocode}
\newcommand{\childdocforward}[2][]
{
  \begingroup
    \if?#1?
      \def\childdoctmp
      {
        \def\childdocname{#2}
        \def\childdocjob{#2}
        \def\jobname{#2}
        \input{#2}
        \endinput
      }
    \else
      \def\childdoctmp
      {
        \childdocdisable
        \def\childdocname{#2}
        \childdoctrue
        \includeonly{#2}
        \def\childdocjob{#1}
        \def\jobname{#1}
        \input{#1}
        \endinput
      }
    \fi
    \expandafter
  \endgroup
  \childdoctmp
}
%    \end{macrocode}

% \macro{\childdocforwardprefix}
% The command |\childdocforwardprefix| redirects
% compilation to the main or a child file by means of a pattern.
% The prefix |#1| in the current filename is replaced by |#2|
% and the suffix of the current filename is kept
% (it is assumed that the filename does not contain the substring `|~~~|'
% which is used as a delimiter).
% Compilation is handed over to the new file by |\childdocforward|:
%    \begin{macrocode}
\newcommand{\childdocforwardprefix}[3][]
{
  \begingroup
    \def\childdocextract #2##1~~~{\def\childdoctmp{\childdocforward[#1]{#3##1}}}
    \expandafter\childdocextract\childdocname~~~
    \expandafter
  \endgroup
  \childdoctmp
}
%    \end{macrocode}

% \macro{\childdoc}
% The deprecated macro |\childdoc| is a legacy version of |\childdocmain|:
%    \begin{macrocode}
\newcommand{\childdoc}{\childdocmain}
%    \end{macrocode}

% \macro{\childdocredirect}
% The deprecated macro |\childdocredirect| is a legacy version
% of |\childdocforward| and |\childdocforwardprefix|:
%    \begin{macrocode}
\newcommand{\childdocredirect}[2][]
{
  \begingroup
    \if?#1?
      \def\childdoctmp{\childdocforward{#2}}
    \else
      \def\childdoctmp{\childdocforwardprefix{#1}{#2}}
    \fi
    \expandafter
  \endgroup
  \childdoctmp
}
%    \end{macrocode}

%\iffalse
%</package>
%\fi
%
\endinput
|\\
|\childdocmain{}|\\
\end{tabular}
\end{center}
at the very top of the main \LaTeX{} file,
in particular \emph{before} the |\documentclass| statement!
The argument of |\childdocmain| should be left empty
(but it must be present).

%%%%%%%%%%%%%%%%%%%%%%%%%%%%%%%%%%%%%%%%
\DescribeMacro{\childdocof}
Furthermore, add the commands
\begin{center}
\begin{tabular}{l}
|% \iffalse
%
% childdoc.dtx Copyright (C) 2017-2018 Niklas Beisert
%
% This work may be distributed and/or modified under the
% conditions of the LaTeX Project Public License, either version 1.3
% of this license or (at your option) any later version.
% The latest version of this license is in
%   http://www.latex-project.org/lppl.txt
% and version 1.3 or later is part of all distributions of LaTeX
% version 2005/12/01 or later.
%
% This work has the LPPL maintenance status `maintained'.
%
% The Current Maintainer of this work is Niklas Beisert.
%
% This work consists of the files childdoc.dtx and childdoc.ins
% and the derived files childdoc.def and cdocsamp.tex with
% cdocsch1.tex, cdocsch2.tex, cdocsdrf.tex, cdocsfn1.tex, cdocsfn2.tex.
%
%<package>\ifdefined\childdocmain\endinput\fi
%<package>\ProvidesFile{childdoc.def}[2018/12/30 v2.0 child document driver]
%<samplemain>\ProvidesFile{cdocsamp.tex}[2018/12/30 v2.0 sample for childdoc]
%<*driver>
%\ProvidesFile{childdoc.drv}[2018/12/30 v2.0 childdoc reference manual file]
\PassOptionsToClass{10pt,a4paper}{article}
\documentclass{ltxdoc}

\usepackage[margin=35mm]{geometry}
\usepackage{hyperref}
\usepackage{hyperxmp}
\usepackage[usenames]{color}

\hypersetup{colorlinks=true}
\hypersetup{pdfstartview=FitH}
\hypersetup{pdfpagemode=UseNone}
\hypersetup{pdfsource={}}
\hypersetup{pdflang={en-UK}}
\hypersetup{pdfcopyright={Copyright 2017-2018 Niklas Beisert.
  This work may be distributed and/or modified under the
  conditions of the LaTeX Project Public License, either version 1.3
  of this license or (at your option) any later version.}}
\hypersetup{pdflicenseurl={http://www.latex-project.org/lppl.txt}}
\hypersetup{pdfcontactaddress={ETH Zurich, ITP, HIT K,
  Wolfgang-Pauli-Strasse 27}}
\hypersetup{pdfcontactpostcode={8093}}
\hypersetup{pdfcontactcity={Zurich}}
\hypersetup{pdfcontactcountry={Switzerland}}
\hypersetup{pdfcontactemail={nbeisert@itp.phys.ethz.ch}}
\hypersetup{pdfcontacturl={http://people.phys.ethz.ch/\xmptilde nbeisert/}}

\newcommand{\secref}[1]{\hyperref[#1]{section \ref*{#1}}}

\parskip1ex
\parindent0pt
\let\olditemize\itemize
\def\itemize{\olditemize\parskip0pt}

\begin{document}

\title{The \textsf{childdoc} Package}
\hypersetup{pdftitle={The childdoc Package}}
\author{Niklas Beisert\\[2ex]
  Institut f\"ur Theoretische Physik\\
  Eidgen\"ossische Technische Hochschule Z\"urich\\
  Wolfgang-Pauli-Strasse 27, 8093 Z\"urich, Switzerland\\[1ex]
  \href{mailto:nbeisert@itp.phys.ethz.ch}
  {\texttt{nbeisert@itp.phys.ethz.ch}}}
\hypersetup{pdfauthor={Niklas Beisert}}
\hypersetup{pdfsubject={Manual for the LaTeX2e Package childdoc}}
\date{30 December 2018, \textsf{v2.0}}
\maketitle

\begin{abstract}\noindent
\textsf{childdoc} is a \LaTeXe{} package
that enables the direct compilation
of document sections included by |\include|
to individual files.
\end{abstract}

\begingroup
\parskip0ex
\tableofcontents
\endgroup

%%%%%%%%%%%%%%%%%%%%%%%%%%%%%%%%%%%%%%%%%%%%%%%%%%%%%%%%%%%%%%%%%%%%%%%%%%%%%%%%
%%%%%%%%%%%%%%%%%%%%%%%%%%%%%%%%%%%%%%%%%%%%%%%%%%%%%%%%%%%%%%%%%%%%%%%%%%%%%%%%
\section{Introduction}

\LaTeX{} provides a mechanism to structure a large document (such as a book)
into a main file and several child files (containing the chapters)
using the |\include| command.
This mechanism is beneficial for documents
which span hundreds of pages in order to
make the source file(s) more manageable.
Moreover, compilation can be restricted to
selected child files by means of the |\includeonly| command.
The latter feature can be used to reduce the compilation time while editing
(this was significantly more useful in the earlier days of \LaTeX{})
or to generate a smaller document which is easier to navigate.
Another application of |\includeonly| is to generate
documents consisting of selected parts of the complete document.

However, there are a few drawbacks of the plain |\include| mechanism:
\begin{itemize}
\item
The child files cannot be compiled on their own,
they can only be compiled via the main file.
A naive editing environment
(such as a text editor with an option
to have the current file processed by \LaTeX)
may require one to switch to the main file before compiling;
attempting to compile the child file produces errors.
\item
The main file must be modified (each time)
to adjust the |\includeonly| command
to the present needs. This easily leaves the main file in a messy state.
\item
The generated document will always carry the filename
of the main document. This is inconvenient if
several child files are to be compiled and
to be kept for distribution.
\end{itemize}

The present package provides a simple interface
to make child files individually compilable by \LaTeX{}.
Compiling a child file then has the same effect as compiling
the main file with an |\includeonly| command
to select the appropriate child.
Moreover the generated document will carry the name of the child
rather than the main file.
This resolves all three above issues.

This feature is meant to make the editing of books,
thesis documents and lecture notes somewhat more convenient.
However, the package can also be used efficiently for
composing a series of documents (such as exercise sheets)
which are typically distributed individually.
It then assists the author in generating the individual documents
(potentially in different versions)
as well as a document containing the collected series.
Another application is in developing style files
or other kinds of included material
where compilation of the style file could redirect
to a sample or test file.

%%%%%%%%%%%%%%%%%%%%%%%%%%%%%%%%%%%%%%%%%%%%%%%%%%%%%%%%%%%%%%%%%%%%%%%%%%%%%%%%
%%%%%%%%%%%%%%%%%%%%%%%%%%%%%%%%%%%%%%%%%%%%%%%%%%%%%%%%%%%%%%%%%%%%%%%%%%%%%%%%
\section{Usage}

First of all, the package \textsf{childdoc} is \emph{not} a standard
\LaTeXe{} |.sty| style file! Therefore it needs to be invoked in
a non-standard way.

%%%%%%%%%%%%%%%%%%%%%%%%%%%%%%%%%%%%%%%%%%%%%%%%%%%%%%%%%%%%%%%%%%%%%%%%%%%%%%%%
\subsection{Included Files}
\label{sec:include}

%%%%%%%%%%%%%%%%%%%%%%%%%%%%%%%%%%%%%%%%
\DescribeMacro{\childdocmain}
To use the package, add the commands
\begin{center}
\begin{tabular}{l}
|\input{childdoc.def}|\\
|\childdocmain{}|\\
\end{tabular}
\end{center}
at the very top of the main \LaTeX{} file,
in particular \emph{before} the |\documentclass| statement!
The argument of |\childdocmain| should be left empty
(but it must be present).

%%%%%%%%%%%%%%%%%%%%%%%%%%%%%%%%%%%%%%%%
\DescribeMacro{\childdocof}
Furthermore, add the commands
\begin{center}
\begin{tabular}{l}
|\input{childdoc.def}|\\
|\childdocof{|\textit{main}|}|\\
\end{tabular}
\end{center}
at the top of every child file \textit{child}
which is included by |\include{|\textit{child}|}|
from within the main file
(or at least for those files to be compiled individually).
The argument \textit{main} must be the filename of the main file.

There are a couple of
considerations in setting up the main and child documents:

%%%%%%%%%%%%%%%%%%%%%%%%%%%%%%%%%%%%%%%%
\paragraph{Restrictions.}

Please note the following restrictions:
\begin{itemize}
\item
|\childdocmain| must be called with one argument \textit{main}
to ensure compatibility with earlier version of the package.
It must either be empty (|\childdocmain{}|)
or precisely match the filename of the main file in which it is specified.
See \secref{sec:detection} for further information.
\item
The filename \textit{main} must be specified without the |.tex| extension.
\item
The filename \textit{main} is case sensitive
(even in case-insensitive file systems)
due to internal string comparison.
\item
The argument \textit{main} should be fully expanded, it cannot be a macro.
\item
Subdirectories and special characters should be avoided in filenames.
\item
The command |\childdocmain{|\textit{main}|}| must be followed by a whitespace.
It should not be followed immediately by another command
or by a comment mark `|%|'.
This is because the \TeX{} parser reads the token immediately following
the argument of |\childdocmain| and puts it
at the beginning of every child section;
however, a white\-space is ignored.
\end{itemize}

%%%%%%%%%%%%%%%%%%%%%%%%%%%%%%%%%%%%%%%%
\paragraph{Content of Main File.}

It is advisable to place all content in the child files included by |\include|.
Any output contained in the main file will appear in all child documents
unless suppressed manually;
it cannot be suppressed automatically by the |\includeonly| directive
and thus should normally be avoided.
A method to include some content in the main file
by means of conditional processing is described in \secref{sec:conditional}.

%%%%%%%%%%%%%%%%%%%%%%%%%%%%%%%%%%%%%%%%
\paragraph{Page Numbering.}

When only a part of the document is compiled,
the appropriate numbering of pages
(as well as other status parameters)
is determined from the |.aux| files.
The latter contain information from previous passes.
However this information needs to propagate through
all intermediate child documents.
Therefore the page numbering in child documents may well
be inconsistent until the complete document is compiled at least once.

A useful (if unconventional) way to always ensure a consistent
page numbering is to restart the numbering in each child document
and denote the pages by `\textit{child}|.|\textit{page}'
where \textit{child} represents the chapter/section number of the child file.
This can be achieved by the command
|\numberwithin{page}{|\textit{child}|}|
of the \textsf{amsmath} package
where \textit{child} can be |chapter| or |section|
depending on the chosen structuring.
Alternatively, one can modify the macro |\thepage| appropriately
and reset the counter |page| at the start of each child file.

%%%%%%%%%%%%%%%%%%%%%%%%%%%%%%%%%%%%%%%%%%%%%%%%%%%%%%%%%%%%%%%%%%%%%%%%%%%%%%%%
\subsection{Conditional Processing}
\label{sec:conditional}

The package provides a mechanism to compile different versions
of a document. To customise the versions further some conditional processing
can come in handy to distinguish which version is being compiled.
The package provides two macros to describe the compilation context:

%%%%%%%%%%%%%%%%%%%%%%%%%%%%%%%%%%%%%%%%
\DescribeMacro{\ifchilddoc}
The conditional |\ifchilddoc| distinguishes between the compilation of
child documents and the main document:
%
\begin{center}
|\ifchilddoc |\textit{child-code}| |[|\||else |\textit{main-code}]| \||fi|
\end{center}

%%%%%%%%%%%%%%%%%%%%%%%%%%%%%%%%%%%%%%%%
\DescribeMacro{\childdocname}
\DescribeMacro{\childdocjob}
The macro |\childdocname| contains the filename (without extension)
of the main or child file being processed.
Note that |\childdocjob| will always contain the name of the main file.

%%%%%%%%%%%%%%%%%%%%%%%%%%%%%%%%%%%%%%%%
\paragraph{Title Page.}

Conditional processing can be used to include a title or banner page
in the main document when proper precautions are taken.
Importantly, the code in the main file should ensure that the page counter
(as well as other status parameters which are stored in the |.aux| files)
takes the same value after the conditional processing.
Otherwise the page numbers may take divergent values
depending on which part is compiled.

For example, a title page could be declared by:
%
\begin{center}
\begin{tabular}{l}
|\ifchilddoc\||else|\\
|\addtocounter{page}{-1}|\\
\textit{code for title page}\\
|\newpage|\\
|\||fi|
\end{tabular}
\end{center}
%
A banner page for the child documents can be generated by:
%
\begin{center}
\begin{tabular}{l}
|\ifchilddoc|\\
|\addtocounter{page}{-1}|\\
\textit{code for banner page}\\
|\newpage|\\
|\||fi|
\end{tabular}
\end{center}
%
Here one could write a message such as:
\begin{center}
|This is the part \childdocname{} of \childdocjob{}.|
\end{center}

%%%%%%%%%%%%%%%%%%%%%%%%%%%%%%%%%%%%%%%%%%%%%%%%%%%%%%%%%%%%%%%%%%%%%%%%%%%%%%%%
\subsection{Flags}
\label{sec:flags}

The package makes it easy to generate different versions
of the main or child documents.
To this end compilation flags can be defined
and assigned different default values.
They will be particularly useful in conjunction
with the forwarding mechanism described in \secref{sec:forward}.

For example, it may be useful to have a flag |\version|
which can be set to |draft| or |final|.
The document source will contain some conditional code
depending on the value of |\version|.
Suppose further, the flag should default to |final| for the main file
and to |draft| for child files
which is a natural assignment for editing the document.
This is achieved by placing the following code
in the preamble of the main document
(below the |\childdocmain| directive):
%
\begin{center}
\begin{tabular}{l}
|\ifchilddoc|\\
|\providecommand{\version}{draft}|\\
|\||else|\\
|\providecommand{\version}{final}|\\
|\||fi|
\end{tabular}
\end{center}
%
The definition by |\providecommand| makes sure
that previous definitions are not overwritten.
Further statements |\providecommand{\version}{...}|
can thus be added before the above code to override it.

For the main file, one might add a line
(between |\childdocmain| and the above block)
%
\begin{center}
|%\ifchilddoc\||else\providecommand{\version}{draft}\||fi|
\end{center}
%
which can be uncommented to produce a draft version.
Likewise one can add a line to the very top of a child file
(above the |\childdocof{|\textit{main}|}| directive)
%
\begin{center}
|%\providecommand{\version}{final}|
\end{center}
%
which can be uncommented to produce the final version of this child document.

%%%%%%%%%%%%%%%%%%%%%%%%%%%%%%%%%%%%%%%%%%%%%%%%%%%%%%%%%%%%%%%%%%%%%%%%%%%%%%%%
\subsection{Forwarding}
\label{sec:forward}

Different versions of the main or child documents
using compilation flags as described in \secref{sec:flags}
can be (permanently) stored in different files
for convenient compilation, viewing and distribution.
To this end, the package defines a command
to pass on compilation to a different file:

%%%%%%%%%%%%%%%%%%%%%%%%%%%%%%%%%%%%%%%%
\DescribeMacro{\childdocforward}
The command |\childdocforward| redirects processing to
another source file:
%
\begin{center}
\begin{tabular}{l}
|\input{childdoc.def}|\\
|\childdocforward[|\textit{main}|]{|\textit{dest}|}|\\
\end{tabular}
\end{center}
%
The argument \textit{dest} is the destination file
(without extension).
It should be the main file or one of the child files.
Note that further \textsf{childdoc} directives
such as |\childdocof| and |\childdocforward|
in the indicated file will be processed in this form.
The optional argument \textit{main}
passes on directly to the main file \textit{main}
while pretending to compile the child \textit{dest}.
This form behaves as if \textit{dest}
issues |\childdocof{|\textit{main}|}| right away,
and no further \textsf{childdoc} directives will be processed.

%%%%%%%%%%%%%%%%%%%%%%%%%%%%%%%%%%%%%%%%
\DescribeMacro{\...prefix}
In the alternative form |\childdocforwardprefix|,
%
\begin{center}
\begin{tabular}{l}
|\input{childdoc.def}|\\
|\childdocforwardprefix[|\textit{main}|]{|\textit{prefix}|}{|\textit{dest}|}|
\end{tabular}
\end{center}
%
the destination file is determined by a pattern
depending on the current file:
To make this work, the current file must be called
`{\textit{prefix}\hspace{0.2em}\textit{suffix}}'
with \textit{prefix} matching precisely the argument.
Processing is then passed on to the file
`{\textit{dest}\hspace{0.2em}\textit{suffix}}'.
Surely, the same effect is achieved by
directly specifying the
argument `{\textit{dest}\hspace{0.2em}\textit{suffix}}'
in the first form.
However, that requires to set up a different file
for each child. With the alternative form of the command
all these files can have exactly the same content
which simplifies setting them up and maintaining them.

For example, the following file |draft.tex|
with a compilation flag |\version| as described in \secref{sec:flags}
compiles the main document as a draft:
%
\begin{center}
\begin{tabular}{l}
|\def\version{draft}|\\
|\input{childdoc.def}|\\
|\childdocforward{|\textit{main}|}|
\end{tabular}
\end{center}
%
Likewise, the following files |final|\textit{nn}|.tex|
compile the final version of the child document
|child|\textit{nn}|.tex|:
%
\begin{center}
\begin{tabular}{l}
|\def\version{final}|\\
|\input{childdoc.def}|\\
|\childdocforwardprefix{final}{child}|
\end{tabular}
\end{center}
%

Note that when several versions of a main file and/or of each child file
are to be generated, it may be convenient to set up a |Makefile| or
shell script to automatise the process.

%%%%%%%%%%%%%%%%%%%%%%%%%%%%%%%%%%%%%%%%%%%%%%%%%%%%%%%%%%%%%%%%%%%%%%%%%%%%%%%%
\subsection{Command Line Processing}
\label{sec:commandline}

The effect of redirection files can also be achieved by invoking
the \LaTeX{} compiler with a more elaborate command line.
Most conveniently this should be done as part
of a shell script or a |Makefile|.

When using \textsf{childdoc} in the main file, the following
command lines effectively perform a redirection
(note that depending on the shell being used,
backslashes may have to be doubled: `|\|' $\to$ `|\\|'):
%
\begin{center}
|... -jobname "|\textit{target}|" |\\|"|[\textit{flags}]%
|\input{childdoc.def}\childdocforward[|\textit{main}|]{|\textit{dest}|}"|
\end{center}
%
Here \textit{target} is the name of the output file,
\textit{main} is the name of the main file
and \textit{dest} is the name of the main or child file to be processed
(all filenames without extensions).
The optional argument \textit{main} can be omitted
if \textit{main} matches \textit{dest}.
Optionally, compilation \textit{flags} can be defined via |\def| commands.
This command line makes the \TeX{} engine believe
it is compiling the file \textit{target}
whose content is specified as the latter parameter.
The provided code then forwards the processing to
\textit{main} or \textit{dest} as described in \secref{sec:forward}.

%%%%%%%%%%%%%%%%%%%%%%%%%%%%%%%%%%%%%%%%%%%%%%%%%%%%%%%%%%%%%%%%%%%%%%%%%%%%%%%%
\subsection{Include by Input}
\label{sec:input}

Including child documents by |\include| has some restrictions by design.
Most notably, the content of a child document always occupies
its own set of pages; pages cannot be shared between child documents.
Usually, this behaviour makes perfect sense
because each child document contain an essential part of the document.
However, in some situations it may be desirable to compose
a document from a collection of parts
without having mandatory page breaks between then.
For this case, the package
provides a mechanism to include parts
by |\input| which can also be processed individually.
However, by construction this mechanism
requires manual handling of the content to be output.

%%%%%%%%%%%%%%%%%%%%%%%%%%%%%%%%%%%%%%%%
\DescribeMacro{\ifchilddocmanual}
The main file should be prepared as usual, see \secref{sec:include}.
However, the document body must make a distinction
between processing of an individual part and of the main document, e.g.:
%
\begin{center}
\begin{tabular}{l}
|\ifchilddocmanual|\\
|\input{\childdocname}|\\
|\||else|\\
\textit{document body with }|\input{|\textit{part}|}|\\
|\||fi|
\end{tabular}
\end{center}
%
The conditional |\ifchilddocmanual| is true whenever
a part to be included by |\input| is being compiled,
and the name of the part is stored in |\childdocname|.

%%%%%%%%%%%%%%%%%%%%%%%%%%%%%%%%%%%%%%%%
\DescribeMacro{\childdocby}
Each part to be included by |\input| should start with:
%
\begin{center}
\begin{tabular}{l}
|\input{childdoc.def}|\\
|\childdocby{|\textit{main}|}|\\
\end{tabular}
\end{center}
%
The directive |\childdocby| is similar to |\childdocof|
described in \secref{sec:include},
but the subsequent selection of content must be done manually.
To that end, both |\ifchilddoc| and |\ifchilddocmanual|
will be true upon processing of a part,
and the name of the part is stored in |\childdocname|.
Note that |\jobname| will be set to the filename of the current part
so that each part receives an individual |.aux| file
that does not interfere with the |.aux| file(s) of the main document.
This behaviour can be altered by the alternative form
|\childdocby[*]{|\textit{main}|}| (with a non-empty optional argument)
which uses the |.aux| file of the main document
by setting |\jobname| to \textit{main}.

%%%%%%%%%%%%%%%%%%%%%%%%%%%%%%%%%%%%%%%%%%%%%%%%%%%%%%%%%%%%%%%%%%%%%%%%%%%%%%%%
\subsection{Driver Development}
\label{sec:driver}

The \textsf{childdoc} mechanism can also be use for the development
of definition files such as \LaTeX{} styles or classes.
This case differs from the above setup with multiple parts
included by |\include| in that no |\includeonly| should be invoked.
This can be achieved by starting the include file
(before |\ProvidesPackage|) with:
%
\begin{center}
\begin{tabular}{l}
|\input{childdoc.def}|\\
|\childdocforward{|\textit{main}|}|\\
\end{tabular}
\end{center}
%
or alternatively with:
%
\begin{center}
\begin{tabular}{l}
|\input{childdoc.def}|\\
|\childdocby{|\textit{main}|}|\\
\end{tabular}
\end{center}
%
Both forms have slightly different effects as described above.
The main file is prepared as usual, see \secref{sec:include}.

%%%%%%%%%%%%%%%%%%%%%%%%%%%%%%%%%%%%%%%%%%%%%%%%%%%%%%%%%%%%%%%%%%%%%%%%%%%%%%%%
\subsection{Legacy Detection}
\label{sec:detection}

The directive |\childdocmain| in the main file can detect
whether the complete document or merely a child is to be compiled
even without using the directive |\childdocof|.
This method is deprecated because it is less robust
and there is no compelling reason to use it;
it is merely provided for backward compatibility
and it may be removed in future versions.

If the detection mechanism is to be used,
it is mandatory to correctly specify
the filename of the main file as the argument of |\childdocmain|:
%
\begin{center}
\begin{tabular}{l}
|\input{childdoc.def}|\\
|\childdocmain{|\textit{main}|}|\\
\end{tabular}
\end{center}
%
If |\jobname| does not match the argument \textit{main} of |\childdocmain|,
it is assumed that |\jobname| points to the child file to be compiled.
When using |\childdocmain| with the main file specified as argument,
it suffices to start a child file
with just |\input{|\textit{main}|}|
without loading of the package and using |\childdocof|.
If instead all processing is done
with the appropriate \textsf{childdoc} directives,
the argument of \textit{main} of |\childdocmain| can be empty.

An alternative version of the command line processing described
in \secref{sec:commandline} using the detection mechanism reads:
%
\begin{center}
|... -jobname "|\textit{target}|" "|[\textit{flags}]%
[|\def\jobname{|\textit{dest}|}|]|\input{|\textit{main}|}"|
\end{center}

%%%%%%%%%%%%%%%%%%%%%%%%%%%%%%%%%%%%%%%%%%%%%%%%%%%%%%%%%%%%%%%%%%%%%%%%%%%%%%%%
\subsection{Manual Code}
\label{sec:manual}

In case one cannot be certain whether the definitions file |childdoc.def|
is installed on the target \TeX{} distribution
and one prefers not to ship it,
it is conceivable to paste a few relevant commands into the sources.

To that end, drop all statements |\input{childdoc.def}|
and perform the replacements as outlined below.
Instead of |\childdocmain{|\textit{main}|}| add the following code
to the top of the main file:
%
\begin{center}
\begin{tabular}{l}
|\||ifdefined\childdocname\endinput\||fi\newif\ifchilddoc|\\
|\edef\childdocname{\scantokens\expandafter{\jobname\noexpand}}|\\
|\def\childdocmain{|\textit{main}|}\||ifx\childdocmain\childdocname\||else|\\
|\childdoctrue\includeonly{\childdocname}\let\jobname\childdocmain\||fi|\\
\end{tabular}
\end{center}
%
Instead of |\childdocof{|\textit{main}|}| just include the main file
at the top of each child file:
%
\begin{center}
|\input{|\textit{main}|}|
\end{center}
%
A simple redirection |\childdocforward{|\textit{dest}|}| is achieved by:
%
\begin{center}
|\def\jobname{|\textit{dest}|}\input{\jobname}|
\end{center}
%
The redirection with prefix
|\childdocforwardprefix[|\textit{prefix}|]{|\textit{dest}|}|
is accomplished by:
%
\begin{center}
\begin{tabular}{l}
|{\edef\jobname{\scantokens\expandafter{\jobname\noexpand}}|\\
|\def\redirectjob |\textit{prefix}|#1~~~{\gdef\jobname{|\textit{dest}|#1}}|\\
|\expandafter\redirectjob\jobname~~~}\input{\jobname}|
\end{tabular}
\end{center}

In an alternative approach,
child documents can be compiled by a specific command line
without additional code or specific definitions:
%
\begin{center}
|... -jobname "|\textit{target}|" "|[\textit{flags}]%
|\includeonly{|\textit{dest}|}\input{|\textit{main}|}"|
\end{center}
%

%%%%%%%%%%%%%%%%%%%%%%%%%%%%%%%%%%%%%%%%%%%%%%%%%%%%%%%%%%%%%%%%%%%%%%%%%%%%%%%%
%%%%%%%%%%%%%%%%%%%%%%%%%%%%%%%%%%%%%%%%%%%%%%%%%%%%%%%%%%%%%%%%%%%%%%%%%%%%%%%%
\section{Information}

%%%%%%%%%%%%%%%%%%%%%%%%%%%%%%%%%%%%%%%%%%%%%%%%%%%%%%%%%%%%%%%%%%%%%%%%%%%%%%%%
\subsection{Copyright}

Copyright \copyright{} 2017--2018 Niklas Beisert

This work may be distributed and/or modified under the
conditions of the \LaTeX{} Project Public License, either version 1.3
of this license or (at your option) any later version.
The latest version of this license is in
  \url{http://www.latex-project.org/lppl.txt}
and version 1.3 or later is part of all distributions of \LaTeX{}
version 2005/12/01 or later.

This work has the LPPL maintenance status `maintained'.

The Current Maintainer of this work is Niklas Beisert.

This work consists of the files |README.txt|, |childdoc.ins| and |childdoc.dtx|
as well as the derived files |childdoc.def|, |cdocsamp.tex|
with |cdocsch1.tex|, |cdocsch2.tex|, |cdocspt3.tex|, |cdocspt4.tex|,
|cdocsdrf.tex|, |cdocsfn1.tex|, |cdocsfn2.tex|
as well as |childdoc.pdf|.

%%%%%%%%%%%%%%%%%%%%%%%%%%%%%%%%%%%%%%%%%%%%%%%%%%%%%%%%%%%%%%%%%%%%%%%%%%%%%%%%
\subsection{Files and Installation}

The package consists of the files:
%
\begin{center}
\begin{tabular}{ll}
    |README.txt|   & readme file \\
    |childdoc.ins| & installation file \\
    |childdoc.dtx| & source file \\
    |childdoc.def| & definition file \\
    |cdocsamp.tex| & sample main file \\
    |cdocsch1.tex| & sample include file \\
    |cdocsch2.tex| & sample include file \\
    |cdocspt3.tex| & sample part file \\
    |cdocspt4.tex| & sample part file \\
    |cdocsdrf.tex| & sample redirection file \\
    |cdocsfn1.tex| & sample redirection file \\
    |cdocsfn2.tex| & sample redirection file \\
    |childdoc.pdf| & manual
\end{tabular}
\end{center}
%
The distribution consists of the files
|README.txt|, |childdoc.ins| and |childdoc.dtx|.
%
\begin{itemize}
\item
Run (pdf)\LaTeX{} on |childdoc.dtx|
to compile the manual |childdoc.pdf| (this file).
\item
Run \LaTeX{} on |childdoc.ins| to create the definitions file |childdoc.def|
and the sample |cdocsamp.tex| with include files
|cdocsch1.tex|, |cdocsch2.tex|, |cdocspt3.tex|, |cdocspt4.tex|,
|cdocsdrf.tex|, |cdocsfn1.tex|, |cdocsfn2.tex|.
Then copy the file |childdoc.def| to an appropriate directory of your \LaTeX{}
distribution, e.g.\ \textit{texmf-root}|/tex/latex/childdoc|.
\end{itemize}

%%%%%%%%%%%%%%%%%%%%%%%%%%%%%%%%%%%%%%%%%%%%%%%%%%%%%%%%%%%%%%%%%%%%%%%%%%%%%%%%
\subsection{Related CTAN Packages}

There are several other packages which offer a similar functionality:
%
\begin{itemize}
\item
The packages
\href{http://ctan.org/pkg/docmute}{\textsf{docmute}},
\href{http://ctan.org/pkg/includex}{\textsf{includex}} and
\href{http://ctan.org/pkg/standalone}{\textsf{standalone}}
provide commands to include only the document body of
a child file thus allowing both files to be compiled individually.
\item
The packages \href{http://ctan.org/pkg/subdocs}{\textsf{subdocs}}
and \href{http://ctan.org/pkg/subfiles}{\textsf{subfiles}}
provide structures in which the main and child documents can be
encapsulated and allowing them to be compiled individually.
The inclusion mechanism is different from the conventional |\include|.
\item
The package \href{http://ctan.org/pkg/combine}{\textsf{combine}}
is an elaborate solution to combine several documents into one.
\end{itemize}
%
See also the CTAN topic \href{http://ctan.org/topic/subdocs}{\textsf{subdocs}}
for further related packages.
The present package differs from the above solutions in that
a document structure constructed with the conventional |\include| mechanism
just needs two extra commands at the top of every file
such that all constituent files can be compiled individually.

%%%%%%%%%%%%%%%%%%%%%%%%%%%%%%%%%%%%%%%%%%%%%%%%%%%%%%%%%%%%%%%%%%%%%%%%%%%%%%%%
%\subsection{Feature Suggestions}
%
%The following is a list of features which may be useful for future
%versions of this package:
%%
%\begin{itemize}
%\item
%\ldots
%\end{itemize}

%%%%%%%%%%%%%%%%%%%%%%%%%%%%%%%%%%%%%%%%%%%%%%%%%%%%%%%%%%%%%%%%%%%%%%%%%%%%%%%%
\subsection{Revision History}

%%%%%%%%%%%%%%%%%%%%%%%%%%%%%%%%%%%%%%%%
\paragraph{v2.0:} 2018/12/30

\begin{itemize}
\item
immediate forward processing
\item
added |\childdocby| mechanism
\item
manual restructured
\end{itemize}

%%%%%%%%%%%%%%%%%%%%%%%%%%%%%%%%%%%%%%%%
\paragraph{v1.6:} 2018/01/17

\begin{itemize}
\item
application for development of include files
\item
corrections to manual
\end{itemize}

%%%%%%%%%%%%%%%%%%%%%%%%%%%%%%%%%%%%%%%%
\paragraph{v1.5:} 2017/05/21

\begin{itemize}
\item
more complete structuring introduced
\item
|\childdocof| introduced
\item
|\childdoc| renamed to |\childdocmain|
\item
|\childredirect| renamed to |\childdocforward| and |\childdocforwardprefix|
and functionality expanded
\end{itemize}

%%%%%%%%%%%%%%%%%%%%%%%%%%%%%%%%%%%%%%%%
\paragraph{v1.0:} 2017/04/27

\begin{itemize}
\item
manual and install package
\item
first version published on CTAN
\end{itemize}

%%%%%%%%%%%%%%%%%%%%%%%%%%%%%%%%%%%%%%%%
\paragraph{v0.6:} 2017/04/26

\begin{itemize}
\item
redirection mechanism added
\end{itemize}

%%%%%%%%%%%%%%%%%%%%%%%%%%%%%%%%%%%%%%%%
\paragraph{v0.5:} 2017/04/26

\begin{itemize}
\item
functionality in definition file
\end{itemize}


%%%%%%%%%%%%%%%%%%%%%%%%%%%%%%%%%%%%%%%%%%%%%%%%%%%%%%%%%%%%%%%%%%%%%%%%%%%%%%%%
%%%%%%%%%%%%%%%%%%%%%%%%%%%%%%%%%%%%%%%%%%%%%%%%%%%%%%%%%%%%%%%%%%%%%%%%%%%%%%%%
%%%%%%%%%%%%%%%%%%%%%%%%%%%%%%%%%%%%%%%%%%%%%%%%%%%%%%%%%%%%%%%%%%%%%%%%%%%%%%%%
\appendix

\settowidth\MacroIndent{\rmfamily\scriptsize 000\ }

 \DocInput{childdoc.dtx}

\end{document}
%</driver>
% \fi
%
% %%%%%%%%%%%%%%%%%%%%%%%%%%%%%%%%%%%%%%%%%%%%%%%%%%%%%%%%%%%%%%%%%%%%%%%%%%%%%%
% %%%%%%%%%%%%%%%%%%%%%%%%%%%%%%%%%%%%%%%%%%%%%%%%%%%%%%%%%%%%%%%%%%%%%%%%%%%%%%
% \section{Sample}
%\iffalse
%<*samplemain>
%\fi
%
% The following presents a sample document
% with two chapters, two parts, a title page,
% a compile flag as well as three forwarding files to set the flag.
% It consists of eight |.tex| files:
% \begin{center}
% \begin{tabular}{ll}
% |cdocsamp.tex|&main file\\
% |cdocsch1.tex|&include file for chapter 1\\
% |cdocsch2.tex|&include file for chapter 2\\
% |cdocspt3.tex|&include file for part 3\\
% |cdocspt4.tex|&include file for part 4\\
% |cdocsdrf.tex|&forwarding file for main file in draft mode\\
% |cdocsfi1.tex|&forwarding file for final version of chapter 1\\
% |cdocsfi2.tex|&forwarding file for final version of chapter 2\\
% \end{tabular}
% \end{center}
% Each of the eight files can be compiled directly by the \LaTeX{} compiler.
%
% %%%%%%%%%%%%%%%%%%%%%%%%%%%%%%%%%%%%%%
% \paragraph{Main File.}
%
% The main file is called |cdocsamp.tex|.
%
% Load the \textsf{childdoc} definitions and
% declare the filename for the main document:
%    \begin{macrocode}
\input{childdoc.def}
\childdocmain{}
%    \end{macrocode}

% Optional override for |\version| flag:
%    \begin{macrocode}
%%\ifchilddoc\else\providecommand{\version}{draft}\fi
%    \end{macrocode}

% Define the default values for the |\version| flag
% (|final| for the main file and |draft| for childs):
%    \begin{macrocode}
\ifchilddoc
\providecommand{\version}{draft}
\else
\providecommand{\version}{final}
\fi
%    \end{macrocode}

% Load the standard document class:
%    \begin{macrocode}
\documentclass[12pt]{article}
%    \end{macrocode}

% Start the document body:
%    \begin{macrocode}
\begin{document}
%    \end{macrocode}

% Declare a title page.
% Print title, part of document being processed and version flag:
%    \begin{macrocode}
\addtocounter{page}{-1}
\begin{center}
{\LARGE\bfseries{}childdoc example\par}
\vspace{1cm}
\ifchilddoc
\ifchilddocmanual part\else chapter\fi:
`\childdocname' of `\childdocjob'\par
\else
main document: `\childdocjob'\par
\fi
version: \version\par
\end{center}
\newpage
%    \end{macrocode}

% Manually include selected file,
% otherwise process as usual:
%    \begin{macrocode}
\ifchilddocmanual
\section*{part `\childdocname'}
\input{\childdocname}
\else
%    \end{macrocode}

% Include the two chapters:
%    \begin{macrocode}
\include{cdocsch1}
\include{cdocsch2}
%    \end{macrocode}

% Include the two parts unless only chapters should be displayed:
%    \begin{macrocode}
\ifchilddoc\else
\section{part three}
\input{cdocspt3}
\section{part four}
\input{cdocspt4}
\fi
%    \end{macrocode}

% Process as usual until here:
%    \begin{macrocode}
\fi
%    \end{macrocode}

% End of document body:
%    \begin{macrocode}
\end{document}
%    \end{macrocode}
%\iffalse
%</samplemain>
%\fi
%
% %%%%%%%%%%%%%%%%%%%%%%%%%%%%%%%%%%%%%%
% \paragraph{Chapter Include Files.}
%
% The include files are called |cdocsch1.tex| and |cdocsch2.tex|.
%
%\iffalse
%<*samplechap1|samplechap2>
%\fi

% Optional override for |\version| flag:
%    \begin{macrocode}
%%\providecommand{\version}{final}
%    \end{macrocode}

% Include the main document:
%    \begin{macrocode}
\input{childdoc.def}
\childdocof{cdocsamp}
%    \end{macrocode}

%\iffalse
%</samplechap1|samplechap2>
%\fi
%
%\iffalse
%<*samplechap1>
%\fi
% Some text for chapter 1:
%    \begin{macrocode}
\section{one}
some text in chapter one
%    \end{macrocode}

%\iffalse
%</samplechap1>
%\fi
% Some text for chapter 2:
%\iffalse
%<*samplechap2>
%\fi
%    \begin{macrocode}
\section{two}
more text in chapter two
%    \end{macrocode}

%\iffalse
%</samplechap2>
%\fi
%
% %%%%%%%%%%%%%%%%%%%%%%%%%%%%%%%%%%%%%%
% \paragraph{Part Include Files.}
%
% The include files are called |cdocspt3.tex| and |cdocspt4.tex|.
%
%\iffalse
%<*samplepart3|samplepart4>
%\fi

% Optional override for |\version| flag:
%    \begin{macrocode}
%%\providecommand{\version}{final}
%    \end{macrocode}

% Include the main document:
%    \begin{macrocode}
\input{childdoc.def}
\childdocby{cdocsamp}
%    \end{macrocode}

%\iffalse
%</samplepart3|samplepart4>
%\fi
%
%\iffalse
%<*samplepart3>
%\fi
% Some text for part 3:
%    \begin{macrocode}
some text in part three
%    \end{macrocode}

%\iffalse
%</samplepart3>
%\fi
% Some text for part 4:
%\iffalse
%<*samplepart4>
%\fi
%    \begin{macrocode}
more text in part four
%    \end{macrocode}

%\iffalse
%</samplepart4>
%\fi
%
% %%%%%%%%%%%%%%%%%%%%%%%%%%%%%%%%%%%%%%
% \paragraph{Forwarding for a Complete Draft.}
%
% The following forwarding file |cdocsdrf.tex|
% compiles the main document in draft mode:
%\iffalse
%<*sampledraft>
%\fi
%    \begin{macrocode}
\def\version{draft}
\input{childdoc.def}
\childdocforward{cdocsamp}
%    \end{macrocode}

%\iffalse
%</sampledraft>
%\fi
%
% %%%%%%%%%%%%%%%%%%%%%%%%%%%%%%%%%%%%%%
% \paragraph{Forwarding for Final Version of the Chapters.}
%
% The following forwarding files |cdocsfn1.tex| and |cdocsfn2.tex|
% (with identical content)
% compile the final versions of the child documents
% |cdocsch1.tex| and |cdocsch2.tex|, respectively:
%\iffalse
%<*samplefinal>
%\fi
%    \begin{macrocode}
\def\version{final}
\input{childdoc.def}
\childdocforwardprefix[cdocsamp]{cdocsfn}{cdocsch}
%    \end{macrocode}

%\iffalse
%</samplefinal>
%\fi
%
% %%%%%%%%%%%%%%%%%%%%%%%%%%%%%%%%%%%%%%
% \paragraph{Command Line Processing.}
%
% The following three command lines generate the output files
% |cdocscld|, |cdocscl1| and |cdocscl2|
% which should be identical to
% |cdocsdrf|, |cdocsch1| and |cdocsfn2|, respectively:
% \begin{center}
% \begin{tabular}{l}
% |latex -jobname cdocscld \|\\
% |  "\def\version{draft}\input{childdoc.def}\childdocforward{cdocsamp}"|\\
% |latex -jobname cdocscl1 \|\\
% |  "\input{childdoc.def}\childdocforward[cdocsamp]{cdocsch1}"|\\
% |latex -jobname cdocscl2 \|\\
% |  "\def\version{final}\input{childdoc.def}\childdocforward{cdocsch2}"|
% \end{tabular}
% \end{center}
% Note that the trailing backslash on each first line
% merely continues the input to the second line
% (for convenient cut ant paste).
% Furthermore, the command |latex| can be replaced by any
% of its alternative versions such as |pdflatex|.
%
% %%%%%%%%%%%%%%%%%%%%%%%%%%%%%%%%%%%%%%%%%%%%%%%%%%%%%%%%%%%%%%%%%%%%%%%%%%%%%%
% %%%%%%%%%%%%%%%%%%%%%%%%%%%%%%%%%%%%%%%%%%%%%%%%%%%%%%%%%%%%%%%%%%%%%%%%%%%%%%
% \section{Implementation}
%\iffalse
%<*package>
%\fi
%
% This section describes the definitions file |childdoc.def|.

% The definitions cannot be loaded using |\usepackage| or |\RequirePackage|
% which has a mechanism to prevent loading a style file more than once.
% When loading the definitions by means of |\input|
% multiple instances have to be prevented manually:
%\iffalse
%This code needs to be before the `\ProvidesFile' directive
%which is defined at the beginning of this file.
%Therefore it is also placed there and commented out here.
%</package>
%<*discard>
%\fi
%    \begin{macrocode}
\ifdefined\childdocmain\endinput\fi
%    \end{macrocode}
%\iffalse
%</discard>
%<*package>
%\fi
%
% \macro{\ifchilddoc}
% \macro{\ifchilddocmanual}
% The conditional |\ifchilddoc| tells whether a
% child (true) or main (false) document is being compiled.
% The conditional |\ifchilddocmanual| tells whether
% the |\includeonly| mechanism is used (false) or
% the selection of child files must be performed manually (true).
% The definitions initialise to false:
%    \begin{macrocode}
\newif\ifchilddoc
\newif\ifchilddocmanual
%    \end{macrocode}

% \macro{\childdocname}
% \macro{\childdocjob}
% The macro |\childdocname| stores the name of the main document
% to be compiled. The macro |\childdocjob| stores the name of
% the document on which the \LaTeX{} compiler was originally invoked.
% The content of |\jobname| cannot be compared
% to filenames specified in the source due to different catcodes.
% The following code rescans |\jobname|, stores the result
% in |\childdocname| and saves a copy in |\childdocjob|:
%    \begin{macrocode}
\edef\childdocname{\scantokens\expandafter{\jobname\noexpand}}
\let\childdocjob\childdocname
%    \end{macrocode}

% \macro{\childdocdisable}
% The macro |\childdocdisable| prevents the main file
% from being processed more than once.
% At this stage, the main document command |\childdocmain|
% is assumed to be called once again where it should do nothing.
% Any subsequent call to it should prevent
% a secondary processing of the main document
% It overwrites the forwarding commands
% |\childdocof| and |\childdocforward|
% with empty macros to prevent further inclusions of the main document:
%    \begin{macrocode}
\newcommand{\childdocdisable}
{
  \renewcommand{\childdocmain}[1]{\renewcommand{\childdocmain}[1]{\endinput}}
  \renewcommand{\childdocof}[1]{}
  \renewcommand{\childdocby}[2][]{}
  \renewcommand{\childdocforward}[2][]{}
  \renewcommand{\childdocdisable}{}
}
%    \end{macrocode}

% \macro{\childdocmain}
% The macro |\childdocmain| is to be called at the top of the main file
% with nothing or the main filename (without extension) as argument.
% First, it breaks loops.
% If the argument is not empty and does not match |\childdocname|
% (which is set by the first inclusion of |childdoc.def|),
% |\ifchilddoc| is set to true, |\includeonly| is applied to the child file
% and |\jobname| is set to the main file
% (for proper handling of |.aux| files):
%    \begin{macrocode}
\newcommand{\childdocmain}[1]
{
  \childdocdisable\childdocmain{}
  \if?#1?\else
    \begingroup
      \def\childdoctmp{#1}
      \ifx\childdoctmp\childdocname
        \def\childdoctmp{}
      \else
        \def\childdoctmp
        {
          \childdoctrue
          \includeonly{\childdocname}
          \def\childdocjob{#1}
          \def\jobname{#1}
        }
      \fi
      \expandafter
    \endgroup
    \childdoctmp
  \fi
}
%    \end{macrocode}

% \macro{\childdocof}
% The command |\childdocof| redirects
% compilation to the main file |#1|.
%    \begin{macrocode}
\newcommand{\childdocof}[1]
{
  \childdocdisable
  \childdoctrue
  \includeonly{\childdocname}
  \def\jobname{#1}
  \def\childdocjob{#1}
  \input{#1}
}
%    \end{macrocode}

% \macro{\childdocby}
% The command |\childdocby| ....
%    \begin{macrocode}
\newcommand{\childdocby}[2][]
{
  \childdocdisable
  \childdoctrue
  \childdocmanualtrue
  \if?#1?\else
    \def\jobname{#2}
  \fi
  \def\childdocjob{#2}
  \input{#2}
  \endinput
}
%    \end{macrocode}

% \macro{\childdocforward}
% The command |\childdocforward| redirects
% compilation to the main file or
% (if the optional argument is given) a child file.
% Parameters are set as if the main file
% or a child file starting with |\childdocof| was compiled.
% Then compilation is handed over to the main file:
%    \begin{macrocode}
\newcommand{\childdocforward}[2][]
{
  \begingroup
    \if?#1?
      \def\childdoctmp
      {
        \def\childdocname{#2}
        \def\childdocjob{#2}
        \def\jobname{#2}
        \input{#2}
        \endinput
      }
    \else
      \def\childdoctmp
      {
        \childdocdisable
        \def\childdocname{#2}
        \childdoctrue
        \includeonly{#2}
        \def\childdocjob{#1}
        \def\jobname{#1}
        \input{#1}
        \endinput
      }
    \fi
    \expandafter
  \endgroup
  \childdoctmp
}
%    \end{macrocode}

% \macro{\childdocforwardprefix}
% The command |\childdocforwardprefix| redirects
% compilation to the main or a child file by means of a pattern.
% The prefix |#1| in the current filename is replaced by |#2|
% and the suffix of the current filename is kept
% (it is assumed that the filename does not contain the substring `|~~~|'
% which is used as a delimiter).
% Compilation is handed over to the new file by |\childdocforward|:
%    \begin{macrocode}
\newcommand{\childdocforwardprefix}[3][]
{
  \begingroup
    \def\childdocextract #2##1~~~{\def\childdoctmp{\childdocforward[#1]{#3##1}}}
    \expandafter\childdocextract\childdocname~~~
    \expandafter
  \endgroup
  \childdoctmp
}
%    \end{macrocode}

% \macro{\childdoc}
% The deprecated macro |\childdoc| is a legacy version of |\childdocmain|:
%    \begin{macrocode}
\newcommand{\childdoc}{\childdocmain}
%    \end{macrocode}

% \macro{\childdocredirect}
% The deprecated macro |\childdocredirect| is a legacy version
% of |\childdocforward| and |\childdocforwardprefix|:
%    \begin{macrocode}
\newcommand{\childdocredirect}[2][]
{
  \begingroup
    \if?#1?
      \def\childdoctmp{\childdocforward{#2}}
    \else
      \def\childdoctmp{\childdocforwardprefix{#1}{#2}}
    \fi
    \expandafter
  \endgroup
  \childdoctmp
}
%    \end{macrocode}

%\iffalse
%</package>
%\fi
%
\endinput
|\\
|\childdocof{|\textit{main}|}|\\
\end{tabular}
\end{center}
at the top of every child file \textit{child}
which is included by |\include{|\textit{child}|}|
from within the main file
(or at least for those files to be compiled individually).
The argument \textit{main} must be the filename of the main file.

There are a couple of
considerations in setting up the main and child documents:

%%%%%%%%%%%%%%%%%%%%%%%%%%%%%%%%%%%%%%%%
\paragraph{Restrictions.}

Please note the following restrictions:
\begin{itemize}
\item
|\childdocmain| must be called with one argument \textit{main}
to ensure compatibility with earlier version of the package.
It must either be empty (|\childdocmain{}|)
or precisely match the filename of the main file in which it is specified.
See \secref{sec:detection} for further information.
\item
The filename \textit{main} must be specified without the |.tex| extension.
\item
The filename \textit{main} is case sensitive
(even in case-insensitive file systems)
due to internal string comparison.
\item
The argument \textit{main} should be fully expanded, it cannot be a macro.
\item
Subdirectories and special characters should be avoided in filenames.
\item
The command |\childdocmain{|\textit{main}|}| must be followed by a whitespace.
It should not be followed immediately by another command
or by a comment mark `|%|'.
This is because the \TeX{} parser reads the token immediately following
the argument of |\childdocmain| and puts it
at the beginning of every child section;
however, a white\-space is ignored.
\end{itemize}

%%%%%%%%%%%%%%%%%%%%%%%%%%%%%%%%%%%%%%%%
\paragraph{Content of Main File.}

It is advisable to place all content in the child files included by |\include|.
Any output contained in the main file will appear in all child documents
unless suppressed manually;
it cannot be suppressed automatically by the |\includeonly| directive
and thus should normally be avoided.
A method to include some content in the main file
by means of conditional processing is described in \secref{sec:conditional}.

%%%%%%%%%%%%%%%%%%%%%%%%%%%%%%%%%%%%%%%%
\paragraph{Page Numbering.}

When only a part of the document is compiled,
the appropriate numbering of pages
(as well as other status parameters)
is determined from the |.aux| files.
The latter contain information from previous passes.
However this information needs to propagate through
all intermediate child documents.
Therefore the page numbering in child documents may well
be inconsistent until the complete document is compiled at least once.

A useful (if unconventional) way to always ensure a consistent
page numbering is to restart the numbering in each child document
and denote the pages by `\textit{child}|.|\textit{page}'
where \textit{child} represents the chapter/section number of the child file.
This can be achieved by the command
|\numberwithin{page}{|\textit{child}|}|
of the \textsf{amsmath} package
where \textit{child} can be |chapter| or |section|
depending on the chosen structuring.
Alternatively, one can modify the macro |\thepage| appropriately
and reset the counter |page| at the start of each child file.

%%%%%%%%%%%%%%%%%%%%%%%%%%%%%%%%%%%%%%%%%%%%%%%%%%%%%%%%%%%%%%%%%%%%%%%%%%%%%%%%
\subsection{Conditional Processing}
\label{sec:conditional}

The package provides a mechanism to compile different versions
of a document. To customise the versions further some conditional processing
can come in handy to distinguish which version is being compiled.
The package provides two macros to describe the compilation context:

%%%%%%%%%%%%%%%%%%%%%%%%%%%%%%%%%%%%%%%%
\DescribeMacro{\ifchilddoc}
The conditional |\ifchilddoc| distinguishes between the compilation of
child documents and the main document:
%
\begin{center}
|\ifchilddoc |\textit{child-code}| |[|\||else |\textit{main-code}]| \||fi|
\end{center}

%%%%%%%%%%%%%%%%%%%%%%%%%%%%%%%%%%%%%%%%
\DescribeMacro{\childdocname}
\DescribeMacro{\childdocjob}
The macro |\childdocname| contains the filename (without extension)
of the main or child file being processed.
Note that |\childdocjob| will always contain the name of the main file.

%%%%%%%%%%%%%%%%%%%%%%%%%%%%%%%%%%%%%%%%
\paragraph{Title Page.}

Conditional processing can be used to include a title or banner page
in the main document when proper precautions are taken.
Importantly, the code in the main file should ensure that the page counter
(as well as other status parameters which are stored in the |.aux| files)
takes the same value after the conditional processing.
Otherwise the page numbers may take divergent values
depending on which part is compiled.

For example, a title page could be declared by:
%
\begin{center}
\begin{tabular}{l}
|\ifchilddoc\||else|\\
|\addtocounter{page}{-1}|\\
\textit{code for title page}\\
|\newpage|\\
|\||fi|
\end{tabular}
\end{center}
%
A banner page for the child documents can be generated by:
%
\begin{center}
\begin{tabular}{l}
|\ifchilddoc|\\
|\addtocounter{page}{-1}|\\
\textit{code for banner page}\\
|\newpage|\\
|\||fi|
\end{tabular}
\end{center}
%
Here one could write a message such as:
\begin{center}
|This is the part \childdocname{} of \childdocjob{}.|
\end{center}

%%%%%%%%%%%%%%%%%%%%%%%%%%%%%%%%%%%%%%%%%%%%%%%%%%%%%%%%%%%%%%%%%%%%%%%%%%%%%%%%
\subsection{Flags}
\label{sec:flags}

The package makes it easy to generate different versions
of the main or child documents.
To this end compilation flags can be defined
and assigned different default values.
They will be particularly useful in conjunction
with the forwarding mechanism described in \secref{sec:forward}.

For example, it may be useful to have a flag |\version|
which can be set to |draft| or |final|.
The document source will contain some conditional code
depending on the value of |\version|.
Suppose further, the flag should default to |final| for the main file
and to |draft| for child files
which is a natural assignment for editing the document.
This is achieved by placing the following code
in the preamble of the main document
(below the |\childdocmain| directive):
%
\begin{center}
\begin{tabular}{l}
|\ifchilddoc|\\
|\providecommand{\version}{draft}|\\
|\||else|\\
|\providecommand{\version}{final}|\\
|\||fi|
\end{tabular}
\end{center}
%
The definition by |\providecommand| makes sure
that previous definitions are not overwritten.
Further statements |\providecommand{\version}{...}|
can thus be added before the above code to override it.

For the main file, one might add a line
(between |\childdocmain| and the above block)
%
\begin{center}
|%\ifchilddoc\||else\providecommand{\version}{draft}\||fi|
\end{center}
%
which can be uncommented to produce a draft version.
Likewise one can add a line to the very top of a child file
(above the |\childdocof{|\textit{main}|}| directive)
%
\begin{center}
|%\providecommand{\version}{final}|
\end{center}
%
which can be uncommented to produce the final version of this child document.

%%%%%%%%%%%%%%%%%%%%%%%%%%%%%%%%%%%%%%%%%%%%%%%%%%%%%%%%%%%%%%%%%%%%%%%%%%%%%%%%
\subsection{Forwarding}
\label{sec:forward}

Different versions of the main or child documents
using compilation flags as described in \secref{sec:flags}
can be (permanently) stored in different files
for convenient compilation, viewing and distribution.
To this end, the package defines a command
to pass on compilation to a different file:

%%%%%%%%%%%%%%%%%%%%%%%%%%%%%%%%%%%%%%%%
\DescribeMacro{\childdocforward}
The command |\childdocforward| redirects processing to
another source file:
%
\begin{center}
\begin{tabular}{l}
|% \iffalse
%
% childdoc.dtx Copyright (C) 2017-2018 Niklas Beisert
%
% This work may be distributed and/or modified under the
% conditions of the LaTeX Project Public License, either version 1.3
% of this license or (at your option) any later version.
% The latest version of this license is in
%   http://www.latex-project.org/lppl.txt
% and version 1.3 or later is part of all distributions of LaTeX
% version 2005/12/01 or later.
%
% This work has the LPPL maintenance status `maintained'.
%
% The Current Maintainer of this work is Niklas Beisert.
%
% This work consists of the files childdoc.dtx and childdoc.ins
% and the derived files childdoc.def and cdocsamp.tex with
% cdocsch1.tex, cdocsch2.tex, cdocsdrf.tex, cdocsfn1.tex, cdocsfn2.tex.
%
%<package>\ifdefined\childdocmain\endinput\fi
%<package>\ProvidesFile{childdoc.def}[2018/12/30 v2.0 child document driver]
%<samplemain>\ProvidesFile{cdocsamp.tex}[2018/12/30 v2.0 sample for childdoc]
%<*driver>
%\ProvidesFile{childdoc.drv}[2018/12/30 v2.0 childdoc reference manual file]
\PassOptionsToClass{10pt,a4paper}{article}
\documentclass{ltxdoc}

\usepackage[margin=35mm]{geometry}
\usepackage{hyperref}
\usepackage{hyperxmp}
\usepackage[usenames]{color}

\hypersetup{colorlinks=true}
\hypersetup{pdfstartview=FitH}
\hypersetup{pdfpagemode=UseNone}
\hypersetup{pdfsource={}}
\hypersetup{pdflang={en-UK}}
\hypersetup{pdfcopyright={Copyright 2017-2018 Niklas Beisert.
  This work may be distributed and/or modified under the
  conditions of the LaTeX Project Public License, either version 1.3
  of this license or (at your option) any later version.}}
\hypersetup{pdflicenseurl={http://www.latex-project.org/lppl.txt}}
\hypersetup{pdfcontactaddress={ETH Zurich, ITP, HIT K,
  Wolfgang-Pauli-Strasse 27}}
\hypersetup{pdfcontactpostcode={8093}}
\hypersetup{pdfcontactcity={Zurich}}
\hypersetup{pdfcontactcountry={Switzerland}}
\hypersetup{pdfcontactemail={nbeisert@itp.phys.ethz.ch}}
\hypersetup{pdfcontacturl={http://people.phys.ethz.ch/\xmptilde nbeisert/}}

\newcommand{\secref}[1]{\hyperref[#1]{section \ref*{#1}}}

\parskip1ex
\parindent0pt
\let\olditemize\itemize
\def\itemize{\olditemize\parskip0pt}

\begin{document}

\title{The \textsf{childdoc} Package}
\hypersetup{pdftitle={The childdoc Package}}
\author{Niklas Beisert\\[2ex]
  Institut f\"ur Theoretische Physik\\
  Eidgen\"ossische Technische Hochschule Z\"urich\\
  Wolfgang-Pauli-Strasse 27, 8093 Z\"urich, Switzerland\\[1ex]
  \href{mailto:nbeisert@itp.phys.ethz.ch}
  {\texttt{nbeisert@itp.phys.ethz.ch}}}
\hypersetup{pdfauthor={Niklas Beisert}}
\hypersetup{pdfsubject={Manual for the LaTeX2e Package childdoc}}
\date{30 December 2018, \textsf{v2.0}}
\maketitle

\begin{abstract}\noindent
\textsf{childdoc} is a \LaTeXe{} package
that enables the direct compilation
of document sections included by |\include|
to individual files.
\end{abstract}

\begingroup
\parskip0ex
\tableofcontents
\endgroup

%%%%%%%%%%%%%%%%%%%%%%%%%%%%%%%%%%%%%%%%%%%%%%%%%%%%%%%%%%%%%%%%%%%%%%%%%%%%%%%%
%%%%%%%%%%%%%%%%%%%%%%%%%%%%%%%%%%%%%%%%%%%%%%%%%%%%%%%%%%%%%%%%%%%%%%%%%%%%%%%%
\section{Introduction}

\LaTeX{} provides a mechanism to structure a large document (such as a book)
into a main file and several child files (containing the chapters)
using the |\include| command.
This mechanism is beneficial for documents
which span hundreds of pages in order to
make the source file(s) more manageable.
Moreover, compilation can be restricted to
selected child files by means of the |\includeonly| command.
The latter feature can be used to reduce the compilation time while editing
(this was significantly more useful in the earlier days of \LaTeX{})
or to generate a smaller document which is easier to navigate.
Another application of |\includeonly| is to generate
documents consisting of selected parts of the complete document.

However, there are a few drawbacks of the plain |\include| mechanism:
\begin{itemize}
\item
The child files cannot be compiled on their own,
they can only be compiled via the main file.
A naive editing environment
(such as a text editor with an option
to have the current file processed by \LaTeX)
may require one to switch to the main file before compiling;
attempting to compile the child file produces errors.
\item
The main file must be modified (each time)
to adjust the |\includeonly| command
to the present needs. This easily leaves the main file in a messy state.
\item
The generated document will always carry the filename
of the main document. This is inconvenient if
several child files are to be compiled and
to be kept for distribution.
\end{itemize}

The present package provides a simple interface
to make child files individually compilable by \LaTeX{}.
Compiling a child file then has the same effect as compiling
the main file with an |\includeonly| command
to select the appropriate child.
Moreover the generated document will carry the name of the child
rather than the main file.
This resolves all three above issues.

This feature is meant to make the editing of books,
thesis documents and lecture notes somewhat more convenient.
However, the package can also be used efficiently for
composing a series of documents (such as exercise sheets)
which are typically distributed individually.
It then assists the author in generating the individual documents
(potentially in different versions)
as well as a document containing the collected series.
Another application is in developing style files
or other kinds of included material
where compilation of the style file could redirect
to a sample or test file.

%%%%%%%%%%%%%%%%%%%%%%%%%%%%%%%%%%%%%%%%%%%%%%%%%%%%%%%%%%%%%%%%%%%%%%%%%%%%%%%%
%%%%%%%%%%%%%%%%%%%%%%%%%%%%%%%%%%%%%%%%%%%%%%%%%%%%%%%%%%%%%%%%%%%%%%%%%%%%%%%%
\section{Usage}

First of all, the package \textsf{childdoc} is \emph{not} a standard
\LaTeXe{} |.sty| style file! Therefore it needs to be invoked in
a non-standard way.

%%%%%%%%%%%%%%%%%%%%%%%%%%%%%%%%%%%%%%%%%%%%%%%%%%%%%%%%%%%%%%%%%%%%%%%%%%%%%%%%
\subsection{Included Files}
\label{sec:include}

%%%%%%%%%%%%%%%%%%%%%%%%%%%%%%%%%%%%%%%%
\DescribeMacro{\childdocmain}
To use the package, add the commands
\begin{center}
\begin{tabular}{l}
|\input{childdoc.def}|\\
|\childdocmain{}|\\
\end{tabular}
\end{center}
at the very top of the main \LaTeX{} file,
in particular \emph{before} the |\documentclass| statement!
The argument of |\childdocmain| should be left empty
(but it must be present).

%%%%%%%%%%%%%%%%%%%%%%%%%%%%%%%%%%%%%%%%
\DescribeMacro{\childdocof}
Furthermore, add the commands
\begin{center}
\begin{tabular}{l}
|\input{childdoc.def}|\\
|\childdocof{|\textit{main}|}|\\
\end{tabular}
\end{center}
at the top of every child file \textit{child}
which is included by |\include{|\textit{child}|}|
from within the main file
(or at least for those files to be compiled individually).
The argument \textit{main} must be the filename of the main file.

There are a couple of
considerations in setting up the main and child documents:

%%%%%%%%%%%%%%%%%%%%%%%%%%%%%%%%%%%%%%%%
\paragraph{Restrictions.}

Please note the following restrictions:
\begin{itemize}
\item
|\childdocmain| must be called with one argument \textit{main}
to ensure compatibility with earlier version of the package.
It must either be empty (|\childdocmain{}|)
or precisely match the filename of the main file in which it is specified.
See \secref{sec:detection} for further information.
\item
The filename \textit{main} must be specified without the |.tex| extension.
\item
The filename \textit{main} is case sensitive
(even in case-insensitive file systems)
due to internal string comparison.
\item
The argument \textit{main} should be fully expanded, it cannot be a macro.
\item
Subdirectories and special characters should be avoided in filenames.
\item
The command |\childdocmain{|\textit{main}|}| must be followed by a whitespace.
It should not be followed immediately by another command
or by a comment mark `|%|'.
This is because the \TeX{} parser reads the token immediately following
the argument of |\childdocmain| and puts it
at the beginning of every child section;
however, a white\-space is ignored.
\end{itemize}

%%%%%%%%%%%%%%%%%%%%%%%%%%%%%%%%%%%%%%%%
\paragraph{Content of Main File.}

It is advisable to place all content in the child files included by |\include|.
Any output contained in the main file will appear in all child documents
unless suppressed manually;
it cannot be suppressed automatically by the |\includeonly| directive
and thus should normally be avoided.
A method to include some content in the main file
by means of conditional processing is described in \secref{sec:conditional}.

%%%%%%%%%%%%%%%%%%%%%%%%%%%%%%%%%%%%%%%%
\paragraph{Page Numbering.}

When only a part of the document is compiled,
the appropriate numbering of pages
(as well as other status parameters)
is determined from the |.aux| files.
The latter contain information from previous passes.
However this information needs to propagate through
all intermediate child documents.
Therefore the page numbering in child documents may well
be inconsistent until the complete document is compiled at least once.

A useful (if unconventional) way to always ensure a consistent
page numbering is to restart the numbering in each child document
and denote the pages by `\textit{child}|.|\textit{page}'
where \textit{child} represents the chapter/section number of the child file.
This can be achieved by the command
|\numberwithin{page}{|\textit{child}|}|
of the \textsf{amsmath} package
where \textit{child} can be |chapter| or |section|
depending on the chosen structuring.
Alternatively, one can modify the macro |\thepage| appropriately
and reset the counter |page| at the start of each child file.

%%%%%%%%%%%%%%%%%%%%%%%%%%%%%%%%%%%%%%%%%%%%%%%%%%%%%%%%%%%%%%%%%%%%%%%%%%%%%%%%
\subsection{Conditional Processing}
\label{sec:conditional}

The package provides a mechanism to compile different versions
of a document. To customise the versions further some conditional processing
can come in handy to distinguish which version is being compiled.
The package provides two macros to describe the compilation context:

%%%%%%%%%%%%%%%%%%%%%%%%%%%%%%%%%%%%%%%%
\DescribeMacro{\ifchilddoc}
The conditional |\ifchilddoc| distinguishes between the compilation of
child documents and the main document:
%
\begin{center}
|\ifchilddoc |\textit{child-code}| |[|\||else |\textit{main-code}]| \||fi|
\end{center}

%%%%%%%%%%%%%%%%%%%%%%%%%%%%%%%%%%%%%%%%
\DescribeMacro{\childdocname}
\DescribeMacro{\childdocjob}
The macro |\childdocname| contains the filename (without extension)
of the main or child file being processed.
Note that |\childdocjob| will always contain the name of the main file.

%%%%%%%%%%%%%%%%%%%%%%%%%%%%%%%%%%%%%%%%
\paragraph{Title Page.}

Conditional processing can be used to include a title or banner page
in the main document when proper precautions are taken.
Importantly, the code in the main file should ensure that the page counter
(as well as other status parameters which are stored in the |.aux| files)
takes the same value after the conditional processing.
Otherwise the page numbers may take divergent values
depending on which part is compiled.

For example, a title page could be declared by:
%
\begin{center}
\begin{tabular}{l}
|\ifchilddoc\||else|\\
|\addtocounter{page}{-1}|\\
\textit{code for title page}\\
|\newpage|\\
|\||fi|
\end{tabular}
\end{center}
%
A banner page for the child documents can be generated by:
%
\begin{center}
\begin{tabular}{l}
|\ifchilddoc|\\
|\addtocounter{page}{-1}|\\
\textit{code for banner page}\\
|\newpage|\\
|\||fi|
\end{tabular}
\end{center}
%
Here one could write a message such as:
\begin{center}
|This is the part \childdocname{} of \childdocjob{}.|
\end{center}

%%%%%%%%%%%%%%%%%%%%%%%%%%%%%%%%%%%%%%%%%%%%%%%%%%%%%%%%%%%%%%%%%%%%%%%%%%%%%%%%
\subsection{Flags}
\label{sec:flags}

The package makes it easy to generate different versions
of the main or child documents.
To this end compilation flags can be defined
and assigned different default values.
They will be particularly useful in conjunction
with the forwarding mechanism described in \secref{sec:forward}.

For example, it may be useful to have a flag |\version|
which can be set to |draft| or |final|.
The document source will contain some conditional code
depending on the value of |\version|.
Suppose further, the flag should default to |final| for the main file
and to |draft| for child files
which is a natural assignment for editing the document.
This is achieved by placing the following code
in the preamble of the main document
(below the |\childdocmain| directive):
%
\begin{center}
\begin{tabular}{l}
|\ifchilddoc|\\
|\providecommand{\version}{draft}|\\
|\||else|\\
|\providecommand{\version}{final}|\\
|\||fi|
\end{tabular}
\end{center}
%
The definition by |\providecommand| makes sure
that previous definitions are not overwritten.
Further statements |\providecommand{\version}{...}|
can thus be added before the above code to override it.

For the main file, one might add a line
(between |\childdocmain| and the above block)
%
\begin{center}
|%\ifchilddoc\||else\providecommand{\version}{draft}\||fi|
\end{center}
%
which can be uncommented to produce a draft version.
Likewise one can add a line to the very top of a child file
(above the |\childdocof{|\textit{main}|}| directive)
%
\begin{center}
|%\providecommand{\version}{final}|
\end{center}
%
which can be uncommented to produce the final version of this child document.

%%%%%%%%%%%%%%%%%%%%%%%%%%%%%%%%%%%%%%%%%%%%%%%%%%%%%%%%%%%%%%%%%%%%%%%%%%%%%%%%
\subsection{Forwarding}
\label{sec:forward}

Different versions of the main or child documents
using compilation flags as described in \secref{sec:flags}
can be (permanently) stored in different files
for convenient compilation, viewing and distribution.
To this end, the package defines a command
to pass on compilation to a different file:

%%%%%%%%%%%%%%%%%%%%%%%%%%%%%%%%%%%%%%%%
\DescribeMacro{\childdocforward}
The command |\childdocforward| redirects processing to
another source file:
%
\begin{center}
\begin{tabular}{l}
|\input{childdoc.def}|\\
|\childdocforward[|\textit{main}|]{|\textit{dest}|}|\\
\end{tabular}
\end{center}
%
The argument \textit{dest} is the destination file
(without extension).
It should be the main file or one of the child files.
Note that further \textsf{childdoc} directives
such as |\childdocof| and |\childdocforward|
in the indicated file will be processed in this form.
The optional argument \textit{main}
passes on directly to the main file \textit{main}
while pretending to compile the child \textit{dest}.
This form behaves as if \textit{dest}
issues |\childdocof{|\textit{main}|}| right away,
and no further \textsf{childdoc} directives will be processed.

%%%%%%%%%%%%%%%%%%%%%%%%%%%%%%%%%%%%%%%%
\DescribeMacro{\...prefix}
In the alternative form |\childdocforwardprefix|,
%
\begin{center}
\begin{tabular}{l}
|\input{childdoc.def}|\\
|\childdocforwardprefix[|\textit{main}|]{|\textit{prefix}|}{|\textit{dest}|}|
\end{tabular}
\end{center}
%
the destination file is determined by a pattern
depending on the current file:
To make this work, the current file must be called
`{\textit{prefix}\hspace{0.2em}\textit{suffix}}'
with \textit{prefix} matching precisely the argument.
Processing is then passed on to the file
`{\textit{dest}\hspace{0.2em}\textit{suffix}}'.
Surely, the same effect is achieved by
directly specifying the
argument `{\textit{dest}\hspace{0.2em}\textit{suffix}}'
in the first form.
However, that requires to set up a different file
for each child. With the alternative form of the command
all these files can have exactly the same content
which simplifies setting them up and maintaining them.

For example, the following file |draft.tex|
with a compilation flag |\version| as described in \secref{sec:flags}
compiles the main document as a draft:
%
\begin{center}
\begin{tabular}{l}
|\def\version{draft}|\\
|\input{childdoc.def}|\\
|\childdocforward{|\textit{main}|}|
\end{tabular}
\end{center}
%
Likewise, the following files |final|\textit{nn}|.tex|
compile the final version of the child document
|child|\textit{nn}|.tex|:
%
\begin{center}
\begin{tabular}{l}
|\def\version{final}|\\
|\input{childdoc.def}|\\
|\childdocforwardprefix{final}{child}|
\end{tabular}
\end{center}
%

Note that when several versions of a main file and/or of each child file
are to be generated, it may be convenient to set up a |Makefile| or
shell script to automatise the process.

%%%%%%%%%%%%%%%%%%%%%%%%%%%%%%%%%%%%%%%%%%%%%%%%%%%%%%%%%%%%%%%%%%%%%%%%%%%%%%%%
\subsection{Command Line Processing}
\label{sec:commandline}

The effect of redirection files can also be achieved by invoking
the \LaTeX{} compiler with a more elaborate command line.
Most conveniently this should be done as part
of a shell script or a |Makefile|.

When using \textsf{childdoc} in the main file, the following
command lines effectively perform a redirection
(note that depending on the shell being used,
backslashes may have to be doubled: `|\|' $\to$ `|\\|'):
%
\begin{center}
|... -jobname "|\textit{target}|" |\\|"|[\textit{flags}]%
|\input{childdoc.def}\childdocforward[|\textit{main}|]{|\textit{dest}|}"|
\end{center}
%
Here \textit{target} is the name of the output file,
\textit{main} is the name of the main file
and \textit{dest} is the name of the main or child file to be processed
(all filenames without extensions).
The optional argument \textit{main} can be omitted
if \textit{main} matches \textit{dest}.
Optionally, compilation \textit{flags} can be defined via |\def| commands.
This command line makes the \TeX{} engine believe
it is compiling the file \textit{target}
whose content is specified as the latter parameter.
The provided code then forwards the processing to
\textit{main} or \textit{dest} as described in \secref{sec:forward}.

%%%%%%%%%%%%%%%%%%%%%%%%%%%%%%%%%%%%%%%%%%%%%%%%%%%%%%%%%%%%%%%%%%%%%%%%%%%%%%%%
\subsection{Include by Input}
\label{sec:input}

Including child documents by |\include| has some restrictions by design.
Most notably, the content of a child document always occupies
its own set of pages; pages cannot be shared between child documents.
Usually, this behaviour makes perfect sense
because each child document contain an essential part of the document.
However, in some situations it may be desirable to compose
a document from a collection of parts
without having mandatory page breaks between then.
For this case, the package
provides a mechanism to include parts
by |\input| which can also be processed individually.
However, by construction this mechanism
requires manual handling of the content to be output.

%%%%%%%%%%%%%%%%%%%%%%%%%%%%%%%%%%%%%%%%
\DescribeMacro{\ifchilddocmanual}
The main file should be prepared as usual, see \secref{sec:include}.
However, the document body must make a distinction
between processing of an individual part and of the main document, e.g.:
%
\begin{center}
\begin{tabular}{l}
|\ifchilddocmanual|\\
|\input{\childdocname}|\\
|\||else|\\
\textit{document body with }|\input{|\textit{part}|}|\\
|\||fi|
\end{tabular}
\end{center}
%
The conditional |\ifchilddocmanual| is true whenever
a part to be included by |\input| is being compiled,
and the name of the part is stored in |\childdocname|.

%%%%%%%%%%%%%%%%%%%%%%%%%%%%%%%%%%%%%%%%
\DescribeMacro{\childdocby}
Each part to be included by |\input| should start with:
%
\begin{center}
\begin{tabular}{l}
|\input{childdoc.def}|\\
|\childdocby{|\textit{main}|}|\\
\end{tabular}
\end{center}
%
The directive |\childdocby| is similar to |\childdocof|
described in \secref{sec:include},
but the subsequent selection of content must be done manually.
To that end, both |\ifchilddoc| and |\ifchilddocmanual|
will be true upon processing of a part,
and the name of the part is stored in |\childdocname|.
Note that |\jobname| will be set to the filename of the current part
so that each part receives an individual |.aux| file
that does not interfere with the |.aux| file(s) of the main document.
This behaviour can be altered by the alternative form
|\childdocby[*]{|\textit{main}|}| (with a non-empty optional argument)
which uses the |.aux| file of the main document
by setting |\jobname| to \textit{main}.

%%%%%%%%%%%%%%%%%%%%%%%%%%%%%%%%%%%%%%%%%%%%%%%%%%%%%%%%%%%%%%%%%%%%%%%%%%%%%%%%
\subsection{Driver Development}
\label{sec:driver}

The \textsf{childdoc} mechanism can also be use for the development
of definition files such as \LaTeX{} styles or classes.
This case differs from the above setup with multiple parts
included by |\include| in that no |\includeonly| should be invoked.
This can be achieved by starting the include file
(before |\ProvidesPackage|) with:
%
\begin{center}
\begin{tabular}{l}
|\input{childdoc.def}|\\
|\childdocforward{|\textit{main}|}|\\
\end{tabular}
\end{center}
%
or alternatively with:
%
\begin{center}
\begin{tabular}{l}
|\input{childdoc.def}|\\
|\childdocby{|\textit{main}|}|\\
\end{tabular}
\end{center}
%
Both forms have slightly different effects as described above.
The main file is prepared as usual, see \secref{sec:include}.

%%%%%%%%%%%%%%%%%%%%%%%%%%%%%%%%%%%%%%%%%%%%%%%%%%%%%%%%%%%%%%%%%%%%%%%%%%%%%%%%
\subsection{Legacy Detection}
\label{sec:detection}

The directive |\childdocmain| in the main file can detect
whether the complete document or merely a child is to be compiled
even without using the directive |\childdocof|.
This method is deprecated because it is less robust
and there is no compelling reason to use it;
it is merely provided for backward compatibility
and it may be removed in future versions.

If the detection mechanism is to be used,
it is mandatory to correctly specify
the filename of the main file as the argument of |\childdocmain|:
%
\begin{center}
\begin{tabular}{l}
|\input{childdoc.def}|\\
|\childdocmain{|\textit{main}|}|\\
\end{tabular}
\end{center}
%
If |\jobname| does not match the argument \textit{main} of |\childdocmain|,
it is assumed that |\jobname| points to the child file to be compiled.
When using |\childdocmain| with the main file specified as argument,
it suffices to start a child file
with just |\input{|\textit{main}|}|
without loading of the package and using |\childdocof|.
If instead all processing is done
with the appropriate \textsf{childdoc} directives,
the argument of \textit{main} of |\childdocmain| can be empty.

An alternative version of the command line processing described
in \secref{sec:commandline} using the detection mechanism reads:
%
\begin{center}
|... -jobname "|\textit{target}|" "|[\textit{flags}]%
[|\def\jobname{|\textit{dest}|}|]|\input{|\textit{main}|}"|
\end{center}

%%%%%%%%%%%%%%%%%%%%%%%%%%%%%%%%%%%%%%%%%%%%%%%%%%%%%%%%%%%%%%%%%%%%%%%%%%%%%%%%
\subsection{Manual Code}
\label{sec:manual}

In case one cannot be certain whether the definitions file |childdoc.def|
is installed on the target \TeX{} distribution
and one prefers not to ship it,
it is conceivable to paste a few relevant commands into the sources.

To that end, drop all statements |\input{childdoc.def}|
and perform the replacements as outlined below.
Instead of |\childdocmain{|\textit{main}|}| add the following code
to the top of the main file:
%
\begin{center}
\begin{tabular}{l}
|\||ifdefined\childdocname\endinput\||fi\newif\ifchilddoc|\\
|\edef\childdocname{\scantokens\expandafter{\jobname\noexpand}}|\\
|\def\childdocmain{|\textit{main}|}\||ifx\childdocmain\childdocname\||else|\\
|\childdoctrue\includeonly{\childdocname}\let\jobname\childdocmain\||fi|\\
\end{tabular}
\end{center}
%
Instead of |\childdocof{|\textit{main}|}| just include the main file
at the top of each child file:
%
\begin{center}
|\input{|\textit{main}|}|
\end{center}
%
A simple redirection |\childdocforward{|\textit{dest}|}| is achieved by:
%
\begin{center}
|\def\jobname{|\textit{dest}|}\input{\jobname}|
\end{center}
%
The redirection with prefix
|\childdocforwardprefix[|\textit{prefix}|]{|\textit{dest}|}|
is accomplished by:
%
\begin{center}
\begin{tabular}{l}
|{\edef\jobname{\scantokens\expandafter{\jobname\noexpand}}|\\
|\def\redirectjob |\textit{prefix}|#1~~~{\gdef\jobname{|\textit{dest}|#1}}|\\
|\expandafter\redirectjob\jobname~~~}\input{\jobname}|
\end{tabular}
\end{center}

In an alternative approach,
child documents can be compiled by a specific command line
without additional code or specific definitions:
%
\begin{center}
|... -jobname "|\textit{target}|" "|[\textit{flags}]%
|\includeonly{|\textit{dest}|}\input{|\textit{main}|}"|
\end{center}
%

%%%%%%%%%%%%%%%%%%%%%%%%%%%%%%%%%%%%%%%%%%%%%%%%%%%%%%%%%%%%%%%%%%%%%%%%%%%%%%%%
%%%%%%%%%%%%%%%%%%%%%%%%%%%%%%%%%%%%%%%%%%%%%%%%%%%%%%%%%%%%%%%%%%%%%%%%%%%%%%%%
\section{Information}

%%%%%%%%%%%%%%%%%%%%%%%%%%%%%%%%%%%%%%%%%%%%%%%%%%%%%%%%%%%%%%%%%%%%%%%%%%%%%%%%
\subsection{Copyright}

Copyright \copyright{} 2017--2018 Niklas Beisert

This work may be distributed and/or modified under the
conditions of the \LaTeX{} Project Public License, either version 1.3
of this license or (at your option) any later version.
The latest version of this license is in
  \url{http://www.latex-project.org/lppl.txt}
and version 1.3 or later is part of all distributions of \LaTeX{}
version 2005/12/01 or later.

This work has the LPPL maintenance status `maintained'.

The Current Maintainer of this work is Niklas Beisert.

This work consists of the files |README.txt|, |childdoc.ins| and |childdoc.dtx|
as well as the derived files |childdoc.def|, |cdocsamp.tex|
with |cdocsch1.tex|, |cdocsch2.tex|, |cdocspt3.tex|, |cdocspt4.tex|,
|cdocsdrf.tex|, |cdocsfn1.tex|, |cdocsfn2.tex|
as well as |childdoc.pdf|.

%%%%%%%%%%%%%%%%%%%%%%%%%%%%%%%%%%%%%%%%%%%%%%%%%%%%%%%%%%%%%%%%%%%%%%%%%%%%%%%%
\subsection{Files and Installation}

The package consists of the files:
%
\begin{center}
\begin{tabular}{ll}
    |README.txt|   & readme file \\
    |childdoc.ins| & installation file \\
    |childdoc.dtx| & source file \\
    |childdoc.def| & definition file \\
    |cdocsamp.tex| & sample main file \\
    |cdocsch1.tex| & sample include file \\
    |cdocsch2.tex| & sample include file \\
    |cdocspt3.tex| & sample part file \\
    |cdocspt4.tex| & sample part file \\
    |cdocsdrf.tex| & sample redirection file \\
    |cdocsfn1.tex| & sample redirection file \\
    |cdocsfn2.tex| & sample redirection file \\
    |childdoc.pdf| & manual
\end{tabular}
\end{center}
%
The distribution consists of the files
|README.txt|, |childdoc.ins| and |childdoc.dtx|.
%
\begin{itemize}
\item
Run (pdf)\LaTeX{} on |childdoc.dtx|
to compile the manual |childdoc.pdf| (this file).
\item
Run \LaTeX{} on |childdoc.ins| to create the definitions file |childdoc.def|
and the sample |cdocsamp.tex| with include files
|cdocsch1.tex|, |cdocsch2.tex|, |cdocspt3.tex|, |cdocspt4.tex|,
|cdocsdrf.tex|, |cdocsfn1.tex|, |cdocsfn2.tex|.
Then copy the file |childdoc.def| to an appropriate directory of your \LaTeX{}
distribution, e.g.\ \textit{texmf-root}|/tex/latex/childdoc|.
\end{itemize}

%%%%%%%%%%%%%%%%%%%%%%%%%%%%%%%%%%%%%%%%%%%%%%%%%%%%%%%%%%%%%%%%%%%%%%%%%%%%%%%%
\subsection{Related CTAN Packages}

There are several other packages which offer a similar functionality:
%
\begin{itemize}
\item
The packages
\href{http://ctan.org/pkg/docmute}{\textsf{docmute}},
\href{http://ctan.org/pkg/includex}{\textsf{includex}} and
\href{http://ctan.org/pkg/standalone}{\textsf{standalone}}
provide commands to include only the document body of
a child file thus allowing both files to be compiled individually.
\item
The packages \href{http://ctan.org/pkg/subdocs}{\textsf{subdocs}}
and \href{http://ctan.org/pkg/subfiles}{\textsf{subfiles}}
provide structures in which the main and child documents can be
encapsulated and allowing them to be compiled individually.
The inclusion mechanism is different from the conventional |\include|.
\item
The package \href{http://ctan.org/pkg/combine}{\textsf{combine}}
is an elaborate solution to combine several documents into one.
\end{itemize}
%
See also the CTAN topic \href{http://ctan.org/topic/subdocs}{\textsf{subdocs}}
for further related packages.
The present package differs from the above solutions in that
a document structure constructed with the conventional |\include| mechanism
just needs two extra commands at the top of every file
such that all constituent files can be compiled individually.

%%%%%%%%%%%%%%%%%%%%%%%%%%%%%%%%%%%%%%%%%%%%%%%%%%%%%%%%%%%%%%%%%%%%%%%%%%%%%%%%
%\subsection{Feature Suggestions}
%
%The following is a list of features which may be useful for future
%versions of this package:
%%
%\begin{itemize}
%\item
%\ldots
%\end{itemize}

%%%%%%%%%%%%%%%%%%%%%%%%%%%%%%%%%%%%%%%%%%%%%%%%%%%%%%%%%%%%%%%%%%%%%%%%%%%%%%%%
\subsection{Revision History}

%%%%%%%%%%%%%%%%%%%%%%%%%%%%%%%%%%%%%%%%
\paragraph{v2.0:} 2018/12/30

\begin{itemize}
\item
immediate forward processing
\item
added |\childdocby| mechanism
\item
manual restructured
\end{itemize}

%%%%%%%%%%%%%%%%%%%%%%%%%%%%%%%%%%%%%%%%
\paragraph{v1.6:} 2018/01/17

\begin{itemize}
\item
application for development of include files
\item
corrections to manual
\end{itemize}

%%%%%%%%%%%%%%%%%%%%%%%%%%%%%%%%%%%%%%%%
\paragraph{v1.5:} 2017/05/21

\begin{itemize}
\item
more complete structuring introduced
\item
|\childdocof| introduced
\item
|\childdoc| renamed to |\childdocmain|
\item
|\childredirect| renamed to |\childdocforward| and |\childdocforwardprefix|
and functionality expanded
\end{itemize}

%%%%%%%%%%%%%%%%%%%%%%%%%%%%%%%%%%%%%%%%
\paragraph{v1.0:} 2017/04/27

\begin{itemize}
\item
manual and install package
\item
first version published on CTAN
\end{itemize}

%%%%%%%%%%%%%%%%%%%%%%%%%%%%%%%%%%%%%%%%
\paragraph{v0.6:} 2017/04/26

\begin{itemize}
\item
redirection mechanism added
\end{itemize}

%%%%%%%%%%%%%%%%%%%%%%%%%%%%%%%%%%%%%%%%
\paragraph{v0.5:} 2017/04/26

\begin{itemize}
\item
functionality in definition file
\end{itemize}


%%%%%%%%%%%%%%%%%%%%%%%%%%%%%%%%%%%%%%%%%%%%%%%%%%%%%%%%%%%%%%%%%%%%%%%%%%%%%%%%
%%%%%%%%%%%%%%%%%%%%%%%%%%%%%%%%%%%%%%%%%%%%%%%%%%%%%%%%%%%%%%%%%%%%%%%%%%%%%%%%
%%%%%%%%%%%%%%%%%%%%%%%%%%%%%%%%%%%%%%%%%%%%%%%%%%%%%%%%%%%%%%%%%%%%%%%%%%%%%%%%
\appendix

\settowidth\MacroIndent{\rmfamily\scriptsize 000\ }

 \DocInput{childdoc.dtx}

\end{document}
%</driver>
% \fi
%
% %%%%%%%%%%%%%%%%%%%%%%%%%%%%%%%%%%%%%%%%%%%%%%%%%%%%%%%%%%%%%%%%%%%%%%%%%%%%%%
% %%%%%%%%%%%%%%%%%%%%%%%%%%%%%%%%%%%%%%%%%%%%%%%%%%%%%%%%%%%%%%%%%%%%%%%%%%%%%%
% \section{Sample}
%\iffalse
%<*samplemain>
%\fi
%
% The following presents a sample document
% with two chapters, two parts, a title page,
% a compile flag as well as three forwarding files to set the flag.
% It consists of eight |.tex| files:
% \begin{center}
% \begin{tabular}{ll}
% |cdocsamp.tex|&main file\\
% |cdocsch1.tex|&include file for chapter 1\\
% |cdocsch2.tex|&include file for chapter 2\\
% |cdocspt3.tex|&include file for part 3\\
% |cdocspt4.tex|&include file for part 4\\
% |cdocsdrf.tex|&forwarding file for main file in draft mode\\
% |cdocsfi1.tex|&forwarding file for final version of chapter 1\\
% |cdocsfi2.tex|&forwarding file for final version of chapter 2\\
% \end{tabular}
% \end{center}
% Each of the eight files can be compiled directly by the \LaTeX{} compiler.
%
% %%%%%%%%%%%%%%%%%%%%%%%%%%%%%%%%%%%%%%
% \paragraph{Main File.}
%
% The main file is called |cdocsamp.tex|.
%
% Load the \textsf{childdoc} definitions and
% declare the filename for the main document:
%    \begin{macrocode}
\input{childdoc.def}
\childdocmain{}
%    \end{macrocode}

% Optional override for |\version| flag:
%    \begin{macrocode}
%%\ifchilddoc\else\providecommand{\version}{draft}\fi
%    \end{macrocode}

% Define the default values for the |\version| flag
% (|final| for the main file and |draft| for childs):
%    \begin{macrocode}
\ifchilddoc
\providecommand{\version}{draft}
\else
\providecommand{\version}{final}
\fi
%    \end{macrocode}

% Load the standard document class:
%    \begin{macrocode}
\documentclass[12pt]{article}
%    \end{macrocode}

% Start the document body:
%    \begin{macrocode}
\begin{document}
%    \end{macrocode}

% Declare a title page.
% Print title, part of document being processed and version flag:
%    \begin{macrocode}
\addtocounter{page}{-1}
\begin{center}
{\LARGE\bfseries{}childdoc example\par}
\vspace{1cm}
\ifchilddoc
\ifchilddocmanual part\else chapter\fi:
`\childdocname' of `\childdocjob'\par
\else
main document: `\childdocjob'\par
\fi
version: \version\par
\end{center}
\newpage
%    \end{macrocode}

% Manually include selected file,
% otherwise process as usual:
%    \begin{macrocode}
\ifchilddocmanual
\section*{part `\childdocname'}
\input{\childdocname}
\else
%    \end{macrocode}

% Include the two chapters:
%    \begin{macrocode}
\include{cdocsch1}
\include{cdocsch2}
%    \end{macrocode}

% Include the two parts unless only chapters should be displayed:
%    \begin{macrocode}
\ifchilddoc\else
\section{part three}
\input{cdocspt3}
\section{part four}
\input{cdocspt4}
\fi
%    \end{macrocode}

% Process as usual until here:
%    \begin{macrocode}
\fi
%    \end{macrocode}

% End of document body:
%    \begin{macrocode}
\end{document}
%    \end{macrocode}
%\iffalse
%</samplemain>
%\fi
%
% %%%%%%%%%%%%%%%%%%%%%%%%%%%%%%%%%%%%%%
% \paragraph{Chapter Include Files.}
%
% The include files are called |cdocsch1.tex| and |cdocsch2.tex|.
%
%\iffalse
%<*samplechap1|samplechap2>
%\fi

% Optional override for |\version| flag:
%    \begin{macrocode}
%%\providecommand{\version}{final}
%    \end{macrocode}

% Include the main document:
%    \begin{macrocode}
\input{childdoc.def}
\childdocof{cdocsamp}
%    \end{macrocode}

%\iffalse
%</samplechap1|samplechap2>
%\fi
%
%\iffalse
%<*samplechap1>
%\fi
% Some text for chapter 1:
%    \begin{macrocode}
\section{one}
some text in chapter one
%    \end{macrocode}

%\iffalse
%</samplechap1>
%\fi
% Some text for chapter 2:
%\iffalse
%<*samplechap2>
%\fi
%    \begin{macrocode}
\section{two}
more text in chapter two
%    \end{macrocode}

%\iffalse
%</samplechap2>
%\fi
%
% %%%%%%%%%%%%%%%%%%%%%%%%%%%%%%%%%%%%%%
% \paragraph{Part Include Files.}
%
% The include files are called |cdocspt3.tex| and |cdocspt4.tex|.
%
%\iffalse
%<*samplepart3|samplepart4>
%\fi

% Optional override for |\version| flag:
%    \begin{macrocode}
%%\providecommand{\version}{final}
%    \end{macrocode}

% Include the main document:
%    \begin{macrocode}
\input{childdoc.def}
\childdocby{cdocsamp}
%    \end{macrocode}

%\iffalse
%</samplepart3|samplepart4>
%\fi
%
%\iffalse
%<*samplepart3>
%\fi
% Some text for part 3:
%    \begin{macrocode}
some text in part three
%    \end{macrocode}

%\iffalse
%</samplepart3>
%\fi
% Some text for part 4:
%\iffalse
%<*samplepart4>
%\fi
%    \begin{macrocode}
more text in part four
%    \end{macrocode}

%\iffalse
%</samplepart4>
%\fi
%
% %%%%%%%%%%%%%%%%%%%%%%%%%%%%%%%%%%%%%%
% \paragraph{Forwarding for a Complete Draft.}
%
% The following forwarding file |cdocsdrf.tex|
% compiles the main document in draft mode:
%\iffalse
%<*sampledraft>
%\fi
%    \begin{macrocode}
\def\version{draft}
\input{childdoc.def}
\childdocforward{cdocsamp}
%    \end{macrocode}

%\iffalse
%</sampledraft>
%\fi
%
% %%%%%%%%%%%%%%%%%%%%%%%%%%%%%%%%%%%%%%
% \paragraph{Forwarding for Final Version of the Chapters.}
%
% The following forwarding files |cdocsfn1.tex| and |cdocsfn2.tex|
% (with identical content)
% compile the final versions of the child documents
% |cdocsch1.tex| and |cdocsch2.tex|, respectively:
%\iffalse
%<*samplefinal>
%\fi
%    \begin{macrocode}
\def\version{final}
\input{childdoc.def}
\childdocforwardprefix[cdocsamp]{cdocsfn}{cdocsch}
%    \end{macrocode}

%\iffalse
%</samplefinal>
%\fi
%
% %%%%%%%%%%%%%%%%%%%%%%%%%%%%%%%%%%%%%%
% \paragraph{Command Line Processing.}
%
% The following three command lines generate the output files
% |cdocscld|, |cdocscl1| and |cdocscl2|
% which should be identical to
% |cdocsdrf|, |cdocsch1| and |cdocsfn2|, respectively:
% \begin{center}
% \begin{tabular}{l}
% |latex -jobname cdocscld \|\\
% |  "\def\version{draft}\input{childdoc.def}\childdocforward{cdocsamp}"|\\
% |latex -jobname cdocscl1 \|\\
% |  "\input{childdoc.def}\childdocforward[cdocsamp]{cdocsch1}"|\\
% |latex -jobname cdocscl2 \|\\
% |  "\def\version{final}\input{childdoc.def}\childdocforward{cdocsch2}"|
% \end{tabular}
% \end{center}
% Note that the trailing backslash on each first line
% merely continues the input to the second line
% (for convenient cut ant paste).
% Furthermore, the command |latex| can be replaced by any
% of its alternative versions such as |pdflatex|.
%
% %%%%%%%%%%%%%%%%%%%%%%%%%%%%%%%%%%%%%%%%%%%%%%%%%%%%%%%%%%%%%%%%%%%%%%%%%%%%%%
% %%%%%%%%%%%%%%%%%%%%%%%%%%%%%%%%%%%%%%%%%%%%%%%%%%%%%%%%%%%%%%%%%%%%%%%%%%%%%%
% \section{Implementation}
%\iffalse
%<*package>
%\fi
%
% This section describes the definitions file |childdoc.def|.

% The definitions cannot be loaded using |\usepackage| or |\RequirePackage|
% which has a mechanism to prevent loading a style file more than once.
% When loading the definitions by means of |\input|
% multiple instances have to be prevented manually:
%\iffalse
%This code needs to be before the `\ProvidesFile' directive
%which is defined at the beginning of this file.
%Therefore it is also placed there and commented out here.
%</package>
%<*discard>
%\fi
%    \begin{macrocode}
\ifdefined\childdocmain\endinput\fi
%    \end{macrocode}
%\iffalse
%</discard>
%<*package>
%\fi
%
% \macro{\ifchilddoc}
% \macro{\ifchilddocmanual}
% The conditional |\ifchilddoc| tells whether a
% child (true) or main (false) document is being compiled.
% The conditional |\ifchilddocmanual| tells whether
% the |\includeonly| mechanism is used (false) or
% the selection of child files must be performed manually (true).
% The definitions initialise to false:
%    \begin{macrocode}
\newif\ifchilddoc
\newif\ifchilddocmanual
%    \end{macrocode}

% \macro{\childdocname}
% \macro{\childdocjob}
% The macro |\childdocname| stores the name of the main document
% to be compiled. The macro |\childdocjob| stores the name of
% the document on which the \LaTeX{} compiler was originally invoked.
% The content of |\jobname| cannot be compared
% to filenames specified in the source due to different catcodes.
% The following code rescans |\jobname|, stores the result
% in |\childdocname| and saves a copy in |\childdocjob|:
%    \begin{macrocode}
\edef\childdocname{\scantokens\expandafter{\jobname\noexpand}}
\let\childdocjob\childdocname
%    \end{macrocode}

% \macro{\childdocdisable}
% The macro |\childdocdisable| prevents the main file
% from being processed more than once.
% At this stage, the main document command |\childdocmain|
% is assumed to be called once again where it should do nothing.
% Any subsequent call to it should prevent
% a secondary processing of the main document
% It overwrites the forwarding commands
% |\childdocof| and |\childdocforward|
% with empty macros to prevent further inclusions of the main document:
%    \begin{macrocode}
\newcommand{\childdocdisable}
{
  \renewcommand{\childdocmain}[1]{\renewcommand{\childdocmain}[1]{\endinput}}
  \renewcommand{\childdocof}[1]{}
  \renewcommand{\childdocby}[2][]{}
  \renewcommand{\childdocforward}[2][]{}
  \renewcommand{\childdocdisable}{}
}
%    \end{macrocode}

% \macro{\childdocmain}
% The macro |\childdocmain| is to be called at the top of the main file
% with nothing or the main filename (without extension) as argument.
% First, it breaks loops.
% If the argument is not empty and does not match |\childdocname|
% (which is set by the first inclusion of |childdoc.def|),
% |\ifchilddoc| is set to true, |\includeonly| is applied to the child file
% and |\jobname| is set to the main file
% (for proper handling of |.aux| files):
%    \begin{macrocode}
\newcommand{\childdocmain}[1]
{
  \childdocdisable\childdocmain{}
  \if?#1?\else
    \begingroup
      \def\childdoctmp{#1}
      \ifx\childdoctmp\childdocname
        \def\childdoctmp{}
      \else
        \def\childdoctmp
        {
          \childdoctrue
          \includeonly{\childdocname}
          \def\childdocjob{#1}
          \def\jobname{#1}
        }
      \fi
      \expandafter
    \endgroup
    \childdoctmp
  \fi
}
%    \end{macrocode}

% \macro{\childdocof}
% The command |\childdocof| redirects
% compilation to the main file |#1|.
%    \begin{macrocode}
\newcommand{\childdocof}[1]
{
  \childdocdisable
  \childdoctrue
  \includeonly{\childdocname}
  \def\jobname{#1}
  \def\childdocjob{#1}
  \input{#1}
}
%    \end{macrocode}

% \macro{\childdocby}
% The command |\childdocby| ....
%    \begin{macrocode}
\newcommand{\childdocby}[2][]
{
  \childdocdisable
  \childdoctrue
  \childdocmanualtrue
  \if?#1?\else
    \def\jobname{#2}
  \fi
  \def\childdocjob{#2}
  \input{#2}
  \endinput
}
%    \end{macrocode}

% \macro{\childdocforward}
% The command |\childdocforward| redirects
% compilation to the main file or
% (if the optional argument is given) a child file.
% Parameters are set as if the main file
% or a child file starting with |\childdocof| was compiled.
% Then compilation is handed over to the main file:
%    \begin{macrocode}
\newcommand{\childdocforward}[2][]
{
  \begingroup
    \if?#1?
      \def\childdoctmp
      {
        \def\childdocname{#2}
        \def\childdocjob{#2}
        \def\jobname{#2}
        \input{#2}
        \endinput
      }
    \else
      \def\childdoctmp
      {
        \childdocdisable
        \def\childdocname{#2}
        \childdoctrue
        \includeonly{#2}
        \def\childdocjob{#1}
        \def\jobname{#1}
        \input{#1}
        \endinput
      }
    \fi
    \expandafter
  \endgroup
  \childdoctmp
}
%    \end{macrocode}

% \macro{\childdocforwardprefix}
% The command |\childdocforwardprefix| redirects
% compilation to the main or a child file by means of a pattern.
% The prefix |#1| in the current filename is replaced by |#2|
% and the suffix of the current filename is kept
% (it is assumed that the filename does not contain the substring `|~~~|'
% which is used as a delimiter).
% Compilation is handed over to the new file by |\childdocforward|:
%    \begin{macrocode}
\newcommand{\childdocforwardprefix}[3][]
{
  \begingroup
    \def\childdocextract #2##1~~~{\def\childdoctmp{\childdocforward[#1]{#3##1}}}
    \expandafter\childdocextract\childdocname~~~
    \expandafter
  \endgroup
  \childdoctmp
}
%    \end{macrocode}

% \macro{\childdoc}
% The deprecated macro |\childdoc| is a legacy version of |\childdocmain|:
%    \begin{macrocode}
\newcommand{\childdoc}{\childdocmain}
%    \end{macrocode}

% \macro{\childdocredirect}
% The deprecated macro |\childdocredirect| is a legacy version
% of |\childdocforward| and |\childdocforwardprefix|:
%    \begin{macrocode}
\newcommand{\childdocredirect}[2][]
{
  \begingroup
    \if?#1?
      \def\childdoctmp{\childdocforward{#2}}
    \else
      \def\childdoctmp{\childdocforwardprefix{#1}{#2}}
    \fi
    \expandafter
  \endgroup
  \childdoctmp
}
%    \end{macrocode}

%\iffalse
%</package>
%\fi
%
\endinput
|\\
|\childdocforward[|\textit{main}|]{|\textit{dest}|}|\\
\end{tabular}
\end{center}
%
The argument \textit{dest} is the destination file
(without extension).
It should be the main file or one of the child files.
Note that further \textsf{childdoc} directives
such as |\childdocof| and |\childdocforward|
in the indicated file will be processed in this form.
The optional argument \textit{main}
passes on directly to the main file \textit{main}
while pretending to compile the child \textit{dest}.
This form behaves as if \textit{dest}
issues |\childdocof{|\textit{main}|}| right away,
and no further \textsf{childdoc} directives will be processed.

%%%%%%%%%%%%%%%%%%%%%%%%%%%%%%%%%%%%%%%%
\DescribeMacro{\...prefix}
In the alternative form |\childdocforwardprefix|,
%
\begin{center}
\begin{tabular}{l}
|% \iffalse
%
% childdoc.dtx Copyright (C) 2017-2018 Niklas Beisert
%
% This work may be distributed and/or modified under the
% conditions of the LaTeX Project Public License, either version 1.3
% of this license or (at your option) any later version.
% The latest version of this license is in
%   http://www.latex-project.org/lppl.txt
% and version 1.3 or later is part of all distributions of LaTeX
% version 2005/12/01 or later.
%
% This work has the LPPL maintenance status `maintained'.
%
% The Current Maintainer of this work is Niklas Beisert.
%
% This work consists of the files childdoc.dtx and childdoc.ins
% and the derived files childdoc.def and cdocsamp.tex with
% cdocsch1.tex, cdocsch2.tex, cdocsdrf.tex, cdocsfn1.tex, cdocsfn2.tex.
%
%<package>\ifdefined\childdocmain\endinput\fi
%<package>\ProvidesFile{childdoc.def}[2018/12/30 v2.0 child document driver]
%<samplemain>\ProvidesFile{cdocsamp.tex}[2018/12/30 v2.0 sample for childdoc]
%<*driver>
%\ProvidesFile{childdoc.drv}[2018/12/30 v2.0 childdoc reference manual file]
\PassOptionsToClass{10pt,a4paper}{article}
\documentclass{ltxdoc}

\usepackage[margin=35mm]{geometry}
\usepackage{hyperref}
\usepackage{hyperxmp}
\usepackage[usenames]{color}

\hypersetup{colorlinks=true}
\hypersetup{pdfstartview=FitH}
\hypersetup{pdfpagemode=UseNone}
\hypersetup{pdfsource={}}
\hypersetup{pdflang={en-UK}}
\hypersetup{pdfcopyright={Copyright 2017-2018 Niklas Beisert.
  This work may be distributed and/or modified under the
  conditions of the LaTeX Project Public License, either version 1.3
  of this license or (at your option) any later version.}}
\hypersetup{pdflicenseurl={http://www.latex-project.org/lppl.txt}}
\hypersetup{pdfcontactaddress={ETH Zurich, ITP, HIT K,
  Wolfgang-Pauli-Strasse 27}}
\hypersetup{pdfcontactpostcode={8093}}
\hypersetup{pdfcontactcity={Zurich}}
\hypersetup{pdfcontactcountry={Switzerland}}
\hypersetup{pdfcontactemail={nbeisert@itp.phys.ethz.ch}}
\hypersetup{pdfcontacturl={http://people.phys.ethz.ch/\xmptilde nbeisert/}}

\newcommand{\secref}[1]{\hyperref[#1]{section \ref*{#1}}}

\parskip1ex
\parindent0pt
\let\olditemize\itemize
\def\itemize{\olditemize\parskip0pt}

\begin{document}

\title{The \textsf{childdoc} Package}
\hypersetup{pdftitle={The childdoc Package}}
\author{Niklas Beisert\\[2ex]
  Institut f\"ur Theoretische Physik\\
  Eidgen\"ossische Technische Hochschule Z\"urich\\
  Wolfgang-Pauli-Strasse 27, 8093 Z\"urich, Switzerland\\[1ex]
  \href{mailto:nbeisert@itp.phys.ethz.ch}
  {\texttt{nbeisert@itp.phys.ethz.ch}}}
\hypersetup{pdfauthor={Niklas Beisert}}
\hypersetup{pdfsubject={Manual for the LaTeX2e Package childdoc}}
\date{30 December 2018, \textsf{v2.0}}
\maketitle

\begin{abstract}\noindent
\textsf{childdoc} is a \LaTeXe{} package
that enables the direct compilation
of document sections included by |\include|
to individual files.
\end{abstract}

\begingroup
\parskip0ex
\tableofcontents
\endgroup

%%%%%%%%%%%%%%%%%%%%%%%%%%%%%%%%%%%%%%%%%%%%%%%%%%%%%%%%%%%%%%%%%%%%%%%%%%%%%%%%
%%%%%%%%%%%%%%%%%%%%%%%%%%%%%%%%%%%%%%%%%%%%%%%%%%%%%%%%%%%%%%%%%%%%%%%%%%%%%%%%
\section{Introduction}

\LaTeX{} provides a mechanism to structure a large document (such as a book)
into a main file and several child files (containing the chapters)
using the |\include| command.
This mechanism is beneficial for documents
which span hundreds of pages in order to
make the source file(s) more manageable.
Moreover, compilation can be restricted to
selected child files by means of the |\includeonly| command.
The latter feature can be used to reduce the compilation time while editing
(this was significantly more useful in the earlier days of \LaTeX{})
or to generate a smaller document which is easier to navigate.
Another application of |\includeonly| is to generate
documents consisting of selected parts of the complete document.

However, there are a few drawbacks of the plain |\include| mechanism:
\begin{itemize}
\item
The child files cannot be compiled on their own,
they can only be compiled via the main file.
A naive editing environment
(such as a text editor with an option
to have the current file processed by \LaTeX)
may require one to switch to the main file before compiling;
attempting to compile the child file produces errors.
\item
The main file must be modified (each time)
to adjust the |\includeonly| command
to the present needs. This easily leaves the main file in a messy state.
\item
The generated document will always carry the filename
of the main document. This is inconvenient if
several child files are to be compiled and
to be kept for distribution.
\end{itemize}

The present package provides a simple interface
to make child files individually compilable by \LaTeX{}.
Compiling a child file then has the same effect as compiling
the main file with an |\includeonly| command
to select the appropriate child.
Moreover the generated document will carry the name of the child
rather than the main file.
This resolves all three above issues.

This feature is meant to make the editing of books,
thesis documents and lecture notes somewhat more convenient.
However, the package can also be used efficiently for
composing a series of documents (such as exercise sheets)
which are typically distributed individually.
It then assists the author in generating the individual documents
(potentially in different versions)
as well as a document containing the collected series.
Another application is in developing style files
or other kinds of included material
where compilation of the style file could redirect
to a sample or test file.

%%%%%%%%%%%%%%%%%%%%%%%%%%%%%%%%%%%%%%%%%%%%%%%%%%%%%%%%%%%%%%%%%%%%%%%%%%%%%%%%
%%%%%%%%%%%%%%%%%%%%%%%%%%%%%%%%%%%%%%%%%%%%%%%%%%%%%%%%%%%%%%%%%%%%%%%%%%%%%%%%
\section{Usage}

First of all, the package \textsf{childdoc} is \emph{not} a standard
\LaTeXe{} |.sty| style file! Therefore it needs to be invoked in
a non-standard way.

%%%%%%%%%%%%%%%%%%%%%%%%%%%%%%%%%%%%%%%%%%%%%%%%%%%%%%%%%%%%%%%%%%%%%%%%%%%%%%%%
\subsection{Included Files}
\label{sec:include}

%%%%%%%%%%%%%%%%%%%%%%%%%%%%%%%%%%%%%%%%
\DescribeMacro{\childdocmain}
To use the package, add the commands
\begin{center}
\begin{tabular}{l}
|\input{childdoc.def}|\\
|\childdocmain{}|\\
\end{tabular}
\end{center}
at the very top of the main \LaTeX{} file,
in particular \emph{before} the |\documentclass| statement!
The argument of |\childdocmain| should be left empty
(but it must be present).

%%%%%%%%%%%%%%%%%%%%%%%%%%%%%%%%%%%%%%%%
\DescribeMacro{\childdocof}
Furthermore, add the commands
\begin{center}
\begin{tabular}{l}
|\input{childdoc.def}|\\
|\childdocof{|\textit{main}|}|\\
\end{tabular}
\end{center}
at the top of every child file \textit{child}
which is included by |\include{|\textit{child}|}|
from within the main file
(or at least for those files to be compiled individually).
The argument \textit{main} must be the filename of the main file.

There are a couple of
considerations in setting up the main and child documents:

%%%%%%%%%%%%%%%%%%%%%%%%%%%%%%%%%%%%%%%%
\paragraph{Restrictions.}

Please note the following restrictions:
\begin{itemize}
\item
|\childdocmain| must be called with one argument \textit{main}
to ensure compatibility with earlier version of the package.
It must either be empty (|\childdocmain{}|)
or precisely match the filename of the main file in which it is specified.
See \secref{sec:detection} for further information.
\item
The filename \textit{main} must be specified without the |.tex| extension.
\item
The filename \textit{main} is case sensitive
(even in case-insensitive file systems)
due to internal string comparison.
\item
The argument \textit{main} should be fully expanded, it cannot be a macro.
\item
Subdirectories and special characters should be avoided in filenames.
\item
The command |\childdocmain{|\textit{main}|}| must be followed by a whitespace.
It should not be followed immediately by another command
or by a comment mark `|%|'.
This is because the \TeX{} parser reads the token immediately following
the argument of |\childdocmain| and puts it
at the beginning of every child section;
however, a white\-space is ignored.
\end{itemize}

%%%%%%%%%%%%%%%%%%%%%%%%%%%%%%%%%%%%%%%%
\paragraph{Content of Main File.}

It is advisable to place all content in the child files included by |\include|.
Any output contained in the main file will appear in all child documents
unless suppressed manually;
it cannot be suppressed automatically by the |\includeonly| directive
and thus should normally be avoided.
A method to include some content in the main file
by means of conditional processing is described in \secref{sec:conditional}.

%%%%%%%%%%%%%%%%%%%%%%%%%%%%%%%%%%%%%%%%
\paragraph{Page Numbering.}

When only a part of the document is compiled,
the appropriate numbering of pages
(as well as other status parameters)
is determined from the |.aux| files.
The latter contain information from previous passes.
However this information needs to propagate through
all intermediate child documents.
Therefore the page numbering in child documents may well
be inconsistent until the complete document is compiled at least once.

A useful (if unconventional) way to always ensure a consistent
page numbering is to restart the numbering in each child document
and denote the pages by `\textit{child}|.|\textit{page}'
where \textit{child} represents the chapter/section number of the child file.
This can be achieved by the command
|\numberwithin{page}{|\textit{child}|}|
of the \textsf{amsmath} package
where \textit{child} can be |chapter| or |section|
depending on the chosen structuring.
Alternatively, one can modify the macro |\thepage| appropriately
and reset the counter |page| at the start of each child file.

%%%%%%%%%%%%%%%%%%%%%%%%%%%%%%%%%%%%%%%%%%%%%%%%%%%%%%%%%%%%%%%%%%%%%%%%%%%%%%%%
\subsection{Conditional Processing}
\label{sec:conditional}

The package provides a mechanism to compile different versions
of a document. To customise the versions further some conditional processing
can come in handy to distinguish which version is being compiled.
The package provides two macros to describe the compilation context:

%%%%%%%%%%%%%%%%%%%%%%%%%%%%%%%%%%%%%%%%
\DescribeMacro{\ifchilddoc}
The conditional |\ifchilddoc| distinguishes between the compilation of
child documents and the main document:
%
\begin{center}
|\ifchilddoc |\textit{child-code}| |[|\||else |\textit{main-code}]| \||fi|
\end{center}

%%%%%%%%%%%%%%%%%%%%%%%%%%%%%%%%%%%%%%%%
\DescribeMacro{\childdocname}
\DescribeMacro{\childdocjob}
The macro |\childdocname| contains the filename (without extension)
of the main or child file being processed.
Note that |\childdocjob| will always contain the name of the main file.

%%%%%%%%%%%%%%%%%%%%%%%%%%%%%%%%%%%%%%%%
\paragraph{Title Page.}

Conditional processing can be used to include a title or banner page
in the main document when proper precautions are taken.
Importantly, the code in the main file should ensure that the page counter
(as well as other status parameters which are stored in the |.aux| files)
takes the same value after the conditional processing.
Otherwise the page numbers may take divergent values
depending on which part is compiled.

For example, a title page could be declared by:
%
\begin{center}
\begin{tabular}{l}
|\ifchilddoc\||else|\\
|\addtocounter{page}{-1}|\\
\textit{code for title page}\\
|\newpage|\\
|\||fi|
\end{tabular}
\end{center}
%
A banner page for the child documents can be generated by:
%
\begin{center}
\begin{tabular}{l}
|\ifchilddoc|\\
|\addtocounter{page}{-1}|\\
\textit{code for banner page}\\
|\newpage|\\
|\||fi|
\end{tabular}
\end{center}
%
Here one could write a message such as:
\begin{center}
|This is the part \childdocname{} of \childdocjob{}.|
\end{center}

%%%%%%%%%%%%%%%%%%%%%%%%%%%%%%%%%%%%%%%%%%%%%%%%%%%%%%%%%%%%%%%%%%%%%%%%%%%%%%%%
\subsection{Flags}
\label{sec:flags}

The package makes it easy to generate different versions
of the main or child documents.
To this end compilation flags can be defined
and assigned different default values.
They will be particularly useful in conjunction
with the forwarding mechanism described in \secref{sec:forward}.

For example, it may be useful to have a flag |\version|
which can be set to |draft| or |final|.
The document source will contain some conditional code
depending on the value of |\version|.
Suppose further, the flag should default to |final| for the main file
and to |draft| for child files
which is a natural assignment for editing the document.
This is achieved by placing the following code
in the preamble of the main document
(below the |\childdocmain| directive):
%
\begin{center}
\begin{tabular}{l}
|\ifchilddoc|\\
|\providecommand{\version}{draft}|\\
|\||else|\\
|\providecommand{\version}{final}|\\
|\||fi|
\end{tabular}
\end{center}
%
The definition by |\providecommand| makes sure
that previous definitions are not overwritten.
Further statements |\providecommand{\version}{...}|
can thus be added before the above code to override it.

For the main file, one might add a line
(between |\childdocmain| and the above block)
%
\begin{center}
|%\ifchilddoc\||else\providecommand{\version}{draft}\||fi|
\end{center}
%
which can be uncommented to produce a draft version.
Likewise one can add a line to the very top of a child file
(above the |\childdocof{|\textit{main}|}| directive)
%
\begin{center}
|%\providecommand{\version}{final}|
\end{center}
%
which can be uncommented to produce the final version of this child document.

%%%%%%%%%%%%%%%%%%%%%%%%%%%%%%%%%%%%%%%%%%%%%%%%%%%%%%%%%%%%%%%%%%%%%%%%%%%%%%%%
\subsection{Forwarding}
\label{sec:forward}

Different versions of the main or child documents
using compilation flags as described in \secref{sec:flags}
can be (permanently) stored in different files
for convenient compilation, viewing and distribution.
To this end, the package defines a command
to pass on compilation to a different file:

%%%%%%%%%%%%%%%%%%%%%%%%%%%%%%%%%%%%%%%%
\DescribeMacro{\childdocforward}
The command |\childdocforward| redirects processing to
another source file:
%
\begin{center}
\begin{tabular}{l}
|\input{childdoc.def}|\\
|\childdocforward[|\textit{main}|]{|\textit{dest}|}|\\
\end{tabular}
\end{center}
%
The argument \textit{dest} is the destination file
(without extension).
It should be the main file or one of the child files.
Note that further \textsf{childdoc} directives
such as |\childdocof| and |\childdocforward|
in the indicated file will be processed in this form.
The optional argument \textit{main}
passes on directly to the main file \textit{main}
while pretending to compile the child \textit{dest}.
This form behaves as if \textit{dest}
issues |\childdocof{|\textit{main}|}| right away,
and no further \textsf{childdoc} directives will be processed.

%%%%%%%%%%%%%%%%%%%%%%%%%%%%%%%%%%%%%%%%
\DescribeMacro{\...prefix}
In the alternative form |\childdocforwardprefix|,
%
\begin{center}
\begin{tabular}{l}
|\input{childdoc.def}|\\
|\childdocforwardprefix[|\textit{main}|]{|\textit{prefix}|}{|\textit{dest}|}|
\end{tabular}
\end{center}
%
the destination file is determined by a pattern
depending on the current file:
To make this work, the current file must be called
`{\textit{prefix}\hspace{0.2em}\textit{suffix}}'
with \textit{prefix} matching precisely the argument.
Processing is then passed on to the file
`{\textit{dest}\hspace{0.2em}\textit{suffix}}'.
Surely, the same effect is achieved by
directly specifying the
argument `{\textit{dest}\hspace{0.2em}\textit{suffix}}'
in the first form.
However, that requires to set up a different file
for each child. With the alternative form of the command
all these files can have exactly the same content
which simplifies setting them up and maintaining them.

For example, the following file |draft.tex|
with a compilation flag |\version| as described in \secref{sec:flags}
compiles the main document as a draft:
%
\begin{center}
\begin{tabular}{l}
|\def\version{draft}|\\
|\input{childdoc.def}|\\
|\childdocforward{|\textit{main}|}|
\end{tabular}
\end{center}
%
Likewise, the following files |final|\textit{nn}|.tex|
compile the final version of the child document
|child|\textit{nn}|.tex|:
%
\begin{center}
\begin{tabular}{l}
|\def\version{final}|\\
|\input{childdoc.def}|\\
|\childdocforwardprefix{final}{child}|
\end{tabular}
\end{center}
%

Note that when several versions of a main file and/or of each child file
are to be generated, it may be convenient to set up a |Makefile| or
shell script to automatise the process.

%%%%%%%%%%%%%%%%%%%%%%%%%%%%%%%%%%%%%%%%%%%%%%%%%%%%%%%%%%%%%%%%%%%%%%%%%%%%%%%%
\subsection{Command Line Processing}
\label{sec:commandline}

The effect of redirection files can also be achieved by invoking
the \LaTeX{} compiler with a more elaborate command line.
Most conveniently this should be done as part
of a shell script or a |Makefile|.

When using \textsf{childdoc} in the main file, the following
command lines effectively perform a redirection
(note that depending on the shell being used,
backslashes may have to be doubled: `|\|' $\to$ `|\\|'):
%
\begin{center}
|... -jobname "|\textit{target}|" |\\|"|[\textit{flags}]%
|\input{childdoc.def}\childdocforward[|\textit{main}|]{|\textit{dest}|}"|
\end{center}
%
Here \textit{target} is the name of the output file,
\textit{main} is the name of the main file
and \textit{dest} is the name of the main or child file to be processed
(all filenames without extensions).
The optional argument \textit{main} can be omitted
if \textit{main} matches \textit{dest}.
Optionally, compilation \textit{flags} can be defined via |\def| commands.
This command line makes the \TeX{} engine believe
it is compiling the file \textit{target}
whose content is specified as the latter parameter.
The provided code then forwards the processing to
\textit{main} or \textit{dest} as described in \secref{sec:forward}.

%%%%%%%%%%%%%%%%%%%%%%%%%%%%%%%%%%%%%%%%%%%%%%%%%%%%%%%%%%%%%%%%%%%%%%%%%%%%%%%%
\subsection{Include by Input}
\label{sec:input}

Including child documents by |\include| has some restrictions by design.
Most notably, the content of a child document always occupies
its own set of pages; pages cannot be shared between child documents.
Usually, this behaviour makes perfect sense
because each child document contain an essential part of the document.
However, in some situations it may be desirable to compose
a document from a collection of parts
without having mandatory page breaks between then.
For this case, the package
provides a mechanism to include parts
by |\input| which can also be processed individually.
However, by construction this mechanism
requires manual handling of the content to be output.

%%%%%%%%%%%%%%%%%%%%%%%%%%%%%%%%%%%%%%%%
\DescribeMacro{\ifchilddocmanual}
The main file should be prepared as usual, see \secref{sec:include}.
However, the document body must make a distinction
between processing of an individual part and of the main document, e.g.:
%
\begin{center}
\begin{tabular}{l}
|\ifchilddocmanual|\\
|\input{\childdocname}|\\
|\||else|\\
\textit{document body with }|\input{|\textit{part}|}|\\
|\||fi|
\end{tabular}
\end{center}
%
The conditional |\ifchilddocmanual| is true whenever
a part to be included by |\input| is being compiled,
and the name of the part is stored in |\childdocname|.

%%%%%%%%%%%%%%%%%%%%%%%%%%%%%%%%%%%%%%%%
\DescribeMacro{\childdocby}
Each part to be included by |\input| should start with:
%
\begin{center}
\begin{tabular}{l}
|\input{childdoc.def}|\\
|\childdocby{|\textit{main}|}|\\
\end{tabular}
\end{center}
%
The directive |\childdocby| is similar to |\childdocof|
described in \secref{sec:include},
but the subsequent selection of content must be done manually.
To that end, both |\ifchilddoc| and |\ifchilddocmanual|
will be true upon processing of a part,
and the name of the part is stored in |\childdocname|.
Note that |\jobname| will be set to the filename of the current part
so that each part receives an individual |.aux| file
that does not interfere with the |.aux| file(s) of the main document.
This behaviour can be altered by the alternative form
|\childdocby[*]{|\textit{main}|}| (with a non-empty optional argument)
which uses the |.aux| file of the main document
by setting |\jobname| to \textit{main}.

%%%%%%%%%%%%%%%%%%%%%%%%%%%%%%%%%%%%%%%%%%%%%%%%%%%%%%%%%%%%%%%%%%%%%%%%%%%%%%%%
\subsection{Driver Development}
\label{sec:driver}

The \textsf{childdoc} mechanism can also be use for the development
of definition files such as \LaTeX{} styles or classes.
This case differs from the above setup with multiple parts
included by |\include| in that no |\includeonly| should be invoked.
This can be achieved by starting the include file
(before |\ProvidesPackage|) with:
%
\begin{center}
\begin{tabular}{l}
|\input{childdoc.def}|\\
|\childdocforward{|\textit{main}|}|\\
\end{tabular}
\end{center}
%
or alternatively with:
%
\begin{center}
\begin{tabular}{l}
|\input{childdoc.def}|\\
|\childdocby{|\textit{main}|}|\\
\end{tabular}
\end{center}
%
Both forms have slightly different effects as described above.
The main file is prepared as usual, see \secref{sec:include}.

%%%%%%%%%%%%%%%%%%%%%%%%%%%%%%%%%%%%%%%%%%%%%%%%%%%%%%%%%%%%%%%%%%%%%%%%%%%%%%%%
\subsection{Legacy Detection}
\label{sec:detection}

The directive |\childdocmain| in the main file can detect
whether the complete document or merely a child is to be compiled
even without using the directive |\childdocof|.
This method is deprecated because it is less robust
and there is no compelling reason to use it;
it is merely provided for backward compatibility
and it may be removed in future versions.

If the detection mechanism is to be used,
it is mandatory to correctly specify
the filename of the main file as the argument of |\childdocmain|:
%
\begin{center}
\begin{tabular}{l}
|\input{childdoc.def}|\\
|\childdocmain{|\textit{main}|}|\\
\end{tabular}
\end{center}
%
If |\jobname| does not match the argument \textit{main} of |\childdocmain|,
it is assumed that |\jobname| points to the child file to be compiled.
When using |\childdocmain| with the main file specified as argument,
it suffices to start a child file
with just |\input{|\textit{main}|}|
without loading of the package and using |\childdocof|.
If instead all processing is done
with the appropriate \textsf{childdoc} directives,
the argument of \textit{main} of |\childdocmain| can be empty.

An alternative version of the command line processing described
in \secref{sec:commandline} using the detection mechanism reads:
%
\begin{center}
|... -jobname "|\textit{target}|" "|[\textit{flags}]%
[|\def\jobname{|\textit{dest}|}|]|\input{|\textit{main}|}"|
\end{center}

%%%%%%%%%%%%%%%%%%%%%%%%%%%%%%%%%%%%%%%%%%%%%%%%%%%%%%%%%%%%%%%%%%%%%%%%%%%%%%%%
\subsection{Manual Code}
\label{sec:manual}

In case one cannot be certain whether the definitions file |childdoc.def|
is installed on the target \TeX{} distribution
and one prefers not to ship it,
it is conceivable to paste a few relevant commands into the sources.

To that end, drop all statements |\input{childdoc.def}|
and perform the replacements as outlined below.
Instead of |\childdocmain{|\textit{main}|}| add the following code
to the top of the main file:
%
\begin{center}
\begin{tabular}{l}
|\||ifdefined\childdocname\endinput\||fi\newif\ifchilddoc|\\
|\edef\childdocname{\scantokens\expandafter{\jobname\noexpand}}|\\
|\def\childdocmain{|\textit{main}|}\||ifx\childdocmain\childdocname\||else|\\
|\childdoctrue\includeonly{\childdocname}\let\jobname\childdocmain\||fi|\\
\end{tabular}
\end{center}
%
Instead of |\childdocof{|\textit{main}|}| just include the main file
at the top of each child file:
%
\begin{center}
|\input{|\textit{main}|}|
\end{center}
%
A simple redirection |\childdocforward{|\textit{dest}|}| is achieved by:
%
\begin{center}
|\def\jobname{|\textit{dest}|}\input{\jobname}|
\end{center}
%
The redirection with prefix
|\childdocforwardprefix[|\textit{prefix}|]{|\textit{dest}|}|
is accomplished by:
%
\begin{center}
\begin{tabular}{l}
|{\edef\jobname{\scantokens\expandafter{\jobname\noexpand}}|\\
|\def\redirectjob |\textit{prefix}|#1~~~{\gdef\jobname{|\textit{dest}|#1}}|\\
|\expandafter\redirectjob\jobname~~~}\input{\jobname}|
\end{tabular}
\end{center}

In an alternative approach,
child documents can be compiled by a specific command line
without additional code or specific definitions:
%
\begin{center}
|... -jobname "|\textit{target}|" "|[\textit{flags}]%
|\includeonly{|\textit{dest}|}\input{|\textit{main}|}"|
\end{center}
%

%%%%%%%%%%%%%%%%%%%%%%%%%%%%%%%%%%%%%%%%%%%%%%%%%%%%%%%%%%%%%%%%%%%%%%%%%%%%%%%%
%%%%%%%%%%%%%%%%%%%%%%%%%%%%%%%%%%%%%%%%%%%%%%%%%%%%%%%%%%%%%%%%%%%%%%%%%%%%%%%%
\section{Information}

%%%%%%%%%%%%%%%%%%%%%%%%%%%%%%%%%%%%%%%%%%%%%%%%%%%%%%%%%%%%%%%%%%%%%%%%%%%%%%%%
\subsection{Copyright}

Copyright \copyright{} 2017--2018 Niklas Beisert

This work may be distributed and/or modified under the
conditions of the \LaTeX{} Project Public License, either version 1.3
of this license or (at your option) any later version.
The latest version of this license is in
  \url{http://www.latex-project.org/lppl.txt}
and version 1.3 or later is part of all distributions of \LaTeX{}
version 2005/12/01 or later.

This work has the LPPL maintenance status `maintained'.

The Current Maintainer of this work is Niklas Beisert.

This work consists of the files |README.txt|, |childdoc.ins| and |childdoc.dtx|
as well as the derived files |childdoc.def|, |cdocsamp.tex|
with |cdocsch1.tex|, |cdocsch2.tex|, |cdocspt3.tex|, |cdocspt4.tex|,
|cdocsdrf.tex|, |cdocsfn1.tex|, |cdocsfn2.tex|
as well as |childdoc.pdf|.

%%%%%%%%%%%%%%%%%%%%%%%%%%%%%%%%%%%%%%%%%%%%%%%%%%%%%%%%%%%%%%%%%%%%%%%%%%%%%%%%
\subsection{Files and Installation}

The package consists of the files:
%
\begin{center}
\begin{tabular}{ll}
    |README.txt|   & readme file \\
    |childdoc.ins| & installation file \\
    |childdoc.dtx| & source file \\
    |childdoc.def| & definition file \\
    |cdocsamp.tex| & sample main file \\
    |cdocsch1.tex| & sample include file \\
    |cdocsch2.tex| & sample include file \\
    |cdocspt3.tex| & sample part file \\
    |cdocspt4.tex| & sample part file \\
    |cdocsdrf.tex| & sample redirection file \\
    |cdocsfn1.tex| & sample redirection file \\
    |cdocsfn2.tex| & sample redirection file \\
    |childdoc.pdf| & manual
\end{tabular}
\end{center}
%
The distribution consists of the files
|README.txt|, |childdoc.ins| and |childdoc.dtx|.
%
\begin{itemize}
\item
Run (pdf)\LaTeX{} on |childdoc.dtx|
to compile the manual |childdoc.pdf| (this file).
\item
Run \LaTeX{} on |childdoc.ins| to create the definitions file |childdoc.def|
and the sample |cdocsamp.tex| with include files
|cdocsch1.tex|, |cdocsch2.tex|, |cdocspt3.tex|, |cdocspt4.tex|,
|cdocsdrf.tex|, |cdocsfn1.tex|, |cdocsfn2.tex|.
Then copy the file |childdoc.def| to an appropriate directory of your \LaTeX{}
distribution, e.g.\ \textit{texmf-root}|/tex/latex/childdoc|.
\end{itemize}

%%%%%%%%%%%%%%%%%%%%%%%%%%%%%%%%%%%%%%%%%%%%%%%%%%%%%%%%%%%%%%%%%%%%%%%%%%%%%%%%
\subsection{Related CTAN Packages}

There are several other packages which offer a similar functionality:
%
\begin{itemize}
\item
The packages
\href{http://ctan.org/pkg/docmute}{\textsf{docmute}},
\href{http://ctan.org/pkg/includex}{\textsf{includex}} and
\href{http://ctan.org/pkg/standalone}{\textsf{standalone}}
provide commands to include only the document body of
a child file thus allowing both files to be compiled individually.
\item
The packages \href{http://ctan.org/pkg/subdocs}{\textsf{subdocs}}
and \href{http://ctan.org/pkg/subfiles}{\textsf{subfiles}}
provide structures in which the main and child documents can be
encapsulated and allowing them to be compiled individually.
The inclusion mechanism is different from the conventional |\include|.
\item
The package \href{http://ctan.org/pkg/combine}{\textsf{combine}}
is an elaborate solution to combine several documents into one.
\end{itemize}
%
See also the CTAN topic \href{http://ctan.org/topic/subdocs}{\textsf{subdocs}}
for further related packages.
The present package differs from the above solutions in that
a document structure constructed with the conventional |\include| mechanism
just needs two extra commands at the top of every file
such that all constituent files can be compiled individually.

%%%%%%%%%%%%%%%%%%%%%%%%%%%%%%%%%%%%%%%%%%%%%%%%%%%%%%%%%%%%%%%%%%%%%%%%%%%%%%%%
%\subsection{Feature Suggestions}
%
%The following is a list of features which may be useful for future
%versions of this package:
%%
%\begin{itemize}
%\item
%\ldots
%\end{itemize}

%%%%%%%%%%%%%%%%%%%%%%%%%%%%%%%%%%%%%%%%%%%%%%%%%%%%%%%%%%%%%%%%%%%%%%%%%%%%%%%%
\subsection{Revision History}

%%%%%%%%%%%%%%%%%%%%%%%%%%%%%%%%%%%%%%%%
\paragraph{v2.0:} 2018/12/30

\begin{itemize}
\item
immediate forward processing
\item
added |\childdocby| mechanism
\item
manual restructured
\end{itemize}

%%%%%%%%%%%%%%%%%%%%%%%%%%%%%%%%%%%%%%%%
\paragraph{v1.6:} 2018/01/17

\begin{itemize}
\item
application for development of include files
\item
corrections to manual
\end{itemize}

%%%%%%%%%%%%%%%%%%%%%%%%%%%%%%%%%%%%%%%%
\paragraph{v1.5:} 2017/05/21

\begin{itemize}
\item
more complete structuring introduced
\item
|\childdocof| introduced
\item
|\childdoc| renamed to |\childdocmain|
\item
|\childredirect| renamed to |\childdocforward| and |\childdocforwardprefix|
and functionality expanded
\end{itemize}

%%%%%%%%%%%%%%%%%%%%%%%%%%%%%%%%%%%%%%%%
\paragraph{v1.0:} 2017/04/27

\begin{itemize}
\item
manual and install package
\item
first version published on CTAN
\end{itemize}

%%%%%%%%%%%%%%%%%%%%%%%%%%%%%%%%%%%%%%%%
\paragraph{v0.6:} 2017/04/26

\begin{itemize}
\item
redirection mechanism added
\end{itemize}

%%%%%%%%%%%%%%%%%%%%%%%%%%%%%%%%%%%%%%%%
\paragraph{v0.5:} 2017/04/26

\begin{itemize}
\item
functionality in definition file
\end{itemize}


%%%%%%%%%%%%%%%%%%%%%%%%%%%%%%%%%%%%%%%%%%%%%%%%%%%%%%%%%%%%%%%%%%%%%%%%%%%%%%%%
%%%%%%%%%%%%%%%%%%%%%%%%%%%%%%%%%%%%%%%%%%%%%%%%%%%%%%%%%%%%%%%%%%%%%%%%%%%%%%%%
%%%%%%%%%%%%%%%%%%%%%%%%%%%%%%%%%%%%%%%%%%%%%%%%%%%%%%%%%%%%%%%%%%%%%%%%%%%%%%%%
\appendix

\settowidth\MacroIndent{\rmfamily\scriptsize 000\ }

 \DocInput{childdoc.dtx}

\end{document}
%</driver>
% \fi
%
% %%%%%%%%%%%%%%%%%%%%%%%%%%%%%%%%%%%%%%%%%%%%%%%%%%%%%%%%%%%%%%%%%%%%%%%%%%%%%%
% %%%%%%%%%%%%%%%%%%%%%%%%%%%%%%%%%%%%%%%%%%%%%%%%%%%%%%%%%%%%%%%%%%%%%%%%%%%%%%
% \section{Sample}
%\iffalse
%<*samplemain>
%\fi
%
% The following presents a sample document
% with two chapters, two parts, a title page,
% a compile flag as well as three forwarding files to set the flag.
% It consists of eight |.tex| files:
% \begin{center}
% \begin{tabular}{ll}
% |cdocsamp.tex|&main file\\
% |cdocsch1.tex|&include file for chapter 1\\
% |cdocsch2.tex|&include file for chapter 2\\
% |cdocspt3.tex|&include file for part 3\\
% |cdocspt4.tex|&include file for part 4\\
% |cdocsdrf.tex|&forwarding file for main file in draft mode\\
% |cdocsfi1.tex|&forwarding file for final version of chapter 1\\
% |cdocsfi2.tex|&forwarding file for final version of chapter 2\\
% \end{tabular}
% \end{center}
% Each of the eight files can be compiled directly by the \LaTeX{} compiler.
%
% %%%%%%%%%%%%%%%%%%%%%%%%%%%%%%%%%%%%%%
% \paragraph{Main File.}
%
% The main file is called |cdocsamp.tex|.
%
% Load the \textsf{childdoc} definitions and
% declare the filename for the main document:
%    \begin{macrocode}
\input{childdoc.def}
\childdocmain{}
%    \end{macrocode}

% Optional override for |\version| flag:
%    \begin{macrocode}
%%\ifchilddoc\else\providecommand{\version}{draft}\fi
%    \end{macrocode}

% Define the default values for the |\version| flag
% (|final| for the main file and |draft| for childs):
%    \begin{macrocode}
\ifchilddoc
\providecommand{\version}{draft}
\else
\providecommand{\version}{final}
\fi
%    \end{macrocode}

% Load the standard document class:
%    \begin{macrocode}
\documentclass[12pt]{article}
%    \end{macrocode}

% Start the document body:
%    \begin{macrocode}
\begin{document}
%    \end{macrocode}

% Declare a title page.
% Print title, part of document being processed and version flag:
%    \begin{macrocode}
\addtocounter{page}{-1}
\begin{center}
{\LARGE\bfseries{}childdoc example\par}
\vspace{1cm}
\ifchilddoc
\ifchilddocmanual part\else chapter\fi:
`\childdocname' of `\childdocjob'\par
\else
main document: `\childdocjob'\par
\fi
version: \version\par
\end{center}
\newpage
%    \end{macrocode}

% Manually include selected file,
% otherwise process as usual:
%    \begin{macrocode}
\ifchilddocmanual
\section*{part `\childdocname'}
\input{\childdocname}
\else
%    \end{macrocode}

% Include the two chapters:
%    \begin{macrocode}
\include{cdocsch1}
\include{cdocsch2}
%    \end{macrocode}

% Include the two parts unless only chapters should be displayed:
%    \begin{macrocode}
\ifchilddoc\else
\section{part three}
\input{cdocspt3}
\section{part four}
\input{cdocspt4}
\fi
%    \end{macrocode}

% Process as usual until here:
%    \begin{macrocode}
\fi
%    \end{macrocode}

% End of document body:
%    \begin{macrocode}
\end{document}
%    \end{macrocode}
%\iffalse
%</samplemain>
%\fi
%
% %%%%%%%%%%%%%%%%%%%%%%%%%%%%%%%%%%%%%%
% \paragraph{Chapter Include Files.}
%
% The include files are called |cdocsch1.tex| and |cdocsch2.tex|.
%
%\iffalse
%<*samplechap1|samplechap2>
%\fi

% Optional override for |\version| flag:
%    \begin{macrocode}
%%\providecommand{\version}{final}
%    \end{macrocode}

% Include the main document:
%    \begin{macrocode}
\input{childdoc.def}
\childdocof{cdocsamp}
%    \end{macrocode}

%\iffalse
%</samplechap1|samplechap2>
%\fi
%
%\iffalse
%<*samplechap1>
%\fi
% Some text for chapter 1:
%    \begin{macrocode}
\section{one}
some text in chapter one
%    \end{macrocode}

%\iffalse
%</samplechap1>
%\fi
% Some text for chapter 2:
%\iffalse
%<*samplechap2>
%\fi
%    \begin{macrocode}
\section{two}
more text in chapter two
%    \end{macrocode}

%\iffalse
%</samplechap2>
%\fi
%
% %%%%%%%%%%%%%%%%%%%%%%%%%%%%%%%%%%%%%%
% \paragraph{Part Include Files.}
%
% The include files are called |cdocspt3.tex| and |cdocspt4.tex|.
%
%\iffalse
%<*samplepart3|samplepart4>
%\fi

% Optional override for |\version| flag:
%    \begin{macrocode}
%%\providecommand{\version}{final}
%    \end{macrocode}

% Include the main document:
%    \begin{macrocode}
\input{childdoc.def}
\childdocby{cdocsamp}
%    \end{macrocode}

%\iffalse
%</samplepart3|samplepart4>
%\fi
%
%\iffalse
%<*samplepart3>
%\fi
% Some text for part 3:
%    \begin{macrocode}
some text in part three
%    \end{macrocode}

%\iffalse
%</samplepart3>
%\fi
% Some text for part 4:
%\iffalse
%<*samplepart4>
%\fi
%    \begin{macrocode}
more text in part four
%    \end{macrocode}

%\iffalse
%</samplepart4>
%\fi
%
% %%%%%%%%%%%%%%%%%%%%%%%%%%%%%%%%%%%%%%
% \paragraph{Forwarding for a Complete Draft.}
%
% The following forwarding file |cdocsdrf.tex|
% compiles the main document in draft mode:
%\iffalse
%<*sampledraft>
%\fi
%    \begin{macrocode}
\def\version{draft}
\input{childdoc.def}
\childdocforward{cdocsamp}
%    \end{macrocode}

%\iffalse
%</sampledraft>
%\fi
%
% %%%%%%%%%%%%%%%%%%%%%%%%%%%%%%%%%%%%%%
% \paragraph{Forwarding for Final Version of the Chapters.}
%
% The following forwarding files |cdocsfn1.tex| and |cdocsfn2.tex|
% (with identical content)
% compile the final versions of the child documents
% |cdocsch1.tex| and |cdocsch2.tex|, respectively:
%\iffalse
%<*samplefinal>
%\fi
%    \begin{macrocode}
\def\version{final}
\input{childdoc.def}
\childdocforwardprefix[cdocsamp]{cdocsfn}{cdocsch}
%    \end{macrocode}

%\iffalse
%</samplefinal>
%\fi
%
% %%%%%%%%%%%%%%%%%%%%%%%%%%%%%%%%%%%%%%
% \paragraph{Command Line Processing.}
%
% The following three command lines generate the output files
% |cdocscld|, |cdocscl1| and |cdocscl2|
% which should be identical to
% |cdocsdrf|, |cdocsch1| and |cdocsfn2|, respectively:
% \begin{center}
% \begin{tabular}{l}
% |latex -jobname cdocscld \|\\
% |  "\def\version{draft}\input{childdoc.def}\childdocforward{cdocsamp}"|\\
% |latex -jobname cdocscl1 \|\\
% |  "\input{childdoc.def}\childdocforward[cdocsamp]{cdocsch1}"|\\
% |latex -jobname cdocscl2 \|\\
% |  "\def\version{final}\input{childdoc.def}\childdocforward{cdocsch2}"|
% \end{tabular}
% \end{center}
% Note that the trailing backslash on each first line
% merely continues the input to the second line
% (for convenient cut ant paste).
% Furthermore, the command |latex| can be replaced by any
% of its alternative versions such as |pdflatex|.
%
% %%%%%%%%%%%%%%%%%%%%%%%%%%%%%%%%%%%%%%%%%%%%%%%%%%%%%%%%%%%%%%%%%%%%%%%%%%%%%%
% %%%%%%%%%%%%%%%%%%%%%%%%%%%%%%%%%%%%%%%%%%%%%%%%%%%%%%%%%%%%%%%%%%%%%%%%%%%%%%
% \section{Implementation}
%\iffalse
%<*package>
%\fi
%
% This section describes the definitions file |childdoc.def|.

% The definitions cannot be loaded using |\usepackage| or |\RequirePackage|
% which has a mechanism to prevent loading a style file more than once.
% When loading the definitions by means of |\input|
% multiple instances have to be prevented manually:
%\iffalse
%This code needs to be before the `\ProvidesFile' directive
%which is defined at the beginning of this file.
%Therefore it is also placed there and commented out here.
%</package>
%<*discard>
%\fi
%    \begin{macrocode}
\ifdefined\childdocmain\endinput\fi
%    \end{macrocode}
%\iffalse
%</discard>
%<*package>
%\fi
%
% \macro{\ifchilddoc}
% \macro{\ifchilddocmanual}
% The conditional |\ifchilddoc| tells whether a
% child (true) or main (false) document is being compiled.
% The conditional |\ifchilddocmanual| tells whether
% the |\includeonly| mechanism is used (false) or
% the selection of child files must be performed manually (true).
% The definitions initialise to false:
%    \begin{macrocode}
\newif\ifchilddoc
\newif\ifchilddocmanual
%    \end{macrocode}

% \macro{\childdocname}
% \macro{\childdocjob}
% The macro |\childdocname| stores the name of the main document
% to be compiled. The macro |\childdocjob| stores the name of
% the document on which the \LaTeX{} compiler was originally invoked.
% The content of |\jobname| cannot be compared
% to filenames specified in the source due to different catcodes.
% The following code rescans |\jobname|, stores the result
% in |\childdocname| and saves a copy in |\childdocjob|:
%    \begin{macrocode}
\edef\childdocname{\scantokens\expandafter{\jobname\noexpand}}
\let\childdocjob\childdocname
%    \end{macrocode}

% \macro{\childdocdisable}
% The macro |\childdocdisable| prevents the main file
% from being processed more than once.
% At this stage, the main document command |\childdocmain|
% is assumed to be called once again where it should do nothing.
% Any subsequent call to it should prevent
% a secondary processing of the main document
% It overwrites the forwarding commands
% |\childdocof| and |\childdocforward|
% with empty macros to prevent further inclusions of the main document:
%    \begin{macrocode}
\newcommand{\childdocdisable}
{
  \renewcommand{\childdocmain}[1]{\renewcommand{\childdocmain}[1]{\endinput}}
  \renewcommand{\childdocof}[1]{}
  \renewcommand{\childdocby}[2][]{}
  \renewcommand{\childdocforward}[2][]{}
  \renewcommand{\childdocdisable}{}
}
%    \end{macrocode}

% \macro{\childdocmain}
% The macro |\childdocmain| is to be called at the top of the main file
% with nothing or the main filename (without extension) as argument.
% First, it breaks loops.
% If the argument is not empty and does not match |\childdocname|
% (which is set by the first inclusion of |childdoc.def|),
% |\ifchilddoc| is set to true, |\includeonly| is applied to the child file
% and |\jobname| is set to the main file
% (for proper handling of |.aux| files):
%    \begin{macrocode}
\newcommand{\childdocmain}[1]
{
  \childdocdisable\childdocmain{}
  \if?#1?\else
    \begingroup
      \def\childdoctmp{#1}
      \ifx\childdoctmp\childdocname
        \def\childdoctmp{}
      \else
        \def\childdoctmp
        {
          \childdoctrue
          \includeonly{\childdocname}
          \def\childdocjob{#1}
          \def\jobname{#1}
        }
      \fi
      \expandafter
    \endgroup
    \childdoctmp
  \fi
}
%    \end{macrocode}

% \macro{\childdocof}
% The command |\childdocof| redirects
% compilation to the main file |#1|.
%    \begin{macrocode}
\newcommand{\childdocof}[1]
{
  \childdocdisable
  \childdoctrue
  \includeonly{\childdocname}
  \def\jobname{#1}
  \def\childdocjob{#1}
  \input{#1}
}
%    \end{macrocode}

% \macro{\childdocby}
% The command |\childdocby| ....
%    \begin{macrocode}
\newcommand{\childdocby}[2][]
{
  \childdocdisable
  \childdoctrue
  \childdocmanualtrue
  \if?#1?\else
    \def\jobname{#2}
  \fi
  \def\childdocjob{#2}
  \input{#2}
  \endinput
}
%    \end{macrocode}

% \macro{\childdocforward}
% The command |\childdocforward| redirects
% compilation to the main file or
% (if the optional argument is given) a child file.
% Parameters are set as if the main file
% or a child file starting with |\childdocof| was compiled.
% Then compilation is handed over to the main file:
%    \begin{macrocode}
\newcommand{\childdocforward}[2][]
{
  \begingroup
    \if?#1?
      \def\childdoctmp
      {
        \def\childdocname{#2}
        \def\childdocjob{#2}
        \def\jobname{#2}
        \input{#2}
        \endinput
      }
    \else
      \def\childdoctmp
      {
        \childdocdisable
        \def\childdocname{#2}
        \childdoctrue
        \includeonly{#2}
        \def\childdocjob{#1}
        \def\jobname{#1}
        \input{#1}
        \endinput
      }
    \fi
    \expandafter
  \endgroup
  \childdoctmp
}
%    \end{macrocode}

% \macro{\childdocforwardprefix}
% The command |\childdocforwardprefix| redirects
% compilation to the main or a child file by means of a pattern.
% The prefix |#1| in the current filename is replaced by |#2|
% and the suffix of the current filename is kept
% (it is assumed that the filename does not contain the substring `|~~~|'
% which is used as a delimiter).
% Compilation is handed over to the new file by |\childdocforward|:
%    \begin{macrocode}
\newcommand{\childdocforwardprefix}[3][]
{
  \begingroup
    \def\childdocextract #2##1~~~{\def\childdoctmp{\childdocforward[#1]{#3##1}}}
    \expandafter\childdocextract\childdocname~~~
    \expandafter
  \endgroup
  \childdoctmp
}
%    \end{macrocode}

% \macro{\childdoc}
% The deprecated macro |\childdoc| is a legacy version of |\childdocmain|:
%    \begin{macrocode}
\newcommand{\childdoc}{\childdocmain}
%    \end{macrocode}

% \macro{\childdocredirect}
% The deprecated macro |\childdocredirect| is a legacy version
% of |\childdocforward| and |\childdocforwardprefix|:
%    \begin{macrocode}
\newcommand{\childdocredirect}[2][]
{
  \begingroup
    \if?#1?
      \def\childdoctmp{\childdocforward{#2}}
    \else
      \def\childdoctmp{\childdocforwardprefix{#1}{#2}}
    \fi
    \expandafter
  \endgroup
  \childdoctmp
}
%    \end{macrocode}

%\iffalse
%</package>
%\fi
%
\endinput
|\\
|\childdocforwardprefix[|\textit{main}|]{|\textit{prefix}|}{|\textit{dest}|}|
\end{tabular}
\end{center}
%
the destination file is determined by a pattern
depending on the current file:
To make this work, the current file must be called
`{\textit{prefix}\hspace{0.2em}\textit{suffix}}'
with \textit{prefix} matching precisely the argument.
Processing is then passed on to the file
`{\textit{dest}\hspace{0.2em}\textit{suffix}}'.
Surely, the same effect is achieved by
directly specifying the
argument `{\textit{dest}\hspace{0.2em}\textit{suffix}}'
in the first form.
However, that requires to set up a different file
for each child. With the alternative form of the command
all these files can have exactly the same content
which simplifies setting them up and maintaining them.

For example, the following file |draft.tex|
with a compilation flag |\version| as described in \secref{sec:flags}
compiles the main document as a draft:
%
\begin{center}
\begin{tabular}{l}
|\def\version{draft}|\\
|% \iffalse
%
% childdoc.dtx Copyright (C) 2017-2018 Niklas Beisert
%
% This work may be distributed and/or modified under the
% conditions of the LaTeX Project Public License, either version 1.3
% of this license or (at your option) any later version.
% The latest version of this license is in
%   http://www.latex-project.org/lppl.txt
% and version 1.3 or later is part of all distributions of LaTeX
% version 2005/12/01 or later.
%
% This work has the LPPL maintenance status `maintained'.
%
% The Current Maintainer of this work is Niklas Beisert.
%
% This work consists of the files childdoc.dtx and childdoc.ins
% and the derived files childdoc.def and cdocsamp.tex with
% cdocsch1.tex, cdocsch2.tex, cdocsdrf.tex, cdocsfn1.tex, cdocsfn2.tex.
%
%<package>\ifdefined\childdocmain\endinput\fi
%<package>\ProvidesFile{childdoc.def}[2018/12/30 v2.0 child document driver]
%<samplemain>\ProvidesFile{cdocsamp.tex}[2018/12/30 v2.0 sample for childdoc]
%<*driver>
%\ProvidesFile{childdoc.drv}[2018/12/30 v2.0 childdoc reference manual file]
\PassOptionsToClass{10pt,a4paper}{article}
\documentclass{ltxdoc}

\usepackage[margin=35mm]{geometry}
\usepackage{hyperref}
\usepackage{hyperxmp}
\usepackage[usenames]{color}

\hypersetup{colorlinks=true}
\hypersetup{pdfstartview=FitH}
\hypersetup{pdfpagemode=UseNone}
\hypersetup{pdfsource={}}
\hypersetup{pdflang={en-UK}}
\hypersetup{pdfcopyright={Copyright 2017-2018 Niklas Beisert.
  This work may be distributed and/or modified under the
  conditions of the LaTeX Project Public License, either version 1.3
  of this license or (at your option) any later version.}}
\hypersetup{pdflicenseurl={http://www.latex-project.org/lppl.txt}}
\hypersetup{pdfcontactaddress={ETH Zurich, ITP, HIT K,
  Wolfgang-Pauli-Strasse 27}}
\hypersetup{pdfcontactpostcode={8093}}
\hypersetup{pdfcontactcity={Zurich}}
\hypersetup{pdfcontactcountry={Switzerland}}
\hypersetup{pdfcontactemail={nbeisert@itp.phys.ethz.ch}}
\hypersetup{pdfcontacturl={http://people.phys.ethz.ch/\xmptilde nbeisert/}}

\newcommand{\secref}[1]{\hyperref[#1]{section \ref*{#1}}}

\parskip1ex
\parindent0pt
\let\olditemize\itemize
\def\itemize{\olditemize\parskip0pt}

\begin{document}

\title{The \textsf{childdoc} Package}
\hypersetup{pdftitle={The childdoc Package}}
\author{Niklas Beisert\\[2ex]
  Institut f\"ur Theoretische Physik\\
  Eidgen\"ossische Technische Hochschule Z\"urich\\
  Wolfgang-Pauli-Strasse 27, 8093 Z\"urich, Switzerland\\[1ex]
  \href{mailto:nbeisert@itp.phys.ethz.ch}
  {\texttt{nbeisert@itp.phys.ethz.ch}}}
\hypersetup{pdfauthor={Niklas Beisert}}
\hypersetup{pdfsubject={Manual for the LaTeX2e Package childdoc}}
\date{30 December 2018, \textsf{v2.0}}
\maketitle

\begin{abstract}\noindent
\textsf{childdoc} is a \LaTeXe{} package
that enables the direct compilation
of document sections included by |\include|
to individual files.
\end{abstract}

\begingroup
\parskip0ex
\tableofcontents
\endgroup

%%%%%%%%%%%%%%%%%%%%%%%%%%%%%%%%%%%%%%%%%%%%%%%%%%%%%%%%%%%%%%%%%%%%%%%%%%%%%%%%
%%%%%%%%%%%%%%%%%%%%%%%%%%%%%%%%%%%%%%%%%%%%%%%%%%%%%%%%%%%%%%%%%%%%%%%%%%%%%%%%
\section{Introduction}

\LaTeX{} provides a mechanism to structure a large document (such as a book)
into a main file and several child files (containing the chapters)
using the |\include| command.
This mechanism is beneficial for documents
which span hundreds of pages in order to
make the source file(s) more manageable.
Moreover, compilation can be restricted to
selected child files by means of the |\includeonly| command.
The latter feature can be used to reduce the compilation time while editing
(this was significantly more useful in the earlier days of \LaTeX{})
or to generate a smaller document which is easier to navigate.
Another application of |\includeonly| is to generate
documents consisting of selected parts of the complete document.

However, there are a few drawbacks of the plain |\include| mechanism:
\begin{itemize}
\item
The child files cannot be compiled on their own,
they can only be compiled via the main file.
A naive editing environment
(such as a text editor with an option
to have the current file processed by \LaTeX)
may require one to switch to the main file before compiling;
attempting to compile the child file produces errors.
\item
The main file must be modified (each time)
to adjust the |\includeonly| command
to the present needs. This easily leaves the main file in a messy state.
\item
The generated document will always carry the filename
of the main document. This is inconvenient if
several child files are to be compiled and
to be kept for distribution.
\end{itemize}

The present package provides a simple interface
to make child files individually compilable by \LaTeX{}.
Compiling a child file then has the same effect as compiling
the main file with an |\includeonly| command
to select the appropriate child.
Moreover the generated document will carry the name of the child
rather than the main file.
This resolves all three above issues.

This feature is meant to make the editing of books,
thesis documents and lecture notes somewhat more convenient.
However, the package can also be used efficiently for
composing a series of documents (such as exercise sheets)
which are typically distributed individually.
It then assists the author in generating the individual documents
(potentially in different versions)
as well as a document containing the collected series.
Another application is in developing style files
or other kinds of included material
where compilation of the style file could redirect
to a sample or test file.

%%%%%%%%%%%%%%%%%%%%%%%%%%%%%%%%%%%%%%%%%%%%%%%%%%%%%%%%%%%%%%%%%%%%%%%%%%%%%%%%
%%%%%%%%%%%%%%%%%%%%%%%%%%%%%%%%%%%%%%%%%%%%%%%%%%%%%%%%%%%%%%%%%%%%%%%%%%%%%%%%
\section{Usage}

First of all, the package \textsf{childdoc} is \emph{not} a standard
\LaTeXe{} |.sty| style file! Therefore it needs to be invoked in
a non-standard way.

%%%%%%%%%%%%%%%%%%%%%%%%%%%%%%%%%%%%%%%%%%%%%%%%%%%%%%%%%%%%%%%%%%%%%%%%%%%%%%%%
\subsection{Included Files}
\label{sec:include}

%%%%%%%%%%%%%%%%%%%%%%%%%%%%%%%%%%%%%%%%
\DescribeMacro{\childdocmain}
To use the package, add the commands
\begin{center}
\begin{tabular}{l}
|\input{childdoc.def}|\\
|\childdocmain{}|\\
\end{tabular}
\end{center}
at the very top of the main \LaTeX{} file,
in particular \emph{before} the |\documentclass| statement!
The argument of |\childdocmain| should be left empty
(but it must be present).

%%%%%%%%%%%%%%%%%%%%%%%%%%%%%%%%%%%%%%%%
\DescribeMacro{\childdocof}
Furthermore, add the commands
\begin{center}
\begin{tabular}{l}
|\input{childdoc.def}|\\
|\childdocof{|\textit{main}|}|\\
\end{tabular}
\end{center}
at the top of every child file \textit{child}
which is included by |\include{|\textit{child}|}|
from within the main file
(or at least for those files to be compiled individually).
The argument \textit{main} must be the filename of the main file.

There are a couple of
considerations in setting up the main and child documents:

%%%%%%%%%%%%%%%%%%%%%%%%%%%%%%%%%%%%%%%%
\paragraph{Restrictions.}

Please note the following restrictions:
\begin{itemize}
\item
|\childdocmain| must be called with one argument \textit{main}
to ensure compatibility with earlier version of the package.
It must either be empty (|\childdocmain{}|)
or precisely match the filename of the main file in which it is specified.
See \secref{sec:detection} for further information.
\item
The filename \textit{main} must be specified without the |.tex| extension.
\item
The filename \textit{main} is case sensitive
(even in case-insensitive file systems)
due to internal string comparison.
\item
The argument \textit{main} should be fully expanded, it cannot be a macro.
\item
Subdirectories and special characters should be avoided in filenames.
\item
The command |\childdocmain{|\textit{main}|}| must be followed by a whitespace.
It should not be followed immediately by another command
or by a comment mark `|%|'.
This is because the \TeX{} parser reads the token immediately following
the argument of |\childdocmain| and puts it
at the beginning of every child section;
however, a white\-space is ignored.
\end{itemize}

%%%%%%%%%%%%%%%%%%%%%%%%%%%%%%%%%%%%%%%%
\paragraph{Content of Main File.}

It is advisable to place all content in the child files included by |\include|.
Any output contained in the main file will appear in all child documents
unless suppressed manually;
it cannot be suppressed automatically by the |\includeonly| directive
and thus should normally be avoided.
A method to include some content in the main file
by means of conditional processing is described in \secref{sec:conditional}.

%%%%%%%%%%%%%%%%%%%%%%%%%%%%%%%%%%%%%%%%
\paragraph{Page Numbering.}

When only a part of the document is compiled,
the appropriate numbering of pages
(as well as other status parameters)
is determined from the |.aux| files.
The latter contain information from previous passes.
However this information needs to propagate through
all intermediate child documents.
Therefore the page numbering in child documents may well
be inconsistent until the complete document is compiled at least once.

A useful (if unconventional) way to always ensure a consistent
page numbering is to restart the numbering in each child document
and denote the pages by `\textit{child}|.|\textit{page}'
where \textit{child} represents the chapter/section number of the child file.
This can be achieved by the command
|\numberwithin{page}{|\textit{child}|}|
of the \textsf{amsmath} package
where \textit{child} can be |chapter| or |section|
depending on the chosen structuring.
Alternatively, one can modify the macro |\thepage| appropriately
and reset the counter |page| at the start of each child file.

%%%%%%%%%%%%%%%%%%%%%%%%%%%%%%%%%%%%%%%%%%%%%%%%%%%%%%%%%%%%%%%%%%%%%%%%%%%%%%%%
\subsection{Conditional Processing}
\label{sec:conditional}

The package provides a mechanism to compile different versions
of a document. To customise the versions further some conditional processing
can come in handy to distinguish which version is being compiled.
The package provides two macros to describe the compilation context:

%%%%%%%%%%%%%%%%%%%%%%%%%%%%%%%%%%%%%%%%
\DescribeMacro{\ifchilddoc}
The conditional |\ifchilddoc| distinguishes between the compilation of
child documents and the main document:
%
\begin{center}
|\ifchilddoc |\textit{child-code}| |[|\||else |\textit{main-code}]| \||fi|
\end{center}

%%%%%%%%%%%%%%%%%%%%%%%%%%%%%%%%%%%%%%%%
\DescribeMacro{\childdocname}
\DescribeMacro{\childdocjob}
The macro |\childdocname| contains the filename (without extension)
of the main or child file being processed.
Note that |\childdocjob| will always contain the name of the main file.

%%%%%%%%%%%%%%%%%%%%%%%%%%%%%%%%%%%%%%%%
\paragraph{Title Page.}

Conditional processing can be used to include a title or banner page
in the main document when proper precautions are taken.
Importantly, the code in the main file should ensure that the page counter
(as well as other status parameters which are stored in the |.aux| files)
takes the same value after the conditional processing.
Otherwise the page numbers may take divergent values
depending on which part is compiled.

For example, a title page could be declared by:
%
\begin{center}
\begin{tabular}{l}
|\ifchilddoc\||else|\\
|\addtocounter{page}{-1}|\\
\textit{code for title page}\\
|\newpage|\\
|\||fi|
\end{tabular}
\end{center}
%
A banner page for the child documents can be generated by:
%
\begin{center}
\begin{tabular}{l}
|\ifchilddoc|\\
|\addtocounter{page}{-1}|\\
\textit{code for banner page}\\
|\newpage|\\
|\||fi|
\end{tabular}
\end{center}
%
Here one could write a message such as:
\begin{center}
|This is the part \childdocname{} of \childdocjob{}.|
\end{center}

%%%%%%%%%%%%%%%%%%%%%%%%%%%%%%%%%%%%%%%%%%%%%%%%%%%%%%%%%%%%%%%%%%%%%%%%%%%%%%%%
\subsection{Flags}
\label{sec:flags}

The package makes it easy to generate different versions
of the main or child documents.
To this end compilation flags can be defined
and assigned different default values.
They will be particularly useful in conjunction
with the forwarding mechanism described in \secref{sec:forward}.

For example, it may be useful to have a flag |\version|
which can be set to |draft| or |final|.
The document source will contain some conditional code
depending on the value of |\version|.
Suppose further, the flag should default to |final| for the main file
and to |draft| for child files
which is a natural assignment for editing the document.
This is achieved by placing the following code
in the preamble of the main document
(below the |\childdocmain| directive):
%
\begin{center}
\begin{tabular}{l}
|\ifchilddoc|\\
|\providecommand{\version}{draft}|\\
|\||else|\\
|\providecommand{\version}{final}|\\
|\||fi|
\end{tabular}
\end{center}
%
The definition by |\providecommand| makes sure
that previous definitions are not overwritten.
Further statements |\providecommand{\version}{...}|
can thus be added before the above code to override it.

For the main file, one might add a line
(between |\childdocmain| and the above block)
%
\begin{center}
|%\ifchilddoc\||else\providecommand{\version}{draft}\||fi|
\end{center}
%
which can be uncommented to produce a draft version.
Likewise one can add a line to the very top of a child file
(above the |\childdocof{|\textit{main}|}| directive)
%
\begin{center}
|%\providecommand{\version}{final}|
\end{center}
%
which can be uncommented to produce the final version of this child document.

%%%%%%%%%%%%%%%%%%%%%%%%%%%%%%%%%%%%%%%%%%%%%%%%%%%%%%%%%%%%%%%%%%%%%%%%%%%%%%%%
\subsection{Forwarding}
\label{sec:forward}

Different versions of the main or child documents
using compilation flags as described in \secref{sec:flags}
can be (permanently) stored in different files
for convenient compilation, viewing and distribution.
To this end, the package defines a command
to pass on compilation to a different file:

%%%%%%%%%%%%%%%%%%%%%%%%%%%%%%%%%%%%%%%%
\DescribeMacro{\childdocforward}
The command |\childdocforward| redirects processing to
another source file:
%
\begin{center}
\begin{tabular}{l}
|\input{childdoc.def}|\\
|\childdocforward[|\textit{main}|]{|\textit{dest}|}|\\
\end{tabular}
\end{center}
%
The argument \textit{dest} is the destination file
(without extension).
It should be the main file or one of the child files.
Note that further \textsf{childdoc} directives
such as |\childdocof| and |\childdocforward|
in the indicated file will be processed in this form.
The optional argument \textit{main}
passes on directly to the main file \textit{main}
while pretending to compile the child \textit{dest}.
This form behaves as if \textit{dest}
issues |\childdocof{|\textit{main}|}| right away,
and no further \textsf{childdoc} directives will be processed.

%%%%%%%%%%%%%%%%%%%%%%%%%%%%%%%%%%%%%%%%
\DescribeMacro{\...prefix}
In the alternative form |\childdocforwardprefix|,
%
\begin{center}
\begin{tabular}{l}
|\input{childdoc.def}|\\
|\childdocforwardprefix[|\textit{main}|]{|\textit{prefix}|}{|\textit{dest}|}|
\end{tabular}
\end{center}
%
the destination file is determined by a pattern
depending on the current file:
To make this work, the current file must be called
`{\textit{prefix}\hspace{0.2em}\textit{suffix}}'
with \textit{prefix} matching precisely the argument.
Processing is then passed on to the file
`{\textit{dest}\hspace{0.2em}\textit{suffix}}'.
Surely, the same effect is achieved by
directly specifying the
argument `{\textit{dest}\hspace{0.2em}\textit{suffix}}'
in the first form.
However, that requires to set up a different file
for each child. With the alternative form of the command
all these files can have exactly the same content
which simplifies setting them up and maintaining them.

For example, the following file |draft.tex|
with a compilation flag |\version| as described in \secref{sec:flags}
compiles the main document as a draft:
%
\begin{center}
\begin{tabular}{l}
|\def\version{draft}|\\
|\input{childdoc.def}|\\
|\childdocforward{|\textit{main}|}|
\end{tabular}
\end{center}
%
Likewise, the following files |final|\textit{nn}|.tex|
compile the final version of the child document
|child|\textit{nn}|.tex|:
%
\begin{center}
\begin{tabular}{l}
|\def\version{final}|\\
|\input{childdoc.def}|\\
|\childdocforwardprefix{final}{child}|
\end{tabular}
\end{center}
%

Note that when several versions of a main file and/or of each child file
are to be generated, it may be convenient to set up a |Makefile| or
shell script to automatise the process.

%%%%%%%%%%%%%%%%%%%%%%%%%%%%%%%%%%%%%%%%%%%%%%%%%%%%%%%%%%%%%%%%%%%%%%%%%%%%%%%%
\subsection{Command Line Processing}
\label{sec:commandline}

The effect of redirection files can also be achieved by invoking
the \LaTeX{} compiler with a more elaborate command line.
Most conveniently this should be done as part
of a shell script or a |Makefile|.

When using \textsf{childdoc} in the main file, the following
command lines effectively perform a redirection
(note that depending on the shell being used,
backslashes may have to be doubled: `|\|' $\to$ `|\\|'):
%
\begin{center}
|... -jobname "|\textit{target}|" |\\|"|[\textit{flags}]%
|\input{childdoc.def}\childdocforward[|\textit{main}|]{|\textit{dest}|}"|
\end{center}
%
Here \textit{target} is the name of the output file,
\textit{main} is the name of the main file
and \textit{dest} is the name of the main or child file to be processed
(all filenames without extensions).
The optional argument \textit{main} can be omitted
if \textit{main} matches \textit{dest}.
Optionally, compilation \textit{flags} can be defined via |\def| commands.
This command line makes the \TeX{} engine believe
it is compiling the file \textit{target}
whose content is specified as the latter parameter.
The provided code then forwards the processing to
\textit{main} or \textit{dest} as described in \secref{sec:forward}.

%%%%%%%%%%%%%%%%%%%%%%%%%%%%%%%%%%%%%%%%%%%%%%%%%%%%%%%%%%%%%%%%%%%%%%%%%%%%%%%%
\subsection{Include by Input}
\label{sec:input}

Including child documents by |\include| has some restrictions by design.
Most notably, the content of a child document always occupies
its own set of pages; pages cannot be shared between child documents.
Usually, this behaviour makes perfect sense
because each child document contain an essential part of the document.
However, in some situations it may be desirable to compose
a document from a collection of parts
without having mandatory page breaks between then.
For this case, the package
provides a mechanism to include parts
by |\input| which can also be processed individually.
However, by construction this mechanism
requires manual handling of the content to be output.

%%%%%%%%%%%%%%%%%%%%%%%%%%%%%%%%%%%%%%%%
\DescribeMacro{\ifchilddocmanual}
The main file should be prepared as usual, see \secref{sec:include}.
However, the document body must make a distinction
between processing of an individual part and of the main document, e.g.:
%
\begin{center}
\begin{tabular}{l}
|\ifchilddocmanual|\\
|\input{\childdocname}|\\
|\||else|\\
\textit{document body with }|\input{|\textit{part}|}|\\
|\||fi|
\end{tabular}
\end{center}
%
The conditional |\ifchilddocmanual| is true whenever
a part to be included by |\input| is being compiled,
and the name of the part is stored in |\childdocname|.

%%%%%%%%%%%%%%%%%%%%%%%%%%%%%%%%%%%%%%%%
\DescribeMacro{\childdocby}
Each part to be included by |\input| should start with:
%
\begin{center}
\begin{tabular}{l}
|\input{childdoc.def}|\\
|\childdocby{|\textit{main}|}|\\
\end{tabular}
\end{center}
%
The directive |\childdocby| is similar to |\childdocof|
described in \secref{sec:include},
but the subsequent selection of content must be done manually.
To that end, both |\ifchilddoc| and |\ifchilddocmanual|
will be true upon processing of a part,
and the name of the part is stored in |\childdocname|.
Note that |\jobname| will be set to the filename of the current part
so that each part receives an individual |.aux| file
that does not interfere with the |.aux| file(s) of the main document.
This behaviour can be altered by the alternative form
|\childdocby[*]{|\textit{main}|}| (with a non-empty optional argument)
which uses the |.aux| file of the main document
by setting |\jobname| to \textit{main}.

%%%%%%%%%%%%%%%%%%%%%%%%%%%%%%%%%%%%%%%%%%%%%%%%%%%%%%%%%%%%%%%%%%%%%%%%%%%%%%%%
\subsection{Driver Development}
\label{sec:driver}

The \textsf{childdoc} mechanism can also be use for the development
of definition files such as \LaTeX{} styles or classes.
This case differs from the above setup with multiple parts
included by |\include| in that no |\includeonly| should be invoked.
This can be achieved by starting the include file
(before |\ProvidesPackage|) with:
%
\begin{center}
\begin{tabular}{l}
|\input{childdoc.def}|\\
|\childdocforward{|\textit{main}|}|\\
\end{tabular}
\end{center}
%
or alternatively with:
%
\begin{center}
\begin{tabular}{l}
|\input{childdoc.def}|\\
|\childdocby{|\textit{main}|}|\\
\end{tabular}
\end{center}
%
Both forms have slightly different effects as described above.
The main file is prepared as usual, see \secref{sec:include}.

%%%%%%%%%%%%%%%%%%%%%%%%%%%%%%%%%%%%%%%%%%%%%%%%%%%%%%%%%%%%%%%%%%%%%%%%%%%%%%%%
\subsection{Legacy Detection}
\label{sec:detection}

The directive |\childdocmain| in the main file can detect
whether the complete document or merely a child is to be compiled
even without using the directive |\childdocof|.
This method is deprecated because it is less robust
and there is no compelling reason to use it;
it is merely provided for backward compatibility
and it may be removed in future versions.

If the detection mechanism is to be used,
it is mandatory to correctly specify
the filename of the main file as the argument of |\childdocmain|:
%
\begin{center}
\begin{tabular}{l}
|\input{childdoc.def}|\\
|\childdocmain{|\textit{main}|}|\\
\end{tabular}
\end{center}
%
If |\jobname| does not match the argument \textit{main} of |\childdocmain|,
it is assumed that |\jobname| points to the child file to be compiled.
When using |\childdocmain| with the main file specified as argument,
it suffices to start a child file
with just |\input{|\textit{main}|}|
without loading of the package and using |\childdocof|.
If instead all processing is done
with the appropriate \textsf{childdoc} directives,
the argument of \textit{main} of |\childdocmain| can be empty.

An alternative version of the command line processing described
in \secref{sec:commandline} using the detection mechanism reads:
%
\begin{center}
|... -jobname "|\textit{target}|" "|[\textit{flags}]%
[|\def\jobname{|\textit{dest}|}|]|\input{|\textit{main}|}"|
\end{center}

%%%%%%%%%%%%%%%%%%%%%%%%%%%%%%%%%%%%%%%%%%%%%%%%%%%%%%%%%%%%%%%%%%%%%%%%%%%%%%%%
\subsection{Manual Code}
\label{sec:manual}

In case one cannot be certain whether the definitions file |childdoc.def|
is installed on the target \TeX{} distribution
and one prefers not to ship it,
it is conceivable to paste a few relevant commands into the sources.

To that end, drop all statements |\input{childdoc.def}|
and perform the replacements as outlined below.
Instead of |\childdocmain{|\textit{main}|}| add the following code
to the top of the main file:
%
\begin{center}
\begin{tabular}{l}
|\||ifdefined\childdocname\endinput\||fi\newif\ifchilddoc|\\
|\edef\childdocname{\scantokens\expandafter{\jobname\noexpand}}|\\
|\def\childdocmain{|\textit{main}|}\||ifx\childdocmain\childdocname\||else|\\
|\childdoctrue\includeonly{\childdocname}\let\jobname\childdocmain\||fi|\\
\end{tabular}
\end{center}
%
Instead of |\childdocof{|\textit{main}|}| just include the main file
at the top of each child file:
%
\begin{center}
|\input{|\textit{main}|}|
\end{center}
%
A simple redirection |\childdocforward{|\textit{dest}|}| is achieved by:
%
\begin{center}
|\def\jobname{|\textit{dest}|}\input{\jobname}|
\end{center}
%
The redirection with prefix
|\childdocforwardprefix[|\textit{prefix}|]{|\textit{dest}|}|
is accomplished by:
%
\begin{center}
\begin{tabular}{l}
|{\edef\jobname{\scantokens\expandafter{\jobname\noexpand}}|\\
|\def\redirectjob |\textit{prefix}|#1~~~{\gdef\jobname{|\textit{dest}|#1}}|\\
|\expandafter\redirectjob\jobname~~~}\input{\jobname}|
\end{tabular}
\end{center}

In an alternative approach,
child documents can be compiled by a specific command line
without additional code or specific definitions:
%
\begin{center}
|... -jobname "|\textit{target}|" "|[\textit{flags}]%
|\includeonly{|\textit{dest}|}\input{|\textit{main}|}"|
\end{center}
%

%%%%%%%%%%%%%%%%%%%%%%%%%%%%%%%%%%%%%%%%%%%%%%%%%%%%%%%%%%%%%%%%%%%%%%%%%%%%%%%%
%%%%%%%%%%%%%%%%%%%%%%%%%%%%%%%%%%%%%%%%%%%%%%%%%%%%%%%%%%%%%%%%%%%%%%%%%%%%%%%%
\section{Information}

%%%%%%%%%%%%%%%%%%%%%%%%%%%%%%%%%%%%%%%%%%%%%%%%%%%%%%%%%%%%%%%%%%%%%%%%%%%%%%%%
\subsection{Copyright}

Copyright \copyright{} 2017--2018 Niklas Beisert

This work may be distributed and/or modified under the
conditions of the \LaTeX{} Project Public License, either version 1.3
of this license or (at your option) any later version.
The latest version of this license is in
  \url{http://www.latex-project.org/lppl.txt}
and version 1.3 or later is part of all distributions of \LaTeX{}
version 2005/12/01 or later.

This work has the LPPL maintenance status `maintained'.

The Current Maintainer of this work is Niklas Beisert.

This work consists of the files |README.txt|, |childdoc.ins| and |childdoc.dtx|
as well as the derived files |childdoc.def|, |cdocsamp.tex|
with |cdocsch1.tex|, |cdocsch2.tex|, |cdocspt3.tex|, |cdocspt4.tex|,
|cdocsdrf.tex|, |cdocsfn1.tex|, |cdocsfn2.tex|
as well as |childdoc.pdf|.

%%%%%%%%%%%%%%%%%%%%%%%%%%%%%%%%%%%%%%%%%%%%%%%%%%%%%%%%%%%%%%%%%%%%%%%%%%%%%%%%
\subsection{Files and Installation}

The package consists of the files:
%
\begin{center}
\begin{tabular}{ll}
    |README.txt|   & readme file \\
    |childdoc.ins| & installation file \\
    |childdoc.dtx| & source file \\
    |childdoc.def| & definition file \\
    |cdocsamp.tex| & sample main file \\
    |cdocsch1.tex| & sample include file \\
    |cdocsch2.tex| & sample include file \\
    |cdocspt3.tex| & sample part file \\
    |cdocspt4.tex| & sample part file \\
    |cdocsdrf.tex| & sample redirection file \\
    |cdocsfn1.tex| & sample redirection file \\
    |cdocsfn2.tex| & sample redirection file \\
    |childdoc.pdf| & manual
\end{tabular}
\end{center}
%
The distribution consists of the files
|README.txt|, |childdoc.ins| and |childdoc.dtx|.
%
\begin{itemize}
\item
Run (pdf)\LaTeX{} on |childdoc.dtx|
to compile the manual |childdoc.pdf| (this file).
\item
Run \LaTeX{} on |childdoc.ins| to create the definitions file |childdoc.def|
and the sample |cdocsamp.tex| with include files
|cdocsch1.tex|, |cdocsch2.tex|, |cdocspt3.tex|, |cdocspt4.tex|,
|cdocsdrf.tex|, |cdocsfn1.tex|, |cdocsfn2.tex|.
Then copy the file |childdoc.def| to an appropriate directory of your \LaTeX{}
distribution, e.g.\ \textit{texmf-root}|/tex/latex/childdoc|.
\end{itemize}

%%%%%%%%%%%%%%%%%%%%%%%%%%%%%%%%%%%%%%%%%%%%%%%%%%%%%%%%%%%%%%%%%%%%%%%%%%%%%%%%
\subsection{Related CTAN Packages}

There are several other packages which offer a similar functionality:
%
\begin{itemize}
\item
The packages
\href{http://ctan.org/pkg/docmute}{\textsf{docmute}},
\href{http://ctan.org/pkg/includex}{\textsf{includex}} and
\href{http://ctan.org/pkg/standalone}{\textsf{standalone}}
provide commands to include only the document body of
a child file thus allowing both files to be compiled individually.
\item
The packages \href{http://ctan.org/pkg/subdocs}{\textsf{subdocs}}
and \href{http://ctan.org/pkg/subfiles}{\textsf{subfiles}}
provide structures in which the main and child documents can be
encapsulated and allowing them to be compiled individually.
The inclusion mechanism is different from the conventional |\include|.
\item
The package \href{http://ctan.org/pkg/combine}{\textsf{combine}}
is an elaborate solution to combine several documents into one.
\end{itemize}
%
See also the CTAN topic \href{http://ctan.org/topic/subdocs}{\textsf{subdocs}}
for further related packages.
The present package differs from the above solutions in that
a document structure constructed with the conventional |\include| mechanism
just needs two extra commands at the top of every file
such that all constituent files can be compiled individually.

%%%%%%%%%%%%%%%%%%%%%%%%%%%%%%%%%%%%%%%%%%%%%%%%%%%%%%%%%%%%%%%%%%%%%%%%%%%%%%%%
%\subsection{Feature Suggestions}
%
%The following is a list of features which may be useful for future
%versions of this package:
%%
%\begin{itemize}
%\item
%\ldots
%\end{itemize}

%%%%%%%%%%%%%%%%%%%%%%%%%%%%%%%%%%%%%%%%%%%%%%%%%%%%%%%%%%%%%%%%%%%%%%%%%%%%%%%%
\subsection{Revision History}

%%%%%%%%%%%%%%%%%%%%%%%%%%%%%%%%%%%%%%%%
\paragraph{v2.0:} 2018/12/30

\begin{itemize}
\item
immediate forward processing
\item
added |\childdocby| mechanism
\item
manual restructured
\end{itemize}

%%%%%%%%%%%%%%%%%%%%%%%%%%%%%%%%%%%%%%%%
\paragraph{v1.6:} 2018/01/17

\begin{itemize}
\item
application for development of include files
\item
corrections to manual
\end{itemize}

%%%%%%%%%%%%%%%%%%%%%%%%%%%%%%%%%%%%%%%%
\paragraph{v1.5:} 2017/05/21

\begin{itemize}
\item
more complete structuring introduced
\item
|\childdocof| introduced
\item
|\childdoc| renamed to |\childdocmain|
\item
|\childredirect| renamed to |\childdocforward| and |\childdocforwardprefix|
and functionality expanded
\end{itemize}

%%%%%%%%%%%%%%%%%%%%%%%%%%%%%%%%%%%%%%%%
\paragraph{v1.0:} 2017/04/27

\begin{itemize}
\item
manual and install package
\item
first version published on CTAN
\end{itemize}

%%%%%%%%%%%%%%%%%%%%%%%%%%%%%%%%%%%%%%%%
\paragraph{v0.6:} 2017/04/26

\begin{itemize}
\item
redirection mechanism added
\end{itemize}

%%%%%%%%%%%%%%%%%%%%%%%%%%%%%%%%%%%%%%%%
\paragraph{v0.5:} 2017/04/26

\begin{itemize}
\item
functionality in definition file
\end{itemize}


%%%%%%%%%%%%%%%%%%%%%%%%%%%%%%%%%%%%%%%%%%%%%%%%%%%%%%%%%%%%%%%%%%%%%%%%%%%%%%%%
%%%%%%%%%%%%%%%%%%%%%%%%%%%%%%%%%%%%%%%%%%%%%%%%%%%%%%%%%%%%%%%%%%%%%%%%%%%%%%%%
%%%%%%%%%%%%%%%%%%%%%%%%%%%%%%%%%%%%%%%%%%%%%%%%%%%%%%%%%%%%%%%%%%%%%%%%%%%%%%%%
\appendix

\settowidth\MacroIndent{\rmfamily\scriptsize 000\ }

 \DocInput{childdoc.dtx}

\end{document}
%</driver>
% \fi
%
% %%%%%%%%%%%%%%%%%%%%%%%%%%%%%%%%%%%%%%%%%%%%%%%%%%%%%%%%%%%%%%%%%%%%%%%%%%%%%%
% %%%%%%%%%%%%%%%%%%%%%%%%%%%%%%%%%%%%%%%%%%%%%%%%%%%%%%%%%%%%%%%%%%%%%%%%%%%%%%
% \section{Sample}
%\iffalse
%<*samplemain>
%\fi
%
% The following presents a sample document
% with two chapters, two parts, a title page,
% a compile flag as well as three forwarding files to set the flag.
% It consists of eight |.tex| files:
% \begin{center}
% \begin{tabular}{ll}
% |cdocsamp.tex|&main file\\
% |cdocsch1.tex|&include file for chapter 1\\
% |cdocsch2.tex|&include file for chapter 2\\
% |cdocspt3.tex|&include file for part 3\\
% |cdocspt4.tex|&include file for part 4\\
% |cdocsdrf.tex|&forwarding file for main file in draft mode\\
% |cdocsfi1.tex|&forwarding file for final version of chapter 1\\
% |cdocsfi2.tex|&forwarding file for final version of chapter 2\\
% \end{tabular}
% \end{center}
% Each of the eight files can be compiled directly by the \LaTeX{} compiler.
%
% %%%%%%%%%%%%%%%%%%%%%%%%%%%%%%%%%%%%%%
% \paragraph{Main File.}
%
% The main file is called |cdocsamp.tex|.
%
% Load the \textsf{childdoc} definitions and
% declare the filename for the main document:
%    \begin{macrocode}
\input{childdoc.def}
\childdocmain{}
%    \end{macrocode}

% Optional override for |\version| flag:
%    \begin{macrocode}
%%\ifchilddoc\else\providecommand{\version}{draft}\fi
%    \end{macrocode}

% Define the default values for the |\version| flag
% (|final| for the main file and |draft| for childs):
%    \begin{macrocode}
\ifchilddoc
\providecommand{\version}{draft}
\else
\providecommand{\version}{final}
\fi
%    \end{macrocode}

% Load the standard document class:
%    \begin{macrocode}
\documentclass[12pt]{article}
%    \end{macrocode}

% Start the document body:
%    \begin{macrocode}
\begin{document}
%    \end{macrocode}

% Declare a title page.
% Print title, part of document being processed and version flag:
%    \begin{macrocode}
\addtocounter{page}{-1}
\begin{center}
{\LARGE\bfseries{}childdoc example\par}
\vspace{1cm}
\ifchilddoc
\ifchilddocmanual part\else chapter\fi:
`\childdocname' of `\childdocjob'\par
\else
main document: `\childdocjob'\par
\fi
version: \version\par
\end{center}
\newpage
%    \end{macrocode}

% Manually include selected file,
% otherwise process as usual:
%    \begin{macrocode}
\ifchilddocmanual
\section*{part `\childdocname'}
\input{\childdocname}
\else
%    \end{macrocode}

% Include the two chapters:
%    \begin{macrocode}
\include{cdocsch1}
\include{cdocsch2}
%    \end{macrocode}

% Include the two parts unless only chapters should be displayed:
%    \begin{macrocode}
\ifchilddoc\else
\section{part three}
\input{cdocspt3}
\section{part four}
\input{cdocspt4}
\fi
%    \end{macrocode}

% Process as usual until here:
%    \begin{macrocode}
\fi
%    \end{macrocode}

% End of document body:
%    \begin{macrocode}
\end{document}
%    \end{macrocode}
%\iffalse
%</samplemain>
%\fi
%
% %%%%%%%%%%%%%%%%%%%%%%%%%%%%%%%%%%%%%%
% \paragraph{Chapter Include Files.}
%
% The include files are called |cdocsch1.tex| and |cdocsch2.tex|.
%
%\iffalse
%<*samplechap1|samplechap2>
%\fi

% Optional override for |\version| flag:
%    \begin{macrocode}
%%\providecommand{\version}{final}
%    \end{macrocode}

% Include the main document:
%    \begin{macrocode}
\input{childdoc.def}
\childdocof{cdocsamp}
%    \end{macrocode}

%\iffalse
%</samplechap1|samplechap2>
%\fi
%
%\iffalse
%<*samplechap1>
%\fi
% Some text for chapter 1:
%    \begin{macrocode}
\section{one}
some text in chapter one
%    \end{macrocode}

%\iffalse
%</samplechap1>
%\fi
% Some text for chapter 2:
%\iffalse
%<*samplechap2>
%\fi
%    \begin{macrocode}
\section{two}
more text in chapter two
%    \end{macrocode}

%\iffalse
%</samplechap2>
%\fi
%
% %%%%%%%%%%%%%%%%%%%%%%%%%%%%%%%%%%%%%%
% \paragraph{Part Include Files.}
%
% The include files are called |cdocspt3.tex| and |cdocspt4.tex|.
%
%\iffalse
%<*samplepart3|samplepart4>
%\fi

% Optional override for |\version| flag:
%    \begin{macrocode}
%%\providecommand{\version}{final}
%    \end{macrocode}

% Include the main document:
%    \begin{macrocode}
\input{childdoc.def}
\childdocby{cdocsamp}
%    \end{macrocode}

%\iffalse
%</samplepart3|samplepart4>
%\fi
%
%\iffalse
%<*samplepart3>
%\fi
% Some text for part 3:
%    \begin{macrocode}
some text in part three
%    \end{macrocode}

%\iffalse
%</samplepart3>
%\fi
% Some text for part 4:
%\iffalse
%<*samplepart4>
%\fi
%    \begin{macrocode}
more text in part four
%    \end{macrocode}

%\iffalse
%</samplepart4>
%\fi
%
% %%%%%%%%%%%%%%%%%%%%%%%%%%%%%%%%%%%%%%
% \paragraph{Forwarding for a Complete Draft.}
%
% The following forwarding file |cdocsdrf.tex|
% compiles the main document in draft mode:
%\iffalse
%<*sampledraft>
%\fi
%    \begin{macrocode}
\def\version{draft}
\input{childdoc.def}
\childdocforward{cdocsamp}
%    \end{macrocode}

%\iffalse
%</sampledraft>
%\fi
%
% %%%%%%%%%%%%%%%%%%%%%%%%%%%%%%%%%%%%%%
% \paragraph{Forwarding for Final Version of the Chapters.}
%
% The following forwarding files |cdocsfn1.tex| and |cdocsfn2.tex|
% (with identical content)
% compile the final versions of the child documents
% |cdocsch1.tex| and |cdocsch2.tex|, respectively:
%\iffalse
%<*samplefinal>
%\fi
%    \begin{macrocode}
\def\version{final}
\input{childdoc.def}
\childdocforwardprefix[cdocsamp]{cdocsfn}{cdocsch}
%    \end{macrocode}

%\iffalse
%</samplefinal>
%\fi
%
% %%%%%%%%%%%%%%%%%%%%%%%%%%%%%%%%%%%%%%
% \paragraph{Command Line Processing.}
%
% The following three command lines generate the output files
% |cdocscld|, |cdocscl1| and |cdocscl2|
% which should be identical to
% |cdocsdrf|, |cdocsch1| and |cdocsfn2|, respectively:
% \begin{center}
% \begin{tabular}{l}
% |latex -jobname cdocscld \|\\
% |  "\def\version{draft}\input{childdoc.def}\childdocforward{cdocsamp}"|\\
% |latex -jobname cdocscl1 \|\\
% |  "\input{childdoc.def}\childdocforward[cdocsamp]{cdocsch1}"|\\
% |latex -jobname cdocscl2 \|\\
% |  "\def\version{final}\input{childdoc.def}\childdocforward{cdocsch2}"|
% \end{tabular}
% \end{center}
% Note that the trailing backslash on each first line
% merely continues the input to the second line
% (for convenient cut ant paste).
% Furthermore, the command |latex| can be replaced by any
% of its alternative versions such as |pdflatex|.
%
% %%%%%%%%%%%%%%%%%%%%%%%%%%%%%%%%%%%%%%%%%%%%%%%%%%%%%%%%%%%%%%%%%%%%%%%%%%%%%%
% %%%%%%%%%%%%%%%%%%%%%%%%%%%%%%%%%%%%%%%%%%%%%%%%%%%%%%%%%%%%%%%%%%%%%%%%%%%%%%
% \section{Implementation}
%\iffalse
%<*package>
%\fi
%
% This section describes the definitions file |childdoc.def|.

% The definitions cannot be loaded using |\usepackage| or |\RequirePackage|
% which has a mechanism to prevent loading a style file more than once.
% When loading the definitions by means of |\input|
% multiple instances have to be prevented manually:
%\iffalse
%This code needs to be before the `\ProvidesFile' directive
%which is defined at the beginning of this file.
%Therefore it is also placed there and commented out here.
%</package>
%<*discard>
%\fi
%    \begin{macrocode}
\ifdefined\childdocmain\endinput\fi
%    \end{macrocode}
%\iffalse
%</discard>
%<*package>
%\fi
%
% \macro{\ifchilddoc}
% \macro{\ifchilddocmanual}
% The conditional |\ifchilddoc| tells whether a
% child (true) or main (false) document is being compiled.
% The conditional |\ifchilddocmanual| tells whether
% the |\includeonly| mechanism is used (false) or
% the selection of child files must be performed manually (true).
% The definitions initialise to false:
%    \begin{macrocode}
\newif\ifchilddoc
\newif\ifchilddocmanual
%    \end{macrocode}

% \macro{\childdocname}
% \macro{\childdocjob}
% The macro |\childdocname| stores the name of the main document
% to be compiled. The macro |\childdocjob| stores the name of
% the document on which the \LaTeX{} compiler was originally invoked.
% The content of |\jobname| cannot be compared
% to filenames specified in the source due to different catcodes.
% The following code rescans |\jobname|, stores the result
% in |\childdocname| and saves a copy in |\childdocjob|:
%    \begin{macrocode}
\edef\childdocname{\scantokens\expandafter{\jobname\noexpand}}
\let\childdocjob\childdocname
%    \end{macrocode}

% \macro{\childdocdisable}
% The macro |\childdocdisable| prevents the main file
% from being processed more than once.
% At this stage, the main document command |\childdocmain|
% is assumed to be called once again where it should do nothing.
% Any subsequent call to it should prevent
% a secondary processing of the main document
% It overwrites the forwarding commands
% |\childdocof| and |\childdocforward|
% with empty macros to prevent further inclusions of the main document:
%    \begin{macrocode}
\newcommand{\childdocdisable}
{
  \renewcommand{\childdocmain}[1]{\renewcommand{\childdocmain}[1]{\endinput}}
  \renewcommand{\childdocof}[1]{}
  \renewcommand{\childdocby}[2][]{}
  \renewcommand{\childdocforward}[2][]{}
  \renewcommand{\childdocdisable}{}
}
%    \end{macrocode}

% \macro{\childdocmain}
% The macro |\childdocmain| is to be called at the top of the main file
% with nothing or the main filename (without extension) as argument.
% First, it breaks loops.
% If the argument is not empty and does not match |\childdocname|
% (which is set by the first inclusion of |childdoc.def|),
% |\ifchilddoc| is set to true, |\includeonly| is applied to the child file
% and |\jobname| is set to the main file
% (for proper handling of |.aux| files):
%    \begin{macrocode}
\newcommand{\childdocmain}[1]
{
  \childdocdisable\childdocmain{}
  \if?#1?\else
    \begingroup
      \def\childdoctmp{#1}
      \ifx\childdoctmp\childdocname
        \def\childdoctmp{}
      \else
        \def\childdoctmp
        {
          \childdoctrue
          \includeonly{\childdocname}
          \def\childdocjob{#1}
          \def\jobname{#1}
        }
      \fi
      \expandafter
    \endgroup
    \childdoctmp
  \fi
}
%    \end{macrocode}

% \macro{\childdocof}
% The command |\childdocof| redirects
% compilation to the main file |#1|.
%    \begin{macrocode}
\newcommand{\childdocof}[1]
{
  \childdocdisable
  \childdoctrue
  \includeonly{\childdocname}
  \def\jobname{#1}
  \def\childdocjob{#1}
  \input{#1}
}
%    \end{macrocode}

% \macro{\childdocby}
% The command |\childdocby| ....
%    \begin{macrocode}
\newcommand{\childdocby}[2][]
{
  \childdocdisable
  \childdoctrue
  \childdocmanualtrue
  \if?#1?\else
    \def\jobname{#2}
  \fi
  \def\childdocjob{#2}
  \input{#2}
  \endinput
}
%    \end{macrocode}

% \macro{\childdocforward}
% The command |\childdocforward| redirects
% compilation to the main file or
% (if the optional argument is given) a child file.
% Parameters are set as if the main file
% or a child file starting with |\childdocof| was compiled.
% Then compilation is handed over to the main file:
%    \begin{macrocode}
\newcommand{\childdocforward}[2][]
{
  \begingroup
    \if?#1?
      \def\childdoctmp
      {
        \def\childdocname{#2}
        \def\childdocjob{#2}
        \def\jobname{#2}
        \input{#2}
        \endinput
      }
    \else
      \def\childdoctmp
      {
        \childdocdisable
        \def\childdocname{#2}
        \childdoctrue
        \includeonly{#2}
        \def\childdocjob{#1}
        \def\jobname{#1}
        \input{#1}
        \endinput
      }
    \fi
    \expandafter
  \endgroup
  \childdoctmp
}
%    \end{macrocode}

% \macro{\childdocforwardprefix}
% The command |\childdocforwardprefix| redirects
% compilation to the main or a child file by means of a pattern.
% The prefix |#1| in the current filename is replaced by |#2|
% and the suffix of the current filename is kept
% (it is assumed that the filename does not contain the substring `|~~~|'
% which is used as a delimiter).
% Compilation is handed over to the new file by |\childdocforward|:
%    \begin{macrocode}
\newcommand{\childdocforwardprefix}[3][]
{
  \begingroup
    \def\childdocextract #2##1~~~{\def\childdoctmp{\childdocforward[#1]{#3##1}}}
    \expandafter\childdocextract\childdocname~~~
    \expandafter
  \endgroup
  \childdoctmp
}
%    \end{macrocode}

% \macro{\childdoc}
% The deprecated macro |\childdoc| is a legacy version of |\childdocmain|:
%    \begin{macrocode}
\newcommand{\childdoc}{\childdocmain}
%    \end{macrocode}

% \macro{\childdocredirect}
% The deprecated macro |\childdocredirect| is a legacy version
% of |\childdocforward| and |\childdocforwardprefix|:
%    \begin{macrocode}
\newcommand{\childdocredirect}[2][]
{
  \begingroup
    \if?#1?
      \def\childdoctmp{\childdocforward{#2}}
    \else
      \def\childdoctmp{\childdocforwardprefix{#1}{#2}}
    \fi
    \expandafter
  \endgroup
  \childdoctmp
}
%    \end{macrocode}

%\iffalse
%</package>
%\fi
%
\endinput
|\\
|\childdocforward{|\textit{main}|}|
\end{tabular}
\end{center}
%
Likewise, the following files |final|\textit{nn}|.tex|
compile the final version of the child document
|child|\textit{nn}|.tex|:
%
\begin{center}
\begin{tabular}{l}
|\def\version{final}|\\
|% \iffalse
%
% childdoc.dtx Copyright (C) 2017-2018 Niklas Beisert
%
% This work may be distributed and/or modified under the
% conditions of the LaTeX Project Public License, either version 1.3
% of this license or (at your option) any later version.
% The latest version of this license is in
%   http://www.latex-project.org/lppl.txt
% and version 1.3 or later is part of all distributions of LaTeX
% version 2005/12/01 or later.
%
% This work has the LPPL maintenance status `maintained'.
%
% The Current Maintainer of this work is Niklas Beisert.
%
% This work consists of the files childdoc.dtx and childdoc.ins
% and the derived files childdoc.def and cdocsamp.tex with
% cdocsch1.tex, cdocsch2.tex, cdocsdrf.tex, cdocsfn1.tex, cdocsfn2.tex.
%
%<package>\ifdefined\childdocmain\endinput\fi
%<package>\ProvidesFile{childdoc.def}[2018/12/30 v2.0 child document driver]
%<samplemain>\ProvidesFile{cdocsamp.tex}[2018/12/30 v2.0 sample for childdoc]
%<*driver>
%\ProvidesFile{childdoc.drv}[2018/12/30 v2.0 childdoc reference manual file]
\PassOptionsToClass{10pt,a4paper}{article}
\documentclass{ltxdoc}

\usepackage[margin=35mm]{geometry}
\usepackage{hyperref}
\usepackage{hyperxmp}
\usepackage[usenames]{color}

\hypersetup{colorlinks=true}
\hypersetup{pdfstartview=FitH}
\hypersetup{pdfpagemode=UseNone}
\hypersetup{pdfsource={}}
\hypersetup{pdflang={en-UK}}
\hypersetup{pdfcopyright={Copyright 2017-2018 Niklas Beisert.
  This work may be distributed and/or modified under the
  conditions of the LaTeX Project Public License, either version 1.3
  of this license or (at your option) any later version.}}
\hypersetup{pdflicenseurl={http://www.latex-project.org/lppl.txt}}
\hypersetup{pdfcontactaddress={ETH Zurich, ITP, HIT K,
  Wolfgang-Pauli-Strasse 27}}
\hypersetup{pdfcontactpostcode={8093}}
\hypersetup{pdfcontactcity={Zurich}}
\hypersetup{pdfcontactcountry={Switzerland}}
\hypersetup{pdfcontactemail={nbeisert@itp.phys.ethz.ch}}
\hypersetup{pdfcontacturl={http://people.phys.ethz.ch/\xmptilde nbeisert/}}

\newcommand{\secref}[1]{\hyperref[#1]{section \ref*{#1}}}

\parskip1ex
\parindent0pt
\let\olditemize\itemize
\def\itemize{\olditemize\parskip0pt}

\begin{document}

\title{The \textsf{childdoc} Package}
\hypersetup{pdftitle={The childdoc Package}}
\author{Niklas Beisert\\[2ex]
  Institut f\"ur Theoretische Physik\\
  Eidgen\"ossische Technische Hochschule Z\"urich\\
  Wolfgang-Pauli-Strasse 27, 8093 Z\"urich, Switzerland\\[1ex]
  \href{mailto:nbeisert@itp.phys.ethz.ch}
  {\texttt{nbeisert@itp.phys.ethz.ch}}}
\hypersetup{pdfauthor={Niklas Beisert}}
\hypersetup{pdfsubject={Manual for the LaTeX2e Package childdoc}}
\date{30 December 2018, \textsf{v2.0}}
\maketitle

\begin{abstract}\noindent
\textsf{childdoc} is a \LaTeXe{} package
that enables the direct compilation
of document sections included by |\include|
to individual files.
\end{abstract}

\begingroup
\parskip0ex
\tableofcontents
\endgroup

%%%%%%%%%%%%%%%%%%%%%%%%%%%%%%%%%%%%%%%%%%%%%%%%%%%%%%%%%%%%%%%%%%%%%%%%%%%%%%%%
%%%%%%%%%%%%%%%%%%%%%%%%%%%%%%%%%%%%%%%%%%%%%%%%%%%%%%%%%%%%%%%%%%%%%%%%%%%%%%%%
\section{Introduction}

\LaTeX{} provides a mechanism to structure a large document (such as a book)
into a main file and several child files (containing the chapters)
using the |\include| command.
This mechanism is beneficial for documents
which span hundreds of pages in order to
make the source file(s) more manageable.
Moreover, compilation can be restricted to
selected child files by means of the |\includeonly| command.
The latter feature can be used to reduce the compilation time while editing
(this was significantly more useful in the earlier days of \LaTeX{})
or to generate a smaller document which is easier to navigate.
Another application of |\includeonly| is to generate
documents consisting of selected parts of the complete document.

However, there are a few drawbacks of the plain |\include| mechanism:
\begin{itemize}
\item
The child files cannot be compiled on their own,
they can only be compiled via the main file.
A naive editing environment
(such as a text editor with an option
to have the current file processed by \LaTeX)
may require one to switch to the main file before compiling;
attempting to compile the child file produces errors.
\item
The main file must be modified (each time)
to adjust the |\includeonly| command
to the present needs. This easily leaves the main file in a messy state.
\item
The generated document will always carry the filename
of the main document. This is inconvenient if
several child files are to be compiled and
to be kept for distribution.
\end{itemize}

The present package provides a simple interface
to make child files individually compilable by \LaTeX{}.
Compiling a child file then has the same effect as compiling
the main file with an |\includeonly| command
to select the appropriate child.
Moreover the generated document will carry the name of the child
rather than the main file.
This resolves all three above issues.

This feature is meant to make the editing of books,
thesis documents and lecture notes somewhat more convenient.
However, the package can also be used efficiently for
composing a series of documents (such as exercise sheets)
which are typically distributed individually.
It then assists the author in generating the individual documents
(potentially in different versions)
as well as a document containing the collected series.
Another application is in developing style files
or other kinds of included material
where compilation of the style file could redirect
to a sample or test file.

%%%%%%%%%%%%%%%%%%%%%%%%%%%%%%%%%%%%%%%%%%%%%%%%%%%%%%%%%%%%%%%%%%%%%%%%%%%%%%%%
%%%%%%%%%%%%%%%%%%%%%%%%%%%%%%%%%%%%%%%%%%%%%%%%%%%%%%%%%%%%%%%%%%%%%%%%%%%%%%%%
\section{Usage}

First of all, the package \textsf{childdoc} is \emph{not} a standard
\LaTeXe{} |.sty| style file! Therefore it needs to be invoked in
a non-standard way.

%%%%%%%%%%%%%%%%%%%%%%%%%%%%%%%%%%%%%%%%%%%%%%%%%%%%%%%%%%%%%%%%%%%%%%%%%%%%%%%%
\subsection{Included Files}
\label{sec:include}

%%%%%%%%%%%%%%%%%%%%%%%%%%%%%%%%%%%%%%%%
\DescribeMacro{\childdocmain}
To use the package, add the commands
\begin{center}
\begin{tabular}{l}
|\input{childdoc.def}|\\
|\childdocmain{}|\\
\end{tabular}
\end{center}
at the very top of the main \LaTeX{} file,
in particular \emph{before} the |\documentclass| statement!
The argument of |\childdocmain| should be left empty
(but it must be present).

%%%%%%%%%%%%%%%%%%%%%%%%%%%%%%%%%%%%%%%%
\DescribeMacro{\childdocof}
Furthermore, add the commands
\begin{center}
\begin{tabular}{l}
|\input{childdoc.def}|\\
|\childdocof{|\textit{main}|}|\\
\end{tabular}
\end{center}
at the top of every child file \textit{child}
which is included by |\include{|\textit{child}|}|
from within the main file
(or at least for those files to be compiled individually).
The argument \textit{main} must be the filename of the main file.

There are a couple of
considerations in setting up the main and child documents:

%%%%%%%%%%%%%%%%%%%%%%%%%%%%%%%%%%%%%%%%
\paragraph{Restrictions.}

Please note the following restrictions:
\begin{itemize}
\item
|\childdocmain| must be called with one argument \textit{main}
to ensure compatibility with earlier version of the package.
It must either be empty (|\childdocmain{}|)
or precisely match the filename of the main file in which it is specified.
See \secref{sec:detection} for further information.
\item
The filename \textit{main} must be specified without the |.tex| extension.
\item
The filename \textit{main} is case sensitive
(even in case-insensitive file systems)
due to internal string comparison.
\item
The argument \textit{main} should be fully expanded, it cannot be a macro.
\item
Subdirectories and special characters should be avoided in filenames.
\item
The command |\childdocmain{|\textit{main}|}| must be followed by a whitespace.
It should not be followed immediately by another command
or by a comment mark `|%|'.
This is because the \TeX{} parser reads the token immediately following
the argument of |\childdocmain| and puts it
at the beginning of every child section;
however, a white\-space is ignored.
\end{itemize}

%%%%%%%%%%%%%%%%%%%%%%%%%%%%%%%%%%%%%%%%
\paragraph{Content of Main File.}

It is advisable to place all content in the child files included by |\include|.
Any output contained in the main file will appear in all child documents
unless suppressed manually;
it cannot be suppressed automatically by the |\includeonly| directive
and thus should normally be avoided.
A method to include some content in the main file
by means of conditional processing is described in \secref{sec:conditional}.

%%%%%%%%%%%%%%%%%%%%%%%%%%%%%%%%%%%%%%%%
\paragraph{Page Numbering.}

When only a part of the document is compiled,
the appropriate numbering of pages
(as well as other status parameters)
is determined from the |.aux| files.
The latter contain information from previous passes.
However this information needs to propagate through
all intermediate child documents.
Therefore the page numbering in child documents may well
be inconsistent until the complete document is compiled at least once.

A useful (if unconventional) way to always ensure a consistent
page numbering is to restart the numbering in each child document
and denote the pages by `\textit{child}|.|\textit{page}'
where \textit{child} represents the chapter/section number of the child file.
This can be achieved by the command
|\numberwithin{page}{|\textit{child}|}|
of the \textsf{amsmath} package
where \textit{child} can be |chapter| or |section|
depending on the chosen structuring.
Alternatively, one can modify the macro |\thepage| appropriately
and reset the counter |page| at the start of each child file.

%%%%%%%%%%%%%%%%%%%%%%%%%%%%%%%%%%%%%%%%%%%%%%%%%%%%%%%%%%%%%%%%%%%%%%%%%%%%%%%%
\subsection{Conditional Processing}
\label{sec:conditional}

The package provides a mechanism to compile different versions
of a document. To customise the versions further some conditional processing
can come in handy to distinguish which version is being compiled.
The package provides two macros to describe the compilation context:

%%%%%%%%%%%%%%%%%%%%%%%%%%%%%%%%%%%%%%%%
\DescribeMacro{\ifchilddoc}
The conditional |\ifchilddoc| distinguishes between the compilation of
child documents and the main document:
%
\begin{center}
|\ifchilddoc |\textit{child-code}| |[|\||else |\textit{main-code}]| \||fi|
\end{center}

%%%%%%%%%%%%%%%%%%%%%%%%%%%%%%%%%%%%%%%%
\DescribeMacro{\childdocname}
\DescribeMacro{\childdocjob}
The macro |\childdocname| contains the filename (without extension)
of the main or child file being processed.
Note that |\childdocjob| will always contain the name of the main file.

%%%%%%%%%%%%%%%%%%%%%%%%%%%%%%%%%%%%%%%%
\paragraph{Title Page.}

Conditional processing can be used to include a title or banner page
in the main document when proper precautions are taken.
Importantly, the code in the main file should ensure that the page counter
(as well as other status parameters which are stored in the |.aux| files)
takes the same value after the conditional processing.
Otherwise the page numbers may take divergent values
depending on which part is compiled.

For example, a title page could be declared by:
%
\begin{center}
\begin{tabular}{l}
|\ifchilddoc\||else|\\
|\addtocounter{page}{-1}|\\
\textit{code for title page}\\
|\newpage|\\
|\||fi|
\end{tabular}
\end{center}
%
A banner page for the child documents can be generated by:
%
\begin{center}
\begin{tabular}{l}
|\ifchilddoc|\\
|\addtocounter{page}{-1}|\\
\textit{code for banner page}\\
|\newpage|\\
|\||fi|
\end{tabular}
\end{center}
%
Here one could write a message such as:
\begin{center}
|This is the part \childdocname{} of \childdocjob{}.|
\end{center}

%%%%%%%%%%%%%%%%%%%%%%%%%%%%%%%%%%%%%%%%%%%%%%%%%%%%%%%%%%%%%%%%%%%%%%%%%%%%%%%%
\subsection{Flags}
\label{sec:flags}

The package makes it easy to generate different versions
of the main or child documents.
To this end compilation flags can be defined
and assigned different default values.
They will be particularly useful in conjunction
with the forwarding mechanism described in \secref{sec:forward}.

For example, it may be useful to have a flag |\version|
which can be set to |draft| or |final|.
The document source will contain some conditional code
depending on the value of |\version|.
Suppose further, the flag should default to |final| for the main file
and to |draft| for child files
which is a natural assignment for editing the document.
This is achieved by placing the following code
in the preamble of the main document
(below the |\childdocmain| directive):
%
\begin{center}
\begin{tabular}{l}
|\ifchilddoc|\\
|\providecommand{\version}{draft}|\\
|\||else|\\
|\providecommand{\version}{final}|\\
|\||fi|
\end{tabular}
\end{center}
%
The definition by |\providecommand| makes sure
that previous definitions are not overwritten.
Further statements |\providecommand{\version}{...}|
can thus be added before the above code to override it.

For the main file, one might add a line
(between |\childdocmain| and the above block)
%
\begin{center}
|%\ifchilddoc\||else\providecommand{\version}{draft}\||fi|
\end{center}
%
which can be uncommented to produce a draft version.
Likewise one can add a line to the very top of a child file
(above the |\childdocof{|\textit{main}|}| directive)
%
\begin{center}
|%\providecommand{\version}{final}|
\end{center}
%
which can be uncommented to produce the final version of this child document.

%%%%%%%%%%%%%%%%%%%%%%%%%%%%%%%%%%%%%%%%%%%%%%%%%%%%%%%%%%%%%%%%%%%%%%%%%%%%%%%%
\subsection{Forwarding}
\label{sec:forward}

Different versions of the main or child documents
using compilation flags as described in \secref{sec:flags}
can be (permanently) stored in different files
for convenient compilation, viewing and distribution.
To this end, the package defines a command
to pass on compilation to a different file:

%%%%%%%%%%%%%%%%%%%%%%%%%%%%%%%%%%%%%%%%
\DescribeMacro{\childdocforward}
The command |\childdocforward| redirects processing to
another source file:
%
\begin{center}
\begin{tabular}{l}
|\input{childdoc.def}|\\
|\childdocforward[|\textit{main}|]{|\textit{dest}|}|\\
\end{tabular}
\end{center}
%
The argument \textit{dest} is the destination file
(without extension).
It should be the main file or one of the child files.
Note that further \textsf{childdoc} directives
such as |\childdocof| and |\childdocforward|
in the indicated file will be processed in this form.
The optional argument \textit{main}
passes on directly to the main file \textit{main}
while pretending to compile the child \textit{dest}.
This form behaves as if \textit{dest}
issues |\childdocof{|\textit{main}|}| right away,
and no further \textsf{childdoc} directives will be processed.

%%%%%%%%%%%%%%%%%%%%%%%%%%%%%%%%%%%%%%%%
\DescribeMacro{\...prefix}
In the alternative form |\childdocforwardprefix|,
%
\begin{center}
\begin{tabular}{l}
|\input{childdoc.def}|\\
|\childdocforwardprefix[|\textit{main}|]{|\textit{prefix}|}{|\textit{dest}|}|
\end{tabular}
\end{center}
%
the destination file is determined by a pattern
depending on the current file:
To make this work, the current file must be called
`{\textit{prefix}\hspace{0.2em}\textit{suffix}}'
with \textit{prefix} matching precisely the argument.
Processing is then passed on to the file
`{\textit{dest}\hspace{0.2em}\textit{suffix}}'.
Surely, the same effect is achieved by
directly specifying the
argument `{\textit{dest}\hspace{0.2em}\textit{suffix}}'
in the first form.
However, that requires to set up a different file
for each child. With the alternative form of the command
all these files can have exactly the same content
which simplifies setting them up and maintaining them.

For example, the following file |draft.tex|
with a compilation flag |\version| as described in \secref{sec:flags}
compiles the main document as a draft:
%
\begin{center}
\begin{tabular}{l}
|\def\version{draft}|\\
|\input{childdoc.def}|\\
|\childdocforward{|\textit{main}|}|
\end{tabular}
\end{center}
%
Likewise, the following files |final|\textit{nn}|.tex|
compile the final version of the child document
|child|\textit{nn}|.tex|:
%
\begin{center}
\begin{tabular}{l}
|\def\version{final}|\\
|\input{childdoc.def}|\\
|\childdocforwardprefix{final}{child}|
\end{tabular}
\end{center}
%

Note that when several versions of a main file and/or of each child file
are to be generated, it may be convenient to set up a |Makefile| or
shell script to automatise the process.

%%%%%%%%%%%%%%%%%%%%%%%%%%%%%%%%%%%%%%%%%%%%%%%%%%%%%%%%%%%%%%%%%%%%%%%%%%%%%%%%
\subsection{Command Line Processing}
\label{sec:commandline}

The effect of redirection files can also be achieved by invoking
the \LaTeX{} compiler with a more elaborate command line.
Most conveniently this should be done as part
of a shell script or a |Makefile|.

When using \textsf{childdoc} in the main file, the following
command lines effectively perform a redirection
(note that depending on the shell being used,
backslashes may have to be doubled: `|\|' $\to$ `|\\|'):
%
\begin{center}
|... -jobname "|\textit{target}|" |\\|"|[\textit{flags}]%
|\input{childdoc.def}\childdocforward[|\textit{main}|]{|\textit{dest}|}"|
\end{center}
%
Here \textit{target} is the name of the output file,
\textit{main} is the name of the main file
and \textit{dest} is the name of the main or child file to be processed
(all filenames without extensions).
The optional argument \textit{main} can be omitted
if \textit{main} matches \textit{dest}.
Optionally, compilation \textit{flags} can be defined via |\def| commands.
This command line makes the \TeX{} engine believe
it is compiling the file \textit{target}
whose content is specified as the latter parameter.
The provided code then forwards the processing to
\textit{main} or \textit{dest} as described in \secref{sec:forward}.

%%%%%%%%%%%%%%%%%%%%%%%%%%%%%%%%%%%%%%%%%%%%%%%%%%%%%%%%%%%%%%%%%%%%%%%%%%%%%%%%
\subsection{Include by Input}
\label{sec:input}

Including child documents by |\include| has some restrictions by design.
Most notably, the content of a child document always occupies
its own set of pages; pages cannot be shared between child documents.
Usually, this behaviour makes perfect sense
because each child document contain an essential part of the document.
However, in some situations it may be desirable to compose
a document from a collection of parts
without having mandatory page breaks between then.
For this case, the package
provides a mechanism to include parts
by |\input| which can also be processed individually.
However, by construction this mechanism
requires manual handling of the content to be output.

%%%%%%%%%%%%%%%%%%%%%%%%%%%%%%%%%%%%%%%%
\DescribeMacro{\ifchilddocmanual}
The main file should be prepared as usual, see \secref{sec:include}.
However, the document body must make a distinction
between processing of an individual part and of the main document, e.g.:
%
\begin{center}
\begin{tabular}{l}
|\ifchilddocmanual|\\
|\input{\childdocname}|\\
|\||else|\\
\textit{document body with }|\input{|\textit{part}|}|\\
|\||fi|
\end{tabular}
\end{center}
%
The conditional |\ifchilddocmanual| is true whenever
a part to be included by |\input| is being compiled,
and the name of the part is stored in |\childdocname|.

%%%%%%%%%%%%%%%%%%%%%%%%%%%%%%%%%%%%%%%%
\DescribeMacro{\childdocby}
Each part to be included by |\input| should start with:
%
\begin{center}
\begin{tabular}{l}
|\input{childdoc.def}|\\
|\childdocby{|\textit{main}|}|\\
\end{tabular}
\end{center}
%
The directive |\childdocby| is similar to |\childdocof|
described in \secref{sec:include},
but the subsequent selection of content must be done manually.
To that end, both |\ifchilddoc| and |\ifchilddocmanual|
will be true upon processing of a part,
and the name of the part is stored in |\childdocname|.
Note that |\jobname| will be set to the filename of the current part
so that each part receives an individual |.aux| file
that does not interfere with the |.aux| file(s) of the main document.
This behaviour can be altered by the alternative form
|\childdocby[*]{|\textit{main}|}| (with a non-empty optional argument)
which uses the |.aux| file of the main document
by setting |\jobname| to \textit{main}.

%%%%%%%%%%%%%%%%%%%%%%%%%%%%%%%%%%%%%%%%%%%%%%%%%%%%%%%%%%%%%%%%%%%%%%%%%%%%%%%%
\subsection{Driver Development}
\label{sec:driver}

The \textsf{childdoc} mechanism can also be use for the development
of definition files such as \LaTeX{} styles or classes.
This case differs from the above setup with multiple parts
included by |\include| in that no |\includeonly| should be invoked.
This can be achieved by starting the include file
(before |\ProvidesPackage|) with:
%
\begin{center}
\begin{tabular}{l}
|\input{childdoc.def}|\\
|\childdocforward{|\textit{main}|}|\\
\end{tabular}
\end{center}
%
or alternatively with:
%
\begin{center}
\begin{tabular}{l}
|\input{childdoc.def}|\\
|\childdocby{|\textit{main}|}|\\
\end{tabular}
\end{center}
%
Both forms have slightly different effects as described above.
The main file is prepared as usual, see \secref{sec:include}.

%%%%%%%%%%%%%%%%%%%%%%%%%%%%%%%%%%%%%%%%%%%%%%%%%%%%%%%%%%%%%%%%%%%%%%%%%%%%%%%%
\subsection{Legacy Detection}
\label{sec:detection}

The directive |\childdocmain| in the main file can detect
whether the complete document or merely a child is to be compiled
even without using the directive |\childdocof|.
This method is deprecated because it is less robust
and there is no compelling reason to use it;
it is merely provided for backward compatibility
and it may be removed in future versions.

If the detection mechanism is to be used,
it is mandatory to correctly specify
the filename of the main file as the argument of |\childdocmain|:
%
\begin{center}
\begin{tabular}{l}
|\input{childdoc.def}|\\
|\childdocmain{|\textit{main}|}|\\
\end{tabular}
\end{center}
%
If |\jobname| does not match the argument \textit{main} of |\childdocmain|,
it is assumed that |\jobname| points to the child file to be compiled.
When using |\childdocmain| with the main file specified as argument,
it suffices to start a child file
with just |\input{|\textit{main}|}|
without loading of the package and using |\childdocof|.
If instead all processing is done
with the appropriate \textsf{childdoc} directives,
the argument of \textit{main} of |\childdocmain| can be empty.

An alternative version of the command line processing described
in \secref{sec:commandline} using the detection mechanism reads:
%
\begin{center}
|... -jobname "|\textit{target}|" "|[\textit{flags}]%
[|\def\jobname{|\textit{dest}|}|]|\input{|\textit{main}|}"|
\end{center}

%%%%%%%%%%%%%%%%%%%%%%%%%%%%%%%%%%%%%%%%%%%%%%%%%%%%%%%%%%%%%%%%%%%%%%%%%%%%%%%%
\subsection{Manual Code}
\label{sec:manual}

In case one cannot be certain whether the definitions file |childdoc.def|
is installed on the target \TeX{} distribution
and one prefers not to ship it,
it is conceivable to paste a few relevant commands into the sources.

To that end, drop all statements |\input{childdoc.def}|
and perform the replacements as outlined below.
Instead of |\childdocmain{|\textit{main}|}| add the following code
to the top of the main file:
%
\begin{center}
\begin{tabular}{l}
|\||ifdefined\childdocname\endinput\||fi\newif\ifchilddoc|\\
|\edef\childdocname{\scantokens\expandafter{\jobname\noexpand}}|\\
|\def\childdocmain{|\textit{main}|}\||ifx\childdocmain\childdocname\||else|\\
|\childdoctrue\includeonly{\childdocname}\let\jobname\childdocmain\||fi|\\
\end{tabular}
\end{center}
%
Instead of |\childdocof{|\textit{main}|}| just include the main file
at the top of each child file:
%
\begin{center}
|\input{|\textit{main}|}|
\end{center}
%
A simple redirection |\childdocforward{|\textit{dest}|}| is achieved by:
%
\begin{center}
|\def\jobname{|\textit{dest}|}\input{\jobname}|
\end{center}
%
The redirection with prefix
|\childdocforwardprefix[|\textit{prefix}|]{|\textit{dest}|}|
is accomplished by:
%
\begin{center}
\begin{tabular}{l}
|{\edef\jobname{\scantokens\expandafter{\jobname\noexpand}}|\\
|\def\redirectjob |\textit{prefix}|#1~~~{\gdef\jobname{|\textit{dest}|#1}}|\\
|\expandafter\redirectjob\jobname~~~}\input{\jobname}|
\end{tabular}
\end{center}

In an alternative approach,
child documents can be compiled by a specific command line
without additional code or specific definitions:
%
\begin{center}
|... -jobname "|\textit{target}|" "|[\textit{flags}]%
|\includeonly{|\textit{dest}|}\input{|\textit{main}|}"|
\end{center}
%

%%%%%%%%%%%%%%%%%%%%%%%%%%%%%%%%%%%%%%%%%%%%%%%%%%%%%%%%%%%%%%%%%%%%%%%%%%%%%%%%
%%%%%%%%%%%%%%%%%%%%%%%%%%%%%%%%%%%%%%%%%%%%%%%%%%%%%%%%%%%%%%%%%%%%%%%%%%%%%%%%
\section{Information}

%%%%%%%%%%%%%%%%%%%%%%%%%%%%%%%%%%%%%%%%%%%%%%%%%%%%%%%%%%%%%%%%%%%%%%%%%%%%%%%%
\subsection{Copyright}

Copyright \copyright{} 2017--2018 Niklas Beisert

This work may be distributed and/or modified under the
conditions of the \LaTeX{} Project Public License, either version 1.3
of this license or (at your option) any later version.
The latest version of this license is in
  \url{http://www.latex-project.org/lppl.txt}
and version 1.3 or later is part of all distributions of \LaTeX{}
version 2005/12/01 or later.

This work has the LPPL maintenance status `maintained'.

The Current Maintainer of this work is Niklas Beisert.

This work consists of the files |README.txt|, |childdoc.ins| and |childdoc.dtx|
as well as the derived files |childdoc.def|, |cdocsamp.tex|
with |cdocsch1.tex|, |cdocsch2.tex|, |cdocspt3.tex|, |cdocspt4.tex|,
|cdocsdrf.tex|, |cdocsfn1.tex|, |cdocsfn2.tex|
as well as |childdoc.pdf|.

%%%%%%%%%%%%%%%%%%%%%%%%%%%%%%%%%%%%%%%%%%%%%%%%%%%%%%%%%%%%%%%%%%%%%%%%%%%%%%%%
\subsection{Files and Installation}

The package consists of the files:
%
\begin{center}
\begin{tabular}{ll}
    |README.txt|   & readme file \\
    |childdoc.ins| & installation file \\
    |childdoc.dtx| & source file \\
    |childdoc.def| & definition file \\
    |cdocsamp.tex| & sample main file \\
    |cdocsch1.tex| & sample include file \\
    |cdocsch2.tex| & sample include file \\
    |cdocspt3.tex| & sample part file \\
    |cdocspt4.tex| & sample part file \\
    |cdocsdrf.tex| & sample redirection file \\
    |cdocsfn1.tex| & sample redirection file \\
    |cdocsfn2.tex| & sample redirection file \\
    |childdoc.pdf| & manual
\end{tabular}
\end{center}
%
The distribution consists of the files
|README.txt|, |childdoc.ins| and |childdoc.dtx|.
%
\begin{itemize}
\item
Run (pdf)\LaTeX{} on |childdoc.dtx|
to compile the manual |childdoc.pdf| (this file).
\item
Run \LaTeX{} on |childdoc.ins| to create the definitions file |childdoc.def|
and the sample |cdocsamp.tex| with include files
|cdocsch1.tex|, |cdocsch2.tex|, |cdocspt3.tex|, |cdocspt4.tex|,
|cdocsdrf.tex|, |cdocsfn1.tex|, |cdocsfn2.tex|.
Then copy the file |childdoc.def| to an appropriate directory of your \LaTeX{}
distribution, e.g.\ \textit{texmf-root}|/tex/latex/childdoc|.
\end{itemize}

%%%%%%%%%%%%%%%%%%%%%%%%%%%%%%%%%%%%%%%%%%%%%%%%%%%%%%%%%%%%%%%%%%%%%%%%%%%%%%%%
\subsection{Related CTAN Packages}

There are several other packages which offer a similar functionality:
%
\begin{itemize}
\item
The packages
\href{http://ctan.org/pkg/docmute}{\textsf{docmute}},
\href{http://ctan.org/pkg/includex}{\textsf{includex}} and
\href{http://ctan.org/pkg/standalone}{\textsf{standalone}}
provide commands to include only the document body of
a child file thus allowing both files to be compiled individually.
\item
The packages \href{http://ctan.org/pkg/subdocs}{\textsf{subdocs}}
and \href{http://ctan.org/pkg/subfiles}{\textsf{subfiles}}
provide structures in which the main and child documents can be
encapsulated and allowing them to be compiled individually.
The inclusion mechanism is different from the conventional |\include|.
\item
The package \href{http://ctan.org/pkg/combine}{\textsf{combine}}
is an elaborate solution to combine several documents into one.
\end{itemize}
%
See also the CTAN topic \href{http://ctan.org/topic/subdocs}{\textsf{subdocs}}
for further related packages.
The present package differs from the above solutions in that
a document structure constructed with the conventional |\include| mechanism
just needs two extra commands at the top of every file
such that all constituent files can be compiled individually.

%%%%%%%%%%%%%%%%%%%%%%%%%%%%%%%%%%%%%%%%%%%%%%%%%%%%%%%%%%%%%%%%%%%%%%%%%%%%%%%%
%\subsection{Feature Suggestions}
%
%The following is a list of features which may be useful for future
%versions of this package:
%%
%\begin{itemize}
%\item
%\ldots
%\end{itemize}

%%%%%%%%%%%%%%%%%%%%%%%%%%%%%%%%%%%%%%%%%%%%%%%%%%%%%%%%%%%%%%%%%%%%%%%%%%%%%%%%
\subsection{Revision History}

%%%%%%%%%%%%%%%%%%%%%%%%%%%%%%%%%%%%%%%%
\paragraph{v2.0:} 2018/12/30

\begin{itemize}
\item
immediate forward processing
\item
added |\childdocby| mechanism
\item
manual restructured
\end{itemize}

%%%%%%%%%%%%%%%%%%%%%%%%%%%%%%%%%%%%%%%%
\paragraph{v1.6:} 2018/01/17

\begin{itemize}
\item
application for development of include files
\item
corrections to manual
\end{itemize}

%%%%%%%%%%%%%%%%%%%%%%%%%%%%%%%%%%%%%%%%
\paragraph{v1.5:} 2017/05/21

\begin{itemize}
\item
more complete structuring introduced
\item
|\childdocof| introduced
\item
|\childdoc| renamed to |\childdocmain|
\item
|\childredirect| renamed to |\childdocforward| and |\childdocforwardprefix|
and functionality expanded
\end{itemize}

%%%%%%%%%%%%%%%%%%%%%%%%%%%%%%%%%%%%%%%%
\paragraph{v1.0:} 2017/04/27

\begin{itemize}
\item
manual and install package
\item
first version published on CTAN
\end{itemize}

%%%%%%%%%%%%%%%%%%%%%%%%%%%%%%%%%%%%%%%%
\paragraph{v0.6:} 2017/04/26

\begin{itemize}
\item
redirection mechanism added
\end{itemize}

%%%%%%%%%%%%%%%%%%%%%%%%%%%%%%%%%%%%%%%%
\paragraph{v0.5:} 2017/04/26

\begin{itemize}
\item
functionality in definition file
\end{itemize}


%%%%%%%%%%%%%%%%%%%%%%%%%%%%%%%%%%%%%%%%%%%%%%%%%%%%%%%%%%%%%%%%%%%%%%%%%%%%%%%%
%%%%%%%%%%%%%%%%%%%%%%%%%%%%%%%%%%%%%%%%%%%%%%%%%%%%%%%%%%%%%%%%%%%%%%%%%%%%%%%%
%%%%%%%%%%%%%%%%%%%%%%%%%%%%%%%%%%%%%%%%%%%%%%%%%%%%%%%%%%%%%%%%%%%%%%%%%%%%%%%%
\appendix

\settowidth\MacroIndent{\rmfamily\scriptsize 000\ }

 \DocInput{childdoc.dtx}

\end{document}
%</driver>
% \fi
%
% %%%%%%%%%%%%%%%%%%%%%%%%%%%%%%%%%%%%%%%%%%%%%%%%%%%%%%%%%%%%%%%%%%%%%%%%%%%%%%
% %%%%%%%%%%%%%%%%%%%%%%%%%%%%%%%%%%%%%%%%%%%%%%%%%%%%%%%%%%%%%%%%%%%%%%%%%%%%%%
% \section{Sample}
%\iffalse
%<*samplemain>
%\fi
%
% The following presents a sample document
% with two chapters, two parts, a title page,
% a compile flag as well as three forwarding files to set the flag.
% It consists of eight |.tex| files:
% \begin{center}
% \begin{tabular}{ll}
% |cdocsamp.tex|&main file\\
% |cdocsch1.tex|&include file for chapter 1\\
% |cdocsch2.tex|&include file for chapter 2\\
% |cdocspt3.tex|&include file for part 3\\
% |cdocspt4.tex|&include file for part 4\\
% |cdocsdrf.tex|&forwarding file for main file in draft mode\\
% |cdocsfi1.tex|&forwarding file for final version of chapter 1\\
% |cdocsfi2.tex|&forwarding file for final version of chapter 2\\
% \end{tabular}
% \end{center}
% Each of the eight files can be compiled directly by the \LaTeX{} compiler.
%
% %%%%%%%%%%%%%%%%%%%%%%%%%%%%%%%%%%%%%%
% \paragraph{Main File.}
%
% The main file is called |cdocsamp.tex|.
%
% Load the \textsf{childdoc} definitions and
% declare the filename for the main document:
%    \begin{macrocode}
\input{childdoc.def}
\childdocmain{}
%    \end{macrocode}

% Optional override for |\version| flag:
%    \begin{macrocode}
%%\ifchilddoc\else\providecommand{\version}{draft}\fi
%    \end{macrocode}

% Define the default values for the |\version| flag
% (|final| for the main file and |draft| for childs):
%    \begin{macrocode}
\ifchilddoc
\providecommand{\version}{draft}
\else
\providecommand{\version}{final}
\fi
%    \end{macrocode}

% Load the standard document class:
%    \begin{macrocode}
\documentclass[12pt]{article}
%    \end{macrocode}

% Start the document body:
%    \begin{macrocode}
\begin{document}
%    \end{macrocode}

% Declare a title page.
% Print title, part of document being processed and version flag:
%    \begin{macrocode}
\addtocounter{page}{-1}
\begin{center}
{\LARGE\bfseries{}childdoc example\par}
\vspace{1cm}
\ifchilddoc
\ifchilddocmanual part\else chapter\fi:
`\childdocname' of `\childdocjob'\par
\else
main document: `\childdocjob'\par
\fi
version: \version\par
\end{center}
\newpage
%    \end{macrocode}

% Manually include selected file,
% otherwise process as usual:
%    \begin{macrocode}
\ifchilddocmanual
\section*{part `\childdocname'}
\input{\childdocname}
\else
%    \end{macrocode}

% Include the two chapters:
%    \begin{macrocode}
\include{cdocsch1}
\include{cdocsch2}
%    \end{macrocode}

% Include the two parts unless only chapters should be displayed:
%    \begin{macrocode}
\ifchilddoc\else
\section{part three}
\input{cdocspt3}
\section{part four}
\input{cdocspt4}
\fi
%    \end{macrocode}

% Process as usual until here:
%    \begin{macrocode}
\fi
%    \end{macrocode}

% End of document body:
%    \begin{macrocode}
\end{document}
%    \end{macrocode}
%\iffalse
%</samplemain>
%\fi
%
% %%%%%%%%%%%%%%%%%%%%%%%%%%%%%%%%%%%%%%
% \paragraph{Chapter Include Files.}
%
% The include files are called |cdocsch1.tex| and |cdocsch2.tex|.
%
%\iffalse
%<*samplechap1|samplechap2>
%\fi

% Optional override for |\version| flag:
%    \begin{macrocode}
%%\providecommand{\version}{final}
%    \end{macrocode}

% Include the main document:
%    \begin{macrocode}
\input{childdoc.def}
\childdocof{cdocsamp}
%    \end{macrocode}

%\iffalse
%</samplechap1|samplechap2>
%\fi
%
%\iffalse
%<*samplechap1>
%\fi
% Some text for chapter 1:
%    \begin{macrocode}
\section{one}
some text in chapter one
%    \end{macrocode}

%\iffalse
%</samplechap1>
%\fi
% Some text for chapter 2:
%\iffalse
%<*samplechap2>
%\fi
%    \begin{macrocode}
\section{two}
more text in chapter two
%    \end{macrocode}

%\iffalse
%</samplechap2>
%\fi
%
% %%%%%%%%%%%%%%%%%%%%%%%%%%%%%%%%%%%%%%
% \paragraph{Part Include Files.}
%
% The include files are called |cdocspt3.tex| and |cdocspt4.tex|.
%
%\iffalse
%<*samplepart3|samplepart4>
%\fi

% Optional override for |\version| flag:
%    \begin{macrocode}
%%\providecommand{\version}{final}
%    \end{macrocode}

% Include the main document:
%    \begin{macrocode}
\input{childdoc.def}
\childdocby{cdocsamp}
%    \end{macrocode}

%\iffalse
%</samplepart3|samplepart4>
%\fi
%
%\iffalse
%<*samplepart3>
%\fi
% Some text for part 3:
%    \begin{macrocode}
some text in part three
%    \end{macrocode}

%\iffalse
%</samplepart3>
%\fi
% Some text for part 4:
%\iffalse
%<*samplepart4>
%\fi
%    \begin{macrocode}
more text in part four
%    \end{macrocode}

%\iffalse
%</samplepart4>
%\fi
%
% %%%%%%%%%%%%%%%%%%%%%%%%%%%%%%%%%%%%%%
% \paragraph{Forwarding for a Complete Draft.}
%
% The following forwarding file |cdocsdrf.tex|
% compiles the main document in draft mode:
%\iffalse
%<*sampledraft>
%\fi
%    \begin{macrocode}
\def\version{draft}
\input{childdoc.def}
\childdocforward{cdocsamp}
%    \end{macrocode}

%\iffalse
%</sampledraft>
%\fi
%
% %%%%%%%%%%%%%%%%%%%%%%%%%%%%%%%%%%%%%%
% \paragraph{Forwarding for Final Version of the Chapters.}
%
% The following forwarding files |cdocsfn1.tex| and |cdocsfn2.tex|
% (with identical content)
% compile the final versions of the child documents
% |cdocsch1.tex| and |cdocsch2.tex|, respectively:
%\iffalse
%<*samplefinal>
%\fi
%    \begin{macrocode}
\def\version{final}
\input{childdoc.def}
\childdocforwardprefix[cdocsamp]{cdocsfn}{cdocsch}
%    \end{macrocode}

%\iffalse
%</samplefinal>
%\fi
%
% %%%%%%%%%%%%%%%%%%%%%%%%%%%%%%%%%%%%%%
% \paragraph{Command Line Processing.}
%
% The following three command lines generate the output files
% |cdocscld|, |cdocscl1| and |cdocscl2|
% which should be identical to
% |cdocsdrf|, |cdocsch1| and |cdocsfn2|, respectively:
% \begin{center}
% \begin{tabular}{l}
% |latex -jobname cdocscld \|\\
% |  "\def\version{draft}\input{childdoc.def}\childdocforward{cdocsamp}"|\\
% |latex -jobname cdocscl1 \|\\
% |  "\input{childdoc.def}\childdocforward[cdocsamp]{cdocsch1}"|\\
% |latex -jobname cdocscl2 \|\\
% |  "\def\version{final}\input{childdoc.def}\childdocforward{cdocsch2}"|
% \end{tabular}
% \end{center}
% Note that the trailing backslash on each first line
% merely continues the input to the second line
% (for convenient cut ant paste).
% Furthermore, the command |latex| can be replaced by any
% of its alternative versions such as |pdflatex|.
%
% %%%%%%%%%%%%%%%%%%%%%%%%%%%%%%%%%%%%%%%%%%%%%%%%%%%%%%%%%%%%%%%%%%%%%%%%%%%%%%
% %%%%%%%%%%%%%%%%%%%%%%%%%%%%%%%%%%%%%%%%%%%%%%%%%%%%%%%%%%%%%%%%%%%%%%%%%%%%%%
% \section{Implementation}
%\iffalse
%<*package>
%\fi
%
% This section describes the definitions file |childdoc.def|.

% The definitions cannot be loaded using |\usepackage| or |\RequirePackage|
% which has a mechanism to prevent loading a style file more than once.
% When loading the definitions by means of |\input|
% multiple instances have to be prevented manually:
%\iffalse
%This code needs to be before the `\ProvidesFile' directive
%which is defined at the beginning of this file.
%Therefore it is also placed there and commented out here.
%</package>
%<*discard>
%\fi
%    \begin{macrocode}
\ifdefined\childdocmain\endinput\fi
%    \end{macrocode}
%\iffalse
%</discard>
%<*package>
%\fi
%
% \macro{\ifchilddoc}
% \macro{\ifchilddocmanual}
% The conditional |\ifchilddoc| tells whether a
% child (true) or main (false) document is being compiled.
% The conditional |\ifchilddocmanual| tells whether
% the |\includeonly| mechanism is used (false) or
% the selection of child files must be performed manually (true).
% The definitions initialise to false:
%    \begin{macrocode}
\newif\ifchilddoc
\newif\ifchilddocmanual
%    \end{macrocode}

% \macro{\childdocname}
% \macro{\childdocjob}
% The macro |\childdocname| stores the name of the main document
% to be compiled. The macro |\childdocjob| stores the name of
% the document on which the \LaTeX{} compiler was originally invoked.
% The content of |\jobname| cannot be compared
% to filenames specified in the source due to different catcodes.
% The following code rescans |\jobname|, stores the result
% in |\childdocname| and saves a copy in |\childdocjob|:
%    \begin{macrocode}
\edef\childdocname{\scantokens\expandafter{\jobname\noexpand}}
\let\childdocjob\childdocname
%    \end{macrocode}

% \macro{\childdocdisable}
% The macro |\childdocdisable| prevents the main file
% from being processed more than once.
% At this stage, the main document command |\childdocmain|
% is assumed to be called once again where it should do nothing.
% Any subsequent call to it should prevent
% a secondary processing of the main document
% It overwrites the forwarding commands
% |\childdocof| and |\childdocforward|
% with empty macros to prevent further inclusions of the main document:
%    \begin{macrocode}
\newcommand{\childdocdisable}
{
  \renewcommand{\childdocmain}[1]{\renewcommand{\childdocmain}[1]{\endinput}}
  \renewcommand{\childdocof}[1]{}
  \renewcommand{\childdocby}[2][]{}
  \renewcommand{\childdocforward}[2][]{}
  \renewcommand{\childdocdisable}{}
}
%    \end{macrocode}

% \macro{\childdocmain}
% The macro |\childdocmain| is to be called at the top of the main file
% with nothing or the main filename (without extension) as argument.
% First, it breaks loops.
% If the argument is not empty and does not match |\childdocname|
% (which is set by the first inclusion of |childdoc.def|),
% |\ifchilddoc| is set to true, |\includeonly| is applied to the child file
% and |\jobname| is set to the main file
% (for proper handling of |.aux| files):
%    \begin{macrocode}
\newcommand{\childdocmain}[1]
{
  \childdocdisable\childdocmain{}
  \if?#1?\else
    \begingroup
      \def\childdoctmp{#1}
      \ifx\childdoctmp\childdocname
        \def\childdoctmp{}
      \else
        \def\childdoctmp
        {
          \childdoctrue
          \includeonly{\childdocname}
          \def\childdocjob{#1}
          \def\jobname{#1}
        }
      \fi
      \expandafter
    \endgroup
    \childdoctmp
  \fi
}
%    \end{macrocode}

% \macro{\childdocof}
% The command |\childdocof| redirects
% compilation to the main file |#1|.
%    \begin{macrocode}
\newcommand{\childdocof}[1]
{
  \childdocdisable
  \childdoctrue
  \includeonly{\childdocname}
  \def\jobname{#1}
  \def\childdocjob{#1}
  \input{#1}
}
%    \end{macrocode}

% \macro{\childdocby}
% The command |\childdocby| ....
%    \begin{macrocode}
\newcommand{\childdocby}[2][]
{
  \childdocdisable
  \childdoctrue
  \childdocmanualtrue
  \if?#1?\else
    \def\jobname{#2}
  \fi
  \def\childdocjob{#2}
  \input{#2}
  \endinput
}
%    \end{macrocode}

% \macro{\childdocforward}
% The command |\childdocforward| redirects
% compilation to the main file or
% (if the optional argument is given) a child file.
% Parameters are set as if the main file
% or a child file starting with |\childdocof| was compiled.
% Then compilation is handed over to the main file:
%    \begin{macrocode}
\newcommand{\childdocforward}[2][]
{
  \begingroup
    \if?#1?
      \def\childdoctmp
      {
        \def\childdocname{#2}
        \def\childdocjob{#2}
        \def\jobname{#2}
        \input{#2}
        \endinput
      }
    \else
      \def\childdoctmp
      {
        \childdocdisable
        \def\childdocname{#2}
        \childdoctrue
        \includeonly{#2}
        \def\childdocjob{#1}
        \def\jobname{#1}
        \input{#1}
        \endinput
      }
    \fi
    \expandafter
  \endgroup
  \childdoctmp
}
%    \end{macrocode}

% \macro{\childdocforwardprefix}
% The command |\childdocforwardprefix| redirects
% compilation to the main or a child file by means of a pattern.
% The prefix |#1| in the current filename is replaced by |#2|
% and the suffix of the current filename is kept
% (it is assumed that the filename does not contain the substring `|~~~|'
% which is used as a delimiter).
% Compilation is handed over to the new file by |\childdocforward|:
%    \begin{macrocode}
\newcommand{\childdocforwardprefix}[3][]
{
  \begingroup
    \def\childdocextract #2##1~~~{\def\childdoctmp{\childdocforward[#1]{#3##1}}}
    \expandafter\childdocextract\childdocname~~~
    \expandafter
  \endgroup
  \childdoctmp
}
%    \end{macrocode}

% \macro{\childdoc}
% The deprecated macro |\childdoc| is a legacy version of |\childdocmain|:
%    \begin{macrocode}
\newcommand{\childdoc}{\childdocmain}
%    \end{macrocode}

% \macro{\childdocredirect}
% The deprecated macro |\childdocredirect| is a legacy version
% of |\childdocforward| and |\childdocforwardprefix|:
%    \begin{macrocode}
\newcommand{\childdocredirect}[2][]
{
  \begingroup
    \if?#1?
      \def\childdoctmp{\childdocforward{#2}}
    \else
      \def\childdoctmp{\childdocforwardprefix{#1}{#2}}
    \fi
    \expandafter
  \endgroup
  \childdoctmp
}
%    \end{macrocode}

%\iffalse
%</package>
%\fi
%
\endinput
|\\
|\childdocforwardprefix{final}{child}|
\end{tabular}
\end{center}
%

Note that when several versions of a main file and/or of each child file
are to be generated, it may be convenient to set up a |Makefile| or
shell script to automatise the process.

%%%%%%%%%%%%%%%%%%%%%%%%%%%%%%%%%%%%%%%%%%%%%%%%%%%%%%%%%%%%%%%%%%%%%%%%%%%%%%%%
\subsection{Command Line Processing}
\label{sec:commandline}

The effect of redirection files can also be achieved by invoking
the \LaTeX{} compiler with a more elaborate command line.
Most conveniently this should be done as part
of a shell script or a |Makefile|.

When using \textsf{childdoc} in the main file, the following
command lines effectively perform a redirection
(note that depending on the shell being used,
backslashes may have to be doubled: `|\|' $\to$ `|\\|'):
%
\begin{center}
|... -jobname "|\textit{target}|" |\\|"|[\textit{flags}]%
|% \iffalse
%
% childdoc.dtx Copyright (C) 2017-2018 Niklas Beisert
%
% This work may be distributed and/or modified under the
% conditions of the LaTeX Project Public License, either version 1.3
% of this license or (at your option) any later version.
% The latest version of this license is in
%   http://www.latex-project.org/lppl.txt
% and version 1.3 or later is part of all distributions of LaTeX
% version 2005/12/01 or later.
%
% This work has the LPPL maintenance status `maintained'.
%
% The Current Maintainer of this work is Niklas Beisert.
%
% This work consists of the files childdoc.dtx and childdoc.ins
% and the derived files childdoc.def and cdocsamp.tex with
% cdocsch1.tex, cdocsch2.tex, cdocsdrf.tex, cdocsfn1.tex, cdocsfn2.tex.
%
%<package>\ifdefined\childdocmain\endinput\fi
%<package>\ProvidesFile{childdoc.def}[2018/12/30 v2.0 child document driver]
%<samplemain>\ProvidesFile{cdocsamp.tex}[2018/12/30 v2.0 sample for childdoc]
%<*driver>
%\ProvidesFile{childdoc.drv}[2018/12/30 v2.0 childdoc reference manual file]
\PassOptionsToClass{10pt,a4paper}{article}
\documentclass{ltxdoc}

\usepackage[margin=35mm]{geometry}
\usepackage{hyperref}
\usepackage{hyperxmp}
\usepackage[usenames]{color}

\hypersetup{colorlinks=true}
\hypersetup{pdfstartview=FitH}
\hypersetup{pdfpagemode=UseNone}
\hypersetup{pdfsource={}}
\hypersetup{pdflang={en-UK}}
\hypersetup{pdfcopyright={Copyright 2017-2018 Niklas Beisert.
  This work may be distributed and/or modified under the
  conditions of the LaTeX Project Public License, either version 1.3
  of this license or (at your option) any later version.}}
\hypersetup{pdflicenseurl={http://www.latex-project.org/lppl.txt}}
\hypersetup{pdfcontactaddress={ETH Zurich, ITP, HIT K,
  Wolfgang-Pauli-Strasse 27}}
\hypersetup{pdfcontactpostcode={8093}}
\hypersetup{pdfcontactcity={Zurich}}
\hypersetup{pdfcontactcountry={Switzerland}}
\hypersetup{pdfcontactemail={nbeisert@itp.phys.ethz.ch}}
\hypersetup{pdfcontacturl={http://people.phys.ethz.ch/\xmptilde nbeisert/}}

\newcommand{\secref}[1]{\hyperref[#1]{section \ref*{#1}}}

\parskip1ex
\parindent0pt
\let\olditemize\itemize
\def\itemize{\olditemize\parskip0pt}

\begin{document}

\title{The \textsf{childdoc} Package}
\hypersetup{pdftitle={The childdoc Package}}
\author{Niklas Beisert\\[2ex]
  Institut f\"ur Theoretische Physik\\
  Eidgen\"ossische Technische Hochschule Z\"urich\\
  Wolfgang-Pauli-Strasse 27, 8093 Z\"urich, Switzerland\\[1ex]
  \href{mailto:nbeisert@itp.phys.ethz.ch}
  {\texttt{nbeisert@itp.phys.ethz.ch}}}
\hypersetup{pdfauthor={Niklas Beisert}}
\hypersetup{pdfsubject={Manual for the LaTeX2e Package childdoc}}
\date{30 December 2018, \textsf{v2.0}}
\maketitle

\begin{abstract}\noindent
\textsf{childdoc} is a \LaTeXe{} package
that enables the direct compilation
of document sections included by |\include|
to individual files.
\end{abstract}

\begingroup
\parskip0ex
\tableofcontents
\endgroup

%%%%%%%%%%%%%%%%%%%%%%%%%%%%%%%%%%%%%%%%%%%%%%%%%%%%%%%%%%%%%%%%%%%%%%%%%%%%%%%%
%%%%%%%%%%%%%%%%%%%%%%%%%%%%%%%%%%%%%%%%%%%%%%%%%%%%%%%%%%%%%%%%%%%%%%%%%%%%%%%%
\section{Introduction}

\LaTeX{} provides a mechanism to structure a large document (such as a book)
into a main file and several child files (containing the chapters)
using the |\include| command.
This mechanism is beneficial for documents
which span hundreds of pages in order to
make the source file(s) more manageable.
Moreover, compilation can be restricted to
selected child files by means of the |\includeonly| command.
The latter feature can be used to reduce the compilation time while editing
(this was significantly more useful in the earlier days of \LaTeX{})
or to generate a smaller document which is easier to navigate.
Another application of |\includeonly| is to generate
documents consisting of selected parts of the complete document.

However, there are a few drawbacks of the plain |\include| mechanism:
\begin{itemize}
\item
The child files cannot be compiled on their own,
they can only be compiled via the main file.
A naive editing environment
(such as a text editor with an option
to have the current file processed by \LaTeX)
may require one to switch to the main file before compiling;
attempting to compile the child file produces errors.
\item
The main file must be modified (each time)
to adjust the |\includeonly| command
to the present needs. This easily leaves the main file in a messy state.
\item
The generated document will always carry the filename
of the main document. This is inconvenient if
several child files are to be compiled and
to be kept for distribution.
\end{itemize}

The present package provides a simple interface
to make child files individually compilable by \LaTeX{}.
Compiling a child file then has the same effect as compiling
the main file with an |\includeonly| command
to select the appropriate child.
Moreover the generated document will carry the name of the child
rather than the main file.
This resolves all three above issues.

This feature is meant to make the editing of books,
thesis documents and lecture notes somewhat more convenient.
However, the package can also be used efficiently for
composing a series of documents (such as exercise sheets)
which are typically distributed individually.
It then assists the author in generating the individual documents
(potentially in different versions)
as well as a document containing the collected series.
Another application is in developing style files
or other kinds of included material
where compilation of the style file could redirect
to a sample or test file.

%%%%%%%%%%%%%%%%%%%%%%%%%%%%%%%%%%%%%%%%%%%%%%%%%%%%%%%%%%%%%%%%%%%%%%%%%%%%%%%%
%%%%%%%%%%%%%%%%%%%%%%%%%%%%%%%%%%%%%%%%%%%%%%%%%%%%%%%%%%%%%%%%%%%%%%%%%%%%%%%%
\section{Usage}

First of all, the package \textsf{childdoc} is \emph{not} a standard
\LaTeXe{} |.sty| style file! Therefore it needs to be invoked in
a non-standard way.

%%%%%%%%%%%%%%%%%%%%%%%%%%%%%%%%%%%%%%%%%%%%%%%%%%%%%%%%%%%%%%%%%%%%%%%%%%%%%%%%
\subsection{Included Files}
\label{sec:include}

%%%%%%%%%%%%%%%%%%%%%%%%%%%%%%%%%%%%%%%%
\DescribeMacro{\childdocmain}
To use the package, add the commands
\begin{center}
\begin{tabular}{l}
|\input{childdoc.def}|\\
|\childdocmain{}|\\
\end{tabular}
\end{center}
at the very top of the main \LaTeX{} file,
in particular \emph{before} the |\documentclass| statement!
The argument of |\childdocmain| should be left empty
(but it must be present).

%%%%%%%%%%%%%%%%%%%%%%%%%%%%%%%%%%%%%%%%
\DescribeMacro{\childdocof}
Furthermore, add the commands
\begin{center}
\begin{tabular}{l}
|\input{childdoc.def}|\\
|\childdocof{|\textit{main}|}|\\
\end{tabular}
\end{center}
at the top of every child file \textit{child}
which is included by |\include{|\textit{child}|}|
from within the main file
(or at least for those files to be compiled individually).
The argument \textit{main} must be the filename of the main file.

There are a couple of
considerations in setting up the main and child documents:

%%%%%%%%%%%%%%%%%%%%%%%%%%%%%%%%%%%%%%%%
\paragraph{Restrictions.}

Please note the following restrictions:
\begin{itemize}
\item
|\childdocmain| must be called with one argument \textit{main}
to ensure compatibility with earlier version of the package.
It must either be empty (|\childdocmain{}|)
or precisely match the filename of the main file in which it is specified.
See \secref{sec:detection} for further information.
\item
The filename \textit{main} must be specified without the |.tex| extension.
\item
The filename \textit{main} is case sensitive
(even in case-insensitive file systems)
due to internal string comparison.
\item
The argument \textit{main} should be fully expanded, it cannot be a macro.
\item
Subdirectories and special characters should be avoided in filenames.
\item
The command |\childdocmain{|\textit{main}|}| must be followed by a whitespace.
It should not be followed immediately by another command
or by a comment mark `|%|'.
This is because the \TeX{} parser reads the token immediately following
the argument of |\childdocmain| and puts it
at the beginning of every child section;
however, a white\-space is ignored.
\end{itemize}

%%%%%%%%%%%%%%%%%%%%%%%%%%%%%%%%%%%%%%%%
\paragraph{Content of Main File.}

It is advisable to place all content in the child files included by |\include|.
Any output contained in the main file will appear in all child documents
unless suppressed manually;
it cannot be suppressed automatically by the |\includeonly| directive
and thus should normally be avoided.
A method to include some content in the main file
by means of conditional processing is described in \secref{sec:conditional}.

%%%%%%%%%%%%%%%%%%%%%%%%%%%%%%%%%%%%%%%%
\paragraph{Page Numbering.}

When only a part of the document is compiled,
the appropriate numbering of pages
(as well as other status parameters)
is determined from the |.aux| files.
The latter contain information from previous passes.
However this information needs to propagate through
all intermediate child documents.
Therefore the page numbering in child documents may well
be inconsistent until the complete document is compiled at least once.

A useful (if unconventional) way to always ensure a consistent
page numbering is to restart the numbering in each child document
and denote the pages by `\textit{child}|.|\textit{page}'
where \textit{child} represents the chapter/section number of the child file.
This can be achieved by the command
|\numberwithin{page}{|\textit{child}|}|
of the \textsf{amsmath} package
where \textit{child} can be |chapter| or |section|
depending on the chosen structuring.
Alternatively, one can modify the macro |\thepage| appropriately
and reset the counter |page| at the start of each child file.

%%%%%%%%%%%%%%%%%%%%%%%%%%%%%%%%%%%%%%%%%%%%%%%%%%%%%%%%%%%%%%%%%%%%%%%%%%%%%%%%
\subsection{Conditional Processing}
\label{sec:conditional}

The package provides a mechanism to compile different versions
of a document. To customise the versions further some conditional processing
can come in handy to distinguish which version is being compiled.
The package provides two macros to describe the compilation context:

%%%%%%%%%%%%%%%%%%%%%%%%%%%%%%%%%%%%%%%%
\DescribeMacro{\ifchilddoc}
The conditional |\ifchilddoc| distinguishes between the compilation of
child documents and the main document:
%
\begin{center}
|\ifchilddoc |\textit{child-code}| |[|\||else |\textit{main-code}]| \||fi|
\end{center}

%%%%%%%%%%%%%%%%%%%%%%%%%%%%%%%%%%%%%%%%
\DescribeMacro{\childdocname}
\DescribeMacro{\childdocjob}
The macro |\childdocname| contains the filename (without extension)
of the main or child file being processed.
Note that |\childdocjob| will always contain the name of the main file.

%%%%%%%%%%%%%%%%%%%%%%%%%%%%%%%%%%%%%%%%
\paragraph{Title Page.}

Conditional processing can be used to include a title or banner page
in the main document when proper precautions are taken.
Importantly, the code in the main file should ensure that the page counter
(as well as other status parameters which are stored in the |.aux| files)
takes the same value after the conditional processing.
Otherwise the page numbers may take divergent values
depending on which part is compiled.

For example, a title page could be declared by:
%
\begin{center}
\begin{tabular}{l}
|\ifchilddoc\||else|\\
|\addtocounter{page}{-1}|\\
\textit{code for title page}\\
|\newpage|\\
|\||fi|
\end{tabular}
\end{center}
%
A banner page for the child documents can be generated by:
%
\begin{center}
\begin{tabular}{l}
|\ifchilddoc|\\
|\addtocounter{page}{-1}|\\
\textit{code for banner page}\\
|\newpage|\\
|\||fi|
\end{tabular}
\end{center}
%
Here one could write a message such as:
\begin{center}
|This is the part \childdocname{} of \childdocjob{}.|
\end{center}

%%%%%%%%%%%%%%%%%%%%%%%%%%%%%%%%%%%%%%%%%%%%%%%%%%%%%%%%%%%%%%%%%%%%%%%%%%%%%%%%
\subsection{Flags}
\label{sec:flags}

The package makes it easy to generate different versions
of the main or child documents.
To this end compilation flags can be defined
and assigned different default values.
They will be particularly useful in conjunction
with the forwarding mechanism described in \secref{sec:forward}.

For example, it may be useful to have a flag |\version|
which can be set to |draft| or |final|.
The document source will contain some conditional code
depending on the value of |\version|.
Suppose further, the flag should default to |final| for the main file
and to |draft| for child files
which is a natural assignment for editing the document.
This is achieved by placing the following code
in the preamble of the main document
(below the |\childdocmain| directive):
%
\begin{center}
\begin{tabular}{l}
|\ifchilddoc|\\
|\providecommand{\version}{draft}|\\
|\||else|\\
|\providecommand{\version}{final}|\\
|\||fi|
\end{tabular}
\end{center}
%
The definition by |\providecommand| makes sure
that previous definitions are not overwritten.
Further statements |\providecommand{\version}{...}|
can thus be added before the above code to override it.

For the main file, one might add a line
(between |\childdocmain| and the above block)
%
\begin{center}
|%\ifchilddoc\||else\providecommand{\version}{draft}\||fi|
\end{center}
%
which can be uncommented to produce a draft version.
Likewise one can add a line to the very top of a child file
(above the |\childdocof{|\textit{main}|}| directive)
%
\begin{center}
|%\providecommand{\version}{final}|
\end{center}
%
which can be uncommented to produce the final version of this child document.

%%%%%%%%%%%%%%%%%%%%%%%%%%%%%%%%%%%%%%%%%%%%%%%%%%%%%%%%%%%%%%%%%%%%%%%%%%%%%%%%
\subsection{Forwarding}
\label{sec:forward}

Different versions of the main or child documents
using compilation flags as described in \secref{sec:flags}
can be (permanently) stored in different files
for convenient compilation, viewing and distribution.
To this end, the package defines a command
to pass on compilation to a different file:

%%%%%%%%%%%%%%%%%%%%%%%%%%%%%%%%%%%%%%%%
\DescribeMacro{\childdocforward}
The command |\childdocforward| redirects processing to
another source file:
%
\begin{center}
\begin{tabular}{l}
|\input{childdoc.def}|\\
|\childdocforward[|\textit{main}|]{|\textit{dest}|}|\\
\end{tabular}
\end{center}
%
The argument \textit{dest} is the destination file
(without extension).
It should be the main file or one of the child files.
Note that further \textsf{childdoc} directives
such as |\childdocof| and |\childdocforward|
in the indicated file will be processed in this form.
The optional argument \textit{main}
passes on directly to the main file \textit{main}
while pretending to compile the child \textit{dest}.
This form behaves as if \textit{dest}
issues |\childdocof{|\textit{main}|}| right away,
and no further \textsf{childdoc} directives will be processed.

%%%%%%%%%%%%%%%%%%%%%%%%%%%%%%%%%%%%%%%%
\DescribeMacro{\...prefix}
In the alternative form |\childdocforwardprefix|,
%
\begin{center}
\begin{tabular}{l}
|\input{childdoc.def}|\\
|\childdocforwardprefix[|\textit{main}|]{|\textit{prefix}|}{|\textit{dest}|}|
\end{tabular}
\end{center}
%
the destination file is determined by a pattern
depending on the current file:
To make this work, the current file must be called
`{\textit{prefix}\hspace{0.2em}\textit{suffix}}'
with \textit{prefix} matching precisely the argument.
Processing is then passed on to the file
`{\textit{dest}\hspace{0.2em}\textit{suffix}}'.
Surely, the same effect is achieved by
directly specifying the
argument `{\textit{dest}\hspace{0.2em}\textit{suffix}}'
in the first form.
However, that requires to set up a different file
for each child. With the alternative form of the command
all these files can have exactly the same content
which simplifies setting them up and maintaining them.

For example, the following file |draft.tex|
with a compilation flag |\version| as described in \secref{sec:flags}
compiles the main document as a draft:
%
\begin{center}
\begin{tabular}{l}
|\def\version{draft}|\\
|\input{childdoc.def}|\\
|\childdocforward{|\textit{main}|}|
\end{tabular}
\end{center}
%
Likewise, the following files |final|\textit{nn}|.tex|
compile the final version of the child document
|child|\textit{nn}|.tex|:
%
\begin{center}
\begin{tabular}{l}
|\def\version{final}|\\
|\input{childdoc.def}|\\
|\childdocforwardprefix{final}{child}|
\end{tabular}
\end{center}
%

Note that when several versions of a main file and/or of each child file
are to be generated, it may be convenient to set up a |Makefile| or
shell script to automatise the process.

%%%%%%%%%%%%%%%%%%%%%%%%%%%%%%%%%%%%%%%%%%%%%%%%%%%%%%%%%%%%%%%%%%%%%%%%%%%%%%%%
\subsection{Command Line Processing}
\label{sec:commandline}

The effect of redirection files can also be achieved by invoking
the \LaTeX{} compiler with a more elaborate command line.
Most conveniently this should be done as part
of a shell script or a |Makefile|.

When using \textsf{childdoc} in the main file, the following
command lines effectively perform a redirection
(note that depending on the shell being used,
backslashes may have to be doubled: `|\|' $\to$ `|\\|'):
%
\begin{center}
|... -jobname "|\textit{target}|" |\\|"|[\textit{flags}]%
|\input{childdoc.def}\childdocforward[|\textit{main}|]{|\textit{dest}|}"|
\end{center}
%
Here \textit{target} is the name of the output file,
\textit{main} is the name of the main file
and \textit{dest} is the name of the main or child file to be processed
(all filenames without extensions).
The optional argument \textit{main} can be omitted
if \textit{main} matches \textit{dest}.
Optionally, compilation \textit{flags} can be defined via |\def| commands.
This command line makes the \TeX{} engine believe
it is compiling the file \textit{target}
whose content is specified as the latter parameter.
The provided code then forwards the processing to
\textit{main} or \textit{dest} as described in \secref{sec:forward}.

%%%%%%%%%%%%%%%%%%%%%%%%%%%%%%%%%%%%%%%%%%%%%%%%%%%%%%%%%%%%%%%%%%%%%%%%%%%%%%%%
\subsection{Include by Input}
\label{sec:input}

Including child documents by |\include| has some restrictions by design.
Most notably, the content of a child document always occupies
its own set of pages; pages cannot be shared between child documents.
Usually, this behaviour makes perfect sense
because each child document contain an essential part of the document.
However, in some situations it may be desirable to compose
a document from a collection of parts
without having mandatory page breaks between then.
For this case, the package
provides a mechanism to include parts
by |\input| which can also be processed individually.
However, by construction this mechanism
requires manual handling of the content to be output.

%%%%%%%%%%%%%%%%%%%%%%%%%%%%%%%%%%%%%%%%
\DescribeMacro{\ifchilddocmanual}
The main file should be prepared as usual, see \secref{sec:include}.
However, the document body must make a distinction
between processing of an individual part and of the main document, e.g.:
%
\begin{center}
\begin{tabular}{l}
|\ifchilddocmanual|\\
|\input{\childdocname}|\\
|\||else|\\
\textit{document body with }|\input{|\textit{part}|}|\\
|\||fi|
\end{tabular}
\end{center}
%
The conditional |\ifchilddocmanual| is true whenever
a part to be included by |\input| is being compiled,
and the name of the part is stored in |\childdocname|.

%%%%%%%%%%%%%%%%%%%%%%%%%%%%%%%%%%%%%%%%
\DescribeMacro{\childdocby}
Each part to be included by |\input| should start with:
%
\begin{center}
\begin{tabular}{l}
|\input{childdoc.def}|\\
|\childdocby{|\textit{main}|}|\\
\end{tabular}
\end{center}
%
The directive |\childdocby| is similar to |\childdocof|
described in \secref{sec:include},
but the subsequent selection of content must be done manually.
To that end, both |\ifchilddoc| and |\ifchilddocmanual|
will be true upon processing of a part,
and the name of the part is stored in |\childdocname|.
Note that |\jobname| will be set to the filename of the current part
so that each part receives an individual |.aux| file
that does not interfere with the |.aux| file(s) of the main document.
This behaviour can be altered by the alternative form
|\childdocby[*]{|\textit{main}|}| (with a non-empty optional argument)
which uses the |.aux| file of the main document
by setting |\jobname| to \textit{main}.

%%%%%%%%%%%%%%%%%%%%%%%%%%%%%%%%%%%%%%%%%%%%%%%%%%%%%%%%%%%%%%%%%%%%%%%%%%%%%%%%
\subsection{Driver Development}
\label{sec:driver}

The \textsf{childdoc} mechanism can also be use for the development
of definition files such as \LaTeX{} styles or classes.
This case differs from the above setup with multiple parts
included by |\include| in that no |\includeonly| should be invoked.
This can be achieved by starting the include file
(before |\ProvidesPackage|) with:
%
\begin{center}
\begin{tabular}{l}
|\input{childdoc.def}|\\
|\childdocforward{|\textit{main}|}|\\
\end{tabular}
\end{center}
%
or alternatively with:
%
\begin{center}
\begin{tabular}{l}
|\input{childdoc.def}|\\
|\childdocby{|\textit{main}|}|\\
\end{tabular}
\end{center}
%
Both forms have slightly different effects as described above.
The main file is prepared as usual, see \secref{sec:include}.

%%%%%%%%%%%%%%%%%%%%%%%%%%%%%%%%%%%%%%%%%%%%%%%%%%%%%%%%%%%%%%%%%%%%%%%%%%%%%%%%
\subsection{Legacy Detection}
\label{sec:detection}

The directive |\childdocmain| in the main file can detect
whether the complete document or merely a child is to be compiled
even without using the directive |\childdocof|.
This method is deprecated because it is less robust
and there is no compelling reason to use it;
it is merely provided for backward compatibility
and it may be removed in future versions.

If the detection mechanism is to be used,
it is mandatory to correctly specify
the filename of the main file as the argument of |\childdocmain|:
%
\begin{center}
\begin{tabular}{l}
|\input{childdoc.def}|\\
|\childdocmain{|\textit{main}|}|\\
\end{tabular}
\end{center}
%
If |\jobname| does not match the argument \textit{main} of |\childdocmain|,
it is assumed that |\jobname| points to the child file to be compiled.
When using |\childdocmain| with the main file specified as argument,
it suffices to start a child file
with just |\input{|\textit{main}|}|
without loading of the package and using |\childdocof|.
If instead all processing is done
with the appropriate \textsf{childdoc} directives,
the argument of \textit{main} of |\childdocmain| can be empty.

An alternative version of the command line processing described
in \secref{sec:commandline} using the detection mechanism reads:
%
\begin{center}
|... -jobname "|\textit{target}|" "|[\textit{flags}]%
[|\def\jobname{|\textit{dest}|}|]|\input{|\textit{main}|}"|
\end{center}

%%%%%%%%%%%%%%%%%%%%%%%%%%%%%%%%%%%%%%%%%%%%%%%%%%%%%%%%%%%%%%%%%%%%%%%%%%%%%%%%
\subsection{Manual Code}
\label{sec:manual}

In case one cannot be certain whether the definitions file |childdoc.def|
is installed on the target \TeX{} distribution
and one prefers not to ship it,
it is conceivable to paste a few relevant commands into the sources.

To that end, drop all statements |\input{childdoc.def}|
and perform the replacements as outlined below.
Instead of |\childdocmain{|\textit{main}|}| add the following code
to the top of the main file:
%
\begin{center}
\begin{tabular}{l}
|\||ifdefined\childdocname\endinput\||fi\newif\ifchilddoc|\\
|\edef\childdocname{\scantokens\expandafter{\jobname\noexpand}}|\\
|\def\childdocmain{|\textit{main}|}\||ifx\childdocmain\childdocname\||else|\\
|\childdoctrue\includeonly{\childdocname}\let\jobname\childdocmain\||fi|\\
\end{tabular}
\end{center}
%
Instead of |\childdocof{|\textit{main}|}| just include the main file
at the top of each child file:
%
\begin{center}
|\input{|\textit{main}|}|
\end{center}
%
A simple redirection |\childdocforward{|\textit{dest}|}| is achieved by:
%
\begin{center}
|\def\jobname{|\textit{dest}|}\input{\jobname}|
\end{center}
%
The redirection with prefix
|\childdocforwardprefix[|\textit{prefix}|]{|\textit{dest}|}|
is accomplished by:
%
\begin{center}
\begin{tabular}{l}
|{\edef\jobname{\scantokens\expandafter{\jobname\noexpand}}|\\
|\def\redirectjob |\textit{prefix}|#1~~~{\gdef\jobname{|\textit{dest}|#1}}|\\
|\expandafter\redirectjob\jobname~~~}\input{\jobname}|
\end{tabular}
\end{center}

In an alternative approach,
child documents can be compiled by a specific command line
without additional code or specific definitions:
%
\begin{center}
|... -jobname "|\textit{target}|" "|[\textit{flags}]%
|\includeonly{|\textit{dest}|}\input{|\textit{main}|}"|
\end{center}
%

%%%%%%%%%%%%%%%%%%%%%%%%%%%%%%%%%%%%%%%%%%%%%%%%%%%%%%%%%%%%%%%%%%%%%%%%%%%%%%%%
%%%%%%%%%%%%%%%%%%%%%%%%%%%%%%%%%%%%%%%%%%%%%%%%%%%%%%%%%%%%%%%%%%%%%%%%%%%%%%%%
\section{Information}

%%%%%%%%%%%%%%%%%%%%%%%%%%%%%%%%%%%%%%%%%%%%%%%%%%%%%%%%%%%%%%%%%%%%%%%%%%%%%%%%
\subsection{Copyright}

Copyright \copyright{} 2017--2018 Niklas Beisert

This work may be distributed and/or modified under the
conditions of the \LaTeX{} Project Public License, either version 1.3
of this license or (at your option) any later version.
The latest version of this license is in
  \url{http://www.latex-project.org/lppl.txt}
and version 1.3 or later is part of all distributions of \LaTeX{}
version 2005/12/01 or later.

This work has the LPPL maintenance status `maintained'.

The Current Maintainer of this work is Niklas Beisert.

This work consists of the files |README.txt|, |childdoc.ins| and |childdoc.dtx|
as well as the derived files |childdoc.def|, |cdocsamp.tex|
with |cdocsch1.tex|, |cdocsch2.tex|, |cdocspt3.tex|, |cdocspt4.tex|,
|cdocsdrf.tex|, |cdocsfn1.tex|, |cdocsfn2.tex|
as well as |childdoc.pdf|.

%%%%%%%%%%%%%%%%%%%%%%%%%%%%%%%%%%%%%%%%%%%%%%%%%%%%%%%%%%%%%%%%%%%%%%%%%%%%%%%%
\subsection{Files and Installation}

The package consists of the files:
%
\begin{center}
\begin{tabular}{ll}
    |README.txt|   & readme file \\
    |childdoc.ins| & installation file \\
    |childdoc.dtx| & source file \\
    |childdoc.def| & definition file \\
    |cdocsamp.tex| & sample main file \\
    |cdocsch1.tex| & sample include file \\
    |cdocsch2.tex| & sample include file \\
    |cdocspt3.tex| & sample part file \\
    |cdocspt4.tex| & sample part file \\
    |cdocsdrf.tex| & sample redirection file \\
    |cdocsfn1.tex| & sample redirection file \\
    |cdocsfn2.tex| & sample redirection file \\
    |childdoc.pdf| & manual
\end{tabular}
\end{center}
%
The distribution consists of the files
|README.txt|, |childdoc.ins| and |childdoc.dtx|.
%
\begin{itemize}
\item
Run (pdf)\LaTeX{} on |childdoc.dtx|
to compile the manual |childdoc.pdf| (this file).
\item
Run \LaTeX{} on |childdoc.ins| to create the definitions file |childdoc.def|
and the sample |cdocsamp.tex| with include files
|cdocsch1.tex|, |cdocsch2.tex|, |cdocspt3.tex|, |cdocspt4.tex|,
|cdocsdrf.tex|, |cdocsfn1.tex|, |cdocsfn2.tex|.
Then copy the file |childdoc.def| to an appropriate directory of your \LaTeX{}
distribution, e.g.\ \textit{texmf-root}|/tex/latex/childdoc|.
\end{itemize}

%%%%%%%%%%%%%%%%%%%%%%%%%%%%%%%%%%%%%%%%%%%%%%%%%%%%%%%%%%%%%%%%%%%%%%%%%%%%%%%%
\subsection{Related CTAN Packages}

There are several other packages which offer a similar functionality:
%
\begin{itemize}
\item
The packages
\href{http://ctan.org/pkg/docmute}{\textsf{docmute}},
\href{http://ctan.org/pkg/includex}{\textsf{includex}} and
\href{http://ctan.org/pkg/standalone}{\textsf{standalone}}
provide commands to include only the document body of
a child file thus allowing both files to be compiled individually.
\item
The packages \href{http://ctan.org/pkg/subdocs}{\textsf{subdocs}}
and \href{http://ctan.org/pkg/subfiles}{\textsf{subfiles}}
provide structures in which the main and child documents can be
encapsulated and allowing them to be compiled individually.
The inclusion mechanism is different from the conventional |\include|.
\item
The package \href{http://ctan.org/pkg/combine}{\textsf{combine}}
is an elaborate solution to combine several documents into one.
\end{itemize}
%
See also the CTAN topic \href{http://ctan.org/topic/subdocs}{\textsf{subdocs}}
for further related packages.
The present package differs from the above solutions in that
a document structure constructed with the conventional |\include| mechanism
just needs two extra commands at the top of every file
such that all constituent files can be compiled individually.

%%%%%%%%%%%%%%%%%%%%%%%%%%%%%%%%%%%%%%%%%%%%%%%%%%%%%%%%%%%%%%%%%%%%%%%%%%%%%%%%
%\subsection{Feature Suggestions}
%
%The following is a list of features which may be useful for future
%versions of this package:
%%
%\begin{itemize}
%\item
%\ldots
%\end{itemize}

%%%%%%%%%%%%%%%%%%%%%%%%%%%%%%%%%%%%%%%%%%%%%%%%%%%%%%%%%%%%%%%%%%%%%%%%%%%%%%%%
\subsection{Revision History}

%%%%%%%%%%%%%%%%%%%%%%%%%%%%%%%%%%%%%%%%
\paragraph{v2.0:} 2018/12/30

\begin{itemize}
\item
immediate forward processing
\item
added |\childdocby| mechanism
\item
manual restructured
\end{itemize}

%%%%%%%%%%%%%%%%%%%%%%%%%%%%%%%%%%%%%%%%
\paragraph{v1.6:} 2018/01/17

\begin{itemize}
\item
application for development of include files
\item
corrections to manual
\end{itemize}

%%%%%%%%%%%%%%%%%%%%%%%%%%%%%%%%%%%%%%%%
\paragraph{v1.5:} 2017/05/21

\begin{itemize}
\item
more complete structuring introduced
\item
|\childdocof| introduced
\item
|\childdoc| renamed to |\childdocmain|
\item
|\childredirect| renamed to |\childdocforward| and |\childdocforwardprefix|
and functionality expanded
\end{itemize}

%%%%%%%%%%%%%%%%%%%%%%%%%%%%%%%%%%%%%%%%
\paragraph{v1.0:} 2017/04/27

\begin{itemize}
\item
manual and install package
\item
first version published on CTAN
\end{itemize}

%%%%%%%%%%%%%%%%%%%%%%%%%%%%%%%%%%%%%%%%
\paragraph{v0.6:} 2017/04/26

\begin{itemize}
\item
redirection mechanism added
\end{itemize}

%%%%%%%%%%%%%%%%%%%%%%%%%%%%%%%%%%%%%%%%
\paragraph{v0.5:} 2017/04/26

\begin{itemize}
\item
functionality in definition file
\end{itemize}


%%%%%%%%%%%%%%%%%%%%%%%%%%%%%%%%%%%%%%%%%%%%%%%%%%%%%%%%%%%%%%%%%%%%%%%%%%%%%%%%
%%%%%%%%%%%%%%%%%%%%%%%%%%%%%%%%%%%%%%%%%%%%%%%%%%%%%%%%%%%%%%%%%%%%%%%%%%%%%%%%
%%%%%%%%%%%%%%%%%%%%%%%%%%%%%%%%%%%%%%%%%%%%%%%%%%%%%%%%%%%%%%%%%%%%%%%%%%%%%%%%
\appendix

\settowidth\MacroIndent{\rmfamily\scriptsize 000\ }

 \DocInput{childdoc.dtx}

\end{document}
%</driver>
% \fi
%
% %%%%%%%%%%%%%%%%%%%%%%%%%%%%%%%%%%%%%%%%%%%%%%%%%%%%%%%%%%%%%%%%%%%%%%%%%%%%%%
% %%%%%%%%%%%%%%%%%%%%%%%%%%%%%%%%%%%%%%%%%%%%%%%%%%%%%%%%%%%%%%%%%%%%%%%%%%%%%%
% \section{Sample}
%\iffalse
%<*samplemain>
%\fi
%
% The following presents a sample document
% with two chapters, two parts, a title page,
% a compile flag as well as three forwarding files to set the flag.
% It consists of eight |.tex| files:
% \begin{center}
% \begin{tabular}{ll}
% |cdocsamp.tex|&main file\\
% |cdocsch1.tex|&include file for chapter 1\\
% |cdocsch2.tex|&include file for chapter 2\\
% |cdocspt3.tex|&include file for part 3\\
% |cdocspt4.tex|&include file for part 4\\
% |cdocsdrf.tex|&forwarding file for main file in draft mode\\
% |cdocsfi1.tex|&forwarding file for final version of chapter 1\\
% |cdocsfi2.tex|&forwarding file for final version of chapter 2\\
% \end{tabular}
% \end{center}
% Each of the eight files can be compiled directly by the \LaTeX{} compiler.
%
% %%%%%%%%%%%%%%%%%%%%%%%%%%%%%%%%%%%%%%
% \paragraph{Main File.}
%
% The main file is called |cdocsamp.tex|.
%
% Load the \textsf{childdoc} definitions and
% declare the filename for the main document:
%    \begin{macrocode}
\input{childdoc.def}
\childdocmain{}
%    \end{macrocode}

% Optional override for |\version| flag:
%    \begin{macrocode}
%%\ifchilddoc\else\providecommand{\version}{draft}\fi
%    \end{macrocode}

% Define the default values for the |\version| flag
% (|final| for the main file and |draft| for childs):
%    \begin{macrocode}
\ifchilddoc
\providecommand{\version}{draft}
\else
\providecommand{\version}{final}
\fi
%    \end{macrocode}

% Load the standard document class:
%    \begin{macrocode}
\documentclass[12pt]{article}
%    \end{macrocode}

% Start the document body:
%    \begin{macrocode}
\begin{document}
%    \end{macrocode}

% Declare a title page.
% Print title, part of document being processed and version flag:
%    \begin{macrocode}
\addtocounter{page}{-1}
\begin{center}
{\LARGE\bfseries{}childdoc example\par}
\vspace{1cm}
\ifchilddoc
\ifchilddocmanual part\else chapter\fi:
`\childdocname' of `\childdocjob'\par
\else
main document: `\childdocjob'\par
\fi
version: \version\par
\end{center}
\newpage
%    \end{macrocode}

% Manually include selected file,
% otherwise process as usual:
%    \begin{macrocode}
\ifchilddocmanual
\section*{part `\childdocname'}
\input{\childdocname}
\else
%    \end{macrocode}

% Include the two chapters:
%    \begin{macrocode}
\include{cdocsch1}
\include{cdocsch2}
%    \end{macrocode}

% Include the two parts unless only chapters should be displayed:
%    \begin{macrocode}
\ifchilddoc\else
\section{part three}
\input{cdocspt3}
\section{part four}
\input{cdocspt4}
\fi
%    \end{macrocode}

% Process as usual until here:
%    \begin{macrocode}
\fi
%    \end{macrocode}

% End of document body:
%    \begin{macrocode}
\end{document}
%    \end{macrocode}
%\iffalse
%</samplemain>
%\fi
%
% %%%%%%%%%%%%%%%%%%%%%%%%%%%%%%%%%%%%%%
% \paragraph{Chapter Include Files.}
%
% The include files are called |cdocsch1.tex| and |cdocsch2.tex|.
%
%\iffalse
%<*samplechap1|samplechap2>
%\fi

% Optional override for |\version| flag:
%    \begin{macrocode}
%%\providecommand{\version}{final}
%    \end{macrocode}

% Include the main document:
%    \begin{macrocode}
\input{childdoc.def}
\childdocof{cdocsamp}
%    \end{macrocode}

%\iffalse
%</samplechap1|samplechap2>
%\fi
%
%\iffalse
%<*samplechap1>
%\fi
% Some text for chapter 1:
%    \begin{macrocode}
\section{one}
some text in chapter one
%    \end{macrocode}

%\iffalse
%</samplechap1>
%\fi
% Some text for chapter 2:
%\iffalse
%<*samplechap2>
%\fi
%    \begin{macrocode}
\section{two}
more text in chapter two
%    \end{macrocode}

%\iffalse
%</samplechap2>
%\fi
%
% %%%%%%%%%%%%%%%%%%%%%%%%%%%%%%%%%%%%%%
% \paragraph{Part Include Files.}
%
% The include files are called |cdocspt3.tex| and |cdocspt4.tex|.
%
%\iffalse
%<*samplepart3|samplepart4>
%\fi

% Optional override for |\version| flag:
%    \begin{macrocode}
%%\providecommand{\version}{final}
%    \end{macrocode}

% Include the main document:
%    \begin{macrocode}
\input{childdoc.def}
\childdocby{cdocsamp}
%    \end{macrocode}

%\iffalse
%</samplepart3|samplepart4>
%\fi
%
%\iffalse
%<*samplepart3>
%\fi
% Some text for part 3:
%    \begin{macrocode}
some text in part three
%    \end{macrocode}

%\iffalse
%</samplepart3>
%\fi
% Some text for part 4:
%\iffalse
%<*samplepart4>
%\fi
%    \begin{macrocode}
more text in part four
%    \end{macrocode}

%\iffalse
%</samplepart4>
%\fi
%
% %%%%%%%%%%%%%%%%%%%%%%%%%%%%%%%%%%%%%%
% \paragraph{Forwarding for a Complete Draft.}
%
% The following forwarding file |cdocsdrf.tex|
% compiles the main document in draft mode:
%\iffalse
%<*sampledraft>
%\fi
%    \begin{macrocode}
\def\version{draft}
\input{childdoc.def}
\childdocforward{cdocsamp}
%    \end{macrocode}

%\iffalse
%</sampledraft>
%\fi
%
% %%%%%%%%%%%%%%%%%%%%%%%%%%%%%%%%%%%%%%
% \paragraph{Forwarding for Final Version of the Chapters.}
%
% The following forwarding files |cdocsfn1.tex| and |cdocsfn2.tex|
% (with identical content)
% compile the final versions of the child documents
% |cdocsch1.tex| and |cdocsch2.tex|, respectively:
%\iffalse
%<*samplefinal>
%\fi
%    \begin{macrocode}
\def\version{final}
\input{childdoc.def}
\childdocforwardprefix[cdocsamp]{cdocsfn}{cdocsch}
%    \end{macrocode}

%\iffalse
%</samplefinal>
%\fi
%
% %%%%%%%%%%%%%%%%%%%%%%%%%%%%%%%%%%%%%%
% \paragraph{Command Line Processing.}
%
% The following three command lines generate the output files
% |cdocscld|, |cdocscl1| and |cdocscl2|
% which should be identical to
% |cdocsdrf|, |cdocsch1| and |cdocsfn2|, respectively:
% \begin{center}
% \begin{tabular}{l}
% |latex -jobname cdocscld \|\\
% |  "\def\version{draft}\input{childdoc.def}\childdocforward{cdocsamp}"|\\
% |latex -jobname cdocscl1 \|\\
% |  "\input{childdoc.def}\childdocforward[cdocsamp]{cdocsch1}"|\\
% |latex -jobname cdocscl2 \|\\
% |  "\def\version{final}\input{childdoc.def}\childdocforward{cdocsch2}"|
% \end{tabular}
% \end{center}
% Note that the trailing backslash on each first line
% merely continues the input to the second line
% (for convenient cut ant paste).
% Furthermore, the command |latex| can be replaced by any
% of its alternative versions such as |pdflatex|.
%
% %%%%%%%%%%%%%%%%%%%%%%%%%%%%%%%%%%%%%%%%%%%%%%%%%%%%%%%%%%%%%%%%%%%%%%%%%%%%%%
% %%%%%%%%%%%%%%%%%%%%%%%%%%%%%%%%%%%%%%%%%%%%%%%%%%%%%%%%%%%%%%%%%%%%%%%%%%%%%%
% \section{Implementation}
%\iffalse
%<*package>
%\fi
%
% This section describes the definitions file |childdoc.def|.

% The definitions cannot be loaded using |\usepackage| or |\RequirePackage|
% which has a mechanism to prevent loading a style file more than once.
% When loading the definitions by means of |\input|
% multiple instances have to be prevented manually:
%\iffalse
%This code needs to be before the `\ProvidesFile' directive
%which is defined at the beginning of this file.
%Therefore it is also placed there and commented out here.
%</package>
%<*discard>
%\fi
%    \begin{macrocode}
\ifdefined\childdocmain\endinput\fi
%    \end{macrocode}
%\iffalse
%</discard>
%<*package>
%\fi
%
% \macro{\ifchilddoc}
% \macro{\ifchilddocmanual}
% The conditional |\ifchilddoc| tells whether a
% child (true) or main (false) document is being compiled.
% The conditional |\ifchilddocmanual| tells whether
% the |\includeonly| mechanism is used (false) or
% the selection of child files must be performed manually (true).
% The definitions initialise to false:
%    \begin{macrocode}
\newif\ifchilddoc
\newif\ifchilddocmanual
%    \end{macrocode}

% \macro{\childdocname}
% \macro{\childdocjob}
% The macro |\childdocname| stores the name of the main document
% to be compiled. The macro |\childdocjob| stores the name of
% the document on which the \LaTeX{} compiler was originally invoked.
% The content of |\jobname| cannot be compared
% to filenames specified in the source due to different catcodes.
% The following code rescans |\jobname|, stores the result
% in |\childdocname| and saves a copy in |\childdocjob|:
%    \begin{macrocode}
\edef\childdocname{\scantokens\expandafter{\jobname\noexpand}}
\let\childdocjob\childdocname
%    \end{macrocode}

% \macro{\childdocdisable}
% The macro |\childdocdisable| prevents the main file
% from being processed more than once.
% At this stage, the main document command |\childdocmain|
% is assumed to be called once again where it should do nothing.
% Any subsequent call to it should prevent
% a secondary processing of the main document
% It overwrites the forwarding commands
% |\childdocof| and |\childdocforward|
% with empty macros to prevent further inclusions of the main document:
%    \begin{macrocode}
\newcommand{\childdocdisable}
{
  \renewcommand{\childdocmain}[1]{\renewcommand{\childdocmain}[1]{\endinput}}
  \renewcommand{\childdocof}[1]{}
  \renewcommand{\childdocby}[2][]{}
  \renewcommand{\childdocforward}[2][]{}
  \renewcommand{\childdocdisable}{}
}
%    \end{macrocode}

% \macro{\childdocmain}
% The macro |\childdocmain| is to be called at the top of the main file
% with nothing or the main filename (without extension) as argument.
% First, it breaks loops.
% If the argument is not empty and does not match |\childdocname|
% (which is set by the first inclusion of |childdoc.def|),
% |\ifchilddoc| is set to true, |\includeonly| is applied to the child file
% and |\jobname| is set to the main file
% (for proper handling of |.aux| files):
%    \begin{macrocode}
\newcommand{\childdocmain}[1]
{
  \childdocdisable\childdocmain{}
  \if?#1?\else
    \begingroup
      \def\childdoctmp{#1}
      \ifx\childdoctmp\childdocname
        \def\childdoctmp{}
      \else
        \def\childdoctmp
        {
          \childdoctrue
          \includeonly{\childdocname}
          \def\childdocjob{#1}
          \def\jobname{#1}
        }
      \fi
      \expandafter
    \endgroup
    \childdoctmp
  \fi
}
%    \end{macrocode}

% \macro{\childdocof}
% The command |\childdocof| redirects
% compilation to the main file |#1|.
%    \begin{macrocode}
\newcommand{\childdocof}[1]
{
  \childdocdisable
  \childdoctrue
  \includeonly{\childdocname}
  \def\jobname{#1}
  \def\childdocjob{#1}
  \input{#1}
}
%    \end{macrocode}

% \macro{\childdocby}
% The command |\childdocby| ....
%    \begin{macrocode}
\newcommand{\childdocby}[2][]
{
  \childdocdisable
  \childdoctrue
  \childdocmanualtrue
  \if?#1?\else
    \def\jobname{#2}
  \fi
  \def\childdocjob{#2}
  \input{#2}
  \endinput
}
%    \end{macrocode}

% \macro{\childdocforward}
% The command |\childdocforward| redirects
% compilation to the main file or
% (if the optional argument is given) a child file.
% Parameters are set as if the main file
% or a child file starting with |\childdocof| was compiled.
% Then compilation is handed over to the main file:
%    \begin{macrocode}
\newcommand{\childdocforward}[2][]
{
  \begingroup
    \if?#1?
      \def\childdoctmp
      {
        \def\childdocname{#2}
        \def\childdocjob{#2}
        \def\jobname{#2}
        \input{#2}
        \endinput
      }
    \else
      \def\childdoctmp
      {
        \childdocdisable
        \def\childdocname{#2}
        \childdoctrue
        \includeonly{#2}
        \def\childdocjob{#1}
        \def\jobname{#1}
        \input{#1}
        \endinput
      }
    \fi
    \expandafter
  \endgroup
  \childdoctmp
}
%    \end{macrocode}

% \macro{\childdocforwardprefix}
% The command |\childdocforwardprefix| redirects
% compilation to the main or a child file by means of a pattern.
% The prefix |#1| in the current filename is replaced by |#2|
% and the suffix of the current filename is kept
% (it is assumed that the filename does not contain the substring `|~~~|'
% which is used as a delimiter).
% Compilation is handed over to the new file by |\childdocforward|:
%    \begin{macrocode}
\newcommand{\childdocforwardprefix}[3][]
{
  \begingroup
    \def\childdocextract #2##1~~~{\def\childdoctmp{\childdocforward[#1]{#3##1}}}
    \expandafter\childdocextract\childdocname~~~
    \expandafter
  \endgroup
  \childdoctmp
}
%    \end{macrocode}

% \macro{\childdoc}
% The deprecated macro |\childdoc| is a legacy version of |\childdocmain|:
%    \begin{macrocode}
\newcommand{\childdoc}{\childdocmain}
%    \end{macrocode}

% \macro{\childdocredirect}
% The deprecated macro |\childdocredirect| is a legacy version
% of |\childdocforward| and |\childdocforwardprefix|:
%    \begin{macrocode}
\newcommand{\childdocredirect}[2][]
{
  \begingroup
    \if?#1?
      \def\childdoctmp{\childdocforward{#2}}
    \else
      \def\childdoctmp{\childdocforwardprefix{#1}{#2}}
    \fi
    \expandafter
  \endgroup
  \childdoctmp
}
%    \end{macrocode}

%\iffalse
%</package>
%\fi
%
\endinput
\childdocforward[|\textit{main}|]{|\textit{dest}|}"|
\end{center}
%
Here \textit{target} is the name of the output file,
\textit{main} is the name of the main file
and \textit{dest} is the name of the main or child file to be processed
(all filenames without extensions).
The optional argument \textit{main} can be omitted
if \textit{main} matches \textit{dest}.
Optionally, compilation \textit{flags} can be defined via |\def| commands.
This command line makes the \TeX{} engine believe
it is compiling the file \textit{target}
whose content is specified as the latter parameter.
The provided code then forwards the processing to
\textit{main} or \textit{dest} as described in \secref{sec:forward}.

%%%%%%%%%%%%%%%%%%%%%%%%%%%%%%%%%%%%%%%%%%%%%%%%%%%%%%%%%%%%%%%%%%%%%%%%%%%%%%%%
\subsection{Include by Input}
\label{sec:input}

Including child documents by |\include| has some restrictions by design.
Most notably, the content of a child document always occupies
its own set of pages; pages cannot be shared between child documents.
Usually, this behaviour makes perfect sense
because each child document contain an essential part of the document.
However, in some situations it may be desirable to compose
a document from a collection of parts
without having mandatory page breaks between then.
For this case, the package
provides a mechanism to include parts
by |\input| which can also be processed individually.
However, by construction this mechanism
requires manual handling of the content to be output.

%%%%%%%%%%%%%%%%%%%%%%%%%%%%%%%%%%%%%%%%
\DescribeMacro{\ifchilddocmanual}
The main file should be prepared as usual, see \secref{sec:include}.
However, the document body must make a distinction
between processing of an individual part and of the main document, e.g.:
%
\begin{center}
\begin{tabular}{l}
|\ifchilddocmanual|\\
|\input{\childdocname}|\\
|\||else|\\
\textit{document body with }|\input{|\textit{part}|}|\\
|\||fi|
\end{tabular}
\end{center}
%
The conditional |\ifchilddocmanual| is true whenever
a part to be included by |\input| is being compiled,
and the name of the part is stored in |\childdocname|.

%%%%%%%%%%%%%%%%%%%%%%%%%%%%%%%%%%%%%%%%
\DescribeMacro{\childdocby}
Each part to be included by |\input| should start with:
%
\begin{center}
\begin{tabular}{l}
|% \iffalse
%
% childdoc.dtx Copyright (C) 2017-2018 Niklas Beisert
%
% This work may be distributed and/or modified under the
% conditions of the LaTeX Project Public License, either version 1.3
% of this license or (at your option) any later version.
% The latest version of this license is in
%   http://www.latex-project.org/lppl.txt
% and version 1.3 or later is part of all distributions of LaTeX
% version 2005/12/01 or later.
%
% This work has the LPPL maintenance status `maintained'.
%
% The Current Maintainer of this work is Niklas Beisert.
%
% This work consists of the files childdoc.dtx and childdoc.ins
% and the derived files childdoc.def and cdocsamp.tex with
% cdocsch1.tex, cdocsch2.tex, cdocsdrf.tex, cdocsfn1.tex, cdocsfn2.tex.
%
%<package>\ifdefined\childdocmain\endinput\fi
%<package>\ProvidesFile{childdoc.def}[2018/12/30 v2.0 child document driver]
%<samplemain>\ProvidesFile{cdocsamp.tex}[2018/12/30 v2.0 sample for childdoc]
%<*driver>
%\ProvidesFile{childdoc.drv}[2018/12/30 v2.0 childdoc reference manual file]
\PassOptionsToClass{10pt,a4paper}{article}
\documentclass{ltxdoc}

\usepackage[margin=35mm]{geometry}
\usepackage{hyperref}
\usepackage{hyperxmp}
\usepackage[usenames]{color}

\hypersetup{colorlinks=true}
\hypersetup{pdfstartview=FitH}
\hypersetup{pdfpagemode=UseNone}
\hypersetup{pdfsource={}}
\hypersetup{pdflang={en-UK}}
\hypersetup{pdfcopyright={Copyright 2017-2018 Niklas Beisert.
  This work may be distributed and/or modified under the
  conditions of the LaTeX Project Public License, either version 1.3
  of this license or (at your option) any later version.}}
\hypersetup{pdflicenseurl={http://www.latex-project.org/lppl.txt}}
\hypersetup{pdfcontactaddress={ETH Zurich, ITP, HIT K,
  Wolfgang-Pauli-Strasse 27}}
\hypersetup{pdfcontactpostcode={8093}}
\hypersetup{pdfcontactcity={Zurich}}
\hypersetup{pdfcontactcountry={Switzerland}}
\hypersetup{pdfcontactemail={nbeisert@itp.phys.ethz.ch}}
\hypersetup{pdfcontacturl={http://people.phys.ethz.ch/\xmptilde nbeisert/}}

\newcommand{\secref}[1]{\hyperref[#1]{section \ref*{#1}}}

\parskip1ex
\parindent0pt
\let\olditemize\itemize
\def\itemize{\olditemize\parskip0pt}

\begin{document}

\title{The \textsf{childdoc} Package}
\hypersetup{pdftitle={The childdoc Package}}
\author{Niklas Beisert\\[2ex]
  Institut f\"ur Theoretische Physik\\
  Eidgen\"ossische Technische Hochschule Z\"urich\\
  Wolfgang-Pauli-Strasse 27, 8093 Z\"urich, Switzerland\\[1ex]
  \href{mailto:nbeisert@itp.phys.ethz.ch}
  {\texttt{nbeisert@itp.phys.ethz.ch}}}
\hypersetup{pdfauthor={Niklas Beisert}}
\hypersetup{pdfsubject={Manual for the LaTeX2e Package childdoc}}
\date{30 December 2018, \textsf{v2.0}}
\maketitle

\begin{abstract}\noindent
\textsf{childdoc} is a \LaTeXe{} package
that enables the direct compilation
of document sections included by |\include|
to individual files.
\end{abstract}

\begingroup
\parskip0ex
\tableofcontents
\endgroup

%%%%%%%%%%%%%%%%%%%%%%%%%%%%%%%%%%%%%%%%%%%%%%%%%%%%%%%%%%%%%%%%%%%%%%%%%%%%%%%%
%%%%%%%%%%%%%%%%%%%%%%%%%%%%%%%%%%%%%%%%%%%%%%%%%%%%%%%%%%%%%%%%%%%%%%%%%%%%%%%%
\section{Introduction}

\LaTeX{} provides a mechanism to structure a large document (such as a book)
into a main file and several child files (containing the chapters)
using the |\include| command.
This mechanism is beneficial for documents
which span hundreds of pages in order to
make the source file(s) more manageable.
Moreover, compilation can be restricted to
selected child files by means of the |\includeonly| command.
The latter feature can be used to reduce the compilation time while editing
(this was significantly more useful in the earlier days of \LaTeX{})
or to generate a smaller document which is easier to navigate.
Another application of |\includeonly| is to generate
documents consisting of selected parts of the complete document.

However, there are a few drawbacks of the plain |\include| mechanism:
\begin{itemize}
\item
The child files cannot be compiled on their own,
they can only be compiled via the main file.
A naive editing environment
(such as a text editor with an option
to have the current file processed by \LaTeX)
may require one to switch to the main file before compiling;
attempting to compile the child file produces errors.
\item
The main file must be modified (each time)
to adjust the |\includeonly| command
to the present needs. This easily leaves the main file in a messy state.
\item
The generated document will always carry the filename
of the main document. This is inconvenient if
several child files are to be compiled and
to be kept for distribution.
\end{itemize}

The present package provides a simple interface
to make child files individually compilable by \LaTeX{}.
Compiling a child file then has the same effect as compiling
the main file with an |\includeonly| command
to select the appropriate child.
Moreover the generated document will carry the name of the child
rather than the main file.
This resolves all three above issues.

This feature is meant to make the editing of books,
thesis documents and lecture notes somewhat more convenient.
However, the package can also be used efficiently for
composing a series of documents (such as exercise sheets)
which are typically distributed individually.
It then assists the author in generating the individual documents
(potentially in different versions)
as well as a document containing the collected series.
Another application is in developing style files
or other kinds of included material
where compilation of the style file could redirect
to a sample or test file.

%%%%%%%%%%%%%%%%%%%%%%%%%%%%%%%%%%%%%%%%%%%%%%%%%%%%%%%%%%%%%%%%%%%%%%%%%%%%%%%%
%%%%%%%%%%%%%%%%%%%%%%%%%%%%%%%%%%%%%%%%%%%%%%%%%%%%%%%%%%%%%%%%%%%%%%%%%%%%%%%%
\section{Usage}

First of all, the package \textsf{childdoc} is \emph{not} a standard
\LaTeXe{} |.sty| style file! Therefore it needs to be invoked in
a non-standard way.

%%%%%%%%%%%%%%%%%%%%%%%%%%%%%%%%%%%%%%%%%%%%%%%%%%%%%%%%%%%%%%%%%%%%%%%%%%%%%%%%
\subsection{Included Files}
\label{sec:include}

%%%%%%%%%%%%%%%%%%%%%%%%%%%%%%%%%%%%%%%%
\DescribeMacro{\childdocmain}
To use the package, add the commands
\begin{center}
\begin{tabular}{l}
|\input{childdoc.def}|\\
|\childdocmain{}|\\
\end{tabular}
\end{center}
at the very top of the main \LaTeX{} file,
in particular \emph{before} the |\documentclass| statement!
The argument of |\childdocmain| should be left empty
(but it must be present).

%%%%%%%%%%%%%%%%%%%%%%%%%%%%%%%%%%%%%%%%
\DescribeMacro{\childdocof}
Furthermore, add the commands
\begin{center}
\begin{tabular}{l}
|\input{childdoc.def}|\\
|\childdocof{|\textit{main}|}|\\
\end{tabular}
\end{center}
at the top of every child file \textit{child}
which is included by |\include{|\textit{child}|}|
from within the main file
(or at least for those files to be compiled individually).
The argument \textit{main} must be the filename of the main file.

There are a couple of
considerations in setting up the main and child documents:

%%%%%%%%%%%%%%%%%%%%%%%%%%%%%%%%%%%%%%%%
\paragraph{Restrictions.}

Please note the following restrictions:
\begin{itemize}
\item
|\childdocmain| must be called with one argument \textit{main}
to ensure compatibility with earlier version of the package.
It must either be empty (|\childdocmain{}|)
or precisely match the filename of the main file in which it is specified.
See \secref{sec:detection} for further information.
\item
The filename \textit{main} must be specified without the |.tex| extension.
\item
The filename \textit{main} is case sensitive
(even in case-insensitive file systems)
due to internal string comparison.
\item
The argument \textit{main} should be fully expanded, it cannot be a macro.
\item
Subdirectories and special characters should be avoided in filenames.
\item
The command |\childdocmain{|\textit{main}|}| must be followed by a whitespace.
It should not be followed immediately by another command
or by a comment mark `|%|'.
This is because the \TeX{} parser reads the token immediately following
the argument of |\childdocmain| and puts it
at the beginning of every child section;
however, a white\-space is ignored.
\end{itemize}

%%%%%%%%%%%%%%%%%%%%%%%%%%%%%%%%%%%%%%%%
\paragraph{Content of Main File.}

It is advisable to place all content in the child files included by |\include|.
Any output contained in the main file will appear in all child documents
unless suppressed manually;
it cannot be suppressed automatically by the |\includeonly| directive
and thus should normally be avoided.
A method to include some content in the main file
by means of conditional processing is described in \secref{sec:conditional}.

%%%%%%%%%%%%%%%%%%%%%%%%%%%%%%%%%%%%%%%%
\paragraph{Page Numbering.}

When only a part of the document is compiled,
the appropriate numbering of pages
(as well as other status parameters)
is determined from the |.aux| files.
The latter contain information from previous passes.
However this information needs to propagate through
all intermediate child documents.
Therefore the page numbering in child documents may well
be inconsistent until the complete document is compiled at least once.

A useful (if unconventional) way to always ensure a consistent
page numbering is to restart the numbering in each child document
and denote the pages by `\textit{child}|.|\textit{page}'
where \textit{child} represents the chapter/section number of the child file.
This can be achieved by the command
|\numberwithin{page}{|\textit{child}|}|
of the \textsf{amsmath} package
where \textit{child} can be |chapter| or |section|
depending on the chosen structuring.
Alternatively, one can modify the macro |\thepage| appropriately
and reset the counter |page| at the start of each child file.

%%%%%%%%%%%%%%%%%%%%%%%%%%%%%%%%%%%%%%%%%%%%%%%%%%%%%%%%%%%%%%%%%%%%%%%%%%%%%%%%
\subsection{Conditional Processing}
\label{sec:conditional}

The package provides a mechanism to compile different versions
of a document. To customise the versions further some conditional processing
can come in handy to distinguish which version is being compiled.
The package provides two macros to describe the compilation context:

%%%%%%%%%%%%%%%%%%%%%%%%%%%%%%%%%%%%%%%%
\DescribeMacro{\ifchilddoc}
The conditional |\ifchilddoc| distinguishes between the compilation of
child documents and the main document:
%
\begin{center}
|\ifchilddoc |\textit{child-code}| |[|\||else |\textit{main-code}]| \||fi|
\end{center}

%%%%%%%%%%%%%%%%%%%%%%%%%%%%%%%%%%%%%%%%
\DescribeMacro{\childdocname}
\DescribeMacro{\childdocjob}
The macro |\childdocname| contains the filename (without extension)
of the main or child file being processed.
Note that |\childdocjob| will always contain the name of the main file.

%%%%%%%%%%%%%%%%%%%%%%%%%%%%%%%%%%%%%%%%
\paragraph{Title Page.}

Conditional processing can be used to include a title or banner page
in the main document when proper precautions are taken.
Importantly, the code in the main file should ensure that the page counter
(as well as other status parameters which are stored in the |.aux| files)
takes the same value after the conditional processing.
Otherwise the page numbers may take divergent values
depending on which part is compiled.

For example, a title page could be declared by:
%
\begin{center}
\begin{tabular}{l}
|\ifchilddoc\||else|\\
|\addtocounter{page}{-1}|\\
\textit{code for title page}\\
|\newpage|\\
|\||fi|
\end{tabular}
\end{center}
%
A banner page for the child documents can be generated by:
%
\begin{center}
\begin{tabular}{l}
|\ifchilddoc|\\
|\addtocounter{page}{-1}|\\
\textit{code for banner page}\\
|\newpage|\\
|\||fi|
\end{tabular}
\end{center}
%
Here one could write a message such as:
\begin{center}
|This is the part \childdocname{} of \childdocjob{}.|
\end{center}

%%%%%%%%%%%%%%%%%%%%%%%%%%%%%%%%%%%%%%%%%%%%%%%%%%%%%%%%%%%%%%%%%%%%%%%%%%%%%%%%
\subsection{Flags}
\label{sec:flags}

The package makes it easy to generate different versions
of the main or child documents.
To this end compilation flags can be defined
and assigned different default values.
They will be particularly useful in conjunction
with the forwarding mechanism described in \secref{sec:forward}.

For example, it may be useful to have a flag |\version|
which can be set to |draft| or |final|.
The document source will contain some conditional code
depending on the value of |\version|.
Suppose further, the flag should default to |final| for the main file
and to |draft| for child files
which is a natural assignment for editing the document.
This is achieved by placing the following code
in the preamble of the main document
(below the |\childdocmain| directive):
%
\begin{center}
\begin{tabular}{l}
|\ifchilddoc|\\
|\providecommand{\version}{draft}|\\
|\||else|\\
|\providecommand{\version}{final}|\\
|\||fi|
\end{tabular}
\end{center}
%
The definition by |\providecommand| makes sure
that previous definitions are not overwritten.
Further statements |\providecommand{\version}{...}|
can thus be added before the above code to override it.

For the main file, one might add a line
(between |\childdocmain| and the above block)
%
\begin{center}
|%\ifchilddoc\||else\providecommand{\version}{draft}\||fi|
\end{center}
%
which can be uncommented to produce a draft version.
Likewise one can add a line to the very top of a child file
(above the |\childdocof{|\textit{main}|}| directive)
%
\begin{center}
|%\providecommand{\version}{final}|
\end{center}
%
which can be uncommented to produce the final version of this child document.

%%%%%%%%%%%%%%%%%%%%%%%%%%%%%%%%%%%%%%%%%%%%%%%%%%%%%%%%%%%%%%%%%%%%%%%%%%%%%%%%
\subsection{Forwarding}
\label{sec:forward}

Different versions of the main or child documents
using compilation flags as described in \secref{sec:flags}
can be (permanently) stored in different files
for convenient compilation, viewing and distribution.
To this end, the package defines a command
to pass on compilation to a different file:

%%%%%%%%%%%%%%%%%%%%%%%%%%%%%%%%%%%%%%%%
\DescribeMacro{\childdocforward}
The command |\childdocforward| redirects processing to
another source file:
%
\begin{center}
\begin{tabular}{l}
|\input{childdoc.def}|\\
|\childdocforward[|\textit{main}|]{|\textit{dest}|}|\\
\end{tabular}
\end{center}
%
The argument \textit{dest} is the destination file
(without extension).
It should be the main file or one of the child files.
Note that further \textsf{childdoc} directives
such as |\childdocof| and |\childdocforward|
in the indicated file will be processed in this form.
The optional argument \textit{main}
passes on directly to the main file \textit{main}
while pretending to compile the child \textit{dest}.
This form behaves as if \textit{dest}
issues |\childdocof{|\textit{main}|}| right away,
and no further \textsf{childdoc} directives will be processed.

%%%%%%%%%%%%%%%%%%%%%%%%%%%%%%%%%%%%%%%%
\DescribeMacro{\...prefix}
In the alternative form |\childdocforwardprefix|,
%
\begin{center}
\begin{tabular}{l}
|\input{childdoc.def}|\\
|\childdocforwardprefix[|\textit{main}|]{|\textit{prefix}|}{|\textit{dest}|}|
\end{tabular}
\end{center}
%
the destination file is determined by a pattern
depending on the current file:
To make this work, the current file must be called
`{\textit{prefix}\hspace{0.2em}\textit{suffix}}'
with \textit{prefix} matching precisely the argument.
Processing is then passed on to the file
`{\textit{dest}\hspace{0.2em}\textit{suffix}}'.
Surely, the same effect is achieved by
directly specifying the
argument `{\textit{dest}\hspace{0.2em}\textit{suffix}}'
in the first form.
However, that requires to set up a different file
for each child. With the alternative form of the command
all these files can have exactly the same content
which simplifies setting them up and maintaining them.

For example, the following file |draft.tex|
with a compilation flag |\version| as described in \secref{sec:flags}
compiles the main document as a draft:
%
\begin{center}
\begin{tabular}{l}
|\def\version{draft}|\\
|\input{childdoc.def}|\\
|\childdocforward{|\textit{main}|}|
\end{tabular}
\end{center}
%
Likewise, the following files |final|\textit{nn}|.tex|
compile the final version of the child document
|child|\textit{nn}|.tex|:
%
\begin{center}
\begin{tabular}{l}
|\def\version{final}|\\
|\input{childdoc.def}|\\
|\childdocforwardprefix{final}{child}|
\end{tabular}
\end{center}
%

Note that when several versions of a main file and/or of each child file
are to be generated, it may be convenient to set up a |Makefile| or
shell script to automatise the process.

%%%%%%%%%%%%%%%%%%%%%%%%%%%%%%%%%%%%%%%%%%%%%%%%%%%%%%%%%%%%%%%%%%%%%%%%%%%%%%%%
\subsection{Command Line Processing}
\label{sec:commandline}

The effect of redirection files can also be achieved by invoking
the \LaTeX{} compiler with a more elaborate command line.
Most conveniently this should be done as part
of a shell script or a |Makefile|.

When using \textsf{childdoc} in the main file, the following
command lines effectively perform a redirection
(note that depending on the shell being used,
backslashes may have to be doubled: `|\|' $\to$ `|\\|'):
%
\begin{center}
|... -jobname "|\textit{target}|" |\\|"|[\textit{flags}]%
|\input{childdoc.def}\childdocforward[|\textit{main}|]{|\textit{dest}|}"|
\end{center}
%
Here \textit{target} is the name of the output file,
\textit{main} is the name of the main file
and \textit{dest} is the name of the main or child file to be processed
(all filenames without extensions).
The optional argument \textit{main} can be omitted
if \textit{main} matches \textit{dest}.
Optionally, compilation \textit{flags} can be defined via |\def| commands.
This command line makes the \TeX{} engine believe
it is compiling the file \textit{target}
whose content is specified as the latter parameter.
The provided code then forwards the processing to
\textit{main} or \textit{dest} as described in \secref{sec:forward}.

%%%%%%%%%%%%%%%%%%%%%%%%%%%%%%%%%%%%%%%%%%%%%%%%%%%%%%%%%%%%%%%%%%%%%%%%%%%%%%%%
\subsection{Include by Input}
\label{sec:input}

Including child documents by |\include| has some restrictions by design.
Most notably, the content of a child document always occupies
its own set of pages; pages cannot be shared between child documents.
Usually, this behaviour makes perfect sense
because each child document contain an essential part of the document.
However, in some situations it may be desirable to compose
a document from a collection of parts
without having mandatory page breaks between then.
For this case, the package
provides a mechanism to include parts
by |\input| which can also be processed individually.
However, by construction this mechanism
requires manual handling of the content to be output.

%%%%%%%%%%%%%%%%%%%%%%%%%%%%%%%%%%%%%%%%
\DescribeMacro{\ifchilddocmanual}
The main file should be prepared as usual, see \secref{sec:include}.
However, the document body must make a distinction
between processing of an individual part and of the main document, e.g.:
%
\begin{center}
\begin{tabular}{l}
|\ifchilddocmanual|\\
|\input{\childdocname}|\\
|\||else|\\
\textit{document body with }|\input{|\textit{part}|}|\\
|\||fi|
\end{tabular}
\end{center}
%
The conditional |\ifchilddocmanual| is true whenever
a part to be included by |\input| is being compiled,
and the name of the part is stored in |\childdocname|.

%%%%%%%%%%%%%%%%%%%%%%%%%%%%%%%%%%%%%%%%
\DescribeMacro{\childdocby}
Each part to be included by |\input| should start with:
%
\begin{center}
\begin{tabular}{l}
|\input{childdoc.def}|\\
|\childdocby{|\textit{main}|}|\\
\end{tabular}
\end{center}
%
The directive |\childdocby| is similar to |\childdocof|
described in \secref{sec:include},
but the subsequent selection of content must be done manually.
To that end, both |\ifchilddoc| and |\ifchilddocmanual|
will be true upon processing of a part,
and the name of the part is stored in |\childdocname|.
Note that |\jobname| will be set to the filename of the current part
so that each part receives an individual |.aux| file
that does not interfere with the |.aux| file(s) of the main document.
This behaviour can be altered by the alternative form
|\childdocby[*]{|\textit{main}|}| (with a non-empty optional argument)
which uses the |.aux| file of the main document
by setting |\jobname| to \textit{main}.

%%%%%%%%%%%%%%%%%%%%%%%%%%%%%%%%%%%%%%%%%%%%%%%%%%%%%%%%%%%%%%%%%%%%%%%%%%%%%%%%
\subsection{Driver Development}
\label{sec:driver}

The \textsf{childdoc} mechanism can also be use for the development
of definition files such as \LaTeX{} styles or classes.
This case differs from the above setup with multiple parts
included by |\include| in that no |\includeonly| should be invoked.
This can be achieved by starting the include file
(before |\ProvidesPackage|) with:
%
\begin{center}
\begin{tabular}{l}
|\input{childdoc.def}|\\
|\childdocforward{|\textit{main}|}|\\
\end{tabular}
\end{center}
%
or alternatively with:
%
\begin{center}
\begin{tabular}{l}
|\input{childdoc.def}|\\
|\childdocby{|\textit{main}|}|\\
\end{tabular}
\end{center}
%
Both forms have slightly different effects as described above.
The main file is prepared as usual, see \secref{sec:include}.

%%%%%%%%%%%%%%%%%%%%%%%%%%%%%%%%%%%%%%%%%%%%%%%%%%%%%%%%%%%%%%%%%%%%%%%%%%%%%%%%
\subsection{Legacy Detection}
\label{sec:detection}

The directive |\childdocmain| in the main file can detect
whether the complete document or merely a child is to be compiled
even without using the directive |\childdocof|.
This method is deprecated because it is less robust
and there is no compelling reason to use it;
it is merely provided for backward compatibility
and it may be removed in future versions.

If the detection mechanism is to be used,
it is mandatory to correctly specify
the filename of the main file as the argument of |\childdocmain|:
%
\begin{center}
\begin{tabular}{l}
|\input{childdoc.def}|\\
|\childdocmain{|\textit{main}|}|\\
\end{tabular}
\end{center}
%
If |\jobname| does not match the argument \textit{main} of |\childdocmain|,
it is assumed that |\jobname| points to the child file to be compiled.
When using |\childdocmain| with the main file specified as argument,
it suffices to start a child file
with just |\input{|\textit{main}|}|
without loading of the package and using |\childdocof|.
If instead all processing is done
with the appropriate \textsf{childdoc} directives,
the argument of \textit{main} of |\childdocmain| can be empty.

An alternative version of the command line processing described
in \secref{sec:commandline} using the detection mechanism reads:
%
\begin{center}
|... -jobname "|\textit{target}|" "|[\textit{flags}]%
[|\def\jobname{|\textit{dest}|}|]|\input{|\textit{main}|}"|
\end{center}

%%%%%%%%%%%%%%%%%%%%%%%%%%%%%%%%%%%%%%%%%%%%%%%%%%%%%%%%%%%%%%%%%%%%%%%%%%%%%%%%
\subsection{Manual Code}
\label{sec:manual}

In case one cannot be certain whether the definitions file |childdoc.def|
is installed on the target \TeX{} distribution
and one prefers not to ship it,
it is conceivable to paste a few relevant commands into the sources.

To that end, drop all statements |\input{childdoc.def}|
and perform the replacements as outlined below.
Instead of |\childdocmain{|\textit{main}|}| add the following code
to the top of the main file:
%
\begin{center}
\begin{tabular}{l}
|\||ifdefined\childdocname\endinput\||fi\newif\ifchilddoc|\\
|\edef\childdocname{\scantokens\expandafter{\jobname\noexpand}}|\\
|\def\childdocmain{|\textit{main}|}\||ifx\childdocmain\childdocname\||else|\\
|\childdoctrue\includeonly{\childdocname}\let\jobname\childdocmain\||fi|\\
\end{tabular}
\end{center}
%
Instead of |\childdocof{|\textit{main}|}| just include the main file
at the top of each child file:
%
\begin{center}
|\input{|\textit{main}|}|
\end{center}
%
A simple redirection |\childdocforward{|\textit{dest}|}| is achieved by:
%
\begin{center}
|\def\jobname{|\textit{dest}|}\input{\jobname}|
\end{center}
%
The redirection with prefix
|\childdocforwardprefix[|\textit{prefix}|]{|\textit{dest}|}|
is accomplished by:
%
\begin{center}
\begin{tabular}{l}
|{\edef\jobname{\scantokens\expandafter{\jobname\noexpand}}|\\
|\def\redirectjob |\textit{prefix}|#1~~~{\gdef\jobname{|\textit{dest}|#1}}|\\
|\expandafter\redirectjob\jobname~~~}\input{\jobname}|
\end{tabular}
\end{center}

In an alternative approach,
child documents can be compiled by a specific command line
without additional code or specific definitions:
%
\begin{center}
|... -jobname "|\textit{target}|" "|[\textit{flags}]%
|\includeonly{|\textit{dest}|}\input{|\textit{main}|}"|
\end{center}
%

%%%%%%%%%%%%%%%%%%%%%%%%%%%%%%%%%%%%%%%%%%%%%%%%%%%%%%%%%%%%%%%%%%%%%%%%%%%%%%%%
%%%%%%%%%%%%%%%%%%%%%%%%%%%%%%%%%%%%%%%%%%%%%%%%%%%%%%%%%%%%%%%%%%%%%%%%%%%%%%%%
\section{Information}

%%%%%%%%%%%%%%%%%%%%%%%%%%%%%%%%%%%%%%%%%%%%%%%%%%%%%%%%%%%%%%%%%%%%%%%%%%%%%%%%
\subsection{Copyright}

Copyright \copyright{} 2017--2018 Niklas Beisert

This work may be distributed and/or modified under the
conditions of the \LaTeX{} Project Public License, either version 1.3
of this license or (at your option) any later version.
The latest version of this license is in
  \url{http://www.latex-project.org/lppl.txt}
and version 1.3 or later is part of all distributions of \LaTeX{}
version 2005/12/01 or later.

This work has the LPPL maintenance status `maintained'.

The Current Maintainer of this work is Niklas Beisert.

This work consists of the files |README.txt|, |childdoc.ins| and |childdoc.dtx|
as well as the derived files |childdoc.def|, |cdocsamp.tex|
with |cdocsch1.tex|, |cdocsch2.tex|, |cdocspt3.tex|, |cdocspt4.tex|,
|cdocsdrf.tex|, |cdocsfn1.tex|, |cdocsfn2.tex|
as well as |childdoc.pdf|.

%%%%%%%%%%%%%%%%%%%%%%%%%%%%%%%%%%%%%%%%%%%%%%%%%%%%%%%%%%%%%%%%%%%%%%%%%%%%%%%%
\subsection{Files and Installation}

The package consists of the files:
%
\begin{center}
\begin{tabular}{ll}
    |README.txt|   & readme file \\
    |childdoc.ins| & installation file \\
    |childdoc.dtx| & source file \\
    |childdoc.def| & definition file \\
    |cdocsamp.tex| & sample main file \\
    |cdocsch1.tex| & sample include file \\
    |cdocsch2.tex| & sample include file \\
    |cdocspt3.tex| & sample part file \\
    |cdocspt4.tex| & sample part file \\
    |cdocsdrf.tex| & sample redirection file \\
    |cdocsfn1.tex| & sample redirection file \\
    |cdocsfn2.tex| & sample redirection file \\
    |childdoc.pdf| & manual
\end{tabular}
\end{center}
%
The distribution consists of the files
|README.txt|, |childdoc.ins| and |childdoc.dtx|.
%
\begin{itemize}
\item
Run (pdf)\LaTeX{} on |childdoc.dtx|
to compile the manual |childdoc.pdf| (this file).
\item
Run \LaTeX{} on |childdoc.ins| to create the definitions file |childdoc.def|
and the sample |cdocsamp.tex| with include files
|cdocsch1.tex|, |cdocsch2.tex|, |cdocspt3.tex|, |cdocspt4.tex|,
|cdocsdrf.tex|, |cdocsfn1.tex|, |cdocsfn2.tex|.
Then copy the file |childdoc.def| to an appropriate directory of your \LaTeX{}
distribution, e.g.\ \textit{texmf-root}|/tex/latex/childdoc|.
\end{itemize}

%%%%%%%%%%%%%%%%%%%%%%%%%%%%%%%%%%%%%%%%%%%%%%%%%%%%%%%%%%%%%%%%%%%%%%%%%%%%%%%%
\subsection{Related CTAN Packages}

There are several other packages which offer a similar functionality:
%
\begin{itemize}
\item
The packages
\href{http://ctan.org/pkg/docmute}{\textsf{docmute}},
\href{http://ctan.org/pkg/includex}{\textsf{includex}} and
\href{http://ctan.org/pkg/standalone}{\textsf{standalone}}
provide commands to include only the document body of
a child file thus allowing both files to be compiled individually.
\item
The packages \href{http://ctan.org/pkg/subdocs}{\textsf{subdocs}}
and \href{http://ctan.org/pkg/subfiles}{\textsf{subfiles}}
provide structures in which the main and child documents can be
encapsulated and allowing them to be compiled individually.
The inclusion mechanism is different from the conventional |\include|.
\item
The package \href{http://ctan.org/pkg/combine}{\textsf{combine}}
is an elaborate solution to combine several documents into one.
\end{itemize}
%
See also the CTAN topic \href{http://ctan.org/topic/subdocs}{\textsf{subdocs}}
for further related packages.
The present package differs from the above solutions in that
a document structure constructed with the conventional |\include| mechanism
just needs two extra commands at the top of every file
such that all constituent files can be compiled individually.

%%%%%%%%%%%%%%%%%%%%%%%%%%%%%%%%%%%%%%%%%%%%%%%%%%%%%%%%%%%%%%%%%%%%%%%%%%%%%%%%
%\subsection{Feature Suggestions}
%
%The following is a list of features which may be useful for future
%versions of this package:
%%
%\begin{itemize}
%\item
%\ldots
%\end{itemize}

%%%%%%%%%%%%%%%%%%%%%%%%%%%%%%%%%%%%%%%%%%%%%%%%%%%%%%%%%%%%%%%%%%%%%%%%%%%%%%%%
\subsection{Revision History}

%%%%%%%%%%%%%%%%%%%%%%%%%%%%%%%%%%%%%%%%
\paragraph{v2.0:} 2018/12/30

\begin{itemize}
\item
immediate forward processing
\item
added |\childdocby| mechanism
\item
manual restructured
\end{itemize}

%%%%%%%%%%%%%%%%%%%%%%%%%%%%%%%%%%%%%%%%
\paragraph{v1.6:} 2018/01/17

\begin{itemize}
\item
application for development of include files
\item
corrections to manual
\end{itemize}

%%%%%%%%%%%%%%%%%%%%%%%%%%%%%%%%%%%%%%%%
\paragraph{v1.5:} 2017/05/21

\begin{itemize}
\item
more complete structuring introduced
\item
|\childdocof| introduced
\item
|\childdoc| renamed to |\childdocmain|
\item
|\childredirect| renamed to |\childdocforward| and |\childdocforwardprefix|
and functionality expanded
\end{itemize}

%%%%%%%%%%%%%%%%%%%%%%%%%%%%%%%%%%%%%%%%
\paragraph{v1.0:} 2017/04/27

\begin{itemize}
\item
manual and install package
\item
first version published on CTAN
\end{itemize}

%%%%%%%%%%%%%%%%%%%%%%%%%%%%%%%%%%%%%%%%
\paragraph{v0.6:} 2017/04/26

\begin{itemize}
\item
redirection mechanism added
\end{itemize}

%%%%%%%%%%%%%%%%%%%%%%%%%%%%%%%%%%%%%%%%
\paragraph{v0.5:} 2017/04/26

\begin{itemize}
\item
functionality in definition file
\end{itemize}


%%%%%%%%%%%%%%%%%%%%%%%%%%%%%%%%%%%%%%%%%%%%%%%%%%%%%%%%%%%%%%%%%%%%%%%%%%%%%%%%
%%%%%%%%%%%%%%%%%%%%%%%%%%%%%%%%%%%%%%%%%%%%%%%%%%%%%%%%%%%%%%%%%%%%%%%%%%%%%%%%
%%%%%%%%%%%%%%%%%%%%%%%%%%%%%%%%%%%%%%%%%%%%%%%%%%%%%%%%%%%%%%%%%%%%%%%%%%%%%%%%
\appendix

\settowidth\MacroIndent{\rmfamily\scriptsize 000\ }

 \DocInput{childdoc.dtx}

\end{document}
%</driver>
% \fi
%
% %%%%%%%%%%%%%%%%%%%%%%%%%%%%%%%%%%%%%%%%%%%%%%%%%%%%%%%%%%%%%%%%%%%%%%%%%%%%%%
% %%%%%%%%%%%%%%%%%%%%%%%%%%%%%%%%%%%%%%%%%%%%%%%%%%%%%%%%%%%%%%%%%%%%%%%%%%%%%%
% \section{Sample}
%\iffalse
%<*samplemain>
%\fi
%
% The following presents a sample document
% with two chapters, two parts, a title page,
% a compile flag as well as three forwarding files to set the flag.
% It consists of eight |.tex| files:
% \begin{center}
% \begin{tabular}{ll}
% |cdocsamp.tex|&main file\\
% |cdocsch1.tex|&include file for chapter 1\\
% |cdocsch2.tex|&include file for chapter 2\\
% |cdocspt3.tex|&include file for part 3\\
% |cdocspt4.tex|&include file for part 4\\
% |cdocsdrf.tex|&forwarding file for main file in draft mode\\
% |cdocsfi1.tex|&forwarding file for final version of chapter 1\\
% |cdocsfi2.tex|&forwarding file for final version of chapter 2\\
% \end{tabular}
% \end{center}
% Each of the eight files can be compiled directly by the \LaTeX{} compiler.
%
% %%%%%%%%%%%%%%%%%%%%%%%%%%%%%%%%%%%%%%
% \paragraph{Main File.}
%
% The main file is called |cdocsamp.tex|.
%
% Load the \textsf{childdoc} definitions and
% declare the filename for the main document:
%    \begin{macrocode}
\input{childdoc.def}
\childdocmain{}
%    \end{macrocode}

% Optional override for |\version| flag:
%    \begin{macrocode}
%%\ifchilddoc\else\providecommand{\version}{draft}\fi
%    \end{macrocode}

% Define the default values for the |\version| flag
% (|final| for the main file and |draft| for childs):
%    \begin{macrocode}
\ifchilddoc
\providecommand{\version}{draft}
\else
\providecommand{\version}{final}
\fi
%    \end{macrocode}

% Load the standard document class:
%    \begin{macrocode}
\documentclass[12pt]{article}
%    \end{macrocode}

% Start the document body:
%    \begin{macrocode}
\begin{document}
%    \end{macrocode}

% Declare a title page.
% Print title, part of document being processed and version flag:
%    \begin{macrocode}
\addtocounter{page}{-1}
\begin{center}
{\LARGE\bfseries{}childdoc example\par}
\vspace{1cm}
\ifchilddoc
\ifchilddocmanual part\else chapter\fi:
`\childdocname' of `\childdocjob'\par
\else
main document: `\childdocjob'\par
\fi
version: \version\par
\end{center}
\newpage
%    \end{macrocode}

% Manually include selected file,
% otherwise process as usual:
%    \begin{macrocode}
\ifchilddocmanual
\section*{part `\childdocname'}
\input{\childdocname}
\else
%    \end{macrocode}

% Include the two chapters:
%    \begin{macrocode}
\include{cdocsch1}
\include{cdocsch2}
%    \end{macrocode}

% Include the two parts unless only chapters should be displayed:
%    \begin{macrocode}
\ifchilddoc\else
\section{part three}
\input{cdocspt3}
\section{part four}
\input{cdocspt4}
\fi
%    \end{macrocode}

% Process as usual until here:
%    \begin{macrocode}
\fi
%    \end{macrocode}

% End of document body:
%    \begin{macrocode}
\end{document}
%    \end{macrocode}
%\iffalse
%</samplemain>
%\fi
%
% %%%%%%%%%%%%%%%%%%%%%%%%%%%%%%%%%%%%%%
% \paragraph{Chapter Include Files.}
%
% The include files are called |cdocsch1.tex| and |cdocsch2.tex|.
%
%\iffalse
%<*samplechap1|samplechap2>
%\fi

% Optional override for |\version| flag:
%    \begin{macrocode}
%%\providecommand{\version}{final}
%    \end{macrocode}

% Include the main document:
%    \begin{macrocode}
\input{childdoc.def}
\childdocof{cdocsamp}
%    \end{macrocode}

%\iffalse
%</samplechap1|samplechap2>
%\fi
%
%\iffalse
%<*samplechap1>
%\fi
% Some text for chapter 1:
%    \begin{macrocode}
\section{one}
some text in chapter one
%    \end{macrocode}

%\iffalse
%</samplechap1>
%\fi
% Some text for chapter 2:
%\iffalse
%<*samplechap2>
%\fi
%    \begin{macrocode}
\section{two}
more text in chapter two
%    \end{macrocode}

%\iffalse
%</samplechap2>
%\fi
%
% %%%%%%%%%%%%%%%%%%%%%%%%%%%%%%%%%%%%%%
% \paragraph{Part Include Files.}
%
% The include files are called |cdocspt3.tex| and |cdocspt4.tex|.
%
%\iffalse
%<*samplepart3|samplepart4>
%\fi

% Optional override for |\version| flag:
%    \begin{macrocode}
%%\providecommand{\version}{final}
%    \end{macrocode}

% Include the main document:
%    \begin{macrocode}
\input{childdoc.def}
\childdocby{cdocsamp}
%    \end{macrocode}

%\iffalse
%</samplepart3|samplepart4>
%\fi
%
%\iffalse
%<*samplepart3>
%\fi
% Some text for part 3:
%    \begin{macrocode}
some text in part three
%    \end{macrocode}

%\iffalse
%</samplepart3>
%\fi
% Some text for part 4:
%\iffalse
%<*samplepart4>
%\fi
%    \begin{macrocode}
more text in part four
%    \end{macrocode}

%\iffalse
%</samplepart4>
%\fi
%
% %%%%%%%%%%%%%%%%%%%%%%%%%%%%%%%%%%%%%%
% \paragraph{Forwarding for a Complete Draft.}
%
% The following forwarding file |cdocsdrf.tex|
% compiles the main document in draft mode:
%\iffalse
%<*sampledraft>
%\fi
%    \begin{macrocode}
\def\version{draft}
\input{childdoc.def}
\childdocforward{cdocsamp}
%    \end{macrocode}

%\iffalse
%</sampledraft>
%\fi
%
% %%%%%%%%%%%%%%%%%%%%%%%%%%%%%%%%%%%%%%
% \paragraph{Forwarding for Final Version of the Chapters.}
%
% The following forwarding files |cdocsfn1.tex| and |cdocsfn2.tex|
% (with identical content)
% compile the final versions of the child documents
% |cdocsch1.tex| and |cdocsch2.tex|, respectively:
%\iffalse
%<*samplefinal>
%\fi
%    \begin{macrocode}
\def\version{final}
\input{childdoc.def}
\childdocforwardprefix[cdocsamp]{cdocsfn}{cdocsch}
%    \end{macrocode}

%\iffalse
%</samplefinal>
%\fi
%
% %%%%%%%%%%%%%%%%%%%%%%%%%%%%%%%%%%%%%%
% \paragraph{Command Line Processing.}
%
% The following three command lines generate the output files
% |cdocscld|, |cdocscl1| and |cdocscl2|
% which should be identical to
% |cdocsdrf|, |cdocsch1| and |cdocsfn2|, respectively:
% \begin{center}
% \begin{tabular}{l}
% |latex -jobname cdocscld \|\\
% |  "\def\version{draft}\input{childdoc.def}\childdocforward{cdocsamp}"|\\
% |latex -jobname cdocscl1 \|\\
% |  "\input{childdoc.def}\childdocforward[cdocsamp]{cdocsch1}"|\\
% |latex -jobname cdocscl2 \|\\
% |  "\def\version{final}\input{childdoc.def}\childdocforward{cdocsch2}"|
% \end{tabular}
% \end{center}
% Note that the trailing backslash on each first line
% merely continues the input to the second line
% (for convenient cut ant paste).
% Furthermore, the command |latex| can be replaced by any
% of its alternative versions such as |pdflatex|.
%
% %%%%%%%%%%%%%%%%%%%%%%%%%%%%%%%%%%%%%%%%%%%%%%%%%%%%%%%%%%%%%%%%%%%%%%%%%%%%%%
% %%%%%%%%%%%%%%%%%%%%%%%%%%%%%%%%%%%%%%%%%%%%%%%%%%%%%%%%%%%%%%%%%%%%%%%%%%%%%%
% \section{Implementation}
%\iffalse
%<*package>
%\fi
%
% This section describes the definitions file |childdoc.def|.

% The definitions cannot be loaded using |\usepackage| or |\RequirePackage|
% which has a mechanism to prevent loading a style file more than once.
% When loading the definitions by means of |\input|
% multiple instances have to be prevented manually:
%\iffalse
%This code needs to be before the `\ProvidesFile' directive
%which is defined at the beginning of this file.
%Therefore it is also placed there and commented out here.
%</package>
%<*discard>
%\fi
%    \begin{macrocode}
\ifdefined\childdocmain\endinput\fi
%    \end{macrocode}
%\iffalse
%</discard>
%<*package>
%\fi
%
% \macro{\ifchilddoc}
% \macro{\ifchilddocmanual}
% The conditional |\ifchilddoc| tells whether a
% child (true) or main (false) document is being compiled.
% The conditional |\ifchilddocmanual| tells whether
% the |\includeonly| mechanism is used (false) or
% the selection of child files must be performed manually (true).
% The definitions initialise to false:
%    \begin{macrocode}
\newif\ifchilddoc
\newif\ifchilddocmanual
%    \end{macrocode}

% \macro{\childdocname}
% \macro{\childdocjob}
% The macro |\childdocname| stores the name of the main document
% to be compiled. The macro |\childdocjob| stores the name of
% the document on which the \LaTeX{} compiler was originally invoked.
% The content of |\jobname| cannot be compared
% to filenames specified in the source due to different catcodes.
% The following code rescans |\jobname|, stores the result
% in |\childdocname| and saves a copy in |\childdocjob|:
%    \begin{macrocode}
\edef\childdocname{\scantokens\expandafter{\jobname\noexpand}}
\let\childdocjob\childdocname
%    \end{macrocode}

% \macro{\childdocdisable}
% The macro |\childdocdisable| prevents the main file
% from being processed more than once.
% At this stage, the main document command |\childdocmain|
% is assumed to be called once again where it should do nothing.
% Any subsequent call to it should prevent
% a secondary processing of the main document
% It overwrites the forwarding commands
% |\childdocof| and |\childdocforward|
% with empty macros to prevent further inclusions of the main document:
%    \begin{macrocode}
\newcommand{\childdocdisable}
{
  \renewcommand{\childdocmain}[1]{\renewcommand{\childdocmain}[1]{\endinput}}
  \renewcommand{\childdocof}[1]{}
  \renewcommand{\childdocby}[2][]{}
  \renewcommand{\childdocforward}[2][]{}
  \renewcommand{\childdocdisable}{}
}
%    \end{macrocode}

% \macro{\childdocmain}
% The macro |\childdocmain| is to be called at the top of the main file
% with nothing or the main filename (without extension) as argument.
% First, it breaks loops.
% If the argument is not empty and does not match |\childdocname|
% (which is set by the first inclusion of |childdoc.def|),
% |\ifchilddoc| is set to true, |\includeonly| is applied to the child file
% and |\jobname| is set to the main file
% (for proper handling of |.aux| files):
%    \begin{macrocode}
\newcommand{\childdocmain}[1]
{
  \childdocdisable\childdocmain{}
  \if?#1?\else
    \begingroup
      \def\childdoctmp{#1}
      \ifx\childdoctmp\childdocname
        \def\childdoctmp{}
      \else
        \def\childdoctmp
        {
          \childdoctrue
          \includeonly{\childdocname}
          \def\childdocjob{#1}
          \def\jobname{#1}
        }
      \fi
      \expandafter
    \endgroup
    \childdoctmp
  \fi
}
%    \end{macrocode}

% \macro{\childdocof}
% The command |\childdocof| redirects
% compilation to the main file |#1|.
%    \begin{macrocode}
\newcommand{\childdocof}[1]
{
  \childdocdisable
  \childdoctrue
  \includeonly{\childdocname}
  \def\jobname{#1}
  \def\childdocjob{#1}
  \input{#1}
}
%    \end{macrocode}

% \macro{\childdocby}
% The command |\childdocby| ....
%    \begin{macrocode}
\newcommand{\childdocby}[2][]
{
  \childdocdisable
  \childdoctrue
  \childdocmanualtrue
  \if?#1?\else
    \def\jobname{#2}
  \fi
  \def\childdocjob{#2}
  \input{#2}
  \endinput
}
%    \end{macrocode}

% \macro{\childdocforward}
% The command |\childdocforward| redirects
% compilation to the main file or
% (if the optional argument is given) a child file.
% Parameters are set as if the main file
% or a child file starting with |\childdocof| was compiled.
% Then compilation is handed over to the main file:
%    \begin{macrocode}
\newcommand{\childdocforward}[2][]
{
  \begingroup
    \if?#1?
      \def\childdoctmp
      {
        \def\childdocname{#2}
        \def\childdocjob{#2}
        \def\jobname{#2}
        \input{#2}
        \endinput
      }
    \else
      \def\childdoctmp
      {
        \childdocdisable
        \def\childdocname{#2}
        \childdoctrue
        \includeonly{#2}
        \def\childdocjob{#1}
        \def\jobname{#1}
        \input{#1}
        \endinput
      }
    \fi
    \expandafter
  \endgroup
  \childdoctmp
}
%    \end{macrocode}

% \macro{\childdocforwardprefix}
% The command |\childdocforwardprefix| redirects
% compilation to the main or a child file by means of a pattern.
% The prefix |#1| in the current filename is replaced by |#2|
% and the suffix of the current filename is kept
% (it is assumed that the filename does not contain the substring `|~~~|'
% which is used as a delimiter).
% Compilation is handed over to the new file by |\childdocforward|:
%    \begin{macrocode}
\newcommand{\childdocforwardprefix}[3][]
{
  \begingroup
    \def\childdocextract #2##1~~~{\def\childdoctmp{\childdocforward[#1]{#3##1}}}
    \expandafter\childdocextract\childdocname~~~
    \expandafter
  \endgroup
  \childdoctmp
}
%    \end{macrocode}

% \macro{\childdoc}
% The deprecated macro |\childdoc| is a legacy version of |\childdocmain|:
%    \begin{macrocode}
\newcommand{\childdoc}{\childdocmain}
%    \end{macrocode}

% \macro{\childdocredirect}
% The deprecated macro |\childdocredirect| is a legacy version
% of |\childdocforward| and |\childdocforwardprefix|:
%    \begin{macrocode}
\newcommand{\childdocredirect}[2][]
{
  \begingroup
    \if?#1?
      \def\childdoctmp{\childdocforward{#2}}
    \else
      \def\childdoctmp{\childdocforwardprefix{#1}{#2}}
    \fi
    \expandafter
  \endgroup
  \childdoctmp
}
%    \end{macrocode}

%\iffalse
%</package>
%\fi
%
\endinput
|\\
|\childdocby{|\textit{main}|}|\\
\end{tabular}
\end{center}
%
The directive |\childdocby| is similar to |\childdocof|
described in \secref{sec:include},
but the subsequent selection of content must be done manually.
To that end, both |\ifchilddoc| and |\ifchilddocmanual|
will be true upon processing of a part,
and the name of the part is stored in |\childdocname|.
Note that |\jobname| will be set to the filename of the current part
so that each part receives an individual |.aux| file
that does not interfere with the |.aux| file(s) of the main document.
This behaviour can be altered by the alternative form
|\childdocby[*]{|\textit{main}|}| (with a non-empty optional argument)
which uses the |.aux| file of the main document
by setting |\jobname| to \textit{main}.

%%%%%%%%%%%%%%%%%%%%%%%%%%%%%%%%%%%%%%%%%%%%%%%%%%%%%%%%%%%%%%%%%%%%%%%%%%%%%%%%
\subsection{Driver Development}
\label{sec:driver}

The \textsf{childdoc} mechanism can also be use for the development
of definition files such as \LaTeX{} styles or classes.
This case differs from the above setup with multiple parts
included by |\include| in that no |\includeonly| should be invoked.
This can be achieved by starting the include file
(before |\ProvidesPackage|) with:
%
\begin{center}
\begin{tabular}{l}
|% \iffalse
%
% childdoc.dtx Copyright (C) 2017-2018 Niklas Beisert
%
% This work may be distributed and/or modified under the
% conditions of the LaTeX Project Public License, either version 1.3
% of this license or (at your option) any later version.
% The latest version of this license is in
%   http://www.latex-project.org/lppl.txt
% and version 1.3 or later is part of all distributions of LaTeX
% version 2005/12/01 or later.
%
% This work has the LPPL maintenance status `maintained'.
%
% The Current Maintainer of this work is Niklas Beisert.
%
% This work consists of the files childdoc.dtx and childdoc.ins
% and the derived files childdoc.def and cdocsamp.tex with
% cdocsch1.tex, cdocsch2.tex, cdocsdrf.tex, cdocsfn1.tex, cdocsfn2.tex.
%
%<package>\ifdefined\childdocmain\endinput\fi
%<package>\ProvidesFile{childdoc.def}[2018/12/30 v2.0 child document driver]
%<samplemain>\ProvidesFile{cdocsamp.tex}[2018/12/30 v2.0 sample for childdoc]
%<*driver>
%\ProvidesFile{childdoc.drv}[2018/12/30 v2.0 childdoc reference manual file]
\PassOptionsToClass{10pt,a4paper}{article}
\documentclass{ltxdoc}

\usepackage[margin=35mm]{geometry}
\usepackage{hyperref}
\usepackage{hyperxmp}
\usepackage[usenames]{color}

\hypersetup{colorlinks=true}
\hypersetup{pdfstartview=FitH}
\hypersetup{pdfpagemode=UseNone}
\hypersetup{pdfsource={}}
\hypersetup{pdflang={en-UK}}
\hypersetup{pdfcopyright={Copyright 2017-2018 Niklas Beisert.
  This work may be distributed and/or modified under the
  conditions of the LaTeX Project Public License, either version 1.3
  of this license or (at your option) any later version.}}
\hypersetup{pdflicenseurl={http://www.latex-project.org/lppl.txt}}
\hypersetup{pdfcontactaddress={ETH Zurich, ITP, HIT K,
  Wolfgang-Pauli-Strasse 27}}
\hypersetup{pdfcontactpostcode={8093}}
\hypersetup{pdfcontactcity={Zurich}}
\hypersetup{pdfcontactcountry={Switzerland}}
\hypersetup{pdfcontactemail={nbeisert@itp.phys.ethz.ch}}
\hypersetup{pdfcontacturl={http://people.phys.ethz.ch/\xmptilde nbeisert/}}

\newcommand{\secref}[1]{\hyperref[#1]{section \ref*{#1}}}

\parskip1ex
\parindent0pt
\let\olditemize\itemize
\def\itemize{\olditemize\parskip0pt}

\begin{document}

\title{The \textsf{childdoc} Package}
\hypersetup{pdftitle={The childdoc Package}}
\author{Niklas Beisert\\[2ex]
  Institut f\"ur Theoretische Physik\\
  Eidgen\"ossische Technische Hochschule Z\"urich\\
  Wolfgang-Pauli-Strasse 27, 8093 Z\"urich, Switzerland\\[1ex]
  \href{mailto:nbeisert@itp.phys.ethz.ch}
  {\texttt{nbeisert@itp.phys.ethz.ch}}}
\hypersetup{pdfauthor={Niklas Beisert}}
\hypersetup{pdfsubject={Manual for the LaTeX2e Package childdoc}}
\date{30 December 2018, \textsf{v2.0}}
\maketitle

\begin{abstract}\noindent
\textsf{childdoc} is a \LaTeXe{} package
that enables the direct compilation
of document sections included by |\include|
to individual files.
\end{abstract}

\begingroup
\parskip0ex
\tableofcontents
\endgroup

%%%%%%%%%%%%%%%%%%%%%%%%%%%%%%%%%%%%%%%%%%%%%%%%%%%%%%%%%%%%%%%%%%%%%%%%%%%%%%%%
%%%%%%%%%%%%%%%%%%%%%%%%%%%%%%%%%%%%%%%%%%%%%%%%%%%%%%%%%%%%%%%%%%%%%%%%%%%%%%%%
\section{Introduction}

\LaTeX{} provides a mechanism to structure a large document (such as a book)
into a main file and several child files (containing the chapters)
using the |\include| command.
This mechanism is beneficial for documents
which span hundreds of pages in order to
make the source file(s) more manageable.
Moreover, compilation can be restricted to
selected child files by means of the |\includeonly| command.
The latter feature can be used to reduce the compilation time while editing
(this was significantly more useful in the earlier days of \LaTeX{})
or to generate a smaller document which is easier to navigate.
Another application of |\includeonly| is to generate
documents consisting of selected parts of the complete document.

However, there are a few drawbacks of the plain |\include| mechanism:
\begin{itemize}
\item
The child files cannot be compiled on their own,
they can only be compiled via the main file.
A naive editing environment
(such as a text editor with an option
to have the current file processed by \LaTeX)
may require one to switch to the main file before compiling;
attempting to compile the child file produces errors.
\item
The main file must be modified (each time)
to adjust the |\includeonly| command
to the present needs. This easily leaves the main file in a messy state.
\item
The generated document will always carry the filename
of the main document. This is inconvenient if
several child files are to be compiled and
to be kept for distribution.
\end{itemize}

The present package provides a simple interface
to make child files individually compilable by \LaTeX{}.
Compiling a child file then has the same effect as compiling
the main file with an |\includeonly| command
to select the appropriate child.
Moreover the generated document will carry the name of the child
rather than the main file.
This resolves all three above issues.

This feature is meant to make the editing of books,
thesis documents and lecture notes somewhat more convenient.
However, the package can also be used efficiently for
composing a series of documents (such as exercise sheets)
which are typically distributed individually.
It then assists the author in generating the individual documents
(potentially in different versions)
as well as a document containing the collected series.
Another application is in developing style files
or other kinds of included material
where compilation of the style file could redirect
to a sample or test file.

%%%%%%%%%%%%%%%%%%%%%%%%%%%%%%%%%%%%%%%%%%%%%%%%%%%%%%%%%%%%%%%%%%%%%%%%%%%%%%%%
%%%%%%%%%%%%%%%%%%%%%%%%%%%%%%%%%%%%%%%%%%%%%%%%%%%%%%%%%%%%%%%%%%%%%%%%%%%%%%%%
\section{Usage}

First of all, the package \textsf{childdoc} is \emph{not} a standard
\LaTeXe{} |.sty| style file! Therefore it needs to be invoked in
a non-standard way.

%%%%%%%%%%%%%%%%%%%%%%%%%%%%%%%%%%%%%%%%%%%%%%%%%%%%%%%%%%%%%%%%%%%%%%%%%%%%%%%%
\subsection{Included Files}
\label{sec:include}

%%%%%%%%%%%%%%%%%%%%%%%%%%%%%%%%%%%%%%%%
\DescribeMacro{\childdocmain}
To use the package, add the commands
\begin{center}
\begin{tabular}{l}
|\input{childdoc.def}|\\
|\childdocmain{}|\\
\end{tabular}
\end{center}
at the very top of the main \LaTeX{} file,
in particular \emph{before} the |\documentclass| statement!
The argument of |\childdocmain| should be left empty
(but it must be present).

%%%%%%%%%%%%%%%%%%%%%%%%%%%%%%%%%%%%%%%%
\DescribeMacro{\childdocof}
Furthermore, add the commands
\begin{center}
\begin{tabular}{l}
|\input{childdoc.def}|\\
|\childdocof{|\textit{main}|}|\\
\end{tabular}
\end{center}
at the top of every child file \textit{child}
which is included by |\include{|\textit{child}|}|
from within the main file
(or at least for those files to be compiled individually).
The argument \textit{main} must be the filename of the main file.

There are a couple of
considerations in setting up the main and child documents:

%%%%%%%%%%%%%%%%%%%%%%%%%%%%%%%%%%%%%%%%
\paragraph{Restrictions.}

Please note the following restrictions:
\begin{itemize}
\item
|\childdocmain| must be called with one argument \textit{main}
to ensure compatibility with earlier version of the package.
It must either be empty (|\childdocmain{}|)
or precisely match the filename of the main file in which it is specified.
See \secref{sec:detection} for further information.
\item
The filename \textit{main} must be specified without the |.tex| extension.
\item
The filename \textit{main} is case sensitive
(even in case-insensitive file systems)
due to internal string comparison.
\item
The argument \textit{main} should be fully expanded, it cannot be a macro.
\item
Subdirectories and special characters should be avoided in filenames.
\item
The command |\childdocmain{|\textit{main}|}| must be followed by a whitespace.
It should not be followed immediately by another command
or by a comment mark `|%|'.
This is because the \TeX{} parser reads the token immediately following
the argument of |\childdocmain| and puts it
at the beginning of every child section;
however, a white\-space is ignored.
\end{itemize}

%%%%%%%%%%%%%%%%%%%%%%%%%%%%%%%%%%%%%%%%
\paragraph{Content of Main File.}

It is advisable to place all content in the child files included by |\include|.
Any output contained in the main file will appear in all child documents
unless suppressed manually;
it cannot be suppressed automatically by the |\includeonly| directive
and thus should normally be avoided.
A method to include some content in the main file
by means of conditional processing is described in \secref{sec:conditional}.

%%%%%%%%%%%%%%%%%%%%%%%%%%%%%%%%%%%%%%%%
\paragraph{Page Numbering.}

When only a part of the document is compiled,
the appropriate numbering of pages
(as well as other status parameters)
is determined from the |.aux| files.
The latter contain information from previous passes.
However this information needs to propagate through
all intermediate child documents.
Therefore the page numbering in child documents may well
be inconsistent until the complete document is compiled at least once.

A useful (if unconventional) way to always ensure a consistent
page numbering is to restart the numbering in each child document
and denote the pages by `\textit{child}|.|\textit{page}'
where \textit{child} represents the chapter/section number of the child file.
This can be achieved by the command
|\numberwithin{page}{|\textit{child}|}|
of the \textsf{amsmath} package
where \textit{child} can be |chapter| or |section|
depending on the chosen structuring.
Alternatively, one can modify the macro |\thepage| appropriately
and reset the counter |page| at the start of each child file.

%%%%%%%%%%%%%%%%%%%%%%%%%%%%%%%%%%%%%%%%%%%%%%%%%%%%%%%%%%%%%%%%%%%%%%%%%%%%%%%%
\subsection{Conditional Processing}
\label{sec:conditional}

The package provides a mechanism to compile different versions
of a document. To customise the versions further some conditional processing
can come in handy to distinguish which version is being compiled.
The package provides two macros to describe the compilation context:

%%%%%%%%%%%%%%%%%%%%%%%%%%%%%%%%%%%%%%%%
\DescribeMacro{\ifchilddoc}
The conditional |\ifchilddoc| distinguishes between the compilation of
child documents and the main document:
%
\begin{center}
|\ifchilddoc |\textit{child-code}| |[|\||else |\textit{main-code}]| \||fi|
\end{center}

%%%%%%%%%%%%%%%%%%%%%%%%%%%%%%%%%%%%%%%%
\DescribeMacro{\childdocname}
\DescribeMacro{\childdocjob}
The macro |\childdocname| contains the filename (without extension)
of the main or child file being processed.
Note that |\childdocjob| will always contain the name of the main file.

%%%%%%%%%%%%%%%%%%%%%%%%%%%%%%%%%%%%%%%%
\paragraph{Title Page.}

Conditional processing can be used to include a title or banner page
in the main document when proper precautions are taken.
Importantly, the code in the main file should ensure that the page counter
(as well as other status parameters which are stored in the |.aux| files)
takes the same value after the conditional processing.
Otherwise the page numbers may take divergent values
depending on which part is compiled.

For example, a title page could be declared by:
%
\begin{center}
\begin{tabular}{l}
|\ifchilddoc\||else|\\
|\addtocounter{page}{-1}|\\
\textit{code for title page}\\
|\newpage|\\
|\||fi|
\end{tabular}
\end{center}
%
A banner page for the child documents can be generated by:
%
\begin{center}
\begin{tabular}{l}
|\ifchilddoc|\\
|\addtocounter{page}{-1}|\\
\textit{code for banner page}\\
|\newpage|\\
|\||fi|
\end{tabular}
\end{center}
%
Here one could write a message such as:
\begin{center}
|This is the part \childdocname{} of \childdocjob{}.|
\end{center}

%%%%%%%%%%%%%%%%%%%%%%%%%%%%%%%%%%%%%%%%%%%%%%%%%%%%%%%%%%%%%%%%%%%%%%%%%%%%%%%%
\subsection{Flags}
\label{sec:flags}

The package makes it easy to generate different versions
of the main or child documents.
To this end compilation flags can be defined
and assigned different default values.
They will be particularly useful in conjunction
with the forwarding mechanism described in \secref{sec:forward}.

For example, it may be useful to have a flag |\version|
which can be set to |draft| or |final|.
The document source will contain some conditional code
depending on the value of |\version|.
Suppose further, the flag should default to |final| for the main file
and to |draft| for child files
which is a natural assignment for editing the document.
This is achieved by placing the following code
in the preamble of the main document
(below the |\childdocmain| directive):
%
\begin{center}
\begin{tabular}{l}
|\ifchilddoc|\\
|\providecommand{\version}{draft}|\\
|\||else|\\
|\providecommand{\version}{final}|\\
|\||fi|
\end{tabular}
\end{center}
%
The definition by |\providecommand| makes sure
that previous definitions are not overwritten.
Further statements |\providecommand{\version}{...}|
can thus be added before the above code to override it.

For the main file, one might add a line
(between |\childdocmain| and the above block)
%
\begin{center}
|%\ifchilddoc\||else\providecommand{\version}{draft}\||fi|
\end{center}
%
which can be uncommented to produce a draft version.
Likewise one can add a line to the very top of a child file
(above the |\childdocof{|\textit{main}|}| directive)
%
\begin{center}
|%\providecommand{\version}{final}|
\end{center}
%
which can be uncommented to produce the final version of this child document.

%%%%%%%%%%%%%%%%%%%%%%%%%%%%%%%%%%%%%%%%%%%%%%%%%%%%%%%%%%%%%%%%%%%%%%%%%%%%%%%%
\subsection{Forwarding}
\label{sec:forward}

Different versions of the main or child documents
using compilation flags as described in \secref{sec:flags}
can be (permanently) stored in different files
for convenient compilation, viewing and distribution.
To this end, the package defines a command
to pass on compilation to a different file:

%%%%%%%%%%%%%%%%%%%%%%%%%%%%%%%%%%%%%%%%
\DescribeMacro{\childdocforward}
The command |\childdocforward| redirects processing to
another source file:
%
\begin{center}
\begin{tabular}{l}
|\input{childdoc.def}|\\
|\childdocforward[|\textit{main}|]{|\textit{dest}|}|\\
\end{tabular}
\end{center}
%
The argument \textit{dest} is the destination file
(without extension).
It should be the main file or one of the child files.
Note that further \textsf{childdoc} directives
such as |\childdocof| and |\childdocforward|
in the indicated file will be processed in this form.
The optional argument \textit{main}
passes on directly to the main file \textit{main}
while pretending to compile the child \textit{dest}.
This form behaves as if \textit{dest}
issues |\childdocof{|\textit{main}|}| right away,
and no further \textsf{childdoc} directives will be processed.

%%%%%%%%%%%%%%%%%%%%%%%%%%%%%%%%%%%%%%%%
\DescribeMacro{\...prefix}
In the alternative form |\childdocforwardprefix|,
%
\begin{center}
\begin{tabular}{l}
|\input{childdoc.def}|\\
|\childdocforwardprefix[|\textit{main}|]{|\textit{prefix}|}{|\textit{dest}|}|
\end{tabular}
\end{center}
%
the destination file is determined by a pattern
depending on the current file:
To make this work, the current file must be called
`{\textit{prefix}\hspace{0.2em}\textit{suffix}}'
with \textit{prefix} matching precisely the argument.
Processing is then passed on to the file
`{\textit{dest}\hspace{0.2em}\textit{suffix}}'.
Surely, the same effect is achieved by
directly specifying the
argument `{\textit{dest}\hspace{0.2em}\textit{suffix}}'
in the first form.
However, that requires to set up a different file
for each child. With the alternative form of the command
all these files can have exactly the same content
which simplifies setting them up and maintaining them.

For example, the following file |draft.tex|
with a compilation flag |\version| as described in \secref{sec:flags}
compiles the main document as a draft:
%
\begin{center}
\begin{tabular}{l}
|\def\version{draft}|\\
|\input{childdoc.def}|\\
|\childdocforward{|\textit{main}|}|
\end{tabular}
\end{center}
%
Likewise, the following files |final|\textit{nn}|.tex|
compile the final version of the child document
|child|\textit{nn}|.tex|:
%
\begin{center}
\begin{tabular}{l}
|\def\version{final}|\\
|\input{childdoc.def}|\\
|\childdocforwardprefix{final}{child}|
\end{tabular}
\end{center}
%

Note that when several versions of a main file and/or of each child file
are to be generated, it may be convenient to set up a |Makefile| or
shell script to automatise the process.

%%%%%%%%%%%%%%%%%%%%%%%%%%%%%%%%%%%%%%%%%%%%%%%%%%%%%%%%%%%%%%%%%%%%%%%%%%%%%%%%
\subsection{Command Line Processing}
\label{sec:commandline}

The effect of redirection files can also be achieved by invoking
the \LaTeX{} compiler with a more elaborate command line.
Most conveniently this should be done as part
of a shell script or a |Makefile|.

When using \textsf{childdoc} in the main file, the following
command lines effectively perform a redirection
(note that depending on the shell being used,
backslashes may have to be doubled: `|\|' $\to$ `|\\|'):
%
\begin{center}
|... -jobname "|\textit{target}|" |\\|"|[\textit{flags}]%
|\input{childdoc.def}\childdocforward[|\textit{main}|]{|\textit{dest}|}"|
\end{center}
%
Here \textit{target} is the name of the output file,
\textit{main} is the name of the main file
and \textit{dest} is the name of the main or child file to be processed
(all filenames without extensions).
The optional argument \textit{main} can be omitted
if \textit{main} matches \textit{dest}.
Optionally, compilation \textit{flags} can be defined via |\def| commands.
This command line makes the \TeX{} engine believe
it is compiling the file \textit{target}
whose content is specified as the latter parameter.
The provided code then forwards the processing to
\textit{main} or \textit{dest} as described in \secref{sec:forward}.

%%%%%%%%%%%%%%%%%%%%%%%%%%%%%%%%%%%%%%%%%%%%%%%%%%%%%%%%%%%%%%%%%%%%%%%%%%%%%%%%
\subsection{Include by Input}
\label{sec:input}

Including child documents by |\include| has some restrictions by design.
Most notably, the content of a child document always occupies
its own set of pages; pages cannot be shared between child documents.
Usually, this behaviour makes perfect sense
because each child document contain an essential part of the document.
However, in some situations it may be desirable to compose
a document from a collection of parts
without having mandatory page breaks between then.
For this case, the package
provides a mechanism to include parts
by |\input| which can also be processed individually.
However, by construction this mechanism
requires manual handling of the content to be output.

%%%%%%%%%%%%%%%%%%%%%%%%%%%%%%%%%%%%%%%%
\DescribeMacro{\ifchilddocmanual}
The main file should be prepared as usual, see \secref{sec:include}.
However, the document body must make a distinction
between processing of an individual part and of the main document, e.g.:
%
\begin{center}
\begin{tabular}{l}
|\ifchilddocmanual|\\
|\input{\childdocname}|\\
|\||else|\\
\textit{document body with }|\input{|\textit{part}|}|\\
|\||fi|
\end{tabular}
\end{center}
%
The conditional |\ifchilddocmanual| is true whenever
a part to be included by |\input| is being compiled,
and the name of the part is stored in |\childdocname|.

%%%%%%%%%%%%%%%%%%%%%%%%%%%%%%%%%%%%%%%%
\DescribeMacro{\childdocby}
Each part to be included by |\input| should start with:
%
\begin{center}
\begin{tabular}{l}
|\input{childdoc.def}|\\
|\childdocby{|\textit{main}|}|\\
\end{tabular}
\end{center}
%
The directive |\childdocby| is similar to |\childdocof|
described in \secref{sec:include},
but the subsequent selection of content must be done manually.
To that end, both |\ifchilddoc| and |\ifchilddocmanual|
will be true upon processing of a part,
and the name of the part is stored in |\childdocname|.
Note that |\jobname| will be set to the filename of the current part
so that each part receives an individual |.aux| file
that does not interfere with the |.aux| file(s) of the main document.
This behaviour can be altered by the alternative form
|\childdocby[*]{|\textit{main}|}| (with a non-empty optional argument)
which uses the |.aux| file of the main document
by setting |\jobname| to \textit{main}.

%%%%%%%%%%%%%%%%%%%%%%%%%%%%%%%%%%%%%%%%%%%%%%%%%%%%%%%%%%%%%%%%%%%%%%%%%%%%%%%%
\subsection{Driver Development}
\label{sec:driver}

The \textsf{childdoc} mechanism can also be use for the development
of definition files such as \LaTeX{} styles or classes.
This case differs from the above setup with multiple parts
included by |\include| in that no |\includeonly| should be invoked.
This can be achieved by starting the include file
(before |\ProvidesPackage|) with:
%
\begin{center}
\begin{tabular}{l}
|\input{childdoc.def}|\\
|\childdocforward{|\textit{main}|}|\\
\end{tabular}
\end{center}
%
or alternatively with:
%
\begin{center}
\begin{tabular}{l}
|\input{childdoc.def}|\\
|\childdocby{|\textit{main}|}|\\
\end{tabular}
\end{center}
%
Both forms have slightly different effects as described above.
The main file is prepared as usual, see \secref{sec:include}.

%%%%%%%%%%%%%%%%%%%%%%%%%%%%%%%%%%%%%%%%%%%%%%%%%%%%%%%%%%%%%%%%%%%%%%%%%%%%%%%%
\subsection{Legacy Detection}
\label{sec:detection}

The directive |\childdocmain| in the main file can detect
whether the complete document or merely a child is to be compiled
even without using the directive |\childdocof|.
This method is deprecated because it is less robust
and there is no compelling reason to use it;
it is merely provided for backward compatibility
and it may be removed in future versions.

If the detection mechanism is to be used,
it is mandatory to correctly specify
the filename of the main file as the argument of |\childdocmain|:
%
\begin{center}
\begin{tabular}{l}
|\input{childdoc.def}|\\
|\childdocmain{|\textit{main}|}|\\
\end{tabular}
\end{center}
%
If |\jobname| does not match the argument \textit{main} of |\childdocmain|,
it is assumed that |\jobname| points to the child file to be compiled.
When using |\childdocmain| with the main file specified as argument,
it suffices to start a child file
with just |\input{|\textit{main}|}|
without loading of the package and using |\childdocof|.
If instead all processing is done
with the appropriate \textsf{childdoc} directives,
the argument of \textit{main} of |\childdocmain| can be empty.

An alternative version of the command line processing described
in \secref{sec:commandline} using the detection mechanism reads:
%
\begin{center}
|... -jobname "|\textit{target}|" "|[\textit{flags}]%
[|\def\jobname{|\textit{dest}|}|]|\input{|\textit{main}|}"|
\end{center}

%%%%%%%%%%%%%%%%%%%%%%%%%%%%%%%%%%%%%%%%%%%%%%%%%%%%%%%%%%%%%%%%%%%%%%%%%%%%%%%%
\subsection{Manual Code}
\label{sec:manual}

In case one cannot be certain whether the definitions file |childdoc.def|
is installed on the target \TeX{} distribution
and one prefers not to ship it,
it is conceivable to paste a few relevant commands into the sources.

To that end, drop all statements |\input{childdoc.def}|
and perform the replacements as outlined below.
Instead of |\childdocmain{|\textit{main}|}| add the following code
to the top of the main file:
%
\begin{center}
\begin{tabular}{l}
|\||ifdefined\childdocname\endinput\||fi\newif\ifchilddoc|\\
|\edef\childdocname{\scantokens\expandafter{\jobname\noexpand}}|\\
|\def\childdocmain{|\textit{main}|}\||ifx\childdocmain\childdocname\||else|\\
|\childdoctrue\includeonly{\childdocname}\let\jobname\childdocmain\||fi|\\
\end{tabular}
\end{center}
%
Instead of |\childdocof{|\textit{main}|}| just include the main file
at the top of each child file:
%
\begin{center}
|\input{|\textit{main}|}|
\end{center}
%
A simple redirection |\childdocforward{|\textit{dest}|}| is achieved by:
%
\begin{center}
|\def\jobname{|\textit{dest}|}\input{\jobname}|
\end{center}
%
The redirection with prefix
|\childdocforwardprefix[|\textit{prefix}|]{|\textit{dest}|}|
is accomplished by:
%
\begin{center}
\begin{tabular}{l}
|{\edef\jobname{\scantokens\expandafter{\jobname\noexpand}}|\\
|\def\redirectjob |\textit{prefix}|#1~~~{\gdef\jobname{|\textit{dest}|#1}}|\\
|\expandafter\redirectjob\jobname~~~}\input{\jobname}|
\end{tabular}
\end{center}

In an alternative approach,
child documents can be compiled by a specific command line
without additional code or specific definitions:
%
\begin{center}
|... -jobname "|\textit{target}|" "|[\textit{flags}]%
|\includeonly{|\textit{dest}|}\input{|\textit{main}|}"|
\end{center}
%

%%%%%%%%%%%%%%%%%%%%%%%%%%%%%%%%%%%%%%%%%%%%%%%%%%%%%%%%%%%%%%%%%%%%%%%%%%%%%%%%
%%%%%%%%%%%%%%%%%%%%%%%%%%%%%%%%%%%%%%%%%%%%%%%%%%%%%%%%%%%%%%%%%%%%%%%%%%%%%%%%
\section{Information}

%%%%%%%%%%%%%%%%%%%%%%%%%%%%%%%%%%%%%%%%%%%%%%%%%%%%%%%%%%%%%%%%%%%%%%%%%%%%%%%%
\subsection{Copyright}

Copyright \copyright{} 2017--2018 Niklas Beisert

This work may be distributed and/or modified under the
conditions of the \LaTeX{} Project Public License, either version 1.3
of this license or (at your option) any later version.
The latest version of this license is in
  \url{http://www.latex-project.org/lppl.txt}
and version 1.3 or later is part of all distributions of \LaTeX{}
version 2005/12/01 or later.

This work has the LPPL maintenance status `maintained'.

The Current Maintainer of this work is Niklas Beisert.

This work consists of the files |README.txt|, |childdoc.ins| and |childdoc.dtx|
as well as the derived files |childdoc.def|, |cdocsamp.tex|
with |cdocsch1.tex|, |cdocsch2.tex|, |cdocspt3.tex|, |cdocspt4.tex|,
|cdocsdrf.tex|, |cdocsfn1.tex|, |cdocsfn2.tex|
as well as |childdoc.pdf|.

%%%%%%%%%%%%%%%%%%%%%%%%%%%%%%%%%%%%%%%%%%%%%%%%%%%%%%%%%%%%%%%%%%%%%%%%%%%%%%%%
\subsection{Files and Installation}

The package consists of the files:
%
\begin{center}
\begin{tabular}{ll}
    |README.txt|   & readme file \\
    |childdoc.ins| & installation file \\
    |childdoc.dtx| & source file \\
    |childdoc.def| & definition file \\
    |cdocsamp.tex| & sample main file \\
    |cdocsch1.tex| & sample include file \\
    |cdocsch2.tex| & sample include file \\
    |cdocspt3.tex| & sample part file \\
    |cdocspt4.tex| & sample part file \\
    |cdocsdrf.tex| & sample redirection file \\
    |cdocsfn1.tex| & sample redirection file \\
    |cdocsfn2.tex| & sample redirection file \\
    |childdoc.pdf| & manual
\end{tabular}
\end{center}
%
The distribution consists of the files
|README.txt|, |childdoc.ins| and |childdoc.dtx|.
%
\begin{itemize}
\item
Run (pdf)\LaTeX{} on |childdoc.dtx|
to compile the manual |childdoc.pdf| (this file).
\item
Run \LaTeX{} on |childdoc.ins| to create the definitions file |childdoc.def|
and the sample |cdocsamp.tex| with include files
|cdocsch1.tex|, |cdocsch2.tex|, |cdocspt3.tex|, |cdocspt4.tex|,
|cdocsdrf.tex|, |cdocsfn1.tex|, |cdocsfn2.tex|.
Then copy the file |childdoc.def| to an appropriate directory of your \LaTeX{}
distribution, e.g.\ \textit{texmf-root}|/tex/latex/childdoc|.
\end{itemize}

%%%%%%%%%%%%%%%%%%%%%%%%%%%%%%%%%%%%%%%%%%%%%%%%%%%%%%%%%%%%%%%%%%%%%%%%%%%%%%%%
\subsection{Related CTAN Packages}

There are several other packages which offer a similar functionality:
%
\begin{itemize}
\item
The packages
\href{http://ctan.org/pkg/docmute}{\textsf{docmute}},
\href{http://ctan.org/pkg/includex}{\textsf{includex}} and
\href{http://ctan.org/pkg/standalone}{\textsf{standalone}}
provide commands to include only the document body of
a child file thus allowing both files to be compiled individually.
\item
The packages \href{http://ctan.org/pkg/subdocs}{\textsf{subdocs}}
and \href{http://ctan.org/pkg/subfiles}{\textsf{subfiles}}
provide structures in which the main and child documents can be
encapsulated and allowing them to be compiled individually.
The inclusion mechanism is different from the conventional |\include|.
\item
The package \href{http://ctan.org/pkg/combine}{\textsf{combine}}
is an elaborate solution to combine several documents into one.
\end{itemize}
%
See also the CTAN topic \href{http://ctan.org/topic/subdocs}{\textsf{subdocs}}
for further related packages.
The present package differs from the above solutions in that
a document structure constructed with the conventional |\include| mechanism
just needs two extra commands at the top of every file
such that all constituent files can be compiled individually.

%%%%%%%%%%%%%%%%%%%%%%%%%%%%%%%%%%%%%%%%%%%%%%%%%%%%%%%%%%%%%%%%%%%%%%%%%%%%%%%%
%\subsection{Feature Suggestions}
%
%The following is a list of features which may be useful for future
%versions of this package:
%%
%\begin{itemize}
%\item
%\ldots
%\end{itemize}

%%%%%%%%%%%%%%%%%%%%%%%%%%%%%%%%%%%%%%%%%%%%%%%%%%%%%%%%%%%%%%%%%%%%%%%%%%%%%%%%
\subsection{Revision History}

%%%%%%%%%%%%%%%%%%%%%%%%%%%%%%%%%%%%%%%%
\paragraph{v2.0:} 2018/12/30

\begin{itemize}
\item
immediate forward processing
\item
added |\childdocby| mechanism
\item
manual restructured
\end{itemize}

%%%%%%%%%%%%%%%%%%%%%%%%%%%%%%%%%%%%%%%%
\paragraph{v1.6:} 2018/01/17

\begin{itemize}
\item
application for development of include files
\item
corrections to manual
\end{itemize}

%%%%%%%%%%%%%%%%%%%%%%%%%%%%%%%%%%%%%%%%
\paragraph{v1.5:} 2017/05/21

\begin{itemize}
\item
more complete structuring introduced
\item
|\childdocof| introduced
\item
|\childdoc| renamed to |\childdocmain|
\item
|\childredirect| renamed to |\childdocforward| and |\childdocforwardprefix|
and functionality expanded
\end{itemize}

%%%%%%%%%%%%%%%%%%%%%%%%%%%%%%%%%%%%%%%%
\paragraph{v1.0:} 2017/04/27

\begin{itemize}
\item
manual and install package
\item
first version published on CTAN
\end{itemize}

%%%%%%%%%%%%%%%%%%%%%%%%%%%%%%%%%%%%%%%%
\paragraph{v0.6:} 2017/04/26

\begin{itemize}
\item
redirection mechanism added
\end{itemize}

%%%%%%%%%%%%%%%%%%%%%%%%%%%%%%%%%%%%%%%%
\paragraph{v0.5:} 2017/04/26

\begin{itemize}
\item
functionality in definition file
\end{itemize}


%%%%%%%%%%%%%%%%%%%%%%%%%%%%%%%%%%%%%%%%%%%%%%%%%%%%%%%%%%%%%%%%%%%%%%%%%%%%%%%%
%%%%%%%%%%%%%%%%%%%%%%%%%%%%%%%%%%%%%%%%%%%%%%%%%%%%%%%%%%%%%%%%%%%%%%%%%%%%%%%%
%%%%%%%%%%%%%%%%%%%%%%%%%%%%%%%%%%%%%%%%%%%%%%%%%%%%%%%%%%%%%%%%%%%%%%%%%%%%%%%%
\appendix

\settowidth\MacroIndent{\rmfamily\scriptsize 000\ }

 \DocInput{childdoc.dtx}

\end{document}
%</driver>
% \fi
%
% %%%%%%%%%%%%%%%%%%%%%%%%%%%%%%%%%%%%%%%%%%%%%%%%%%%%%%%%%%%%%%%%%%%%%%%%%%%%%%
% %%%%%%%%%%%%%%%%%%%%%%%%%%%%%%%%%%%%%%%%%%%%%%%%%%%%%%%%%%%%%%%%%%%%%%%%%%%%%%
% \section{Sample}
%\iffalse
%<*samplemain>
%\fi
%
% The following presents a sample document
% with two chapters, two parts, a title page,
% a compile flag as well as three forwarding files to set the flag.
% It consists of eight |.tex| files:
% \begin{center}
% \begin{tabular}{ll}
% |cdocsamp.tex|&main file\\
% |cdocsch1.tex|&include file for chapter 1\\
% |cdocsch2.tex|&include file for chapter 2\\
% |cdocspt3.tex|&include file for part 3\\
% |cdocspt4.tex|&include file for part 4\\
% |cdocsdrf.tex|&forwarding file for main file in draft mode\\
% |cdocsfi1.tex|&forwarding file for final version of chapter 1\\
% |cdocsfi2.tex|&forwarding file for final version of chapter 2\\
% \end{tabular}
% \end{center}
% Each of the eight files can be compiled directly by the \LaTeX{} compiler.
%
% %%%%%%%%%%%%%%%%%%%%%%%%%%%%%%%%%%%%%%
% \paragraph{Main File.}
%
% The main file is called |cdocsamp.tex|.
%
% Load the \textsf{childdoc} definitions and
% declare the filename for the main document:
%    \begin{macrocode}
\input{childdoc.def}
\childdocmain{}
%    \end{macrocode}

% Optional override for |\version| flag:
%    \begin{macrocode}
%%\ifchilddoc\else\providecommand{\version}{draft}\fi
%    \end{macrocode}

% Define the default values for the |\version| flag
% (|final| for the main file and |draft| for childs):
%    \begin{macrocode}
\ifchilddoc
\providecommand{\version}{draft}
\else
\providecommand{\version}{final}
\fi
%    \end{macrocode}

% Load the standard document class:
%    \begin{macrocode}
\documentclass[12pt]{article}
%    \end{macrocode}

% Start the document body:
%    \begin{macrocode}
\begin{document}
%    \end{macrocode}

% Declare a title page.
% Print title, part of document being processed and version flag:
%    \begin{macrocode}
\addtocounter{page}{-1}
\begin{center}
{\LARGE\bfseries{}childdoc example\par}
\vspace{1cm}
\ifchilddoc
\ifchilddocmanual part\else chapter\fi:
`\childdocname' of `\childdocjob'\par
\else
main document: `\childdocjob'\par
\fi
version: \version\par
\end{center}
\newpage
%    \end{macrocode}

% Manually include selected file,
% otherwise process as usual:
%    \begin{macrocode}
\ifchilddocmanual
\section*{part `\childdocname'}
\input{\childdocname}
\else
%    \end{macrocode}

% Include the two chapters:
%    \begin{macrocode}
\include{cdocsch1}
\include{cdocsch2}
%    \end{macrocode}

% Include the two parts unless only chapters should be displayed:
%    \begin{macrocode}
\ifchilddoc\else
\section{part three}
\input{cdocspt3}
\section{part four}
\input{cdocspt4}
\fi
%    \end{macrocode}

% Process as usual until here:
%    \begin{macrocode}
\fi
%    \end{macrocode}

% End of document body:
%    \begin{macrocode}
\end{document}
%    \end{macrocode}
%\iffalse
%</samplemain>
%\fi
%
% %%%%%%%%%%%%%%%%%%%%%%%%%%%%%%%%%%%%%%
% \paragraph{Chapter Include Files.}
%
% The include files are called |cdocsch1.tex| and |cdocsch2.tex|.
%
%\iffalse
%<*samplechap1|samplechap2>
%\fi

% Optional override for |\version| flag:
%    \begin{macrocode}
%%\providecommand{\version}{final}
%    \end{macrocode}

% Include the main document:
%    \begin{macrocode}
\input{childdoc.def}
\childdocof{cdocsamp}
%    \end{macrocode}

%\iffalse
%</samplechap1|samplechap2>
%\fi
%
%\iffalse
%<*samplechap1>
%\fi
% Some text for chapter 1:
%    \begin{macrocode}
\section{one}
some text in chapter one
%    \end{macrocode}

%\iffalse
%</samplechap1>
%\fi
% Some text for chapter 2:
%\iffalse
%<*samplechap2>
%\fi
%    \begin{macrocode}
\section{two}
more text in chapter two
%    \end{macrocode}

%\iffalse
%</samplechap2>
%\fi
%
% %%%%%%%%%%%%%%%%%%%%%%%%%%%%%%%%%%%%%%
% \paragraph{Part Include Files.}
%
% The include files are called |cdocspt3.tex| and |cdocspt4.tex|.
%
%\iffalse
%<*samplepart3|samplepart4>
%\fi

% Optional override for |\version| flag:
%    \begin{macrocode}
%%\providecommand{\version}{final}
%    \end{macrocode}

% Include the main document:
%    \begin{macrocode}
\input{childdoc.def}
\childdocby{cdocsamp}
%    \end{macrocode}

%\iffalse
%</samplepart3|samplepart4>
%\fi
%
%\iffalse
%<*samplepart3>
%\fi
% Some text for part 3:
%    \begin{macrocode}
some text in part three
%    \end{macrocode}

%\iffalse
%</samplepart3>
%\fi
% Some text for part 4:
%\iffalse
%<*samplepart4>
%\fi
%    \begin{macrocode}
more text in part four
%    \end{macrocode}

%\iffalse
%</samplepart4>
%\fi
%
% %%%%%%%%%%%%%%%%%%%%%%%%%%%%%%%%%%%%%%
% \paragraph{Forwarding for a Complete Draft.}
%
% The following forwarding file |cdocsdrf.tex|
% compiles the main document in draft mode:
%\iffalse
%<*sampledraft>
%\fi
%    \begin{macrocode}
\def\version{draft}
\input{childdoc.def}
\childdocforward{cdocsamp}
%    \end{macrocode}

%\iffalse
%</sampledraft>
%\fi
%
% %%%%%%%%%%%%%%%%%%%%%%%%%%%%%%%%%%%%%%
% \paragraph{Forwarding for Final Version of the Chapters.}
%
% The following forwarding files |cdocsfn1.tex| and |cdocsfn2.tex|
% (with identical content)
% compile the final versions of the child documents
% |cdocsch1.tex| and |cdocsch2.tex|, respectively:
%\iffalse
%<*samplefinal>
%\fi
%    \begin{macrocode}
\def\version{final}
\input{childdoc.def}
\childdocforwardprefix[cdocsamp]{cdocsfn}{cdocsch}
%    \end{macrocode}

%\iffalse
%</samplefinal>
%\fi
%
% %%%%%%%%%%%%%%%%%%%%%%%%%%%%%%%%%%%%%%
% \paragraph{Command Line Processing.}
%
% The following three command lines generate the output files
% |cdocscld|, |cdocscl1| and |cdocscl2|
% which should be identical to
% |cdocsdrf|, |cdocsch1| and |cdocsfn2|, respectively:
% \begin{center}
% \begin{tabular}{l}
% |latex -jobname cdocscld \|\\
% |  "\def\version{draft}\input{childdoc.def}\childdocforward{cdocsamp}"|\\
% |latex -jobname cdocscl1 \|\\
% |  "\input{childdoc.def}\childdocforward[cdocsamp]{cdocsch1}"|\\
% |latex -jobname cdocscl2 \|\\
% |  "\def\version{final}\input{childdoc.def}\childdocforward{cdocsch2}"|
% \end{tabular}
% \end{center}
% Note that the trailing backslash on each first line
% merely continues the input to the second line
% (for convenient cut ant paste).
% Furthermore, the command |latex| can be replaced by any
% of its alternative versions such as |pdflatex|.
%
% %%%%%%%%%%%%%%%%%%%%%%%%%%%%%%%%%%%%%%%%%%%%%%%%%%%%%%%%%%%%%%%%%%%%%%%%%%%%%%
% %%%%%%%%%%%%%%%%%%%%%%%%%%%%%%%%%%%%%%%%%%%%%%%%%%%%%%%%%%%%%%%%%%%%%%%%%%%%%%
% \section{Implementation}
%\iffalse
%<*package>
%\fi
%
% This section describes the definitions file |childdoc.def|.

% The definitions cannot be loaded using |\usepackage| or |\RequirePackage|
% which has a mechanism to prevent loading a style file more than once.
% When loading the definitions by means of |\input|
% multiple instances have to be prevented manually:
%\iffalse
%This code needs to be before the `\ProvidesFile' directive
%which is defined at the beginning of this file.
%Therefore it is also placed there and commented out here.
%</package>
%<*discard>
%\fi
%    \begin{macrocode}
\ifdefined\childdocmain\endinput\fi
%    \end{macrocode}
%\iffalse
%</discard>
%<*package>
%\fi
%
% \macro{\ifchilddoc}
% \macro{\ifchilddocmanual}
% The conditional |\ifchilddoc| tells whether a
% child (true) or main (false) document is being compiled.
% The conditional |\ifchilddocmanual| tells whether
% the |\includeonly| mechanism is used (false) or
% the selection of child files must be performed manually (true).
% The definitions initialise to false:
%    \begin{macrocode}
\newif\ifchilddoc
\newif\ifchilddocmanual
%    \end{macrocode}

% \macro{\childdocname}
% \macro{\childdocjob}
% The macro |\childdocname| stores the name of the main document
% to be compiled. The macro |\childdocjob| stores the name of
% the document on which the \LaTeX{} compiler was originally invoked.
% The content of |\jobname| cannot be compared
% to filenames specified in the source due to different catcodes.
% The following code rescans |\jobname|, stores the result
% in |\childdocname| and saves a copy in |\childdocjob|:
%    \begin{macrocode}
\edef\childdocname{\scantokens\expandafter{\jobname\noexpand}}
\let\childdocjob\childdocname
%    \end{macrocode}

% \macro{\childdocdisable}
% The macro |\childdocdisable| prevents the main file
% from being processed more than once.
% At this stage, the main document command |\childdocmain|
% is assumed to be called once again where it should do nothing.
% Any subsequent call to it should prevent
% a secondary processing of the main document
% It overwrites the forwarding commands
% |\childdocof| and |\childdocforward|
% with empty macros to prevent further inclusions of the main document:
%    \begin{macrocode}
\newcommand{\childdocdisable}
{
  \renewcommand{\childdocmain}[1]{\renewcommand{\childdocmain}[1]{\endinput}}
  \renewcommand{\childdocof}[1]{}
  \renewcommand{\childdocby}[2][]{}
  \renewcommand{\childdocforward}[2][]{}
  \renewcommand{\childdocdisable}{}
}
%    \end{macrocode}

% \macro{\childdocmain}
% The macro |\childdocmain| is to be called at the top of the main file
% with nothing or the main filename (without extension) as argument.
% First, it breaks loops.
% If the argument is not empty and does not match |\childdocname|
% (which is set by the first inclusion of |childdoc.def|),
% |\ifchilddoc| is set to true, |\includeonly| is applied to the child file
% and |\jobname| is set to the main file
% (for proper handling of |.aux| files):
%    \begin{macrocode}
\newcommand{\childdocmain}[1]
{
  \childdocdisable\childdocmain{}
  \if?#1?\else
    \begingroup
      \def\childdoctmp{#1}
      \ifx\childdoctmp\childdocname
        \def\childdoctmp{}
      \else
        \def\childdoctmp
        {
          \childdoctrue
          \includeonly{\childdocname}
          \def\childdocjob{#1}
          \def\jobname{#1}
        }
      \fi
      \expandafter
    \endgroup
    \childdoctmp
  \fi
}
%    \end{macrocode}

% \macro{\childdocof}
% The command |\childdocof| redirects
% compilation to the main file |#1|.
%    \begin{macrocode}
\newcommand{\childdocof}[1]
{
  \childdocdisable
  \childdoctrue
  \includeonly{\childdocname}
  \def\jobname{#1}
  \def\childdocjob{#1}
  \input{#1}
}
%    \end{macrocode}

% \macro{\childdocby}
% The command |\childdocby| ....
%    \begin{macrocode}
\newcommand{\childdocby}[2][]
{
  \childdocdisable
  \childdoctrue
  \childdocmanualtrue
  \if?#1?\else
    \def\jobname{#2}
  \fi
  \def\childdocjob{#2}
  \input{#2}
  \endinput
}
%    \end{macrocode}

% \macro{\childdocforward}
% The command |\childdocforward| redirects
% compilation to the main file or
% (if the optional argument is given) a child file.
% Parameters are set as if the main file
% or a child file starting with |\childdocof| was compiled.
% Then compilation is handed over to the main file:
%    \begin{macrocode}
\newcommand{\childdocforward}[2][]
{
  \begingroup
    \if?#1?
      \def\childdoctmp
      {
        \def\childdocname{#2}
        \def\childdocjob{#2}
        \def\jobname{#2}
        \input{#2}
        \endinput
      }
    \else
      \def\childdoctmp
      {
        \childdocdisable
        \def\childdocname{#2}
        \childdoctrue
        \includeonly{#2}
        \def\childdocjob{#1}
        \def\jobname{#1}
        \input{#1}
        \endinput
      }
    \fi
    \expandafter
  \endgroup
  \childdoctmp
}
%    \end{macrocode}

% \macro{\childdocforwardprefix}
% The command |\childdocforwardprefix| redirects
% compilation to the main or a child file by means of a pattern.
% The prefix |#1| in the current filename is replaced by |#2|
% and the suffix of the current filename is kept
% (it is assumed that the filename does not contain the substring `|~~~|'
% which is used as a delimiter).
% Compilation is handed over to the new file by |\childdocforward|:
%    \begin{macrocode}
\newcommand{\childdocforwardprefix}[3][]
{
  \begingroup
    \def\childdocextract #2##1~~~{\def\childdoctmp{\childdocforward[#1]{#3##1}}}
    \expandafter\childdocextract\childdocname~~~
    \expandafter
  \endgroup
  \childdoctmp
}
%    \end{macrocode}

% \macro{\childdoc}
% The deprecated macro |\childdoc| is a legacy version of |\childdocmain|:
%    \begin{macrocode}
\newcommand{\childdoc}{\childdocmain}
%    \end{macrocode}

% \macro{\childdocredirect}
% The deprecated macro |\childdocredirect| is a legacy version
% of |\childdocforward| and |\childdocforwardprefix|:
%    \begin{macrocode}
\newcommand{\childdocredirect}[2][]
{
  \begingroup
    \if?#1?
      \def\childdoctmp{\childdocforward{#2}}
    \else
      \def\childdoctmp{\childdocforwardprefix{#1}{#2}}
    \fi
    \expandafter
  \endgroup
  \childdoctmp
}
%    \end{macrocode}

%\iffalse
%</package>
%\fi
%
\endinput
|\\
|\childdocforward{|\textit{main}|}|\\
\end{tabular}
\end{center}
%
or alternatively with:
%
\begin{center}
\begin{tabular}{l}
|% \iffalse
%
% childdoc.dtx Copyright (C) 2017-2018 Niklas Beisert
%
% This work may be distributed and/or modified under the
% conditions of the LaTeX Project Public License, either version 1.3
% of this license or (at your option) any later version.
% The latest version of this license is in
%   http://www.latex-project.org/lppl.txt
% and version 1.3 or later is part of all distributions of LaTeX
% version 2005/12/01 or later.
%
% This work has the LPPL maintenance status `maintained'.
%
% The Current Maintainer of this work is Niklas Beisert.
%
% This work consists of the files childdoc.dtx and childdoc.ins
% and the derived files childdoc.def and cdocsamp.tex with
% cdocsch1.tex, cdocsch2.tex, cdocsdrf.tex, cdocsfn1.tex, cdocsfn2.tex.
%
%<package>\ifdefined\childdocmain\endinput\fi
%<package>\ProvidesFile{childdoc.def}[2018/12/30 v2.0 child document driver]
%<samplemain>\ProvidesFile{cdocsamp.tex}[2018/12/30 v2.0 sample for childdoc]
%<*driver>
%\ProvidesFile{childdoc.drv}[2018/12/30 v2.0 childdoc reference manual file]
\PassOptionsToClass{10pt,a4paper}{article}
\documentclass{ltxdoc}

\usepackage[margin=35mm]{geometry}
\usepackage{hyperref}
\usepackage{hyperxmp}
\usepackage[usenames]{color}

\hypersetup{colorlinks=true}
\hypersetup{pdfstartview=FitH}
\hypersetup{pdfpagemode=UseNone}
\hypersetup{pdfsource={}}
\hypersetup{pdflang={en-UK}}
\hypersetup{pdfcopyright={Copyright 2017-2018 Niklas Beisert.
  This work may be distributed and/or modified under the
  conditions of the LaTeX Project Public License, either version 1.3
  of this license or (at your option) any later version.}}
\hypersetup{pdflicenseurl={http://www.latex-project.org/lppl.txt}}
\hypersetup{pdfcontactaddress={ETH Zurich, ITP, HIT K,
  Wolfgang-Pauli-Strasse 27}}
\hypersetup{pdfcontactpostcode={8093}}
\hypersetup{pdfcontactcity={Zurich}}
\hypersetup{pdfcontactcountry={Switzerland}}
\hypersetup{pdfcontactemail={nbeisert@itp.phys.ethz.ch}}
\hypersetup{pdfcontacturl={http://people.phys.ethz.ch/\xmptilde nbeisert/}}

\newcommand{\secref}[1]{\hyperref[#1]{section \ref*{#1}}}

\parskip1ex
\parindent0pt
\let\olditemize\itemize
\def\itemize{\olditemize\parskip0pt}

\begin{document}

\title{The \textsf{childdoc} Package}
\hypersetup{pdftitle={The childdoc Package}}
\author{Niklas Beisert\\[2ex]
  Institut f\"ur Theoretische Physik\\
  Eidgen\"ossische Technische Hochschule Z\"urich\\
  Wolfgang-Pauli-Strasse 27, 8093 Z\"urich, Switzerland\\[1ex]
  \href{mailto:nbeisert@itp.phys.ethz.ch}
  {\texttt{nbeisert@itp.phys.ethz.ch}}}
\hypersetup{pdfauthor={Niklas Beisert}}
\hypersetup{pdfsubject={Manual for the LaTeX2e Package childdoc}}
\date{30 December 2018, \textsf{v2.0}}
\maketitle

\begin{abstract}\noindent
\textsf{childdoc} is a \LaTeXe{} package
that enables the direct compilation
of document sections included by |\include|
to individual files.
\end{abstract}

\begingroup
\parskip0ex
\tableofcontents
\endgroup

%%%%%%%%%%%%%%%%%%%%%%%%%%%%%%%%%%%%%%%%%%%%%%%%%%%%%%%%%%%%%%%%%%%%%%%%%%%%%%%%
%%%%%%%%%%%%%%%%%%%%%%%%%%%%%%%%%%%%%%%%%%%%%%%%%%%%%%%%%%%%%%%%%%%%%%%%%%%%%%%%
\section{Introduction}

\LaTeX{} provides a mechanism to structure a large document (such as a book)
into a main file and several child files (containing the chapters)
using the |\include| command.
This mechanism is beneficial for documents
which span hundreds of pages in order to
make the source file(s) more manageable.
Moreover, compilation can be restricted to
selected child files by means of the |\includeonly| command.
The latter feature can be used to reduce the compilation time while editing
(this was significantly more useful in the earlier days of \LaTeX{})
or to generate a smaller document which is easier to navigate.
Another application of |\includeonly| is to generate
documents consisting of selected parts of the complete document.

However, there are a few drawbacks of the plain |\include| mechanism:
\begin{itemize}
\item
The child files cannot be compiled on their own,
they can only be compiled via the main file.
A naive editing environment
(such as a text editor with an option
to have the current file processed by \LaTeX)
may require one to switch to the main file before compiling;
attempting to compile the child file produces errors.
\item
The main file must be modified (each time)
to adjust the |\includeonly| command
to the present needs. This easily leaves the main file in a messy state.
\item
The generated document will always carry the filename
of the main document. This is inconvenient if
several child files are to be compiled and
to be kept for distribution.
\end{itemize}

The present package provides a simple interface
to make child files individually compilable by \LaTeX{}.
Compiling a child file then has the same effect as compiling
the main file with an |\includeonly| command
to select the appropriate child.
Moreover the generated document will carry the name of the child
rather than the main file.
This resolves all three above issues.

This feature is meant to make the editing of books,
thesis documents and lecture notes somewhat more convenient.
However, the package can also be used efficiently for
composing a series of documents (such as exercise sheets)
which are typically distributed individually.
It then assists the author in generating the individual documents
(potentially in different versions)
as well as a document containing the collected series.
Another application is in developing style files
or other kinds of included material
where compilation of the style file could redirect
to a sample or test file.

%%%%%%%%%%%%%%%%%%%%%%%%%%%%%%%%%%%%%%%%%%%%%%%%%%%%%%%%%%%%%%%%%%%%%%%%%%%%%%%%
%%%%%%%%%%%%%%%%%%%%%%%%%%%%%%%%%%%%%%%%%%%%%%%%%%%%%%%%%%%%%%%%%%%%%%%%%%%%%%%%
\section{Usage}

First of all, the package \textsf{childdoc} is \emph{not} a standard
\LaTeXe{} |.sty| style file! Therefore it needs to be invoked in
a non-standard way.

%%%%%%%%%%%%%%%%%%%%%%%%%%%%%%%%%%%%%%%%%%%%%%%%%%%%%%%%%%%%%%%%%%%%%%%%%%%%%%%%
\subsection{Included Files}
\label{sec:include}

%%%%%%%%%%%%%%%%%%%%%%%%%%%%%%%%%%%%%%%%
\DescribeMacro{\childdocmain}
To use the package, add the commands
\begin{center}
\begin{tabular}{l}
|\input{childdoc.def}|\\
|\childdocmain{}|\\
\end{tabular}
\end{center}
at the very top of the main \LaTeX{} file,
in particular \emph{before} the |\documentclass| statement!
The argument of |\childdocmain| should be left empty
(but it must be present).

%%%%%%%%%%%%%%%%%%%%%%%%%%%%%%%%%%%%%%%%
\DescribeMacro{\childdocof}
Furthermore, add the commands
\begin{center}
\begin{tabular}{l}
|\input{childdoc.def}|\\
|\childdocof{|\textit{main}|}|\\
\end{tabular}
\end{center}
at the top of every child file \textit{child}
which is included by |\include{|\textit{child}|}|
from within the main file
(or at least for those files to be compiled individually).
The argument \textit{main} must be the filename of the main file.

There are a couple of
considerations in setting up the main and child documents:

%%%%%%%%%%%%%%%%%%%%%%%%%%%%%%%%%%%%%%%%
\paragraph{Restrictions.}

Please note the following restrictions:
\begin{itemize}
\item
|\childdocmain| must be called with one argument \textit{main}
to ensure compatibility with earlier version of the package.
It must either be empty (|\childdocmain{}|)
or precisely match the filename of the main file in which it is specified.
See \secref{sec:detection} for further information.
\item
The filename \textit{main} must be specified without the |.tex| extension.
\item
The filename \textit{main} is case sensitive
(even in case-insensitive file systems)
due to internal string comparison.
\item
The argument \textit{main} should be fully expanded, it cannot be a macro.
\item
Subdirectories and special characters should be avoided in filenames.
\item
The command |\childdocmain{|\textit{main}|}| must be followed by a whitespace.
It should not be followed immediately by another command
or by a comment mark `|%|'.
This is because the \TeX{} parser reads the token immediately following
the argument of |\childdocmain| and puts it
at the beginning of every child section;
however, a white\-space is ignored.
\end{itemize}

%%%%%%%%%%%%%%%%%%%%%%%%%%%%%%%%%%%%%%%%
\paragraph{Content of Main File.}

It is advisable to place all content in the child files included by |\include|.
Any output contained in the main file will appear in all child documents
unless suppressed manually;
it cannot be suppressed automatically by the |\includeonly| directive
and thus should normally be avoided.
A method to include some content in the main file
by means of conditional processing is described in \secref{sec:conditional}.

%%%%%%%%%%%%%%%%%%%%%%%%%%%%%%%%%%%%%%%%
\paragraph{Page Numbering.}

When only a part of the document is compiled,
the appropriate numbering of pages
(as well as other status parameters)
is determined from the |.aux| files.
The latter contain information from previous passes.
However this information needs to propagate through
all intermediate child documents.
Therefore the page numbering in child documents may well
be inconsistent until the complete document is compiled at least once.

A useful (if unconventional) way to always ensure a consistent
page numbering is to restart the numbering in each child document
and denote the pages by `\textit{child}|.|\textit{page}'
where \textit{child} represents the chapter/section number of the child file.
This can be achieved by the command
|\numberwithin{page}{|\textit{child}|}|
of the \textsf{amsmath} package
where \textit{child} can be |chapter| or |section|
depending on the chosen structuring.
Alternatively, one can modify the macro |\thepage| appropriately
and reset the counter |page| at the start of each child file.

%%%%%%%%%%%%%%%%%%%%%%%%%%%%%%%%%%%%%%%%%%%%%%%%%%%%%%%%%%%%%%%%%%%%%%%%%%%%%%%%
\subsection{Conditional Processing}
\label{sec:conditional}

The package provides a mechanism to compile different versions
of a document. To customise the versions further some conditional processing
can come in handy to distinguish which version is being compiled.
The package provides two macros to describe the compilation context:

%%%%%%%%%%%%%%%%%%%%%%%%%%%%%%%%%%%%%%%%
\DescribeMacro{\ifchilddoc}
The conditional |\ifchilddoc| distinguishes between the compilation of
child documents and the main document:
%
\begin{center}
|\ifchilddoc |\textit{child-code}| |[|\||else |\textit{main-code}]| \||fi|
\end{center}

%%%%%%%%%%%%%%%%%%%%%%%%%%%%%%%%%%%%%%%%
\DescribeMacro{\childdocname}
\DescribeMacro{\childdocjob}
The macro |\childdocname| contains the filename (without extension)
of the main or child file being processed.
Note that |\childdocjob| will always contain the name of the main file.

%%%%%%%%%%%%%%%%%%%%%%%%%%%%%%%%%%%%%%%%
\paragraph{Title Page.}

Conditional processing can be used to include a title or banner page
in the main document when proper precautions are taken.
Importantly, the code in the main file should ensure that the page counter
(as well as other status parameters which are stored in the |.aux| files)
takes the same value after the conditional processing.
Otherwise the page numbers may take divergent values
depending on which part is compiled.

For example, a title page could be declared by:
%
\begin{center}
\begin{tabular}{l}
|\ifchilddoc\||else|\\
|\addtocounter{page}{-1}|\\
\textit{code for title page}\\
|\newpage|\\
|\||fi|
\end{tabular}
\end{center}
%
A banner page for the child documents can be generated by:
%
\begin{center}
\begin{tabular}{l}
|\ifchilddoc|\\
|\addtocounter{page}{-1}|\\
\textit{code for banner page}\\
|\newpage|\\
|\||fi|
\end{tabular}
\end{center}
%
Here one could write a message such as:
\begin{center}
|This is the part \childdocname{} of \childdocjob{}.|
\end{center}

%%%%%%%%%%%%%%%%%%%%%%%%%%%%%%%%%%%%%%%%%%%%%%%%%%%%%%%%%%%%%%%%%%%%%%%%%%%%%%%%
\subsection{Flags}
\label{sec:flags}

The package makes it easy to generate different versions
of the main or child documents.
To this end compilation flags can be defined
and assigned different default values.
They will be particularly useful in conjunction
with the forwarding mechanism described in \secref{sec:forward}.

For example, it may be useful to have a flag |\version|
which can be set to |draft| or |final|.
The document source will contain some conditional code
depending on the value of |\version|.
Suppose further, the flag should default to |final| for the main file
and to |draft| for child files
which is a natural assignment for editing the document.
This is achieved by placing the following code
in the preamble of the main document
(below the |\childdocmain| directive):
%
\begin{center}
\begin{tabular}{l}
|\ifchilddoc|\\
|\providecommand{\version}{draft}|\\
|\||else|\\
|\providecommand{\version}{final}|\\
|\||fi|
\end{tabular}
\end{center}
%
The definition by |\providecommand| makes sure
that previous definitions are not overwritten.
Further statements |\providecommand{\version}{...}|
can thus be added before the above code to override it.

For the main file, one might add a line
(between |\childdocmain| and the above block)
%
\begin{center}
|%\ifchilddoc\||else\providecommand{\version}{draft}\||fi|
\end{center}
%
which can be uncommented to produce a draft version.
Likewise one can add a line to the very top of a child file
(above the |\childdocof{|\textit{main}|}| directive)
%
\begin{center}
|%\providecommand{\version}{final}|
\end{center}
%
which can be uncommented to produce the final version of this child document.

%%%%%%%%%%%%%%%%%%%%%%%%%%%%%%%%%%%%%%%%%%%%%%%%%%%%%%%%%%%%%%%%%%%%%%%%%%%%%%%%
\subsection{Forwarding}
\label{sec:forward}

Different versions of the main or child documents
using compilation flags as described in \secref{sec:flags}
can be (permanently) stored in different files
for convenient compilation, viewing and distribution.
To this end, the package defines a command
to pass on compilation to a different file:

%%%%%%%%%%%%%%%%%%%%%%%%%%%%%%%%%%%%%%%%
\DescribeMacro{\childdocforward}
The command |\childdocforward| redirects processing to
another source file:
%
\begin{center}
\begin{tabular}{l}
|\input{childdoc.def}|\\
|\childdocforward[|\textit{main}|]{|\textit{dest}|}|\\
\end{tabular}
\end{center}
%
The argument \textit{dest} is the destination file
(without extension).
It should be the main file or one of the child files.
Note that further \textsf{childdoc} directives
such as |\childdocof| and |\childdocforward|
in the indicated file will be processed in this form.
The optional argument \textit{main}
passes on directly to the main file \textit{main}
while pretending to compile the child \textit{dest}.
This form behaves as if \textit{dest}
issues |\childdocof{|\textit{main}|}| right away,
and no further \textsf{childdoc} directives will be processed.

%%%%%%%%%%%%%%%%%%%%%%%%%%%%%%%%%%%%%%%%
\DescribeMacro{\...prefix}
In the alternative form |\childdocforwardprefix|,
%
\begin{center}
\begin{tabular}{l}
|\input{childdoc.def}|\\
|\childdocforwardprefix[|\textit{main}|]{|\textit{prefix}|}{|\textit{dest}|}|
\end{tabular}
\end{center}
%
the destination file is determined by a pattern
depending on the current file:
To make this work, the current file must be called
`{\textit{prefix}\hspace{0.2em}\textit{suffix}}'
with \textit{prefix} matching precisely the argument.
Processing is then passed on to the file
`{\textit{dest}\hspace{0.2em}\textit{suffix}}'.
Surely, the same effect is achieved by
directly specifying the
argument `{\textit{dest}\hspace{0.2em}\textit{suffix}}'
in the first form.
However, that requires to set up a different file
for each child. With the alternative form of the command
all these files can have exactly the same content
which simplifies setting them up and maintaining them.

For example, the following file |draft.tex|
with a compilation flag |\version| as described in \secref{sec:flags}
compiles the main document as a draft:
%
\begin{center}
\begin{tabular}{l}
|\def\version{draft}|\\
|\input{childdoc.def}|\\
|\childdocforward{|\textit{main}|}|
\end{tabular}
\end{center}
%
Likewise, the following files |final|\textit{nn}|.tex|
compile the final version of the child document
|child|\textit{nn}|.tex|:
%
\begin{center}
\begin{tabular}{l}
|\def\version{final}|\\
|\input{childdoc.def}|\\
|\childdocforwardprefix{final}{child}|
\end{tabular}
\end{center}
%

Note that when several versions of a main file and/or of each child file
are to be generated, it may be convenient to set up a |Makefile| or
shell script to automatise the process.

%%%%%%%%%%%%%%%%%%%%%%%%%%%%%%%%%%%%%%%%%%%%%%%%%%%%%%%%%%%%%%%%%%%%%%%%%%%%%%%%
\subsection{Command Line Processing}
\label{sec:commandline}

The effect of redirection files can also be achieved by invoking
the \LaTeX{} compiler with a more elaborate command line.
Most conveniently this should be done as part
of a shell script or a |Makefile|.

When using \textsf{childdoc} in the main file, the following
command lines effectively perform a redirection
(note that depending on the shell being used,
backslashes may have to be doubled: `|\|' $\to$ `|\\|'):
%
\begin{center}
|... -jobname "|\textit{target}|" |\\|"|[\textit{flags}]%
|\input{childdoc.def}\childdocforward[|\textit{main}|]{|\textit{dest}|}"|
\end{center}
%
Here \textit{target} is the name of the output file,
\textit{main} is the name of the main file
and \textit{dest} is the name of the main or child file to be processed
(all filenames without extensions).
The optional argument \textit{main} can be omitted
if \textit{main} matches \textit{dest}.
Optionally, compilation \textit{flags} can be defined via |\def| commands.
This command line makes the \TeX{} engine believe
it is compiling the file \textit{target}
whose content is specified as the latter parameter.
The provided code then forwards the processing to
\textit{main} or \textit{dest} as described in \secref{sec:forward}.

%%%%%%%%%%%%%%%%%%%%%%%%%%%%%%%%%%%%%%%%%%%%%%%%%%%%%%%%%%%%%%%%%%%%%%%%%%%%%%%%
\subsection{Include by Input}
\label{sec:input}

Including child documents by |\include| has some restrictions by design.
Most notably, the content of a child document always occupies
its own set of pages; pages cannot be shared between child documents.
Usually, this behaviour makes perfect sense
because each child document contain an essential part of the document.
However, in some situations it may be desirable to compose
a document from a collection of parts
without having mandatory page breaks between then.
For this case, the package
provides a mechanism to include parts
by |\input| which can also be processed individually.
However, by construction this mechanism
requires manual handling of the content to be output.

%%%%%%%%%%%%%%%%%%%%%%%%%%%%%%%%%%%%%%%%
\DescribeMacro{\ifchilddocmanual}
The main file should be prepared as usual, see \secref{sec:include}.
However, the document body must make a distinction
between processing of an individual part and of the main document, e.g.:
%
\begin{center}
\begin{tabular}{l}
|\ifchilddocmanual|\\
|\input{\childdocname}|\\
|\||else|\\
\textit{document body with }|\input{|\textit{part}|}|\\
|\||fi|
\end{tabular}
\end{center}
%
The conditional |\ifchilddocmanual| is true whenever
a part to be included by |\input| is being compiled,
and the name of the part is stored in |\childdocname|.

%%%%%%%%%%%%%%%%%%%%%%%%%%%%%%%%%%%%%%%%
\DescribeMacro{\childdocby}
Each part to be included by |\input| should start with:
%
\begin{center}
\begin{tabular}{l}
|\input{childdoc.def}|\\
|\childdocby{|\textit{main}|}|\\
\end{tabular}
\end{center}
%
The directive |\childdocby| is similar to |\childdocof|
described in \secref{sec:include},
but the subsequent selection of content must be done manually.
To that end, both |\ifchilddoc| and |\ifchilddocmanual|
will be true upon processing of a part,
and the name of the part is stored in |\childdocname|.
Note that |\jobname| will be set to the filename of the current part
so that each part receives an individual |.aux| file
that does not interfere with the |.aux| file(s) of the main document.
This behaviour can be altered by the alternative form
|\childdocby[*]{|\textit{main}|}| (with a non-empty optional argument)
which uses the |.aux| file of the main document
by setting |\jobname| to \textit{main}.

%%%%%%%%%%%%%%%%%%%%%%%%%%%%%%%%%%%%%%%%%%%%%%%%%%%%%%%%%%%%%%%%%%%%%%%%%%%%%%%%
\subsection{Driver Development}
\label{sec:driver}

The \textsf{childdoc} mechanism can also be use for the development
of definition files such as \LaTeX{} styles or classes.
This case differs from the above setup with multiple parts
included by |\include| in that no |\includeonly| should be invoked.
This can be achieved by starting the include file
(before |\ProvidesPackage|) with:
%
\begin{center}
\begin{tabular}{l}
|\input{childdoc.def}|\\
|\childdocforward{|\textit{main}|}|\\
\end{tabular}
\end{center}
%
or alternatively with:
%
\begin{center}
\begin{tabular}{l}
|\input{childdoc.def}|\\
|\childdocby{|\textit{main}|}|\\
\end{tabular}
\end{center}
%
Both forms have slightly different effects as described above.
The main file is prepared as usual, see \secref{sec:include}.

%%%%%%%%%%%%%%%%%%%%%%%%%%%%%%%%%%%%%%%%%%%%%%%%%%%%%%%%%%%%%%%%%%%%%%%%%%%%%%%%
\subsection{Legacy Detection}
\label{sec:detection}

The directive |\childdocmain| in the main file can detect
whether the complete document or merely a child is to be compiled
even without using the directive |\childdocof|.
This method is deprecated because it is less robust
and there is no compelling reason to use it;
it is merely provided for backward compatibility
and it may be removed in future versions.

If the detection mechanism is to be used,
it is mandatory to correctly specify
the filename of the main file as the argument of |\childdocmain|:
%
\begin{center}
\begin{tabular}{l}
|\input{childdoc.def}|\\
|\childdocmain{|\textit{main}|}|\\
\end{tabular}
\end{center}
%
If |\jobname| does not match the argument \textit{main} of |\childdocmain|,
it is assumed that |\jobname| points to the child file to be compiled.
When using |\childdocmain| with the main file specified as argument,
it suffices to start a child file
with just |\input{|\textit{main}|}|
without loading of the package and using |\childdocof|.
If instead all processing is done
with the appropriate \textsf{childdoc} directives,
the argument of \textit{main} of |\childdocmain| can be empty.

An alternative version of the command line processing described
in \secref{sec:commandline} using the detection mechanism reads:
%
\begin{center}
|... -jobname "|\textit{target}|" "|[\textit{flags}]%
[|\def\jobname{|\textit{dest}|}|]|\input{|\textit{main}|}"|
\end{center}

%%%%%%%%%%%%%%%%%%%%%%%%%%%%%%%%%%%%%%%%%%%%%%%%%%%%%%%%%%%%%%%%%%%%%%%%%%%%%%%%
\subsection{Manual Code}
\label{sec:manual}

In case one cannot be certain whether the definitions file |childdoc.def|
is installed on the target \TeX{} distribution
and one prefers not to ship it,
it is conceivable to paste a few relevant commands into the sources.

To that end, drop all statements |\input{childdoc.def}|
and perform the replacements as outlined below.
Instead of |\childdocmain{|\textit{main}|}| add the following code
to the top of the main file:
%
\begin{center}
\begin{tabular}{l}
|\||ifdefined\childdocname\endinput\||fi\newif\ifchilddoc|\\
|\edef\childdocname{\scantokens\expandafter{\jobname\noexpand}}|\\
|\def\childdocmain{|\textit{main}|}\||ifx\childdocmain\childdocname\||else|\\
|\childdoctrue\includeonly{\childdocname}\let\jobname\childdocmain\||fi|\\
\end{tabular}
\end{center}
%
Instead of |\childdocof{|\textit{main}|}| just include the main file
at the top of each child file:
%
\begin{center}
|\input{|\textit{main}|}|
\end{center}
%
A simple redirection |\childdocforward{|\textit{dest}|}| is achieved by:
%
\begin{center}
|\def\jobname{|\textit{dest}|}\input{\jobname}|
\end{center}
%
The redirection with prefix
|\childdocforwardprefix[|\textit{prefix}|]{|\textit{dest}|}|
is accomplished by:
%
\begin{center}
\begin{tabular}{l}
|{\edef\jobname{\scantokens\expandafter{\jobname\noexpand}}|\\
|\def\redirectjob |\textit{prefix}|#1~~~{\gdef\jobname{|\textit{dest}|#1}}|\\
|\expandafter\redirectjob\jobname~~~}\input{\jobname}|
\end{tabular}
\end{center}

In an alternative approach,
child documents can be compiled by a specific command line
without additional code or specific definitions:
%
\begin{center}
|... -jobname "|\textit{target}|" "|[\textit{flags}]%
|\includeonly{|\textit{dest}|}\input{|\textit{main}|}"|
\end{center}
%

%%%%%%%%%%%%%%%%%%%%%%%%%%%%%%%%%%%%%%%%%%%%%%%%%%%%%%%%%%%%%%%%%%%%%%%%%%%%%%%%
%%%%%%%%%%%%%%%%%%%%%%%%%%%%%%%%%%%%%%%%%%%%%%%%%%%%%%%%%%%%%%%%%%%%%%%%%%%%%%%%
\section{Information}

%%%%%%%%%%%%%%%%%%%%%%%%%%%%%%%%%%%%%%%%%%%%%%%%%%%%%%%%%%%%%%%%%%%%%%%%%%%%%%%%
\subsection{Copyright}

Copyright \copyright{} 2017--2018 Niklas Beisert

This work may be distributed and/or modified under the
conditions of the \LaTeX{} Project Public License, either version 1.3
of this license or (at your option) any later version.
The latest version of this license is in
  \url{http://www.latex-project.org/lppl.txt}
and version 1.3 or later is part of all distributions of \LaTeX{}
version 2005/12/01 or later.

This work has the LPPL maintenance status `maintained'.

The Current Maintainer of this work is Niklas Beisert.

This work consists of the files |README.txt|, |childdoc.ins| and |childdoc.dtx|
as well as the derived files |childdoc.def|, |cdocsamp.tex|
with |cdocsch1.tex|, |cdocsch2.tex|, |cdocspt3.tex|, |cdocspt4.tex|,
|cdocsdrf.tex|, |cdocsfn1.tex|, |cdocsfn2.tex|
as well as |childdoc.pdf|.

%%%%%%%%%%%%%%%%%%%%%%%%%%%%%%%%%%%%%%%%%%%%%%%%%%%%%%%%%%%%%%%%%%%%%%%%%%%%%%%%
\subsection{Files and Installation}

The package consists of the files:
%
\begin{center}
\begin{tabular}{ll}
    |README.txt|   & readme file \\
    |childdoc.ins| & installation file \\
    |childdoc.dtx| & source file \\
    |childdoc.def| & definition file \\
    |cdocsamp.tex| & sample main file \\
    |cdocsch1.tex| & sample include file \\
    |cdocsch2.tex| & sample include file \\
    |cdocspt3.tex| & sample part file \\
    |cdocspt4.tex| & sample part file \\
    |cdocsdrf.tex| & sample redirection file \\
    |cdocsfn1.tex| & sample redirection file \\
    |cdocsfn2.tex| & sample redirection file \\
    |childdoc.pdf| & manual
\end{tabular}
\end{center}
%
The distribution consists of the files
|README.txt|, |childdoc.ins| and |childdoc.dtx|.
%
\begin{itemize}
\item
Run (pdf)\LaTeX{} on |childdoc.dtx|
to compile the manual |childdoc.pdf| (this file).
\item
Run \LaTeX{} on |childdoc.ins| to create the definitions file |childdoc.def|
and the sample |cdocsamp.tex| with include files
|cdocsch1.tex|, |cdocsch2.tex|, |cdocspt3.tex|, |cdocspt4.tex|,
|cdocsdrf.tex|, |cdocsfn1.tex|, |cdocsfn2.tex|.
Then copy the file |childdoc.def| to an appropriate directory of your \LaTeX{}
distribution, e.g.\ \textit{texmf-root}|/tex/latex/childdoc|.
\end{itemize}

%%%%%%%%%%%%%%%%%%%%%%%%%%%%%%%%%%%%%%%%%%%%%%%%%%%%%%%%%%%%%%%%%%%%%%%%%%%%%%%%
\subsection{Related CTAN Packages}

There are several other packages which offer a similar functionality:
%
\begin{itemize}
\item
The packages
\href{http://ctan.org/pkg/docmute}{\textsf{docmute}},
\href{http://ctan.org/pkg/includex}{\textsf{includex}} and
\href{http://ctan.org/pkg/standalone}{\textsf{standalone}}
provide commands to include only the document body of
a child file thus allowing both files to be compiled individually.
\item
The packages \href{http://ctan.org/pkg/subdocs}{\textsf{subdocs}}
and \href{http://ctan.org/pkg/subfiles}{\textsf{subfiles}}
provide structures in which the main and child documents can be
encapsulated and allowing them to be compiled individually.
The inclusion mechanism is different from the conventional |\include|.
\item
The package \href{http://ctan.org/pkg/combine}{\textsf{combine}}
is an elaborate solution to combine several documents into one.
\end{itemize}
%
See also the CTAN topic \href{http://ctan.org/topic/subdocs}{\textsf{subdocs}}
for further related packages.
The present package differs from the above solutions in that
a document structure constructed with the conventional |\include| mechanism
just needs two extra commands at the top of every file
such that all constituent files can be compiled individually.

%%%%%%%%%%%%%%%%%%%%%%%%%%%%%%%%%%%%%%%%%%%%%%%%%%%%%%%%%%%%%%%%%%%%%%%%%%%%%%%%
%\subsection{Feature Suggestions}
%
%The following is a list of features which may be useful for future
%versions of this package:
%%
%\begin{itemize}
%\item
%\ldots
%\end{itemize}

%%%%%%%%%%%%%%%%%%%%%%%%%%%%%%%%%%%%%%%%%%%%%%%%%%%%%%%%%%%%%%%%%%%%%%%%%%%%%%%%
\subsection{Revision History}

%%%%%%%%%%%%%%%%%%%%%%%%%%%%%%%%%%%%%%%%
\paragraph{v2.0:} 2018/12/30

\begin{itemize}
\item
immediate forward processing
\item
added |\childdocby| mechanism
\item
manual restructured
\end{itemize}

%%%%%%%%%%%%%%%%%%%%%%%%%%%%%%%%%%%%%%%%
\paragraph{v1.6:} 2018/01/17

\begin{itemize}
\item
application for development of include files
\item
corrections to manual
\end{itemize}

%%%%%%%%%%%%%%%%%%%%%%%%%%%%%%%%%%%%%%%%
\paragraph{v1.5:} 2017/05/21

\begin{itemize}
\item
more complete structuring introduced
\item
|\childdocof| introduced
\item
|\childdoc| renamed to |\childdocmain|
\item
|\childredirect| renamed to |\childdocforward| and |\childdocforwardprefix|
and functionality expanded
\end{itemize}

%%%%%%%%%%%%%%%%%%%%%%%%%%%%%%%%%%%%%%%%
\paragraph{v1.0:} 2017/04/27

\begin{itemize}
\item
manual and install package
\item
first version published on CTAN
\end{itemize}

%%%%%%%%%%%%%%%%%%%%%%%%%%%%%%%%%%%%%%%%
\paragraph{v0.6:} 2017/04/26

\begin{itemize}
\item
redirection mechanism added
\end{itemize}

%%%%%%%%%%%%%%%%%%%%%%%%%%%%%%%%%%%%%%%%
\paragraph{v0.5:} 2017/04/26

\begin{itemize}
\item
functionality in definition file
\end{itemize}


%%%%%%%%%%%%%%%%%%%%%%%%%%%%%%%%%%%%%%%%%%%%%%%%%%%%%%%%%%%%%%%%%%%%%%%%%%%%%%%%
%%%%%%%%%%%%%%%%%%%%%%%%%%%%%%%%%%%%%%%%%%%%%%%%%%%%%%%%%%%%%%%%%%%%%%%%%%%%%%%%
%%%%%%%%%%%%%%%%%%%%%%%%%%%%%%%%%%%%%%%%%%%%%%%%%%%%%%%%%%%%%%%%%%%%%%%%%%%%%%%%
\appendix

\settowidth\MacroIndent{\rmfamily\scriptsize 000\ }

 \DocInput{childdoc.dtx}

\end{document}
%</driver>
% \fi
%
% %%%%%%%%%%%%%%%%%%%%%%%%%%%%%%%%%%%%%%%%%%%%%%%%%%%%%%%%%%%%%%%%%%%%%%%%%%%%%%
% %%%%%%%%%%%%%%%%%%%%%%%%%%%%%%%%%%%%%%%%%%%%%%%%%%%%%%%%%%%%%%%%%%%%%%%%%%%%%%
% \section{Sample}
%\iffalse
%<*samplemain>
%\fi
%
% The following presents a sample document
% with two chapters, two parts, a title page,
% a compile flag as well as three forwarding files to set the flag.
% It consists of eight |.tex| files:
% \begin{center}
% \begin{tabular}{ll}
% |cdocsamp.tex|&main file\\
% |cdocsch1.tex|&include file for chapter 1\\
% |cdocsch2.tex|&include file for chapter 2\\
% |cdocspt3.tex|&include file for part 3\\
% |cdocspt4.tex|&include file for part 4\\
% |cdocsdrf.tex|&forwarding file for main file in draft mode\\
% |cdocsfi1.tex|&forwarding file for final version of chapter 1\\
% |cdocsfi2.tex|&forwarding file for final version of chapter 2\\
% \end{tabular}
% \end{center}
% Each of the eight files can be compiled directly by the \LaTeX{} compiler.
%
% %%%%%%%%%%%%%%%%%%%%%%%%%%%%%%%%%%%%%%
% \paragraph{Main File.}
%
% The main file is called |cdocsamp.tex|.
%
% Load the \textsf{childdoc} definitions and
% declare the filename for the main document:
%    \begin{macrocode}
\input{childdoc.def}
\childdocmain{}
%    \end{macrocode}

% Optional override for |\version| flag:
%    \begin{macrocode}
%%\ifchilddoc\else\providecommand{\version}{draft}\fi
%    \end{macrocode}

% Define the default values for the |\version| flag
% (|final| for the main file and |draft| for childs):
%    \begin{macrocode}
\ifchilddoc
\providecommand{\version}{draft}
\else
\providecommand{\version}{final}
\fi
%    \end{macrocode}

% Load the standard document class:
%    \begin{macrocode}
\documentclass[12pt]{article}
%    \end{macrocode}

% Start the document body:
%    \begin{macrocode}
\begin{document}
%    \end{macrocode}

% Declare a title page.
% Print title, part of document being processed and version flag:
%    \begin{macrocode}
\addtocounter{page}{-1}
\begin{center}
{\LARGE\bfseries{}childdoc example\par}
\vspace{1cm}
\ifchilddoc
\ifchilddocmanual part\else chapter\fi:
`\childdocname' of `\childdocjob'\par
\else
main document: `\childdocjob'\par
\fi
version: \version\par
\end{center}
\newpage
%    \end{macrocode}

% Manually include selected file,
% otherwise process as usual:
%    \begin{macrocode}
\ifchilddocmanual
\section*{part `\childdocname'}
\input{\childdocname}
\else
%    \end{macrocode}

% Include the two chapters:
%    \begin{macrocode}
\include{cdocsch1}
\include{cdocsch2}
%    \end{macrocode}

% Include the two parts unless only chapters should be displayed:
%    \begin{macrocode}
\ifchilddoc\else
\section{part three}
\input{cdocspt3}
\section{part four}
\input{cdocspt4}
\fi
%    \end{macrocode}

% Process as usual until here:
%    \begin{macrocode}
\fi
%    \end{macrocode}

% End of document body:
%    \begin{macrocode}
\end{document}
%    \end{macrocode}
%\iffalse
%</samplemain>
%\fi
%
% %%%%%%%%%%%%%%%%%%%%%%%%%%%%%%%%%%%%%%
% \paragraph{Chapter Include Files.}
%
% The include files are called |cdocsch1.tex| and |cdocsch2.tex|.
%
%\iffalse
%<*samplechap1|samplechap2>
%\fi

% Optional override for |\version| flag:
%    \begin{macrocode}
%%\providecommand{\version}{final}
%    \end{macrocode}

% Include the main document:
%    \begin{macrocode}
\input{childdoc.def}
\childdocof{cdocsamp}
%    \end{macrocode}

%\iffalse
%</samplechap1|samplechap2>
%\fi
%
%\iffalse
%<*samplechap1>
%\fi
% Some text for chapter 1:
%    \begin{macrocode}
\section{one}
some text in chapter one
%    \end{macrocode}

%\iffalse
%</samplechap1>
%\fi
% Some text for chapter 2:
%\iffalse
%<*samplechap2>
%\fi
%    \begin{macrocode}
\section{two}
more text in chapter two
%    \end{macrocode}

%\iffalse
%</samplechap2>
%\fi
%
% %%%%%%%%%%%%%%%%%%%%%%%%%%%%%%%%%%%%%%
% \paragraph{Part Include Files.}
%
% The include files are called |cdocspt3.tex| and |cdocspt4.tex|.
%
%\iffalse
%<*samplepart3|samplepart4>
%\fi

% Optional override for |\version| flag:
%    \begin{macrocode}
%%\providecommand{\version}{final}
%    \end{macrocode}

% Include the main document:
%    \begin{macrocode}
\input{childdoc.def}
\childdocby{cdocsamp}
%    \end{macrocode}

%\iffalse
%</samplepart3|samplepart4>
%\fi
%
%\iffalse
%<*samplepart3>
%\fi
% Some text for part 3:
%    \begin{macrocode}
some text in part three
%    \end{macrocode}

%\iffalse
%</samplepart3>
%\fi
% Some text for part 4:
%\iffalse
%<*samplepart4>
%\fi
%    \begin{macrocode}
more text in part four
%    \end{macrocode}

%\iffalse
%</samplepart4>
%\fi
%
% %%%%%%%%%%%%%%%%%%%%%%%%%%%%%%%%%%%%%%
% \paragraph{Forwarding for a Complete Draft.}
%
% The following forwarding file |cdocsdrf.tex|
% compiles the main document in draft mode:
%\iffalse
%<*sampledraft>
%\fi
%    \begin{macrocode}
\def\version{draft}
\input{childdoc.def}
\childdocforward{cdocsamp}
%    \end{macrocode}

%\iffalse
%</sampledraft>
%\fi
%
% %%%%%%%%%%%%%%%%%%%%%%%%%%%%%%%%%%%%%%
% \paragraph{Forwarding for Final Version of the Chapters.}
%
% The following forwarding files |cdocsfn1.tex| and |cdocsfn2.tex|
% (with identical content)
% compile the final versions of the child documents
% |cdocsch1.tex| and |cdocsch2.tex|, respectively:
%\iffalse
%<*samplefinal>
%\fi
%    \begin{macrocode}
\def\version{final}
\input{childdoc.def}
\childdocforwardprefix[cdocsamp]{cdocsfn}{cdocsch}
%    \end{macrocode}

%\iffalse
%</samplefinal>
%\fi
%
% %%%%%%%%%%%%%%%%%%%%%%%%%%%%%%%%%%%%%%
% \paragraph{Command Line Processing.}
%
% The following three command lines generate the output files
% |cdocscld|, |cdocscl1| and |cdocscl2|
% which should be identical to
% |cdocsdrf|, |cdocsch1| and |cdocsfn2|, respectively:
% \begin{center}
% \begin{tabular}{l}
% |latex -jobname cdocscld \|\\
% |  "\def\version{draft}\input{childdoc.def}\childdocforward{cdocsamp}"|\\
% |latex -jobname cdocscl1 \|\\
% |  "\input{childdoc.def}\childdocforward[cdocsamp]{cdocsch1}"|\\
% |latex -jobname cdocscl2 \|\\
% |  "\def\version{final}\input{childdoc.def}\childdocforward{cdocsch2}"|
% \end{tabular}
% \end{center}
% Note that the trailing backslash on each first line
% merely continues the input to the second line
% (for convenient cut ant paste).
% Furthermore, the command |latex| can be replaced by any
% of its alternative versions such as |pdflatex|.
%
% %%%%%%%%%%%%%%%%%%%%%%%%%%%%%%%%%%%%%%%%%%%%%%%%%%%%%%%%%%%%%%%%%%%%%%%%%%%%%%
% %%%%%%%%%%%%%%%%%%%%%%%%%%%%%%%%%%%%%%%%%%%%%%%%%%%%%%%%%%%%%%%%%%%%%%%%%%%%%%
% \section{Implementation}
%\iffalse
%<*package>
%\fi
%
% This section describes the definitions file |childdoc.def|.

% The definitions cannot be loaded using |\usepackage| or |\RequirePackage|
% which has a mechanism to prevent loading a style file more than once.
% When loading the definitions by means of |\input|
% multiple instances have to be prevented manually:
%\iffalse
%This code needs to be before the `\ProvidesFile' directive
%which is defined at the beginning of this file.
%Therefore it is also placed there and commented out here.
%</package>
%<*discard>
%\fi
%    \begin{macrocode}
\ifdefined\childdocmain\endinput\fi
%    \end{macrocode}
%\iffalse
%</discard>
%<*package>
%\fi
%
% \macro{\ifchilddoc}
% \macro{\ifchilddocmanual}
% The conditional |\ifchilddoc| tells whether a
% child (true) or main (false) document is being compiled.
% The conditional |\ifchilddocmanual| tells whether
% the |\includeonly| mechanism is used (false) or
% the selection of child files must be performed manually (true).
% The definitions initialise to false:
%    \begin{macrocode}
\newif\ifchilddoc
\newif\ifchilddocmanual
%    \end{macrocode}

% \macro{\childdocname}
% \macro{\childdocjob}
% The macro |\childdocname| stores the name of the main document
% to be compiled. The macro |\childdocjob| stores the name of
% the document on which the \LaTeX{} compiler was originally invoked.
% The content of |\jobname| cannot be compared
% to filenames specified in the source due to different catcodes.
% The following code rescans |\jobname|, stores the result
% in |\childdocname| and saves a copy in |\childdocjob|:
%    \begin{macrocode}
\edef\childdocname{\scantokens\expandafter{\jobname\noexpand}}
\let\childdocjob\childdocname
%    \end{macrocode}

% \macro{\childdocdisable}
% The macro |\childdocdisable| prevents the main file
% from being processed more than once.
% At this stage, the main document command |\childdocmain|
% is assumed to be called once again where it should do nothing.
% Any subsequent call to it should prevent
% a secondary processing of the main document
% It overwrites the forwarding commands
% |\childdocof| and |\childdocforward|
% with empty macros to prevent further inclusions of the main document:
%    \begin{macrocode}
\newcommand{\childdocdisable}
{
  \renewcommand{\childdocmain}[1]{\renewcommand{\childdocmain}[1]{\endinput}}
  \renewcommand{\childdocof}[1]{}
  \renewcommand{\childdocby}[2][]{}
  \renewcommand{\childdocforward}[2][]{}
  \renewcommand{\childdocdisable}{}
}
%    \end{macrocode}

% \macro{\childdocmain}
% The macro |\childdocmain| is to be called at the top of the main file
% with nothing or the main filename (without extension) as argument.
% First, it breaks loops.
% If the argument is not empty and does not match |\childdocname|
% (which is set by the first inclusion of |childdoc.def|),
% |\ifchilddoc| is set to true, |\includeonly| is applied to the child file
% and |\jobname| is set to the main file
% (for proper handling of |.aux| files):
%    \begin{macrocode}
\newcommand{\childdocmain}[1]
{
  \childdocdisable\childdocmain{}
  \if?#1?\else
    \begingroup
      \def\childdoctmp{#1}
      \ifx\childdoctmp\childdocname
        \def\childdoctmp{}
      \else
        \def\childdoctmp
        {
          \childdoctrue
          \includeonly{\childdocname}
          \def\childdocjob{#1}
          \def\jobname{#1}
        }
      \fi
      \expandafter
    \endgroup
    \childdoctmp
  \fi
}
%    \end{macrocode}

% \macro{\childdocof}
% The command |\childdocof| redirects
% compilation to the main file |#1|.
%    \begin{macrocode}
\newcommand{\childdocof}[1]
{
  \childdocdisable
  \childdoctrue
  \includeonly{\childdocname}
  \def\jobname{#1}
  \def\childdocjob{#1}
  \input{#1}
}
%    \end{macrocode}

% \macro{\childdocby}
% The command |\childdocby| ....
%    \begin{macrocode}
\newcommand{\childdocby}[2][]
{
  \childdocdisable
  \childdoctrue
  \childdocmanualtrue
  \if?#1?\else
    \def\jobname{#2}
  \fi
  \def\childdocjob{#2}
  \input{#2}
  \endinput
}
%    \end{macrocode}

% \macro{\childdocforward}
% The command |\childdocforward| redirects
% compilation to the main file or
% (if the optional argument is given) a child file.
% Parameters are set as if the main file
% or a child file starting with |\childdocof| was compiled.
% Then compilation is handed over to the main file:
%    \begin{macrocode}
\newcommand{\childdocforward}[2][]
{
  \begingroup
    \if?#1?
      \def\childdoctmp
      {
        \def\childdocname{#2}
        \def\childdocjob{#2}
        \def\jobname{#2}
        \input{#2}
        \endinput
      }
    \else
      \def\childdoctmp
      {
        \childdocdisable
        \def\childdocname{#2}
        \childdoctrue
        \includeonly{#2}
        \def\childdocjob{#1}
        \def\jobname{#1}
        \input{#1}
        \endinput
      }
    \fi
    \expandafter
  \endgroup
  \childdoctmp
}
%    \end{macrocode}

% \macro{\childdocforwardprefix}
% The command |\childdocforwardprefix| redirects
% compilation to the main or a child file by means of a pattern.
% The prefix |#1| in the current filename is replaced by |#2|
% and the suffix of the current filename is kept
% (it is assumed that the filename does not contain the substring `|~~~|'
% which is used as a delimiter).
% Compilation is handed over to the new file by |\childdocforward|:
%    \begin{macrocode}
\newcommand{\childdocforwardprefix}[3][]
{
  \begingroup
    \def\childdocextract #2##1~~~{\def\childdoctmp{\childdocforward[#1]{#3##1}}}
    \expandafter\childdocextract\childdocname~~~
    \expandafter
  \endgroup
  \childdoctmp
}
%    \end{macrocode}

% \macro{\childdoc}
% The deprecated macro |\childdoc| is a legacy version of |\childdocmain|:
%    \begin{macrocode}
\newcommand{\childdoc}{\childdocmain}
%    \end{macrocode}

% \macro{\childdocredirect}
% The deprecated macro |\childdocredirect| is a legacy version
% of |\childdocforward| and |\childdocforwardprefix|:
%    \begin{macrocode}
\newcommand{\childdocredirect}[2][]
{
  \begingroup
    \if?#1?
      \def\childdoctmp{\childdocforward{#2}}
    \else
      \def\childdoctmp{\childdocforwardprefix{#1}{#2}}
    \fi
    \expandafter
  \endgroup
  \childdoctmp
}
%    \end{macrocode}

%\iffalse
%</package>
%\fi
%
\endinput
|\\
|\childdocby{|\textit{main}|}|\\
\end{tabular}
\end{center}
%
Both forms have slightly different effects as described above.
The main file is prepared as usual, see \secref{sec:include}.

%%%%%%%%%%%%%%%%%%%%%%%%%%%%%%%%%%%%%%%%%%%%%%%%%%%%%%%%%%%%%%%%%%%%%%%%%%%%%%%%
\subsection{Legacy Detection}
\label{sec:detection}

The directive |\childdocmain| in the main file can detect
whether the complete document or merely a child is to be compiled
even without using the directive |\childdocof|.
This method is deprecated because it is less robust
and there is no compelling reason to use it;
it is merely provided for backward compatibility
and it may be removed in future versions.

If the detection mechanism is to be used,
it is mandatory to correctly specify
the filename of the main file as the argument of |\childdocmain|:
%
\begin{center}
\begin{tabular}{l}
|% \iffalse
%
% childdoc.dtx Copyright (C) 2017-2018 Niklas Beisert
%
% This work may be distributed and/or modified under the
% conditions of the LaTeX Project Public License, either version 1.3
% of this license or (at your option) any later version.
% The latest version of this license is in
%   http://www.latex-project.org/lppl.txt
% and version 1.3 or later is part of all distributions of LaTeX
% version 2005/12/01 or later.
%
% This work has the LPPL maintenance status `maintained'.
%
% The Current Maintainer of this work is Niklas Beisert.
%
% This work consists of the files childdoc.dtx and childdoc.ins
% and the derived files childdoc.def and cdocsamp.tex with
% cdocsch1.tex, cdocsch2.tex, cdocsdrf.tex, cdocsfn1.tex, cdocsfn2.tex.
%
%<package>\ifdefined\childdocmain\endinput\fi
%<package>\ProvidesFile{childdoc.def}[2018/12/30 v2.0 child document driver]
%<samplemain>\ProvidesFile{cdocsamp.tex}[2018/12/30 v2.0 sample for childdoc]
%<*driver>
%\ProvidesFile{childdoc.drv}[2018/12/30 v2.0 childdoc reference manual file]
\PassOptionsToClass{10pt,a4paper}{article}
\documentclass{ltxdoc}

\usepackage[margin=35mm]{geometry}
\usepackage{hyperref}
\usepackage{hyperxmp}
\usepackage[usenames]{color}

\hypersetup{colorlinks=true}
\hypersetup{pdfstartview=FitH}
\hypersetup{pdfpagemode=UseNone}
\hypersetup{pdfsource={}}
\hypersetup{pdflang={en-UK}}
\hypersetup{pdfcopyright={Copyright 2017-2018 Niklas Beisert.
  This work may be distributed and/or modified under the
  conditions of the LaTeX Project Public License, either version 1.3
  of this license or (at your option) any later version.}}
\hypersetup{pdflicenseurl={http://www.latex-project.org/lppl.txt}}
\hypersetup{pdfcontactaddress={ETH Zurich, ITP, HIT K,
  Wolfgang-Pauli-Strasse 27}}
\hypersetup{pdfcontactpostcode={8093}}
\hypersetup{pdfcontactcity={Zurich}}
\hypersetup{pdfcontactcountry={Switzerland}}
\hypersetup{pdfcontactemail={nbeisert@itp.phys.ethz.ch}}
\hypersetup{pdfcontacturl={http://people.phys.ethz.ch/\xmptilde nbeisert/}}

\newcommand{\secref}[1]{\hyperref[#1]{section \ref*{#1}}}

\parskip1ex
\parindent0pt
\let\olditemize\itemize
\def\itemize{\olditemize\parskip0pt}

\begin{document}

\title{The \textsf{childdoc} Package}
\hypersetup{pdftitle={The childdoc Package}}
\author{Niklas Beisert\\[2ex]
  Institut f\"ur Theoretische Physik\\
  Eidgen\"ossische Technische Hochschule Z\"urich\\
  Wolfgang-Pauli-Strasse 27, 8093 Z\"urich, Switzerland\\[1ex]
  \href{mailto:nbeisert@itp.phys.ethz.ch}
  {\texttt{nbeisert@itp.phys.ethz.ch}}}
\hypersetup{pdfauthor={Niklas Beisert}}
\hypersetup{pdfsubject={Manual for the LaTeX2e Package childdoc}}
\date{30 December 2018, \textsf{v2.0}}
\maketitle

\begin{abstract}\noindent
\textsf{childdoc} is a \LaTeXe{} package
that enables the direct compilation
of document sections included by |\include|
to individual files.
\end{abstract}

\begingroup
\parskip0ex
\tableofcontents
\endgroup

%%%%%%%%%%%%%%%%%%%%%%%%%%%%%%%%%%%%%%%%%%%%%%%%%%%%%%%%%%%%%%%%%%%%%%%%%%%%%%%%
%%%%%%%%%%%%%%%%%%%%%%%%%%%%%%%%%%%%%%%%%%%%%%%%%%%%%%%%%%%%%%%%%%%%%%%%%%%%%%%%
\section{Introduction}

\LaTeX{} provides a mechanism to structure a large document (such as a book)
into a main file and several child files (containing the chapters)
using the |\include| command.
This mechanism is beneficial for documents
which span hundreds of pages in order to
make the source file(s) more manageable.
Moreover, compilation can be restricted to
selected child files by means of the |\includeonly| command.
The latter feature can be used to reduce the compilation time while editing
(this was significantly more useful in the earlier days of \LaTeX{})
or to generate a smaller document which is easier to navigate.
Another application of |\includeonly| is to generate
documents consisting of selected parts of the complete document.

However, there are a few drawbacks of the plain |\include| mechanism:
\begin{itemize}
\item
The child files cannot be compiled on their own,
they can only be compiled via the main file.
A naive editing environment
(such as a text editor with an option
to have the current file processed by \LaTeX)
may require one to switch to the main file before compiling;
attempting to compile the child file produces errors.
\item
The main file must be modified (each time)
to adjust the |\includeonly| command
to the present needs. This easily leaves the main file in a messy state.
\item
The generated document will always carry the filename
of the main document. This is inconvenient if
several child files are to be compiled and
to be kept for distribution.
\end{itemize}

The present package provides a simple interface
to make child files individually compilable by \LaTeX{}.
Compiling a child file then has the same effect as compiling
the main file with an |\includeonly| command
to select the appropriate child.
Moreover the generated document will carry the name of the child
rather than the main file.
This resolves all three above issues.

This feature is meant to make the editing of books,
thesis documents and lecture notes somewhat more convenient.
However, the package can also be used efficiently for
composing a series of documents (such as exercise sheets)
which are typically distributed individually.
It then assists the author in generating the individual documents
(potentially in different versions)
as well as a document containing the collected series.
Another application is in developing style files
or other kinds of included material
where compilation of the style file could redirect
to a sample or test file.

%%%%%%%%%%%%%%%%%%%%%%%%%%%%%%%%%%%%%%%%%%%%%%%%%%%%%%%%%%%%%%%%%%%%%%%%%%%%%%%%
%%%%%%%%%%%%%%%%%%%%%%%%%%%%%%%%%%%%%%%%%%%%%%%%%%%%%%%%%%%%%%%%%%%%%%%%%%%%%%%%
\section{Usage}

First of all, the package \textsf{childdoc} is \emph{not} a standard
\LaTeXe{} |.sty| style file! Therefore it needs to be invoked in
a non-standard way.

%%%%%%%%%%%%%%%%%%%%%%%%%%%%%%%%%%%%%%%%%%%%%%%%%%%%%%%%%%%%%%%%%%%%%%%%%%%%%%%%
\subsection{Included Files}
\label{sec:include}

%%%%%%%%%%%%%%%%%%%%%%%%%%%%%%%%%%%%%%%%
\DescribeMacro{\childdocmain}
To use the package, add the commands
\begin{center}
\begin{tabular}{l}
|\input{childdoc.def}|\\
|\childdocmain{}|\\
\end{tabular}
\end{center}
at the very top of the main \LaTeX{} file,
in particular \emph{before} the |\documentclass| statement!
The argument of |\childdocmain| should be left empty
(but it must be present).

%%%%%%%%%%%%%%%%%%%%%%%%%%%%%%%%%%%%%%%%
\DescribeMacro{\childdocof}
Furthermore, add the commands
\begin{center}
\begin{tabular}{l}
|\input{childdoc.def}|\\
|\childdocof{|\textit{main}|}|\\
\end{tabular}
\end{center}
at the top of every child file \textit{child}
which is included by |\include{|\textit{child}|}|
from within the main file
(or at least for those files to be compiled individually).
The argument \textit{main} must be the filename of the main file.

There are a couple of
considerations in setting up the main and child documents:

%%%%%%%%%%%%%%%%%%%%%%%%%%%%%%%%%%%%%%%%
\paragraph{Restrictions.}

Please note the following restrictions:
\begin{itemize}
\item
|\childdocmain| must be called with one argument \textit{main}
to ensure compatibility with earlier version of the package.
It must either be empty (|\childdocmain{}|)
or precisely match the filename of the main file in which it is specified.
See \secref{sec:detection} for further information.
\item
The filename \textit{main} must be specified without the |.tex| extension.
\item
The filename \textit{main} is case sensitive
(even in case-insensitive file systems)
due to internal string comparison.
\item
The argument \textit{main} should be fully expanded, it cannot be a macro.
\item
Subdirectories and special characters should be avoided in filenames.
\item
The command |\childdocmain{|\textit{main}|}| must be followed by a whitespace.
It should not be followed immediately by another command
or by a comment mark `|%|'.
This is because the \TeX{} parser reads the token immediately following
the argument of |\childdocmain| and puts it
at the beginning of every child section;
however, a white\-space is ignored.
\end{itemize}

%%%%%%%%%%%%%%%%%%%%%%%%%%%%%%%%%%%%%%%%
\paragraph{Content of Main File.}

It is advisable to place all content in the child files included by |\include|.
Any output contained in the main file will appear in all child documents
unless suppressed manually;
it cannot be suppressed automatically by the |\includeonly| directive
and thus should normally be avoided.
A method to include some content in the main file
by means of conditional processing is described in \secref{sec:conditional}.

%%%%%%%%%%%%%%%%%%%%%%%%%%%%%%%%%%%%%%%%
\paragraph{Page Numbering.}

When only a part of the document is compiled,
the appropriate numbering of pages
(as well as other status parameters)
is determined from the |.aux| files.
The latter contain information from previous passes.
However this information needs to propagate through
all intermediate child documents.
Therefore the page numbering in child documents may well
be inconsistent until the complete document is compiled at least once.

A useful (if unconventional) way to always ensure a consistent
page numbering is to restart the numbering in each child document
and denote the pages by `\textit{child}|.|\textit{page}'
where \textit{child} represents the chapter/section number of the child file.
This can be achieved by the command
|\numberwithin{page}{|\textit{child}|}|
of the \textsf{amsmath} package
where \textit{child} can be |chapter| or |section|
depending on the chosen structuring.
Alternatively, one can modify the macro |\thepage| appropriately
and reset the counter |page| at the start of each child file.

%%%%%%%%%%%%%%%%%%%%%%%%%%%%%%%%%%%%%%%%%%%%%%%%%%%%%%%%%%%%%%%%%%%%%%%%%%%%%%%%
\subsection{Conditional Processing}
\label{sec:conditional}

The package provides a mechanism to compile different versions
of a document. To customise the versions further some conditional processing
can come in handy to distinguish which version is being compiled.
The package provides two macros to describe the compilation context:

%%%%%%%%%%%%%%%%%%%%%%%%%%%%%%%%%%%%%%%%
\DescribeMacro{\ifchilddoc}
The conditional |\ifchilddoc| distinguishes between the compilation of
child documents and the main document:
%
\begin{center}
|\ifchilddoc |\textit{child-code}| |[|\||else |\textit{main-code}]| \||fi|
\end{center}

%%%%%%%%%%%%%%%%%%%%%%%%%%%%%%%%%%%%%%%%
\DescribeMacro{\childdocname}
\DescribeMacro{\childdocjob}
The macro |\childdocname| contains the filename (without extension)
of the main or child file being processed.
Note that |\childdocjob| will always contain the name of the main file.

%%%%%%%%%%%%%%%%%%%%%%%%%%%%%%%%%%%%%%%%
\paragraph{Title Page.}

Conditional processing can be used to include a title or banner page
in the main document when proper precautions are taken.
Importantly, the code in the main file should ensure that the page counter
(as well as other status parameters which are stored in the |.aux| files)
takes the same value after the conditional processing.
Otherwise the page numbers may take divergent values
depending on which part is compiled.

For example, a title page could be declared by:
%
\begin{center}
\begin{tabular}{l}
|\ifchilddoc\||else|\\
|\addtocounter{page}{-1}|\\
\textit{code for title page}\\
|\newpage|\\
|\||fi|
\end{tabular}
\end{center}
%
A banner page for the child documents can be generated by:
%
\begin{center}
\begin{tabular}{l}
|\ifchilddoc|\\
|\addtocounter{page}{-1}|\\
\textit{code for banner page}\\
|\newpage|\\
|\||fi|
\end{tabular}
\end{center}
%
Here one could write a message such as:
\begin{center}
|This is the part \childdocname{} of \childdocjob{}.|
\end{center}

%%%%%%%%%%%%%%%%%%%%%%%%%%%%%%%%%%%%%%%%%%%%%%%%%%%%%%%%%%%%%%%%%%%%%%%%%%%%%%%%
\subsection{Flags}
\label{sec:flags}

The package makes it easy to generate different versions
of the main or child documents.
To this end compilation flags can be defined
and assigned different default values.
They will be particularly useful in conjunction
with the forwarding mechanism described in \secref{sec:forward}.

For example, it may be useful to have a flag |\version|
which can be set to |draft| or |final|.
The document source will contain some conditional code
depending on the value of |\version|.
Suppose further, the flag should default to |final| for the main file
and to |draft| for child files
which is a natural assignment for editing the document.
This is achieved by placing the following code
in the preamble of the main document
(below the |\childdocmain| directive):
%
\begin{center}
\begin{tabular}{l}
|\ifchilddoc|\\
|\providecommand{\version}{draft}|\\
|\||else|\\
|\providecommand{\version}{final}|\\
|\||fi|
\end{tabular}
\end{center}
%
The definition by |\providecommand| makes sure
that previous definitions are not overwritten.
Further statements |\providecommand{\version}{...}|
can thus be added before the above code to override it.

For the main file, one might add a line
(between |\childdocmain| and the above block)
%
\begin{center}
|%\ifchilddoc\||else\providecommand{\version}{draft}\||fi|
\end{center}
%
which can be uncommented to produce a draft version.
Likewise one can add a line to the very top of a child file
(above the |\childdocof{|\textit{main}|}| directive)
%
\begin{center}
|%\providecommand{\version}{final}|
\end{center}
%
which can be uncommented to produce the final version of this child document.

%%%%%%%%%%%%%%%%%%%%%%%%%%%%%%%%%%%%%%%%%%%%%%%%%%%%%%%%%%%%%%%%%%%%%%%%%%%%%%%%
\subsection{Forwarding}
\label{sec:forward}

Different versions of the main or child documents
using compilation flags as described in \secref{sec:flags}
can be (permanently) stored in different files
for convenient compilation, viewing and distribution.
To this end, the package defines a command
to pass on compilation to a different file:

%%%%%%%%%%%%%%%%%%%%%%%%%%%%%%%%%%%%%%%%
\DescribeMacro{\childdocforward}
The command |\childdocforward| redirects processing to
another source file:
%
\begin{center}
\begin{tabular}{l}
|\input{childdoc.def}|\\
|\childdocforward[|\textit{main}|]{|\textit{dest}|}|\\
\end{tabular}
\end{center}
%
The argument \textit{dest} is the destination file
(without extension).
It should be the main file or one of the child files.
Note that further \textsf{childdoc} directives
such as |\childdocof| and |\childdocforward|
in the indicated file will be processed in this form.
The optional argument \textit{main}
passes on directly to the main file \textit{main}
while pretending to compile the child \textit{dest}.
This form behaves as if \textit{dest}
issues |\childdocof{|\textit{main}|}| right away,
and no further \textsf{childdoc} directives will be processed.

%%%%%%%%%%%%%%%%%%%%%%%%%%%%%%%%%%%%%%%%
\DescribeMacro{\...prefix}
In the alternative form |\childdocforwardprefix|,
%
\begin{center}
\begin{tabular}{l}
|\input{childdoc.def}|\\
|\childdocforwardprefix[|\textit{main}|]{|\textit{prefix}|}{|\textit{dest}|}|
\end{tabular}
\end{center}
%
the destination file is determined by a pattern
depending on the current file:
To make this work, the current file must be called
`{\textit{prefix}\hspace{0.2em}\textit{suffix}}'
with \textit{prefix} matching precisely the argument.
Processing is then passed on to the file
`{\textit{dest}\hspace{0.2em}\textit{suffix}}'.
Surely, the same effect is achieved by
directly specifying the
argument `{\textit{dest}\hspace{0.2em}\textit{suffix}}'
in the first form.
However, that requires to set up a different file
for each child. With the alternative form of the command
all these files can have exactly the same content
which simplifies setting them up and maintaining them.

For example, the following file |draft.tex|
with a compilation flag |\version| as described in \secref{sec:flags}
compiles the main document as a draft:
%
\begin{center}
\begin{tabular}{l}
|\def\version{draft}|\\
|\input{childdoc.def}|\\
|\childdocforward{|\textit{main}|}|
\end{tabular}
\end{center}
%
Likewise, the following files |final|\textit{nn}|.tex|
compile the final version of the child document
|child|\textit{nn}|.tex|:
%
\begin{center}
\begin{tabular}{l}
|\def\version{final}|\\
|\input{childdoc.def}|\\
|\childdocforwardprefix{final}{child}|
\end{tabular}
\end{center}
%

Note that when several versions of a main file and/or of each child file
are to be generated, it may be convenient to set up a |Makefile| or
shell script to automatise the process.

%%%%%%%%%%%%%%%%%%%%%%%%%%%%%%%%%%%%%%%%%%%%%%%%%%%%%%%%%%%%%%%%%%%%%%%%%%%%%%%%
\subsection{Command Line Processing}
\label{sec:commandline}

The effect of redirection files can also be achieved by invoking
the \LaTeX{} compiler with a more elaborate command line.
Most conveniently this should be done as part
of a shell script or a |Makefile|.

When using \textsf{childdoc} in the main file, the following
command lines effectively perform a redirection
(note that depending on the shell being used,
backslashes may have to be doubled: `|\|' $\to$ `|\\|'):
%
\begin{center}
|... -jobname "|\textit{target}|" |\\|"|[\textit{flags}]%
|\input{childdoc.def}\childdocforward[|\textit{main}|]{|\textit{dest}|}"|
\end{center}
%
Here \textit{target} is the name of the output file,
\textit{main} is the name of the main file
and \textit{dest} is the name of the main or child file to be processed
(all filenames without extensions).
The optional argument \textit{main} can be omitted
if \textit{main} matches \textit{dest}.
Optionally, compilation \textit{flags} can be defined via |\def| commands.
This command line makes the \TeX{} engine believe
it is compiling the file \textit{target}
whose content is specified as the latter parameter.
The provided code then forwards the processing to
\textit{main} or \textit{dest} as described in \secref{sec:forward}.

%%%%%%%%%%%%%%%%%%%%%%%%%%%%%%%%%%%%%%%%%%%%%%%%%%%%%%%%%%%%%%%%%%%%%%%%%%%%%%%%
\subsection{Include by Input}
\label{sec:input}

Including child documents by |\include| has some restrictions by design.
Most notably, the content of a child document always occupies
its own set of pages; pages cannot be shared between child documents.
Usually, this behaviour makes perfect sense
because each child document contain an essential part of the document.
However, in some situations it may be desirable to compose
a document from a collection of parts
without having mandatory page breaks between then.
For this case, the package
provides a mechanism to include parts
by |\input| which can also be processed individually.
However, by construction this mechanism
requires manual handling of the content to be output.

%%%%%%%%%%%%%%%%%%%%%%%%%%%%%%%%%%%%%%%%
\DescribeMacro{\ifchilddocmanual}
The main file should be prepared as usual, see \secref{sec:include}.
However, the document body must make a distinction
between processing of an individual part and of the main document, e.g.:
%
\begin{center}
\begin{tabular}{l}
|\ifchilddocmanual|\\
|\input{\childdocname}|\\
|\||else|\\
\textit{document body with }|\input{|\textit{part}|}|\\
|\||fi|
\end{tabular}
\end{center}
%
The conditional |\ifchilddocmanual| is true whenever
a part to be included by |\input| is being compiled,
and the name of the part is stored in |\childdocname|.

%%%%%%%%%%%%%%%%%%%%%%%%%%%%%%%%%%%%%%%%
\DescribeMacro{\childdocby}
Each part to be included by |\input| should start with:
%
\begin{center}
\begin{tabular}{l}
|\input{childdoc.def}|\\
|\childdocby{|\textit{main}|}|\\
\end{tabular}
\end{center}
%
The directive |\childdocby| is similar to |\childdocof|
described in \secref{sec:include},
but the subsequent selection of content must be done manually.
To that end, both |\ifchilddoc| and |\ifchilddocmanual|
will be true upon processing of a part,
and the name of the part is stored in |\childdocname|.
Note that |\jobname| will be set to the filename of the current part
so that each part receives an individual |.aux| file
that does not interfere with the |.aux| file(s) of the main document.
This behaviour can be altered by the alternative form
|\childdocby[*]{|\textit{main}|}| (with a non-empty optional argument)
which uses the |.aux| file of the main document
by setting |\jobname| to \textit{main}.

%%%%%%%%%%%%%%%%%%%%%%%%%%%%%%%%%%%%%%%%%%%%%%%%%%%%%%%%%%%%%%%%%%%%%%%%%%%%%%%%
\subsection{Driver Development}
\label{sec:driver}

The \textsf{childdoc} mechanism can also be use for the development
of definition files such as \LaTeX{} styles or classes.
This case differs from the above setup with multiple parts
included by |\include| in that no |\includeonly| should be invoked.
This can be achieved by starting the include file
(before |\ProvidesPackage|) with:
%
\begin{center}
\begin{tabular}{l}
|\input{childdoc.def}|\\
|\childdocforward{|\textit{main}|}|\\
\end{tabular}
\end{center}
%
or alternatively with:
%
\begin{center}
\begin{tabular}{l}
|\input{childdoc.def}|\\
|\childdocby{|\textit{main}|}|\\
\end{tabular}
\end{center}
%
Both forms have slightly different effects as described above.
The main file is prepared as usual, see \secref{sec:include}.

%%%%%%%%%%%%%%%%%%%%%%%%%%%%%%%%%%%%%%%%%%%%%%%%%%%%%%%%%%%%%%%%%%%%%%%%%%%%%%%%
\subsection{Legacy Detection}
\label{sec:detection}

The directive |\childdocmain| in the main file can detect
whether the complete document or merely a child is to be compiled
even without using the directive |\childdocof|.
This method is deprecated because it is less robust
and there is no compelling reason to use it;
it is merely provided for backward compatibility
and it may be removed in future versions.

If the detection mechanism is to be used,
it is mandatory to correctly specify
the filename of the main file as the argument of |\childdocmain|:
%
\begin{center}
\begin{tabular}{l}
|\input{childdoc.def}|\\
|\childdocmain{|\textit{main}|}|\\
\end{tabular}
\end{center}
%
If |\jobname| does not match the argument \textit{main} of |\childdocmain|,
it is assumed that |\jobname| points to the child file to be compiled.
When using |\childdocmain| with the main file specified as argument,
it suffices to start a child file
with just |\input{|\textit{main}|}|
without loading of the package and using |\childdocof|.
If instead all processing is done
with the appropriate \textsf{childdoc} directives,
the argument of \textit{main} of |\childdocmain| can be empty.

An alternative version of the command line processing described
in \secref{sec:commandline} using the detection mechanism reads:
%
\begin{center}
|... -jobname "|\textit{target}|" "|[\textit{flags}]%
[|\def\jobname{|\textit{dest}|}|]|\input{|\textit{main}|}"|
\end{center}

%%%%%%%%%%%%%%%%%%%%%%%%%%%%%%%%%%%%%%%%%%%%%%%%%%%%%%%%%%%%%%%%%%%%%%%%%%%%%%%%
\subsection{Manual Code}
\label{sec:manual}

In case one cannot be certain whether the definitions file |childdoc.def|
is installed on the target \TeX{} distribution
and one prefers not to ship it,
it is conceivable to paste a few relevant commands into the sources.

To that end, drop all statements |\input{childdoc.def}|
and perform the replacements as outlined below.
Instead of |\childdocmain{|\textit{main}|}| add the following code
to the top of the main file:
%
\begin{center}
\begin{tabular}{l}
|\||ifdefined\childdocname\endinput\||fi\newif\ifchilddoc|\\
|\edef\childdocname{\scantokens\expandafter{\jobname\noexpand}}|\\
|\def\childdocmain{|\textit{main}|}\||ifx\childdocmain\childdocname\||else|\\
|\childdoctrue\includeonly{\childdocname}\let\jobname\childdocmain\||fi|\\
\end{tabular}
\end{center}
%
Instead of |\childdocof{|\textit{main}|}| just include the main file
at the top of each child file:
%
\begin{center}
|\input{|\textit{main}|}|
\end{center}
%
A simple redirection |\childdocforward{|\textit{dest}|}| is achieved by:
%
\begin{center}
|\def\jobname{|\textit{dest}|}\input{\jobname}|
\end{center}
%
The redirection with prefix
|\childdocforwardprefix[|\textit{prefix}|]{|\textit{dest}|}|
is accomplished by:
%
\begin{center}
\begin{tabular}{l}
|{\edef\jobname{\scantokens\expandafter{\jobname\noexpand}}|\\
|\def\redirectjob |\textit{prefix}|#1~~~{\gdef\jobname{|\textit{dest}|#1}}|\\
|\expandafter\redirectjob\jobname~~~}\input{\jobname}|
\end{tabular}
\end{center}

In an alternative approach,
child documents can be compiled by a specific command line
without additional code or specific definitions:
%
\begin{center}
|... -jobname "|\textit{target}|" "|[\textit{flags}]%
|\includeonly{|\textit{dest}|}\input{|\textit{main}|}"|
\end{center}
%

%%%%%%%%%%%%%%%%%%%%%%%%%%%%%%%%%%%%%%%%%%%%%%%%%%%%%%%%%%%%%%%%%%%%%%%%%%%%%%%%
%%%%%%%%%%%%%%%%%%%%%%%%%%%%%%%%%%%%%%%%%%%%%%%%%%%%%%%%%%%%%%%%%%%%%%%%%%%%%%%%
\section{Information}

%%%%%%%%%%%%%%%%%%%%%%%%%%%%%%%%%%%%%%%%%%%%%%%%%%%%%%%%%%%%%%%%%%%%%%%%%%%%%%%%
\subsection{Copyright}

Copyright \copyright{} 2017--2018 Niklas Beisert

This work may be distributed and/or modified under the
conditions of the \LaTeX{} Project Public License, either version 1.3
of this license or (at your option) any later version.
The latest version of this license is in
  \url{http://www.latex-project.org/lppl.txt}
and version 1.3 or later is part of all distributions of \LaTeX{}
version 2005/12/01 or later.

This work has the LPPL maintenance status `maintained'.

The Current Maintainer of this work is Niklas Beisert.

This work consists of the files |README.txt|, |childdoc.ins| and |childdoc.dtx|
as well as the derived files |childdoc.def|, |cdocsamp.tex|
with |cdocsch1.tex|, |cdocsch2.tex|, |cdocspt3.tex|, |cdocspt4.tex|,
|cdocsdrf.tex|, |cdocsfn1.tex|, |cdocsfn2.tex|
as well as |childdoc.pdf|.

%%%%%%%%%%%%%%%%%%%%%%%%%%%%%%%%%%%%%%%%%%%%%%%%%%%%%%%%%%%%%%%%%%%%%%%%%%%%%%%%
\subsection{Files and Installation}

The package consists of the files:
%
\begin{center}
\begin{tabular}{ll}
    |README.txt|   & readme file \\
    |childdoc.ins| & installation file \\
    |childdoc.dtx| & source file \\
    |childdoc.def| & definition file \\
    |cdocsamp.tex| & sample main file \\
    |cdocsch1.tex| & sample include file \\
    |cdocsch2.tex| & sample include file \\
    |cdocspt3.tex| & sample part file \\
    |cdocspt4.tex| & sample part file \\
    |cdocsdrf.tex| & sample redirection file \\
    |cdocsfn1.tex| & sample redirection file \\
    |cdocsfn2.tex| & sample redirection file \\
    |childdoc.pdf| & manual
\end{tabular}
\end{center}
%
The distribution consists of the files
|README.txt|, |childdoc.ins| and |childdoc.dtx|.
%
\begin{itemize}
\item
Run (pdf)\LaTeX{} on |childdoc.dtx|
to compile the manual |childdoc.pdf| (this file).
\item
Run \LaTeX{} on |childdoc.ins| to create the definitions file |childdoc.def|
and the sample |cdocsamp.tex| with include files
|cdocsch1.tex|, |cdocsch2.tex|, |cdocspt3.tex|, |cdocspt4.tex|,
|cdocsdrf.tex|, |cdocsfn1.tex|, |cdocsfn2.tex|.
Then copy the file |childdoc.def| to an appropriate directory of your \LaTeX{}
distribution, e.g.\ \textit{texmf-root}|/tex/latex/childdoc|.
\end{itemize}

%%%%%%%%%%%%%%%%%%%%%%%%%%%%%%%%%%%%%%%%%%%%%%%%%%%%%%%%%%%%%%%%%%%%%%%%%%%%%%%%
\subsection{Related CTAN Packages}

There are several other packages which offer a similar functionality:
%
\begin{itemize}
\item
The packages
\href{http://ctan.org/pkg/docmute}{\textsf{docmute}},
\href{http://ctan.org/pkg/includex}{\textsf{includex}} and
\href{http://ctan.org/pkg/standalone}{\textsf{standalone}}
provide commands to include only the document body of
a child file thus allowing both files to be compiled individually.
\item
The packages \href{http://ctan.org/pkg/subdocs}{\textsf{subdocs}}
and \href{http://ctan.org/pkg/subfiles}{\textsf{subfiles}}
provide structures in which the main and child documents can be
encapsulated and allowing them to be compiled individually.
The inclusion mechanism is different from the conventional |\include|.
\item
The package \href{http://ctan.org/pkg/combine}{\textsf{combine}}
is an elaborate solution to combine several documents into one.
\end{itemize}
%
See also the CTAN topic \href{http://ctan.org/topic/subdocs}{\textsf{subdocs}}
for further related packages.
The present package differs from the above solutions in that
a document structure constructed with the conventional |\include| mechanism
just needs two extra commands at the top of every file
such that all constituent files can be compiled individually.

%%%%%%%%%%%%%%%%%%%%%%%%%%%%%%%%%%%%%%%%%%%%%%%%%%%%%%%%%%%%%%%%%%%%%%%%%%%%%%%%
%\subsection{Feature Suggestions}
%
%The following is a list of features which may be useful for future
%versions of this package:
%%
%\begin{itemize}
%\item
%\ldots
%\end{itemize}

%%%%%%%%%%%%%%%%%%%%%%%%%%%%%%%%%%%%%%%%%%%%%%%%%%%%%%%%%%%%%%%%%%%%%%%%%%%%%%%%
\subsection{Revision History}

%%%%%%%%%%%%%%%%%%%%%%%%%%%%%%%%%%%%%%%%
\paragraph{v2.0:} 2018/12/30

\begin{itemize}
\item
immediate forward processing
\item
added |\childdocby| mechanism
\item
manual restructured
\end{itemize}

%%%%%%%%%%%%%%%%%%%%%%%%%%%%%%%%%%%%%%%%
\paragraph{v1.6:} 2018/01/17

\begin{itemize}
\item
application for development of include files
\item
corrections to manual
\end{itemize}

%%%%%%%%%%%%%%%%%%%%%%%%%%%%%%%%%%%%%%%%
\paragraph{v1.5:} 2017/05/21

\begin{itemize}
\item
more complete structuring introduced
\item
|\childdocof| introduced
\item
|\childdoc| renamed to |\childdocmain|
\item
|\childredirect| renamed to |\childdocforward| and |\childdocforwardprefix|
and functionality expanded
\end{itemize}

%%%%%%%%%%%%%%%%%%%%%%%%%%%%%%%%%%%%%%%%
\paragraph{v1.0:} 2017/04/27

\begin{itemize}
\item
manual and install package
\item
first version published on CTAN
\end{itemize}

%%%%%%%%%%%%%%%%%%%%%%%%%%%%%%%%%%%%%%%%
\paragraph{v0.6:} 2017/04/26

\begin{itemize}
\item
redirection mechanism added
\end{itemize}

%%%%%%%%%%%%%%%%%%%%%%%%%%%%%%%%%%%%%%%%
\paragraph{v0.5:} 2017/04/26

\begin{itemize}
\item
functionality in definition file
\end{itemize}


%%%%%%%%%%%%%%%%%%%%%%%%%%%%%%%%%%%%%%%%%%%%%%%%%%%%%%%%%%%%%%%%%%%%%%%%%%%%%%%%
%%%%%%%%%%%%%%%%%%%%%%%%%%%%%%%%%%%%%%%%%%%%%%%%%%%%%%%%%%%%%%%%%%%%%%%%%%%%%%%%
%%%%%%%%%%%%%%%%%%%%%%%%%%%%%%%%%%%%%%%%%%%%%%%%%%%%%%%%%%%%%%%%%%%%%%%%%%%%%%%%
\appendix

\settowidth\MacroIndent{\rmfamily\scriptsize 000\ }

 \DocInput{childdoc.dtx}

\end{document}
%</driver>
% \fi
%
% %%%%%%%%%%%%%%%%%%%%%%%%%%%%%%%%%%%%%%%%%%%%%%%%%%%%%%%%%%%%%%%%%%%%%%%%%%%%%%
% %%%%%%%%%%%%%%%%%%%%%%%%%%%%%%%%%%%%%%%%%%%%%%%%%%%%%%%%%%%%%%%%%%%%%%%%%%%%%%
% \section{Sample}
%\iffalse
%<*samplemain>
%\fi
%
% The following presents a sample document
% with two chapters, two parts, a title page,
% a compile flag as well as three forwarding files to set the flag.
% It consists of eight |.tex| files:
% \begin{center}
% \begin{tabular}{ll}
% |cdocsamp.tex|&main file\\
% |cdocsch1.tex|&include file for chapter 1\\
% |cdocsch2.tex|&include file for chapter 2\\
% |cdocspt3.tex|&include file for part 3\\
% |cdocspt4.tex|&include file for part 4\\
% |cdocsdrf.tex|&forwarding file for main file in draft mode\\
% |cdocsfi1.tex|&forwarding file for final version of chapter 1\\
% |cdocsfi2.tex|&forwarding file for final version of chapter 2\\
% \end{tabular}
% \end{center}
% Each of the eight files can be compiled directly by the \LaTeX{} compiler.
%
% %%%%%%%%%%%%%%%%%%%%%%%%%%%%%%%%%%%%%%
% \paragraph{Main File.}
%
% The main file is called |cdocsamp.tex|.
%
% Load the \textsf{childdoc} definitions and
% declare the filename for the main document:
%    \begin{macrocode}
\input{childdoc.def}
\childdocmain{}
%    \end{macrocode}

% Optional override for |\version| flag:
%    \begin{macrocode}
%%\ifchilddoc\else\providecommand{\version}{draft}\fi
%    \end{macrocode}

% Define the default values for the |\version| flag
% (|final| for the main file and |draft| for childs):
%    \begin{macrocode}
\ifchilddoc
\providecommand{\version}{draft}
\else
\providecommand{\version}{final}
\fi
%    \end{macrocode}

% Load the standard document class:
%    \begin{macrocode}
\documentclass[12pt]{article}
%    \end{macrocode}

% Start the document body:
%    \begin{macrocode}
\begin{document}
%    \end{macrocode}

% Declare a title page.
% Print title, part of document being processed and version flag:
%    \begin{macrocode}
\addtocounter{page}{-1}
\begin{center}
{\LARGE\bfseries{}childdoc example\par}
\vspace{1cm}
\ifchilddoc
\ifchilddocmanual part\else chapter\fi:
`\childdocname' of `\childdocjob'\par
\else
main document: `\childdocjob'\par
\fi
version: \version\par
\end{center}
\newpage
%    \end{macrocode}

% Manually include selected file,
% otherwise process as usual:
%    \begin{macrocode}
\ifchilddocmanual
\section*{part `\childdocname'}
\input{\childdocname}
\else
%    \end{macrocode}

% Include the two chapters:
%    \begin{macrocode}
\include{cdocsch1}
\include{cdocsch2}
%    \end{macrocode}

% Include the two parts unless only chapters should be displayed:
%    \begin{macrocode}
\ifchilddoc\else
\section{part three}
\input{cdocspt3}
\section{part four}
\input{cdocspt4}
\fi
%    \end{macrocode}

% Process as usual until here:
%    \begin{macrocode}
\fi
%    \end{macrocode}

% End of document body:
%    \begin{macrocode}
\end{document}
%    \end{macrocode}
%\iffalse
%</samplemain>
%\fi
%
% %%%%%%%%%%%%%%%%%%%%%%%%%%%%%%%%%%%%%%
% \paragraph{Chapter Include Files.}
%
% The include files are called |cdocsch1.tex| and |cdocsch2.tex|.
%
%\iffalse
%<*samplechap1|samplechap2>
%\fi

% Optional override for |\version| flag:
%    \begin{macrocode}
%%\providecommand{\version}{final}
%    \end{macrocode}

% Include the main document:
%    \begin{macrocode}
\input{childdoc.def}
\childdocof{cdocsamp}
%    \end{macrocode}

%\iffalse
%</samplechap1|samplechap2>
%\fi
%
%\iffalse
%<*samplechap1>
%\fi
% Some text for chapter 1:
%    \begin{macrocode}
\section{one}
some text in chapter one
%    \end{macrocode}

%\iffalse
%</samplechap1>
%\fi
% Some text for chapter 2:
%\iffalse
%<*samplechap2>
%\fi
%    \begin{macrocode}
\section{two}
more text in chapter two
%    \end{macrocode}

%\iffalse
%</samplechap2>
%\fi
%
% %%%%%%%%%%%%%%%%%%%%%%%%%%%%%%%%%%%%%%
% \paragraph{Part Include Files.}
%
% The include files are called |cdocspt3.tex| and |cdocspt4.tex|.
%
%\iffalse
%<*samplepart3|samplepart4>
%\fi

% Optional override for |\version| flag:
%    \begin{macrocode}
%%\providecommand{\version}{final}
%    \end{macrocode}

% Include the main document:
%    \begin{macrocode}
\input{childdoc.def}
\childdocby{cdocsamp}
%    \end{macrocode}

%\iffalse
%</samplepart3|samplepart4>
%\fi
%
%\iffalse
%<*samplepart3>
%\fi
% Some text for part 3:
%    \begin{macrocode}
some text in part three
%    \end{macrocode}

%\iffalse
%</samplepart3>
%\fi
% Some text for part 4:
%\iffalse
%<*samplepart4>
%\fi
%    \begin{macrocode}
more text in part four
%    \end{macrocode}

%\iffalse
%</samplepart4>
%\fi
%
% %%%%%%%%%%%%%%%%%%%%%%%%%%%%%%%%%%%%%%
% \paragraph{Forwarding for a Complete Draft.}
%
% The following forwarding file |cdocsdrf.tex|
% compiles the main document in draft mode:
%\iffalse
%<*sampledraft>
%\fi
%    \begin{macrocode}
\def\version{draft}
\input{childdoc.def}
\childdocforward{cdocsamp}
%    \end{macrocode}

%\iffalse
%</sampledraft>
%\fi
%
% %%%%%%%%%%%%%%%%%%%%%%%%%%%%%%%%%%%%%%
% \paragraph{Forwarding for Final Version of the Chapters.}
%
% The following forwarding files |cdocsfn1.tex| and |cdocsfn2.tex|
% (with identical content)
% compile the final versions of the child documents
% |cdocsch1.tex| and |cdocsch2.tex|, respectively:
%\iffalse
%<*samplefinal>
%\fi
%    \begin{macrocode}
\def\version{final}
\input{childdoc.def}
\childdocforwardprefix[cdocsamp]{cdocsfn}{cdocsch}
%    \end{macrocode}

%\iffalse
%</samplefinal>
%\fi
%
% %%%%%%%%%%%%%%%%%%%%%%%%%%%%%%%%%%%%%%
% \paragraph{Command Line Processing.}
%
% The following three command lines generate the output files
% |cdocscld|, |cdocscl1| and |cdocscl2|
% which should be identical to
% |cdocsdrf|, |cdocsch1| and |cdocsfn2|, respectively:
% \begin{center}
% \begin{tabular}{l}
% |latex -jobname cdocscld \|\\
% |  "\def\version{draft}\input{childdoc.def}\childdocforward{cdocsamp}"|\\
% |latex -jobname cdocscl1 \|\\
% |  "\input{childdoc.def}\childdocforward[cdocsamp]{cdocsch1}"|\\
% |latex -jobname cdocscl2 \|\\
% |  "\def\version{final}\input{childdoc.def}\childdocforward{cdocsch2}"|
% \end{tabular}
% \end{center}
% Note that the trailing backslash on each first line
% merely continues the input to the second line
% (for convenient cut ant paste).
% Furthermore, the command |latex| can be replaced by any
% of its alternative versions such as |pdflatex|.
%
% %%%%%%%%%%%%%%%%%%%%%%%%%%%%%%%%%%%%%%%%%%%%%%%%%%%%%%%%%%%%%%%%%%%%%%%%%%%%%%
% %%%%%%%%%%%%%%%%%%%%%%%%%%%%%%%%%%%%%%%%%%%%%%%%%%%%%%%%%%%%%%%%%%%%%%%%%%%%%%
% \section{Implementation}
%\iffalse
%<*package>
%\fi
%
% This section describes the definitions file |childdoc.def|.

% The definitions cannot be loaded using |\usepackage| or |\RequirePackage|
% which has a mechanism to prevent loading a style file more than once.
% When loading the definitions by means of |\input|
% multiple instances have to be prevented manually:
%\iffalse
%This code needs to be before the `\ProvidesFile' directive
%which is defined at the beginning of this file.
%Therefore it is also placed there and commented out here.
%</package>
%<*discard>
%\fi
%    \begin{macrocode}
\ifdefined\childdocmain\endinput\fi
%    \end{macrocode}
%\iffalse
%</discard>
%<*package>
%\fi
%
% \macro{\ifchilddoc}
% \macro{\ifchilddocmanual}
% The conditional |\ifchilddoc| tells whether a
% child (true) or main (false) document is being compiled.
% The conditional |\ifchilddocmanual| tells whether
% the |\includeonly| mechanism is used (false) or
% the selection of child files must be performed manually (true).
% The definitions initialise to false:
%    \begin{macrocode}
\newif\ifchilddoc
\newif\ifchilddocmanual
%    \end{macrocode}

% \macro{\childdocname}
% \macro{\childdocjob}
% The macro |\childdocname| stores the name of the main document
% to be compiled. The macro |\childdocjob| stores the name of
% the document on which the \LaTeX{} compiler was originally invoked.
% The content of |\jobname| cannot be compared
% to filenames specified in the source due to different catcodes.
% The following code rescans |\jobname|, stores the result
% in |\childdocname| and saves a copy in |\childdocjob|:
%    \begin{macrocode}
\edef\childdocname{\scantokens\expandafter{\jobname\noexpand}}
\let\childdocjob\childdocname
%    \end{macrocode}

% \macro{\childdocdisable}
% The macro |\childdocdisable| prevents the main file
% from being processed more than once.
% At this stage, the main document command |\childdocmain|
% is assumed to be called once again where it should do nothing.
% Any subsequent call to it should prevent
% a secondary processing of the main document
% It overwrites the forwarding commands
% |\childdocof| and |\childdocforward|
% with empty macros to prevent further inclusions of the main document:
%    \begin{macrocode}
\newcommand{\childdocdisable}
{
  \renewcommand{\childdocmain}[1]{\renewcommand{\childdocmain}[1]{\endinput}}
  \renewcommand{\childdocof}[1]{}
  \renewcommand{\childdocby}[2][]{}
  \renewcommand{\childdocforward}[2][]{}
  \renewcommand{\childdocdisable}{}
}
%    \end{macrocode}

% \macro{\childdocmain}
% The macro |\childdocmain| is to be called at the top of the main file
% with nothing or the main filename (without extension) as argument.
% First, it breaks loops.
% If the argument is not empty and does not match |\childdocname|
% (which is set by the first inclusion of |childdoc.def|),
% |\ifchilddoc| is set to true, |\includeonly| is applied to the child file
% and |\jobname| is set to the main file
% (for proper handling of |.aux| files):
%    \begin{macrocode}
\newcommand{\childdocmain}[1]
{
  \childdocdisable\childdocmain{}
  \if?#1?\else
    \begingroup
      \def\childdoctmp{#1}
      \ifx\childdoctmp\childdocname
        \def\childdoctmp{}
      \else
        \def\childdoctmp
        {
          \childdoctrue
          \includeonly{\childdocname}
          \def\childdocjob{#1}
          \def\jobname{#1}
        }
      \fi
      \expandafter
    \endgroup
    \childdoctmp
  \fi
}
%    \end{macrocode}

% \macro{\childdocof}
% The command |\childdocof| redirects
% compilation to the main file |#1|.
%    \begin{macrocode}
\newcommand{\childdocof}[1]
{
  \childdocdisable
  \childdoctrue
  \includeonly{\childdocname}
  \def\jobname{#1}
  \def\childdocjob{#1}
  \input{#1}
}
%    \end{macrocode}

% \macro{\childdocby}
% The command |\childdocby| ....
%    \begin{macrocode}
\newcommand{\childdocby}[2][]
{
  \childdocdisable
  \childdoctrue
  \childdocmanualtrue
  \if?#1?\else
    \def\jobname{#2}
  \fi
  \def\childdocjob{#2}
  \input{#2}
  \endinput
}
%    \end{macrocode}

% \macro{\childdocforward}
% The command |\childdocforward| redirects
% compilation to the main file or
% (if the optional argument is given) a child file.
% Parameters are set as if the main file
% or a child file starting with |\childdocof| was compiled.
% Then compilation is handed over to the main file:
%    \begin{macrocode}
\newcommand{\childdocforward}[2][]
{
  \begingroup
    \if?#1?
      \def\childdoctmp
      {
        \def\childdocname{#2}
        \def\childdocjob{#2}
        \def\jobname{#2}
        \input{#2}
        \endinput
      }
    \else
      \def\childdoctmp
      {
        \childdocdisable
        \def\childdocname{#2}
        \childdoctrue
        \includeonly{#2}
        \def\childdocjob{#1}
        \def\jobname{#1}
        \input{#1}
        \endinput
      }
    \fi
    \expandafter
  \endgroup
  \childdoctmp
}
%    \end{macrocode}

% \macro{\childdocforwardprefix}
% The command |\childdocforwardprefix| redirects
% compilation to the main or a child file by means of a pattern.
% The prefix |#1| in the current filename is replaced by |#2|
% and the suffix of the current filename is kept
% (it is assumed that the filename does not contain the substring `|~~~|'
% which is used as a delimiter).
% Compilation is handed over to the new file by |\childdocforward|:
%    \begin{macrocode}
\newcommand{\childdocforwardprefix}[3][]
{
  \begingroup
    \def\childdocextract #2##1~~~{\def\childdoctmp{\childdocforward[#1]{#3##1}}}
    \expandafter\childdocextract\childdocname~~~
    \expandafter
  \endgroup
  \childdoctmp
}
%    \end{macrocode}

% \macro{\childdoc}
% The deprecated macro |\childdoc| is a legacy version of |\childdocmain|:
%    \begin{macrocode}
\newcommand{\childdoc}{\childdocmain}
%    \end{macrocode}

% \macro{\childdocredirect}
% The deprecated macro |\childdocredirect| is a legacy version
% of |\childdocforward| and |\childdocforwardprefix|:
%    \begin{macrocode}
\newcommand{\childdocredirect}[2][]
{
  \begingroup
    \if?#1?
      \def\childdoctmp{\childdocforward{#2}}
    \else
      \def\childdoctmp{\childdocforwardprefix{#1}{#2}}
    \fi
    \expandafter
  \endgroup
  \childdoctmp
}
%    \end{macrocode}

%\iffalse
%</package>
%\fi
%
\endinput
|\\
|\childdocmain{|\textit{main}|}|\\
\end{tabular}
\end{center}
%
If |\jobname| does not match the argument \textit{main} of |\childdocmain|,
it is assumed that |\jobname| points to the child file to be compiled.
When using |\childdocmain| with the main file specified as argument,
it suffices to start a child file
with just |\input{|\textit{main}|}|
without loading of the package and using |\childdocof|.
If instead all processing is done
with the appropriate \textsf{childdoc} directives,
the argument of \textit{main} of |\childdocmain| can be empty.

An alternative version of the command line processing described
in \secref{sec:commandline} using the detection mechanism reads:
%
\begin{center}
|... -jobname "|\textit{target}|" "|[\textit{flags}]%
[|\def\jobname{|\textit{dest}|}|]|\input{|\textit{main}|}"|
\end{center}

%%%%%%%%%%%%%%%%%%%%%%%%%%%%%%%%%%%%%%%%%%%%%%%%%%%%%%%%%%%%%%%%%%%%%%%%%%%%%%%%
\subsection{Manual Code}
\label{sec:manual}

In case one cannot be certain whether the definitions file |childdoc.def|
is installed on the target \TeX{} distribution
and one prefers not to ship it,
it is conceivable to paste a few relevant commands into the sources.

To that end, drop all statements |% \iffalse
%
% childdoc.dtx Copyright (C) 2017-2018 Niklas Beisert
%
% This work may be distributed and/or modified under the
% conditions of the LaTeX Project Public License, either version 1.3
% of this license or (at your option) any later version.
% The latest version of this license is in
%   http://www.latex-project.org/lppl.txt
% and version 1.3 or later is part of all distributions of LaTeX
% version 2005/12/01 or later.
%
% This work has the LPPL maintenance status `maintained'.
%
% The Current Maintainer of this work is Niklas Beisert.
%
% This work consists of the files childdoc.dtx and childdoc.ins
% and the derived files childdoc.def and cdocsamp.tex with
% cdocsch1.tex, cdocsch2.tex, cdocsdrf.tex, cdocsfn1.tex, cdocsfn2.tex.
%
%<package>\ifdefined\childdocmain\endinput\fi
%<package>\ProvidesFile{childdoc.def}[2018/12/30 v2.0 child document driver]
%<samplemain>\ProvidesFile{cdocsamp.tex}[2018/12/30 v2.0 sample for childdoc]
%<*driver>
%\ProvidesFile{childdoc.drv}[2018/12/30 v2.0 childdoc reference manual file]
\PassOptionsToClass{10pt,a4paper}{article}
\documentclass{ltxdoc}

\usepackage[margin=35mm]{geometry}
\usepackage{hyperref}
\usepackage{hyperxmp}
\usepackage[usenames]{color}

\hypersetup{colorlinks=true}
\hypersetup{pdfstartview=FitH}
\hypersetup{pdfpagemode=UseNone}
\hypersetup{pdfsource={}}
\hypersetup{pdflang={en-UK}}
\hypersetup{pdfcopyright={Copyright 2017-2018 Niklas Beisert.
  This work may be distributed and/or modified under the
  conditions of the LaTeX Project Public License, either version 1.3
  of this license or (at your option) any later version.}}
\hypersetup{pdflicenseurl={http://www.latex-project.org/lppl.txt}}
\hypersetup{pdfcontactaddress={ETH Zurich, ITP, HIT K,
  Wolfgang-Pauli-Strasse 27}}
\hypersetup{pdfcontactpostcode={8093}}
\hypersetup{pdfcontactcity={Zurich}}
\hypersetup{pdfcontactcountry={Switzerland}}
\hypersetup{pdfcontactemail={nbeisert@itp.phys.ethz.ch}}
\hypersetup{pdfcontacturl={http://people.phys.ethz.ch/\xmptilde nbeisert/}}

\newcommand{\secref}[1]{\hyperref[#1]{section \ref*{#1}}}

\parskip1ex
\parindent0pt
\let\olditemize\itemize
\def\itemize{\olditemize\parskip0pt}

\begin{document}

\title{The \textsf{childdoc} Package}
\hypersetup{pdftitle={The childdoc Package}}
\author{Niklas Beisert\\[2ex]
  Institut f\"ur Theoretische Physik\\
  Eidgen\"ossische Technische Hochschule Z\"urich\\
  Wolfgang-Pauli-Strasse 27, 8093 Z\"urich, Switzerland\\[1ex]
  \href{mailto:nbeisert@itp.phys.ethz.ch}
  {\texttt{nbeisert@itp.phys.ethz.ch}}}
\hypersetup{pdfauthor={Niklas Beisert}}
\hypersetup{pdfsubject={Manual for the LaTeX2e Package childdoc}}
\date{30 December 2018, \textsf{v2.0}}
\maketitle

\begin{abstract}\noindent
\textsf{childdoc} is a \LaTeXe{} package
that enables the direct compilation
of document sections included by |\include|
to individual files.
\end{abstract}

\begingroup
\parskip0ex
\tableofcontents
\endgroup

%%%%%%%%%%%%%%%%%%%%%%%%%%%%%%%%%%%%%%%%%%%%%%%%%%%%%%%%%%%%%%%%%%%%%%%%%%%%%%%%
%%%%%%%%%%%%%%%%%%%%%%%%%%%%%%%%%%%%%%%%%%%%%%%%%%%%%%%%%%%%%%%%%%%%%%%%%%%%%%%%
\section{Introduction}

\LaTeX{} provides a mechanism to structure a large document (such as a book)
into a main file and several child files (containing the chapters)
using the |\include| command.
This mechanism is beneficial for documents
which span hundreds of pages in order to
make the source file(s) more manageable.
Moreover, compilation can be restricted to
selected child files by means of the |\includeonly| command.
The latter feature can be used to reduce the compilation time while editing
(this was significantly more useful in the earlier days of \LaTeX{})
or to generate a smaller document which is easier to navigate.
Another application of |\includeonly| is to generate
documents consisting of selected parts of the complete document.

However, there are a few drawbacks of the plain |\include| mechanism:
\begin{itemize}
\item
The child files cannot be compiled on their own,
they can only be compiled via the main file.
A naive editing environment
(such as a text editor with an option
to have the current file processed by \LaTeX)
may require one to switch to the main file before compiling;
attempting to compile the child file produces errors.
\item
The main file must be modified (each time)
to adjust the |\includeonly| command
to the present needs. This easily leaves the main file in a messy state.
\item
The generated document will always carry the filename
of the main document. This is inconvenient if
several child files are to be compiled and
to be kept for distribution.
\end{itemize}

The present package provides a simple interface
to make child files individually compilable by \LaTeX{}.
Compiling a child file then has the same effect as compiling
the main file with an |\includeonly| command
to select the appropriate child.
Moreover the generated document will carry the name of the child
rather than the main file.
This resolves all three above issues.

This feature is meant to make the editing of books,
thesis documents and lecture notes somewhat more convenient.
However, the package can also be used efficiently for
composing a series of documents (such as exercise sheets)
which are typically distributed individually.
It then assists the author in generating the individual documents
(potentially in different versions)
as well as a document containing the collected series.
Another application is in developing style files
or other kinds of included material
where compilation of the style file could redirect
to a sample or test file.

%%%%%%%%%%%%%%%%%%%%%%%%%%%%%%%%%%%%%%%%%%%%%%%%%%%%%%%%%%%%%%%%%%%%%%%%%%%%%%%%
%%%%%%%%%%%%%%%%%%%%%%%%%%%%%%%%%%%%%%%%%%%%%%%%%%%%%%%%%%%%%%%%%%%%%%%%%%%%%%%%
\section{Usage}

First of all, the package \textsf{childdoc} is \emph{not} a standard
\LaTeXe{} |.sty| style file! Therefore it needs to be invoked in
a non-standard way.

%%%%%%%%%%%%%%%%%%%%%%%%%%%%%%%%%%%%%%%%%%%%%%%%%%%%%%%%%%%%%%%%%%%%%%%%%%%%%%%%
\subsection{Included Files}
\label{sec:include}

%%%%%%%%%%%%%%%%%%%%%%%%%%%%%%%%%%%%%%%%
\DescribeMacro{\childdocmain}
To use the package, add the commands
\begin{center}
\begin{tabular}{l}
|\input{childdoc.def}|\\
|\childdocmain{}|\\
\end{tabular}
\end{center}
at the very top of the main \LaTeX{} file,
in particular \emph{before} the |\documentclass| statement!
The argument of |\childdocmain| should be left empty
(but it must be present).

%%%%%%%%%%%%%%%%%%%%%%%%%%%%%%%%%%%%%%%%
\DescribeMacro{\childdocof}
Furthermore, add the commands
\begin{center}
\begin{tabular}{l}
|\input{childdoc.def}|\\
|\childdocof{|\textit{main}|}|\\
\end{tabular}
\end{center}
at the top of every child file \textit{child}
which is included by |\include{|\textit{child}|}|
from within the main file
(or at least for those files to be compiled individually).
The argument \textit{main} must be the filename of the main file.

There are a couple of
considerations in setting up the main and child documents:

%%%%%%%%%%%%%%%%%%%%%%%%%%%%%%%%%%%%%%%%
\paragraph{Restrictions.}

Please note the following restrictions:
\begin{itemize}
\item
|\childdocmain| must be called with one argument \textit{main}
to ensure compatibility with earlier version of the package.
It must either be empty (|\childdocmain{}|)
or precisely match the filename of the main file in which it is specified.
See \secref{sec:detection} for further information.
\item
The filename \textit{main} must be specified without the |.tex| extension.
\item
The filename \textit{main} is case sensitive
(even in case-insensitive file systems)
due to internal string comparison.
\item
The argument \textit{main} should be fully expanded, it cannot be a macro.
\item
Subdirectories and special characters should be avoided in filenames.
\item
The command |\childdocmain{|\textit{main}|}| must be followed by a whitespace.
It should not be followed immediately by another command
or by a comment mark `|%|'.
This is because the \TeX{} parser reads the token immediately following
the argument of |\childdocmain| and puts it
at the beginning of every child section;
however, a white\-space is ignored.
\end{itemize}

%%%%%%%%%%%%%%%%%%%%%%%%%%%%%%%%%%%%%%%%
\paragraph{Content of Main File.}

It is advisable to place all content in the child files included by |\include|.
Any output contained in the main file will appear in all child documents
unless suppressed manually;
it cannot be suppressed automatically by the |\includeonly| directive
and thus should normally be avoided.
A method to include some content in the main file
by means of conditional processing is described in \secref{sec:conditional}.

%%%%%%%%%%%%%%%%%%%%%%%%%%%%%%%%%%%%%%%%
\paragraph{Page Numbering.}

When only a part of the document is compiled,
the appropriate numbering of pages
(as well as other status parameters)
is determined from the |.aux| files.
The latter contain information from previous passes.
However this information needs to propagate through
all intermediate child documents.
Therefore the page numbering in child documents may well
be inconsistent until the complete document is compiled at least once.

A useful (if unconventional) way to always ensure a consistent
page numbering is to restart the numbering in each child document
and denote the pages by `\textit{child}|.|\textit{page}'
where \textit{child} represents the chapter/section number of the child file.
This can be achieved by the command
|\numberwithin{page}{|\textit{child}|}|
of the \textsf{amsmath} package
where \textit{child} can be |chapter| or |section|
depending on the chosen structuring.
Alternatively, one can modify the macro |\thepage| appropriately
and reset the counter |page| at the start of each child file.

%%%%%%%%%%%%%%%%%%%%%%%%%%%%%%%%%%%%%%%%%%%%%%%%%%%%%%%%%%%%%%%%%%%%%%%%%%%%%%%%
\subsection{Conditional Processing}
\label{sec:conditional}

The package provides a mechanism to compile different versions
of a document. To customise the versions further some conditional processing
can come in handy to distinguish which version is being compiled.
The package provides two macros to describe the compilation context:

%%%%%%%%%%%%%%%%%%%%%%%%%%%%%%%%%%%%%%%%
\DescribeMacro{\ifchilddoc}
The conditional |\ifchilddoc| distinguishes between the compilation of
child documents and the main document:
%
\begin{center}
|\ifchilddoc |\textit{child-code}| |[|\||else |\textit{main-code}]| \||fi|
\end{center}

%%%%%%%%%%%%%%%%%%%%%%%%%%%%%%%%%%%%%%%%
\DescribeMacro{\childdocname}
\DescribeMacro{\childdocjob}
The macro |\childdocname| contains the filename (without extension)
of the main or child file being processed.
Note that |\childdocjob| will always contain the name of the main file.

%%%%%%%%%%%%%%%%%%%%%%%%%%%%%%%%%%%%%%%%
\paragraph{Title Page.}

Conditional processing can be used to include a title or banner page
in the main document when proper precautions are taken.
Importantly, the code in the main file should ensure that the page counter
(as well as other status parameters which are stored in the |.aux| files)
takes the same value after the conditional processing.
Otherwise the page numbers may take divergent values
depending on which part is compiled.

For example, a title page could be declared by:
%
\begin{center}
\begin{tabular}{l}
|\ifchilddoc\||else|\\
|\addtocounter{page}{-1}|\\
\textit{code for title page}\\
|\newpage|\\
|\||fi|
\end{tabular}
\end{center}
%
A banner page for the child documents can be generated by:
%
\begin{center}
\begin{tabular}{l}
|\ifchilddoc|\\
|\addtocounter{page}{-1}|\\
\textit{code for banner page}\\
|\newpage|\\
|\||fi|
\end{tabular}
\end{center}
%
Here one could write a message such as:
\begin{center}
|This is the part \childdocname{} of \childdocjob{}.|
\end{center}

%%%%%%%%%%%%%%%%%%%%%%%%%%%%%%%%%%%%%%%%%%%%%%%%%%%%%%%%%%%%%%%%%%%%%%%%%%%%%%%%
\subsection{Flags}
\label{sec:flags}

The package makes it easy to generate different versions
of the main or child documents.
To this end compilation flags can be defined
and assigned different default values.
They will be particularly useful in conjunction
with the forwarding mechanism described in \secref{sec:forward}.

For example, it may be useful to have a flag |\version|
which can be set to |draft| or |final|.
The document source will contain some conditional code
depending on the value of |\version|.
Suppose further, the flag should default to |final| for the main file
and to |draft| for child files
which is a natural assignment for editing the document.
This is achieved by placing the following code
in the preamble of the main document
(below the |\childdocmain| directive):
%
\begin{center}
\begin{tabular}{l}
|\ifchilddoc|\\
|\providecommand{\version}{draft}|\\
|\||else|\\
|\providecommand{\version}{final}|\\
|\||fi|
\end{tabular}
\end{center}
%
The definition by |\providecommand| makes sure
that previous definitions are not overwritten.
Further statements |\providecommand{\version}{...}|
can thus be added before the above code to override it.

For the main file, one might add a line
(between |\childdocmain| and the above block)
%
\begin{center}
|%\ifchilddoc\||else\providecommand{\version}{draft}\||fi|
\end{center}
%
which can be uncommented to produce a draft version.
Likewise one can add a line to the very top of a child file
(above the |\childdocof{|\textit{main}|}| directive)
%
\begin{center}
|%\providecommand{\version}{final}|
\end{center}
%
which can be uncommented to produce the final version of this child document.

%%%%%%%%%%%%%%%%%%%%%%%%%%%%%%%%%%%%%%%%%%%%%%%%%%%%%%%%%%%%%%%%%%%%%%%%%%%%%%%%
\subsection{Forwarding}
\label{sec:forward}

Different versions of the main or child documents
using compilation flags as described in \secref{sec:flags}
can be (permanently) stored in different files
for convenient compilation, viewing and distribution.
To this end, the package defines a command
to pass on compilation to a different file:

%%%%%%%%%%%%%%%%%%%%%%%%%%%%%%%%%%%%%%%%
\DescribeMacro{\childdocforward}
The command |\childdocforward| redirects processing to
another source file:
%
\begin{center}
\begin{tabular}{l}
|\input{childdoc.def}|\\
|\childdocforward[|\textit{main}|]{|\textit{dest}|}|\\
\end{tabular}
\end{center}
%
The argument \textit{dest} is the destination file
(without extension).
It should be the main file or one of the child files.
Note that further \textsf{childdoc} directives
such as |\childdocof| and |\childdocforward|
in the indicated file will be processed in this form.
The optional argument \textit{main}
passes on directly to the main file \textit{main}
while pretending to compile the child \textit{dest}.
This form behaves as if \textit{dest}
issues |\childdocof{|\textit{main}|}| right away,
and no further \textsf{childdoc} directives will be processed.

%%%%%%%%%%%%%%%%%%%%%%%%%%%%%%%%%%%%%%%%
\DescribeMacro{\...prefix}
In the alternative form |\childdocforwardprefix|,
%
\begin{center}
\begin{tabular}{l}
|\input{childdoc.def}|\\
|\childdocforwardprefix[|\textit{main}|]{|\textit{prefix}|}{|\textit{dest}|}|
\end{tabular}
\end{center}
%
the destination file is determined by a pattern
depending on the current file:
To make this work, the current file must be called
`{\textit{prefix}\hspace{0.2em}\textit{suffix}}'
with \textit{prefix} matching precisely the argument.
Processing is then passed on to the file
`{\textit{dest}\hspace{0.2em}\textit{suffix}}'.
Surely, the same effect is achieved by
directly specifying the
argument `{\textit{dest}\hspace{0.2em}\textit{suffix}}'
in the first form.
However, that requires to set up a different file
for each child. With the alternative form of the command
all these files can have exactly the same content
which simplifies setting them up and maintaining them.

For example, the following file |draft.tex|
with a compilation flag |\version| as described in \secref{sec:flags}
compiles the main document as a draft:
%
\begin{center}
\begin{tabular}{l}
|\def\version{draft}|\\
|\input{childdoc.def}|\\
|\childdocforward{|\textit{main}|}|
\end{tabular}
\end{center}
%
Likewise, the following files |final|\textit{nn}|.tex|
compile the final version of the child document
|child|\textit{nn}|.tex|:
%
\begin{center}
\begin{tabular}{l}
|\def\version{final}|\\
|\input{childdoc.def}|\\
|\childdocforwardprefix{final}{child}|
\end{tabular}
\end{center}
%

Note that when several versions of a main file and/or of each child file
are to be generated, it may be convenient to set up a |Makefile| or
shell script to automatise the process.

%%%%%%%%%%%%%%%%%%%%%%%%%%%%%%%%%%%%%%%%%%%%%%%%%%%%%%%%%%%%%%%%%%%%%%%%%%%%%%%%
\subsection{Command Line Processing}
\label{sec:commandline}

The effect of redirection files can also be achieved by invoking
the \LaTeX{} compiler with a more elaborate command line.
Most conveniently this should be done as part
of a shell script or a |Makefile|.

When using \textsf{childdoc} in the main file, the following
command lines effectively perform a redirection
(note that depending on the shell being used,
backslashes may have to be doubled: `|\|' $\to$ `|\\|'):
%
\begin{center}
|... -jobname "|\textit{target}|" |\\|"|[\textit{flags}]%
|\input{childdoc.def}\childdocforward[|\textit{main}|]{|\textit{dest}|}"|
\end{center}
%
Here \textit{target} is the name of the output file,
\textit{main} is the name of the main file
and \textit{dest} is the name of the main or child file to be processed
(all filenames without extensions).
The optional argument \textit{main} can be omitted
if \textit{main} matches \textit{dest}.
Optionally, compilation \textit{flags} can be defined via |\def| commands.
This command line makes the \TeX{} engine believe
it is compiling the file \textit{target}
whose content is specified as the latter parameter.
The provided code then forwards the processing to
\textit{main} or \textit{dest} as described in \secref{sec:forward}.

%%%%%%%%%%%%%%%%%%%%%%%%%%%%%%%%%%%%%%%%%%%%%%%%%%%%%%%%%%%%%%%%%%%%%%%%%%%%%%%%
\subsection{Include by Input}
\label{sec:input}

Including child documents by |\include| has some restrictions by design.
Most notably, the content of a child document always occupies
its own set of pages; pages cannot be shared between child documents.
Usually, this behaviour makes perfect sense
because each child document contain an essential part of the document.
However, in some situations it may be desirable to compose
a document from a collection of parts
without having mandatory page breaks between then.
For this case, the package
provides a mechanism to include parts
by |\input| which can also be processed individually.
However, by construction this mechanism
requires manual handling of the content to be output.

%%%%%%%%%%%%%%%%%%%%%%%%%%%%%%%%%%%%%%%%
\DescribeMacro{\ifchilddocmanual}
The main file should be prepared as usual, see \secref{sec:include}.
However, the document body must make a distinction
between processing of an individual part and of the main document, e.g.:
%
\begin{center}
\begin{tabular}{l}
|\ifchilddocmanual|\\
|\input{\childdocname}|\\
|\||else|\\
\textit{document body with }|\input{|\textit{part}|}|\\
|\||fi|
\end{tabular}
\end{center}
%
The conditional |\ifchilddocmanual| is true whenever
a part to be included by |\input| is being compiled,
and the name of the part is stored in |\childdocname|.

%%%%%%%%%%%%%%%%%%%%%%%%%%%%%%%%%%%%%%%%
\DescribeMacro{\childdocby}
Each part to be included by |\input| should start with:
%
\begin{center}
\begin{tabular}{l}
|\input{childdoc.def}|\\
|\childdocby{|\textit{main}|}|\\
\end{tabular}
\end{center}
%
The directive |\childdocby| is similar to |\childdocof|
described in \secref{sec:include},
but the subsequent selection of content must be done manually.
To that end, both |\ifchilddoc| and |\ifchilddocmanual|
will be true upon processing of a part,
and the name of the part is stored in |\childdocname|.
Note that |\jobname| will be set to the filename of the current part
so that each part receives an individual |.aux| file
that does not interfere with the |.aux| file(s) of the main document.
This behaviour can be altered by the alternative form
|\childdocby[*]{|\textit{main}|}| (with a non-empty optional argument)
which uses the |.aux| file of the main document
by setting |\jobname| to \textit{main}.

%%%%%%%%%%%%%%%%%%%%%%%%%%%%%%%%%%%%%%%%%%%%%%%%%%%%%%%%%%%%%%%%%%%%%%%%%%%%%%%%
\subsection{Driver Development}
\label{sec:driver}

The \textsf{childdoc} mechanism can also be use for the development
of definition files such as \LaTeX{} styles or classes.
This case differs from the above setup with multiple parts
included by |\include| in that no |\includeonly| should be invoked.
This can be achieved by starting the include file
(before |\ProvidesPackage|) with:
%
\begin{center}
\begin{tabular}{l}
|\input{childdoc.def}|\\
|\childdocforward{|\textit{main}|}|\\
\end{tabular}
\end{center}
%
or alternatively with:
%
\begin{center}
\begin{tabular}{l}
|\input{childdoc.def}|\\
|\childdocby{|\textit{main}|}|\\
\end{tabular}
\end{center}
%
Both forms have slightly different effects as described above.
The main file is prepared as usual, see \secref{sec:include}.

%%%%%%%%%%%%%%%%%%%%%%%%%%%%%%%%%%%%%%%%%%%%%%%%%%%%%%%%%%%%%%%%%%%%%%%%%%%%%%%%
\subsection{Legacy Detection}
\label{sec:detection}

The directive |\childdocmain| in the main file can detect
whether the complete document or merely a child is to be compiled
even without using the directive |\childdocof|.
This method is deprecated because it is less robust
and there is no compelling reason to use it;
it is merely provided for backward compatibility
and it may be removed in future versions.

If the detection mechanism is to be used,
it is mandatory to correctly specify
the filename of the main file as the argument of |\childdocmain|:
%
\begin{center}
\begin{tabular}{l}
|\input{childdoc.def}|\\
|\childdocmain{|\textit{main}|}|\\
\end{tabular}
\end{center}
%
If |\jobname| does not match the argument \textit{main} of |\childdocmain|,
it is assumed that |\jobname| points to the child file to be compiled.
When using |\childdocmain| with the main file specified as argument,
it suffices to start a child file
with just |\input{|\textit{main}|}|
without loading of the package and using |\childdocof|.
If instead all processing is done
with the appropriate \textsf{childdoc} directives,
the argument of \textit{main} of |\childdocmain| can be empty.

An alternative version of the command line processing described
in \secref{sec:commandline} using the detection mechanism reads:
%
\begin{center}
|... -jobname "|\textit{target}|" "|[\textit{flags}]%
[|\def\jobname{|\textit{dest}|}|]|\input{|\textit{main}|}"|
\end{center}

%%%%%%%%%%%%%%%%%%%%%%%%%%%%%%%%%%%%%%%%%%%%%%%%%%%%%%%%%%%%%%%%%%%%%%%%%%%%%%%%
\subsection{Manual Code}
\label{sec:manual}

In case one cannot be certain whether the definitions file |childdoc.def|
is installed on the target \TeX{} distribution
and one prefers not to ship it,
it is conceivable to paste a few relevant commands into the sources.

To that end, drop all statements |\input{childdoc.def}|
and perform the replacements as outlined below.
Instead of |\childdocmain{|\textit{main}|}| add the following code
to the top of the main file:
%
\begin{center}
\begin{tabular}{l}
|\||ifdefined\childdocname\endinput\||fi\newif\ifchilddoc|\\
|\edef\childdocname{\scantokens\expandafter{\jobname\noexpand}}|\\
|\def\childdocmain{|\textit{main}|}\||ifx\childdocmain\childdocname\||else|\\
|\childdoctrue\includeonly{\childdocname}\let\jobname\childdocmain\||fi|\\
\end{tabular}
\end{center}
%
Instead of |\childdocof{|\textit{main}|}| just include the main file
at the top of each child file:
%
\begin{center}
|\input{|\textit{main}|}|
\end{center}
%
A simple redirection |\childdocforward{|\textit{dest}|}| is achieved by:
%
\begin{center}
|\def\jobname{|\textit{dest}|}\input{\jobname}|
\end{center}
%
The redirection with prefix
|\childdocforwardprefix[|\textit{prefix}|]{|\textit{dest}|}|
is accomplished by:
%
\begin{center}
\begin{tabular}{l}
|{\edef\jobname{\scantokens\expandafter{\jobname\noexpand}}|\\
|\def\redirectjob |\textit{prefix}|#1~~~{\gdef\jobname{|\textit{dest}|#1}}|\\
|\expandafter\redirectjob\jobname~~~}\input{\jobname}|
\end{tabular}
\end{center}

In an alternative approach,
child documents can be compiled by a specific command line
without additional code or specific definitions:
%
\begin{center}
|... -jobname "|\textit{target}|" "|[\textit{flags}]%
|\includeonly{|\textit{dest}|}\input{|\textit{main}|}"|
\end{center}
%

%%%%%%%%%%%%%%%%%%%%%%%%%%%%%%%%%%%%%%%%%%%%%%%%%%%%%%%%%%%%%%%%%%%%%%%%%%%%%%%%
%%%%%%%%%%%%%%%%%%%%%%%%%%%%%%%%%%%%%%%%%%%%%%%%%%%%%%%%%%%%%%%%%%%%%%%%%%%%%%%%
\section{Information}

%%%%%%%%%%%%%%%%%%%%%%%%%%%%%%%%%%%%%%%%%%%%%%%%%%%%%%%%%%%%%%%%%%%%%%%%%%%%%%%%
\subsection{Copyright}

Copyright \copyright{} 2017--2018 Niklas Beisert

This work may be distributed and/or modified under the
conditions of the \LaTeX{} Project Public License, either version 1.3
of this license or (at your option) any later version.
The latest version of this license is in
  \url{http://www.latex-project.org/lppl.txt}
and version 1.3 or later is part of all distributions of \LaTeX{}
version 2005/12/01 or later.

This work has the LPPL maintenance status `maintained'.

The Current Maintainer of this work is Niklas Beisert.

This work consists of the files |README.txt|, |childdoc.ins| and |childdoc.dtx|
as well as the derived files |childdoc.def|, |cdocsamp.tex|
with |cdocsch1.tex|, |cdocsch2.tex|, |cdocspt3.tex|, |cdocspt4.tex|,
|cdocsdrf.tex|, |cdocsfn1.tex|, |cdocsfn2.tex|
as well as |childdoc.pdf|.

%%%%%%%%%%%%%%%%%%%%%%%%%%%%%%%%%%%%%%%%%%%%%%%%%%%%%%%%%%%%%%%%%%%%%%%%%%%%%%%%
\subsection{Files and Installation}

The package consists of the files:
%
\begin{center}
\begin{tabular}{ll}
    |README.txt|   & readme file \\
    |childdoc.ins| & installation file \\
    |childdoc.dtx| & source file \\
    |childdoc.def| & definition file \\
    |cdocsamp.tex| & sample main file \\
    |cdocsch1.tex| & sample include file \\
    |cdocsch2.tex| & sample include file \\
    |cdocspt3.tex| & sample part file \\
    |cdocspt4.tex| & sample part file \\
    |cdocsdrf.tex| & sample redirection file \\
    |cdocsfn1.tex| & sample redirection file \\
    |cdocsfn2.tex| & sample redirection file \\
    |childdoc.pdf| & manual
\end{tabular}
\end{center}
%
The distribution consists of the files
|README.txt|, |childdoc.ins| and |childdoc.dtx|.
%
\begin{itemize}
\item
Run (pdf)\LaTeX{} on |childdoc.dtx|
to compile the manual |childdoc.pdf| (this file).
\item
Run \LaTeX{} on |childdoc.ins| to create the definitions file |childdoc.def|
and the sample |cdocsamp.tex| with include files
|cdocsch1.tex|, |cdocsch2.tex|, |cdocspt3.tex|, |cdocspt4.tex|,
|cdocsdrf.tex|, |cdocsfn1.tex|, |cdocsfn2.tex|.
Then copy the file |childdoc.def| to an appropriate directory of your \LaTeX{}
distribution, e.g.\ \textit{texmf-root}|/tex/latex/childdoc|.
\end{itemize}

%%%%%%%%%%%%%%%%%%%%%%%%%%%%%%%%%%%%%%%%%%%%%%%%%%%%%%%%%%%%%%%%%%%%%%%%%%%%%%%%
\subsection{Related CTAN Packages}

There are several other packages which offer a similar functionality:
%
\begin{itemize}
\item
The packages
\href{http://ctan.org/pkg/docmute}{\textsf{docmute}},
\href{http://ctan.org/pkg/includex}{\textsf{includex}} and
\href{http://ctan.org/pkg/standalone}{\textsf{standalone}}
provide commands to include only the document body of
a child file thus allowing both files to be compiled individually.
\item
The packages \href{http://ctan.org/pkg/subdocs}{\textsf{subdocs}}
and \href{http://ctan.org/pkg/subfiles}{\textsf{subfiles}}
provide structures in which the main and child documents can be
encapsulated and allowing them to be compiled individually.
The inclusion mechanism is different from the conventional |\include|.
\item
The package \href{http://ctan.org/pkg/combine}{\textsf{combine}}
is an elaborate solution to combine several documents into one.
\end{itemize}
%
See also the CTAN topic \href{http://ctan.org/topic/subdocs}{\textsf{subdocs}}
for further related packages.
The present package differs from the above solutions in that
a document structure constructed with the conventional |\include| mechanism
just needs two extra commands at the top of every file
such that all constituent files can be compiled individually.

%%%%%%%%%%%%%%%%%%%%%%%%%%%%%%%%%%%%%%%%%%%%%%%%%%%%%%%%%%%%%%%%%%%%%%%%%%%%%%%%
%\subsection{Feature Suggestions}
%
%The following is a list of features which may be useful for future
%versions of this package:
%%
%\begin{itemize}
%\item
%\ldots
%\end{itemize}

%%%%%%%%%%%%%%%%%%%%%%%%%%%%%%%%%%%%%%%%%%%%%%%%%%%%%%%%%%%%%%%%%%%%%%%%%%%%%%%%
\subsection{Revision History}

%%%%%%%%%%%%%%%%%%%%%%%%%%%%%%%%%%%%%%%%
\paragraph{v2.0:} 2018/12/30

\begin{itemize}
\item
immediate forward processing
\item
added |\childdocby| mechanism
\item
manual restructured
\end{itemize}

%%%%%%%%%%%%%%%%%%%%%%%%%%%%%%%%%%%%%%%%
\paragraph{v1.6:} 2018/01/17

\begin{itemize}
\item
application for development of include files
\item
corrections to manual
\end{itemize}

%%%%%%%%%%%%%%%%%%%%%%%%%%%%%%%%%%%%%%%%
\paragraph{v1.5:} 2017/05/21

\begin{itemize}
\item
more complete structuring introduced
\item
|\childdocof| introduced
\item
|\childdoc| renamed to |\childdocmain|
\item
|\childredirect| renamed to |\childdocforward| and |\childdocforwardprefix|
and functionality expanded
\end{itemize}

%%%%%%%%%%%%%%%%%%%%%%%%%%%%%%%%%%%%%%%%
\paragraph{v1.0:} 2017/04/27

\begin{itemize}
\item
manual and install package
\item
first version published on CTAN
\end{itemize}

%%%%%%%%%%%%%%%%%%%%%%%%%%%%%%%%%%%%%%%%
\paragraph{v0.6:} 2017/04/26

\begin{itemize}
\item
redirection mechanism added
\end{itemize}

%%%%%%%%%%%%%%%%%%%%%%%%%%%%%%%%%%%%%%%%
\paragraph{v0.5:} 2017/04/26

\begin{itemize}
\item
functionality in definition file
\end{itemize}


%%%%%%%%%%%%%%%%%%%%%%%%%%%%%%%%%%%%%%%%%%%%%%%%%%%%%%%%%%%%%%%%%%%%%%%%%%%%%%%%
%%%%%%%%%%%%%%%%%%%%%%%%%%%%%%%%%%%%%%%%%%%%%%%%%%%%%%%%%%%%%%%%%%%%%%%%%%%%%%%%
%%%%%%%%%%%%%%%%%%%%%%%%%%%%%%%%%%%%%%%%%%%%%%%%%%%%%%%%%%%%%%%%%%%%%%%%%%%%%%%%
\appendix

\settowidth\MacroIndent{\rmfamily\scriptsize 000\ }

 \DocInput{childdoc.dtx}

\end{document}
%</driver>
% \fi
%
% %%%%%%%%%%%%%%%%%%%%%%%%%%%%%%%%%%%%%%%%%%%%%%%%%%%%%%%%%%%%%%%%%%%%%%%%%%%%%%
% %%%%%%%%%%%%%%%%%%%%%%%%%%%%%%%%%%%%%%%%%%%%%%%%%%%%%%%%%%%%%%%%%%%%%%%%%%%%%%
% \section{Sample}
%\iffalse
%<*samplemain>
%\fi
%
% The following presents a sample document
% with two chapters, two parts, a title page,
% a compile flag as well as three forwarding files to set the flag.
% It consists of eight |.tex| files:
% \begin{center}
% \begin{tabular}{ll}
% |cdocsamp.tex|&main file\\
% |cdocsch1.tex|&include file for chapter 1\\
% |cdocsch2.tex|&include file for chapter 2\\
% |cdocspt3.tex|&include file for part 3\\
% |cdocspt4.tex|&include file for part 4\\
% |cdocsdrf.tex|&forwarding file for main file in draft mode\\
% |cdocsfi1.tex|&forwarding file for final version of chapter 1\\
% |cdocsfi2.tex|&forwarding file for final version of chapter 2\\
% \end{tabular}
% \end{center}
% Each of the eight files can be compiled directly by the \LaTeX{} compiler.
%
% %%%%%%%%%%%%%%%%%%%%%%%%%%%%%%%%%%%%%%
% \paragraph{Main File.}
%
% The main file is called |cdocsamp.tex|.
%
% Load the \textsf{childdoc} definitions and
% declare the filename for the main document:
%    \begin{macrocode}
\input{childdoc.def}
\childdocmain{}
%    \end{macrocode}

% Optional override for |\version| flag:
%    \begin{macrocode}
%%\ifchilddoc\else\providecommand{\version}{draft}\fi
%    \end{macrocode}

% Define the default values for the |\version| flag
% (|final| for the main file and |draft| for childs):
%    \begin{macrocode}
\ifchilddoc
\providecommand{\version}{draft}
\else
\providecommand{\version}{final}
\fi
%    \end{macrocode}

% Load the standard document class:
%    \begin{macrocode}
\documentclass[12pt]{article}
%    \end{macrocode}

% Start the document body:
%    \begin{macrocode}
\begin{document}
%    \end{macrocode}

% Declare a title page.
% Print title, part of document being processed and version flag:
%    \begin{macrocode}
\addtocounter{page}{-1}
\begin{center}
{\LARGE\bfseries{}childdoc example\par}
\vspace{1cm}
\ifchilddoc
\ifchilddocmanual part\else chapter\fi:
`\childdocname' of `\childdocjob'\par
\else
main document: `\childdocjob'\par
\fi
version: \version\par
\end{center}
\newpage
%    \end{macrocode}

% Manually include selected file,
% otherwise process as usual:
%    \begin{macrocode}
\ifchilddocmanual
\section*{part `\childdocname'}
\input{\childdocname}
\else
%    \end{macrocode}

% Include the two chapters:
%    \begin{macrocode}
\include{cdocsch1}
\include{cdocsch2}
%    \end{macrocode}

% Include the two parts unless only chapters should be displayed:
%    \begin{macrocode}
\ifchilddoc\else
\section{part three}
\input{cdocspt3}
\section{part four}
\input{cdocspt4}
\fi
%    \end{macrocode}

% Process as usual until here:
%    \begin{macrocode}
\fi
%    \end{macrocode}

% End of document body:
%    \begin{macrocode}
\end{document}
%    \end{macrocode}
%\iffalse
%</samplemain>
%\fi
%
% %%%%%%%%%%%%%%%%%%%%%%%%%%%%%%%%%%%%%%
% \paragraph{Chapter Include Files.}
%
% The include files are called |cdocsch1.tex| and |cdocsch2.tex|.
%
%\iffalse
%<*samplechap1|samplechap2>
%\fi

% Optional override for |\version| flag:
%    \begin{macrocode}
%%\providecommand{\version}{final}
%    \end{macrocode}

% Include the main document:
%    \begin{macrocode}
\input{childdoc.def}
\childdocof{cdocsamp}
%    \end{macrocode}

%\iffalse
%</samplechap1|samplechap2>
%\fi
%
%\iffalse
%<*samplechap1>
%\fi
% Some text for chapter 1:
%    \begin{macrocode}
\section{one}
some text in chapter one
%    \end{macrocode}

%\iffalse
%</samplechap1>
%\fi
% Some text for chapter 2:
%\iffalse
%<*samplechap2>
%\fi
%    \begin{macrocode}
\section{two}
more text in chapter two
%    \end{macrocode}

%\iffalse
%</samplechap2>
%\fi
%
% %%%%%%%%%%%%%%%%%%%%%%%%%%%%%%%%%%%%%%
% \paragraph{Part Include Files.}
%
% The include files are called |cdocspt3.tex| and |cdocspt4.tex|.
%
%\iffalse
%<*samplepart3|samplepart4>
%\fi

% Optional override for |\version| flag:
%    \begin{macrocode}
%%\providecommand{\version}{final}
%    \end{macrocode}

% Include the main document:
%    \begin{macrocode}
\input{childdoc.def}
\childdocby{cdocsamp}
%    \end{macrocode}

%\iffalse
%</samplepart3|samplepart4>
%\fi
%
%\iffalse
%<*samplepart3>
%\fi
% Some text for part 3:
%    \begin{macrocode}
some text in part three
%    \end{macrocode}

%\iffalse
%</samplepart3>
%\fi
% Some text for part 4:
%\iffalse
%<*samplepart4>
%\fi
%    \begin{macrocode}
more text in part four
%    \end{macrocode}

%\iffalse
%</samplepart4>
%\fi
%
% %%%%%%%%%%%%%%%%%%%%%%%%%%%%%%%%%%%%%%
% \paragraph{Forwarding for a Complete Draft.}
%
% The following forwarding file |cdocsdrf.tex|
% compiles the main document in draft mode:
%\iffalse
%<*sampledraft>
%\fi
%    \begin{macrocode}
\def\version{draft}
\input{childdoc.def}
\childdocforward{cdocsamp}
%    \end{macrocode}

%\iffalse
%</sampledraft>
%\fi
%
% %%%%%%%%%%%%%%%%%%%%%%%%%%%%%%%%%%%%%%
% \paragraph{Forwarding for Final Version of the Chapters.}
%
% The following forwarding files |cdocsfn1.tex| and |cdocsfn2.tex|
% (with identical content)
% compile the final versions of the child documents
% |cdocsch1.tex| and |cdocsch2.tex|, respectively:
%\iffalse
%<*samplefinal>
%\fi
%    \begin{macrocode}
\def\version{final}
\input{childdoc.def}
\childdocforwardprefix[cdocsamp]{cdocsfn}{cdocsch}
%    \end{macrocode}

%\iffalse
%</samplefinal>
%\fi
%
% %%%%%%%%%%%%%%%%%%%%%%%%%%%%%%%%%%%%%%
% \paragraph{Command Line Processing.}
%
% The following three command lines generate the output files
% |cdocscld|, |cdocscl1| and |cdocscl2|
% which should be identical to
% |cdocsdrf|, |cdocsch1| and |cdocsfn2|, respectively:
% \begin{center}
% \begin{tabular}{l}
% |latex -jobname cdocscld \|\\
% |  "\def\version{draft}\input{childdoc.def}\childdocforward{cdocsamp}"|\\
% |latex -jobname cdocscl1 \|\\
% |  "\input{childdoc.def}\childdocforward[cdocsamp]{cdocsch1}"|\\
% |latex -jobname cdocscl2 \|\\
% |  "\def\version{final}\input{childdoc.def}\childdocforward{cdocsch2}"|
% \end{tabular}
% \end{center}
% Note that the trailing backslash on each first line
% merely continues the input to the second line
% (for convenient cut ant paste).
% Furthermore, the command |latex| can be replaced by any
% of its alternative versions such as |pdflatex|.
%
% %%%%%%%%%%%%%%%%%%%%%%%%%%%%%%%%%%%%%%%%%%%%%%%%%%%%%%%%%%%%%%%%%%%%%%%%%%%%%%
% %%%%%%%%%%%%%%%%%%%%%%%%%%%%%%%%%%%%%%%%%%%%%%%%%%%%%%%%%%%%%%%%%%%%%%%%%%%%%%
% \section{Implementation}
%\iffalse
%<*package>
%\fi
%
% This section describes the definitions file |childdoc.def|.

% The definitions cannot be loaded using |\usepackage| or |\RequirePackage|
% which has a mechanism to prevent loading a style file more than once.
% When loading the definitions by means of |\input|
% multiple instances have to be prevented manually:
%\iffalse
%This code needs to be before the `\ProvidesFile' directive
%which is defined at the beginning of this file.
%Therefore it is also placed there and commented out here.
%</package>
%<*discard>
%\fi
%    \begin{macrocode}
\ifdefined\childdocmain\endinput\fi
%    \end{macrocode}
%\iffalse
%</discard>
%<*package>
%\fi
%
% \macro{\ifchilddoc}
% \macro{\ifchilddocmanual}
% The conditional |\ifchilddoc| tells whether a
% child (true) or main (false) document is being compiled.
% The conditional |\ifchilddocmanual| tells whether
% the |\includeonly| mechanism is used (false) or
% the selection of child files must be performed manually (true).
% The definitions initialise to false:
%    \begin{macrocode}
\newif\ifchilddoc
\newif\ifchilddocmanual
%    \end{macrocode}

% \macro{\childdocname}
% \macro{\childdocjob}
% The macro |\childdocname| stores the name of the main document
% to be compiled. The macro |\childdocjob| stores the name of
% the document on which the \LaTeX{} compiler was originally invoked.
% The content of |\jobname| cannot be compared
% to filenames specified in the source due to different catcodes.
% The following code rescans |\jobname|, stores the result
% in |\childdocname| and saves a copy in |\childdocjob|:
%    \begin{macrocode}
\edef\childdocname{\scantokens\expandafter{\jobname\noexpand}}
\let\childdocjob\childdocname
%    \end{macrocode}

% \macro{\childdocdisable}
% The macro |\childdocdisable| prevents the main file
% from being processed more than once.
% At this stage, the main document command |\childdocmain|
% is assumed to be called once again where it should do nothing.
% Any subsequent call to it should prevent
% a secondary processing of the main document
% It overwrites the forwarding commands
% |\childdocof| and |\childdocforward|
% with empty macros to prevent further inclusions of the main document:
%    \begin{macrocode}
\newcommand{\childdocdisable}
{
  \renewcommand{\childdocmain}[1]{\renewcommand{\childdocmain}[1]{\endinput}}
  \renewcommand{\childdocof}[1]{}
  \renewcommand{\childdocby}[2][]{}
  \renewcommand{\childdocforward}[2][]{}
  \renewcommand{\childdocdisable}{}
}
%    \end{macrocode}

% \macro{\childdocmain}
% The macro |\childdocmain| is to be called at the top of the main file
% with nothing or the main filename (without extension) as argument.
% First, it breaks loops.
% If the argument is not empty and does not match |\childdocname|
% (which is set by the first inclusion of |childdoc.def|),
% |\ifchilddoc| is set to true, |\includeonly| is applied to the child file
% and |\jobname| is set to the main file
% (for proper handling of |.aux| files):
%    \begin{macrocode}
\newcommand{\childdocmain}[1]
{
  \childdocdisable\childdocmain{}
  \if?#1?\else
    \begingroup
      \def\childdoctmp{#1}
      \ifx\childdoctmp\childdocname
        \def\childdoctmp{}
      \else
        \def\childdoctmp
        {
          \childdoctrue
          \includeonly{\childdocname}
          \def\childdocjob{#1}
          \def\jobname{#1}
        }
      \fi
      \expandafter
    \endgroup
    \childdoctmp
  \fi
}
%    \end{macrocode}

% \macro{\childdocof}
% The command |\childdocof| redirects
% compilation to the main file |#1|.
%    \begin{macrocode}
\newcommand{\childdocof}[1]
{
  \childdocdisable
  \childdoctrue
  \includeonly{\childdocname}
  \def\jobname{#1}
  \def\childdocjob{#1}
  \input{#1}
}
%    \end{macrocode}

% \macro{\childdocby}
% The command |\childdocby| ....
%    \begin{macrocode}
\newcommand{\childdocby}[2][]
{
  \childdocdisable
  \childdoctrue
  \childdocmanualtrue
  \if?#1?\else
    \def\jobname{#2}
  \fi
  \def\childdocjob{#2}
  \input{#2}
  \endinput
}
%    \end{macrocode}

% \macro{\childdocforward}
% The command |\childdocforward| redirects
% compilation to the main file or
% (if the optional argument is given) a child file.
% Parameters are set as if the main file
% or a child file starting with |\childdocof| was compiled.
% Then compilation is handed over to the main file:
%    \begin{macrocode}
\newcommand{\childdocforward}[2][]
{
  \begingroup
    \if?#1?
      \def\childdoctmp
      {
        \def\childdocname{#2}
        \def\childdocjob{#2}
        \def\jobname{#2}
        \input{#2}
        \endinput
      }
    \else
      \def\childdoctmp
      {
        \childdocdisable
        \def\childdocname{#2}
        \childdoctrue
        \includeonly{#2}
        \def\childdocjob{#1}
        \def\jobname{#1}
        \input{#1}
        \endinput
      }
    \fi
    \expandafter
  \endgroup
  \childdoctmp
}
%    \end{macrocode}

% \macro{\childdocforwardprefix}
% The command |\childdocforwardprefix| redirects
% compilation to the main or a child file by means of a pattern.
% The prefix |#1| in the current filename is replaced by |#2|
% and the suffix of the current filename is kept
% (it is assumed that the filename does not contain the substring `|~~~|'
% which is used as a delimiter).
% Compilation is handed over to the new file by |\childdocforward|:
%    \begin{macrocode}
\newcommand{\childdocforwardprefix}[3][]
{
  \begingroup
    \def\childdocextract #2##1~~~{\def\childdoctmp{\childdocforward[#1]{#3##1}}}
    \expandafter\childdocextract\childdocname~~~
    \expandafter
  \endgroup
  \childdoctmp
}
%    \end{macrocode}

% \macro{\childdoc}
% The deprecated macro |\childdoc| is a legacy version of |\childdocmain|:
%    \begin{macrocode}
\newcommand{\childdoc}{\childdocmain}
%    \end{macrocode}

% \macro{\childdocredirect}
% The deprecated macro |\childdocredirect| is a legacy version
% of |\childdocforward| and |\childdocforwardprefix|:
%    \begin{macrocode}
\newcommand{\childdocredirect}[2][]
{
  \begingroup
    \if?#1?
      \def\childdoctmp{\childdocforward{#2}}
    \else
      \def\childdoctmp{\childdocforwardprefix{#1}{#2}}
    \fi
    \expandafter
  \endgroup
  \childdoctmp
}
%    \end{macrocode}

%\iffalse
%</package>
%\fi
%
\endinput
|
and perform the replacements as outlined below.
Instead of |\childdocmain{|\textit{main}|}| add the following code
to the top of the main file:
%
\begin{center}
\begin{tabular}{l}
|\||ifdefined\childdocname\endinput\||fi\newif\ifchilddoc|\\
|\edef\childdocname{\scantokens\expandafter{\jobname\noexpand}}|\\
|\def\childdocmain{|\textit{main}|}\||ifx\childdocmain\childdocname\||else|\\
|\childdoctrue\includeonly{\childdocname}\let\jobname\childdocmain\||fi|\\
\end{tabular}
\end{center}
%
Instead of |\childdocof{|\textit{main}|}| just include the main file
at the top of each child file:
%
\begin{center}
|\input{|\textit{main}|}|
\end{center}
%
A simple redirection |\childdocforward{|\textit{dest}|}| is achieved by:
%
\begin{center}
|\def\jobname{|\textit{dest}|}\input{\jobname}|
\end{center}
%
The redirection with prefix
|\childdocforwardprefix[|\textit{prefix}|]{|\textit{dest}|}|
is accomplished by:
%
\begin{center}
\begin{tabular}{l}
|{\edef\jobname{\scantokens\expandafter{\jobname\noexpand}}|\\
|\def\redirectjob |\textit{prefix}|#1~~~{\gdef\jobname{|\textit{dest}|#1}}|\\
|\expandafter\redirectjob\jobname~~~}\input{\jobname}|
\end{tabular}
\end{center}

In an alternative approach,
child documents can be compiled by a specific command line
without additional code or specific definitions:
%
\begin{center}
|... -jobname "|\textit{target}|" "|[\textit{flags}]%
|\includeonly{|\textit{dest}|}\input{|\textit{main}|}"|
\end{center}
%

%%%%%%%%%%%%%%%%%%%%%%%%%%%%%%%%%%%%%%%%%%%%%%%%%%%%%%%%%%%%%%%%%%%%%%%%%%%%%%%%
%%%%%%%%%%%%%%%%%%%%%%%%%%%%%%%%%%%%%%%%%%%%%%%%%%%%%%%%%%%%%%%%%%%%%%%%%%%%%%%%
\section{Information}

%%%%%%%%%%%%%%%%%%%%%%%%%%%%%%%%%%%%%%%%%%%%%%%%%%%%%%%%%%%%%%%%%%%%%%%%%%%%%%%%
\subsection{Copyright}

Copyright \copyright{} 2017--2018 Niklas Beisert

This work may be distributed and/or modified under the
conditions of the \LaTeX{} Project Public License, either version 1.3
of this license or (at your option) any later version.
The latest version of this license is in
  \url{http://www.latex-project.org/lppl.txt}
and version 1.3 or later is part of all distributions of \LaTeX{}
version 2005/12/01 or later.

This work has the LPPL maintenance status `maintained'.

The Current Maintainer of this work is Niklas Beisert.

This work consists of the files |README.txt|, |childdoc.ins| and |childdoc.dtx|
as well as the derived files |childdoc.def|, |cdocsamp.tex|
with |cdocsch1.tex|, |cdocsch2.tex|, |cdocspt3.tex|, |cdocspt4.tex|,
|cdocsdrf.tex|, |cdocsfn1.tex|, |cdocsfn2.tex|
as well as |childdoc.pdf|.

%%%%%%%%%%%%%%%%%%%%%%%%%%%%%%%%%%%%%%%%%%%%%%%%%%%%%%%%%%%%%%%%%%%%%%%%%%%%%%%%
\subsection{Files and Installation}

The package consists of the files:
%
\begin{center}
\begin{tabular}{ll}
    |README.txt|   & readme file \\
    |childdoc.ins| & installation file \\
    |childdoc.dtx| & source file \\
    |childdoc.def| & definition file \\
    |cdocsamp.tex| & sample main file \\
    |cdocsch1.tex| & sample include file \\
    |cdocsch2.tex| & sample include file \\
    |cdocspt3.tex| & sample part file \\
    |cdocspt4.tex| & sample part file \\
    |cdocsdrf.tex| & sample redirection file \\
    |cdocsfn1.tex| & sample redirection file \\
    |cdocsfn2.tex| & sample redirection file \\
    |childdoc.pdf| & manual
\end{tabular}
\end{center}
%
The distribution consists of the files
|README.txt|, |childdoc.ins| and |childdoc.dtx|.
%
\begin{itemize}
\item
Run (pdf)\LaTeX{} on |childdoc.dtx|
to compile the manual |childdoc.pdf| (this file).
\item
Run \LaTeX{} on |childdoc.ins| to create the definitions file |childdoc.def|
and the sample |cdocsamp.tex| with include files
|cdocsch1.tex|, |cdocsch2.tex|, |cdocspt3.tex|, |cdocspt4.tex|,
|cdocsdrf.tex|, |cdocsfn1.tex|, |cdocsfn2.tex|.
Then copy the file |childdoc.def| to an appropriate directory of your \LaTeX{}
distribution, e.g.\ \textit{texmf-root}|/tex/latex/childdoc|.
\end{itemize}

%%%%%%%%%%%%%%%%%%%%%%%%%%%%%%%%%%%%%%%%%%%%%%%%%%%%%%%%%%%%%%%%%%%%%%%%%%%%%%%%
\subsection{Related CTAN Packages}

There are several other packages which offer a similar functionality:
%
\begin{itemize}
\item
The packages
\href{http://ctan.org/pkg/docmute}{\textsf{docmute}},
\href{http://ctan.org/pkg/includex}{\textsf{includex}} and
\href{http://ctan.org/pkg/standalone}{\textsf{standalone}}
provide commands to include only the document body of
a child file thus allowing both files to be compiled individually.
\item
The packages \href{http://ctan.org/pkg/subdocs}{\textsf{subdocs}}
and \href{http://ctan.org/pkg/subfiles}{\textsf{subfiles}}
provide structures in which the main and child documents can be
encapsulated and allowing them to be compiled individually.
The inclusion mechanism is different from the conventional |\include|.
\item
The package \href{http://ctan.org/pkg/combine}{\textsf{combine}}
is an elaborate solution to combine several documents into one.
\end{itemize}
%
See also the CTAN topic \href{http://ctan.org/topic/subdocs}{\textsf{subdocs}}
for further related packages.
The present package differs from the above solutions in that
a document structure constructed with the conventional |\include| mechanism
just needs two extra commands at the top of every file
such that all constituent files can be compiled individually.

%%%%%%%%%%%%%%%%%%%%%%%%%%%%%%%%%%%%%%%%%%%%%%%%%%%%%%%%%%%%%%%%%%%%%%%%%%%%%%%%
%\subsection{Feature Suggestions}
%
%The following is a list of features which may be useful for future
%versions of this package:
%%
%\begin{itemize}
%\item
%\ldots
%\end{itemize}

%%%%%%%%%%%%%%%%%%%%%%%%%%%%%%%%%%%%%%%%%%%%%%%%%%%%%%%%%%%%%%%%%%%%%%%%%%%%%%%%
\subsection{Revision History}

%%%%%%%%%%%%%%%%%%%%%%%%%%%%%%%%%%%%%%%%
\paragraph{v2.0:} 2018/12/30

\begin{itemize}
\item
immediate forward processing
\item
added |\childdocby| mechanism
\item
manual restructured
\end{itemize}

%%%%%%%%%%%%%%%%%%%%%%%%%%%%%%%%%%%%%%%%
\paragraph{v1.6:} 2018/01/17

\begin{itemize}
\item
application for development of include files
\item
corrections to manual
\end{itemize}

%%%%%%%%%%%%%%%%%%%%%%%%%%%%%%%%%%%%%%%%
\paragraph{v1.5:} 2017/05/21

\begin{itemize}
\item
more complete structuring introduced
\item
|\childdocof| introduced
\item
|\childdoc| renamed to |\childdocmain|
\item
|\childredirect| renamed to |\childdocforward| and |\childdocforwardprefix|
and functionality expanded
\end{itemize}

%%%%%%%%%%%%%%%%%%%%%%%%%%%%%%%%%%%%%%%%
\paragraph{v1.0:} 2017/04/27

\begin{itemize}
\item
manual and install package
\item
first version published on CTAN
\end{itemize}

%%%%%%%%%%%%%%%%%%%%%%%%%%%%%%%%%%%%%%%%
\paragraph{v0.6:} 2017/04/26

\begin{itemize}
\item
redirection mechanism added
\end{itemize}

%%%%%%%%%%%%%%%%%%%%%%%%%%%%%%%%%%%%%%%%
\paragraph{v0.5:} 2017/04/26

\begin{itemize}
\item
functionality in definition file
\end{itemize}


%%%%%%%%%%%%%%%%%%%%%%%%%%%%%%%%%%%%%%%%%%%%%%%%%%%%%%%%%%%%%%%%%%%%%%%%%%%%%%%%
%%%%%%%%%%%%%%%%%%%%%%%%%%%%%%%%%%%%%%%%%%%%%%%%%%%%%%%%%%%%%%%%%%%%%%%%%%%%%%%%
%%%%%%%%%%%%%%%%%%%%%%%%%%%%%%%%%%%%%%%%%%%%%%%%%%%%%%%%%%%%%%%%%%%%%%%%%%%%%%%%
\appendix

\settowidth\MacroIndent{\rmfamily\scriptsize 000\ }

 \DocInput{childdoc.dtx}

\end{document}
%</driver>
% \fi
%
% %%%%%%%%%%%%%%%%%%%%%%%%%%%%%%%%%%%%%%%%%%%%%%%%%%%%%%%%%%%%%%%%%%%%%%%%%%%%%%
% %%%%%%%%%%%%%%%%%%%%%%%%%%%%%%%%%%%%%%%%%%%%%%%%%%%%%%%%%%%%%%%%%%%%%%%%%%%%%%
% \section{Sample}
%\iffalse
%<*samplemain>
%\fi
%
% The following presents a sample document
% with two chapters, two parts, a title page,
% a compile flag as well as three forwarding files to set the flag.
% It consists of eight |.tex| files:
% \begin{center}
% \begin{tabular}{ll}
% |cdocsamp.tex|&main file\\
% |cdocsch1.tex|&include file for chapter 1\\
% |cdocsch2.tex|&include file for chapter 2\\
% |cdocspt3.tex|&include file for part 3\\
% |cdocspt4.tex|&include file for part 4\\
% |cdocsdrf.tex|&forwarding file for main file in draft mode\\
% |cdocsfi1.tex|&forwarding file for final version of chapter 1\\
% |cdocsfi2.tex|&forwarding file for final version of chapter 2\\
% \end{tabular}
% \end{center}
% Each of the eight files can be compiled directly by the \LaTeX{} compiler.
%
% %%%%%%%%%%%%%%%%%%%%%%%%%%%%%%%%%%%%%%
% \paragraph{Main File.}
%
% The main file is called |cdocsamp.tex|.
%
% Load the \textsf{childdoc} definitions and
% declare the filename for the main document:
%    \begin{macrocode}
% \iffalse
%
% childdoc.dtx Copyright (C) 2017-2018 Niklas Beisert
%
% This work may be distributed and/or modified under the
% conditions of the LaTeX Project Public License, either version 1.3
% of this license or (at your option) any later version.
% The latest version of this license is in
%   http://www.latex-project.org/lppl.txt
% and version 1.3 or later is part of all distributions of LaTeX
% version 2005/12/01 or later.
%
% This work has the LPPL maintenance status `maintained'.
%
% The Current Maintainer of this work is Niklas Beisert.
%
% This work consists of the files childdoc.dtx and childdoc.ins
% and the derived files childdoc.def and cdocsamp.tex with
% cdocsch1.tex, cdocsch2.tex, cdocsdrf.tex, cdocsfn1.tex, cdocsfn2.tex.
%
%<package>\ifdefined\childdocmain\endinput\fi
%<package>\ProvidesFile{childdoc.def}[2018/12/30 v2.0 child document driver]
%<samplemain>\ProvidesFile{cdocsamp.tex}[2018/12/30 v2.0 sample for childdoc]
%<*driver>
%\ProvidesFile{childdoc.drv}[2018/12/30 v2.0 childdoc reference manual file]
\PassOptionsToClass{10pt,a4paper}{article}
\documentclass{ltxdoc}

\usepackage[margin=35mm]{geometry}
\usepackage{hyperref}
\usepackage{hyperxmp}
\usepackage[usenames]{color}

\hypersetup{colorlinks=true}
\hypersetup{pdfstartview=FitH}
\hypersetup{pdfpagemode=UseNone}
\hypersetup{pdfsource={}}
\hypersetup{pdflang={en-UK}}
\hypersetup{pdfcopyright={Copyright 2017-2018 Niklas Beisert.
  This work may be distributed and/or modified under the
  conditions of the LaTeX Project Public License, either version 1.3
  of this license or (at your option) any later version.}}
\hypersetup{pdflicenseurl={http://www.latex-project.org/lppl.txt}}
\hypersetup{pdfcontactaddress={ETH Zurich, ITP, HIT K,
  Wolfgang-Pauli-Strasse 27}}
\hypersetup{pdfcontactpostcode={8093}}
\hypersetup{pdfcontactcity={Zurich}}
\hypersetup{pdfcontactcountry={Switzerland}}
\hypersetup{pdfcontactemail={nbeisert@itp.phys.ethz.ch}}
\hypersetup{pdfcontacturl={http://people.phys.ethz.ch/\xmptilde nbeisert/}}

\newcommand{\secref}[1]{\hyperref[#1]{section \ref*{#1}}}

\parskip1ex
\parindent0pt
\let\olditemize\itemize
\def\itemize{\olditemize\parskip0pt}

\begin{document}

\title{The \textsf{childdoc} Package}
\hypersetup{pdftitle={The childdoc Package}}
\author{Niklas Beisert\\[2ex]
  Institut f\"ur Theoretische Physik\\
  Eidgen\"ossische Technische Hochschule Z\"urich\\
  Wolfgang-Pauli-Strasse 27, 8093 Z\"urich, Switzerland\\[1ex]
  \href{mailto:nbeisert@itp.phys.ethz.ch}
  {\texttt{nbeisert@itp.phys.ethz.ch}}}
\hypersetup{pdfauthor={Niklas Beisert}}
\hypersetup{pdfsubject={Manual for the LaTeX2e Package childdoc}}
\date{30 December 2018, \textsf{v2.0}}
\maketitle

\begin{abstract}\noindent
\textsf{childdoc} is a \LaTeXe{} package
that enables the direct compilation
of document sections included by |\include|
to individual files.
\end{abstract}

\begingroup
\parskip0ex
\tableofcontents
\endgroup

%%%%%%%%%%%%%%%%%%%%%%%%%%%%%%%%%%%%%%%%%%%%%%%%%%%%%%%%%%%%%%%%%%%%%%%%%%%%%%%%
%%%%%%%%%%%%%%%%%%%%%%%%%%%%%%%%%%%%%%%%%%%%%%%%%%%%%%%%%%%%%%%%%%%%%%%%%%%%%%%%
\section{Introduction}

\LaTeX{} provides a mechanism to structure a large document (such as a book)
into a main file and several child files (containing the chapters)
using the |\include| command.
This mechanism is beneficial for documents
which span hundreds of pages in order to
make the source file(s) more manageable.
Moreover, compilation can be restricted to
selected child files by means of the |\includeonly| command.
The latter feature can be used to reduce the compilation time while editing
(this was significantly more useful in the earlier days of \LaTeX{})
or to generate a smaller document which is easier to navigate.
Another application of |\includeonly| is to generate
documents consisting of selected parts of the complete document.

However, there are a few drawbacks of the plain |\include| mechanism:
\begin{itemize}
\item
The child files cannot be compiled on their own,
they can only be compiled via the main file.
A naive editing environment
(such as a text editor with an option
to have the current file processed by \LaTeX)
may require one to switch to the main file before compiling;
attempting to compile the child file produces errors.
\item
The main file must be modified (each time)
to adjust the |\includeonly| command
to the present needs. This easily leaves the main file in a messy state.
\item
The generated document will always carry the filename
of the main document. This is inconvenient if
several child files are to be compiled and
to be kept for distribution.
\end{itemize}

The present package provides a simple interface
to make child files individually compilable by \LaTeX{}.
Compiling a child file then has the same effect as compiling
the main file with an |\includeonly| command
to select the appropriate child.
Moreover the generated document will carry the name of the child
rather than the main file.
This resolves all three above issues.

This feature is meant to make the editing of books,
thesis documents and lecture notes somewhat more convenient.
However, the package can also be used efficiently for
composing a series of documents (such as exercise sheets)
which are typically distributed individually.
It then assists the author in generating the individual documents
(potentially in different versions)
as well as a document containing the collected series.
Another application is in developing style files
or other kinds of included material
where compilation of the style file could redirect
to a sample or test file.

%%%%%%%%%%%%%%%%%%%%%%%%%%%%%%%%%%%%%%%%%%%%%%%%%%%%%%%%%%%%%%%%%%%%%%%%%%%%%%%%
%%%%%%%%%%%%%%%%%%%%%%%%%%%%%%%%%%%%%%%%%%%%%%%%%%%%%%%%%%%%%%%%%%%%%%%%%%%%%%%%
\section{Usage}

First of all, the package \textsf{childdoc} is \emph{not} a standard
\LaTeXe{} |.sty| style file! Therefore it needs to be invoked in
a non-standard way.

%%%%%%%%%%%%%%%%%%%%%%%%%%%%%%%%%%%%%%%%%%%%%%%%%%%%%%%%%%%%%%%%%%%%%%%%%%%%%%%%
\subsection{Included Files}
\label{sec:include}

%%%%%%%%%%%%%%%%%%%%%%%%%%%%%%%%%%%%%%%%
\DescribeMacro{\childdocmain}
To use the package, add the commands
\begin{center}
\begin{tabular}{l}
|\input{childdoc.def}|\\
|\childdocmain{}|\\
\end{tabular}
\end{center}
at the very top of the main \LaTeX{} file,
in particular \emph{before} the |\documentclass| statement!
The argument of |\childdocmain| should be left empty
(but it must be present).

%%%%%%%%%%%%%%%%%%%%%%%%%%%%%%%%%%%%%%%%
\DescribeMacro{\childdocof}
Furthermore, add the commands
\begin{center}
\begin{tabular}{l}
|\input{childdoc.def}|\\
|\childdocof{|\textit{main}|}|\\
\end{tabular}
\end{center}
at the top of every child file \textit{child}
which is included by |\include{|\textit{child}|}|
from within the main file
(or at least for those files to be compiled individually).
The argument \textit{main} must be the filename of the main file.

There are a couple of
considerations in setting up the main and child documents:

%%%%%%%%%%%%%%%%%%%%%%%%%%%%%%%%%%%%%%%%
\paragraph{Restrictions.}

Please note the following restrictions:
\begin{itemize}
\item
|\childdocmain| must be called with one argument \textit{main}
to ensure compatibility with earlier version of the package.
It must either be empty (|\childdocmain{}|)
or precisely match the filename of the main file in which it is specified.
See \secref{sec:detection} for further information.
\item
The filename \textit{main} must be specified without the |.tex| extension.
\item
The filename \textit{main} is case sensitive
(even in case-insensitive file systems)
due to internal string comparison.
\item
The argument \textit{main} should be fully expanded, it cannot be a macro.
\item
Subdirectories and special characters should be avoided in filenames.
\item
The command |\childdocmain{|\textit{main}|}| must be followed by a whitespace.
It should not be followed immediately by another command
or by a comment mark `|%|'.
This is because the \TeX{} parser reads the token immediately following
the argument of |\childdocmain| and puts it
at the beginning of every child section;
however, a white\-space is ignored.
\end{itemize}

%%%%%%%%%%%%%%%%%%%%%%%%%%%%%%%%%%%%%%%%
\paragraph{Content of Main File.}

It is advisable to place all content in the child files included by |\include|.
Any output contained in the main file will appear in all child documents
unless suppressed manually;
it cannot be suppressed automatically by the |\includeonly| directive
and thus should normally be avoided.
A method to include some content in the main file
by means of conditional processing is described in \secref{sec:conditional}.

%%%%%%%%%%%%%%%%%%%%%%%%%%%%%%%%%%%%%%%%
\paragraph{Page Numbering.}

When only a part of the document is compiled,
the appropriate numbering of pages
(as well as other status parameters)
is determined from the |.aux| files.
The latter contain information from previous passes.
However this information needs to propagate through
all intermediate child documents.
Therefore the page numbering in child documents may well
be inconsistent until the complete document is compiled at least once.

A useful (if unconventional) way to always ensure a consistent
page numbering is to restart the numbering in each child document
and denote the pages by `\textit{child}|.|\textit{page}'
where \textit{child} represents the chapter/section number of the child file.
This can be achieved by the command
|\numberwithin{page}{|\textit{child}|}|
of the \textsf{amsmath} package
where \textit{child} can be |chapter| or |section|
depending on the chosen structuring.
Alternatively, one can modify the macro |\thepage| appropriately
and reset the counter |page| at the start of each child file.

%%%%%%%%%%%%%%%%%%%%%%%%%%%%%%%%%%%%%%%%%%%%%%%%%%%%%%%%%%%%%%%%%%%%%%%%%%%%%%%%
\subsection{Conditional Processing}
\label{sec:conditional}

The package provides a mechanism to compile different versions
of a document. To customise the versions further some conditional processing
can come in handy to distinguish which version is being compiled.
The package provides two macros to describe the compilation context:

%%%%%%%%%%%%%%%%%%%%%%%%%%%%%%%%%%%%%%%%
\DescribeMacro{\ifchilddoc}
The conditional |\ifchilddoc| distinguishes between the compilation of
child documents and the main document:
%
\begin{center}
|\ifchilddoc |\textit{child-code}| |[|\||else |\textit{main-code}]| \||fi|
\end{center}

%%%%%%%%%%%%%%%%%%%%%%%%%%%%%%%%%%%%%%%%
\DescribeMacro{\childdocname}
\DescribeMacro{\childdocjob}
The macro |\childdocname| contains the filename (without extension)
of the main or child file being processed.
Note that |\childdocjob| will always contain the name of the main file.

%%%%%%%%%%%%%%%%%%%%%%%%%%%%%%%%%%%%%%%%
\paragraph{Title Page.}

Conditional processing can be used to include a title or banner page
in the main document when proper precautions are taken.
Importantly, the code in the main file should ensure that the page counter
(as well as other status parameters which are stored in the |.aux| files)
takes the same value after the conditional processing.
Otherwise the page numbers may take divergent values
depending on which part is compiled.

For example, a title page could be declared by:
%
\begin{center}
\begin{tabular}{l}
|\ifchilddoc\||else|\\
|\addtocounter{page}{-1}|\\
\textit{code for title page}\\
|\newpage|\\
|\||fi|
\end{tabular}
\end{center}
%
A banner page for the child documents can be generated by:
%
\begin{center}
\begin{tabular}{l}
|\ifchilddoc|\\
|\addtocounter{page}{-1}|\\
\textit{code for banner page}\\
|\newpage|\\
|\||fi|
\end{tabular}
\end{center}
%
Here one could write a message such as:
\begin{center}
|This is the part \childdocname{} of \childdocjob{}.|
\end{center}

%%%%%%%%%%%%%%%%%%%%%%%%%%%%%%%%%%%%%%%%%%%%%%%%%%%%%%%%%%%%%%%%%%%%%%%%%%%%%%%%
\subsection{Flags}
\label{sec:flags}

The package makes it easy to generate different versions
of the main or child documents.
To this end compilation flags can be defined
and assigned different default values.
They will be particularly useful in conjunction
with the forwarding mechanism described in \secref{sec:forward}.

For example, it may be useful to have a flag |\version|
which can be set to |draft| or |final|.
The document source will contain some conditional code
depending on the value of |\version|.
Suppose further, the flag should default to |final| for the main file
and to |draft| for child files
which is a natural assignment for editing the document.
This is achieved by placing the following code
in the preamble of the main document
(below the |\childdocmain| directive):
%
\begin{center}
\begin{tabular}{l}
|\ifchilddoc|\\
|\providecommand{\version}{draft}|\\
|\||else|\\
|\providecommand{\version}{final}|\\
|\||fi|
\end{tabular}
\end{center}
%
The definition by |\providecommand| makes sure
that previous definitions are not overwritten.
Further statements |\providecommand{\version}{...}|
can thus be added before the above code to override it.

For the main file, one might add a line
(between |\childdocmain| and the above block)
%
\begin{center}
|%\ifchilddoc\||else\providecommand{\version}{draft}\||fi|
\end{center}
%
which can be uncommented to produce a draft version.
Likewise one can add a line to the very top of a child file
(above the |\childdocof{|\textit{main}|}| directive)
%
\begin{center}
|%\providecommand{\version}{final}|
\end{center}
%
which can be uncommented to produce the final version of this child document.

%%%%%%%%%%%%%%%%%%%%%%%%%%%%%%%%%%%%%%%%%%%%%%%%%%%%%%%%%%%%%%%%%%%%%%%%%%%%%%%%
\subsection{Forwarding}
\label{sec:forward}

Different versions of the main or child documents
using compilation flags as described in \secref{sec:flags}
can be (permanently) stored in different files
for convenient compilation, viewing and distribution.
To this end, the package defines a command
to pass on compilation to a different file:

%%%%%%%%%%%%%%%%%%%%%%%%%%%%%%%%%%%%%%%%
\DescribeMacro{\childdocforward}
The command |\childdocforward| redirects processing to
another source file:
%
\begin{center}
\begin{tabular}{l}
|\input{childdoc.def}|\\
|\childdocforward[|\textit{main}|]{|\textit{dest}|}|\\
\end{tabular}
\end{center}
%
The argument \textit{dest} is the destination file
(without extension).
It should be the main file or one of the child files.
Note that further \textsf{childdoc} directives
such as |\childdocof| and |\childdocforward|
in the indicated file will be processed in this form.
The optional argument \textit{main}
passes on directly to the main file \textit{main}
while pretending to compile the child \textit{dest}.
This form behaves as if \textit{dest}
issues |\childdocof{|\textit{main}|}| right away,
and no further \textsf{childdoc} directives will be processed.

%%%%%%%%%%%%%%%%%%%%%%%%%%%%%%%%%%%%%%%%
\DescribeMacro{\...prefix}
In the alternative form |\childdocforwardprefix|,
%
\begin{center}
\begin{tabular}{l}
|\input{childdoc.def}|\\
|\childdocforwardprefix[|\textit{main}|]{|\textit{prefix}|}{|\textit{dest}|}|
\end{tabular}
\end{center}
%
the destination file is determined by a pattern
depending on the current file:
To make this work, the current file must be called
`{\textit{prefix}\hspace{0.2em}\textit{suffix}}'
with \textit{prefix} matching precisely the argument.
Processing is then passed on to the file
`{\textit{dest}\hspace{0.2em}\textit{suffix}}'.
Surely, the same effect is achieved by
directly specifying the
argument `{\textit{dest}\hspace{0.2em}\textit{suffix}}'
in the first form.
However, that requires to set up a different file
for each child. With the alternative form of the command
all these files can have exactly the same content
which simplifies setting them up and maintaining them.

For example, the following file |draft.tex|
with a compilation flag |\version| as described in \secref{sec:flags}
compiles the main document as a draft:
%
\begin{center}
\begin{tabular}{l}
|\def\version{draft}|\\
|\input{childdoc.def}|\\
|\childdocforward{|\textit{main}|}|
\end{tabular}
\end{center}
%
Likewise, the following files |final|\textit{nn}|.tex|
compile the final version of the child document
|child|\textit{nn}|.tex|:
%
\begin{center}
\begin{tabular}{l}
|\def\version{final}|\\
|\input{childdoc.def}|\\
|\childdocforwardprefix{final}{child}|
\end{tabular}
\end{center}
%

Note that when several versions of a main file and/or of each child file
are to be generated, it may be convenient to set up a |Makefile| or
shell script to automatise the process.

%%%%%%%%%%%%%%%%%%%%%%%%%%%%%%%%%%%%%%%%%%%%%%%%%%%%%%%%%%%%%%%%%%%%%%%%%%%%%%%%
\subsection{Command Line Processing}
\label{sec:commandline}

The effect of redirection files can also be achieved by invoking
the \LaTeX{} compiler with a more elaborate command line.
Most conveniently this should be done as part
of a shell script or a |Makefile|.

When using \textsf{childdoc} in the main file, the following
command lines effectively perform a redirection
(note that depending on the shell being used,
backslashes may have to be doubled: `|\|' $\to$ `|\\|'):
%
\begin{center}
|... -jobname "|\textit{target}|" |\\|"|[\textit{flags}]%
|\input{childdoc.def}\childdocforward[|\textit{main}|]{|\textit{dest}|}"|
\end{center}
%
Here \textit{target} is the name of the output file,
\textit{main} is the name of the main file
and \textit{dest} is the name of the main or child file to be processed
(all filenames without extensions).
The optional argument \textit{main} can be omitted
if \textit{main} matches \textit{dest}.
Optionally, compilation \textit{flags} can be defined via |\def| commands.
This command line makes the \TeX{} engine believe
it is compiling the file \textit{target}
whose content is specified as the latter parameter.
The provided code then forwards the processing to
\textit{main} or \textit{dest} as described in \secref{sec:forward}.

%%%%%%%%%%%%%%%%%%%%%%%%%%%%%%%%%%%%%%%%%%%%%%%%%%%%%%%%%%%%%%%%%%%%%%%%%%%%%%%%
\subsection{Include by Input}
\label{sec:input}

Including child documents by |\include| has some restrictions by design.
Most notably, the content of a child document always occupies
its own set of pages; pages cannot be shared between child documents.
Usually, this behaviour makes perfect sense
because each child document contain an essential part of the document.
However, in some situations it may be desirable to compose
a document from a collection of parts
without having mandatory page breaks between then.
For this case, the package
provides a mechanism to include parts
by |\input| which can also be processed individually.
However, by construction this mechanism
requires manual handling of the content to be output.

%%%%%%%%%%%%%%%%%%%%%%%%%%%%%%%%%%%%%%%%
\DescribeMacro{\ifchilddocmanual}
The main file should be prepared as usual, see \secref{sec:include}.
However, the document body must make a distinction
between processing of an individual part and of the main document, e.g.:
%
\begin{center}
\begin{tabular}{l}
|\ifchilddocmanual|\\
|\input{\childdocname}|\\
|\||else|\\
\textit{document body with }|\input{|\textit{part}|}|\\
|\||fi|
\end{tabular}
\end{center}
%
The conditional |\ifchilddocmanual| is true whenever
a part to be included by |\input| is being compiled,
and the name of the part is stored in |\childdocname|.

%%%%%%%%%%%%%%%%%%%%%%%%%%%%%%%%%%%%%%%%
\DescribeMacro{\childdocby}
Each part to be included by |\input| should start with:
%
\begin{center}
\begin{tabular}{l}
|\input{childdoc.def}|\\
|\childdocby{|\textit{main}|}|\\
\end{tabular}
\end{center}
%
The directive |\childdocby| is similar to |\childdocof|
described in \secref{sec:include},
but the subsequent selection of content must be done manually.
To that end, both |\ifchilddoc| and |\ifchilddocmanual|
will be true upon processing of a part,
and the name of the part is stored in |\childdocname|.
Note that |\jobname| will be set to the filename of the current part
so that each part receives an individual |.aux| file
that does not interfere with the |.aux| file(s) of the main document.
This behaviour can be altered by the alternative form
|\childdocby[*]{|\textit{main}|}| (with a non-empty optional argument)
which uses the |.aux| file of the main document
by setting |\jobname| to \textit{main}.

%%%%%%%%%%%%%%%%%%%%%%%%%%%%%%%%%%%%%%%%%%%%%%%%%%%%%%%%%%%%%%%%%%%%%%%%%%%%%%%%
\subsection{Driver Development}
\label{sec:driver}

The \textsf{childdoc} mechanism can also be use for the development
of definition files such as \LaTeX{} styles or classes.
This case differs from the above setup with multiple parts
included by |\include| in that no |\includeonly| should be invoked.
This can be achieved by starting the include file
(before |\ProvidesPackage|) with:
%
\begin{center}
\begin{tabular}{l}
|\input{childdoc.def}|\\
|\childdocforward{|\textit{main}|}|\\
\end{tabular}
\end{center}
%
or alternatively with:
%
\begin{center}
\begin{tabular}{l}
|\input{childdoc.def}|\\
|\childdocby{|\textit{main}|}|\\
\end{tabular}
\end{center}
%
Both forms have slightly different effects as described above.
The main file is prepared as usual, see \secref{sec:include}.

%%%%%%%%%%%%%%%%%%%%%%%%%%%%%%%%%%%%%%%%%%%%%%%%%%%%%%%%%%%%%%%%%%%%%%%%%%%%%%%%
\subsection{Legacy Detection}
\label{sec:detection}

The directive |\childdocmain| in the main file can detect
whether the complete document or merely a child is to be compiled
even without using the directive |\childdocof|.
This method is deprecated because it is less robust
and there is no compelling reason to use it;
it is merely provided for backward compatibility
and it may be removed in future versions.

If the detection mechanism is to be used,
it is mandatory to correctly specify
the filename of the main file as the argument of |\childdocmain|:
%
\begin{center}
\begin{tabular}{l}
|\input{childdoc.def}|\\
|\childdocmain{|\textit{main}|}|\\
\end{tabular}
\end{center}
%
If |\jobname| does not match the argument \textit{main} of |\childdocmain|,
it is assumed that |\jobname| points to the child file to be compiled.
When using |\childdocmain| with the main file specified as argument,
it suffices to start a child file
with just |\input{|\textit{main}|}|
without loading of the package and using |\childdocof|.
If instead all processing is done
with the appropriate \textsf{childdoc} directives,
the argument of \textit{main} of |\childdocmain| can be empty.

An alternative version of the command line processing described
in \secref{sec:commandline} using the detection mechanism reads:
%
\begin{center}
|... -jobname "|\textit{target}|" "|[\textit{flags}]%
[|\def\jobname{|\textit{dest}|}|]|\input{|\textit{main}|}"|
\end{center}

%%%%%%%%%%%%%%%%%%%%%%%%%%%%%%%%%%%%%%%%%%%%%%%%%%%%%%%%%%%%%%%%%%%%%%%%%%%%%%%%
\subsection{Manual Code}
\label{sec:manual}

In case one cannot be certain whether the definitions file |childdoc.def|
is installed on the target \TeX{} distribution
and one prefers not to ship it,
it is conceivable to paste a few relevant commands into the sources.

To that end, drop all statements |\input{childdoc.def}|
and perform the replacements as outlined below.
Instead of |\childdocmain{|\textit{main}|}| add the following code
to the top of the main file:
%
\begin{center}
\begin{tabular}{l}
|\||ifdefined\childdocname\endinput\||fi\newif\ifchilddoc|\\
|\edef\childdocname{\scantokens\expandafter{\jobname\noexpand}}|\\
|\def\childdocmain{|\textit{main}|}\||ifx\childdocmain\childdocname\||else|\\
|\childdoctrue\includeonly{\childdocname}\let\jobname\childdocmain\||fi|\\
\end{tabular}
\end{center}
%
Instead of |\childdocof{|\textit{main}|}| just include the main file
at the top of each child file:
%
\begin{center}
|\input{|\textit{main}|}|
\end{center}
%
A simple redirection |\childdocforward{|\textit{dest}|}| is achieved by:
%
\begin{center}
|\def\jobname{|\textit{dest}|}\input{\jobname}|
\end{center}
%
The redirection with prefix
|\childdocforwardprefix[|\textit{prefix}|]{|\textit{dest}|}|
is accomplished by:
%
\begin{center}
\begin{tabular}{l}
|{\edef\jobname{\scantokens\expandafter{\jobname\noexpand}}|\\
|\def\redirectjob |\textit{prefix}|#1~~~{\gdef\jobname{|\textit{dest}|#1}}|\\
|\expandafter\redirectjob\jobname~~~}\input{\jobname}|
\end{tabular}
\end{center}

In an alternative approach,
child documents can be compiled by a specific command line
without additional code or specific definitions:
%
\begin{center}
|... -jobname "|\textit{target}|" "|[\textit{flags}]%
|\includeonly{|\textit{dest}|}\input{|\textit{main}|}"|
\end{center}
%

%%%%%%%%%%%%%%%%%%%%%%%%%%%%%%%%%%%%%%%%%%%%%%%%%%%%%%%%%%%%%%%%%%%%%%%%%%%%%%%%
%%%%%%%%%%%%%%%%%%%%%%%%%%%%%%%%%%%%%%%%%%%%%%%%%%%%%%%%%%%%%%%%%%%%%%%%%%%%%%%%
\section{Information}

%%%%%%%%%%%%%%%%%%%%%%%%%%%%%%%%%%%%%%%%%%%%%%%%%%%%%%%%%%%%%%%%%%%%%%%%%%%%%%%%
\subsection{Copyright}

Copyright \copyright{} 2017--2018 Niklas Beisert

This work may be distributed and/or modified under the
conditions of the \LaTeX{} Project Public License, either version 1.3
of this license or (at your option) any later version.
The latest version of this license is in
  \url{http://www.latex-project.org/lppl.txt}
and version 1.3 or later is part of all distributions of \LaTeX{}
version 2005/12/01 or later.

This work has the LPPL maintenance status `maintained'.

The Current Maintainer of this work is Niklas Beisert.

This work consists of the files |README.txt|, |childdoc.ins| and |childdoc.dtx|
as well as the derived files |childdoc.def|, |cdocsamp.tex|
with |cdocsch1.tex|, |cdocsch2.tex|, |cdocspt3.tex|, |cdocspt4.tex|,
|cdocsdrf.tex|, |cdocsfn1.tex|, |cdocsfn2.tex|
as well as |childdoc.pdf|.

%%%%%%%%%%%%%%%%%%%%%%%%%%%%%%%%%%%%%%%%%%%%%%%%%%%%%%%%%%%%%%%%%%%%%%%%%%%%%%%%
\subsection{Files and Installation}

The package consists of the files:
%
\begin{center}
\begin{tabular}{ll}
    |README.txt|   & readme file \\
    |childdoc.ins| & installation file \\
    |childdoc.dtx| & source file \\
    |childdoc.def| & definition file \\
    |cdocsamp.tex| & sample main file \\
    |cdocsch1.tex| & sample include file \\
    |cdocsch2.tex| & sample include file \\
    |cdocspt3.tex| & sample part file \\
    |cdocspt4.tex| & sample part file \\
    |cdocsdrf.tex| & sample redirection file \\
    |cdocsfn1.tex| & sample redirection file \\
    |cdocsfn2.tex| & sample redirection file \\
    |childdoc.pdf| & manual
\end{tabular}
\end{center}
%
The distribution consists of the files
|README.txt|, |childdoc.ins| and |childdoc.dtx|.
%
\begin{itemize}
\item
Run (pdf)\LaTeX{} on |childdoc.dtx|
to compile the manual |childdoc.pdf| (this file).
\item
Run \LaTeX{} on |childdoc.ins| to create the definitions file |childdoc.def|
and the sample |cdocsamp.tex| with include files
|cdocsch1.tex|, |cdocsch2.tex|, |cdocspt3.tex|, |cdocspt4.tex|,
|cdocsdrf.tex|, |cdocsfn1.tex|, |cdocsfn2.tex|.
Then copy the file |childdoc.def| to an appropriate directory of your \LaTeX{}
distribution, e.g.\ \textit{texmf-root}|/tex/latex/childdoc|.
\end{itemize}

%%%%%%%%%%%%%%%%%%%%%%%%%%%%%%%%%%%%%%%%%%%%%%%%%%%%%%%%%%%%%%%%%%%%%%%%%%%%%%%%
\subsection{Related CTAN Packages}

There are several other packages which offer a similar functionality:
%
\begin{itemize}
\item
The packages
\href{http://ctan.org/pkg/docmute}{\textsf{docmute}},
\href{http://ctan.org/pkg/includex}{\textsf{includex}} and
\href{http://ctan.org/pkg/standalone}{\textsf{standalone}}
provide commands to include only the document body of
a child file thus allowing both files to be compiled individually.
\item
The packages \href{http://ctan.org/pkg/subdocs}{\textsf{subdocs}}
and \href{http://ctan.org/pkg/subfiles}{\textsf{subfiles}}
provide structures in which the main and child documents can be
encapsulated and allowing them to be compiled individually.
The inclusion mechanism is different from the conventional |\include|.
\item
The package \href{http://ctan.org/pkg/combine}{\textsf{combine}}
is an elaborate solution to combine several documents into one.
\end{itemize}
%
See also the CTAN topic \href{http://ctan.org/topic/subdocs}{\textsf{subdocs}}
for further related packages.
The present package differs from the above solutions in that
a document structure constructed with the conventional |\include| mechanism
just needs two extra commands at the top of every file
such that all constituent files can be compiled individually.

%%%%%%%%%%%%%%%%%%%%%%%%%%%%%%%%%%%%%%%%%%%%%%%%%%%%%%%%%%%%%%%%%%%%%%%%%%%%%%%%
%\subsection{Feature Suggestions}
%
%The following is a list of features which may be useful for future
%versions of this package:
%%
%\begin{itemize}
%\item
%\ldots
%\end{itemize}

%%%%%%%%%%%%%%%%%%%%%%%%%%%%%%%%%%%%%%%%%%%%%%%%%%%%%%%%%%%%%%%%%%%%%%%%%%%%%%%%
\subsection{Revision History}

%%%%%%%%%%%%%%%%%%%%%%%%%%%%%%%%%%%%%%%%
\paragraph{v2.0:} 2018/12/30

\begin{itemize}
\item
immediate forward processing
\item
added |\childdocby| mechanism
\item
manual restructured
\end{itemize}

%%%%%%%%%%%%%%%%%%%%%%%%%%%%%%%%%%%%%%%%
\paragraph{v1.6:} 2018/01/17

\begin{itemize}
\item
application for development of include files
\item
corrections to manual
\end{itemize}

%%%%%%%%%%%%%%%%%%%%%%%%%%%%%%%%%%%%%%%%
\paragraph{v1.5:} 2017/05/21

\begin{itemize}
\item
more complete structuring introduced
\item
|\childdocof| introduced
\item
|\childdoc| renamed to |\childdocmain|
\item
|\childredirect| renamed to |\childdocforward| and |\childdocforwardprefix|
and functionality expanded
\end{itemize}

%%%%%%%%%%%%%%%%%%%%%%%%%%%%%%%%%%%%%%%%
\paragraph{v1.0:} 2017/04/27

\begin{itemize}
\item
manual and install package
\item
first version published on CTAN
\end{itemize}

%%%%%%%%%%%%%%%%%%%%%%%%%%%%%%%%%%%%%%%%
\paragraph{v0.6:} 2017/04/26

\begin{itemize}
\item
redirection mechanism added
\end{itemize}

%%%%%%%%%%%%%%%%%%%%%%%%%%%%%%%%%%%%%%%%
\paragraph{v0.5:} 2017/04/26

\begin{itemize}
\item
functionality in definition file
\end{itemize}


%%%%%%%%%%%%%%%%%%%%%%%%%%%%%%%%%%%%%%%%%%%%%%%%%%%%%%%%%%%%%%%%%%%%%%%%%%%%%%%%
%%%%%%%%%%%%%%%%%%%%%%%%%%%%%%%%%%%%%%%%%%%%%%%%%%%%%%%%%%%%%%%%%%%%%%%%%%%%%%%%
%%%%%%%%%%%%%%%%%%%%%%%%%%%%%%%%%%%%%%%%%%%%%%%%%%%%%%%%%%%%%%%%%%%%%%%%%%%%%%%%
\appendix

\settowidth\MacroIndent{\rmfamily\scriptsize 000\ }

 \DocInput{childdoc.dtx}

\end{document}
%</driver>
% \fi
%
% %%%%%%%%%%%%%%%%%%%%%%%%%%%%%%%%%%%%%%%%%%%%%%%%%%%%%%%%%%%%%%%%%%%%%%%%%%%%%%
% %%%%%%%%%%%%%%%%%%%%%%%%%%%%%%%%%%%%%%%%%%%%%%%%%%%%%%%%%%%%%%%%%%%%%%%%%%%%%%
% \section{Sample}
%\iffalse
%<*samplemain>
%\fi
%
% The following presents a sample document
% with two chapters, two parts, a title page,
% a compile flag as well as three forwarding files to set the flag.
% It consists of eight |.tex| files:
% \begin{center}
% \begin{tabular}{ll}
% |cdocsamp.tex|&main file\\
% |cdocsch1.tex|&include file for chapter 1\\
% |cdocsch2.tex|&include file for chapter 2\\
% |cdocspt3.tex|&include file for part 3\\
% |cdocspt4.tex|&include file for part 4\\
% |cdocsdrf.tex|&forwarding file for main file in draft mode\\
% |cdocsfi1.tex|&forwarding file for final version of chapter 1\\
% |cdocsfi2.tex|&forwarding file for final version of chapter 2\\
% \end{tabular}
% \end{center}
% Each of the eight files can be compiled directly by the \LaTeX{} compiler.
%
% %%%%%%%%%%%%%%%%%%%%%%%%%%%%%%%%%%%%%%
% \paragraph{Main File.}
%
% The main file is called |cdocsamp.tex|.
%
% Load the \textsf{childdoc} definitions and
% declare the filename for the main document:
%    \begin{macrocode}
\input{childdoc.def}
\childdocmain{}
%    \end{macrocode}

% Optional override for |\version| flag:
%    \begin{macrocode}
%%\ifchilddoc\else\providecommand{\version}{draft}\fi
%    \end{macrocode}

% Define the default values for the |\version| flag
% (|final| for the main file and |draft| for childs):
%    \begin{macrocode}
\ifchilddoc
\providecommand{\version}{draft}
\else
\providecommand{\version}{final}
\fi
%    \end{macrocode}

% Load the standard document class:
%    \begin{macrocode}
\documentclass[12pt]{article}
%    \end{macrocode}

% Start the document body:
%    \begin{macrocode}
\begin{document}
%    \end{macrocode}

% Declare a title page.
% Print title, part of document being processed and version flag:
%    \begin{macrocode}
\addtocounter{page}{-1}
\begin{center}
{\LARGE\bfseries{}childdoc example\par}
\vspace{1cm}
\ifchilddoc
\ifchilddocmanual part\else chapter\fi:
`\childdocname' of `\childdocjob'\par
\else
main document: `\childdocjob'\par
\fi
version: \version\par
\end{center}
\newpage
%    \end{macrocode}

% Manually include selected file,
% otherwise process as usual:
%    \begin{macrocode}
\ifchilddocmanual
\section*{part `\childdocname'}
\input{\childdocname}
\else
%    \end{macrocode}

% Include the two chapters:
%    \begin{macrocode}
\include{cdocsch1}
\include{cdocsch2}
%    \end{macrocode}

% Include the two parts unless only chapters should be displayed:
%    \begin{macrocode}
\ifchilddoc\else
\section{part three}
\input{cdocspt3}
\section{part four}
\input{cdocspt4}
\fi
%    \end{macrocode}

% Process as usual until here:
%    \begin{macrocode}
\fi
%    \end{macrocode}

% End of document body:
%    \begin{macrocode}
\end{document}
%    \end{macrocode}
%\iffalse
%</samplemain>
%\fi
%
% %%%%%%%%%%%%%%%%%%%%%%%%%%%%%%%%%%%%%%
% \paragraph{Chapter Include Files.}
%
% The include files are called |cdocsch1.tex| and |cdocsch2.tex|.
%
%\iffalse
%<*samplechap1|samplechap2>
%\fi

% Optional override for |\version| flag:
%    \begin{macrocode}
%%\providecommand{\version}{final}
%    \end{macrocode}

% Include the main document:
%    \begin{macrocode}
\input{childdoc.def}
\childdocof{cdocsamp}
%    \end{macrocode}

%\iffalse
%</samplechap1|samplechap2>
%\fi
%
%\iffalse
%<*samplechap1>
%\fi
% Some text for chapter 1:
%    \begin{macrocode}
\section{one}
some text in chapter one
%    \end{macrocode}

%\iffalse
%</samplechap1>
%\fi
% Some text for chapter 2:
%\iffalse
%<*samplechap2>
%\fi
%    \begin{macrocode}
\section{two}
more text in chapter two
%    \end{macrocode}

%\iffalse
%</samplechap2>
%\fi
%
% %%%%%%%%%%%%%%%%%%%%%%%%%%%%%%%%%%%%%%
% \paragraph{Part Include Files.}
%
% The include files are called |cdocspt3.tex| and |cdocspt4.tex|.
%
%\iffalse
%<*samplepart3|samplepart4>
%\fi

% Optional override for |\version| flag:
%    \begin{macrocode}
%%\providecommand{\version}{final}
%    \end{macrocode}

% Include the main document:
%    \begin{macrocode}
\input{childdoc.def}
\childdocby{cdocsamp}
%    \end{macrocode}

%\iffalse
%</samplepart3|samplepart4>
%\fi
%
%\iffalse
%<*samplepart3>
%\fi
% Some text for part 3:
%    \begin{macrocode}
some text in part three
%    \end{macrocode}

%\iffalse
%</samplepart3>
%\fi
% Some text for part 4:
%\iffalse
%<*samplepart4>
%\fi
%    \begin{macrocode}
more text in part four
%    \end{macrocode}

%\iffalse
%</samplepart4>
%\fi
%
% %%%%%%%%%%%%%%%%%%%%%%%%%%%%%%%%%%%%%%
% \paragraph{Forwarding for a Complete Draft.}
%
% The following forwarding file |cdocsdrf.tex|
% compiles the main document in draft mode:
%\iffalse
%<*sampledraft>
%\fi
%    \begin{macrocode}
\def\version{draft}
\input{childdoc.def}
\childdocforward{cdocsamp}
%    \end{macrocode}

%\iffalse
%</sampledraft>
%\fi
%
% %%%%%%%%%%%%%%%%%%%%%%%%%%%%%%%%%%%%%%
% \paragraph{Forwarding for Final Version of the Chapters.}
%
% The following forwarding files |cdocsfn1.tex| and |cdocsfn2.tex|
% (with identical content)
% compile the final versions of the child documents
% |cdocsch1.tex| and |cdocsch2.tex|, respectively:
%\iffalse
%<*samplefinal>
%\fi
%    \begin{macrocode}
\def\version{final}
\input{childdoc.def}
\childdocforwardprefix[cdocsamp]{cdocsfn}{cdocsch}
%    \end{macrocode}

%\iffalse
%</samplefinal>
%\fi
%
% %%%%%%%%%%%%%%%%%%%%%%%%%%%%%%%%%%%%%%
% \paragraph{Command Line Processing.}
%
% The following three command lines generate the output files
% |cdocscld|, |cdocscl1| and |cdocscl2|
% which should be identical to
% |cdocsdrf|, |cdocsch1| and |cdocsfn2|, respectively:
% \begin{center}
% \begin{tabular}{l}
% |latex -jobname cdocscld \|\\
% |  "\def\version{draft}\input{childdoc.def}\childdocforward{cdocsamp}"|\\
% |latex -jobname cdocscl1 \|\\
% |  "\input{childdoc.def}\childdocforward[cdocsamp]{cdocsch1}"|\\
% |latex -jobname cdocscl2 \|\\
% |  "\def\version{final}\input{childdoc.def}\childdocforward{cdocsch2}"|
% \end{tabular}
% \end{center}
% Note that the trailing backslash on each first line
% merely continues the input to the second line
% (for convenient cut ant paste).
% Furthermore, the command |latex| can be replaced by any
% of its alternative versions such as |pdflatex|.
%
% %%%%%%%%%%%%%%%%%%%%%%%%%%%%%%%%%%%%%%%%%%%%%%%%%%%%%%%%%%%%%%%%%%%%%%%%%%%%%%
% %%%%%%%%%%%%%%%%%%%%%%%%%%%%%%%%%%%%%%%%%%%%%%%%%%%%%%%%%%%%%%%%%%%%%%%%%%%%%%
% \section{Implementation}
%\iffalse
%<*package>
%\fi
%
% This section describes the definitions file |childdoc.def|.

% The definitions cannot be loaded using |\usepackage| or |\RequirePackage|
% which has a mechanism to prevent loading a style file more than once.
% When loading the definitions by means of |\input|
% multiple instances have to be prevented manually:
%\iffalse
%This code needs to be before the `\ProvidesFile' directive
%which is defined at the beginning of this file.
%Therefore it is also placed there and commented out here.
%</package>
%<*discard>
%\fi
%    \begin{macrocode}
\ifdefined\childdocmain\endinput\fi
%    \end{macrocode}
%\iffalse
%</discard>
%<*package>
%\fi
%
% \macro{\ifchilddoc}
% \macro{\ifchilddocmanual}
% The conditional |\ifchilddoc| tells whether a
% child (true) or main (false) document is being compiled.
% The conditional |\ifchilddocmanual| tells whether
% the |\includeonly| mechanism is used (false) or
% the selection of child files must be performed manually (true).
% The definitions initialise to false:
%    \begin{macrocode}
\newif\ifchilddoc
\newif\ifchilddocmanual
%    \end{macrocode}

% \macro{\childdocname}
% \macro{\childdocjob}
% The macro |\childdocname| stores the name of the main document
% to be compiled. The macro |\childdocjob| stores the name of
% the document on which the \LaTeX{} compiler was originally invoked.
% The content of |\jobname| cannot be compared
% to filenames specified in the source due to different catcodes.
% The following code rescans |\jobname|, stores the result
% in |\childdocname| and saves a copy in |\childdocjob|:
%    \begin{macrocode}
\edef\childdocname{\scantokens\expandafter{\jobname\noexpand}}
\let\childdocjob\childdocname
%    \end{macrocode}

% \macro{\childdocdisable}
% The macro |\childdocdisable| prevents the main file
% from being processed more than once.
% At this stage, the main document command |\childdocmain|
% is assumed to be called once again where it should do nothing.
% Any subsequent call to it should prevent
% a secondary processing of the main document
% It overwrites the forwarding commands
% |\childdocof| and |\childdocforward|
% with empty macros to prevent further inclusions of the main document:
%    \begin{macrocode}
\newcommand{\childdocdisable}
{
  \renewcommand{\childdocmain}[1]{\renewcommand{\childdocmain}[1]{\endinput}}
  \renewcommand{\childdocof}[1]{}
  \renewcommand{\childdocby}[2][]{}
  \renewcommand{\childdocforward}[2][]{}
  \renewcommand{\childdocdisable}{}
}
%    \end{macrocode}

% \macro{\childdocmain}
% The macro |\childdocmain| is to be called at the top of the main file
% with nothing or the main filename (without extension) as argument.
% First, it breaks loops.
% If the argument is not empty and does not match |\childdocname|
% (which is set by the first inclusion of |childdoc.def|),
% |\ifchilddoc| is set to true, |\includeonly| is applied to the child file
% and |\jobname| is set to the main file
% (for proper handling of |.aux| files):
%    \begin{macrocode}
\newcommand{\childdocmain}[1]
{
  \childdocdisable\childdocmain{}
  \if?#1?\else
    \begingroup
      \def\childdoctmp{#1}
      \ifx\childdoctmp\childdocname
        \def\childdoctmp{}
      \else
        \def\childdoctmp
        {
          \childdoctrue
          \includeonly{\childdocname}
          \def\childdocjob{#1}
          \def\jobname{#1}
        }
      \fi
      \expandafter
    \endgroup
    \childdoctmp
  \fi
}
%    \end{macrocode}

% \macro{\childdocof}
% The command |\childdocof| redirects
% compilation to the main file |#1|.
%    \begin{macrocode}
\newcommand{\childdocof}[1]
{
  \childdocdisable
  \childdoctrue
  \includeonly{\childdocname}
  \def\jobname{#1}
  \def\childdocjob{#1}
  \input{#1}
}
%    \end{macrocode}

% \macro{\childdocby}
% The command |\childdocby| ....
%    \begin{macrocode}
\newcommand{\childdocby}[2][]
{
  \childdocdisable
  \childdoctrue
  \childdocmanualtrue
  \if?#1?\else
    \def\jobname{#2}
  \fi
  \def\childdocjob{#2}
  \input{#2}
  \endinput
}
%    \end{macrocode}

% \macro{\childdocforward}
% The command |\childdocforward| redirects
% compilation to the main file or
% (if the optional argument is given) a child file.
% Parameters are set as if the main file
% or a child file starting with |\childdocof| was compiled.
% Then compilation is handed over to the main file:
%    \begin{macrocode}
\newcommand{\childdocforward}[2][]
{
  \begingroup
    \if?#1?
      \def\childdoctmp
      {
        \def\childdocname{#2}
        \def\childdocjob{#2}
        \def\jobname{#2}
        \input{#2}
        \endinput
      }
    \else
      \def\childdoctmp
      {
        \childdocdisable
        \def\childdocname{#2}
        \childdoctrue
        \includeonly{#2}
        \def\childdocjob{#1}
        \def\jobname{#1}
        \input{#1}
        \endinput
      }
    \fi
    \expandafter
  \endgroup
  \childdoctmp
}
%    \end{macrocode}

% \macro{\childdocforwardprefix}
% The command |\childdocforwardprefix| redirects
% compilation to the main or a child file by means of a pattern.
% The prefix |#1| in the current filename is replaced by |#2|
% and the suffix of the current filename is kept
% (it is assumed that the filename does not contain the substring `|~~~|'
% which is used as a delimiter).
% Compilation is handed over to the new file by |\childdocforward|:
%    \begin{macrocode}
\newcommand{\childdocforwardprefix}[3][]
{
  \begingroup
    \def\childdocextract #2##1~~~{\def\childdoctmp{\childdocforward[#1]{#3##1}}}
    \expandafter\childdocextract\childdocname~~~
    \expandafter
  \endgroup
  \childdoctmp
}
%    \end{macrocode}

% \macro{\childdoc}
% The deprecated macro |\childdoc| is a legacy version of |\childdocmain|:
%    \begin{macrocode}
\newcommand{\childdoc}{\childdocmain}
%    \end{macrocode}

% \macro{\childdocredirect}
% The deprecated macro |\childdocredirect| is a legacy version
% of |\childdocforward| and |\childdocforwardprefix|:
%    \begin{macrocode}
\newcommand{\childdocredirect}[2][]
{
  \begingroup
    \if?#1?
      \def\childdoctmp{\childdocforward{#2}}
    \else
      \def\childdoctmp{\childdocforwardprefix{#1}{#2}}
    \fi
    \expandafter
  \endgroup
  \childdoctmp
}
%    \end{macrocode}

%\iffalse
%</package>
%\fi
%
\endinput

\childdocmain{}
%    \end{macrocode}

% Optional override for |\version| flag:
%    \begin{macrocode}
%%\ifchilddoc\else\providecommand{\version}{draft}\fi
%    \end{macrocode}

% Define the default values for the |\version| flag
% (|final| for the main file and |draft| for childs):
%    \begin{macrocode}
\ifchilddoc
\providecommand{\version}{draft}
\else
\providecommand{\version}{final}
\fi
%    \end{macrocode}

% Load the standard document class:
%    \begin{macrocode}
\documentclass[12pt]{article}
%    \end{macrocode}

% Start the document body:
%    \begin{macrocode}
\begin{document}
%    \end{macrocode}

% Declare a title page.
% Print title, part of document being processed and version flag:
%    \begin{macrocode}
\addtocounter{page}{-1}
\begin{center}
{\LARGE\bfseries{}childdoc example\par}
\vspace{1cm}
\ifchilddoc
\ifchilddocmanual part\else chapter\fi:
`\childdocname' of `\childdocjob'\par
\else
main document: `\childdocjob'\par
\fi
version: \version\par
\end{center}
\newpage
%    \end{macrocode}

% Manually include selected file,
% otherwise process as usual:
%    \begin{macrocode}
\ifchilddocmanual
\section*{part `\childdocname'}
\input{\childdocname}
\else
%    \end{macrocode}

% Include the two chapters:
%    \begin{macrocode}
\include{cdocsch1}
\include{cdocsch2}
%    \end{macrocode}

% Include the two parts unless only chapters should be displayed:
%    \begin{macrocode}
\ifchilddoc\else
\section{part three}
\input{cdocspt3}
\section{part four}
\input{cdocspt4}
\fi
%    \end{macrocode}

% Process as usual until here:
%    \begin{macrocode}
\fi
%    \end{macrocode}

% End of document body:
%    \begin{macrocode}
\end{document}
%    \end{macrocode}
%\iffalse
%</samplemain>
%\fi
%
% %%%%%%%%%%%%%%%%%%%%%%%%%%%%%%%%%%%%%%
% \paragraph{Chapter Include Files.}
%
% The include files are called |cdocsch1.tex| and |cdocsch2.tex|.
%
%\iffalse
%<*samplechap1|samplechap2>
%\fi

% Optional override for |\version| flag:
%    \begin{macrocode}
%%\providecommand{\version}{final}
%    \end{macrocode}

% Include the main document:
%    \begin{macrocode}
% \iffalse
%
% childdoc.dtx Copyright (C) 2017-2018 Niklas Beisert
%
% This work may be distributed and/or modified under the
% conditions of the LaTeX Project Public License, either version 1.3
% of this license or (at your option) any later version.
% The latest version of this license is in
%   http://www.latex-project.org/lppl.txt
% and version 1.3 or later is part of all distributions of LaTeX
% version 2005/12/01 or later.
%
% This work has the LPPL maintenance status `maintained'.
%
% The Current Maintainer of this work is Niklas Beisert.
%
% This work consists of the files childdoc.dtx and childdoc.ins
% and the derived files childdoc.def and cdocsamp.tex with
% cdocsch1.tex, cdocsch2.tex, cdocsdrf.tex, cdocsfn1.tex, cdocsfn2.tex.
%
%<package>\ifdefined\childdocmain\endinput\fi
%<package>\ProvidesFile{childdoc.def}[2018/12/30 v2.0 child document driver]
%<samplemain>\ProvidesFile{cdocsamp.tex}[2018/12/30 v2.0 sample for childdoc]
%<*driver>
%\ProvidesFile{childdoc.drv}[2018/12/30 v2.0 childdoc reference manual file]
\PassOptionsToClass{10pt,a4paper}{article}
\documentclass{ltxdoc}

\usepackage[margin=35mm]{geometry}
\usepackage{hyperref}
\usepackage{hyperxmp}
\usepackage[usenames]{color}

\hypersetup{colorlinks=true}
\hypersetup{pdfstartview=FitH}
\hypersetup{pdfpagemode=UseNone}
\hypersetup{pdfsource={}}
\hypersetup{pdflang={en-UK}}
\hypersetup{pdfcopyright={Copyright 2017-2018 Niklas Beisert.
  This work may be distributed and/or modified under the
  conditions of the LaTeX Project Public License, either version 1.3
  of this license or (at your option) any later version.}}
\hypersetup{pdflicenseurl={http://www.latex-project.org/lppl.txt}}
\hypersetup{pdfcontactaddress={ETH Zurich, ITP, HIT K,
  Wolfgang-Pauli-Strasse 27}}
\hypersetup{pdfcontactpostcode={8093}}
\hypersetup{pdfcontactcity={Zurich}}
\hypersetup{pdfcontactcountry={Switzerland}}
\hypersetup{pdfcontactemail={nbeisert@itp.phys.ethz.ch}}
\hypersetup{pdfcontacturl={http://people.phys.ethz.ch/\xmptilde nbeisert/}}

\newcommand{\secref}[1]{\hyperref[#1]{section \ref*{#1}}}

\parskip1ex
\parindent0pt
\let\olditemize\itemize
\def\itemize{\olditemize\parskip0pt}

\begin{document}

\title{The \textsf{childdoc} Package}
\hypersetup{pdftitle={The childdoc Package}}
\author{Niklas Beisert\\[2ex]
  Institut f\"ur Theoretische Physik\\
  Eidgen\"ossische Technische Hochschule Z\"urich\\
  Wolfgang-Pauli-Strasse 27, 8093 Z\"urich, Switzerland\\[1ex]
  \href{mailto:nbeisert@itp.phys.ethz.ch}
  {\texttt{nbeisert@itp.phys.ethz.ch}}}
\hypersetup{pdfauthor={Niklas Beisert}}
\hypersetup{pdfsubject={Manual for the LaTeX2e Package childdoc}}
\date{30 December 2018, \textsf{v2.0}}
\maketitle

\begin{abstract}\noindent
\textsf{childdoc} is a \LaTeXe{} package
that enables the direct compilation
of document sections included by |\include|
to individual files.
\end{abstract}

\begingroup
\parskip0ex
\tableofcontents
\endgroup

%%%%%%%%%%%%%%%%%%%%%%%%%%%%%%%%%%%%%%%%%%%%%%%%%%%%%%%%%%%%%%%%%%%%%%%%%%%%%%%%
%%%%%%%%%%%%%%%%%%%%%%%%%%%%%%%%%%%%%%%%%%%%%%%%%%%%%%%%%%%%%%%%%%%%%%%%%%%%%%%%
\section{Introduction}

\LaTeX{} provides a mechanism to structure a large document (such as a book)
into a main file and several child files (containing the chapters)
using the |\include| command.
This mechanism is beneficial for documents
which span hundreds of pages in order to
make the source file(s) more manageable.
Moreover, compilation can be restricted to
selected child files by means of the |\includeonly| command.
The latter feature can be used to reduce the compilation time while editing
(this was significantly more useful in the earlier days of \LaTeX{})
or to generate a smaller document which is easier to navigate.
Another application of |\includeonly| is to generate
documents consisting of selected parts of the complete document.

However, there are a few drawbacks of the plain |\include| mechanism:
\begin{itemize}
\item
The child files cannot be compiled on their own,
they can only be compiled via the main file.
A naive editing environment
(such as a text editor with an option
to have the current file processed by \LaTeX)
may require one to switch to the main file before compiling;
attempting to compile the child file produces errors.
\item
The main file must be modified (each time)
to adjust the |\includeonly| command
to the present needs. This easily leaves the main file in a messy state.
\item
The generated document will always carry the filename
of the main document. This is inconvenient if
several child files are to be compiled and
to be kept for distribution.
\end{itemize}

The present package provides a simple interface
to make child files individually compilable by \LaTeX{}.
Compiling a child file then has the same effect as compiling
the main file with an |\includeonly| command
to select the appropriate child.
Moreover the generated document will carry the name of the child
rather than the main file.
This resolves all three above issues.

This feature is meant to make the editing of books,
thesis documents and lecture notes somewhat more convenient.
However, the package can also be used efficiently for
composing a series of documents (such as exercise sheets)
which are typically distributed individually.
It then assists the author in generating the individual documents
(potentially in different versions)
as well as a document containing the collected series.
Another application is in developing style files
or other kinds of included material
where compilation of the style file could redirect
to a sample or test file.

%%%%%%%%%%%%%%%%%%%%%%%%%%%%%%%%%%%%%%%%%%%%%%%%%%%%%%%%%%%%%%%%%%%%%%%%%%%%%%%%
%%%%%%%%%%%%%%%%%%%%%%%%%%%%%%%%%%%%%%%%%%%%%%%%%%%%%%%%%%%%%%%%%%%%%%%%%%%%%%%%
\section{Usage}

First of all, the package \textsf{childdoc} is \emph{not} a standard
\LaTeXe{} |.sty| style file! Therefore it needs to be invoked in
a non-standard way.

%%%%%%%%%%%%%%%%%%%%%%%%%%%%%%%%%%%%%%%%%%%%%%%%%%%%%%%%%%%%%%%%%%%%%%%%%%%%%%%%
\subsection{Included Files}
\label{sec:include}

%%%%%%%%%%%%%%%%%%%%%%%%%%%%%%%%%%%%%%%%
\DescribeMacro{\childdocmain}
To use the package, add the commands
\begin{center}
\begin{tabular}{l}
|\input{childdoc.def}|\\
|\childdocmain{}|\\
\end{tabular}
\end{center}
at the very top of the main \LaTeX{} file,
in particular \emph{before} the |\documentclass| statement!
The argument of |\childdocmain| should be left empty
(but it must be present).

%%%%%%%%%%%%%%%%%%%%%%%%%%%%%%%%%%%%%%%%
\DescribeMacro{\childdocof}
Furthermore, add the commands
\begin{center}
\begin{tabular}{l}
|\input{childdoc.def}|\\
|\childdocof{|\textit{main}|}|\\
\end{tabular}
\end{center}
at the top of every child file \textit{child}
which is included by |\include{|\textit{child}|}|
from within the main file
(or at least for those files to be compiled individually).
The argument \textit{main} must be the filename of the main file.

There are a couple of
considerations in setting up the main and child documents:

%%%%%%%%%%%%%%%%%%%%%%%%%%%%%%%%%%%%%%%%
\paragraph{Restrictions.}

Please note the following restrictions:
\begin{itemize}
\item
|\childdocmain| must be called with one argument \textit{main}
to ensure compatibility with earlier version of the package.
It must either be empty (|\childdocmain{}|)
or precisely match the filename of the main file in which it is specified.
See \secref{sec:detection} for further information.
\item
The filename \textit{main} must be specified without the |.tex| extension.
\item
The filename \textit{main} is case sensitive
(even in case-insensitive file systems)
due to internal string comparison.
\item
The argument \textit{main} should be fully expanded, it cannot be a macro.
\item
Subdirectories and special characters should be avoided in filenames.
\item
The command |\childdocmain{|\textit{main}|}| must be followed by a whitespace.
It should not be followed immediately by another command
or by a comment mark `|%|'.
This is because the \TeX{} parser reads the token immediately following
the argument of |\childdocmain| and puts it
at the beginning of every child section;
however, a white\-space is ignored.
\end{itemize}

%%%%%%%%%%%%%%%%%%%%%%%%%%%%%%%%%%%%%%%%
\paragraph{Content of Main File.}

It is advisable to place all content in the child files included by |\include|.
Any output contained in the main file will appear in all child documents
unless suppressed manually;
it cannot be suppressed automatically by the |\includeonly| directive
and thus should normally be avoided.
A method to include some content in the main file
by means of conditional processing is described in \secref{sec:conditional}.

%%%%%%%%%%%%%%%%%%%%%%%%%%%%%%%%%%%%%%%%
\paragraph{Page Numbering.}

When only a part of the document is compiled,
the appropriate numbering of pages
(as well as other status parameters)
is determined from the |.aux| files.
The latter contain information from previous passes.
However this information needs to propagate through
all intermediate child documents.
Therefore the page numbering in child documents may well
be inconsistent until the complete document is compiled at least once.

A useful (if unconventional) way to always ensure a consistent
page numbering is to restart the numbering in each child document
and denote the pages by `\textit{child}|.|\textit{page}'
where \textit{child} represents the chapter/section number of the child file.
This can be achieved by the command
|\numberwithin{page}{|\textit{child}|}|
of the \textsf{amsmath} package
where \textit{child} can be |chapter| or |section|
depending on the chosen structuring.
Alternatively, one can modify the macro |\thepage| appropriately
and reset the counter |page| at the start of each child file.

%%%%%%%%%%%%%%%%%%%%%%%%%%%%%%%%%%%%%%%%%%%%%%%%%%%%%%%%%%%%%%%%%%%%%%%%%%%%%%%%
\subsection{Conditional Processing}
\label{sec:conditional}

The package provides a mechanism to compile different versions
of a document. To customise the versions further some conditional processing
can come in handy to distinguish which version is being compiled.
The package provides two macros to describe the compilation context:

%%%%%%%%%%%%%%%%%%%%%%%%%%%%%%%%%%%%%%%%
\DescribeMacro{\ifchilddoc}
The conditional |\ifchilddoc| distinguishes between the compilation of
child documents and the main document:
%
\begin{center}
|\ifchilddoc |\textit{child-code}| |[|\||else |\textit{main-code}]| \||fi|
\end{center}

%%%%%%%%%%%%%%%%%%%%%%%%%%%%%%%%%%%%%%%%
\DescribeMacro{\childdocname}
\DescribeMacro{\childdocjob}
The macro |\childdocname| contains the filename (without extension)
of the main or child file being processed.
Note that |\childdocjob| will always contain the name of the main file.

%%%%%%%%%%%%%%%%%%%%%%%%%%%%%%%%%%%%%%%%
\paragraph{Title Page.}

Conditional processing can be used to include a title or banner page
in the main document when proper precautions are taken.
Importantly, the code in the main file should ensure that the page counter
(as well as other status parameters which are stored in the |.aux| files)
takes the same value after the conditional processing.
Otherwise the page numbers may take divergent values
depending on which part is compiled.

For example, a title page could be declared by:
%
\begin{center}
\begin{tabular}{l}
|\ifchilddoc\||else|\\
|\addtocounter{page}{-1}|\\
\textit{code for title page}\\
|\newpage|\\
|\||fi|
\end{tabular}
\end{center}
%
A banner page for the child documents can be generated by:
%
\begin{center}
\begin{tabular}{l}
|\ifchilddoc|\\
|\addtocounter{page}{-1}|\\
\textit{code for banner page}\\
|\newpage|\\
|\||fi|
\end{tabular}
\end{center}
%
Here one could write a message such as:
\begin{center}
|This is the part \childdocname{} of \childdocjob{}.|
\end{center}

%%%%%%%%%%%%%%%%%%%%%%%%%%%%%%%%%%%%%%%%%%%%%%%%%%%%%%%%%%%%%%%%%%%%%%%%%%%%%%%%
\subsection{Flags}
\label{sec:flags}

The package makes it easy to generate different versions
of the main or child documents.
To this end compilation flags can be defined
and assigned different default values.
They will be particularly useful in conjunction
with the forwarding mechanism described in \secref{sec:forward}.

For example, it may be useful to have a flag |\version|
which can be set to |draft| or |final|.
The document source will contain some conditional code
depending on the value of |\version|.
Suppose further, the flag should default to |final| for the main file
and to |draft| for child files
which is a natural assignment for editing the document.
This is achieved by placing the following code
in the preamble of the main document
(below the |\childdocmain| directive):
%
\begin{center}
\begin{tabular}{l}
|\ifchilddoc|\\
|\providecommand{\version}{draft}|\\
|\||else|\\
|\providecommand{\version}{final}|\\
|\||fi|
\end{tabular}
\end{center}
%
The definition by |\providecommand| makes sure
that previous definitions are not overwritten.
Further statements |\providecommand{\version}{...}|
can thus be added before the above code to override it.

For the main file, one might add a line
(between |\childdocmain| and the above block)
%
\begin{center}
|%\ifchilddoc\||else\providecommand{\version}{draft}\||fi|
\end{center}
%
which can be uncommented to produce a draft version.
Likewise one can add a line to the very top of a child file
(above the |\childdocof{|\textit{main}|}| directive)
%
\begin{center}
|%\providecommand{\version}{final}|
\end{center}
%
which can be uncommented to produce the final version of this child document.

%%%%%%%%%%%%%%%%%%%%%%%%%%%%%%%%%%%%%%%%%%%%%%%%%%%%%%%%%%%%%%%%%%%%%%%%%%%%%%%%
\subsection{Forwarding}
\label{sec:forward}

Different versions of the main or child documents
using compilation flags as described in \secref{sec:flags}
can be (permanently) stored in different files
for convenient compilation, viewing and distribution.
To this end, the package defines a command
to pass on compilation to a different file:

%%%%%%%%%%%%%%%%%%%%%%%%%%%%%%%%%%%%%%%%
\DescribeMacro{\childdocforward}
The command |\childdocforward| redirects processing to
another source file:
%
\begin{center}
\begin{tabular}{l}
|\input{childdoc.def}|\\
|\childdocforward[|\textit{main}|]{|\textit{dest}|}|\\
\end{tabular}
\end{center}
%
The argument \textit{dest} is the destination file
(without extension).
It should be the main file or one of the child files.
Note that further \textsf{childdoc} directives
such as |\childdocof| and |\childdocforward|
in the indicated file will be processed in this form.
The optional argument \textit{main}
passes on directly to the main file \textit{main}
while pretending to compile the child \textit{dest}.
This form behaves as if \textit{dest}
issues |\childdocof{|\textit{main}|}| right away,
and no further \textsf{childdoc} directives will be processed.

%%%%%%%%%%%%%%%%%%%%%%%%%%%%%%%%%%%%%%%%
\DescribeMacro{\...prefix}
In the alternative form |\childdocforwardprefix|,
%
\begin{center}
\begin{tabular}{l}
|\input{childdoc.def}|\\
|\childdocforwardprefix[|\textit{main}|]{|\textit{prefix}|}{|\textit{dest}|}|
\end{tabular}
\end{center}
%
the destination file is determined by a pattern
depending on the current file:
To make this work, the current file must be called
`{\textit{prefix}\hspace{0.2em}\textit{suffix}}'
with \textit{prefix} matching precisely the argument.
Processing is then passed on to the file
`{\textit{dest}\hspace{0.2em}\textit{suffix}}'.
Surely, the same effect is achieved by
directly specifying the
argument `{\textit{dest}\hspace{0.2em}\textit{suffix}}'
in the first form.
However, that requires to set up a different file
for each child. With the alternative form of the command
all these files can have exactly the same content
which simplifies setting them up and maintaining them.

For example, the following file |draft.tex|
with a compilation flag |\version| as described in \secref{sec:flags}
compiles the main document as a draft:
%
\begin{center}
\begin{tabular}{l}
|\def\version{draft}|\\
|\input{childdoc.def}|\\
|\childdocforward{|\textit{main}|}|
\end{tabular}
\end{center}
%
Likewise, the following files |final|\textit{nn}|.tex|
compile the final version of the child document
|child|\textit{nn}|.tex|:
%
\begin{center}
\begin{tabular}{l}
|\def\version{final}|\\
|\input{childdoc.def}|\\
|\childdocforwardprefix{final}{child}|
\end{tabular}
\end{center}
%

Note that when several versions of a main file and/or of each child file
are to be generated, it may be convenient to set up a |Makefile| or
shell script to automatise the process.

%%%%%%%%%%%%%%%%%%%%%%%%%%%%%%%%%%%%%%%%%%%%%%%%%%%%%%%%%%%%%%%%%%%%%%%%%%%%%%%%
\subsection{Command Line Processing}
\label{sec:commandline}

The effect of redirection files can also be achieved by invoking
the \LaTeX{} compiler with a more elaborate command line.
Most conveniently this should be done as part
of a shell script or a |Makefile|.

When using \textsf{childdoc} in the main file, the following
command lines effectively perform a redirection
(note that depending on the shell being used,
backslashes may have to be doubled: `|\|' $\to$ `|\\|'):
%
\begin{center}
|... -jobname "|\textit{target}|" |\\|"|[\textit{flags}]%
|\input{childdoc.def}\childdocforward[|\textit{main}|]{|\textit{dest}|}"|
\end{center}
%
Here \textit{target} is the name of the output file,
\textit{main} is the name of the main file
and \textit{dest} is the name of the main or child file to be processed
(all filenames without extensions).
The optional argument \textit{main} can be omitted
if \textit{main} matches \textit{dest}.
Optionally, compilation \textit{flags} can be defined via |\def| commands.
This command line makes the \TeX{} engine believe
it is compiling the file \textit{target}
whose content is specified as the latter parameter.
The provided code then forwards the processing to
\textit{main} or \textit{dest} as described in \secref{sec:forward}.

%%%%%%%%%%%%%%%%%%%%%%%%%%%%%%%%%%%%%%%%%%%%%%%%%%%%%%%%%%%%%%%%%%%%%%%%%%%%%%%%
\subsection{Include by Input}
\label{sec:input}

Including child documents by |\include| has some restrictions by design.
Most notably, the content of a child document always occupies
its own set of pages; pages cannot be shared between child documents.
Usually, this behaviour makes perfect sense
because each child document contain an essential part of the document.
However, in some situations it may be desirable to compose
a document from a collection of parts
without having mandatory page breaks between then.
For this case, the package
provides a mechanism to include parts
by |\input| which can also be processed individually.
However, by construction this mechanism
requires manual handling of the content to be output.

%%%%%%%%%%%%%%%%%%%%%%%%%%%%%%%%%%%%%%%%
\DescribeMacro{\ifchilddocmanual}
The main file should be prepared as usual, see \secref{sec:include}.
However, the document body must make a distinction
between processing of an individual part and of the main document, e.g.:
%
\begin{center}
\begin{tabular}{l}
|\ifchilddocmanual|\\
|\input{\childdocname}|\\
|\||else|\\
\textit{document body with }|\input{|\textit{part}|}|\\
|\||fi|
\end{tabular}
\end{center}
%
The conditional |\ifchilddocmanual| is true whenever
a part to be included by |\input| is being compiled,
and the name of the part is stored in |\childdocname|.

%%%%%%%%%%%%%%%%%%%%%%%%%%%%%%%%%%%%%%%%
\DescribeMacro{\childdocby}
Each part to be included by |\input| should start with:
%
\begin{center}
\begin{tabular}{l}
|\input{childdoc.def}|\\
|\childdocby{|\textit{main}|}|\\
\end{tabular}
\end{center}
%
The directive |\childdocby| is similar to |\childdocof|
described in \secref{sec:include},
but the subsequent selection of content must be done manually.
To that end, both |\ifchilddoc| and |\ifchilddocmanual|
will be true upon processing of a part,
and the name of the part is stored in |\childdocname|.
Note that |\jobname| will be set to the filename of the current part
so that each part receives an individual |.aux| file
that does not interfere with the |.aux| file(s) of the main document.
This behaviour can be altered by the alternative form
|\childdocby[*]{|\textit{main}|}| (with a non-empty optional argument)
which uses the |.aux| file of the main document
by setting |\jobname| to \textit{main}.

%%%%%%%%%%%%%%%%%%%%%%%%%%%%%%%%%%%%%%%%%%%%%%%%%%%%%%%%%%%%%%%%%%%%%%%%%%%%%%%%
\subsection{Driver Development}
\label{sec:driver}

The \textsf{childdoc} mechanism can also be use for the development
of definition files such as \LaTeX{} styles or classes.
This case differs from the above setup with multiple parts
included by |\include| in that no |\includeonly| should be invoked.
This can be achieved by starting the include file
(before |\ProvidesPackage|) with:
%
\begin{center}
\begin{tabular}{l}
|\input{childdoc.def}|\\
|\childdocforward{|\textit{main}|}|\\
\end{tabular}
\end{center}
%
or alternatively with:
%
\begin{center}
\begin{tabular}{l}
|\input{childdoc.def}|\\
|\childdocby{|\textit{main}|}|\\
\end{tabular}
\end{center}
%
Both forms have slightly different effects as described above.
The main file is prepared as usual, see \secref{sec:include}.

%%%%%%%%%%%%%%%%%%%%%%%%%%%%%%%%%%%%%%%%%%%%%%%%%%%%%%%%%%%%%%%%%%%%%%%%%%%%%%%%
\subsection{Legacy Detection}
\label{sec:detection}

The directive |\childdocmain| in the main file can detect
whether the complete document or merely a child is to be compiled
even without using the directive |\childdocof|.
This method is deprecated because it is less robust
and there is no compelling reason to use it;
it is merely provided for backward compatibility
and it may be removed in future versions.

If the detection mechanism is to be used,
it is mandatory to correctly specify
the filename of the main file as the argument of |\childdocmain|:
%
\begin{center}
\begin{tabular}{l}
|\input{childdoc.def}|\\
|\childdocmain{|\textit{main}|}|\\
\end{tabular}
\end{center}
%
If |\jobname| does not match the argument \textit{main} of |\childdocmain|,
it is assumed that |\jobname| points to the child file to be compiled.
When using |\childdocmain| with the main file specified as argument,
it suffices to start a child file
with just |\input{|\textit{main}|}|
without loading of the package and using |\childdocof|.
If instead all processing is done
with the appropriate \textsf{childdoc} directives,
the argument of \textit{main} of |\childdocmain| can be empty.

An alternative version of the command line processing described
in \secref{sec:commandline} using the detection mechanism reads:
%
\begin{center}
|... -jobname "|\textit{target}|" "|[\textit{flags}]%
[|\def\jobname{|\textit{dest}|}|]|\input{|\textit{main}|}"|
\end{center}

%%%%%%%%%%%%%%%%%%%%%%%%%%%%%%%%%%%%%%%%%%%%%%%%%%%%%%%%%%%%%%%%%%%%%%%%%%%%%%%%
\subsection{Manual Code}
\label{sec:manual}

In case one cannot be certain whether the definitions file |childdoc.def|
is installed on the target \TeX{} distribution
and one prefers not to ship it,
it is conceivable to paste a few relevant commands into the sources.

To that end, drop all statements |\input{childdoc.def}|
and perform the replacements as outlined below.
Instead of |\childdocmain{|\textit{main}|}| add the following code
to the top of the main file:
%
\begin{center}
\begin{tabular}{l}
|\||ifdefined\childdocname\endinput\||fi\newif\ifchilddoc|\\
|\edef\childdocname{\scantokens\expandafter{\jobname\noexpand}}|\\
|\def\childdocmain{|\textit{main}|}\||ifx\childdocmain\childdocname\||else|\\
|\childdoctrue\includeonly{\childdocname}\let\jobname\childdocmain\||fi|\\
\end{tabular}
\end{center}
%
Instead of |\childdocof{|\textit{main}|}| just include the main file
at the top of each child file:
%
\begin{center}
|\input{|\textit{main}|}|
\end{center}
%
A simple redirection |\childdocforward{|\textit{dest}|}| is achieved by:
%
\begin{center}
|\def\jobname{|\textit{dest}|}\input{\jobname}|
\end{center}
%
The redirection with prefix
|\childdocforwardprefix[|\textit{prefix}|]{|\textit{dest}|}|
is accomplished by:
%
\begin{center}
\begin{tabular}{l}
|{\edef\jobname{\scantokens\expandafter{\jobname\noexpand}}|\\
|\def\redirectjob |\textit{prefix}|#1~~~{\gdef\jobname{|\textit{dest}|#1}}|\\
|\expandafter\redirectjob\jobname~~~}\input{\jobname}|
\end{tabular}
\end{center}

In an alternative approach,
child documents can be compiled by a specific command line
without additional code or specific definitions:
%
\begin{center}
|... -jobname "|\textit{target}|" "|[\textit{flags}]%
|\includeonly{|\textit{dest}|}\input{|\textit{main}|}"|
\end{center}
%

%%%%%%%%%%%%%%%%%%%%%%%%%%%%%%%%%%%%%%%%%%%%%%%%%%%%%%%%%%%%%%%%%%%%%%%%%%%%%%%%
%%%%%%%%%%%%%%%%%%%%%%%%%%%%%%%%%%%%%%%%%%%%%%%%%%%%%%%%%%%%%%%%%%%%%%%%%%%%%%%%
\section{Information}

%%%%%%%%%%%%%%%%%%%%%%%%%%%%%%%%%%%%%%%%%%%%%%%%%%%%%%%%%%%%%%%%%%%%%%%%%%%%%%%%
\subsection{Copyright}

Copyright \copyright{} 2017--2018 Niklas Beisert

This work may be distributed and/or modified under the
conditions of the \LaTeX{} Project Public License, either version 1.3
of this license or (at your option) any later version.
The latest version of this license is in
  \url{http://www.latex-project.org/lppl.txt}
and version 1.3 or later is part of all distributions of \LaTeX{}
version 2005/12/01 or later.

This work has the LPPL maintenance status `maintained'.

The Current Maintainer of this work is Niklas Beisert.

This work consists of the files |README.txt|, |childdoc.ins| and |childdoc.dtx|
as well as the derived files |childdoc.def|, |cdocsamp.tex|
with |cdocsch1.tex|, |cdocsch2.tex|, |cdocspt3.tex|, |cdocspt4.tex|,
|cdocsdrf.tex|, |cdocsfn1.tex|, |cdocsfn2.tex|
as well as |childdoc.pdf|.

%%%%%%%%%%%%%%%%%%%%%%%%%%%%%%%%%%%%%%%%%%%%%%%%%%%%%%%%%%%%%%%%%%%%%%%%%%%%%%%%
\subsection{Files and Installation}

The package consists of the files:
%
\begin{center}
\begin{tabular}{ll}
    |README.txt|   & readme file \\
    |childdoc.ins| & installation file \\
    |childdoc.dtx| & source file \\
    |childdoc.def| & definition file \\
    |cdocsamp.tex| & sample main file \\
    |cdocsch1.tex| & sample include file \\
    |cdocsch2.tex| & sample include file \\
    |cdocspt3.tex| & sample part file \\
    |cdocspt4.tex| & sample part file \\
    |cdocsdrf.tex| & sample redirection file \\
    |cdocsfn1.tex| & sample redirection file \\
    |cdocsfn2.tex| & sample redirection file \\
    |childdoc.pdf| & manual
\end{tabular}
\end{center}
%
The distribution consists of the files
|README.txt|, |childdoc.ins| and |childdoc.dtx|.
%
\begin{itemize}
\item
Run (pdf)\LaTeX{} on |childdoc.dtx|
to compile the manual |childdoc.pdf| (this file).
\item
Run \LaTeX{} on |childdoc.ins| to create the definitions file |childdoc.def|
and the sample |cdocsamp.tex| with include files
|cdocsch1.tex|, |cdocsch2.tex|, |cdocspt3.tex|, |cdocspt4.tex|,
|cdocsdrf.tex|, |cdocsfn1.tex|, |cdocsfn2.tex|.
Then copy the file |childdoc.def| to an appropriate directory of your \LaTeX{}
distribution, e.g.\ \textit{texmf-root}|/tex/latex/childdoc|.
\end{itemize}

%%%%%%%%%%%%%%%%%%%%%%%%%%%%%%%%%%%%%%%%%%%%%%%%%%%%%%%%%%%%%%%%%%%%%%%%%%%%%%%%
\subsection{Related CTAN Packages}

There are several other packages which offer a similar functionality:
%
\begin{itemize}
\item
The packages
\href{http://ctan.org/pkg/docmute}{\textsf{docmute}},
\href{http://ctan.org/pkg/includex}{\textsf{includex}} and
\href{http://ctan.org/pkg/standalone}{\textsf{standalone}}
provide commands to include only the document body of
a child file thus allowing both files to be compiled individually.
\item
The packages \href{http://ctan.org/pkg/subdocs}{\textsf{subdocs}}
and \href{http://ctan.org/pkg/subfiles}{\textsf{subfiles}}
provide structures in which the main and child documents can be
encapsulated and allowing them to be compiled individually.
The inclusion mechanism is different from the conventional |\include|.
\item
The package \href{http://ctan.org/pkg/combine}{\textsf{combine}}
is an elaborate solution to combine several documents into one.
\end{itemize}
%
See also the CTAN topic \href{http://ctan.org/topic/subdocs}{\textsf{subdocs}}
for further related packages.
The present package differs from the above solutions in that
a document structure constructed with the conventional |\include| mechanism
just needs two extra commands at the top of every file
such that all constituent files can be compiled individually.

%%%%%%%%%%%%%%%%%%%%%%%%%%%%%%%%%%%%%%%%%%%%%%%%%%%%%%%%%%%%%%%%%%%%%%%%%%%%%%%%
%\subsection{Feature Suggestions}
%
%The following is a list of features which may be useful for future
%versions of this package:
%%
%\begin{itemize}
%\item
%\ldots
%\end{itemize}

%%%%%%%%%%%%%%%%%%%%%%%%%%%%%%%%%%%%%%%%%%%%%%%%%%%%%%%%%%%%%%%%%%%%%%%%%%%%%%%%
\subsection{Revision History}

%%%%%%%%%%%%%%%%%%%%%%%%%%%%%%%%%%%%%%%%
\paragraph{v2.0:} 2018/12/30

\begin{itemize}
\item
immediate forward processing
\item
added |\childdocby| mechanism
\item
manual restructured
\end{itemize}

%%%%%%%%%%%%%%%%%%%%%%%%%%%%%%%%%%%%%%%%
\paragraph{v1.6:} 2018/01/17

\begin{itemize}
\item
application for development of include files
\item
corrections to manual
\end{itemize}

%%%%%%%%%%%%%%%%%%%%%%%%%%%%%%%%%%%%%%%%
\paragraph{v1.5:} 2017/05/21

\begin{itemize}
\item
more complete structuring introduced
\item
|\childdocof| introduced
\item
|\childdoc| renamed to |\childdocmain|
\item
|\childredirect| renamed to |\childdocforward| and |\childdocforwardprefix|
and functionality expanded
\end{itemize}

%%%%%%%%%%%%%%%%%%%%%%%%%%%%%%%%%%%%%%%%
\paragraph{v1.0:} 2017/04/27

\begin{itemize}
\item
manual and install package
\item
first version published on CTAN
\end{itemize}

%%%%%%%%%%%%%%%%%%%%%%%%%%%%%%%%%%%%%%%%
\paragraph{v0.6:} 2017/04/26

\begin{itemize}
\item
redirection mechanism added
\end{itemize}

%%%%%%%%%%%%%%%%%%%%%%%%%%%%%%%%%%%%%%%%
\paragraph{v0.5:} 2017/04/26

\begin{itemize}
\item
functionality in definition file
\end{itemize}


%%%%%%%%%%%%%%%%%%%%%%%%%%%%%%%%%%%%%%%%%%%%%%%%%%%%%%%%%%%%%%%%%%%%%%%%%%%%%%%%
%%%%%%%%%%%%%%%%%%%%%%%%%%%%%%%%%%%%%%%%%%%%%%%%%%%%%%%%%%%%%%%%%%%%%%%%%%%%%%%%
%%%%%%%%%%%%%%%%%%%%%%%%%%%%%%%%%%%%%%%%%%%%%%%%%%%%%%%%%%%%%%%%%%%%%%%%%%%%%%%%
\appendix

\settowidth\MacroIndent{\rmfamily\scriptsize 000\ }

 \DocInput{childdoc.dtx}

\end{document}
%</driver>
% \fi
%
% %%%%%%%%%%%%%%%%%%%%%%%%%%%%%%%%%%%%%%%%%%%%%%%%%%%%%%%%%%%%%%%%%%%%%%%%%%%%%%
% %%%%%%%%%%%%%%%%%%%%%%%%%%%%%%%%%%%%%%%%%%%%%%%%%%%%%%%%%%%%%%%%%%%%%%%%%%%%%%
% \section{Sample}
%\iffalse
%<*samplemain>
%\fi
%
% The following presents a sample document
% with two chapters, two parts, a title page,
% a compile flag as well as three forwarding files to set the flag.
% It consists of eight |.tex| files:
% \begin{center}
% \begin{tabular}{ll}
% |cdocsamp.tex|&main file\\
% |cdocsch1.tex|&include file for chapter 1\\
% |cdocsch2.tex|&include file for chapter 2\\
% |cdocspt3.tex|&include file for part 3\\
% |cdocspt4.tex|&include file for part 4\\
% |cdocsdrf.tex|&forwarding file for main file in draft mode\\
% |cdocsfi1.tex|&forwarding file for final version of chapter 1\\
% |cdocsfi2.tex|&forwarding file for final version of chapter 2\\
% \end{tabular}
% \end{center}
% Each of the eight files can be compiled directly by the \LaTeX{} compiler.
%
% %%%%%%%%%%%%%%%%%%%%%%%%%%%%%%%%%%%%%%
% \paragraph{Main File.}
%
% The main file is called |cdocsamp.tex|.
%
% Load the \textsf{childdoc} definitions and
% declare the filename for the main document:
%    \begin{macrocode}
\input{childdoc.def}
\childdocmain{}
%    \end{macrocode}

% Optional override for |\version| flag:
%    \begin{macrocode}
%%\ifchilddoc\else\providecommand{\version}{draft}\fi
%    \end{macrocode}

% Define the default values for the |\version| flag
% (|final| for the main file and |draft| for childs):
%    \begin{macrocode}
\ifchilddoc
\providecommand{\version}{draft}
\else
\providecommand{\version}{final}
\fi
%    \end{macrocode}

% Load the standard document class:
%    \begin{macrocode}
\documentclass[12pt]{article}
%    \end{macrocode}

% Start the document body:
%    \begin{macrocode}
\begin{document}
%    \end{macrocode}

% Declare a title page.
% Print title, part of document being processed and version flag:
%    \begin{macrocode}
\addtocounter{page}{-1}
\begin{center}
{\LARGE\bfseries{}childdoc example\par}
\vspace{1cm}
\ifchilddoc
\ifchilddocmanual part\else chapter\fi:
`\childdocname' of `\childdocjob'\par
\else
main document: `\childdocjob'\par
\fi
version: \version\par
\end{center}
\newpage
%    \end{macrocode}

% Manually include selected file,
% otherwise process as usual:
%    \begin{macrocode}
\ifchilddocmanual
\section*{part `\childdocname'}
\input{\childdocname}
\else
%    \end{macrocode}

% Include the two chapters:
%    \begin{macrocode}
\include{cdocsch1}
\include{cdocsch2}
%    \end{macrocode}

% Include the two parts unless only chapters should be displayed:
%    \begin{macrocode}
\ifchilddoc\else
\section{part three}
\input{cdocspt3}
\section{part four}
\input{cdocspt4}
\fi
%    \end{macrocode}

% Process as usual until here:
%    \begin{macrocode}
\fi
%    \end{macrocode}

% End of document body:
%    \begin{macrocode}
\end{document}
%    \end{macrocode}
%\iffalse
%</samplemain>
%\fi
%
% %%%%%%%%%%%%%%%%%%%%%%%%%%%%%%%%%%%%%%
% \paragraph{Chapter Include Files.}
%
% The include files are called |cdocsch1.tex| and |cdocsch2.tex|.
%
%\iffalse
%<*samplechap1|samplechap2>
%\fi

% Optional override for |\version| flag:
%    \begin{macrocode}
%%\providecommand{\version}{final}
%    \end{macrocode}

% Include the main document:
%    \begin{macrocode}
\input{childdoc.def}
\childdocof{cdocsamp}
%    \end{macrocode}

%\iffalse
%</samplechap1|samplechap2>
%\fi
%
%\iffalse
%<*samplechap1>
%\fi
% Some text for chapter 1:
%    \begin{macrocode}
\section{one}
some text in chapter one
%    \end{macrocode}

%\iffalse
%</samplechap1>
%\fi
% Some text for chapter 2:
%\iffalse
%<*samplechap2>
%\fi
%    \begin{macrocode}
\section{two}
more text in chapter two
%    \end{macrocode}

%\iffalse
%</samplechap2>
%\fi
%
% %%%%%%%%%%%%%%%%%%%%%%%%%%%%%%%%%%%%%%
% \paragraph{Part Include Files.}
%
% The include files are called |cdocspt3.tex| and |cdocspt4.tex|.
%
%\iffalse
%<*samplepart3|samplepart4>
%\fi

% Optional override for |\version| flag:
%    \begin{macrocode}
%%\providecommand{\version}{final}
%    \end{macrocode}

% Include the main document:
%    \begin{macrocode}
\input{childdoc.def}
\childdocby{cdocsamp}
%    \end{macrocode}

%\iffalse
%</samplepart3|samplepart4>
%\fi
%
%\iffalse
%<*samplepart3>
%\fi
% Some text for part 3:
%    \begin{macrocode}
some text in part three
%    \end{macrocode}

%\iffalse
%</samplepart3>
%\fi
% Some text for part 4:
%\iffalse
%<*samplepart4>
%\fi
%    \begin{macrocode}
more text in part four
%    \end{macrocode}

%\iffalse
%</samplepart4>
%\fi
%
% %%%%%%%%%%%%%%%%%%%%%%%%%%%%%%%%%%%%%%
% \paragraph{Forwarding for a Complete Draft.}
%
% The following forwarding file |cdocsdrf.tex|
% compiles the main document in draft mode:
%\iffalse
%<*sampledraft>
%\fi
%    \begin{macrocode}
\def\version{draft}
\input{childdoc.def}
\childdocforward{cdocsamp}
%    \end{macrocode}

%\iffalse
%</sampledraft>
%\fi
%
% %%%%%%%%%%%%%%%%%%%%%%%%%%%%%%%%%%%%%%
% \paragraph{Forwarding for Final Version of the Chapters.}
%
% The following forwarding files |cdocsfn1.tex| and |cdocsfn2.tex|
% (with identical content)
% compile the final versions of the child documents
% |cdocsch1.tex| and |cdocsch2.tex|, respectively:
%\iffalse
%<*samplefinal>
%\fi
%    \begin{macrocode}
\def\version{final}
\input{childdoc.def}
\childdocforwardprefix[cdocsamp]{cdocsfn}{cdocsch}
%    \end{macrocode}

%\iffalse
%</samplefinal>
%\fi
%
% %%%%%%%%%%%%%%%%%%%%%%%%%%%%%%%%%%%%%%
% \paragraph{Command Line Processing.}
%
% The following three command lines generate the output files
% |cdocscld|, |cdocscl1| and |cdocscl2|
% which should be identical to
% |cdocsdrf|, |cdocsch1| and |cdocsfn2|, respectively:
% \begin{center}
% \begin{tabular}{l}
% |latex -jobname cdocscld \|\\
% |  "\def\version{draft}\input{childdoc.def}\childdocforward{cdocsamp}"|\\
% |latex -jobname cdocscl1 \|\\
% |  "\input{childdoc.def}\childdocforward[cdocsamp]{cdocsch1}"|\\
% |latex -jobname cdocscl2 \|\\
% |  "\def\version{final}\input{childdoc.def}\childdocforward{cdocsch2}"|
% \end{tabular}
% \end{center}
% Note that the trailing backslash on each first line
% merely continues the input to the second line
% (for convenient cut ant paste).
% Furthermore, the command |latex| can be replaced by any
% of its alternative versions such as |pdflatex|.
%
% %%%%%%%%%%%%%%%%%%%%%%%%%%%%%%%%%%%%%%%%%%%%%%%%%%%%%%%%%%%%%%%%%%%%%%%%%%%%%%
% %%%%%%%%%%%%%%%%%%%%%%%%%%%%%%%%%%%%%%%%%%%%%%%%%%%%%%%%%%%%%%%%%%%%%%%%%%%%%%
% \section{Implementation}
%\iffalse
%<*package>
%\fi
%
% This section describes the definitions file |childdoc.def|.

% The definitions cannot be loaded using |\usepackage| or |\RequirePackage|
% which has a mechanism to prevent loading a style file more than once.
% When loading the definitions by means of |\input|
% multiple instances have to be prevented manually:
%\iffalse
%This code needs to be before the `\ProvidesFile' directive
%which is defined at the beginning of this file.
%Therefore it is also placed there and commented out here.
%</package>
%<*discard>
%\fi
%    \begin{macrocode}
\ifdefined\childdocmain\endinput\fi
%    \end{macrocode}
%\iffalse
%</discard>
%<*package>
%\fi
%
% \macro{\ifchilddoc}
% \macro{\ifchilddocmanual}
% The conditional |\ifchilddoc| tells whether a
% child (true) or main (false) document is being compiled.
% The conditional |\ifchilddocmanual| tells whether
% the |\includeonly| mechanism is used (false) or
% the selection of child files must be performed manually (true).
% The definitions initialise to false:
%    \begin{macrocode}
\newif\ifchilddoc
\newif\ifchilddocmanual
%    \end{macrocode}

% \macro{\childdocname}
% \macro{\childdocjob}
% The macro |\childdocname| stores the name of the main document
% to be compiled. The macro |\childdocjob| stores the name of
% the document on which the \LaTeX{} compiler was originally invoked.
% The content of |\jobname| cannot be compared
% to filenames specified in the source due to different catcodes.
% The following code rescans |\jobname|, stores the result
% in |\childdocname| and saves a copy in |\childdocjob|:
%    \begin{macrocode}
\edef\childdocname{\scantokens\expandafter{\jobname\noexpand}}
\let\childdocjob\childdocname
%    \end{macrocode}

% \macro{\childdocdisable}
% The macro |\childdocdisable| prevents the main file
% from being processed more than once.
% At this stage, the main document command |\childdocmain|
% is assumed to be called once again where it should do nothing.
% Any subsequent call to it should prevent
% a secondary processing of the main document
% It overwrites the forwarding commands
% |\childdocof| and |\childdocforward|
% with empty macros to prevent further inclusions of the main document:
%    \begin{macrocode}
\newcommand{\childdocdisable}
{
  \renewcommand{\childdocmain}[1]{\renewcommand{\childdocmain}[1]{\endinput}}
  \renewcommand{\childdocof}[1]{}
  \renewcommand{\childdocby}[2][]{}
  \renewcommand{\childdocforward}[2][]{}
  \renewcommand{\childdocdisable}{}
}
%    \end{macrocode}

% \macro{\childdocmain}
% The macro |\childdocmain| is to be called at the top of the main file
% with nothing or the main filename (without extension) as argument.
% First, it breaks loops.
% If the argument is not empty and does not match |\childdocname|
% (which is set by the first inclusion of |childdoc.def|),
% |\ifchilddoc| is set to true, |\includeonly| is applied to the child file
% and |\jobname| is set to the main file
% (for proper handling of |.aux| files):
%    \begin{macrocode}
\newcommand{\childdocmain}[1]
{
  \childdocdisable\childdocmain{}
  \if?#1?\else
    \begingroup
      \def\childdoctmp{#1}
      \ifx\childdoctmp\childdocname
        \def\childdoctmp{}
      \else
        \def\childdoctmp
        {
          \childdoctrue
          \includeonly{\childdocname}
          \def\childdocjob{#1}
          \def\jobname{#1}
        }
      \fi
      \expandafter
    \endgroup
    \childdoctmp
  \fi
}
%    \end{macrocode}

% \macro{\childdocof}
% The command |\childdocof| redirects
% compilation to the main file |#1|.
%    \begin{macrocode}
\newcommand{\childdocof}[1]
{
  \childdocdisable
  \childdoctrue
  \includeonly{\childdocname}
  \def\jobname{#1}
  \def\childdocjob{#1}
  \input{#1}
}
%    \end{macrocode}

% \macro{\childdocby}
% The command |\childdocby| ....
%    \begin{macrocode}
\newcommand{\childdocby}[2][]
{
  \childdocdisable
  \childdoctrue
  \childdocmanualtrue
  \if?#1?\else
    \def\jobname{#2}
  \fi
  \def\childdocjob{#2}
  \input{#2}
  \endinput
}
%    \end{macrocode}

% \macro{\childdocforward}
% The command |\childdocforward| redirects
% compilation to the main file or
% (if the optional argument is given) a child file.
% Parameters are set as if the main file
% or a child file starting with |\childdocof| was compiled.
% Then compilation is handed over to the main file:
%    \begin{macrocode}
\newcommand{\childdocforward}[2][]
{
  \begingroup
    \if?#1?
      \def\childdoctmp
      {
        \def\childdocname{#2}
        \def\childdocjob{#2}
        \def\jobname{#2}
        \input{#2}
        \endinput
      }
    \else
      \def\childdoctmp
      {
        \childdocdisable
        \def\childdocname{#2}
        \childdoctrue
        \includeonly{#2}
        \def\childdocjob{#1}
        \def\jobname{#1}
        \input{#1}
        \endinput
      }
    \fi
    \expandafter
  \endgroup
  \childdoctmp
}
%    \end{macrocode}

% \macro{\childdocforwardprefix}
% The command |\childdocforwardprefix| redirects
% compilation to the main or a child file by means of a pattern.
% The prefix |#1| in the current filename is replaced by |#2|
% and the suffix of the current filename is kept
% (it is assumed that the filename does not contain the substring `|~~~|'
% which is used as a delimiter).
% Compilation is handed over to the new file by |\childdocforward|:
%    \begin{macrocode}
\newcommand{\childdocforwardprefix}[3][]
{
  \begingroup
    \def\childdocextract #2##1~~~{\def\childdoctmp{\childdocforward[#1]{#3##1}}}
    \expandafter\childdocextract\childdocname~~~
    \expandafter
  \endgroup
  \childdoctmp
}
%    \end{macrocode}

% \macro{\childdoc}
% The deprecated macro |\childdoc| is a legacy version of |\childdocmain|:
%    \begin{macrocode}
\newcommand{\childdoc}{\childdocmain}
%    \end{macrocode}

% \macro{\childdocredirect}
% The deprecated macro |\childdocredirect| is a legacy version
% of |\childdocforward| and |\childdocforwardprefix|:
%    \begin{macrocode}
\newcommand{\childdocredirect}[2][]
{
  \begingroup
    \if?#1?
      \def\childdoctmp{\childdocforward{#2}}
    \else
      \def\childdoctmp{\childdocforwardprefix{#1}{#2}}
    \fi
    \expandafter
  \endgroup
  \childdoctmp
}
%    \end{macrocode}

%\iffalse
%</package>
%\fi
%
\endinput

\childdocof{cdocsamp}
%    \end{macrocode}

%\iffalse
%</samplechap1|samplechap2>
%\fi
%
%\iffalse
%<*samplechap1>
%\fi
% Some text for chapter 1:
%    \begin{macrocode}
\section{one}
some text in chapter one
%    \end{macrocode}

%\iffalse
%</samplechap1>
%\fi
% Some text for chapter 2:
%\iffalse
%<*samplechap2>
%\fi
%    \begin{macrocode}
\section{two}
more text in chapter two
%    \end{macrocode}

%\iffalse
%</samplechap2>
%\fi
%
% %%%%%%%%%%%%%%%%%%%%%%%%%%%%%%%%%%%%%%
% \paragraph{Part Include Files.}
%
% The include files are called |cdocspt3.tex| and |cdocspt4.tex|.
%
%\iffalse
%<*samplepart3|samplepart4>
%\fi

% Optional override for |\version| flag:
%    \begin{macrocode}
%%\providecommand{\version}{final}
%    \end{macrocode}

% Include the main document:
%    \begin{macrocode}
% \iffalse
%
% childdoc.dtx Copyright (C) 2017-2018 Niklas Beisert
%
% This work may be distributed and/or modified under the
% conditions of the LaTeX Project Public License, either version 1.3
% of this license or (at your option) any later version.
% The latest version of this license is in
%   http://www.latex-project.org/lppl.txt
% and version 1.3 or later is part of all distributions of LaTeX
% version 2005/12/01 or later.
%
% This work has the LPPL maintenance status `maintained'.
%
% The Current Maintainer of this work is Niklas Beisert.
%
% This work consists of the files childdoc.dtx and childdoc.ins
% and the derived files childdoc.def and cdocsamp.tex with
% cdocsch1.tex, cdocsch2.tex, cdocsdrf.tex, cdocsfn1.tex, cdocsfn2.tex.
%
%<package>\ifdefined\childdocmain\endinput\fi
%<package>\ProvidesFile{childdoc.def}[2018/12/30 v2.0 child document driver]
%<samplemain>\ProvidesFile{cdocsamp.tex}[2018/12/30 v2.0 sample for childdoc]
%<*driver>
%\ProvidesFile{childdoc.drv}[2018/12/30 v2.0 childdoc reference manual file]
\PassOptionsToClass{10pt,a4paper}{article}
\documentclass{ltxdoc}

\usepackage[margin=35mm]{geometry}
\usepackage{hyperref}
\usepackage{hyperxmp}
\usepackage[usenames]{color}

\hypersetup{colorlinks=true}
\hypersetup{pdfstartview=FitH}
\hypersetup{pdfpagemode=UseNone}
\hypersetup{pdfsource={}}
\hypersetup{pdflang={en-UK}}
\hypersetup{pdfcopyright={Copyright 2017-2018 Niklas Beisert.
  This work may be distributed and/or modified under the
  conditions of the LaTeX Project Public License, either version 1.3
  of this license or (at your option) any later version.}}
\hypersetup{pdflicenseurl={http://www.latex-project.org/lppl.txt}}
\hypersetup{pdfcontactaddress={ETH Zurich, ITP, HIT K,
  Wolfgang-Pauli-Strasse 27}}
\hypersetup{pdfcontactpostcode={8093}}
\hypersetup{pdfcontactcity={Zurich}}
\hypersetup{pdfcontactcountry={Switzerland}}
\hypersetup{pdfcontactemail={nbeisert@itp.phys.ethz.ch}}
\hypersetup{pdfcontacturl={http://people.phys.ethz.ch/\xmptilde nbeisert/}}

\newcommand{\secref}[1]{\hyperref[#1]{section \ref*{#1}}}

\parskip1ex
\parindent0pt
\let\olditemize\itemize
\def\itemize{\olditemize\parskip0pt}

\begin{document}

\title{The \textsf{childdoc} Package}
\hypersetup{pdftitle={The childdoc Package}}
\author{Niklas Beisert\\[2ex]
  Institut f\"ur Theoretische Physik\\
  Eidgen\"ossische Technische Hochschule Z\"urich\\
  Wolfgang-Pauli-Strasse 27, 8093 Z\"urich, Switzerland\\[1ex]
  \href{mailto:nbeisert@itp.phys.ethz.ch}
  {\texttt{nbeisert@itp.phys.ethz.ch}}}
\hypersetup{pdfauthor={Niklas Beisert}}
\hypersetup{pdfsubject={Manual for the LaTeX2e Package childdoc}}
\date{30 December 2018, \textsf{v2.0}}
\maketitle

\begin{abstract}\noindent
\textsf{childdoc} is a \LaTeXe{} package
that enables the direct compilation
of document sections included by |\include|
to individual files.
\end{abstract}

\begingroup
\parskip0ex
\tableofcontents
\endgroup

%%%%%%%%%%%%%%%%%%%%%%%%%%%%%%%%%%%%%%%%%%%%%%%%%%%%%%%%%%%%%%%%%%%%%%%%%%%%%%%%
%%%%%%%%%%%%%%%%%%%%%%%%%%%%%%%%%%%%%%%%%%%%%%%%%%%%%%%%%%%%%%%%%%%%%%%%%%%%%%%%
\section{Introduction}

\LaTeX{} provides a mechanism to structure a large document (such as a book)
into a main file and several child files (containing the chapters)
using the |\include| command.
This mechanism is beneficial for documents
which span hundreds of pages in order to
make the source file(s) more manageable.
Moreover, compilation can be restricted to
selected child files by means of the |\includeonly| command.
The latter feature can be used to reduce the compilation time while editing
(this was significantly more useful in the earlier days of \LaTeX{})
or to generate a smaller document which is easier to navigate.
Another application of |\includeonly| is to generate
documents consisting of selected parts of the complete document.

However, there are a few drawbacks of the plain |\include| mechanism:
\begin{itemize}
\item
The child files cannot be compiled on their own,
they can only be compiled via the main file.
A naive editing environment
(such as a text editor with an option
to have the current file processed by \LaTeX)
may require one to switch to the main file before compiling;
attempting to compile the child file produces errors.
\item
The main file must be modified (each time)
to adjust the |\includeonly| command
to the present needs. This easily leaves the main file in a messy state.
\item
The generated document will always carry the filename
of the main document. This is inconvenient if
several child files are to be compiled and
to be kept for distribution.
\end{itemize}

The present package provides a simple interface
to make child files individually compilable by \LaTeX{}.
Compiling a child file then has the same effect as compiling
the main file with an |\includeonly| command
to select the appropriate child.
Moreover the generated document will carry the name of the child
rather than the main file.
This resolves all three above issues.

This feature is meant to make the editing of books,
thesis documents and lecture notes somewhat more convenient.
However, the package can also be used efficiently for
composing a series of documents (such as exercise sheets)
which are typically distributed individually.
It then assists the author in generating the individual documents
(potentially in different versions)
as well as a document containing the collected series.
Another application is in developing style files
or other kinds of included material
where compilation of the style file could redirect
to a sample or test file.

%%%%%%%%%%%%%%%%%%%%%%%%%%%%%%%%%%%%%%%%%%%%%%%%%%%%%%%%%%%%%%%%%%%%%%%%%%%%%%%%
%%%%%%%%%%%%%%%%%%%%%%%%%%%%%%%%%%%%%%%%%%%%%%%%%%%%%%%%%%%%%%%%%%%%%%%%%%%%%%%%
\section{Usage}

First of all, the package \textsf{childdoc} is \emph{not} a standard
\LaTeXe{} |.sty| style file! Therefore it needs to be invoked in
a non-standard way.

%%%%%%%%%%%%%%%%%%%%%%%%%%%%%%%%%%%%%%%%%%%%%%%%%%%%%%%%%%%%%%%%%%%%%%%%%%%%%%%%
\subsection{Included Files}
\label{sec:include}

%%%%%%%%%%%%%%%%%%%%%%%%%%%%%%%%%%%%%%%%
\DescribeMacro{\childdocmain}
To use the package, add the commands
\begin{center}
\begin{tabular}{l}
|\input{childdoc.def}|\\
|\childdocmain{}|\\
\end{tabular}
\end{center}
at the very top of the main \LaTeX{} file,
in particular \emph{before} the |\documentclass| statement!
The argument of |\childdocmain| should be left empty
(but it must be present).

%%%%%%%%%%%%%%%%%%%%%%%%%%%%%%%%%%%%%%%%
\DescribeMacro{\childdocof}
Furthermore, add the commands
\begin{center}
\begin{tabular}{l}
|\input{childdoc.def}|\\
|\childdocof{|\textit{main}|}|\\
\end{tabular}
\end{center}
at the top of every child file \textit{child}
which is included by |\include{|\textit{child}|}|
from within the main file
(or at least for those files to be compiled individually).
The argument \textit{main} must be the filename of the main file.

There are a couple of
considerations in setting up the main and child documents:

%%%%%%%%%%%%%%%%%%%%%%%%%%%%%%%%%%%%%%%%
\paragraph{Restrictions.}

Please note the following restrictions:
\begin{itemize}
\item
|\childdocmain| must be called with one argument \textit{main}
to ensure compatibility with earlier version of the package.
It must either be empty (|\childdocmain{}|)
or precisely match the filename of the main file in which it is specified.
See \secref{sec:detection} for further information.
\item
The filename \textit{main} must be specified without the |.tex| extension.
\item
The filename \textit{main} is case sensitive
(even in case-insensitive file systems)
due to internal string comparison.
\item
The argument \textit{main} should be fully expanded, it cannot be a macro.
\item
Subdirectories and special characters should be avoided in filenames.
\item
The command |\childdocmain{|\textit{main}|}| must be followed by a whitespace.
It should not be followed immediately by another command
or by a comment mark `|%|'.
This is because the \TeX{} parser reads the token immediately following
the argument of |\childdocmain| and puts it
at the beginning of every child section;
however, a white\-space is ignored.
\end{itemize}

%%%%%%%%%%%%%%%%%%%%%%%%%%%%%%%%%%%%%%%%
\paragraph{Content of Main File.}

It is advisable to place all content in the child files included by |\include|.
Any output contained in the main file will appear in all child documents
unless suppressed manually;
it cannot be suppressed automatically by the |\includeonly| directive
and thus should normally be avoided.
A method to include some content in the main file
by means of conditional processing is described in \secref{sec:conditional}.

%%%%%%%%%%%%%%%%%%%%%%%%%%%%%%%%%%%%%%%%
\paragraph{Page Numbering.}

When only a part of the document is compiled,
the appropriate numbering of pages
(as well as other status parameters)
is determined from the |.aux| files.
The latter contain information from previous passes.
However this information needs to propagate through
all intermediate child documents.
Therefore the page numbering in child documents may well
be inconsistent until the complete document is compiled at least once.

A useful (if unconventional) way to always ensure a consistent
page numbering is to restart the numbering in each child document
and denote the pages by `\textit{child}|.|\textit{page}'
where \textit{child} represents the chapter/section number of the child file.
This can be achieved by the command
|\numberwithin{page}{|\textit{child}|}|
of the \textsf{amsmath} package
where \textit{child} can be |chapter| or |section|
depending on the chosen structuring.
Alternatively, one can modify the macro |\thepage| appropriately
and reset the counter |page| at the start of each child file.

%%%%%%%%%%%%%%%%%%%%%%%%%%%%%%%%%%%%%%%%%%%%%%%%%%%%%%%%%%%%%%%%%%%%%%%%%%%%%%%%
\subsection{Conditional Processing}
\label{sec:conditional}

The package provides a mechanism to compile different versions
of a document. To customise the versions further some conditional processing
can come in handy to distinguish which version is being compiled.
The package provides two macros to describe the compilation context:

%%%%%%%%%%%%%%%%%%%%%%%%%%%%%%%%%%%%%%%%
\DescribeMacro{\ifchilddoc}
The conditional |\ifchilddoc| distinguishes between the compilation of
child documents and the main document:
%
\begin{center}
|\ifchilddoc |\textit{child-code}| |[|\||else |\textit{main-code}]| \||fi|
\end{center}

%%%%%%%%%%%%%%%%%%%%%%%%%%%%%%%%%%%%%%%%
\DescribeMacro{\childdocname}
\DescribeMacro{\childdocjob}
The macro |\childdocname| contains the filename (without extension)
of the main or child file being processed.
Note that |\childdocjob| will always contain the name of the main file.

%%%%%%%%%%%%%%%%%%%%%%%%%%%%%%%%%%%%%%%%
\paragraph{Title Page.}

Conditional processing can be used to include a title or banner page
in the main document when proper precautions are taken.
Importantly, the code in the main file should ensure that the page counter
(as well as other status parameters which are stored in the |.aux| files)
takes the same value after the conditional processing.
Otherwise the page numbers may take divergent values
depending on which part is compiled.

For example, a title page could be declared by:
%
\begin{center}
\begin{tabular}{l}
|\ifchilddoc\||else|\\
|\addtocounter{page}{-1}|\\
\textit{code for title page}\\
|\newpage|\\
|\||fi|
\end{tabular}
\end{center}
%
A banner page for the child documents can be generated by:
%
\begin{center}
\begin{tabular}{l}
|\ifchilddoc|\\
|\addtocounter{page}{-1}|\\
\textit{code for banner page}\\
|\newpage|\\
|\||fi|
\end{tabular}
\end{center}
%
Here one could write a message such as:
\begin{center}
|This is the part \childdocname{} of \childdocjob{}.|
\end{center}

%%%%%%%%%%%%%%%%%%%%%%%%%%%%%%%%%%%%%%%%%%%%%%%%%%%%%%%%%%%%%%%%%%%%%%%%%%%%%%%%
\subsection{Flags}
\label{sec:flags}

The package makes it easy to generate different versions
of the main or child documents.
To this end compilation flags can be defined
and assigned different default values.
They will be particularly useful in conjunction
with the forwarding mechanism described in \secref{sec:forward}.

For example, it may be useful to have a flag |\version|
which can be set to |draft| or |final|.
The document source will contain some conditional code
depending on the value of |\version|.
Suppose further, the flag should default to |final| for the main file
and to |draft| for child files
which is a natural assignment for editing the document.
This is achieved by placing the following code
in the preamble of the main document
(below the |\childdocmain| directive):
%
\begin{center}
\begin{tabular}{l}
|\ifchilddoc|\\
|\providecommand{\version}{draft}|\\
|\||else|\\
|\providecommand{\version}{final}|\\
|\||fi|
\end{tabular}
\end{center}
%
The definition by |\providecommand| makes sure
that previous definitions are not overwritten.
Further statements |\providecommand{\version}{...}|
can thus be added before the above code to override it.

For the main file, one might add a line
(between |\childdocmain| and the above block)
%
\begin{center}
|%\ifchilddoc\||else\providecommand{\version}{draft}\||fi|
\end{center}
%
which can be uncommented to produce a draft version.
Likewise one can add a line to the very top of a child file
(above the |\childdocof{|\textit{main}|}| directive)
%
\begin{center}
|%\providecommand{\version}{final}|
\end{center}
%
which can be uncommented to produce the final version of this child document.

%%%%%%%%%%%%%%%%%%%%%%%%%%%%%%%%%%%%%%%%%%%%%%%%%%%%%%%%%%%%%%%%%%%%%%%%%%%%%%%%
\subsection{Forwarding}
\label{sec:forward}

Different versions of the main or child documents
using compilation flags as described in \secref{sec:flags}
can be (permanently) stored in different files
for convenient compilation, viewing and distribution.
To this end, the package defines a command
to pass on compilation to a different file:

%%%%%%%%%%%%%%%%%%%%%%%%%%%%%%%%%%%%%%%%
\DescribeMacro{\childdocforward}
The command |\childdocforward| redirects processing to
another source file:
%
\begin{center}
\begin{tabular}{l}
|\input{childdoc.def}|\\
|\childdocforward[|\textit{main}|]{|\textit{dest}|}|\\
\end{tabular}
\end{center}
%
The argument \textit{dest} is the destination file
(without extension).
It should be the main file or one of the child files.
Note that further \textsf{childdoc} directives
such as |\childdocof| and |\childdocforward|
in the indicated file will be processed in this form.
The optional argument \textit{main}
passes on directly to the main file \textit{main}
while pretending to compile the child \textit{dest}.
This form behaves as if \textit{dest}
issues |\childdocof{|\textit{main}|}| right away,
and no further \textsf{childdoc} directives will be processed.

%%%%%%%%%%%%%%%%%%%%%%%%%%%%%%%%%%%%%%%%
\DescribeMacro{\...prefix}
In the alternative form |\childdocforwardprefix|,
%
\begin{center}
\begin{tabular}{l}
|\input{childdoc.def}|\\
|\childdocforwardprefix[|\textit{main}|]{|\textit{prefix}|}{|\textit{dest}|}|
\end{tabular}
\end{center}
%
the destination file is determined by a pattern
depending on the current file:
To make this work, the current file must be called
`{\textit{prefix}\hspace{0.2em}\textit{suffix}}'
with \textit{prefix} matching precisely the argument.
Processing is then passed on to the file
`{\textit{dest}\hspace{0.2em}\textit{suffix}}'.
Surely, the same effect is achieved by
directly specifying the
argument `{\textit{dest}\hspace{0.2em}\textit{suffix}}'
in the first form.
However, that requires to set up a different file
for each child. With the alternative form of the command
all these files can have exactly the same content
which simplifies setting them up and maintaining them.

For example, the following file |draft.tex|
with a compilation flag |\version| as described in \secref{sec:flags}
compiles the main document as a draft:
%
\begin{center}
\begin{tabular}{l}
|\def\version{draft}|\\
|\input{childdoc.def}|\\
|\childdocforward{|\textit{main}|}|
\end{tabular}
\end{center}
%
Likewise, the following files |final|\textit{nn}|.tex|
compile the final version of the child document
|child|\textit{nn}|.tex|:
%
\begin{center}
\begin{tabular}{l}
|\def\version{final}|\\
|\input{childdoc.def}|\\
|\childdocforwardprefix{final}{child}|
\end{tabular}
\end{center}
%

Note that when several versions of a main file and/or of each child file
are to be generated, it may be convenient to set up a |Makefile| or
shell script to automatise the process.

%%%%%%%%%%%%%%%%%%%%%%%%%%%%%%%%%%%%%%%%%%%%%%%%%%%%%%%%%%%%%%%%%%%%%%%%%%%%%%%%
\subsection{Command Line Processing}
\label{sec:commandline}

The effect of redirection files can also be achieved by invoking
the \LaTeX{} compiler with a more elaborate command line.
Most conveniently this should be done as part
of a shell script or a |Makefile|.

When using \textsf{childdoc} in the main file, the following
command lines effectively perform a redirection
(note that depending on the shell being used,
backslashes may have to be doubled: `|\|' $\to$ `|\\|'):
%
\begin{center}
|... -jobname "|\textit{target}|" |\\|"|[\textit{flags}]%
|\input{childdoc.def}\childdocforward[|\textit{main}|]{|\textit{dest}|}"|
\end{center}
%
Here \textit{target} is the name of the output file,
\textit{main} is the name of the main file
and \textit{dest} is the name of the main or child file to be processed
(all filenames without extensions).
The optional argument \textit{main} can be omitted
if \textit{main} matches \textit{dest}.
Optionally, compilation \textit{flags} can be defined via |\def| commands.
This command line makes the \TeX{} engine believe
it is compiling the file \textit{target}
whose content is specified as the latter parameter.
The provided code then forwards the processing to
\textit{main} or \textit{dest} as described in \secref{sec:forward}.

%%%%%%%%%%%%%%%%%%%%%%%%%%%%%%%%%%%%%%%%%%%%%%%%%%%%%%%%%%%%%%%%%%%%%%%%%%%%%%%%
\subsection{Include by Input}
\label{sec:input}

Including child documents by |\include| has some restrictions by design.
Most notably, the content of a child document always occupies
its own set of pages; pages cannot be shared between child documents.
Usually, this behaviour makes perfect sense
because each child document contain an essential part of the document.
However, in some situations it may be desirable to compose
a document from a collection of parts
without having mandatory page breaks between then.
For this case, the package
provides a mechanism to include parts
by |\input| which can also be processed individually.
However, by construction this mechanism
requires manual handling of the content to be output.

%%%%%%%%%%%%%%%%%%%%%%%%%%%%%%%%%%%%%%%%
\DescribeMacro{\ifchilddocmanual}
The main file should be prepared as usual, see \secref{sec:include}.
However, the document body must make a distinction
between processing of an individual part and of the main document, e.g.:
%
\begin{center}
\begin{tabular}{l}
|\ifchilddocmanual|\\
|\input{\childdocname}|\\
|\||else|\\
\textit{document body with }|\input{|\textit{part}|}|\\
|\||fi|
\end{tabular}
\end{center}
%
The conditional |\ifchilddocmanual| is true whenever
a part to be included by |\input| is being compiled,
and the name of the part is stored in |\childdocname|.

%%%%%%%%%%%%%%%%%%%%%%%%%%%%%%%%%%%%%%%%
\DescribeMacro{\childdocby}
Each part to be included by |\input| should start with:
%
\begin{center}
\begin{tabular}{l}
|\input{childdoc.def}|\\
|\childdocby{|\textit{main}|}|\\
\end{tabular}
\end{center}
%
The directive |\childdocby| is similar to |\childdocof|
described in \secref{sec:include},
but the subsequent selection of content must be done manually.
To that end, both |\ifchilddoc| and |\ifchilddocmanual|
will be true upon processing of a part,
and the name of the part is stored in |\childdocname|.
Note that |\jobname| will be set to the filename of the current part
so that each part receives an individual |.aux| file
that does not interfere with the |.aux| file(s) of the main document.
This behaviour can be altered by the alternative form
|\childdocby[*]{|\textit{main}|}| (with a non-empty optional argument)
which uses the |.aux| file of the main document
by setting |\jobname| to \textit{main}.

%%%%%%%%%%%%%%%%%%%%%%%%%%%%%%%%%%%%%%%%%%%%%%%%%%%%%%%%%%%%%%%%%%%%%%%%%%%%%%%%
\subsection{Driver Development}
\label{sec:driver}

The \textsf{childdoc} mechanism can also be use for the development
of definition files such as \LaTeX{} styles or classes.
This case differs from the above setup with multiple parts
included by |\include| in that no |\includeonly| should be invoked.
This can be achieved by starting the include file
(before |\ProvidesPackage|) with:
%
\begin{center}
\begin{tabular}{l}
|\input{childdoc.def}|\\
|\childdocforward{|\textit{main}|}|\\
\end{tabular}
\end{center}
%
or alternatively with:
%
\begin{center}
\begin{tabular}{l}
|\input{childdoc.def}|\\
|\childdocby{|\textit{main}|}|\\
\end{tabular}
\end{center}
%
Both forms have slightly different effects as described above.
The main file is prepared as usual, see \secref{sec:include}.

%%%%%%%%%%%%%%%%%%%%%%%%%%%%%%%%%%%%%%%%%%%%%%%%%%%%%%%%%%%%%%%%%%%%%%%%%%%%%%%%
\subsection{Legacy Detection}
\label{sec:detection}

The directive |\childdocmain| in the main file can detect
whether the complete document or merely a child is to be compiled
even without using the directive |\childdocof|.
This method is deprecated because it is less robust
and there is no compelling reason to use it;
it is merely provided for backward compatibility
and it may be removed in future versions.

If the detection mechanism is to be used,
it is mandatory to correctly specify
the filename of the main file as the argument of |\childdocmain|:
%
\begin{center}
\begin{tabular}{l}
|\input{childdoc.def}|\\
|\childdocmain{|\textit{main}|}|\\
\end{tabular}
\end{center}
%
If |\jobname| does not match the argument \textit{main} of |\childdocmain|,
it is assumed that |\jobname| points to the child file to be compiled.
When using |\childdocmain| with the main file specified as argument,
it suffices to start a child file
with just |\input{|\textit{main}|}|
without loading of the package and using |\childdocof|.
If instead all processing is done
with the appropriate \textsf{childdoc} directives,
the argument of \textit{main} of |\childdocmain| can be empty.

An alternative version of the command line processing described
in \secref{sec:commandline} using the detection mechanism reads:
%
\begin{center}
|... -jobname "|\textit{target}|" "|[\textit{flags}]%
[|\def\jobname{|\textit{dest}|}|]|\input{|\textit{main}|}"|
\end{center}

%%%%%%%%%%%%%%%%%%%%%%%%%%%%%%%%%%%%%%%%%%%%%%%%%%%%%%%%%%%%%%%%%%%%%%%%%%%%%%%%
\subsection{Manual Code}
\label{sec:manual}

In case one cannot be certain whether the definitions file |childdoc.def|
is installed on the target \TeX{} distribution
and one prefers not to ship it,
it is conceivable to paste a few relevant commands into the sources.

To that end, drop all statements |\input{childdoc.def}|
and perform the replacements as outlined below.
Instead of |\childdocmain{|\textit{main}|}| add the following code
to the top of the main file:
%
\begin{center}
\begin{tabular}{l}
|\||ifdefined\childdocname\endinput\||fi\newif\ifchilddoc|\\
|\edef\childdocname{\scantokens\expandafter{\jobname\noexpand}}|\\
|\def\childdocmain{|\textit{main}|}\||ifx\childdocmain\childdocname\||else|\\
|\childdoctrue\includeonly{\childdocname}\let\jobname\childdocmain\||fi|\\
\end{tabular}
\end{center}
%
Instead of |\childdocof{|\textit{main}|}| just include the main file
at the top of each child file:
%
\begin{center}
|\input{|\textit{main}|}|
\end{center}
%
A simple redirection |\childdocforward{|\textit{dest}|}| is achieved by:
%
\begin{center}
|\def\jobname{|\textit{dest}|}\input{\jobname}|
\end{center}
%
The redirection with prefix
|\childdocforwardprefix[|\textit{prefix}|]{|\textit{dest}|}|
is accomplished by:
%
\begin{center}
\begin{tabular}{l}
|{\edef\jobname{\scantokens\expandafter{\jobname\noexpand}}|\\
|\def\redirectjob |\textit{prefix}|#1~~~{\gdef\jobname{|\textit{dest}|#1}}|\\
|\expandafter\redirectjob\jobname~~~}\input{\jobname}|
\end{tabular}
\end{center}

In an alternative approach,
child documents can be compiled by a specific command line
without additional code or specific definitions:
%
\begin{center}
|... -jobname "|\textit{target}|" "|[\textit{flags}]%
|\includeonly{|\textit{dest}|}\input{|\textit{main}|}"|
\end{center}
%

%%%%%%%%%%%%%%%%%%%%%%%%%%%%%%%%%%%%%%%%%%%%%%%%%%%%%%%%%%%%%%%%%%%%%%%%%%%%%%%%
%%%%%%%%%%%%%%%%%%%%%%%%%%%%%%%%%%%%%%%%%%%%%%%%%%%%%%%%%%%%%%%%%%%%%%%%%%%%%%%%
\section{Information}

%%%%%%%%%%%%%%%%%%%%%%%%%%%%%%%%%%%%%%%%%%%%%%%%%%%%%%%%%%%%%%%%%%%%%%%%%%%%%%%%
\subsection{Copyright}

Copyright \copyright{} 2017--2018 Niklas Beisert

This work may be distributed and/or modified under the
conditions of the \LaTeX{} Project Public License, either version 1.3
of this license or (at your option) any later version.
The latest version of this license is in
  \url{http://www.latex-project.org/lppl.txt}
and version 1.3 or later is part of all distributions of \LaTeX{}
version 2005/12/01 or later.

This work has the LPPL maintenance status `maintained'.

The Current Maintainer of this work is Niklas Beisert.

This work consists of the files |README.txt|, |childdoc.ins| and |childdoc.dtx|
as well as the derived files |childdoc.def|, |cdocsamp.tex|
with |cdocsch1.tex|, |cdocsch2.tex|, |cdocspt3.tex|, |cdocspt4.tex|,
|cdocsdrf.tex|, |cdocsfn1.tex|, |cdocsfn2.tex|
as well as |childdoc.pdf|.

%%%%%%%%%%%%%%%%%%%%%%%%%%%%%%%%%%%%%%%%%%%%%%%%%%%%%%%%%%%%%%%%%%%%%%%%%%%%%%%%
\subsection{Files and Installation}

The package consists of the files:
%
\begin{center}
\begin{tabular}{ll}
    |README.txt|   & readme file \\
    |childdoc.ins| & installation file \\
    |childdoc.dtx| & source file \\
    |childdoc.def| & definition file \\
    |cdocsamp.tex| & sample main file \\
    |cdocsch1.tex| & sample include file \\
    |cdocsch2.tex| & sample include file \\
    |cdocspt3.tex| & sample part file \\
    |cdocspt4.tex| & sample part file \\
    |cdocsdrf.tex| & sample redirection file \\
    |cdocsfn1.tex| & sample redirection file \\
    |cdocsfn2.tex| & sample redirection file \\
    |childdoc.pdf| & manual
\end{tabular}
\end{center}
%
The distribution consists of the files
|README.txt|, |childdoc.ins| and |childdoc.dtx|.
%
\begin{itemize}
\item
Run (pdf)\LaTeX{} on |childdoc.dtx|
to compile the manual |childdoc.pdf| (this file).
\item
Run \LaTeX{} on |childdoc.ins| to create the definitions file |childdoc.def|
and the sample |cdocsamp.tex| with include files
|cdocsch1.tex|, |cdocsch2.tex|, |cdocspt3.tex|, |cdocspt4.tex|,
|cdocsdrf.tex|, |cdocsfn1.tex|, |cdocsfn2.tex|.
Then copy the file |childdoc.def| to an appropriate directory of your \LaTeX{}
distribution, e.g.\ \textit{texmf-root}|/tex/latex/childdoc|.
\end{itemize}

%%%%%%%%%%%%%%%%%%%%%%%%%%%%%%%%%%%%%%%%%%%%%%%%%%%%%%%%%%%%%%%%%%%%%%%%%%%%%%%%
\subsection{Related CTAN Packages}

There are several other packages which offer a similar functionality:
%
\begin{itemize}
\item
The packages
\href{http://ctan.org/pkg/docmute}{\textsf{docmute}},
\href{http://ctan.org/pkg/includex}{\textsf{includex}} and
\href{http://ctan.org/pkg/standalone}{\textsf{standalone}}
provide commands to include only the document body of
a child file thus allowing both files to be compiled individually.
\item
The packages \href{http://ctan.org/pkg/subdocs}{\textsf{subdocs}}
and \href{http://ctan.org/pkg/subfiles}{\textsf{subfiles}}
provide structures in which the main and child documents can be
encapsulated and allowing them to be compiled individually.
The inclusion mechanism is different from the conventional |\include|.
\item
The package \href{http://ctan.org/pkg/combine}{\textsf{combine}}
is an elaborate solution to combine several documents into one.
\end{itemize}
%
See also the CTAN topic \href{http://ctan.org/topic/subdocs}{\textsf{subdocs}}
for further related packages.
The present package differs from the above solutions in that
a document structure constructed with the conventional |\include| mechanism
just needs two extra commands at the top of every file
such that all constituent files can be compiled individually.

%%%%%%%%%%%%%%%%%%%%%%%%%%%%%%%%%%%%%%%%%%%%%%%%%%%%%%%%%%%%%%%%%%%%%%%%%%%%%%%%
%\subsection{Feature Suggestions}
%
%The following is a list of features which may be useful for future
%versions of this package:
%%
%\begin{itemize}
%\item
%\ldots
%\end{itemize}

%%%%%%%%%%%%%%%%%%%%%%%%%%%%%%%%%%%%%%%%%%%%%%%%%%%%%%%%%%%%%%%%%%%%%%%%%%%%%%%%
\subsection{Revision History}

%%%%%%%%%%%%%%%%%%%%%%%%%%%%%%%%%%%%%%%%
\paragraph{v2.0:} 2018/12/30

\begin{itemize}
\item
immediate forward processing
\item
added |\childdocby| mechanism
\item
manual restructured
\end{itemize}

%%%%%%%%%%%%%%%%%%%%%%%%%%%%%%%%%%%%%%%%
\paragraph{v1.6:} 2018/01/17

\begin{itemize}
\item
application for development of include files
\item
corrections to manual
\end{itemize}

%%%%%%%%%%%%%%%%%%%%%%%%%%%%%%%%%%%%%%%%
\paragraph{v1.5:} 2017/05/21

\begin{itemize}
\item
more complete structuring introduced
\item
|\childdocof| introduced
\item
|\childdoc| renamed to |\childdocmain|
\item
|\childredirect| renamed to |\childdocforward| and |\childdocforwardprefix|
and functionality expanded
\end{itemize}

%%%%%%%%%%%%%%%%%%%%%%%%%%%%%%%%%%%%%%%%
\paragraph{v1.0:} 2017/04/27

\begin{itemize}
\item
manual and install package
\item
first version published on CTAN
\end{itemize}

%%%%%%%%%%%%%%%%%%%%%%%%%%%%%%%%%%%%%%%%
\paragraph{v0.6:} 2017/04/26

\begin{itemize}
\item
redirection mechanism added
\end{itemize}

%%%%%%%%%%%%%%%%%%%%%%%%%%%%%%%%%%%%%%%%
\paragraph{v0.5:} 2017/04/26

\begin{itemize}
\item
functionality in definition file
\end{itemize}


%%%%%%%%%%%%%%%%%%%%%%%%%%%%%%%%%%%%%%%%%%%%%%%%%%%%%%%%%%%%%%%%%%%%%%%%%%%%%%%%
%%%%%%%%%%%%%%%%%%%%%%%%%%%%%%%%%%%%%%%%%%%%%%%%%%%%%%%%%%%%%%%%%%%%%%%%%%%%%%%%
%%%%%%%%%%%%%%%%%%%%%%%%%%%%%%%%%%%%%%%%%%%%%%%%%%%%%%%%%%%%%%%%%%%%%%%%%%%%%%%%
\appendix

\settowidth\MacroIndent{\rmfamily\scriptsize 000\ }

 \DocInput{childdoc.dtx}

\end{document}
%</driver>
% \fi
%
% %%%%%%%%%%%%%%%%%%%%%%%%%%%%%%%%%%%%%%%%%%%%%%%%%%%%%%%%%%%%%%%%%%%%%%%%%%%%%%
% %%%%%%%%%%%%%%%%%%%%%%%%%%%%%%%%%%%%%%%%%%%%%%%%%%%%%%%%%%%%%%%%%%%%%%%%%%%%%%
% \section{Sample}
%\iffalse
%<*samplemain>
%\fi
%
% The following presents a sample document
% with two chapters, two parts, a title page,
% a compile flag as well as three forwarding files to set the flag.
% It consists of eight |.tex| files:
% \begin{center}
% \begin{tabular}{ll}
% |cdocsamp.tex|&main file\\
% |cdocsch1.tex|&include file for chapter 1\\
% |cdocsch2.tex|&include file for chapter 2\\
% |cdocspt3.tex|&include file for part 3\\
% |cdocspt4.tex|&include file for part 4\\
% |cdocsdrf.tex|&forwarding file for main file in draft mode\\
% |cdocsfi1.tex|&forwarding file for final version of chapter 1\\
% |cdocsfi2.tex|&forwarding file for final version of chapter 2\\
% \end{tabular}
% \end{center}
% Each of the eight files can be compiled directly by the \LaTeX{} compiler.
%
% %%%%%%%%%%%%%%%%%%%%%%%%%%%%%%%%%%%%%%
% \paragraph{Main File.}
%
% The main file is called |cdocsamp.tex|.
%
% Load the \textsf{childdoc} definitions and
% declare the filename for the main document:
%    \begin{macrocode}
\input{childdoc.def}
\childdocmain{}
%    \end{macrocode}

% Optional override for |\version| flag:
%    \begin{macrocode}
%%\ifchilddoc\else\providecommand{\version}{draft}\fi
%    \end{macrocode}

% Define the default values for the |\version| flag
% (|final| for the main file and |draft| for childs):
%    \begin{macrocode}
\ifchilddoc
\providecommand{\version}{draft}
\else
\providecommand{\version}{final}
\fi
%    \end{macrocode}

% Load the standard document class:
%    \begin{macrocode}
\documentclass[12pt]{article}
%    \end{macrocode}

% Start the document body:
%    \begin{macrocode}
\begin{document}
%    \end{macrocode}

% Declare a title page.
% Print title, part of document being processed and version flag:
%    \begin{macrocode}
\addtocounter{page}{-1}
\begin{center}
{\LARGE\bfseries{}childdoc example\par}
\vspace{1cm}
\ifchilddoc
\ifchilddocmanual part\else chapter\fi:
`\childdocname' of `\childdocjob'\par
\else
main document: `\childdocjob'\par
\fi
version: \version\par
\end{center}
\newpage
%    \end{macrocode}

% Manually include selected file,
% otherwise process as usual:
%    \begin{macrocode}
\ifchilddocmanual
\section*{part `\childdocname'}
\input{\childdocname}
\else
%    \end{macrocode}

% Include the two chapters:
%    \begin{macrocode}
\include{cdocsch1}
\include{cdocsch2}
%    \end{macrocode}

% Include the two parts unless only chapters should be displayed:
%    \begin{macrocode}
\ifchilddoc\else
\section{part three}
\input{cdocspt3}
\section{part four}
\input{cdocspt4}
\fi
%    \end{macrocode}

% Process as usual until here:
%    \begin{macrocode}
\fi
%    \end{macrocode}

% End of document body:
%    \begin{macrocode}
\end{document}
%    \end{macrocode}
%\iffalse
%</samplemain>
%\fi
%
% %%%%%%%%%%%%%%%%%%%%%%%%%%%%%%%%%%%%%%
% \paragraph{Chapter Include Files.}
%
% The include files are called |cdocsch1.tex| and |cdocsch2.tex|.
%
%\iffalse
%<*samplechap1|samplechap2>
%\fi

% Optional override for |\version| flag:
%    \begin{macrocode}
%%\providecommand{\version}{final}
%    \end{macrocode}

% Include the main document:
%    \begin{macrocode}
\input{childdoc.def}
\childdocof{cdocsamp}
%    \end{macrocode}

%\iffalse
%</samplechap1|samplechap2>
%\fi
%
%\iffalse
%<*samplechap1>
%\fi
% Some text for chapter 1:
%    \begin{macrocode}
\section{one}
some text in chapter one
%    \end{macrocode}

%\iffalse
%</samplechap1>
%\fi
% Some text for chapter 2:
%\iffalse
%<*samplechap2>
%\fi
%    \begin{macrocode}
\section{two}
more text in chapter two
%    \end{macrocode}

%\iffalse
%</samplechap2>
%\fi
%
% %%%%%%%%%%%%%%%%%%%%%%%%%%%%%%%%%%%%%%
% \paragraph{Part Include Files.}
%
% The include files are called |cdocspt3.tex| and |cdocspt4.tex|.
%
%\iffalse
%<*samplepart3|samplepart4>
%\fi

% Optional override for |\version| flag:
%    \begin{macrocode}
%%\providecommand{\version}{final}
%    \end{macrocode}

% Include the main document:
%    \begin{macrocode}
\input{childdoc.def}
\childdocby{cdocsamp}
%    \end{macrocode}

%\iffalse
%</samplepart3|samplepart4>
%\fi
%
%\iffalse
%<*samplepart3>
%\fi
% Some text for part 3:
%    \begin{macrocode}
some text in part three
%    \end{macrocode}

%\iffalse
%</samplepart3>
%\fi
% Some text for part 4:
%\iffalse
%<*samplepart4>
%\fi
%    \begin{macrocode}
more text in part four
%    \end{macrocode}

%\iffalse
%</samplepart4>
%\fi
%
% %%%%%%%%%%%%%%%%%%%%%%%%%%%%%%%%%%%%%%
% \paragraph{Forwarding for a Complete Draft.}
%
% The following forwarding file |cdocsdrf.tex|
% compiles the main document in draft mode:
%\iffalse
%<*sampledraft>
%\fi
%    \begin{macrocode}
\def\version{draft}
\input{childdoc.def}
\childdocforward{cdocsamp}
%    \end{macrocode}

%\iffalse
%</sampledraft>
%\fi
%
% %%%%%%%%%%%%%%%%%%%%%%%%%%%%%%%%%%%%%%
% \paragraph{Forwarding for Final Version of the Chapters.}
%
% The following forwarding files |cdocsfn1.tex| and |cdocsfn2.tex|
% (with identical content)
% compile the final versions of the child documents
% |cdocsch1.tex| and |cdocsch2.tex|, respectively:
%\iffalse
%<*samplefinal>
%\fi
%    \begin{macrocode}
\def\version{final}
\input{childdoc.def}
\childdocforwardprefix[cdocsamp]{cdocsfn}{cdocsch}
%    \end{macrocode}

%\iffalse
%</samplefinal>
%\fi
%
% %%%%%%%%%%%%%%%%%%%%%%%%%%%%%%%%%%%%%%
% \paragraph{Command Line Processing.}
%
% The following three command lines generate the output files
% |cdocscld|, |cdocscl1| and |cdocscl2|
% which should be identical to
% |cdocsdrf|, |cdocsch1| and |cdocsfn2|, respectively:
% \begin{center}
% \begin{tabular}{l}
% |latex -jobname cdocscld \|\\
% |  "\def\version{draft}\input{childdoc.def}\childdocforward{cdocsamp}"|\\
% |latex -jobname cdocscl1 \|\\
% |  "\input{childdoc.def}\childdocforward[cdocsamp]{cdocsch1}"|\\
% |latex -jobname cdocscl2 \|\\
% |  "\def\version{final}\input{childdoc.def}\childdocforward{cdocsch2}"|
% \end{tabular}
% \end{center}
% Note that the trailing backslash on each first line
% merely continues the input to the second line
% (for convenient cut ant paste).
% Furthermore, the command |latex| can be replaced by any
% of its alternative versions such as |pdflatex|.
%
% %%%%%%%%%%%%%%%%%%%%%%%%%%%%%%%%%%%%%%%%%%%%%%%%%%%%%%%%%%%%%%%%%%%%%%%%%%%%%%
% %%%%%%%%%%%%%%%%%%%%%%%%%%%%%%%%%%%%%%%%%%%%%%%%%%%%%%%%%%%%%%%%%%%%%%%%%%%%%%
% \section{Implementation}
%\iffalse
%<*package>
%\fi
%
% This section describes the definitions file |childdoc.def|.

% The definitions cannot be loaded using |\usepackage| or |\RequirePackage|
% which has a mechanism to prevent loading a style file more than once.
% When loading the definitions by means of |\input|
% multiple instances have to be prevented manually:
%\iffalse
%This code needs to be before the `\ProvidesFile' directive
%which is defined at the beginning of this file.
%Therefore it is also placed there and commented out here.
%</package>
%<*discard>
%\fi
%    \begin{macrocode}
\ifdefined\childdocmain\endinput\fi
%    \end{macrocode}
%\iffalse
%</discard>
%<*package>
%\fi
%
% \macro{\ifchilddoc}
% \macro{\ifchilddocmanual}
% The conditional |\ifchilddoc| tells whether a
% child (true) or main (false) document is being compiled.
% The conditional |\ifchilddocmanual| tells whether
% the |\includeonly| mechanism is used (false) or
% the selection of child files must be performed manually (true).
% The definitions initialise to false:
%    \begin{macrocode}
\newif\ifchilddoc
\newif\ifchilddocmanual
%    \end{macrocode}

% \macro{\childdocname}
% \macro{\childdocjob}
% The macro |\childdocname| stores the name of the main document
% to be compiled. The macro |\childdocjob| stores the name of
% the document on which the \LaTeX{} compiler was originally invoked.
% The content of |\jobname| cannot be compared
% to filenames specified in the source due to different catcodes.
% The following code rescans |\jobname|, stores the result
% in |\childdocname| and saves a copy in |\childdocjob|:
%    \begin{macrocode}
\edef\childdocname{\scantokens\expandafter{\jobname\noexpand}}
\let\childdocjob\childdocname
%    \end{macrocode}

% \macro{\childdocdisable}
% The macro |\childdocdisable| prevents the main file
% from being processed more than once.
% At this stage, the main document command |\childdocmain|
% is assumed to be called once again where it should do nothing.
% Any subsequent call to it should prevent
% a secondary processing of the main document
% It overwrites the forwarding commands
% |\childdocof| and |\childdocforward|
% with empty macros to prevent further inclusions of the main document:
%    \begin{macrocode}
\newcommand{\childdocdisable}
{
  \renewcommand{\childdocmain}[1]{\renewcommand{\childdocmain}[1]{\endinput}}
  \renewcommand{\childdocof}[1]{}
  \renewcommand{\childdocby}[2][]{}
  \renewcommand{\childdocforward}[2][]{}
  \renewcommand{\childdocdisable}{}
}
%    \end{macrocode}

% \macro{\childdocmain}
% The macro |\childdocmain| is to be called at the top of the main file
% with nothing or the main filename (without extension) as argument.
% First, it breaks loops.
% If the argument is not empty and does not match |\childdocname|
% (which is set by the first inclusion of |childdoc.def|),
% |\ifchilddoc| is set to true, |\includeonly| is applied to the child file
% and |\jobname| is set to the main file
% (for proper handling of |.aux| files):
%    \begin{macrocode}
\newcommand{\childdocmain}[1]
{
  \childdocdisable\childdocmain{}
  \if?#1?\else
    \begingroup
      \def\childdoctmp{#1}
      \ifx\childdoctmp\childdocname
        \def\childdoctmp{}
      \else
        \def\childdoctmp
        {
          \childdoctrue
          \includeonly{\childdocname}
          \def\childdocjob{#1}
          \def\jobname{#1}
        }
      \fi
      \expandafter
    \endgroup
    \childdoctmp
  \fi
}
%    \end{macrocode}

% \macro{\childdocof}
% The command |\childdocof| redirects
% compilation to the main file |#1|.
%    \begin{macrocode}
\newcommand{\childdocof}[1]
{
  \childdocdisable
  \childdoctrue
  \includeonly{\childdocname}
  \def\jobname{#1}
  \def\childdocjob{#1}
  \input{#1}
}
%    \end{macrocode}

% \macro{\childdocby}
% The command |\childdocby| ....
%    \begin{macrocode}
\newcommand{\childdocby}[2][]
{
  \childdocdisable
  \childdoctrue
  \childdocmanualtrue
  \if?#1?\else
    \def\jobname{#2}
  \fi
  \def\childdocjob{#2}
  \input{#2}
  \endinput
}
%    \end{macrocode}

% \macro{\childdocforward}
% The command |\childdocforward| redirects
% compilation to the main file or
% (if the optional argument is given) a child file.
% Parameters are set as if the main file
% or a child file starting with |\childdocof| was compiled.
% Then compilation is handed over to the main file:
%    \begin{macrocode}
\newcommand{\childdocforward}[2][]
{
  \begingroup
    \if?#1?
      \def\childdoctmp
      {
        \def\childdocname{#2}
        \def\childdocjob{#2}
        \def\jobname{#2}
        \input{#2}
        \endinput
      }
    \else
      \def\childdoctmp
      {
        \childdocdisable
        \def\childdocname{#2}
        \childdoctrue
        \includeonly{#2}
        \def\childdocjob{#1}
        \def\jobname{#1}
        \input{#1}
        \endinput
      }
    \fi
    \expandafter
  \endgroup
  \childdoctmp
}
%    \end{macrocode}

% \macro{\childdocforwardprefix}
% The command |\childdocforwardprefix| redirects
% compilation to the main or a child file by means of a pattern.
% The prefix |#1| in the current filename is replaced by |#2|
% and the suffix of the current filename is kept
% (it is assumed that the filename does not contain the substring `|~~~|'
% which is used as a delimiter).
% Compilation is handed over to the new file by |\childdocforward|:
%    \begin{macrocode}
\newcommand{\childdocforwardprefix}[3][]
{
  \begingroup
    \def\childdocextract #2##1~~~{\def\childdoctmp{\childdocforward[#1]{#3##1}}}
    \expandafter\childdocextract\childdocname~~~
    \expandafter
  \endgroup
  \childdoctmp
}
%    \end{macrocode}

% \macro{\childdoc}
% The deprecated macro |\childdoc| is a legacy version of |\childdocmain|:
%    \begin{macrocode}
\newcommand{\childdoc}{\childdocmain}
%    \end{macrocode}

% \macro{\childdocredirect}
% The deprecated macro |\childdocredirect| is a legacy version
% of |\childdocforward| and |\childdocforwardprefix|:
%    \begin{macrocode}
\newcommand{\childdocredirect}[2][]
{
  \begingroup
    \if?#1?
      \def\childdoctmp{\childdocforward{#2}}
    \else
      \def\childdoctmp{\childdocforwardprefix{#1}{#2}}
    \fi
    \expandafter
  \endgroup
  \childdoctmp
}
%    \end{macrocode}

%\iffalse
%</package>
%\fi
%
\endinput

\childdocby{cdocsamp}
%    \end{macrocode}

%\iffalse
%</samplepart3|samplepart4>
%\fi
%
%\iffalse
%<*samplepart3>
%\fi
% Some text for part 3:
%    \begin{macrocode}
some text in part three
%    \end{macrocode}

%\iffalse
%</samplepart3>
%\fi
% Some text for part 4:
%\iffalse
%<*samplepart4>
%\fi
%    \begin{macrocode}
more text in part four
%    \end{macrocode}

%\iffalse
%</samplepart4>
%\fi
%
% %%%%%%%%%%%%%%%%%%%%%%%%%%%%%%%%%%%%%%
% \paragraph{Forwarding for a Complete Draft.}
%
% The following forwarding file |cdocsdrf.tex|
% compiles the main document in draft mode:
%\iffalse
%<*sampledraft>
%\fi
%    \begin{macrocode}
\def\version{draft}
% \iffalse
%
% childdoc.dtx Copyright (C) 2017-2018 Niklas Beisert
%
% This work may be distributed and/or modified under the
% conditions of the LaTeX Project Public License, either version 1.3
% of this license or (at your option) any later version.
% The latest version of this license is in
%   http://www.latex-project.org/lppl.txt
% and version 1.3 or later is part of all distributions of LaTeX
% version 2005/12/01 or later.
%
% This work has the LPPL maintenance status `maintained'.
%
% The Current Maintainer of this work is Niklas Beisert.
%
% This work consists of the files childdoc.dtx and childdoc.ins
% and the derived files childdoc.def and cdocsamp.tex with
% cdocsch1.tex, cdocsch2.tex, cdocsdrf.tex, cdocsfn1.tex, cdocsfn2.tex.
%
%<package>\ifdefined\childdocmain\endinput\fi
%<package>\ProvidesFile{childdoc.def}[2018/12/30 v2.0 child document driver]
%<samplemain>\ProvidesFile{cdocsamp.tex}[2018/12/30 v2.0 sample for childdoc]
%<*driver>
%\ProvidesFile{childdoc.drv}[2018/12/30 v2.0 childdoc reference manual file]
\PassOptionsToClass{10pt,a4paper}{article}
\documentclass{ltxdoc}

\usepackage[margin=35mm]{geometry}
\usepackage{hyperref}
\usepackage{hyperxmp}
\usepackage[usenames]{color}

\hypersetup{colorlinks=true}
\hypersetup{pdfstartview=FitH}
\hypersetup{pdfpagemode=UseNone}
\hypersetup{pdfsource={}}
\hypersetup{pdflang={en-UK}}
\hypersetup{pdfcopyright={Copyright 2017-2018 Niklas Beisert.
  This work may be distributed and/or modified under the
  conditions of the LaTeX Project Public License, either version 1.3
  of this license or (at your option) any later version.}}
\hypersetup{pdflicenseurl={http://www.latex-project.org/lppl.txt}}
\hypersetup{pdfcontactaddress={ETH Zurich, ITP, HIT K,
  Wolfgang-Pauli-Strasse 27}}
\hypersetup{pdfcontactpostcode={8093}}
\hypersetup{pdfcontactcity={Zurich}}
\hypersetup{pdfcontactcountry={Switzerland}}
\hypersetup{pdfcontactemail={nbeisert@itp.phys.ethz.ch}}
\hypersetup{pdfcontacturl={http://people.phys.ethz.ch/\xmptilde nbeisert/}}

\newcommand{\secref}[1]{\hyperref[#1]{section \ref*{#1}}}

\parskip1ex
\parindent0pt
\let\olditemize\itemize
\def\itemize{\olditemize\parskip0pt}

\begin{document}

\title{The \textsf{childdoc} Package}
\hypersetup{pdftitle={The childdoc Package}}
\author{Niklas Beisert\\[2ex]
  Institut f\"ur Theoretische Physik\\
  Eidgen\"ossische Technische Hochschule Z\"urich\\
  Wolfgang-Pauli-Strasse 27, 8093 Z\"urich, Switzerland\\[1ex]
  \href{mailto:nbeisert@itp.phys.ethz.ch}
  {\texttt{nbeisert@itp.phys.ethz.ch}}}
\hypersetup{pdfauthor={Niklas Beisert}}
\hypersetup{pdfsubject={Manual for the LaTeX2e Package childdoc}}
\date{30 December 2018, \textsf{v2.0}}
\maketitle

\begin{abstract}\noindent
\textsf{childdoc} is a \LaTeXe{} package
that enables the direct compilation
of document sections included by |\include|
to individual files.
\end{abstract}

\begingroup
\parskip0ex
\tableofcontents
\endgroup

%%%%%%%%%%%%%%%%%%%%%%%%%%%%%%%%%%%%%%%%%%%%%%%%%%%%%%%%%%%%%%%%%%%%%%%%%%%%%%%%
%%%%%%%%%%%%%%%%%%%%%%%%%%%%%%%%%%%%%%%%%%%%%%%%%%%%%%%%%%%%%%%%%%%%%%%%%%%%%%%%
\section{Introduction}

\LaTeX{} provides a mechanism to structure a large document (such as a book)
into a main file and several child files (containing the chapters)
using the |\include| command.
This mechanism is beneficial for documents
which span hundreds of pages in order to
make the source file(s) more manageable.
Moreover, compilation can be restricted to
selected child files by means of the |\includeonly| command.
The latter feature can be used to reduce the compilation time while editing
(this was significantly more useful in the earlier days of \LaTeX{})
or to generate a smaller document which is easier to navigate.
Another application of |\includeonly| is to generate
documents consisting of selected parts of the complete document.

However, there are a few drawbacks of the plain |\include| mechanism:
\begin{itemize}
\item
The child files cannot be compiled on their own,
they can only be compiled via the main file.
A naive editing environment
(such as a text editor with an option
to have the current file processed by \LaTeX)
may require one to switch to the main file before compiling;
attempting to compile the child file produces errors.
\item
The main file must be modified (each time)
to adjust the |\includeonly| command
to the present needs. This easily leaves the main file in a messy state.
\item
The generated document will always carry the filename
of the main document. This is inconvenient if
several child files are to be compiled and
to be kept for distribution.
\end{itemize}

The present package provides a simple interface
to make child files individually compilable by \LaTeX{}.
Compiling a child file then has the same effect as compiling
the main file with an |\includeonly| command
to select the appropriate child.
Moreover the generated document will carry the name of the child
rather than the main file.
This resolves all three above issues.

This feature is meant to make the editing of books,
thesis documents and lecture notes somewhat more convenient.
However, the package can also be used efficiently for
composing a series of documents (such as exercise sheets)
which are typically distributed individually.
It then assists the author in generating the individual documents
(potentially in different versions)
as well as a document containing the collected series.
Another application is in developing style files
or other kinds of included material
where compilation of the style file could redirect
to a sample or test file.

%%%%%%%%%%%%%%%%%%%%%%%%%%%%%%%%%%%%%%%%%%%%%%%%%%%%%%%%%%%%%%%%%%%%%%%%%%%%%%%%
%%%%%%%%%%%%%%%%%%%%%%%%%%%%%%%%%%%%%%%%%%%%%%%%%%%%%%%%%%%%%%%%%%%%%%%%%%%%%%%%
\section{Usage}

First of all, the package \textsf{childdoc} is \emph{not} a standard
\LaTeXe{} |.sty| style file! Therefore it needs to be invoked in
a non-standard way.

%%%%%%%%%%%%%%%%%%%%%%%%%%%%%%%%%%%%%%%%%%%%%%%%%%%%%%%%%%%%%%%%%%%%%%%%%%%%%%%%
\subsection{Included Files}
\label{sec:include}

%%%%%%%%%%%%%%%%%%%%%%%%%%%%%%%%%%%%%%%%
\DescribeMacro{\childdocmain}
To use the package, add the commands
\begin{center}
\begin{tabular}{l}
|\input{childdoc.def}|\\
|\childdocmain{}|\\
\end{tabular}
\end{center}
at the very top of the main \LaTeX{} file,
in particular \emph{before} the |\documentclass| statement!
The argument of |\childdocmain| should be left empty
(but it must be present).

%%%%%%%%%%%%%%%%%%%%%%%%%%%%%%%%%%%%%%%%
\DescribeMacro{\childdocof}
Furthermore, add the commands
\begin{center}
\begin{tabular}{l}
|\input{childdoc.def}|\\
|\childdocof{|\textit{main}|}|\\
\end{tabular}
\end{center}
at the top of every child file \textit{child}
which is included by |\include{|\textit{child}|}|
from within the main file
(or at least for those files to be compiled individually).
The argument \textit{main} must be the filename of the main file.

There are a couple of
considerations in setting up the main and child documents:

%%%%%%%%%%%%%%%%%%%%%%%%%%%%%%%%%%%%%%%%
\paragraph{Restrictions.}

Please note the following restrictions:
\begin{itemize}
\item
|\childdocmain| must be called with one argument \textit{main}
to ensure compatibility with earlier version of the package.
It must either be empty (|\childdocmain{}|)
or precisely match the filename of the main file in which it is specified.
See \secref{sec:detection} for further information.
\item
The filename \textit{main} must be specified without the |.tex| extension.
\item
The filename \textit{main} is case sensitive
(even in case-insensitive file systems)
due to internal string comparison.
\item
The argument \textit{main} should be fully expanded, it cannot be a macro.
\item
Subdirectories and special characters should be avoided in filenames.
\item
The command |\childdocmain{|\textit{main}|}| must be followed by a whitespace.
It should not be followed immediately by another command
or by a comment mark `|%|'.
This is because the \TeX{} parser reads the token immediately following
the argument of |\childdocmain| and puts it
at the beginning of every child section;
however, a white\-space is ignored.
\end{itemize}

%%%%%%%%%%%%%%%%%%%%%%%%%%%%%%%%%%%%%%%%
\paragraph{Content of Main File.}

It is advisable to place all content in the child files included by |\include|.
Any output contained in the main file will appear in all child documents
unless suppressed manually;
it cannot be suppressed automatically by the |\includeonly| directive
and thus should normally be avoided.
A method to include some content in the main file
by means of conditional processing is described in \secref{sec:conditional}.

%%%%%%%%%%%%%%%%%%%%%%%%%%%%%%%%%%%%%%%%
\paragraph{Page Numbering.}

When only a part of the document is compiled,
the appropriate numbering of pages
(as well as other status parameters)
is determined from the |.aux| files.
The latter contain information from previous passes.
However this information needs to propagate through
all intermediate child documents.
Therefore the page numbering in child documents may well
be inconsistent until the complete document is compiled at least once.

A useful (if unconventional) way to always ensure a consistent
page numbering is to restart the numbering in each child document
and denote the pages by `\textit{child}|.|\textit{page}'
where \textit{child} represents the chapter/section number of the child file.
This can be achieved by the command
|\numberwithin{page}{|\textit{child}|}|
of the \textsf{amsmath} package
where \textit{child} can be |chapter| or |section|
depending on the chosen structuring.
Alternatively, one can modify the macro |\thepage| appropriately
and reset the counter |page| at the start of each child file.

%%%%%%%%%%%%%%%%%%%%%%%%%%%%%%%%%%%%%%%%%%%%%%%%%%%%%%%%%%%%%%%%%%%%%%%%%%%%%%%%
\subsection{Conditional Processing}
\label{sec:conditional}

The package provides a mechanism to compile different versions
of a document. To customise the versions further some conditional processing
can come in handy to distinguish which version is being compiled.
The package provides two macros to describe the compilation context:

%%%%%%%%%%%%%%%%%%%%%%%%%%%%%%%%%%%%%%%%
\DescribeMacro{\ifchilddoc}
The conditional |\ifchilddoc| distinguishes between the compilation of
child documents and the main document:
%
\begin{center}
|\ifchilddoc |\textit{child-code}| |[|\||else |\textit{main-code}]| \||fi|
\end{center}

%%%%%%%%%%%%%%%%%%%%%%%%%%%%%%%%%%%%%%%%
\DescribeMacro{\childdocname}
\DescribeMacro{\childdocjob}
The macro |\childdocname| contains the filename (without extension)
of the main or child file being processed.
Note that |\childdocjob| will always contain the name of the main file.

%%%%%%%%%%%%%%%%%%%%%%%%%%%%%%%%%%%%%%%%
\paragraph{Title Page.}

Conditional processing can be used to include a title or banner page
in the main document when proper precautions are taken.
Importantly, the code in the main file should ensure that the page counter
(as well as other status parameters which are stored in the |.aux| files)
takes the same value after the conditional processing.
Otherwise the page numbers may take divergent values
depending on which part is compiled.

For example, a title page could be declared by:
%
\begin{center}
\begin{tabular}{l}
|\ifchilddoc\||else|\\
|\addtocounter{page}{-1}|\\
\textit{code for title page}\\
|\newpage|\\
|\||fi|
\end{tabular}
\end{center}
%
A banner page for the child documents can be generated by:
%
\begin{center}
\begin{tabular}{l}
|\ifchilddoc|\\
|\addtocounter{page}{-1}|\\
\textit{code for banner page}\\
|\newpage|\\
|\||fi|
\end{tabular}
\end{center}
%
Here one could write a message such as:
\begin{center}
|This is the part \childdocname{} of \childdocjob{}.|
\end{center}

%%%%%%%%%%%%%%%%%%%%%%%%%%%%%%%%%%%%%%%%%%%%%%%%%%%%%%%%%%%%%%%%%%%%%%%%%%%%%%%%
\subsection{Flags}
\label{sec:flags}

The package makes it easy to generate different versions
of the main or child documents.
To this end compilation flags can be defined
and assigned different default values.
They will be particularly useful in conjunction
with the forwarding mechanism described in \secref{sec:forward}.

For example, it may be useful to have a flag |\version|
which can be set to |draft| or |final|.
The document source will contain some conditional code
depending on the value of |\version|.
Suppose further, the flag should default to |final| for the main file
and to |draft| for child files
which is a natural assignment for editing the document.
This is achieved by placing the following code
in the preamble of the main document
(below the |\childdocmain| directive):
%
\begin{center}
\begin{tabular}{l}
|\ifchilddoc|\\
|\providecommand{\version}{draft}|\\
|\||else|\\
|\providecommand{\version}{final}|\\
|\||fi|
\end{tabular}
\end{center}
%
The definition by |\providecommand| makes sure
that previous definitions are not overwritten.
Further statements |\providecommand{\version}{...}|
can thus be added before the above code to override it.

For the main file, one might add a line
(between |\childdocmain| and the above block)
%
\begin{center}
|%\ifchilddoc\||else\providecommand{\version}{draft}\||fi|
\end{center}
%
which can be uncommented to produce a draft version.
Likewise one can add a line to the very top of a child file
(above the |\childdocof{|\textit{main}|}| directive)
%
\begin{center}
|%\providecommand{\version}{final}|
\end{center}
%
which can be uncommented to produce the final version of this child document.

%%%%%%%%%%%%%%%%%%%%%%%%%%%%%%%%%%%%%%%%%%%%%%%%%%%%%%%%%%%%%%%%%%%%%%%%%%%%%%%%
\subsection{Forwarding}
\label{sec:forward}

Different versions of the main or child documents
using compilation flags as described in \secref{sec:flags}
can be (permanently) stored in different files
for convenient compilation, viewing and distribution.
To this end, the package defines a command
to pass on compilation to a different file:

%%%%%%%%%%%%%%%%%%%%%%%%%%%%%%%%%%%%%%%%
\DescribeMacro{\childdocforward}
The command |\childdocforward| redirects processing to
another source file:
%
\begin{center}
\begin{tabular}{l}
|\input{childdoc.def}|\\
|\childdocforward[|\textit{main}|]{|\textit{dest}|}|\\
\end{tabular}
\end{center}
%
The argument \textit{dest} is the destination file
(without extension).
It should be the main file or one of the child files.
Note that further \textsf{childdoc} directives
such as |\childdocof| and |\childdocforward|
in the indicated file will be processed in this form.
The optional argument \textit{main}
passes on directly to the main file \textit{main}
while pretending to compile the child \textit{dest}.
This form behaves as if \textit{dest}
issues |\childdocof{|\textit{main}|}| right away,
and no further \textsf{childdoc} directives will be processed.

%%%%%%%%%%%%%%%%%%%%%%%%%%%%%%%%%%%%%%%%
\DescribeMacro{\...prefix}
In the alternative form |\childdocforwardprefix|,
%
\begin{center}
\begin{tabular}{l}
|\input{childdoc.def}|\\
|\childdocforwardprefix[|\textit{main}|]{|\textit{prefix}|}{|\textit{dest}|}|
\end{tabular}
\end{center}
%
the destination file is determined by a pattern
depending on the current file:
To make this work, the current file must be called
`{\textit{prefix}\hspace{0.2em}\textit{suffix}}'
with \textit{prefix} matching precisely the argument.
Processing is then passed on to the file
`{\textit{dest}\hspace{0.2em}\textit{suffix}}'.
Surely, the same effect is achieved by
directly specifying the
argument `{\textit{dest}\hspace{0.2em}\textit{suffix}}'
in the first form.
However, that requires to set up a different file
for each child. With the alternative form of the command
all these files can have exactly the same content
which simplifies setting them up and maintaining them.

For example, the following file |draft.tex|
with a compilation flag |\version| as described in \secref{sec:flags}
compiles the main document as a draft:
%
\begin{center}
\begin{tabular}{l}
|\def\version{draft}|\\
|\input{childdoc.def}|\\
|\childdocforward{|\textit{main}|}|
\end{tabular}
\end{center}
%
Likewise, the following files |final|\textit{nn}|.tex|
compile the final version of the child document
|child|\textit{nn}|.tex|:
%
\begin{center}
\begin{tabular}{l}
|\def\version{final}|\\
|\input{childdoc.def}|\\
|\childdocforwardprefix{final}{child}|
\end{tabular}
\end{center}
%

Note that when several versions of a main file and/or of each child file
are to be generated, it may be convenient to set up a |Makefile| or
shell script to automatise the process.

%%%%%%%%%%%%%%%%%%%%%%%%%%%%%%%%%%%%%%%%%%%%%%%%%%%%%%%%%%%%%%%%%%%%%%%%%%%%%%%%
\subsection{Command Line Processing}
\label{sec:commandline}

The effect of redirection files can also be achieved by invoking
the \LaTeX{} compiler with a more elaborate command line.
Most conveniently this should be done as part
of a shell script or a |Makefile|.

When using \textsf{childdoc} in the main file, the following
command lines effectively perform a redirection
(note that depending on the shell being used,
backslashes may have to be doubled: `|\|' $\to$ `|\\|'):
%
\begin{center}
|... -jobname "|\textit{target}|" |\\|"|[\textit{flags}]%
|\input{childdoc.def}\childdocforward[|\textit{main}|]{|\textit{dest}|}"|
\end{center}
%
Here \textit{target} is the name of the output file,
\textit{main} is the name of the main file
and \textit{dest} is the name of the main or child file to be processed
(all filenames without extensions).
The optional argument \textit{main} can be omitted
if \textit{main} matches \textit{dest}.
Optionally, compilation \textit{flags} can be defined via |\def| commands.
This command line makes the \TeX{} engine believe
it is compiling the file \textit{target}
whose content is specified as the latter parameter.
The provided code then forwards the processing to
\textit{main} or \textit{dest} as described in \secref{sec:forward}.

%%%%%%%%%%%%%%%%%%%%%%%%%%%%%%%%%%%%%%%%%%%%%%%%%%%%%%%%%%%%%%%%%%%%%%%%%%%%%%%%
\subsection{Include by Input}
\label{sec:input}

Including child documents by |\include| has some restrictions by design.
Most notably, the content of a child document always occupies
its own set of pages; pages cannot be shared between child documents.
Usually, this behaviour makes perfect sense
because each child document contain an essential part of the document.
However, in some situations it may be desirable to compose
a document from a collection of parts
without having mandatory page breaks between then.
For this case, the package
provides a mechanism to include parts
by |\input| which can also be processed individually.
However, by construction this mechanism
requires manual handling of the content to be output.

%%%%%%%%%%%%%%%%%%%%%%%%%%%%%%%%%%%%%%%%
\DescribeMacro{\ifchilddocmanual}
The main file should be prepared as usual, see \secref{sec:include}.
However, the document body must make a distinction
between processing of an individual part and of the main document, e.g.:
%
\begin{center}
\begin{tabular}{l}
|\ifchilddocmanual|\\
|\input{\childdocname}|\\
|\||else|\\
\textit{document body with }|\input{|\textit{part}|}|\\
|\||fi|
\end{tabular}
\end{center}
%
The conditional |\ifchilddocmanual| is true whenever
a part to be included by |\input| is being compiled,
and the name of the part is stored in |\childdocname|.

%%%%%%%%%%%%%%%%%%%%%%%%%%%%%%%%%%%%%%%%
\DescribeMacro{\childdocby}
Each part to be included by |\input| should start with:
%
\begin{center}
\begin{tabular}{l}
|\input{childdoc.def}|\\
|\childdocby{|\textit{main}|}|\\
\end{tabular}
\end{center}
%
The directive |\childdocby| is similar to |\childdocof|
described in \secref{sec:include},
but the subsequent selection of content must be done manually.
To that end, both |\ifchilddoc| and |\ifchilddocmanual|
will be true upon processing of a part,
and the name of the part is stored in |\childdocname|.
Note that |\jobname| will be set to the filename of the current part
so that each part receives an individual |.aux| file
that does not interfere with the |.aux| file(s) of the main document.
This behaviour can be altered by the alternative form
|\childdocby[*]{|\textit{main}|}| (with a non-empty optional argument)
which uses the |.aux| file of the main document
by setting |\jobname| to \textit{main}.

%%%%%%%%%%%%%%%%%%%%%%%%%%%%%%%%%%%%%%%%%%%%%%%%%%%%%%%%%%%%%%%%%%%%%%%%%%%%%%%%
\subsection{Driver Development}
\label{sec:driver}

The \textsf{childdoc} mechanism can also be use for the development
of definition files such as \LaTeX{} styles or classes.
This case differs from the above setup with multiple parts
included by |\include| in that no |\includeonly| should be invoked.
This can be achieved by starting the include file
(before |\ProvidesPackage|) with:
%
\begin{center}
\begin{tabular}{l}
|\input{childdoc.def}|\\
|\childdocforward{|\textit{main}|}|\\
\end{tabular}
\end{center}
%
or alternatively with:
%
\begin{center}
\begin{tabular}{l}
|\input{childdoc.def}|\\
|\childdocby{|\textit{main}|}|\\
\end{tabular}
\end{center}
%
Both forms have slightly different effects as described above.
The main file is prepared as usual, see \secref{sec:include}.

%%%%%%%%%%%%%%%%%%%%%%%%%%%%%%%%%%%%%%%%%%%%%%%%%%%%%%%%%%%%%%%%%%%%%%%%%%%%%%%%
\subsection{Legacy Detection}
\label{sec:detection}

The directive |\childdocmain| in the main file can detect
whether the complete document or merely a child is to be compiled
even without using the directive |\childdocof|.
This method is deprecated because it is less robust
and there is no compelling reason to use it;
it is merely provided for backward compatibility
and it may be removed in future versions.

If the detection mechanism is to be used,
it is mandatory to correctly specify
the filename of the main file as the argument of |\childdocmain|:
%
\begin{center}
\begin{tabular}{l}
|\input{childdoc.def}|\\
|\childdocmain{|\textit{main}|}|\\
\end{tabular}
\end{center}
%
If |\jobname| does not match the argument \textit{main} of |\childdocmain|,
it is assumed that |\jobname| points to the child file to be compiled.
When using |\childdocmain| with the main file specified as argument,
it suffices to start a child file
with just |\input{|\textit{main}|}|
without loading of the package and using |\childdocof|.
If instead all processing is done
with the appropriate \textsf{childdoc} directives,
the argument of \textit{main} of |\childdocmain| can be empty.

An alternative version of the command line processing described
in \secref{sec:commandline} using the detection mechanism reads:
%
\begin{center}
|... -jobname "|\textit{target}|" "|[\textit{flags}]%
[|\def\jobname{|\textit{dest}|}|]|\input{|\textit{main}|}"|
\end{center}

%%%%%%%%%%%%%%%%%%%%%%%%%%%%%%%%%%%%%%%%%%%%%%%%%%%%%%%%%%%%%%%%%%%%%%%%%%%%%%%%
\subsection{Manual Code}
\label{sec:manual}

In case one cannot be certain whether the definitions file |childdoc.def|
is installed on the target \TeX{} distribution
and one prefers not to ship it,
it is conceivable to paste a few relevant commands into the sources.

To that end, drop all statements |\input{childdoc.def}|
and perform the replacements as outlined below.
Instead of |\childdocmain{|\textit{main}|}| add the following code
to the top of the main file:
%
\begin{center}
\begin{tabular}{l}
|\||ifdefined\childdocname\endinput\||fi\newif\ifchilddoc|\\
|\edef\childdocname{\scantokens\expandafter{\jobname\noexpand}}|\\
|\def\childdocmain{|\textit{main}|}\||ifx\childdocmain\childdocname\||else|\\
|\childdoctrue\includeonly{\childdocname}\let\jobname\childdocmain\||fi|\\
\end{tabular}
\end{center}
%
Instead of |\childdocof{|\textit{main}|}| just include the main file
at the top of each child file:
%
\begin{center}
|\input{|\textit{main}|}|
\end{center}
%
A simple redirection |\childdocforward{|\textit{dest}|}| is achieved by:
%
\begin{center}
|\def\jobname{|\textit{dest}|}\input{\jobname}|
\end{center}
%
The redirection with prefix
|\childdocforwardprefix[|\textit{prefix}|]{|\textit{dest}|}|
is accomplished by:
%
\begin{center}
\begin{tabular}{l}
|{\edef\jobname{\scantokens\expandafter{\jobname\noexpand}}|\\
|\def\redirectjob |\textit{prefix}|#1~~~{\gdef\jobname{|\textit{dest}|#1}}|\\
|\expandafter\redirectjob\jobname~~~}\input{\jobname}|
\end{tabular}
\end{center}

In an alternative approach,
child documents can be compiled by a specific command line
without additional code or specific definitions:
%
\begin{center}
|... -jobname "|\textit{target}|" "|[\textit{flags}]%
|\includeonly{|\textit{dest}|}\input{|\textit{main}|}"|
\end{center}
%

%%%%%%%%%%%%%%%%%%%%%%%%%%%%%%%%%%%%%%%%%%%%%%%%%%%%%%%%%%%%%%%%%%%%%%%%%%%%%%%%
%%%%%%%%%%%%%%%%%%%%%%%%%%%%%%%%%%%%%%%%%%%%%%%%%%%%%%%%%%%%%%%%%%%%%%%%%%%%%%%%
\section{Information}

%%%%%%%%%%%%%%%%%%%%%%%%%%%%%%%%%%%%%%%%%%%%%%%%%%%%%%%%%%%%%%%%%%%%%%%%%%%%%%%%
\subsection{Copyright}

Copyright \copyright{} 2017--2018 Niklas Beisert

This work may be distributed and/or modified under the
conditions of the \LaTeX{} Project Public License, either version 1.3
of this license or (at your option) any later version.
The latest version of this license is in
  \url{http://www.latex-project.org/lppl.txt}
and version 1.3 or later is part of all distributions of \LaTeX{}
version 2005/12/01 or later.

This work has the LPPL maintenance status `maintained'.

The Current Maintainer of this work is Niklas Beisert.

This work consists of the files |README.txt|, |childdoc.ins| and |childdoc.dtx|
as well as the derived files |childdoc.def|, |cdocsamp.tex|
with |cdocsch1.tex|, |cdocsch2.tex|, |cdocspt3.tex|, |cdocspt4.tex|,
|cdocsdrf.tex|, |cdocsfn1.tex|, |cdocsfn2.tex|
as well as |childdoc.pdf|.

%%%%%%%%%%%%%%%%%%%%%%%%%%%%%%%%%%%%%%%%%%%%%%%%%%%%%%%%%%%%%%%%%%%%%%%%%%%%%%%%
\subsection{Files and Installation}

The package consists of the files:
%
\begin{center}
\begin{tabular}{ll}
    |README.txt|   & readme file \\
    |childdoc.ins| & installation file \\
    |childdoc.dtx| & source file \\
    |childdoc.def| & definition file \\
    |cdocsamp.tex| & sample main file \\
    |cdocsch1.tex| & sample include file \\
    |cdocsch2.tex| & sample include file \\
    |cdocspt3.tex| & sample part file \\
    |cdocspt4.tex| & sample part file \\
    |cdocsdrf.tex| & sample redirection file \\
    |cdocsfn1.tex| & sample redirection file \\
    |cdocsfn2.tex| & sample redirection file \\
    |childdoc.pdf| & manual
\end{tabular}
\end{center}
%
The distribution consists of the files
|README.txt|, |childdoc.ins| and |childdoc.dtx|.
%
\begin{itemize}
\item
Run (pdf)\LaTeX{} on |childdoc.dtx|
to compile the manual |childdoc.pdf| (this file).
\item
Run \LaTeX{} on |childdoc.ins| to create the definitions file |childdoc.def|
and the sample |cdocsamp.tex| with include files
|cdocsch1.tex|, |cdocsch2.tex|, |cdocspt3.tex|, |cdocspt4.tex|,
|cdocsdrf.tex|, |cdocsfn1.tex|, |cdocsfn2.tex|.
Then copy the file |childdoc.def| to an appropriate directory of your \LaTeX{}
distribution, e.g.\ \textit{texmf-root}|/tex/latex/childdoc|.
\end{itemize}

%%%%%%%%%%%%%%%%%%%%%%%%%%%%%%%%%%%%%%%%%%%%%%%%%%%%%%%%%%%%%%%%%%%%%%%%%%%%%%%%
\subsection{Related CTAN Packages}

There are several other packages which offer a similar functionality:
%
\begin{itemize}
\item
The packages
\href{http://ctan.org/pkg/docmute}{\textsf{docmute}},
\href{http://ctan.org/pkg/includex}{\textsf{includex}} and
\href{http://ctan.org/pkg/standalone}{\textsf{standalone}}
provide commands to include only the document body of
a child file thus allowing both files to be compiled individually.
\item
The packages \href{http://ctan.org/pkg/subdocs}{\textsf{subdocs}}
and \href{http://ctan.org/pkg/subfiles}{\textsf{subfiles}}
provide structures in which the main and child documents can be
encapsulated and allowing them to be compiled individually.
The inclusion mechanism is different from the conventional |\include|.
\item
The package \href{http://ctan.org/pkg/combine}{\textsf{combine}}
is an elaborate solution to combine several documents into one.
\end{itemize}
%
See also the CTAN topic \href{http://ctan.org/topic/subdocs}{\textsf{subdocs}}
for further related packages.
The present package differs from the above solutions in that
a document structure constructed with the conventional |\include| mechanism
just needs two extra commands at the top of every file
such that all constituent files can be compiled individually.

%%%%%%%%%%%%%%%%%%%%%%%%%%%%%%%%%%%%%%%%%%%%%%%%%%%%%%%%%%%%%%%%%%%%%%%%%%%%%%%%
%\subsection{Feature Suggestions}
%
%The following is a list of features which may be useful for future
%versions of this package:
%%
%\begin{itemize}
%\item
%\ldots
%\end{itemize}

%%%%%%%%%%%%%%%%%%%%%%%%%%%%%%%%%%%%%%%%%%%%%%%%%%%%%%%%%%%%%%%%%%%%%%%%%%%%%%%%
\subsection{Revision History}

%%%%%%%%%%%%%%%%%%%%%%%%%%%%%%%%%%%%%%%%
\paragraph{v2.0:} 2018/12/30

\begin{itemize}
\item
immediate forward processing
\item
added |\childdocby| mechanism
\item
manual restructured
\end{itemize}

%%%%%%%%%%%%%%%%%%%%%%%%%%%%%%%%%%%%%%%%
\paragraph{v1.6:} 2018/01/17

\begin{itemize}
\item
application for development of include files
\item
corrections to manual
\end{itemize}

%%%%%%%%%%%%%%%%%%%%%%%%%%%%%%%%%%%%%%%%
\paragraph{v1.5:} 2017/05/21

\begin{itemize}
\item
more complete structuring introduced
\item
|\childdocof| introduced
\item
|\childdoc| renamed to |\childdocmain|
\item
|\childredirect| renamed to |\childdocforward| and |\childdocforwardprefix|
and functionality expanded
\end{itemize}

%%%%%%%%%%%%%%%%%%%%%%%%%%%%%%%%%%%%%%%%
\paragraph{v1.0:} 2017/04/27

\begin{itemize}
\item
manual and install package
\item
first version published on CTAN
\end{itemize}

%%%%%%%%%%%%%%%%%%%%%%%%%%%%%%%%%%%%%%%%
\paragraph{v0.6:} 2017/04/26

\begin{itemize}
\item
redirection mechanism added
\end{itemize}

%%%%%%%%%%%%%%%%%%%%%%%%%%%%%%%%%%%%%%%%
\paragraph{v0.5:} 2017/04/26

\begin{itemize}
\item
functionality in definition file
\end{itemize}


%%%%%%%%%%%%%%%%%%%%%%%%%%%%%%%%%%%%%%%%%%%%%%%%%%%%%%%%%%%%%%%%%%%%%%%%%%%%%%%%
%%%%%%%%%%%%%%%%%%%%%%%%%%%%%%%%%%%%%%%%%%%%%%%%%%%%%%%%%%%%%%%%%%%%%%%%%%%%%%%%
%%%%%%%%%%%%%%%%%%%%%%%%%%%%%%%%%%%%%%%%%%%%%%%%%%%%%%%%%%%%%%%%%%%%%%%%%%%%%%%%
\appendix

\settowidth\MacroIndent{\rmfamily\scriptsize 000\ }

 \DocInput{childdoc.dtx}

\end{document}
%</driver>
% \fi
%
% %%%%%%%%%%%%%%%%%%%%%%%%%%%%%%%%%%%%%%%%%%%%%%%%%%%%%%%%%%%%%%%%%%%%%%%%%%%%%%
% %%%%%%%%%%%%%%%%%%%%%%%%%%%%%%%%%%%%%%%%%%%%%%%%%%%%%%%%%%%%%%%%%%%%%%%%%%%%%%
% \section{Sample}
%\iffalse
%<*samplemain>
%\fi
%
% The following presents a sample document
% with two chapters, two parts, a title page,
% a compile flag as well as three forwarding files to set the flag.
% It consists of eight |.tex| files:
% \begin{center}
% \begin{tabular}{ll}
% |cdocsamp.tex|&main file\\
% |cdocsch1.tex|&include file for chapter 1\\
% |cdocsch2.tex|&include file for chapter 2\\
% |cdocspt3.tex|&include file for part 3\\
% |cdocspt4.tex|&include file for part 4\\
% |cdocsdrf.tex|&forwarding file for main file in draft mode\\
% |cdocsfi1.tex|&forwarding file for final version of chapter 1\\
% |cdocsfi2.tex|&forwarding file for final version of chapter 2\\
% \end{tabular}
% \end{center}
% Each of the eight files can be compiled directly by the \LaTeX{} compiler.
%
% %%%%%%%%%%%%%%%%%%%%%%%%%%%%%%%%%%%%%%
% \paragraph{Main File.}
%
% The main file is called |cdocsamp.tex|.
%
% Load the \textsf{childdoc} definitions and
% declare the filename for the main document:
%    \begin{macrocode}
\input{childdoc.def}
\childdocmain{}
%    \end{macrocode}

% Optional override for |\version| flag:
%    \begin{macrocode}
%%\ifchilddoc\else\providecommand{\version}{draft}\fi
%    \end{macrocode}

% Define the default values for the |\version| flag
% (|final| for the main file and |draft| for childs):
%    \begin{macrocode}
\ifchilddoc
\providecommand{\version}{draft}
\else
\providecommand{\version}{final}
\fi
%    \end{macrocode}

% Load the standard document class:
%    \begin{macrocode}
\documentclass[12pt]{article}
%    \end{macrocode}

% Start the document body:
%    \begin{macrocode}
\begin{document}
%    \end{macrocode}

% Declare a title page.
% Print title, part of document being processed and version flag:
%    \begin{macrocode}
\addtocounter{page}{-1}
\begin{center}
{\LARGE\bfseries{}childdoc example\par}
\vspace{1cm}
\ifchilddoc
\ifchilddocmanual part\else chapter\fi:
`\childdocname' of `\childdocjob'\par
\else
main document: `\childdocjob'\par
\fi
version: \version\par
\end{center}
\newpage
%    \end{macrocode}

% Manually include selected file,
% otherwise process as usual:
%    \begin{macrocode}
\ifchilddocmanual
\section*{part `\childdocname'}
\input{\childdocname}
\else
%    \end{macrocode}

% Include the two chapters:
%    \begin{macrocode}
\include{cdocsch1}
\include{cdocsch2}
%    \end{macrocode}

% Include the two parts unless only chapters should be displayed:
%    \begin{macrocode}
\ifchilddoc\else
\section{part three}
\input{cdocspt3}
\section{part four}
\input{cdocspt4}
\fi
%    \end{macrocode}

% Process as usual until here:
%    \begin{macrocode}
\fi
%    \end{macrocode}

% End of document body:
%    \begin{macrocode}
\end{document}
%    \end{macrocode}
%\iffalse
%</samplemain>
%\fi
%
% %%%%%%%%%%%%%%%%%%%%%%%%%%%%%%%%%%%%%%
% \paragraph{Chapter Include Files.}
%
% The include files are called |cdocsch1.tex| and |cdocsch2.tex|.
%
%\iffalse
%<*samplechap1|samplechap2>
%\fi

% Optional override for |\version| flag:
%    \begin{macrocode}
%%\providecommand{\version}{final}
%    \end{macrocode}

% Include the main document:
%    \begin{macrocode}
\input{childdoc.def}
\childdocof{cdocsamp}
%    \end{macrocode}

%\iffalse
%</samplechap1|samplechap2>
%\fi
%
%\iffalse
%<*samplechap1>
%\fi
% Some text for chapter 1:
%    \begin{macrocode}
\section{one}
some text in chapter one
%    \end{macrocode}

%\iffalse
%</samplechap1>
%\fi
% Some text for chapter 2:
%\iffalse
%<*samplechap2>
%\fi
%    \begin{macrocode}
\section{two}
more text in chapter two
%    \end{macrocode}

%\iffalse
%</samplechap2>
%\fi
%
% %%%%%%%%%%%%%%%%%%%%%%%%%%%%%%%%%%%%%%
% \paragraph{Part Include Files.}
%
% The include files are called |cdocspt3.tex| and |cdocspt4.tex|.
%
%\iffalse
%<*samplepart3|samplepart4>
%\fi

% Optional override for |\version| flag:
%    \begin{macrocode}
%%\providecommand{\version}{final}
%    \end{macrocode}

% Include the main document:
%    \begin{macrocode}
\input{childdoc.def}
\childdocby{cdocsamp}
%    \end{macrocode}

%\iffalse
%</samplepart3|samplepart4>
%\fi
%
%\iffalse
%<*samplepart3>
%\fi
% Some text for part 3:
%    \begin{macrocode}
some text in part three
%    \end{macrocode}

%\iffalse
%</samplepart3>
%\fi
% Some text for part 4:
%\iffalse
%<*samplepart4>
%\fi
%    \begin{macrocode}
more text in part four
%    \end{macrocode}

%\iffalse
%</samplepart4>
%\fi
%
% %%%%%%%%%%%%%%%%%%%%%%%%%%%%%%%%%%%%%%
% \paragraph{Forwarding for a Complete Draft.}
%
% The following forwarding file |cdocsdrf.tex|
% compiles the main document in draft mode:
%\iffalse
%<*sampledraft>
%\fi
%    \begin{macrocode}
\def\version{draft}
\input{childdoc.def}
\childdocforward{cdocsamp}
%    \end{macrocode}

%\iffalse
%</sampledraft>
%\fi
%
% %%%%%%%%%%%%%%%%%%%%%%%%%%%%%%%%%%%%%%
% \paragraph{Forwarding for Final Version of the Chapters.}
%
% The following forwarding files |cdocsfn1.tex| and |cdocsfn2.tex|
% (with identical content)
% compile the final versions of the child documents
% |cdocsch1.tex| and |cdocsch2.tex|, respectively:
%\iffalse
%<*samplefinal>
%\fi
%    \begin{macrocode}
\def\version{final}
\input{childdoc.def}
\childdocforwardprefix[cdocsamp]{cdocsfn}{cdocsch}
%    \end{macrocode}

%\iffalse
%</samplefinal>
%\fi
%
% %%%%%%%%%%%%%%%%%%%%%%%%%%%%%%%%%%%%%%
% \paragraph{Command Line Processing.}
%
% The following three command lines generate the output files
% |cdocscld|, |cdocscl1| and |cdocscl2|
% which should be identical to
% |cdocsdrf|, |cdocsch1| and |cdocsfn2|, respectively:
% \begin{center}
% \begin{tabular}{l}
% |latex -jobname cdocscld \|\\
% |  "\def\version{draft}\input{childdoc.def}\childdocforward{cdocsamp}"|\\
% |latex -jobname cdocscl1 \|\\
% |  "\input{childdoc.def}\childdocforward[cdocsamp]{cdocsch1}"|\\
% |latex -jobname cdocscl2 \|\\
% |  "\def\version{final}\input{childdoc.def}\childdocforward{cdocsch2}"|
% \end{tabular}
% \end{center}
% Note that the trailing backslash on each first line
% merely continues the input to the second line
% (for convenient cut ant paste).
% Furthermore, the command |latex| can be replaced by any
% of its alternative versions such as |pdflatex|.
%
% %%%%%%%%%%%%%%%%%%%%%%%%%%%%%%%%%%%%%%%%%%%%%%%%%%%%%%%%%%%%%%%%%%%%%%%%%%%%%%
% %%%%%%%%%%%%%%%%%%%%%%%%%%%%%%%%%%%%%%%%%%%%%%%%%%%%%%%%%%%%%%%%%%%%%%%%%%%%%%
% \section{Implementation}
%\iffalse
%<*package>
%\fi
%
% This section describes the definitions file |childdoc.def|.

% The definitions cannot be loaded using |\usepackage| or |\RequirePackage|
% which has a mechanism to prevent loading a style file more than once.
% When loading the definitions by means of |\input|
% multiple instances have to be prevented manually:
%\iffalse
%This code needs to be before the `\ProvidesFile' directive
%which is defined at the beginning of this file.
%Therefore it is also placed there and commented out here.
%</package>
%<*discard>
%\fi
%    \begin{macrocode}
\ifdefined\childdocmain\endinput\fi
%    \end{macrocode}
%\iffalse
%</discard>
%<*package>
%\fi
%
% \macro{\ifchilddoc}
% \macro{\ifchilddocmanual}
% The conditional |\ifchilddoc| tells whether a
% child (true) or main (false) document is being compiled.
% The conditional |\ifchilddocmanual| tells whether
% the |\includeonly| mechanism is used (false) or
% the selection of child files must be performed manually (true).
% The definitions initialise to false:
%    \begin{macrocode}
\newif\ifchilddoc
\newif\ifchilddocmanual
%    \end{macrocode}

% \macro{\childdocname}
% \macro{\childdocjob}
% The macro |\childdocname| stores the name of the main document
% to be compiled. The macro |\childdocjob| stores the name of
% the document on which the \LaTeX{} compiler was originally invoked.
% The content of |\jobname| cannot be compared
% to filenames specified in the source due to different catcodes.
% The following code rescans |\jobname|, stores the result
% in |\childdocname| and saves a copy in |\childdocjob|:
%    \begin{macrocode}
\edef\childdocname{\scantokens\expandafter{\jobname\noexpand}}
\let\childdocjob\childdocname
%    \end{macrocode}

% \macro{\childdocdisable}
% The macro |\childdocdisable| prevents the main file
% from being processed more than once.
% At this stage, the main document command |\childdocmain|
% is assumed to be called once again where it should do nothing.
% Any subsequent call to it should prevent
% a secondary processing of the main document
% It overwrites the forwarding commands
% |\childdocof| and |\childdocforward|
% with empty macros to prevent further inclusions of the main document:
%    \begin{macrocode}
\newcommand{\childdocdisable}
{
  \renewcommand{\childdocmain}[1]{\renewcommand{\childdocmain}[1]{\endinput}}
  \renewcommand{\childdocof}[1]{}
  \renewcommand{\childdocby}[2][]{}
  \renewcommand{\childdocforward}[2][]{}
  \renewcommand{\childdocdisable}{}
}
%    \end{macrocode}

% \macro{\childdocmain}
% The macro |\childdocmain| is to be called at the top of the main file
% with nothing or the main filename (without extension) as argument.
% First, it breaks loops.
% If the argument is not empty and does not match |\childdocname|
% (which is set by the first inclusion of |childdoc.def|),
% |\ifchilddoc| is set to true, |\includeonly| is applied to the child file
% and |\jobname| is set to the main file
% (for proper handling of |.aux| files):
%    \begin{macrocode}
\newcommand{\childdocmain}[1]
{
  \childdocdisable\childdocmain{}
  \if?#1?\else
    \begingroup
      \def\childdoctmp{#1}
      \ifx\childdoctmp\childdocname
        \def\childdoctmp{}
      \else
        \def\childdoctmp
        {
          \childdoctrue
          \includeonly{\childdocname}
          \def\childdocjob{#1}
          \def\jobname{#1}
        }
      \fi
      \expandafter
    \endgroup
    \childdoctmp
  \fi
}
%    \end{macrocode}

% \macro{\childdocof}
% The command |\childdocof| redirects
% compilation to the main file |#1|.
%    \begin{macrocode}
\newcommand{\childdocof}[1]
{
  \childdocdisable
  \childdoctrue
  \includeonly{\childdocname}
  \def\jobname{#1}
  \def\childdocjob{#1}
  \input{#1}
}
%    \end{macrocode}

% \macro{\childdocby}
% The command |\childdocby| ....
%    \begin{macrocode}
\newcommand{\childdocby}[2][]
{
  \childdocdisable
  \childdoctrue
  \childdocmanualtrue
  \if?#1?\else
    \def\jobname{#2}
  \fi
  \def\childdocjob{#2}
  \input{#2}
  \endinput
}
%    \end{macrocode}

% \macro{\childdocforward}
% The command |\childdocforward| redirects
% compilation to the main file or
% (if the optional argument is given) a child file.
% Parameters are set as if the main file
% or a child file starting with |\childdocof| was compiled.
% Then compilation is handed over to the main file:
%    \begin{macrocode}
\newcommand{\childdocforward}[2][]
{
  \begingroup
    \if?#1?
      \def\childdoctmp
      {
        \def\childdocname{#2}
        \def\childdocjob{#2}
        \def\jobname{#2}
        \input{#2}
        \endinput
      }
    \else
      \def\childdoctmp
      {
        \childdocdisable
        \def\childdocname{#2}
        \childdoctrue
        \includeonly{#2}
        \def\childdocjob{#1}
        \def\jobname{#1}
        \input{#1}
        \endinput
      }
    \fi
    \expandafter
  \endgroup
  \childdoctmp
}
%    \end{macrocode}

% \macro{\childdocforwardprefix}
% The command |\childdocforwardprefix| redirects
% compilation to the main or a child file by means of a pattern.
% The prefix |#1| in the current filename is replaced by |#2|
% and the suffix of the current filename is kept
% (it is assumed that the filename does not contain the substring `|~~~|'
% which is used as a delimiter).
% Compilation is handed over to the new file by |\childdocforward|:
%    \begin{macrocode}
\newcommand{\childdocforwardprefix}[3][]
{
  \begingroup
    \def\childdocextract #2##1~~~{\def\childdoctmp{\childdocforward[#1]{#3##1}}}
    \expandafter\childdocextract\childdocname~~~
    \expandafter
  \endgroup
  \childdoctmp
}
%    \end{macrocode}

% \macro{\childdoc}
% The deprecated macro |\childdoc| is a legacy version of |\childdocmain|:
%    \begin{macrocode}
\newcommand{\childdoc}{\childdocmain}
%    \end{macrocode}

% \macro{\childdocredirect}
% The deprecated macro |\childdocredirect| is a legacy version
% of |\childdocforward| and |\childdocforwardprefix|:
%    \begin{macrocode}
\newcommand{\childdocredirect}[2][]
{
  \begingroup
    \if?#1?
      \def\childdoctmp{\childdocforward{#2}}
    \else
      \def\childdoctmp{\childdocforwardprefix{#1}{#2}}
    \fi
    \expandafter
  \endgroup
  \childdoctmp
}
%    \end{macrocode}

%\iffalse
%</package>
%\fi
%
\endinput

\childdocforward{cdocsamp}
%    \end{macrocode}

%\iffalse
%</sampledraft>
%\fi
%
% %%%%%%%%%%%%%%%%%%%%%%%%%%%%%%%%%%%%%%
% \paragraph{Forwarding for Final Version of the Chapters.}
%
% The following forwarding files |cdocsfn1.tex| and |cdocsfn2.tex|
% (with identical content)
% compile the final versions of the child documents
% |cdocsch1.tex| and |cdocsch2.tex|, respectively:
%\iffalse
%<*samplefinal>
%\fi
%    \begin{macrocode}
\def\version{final}
% \iffalse
%
% childdoc.dtx Copyright (C) 2017-2018 Niklas Beisert
%
% This work may be distributed and/or modified under the
% conditions of the LaTeX Project Public License, either version 1.3
% of this license or (at your option) any later version.
% The latest version of this license is in
%   http://www.latex-project.org/lppl.txt
% and version 1.3 or later is part of all distributions of LaTeX
% version 2005/12/01 or later.
%
% This work has the LPPL maintenance status `maintained'.
%
% The Current Maintainer of this work is Niklas Beisert.
%
% This work consists of the files childdoc.dtx and childdoc.ins
% and the derived files childdoc.def and cdocsamp.tex with
% cdocsch1.tex, cdocsch2.tex, cdocsdrf.tex, cdocsfn1.tex, cdocsfn2.tex.
%
%<package>\ifdefined\childdocmain\endinput\fi
%<package>\ProvidesFile{childdoc.def}[2018/12/30 v2.0 child document driver]
%<samplemain>\ProvidesFile{cdocsamp.tex}[2018/12/30 v2.0 sample for childdoc]
%<*driver>
%\ProvidesFile{childdoc.drv}[2018/12/30 v2.0 childdoc reference manual file]
\PassOptionsToClass{10pt,a4paper}{article}
\documentclass{ltxdoc}

\usepackage[margin=35mm]{geometry}
\usepackage{hyperref}
\usepackage{hyperxmp}
\usepackage[usenames]{color}

\hypersetup{colorlinks=true}
\hypersetup{pdfstartview=FitH}
\hypersetup{pdfpagemode=UseNone}
\hypersetup{pdfsource={}}
\hypersetup{pdflang={en-UK}}
\hypersetup{pdfcopyright={Copyright 2017-2018 Niklas Beisert.
  This work may be distributed and/or modified under the
  conditions of the LaTeX Project Public License, either version 1.3
  of this license or (at your option) any later version.}}
\hypersetup{pdflicenseurl={http://www.latex-project.org/lppl.txt}}
\hypersetup{pdfcontactaddress={ETH Zurich, ITP, HIT K,
  Wolfgang-Pauli-Strasse 27}}
\hypersetup{pdfcontactpostcode={8093}}
\hypersetup{pdfcontactcity={Zurich}}
\hypersetup{pdfcontactcountry={Switzerland}}
\hypersetup{pdfcontactemail={nbeisert@itp.phys.ethz.ch}}
\hypersetup{pdfcontacturl={http://people.phys.ethz.ch/\xmptilde nbeisert/}}

\newcommand{\secref}[1]{\hyperref[#1]{section \ref*{#1}}}

\parskip1ex
\parindent0pt
\let\olditemize\itemize
\def\itemize{\olditemize\parskip0pt}

\begin{document}

\title{The \textsf{childdoc} Package}
\hypersetup{pdftitle={The childdoc Package}}
\author{Niklas Beisert\\[2ex]
  Institut f\"ur Theoretische Physik\\
  Eidgen\"ossische Technische Hochschule Z\"urich\\
  Wolfgang-Pauli-Strasse 27, 8093 Z\"urich, Switzerland\\[1ex]
  \href{mailto:nbeisert@itp.phys.ethz.ch}
  {\texttt{nbeisert@itp.phys.ethz.ch}}}
\hypersetup{pdfauthor={Niklas Beisert}}
\hypersetup{pdfsubject={Manual for the LaTeX2e Package childdoc}}
\date{30 December 2018, \textsf{v2.0}}
\maketitle

\begin{abstract}\noindent
\textsf{childdoc} is a \LaTeXe{} package
that enables the direct compilation
of document sections included by |\include|
to individual files.
\end{abstract}

\begingroup
\parskip0ex
\tableofcontents
\endgroup

%%%%%%%%%%%%%%%%%%%%%%%%%%%%%%%%%%%%%%%%%%%%%%%%%%%%%%%%%%%%%%%%%%%%%%%%%%%%%%%%
%%%%%%%%%%%%%%%%%%%%%%%%%%%%%%%%%%%%%%%%%%%%%%%%%%%%%%%%%%%%%%%%%%%%%%%%%%%%%%%%
\section{Introduction}

\LaTeX{} provides a mechanism to structure a large document (such as a book)
into a main file and several child files (containing the chapters)
using the |\include| command.
This mechanism is beneficial for documents
which span hundreds of pages in order to
make the source file(s) more manageable.
Moreover, compilation can be restricted to
selected child files by means of the |\includeonly| command.
The latter feature can be used to reduce the compilation time while editing
(this was significantly more useful in the earlier days of \LaTeX{})
or to generate a smaller document which is easier to navigate.
Another application of |\includeonly| is to generate
documents consisting of selected parts of the complete document.

However, there are a few drawbacks of the plain |\include| mechanism:
\begin{itemize}
\item
The child files cannot be compiled on their own,
they can only be compiled via the main file.
A naive editing environment
(such as a text editor with an option
to have the current file processed by \LaTeX)
may require one to switch to the main file before compiling;
attempting to compile the child file produces errors.
\item
The main file must be modified (each time)
to adjust the |\includeonly| command
to the present needs. This easily leaves the main file in a messy state.
\item
The generated document will always carry the filename
of the main document. This is inconvenient if
several child files are to be compiled and
to be kept for distribution.
\end{itemize}

The present package provides a simple interface
to make child files individually compilable by \LaTeX{}.
Compiling a child file then has the same effect as compiling
the main file with an |\includeonly| command
to select the appropriate child.
Moreover the generated document will carry the name of the child
rather than the main file.
This resolves all three above issues.

This feature is meant to make the editing of books,
thesis documents and lecture notes somewhat more convenient.
However, the package can also be used efficiently for
composing a series of documents (such as exercise sheets)
which are typically distributed individually.
It then assists the author in generating the individual documents
(potentially in different versions)
as well as a document containing the collected series.
Another application is in developing style files
or other kinds of included material
where compilation of the style file could redirect
to a sample or test file.

%%%%%%%%%%%%%%%%%%%%%%%%%%%%%%%%%%%%%%%%%%%%%%%%%%%%%%%%%%%%%%%%%%%%%%%%%%%%%%%%
%%%%%%%%%%%%%%%%%%%%%%%%%%%%%%%%%%%%%%%%%%%%%%%%%%%%%%%%%%%%%%%%%%%%%%%%%%%%%%%%
\section{Usage}

First of all, the package \textsf{childdoc} is \emph{not} a standard
\LaTeXe{} |.sty| style file! Therefore it needs to be invoked in
a non-standard way.

%%%%%%%%%%%%%%%%%%%%%%%%%%%%%%%%%%%%%%%%%%%%%%%%%%%%%%%%%%%%%%%%%%%%%%%%%%%%%%%%
\subsection{Included Files}
\label{sec:include}

%%%%%%%%%%%%%%%%%%%%%%%%%%%%%%%%%%%%%%%%
\DescribeMacro{\childdocmain}
To use the package, add the commands
\begin{center}
\begin{tabular}{l}
|\input{childdoc.def}|\\
|\childdocmain{}|\\
\end{tabular}
\end{center}
at the very top of the main \LaTeX{} file,
in particular \emph{before} the |\documentclass| statement!
The argument of |\childdocmain| should be left empty
(but it must be present).

%%%%%%%%%%%%%%%%%%%%%%%%%%%%%%%%%%%%%%%%
\DescribeMacro{\childdocof}
Furthermore, add the commands
\begin{center}
\begin{tabular}{l}
|\input{childdoc.def}|\\
|\childdocof{|\textit{main}|}|\\
\end{tabular}
\end{center}
at the top of every child file \textit{child}
which is included by |\include{|\textit{child}|}|
from within the main file
(or at least for those files to be compiled individually).
The argument \textit{main} must be the filename of the main file.

There are a couple of
considerations in setting up the main and child documents:

%%%%%%%%%%%%%%%%%%%%%%%%%%%%%%%%%%%%%%%%
\paragraph{Restrictions.}

Please note the following restrictions:
\begin{itemize}
\item
|\childdocmain| must be called with one argument \textit{main}
to ensure compatibility with earlier version of the package.
It must either be empty (|\childdocmain{}|)
or precisely match the filename of the main file in which it is specified.
See \secref{sec:detection} for further information.
\item
The filename \textit{main} must be specified without the |.tex| extension.
\item
The filename \textit{main} is case sensitive
(even in case-insensitive file systems)
due to internal string comparison.
\item
The argument \textit{main} should be fully expanded, it cannot be a macro.
\item
Subdirectories and special characters should be avoided in filenames.
\item
The command |\childdocmain{|\textit{main}|}| must be followed by a whitespace.
It should not be followed immediately by another command
or by a comment mark `|%|'.
This is because the \TeX{} parser reads the token immediately following
the argument of |\childdocmain| and puts it
at the beginning of every child section;
however, a white\-space is ignored.
\end{itemize}

%%%%%%%%%%%%%%%%%%%%%%%%%%%%%%%%%%%%%%%%
\paragraph{Content of Main File.}

It is advisable to place all content in the child files included by |\include|.
Any output contained in the main file will appear in all child documents
unless suppressed manually;
it cannot be suppressed automatically by the |\includeonly| directive
and thus should normally be avoided.
A method to include some content in the main file
by means of conditional processing is described in \secref{sec:conditional}.

%%%%%%%%%%%%%%%%%%%%%%%%%%%%%%%%%%%%%%%%
\paragraph{Page Numbering.}

When only a part of the document is compiled,
the appropriate numbering of pages
(as well as other status parameters)
is determined from the |.aux| files.
The latter contain information from previous passes.
However this information needs to propagate through
all intermediate child documents.
Therefore the page numbering in child documents may well
be inconsistent until the complete document is compiled at least once.

A useful (if unconventional) way to always ensure a consistent
page numbering is to restart the numbering in each child document
and denote the pages by `\textit{child}|.|\textit{page}'
where \textit{child} represents the chapter/section number of the child file.
This can be achieved by the command
|\numberwithin{page}{|\textit{child}|}|
of the \textsf{amsmath} package
where \textit{child} can be |chapter| or |section|
depending on the chosen structuring.
Alternatively, one can modify the macro |\thepage| appropriately
and reset the counter |page| at the start of each child file.

%%%%%%%%%%%%%%%%%%%%%%%%%%%%%%%%%%%%%%%%%%%%%%%%%%%%%%%%%%%%%%%%%%%%%%%%%%%%%%%%
\subsection{Conditional Processing}
\label{sec:conditional}

The package provides a mechanism to compile different versions
of a document. To customise the versions further some conditional processing
can come in handy to distinguish which version is being compiled.
The package provides two macros to describe the compilation context:

%%%%%%%%%%%%%%%%%%%%%%%%%%%%%%%%%%%%%%%%
\DescribeMacro{\ifchilddoc}
The conditional |\ifchilddoc| distinguishes between the compilation of
child documents and the main document:
%
\begin{center}
|\ifchilddoc |\textit{child-code}| |[|\||else |\textit{main-code}]| \||fi|
\end{center}

%%%%%%%%%%%%%%%%%%%%%%%%%%%%%%%%%%%%%%%%
\DescribeMacro{\childdocname}
\DescribeMacro{\childdocjob}
The macro |\childdocname| contains the filename (without extension)
of the main or child file being processed.
Note that |\childdocjob| will always contain the name of the main file.

%%%%%%%%%%%%%%%%%%%%%%%%%%%%%%%%%%%%%%%%
\paragraph{Title Page.}

Conditional processing can be used to include a title or banner page
in the main document when proper precautions are taken.
Importantly, the code in the main file should ensure that the page counter
(as well as other status parameters which are stored in the |.aux| files)
takes the same value after the conditional processing.
Otherwise the page numbers may take divergent values
depending on which part is compiled.

For example, a title page could be declared by:
%
\begin{center}
\begin{tabular}{l}
|\ifchilddoc\||else|\\
|\addtocounter{page}{-1}|\\
\textit{code for title page}\\
|\newpage|\\
|\||fi|
\end{tabular}
\end{center}
%
A banner page for the child documents can be generated by:
%
\begin{center}
\begin{tabular}{l}
|\ifchilddoc|\\
|\addtocounter{page}{-1}|\\
\textit{code for banner page}\\
|\newpage|\\
|\||fi|
\end{tabular}
\end{center}
%
Here one could write a message such as:
\begin{center}
|This is the part \childdocname{} of \childdocjob{}.|
\end{center}

%%%%%%%%%%%%%%%%%%%%%%%%%%%%%%%%%%%%%%%%%%%%%%%%%%%%%%%%%%%%%%%%%%%%%%%%%%%%%%%%
\subsection{Flags}
\label{sec:flags}

The package makes it easy to generate different versions
of the main or child documents.
To this end compilation flags can be defined
and assigned different default values.
They will be particularly useful in conjunction
with the forwarding mechanism described in \secref{sec:forward}.

For example, it may be useful to have a flag |\version|
which can be set to |draft| or |final|.
The document source will contain some conditional code
depending on the value of |\version|.
Suppose further, the flag should default to |final| for the main file
and to |draft| for child files
which is a natural assignment for editing the document.
This is achieved by placing the following code
in the preamble of the main document
(below the |\childdocmain| directive):
%
\begin{center}
\begin{tabular}{l}
|\ifchilddoc|\\
|\providecommand{\version}{draft}|\\
|\||else|\\
|\providecommand{\version}{final}|\\
|\||fi|
\end{tabular}
\end{center}
%
The definition by |\providecommand| makes sure
that previous definitions are not overwritten.
Further statements |\providecommand{\version}{...}|
can thus be added before the above code to override it.

For the main file, one might add a line
(between |\childdocmain| and the above block)
%
\begin{center}
|%\ifchilddoc\||else\providecommand{\version}{draft}\||fi|
\end{center}
%
which can be uncommented to produce a draft version.
Likewise one can add a line to the very top of a child file
(above the |\childdocof{|\textit{main}|}| directive)
%
\begin{center}
|%\providecommand{\version}{final}|
\end{center}
%
which can be uncommented to produce the final version of this child document.

%%%%%%%%%%%%%%%%%%%%%%%%%%%%%%%%%%%%%%%%%%%%%%%%%%%%%%%%%%%%%%%%%%%%%%%%%%%%%%%%
\subsection{Forwarding}
\label{sec:forward}

Different versions of the main or child documents
using compilation flags as described in \secref{sec:flags}
can be (permanently) stored in different files
for convenient compilation, viewing and distribution.
To this end, the package defines a command
to pass on compilation to a different file:

%%%%%%%%%%%%%%%%%%%%%%%%%%%%%%%%%%%%%%%%
\DescribeMacro{\childdocforward}
The command |\childdocforward| redirects processing to
another source file:
%
\begin{center}
\begin{tabular}{l}
|\input{childdoc.def}|\\
|\childdocforward[|\textit{main}|]{|\textit{dest}|}|\\
\end{tabular}
\end{center}
%
The argument \textit{dest} is the destination file
(without extension).
It should be the main file or one of the child files.
Note that further \textsf{childdoc} directives
such as |\childdocof| and |\childdocforward|
in the indicated file will be processed in this form.
The optional argument \textit{main}
passes on directly to the main file \textit{main}
while pretending to compile the child \textit{dest}.
This form behaves as if \textit{dest}
issues |\childdocof{|\textit{main}|}| right away,
and no further \textsf{childdoc} directives will be processed.

%%%%%%%%%%%%%%%%%%%%%%%%%%%%%%%%%%%%%%%%
\DescribeMacro{\...prefix}
In the alternative form |\childdocforwardprefix|,
%
\begin{center}
\begin{tabular}{l}
|\input{childdoc.def}|\\
|\childdocforwardprefix[|\textit{main}|]{|\textit{prefix}|}{|\textit{dest}|}|
\end{tabular}
\end{center}
%
the destination file is determined by a pattern
depending on the current file:
To make this work, the current file must be called
`{\textit{prefix}\hspace{0.2em}\textit{suffix}}'
with \textit{prefix} matching precisely the argument.
Processing is then passed on to the file
`{\textit{dest}\hspace{0.2em}\textit{suffix}}'.
Surely, the same effect is achieved by
directly specifying the
argument `{\textit{dest}\hspace{0.2em}\textit{suffix}}'
in the first form.
However, that requires to set up a different file
for each child. With the alternative form of the command
all these files can have exactly the same content
which simplifies setting them up and maintaining them.

For example, the following file |draft.tex|
with a compilation flag |\version| as described in \secref{sec:flags}
compiles the main document as a draft:
%
\begin{center}
\begin{tabular}{l}
|\def\version{draft}|\\
|\input{childdoc.def}|\\
|\childdocforward{|\textit{main}|}|
\end{tabular}
\end{center}
%
Likewise, the following files |final|\textit{nn}|.tex|
compile the final version of the child document
|child|\textit{nn}|.tex|:
%
\begin{center}
\begin{tabular}{l}
|\def\version{final}|\\
|\input{childdoc.def}|\\
|\childdocforwardprefix{final}{child}|
\end{tabular}
\end{center}
%

Note that when several versions of a main file and/or of each child file
are to be generated, it may be convenient to set up a |Makefile| or
shell script to automatise the process.

%%%%%%%%%%%%%%%%%%%%%%%%%%%%%%%%%%%%%%%%%%%%%%%%%%%%%%%%%%%%%%%%%%%%%%%%%%%%%%%%
\subsection{Command Line Processing}
\label{sec:commandline}

The effect of redirection files can also be achieved by invoking
the \LaTeX{} compiler with a more elaborate command line.
Most conveniently this should be done as part
of a shell script or a |Makefile|.

When using \textsf{childdoc} in the main file, the following
command lines effectively perform a redirection
(note that depending on the shell being used,
backslashes may have to be doubled: `|\|' $\to$ `|\\|'):
%
\begin{center}
|... -jobname "|\textit{target}|" |\\|"|[\textit{flags}]%
|\input{childdoc.def}\childdocforward[|\textit{main}|]{|\textit{dest}|}"|
\end{center}
%
Here \textit{target} is the name of the output file,
\textit{main} is the name of the main file
and \textit{dest} is the name of the main or child file to be processed
(all filenames without extensions).
The optional argument \textit{main} can be omitted
if \textit{main} matches \textit{dest}.
Optionally, compilation \textit{flags} can be defined via |\def| commands.
This command line makes the \TeX{} engine believe
it is compiling the file \textit{target}
whose content is specified as the latter parameter.
The provided code then forwards the processing to
\textit{main} or \textit{dest} as described in \secref{sec:forward}.

%%%%%%%%%%%%%%%%%%%%%%%%%%%%%%%%%%%%%%%%%%%%%%%%%%%%%%%%%%%%%%%%%%%%%%%%%%%%%%%%
\subsection{Include by Input}
\label{sec:input}

Including child documents by |\include| has some restrictions by design.
Most notably, the content of a child document always occupies
its own set of pages; pages cannot be shared between child documents.
Usually, this behaviour makes perfect sense
because each child document contain an essential part of the document.
However, in some situations it may be desirable to compose
a document from a collection of parts
without having mandatory page breaks between then.
For this case, the package
provides a mechanism to include parts
by |\input| which can also be processed individually.
However, by construction this mechanism
requires manual handling of the content to be output.

%%%%%%%%%%%%%%%%%%%%%%%%%%%%%%%%%%%%%%%%
\DescribeMacro{\ifchilddocmanual}
The main file should be prepared as usual, see \secref{sec:include}.
However, the document body must make a distinction
between processing of an individual part and of the main document, e.g.:
%
\begin{center}
\begin{tabular}{l}
|\ifchilddocmanual|\\
|\input{\childdocname}|\\
|\||else|\\
\textit{document body with }|\input{|\textit{part}|}|\\
|\||fi|
\end{tabular}
\end{center}
%
The conditional |\ifchilddocmanual| is true whenever
a part to be included by |\input| is being compiled,
and the name of the part is stored in |\childdocname|.

%%%%%%%%%%%%%%%%%%%%%%%%%%%%%%%%%%%%%%%%
\DescribeMacro{\childdocby}
Each part to be included by |\input| should start with:
%
\begin{center}
\begin{tabular}{l}
|\input{childdoc.def}|\\
|\childdocby{|\textit{main}|}|\\
\end{tabular}
\end{center}
%
The directive |\childdocby| is similar to |\childdocof|
described in \secref{sec:include},
but the subsequent selection of content must be done manually.
To that end, both |\ifchilddoc| and |\ifchilddocmanual|
will be true upon processing of a part,
and the name of the part is stored in |\childdocname|.
Note that |\jobname| will be set to the filename of the current part
so that each part receives an individual |.aux| file
that does not interfere with the |.aux| file(s) of the main document.
This behaviour can be altered by the alternative form
|\childdocby[*]{|\textit{main}|}| (with a non-empty optional argument)
which uses the |.aux| file of the main document
by setting |\jobname| to \textit{main}.

%%%%%%%%%%%%%%%%%%%%%%%%%%%%%%%%%%%%%%%%%%%%%%%%%%%%%%%%%%%%%%%%%%%%%%%%%%%%%%%%
\subsection{Driver Development}
\label{sec:driver}

The \textsf{childdoc} mechanism can also be use for the development
of definition files such as \LaTeX{} styles or classes.
This case differs from the above setup with multiple parts
included by |\include| in that no |\includeonly| should be invoked.
This can be achieved by starting the include file
(before |\ProvidesPackage|) with:
%
\begin{center}
\begin{tabular}{l}
|\input{childdoc.def}|\\
|\childdocforward{|\textit{main}|}|\\
\end{tabular}
\end{center}
%
or alternatively with:
%
\begin{center}
\begin{tabular}{l}
|\input{childdoc.def}|\\
|\childdocby{|\textit{main}|}|\\
\end{tabular}
\end{center}
%
Both forms have slightly different effects as described above.
The main file is prepared as usual, see \secref{sec:include}.

%%%%%%%%%%%%%%%%%%%%%%%%%%%%%%%%%%%%%%%%%%%%%%%%%%%%%%%%%%%%%%%%%%%%%%%%%%%%%%%%
\subsection{Legacy Detection}
\label{sec:detection}

The directive |\childdocmain| in the main file can detect
whether the complete document or merely a child is to be compiled
even without using the directive |\childdocof|.
This method is deprecated because it is less robust
and there is no compelling reason to use it;
it is merely provided for backward compatibility
and it may be removed in future versions.

If the detection mechanism is to be used,
it is mandatory to correctly specify
the filename of the main file as the argument of |\childdocmain|:
%
\begin{center}
\begin{tabular}{l}
|\input{childdoc.def}|\\
|\childdocmain{|\textit{main}|}|\\
\end{tabular}
\end{center}
%
If |\jobname| does not match the argument \textit{main} of |\childdocmain|,
it is assumed that |\jobname| points to the child file to be compiled.
When using |\childdocmain| with the main file specified as argument,
it suffices to start a child file
with just |\input{|\textit{main}|}|
without loading of the package and using |\childdocof|.
If instead all processing is done
with the appropriate \textsf{childdoc} directives,
the argument of \textit{main} of |\childdocmain| can be empty.

An alternative version of the command line processing described
in \secref{sec:commandline} using the detection mechanism reads:
%
\begin{center}
|... -jobname "|\textit{target}|" "|[\textit{flags}]%
[|\def\jobname{|\textit{dest}|}|]|\input{|\textit{main}|}"|
\end{center}

%%%%%%%%%%%%%%%%%%%%%%%%%%%%%%%%%%%%%%%%%%%%%%%%%%%%%%%%%%%%%%%%%%%%%%%%%%%%%%%%
\subsection{Manual Code}
\label{sec:manual}

In case one cannot be certain whether the definitions file |childdoc.def|
is installed on the target \TeX{} distribution
and one prefers not to ship it,
it is conceivable to paste a few relevant commands into the sources.

To that end, drop all statements |\input{childdoc.def}|
and perform the replacements as outlined below.
Instead of |\childdocmain{|\textit{main}|}| add the following code
to the top of the main file:
%
\begin{center}
\begin{tabular}{l}
|\||ifdefined\childdocname\endinput\||fi\newif\ifchilddoc|\\
|\edef\childdocname{\scantokens\expandafter{\jobname\noexpand}}|\\
|\def\childdocmain{|\textit{main}|}\||ifx\childdocmain\childdocname\||else|\\
|\childdoctrue\includeonly{\childdocname}\let\jobname\childdocmain\||fi|\\
\end{tabular}
\end{center}
%
Instead of |\childdocof{|\textit{main}|}| just include the main file
at the top of each child file:
%
\begin{center}
|\input{|\textit{main}|}|
\end{center}
%
A simple redirection |\childdocforward{|\textit{dest}|}| is achieved by:
%
\begin{center}
|\def\jobname{|\textit{dest}|}\input{\jobname}|
\end{center}
%
The redirection with prefix
|\childdocforwardprefix[|\textit{prefix}|]{|\textit{dest}|}|
is accomplished by:
%
\begin{center}
\begin{tabular}{l}
|{\edef\jobname{\scantokens\expandafter{\jobname\noexpand}}|\\
|\def\redirectjob |\textit{prefix}|#1~~~{\gdef\jobname{|\textit{dest}|#1}}|\\
|\expandafter\redirectjob\jobname~~~}\input{\jobname}|
\end{tabular}
\end{center}

In an alternative approach,
child documents can be compiled by a specific command line
without additional code or specific definitions:
%
\begin{center}
|... -jobname "|\textit{target}|" "|[\textit{flags}]%
|\includeonly{|\textit{dest}|}\input{|\textit{main}|}"|
\end{center}
%

%%%%%%%%%%%%%%%%%%%%%%%%%%%%%%%%%%%%%%%%%%%%%%%%%%%%%%%%%%%%%%%%%%%%%%%%%%%%%%%%
%%%%%%%%%%%%%%%%%%%%%%%%%%%%%%%%%%%%%%%%%%%%%%%%%%%%%%%%%%%%%%%%%%%%%%%%%%%%%%%%
\section{Information}

%%%%%%%%%%%%%%%%%%%%%%%%%%%%%%%%%%%%%%%%%%%%%%%%%%%%%%%%%%%%%%%%%%%%%%%%%%%%%%%%
\subsection{Copyright}

Copyright \copyright{} 2017--2018 Niklas Beisert

This work may be distributed and/or modified under the
conditions of the \LaTeX{} Project Public License, either version 1.3
of this license or (at your option) any later version.
The latest version of this license is in
  \url{http://www.latex-project.org/lppl.txt}
and version 1.3 or later is part of all distributions of \LaTeX{}
version 2005/12/01 or later.

This work has the LPPL maintenance status `maintained'.

The Current Maintainer of this work is Niklas Beisert.

This work consists of the files |README.txt|, |childdoc.ins| and |childdoc.dtx|
as well as the derived files |childdoc.def|, |cdocsamp.tex|
with |cdocsch1.tex|, |cdocsch2.tex|, |cdocspt3.tex|, |cdocspt4.tex|,
|cdocsdrf.tex|, |cdocsfn1.tex|, |cdocsfn2.tex|
as well as |childdoc.pdf|.

%%%%%%%%%%%%%%%%%%%%%%%%%%%%%%%%%%%%%%%%%%%%%%%%%%%%%%%%%%%%%%%%%%%%%%%%%%%%%%%%
\subsection{Files and Installation}

The package consists of the files:
%
\begin{center}
\begin{tabular}{ll}
    |README.txt|   & readme file \\
    |childdoc.ins| & installation file \\
    |childdoc.dtx| & source file \\
    |childdoc.def| & definition file \\
    |cdocsamp.tex| & sample main file \\
    |cdocsch1.tex| & sample include file \\
    |cdocsch2.tex| & sample include file \\
    |cdocspt3.tex| & sample part file \\
    |cdocspt4.tex| & sample part file \\
    |cdocsdrf.tex| & sample redirection file \\
    |cdocsfn1.tex| & sample redirection file \\
    |cdocsfn2.tex| & sample redirection file \\
    |childdoc.pdf| & manual
\end{tabular}
\end{center}
%
The distribution consists of the files
|README.txt|, |childdoc.ins| and |childdoc.dtx|.
%
\begin{itemize}
\item
Run (pdf)\LaTeX{} on |childdoc.dtx|
to compile the manual |childdoc.pdf| (this file).
\item
Run \LaTeX{} on |childdoc.ins| to create the definitions file |childdoc.def|
and the sample |cdocsamp.tex| with include files
|cdocsch1.tex|, |cdocsch2.tex|, |cdocspt3.tex|, |cdocspt4.tex|,
|cdocsdrf.tex|, |cdocsfn1.tex|, |cdocsfn2.tex|.
Then copy the file |childdoc.def| to an appropriate directory of your \LaTeX{}
distribution, e.g.\ \textit{texmf-root}|/tex/latex/childdoc|.
\end{itemize}

%%%%%%%%%%%%%%%%%%%%%%%%%%%%%%%%%%%%%%%%%%%%%%%%%%%%%%%%%%%%%%%%%%%%%%%%%%%%%%%%
\subsection{Related CTAN Packages}

There are several other packages which offer a similar functionality:
%
\begin{itemize}
\item
The packages
\href{http://ctan.org/pkg/docmute}{\textsf{docmute}},
\href{http://ctan.org/pkg/includex}{\textsf{includex}} and
\href{http://ctan.org/pkg/standalone}{\textsf{standalone}}
provide commands to include only the document body of
a child file thus allowing both files to be compiled individually.
\item
The packages \href{http://ctan.org/pkg/subdocs}{\textsf{subdocs}}
and \href{http://ctan.org/pkg/subfiles}{\textsf{subfiles}}
provide structures in which the main and child documents can be
encapsulated and allowing them to be compiled individually.
The inclusion mechanism is different from the conventional |\include|.
\item
The package \href{http://ctan.org/pkg/combine}{\textsf{combine}}
is an elaborate solution to combine several documents into one.
\end{itemize}
%
See also the CTAN topic \href{http://ctan.org/topic/subdocs}{\textsf{subdocs}}
for further related packages.
The present package differs from the above solutions in that
a document structure constructed with the conventional |\include| mechanism
just needs two extra commands at the top of every file
such that all constituent files can be compiled individually.

%%%%%%%%%%%%%%%%%%%%%%%%%%%%%%%%%%%%%%%%%%%%%%%%%%%%%%%%%%%%%%%%%%%%%%%%%%%%%%%%
%\subsection{Feature Suggestions}
%
%The following is a list of features which may be useful for future
%versions of this package:
%%
%\begin{itemize}
%\item
%\ldots
%\end{itemize}

%%%%%%%%%%%%%%%%%%%%%%%%%%%%%%%%%%%%%%%%%%%%%%%%%%%%%%%%%%%%%%%%%%%%%%%%%%%%%%%%
\subsection{Revision History}

%%%%%%%%%%%%%%%%%%%%%%%%%%%%%%%%%%%%%%%%
\paragraph{v2.0:} 2018/12/30

\begin{itemize}
\item
immediate forward processing
\item
added |\childdocby| mechanism
\item
manual restructured
\end{itemize}

%%%%%%%%%%%%%%%%%%%%%%%%%%%%%%%%%%%%%%%%
\paragraph{v1.6:} 2018/01/17

\begin{itemize}
\item
application for development of include files
\item
corrections to manual
\end{itemize}

%%%%%%%%%%%%%%%%%%%%%%%%%%%%%%%%%%%%%%%%
\paragraph{v1.5:} 2017/05/21

\begin{itemize}
\item
more complete structuring introduced
\item
|\childdocof| introduced
\item
|\childdoc| renamed to |\childdocmain|
\item
|\childredirect| renamed to |\childdocforward| and |\childdocforwardprefix|
and functionality expanded
\end{itemize}

%%%%%%%%%%%%%%%%%%%%%%%%%%%%%%%%%%%%%%%%
\paragraph{v1.0:} 2017/04/27

\begin{itemize}
\item
manual and install package
\item
first version published on CTAN
\end{itemize}

%%%%%%%%%%%%%%%%%%%%%%%%%%%%%%%%%%%%%%%%
\paragraph{v0.6:} 2017/04/26

\begin{itemize}
\item
redirection mechanism added
\end{itemize}

%%%%%%%%%%%%%%%%%%%%%%%%%%%%%%%%%%%%%%%%
\paragraph{v0.5:} 2017/04/26

\begin{itemize}
\item
functionality in definition file
\end{itemize}


%%%%%%%%%%%%%%%%%%%%%%%%%%%%%%%%%%%%%%%%%%%%%%%%%%%%%%%%%%%%%%%%%%%%%%%%%%%%%%%%
%%%%%%%%%%%%%%%%%%%%%%%%%%%%%%%%%%%%%%%%%%%%%%%%%%%%%%%%%%%%%%%%%%%%%%%%%%%%%%%%
%%%%%%%%%%%%%%%%%%%%%%%%%%%%%%%%%%%%%%%%%%%%%%%%%%%%%%%%%%%%%%%%%%%%%%%%%%%%%%%%
\appendix

\settowidth\MacroIndent{\rmfamily\scriptsize 000\ }

 \DocInput{childdoc.dtx}

\end{document}
%</driver>
% \fi
%
% %%%%%%%%%%%%%%%%%%%%%%%%%%%%%%%%%%%%%%%%%%%%%%%%%%%%%%%%%%%%%%%%%%%%%%%%%%%%%%
% %%%%%%%%%%%%%%%%%%%%%%%%%%%%%%%%%%%%%%%%%%%%%%%%%%%%%%%%%%%%%%%%%%%%%%%%%%%%%%
% \section{Sample}
%\iffalse
%<*samplemain>
%\fi
%
% The following presents a sample document
% with two chapters, two parts, a title page,
% a compile flag as well as three forwarding files to set the flag.
% It consists of eight |.tex| files:
% \begin{center}
% \begin{tabular}{ll}
% |cdocsamp.tex|&main file\\
% |cdocsch1.tex|&include file for chapter 1\\
% |cdocsch2.tex|&include file for chapter 2\\
% |cdocspt3.tex|&include file for part 3\\
% |cdocspt4.tex|&include file for part 4\\
% |cdocsdrf.tex|&forwarding file for main file in draft mode\\
% |cdocsfi1.tex|&forwarding file for final version of chapter 1\\
% |cdocsfi2.tex|&forwarding file for final version of chapter 2\\
% \end{tabular}
% \end{center}
% Each of the eight files can be compiled directly by the \LaTeX{} compiler.
%
% %%%%%%%%%%%%%%%%%%%%%%%%%%%%%%%%%%%%%%
% \paragraph{Main File.}
%
% The main file is called |cdocsamp.tex|.
%
% Load the \textsf{childdoc} definitions and
% declare the filename for the main document:
%    \begin{macrocode}
\input{childdoc.def}
\childdocmain{}
%    \end{macrocode}

% Optional override for |\version| flag:
%    \begin{macrocode}
%%\ifchilddoc\else\providecommand{\version}{draft}\fi
%    \end{macrocode}

% Define the default values for the |\version| flag
% (|final| for the main file and |draft| for childs):
%    \begin{macrocode}
\ifchilddoc
\providecommand{\version}{draft}
\else
\providecommand{\version}{final}
\fi
%    \end{macrocode}

% Load the standard document class:
%    \begin{macrocode}
\documentclass[12pt]{article}
%    \end{macrocode}

% Start the document body:
%    \begin{macrocode}
\begin{document}
%    \end{macrocode}

% Declare a title page.
% Print title, part of document being processed and version flag:
%    \begin{macrocode}
\addtocounter{page}{-1}
\begin{center}
{\LARGE\bfseries{}childdoc example\par}
\vspace{1cm}
\ifchilddoc
\ifchilddocmanual part\else chapter\fi:
`\childdocname' of `\childdocjob'\par
\else
main document: `\childdocjob'\par
\fi
version: \version\par
\end{center}
\newpage
%    \end{macrocode}

% Manually include selected file,
% otherwise process as usual:
%    \begin{macrocode}
\ifchilddocmanual
\section*{part `\childdocname'}
\input{\childdocname}
\else
%    \end{macrocode}

% Include the two chapters:
%    \begin{macrocode}
\include{cdocsch1}
\include{cdocsch2}
%    \end{macrocode}

% Include the two parts unless only chapters should be displayed:
%    \begin{macrocode}
\ifchilddoc\else
\section{part three}
\input{cdocspt3}
\section{part four}
\input{cdocspt4}
\fi
%    \end{macrocode}

% Process as usual until here:
%    \begin{macrocode}
\fi
%    \end{macrocode}

% End of document body:
%    \begin{macrocode}
\end{document}
%    \end{macrocode}
%\iffalse
%</samplemain>
%\fi
%
% %%%%%%%%%%%%%%%%%%%%%%%%%%%%%%%%%%%%%%
% \paragraph{Chapter Include Files.}
%
% The include files are called |cdocsch1.tex| and |cdocsch2.tex|.
%
%\iffalse
%<*samplechap1|samplechap2>
%\fi

% Optional override for |\version| flag:
%    \begin{macrocode}
%%\providecommand{\version}{final}
%    \end{macrocode}

% Include the main document:
%    \begin{macrocode}
\input{childdoc.def}
\childdocof{cdocsamp}
%    \end{macrocode}

%\iffalse
%</samplechap1|samplechap2>
%\fi
%
%\iffalse
%<*samplechap1>
%\fi
% Some text for chapter 1:
%    \begin{macrocode}
\section{one}
some text in chapter one
%    \end{macrocode}

%\iffalse
%</samplechap1>
%\fi
% Some text for chapter 2:
%\iffalse
%<*samplechap2>
%\fi
%    \begin{macrocode}
\section{two}
more text in chapter two
%    \end{macrocode}

%\iffalse
%</samplechap2>
%\fi
%
% %%%%%%%%%%%%%%%%%%%%%%%%%%%%%%%%%%%%%%
% \paragraph{Part Include Files.}
%
% The include files are called |cdocspt3.tex| and |cdocspt4.tex|.
%
%\iffalse
%<*samplepart3|samplepart4>
%\fi

% Optional override for |\version| flag:
%    \begin{macrocode}
%%\providecommand{\version}{final}
%    \end{macrocode}

% Include the main document:
%    \begin{macrocode}
\input{childdoc.def}
\childdocby{cdocsamp}
%    \end{macrocode}

%\iffalse
%</samplepart3|samplepart4>
%\fi
%
%\iffalse
%<*samplepart3>
%\fi
% Some text for part 3:
%    \begin{macrocode}
some text in part three
%    \end{macrocode}

%\iffalse
%</samplepart3>
%\fi
% Some text for part 4:
%\iffalse
%<*samplepart4>
%\fi
%    \begin{macrocode}
more text in part four
%    \end{macrocode}

%\iffalse
%</samplepart4>
%\fi
%
% %%%%%%%%%%%%%%%%%%%%%%%%%%%%%%%%%%%%%%
% \paragraph{Forwarding for a Complete Draft.}
%
% The following forwarding file |cdocsdrf.tex|
% compiles the main document in draft mode:
%\iffalse
%<*sampledraft>
%\fi
%    \begin{macrocode}
\def\version{draft}
\input{childdoc.def}
\childdocforward{cdocsamp}
%    \end{macrocode}

%\iffalse
%</sampledraft>
%\fi
%
% %%%%%%%%%%%%%%%%%%%%%%%%%%%%%%%%%%%%%%
% \paragraph{Forwarding for Final Version of the Chapters.}
%
% The following forwarding files |cdocsfn1.tex| and |cdocsfn2.tex|
% (with identical content)
% compile the final versions of the child documents
% |cdocsch1.tex| and |cdocsch2.tex|, respectively:
%\iffalse
%<*samplefinal>
%\fi
%    \begin{macrocode}
\def\version{final}
\input{childdoc.def}
\childdocforwardprefix[cdocsamp]{cdocsfn}{cdocsch}
%    \end{macrocode}

%\iffalse
%</samplefinal>
%\fi
%
% %%%%%%%%%%%%%%%%%%%%%%%%%%%%%%%%%%%%%%
% \paragraph{Command Line Processing.}
%
% The following three command lines generate the output files
% |cdocscld|, |cdocscl1| and |cdocscl2|
% which should be identical to
% |cdocsdrf|, |cdocsch1| and |cdocsfn2|, respectively:
% \begin{center}
% \begin{tabular}{l}
% |latex -jobname cdocscld \|\\
% |  "\def\version{draft}\input{childdoc.def}\childdocforward{cdocsamp}"|\\
% |latex -jobname cdocscl1 \|\\
% |  "\input{childdoc.def}\childdocforward[cdocsamp]{cdocsch1}"|\\
% |latex -jobname cdocscl2 \|\\
% |  "\def\version{final}\input{childdoc.def}\childdocforward{cdocsch2}"|
% \end{tabular}
% \end{center}
% Note that the trailing backslash on each first line
% merely continues the input to the second line
% (for convenient cut ant paste).
% Furthermore, the command |latex| can be replaced by any
% of its alternative versions such as |pdflatex|.
%
% %%%%%%%%%%%%%%%%%%%%%%%%%%%%%%%%%%%%%%%%%%%%%%%%%%%%%%%%%%%%%%%%%%%%%%%%%%%%%%
% %%%%%%%%%%%%%%%%%%%%%%%%%%%%%%%%%%%%%%%%%%%%%%%%%%%%%%%%%%%%%%%%%%%%%%%%%%%%%%
% \section{Implementation}
%\iffalse
%<*package>
%\fi
%
% This section describes the definitions file |childdoc.def|.

% The definitions cannot be loaded using |\usepackage| or |\RequirePackage|
% which has a mechanism to prevent loading a style file more than once.
% When loading the definitions by means of |\input|
% multiple instances have to be prevented manually:
%\iffalse
%This code needs to be before the `\ProvidesFile' directive
%which is defined at the beginning of this file.
%Therefore it is also placed there and commented out here.
%</package>
%<*discard>
%\fi
%    \begin{macrocode}
\ifdefined\childdocmain\endinput\fi
%    \end{macrocode}
%\iffalse
%</discard>
%<*package>
%\fi
%
% \macro{\ifchilddoc}
% \macro{\ifchilddocmanual}
% The conditional |\ifchilddoc| tells whether a
% child (true) or main (false) document is being compiled.
% The conditional |\ifchilddocmanual| tells whether
% the |\includeonly| mechanism is used (false) or
% the selection of child files must be performed manually (true).
% The definitions initialise to false:
%    \begin{macrocode}
\newif\ifchilddoc
\newif\ifchilddocmanual
%    \end{macrocode}

% \macro{\childdocname}
% \macro{\childdocjob}
% The macro |\childdocname| stores the name of the main document
% to be compiled. The macro |\childdocjob| stores the name of
% the document on which the \LaTeX{} compiler was originally invoked.
% The content of |\jobname| cannot be compared
% to filenames specified in the source due to different catcodes.
% The following code rescans |\jobname|, stores the result
% in |\childdocname| and saves a copy in |\childdocjob|:
%    \begin{macrocode}
\edef\childdocname{\scantokens\expandafter{\jobname\noexpand}}
\let\childdocjob\childdocname
%    \end{macrocode}

% \macro{\childdocdisable}
% The macro |\childdocdisable| prevents the main file
% from being processed more than once.
% At this stage, the main document command |\childdocmain|
% is assumed to be called once again where it should do nothing.
% Any subsequent call to it should prevent
% a secondary processing of the main document
% It overwrites the forwarding commands
% |\childdocof| and |\childdocforward|
% with empty macros to prevent further inclusions of the main document:
%    \begin{macrocode}
\newcommand{\childdocdisable}
{
  \renewcommand{\childdocmain}[1]{\renewcommand{\childdocmain}[1]{\endinput}}
  \renewcommand{\childdocof}[1]{}
  \renewcommand{\childdocby}[2][]{}
  \renewcommand{\childdocforward}[2][]{}
  \renewcommand{\childdocdisable}{}
}
%    \end{macrocode}

% \macro{\childdocmain}
% The macro |\childdocmain| is to be called at the top of the main file
% with nothing or the main filename (without extension) as argument.
% First, it breaks loops.
% If the argument is not empty and does not match |\childdocname|
% (which is set by the first inclusion of |childdoc.def|),
% |\ifchilddoc| is set to true, |\includeonly| is applied to the child file
% and |\jobname| is set to the main file
% (for proper handling of |.aux| files):
%    \begin{macrocode}
\newcommand{\childdocmain}[1]
{
  \childdocdisable\childdocmain{}
  \if?#1?\else
    \begingroup
      \def\childdoctmp{#1}
      \ifx\childdoctmp\childdocname
        \def\childdoctmp{}
      \else
        \def\childdoctmp
        {
          \childdoctrue
          \includeonly{\childdocname}
          \def\childdocjob{#1}
          \def\jobname{#1}
        }
      \fi
      \expandafter
    \endgroup
    \childdoctmp
  \fi
}
%    \end{macrocode}

% \macro{\childdocof}
% The command |\childdocof| redirects
% compilation to the main file |#1|.
%    \begin{macrocode}
\newcommand{\childdocof}[1]
{
  \childdocdisable
  \childdoctrue
  \includeonly{\childdocname}
  \def\jobname{#1}
  \def\childdocjob{#1}
  \input{#1}
}
%    \end{macrocode}

% \macro{\childdocby}
% The command |\childdocby| ....
%    \begin{macrocode}
\newcommand{\childdocby}[2][]
{
  \childdocdisable
  \childdoctrue
  \childdocmanualtrue
  \if?#1?\else
    \def\jobname{#2}
  \fi
  \def\childdocjob{#2}
  \input{#2}
  \endinput
}
%    \end{macrocode}

% \macro{\childdocforward}
% The command |\childdocforward| redirects
% compilation to the main file or
% (if the optional argument is given) a child file.
% Parameters are set as if the main file
% or a child file starting with |\childdocof| was compiled.
% Then compilation is handed over to the main file:
%    \begin{macrocode}
\newcommand{\childdocforward}[2][]
{
  \begingroup
    \if?#1?
      \def\childdoctmp
      {
        \def\childdocname{#2}
        \def\childdocjob{#2}
        \def\jobname{#2}
        \input{#2}
        \endinput
      }
    \else
      \def\childdoctmp
      {
        \childdocdisable
        \def\childdocname{#2}
        \childdoctrue
        \includeonly{#2}
        \def\childdocjob{#1}
        \def\jobname{#1}
        \input{#1}
        \endinput
      }
    \fi
    \expandafter
  \endgroup
  \childdoctmp
}
%    \end{macrocode}

% \macro{\childdocforwardprefix}
% The command |\childdocforwardprefix| redirects
% compilation to the main or a child file by means of a pattern.
% The prefix |#1| in the current filename is replaced by |#2|
% and the suffix of the current filename is kept
% (it is assumed that the filename does not contain the substring `|~~~|'
% which is used as a delimiter).
% Compilation is handed over to the new file by |\childdocforward|:
%    \begin{macrocode}
\newcommand{\childdocforwardprefix}[3][]
{
  \begingroup
    \def\childdocextract #2##1~~~{\def\childdoctmp{\childdocforward[#1]{#3##1}}}
    \expandafter\childdocextract\childdocname~~~
    \expandafter
  \endgroup
  \childdoctmp
}
%    \end{macrocode}

% \macro{\childdoc}
% The deprecated macro |\childdoc| is a legacy version of |\childdocmain|:
%    \begin{macrocode}
\newcommand{\childdoc}{\childdocmain}
%    \end{macrocode}

% \macro{\childdocredirect}
% The deprecated macro |\childdocredirect| is a legacy version
% of |\childdocforward| and |\childdocforwardprefix|:
%    \begin{macrocode}
\newcommand{\childdocredirect}[2][]
{
  \begingroup
    \if?#1?
      \def\childdoctmp{\childdocforward{#2}}
    \else
      \def\childdoctmp{\childdocforwardprefix{#1}{#2}}
    \fi
    \expandafter
  \endgroup
  \childdoctmp
}
%    \end{macrocode}

%\iffalse
%</package>
%\fi
%
\endinput

\childdocforwardprefix[cdocsamp]{cdocsfn}{cdocsch}
%    \end{macrocode}

%\iffalse
%</samplefinal>
%\fi
%
% %%%%%%%%%%%%%%%%%%%%%%%%%%%%%%%%%%%%%%
% \paragraph{Command Line Processing.}
%
% The following three command lines generate the output files
% |cdocscld|, |cdocscl1| and |cdocscl2|
% which should be identical to
% |cdocsdrf|, |cdocsch1| and |cdocsfn2|, respectively:
% \begin{center}
% \begin{tabular}{l}
% |latex -jobname cdocscld \|\\
% |  "\def\version{draft}% \iffalse
%
% childdoc.dtx Copyright (C) 2017-2018 Niklas Beisert
%
% This work may be distributed and/or modified under the
% conditions of the LaTeX Project Public License, either version 1.3
% of this license or (at your option) any later version.
% The latest version of this license is in
%   http://www.latex-project.org/lppl.txt
% and version 1.3 or later is part of all distributions of LaTeX
% version 2005/12/01 or later.
%
% This work has the LPPL maintenance status `maintained'.
%
% The Current Maintainer of this work is Niklas Beisert.
%
% This work consists of the files childdoc.dtx and childdoc.ins
% and the derived files childdoc.def and cdocsamp.tex with
% cdocsch1.tex, cdocsch2.tex, cdocsdrf.tex, cdocsfn1.tex, cdocsfn2.tex.
%
%<package>\ifdefined\childdocmain\endinput\fi
%<package>\ProvidesFile{childdoc.def}[2018/12/30 v2.0 child document driver]
%<samplemain>\ProvidesFile{cdocsamp.tex}[2018/12/30 v2.0 sample for childdoc]
%<*driver>
%\ProvidesFile{childdoc.drv}[2018/12/30 v2.0 childdoc reference manual file]
\PassOptionsToClass{10pt,a4paper}{article}
\documentclass{ltxdoc}

\usepackage[margin=35mm]{geometry}
\usepackage{hyperref}
\usepackage{hyperxmp}
\usepackage[usenames]{color}

\hypersetup{colorlinks=true}
\hypersetup{pdfstartview=FitH}
\hypersetup{pdfpagemode=UseNone}
\hypersetup{pdfsource={}}
\hypersetup{pdflang={en-UK}}
\hypersetup{pdfcopyright={Copyright 2017-2018 Niklas Beisert.
  This work may be distributed and/or modified under the
  conditions of the LaTeX Project Public License, either version 1.3
  of this license or (at your option) any later version.}}
\hypersetup{pdflicenseurl={http://www.latex-project.org/lppl.txt}}
\hypersetup{pdfcontactaddress={ETH Zurich, ITP, HIT K,
  Wolfgang-Pauli-Strasse 27}}
\hypersetup{pdfcontactpostcode={8093}}
\hypersetup{pdfcontactcity={Zurich}}
\hypersetup{pdfcontactcountry={Switzerland}}
\hypersetup{pdfcontactemail={nbeisert@itp.phys.ethz.ch}}
\hypersetup{pdfcontacturl={http://people.phys.ethz.ch/\xmptilde nbeisert/}}

\newcommand{\secref}[1]{\hyperref[#1]{section \ref*{#1}}}

\parskip1ex
\parindent0pt
\let\olditemize\itemize
\def\itemize{\olditemize\parskip0pt}

\begin{document}

\title{The \textsf{childdoc} Package}
\hypersetup{pdftitle={The childdoc Package}}
\author{Niklas Beisert\\[2ex]
  Institut f\"ur Theoretische Physik\\
  Eidgen\"ossische Technische Hochschule Z\"urich\\
  Wolfgang-Pauli-Strasse 27, 8093 Z\"urich, Switzerland\\[1ex]
  \href{mailto:nbeisert@itp.phys.ethz.ch}
  {\texttt{nbeisert@itp.phys.ethz.ch}}}
\hypersetup{pdfauthor={Niklas Beisert}}
\hypersetup{pdfsubject={Manual for the LaTeX2e Package childdoc}}
\date{30 December 2018, \textsf{v2.0}}
\maketitle

\begin{abstract}\noindent
\textsf{childdoc} is a \LaTeXe{} package
that enables the direct compilation
of document sections included by |\include|
to individual files.
\end{abstract}

\begingroup
\parskip0ex
\tableofcontents
\endgroup

%%%%%%%%%%%%%%%%%%%%%%%%%%%%%%%%%%%%%%%%%%%%%%%%%%%%%%%%%%%%%%%%%%%%%%%%%%%%%%%%
%%%%%%%%%%%%%%%%%%%%%%%%%%%%%%%%%%%%%%%%%%%%%%%%%%%%%%%%%%%%%%%%%%%%%%%%%%%%%%%%
\section{Introduction}

\LaTeX{} provides a mechanism to structure a large document (such as a book)
into a main file and several child files (containing the chapters)
using the |\include| command.
This mechanism is beneficial for documents
which span hundreds of pages in order to
make the source file(s) more manageable.
Moreover, compilation can be restricted to
selected child files by means of the |\includeonly| command.
The latter feature can be used to reduce the compilation time while editing
(this was significantly more useful in the earlier days of \LaTeX{})
or to generate a smaller document which is easier to navigate.
Another application of |\includeonly| is to generate
documents consisting of selected parts of the complete document.

However, there are a few drawbacks of the plain |\include| mechanism:
\begin{itemize}
\item
The child files cannot be compiled on their own,
they can only be compiled via the main file.
A naive editing environment
(such as a text editor with an option
to have the current file processed by \LaTeX)
may require one to switch to the main file before compiling;
attempting to compile the child file produces errors.
\item
The main file must be modified (each time)
to adjust the |\includeonly| command
to the present needs. This easily leaves the main file in a messy state.
\item
The generated document will always carry the filename
of the main document. This is inconvenient if
several child files are to be compiled and
to be kept for distribution.
\end{itemize}

The present package provides a simple interface
to make child files individually compilable by \LaTeX{}.
Compiling a child file then has the same effect as compiling
the main file with an |\includeonly| command
to select the appropriate child.
Moreover the generated document will carry the name of the child
rather than the main file.
This resolves all three above issues.

This feature is meant to make the editing of books,
thesis documents and lecture notes somewhat more convenient.
However, the package can also be used efficiently for
composing a series of documents (such as exercise sheets)
which are typically distributed individually.
It then assists the author in generating the individual documents
(potentially in different versions)
as well as a document containing the collected series.
Another application is in developing style files
or other kinds of included material
where compilation of the style file could redirect
to a sample or test file.

%%%%%%%%%%%%%%%%%%%%%%%%%%%%%%%%%%%%%%%%%%%%%%%%%%%%%%%%%%%%%%%%%%%%%%%%%%%%%%%%
%%%%%%%%%%%%%%%%%%%%%%%%%%%%%%%%%%%%%%%%%%%%%%%%%%%%%%%%%%%%%%%%%%%%%%%%%%%%%%%%
\section{Usage}

First of all, the package \textsf{childdoc} is \emph{not} a standard
\LaTeXe{} |.sty| style file! Therefore it needs to be invoked in
a non-standard way.

%%%%%%%%%%%%%%%%%%%%%%%%%%%%%%%%%%%%%%%%%%%%%%%%%%%%%%%%%%%%%%%%%%%%%%%%%%%%%%%%
\subsection{Included Files}
\label{sec:include}

%%%%%%%%%%%%%%%%%%%%%%%%%%%%%%%%%%%%%%%%
\DescribeMacro{\childdocmain}
To use the package, add the commands
\begin{center}
\begin{tabular}{l}
|\input{childdoc.def}|\\
|\childdocmain{}|\\
\end{tabular}
\end{center}
at the very top of the main \LaTeX{} file,
in particular \emph{before} the |\documentclass| statement!
The argument of |\childdocmain| should be left empty
(but it must be present).

%%%%%%%%%%%%%%%%%%%%%%%%%%%%%%%%%%%%%%%%
\DescribeMacro{\childdocof}
Furthermore, add the commands
\begin{center}
\begin{tabular}{l}
|\input{childdoc.def}|\\
|\childdocof{|\textit{main}|}|\\
\end{tabular}
\end{center}
at the top of every child file \textit{child}
which is included by |\include{|\textit{child}|}|
from within the main file
(or at least for those files to be compiled individually).
The argument \textit{main} must be the filename of the main file.

There are a couple of
considerations in setting up the main and child documents:

%%%%%%%%%%%%%%%%%%%%%%%%%%%%%%%%%%%%%%%%
\paragraph{Restrictions.}

Please note the following restrictions:
\begin{itemize}
\item
|\childdocmain| must be called with one argument \textit{main}
to ensure compatibility with earlier version of the package.
It must either be empty (|\childdocmain{}|)
or precisely match the filename of the main file in which it is specified.
See \secref{sec:detection} for further information.
\item
The filename \textit{main} must be specified without the |.tex| extension.
\item
The filename \textit{main} is case sensitive
(even in case-insensitive file systems)
due to internal string comparison.
\item
The argument \textit{main} should be fully expanded, it cannot be a macro.
\item
Subdirectories and special characters should be avoided in filenames.
\item
The command |\childdocmain{|\textit{main}|}| must be followed by a whitespace.
It should not be followed immediately by another command
or by a comment mark `|%|'.
This is because the \TeX{} parser reads the token immediately following
the argument of |\childdocmain| and puts it
at the beginning of every child section;
however, a white\-space is ignored.
\end{itemize}

%%%%%%%%%%%%%%%%%%%%%%%%%%%%%%%%%%%%%%%%
\paragraph{Content of Main File.}

It is advisable to place all content in the child files included by |\include|.
Any output contained in the main file will appear in all child documents
unless suppressed manually;
it cannot be suppressed automatically by the |\includeonly| directive
and thus should normally be avoided.
A method to include some content in the main file
by means of conditional processing is described in \secref{sec:conditional}.

%%%%%%%%%%%%%%%%%%%%%%%%%%%%%%%%%%%%%%%%
\paragraph{Page Numbering.}

When only a part of the document is compiled,
the appropriate numbering of pages
(as well as other status parameters)
is determined from the |.aux| files.
The latter contain information from previous passes.
However this information needs to propagate through
all intermediate child documents.
Therefore the page numbering in child documents may well
be inconsistent until the complete document is compiled at least once.

A useful (if unconventional) way to always ensure a consistent
page numbering is to restart the numbering in each child document
and denote the pages by `\textit{child}|.|\textit{page}'
where \textit{child} represents the chapter/section number of the child file.
This can be achieved by the command
|\numberwithin{page}{|\textit{child}|}|
of the \textsf{amsmath} package
where \textit{child} can be |chapter| or |section|
depending on the chosen structuring.
Alternatively, one can modify the macro |\thepage| appropriately
and reset the counter |page| at the start of each child file.

%%%%%%%%%%%%%%%%%%%%%%%%%%%%%%%%%%%%%%%%%%%%%%%%%%%%%%%%%%%%%%%%%%%%%%%%%%%%%%%%
\subsection{Conditional Processing}
\label{sec:conditional}

The package provides a mechanism to compile different versions
of a document. To customise the versions further some conditional processing
can come in handy to distinguish which version is being compiled.
The package provides two macros to describe the compilation context:

%%%%%%%%%%%%%%%%%%%%%%%%%%%%%%%%%%%%%%%%
\DescribeMacro{\ifchilddoc}
The conditional |\ifchilddoc| distinguishes between the compilation of
child documents and the main document:
%
\begin{center}
|\ifchilddoc |\textit{child-code}| |[|\||else |\textit{main-code}]| \||fi|
\end{center}

%%%%%%%%%%%%%%%%%%%%%%%%%%%%%%%%%%%%%%%%
\DescribeMacro{\childdocname}
\DescribeMacro{\childdocjob}
The macro |\childdocname| contains the filename (without extension)
of the main or child file being processed.
Note that |\childdocjob| will always contain the name of the main file.

%%%%%%%%%%%%%%%%%%%%%%%%%%%%%%%%%%%%%%%%
\paragraph{Title Page.}

Conditional processing can be used to include a title or banner page
in the main document when proper precautions are taken.
Importantly, the code in the main file should ensure that the page counter
(as well as other status parameters which are stored in the |.aux| files)
takes the same value after the conditional processing.
Otherwise the page numbers may take divergent values
depending on which part is compiled.

For example, a title page could be declared by:
%
\begin{center}
\begin{tabular}{l}
|\ifchilddoc\||else|\\
|\addtocounter{page}{-1}|\\
\textit{code for title page}\\
|\newpage|\\
|\||fi|
\end{tabular}
\end{center}
%
A banner page for the child documents can be generated by:
%
\begin{center}
\begin{tabular}{l}
|\ifchilddoc|\\
|\addtocounter{page}{-1}|\\
\textit{code for banner page}\\
|\newpage|\\
|\||fi|
\end{tabular}
\end{center}
%
Here one could write a message such as:
\begin{center}
|This is the part \childdocname{} of \childdocjob{}.|
\end{center}

%%%%%%%%%%%%%%%%%%%%%%%%%%%%%%%%%%%%%%%%%%%%%%%%%%%%%%%%%%%%%%%%%%%%%%%%%%%%%%%%
\subsection{Flags}
\label{sec:flags}

The package makes it easy to generate different versions
of the main or child documents.
To this end compilation flags can be defined
and assigned different default values.
They will be particularly useful in conjunction
with the forwarding mechanism described in \secref{sec:forward}.

For example, it may be useful to have a flag |\version|
which can be set to |draft| or |final|.
The document source will contain some conditional code
depending on the value of |\version|.
Suppose further, the flag should default to |final| for the main file
and to |draft| for child files
which is a natural assignment for editing the document.
This is achieved by placing the following code
in the preamble of the main document
(below the |\childdocmain| directive):
%
\begin{center}
\begin{tabular}{l}
|\ifchilddoc|\\
|\providecommand{\version}{draft}|\\
|\||else|\\
|\providecommand{\version}{final}|\\
|\||fi|
\end{tabular}
\end{center}
%
The definition by |\providecommand| makes sure
that previous definitions are not overwritten.
Further statements |\providecommand{\version}{...}|
can thus be added before the above code to override it.

For the main file, one might add a line
(between |\childdocmain| and the above block)
%
\begin{center}
|%\ifchilddoc\||else\providecommand{\version}{draft}\||fi|
\end{center}
%
which can be uncommented to produce a draft version.
Likewise one can add a line to the very top of a child file
(above the |\childdocof{|\textit{main}|}| directive)
%
\begin{center}
|%\providecommand{\version}{final}|
\end{center}
%
which can be uncommented to produce the final version of this child document.

%%%%%%%%%%%%%%%%%%%%%%%%%%%%%%%%%%%%%%%%%%%%%%%%%%%%%%%%%%%%%%%%%%%%%%%%%%%%%%%%
\subsection{Forwarding}
\label{sec:forward}

Different versions of the main or child documents
using compilation flags as described in \secref{sec:flags}
can be (permanently) stored in different files
for convenient compilation, viewing and distribution.
To this end, the package defines a command
to pass on compilation to a different file:

%%%%%%%%%%%%%%%%%%%%%%%%%%%%%%%%%%%%%%%%
\DescribeMacro{\childdocforward}
The command |\childdocforward| redirects processing to
another source file:
%
\begin{center}
\begin{tabular}{l}
|\input{childdoc.def}|\\
|\childdocforward[|\textit{main}|]{|\textit{dest}|}|\\
\end{tabular}
\end{center}
%
The argument \textit{dest} is the destination file
(without extension).
It should be the main file or one of the child files.
Note that further \textsf{childdoc} directives
such as |\childdocof| and |\childdocforward|
in the indicated file will be processed in this form.
The optional argument \textit{main}
passes on directly to the main file \textit{main}
while pretending to compile the child \textit{dest}.
This form behaves as if \textit{dest}
issues |\childdocof{|\textit{main}|}| right away,
and no further \textsf{childdoc} directives will be processed.

%%%%%%%%%%%%%%%%%%%%%%%%%%%%%%%%%%%%%%%%
\DescribeMacro{\...prefix}
In the alternative form |\childdocforwardprefix|,
%
\begin{center}
\begin{tabular}{l}
|\input{childdoc.def}|\\
|\childdocforwardprefix[|\textit{main}|]{|\textit{prefix}|}{|\textit{dest}|}|
\end{tabular}
\end{center}
%
the destination file is determined by a pattern
depending on the current file:
To make this work, the current file must be called
`{\textit{prefix}\hspace{0.2em}\textit{suffix}}'
with \textit{prefix} matching precisely the argument.
Processing is then passed on to the file
`{\textit{dest}\hspace{0.2em}\textit{suffix}}'.
Surely, the same effect is achieved by
directly specifying the
argument `{\textit{dest}\hspace{0.2em}\textit{suffix}}'
in the first form.
However, that requires to set up a different file
for each child. With the alternative form of the command
all these files can have exactly the same content
which simplifies setting them up and maintaining them.

For example, the following file |draft.tex|
with a compilation flag |\version| as described in \secref{sec:flags}
compiles the main document as a draft:
%
\begin{center}
\begin{tabular}{l}
|\def\version{draft}|\\
|\input{childdoc.def}|\\
|\childdocforward{|\textit{main}|}|
\end{tabular}
\end{center}
%
Likewise, the following files |final|\textit{nn}|.tex|
compile the final version of the child document
|child|\textit{nn}|.tex|:
%
\begin{center}
\begin{tabular}{l}
|\def\version{final}|\\
|\input{childdoc.def}|\\
|\childdocforwardprefix{final}{child}|
\end{tabular}
\end{center}
%

Note that when several versions of a main file and/or of each child file
are to be generated, it may be convenient to set up a |Makefile| or
shell script to automatise the process.

%%%%%%%%%%%%%%%%%%%%%%%%%%%%%%%%%%%%%%%%%%%%%%%%%%%%%%%%%%%%%%%%%%%%%%%%%%%%%%%%
\subsection{Command Line Processing}
\label{sec:commandline}

The effect of redirection files can also be achieved by invoking
the \LaTeX{} compiler with a more elaborate command line.
Most conveniently this should be done as part
of a shell script or a |Makefile|.

When using \textsf{childdoc} in the main file, the following
command lines effectively perform a redirection
(note that depending on the shell being used,
backslashes may have to be doubled: `|\|' $\to$ `|\\|'):
%
\begin{center}
|... -jobname "|\textit{target}|" |\\|"|[\textit{flags}]%
|\input{childdoc.def}\childdocforward[|\textit{main}|]{|\textit{dest}|}"|
\end{center}
%
Here \textit{target} is the name of the output file,
\textit{main} is the name of the main file
and \textit{dest} is the name of the main or child file to be processed
(all filenames without extensions).
The optional argument \textit{main} can be omitted
if \textit{main} matches \textit{dest}.
Optionally, compilation \textit{flags} can be defined via |\def| commands.
This command line makes the \TeX{} engine believe
it is compiling the file \textit{target}
whose content is specified as the latter parameter.
The provided code then forwards the processing to
\textit{main} or \textit{dest} as described in \secref{sec:forward}.

%%%%%%%%%%%%%%%%%%%%%%%%%%%%%%%%%%%%%%%%%%%%%%%%%%%%%%%%%%%%%%%%%%%%%%%%%%%%%%%%
\subsection{Include by Input}
\label{sec:input}

Including child documents by |\include| has some restrictions by design.
Most notably, the content of a child document always occupies
its own set of pages; pages cannot be shared between child documents.
Usually, this behaviour makes perfect sense
because each child document contain an essential part of the document.
However, in some situations it may be desirable to compose
a document from a collection of parts
without having mandatory page breaks between then.
For this case, the package
provides a mechanism to include parts
by |\input| which can also be processed individually.
However, by construction this mechanism
requires manual handling of the content to be output.

%%%%%%%%%%%%%%%%%%%%%%%%%%%%%%%%%%%%%%%%
\DescribeMacro{\ifchilddocmanual}
The main file should be prepared as usual, see \secref{sec:include}.
However, the document body must make a distinction
between processing of an individual part and of the main document, e.g.:
%
\begin{center}
\begin{tabular}{l}
|\ifchilddocmanual|\\
|\input{\childdocname}|\\
|\||else|\\
\textit{document body with }|\input{|\textit{part}|}|\\
|\||fi|
\end{tabular}
\end{center}
%
The conditional |\ifchilddocmanual| is true whenever
a part to be included by |\input| is being compiled,
and the name of the part is stored in |\childdocname|.

%%%%%%%%%%%%%%%%%%%%%%%%%%%%%%%%%%%%%%%%
\DescribeMacro{\childdocby}
Each part to be included by |\input| should start with:
%
\begin{center}
\begin{tabular}{l}
|\input{childdoc.def}|\\
|\childdocby{|\textit{main}|}|\\
\end{tabular}
\end{center}
%
The directive |\childdocby| is similar to |\childdocof|
described in \secref{sec:include},
but the subsequent selection of content must be done manually.
To that end, both |\ifchilddoc| and |\ifchilddocmanual|
will be true upon processing of a part,
and the name of the part is stored in |\childdocname|.
Note that |\jobname| will be set to the filename of the current part
so that each part receives an individual |.aux| file
that does not interfere with the |.aux| file(s) of the main document.
This behaviour can be altered by the alternative form
|\childdocby[*]{|\textit{main}|}| (with a non-empty optional argument)
which uses the |.aux| file of the main document
by setting |\jobname| to \textit{main}.

%%%%%%%%%%%%%%%%%%%%%%%%%%%%%%%%%%%%%%%%%%%%%%%%%%%%%%%%%%%%%%%%%%%%%%%%%%%%%%%%
\subsection{Driver Development}
\label{sec:driver}

The \textsf{childdoc} mechanism can also be use for the development
of definition files such as \LaTeX{} styles or classes.
This case differs from the above setup with multiple parts
included by |\include| in that no |\includeonly| should be invoked.
This can be achieved by starting the include file
(before |\ProvidesPackage|) with:
%
\begin{center}
\begin{tabular}{l}
|\input{childdoc.def}|\\
|\childdocforward{|\textit{main}|}|\\
\end{tabular}
\end{center}
%
or alternatively with:
%
\begin{center}
\begin{tabular}{l}
|\input{childdoc.def}|\\
|\childdocby{|\textit{main}|}|\\
\end{tabular}
\end{center}
%
Both forms have slightly different effects as described above.
The main file is prepared as usual, see \secref{sec:include}.

%%%%%%%%%%%%%%%%%%%%%%%%%%%%%%%%%%%%%%%%%%%%%%%%%%%%%%%%%%%%%%%%%%%%%%%%%%%%%%%%
\subsection{Legacy Detection}
\label{sec:detection}

The directive |\childdocmain| in the main file can detect
whether the complete document or merely a child is to be compiled
even without using the directive |\childdocof|.
This method is deprecated because it is less robust
and there is no compelling reason to use it;
it is merely provided for backward compatibility
and it may be removed in future versions.

If the detection mechanism is to be used,
it is mandatory to correctly specify
the filename of the main file as the argument of |\childdocmain|:
%
\begin{center}
\begin{tabular}{l}
|\input{childdoc.def}|\\
|\childdocmain{|\textit{main}|}|\\
\end{tabular}
\end{center}
%
If |\jobname| does not match the argument \textit{main} of |\childdocmain|,
it is assumed that |\jobname| points to the child file to be compiled.
When using |\childdocmain| with the main file specified as argument,
it suffices to start a child file
with just |\input{|\textit{main}|}|
without loading of the package and using |\childdocof|.
If instead all processing is done
with the appropriate \textsf{childdoc} directives,
the argument of \textit{main} of |\childdocmain| can be empty.

An alternative version of the command line processing described
in \secref{sec:commandline} using the detection mechanism reads:
%
\begin{center}
|... -jobname "|\textit{target}|" "|[\textit{flags}]%
[|\def\jobname{|\textit{dest}|}|]|\input{|\textit{main}|}"|
\end{center}

%%%%%%%%%%%%%%%%%%%%%%%%%%%%%%%%%%%%%%%%%%%%%%%%%%%%%%%%%%%%%%%%%%%%%%%%%%%%%%%%
\subsection{Manual Code}
\label{sec:manual}

In case one cannot be certain whether the definitions file |childdoc.def|
is installed on the target \TeX{} distribution
and one prefers not to ship it,
it is conceivable to paste a few relevant commands into the sources.

To that end, drop all statements |\input{childdoc.def}|
and perform the replacements as outlined below.
Instead of |\childdocmain{|\textit{main}|}| add the following code
to the top of the main file:
%
\begin{center}
\begin{tabular}{l}
|\||ifdefined\childdocname\endinput\||fi\newif\ifchilddoc|\\
|\edef\childdocname{\scantokens\expandafter{\jobname\noexpand}}|\\
|\def\childdocmain{|\textit{main}|}\||ifx\childdocmain\childdocname\||else|\\
|\childdoctrue\includeonly{\childdocname}\let\jobname\childdocmain\||fi|\\
\end{tabular}
\end{center}
%
Instead of |\childdocof{|\textit{main}|}| just include the main file
at the top of each child file:
%
\begin{center}
|\input{|\textit{main}|}|
\end{center}
%
A simple redirection |\childdocforward{|\textit{dest}|}| is achieved by:
%
\begin{center}
|\def\jobname{|\textit{dest}|}\input{\jobname}|
\end{center}
%
The redirection with prefix
|\childdocforwardprefix[|\textit{prefix}|]{|\textit{dest}|}|
is accomplished by:
%
\begin{center}
\begin{tabular}{l}
|{\edef\jobname{\scantokens\expandafter{\jobname\noexpand}}|\\
|\def\redirectjob |\textit{prefix}|#1~~~{\gdef\jobname{|\textit{dest}|#1}}|\\
|\expandafter\redirectjob\jobname~~~}\input{\jobname}|
\end{tabular}
\end{center}

In an alternative approach,
child documents can be compiled by a specific command line
without additional code or specific definitions:
%
\begin{center}
|... -jobname "|\textit{target}|" "|[\textit{flags}]%
|\includeonly{|\textit{dest}|}\input{|\textit{main}|}"|
\end{center}
%

%%%%%%%%%%%%%%%%%%%%%%%%%%%%%%%%%%%%%%%%%%%%%%%%%%%%%%%%%%%%%%%%%%%%%%%%%%%%%%%%
%%%%%%%%%%%%%%%%%%%%%%%%%%%%%%%%%%%%%%%%%%%%%%%%%%%%%%%%%%%%%%%%%%%%%%%%%%%%%%%%
\section{Information}

%%%%%%%%%%%%%%%%%%%%%%%%%%%%%%%%%%%%%%%%%%%%%%%%%%%%%%%%%%%%%%%%%%%%%%%%%%%%%%%%
\subsection{Copyright}

Copyright \copyright{} 2017--2018 Niklas Beisert

This work may be distributed and/or modified under the
conditions of the \LaTeX{} Project Public License, either version 1.3
of this license or (at your option) any later version.
The latest version of this license is in
  \url{http://www.latex-project.org/lppl.txt}
and version 1.3 or later is part of all distributions of \LaTeX{}
version 2005/12/01 or later.

This work has the LPPL maintenance status `maintained'.

The Current Maintainer of this work is Niklas Beisert.

This work consists of the files |README.txt|, |childdoc.ins| and |childdoc.dtx|
as well as the derived files |childdoc.def|, |cdocsamp.tex|
with |cdocsch1.tex|, |cdocsch2.tex|, |cdocspt3.tex|, |cdocspt4.tex|,
|cdocsdrf.tex|, |cdocsfn1.tex|, |cdocsfn2.tex|
as well as |childdoc.pdf|.

%%%%%%%%%%%%%%%%%%%%%%%%%%%%%%%%%%%%%%%%%%%%%%%%%%%%%%%%%%%%%%%%%%%%%%%%%%%%%%%%
\subsection{Files and Installation}

The package consists of the files:
%
\begin{center}
\begin{tabular}{ll}
    |README.txt|   & readme file \\
    |childdoc.ins| & installation file \\
    |childdoc.dtx| & source file \\
    |childdoc.def| & definition file \\
    |cdocsamp.tex| & sample main file \\
    |cdocsch1.tex| & sample include file \\
    |cdocsch2.tex| & sample include file \\
    |cdocspt3.tex| & sample part file \\
    |cdocspt4.tex| & sample part file \\
    |cdocsdrf.tex| & sample redirection file \\
    |cdocsfn1.tex| & sample redirection file \\
    |cdocsfn2.tex| & sample redirection file \\
    |childdoc.pdf| & manual
\end{tabular}
\end{center}
%
The distribution consists of the files
|README.txt|, |childdoc.ins| and |childdoc.dtx|.
%
\begin{itemize}
\item
Run (pdf)\LaTeX{} on |childdoc.dtx|
to compile the manual |childdoc.pdf| (this file).
\item
Run \LaTeX{} on |childdoc.ins| to create the definitions file |childdoc.def|
and the sample |cdocsamp.tex| with include files
|cdocsch1.tex|, |cdocsch2.tex|, |cdocspt3.tex|, |cdocspt4.tex|,
|cdocsdrf.tex|, |cdocsfn1.tex|, |cdocsfn2.tex|.
Then copy the file |childdoc.def| to an appropriate directory of your \LaTeX{}
distribution, e.g.\ \textit{texmf-root}|/tex/latex/childdoc|.
\end{itemize}

%%%%%%%%%%%%%%%%%%%%%%%%%%%%%%%%%%%%%%%%%%%%%%%%%%%%%%%%%%%%%%%%%%%%%%%%%%%%%%%%
\subsection{Related CTAN Packages}

There are several other packages which offer a similar functionality:
%
\begin{itemize}
\item
The packages
\href{http://ctan.org/pkg/docmute}{\textsf{docmute}},
\href{http://ctan.org/pkg/includex}{\textsf{includex}} and
\href{http://ctan.org/pkg/standalone}{\textsf{standalone}}
provide commands to include only the document body of
a child file thus allowing both files to be compiled individually.
\item
The packages \href{http://ctan.org/pkg/subdocs}{\textsf{subdocs}}
and \href{http://ctan.org/pkg/subfiles}{\textsf{subfiles}}
provide structures in which the main and child documents can be
encapsulated and allowing them to be compiled individually.
The inclusion mechanism is different from the conventional |\include|.
\item
The package \href{http://ctan.org/pkg/combine}{\textsf{combine}}
is an elaborate solution to combine several documents into one.
\end{itemize}
%
See also the CTAN topic \href{http://ctan.org/topic/subdocs}{\textsf{subdocs}}
for further related packages.
The present package differs from the above solutions in that
a document structure constructed with the conventional |\include| mechanism
just needs two extra commands at the top of every file
such that all constituent files can be compiled individually.

%%%%%%%%%%%%%%%%%%%%%%%%%%%%%%%%%%%%%%%%%%%%%%%%%%%%%%%%%%%%%%%%%%%%%%%%%%%%%%%%
%\subsection{Feature Suggestions}
%
%The following is a list of features which may be useful for future
%versions of this package:
%%
%\begin{itemize}
%\item
%\ldots
%\end{itemize}

%%%%%%%%%%%%%%%%%%%%%%%%%%%%%%%%%%%%%%%%%%%%%%%%%%%%%%%%%%%%%%%%%%%%%%%%%%%%%%%%
\subsection{Revision History}

%%%%%%%%%%%%%%%%%%%%%%%%%%%%%%%%%%%%%%%%
\paragraph{v2.0:} 2018/12/30

\begin{itemize}
\item
immediate forward processing
\item
added |\childdocby| mechanism
\item
manual restructured
\end{itemize}

%%%%%%%%%%%%%%%%%%%%%%%%%%%%%%%%%%%%%%%%
\paragraph{v1.6:} 2018/01/17

\begin{itemize}
\item
application for development of include files
\item
corrections to manual
\end{itemize}

%%%%%%%%%%%%%%%%%%%%%%%%%%%%%%%%%%%%%%%%
\paragraph{v1.5:} 2017/05/21

\begin{itemize}
\item
more complete structuring introduced
\item
|\childdocof| introduced
\item
|\childdoc| renamed to |\childdocmain|
\item
|\childredirect| renamed to |\childdocforward| and |\childdocforwardprefix|
and functionality expanded
\end{itemize}

%%%%%%%%%%%%%%%%%%%%%%%%%%%%%%%%%%%%%%%%
\paragraph{v1.0:} 2017/04/27

\begin{itemize}
\item
manual and install package
\item
first version published on CTAN
\end{itemize}

%%%%%%%%%%%%%%%%%%%%%%%%%%%%%%%%%%%%%%%%
\paragraph{v0.6:} 2017/04/26

\begin{itemize}
\item
redirection mechanism added
\end{itemize}

%%%%%%%%%%%%%%%%%%%%%%%%%%%%%%%%%%%%%%%%
\paragraph{v0.5:} 2017/04/26

\begin{itemize}
\item
functionality in definition file
\end{itemize}


%%%%%%%%%%%%%%%%%%%%%%%%%%%%%%%%%%%%%%%%%%%%%%%%%%%%%%%%%%%%%%%%%%%%%%%%%%%%%%%%
%%%%%%%%%%%%%%%%%%%%%%%%%%%%%%%%%%%%%%%%%%%%%%%%%%%%%%%%%%%%%%%%%%%%%%%%%%%%%%%%
%%%%%%%%%%%%%%%%%%%%%%%%%%%%%%%%%%%%%%%%%%%%%%%%%%%%%%%%%%%%%%%%%%%%%%%%%%%%%%%%
\appendix

\settowidth\MacroIndent{\rmfamily\scriptsize 000\ }

 \DocInput{childdoc.dtx}

\end{document}
%</driver>
% \fi
%
% %%%%%%%%%%%%%%%%%%%%%%%%%%%%%%%%%%%%%%%%%%%%%%%%%%%%%%%%%%%%%%%%%%%%%%%%%%%%%%
% %%%%%%%%%%%%%%%%%%%%%%%%%%%%%%%%%%%%%%%%%%%%%%%%%%%%%%%%%%%%%%%%%%%%%%%%%%%%%%
% \section{Sample}
%\iffalse
%<*samplemain>
%\fi
%
% The following presents a sample document
% with two chapters, two parts, a title page,
% a compile flag as well as three forwarding files to set the flag.
% It consists of eight |.tex| files:
% \begin{center}
% \begin{tabular}{ll}
% |cdocsamp.tex|&main file\\
% |cdocsch1.tex|&include file for chapter 1\\
% |cdocsch2.tex|&include file for chapter 2\\
% |cdocspt3.tex|&include file for part 3\\
% |cdocspt4.tex|&include file for part 4\\
% |cdocsdrf.tex|&forwarding file for main file in draft mode\\
% |cdocsfi1.tex|&forwarding file for final version of chapter 1\\
% |cdocsfi2.tex|&forwarding file for final version of chapter 2\\
% \end{tabular}
% \end{center}
% Each of the eight files can be compiled directly by the \LaTeX{} compiler.
%
% %%%%%%%%%%%%%%%%%%%%%%%%%%%%%%%%%%%%%%
% \paragraph{Main File.}
%
% The main file is called |cdocsamp.tex|.
%
% Load the \textsf{childdoc} definitions and
% declare the filename for the main document:
%    \begin{macrocode}
\input{childdoc.def}
\childdocmain{}
%    \end{macrocode}

% Optional override for |\version| flag:
%    \begin{macrocode}
%%\ifchilddoc\else\providecommand{\version}{draft}\fi
%    \end{macrocode}

% Define the default values for the |\version| flag
% (|final| for the main file and |draft| for childs):
%    \begin{macrocode}
\ifchilddoc
\providecommand{\version}{draft}
\else
\providecommand{\version}{final}
\fi
%    \end{macrocode}

% Load the standard document class:
%    \begin{macrocode}
\documentclass[12pt]{article}
%    \end{macrocode}

% Start the document body:
%    \begin{macrocode}
\begin{document}
%    \end{macrocode}

% Declare a title page.
% Print title, part of document being processed and version flag:
%    \begin{macrocode}
\addtocounter{page}{-1}
\begin{center}
{\LARGE\bfseries{}childdoc example\par}
\vspace{1cm}
\ifchilddoc
\ifchilddocmanual part\else chapter\fi:
`\childdocname' of `\childdocjob'\par
\else
main document: `\childdocjob'\par
\fi
version: \version\par
\end{center}
\newpage
%    \end{macrocode}

% Manually include selected file,
% otherwise process as usual:
%    \begin{macrocode}
\ifchilddocmanual
\section*{part `\childdocname'}
\input{\childdocname}
\else
%    \end{macrocode}

% Include the two chapters:
%    \begin{macrocode}
\include{cdocsch1}
\include{cdocsch2}
%    \end{macrocode}

% Include the two parts unless only chapters should be displayed:
%    \begin{macrocode}
\ifchilddoc\else
\section{part three}
\input{cdocspt3}
\section{part four}
\input{cdocspt4}
\fi
%    \end{macrocode}

% Process as usual until here:
%    \begin{macrocode}
\fi
%    \end{macrocode}

% End of document body:
%    \begin{macrocode}
\end{document}
%    \end{macrocode}
%\iffalse
%</samplemain>
%\fi
%
% %%%%%%%%%%%%%%%%%%%%%%%%%%%%%%%%%%%%%%
% \paragraph{Chapter Include Files.}
%
% The include files are called |cdocsch1.tex| and |cdocsch2.tex|.
%
%\iffalse
%<*samplechap1|samplechap2>
%\fi

% Optional override for |\version| flag:
%    \begin{macrocode}
%%\providecommand{\version}{final}
%    \end{macrocode}

% Include the main document:
%    \begin{macrocode}
\input{childdoc.def}
\childdocof{cdocsamp}
%    \end{macrocode}

%\iffalse
%</samplechap1|samplechap2>
%\fi
%
%\iffalse
%<*samplechap1>
%\fi
% Some text for chapter 1:
%    \begin{macrocode}
\section{one}
some text in chapter one
%    \end{macrocode}

%\iffalse
%</samplechap1>
%\fi
% Some text for chapter 2:
%\iffalse
%<*samplechap2>
%\fi
%    \begin{macrocode}
\section{two}
more text in chapter two
%    \end{macrocode}

%\iffalse
%</samplechap2>
%\fi
%
% %%%%%%%%%%%%%%%%%%%%%%%%%%%%%%%%%%%%%%
% \paragraph{Part Include Files.}
%
% The include files are called |cdocspt3.tex| and |cdocspt4.tex|.
%
%\iffalse
%<*samplepart3|samplepart4>
%\fi

% Optional override for |\version| flag:
%    \begin{macrocode}
%%\providecommand{\version}{final}
%    \end{macrocode}

% Include the main document:
%    \begin{macrocode}
\input{childdoc.def}
\childdocby{cdocsamp}
%    \end{macrocode}

%\iffalse
%</samplepart3|samplepart4>
%\fi
%
%\iffalse
%<*samplepart3>
%\fi
% Some text for part 3:
%    \begin{macrocode}
some text in part three
%    \end{macrocode}

%\iffalse
%</samplepart3>
%\fi
% Some text for part 4:
%\iffalse
%<*samplepart4>
%\fi
%    \begin{macrocode}
more text in part four
%    \end{macrocode}

%\iffalse
%</samplepart4>
%\fi
%
% %%%%%%%%%%%%%%%%%%%%%%%%%%%%%%%%%%%%%%
% \paragraph{Forwarding for a Complete Draft.}
%
% The following forwarding file |cdocsdrf.tex|
% compiles the main document in draft mode:
%\iffalse
%<*sampledraft>
%\fi
%    \begin{macrocode}
\def\version{draft}
\input{childdoc.def}
\childdocforward{cdocsamp}
%    \end{macrocode}

%\iffalse
%</sampledraft>
%\fi
%
% %%%%%%%%%%%%%%%%%%%%%%%%%%%%%%%%%%%%%%
% \paragraph{Forwarding for Final Version of the Chapters.}
%
% The following forwarding files |cdocsfn1.tex| and |cdocsfn2.tex|
% (with identical content)
% compile the final versions of the child documents
% |cdocsch1.tex| and |cdocsch2.tex|, respectively:
%\iffalse
%<*samplefinal>
%\fi
%    \begin{macrocode}
\def\version{final}
\input{childdoc.def}
\childdocforwardprefix[cdocsamp]{cdocsfn}{cdocsch}
%    \end{macrocode}

%\iffalse
%</samplefinal>
%\fi
%
% %%%%%%%%%%%%%%%%%%%%%%%%%%%%%%%%%%%%%%
% \paragraph{Command Line Processing.}
%
% The following three command lines generate the output files
% |cdocscld|, |cdocscl1| and |cdocscl2|
% which should be identical to
% |cdocsdrf|, |cdocsch1| and |cdocsfn2|, respectively:
% \begin{center}
% \begin{tabular}{l}
% |latex -jobname cdocscld \|\\
% |  "\def\version{draft}\input{childdoc.def}\childdocforward{cdocsamp}"|\\
% |latex -jobname cdocscl1 \|\\
% |  "\input{childdoc.def}\childdocforward[cdocsamp]{cdocsch1}"|\\
% |latex -jobname cdocscl2 \|\\
% |  "\def\version{final}\input{childdoc.def}\childdocforward{cdocsch2}"|
% \end{tabular}
% \end{center}
% Note that the trailing backslash on each first line
% merely continues the input to the second line
% (for convenient cut ant paste).
% Furthermore, the command |latex| can be replaced by any
% of its alternative versions such as |pdflatex|.
%
% %%%%%%%%%%%%%%%%%%%%%%%%%%%%%%%%%%%%%%%%%%%%%%%%%%%%%%%%%%%%%%%%%%%%%%%%%%%%%%
% %%%%%%%%%%%%%%%%%%%%%%%%%%%%%%%%%%%%%%%%%%%%%%%%%%%%%%%%%%%%%%%%%%%%%%%%%%%%%%
% \section{Implementation}
%\iffalse
%<*package>
%\fi
%
% This section describes the definitions file |childdoc.def|.

% The definitions cannot be loaded using |\usepackage| or |\RequirePackage|
% which has a mechanism to prevent loading a style file more than once.
% When loading the definitions by means of |\input|
% multiple instances have to be prevented manually:
%\iffalse
%This code needs to be before the `\ProvidesFile' directive
%which is defined at the beginning of this file.
%Therefore it is also placed there and commented out here.
%</package>
%<*discard>
%\fi
%    \begin{macrocode}
\ifdefined\childdocmain\endinput\fi
%    \end{macrocode}
%\iffalse
%</discard>
%<*package>
%\fi
%
% \macro{\ifchilddoc}
% \macro{\ifchilddocmanual}
% The conditional |\ifchilddoc| tells whether a
% child (true) or main (false) document is being compiled.
% The conditional |\ifchilddocmanual| tells whether
% the |\includeonly| mechanism is used (false) or
% the selection of child files must be performed manually (true).
% The definitions initialise to false:
%    \begin{macrocode}
\newif\ifchilddoc
\newif\ifchilddocmanual
%    \end{macrocode}

% \macro{\childdocname}
% \macro{\childdocjob}
% The macro |\childdocname| stores the name of the main document
% to be compiled. The macro |\childdocjob| stores the name of
% the document on which the \LaTeX{} compiler was originally invoked.
% The content of |\jobname| cannot be compared
% to filenames specified in the source due to different catcodes.
% The following code rescans |\jobname|, stores the result
% in |\childdocname| and saves a copy in |\childdocjob|:
%    \begin{macrocode}
\edef\childdocname{\scantokens\expandafter{\jobname\noexpand}}
\let\childdocjob\childdocname
%    \end{macrocode}

% \macro{\childdocdisable}
% The macro |\childdocdisable| prevents the main file
% from being processed more than once.
% At this stage, the main document command |\childdocmain|
% is assumed to be called once again where it should do nothing.
% Any subsequent call to it should prevent
% a secondary processing of the main document
% It overwrites the forwarding commands
% |\childdocof| and |\childdocforward|
% with empty macros to prevent further inclusions of the main document:
%    \begin{macrocode}
\newcommand{\childdocdisable}
{
  \renewcommand{\childdocmain}[1]{\renewcommand{\childdocmain}[1]{\endinput}}
  \renewcommand{\childdocof}[1]{}
  \renewcommand{\childdocby}[2][]{}
  \renewcommand{\childdocforward}[2][]{}
  \renewcommand{\childdocdisable}{}
}
%    \end{macrocode}

% \macro{\childdocmain}
% The macro |\childdocmain| is to be called at the top of the main file
% with nothing or the main filename (without extension) as argument.
% First, it breaks loops.
% If the argument is not empty and does not match |\childdocname|
% (which is set by the first inclusion of |childdoc.def|),
% |\ifchilddoc| is set to true, |\includeonly| is applied to the child file
% and |\jobname| is set to the main file
% (for proper handling of |.aux| files):
%    \begin{macrocode}
\newcommand{\childdocmain}[1]
{
  \childdocdisable\childdocmain{}
  \if?#1?\else
    \begingroup
      \def\childdoctmp{#1}
      \ifx\childdoctmp\childdocname
        \def\childdoctmp{}
      \else
        \def\childdoctmp
        {
          \childdoctrue
          \includeonly{\childdocname}
          \def\childdocjob{#1}
          \def\jobname{#1}
        }
      \fi
      \expandafter
    \endgroup
    \childdoctmp
  \fi
}
%    \end{macrocode}

% \macro{\childdocof}
% The command |\childdocof| redirects
% compilation to the main file |#1|.
%    \begin{macrocode}
\newcommand{\childdocof}[1]
{
  \childdocdisable
  \childdoctrue
  \includeonly{\childdocname}
  \def\jobname{#1}
  \def\childdocjob{#1}
  \input{#1}
}
%    \end{macrocode}

% \macro{\childdocby}
% The command |\childdocby| ....
%    \begin{macrocode}
\newcommand{\childdocby}[2][]
{
  \childdocdisable
  \childdoctrue
  \childdocmanualtrue
  \if?#1?\else
    \def\jobname{#2}
  \fi
  \def\childdocjob{#2}
  \input{#2}
  \endinput
}
%    \end{macrocode}

% \macro{\childdocforward}
% The command |\childdocforward| redirects
% compilation to the main file or
% (if the optional argument is given) a child file.
% Parameters are set as if the main file
% or a child file starting with |\childdocof| was compiled.
% Then compilation is handed over to the main file:
%    \begin{macrocode}
\newcommand{\childdocforward}[2][]
{
  \begingroup
    \if?#1?
      \def\childdoctmp
      {
        \def\childdocname{#2}
        \def\childdocjob{#2}
        \def\jobname{#2}
        \input{#2}
        \endinput
      }
    \else
      \def\childdoctmp
      {
        \childdocdisable
        \def\childdocname{#2}
        \childdoctrue
        \includeonly{#2}
        \def\childdocjob{#1}
        \def\jobname{#1}
        \input{#1}
        \endinput
      }
    \fi
    \expandafter
  \endgroup
  \childdoctmp
}
%    \end{macrocode}

% \macro{\childdocforwardprefix}
% The command |\childdocforwardprefix| redirects
% compilation to the main or a child file by means of a pattern.
% The prefix |#1| in the current filename is replaced by |#2|
% and the suffix of the current filename is kept
% (it is assumed that the filename does not contain the substring `|~~~|'
% which is used as a delimiter).
% Compilation is handed over to the new file by |\childdocforward|:
%    \begin{macrocode}
\newcommand{\childdocforwardprefix}[3][]
{
  \begingroup
    \def\childdocextract #2##1~~~{\def\childdoctmp{\childdocforward[#1]{#3##1}}}
    \expandafter\childdocextract\childdocname~~~
    \expandafter
  \endgroup
  \childdoctmp
}
%    \end{macrocode}

% \macro{\childdoc}
% The deprecated macro |\childdoc| is a legacy version of |\childdocmain|:
%    \begin{macrocode}
\newcommand{\childdoc}{\childdocmain}
%    \end{macrocode}

% \macro{\childdocredirect}
% The deprecated macro |\childdocredirect| is a legacy version
% of |\childdocforward| and |\childdocforwardprefix|:
%    \begin{macrocode}
\newcommand{\childdocredirect}[2][]
{
  \begingroup
    \if?#1?
      \def\childdoctmp{\childdocforward{#2}}
    \else
      \def\childdoctmp{\childdocforwardprefix{#1}{#2}}
    \fi
    \expandafter
  \endgroup
  \childdoctmp
}
%    \end{macrocode}

%\iffalse
%</package>
%\fi
%
\endinput
\childdocforward{cdocsamp}"|\\
% |latex -jobname cdocscl1 \|\\
% |  "% \iffalse
%
% childdoc.dtx Copyright (C) 2017-2018 Niklas Beisert
%
% This work may be distributed and/or modified under the
% conditions of the LaTeX Project Public License, either version 1.3
% of this license or (at your option) any later version.
% The latest version of this license is in
%   http://www.latex-project.org/lppl.txt
% and version 1.3 or later is part of all distributions of LaTeX
% version 2005/12/01 or later.
%
% This work has the LPPL maintenance status `maintained'.
%
% The Current Maintainer of this work is Niklas Beisert.
%
% This work consists of the files childdoc.dtx and childdoc.ins
% and the derived files childdoc.def and cdocsamp.tex with
% cdocsch1.tex, cdocsch2.tex, cdocsdrf.tex, cdocsfn1.tex, cdocsfn2.tex.
%
%<package>\ifdefined\childdocmain\endinput\fi
%<package>\ProvidesFile{childdoc.def}[2018/12/30 v2.0 child document driver]
%<samplemain>\ProvidesFile{cdocsamp.tex}[2018/12/30 v2.0 sample for childdoc]
%<*driver>
%\ProvidesFile{childdoc.drv}[2018/12/30 v2.0 childdoc reference manual file]
\PassOptionsToClass{10pt,a4paper}{article}
\documentclass{ltxdoc}

\usepackage[margin=35mm]{geometry}
\usepackage{hyperref}
\usepackage{hyperxmp}
\usepackage[usenames]{color}

\hypersetup{colorlinks=true}
\hypersetup{pdfstartview=FitH}
\hypersetup{pdfpagemode=UseNone}
\hypersetup{pdfsource={}}
\hypersetup{pdflang={en-UK}}
\hypersetup{pdfcopyright={Copyright 2017-2018 Niklas Beisert.
  This work may be distributed and/or modified under the
  conditions of the LaTeX Project Public License, either version 1.3
  of this license or (at your option) any later version.}}
\hypersetup{pdflicenseurl={http://www.latex-project.org/lppl.txt}}
\hypersetup{pdfcontactaddress={ETH Zurich, ITP, HIT K,
  Wolfgang-Pauli-Strasse 27}}
\hypersetup{pdfcontactpostcode={8093}}
\hypersetup{pdfcontactcity={Zurich}}
\hypersetup{pdfcontactcountry={Switzerland}}
\hypersetup{pdfcontactemail={nbeisert@itp.phys.ethz.ch}}
\hypersetup{pdfcontacturl={http://people.phys.ethz.ch/\xmptilde nbeisert/}}

\newcommand{\secref}[1]{\hyperref[#1]{section \ref*{#1}}}

\parskip1ex
\parindent0pt
\let\olditemize\itemize
\def\itemize{\olditemize\parskip0pt}

\begin{document}

\title{The \textsf{childdoc} Package}
\hypersetup{pdftitle={The childdoc Package}}
\author{Niklas Beisert\\[2ex]
  Institut f\"ur Theoretische Physik\\
  Eidgen\"ossische Technische Hochschule Z\"urich\\
  Wolfgang-Pauli-Strasse 27, 8093 Z\"urich, Switzerland\\[1ex]
  \href{mailto:nbeisert@itp.phys.ethz.ch}
  {\texttt{nbeisert@itp.phys.ethz.ch}}}
\hypersetup{pdfauthor={Niklas Beisert}}
\hypersetup{pdfsubject={Manual for the LaTeX2e Package childdoc}}
\date{30 December 2018, \textsf{v2.0}}
\maketitle

\begin{abstract}\noindent
\textsf{childdoc} is a \LaTeXe{} package
that enables the direct compilation
of document sections included by |\include|
to individual files.
\end{abstract}

\begingroup
\parskip0ex
\tableofcontents
\endgroup

%%%%%%%%%%%%%%%%%%%%%%%%%%%%%%%%%%%%%%%%%%%%%%%%%%%%%%%%%%%%%%%%%%%%%%%%%%%%%%%%
%%%%%%%%%%%%%%%%%%%%%%%%%%%%%%%%%%%%%%%%%%%%%%%%%%%%%%%%%%%%%%%%%%%%%%%%%%%%%%%%
\section{Introduction}

\LaTeX{} provides a mechanism to structure a large document (such as a book)
into a main file and several child files (containing the chapters)
using the |\include| command.
This mechanism is beneficial for documents
which span hundreds of pages in order to
make the source file(s) more manageable.
Moreover, compilation can be restricted to
selected child files by means of the |\includeonly| command.
The latter feature can be used to reduce the compilation time while editing
(this was significantly more useful in the earlier days of \LaTeX{})
or to generate a smaller document which is easier to navigate.
Another application of |\includeonly| is to generate
documents consisting of selected parts of the complete document.

However, there are a few drawbacks of the plain |\include| mechanism:
\begin{itemize}
\item
The child files cannot be compiled on their own,
they can only be compiled via the main file.
A naive editing environment
(such as a text editor with an option
to have the current file processed by \LaTeX)
may require one to switch to the main file before compiling;
attempting to compile the child file produces errors.
\item
The main file must be modified (each time)
to adjust the |\includeonly| command
to the present needs. This easily leaves the main file in a messy state.
\item
The generated document will always carry the filename
of the main document. This is inconvenient if
several child files are to be compiled and
to be kept for distribution.
\end{itemize}

The present package provides a simple interface
to make child files individually compilable by \LaTeX{}.
Compiling a child file then has the same effect as compiling
the main file with an |\includeonly| command
to select the appropriate child.
Moreover the generated document will carry the name of the child
rather than the main file.
This resolves all three above issues.

This feature is meant to make the editing of books,
thesis documents and lecture notes somewhat more convenient.
However, the package can also be used efficiently for
composing a series of documents (such as exercise sheets)
which are typically distributed individually.
It then assists the author in generating the individual documents
(potentially in different versions)
as well as a document containing the collected series.
Another application is in developing style files
or other kinds of included material
where compilation of the style file could redirect
to a sample or test file.

%%%%%%%%%%%%%%%%%%%%%%%%%%%%%%%%%%%%%%%%%%%%%%%%%%%%%%%%%%%%%%%%%%%%%%%%%%%%%%%%
%%%%%%%%%%%%%%%%%%%%%%%%%%%%%%%%%%%%%%%%%%%%%%%%%%%%%%%%%%%%%%%%%%%%%%%%%%%%%%%%
\section{Usage}

First of all, the package \textsf{childdoc} is \emph{not} a standard
\LaTeXe{} |.sty| style file! Therefore it needs to be invoked in
a non-standard way.

%%%%%%%%%%%%%%%%%%%%%%%%%%%%%%%%%%%%%%%%%%%%%%%%%%%%%%%%%%%%%%%%%%%%%%%%%%%%%%%%
\subsection{Included Files}
\label{sec:include}

%%%%%%%%%%%%%%%%%%%%%%%%%%%%%%%%%%%%%%%%
\DescribeMacro{\childdocmain}
To use the package, add the commands
\begin{center}
\begin{tabular}{l}
|\input{childdoc.def}|\\
|\childdocmain{}|\\
\end{tabular}
\end{center}
at the very top of the main \LaTeX{} file,
in particular \emph{before} the |\documentclass| statement!
The argument of |\childdocmain| should be left empty
(but it must be present).

%%%%%%%%%%%%%%%%%%%%%%%%%%%%%%%%%%%%%%%%
\DescribeMacro{\childdocof}
Furthermore, add the commands
\begin{center}
\begin{tabular}{l}
|\input{childdoc.def}|\\
|\childdocof{|\textit{main}|}|\\
\end{tabular}
\end{center}
at the top of every child file \textit{child}
which is included by |\include{|\textit{child}|}|
from within the main file
(or at least for those files to be compiled individually).
The argument \textit{main} must be the filename of the main file.

There are a couple of
considerations in setting up the main and child documents:

%%%%%%%%%%%%%%%%%%%%%%%%%%%%%%%%%%%%%%%%
\paragraph{Restrictions.}

Please note the following restrictions:
\begin{itemize}
\item
|\childdocmain| must be called with one argument \textit{main}
to ensure compatibility with earlier version of the package.
It must either be empty (|\childdocmain{}|)
or precisely match the filename of the main file in which it is specified.
See \secref{sec:detection} for further information.
\item
The filename \textit{main} must be specified without the |.tex| extension.
\item
The filename \textit{main} is case sensitive
(even in case-insensitive file systems)
due to internal string comparison.
\item
The argument \textit{main} should be fully expanded, it cannot be a macro.
\item
Subdirectories and special characters should be avoided in filenames.
\item
The command |\childdocmain{|\textit{main}|}| must be followed by a whitespace.
It should not be followed immediately by another command
or by a comment mark `|%|'.
This is because the \TeX{} parser reads the token immediately following
the argument of |\childdocmain| and puts it
at the beginning of every child section;
however, a white\-space is ignored.
\end{itemize}

%%%%%%%%%%%%%%%%%%%%%%%%%%%%%%%%%%%%%%%%
\paragraph{Content of Main File.}

It is advisable to place all content in the child files included by |\include|.
Any output contained in the main file will appear in all child documents
unless suppressed manually;
it cannot be suppressed automatically by the |\includeonly| directive
and thus should normally be avoided.
A method to include some content in the main file
by means of conditional processing is described in \secref{sec:conditional}.

%%%%%%%%%%%%%%%%%%%%%%%%%%%%%%%%%%%%%%%%
\paragraph{Page Numbering.}

When only a part of the document is compiled,
the appropriate numbering of pages
(as well as other status parameters)
is determined from the |.aux| files.
The latter contain information from previous passes.
However this information needs to propagate through
all intermediate child documents.
Therefore the page numbering in child documents may well
be inconsistent until the complete document is compiled at least once.

A useful (if unconventional) way to always ensure a consistent
page numbering is to restart the numbering in each child document
and denote the pages by `\textit{child}|.|\textit{page}'
where \textit{child} represents the chapter/section number of the child file.
This can be achieved by the command
|\numberwithin{page}{|\textit{child}|}|
of the \textsf{amsmath} package
where \textit{child} can be |chapter| or |section|
depending on the chosen structuring.
Alternatively, one can modify the macro |\thepage| appropriately
and reset the counter |page| at the start of each child file.

%%%%%%%%%%%%%%%%%%%%%%%%%%%%%%%%%%%%%%%%%%%%%%%%%%%%%%%%%%%%%%%%%%%%%%%%%%%%%%%%
\subsection{Conditional Processing}
\label{sec:conditional}

The package provides a mechanism to compile different versions
of a document. To customise the versions further some conditional processing
can come in handy to distinguish which version is being compiled.
The package provides two macros to describe the compilation context:

%%%%%%%%%%%%%%%%%%%%%%%%%%%%%%%%%%%%%%%%
\DescribeMacro{\ifchilddoc}
The conditional |\ifchilddoc| distinguishes between the compilation of
child documents and the main document:
%
\begin{center}
|\ifchilddoc |\textit{child-code}| |[|\||else |\textit{main-code}]| \||fi|
\end{center}

%%%%%%%%%%%%%%%%%%%%%%%%%%%%%%%%%%%%%%%%
\DescribeMacro{\childdocname}
\DescribeMacro{\childdocjob}
The macro |\childdocname| contains the filename (without extension)
of the main or child file being processed.
Note that |\childdocjob| will always contain the name of the main file.

%%%%%%%%%%%%%%%%%%%%%%%%%%%%%%%%%%%%%%%%
\paragraph{Title Page.}

Conditional processing can be used to include a title or banner page
in the main document when proper precautions are taken.
Importantly, the code in the main file should ensure that the page counter
(as well as other status parameters which are stored in the |.aux| files)
takes the same value after the conditional processing.
Otherwise the page numbers may take divergent values
depending on which part is compiled.

For example, a title page could be declared by:
%
\begin{center}
\begin{tabular}{l}
|\ifchilddoc\||else|\\
|\addtocounter{page}{-1}|\\
\textit{code for title page}\\
|\newpage|\\
|\||fi|
\end{tabular}
\end{center}
%
A banner page for the child documents can be generated by:
%
\begin{center}
\begin{tabular}{l}
|\ifchilddoc|\\
|\addtocounter{page}{-1}|\\
\textit{code for banner page}\\
|\newpage|\\
|\||fi|
\end{tabular}
\end{center}
%
Here one could write a message such as:
\begin{center}
|This is the part \childdocname{} of \childdocjob{}.|
\end{center}

%%%%%%%%%%%%%%%%%%%%%%%%%%%%%%%%%%%%%%%%%%%%%%%%%%%%%%%%%%%%%%%%%%%%%%%%%%%%%%%%
\subsection{Flags}
\label{sec:flags}

The package makes it easy to generate different versions
of the main or child documents.
To this end compilation flags can be defined
and assigned different default values.
They will be particularly useful in conjunction
with the forwarding mechanism described in \secref{sec:forward}.

For example, it may be useful to have a flag |\version|
which can be set to |draft| or |final|.
The document source will contain some conditional code
depending on the value of |\version|.
Suppose further, the flag should default to |final| for the main file
and to |draft| for child files
which is a natural assignment for editing the document.
This is achieved by placing the following code
in the preamble of the main document
(below the |\childdocmain| directive):
%
\begin{center}
\begin{tabular}{l}
|\ifchilddoc|\\
|\providecommand{\version}{draft}|\\
|\||else|\\
|\providecommand{\version}{final}|\\
|\||fi|
\end{tabular}
\end{center}
%
The definition by |\providecommand| makes sure
that previous definitions are not overwritten.
Further statements |\providecommand{\version}{...}|
can thus be added before the above code to override it.

For the main file, one might add a line
(between |\childdocmain| and the above block)
%
\begin{center}
|%\ifchilddoc\||else\providecommand{\version}{draft}\||fi|
\end{center}
%
which can be uncommented to produce a draft version.
Likewise one can add a line to the very top of a child file
(above the |\childdocof{|\textit{main}|}| directive)
%
\begin{center}
|%\providecommand{\version}{final}|
\end{center}
%
which can be uncommented to produce the final version of this child document.

%%%%%%%%%%%%%%%%%%%%%%%%%%%%%%%%%%%%%%%%%%%%%%%%%%%%%%%%%%%%%%%%%%%%%%%%%%%%%%%%
\subsection{Forwarding}
\label{sec:forward}

Different versions of the main or child documents
using compilation flags as described in \secref{sec:flags}
can be (permanently) stored in different files
for convenient compilation, viewing and distribution.
To this end, the package defines a command
to pass on compilation to a different file:

%%%%%%%%%%%%%%%%%%%%%%%%%%%%%%%%%%%%%%%%
\DescribeMacro{\childdocforward}
The command |\childdocforward| redirects processing to
another source file:
%
\begin{center}
\begin{tabular}{l}
|\input{childdoc.def}|\\
|\childdocforward[|\textit{main}|]{|\textit{dest}|}|\\
\end{tabular}
\end{center}
%
The argument \textit{dest} is the destination file
(without extension).
It should be the main file or one of the child files.
Note that further \textsf{childdoc} directives
such as |\childdocof| and |\childdocforward|
in the indicated file will be processed in this form.
The optional argument \textit{main}
passes on directly to the main file \textit{main}
while pretending to compile the child \textit{dest}.
This form behaves as if \textit{dest}
issues |\childdocof{|\textit{main}|}| right away,
and no further \textsf{childdoc} directives will be processed.

%%%%%%%%%%%%%%%%%%%%%%%%%%%%%%%%%%%%%%%%
\DescribeMacro{\...prefix}
In the alternative form |\childdocforwardprefix|,
%
\begin{center}
\begin{tabular}{l}
|\input{childdoc.def}|\\
|\childdocforwardprefix[|\textit{main}|]{|\textit{prefix}|}{|\textit{dest}|}|
\end{tabular}
\end{center}
%
the destination file is determined by a pattern
depending on the current file:
To make this work, the current file must be called
`{\textit{prefix}\hspace{0.2em}\textit{suffix}}'
with \textit{prefix} matching precisely the argument.
Processing is then passed on to the file
`{\textit{dest}\hspace{0.2em}\textit{suffix}}'.
Surely, the same effect is achieved by
directly specifying the
argument `{\textit{dest}\hspace{0.2em}\textit{suffix}}'
in the first form.
However, that requires to set up a different file
for each child. With the alternative form of the command
all these files can have exactly the same content
which simplifies setting them up and maintaining them.

For example, the following file |draft.tex|
with a compilation flag |\version| as described in \secref{sec:flags}
compiles the main document as a draft:
%
\begin{center}
\begin{tabular}{l}
|\def\version{draft}|\\
|\input{childdoc.def}|\\
|\childdocforward{|\textit{main}|}|
\end{tabular}
\end{center}
%
Likewise, the following files |final|\textit{nn}|.tex|
compile the final version of the child document
|child|\textit{nn}|.tex|:
%
\begin{center}
\begin{tabular}{l}
|\def\version{final}|\\
|\input{childdoc.def}|\\
|\childdocforwardprefix{final}{child}|
\end{tabular}
\end{center}
%

Note that when several versions of a main file and/or of each child file
are to be generated, it may be convenient to set up a |Makefile| or
shell script to automatise the process.

%%%%%%%%%%%%%%%%%%%%%%%%%%%%%%%%%%%%%%%%%%%%%%%%%%%%%%%%%%%%%%%%%%%%%%%%%%%%%%%%
\subsection{Command Line Processing}
\label{sec:commandline}

The effect of redirection files can also be achieved by invoking
the \LaTeX{} compiler with a more elaborate command line.
Most conveniently this should be done as part
of a shell script or a |Makefile|.

When using \textsf{childdoc} in the main file, the following
command lines effectively perform a redirection
(note that depending on the shell being used,
backslashes may have to be doubled: `|\|' $\to$ `|\\|'):
%
\begin{center}
|... -jobname "|\textit{target}|" |\\|"|[\textit{flags}]%
|\input{childdoc.def}\childdocforward[|\textit{main}|]{|\textit{dest}|}"|
\end{center}
%
Here \textit{target} is the name of the output file,
\textit{main} is the name of the main file
and \textit{dest} is the name of the main or child file to be processed
(all filenames without extensions).
The optional argument \textit{main} can be omitted
if \textit{main} matches \textit{dest}.
Optionally, compilation \textit{flags} can be defined via |\def| commands.
This command line makes the \TeX{} engine believe
it is compiling the file \textit{target}
whose content is specified as the latter parameter.
The provided code then forwards the processing to
\textit{main} or \textit{dest} as described in \secref{sec:forward}.

%%%%%%%%%%%%%%%%%%%%%%%%%%%%%%%%%%%%%%%%%%%%%%%%%%%%%%%%%%%%%%%%%%%%%%%%%%%%%%%%
\subsection{Include by Input}
\label{sec:input}

Including child documents by |\include| has some restrictions by design.
Most notably, the content of a child document always occupies
its own set of pages; pages cannot be shared between child documents.
Usually, this behaviour makes perfect sense
because each child document contain an essential part of the document.
However, in some situations it may be desirable to compose
a document from a collection of parts
without having mandatory page breaks between then.
For this case, the package
provides a mechanism to include parts
by |\input| which can also be processed individually.
However, by construction this mechanism
requires manual handling of the content to be output.

%%%%%%%%%%%%%%%%%%%%%%%%%%%%%%%%%%%%%%%%
\DescribeMacro{\ifchilddocmanual}
The main file should be prepared as usual, see \secref{sec:include}.
However, the document body must make a distinction
between processing of an individual part and of the main document, e.g.:
%
\begin{center}
\begin{tabular}{l}
|\ifchilddocmanual|\\
|\input{\childdocname}|\\
|\||else|\\
\textit{document body with }|\input{|\textit{part}|}|\\
|\||fi|
\end{tabular}
\end{center}
%
The conditional |\ifchilddocmanual| is true whenever
a part to be included by |\input| is being compiled,
and the name of the part is stored in |\childdocname|.

%%%%%%%%%%%%%%%%%%%%%%%%%%%%%%%%%%%%%%%%
\DescribeMacro{\childdocby}
Each part to be included by |\input| should start with:
%
\begin{center}
\begin{tabular}{l}
|\input{childdoc.def}|\\
|\childdocby{|\textit{main}|}|\\
\end{tabular}
\end{center}
%
The directive |\childdocby| is similar to |\childdocof|
described in \secref{sec:include},
but the subsequent selection of content must be done manually.
To that end, both |\ifchilddoc| and |\ifchilddocmanual|
will be true upon processing of a part,
and the name of the part is stored in |\childdocname|.
Note that |\jobname| will be set to the filename of the current part
so that each part receives an individual |.aux| file
that does not interfere with the |.aux| file(s) of the main document.
This behaviour can be altered by the alternative form
|\childdocby[*]{|\textit{main}|}| (with a non-empty optional argument)
which uses the |.aux| file of the main document
by setting |\jobname| to \textit{main}.

%%%%%%%%%%%%%%%%%%%%%%%%%%%%%%%%%%%%%%%%%%%%%%%%%%%%%%%%%%%%%%%%%%%%%%%%%%%%%%%%
\subsection{Driver Development}
\label{sec:driver}

The \textsf{childdoc} mechanism can also be use for the development
of definition files such as \LaTeX{} styles or classes.
This case differs from the above setup with multiple parts
included by |\include| in that no |\includeonly| should be invoked.
This can be achieved by starting the include file
(before |\ProvidesPackage|) with:
%
\begin{center}
\begin{tabular}{l}
|\input{childdoc.def}|\\
|\childdocforward{|\textit{main}|}|\\
\end{tabular}
\end{center}
%
or alternatively with:
%
\begin{center}
\begin{tabular}{l}
|\input{childdoc.def}|\\
|\childdocby{|\textit{main}|}|\\
\end{tabular}
\end{center}
%
Both forms have slightly different effects as described above.
The main file is prepared as usual, see \secref{sec:include}.

%%%%%%%%%%%%%%%%%%%%%%%%%%%%%%%%%%%%%%%%%%%%%%%%%%%%%%%%%%%%%%%%%%%%%%%%%%%%%%%%
\subsection{Legacy Detection}
\label{sec:detection}

The directive |\childdocmain| in the main file can detect
whether the complete document or merely a child is to be compiled
even without using the directive |\childdocof|.
This method is deprecated because it is less robust
and there is no compelling reason to use it;
it is merely provided for backward compatibility
and it may be removed in future versions.

If the detection mechanism is to be used,
it is mandatory to correctly specify
the filename of the main file as the argument of |\childdocmain|:
%
\begin{center}
\begin{tabular}{l}
|\input{childdoc.def}|\\
|\childdocmain{|\textit{main}|}|\\
\end{tabular}
\end{center}
%
If |\jobname| does not match the argument \textit{main} of |\childdocmain|,
it is assumed that |\jobname| points to the child file to be compiled.
When using |\childdocmain| with the main file specified as argument,
it suffices to start a child file
with just |\input{|\textit{main}|}|
without loading of the package and using |\childdocof|.
If instead all processing is done
with the appropriate \textsf{childdoc} directives,
the argument of \textit{main} of |\childdocmain| can be empty.

An alternative version of the command line processing described
in \secref{sec:commandline} using the detection mechanism reads:
%
\begin{center}
|... -jobname "|\textit{target}|" "|[\textit{flags}]%
[|\def\jobname{|\textit{dest}|}|]|\input{|\textit{main}|}"|
\end{center}

%%%%%%%%%%%%%%%%%%%%%%%%%%%%%%%%%%%%%%%%%%%%%%%%%%%%%%%%%%%%%%%%%%%%%%%%%%%%%%%%
\subsection{Manual Code}
\label{sec:manual}

In case one cannot be certain whether the definitions file |childdoc.def|
is installed on the target \TeX{} distribution
and one prefers not to ship it,
it is conceivable to paste a few relevant commands into the sources.

To that end, drop all statements |\input{childdoc.def}|
and perform the replacements as outlined below.
Instead of |\childdocmain{|\textit{main}|}| add the following code
to the top of the main file:
%
\begin{center}
\begin{tabular}{l}
|\||ifdefined\childdocname\endinput\||fi\newif\ifchilddoc|\\
|\edef\childdocname{\scantokens\expandafter{\jobname\noexpand}}|\\
|\def\childdocmain{|\textit{main}|}\||ifx\childdocmain\childdocname\||else|\\
|\childdoctrue\includeonly{\childdocname}\let\jobname\childdocmain\||fi|\\
\end{tabular}
\end{center}
%
Instead of |\childdocof{|\textit{main}|}| just include the main file
at the top of each child file:
%
\begin{center}
|\input{|\textit{main}|}|
\end{center}
%
A simple redirection |\childdocforward{|\textit{dest}|}| is achieved by:
%
\begin{center}
|\def\jobname{|\textit{dest}|}\input{\jobname}|
\end{center}
%
The redirection with prefix
|\childdocforwardprefix[|\textit{prefix}|]{|\textit{dest}|}|
is accomplished by:
%
\begin{center}
\begin{tabular}{l}
|{\edef\jobname{\scantokens\expandafter{\jobname\noexpand}}|\\
|\def\redirectjob |\textit{prefix}|#1~~~{\gdef\jobname{|\textit{dest}|#1}}|\\
|\expandafter\redirectjob\jobname~~~}\input{\jobname}|
\end{tabular}
\end{center}

In an alternative approach,
child documents can be compiled by a specific command line
without additional code or specific definitions:
%
\begin{center}
|... -jobname "|\textit{target}|" "|[\textit{flags}]%
|\includeonly{|\textit{dest}|}\input{|\textit{main}|}"|
\end{center}
%

%%%%%%%%%%%%%%%%%%%%%%%%%%%%%%%%%%%%%%%%%%%%%%%%%%%%%%%%%%%%%%%%%%%%%%%%%%%%%%%%
%%%%%%%%%%%%%%%%%%%%%%%%%%%%%%%%%%%%%%%%%%%%%%%%%%%%%%%%%%%%%%%%%%%%%%%%%%%%%%%%
\section{Information}

%%%%%%%%%%%%%%%%%%%%%%%%%%%%%%%%%%%%%%%%%%%%%%%%%%%%%%%%%%%%%%%%%%%%%%%%%%%%%%%%
\subsection{Copyright}

Copyright \copyright{} 2017--2018 Niklas Beisert

This work may be distributed and/or modified under the
conditions of the \LaTeX{} Project Public License, either version 1.3
of this license or (at your option) any later version.
The latest version of this license is in
  \url{http://www.latex-project.org/lppl.txt}
and version 1.3 or later is part of all distributions of \LaTeX{}
version 2005/12/01 or later.

This work has the LPPL maintenance status `maintained'.

The Current Maintainer of this work is Niklas Beisert.

This work consists of the files |README.txt|, |childdoc.ins| and |childdoc.dtx|
as well as the derived files |childdoc.def|, |cdocsamp.tex|
with |cdocsch1.tex|, |cdocsch2.tex|, |cdocspt3.tex|, |cdocspt4.tex|,
|cdocsdrf.tex|, |cdocsfn1.tex|, |cdocsfn2.tex|
as well as |childdoc.pdf|.

%%%%%%%%%%%%%%%%%%%%%%%%%%%%%%%%%%%%%%%%%%%%%%%%%%%%%%%%%%%%%%%%%%%%%%%%%%%%%%%%
\subsection{Files and Installation}

The package consists of the files:
%
\begin{center}
\begin{tabular}{ll}
    |README.txt|   & readme file \\
    |childdoc.ins| & installation file \\
    |childdoc.dtx| & source file \\
    |childdoc.def| & definition file \\
    |cdocsamp.tex| & sample main file \\
    |cdocsch1.tex| & sample include file \\
    |cdocsch2.tex| & sample include file \\
    |cdocspt3.tex| & sample part file \\
    |cdocspt4.tex| & sample part file \\
    |cdocsdrf.tex| & sample redirection file \\
    |cdocsfn1.tex| & sample redirection file \\
    |cdocsfn2.tex| & sample redirection file \\
    |childdoc.pdf| & manual
\end{tabular}
\end{center}
%
The distribution consists of the files
|README.txt|, |childdoc.ins| and |childdoc.dtx|.
%
\begin{itemize}
\item
Run (pdf)\LaTeX{} on |childdoc.dtx|
to compile the manual |childdoc.pdf| (this file).
\item
Run \LaTeX{} on |childdoc.ins| to create the definitions file |childdoc.def|
and the sample |cdocsamp.tex| with include files
|cdocsch1.tex|, |cdocsch2.tex|, |cdocspt3.tex|, |cdocspt4.tex|,
|cdocsdrf.tex|, |cdocsfn1.tex|, |cdocsfn2.tex|.
Then copy the file |childdoc.def| to an appropriate directory of your \LaTeX{}
distribution, e.g.\ \textit{texmf-root}|/tex/latex/childdoc|.
\end{itemize}

%%%%%%%%%%%%%%%%%%%%%%%%%%%%%%%%%%%%%%%%%%%%%%%%%%%%%%%%%%%%%%%%%%%%%%%%%%%%%%%%
\subsection{Related CTAN Packages}

There are several other packages which offer a similar functionality:
%
\begin{itemize}
\item
The packages
\href{http://ctan.org/pkg/docmute}{\textsf{docmute}},
\href{http://ctan.org/pkg/includex}{\textsf{includex}} and
\href{http://ctan.org/pkg/standalone}{\textsf{standalone}}
provide commands to include only the document body of
a child file thus allowing both files to be compiled individually.
\item
The packages \href{http://ctan.org/pkg/subdocs}{\textsf{subdocs}}
and \href{http://ctan.org/pkg/subfiles}{\textsf{subfiles}}
provide structures in which the main and child documents can be
encapsulated and allowing them to be compiled individually.
The inclusion mechanism is different from the conventional |\include|.
\item
The package \href{http://ctan.org/pkg/combine}{\textsf{combine}}
is an elaborate solution to combine several documents into one.
\end{itemize}
%
See also the CTAN topic \href{http://ctan.org/topic/subdocs}{\textsf{subdocs}}
for further related packages.
The present package differs from the above solutions in that
a document structure constructed with the conventional |\include| mechanism
just needs two extra commands at the top of every file
such that all constituent files can be compiled individually.

%%%%%%%%%%%%%%%%%%%%%%%%%%%%%%%%%%%%%%%%%%%%%%%%%%%%%%%%%%%%%%%%%%%%%%%%%%%%%%%%
%\subsection{Feature Suggestions}
%
%The following is a list of features which may be useful for future
%versions of this package:
%%
%\begin{itemize}
%\item
%\ldots
%\end{itemize}

%%%%%%%%%%%%%%%%%%%%%%%%%%%%%%%%%%%%%%%%%%%%%%%%%%%%%%%%%%%%%%%%%%%%%%%%%%%%%%%%
\subsection{Revision History}

%%%%%%%%%%%%%%%%%%%%%%%%%%%%%%%%%%%%%%%%
\paragraph{v2.0:} 2018/12/30

\begin{itemize}
\item
immediate forward processing
\item
added |\childdocby| mechanism
\item
manual restructured
\end{itemize}

%%%%%%%%%%%%%%%%%%%%%%%%%%%%%%%%%%%%%%%%
\paragraph{v1.6:} 2018/01/17

\begin{itemize}
\item
application for development of include files
\item
corrections to manual
\end{itemize}

%%%%%%%%%%%%%%%%%%%%%%%%%%%%%%%%%%%%%%%%
\paragraph{v1.5:} 2017/05/21

\begin{itemize}
\item
more complete structuring introduced
\item
|\childdocof| introduced
\item
|\childdoc| renamed to |\childdocmain|
\item
|\childredirect| renamed to |\childdocforward| and |\childdocforwardprefix|
and functionality expanded
\end{itemize}

%%%%%%%%%%%%%%%%%%%%%%%%%%%%%%%%%%%%%%%%
\paragraph{v1.0:} 2017/04/27

\begin{itemize}
\item
manual and install package
\item
first version published on CTAN
\end{itemize}

%%%%%%%%%%%%%%%%%%%%%%%%%%%%%%%%%%%%%%%%
\paragraph{v0.6:} 2017/04/26

\begin{itemize}
\item
redirection mechanism added
\end{itemize}

%%%%%%%%%%%%%%%%%%%%%%%%%%%%%%%%%%%%%%%%
\paragraph{v0.5:} 2017/04/26

\begin{itemize}
\item
functionality in definition file
\end{itemize}


%%%%%%%%%%%%%%%%%%%%%%%%%%%%%%%%%%%%%%%%%%%%%%%%%%%%%%%%%%%%%%%%%%%%%%%%%%%%%%%%
%%%%%%%%%%%%%%%%%%%%%%%%%%%%%%%%%%%%%%%%%%%%%%%%%%%%%%%%%%%%%%%%%%%%%%%%%%%%%%%%
%%%%%%%%%%%%%%%%%%%%%%%%%%%%%%%%%%%%%%%%%%%%%%%%%%%%%%%%%%%%%%%%%%%%%%%%%%%%%%%%
\appendix

\settowidth\MacroIndent{\rmfamily\scriptsize 000\ }

 \DocInput{childdoc.dtx}

\end{document}
%</driver>
% \fi
%
% %%%%%%%%%%%%%%%%%%%%%%%%%%%%%%%%%%%%%%%%%%%%%%%%%%%%%%%%%%%%%%%%%%%%%%%%%%%%%%
% %%%%%%%%%%%%%%%%%%%%%%%%%%%%%%%%%%%%%%%%%%%%%%%%%%%%%%%%%%%%%%%%%%%%%%%%%%%%%%
% \section{Sample}
%\iffalse
%<*samplemain>
%\fi
%
% The following presents a sample document
% with two chapters, two parts, a title page,
% a compile flag as well as three forwarding files to set the flag.
% It consists of eight |.tex| files:
% \begin{center}
% \begin{tabular}{ll}
% |cdocsamp.tex|&main file\\
% |cdocsch1.tex|&include file for chapter 1\\
% |cdocsch2.tex|&include file for chapter 2\\
% |cdocspt3.tex|&include file for part 3\\
% |cdocspt4.tex|&include file for part 4\\
% |cdocsdrf.tex|&forwarding file for main file in draft mode\\
% |cdocsfi1.tex|&forwarding file for final version of chapter 1\\
% |cdocsfi2.tex|&forwarding file for final version of chapter 2\\
% \end{tabular}
% \end{center}
% Each of the eight files can be compiled directly by the \LaTeX{} compiler.
%
% %%%%%%%%%%%%%%%%%%%%%%%%%%%%%%%%%%%%%%
% \paragraph{Main File.}
%
% The main file is called |cdocsamp.tex|.
%
% Load the \textsf{childdoc} definitions and
% declare the filename for the main document:
%    \begin{macrocode}
\input{childdoc.def}
\childdocmain{}
%    \end{macrocode}

% Optional override for |\version| flag:
%    \begin{macrocode}
%%\ifchilddoc\else\providecommand{\version}{draft}\fi
%    \end{macrocode}

% Define the default values for the |\version| flag
% (|final| for the main file and |draft| for childs):
%    \begin{macrocode}
\ifchilddoc
\providecommand{\version}{draft}
\else
\providecommand{\version}{final}
\fi
%    \end{macrocode}

% Load the standard document class:
%    \begin{macrocode}
\documentclass[12pt]{article}
%    \end{macrocode}

% Start the document body:
%    \begin{macrocode}
\begin{document}
%    \end{macrocode}

% Declare a title page.
% Print title, part of document being processed and version flag:
%    \begin{macrocode}
\addtocounter{page}{-1}
\begin{center}
{\LARGE\bfseries{}childdoc example\par}
\vspace{1cm}
\ifchilddoc
\ifchilddocmanual part\else chapter\fi:
`\childdocname' of `\childdocjob'\par
\else
main document: `\childdocjob'\par
\fi
version: \version\par
\end{center}
\newpage
%    \end{macrocode}

% Manually include selected file,
% otherwise process as usual:
%    \begin{macrocode}
\ifchilddocmanual
\section*{part `\childdocname'}
\input{\childdocname}
\else
%    \end{macrocode}

% Include the two chapters:
%    \begin{macrocode}
\include{cdocsch1}
\include{cdocsch2}
%    \end{macrocode}

% Include the two parts unless only chapters should be displayed:
%    \begin{macrocode}
\ifchilddoc\else
\section{part three}
\input{cdocspt3}
\section{part four}
\input{cdocspt4}
\fi
%    \end{macrocode}

% Process as usual until here:
%    \begin{macrocode}
\fi
%    \end{macrocode}

% End of document body:
%    \begin{macrocode}
\end{document}
%    \end{macrocode}
%\iffalse
%</samplemain>
%\fi
%
% %%%%%%%%%%%%%%%%%%%%%%%%%%%%%%%%%%%%%%
% \paragraph{Chapter Include Files.}
%
% The include files are called |cdocsch1.tex| and |cdocsch2.tex|.
%
%\iffalse
%<*samplechap1|samplechap2>
%\fi

% Optional override for |\version| flag:
%    \begin{macrocode}
%%\providecommand{\version}{final}
%    \end{macrocode}

% Include the main document:
%    \begin{macrocode}
\input{childdoc.def}
\childdocof{cdocsamp}
%    \end{macrocode}

%\iffalse
%</samplechap1|samplechap2>
%\fi
%
%\iffalse
%<*samplechap1>
%\fi
% Some text for chapter 1:
%    \begin{macrocode}
\section{one}
some text in chapter one
%    \end{macrocode}

%\iffalse
%</samplechap1>
%\fi
% Some text for chapter 2:
%\iffalse
%<*samplechap2>
%\fi
%    \begin{macrocode}
\section{two}
more text in chapter two
%    \end{macrocode}

%\iffalse
%</samplechap2>
%\fi
%
% %%%%%%%%%%%%%%%%%%%%%%%%%%%%%%%%%%%%%%
% \paragraph{Part Include Files.}
%
% The include files are called |cdocspt3.tex| and |cdocspt4.tex|.
%
%\iffalse
%<*samplepart3|samplepart4>
%\fi

% Optional override for |\version| flag:
%    \begin{macrocode}
%%\providecommand{\version}{final}
%    \end{macrocode}

% Include the main document:
%    \begin{macrocode}
\input{childdoc.def}
\childdocby{cdocsamp}
%    \end{macrocode}

%\iffalse
%</samplepart3|samplepart4>
%\fi
%
%\iffalse
%<*samplepart3>
%\fi
% Some text for part 3:
%    \begin{macrocode}
some text in part three
%    \end{macrocode}

%\iffalse
%</samplepart3>
%\fi
% Some text for part 4:
%\iffalse
%<*samplepart4>
%\fi
%    \begin{macrocode}
more text in part four
%    \end{macrocode}

%\iffalse
%</samplepart4>
%\fi
%
% %%%%%%%%%%%%%%%%%%%%%%%%%%%%%%%%%%%%%%
% \paragraph{Forwarding for a Complete Draft.}
%
% The following forwarding file |cdocsdrf.tex|
% compiles the main document in draft mode:
%\iffalse
%<*sampledraft>
%\fi
%    \begin{macrocode}
\def\version{draft}
\input{childdoc.def}
\childdocforward{cdocsamp}
%    \end{macrocode}

%\iffalse
%</sampledraft>
%\fi
%
% %%%%%%%%%%%%%%%%%%%%%%%%%%%%%%%%%%%%%%
% \paragraph{Forwarding for Final Version of the Chapters.}
%
% The following forwarding files |cdocsfn1.tex| and |cdocsfn2.tex|
% (with identical content)
% compile the final versions of the child documents
% |cdocsch1.tex| and |cdocsch2.tex|, respectively:
%\iffalse
%<*samplefinal>
%\fi
%    \begin{macrocode}
\def\version{final}
\input{childdoc.def}
\childdocforwardprefix[cdocsamp]{cdocsfn}{cdocsch}
%    \end{macrocode}

%\iffalse
%</samplefinal>
%\fi
%
% %%%%%%%%%%%%%%%%%%%%%%%%%%%%%%%%%%%%%%
% \paragraph{Command Line Processing.}
%
% The following three command lines generate the output files
% |cdocscld|, |cdocscl1| and |cdocscl2|
% which should be identical to
% |cdocsdrf|, |cdocsch1| and |cdocsfn2|, respectively:
% \begin{center}
% \begin{tabular}{l}
% |latex -jobname cdocscld \|\\
% |  "\def\version{draft}\input{childdoc.def}\childdocforward{cdocsamp}"|\\
% |latex -jobname cdocscl1 \|\\
% |  "\input{childdoc.def}\childdocforward[cdocsamp]{cdocsch1}"|\\
% |latex -jobname cdocscl2 \|\\
% |  "\def\version{final}\input{childdoc.def}\childdocforward{cdocsch2}"|
% \end{tabular}
% \end{center}
% Note that the trailing backslash on each first line
% merely continues the input to the second line
% (for convenient cut ant paste).
% Furthermore, the command |latex| can be replaced by any
% of its alternative versions such as |pdflatex|.
%
% %%%%%%%%%%%%%%%%%%%%%%%%%%%%%%%%%%%%%%%%%%%%%%%%%%%%%%%%%%%%%%%%%%%%%%%%%%%%%%
% %%%%%%%%%%%%%%%%%%%%%%%%%%%%%%%%%%%%%%%%%%%%%%%%%%%%%%%%%%%%%%%%%%%%%%%%%%%%%%
% \section{Implementation}
%\iffalse
%<*package>
%\fi
%
% This section describes the definitions file |childdoc.def|.

% The definitions cannot be loaded using |\usepackage| or |\RequirePackage|
% which has a mechanism to prevent loading a style file more than once.
% When loading the definitions by means of |\input|
% multiple instances have to be prevented manually:
%\iffalse
%This code needs to be before the `\ProvidesFile' directive
%which is defined at the beginning of this file.
%Therefore it is also placed there and commented out here.
%</package>
%<*discard>
%\fi
%    \begin{macrocode}
\ifdefined\childdocmain\endinput\fi
%    \end{macrocode}
%\iffalse
%</discard>
%<*package>
%\fi
%
% \macro{\ifchilddoc}
% \macro{\ifchilddocmanual}
% The conditional |\ifchilddoc| tells whether a
% child (true) or main (false) document is being compiled.
% The conditional |\ifchilddocmanual| tells whether
% the |\includeonly| mechanism is used (false) or
% the selection of child files must be performed manually (true).
% The definitions initialise to false:
%    \begin{macrocode}
\newif\ifchilddoc
\newif\ifchilddocmanual
%    \end{macrocode}

% \macro{\childdocname}
% \macro{\childdocjob}
% The macro |\childdocname| stores the name of the main document
% to be compiled. The macro |\childdocjob| stores the name of
% the document on which the \LaTeX{} compiler was originally invoked.
% The content of |\jobname| cannot be compared
% to filenames specified in the source due to different catcodes.
% The following code rescans |\jobname|, stores the result
% in |\childdocname| and saves a copy in |\childdocjob|:
%    \begin{macrocode}
\edef\childdocname{\scantokens\expandafter{\jobname\noexpand}}
\let\childdocjob\childdocname
%    \end{macrocode}

% \macro{\childdocdisable}
% The macro |\childdocdisable| prevents the main file
% from being processed more than once.
% At this stage, the main document command |\childdocmain|
% is assumed to be called once again where it should do nothing.
% Any subsequent call to it should prevent
% a secondary processing of the main document
% It overwrites the forwarding commands
% |\childdocof| and |\childdocforward|
% with empty macros to prevent further inclusions of the main document:
%    \begin{macrocode}
\newcommand{\childdocdisable}
{
  \renewcommand{\childdocmain}[1]{\renewcommand{\childdocmain}[1]{\endinput}}
  \renewcommand{\childdocof}[1]{}
  \renewcommand{\childdocby}[2][]{}
  \renewcommand{\childdocforward}[2][]{}
  \renewcommand{\childdocdisable}{}
}
%    \end{macrocode}

% \macro{\childdocmain}
% The macro |\childdocmain| is to be called at the top of the main file
% with nothing or the main filename (without extension) as argument.
% First, it breaks loops.
% If the argument is not empty and does not match |\childdocname|
% (which is set by the first inclusion of |childdoc.def|),
% |\ifchilddoc| is set to true, |\includeonly| is applied to the child file
% and |\jobname| is set to the main file
% (for proper handling of |.aux| files):
%    \begin{macrocode}
\newcommand{\childdocmain}[1]
{
  \childdocdisable\childdocmain{}
  \if?#1?\else
    \begingroup
      \def\childdoctmp{#1}
      \ifx\childdoctmp\childdocname
        \def\childdoctmp{}
      \else
        \def\childdoctmp
        {
          \childdoctrue
          \includeonly{\childdocname}
          \def\childdocjob{#1}
          \def\jobname{#1}
        }
      \fi
      \expandafter
    \endgroup
    \childdoctmp
  \fi
}
%    \end{macrocode}

% \macro{\childdocof}
% The command |\childdocof| redirects
% compilation to the main file |#1|.
%    \begin{macrocode}
\newcommand{\childdocof}[1]
{
  \childdocdisable
  \childdoctrue
  \includeonly{\childdocname}
  \def\jobname{#1}
  \def\childdocjob{#1}
  \input{#1}
}
%    \end{macrocode}

% \macro{\childdocby}
% The command |\childdocby| ....
%    \begin{macrocode}
\newcommand{\childdocby}[2][]
{
  \childdocdisable
  \childdoctrue
  \childdocmanualtrue
  \if?#1?\else
    \def\jobname{#2}
  \fi
  \def\childdocjob{#2}
  \input{#2}
  \endinput
}
%    \end{macrocode}

% \macro{\childdocforward}
% The command |\childdocforward| redirects
% compilation to the main file or
% (if the optional argument is given) a child file.
% Parameters are set as if the main file
% or a child file starting with |\childdocof| was compiled.
% Then compilation is handed over to the main file:
%    \begin{macrocode}
\newcommand{\childdocforward}[2][]
{
  \begingroup
    \if?#1?
      \def\childdoctmp
      {
        \def\childdocname{#2}
        \def\childdocjob{#2}
        \def\jobname{#2}
        \input{#2}
        \endinput
      }
    \else
      \def\childdoctmp
      {
        \childdocdisable
        \def\childdocname{#2}
        \childdoctrue
        \includeonly{#2}
        \def\childdocjob{#1}
        \def\jobname{#1}
        \input{#1}
        \endinput
      }
    \fi
    \expandafter
  \endgroup
  \childdoctmp
}
%    \end{macrocode}

% \macro{\childdocforwardprefix}
% The command |\childdocforwardprefix| redirects
% compilation to the main or a child file by means of a pattern.
% The prefix |#1| in the current filename is replaced by |#2|
% and the suffix of the current filename is kept
% (it is assumed that the filename does not contain the substring `|~~~|'
% which is used as a delimiter).
% Compilation is handed over to the new file by |\childdocforward|:
%    \begin{macrocode}
\newcommand{\childdocforwardprefix}[3][]
{
  \begingroup
    \def\childdocextract #2##1~~~{\def\childdoctmp{\childdocforward[#1]{#3##1}}}
    \expandafter\childdocextract\childdocname~~~
    \expandafter
  \endgroup
  \childdoctmp
}
%    \end{macrocode}

% \macro{\childdoc}
% The deprecated macro |\childdoc| is a legacy version of |\childdocmain|:
%    \begin{macrocode}
\newcommand{\childdoc}{\childdocmain}
%    \end{macrocode}

% \macro{\childdocredirect}
% The deprecated macro |\childdocredirect| is a legacy version
% of |\childdocforward| and |\childdocforwardprefix|:
%    \begin{macrocode}
\newcommand{\childdocredirect}[2][]
{
  \begingroup
    \if?#1?
      \def\childdoctmp{\childdocforward{#2}}
    \else
      \def\childdoctmp{\childdocforwardprefix{#1}{#2}}
    \fi
    \expandafter
  \endgroup
  \childdoctmp
}
%    \end{macrocode}

%\iffalse
%</package>
%\fi
%
\endinput
\childdocforward[cdocsamp]{cdocsch1}"|\\
% |latex -jobname cdocscl2 \|\\
% |  "\def\version{final}% \iffalse
%
% childdoc.dtx Copyright (C) 2017-2018 Niklas Beisert
%
% This work may be distributed and/or modified under the
% conditions of the LaTeX Project Public License, either version 1.3
% of this license or (at your option) any later version.
% The latest version of this license is in
%   http://www.latex-project.org/lppl.txt
% and version 1.3 or later is part of all distributions of LaTeX
% version 2005/12/01 or later.
%
% This work has the LPPL maintenance status `maintained'.
%
% The Current Maintainer of this work is Niklas Beisert.
%
% This work consists of the files childdoc.dtx and childdoc.ins
% and the derived files childdoc.def and cdocsamp.tex with
% cdocsch1.tex, cdocsch2.tex, cdocsdrf.tex, cdocsfn1.tex, cdocsfn2.tex.
%
%<package>\ifdefined\childdocmain\endinput\fi
%<package>\ProvidesFile{childdoc.def}[2018/12/30 v2.0 child document driver]
%<samplemain>\ProvidesFile{cdocsamp.tex}[2018/12/30 v2.0 sample for childdoc]
%<*driver>
%\ProvidesFile{childdoc.drv}[2018/12/30 v2.0 childdoc reference manual file]
\PassOptionsToClass{10pt,a4paper}{article}
\documentclass{ltxdoc}

\usepackage[margin=35mm]{geometry}
\usepackage{hyperref}
\usepackage{hyperxmp}
\usepackage[usenames]{color}

\hypersetup{colorlinks=true}
\hypersetup{pdfstartview=FitH}
\hypersetup{pdfpagemode=UseNone}
\hypersetup{pdfsource={}}
\hypersetup{pdflang={en-UK}}
\hypersetup{pdfcopyright={Copyright 2017-2018 Niklas Beisert.
  This work may be distributed and/or modified under the
  conditions of the LaTeX Project Public License, either version 1.3
  of this license or (at your option) any later version.}}
\hypersetup{pdflicenseurl={http://www.latex-project.org/lppl.txt}}
\hypersetup{pdfcontactaddress={ETH Zurich, ITP, HIT K,
  Wolfgang-Pauli-Strasse 27}}
\hypersetup{pdfcontactpostcode={8093}}
\hypersetup{pdfcontactcity={Zurich}}
\hypersetup{pdfcontactcountry={Switzerland}}
\hypersetup{pdfcontactemail={nbeisert@itp.phys.ethz.ch}}
\hypersetup{pdfcontacturl={http://people.phys.ethz.ch/\xmptilde nbeisert/}}

\newcommand{\secref}[1]{\hyperref[#1]{section \ref*{#1}}}

\parskip1ex
\parindent0pt
\let\olditemize\itemize
\def\itemize{\olditemize\parskip0pt}

\begin{document}

\title{The \textsf{childdoc} Package}
\hypersetup{pdftitle={The childdoc Package}}
\author{Niklas Beisert\\[2ex]
  Institut f\"ur Theoretische Physik\\
  Eidgen\"ossische Technische Hochschule Z\"urich\\
  Wolfgang-Pauli-Strasse 27, 8093 Z\"urich, Switzerland\\[1ex]
  \href{mailto:nbeisert@itp.phys.ethz.ch}
  {\texttt{nbeisert@itp.phys.ethz.ch}}}
\hypersetup{pdfauthor={Niklas Beisert}}
\hypersetup{pdfsubject={Manual for the LaTeX2e Package childdoc}}
\date{30 December 2018, \textsf{v2.0}}
\maketitle

\begin{abstract}\noindent
\textsf{childdoc} is a \LaTeXe{} package
that enables the direct compilation
of document sections included by |\include|
to individual files.
\end{abstract}

\begingroup
\parskip0ex
\tableofcontents
\endgroup

%%%%%%%%%%%%%%%%%%%%%%%%%%%%%%%%%%%%%%%%%%%%%%%%%%%%%%%%%%%%%%%%%%%%%%%%%%%%%%%%
%%%%%%%%%%%%%%%%%%%%%%%%%%%%%%%%%%%%%%%%%%%%%%%%%%%%%%%%%%%%%%%%%%%%%%%%%%%%%%%%
\section{Introduction}

\LaTeX{} provides a mechanism to structure a large document (such as a book)
into a main file and several child files (containing the chapters)
using the |\include| command.
This mechanism is beneficial for documents
which span hundreds of pages in order to
make the source file(s) more manageable.
Moreover, compilation can be restricted to
selected child files by means of the |\includeonly| command.
The latter feature can be used to reduce the compilation time while editing
(this was significantly more useful in the earlier days of \LaTeX{})
or to generate a smaller document which is easier to navigate.
Another application of |\includeonly| is to generate
documents consisting of selected parts of the complete document.

However, there are a few drawbacks of the plain |\include| mechanism:
\begin{itemize}
\item
The child files cannot be compiled on their own,
they can only be compiled via the main file.
A naive editing environment
(such as a text editor with an option
to have the current file processed by \LaTeX)
may require one to switch to the main file before compiling;
attempting to compile the child file produces errors.
\item
The main file must be modified (each time)
to adjust the |\includeonly| command
to the present needs. This easily leaves the main file in a messy state.
\item
The generated document will always carry the filename
of the main document. This is inconvenient if
several child files are to be compiled and
to be kept for distribution.
\end{itemize}

The present package provides a simple interface
to make child files individually compilable by \LaTeX{}.
Compiling a child file then has the same effect as compiling
the main file with an |\includeonly| command
to select the appropriate child.
Moreover the generated document will carry the name of the child
rather than the main file.
This resolves all three above issues.

This feature is meant to make the editing of books,
thesis documents and lecture notes somewhat more convenient.
However, the package can also be used efficiently for
composing a series of documents (such as exercise sheets)
which are typically distributed individually.
It then assists the author in generating the individual documents
(potentially in different versions)
as well as a document containing the collected series.
Another application is in developing style files
or other kinds of included material
where compilation of the style file could redirect
to a sample or test file.

%%%%%%%%%%%%%%%%%%%%%%%%%%%%%%%%%%%%%%%%%%%%%%%%%%%%%%%%%%%%%%%%%%%%%%%%%%%%%%%%
%%%%%%%%%%%%%%%%%%%%%%%%%%%%%%%%%%%%%%%%%%%%%%%%%%%%%%%%%%%%%%%%%%%%%%%%%%%%%%%%
\section{Usage}

First of all, the package \textsf{childdoc} is \emph{not} a standard
\LaTeXe{} |.sty| style file! Therefore it needs to be invoked in
a non-standard way.

%%%%%%%%%%%%%%%%%%%%%%%%%%%%%%%%%%%%%%%%%%%%%%%%%%%%%%%%%%%%%%%%%%%%%%%%%%%%%%%%
\subsection{Included Files}
\label{sec:include}

%%%%%%%%%%%%%%%%%%%%%%%%%%%%%%%%%%%%%%%%
\DescribeMacro{\childdocmain}
To use the package, add the commands
\begin{center}
\begin{tabular}{l}
|\input{childdoc.def}|\\
|\childdocmain{}|\\
\end{tabular}
\end{center}
at the very top of the main \LaTeX{} file,
in particular \emph{before} the |\documentclass| statement!
The argument of |\childdocmain| should be left empty
(but it must be present).

%%%%%%%%%%%%%%%%%%%%%%%%%%%%%%%%%%%%%%%%
\DescribeMacro{\childdocof}
Furthermore, add the commands
\begin{center}
\begin{tabular}{l}
|\input{childdoc.def}|\\
|\childdocof{|\textit{main}|}|\\
\end{tabular}
\end{center}
at the top of every child file \textit{child}
which is included by |\include{|\textit{child}|}|
from within the main file
(or at least for those files to be compiled individually).
The argument \textit{main} must be the filename of the main file.

There are a couple of
considerations in setting up the main and child documents:

%%%%%%%%%%%%%%%%%%%%%%%%%%%%%%%%%%%%%%%%
\paragraph{Restrictions.}

Please note the following restrictions:
\begin{itemize}
\item
|\childdocmain| must be called with one argument \textit{main}
to ensure compatibility with earlier version of the package.
It must either be empty (|\childdocmain{}|)
or precisely match the filename of the main file in which it is specified.
See \secref{sec:detection} for further information.
\item
The filename \textit{main} must be specified without the |.tex| extension.
\item
The filename \textit{main} is case sensitive
(even in case-insensitive file systems)
due to internal string comparison.
\item
The argument \textit{main} should be fully expanded, it cannot be a macro.
\item
Subdirectories and special characters should be avoided in filenames.
\item
The command |\childdocmain{|\textit{main}|}| must be followed by a whitespace.
It should not be followed immediately by another command
or by a comment mark `|%|'.
This is because the \TeX{} parser reads the token immediately following
the argument of |\childdocmain| and puts it
at the beginning of every child section;
however, a white\-space is ignored.
\end{itemize}

%%%%%%%%%%%%%%%%%%%%%%%%%%%%%%%%%%%%%%%%
\paragraph{Content of Main File.}

It is advisable to place all content in the child files included by |\include|.
Any output contained in the main file will appear in all child documents
unless suppressed manually;
it cannot be suppressed automatically by the |\includeonly| directive
and thus should normally be avoided.
A method to include some content in the main file
by means of conditional processing is described in \secref{sec:conditional}.

%%%%%%%%%%%%%%%%%%%%%%%%%%%%%%%%%%%%%%%%
\paragraph{Page Numbering.}

When only a part of the document is compiled,
the appropriate numbering of pages
(as well as other status parameters)
is determined from the |.aux| files.
The latter contain information from previous passes.
However this information needs to propagate through
all intermediate child documents.
Therefore the page numbering in child documents may well
be inconsistent until the complete document is compiled at least once.

A useful (if unconventional) way to always ensure a consistent
page numbering is to restart the numbering in each child document
and denote the pages by `\textit{child}|.|\textit{page}'
where \textit{child} represents the chapter/section number of the child file.
This can be achieved by the command
|\numberwithin{page}{|\textit{child}|}|
of the \textsf{amsmath} package
where \textit{child} can be |chapter| or |section|
depending on the chosen structuring.
Alternatively, one can modify the macro |\thepage| appropriately
and reset the counter |page| at the start of each child file.

%%%%%%%%%%%%%%%%%%%%%%%%%%%%%%%%%%%%%%%%%%%%%%%%%%%%%%%%%%%%%%%%%%%%%%%%%%%%%%%%
\subsection{Conditional Processing}
\label{sec:conditional}

The package provides a mechanism to compile different versions
of a document. To customise the versions further some conditional processing
can come in handy to distinguish which version is being compiled.
The package provides two macros to describe the compilation context:

%%%%%%%%%%%%%%%%%%%%%%%%%%%%%%%%%%%%%%%%
\DescribeMacro{\ifchilddoc}
The conditional |\ifchilddoc| distinguishes between the compilation of
child documents and the main document:
%
\begin{center}
|\ifchilddoc |\textit{child-code}| |[|\||else |\textit{main-code}]| \||fi|
\end{center}

%%%%%%%%%%%%%%%%%%%%%%%%%%%%%%%%%%%%%%%%
\DescribeMacro{\childdocname}
\DescribeMacro{\childdocjob}
The macro |\childdocname| contains the filename (without extension)
of the main or child file being processed.
Note that |\childdocjob| will always contain the name of the main file.

%%%%%%%%%%%%%%%%%%%%%%%%%%%%%%%%%%%%%%%%
\paragraph{Title Page.}

Conditional processing can be used to include a title or banner page
in the main document when proper precautions are taken.
Importantly, the code in the main file should ensure that the page counter
(as well as other status parameters which are stored in the |.aux| files)
takes the same value after the conditional processing.
Otherwise the page numbers may take divergent values
depending on which part is compiled.

For example, a title page could be declared by:
%
\begin{center}
\begin{tabular}{l}
|\ifchilddoc\||else|\\
|\addtocounter{page}{-1}|\\
\textit{code for title page}\\
|\newpage|\\
|\||fi|
\end{tabular}
\end{center}
%
A banner page for the child documents can be generated by:
%
\begin{center}
\begin{tabular}{l}
|\ifchilddoc|\\
|\addtocounter{page}{-1}|\\
\textit{code for banner page}\\
|\newpage|\\
|\||fi|
\end{tabular}
\end{center}
%
Here one could write a message such as:
\begin{center}
|This is the part \childdocname{} of \childdocjob{}.|
\end{center}

%%%%%%%%%%%%%%%%%%%%%%%%%%%%%%%%%%%%%%%%%%%%%%%%%%%%%%%%%%%%%%%%%%%%%%%%%%%%%%%%
\subsection{Flags}
\label{sec:flags}

The package makes it easy to generate different versions
of the main or child documents.
To this end compilation flags can be defined
and assigned different default values.
They will be particularly useful in conjunction
with the forwarding mechanism described in \secref{sec:forward}.

For example, it may be useful to have a flag |\version|
which can be set to |draft| or |final|.
The document source will contain some conditional code
depending on the value of |\version|.
Suppose further, the flag should default to |final| for the main file
and to |draft| for child files
which is a natural assignment for editing the document.
This is achieved by placing the following code
in the preamble of the main document
(below the |\childdocmain| directive):
%
\begin{center}
\begin{tabular}{l}
|\ifchilddoc|\\
|\providecommand{\version}{draft}|\\
|\||else|\\
|\providecommand{\version}{final}|\\
|\||fi|
\end{tabular}
\end{center}
%
The definition by |\providecommand| makes sure
that previous definitions are not overwritten.
Further statements |\providecommand{\version}{...}|
can thus be added before the above code to override it.

For the main file, one might add a line
(between |\childdocmain| and the above block)
%
\begin{center}
|%\ifchilddoc\||else\providecommand{\version}{draft}\||fi|
\end{center}
%
which can be uncommented to produce a draft version.
Likewise one can add a line to the very top of a child file
(above the |\childdocof{|\textit{main}|}| directive)
%
\begin{center}
|%\providecommand{\version}{final}|
\end{center}
%
which can be uncommented to produce the final version of this child document.

%%%%%%%%%%%%%%%%%%%%%%%%%%%%%%%%%%%%%%%%%%%%%%%%%%%%%%%%%%%%%%%%%%%%%%%%%%%%%%%%
\subsection{Forwarding}
\label{sec:forward}

Different versions of the main or child documents
using compilation flags as described in \secref{sec:flags}
can be (permanently) stored in different files
for convenient compilation, viewing and distribution.
To this end, the package defines a command
to pass on compilation to a different file:

%%%%%%%%%%%%%%%%%%%%%%%%%%%%%%%%%%%%%%%%
\DescribeMacro{\childdocforward}
The command |\childdocforward| redirects processing to
another source file:
%
\begin{center}
\begin{tabular}{l}
|\input{childdoc.def}|\\
|\childdocforward[|\textit{main}|]{|\textit{dest}|}|\\
\end{tabular}
\end{center}
%
The argument \textit{dest} is the destination file
(without extension).
It should be the main file or one of the child files.
Note that further \textsf{childdoc} directives
such as |\childdocof| and |\childdocforward|
in the indicated file will be processed in this form.
The optional argument \textit{main}
passes on directly to the main file \textit{main}
while pretending to compile the child \textit{dest}.
This form behaves as if \textit{dest}
issues |\childdocof{|\textit{main}|}| right away,
and no further \textsf{childdoc} directives will be processed.

%%%%%%%%%%%%%%%%%%%%%%%%%%%%%%%%%%%%%%%%
\DescribeMacro{\...prefix}
In the alternative form |\childdocforwardprefix|,
%
\begin{center}
\begin{tabular}{l}
|\input{childdoc.def}|\\
|\childdocforwardprefix[|\textit{main}|]{|\textit{prefix}|}{|\textit{dest}|}|
\end{tabular}
\end{center}
%
the destination file is determined by a pattern
depending on the current file:
To make this work, the current file must be called
`{\textit{prefix}\hspace{0.2em}\textit{suffix}}'
with \textit{prefix} matching precisely the argument.
Processing is then passed on to the file
`{\textit{dest}\hspace{0.2em}\textit{suffix}}'.
Surely, the same effect is achieved by
directly specifying the
argument `{\textit{dest}\hspace{0.2em}\textit{suffix}}'
in the first form.
However, that requires to set up a different file
for each child. With the alternative form of the command
all these files can have exactly the same content
which simplifies setting them up and maintaining them.

For example, the following file |draft.tex|
with a compilation flag |\version| as described in \secref{sec:flags}
compiles the main document as a draft:
%
\begin{center}
\begin{tabular}{l}
|\def\version{draft}|\\
|\input{childdoc.def}|\\
|\childdocforward{|\textit{main}|}|
\end{tabular}
\end{center}
%
Likewise, the following files |final|\textit{nn}|.tex|
compile the final version of the child document
|child|\textit{nn}|.tex|:
%
\begin{center}
\begin{tabular}{l}
|\def\version{final}|\\
|\input{childdoc.def}|\\
|\childdocforwardprefix{final}{child}|
\end{tabular}
\end{center}
%

Note that when several versions of a main file and/or of each child file
are to be generated, it may be convenient to set up a |Makefile| or
shell script to automatise the process.

%%%%%%%%%%%%%%%%%%%%%%%%%%%%%%%%%%%%%%%%%%%%%%%%%%%%%%%%%%%%%%%%%%%%%%%%%%%%%%%%
\subsection{Command Line Processing}
\label{sec:commandline}

The effect of redirection files can also be achieved by invoking
the \LaTeX{} compiler with a more elaborate command line.
Most conveniently this should be done as part
of a shell script or a |Makefile|.

When using \textsf{childdoc} in the main file, the following
command lines effectively perform a redirection
(note that depending on the shell being used,
backslashes may have to be doubled: `|\|' $\to$ `|\\|'):
%
\begin{center}
|... -jobname "|\textit{target}|" |\\|"|[\textit{flags}]%
|\input{childdoc.def}\childdocforward[|\textit{main}|]{|\textit{dest}|}"|
\end{center}
%
Here \textit{target} is the name of the output file,
\textit{main} is the name of the main file
and \textit{dest} is the name of the main or child file to be processed
(all filenames without extensions).
The optional argument \textit{main} can be omitted
if \textit{main} matches \textit{dest}.
Optionally, compilation \textit{flags} can be defined via |\def| commands.
This command line makes the \TeX{} engine believe
it is compiling the file \textit{target}
whose content is specified as the latter parameter.
The provided code then forwards the processing to
\textit{main} or \textit{dest} as described in \secref{sec:forward}.

%%%%%%%%%%%%%%%%%%%%%%%%%%%%%%%%%%%%%%%%%%%%%%%%%%%%%%%%%%%%%%%%%%%%%%%%%%%%%%%%
\subsection{Include by Input}
\label{sec:input}

Including child documents by |\include| has some restrictions by design.
Most notably, the content of a child document always occupies
its own set of pages; pages cannot be shared between child documents.
Usually, this behaviour makes perfect sense
because each child document contain an essential part of the document.
However, in some situations it may be desirable to compose
a document from a collection of parts
without having mandatory page breaks between then.
For this case, the package
provides a mechanism to include parts
by |\input| which can also be processed individually.
However, by construction this mechanism
requires manual handling of the content to be output.

%%%%%%%%%%%%%%%%%%%%%%%%%%%%%%%%%%%%%%%%
\DescribeMacro{\ifchilddocmanual}
The main file should be prepared as usual, see \secref{sec:include}.
However, the document body must make a distinction
between processing of an individual part and of the main document, e.g.:
%
\begin{center}
\begin{tabular}{l}
|\ifchilddocmanual|\\
|\input{\childdocname}|\\
|\||else|\\
\textit{document body with }|\input{|\textit{part}|}|\\
|\||fi|
\end{tabular}
\end{center}
%
The conditional |\ifchilddocmanual| is true whenever
a part to be included by |\input| is being compiled,
and the name of the part is stored in |\childdocname|.

%%%%%%%%%%%%%%%%%%%%%%%%%%%%%%%%%%%%%%%%
\DescribeMacro{\childdocby}
Each part to be included by |\input| should start with:
%
\begin{center}
\begin{tabular}{l}
|\input{childdoc.def}|\\
|\childdocby{|\textit{main}|}|\\
\end{tabular}
\end{center}
%
The directive |\childdocby| is similar to |\childdocof|
described in \secref{sec:include},
but the subsequent selection of content must be done manually.
To that end, both |\ifchilddoc| and |\ifchilddocmanual|
will be true upon processing of a part,
and the name of the part is stored in |\childdocname|.
Note that |\jobname| will be set to the filename of the current part
so that each part receives an individual |.aux| file
that does not interfere with the |.aux| file(s) of the main document.
This behaviour can be altered by the alternative form
|\childdocby[*]{|\textit{main}|}| (with a non-empty optional argument)
which uses the |.aux| file of the main document
by setting |\jobname| to \textit{main}.

%%%%%%%%%%%%%%%%%%%%%%%%%%%%%%%%%%%%%%%%%%%%%%%%%%%%%%%%%%%%%%%%%%%%%%%%%%%%%%%%
\subsection{Driver Development}
\label{sec:driver}

The \textsf{childdoc} mechanism can also be use for the development
of definition files such as \LaTeX{} styles or classes.
This case differs from the above setup with multiple parts
included by |\include| in that no |\includeonly| should be invoked.
This can be achieved by starting the include file
(before |\ProvidesPackage|) with:
%
\begin{center}
\begin{tabular}{l}
|\input{childdoc.def}|\\
|\childdocforward{|\textit{main}|}|\\
\end{tabular}
\end{center}
%
or alternatively with:
%
\begin{center}
\begin{tabular}{l}
|\input{childdoc.def}|\\
|\childdocby{|\textit{main}|}|\\
\end{tabular}
\end{center}
%
Both forms have slightly different effects as described above.
The main file is prepared as usual, see \secref{sec:include}.

%%%%%%%%%%%%%%%%%%%%%%%%%%%%%%%%%%%%%%%%%%%%%%%%%%%%%%%%%%%%%%%%%%%%%%%%%%%%%%%%
\subsection{Legacy Detection}
\label{sec:detection}

The directive |\childdocmain| in the main file can detect
whether the complete document or merely a child is to be compiled
even without using the directive |\childdocof|.
This method is deprecated because it is less robust
and there is no compelling reason to use it;
it is merely provided for backward compatibility
and it may be removed in future versions.

If the detection mechanism is to be used,
it is mandatory to correctly specify
the filename of the main file as the argument of |\childdocmain|:
%
\begin{center}
\begin{tabular}{l}
|\input{childdoc.def}|\\
|\childdocmain{|\textit{main}|}|\\
\end{tabular}
\end{center}
%
If |\jobname| does not match the argument \textit{main} of |\childdocmain|,
it is assumed that |\jobname| points to the child file to be compiled.
When using |\childdocmain| with the main file specified as argument,
it suffices to start a child file
with just |\input{|\textit{main}|}|
without loading of the package and using |\childdocof|.
If instead all processing is done
with the appropriate \textsf{childdoc} directives,
the argument of \textit{main} of |\childdocmain| can be empty.

An alternative version of the command line processing described
in \secref{sec:commandline} using the detection mechanism reads:
%
\begin{center}
|... -jobname "|\textit{target}|" "|[\textit{flags}]%
[|\def\jobname{|\textit{dest}|}|]|\input{|\textit{main}|}"|
\end{center}

%%%%%%%%%%%%%%%%%%%%%%%%%%%%%%%%%%%%%%%%%%%%%%%%%%%%%%%%%%%%%%%%%%%%%%%%%%%%%%%%
\subsection{Manual Code}
\label{sec:manual}

In case one cannot be certain whether the definitions file |childdoc.def|
is installed on the target \TeX{} distribution
and one prefers not to ship it,
it is conceivable to paste a few relevant commands into the sources.

To that end, drop all statements |\input{childdoc.def}|
and perform the replacements as outlined below.
Instead of |\childdocmain{|\textit{main}|}| add the following code
to the top of the main file:
%
\begin{center}
\begin{tabular}{l}
|\||ifdefined\childdocname\endinput\||fi\newif\ifchilddoc|\\
|\edef\childdocname{\scantokens\expandafter{\jobname\noexpand}}|\\
|\def\childdocmain{|\textit{main}|}\||ifx\childdocmain\childdocname\||else|\\
|\childdoctrue\includeonly{\childdocname}\let\jobname\childdocmain\||fi|\\
\end{tabular}
\end{center}
%
Instead of |\childdocof{|\textit{main}|}| just include the main file
at the top of each child file:
%
\begin{center}
|\input{|\textit{main}|}|
\end{center}
%
A simple redirection |\childdocforward{|\textit{dest}|}| is achieved by:
%
\begin{center}
|\def\jobname{|\textit{dest}|}\input{\jobname}|
\end{center}
%
The redirection with prefix
|\childdocforwardprefix[|\textit{prefix}|]{|\textit{dest}|}|
is accomplished by:
%
\begin{center}
\begin{tabular}{l}
|{\edef\jobname{\scantokens\expandafter{\jobname\noexpand}}|\\
|\def\redirectjob |\textit{prefix}|#1~~~{\gdef\jobname{|\textit{dest}|#1}}|\\
|\expandafter\redirectjob\jobname~~~}\input{\jobname}|
\end{tabular}
\end{center}

In an alternative approach,
child documents can be compiled by a specific command line
without additional code or specific definitions:
%
\begin{center}
|... -jobname "|\textit{target}|" "|[\textit{flags}]%
|\includeonly{|\textit{dest}|}\input{|\textit{main}|}"|
\end{center}
%

%%%%%%%%%%%%%%%%%%%%%%%%%%%%%%%%%%%%%%%%%%%%%%%%%%%%%%%%%%%%%%%%%%%%%%%%%%%%%%%%
%%%%%%%%%%%%%%%%%%%%%%%%%%%%%%%%%%%%%%%%%%%%%%%%%%%%%%%%%%%%%%%%%%%%%%%%%%%%%%%%
\section{Information}

%%%%%%%%%%%%%%%%%%%%%%%%%%%%%%%%%%%%%%%%%%%%%%%%%%%%%%%%%%%%%%%%%%%%%%%%%%%%%%%%
\subsection{Copyright}

Copyright \copyright{} 2017--2018 Niklas Beisert

This work may be distributed and/or modified under the
conditions of the \LaTeX{} Project Public License, either version 1.3
of this license or (at your option) any later version.
The latest version of this license is in
  \url{http://www.latex-project.org/lppl.txt}
and version 1.3 or later is part of all distributions of \LaTeX{}
version 2005/12/01 or later.

This work has the LPPL maintenance status `maintained'.

The Current Maintainer of this work is Niklas Beisert.

This work consists of the files |README.txt|, |childdoc.ins| and |childdoc.dtx|
as well as the derived files |childdoc.def|, |cdocsamp.tex|
with |cdocsch1.tex|, |cdocsch2.tex|, |cdocspt3.tex|, |cdocspt4.tex|,
|cdocsdrf.tex|, |cdocsfn1.tex|, |cdocsfn2.tex|
as well as |childdoc.pdf|.

%%%%%%%%%%%%%%%%%%%%%%%%%%%%%%%%%%%%%%%%%%%%%%%%%%%%%%%%%%%%%%%%%%%%%%%%%%%%%%%%
\subsection{Files and Installation}

The package consists of the files:
%
\begin{center}
\begin{tabular}{ll}
    |README.txt|   & readme file \\
    |childdoc.ins| & installation file \\
    |childdoc.dtx| & source file \\
    |childdoc.def| & definition file \\
    |cdocsamp.tex| & sample main file \\
    |cdocsch1.tex| & sample include file \\
    |cdocsch2.tex| & sample include file \\
    |cdocspt3.tex| & sample part file \\
    |cdocspt4.tex| & sample part file \\
    |cdocsdrf.tex| & sample redirection file \\
    |cdocsfn1.tex| & sample redirection file \\
    |cdocsfn2.tex| & sample redirection file \\
    |childdoc.pdf| & manual
\end{tabular}
\end{center}
%
The distribution consists of the files
|README.txt|, |childdoc.ins| and |childdoc.dtx|.
%
\begin{itemize}
\item
Run (pdf)\LaTeX{} on |childdoc.dtx|
to compile the manual |childdoc.pdf| (this file).
\item
Run \LaTeX{} on |childdoc.ins| to create the definitions file |childdoc.def|
and the sample |cdocsamp.tex| with include files
|cdocsch1.tex|, |cdocsch2.tex|, |cdocspt3.tex|, |cdocspt4.tex|,
|cdocsdrf.tex|, |cdocsfn1.tex|, |cdocsfn2.tex|.
Then copy the file |childdoc.def| to an appropriate directory of your \LaTeX{}
distribution, e.g.\ \textit{texmf-root}|/tex/latex/childdoc|.
\end{itemize}

%%%%%%%%%%%%%%%%%%%%%%%%%%%%%%%%%%%%%%%%%%%%%%%%%%%%%%%%%%%%%%%%%%%%%%%%%%%%%%%%
\subsection{Related CTAN Packages}

There are several other packages which offer a similar functionality:
%
\begin{itemize}
\item
The packages
\href{http://ctan.org/pkg/docmute}{\textsf{docmute}},
\href{http://ctan.org/pkg/includex}{\textsf{includex}} and
\href{http://ctan.org/pkg/standalone}{\textsf{standalone}}
provide commands to include only the document body of
a child file thus allowing both files to be compiled individually.
\item
The packages \href{http://ctan.org/pkg/subdocs}{\textsf{subdocs}}
and \href{http://ctan.org/pkg/subfiles}{\textsf{subfiles}}
provide structures in which the main and child documents can be
encapsulated and allowing them to be compiled individually.
The inclusion mechanism is different from the conventional |\include|.
\item
The package \href{http://ctan.org/pkg/combine}{\textsf{combine}}
is an elaborate solution to combine several documents into one.
\end{itemize}
%
See also the CTAN topic \href{http://ctan.org/topic/subdocs}{\textsf{subdocs}}
for further related packages.
The present package differs from the above solutions in that
a document structure constructed with the conventional |\include| mechanism
just needs two extra commands at the top of every file
such that all constituent files can be compiled individually.

%%%%%%%%%%%%%%%%%%%%%%%%%%%%%%%%%%%%%%%%%%%%%%%%%%%%%%%%%%%%%%%%%%%%%%%%%%%%%%%%
%\subsection{Feature Suggestions}
%
%The following is a list of features which may be useful for future
%versions of this package:
%%
%\begin{itemize}
%\item
%\ldots
%\end{itemize}

%%%%%%%%%%%%%%%%%%%%%%%%%%%%%%%%%%%%%%%%%%%%%%%%%%%%%%%%%%%%%%%%%%%%%%%%%%%%%%%%
\subsection{Revision History}

%%%%%%%%%%%%%%%%%%%%%%%%%%%%%%%%%%%%%%%%
\paragraph{v2.0:} 2018/12/30

\begin{itemize}
\item
immediate forward processing
\item
added |\childdocby| mechanism
\item
manual restructured
\end{itemize}

%%%%%%%%%%%%%%%%%%%%%%%%%%%%%%%%%%%%%%%%
\paragraph{v1.6:} 2018/01/17

\begin{itemize}
\item
application for development of include files
\item
corrections to manual
\end{itemize}

%%%%%%%%%%%%%%%%%%%%%%%%%%%%%%%%%%%%%%%%
\paragraph{v1.5:} 2017/05/21

\begin{itemize}
\item
more complete structuring introduced
\item
|\childdocof| introduced
\item
|\childdoc| renamed to |\childdocmain|
\item
|\childredirect| renamed to |\childdocforward| and |\childdocforwardprefix|
and functionality expanded
\end{itemize}

%%%%%%%%%%%%%%%%%%%%%%%%%%%%%%%%%%%%%%%%
\paragraph{v1.0:} 2017/04/27

\begin{itemize}
\item
manual and install package
\item
first version published on CTAN
\end{itemize}

%%%%%%%%%%%%%%%%%%%%%%%%%%%%%%%%%%%%%%%%
\paragraph{v0.6:} 2017/04/26

\begin{itemize}
\item
redirection mechanism added
\end{itemize}

%%%%%%%%%%%%%%%%%%%%%%%%%%%%%%%%%%%%%%%%
\paragraph{v0.5:} 2017/04/26

\begin{itemize}
\item
functionality in definition file
\end{itemize}


%%%%%%%%%%%%%%%%%%%%%%%%%%%%%%%%%%%%%%%%%%%%%%%%%%%%%%%%%%%%%%%%%%%%%%%%%%%%%%%%
%%%%%%%%%%%%%%%%%%%%%%%%%%%%%%%%%%%%%%%%%%%%%%%%%%%%%%%%%%%%%%%%%%%%%%%%%%%%%%%%
%%%%%%%%%%%%%%%%%%%%%%%%%%%%%%%%%%%%%%%%%%%%%%%%%%%%%%%%%%%%%%%%%%%%%%%%%%%%%%%%
\appendix

\settowidth\MacroIndent{\rmfamily\scriptsize 000\ }

 \DocInput{childdoc.dtx}

\end{document}
%</driver>
% \fi
%
% %%%%%%%%%%%%%%%%%%%%%%%%%%%%%%%%%%%%%%%%%%%%%%%%%%%%%%%%%%%%%%%%%%%%%%%%%%%%%%
% %%%%%%%%%%%%%%%%%%%%%%%%%%%%%%%%%%%%%%%%%%%%%%%%%%%%%%%%%%%%%%%%%%%%%%%%%%%%%%
% \section{Sample}
%\iffalse
%<*samplemain>
%\fi
%
% The following presents a sample document
% with two chapters, two parts, a title page,
% a compile flag as well as three forwarding files to set the flag.
% It consists of eight |.tex| files:
% \begin{center}
% \begin{tabular}{ll}
% |cdocsamp.tex|&main file\\
% |cdocsch1.tex|&include file for chapter 1\\
% |cdocsch2.tex|&include file for chapter 2\\
% |cdocspt3.tex|&include file for part 3\\
% |cdocspt4.tex|&include file for part 4\\
% |cdocsdrf.tex|&forwarding file for main file in draft mode\\
% |cdocsfi1.tex|&forwarding file for final version of chapter 1\\
% |cdocsfi2.tex|&forwarding file for final version of chapter 2\\
% \end{tabular}
% \end{center}
% Each of the eight files can be compiled directly by the \LaTeX{} compiler.
%
% %%%%%%%%%%%%%%%%%%%%%%%%%%%%%%%%%%%%%%
% \paragraph{Main File.}
%
% The main file is called |cdocsamp.tex|.
%
% Load the \textsf{childdoc} definitions and
% declare the filename for the main document:
%    \begin{macrocode}
\input{childdoc.def}
\childdocmain{}
%    \end{macrocode}

% Optional override for |\version| flag:
%    \begin{macrocode}
%%\ifchilddoc\else\providecommand{\version}{draft}\fi
%    \end{macrocode}

% Define the default values for the |\version| flag
% (|final| for the main file and |draft| for childs):
%    \begin{macrocode}
\ifchilddoc
\providecommand{\version}{draft}
\else
\providecommand{\version}{final}
\fi
%    \end{macrocode}

% Load the standard document class:
%    \begin{macrocode}
\documentclass[12pt]{article}
%    \end{macrocode}

% Start the document body:
%    \begin{macrocode}
\begin{document}
%    \end{macrocode}

% Declare a title page.
% Print title, part of document being processed and version flag:
%    \begin{macrocode}
\addtocounter{page}{-1}
\begin{center}
{\LARGE\bfseries{}childdoc example\par}
\vspace{1cm}
\ifchilddoc
\ifchilddocmanual part\else chapter\fi:
`\childdocname' of `\childdocjob'\par
\else
main document: `\childdocjob'\par
\fi
version: \version\par
\end{center}
\newpage
%    \end{macrocode}

% Manually include selected file,
% otherwise process as usual:
%    \begin{macrocode}
\ifchilddocmanual
\section*{part `\childdocname'}
\input{\childdocname}
\else
%    \end{macrocode}

% Include the two chapters:
%    \begin{macrocode}
\include{cdocsch1}
\include{cdocsch2}
%    \end{macrocode}

% Include the two parts unless only chapters should be displayed:
%    \begin{macrocode}
\ifchilddoc\else
\section{part three}
\input{cdocspt3}
\section{part four}
\input{cdocspt4}
\fi
%    \end{macrocode}

% Process as usual until here:
%    \begin{macrocode}
\fi
%    \end{macrocode}

% End of document body:
%    \begin{macrocode}
\end{document}
%    \end{macrocode}
%\iffalse
%</samplemain>
%\fi
%
% %%%%%%%%%%%%%%%%%%%%%%%%%%%%%%%%%%%%%%
% \paragraph{Chapter Include Files.}
%
% The include files are called |cdocsch1.tex| and |cdocsch2.tex|.
%
%\iffalse
%<*samplechap1|samplechap2>
%\fi

% Optional override for |\version| flag:
%    \begin{macrocode}
%%\providecommand{\version}{final}
%    \end{macrocode}

% Include the main document:
%    \begin{macrocode}
\input{childdoc.def}
\childdocof{cdocsamp}
%    \end{macrocode}

%\iffalse
%</samplechap1|samplechap2>
%\fi
%
%\iffalse
%<*samplechap1>
%\fi
% Some text for chapter 1:
%    \begin{macrocode}
\section{one}
some text in chapter one
%    \end{macrocode}

%\iffalse
%</samplechap1>
%\fi
% Some text for chapter 2:
%\iffalse
%<*samplechap2>
%\fi
%    \begin{macrocode}
\section{two}
more text in chapter two
%    \end{macrocode}

%\iffalse
%</samplechap2>
%\fi
%
% %%%%%%%%%%%%%%%%%%%%%%%%%%%%%%%%%%%%%%
% \paragraph{Part Include Files.}
%
% The include files are called |cdocspt3.tex| and |cdocspt4.tex|.
%
%\iffalse
%<*samplepart3|samplepart4>
%\fi

% Optional override for |\version| flag:
%    \begin{macrocode}
%%\providecommand{\version}{final}
%    \end{macrocode}

% Include the main document:
%    \begin{macrocode}
\input{childdoc.def}
\childdocby{cdocsamp}
%    \end{macrocode}

%\iffalse
%</samplepart3|samplepart4>
%\fi
%
%\iffalse
%<*samplepart3>
%\fi
% Some text for part 3:
%    \begin{macrocode}
some text in part three
%    \end{macrocode}

%\iffalse
%</samplepart3>
%\fi
% Some text for part 4:
%\iffalse
%<*samplepart4>
%\fi
%    \begin{macrocode}
more text in part four
%    \end{macrocode}

%\iffalse
%</samplepart4>
%\fi
%
% %%%%%%%%%%%%%%%%%%%%%%%%%%%%%%%%%%%%%%
% \paragraph{Forwarding for a Complete Draft.}
%
% The following forwarding file |cdocsdrf.tex|
% compiles the main document in draft mode:
%\iffalse
%<*sampledraft>
%\fi
%    \begin{macrocode}
\def\version{draft}
\input{childdoc.def}
\childdocforward{cdocsamp}
%    \end{macrocode}

%\iffalse
%</sampledraft>
%\fi
%
% %%%%%%%%%%%%%%%%%%%%%%%%%%%%%%%%%%%%%%
% \paragraph{Forwarding for Final Version of the Chapters.}
%
% The following forwarding files |cdocsfn1.tex| and |cdocsfn2.tex|
% (with identical content)
% compile the final versions of the child documents
% |cdocsch1.tex| and |cdocsch2.tex|, respectively:
%\iffalse
%<*samplefinal>
%\fi
%    \begin{macrocode}
\def\version{final}
\input{childdoc.def}
\childdocforwardprefix[cdocsamp]{cdocsfn}{cdocsch}
%    \end{macrocode}

%\iffalse
%</samplefinal>
%\fi
%
% %%%%%%%%%%%%%%%%%%%%%%%%%%%%%%%%%%%%%%
% \paragraph{Command Line Processing.}
%
% The following three command lines generate the output files
% |cdocscld|, |cdocscl1| and |cdocscl2|
% which should be identical to
% |cdocsdrf|, |cdocsch1| and |cdocsfn2|, respectively:
% \begin{center}
% \begin{tabular}{l}
% |latex -jobname cdocscld \|\\
% |  "\def\version{draft}\input{childdoc.def}\childdocforward{cdocsamp}"|\\
% |latex -jobname cdocscl1 \|\\
% |  "\input{childdoc.def}\childdocforward[cdocsamp]{cdocsch1}"|\\
% |latex -jobname cdocscl2 \|\\
% |  "\def\version{final}\input{childdoc.def}\childdocforward{cdocsch2}"|
% \end{tabular}
% \end{center}
% Note that the trailing backslash on each first line
% merely continues the input to the second line
% (for convenient cut ant paste).
% Furthermore, the command |latex| can be replaced by any
% of its alternative versions such as |pdflatex|.
%
% %%%%%%%%%%%%%%%%%%%%%%%%%%%%%%%%%%%%%%%%%%%%%%%%%%%%%%%%%%%%%%%%%%%%%%%%%%%%%%
% %%%%%%%%%%%%%%%%%%%%%%%%%%%%%%%%%%%%%%%%%%%%%%%%%%%%%%%%%%%%%%%%%%%%%%%%%%%%%%
% \section{Implementation}
%\iffalse
%<*package>
%\fi
%
% This section describes the definitions file |childdoc.def|.

% The definitions cannot be loaded using |\usepackage| or |\RequirePackage|
% which has a mechanism to prevent loading a style file more than once.
% When loading the definitions by means of |\input|
% multiple instances have to be prevented manually:
%\iffalse
%This code needs to be before the `\ProvidesFile' directive
%which is defined at the beginning of this file.
%Therefore it is also placed there and commented out here.
%</package>
%<*discard>
%\fi
%    \begin{macrocode}
\ifdefined\childdocmain\endinput\fi
%    \end{macrocode}
%\iffalse
%</discard>
%<*package>
%\fi
%
% \macro{\ifchilddoc}
% \macro{\ifchilddocmanual}
% The conditional |\ifchilddoc| tells whether a
% child (true) or main (false) document is being compiled.
% The conditional |\ifchilddocmanual| tells whether
% the |\includeonly| mechanism is used (false) or
% the selection of child files must be performed manually (true).
% The definitions initialise to false:
%    \begin{macrocode}
\newif\ifchilddoc
\newif\ifchilddocmanual
%    \end{macrocode}

% \macro{\childdocname}
% \macro{\childdocjob}
% The macro |\childdocname| stores the name of the main document
% to be compiled. The macro |\childdocjob| stores the name of
% the document on which the \LaTeX{} compiler was originally invoked.
% The content of |\jobname| cannot be compared
% to filenames specified in the source due to different catcodes.
% The following code rescans |\jobname|, stores the result
% in |\childdocname| and saves a copy in |\childdocjob|:
%    \begin{macrocode}
\edef\childdocname{\scantokens\expandafter{\jobname\noexpand}}
\let\childdocjob\childdocname
%    \end{macrocode}

% \macro{\childdocdisable}
% The macro |\childdocdisable| prevents the main file
% from being processed more than once.
% At this stage, the main document command |\childdocmain|
% is assumed to be called once again where it should do nothing.
% Any subsequent call to it should prevent
% a secondary processing of the main document
% It overwrites the forwarding commands
% |\childdocof| and |\childdocforward|
% with empty macros to prevent further inclusions of the main document:
%    \begin{macrocode}
\newcommand{\childdocdisable}
{
  \renewcommand{\childdocmain}[1]{\renewcommand{\childdocmain}[1]{\endinput}}
  \renewcommand{\childdocof}[1]{}
  \renewcommand{\childdocby}[2][]{}
  \renewcommand{\childdocforward}[2][]{}
  \renewcommand{\childdocdisable}{}
}
%    \end{macrocode}

% \macro{\childdocmain}
% The macro |\childdocmain| is to be called at the top of the main file
% with nothing or the main filename (without extension) as argument.
% First, it breaks loops.
% If the argument is not empty and does not match |\childdocname|
% (which is set by the first inclusion of |childdoc.def|),
% |\ifchilddoc| is set to true, |\includeonly| is applied to the child file
% and |\jobname| is set to the main file
% (for proper handling of |.aux| files):
%    \begin{macrocode}
\newcommand{\childdocmain}[1]
{
  \childdocdisable\childdocmain{}
  \if?#1?\else
    \begingroup
      \def\childdoctmp{#1}
      \ifx\childdoctmp\childdocname
        \def\childdoctmp{}
      \else
        \def\childdoctmp
        {
          \childdoctrue
          \includeonly{\childdocname}
          \def\childdocjob{#1}
          \def\jobname{#1}
        }
      \fi
      \expandafter
    \endgroup
    \childdoctmp
  \fi
}
%    \end{macrocode}

% \macro{\childdocof}
% The command |\childdocof| redirects
% compilation to the main file |#1|.
%    \begin{macrocode}
\newcommand{\childdocof}[1]
{
  \childdocdisable
  \childdoctrue
  \includeonly{\childdocname}
  \def\jobname{#1}
  \def\childdocjob{#1}
  \input{#1}
}
%    \end{macrocode}

% \macro{\childdocby}
% The command |\childdocby| ....
%    \begin{macrocode}
\newcommand{\childdocby}[2][]
{
  \childdocdisable
  \childdoctrue
  \childdocmanualtrue
  \if?#1?\else
    \def\jobname{#2}
  \fi
  \def\childdocjob{#2}
  \input{#2}
  \endinput
}
%    \end{macrocode}

% \macro{\childdocforward}
% The command |\childdocforward| redirects
% compilation to the main file or
% (if the optional argument is given) a child file.
% Parameters are set as if the main file
% or a child file starting with |\childdocof| was compiled.
% Then compilation is handed over to the main file:
%    \begin{macrocode}
\newcommand{\childdocforward}[2][]
{
  \begingroup
    \if?#1?
      \def\childdoctmp
      {
        \def\childdocname{#2}
        \def\childdocjob{#2}
        \def\jobname{#2}
        \input{#2}
        \endinput
      }
    \else
      \def\childdoctmp
      {
        \childdocdisable
        \def\childdocname{#2}
        \childdoctrue
        \includeonly{#2}
        \def\childdocjob{#1}
        \def\jobname{#1}
        \input{#1}
        \endinput
      }
    \fi
    \expandafter
  \endgroup
  \childdoctmp
}
%    \end{macrocode}

% \macro{\childdocforwardprefix}
% The command |\childdocforwardprefix| redirects
% compilation to the main or a child file by means of a pattern.
% The prefix |#1| in the current filename is replaced by |#2|
% and the suffix of the current filename is kept
% (it is assumed that the filename does not contain the substring `|~~~|'
% which is used as a delimiter).
% Compilation is handed over to the new file by |\childdocforward|:
%    \begin{macrocode}
\newcommand{\childdocforwardprefix}[3][]
{
  \begingroup
    \def\childdocextract #2##1~~~{\def\childdoctmp{\childdocforward[#1]{#3##1}}}
    \expandafter\childdocextract\childdocname~~~
    \expandafter
  \endgroup
  \childdoctmp
}
%    \end{macrocode}

% \macro{\childdoc}
% The deprecated macro |\childdoc| is a legacy version of |\childdocmain|:
%    \begin{macrocode}
\newcommand{\childdoc}{\childdocmain}
%    \end{macrocode}

% \macro{\childdocredirect}
% The deprecated macro |\childdocredirect| is a legacy version
% of |\childdocforward| and |\childdocforwardprefix|:
%    \begin{macrocode}
\newcommand{\childdocredirect}[2][]
{
  \begingroup
    \if?#1?
      \def\childdoctmp{\childdocforward{#2}}
    \else
      \def\childdoctmp{\childdocforwardprefix{#1}{#2}}
    \fi
    \expandafter
  \endgroup
  \childdoctmp
}
%    \end{macrocode}

%\iffalse
%</package>
%\fi
%
\endinput
\childdocforward{cdocsch2}"|
% \end{tabular}
% \end{center}
% Note that the trailing backslash on each first line
% merely continues the input to the second line
% (for convenient cut ant paste).
% Furthermore, the command |latex| can be replaced by any
% of its alternative versions such as |pdflatex|.
%
% %%%%%%%%%%%%%%%%%%%%%%%%%%%%%%%%%%%%%%%%%%%%%%%%%%%%%%%%%%%%%%%%%%%%%%%%%%%%%%
% %%%%%%%%%%%%%%%%%%%%%%%%%%%%%%%%%%%%%%%%%%%%%%%%%%%%%%%%%%%%%%%%%%%%%%%%%%%%%%
% \section{Implementation}
%\iffalse
%<*package>
%\fi
%
% This section describes the definitions file |childdoc.def|.

% The definitions cannot be loaded using |\usepackage| or |\RequirePackage|
% which has a mechanism to prevent loading a style file more than once.
% When loading the definitions by means of |\input|
% multiple instances have to be prevented manually:
%\iffalse
%This code needs to be before the `\ProvidesFile' directive
%which is defined at the beginning of this file.
%Therefore it is also placed there and commented out here.
%</package>
%<*discard>
%\fi
%    \begin{macrocode}
\ifdefined\childdocmain\endinput\fi
%    \end{macrocode}
%\iffalse
%</discard>
%<*package>
%\fi
%
% \macro{\ifchilddoc}
% \macro{\ifchilddocmanual}
% The conditional |\ifchilddoc| tells whether a
% child (true) or main (false) document is being compiled.
% The conditional |\ifchilddocmanual| tells whether
% the |\includeonly| mechanism is used (false) or
% the selection of child files must be performed manually (true).
% The definitions initialise to false:
%    \begin{macrocode}
\newif\ifchilddoc
\newif\ifchilddocmanual
%    \end{macrocode}

% \macro{\childdocname}
% \macro{\childdocjob}
% The macro |\childdocname| stores the name of the main document
% to be compiled. The macro |\childdocjob| stores the name of
% the document on which the \LaTeX{} compiler was originally invoked.
% The content of |\jobname| cannot be compared
% to filenames specified in the source due to different catcodes.
% The following code rescans |\jobname|, stores the result
% in |\childdocname| and saves a copy in |\childdocjob|:
%    \begin{macrocode}
\edef\childdocname{\scantokens\expandafter{\jobname\noexpand}}
\let\childdocjob\childdocname
%    \end{macrocode}

% \macro{\childdocdisable}
% The macro |\childdocdisable| prevents the main file
% from being processed more than once.
% At this stage, the main document command |\childdocmain|
% is assumed to be called once again where it should do nothing.
% Any subsequent call to it should prevent
% a secondary processing of the main document
% It overwrites the forwarding commands
% |\childdocof| and |\childdocforward|
% with empty macros to prevent further inclusions of the main document:
%    \begin{macrocode}
\newcommand{\childdocdisable}
{
  \renewcommand{\childdocmain}[1]{\renewcommand{\childdocmain}[1]{\endinput}}
  \renewcommand{\childdocof}[1]{}
  \renewcommand{\childdocby}[2][]{}
  \renewcommand{\childdocforward}[2][]{}
  \renewcommand{\childdocdisable}{}
}
%    \end{macrocode}

% \macro{\childdocmain}
% The macro |\childdocmain| is to be called at the top of the main file
% with nothing or the main filename (without extension) as argument.
% First, it breaks loops.
% If the argument is not empty and does not match |\childdocname|
% (which is set by the first inclusion of |childdoc.def|),
% |\ifchilddoc| is set to true, |\includeonly| is applied to the child file
% and |\jobname| is set to the main file
% (for proper handling of |.aux| files):
%    \begin{macrocode}
\newcommand{\childdocmain}[1]
{
  \childdocdisable\childdocmain{}
  \if?#1?\else
    \begingroup
      \def\childdoctmp{#1}
      \ifx\childdoctmp\childdocname
        \def\childdoctmp{}
      \else
        \def\childdoctmp
        {
          \childdoctrue
          \includeonly{\childdocname}
          \def\childdocjob{#1}
          \def\jobname{#1}
        }
      \fi
      \expandafter
    \endgroup
    \childdoctmp
  \fi
}
%    \end{macrocode}

% \macro{\childdocof}
% The command |\childdocof| redirects
% compilation to the main file |#1|.
%    \begin{macrocode}
\newcommand{\childdocof}[1]
{
  \childdocdisable
  \childdoctrue
  \includeonly{\childdocname}
  \def\jobname{#1}
  \def\childdocjob{#1}
  \input{#1}
}
%    \end{macrocode}

% \macro{\childdocby}
% The command |\childdocby| ....
%    \begin{macrocode}
\newcommand{\childdocby}[2][]
{
  \childdocdisable
  \childdoctrue
  \childdocmanualtrue
  \if?#1?\else
    \def\jobname{#2}
  \fi
  \def\childdocjob{#2}
  \input{#2}
  \endinput
}
%    \end{macrocode}

% \macro{\childdocforward}
% The command |\childdocforward| redirects
% compilation to the main file or
% (if the optional argument is given) a child file.
% Parameters are set as if the main file
% or a child file starting with |\childdocof| was compiled.
% Then compilation is handed over to the main file:
%    \begin{macrocode}
\newcommand{\childdocforward}[2][]
{
  \begingroup
    \if?#1?
      \def\childdoctmp
      {
        \def\childdocname{#2}
        \def\childdocjob{#2}
        \def\jobname{#2}
        \input{#2}
        \endinput
      }
    \else
      \def\childdoctmp
      {
        \childdocdisable
        \def\childdocname{#2}
        \childdoctrue
        \includeonly{#2}
        \def\childdocjob{#1}
        \def\jobname{#1}
        \input{#1}
        \endinput
      }
    \fi
    \expandafter
  \endgroup
  \childdoctmp
}
%    \end{macrocode}

% \macro{\childdocforwardprefix}
% The command |\childdocforwardprefix| redirects
% compilation to the main or a child file by means of a pattern.
% The prefix |#1| in the current filename is replaced by |#2|
% and the suffix of the current filename is kept
% (it is assumed that the filename does not contain the substring `|~~~|'
% which is used as a delimiter).
% Compilation is handed over to the new file by |\childdocforward|:
%    \begin{macrocode}
\newcommand{\childdocforwardprefix}[3][]
{
  \begingroup
    \def\childdocextract #2##1~~~{\def\childdoctmp{\childdocforward[#1]{#3##1}}}
    \expandafter\childdocextract\childdocname~~~
    \expandafter
  \endgroup
  \childdoctmp
}
%    \end{macrocode}

% \macro{\childdoc}
% The deprecated macro |\childdoc| is a legacy version of |\childdocmain|:
%    \begin{macrocode}
\newcommand{\childdoc}{\childdocmain}
%    \end{macrocode}

% \macro{\childdocredirect}
% The deprecated macro |\childdocredirect| is a legacy version
% of |\childdocforward| and |\childdocforwardprefix|:
%    \begin{macrocode}
\newcommand{\childdocredirect}[2][]
{
  \begingroup
    \if?#1?
      \def\childdoctmp{\childdocforward{#2}}
    \else
      \def\childdoctmp{\childdocforwardprefix{#1}{#2}}
    \fi
    \expandafter
  \endgroup
  \childdoctmp
}
%    \end{macrocode}

%\iffalse
%</package>
%\fi
%
\endinput
|\\
|\childdocof{|\textit{main}|}|\\
\end{tabular}
\end{center}
at the top of every child file \textit{child}
which is included by |\include{|\textit{child}|}|
from within the main file
(or at least for those files to be compiled individually).
The argument \textit{main} must be the filename of the main file.

There are a couple of
considerations in setting up the main and child documents:

%%%%%%%%%%%%%%%%%%%%%%%%%%%%%%%%%%%%%%%%
\paragraph{Restrictions.}

Please note the following restrictions:
\begin{itemize}
\item
|\childdocmain| must be called with one argument \textit{main}
to ensure compatibility with earlier version of the package.
It must either be empty (|\childdocmain{}|)
or precisely match the filename of the main file in which it is specified.
See \secref{sec:detection} for further information.
\item
The filename \textit{main} must be specified without the |.tex| extension.
\item
The filename \textit{main} is case sensitive
(even in case-insensitive file systems)
due to internal string comparison.
\item
The argument \textit{main} should be fully expanded, it cannot be a macro.
\item
Subdirectories and special characters should be avoided in filenames.
\item
The command |\childdocmain{|\textit{main}|}| must be followed by a whitespace.
It should not be followed immediately by another command
or by a comment mark `|%|'.
This is because the \TeX{} parser reads the token immediately following
the argument of |\childdocmain| and puts it
at the beginning of every child section;
however, a white\-space is ignored.
\end{itemize}

%%%%%%%%%%%%%%%%%%%%%%%%%%%%%%%%%%%%%%%%
\paragraph{Content of Main File.}

It is advisable to place all content in the child files included by |\include|.
Any output contained in the main file will appear in all child documents
unless suppressed manually;
it cannot be suppressed automatically by the |\includeonly| directive
and thus should normally be avoided.
A method to include some content in the main file
by means of conditional processing is described in \secref{sec:conditional}.

%%%%%%%%%%%%%%%%%%%%%%%%%%%%%%%%%%%%%%%%
\paragraph{Page Numbering.}

When only a part of the document is compiled,
the appropriate numbering of pages
(as well as other status parameters)
is determined from the |.aux| files.
The latter contain information from previous passes.
However this information needs to propagate through
all intermediate child documents.
Therefore the page numbering in child documents may well
be inconsistent until the complete document is compiled at least once.

A useful (if unconventional) way to always ensure a consistent
page numbering is to restart the numbering in each child document
and denote the pages by `\textit{child}|.|\textit{page}'
where \textit{child} represents the chapter/section number of the child file.
This can be achieved by the command
|\numberwithin{page}{|\textit{child}|}|
of the \textsf{amsmath} package
where \textit{child} can be |chapter| or |section|
depending on the chosen structuring.
Alternatively, one can modify the macro |\thepage| appropriately
and reset the counter |page| at the start of each child file.

%%%%%%%%%%%%%%%%%%%%%%%%%%%%%%%%%%%%%%%%%%%%%%%%%%%%%%%%%%%%%%%%%%%%%%%%%%%%%%%%
\subsection{Conditional Processing}
\label{sec:conditional}

The package provides a mechanism to compile different versions
of a document. To customise the versions further some conditional processing
can come in handy to distinguish which version is being compiled.
The package provides two macros to describe the compilation context:

%%%%%%%%%%%%%%%%%%%%%%%%%%%%%%%%%%%%%%%%
\DescribeMacro{\ifchilddoc}
The conditional |\ifchilddoc| distinguishes between the compilation of
child documents and the main document:
%
\begin{center}
|\ifchilddoc |\textit{child-code}| |[|\||else |\textit{main-code}]| \||fi|
\end{center}

%%%%%%%%%%%%%%%%%%%%%%%%%%%%%%%%%%%%%%%%
\DescribeMacro{\childdocname}
\DescribeMacro{\childdocjob}
The macro |\childdocname| contains the filename (without extension)
of the main or child file being processed.
Note that |\childdocjob| will always contain the name of the main file.

%%%%%%%%%%%%%%%%%%%%%%%%%%%%%%%%%%%%%%%%
\paragraph{Title Page.}

Conditional processing can be used to include a title or banner page
in the main document when proper precautions are taken.
Importantly, the code in the main file should ensure that the page counter
(as well as other status parameters which are stored in the |.aux| files)
takes the same value after the conditional processing.
Otherwise the page numbers may take divergent values
depending on which part is compiled.

For example, a title page could be declared by:
%
\begin{center}
\begin{tabular}{l}
|\ifchilddoc\||else|\\
|\addtocounter{page}{-1}|\\
\textit{code for title page}\\
|\newpage|\\
|\||fi|
\end{tabular}
\end{center}
%
A banner page for the child documents can be generated by:
%
\begin{center}
\begin{tabular}{l}
|\ifchilddoc|\\
|\addtocounter{page}{-1}|\\
\textit{code for banner page}\\
|\newpage|\\
|\||fi|
\end{tabular}
\end{center}
%
Here one could write a message such as:
\begin{center}
|This is the part \childdocname{} of \childdocjob{}.|
\end{center}

%%%%%%%%%%%%%%%%%%%%%%%%%%%%%%%%%%%%%%%%%%%%%%%%%%%%%%%%%%%%%%%%%%%%%%%%%%%%%%%%
\subsection{Flags}
\label{sec:flags}

The package makes it easy to generate different versions
of the main or child documents.
To this end compilation flags can be defined
and assigned different default values.
They will be particularly useful in conjunction
with the forwarding mechanism described in \secref{sec:forward}.

For example, it may be useful to have a flag |\version|
which can be set to |draft| or |final|.
The document source will contain some conditional code
depending on the value of |\version|.
Suppose further, the flag should default to |final| for the main file
and to |draft| for child files
which is a natural assignment for editing the document.
This is achieved by placing the following code
in the preamble of the main document
(below the |\childdocmain| directive):
%
\begin{center}
\begin{tabular}{l}
|\ifchilddoc|\\
|\providecommand{\version}{draft}|\\
|\||else|\\
|\providecommand{\version}{final}|\\
|\||fi|
\end{tabular}
\end{center}
%
The definition by |\providecommand| makes sure
that previous definitions are not overwritten.
Further statements |\providecommand{\version}{...}|
can thus be added before the above code to override it.

For the main file, one might add a line
(between |\childdocmain| and the above block)
%
\begin{center}
|%\ifchilddoc\||else\providecommand{\version}{draft}\||fi|
\end{center}
%
which can be uncommented to produce a draft version.
Likewise one can add a line to the very top of a child file
(above the |\childdocof{|\textit{main}|}| directive)
%
\begin{center}
|%\providecommand{\version}{final}|
\end{center}
%
which can be uncommented to produce the final version of this child document.

%%%%%%%%%%%%%%%%%%%%%%%%%%%%%%%%%%%%%%%%%%%%%%%%%%%%%%%%%%%%%%%%%%%%%%%%%%%%%%%%
\subsection{Forwarding}
\label{sec:forward}

Different versions of the main or child documents
using compilation flags as described in \secref{sec:flags}
can be (permanently) stored in different files
for convenient compilation, viewing and distribution.
To this end, the package defines a command
to pass on compilation to a different file:

%%%%%%%%%%%%%%%%%%%%%%%%%%%%%%%%%%%%%%%%
\DescribeMacro{\childdocforward}
The command |\childdocforward| redirects processing to
another source file:
%
\begin{center}
\begin{tabular}{l}
|% \iffalse
%
% childdoc.dtx Copyright (C) 2017-2018 Niklas Beisert
%
% This work may be distributed and/or modified under the
% conditions of the LaTeX Project Public License, either version 1.3
% of this license or (at your option) any later version.
% The latest version of this license is in
%   http://www.latex-project.org/lppl.txt
% and version 1.3 or later is part of all distributions of LaTeX
% version 2005/12/01 or later.
%
% This work has the LPPL maintenance status `maintained'.
%
% The Current Maintainer of this work is Niklas Beisert.
%
% This work consists of the files childdoc.dtx and childdoc.ins
% and the derived files childdoc.def and cdocsamp.tex with
% cdocsch1.tex, cdocsch2.tex, cdocsdrf.tex, cdocsfn1.tex, cdocsfn2.tex.
%
%<package>\ifdefined\childdocmain\endinput\fi
%<package>\ProvidesFile{childdoc.def}[2018/12/30 v2.0 child document driver]
%<samplemain>\ProvidesFile{cdocsamp.tex}[2018/12/30 v2.0 sample for childdoc]
%<*driver>
%\ProvidesFile{childdoc.drv}[2018/12/30 v2.0 childdoc reference manual file]
\PassOptionsToClass{10pt,a4paper}{article}
\documentclass{ltxdoc}

\usepackage[margin=35mm]{geometry}
\usepackage{hyperref}
\usepackage{hyperxmp}
\usepackage[usenames]{color}

\hypersetup{colorlinks=true}
\hypersetup{pdfstartview=FitH}
\hypersetup{pdfpagemode=UseNone}
\hypersetup{pdfsource={}}
\hypersetup{pdflang={en-UK}}
\hypersetup{pdfcopyright={Copyright 2017-2018 Niklas Beisert.
  This work may be distributed and/or modified under the
  conditions of the LaTeX Project Public License, either version 1.3
  of this license or (at your option) any later version.}}
\hypersetup{pdflicenseurl={http://www.latex-project.org/lppl.txt}}
\hypersetup{pdfcontactaddress={ETH Zurich, ITP, HIT K,
  Wolfgang-Pauli-Strasse 27}}
\hypersetup{pdfcontactpostcode={8093}}
\hypersetup{pdfcontactcity={Zurich}}
\hypersetup{pdfcontactcountry={Switzerland}}
\hypersetup{pdfcontactemail={nbeisert@itp.phys.ethz.ch}}
\hypersetup{pdfcontacturl={http://people.phys.ethz.ch/\xmptilde nbeisert/}}

\newcommand{\secref}[1]{\hyperref[#1]{section \ref*{#1}}}

\parskip1ex
\parindent0pt
\let\olditemize\itemize
\def\itemize{\olditemize\parskip0pt}

\begin{document}

\title{The \textsf{childdoc} Package}
\hypersetup{pdftitle={The childdoc Package}}
\author{Niklas Beisert\\[2ex]
  Institut f\"ur Theoretische Physik\\
  Eidgen\"ossische Technische Hochschule Z\"urich\\
  Wolfgang-Pauli-Strasse 27, 8093 Z\"urich, Switzerland\\[1ex]
  \href{mailto:nbeisert@itp.phys.ethz.ch}
  {\texttt{nbeisert@itp.phys.ethz.ch}}}
\hypersetup{pdfauthor={Niklas Beisert}}
\hypersetup{pdfsubject={Manual for the LaTeX2e Package childdoc}}
\date{30 December 2018, \textsf{v2.0}}
\maketitle

\begin{abstract}\noindent
\textsf{childdoc} is a \LaTeXe{} package
that enables the direct compilation
of document sections included by |\include|
to individual files.
\end{abstract}

\begingroup
\parskip0ex
\tableofcontents
\endgroup

%%%%%%%%%%%%%%%%%%%%%%%%%%%%%%%%%%%%%%%%%%%%%%%%%%%%%%%%%%%%%%%%%%%%%%%%%%%%%%%%
%%%%%%%%%%%%%%%%%%%%%%%%%%%%%%%%%%%%%%%%%%%%%%%%%%%%%%%%%%%%%%%%%%%%%%%%%%%%%%%%
\section{Introduction}

\LaTeX{} provides a mechanism to structure a large document (such as a book)
into a main file and several child files (containing the chapters)
using the |\include| command.
This mechanism is beneficial for documents
which span hundreds of pages in order to
make the source file(s) more manageable.
Moreover, compilation can be restricted to
selected child files by means of the |\includeonly| command.
The latter feature can be used to reduce the compilation time while editing
(this was significantly more useful in the earlier days of \LaTeX{})
or to generate a smaller document which is easier to navigate.
Another application of |\includeonly| is to generate
documents consisting of selected parts of the complete document.

However, there are a few drawbacks of the plain |\include| mechanism:
\begin{itemize}
\item
The child files cannot be compiled on their own,
they can only be compiled via the main file.
A naive editing environment
(such as a text editor with an option
to have the current file processed by \LaTeX)
may require one to switch to the main file before compiling;
attempting to compile the child file produces errors.
\item
The main file must be modified (each time)
to adjust the |\includeonly| command
to the present needs. This easily leaves the main file in a messy state.
\item
The generated document will always carry the filename
of the main document. This is inconvenient if
several child files are to be compiled and
to be kept for distribution.
\end{itemize}

The present package provides a simple interface
to make child files individually compilable by \LaTeX{}.
Compiling a child file then has the same effect as compiling
the main file with an |\includeonly| command
to select the appropriate child.
Moreover the generated document will carry the name of the child
rather than the main file.
This resolves all three above issues.

This feature is meant to make the editing of books,
thesis documents and lecture notes somewhat more convenient.
However, the package can also be used efficiently for
composing a series of documents (such as exercise sheets)
which are typically distributed individually.
It then assists the author in generating the individual documents
(potentially in different versions)
as well as a document containing the collected series.
Another application is in developing style files
or other kinds of included material
where compilation of the style file could redirect
to a sample or test file.

%%%%%%%%%%%%%%%%%%%%%%%%%%%%%%%%%%%%%%%%%%%%%%%%%%%%%%%%%%%%%%%%%%%%%%%%%%%%%%%%
%%%%%%%%%%%%%%%%%%%%%%%%%%%%%%%%%%%%%%%%%%%%%%%%%%%%%%%%%%%%%%%%%%%%%%%%%%%%%%%%
\section{Usage}

First of all, the package \textsf{childdoc} is \emph{not} a standard
\LaTeXe{} |.sty| style file! Therefore it needs to be invoked in
a non-standard way.

%%%%%%%%%%%%%%%%%%%%%%%%%%%%%%%%%%%%%%%%%%%%%%%%%%%%%%%%%%%%%%%%%%%%%%%%%%%%%%%%
\subsection{Included Files}
\label{sec:include}

%%%%%%%%%%%%%%%%%%%%%%%%%%%%%%%%%%%%%%%%
\DescribeMacro{\childdocmain}
To use the package, add the commands
\begin{center}
\begin{tabular}{l}
|% \iffalse
%
% childdoc.dtx Copyright (C) 2017-2018 Niklas Beisert
%
% This work may be distributed and/or modified under the
% conditions of the LaTeX Project Public License, either version 1.3
% of this license or (at your option) any later version.
% The latest version of this license is in
%   http://www.latex-project.org/lppl.txt
% and version 1.3 or later is part of all distributions of LaTeX
% version 2005/12/01 or later.
%
% This work has the LPPL maintenance status `maintained'.
%
% The Current Maintainer of this work is Niklas Beisert.
%
% This work consists of the files childdoc.dtx and childdoc.ins
% and the derived files childdoc.def and cdocsamp.tex with
% cdocsch1.tex, cdocsch2.tex, cdocsdrf.tex, cdocsfn1.tex, cdocsfn2.tex.
%
%<package>\ifdefined\childdocmain\endinput\fi
%<package>\ProvidesFile{childdoc.def}[2018/12/30 v2.0 child document driver]
%<samplemain>\ProvidesFile{cdocsamp.tex}[2018/12/30 v2.0 sample for childdoc]
%<*driver>
%\ProvidesFile{childdoc.drv}[2018/12/30 v2.0 childdoc reference manual file]
\PassOptionsToClass{10pt,a4paper}{article}
\documentclass{ltxdoc}

\usepackage[margin=35mm]{geometry}
\usepackage{hyperref}
\usepackage{hyperxmp}
\usepackage[usenames]{color}

\hypersetup{colorlinks=true}
\hypersetup{pdfstartview=FitH}
\hypersetup{pdfpagemode=UseNone}
\hypersetup{pdfsource={}}
\hypersetup{pdflang={en-UK}}
\hypersetup{pdfcopyright={Copyright 2017-2018 Niklas Beisert.
  This work may be distributed and/or modified under the
  conditions of the LaTeX Project Public License, either version 1.3
  of this license or (at your option) any later version.}}
\hypersetup{pdflicenseurl={http://www.latex-project.org/lppl.txt}}
\hypersetup{pdfcontactaddress={ETH Zurich, ITP, HIT K,
  Wolfgang-Pauli-Strasse 27}}
\hypersetup{pdfcontactpostcode={8093}}
\hypersetup{pdfcontactcity={Zurich}}
\hypersetup{pdfcontactcountry={Switzerland}}
\hypersetup{pdfcontactemail={nbeisert@itp.phys.ethz.ch}}
\hypersetup{pdfcontacturl={http://people.phys.ethz.ch/\xmptilde nbeisert/}}

\newcommand{\secref}[1]{\hyperref[#1]{section \ref*{#1}}}

\parskip1ex
\parindent0pt
\let\olditemize\itemize
\def\itemize{\olditemize\parskip0pt}

\begin{document}

\title{The \textsf{childdoc} Package}
\hypersetup{pdftitle={The childdoc Package}}
\author{Niklas Beisert\\[2ex]
  Institut f\"ur Theoretische Physik\\
  Eidgen\"ossische Technische Hochschule Z\"urich\\
  Wolfgang-Pauli-Strasse 27, 8093 Z\"urich, Switzerland\\[1ex]
  \href{mailto:nbeisert@itp.phys.ethz.ch}
  {\texttt{nbeisert@itp.phys.ethz.ch}}}
\hypersetup{pdfauthor={Niklas Beisert}}
\hypersetup{pdfsubject={Manual for the LaTeX2e Package childdoc}}
\date{30 December 2018, \textsf{v2.0}}
\maketitle

\begin{abstract}\noindent
\textsf{childdoc} is a \LaTeXe{} package
that enables the direct compilation
of document sections included by |\include|
to individual files.
\end{abstract}

\begingroup
\parskip0ex
\tableofcontents
\endgroup

%%%%%%%%%%%%%%%%%%%%%%%%%%%%%%%%%%%%%%%%%%%%%%%%%%%%%%%%%%%%%%%%%%%%%%%%%%%%%%%%
%%%%%%%%%%%%%%%%%%%%%%%%%%%%%%%%%%%%%%%%%%%%%%%%%%%%%%%%%%%%%%%%%%%%%%%%%%%%%%%%
\section{Introduction}

\LaTeX{} provides a mechanism to structure a large document (such as a book)
into a main file and several child files (containing the chapters)
using the |\include| command.
This mechanism is beneficial for documents
which span hundreds of pages in order to
make the source file(s) more manageable.
Moreover, compilation can be restricted to
selected child files by means of the |\includeonly| command.
The latter feature can be used to reduce the compilation time while editing
(this was significantly more useful in the earlier days of \LaTeX{})
or to generate a smaller document which is easier to navigate.
Another application of |\includeonly| is to generate
documents consisting of selected parts of the complete document.

However, there are a few drawbacks of the plain |\include| mechanism:
\begin{itemize}
\item
The child files cannot be compiled on their own,
they can only be compiled via the main file.
A naive editing environment
(such as a text editor with an option
to have the current file processed by \LaTeX)
may require one to switch to the main file before compiling;
attempting to compile the child file produces errors.
\item
The main file must be modified (each time)
to adjust the |\includeonly| command
to the present needs. This easily leaves the main file in a messy state.
\item
The generated document will always carry the filename
of the main document. This is inconvenient if
several child files are to be compiled and
to be kept for distribution.
\end{itemize}

The present package provides a simple interface
to make child files individually compilable by \LaTeX{}.
Compiling a child file then has the same effect as compiling
the main file with an |\includeonly| command
to select the appropriate child.
Moreover the generated document will carry the name of the child
rather than the main file.
This resolves all three above issues.

This feature is meant to make the editing of books,
thesis documents and lecture notes somewhat more convenient.
However, the package can also be used efficiently for
composing a series of documents (such as exercise sheets)
which are typically distributed individually.
It then assists the author in generating the individual documents
(potentially in different versions)
as well as a document containing the collected series.
Another application is in developing style files
or other kinds of included material
where compilation of the style file could redirect
to a sample or test file.

%%%%%%%%%%%%%%%%%%%%%%%%%%%%%%%%%%%%%%%%%%%%%%%%%%%%%%%%%%%%%%%%%%%%%%%%%%%%%%%%
%%%%%%%%%%%%%%%%%%%%%%%%%%%%%%%%%%%%%%%%%%%%%%%%%%%%%%%%%%%%%%%%%%%%%%%%%%%%%%%%
\section{Usage}

First of all, the package \textsf{childdoc} is \emph{not} a standard
\LaTeXe{} |.sty| style file! Therefore it needs to be invoked in
a non-standard way.

%%%%%%%%%%%%%%%%%%%%%%%%%%%%%%%%%%%%%%%%%%%%%%%%%%%%%%%%%%%%%%%%%%%%%%%%%%%%%%%%
\subsection{Included Files}
\label{sec:include}

%%%%%%%%%%%%%%%%%%%%%%%%%%%%%%%%%%%%%%%%
\DescribeMacro{\childdocmain}
To use the package, add the commands
\begin{center}
\begin{tabular}{l}
|\input{childdoc.def}|\\
|\childdocmain{}|\\
\end{tabular}
\end{center}
at the very top of the main \LaTeX{} file,
in particular \emph{before} the |\documentclass| statement!
The argument of |\childdocmain| should be left empty
(but it must be present).

%%%%%%%%%%%%%%%%%%%%%%%%%%%%%%%%%%%%%%%%
\DescribeMacro{\childdocof}
Furthermore, add the commands
\begin{center}
\begin{tabular}{l}
|\input{childdoc.def}|\\
|\childdocof{|\textit{main}|}|\\
\end{tabular}
\end{center}
at the top of every child file \textit{child}
which is included by |\include{|\textit{child}|}|
from within the main file
(or at least for those files to be compiled individually).
The argument \textit{main} must be the filename of the main file.

There are a couple of
considerations in setting up the main and child documents:

%%%%%%%%%%%%%%%%%%%%%%%%%%%%%%%%%%%%%%%%
\paragraph{Restrictions.}

Please note the following restrictions:
\begin{itemize}
\item
|\childdocmain| must be called with one argument \textit{main}
to ensure compatibility with earlier version of the package.
It must either be empty (|\childdocmain{}|)
or precisely match the filename of the main file in which it is specified.
See \secref{sec:detection} for further information.
\item
The filename \textit{main} must be specified without the |.tex| extension.
\item
The filename \textit{main} is case sensitive
(even in case-insensitive file systems)
due to internal string comparison.
\item
The argument \textit{main} should be fully expanded, it cannot be a macro.
\item
Subdirectories and special characters should be avoided in filenames.
\item
The command |\childdocmain{|\textit{main}|}| must be followed by a whitespace.
It should not be followed immediately by another command
or by a comment mark `|%|'.
This is because the \TeX{} parser reads the token immediately following
the argument of |\childdocmain| and puts it
at the beginning of every child section;
however, a white\-space is ignored.
\end{itemize}

%%%%%%%%%%%%%%%%%%%%%%%%%%%%%%%%%%%%%%%%
\paragraph{Content of Main File.}

It is advisable to place all content in the child files included by |\include|.
Any output contained in the main file will appear in all child documents
unless suppressed manually;
it cannot be suppressed automatically by the |\includeonly| directive
and thus should normally be avoided.
A method to include some content in the main file
by means of conditional processing is described in \secref{sec:conditional}.

%%%%%%%%%%%%%%%%%%%%%%%%%%%%%%%%%%%%%%%%
\paragraph{Page Numbering.}

When only a part of the document is compiled,
the appropriate numbering of pages
(as well as other status parameters)
is determined from the |.aux| files.
The latter contain information from previous passes.
However this information needs to propagate through
all intermediate child documents.
Therefore the page numbering in child documents may well
be inconsistent until the complete document is compiled at least once.

A useful (if unconventional) way to always ensure a consistent
page numbering is to restart the numbering in each child document
and denote the pages by `\textit{child}|.|\textit{page}'
where \textit{child} represents the chapter/section number of the child file.
This can be achieved by the command
|\numberwithin{page}{|\textit{child}|}|
of the \textsf{amsmath} package
where \textit{child} can be |chapter| or |section|
depending on the chosen structuring.
Alternatively, one can modify the macro |\thepage| appropriately
and reset the counter |page| at the start of each child file.

%%%%%%%%%%%%%%%%%%%%%%%%%%%%%%%%%%%%%%%%%%%%%%%%%%%%%%%%%%%%%%%%%%%%%%%%%%%%%%%%
\subsection{Conditional Processing}
\label{sec:conditional}

The package provides a mechanism to compile different versions
of a document. To customise the versions further some conditional processing
can come in handy to distinguish which version is being compiled.
The package provides two macros to describe the compilation context:

%%%%%%%%%%%%%%%%%%%%%%%%%%%%%%%%%%%%%%%%
\DescribeMacro{\ifchilddoc}
The conditional |\ifchilddoc| distinguishes between the compilation of
child documents and the main document:
%
\begin{center}
|\ifchilddoc |\textit{child-code}| |[|\||else |\textit{main-code}]| \||fi|
\end{center}

%%%%%%%%%%%%%%%%%%%%%%%%%%%%%%%%%%%%%%%%
\DescribeMacro{\childdocname}
\DescribeMacro{\childdocjob}
The macro |\childdocname| contains the filename (without extension)
of the main or child file being processed.
Note that |\childdocjob| will always contain the name of the main file.

%%%%%%%%%%%%%%%%%%%%%%%%%%%%%%%%%%%%%%%%
\paragraph{Title Page.}

Conditional processing can be used to include a title or banner page
in the main document when proper precautions are taken.
Importantly, the code in the main file should ensure that the page counter
(as well as other status parameters which are stored in the |.aux| files)
takes the same value after the conditional processing.
Otherwise the page numbers may take divergent values
depending on which part is compiled.

For example, a title page could be declared by:
%
\begin{center}
\begin{tabular}{l}
|\ifchilddoc\||else|\\
|\addtocounter{page}{-1}|\\
\textit{code for title page}\\
|\newpage|\\
|\||fi|
\end{tabular}
\end{center}
%
A banner page for the child documents can be generated by:
%
\begin{center}
\begin{tabular}{l}
|\ifchilddoc|\\
|\addtocounter{page}{-1}|\\
\textit{code for banner page}\\
|\newpage|\\
|\||fi|
\end{tabular}
\end{center}
%
Here one could write a message such as:
\begin{center}
|This is the part \childdocname{} of \childdocjob{}.|
\end{center}

%%%%%%%%%%%%%%%%%%%%%%%%%%%%%%%%%%%%%%%%%%%%%%%%%%%%%%%%%%%%%%%%%%%%%%%%%%%%%%%%
\subsection{Flags}
\label{sec:flags}

The package makes it easy to generate different versions
of the main or child documents.
To this end compilation flags can be defined
and assigned different default values.
They will be particularly useful in conjunction
with the forwarding mechanism described in \secref{sec:forward}.

For example, it may be useful to have a flag |\version|
which can be set to |draft| or |final|.
The document source will contain some conditional code
depending on the value of |\version|.
Suppose further, the flag should default to |final| for the main file
and to |draft| for child files
which is a natural assignment for editing the document.
This is achieved by placing the following code
in the preamble of the main document
(below the |\childdocmain| directive):
%
\begin{center}
\begin{tabular}{l}
|\ifchilddoc|\\
|\providecommand{\version}{draft}|\\
|\||else|\\
|\providecommand{\version}{final}|\\
|\||fi|
\end{tabular}
\end{center}
%
The definition by |\providecommand| makes sure
that previous definitions are not overwritten.
Further statements |\providecommand{\version}{...}|
can thus be added before the above code to override it.

For the main file, one might add a line
(between |\childdocmain| and the above block)
%
\begin{center}
|%\ifchilddoc\||else\providecommand{\version}{draft}\||fi|
\end{center}
%
which can be uncommented to produce a draft version.
Likewise one can add a line to the very top of a child file
(above the |\childdocof{|\textit{main}|}| directive)
%
\begin{center}
|%\providecommand{\version}{final}|
\end{center}
%
which can be uncommented to produce the final version of this child document.

%%%%%%%%%%%%%%%%%%%%%%%%%%%%%%%%%%%%%%%%%%%%%%%%%%%%%%%%%%%%%%%%%%%%%%%%%%%%%%%%
\subsection{Forwarding}
\label{sec:forward}

Different versions of the main or child documents
using compilation flags as described in \secref{sec:flags}
can be (permanently) stored in different files
for convenient compilation, viewing and distribution.
To this end, the package defines a command
to pass on compilation to a different file:

%%%%%%%%%%%%%%%%%%%%%%%%%%%%%%%%%%%%%%%%
\DescribeMacro{\childdocforward}
The command |\childdocforward| redirects processing to
another source file:
%
\begin{center}
\begin{tabular}{l}
|\input{childdoc.def}|\\
|\childdocforward[|\textit{main}|]{|\textit{dest}|}|\\
\end{tabular}
\end{center}
%
The argument \textit{dest} is the destination file
(without extension).
It should be the main file or one of the child files.
Note that further \textsf{childdoc} directives
such as |\childdocof| and |\childdocforward|
in the indicated file will be processed in this form.
The optional argument \textit{main}
passes on directly to the main file \textit{main}
while pretending to compile the child \textit{dest}.
This form behaves as if \textit{dest}
issues |\childdocof{|\textit{main}|}| right away,
and no further \textsf{childdoc} directives will be processed.

%%%%%%%%%%%%%%%%%%%%%%%%%%%%%%%%%%%%%%%%
\DescribeMacro{\...prefix}
In the alternative form |\childdocforwardprefix|,
%
\begin{center}
\begin{tabular}{l}
|\input{childdoc.def}|\\
|\childdocforwardprefix[|\textit{main}|]{|\textit{prefix}|}{|\textit{dest}|}|
\end{tabular}
\end{center}
%
the destination file is determined by a pattern
depending on the current file:
To make this work, the current file must be called
`{\textit{prefix}\hspace{0.2em}\textit{suffix}}'
with \textit{prefix} matching precisely the argument.
Processing is then passed on to the file
`{\textit{dest}\hspace{0.2em}\textit{suffix}}'.
Surely, the same effect is achieved by
directly specifying the
argument `{\textit{dest}\hspace{0.2em}\textit{suffix}}'
in the first form.
However, that requires to set up a different file
for each child. With the alternative form of the command
all these files can have exactly the same content
which simplifies setting them up and maintaining them.

For example, the following file |draft.tex|
with a compilation flag |\version| as described in \secref{sec:flags}
compiles the main document as a draft:
%
\begin{center}
\begin{tabular}{l}
|\def\version{draft}|\\
|\input{childdoc.def}|\\
|\childdocforward{|\textit{main}|}|
\end{tabular}
\end{center}
%
Likewise, the following files |final|\textit{nn}|.tex|
compile the final version of the child document
|child|\textit{nn}|.tex|:
%
\begin{center}
\begin{tabular}{l}
|\def\version{final}|\\
|\input{childdoc.def}|\\
|\childdocforwardprefix{final}{child}|
\end{tabular}
\end{center}
%

Note that when several versions of a main file and/or of each child file
are to be generated, it may be convenient to set up a |Makefile| or
shell script to automatise the process.

%%%%%%%%%%%%%%%%%%%%%%%%%%%%%%%%%%%%%%%%%%%%%%%%%%%%%%%%%%%%%%%%%%%%%%%%%%%%%%%%
\subsection{Command Line Processing}
\label{sec:commandline}

The effect of redirection files can also be achieved by invoking
the \LaTeX{} compiler with a more elaborate command line.
Most conveniently this should be done as part
of a shell script or a |Makefile|.

When using \textsf{childdoc} in the main file, the following
command lines effectively perform a redirection
(note that depending on the shell being used,
backslashes may have to be doubled: `|\|' $\to$ `|\\|'):
%
\begin{center}
|... -jobname "|\textit{target}|" |\\|"|[\textit{flags}]%
|\input{childdoc.def}\childdocforward[|\textit{main}|]{|\textit{dest}|}"|
\end{center}
%
Here \textit{target} is the name of the output file,
\textit{main} is the name of the main file
and \textit{dest} is the name of the main or child file to be processed
(all filenames without extensions).
The optional argument \textit{main} can be omitted
if \textit{main} matches \textit{dest}.
Optionally, compilation \textit{flags} can be defined via |\def| commands.
This command line makes the \TeX{} engine believe
it is compiling the file \textit{target}
whose content is specified as the latter parameter.
The provided code then forwards the processing to
\textit{main} or \textit{dest} as described in \secref{sec:forward}.

%%%%%%%%%%%%%%%%%%%%%%%%%%%%%%%%%%%%%%%%%%%%%%%%%%%%%%%%%%%%%%%%%%%%%%%%%%%%%%%%
\subsection{Include by Input}
\label{sec:input}

Including child documents by |\include| has some restrictions by design.
Most notably, the content of a child document always occupies
its own set of pages; pages cannot be shared between child documents.
Usually, this behaviour makes perfect sense
because each child document contain an essential part of the document.
However, in some situations it may be desirable to compose
a document from a collection of parts
without having mandatory page breaks between then.
For this case, the package
provides a mechanism to include parts
by |\input| which can also be processed individually.
However, by construction this mechanism
requires manual handling of the content to be output.

%%%%%%%%%%%%%%%%%%%%%%%%%%%%%%%%%%%%%%%%
\DescribeMacro{\ifchilddocmanual}
The main file should be prepared as usual, see \secref{sec:include}.
However, the document body must make a distinction
between processing of an individual part and of the main document, e.g.:
%
\begin{center}
\begin{tabular}{l}
|\ifchilddocmanual|\\
|\input{\childdocname}|\\
|\||else|\\
\textit{document body with }|\input{|\textit{part}|}|\\
|\||fi|
\end{tabular}
\end{center}
%
The conditional |\ifchilddocmanual| is true whenever
a part to be included by |\input| is being compiled,
and the name of the part is stored in |\childdocname|.

%%%%%%%%%%%%%%%%%%%%%%%%%%%%%%%%%%%%%%%%
\DescribeMacro{\childdocby}
Each part to be included by |\input| should start with:
%
\begin{center}
\begin{tabular}{l}
|\input{childdoc.def}|\\
|\childdocby{|\textit{main}|}|\\
\end{tabular}
\end{center}
%
The directive |\childdocby| is similar to |\childdocof|
described in \secref{sec:include},
but the subsequent selection of content must be done manually.
To that end, both |\ifchilddoc| and |\ifchilddocmanual|
will be true upon processing of a part,
and the name of the part is stored in |\childdocname|.
Note that |\jobname| will be set to the filename of the current part
so that each part receives an individual |.aux| file
that does not interfere with the |.aux| file(s) of the main document.
This behaviour can be altered by the alternative form
|\childdocby[*]{|\textit{main}|}| (with a non-empty optional argument)
which uses the |.aux| file of the main document
by setting |\jobname| to \textit{main}.

%%%%%%%%%%%%%%%%%%%%%%%%%%%%%%%%%%%%%%%%%%%%%%%%%%%%%%%%%%%%%%%%%%%%%%%%%%%%%%%%
\subsection{Driver Development}
\label{sec:driver}

The \textsf{childdoc} mechanism can also be use for the development
of definition files such as \LaTeX{} styles or classes.
This case differs from the above setup with multiple parts
included by |\include| in that no |\includeonly| should be invoked.
This can be achieved by starting the include file
(before |\ProvidesPackage|) with:
%
\begin{center}
\begin{tabular}{l}
|\input{childdoc.def}|\\
|\childdocforward{|\textit{main}|}|\\
\end{tabular}
\end{center}
%
or alternatively with:
%
\begin{center}
\begin{tabular}{l}
|\input{childdoc.def}|\\
|\childdocby{|\textit{main}|}|\\
\end{tabular}
\end{center}
%
Both forms have slightly different effects as described above.
The main file is prepared as usual, see \secref{sec:include}.

%%%%%%%%%%%%%%%%%%%%%%%%%%%%%%%%%%%%%%%%%%%%%%%%%%%%%%%%%%%%%%%%%%%%%%%%%%%%%%%%
\subsection{Legacy Detection}
\label{sec:detection}

The directive |\childdocmain| in the main file can detect
whether the complete document or merely a child is to be compiled
even without using the directive |\childdocof|.
This method is deprecated because it is less robust
and there is no compelling reason to use it;
it is merely provided for backward compatibility
and it may be removed in future versions.

If the detection mechanism is to be used,
it is mandatory to correctly specify
the filename of the main file as the argument of |\childdocmain|:
%
\begin{center}
\begin{tabular}{l}
|\input{childdoc.def}|\\
|\childdocmain{|\textit{main}|}|\\
\end{tabular}
\end{center}
%
If |\jobname| does not match the argument \textit{main} of |\childdocmain|,
it is assumed that |\jobname| points to the child file to be compiled.
When using |\childdocmain| with the main file specified as argument,
it suffices to start a child file
with just |\input{|\textit{main}|}|
without loading of the package and using |\childdocof|.
If instead all processing is done
with the appropriate \textsf{childdoc} directives,
the argument of \textit{main} of |\childdocmain| can be empty.

An alternative version of the command line processing described
in \secref{sec:commandline} using the detection mechanism reads:
%
\begin{center}
|... -jobname "|\textit{target}|" "|[\textit{flags}]%
[|\def\jobname{|\textit{dest}|}|]|\input{|\textit{main}|}"|
\end{center}

%%%%%%%%%%%%%%%%%%%%%%%%%%%%%%%%%%%%%%%%%%%%%%%%%%%%%%%%%%%%%%%%%%%%%%%%%%%%%%%%
\subsection{Manual Code}
\label{sec:manual}

In case one cannot be certain whether the definitions file |childdoc.def|
is installed on the target \TeX{} distribution
and one prefers not to ship it,
it is conceivable to paste a few relevant commands into the sources.

To that end, drop all statements |\input{childdoc.def}|
and perform the replacements as outlined below.
Instead of |\childdocmain{|\textit{main}|}| add the following code
to the top of the main file:
%
\begin{center}
\begin{tabular}{l}
|\||ifdefined\childdocname\endinput\||fi\newif\ifchilddoc|\\
|\edef\childdocname{\scantokens\expandafter{\jobname\noexpand}}|\\
|\def\childdocmain{|\textit{main}|}\||ifx\childdocmain\childdocname\||else|\\
|\childdoctrue\includeonly{\childdocname}\let\jobname\childdocmain\||fi|\\
\end{tabular}
\end{center}
%
Instead of |\childdocof{|\textit{main}|}| just include the main file
at the top of each child file:
%
\begin{center}
|\input{|\textit{main}|}|
\end{center}
%
A simple redirection |\childdocforward{|\textit{dest}|}| is achieved by:
%
\begin{center}
|\def\jobname{|\textit{dest}|}\input{\jobname}|
\end{center}
%
The redirection with prefix
|\childdocforwardprefix[|\textit{prefix}|]{|\textit{dest}|}|
is accomplished by:
%
\begin{center}
\begin{tabular}{l}
|{\edef\jobname{\scantokens\expandafter{\jobname\noexpand}}|\\
|\def\redirectjob |\textit{prefix}|#1~~~{\gdef\jobname{|\textit{dest}|#1}}|\\
|\expandafter\redirectjob\jobname~~~}\input{\jobname}|
\end{tabular}
\end{center}

In an alternative approach,
child documents can be compiled by a specific command line
without additional code or specific definitions:
%
\begin{center}
|... -jobname "|\textit{target}|" "|[\textit{flags}]%
|\includeonly{|\textit{dest}|}\input{|\textit{main}|}"|
\end{center}
%

%%%%%%%%%%%%%%%%%%%%%%%%%%%%%%%%%%%%%%%%%%%%%%%%%%%%%%%%%%%%%%%%%%%%%%%%%%%%%%%%
%%%%%%%%%%%%%%%%%%%%%%%%%%%%%%%%%%%%%%%%%%%%%%%%%%%%%%%%%%%%%%%%%%%%%%%%%%%%%%%%
\section{Information}

%%%%%%%%%%%%%%%%%%%%%%%%%%%%%%%%%%%%%%%%%%%%%%%%%%%%%%%%%%%%%%%%%%%%%%%%%%%%%%%%
\subsection{Copyright}

Copyright \copyright{} 2017--2018 Niklas Beisert

This work may be distributed and/or modified under the
conditions of the \LaTeX{} Project Public License, either version 1.3
of this license or (at your option) any later version.
The latest version of this license is in
  \url{http://www.latex-project.org/lppl.txt}
and version 1.3 or later is part of all distributions of \LaTeX{}
version 2005/12/01 or later.

This work has the LPPL maintenance status `maintained'.

The Current Maintainer of this work is Niklas Beisert.

This work consists of the files |README.txt|, |childdoc.ins| and |childdoc.dtx|
as well as the derived files |childdoc.def|, |cdocsamp.tex|
with |cdocsch1.tex|, |cdocsch2.tex|, |cdocspt3.tex|, |cdocspt4.tex|,
|cdocsdrf.tex|, |cdocsfn1.tex|, |cdocsfn2.tex|
as well as |childdoc.pdf|.

%%%%%%%%%%%%%%%%%%%%%%%%%%%%%%%%%%%%%%%%%%%%%%%%%%%%%%%%%%%%%%%%%%%%%%%%%%%%%%%%
\subsection{Files and Installation}

The package consists of the files:
%
\begin{center}
\begin{tabular}{ll}
    |README.txt|   & readme file \\
    |childdoc.ins| & installation file \\
    |childdoc.dtx| & source file \\
    |childdoc.def| & definition file \\
    |cdocsamp.tex| & sample main file \\
    |cdocsch1.tex| & sample include file \\
    |cdocsch2.tex| & sample include file \\
    |cdocspt3.tex| & sample part file \\
    |cdocspt4.tex| & sample part file \\
    |cdocsdrf.tex| & sample redirection file \\
    |cdocsfn1.tex| & sample redirection file \\
    |cdocsfn2.tex| & sample redirection file \\
    |childdoc.pdf| & manual
\end{tabular}
\end{center}
%
The distribution consists of the files
|README.txt|, |childdoc.ins| and |childdoc.dtx|.
%
\begin{itemize}
\item
Run (pdf)\LaTeX{} on |childdoc.dtx|
to compile the manual |childdoc.pdf| (this file).
\item
Run \LaTeX{} on |childdoc.ins| to create the definitions file |childdoc.def|
and the sample |cdocsamp.tex| with include files
|cdocsch1.tex|, |cdocsch2.tex|, |cdocspt3.tex|, |cdocspt4.tex|,
|cdocsdrf.tex|, |cdocsfn1.tex|, |cdocsfn2.tex|.
Then copy the file |childdoc.def| to an appropriate directory of your \LaTeX{}
distribution, e.g.\ \textit{texmf-root}|/tex/latex/childdoc|.
\end{itemize}

%%%%%%%%%%%%%%%%%%%%%%%%%%%%%%%%%%%%%%%%%%%%%%%%%%%%%%%%%%%%%%%%%%%%%%%%%%%%%%%%
\subsection{Related CTAN Packages}

There are several other packages which offer a similar functionality:
%
\begin{itemize}
\item
The packages
\href{http://ctan.org/pkg/docmute}{\textsf{docmute}},
\href{http://ctan.org/pkg/includex}{\textsf{includex}} and
\href{http://ctan.org/pkg/standalone}{\textsf{standalone}}
provide commands to include only the document body of
a child file thus allowing both files to be compiled individually.
\item
The packages \href{http://ctan.org/pkg/subdocs}{\textsf{subdocs}}
and \href{http://ctan.org/pkg/subfiles}{\textsf{subfiles}}
provide structures in which the main and child documents can be
encapsulated and allowing them to be compiled individually.
The inclusion mechanism is different from the conventional |\include|.
\item
The package \href{http://ctan.org/pkg/combine}{\textsf{combine}}
is an elaborate solution to combine several documents into one.
\end{itemize}
%
See also the CTAN topic \href{http://ctan.org/topic/subdocs}{\textsf{subdocs}}
for further related packages.
The present package differs from the above solutions in that
a document structure constructed with the conventional |\include| mechanism
just needs two extra commands at the top of every file
such that all constituent files can be compiled individually.

%%%%%%%%%%%%%%%%%%%%%%%%%%%%%%%%%%%%%%%%%%%%%%%%%%%%%%%%%%%%%%%%%%%%%%%%%%%%%%%%
%\subsection{Feature Suggestions}
%
%The following is a list of features which may be useful for future
%versions of this package:
%%
%\begin{itemize}
%\item
%\ldots
%\end{itemize}

%%%%%%%%%%%%%%%%%%%%%%%%%%%%%%%%%%%%%%%%%%%%%%%%%%%%%%%%%%%%%%%%%%%%%%%%%%%%%%%%
\subsection{Revision History}

%%%%%%%%%%%%%%%%%%%%%%%%%%%%%%%%%%%%%%%%
\paragraph{v2.0:} 2018/12/30

\begin{itemize}
\item
immediate forward processing
\item
added |\childdocby| mechanism
\item
manual restructured
\end{itemize}

%%%%%%%%%%%%%%%%%%%%%%%%%%%%%%%%%%%%%%%%
\paragraph{v1.6:} 2018/01/17

\begin{itemize}
\item
application for development of include files
\item
corrections to manual
\end{itemize}

%%%%%%%%%%%%%%%%%%%%%%%%%%%%%%%%%%%%%%%%
\paragraph{v1.5:} 2017/05/21

\begin{itemize}
\item
more complete structuring introduced
\item
|\childdocof| introduced
\item
|\childdoc| renamed to |\childdocmain|
\item
|\childredirect| renamed to |\childdocforward| and |\childdocforwardprefix|
and functionality expanded
\end{itemize}

%%%%%%%%%%%%%%%%%%%%%%%%%%%%%%%%%%%%%%%%
\paragraph{v1.0:} 2017/04/27

\begin{itemize}
\item
manual and install package
\item
first version published on CTAN
\end{itemize}

%%%%%%%%%%%%%%%%%%%%%%%%%%%%%%%%%%%%%%%%
\paragraph{v0.6:} 2017/04/26

\begin{itemize}
\item
redirection mechanism added
\end{itemize}

%%%%%%%%%%%%%%%%%%%%%%%%%%%%%%%%%%%%%%%%
\paragraph{v0.5:} 2017/04/26

\begin{itemize}
\item
functionality in definition file
\end{itemize}


%%%%%%%%%%%%%%%%%%%%%%%%%%%%%%%%%%%%%%%%%%%%%%%%%%%%%%%%%%%%%%%%%%%%%%%%%%%%%%%%
%%%%%%%%%%%%%%%%%%%%%%%%%%%%%%%%%%%%%%%%%%%%%%%%%%%%%%%%%%%%%%%%%%%%%%%%%%%%%%%%
%%%%%%%%%%%%%%%%%%%%%%%%%%%%%%%%%%%%%%%%%%%%%%%%%%%%%%%%%%%%%%%%%%%%%%%%%%%%%%%%
\appendix

\settowidth\MacroIndent{\rmfamily\scriptsize 000\ }

 \DocInput{childdoc.dtx}

\end{document}
%</driver>
% \fi
%
% %%%%%%%%%%%%%%%%%%%%%%%%%%%%%%%%%%%%%%%%%%%%%%%%%%%%%%%%%%%%%%%%%%%%%%%%%%%%%%
% %%%%%%%%%%%%%%%%%%%%%%%%%%%%%%%%%%%%%%%%%%%%%%%%%%%%%%%%%%%%%%%%%%%%%%%%%%%%%%
% \section{Sample}
%\iffalse
%<*samplemain>
%\fi
%
% The following presents a sample document
% with two chapters, two parts, a title page,
% a compile flag as well as three forwarding files to set the flag.
% It consists of eight |.tex| files:
% \begin{center}
% \begin{tabular}{ll}
% |cdocsamp.tex|&main file\\
% |cdocsch1.tex|&include file for chapter 1\\
% |cdocsch2.tex|&include file for chapter 2\\
% |cdocspt3.tex|&include file for part 3\\
% |cdocspt4.tex|&include file for part 4\\
% |cdocsdrf.tex|&forwarding file for main file in draft mode\\
% |cdocsfi1.tex|&forwarding file for final version of chapter 1\\
% |cdocsfi2.tex|&forwarding file for final version of chapter 2\\
% \end{tabular}
% \end{center}
% Each of the eight files can be compiled directly by the \LaTeX{} compiler.
%
% %%%%%%%%%%%%%%%%%%%%%%%%%%%%%%%%%%%%%%
% \paragraph{Main File.}
%
% The main file is called |cdocsamp.tex|.
%
% Load the \textsf{childdoc} definitions and
% declare the filename for the main document:
%    \begin{macrocode}
\input{childdoc.def}
\childdocmain{}
%    \end{macrocode}

% Optional override for |\version| flag:
%    \begin{macrocode}
%%\ifchilddoc\else\providecommand{\version}{draft}\fi
%    \end{macrocode}

% Define the default values for the |\version| flag
% (|final| for the main file and |draft| for childs):
%    \begin{macrocode}
\ifchilddoc
\providecommand{\version}{draft}
\else
\providecommand{\version}{final}
\fi
%    \end{macrocode}

% Load the standard document class:
%    \begin{macrocode}
\documentclass[12pt]{article}
%    \end{macrocode}

% Start the document body:
%    \begin{macrocode}
\begin{document}
%    \end{macrocode}

% Declare a title page.
% Print title, part of document being processed and version flag:
%    \begin{macrocode}
\addtocounter{page}{-1}
\begin{center}
{\LARGE\bfseries{}childdoc example\par}
\vspace{1cm}
\ifchilddoc
\ifchilddocmanual part\else chapter\fi:
`\childdocname' of `\childdocjob'\par
\else
main document: `\childdocjob'\par
\fi
version: \version\par
\end{center}
\newpage
%    \end{macrocode}

% Manually include selected file,
% otherwise process as usual:
%    \begin{macrocode}
\ifchilddocmanual
\section*{part `\childdocname'}
\input{\childdocname}
\else
%    \end{macrocode}

% Include the two chapters:
%    \begin{macrocode}
\include{cdocsch1}
\include{cdocsch2}
%    \end{macrocode}

% Include the two parts unless only chapters should be displayed:
%    \begin{macrocode}
\ifchilddoc\else
\section{part three}
\input{cdocspt3}
\section{part four}
\input{cdocspt4}
\fi
%    \end{macrocode}

% Process as usual until here:
%    \begin{macrocode}
\fi
%    \end{macrocode}

% End of document body:
%    \begin{macrocode}
\end{document}
%    \end{macrocode}
%\iffalse
%</samplemain>
%\fi
%
% %%%%%%%%%%%%%%%%%%%%%%%%%%%%%%%%%%%%%%
% \paragraph{Chapter Include Files.}
%
% The include files are called |cdocsch1.tex| and |cdocsch2.tex|.
%
%\iffalse
%<*samplechap1|samplechap2>
%\fi

% Optional override for |\version| flag:
%    \begin{macrocode}
%%\providecommand{\version}{final}
%    \end{macrocode}

% Include the main document:
%    \begin{macrocode}
\input{childdoc.def}
\childdocof{cdocsamp}
%    \end{macrocode}

%\iffalse
%</samplechap1|samplechap2>
%\fi
%
%\iffalse
%<*samplechap1>
%\fi
% Some text for chapter 1:
%    \begin{macrocode}
\section{one}
some text in chapter one
%    \end{macrocode}

%\iffalse
%</samplechap1>
%\fi
% Some text for chapter 2:
%\iffalse
%<*samplechap2>
%\fi
%    \begin{macrocode}
\section{two}
more text in chapter two
%    \end{macrocode}

%\iffalse
%</samplechap2>
%\fi
%
% %%%%%%%%%%%%%%%%%%%%%%%%%%%%%%%%%%%%%%
% \paragraph{Part Include Files.}
%
% The include files are called |cdocspt3.tex| and |cdocspt4.tex|.
%
%\iffalse
%<*samplepart3|samplepart4>
%\fi

% Optional override for |\version| flag:
%    \begin{macrocode}
%%\providecommand{\version}{final}
%    \end{macrocode}

% Include the main document:
%    \begin{macrocode}
\input{childdoc.def}
\childdocby{cdocsamp}
%    \end{macrocode}

%\iffalse
%</samplepart3|samplepart4>
%\fi
%
%\iffalse
%<*samplepart3>
%\fi
% Some text for part 3:
%    \begin{macrocode}
some text in part three
%    \end{macrocode}

%\iffalse
%</samplepart3>
%\fi
% Some text for part 4:
%\iffalse
%<*samplepart4>
%\fi
%    \begin{macrocode}
more text in part four
%    \end{macrocode}

%\iffalse
%</samplepart4>
%\fi
%
% %%%%%%%%%%%%%%%%%%%%%%%%%%%%%%%%%%%%%%
% \paragraph{Forwarding for a Complete Draft.}
%
% The following forwarding file |cdocsdrf.tex|
% compiles the main document in draft mode:
%\iffalse
%<*sampledraft>
%\fi
%    \begin{macrocode}
\def\version{draft}
\input{childdoc.def}
\childdocforward{cdocsamp}
%    \end{macrocode}

%\iffalse
%</sampledraft>
%\fi
%
% %%%%%%%%%%%%%%%%%%%%%%%%%%%%%%%%%%%%%%
% \paragraph{Forwarding for Final Version of the Chapters.}
%
% The following forwarding files |cdocsfn1.tex| and |cdocsfn2.tex|
% (with identical content)
% compile the final versions of the child documents
% |cdocsch1.tex| and |cdocsch2.tex|, respectively:
%\iffalse
%<*samplefinal>
%\fi
%    \begin{macrocode}
\def\version{final}
\input{childdoc.def}
\childdocforwardprefix[cdocsamp]{cdocsfn}{cdocsch}
%    \end{macrocode}

%\iffalse
%</samplefinal>
%\fi
%
% %%%%%%%%%%%%%%%%%%%%%%%%%%%%%%%%%%%%%%
% \paragraph{Command Line Processing.}
%
% The following three command lines generate the output files
% |cdocscld|, |cdocscl1| and |cdocscl2|
% which should be identical to
% |cdocsdrf|, |cdocsch1| and |cdocsfn2|, respectively:
% \begin{center}
% \begin{tabular}{l}
% |latex -jobname cdocscld \|\\
% |  "\def\version{draft}\input{childdoc.def}\childdocforward{cdocsamp}"|\\
% |latex -jobname cdocscl1 \|\\
% |  "\input{childdoc.def}\childdocforward[cdocsamp]{cdocsch1}"|\\
% |latex -jobname cdocscl2 \|\\
% |  "\def\version{final}\input{childdoc.def}\childdocforward{cdocsch2}"|
% \end{tabular}
% \end{center}
% Note that the trailing backslash on each first line
% merely continues the input to the second line
% (for convenient cut ant paste).
% Furthermore, the command |latex| can be replaced by any
% of its alternative versions such as |pdflatex|.
%
% %%%%%%%%%%%%%%%%%%%%%%%%%%%%%%%%%%%%%%%%%%%%%%%%%%%%%%%%%%%%%%%%%%%%%%%%%%%%%%
% %%%%%%%%%%%%%%%%%%%%%%%%%%%%%%%%%%%%%%%%%%%%%%%%%%%%%%%%%%%%%%%%%%%%%%%%%%%%%%
% \section{Implementation}
%\iffalse
%<*package>
%\fi
%
% This section describes the definitions file |childdoc.def|.

% The definitions cannot be loaded using |\usepackage| or |\RequirePackage|
% which has a mechanism to prevent loading a style file more than once.
% When loading the definitions by means of |\input|
% multiple instances have to be prevented manually:
%\iffalse
%This code needs to be before the `\ProvidesFile' directive
%which is defined at the beginning of this file.
%Therefore it is also placed there and commented out here.
%</package>
%<*discard>
%\fi
%    \begin{macrocode}
\ifdefined\childdocmain\endinput\fi
%    \end{macrocode}
%\iffalse
%</discard>
%<*package>
%\fi
%
% \macro{\ifchilddoc}
% \macro{\ifchilddocmanual}
% The conditional |\ifchilddoc| tells whether a
% child (true) or main (false) document is being compiled.
% The conditional |\ifchilddocmanual| tells whether
% the |\includeonly| mechanism is used (false) or
% the selection of child files must be performed manually (true).
% The definitions initialise to false:
%    \begin{macrocode}
\newif\ifchilddoc
\newif\ifchilddocmanual
%    \end{macrocode}

% \macro{\childdocname}
% \macro{\childdocjob}
% The macro |\childdocname| stores the name of the main document
% to be compiled. The macro |\childdocjob| stores the name of
% the document on which the \LaTeX{} compiler was originally invoked.
% The content of |\jobname| cannot be compared
% to filenames specified in the source due to different catcodes.
% The following code rescans |\jobname|, stores the result
% in |\childdocname| and saves a copy in |\childdocjob|:
%    \begin{macrocode}
\edef\childdocname{\scantokens\expandafter{\jobname\noexpand}}
\let\childdocjob\childdocname
%    \end{macrocode}

% \macro{\childdocdisable}
% The macro |\childdocdisable| prevents the main file
% from being processed more than once.
% At this stage, the main document command |\childdocmain|
% is assumed to be called once again where it should do nothing.
% Any subsequent call to it should prevent
% a secondary processing of the main document
% It overwrites the forwarding commands
% |\childdocof| and |\childdocforward|
% with empty macros to prevent further inclusions of the main document:
%    \begin{macrocode}
\newcommand{\childdocdisable}
{
  \renewcommand{\childdocmain}[1]{\renewcommand{\childdocmain}[1]{\endinput}}
  \renewcommand{\childdocof}[1]{}
  \renewcommand{\childdocby}[2][]{}
  \renewcommand{\childdocforward}[2][]{}
  \renewcommand{\childdocdisable}{}
}
%    \end{macrocode}

% \macro{\childdocmain}
% The macro |\childdocmain| is to be called at the top of the main file
% with nothing or the main filename (without extension) as argument.
% First, it breaks loops.
% If the argument is not empty and does not match |\childdocname|
% (which is set by the first inclusion of |childdoc.def|),
% |\ifchilddoc| is set to true, |\includeonly| is applied to the child file
% and |\jobname| is set to the main file
% (for proper handling of |.aux| files):
%    \begin{macrocode}
\newcommand{\childdocmain}[1]
{
  \childdocdisable\childdocmain{}
  \if?#1?\else
    \begingroup
      \def\childdoctmp{#1}
      \ifx\childdoctmp\childdocname
        \def\childdoctmp{}
      \else
        \def\childdoctmp
        {
          \childdoctrue
          \includeonly{\childdocname}
          \def\childdocjob{#1}
          \def\jobname{#1}
        }
      \fi
      \expandafter
    \endgroup
    \childdoctmp
  \fi
}
%    \end{macrocode}

% \macro{\childdocof}
% The command |\childdocof| redirects
% compilation to the main file |#1|.
%    \begin{macrocode}
\newcommand{\childdocof}[1]
{
  \childdocdisable
  \childdoctrue
  \includeonly{\childdocname}
  \def\jobname{#1}
  \def\childdocjob{#1}
  \input{#1}
}
%    \end{macrocode}

% \macro{\childdocby}
% The command |\childdocby| ....
%    \begin{macrocode}
\newcommand{\childdocby}[2][]
{
  \childdocdisable
  \childdoctrue
  \childdocmanualtrue
  \if?#1?\else
    \def\jobname{#2}
  \fi
  \def\childdocjob{#2}
  \input{#2}
  \endinput
}
%    \end{macrocode}

% \macro{\childdocforward}
% The command |\childdocforward| redirects
% compilation to the main file or
% (if the optional argument is given) a child file.
% Parameters are set as if the main file
% or a child file starting with |\childdocof| was compiled.
% Then compilation is handed over to the main file:
%    \begin{macrocode}
\newcommand{\childdocforward}[2][]
{
  \begingroup
    \if?#1?
      \def\childdoctmp
      {
        \def\childdocname{#2}
        \def\childdocjob{#2}
        \def\jobname{#2}
        \input{#2}
        \endinput
      }
    \else
      \def\childdoctmp
      {
        \childdocdisable
        \def\childdocname{#2}
        \childdoctrue
        \includeonly{#2}
        \def\childdocjob{#1}
        \def\jobname{#1}
        \input{#1}
        \endinput
      }
    \fi
    \expandafter
  \endgroup
  \childdoctmp
}
%    \end{macrocode}

% \macro{\childdocforwardprefix}
% The command |\childdocforwardprefix| redirects
% compilation to the main or a child file by means of a pattern.
% The prefix |#1| in the current filename is replaced by |#2|
% and the suffix of the current filename is kept
% (it is assumed that the filename does not contain the substring `|~~~|'
% which is used as a delimiter).
% Compilation is handed over to the new file by |\childdocforward|:
%    \begin{macrocode}
\newcommand{\childdocforwardprefix}[3][]
{
  \begingroup
    \def\childdocextract #2##1~~~{\def\childdoctmp{\childdocforward[#1]{#3##1}}}
    \expandafter\childdocextract\childdocname~~~
    \expandafter
  \endgroup
  \childdoctmp
}
%    \end{macrocode}

% \macro{\childdoc}
% The deprecated macro |\childdoc| is a legacy version of |\childdocmain|:
%    \begin{macrocode}
\newcommand{\childdoc}{\childdocmain}
%    \end{macrocode}

% \macro{\childdocredirect}
% The deprecated macro |\childdocredirect| is a legacy version
% of |\childdocforward| and |\childdocforwardprefix|:
%    \begin{macrocode}
\newcommand{\childdocredirect}[2][]
{
  \begingroup
    \if?#1?
      \def\childdoctmp{\childdocforward{#2}}
    \else
      \def\childdoctmp{\childdocforwardprefix{#1}{#2}}
    \fi
    \expandafter
  \endgroup
  \childdoctmp
}
%    \end{macrocode}

%\iffalse
%</package>
%\fi
%
\endinput
|\\
|\childdocmain{}|\\
\end{tabular}
\end{center}
at the very top of the main \LaTeX{} file,
in particular \emph{before} the |\documentclass| statement!
The argument of |\childdocmain| should be left empty
(but it must be present).

%%%%%%%%%%%%%%%%%%%%%%%%%%%%%%%%%%%%%%%%
\DescribeMacro{\childdocof}
Furthermore, add the commands
\begin{center}
\begin{tabular}{l}
|% \iffalse
%
% childdoc.dtx Copyright (C) 2017-2018 Niklas Beisert
%
% This work may be distributed and/or modified under the
% conditions of the LaTeX Project Public License, either version 1.3
% of this license or (at your option) any later version.
% The latest version of this license is in
%   http://www.latex-project.org/lppl.txt
% and version 1.3 or later is part of all distributions of LaTeX
% version 2005/12/01 or later.
%
% This work has the LPPL maintenance status `maintained'.
%
% The Current Maintainer of this work is Niklas Beisert.
%
% This work consists of the files childdoc.dtx and childdoc.ins
% and the derived files childdoc.def and cdocsamp.tex with
% cdocsch1.tex, cdocsch2.tex, cdocsdrf.tex, cdocsfn1.tex, cdocsfn2.tex.
%
%<package>\ifdefined\childdocmain\endinput\fi
%<package>\ProvidesFile{childdoc.def}[2018/12/30 v2.0 child document driver]
%<samplemain>\ProvidesFile{cdocsamp.tex}[2018/12/30 v2.0 sample for childdoc]
%<*driver>
%\ProvidesFile{childdoc.drv}[2018/12/30 v2.0 childdoc reference manual file]
\PassOptionsToClass{10pt,a4paper}{article}
\documentclass{ltxdoc}

\usepackage[margin=35mm]{geometry}
\usepackage{hyperref}
\usepackage{hyperxmp}
\usepackage[usenames]{color}

\hypersetup{colorlinks=true}
\hypersetup{pdfstartview=FitH}
\hypersetup{pdfpagemode=UseNone}
\hypersetup{pdfsource={}}
\hypersetup{pdflang={en-UK}}
\hypersetup{pdfcopyright={Copyright 2017-2018 Niklas Beisert.
  This work may be distributed and/or modified under the
  conditions of the LaTeX Project Public License, either version 1.3
  of this license or (at your option) any later version.}}
\hypersetup{pdflicenseurl={http://www.latex-project.org/lppl.txt}}
\hypersetup{pdfcontactaddress={ETH Zurich, ITP, HIT K,
  Wolfgang-Pauli-Strasse 27}}
\hypersetup{pdfcontactpostcode={8093}}
\hypersetup{pdfcontactcity={Zurich}}
\hypersetup{pdfcontactcountry={Switzerland}}
\hypersetup{pdfcontactemail={nbeisert@itp.phys.ethz.ch}}
\hypersetup{pdfcontacturl={http://people.phys.ethz.ch/\xmptilde nbeisert/}}

\newcommand{\secref}[1]{\hyperref[#1]{section \ref*{#1}}}

\parskip1ex
\parindent0pt
\let\olditemize\itemize
\def\itemize{\olditemize\parskip0pt}

\begin{document}

\title{The \textsf{childdoc} Package}
\hypersetup{pdftitle={The childdoc Package}}
\author{Niklas Beisert\\[2ex]
  Institut f\"ur Theoretische Physik\\
  Eidgen\"ossische Technische Hochschule Z\"urich\\
  Wolfgang-Pauli-Strasse 27, 8093 Z\"urich, Switzerland\\[1ex]
  \href{mailto:nbeisert@itp.phys.ethz.ch}
  {\texttt{nbeisert@itp.phys.ethz.ch}}}
\hypersetup{pdfauthor={Niklas Beisert}}
\hypersetup{pdfsubject={Manual for the LaTeX2e Package childdoc}}
\date{30 December 2018, \textsf{v2.0}}
\maketitle

\begin{abstract}\noindent
\textsf{childdoc} is a \LaTeXe{} package
that enables the direct compilation
of document sections included by |\include|
to individual files.
\end{abstract}

\begingroup
\parskip0ex
\tableofcontents
\endgroup

%%%%%%%%%%%%%%%%%%%%%%%%%%%%%%%%%%%%%%%%%%%%%%%%%%%%%%%%%%%%%%%%%%%%%%%%%%%%%%%%
%%%%%%%%%%%%%%%%%%%%%%%%%%%%%%%%%%%%%%%%%%%%%%%%%%%%%%%%%%%%%%%%%%%%%%%%%%%%%%%%
\section{Introduction}

\LaTeX{} provides a mechanism to structure a large document (such as a book)
into a main file and several child files (containing the chapters)
using the |\include| command.
This mechanism is beneficial for documents
which span hundreds of pages in order to
make the source file(s) more manageable.
Moreover, compilation can be restricted to
selected child files by means of the |\includeonly| command.
The latter feature can be used to reduce the compilation time while editing
(this was significantly more useful in the earlier days of \LaTeX{})
or to generate a smaller document which is easier to navigate.
Another application of |\includeonly| is to generate
documents consisting of selected parts of the complete document.

However, there are a few drawbacks of the plain |\include| mechanism:
\begin{itemize}
\item
The child files cannot be compiled on their own,
they can only be compiled via the main file.
A naive editing environment
(such as a text editor with an option
to have the current file processed by \LaTeX)
may require one to switch to the main file before compiling;
attempting to compile the child file produces errors.
\item
The main file must be modified (each time)
to adjust the |\includeonly| command
to the present needs. This easily leaves the main file in a messy state.
\item
The generated document will always carry the filename
of the main document. This is inconvenient if
several child files are to be compiled and
to be kept for distribution.
\end{itemize}

The present package provides a simple interface
to make child files individually compilable by \LaTeX{}.
Compiling a child file then has the same effect as compiling
the main file with an |\includeonly| command
to select the appropriate child.
Moreover the generated document will carry the name of the child
rather than the main file.
This resolves all three above issues.

This feature is meant to make the editing of books,
thesis documents and lecture notes somewhat more convenient.
However, the package can also be used efficiently for
composing a series of documents (such as exercise sheets)
which are typically distributed individually.
It then assists the author in generating the individual documents
(potentially in different versions)
as well as a document containing the collected series.
Another application is in developing style files
or other kinds of included material
where compilation of the style file could redirect
to a sample or test file.

%%%%%%%%%%%%%%%%%%%%%%%%%%%%%%%%%%%%%%%%%%%%%%%%%%%%%%%%%%%%%%%%%%%%%%%%%%%%%%%%
%%%%%%%%%%%%%%%%%%%%%%%%%%%%%%%%%%%%%%%%%%%%%%%%%%%%%%%%%%%%%%%%%%%%%%%%%%%%%%%%
\section{Usage}

First of all, the package \textsf{childdoc} is \emph{not} a standard
\LaTeXe{} |.sty| style file! Therefore it needs to be invoked in
a non-standard way.

%%%%%%%%%%%%%%%%%%%%%%%%%%%%%%%%%%%%%%%%%%%%%%%%%%%%%%%%%%%%%%%%%%%%%%%%%%%%%%%%
\subsection{Included Files}
\label{sec:include}

%%%%%%%%%%%%%%%%%%%%%%%%%%%%%%%%%%%%%%%%
\DescribeMacro{\childdocmain}
To use the package, add the commands
\begin{center}
\begin{tabular}{l}
|\input{childdoc.def}|\\
|\childdocmain{}|\\
\end{tabular}
\end{center}
at the very top of the main \LaTeX{} file,
in particular \emph{before} the |\documentclass| statement!
The argument of |\childdocmain| should be left empty
(but it must be present).

%%%%%%%%%%%%%%%%%%%%%%%%%%%%%%%%%%%%%%%%
\DescribeMacro{\childdocof}
Furthermore, add the commands
\begin{center}
\begin{tabular}{l}
|\input{childdoc.def}|\\
|\childdocof{|\textit{main}|}|\\
\end{tabular}
\end{center}
at the top of every child file \textit{child}
which is included by |\include{|\textit{child}|}|
from within the main file
(or at least for those files to be compiled individually).
The argument \textit{main} must be the filename of the main file.

There are a couple of
considerations in setting up the main and child documents:

%%%%%%%%%%%%%%%%%%%%%%%%%%%%%%%%%%%%%%%%
\paragraph{Restrictions.}

Please note the following restrictions:
\begin{itemize}
\item
|\childdocmain| must be called with one argument \textit{main}
to ensure compatibility with earlier version of the package.
It must either be empty (|\childdocmain{}|)
or precisely match the filename of the main file in which it is specified.
See \secref{sec:detection} for further information.
\item
The filename \textit{main} must be specified without the |.tex| extension.
\item
The filename \textit{main} is case sensitive
(even in case-insensitive file systems)
due to internal string comparison.
\item
The argument \textit{main} should be fully expanded, it cannot be a macro.
\item
Subdirectories and special characters should be avoided in filenames.
\item
The command |\childdocmain{|\textit{main}|}| must be followed by a whitespace.
It should not be followed immediately by another command
or by a comment mark `|%|'.
This is because the \TeX{} parser reads the token immediately following
the argument of |\childdocmain| and puts it
at the beginning of every child section;
however, a white\-space is ignored.
\end{itemize}

%%%%%%%%%%%%%%%%%%%%%%%%%%%%%%%%%%%%%%%%
\paragraph{Content of Main File.}

It is advisable to place all content in the child files included by |\include|.
Any output contained in the main file will appear in all child documents
unless suppressed manually;
it cannot be suppressed automatically by the |\includeonly| directive
and thus should normally be avoided.
A method to include some content in the main file
by means of conditional processing is described in \secref{sec:conditional}.

%%%%%%%%%%%%%%%%%%%%%%%%%%%%%%%%%%%%%%%%
\paragraph{Page Numbering.}

When only a part of the document is compiled,
the appropriate numbering of pages
(as well as other status parameters)
is determined from the |.aux| files.
The latter contain information from previous passes.
However this information needs to propagate through
all intermediate child documents.
Therefore the page numbering in child documents may well
be inconsistent until the complete document is compiled at least once.

A useful (if unconventional) way to always ensure a consistent
page numbering is to restart the numbering in each child document
and denote the pages by `\textit{child}|.|\textit{page}'
where \textit{child} represents the chapter/section number of the child file.
This can be achieved by the command
|\numberwithin{page}{|\textit{child}|}|
of the \textsf{amsmath} package
where \textit{child} can be |chapter| or |section|
depending on the chosen structuring.
Alternatively, one can modify the macro |\thepage| appropriately
and reset the counter |page| at the start of each child file.

%%%%%%%%%%%%%%%%%%%%%%%%%%%%%%%%%%%%%%%%%%%%%%%%%%%%%%%%%%%%%%%%%%%%%%%%%%%%%%%%
\subsection{Conditional Processing}
\label{sec:conditional}

The package provides a mechanism to compile different versions
of a document. To customise the versions further some conditional processing
can come in handy to distinguish which version is being compiled.
The package provides two macros to describe the compilation context:

%%%%%%%%%%%%%%%%%%%%%%%%%%%%%%%%%%%%%%%%
\DescribeMacro{\ifchilddoc}
The conditional |\ifchilddoc| distinguishes between the compilation of
child documents and the main document:
%
\begin{center}
|\ifchilddoc |\textit{child-code}| |[|\||else |\textit{main-code}]| \||fi|
\end{center}

%%%%%%%%%%%%%%%%%%%%%%%%%%%%%%%%%%%%%%%%
\DescribeMacro{\childdocname}
\DescribeMacro{\childdocjob}
The macro |\childdocname| contains the filename (without extension)
of the main or child file being processed.
Note that |\childdocjob| will always contain the name of the main file.

%%%%%%%%%%%%%%%%%%%%%%%%%%%%%%%%%%%%%%%%
\paragraph{Title Page.}

Conditional processing can be used to include a title or banner page
in the main document when proper precautions are taken.
Importantly, the code in the main file should ensure that the page counter
(as well as other status parameters which are stored in the |.aux| files)
takes the same value after the conditional processing.
Otherwise the page numbers may take divergent values
depending on which part is compiled.

For example, a title page could be declared by:
%
\begin{center}
\begin{tabular}{l}
|\ifchilddoc\||else|\\
|\addtocounter{page}{-1}|\\
\textit{code for title page}\\
|\newpage|\\
|\||fi|
\end{tabular}
\end{center}
%
A banner page for the child documents can be generated by:
%
\begin{center}
\begin{tabular}{l}
|\ifchilddoc|\\
|\addtocounter{page}{-1}|\\
\textit{code for banner page}\\
|\newpage|\\
|\||fi|
\end{tabular}
\end{center}
%
Here one could write a message such as:
\begin{center}
|This is the part \childdocname{} of \childdocjob{}.|
\end{center}

%%%%%%%%%%%%%%%%%%%%%%%%%%%%%%%%%%%%%%%%%%%%%%%%%%%%%%%%%%%%%%%%%%%%%%%%%%%%%%%%
\subsection{Flags}
\label{sec:flags}

The package makes it easy to generate different versions
of the main or child documents.
To this end compilation flags can be defined
and assigned different default values.
They will be particularly useful in conjunction
with the forwarding mechanism described in \secref{sec:forward}.

For example, it may be useful to have a flag |\version|
which can be set to |draft| or |final|.
The document source will contain some conditional code
depending on the value of |\version|.
Suppose further, the flag should default to |final| for the main file
and to |draft| for child files
which is a natural assignment for editing the document.
This is achieved by placing the following code
in the preamble of the main document
(below the |\childdocmain| directive):
%
\begin{center}
\begin{tabular}{l}
|\ifchilddoc|\\
|\providecommand{\version}{draft}|\\
|\||else|\\
|\providecommand{\version}{final}|\\
|\||fi|
\end{tabular}
\end{center}
%
The definition by |\providecommand| makes sure
that previous definitions are not overwritten.
Further statements |\providecommand{\version}{...}|
can thus be added before the above code to override it.

For the main file, one might add a line
(between |\childdocmain| and the above block)
%
\begin{center}
|%\ifchilddoc\||else\providecommand{\version}{draft}\||fi|
\end{center}
%
which can be uncommented to produce a draft version.
Likewise one can add a line to the very top of a child file
(above the |\childdocof{|\textit{main}|}| directive)
%
\begin{center}
|%\providecommand{\version}{final}|
\end{center}
%
which can be uncommented to produce the final version of this child document.

%%%%%%%%%%%%%%%%%%%%%%%%%%%%%%%%%%%%%%%%%%%%%%%%%%%%%%%%%%%%%%%%%%%%%%%%%%%%%%%%
\subsection{Forwarding}
\label{sec:forward}

Different versions of the main or child documents
using compilation flags as described in \secref{sec:flags}
can be (permanently) stored in different files
for convenient compilation, viewing and distribution.
To this end, the package defines a command
to pass on compilation to a different file:

%%%%%%%%%%%%%%%%%%%%%%%%%%%%%%%%%%%%%%%%
\DescribeMacro{\childdocforward}
The command |\childdocforward| redirects processing to
another source file:
%
\begin{center}
\begin{tabular}{l}
|\input{childdoc.def}|\\
|\childdocforward[|\textit{main}|]{|\textit{dest}|}|\\
\end{tabular}
\end{center}
%
The argument \textit{dest} is the destination file
(without extension).
It should be the main file or one of the child files.
Note that further \textsf{childdoc} directives
such as |\childdocof| and |\childdocforward|
in the indicated file will be processed in this form.
The optional argument \textit{main}
passes on directly to the main file \textit{main}
while pretending to compile the child \textit{dest}.
This form behaves as if \textit{dest}
issues |\childdocof{|\textit{main}|}| right away,
and no further \textsf{childdoc} directives will be processed.

%%%%%%%%%%%%%%%%%%%%%%%%%%%%%%%%%%%%%%%%
\DescribeMacro{\...prefix}
In the alternative form |\childdocforwardprefix|,
%
\begin{center}
\begin{tabular}{l}
|\input{childdoc.def}|\\
|\childdocforwardprefix[|\textit{main}|]{|\textit{prefix}|}{|\textit{dest}|}|
\end{tabular}
\end{center}
%
the destination file is determined by a pattern
depending on the current file:
To make this work, the current file must be called
`{\textit{prefix}\hspace{0.2em}\textit{suffix}}'
with \textit{prefix} matching precisely the argument.
Processing is then passed on to the file
`{\textit{dest}\hspace{0.2em}\textit{suffix}}'.
Surely, the same effect is achieved by
directly specifying the
argument `{\textit{dest}\hspace{0.2em}\textit{suffix}}'
in the first form.
However, that requires to set up a different file
for each child. With the alternative form of the command
all these files can have exactly the same content
which simplifies setting them up and maintaining them.

For example, the following file |draft.tex|
with a compilation flag |\version| as described in \secref{sec:flags}
compiles the main document as a draft:
%
\begin{center}
\begin{tabular}{l}
|\def\version{draft}|\\
|\input{childdoc.def}|\\
|\childdocforward{|\textit{main}|}|
\end{tabular}
\end{center}
%
Likewise, the following files |final|\textit{nn}|.tex|
compile the final version of the child document
|child|\textit{nn}|.tex|:
%
\begin{center}
\begin{tabular}{l}
|\def\version{final}|\\
|\input{childdoc.def}|\\
|\childdocforwardprefix{final}{child}|
\end{tabular}
\end{center}
%

Note that when several versions of a main file and/or of each child file
are to be generated, it may be convenient to set up a |Makefile| or
shell script to automatise the process.

%%%%%%%%%%%%%%%%%%%%%%%%%%%%%%%%%%%%%%%%%%%%%%%%%%%%%%%%%%%%%%%%%%%%%%%%%%%%%%%%
\subsection{Command Line Processing}
\label{sec:commandline}

The effect of redirection files can also be achieved by invoking
the \LaTeX{} compiler with a more elaborate command line.
Most conveniently this should be done as part
of a shell script or a |Makefile|.

When using \textsf{childdoc} in the main file, the following
command lines effectively perform a redirection
(note that depending on the shell being used,
backslashes may have to be doubled: `|\|' $\to$ `|\\|'):
%
\begin{center}
|... -jobname "|\textit{target}|" |\\|"|[\textit{flags}]%
|\input{childdoc.def}\childdocforward[|\textit{main}|]{|\textit{dest}|}"|
\end{center}
%
Here \textit{target} is the name of the output file,
\textit{main} is the name of the main file
and \textit{dest} is the name of the main or child file to be processed
(all filenames without extensions).
The optional argument \textit{main} can be omitted
if \textit{main} matches \textit{dest}.
Optionally, compilation \textit{flags} can be defined via |\def| commands.
This command line makes the \TeX{} engine believe
it is compiling the file \textit{target}
whose content is specified as the latter parameter.
The provided code then forwards the processing to
\textit{main} or \textit{dest} as described in \secref{sec:forward}.

%%%%%%%%%%%%%%%%%%%%%%%%%%%%%%%%%%%%%%%%%%%%%%%%%%%%%%%%%%%%%%%%%%%%%%%%%%%%%%%%
\subsection{Include by Input}
\label{sec:input}

Including child documents by |\include| has some restrictions by design.
Most notably, the content of a child document always occupies
its own set of pages; pages cannot be shared between child documents.
Usually, this behaviour makes perfect sense
because each child document contain an essential part of the document.
However, in some situations it may be desirable to compose
a document from a collection of parts
without having mandatory page breaks between then.
For this case, the package
provides a mechanism to include parts
by |\input| which can also be processed individually.
However, by construction this mechanism
requires manual handling of the content to be output.

%%%%%%%%%%%%%%%%%%%%%%%%%%%%%%%%%%%%%%%%
\DescribeMacro{\ifchilddocmanual}
The main file should be prepared as usual, see \secref{sec:include}.
However, the document body must make a distinction
between processing of an individual part and of the main document, e.g.:
%
\begin{center}
\begin{tabular}{l}
|\ifchilddocmanual|\\
|\input{\childdocname}|\\
|\||else|\\
\textit{document body with }|\input{|\textit{part}|}|\\
|\||fi|
\end{tabular}
\end{center}
%
The conditional |\ifchilddocmanual| is true whenever
a part to be included by |\input| is being compiled,
and the name of the part is stored in |\childdocname|.

%%%%%%%%%%%%%%%%%%%%%%%%%%%%%%%%%%%%%%%%
\DescribeMacro{\childdocby}
Each part to be included by |\input| should start with:
%
\begin{center}
\begin{tabular}{l}
|\input{childdoc.def}|\\
|\childdocby{|\textit{main}|}|\\
\end{tabular}
\end{center}
%
The directive |\childdocby| is similar to |\childdocof|
described in \secref{sec:include},
but the subsequent selection of content must be done manually.
To that end, both |\ifchilddoc| and |\ifchilddocmanual|
will be true upon processing of a part,
and the name of the part is stored in |\childdocname|.
Note that |\jobname| will be set to the filename of the current part
so that each part receives an individual |.aux| file
that does not interfere with the |.aux| file(s) of the main document.
This behaviour can be altered by the alternative form
|\childdocby[*]{|\textit{main}|}| (with a non-empty optional argument)
which uses the |.aux| file of the main document
by setting |\jobname| to \textit{main}.

%%%%%%%%%%%%%%%%%%%%%%%%%%%%%%%%%%%%%%%%%%%%%%%%%%%%%%%%%%%%%%%%%%%%%%%%%%%%%%%%
\subsection{Driver Development}
\label{sec:driver}

The \textsf{childdoc} mechanism can also be use for the development
of definition files such as \LaTeX{} styles or classes.
This case differs from the above setup with multiple parts
included by |\include| in that no |\includeonly| should be invoked.
This can be achieved by starting the include file
(before |\ProvidesPackage|) with:
%
\begin{center}
\begin{tabular}{l}
|\input{childdoc.def}|\\
|\childdocforward{|\textit{main}|}|\\
\end{tabular}
\end{center}
%
or alternatively with:
%
\begin{center}
\begin{tabular}{l}
|\input{childdoc.def}|\\
|\childdocby{|\textit{main}|}|\\
\end{tabular}
\end{center}
%
Both forms have slightly different effects as described above.
The main file is prepared as usual, see \secref{sec:include}.

%%%%%%%%%%%%%%%%%%%%%%%%%%%%%%%%%%%%%%%%%%%%%%%%%%%%%%%%%%%%%%%%%%%%%%%%%%%%%%%%
\subsection{Legacy Detection}
\label{sec:detection}

The directive |\childdocmain| in the main file can detect
whether the complete document or merely a child is to be compiled
even without using the directive |\childdocof|.
This method is deprecated because it is less robust
and there is no compelling reason to use it;
it is merely provided for backward compatibility
and it may be removed in future versions.

If the detection mechanism is to be used,
it is mandatory to correctly specify
the filename of the main file as the argument of |\childdocmain|:
%
\begin{center}
\begin{tabular}{l}
|\input{childdoc.def}|\\
|\childdocmain{|\textit{main}|}|\\
\end{tabular}
\end{center}
%
If |\jobname| does not match the argument \textit{main} of |\childdocmain|,
it is assumed that |\jobname| points to the child file to be compiled.
When using |\childdocmain| with the main file specified as argument,
it suffices to start a child file
with just |\input{|\textit{main}|}|
without loading of the package and using |\childdocof|.
If instead all processing is done
with the appropriate \textsf{childdoc} directives,
the argument of \textit{main} of |\childdocmain| can be empty.

An alternative version of the command line processing described
in \secref{sec:commandline} using the detection mechanism reads:
%
\begin{center}
|... -jobname "|\textit{target}|" "|[\textit{flags}]%
[|\def\jobname{|\textit{dest}|}|]|\input{|\textit{main}|}"|
\end{center}

%%%%%%%%%%%%%%%%%%%%%%%%%%%%%%%%%%%%%%%%%%%%%%%%%%%%%%%%%%%%%%%%%%%%%%%%%%%%%%%%
\subsection{Manual Code}
\label{sec:manual}

In case one cannot be certain whether the definitions file |childdoc.def|
is installed on the target \TeX{} distribution
and one prefers not to ship it,
it is conceivable to paste a few relevant commands into the sources.

To that end, drop all statements |\input{childdoc.def}|
and perform the replacements as outlined below.
Instead of |\childdocmain{|\textit{main}|}| add the following code
to the top of the main file:
%
\begin{center}
\begin{tabular}{l}
|\||ifdefined\childdocname\endinput\||fi\newif\ifchilddoc|\\
|\edef\childdocname{\scantokens\expandafter{\jobname\noexpand}}|\\
|\def\childdocmain{|\textit{main}|}\||ifx\childdocmain\childdocname\||else|\\
|\childdoctrue\includeonly{\childdocname}\let\jobname\childdocmain\||fi|\\
\end{tabular}
\end{center}
%
Instead of |\childdocof{|\textit{main}|}| just include the main file
at the top of each child file:
%
\begin{center}
|\input{|\textit{main}|}|
\end{center}
%
A simple redirection |\childdocforward{|\textit{dest}|}| is achieved by:
%
\begin{center}
|\def\jobname{|\textit{dest}|}\input{\jobname}|
\end{center}
%
The redirection with prefix
|\childdocforwardprefix[|\textit{prefix}|]{|\textit{dest}|}|
is accomplished by:
%
\begin{center}
\begin{tabular}{l}
|{\edef\jobname{\scantokens\expandafter{\jobname\noexpand}}|\\
|\def\redirectjob |\textit{prefix}|#1~~~{\gdef\jobname{|\textit{dest}|#1}}|\\
|\expandafter\redirectjob\jobname~~~}\input{\jobname}|
\end{tabular}
\end{center}

In an alternative approach,
child documents can be compiled by a specific command line
without additional code or specific definitions:
%
\begin{center}
|... -jobname "|\textit{target}|" "|[\textit{flags}]%
|\includeonly{|\textit{dest}|}\input{|\textit{main}|}"|
\end{center}
%

%%%%%%%%%%%%%%%%%%%%%%%%%%%%%%%%%%%%%%%%%%%%%%%%%%%%%%%%%%%%%%%%%%%%%%%%%%%%%%%%
%%%%%%%%%%%%%%%%%%%%%%%%%%%%%%%%%%%%%%%%%%%%%%%%%%%%%%%%%%%%%%%%%%%%%%%%%%%%%%%%
\section{Information}

%%%%%%%%%%%%%%%%%%%%%%%%%%%%%%%%%%%%%%%%%%%%%%%%%%%%%%%%%%%%%%%%%%%%%%%%%%%%%%%%
\subsection{Copyright}

Copyright \copyright{} 2017--2018 Niklas Beisert

This work may be distributed and/or modified under the
conditions of the \LaTeX{} Project Public License, either version 1.3
of this license or (at your option) any later version.
The latest version of this license is in
  \url{http://www.latex-project.org/lppl.txt}
and version 1.3 or later is part of all distributions of \LaTeX{}
version 2005/12/01 or later.

This work has the LPPL maintenance status `maintained'.

The Current Maintainer of this work is Niklas Beisert.

This work consists of the files |README.txt|, |childdoc.ins| and |childdoc.dtx|
as well as the derived files |childdoc.def|, |cdocsamp.tex|
with |cdocsch1.tex|, |cdocsch2.tex|, |cdocspt3.tex|, |cdocspt4.tex|,
|cdocsdrf.tex|, |cdocsfn1.tex|, |cdocsfn2.tex|
as well as |childdoc.pdf|.

%%%%%%%%%%%%%%%%%%%%%%%%%%%%%%%%%%%%%%%%%%%%%%%%%%%%%%%%%%%%%%%%%%%%%%%%%%%%%%%%
\subsection{Files and Installation}

The package consists of the files:
%
\begin{center}
\begin{tabular}{ll}
    |README.txt|   & readme file \\
    |childdoc.ins| & installation file \\
    |childdoc.dtx| & source file \\
    |childdoc.def| & definition file \\
    |cdocsamp.tex| & sample main file \\
    |cdocsch1.tex| & sample include file \\
    |cdocsch2.tex| & sample include file \\
    |cdocspt3.tex| & sample part file \\
    |cdocspt4.tex| & sample part file \\
    |cdocsdrf.tex| & sample redirection file \\
    |cdocsfn1.tex| & sample redirection file \\
    |cdocsfn2.tex| & sample redirection file \\
    |childdoc.pdf| & manual
\end{tabular}
\end{center}
%
The distribution consists of the files
|README.txt|, |childdoc.ins| and |childdoc.dtx|.
%
\begin{itemize}
\item
Run (pdf)\LaTeX{} on |childdoc.dtx|
to compile the manual |childdoc.pdf| (this file).
\item
Run \LaTeX{} on |childdoc.ins| to create the definitions file |childdoc.def|
and the sample |cdocsamp.tex| with include files
|cdocsch1.tex|, |cdocsch2.tex|, |cdocspt3.tex|, |cdocspt4.tex|,
|cdocsdrf.tex|, |cdocsfn1.tex|, |cdocsfn2.tex|.
Then copy the file |childdoc.def| to an appropriate directory of your \LaTeX{}
distribution, e.g.\ \textit{texmf-root}|/tex/latex/childdoc|.
\end{itemize}

%%%%%%%%%%%%%%%%%%%%%%%%%%%%%%%%%%%%%%%%%%%%%%%%%%%%%%%%%%%%%%%%%%%%%%%%%%%%%%%%
\subsection{Related CTAN Packages}

There are several other packages which offer a similar functionality:
%
\begin{itemize}
\item
The packages
\href{http://ctan.org/pkg/docmute}{\textsf{docmute}},
\href{http://ctan.org/pkg/includex}{\textsf{includex}} and
\href{http://ctan.org/pkg/standalone}{\textsf{standalone}}
provide commands to include only the document body of
a child file thus allowing both files to be compiled individually.
\item
The packages \href{http://ctan.org/pkg/subdocs}{\textsf{subdocs}}
and \href{http://ctan.org/pkg/subfiles}{\textsf{subfiles}}
provide structures in which the main and child documents can be
encapsulated and allowing them to be compiled individually.
The inclusion mechanism is different from the conventional |\include|.
\item
The package \href{http://ctan.org/pkg/combine}{\textsf{combine}}
is an elaborate solution to combine several documents into one.
\end{itemize}
%
See also the CTAN topic \href{http://ctan.org/topic/subdocs}{\textsf{subdocs}}
for further related packages.
The present package differs from the above solutions in that
a document structure constructed with the conventional |\include| mechanism
just needs two extra commands at the top of every file
such that all constituent files can be compiled individually.

%%%%%%%%%%%%%%%%%%%%%%%%%%%%%%%%%%%%%%%%%%%%%%%%%%%%%%%%%%%%%%%%%%%%%%%%%%%%%%%%
%\subsection{Feature Suggestions}
%
%The following is a list of features which may be useful for future
%versions of this package:
%%
%\begin{itemize}
%\item
%\ldots
%\end{itemize}

%%%%%%%%%%%%%%%%%%%%%%%%%%%%%%%%%%%%%%%%%%%%%%%%%%%%%%%%%%%%%%%%%%%%%%%%%%%%%%%%
\subsection{Revision History}

%%%%%%%%%%%%%%%%%%%%%%%%%%%%%%%%%%%%%%%%
\paragraph{v2.0:} 2018/12/30

\begin{itemize}
\item
immediate forward processing
\item
added |\childdocby| mechanism
\item
manual restructured
\end{itemize}

%%%%%%%%%%%%%%%%%%%%%%%%%%%%%%%%%%%%%%%%
\paragraph{v1.6:} 2018/01/17

\begin{itemize}
\item
application for development of include files
\item
corrections to manual
\end{itemize}

%%%%%%%%%%%%%%%%%%%%%%%%%%%%%%%%%%%%%%%%
\paragraph{v1.5:} 2017/05/21

\begin{itemize}
\item
more complete structuring introduced
\item
|\childdocof| introduced
\item
|\childdoc| renamed to |\childdocmain|
\item
|\childredirect| renamed to |\childdocforward| and |\childdocforwardprefix|
and functionality expanded
\end{itemize}

%%%%%%%%%%%%%%%%%%%%%%%%%%%%%%%%%%%%%%%%
\paragraph{v1.0:} 2017/04/27

\begin{itemize}
\item
manual and install package
\item
first version published on CTAN
\end{itemize}

%%%%%%%%%%%%%%%%%%%%%%%%%%%%%%%%%%%%%%%%
\paragraph{v0.6:} 2017/04/26

\begin{itemize}
\item
redirection mechanism added
\end{itemize}

%%%%%%%%%%%%%%%%%%%%%%%%%%%%%%%%%%%%%%%%
\paragraph{v0.5:} 2017/04/26

\begin{itemize}
\item
functionality in definition file
\end{itemize}


%%%%%%%%%%%%%%%%%%%%%%%%%%%%%%%%%%%%%%%%%%%%%%%%%%%%%%%%%%%%%%%%%%%%%%%%%%%%%%%%
%%%%%%%%%%%%%%%%%%%%%%%%%%%%%%%%%%%%%%%%%%%%%%%%%%%%%%%%%%%%%%%%%%%%%%%%%%%%%%%%
%%%%%%%%%%%%%%%%%%%%%%%%%%%%%%%%%%%%%%%%%%%%%%%%%%%%%%%%%%%%%%%%%%%%%%%%%%%%%%%%
\appendix

\settowidth\MacroIndent{\rmfamily\scriptsize 000\ }

 \DocInput{childdoc.dtx}

\end{document}
%</driver>
% \fi
%
% %%%%%%%%%%%%%%%%%%%%%%%%%%%%%%%%%%%%%%%%%%%%%%%%%%%%%%%%%%%%%%%%%%%%%%%%%%%%%%
% %%%%%%%%%%%%%%%%%%%%%%%%%%%%%%%%%%%%%%%%%%%%%%%%%%%%%%%%%%%%%%%%%%%%%%%%%%%%%%
% \section{Sample}
%\iffalse
%<*samplemain>
%\fi
%
% The following presents a sample document
% with two chapters, two parts, a title page,
% a compile flag as well as three forwarding files to set the flag.
% It consists of eight |.tex| files:
% \begin{center}
% \begin{tabular}{ll}
% |cdocsamp.tex|&main file\\
% |cdocsch1.tex|&include file for chapter 1\\
% |cdocsch2.tex|&include file for chapter 2\\
% |cdocspt3.tex|&include file for part 3\\
% |cdocspt4.tex|&include file for part 4\\
% |cdocsdrf.tex|&forwarding file for main file in draft mode\\
% |cdocsfi1.tex|&forwarding file for final version of chapter 1\\
% |cdocsfi2.tex|&forwarding file for final version of chapter 2\\
% \end{tabular}
% \end{center}
% Each of the eight files can be compiled directly by the \LaTeX{} compiler.
%
% %%%%%%%%%%%%%%%%%%%%%%%%%%%%%%%%%%%%%%
% \paragraph{Main File.}
%
% The main file is called |cdocsamp.tex|.
%
% Load the \textsf{childdoc} definitions and
% declare the filename for the main document:
%    \begin{macrocode}
\input{childdoc.def}
\childdocmain{}
%    \end{macrocode}

% Optional override for |\version| flag:
%    \begin{macrocode}
%%\ifchilddoc\else\providecommand{\version}{draft}\fi
%    \end{macrocode}

% Define the default values for the |\version| flag
% (|final| for the main file and |draft| for childs):
%    \begin{macrocode}
\ifchilddoc
\providecommand{\version}{draft}
\else
\providecommand{\version}{final}
\fi
%    \end{macrocode}

% Load the standard document class:
%    \begin{macrocode}
\documentclass[12pt]{article}
%    \end{macrocode}

% Start the document body:
%    \begin{macrocode}
\begin{document}
%    \end{macrocode}

% Declare a title page.
% Print title, part of document being processed and version flag:
%    \begin{macrocode}
\addtocounter{page}{-1}
\begin{center}
{\LARGE\bfseries{}childdoc example\par}
\vspace{1cm}
\ifchilddoc
\ifchilddocmanual part\else chapter\fi:
`\childdocname' of `\childdocjob'\par
\else
main document: `\childdocjob'\par
\fi
version: \version\par
\end{center}
\newpage
%    \end{macrocode}

% Manually include selected file,
% otherwise process as usual:
%    \begin{macrocode}
\ifchilddocmanual
\section*{part `\childdocname'}
\input{\childdocname}
\else
%    \end{macrocode}

% Include the two chapters:
%    \begin{macrocode}
\include{cdocsch1}
\include{cdocsch2}
%    \end{macrocode}

% Include the two parts unless only chapters should be displayed:
%    \begin{macrocode}
\ifchilddoc\else
\section{part three}
\input{cdocspt3}
\section{part four}
\input{cdocspt4}
\fi
%    \end{macrocode}

% Process as usual until here:
%    \begin{macrocode}
\fi
%    \end{macrocode}

% End of document body:
%    \begin{macrocode}
\end{document}
%    \end{macrocode}
%\iffalse
%</samplemain>
%\fi
%
% %%%%%%%%%%%%%%%%%%%%%%%%%%%%%%%%%%%%%%
% \paragraph{Chapter Include Files.}
%
% The include files are called |cdocsch1.tex| and |cdocsch2.tex|.
%
%\iffalse
%<*samplechap1|samplechap2>
%\fi

% Optional override for |\version| flag:
%    \begin{macrocode}
%%\providecommand{\version}{final}
%    \end{macrocode}

% Include the main document:
%    \begin{macrocode}
\input{childdoc.def}
\childdocof{cdocsamp}
%    \end{macrocode}

%\iffalse
%</samplechap1|samplechap2>
%\fi
%
%\iffalse
%<*samplechap1>
%\fi
% Some text for chapter 1:
%    \begin{macrocode}
\section{one}
some text in chapter one
%    \end{macrocode}

%\iffalse
%</samplechap1>
%\fi
% Some text for chapter 2:
%\iffalse
%<*samplechap2>
%\fi
%    \begin{macrocode}
\section{two}
more text in chapter two
%    \end{macrocode}

%\iffalse
%</samplechap2>
%\fi
%
% %%%%%%%%%%%%%%%%%%%%%%%%%%%%%%%%%%%%%%
% \paragraph{Part Include Files.}
%
% The include files are called |cdocspt3.tex| and |cdocspt4.tex|.
%
%\iffalse
%<*samplepart3|samplepart4>
%\fi

% Optional override for |\version| flag:
%    \begin{macrocode}
%%\providecommand{\version}{final}
%    \end{macrocode}

% Include the main document:
%    \begin{macrocode}
\input{childdoc.def}
\childdocby{cdocsamp}
%    \end{macrocode}

%\iffalse
%</samplepart3|samplepart4>
%\fi
%
%\iffalse
%<*samplepart3>
%\fi
% Some text for part 3:
%    \begin{macrocode}
some text in part three
%    \end{macrocode}

%\iffalse
%</samplepart3>
%\fi
% Some text for part 4:
%\iffalse
%<*samplepart4>
%\fi
%    \begin{macrocode}
more text in part four
%    \end{macrocode}

%\iffalse
%</samplepart4>
%\fi
%
% %%%%%%%%%%%%%%%%%%%%%%%%%%%%%%%%%%%%%%
% \paragraph{Forwarding for a Complete Draft.}
%
% The following forwarding file |cdocsdrf.tex|
% compiles the main document in draft mode:
%\iffalse
%<*sampledraft>
%\fi
%    \begin{macrocode}
\def\version{draft}
\input{childdoc.def}
\childdocforward{cdocsamp}
%    \end{macrocode}

%\iffalse
%</sampledraft>
%\fi
%
% %%%%%%%%%%%%%%%%%%%%%%%%%%%%%%%%%%%%%%
% \paragraph{Forwarding for Final Version of the Chapters.}
%
% The following forwarding files |cdocsfn1.tex| and |cdocsfn2.tex|
% (with identical content)
% compile the final versions of the child documents
% |cdocsch1.tex| and |cdocsch2.tex|, respectively:
%\iffalse
%<*samplefinal>
%\fi
%    \begin{macrocode}
\def\version{final}
\input{childdoc.def}
\childdocforwardprefix[cdocsamp]{cdocsfn}{cdocsch}
%    \end{macrocode}

%\iffalse
%</samplefinal>
%\fi
%
% %%%%%%%%%%%%%%%%%%%%%%%%%%%%%%%%%%%%%%
% \paragraph{Command Line Processing.}
%
% The following three command lines generate the output files
% |cdocscld|, |cdocscl1| and |cdocscl2|
% which should be identical to
% |cdocsdrf|, |cdocsch1| and |cdocsfn2|, respectively:
% \begin{center}
% \begin{tabular}{l}
% |latex -jobname cdocscld \|\\
% |  "\def\version{draft}\input{childdoc.def}\childdocforward{cdocsamp}"|\\
% |latex -jobname cdocscl1 \|\\
% |  "\input{childdoc.def}\childdocforward[cdocsamp]{cdocsch1}"|\\
% |latex -jobname cdocscl2 \|\\
% |  "\def\version{final}\input{childdoc.def}\childdocforward{cdocsch2}"|
% \end{tabular}
% \end{center}
% Note that the trailing backslash on each first line
% merely continues the input to the second line
% (for convenient cut ant paste).
% Furthermore, the command |latex| can be replaced by any
% of its alternative versions such as |pdflatex|.
%
% %%%%%%%%%%%%%%%%%%%%%%%%%%%%%%%%%%%%%%%%%%%%%%%%%%%%%%%%%%%%%%%%%%%%%%%%%%%%%%
% %%%%%%%%%%%%%%%%%%%%%%%%%%%%%%%%%%%%%%%%%%%%%%%%%%%%%%%%%%%%%%%%%%%%%%%%%%%%%%
% \section{Implementation}
%\iffalse
%<*package>
%\fi
%
% This section describes the definitions file |childdoc.def|.

% The definitions cannot be loaded using |\usepackage| or |\RequirePackage|
% which has a mechanism to prevent loading a style file more than once.
% When loading the definitions by means of |\input|
% multiple instances have to be prevented manually:
%\iffalse
%This code needs to be before the `\ProvidesFile' directive
%which is defined at the beginning of this file.
%Therefore it is also placed there and commented out here.
%</package>
%<*discard>
%\fi
%    \begin{macrocode}
\ifdefined\childdocmain\endinput\fi
%    \end{macrocode}
%\iffalse
%</discard>
%<*package>
%\fi
%
% \macro{\ifchilddoc}
% \macro{\ifchilddocmanual}
% The conditional |\ifchilddoc| tells whether a
% child (true) or main (false) document is being compiled.
% The conditional |\ifchilddocmanual| tells whether
% the |\includeonly| mechanism is used (false) or
% the selection of child files must be performed manually (true).
% The definitions initialise to false:
%    \begin{macrocode}
\newif\ifchilddoc
\newif\ifchilddocmanual
%    \end{macrocode}

% \macro{\childdocname}
% \macro{\childdocjob}
% The macro |\childdocname| stores the name of the main document
% to be compiled. The macro |\childdocjob| stores the name of
% the document on which the \LaTeX{} compiler was originally invoked.
% The content of |\jobname| cannot be compared
% to filenames specified in the source due to different catcodes.
% The following code rescans |\jobname|, stores the result
% in |\childdocname| and saves a copy in |\childdocjob|:
%    \begin{macrocode}
\edef\childdocname{\scantokens\expandafter{\jobname\noexpand}}
\let\childdocjob\childdocname
%    \end{macrocode}

% \macro{\childdocdisable}
% The macro |\childdocdisable| prevents the main file
% from being processed more than once.
% At this stage, the main document command |\childdocmain|
% is assumed to be called once again where it should do nothing.
% Any subsequent call to it should prevent
% a secondary processing of the main document
% It overwrites the forwarding commands
% |\childdocof| and |\childdocforward|
% with empty macros to prevent further inclusions of the main document:
%    \begin{macrocode}
\newcommand{\childdocdisable}
{
  \renewcommand{\childdocmain}[1]{\renewcommand{\childdocmain}[1]{\endinput}}
  \renewcommand{\childdocof}[1]{}
  \renewcommand{\childdocby}[2][]{}
  \renewcommand{\childdocforward}[2][]{}
  \renewcommand{\childdocdisable}{}
}
%    \end{macrocode}

% \macro{\childdocmain}
% The macro |\childdocmain| is to be called at the top of the main file
% with nothing or the main filename (without extension) as argument.
% First, it breaks loops.
% If the argument is not empty and does not match |\childdocname|
% (which is set by the first inclusion of |childdoc.def|),
% |\ifchilddoc| is set to true, |\includeonly| is applied to the child file
% and |\jobname| is set to the main file
% (for proper handling of |.aux| files):
%    \begin{macrocode}
\newcommand{\childdocmain}[1]
{
  \childdocdisable\childdocmain{}
  \if?#1?\else
    \begingroup
      \def\childdoctmp{#1}
      \ifx\childdoctmp\childdocname
        \def\childdoctmp{}
      \else
        \def\childdoctmp
        {
          \childdoctrue
          \includeonly{\childdocname}
          \def\childdocjob{#1}
          \def\jobname{#1}
        }
      \fi
      \expandafter
    \endgroup
    \childdoctmp
  \fi
}
%    \end{macrocode}

% \macro{\childdocof}
% The command |\childdocof| redirects
% compilation to the main file |#1|.
%    \begin{macrocode}
\newcommand{\childdocof}[1]
{
  \childdocdisable
  \childdoctrue
  \includeonly{\childdocname}
  \def\jobname{#1}
  \def\childdocjob{#1}
  \input{#1}
}
%    \end{macrocode}

% \macro{\childdocby}
% The command |\childdocby| ....
%    \begin{macrocode}
\newcommand{\childdocby}[2][]
{
  \childdocdisable
  \childdoctrue
  \childdocmanualtrue
  \if?#1?\else
    \def\jobname{#2}
  \fi
  \def\childdocjob{#2}
  \input{#2}
  \endinput
}
%    \end{macrocode}

% \macro{\childdocforward}
% The command |\childdocforward| redirects
% compilation to the main file or
% (if the optional argument is given) a child file.
% Parameters are set as if the main file
% or a child file starting with |\childdocof| was compiled.
% Then compilation is handed over to the main file:
%    \begin{macrocode}
\newcommand{\childdocforward}[2][]
{
  \begingroup
    \if?#1?
      \def\childdoctmp
      {
        \def\childdocname{#2}
        \def\childdocjob{#2}
        \def\jobname{#2}
        \input{#2}
        \endinput
      }
    \else
      \def\childdoctmp
      {
        \childdocdisable
        \def\childdocname{#2}
        \childdoctrue
        \includeonly{#2}
        \def\childdocjob{#1}
        \def\jobname{#1}
        \input{#1}
        \endinput
      }
    \fi
    \expandafter
  \endgroup
  \childdoctmp
}
%    \end{macrocode}

% \macro{\childdocforwardprefix}
% The command |\childdocforwardprefix| redirects
% compilation to the main or a child file by means of a pattern.
% The prefix |#1| in the current filename is replaced by |#2|
% and the suffix of the current filename is kept
% (it is assumed that the filename does not contain the substring `|~~~|'
% which is used as a delimiter).
% Compilation is handed over to the new file by |\childdocforward|:
%    \begin{macrocode}
\newcommand{\childdocforwardprefix}[3][]
{
  \begingroup
    \def\childdocextract #2##1~~~{\def\childdoctmp{\childdocforward[#1]{#3##1}}}
    \expandafter\childdocextract\childdocname~~~
    \expandafter
  \endgroup
  \childdoctmp
}
%    \end{macrocode}

% \macro{\childdoc}
% The deprecated macro |\childdoc| is a legacy version of |\childdocmain|:
%    \begin{macrocode}
\newcommand{\childdoc}{\childdocmain}
%    \end{macrocode}

% \macro{\childdocredirect}
% The deprecated macro |\childdocredirect| is a legacy version
% of |\childdocforward| and |\childdocforwardprefix|:
%    \begin{macrocode}
\newcommand{\childdocredirect}[2][]
{
  \begingroup
    \if?#1?
      \def\childdoctmp{\childdocforward{#2}}
    \else
      \def\childdoctmp{\childdocforwardprefix{#1}{#2}}
    \fi
    \expandafter
  \endgroup
  \childdoctmp
}
%    \end{macrocode}

%\iffalse
%</package>
%\fi
%
\endinput
|\\
|\childdocof{|\textit{main}|}|\\
\end{tabular}
\end{center}
at the top of every child file \textit{child}
which is included by |\include{|\textit{child}|}|
from within the main file
(or at least for those files to be compiled individually).
The argument \textit{main} must be the filename of the main file.

There are a couple of
considerations in setting up the main and child documents:

%%%%%%%%%%%%%%%%%%%%%%%%%%%%%%%%%%%%%%%%
\paragraph{Restrictions.}

Please note the following restrictions:
\begin{itemize}
\item
|\childdocmain| must be called with one argument \textit{main}
to ensure compatibility with earlier version of the package.
It must either be empty (|\childdocmain{}|)
or precisely match the filename of the main file in which it is specified.
See \secref{sec:detection} for further information.
\item
The filename \textit{main} must be specified without the |.tex| extension.
\item
The filename \textit{main} is case sensitive
(even in case-insensitive file systems)
due to internal string comparison.
\item
The argument \textit{main} should be fully expanded, it cannot be a macro.
\item
Subdirectories and special characters should be avoided in filenames.
\item
The command |\childdocmain{|\textit{main}|}| must be followed by a whitespace.
It should not be followed immediately by another command
or by a comment mark `|%|'.
This is because the \TeX{} parser reads the token immediately following
the argument of |\childdocmain| and puts it
at the beginning of every child section;
however, a white\-space is ignored.
\end{itemize}

%%%%%%%%%%%%%%%%%%%%%%%%%%%%%%%%%%%%%%%%
\paragraph{Content of Main File.}

It is advisable to place all content in the child files included by |\include|.
Any output contained in the main file will appear in all child documents
unless suppressed manually;
it cannot be suppressed automatically by the |\includeonly| directive
and thus should normally be avoided.
A method to include some content in the main file
by means of conditional processing is described in \secref{sec:conditional}.

%%%%%%%%%%%%%%%%%%%%%%%%%%%%%%%%%%%%%%%%
\paragraph{Page Numbering.}

When only a part of the document is compiled,
the appropriate numbering of pages
(as well as other status parameters)
is determined from the |.aux| files.
The latter contain information from previous passes.
However this information needs to propagate through
all intermediate child documents.
Therefore the page numbering in child documents may well
be inconsistent until the complete document is compiled at least once.

A useful (if unconventional) way to always ensure a consistent
page numbering is to restart the numbering in each child document
and denote the pages by `\textit{child}|.|\textit{page}'
where \textit{child} represents the chapter/section number of the child file.
This can be achieved by the command
|\numberwithin{page}{|\textit{child}|}|
of the \textsf{amsmath} package
where \textit{child} can be |chapter| or |section|
depending on the chosen structuring.
Alternatively, one can modify the macro |\thepage| appropriately
and reset the counter |page| at the start of each child file.

%%%%%%%%%%%%%%%%%%%%%%%%%%%%%%%%%%%%%%%%%%%%%%%%%%%%%%%%%%%%%%%%%%%%%%%%%%%%%%%%
\subsection{Conditional Processing}
\label{sec:conditional}

The package provides a mechanism to compile different versions
of a document. To customise the versions further some conditional processing
can come in handy to distinguish which version is being compiled.
The package provides two macros to describe the compilation context:

%%%%%%%%%%%%%%%%%%%%%%%%%%%%%%%%%%%%%%%%
\DescribeMacro{\ifchilddoc}
The conditional |\ifchilddoc| distinguishes between the compilation of
child documents and the main document:
%
\begin{center}
|\ifchilddoc |\textit{child-code}| |[|\||else |\textit{main-code}]| \||fi|
\end{center}

%%%%%%%%%%%%%%%%%%%%%%%%%%%%%%%%%%%%%%%%
\DescribeMacro{\childdocname}
\DescribeMacro{\childdocjob}
The macro |\childdocname| contains the filename (without extension)
of the main or child file being processed.
Note that |\childdocjob| will always contain the name of the main file.

%%%%%%%%%%%%%%%%%%%%%%%%%%%%%%%%%%%%%%%%
\paragraph{Title Page.}

Conditional processing can be used to include a title or banner page
in the main document when proper precautions are taken.
Importantly, the code in the main file should ensure that the page counter
(as well as other status parameters which are stored in the |.aux| files)
takes the same value after the conditional processing.
Otherwise the page numbers may take divergent values
depending on which part is compiled.

For example, a title page could be declared by:
%
\begin{center}
\begin{tabular}{l}
|\ifchilddoc\||else|\\
|\addtocounter{page}{-1}|\\
\textit{code for title page}\\
|\newpage|\\
|\||fi|
\end{tabular}
\end{center}
%
A banner page for the child documents can be generated by:
%
\begin{center}
\begin{tabular}{l}
|\ifchilddoc|\\
|\addtocounter{page}{-1}|\\
\textit{code for banner page}\\
|\newpage|\\
|\||fi|
\end{tabular}
\end{center}
%
Here one could write a message such as:
\begin{center}
|This is the part \childdocname{} of \childdocjob{}.|
\end{center}

%%%%%%%%%%%%%%%%%%%%%%%%%%%%%%%%%%%%%%%%%%%%%%%%%%%%%%%%%%%%%%%%%%%%%%%%%%%%%%%%
\subsection{Flags}
\label{sec:flags}

The package makes it easy to generate different versions
of the main or child documents.
To this end compilation flags can be defined
and assigned different default values.
They will be particularly useful in conjunction
with the forwarding mechanism described in \secref{sec:forward}.

For example, it may be useful to have a flag |\version|
which can be set to |draft| or |final|.
The document source will contain some conditional code
depending on the value of |\version|.
Suppose further, the flag should default to |final| for the main file
and to |draft| for child files
which is a natural assignment for editing the document.
This is achieved by placing the following code
in the preamble of the main document
(below the |\childdocmain| directive):
%
\begin{center}
\begin{tabular}{l}
|\ifchilddoc|\\
|\providecommand{\version}{draft}|\\
|\||else|\\
|\providecommand{\version}{final}|\\
|\||fi|
\end{tabular}
\end{center}
%
The definition by |\providecommand| makes sure
that previous definitions are not overwritten.
Further statements |\providecommand{\version}{...}|
can thus be added before the above code to override it.

For the main file, one might add a line
(between |\childdocmain| and the above block)
%
\begin{center}
|%\ifchilddoc\||else\providecommand{\version}{draft}\||fi|
\end{center}
%
which can be uncommented to produce a draft version.
Likewise one can add a line to the very top of a child file
(above the |\childdocof{|\textit{main}|}| directive)
%
\begin{center}
|%\providecommand{\version}{final}|
\end{center}
%
which can be uncommented to produce the final version of this child document.

%%%%%%%%%%%%%%%%%%%%%%%%%%%%%%%%%%%%%%%%%%%%%%%%%%%%%%%%%%%%%%%%%%%%%%%%%%%%%%%%
\subsection{Forwarding}
\label{sec:forward}

Different versions of the main or child documents
using compilation flags as described in \secref{sec:flags}
can be (permanently) stored in different files
for convenient compilation, viewing and distribution.
To this end, the package defines a command
to pass on compilation to a different file:

%%%%%%%%%%%%%%%%%%%%%%%%%%%%%%%%%%%%%%%%
\DescribeMacro{\childdocforward}
The command |\childdocforward| redirects processing to
another source file:
%
\begin{center}
\begin{tabular}{l}
|% \iffalse
%
% childdoc.dtx Copyright (C) 2017-2018 Niklas Beisert
%
% This work may be distributed and/or modified under the
% conditions of the LaTeX Project Public License, either version 1.3
% of this license or (at your option) any later version.
% The latest version of this license is in
%   http://www.latex-project.org/lppl.txt
% and version 1.3 or later is part of all distributions of LaTeX
% version 2005/12/01 or later.
%
% This work has the LPPL maintenance status `maintained'.
%
% The Current Maintainer of this work is Niklas Beisert.
%
% This work consists of the files childdoc.dtx and childdoc.ins
% and the derived files childdoc.def and cdocsamp.tex with
% cdocsch1.tex, cdocsch2.tex, cdocsdrf.tex, cdocsfn1.tex, cdocsfn2.tex.
%
%<package>\ifdefined\childdocmain\endinput\fi
%<package>\ProvidesFile{childdoc.def}[2018/12/30 v2.0 child document driver]
%<samplemain>\ProvidesFile{cdocsamp.tex}[2018/12/30 v2.0 sample for childdoc]
%<*driver>
%\ProvidesFile{childdoc.drv}[2018/12/30 v2.0 childdoc reference manual file]
\PassOptionsToClass{10pt,a4paper}{article}
\documentclass{ltxdoc}

\usepackage[margin=35mm]{geometry}
\usepackage{hyperref}
\usepackage{hyperxmp}
\usepackage[usenames]{color}

\hypersetup{colorlinks=true}
\hypersetup{pdfstartview=FitH}
\hypersetup{pdfpagemode=UseNone}
\hypersetup{pdfsource={}}
\hypersetup{pdflang={en-UK}}
\hypersetup{pdfcopyright={Copyright 2017-2018 Niklas Beisert.
  This work may be distributed and/or modified under the
  conditions of the LaTeX Project Public License, either version 1.3
  of this license or (at your option) any later version.}}
\hypersetup{pdflicenseurl={http://www.latex-project.org/lppl.txt}}
\hypersetup{pdfcontactaddress={ETH Zurich, ITP, HIT K,
  Wolfgang-Pauli-Strasse 27}}
\hypersetup{pdfcontactpostcode={8093}}
\hypersetup{pdfcontactcity={Zurich}}
\hypersetup{pdfcontactcountry={Switzerland}}
\hypersetup{pdfcontactemail={nbeisert@itp.phys.ethz.ch}}
\hypersetup{pdfcontacturl={http://people.phys.ethz.ch/\xmptilde nbeisert/}}

\newcommand{\secref}[1]{\hyperref[#1]{section \ref*{#1}}}

\parskip1ex
\parindent0pt
\let\olditemize\itemize
\def\itemize{\olditemize\parskip0pt}

\begin{document}

\title{The \textsf{childdoc} Package}
\hypersetup{pdftitle={The childdoc Package}}
\author{Niklas Beisert\\[2ex]
  Institut f\"ur Theoretische Physik\\
  Eidgen\"ossische Technische Hochschule Z\"urich\\
  Wolfgang-Pauli-Strasse 27, 8093 Z\"urich, Switzerland\\[1ex]
  \href{mailto:nbeisert@itp.phys.ethz.ch}
  {\texttt{nbeisert@itp.phys.ethz.ch}}}
\hypersetup{pdfauthor={Niklas Beisert}}
\hypersetup{pdfsubject={Manual for the LaTeX2e Package childdoc}}
\date{30 December 2018, \textsf{v2.0}}
\maketitle

\begin{abstract}\noindent
\textsf{childdoc} is a \LaTeXe{} package
that enables the direct compilation
of document sections included by |\include|
to individual files.
\end{abstract}

\begingroup
\parskip0ex
\tableofcontents
\endgroup

%%%%%%%%%%%%%%%%%%%%%%%%%%%%%%%%%%%%%%%%%%%%%%%%%%%%%%%%%%%%%%%%%%%%%%%%%%%%%%%%
%%%%%%%%%%%%%%%%%%%%%%%%%%%%%%%%%%%%%%%%%%%%%%%%%%%%%%%%%%%%%%%%%%%%%%%%%%%%%%%%
\section{Introduction}

\LaTeX{} provides a mechanism to structure a large document (such as a book)
into a main file and several child files (containing the chapters)
using the |\include| command.
This mechanism is beneficial for documents
which span hundreds of pages in order to
make the source file(s) more manageable.
Moreover, compilation can be restricted to
selected child files by means of the |\includeonly| command.
The latter feature can be used to reduce the compilation time while editing
(this was significantly more useful in the earlier days of \LaTeX{})
or to generate a smaller document which is easier to navigate.
Another application of |\includeonly| is to generate
documents consisting of selected parts of the complete document.

However, there are a few drawbacks of the plain |\include| mechanism:
\begin{itemize}
\item
The child files cannot be compiled on their own,
they can only be compiled via the main file.
A naive editing environment
(such as a text editor with an option
to have the current file processed by \LaTeX)
may require one to switch to the main file before compiling;
attempting to compile the child file produces errors.
\item
The main file must be modified (each time)
to adjust the |\includeonly| command
to the present needs. This easily leaves the main file in a messy state.
\item
The generated document will always carry the filename
of the main document. This is inconvenient if
several child files are to be compiled and
to be kept for distribution.
\end{itemize}

The present package provides a simple interface
to make child files individually compilable by \LaTeX{}.
Compiling a child file then has the same effect as compiling
the main file with an |\includeonly| command
to select the appropriate child.
Moreover the generated document will carry the name of the child
rather than the main file.
This resolves all three above issues.

This feature is meant to make the editing of books,
thesis documents and lecture notes somewhat more convenient.
However, the package can also be used efficiently for
composing a series of documents (such as exercise sheets)
which are typically distributed individually.
It then assists the author in generating the individual documents
(potentially in different versions)
as well as a document containing the collected series.
Another application is in developing style files
or other kinds of included material
where compilation of the style file could redirect
to a sample or test file.

%%%%%%%%%%%%%%%%%%%%%%%%%%%%%%%%%%%%%%%%%%%%%%%%%%%%%%%%%%%%%%%%%%%%%%%%%%%%%%%%
%%%%%%%%%%%%%%%%%%%%%%%%%%%%%%%%%%%%%%%%%%%%%%%%%%%%%%%%%%%%%%%%%%%%%%%%%%%%%%%%
\section{Usage}

First of all, the package \textsf{childdoc} is \emph{not} a standard
\LaTeXe{} |.sty| style file! Therefore it needs to be invoked in
a non-standard way.

%%%%%%%%%%%%%%%%%%%%%%%%%%%%%%%%%%%%%%%%%%%%%%%%%%%%%%%%%%%%%%%%%%%%%%%%%%%%%%%%
\subsection{Included Files}
\label{sec:include}

%%%%%%%%%%%%%%%%%%%%%%%%%%%%%%%%%%%%%%%%
\DescribeMacro{\childdocmain}
To use the package, add the commands
\begin{center}
\begin{tabular}{l}
|\input{childdoc.def}|\\
|\childdocmain{}|\\
\end{tabular}
\end{center}
at the very top of the main \LaTeX{} file,
in particular \emph{before} the |\documentclass| statement!
The argument of |\childdocmain| should be left empty
(but it must be present).

%%%%%%%%%%%%%%%%%%%%%%%%%%%%%%%%%%%%%%%%
\DescribeMacro{\childdocof}
Furthermore, add the commands
\begin{center}
\begin{tabular}{l}
|\input{childdoc.def}|\\
|\childdocof{|\textit{main}|}|\\
\end{tabular}
\end{center}
at the top of every child file \textit{child}
which is included by |\include{|\textit{child}|}|
from within the main file
(or at least for those files to be compiled individually).
The argument \textit{main} must be the filename of the main file.

There are a couple of
considerations in setting up the main and child documents:

%%%%%%%%%%%%%%%%%%%%%%%%%%%%%%%%%%%%%%%%
\paragraph{Restrictions.}

Please note the following restrictions:
\begin{itemize}
\item
|\childdocmain| must be called with one argument \textit{main}
to ensure compatibility with earlier version of the package.
It must either be empty (|\childdocmain{}|)
or precisely match the filename of the main file in which it is specified.
See \secref{sec:detection} for further information.
\item
The filename \textit{main} must be specified without the |.tex| extension.
\item
The filename \textit{main} is case sensitive
(even in case-insensitive file systems)
due to internal string comparison.
\item
The argument \textit{main} should be fully expanded, it cannot be a macro.
\item
Subdirectories and special characters should be avoided in filenames.
\item
The command |\childdocmain{|\textit{main}|}| must be followed by a whitespace.
It should not be followed immediately by another command
or by a comment mark `|%|'.
This is because the \TeX{} parser reads the token immediately following
the argument of |\childdocmain| and puts it
at the beginning of every child section;
however, a white\-space is ignored.
\end{itemize}

%%%%%%%%%%%%%%%%%%%%%%%%%%%%%%%%%%%%%%%%
\paragraph{Content of Main File.}

It is advisable to place all content in the child files included by |\include|.
Any output contained in the main file will appear in all child documents
unless suppressed manually;
it cannot be suppressed automatically by the |\includeonly| directive
and thus should normally be avoided.
A method to include some content in the main file
by means of conditional processing is described in \secref{sec:conditional}.

%%%%%%%%%%%%%%%%%%%%%%%%%%%%%%%%%%%%%%%%
\paragraph{Page Numbering.}

When only a part of the document is compiled,
the appropriate numbering of pages
(as well as other status parameters)
is determined from the |.aux| files.
The latter contain information from previous passes.
However this information needs to propagate through
all intermediate child documents.
Therefore the page numbering in child documents may well
be inconsistent until the complete document is compiled at least once.

A useful (if unconventional) way to always ensure a consistent
page numbering is to restart the numbering in each child document
and denote the pages by `\textit{child}|.|\textit{page}'
where \textit{child} represents the chapter/section number of the child file.
This can be achieved by the command
|\numberwithin{page}{|\textit{child}|}|
of the \textsf{amsmath} package
where \textit{child} can be |chapter| or |section|
depending on the chosen structuring.
Alternatively, one can modify the macro |\thepage| appropriately
and reset the counter |page| at the start of each child file.

%%%%%%%%%%%%%%%%%%%%%%%%%%%%%%%%%%%%%%%%%%%%%%%%%%%%%%%%%%%%%%%%%%%%%%%%%%%%%%%%
\subsection{Conditional Processing}
\label{sec:conditional}

The package provides a mechanism to compile different versions
of a document. To customise the versions further some conditional processing
can come in handy to distinguish which version is being compiled.
The package provides two macros to describe the compilation context:

%%%%%%%%%%%%%%%%%%%%%%%%%%%%%%%%%%%%%%%%
\DescribeMacro{\ifchilddoc}
The conditional |\ifchilddoc| distinguishes between the compilation of
child documents and the main document:
%
\begin{center}
|\ifchilddoc |\textit{child-code}| |[|\||else |\textit{main-code}]| \||fi|
\end{center}

%%%%%%%%%%%%%%%%%%%%%%%%%%%%%%%%%%%%%%%%
\DescribeMacro{\childdocname}
\DescribeMacro{\childdocjob}
The macro |\childdocname| contains the filename (without extension)
of the main or child file being processed.
Note that |\childdocjob| will always contain the name of the main file.

%%%%%%%%%%%%%%%%%%%%%%%%%%%%%%%%%%%%%%%%
\paragraph{Title Page.}

Conditional processing can be used to include a title or banner page
in the main document when proper precautions are taken.
Importantly, the code in the main file should ensure that the page counter
(as well as other status parameters which are stored in the |.aux| files)
takes the same value after the conditional processing.
Otherwise the page numbers may take divergent values
depending on which part is compiled.

For example, a title page could be declared by:
%
\begin{center}
\begin{tabular}{l}
|\ifchilddoc\||else|\\
|\addtocounter{page}{-1}|\\
\textit{code for title page}\\
|\newpage|\\
|\||fi|
\end{tabular}
\end{center}
%
A banner page for the child documents can be generated by:
%
\begin{center}
\begin{tabular}{l}
|\ifchilddoc|\\
|\addtocounter{page}{-1}|\\
\textit{code for banner page}\\
|\newpage|\\
|\||fi|
\end{tabular}
\end{center}
%
Here one could write a message such as:
\begin{center}
|This is the part \childdocname{} of \childdocjob{}.|
\end{center}

%%%%%%%%%%%%%%%%%%%%%%%%%%%%%%%%%%%%%%%%%%%%%%%%%%%%%%%%%%%%%%%%%%%%%%%%%%%%%%%%
\subsection{Flags}
\label{sec:flags}

The package makes it easy to generate different versions
of the main or child documents.
To this end compilation flags can be defined
and assigned different default values.
They will be particularly useful in conjunction
with the forwarding mechanism described in \secref{sec:forward}.

For example, it may be useful to have a flag |\version|
which can be set to |draft| or |final|.
The document source will contain some conditional code
depending on the value of |\version|.
Suppose further, the flag should default to |final| for the main file
and to |draft| for child files
which is a natural assignment for editing the document.
This is achieved by placing the following code
in the preamble of the main document
(below the |\childdocmain| directive):
%
\begin{center}
\begin{tabular}{l}
|\ifchilddoc|\\
|\providecommand{\version}{draft}|\\
|\||else|\\
|\providecommand{\version}{final}|\\
|\||fi|
\end{tabular}
\end{center}
%
The definition by |\providecommand| makes sure
that previous definitions are not overwritten.
Further statements |\providecommand{\version}{...}|
can thus be added before the above code to override it.

For the main file, one might add a line
(between |\childdocmain| and the above block)
%
\begin{center}
|%\ifchilddoc\||else\providecommand{\version}{draft}\||fi|
\end{center}
%
which can be uncommented to produce a draft version.
Likewise one can add a line to the very top of a child file
(above the |\childdocof{|\textit{main}|}| directive)
%
\begin{center}
|%\providecommand{\version}{final}|
\end{center}
%
which can be uncommented to produce the final version of this child document.

%%%%%%%%%%%%%%%%%%%%%%%%%%%%%%%%%%%%%%%%%%%%%%%%%%%%%%%%%%%%%%%%%%%%%%%%%%%%%%%%
\subsection{Forwarding}
\label{sec:forward}

Different versions of the main or child documents
using compilation flags as described in \secref{sec:flags}
can be (permanently) stored in different files
for convenient compilation, viewing and distribution.
To this end, the package defines a command
to pass on compilation to a different file:

%%%%%%%%%%%%%%%%%%%%%%%%%%%%%%%%%%%%%%%%
\DescribeMacro{\childdocforward}
The command |\childdocforward| redirects processing to
another source file:
%
\begin{center}
\begin{tabular}{l}
|\input{childdoc.def}|\\
|\childdocforward[|\textit{main}|]{|\textit{dest}|}|\\
\end{tabular}
\end{center}
%
The argument \textit{dest} is the destination file
(without extension).
It should be the main file or one of the child files.
Note that further \textsf{childdoc} directives
such as |\childdocof| and |\childdocforward|
in the indicated file will be processed in this form.
The optional argument \textit{main}
passes on directly to the main file \textit{main}
while pretending to compile the child \textit{dest}.
This form behaves as if \textit{dest}
issues |\childdocof{|\textit{main}|}| right away,
and no further \textsf{childdoc} directives will be processed.

%%%%%%%%%%%%%%%%%%%%%%%%%%%%%%%%%%%%%%%%
\DescribeMacro{\...prefix}
In the alternative form |\childdocforwardprefix|,
%
\begin{center}
\begin{tabular}{l}
|\input{childdoc.def}|\\
|\childdocforwardprefix[|\textit{main}|]{|\textit{prefix}|}{|\textit{dest}|}|
\end{tabular}
\end{center}
%
the destination file is determined by a pattern
depending on the current file:
To make this work, the current file must be called
`{\textit{prefix}\hspace{0.2em}\textit{suffix}}'
with \textit{prefix} matching precisely the argument.
Processing is then passed on to the file
`{\textit{dest}\hspace{0.2em}\textit{suffix}}'.
Surely, the same effect is achieved by
directly specifying the
argument `{\textit{dest}\hspace{0.2em}\textit{suffix}}'
in the first form.
However, that requires to set up a different file
for each child. With the alternative form of the command
all these files can have exactly the same content
which simplifies setting them up and maintaining them.

For example, the following file |draft.tex|
with a compilation flag |\version| as described in \secref{sec:flags}
compiles the main document as a draft:
%
\begin{center}
\begin{tabular}{l}
|\def\version{draft}|\\
|\input{childdoc.def}|\\
|\childdocforward{|\textit{main}|}|
\end{tabular}
\end{center}
%
Likewise, the following files |final|\textit{nn}|.tex|
compile the final version of the child document
|child|\textit{nn}|.tex|:
%
\begin{center}
\begin{tabular}{l}
|\def\version{final}|\\
|\input{childdoc.def}|\\
|\childdocforwardprefix{final}{child}|
\end{tabular}
\end{center}
%

Note that when several versions of a main file and/or of each child file
are to be generated, it may be convenient to set up a |Makefile| or
shell script to automatise the process.

%%%%%%%%%%%%%%%%%%%%%%%%%%%%%%%%%%%%%%%%%%%%%%%%%%%%%%%%%%%%%%%%%%%%%%%%%%%%%%%%
\subsection{Command Line Processing}
\label{sec:commandline}

The effect of redirection files can also be achieved by invoking
the \LaTeX{} compiler with a more elaborate command line.
Most conveniently this should be done as part
of a shell script or a |Makefile|.

When using \textsf{childdoc} in the main file, the following
command lines effectively perform a redirection
(note that depending on the shell being used,
backslashes may have to be doubled: `|\|' $\to$ `|\\|'):
%
\begin{center}
|... -jobname "|\textit{target}|" |\\|"|[\textit{flags}]%
|\input{childdoc.def}\childdocforward[|\textit{main}|]{|\textit{dest}|}"|
\end{center}
%
Here \textit{target} is the name of the output file,
\textit{main} is the name of the main file
and \textit{dest} is the name of the main or child file to be processed
(all filenames without extensions).
The optional argument \textit{main} can be omitted
if \textit{main} matches \textit{dest}.
Optionally, compilation \textit{flags} can be defined via |\def| commands.
This command line makes the \TeX{} engine believe
it is compiling the file \textit{target}
whose content is specified as the latter parameter.
The provided code then forwards the processing to
\textit{main} or \textit{dest} as described in \secref{sec:forward}.

%%%%%%%%%%%%%%%%%%%%%%%%%%%%%%%%%%%%%%%%%%%%%%%%%%%%%%%%%%%%%%%%%%%%%%%%%%%%%%%%
\subsection{Include by Input}
\label{sec:input}

Including child documents by |\include| has some restrictions by design.
Most notably, the content of a child document always occupies
its own set of pages; pages cannot be shared between child documents.
Usually, this behaviour makes perfect sense
because each child document contain an essential part of the document.
However, in some situations it may be desirable to compose
a document from a collection of parts
without having mandatory page breaks between then.
For this case, the package
provides a mechanism to include parts
by |\input| which can also be processed individually.
However, by construction this mechanism
requires manual handling of the content to be output.

%%%%%%%%%%%%%%%%%%%%%%%%%%%%%%%%%%%%%%%%
\DescribeMacro{\ifchilddocmanual}
The main file should be prepared as usual, see \secref{sec:include}.
However, the document body must make a distinction
between processing of an individual part and of the main document, e.g.:
%
\begin{center}
\begin{tabular}{l}
|\ifchilddocmanual|\\
|\input{\childdocname}|\\
|\||else|\\
\textit{document body with }|\input{|\textit{part}|}|\\
|\||fi|
\end{tabular}
\end{center}
%
The conditional |\ifchilddocmanual| is true whenever
a part to be included by |\input| is being compiled,
and the name of the part is stored in |\childdocname|.

%%%%%%%%%%%%%%%%%%%%%%%%%%%%%%%%%%%%%%%%
\DescribeMacro{\childdocby}
Each part to be included by |\input| should start with:
%
\begin{center}
\begin{tabular}{l}
|\input{childdoc.def}|\\
|\childdocby{|\textit{main}|}|\\
\end{tabular}
\end{center}
%
The directive |\childdocby| is similar to |\childdocof|
described in \secref{sec:include},
but the subsequent selection of content must be done manually.
To that end, both |\ifchilddoc| and |\ifchilddocmanual|
will be true upon processing of a part,
and the name of the part is stored in |\childdocname|.
Note that |\jobname| will be set to the filename of the current part
so that each part receives an individual |.aux| file
that does not interfere with the |.aux| file(s) of the main document.
This behaviour can be altered by the alternative form
|\childdocby[*]{|\textit{main}|}| (with a non-empty optional argument)
which uses the |.aux| file of the main document
by setting |\jobname| to \textit{main}.

%%%%%%%%%%%%%%%%%%%%%%%%%%%%%%%%%%%%%%%%%%%%%%%%%%%%%%%%%%%%%%%%%%%%%%%%%%%%%%%%
\subsection{Driver Development}
\label{sec:driver}

The \textsf{childdoc} mechanism can also be use for the development
of definition files such as \LaTeX{} styles or classes.
This case differs from the above setup with multiple parts
included by |\include| in that no |\includeonly| should be invoked.
This can be achieved by starting the include file
(before |\ProvidesPackage|) with:
%
\begin{center}
\begin{tabular}{l}
|\input{childdoc.def}|\\
|\childdocforward{|\textit{main}|}|\\
\end{tabular}
\end{center}
%
or alternatively with:
%
\begin{center}
\begin{tabular}{l}
|\input{childdoc.def}|\\
|\childdocby{|\textit{main}|}|\\
\end{tabular}
\end{center}
%
Both forms have slightly different effects as described above.
The main file is prepared as usual, see \secref{sec:include}.

%%%%%%%%%%%%%%%%%%%%%%%%%%%%%%%%%%%%%%%%%%%%%%%%%%%%%%%%%%%%%%%%%%%%%%%%%%%%%%%%
\subsection{Legacy Detection}
\label{sec:detection}

The directive |\childdocmain| in the main file can detect
whether the complete document or merely a child is to be compiled
even without using the directive |\childdocof|.
This method is deprecated because it is less robust
and there is no compelling reason to use it;
it is merely provided for backward compatibility
and it may be removed in future versions.

If the detection mechanism is to be used,
it is mandatory to correctly specify
the filename of the main file as the argument of |\childdocmain|:
%
\begin{center}
\begin{tabular}{l}
|\input{childdoc.def}|\\
|\childdocmain{|\textit{main}|}|\\
\end{tabular}
\end{center}
%
If |\jobname| does not match the argument \textit{main} of |\childdocmain|,
it is assumed that |\jobname| points to the child file to be compiled.
When using |\childdocmain| with the main file specified as argument,
it suffices to start a child file
with just |\input{|\textit{main}|}|
without loading of the package and using |\childdocof|.
If instead all processing is done
with the appropriate \textsf{childdoc} directives,
the argument of \textit{main} of |\childdocmain| can be empty.

An alternative version of the command line processing described
in \secref{sec:commandline} using the detection mechanism reads:
%
\begin{center}
|... -jobname "|\textit{target}|" "|[\textit{flags}]%
[|\def\jobname{|\textit{dest}|}|]|\input{|\textit{main}|}"|
\end{center}

%%%%%%%%%%%%%%%%%%%%%%%%%%%%%%%%%%%%%%%%%%%%%%%%%%%%%%%%%%%%%%%%%%%%%%%%%%%%%%%%
\subsection{Manual Code}
\label{sec:manual}

In case one cannot be certain whether the definitions file |childdoc.def|
is installed on the target \TeX{} distribution
and one prefers not to ship it,
it is conceivable to paste a few relevant commands into the sources.

To that end, drop all statements |\input{childdoc.def}|
and perform the replacements as outlined below.
Instead of |\childdocmain{|\textit{main}|}| add the following code
to the top of the main file:
%
\begin{center}
\begin{tabular}{l}
|\||ifdefined\childdocname\endinput\||fi\newif\ifchilddoc|\\
|\edef\childdocname{\scantokens\expandafter{\jobname\noexpand}}|\\
|\def\childdocmain{|\textit{main}|}\||ifx\childdocmain\childdocname\||else|\\
|\childdoctrue\includeonly{\childdocname}\let\jobname\childdocmain\||fi|\\
\end{tabular}
\end{center}
%
Instead of |\childdocof{|\textit{main}|}| just include the main file
at the top of each child file:
%
\begin{center}
|\input{|\textit{main}|}|
\end{center}
%
A simple redirection |\childdocforward{|\textit{dest}|}| is achieved by:
%
\begin{center}
|\def\jobname{|\textit{dest}|}\input{\jobname}|
\end{center}
%
The redirection with prefix
|\childdocforwardprefix[|\textit{prefix}|]{|\textit{dest}|}|
is accomplished by:
%
\begin{center}
\begin{tabular}{l}
|{\edef\jobname{\scantokens\expandafter{\jobname\noexpand}}|\\
|\def\redirectjob |\textit{prefix}|#1~~~{\gdef\jobname{|\textit{dest}|#1}}|\\
|\expandafter\redirectjob\jobname~~~}\input{\jobname}|
\end{tabular}
\end{center}

In an alternative approach,
child documents can be compiled by a specific command line
without additional code or specific definitions:
%
\begin{center}
|... -jobname "|\textit{target}|" "|[\textit{flags}]%
|\includeonly{|\textit{dest}|}\input{|\textit{main}|}"|
\end{center}
%

%%%%%%%%%%%%%%%%%%%%%%%%%%%%%%%%%%%%%%%%%%%%%%%%%%%%%%%%%%%%%%%%%%%%%%%%%%%%%%%%
%%%%%%%%%%%%%%%%%%%%%%%%%%%%%%%%%%%%%%%%%%%%%%%%%%%%%%%%%%%%%%%%%%%%%%%%%%%%%%%%
\section{Information}

%%%%%%%%%%%%%%%%%%%%%%%%%%%%%%%%%%%%%%%%%%%%%%%%%%%%%%%%%%%%%%%%%%%%%%%%%%%%%%%%
\subsection{Copyright}

Copyright \copyright{} 2017--2018 Niklas Beisert

This work may be distributed and/or modified under the
conditions of the \LaTeX{} Project Public License, either version 1.3
of this license or (at your option) any later version.
The latest version of this license is in
  \url{http://www.latex-project.org/lppl.txt}
and version 1.3 or later is part of all distributions of \LaTeX{}
version 2005/12/01 or later.

This work has the LPPL maintenance status `maintained'.

The Current Maintainer of this work is Niklas Beisert.

This work consists of the files |README.txt|, |childdoc.ins| and |childdoc.dtx|
as well as the derived files |childdoc.def|, |cdocsamp.tex|
with |cdocsch1.tex|, |cdocsch2.tex|, |cdocspt3.tex|, |cdocspt4.tex|,
|cdocsdrf.tex|, |cdocsfn1.tex|, |cdocsfn2.tex|
as well as |childdoc.pdf|.

%%%%%%%%%%%%%%%%%%%%%%%%%%%%%%%%%%%%%%%%%%%%%%%%%%%%%%%%%%%%%%%%%%%%%%%%%%%%%%%%
\subsection{Files and Installation}

The package consists of the files:
%
\begin{center}
\begin{tabular}{ll}
    |README.txt|   & readme file \\
    |childdoc.ins| & installation file \\
    |childdoc.dtx| & source file \\
    |childdoc.def| & definition file \\
    |cdocsamp.tex| & sample main file \\
    |cdocsch1.tex| & sample include file \\
    |cdocsch2.tex| & sample include file \\
    |cdocspt3.tex| & sample part file \\
    |cdocspt4.tex| & sample part file \\
    |cdocsdrf.tex| & sample redirection file \\
    |cdocsfn1.tex| & sample redirection file \\
    |cdocsfn2.tex| & sample redirection file \\
    |childdoc.pdf| & manual
\end{tabular}
\end{center}
%
The distribution consists of the files
|README.txt|, |childdoc.ins| and |childdoc.dtx|.
%
\begin{itemize}
\item
Run (pdf)\LaTeX{} on |childdoc.dtx|
to compile the manual |childdoc.pdf| (this file).
\item
Run \LaTeX{} on |childdoc.ins| to create the definitions file |childdoc.def|
and the sample |cdocsamp.tex| with include files
|cdocsch1.tex|, |cdocsch2.tex|, |cdocspt3.tex|, |cdocspt4.tex|,
|cdocsdrf.tex|, |cdocsfn1.tex|, |cdocsfn2.tex|.
Then copy the file |childdoc.def| to an appropriate directory of your \LaTeX{}
distribution, e.g.\ \textit{texmf-root}|/tex/latex/childdoc|.
\end{itemize}

%%%%%%%%%%%%%%%%%%%%%%%%%%%%%%%%%%%%%%%%%%%%%%%%%%%%%%%%%%%%%%%%%%%%%%%%%%%%%%%%
\subsection{Related CTAN Packages}

There are several other packages which offer a similar functionality:
%
\begin{itemize}
\item
The packages
\href{http://ctan.org/pkg/docmute}{\textsf{docmute}},
\href{http://ctan.org/pkg/includex}{\textsf{includex}} and
\href{http://ctan.org/pkg/standalone}{\textsf{standalone}}
provide commands to include only the document body of
a child file thus allowing both files to be compiled individually.
\item
The packages \href{http://ctan.org/pkg/subdocs}{\textsf{subdocs}}
and \href{http://ctan.org/pkg/subfiles}{\textsf{subfiles}}
provide structures in which the main and child documents can be
encapsulated and allowing them to be compiled individually.
The inclusion mechanism is different from the conventional |\include|.
\item
The package \href{http://ctan.org/pkg/combine}{\textsf{combine}}
is an elaborate solution to combine several documents into one.
\end{itemize}
%
See also the CTAN topic \href{http://ctan.org/topic/subdocs}{\textsf{subdocs}}
for further related packages.
The present package differs from the above solutions in that
a document structure constructed with the conventional |\include| mechanism
just needs two extra commands at the top of every file
such that all constituent files can be compiled individually.

%%%%%%%%%%%%%%%%%%%%%%%%%%%%%%%%%%%%%%%%%%%%%%%%%%%%%%%%%%%%%%%%%%%%%%%%%%%%%%%%
%\subsection{Feature Suggestions}
%
%The following is a list of features which may be useful for future
%versions of this package:
%%
%\begin{itemize}
%\item
%\ldots
%\end{itemize}

%%%%%%%%%%%%%%%%%%%%%%%%%%%%%%%%%%%%%%%%%%%%%%%%%%%%%%%%%%%%%%%%%%%%%%%%%%%%%%%%
\subsection{Revision History}

%%%%%%%%%%%%%%%%%%%%%%%%%%%%%%%%%%%%%%%%
\paragraph{v2.0:} 2018/12/30

\begin{itemize}
\item
immediate forward processing
\item
added |\childdocby| mechanism
\item
manual restructured
\end{itemize}

%%%%%%%%%%%%%%%%%%%%%%%%%%%%%%%%%%%%%%%%
\paragraph{v1.6:} 2018/01/17

\begin{itemize}
\item
application for development of include files
\item
corrections to manual
\end{itemize}

%%%%%%%%%%%%%%%%%%%%%%%%%%%%%%%%%%%%%%%%
\paragraph{v1.5:} 2017/05/21

\begin{itemize}
\item
more complete structuring introduced
\item
|\childdocof| introduced
\item
|\childdoc| renamed to |\childdocmain|
\item
|\childredirect| renamed to |\childdocforward| and |\childdocforwardprefix|
and functionality expanded
\end{itemize}

%%%%%%%%%%%%%%%%%%%%%%%%%%%%%%%%%%%%%%%%
\paragraph{v1.0:} 2017/04/27

\begin{itemize}
\item
manual and install package
\item
first version published on CTAN
\end{itemize}

%%%%%%%%%%%%%%%%%%%%%%%%%%%%%%%%%%%%%%%%
\paragraph{v0.6:} 2017/04/26

\begin{itemize}
\item
redirection mechanism added
\end{itemize}

%%%%%%%%%%%%%%%%%%%%%%%%%%%%%%%%%%%%%%%%
\paragraph{v0.5:} 2017/04/26

\begin{itemize}
\item
functionality in definition file
\end{itemize}


%%%%%%%%%%%%%%%%%%%%%%%%%%%%%%%%%%%%%%%%%%%%%%%%%%%%%%%%%%%%%%%%%%%%%%%%%%%%%%%%
%%%%%%%%%%%%%%%%%%%%%%%%%%%%%%%%%%%%%%%%%%%%%%%%%%%%%%%%%%%%%%%%%%%%%%%%%%%%%%%%
%%%%%%%%%%%%%%%%%%%%%%%%%%%%%%%%%%%%%%%%%%%%%%%%%%%%%%%%%%%%%%%%%%%%%%%%%%%%%%%%
\appendix

\settowidth\MacroIndent{\rmfamily\scriptsize 000\ }

 \DocInput{childdoc.dtx}

\end{document}
%</driver>
% \fi
%
% %%%%%%%%%%%%%%%%%%%%%%%%%%%%%%%%%%%%%%%%%%%%%%%%%%%%%%%%%%%%%%%%%%%%%%%%%%%%%%
% %%%%%%%%%%%%%%%%%%%%%%%%%%%%%%%%%%%%%%%%%%%%%%%%%%%%%%%%%%%%%%%%%%%%%%%%%%%%%%
% \section{Sample}
%\iffalse
%<*samplemain>
%\fi
%
% The following presents a sample document
% with two chapters, two parts, a title page,
% a compile flag as well as three forwarding files to set the flag.
% It consists of eight |.tex| files:
% \begin{center}
% \begin{tabular}{ll}
% |cdocsamp.tex|&main file\\
% |cdocsch1.tex|&include file for chapter 1\\
% |cdocsch2.tex|&include file for chapter 2\\
% |cdocspt3.tex|&include file for part 3\\
% |cdocspt4.tex|&include file for part 4\\
% |cdocsdrf.tex|&forwarding file for main file in draft mode\\
% |cdocsfi1.tex|&forwarding file for final version of chapter 1\\
% |cdocsfi2.tex|&forwarding file for final version of chapter 2\\
% \end{tabular}
% \end{center}
% Each of the eight files can be compiled directly by the \LaTeX{} compiler.
%
% %%%%%%%%%%%%%%%%%%%%%%%%%%%%%%%%%%%%%%
% \paragraph{Main File.}
%
% The main file is called |cdocsamp.tex|.
%
% Load the \textsf{childdoc} definitions and
% declare the filename for the main document:
%    \begin{macrocode}
\input{childdoc.def}
\childdocmain{}
%    \end{macrocode}

% Optional override for |\version| flag:
%    \begin{macrocode}
%%\ifchilddoc\else\providecommand{\version}{draft}\fi
%    \end{macrocode}

% Define the default values for the |\version| flag
% (|final| for the main file and |draft| for childs):
%    \begin{macrocode}
\ifchilddoc
\providecommand{\version}{draft}
\else
\providecommand{\version}{final}
\fi
%    \end{macrocode}

% Load the standard document class:
%    \begin{macrocode}
\documentclass[12pt]{article}
%    \end{macrocode}

% Start the document body:
%    \begin{macrocode}
\begin{document}
%    \end{macrocode}

% Declare a title page.
% Print title, part of document being processed and version flag:
%    \begin{macrocode}
\addtocounter{page}{-1}
\begin{center}
{\LARGE\bfseries{}childdoc example\par}
\vspace{1cm}
\ifchilddoc
\ifchilddocmanual part\else chapter\fi:
`\childdocname' of `\childdocjob'\par
\else
main document: `\childdocjob'\par
\fi
version: \version\par
\end{center}
\newpage
%    \end{macrocode}

% Manually include selected file,
% otherwise process as usual:
%    \begin{macrocode}
\ifchilddocmanual
\section*{part `\childdocname'}
\input{\childdocname}
\else
%    \end{macrocode}

% Include the two chapters:
%    \begin{macrocode}
\include{cdocsch1}
\include{cdocsch2}
%    \end{macrocode}

% Include the two parts unless only chapters should be displayed:
%    \begin{macrocode}
\ifchilddoc\else
\section{part three}
\input{cdocspt3}
\section{part four}
\input{cdocspt4}
\fi
%    \end{macrocode}

% Process as usual until here:
%    \begin{macrocode}
\fi
%    \end{macrocode}

% End of document body:
%    \begin{macrocode}
\end{document}
%    \end{macrocode}
%\iffalse
%</samplemain>
%\fi
%
% %%%%%%%%%%%%%%%%%%%%%%%%%%%%%%%%%%%%%%
% \paragraph{Chapter Include Files.}
%
% The include files are called |cdocsch1.tex| and |cdocsch2.tex|.
%
%\iffalse
%<*samplechap1|samplechap2>
%\fi

% Optional override for |\version| flag:
%    \begin{macrocode}
%%\providecommand{\version}{final}
%    \end{macrocode}

% Include the main document:
%    \begin{macrocode}
\input{childdoc.def}
\childdocof{cdocsamp}
%    \end{macrocode}

%\iffalse
%</samplechap1|samplechap2>
%\fi
%
%\iffalse
%<*samplechap1>
%\fi
% Some text for chapter 1:
%    \begin{macrocode}
\section{one}
some text in chapter one
%    \end{macrocode}

%\iffalse
%</samplechap1>
%\fi
% Some text for chapter 2:
%\iffalse
%<*samplechap2>
%\fi
%    \begin{macrocode}
\section{two}
more text in chapter two
%    \end{macrocode}

%\iffalse
%</samplechap2>
%\fi
%
% %%%%%%%%%%%%%%%%%%%%%%%%%%%%%%%%%%%%%%
% \paragraph{Part Include Files.}
%
% The include files are called |cdocspt3.tex| and |cdocspt4.tex|.
%
%\iffalse
%<*samplepart3|samplepart4>
%\fi

% Optional override for |\version| flag:
%    \begin{macrocode}
%%\providecommand{\version}{final}
%    \end{macrocode}

% Include the main document:
%    \begin{macrocode}
\input{childdoc.def}
\childdocby{cdocsamp}
%    \end{macrocode}

%\iffalse
%</samplepart3|samplepart4>
%\fi
%
%\iffalse
%<*samplepart3>
%\fi
% Some text for part 3:
%    \begin{macrocode}
some text in part three
%    \end{macrocode}

%\iffalse
%</samplepart3>
%\fi
% Some text for part 4:
%\iffalse
%<*samplepart4>
%\fi
%    \begin{macrocode}
more text in part four
%    \end{macrocode}

%\iffalse
%</samplepart4>
%\fi
%
% %%%%%%%%%%%%%%%%%%%%%%%%%%%%%%%%%%%%%%
% \paragraph{Forwarding for a Complete Draft.}
%
% The following forwarding file |cdocsdrf.tex|
% compiles the main document in draft mode:
%\iffalse
%<*sampledraft>
%\fi
%    \begin{macrocode}
\def\version{draft}
\input{childdoc.def}
\childdocforward{cdocsamp}
%    \end{macrocode}

%\iffalse
%</sampledraft>
%\fi
%
% %%%%%%%%%%%%%%%%%%%%%%%%%%%%%%%%%%%%%%
% \paragraph{Forwarding for Final Version of the Chapters.}
%
% The following forwarding files |cdocsfn1.tex| and |cdocsfn2.tex|
% (with identical content)
% compile the final versions of the child documents
% |cdocsch1.tex| and |cdocsch2.tex|, respectively:
%\iffalse
%<*samplefinal>
%\fi
%    \begin{macrocode}
\def\version{final}
\input{childdoc.def}
\childdocforwardprefix[cdocsamp]{cdocsfn}{cdocsch}
%    \end{macrocode}

%\iffalse
%</samplefinal>
%\fi
%
% %%%%%%%%%%%%%%%%%%%%%%%%%%%%%%%%%%%%%%
% \paragraph{Command Line Processing.}
%
% The following three command lines generate the output files
% |cdocscld|, |cdocscl1| and |cdocscl2|
% which should be identical to
% |cdocsdrf|, |cdocsch1| and |cdocsfn2|, respectively:
% \begin{center}
% \begin{tabular}{l}
% |latex -jobname cdocscld \|\\
% |  "\def\version{draft}\input{childdoc.def}\childdocforward{cdocsamp}"|\\
% |latex -jobname cdocscl1 \|\\
% |  "\input{childdoc.def}\childdocforward[cdocsamp]{cdocsch1}"|\\
% |latex -jobname cdocscl2 \|\\
% |  "\def\version{final}\input{childdoc.def}\childdocforward{cdocsch2}"|
% \end{tabular}
% \end{center}
% Note that the trailing backslash on each first line
% merely continues the input to the second line
% (for convenient cut ant paste).
% Furthermore, the command |latex| can be replaced by any
% of its alternative versions such as |pdflatex|.
%
% %%%%%%%%%%%%%%%%%%%%%%%%%%%%%%%%%%%%%%%%%%%%%%%%%%%%%%%%%%%%%%%%%%%%%%%%%%%%%%
% %%%%%%%%%%%%%%%%%%%%%%%%%%%%%%%%%%%%%%%%%%%%%%%%%%%%%%%%%%%%%%%%%%%%%%%%%%%%%%
% \section{Implementation}
%\iffalse
%<*package>
%\fi
%
% This section describes the definitions file |childdoc.def|.

% The definitions cannot be loaded using |\usepackage| or |\RequirePackage|
% which has a mechanism to prevent loading a style file more than once.
% When loading the definitions by means of |\input|
% multiple instances have to be prevented manually:
%\iffalse
%This code needs to be before the `\ProvidesFile' directive
%which is defined at the beginning of this file.
%Therefore it is also placed there and commented out here.
%</package>
%<*discard>
%\fi
%    \begin{macrocode}
\ifdefined\childdocmain\endinput\fi
%    \end{macrocode}
%\iffalse
%</discard>
%<*package>
%\fi
%
% \macro{\ifchilddoc}
% \macro{\ifchilddocmanual}
% The conditional |\ifchilddoc| tells whether a
% child (true) or main (false) document is being compiled.
% The conditional |\ifchilddocmanual| tells whether
% the |\includeonly| mechanism is used (false) or
% the selection of child files must be performed manually (true).
% The definitions initialise to false:
%    \begin{macrocode}
\newif\ifchilddoc
\newif\ifchilddocmanual
%    \end{macrocode}

% \macro{\childdocname}
% \macro{\childdocjob}
% The macro |\childdocname| stores the name of the main document
% to be compiled. The macro |\childdocjob| stores the name of
% the document on which the \LaTeX{} compiler was originally invoked.
% The content of |\jobname| cannot be compared
% to filenames specified in the source due to different catcodes.
% The following code rescans |\jobname|, stores the result
% in |\childdocname| and saves a copy in |\childdocjob|:
%    \begin{macrocode}
\edef\childdocname{\scantokens\expandafter{\jobname\noexpand}}
\let\childdocjob\childdocname
%    \end{macrocode}

% \macro{\childdocdisable}
% The macro |\childdocdisable| prevents the main file
% from being processed more than once.
% At this stage, the main document command |\childdocmain|
% is assumed to be called once again where it should do nothing.
% Any subsequent call to it should prevent
% a secondary processing of the main document
% It overwrites the forwarding commands
% |\childdocof| and |\childdocforward|
% with empty macros to prevent further inclusions of the main document:
%    \begin{macrocode}
\newcommand{\childdocdisable}
{
  \renewcommand{\childdocmain}[1]{\renewcommand{\childdocmain}[1]{\endinput}}
  \renewcommand{\childdocof}[1]{}
  \renewcommand{\childdocby}[2][]{}
  \renewcommand{\childdocforward}[2][]{}
  \renewcommand{\childdocdisable}{}
}
%    \end{macrocode}

% \macro{\childdocmain}
% The macro |\childdocmain| is to be called at the top of the main file
% with nothing or the main filename (without extension) as argument.
% First, it breaks loops.
% If the argument is not empty and does not match |\childdocname|
% (which is set by the first inclusion of |childdoc.def|),
% |\ifchilddoc| is set to true, |\includeonly| is applied to the child file
% and |\jobname| is set to the main file
% (for proper handling of |.aux| files):
%    \begin{macrocode}
\newcommand{\childdocmain}[1]
{
  \childdocdisable\childdocmain{}
  \if?#1?\else
    \begingroup
      \def\childdoctmp{#1}
      \ifx\childdoctmp\childdocname
        \def\childdoctmp{}
      \else
        \def\childdoctmp
        {
          \childdoctrue
          \includeonly{\childdocname}
          \def\childdocjob{#1}
          \def\jobname{#1}
        }
      \fi
      \expandafter
    \endgroup
    \childdoctmp
  \fi
}
%    \end{macrocode}

% \macro{\childdocof}
% The command |\childdocof| redirects
% compilation to the main file |#1|.
%    \begin{macrocode}
\newcommand{\childdocof}[1]
{
  \childdocdisable
  \childdoctrue
  \includeonly{\childdocname}
  \def\jobname{#1}
  \def\childdocjob{#1}
  \input{#1}
}
%    \end{macrocode}

% \macro{\childdocby}
% The command |\childdocby| ....
%    \begin{macrocode}
\newcommand{\childdocby}[2][]
{
  \childdocdisable
  \childdoctrue
  \childdocmanualtrue
  \if?#1?\else
    \def\jobname{#2}
  \fi
  \def\childdocjob{#2}
  \input{#2}
  \endinput
}
%    \end{macrocode}

% \macro{\childdocforward}
% The command |\childdocforward| redirects
% compilation to the main file or
% (if the optional argument is given) a child file.
% Parameters are set as if the main file
% or a child file starting with |\childdocof| was compiled.
% Then compilation is handed over to the main file:
%    \begin{macrocode}
\newcommand{\childdocforward}[2][]
{
  \begingroup
    \if?#1?
      \def\childdoctmp
      {
        \def\childdocname{#2}
        \def\childdocjob{#2}
        \def\jobname{#2}
        \input{#2}
        \endinput
      }
    \else
      \def\childdoctmp
      {
        \childdocdisable
        \def\childdocname{#2}
        \childdoctrue
        \includeonly{#2}
        \def\childdocjob{#1}
        \def\jobname{#1}
        \input{#1}
        \endinput
      }
    \fi
    \expandafter
  \endgroup
  \childdoctmp
}
%    \end{macrocode}

% \macro{\childdocforwardprefix}
% The command |\childdocforwardprefix| redirects
% compilation to the main or a child file by means of a pattern.
% The prefix |#1| in the current filename is replaced by |#2|
% and the suffix of the current filename is kept
% (it is assumed that the filename does not contain the substring `|~~~|'
% which is used as a delimiter).
% Compilation is handed over to the new file by |\childdocforward|:
%    \begin{macrocode}
\newcommand{\childdocforwardprefix}[3][]
{
  \begingroup
    \def\childdocextract #2##1~~~{\def\childdoctmp{\childdocforward[#1]{#3##1}}}
    \expandafter\childdocextract\childdocname~~~
    \expandafter
  \endgroup
  \childdoctmp
}
%    \end{macrocode}

% \macro{\childdoc}
% The deprecated macro |\childdoc| is a legacy version of |\childdocmain|:
%    \begin{macrocode}
\newcommand{\childdoc}{\childdocmain}
%    \end{macrocode}

% \macro{\childdocredirect}
% The deprecated macro |\childdocredirect| is a legacy version
% of |\childdocforward| and |\childdocforwardprefix|:
%    \begin{macrocode}
\newcommand{\childdocredirect}[2][]
{
  \begingroup
    \if?#1?
      \def\childdoctmp{\childdocforward{#2}}
    \else
      \def\childdoctmp{\childdocforwardprefix{#1}{#2}}
    \fi
    \expandafter
  \endgroup
  \childdoctmp
}
%    \end{macrocode}

%\iffalse
%</package>
%\fi
%
\endinput
|\\
|\childdocforward[|\textit{main}|]{|\textit{dest}|}|\\
\end{tabular}
\end{center}
%
The argument \textit{dest} is the destination file
(without extension).
It should be the main file or one of the child files.
Note that further \textsf{childdoc} directives
such as |\childdocof| and |\childdocforward|
in the indicated file will be processed in this form.
The optional argument \textit{main}
passes on directly to the main file \textit{main}
while pretending to compile the child \textit{dest}.
This form behaves as if \textit{dest}
issues |\childdocof{|\textit{main}|}| right away,
and no further \textsf{childdoc} directives will be processed.

%%%%%%%%%%%%%%%%%%%%%%%%%%%%%%%%%%%%%%%%
\DescribeMacro{\...prefix}
In the alternative form |\childdocforwardprefix|,
%
\begin{center}
\begin{tabular}{l}
|% \iffalse
%
% childdoc.dtx Copyright (C) 2017-2018 Niklas Beisert
%
% This work may be distributed and/or modified under the
% conditions of the LaTeX Project Public License, either version 1.3
% of this license or (at your option) any later version.
% The latest version of this license is in
%   http://www.latex-project.org/lppl.txt
% and version 1.3 or later is part of all distributions of LaTeX
% version 2005/12/01 or later.
%
% This work has the LPPL maintenance status `maintained'.
%
% The Current Maintainer of this work is Niklas Beisert.
%
% This work consists of the files childdoc.dtx and childdoc.ins
% and the derived files childdoc.def and cdocsamp.tex with
% cdocsch1.tex, cdocsch2.tex, cdocsdrf.tex, cdocsfn1.tex, cdocsfn2.tex.
%
%<package>\ifdefined\childdocmain\endinput\fi
%<package>\ProvidesFile{childdoc.def}[2018/12/30 v2.0 child document driver]
%<samplemain>\ProvidesFile{cdocsamp.tex}[2018/12/30 v2.0 sample for childdoc]
%<*driver>
%\ProvidesFile{childdoc.drv}[2018/12/30 v2.0 childdoc reference manual file]
\PassOptionsToClass{10pt,a4paper}{article}
\documentclass{ltxdoc}

\usepackage[margin=35mm]{geometry}
\usepackage{hyperref}
\usepackage{hyperxmp}
\usepackage[usenames]{color}

\hypersetup{colorlinks=true}
\hypersetup{pdfstartview=FitH}
\hypersetup{pdfpagemode=UseNone}
\hypersetup{pdfsource={}}
\hypersetup{pdflang={en-UK}}
\hypersetup{pdfcopyright={Copyright 2017-2018 Niklas Beisert.
  This work may be distributed and/or modified under the
  conditions of the LaTeX Project Public License, either version 1.3
  of this license or (at your option) any later version.}}
\hypersetup{pdflicenseurl={http://www.latex-project.org/lppl.txt}}
\hypersetup{pdfcontactaddress={ETH Zurich, ITP, HIT K,
  Wolfgang-Pauli-Strasse 27}}
\hypersetup{pdfcontactpostcode={8093}}
\hypersetup{pdfcontactcity={Zurich}}
\hypersetup{pdfcontactcountry={Switzerland}}
\hypersetup{pdfcontactemail={nbeisert@itp.phys.ethz.ch}}
\hypersetup{pdfcontacturl={http://people.phys.ethz.ch/\xmptilde nbeisert/}}

\newcommand{\secref}[1]{\hyperref[#1]{section \ref*{#1}}}

\parskip1ex
\parindent0pt
\let\olditemize\itemize
\def\itemize{\olditemize\parskip0pt}

\begin{document}

\title{The \textsf{childdoc} Package}
\hypersetup{pdftitle={The childdoc Package}}
\author{Niklas Beisert\\[2ex]
  Institut f\"ur Theoretische Physik\\
  Eidgen\"ossische Technische Hochschule Z\"urich\\
  Wolfgang-Pauli-Strasse 27, 8093 Z\"urich, Switzerland\\[1ex]
  \href{mailto:nbeisert@itp.phys.ethz.ch}
  {\texttt{nbeisert@itp.phys.ethz.ch}}}
\hypersetup{pdfauthor={Niklas Beisert}}
\hypersetup{pdfsubject={Manual for the LaTeX2e Package childdoc}}
\date{30 December 2018, \textsf{v2.0}}
\maketitle

\begin{abstract}\noindent
\textsf{childdoc} is a \LaTeXe{} package
that enables the direct compilation
of document sections included by |\include|
to individual files.
\end{abstract}

\begingroup
\parskip0ex
\tableofcontents
\endgroup

%%%%%%%%%%%%%%%%%%%%%%%%%%%%%%%%%%%%%%%%%%%%%%%%%%%%%%%%%%%%%%%%%%%%%%%%%%%%%%%%
%%%%%%%%%%%%%%%%%%%%%%%%%%%%%%%%%%%%%%%%%%%%%%%%%%%%%%%%%%%%%%%%%%%%%%%%%%%%%%%%
\section{Introduction}

\LaTeX{} provides a mechanism to structure a large document (such as a book)
into a main file and several child files (containing the chapters)
using the |\include| command.
This mechanism is beneficial for documents
which span hundreds of pages in order to
make the source file(s) more manageable.
Moreover, compilation can be restricted to
selected child files by means of the |\includeonly| command.
The latter feature can be used to reduce the compilation time while editing
(this was significantly more useful in the earlier days of \LaTeX{})
or to generate a smaller document which is easier to navigate.
Another application of |\includeonly| is to generate
documents consisting of selected parts of the complete document.

However, there are a few drawbacks of the plain |\include| mechanism:
\begin{itemize}
\item
The child files cannot be compiled on their own,
they can only be compiled via the main file.
A naive editing environment
(such as a text editor with an option
to have the current file processed by \LaTeX)
may require one to switch to the main file before compiling;
attempting to compile the child file produces errors.
\item
The main file must be modified (each time)
to adjust the |\includeonly| command
to the present needs. This easily leaves the main file in a messy state.
\item
The generated document will always carry the filename
of the main document. This is inconvenient if
several child files are to be compiled and
to be kept for distribution.
\end{itemize}

The present package provides a simple interface
to make child files individually compilable by \LaTeX{}.
Compiling a child file then has the same effect as compiling
the main file with an |\includeonly| command
to select the appropriate child.
Moreover the generated document will carry the name of the child
rather than the main file.
This resolves all three above issues.

This feature is meant to make the editing of books,
thesis documents and lecture notes somewhat more convenient.
However, the package can also be used efficiently for
composing a series of documents (such as exercise sheets)
which are typically distributed individually.
It then assists the author in generating the individual documents
(potentially in different versions)
as well as a document containing the collected series.
Another application is in developing style files
or other kinds of included material
where compilation of the style file could redirect
to a sample or test file.

%%%%%%%%%%%%%%%%%%%%%%%%%%%%%%%%%%%%%%%%%%%%%%%%%%%%%%%%%%%%%%%%%%%%%%%%%%%%%%%%
%%%%%%%%%%%%%%%%%%%%%%%%%%%%%%%%%%%%%%%%%%%%%%%%%%%%%%%%%%%%%%%%%%%%%%%%%%%%%%%%
\section{Usage}

First of all, the package \textsf{childdoc} is \emph{not} a standard
\LaTeXe{} |.sty| style file! Therefore it needs to be invoked in
a non-standard way.

%%%%%%%%%%%%%%%%%%%%%%%%%%%%%%%%%%%%%%%%%%%%%%%%%%%%%%%%%%%%%%%%%%%%%%%%%%%%%%%%
\subsection{Included Files}
\label{sec:include}

%%%%%%%%%%%%%%%%%%%%%%%%%%%%%%%%%%%%%%%%
\DescribeMacro{\childdocmain}
To use the package, add the commands
\begin{center}
\begin{tabular}{l}
|\input{childdoc.def}|\\
|\childdocmain{}|\\
\end{tabular}
\end{center}
at the very top of the main \LaTeX{} file,
in particular \emph{before} the |\documentclass| statement!
The argument of |\childdocmain| should be left empty
(but it must be present).

%%%%%%%%%%%%%%%%%%%%%%%%%%%%%%%%%%%%%%%%
\DescribeMacro{\childdocof}
Furthermore, add the commands
\begin{center}
\begin{tabular}{l}
|\input{childdoc.def}|\\
|\childdocof{|\textit{main}|}|\\
\end{tabular}
\end{center}
at the top of every child file \textit{child}
which is included by |\include{|\textit{child}|}|
from within the main file
(or at least for those files to be compiled individually).
The argument \textit{main} must be the filename of the main file.

There are a couple of
considerations in setting up the main and child documents:

%%%%%%%%%%%%%%%%%%%%%%%%%%%%%%%%%%%%%%%%
\paragraph{Restrictions.}

Please note the following restrictions:
\begin{itemize}
\item
|\childdocmain| must be called with one argument \textit{main}
to ensure compatibility with earlier version of the package.
It must either be empty (|\childdocmain{}|)
or precisely match the filename of the main file in which it is specified.
See \secref{sec:detection} for further information.
\item
The filename \textit{main} must be specified without the |.tex| extension.
\item
The filename \textit{main} is case sensitive
(even in case-insensitive file systems)
due to internal string comparison.
\item
The argument \textit{main} should be fully expanded, it cannot be a macro.
\item
Subdirectories and special characters should be avoided in filenames.
\item
The command |\childdocmain{|\textit{main}|}| must be followed by a whitespace.
It should not be followed immediately by another command
or by a comment mark `|%|'.
This is because the \TeX{} parser reads the token immediately following
the argument of |\childdocmain| and puts it
at the beginning of every child section;
however, a white\-space is ignored.
\end{itemize}

%%%%%%%%%%%%%%%%%%%%%%%%%%%%%%%%%%%%%%%%
\paragraph{Content of Main File.}

It is advisable to place all content in the child files included by |\include|.
Any output contained in the main file will appear in all child documents
unless suppressed manually;
it cannot be suppressed automatically by the |\includeonly| directive
and thus should normally be avoided.
A method to include some content in the main file
by means of conditional processing is described in \secref{sec:conditional}.

%%%%%%%%%%%%%%%%%%%%%%%%%%%%%%%%%%%%%%%%
\paragraph{Page Numbering.}

When only a part of the document is compiled,
the appropriate numbering of pages
(as well as other status parameters)
is determined from the |.aux| files.
The latter contain information from previous passes.
However this information needs to propagate through
all intermediate child documents.
Therefore the page numbering in child documents may well
be inconsistent until the complete document is compiled at least once.

A useful (if unconventional) way to always ensure a consistent
page numbering is to restart the numbering in each child document
and denote the pages by `\textit{child}|.|\textit{page}'
where \textit{child} represents the chapter/section number of the child file.
This can be achieved by the command
|\numberwithin{page}{|\textit{child}|}|
of the \textsf{amsmath} package
where \textit{child} can be |chapter| or |section|
depending on the chosen structuring.
Alternatively, one can modify the macro |\thepage| appropriately
and reset the counter |page| at the start of each child file.

%%%%%%%%%%%%%%%%%%%%%%%%%%%%%%%%%%%%%%%%%%%%%%%%%%%%%%%%%%%%%%%%%%%%%%%%%%%%%%%%
\subsection{Conditional Processing}
\label{sec:conditional}

The package provides a mechanism to compile different versions
of a document. To customise the versions further some conditional processing
can come in handy to distinguish which version is being compiled.
The package provides two macros to describe the compilation context:

%%%%%%%%%%%%%%%%%%%%%%%%%%%%%%%%%%%%%%%%
\DescribeMacro{\ifchilddoc}
The conditional |\ifchilddoc| distinguishes between the compilation of
child documents and the main document:
%
\begin{center}
|\ifchilddoc |\textit{child-code}| |[|\||else |\textit{main-code}]| \||fi|
\end{center}

%%%%%%%%%%%%%%%%%%%%%%%%%%%%%%%%%%%%%%%%
\DescribeMacro{\childdocname}
\DescribeMacro{\childdocjob}
The macro |\childdocname| contains the filename (without extension)
of the main or child file being processed.
Note that |\childdocjob| will always contain the name of the main file.

%%%%%%%%%%%%%%%%%%%%%%%%%%%%%%%%%%%%%%%%
\paragraph{Title Page.}

Conditional processing can be used to include a title or banner page
in the main document when proper precautions are taken.
Importantly, the code in the main file should ensure that the page counter
(as well as other status parameters which are stored in the |.aux| files)
takes the same value after the conditional processing.
Otherwise the page numbers may take divergent values
depending on which part is compiled.

For example, a title page could be declared by:
%
\begin{center}
\begin{tabular}{l}
|\ifchilddoc\||else|\\
|\addtocounter{page}{-1}|\\
\textit{code for title page}\\
|\newpage|\\
|\||fi|
\end{tabular}
\end{center}
%
A banner page for the child documents can be generated by:
%
\begin{center}
\begin{tabular}{l}
|\ifchilddoc|\\
|\addtocounter{page}{-1}|\\
\textit{code for banner page}\\
|\newpage|\\
|\||fi|
\end{tabular}
\end{center}
%
Here one could write a message such as:
\begin{center}
|This is the part \childdocname{} of \childdocjob{}.|
\end{center}

%%%%%%%%%%%%%%%%%%%%%%%%%%%%%%%%%%%%%%%%%%%%%%%%%%%%%%%%%%%%%%%%%%%%%%%%%%%%%%%%
\subsection{Flags}
\label{sec:flags}

The package makes it easy to generate different versions
of the main or child documents.
To this end compilation flags can be defined
and assigned different default values.
They will be particularly useful in conjunction
with the forwarding mechanism described in \secref{sec:forward}.

For example, it may be useful to have a flag |\version|
which can be set to |draft| or |final|.
The document source will contain some conditional code
depending on the value of |\version|.
Suppose further, the flag should default to |final| for the main file
and to |draft| for child files
which is a natural assignment for editing the document.
This is achieved by placing the following code
in the preamble of the main document
(below the |\childdocmain| directive):
%
\begin{center}
\begin{tabular}{l}
|\ifchilddoc|\\
|\providecommand{\version}{draft}|\\
|\||else|\\
|\providecommand{\version}{final}|\\
|\||fi|
\end{tabular}
\end{center}
%
The definition by |\providecommand| makes sure
that previous definitions are not overwritten.
Further statements |\providecommand{\version}{...}|
can thus be added before the above code to override it.

For the main file, one might add a line
(between |\childdocmain| and the above block)
%
\begin{center}
|%\ifchilddoc\||else\providecommand{\version}{draft}\||fi|
\end{center}
%
which can be uncommented to produce a draft version.
Likewise one can add a line to the very top of a child file
(above the |\childdocof{|\textit{main}|}| directive)
%
\begin{center}
|%\providecommand{\version}{final}|
\end{center}
%
which can be uncommented to produce the final version of this child document.

%%%%%%%%%%%%%%%%%%%%%%%%%%%%%%%%%%%%%%%%%%%%%%%%%%%%%%%%%%%%%%%%%%%%%%%%%%%%%%%%
\subsection{Forwarding}
\label{sec:forward}

Different versions of the main or child documents
using compilation flags as described in \secref{sec:flags}
can be (permanently) stored in different files
for convenient compilation, viewing and distribution.
To this end, the package defines a command
to pass on compilation to a different file:

%%%%%%%%%%%%%%%%%%%%%%%%%%%%%%%%%%%%%%%%
\DescribeMacro{\childdocforward}
The command |\childdocforward| redirects processing to
another source file:
%
\begin{center}
\begin{tabular}{l}
|\input{childdoc.def}|\\
|\childdocforward[|\textit{main}|]{|\textit{dest}|}|\\
\end{tabular}
\end{center}
%
The argument \textit{dest} is the destination file
(without extension).
It should be the main file or one of the child files.
Note that further \textsf{childdoc} directives
such as |\childdocof| and |\childdocforward|
in the indicated file will be processed in this form.
The optional argument \textit{main}
passes on directly to the main file \textit{main}
while pretending to compile the child \textit{dest}.
This form behaves as if \textit{dest}
issues |\childdocof{|\textit{main}|}| right away,
and no further \textsf{childdoc} directives will be processed.

%%%%%%%%%%%%%%%%%%%%%%%%%%%%%%%%%%%%%%%%
\DescribeMacro{\...prefix}
In the alternative form |\childdocforwardprefix|,
%
\begin{center}
\begin{tabular}{l}
|\input{childdoc.def}|\\
|\childdocforwardprefix[|\textit{main}|]{|\textit{prefix}|}{|\textit{dest}|}|
\end{tabular}
\end{center}
%
the destination file is determined by a pattern
depending on the current file:
To make this work, the current file must be called
`{\textit{prefix}\hspace{0.2em}\textit{suffix}}'
with \textit{prefix} matching precisely the argument.
Processing is then passed on to the file
`{\textit{dest}\hspace{0.2em}\textit{suffix}}'.
Surely, the same effect is achieved by
directly specifying the
argument `{\textit{dest}\hspace{0.2em}\textit{suffix}}'
in the first form.
However, that requires to set up a different file
for each child. With the alternative form of the command
all these files can have exactly the same content
which simplifies setting them up and maintaining them.

For example, the following file |draft.tex|
with a compilation flag |\version| as described in \secref{sec:flags}
compiles the main document as a draft:
%
\begin{center}
\begin{tabular}{l}
|\def\version{draft}|\\
|\input{childdoc.def}|\\
|\childdocforward{|\textit{main}|}|
\end{tabular}
\end{center}
%
Likewise, the following files |final|\textit{nn}|.tex|
compile the final version of the child document
|child|\textit{nn}|.tex|:
%
\begin{center}
\begin{tabular}{l}
|\def\version{final}|\\
|\input{childdoc.def}|\\
|\childdocforwardprefix{final}{child}|
\end{tabular}
\end{center}
%

Note that when several versions of a main file and/or of each child file
are to be generated, it may be convenient to set up a |Makefile| or
shell script to automatise the process.

%%%%%%%%%%%%%%%%%%%%%%%%%%%%%%%%%%%%%%%%%%%%%%%%%%%%%%%%%%%%%%%%%%%%%%%%%%%%%%%%
\subsection{Command Line Processing}
\label{sec:commandline}

The effect of redirection files can also be achieved by invoking
the \LaTeX{} compiler with a more elaborate command line.
Most conveniently this should be done as part
of a shell script or a |Makefile|.

When using \textsf{childdoc} in the main file, the following
command lines effectively perform a redirection
(note that depending on the shell being used,
backslashes may have to be doubled: `|\|' $\to$ `|\\|'):
%
\begin{center}
|... -jobname "|\textit{target}|" |\\|"|[\textit{flags}]%
|\input{childdoc.def}\childdocforward[|\textit{main}|]{|\textit{dest}|}"|
\end{center}
%
Here \textit{target} is the name of the output file,
\textit{main} is the name of the main file
and \textit{dest} is the name of the main or child file to be processed
(all filenames without extensions).
The optional argument \textit{main} can be omitted
if \textit{main} matches \textit{dest}.
Optionally, compilation \textit{flags} can be defined via |\def| commands.
This command line makes the \TeX{} engine believe
it is compiling the file \textit{target}
whose content is specified as the latter parameter.
The provided code then forwards the processing to
\textit{main} or \textit{dest} as described in \secref{sec:forward}.

%%%%%%%%%%%%%%%%%%%%%%%%%%%%%%%%%%%%%%%%%%%%%%%%%%%%%%%%%%%%%%%%%%%%%%%%%%%%%%%%
\subsection{Include by Input}
\label{sec:input}

Including child documents by |\include| has some restrictions by design.
Most notably, the content of a child document always occupies
its own set of pages; pages cannot be shared between child documents.
Usually, this behaviour makes perfect sense
because each child document contain an essential part of the document.
However, in some situations it may be desirable to compose
a document from a collection of parts
without having mandatory page breaks between then.
For this case, the package
provides a mechanism to include parts
by |\input| which can also be processed individually.
However, by construction this mechanism
requires manual handling of the content to be output.

%%%%%%%%%%%%%%%%%%%%%%%%%%%%%%%%%%%%%%%%
\DescribeMacro{\ifchilddocmanual}
The main file should be prepared as usual, see \secref{sec:include}.
However, the document body must make a distinction
between processing of an individual part and of the main document, e.g.:
%
\begin{center}
\begin{tabular}{l}
|\ifchilddocmanual|\\
|\input{\childdocname}|\\
|\||else|\\
\textit{document body with }|\input{|\textit{part}|}|\\
|\||fi|
\end{tabular}
\end{center}
%
The conditional |\ifchilddocmanual| is true whenever
a part to be included by |\input| is being compiled,
and the name of the part is stored in |\childdocname|.

%%%%%%%%%%%%%%%%%%%%%%%%%%%%%%%%%%%%%%%%
\DescribeMacro{\childdocby}
Each part to be included by |\input| should start with:
%
\begin{center}
\begin{tabular}{l}
|\input{childdoc.def}|\\
|\childdocby{|\textit{main}|}|\\
\end{tabular}
\end{center}
%
The directive |\childdocby| is similar to |\childdocof|
described in \secref{sec:include},
but the subsequent selection of content must be done manually.
To that end, both |\ifchilddoc| and |\ifchilddocmanual|
will be true upon processing of a part,
and the name of the part is stored in |\childdocname|.
Note that |\jobname| will be set to the filename of the current part
so that each part receives an individual |.aux| file
that does not interfere with the |.aux| file(s) of the main document.
This behaviour can be altered by the alternative form
|\childdocby[*]{|\textit{main}|}| (with a non-empty optional argument)
which uses the |.aux| file of the main document
by setting |\jobname| to \textit{main}.

%%%%%%%%%%%%%%%%%%%%%%%%%%%%%%%%%%%%%%%%%%%%%%%%%%%%%%%%%%%%%%%%%%%%%%%%%%%%%%%%
\subsection{Driver Development}
\label{sec:driver}

The \textsf{childdoc} mechanism can also be use for the development
of definition files such as \LaTeX{} styles or classes.
This case differs from the above setup with multiple parts
included by |\include| in that no |\includeonly| should be invoked.
This can be achieved by starting the include file
(before |\ProvidesPackage|) with:
%
\begin{center}
\begin{tabular}{l}
|\input{childdoc.def}|\\
|\childdocforward{|\textit{main}|}|\\
\end{tabular}
\end{center}
%
or alternatively with:
%
\begin{center}
\begin{tabular}{l}
|\input{childdoc.def}|\\
|\childdocby{|\textit{main}|}|\\
\end{tabular}
\end{center}
%
Both forms have slightly different effects as described above.
The main file is prepared as usual, see \secref{sec:include}.

%%%%%%%%%%%%%%%%%%%%%%%%%%%%%%%%%%%%%%%%%%%%%%%%%%%%%%%%%%%%%%%%%%%%%%%%%%%%%%%%
\subsection{Legacy Detection}
\label{sec:detection}

The directive |\childdocmain| in the main file can detect
whether the complete document or merely a child is to be compiled
even without using the directive |\childdocof|.
This method is deprecated because it is less robust
and there is no compelling reason to use it;
it is merely provided for backward compatibility
and it may be removed in future versions.

If the detection mechanism is to be used,
it is mandatory to correctly specify
the filename of the main file as the argument of |\childdocmain|:
%
\begin{center}
\begin{tabular}{l}
|\input{childdoc.def}|\\
|\childdocmain{|\textit{main}|}|\\
\end{tabular}
\end{center}
%
If |\jobname| does not match the argument \textit{main} of |\childdocmain|,
it is assumed that |\jobname| points to the child file to be compiled.
When using |\childdocmain| with the main file specified as argument,
it suffices to start a child file
with just |\input{|\textit{main}|}|
without loading of the package and using |\childdocof|.
If instead all processing is done
with the appropriate \textsf{childdoc} directives,
the argument of \textit{main} of |\childdocmain| can be empty.

An alternative version of the command line processing described
in \secref{sec:commandline} using the detection mechanism reads:
%
\begin{center}
|... -jobname "|\textit{target}|" "|[\textit{flags}]%
[|\def\jobname{|\textit{dest}|}|]|\input{|\textit{main}|}"|
\end{center}

%%%%%%%%%%%%%%%%%%%%%%%%%%%%%%%%%%%%%%%%%%%%%%%%%%%%%%%%%%%%%%%%%%%%%%%%%%%%%%%%
\subsection{Manual Code}
\label{sec:manual}

In case one cannot be certain whether the definitions file |childdoc.def|
is installed on the target \TeX{} distribution
and one prefers not to ship it,
it is conceivable to paste a few relevant commands into the sources.

To that end, drop all statements |\input{childdoc.def}|
and perform the replacements as outlined below.
Instead of |\childdocmain{|\textit{main}|}| add the following code
to the top of the main file:
%
\begin{center}
\begin{tabular}{l}
|\||ifdefined\childdocname\endinput\||fi\newif\ifchilddoc|\\
|\edef\childdocname{\scantokens\expandafter{\jobname\noexpand}}|\\
|\def\childdocmain{|\textit{main}|}\||ifx\childdocmain\childdocname\||else|\\
|\childdoctrue\includeonly{\childdocname}\let\jobname\childdocmain\||fi|\\
\end{tabular}
\end{center}
%
Instead of |\childdocof{|\textit{main}|}| just include the main file
at the top of each child file:
%
\begin{center}
|\input{|\textit{main}|}|
\end{center}
%
A simple redirection |\childdocforward{|\textit{dest}|}| is achieved by:
%
\begin{center}
|\def\jobname{|\textit{dest}|}\input{\jobname}|
\end{center}
%
The redirection with prefix
|\childdocforwardprefix[|\textit{prefix}|]{|\textit{dest}|}|
is accomplished by:
%
\begin{center}
\begin{tabular}{l}
|{\edef\jobname{\scantokens\expandafter{\jobname\noexpand}}|\\
|\def\redirectjob |\textit{prefix}|#1~~~{\gdef\jobname{|\textit{dest}|#1}}|\\
|\expandafter\redirectjob\jobname~~~}\input{\jobname}|
\end{tabular}
\end{center}

In an alternative approach,
child documents can be compiled by a specific command line
without additional code or specific definitions:
%
\begin{center}
|... -jobname "|\textit{target}|" "|[\textit{flags}]%
|\includeonly{|\textit{dest}|}\input{|\textit{main}|}"|
\end{center}
%

%%%%%%%%%%%%%%%%%%%%%%%%%%%%%%%%%%%%%%%%%%%%%%%%%%%%%%%%%%%%%%%%%%%%%%%%%%%%%%%%
%%%%%%%%%%%%%%%%%%%%%%%%%%%%%%%%%%%%%%%%%%%%%%%%%%%%%%%%%%%%%%%%%%%%%%%%%%%%%%%%
\section{Information}

%%%%%%%%%%%%%%%%%%%%%%%%%%%%%%%%%%%%%%%%%%%%%%%%%%%%%%%%%%%%%%%%%%%%%%%%%%%%%%%%
\subsection{Copyright}

Copyright \copyright{} 2017--2018 Niklas Beisert

This work may be distributed and/or modified under the
conditions of the \LaTeX{} Project Public License, either version 1.3
of this license or (at your option) any later version.
The latest version of this license is in
  \url{http://www.latex-project.org/lppl.txt}
and version 1.3 or later is part of all distributions of \LaTeX{}
version 2005/12/01 or later.

This work has the LPPL maintenance status `maintained'.

The Current Maintainer of this work is Niklas Beisert.

This work consists of the files |README.txt|, |childdoc.ins| and |childdoc.dtx|
as well as the derived files |childdoc.def|, |cdocsamp.tex|
with |cdocsch1.tex|, |cdocsch2.tex|, |cdocspt3.tex|, |cdocspt4.tex|,
|cdocsdrf.tex|, |cdocsfn1.tex|, |cdocsfn2.tex|
as well as |childdoc.pdf|.

%%%%%%%%%%%%%%%%%%%%%%%%%%%%%%%%%%%%%%%%%%%%%%%%%%%%%%%%%%%%%%%%%%%%%%%%%%%%%%%%
\subsection{Files and Installation}

The package consists of the files:
%
\begin{center}
\begin{tabular}{ll}
    |README.txt|   & readme file \\
    |childdoc.ins| & installation file \\
    |childdoc.dtx| & source file \\
    |childdoc.def| & definition file \\
    |cdocsamp.tex| & sample main file \\
    |cdocsch1.tex| & sample include file \\
    |cdocsch2.tex| & sample include file \\
    |cdocspt3.tex| & sample part file \\
    |cdocspt4.tex| & sample part file \\
    |cdocsdrf.tex| & sample redirection file \\
    |cdocsfn1.tex| & sample redirection file \\
    |cdocsfn2.tex| & sample redirection file \\
    |childdoc.pdf| & manual
\end{tabular}
\end{center}
%
The distribution consists of the files
|README.txt|, |childdoc.ins| and |childdoc.dtx|.
%
\begin{itemize}
\item
Run (pdf)\LaTeX{} on |childdoc.dtx|
to compile the manual |childdoc.pdf| (this file).
\item
Run \LaTeX{} on |childdoc.ins| to create the definitions file |childdoc.def|
and the sample |cdocsamp.tex| with include files
|cdocsch1.tex|, |cdocsch2.tex|, |cdocspt3.tex|, |cdocspt4.tex|,
|cdocsdrf.tex|, |cdocsfn1.tex|, |cdocsfn2.tex|.
Then copy the file |childdoc.def| to an appropriate directory of your \LaTeX{}
distribution, e.g.\ \textit{texmf-root}|/tex/latex/childdoc|.
\end{itemize}

%%%%%%%%%%%%%%%%%%%%%%%%%%%%%%%%%%%%%%%%%%%%%%%%%%%%%%%%%%%%%%%%%%%%%%%%%%%%%%%%
\subsection{Related CTAN Packages}

There are several other packages which offer a similar functionality:
%
\begin{itemize}
\item
The packages
\href{http://ctan.org/pkg/docmute}{\textsf{docmute}},
\href{http://ctan.org/pkg/includex}{\textsf{includex}} and
\href{http://ctan.org/pkg/standalone}{\textsf{standalone}}
provide commands to include only the document body of
a child file thus allowing both files to be compiled individually.
\item
The packages \href{http://ctan.org/pkg/subdocs}{\textsf{subdocs}}
and \href{http://ctan.org/pkg/subfiles}{\textsf{subfiles}}
provide structures in which the main and child documents can be
encapsulated and allowing them to be compiled individually.
The inclusion mechanism is different from the conventional |\include|.
\item
The package \href{http://ctan.org/pkg/combine}{\textsf{combine}}
is an elaborate solution to combine several documents into one.
\end{itemize}
%
See also the CTAN topic \href{http://ctan.org/topic/subdocs}{\textsf{subdocs}}
for further related packages.
The present package differs from the above solutions in that
a document structure constructed with the conventional |\include| mechanism
just needs two extra commands at the top of every file
such that all constituent files can be compiled individually.

%%%%%%%%%%%%%%%%%%%%%%%%%%%%%%%%%%%%%%%%%%%%%%%%%%%%%%%%%%%%%%%%%%%%%%%%%%%%%%%%
%\subsection{Feature Suggestions}
%
%The following is a list of features which may be useful for future
%versions of this package:
%%
%\begin{itemize}
%\item
%\ldots
%\end{itemize}

%%%%%%%%%%%%%%%%%%%%%%%%%%%%%%%%%%%%%%%%%%%%%%%%%%%%%%%%%%%%%%%%%%%%%%%%%%%%%%%%
\subsection{Revision History}

%%%%%%%%%%%%%%%%%%%%%%%%%%%%%%%%%%%%%%%%
\paragraph{v2.0:} 2018/12/30

\begin{itemize}
\item
immediate forward processing
\item
added |\childdocby| mechanism
\item
manual restructured
\end{itemize}

%%%%%%%%%%%%%%%%%%%%%%%%%%%%%%%%%%%%%%%%
\paragraph{v1.6:} 2018/01/17

\begin{itemize}
\item
application for development of include files
\item
corrections to manual
\end{itemize}

%%%%%%%%%%%%%%%%%%%%%%%%%%%%%%%%%%%%%%%%
\paragraph{v1.5:} 2017/05/21

\begin{itemize}
\item
more complete structuring introduced
\item
|\childdocof| introduced
\item
|\childdoc| renamed to |\childdocmain|
\item
|\childredirect| renamed to |\childdocforward| and |\childdocforwardprefix|
and functionality expanded
\end{itemize}

%%%%%%%%%%%%%%%%%%%%%%%%%%%%%%%%%%%%%%%%
\paragraph{v1.0:} 2017/04/27

\begin{itemize}
\item
manual and install package
\item
first version published on CTAN
\end{itemize}

%%%%%%%%%%%%%%%%%%%%%%%%%%%%%%%%%%%%%%%%
\paragraph{v0.6:} 2017/04/26

\begin{itemize}
\item
redirection mechanism added
\end{itemize}

%%%%%%%%%%%%%%%%%%%%%%%%%%%%%%%%%%%%%%%%
\paragraph{v0.5:} 2017/04/26

\begin{itemize}
\item
functionality in definition file
\end{itemize}


%%%%%%%%%%%%%%%%%%%%%%%%%%%%%%%%%%%%%%%%%%%%%%%%%%%%%%%%%%%%%%%%%%%%%%%%%%%%%%%%
%%%%%%%%%%%%%%%%%%%%%%%%%%%%%%%%%%%%%%%%%%%%%%%%%%%%%%%%%%%%%%%%%%%%%%%%%%%%%%%%
%%%%%%%%%%%%%%%%%%%%%%%%%%%%%%%%%%%%%%%%%%%%%%%%%%%%%%%%%%%%%%%%%%%%%%%%%%%%%%%%
\appendix

\settowidth\MacroIndent{\rmfamily\scriptsize 000\ }

 \DocInput{childdoc.dtx}

\end{document}
%</driver>
% \fi
%
% %%%%%%%%%%%%%%%%%%%%%%%%%%%%%%%%%%%%%%%%%%%%%%%%%%%%%%%%%%%%%%%%%%%%%%%%%%%%%%
% %%%%%%%%%%%%%%%%%%%%%%%%%%%%%%%%%%%%%%%%%%%%%%%%%%%%%%%%%%%%%%%%%%%%%%%%%%%%%%
% \section{Sample}
%\iffalse
%<*samplemain>
%\fi
%
% The following presents a sample document
% with two chapters, two parts, a title page,
% a compile flag as well as three forwarding files to set the flag.
% It consists of eight |.tex| files:
% \begin{center}
% \begin{tabular}{ll}
% |cdocsamp.tex|&main file\\
% |cdocsch1.tex|&include file for chapter 1\\
% |cdocsch2.tex|&include file for chapter 2\\
% |cdocspt3.tex|&include file for part 3\\
% |cdocspt4.tex|&include file for part 4\\
% |cdocsdrf.tex|&forwarding file for main file in draft mode\\
% |cdocsfi1.tex|&forwarding file for final version of chapter 1\\
% |cdocsfi2.tex|&forwarding file for final version of chapter 2\\
% \end{tabular}
% \end{center}
% Each of the eight files can be compiled directly by the \LaTeX{} compiler.
%
% %%%%%%%%%%%%%%%%%%%%%%%%%%%%%%%%%%%%%%
% \paragraph{Main File.}
%
% The main file is called |cdocsamp.tex|.
%
% Load the \textsf{childdoc} definitions and
% declare the filename for the main document:
%    \begin{macrocode}
\input{childdoc.def}
\childdocmain{}
%    \end{macrocode}

% Optional override for |\version| flag:
%    \begin{macrocode}
%%\ifchilddoc\else\providecommand{\version}{draft}\fi
%    \end{macrocode}

% Define the default values for the |\version| flag
% (|final| for the main file and |draft| for childs):
%    \begin{macrocode}
\ifchilddoc
\providecommand{\version}{draft}
\else
\providecommand{\version}{final}
\fi
%    \end{macrocode}

% Load the standard document class:
%    \begin{macrocode}
\documentclass[12pt]{article}
%    \end{macrocode}

% Start the document body:
%    \begin{macrocode}
\begin{document}
%    \end{macrocode}

% Declare a title page.
% Print title, part of document being processed and version flag:
%    \begin{macrocode}
\addtocounter{page}{-1}
\begin{center}
{\LARGE\bfseries{}childdoc example\par}
\vspace{1cm}
\ifchilddoc
\ifchilddocmanual part\else chapter\fi:
`\childdocname' of `\childdocjob'\par
\else
main document: `\childdocjob'\par
\fi
version: \version\par
\end{center}
\newpage
%    \end{macrocode}

% Manually include selected file,
% otherwise process as usual:
%    \begin{macrocode}
\ifchilddocmanual
\section*{part `\childdocname'}
\input{\childdocname}
\else
%    \end{macrocode}

% Include the two chapters:
%    \begin{macrocode}
\include{cdocsch1}
\include{cdocsch2}
%    \end{macrocode}

% Include the two parts unless only chapters should be displayed:
%    \begin{macrocode}
\ifchilddoc\else
\section{part three}
\input{cdocspt3}
\section{part four}
\input{cdocspt4}
\fi
%    \end{macrocode}

% Process as usual until here:
%    \begin{macrocode}
\fi
%    \end{macrocode}

% End of document body:
%    \begin{macrocode}
\end{document}
%    \end{macrocode}
%\iffalse
%</samplemain>
%\fi
%
% %%%%%%%%%%%%%%%%%%%%%%%%%%%%%%%%%%%%%%
% \paragraph{Chapter Include Files.}
%
% The include files are called |cdocsch1.tex| and |cdocsch2.tex|.
%
%\iffalse
%<*samplechap1|samplechap2>
%\fi

% Optional override for |\version| flag:
%    \begin{macrocode}
%%\providecommand{\version}{final}
%    \end{macrocode}

% Include the main document:
%    \begin{macrocode}
\input{childdoc.def}
\childdocof{cdocsamp}
%    \end{macrocode}

%\iffalse
%</samplechap1|samplechap2>
%\fi
%
%\iffalse
%<*samplechap1>
%\fi
% Some text for chapter 1:
%    \begin{macrocode}
\section{one}
some text in chapter one
%    \end{macrocode}

%\iffalse
%</samplechap1>
%\fi
% Some text for chapter 2:
%\iffalse
%<*samplechap2>
%\fi
%    \begin{macrocode}
\section{two}
more text in chapter two
%    \end{macrocode}

%\iffalse
%</samplechap2>
%\fi
%
% %%%%%%%%%%%%%%%%%%%%%%%%%%%%%%%%%%%%%%
% \paragraph{Part Include Files.}
%
% The include files are called |cdocspt3.tex| and |cdocspt4.tex|.
%
%\iffalse
%<*samplepart3|samplepart4>
%\fi

% Optional override for |\version| flag:
%    \begin{macrocode}
%%\providecommand{\version}{final}
%    \end{macrocode}

% Include the main document:
%    \begin{macrocode}
\input{childdoc.def}
\childdocby{cdocsamp}
%    \end{macrocode}

%\iffalse
%</samplepart3|samplepart4>
%\fi
%
%\iffalse
%<*samplepart3>
%\fi
% Some text for part 3:
%    \begin{macrocode}
some text in part three
%    \end{macrocode}

%\iffalse
%</samplepart3>
%\fi
% Some text for part 4:
%\iffalse
%<*samplepart4>
%\fi
%    \begin{macrocode}
more text in part four
%    \end{macrocode}

%\iffalse
%</samplepart4>
%\fi
%
% %%%%%%%%%%%%%%%%%%%%%%%%%%%%%%%%%%%%%%
% \paragraph{Forwarding for a Complete Draft.}
%
% The following forwarding file |cdocsdrf.tex|
% compiles the main document in draft mode:
%\iffalse
%<*sampledraft>
%\fi
%    \begin{macrocode}
\def\version{draft}
\input{childdoc.def}
\childdocforward{cdocsamp}
%    \end{macrocode}

%\iffalse
%</sampledraft>
%\fi
%
% %%%%%%%%%%%%%%%%%%%%%%%%%%%%%%%%%%%%%%
% \paragraph{Forwarding for Final Version of the Chapters.}
%
% The following forwarding files |cdocsfn1.tex| and |cdocsfn2.tex|
% (with identical content)
% compile the final versions of the child documents
% |cdocsch1.tex| and |cdocsch2.tex|, respectively:
%\iffalse
%<*samplefinal>
%\fi
%    \begin{macrocode}
\def\version{final}
\input{childdoc.def}
\childdocforwardprefix[cdocsamp]{cdocsfn}{cdocsch}
%    \end{macrocode}

%\iffalse
%</samplefinal>
%\fi
%
% %%%%%%%%%%%%%%%%%%%%%%%%%%%%%%%%%%%%%%
% \paragraph{Command Line Processing.}
%
% The following three command lines generate the output files
% |cdocscld|, |cdocscl1| and |cdocscl2|
% which should be identical to
% |cdocsdrf|, |cdocsch1| and |cdocsfn2|, respectively:
% \begin{center}
% \begin{tabular}{l}
% |latex -jobname cdocscld \|\\
% |  "\def\version{draft}\input{childdoc.def}\childdocforward{cdocsamp}"|\\
% |latex -jobname cdocscl1 \|\\
% |  "\input{childdoc.def}\childdocforward[cdocsamp]{cdocsch1}"|\\
% |latex -jobname cdocscl2 \|\\
% |  "\def\version{final}\input{childdoc.def}\childdocforward{cdocsch2}"|
% \end{tabular}
% \end{center}
% Note that the trailing backslash on each first line
% merely continues the input to the second line
% (for convenient cut ant paste).
% Furthermore, the command |latex| can be replaced by any
% of its alternative versions such as |pdflatex|.
%
% %%%%%%%%%%%%%%%%%%%%%%%%%%%%%%%%%%%%%%%%%%%%%%%%%%%%%%%%%%%%%%%%%%%%%%%%%%%%%%
% %%%%%%%%%%%%%%%%%%%%%%%%%%%%%%%%%%%%%%%%%%%%%%%%%%%%%%%%%%%%%%%%%%%%%%%%%%%%%%
% \section{Implementation}
%\iffalse
%<*package>
%\fi
%
% This section describes the definitions file |childdoc.def|.

% The definitions cannot be loaded using |\usepackage| or |\RequirePackage|
% which has a mechanism to prevent loading a style file more than once.
% When loading the definitions by means of |\input|
% multiple instances have to be prevented manually:
%\iffalse
%This code needs to be before the `\ProvidesFile' directive
%which is defined at the beginning of this file.
%Therefore it is also placed there and commented out here.
%</package>
%<*discard>
%\fi
%    \begin{macrocode}
\ifdefined\childdocmain\endinput\fi
%    \end{macrocode}
%\iffalse
%</discard>
%<*package>
%\fi
%
% \macro{\ifchilddoc}
% \macro{\ifchilddocmanual}
% The conditional |\ifchilddoc| tells whether a
% child (true) or main (false) document is being compiled.
% The conditional |\ifchilddocmanual| tells whether
% the |\includeonly| mechanism is used (false) or
% the selection of child files must be performed manually (true).
% The definitions initialise to false:
%    \begin{macrocode}
\newif\ifchilddoc
\newif\ifchilddocmanual
%    \end{macrocode}

% \macro{\childdocname}
% \macro{\childdocjob}
% The macro |\childdocname| stores the name of the main document
% to be compiled. The macro |\childdocjob| stores the name of
% the document on which the \LaTeX{} compiler was originally invoked.
% The content of |\jobname| cannot be compared
% to filenames specified in the source due to different catcodes.
% The following code rescans |\jobname|, stores the result
% in |\childdocname| and saves a copy in |\childdocjob|:
%    \begin{macrocode}
\edef\childdocname{\scantokens\expandafter{\jobname\noexpand}}
\let\childdocjob\childdocname
%    \end{macrocode}

% \macro{\childdocdisable}
% The macro |\childdocdisable| prevents the main file
% from being processed more than once.
% At this stage, the main document command |\childdocmain|
% is assumed to be called once again where it should do nothing.
% Any subsequent call to it should prevent
% a secondary processing of the main document
% It overwrites the forwarding commands
% |\childdocof| and |\childdocforward|
% with empty macros to prevent further inclusions of the main document:
%    \begin{macrocode}
\newcommand{\childdocdisable}
{
  \renewcommand{\childdocmain}[1]{\renewcommand{\childdocmain}[1]{\endinput}}
  \renewcommand{\childdocof}[1]{}
  \renewcommand{\childdocby}[2][]{}
  \renewcommand{\childdocforward}[2][]{}
  \renewcommand{\childdocdisable}{}
}
%    \end{macrocode}

% \macro{\childdocmain}
% The macro |\childdocmain| is to be called at the top of the main file
% with nothing or the main filename (without extension) as argument.
% First, it breaks loops.
% If the argument is not empty and does not match |\childdocname|
% (which is set by the first inclusion of |childdoc.def|),
% |\ifchilddoc| is set to true, |\includeonly| is applied to the child file
% and |\jobname| is set to the main file
% (for proper handling of |.aux| files):
%    \begin{macrocode}
\newcommand{\childdocmain}[1]
{
  \childdocdisable\childdocmain{}
  \if?#1?\else
    \begingroup
      \def\childdoctmp{#1}
      \ifx\childdoctmp\childdocname
        \def\childdoctmp{}
      \else
        \def\childdoctmp
        {
          \childdoctrue
          \includeonly{\childdocname}
          \def\childdocjob{#1}
          \def\jobname{#1}
        }
      \fi
      \expandafter
    \endgroup
    \childdoctmp
  \fi
}
%    \end{macrocode}

% \macro{\childdocof}
% The command |\childdocof| redirects
% compilation to the main file |#1|.
%    \begin{macrocode}
\newcommand{\childdocof}[1]
{
  \childdocdisable
  \childdoctrue
  \includeonly{\childdocname}
  \def\jobname{#1}
  \def\childdocjob{#1}
  \input{#1}
}
%    \end{macrocode}

% \macro{\childdocby}
% The command |\childdocby| ....
%    \begin{macrocode}
\newcommand{\childdocby}[2][]
{
  \childdocdisable
  \childdoctrue
  \childdocmanualtrue
  \if?#1?\else
    \def\jobname{#2}
  \fi
  \def\childdocjob{#2}
  \input{#2}
  \endinput
}
%    \end{macrocode}

% \macro{\childdocforward}
% The command |\childdocforward| redirects
% compilation to the main file or
% (if the optional argument is given) a child file.
% Parameters are set as if the main file
% or a child file starting with |\childdocof| was compiled.
% Then compilation is handed over to the main file:
%    \begin{macrocode}
\newcommand{\childdocforward}[2][]
{
  \begingroup
    \if?#1?
      \def\childdoctmp
      {
        \def\childdocname{#2}
        \def\childdocjob{#2}
        \def\jobname{#2}
        \input{#2}
        \endinput
      }
    \else
      \def\childdoctmp
      {
        \childdocdisable
        \def\childdocname{#2}
        \childdoctrue
        \includeonly{#2}
        \def\childdocjob{#1}
        \def\jobname{#1}
        \input{#1}
        \endinput
      }
    \fi
    \expandafter
  \endgroup
  \childdoctmp
}
%    \end{macrocode}

% \macro{\childdocforwardprefix}
% The command |\childdocforwardprefix| redirects
% compilation to the main or a child file by means of a pattern.
% The prefix |#1| in the current filename is replaced by |#2|
% and the suffix of the current filename is kept
% (it is assumed that the filename does not contain the substring `|~~~|'
% which is used as a delimiter).
% Compilation is handed over to the new file by |\childdocforward|:
%    \begin{macrocode}
\newcommand{\childdocforwardprefix}[3][]
{
  \begingroup
    \def\childdocextract #2##1~~~{\def\childdoctmp{\childdocforward[#1]{#3##1}}}
    \expandafter\childdocextract\childdocname~~~
    \expandafter
  \endgroup
  \childdoctmp
}
%    \end{macrocode}

% \macro{\childdoc}
% The deprecated macro |\childdoc| is a legacy version of |\childdocmain|:
%    \begin{macrocode}
\newcommand{\childdoc}{\childdocmain}
%    \end{macrocode}

% \macro{\childdocredirect}
% The deprecated macro |\childdocredirect| is a legacy version
% of |\childdocforward| and |\childdocforwardprefix|:
%    \begin{macrocode}
\newcommand{\childdocredirect}[2][]
{
  \begingroup
    \if?#1?
      \def\childdoctmp{\childdocforward{#2}}
    \else
      \def\childdoctmp{\childdocforwardprefix{#1}{#2}}
    \fi
    \expandafter
  \endgroup
  \childdoctmp
}
%    \end{macrocode}

%\iffalse
%</package>
%\fi
%
\endinput
|\\
|\childdocforwardprefix[|\textit{main}|]{|\textit{prefix}|}{|\textit{dest}|}|
\end{tabular}
\end{center}
%
the destination file is determined by a pattern
depending on the current file:
To make this work, the current file must be called
`{\textit{prefix}\hspace{0.2em}\textit{suffix}}'
with \textit{prefix} matching precisely the argument.
Processing is then passed on to the file
`{\textit{dest}\hspace{0.2em}\textit{suffix}}'.
Surely, the same effect is achieved by
directly specifying the
argument `{\textit{dest}\hspace{0.2em}\textit{suffix}}'
in the first form.
However, that requires to set up a different file
for each child. With the alternative form of the command
all these files can have exactly the same content
which simplifies setting them up and maintaining them.

For example, the following file |draft.tex|
with a compilation flag |\version| as described in \secref{sec:flags}
compiles the main document as a draft:
%
\begin{center}
\begin{tabular}{l}
|\def\version{draft}|\\
|% \iffalse
%
% childdoc.dtx Copyright (C) 2017-2018 Niklas Beisert
%
% This work may be distributed and/or modified under the
% conditions of the LaTeX Project Public License, either version 1.3
% of this license or (at your option) any later version.
% The latest version of this license is in
%   http://www.latex-project.org/lppl.txt
% and version 1.3 or later is part of all distributions of LaTeX
% version 2005/12/01 or later.
%
% This work has the LPPL maintenance status `maintained'.
%
% The Current Maintainer of this work is Niklas Beisert.
%
% This work consists of the files childdoc.dtx and childdoc.ins
% and the derived files childdoc.def and cdocsamp.tex with
% cdocsch1.tex, cdocsch2.tex, cdocsdrf.tex, cdocsfn1.tex, cdocsfn2.tex.
%
%<package>\ifdefined\childdocmain\endinput\fi
%<package>\ProvidesFile{childdoc.def}[2018/12/30 v2.0 child document driver]
%<samplemain>\ProvidesFile{cdocsamp.tex}[2018/12/30 v2.0 sample for childdoc]
%<*driver>
%\ProvidesFile{childdoc.drv}[2018/12/30 v2.0 childdoc reference manual file]
\PassOptionsToClass{10pt,a4paper}{article}
\documentclass{ltxdoc}

\usepackage[margin=35mm]{geometry}
\usepackage{hyperref}
\usepackage{hyperxmp}
\usepackage[usenames]{color}

\hypersetup{colorlinks=true}
\hypersetup{pdfstartview=FitH}
\hypersetup{pdfpagemode=UseNone}
\hypersetup{pdfsource={}}
\hypersetup{pdflang={en-UK}}
\hypersetup{pdfcopyright={Copyright 2017-2018 Niklas Beisert.
  This work may be distributed and/or modified under the
  conditions of the LaTeX Project Public License, either version 1.3
  of this license or (at your option) any later version.}}
\hypersetup{pdflicenseurl={http://www.latex-project.org/lppl.txt}}
\hypersetup{pdfcontactaddress={ETH Zurich, ITP, HIT K,
  Wolfgang-Pauli-Strasse 27}}
\hypersetup{pdfcontactpostcode={8093}}
\hypersetup{pdfcontactcity={Zurich}}
\hypersetup{pdfcontactcountry={Switzerland}}
\hypersetup{pdfcontactemail={nbeisert@itp.phys.ethz.ch}}
\hypersetup{pdfcontacturl={http://people.phys.ethz.ch/\xmptilde nbeisert/}}

\newcommand{\secref}[1]{\hyperref[#1]{section \ref*{#1}}}

\parskip1ex
\parindent0pt
\let\olditemize\itemize
\def\itemize{\olditemize\parskip0pt}

\begin{document}

\title{The \textsf{childdoc} Package}
\hypersetup{pdftitle={The childdoc Package}}
\author{Niklas Beisert\\[2ex]
  Institut f\"ur Theoretische Physik\\
  Eidgen\"ossische Technische Hochschule Z\"urich\\
  Wolfgang-Pauli-Strasse 27, 8093 Z\"urich, Switzerland\\[1ex]
  \href{mailto:nbeisert@itp.phys.ethz.ch}
  {\texttt{nbeisert@itp.phys.ethz.ch}}}
\hypersetup{pdfauthor={Niklas Beisert}}
\hypersetup{pdfsubject={Manual for the LaTeX2e Package childdoc}}
\date{30 December 2018, \textsf{v2.0}}
\maketitle

\begin{abstract}\noindent
\textsf{childdoc} is a \LaTeXe{} package
that enables the direct compilation
of document sections included by |\include|
to individual files.
\end{abstract}

\begingroup
\parskip0ex
\tableofcontents
\endgroup

%%%%%%%%%%%%%%%%%%%%%%%%%%%%%%%%%%%%%%%%%%%%%%%%%%%%%%%%%%%%%%%%%%%%%%%%%%%%%%%%
%%%%%%%%%%%%%%%%%%%%%%%%%%%%%%%%%%%%%%%%%%%%%%%%%%%%%%%%%%%%%%%%%%%%%%%%%%%%%%%%
\section{Introduction}

\LaTeX{} provides a mechanism to structure a large document (such as a book)
into a main file and several child files (containing the chapters)
using the |\include| command.
This mechanism is beneficial for documents
which span hundreds of pages in order to
make the source file(s) more manageable.
Moreover, compilation can be restricted to
selected child files by means of the |\includeonly| command.
The latter feature can be used to reduce the compilation time while editing
(this was significantly more useful in the earlier days of \LaTeX{})
or to generate a smaller document which is easier to navigate.
Another application of |\includeonly| is to generate
documents consisting of selected parts of the complete document.

However, there are a few drawbacks of the plain |\include| mechanism:
\begin{itemize}
\item
The child files cannot be compiled on their own,
they can only be compiled via the main file.
A naive editing environment
(such as a text editor with an option
to have the current file processed by \LaTeX)
may require one to switch to the main file before compiling;
attempting to compile the child file produces errors.
\item
The main file must be modified (each time)
to adjust the |\includeonly| command
to the present needs. This easily leaves the main file in a messy state.
\item
The generated document will always carry the filename
of the main document. This is inconvenient if
several child files are to be compiled and
to be kept for distribution.
\end{itemize}

The present package provides a simple interface
to make child files individually compilable by \LaTeX{}.
Compiling a child file then has the same effect as compiling
the main file with an |\includeonly| command
to select the appropriate child.
Moreover the generated document will carry the name of the child
rather than the main file.
This resolves all three above issues.

This feature is meant to make the editing of books,
thesis documents and lecture notes somewhat more convenient.
However, the package can also be used efficiently for
composing a series of documents (such as exercise sheets)
which are typically distributed individually.
It then assists the author in generating the individual documents
(potentially in different versions)
as well as a document containing the collected series.
Another application is in developing style files
or other kinds of included material
where compilation of the style file could redirect
to a sample or test file.

%%%%%%%%%%%%%%%%%%%%%%%%%%%%%%%%%%%%%%%%%%%%%%%%%%%%%%%%%%%%%%%%%%%%%%%%%%%%%%%%
%%%%%%%%%%%%%%%%%%%%%%%%%%%%%%%%%%%%%%%%%%%%%%%%%%%%%%%%%%%%%%%%%%%%%%%%%%%%%%%%
\section{Usage}

First of all, the package \textsf{childdoc} is \emph{not} a standard
\LaTeXe{} |.sty| style file! Therefore it needs to be invoked in
a non-standard way.

%%%%%%%%%%%%%%%%%%%%%%%%%%%%%%%%%%%%%%%%%%%%%%%%%%%%%%%%%%%%%%%%%%%%%%%%%%%%%%%%
\subsection{Included Files}
\label{sec:include}

%%%%%%%%%%%%%%%%%%%%%%%%%%%%%%%%%%%%%%%%
\DescribeMacro{\childdocmain}
To use the package, add the commands
\begin{center}
\begin{tabular}{l}
|\input{childdoc.def}|\\
|\childdocmain{}|\\
\end{tabular}
\end{center}
at the very top of the main \LaTeX{} file,
in particular \emph{before} the |\documentclass| statement!
The argument of |\childdocmain| should be left empty
(but it must be present).

%%%%%%%%%%%%%%%%%%%%%%%%%%%%%%%%%%%%%%%%
\DescribeMacro{\childdocof}
Furthermore, add the commands
\begin{center}
\begin{tabular}{l}
|\input{childdoc.def}|\\
|\childdocof{|\textit{main}|}|\\
\end{tabular}
\end{center}
at the top of every child file \textit{child}
which is included by |\include{|\textit{child}|}|
from within the main file
(or at least for those files to be compiled individually).
The argument \textit{main} must be the filename of the main file.

There are a couple of
considerations in setting up the main and child documents:

%%%%%%%%%%%%%%%%%%%%%%%%%%%%%%%%%%%%%%%%
\paragraph{Restrictions.}

Please note the following restrictions:
\begin{itemize}
\item
|\childdocmain| must be called with one argument \textit{main}
to ensure compatibility with earlier version of the package.
It must either be empty (|\childdocmain{}|)
or precisely match the filename of the main file in which it is specified.
See \secref{sec:detection} for further information.
\item
The filename \textit{main} must be specified without the |.tex| extension.
\item
The filename \textit{main} is case sensitive
(even in case-insensitive file systems)
due to internal string comparison.
\item
The argument \textit{main} should be fully expanded, it cannot be a macro.
\item
Subdirectories and special characters should be avoided in filenames.
\item
The command |\childdocmain{|\textit{main}|}| must be followed by a whitespace.
It should not be followed immediately by another command
or by a comment mark `|%|'.
This is because the \TeX{} parser reads the token immediately following
the argument of |\childdocmain| and puts it
at the beginning of every child section;
however, a white\-space is ignored.
\end{itemize}

%%%%%%%%%%%%%%%%%%%%%%%%%%%%%%%%%%%%%%%%
\paragraph{Content of Main File.}

It is advisable to place all content in the child files included by |\include|.
Any output contained in the main file will appear in all child documents
unless suppressed manually;
it cannot be suppressed automatically by the |\includeonly| directive
and thus should normally be avoided.
A method to include some content in the main file
by means of conditional processing is described in \secref{sec:conditional}.

%%%%%%%%%%%%%%%%%%%%%%%%%%%%%%%%%%%%%%%%
\paragraph{Page Numbering.}

When only a part of the document is compiled,
the appropriate numbering of pages
(as well as other status parameters)
is determined from the |.aux| files.
The latter contain information from previous passes.
However this information needs to propagate through
all intermediate child documents.
Therefore the page numbering in child documents may well
be inconsistent until the complete document is compiled at least once.

A useful (if unconventional) way to always ensure a consistent
page numbering is to restart the numbering in each child document
and denote the pages by `\textit{child}|.|\textit{page}'
where \textit{child} represents the chapter/section number of the child file.
This can be achieved by the command
|\numberwithin{page}{|\textit{child}|}|
of the \textsf{amsmath} package
where \textit{child} can be |chapter| or |section|
depending on the chosen structuring.
Alternatively, one can modify the macro |\thepage| appropriately
and reset the counter |page| at the start of each child file.

%%%%%%%%%%%%%%%%%%%%%%%%%%%%%%%%%%%%%%%%%%%%%%%%%%%%%%%%%%%%%%%%%%%%%%%%%%%%%%%%
\subsection{Conditional Processing}
\label{sec:conditional}

The package provides a mechanism to compile different versions
of a document. To customise the versions further some conditional processing
can come in handy to distinguish which version is being compiled.
The package provides two macros to describe the compilation context:

%%%%%%%%%%%%%%%%%%%%%%%%%%%%%%%%%%%%%%%%
\DescribeMacro{\ifchilddoc}
The conditional |\ifchilddoc| distinguishes between the compilation of
child documents and the main document:
%
\begin{center}
|\ifchilddoc |\textit{child-code}| |[|\||else |\textit{main-code}]| \||fi|
\end{center}

%%%%%%%%%%%%%%%%%%%%%%%%%%%%%%%%%%%%%%%%
\DescribeMacro{\childdocname}
\DescribeMacro{\childdocjob}
The macro |\childdocname| contains the filename (without extension)
of the main or child file being processed.
Note that |\childdocjob| will always contain the name of the main file.

%%%%%%%%%%%%%%%%%%%%%%%%%%%%%%%%%%%%%%%%
\paragraph{Title Page.}

Conditional processing can be used to include a title or banner page
in the main document when proper precautions are taken.
Importantly, the code in the main file should ensure that the page counter
(as well as other status parameters which are stored in the |.aux| files)
takes the same value after the conditional processing.
Otherwise the page numbers may take divergent values
depending on which part is compiled.

For example, a title page could be declared by:
%
\begin{center}
\begin{tabular}{l}
|\ifchilddoc\||else|\\
|\addtocounter{page}{-1}|\\
\textit{code for title page}\\
|\newpage|\\
|\||fi|
\end{tabular}
\end{center}
%
A banner page for the child documents can be generated by:
%
\begin{center}
\begin{tabular}{l}
|\ifchilddoc|\\
|\addtocounter{page}{-1}|\\
\textit{code for banner page}\\
|\newpage|\\
|\||fi|
\end{tabular}
\end{center}
%
Here one could write a message such as:
\begin{center}
|This is the part \childdocname{} of \childdocjob{}.|
\end{center}

%%%%%%%%%%%%%%%%%%%%%%%%%%%%%%%%%%%%%%%%%%%%%%%%%%%%%%%%%%%%%%%%%%%%%%%%%%%%%%%%
\subsection{Flags}
\label{sec:flags}

The package makes it easy to generate different versions
of the main or child documents.
To this end compilation flags can be defined
and assigned different default values.
They will be particularly useful in conjunction
with the forwarding mechanism described in \secref{sec:forward}.

For example, it may be useful to have a flag |\version|
which can be set to |draft| or |final|.
The document source will contain some conditional code
depending on the value of |\version|.
Suppose further, the flag should default to |final| for the main file
and to |draft| for child files
which is a natural assignment for editing the document.
This is achieved by placing the following code
in the preamble of the main document
(below the |\childdocmain| directive):
%
\begin{center}
\begin{tabular}{l}
|\ifchilddoc|\\
|\providecommand{\version}{draft}|\\
|\||else|\\
|\providecommand{\version}{final}|\\
|\||fi|
\end{tabular}
\end{center}
%
The definition by |\providecommand| makes sure
that previous definitions are not overwritten.
Further statements |\providecommand{\version}{...}|
can thus be added before the above code to override it.

For the main file, one might add a line
(between |\childdocmain| and the above block)
%
\begin{center}
|%\ifchilddoc\||else\providecommand{\version}{draft}\||fi|
\end{center}
%
which can be uncommented to produce a draft version.
Likewise one can add a line to the very top of a child file
(above the |\childdocof{|\textit{main}|}| directive)
%
\begin{center}
|%\providecommand{\version}{final}|
\end{center}
%
which can be uncommented to produce the final version of this child document.

%%%%%%%%%%%%%%%%%%%%%%%%%%%%%%%%%%%%%%%%%%%%%%%%%%%%%%%%%%%%%%%%%%%%%%%%%%%%%%%%
\subsection{Forwarding}
\label{sec:forward}

Different versions of the main or child documents
using compilation flags as described in \secref{sec:flags}
can be (permanently) stored in different files
for convenient compilation, viewing and distribution.
To this end, the package defines a command
to pass on compilation to a different file:

%%%%%%%%%%%%%%%%%%%%%%%%%%%%%%%%%%%%%%%%
\DescribeMacro{\childdocforward}
The command |\childdocforward| redirects processing to
another source file:
%
\begin{center}
\begin{tabular}{l}
|\input{childdoc.def}|\\
|\childdocforward[|\textit{main}|]{|\textit{dest}|}|\\
\end{tabular}
\end{center}
%
The argument \textit{dest} is the destination file
(without extension).
It should be the main file or one of the child files.
Note that further \textsf{childdoc} directives
such as |\childdocof| and |\childdocforward|
in the indicated file will be processed in this form.
The optional argument \textit{main}
passes on directly to the main file \textit{main}
while pretending to compile the child \textit{dest}.
This form behaves as if \textit{dest}
issues |\childdocof{|\textit{main}|}| right away,
and no further \textsf{childdoc} directives will be processed.

%%%%%%%%%%%%%%%%%%%%%%%%%%%%%%%%%%%%%%%%
\DescribeMacro{\...prefix}
In the alternative form |\childdocforwardprefix|,
%
\begin{center}
\begin{tabular}{l}
|\input{childdoc.def}|\\
|\childdocforwardprefix[|\textit{main}|]{|\textit{prefix}|}{|\textit{dest}|}|
\end{tabular}
\end{center}
%
the destination file is determined by a pattern
depending on the current file:
To make this work, the current file must be called
`{\textit{prefix}\hspace{0.2em}\textit{suffix}}'
with \textit{prefix} matching precisely the argument.
Processing is then passed on to the file
`{\textit{dest}\hspace{0.2em}\textit{suffix}}'.
Surely, the same effect is achieved by
directly specifying the
argument `{\textit{dest}\hspace{0.2em}\textit{suffix}}'
in the first form.
However, that requires to set up a different file
for each child. With the alternative form of the command
all these files can have exactly the same content
which simplifies setting them up and maintaining them.

For example, the following file |draft.tex|
with a compilation flag |\version| as described in \secref{sec:flags}
compiles the main document as a draft:
%
\begin{center}
\begin{tabular}{l}
|\def\version{draft}|\\
|\input{childdoc.def}|\\
|\childdocforward{|\textit{main}|}|
\end{tabular}
\end{center}
%
Likewise, the following files |final|\textit{nn}|.tex|
compile the final version of the child document
|child|\textit{nn}|.tex|:
%
\begin{center}
\begin{tabular}{l}
|\def\version{final}|\\
|\input{childdoc.def}|\\
|\childdocforwardprefix{final}{child}|
\end{tabular}
\end{center}
%

Note that when several versions of a main file and/or of each child file
are to be generated, it may be convenient to set up a |Makefile| or
shell script to automatise the process.

%%%%%%%%%%%%%%%%%%%%%%%%%%%%%%%%%%%%%%%%%%%%%%%%%%%%%%%%%%%%%%%%%%%%%%%%%%%%%%%%
\subsection{Command Line Processing}
\label{sec:commandline}

The effect of redirection files can also be achieved by invoking
the \LaTeX{} compiler with a more elaborate command line.
Most conveniently this should be done as part
of a shell script or a |Makefile|.

When using \textsf{childdoc} in the main file, the following
command lines effectively perform a redirection
(note that depending on the shell being used,
backslashes may have to be doubled: `|\|' $\to$ `|\\|'):
%
\begin{center}
|... -jobname "|\textit{target}|" |\\|"|[\textit{flags}]%
|\input{childdoc.def}\childdocforward[|\textit{main}|]{|\textit{dest}|}"|
\end{center}
%
Here \textit{target} is the name of the output file,
\textit{main} is the name of the main file
and \textit{dest} is the name of the main or child file to be processed
(all filenames without extensions).
The optional argument \textit{main} can be omitted
if \textit{main} matches \textit{dest}.
Optionally, compilation \textit{flags} can be defined via |\def| commands.
This command line makes the \TeX{} engine believe
it is compiling the file \textit{target}
whose content is specified as the latter parameter.
The provided code then forwards the processing to
\textit{main} or \textit{dest} as described in \secref{sec:forward}.

%%%%%%%%%%%%%%%%%%%%%%%%%%%%%%%%%%%%%%%%%%%%%%%%%%%%%%%%%%%%%%%%%%%%%%%%%%%%%%%%
\subsection{Include by Input}
\label{sec:input}

Including child documents by |\include| has some restrictions by design.
Most notably, the content of a child document always occupies
its own set of pages; pages cannot be shared between child documents.
Usually, this behaviour makes perfect sense
because each child document contain an essential part of the document.
However, in some situations it may be desirable to compose
a document from a collection of parts
without having mandatory page breaks between then.
For this case, the package
provides a mechanism to include parts
by |\input| which can also be processed individually.
However, by construction this mechanism
requires manual handling of the content to be output.

%%%%%%%%%%%%%%%%%%%%%%%%%%%%%%%%%%%%%%%%
\DescribeMacro{\ifchilddocmanual}
The main file should be prepared as usual, see \secref{sec:include}.
However, the document body must make a distinction
between processing of an individual part and of the main document, e.g.:
%
\begin{center}
\begin{tabular}{l}
|\ifchilddocmanual|\\
|\input{\childdocname}|\\
|\||else|\\
\textit{document body with }|\input{|\textit{part}|}|\\
|\||fi|
\end{tabular}
\end{center}
%
The conditional |\ifchilddocmanual| is true whenever
a part to be included by |\input| is being compiled,
and the name of the part is stored in |\childdocname|.

%%%%%%%%%%%%%%%%%%%%%%%%%%%%%%%%%%%%%%%%
\DescribeMacro{\childdocby}
Each part to be included by |\input| should start with:
%
\begin{center}
\begin{tabular}{l}
|\input{childdoc.def}|\\
|\childdocby{|\textit{main}|}|\\
\end{tabular}
\end{center}
%
The directive |\childdocby| is similar to |\childdocof|
described in \secref{sec:include},
but the subsequent selection of content must be done manually.
To that end, both |\ifchilddoc| and |\ifchilddocmanual|
will be true upon processing of a part,
and the name of the part is stored in |\childdocname|.
Note that |\jobname| will be set to the filename of the current part
so that each part receives an individual |.aux| file
that does not interfere with the |.aux| file(s) of the main document.
This behaviour can be altered by the alternative form
|\childdocby[*]{|\textit{main}|}| (with a non-empty optional argument)
which uses the |.aux| file of the main document
by setting |\jobname| to \textit{main}.

%%%%%%%%%%%%%%%%%%%%%%%%%%%%%%%%%%%%%%%%%%%%%%%%%%%%%%%%%%%%%%%%%%%%%%%%%%%%%%%%
\subsection{Driver Development}
\label{sec:driver}

The \textsf{childdoc} mechanism can also be use for the development
of definition files such as \LaTeX{} styles or classes.
This case differs from the above setup with multiple parts
included by |\include| in that no |\includeonly| should be invoked.
This can be achieved by starting the include file
(before |\ProvidesPackage|) with:
%
\begin{center}
\begin{tabular}{l}
|\input{childdoc.def}|\\
|\childdocforward{|\textit{main}|}|\\
\end{tabular}
\end{center}
%
or alternatively with:
%
\begin{center}
\begin{tabular}{l}
|\input{childdoc.def}|\\
|\childdocby{|\textit{main}|}|\\
\end{tabular}
\end{center}
%
Both forms have slightly different effects as described above.
The main file is prepared as usual, see \secref{sec:include}.

%%%%%%%%%%%%%%%%%%%%%%%%%%%%%%%%%%%%%%%%%%%%%%%%%%%%%%%%%%%%%%%%%%%%%%%%%%%%%%%%
\subsection{Legacy Detection}
\label{sec:detection}

The directive |\childdocmain| in the main file can detect
whether the complete document or merely a child is to be compiled
even without using the directive |\childdocof|.
This method is deprecated because it is less robust
and there is no compelling reason to use it;
it is merely provided for backward compatibility
and it may be removed in future versions.

If the detection mechanism is to be used,
it is mandatory to correctly specify
the filename of the main file as the argument of |\childdocmain|:
%
\begin{center}
\begin{tabular}{l}
|\input{childdoc.def}|\\
|\childdocmain{|\textit{main}|}|\\
\end{tabular}
\end{center}
%
If |\jobname| does not match the argument \textit{main} of |\childdocmain|,
it is assumed that |\jobname| points to the child file to be compiled.
When using |\childdocmain| with the main file specified as argument,
it suffices to start a child file
with just |\input{|\textit{main}|}|
without loading of the package and using |\childdocof|.
If instead all processing is done
with the appropriate \textsf{childdoc} directives,
the argument of \textit{main} of |\childdocmain| can be empty.

An alternative version of the command line processing described
in \secref{sec:commandline} using the detection mechanism reads:
%
\begin{center}
|... -jobname "|\textit{target}|" "|[\textit{flags}]%
[|\def\jobname{|\textit{dest}|}|]|\input{|\textit{main}|}"|
\end{center}

%%%%%%%%%%%%%%%%%%%%%%%%%%%%%%%%%%%%%%%%%%%%%%%%%%%%%%%%%%%%%%%%%%%%%%%%%%%%%%%%
\subsection{Manual Code}
\label{sec:manual}

In case one cannot be certain whether the definitions file |childdoc.def|
is installed on the target \TeX{} distribution
and one prefers not to ship it,
it is conceivable to paste a few relevant commands into the sources.

To that end, drop all statements |\input{childdoc.def}|
and perform the replacements as outlined below.
Instead of |\childdocmain{|\textit{main}|}| add the following code
to the top of the main file:
%
\begin{center}
\begin{tabular}{l}
|\||ifdefined\childdocname\endinput\||fi\newif\ifchilddoc|\\
|\edef\childdocname{\scantokens\expandafter{\jobname\noexpand}}|\\
|\def\childdocmain{|\textit{main}|}\||ifx\childdocmain\childdocname\||else|\\
|\childdoctrue\includeonly{\childdocname}\let\jobname\childdocmain\||fi|\\
\end{tabular}
\end{center}
%
Instead of |\childdocof{|\textit{main}|}| just include the main file
at the top of each child file:
%
\begin{center}
|\input{|\textit{main}|}|
\end{center}
%
A simple redirection |\childdocforward{|\textit{dest}|}| is achieved by:
%
\begin{center}
|\def\jobname{|\textit{dest}|}\input{\jobname}|
\end{center}
%
The redirection with prefix
|\childdocforwardprefix[|\textit{prefix}|]{|\textit{dest}|}|
is accomplished by:
%
\begin{center}
\begin{tabular}{l}
|{\edef\jobname{\scantokens\expandafter{\jobname\noexpand}}|\\
|\def\redirectjob |\textit{prefix}|#1~~~{\gdef\jobname{|\textit{dest}|#1}}|\\
|\expandafter\redirectjob\jobname~~~}\input{\jobname}|
\end{tabular}
\end{center}

In an alternative approach,
child documents can be compiled by a specific command line
without additional code or specific definitions:
%
\begin{center}
|... -jobname "|\textit{target}|" "|[\textit{flags}]%
|\includeonly{|\textit{dest}|}\input{|\textit{main}|}"|
\end{center}
%

%%%%%%%%%%%%%%%%%%%%%%%%%%%%%%%%%%%%%%%%%%%%%%%%%%%%%%%%%%%%%%%%%%%%%%%%%%%%%%%%
%%%%%%%%%%%%%%%%%%%%%%%%%%%%%%%%%%%%%%%%%%%%%%%%%%%%%%%%%%%%%%%%%%%%%%%%%%%%%%%%
\section{Information}

%%%%%%%%%%%%%%%%%%%%%%%%%%%%%%%%%%%%%%%%%%%%%%%%%%%%%%%%%%%%%%%%%%%%%%%%%%%%%%%%
\subsection{Copyright}

Copyright \copyright{} 2017--2018 Niklas Beisert

This work may be distributed and/or modified under the
conditions of the \LaTeX{} Project Public License, either version 1.3
of this license or (at your option) any later version.
The latest version of this license is in
  \url{http://www.latex-project.org/lppl.txt}
and version 1.3 or later is part of all distributions of \LaTeX{}
version 2005/12/01 or later.

This work has the LPPL maintenance status `maintained'.

The Current Maintainer of this work is Niklas Beisert.

This work consists of the files |README.txt|, |childdoc.ins| and |childdoc.dtx|
as well as the derived files |childdoc.def|, |cdocsamp.tex|
with |cdocsch1.tex|, |cdocsch2.tex|, |cdocspt3.tex|, |cdocspt4.tex|,
|cdocsdrf.tex|, |cdocsfn1.tex|, |cdocsfn2.tex|
as well as |childdoc.pdf|.

%%%%%%%%%%%%%%%%%%%%%%%%%%%%%%%%%%%%%%%%%%%%%%%%%%%%%%%%%%%%%%%%%%%%%%%%%%%%%%%%
\subsection{Files and Installation}

The package consists of the files:
%
\begin{center}
\begin{tabular}{ll}
    |README.txt|   & readme file \\
    |childdoc.ins| & installation file \\
    |childdoc.dtx| & source file \\
    |childdoc.def| & definition file \\
    |cdocsamp.tex| & sample main file \\
    |cdocsch1.tex| & sample include file \\
    |cdocsch2.tex| & sample include file \\
    |cdocspt3.tex| & sample part file \\
    |cdocspt4.tex| & sample part file \\
    |cdocsdrf.tex| & sample redirection file \\
    |cdocsfn1.tex| & sample redirection file \\
    |cdocsfn2.tex| & sample redirection file \\
    |childdoc.pdf| & manual
\end{tabular}
\end{center}
%
The distribution consists of the files
|README.txt|, |childdoc.ins| and |childdoc.dtx|.
%
\begin{itemize}
\item
Run (pdf)\LaTeX{} on |childdoc.dtx|
to compile the manual |childdoc.pdf| (this file).
\item
Run \LaTeX{} on |childdoc.ins| to create the definitions file |childdoc.def|
and the sample |cdocsamp.tex| with include files
|cdocsch1.tex|, |cdocsch2.tex|, |cdocspt3.tex|, |cdocspt4.tex|,
|cdocsdrf.tex|, |cdocsfn1.tex|, |cdocsfn2.tex|.
Then copy the file |childdoc.def| to an appropriate directory of your \LaTeX{}
distribution, e.g.\ \textit{texmf-root}|/tex/latex/childdoc|.
\end{itemize}

%%%%%%%%%%%%%%%%%%%%%%%%%%%%%%%%%%%%%%%%%%%%%%%%%%%%%%%%%%%%%%%%%%%%%%%%%%%%%%%%
\subsection{Related CTAN Packages}

There are several other packages which offer a similar functionality:
%
\begin{itemize}
\item
The packages
\href{http://ctan.org/pkg/docmute}{\textsf{docmute}},
\href{http://ctan.org/pkg/includex}{\textsf{includex}} and
\href{http://ctan.org/pkg/standalone}{\textsf{standalone}}
provide commands to include only the document body of
a child file thus allowing both files to be compiled individually.
\item
The packages \href{http://ctan.org/pkg/subdocs}{\textsf{subdocs}}
and \href{http://ctan.org/pkg/subfiles}{\textsf{subfiles}}
provide structures in which the main and child documents can be
encapsulated and allowing them to be compiled individually.
The inclusion mechanism is different from the conventional |\include|.
\item
The package \href{http://ctan.org/pkg/combine}{\textsf{combine}}
is an elaborate solution to combine several documents into one.
\end{itemize}
%
See also the CTAN topic \href{http://ctan.org/topic/subdocs}{\textsf{subdocs}}
for further related packages.
The present package differs from the above solutions in that
a document structure constructed with the conventional |\include| mechanism
just needs two extra commands at the top of every file
such that all constituent files can be compiled individually.

%%%%%%%%%%%%%%%%%%%%%%%%%%%%%%%%%%%%%%%%%%%%%%%%%%%%%%%%%%%%%%%%%%%%%%%%%%%%%%%%
%\subsection{Feature Suggestions}
%
%The following is a list of features which may be useful for future
%versions of this package:
%%
%\begin{itemize}
%\item
%\ldots
%\end{itemize}

%%%%%%%%%%%%%%%%%%%%%%%%%%%%%%%%%%%%%%%%%%%%%%%%%%%%%%%%%%%%%%%%%%%%%%%%%%%%%%%%
\subsection{Revision History}

%%%%%%%%%%%%%%%%%%%%%%%%%%%%%%%%%%%%%%%%
\paragraph{v2.0:} 2018/12/30

\begin{itemize}
\item
immediate forward processing
\item
added |\childdocby| mechanism
\item
manual restructured
\end{itemize}

%%%%%%%%%%%%%%%%%%%%%%%%%%%%%%%%%%%%%%%%
\paragraph{v1.6:} 2018/01/17

\begin{itemize}
\item
application for development of include files
\item
corrections to manual
\end{itemize}

%%%%%%%%%%%%%%%%%%%%%%%%%%%%%%%%%%%%%%%%
\paragraph{v1.5:} 2017/05/21

\begin{itemize}
\item
more complete structuring introduced
\item
|\childdocof| introduced
\item
|\childdoc| renamed to |\childdocmain|
\item
|\childredirect| renamed to |\childdocforward| and |\childdocforwardprefix|
and functionality expanded
\end{itemize}

%%%%%%%%%%%%%%%%%%%%%%%%%%%%%%%%%%%%%%%%
\paragraph{v1.0:} 2017/04/27

\begin{itemize}
\item
manual and install package
\item
first version published on CTAN
\end{itemize}

%%%%%%%%%%%%%%%%%%%%%%%%%%%%%%%%%%%%%%%%
\paragraph{v0.6:} 2017/04/26

\begin{itemize}
\item
redirection mechanism added
\end{itemize}

%%%%%%%%%%%%%%%%%%%%%%%%%%%%%%%%%%%%%%%%
\paragraph{v0.5:} 2017/04/26

\begin{itemize}
\item
functionality in definition file
\end{itemize}


%%%%%%%%%%%%%%%%%%%%%%%%%%%%%%%%%%%%%%%%%%%%%%%%%%%%%%%%%%%%%%%%%%%%%%%%%%%%%%%%
%%%%%%%%%%%%%%%%%%%%%%%%%%%%%%%%%%%%%%%%%%%%%%%%%%%%%%%%%%%%%%%%%%%%%%%%%%%%%%%%
%%%%%%%%%%%%%%%%%%%%%%%%%%%%%%%%%%%%%%%%%%%%%%%%%%%%%%%%%%%%%%%%%%%%%%%%%%%%%%%%
\appendix

\settowidth\MacroIndent{\rmfamily\scriptsize 000\ }

 \DocInput{childdoc.dtx}

\end{document}
%</driver>
% \fi
%
% %%%%%%%%%%%%%%%%%%%%%%%%%%%%%%%%%%%%%%%%%%%%%%%%%%%%%%%%%%%%%%%%%%%%%%%%%%%%%%
% %%%%%%%%%%%%%%%%%%%%%%%%%%%%%%%%%%%%%%%%%%%%%%%%%%%%%%%%%%%%%%%%%%%%%%%%%%%%%%
% \section{Sample}
%\iffalse
%<*samplemain>
%\fi
%
% The following presents a sample document
% with two chapters, two parts, a title page,
% a compile flag as well as three forwarding files to set the flag.
% It consists of eight |.tex| files:
% \begin{center}
% \begin{tabular}{ll}
% |cdocsamp.tex|&main file\\
% |cdocsch1.tex|&include file for chapter 1\\
% |cdocsch2.tex|&include file for chapter 2\\
% |cdocspt3.tex|&include file for part 3\\
% |cdocspt4.tex|&include file for part 4\\
% |cdocsdrf.tex|&forwarding file for main file in draft mode\\
% |cdocsfi1.tex|&forwarding file for final version of chapter 1\\
% |cdocsfi2.tex|&forwarding file for final version of chapter 2\\
% \end{tabular}
% \end{center}
% Each of the eight files can be compiled directly by the \LaTeX{} compiler.
%
% %%%%%%%%%%%%%%%%%%%%%%%%%%%%%%%%%%%%%%
% \paragraph{Main File.}
%
% The main file is called |cdocsamp.tex|.
%
% Load the \textsf{childdoc} definitions and
% declare the filename for the main document:
%    \begin{macrocode}
\input{childdoc.def}
\childdocmain{}
%    \end{macrocode}

% Optional override for |\version| flag:
%    \begin{macrocode}
%%\ifchilddoc\else\providecommand{\version}{draft}\fi
%    \end{macrocode}

% Define the default values for the |\version| flag
% (|final| for the main file and |draft| for childs):
%    \begin{macrocode}
\ifchilddoc
\providecommand{\version}{draft}
\else
\providecommand{\version}{final}
\fi
%    \end{macrocode}

% Load the standard document class:
%    \begin{macrocode}
\documentclass[12pt]{article}
%    \end{macrocode}

% Start the document body:
%    \begin{macrocode}
\begin{document}
%    \end{macrocode}

% Declare a title page.
% Print title, part of document being processed and version flag:
%    \begin{macrocode}
\addtocounter{page}{-1}
\begin{center}
{\LARGE\bfseries{}childdoc example\par}
\vspace{1cm}
\ifchilddoc
\ifchilddocmanual part\else chapter\fi:
`\childdocname' of `\childdocjob'\par
\else
main document: `\childdocjob'\par
\fi
version: \version\par
\end{center}
\newpage
%    \end{macrocode}

% Manually include selected file,
% otherwise process as usual:
%    \begin{macrocode}
\ifchilddocmanual
\section*{part `\childdocname'}
\input{\childdocname}
\else
%    \end{macrocode}

% Include the two chapters:
%    \begin{macrocode}
\include{cdocsch1}
\include{cdocsch2}
%    \end{macrocode}

% Include the two parts unless only chapters should be displayed:
%    \begin{macrocode}
\ifchilddoc\else
\section{part three}
\input{cdocspt3}
\section{part four}
\input{cdocspt4}
\fi
%    \end{macrocode}

% Process as usual until here:
%    \begin{macrocode}
\fi
%    \end{macrocode}

% End of document body:
%    \begin{macrocode}
\end{document}
%    \end{macrocode}
%\iffalse
%</samplemain>
%\fi
%
% %%%%%%%%%%%%%%%%%%%%%%%%%%%%%%%%%%%%%%
% \paragraph{Chapter Include Files.}
%
% The include files are called |cdocsch1.tex| and |cdocsch2.tex|.
%
%\iffalse
%<*samplechap1|samplechap2>
%\fi

% Optional override for |\version| flag:
%    \begin{macrocode}
%%\providecommand{\version}{final}
%    \end{macrocode}

% Include the main document:
%    \begin{macrocode}
\input{childdoc.def}
\childdocof{cdocsamp}
%    \end{macrocode}

%\iffalse
%</samplechap1|samplechap2>
%\fi
%
%\iffalse
%<*samplechap1>
%\fi
% Some text for chapter 1:
%    \begin{macrocode}
\section{one}
some text in chapter one
%    \end{macrocode}

%\iffalse
%</samplechap1>
%\fi
% Some text for chapter 2:
%\iffalse
%<*samplechap2>
%\fi
%    \begin{macrocode}
\section{two}
more text in chapter two
%    \end{macrocode}

%\iffalse
%</samplechap2>
%\fi
%
% %%%%%%%%%%%%%%%%%%%%%%%%%%%%%%%%%%%%%%
% \paragraph{Part Include Files.}
%
% The include files are called |cdocspt3.tex| and |cdocspt4.tex|.
%
%\iffalse
%<*samplepart3|samplepart4>
%\fi

% Optional override for |\version| flag:
%    \begin{macrocode}
%%\providecommand{\version}{final}
%    \end{macrocode}

% Include the main document:
%    \begin{macrocode}
\input{childdoc.def}
\childdocby{cdocsamp}
%    \end{macrocode}

%\iffalse
%</samplepart3|samplepart4>
%\fi
%
%\iffalse
%<*samplepart3>
%\fi
% Some text for part 3:
%    \begin{macrocode}
some text in part three
%    \end{macrocode}

%\iffalse
%</samplepart3>
%\fi
% Some text for part 4:
%\iffalse
%<*samplepart4>
%\fi
%    \begin{macrocode}
more text in part four
%    \end{macrocode}

%\iffalse
%</samplepart4>
%\fi
%
% %%%%%%%%%%%%%%%%%%%%%%%%%%%%%%%%%%%%%%
% \paragraph{Forwarding for a Complete Draft.}
%
% The following forwarding file |cdocsdrf.tex|
% compiles the main document in draft mode:
%\iffalse
%<*sampledraft>
%\fi
%    \begin{macrocode}
\def\version{draft}
\input{childdoc.def}
\childdocforward{cdocsamp}
%    \end{macrocode}

%\iffalse
%</sampledraft>
%\fi
%
% %%%%%%%%%%%%%%%%%%%%%%%%%%%%%%%%%%%%%%
% \paragraph{Forwarding for Final Version of the Chapters.}
%
% The following forwarding files |cdocsfn1.tex| and |cdocsfn2.tex|
% (with identical content)
% compile the final versions of the child documents
% |cdocsch1.tex| and |cdocsch2.tex|, respectively:
%\iffalse
%<*samplefinal>
%\fi
%    \begin{macrocode}
\def\version{final}
\input{childdoc.def}
\childdocforwardprefix[cdocsamp]{cdocsfn}{cdocsch}
%    \end{macrocode}

%\iffalse
%</samplefinal>
%\fi
%
% %%%%%%%%%%%%%%%%%%%%%%%%%%%%%%%%%%%%%%
% \paragraph{Command Line Processing.}
%
% The following three command lines generate the output files
% |cdocscld|, |cdocscl1| and |cdocscl2|
% which should be identical to
% |cdocsdrf|, |cdocsch1| and |cdocsfn2|, respectively:
% \begin{center}
% \begin{tabular}{l}
% |latex -jobname cdocscld \|\\
% |  "\def\version{draft}\input{childdoc.def}\childdocforward{cdocsamp}"|\\
% |latex -jobname cdocscl1 \|\\
% |  "\input{childdoc.def}\childdocforward[cdocsamp]{cdocsch1}"|\\
% |latex -jobname cdocscl2 \|\\
% |  "\def\version{final}\input{childdoc.def}\childdocforward{cdocsch2}"|
% \end{tabular}
% \end{center}
% Note that the trailing backslash on each first line
% merely continues the input to the second line
% (for convenient cut ant paste).
% Furthermore, the command |latex| can be replaced by any
% of its alternative versions such as |pdflatex|.
%
% %%%%%%%%%%%%%%%%%%%%%%%%%%%%%%%%%%%%%%%%%%%%%%%%%%%%%%%%%%%%%%%%%%%%%%%%%%%%%%
% %%%%%%%%%%%%%%%%%%%%%%%%%%%%%%%%%%%%%%%%%%%%%%%%%%%%%%%%%%%%%%%%%%%%%%%%%%%%%%
% \section{Implementation}
%\iffalse
%<*package>
%\fi
%
% This section describes the definitions file |childdoc.def|.

% The definitions cannot be loaded using |\usepackage| or |\RequirePackage|
% which has a mechanism to prevent loading a style file more than once.
% When loading the definitions by means of |\input|
% multiple instances have to be prevented manually:
%\iffalse
%This code needs to be before the `\ProvidesFile' directive
%which is defined at the beginning of this file.
%Therefore it is also placed there and commented out here.
%</package>
%<*discard>
%\fi
%    \begin{macrocode}
\ifdefined\childdocmain\endinput\fi
%    \end{macrocode}
%\iffalse
%</discard>
%<*package>
%\fi
%
% \macro{\ifchilddoc}
% \macro{\ifchilddocmanual}
% The conditional |\ifchilddoc| tells whether a
% child (true) or main (false) document is being compiled.
% The conditional |\ifchilddocmanual| tells whether
% the |\includeonly| mechanism is used (false) or
% the selection of child files must be performed manually (true).
% The definitions initialise to false:
%    \begin{macrocode}
\newif\ifchilddoc
\newif\ifchilddocmanual
%    \end{macrocode}

% \macro{\childdocname}
% \macro{\childdocjob}
% The macro |\childdocname| stores the name of the main document
% to be compiled. The macro |\childdocjob| stores the name of
% the document on which the \LaTeX{} compiler was originally invoked.
% The content of |\jobname| cannot be compared
% to filenames specified in the source due to different catcodes.
% The following code rescans |\jobname|, stores the result
% in |\childdocname| and saves a copy in |\childdocjob|:
%    \begin{macrocode}
\edef\childdocname{\scantokens\expandafter{\jobname\noexpand}}
\let\childdocjob\childdocname
%    \end{macrocode}

% \macro{\childdocdisable}
% The macro |\childdocdisable| prevents the main file
% from being processed more than once.
% At this stage, the main document command |\childdocmain|
% is assumed to be called once again where it should do nothing.
% Any subsequent call to it should prevent
% a secondary processing of the main document
% It overwrites the forwarding commands
% |\childdocof| and |\childdocforward|
% with empty macros to prevent further inclusions of the main document:
%    \begin{macrocode}
\newcommand{\childdocdisable}
{
  \renewcommand{\childdocmain}[1]{\renewcommand{\childdocmain}[1]{\endinput}}
  \renewcommand{\childdocof}[1]{}
  \renewcommand{\childdocby}[2][]{}
  \renewcommand{\childdocforward}[2][]{}
  \renewcommand{\childdocdisable}{}
}
%    \end{macrocode}

% \macro{\childdocmain}
% The macro |\childdocmain| is to be called at the top of the main file
% with nothing or the main filename (without extension) as argument.
% First, it breaks loops.
% If the argument is not empty and does not match |\childdocname|
% (which is set by the first inclusion of |childdoc.def|),
% |\ifchilddoc| is set to true, |\includeonly| is applied to the child file
% and |\jobname| is set to the main file
% (for proper handling of |.aux| files):
%    \begin{macrocode}
\newcommand{\childdocmain}[1]
{
  \childdocdisable\childdocmain{}
  \if?#1?\else
    \begingroup
      \def\childdoctmp{#1}
      \ifx\childdoctmp\childdocname
        \def\childdoctmp{}
      \else
        \def\childdoctmp
        {
          \childdoctrue
          \includeonly{\childdocname}
          \def\childdocjob{#1}
          \def\jobname{#1}
        }
      \fi
      \expandafter
    \endgroup
    \childdoctmp
  \fi
}
%    \end{macrocode}

% \macro{\childdocof}
% The command |\childdocof| redirects
% compilation to the main file |#1|.
%    \begin{macrocode}
\newcommand{\childdocof}[1]
{
  \childdocdisable
  \childdoctrue
  \includeonly{\childdocname}
  \def\jobname{#1}
  \def\childdocjob{#1}
  \input{#1}
}
%    \end{macrocode}

% \macro{\childdocby}
% The command |\childdocby| ....
%    \begin{macrocode}
\newcommand{\childdocby}[2][]
{
  \childdocdisable
  \childdoctrue
  \childdocmanualtrue
  \if?#1?\else
    \def\jobname{#2}
  \fi
  \def\childdocjob{#2}
  \input{#2}
  \endinput
}
%    \end{macrocode}

% \macro{\childdocforward}
% The command |\childdocforward| redirects
% compilation to the main file or
% (if the optional argument is given) a child file.
% Parameters are set as if the main file
% or a child file starting with |\childdocof| was compiled.
% Then compilation is handed over to the main file:
%    \begin{macrocode}
\newcommand{\childdocforward}[2][]
{
  \begingroup
    \if?#1?
      \def\childdoctmp
      {
        \def\childdocname{#2}
        \def\childdocjob{#2}
        \def\jobname{#2}
        \input{#2}
        \endinput
      }
    \else
      \def\childdoctmp
      {
        \childdocdisable
        \def\childdocname{#2}
        \childdoctrue
        \includeonly{#2}
        \def\childdocjob{#1}
        \def\jobname{#1}
        \input{#1}
        \endinput
      }
    \fi
    \expandafter
  \endgroup
  \childdoctmp
}
%    \end{macrocode}

% \macro{\childdocforwardprefix}
% The command |\childdocforwardprefix| redirects
% compilation to the main or a child file by means of a pattern.
% The prefix |#1| in the current filename is replaced by |#2|
% and the suffix of the current filename is kept
% (it is assumed that the filename does not contain the substring `|~~~|'
% which is used as a delimiter).
% Compilation is handed over to the new file by |\childdocforward|:
%    \begin{macrocode}
\newcommand{\childdocforwardprefix}[3][]
{
  \begingroup
    \def\childdocextract #2##1~~~{\def\childdoctmp{\childdocforward[#1]{#3##1}}}
    \expandafter\childdocextract\childdocname~~~
    \expandafter
  \endgroup
  \childdoctmp
}
%    \end{macrocode}

% \macro{\childdoc}
% The deprecated macro |\childdoc| is a legacy version of |\childdocmain|:
%    \begin{macrocode}
\newcommand{\childdoc}{\childdocmain}
%    \end{macrocode}

% \macro{\childdocredirect}
% The deprecated macro |\childdocredirect| is a legacy version
% of |\childdocforward| and |\childdocforwardprefix|:
%    \begin{macrocode}
\newcommand{\childdocredirect}[2][]
{
  \begingroup
    \if?#1?
      \def\childdoctmp{\childdocforward{#2}}
    \else
      \def\childdoctmp{\childdocforwardprefix{#1}{#2}}
    \fi
    \expandafter
  \endgroup
  \childdoctmp
}
%    \end{macrocode}

%\iffalse
%</package>
%\fi
%
\endinput
|\\
|\childdocforward{|\textit{main}|}|
\end{tabular}
\end{center}
%
Likewise, the following files |final|\textit{nn}|.tex|
compile the final version of the child document
|child|\textit{nn}|.tex|:
%
\begin{center}
\begin{tabular}{l}
|\def\version{final}|\\
|% \iffalse
%
% childdoc.dtx Copyright (C) 2017-2018 Niklas Beisert
%
% This work may be distributed and/or modified under the
% conditions of the LaTeX Project Public License, either version 1.3
% of this license or (at your option) any later version.
% The latest version of this license is in
%   http://www.latex-project.org/lppl.txt
% and version 1.3 or later is part of all distributions of LaTeX
% version 2005/12/01 or later.
%
% This work has the LPPL maintenance status `maintained'.
%
% The Current Maintainer of this work is Niklas Beisert.
%
% This work consists of the files childdoc.dtx and childdoc.ins
% and the derived files childdoc.def and cdocsamp.tex with
% cdocsch1.tex, cdocsch2.tex, cdocsdrf.tex, cdocsfn1.tex, cdocsfn2.tex.
%
%<package>\ifdefined\childdocmain\endinput\fi
%<package>\ProvidesFile{childdoc.def}[2018/12/30 v2.0 child document driver]
%<samplemain>\ProvidesFile{cdocsamp.tex}[2018/12/30 v2.0 sample for childdoc]
%<*driver>
%\ProvidesFile{childdoc.drv}[2018/12/30 v2.0 childdoc reference manual file]
\PassOptionsToClass{10pt,a4paper}{article}
\documentclass{ltxdoc}

\usepackage[margin=35mm]{geometry}
\usepackage{hyperref}
\usepackage{hyperxmp}
\usepackage[usenames]{color}

\hypersetup{colorlinks=true}
\hypersetup{pdfstartview=FitH}
\hypersetup{pdfpagemode=UseNone}
\hypersetup{pdfsource={}}
\hypersetup{pdflang={en-UK}}
\hypersetup{pdfcopyright={Copyright 2017-2018 Niklas Beisert.
  This work may be distributed and/or modified under the
  conditions of the LaTeX Project Public License, either version 1.3
  of this license or (at your option) any later version.}}
\hypersetup{pdflicenseurl={http://www.latex-project.org/lppl.txt}}
\hypersetup{pdfcontactaddress={ETH Zurich, ITP, HIT K,
  Wolfgang-Pauli-Strasse 27}}
\hypersetup{pdfcontactpostcode={8093}}
\hypersetup{pdfcontactcity={Zurich}}
\hypersetup{pdfcontactcountry={Switzerland}}
\hypersetup{pdfcontactemail={nbeisert@itp.phys.ethz.ch}}
\hypersetup{pdfcontacturl={http://people.phys.ethz.ch/\xmptilde nbeisert/}}

\newcommand{\secref}[1]{\hyperref[#1]{section \ref*{#1}}}

\parskip1ex
\parindent0pt
\let\olditemize\itemize
\def\itemize{\olditemize\parskip0pt}

\begin{document}

\title{The \textsf{childdoc} Package}
\hypersetup{pdftitle={The childdoc Package}}
\author{Niklas Beisert\\[2ex]
  Institut f\"ur Theoretische Physik\\
  Eidgen\"ossische Technische Hochschule Z\"urich\\
  Wolfgang-Pauli-Strasse 27, 8093 Z\"urich, Switzerland\\[1ex]
  \href{mailto:nbeisert@itp.phys.ethz.ch}
  {\texttt{nbeisert@itp.phys.ethz.ch}}}
\hypersetup{pdfauthor={Niklas Beisert}}
\hypersetup{pdfsubject={Manual for the LaTeX2e Package childdoc}}
\date{30 December 2018, \textsf{v2.0}}
\maketitle

\begin{abstract}\noindent
\textsf{childdoc} is a \LaTeXe{} package
that enables the direct compilation
of document sections included by |\include|
to individual files.
\end{abstract}

\begingroup
\parskip0ex
\tableofcontents
\endgroup

%%%%%%%%%%%%%%%%%%%%%%%%%%%%%%%%%%%%%%%%%%%%%%%%%%%%%%%%%%%%%%%%%%%%%%%%%%%%%%%%
%%%%%%%%%%%%%%%%%%%%%%%%%%%%%%%%%%%%%%%%%%%%%%%%%%%%%%%%%%%%%%%%%%%%%%%%%%%%%%%%
\section{Introduction}

\LaTeX{} provides a mechanism to structure a large document (such as a book)
into a main file and several child files (containing the chapters)
using the |\include| command.
This mechanism is beneficial for documents
which span hundreds of pages in order to
make the source file(s) more manageable.
Moreover, compilation can be restricted to
selected child files by means of the |\includeonly| command.
The latter feature can be used to reduce the compilation time while editing
(this was significantly more useful in the earlier days of \LaTeX{})
or to generate a smaller document which is easier to navigate.
Another application of |\includeonly| is to generate
documents consisting of selected parts of the complete document.

However, there are a few drawbacks of the plain |\include| mechanism:
\begin{itemize}
\item
The child files cannot be compiled on their own,
they can only be compiled via the main file.
A naive editing environment
(such as a text editor with an option
to have the current file processed by \LaTeX)
may require one to switch to the main file before compiling;
attempting to compile the child file produces errors.
\item
The main file must be modified (each time)
to adjust the |\includeonly| command
to the present needs. This easily leaves the main file in a messy state.
\item
The generated document will always carry the filename
of the main document. This is inconvenient if
several child files are to be compiled and
to be kept for distribution.
\end{itemize}

The present package provides a simple interface
to make child files individually compilable by \LaTeX{}.
Compiling a child file then has the same effect as compiling
the main file with an |\includeonly| command
to select the appropriate child.
Moreover the generated document will carry the name of the child
rather than the main file.
This resolves all three above issues.

This feature is meant to make the editing of books,
thesis documents and lecture notes somewhat more convenient.
However, the package can also be used efficiently for
composing a series of documents (such as exercise sheets)
which are typically distributed individually.
It then assists the author in generating the individual documents
(potentially in different versions)
as well as a document containing the collected series.
Another application is in developing style files
or other kinds of included material
where compilation of the style file could redirect
to a sample or test file.

%%%%%%%%%%%%%%%%%%%%%%%%%%%%%%%%%%%%%%%%%%%%%%%%%%%%%%%%%%%%%%%%%%%%%%%%%%%%%%%%
%%%%%%%%%%%%%%%%%%%%%%%%%%%%%%%%%%%%%%%%%%%%%%%%%%%%%%%%%%%%%%%%%%%%%%%%%%%%%%%%
\section{Usage}

First of all, the package \textsf{childdoc} is \emph{not} a standard
\LaTeXe{} |.sty| style file! Therefore it needs to be invoked in
a non-standard way.

%%%%%%%%%%%%%%%%%%%%%%%%%%%%%%%%%%%%%%%%%%%%%%%%%%%%%%%%%%%%%%%%%%%%%%%%%%%%%%%%
\subsection{Included Files}
\label{sec:include}

%%%%%%%%%%%%%%%%%%%%%%%%%%%%%%%%%%%%%%%%
\DescribeMacro{\childdocmain}
To use the package, add the commands
\begin{center}
\begin{tabular}{l}
|\input{childdoc.def}|\\
|\childdocmain{}|\\
\end{tabular}
\end{center}
at the very top of the main \LaTeX{} file,
in particular \emph{before} the |\documentclass| statement!
The argument of |\childdocmain| should be left empty
(but it must be present).

%%%%%%%%%%%%%%%%%%%%%%%%%%%%%%%%%%%%%%%%
\DescribeMacro{\childdocof}
Furthermore, add the commands
\begin{center}
\begin{tabular}{l}
|\input{childdoc.def}|\\
|\childdocof{|\textit{main}|}|\\
\end{tabular}
\end{center}
at the top of every child file \textit{child}
which is included by |\include{|\textit{child}|}|
from within the main file
(or at least for those files to be compiled individually).
The argument \textit{main} must be the filename of the main file.

There are a couple of
considerations in setting up the main and child documents:

%%%%%%%%%%%%%%%%%%%%%%%%%%%%%%%%%%%%%%%%
\paragraph{Restrictions.}

Please note the following restrictions:
\begin{itemize}
\item
|\childdocmain| must be called with one argument \textit{main}
to ensure compatibility with earlier version of the package.
It must either be empty (|\childdocmain{}|)
or precisely match the filename of the main file in which it is specified.
See \secref{sec:detection} for further information.
\item
The filename \textit{main} must be specified without the |.tex| extension.
\item
The filename \textit{main} is case sensitive
(even in case-insensitive file systems)
due to internal string comparison.
\item
The argument \textit{main} should be fully expanded, it cannot be a macro.
\item
Subdirectories and special characters should be avoided in filenames.
\item
The command |\childdocmain{|\textit{main}|}| must be followed by a whitespace.
It should not be followed immediately by another command
or by a comment mark `|%|'.
This is because the \TeX{} parser reads the token immediately following
the argument of |\childdocmain| and puts it
at the beginning of every child section;
however, a white\-space is ignored.
\end{itemize}

%%%%%%%%%%%%%%%%%%%%%%%%%%%%%%%%%%%%%%%%
\paragraph{Content of Main File.}

It is advisable to place all content in the child files included by |\include|.
Any output contained in the main file will appear in all child documents
unless suppressed manually;
it cannot be suppressed automatically by the |\includeonly| directive
and thus should normally be avoided.
A method to include some content in the main file
by means of conditional processing is described in \secref{sec:conditional}.

%%%%%%%%%%%%%%%%%%%%%%%%%%%%%%%%%%%%%%%%
\paragraph{Page Numbering.}

When only a part of the document is compiled,
the appropriate numbering of pages
(as well as other status parameters)
is determined from the |.aux| files.
The latter contain information from previous passes.
However this information needs to propagate through
all intermediate child documents.
Therefore the page numbering in child documents may well
be inconsistent until the complete document is compiled at least once.

A useful (if unconventional) way to always ensure a consistent
page numbering is to restart the numbering in each child document
and denote the pages by `\textit{child}|.|\textit{page}'
where \textit{child} represents the chapter/section number of the child file.
This can be achieved by the command
|\numberwithin{page}{|\textit{child}|}|
of the \textsf{amsmath} package
where \textit{child} can be |chapter| or |section|
depending on the chosen structuring.
Alternatively, one can modify the macro |\thepage| appropriately
and reset the counter |page| at the start of each child file.

%%%%%%%%%%%%%%%%%%%%%%%%%%%%%%%%%%%%%%%%%%%%%%%%%%%%%%%%%%%%%%%%%%%%%%%%%%%%%%%%
\subsection{Conditional Processing}
\label{sec:conditional}

The package provides a mechanism to compile different versions
of a document. To customise the versions further some conditional processing
can come in handy to distinguish which version is being compiled.
The package provides two macros to describe the compilation context:

%%%%%%%%%%%%%%%%%%%%%%%%%%%%%%%%%%%%%%%%
\DescribeMacro{\ifchilddoc}
The conditional |\ifchilddoc| distinguishes between the compilation of
child documents and the main document:
%
\begin{center}
|\ifchilddoc |\textit{child-code}| |[|\||else |\textit{main-code}]| \||fi|
\end{center}

%%%%%%%%%%%%%%%%%%%%%%%%%%%%%%%%%%%%%%%%
\DescribeMacro{\childdocname}
\DescribeMacro{\childdocjob}
The macro |\childdocname| contains the filename (without extension)
of the main or child file being processed.
Note that |\childdocjob| will always contain the name of the main file.

%%%%%%%%%%%%%%%%%%%%%%%%%%%%%%%%%%%%%%%%
\paragraph{Title Page.}

Conditional processing can be used to include a title or banner page
in the main document when proper precautions are taken.
Importantly, the code in the main file should ensure that the page counter
(as well as other status parameters which are stored in the |.aux| files)
takes the same value after the conditional processing.
Otherwise the page numbers may take divergent values
depending on which part is compiled.

For example, a title page could be declared by:
%
\begin{center}
\begin{tabular}{l}
|\ifchilddoc\||else|\\
|\addtocounter{page}{-1}|\\
\textit{code for title page}\\
|\newpage|\\
|\||fi|
\end{tabular}
\end{center}
%
A banner page for the child documents can be generated by:
%
\begin{center}
\begin{tabular}{l}
|\ifchilddoc|\\
|\addtocounter{page}{-1}|\\
\textit{code for banner page}\\
|\newpage|\\
|\||fi|
\end{tabular}
\end{center}
%
Here one could write a message such as:
\begin{center}
|This is the part \childdocname{} of \childdocjob{}.|
\end{center}

%%%%%%%%%%%%%%%%%%%%%%%%%%%%%%%%%%%%%%%%%%%%%%%%%%%%%%%%%%%%%%%%%%%%%%%%%%%%%%%%
\subsection{Flags}
\label{sec:flags}

The package makes it easy to generate different versions
of the main or child documents.
To this end compilation flags can be defined
and assigned different default values.
They will be particularly useful in conjunction
with the forwarding mechanism described in \secref{sec:forward}.

For example, it may be useful to have a flag |\version|
which can be set to |draft| or |final|.
The document source will contain some conditional code
depending on the value of |\version|.
Suppose further, the flag should default to |final| for the main file
and to |draft| for child files
which is a natural assignment for editing the document.
This is achieved by placing the following code
in the preamble of the main document
(below the |\childdocmain| directive):
%
\begin{center}
\begin{tabular}{l}
|\ifchilddoc|\\
|\providecommand{\version}{draft}|\\
|\||else|\\
|\providecommand{\version}{final}|\\
|\||fi|
\end{tabular}
\end{center}
%
The definition by |\providecommand| makes sure
that previous definitions are not overwritten.
Further statements |\providecommand{\version}{...}|
can thus be added before the above code to override it.

For the main file, one might add a line
(between |\childdocmain| and the above block)
%
\begin{center}
|%\ifchilddoc\||else\providecommand{\version}{draft}\||fi|
\end{center}
%
which can be uncommented to produce a draft version.
Likewise one can add a line to the very top of a child file
(above the |\childdocof{|\textit{main}|}| directive)
%
\begin{center}
|%\providecommand{\version}{final}|
\end{center}
%
which can be uncommented to produce the final version of this child document.

%%%%%%%%%%%%%%%%%%%%%%%%%%%%%%%%%%%%%%%%%%%%%%%%%%%%%%%%%%%%%%%%%%%%%%%%%%%%%%%%
\subsection{Forwarding}
\label{sec:forward}

Different versions of the main or child documents
using compilation flags as described in \secref{sec:flags}
can be (permanently) stored in different files
for convenient compilation, viewing and distribution.
To this end, the package defines a command
to pass on compilation to a different file:

%%%%%%%%%%%%%%%%%%%%%%%%%%%%%%%%%%%%%%%%
\DescribeMacro{\childdocforward}
The command |\childdocforward| redirects processing to
another source file:
%
\begin{center}
\begin{tabular}{l}
|\input{childdoc.def}|\\
|\childdocforward[|\textit{main}|]{|\textit{dest}|}|\\
\end{tabular}
\end{center}
%
The argument \textit{dest} is the destination file
(without extension).
It should be the main file or one of the child files.
Note that further \textsf{childdoc} directives
such as |\childdocof| and |\childdocforward|
in the indicated file will be processed in this form.
The optional argument \textit{main}
passes on directly to the main file \textit{main}
while pretending to compile the child \textit{dest}.
This form behaves as if \textit{dest}
issues |\childdocof{|\textit{main}|}| right away,
and no further \textsf{childdoc} directives will be processed.

%%%%%%%%%%%%%%%%%%%%%%%%%%%%%%%%%%%%%%%%
\DescribeMacro{\...prefix}
In the alternative form |\childdocforwardprefix|,
%
\begin{center}
\begin{tabular}{l}
|\input{childdoc.def}|\\
|\childdocforwardprefix[|\textit{main}|]{|\textit{prefix}|}{|\textit{dest}|}|
\end{tabular}
\end{center}
%
the destination file is determined by a pattern
depending on the current file:
To make this work, the current file must be called
`{\textit{prefix}\hspace{0.2em}\textit{suffix}}'
with \textit{prefix} matching precisely the argument.
Processing is then passed on to the file
`{\textit{dest}\hspace{0.2em}\textit{suffix}}'.
Surely, the same effect is achieved by
directly specifying the
argument `{\textit{dest}\hspace{0.2em}\textit{suffix}}'
in the first form.
However, that requires to set up a different file
for each child. With the alternative form of the command
all these files can have exactly the same content
which simplifies setting them up and maintaining them.

For example, the following file |draft.tex|
with a compilation flag |\version| as described in \secref{sec:flags}
compiles the main document as a draft:
%
\begin{center}
\begin{tabular}{l}
|\def\version{draft}|\\
|\input{childdoc.def}|\\
|\childdocforward{|\textit{main}|}|
\end{tabular}
\end{center}
%
Likewise, the following files |final|\textit{nn}|.tex|
compile the final version of the child document
|child|\textit{nn}|.tex|:
%
\begin{center}
\begin{tabular}{l}
|\def\version{final}|\\
|\input{childdoc.def}|\\
|\childdocforwardprefix{final}{child}|
\end{tabular}
\end{center}
%

Note that when several versions of a main file and/or of each child file
are to be generated, it may be convenient to set up a |Makefile| or
shell script to automatise the process.

%%%%%%%%%%%%%%%%%%%%%%%%%%%%%%%%%%%%%%%%%%%%%%%%%%%%%%%%%%%%%%%%%%%%%%%%%%%%%%%%
\subsection{Command Line Processing}
\label{sec:commandline}

The effect of redirection files can also be achieved by invoking
the \LaTeX{} compiler with a more elaborate command line.
Most conveniently this should be done as part
of a shell script or a |Makefile|.

When using \textsf{childdoc} in the main file, the following
command lines effectively perform a redirection
(note that depending on the shell being used,
backslashes may have to be doubled: `|\|' $\to$ `|\\|'):
%
\begin{center}
|... -jobname "|\textit{target}|" |\\|"|[\textit{flags}]%
|\input{childdoc.def}\childdocforward[|\textit{main}|]{|\textit{dest}|}"|
\end{center}
%
Here \textit{target} is the name of the output file,
\textit{main} is the name of the main file
and \textit{dest} is the name of the main or child file to be processed
(all filenames without extensions).
The optional argument \textit{main} can be omitted
if \textit{main} matches \textit{dest}.
Optionally, compilation \textit{flags} can be defined via |\def| commands.
This command line makes the \TeX{} engine believe
it is compiling the file \textit{target}
whose content is specified as the latter parameter.
The provided code then forwards the processing to
\textit{main} or \textit{dest} as described in \secref{sec:forward}.

%%%%%%%%%%%%%%%%%%%%%%%%%%%%%%%%%%%%%%%%%%%%%%%%%%%%%%%%%%%%%%%%%%%%%%%%%%%%%%%%
\subsection{Include by Input}
\label{sec:input}

Including child documents by |\include| has some restrictions by design.
Most notably, the content of a child document always occupies
its own set of pages; pages cannot be shared between child documents.
Usually, this behaviour makes perfect sense
because each child document contain an essential part of the document.
However, in some situations it may be desirable to compose
a document from a collection of parts
without having mandatory page breaks between then.
For this case, the package
provides a mechanism to include parts
by |\input| which can also be processed individually.
However, by construction this mechanism
requires manual handling of the content to be output.

%%%%%%%%%%%%%%%%%%%%%%%%%%%%%%%%%%%%%%%%
\DescribeMacro{\ifchilddocmanual}
The main file should be prepared as usual, see \secref{sec:include}.
However, the document body must make a distinction
between processing of an individual part and of the main document, e.g.:
%
\begin{center}
\begin{tabular}{l}
|\ifchilddocmanual|\\
|\input{\childdocname}|\\
|\||else|\\
\textit{document body with }|\input{|\textit{part}|}|\\
|\||fi|
\end{tabular}
\end{center}
%
The conditional |\ifchilddocmanual| is true whenever
a part to be included by |\input| is being compiled,
and the name of the part is stored in |\childdocname|.

%%%%%%%%%%%%%%%%%%%%%%%%%%%%%%%%%%%%%%%%
\DescribeMacro{\childdocby}
Each part to be included by |\input| should start with:
%
\begin{center}
\begin{tabular}{l}
|\input{childdoc.def}|\\
|\childdocby{|\textit{main}|}|\\
\end{tabular}
\end{center}
%
The directive |\childdocby| is similar to |\childdocof|
described in \secref{sec:include},
but the subsequent selection of content must be done manually.
To that end, both |\ifchilddoc| and |\ifchilddocmanual|
will be true upon processing of a part,
and the name of the part is stored in |\childdocname|.
Note that |\jobname| will be set to the filename of the current part
so that each part receives an individual |.aux| file
that does not interfere with the |.aux| file(s) of the main document.
This behaviour can be altered by the alternative form
|\childdocby[*]{|\textit{main}|}| (with a non-empty optional argument)
which uses the |.aux| file of the main document
by setting |\jobname| to \textit{main}.

%%%%%%%%%%%%%%%%%%%%%%%%%%%%%%%%%%%%%%%%%%%%%%%%%%%%%%%%%%%%%%%%%%%%%%%%%%%%%%%%
\subsection{Driver Development}
\label{sec:driver}

The \textsf{childdoc} mechanism can also be use for the development
of definition files such as \LaTeX{} styles or classes.
This case differs from the above setup with multiple parts
included by |\include| in that no |\includeonly| should be invoked.
This can be achieved by starting the include file
(before |\ProvidesPackage|) with:
%
\begin{center}
\begin{tabular}{l}
|\input{childdoc.def}|\\
|\childdocforward{|\textit{main}|}|\\
\end{tabular}
\end{center}
%
or alternatively with:
%
\begin{center}
\begin{tabular}{l}
|\input{childdoc.def}|\\
|\childdocby{|\textit{main}|}|\\
\end{tabular}
\end{center}
%
Both forms have slightly different effects as described above.
The main file is prepared as usual, see \secref{sec:include}.

%%%%%%%%%%%%%%%%%%%%%%%%%%%%%%%%%%%%%%%%%%%%%%%%%%%%%%%%%%%%%%%%%%%%%%%%%%%%%%%%
\subsection{Legacy Detection}
\label{sec:detection}

The directive |\childdocmain| in the main file can detect
whether the complete document or merely a child is to be compiled
even without using the directive |\childdocof|.
This method is deprecated because it is less robust
and there is no compelling reason to use it;
it is merely provided for backward compatibility
and it may be removed in future versions.

If the detection mechanism is to be used,
it is mandatory to correctly specify
the filename of the main file as the argument of |\childdocmain|:
%
\begin{center}
\begin{tabular}{l}
|\input{childdoc.def}|\\
|\childdocmain{|\textit{main}|}|\\
\end{tabular}
\end{center}
%
If |\jobname| does not match the argument \textit{main} of |\childdocmain|,
it is assumed that |\jobname| points to the child file to be compiled.
When using |\childdocmain| with the main file specified as argument,
it suffices to start a child file
with just |\input{|\textit{main}|}|
without loading of the package and using |\childdocof|.
If instead all processing is done
with the appropriate \textsf{childdoc} directives,
the argument of \textit{main} of |\childdocmain| can be empty.

An alternative version of the command line processing described
in \secref{sec:commandline} using the detection mechanism reads:
%
\begin{center}
|... -jobname "|\textit{target}|" "|[\textit{flags}]%
[|\def\jobname{|\textit{dest}|}|]|\input{|\textit{main}|}"|
\end{center}

%%%%%%%%%%%%%%%%%%%%%%%%%%%%%%%%%%%%%%%%%%%%%%%%%%%%%%%%%%%%%%%%%%%%%%%%%%%%%%%%
\subsection{Manual Code}
\label{sec:manual}

In case one cannot be certain whether the definitions file |childdoc.def|
is installed on the target \TeX{} distribution
and one prefers not to ship it,
it is conceivable to paste a few relevant commands into the sources.

To that end, drop all statements |\input{childdoc.def}|
and perform the replacements as outlined below.
Instead of |\childdocmain{|\textit{main}|}| add the following code
to the top of the main file:
%
\begin{center}
\begin{tabular}{l}
|\||ifdefined\childdocname\endinput\||fi\newif\ifchilddoc|\\
|\edef\childdocname{\scantokens\expandafter{\jobname\noexpand}}|\\
|\def\childdocmain{|\textit{main}|}\||ifx\childdocmain\childdocname\||else|\\
|\childdoctrue\includeonly{\childdocname}\let\jobname\childdocmain\||fi|\\
\end{tabular}
\end{center}
%
Instead of |\childdocof{|\textit{main}|}| just include the main file
at the top of each child file:
%
\begin{center}
|\input{|\textit{main}|}|
\end{center}
%
A simple redirection |\childdocforward{|\textit{dest}|}| is achieved by:
%
\begin{center}
|\def\jobname{|\textit{dest}|}\input{\jobname}|
\end{center}
%
The redirection with prefix
|\childdocforwardprefix[|\textit{prefix}|]{|\textit{dest}|}|
is accomplished by:
%
\begin{center}
\begin{tabular}{l}
|{\edef\jobname{\scantokens\expandafter{\jobname\noexpand}}|\\
|\def\redirectjob |\textit{prefix}|#1~~~{\gdef\jobname{|\textit{dest}|#1}}|\\
|\expandafter\redirectjob\jobname~~~}\input{\jobname}|
\end{tabular}
\end{center}

In an alternative approach,
child documents can be compiled by a specific command line
without additional code or specific definitions:
%
\begin{center}
|... -jobname "|\textit{target}|" "|[\textit{flags}]%
|\includeonly{|\textit{dest}|}\input{|\textit{main}|}"|
\end{center}
%

%%%%%%%%%%%%%%%%%%%%%%%%%%%%%%%%%%%%%%%%%%%%%%%%%%%%%%%%%%%%%%%%%%%%%%%%%%%%%%%%
%%%%%%%%%%%%%%%%%%%%%%%%%%%%%%%%%%%%%%%%%%%%%%%%%%%%%%%%%%%%%%%%%%%%%%%%%%%%%%%%
\section{Information}

%%%%%%%%%%%%%%%%%%%%%%%%%%%%%%%%%%%%%%%%%%%%%%%%%%%%%%%%%%%%%%%%%%%%%%%%%%%%%%%%
\subsection{Copyright}

Copyright \copyright{} 2017--2018 Niklas Beisert

This work may be distributed and/or modified under the
conditions of the \LaTeX{} Project Public License, either version 1.3
of this license or (at your option) any later version.
The latest version of this license is in
  \url{http://www.latex-project.org/lppl.txt}
and version 1.3 or later is part of all distributions of \LaTeX{}
version 2005/12/01 or later.

This work has the LPPL maintenance status `maintained'.

The Current Maintainer of this work is Niklas Beisert.

This work consists of the files |README.txt|, |childdoc.ins| and |childdoc.dtx|
as well as the derived files |childdoc.def|, |cdocsamp.tex|
with |cdocsch1.tex|, |cdocsch2.tex|, |cdocspt3.tex|, |cdocspt4.tex|,
|cdocsdrf.tex|, |cdocsfn1.tex|, |cdocsfn2.tex|
as well as |childdoc.pdf|.

%%%%%%%%%%%%%%%%%%%%%%%%%%%%%%%%%%%%%%%%%%%%%%%%%%%%%%%%%%%%%%%%%%%%%%%%%%%%%%%%
\subsection{Files and Installation}

The package consists of the files:
%
\begin{center}
\begin{tabular}{ll}
    |README.txt|   & readme file \\
    |childdoc.ins| & installation file \\
    |childdoc.dtx| & source file \\
    |childdoc.def| & definition file \\
    |cdocsamp.tex| & sample main file \\
    |cdocsch1.tex| & sample include file \\
    |cdocsch2.tex| & sample include file \\
    |cdocspt3.tex| & sample part file \\
    |cdocspt4.tex| & sample part file \\
    |cdocsdrf.tex| & sample redirection file \\
    |cdocsfn1.tex| & sample redirection file \\
    |cdocsfn2.tex| & sample redirection file \\
    |childdoc.pdf| & manual
\end{tabular}
\end{center}
%
The distribution consists of the files
|README.txt|, |childdoc.ins| and |childdoc.dtx|.
%
\begin{itemize}
\item
Run (pdf)\LaTeX{} on |childdoc.dtx|
to compile the manual |childdoc.pdf| (this file).
\item
Run \LaTeX{} on |childdoc.ins| to create the definitions file |childdoc.def|
and the sample |cdocsamp.tex| with include files
|cdocsch1.tex|, |cdocsch2.tex|, |cdocspt3.tex|, |cdocspt4.tex|,
|cdocsdrf.tex|, |cdocsfn1.tex|, |cdocsfn2.tex|.
Then copy the file |childdoc.def| to an appropriate directory of your \LaTeX{}
distribution, e.g.\ \textit{texmf-root}|/tex/latex/childdoc|.
\end{itemize}

%%%%%%%%%%%%%%%%%%%%%%%%%%%%%%%%%%%%%%%%%%%%%%%%%%%%%%%%%%%%%%%%%%%%%%%%%%%%%%%%
\subsection{Related CTAN Packages}

There are several other packages which offer a similar functionality:
%
\begin{itemize}
\item
The packages
\href{http://ctan.org/pkg/docmute}{\textsf{docmute}},
\href{http://ctan.org/pkg/includex}{\textsf{includex}} and
\href{http://ctan.org/pkg/standalone}{\textsf{standalone}}
provide commands to include only the document body of
a child file thus allowing both files to be compiled individually.
\item
The packages \href{http://ctan.org/pkg/subdocs}{\textsf{subdocs}}
and \href{http://ctan.org/pkg/subfiles}{\textsf{subfiles}}
provide structures in which the main and child documents can be
encapsulated and allowing them to be compiled individually.
The inclusion mechanism is different from the conventional |\include|.
\item
The package \href{http://ctan.org/pkg/combine}{\textsf{combine}}
is an elaborate solution to combine several documents into one.
\end{itemize}
%
See also the CTAN topic \href{http://ctan.org/topic/subdocs}{\textsf{subdocs}}
for further related packages.
The present package differs from the above solutions in that
a document structure constructed with the conventional |\include| mechanism
just needs two extra commands at the top of every file
such that all constituent files can be compiled individually.

%%%%%%%%%%%%%%%%%%%%%%%%%%%%%%%%%%%%%%%%%%%%%%%%%%%%%%%%%%%%%%%%%%%%%%%%%%%%%%%%
%\subsection{Feature Suggestions}
%
%The following is a list of features which may be useful for future
%versions of this package:
%%
%\begin{itemize}
%\item
%\ldots
%\end{itemize}

%%%%%%%%%%%%%%%%%%%%%%%%%%%%%%%%%%%%%%%%%%%%%%%%%%%%%%%%%%%%%%%%%%%%%%%%%%%%%%%%
\subsection{Revision History}

%%%%%%%%%%%%%%%%%%%%%%%%%%%%%%%%%%%%%%%%
\paragraph{v2.0:} 2018/12/30

\begin{itemize}
\item
immediate forward processing
\item
added |\childdocby| mechanism
\item
manual restructured
\end{itemize}

%%%%%%%%%%%%%%%%%%%%%%%%%%%%%%%%%%%%%%%%
\paragraph{v1.6:} 2018/01/17

\begin{itemize}
\item
application for development of include files
\item
corrections to manual
\end{itemize}

%%%%%%%%%%%%%%%%%%%%%%%%%%%%%%%%%%%%%%%%
\paragraph{v1.5:} 2017/05/21

\begin{itemize}
\item
more complete structuring introduced
\item
|\childdocof| introduced
\item
|\childdoc| renamed to |\childdocmain|
\item
|\childredirect| renamed to |\childdocforward| and |\childdocforwardprefix|
and functionality expanded
\end{itemize}

%%%%%%%%%%%%%%%%%%%%%%%%%%%%%%%%%%%%%%%%
\paragraph{v1.0:} 2017/04/27

\begin{itemize}
\item
manual and install package
\item
first version published on CTAN
\end{itemize}

%%%%%%%%%%%%%%%%%%%%%%%%%%%%%%%%%%%%%%%%
\paragraph{v0.6:} 2017/04/26

\begin{itemize}
\item
redirection mechanism added
\end{itemize}

%%%%%%%%%%%%%%%%%%%%%%%%%%%%%%%%%%%%%%%%
\paragraph{v0.5:} 2017/04/26

\begin{itemize}
\item
functionality in definition file
\end{itemize}


%%%%%%%%%%%%%%%%%%%%%%%%%%%%%%%%%%%%%%%%%%%%%%%%%%%%%%%%%%%%%%%%%%%%%%%%%%%%%%%%
%%%%%%%%%%%%%%%%%%%%%%%%%%%%%%%%%%%%%%%%%%%%%%%%%%%%%%%%%%%%%%%%%%%%%%%%%%%%%%%%
%%%%%%%%%%%%%%%%%%%%%%%%%%%%%%%%%%%%%%%%%%%%%%%%%%%%%%%%%%%%%%%%%%%%%%%%%%%%%%%%
\appendix

\settowidth\MacroIndent{\rmfamily\scriptsize 000\ }

 \DocInput{childdoc.dtx}

\end{document}
%</driver>
% \fi
%
% %%%%%%%%%%%%%%%%%%%%%%%%%%%%%%%%%%%%%%%%%%%%%%%%%%%%%%%%%%%%%%%%%%%%%%%%%%%%%%
% %%%%%%%%%%%%%%%%%%%%%%%%%%%%%%%%%%%%%%%%%%%%%%%%%%%%%%%%%%%%%%%%%%%%%%%%%%%%%%
% \section{Sample}
%\iffalse
%<*samplemain>
%\fi
%
% The following presents a sample document
% with two chapters, two parts, a title page,
% a compile flag as well as three forwarding files to set the flag.
% It consists of eight |.tex| files:
% \begin{center}
% \begin{tabular}{ll}
% |cdocsamp.tex|&main file\\
% |cdocsch1.tex|&include file for chapter 1\\
% |cdocsch2.tex|&include file for chapter 2\\
% |cdocspt3.tex|&include file for part 3\\
% |cdocspt4.tex|&include file for part 4\\
% |cdocsdrf.tex|&forwarding file for main file in draft mode\\
% |cdocsfi1.tex|&forwarding file for final version of chapter 1\\
% |cdocsfi2.tex|&forwarding file for final version of chapter 2\\
% \end{tabular}
% \end{center}
% Each of the eight files can be compiled directly by the \LaTeX{} compiler.
%
% %%%%%%%%%%%%%%%%%%%%%%%%%%%%%%%%%%%%%%
% \paragraph{Main File.}
%
% The main file is called |cdocsamp.tex|.
%
% Load the \textsf{childdoc} definitions and
% declare the filename for the main document:
%    \begin{macrocode}
\input{childdoc.def}
\childdocmain{}
%    \end{macrocode}

% Optional override for |\version| flag:
%    \begin{macrocode}
%%\ifchilddoc\else\providecommand{\version}{draft}\fi
%    \end{macrocode}

% Define the default values for the |\version| flag
% (|final| for the main file and |draft| for childs):
%    \begin{macrocode}
\ifchilddoc
\providecommand{\version}{draft}
\else
\providecommand{\version}{final}
\fi
%    \end{macrocode}

% Load the standard document class:
%    \begin{macrocode}
\documentclass[12pt]{article}
%    \end{macrocode}

% Start the document body:
%    \begin{macrocode}
\begin{document}
%    \end{macrocode}

% Declare a title page.
% Print title, part of document being processed and version flag:
%    \begin{macrocode}
\addtocounter{page}{-1}
\begin{center}
{\LARGE\bfseries{}childdoc example\par}
\vspace{1cm}
\ifchilddoc
\ifchilddocmanual part\else chapter\fi:
`\childdocname' of `\childdocjob'\par
\else
main document: `\childdocjob'\par
\fi
version: \version\par
\end{center}
\newpage
%    \end{macrocode}

% Manually include selected file,
% otherwise process as usual:
%    \begin{macrocode}
\ifchilddocmanual
\section*{part `\childdocname'}
\input{\childdocname}
\else
%    \end{macrocode}

% Include the two chapters:
%    \begin{macrocode}
\include{cdocsch1}
\include{cdocsch2}
%    \end{macrocode}

% Include the two parts unless only chapters should be displayed:
%    \begin{macrocode}
\ifchilddoc\else
\section{part three}
\input{cdocspt3}
\section{part four}
\input{cdocspt4}
\fi
%    \end{macrocode}

% Process as usual until here:
%    \begin{macrocode}
\fi
%    \end{macrocode}

% End of document body:
%    \begin{macrocode}
\end{document}
%    \end{macrocode}
%\iffalse
%</samplemain>
%\fi
%
% %%%%%%%%%%%%%%%%%%%%%%%%%%%%%%%%%%%%%%
% \paragraph{Chapter Include Files.}
%
% The include files are called |cdocsch1.tex| and |cdocsch2.tex|.
%
%\iffalse
%<*samplechap1|samplechap2>
%\fi

% Optional override for |\version| flag:
%    \begin{macrocode}
%%\providecommand{\version}{final}
%    \end{macrocode}

% Include the main document:
%    \begin{macrocode}
\input{childdoc.def}
\childdocof{cdocsamp}
%    \end{macrocode}

%\iffalse
%</samplechap1|samplechap2>
%\fi
%
%\iffalse
%<*samplechap1>
%\fi
% Some text for chapter 1:
%    \begin{macrocode}
\section{one}
some text in chapter one
%    \end{macrocode}

%\iffalse
%</samplechap1>
%\fi
% Some text for chapter 2:
%\iffalse
%<*samplechap2>
%\fi
%    \begin{macrocode}
\section{two}
more text in chapter two
%    \end{macrocode}

%\iffalse
%</samplechap2>
%\fi
%
% %%%%%%%%%%%%%%%%%%%%%%%%%%%%%%%%%%%%%%
% \paragraph{Part Include Files.}
%
% The include files are called |cdocspt3.tex| and |cdocspt4.tex|.
%
%\iffalse
%<*samplepart3|samplepart4>
%\fi

% Optional override for |\version| flag:
%    \begin{macrocode}
%%\providecommand{\version}{final}
%    \end{macrocode}

% Include the main document:
%    \begin{macrocode}
\input{childdoc.def}
\childdocby{cdocsamp}
%    \end{macrocode}

%\iffalse
%</samplepart3|samplepart4>
%\fi
%
%\iffalse
%<*samplepart3>
%\fi
% Some text for part 3:
%    \begin{macrocode}
some text in part three
%    \end{macrocode}

%\iffalse
%</samplepart3>
%\fi
% Some text for part 4:
%\iffalse
%<*samplepart4>
%\fi
%    \begin{macrocode}
more text in part four
%    \end{macrocode}

%\iffalse
%</samplepart4>
%\fi
%
% %%%%%%%%%%%%%%%%%%%%%%%%%%%%%%%%%%%%%%
% \paragraph{Forwarding for a Complete Draft.}
%
% The following forwarding file |cdocsdrf.tex|
% compiles the main document in draft mode:
%\iffalse
%<*sampledraft>
%\fi
%    \begin{macrocode}
\def\version{draft}
\input{childdoc.def}
\childdocforward{cdocsamp}
%    \end{macrocode}

%\iffalse
%</sampledraft>
%\fi
%
% %%%%%%%%%%%%%%%%%%%%%%%%%%%%%%%%%%%%%%
% \paragraph{Forwarding for Final Version of the Chapters.}
%
% The following forwarding files |cdocsfn1.tex| and |cdocsfn2.tex|
% (with identical content)
% compile the final versions of the child documents
% |cdocsch1.tex| and |cdocsch2.tex|, respectively:
%\iffalse
%<*samplefinal>
%\fi
%    \begin{macrocode}
\def\version{final}
\input{childdoc.def}
\childdocforwardprefix[cdocsamp]{cdocsfn}{cdocsch}
%    \end{macrocode}

%\iffalse
%</samplefinal>
%\fi
%
% %%%%%%%%%%%%%%%%%%%%%%%%%%%%%%%%%%%%%%
% \paragraph{Command Line Processing.}
%
% The following three command lines generate the output files
% |cdocscld|, |cdocscl1| and |cdocscl2|
% which should be identical to
% |cdocsdrf|, |cdocsch1| and |cdocsfn2|, respectively:
% \begin{center}
% \begin{tabular}{l}
% |latex -jobname cdocscld \|\\
% |  "\def\version{draft}\input{childdoc.def}\childdocforward{cdocsamp}"|\\
% |latex -jobname cdocscl1 \|\\
% |  "\input{childdoc.def}\childdocforward[cdocsamp]{cdocsch1}"|\\
% |latex -jobname cdocscl2 \|\\
% |  "\def\version{final}\input{childdoc.def}\childdocforward{cdocsch2}"|
% \end{tabular}
% \end{center}
% Note that the trailing backslash on each first line
% merely continues the input to the second line
% (for convenient cut ant paste).
% Furthermore, the command |latex| can be replaced by any
% of its alternative versions such as |pdflatex|.
%
% %%%%%%%%%%%%%%%%%%%%%%%%%%%%%%%%%%%%%%%%%%%%%%%%%%%%%%%%%%%%%%%%%%%%%%%%%%%%%%
% %%%%%%%%%%%%%%%%%%%%%%%%%%%%%%%%%%%%%%%%%%%%%%%%%%%%%%%%%%%%%%%%%%%%%%%%%%%%%%
% \section{Implementation}
%\iffalse
%<*package>
%\fi
%
% This section describes the definitions file |childdoc.def|.

% The definitions cannot be loaded using |\usepackage| or |\RequirePackage|
% which has a mechanism to prevent loading a style file more than once.
% When loading the definitions by means of |\input|
% multiple instances have to be prevented manually:
%\iffalse
%This code needs to be before the `\ProvidesFile' directive
%which is defined at the beginning of this file.
%Therefore it is also placed there and commented out here.
%</package>
%<*discard>
%\fi
%    \begin{macrocode}
\ifdefined\childdocmain\endinput\fi
%    \end{macrocode}
%\iffalse
%</discard>
%<*package>
%\fi
%
% \macro{\ifchilddoc}
% \macro{\ifchilddocmanual}
% The conditional |\ifchilddoc| tells whether a
% child (true) or main (false) document is being compiled.
% The conditional |\ifchilddocmanual| tells whether
% the |\includeonly| mechanism is used (false) or
% the selection of child files must be performed manually (true).
% The definitions initialise to false:
%    \begin{macrocode}
\newif\ifchilddoc
\newif\ifchilddocmanual
%    \end{macrocode}

% \macro{\childdocname}
% \macro{\childdocjob}
% The macro |\childdocname| stores the name of the main document
% to be compiled. The macro |\childdocjob| stores the name of
% the document on which the \LaTeX{} compiler was originally invoked.
% The content of |\jobname| cannot be compared
% to filenames specified in the source due to different catcodes.
% The following code rescans |\jobname|, stores the result
% in |\childdocname| and saves a copy in |\childdocjob|:
%    \begin{macrocode}
\edef\childdocname{\scantokens\expandafter{\jobname\noexpand}}
\let\childdocjob\childdocname
%    \end{macrocode}

% \macro{\childdocdisable}
% The macro |\childdocdisable| prevents the main file
% from being processed more than once.
% At this stage, the main document command |\childdocmain|
% is assumed to be called once again where it should do nothing.
% Any subsequent call to it should prevent
% a secondary processing of the main document
% It overwrites the forwarding commands
% |\childdocof| and |\childdocforward|
% with empty macros to prevent further inclusions of the main document:
%    \begin{macrocode}
\newcommand{\childdocdisable}
{
  \renewcommand{\childdocmain}[1]{\renewcommand{\childdocmain}[1]{\endinput}}
  \renewcommand{\childdocof}[1]{}
  \renewcommand{\childdocby}[2][]{}
  \renewcommand{\childdocforward}[2][]{}
  \renewcommand{\childdocdisable}{}
}
%    \end{macrocode}

% \macro{\childdocmain}
% The macro |\childdocmain| is to be called at the top of the main file
% with nothing or the main filename (without extension) as argument.
% First, it breaks loops.
% If the argument is not empty and does not match |\childdocname|
% (which is set by the first inclusion of |childdoc.def|),
% |\ifchilddoc| is set to true, |\includeonly| is applied to the child file
% and |\jobname| is set to the main file
% (for proper handling of |.aux| files):
%    \begin{macrocode}
\newcommand{\childdocmain}[1]
{
  \childdocdisable\childdocmain{}
  \if?#1?\else
    \begingroup
      \def\childdoctmp{#1}
      \ifx\childdoctmp\childdocname
        \def\childdoctmp{}
      \else
        \def\childdoctmp
        {
          \childdoctrue
          \includeonly{\childdocname}
          \def\childdocjob{#1}
          \def\jobname{#1}
        }
      \fi
      \expandafter
    \endgroup
    \childdoctmp
  \fi
}
%    \end{macrocode}

% \macro{\childdocof}
% The command |\childdocof| redirects
% compilation to the main file |#1|.
%    \begin{macrocode}
\newcommand{\childdocof}[1]
{
  \childdocdisable
  \childdoctrue
  \includeonly{\childdocname}
  \def\jobname{#1}
  \def\childdocjob{#1}
  \input{#1}
}
%    \end{macrocode}

% \macro{\childdocby}
% The command |\childdocby| ....
%    \begin{macrocode}
\newcommand{\childdocby}[2][]
{
  \childdocdisable
  \childdoctrue
  \childdocmanualtrue
  \if?#1?\else
    \def\jobname{#2}
  \fi
  \def\childdocjob{#2}
  \input{#2}
  \endinput
}
%    \end{macrocode}

% \macro{\childdocforward}
% The command |\childdocforward| redirects
% compilation to the main file or
% (if the optional argument is given) a child file.
% Parameters are set as if the main file
% or a child file starting with |\childdocof| was compiled.
% Then compilation is handed over to the main file:
%    \begin{macrocode}
\newcommand{\childdocforward}[2][]
{
  \begingroup
    \if?#1?
      \def\childdoctmp
      {
        \def\childdocname{#2}
        \def\childdocjob{#2}
        \def\jobname{#2}
        \input{#2}
        \endinput
      }
    \else
      \def\childdoctmp
      {
        \childdocdisable
        \def\childdocname{#2}
        \childdoctrue
        \includeonly{#2}
        \def\childdocjob{#1}
        \def\jobname{#1}
        \input{#1}
        \endinput
      }
    \fi
    \expandafter
  \endgroup
  \childdoctmp
}
%    \end{macrocode}

% \macro{\childdocforwardprefix}
% The command |\childdocforwardprefix| redirects
% compilation to the main or a child file by means of a pattern.
% The prefix |#1| in the current filename is replaced by |#2|
% and the suffix of the current filename is kept
% (it is assumed that the filename does not contain the substring `|~~~|'
% which is used as a delimiter).
% Compilation is handed over to the new file by |\childdocforward|:
%    \begin{macrocode}
\newcommand{\childdocforwardprefix}[3][]
{
  \begingroup
    \def\childdocextract #2##1~~~{\def\childdoctmp{\childdocforward[#1]{#3##1}}}
    \expandafter\childdocextract\childdocname~~~
    \expandafter
  \endgroup
  \childdoctmp
}
%    \end{macrocode}

% \macro{\childdoc}
% The deprecated macro |\childdoc| is a legacy version of |\childdocmain|:
%    \begin{macrocode}
\newcommand{\childdoc}{\childdocmain}
%    \end{macrocode}

% \macro{\childdocredirect}
% The deprecated macro |\childdocredirect| is a legacy version
% of |\childdocforward| and |\childdocforwardprefix|:
%    \begin{macrocode}
\newcommand{\childdocredirect}[2][]
{
  \begingroup
    \if?#1?
      \def\childdoctmp{\childdocforward{#2}}
    \else
      \def\childdoctmp{\childdocforwardprefix{#1}{#2}}
    \fi
    \expandafter
  \endgroup
  \childdoctmp
}
%    \end{macrocode}

%\iffalse
%</package>
%\fi
%
\endinput
|\\
|\childdocforwardprefix{final}{child}|
\end{tabular}
\end{center}
%

Note that when several versions of a main file and/or of each child file
are to be generated, it may be convenient to set up a |Makefile| or
shell script to automatise the process.

%%%%%%%%%%%%%%%%%%%%%%%%%%%%%%%%%%%%%%%%%%%%%%%%%%%%%%%%%%%%%%%%%%%%%%%%%%%%%%%%
\subsection{Command Line Processing}
\label{sec:commandline}

The effect of redirection files can also be achieved by invoking
the \LaTeX{} compiler with a more elaborate command line.
Most conveniently this should be done as part
of a shell script or a |Makefile|.

When using \textsf{childdoc} in the main file, the following
command lines effectively perform a redirection
(note that depending on the shell being used,
backslashes may have to be doubled: `|\|' $\to$ `|\\|'):
%
\begin{center}
|... -jobname "|\textit{target}|" |\\|"|[\textit{flags}]%
|% \iffalse
%
% childdoc.dtx Copyright (C) 2017-2018 Niklas Beisert
%
% This work may be distributed and/or modified under the
% conditions of the LaTeX Project Public License, either version 1.3
% of this license or (at your option) any later version.
% The latest version of this license is in
%   http://www.latex-project.org/lppl.txt
% and version 1.3 or later is part of all distributions of LaTeX
% version 2005/12/01 or later.
%
% This work has the LPPL maintenance status `maintained'.
%
% The Current Maintainer of this work is Niklas Beisert.
%
% This work consists of the files childdoc.dtx and childdoc.ins
% and the derived files childdoc.def and cdocsamp.tex with
% cdocsch1.tex, cdocsch2.tex, cdocsdrf.tex, cdocsfn1.tex, cdocsfn2.tex.
%
%<package>\ifdefined\childdocmain\endinput\fi
%<package>\ProvidesFile{childdoc.def}[2018/12/30 v2.0 child document driver]
%<samplemain>\ProvidesFile{cdocsamp.tex}[2018/12/30 v2.0 sample for childdoc]
%<*driver>
%\ProvidesFile{childdoc.drv}[2018/12/30 v2.0 childdoc reference manual file]
\PassOptionsToClass{10pt,a4paper}{article}
\documentclass{ltxdoc}

\usepackage[margin=35mm]{geometry}
\usepackage{hyperref}
\usepackage{hyperxmp}
\usepackage[usenames]{color}

\hypersetup{colorlinks=true}
\hypersetup{pdfstartview=FitH}
\hypersetup{pdfpagemode=UseNone}
\hypersetup{pdfsource={}}
\hypersetup{pdflang={en-UK}}
\hypersetup{pdfcopyright={Copyright 2017-2018 Niklas Beisert.
  This work may be distributed and/or modified under the
  conditions of the LaTeX Project Public License, either version 1.3
  of this license or (at your option) any later version.}}
\hypersetup{pdflicenseurl={http://www.latex-project.org/lppl.txt}}
\hypersetup{pdfcontactaddress={ETH Zurich, ITP, HIT K,
  Wolfgang-Pauli-Strasse 27}}
\hypersetup{pdfcontactpostcode={8093}}
\hypersetup{pdfcontactcity={Zurich}}
\hypersetup{pdfcontactcountry={Switzerland}}
\hypersetup{pdfcontactemail={nbeisert@itp.phys.ethz.ch}}
\hypersetup{pdfcontacturl={http://people.phys.ethz.ch/\xmptilde nbeisert/}}

\newcommand{\secref}[1]{\hyperref[#1]{section \ref*{#1}}}

\parskip1ex
\parindent0pt
\let\olditemize\itemize
\def\itemize{\olditemize\parskip0pt}

\begin{document}

\title{The \textsf{childdoc} Package}
\hypersetup{pdftitle={The childdoc Package}}
\author{Niklas Beisert\\[2ex]
  Institut f\"ur Theoretische Physik\\
  Eidgen\"ossische Technische Hochschule Z\"urich\\
  Wolfgang-Pauli-Strasse 27, 8093 Z\"urich, Switzerland\\[1ex]
  \href{mailto:nbeisert@itp.phys.ethz.ch}
  {\texttt{nbeisert@itp.phys.ethz.ch}}}
\hypersetup{pdfauthor={Niklas Beisert}}
\hypersetup{pdfsubject={Manual for the LaTeX2e Package childdoc}}
\date{30 December 2018, \textsf{v2.0}}
\maketitle

\begin{abstract}\noindent
\textsf{childdoc} is a \LaTeXe{} package
that enables the direct compilation
of document sections included by |\include|
to individual files.
\end{abstract}

\begingroup
\parskip0ex
\tableofcontents
\endgroup

%%%%%%%%%%%%%%%%%%%%%%%%%%%%%%%%%%%%%%%%%%%%%%%%%%%%%%%%%%%%%%%%%%%%%%%%%%%%%%%%
%%%%%%%%%%%%%%%%%%%%%%%%%%%%%%%%%%%%%%%%%%%%%%%%%%%%%%%%%%%%%%%%%%%%%%%%%%%%%%%%
\section{Introduction}

\LaTeX{} provides a mechanism to structure a large document (such as a book)
into a main file and several child files (containing the chapters)
using the |\include| command.
This mechanism is beneficial for documents
which span hundreds of pages in order to
make the source file(s) more manageable.
Moreover, compilation can be restricted to
selected child files by means of the |\includeonly| command.
The latter feature can be used to reduce the compilation time while editing
(this was significantly more useful in the earlier days of \LaTeX{})
or to generate a smaller document which is easier to navigate.
Another application of |\includeonly| is to generate
documents consisting of selected parts of the complete document.

However, there are a few drawbacks of the plain |\include| mechanism:
\begin{itemize}
\item
The child files cannot be compiled on their own,
they can only be compiled via the main file.
A naive editing environment
(such as a text editor with an option
to have the current file processed by \LaTeX)
may require one to switch to the main file before compiling;
attempting to compile the child file produces errors.
\item
The main file must be modified (each time)
to adjust the |\includeonly| command
to the present needs. This easily leaves the main file in a messy state.
\item
The generated document will always carry the filename
of the main document. This is inconvenient if
several child files are to be compiled and
to be kept for distribution.
\end{itemize}

The present package provides a simple interface
to make child files individually compilable by \LaTeX{}.
Compiling a child file then has the same effect as compiling
the main file with an |\includeonly| command
to select the appropriate child.
Moreover the generated document will carry the name of the child
rather than the main file.
This resolves all three above issues.

This feature is meant to make the editing of books,
thesis documents and lecture notes somewhat more convenient.
However, the package can also be used efficiently for
composing a series of documents (such as exercise sheets)
which are typically distributed individually.
It then assists the author in generating the individual documents
(potentially in different versions)
as well as a document containing the collected series.
Another application is in developing style files
or other kinds of included material
where compilation of the style file could redirect
to a sample or test file.

%%%%%%%%%%%%%%%%%%%%%%%%%%%%%%%%%%%%%%%%%%%%%%%%%%%%%%%%%%%%%%%%%%%%%%%%%%%%%%%%
%%%%%%%%%%%%%%%%%%%%%%%%%%%%%%%%%%%%%%%%%%%%%%%%%%%%%%%%%%%%%%%%%%%%%%%%%%%%%%%%
\section{Usage}

First of all, the package \textsf{childdoc} is \emph{not} a standard
\LaTeXe{} |.sty| style file! Therefore it needs to be invoked in
a non-standard way.

%%%%%%%%%%%%%%%%%%%%%%%%%%%%%%%%%%%%%%%%%%%%%%%%%%%%%%%%%%%%%%%%%%%%%%%%%%%%%%%%
\subsection{Included Files}
\label{sec:include}

%%%%%%%%%%%%%%%%%%%%%%%%%%%%%%%%%%%%%%%%
\DescribeMacro{\childdocmain}
To use the package, add the commands
\begin{center}
\begin{tabular}{l}
|\input{childdoc.def}|\\
|\childdocmain{}|\\
\end{tabular}
\end{center}
at the very top of the main \LaTeX{} file,
in particular \emph{before} the |\documentclass| statement!
The argument of |\childdocmain| should be left empty
(but it must be present).

%%%%%%%%%%%%%%%%%%%%%%%%%%%%%%%%%%%%%%%%
\DescribeMacro{\childdocof}
Furthermore, add the commands
\begin{center}
\begin{tabular}{l}
|\input{childdoc.def}|\\
|\childdocof{|\textit{main}|}|\\
\end{tabular}
\end{center}
at the top of every child file \textit{child}
which is included by |\include{|\textit{child}|}|
from within the main file
(or at least for those files to be compiled individually).
The argument \textit{main} must be the filename of the main file.

There are a couple of
considerations in setting up the main and child documents:

%%%%%%%%%%%%%%%%%%%%%%%%%%%%%%%%%%%%%%%%
\paragraph{Restrictions.}

Please note the following restrictions:
\begin{itemize}
\item
|\childdocmain| must be called with one argument \textit{main}
to ensure compatibility with earlier version of the package.
It must either be empty (|\childdocmain{}|)
or precisely match the filename of the main file in which it is specified.
See \secref{sec:detection} for further information.
\item
The filename \textit{main} must be specified without the |.tex| extension.
\item
The filename \textit{main} is case sensitive
(even in case-insensitive file systems)
due to internal string comparison.
\item
The argument \textit{main} should be fully expanded, it cannot be a macro.
\item
Subdirectories and special characters should be avoided in filenames.
\item
The command |\childdocmain{|\textit{main}|}| must be followed by a whitespace.
It should not be followed immediately by another command
or by a comment mark `|%|'.
This is because the \TeX{} parser reads the token immediately following
the argument of |\childdocmain| and puts it
at the beginning of every child section;
however, a white\-space is ignored.
\end{itemize}

%%%%%%%%%%%%%%%%%%%%%%%%%%%%%%%%%%%%%%%%
\paragraph{Content of Main File.}

It is advisable to place all content in the child files included by |\include|.
Any output contained in the main file will appear in all child documents
unless suppressed manually;
it cannot be suppressed automatically by the |\includeonly| directive
and thus should normally be avoided.
A method to include some content in the main file
by means of conditional processing is described in \secref{sec:conditional}.

%%%%%%%%%%%%%%%%%%%%%%%%%%%%%%%%%%%%%%%%
\paragraph{Page Numbering.}

When only a part of the document is compiled,
the appropriate numbering of pages
(as well as other status parameters)
is determined from the |.aux| files.
The latter contain information from previous passes.
However this information needs to propagate through
all intermediate child documents.
Therefore the page numbering in child documents may well
be inconsistent until the complete document is compiled at least once.

A useful (if unconventional) way to always ensure a consistent
page numbering is to restart the numbering in each child document
and denote the pages by `\textit{child}|.|\textit{page}'
where \textit{child} represents the chapter/section number of the child file.
This can be achieved by the command
|\numberwithin{page}{|\textit{child}|}|
of the \textsf{amsmath} package
where \textit{child} can be |chapter| or |section|
depending on the chosen structuring.
Alternatively, one can modify the macro |\thepage| appropriately
and reset the counter |page| at the start of each child file.

%%%%%%%%%%%%%%%%%%%%%%%%%%%%%%%%%%%%%%%%%%%%%%%%%%%%%%%%%%%%%%%%%%%%%%%%%%%%%%%%
\subsection{Conditional Processing}
\label{sec:conditional}

The package provides a mechanism to compile different versions
of a document. To customise the versions further some conditional processing
can come in handy to distinguish which version is being compiled.
The package provides two macros to describe the compilation context:

%%%%%%%%%%%%%%%%%%%%%%%%%%%%%%%%%%%%%%%%
\DescribeMacro{\ifchilddoc}
The conditional |\ifchilddoc| distinguishes between the compilation of
child documents and the main document:
%
\begin{center}
|\ifchilddoc |\textit{child-code}| |[|\||else |\textit{main-code}]| \||fi|
\end{center}

%%%%%%%%%%%%%%%%%%%%%%%%%%%%%%%%%%%%%%%%
\DescribeMacro{\childdocname}
\DescribeMacro{\childdocjob}
The macro |\childdocname| contains the filename (without extension)
of the main or child file being processed.
Note that |\childdocjob| will always contain the name of the main file.

%%%%%%%%%%%%%%%%%%%%%%%%%%%%%%%%%%%%%%%%
\paragraph{Title Page.}

Conditional processing can be used to include a title or banner page
in the main document when proper precautions are taken.
Importantly, the code in the main file should ensure that the page counter
(as well as other status parameters which are stored in the |.aux| files)
takes the same value after the conditional processing.
Otherwise the page numbers may take divergent values
depending on which part is compiled.

For example, a title page could be declared by:
%
\begin{center}
\begin{tabular}{l}
|\ifchilddoc\||else|\\
|\addtocounter{page}{-1}|\\
\textit{code for title page}\\
|\newpage|\\
|\||fi|
\end{tabular}
\end{center}
%
A banner page for the child documents can be generated by:
%
\begin{center}
\begin{tabular}{l}
|\ifchilddoc|\\
|\addtocounter{page}{-1}|\\
\textit{code for banner page}\\
|\newpage|\\
|\||fi|
\end{tabular}
\end{center}
%
Here one could write a message such as:
\begin{center}
|This is the part \childdocname{} of \childdocjob{}.|
\end{center}

%%%%%%%%%%%%%%%%%%%%%%%%%%%%%%%%%%%%%%%%%%%%%%%%%%%%%%%%%%%%%%%%%%%%%%%%%%%%%%%%
\subsection{Flags}
\label{sec:flags}

The package makes it easy to generate different versions
of the main or child documents.
To this end compilation flags can be defined
and assigned different default values.
They will be particularly useful in conjunction
with the forwarding mechanism described in \secref{sec:forward}.

For example, it may be useful to have a flag |\version|
which can be set to |draft| or |final|.
The document source will contain some conditional code
depending on the value of |\version|.
Suppose further, the flag should default to |final| for the main file
and to |draft| for child files
which is a natural assignment for editing the document.
This is achieved by placing the following code
in the preamble of the main document
(below the |\childdocmain| directive):
%
\begin{center}
\begin{tabular}{l}
|\ifchilddoc|\\
|\providecommand{\version}{draft}|\\
|\||else|\\
|\providecommand{\version}{final}|\\
|\||fi|
\end{tabular}
\end{center}
%
The definition by |\providecommand| makes sure
that previous definitions are not overwritten.
Further statements |\providecommand{\version}{...}|
can thus be added before the above code to override it.

For the main file, one might add a line
(between |\childdocmain| and the above block)
%
\begin{center}
|%\ifchilddoc\||else\providecommand{\version}{draft}\||fi|
\end{center}
%
which can be uncommented to produce a draft version.
Likewise one can add a line to the very top of a child file
(above the |\childdocof{|\textit{main}|}| directive)
%
\begin{center}
|%\providecommand{\version}{final}|
\end{center}
%
which can be uncommented to produce the final version of this child document.

%%%%%%%%%%%%%%%%%%%%%%%%%%%%%%%%%%%%%%%%%%%%%%%%%%%%%%%%%%%%%%%%%%%%%%%%%%%%%%%%
\subsection{Forwarding}
\label{sec:forward}

Different versions of the main or child documents
using compilation flags as described in \secref{sec:flags}
can be (permanently) stored in different files
for convenient compilation, viewing and distribution.
To this end, the package defines a command
to pass on compilation to a different file:

%%%%%%%%%%%%%%%%%%%%%%%%%%%%%%%%%%%%%%%%
\DescribeMacro{\childdocforward}
The command |\childdocforward| redirects processing to
another source file:
%
\begin{center}
\begin{tabular}{l}
|\input{childdoc.def}|\\
|\childdocforward[|\textit{main}|]{|\textit{dest}|}|\\
\end{tabular}
\end{center}
%
The argument \textit{dest} is the destination file
(without extension).
It should be the main file or one of the child files.
Note that further \textsf{childdoc} directives
such as |\childdocof| and |\childdocforward|
in the indicated file will be processed in this form.
The optional argument \textit{main}
passes on directly to the main file \textit{main}
while pretending to compile the child \textit{dest}.
This form behaves as if \textit{dest}
issues |\childdocof{|\textit{main}|}| right away,
and no further \textsf{childdoc} directives will be processed.

%%%%%%%%%%%%%%%%%%%%%%%%%%%%%%%%%%%%%%%%
\DescribeMacro{\...prefix}
In the alternative form |\childdocforwardprefix|,
%
\begin{center}
\begin{tabular}{l}
|\input{childdoc.def}|\\
|\childdocforwardprefix[|\textit{main}|]{|\textit{prefix}|}{|\textit{dest}|}|
\end{tabular}
\end{center}
%
the destination file is determined by a pattern
depending on the current file:
To make this work, the current file must be called
`{\textit{prefix}\hspace{0.2em}\textit{suffix}}'
with \textit{prefix} matching precisely the argument.
Processing is then passed on to the file
`{\textit{dest}\hspace{0.2em}\textit{suffix}}'.
Surely, the same effect is achieved by
directly specifying the
argument `{\textit{dest}\hspace{0.2em}\textit{suffix}}'
in the first form.
However, that requires to set up a different file
for each child. With the alternative form of the command
all these files can have exactly the same content
which simplifies setting them up and maintaining them.

For example, the following file |draft.tex|
with a compilation flag |\version| as described in \secref{sec:flags}
compiles the main document as a draft:
%
\begin{center}
\begin{tabular}{l}
|\def\version{draft}|\\
|\input{childdoc.def}|\\
|\childdocforward{|\textit{main}|}|
\end{tabular}
\end{center}
%
Likewise, the following files |final|\textit{nn}|.tex|
compile the final version of the child document
|child|\textit{nn}|.tex|:
%
\begin{center}
\begin{tabular}{l}
|\def\version{final}|\\
|\input{childdoc.def}|\\
|\childdocforwardprefix{final}{child}|
\end{tabular}
\end{center}
%

Note that when several versions of a main file and/or of each child file
are to be generated, it may be convenient to set up a |Makefile| or
shell script to automatise the process.

%%%%%%%%%%%%%%%%%%%%%%%%%%%%%%%%%%%%%%%%%%%%%%%%%%%%%%%%%%%%%%%%%%%%%%%%%%%%%%%%
\subsection{Command Line Processing}
\label{sec:commandline}

The effect of redirection files can also be achieved by invoking
the \LaTeX{} compiler with a more elaborate command line.
Most conveniently this should be done as part
of a shell script or a |Makefile|.

When using \textsf{childdoc} in the main file, the following
command lines effectively perform a redirection
(note that depending on the shell being used,
backslashes may have to be doubled: `|\|' $\to$ `|\\|'):
%
\begin{center}
|... -jobname "|\textit{target}|" |\\|"|[\textit{flags}]%
|\input{childdoc.def}\childdocforward[|\textit{main}|]{|\textit{dest}|}"|
\end{center}
%
Here \textit{target} is the name of the output file,
\textit{main} is the name of the main file
and \textit{dest} is the name of the main or child file to be processed
(all filenames without extensions).
The optional argument \textit{main} can be omitted
if \textit{main} matches \textit{dest}.
Optionally, compilation \textit{flags} can be defined via |\def| commands.
This command line makes the \TeX{} engine believe
it is compiling the file \textit{target}
whose content is specified as the latter parameter.
The provided code then forwards the processing to
\textit{main} or \textit{dest} as described in \secref{sec:forward}.

%%%%%%%%%%%%%%%%%%%%%%%%%%%%%%%%%%%%%%%%%%%%%%%%%%%%%%%%%%%%%%%%%%%%%%%%%%%%%%%%
\subsection{Include by Input}
\label{sec:input}

Including child documents by |\include| has some restrictions by design.
Most notably, the content of a child document always occupies
its own set of pages; pages cannot be shared between child documents.
Usually, this behaviour makes perfect sense
because each child document contain an essential part of the document.
However, in some situations it may be desirable to compose
a document from a collection of parts
without having mandatory page breaks between then.
For this case, the package
provides a mechanism to include parts
by |\input| which can also be processed individually.
However, by construction this mechanism
requires manual handling of the content to be output.

%%%%%%%%%%%%%%%%%%%%%%%%%%%%%%%%%%%%%%%%
\DescribeMacro{\ifchilddocmanual}
The main file should be prepared as usual, see \secref{sec:include}.
However, the document body must make a distinction
between processing of an individual part and of the main document, e.g.:
%
\begin{center}
\begin{tabular}{l}
|\ifchilddocmanual|\\
|\input{\childdocname}|\\
|\||else|\\
\textit{document body with }|\input{|\textit{part}|}|\\
|\||fi|
\end{tabular}
\end{center}
%
The conditional |\ifchilddocmanual| is true whenever
a part to be included by |\input| is being compiled,
and the name of the part is stored in |\childdocname|.

%%%%%%%%%%%%%%%%%%%%%%%%%%%%%%%%%%%%%%%%
\DescribeMacro{\childdocby}
Each part to be included by |\input| should start with:
%
\begin{center}
\begin{tabular}{l}
|\input{childdoc.def}|\\
|\childdocby{|\textit{main}|}|\\
\end{tabular}
\end{center}
%
The directive |\childdocby| is similar to |\childdocof|
described in \secref{sec:include},
but the subsequent selection of content must be done manually.
To that end, both |\ifchilddoc| and |\ifchilddocmanual|
will be true upon processing of a part,
and the name of the part is stored in |\childdocname|.
Note that |\jobname| will be set to the filename of the current part
so that each part receives an individual |.aux| file
that does not interfere with the |.aux| file(s) of the main document.
This behaviour can be altered by the alternative form
|\childdocby[*]{|\textit{main}|}| (with a non-empty optional argument)
which uses the |.aux| file of the main document
by setting |\jobname| to \textit{main}.

%%%%%%%%%%%%%%%%%%%%%%%%%%%%%%%%%%%%%%%%%%%%%%%%%%%%%%%%%%%%%%%%%%%%%%%%%%%%%%%%
\subsection{Driver Development}
\label{sec:driver}

The \textsf{childdoc} mechanism can also be use for the development
of definition files such as \LaTeX{} styles or classes.
This case differs from the above setup with multiple parts
included by |\include| in that no |\includeonly| should be invoked.
This can be achieved by starting the include file
(before |\ProvidesPackage|) with:
%
\begin{center}
\begin{tabular}{l}
|\input{childdoc.def}|\\
|\childdocforward{|\textit{main}|}|\\
\end{tabular}
\end{center}
%
or alternatively with:
%
\begin{center}
\begin{tabular}{l}
|\input{childdoc.def}|\\
|\childdocby{|\textit{main}|}|\\
\end{tabular}
\end{center}
%
Both forms have slightly different effects as described above.
The main file is prepared as usual, see \secref{sec:include}.

%%%%%%%%%%%%%%%%%%%%%%%%%%%%%%%%%%%%%%%%%%%%%%%%%%%%%%%%%%%%%%%%%%%%%%%%%%%%%%%%
\subsection{Legacy Detection}
\label{sec:detection}

The directive |\childdocmain| in the main file can detect
whether the complete document or merely a child is to be compiled
even without using the directive |\childdocof|.
This method is deprecated because it is less robust
and there is no compelling reason to use it;
it is merely provided for backward compatibility
and it may be removed in future versions.

If the detection mechanism is to be used,
it is mandatory to correctly specify
the filename of the main file as the argument of |\childdocmain|:
%
\begin{center}
\begin{tabular}{l}
|\input{childdoc.def}|\\
|\childdocmain{|\textit{main}|}|\\
\end{tabular}
\end{center}
%
If |\jobname| does not match the argument \textit{main} of |\childdocmain|,
it is assumed that |\jobname| points to the child file to be compiled.
When using |\childdocmain| with the main file specified as argument,
it suffices to start a child file
with just |\input{|\textit{main}|}|
without loading of the package and using |\childdocof|.
If instead all processing is done
with the appropriate \textsf{childdoc} directives,
the argument of \textit{main} of |\childdocmain| can be empty.

An alternative version of the command line processing described
in \secref{sec:commandline} using the detection mechanism reads:
%
\begin{center}
|... -jobname "|\textit{target}|" "|[\textit{flags}]%
[|\def\jobname{|\textit{dest}|}|]|\input{|\textit{main}|}"|
\end{center}

%%%%%%%%%%%%%%%%%%%%%%%%%%%%%%%%%%%%%%%%%%%%%%%%%%%%%%%%%%%%%%%%%%%%%%%%%%%%%%%%
\subsection{Manual Code}
\label{sec:manual}

In case one cannot be certain whether the definitions file |childdoc.def|
is installed on the target \TeX{} distribution
and one prefers not to ship it,
it is conceivable to paste a few relevant commands into the sources.

To that end, drop all statements |\input{childdoc.def}|
and perform the replacements as outlined below.
Instead of |\childdocmain{|\textit{main}|}| add the following code
to the top of the main file:
%
\begin{center}
\begin{tabular}{l}
|\||ifdefined\childdocname\endinput\||fi\newif\ifchilddoc|\\
|\edef\childdocname{\scantokens\expandafter{\jobname\noexpand}}|\\
|\def\childdocmain{|\textit{main}|}\||ifx\childdocmain\childdocname\||else|\\
|\childdoctrue\includeonly{\childdocname}\let\jobname\childdocmain\||fi|\\
\end{tabular}
\end{center}
%
Instead of |\childdocof{|\textit{main}|}| just include the main file
at the top of each child file:
%
\begin{center}
|\input{|\textit{main}|}|
\end{center}
%
A simple redirection |\childdocforward{|\textit{dest}|}| is achieved by:
%
\begin{center}
|\def\jobname{|\textit{dest}|}\input{\jobname}|
\end{center}
%
The redirection with prefix
|\childdocforwardprefix[|\textit{prefix}|]{|\textit{dest}|}|
is accomplished by:
%
\begin{center}
\begin{tabular}{l}
|{\edef\jobname{\scantokens\expandafter{\jobname\noexpand}}|\\
|\def\redirectjob |\textit{prefix}|#1~~~{\gdef\jobname{|\textit{dest}|#1}}|\\
|\expandafter\redirectjob\jobname~~~}\input{\jobname}|
\end{tabular}
\end{center}

In an alternative approach,
child documents can be compiled by a specific command line
without additional code or specific definitions:
%
\begin{center}
|... -jobname "|\textit{target}|" "|[\textit{flags}]%
|\includeonly{|\textit{dest}|}\input{|\textit{main}|}"|
\end{center}
%

%%%%%%%%%%%%%%%%%%%%%%%%%%%%%%%%%%%%%%%%%%%%%%%%%%%%%%%%%%%%%%%%%%%%%%%%%%%%%%%%
%%%%%%%%%%%%%%%%%%%%%%%%%%%%%%%%%%%%%%%%%%%%%%%%%%%%%%%%%%%%%%%%%%%%%%%%%%%%%%%%
\section{Information}

%%%%%%%%%%%%%%%%%%%%%%%%%%%%%%%%%%%%%%%%%%%%%%%%%%%%%%%%%%%%%%%%%%%%%%%%%%%%%%%%
\subsection{Copyright}

Copyright \copyright{} 2017--2018 Niklas Beisert

This work may be distributed and/or modified under the
conditions of the \LaTeX{} Project Public License, either version 1.3
of this license or (at your option) any later version.
The latest version of this license is in
  \url{http://www.latex-project.org/lppl.txt}
and version 1.3 or later is part of all distributions of \LaTeX{}
version 2005/12/01 or later.

This work has the LPPL maintenance status `maintained'.

The Current Maintainer of this work is Niklas Beisert.

This work consists of the files |README.txt|, |childdoc.ins| and |childdoc.dtx|
as well as the derived files |childdoc.def|, |cdocsamp.tex|
with |cdocsch1.tex|, |cdocsch2.tex|, |cdocspt3.tex|, |cdocspt4.tex|,
|cdocsdrf.tex|, |cdocsfn1.tex|, |cdocsfn2.tex|
as well as |childdoc.pdf|.

%%%%%%%%%%%%%%%%%%%%%%%%%%%%%%%%%%%%%%%%%%%%%%%%%%%%%%%%%%%%%%%%%%%%%%%%%%%%%%%%
\subsection{Files and Installation}

The package consists of the files:
%
\begin{center}
\begin{tabular}{ll}
    |README.txt|   & readme file \\
    |childdoc.ins| & installation file \\
    |childdoc.dtx| & source file \\
    |childdoc.def| & definition file \\
    |cdocsamp.tex| & sample main file \\
    |cdocsch1.tex| & sample include file \\
    |cdocsch2.tex| & sample include file \\
    |cdocspt3.tex| & sample part file \\
    |cdocspt4.tex| & sample part file \\
    |cdocsdrf.tex| & sample redirection file \\
    |cdocsfn1.tex| & sample redirection file \\
    |cdocsfn2.tex| & sample redirection file \\
    |childdoc.pdf| & manual
\end{tabular}
\end{center}
%
The distribution consists of the files
|README.txt|, |childdoc.ins| and |childdoc.dtx|.
%
\begin{itemize}
\item
Run (pdf)\LaTeX{} on |childdoc.dtx|
to compile the manual |childdoc.pdf| (this file).
\item
Run \LaTeX{} on |childdoc.ins| to create the definitions file |childdoc.def|
and the sample |cdocsamp.tex| with include files
|cdocsch1.tex|, |cdocsch2.tex|, |cdocspt3.tex|, |cdocspt4.tex|,
|cdocsdrf.tex|, |cdocsfn1.tex|, |cdocsfn2.tex|.
Then copy the file |childdoc.def| to an appropriate directory of your \LaTeX{}
distribution, e.g.\ \textit{texmf-root}|/tex/latex/childdoc|.
\end{itemize}

%%%%%%%%%%%%%%%%%%%%%%%%%%%%%%%%%%%%%%%%%%%%%%%%%%%%%%%%%%%%%%%%%%%%%%%%%%%%%%%%
\subsection{Related CTAN Packages}

There are several other packages which offer a similar functionality:
%
\begin{itemize}
\item
The packages
\href{http://ctan.org/pkg/docmute}{\textsf{docmute}},
\href{http://ctan.org/pkg/includex}{\textsf{includex}} and
\href{http://ctan.org/pkg/standalone}{\textsf{standalone}}
provide commands to include only the document body of
a child file thus allowing both files to be compiled individually.
\item
The packages \href{http://ctan.org/pkg/subdocs}{\textsf{subdocs}}
and \href{http://ctan.org/pkg/subfiles}{\textsf{subfiles}}
provide structures in which the main and child documents can be
encapsulated and allowing them to be compiled individually.
The inclusion mechanism is different from the conventional |\include|.
\item
The package \href{http://ctan.org/pkg/combine}{\textsf{combine}}
is an elaborate solution to combine several documents into one.
\end{itemize}
%
See also the CTAN topic \href{http://ctan.org/topic/subdocs}{\textsf{subdocs}}
for further related packages.
The present package differs from the above solutions in that
a document structure constructed with the conventional |\include| mechanism
just needs two extra commands at the top of every file
such that all constituent files can be compiled individually.

%%%%%%%%%%%%%%%%%%%%%%%%%%%%%%%%%%%%%%%%%%%%%%%%%%%%%%%%%%%%%%%%%%%%%%%%%%%%%%%%
%\subsection{Feature Suggestions}
%
%The following is a list of features which may be useful for future
%versions of this package:
%%
%\begin{itemize}
%\item
%\ldots
%\end{itemize}

%%%%%%%%%%%%%%%%%%%%%%%%%%%%%%%%%%%%%%%%%%%%%%%%%%%%%%%%%%%%%%%%%%%%%%%%%%%%%%%%
\subsection{Revision History}

%%%%%%%%%%%%%%%%%%%%%%%%%%%%%%%%%%%%%%%%
\paragraph{v2.0:} 2018/12/30

\begin{itemize}
\item
immediate forward processing
\item
added |\childdocby| mechanism
\item
manual restructured
\end{itemize}

%%%%%%%%%%%%%%%%%%%%%%%%%%%%%%%%%%%%%%%%
\paragraph{v1.6:} 2018/01/17

\begin{itemize}
\item
application for development of include files
\item
corrections to manual
\end{itemize}

%%%%%%%%%%%%%%%%%%%%%%%%%%%%%%%%%%%%%%%%
\paragraph{v1.5:} 2017/05/21

\begin{itemize}
\item
more complete structuring introduced
\item
|\childdocof| introduced
\item
|\childdoc| renamed to |\childdocmain|
\item
|\childredirect| renamed to |\childdocforward| and |\childdocforwardprefix|
and functionality expanded
\end{itemize}

%%%%%%%%%%%%%%%%%%%%%%%%%%%%%%%%%%%%%%%%
\paragraph{v1.0:} 2017/04/27

\begin{itemize}
\item
manual and install package
\item
first version published on CTAN
\end{itemize}

%%%%%%%%%%%%%%%%%%%%%%%%%%%%%%%%%%%%%%%%
\paragraph{v0.6:} 2017/04/26

\begin{itemize}
\item
redirection mechanism added
\end{itemize}

%%%%%%%%%%%%%%%%%%%%%%%%%%%%%%%%%%%%%%%%
\paragraph{v0.5:} 2017/04/26

\begin{itemize}
\item
functionality in definition file
\end{itemize}


%%%%%%%%%%%%%%%%%%%%%%%%%%%%%%%%%%%%%%%%%%%%%%%%%%%%%%%%%%%%%%%%%%%%%%%%%%%%%%%%
%%%%%%%%%%%%%%%%%%%%%%%%%%%%%%%%%%%%%%%%%%%%%%%%%%%%%%%%%%%%%%%%%%%%%%%%%%%%%%%%
%%%%%%%%%%%%%%%%%%%%%%%%%%%%%%%%%%%%%%%%%%%%%%%%%%%%%%%%%%%%%%%%%%%%%%%%%%%%%%%%
\appendix

\settowidth\MacroIndent{\rmfamily\scriptsize 000\ }

 \DocInput{childdoc.dtx}

\end{document}
%</driver>
% \fi
%
% %%%%%%%%%%%%%%%%%%%%%%%%%%%%%%%%%%%%%%%%%%%%%%%%%%%%%%%%%%%%%%%%%%%%%%%%%%%%%%
% %%%%%%%%%%%%%%%%%%%%%%%%%%%%%%%%%%%%%%%%%%%%%%%%%%%%%%%%%%%%%%%%%%%%%%%%%%%%%%
% \section{Sample}
%\iffalse
%<*samplemain>
%\fi
%
% The following presents a sample document
% with two chapters, two parts, a title page,
% a compile flag as well as three forwarding files to set the flag.
% It consists of eight |.tex| files:
% \begin{center}
% \begin{tabular}{ll}
% |cdocsamp.tex|&main file\\
% |cdocsch1.tex|&include file for chapter 1\\
% |cdocsch2.tex|&include file for chapter 2\\
% |cdocspt3.tex|&include file for part 3\\
% |cdocspt4.tex|&include file for part 4\\
% |cdocsdrf.tex|&forwarding file for main file in draft mode\\
% |cdocsfi1.tex|&forwarding file for final version of chapter 1\\
% |cdocsfi2.tex|&forwarding file for final version of chapter 2\\
% \end{tabular}
% \end{center}
% Each of the eight files can be compiled directly by the \LaTeX{} compiler.
%
% %%%%%%%%%%%%%%%%%%%%%%%%%%%%%%%%%%%%%%
% \paragraph{Main File.}
%
% The main file is called |cdocsamp.tex|.
%
% Load the \textsf{childdoc} definitions and
% declare the filename for the main document:
%    \begin{macrocode}
\input{childdoc.def}
\childdocmain{}
%    \end{macrocode}

% Optional override for |\version| flag:
%    \begin{macrocode}
%%\ifchilddoc\else\providecommand{\version}{draft}\fi
%    \end{macrocode}

% Define the default values for the |\version| flag
% (|final| for the main file and |draft| for childs):
%    \begin{macrocode}
\ifchilddoc
\providecommand{\version}{draft}
\else
\providecommand{\version}{final}
\fi
%    \end{macrocode}

% Load the standard document class:
%    \begin{macrocode}
\documentclass[12pt]{article}
%    \end{macrocode}

% Start the document body:
%    \begin{macrocode}
\begin{document}
%    \end{macrocode}

% Declare a title page.
% Print title, part of document being processed and version flag:
%    \begin{macrocode}
\addtocounter{page}{-1}
\begin{center}
{\LARGE\bfseries{}childdoc example\par}
\vspace{1cm}
\ifchilddoc
\ifchilddocmanual part\else chapter\fi:
`\childdocname' of `\childdocjob'\par
\else
main document: `\childdocjob'\par
\fi
version: \version\par
\end{center}
\newpage
%    \end{macrocode}

% Manually include selected file,
% otherwise process as usual:
%    \begin{macrocode}
\ifchilddocmanual
\section*{part `\childdocname'}
\input{\childdocname}
\else
%    \end{macrocode}

% Include the two chapters:
%    \begin{macrocode}
\include{cdocsch1}
\include{cdocsch2}
%    \end{macrocode}

% Include the two parts unless only chapters should be displayed:
%    \begin{macrocode}
\ifchilddoc\else
\section{part three}
\input{cdocspt3}
\section{part four}
\input{cdocspt4}
\fi
%    \end{macrocode}

% Process as usual until here:
%    \begin{macrocode}
\fi
%    \end{macrocode}

% End of document body:
%    \begin{macrocode}
\end{document}
%    \end{macrocode}
%\iffalse
%</samplemain>
%\fi
%
% %%%%%%%%%%%%%%%%%%%%%%%%%%%%%%%%%%%%%%
% \paragraph{Chapter Include Files.}
%
% The include files are called |cdocsch1.tex| and |cdocsch2.tex|.
%
%\iffalse
%<*samplechap1|samplechap2>
%\fi

% Optional override for |\version| flag:
%    \begin{macrocode}
%%\providecommand{\version}{final}
%    \end{macrocode}

% Include the main document:
%    \begin{macrocode}
\input{childdoc.def}
\childdocof{cdocsamp}
%    \end{macrocode}

%\iffalse
%</samplechap1|samplechap2>
%\fi
%
%\iffalse
%<*samplechap1>
%\fi
% Some text for chapter 1:
%    \begin{macrocode}
\section{one}
some text in chapter one
%    \end{macrocode}

%\iffalse
%</samplechap1>
%\fi
% Some text for chapter 2:
%\iffalse
%<*samplechap2>
%\fi
%    \begin{macrocode}
\section{two}
more text in chapter two
%    \end{macrocode}

%\iffalse
%</samplechap2>
%\fi
%
% %%%%%%%%%%%%%%%%%%%%%%%%%%%%%%%%%%%%%%
% \paragraph{Part Include Files.}
%
% The include files are called |cdocspt3.tex| and |cdocspt4.tex|.
%
%\iffalse
%<*samplepart3|samplepart4>
%\fi

% Optional override for |\version| flag:
%    \begin{macrocode}
%%\providecommand{\version}{final}
%    \end{macrocode}

% Include the main document:
%    \begin{macrocode}
\input{childdoc.def}
\childdocby{cdocsamp}
%    \end{macrocode}

%\iffalse
%</samplepart3|samplepart4>
%\fi
%
%\iffalse
%<*samplepart3>
%\fi
% Some text for part 3:
%    \begin{macrocode}
some text in part three
%    \end{macrocode}

%\iffalse
%</samplepart3>
%\fi
% Some text for part 4:
%\iffalse
%<*samplepart4>
%\fi
%    \begin{macrocode}
more text in part four
%    \end{macrocode}

%\iffalse
%</samplepart4>
%\fi
%
% %%%%%%%%%%%%%%%%%%%%%%%%%%%%%%%%%%%%%%
% \paragraph{Forwarding for a Complete Draft.}
%
% The following forwarding file |cdocsdrf.tex|
% compiles the main document in draft mode:
%\iffalse
%<*sampledraft>
%\fi
%    \begin{macrocode}
\def\version{draft}
\input{childdoc.def}
\childdocforward{cdocsamp}
%    \end{macrocode}

%\iffalse
%</sampledraft>
%\fi
%
% %%%%%%%%%%%%%%%%%%%%%%%%%%%%%%%%%%%%%%
% \paragraph{Forwarding for Final Version of the Chapters.}
%
% The following forwarding files |cdocsfn1.tex| and |cdocsfn2.tex|
% (with identical content)
% compile the final versions of the child documents
% |cdocsch1.tex| and |cdocsch2.tex|, respectively:
%\iffalse
%<*samplefinal>
%\fi
%    \begin{macrocode}
\def\version{final}
\input{childdoc.def}
\childdocforwardprefix[cdocsamp]{cdocsfn}{cdocsch}
%    \end{macrocode}

%\iffalse
%</samplefinal>
%\fi
%
% %%%%%%%%%%%%%%%%%%%%%%%%%%%%%%%%%%%%%%
% \paragraph{Command Line Processing.}
%
% The following three command lines generate the output files
% |cdocscld|, |cdocscl1| and |cdocscl2|
% which should be identical to
% |cdocsdrf|, |cdocsch1| and |cdocsfn2|, respectively:
% \begin{center}
% \begin{tabular}{l}
% |latex -jobname cdocscld \|\\
% |  "\def\version{draft}\input{childdoc.def}\childdocforward{cdocsamp}"|\\
% |latex -jobname cdocscl1 \|\\
% |  "\input{childdoc.def}\childdocforward[cdocsamp]{cdocsch1}"|\\
% |latex -jobname cdocscl2 \|\\
% |  "\def\version{final}\input{childdoc.def}\childdocforward{cdocsch2}"|
% \end{tabular}
% \end{center}
% Note that the trailing backslash on each first line
% merely continues the input to the second line
% (for convenient cut ant paste).
% Furthermore, the command |latex| can be replaced by any
% of its alternative versions such as |pdflatex|.
%
% %%%%%%%%%%%%%%%%%%%%%%%%%%%%%%%%%%%%%%%%%%%%%%%%%%%%%%%%%%%%%%%%%%%%%%%%%%%%%%
% %%%%%%%%%%%%%%%%%%%%%%%%%%%%%%%%%%%%%%%%%%%%%%%%%%%%%%%%%%%%%%%%%%%%%%%%%%%%%%
% \section{Implementation}
%\iffalse
%<*package>
%\fi
%
% This section describes the definitions file |childdoc.def|.

% The definitions cannot be loaded using |\usepackage| or |\RequirePackage|
% which has a mechanism to prevent loading a style file more than once.
% When loading the definitions by means of |\input|
% multiple instances have to be prevented manually:
%\iffalse
%This code needs to be before the `\ProvidesFile' directive
%which is defined at the beginning of this file.
%Therefore it is also placed there and commented out here.
%</package>
%<*discard>
%\fi
%    \begin{macrocode}
\ifdefined\childdocmain\endinput\fi
%    \end{macrocode}
%\iffalse
%</discard>
%<*package>
%\fi
%
% \macro{\ifchilddoc}
% \macro{\ifchilddocmanual}
% The conditional |\ifchilddoc| tells whether a
% child (true) or main (false) document is being compiled.
% The conditional |\ifchilddocmanual| tells whether
% the |\includeonly| mechanism is used (false) or
% the selection of child files must be performed manually (true).
% The definitions initialise to false:
%    \begin{macrocode}
\newif\ifchilddoc
\newif\ifchilddocmanual
%    \end{macrocode}

% \macro{\childdocname}
% \macro{\childdocjob}
% The macro |\childdocname| stores the name of the main document
% to be compiled. The macro |\childdocjob| stores the name of
% the document on which the \LaTeX{} compiler was originally invoked.
% The content of |\jobname| cannot be compared
% to filenames specified in the source due to different catcodes.
% The following code rescans |\jobname|, stores the result
% in |\childdocname| and saves a copy in |\childdocjob|:
%    \begin{macrocode}
\edef\childdocname{\scantokens\expandafter{\jobname\noexpand}}
\let\childdocjob\childdocname
%    \end{macrocode}

% \macro{\childdocdisable}
% The macro |\childdocdisable| prevents the main file
% from being processed more than once.
% At this stage, the main document command |\childdocmain|
% is assumed to be called once again where it should do nothing.
% Any subsequent call to it should prevent
% a secondary processing of the main document
% It overwrites the forwarding commands
% |\childdocof| and |\childdocforward|
% with empty macros to prevent further inclusions of the main document:
%    \begin{macrocode}
\newcommand{\childdocdisable}
{
  \renewcommand{\childdocmain}[1]{\renewcommand{\childdocmain}[1]{\endinput}}
  \renewcommand{\childdocof}[1]{}
  \renewcommand{\childdocby}[2][]{}
  \renewcommand{\childdocforward}[2][]{}
  \renewcommand{\childdocdisable}{}
}
%    \end{macrocode}

% \macro{\childdocmain}
% The macro |\childdocmain| is to be called at the top of the main file
% with nothing or the main filename (without extension) as argument.
% First, it breaks loops.
% If the argument is not empty and does not match |\childdocname|
% (which is set by the first inclusion of |childdoc.def|),
% |\ifchilddoc| is set to true, |\includeonly| is applied to the child file
% and |\jobname| is set to the main file
% (for proper handling of |.aux| files):
%    \begin{macrocode}
\newcommand{\childdocmain}[1]
{
  \childdocdisable\childdocmain{}
  \if?#1?\else
    \begingroup
      \def\childdoctmp{#1}
      \ifx\childdoctmp\childdocname
        \def\childdoctmp{}
      \else
        \def\childdoctmp
        {
          \childdoctrue
          \includeonly{\childdocname}
          \def\childdocjob{#1}
          \def\jobname{#1}
        }
      \fi
      \expandafter
    \endgroup
    \childdoctmp
  \fi
}
%    \end{macrocode}

% \macro{\childdocof}
% The command |\childdocof| redirects
% compilation to the main file |#1|.
%    \begin{macrocode}
\newcommand{\childdocof}[1]
{
  \childdocdisable
  \childdoctrue
  \includeonly{\childdocname}
  \def\jobname{#1}
  \def\childdocjob{#1}
  \input{#1}
}
%    \end{macrocode}

% \macro{\childdocby}
% The command |\childdocby| ....
%    \begin{macrocode}
\newcommand{\childdocby}[2][]
{
  \childdocdisable
  \childdoctrue
  \childdocmanualtrue
  \if?#1?\else
    \def\jobname{#2}
  \fi
  \def\childdocjob{#2}
  \input{#2}
  \endinput
}
%    \end{macrocode}

% \macro{\childdocforward}
% The command |\childdocforward| redirects
% compilation to the main file or
% (if the optional argument is given) a child file.
% Parameters are set as if the main file
% or a child file starting with |\childdocof| was compiled.
% Then compilation is handed over to the main file:
%    \begin{macrocode}
\newcommand{\childdocforward}[2][]
{
  \begingroup
    \if?#1?
      \def\childdoctmp
      {
        \def\childdocname{#2}
        \def\childdocjob{#2}
        \def\jobname{#2}
        \input{#2}
        \endinput
      }
    \else
      \def\childdoctmp
      {
        \childdocdisable
        \def\childdocname{#2}
        \childdoctrue
        \includeonly{#2}
        \def\childdocjob{#1}
        \def\jobname{#1}
        \input{#1}
        \endinput
      }
    \fi
    \expandafter
  \endgroup
  \childdoctmp
}
%    \end{macrocode}

% \macro{\childdocforwardprefix}
% The command |\childdocforwardprefix| redirects
% compilation to the main or a child file by means of a pattern.
% The prefix |#1| in the current filename is replaced by |#2|
% and the suffix of the current filename is kept
% (it is assumed that the filename does not contain the substring `|~~~|'
% which is used as a delimiter).
% Compilation is handed over to the new file by |\childdocforward|:
%    \begin{macrocode}
\newcommand{\childdocforwardprefix}[3][]
{
  \begingroup
    \def\childdocextract #2##1~~~{\def\childdoctmp{\childdocforward[#1]{#3##1}}}
    \expandafter\childdocextract\childdocname~~~
    \expandafter
  \endgroup
  \childdoctmp
}
%    \end{macrocode}

% \macro{\childdoc}
% The deprecated macro |\childdoc| is a legacy version of |\childdocmain|:
%    \begin{macrocode}
\newcommand{\childdoc}{\childdocmain}
%    \end{macrocode}

% \macro{\childdocredirect}
% The deprecated macro |\childdocredirect| is a legacy version
% of |\childdocforward| and |\childdocforwardprefix|:
%    \begin{macrocode}
\newcommand{\childdocredirect}[2][]
{
  \begingroup
    \if?#1?
      \def\childdoctmp{\childdocforward{#2}}
    \else
      \def\childdoctmp{\childdocforwardprefix{#1}{#2}}
    \fi
    \expandafter
  \endgroup
  \childdoctmp
}
%    \end{macrocode}

%\iffalse
%</package>
%\fi
%
\endinput
\childdocforward[|\textit{main}|]{|\textit{dest}|}"|
\end{center}
%
Here \textit{target} is the name of the output file,
\textit{main} is the name of the main file
and \textit{dest} is the name of the main or child file to be processed
(all filenames without extensions).
The optional argument \textit{main} can be omitted
if \textit{main} matches \textit{dest}.
Optionally, compilation \textit{flags} can be defined via |\def| commands.
This command line makes the \TeX{} engine believe
it is compiling the file \textit{target}
whose content is specified as the latter parameter.
The provided code then forwards the processing to
\textit{main} or \textit{dest} as described in \secref{sec:forward}.

%%%%%%%%%%%%%%%%%%%%%%%%%%%%%%%%%%%%%%%%%%%%%%%%%%%%%%%%%%%%%%%%%%%%%%%%%%%%%%%%
\subsection{Include by Input}
\label{sec:input}

Including child documents by |\include| has some restrictions by design.
Most notably, the content of a child document always occupies
its own set of pages; pages cannot be shared between child documents.
Usually, this behaviour makes perfect sense
because each child document contain an essential part of the document.
However, in some situations it may be desirable to compose
a document from a collection of parts
without having mandatory page breaks between then.
For this case, the package
provides a mechanism to include parts
by |\input| which can also be processed individually.
However, by construction this mechanism
requires manual handling of the content to be output.

%%%%%%%%%%%%%%%%%%%%%%%%%%%%%%%%%%%%%%%%
\DescribeMacro{\ifchilddocmanual}
The main file should be prepared as usual, see \secref{sec:include}.
However, the document body must make a distinction
between processing of an individual part and of the main document, e.g.:
%
\begin{center}
\begin{tabular}{l}
|\ifchilddocmanual|\\
|\input{\childdocname}|\\
|\||else|\\
\textit{document body with }|\input{|\textit{part}|}|\\
|\||fi|
\end{tabular}
\end{center}
%
The conditional |\ifchilddocmanual| is true whenever
a part to be included by |\input| is being compiled,
and the name of the part is stored in |\childdocname|.

%%%%%%%%%%%%%%%%%%%%%%%%%%%%%%%%%%%%%%%%
\DescribeMacro{\childdocby}
Each part to be included by |\input| should start with:
%
\begin{center}
\begin{tabular}{l}
|% \iffalse
%
% childdoc.dtx Copyright (C) 2017-2018 Niklas Beisert
%
% This work may be distributed and/or modified under the
% conditions of the LaTeX Project Public License, either version 1.3
% of this license or (at your option) any later version.
% The latest version of this license is in
%   http://www.latex-project.org/lppl.txt
% and version 1.3 or later is part of all distributions of LaTeX
% version 2005/12/01 or later.
%
% This work has the LPPL maintenance status `maintained'.
%
% The Current Maintainer of this work is Niklas Beisert.
%
% This work consists of the files childdoc.dtx and childdoc.ins
% and the derived files childdoc.def and cdocsamp.tex with
% cdocsch1.tex, cdocsch2.tex, cdocsdrf.tex, cdocsfn1.tex, cdocsfn2.tex.
%
%<package>\ifdefined\childdocmain\endinput\fi
%<package>\ProvidesFile{childdoc.def}[2018/12/30 v2.0 child document driver]
%<samplemain>\ProvidesFile{cdocsamp.tex}[2018/12/30 v2.0 sample for childdoc]
%<*driver>
%\ProvidesFile{childdoc.drv}[2018/12/30 v2.0 childdoc reference manual file]
\PassOptionsToClass{10pt,a4paper}{article}
\documentclass{ltxdoc}

\usepackage[margin=35mm]{geometry}
\usepackage{hyperref}
\usepackage{hyperxmp}
\usepackage[usenames]{color}

\hypersetup{colorlinks=true}
\hypersetup{pdfstartview=FitH}
\hypersetup{pdfpagemode=UseNone}
\hypersetup{pdfsource={}}
\hypersetup{pdflang={en-UK}}
\hypersetup{pdfcopyright={Copyright 2017-2018 Niklas Beisert.
  This work may be distributed and/or modified under the
  conditions of the LaTeX Project Public License, either version 1.3
  of this license or (at your option) any later version.}}
\hypersetup{pdflicenseurl={http://www.latex-project.org/lppl.txt}}
\hypersetup{pdfcontactaddress={ETH Zurich, ITP, HIT K,
  Wolfgang-Pauli-Strasse 27}}
\hypersetup{pdfcontactpostcode={8093}}
\hypersetup{pdfcontactcity={Zurich}}
\hypersetup{pdfcontactcountry={Switzerland}}
\hypersetup{pdfcontactemail={nbeisert@itp.phys.ethz.ch}}
\hypersetup{pdfcontacturl={http://people.phys.ethz.ch/\xmptilde nbeisert/}}

\newcommand{\secref}[1]{\hyperref[#1]{section \ref*{#1}}}

\parskip1ex
\parindent0pt
\let\olditemize\itemize
\def\itemize{\olditemize\parskip0pt}

\begin{document}

\title{The \textsf{childdoc} Package}
\hypersetup{pdftitle={The childdoc Package}}
\author{Niklas Beisert\\[2ex]
  Institut f\"ur Theoretische Physik\\
  Eidgen\"ossische Technische Hochschule Z\"urich\\
  Wolfgang-Pauli-Strasse 27, 8093 Z\"urich, Switzerland\\[1ex]
  \href{mailto:nbeisert@itp.phys.ethz.ch}
  {\texttt{nbeisert@itp.phys.ethz.ch}}}
\hypersetup{pdfauthor={Niklas Beisert}}
\hypersetup{pdfsubject={Manual for the LaTeX2e Package childdoc}}
\date{30 December 2018, \textsf{v2.0}}
\maketitle

\begin{abstract}\noindent
\textsf{childdoc} is a \LaTeXe{} package
that enables the direct compilation
of document sections included by |\include|
to individual files.
\end{abstract}

\begingroup
\parskip0ex
\tableofcontents
\endgroup

%%%%%%%%%%%%%%%%%%%%%%%%%%%%%%%%%%%%%%%%%%%%%%%%%%%%%%%%%%%%%%%%%%%%%%%%%%%%%%%%
%%%%%%%%%%%%%%%%%%%%%%%%%%%%%%%%%%%%%%%%%%%%%%%%%%%%%%%%%%%%%%%%%%%%%%%%%%%%%%%%
\section{Introduction}

\LaTeX{} provides a mechanism to structure a large document (such as a book)
into a main file and several child files (containing the chapters)
using the |\include| command.
This mechanism is beneficial for documents
which span hundreds of pages in order to
make the source file(s) more manageable.
Moreover, compilation can be restricted to
selected child files by means of the |\includeonly| command.
The latter feature can be used to reduce the compilation time while editing
(this was significantly more useful in the earlier days of \LaTeX{})
or to generate a smaller document which is easier to navigate.
Another application of |\includeonly| is to generate
documents consisting of selected parts of the complete document.

However, there are a few drawbacks of the plain |\include| mechanism:
\begin{itemize}
\item
The child files cannot be compiled on their own,
they can only be compiled via the main file.
A naive editing environment
(such as a text editor with an option
to have the current file processed by \LaTeX)
may require one to switch to the main file before compiling;
attempting to compile the child file produces errors.
\item
The main file must be modified (each time)
to adjust the |\includeonly| command
to the present needs. This easily leaves the main file in a messy state.
\item
The generated document will always carry the filename
of the main document. This is inconvenient if
several child files are to be compiled and
to be kept for distribution.
\end{itemize}

The present package provides a simple interface
to make child files individually compilable by \LaTeX{}.
Compiling a child file then has the same effect as compiling
the main file with an |\includeonly| command
to select the appropriate child.
Moreover the generated document will carry the name of the child
rather than the main file.
This resolves all three above issues.

This feature is meant to make the editing of books,
thesis documents and lecture notes somewhat more convenient.
However, the package can also be used efficiently for
composing a series of documents (such as exercise sheets)
which are typically distributed individually.
It then assists the author in generating the individual documents
(potentially in different versions)
as well as a document containing the collected series.
Another application is in developing style files
or other kinds of included material
where compilation of the style file could redirect
to a sample or test file.

%%%%%%%%%%%%%%%%%%%%%%%%%%%%%%%%%%%%%%%%%%%%%%%%%%%%%%%%%%%%%%%%%%%%%%%%%%%%%%%%
%%%%%%%%%%%%%%%%%%%%%%%%%%%%%%%%%%%%%%%%%%%%%%%%%%%%%%%%%%%%%%%%%%%%%%%%%%%%%%%%
\section{Usage}

First of all, the package \textsf{childdoc} is \emph{not} a standard
\LaTeXe{} |.sty| style file! Therefore it needs to be invoked in
a non-standard way.

%%%%%%%%%%%%%%%%%%%%%%%%%%%%%%%%%%%%%%%%%%%%%%%%%%%%%%%%%%%%%%%%%%%%%%%%%%%%%%%%
\subsection{Included Files}
\label{sec:include}

%%%%%%%%%%%%%%%%%%%%%%%%%%%%%%%%%%%%%%%%
\DescribeMacro{\childdocmain}
To use the package, add the commands
\begin{center}
\begin{tabular}{l}
|\input{childdoc.def}|\\
|\childdocmain{}|\\
\end{tabular}
\end{center}
at the very top of the main \LaTeX{} file,
in particular \emph{before} the |\documentclass| statement!
The argument of |\childdocmain| should be left empty
(but it must be present).

%%%%%%%%%%%%%%%%%%%%%%%%%%%%%%%%%%%%%%%%
\DescribeMacro{\childdocof}
Furthermore, add the commands
\begin{center}
\begin{tabular}{l}
|\input{childdoc.def}|\\
|\childdocof{|\textit{main}|}|\\
\end{tabular}
\end{center}
at the top of every child file \textit{child}
which is included by |\include{|\textit{child}|}|
from within the main file
(or at least for those files to be compiled individually).
The argument \textit{main} must be the filename of the main file.

There are a couple of
considerations in setting up the main and child documents:

%%%%%%%%%%%%%%%%%%%%%%%%%%%%%%%%%%%%%%%%
\paragraph{Restrictions.}

Please note the following restrictions:
\begin{itemize}
\item
|\childdocmain| must be called with one argument \textit{main}
to ensure compatibility with earlier version of the package.
It must either be empty (|\childdocmain{}|)
or precisely match the filename of the main file in which it is specified.
See \secref{sec:detection} for further information.
\item
The filename \textit{main} must be specified without the |.tex| extension.
\item
The filename \textit{main} is case sensitive
(even in case-insensitive file systems)
due to internal string comparison.
\item
The argument \textit{main} should be fully expanded, it cannot be a macro.
\item
Subdirectories and special characters should be avoided in filenames.
\item
The command |\childdocmain{|\textit{main}|}| must be followed by a whitespace.
It should not be followed immediately by another command
or by a comment mark `|%|'.
This is because the \TeX{} parser reads the token immediately following
the argument of |\childdocmain| and puts it
at the beginning of every child section;
however, a white\-space is ignored.
\end{itemize}

%%%%%%%%%%%%%%%%%%%%%%%%%%%%%%%%%%%%%%%%
\paragraph{Content of Main File.}

It is advisable to place all content in the child files included by |\include|.
Any output contained in the main file will appear in all child documents
unless suppressed manually;
it cannot be suppressed automatically by the |\includeonly| directive
and thus should normally be avoided.
A method to include some content in the main file
by means of conditional processing is described in \secref{sec:conditional}.

%%%%%%%%%%%%%%%%%%%%%%%%%%%%%%%%%%%%%%%%
\paragraph{Page Numbering.}

When only a part of the document is compiled,
the appropriate numbering of pages
(as well as other status parameters)
is determined from the |.aux| files.
The latter contain information from previous passes.
However this information needs to propagate through
all intermediate child documents.
Therefore the page numbering in child documents may well
be inconsistent until the complete document is compiled at least once.

A useful (if unconventional) way to always ensure a consistent
page numbering is to restart the numbering in each child document
and denote the pages by `\textit{child}|.|\textit{page}'
where \textit{child} represents the chapter/section number of the child file.
This can be achieved by the command
|\numberwithin{page}{|\textit{child}|}|
of the \textsf{amsmath} package
where \textit{child} can be |chapter| or |section|
depending on the chosen structuring.
Alternatively, one can modify the macro |\thepage| appropriately
and reset the counter |page| at the start of each child file.

%%%%%%%%%%%%%%%%%%%%%%%%%%%%%%%%%%%%%%%%%%%%%%%%%%%%%%%%%%%%%%%%%%%%%%%%%%%%%%%%
\subsection{Conditional Processing}
\label{sec:conditional}

The package provides a mechanism to compile different versions
of a document. To customise the versions further some conditional processing
can come in handy to distinguish which version is being compiled.
The package provides two macros to describe the compilation context:

%%%%%%%%%%%%%%%%%%%%%%%%%%%%%%%%%%%%%%%%
\DescribeMacro{\ifchilddoc}
The conditional |\ifchilddoc| distinguishes between the compilation of
child documents and the main document:
%
\begin{center}
|\ifchilddoc |\textit{child-code}| |[|\||else |\textit{main-code}]| \||fi|
\end{center}

%%%%%%%%%%%%%%%%%%%%%%%%%%%%%%%%%%%%%%%%
\DescribeMacro{\childdocname}
\DescribeMacro{\childdocjob}
The macro |\childdocname| contains the filename (without extension)
of the main or child file being processed.
Note that |\childdocjob| will always contain the name of the main file.

%%%%%%%%%%%%%%%%%%%%%%%%%%%%%%%%%%%%%%%%
\paragraph{Title Page.}

Conditional processing can be used to include a title or banner page
in the main document when proper precautions are taken.
Importantly, the code in the main file should ensure that the page counter
(as well as other status parameters which are stored in the |.aux| files)
takes the same value after the conditional processing.
Otherwise the page numbers may take divergent values
depending on which part is compiled.

For example, a title page could be declared by:
%
\begin{center}
\begin{tabular}{l}
|\ifchilddoc\||else|\\
|\addtocounter{page}{-1}|\\
\textit{code for title page}\\
|\newpage|\\
|\||fi|
\end{tabular}
\end{center}
%
A banner page for the child documents can be generated by:
%
\begin{center}
\begin{tabular}{l}
|\ifchilddoc|\\
|\addtocounter{page}{-1}|\\
\textit{code for banner page}\\
|\newpage|\\
|\||fi|
\end{tabular}
\end{center}
%
Here one could write a message such as:
\begin{center}
|This is the part \childdocname{} of \childdocjob{}.|
\end{center}

%%%%%%%%%%%%%%%%%%%%%%%%%%%%%%%%%%%%%%%%%%%%%%%%%%%%%%%%%%%%%%%%%%%%%%%%%%%%%%%%
\subsection{Flags}
\label{sec:flags}

The package makes it easy to generate different versions
of the main or child documents.
To this end compilation flags can be defined
and assigned different default values.
They will be particularly useful in conjunction
with the forwarding mechanism described in \secref{sec:forward}.

For example, it may be useful to have a flag |\version|
which can be set to |draft| or |final|.
The document source will contain some conditional code
depending on the value of |\version|.
Suppose further, the flag should default to |final| for the main file
and to |draft| for child files
which is a natural assignment for editing the document.
This is achieved by placing the following code
in the preamble of the main document
(below the |\childdocmain| directive):
%
\begin{center}
\begin{tabular}{l}
|\ifchilddoc|\\
|\providecommand{\version}{draft}|\\
|\||else|\\
|\providecommand{\version}{final}|\\
|\||fi|
\end{tabular}
\end{center}
%
The definition by |\providecommand| makes sure
that previous definitions are not overwritten.
Further statements |\providecommand{\version}{...}|
can thus be added before the above code to override it.

For the main file, one might add a line
(between |\childdocmain| and the above block)
%
\begin{center}
|%\ifchilddoc\||else\providecommand{\version}{draft}\||fi|
\end{center}
%
which can be uncommented to produce a draft version.
Likewise one can add a line to the very top of a child file
(above the |\childdocof{|\textit{main}|}| directive)
%
\begin{center}
|%\providecommand{\version}{final}|
\end{center}
%
which can be uncommented to produce the final version of this child document.

%%%%%%%%%%%%%%%%%%%%%%%%%%%%%%%%%%%%%%%%%%%%%%%%%%%%%%%%%%%%%%%%%%%%%%%%%%%%%%%%
\subsection{Forwarding}
\label{sec:forward}

Different versions of the main or child documents
using compilation flags as described in \secref{sec:flags}
can be (permanently) stored in different files
for convenient compilation, viewing and distribution.
To this end, the package defines a command
to pass on compilation to a different file:

%%%%%%%%%%%%%%%%%%%%%%%%%%%%%%%%%%%%%%%%
\DescribeMacro{\childdocforward}
The command |\childdocforward| redirects processing to
another source file:
%
\begin{center}
\begin{tabular}{l}
|\input{childdoc.def}|\\
|\childdocforward[|\textit{main}|]{|\textit{dest}|}|\\
\end{tabular}
\end{center}
%
The argument \textit{dest} is the destination file
(without extension).
It should be the main file or one of the child files.
Note that further \textsf{childdoc} directives
such as |\childdocof| and |\childdocforward|
in the indicated file will be processed in this form.
The optional argument \textit{main}
passes on directly to the main file \textit{main}
while pretending to compile the child \textit{dest}.
This form behaves as if \textit{dest}
issues |\childdocof{|\textit{main}|}| right away,
and no further \textsf{childdoc} directives will be processed.

%%%%%%%%%%%%%%%%%%%%%%%%%%%%%%%%%%%%%%%%
\DescribeMacro{\...prefix}
In the alternative form |\childdocforwardprefix|,
%
\begin{center}
\begin{tabular}{l}
|\input{childdoc.def}|\\
|\childdocforwardprefix[|\textit{main}|]{|\textit{prefix}|}{|\textit{dest}|}|
\end{tabular}
\end{center}
%
the destination file is determined by a pattern
depending on the current file:
To make this work, the current file must be called
`{\textit{prefix}\hspace{0.2em}\textit{suffix}}'
with \textit{prefix} matching precisely the argument.
Processing is then passed on to the file
`{\textit{dest}\hspace{0.2em}\textit{suffix}}'.
Surely, the same effect is achieved by
directly specifying the
argument `{\textit{dest}\hspace{0.2em}\textit{suffix}}'
in the first form.
However, that requires to set up a different file
for each child. With the alternative form of the command
all these files can have exactly the same content
which simplifies setting them up and maintaining them.

For example, the following file |draft.tex|
with a compilation flag |\version| as described in \secref{sec:flags}
compiles the main document as a draft:
%
\begin{center}
\begin{tabular}{l}
|\def\version{draft}|\\
|\input{childdoc.def}|\\
|\childdocforward{|\textit{main}|}|
\end{tabular}
\end{center}
%
Likewise, the following files |final|\textit{nn}|.tex|
compile the final version of the child document
|child|\textit{nn}|.tex|:
%
\begin{center}
\begin{tabular}{l}
|\def\version{final}|\\
|\input{childdoc.def}|\\
|\childdocforwardprefix{final}{child}|
\end{tabular}
\end{center}
%

Note that when several versions of a main file and/or of each child file
are to be generated, it may be convenient to set up a |Makefile| or
shell script to automatise the process.

%%%%%%%%%%%%%%%%%%%%%%%%%%%%%%%%%%%%%%%%%%%%%%%%%%%%%%%%%%%%%%%%%%%%%%%%%%%%%%%%
\subsection{Command Line Processing}
\label{sec:commandline}

The effect of redirection files can also be achieved by invoking
the \LaTeX{} compiler with a more elaborate command line.
Most conveniently this should be done as part
of a shell script or a |Makefile|.

When using \textsf{childdoc} in the main file, the following
command lines effectively perform a redirection
(note that depending on the shell being used,
backslashes may have to be doubled: `|\|' $\to$ `|\\|'):
%
\begin{center}
|... -jobname "|\textit{target}|" |\\|"|[\textit{flags}]%
|\input{childdoc.def}\childdocforward[|\textit{main}|]{|\textit{dest}|}"|
\end{center}
%
Here \textit{target} is the name of the output file,
\textit{main} is the name of the main file
and \textit{dest} is the name of the main or child file to be processed
(all filenames without extensions).
The optional argument \textit{main} can be omitted
if \textit{main} matches \textit{dest}.
Optionally, compilation \textit{flags} can be defined via |\def| commands.
This command line makes the \TeX{} engine believe
it is compiling the file \textit{target}
whose content is specified as the latter parameter.
The provided code then forwards the processing to
\textit{main} or \textit{dest} as described in \secref{sec:forward}.

%%%%%%%%%%%%%%%%%%%%%%%%%%%%%%%%%%%%%%%%%%%%%%%%%%%%%%%%%%%%%%%%%%%%%%%%%%%%%%%%
\subsection{Include by Input}
\label{sec:input}

Including child documents by |\include| has some restrictions by design.
Most notably, the content of a child document always occupies
its own set of pages; pages cannot be shared between child documents.
Usually, this behaviour makes perfect sense
because each child document contain an essential part of the document.
However, in some situations it may be desirable to compose
a document from a collection of parts
without having mandatory page breaks between then.
For this case, the package
provides a mechanism to include parts
by |\input| which can also be processed individually.
However, by construction this mechanism
requires manual handling of the content to be output.

%%%%%%%%%%%%%%%%%%%%%%%%%%%%%%%%%%%%%%%%
\DescribeMacro{\ifchilddocmanual}
The main file should be prepared as usual, see \secref{sec:include}.
However, the document body must make a distinction
between processing of an individual part and of the main document, e.g.:
%
\begin{center}
\begin{tabular}{l}
|\ifchilddocmanual|\\
|\input{\childdocname}|\\
|\||else|\\
\textit{document body with }|\input{|\textit{part}|}|\\
|\||fi|
\end{tabular}
\end{center}
%
The conditional |\ifchilddocmanual| is true whenever
a part to be included by |\input| is being compiled,
and the name of the part is stored in |\childdocname|.

%%%%%%%%%%%%%%%%%%%%%%%%%%%%%%%%%%%%%%%%
\DescribeMacro{\childdocby}
Each part to be included by |\input| should start with:
%
\begin{center}
\begin{tabular}{l}
|\input{childdoc.def}|\\
|\childdocby{|\textit{main}|}|\\
\end{tabular}
\end{center}
%
The directive |\childdocby| is similar to |\childdocof|
described in \secref{sec:include},
but the subsequent selection of content must be done manually.
To that end, both |\ifchilddoc| and |\ifchilddocmanual|
will be true upon processing of a part,
and the name of the part is stored in |\childdocname|.
Note that |\jobname| will be set to the filename of the current part
so that each part receives an individual |.aux| file
that does not interfere with the |.aux| file(s) of the main document.
This behaviour can be altered by the alternative form
|\childdocby[*]{|\textit{main}|}| (with a non-empty optional argument)
which uses the |.aux| file of the main document
by setting |\jobname| to \textit{main}.

%%%%%%%%%%%%%%%%%%%%%%%%%%%%%%%%%%%%%%%%%%%%%%%%%%%%%%%%%%%%%%%%%%%%%%%%%%%%%%%%
\subsection{Driver Development}
\label{sec:driver}

The \textsf{childdoc} mechanism can also be use for the development
of definition files such as \LaTeX{} styles or classes.
This case differs from the above setup with multiple parts
included by |\include| in that no |\includeonly| should be invoked.
This can be achieved by starting the include file
(before |\ProvidesPackage|) with:
%
\begin{center}
\begin{tabular}{l}
|\input{childdoc.def}|\\
|\childdocforward{|\textit{main}|}|\\
\end{tabular}
\end{center}
%
or alternatively with:
%
\begin{center}
\begin{tabular}{l}
|\input{childdoc.def}|\\
|\childdocby{|\textit{main}|}|\\
\end{tabular}
\end{center}
%
Both forms have slightly different effects as described above.
The main file is prepared as usual, see \secref{sec:include}.

%%%%%%%%%%%%%%%%%%%%%%%%%%%%%%%%%%%%%%%%%%%%%%%%%%%%%%%%%%%%%%%%%%%%%%%%%%%%%%%%
\subsection{Legacy Detection}
\label{sec:detection}

The directive |\childdocmain| in the main file can detect
whether the complete document or merely a child is to be compiled
even without using the directive |\childdocof|.
This method is deprecated because it is less robust
and there is no compelling reason to use it;
it is merely provided for backward compatibility
and it may be removed in future versions.

If the detection mechanism is to be used,
it is mandatory to correctly specify
the filename of the main file as the argument of |\childdocmain|:
%
\begin{center}
\begin{tabular}{l}
|\input{childdoc.def}|\\
|\childdocmain{|\textit{main}|}|\\
\end{tabular}
\end{center}
%
If |\jobname| does not match the argument \textit{main} of |\childdocmain|,
it is assumed that |\jobname| points to the child file to be compiled.
When using |\childdocmain| with the main file specified as argument,
it suffices to start a child file
with just |\input{|\textit{main}|}|
without loading of the package and using |\childdocof|.
If instead all processing is done
with the appropriate \textsf{childdoc} directives,
the argument of \textit{main} of |\childdocmain| can be empty.

An alternative version of the command line processing described
in \secref{sec:commandline} using the detection mechanism reads:
%
\begin{center}
|... -jobname "|\textit{target}|" "|[\textit{flags}]%
[|\def\jobname{|\textit{dest}|}|]|\input{|\textit{main}|}"|
\end{center}

%%%%%%%%%%%%%%%%%%%%%%%%%%%%%%%%%%%%%%%%%%%%%%%%%%%%%%%%%%%%%%%%%%%%%%%%%%%%%%%%
\subsection{Manual Code}
\label{sec:manual}

In case one cannot be certain whether the definitions file |childdoc.def|
is installed on the target \TeX{} distribution
and one prefers not to ship it,
it is conceivable to paste a few relevant commands into the sources.

To that end, drop all statements |\input{childdoc.def}|
and perform the replacements as outlined below.
Instead of |\childdocmain{|\textit{main}|}| add the following code
to the top of the main file:
%
\begin{center}
\begin{tabular}{l}
|\||ifdefined\childdocname\endinput\||fi\newif\ifchilddoc|\\
|\edef\childdocname{\scantokens\expandafter{\jobname\noexpand}}|\\
|\def\childdocmain{|\textit{main}|}\||ifx\childdocmain\childdocname\||else|\\
|\childdoctrue\includeonly{\childdocname}\let\jobname\childdocmain\||fi|\\
\end{tabular}
\end{center}
%
Instead of |\childdocof{|\textit{main}|}| just include the main file
at the top of each child file:
%
\begin{center}
|\input{|\textit{main}|}|
\end{center}
%
A simple redirection |\childdocforward{|\textit{dest}|}| is achieved by:
%
\begin{center}
|\def\jobname{|\textit{dest}|}\input{\jobname}|
\end{center}
%
The redirection with prefix
|\childdocforwardprefix[|\textit{prefix}|]{|\textit{dest}|}|
is accomplished by:
%
\begin{center}
\begin{tabular}{l}
|{\edef\jobname{\scantokens\expandafter{\jobname\noexpand}}|\\
|\def\redirectjob |\textit{prefix}|#1~~~{\gdef\jobname{|\textit{dest}|#1}}|\\
|\expandafter\redirectjob\jobname~~~}\input{\jobname}|
\end{tabular}
\end{center}

In an alternative approach,
child documents can be compiled by a specific command line
without additional code or specific definitions:
%
\begin{center}
|... -jobname "|\textit{target}|" "|[\textit{flags}]%
|\includeonly{|\textit{dest}|}\input{|\textit{main}|}"|
\end{center}
%

%%%%%%%%%%%%%%%%%%%%%%%%%%%%%%%%%%%%%%%%%%%%%%%%%%%%%%%%%%%%%%%%%%%%%%%%%%%%%%%%
%%%%%%%%%%%%%%%%%%%%%%%%%%%%%%%%%%%%%%%%%%%%%%%%%%%%%%%%%%%%%%%%%%%%%%%%%%%%%%%%
\section{Information}

%%%%%%%%%%%%%%%%%%%%%%%%%%%%%%%%%%%%%%%%%%%%%%%%%%%%%%%%%%%%%%%%%%%%%%%%%%%%%%%%
\subsection{Copyright}

Copyright \copyright{} 2017--2018 Niklas Beisert

This work may be distributed and/or modified under the
conditions of the \LaTeX{} Project Public License, either version 1.3
of this license or (at your option) any later version.
The latest version of this license is in
  \url{http://www.latex-project.org/lppl.txt}
and version 1.3 or later is part of all distributions of \LaTeX{}
version 2005/12/01 or later.

This work has the LPPL maintenance status `maintained'.

The Current Maintainer of this work is Niklas Beisert.

This work consists of the files |README.txt|, |childdoc.ins| and |childdoc.dtx|
as well as the derived files |childdoc.def|, |cdocsamp.tex|
with |cdocsch1.tex|, |cdocsch2.tex|, |cdocspt3.tex|, |cdocspt4.tex|,
|cdocsdrf.tex|, |cdocsfn1.tex|, |cdocsfn2.tex|
as well as |childdoc.pdf|.

%%%%%%%%%%%%%%%%%%%%%%%%%%%%%%%%%%%%%%%%%%%%%%%%%%%%%%%%%%%%%%%%%%%%%%%%%%%%%%%%
\subsection{Files and Installation}

The package consists of the files:
%
\begin{center}
\begin{tabular}{ll}
    |README.txt|   & readme file \\
    |childdoc.ins| & installation file \\
    |childdoc.dtx| & source file \\
    |childdoc.def| & definition file \\
    |cdocsamp.tex| & sample main file \\
    |cdocsch1.tex| & sample include file \\
    |cdocsch2.tex| & sample include file \\
    |cdocspt3.tex| & sample part file \\
    |cdocspt4.tex| & sample part file \\
    |cdocsdrf.tex| & sample redirection file \\
    |cdocsfn1.tex| & sample redirection file \\
    |cdocsfn2.tex| & sample redirection file \\
    |childdoc.pdf| & manual
\end{tabular}
\end{center}
%
The distribution consists of the files
|README.txt|, |childdoc.ins| and |childdoc.dtx|.
%
\begin{itemize}
\item
Run (pdf)\LaTeX{} on |childdoc.dtx|
to compile the manual |childdoc.pdf| (this file).
\item
Run \LaTeX{} on |childdoc.ins| to create the definitions file |childdoc.def|
and the sample |cdocsamp.tex| with include files
|cdocsch1.tex|, |cdocsch2.tex|, |cdocspt3.tex|, |cdocspt4.tex|,
|cdocsdrf.tex|, |cdocsfn1.tex|, |cdocsfn2.tex|.
Then copy the file |childdoc.def| to an appropriate directory of your \LaTeX{}
distribution, e.g.\ \textit{texmf-root}|/tex/latex/childdoc|.
\end{itemize}

%%%%%%%%%%%%%%%%%%%%%%%%%%%%%%%%%%%%%%%%%%%%%%%%%%%%%%%%%%%%%%%%%%%%%%%%%%%%%%%%
\subsection{Related CTAN Packages}

There are several other packages which offer a similar functionality:
%
\begin{itemize}
\item
The packages
\href{http://ctan.org/pkg/docmute}{\textsf{docmute}},
\href{http://ctan.org/pkg/includex}{\textsf{includex}} and
\href{http://ctan.org/pkg/standalone}{\textsf{standalone}}
provide commands to include only the document body of
a child file thus allowing both files to be compiled individually.
\item
The packages \href{http://ctan.org/pkg/subdocs}{\textsf{subdocs}}
and \href{http://ctan.org/pkg/subfiles}{\textsf{subfiles}}
provide structures in which the main and child documents can be
encapsulated and allowing them to be compiled individually.
The inclusion mechanism is different from the conventional |\include|.
\item
The package \href{http://ctan.org/pkg/combine}{\textsf{combine}}
is an elaborate solution to combine several documents into one.
\end{itemize}
%
See also the CTAN topic \href{http://ctan.org/topic/subdocs}{\textsf{subdocs}}
for further related packages.
The present package differs from the above solutions in that
a document structure constructed with the conventional |\include| mechanism
just needs two extra commands at the top of every file
such that all constituent files can be compiled individually.

%%%%%%%%%%%%%%%%%%%%%%%%%%%%%%%%%%%%%%%%%%%%%%%%%%%%%%%%%%%%%%%%%%%%%%%%%%%%%%%%
%\subsection{Feature Suggestions}
%
%The following is a list of features which may be useful for future
%versions of this package:
%%
%\begin{itemize}
%\item
%\ldots
%\end{itemize}

%%%%%%%%%%%%%%%%%%%%%%%%%%%%%%%%%%%%%%%%%%%%%%%%%%%%%%%%%%%%%%%%%%%%%%%%%%%%%%%%
\subsection{Revision History}

%%%%%%%%%%%%%%%%%%%%%%%%%%%%%%%%%%%%%%%%
\paragraph{v2.0:} 2018/12/30

\begin{itemize}
\item
immediate forward processing
\item
added |\childdocby| mechanism
\item
manual restructured
\end{itemize}

%%%%%%%%%%%%%%%%%%%%%%%%%%%%%%%%%%%%%%%%
\paragraph{v1.6:} 2018/01/17

\begin{itemize}
\item
application for development of include files
\item
corrections to manual
\end{itemize}

%%%%%%%%%%%%%%%%%%%%%%%%%%%%%%%%%%%%%%%%
\paragraph{v1.5:} 2017/05/21

\begin{itemize}
\item
more complete structuring introduced
\item
|\childdocof| introduced
\item
|\childdoc| renamed to |\childdocmain|
\item
|\childredirect| renamed to |\childdocforward| and |\childdocforwardprefix|
and functionality expanded
\end{itemize}

%%%%%%%%%%%%%%%%%%%%%%%%%%%%%%%%%%%%%%%%
\paragraph{v1.0:} 2017/04/27

\begin{itemize}
\item
manual and install package
\item
first version published on CTAN
\end{itemize}

%%%%%%%%%%%%%%%%%%%%%%%%%%%%%%%%%%%%%%%%
\paragraph{v0.6:} 2017/04/26

\begin{itemize}
\item
redirection mechanism added
\end{itemize}

%%%%%%%%%%%%%%%%%%%%%%%%%%%%%%%%%%%%%%%%
\paragraph{v0.5:} 2017/04/26

\begin{itemize}
\item
functionality in definition file
\end{itemize}


%%%%%%%%%%%%%%%%%%%%%%%%%%%%%%%%%%%%%%%%%%%%%%%%%%%%%%%%%%%%%%%%%%%%%%%%%%%%%%%%
%%%%%%%%%%%%%%%%%%%%%%%%%%%%%%%%%%%%%%%%%%%%%%%%%%%%%%%%%%%%%%%%%%%%%%%%%%%%%%%%
%%%%%%%%%%%%%%%%%%%%%%%%%%%%%%%%%%%%%%%%%%%%%%%%%%%%%%%%%%%%%%%%%%%%%%%%%%%%%%%%
\appendix

\settowidth\MacroIndent{\rmfamily\scriptsize 000\ }

 \DocInput{childdoc.dtx}

\end{document}
%</driver>
% \fi
%
% %%%%%%%%%%%%%%%%%%%%%%%%%%%%%%%%%%%%%%%%%%%%%%%%%%%%%%%%%%%%%%%%%%%%%%%%%%%%%%
% %%%%%%%%%%%%%%%%%%%%%%%%%%%%%%%%%%%%%%%%%%%%%%%%%%%%%%%%%%%%%%%%%%%%%%%%%%%%%%
% \section{Sample}
%\iffalse
%<*samplemain>
%\fi
%
% The following presents a sample document
% with two chapters, two parts, a title page,
% a compile flag as well as three forwarding files to set the flag.
% It consists of eight |.tex| files:
% \begin{center}
% \begin{tabular}{ll}
% |cdocsamp.tex|&main file\\
% |cdocsch1.tex|&include file for chapter 1\\
% |cdocsch2.tex|&include file for chapter 2\\
% |cdocspt3.tex|&include file for part 3\\
% |cdocspt4.tex|&include file for part 4\\
% |cdocsdrf.tex|&forwarding file for main file in draft mode\\
% |cdocsfi1.tex|&forwarding file for final version of chapter 1\\
% |cdocsfi2.tex|&forwarding file for final version of chapter 2\\
% \end{tabular}
% \end{center}
% Each of the eight files can be compiled directly by the \LaTeX{} compiler.
%
% %%%%%%%%%%%%%%%%%%%%%%%%%%%%%%%%%%%%%%
% \paragraph{Main File.}
%
% The main file is called |cdocsamp.tex|.
%
% Load the \textsf{childdoc} definitions and
% declare the filename for the main document:
%    \begin{macrocode}
\input{childdoc.def}
\childdocmain{}
%    \end{macrocode}

% Optional override for |\version| flag:
%    \begin{macrocode}
%%\ifchilddoc\else\providecommand{\version}{draft}\fi
%    \end{macrocode}

% Define the default values for the |\version| flag
% (|final| for the main file and |draft| for childs):
%    \begin{macrocode}
\ifchilddoc
\providecommand{\version}{draft}
\else
\providecommand{\version}{final}
\fi
%    \end{macrocode}

% Load the standard document class:
%    \begin{macrocode}
\documentclass[12pt]{article}
%    \end{macrocode}

% Start the document body:
%    \begin{macrocode}
\begin{document}
%    \end{macrocode}

% Declare a title page.
% Print title, part of document being processed and version flag:
%    \begin{macrocode}
\addtocounter{page}{-1}
\begin{center}
{\LARGE\bfseries{}childdoc example\par}
\vspace{1cm}
\ifchilddoc
\ifchilddocmanual part\else chapter\fi:
`\childdocname' of `\childdocjob'\par
\else
main document: `\childdocjob'\par
\fi
version: \version\par
\end{center}
\newpage
%    \end{macrocode}

% Manually include selected file,
% otherwise process as usual:
%    \begin{macrocode}
\ifchilddocmanual
\section*{part `\childdocname'}
\input{\childdocname}
\else
%    \end{macrocode}

% Include the two chapters:
%    \begin{macrocode}
\include{cdocsch1}
\include{cdocsch2}
%    \end{macrocode}

% Include the two parts unless only chapters should be displayed:
%    \begin{macrocode}
\ifchilddoc\else
\section{part three}
\input{cdocspt3}
\section{part four}
\input{cdocspt4}
\fi
%    \end{macrocode}

% Process as usual until here:
%    \begin{macrocode}
\fi
%    \end{macrocode}

% End of document body:
%    \begin{macrocode}
\end{document}
%    \end{macrocode}
%\iffalse
%</samplemain>
%\fi
%
% %%%%%%%%%%%%%%%%%%%%%%%%%%%%%%%%%%%%%%
% \paragraph{Chapter Include Files.}
%
% The include files are called |cdocsch1.tex| and |cdocsch2.tex|.
%
%\iffalse
%<*samplechap1|samplechap2>
%\fi

% Optional override for |\version| flag:
%    \begin{macrocode}
%%\providecommand{\version}{final}
%    \end{macrocode}

% Include the main document:
%    \begin{macrocode}
\input{childdoc.def}
\childdocof{cdocsamp}
%    \end{macrocode}

%\iffalse
%</samplechap1|samplechap2>
%\fi
%
%\iffalse
%<*samplechap1>
%\fi
% Some text for chapter 1:
%    \begin{macrocode}
\section{one}
some text in chapter one
%    \end{macrocode}

%\iffalse
%</samplechap1>
%\fi
% Some text for chapter 2:
%\iffalse
%<*samplechap2>
%\fi
%    \begin{macrocode}
\section{two}
more text in chapter two
%    \end{macrocode}

%\iffalse
%</samplechap2>
%\fi
%
% %%%%%%%%%%%%%%%%%%%%%%%%%%%%%%%%%%%%%%
% \paragraph{Part Include Files.}
%
% The include files are called |cdocspt3.tex| and |cdocspt4.tex|.
%
%\iffalse
%<*samplepart3|samplepart4>
%\fi

% Optional override for |\version| flag:
%    \begin{macrocode}
%%\providecommand{\version}{final}
%    \end{macrocode}

% Include the main document:
%    \begin{macrocode}
\input{childdoc.def}
\childdocby{cdocsamp}
%    \end{macrocode}

%\iffalse
%</samplepart3|samplepart4>
%\fi
%
%\iffalse
%<*samplepart3>
%\fi
% Some text for part 3:
%    \begin{macrocode}
some text in part three
%    \end{macrocode}

%\iffalse
%</samplepart3>
%\fi
% Some text for part 4:
%\iffalse
%<*samplepart4>
%\fi
%    \begin{macrocode}
more text in part four
%    \end{macrocode}

%\iffalse
%</samplepart4>
%\fi
%
% %%%%%%%%%%%%%%%%%%%%%%%%%%%%%%%%%%%%%%
% \paragraph{Forwarding for a Complete Draft.}
%
% The following forwarding file |cdocsdrf.tex|
% compiles the main document in draft mode:
%\iffalse
%<*sampledraft>
%\fi
%    \begin{macrocode}
\def\version{draft}
\input{childdoc.def}
\childdocforward{cdocsamp}
%    \end{macrocode}

%\iffalse
%</sampledraft>
%\fi
%
% %%%%%%%%%%%%%%%%%%%%%%%%%%%%%%%%%%%%%%
% \paragraph{Forwarding for Final Version of the Chapters.}
%
% The following forwarding files |cdocsfn1.tex| and |cdocsfn2.tex|
% (with identical content)
% compile the final versions of the child documents
% |cdocsch1.tex| and |cdocsch2.tex|, respectively:
%\iffalse
%<*samplefinal>
%\fi
%    \begin{macrocode}
\def\version{final}
\input{childdoc.def}
\childdocforwardprefix[cdocsamp]{cdocsfn}{cdocsch}
%    \end{macrocode}

%\iffalse
%</samplefinal>
%\fi
%
% %%%%%%%%%%%%%%%%%%%%%%%%%%%%%%%%%%%%%%
% \paragraph{Command Line Processing.}
%
% The following three command lines generate the output files
% |cdocscld|, |cdocscl1| and |cdocscl2|
% which should be identical to
% |cdocsdrf|, |cdocsch1| and |cdocsfn2|, respectively:
% \begin{center}
% \begin{tabular}{l}
% |latex -jobname cdocscld \|\\
% |  "\def\version{draft}\input{childdoc.def}\childdocforward{cdocsamp}"|\\
% |latex -jobname cdocscl1 \|\\
% |  "\input{childdoc.def}\childdocforward[cdocsamp]{cdocsch1}"|\\
% |latex -jobname cdocscl2 \|\\
% |  "\def\version{final}\input{childdoc.def}\childdocforward{cdocsch2}"|
% \end{tabular}
% \end{center}
% Note that the trailing backslash on each first line
% merely continues the input to the second line
% (for convenient cut ant paste).
% Furthermore, the command |latex| can be replaced by any
% of its alternative versions such as |pdflatex|.
%
% %%%%%%%%%%%%%%%%%%%%%%%%%%%%%%%%%%%%%%%%%%%%%%%%%%%%%%%%%%%%%%%%%%%%%%%%%%%%%%
% %%%%%%%%%%%%%%%%%%%%%%%%%%%%%%%%%%%%%%%%%%%%%%%%%%%%%%%%%%%%%%%%%%%%%%%%%%%%%%
% \section{Implementation}
%\iffalse
%<*package>
%\fi
%
% This section describes the definitions file |childdoc.def|.

% The definitions cannot be loaded using |\usepackage| or |\RequirePackage|
% which has a mechanism to prevent loading a style file more than once.
% When loading the definitions by means of |\input|
% multiple instances have to be prevented manually:
%\iffalse
%This code needs to be before the `\ProvidesFile' directive
%which is defined at the beginning of this file.
%Therefore it is also placed there and commented out here.
%</package>
%<*discard>
%\fi
%    \begin{macrocode}
\ifdefined\childdocmain\endinput\fi
%    \end{macrocode}
%\iffalse
%</discard>
%<*package>
%\fi
%
% \macro{\ifchilddoc}
% \macro{\ifchilddocmanual}
% The conditional |\ifchilddoc| tells whether a
% child (true) or main (false) document is being compiled.
% The conditional |\ifchilddocmanual| tells whether
% the |\includeonly| mechanism is used (false) or
% the selection of child files must be performed manually (true).
% The definitions initialise to false:
%    \begin{macrocode}
\newif\ifchilddoc
\newif\ifchilddocmanual
%    \end{macrocode}

% \macro{\childdocname}
% \macro{\childdocjob}
% The macro |\childdocname| stores the name of the main document
% to be compiled. The macro |\childdocjob| stores the name of
% the document on which the \LaTeX{} compiler was originally invoked.
% The content of |\jobname| cannot be compared
% to filenames specified in the source due to different catcodes.
% The following code rescans |\jobname|, stores the result
% in |\childdocname| and saves a copy in |\childdocjob|:
%    \begin{macrocode}
\edef\childdocname{\scantokens\expandafter{\jobname\noexpand}}
\let\childdocjob\childdocname
%    \end{macrocode}

% \macro{\childdocdisable}
% The macro |\childdocdisable| prevents the main file
% from being processed more than once.
% At this stage, the main document command |\childdocmain|
% is assumed to be called once again where it should do nothing.
% Any subsequent call to it should prevent
% a secondary processing of the main document
% It overwrites the forwarding commands
% |\childdocof| and |\childdocforward|
% with empty macros to prevent further inclusions of the main document:
%    \begin{macrocode}
\newcommand{\childdocdisable}
{
  \renewcommand{\childdocmain}[1]{\renewcommand{\childdocmain}[1]{\endinput}}
  \renewcommand{\childdocof}[1]{}
  \renewcommand{\childdocby}[2][]{}
  \renewcommand{\childdocforward}[2][]{}
  \renewcommand{\childdocdisable}{}
}
%    \end{macrocode}

% \macro{\childdocmain}
% The macro |\childdocmain| is to be called at the top of the main file
% with nothing or the main filename (without extension) as argument.
% First, it breaks loops.
% If the argument is not empty and does not match |\childdocname|
% (which is set by the first inclusion of |childdoc.def|),
% |\ifchilddoc| is set to true, |\includeonly| is applied to the child file
% and |\jobname| is set to the main file
% (for proper handling of |.aux| files):
%    \begin{macrocode}
\newcommand{\childdocmain}[1]
{
  \childdocdisable\childdocmain{}
  \if?#1?\else
    \begingroup
      \def\childdoctmp{#1}
      \ifx\childdoctmp\childdocname
        \def\childdoctmp{}
      \else
        \def\childdoctmp
        {
          \childdoctrue
          \includeonly{\childdocname}
          \def\childdocjob{#1}
          \def\jobname{#1}
        }
      \fi
      \expandafter
    \endgroup
    \childdoctmp
  \fi
}
%    \end{macrocode}

% \macro{\childdocof}
% The command |\childdocof| redirects
% compilation to the main file |#1|.
%    \begin{macrocode}
\newcommand{\childdocof}[1]
{
  \childdocdisable
  \childdoctrue
  \includeonly{\childdocname}
  \def\jobname{#1}
  \def\childdocjob{#1}
  \input{#1}
}
%    \end{macrocode}

% \macro{\childdocby}
% The command |\childdocby| ....
%    \begin{macrocode}
\newcommand{\childdocby}[2][]
{
  \childdocdisable
  \childdoctrue
  \childdocmanualtrue
  \if?#1?\else
    \def\jobname{#2}
  \fi
  \def\childdocjob{#2}
  \input{#2}
  \endinput
}
%    \end{macrocode}

% \macro{\childdocforward}
% The command |\childdocforward| redirects
% compilation to the main file or
% (if the optional argument is given) a child file.
% Parameters are set as if the main file
% or a child file starting with |\childdocof| was compiled.
% Then compilation is handed over to the main file:
%    \begin{macrocode}
\newcommand{\childdocforward}[2][]
{
  \begingroup
    \if?#1?
      \def\childdoctmp
      {
        \def\childdocname{#2}
        \def\childdocjob{#2}
        \def\jobname{#2}
        \input{#2}
        \endinput
      }
    \else
      \def\childdoctmp
      {
        \childdocdisable
        \def\childdocname{#2}
        \childdoctrue
        \includeonly{#2}
        \def\childdocjob{#1}
        \def\jobname{#1}
        \input{#1}
        \endinput
      }
    \fi
    \expandafter
  \endgroup
  \childdoctmp
}
%    \end{macrocode}

% \macro{\childdocforwardprefix}
% The command |\childdocforwardprefix| redirects
% compilation to the main or a child file by means of a pattern.
% The prefix |#1| in the current filename is replaced by |#2|
% and the suffix of the current filename is kept
% (it is assumed that the filename does not contain the substring `|~~~|'
% which is used as a delimiter).
% Compilation is handed over to the new file by |\childdocforward|:
%    \begin{macrocode}
\newcommand{\childdocforwardprefix}[3][]
{
  \begingroup
    \def\childdocextract #2##1~~~{\def\childdoctmp{\childdocforward[#1]{#3##1}}}
    \expandafter\childdocextract\childdocname~~~
    \expandafter
  \endgroup
  \childdoctmp
}
%    \end{macrocode}

% \macro{\childdoc}
% The deprecated macro |\childdoc| is a legacy version of |\childdocmain|:
%    \begin{macrocode}
\newcommand{\childdoc}{\childdocmain}
%    \end{macrocode}

% \macro{\childdocredirect}
% The deprecated macro |\childdocredirect| is a legacy version
% of |\childdocforward| and |\childdocforwardprefix|:
%    \begin{macrocode}
\newcommand{\childdocredirect}[2][]
{
  \begingroup
    \if?#1?
      \def\childdoctmp{\childdocforward{#2}}
    \else
      \def\childdoctmp{\childdocforwardprefix{#1}{#2}}
    \fi
    \expandafter
  \endgroup
  \childdoctmp
}
%    \end{macrocode}

%\iffalse
%</package>
%\fi
%
\endinput
|\\
|\childdocby{|\textit{main}|}|\\
\end{tabular}
\end{center}
%
The directive |\childdocby| is similar to |\childdocof|
described in \secref{sec:include},
but the subsequent selection of content must be done manually.
To that end, both |\ifchilddoc| and |\ifchilddocmanual|
will be true upon processing of a part,
and the name of the part is stored in |\childdocname|.
Note that |\jobname| will be set to the filename of the current part
so that each part receives an individual |.aux| file
that does not interfere with the |.aux| file(s) of the main document.
This behaviour can be altered by the alternative form
|\childdocby[*]{|\textit{main}|}| (with a non-empty optional argument)
which uses the |.aux| file of the main document
by setting |\jobname| to \textit{main}.

%%%%%%%%%%%%%%%%%%%%%%%%%%%%%%%%%%%%%%%%%%%%%%%%%%%%%%%%%%%%%%%%%%%%%%%%%%%%%%%%
\subsection{Driver Development}
\label{sec:driver}

The \textsf{childdoc} mechanism can also be use for the development
of definition files such as \LaTeX{} styles or classes.
This case differs from the above setup with multiple parts
included by |\include| in that no |\includeonly| should be invoked.
This can be achieved by starting the include file
(before |\ProvidesPackage|) with:
%
\begin{center}
\begin{tabular}{l}
|% \iffalse
%
% childdoc.dtx Copyright (C) 2017-2018 Niklas Beisert
%
% This work may be distributed and/or modified under the
% conditions of the LaTeX Project Public License, either version 1.3
% of this license or (at your option) any later version.
% The latest version of this license is in
%   http://www.latex-project.org/lppl.txt
% and version 1.3 or later is part of all distributions of LaTeX
% version 2005/12/01 or later.
%
% This work has the LPPL maintenance status `maintained'.
%
% The Current Maintainer of this work is Niklas Beisert.
%
% This work consists of the files childdoc.dtx and childdoc.ins
% and the derived files childdoc.def and cdocsamp.tex with
% cdocsch1.tex, cdocsch2.tex, cdocsdrf.tex, cdocsfn1.tex, cdocsfn2.tex.
%
%<package>\ifdefined\childdocmain\endinput\fi
%<package>\ProvidesFile{childdoc.def}[2018/12/30 v2.0 child document driver]
%<samplemain>\ProvidesFile{cdocsamp.tex}[2018/12/30 v2.0 sample for childdoc]
%<*driver>
%\ProvidesFile{childdoc.drv}[2018/12/30 v2.0 childdoc reference manual file]
\PassOptionsToClass{10pt,a4paper}{article}
\documentclass{ltxdoc}

\usepackage[margin=35mm]{geometry}
\usepackage{hyperref}
\usepackage{hyperxmp}
\usepackage[usenames]{color}

\hypersetup{colorlinks=true}
\hypersetup{pdfstartview=FitH}
\hypersetup{pdfpagemode=UseNone}
\hypersetup{pdfsource={}}
\hypersetup{pdflang={en-UK}}
\hypersetup{pdfcopyright={Copyright 2017-2018 Niklas Beisert.
  This work may be distributed and/or modified under the
  conditions of the LaTeX Project Public License, either version 1.3
  of this license or (at your option) any later version.}}
\hypersetup{pdflicenseurl={http://www.latex-project.org/lppl.txt}}
\hypersetup{pdfcontactaddress={ETH Zurich, ITP, HIT K,
  Wolfgang-Pauli-Strasse 27}}
\hypersetup{pdfcontactpostcode={8093}}
\hypersetup{pdfcontactcity={Zurich}}
\hypersetup{pdfcontactcountry={Switzerland}}
\hypersetup{pdfcontactemail={nbeisert@itp.phys.ethz.ch}}
\hypersetup{pdfcontacturl={http://people.phys.ethz.ch/\xmptilde nbeisert/}}

\newcommand{\secref}[1]{\hyperref[#1]{section \ref*{#1}}}

\parskip1ex
\parindent0pt
\let\olditemize\itemize
\def\itemize{\olditemize\parskip0pt}

\begin{document}

\title{The \textsf{childdoc} Package}
\hypersetup{pdftitle={The childdoc Package}}
\author{Niklas Beisert\\[2ex]
  Institut f\"ur Theoretische Physik\\
  Eidgen\"ossische Technische Hochschule Z\"urich\\
  Wolfgang-Pauli-Strasse 27, 8093 Z\"urich, Switzerland\\[1ex]
  \href{mailto:nbeisert@itp.phys.ethz.ch}
  {\texttt{nbeisert@itp.phys.ethz.ch}}}
\hypersetup{pdfauthor={Niklas Beisert}}
\hypersetup{pdfsubject={Manual for the LaTeX2e Package childdoc}}
\date{30 December 2018, \textsf{v2.0}}
\maketitle

\begin{abstract}\noindent
\textsf{childdoc} is a \LaTeXe{} package
that enables the direct compilation
of document sections included by |\include|
to individual files.
\end{abstract}

\begingroup
\parskip0ex
\tableofcontents
\endgroup

%%%%%%%%%%%%%%%%%%%%%%%%%%%%%%%%%%%%%%%%%%%%%%%%%%%%%%%%%%%%%%%%%%%%%%%%%%%%%%%%
%%%%%%%%%%%%%%%%%%%%%%%%%%%%%%%%%%%%%%%%%%%%%%%%%%%%%%%%%%%%%%%%%%%%%%%%%%%%%%%%
\section{Introduction}

\LaTeX{} provides a mechanism to structure a large document (such as a book)
into a main file and several child files (containing the chapters)
using the |\include| command.
This mechanism is beneficial for documents
which span hundreds of pages in order to
make the source file(s) more manageable.
Moreover, compilation can be restricted to
selected child files by means of the |\includeonly| command.
The latter feature can be used to reduce the compilation time while editing
(this was significantly more useful in the earlier days of \LaTeX{})
or to generate a smaller document which is easier to navigate.
Another application of |\includeonly| is to generate
documents consisting of selected parts of the complete document.

However, there are a few drawbacks of the plain |\include| mechanism:
\begin{itemize}
\item
The child files cannot be compiled on their own,
they can only be compiled via the main file.
A naive editing environment
(such as a text editor with an option
to have the current file processed by \LaTeX)
may require one to switch to the main file before compiling;
attempting to compile the child file produces errors.
\item
The main file must be modified (each time)
to adjust the |\includeonly| command
to the present needs. This easily leaves the main file in a messy state.
\item
The generated document will always carry the filename
of the main document. This is inconvenient if
several child files are to be compiled and
to be kept for distribution.
\end{itemize}

The present package provides a simple interface
to make child files individually compilable by \LaTeX{}.
Compiling a child file then has the same effect as compiling
the main file with an |\includeonly| command
to select the appropriate child.
Moreover the generated document will carry the name of the child
rather than the main file.
This resolves all three above issues.

This feature is meant to make the editing of books,
thesis documents and lecture notes somewhat more convenient.
However, the package can also be used efficiently for
composing a series of documents (such as exercise sheets)
which are typically distributed individually.
It then assists the author in generating the individual documents
(potentially in different versions)
as well as a document containing the collected series.
Another application is in developing style files
or other kinds of included material
where compilation of the style file could redirect
to a sample or test file.

%%%%%%%%%%%%%%%%%%%%%%%%%%%%%%%%%%%%%%%%%%%%%%%%%%%%%%%%%%%%%%%%%%%%%%%%%%%%%%%%
%%%%%%%%%%%%%%%%%%%%%%%%%%%%%%%%%%%%%%%%%%%%%%%%%%%%%%%%%%%%%%%%%%%%%%%%%%%%%%%%
\section{Usage}

First of all, the package \textsf{childdoc} is \emph{not} a standard
\LaTeXe{} |.sty| style file! Therefore it needs to be invoked in
a non-standard way.

%%%%%%%%%%%%%%%%%%%%%%%%%%%%%%%%%%%%%%%%%%%%%%%%%%%%%%%%%%%%%%%%%%%%%%%%%%%%%%%%
\subsection{Included Files}
\label{sec:include}

%%%%%%%%%%%%%%%%%%%%%%%%%%%%%%%%%%%%%%%%
\DescribeMacro{\childdocmain}
To use the package, add the commands
\begin{center}
\begin{tabular}{l}
|\input{childdoc.def}|\\
|\childdocmain{}|\\
\end{tabular}
\end{center}
at the very top of the main \LaTeX{} file,
in particular \emph{before} the |\documentclass| statement!
The argument of |\childdocmain| should be left empty
(but it must be present).

%%%%%%%%%%%%%%%%%%%%%%%%%%%%%%%%%%%%%%%%
\DescribeMacro{\childdocof}
Furthermore, add the commands
\begin{center}
\begin{tabular}{l}
|\input{childdoc.def}|\\
|\childdocof{|\textit{main}|}|\\
\end{tabular}
\end{center}
at the top of every child file \textit{child}
which is included by |\include{|\textit{child}|}|
from within the main file
(or at least for those files to be compiled individually).
The argument \textit{main} must be the filename of the main file.

There are a couple of
considerations in setting up the main and child documents:

%%%%%%%%%%%%%%%%%%%%%%%%%%%%%%%%%%%%%%%%
\paragraph{Restrictions.}

Please note the following restrictions:
\begin{itemize}
\item
|\childdocmain| must be called with one argument \textit{main}
to ensure compatibility with earlier version of the package.
It must either be empty (|\childdocmain{}|)
or precisely match the filename of the main file in which it is specified.
See \secref{sec:detection} for further information.
\item
The filename \textit{main} must be specified without the |.tex| extension.
\item
The filename \textit{main} is case sensitive
(even in case-insensitive file systems)
due to internal string comparison.
\item
The argument \textit{main} should be fully expanded, it cannot be a macro.
\item
Subdirectories and special characters should be avoided in filenames.
\item
The command |\childdocmain{|\textit{main}|}| must be followed by a whitespace.
It should not be followed immediately by another command
or by a comment mark `|%|'.
This is because the \TeX{} parser reads the token immediately following
the argument of |\childdocmain| and puts it
at the beginning of every child section;
however, a white\-space is ignored.
\end{itemize}

%%%%%%%%%%%%%%%%%%%%%%%%%%%%%%%%%%%%%%%%
\paragraph{Content of Main File.}

It is advisable to place all content in the child files included by |\include|.
Any output contained in the main file will appear in all child documents
unless suppressed manually;
it cannot be suppressed automatically by the |\includeonly| directive
and thus should normally be avoided.
A method to include some content in the main file
by means of conditional processing is described in \secref{sec:conditional}.

%%%%%%%%%%%%%%%%%%%%%%%%%%%%%%%%%%%%%%%%
\paragraph{Page Numbering.}

When only a part of the document is compiled,
the appropriate numbering of pages
(as well as other status parameters)
is determined from the |.aux| files.
The latter contain information from previous passes.
However this information needs to propagate through
all intermediate child documents.
Therefore the page numbering in child documents may well
be inconsistent until the complete document is compiled at least once.

A useful (if unconventional) way to always ensure a consistent
page numbering is to restart the numbering in each child document
and denote the pages by `\textit{child}|.|\textit{page}'
where \textit{child} represents the chapter/section number of the child file.
This can be achieved by the command
|\numberwithin{page}{|\textit{child}|}|
of the \textsf{amsmath} package
where \textit{child} can be |chapter| or |section|
depending on the chosen structuring.
Alternatively, one can modify the macro |\thepage| appropriately
and reset the counter |page| at the start of each child file.

%%%%%%%%%%%%%%%%%%%%%%%%%%%%%%%%%%%%%%%%%%%%%%%%%%%%%%%%%%%%%%%%%%%%%%%%%%%%%%%%
\subsection{Conditional Processing}
\label{sec:conditional}

The package provides a mechanism to compile different versions
of a document. To customise the versions further some conditional processing
can come in handy to distinguish which version is being compiled.
The package provides two macros to describe the compilation context:

%%%%%%%%%%%%%%%%%%%%%%%%%%%%%%%%%%%%%%%%
\DescribeMacro{\ifchilddoc}
The conditional |\ifchilddoc| distinguishes between the compilation of
child documents and the main document:
%
\begin{center}
|\ifchilddoc |\textit{child-code}| |[|\||else |\textit{main-code}]| \||fi|
\end{center}

%%%%%%%%%%%%%%%%%%%%%%%%%%%%%%%%%%%%%%%%
\DescribeMacro{\childdocname}
\DescribeMacro{\childdocjob}
The macro |\childdocname| contains the filename (without extension)
of the main or child file being processed.
Note that |\childdocjob| will always contain the name of the main file.

%%%%%%%%%%%%%%%%%%%%%%%%%%%%%%%%%%%%%%%%
\paragraph{Title Page.}

Conditional processing can be used to include a title or banner page
in the main document when proper precautions are taken.
Importantly, the code in the main file should ensure that the page counter
(as well as other status parameters which are stored in the |.aux| files)
takes the same value after the conditional processing.
Otherwise the page numbers may take divergent values
depending on which part is compiled.

For example, a title page could be declared by:
%
\begin{center}
\begin{tabular}{l}
|\ifchilddoc\||else|\\
|\addtocounter{page}{-1}|\\
\textit{code for title page}\\
|\newpage|\\
|\||fi|
\end{tabular}
\end{center}
%
A banner page for the child documents can be generated by:
%
\begin{center}
\begin{tabular}{l}
|\ifchilddoc|\\
|\addtocounter{page}{-1}|\\
\textit{code for banner page}\\
|\newpage|\\
|\||fi|
\end{tabular}
\end{center}
%
Here one could write a message such as:
\begin{center}
|This is the part \childdocname{} of \childdocjob{}.|
\end{center}

%%%%%%%%%%%%%%%%%%%%%%%%%%%%%%%%%%%%%%%%%%%%%%%%%%%%%%%%%%%%%%%%%%%%%%%%%%%%%%%%
\subsection{Flags}
\label{sec:flags}

The package makes it easy to generate different versions
of the main or child documents.
To this end compilation flags can be defined
and assigned different default values.
They will be particularly useful in conjunction
with the forwarding mechanism described in \secref{sec:forward}.

For example, it may be useful to have a flag |\version|
which can be set to |draft| or |final|.
The document source will contain some conditional code
depending on the value of |\version|.
Suppose further, the flag should default to |final| for the main file
and to |draft| for child files
which is a natural assignment for editing the document.
This is achieved by placing the following code
in the preamble of the main document
(below the |\childdocmain| directive):
%
\begin{center}
\begin{tabular}{l}
|\ifchilddoc|\\
|\providecommand{\version}{draft}|\\
|\||else|\\
|\providecommand{\version}{final}|\\
|\||fi|
\end{tabular}
\end{center}
%
The definition by |\providecommand| makes sure
that previous definitions are not overwritten.
Further statements |\providecommand{\version}{...}|
can thus be added before the above code to override it.

For the main file, one might add a line
(between |\childdocmain| and the above block)
%
\begin{center}
|%\ifchilddoc\||else\providecommand{\version}{draft}\||fi|
\end{center}
%
which can be uncommented to produce a draft version.
Likewise one can add a line to the very top of a child file
(above the |\childdocof{|\textit{main}|}| directive)
%
\begin{center}
|%\providecommand{\version}{final}|
\end{center}
%
which can be uncommented to produce the final version of this child document.

%%%%%%%%%%%%%%%%%%%%%%%%%%%%%%%%%%%%%%%%%%%%%%%%%%%%%%%%%%%%%%%%%%%%%%%%%%%%%%%%
\subsection{Forwarding}
\label{sec:forward}

Different versions of the main or child documents
using compilation flags as described in \secref{sec:flags}
can be (permanently) stored in different files
for convenient compilation, viewing and distribution.
To this end, the package defines a command
to pass on compilation to a different file:

%%%%%%%%%%%%%%%%%%%%%%%%%%%%%%%%%%%%%%%%
\DescribeMacro{\childdocforward}
The command |\childdocforward| redirects processing to
another source file:
%
\begin{center}
\begin{tabular}{l}
|\input{childdoc.def}|\\
|\childdocforward[|\textit{main}|]{|\textit{dest}|}|\\
\end{tabular}
\end{center}
%
The argument \textit{dest} is the destination file
(without extension).
It should be the main file or one of the child files.
Note that further \textsf{childdoc} directives
such as |\childdocof| and |\childdocforward|
in the indicated file will be processed in this form.
The optional argument \textit{main}
passes on directly to the main file \textit{main}
while pretending to compile the child \textit{dest}.
This form behaves as if \textit{dest}
issues |\childdocof{|\textit{main}|}| right away,
and no further \textsf{childdoc} directives will be processed.

%%%%%%%%%%%%%%%%%%%%%%%%%%%%%%%%%%%%%%%%
\DescribeMacro{\...prefix}
In the alternative form |\childdocforwardprefix|,
%
\begin{center}
\begin{tabular}{l}
|\input{childdoc.def}|\\
|\childdocforwardprefix[|\textit{main}|]{|\textit{prefix}|}{|\textit{dest}|}|
\end{tabular}
\end{center}
%
the destination file is determined by a pattern
depending on the current file:
To make this work, the current file must be called
`{\textit{prefix}\hspace{0.2em}\textit{suffix}}'
with \textit{prefix} matching precisely the argument.
Processing is then passed on to the file
`{\textit{dest}\hspace{0.2em}\textit{suffix}}'.
Surely, the same effect is achieved by
directly specifying the
argument `{\textit{dest}\hspace{0.2em}\textit{suffix}}'
in the first form.
However, that requires to set up a different file
for each child. With the alternative form of the command
all these files can have exactly the same content
which simplifies setting them up and maintaining them.

For example, the following file |draft.tex|
with a compilation flag |\version| as described in \secref{sec:flags}
compiles the main document as a draft:
%
\begin{center}
\begin{tabular}{l}
|\def\version{draft}|\\
|\input{childdoc.def}|\\
|\childdocforward{|\textit{main}|}|
\end{tabular}
\end{center}
%
Likewise, the following files |final|\textit{nn}|.tex|
compile the final version of the child document
|child|\textit{nn}|.tex|:
%
\begin{center}
\begin{tabular}{l}
|\def\version{final}|\\
|\input{childdoc.def}|\\
|\childdocforwardprefix{final}{child}|
\end{tabular}
\end{center}
%

Note that when several versions of a main file and/or of each child file
are to be generated, it may be convenient to set up a |Makefile| or
shell script to automatise the process.

%%%%%%%%%%%%%%%%%%%%%%%%%%%%%%%%%%%%%%%%%%%%%%%%%%%%%%%%%%%%%%%%%%%%%%%%%%%%%%%%
\subsection{Command Line Processing}
\label{sec:commandline}

The effect of redirection files can also be achieved by invoking
the \LaTeX{} compiler with a more elaborate command line.
Most conveniently this should be done as part
of a shell script or a |Makefile|.

When using \textsf{childdoc} in the main file, the following
command lines effectively perform a redirection
(note that depending on the shell being used,
backslashes may have to be doubled: `|\|' $\to$ `|\\|'):
%
\begin{center}
|... -jobname "|\textit{target}|" |\\|"|[\textit{flags}]%
|\input{childdoc.def}\childdocforward[|\textit{main}|]{|\textit{dest}|}"|
\end{center}
%
Here \textit{target} is the name of the output file,
\textit{main} is the name of the main file
and \textit{dest} is the name of the main or child file to be processed
(all filenames without extensions).
The optional argument \textit{main} can be omitted
if \textit{main} matches \textit{dest}.
Optionally, compilation \textit{flags} can be defined via |\def| commands.
This command line makes the \TeX{} engine believe
it is compiling the file \textit{target}
whose content is specified as the latter parameter.
The provided code then forwards the processing to
\textit{main} or \textit{dest} as described in \secref{sec:forward}.

%%%%%%%%%%%%%%%%%%%%%%%%%%%%%%%%%%%%%%%%%%%%%%%%%%%%%%%%%%%%%%%%%%%%%%%%%%%%%%%%
\subsection{Include by Input}
\label{sec:input}

Including child documents by |\include| has some restrictions by design.
Most notably, the content of a child document always occupies
its own set of pages; pages cannot be shared between child documents.
Usually, this behaviour makes perfect sense
because each child document contain an essential part of the document.
However, in some situations it may be desirable to compose
a document from a collection of parts
without having mandatory page breaks between then.
For this case, the package
provides a mechanism to include parts
by |\input| which can also be processed individually.
However, by construction this mechanism
requires manual handling of the content to be output.

%%%%%%%%%%%%%%%%%%%%%%%%%%%%%%%%%%%%%%%%
\DescribeMacro{\ifchilddocmanual}
The main file should be prepared as usual, see \secref{sec:include}.
However, the document body must make a distinction
between processing of an individual part and of the main document, e.g.:
%
\begin{center}
\begin{tabular}{l}
|\ifchilddocmanual|\\
|\input{\childdocname}|\\
|\||else|\\
\textit{document body with }|\input{|\textit{part}|}|\\
|\||fi|
\end{tabular}
\end{center}
%
The conditional |\ifchilddocmanual| is true whenever
a part to be included by |\input| is being compiled,
and the name of the part is stored in |\childdocname|.

%%%%%%%%%%%%%%%%%%%%%%%%%%%%%%%%%%%%%%%%
\DescribeMacro{\childdocby}
Each part to be included by |\input| should start with:
%
\begin{center}
\begin{tabular}{l}
|\input{childdoc.def}|\\
|\childdocby{|\textit{main}|}|\\
\end{tabular}
\end{center}
%
The directive |\childdocby| is similar to |\childdocof|
described in \secref{sec:include},
but the subsequent selection of content must be done manually.
To that end, both |\ifchilddoc| and |\ifchilddocmanual|
will be true upon processing of a part,
and the name of the part is stored in |\childdocname|.
Note that |\jobname| will be set to the filename of the current part
so that each part receives an individual |.aux| file
that does not interfere with the |.aux| file(s) of the main document.
This behaviour can be altered by the alternative form
|\childdocby[*]{|\textit{main}|}| (with a non-empty optional argument)
which uses the |.aux| file of the main document
by setting |\jobname| to \textit{main}.

%%%%%%%%%%%%%%%%%%%%%%%%%%%%%%%%%%%%%%%%%%%%%%%%%%%%%%%%%%%%%%%%%%%%%%%%%%%%%%%%
\subsection{Driver Development}
\label{sec:driver}

The \textsf{childdoc} mechanism can also be use for the development
of definition files such as \LaTeX{} styles or classes.
This case differs from the above setup with multiple parts
included by |\include| in that no |\includeonly| should be invoked.
This can be achieved by starting the include file
(before |\ProvidesPackage|) with:
%
\begin{center}
\begin{tabular}{l}
|\input{childdoc.def}|\\
|\childdocforward{|\textit{main}|}|\\
\end{tabular}
\end{center}
%
or alternatively with:
%
\begin{center}
\begin{tabular}{l}
|\input{childdoc.def}|\\
|\childdocby{|\textit{main}|}|\\
\end{tabular}
\end{center}
%
Both forms have slightly different effects as described above.
The main file is prepared as usual, see \secref{sec:include}.

%%%%%%%%%%%%%%%%%%%%%%%%%%%%%%%%%%%%%%%%%%%%%%%%%%%%%%%%%%%%%%%%%%%%%%%%%%%%%%%%
\subsection{Legacy Detection}
\label{sec:detection}

The directive |\childdocmain| in the main file can detect
whether the complete document or merely a child is to be compiled
even without using the directive |\childdocof|.
This method is deprecated because it is less robust
and there is no compelling reason to use it;
it is merely provided for backward compatibility
and it may be removed in future versions.

If the detection mechanism is to be used,
it is mandatory to correctly specify
the filename of the main file as the argument of |\childdocmain|:
%
\begin{center}
\begin{tabular}{l}
|\input{childdoc.def}|\\
|\childdocmain{|\textit{main}|}|\\
\end{tabular}
\end{center}
%
If |\jobname| does not match the argument \textit{main} of |\childdocmain|,
it is assumed that |\jobname| points to the child file to be compiled.
When using |\childdocmain| with the main file specified as argument,
it suffices to start a child file
with just |\input{|\textit{main}|}|
without loading of the package and using |\childdocof|.
If instead all processing is done
with the appropriate \textsf{childdoc} directives,
the argument of \textit{main} of |\childdocmain| can be empty.

An alternative version of the command line processing described
in \secref{sec:commandline} using the detection mechanism reads:
%
\begin{center}
|... -jobname "|\textit{target}|" "|[\textit{flags}]%
[|\def\jobname{|\textit{dest}|}|]|\input{|\textit{main}|}"|
\end{center}

%%%%%%%%%%%%%%%%%%%%%%%%%%%%%%%%%%%%%%%%%%%%%%%%%%%%%%%%%%%%%%%%%%%%%%%%%%%%%%%%
\subsection{Manual Code}
\label{sec:manual}

In case one cannot be certain whether the definitions file |childdoc.def|
is installed on the target \TeX{} distribution
and one prefers not to ship it,
it is conceivable to paste a few relevant commands into the sources.

To that end, drop all statements |\input{childdoc.def}|
and perform the replacements as outlined below.
Instead of |\childdocmain{|\textit{main}|}| add the following code
to the top of the main file:
%
\begin{center}
\begin{tabular}{l}
|\||ifdefined\childdocname\endinput\||fi\newif\ifchilddoc|\\
|\edef\childdocname{\scantokens\expandafter{\jobname\noexpand}}|\\
|\def\childdocmain{|\textit{main}|}\||ifx\childdocmain\childdocname\||else|\\
|\childdoctrue\includeonly{\childdocname}\let\jobname\childdocmain\||fi|\\
\end{tabular}
\end{center}
%
Instead of |\childdocof{|\textit{main}|}| just include the main file
at the top of each child file:
%
\begin{center}
|\input{|\textit{main}|}|
\end{center}
%
A simple redirection |\childdocforward{|\textit{dest}|}| is achieved by:
%
\begin{center}
|\def\jobname{|\textit{dest}|}\input{\jobname}|
\end{center}
%
The redirection with prefix
|\childdocforwardprefix[|\textit{prefix}|]{|\textit{dest}|}|
is accomplished by:
%
\begin{center}
\begin{tabular}{l}
|{\edef\jobname{\scantokens\expandafter{\jobname\noexpand}}|\\
|\def\redirectjob |\textit{prefix}|#1~~~{\gdef\jobname{|\textit{dest}|#1}}|\\
|\expandafter\redirectjob\jobname~~~}\input{\jobname}|
\end{tabular}
\end{center}

In an alternative approach,
child documents can be compiled by a specific command line
without additional code or specific definitions:
%
\begin{center}
|... -jobname "|\textit{target}|" "|[\textit{flags}]%
|\includeonly{|\textit{dest}|}\input{|\textit{main}|}"|
\end{center}
%

%%%%%%%%%%%%%%%%%%%%%%%%%%%%%%%%%%%%%%%%%%%%%%%%%%%%%%%%%%%%%%%%%%%%%%%%%%%%%%%%
%%%%%%%%%%%%%%%%%%%%%%%%%%%%%%%%%%%%%%%%%%%%%%%%%%%%%%%%%%%%%%%%%%%%%%%%%%%%%%%%
\section{Information}

%%%%%%%%%%%%%%%%%%%%%%%%%%%%%%%%%%%%%%%%%%%%%%%%%%%%%%%%%%%%%%%%%%%%%%%%%%%%%%%%
\subsection{Copyright}

Copyright \copyright{} 2017--2018 Niklas Beisert

This work may be distributed and/or modified under the
conditions of the \LaTeX{} Project Public License, either version 1.3
of this license or (at your option) any later version.
The latest version of this license is in
  \url{http://www.latex-project.org/lppl.txt}
and version 1.3 or later is part of all distributions of \LaTeX{}
version 2005/12/01 or later.

This work has the LPPL maintenance status `maintained'.

The Current Maintainer of this work is Niklas Beisert.

This work consists of the files |README.txt|, |childdoc.ins| and |childdoc.dtx|
as well as the derived files |childdoc.def|, |cdocsamp.tex|
with |cdocsch1.tex|, |cdocsch2.tex|, |cdocspt3.tex|, |cdocspt4.tex|,
|cdocsdrf.tex|, |cdocsfn1.tex|, |cdocsfn2.tex|
as well as |childdoc.pdf|.

%%%%%%%%%%%%%%%%%%%%%%%%%%%%%%%%%%%%%%%%%%%%%%%%%%%%%%%%%%%%%%%%%%%%%%%%%%%%%%%%
\subsection{Files and Installation}

The package consists of the files:
%
\begin{center}
\begin{tabular}{ll}
    |README.txt|   & readme file \\
    |childdoc.ins| & installation file \\
    |childdoc.dtx| & source file \\
    |childdoc.def| & definition file \\
    |cdocsamp.tex| & sample main file \\
    |cdocsch1.tex| & sample include file \\
    |cdocsch2.tex| & sample include file \\
    |cdocspt3.tex| & sample part file \\
    |cdocspt4.tex| & sample part file \\
    |cdocsdrf.tex| & sample redirection file \\
    |cdocsfn1.tex| & sample redirection file \\
    |cdocsfn2.tex| & sample redirection file \\
    |childdoc.pdf| & manual
\end{tabular}
\end{center}
%
The distribution consists of the files
|README.txt|, |childdoc.ins| and |childdoc.dtx|.
%
\begin{itemize}
\item
Run (pdf)\LaTeX{} on |childdoc.dtx|
to compile the manual |childdoc.pdf| (this file).
\item
Run \LaTeX{} on |childdoc.ins| to create the definitions file |childdoc.def|
and the sample |cdocsamp.tex| with include files
|cdocsch1.tex|, |cdocsch2.tex|, |cdocspt3.tex|, |cdocspt4.tex|,
|cdocsdrf.tex|, |cdocsfn1.tex|, |cdocsfn2.tex|.
Then copy the file |childdoc.def| to an appropriate directory of your \LaTeX{}
distribution, e.g.\ \textit{texmf-root}|/tex/latex/childdoc|.
\end{itemize}

%%%%%%%%%%%%%%%%%%%%%%%%%%%%%%%%%%%%%%%%%%%%%%%%%%%%%%%%%%%%%%%%%%%%%%%%%%%%%%%%
\subsection{Related CTAN Packages}

There are several other packages which offer a similar functionality:
%
\begin{itemize}
\item
The packages
\href{http://ctan.org/pkg/docmute}{\textsf{docmute}},
\href{http://ctan.org/pkg/includex}{\textsf{includex}} and
\href{http://ctan.org/pkg/standalone}{\textsf{standalone}}
provide commands to include only the document body of
a child file thus allowing both files to be compiled individually.
\item
The packages \href{http://ctan.org/pkg/subdocs}{\textsf{subdocs}}
and \href{http://ctan.org/pkg/subfiles}{\textsf{subfiles}}
provide structures in which the main and child documents can be
encapsulated and allowing them to be compiled individually.
The inclusion mechanism is different from the conventional |\include|.
\item
The package \href{http://ctan.org/pkg/combine}{\textsf{combine}}
is an elaborate solution to combine several documents into one.
\end{itemize}
%
See also the CTAN topic \href{http://ctan.org/topic/subdocs}{\textsf{subdocs}}
for further related packages.
The present package differs from the above solutions in that
a document structure constructed with the conventional |\include| mechanism
just needs two extra commands at the top of every file
such that all constituent files can be compiled individually.

%%%%%%%%%%%%%%%%%%%%%%%%%%%%%%%%%%%%%%%%%%%%%%%%%%%%%%%%%%%%%%%%%%%%%%%%%%%%%%%%
%\subsection{Feature Suggestions}
%
%The following is a list of features which may be useful for future
%versions of this package:
%%
%\begin{itemize}
%\item
%\ldots
%\end{itemize}

%%%%%%%%%%%%%%%%%%%%%%%%%%%%%%%%%%%%%%%%%%%%%%%%%%%%%%%%%%%%%%%%%%%%%%%%%%%%%%%%
\subsection{Revision History}

%%%%%%%%%%%%%%%%%%%%%%%%%%%%%%%%%%%%%%%%
\paragraph{v2.0:} 2018/12/30

\begin{itemize}
\item
immediate forward processing
\item
added |\childdocby| mechanism
\item
manual restructured
\end{itemize}

%%%%%%%%%%%%%%%%%%%%%%%%%%%%%%%%%%%%%%%%
\paragraph{v1.6:} 2018/01/17

\begin{itemize}
\item
application for development of include files
\item
corrections to manual
\end{itemize}

%%%%%%%%%%%%%%%%%%%%%%%%%%%%%%%%%%%%%%%%
\paragraph{v1.5:} 2017/05/21

\begin{itemize}
\item
more complete structuring introduced
\item
|\childdocof| introduced
\item
|\childdoc| renamed to |\childdocmain|
\item
|\childredirect| renamed to |\childdocforward| and |\childdocforwardprefix|
and functionality expanded
\end{itemize}

%%%%%%%%%%%%%%%%%%%%%%%%%%%%%%%%%%%%%%%%
\paragraph{v1.0:} 2017/04/27

\begin{itemize}
\item
manual and install package
\item
first version published on CTAN
\end{itemize}

%%%%%%%%%%%%%%%%%%%%%%%%%%%%%%%%%%%%%%%%
\paragraph{v0.6:} 2017/04/26

\begin{itemize}
\item
redirection mechanism added
\end{itemize}

%%%%%%%%%%%%%%%%%%%%%%%%%%%%%%%%%%%%%%%%
\paragraph{v0.5:} 2017/04/26

\begin{itemize}
\item
functionality in definition file
\end{itemize}


%%%%%%%%%%%%%%%%%%%%%%%%%%%%%%%%%%%%%%%%%%%%%%%%%%%%%%%%%%%%%%%%%%%%%%%%%%%%%%%%
%%%%%%%%%%%%%%%%%%%%%%%%%%%%%%%%%%%%%%%%%%%%%%%%%%%%%%%%%%%%%%%%%%%%%%%%%%%%%%%%
%%%%%%%%%%%%%%%%%%%%%%%%%%%%%%%%%%%%%%%%%%%%%%%%%%%%%%%%%%%%%%%%%%%%%%%%%%%%%%%%
\appendix

\settowidth\MacroIndent{\rmfamily\scriptsize 000\ }

 \DocInput{childdoc.dtx}

\end{document}
%</driver>
% \fi
%
% %%%%%%%%%%%%%%%%%%%%%%%%%%%%%%%%%%%%%%%%%%%%%%%%%%%%%%%%%%%%%%%%%%%%%%%%%%%%%%
% %%%%%%%%%%%%%%%%%%%%%%%%%%%%%%%%%%%%%%%%%%%%%%%%%%%%%%%%%%%%%%%%%%%%%%%%%%%%%%
% \section{Sample}
%\iffalse
%<*samplemain>
%\fi
%
% The following presents a sample document
% with two chapters, two parts, a title page,
% a compile flag as well as three forwarding files to set the flag.
% It consists of eight |.tex| files:
% \begin{center}
% \begin{tabular}{ll}
% |cdocsamp.tex|&main file\\
% |cdocsch1.tex|&include file for chapter 1\\
% |cdocsch2.tex|&include file for chapter 2\\
% |cdocspt3.tex|&include file for part 3\\
% |cdocspt4.tex|&include file for part 4\\
% |cdocsdrf.tex|&forwarding file for main file in draft mode\\
% |cdocsfi1.tex|&forwarding file for final version of chapter 1\\
% |cdocsfi2.tex|&forwarding file for final version of chapter 2\\
% \end{tabular}
% \end{center}
% Each of the eight files can be compiled directly by the \LaTeX{} compiler.
%
% %%%%%%%%%%%%%%%%%%%%%%%%%%%%%%%%%%%%%%
% \paragraph{Main File.}
%
% The main file is called |cdocsamp.tex|.
%
% Load the \textsf{childdoc} definitions and
% declare the filename for the main document:
%    \begin{macrocode}
\input{childdoc.def}
\childdocmain{}
%    \end{macrocode}

% Optional override for |\version| flag:
%    \begin{macrocode}
%%\ifchilddoc\else\providecommand{\version}{draft}\fi
%    \end{macrocode}

% Define the default values for the |\version| flag
% (|final| for the main file and |draft| for childs):
%    \begin{macrocode}
\ifchilddoc
\providecommand{\version}{draft}
\else
\providecommand{\version}{final}
\fi
%    \end{macrocode}

% Load the standard document class:
%    \begin{macrocode}
\documentclass[12pt]{article}
%    \end{macrocode}

% Start the document body:
%    \begin{macrocode}
\begin{document}
%    \end{macrocode}

% Declare a title page.
% Print title, part of document being processed and version flag:
%    \begin{macrocode}
\addtocounter{page}{-1}
\begin{center}
{\LARGE\bfseries{}childdoc example\par}
\vspace{1cm}
\ifchilddoc
\ifchilddocmanual part\else chapter\fi:
`\childdocname' of `\childdocjob'\par
\else
main document: `\childdocjob'\par
\fi
version: \version\par
\end{center}
\newpage
%    \end{macrocode}

% Manually include selected file,
% otherwise process as usual:
%    \begin{macrocode}
\ifchilddocmanual
\section*{part `\childdocname'}
\input{\childdocname}
\else
%    \end{macrocode}

% Include the two chapters:
%    \begin{macrocode}
\include{cdocsch1}
\include{cdocsch2}
%    \end{macrocode}

% Include the two parts unless only chapters should be displayed:
%    \begin{macrocode}
\ifchilddoc\else
\section{part three}
\input{cdocspt3}
\section{part four}
\input{cdocspt4}
\fi
%    \end{macrocode}

% Process as usual until here:
%    \begin{macrocode}
\fi
%    \end{macrocode}

% End of document body:
%    \begin{macrocode}
\end{document}
%    \end{macrocode}
%\iffalse
%</samplemain>
%\fi
%
% %%%%%%%%%%%%%%%%%%%%%%%%%%%%%%%%%%%%%%
% \paragraph{Chapter Include Files.}
%
% The include files are called |cdocsch1.tex| and |cdocsch2.tex|.
%
%\iffalse
%<*samplechap1|samplechap2>
%\fi

% Optional override for |\version| flag:
%    \begin{macrocode}
%%\providecommand{\version}{final}
%    \end{macrocode}

% Include the main document:
%    \begin{macrocode}
\input{childdoc.def}
\childdocof{cdocsamp}
%    \end{macrocode}

%\iffalse
%</samplechap1|samplechap2>
%\fi
%
%\iffalse
%<*samplechap1>
%\fi
% Some text for chapter 1:
%    \begin{macrocode}
\section{one}
some text in chapter one
%    \end{macrocode}

%\iffalse
%</samplechap1>
%\fi
% Some text for chapter 2:
%\iffalse
%<*samplechap2>
%\fi
%    \begin{macrocode}
\section{two}
more text in chapter two
%    \end{macrocode}

%\iffalse
%</samplechap2>
%\fi
%
% %%%%%%%%%%%%%%%%%%%%%%%%%%%%%%%%%%%%%%
% \paragraph{Part Include Files.}
%
% The include files are called |cdocspt3.tex| and |cdocspt4.tex|.
%
%\iffalse
%<*samplepart3|samplepart4>
%\fi

% Optional override for |\version| flag:
%    \begin{macrocode}
%%\providecommand{\version}{final}
%    \end{macrocode}

% Include the main document:
%    \begin{macrocode}
\input{childdoc.def}
\childdocby{cdocsamp}
%    \end{macrocode}

%\iffalse
%</samplepart3|samplepart4>
%\fi
%
%\iffalse
%<*samplepart3>
%\fi
% Some text for part 3:
%    \begin{macrocode}
some text in part three
%    \end{macrocode}

%\iffalse
%</samplepart3>
%\fi
% Some text for part 4:
%\iffalse
%<*samplepart4>
%\fi
%    \begin{macrocode}
more text in part four
%    \end{macrocode}

%\iffalse
%</samplepart4>
%\fi
%
% %%%%%%%%%%%%%%%%%%%%%%%%%%%%%%%%%%%%%%
% \paragraph{Forwarding for a Complete Draft.}
%
% The following forwarding file |cdocsdrf.tex|
% compiles the main document in draft mode:
%\iffalse
%<*sampledraft>
%\fi
%    \begin{macrocode}
\def\version{draft}
\input{childdoc.def}
\childdocforward{cdocsamp}
%    \end{macrocode}

%\iffalse
%</sampledraft>
%\fi
%
% %%%%%%%%%%%%%%%%%%%%%%%%%%%%%%%%%%%%%%
% \paragraph{Forwarding for Final Version of the Chapters.}
%
% The following forwarding files |cdocsfn1.tex| and |cdocsfn2.tex|
% (with identical content)
% compile the final versions of the child documents
% |cdocsch1.tex| and |cdocsch2.tex|, respectively:
%\iffalse
%<*samplefinal>
%\fi
%    \begin{macrocode}
\def\version{final}
\input{childdoc.def}
\childdocforwardprefix[cdocsamp]{cdocsfn}{cdocsch}
%    \end{macrocode}

%\iffalse
%</samplefinal>
%\fi
%
% %%%%%%%%%%%%%%%%%%%%%%%%%%%%%%%%%%%%%%
% \paragraph{Command Line Processing.}
%
% The following three command lines generate the output files
% |cdocscld|, |cdocscl1| and |cdocscl2|
% which should be identical to
% |cdocsdrf|, |cdocsch1| and |cdocsfn2|, respectively:
% \begin{center}
% \begin{tabular}{l}
% |latex -jobname cdocscld \|\\
% |  "\def\version{draft}\input{childdoc.def}\childdocforward{cdocsamp}"|\\
% |latex -jobname cdocscl1 \|\\
% |  "\input{childdoc.def}\childdocforward[cdocsamp]{cdocsch1}"|\\
% |latex -jobname cdocscl2 \|\\
% |  "\def\version{final}\input{childdoc.def}\childdocforward{cdocsch2}"|
% \end{tabular}
% \end{center}
% Note that the trailing backslash on each first line
% merely continues the input to the second line
% (for convenient cut ant paste).
% Furthermore, the command |latex| can be replaced by any
% of its alternative versions such as |pdflatex|.
%
% %%%%%%%%%%%%%%%%%%%%%%%%%%%%%%%%%%%%%%%%%%%%%%%%%%%%%%%%%%%%%%%%%%%%%%%%%%%%%%
% %%%%%%%%%%%%%%%%%%%%%%%%%%%%%%%%%%%%%%%%%%%%%%%%%%%%%%%%%%%%%%%%%%%%%%%%%%%%%%
% \section{Implementation}
%\iffalse
%<*package>
%\fi
%
% This section describes the definitions file |childdoc.def|.

% The definitions cannot be loaded using |\usepackage| or |\RequirePackage|
% which has a mechanism to prevent loading a style file more than once.
% When loading the definitions by means of |\input|
% multiple instances have to be prevented manually:
%\iffalse
%This code needs to be before the `\ProvidesFile' directive
%which is defined at the beginning of this file.
%Therefore it is also placed there and commented out here.
%</package>
%<*discard>
%\fi
%    \begin{macrocode}
\ifdefined\childdocmain\endinput\fi
%    \end{macrocode}
%\iffalse
%</discard>
%<*package>
%\fi
%
% \macro{\ifchilddoc}
% \macro{\ifchilddocmanual}
% The conditional |\ifchilddoc| tells whether a
% child (true) or main (false) document is being compiled.
% The conditional |\ifchilddocmanual| tells whether
% the |\includeonly| mechanism is used (false) or
% the selection of child files must be performed manually (true).
% The definitions initialise to false:
%    \begin{macrocode}
\newif\ifchilddoc
\newif\ifchilddocmanual
%    \end{macrocode}

% \macro{\childdocname}
% \macro{\childdocjob}
% The macro |\childdocname| stores the name of the main document
% to be compiled. The macro |\childdocjob| stores the name of
% the document on which the \LaTeX{} compiler was originally invoked.
% The content of |\jobname| cannot be compared
% to filenames specified in the source due to different catcodes.
% The following code rescans |\jobname|, stores the result
% in |\childdocname| and saves a copy in |\childdocjob|:
%    \begin{macrocode}
\edef\childdocname{\scantokens\expandafter{\jobname\noexpand}}
\let\childdocjob\childdocname
%    \end{macrocode}

% \macro{\childdocdisable}
% The macro |\childdocdisable| prevents the main file
% from being processed more than once.
% At this stage, the main document command |\childdocmain|
% is assumed to be called once again where it should do nothing.
% Any subsequent call to it should prevent
% a secondary processing of the main document
% It overwrites the forwarding commands
% |\childdocof| and |\childdocforward|
% with empty macros to prevent further inclusions of the main document:
%    \begin{macrocode}
\newcommand{\childdocdisable}
{
  \renewcommand{\childdocmain}[1]{\renewcommand{\childdocmain}[1]{\endinput}}
  \renewcommand{\childdocof}[1]{}
  \renewcommand{\childdocby}[2][]{}
  \renewcommand{\childdocforward}[2][]{}
  \renewcommand{\childdocdisable}{}
}
%    \end{macrocode}

% \macro{\childdocmain}
% The macro |\childdocmain| is to be called at the top of the main file
% with nothing or the main filename (without extension) as argument.
% First, it breaks loops.
% If the argument is not empty and does not match |\childdocname|
% (which is set by the first inclusion of |childdoc.def|),
% |\ifchilddoc| is set to true, |\includeonly| is applied to the child file
% and |\jobname| is set to the main file
% (for proper handling of |.aux| files):
%    \begin{macrocode}
\newcommand{\childdocmain}[1]
{
  \childdocdisable\childdocmain{}
  \if?#1?\else
    \begingroup
      \def\childdoctmp{#1}
      \ifx\childdoctmp\childdocname
        \def\childdoctmp{}
      \else
        \def\childdoctmp
        {
          \childdoctrue
          \includeonly{\childdocname}
          \def\childdocjob{#1}
          \def\jobname{#1}
        }
      \fi
      \expandafter
    \endgroup
    \childdoctmp
  \fi
}
%    \end{macrocode}

% \macro{\childdocof}
% The command |\childdocof| redirects
% compilation to the main file |#1|.
%    \begin{macrocode}
\newcommand{\childdocof}[1]
{
  \childdocdisable
  \childdoctrue
  \includeonly{\childdocname}
  \def\jobname{#1}
  \def\childdocjob{#1}
  \input{#1}
}
%    \end{macrocode}

% \macro{\childdocby}
% The command |\childdocby| ....
%    \begin{macrocode}
\newcommand{\childdocby}[2][]
{
  \childdocdisable
  \childdoctrue
  \childdocmanualtrue
  \if?#1?\else
    \def\jobname{#2}
  \fi
  \def\childdocjob{#2}
  \input{#2}
  \endinput
}
%    \end{macrocode}

% \macro{\childdocforward}
% The command |\childdocforward| redirects
% compilation to the main file or
% (if the optional argument is given) a child file.
% Parameters are set as if the main file
% or a child file starting with |\childdocof| was compiled.
% Then compilation is handed over to the main file:
%    \begin{macrocode}
\newcommand{\childdocforward}[2][]
{
  \begingroup
    \if?#1?
      \def\childdoctmp
      {
        \def\childdocname{#2}
        \def\childdocjob{#2}
        \def\jobname{#2}
        \input{#2}
        \endinput
      }
    \else
      \def\childdoctmp
      {
        \childdocdisable
        \def\childdocname{#2}
        \childdoctrue
        \includeonly{#2}
        \def\childdocjob{#1}
        \def\jobname{#1}
        \input{#1}
        \endinput
      }
    \fi
    \expandafter
  \endgroup
  \childdoctmp
}
%    \end{macrocode}

% \macro{\childdocforwardprefix}
% The command |\childdocforwardprefix| redirects
% compilation to the main or a child file by means of a pattern.
% The prefix |#1| in the current filename is replaced by |#2|
% and the suffix of the current filename is kept
% (it is assumed that the filename does not contain the substring `|~~~|'
% which is used as a delimiter).
% Compilation is handed over to the new file by |\childdocforward|:
%    \begin{macrocode}
\newcommand{\childdocforwardprefix}[3][]
{
  \begingroup
    \def\childdocextract #2##1~~~{\def\childdoctmp{\childdocforward[#1]{#3##1}}}
    \expandafter\childdocextract\childdocname~~~
    \expandafter
  \endgroup
  \childdoctmp
}
%    \end{macrocode}

% \macro{\childdoc}
% The deprecated macro |\childdoc| is a legacy version of |\childdocmain|:
%    \begin{macrocode}
\newcommand{\childdoc}{\childdocmain}
%    \end{macrocode}

% \macro{\childdocredirect}
% The deprecated macro |\childdocredirect| is a legacy version
% of |\childdocforward| and |\childdocforwardprefix|:
%    \begin{macrocode}
\newcommand{\childdocredirect}[2][]
{
  \begingroup
    \if?#1?
      \def\childdoctmp{\childdocforward{#2}}
    \else
      \def\childdoctmp{\childdocforwardprefix{#1}{#2}}
    \fi
    \expandafter
  \endgroup
  \childdoctmp
}
%    \end{macrocode}

%\iffalse
%</package>
%\fi
%
\endinput
|\\
|\childdocforward{|\textit{main}|}|\\
\end{tabular}
\end{center}
%
or alternatively with:
%
\begin{center}
\begin{tabular}{l}
|% \iffalse
%
% childdoc.dtx Copyright (C) 2017-2018 Niklas Beisert
%
% This work may be distributed and/or modified under the
% conditions of the LaTeX Project Public License, either version 1.3
% of this license or (at your option) any later version.
% The latest version of this license is in
%   http://www.latex-project.org/lppl.txt
% and version 1.3 or later is part of all distributions of LaTeX
% version 2005/12/01 or later.
%
% This work has the LPPL maintenance status `maintained'.
%
% The Current Maintainer of this work is Niklas Beisert.
%
% This work consists of the files childdoc.dtx and childdoc.ins
% and the derived files childdoc.def and cdocsamp.tex with
% cdocsch1.tex, cdocsch2.tex, cdocsdrf.tex, cdocsfn1.tex, cdocsfn2.tex.
%
%<package>\ifdefined\childdocmain\endinput\fi
%<package>\ProvidesFile{childdoc.def}[2018/12/30 v2.0 child document driver]
%<samplemain>\ProvidesFile{cdocsamp.tex}[2018/12/30 v2.0 sample for childdoc]
%<*driver>
%\ProvidesFile{childdoc.drv}[2018/12/30 v2.0 childdoc reference manual file]
\PassOptionsToClass{10pt,a4paper}{article}
\documentclass{ltxdoc}

\usepackage[margin=35mm]{geometry}
\usepackage{hyperref}
\usepackage{hyperxmp}
\usepackage[usenames]{color}

\hypersetup{colorlinks=true}
\hypersetup{pdfstartview=FitH}
\hypersetup{pdfpagemode=UseNone}
\hypersetup{pdfsource={}}
\hypersetup{pdflang={en-UK}}
\hypersetup{pdfcopyright={Copyright 2017-2018 Niklas Beisert.
  This work may be distributed and/or modified under the
  conditions of the LaTeX Project Public License, either version 1.3
  of this license or (at your option) any later version.}}
\hypersetup{pdflicenseurl={http://www.latex-project.org/lppl.txt}}
\hypersetup{pdfcontactaddress={ETH Zurich, ITP, HIT K,
  Wolfgang-Pauli-Strasse 27}}
\hypersetup{pdfcontactpostcode={8093}}
\hypersetup{pdfcontactcity={Zurich}}
\hypersetup{pdfcontactcountry={Switzerland}}
\hypersetup{pdfcontactemail={nbeisert@itp.phys.ethz.ch}}
\hypersetup{pdfcontacturl={http://people.phys.ethz.ch/\xmptilde nbeisert/}}

\newcommand{\secref}[1]{\hyperref[#1]{section \ref*{#1}}}

\parskip1ex
\parindent0pt
\let\olditemize\itemize
\def\itemize{\olditemize\parskip0pt}

\begin{document}

\title{The \textsf{childdoc} Package}
\hypersetup{pdftitle={The childdoc Package}}
\author{Niklas Beisert\\[2ex]
  Institut f\"ur Theoretische Physik\\
  Eidgen\"ossische Technische Hochschule Z\"urich\\
  Wolfgang-Pauli-Strasse 27, 8093 Z\"urich, Switzerland\\[1ex]
  \href{mailto:nbeisert@itp.phys.ethz.ch}
  {\texttt{nbeisert@itp.phys.ethz.ch}}}
\hypersetup{pdfauthor={Niklas Beisert}}
\hypersetup{pdfsubject={Manual for the LaTeX2e Package childdoc}}
\date{30 December 2018, \textsf{v2.0}}
\maketitle

\begin{abstract}\noindent
\textsf{childdoc} is a \LaTeXe{} package
that enables the direct compilation
of document sections included by |\include|
to individual files.
\end{abstract}

\begingroup
\parskip0ex
\tableofcontents
\endgroup

%%%%%%%%%%%%%%%%%%%%%%%%%%%%%%%%%%%%%%%%%%%%%%%%%%%%%%%%%%%%%%%%%%%%%%%%%%%%%%%%
%%%%%%%%%%%%%%%%%%%%%%%%%%%%%%%%%%%%%%%%%%%%%%%%%%%%%%%%%%%%%%%%%%%%%%%%%%%%%%%%
\section{Introduction}

\LaTeX{} provides a mechanism to structure a large document (such as a book)
into a main file and several child files (containing the chapters)
using the |\include| command.
This mechanism is beneficial for documents
which span hundreds of pages in order to
make the source file(s) more manageable.
Moreover, compilation can be restricted to
selected child files by means of the |\includeonly| command.
The latter feature can be used to reduce the compilation time while editing
(this was significantly more useful in the earlier days of \LaTeX{})
or to generate a smaller document which is easier to navigate.
Another application of |\includeonly| is to generate
documents consisting of selected parts of the complete document.

However, there are a few drawbacks of the plain |\include| mechanism:
\begin{itemize}
\item
The child files cannot be compiled on their own,
they can only be compiled via the main file.
A naive editing environment
(such as a text editor with an option
to have the current file processed by \LaTeX)
may require one to switch to the main file before compiling;
attempting to compile the child file produces errors.
\item
The main file must be modified (each time)
to adjust the |\includeonly| command
to the present needs. This easily leaves the main file in a messy state.
\item
The generated document will always carry the filename
of the main document. This is inconvenient if
several child files are to be compiled and
to be kept for distribution.
\end{itemize}

The present package provides a simple interface
to make child files individually compilable by \LaTeX{}.
Compiling a child file then has the same effect as compiling
the main file with an |\includeonly| command
to select the appropriate child.
Moreover the generated document will carry the name of the child
rather than the main file.
This resolves all three above issues.

This feature is meant to make the editing of books,
thesis documents and lecture notes somewhat more convenient.
However, the package can also be used efficiently for
composing a series of documents (such as exercise sheets)
which are typically distributed individually.
It then assists the author in generating the individual documents
(potentially in different versions)
as well as a document containing the collected series.
Another application is in developing style files
or other kinds of included material
where compilation of the style file could redirect
to a sample or test file.

%%%%%%%%%%%%%%%%%%%%%%%%%%%%%%%%%%%%%%%%%%%%%%%%%%%%%%%%%%%%%%%%%%%%%%%%%%%%%%%%
%%%%%%%%%%%%%%%%%%%%%%%%%%%%%%%%%%%%%%%%%%%%%%%%%%%%%%%%%%%%%%%%%%%%%%%%%%%%%%%%
\section{Usage}

First of all, the package \textsf{childdoc} is \emph{not} a standard
\LaTeXe{} |.sty| style file! Therefore it needs to be invoked in
a non-standard way.

%%%%%%%%%%%%%%%%%%%%%%%%%%%%%%%%%%%%%%%%%%%%%%%%%%%%%%%%%%%%%%%%%%%%%%%%%%%%%%%%
\subsection{Included Files}
\label{sec:include}

%%%%%%%%%%%%%%%%%%%%%%%%%%%%%%%%%%%%%%%%
\DescribeMacro{\childdocmain}
To use the package, add the commands
\begin{center}
\begin{tabular}{l}
|\input{childdoc.def}|\\
|\childdocmain{}|\\
\end{tabular}
\end{center}
at the very top of the main \LaTeX{} file,
in particular \emph{before} the |\documentclass| statement!
The argument of |\childdocmain| should be left empty
(but it must be present).

%%%%%%%%%%%%%%%%%%%%%%%%%%%%%%%%%%%%%%%%
\DescribeMacro{\childdocof}
Furthermore, add the commands
\begin{center}
\begin{tabular}{l}
|\input{childdoc.def}|\\
|\childdocof{|\textit{main}|}|\\
\end{tabular}
\end{center}
at the top of every child file \textit{child}
which is included by |\include{|\textit{child}|}|
from within the main file
(or at least for those files to be compiled individually).
The argument \textit{main} must be the filename of the main file.

There are a couple of
considerations in setting up the main and child documents:

%%%%%%%%%%%%%%%%%%%%%%%%%%%%%%%%%%%%%%%%
\paragraph{Restrictions.}

Please note the following restrictions:
\begin{itemize}
\item
|\childdocmain| must be called with one argument \textit{main}
to ensure compatibility with earlier version of the package.
It must either be empty (|\childdocmain{}|)
or precisely match the filename of the main file in which it is specified.
See \secref{sec:detection} for further information.
\item
The filename \textit{main} must be specified without the |.tex| extension.
\item
The filename \textit{main} is case sensitive
(even in case-insensitive file systems)
due to internal string comparison.
\item
The argument \textit{main} should be fully expanded, it cannot be a macro.
\item
Subdirectories and special characters should be avoided in filenames.
\item
The command |\childdocmain{|\textit{main}|}| must be followed by a whitespace.
It should not be followed immediately by another command
or by a comment mark `|%|'.
This is because the \TeX{} parser reads the token immediately following
the argument of |\childdocmain| and puts it
at the beginning of every child section;
however, a white\-space is ignored.
\end{itemize}

%%%%%%%%%%%%%%%%%%%%%%%%%%%%%%%%%%%%%%%%
\paragraph{Content of Main File.}

It is advisable to place all content in the child files included by |\include|.
Any output contained in the main file will appear in all child documents
unless suppressed manually;
it cannot be suppressed automatically by the |\includeonly| directive
and thus should normally be avoided.
A method to include some content in the main file
by means of conditional processing is described in \secref{sec:conditional}.

%%%%%%%%%%%%%%%%%%%%%%%%%%%%%%%%%%%%%%%%
\paragraph{Page Numbering.}

When only a part of the document is compiled,
the appropriate numbering of pages
(as well as other status parameters)
is determined from the |.aux| files.
The latter contain information from previous passes.
However this information needs to propagate through
all intermediate child documents.
Therefore the page numbering in child documents may well
be inconsistent until the complete document is compiled at least once.

A useful (if unconventional) way to always ensure a consistent
page numbering is to restart the numbering in each child document
and denote the pages by `\textit{child}|.|\textit{page}'
where \textit{child} represents the chapter/section number of the child file.
This can be achieved by the command
|\numberwithin{page}{|\textit{child}|}|
of the \textsf{amsmath} package
where \textit{child} can be |chapter| or |section|
depending on the chosen structuring.
Alternatively, one can modify the macro |\thepage| appropriately
and reset the counter |page| at the start of each child file.

%%%%%%%%%%%%%%%%%%%%%%%%%%%%%%%%%%%%%%%%%%%%%%%%%%%%%%%%%%%%%%%%%%%%%%%%%%%%%%%%
\subsection{Conditional Processing}
\label{sec:conditional}

The package provides a mechanism to compile different versions
of a document. To customise the versions further some conditional processing
can come in handy to distinguish which version is being compiled.
The package provides two macros to describe the compilation context:

%%%%%%%%%%%%%%%%%%%%%%%%%%%%%%%%%%%%%%%%
\DescribeMacro{\ifchilddoc}
The conditional |\ifchilddoc| distinguishes between the compilation of
child documents and the main document:
%
\begin{center}
|\ifchilddoc |\textit{child-code}| |[|\||else |\textit{main-code}]| \||fi|
\end{center}

%%%%%%%%%%%%%%%%%%%%%%%%%%%%%%%%%%%%%%%%
\DescribeMacro{\childdocname}
\DescribeMacro{\childdocjob}
The macro |\childdocname| contains the filename (without extension)
of the main or child file being processed.
Note that |\childdocjob| will always contain the name of the main file.

%%%%%%%%%%%%%%%%%%%%%%%%%%%%%%%%%%%%%%%%
\paragraph{Title Page.}

Conditional processing can be used to include a title or banner page
in the main document when proper precautions are taken.
Importantly, the code in the main file should ensure that the page counter
(as well as other status parameters which are stored in the |.aux| files)
takes the same value after the conditional processing.
Otherwise the page numbers may take divergent values
depending on which part is compiled.

For example, a title page could be declared by:
%
\begin{center}
\begin{tabular}{l}
|\ifchilddoc\||else|\\
|\addtocounter{page}{-1}|\\
\textit{code for title page}\\
|\newpage|\\
|\||fi|
\end{tabular}
\end{center}
%
A banner page for the child documents can be generated by:
%
\begin{center}
\begin{tabular}{l}
|\ifchilddoc|\\
|\addtocounter{page}{-1}|\\
\textit{code for banner page}\\
|\newpage|\\
|\||fi|
\end{tabular}
\end{center}
%
Here one could write a message such as:
\begin{center}
|This is the part \childdocname{} of \childdocjob{}.|
\end{center}

%%%%%%%%%%%%%%%%%%%%%%%%%%%%%%%%%%%%%%%%%%%%%%%%%%%%%%%%%%%%%%%%%%%%%%%%%%%%%%%%
\subsection{Flags}
\label{sec:flags}

The package makes it easy to generate different versions
of the main or child documents.
To this end compilation flags can be defined
and assigned different default values.
They will be particularly useful in conjunction
with the forwarding mechanism described in \secref{sec:forward}.

For example, it may be useful to have a flag |\version|
which can be set to |draft| or |final|.
The document source will contain some conditional code
depending on the value of |\version|.
Suppose further, the flag should default to |final| for the main file
and to |draft| for child files
which is a natural assignment for editing the document.
This is achieved by placing the following code
in the preamble of the main document
(below the |\childdocmain| directive):
%
\begin{center}
\begin{tabular}{l}
|\ifchilddoc|\\
|\providecommand{\version}{draft}|\\
|\||else|\\
|\providecommand{\version}{final}|\\
|\||fi|
\end{tabular}
\end{center}
%
The definition by |\providecommand| makes sure
that previous definitions are not overwritten.
Further statements |\providecommand{\version}{...}|
can thus be added before the above code to override it.

For the main file, one might add a line
(between |\childdocmain| and the above block)
%
\begin{center}
|%\ifchilddoc\||else\providecommand{\version}{draft}\||fi|
\end{center}
%
which can be uncommented to produce a draft version.
Likewise one can add a line to the very top of a child file
(above the |\childdocof{|\textit{main}|}| directive)
%
\begin{center}
|%\providecommand{\version}{final}|
\end{center}
%
which can be uncommented to produce the final version of this child document.

%%%%%%%%%%%%%%%%%%%%%%%%%%%%%%%%%%%%%%%%%%%%%%%%%%%%%%%%%%%%%%%%%%%%%%%%%%%%%%%%
\subsection{Forwarding}
\label{sec:forward}

Different versions of the main or child documents
using compilation flags as described in \secref{sec:flags}
can be (permanently) stored in different files
for convenient compilation, viewing and distribution.
To this end, the package defines a command
to pass on compilation to a different file:

%%%%%%%%%%%%%%%%%%%%%%%%%%%%%%%%%%%%%%%%
\DescribeMacro{\childdocforward}
The command |\childdocforward| redirects processing to
another source file:
%
\begin{center}
\begin{tabular}{l}
|\input{childdoc.def}|\\
|\childdocforward[|\textit{main}|]{|\textit{dest}|}|\\
\end{tabular}
\end{center}
%
The argument \textit{dest} is the destination file
(without extension).
It should be the main file or one of the child files.
Note that further \textsf{childdoc} directives
such as |\childdocof| and |\childdocforward|
in the indicated file will be processed in this form.
The optional argument \textit{main}
passes on directly to the main file \textit{main}
while pretending to compile the child \textit{dest}.
This form behaves as if \textit{dest}
issues |\childdocof{|\textit{main}|}| right away,
and no further \textsf{childdoc} directives will be processed.

%%%%%%%%%%%%%%%%%%%%%%%%%%%%%%%%%%%%%%%%
\DescribeMacro{\...prefix}
In the alternative form |\childdocforwardprefix|,
%
\begin{center}
\begin{tabular}{l}
|\input{childdoc.def}|\\
|\childdocforwardprefix[|\textit{main}|]{|\textit{prefix}|}{|\textit{dest}|}|
\end{tabular}
\end{center}
%
the destination file is determined by a pattern
depending on the current file:
To make this work, the current file must be called
`{\textit{prefix}\hspace{0.2em}\textit{suffix}}'
with \textit{prefix} matching precisely the argument.
Processing is then passed on to the file
`{\textit{dest}\hspace{0.2em}\textit{suffix}}'.
Surely, the same effect is achieved by
directly specifying the
argument `{\textit{dest}\hspace{0.2em}\textit{suffix}}'
in the first form.
However, that requires to set up a different file
for each child. With the alternative form of the command
all these files can have exactly the same content
which simplifies setting them up and maintaining them.

For example, the following file |draft.tex|
with a compilation flag |\version| as described in \secref{sec:flags}
compiles the main document as a draft:
%
\begin{center}
\begin{tabular}{l}
|\def\version{draft}|\\
|\input{childdoc.def}|\\
|\childdocforward{|\textit{main}|}|
\end{tabular}
\end{center}
%
Likewise, the following files |final|\textit{nn}|.tex|
compile the final version of the child document
|child|\textit{nn}|.tex|:
%
\begin{center}
\begin{tabular}{l}
|\def\version{final}|\\
|\input{childdoc.def}|\\
|\childdocforwardprefix{final}{child}|
\end{tabular}
\end{center}
%

Note that when several versions of a main file and/or of each child file
are to be generated, it may be convenient to set up a |Makefile| or
shell script to automatise the process.

%%%%%%%%%%%%%%%%%%%%%%%%%%%%%%%%%%%%%%%%%%%%%%%%%%%%%%%%%%%%%%%%%%%%%%%%%%%%%%%%
\subsection{Command Line Processing}
\label{sec:commandline}

The effect of redirection files can also be achieved by invoking
the \LaTeX{} compiler with a more elaborate command line.
Most conveniently this should be done as part
of a shell script or a |Makefile|.

When using \textsf{childdoc} in the main file, the following
command lines effectively perform a redirection
(note that depending on the shell being used,
backslashes may have to be doubled: `|\|' $\to$ `|\\|'):
%
\begin{center}
|... -jobname "|\textit{target}|" |\\|"|[\textit{flags}]%
|\input{childdoc.def}\childdocforward[|\textit{main}|]{|\textit{dest}|}"|
\end{center}
%
Here \textit{target} is the name of the output file,
\textit{main} is the name of the main file
and \textit{dest} is the name of the main or child file to be processed
(all filenames without extensions).
The optional argument \textit{main} can be omitted
if \textit{main} matches \textit{dest}.
Optionally, compilation \textit{flags} can be defined via |\def| commands.
This command line makes the \TeX{} engine believe
it is compiling the file \textit{target}
whose content is specified as the latter parameter.
The provided code then forwards the processing to
\textit{main} or \textit{dest} as described in \secref{sec:forward}.

%%%%%%%%%%%%%%%%%%%%%%%%%%%%%%%%%%%%%%%%%%%%%%%%%%%%%%%%%%%%%%%%%%%%%%%%%%%%%%%%
\subsection{Include by Input}
\label{sec:input}

Including child documents by |\include| has some restrictions by design.
Most notably, the content of a child document always occupies
its own set of pages; pages cannot be shared between child documents.
Usually, this behaviour makes perfect sense
because each child document contain an essential part of the document.
However, in some situations it may be desirable to compose
a document from a collection of parts
without having mandatory page breaks between then.
For this case, the package
provides a mechanism to include parts
by |\input| which can also be processed individually.
However, by construction this mechanism
requires manual handling of the content to be output.

%%%%%%%%%%%%%%%%%%%%%%%%%%%%%%%%%%%%%%%%
\DescribeMacro{\ifchilddocmanual}
The main file should be prepared as usual, see \secref{sec:include}.
However, the document body must make a distinction
between processing of an individual part and of the main document, e.g.:
%
\begin{center}
\begin{tabular}{l}
|\ifchilddocmanual|\\
|\input{\childdocname}|\\
|\||else|\\
\textit{document body with }|\input{|\textit{part}|}|\\
|\||fi|
\end{tabular}
\end{center}
%
The conditional |\ifchilddocmanual| is true whenever
a part to be included by |\input| is being compiled,
and the name of the part is stored in |\childdocname|.

%%%%%%%%%%%%%%%%%%%%%%%%%%%%%%%%%%%%%%%%
\DescribeMacro{\childdocby}
Each part to be included by |\input| should start with:
%
\begin{center}
\begin{tabular}{l}
|\input{childdoc.def}|\\
|\childdocby{|\textit{main}|}|\\
\end{tabular}
\end{center}
%
The directive |\childdocby| is similar to |\childdocof|
described in \secref{sec:include},
but the subsequent selection of content must be done manually.
To that end, both |\ifchilddoc| and |\ifchilddocmanual|
will be true upon processing of a part,
and the name of the part is stored in |\childdocname|.
Note that |\jobname| will be set to the filename of the current part
so that each part receives an individual |.aux| file
that does not interfere with the |.aux| file(s) of the main document.
This behaviour can be altered by the alternative form
|\childdocby[*]{|\textit{main}|}| (with a non-empty optional argument)
which uses the |.aux| file of the main document
by setting |\jobname| to \textit{main}.

%%%%%%%%%%%%%%%%%%%%%%%%%%%%%%%%%%%%%%%%%%%%%%%%%%%%%%%%%%%%%%%%%%%%%%%%%%%%%%%%
\subsection{Driver Development}
\label{sec:driver}

The \textsf{childdoc} mechanism can also be use for the development
of definition files such as \LaTeX{} styles or classes.
This case differs from the above setup with multiple parts
included by |\include| in that no |\includeonly| should be invoked.
This can be achieved by starting the include file
(before |\ProvidesPackage|) with:
%
\begin{center}
\begin{tabular}{l}
|\input{childdoc.def}|\\
|\childdocforward{|\textit{main}|}|\\
\end{tabular}
\end{center}
%
or alternatively with:
%
\begin{center}
\begin{tabular}{l}
|\input{childdoc.def}|\\
|\childdocby{|\textit{main}|}|\\
\end{tabular}
\end{center}
%
Both forms have slightly different effects as described above.
The main file is prepared as usual, see \secref{sec:include}.

%%%%%%%%%%%%%%%%%%%%%%%%%%%%%%%%%%%%%%%%%%%%%%%%%%%%%%%%%%%%%%%%%%%%%%%%%%%%%%%%
\subsection{Legacy Detection}
\label{sec:detection}

The directive |\childdocmain| in the main file can detect
whether the complete document or merely a child is to be compiled
even without using the directive |\childdocof|.
This method is deprecated because it is less robust
and there is no compelling reason to use it;
it is merely provided for backward compatibility
and it may be removed in future versions.

If the detection mechanism is to be used,
it is mandatory to correctly specify
the filename of the main file as the argument of |\childdocmain|:
%
\begin{center}
\begin{tabular}{l}
|\input{childdoc.def}|\\
|\childdocmain{|\textit{main}|}|\\
\end{tabular}
\end{center}
%
If |\jobname| does not match the argument \textit{main} of |\childdocmain|,
it is assumed that |\jobname| points to the child file to be compiled.
When using |\childdocmain| with the main file specified as argument,
it suffices to start a child file
with just |\input{|\textit{main}|}|
without loading of the package and using |\childdocof|.
If instead all processing is done
with the appropriate \textsf{childdoc} directives,
the argument of \textit{main} of |\childdocmain| can be empty.

An alternative version of the command line processing described
in \secref{sec:commandline} using the detection mechanism reads:
%
\begin{center}
|... -jobname "|\textit{target}|" "|[\textit{flags}]%
[|\def\jobname{|\textit{dest}|}|]|\input{|\textit{main}|}"|
\end{center}

%%%%%%%%%%%%%%%%%%%%%%%%%%%%%%%%%%%%%%%%%%%%%%%%%%%%%%%%%%%%%%%%%%%%%%%%%%%%%%%%
\subsection{Manual Code}
\label{sec:manual}

In case one cannot be certain whether the definitions file |childdoc.def|
is installed on the target \TeX{} distribution
and one prefers not to ship it,
it is conceivable to paste a few relevant commands into the sources.

To that end, drop all statements |\input{childdoc.def}|
and perform the replacements as outlined below.
Instead of |\childdocmain{|\textit{main}|}| add the following code
to the top of the main file:
%
\begin{center}
\begin{tabular}{l}
|\||ifdefined\childdocname\endinput\||fi\newif\ifchilddoc|\\
|\edef\childdocname{\scantokens\expandafter{\jobname\noexpand}}|\\
|\def\childdocmain{|\textit{main}|}\||ifx\childdocmain\childdocname\||else|\\
|\childdoctrue\includeonly{\childdocname}\let\jobname\childdocmain\||fi|\\
\end{tabular}
\end{center}
%
Instead of |\childdocof{|\textit{main}|}| just include the main file
at the top of each child file:
%
\begin{center}
|\input{|\textit{main}|}|
\end{center}
%
A simple redirection |\childdocforward{|\textit{dest}|}| is achieved by:
%
\begin{center}
|\def\jobname{|\textit{dest}|}\input{\jobname}|
\end{center}
%
The redirection with prefix
|\childdocforwardprefix[|\textit{prefix}|]{|\textit{dest}|}|
is accomplished by:
%
\begin{center}
\begin{tabular}{l}
|{\edef\jobname{\scantokens\expandafter{\jobname\noexpand}}|\\
|\def\redirectjob |\textit{prefix}|#1~~~{\gdef\jobname{|\textit{dest}|#1}}|\\
|\expandafter\redirectjob\jobname~~~}\input{\jobname}|
\end{tabular}
\end{center}

In an alternative approach,
child documents can be compiled by a specific command line
without additional code or specific definitions:
%
\begin{center}
|... -jobname "|\textit{target}|" "|[\textit{flags}]%
|\includeonly{|\textit{dest}|}\input{|\textit{main}|}"|
\end{center}
%

%%%%%%%%%%%%%%%%%%%%%%%%%%%%%%%%%%%%%%%%%%%%%%%%%%%%%%%%%%%%%%%%%%%%%%%%%%%%%%%%
%%%%%%%%%%%%%%%%%%%%%%%%%%%%%%%%%%%%%%%%%%%%%%%%%%%%%%%%%%%%%%%%%%%%%%%%%%%%%%%%
\section{Information}

%%%%%%%%%%%%%%%%%%%%%%%%%%%%%%%%%%%%%%%%%%%%%%%%%%%%%%%%%%%%%%%%%%%%%%%%%%%%%%%%
\subsection{Copyright}

Copyright \copyright{} 2017--2018 Niklas Beisert

This work may be distributed and/or modified under the
conditions of the \LaTeX{} Project Public License, either version 1.3
of this license or (at your option) any later version.
The latest version of this license is in
  \url{http://www.latex-project.org/lppl.txt}
and version 1.3 or later is part of all distributions of \LaTeX{}
version 2005/12/01 or later.

This work has the LPPL maintenance status `maintained'.

The Current Maintainer of this work is Niklas Beisert.

This work consists of the files |README.txt|, |childdoc.ins| and |childdoc.dtx|
as well as the derived files |childdoc.def|, |cdocsamp.tex|
with |cdocsch1.tex|, |cdocsch2.tex|, |cdocspt3.tex|, |cdocspt4.tex|,
|cdocsdrf.tex|, |cdocsfn1.tex|, |cdocsfn2.tex|
as well as |childdoc.pdf|.

%%%%%%%%%%%%%%%%%%%%%%%%%%%%%%%%%%%%%%%%%%%%%%%%%%%%%%%%%%%%%%%%%%%%%%%%%%%%%%%%
\subsection{Files and Installation}

The package consists of the files:
%
\begin{center}
\begin{tabular}{ll}
    |README.txt|   & readme file \\
    |childdoc.ins| & installation file \\
    |childdoc.dtx| & source file \\
    |childdoc.def| & definition file \\
    |cdocsamp.tex| & sample main file \\
    |cdocsch1.tex| & sample include file \\
    |cdocsch2.tex| & sample include file \\
    |cdocspt3.tex| & sample part file \\
    |cdocspt4.tex| & sample part file \\
    |cdocsdrf.tex| & sample redirection file \\
    |cdocsfn1.tex| & sample redirection file \\
    |cdocsfn2.tex| & sample redirection file \\
    |childdoc.pdf| & manual
\end{tabular}
\end{center}
%
The distribution consists of the files
|README.txt|, |childdoc.ins| and |childdoc.dtx|.
%
\begin{itemize}
\item
Run (pdf)\LaTeX{} on |childdoc.dtx|
to compile the manual |childdoc.pdf| (this file).
\item
Run \LaTeX{} on |childdoc.ins| to create the definitions file |childdoc.def|
and the sample |cdocsamp.tex| with include files
|cdocsch1.tex|, |cdocsch2.tex|, |cdocspt3.tex|, |cdocspt4.tex|,
|cdocsdrf.tex|, |cdocsfn1.tex|, |cdocsfn2.tex|.
Then copy the file |childdoc.def| to an appropriate directory of your \LaTeX{}
distribution, e.g.\ \textit{texmf-root}|/tex/latex/childdoc|.
\end{itemize}

%%%%%%%%%%%%%%%%%%%%%%%%%%%%%%%%%%%%%%%%%%%%%%%%%%%%%%%%%%%%%%%%%%%%%%%%%%%%%%%%
\subsection{Related CTAN Packages}

There are several other packages which offer a similar functionality:
%
\begin{itemize}
\item
The packages
\href{http://ctan.org/pkg/docmute}{\textsf{docmute}},
\href{http://ctan.org/pkg/includex}{\textsf{includex}} and
\href{http://ctan.org/pkg/standalone}{\textsf{standalone}}
provide commands to include only the document body of
a child file thus allowing both files to be compiled individually.
\item
The packages \href{http://ctan.org/pkg/subdocs}{\textsf{subdocs}}
and \href{http://ctan.org/pkg/subfiles}{\textsf{subfiles}}
provide structures in which the main and child documents can be
encapsulated and allowing them to be compiled individually.
The inclusion mechanism is different from the conventional |\include|.
\item
The package \href{http://ctan.org/pkg/combine}{\textsf{combine}}
is an elaborate solution to combine several documents into one.
\end{itemize}
%
See also the CTAN topic \href{http://ctan.org/topic/subdocs}{\textsf{subdocs}}
for further related packages.
The present package differs from the above solutions in that
a document structure constructed with the conventional |\include| mechanism
just needs two extra commands at the top of every file
such that all constituent files can be compiled individually.

%%%%%%%%%%%%%%%%%%%%%%%%%%%%%%%%%%%%%%%%%%%%%%%%%%%%%%%%%%%%%%%%%%%%%%%%%%%%%%%%
%\subsection{Feature Suggestions}
%
%The following is a list of features which may be useful for future
%versions of this package:
%%
%\begin{itemize}
%\item
%\ldots
%\end{itemize}

%%%%%%%%%%%%%%%%%%%%%%%%%%%%%%%%%%%%%%%%%%%%%%%%%%%%%%%%%%%%%%%%%%%%%%%%%%%%%%%%
\subsection{Revision History}

%%%%%%%%%%%%%%%%%%%%%%%%%%%%%%%%%%%%%%%%
\paragraph{v2.0:} 2018/12/30

\begin{itemize}
\item
immediate forward processing
\item
added |\childdocby| mechanism
\item
manual restructured
\end{itemize}

%%%%%%%%%%%%%%%%%%%%%%%%%%%%%%%%%%%%%%%%
\paragraph{v1.6:} 2018/01/17

\begin{itemize}
\item
application for development of include files
\item
corrections to manual
\end{itemize}

%%%%%%%%%%%%%%%%%%%%%%%%%%%%%%%%%%%%%%%%
\paragraph{v1.5:} 2017/05/21

\begin{itemize}
\item
more complete structuring introduced
\item
|\childdocof| introduced
\item
|\childdoc| renamed to |\childdocmain|
\item
|\childredirect| renamed to |\childdocforward| and |\childdocforwardprefix|
and functionality expanded
\end{itemize}

%%%%%%%%%%%%%%%%%%%%%%%%%%%%%%%%%%%%%%%%
\paragraph{v1.0:} 2017/04/27

\begin{itemize}
\item
manual and install package
\item
first version published on CTAN
\end{itemize}

%%%%%%%%%%%%%%%%%%%%%%%%%%%%%%%%%%%%%%%%
\paragraph{v0.6:} 2017/04/26

\begin{itemize}
\item
redirection mechanism added
\end{itemize}

%%%%%%%%%%%%%%%%%%%%%%%%%%%%%%%%%%%%%%%%
\paragraph{v0.5:} 2017/04/26

\begin{itemize}
\item
functionality in definition file
\end{itemize}


%%%%%%%%%%%%%%%%%%%%%%%%%%%%%%%%%%%%%%%%%%%%%%%%%%%%%%%%%%%%%%%%%%%%%%%%%%%%%%%%
%%%%%%%%%%%%%%%%%%%%%%%%%%%%%%%%%%%%%%%%%%%%%%%%%%%%%%%%%%%%%%%%%%%%%%%%%%%%%%%%
%%%%%%%%%%%%%%%%%%%%%%%%%%%%%%%%%%%%%%%%%%%%%%%%%%%%%%%%%%%%%%%%%%%%%%%%%%%%%%%%
\appendix

\settowidth\MacroIndent{\rmfamily\scriptsize 000\ }

 \DocInput{childdoc.dtx}

\end{document}
%</driver>
% \fi
%
% %%%%%%%%%%%%%%%%%%%%%%%%%%%%%%%%%%%%%%%%%%%%%%%%%%%%%%%%%%%%%%%%%%%%%%%%%%%%%%
% %%%%%%%%%%%%%%%%%%%%%%%%%%%%%%%%%%%%%%%%%%%%%%%%%%%%%%%%%%%%%%%%%%%%%%%%%%%%%%
% \section{Sample}
%\iffalse
%<*samplemain>
%\fi
%
% The following presents a sample document
% with two chapters, two parts, a title page,
% a compile flag as well as three forwarding files to set the flag.
% It consists of eight |.tex| files:
% \begin{center}
% \begin{tabular}{ll}
% |cdocsamp.tex|&main file\\
% |cdocsch1.tex|&include file for chapter 1\\
% |cdocsch2.tex|&include file for chapter 2\\
% |cdocspt3.tex|&include file for part 3\\
% |cdocspt4.tex|&include file for part 4\\
% |cdocsdrf.tex|&forwarding file for main file in draft mode\\
% |cdocsfi1.tex|&forwarding file for final version of chapter 1\\
% |cdocsfi2.tex|&forwarding file for final version of chapter 2\\
% \end{tabular}
% \end{center}
% Each of the eight files can be compiled directly by the \LaTeX{} compiler.
%
% %%%%%%%%%%%%%%%%%%%%%%%%%%%%%%%%%%%%%%
% \paragraph{Main File.}
%
% The main file is called |cdocsamp.tex|.
%
% Load the \textsf{childdoc} definitions and
% declare the filename for the main document:
%    \begin{macrocode}
\input{childdoc.def}
\childdocmain{}
%    \end{macrocode}

% Optional override for |\version| flag:
%    \begin{macrocode}
%%\ifchilddoc\else\providecommand{\version}{draft}\fi
%    \end{macrocode}

% Define the default values for the |\version| flag
% (|final| for the main file and |draft| for childs):
%    \begin{macrocode}
\ifchilddoc
\providecommand{\version}{draft}
\else
\providecommand{\version}{final}
\fi
%    \end{macrocode}

% Load the standard document class:
%    \begin{macrocode}
\documentclass[12pt]{article}
%    \end{macrocode}

% Start the document body:
%    \begin{macrocode}
\begin{document}
%    \end{macrocode}

% Declare a title page.
% Print title, part of document being processed and version flag:
%    \begin{macrocode}
\addtocounter{page}{-1}
\begin{center}
{\LARGE\bfseries{}childdoc example\par}
\vspace{1cm}
\ifchilddoc
\ifchilddocmanual part\else chapter\fi:
`\childdocname' of `\childdocjob'\par
\else
main document: `\childdocjob'\par
\fi
version: \version\par
\end{center}
\newpage
%    \end{macrocode}

% Manually include selected file,
% otherwise process as usual:
%    \begin{macrocode}
\ifchilddocmanual
\section*{part `\childdocname'}
\input{\childdocname}
\else
%    \end{macrocode}

% Include the two chapters:
%    \begin{macrocode}
\include{cdocsch1}
\include{cdocsch2}
%    \end{macrocode}

% Include the two parts unless only chapters should be displayed:
%    \begin{macrocode}
\ifchilddoc\else
\section{part three}
\input{cdocspt3}
\section{part four}
\input{cdocspt4}
\fi
%    \end{macrocode}

% Process as usual until here:
%    \begin{macrocode}
\fi
%    \end{macrocode}

% End of document body:
%    \begin{macrocode}
\end{document}
%    \end{macrocode}
%\iffalse
%</samplemain>
%\fi
%
% %%%%%%%%%%%%%%%%%%%%%%%%%%%%%%%%%%%%%%
% \paragraph{Chapter Include Files.}
%
% The include files are called |cdocsch1.tex| and |cdocsch2.tex|.
%
%\iffalse
%<*samplechap1|samplechap2>
%\fi

% Optional override for |\version| flag:
%    \begin{macrocode}
%%\providecommand{\version}{final}
%    \end{macrocode}

% Include the main document:
%    \begin{macrocode}
\input{childdoc.def}
\childdocof{cdocsamp}
%    \end{macrocode}

%\iffalse
%</samplechap1|samplechap2>
%\fi
%
%\iffalse
%<*samplechap1>
%\fi
% Some text for chapter 1:
%    \begin{macrocode}
\section{one}
some text in chapter one
%    \end{macrocode}

%\iffalse
%</samplechap1>
%\fi
% Some text for chapter 2:
%\iffalse
%<*samplechap2>
%\fi
%    \begin{macrocode}
\section{two}
more text in chapter two
%    \end{macrocode}

%\iffalse
%</samplechap2>
%\fi
%
% %%%%%%%%%%%%%%%%%%%%%%%%%%%%%%%%%%%%%%
% \paragraph{Part Include Files.}
%
% The include files are called |cdocspt3.tex| and |cdocspt4.tex|.
%
%\iffalse
%<*samplepart3|samplepart4>
%\fi

% Optional override for |\version| flag:
%    \begin{macrocode}
%%\providecommand{\version}{final}
%    \end{macrocode}

% Include the main document:
%    \begin{macrocode}
\input{childdoc.def}
\childdocby{cdocsamp}
%    \end{macrocode}

%\iffalse
%</samplepart3|samplepart4>
%\fi
%
%\iffalse
%<*samplepart3>
%\fi
% Some text for part 3:
%    \begin{macrocode}
some text in part three
%    \end{macrocode}

%\iffalse
%</samplepart3>
%\fi
% Some text for part 4:
%\iffalse
%<*samplepart4>
%\fi
%    \begin{macrocode}
more text in part four
%    \end{macrocode}

%\iffalse
%</samplepart4>
%\fi
%
% %%%%%%%%%%%%%%%%%%%%%%%%%%%%%%%%%%%%%%
% \paragraph{Forwarding for a Complete Draft.}
%
% The following forwarding file |cdocsdrf.tex|
% compiles the main document in draft mode:
%\iffalse
%<*sampledraft>
%\fi
%    \begin{macrocode}
\def\version{draft}
\input{childdoc.def}
\childdocforward{cdocsamp}
%    \end{macrocode}

%\iffalse
%</sampledraft>
%\fi
%
% %%%%%%%%%%%%%%%%%%%%%%%%%%%%%%%%%%%%%%
% \paragraph{Forwarding for Final Version of the Chapters.}
%
% The following forwarding files |cdocsfn1.tex| and |cdocsfn2.tex|
% (with identical content)
% compile the final versions of the child documents
% |cdocsch1.tex| and |cdocsch2.tex|, respectively:
%\iffalse
%<*samplefinal>
%\fi
%    \begin{macrocode}
\def\version{final}
\input{childdoc.def}
\childdocforwardprefix[cdocsamp]{cdocsfn}{cdocsch}
%    \end{macrocode}

%\iffalse
%</samplefinal>
%\fi
%
% %%%%%%%%%%%%%%%%%%%%%%%%%%%%%%%%%%%%%%
% \paragraph{Command Line Processing.}
%
% The following three command lines generate the output files
% |cdocscld|, |cdocscl1| and |cdocscl2|
% which should be identical to
% |cdocsdrf|, |cdocsch1| and |cdocsfn2|, respectively:
% \begin{center}
% \begin{tabular}{l}
% |latex -jobname cdocscld \|\\
% |  "\def\version{draft}\input{childdoc.def}\childdocforward{cdocsamp}"|\\
% |latex -jobname cdocscl1 \|\\
% |  "\input{childdoc.def}\childdocforward[cdocsamp]{cdocsch1}"|\\
% |latex -jobname cdocscl2 \|\\
% |  "\def\version{final}\input{childdoc.def}\childdocforward{cdocsch2}"|
% \end{tabular}
% \end{center}
% Note that the trailing backslash on each first line
% merely continues the input to the second line
% (for convenient cut ant paste).
% Furthermore, the command |latex| can be replaced by any
% of its alternative versions such as |pdflatex|.
%
% %%%%%%%%%%%%%%%%%%%%%%%%%%%%%%%%%%%%%%%%%%%%%%%%%%%%%%%%%%%%%%%%%%%%%%%%%%%%%%
% %%%%%%%%%%%%%%%%%%%%%%%%%%%%%%%%%%%%%%%%%%%%%%%%%%%%%%%%%%%%%%%%%%%%%%%%%%%%%%
% \section{Implementation}
%\iffalse
%<*package>
%\fi
%
% This section describes the definitions file |childdoc.def|.

% The definitions cannot be loaded using |\usepackage| or |\RequirePackage|
% which has a mechanism to prevent loading a style file more than once.
% When loading the definitions by means of |\input|
% multiple instances have to be prevented manually:
%\iffalse
%This code needs to be before the `\ProvidesFile' directive
%which is defined at the beginning of this file.
%Therefore it is also placed there and commented out here.
%</package>
%<*discard>
%\fi
%    \begin{macrocode}
\ifdefined\childdocmain\endinput\fi
%    \end{macrocode}
%\iffalse
%</discard>
%<*package>
%\fi
%
% \macro{\ifchilddoc}
% \macro{\ifchilddocmanual}
% The conditional |\ifchilddoc| tells whether a
% child (true) or main (false) document is being compiled.
% The conditional |\ifchilddocmanual| tells whether
% the |\includeonly| mechanism is used (false) or
% the selection of child files must be performed manually (true).
% The definitions initialise to false:
%    \begin{macrocode}
\newif\ifchilddoc
\newif\ifchilddocmanual
%    \end{macrocode}

% \macro{\childdocname}
% \macro{\childdocjob}
% The macro |\childdocname| stores the name of the main document
% to be compiled. The macro |\childdocjob| stores the name of
% the document on which the \LaTeX{} compiler was originally invoked.
% The content of |\jobname| cannot be compared
% to filenames specified in the source due to different catcodes.
% The following code rescans |\jobname|, stores the result
% in |\childdocname| and saves a copy in |\childdocjob|:
%    \begin{macrocode}
\edef\childdocname{\scantokens\expandafter{\jobname\noexpand}}
\let\childdocjob\childdocname
%    \end{macrocode}

% \macro{\childdocdisable}
% The macro |\childdocdisable| prevents the main file
% from being processed more than once.
% At this stage, the main document command |\childdocmain|
% is assumed to be called once again where it should do nothing.
% Any subsequent call to it should prevent
% a secondary processing of the main document
% It overwrites the forwarding commands
% |\childdocof| and |\childdocforward|
% with empty macros to prevent further inclusions of the main document:
%    \begin{macrocode}
\newcommand{\childdocdisable}
{
  \renewcommand{\childdocmain}[1]{\renewcommand{\childdocmain}[1]{\endinput}}
  \renewcommand{\childdocof}[1]{}
  \renewcommand{\childdocby}[2][]{}
  \renewcommand{\childdocforward}[2][]{}
  \renewcommand{\childdocdisable}{}
}
%    \end{macrocode}

% \macro{\childdocmain}
% The macro |\childdocmain| is to be called at the top of the main file
% with nothing or the main filename (without extension) as argument.
% First, it breaks loops.
% If the argument is not empty and does not match |\childdocname|
% (which is set by the first inclusion of |childdoc.def|),
% |\ifchilddoc| is set to true, |\includeonly| is applied to the child file
% and |\jobname| is set to the main file
% (for proper handling of |.aux| files):
%    \begin{macrocode}
\newcommand{\childdocmain}[1]
{
  \childdocdisable\childdocmain{}
  \if?#1?\else
    \begingroup
      \def\childdoctmp{#1}
      \ifx\childdoctmp\childdocname
        \def\childdoctmp{}
      \else
        \def\childdoctmp
        {
          \childdoctrue
          \includeonly{\childdocname}
          \def\childdocjob{#1}
          \def\jobname{#1}
        }
      \fi
      \expandafter
    \endgroup
    \childdoctmp
  \fi
}
%    \end{macrocode}

% \macro{\childdocof}
% The command |\childdocof| redirects
% compilation to the main file |#1|.
%    \begin{macrocode}
\newcommand{\childdocof}[1]
{
  \childdocdisable
  \childdoctrue
  \includeonly{\childdocname}
  \def\jobname{#1}
  \def\childdocjob{#1}
  \input{#1}
}
%    \end{macrocode}

% \macro{\childdocby}
% The command |\childdocby| ....
%    \begin{macrocode}
\newcommand{\childdocby}[2][]
{
  \childdocdisable
  \childdoctrue
  \childdocmanualtrue
  \if?#1?\else
    \def\jobname{#2}
  \fi
  \def\childdocjob{#2}
  \input{#2}
  \endinput
}
%    \end{macrocode}

% \macro{\childdocforward}
% The command |\childdocforward| redirects
% compilation to the main file or
% (if the optional argument is given) a child file.
% Parameters are set as if the main file
% or a child file starting with |\childdocof| was compiled.
% Then compilation is handed over to the main file:
%    \begin{macrocode}
\newcommand{\childdocforward}[2][]
{
  \begingroup
    \if?#1?
      \def\childdoctmp
      {
        \def\childdocname{#2}
        \def\childdocjob{#2}
        \def\jobname{#2}
        \input{#2}
        \endinput
      }
    \else
      \def\childdoctmp
      {
        \childdocdisable
        \def\childdocname{#2}
        \childdoctrue
        \includeonly{#2}
        \def\childdocjob{#1}
        \def\jobname{#1}
        \input{#1}
        \endinput
      }
    \fi
    \expandafter
  \endgroup
  \childdoctmp
}
%    \end{macrocode}

% \macro{\childdocforwardprefix}
% The command |\childdocforwardprefix| redirects
% compilation to the main or a child file by means of a pattern.
% The prefix |#1| in the current filename is replaced by |#2|
% and the suffix of the current filename is kept
% (it is assumed that the filename does not contain the substring `|~~~|'
% which is used as a delimiter).
% Compilation is handed over to the new file by |\childdocforward|:
%    \begin{macrocode}
\newcommand{\childdocforwardprefix}[3][]
{
  \begingroup
    \def\childdocextract #2##1~~~{\def\childdoctmp{\childdocforward[#1]{#3##1}}}
    \expandafter\childdocextract\childdocname~~~
    \expandafter
  \endgroup
  \childdoctmp
}
%    \end{macrocode}

% \macro{\childdoc}
% The deprecated macro |\childdoc| is a legacy version of |\childdocmain|:
%    \begin{macrocode}
\newcommand{\childdoc}{\childdocmain}
%    \end{macrocode}

% \macro{\childdocredirect}
% The deprecated macro |\childdocredirect| is a legacy version
% of |\childdocforward| and |\childdocforwardprefix|:
%    \begin{macrocode}
\newcommand{\childdocredirect}[2][]
{
  \begingroup
    \if?#1?
      \def\childdoctmp{\childdocforward{#2}}
    \else
      \def\childdoctmp{\childdocforwardprefix{#1}{#2}}
    \fi
    \expandafter
  \endgroup
  \childdoctmp
}
%    \end{macrocode}

%\iffalse
%</package>
%\fi
%
\endinput
|\\
|\childdocby{|\textit{main}|}|\\
\end{tabular}
\end{center}
%
Both forms have slightly different effects as described above.
The main file is prepared as usual, see \secref{sec:include}.

%%%%%%%%%%%%%%%%%%%%%%%%%%%%%%%%%%%%%%%%%%%%%%%%%%%%%%%%%%%%%%%%%%%%%%%%%%%%%%%%
\subsection{Legacy Detection}
\label{sec:detection}

The directive |\childdocmain| in the main file can detect
whether the complete document or merely a child is to be compiled
even without using the directive |\childdocof|.
This method is deprecated because it is less robust
and there is no compelling reason to use it;
it is merely provided for backward compatibility
and it may be removed in future versions.

If the detection mechanism is to be used,
it is mandatory to correctly specify
the filename of the main file as the argument of |\childdocmain|:
%
\begin{center}
\begin{tabular}{l}
|% \iffalse
%
% childdoc.dtx Copyright (C) 2017-2018 Niklas Beisert
%
% This work may be distributed and/or modified under the
% conditions of the LaTeX Project Public License, either version 1.3
% of this license or (at your option) any later version.
% The latest version of this license is in
%   http://www.latex-project.org/lppl.txt
% and version 1.3 or later is part of all distributions of LaTeX
% version 2005/12/01 or later.
%
% This work has the LPPL maintenance status `maintained'.
%
% The Current Maintainer of this work is Niklas Beisert.
%
% This work consists of the files childdoc.dtx and childdoc.ins
% and the derived files childdoc.def and cdocsamp.tex with
% cdocsch1.tex, cdocsch2.tex, cdocsdrf.tex, cdocsfn1.tex, cdocsfn2.tex.
%
%<package>\ifdefined\childdocmain\endinput\fi
%<package>\ProvidesFile{childdoc.def}[2018/12/30 v2.0 child document driver]
%<samplemain>\ProvidesFile{cdocsamp.tex}[2018/12/30 v2.0 sample for childdoc]
%<*driver>
%\ProvidesFile{childdoc.drv}[2018/12/30 v2.0 childdoc reference manual file]
\PassOptionsToClass{10pt,a4paper}{article}
\documentclass{ltxdoc}

\usepackage[margin=35mm]{geometry}
\usepackage{hyperref}
\usepackage{hyperxmp}
\usepackage[usenames]{color}

\hypersetup{colorlinks=true}
\hypersetup{pdfstartview=FitH}
\hypersetup{pdfpagemode=UseNone}
\hypersetup{pdfsource={}}
\hypersetup{pdflang={en-UK}}
\hypersetup{pdfcopyright={Copyright 2017-2018 Niklas Beisert.
  This work may be distributed and/or modified under the
  conditions of the LaTeX Project Public License, either version 1.3
  of this license or (at your option) any later version.}}
\hypersetup{pdflicenseurl={http://www.latex-project.org/lppl.txt}}
\hypersetup{pdfcontactaddress={ETH Zurich, ITP, HIT K,
  Wolfgang-Pauli-Strasse 27}}
\hypersetup{pdfcontactpostcode={8093}}
\hypersetup{pdfcontactcity={Zurich}}
\hypersetup{pdfcontactcountry={Switzerland}}
\hypersetup{pdfcontactemail={nbeisert@itp.phys.ethz.ch}}
\hypersetup{pdfcontacturl={http://people.phys.ethz.ch/\xmptilde nbeisert/}}

\newcommand{\secref}[1]{\hyperref[#1]{section \ref*{#1}}}

\parskip1ex
\parindent0pt
\let\olditemize\itemize
\def\itemize{\olditemize\parskip0pt}

\begin{document}

\title{The \textsf{childdoc} Package}
\hypersetup{pdftitle={The childdoc Package}}
\author{Niklas Beisert\\[2ex]
  Institut f\"ur Theoretische Physik\\
  Eidgen\"ossische Technische Hochschule Z\"urich\\
  Wolfgang-Pauli-Strasse 27, 8093 Z\"urich, Switzerland\\[1ex]
  \href{mailto:nbeisert@itp.phys.ethz.ch}
  {\texttt{nbeisert@itp.phys.ethz.ch}}}
\hypersetup{pdfauthor={Niklas Beisert}}
\hypersetup{pdfsubject={Manual for the LaTeX2e Package childdoc}}
\date{30 December 2018, \textsf{v2.0}}
\maketitle

\begin{abstract}\noindent
\textsf{childdoc} is a \LaTeXe{} package
that enables the direct compilation
of document sections included by |\include|
to individual files.
\end{abstract}

\begingroup
\parskip0ex
\tableofcontents
\endgroup

%%%%%%%%%%%%%%%%%%%%%%%%%%%%%%%%%%%%%%%%%%%%%%%%%%%%%%%%%%%%%%%%%%%%%%%%%%%%%%%%
%%%%%%%%%%%%%%%%%%%%%%%%%%%%%%%%%%%%%%%%%%%%%%%%%%%%%%%%%%%%%%%%%%%%%%%%%%%%%%%%
\section{Introduction}

\LaTeX{} provides a mechanism to structure a large document (such as a book)
into a main file and several child files (containing the chapters)
using the |\include| command.
This mechanism is beneficial for documents
which span hundreds of pages in order to
make the source file(s) more manageable.
Moreover, compilation can be restricted to
selected child files by means of the |\includeonly| command.
The latter feature can be used to reduce the compilation time while editing
(this was significantly more useful in the earlier days of \LaTeX{})
or to generate a smaller document which is easier to navigate.
Another application of |\includeonly| is to generate
documents consisting of selected parts of the complete document.

However, there are a few drawbacks of the plain |\include| mechanism:
\begin{itemize}
\item
The child files cannot be compiled on their own,
they can only be compiled via the main file.
A naive editing environment
(such as a text editor with an option
to have the current file processed by \LaTeX)
may require one to switch to the main file before compiling;
attempting to compile the child file produces errors.
\item
The main file must be modified (each time)
to adjust the |\includeonly| command
to the present needs. This easily leaves the main file in a messy state.
\item
The generated document will always carry the filename
of the main document. This is inconvenient if
several child files are to be compiled and
to be kept for distribution.
\end{itemize}

The present package provides a simple interface
to make child files individually compilable by \LaTeX{}.
Compiling a child file then has the same effect as compiling
the main file with an |\includeonly| command
to select the appropriate child.
Moreover the generated document will carry the name of the child
rather than the main file.
This resolves all three above issues.

This feature is meant to make the editing of books,
thesis documents and lecture notes somewhat more convenient.
However, the package can also be used efficiently for
composing a series of documents (such as exercise sheets)
which are typically distributed individually.
It then assists the author in generating the individual documents
(potentially in different versions)
as well as a document containing the collected series.
Another application is in developing style files
or other kinds of included material
where compilation of the style file could redirect
to a sample or test file.

%%%%%%%%%%%%%%%%%%%%%%%%%%%%%%%%%%%%%%%%%%%%%%%%%%%%%%%%%%%%%%%%%%%%%%%%%%%%%%%%
%%%%%%%%%%%%%%%%%%%%%%%%%%%%%%%%%%%%%%%%%%%%%%%%%%%%%%%%%%%%%%%%%%%%%%%%%%%%%%%%
\section{Usage}

First of all, the package \textsf{childdoc} is \emph{not} a standard
\LaTeXe{} |.sty| style file! Therefore it needs to be invoked in
a non-standard way.

%%%%%%%%%%%%%%%%%%%%%%%%%%%%%%%%%%%%%%%%%%%%%%%%%%%%%%%%%%%%%%%%%%%%%%%%%%%%%%%%
\subsection{Included Files}
\label{sec:include}

%%%%%%%%%%%%%%%%%%%%%%%%%%%%%%%%%%%%%%%%
\DescribeMacro{\childdocmain}
To use the package, add the commands
\begin{center}
\begin{tabular}{l}
|\input{childdoc.def}|\\
|\childdocmain{}|\\
\end{tabular}
\end{center}
at the very top of the main \LaTeX{} file,
in particular \emph{before} the |\documentclass| statement!
The argument of |\childdocmain| should be left empty
(but it must be present).

%%%%%%%%%%%%%%%%%%%%%%%%%%%%%%%%%%%%%%%%
\DescribeMacro{\childdocof}
Furthermore, add the commands
\begin{center}
\begin{tabular}{l}
|\input{childdoc.def}|\\
|\childdocof{|\textit{main}|}|\\
\end{tabular}
\end{center}
at the top of every child file \textit{child}
which is included by |\include{|\textit{child}|}|
from within the main file
(or at least for those files to be compiled individually).
The argument \textit{main} must be the filename of the main file.

There are a couple of
considerations in setting up the main and child documents:

%%%%%%%%%%%%%%%%%%%%%%%%%%%%%%%%%%%%%%%%
\paragraph{Restrictions.}

Please note the following restrictions:
\begin{itemize}
\item
|\childdocmain| must be called with one argument \textit{main}
to ensure compatibility with earlier version of the package.
It must either be empty (|\childdocmain{}|)
or precisely match the filename of the main file in which it is specified.
See \secref{sec:detection} for further information.
\item
The filename \textit{main} must be specified without the |.tex| extension.
\item
The filename \textit{main} is case sensitive
(even in case-insensitive file systems)
due to internal string comparison.
\item
The argument \textit{main} should be fully expanded, it cannot be a macro.
\item
Subdirectories and special characters should be avoided in filenames.
\item
The command |\childdocmain{|\textit{main}|}| must be followed by a whitespace.
It should not be followed immediately by another command
or by a comment mark `|%|'.
This is because the \TeX{} parser reads the token immediately following
the argument of |\childdocmain| and puts it
at the beginning of every child section;
however, a white\-space is ignored.
\end{itemize}

%%%%%%%%%%%%%%%%%%%%%%%%%%%%%%%%%%%%%%%%
\paragraph{Content of Main File.}

It is advisable to place all content in the child files included by |\include|.
Any output contained in the main file will appear in all child documents
unless suppressed manually;
it cannot be suppressed automatically by the |\includeonly| directive
and thus should normally be avoided.
A method to include some content in the main file
by means of conditional processing is described in \secref{sec:conditional}.

%%%%%%%%%%%%%%%%%%%%%%%%%%%%%%%%%%%%%%%%
\paragraph{Page Numbering.}

When only a part of the document is compiled,
the appropriate numbering of pages
(as well as other status parameters)
is determined from the |.aux| files.
The latter contain information from previous passes.
However this information needs to propagate through
all intermediate child documents.
Therefore the page numbering in child documents may well
be inconsistent until the complete document is compiled at least once.

A useful (if unconventional) way to always ensure a consistent
page numbering is to restart the numbering in each child document
and denote the pages by `\textit{child}|.|\textit{page}'
where \textit{child} represents the chapter/section number of the child file.
This can be achieved by the command
|\numberwithin{page}{|\textit{child}|}|
of the \textsf{amsmath} package
where \textit{child} can be |chapter| or |section|
depending on the chosen structuring.
Alternatively, one can modify the macro |\thepage| appropriately
and reset the counter |page| at the start of each child file.

%%%%%%%%%%%%%%%%%%%%%%%%%%%%%%%%%%%%%%%%%%%%%%%%%%%%%%%%%%%%%%%%%%%%%%%%%%%%%%%%
\subsection{Conditional Processing}
\label{sec:conditional}

The package provides a mechanism to compile different versions
of a document. To customise the versions further some conditional processing
can come in handy to distinguish which version is being compiled.
The package provides two macros to describe the compilation context:

%%%%%%%%%%%%%%%%%%%%%%%%%%%%%%%%%%%%%%%%
\DescribeMacro{\ifchilddoc}
The conditional |\ifchilddoc| distinguishes between the compilation of
child documents and the main document:
%
\begin{center}
|\ifchilddoc |\textit{child-code}| |[|\||else |\textit{main-code}]| \||fi|
\end{center}

%%%%%%%%%%%%%%%%%%%%%%%%%%%%%%%%%%%%%%%%
\DescribeMacro{\childdocname}
\DescribeMacro{\childdocjob}
The macro |\childdocname| contains the filename (without extension)
of the main or child file being processed.
Note that |\childdocjob| will always contain the name of the main file.

%%%%%%%%%%%%%%%%%%%%%%%%%%%%%%%%%%%%%%%%
\paragraph{Title Page.}

Conditional processing can be used to include a title or banner page
in the main document when proper precautions are taken.
Importantly, the code in the main file should ensure that the page counter
(as well as other status parameters which are stored in the |.aux| files)
takes the same value after the conditional processing.
Otherwise the page numbers may take divergent values
depending on which part is compiled.

For example, a title page could be declared by:
%
\begin{center}
\begin{tabular}{l}
|\ifchilddoc\||else|\\
|\addtocounter{page}{-1}|\\
\textit{code for title page}\\
|\newpage|\\
|\||fi|
\end{tabular}
\end{center}
%
A banner page for the child documents can be generated by:
%
\begin{center}
\begin{tabular}{l}
|\ifchilddoc|\\
|\addtocounter{page}{-1}|\\
\textit{code for banner page}\\
|\newpage|\\
|\||fi|
\end{tabular}
\end{center}
%
Here one could write a message such as:
\begin{center}
|This is the part \childdocname{} of \childdocjob{}.|
\end{center}

%%%%%%%%%%%%%%%%%%%%%%%%%%%%%%%%%%%%%%%%%%%%%%%%%%%%%%%%%%%%%%%%%%%%%%%%%%%%%%%%
\subsection{Flags}
\label{sec:flags}

The package makes it easy to generate different versions
of the main or child documents.
To this end compilation flags can be defined
and assigned different default values.
They will be particularly useful in conjunction
with the forwarding mechanism described in \secref{sec:forward}.

For example, it may be useful to have a flag |\version|
which can be set to |draft| or |final|.
The document source will contain some conditional code
depending on the value of |\version|.
Suppose further, the flag should default to |final| for the main file
and to |draft| for child files
which is a natural assignment for editing the document.
This is achieved by placing the following code
in the preamble of the main document
(below the |\childdocmain| directive):
%
\begin{center}
\begin{tabular}{l}
|\ifchilddoc|\\
|\providecommand{\version}{draft}|\\
|\||else|\\
|\providecommand{\version}{final}|\\
|\||fi|
\end{tabular}
\end{center}
%
The definition by |\providecommand| makes sure
that previous definitions are not overwritten.
Further statements |\providecommand{\version}{...}|
can thus be added before the above code to override it.

For the main file, one might add a line
(between |\childdocmain| and the above block)
%
\begin{center}
|%\ifchilddoc\||else\providecommand{\version}{draft}\||fi|
\end{center}
%
which can be uncommented to produce a draft version.
Likewise one can add a line to the very top of a child file
(above the |\childdocof{|\textit{main}|}| directive)
%
\begin{center}
|%\providecommand{\version}{final}|
\end{center}
%
which can be uncommented to produce the final version of this child document.

%%%%%%%%%%%%%%%%%%%%%%%%%%%%%%%%%%%%%%%%%%%%%%%%%%%%%%%%%%%%%%%%%%%%%%%%%%%%%%%%
\subsection{Forwarding}
\label{sec:forward}

Different versions of the main or child documents
using compilation flags as described in \secref{sec:flags}
can be (permanently) stored in different files
for convenient compilation, viewing and distribution.
To this end, the package defines a command
to pass on compilation to a different file:

%%%%%%%%%%%%%%%%%%%%%%%%%%%%%%%%%%%%%%%%
\DescribeMacro{\childdocforward}
The command |\childdocforward| redirects processing to
another source file:
%
\begin{center}
\begin{tabular}{l}
|\input{childdoc.def}|\\
|\childdocforward[|\textit{main}|]{|\textit{dest}|}|\\
\end{tabular}
\end{center}
%
The argument \textit{dest} is the destination file
(without extension).
It should be the main file or one of the child files.
Note that further \textsf{childdoc} directives
such as |\childdocof| and |\childdocforward|
in the indicated file will be processed in this form.
The optional argument \textit{main}
passes on directly to the main file \textit{main}
while pretending to compile the child \textit{dest}.
This form behaves as if \textit{dest}
issues |\childdocof{|\textit{main}|}| right away,
and no further \textsf{childdoc} directives will be processed.

%%%%%%%%%%%%%%%%%%%%%%%%%%%%%%%%%%%%%%%%
\DescribeMacro{\...prefix}
In the alternative form |\childdocforwardprefix|,
%
\begin{center}
\begin{tabular}{l}
|\input{childdoc.def}|\\
|\childdocforwardprefix[|\textit{main}|]{|\textit{prefix}|}{|\textit{dest}|}|
\end{tabular}
\end{center}
%
the destination file is determined by a pattern
depending on the current file:
To make this work, the current file must be called
`{\textit{prefix}\hspace{0.2em}\textit{suffix}}'
with \textit{prefix} matching precisely the argument.
Processing is then passed on to the file
`{\textit{dest}\hspace{0.2em}\textit{suffix}}'.
Surely, the same effect is achieved by
directly specifying the
argument `{\textit{dest}\hspace{0.2em}\textit{suffix}}'
in the first form.
However, that requires to set up a different file
for each child. With the alternative form of the command
all these files can have exactly the same content
which simplifies setting them up and maintaining them.

For example, the following file |draft.tex|
with a compilation flag |\version| as described in \secref{sec:flags}
compiles the main document as a draft:
%
\begin{center}
\begin{tabular}{l}
|\def\version{draft}|\\
|\input{childdoc.def}|\\
|\childdocforward{|\textit{main}|}|
\end{tabular}
\end{center}
%
Likewise, the following files |final|\textit{nn}|.tex|
compile the final version of the child document
|child|\textit{nn}|.tex|:
%
\begin{center}
\begin{tabular}{l}
|\def\version{final}|\\
|\input{childdoc.def}|\\
|\childdocforwardprefix{final}{child}|
\end{tabular}
\end{center}
%

Note that when several versions of a main file and/or of each child file
are to be generated, it may be convenient to set up a |Makefile| or
shell script to automatise the process.

%%%%%%%%%%%%%%%%%%%%%%%%%%%%%%%%%%%%%%%%%%%%%%%%%%%%%%%%%%%%%%%%%%%%%%%%%%%%%%%%
\subsection{Command Line Processing}
\label{sec:commandline}

The effect of redirection files can also be achieved by invoking
the \LaTeX{} compiler with a more elaborate command line.
Most conveniently this should be done as part
of a shell script or a |Makefile|.

When using \textsf{childdoc} in the main file, the following
command lines effectively perform a redirection
(note that depending on the shell being used,
backslashes may have to be doubled: `|\|' $\to$ `|\\|'):
%
\begin{center}
|... -jobname "|\textit{target}|" |\\|"|[\textit{flags}]%
|\input{childdoc.def}\childdocforward[|\textit{main}|]{|\textit{dest}|}"|
\end{center}
%
Here \textit{target} is the name of the output file,
\textit{main} is the name of the main file
and \textit{dest} is the name of the main or child file to be processed
(all filenames without extensions).
The optional argument \textit{main} can be omitted
if \textit{main} matches \textit{dest}.
Optionally, compilation \textit{flags} can be defined via |\def| commands.
This command line makes the \TeX{} engine believe
it is compiling the file \textit{target}
whose content is specified as the latter parameter.
The provided code then forwards the processing to
\textit{main} or \textit{dest} as described in \secref{sec:forward}.

%%%%%%%%%%%%%%%%%%%%%%%%%%%%%%%%%%%%%%%%%%%%%%%%%%%%%%%%%%%%%%%%%%%%%%%%%%%%%%%%
\subsection{Include by Input}
\label{sec:input}

Including child documents by |\include| has some restrictions by design.
Most notably, the content of a child document always occupies
its own set of pages; pages cannot be shared between child documents.
Usually, this behaviour makes perfect sense
because each child document contain an essential part of the document.
However, in some situations it may be desirable to compose
a document from a collection of parts
without having mandatory page breaks between then.
For this case, the package
provides a mechanism to include parts
by |\input| which can also be processed individually.
However, by construction this mechanism
requires manual handling of the content to be output.

%%%%%%%%%%%%%%%%%%%%%%%%%%%%%%%%%%%%%%%%
\DescribeMacro{\ifchilddocmanual}
The main file should be prepared as usual, see \secref{sec:include}.
However, the document body must make a distinction
between processing of an individual part and of the main document, e.g.:
%
\begin{center}
\begin{tabular}{l}
|\ifchilddocmanual|\\
|\input{\childdocname}|\\
|\||else|\\
\textit{document body with }|\input{|\textit{part}|}|\\
|\||fi|
\end{tabular}
\end{center}
%
The conditional |\ifchilddocmanual| is true whenever
a part to be included by |\input| is being compiled,
and the name of the part is stored in |\childdocname|.

%%%%%%%%%%%%%%%%%%%%%%%%%%%%%%%%%%%%%%%%
\DescribeMacro{\childdocby}
Each part to be included by |\input| should start with:
%
\begin{center}
\begin{tabular}{l}
|\input{childdoc.def}|\\
|\childdocby{|\textit{main}|}|\\
\end{tabular}
\end{center}
%
The directive |\childdocby| is similar to |\childdocof|
described in \secref{sec:include},
but the subsequent selection of content must be done manually.
To that end, both |\ifchilddoc| and |\ifchilddocmanual|
will be true upon processing of a part,
and the name of the part is stored in |\childdocname|.
Note that |\jobname| will be set to the filename of the current part
so that each part receives an individual |.aux| file
that does not interfere with the |.aux| file(s) of the main document.
This behaviour can be altered by the alternative form
|\childdocby[*]{|\textit{main}|}| (with a non-empty optional argument)
which uses the |.aux| file of the main document
by setting |\jobname| to \textit{main}.

%%%%%%%%%%%%%%%%%%%%%%%%%%%%%%%%%%%%%%%%%%%%%%%%%%%%%%%%%%%%%%%%%%%%%%%%%%%%%%%%
\subsection{Driver Development}
\label{sec:driver}

The \textsf{childdoc} mechanism can also be use for the development
of definition files such as \LaTeX{} styles or classes.
This case differs from the above setup with multiple parts
included by |\include| in that no |\includeonly| should be invoked.
This can be achieved by starting the include file
(before |\ProvidesPackage|) with:
%
\begin{center}
\begin{tabular}{l}
|\input{childdoc.def}|\\
|\childdocforward{|\textit{main}|}|\\
\end{tabular}
\end{center}
%
or alternatively with:
%
\begin{center}
\begin{tabular}{l}
|\input{childdoc.def}|\\
|\childdocby{|\textit{main}|}|\\
\end{tabular}
\end{center}
%
Both forms have slightly different effects as described above.
The main file is prepared as usual, see \secref{sec:include}.

%%%%%%%%%%%%%%%%%%%%%%%%%%%%%%%%%%%%%%%%%%%%%%%%%%%%%%%%%%%%%%%%%%%%%%%%%%%%%%%%
\subsection{Legacy Detection}
\label{sec:detection}

The directive |\childdocmain| in the main file can detect
whether the complete document or merely a child is to be compiled
even without using the directive |\childdocof|.
This method is deprecated because it is less robust
and there is no compelling reason to use it;
it is merely provided for backward compatibility
and it may be removed in future versions.

If the detection mechanism is to be used,
it is mandatory to correctly specify
the filename of the main file as the argument of |\childdocmain|:
%
\begin{center}
\begin{tabular}{l}
|\input{childdoc.def}|\\
|\childdocmain{|\textit{main}|}|\\
\end{tabular}
\end{center}
%
If |\jobname| does not match the argument \textit{main} of |\childdocmain|,
it is assumed that |\jobname| points to the child file to be compiled.
When using |\childdocmain| with the main file specified as argument,
it suffices to start a child file
with just |\input{|\textit{main}|}|
without loading of the package and using |\childdocof|.
If instead all processing is done
with the appropriate \textsf{childdoc} directives,
the argument of \textit{main} of |\childdocmain| can be empty.

An alternative version of the command line processing described
in \secref{sec:commandline} using the detection mechanism reads:
%
\begin{center}
|... -jobname "|\textit{target}|" "|[\textit{flags}]%
[|\def\jobname{|\textit{dest}|}|]|\input{|\textit{main}|}"|
\end{center}

%%%%%%%%%%%%%%%%%%%%%%%%%%%%%%%%%%%%%%%%%%%%%%%%%%%%%%%%%%%%%%%%%%%%%%%%%%%%%%%%
\subsection{Manual Code}
\label{sec:manual}

In case one cannot be certain whether the definitions file |childdoc.def|
is installed on the target \TeX{} distribution
and one prefers not to ship it,
it is conceivable to paste a few relevant commands into the sources.

To that end, drop all statements |\input{childdoc.def}|
and perform the replacements as outlined below.
Instead of |\childdocmain{|\textit{main}|}| add the following code
to the top of the main file:
%
\begin{center}
\begin{tabular}{l}
|\||ifdefined\childdocname\endinput\||fi\newif\ifchilddoc|\\
|\edef\childdocname{\scantokens\expandafter{\jobname\noexpand}}|\\
|\def\childdocmain{|\textit{main}|}\||ifx\childdocmain\childdocname\||else|\\
|\childdoctrue\includeonly{\childdocname}\let\jobname\childdocmain\||fi|\\
\end{tabular}
\end{center}
%
Instead of |\childdocof{|\textit{main}|}| just include the main file
at the top of each child file:
%
\begin{center}
|\input{|\textit{main}|}|
\end{center}
%
A simple redirection |\childdocforward{|\textit{dest}|}| is achieved by:
%
\begin{center}
|\def\jobname{|\textit{dest}|}\input{\jobname}|
\end{center}
%
The redirection with prefix
|\childdocforwardprefix[|\textit{prefix}|]{|\textit{dest}|}|
is accomplished by:
%
\begin{center}
\begin{tabular}{l}
|{\edef\jobname{\scantokens\expandafter{\jobname\noexpand}}|\\
|\def\redirectjob |\textit{prefix}|#1~~~{\gdef\jobname{|\textit{dest}|#1}}|\\
|\expandafter\redirectjob\jobname~~~}\input{\jobname}|
\end{tabular}
\end{center}

In an alternative approach,
child documents can be compiled by a specific command line
without additional code or specific definitions:
%
\begin{center}
|... -jobname "|\textit{target}|" "|[\textit{flags}]%
|\includeonly{|\textit{dest}|}\input{|\textit{main}|}"|
\end{center}
%

%%%%%%%%%%%%%%%%%%%%%%%%%%%%%%%%%%%%%%%%%%%%%%%%%%%%%%%%%%%%%%%%%%%%%%%%%%%%%%%%
%%%%%%%%%%%%%%%%%%%%%%%%%%%%%%%%%%%%%%%%%%%%%%%%%%%%%%%%%%%%%%%%%%%%%%%%%%%%%%%%
\section{Information}

%%%%%%%%%%%%%%%%%%%%%%%%%%%%%%%%%%%%%%%%%%%%%%%%%%%%%%%%%%%%%%%%%%%%%%%%%%%%%%%%
\subsection{Copyright}

Copyright \copyright{} 2017--2018 Niklas Beisert

This work may be distributed and/or modified under the
conditions of the \LaTeX{} Project Public License, either version 1.3
of this license or (at your option) any later version.
The latest version of this license is in
  \url{http://www.latex-project.org/lppl.txt}
and version 1.3 or later is part of all distributions of \LaTeX{}
version 2005/12/01 or later.

This work has the LPPL maintenance status `maintained'.

The Current Maintainer of this work is Niklas Beisert.

This work consists of the files |README.txt|, |childdoc.ins| and |childdoc.dtx|
as well as the derived files |childdoc.def|, |cdocsamp.tex|
with |cdocsch1.tex|, |cdocsch2.tex|, |cdocspt3.tex|, |cdocspt4.tex|,
|cdocsdrf.tex|, |cdocsfn1.tex|, |cdocsfn2.tex|
as well as |childdoc.pdf|.

%%%%%%%%%%%%%%%%%%%%%%%%%%%%%%%%%%%%%%%%%%%%%%%%%%%%%%%%%%%%%%%%%%%%%%%%%%%%%%%%
\subsection{Files and Installation}

The package consists of the files:
%
\begin{center}
\begin{tabular}{ll}
    |README.txt|   & readme file \\
    |childdoc.ins| & installation file \\
    |childdoc.dtx| & source file \\
    |childdoc.def| & definition file \\
    |cdocsamp.tex| & sample main file \\
    |cdocsch1.tex| & sample include file \\
    |cdocsch2.tex| & sample include file \\
    |cdocspt3.tex| & sample part file \\
    |cdocspt4.tex| & sample part file \\
    |cdocsdrf.tex| & sample redirection file \\
    |cdocsfn1.tex| & sample redirection file \\
    |cdocsfn2.tex| & sample redirection file \\
    |childdoc.pdf| & manual
\end{tabular}
\end{center}
%
The distribution consists of the files
|README.txt|, |childdoc.ins| and |childdoc.dtx|.
%
\begin{itemize}
\item
Run (pdf)\LaTeX{} on |childdoc.dtx|
to compile the manual |childdoc.pdf| (this file).
\item
Run \LaTeX{} on |childdoc.ins| to create the definitions file |childdoc.def|
and the sample |cdocsamp.tex| with include files
|cdocsch1.tex|, |cdocsch2.tex|, |cdocspt3.tex|, |cdocspt4.tex|,
|cdocsdrf.tex|, |cdocsfn1.tex|, |cdocsfn2.tex|.
Then copy the file |childdoc.def| to an appropriate directory of your \LaTeX{}
distribution, e.g.\ \textit{texmf-root}|/tex/latex/childdoc|.
\end{itemize}

%%%%%%%%%%%%%%%%%%%%%%%%%%%%%%%%%%%%%%%%%%%%%%%%%%%%%%%%%%%%%%%%%%%%%%%%%%%%%%%%
\subsection{Related CTAN Packages}

There are several other packages which offer a similar functionality:
%
\begin{itemize}
\item
The packages
\href{http://ctan.org/pkg/docmute}{\textsf{docmute}},
\href{http://ctan.org/pkg/includex}{\textsf{includex}} and
\href{http://ctan.org/pkg/standalone}{\textsf{standalone}}
provide commands to include only the document body of
a child file thus allowing both files to be compiled individually.
\item
The packages \href{http://ctan.org/pkg/subdocs}{\textsf{subdocs}}
and \href{http://ctan.org/pkg/subfiles}{\textsf{subfiles}}
provide structures in which the main and child documents can be
encapsulated and allowing them to be compiled individually.
The inclusion mechanism is different from the conventional |\include|.
\item
The package \href{http://ctan.org/pkg/combine}{\textsf{combine}}
is an elaborate solution to combine several documents into one.
\end{itemize}
%
See also the CTAN topic \href{http://ctan.org/topic/subdocs}{\textsf{subdocs}}
for further related packages.
The present package differs from the above solutions in that
a document structure constructed with the conventional |\include| mechanism
just needs two extra commands at the top of every file
such that all constituent files can be compiled individually.

%%%%%%%%%%%%%%%%%%%%%%%%%%%%%%%%%%%%%%%%%%%%%%%%%%%%%%%%%%%%%%%%%%%%%%%%%%%%%%%%
%\subsection{Feature Suggestions}
%
%The following is a list of features which may be useful for future
%versions of this package:
%%
%\begin{itemize}
%\item
%\ldots
%\end{itemize}

%%%%%%%%%%%%%%%%%%%%%%%%%%%%%%%%%%%%%%%%%%%%%%%%%%%%%%%%%%%%%%%%%%%%%%%%%%%%%%%%
\subsection{Revision History}

%%%%%%%%%%%%%%%%%%%%%%%%%%%%%%%%%%%%%%%%
\paragraph{v2.0:} 2018/12/30

\begin{itemize}
\item
immediate forward processing
\item
added |\childdocby| mechanism
\item
manual restructured
\end{itemize}

%%%%%%%%%%%%%%%%%%%%%%%%%%%%%%%%%%%%%%%%
\paragraph{v1.6:} 2018/01/17

\begin{itemize}
\item
application for development of include files
\item
corrections to manual
\end{itemize}

%%%%%%%%%%%%%%%%%%%%%%%%%%%%%%%%%%%%%%%%
\paragraph{v1.5:} 2017/05/21

\begin{itemize}
\item
more complete structuring introduced
\item
|\childdocof| introduced
\item
|\childdoc| renamed to |\childdocmain|
\item
|\childredirect| renamed to |\childdocforward| and |\childdocforwardprefix|
and functionality expanded
\end{itemize}

%%%%%%%%%%%%%%%%%%%%%%%%%%%%%%%%%%%%%%%%
\paragraph{v1.0:} 2017/04/27

\begin{itemize}
\item
manual and install package
\item
first version published on CTAN
\end{itemize}

%%%%%%%%%%%%%%%%%%%%%%%%%%%%%%%%%%%%%%%%
\paragraph{v0.6:} 2017/04/26

\begin{itemize}
\item
redirection mechanism added
\end{itemize}

%%%%%%%%%%%%%%%%%%%%%%%%%%%%%%%%%%%%%%%%
\paragraph{v0.5:} 2017/04/26

\begin{itemize}
\item
functionality in definition file
\end{itemize}


%%%%%%%%%%%%%%%%%%%%%%%%%%%%%%%%%%%%%%%%%%%%%%%%%%%%%%%%%%%%%%%%%%%%%%%%%%%%%%%%
%%%%%%%%%%%%%%%%%%%%%%%%%%%%%%%%%%%%%%%%%%%%%%%%%%%%%%%%%%%%%%%%%%%%%%%%%%%%%%%%
%%%%%%%%%%%%%%%%%%%%%%%%%%%%%%%%%%%%%%%%%%%%%%%%%%%%%%%%%%%%%%%%%%%%%%%%%%%%%%%%
\appendix

\settowidth\MacroIndent{\rmfamily\scriptsize 000\ }

 \DocInput{childdoc.dtx}

\end{document}
%</driver>
% \fi
%
% %%%%%%%%%%%%%%%%%%%%%%%%%%%%%%%%%%%%%%%%%%%%%%%%%%%%%%%%%%%%%%%%%%%%%%%%%%%%%%
% %%%%%%%%%%%%%%%%%%%%%%%%%%%%%%%%%%%%%%%%%%%%%%%%%%%%%%%%%%%%%%%%%%%%%%%%%%%%%%
% \section{Sample}
%\iffalse
%<*samplemain>
%\fi
%
% The following presents a sample document
% with two chapters, two parts, a title page,
% a compile flag as well as three forwarding files to set the flag.
% It consists of eight |.tex| files:
% \begin{center}
% \begin{tabular}{ll}
% |cdocsamp.tex|&main file\\
% |cdocsch1.tex|&include file for chapter 1\\
% |cdocsch2.tex|&include file for chapter 2\\
% |cdocspt3.tex|&include file for part 3\\
% |cdocspt4.tex|&include file for part 4\\
% |cdocsdrf.tex|&forwarding file for main file in draft mode\\
% |cdocsfi1.tex|&forwarding file for final version of chapter 1\\
% |cdocsfi2.tex|&forwarding file for final version of chapter 2\\
% \end{tabular}
% \end{center}
% Each of the eight files can be compiled directly by the \LaTeX{} compiler.
%
% %%%%%%%%%%%%%%%%%%%%%%%%%%%%%%%%%%%%%%
% \paragraph{Main File.}
%
% The main file is called |cdocsamp.tex|.
%
% Load the \textsf{childdoc} definitions and
% declare the filename for the main document:
%    \begin{macrocode}
\input{childdoc.def}
\childdocmain{}
%    \end{macrocode}

% Optional override for |\version| flag:
%    \begin{macrocode}
%%\ifchilddoc\else\providecommand{\version}{draft}\fi
%    \end{macrocode}

% Define the default values for the |\version| flag
% (|final| for the main file and |draft| for childs):
%    \begin{macrocode}
\ifchilddoc
\providecommand{\version}{draft}
\else
\providecommand{\version}{final}
\fi
%    \end{macrocode}

% Load the standard document class:
%    \begin{macrocode}
\documentclass[12pt]{article}
%    \end{macrocode}

% Start the document body:
%    \begin{macrocode}
\begin{document}
%    \end{macrocode}

% Declare a title page.
% Print title, part of document being processed and version flag:
%    \begin{macrocode}
\addtocounter{page}{-1}
\begin{center}
{\LARGE\bfseries{}childdoc example\par}
\vspace{1cm}
\ifchilddoc
\ifchilddocmanual part\else chapter\fi:
`\childdocname' of `\childdocjob'\par
\else
main document: `\childdocjob'\par
\fi
version: \version\par
\end{center}
\newpage
%    \end{macrocode}

% Manually include selected file,
% otherwise process as usual:
%    \begin{macrocode}
\ifchilddocmanual
\section*{part `\childdocname'}
\input{\childdocname}
\else
%    \end{macrocode}

% Include the two chapters:
%    \begin{macrocode}
\include{cdocsch1}
\include{cdocsch2}
%    \end{macrocode}

% Include the two parts unless only chapters should be displayed:
%    \begin{macrocode}
\ifchilddoc\else
\section{part three}
\input{cdocspt3}
\section{part four}
\input{cdocspt4}
\fi
%    \end{macrocode}

% Process as usual until here:
%    \begin{macrocode}
\fi
%    \end{macrocode}

% End of document body:
%    \begin{macrocode}
\end{document}
%    \end{macrocode}
%\iffalse
%</samplemain>
%\fi
%
% %%%%%%%%%%%%%%%%%%%%%%%%%%%%%%%%%%%%%%
% \paragraph{Chapter Include Files.}
%
% The include files are called |cdocsch1.tex| and |cdocsch2.tex|.
%
%\iffalse
%<*samplechap1|samplechap2>
%\fi

% Optional override for |\version| flag:
%    \begin{macrocode}
%%\providecommand{\version}{final}
%    \end{macrocode}

% Include the main document:
%    \begin{macrocode}
\input{childdoc.def}
\childdocof{cdocsamp}
%    \end{macrocode}

%\iffalse
%</samplechap1|samplechap2>
%\fi
%
%\iffalse
%<*samplechap1>
%\fi
% Some text for chapter 1:
%    \begin{macrocode}
\section{one}
some text in chapter one
%    \end{macrocode}

%\iffalse
%</samplechap1>
%\fi
% Some text for chapter 2:
%\iffalse
%<*samplechap2>
%\fi
%    \begin{macrocode}
\section{two}
more text in chapter two
%    \end{macrocode}

%\iffalse
%</samplechap2>
%\fi
%
% %%%%%%%%%%%%%%%%%%%%%%%%%%%%%%%%%%%%%%
% \paragraph{Part Include Files.}
%
% The include files are called |cdocspt3.tex| and |cdocspt4.tex|.
%
%\iffalse
%<*samplepart3|samplepart4>
%\fi

% Optional override for |\version| flag:
%    \begin{macrocode}
%%\providecommand{\version}{final}
%    \end{macrocode}

% Include the main document:
%    \begin{macrocode}
\input{childdoc.def}
\childdocby{cdocsamp}
%    \end{macrocode}

%\iffalse
%</samplepart3|samplepart4>
%\fi
%
%\iffalse
%<*samplepart3>
%\fi
% Some text for part 3:
%    \begin{macrocode}
some text in part three
%    \end{macrocode}

%\iffalse
%</samplepart3>
%\fi
% Some text for part 4:
%\iffalse
%<*samplepart4>
%\fi
%    \begin{macrocode}
more text in part four
%    \end{macrocode}

%\iffalse
%</samplepart4>
%\fi
%
% %%%%%%%%%%%%%%%%%%%%%%%%%%%%%%%%%%%%%%
% \paragraph{Forwarding for a Complete Draft.}
%
% The following forwarding file |cdocsdrf.tex|
% compiles the main document in draft mode:
%\iffalse
%<*sampledraft>
%\fi
%    \begin{macrocode}
\def\version{draft}
\input{childdoc.def}
\childdocforward{cdocsamp}
%    \end{macrocode}

%\iffalse
%</sampledraft>
%\fi
%
% %%%%%%%%%%%%%%%%%%%%%%%%%%%%%%%%%%%%%%
% \paragraph{Forwarding for Final Version of the Chapters.}
%
% The following forwarding files |cdocsfn1.tex| and |cdocsfn2.tex|
% (with identical content)
% compile the final versions of the child documents
% |cdocsch1.tex| and |cdocsch2.tex|, respectively:
%\iffalse
%<*samplefinal>
%\fi
%    \begin{macrocode}
\def\version{final}
\input{childdoc.def}
\childdocforwardprefix[cdocsamp]{cdocsfn}{cdocsch}
%    \end{macrocode}

%\iffalse
%</samplefinal>
%\fi
%
% %%%%%%%%%%%%%%%%%%%%%%%%%%%%%%%%%%%%%%
% \paragraph{Command Line Processing.}
%
% The following three command lines generate the output files
% |cdocscld|, |cdocscl1| and |cdocscl2|
% which should be identical to
% |cdocsdrf|, |cdocsch1| and |cdocsfn2|, respectively:
% \begin{center}
% \begin{tabular}{l}
% |latex -jobname cdocscld \|\\
% |  "\def\version{draft}\input{childdoc.def}\childdocforward{cdocsamp}"|\\
% |latex -jobname cdocscl1 \|\\
% |  "\input{childdoc.def}\childdocforward[cdocsamp]{cdocsch1}"|\\
% |latex -jobname cdocscl2 \|\\
% |  "\def\version{final}\input{childdoc.def}\childdocforward{cdocsch2}"|
% \end{tabular}
% \end{center}
% Note that the trailing backslash on each first line
% merely continues the input to the second line
% (for convenient cut ant paste).
% Furthermore, the command |latex| can be replaced by any
% of its alternative versions such as |pdflatex|.
%
% %%%%%%%%%%%%%%%%%%%%%%%%%%%%%%%%%%%%%%%%%%%%%%%%%%%%%%%%%%%%%%%%%%%%%%%%%%%%%%
% %%%%%%%%%%%%%%%%%%%%%%%%%%%%%%%%%%%%%%%%%%%%%%%%%%%%%%%%%%%%%%%%%%%%%%%%%%%%%%
% \section{Implementation}
%\iffalse
%<*package>
%\fi
%
% This section describes the definitions file |childdoc.def|.

% The definitions cannot be loaded using |\usepackage| or |\RequirePackage|
% which has a mechanism to prevent loading a style file more than once.
% When loading the definitions by means of |\input|
% multiple instances have to be prevented manually:
%\iffalse
%This code needs to be before the `\ProvidesFile' directive
%which is defined at the beginning of this file.
%Therefore it is also placed there and commented out here.
%</package>
%<*discard>
%\fi
%    \begin{macrocode}
\ifdefined\childdocmain\endinput\fi
%    \end{macrocode}
%\iffalse
%</discard>
%<*package>
%\fi
%
% \macro{\ifchilddoc}
% \macro{\ifchilddocmanual}
% The conditional |\ifchilddoc| tells whether a
% child (true) or main (false) document is being compiled.
% The conditional |\ifchilddocmanual| tells whether
% the |\includeonly| mechanism is used (false) or
% the selection of child files must be performed manually (true).
% The definitions initialise to false:
%    \begin{macrocode}
\newif\ifchilddoc
\newif\ifchilddocmanual
%    \end{macrocode}

% \macro{\childdocname}
% \macro{\childdocjob}
% The macro |\childdocname| stores the name of the main document
% to be compiled. The macro |\childdocjob| stores the name of
% the document on which the \LaTeX{} compiler was originally invoked.
% The content of |\jobname| cannot be compared
% to filenames specified in the source due to different catcodes.
% The following code rescans |\jobname|, stores the result
% in |\childdocname| and saves a copy in |\childdocjob|:
%    \begin{macrocode}
\edef\childdocname{\scantokens\expandafter{\jobname\noexpand}}
\let\childdocjob\childdocname
%    \end{macrocode}

% \macro{\childdocdisable}
% The macro |\childdocdisable| prevents the main file
% from being processed more than once.
% At this stage, the main document command |\childdocmain|
% is assumed to be called once again where it should do nothing.
% Any subsequent call to it should prevent
% a secondary processing of the main document
% It overwrites the forwarding commands
% |\childdocof| and |\childdocforward|
% with empty macros to prevent further inclusions of the main document:
%    \begin{macrocode}
\newcommand{\childdocdisable}
{
  \renewcommand{\childdocmain}[1]{\renewcommand{\childdocmain}[1]{\endinput}}
  \renewcommand{\childdocof}[1]{}
  \renewcommand{\childdocby}[2][]{}
  \renewcommand{\childdocforward}[2][]{}
  \renewcommand{\childdocdisable}{}
}
%    \end{macrocode}

% \macro{\childdocmain}
% The macro |\childdocmain| is to be called at the top of the main file
% with nothing or the main filename (without extension) as argument.
% First, it breaks loops.
% If the argument is not empty and does not match |\childdocname|
% (which is set by the first inclusion of |childdoc.def|),
% |\ifchilddoc| is set to true, |\includeonly| is applied to the child file
% and |\jobname| is set to the main file
% (for proper handling of |.aux| files):
%    \begin{macrocode}
\newcommand{\childdocmain}[1]
{
  \childdocdisable\childdocmain{}
  \if?#1?\else
    \begingroup
      \def\childdoctmp{#1}
      \ifx\childdoctmp\childdocname
        \def\childdoctmp{}
      \else
        \def\childdoctmp
        {
          \childdoctrue
          \includeonly{\childdocname}
          \def\childdocjob{#1}
          \def\jobname{#1}
        }
      \fi
      \expandafter
    \endgroup
    \childdoctmp
  \fi
}
%    \end{macrocode}

% \macro{\childdocof}
% The command |\childdocof| redirects
% compilation to the main file |#1|.
%    \begin{macrocode}
\newcommand{\childdocof}[1]
{
  \childdocdisable
  \childdoctrue
  \includeonly{\childdocname}
  \def\jobname{#1}
  \def\childdocjob{#1}
  \input{#1}
}
%    \end{macrocode}

% \macro{\childdocby}
% The command |\childdocby| ....
%    \begin{macrocode}
\newcommand{\childdocby}[2][]
{
  \childdocdisable
  \childdoctrue
  \childdocmanualtrue
  \if?#1?\else
    \def\jobname{#2}
  \fi
  \def\childdocjob{#2}
  \input{#2}
  \endinput
}
%    \end{macrocode}

% \macro{\childdocforward}
% The command |\childdocforward| redirects
% compilation to the main file or
% (if the optional argument is given) a child file.
% Parameters are set as if the main file
% or a child file starting with |\childdocof| was compiled.
% Then compilation is handed over to the main file:
%    \begin{macrocode}
\newcommand{\childdocforward}[2][]
{
  \begingroup
    \if?#1?
      \def\childdoctmp
      {
        \def\childdocname{#2}
        \def\childdocjob{#2}
        \def\jobname{#2}
        \input{#2}
        \endinput
      }
    \else
      \def\childdoctmp
      {
        \childdocdisable
        \def\childdocname{#2}
        \childdoctrue
        \includeonly{#2}
        \def\childdocjob{#1}
        \def\jobname{#1}
        \input{#1}
        \endinput
      }
    \fi
    \expandafter
  \endgroup
  \childdoctmp
}
%    \end{macrocode}

% \macro{\childdocforwardprefix}
% The command |\childdocforwardprefix| redirects
% compilation to the main or a child file by means of a pattern.
% The prefix |#1| in the current filename is replaced by |#2|
% and the suffix of the current filename is kept
% (it is assumed that the filename does not contain the substring `|~~~|'
% which is used as a delimiter).
% Compilation is handed over to the new file by |\childdocforward|:
%    \begin{macrocode}
\newcommand{\childdocforwardprefix}[3][]
{
  \begingroup
    \def\childdocextract #2##1~~~{\def\childdoctmp{\childdocforward[#1]{#3##1}}}
    \expandafter\childdocextract\childdocname~~~
    \expandafter
  \endgroup
  \childdoctmp
}
%    \end{macrocode}

% \macro{\childdoc}
% The deprecated macro |\childdoc| is a legacy version of |\childdocmain|:
%    \begin{macrocode}
\newcommand{\childdoc}{\childdocmain}
%    \end{macrocode}

% \macro{\childdocredirect}
% The deprecated macro |\childdocredirect| is a legacy version
% of |\childdocforward| and |\childdocforwardprefix|:
%    \begin{macrocode}
\newcommand{\childdocredirect}[2][]
{
  \begingroup
    \if?#1?
      \def\childdoctmp{\childdocforward{#2}}
    \else
      \def\childdoctmp{\childdocforwardprefix{#1}{#2}}
    \fi
    \expandafter
  \endgroup
  \childdoctmp
}
%    \end{macrocode}

%\iffalse
%</package>
%\fi
%
\endinput
|\\
|\childdocmain{|\textit{main}|}|\\
\end{tabular}
\end{center}
%
If |\jobname| does not match the argument \textit{main} of |\childdocmain|,
it is assumed that |\jobname| points to the child file to be compiled.
When using |\childdocmain| with the main file specified as argument,
it suffices to start a child file
with just |\input{|\textit{main}|}|
without loading of the package and using |\childdocof|.
If instead all processing is done
with the appropriate \textsf{childdoc} directives,
the argument of \textit{main} of |\childdocmain| can be empty.

An alternative version of the command line processing described
in \secref{sec:commandline} using the detection mechanism reads:
%
\begin{center}
|... -jobname "|\textit{target}|" "|[\textit{flags}]%
[|\def\jobname{|\textit{dest}|}|]|\input{|\textit{main}|}"|
\end{center}

%%%%%%%%%%%%%%%%%%%%%%%%%%%%%%%%%%%%%%%%%%%%%%%%%%%%%%%%%%%%%%%%%%%%%%%%%%%%%%%%
\subsection{Manual Code}
\label{sec:manual}

In case one cannot be certain whether the definitions file |childdoc.def|
is installed on the target \TeX{} distribution
and one prefers not to ship it,
it is conceivable to paste a few relevant commands into the sources.

To that end, drop all statements |% \iffalse
%
% childdoc.dtx Copyright (C) 2017-2018 Niklas Beisert
%
% This work may be distributed and/or modified under the
% conditions of the LaTeX Project Public License, either version 1.3
% of this license or (at your option) any later version.
% The latest version of this license is in
%   http://www.latex-project.org/lppl.txt
% and version 1.3 or later is part of all distributions of LaTeX
% version 2005/12/01 or later.
%
% This work has the LPPL maintenance status `maintained'.
%
% The Current Maintainer of this work is Niklas Beisert.
%
% This work consists of the files childdoc.dtx and childdoc.ins
% and the derived files childdoc.def and cdocsamp.tex with
% cdocsch1.tex, cdocsch2.tex, cdocsdrf.tex, cdocsfn1.tex, cdocsfn2.tex.
%
%<package>\ifdefined\childdocmain\endinput\fi
%<package>\ProvidesFile{childdoc.def}[2018/12/30 v2.0 child document driver]
%<samplemain>\ProvidesFile{cdocsamp.tex}[2018/12/30 v2.0 sample for childdoc]
%<*driver>
%\ProvidesFile{childdoc.drv}[2018/12/30 v2.0 childdoc reference manual file]
\PassOptionsToClass{10pt,a4paper}{article}
\documentclass{ltxdoc}

\usepackage[margin=35mm]{geometry}
\usepackage{hyperref}
\usepackage{hyperxmp}
\usepackage[usenames]{color}

\hypersetup{colorlinks=true}
\hypersetup{pdfstartview=FitH}
\hypersetup{pdfpagemode=UseNone}
\hypersetup{pdfsource={}}
\hypersetup{pdflang={en-UK}}
\hypersetup{pdfcopyright={Copyright 2017-2018 Niklas Beisert.
  This work may be distributed and/or modified under the
  conditions of the LaTeX Project Public License, either version 1.3
  of this license or (at your option) any later version.}}
\hypersetup{pdflicenseurl={http://www.latex-project.org/lppl.txt}}
\hypersetup{pdfcontactaddress={ETH Zurich, ITP, HIT K,
  Wolfgang-Pauli-Strasse 27}}
\hypersetup{pdfcontactpostcode={8093}}
\hypersetup{pdfcontactcity={Zurich}}
\hypersetup{pdfcontactcountry={Switzerland}}
\hypersetup{pdfcontactemail={nbeisert@itp.phys.ethz.ch}}
\hypersetup{pdfcontacturl={http://people.phys.ethz.ch/\xmptilde nbeisert/}}

\newcommand{\secref}[1]{\hyperref[#1]{section \ref*{#1}}}

\parskip1ex
\parindent0pt
\let\olditemize\itemize
\def\itemize{\olditemize\parskip0pt}

\begin{document}

\title{The \textsf{childdoc} Package}
\hypersetup{pdftitle={The childdoc Package}}
\author{Niklas Beisert\\[2ex]
  Institut f\"ur Theoretische Physik\\
  Eidgen\"ossische Technische Hochschule Z\"urich\\
  Wolfgang-Pauli-Strasse 27, 8093 Z\"urich, Switzerland\\[1ex]
  \href{mailto:nbeisert@itp.phys.ethz.ch}
  {\texttt{nbeisert@itp.phys.ethz.ch}}}
\hypersetup{pdfauthor={Niklas Beisert}}
\hypersetup{pdfsubject={Manual for the LaTeX2e Package childdoc}}
\date{30 December 2018, \textsf{v2.0}}
\maketitle

\begin{abstract}\noindent
\textsf{childdoc} is a \LaTeXe{} package
that enables the direct compilation
of document sections included by |\include|
to individual files.
\end{abstract}

\begingroup
\parskip0ex
\tableofcontents
\endgroup

%%%%%%%%%%%%%%%%%%%%%%%%%%%%%%%%%%%%%%%%%%%%%%%%%%%%%%%%%%%%%%%%%%%%%%%%%%%%%%%%
%%%%%%%%%%%%%%%%%%%%%%%%%%%%%%%%%%%%%%%%%%%%%%%%%%%%%%%%%%%%%%%%%%%%%%%%%%%%%%%%
\section{Introduction}

\LaTeX{} provides a mechanism to structure a large document (such as a book)
into a main file and several child files (containing the chapters)
using the |\include| command.
This mechanism is beneficial for documents
which span hundreds of pages in order to
make the source file(s) more manageable.
Moreover, compilation can be restricted to
selected child files by means of the |\includeonly| command.
The latter feature can be used to reduce the compilation time while editing
(this was significantly more useful in the earlier days of \LaTeX{})
or to generate a smaller document which is easier to navigate.
Another application of |\includeonly| is to generate
documents consisting of selected parts of the complete document.

However, there are a few drawbacks of the plain |\include| mechanism:
\begin{itemize}
\item
The child files cannot be compiled on their own,
they can only be compiled via the main file.
A naive editing environment
(such as a text editor with an option
to have the current file processed by \LaTeX)
may require one to switch to the main file before compiling;
attempting to compile the child file produces errors.
\item
The main file must be modified (each time)
to adjust the |\includeonly| command
to the present needs. This easily leaves the main file in a messy state.
\item
The generated document will always carry the filename
of the main document. This is inconvenient if
several child files are to be compiled and
to be kept for distribution.
\end{itemize}

The present package provides a simple interface
to make child files individually compilable by \LaTeX{}.
Compiling a child file then has the same effect as compiling
the main file with an |\includeonly| command
to select the appropriate child.
Moreover the generated document will carry the name of the child
rather than the main file.
This resolves all three above issues.

This feature is meant to make the editing of books,
thesis documents and lecture notes somewhat more convenient.
However, the package can also be used efficiently for
composing a series of documents (such as exercise sheets)
which are typically distributed individually.
It then assists the author in generating the individual documents
(potentially in different versions)
as well as a document containing the collected series.
Another application is in developing style files
or other kinds of included material
where compilation of the style file could redirect
to a sample or test file.

%%%%%%%%%%%%%%%%%%%%%%%%%%%%%%%%%%%%%%%%%%%%%%%%%%%%%%%%%%%%%%%%%%%%%%%%%%%%%%%%
%%%%%%%%%%%%%%%%%%%%%%%%%%%%%%%%%%%%%%%%%%%%%%%%%%%%%%%%%%%%%%%%%%%%%%%%%%%%%%%%
\section{Usage}

First of all, the package \textsf{childdoc} is \emph{not} a standard
\LaTeXe{} |.sty| style file! Therefore it needs to be invoked in
a non-standard way.

%%%%%%%%%%%%%%%%%%%%%%%%%%%%%%%%%%%%%%%%%%%%%%%%%%%%%%%%%%%%%%%%%%%%%%%%%%%%%%%%
\subsection{Included Files}
\label{sec:include}

%%%%%%%%%%%%%%%%%%%%%%%%%%%%%%%%%%%%%%%%
\DescribeMacro{\childdocmain}
To use the package, add the commands
\begin{center}
\begin{tabular}{l}
|\input{childdoc.def}|\\
|\childdocmain{}|\\
\end{tabular}
\end{center}
at the very top of the main \LaTeX{} file,
in particular \emph{before} the |\documentclass| statement!
The argument of |\childdocmain| should be left empty
(but it must be present).

%%%%%%%%%%%%%%%%%%%%%%%%%%%%%%%%%%%%%%%%
\DescribeMacro{\childdocof}
Furthermore, add the commands
\begin{center}
\begin{tabular}{l}
|\input{childdoc.def}|\\
|\childdocof{|\textit{main}|}|\\
\end{tabular}
\end{center}
at the top of every child file \textit{child}
which is included by |\include{|\textit{child}|}|
from within the main file
(or at least for those files to be compiled individually).
The argument \textit{main} must be the filename of the main file.

There are a couple of
considerations in setting up the main and child documents:

%%%%%%%%%%%%%%%%%%%%%%%%%%%%%%%%%%%%%%%%
\paragraph{Restrictions.}

Please note the following restrictions:
\begin{itemize}
\item
|\childdocmain| must be called with one argument \textit{main}
to ensure compatibility with earlier version of the package.
It must either be empty (|\childdocmain{}|)
or precisely match the filename of the main file in which it is specified.
See \secref{sec:detection} for further information.
\item
The filename \textit{main} must be specified without the |.tex| extension.
\item
The filename \textit{main} is case sensitive
(even in case-insensitive file systems)
due to internal string comparison.
\item
The argument \textit{main} should be fully expanded, it cannot be a macro.
\item
Subdirectories and special characters should be avoided in filenames.
\item
The command |\childdocmain{|\textit{main}|}| must be followed by a whitespace.
It should not be followed immediately by another command
or by a comment mark `|%|'.
This is because the \TeX{} parser reads the token immediately following
the argument of |\childdocmain| and puts it
at the beginning of every child section;
however, a white\-space is ignored.
\end{itemize}

%%%%%%%%%%%%%%%%%%%%%%%%%%%%%%%%%%%%%%%%
\paragraph{Content of Main File.}

It is advisable to place all content in the child files included by |\include|.
Any output contained in the main file will appear in all child documents
unless suppressed manually;
it cannot be suppressed automatically by the |\includeonly| directive
and thus should normally be avoided.
A method to include some content in the main file
by means of conditional processing is described in \secref{sec:conditional}.

%%%%%%%%%%%%%%%%%%%%%%%%%%%%%%%%%%%%%%%%
\paragraph{Page Numbering.}

When only a part of the document is compiled,
the appropriate numbering of pages
(as well as other status parameters)
is determined from the |.aux| files.
The latter contain information from previous passes.
However this information needs to propagate through
all intermediate child documents.
Therefore the page numbering in child documents may well
be inconsistent until the complete document is compiled at least once.

A useful (if unconventional) way to always ensure a consistent
page numbering is to restart the numbering in each child document
and denote the pages by `\textit{child}|.|\textit{page}'
where \textit{child} represents the chapter/section number of the child file.
This can be achieved by the command
|\numberwithin{page}{|\textit{child}|}|
of the \textsf{amsmath} package
where \textit{child} can be |chapter| or |section|
depending on the chosen structuring.
Alternatively, one can modify the macro |\thepage| appropriately
and reset the counter |page| at the start of each child file.

%%%%%%%%%%%%%%%%%%%%%%%%%%%%%%%%%%%%%%%%%%%%%%%%%%%%%%%%%%%%%%%%%%%%%%%%%%%%%%%%
\subsection{Conditional Processing}
\label{sec:conditional}

The package provides a mechanism to compile different versions
of a document. To customise the versions further some conditional processing
can come in handy to distinguish which version is being compiled.
The package provides two macros to describe the compilation context:

%%%%%%%%%%%%%%%%%%%%%%%%%%%%%%%%%%%%%%%%
\DescribeMacro{\ifchilddoc}
The conditional |\ifchilddoc| distinguishes between the compilation of
child documents and the main document:
%
\begin{center}
|\ifchilddoc |\textit{child-code}| |[|\||else |\textit{main-code}]| \||fi|
\end{center}

%%%%%%%%%%%%%%%%%%%%%%%%%%%%%%%%%%%%%%%%
\DescribeMacro{\childdocname}
\DescribeMacro{\childdocjob}
The macro |\childdocname| contains the filename (without extension)
of the main or child file being processed.
Note that |\childdocjob| will always contain the name of the main file.

%%%%%%%%%%%%%%%%%%%%%%%%%%%%%%%%%%%%%%%%
\paragraph{Title Page.}

Conditional processing can be used to include a title or banner page
in the main document when proper precautions are taken.
Importantly, the code in the main file should ensure that the page counter
(as well as other status parameters which are stored in the |.aux| files)
takes the same value after the conditional processing.
Otherwise the page numbers may take divergent values
depending on which part is compiled.

For example, a title page could be declared by:
%
\begin{center}
\begin{tabular}{l}
|\ifchilddoc\||else|\\
|\addtocounter{page}{-1}|\\
\textit{code for title page}\\
|\newpage|\\
|\||fi|
\end{tabular}
\end{center}
%
A banner page for the child documents can be generated by:
%
\begin{center}
\begin{tabular}{l}
|\ifchilddoc|\\
|\addtocounter{page}{-1}|\\
\textit{code for banner page}\\
|\newpage|\\
|\||fi|
\end{tabular}
\end{center}
%
Here one could write a message such as:
\begin{center}
|This is the part \childdocname{} of \childdocjob{}.|
\end{center}

%%%%%%%%%%%%%%%%%%%%%%%%%%%%%%%%%%%%%%%%%%%%%%%%%%%%%%%%%%%%%%%%%%%%%%%%%%%%%%%%
\subsection{Flags}
\label{sec:flags}

The package makes it easy to generate different versions
of the main or child documents.
To this end compilation flags can be defined
and assigned different default values.
They will be particularly useful in conjunction
with the forwarding mechanism described in \secref{sec:forward}.

For example, it may be useful to have a flag |\version|
which can be set to |draft| or |final|.
The document source will contain some conditional code
depending on the value of |\version|.
Suppose further, the flag should default to |final| for the main file
and to |draft| for child files
which is a natural assignment for editing the document.
This is achieved by placing the following code
in the preamble of the main document
(below the |\childdocmain| directive):
%
\begin{center}
\begin{tabular}{l}
|\ifchilddoc|\\
|\providecommand{\version}{draft}|\\
|\||else|\\
|\providecommand{\version}{final}|\\
|\||fi|
\end{tabular}
\end{center}
%
The definition by |\providecommand| makes sure
that previous definitions are not overwritten.
Further statements |\providecommand{\version}{...}|
can thus be added before the above code to override it.

For the main file, one might add a line
(between |\childdocmain| and the above block)
%
\begin{center}
|%\ifchilddoc\||else\providecommand{\version}{draft}\||fi|
\end{center}
%
which can be uncommented to produce a draft version.
Likewise one can add a line to the very top of a child file
(above the |\childdocof{|\textit{main}|}| directive)
%
\begin{center}
|%\providecommand{\version}{final}|
\end{center}
%
which can be uncommented to produce the final version of this child document.

%%%%%%%%%%%%%%%%%%%%%%%%%%%%%%%%%%%%%%%%%%%%%%%%%%%%%%%%%%%%%%%%%%%%%%%%%%%%%%%%
\subsection{Forwarding}
\label{sec:forward}

Different versions of the main or child documents
using compilation flags as described in \secref{sec:flags}
can be (permanently) stored in different files
for convenient compilation, viewing and distribution.
To this end, the package defines a command
to pass on compilation to a different file:

%%%%%%%%%%%%%%%%%%%%%%%%%%%%%%%%%%%%%%%%
\DescribeMacro{\childdocforward}
The command |\childdocforward| redirects processing to
another source file:
%
\begin{center}
\begin{tabular}{l}
|\input{childdoc.def}|\\
|\childdocforward[|\textit{main}|]{|\textit{dest}|}|\\
\end{tabular}
\end{center}
%
The argument \textit{dest} is the destination file
(without extension).
It should be the main file or one of the child files.
Note that further \textsf{childdoc} directives
such as |\childdocof| and |\childdocforward|
in the indicated file will be processed in this form.
The optional argument \textit{main}
passes on directly to the main file \textit{main}
while pretending to compile the child \textit{dest}.
This form behaves as if \textit{dest}
issues |\childdocof{|\textit{main}|}| right away,
and no further \textsf{childdoc} directives will be processed.

%%%%%%%%%%%%%%%%%%%%%%%%%%%%%%%%%%%%%%%%
\DescribeMacro{\...prefix}
In the alternative form |\childdocforwardprefix|,
%
\begin{center}
\begin{tabular}{l}
|\input{childdoc.def}|\\
|\childdocforwardprefix[|\textit{main}|]{|\textit{prefix}|}{|\textit{dest}|}|
\end{tabular}
\end{center}
%
the destination file is determined by a pattern
depending on the current file:
To make this work, the current file must be called
`{\textit{prefix}\hspace{0.2em}\textit{suffix}}'
with \textit{prefix} matching precisely the argument.
Processing is then passed on to the file
`{\textit{dest}\hspace{0.2em}\textit{suffix}}'.
Surely, the same effect is achieved by
directly specifying the
argument `{\textit{dest}\hspace{0.2em}\textit{suffix}}'
in the first form.
However, that requires to set up a different file
for each child. With the alternative form of the command
all these files can have exactly the same content
which simplifies setting them up and maintaining them.

For example, the following file |draft.tex|
with a compilation flag |\version| as described in \secref{sec:flags}
compiles the main document as a draft:
%
\begin{center}
\begin{tabular}{l}
|\def\version{draft}|\\
|\input{childdoc.def}|\\
|\childdocforward{|\textit{main}|}|
\end{tabular}
\end{center}
%
Likewise, the following files |final|\textit{nn}|.tex|
compile the final version of the child document
|child|\textit{nn}|.tex|:
%
\begin{center}
\begin{tabular}{l}
|\def\version{final}|\\
|\input{childdoc.def}|\\
|\childdocforwardprefix{final}{child}|
\end{tabular}
\end{center}
%

Note that when several versions of a main file and/or of each child file
are to be generated, it may be convenient to set up a |Makefile| or
shell script to automatise the process.

%%%%%%%%%%%%%%%%%%%%%%%%%%%%%%%%%%%%%%%%%%%%%%%%%%%%%%%%%%%%%%%%%%%%%%%%%%%%%%%%
\subsection{Command Line Processing}
\label{sec:commandline}

The effect of redirection files can also be achieved by invoking
the \LaTeX{} compiler with a more elaborate command line.
Most conveniently this should be done as part
of a shell script or a |Makefile|.

When using \textsf{childdoc} in the main file, the following
command lines effectively perform a redirection
(note that depending on the shell being used,
backslashes may have to be doubled: `|\|' $\to$ `|\\|'):
%
\begin{center}
|... -jobname "|\textit{target}|" |\\|"|[\textit{flags}]%
|\input{childdoc.def}\childdocforward[|\textit{main}|]{|\textit{dest}|}"|
\end{center}
%
Here \textit{target} is the name of the output file,
\textit{main} is the name of the main file
and \textit{dest} is the name of the main or child file to be processed
(all filenames without extensions).
The optional argument \textit{main} can be omitted
if \textit{main} matches \textit{dest}.
Optionally, compilation \textit{flags} can be defined via |\def| commands.
This command line makes the \TeX{} engine believe
it is compiling the file \textit{target}
whose content is specified as the latter parameter.
The provided code then forwards the processing to
\textit{main} or \textit{dest} as described in \secref{sec:forward}.

%%%%%%%%%%%%%%%%%%%%%%%%%%%%%%%%%%%%%%%%%%%%%%%%%%%%%%%%%%%%%%%%%%%%%%%%%%%%%%%%
\subsection{Include by Input}
\label{sec:input}

Including child documents by |\include| has some restrictions by design.
Most notably, the content of a child document always occupies
its own set of pages; pages cannot be shared between child documents.
Usually, this behaviour makes perfect sense
because each child document contain an essential part of the document.
However, in some situations it may be desirable to compose
a document from a collection of parts
without having mandatory page breaks between then.
For this case, the package
provides a mechanism to include parts
by |\input| which can also be processed individually.
However, by construction this mechanism
requires manual handling of the content to be output.

%%%%%%%%%%%%%%%%%%%%%%%%%%%%%%%%%%%%%%%%
\DescribeMacro{\ifchilddocmanual}
The main file should be prepared as usual, see \secref{sec:include}.
However, the document body must make a distinction
between processing of an individual part and of the main document, e.g.:
%
\begin{center}
\begin{tabular}{l}
|\ifchilddocmanual|\\
|\input{\childdocname}|\\
|\||else|\\
\textit{document body with }|\input{|\textit{part}|}|\\
|\||fi|
\end{tabular}
\end{center}
%
The conditional |\ifchilddocmanual| is true whenever
a part to be included by |\input| is being compiled,
and the name of the part is stored in |\childdocname|.

%%%%%%%%%%%%%%%%%%%%%%%%%%%%%%%%%%%%%%%%
\DescribeMacro{\childdocby}
Each part to be included by |\input| should start with:
%
\begin{center}
\begin{tabular}{l}
|\input{childdoc.def}|\\
|\childdocby{|\textit{main}|}|\\
\end{tabular}
\end{center}
%
The directive |\childdocby| is similar to |\childdocof|
described in \secref{sec:include},
but the subsequent selection of content must be done manually.
To that end, both |\ifchilddoc| and |\ifchilddocmanual|
will be true upon processing of a part,
and the name of the part is stored in |\childdocname|.
Note that |\jobname| will be set to the filename of the current part
so that each part receives an individual |.aux| file
that does not interfere with the |.aux| file(s) of the main document.
This behaviour can be altered by the alternative form
|\childdocby[*]{|\textit{main}|}| (with a non-empty optional argument)
which uses the |.aux| file of the main document
by setting |\jobname| to \textit{main}.

%%%%%%%%%%%%%%%%%%%%%%%%%%%%%%%%%%%%%%%%%%%%%%%%%%%%%%%%%%%%%%%%%%%%%%%%%%%%%%%%
\subsection{Driver Development}
\label{sec:driver}

The \textsf{childdoc} mechanism can also be use for the development
of definition files such as \LaTeX{} styles or classes.
This case differs from the above setup with multiple parts
included by |\include| in that no |\includeonly| should be invoked.
This can be achieved by starting the include file
(before |\ProvidesPackage|) with:
%
\begin{center}
\begin{tabular}{l}
|\input{childdoc.def}|\\
|\childdocforward{|\textit{main}|}|\\
\end{tabular}
\end{center}
%
or alternatively with:
%
\begin{center}
\begin{tabular}{l}
|\input{childdoc.def}|\\
|\childdocby{|\textit{main}|}|\\
\end{tabular}
\end{center}
%
Both forms have slightly different effects as described above.
The main file is prepared as usual, see \secref{sec:include}.

%%%%%%%%%%%%%%%%%%%%%%%%%%%%%%%%%%%%%%%%%%%%%%%%%%%%%%%%%%%%%%%%%%%%%%%%%%%%%%%%
\subsection{Legacy Detection}
\label{sec:detection}

The directive |\childdocmain| in the main file can detect
whether the complete document or merely a child is to be compiled
even without using the directive |\childdocof|.
This method is deprecated because it is less robust
and there is no compelling reason to use it;
it is merely provided for backward compatibility
and it may be removed in future versions.

If the detection mechanism is to be used,
it is mandatory to correctly specify
the filename of the main file as the argument of |\childdocmain|:
%
\begin{center}
\begin{tabular}{l}
|\input{childdoc.def}|\\
|\childdocmain{|\textit{main}|}|\\
\end{tabular}
\end{center}
%
If |\jobname| does not match the argument \textit{main} of |\childdocmain|,
it is assumed that |\jobname| points to the child file to be compiled.
When using |\childdocmain| with the main file specified as argument,
it suffices to start a child file
with just |\input{|\textit{main}|}|
without loading of the package and using |\childdocof|.
If instead all processing is done
with the appropriate \textsf{childdoc} directives,
the argument of \textit{main} of |\childdocmain| can be empty.

An alternative version of the command line processing described
in \secref{sec:commandline} using the detection mechanism reads:
%
\begin{center}
|... -jobname "|\textit{target}|" "|[\textit{flags}]%
[|\def\jobname{|\textit{dest}|}|]|\input{|\textit{main}|}"|
\end{center}

%%%%%%%%%%%%%%%%%%%%%%%%%%%%%%%%%%%%%%%%%%%%%%%%%%%%%%%%%%%%%%%%%%%%%%%%%%%%%%%%
\subsection{Manual Code}
\label{sec:manual}

In case one cannot be certain whether the definitions file |childdoc.def|
is installed on the target \TeX{} distribution
and one prefers not to ship it,
it is conceivable to paste a few relevant commands into the sources.

To that end, drop all statements |\input{childdoc.def}|
and perform the replacements as outlined below.
Instead of |\childdocmain{|\textit{main}|}| add the following code
to the top of the main file:
%
\begin{center}
\begin{tabular}{l}
|\||ifdefined\childdocname\endinput\||fi\newif\ifchilddoc|\\
|\edef\childdocname{\scantokens\expandafter{\jobname\noexpand}}|\\
|\def\childdocmain{|\textit{main}|}\||ifx\childdocmain\childdocname\||else|\\
|\childdoctrue\includeonly{\childdocname}\let\jobname\childdocmain\||fi|\\
\end{tabular}
\end{center}
%
Instead of |\childdocof{|\textit{main}|}| just include the main file
at the top of each child file:
%
\begin{center}
|\input{|\textit{main}|}|
\end{center}
%
A simple redirection |\childdocforward{|\textit{dest}|}| is achieved by:
%
\begin{center}
|\def\jobname{|\textit{dest}|}\input{\jobname}|
\end{center}
%
The redirection with prefix
|\childdocforwardprefix[|\textit{prefix}|]{|\textit{dest}|}|
is accomplished by:
%
\begin{center}
\begin{tabular}{l}
|{\edef\jobname{\scantokens\expandafter{\jobname\noexpand}}|\\
|\def\redirectjob |\textit{prefix}|#1~~~{\gdef\jobname{|\textit{dest}|#1}}|\\
|\expandafter\redirectjob\jobname~~~}\input{\jobname}|
\end{tabular}
\end{center}

In an alternative approach,
child documents can be compiled by a specific command line
without additional code or specific definitions:
%
\begin{center}
|... -jobname "|\textit{target}|" "|[\textit{flags}]%
|\includeonly{|\textit{dest}|}\input{|\textit{main}|}"|
\end{center}
%

%%%%%%%%%%%%%%%%%%%%%%%%%%%%%%%%%%%%%%%%%%%%%%%%%%%%%%%%%%%%%%%%%%%%%%%%%%%%%%%%
%%%%%%%%%%%%%%%%%%%%%%%%%%%%%%%%%%%%%%%%%%%%%%%%%%%%%%%%%%%%%%%%%%%%%%%%%%%%%%%%
\section{Information}

%%%%%%%%%%%%%%%%%%%%%%%%%%%%%%%%%%%%%%%%%%%%%%%%%%%%%%%%%%%%%%%%%%%%%%%%%%%%%%%%
\subsection{Copyright}

Copyright \copyright{} 2017--2018 Niklas Beisert

This work may be distributed and/or modified under the
conditions of the \LaTeX{} Project Public License, either version 1.3
of this license or (at your option) any later version.
The latest version of this license is in
  \url{http://www.latex-project.org/lppl.txt}
and version 1.3 or later is part of all distributions of \LaTeX{}
version 2005/12/01 or later.

This work has the LPPL maintenance status `maintained'.

The Current Maintainer of this work is Niklas Beisert.

This work consists of the files |README.txt|, |childdoc.ins| and |childdoc.dtx|
as well as the derived files |childdoc.def|, |cdocsamp.tex|
with |cdocsch1.tex|, |cdocsch2.tex|, |cdocspt3.tex|, |cdocspt4.tex|,
|cdocsdrf.tex|, |cdocsfn1.tex|, |cdocsfn2.tex|
as well as |childdoc.pdf|.

%%%%%%%%%%%%%%%%%%%%%%%%%%%%%%%%%%%%%%%%%%%%%%%%%%%%%%%%%%%%%%%%%%%%%%%%%%%%%%%%
\subsection{Files and Installation}

The package consists of the files:
%
\begin{center}
\begin{tabular}{ll}
    |README.txt|   & readme file \\
    |childdoc.ins| & installation file \\
    |childdoc.dtx| & source file \\
    |childdoc.def| & definition file \\
    |cdocsamp.tex| & sample main file \\
    |cdocsch1.tex| & sample include file \\
    |cdocsch2.tex| & sample include file \\
    |cdocspt3.tex| & sample part file \\
    |cdocspt4.tex| & sample part file \\
    |cdocsdrf.tex| & sample redirection file \\
    |cdocsfn1.tex| & sample redirection file \\
    |cdocsfn2.tex| & sample redirection file \\
    |childdoc.pdf| & manual
\end{tabular}
\end{center}
%
The distribution consists of the files
|README.txt|, |childdoc.ins| and |childdoc.dtx|.
%
\begin{itemize}
\item
Run (pdf)\LaTeX{} on |childdoc.dtx|
to compile the manual |childdoc.pdf| (this file).
\item
Run \LaTeX{} on |childdoc.ins| to create the definitions file |childdoc.def|
and the sample |cdocsamp.tex| with include files
|cdocsch1.tex|, |cdocsch2.tex|, |cdocspt3.tex|, |cdocspt4.tex|,
|cdocsdrf.tex|, |cdocsfn1.tex|, |cdocsfn2.tex|.
Then copy the file |childdoc.def| to an appropriate directory of your \LaTeX{}
distribution, e.g.\ \textit{texmf-root}|/tex/latex/childdoc|.
\end{itemize}

%%%%%%%%%%%%%%%%%%%%%%%%%%%%%%%%%%%%%%%%%%%%%%%%%%%%%%%%%%%%%%%%%%%%%%%%%%%%%%%%
\subsection{Related CTAN Packages}

There are several other packages which offer a similar functionality:
%
\begin{itemize}
\item
The packages
\href{http://ctan.org/pkg/docmute}{\textsf{docmute}},
\href{http://ctan.org/pkg/includex}{\textsf{includex}} and
\href{http://ctan.org/pkg/standalone}{\textsf{standalone}}
provide commands to include only the document body of
a child file thus allowing both files to be compiled individually.
\item
The packages \href{http://ctan.org/pkg/subdocs}{\textsf{subdocs}}
and \href{http://ctan.org/pkg/subfiles}{\textsf{subfiles}}
provide structures in which the main and child documents can be
encapsulated and allowing them to be compiled individually.
The inclusion mechanism is different from the conventional |\include|.
\item
The package \href{http://ctan.org/pkg/combine}{\textsf{combine}}
is an elaborate solution to combine several documents into one.
\end{itemize}
%
See also the CTAN topic \href{http://ctan.org/topic/subdocs}{\textsf{subdocs}}
for further related packages.
The present package differs from the above solutions in that
a document structure constructed with the conventional |\include| mechanism
just needs two extra commands at the top of every file
such that all constituent files can be compiled individually.

%%%%%%%%%%%%%%%%%%%%%%%%%%%%%%%%%%%%%%%%%%%%%%%%%%%%%%%%%%%%%%%%%%%%%%%%%%%%%%%%
%\subsection{Feature Suggestions}
%
%The following is a list of features which may be useful for future
%versions of this package:
%%
%\begin{itemize}
%\item
%\ldots
%\end{itemize}

%%%%%%%%%%%%%%%%%%%%%%%%%%%%%%%%%%%%%%%%%%%%%%%%%%%%%%%%%%%%%%%%%%%%%%%%%%%%%%%%
\subsection{Revision History}

%%%%%%%%%%%%%%%%%%%%%%%%%%%%%%%%%%%%%%%%
\paragraph{v2.0:} 2018/12/30

\begin{itemize}
\item
immediate forward processing
\item
added |\childdocby| mechanism
\item
manual restructured
\end{itemize}

%%%%%%%%%%%%%%%%%%%%%%%%%%%%%%%%%%%%%%%%
\paragraph{v1.6:} 2018/01/17

\begin{itemize}
\item
application for development of include files
\item
corrections to manual
\end{itemize}

%%%%%%%%%%%%%%%%%%%%%%%%%%%%%%%%%%%%%%%%
\paragraph{v1.5:} 2017/05/21

\begin{itemize}
\item
more complete structuring introduced
\item
|\childdocof| introduced
\item
|\childdoc| renamed to |\childdocmain|
\item
|\childredirect| renamed to |\childdocforward| and |\childdocforwardprefix|
and functionality expanded
\end{itemize}

%%%%%%%%%%%%%%%%%%%%%%%%%%%%%%%%%%%%%%%%
\paragraph{v1.0:} 2017/04/27

\begin{itemize}
\item
manual and install package
\item
first version published on CTAN
\end{itemize}

%%%%%%%%%%%%%%%%%%%%%%%%%%%%%%%%%%%%%%%%
\paragraph{v0.6:} 2017/04/26

\begin{itemize}
\item
redirection mechanism added
\end{itemize}

%%%%%%%%%%%%%%%%%%%%%%%%%%%%%%%%%%%%%%%%
\paragraph{v0.5:} 2017/04/26

\begin{itemize}
\item
functionality in definition file
\end{itemize}


%%%%%%%%%%%%%%%%%%%%%%%%%%%%%%%%%%%%%%%%%%%%%%%%%%%%%%%%%%%%%%%%%%%%%%%%%%%%%%%%
%%%%%%%%%%%%%%%%%%%%%%%%%%%%%%%%%%%%%%%%%%%%%%%%%%%%%%%%%%%%%%%%%%%%%%%%%%%%%%%%
%%%%%%%%%%%%%%%%%%%%%%%%%%%%%%%%%%%%%%%%%%%%%%%%%%%%%%%%%%%%%%%%%%%%%%%%%%%%%%%%
\appendix

\settowidth\MacroIndent{\rmfamily\scriptsize 000\ }

 \DocInput{childdoc.dtx}

\end{document}
%</driver>
% \fi
%
% %%%%%%%%%%%%%%%%%%%%%%%%%%%%%%%%%%%%%%%%%%%%%%%%%%%%%%%%%%%%%%%%%%%%%%%%%%%%%%
% %%%%%%%%%%%%%%%%%%%%%%%%%%%%%%%%%%%%%%%%%%%%%%%%%%%%%%%%%%%%%%%%%%%%%%%%%%%%%%
% \section{Sample}
%\iffalse
%<*samplemain>
%\fi
%
% The following presents a sample document
% with two chapters, two parts, a title page,
% a compile flag as well as three forwarding files to set the flag.
% It consists of eight |.tex| files:
% \begin{center}
% \begin{tabular}{ll}
% |cdocsamp.tex|&main file\\
% |cdocsch1.tex|&include file for chapter 1\\
% |cdocsch2.tex|&include file for chapter 2\\
% |cdocspt3.tex|&include file for part 3\\
% |cdocspt4.tex|&include file for part 4\\
% |cdocsdrf.tex|&forwarding file for main file in draft mode\\
% |cdocsfi1.tex|&forwarding file for final version of chapter 1\\
% |cdocsfi2.tex|&forwarding file for final version of chapter 2\\
% \end{tabular}
% \end{center}
% Each of the eight files can be compiled directly by the \LaTeX{} compiler.
%
% %%%%%%%%%%%%%%%%%%%%%%%%%%%%%%%%%%%%%%
% \paragraph{Main File.}
%
% The main file is called |cdocsamp.tex|.
%
% Load the \textsf{childdoc} definitions and
% declare the filename for the main document:
%    \begin{macrocode}
\input{childdoc.def}
\childdocmain{}
%    \end{macrocode}

% Optional override for |\version| flag:
%    \begin{macrocode}
%%\ifchilddoc\else\providecommand{\version}{draft}\fi
%    \end{macrocode}

% Define the default values for the |\version| flag
% (|final| for the main file and |draft| for childs):
%    \begin{macrocode}
\ifchilddoc
\providecommand{\version}{draft}
\else
\providecommand{\version}{final}
\fi
%    \end{macrocode}

% Load the standard document class:
%    \begin{macrocode}
\documentclass[12pt]{article}
%    \end{macrocode}

% Start the document body:
%    \begin{macrocode}
\begin{document}
%    \end{macrocode}

% Declare a title page.
% Print title, part of document being processed and version flag:
%    \begin{macrocode}
\addtocounter{page}{-1}
\begin{center}
{\LARGE\bfseries{}childdoc example\par}
\vspace{1cm}
\ifchilddoc
\ifchilddocmanual part\else chapter\fi:
`\childdocname' of `\childdocjob'\par
\else
main document: `\childdocjob'\par
\fi
version: \version\par
\end{center}
\newpage
%    \end{macrocode}

% Manually include selected file,
% otherwise process as usual:
%    \begin{macrocode}
\ifchilddocmanual
\section*{part `\childdocname'}
\input{\childdocname}
\else
%    \end{macrocode}

% Include the two chapters:
%    \begin{macrocode}
\include{cdocsch1}
\include{cdocsch2}
%    \end{macrocode}

% Include the two parts unless only chapters should be displayed:
%    \begin{macrocode}
\ifchilddoc\else
\section{part three}
\input{cdocspt3}
\section{part four}
\input{cdocspt4}
\fi
%    \end{macrocode}

% Process as usual until here:
%    \begin{macrocode}
\fi
%    \end{macrocode}

% End of document body:
%    \begin{macrocode}
\end{document}
%    \end{macrocode}
%\iffalse
%</samplemain>
%\fi
%
% %%%%%%%%%%%%%%%%%%%%%%%%%%%%%%%%%%%%%%
% \paragraph{Chapter Include Files.}
%
% The include files are called |cdocsch1.tex| and |cdocsch2.tex|.
%
%\iffalse
%<*samplechap1|samplechap2>
%\fi

% Optional override for |\version| flag:
%    \begin{macrocode}
%%\providecommand{\version}{final}
%    \end{macrocode}

% Include the main document:
%    \begin{macrocode}
\input{childdoc.def}
\childdocof{cdocsamp}
%    \end{macrocode}

%\iffalse
%</samplechap1|samplechap2>
%\fi
%
%\iffalse
%<*samplechap1>
%\fi
% Some text for chapter 1:
%    \begin{macrocode}
\section{one}
some text in chapter one
%    \end{macrocode}

%\iffalse
%</samplechap1>
%\fi
% Some text for chapter 2:
%\iffalse
%<*samplechap2>
%\fi
%    \begin{macrocode}
\section{two}
more text in chapter two
%    \end{macrocode}

%\iffalse
%</samplechap2>
%\fi
%
% %%%%%%%%%%%%%%%%%%%%%%%%%%%%%%%%%%%%%%
% \paragraph{Part Include Files.}
%
% The include files are called |cdocspt3.tex| and |cdocspt4.tex|.
%
%\iffalse
%<*samplepart3|samplepart4>
%\fi

% Optional override for |\version| flag:
%    \begin{macrocode}
%%\providecommand{\version}{final}
%    \end{macrocode}

% Include the main document:
%    \begin{macrocode}
\input{childdoc.def}
\childdocby{cdocsamp}
%    \end{macrocode}

%\iffalse
%</samplepart3|samplepart4>
%\fi
%
%\iffalse
%<*samplepart3>
%\fi
% Some text for part 3:
%    \begin{macrocode}
some text in part three
%    \end{macrocode}

%\iffalse
%</samplepart3>
%\fi
% Some text for part 4:
%\iffalse
%<*samplepart4>
%\fi
%    \begin{macrocode}
more text in part four
%    \end{macrocode}

%\iffalse
%</samplepart4>
%\fi
%
% %%%%%%%%%%%%%%%%%%%%%%%%%%%%%%%%%%%%%%
% \paragraph{Forwarding for a Complete Draft.}
%
% The following forwarding file |cdocsdrf.tex|
% compiles the main document in draft mode:
%\iffalse
%<*sampledraft>
%\fi
%    \begin{macrocode}
\def\version{draft}
\input{childdoc.def}
\childdocforward{cdocsamp}
%    \end{macrocode}

%\iffalse
%</sampledraft>
%\fi
%
% %%%%%%%%%%%%%%%%%%%%%%%%%%%%%%%%%%%%%%
% \paragraph{Forwarding for Final Version of the Chapters.}
%
% The following forwarding files |cdocsfn1.tex| and |cdocsfn2.tex|
% (with identical content)
% compile the final versions of the child documents
% |cdocsch1.tex| and |cdocsch2.tex|, respectively:
%\iffalse
%<*samplefinal>
%\fi
%    \begin{macrocode}
\def\version{final}
\input{childdoc.def}
\childdocforwardprefix[cdocsamp]{cdocsfn}{cdocsch}
%    \end{macrocode}

%\iffalse
%</samplefinal>
%\fi
%
% %%%%%%%%%%%%%%%%%%%%%%%%%%%%%%%%%%%%%%
% \paragraph{Command Line Processing.}
%
% The following three command lines generate the output files
% |cdocscld|, |cdocscl1| and |cdocscl2|
% which should be identical to
% |cdocsdrf|, |cdocsch1| and |cdocsfn2|, respectively:
% \begin{center}
% \begin{tabular}{l}
% |latex -jobname cdocscld \|\\
% |  "\def\version{draft}\input{childdoc.def}\childdocforward{cdocsamp}"|\\
% |latex -jobname cdocscl1 \|\\
% |  "\input{childdoc.def}\childdocforward[cdocsamp]{cdocsch1}"|\\
% |latex -jobname cdocscl2 \|\\
% |  "\def\version{final}\input{childdoc.def}\childdocforward{cdocsch2}"|
% \end{tabular}
% \end{center}
% Note that the trailing backslash on each first line
% merely continues the input to the second line
% (for convenient cut ant paste).
% Furthermore, the command |latex| can be replaced by any
% of its alternative versions such as |pdflatex|.
%
% %%%%%%%%%%%%%%%%%%%%%%%%%%%%%%%%%%%%%%%%%%%%%%%%%%%%%%%%%%%%%%%%%%%%%%%%%%%%%%
% %%%%%%%%%%%%%%%%%%%%%%%%%%%%%%%%%%%%%%%%%%%%%%%%%%%%%%%%%%%%%%%%%%%%%%%%%%%%%%
% \section{Implementation}
%\iffalse
%<*package>
%\fi
%
% This section describes the definitions file |childdoc.def|.

% The definitions cannot be loaded using |\usepackage| or |\RequirePackage|
% which has a mechanism to prevent loading a style file more than once.
% When loading the definitions by means of |\input|
% multiple instances have to be prevented manually:
%\iffalse
%This code needs to be before the `\ProvidesFile' directive
%which is defined at the beginning of this file.
%Therefore it is also placed there and commented out here.
%</package>
%<*discard>
%\fi
%    \begin{macrocode}
\ifdefined\childdocmain\endinput\fi
%    \end{macrocode}
%\iffalse
%</discard>
%<*package>
%\fi
%
% \macro{\ifchilddoc}
% \macro{\ifchilddocmanual}
% The conditional |\ifchilddoc| tells whether a
% child (true) or main (false) document is being compiled.
% The conditional |\ifchilddocmanual| tells whether
% the |\includeonly| mechanism is used (false) or
% the selection of child files must be performed manually (true).
% The definitions initialise to false:
%    \begin{macrocode}
\newif\ifchilddoc
\newif\ifchilddocmanual
%    \end{macrocode}

% \macro{\childdocname}
% \macro{\childdocjob}
% The macro |\childdocname| stores the name of the main document
% to be compiled. The macro |\childdocjob| stores the name of
% the document on which the \LaTeX{} compiler was originally invoked.
% The content of |\jobname| cannot be compared
% to filenames specified in the source due to different catcodes.
% The following code rescans |\jobname|, stores the result
% in |\childdocname| and saves a copy in |\childdocjob|:
%    \begin{macrocode}
\edef\childdocname{\scantokens\expandafter{\jobname\noexpand}}
\let\childdocjob\childdocname
%    \end{macrocode}

% \macro{\childdocdisable}
% The macro |\childdocdisable| prevents the main file
% from being processed more than once.
% At this stage, the main document command |\childdocmain|
% is assumed to be called once again where it should do nothing.
% Any subsequent call to it should prevent
% a secondary processing of the main document
% It overwrites the forwarding commands
% |\childdocof| and |\childdocforward|
% with empty macros to prevent further inclusions of the main document:
%    \begin{macrocode}
\newcommand{\childdocdisable}
{
  \renewcommand{\childdocmain}[1]{\renewcommand{\childdocmain}[1]{\endinput}}
  \renewcommand{\childdocof}[1]{}
  \renewcommand{\childdocby}[2][]{}
  \renewcommand{\childdocforward}[2][]{}
  \renewcommand{\childdocdisable}{}
}
%    \end{macrocode}

% \macro{\childdocmain}
% The macro |\childdocmain| is to be called at the top of the main file
% with nothing or the main filename (without extension) as argument.
% First, it breaks loops.
% If the argument is not empty and does not match |\childdocname|
% (which is set by the first inclusion of |childdoc.def|),
% |\ifchilddoc| is set to true, |\includeonly| is applied to the child file
% and |\jobname| is set to the main file
% (for proper handling of |.aux| files):
%    \begin{macrocode}
\newcommand{\childdocmain}[1]
{
  \childdocdisable\childdocmain{}
  \if?#1?\else
    \begingroup
      \def\childdoctmp{#1}
      \ifx\childdoctmp\childdocname
        \def\childdoctmp{}
      \else
        \def\childdoctmp
        {
          \childdoctrue
          \includeonly{\childdocname}
          \def\childdocjob{#1}
          \def\jobname{#1}
        }
      \fi
      \expandafter
    \endgroup
    \childdoctmp
  \fi
}
%    \end{macrocode}

% \macro{\childdocof}
% The command |\childdocof| redirects
% compilation to the main file |#1|.
%    \begin{macrocode}
\newcommand{\childdocof}[1]
{
  \childdocdisable
  \childdoctrue
  \includeonly{\childdocname}
  \def\jobname{#1}
  \def\childdocjob{#1}
  \input{#1}
}
%    \end{macrocode}

% \macro{\childdocby}
% The command |\childdocby| ....
%    \begin{macrocode}
\newcommand{\childdocby}[2][]
{
  \childdocdisable
  \childdoctrue
  \childdocmanualtrue
  \if?#1?\else
    \def\jobname{#2}
  \fi
  \def\childdocjob{#2}
  \input{#2}
  \endinput
}
%    \end{macrocode}

% \macro{\childdocforward}
% The command |\childdocforward| redirects
% compilation to the main file or
% (if the optional argument is given) a child file.
% Parameters are set as if the main file
% or a child file starting with |\childdocof| was compiled.
% Then compilation is handed over to the main file:
%    \begin{macrocode}
\newcommand{\childdocforward}[2][]
{
  \begingroup
    \if?#1?
      \def\childdoctmp
      {
        \def\childdocname{#2}
        \def\childdocjob{#2}
        \def\jobname{#2}
        \input{#2}
        \endinput
      }
    \else
      \def\childdoctmp
      {
        \childdocdisable
        \def\childdocname{#2}
        \childdoctrue
        \includeonly{#2}
        \def\childdocjob{#1}
        \def\jobname{#1}
        \input{#1}
        \endinput
      }
    \fi
    \expandafter
  \endgroup
  \childdoctmp
}
%    \end{macrocode}

% \macro{\childdocforwardprefix}
% The command |\childdocforwardprefix| redirects
% compilation to the main or a child file by means of a pattern.
% The prefix |#1| in the current filename is replaced by |#2|
% and the suffix of the current filename is kept
% (it is assumed that the filename does not contain the substring `|~~~|'
% which is used as a delimiter).
% Compilation is handed over to the new file by |\childdocforward|:
%    \begin{macrocode}
\newcommand{\childdocforwardprefix}[3][]
{
  \begingroup
    \def\childdocextract #2##1~~~{\def\childdoctmp{\childdocforward[#1]{#3##1}}}
    \expandafter\childdocextract\childdocname~~~
    \expandafter
  \endgroup
  \childdoctmp
}
%    \end{macrocode}

% \macro{\childdoc}
% The deprecated macro |\childdoc| is a legacy version of |\childdocmain|:
%    \begin{macrocode}
\newcommand{\childdoc}{\childdocmain}
%    \end{macrocode}

% \macro{\childdocredirect}
% The deprecated macro |\childdocredirect| is a legacy version
% of |\childdocforward| and |\childdocforwardprefix|:
%    \begin{macrocode}
\newcommand{\childdocredirect}[2][]
{
  \begingroup
    \if?#1?
      \def\childdoctmp{\childdocforward{#2}}
    \else
      \def\childdoctmp{\childdocforwardprefix{#1}{#2}}
    \fi
    \expandafter
  \endgroup
  \childdoctmp
}
%    \end{macrocode}

%\iffalse
%</package>
%\fi
%
\endinput
|
and perform the replacements as outlined below.
Instead of |\childdocmain{|\textit{main}|}| add the following code
to the top of the main file:
%
\begin{center}
\begin{tabular}{l}
|\||ifdefined\childdocname\endinput\||fi\newif\ifchilddoc|\\
|\edef\childdocname{\scantokens\expandafter{\jobname\noexpand}}|\\
|\def\childdocmain{|\textit{main}|}\||ifx\childdocmain\childdocname\||else|\\
|\childdoctrue\includeonly{\childdocname}\let\jobname\childdocmain\||fi|\\
\end{tabular}
\end{center}
%
Instead of |\childdocof{|\textit{main}|}| just include the main file
at the top of each child file:
%
\begin{center}
|\input{|\textit{main}|}|
\end{center}
%
A simple redirection |\childdocforward{|\textit{dest}|}| is achieved by:
%
\begin{center}
|\def\jobname{|\textit{dest}|}\input{\jobname}|
\end{center}
%
The redirection with prefix
|\childdocforwardprefix[|\textit{prefix}|]{|\textit{dest}|}|
is accomplished by:
%
\begin{center}
\begin{tabular}{l}
|{\edef\jobname{\scantokens\expandafter{\jobname\noexpand}}|\\
|\def\redirectjob |\textit{prefix}|#1~~~{\gdef\jobname{|\textit{dest}|#1}}|\\
|\expandafter\redirectjob\jobname~~~}\input{\jobname}|
\end{tabular}
\end{center}

In an alternative approach,
child documents can be compiled by a specific command line
without additional code or specific definitions:
%
\begin{center}
|... -jobname "|\textit{target}|" "|[\textit{flags}]%
|\includeonly{|\textit{dest}|}\input{|\textit{main}|}"|
\end{center}
%

%%%%%%%%%%%%%%%%%%%%%%%%%%%%%%%%%%%%%%%%%%%%%%%%%%%%%%%%%%%%%%%%%%%%%%%%%%%%%%%%
%%%%%%%%%%%%%%%%%%%%%%%%%%%%%%%%%%%%%%%%%%%%%%%%%%%%%%%%%%%%%%%%%%%%%%%%%%%%%%%%
\section{Information}

%%%%%%%%%%%%%%%%%%%%%%%%%%%%%%%%%%%%%%%%%%%%%%%%%%%%%%%%%%%%%%%%%%%%%%%%%%%%%%%%
\subsection{Copyright}

Copyright \copyright{} 2017--2018 Niklas Beisert

This work may be distributed and/or modified under the
conditions of the \LaTeX{} Project Public License, either version 1.3
of this license or (at your option) any later version.
The latest version of this license is in
  \url{http://www.latex-project.org/lppl.txt}
and version 1.3 or later is part of all distributions of \LaTeX{}
version 2005/12/01 or later.

This work has the LPPL maintenance status `maintained'.

The Current Maintainer of this work is Niklas Beisert.

This work consists of the files |README.txt|, |childdoc.ins| and |childdoc.dtx|
as well as the derived files |childdoc.def|, |cdocsamp.tex|
with |cdocsch1.tex|, |cdocsch2.tex|, |cdocspt3.tex|, |cdocspt4.tex|,
|cdocsdrf.tex|, |cdocsfn1.tex|, |cdocsfn2.tex|
as well as |childdoc.pdf|.

%%%%%%%%%%%%%%%%%%%%%%%%%%%%%%%%%%%%%%%%%%%%%%%%%%%%%%%%%%%%%%%%%%%%%%%%%%%%%%%%
\subsection{Files and Installation}

The package consists of the files:
%
\begin{center}
\begin{tabular}{ll}
    |README.txt|   & readme file \\
    |childdoc.ins| & installation file \\
    |childdoc.dtx| & source file \\
    |childdoc.def| & definition file \\
    |cdocsamp.tex| & sample main file \\
    |cdocsch1.tex| & sample include file \\
    |cdocsch2.tex| & sample include file \\
    |cdocspt3.tex| & sample part file \\
    |cdocspt4.tex| & sample part file \\
    |cdocsdrf.tex| & sample redirection file \\
    |cdocsfn1.tex| & sample redirection file \\
    |cdocsfn2.tex| & sample redirection file \\
    |childdoc.pdf| & manual
\end{tabular}
\end{center}
%
The distribution consists of the files
|README.txt|, |childdoc.ins| and |childdoc.dtx|.
%
\begin{itemize}
\item
Run (pdf)\LaTeX{} on |childdoc.dtx|
to compile the manual |childdoc.pdf| (this file).
\item
Run \LaTeX{} on |childdoc.ins| to create the definitions file |childdoc.def|
and the sample |cdocsamp.tex| with include files
|cdocsch1.tex|, |cdocsch2.tex|, |cdocspt3.tex|, |cdocspt4.tex|,
|cdocsdrf.tex|, |cdocsfn1.tex|, |cdocsfn2.tex|.
Then copy the file |childdoc.def| to an appropriate directory of your \LaTeX{}
distribution, e.g.\ \textit{texmf-root}|/tex/latex/childdoc|.
\end{itemize}

%%%%%%%%%%%%%%%%%%%%%%%%%%%%%%%%%%%%%%%%%%%%%%%%%%%%%%%%%%%%%%%%%%%%%%%%%%%%%%%%
\subsection{Related CTAN Packages}

There are several other packages which offer a similar functionality:
%
\begin{itemize}
\item
The packages
\href{http://ctan.org/pkg/docmute}{\textsf{docmute}},
\href{http://ctan.org/pkg/includex}{\textsf{includex}} and
\href{http://ctan.org/pkg/standalone}{\textsf{standalone}}
provide commands to include only the document body of
a child file thus allowing both files to be compiled individually.
\item
The packages \href{http://ctan.org/pkg/subdocs}{\textsf{subdocs}}
and \href{http://ctan.org/pkg/subfiles}{\textsf{subfiles}}
provide structures in which the main and child documents can be
encapsulated and allowing them to be compiled individually.
The inclusion mechanism is different from the conventional |\include|.
\item
The package \href{http://ctan.org/pkg/combine}{\textsf{combine}}
is an elaborate solution to combine several documents into one.
\end{itemize}
%
See also the CTAN topic \href{http://ctan.org/topic/subdocs}{\textsf{subdocs}}
for further related packages.
The present package differs from the above solutions in that
a document structure constructed with the conventional |\include| mechanism
just needs two extra commands at the top of every file
such that all constituent files can be compiled individually.

%%%%%%%%%%%%%%%%%%%%%%%%%%%%%%%%%%%%%%%%%%%%%%%%%%%%%%%%%%%%%%%%%%%%%%%%%%%%%%%%
%\subsection{Feature Suggestions}
%
%The following is a list of features which may be useful for future
%versions of this package:
%%
%\begin{itemize}
%\item
%\ldots
%\end{itemize}

%%%%%%%%%%%%%%%%%%%%%%%%%%%%%%%%%%%%%%%%%%%%%%%%%%%%%%%%%%%%%%%%%%%%%%%%%%%%%%%%
\subsection{Revision History}

%%%%%%%%%%%%%%%%%%%%%%%%%%%%%%%%%%%%%%%%
\paragraph{v2.0:} 2018/12/30

\begin{itemize}
\item
immediate forward processing
\item
added |\childdocby| mechanism
\item
manual restructured
\end{itemize}

%%%%%%%%%%%%%%%%%%%%%%%%%%%%%%%%%%%%%%%%
\paragraph{v1.6:} 2018/01/17

\begin{itemize}
\item
application for development of include files
\item
corrections to manual
\end{itemize}

%%%%%%%%%%%%%%%%%%%%%%%%%%%%%%%%%%%%%%%%
\paragraph{v1.5:} 2017/05/21

\begin{itemize}
\item
more complete structuring introduced
\item
|\childdocof| introduced
\item
|\childdoc| renamed to |\childdocmain|
\item
|\childredirect| renamed to |\childdocforward| and |\childdocforwardprefix|
and functionality expanded
\end{itemize}

%%%%%%%%%%%%%%%%%%%%%%%%%%%%%%%%%%%%%%%%
\paragraph{v1.0:} 2017/04/27

\begin{itemize}
\item
manual and install package
\item
first version published on CTAN
\end{itemize}

%%%%%%%%%%%%%%%%%%%%%%%%%%%%%%%%%%%%%%%%
\paragraph{v0.6:} 2017/04/26

\begin{itemize}
\item
redirection mechanism added
\end{itemize}

%%%%%%%%%%%%%%%%%%%%%%%%%%%%%%%%%%%%%%%%
\paragraph{v0.5:} 2017/04/26

\begin{itemize}
\item
functionality in definition file
\end{itemize}


%%%%%%%%%%%%%%%%%%%%%%%%%%%%%%%%%%%%%%%%%%%%%%%%%%%%%%%%%%%%%%%%%%%%%%%%%%%%%%%%
%%%%%%%%%%%%%%%%%%%%%%%%%%%%%%%%%%%%%%%%%%%%%%%%%%%%%%%%%%%%%%%%%%%%%%%%%%%%%%%%
%%%%%%%%%%%%%%%%%%%%%%%%%%%%%%%%%%%%%%%%%%%%%%%%%%%%%%%%%%%%%%%%%%%%%%%%%%%%%%%%
\appendix

\settowidth\MacroIndent{\rmfamily\scriptsize 000\ }

 \DocInput{childdoc.dtx}

\end{document}
%</driver>
% \fi
%
% %%%%%%%%%%%%%%%%%%%%%%%%%%%%%%%%%%%%%%%%%%%%%%%%%%%%%%%%%%%%%%%%%%%%%%%%%%%%%%
% %%%%%%%%%%%%%%%%%%%%%%%%%%%%%%%%%%%%%%%%%%%%%%%%%%%%%%%%%%%%%%%%%%%%%%%%%%%%%%
% \section{Sample}
%\iffalse
%<*samplemain>
%\fi
%
% The following presents a sample document
% with two chapters, two parts, a title page,
% a compile flag as well as three forwarding files to set the flag.
% It consists of eight |.tex| files:
% \begin{center}
% \begin{tabular}{ll}
% |cdocsamp.tex|&main file\\
% |cdocsch1.tex|&include file for chapter 1\\
% |cdocsch2.tex|&include file for chapter 2\\
% |cdocspt3.tex|&include file for part 3\\
% |cdocspt4.tex|&include file for part 4\\
% |cdocsdrf.tex|&forwarding file for main file in draft mode\\
% |cdocsfi1.tex|&forwarding file for final version of chapter 1\\
% |cdocsfi2.tex|&forwarding file for final version of chapter 2\\
% \end{tabular}
% \end{center}
% Each of the eight files can be compiled directly by the \LaTeX{} compiler.
%
% %%%%%%%%%%%%%%%%%%%%%%%%%%%%%%%%%%%%%%
% \paragraph{Main File.}
%
% The main file is called |cdocsamp.tex|.
%
% Load the \textsf{childdoc} definitions and
% declare the filename for the main document:
%    \begin{macrocode}
% \iffalse
%
% childdoc.dtx Copyright (C) 2017-2018 Niklas Beisert
%
% This work may be distributed and/or modified under the
% conditions of the LaTeX Project Public License, either version 1.3
% of this license or (at your option) any later version.
% The latest version of this license is in
%   http://www.latex-project.org/lppl.txt
% and version 1.3 or later is part of all distributions of LaTeX
% version 2005/12/01 or later.
%
% This work has the LPPL maintenance status `maintained'.
%
% The Current Maintainer of this work is Niklas Beisert.
%
% This work consists of the files childdoc.dtx and childdoc.ins
% and the derived files childdoc.def and cdocsamp.tex with
% cdocsch1.tex, cdocsch2.tex, cdocsdrf.tex, cdocsfn1.tex, cdocsfn2.tex.
%
%<package>\ifdefined\childdocmain\endinput\fi
%<package>\ProvidesFile{childdoc.def}[2018/12/30 v2.0 child document driver]
%<samplemain>\ProvidesFile{cdocsamp.tex}[2018/12/30 v2.0 sample for childdoc]
%<*driver>
%\ProvidesFile{childdoc.drv}[2018/12/30 v2.0 childdoc reference manual file]
\PassOptionsToClass{10pt,a4paper}{article}
\documentclass{ltxdoc}

\usepackage[margin=35mm]{geometry}
\usepackage{hyperref}
\usepackage{hyperxmp}
\usepackage[usenames]{color}

\hypersetup{colorlinks=true}
\hypersetup{pdfstartview=FitH}
\hypersetup{pdfpagemode=UseNone}
\hypersetup{pdfsource={}}
\hypersetup{pdflang={en-UK}}
\hypersetup{pdfcopyright={Copyright 2017-2018 Niklas Beisert.
  This work may be distributed and/or modified under the
  conditions of the LaTeX Project Public License, either version 1.3
  of this license or (at your option) any later version.}}
\hypersetup{pdflicenseurl={http://www.latex-project.org/lppl.txt}}
\hypersetup{pdfcontactaddress={ETH Zurich, ITP, HIT K,
  Wolfgang-Pauli-Strasse 27}}
\hypersetup{pdfcontactpostcode={8093}}
\hypersetup{pdfcontactcity={Zurich}}
\hypersetup{pdfcontactcountry={Switzerland}}
\hypersetup{pdfcontactemail={nbeisert@itp.phys.ethz.ch}}
\hypersetup{pdfcontacturl={http://people.phys.ethz.ch/\xmptilde nbeisert/}}

\newcommand{\secref}[1]{\hyperref[#1]{section \ref*{#1}}}

\parskip1ex
\parindent0pt
\let\olditemize\itemize
\def\itemize{\olditemize\parskip0pt}

\begin{document}

\title{The \textsf{childdoc} Package}
\hypersetup{pdftitle={The childdoc Package}}
\author{Niklas Beisert\\[2ex]
  Institut f\"ur Theoretische Physik\\
  Eidgen\"ossische Technische Hochschule Z\"urich\\
  Wolfgang-Pauli-Strasse 27, 8093 Z\"urich, Switzerland\\[1ex]
  \href{mailto:nbeisert@itp.phys.ethz.ch}
  {\texttt{nbeisert@itp.phys.ethz.ch}}}
\hypersetup{pdfauthor={Niklas Beisert}}
\hypersetup{pdfsubject={Manual for the LaTeX2e Package childdoc}}
\date{30 December 2018, \textsf{v2.0}}
\maketitle

\begin{abstract}\noindent
\textsf{childdoc} is a \LaTeXe{} package
that enables the direct compilation
of document sections included by |\include|
to individual files.
\end{abstract}

\begingroup
\parskip0ex
\tableofcontents
\endgroup

%%%%%%%%%%%%%%%%%%%%%%%%%%%%%%%%%%%%%%%%%%%%%%%%%%%%%%%%%%%%%%%%%%%%%%%%%%%%%%%%
%%%%%%%%%%%%%%%%%%%%%%%%%%%%%%%%%%%%%%%%%%%%%%%%%%%%%%%%%%%%%%%%%%%%%%%%%%%%%%%%
\section{Introduction}

\LaTeX{} provides a mechanism to structure a large document (such as a book)
into a main file and several child files (containing the chapters)
using the |\include| command.
This mechanism is beneficial for documents
which span hundreds of pages in order to
make the source file(s) more manageable.
Moreover, compilation can be restricted to
selected child files by means of the |\includeonly| command.
The latter feature can be used to reduce the compilation time while editing
(this was significantly more useful in the earlier days of \LaTeX{})
or to generate a smaller document which is easier to navigate.
Another application of |\includeonly| is to generate
documents consisting of selected parts of the complete document.

However, there are a few drawbacks of the plain |\include| mechanism:
\begin{itemize}
\item
The child files cannot be compiled on their own,
they can only be compiled via the main file.
A naive editing environment
(such as a text editor with an option
to have the current file processed by \LaTeX)
may require one to switch to the main file before compiling;
attempting to compile the child file produces errors.
\item
The main file must be modified (each time)
to adjust the |\includeonly| command
to the present needs. This easily leaves the main file in a messy state.
\item
The generated document will always carry the filename
of the main document. This is inconvenient if
several child files are to be compiled and
to be kept for distribution.
\end{itemize}

The present package provides a simple interface
to make child files individually compilable by \LaTeX{}.
Compiling a child file then has the same effect as compiling
the main file with an |\includeonly| command
to select the appropriate child.
Moreover the generated document will carry the name of the child
rather than the main file.
This resolves all three above issues.

This feature is meant to make the editing of books,
thesis documents and lecture notes somewhat more convenient.
However, the package can also be used efficiently for
composing a series of documents (such as exercise sheets)
which are typically distributed individually.
It then assists the author in generating the individual documents
(potentially in different versions)
as well as a document containing the collected series.
Another application is in developing style files
or other kinds of included material
where compilation of the style file could redirect
to a sample or test file.

%%%%%%%%%%%%%%%%%%%%%%%%%%%%%%%%%%%%%%%%%%%%%%%%%%%%%%%%%%%%%%%%%%%%%%%%%%%%%%%%
%%%%%%%%%%%%%%%%%%%%%%%%%%%%%%%%%%%%%%%%%%%%%%%%%%%%%%%%%%%%%%%%%%%%%%%%%%%%%%%%
\section{Usage}

First of all, the package \textsf{childdoc} is \emph{not} a standard
\LaTeXe{} |.sty| style file! Therefore it needs to be invoked in
a non-standard way.

%%%%%%%%%%%%%%%%%%%%%%%%%%%%%%%%%%%%%%%%%%%%%%%%%%%%%%%%%%%%%%%%%%%%%%%%%%%%%%%%
\subsection{Included Files}
\label{sec:include}

%%%%%%%%%%%%%%%%%%%%%%%%%%%%%%%%%%%%%%%%
\DescribeMacro{\childdocmain}
To use the package, add the commands
\begin{center}
\begin{tabular}{l}
|\input{childdoc.def}|\\
|\childdocmain{}|\\
\end{tabular}
\end{center}
at the very top of the main \LaTeX{} file,
in particular \emph{before} the |\documentclass| statement!
The argument of |\childdocmain| should be left empty
(but it must be present).

%%%%%%%%%%%%%%%%%%%%%%%%%%%%%%%%%%%%%%%%
\DescribeMacro{\childdocof}
Furthermore, add the commands
\begin{center}
\begin{tabular}{l}
|\input{childdoc.def}|\\
|\childdocof{|\textit{main}|}|\\
\end{tabular}
\end{center}
at the top of every child file \textit{child}
which is included by |\include{|\textit{child}|}|
from within the main file
(or at least for those files to be compiled individually).
The argument \textit{main} must be the filename of the main file.

There are a couple of
considerations in setting up the main and child documents:

%%%%%%%%%%%%%%%%%%%%%%%%%%%%%%%%%%%%%%%%
\paragraph{Restrictions.}

Please note the following restrictions:
\begin{itemize}
\item
|\childdocmain| must be called with one argument \textit{main}
to ensure compatibility with earlier version of the package.
It must either be empty (|\childdocmain{}|)
or precisely match the filename of the main file in which it is specified.
See \secref{sec:detection} for further information.
\item
The filename \textit{main} must be specified without the |.tex| extension.
\item
The filename \textit{main} is case sensitive
(even in case-insensitive file systems)
due to internal string comparison.
\item
The argument \textit{main} should be fully expanded, it cannot be a macro.
\item
Subdirectories and special characters should be avoided in filenames.
\item
The command |\childdocmain{|\textit{main}|}| must be followed by a whitespace.
It should not be followed immediately by another command
or by a comment mark `|%|'.
This is because the \TeX{} parser reads the token immediately following
the argument of |\childdocmain| and puts it
at the beginning of every child section;
however, a white\-space is ignored.
\end{itemize}

%%%%%%%%%%%%%%%%%%%%%%%%%%%%%%%%%%%%%%%%
\paragraph{Content of Main File.}

It is advisable to place all content in the child files included by |\include|.
Any output contained in the main file will appear in all child documents
unless suppressed manually;
it cannot be suppressed automatically by the |\includeonly| directive
and thus should normally be avoided.
A method to include some content in the main file
by means of conditional processing is described in \secref{sec:conditional}.

%%%%%%%%%%%%%%%%%%%%%%%%%%%%%%%%%%%%%%%%
\paragraph{Page Numbering.}

When only a part of the document is compiled,
the appropriate numbering of pages
(as well as other status parameters)
is determined from the |.aux| files.
The latter contain information from previous passes.
However this information needs to propagate through
all intermediate child documents.
Therefore the page numbering in child documents may well
be inconsistent until the complete document is compiled at least once.

A useful (if unconventional) way to always ensure a consistent
page numbering is to restart the numbering in each child document
and denote the pages by `\textit{child}|.|\textit{page}'
where \textit{child} represents the chapter/section number of the child file.
This can be achieved by the command
|\numberwithin{page}{|\textit{child}|}|
of the \textsf{amsmath} package
where \textit{child} can be |chapter| or |section|
depending on the chosen structuring.
Alternatively, one can modify the macro |\thepage| appropriately
and reset the counter |page| at the start of each child file.

%%%%%%%%%%%%%%%%%%%%%%%%%%%%%%%%%%%%%%%%%%%%%%%%%%%%%%%%%%%%%%%%%%%%%%%%%%%%%%%%
\subsection{Conditional Processing}
\label{sec:conditional}

The package provides a mechanism to compile different versions
of a document. To customise the versions further some conditional processing
can come in handy to distinguish which version is being compiled.
The package provides two macros to describe the compilation context:

%%%%%%%%%%%%%%%%%%%%%%%%%%%%%%%%%%%%%%%%
\DescribeMacro{\ifchilddoc}
The conditional |\ifchilddoc| distinguishes between the compilation of
child documents and the main document:
%
\begin{center}
|\ifchilddoc |\textit{child-code}| |[|\||else |\textit{main-code}]| \||fi|
\end{center}

%%%%%%%%%%%%%%%%%%%%%%%%%%%%%%%%%%%%%%%%
\DescribeMacro{\childdocname}
\DescribeMacro{\childdocjob}
The macro |\childdocname| contains the filename (without extension)
of the main or child file being processed.
Note that |\childdocjob| will always contain the name of the main file.

%%%%%%%%%%%%%%%%%%%%%%%%%%%%%%%%%%%%%%%%
\paragraph{Title Page.}

Conditional processing can be used to include a title or banner page
in the main document when proper precautions are taken.
Importantly, the code in the main file should ensure that the page counter
(as well as other status parameters which are stored in the |.aux| files)
takes the same value after the conditional processing.
Otherwise the page numbers may take divergent values
depending on which part is compiled.

For example, a title page could be declared by:
%
\begin{center}
\begin{tabular}{l}
|\ifchilddoc\||else|\\
|\addtocounter{page}{-1}|\\
\textit{code for title page}\\
|\newpage|\\
|\||fi|
\end{tabular}
\end{center}
%
A banner page for the child documents can be generated by:
%
\begin{center}
\begin{tabular}{l}
|\ifchilddoc|\\
|\addtocounter{page}{-1}|\\
\textit{code for banner page}\\
|\newpage|\\
|\||fi|
\end{tabular}
\end{center}
%
Here one could write a message such as:
\begin{center}
|This is the part \childdocname{} of \childdocjob{}.|
\end{center}

%%%%%%%%%%%%%%%%%%%%%%%%%%%%%%%%%%%%%%%%%%%%%%%%%%%%%%%%%%%%%%%%%%%%%%%%%%%%%%%%
\subsection{Flags}
\label{sec:flags}

The package makes it easy to generate different versions
of the main or child documents.
To this end compilation flags can be defined
and assigned different default values.
They will be particularly useful in conjunction
with the forwarding mechanism described in \secref{sec:forward}.

For example, it may be useful to have a flag |\version|
which can be set to |draft| or |final|.
The document source will contain some conditional code
depending on the value of |\version|.
Suppose further, the flag should default to |final| for the main file
and to |draft| for child files
which is a natural assignment for editing the document.
This is achieved by placing the following code
in the preamble of the main document
(below the |\childdocmain| directive):
%
\begin{center}
\begin{tabular}{l}
|\ifchilddoc|\\
|\providecommand{\version}{draft}|\\
|\||else|\\
|\providecommand{\version}{final}|\\
|\||fi|
\end{tabular}
\end{center}
%
The definition by |\providecommand| makes sure
that previous definitions are not overwritten.
Further statements |\providecommand{\version}{...}|
can thus be added before the above code to override it.

For the main file, one might add a line
(between |\childdocmain| and the above block)
%
\begin{center}
|%\ifchilddoc\||else\providecommand{\version}{draft}\||fi|
\end{center}
%
which can be uncommented to produce a draft version.
Likewise one can add a line to the very top of a child file
(above the |\childdocof{|\textit{main}|}| directive)
%
\begin{center}
|%\providecommand{\version}{final}|
\end{center}
%
which can be uncommented to produce the final version of this child document.

%%%%%%%%%%%%%%%%%%%%%%%%%%%%%%%%%%%%%%%%%%%%%%%%%%%%%%%%%%%%%%%%%%%%%%%%%%%%%%%%
\subsection{Forwarding}
\label{sec:forward}

Different versions of the main or child documents
using compilation flags as described in \secref{sec:flags}
can be (permanently) stored in different files
for convenient compilation, viewing and distribution.
To this end, the package defines a command
to pass on compilation to a different file:

%%%%%%%%%%%%%%%%%%%%%%%%%%%%%%%%%%%%%%%%
\DescribeMacro{\childdocforward}
The command |\childdocforward| redirects processing to
another source file:
%
\begin{center}
\begin{tabular}{l}
|\input{childdoc.def}|\\
|\childdocforward[|\textit{main}|]{|\textit{dest}|}|\\
\end{tabular}
\end{center}
%
The argument \textit{dest} is the destination file
(without extension).
It should be the main file or one of the child files.
Note that further \textsf{childdoc} directives
such as |\childdocof| and |\childdocforward|
in the indicated file will be processed in this form.
The optional argument \textit{main}
passes on directly to the main file \textit{main}
while pretending to compile the child \textit{dest}.
This form behaves as if \textit{dest}
issues |\childdocof{|\textit{main}|}| right away,
and no further \textsf{childdoc} directives will be processed.

%%%%%%%%%%%%%%%%%%%%%%%%%%%%%%%%%%%%%%%%
\DescribeMacro{\...prefix}
In the alternative form |\childdocforwardprefix|,
%
\begin{center}
\begin{tabular}{l}
|\input{childdoc.def}|\\
|\childdocforwardprefix[|\textit{main}|]{|\textit{prefix}|}{|\textit{dest}|}|
\end{tabular}
\end{center}
%
the destination file is determined by a pattern
depending on the current file:
To make this work, the current file must be called
`{\textit{prefix}\hspace{0.2em}\textit{suffix}}'
with \textit{prefix} matching precisely the argument.
Processing is then passed on to the file
`{\textit{dest}\hspace{0.2em}\textit{suffix}}'.
Surely, the same effect is achieved by
directly specifying the
argument `{\textit{dest}\hspace{0.2em}\textit{suffix}}'
in the first form.
However, that requires to set up a different file
for each child. With the alternative form of the command
all these files can have exactly the same content
which simplifies setting them up and maintaining them.

For example, the following file |draft.tex|
with a compilation flag |\version| as described in \secref{sec:flags}
compiles the main document as a draft:
%
\begin{center}
\begin{tabular}{l}
|\def\version{draft}|\\
|\input{childdoc.def}|\\
|\childdocforward{|\textit{main}|}|
\end{tabular}
\end{center}
%
Likewise, the following files |final|\textit{nn}|.tex|
compile the final version of the child document
|child|\textit{nn}|.tex|:
%
\begin{center}
\begin{tabular}{l}
|\def\version{final}|\\
|\input{childdoc.def}|\\
|\childdocforwardprefix{final}{child}|
\end{tabular}
\end{center}
%

Note that when several versions of a main file and/or of each child file
are to be generated, it may be convenient to set up a |Makefile| or
shell script to automatise the process.

%%%%%%%%%%%%%%%%%%%%%%%%%%%%%%%%%%%%%%%%%%%%%%%%%%%%%%%%%%%%%%%%%%%%%%%%%%%%%%%%
\subsection{Command Line Processing}
\label{sec:commandline}

The effect of redirection files can also be achieved by invoking
the \LaTeX{} compiler with a more elaborate command line.
Most conveniently this should be done as part
of a shell script or a |Makefile|.

When using \textsf{childdoc} in the main file, the following
command lines effectively perform a redirection
(note that depending on the shell being used,
backslashes may have to be doubled: `|\|' $\to$ `|\\|'):
%
\begin{center}
|... -jobname "|\textit{target}|" |\\|"|[\textit{flags}]%
|\input{childdoc.def}\childdocforward[|\textit{main}|]{|\textit{dest}|}"|
\end{center}
%
Here \textit{target} is the name of the output file,
\textit{main} is the name of the main file
and \textit{dest} is the name of the main or child file to be processed
(all filenames without extensions).
The optional argument \textit{main} can be omitted
if \textit{main} matches \textit{dest}.
Optionally, compilation \textit{flags} can be defined via |\def| commands.
This command line makes the \TeX{} engine believe
it is compiling the file \textit{target}
whose content is specified as the latter parameter.
The provided code then forwards the processing to
\textit{main} or \textit{dest} as described in \secref{sec:forward}.

%%%%%%%%%%%%%%%%%%%%%%%%%%%%%%%%%%%%%%%%%%%%%%%%%%%%%%%%%%%%%%%%%%%%%%%%%%%%%%%%
\subsection{Include by Input}
\label{sec:input}

Including child documents by |\include| has some restrictions by design.
Most notably, the content of a child document always occupies
its own set of pages; pages cannot be shared between child documents.
Usually, this behaviour makes perfect sense
because each child document contain an essential part of the document.
However, in some situations it may be desirable to compose
a document from a collection of parts
without having mandatory page breaks between then.
For this case, the package
provides a mechanism to include parts
by |\input| which can also be processed individually.
However, by construction this mechanism
requires manual handling of the content to be output.

%%%%%%%%%%%%%%%%%%%%%%%%%%%%%%%%%%%%%%%%
\DescribeMacro{\ifchilddocmanual}
The main file should be prepared as usual, see \secref{sec:include}.
However, the document body must make a distinction
between processing of an individual part and of the main document, e.g.:
%
\begin{center}
\begin{tabular}{l}
|\ifchilddocmanual|\\
|\input{\childdocname}|\\
|\||else|\\
\textit{document body with }|\input{|\textit{part}|}|\\
|\||fi|
\end{tabular}
\end{center}
%
The conditional |\ifchilddocmanual| is true whenever
a part to be included by |\input| is being compiled,
and the name of the part is stored in |\childdocname|.

%%%%%%%%%%%%%%%%%%%%%%%%%%%%%%%%%%%%%%%%
\DescribeMacro{\childdocby}
Each part to be included by |\input| should start with:
%
\begin{center}
\begin{tabular}{l}
|\input{childdoc.def}|\\
|\childdocby{|\textit{main}|}|\\
\end{tabular}
\end{center}
%
The directive |\childdocby| is similar to |\childdocof|
described in \secref{sec:include},
but the subsequent selection of content must be done manually.
To that end, both |\ifchilddoc| and |\ifchilddocmanual|
will be true upon processing of a part,
and the name of the part is stored in |\childdocname|.
Note that |\jobname| will be set to the filename of the current part
so that each part receives an individual |.aux| file
that does not interfere with the |.aux| file(s) of the main document.
This behaviour can be altered by the alternative form
|\childdocby[*]{|\textit{main}|}| (with a non-empty optional argument)
which uses the |.aux| file of the main document
by setting |\jobname| to \textit{main}.

%%%%%%%%%%%%%%%%%%%%%%%%%%%%%%%%%%%%%%%%%%%%%%%%%%%%%%%%%%%%%%%%%%%%%%%%%%%%%%%%
\subsection{Driver Development}
\label{sec:driver}

The \textsf{childdoc} mechanism can also be use for the development
of definition files such as \LaTeX{} styles or classes.
This case differs from the above setup with multiple parts
included by |\include| in that no |\includeonly| should be invoked.
This can be achieved by starting the include file
(before |\ProvidesPackage|) with:
%
\begin{center}
\begin{tabular}{l}
|\input{childdoc.def}|\\
|\childdocforward{|\textit{main}|}|\\
\end{tabular}
\end{center}
%
or alternatively with:
%
\begin{center}
\begin{tabular}{l}
|\input{childdoc.def}|\\
|\childdocby{|\textit{main}|}|\\
\end{tabular}
\end{center}
%
Both forms have slightly different effects as described above.
The main file is prepared as usual, see \secref{sec:include}.

%%%%%%%%%%%%%%%%%%%%%%%%%%%%%%%%%%%%%%%%%%%%%%%%%%%%%%%%%%%%%%%%%%%%%%%%%%%%%%%%
\subsection{Legacy Detection}
\label{sec:detection}

The directive |\childdocmain| in the main file can detect
whether the complete document or merely a child is to be compiled
even without using the directive |\childdocof|.
This method is deprecated because it is less robust
and there is no compelling reason to use it;
it is merely provided for backward compatibility
and it may be removed in future versions.

If the detection mechanism is to be used,
it is mandatory to correctly specify
the filename of the main file as the argument of |\childdocmain|:
%
\begin{center}
\begin{tabular}{l}
|\input{childdoc.def}|\\
|\childdocmain{|\textit{main}|}|\\
\end{tabular}
\end{center}
%
If |\jobname| does not match the argument \textit{main} of |\childdocmain|,
it is assumed that |\jobname| points to the child file to be compiled.
When using |\childdocmain| with the main file specified as argument,
it suffices to start a child file
with just |\input{|\textit{main}|}|
without loading of the package and using |\childdocof|.
If instead all processing is done
with the appropriate \textsf{childdoc} directives,
the argument of \textit{main} of |\childdocmain| can be empty.

An alternative version of the command line processing described
in \secref{sec:commandline} using the detection mechanism reads:
%
\begin{center}
|... -jobname "|\textit{target}|" "|[\textit{flags}]%
[|\def\jobname{|\textit{dest}|}|]|\input{|\textit{main}|}"|
\end{center}

%%%%%%%%%%%%%%%%%%%%%%%%%%%%%%%%%%%%%%%%%%%%%%%%%%%%%%%%%%%%%%%%%%%%%%%%%%%%%%%%
\subsection{Manual Code}
\label{sec:manual}

In case one cannot be certain whether the definitions file |childdoc.def|
is installed on the target \TeX{} distribution
and one prefers not to ship it,
it is conceivable to paste a few relevant commands into the sources.

To that end, drop all statements |\input{childdoc.def}|
and perform the replacements as outlined below.
Instead of |\childdocmain{|\textit{main}|}| add the following code
to the top of the main file:
%
\begin{center}
\begin{tabular}{l}
|\||ifdefined\childdocname\endinput\||fi\newif\ifchilddoc|\\
|\edef\childdocname{\scantokens\expandafter{\jobname\noexpand}}|\\
|\def\childdocmain{|\textit{main}|}\||ifx\childdocmain\childdocname\||else|\\
|\childdoctrue\includeonly{\childdocname}\let\jobname\childdocmain\||fi|\\
\end{tabular}
\end{center}
%
Instead of |\childdocof{|\textit{main}|}| just include the main file
at the top of each child file:
%
\begin{center}
|\input{|\textit{main}|}|
\end{center}
%
A simple redirection |\childdocforward{|\textit{dest}|}| is achieved by:
%
\begin{center}
|\def\jobname{|\textit{dest}|}\input{\jobname}|
\end{center}
%
The redirection with prefix
|\childdocforwardprefix[|\textit{prefix}|]{|\textit{dest}|}|
is accomplished by:
%
\begin{center}
\begin{tabular}{l}
|{\edef\jobname{\scantokens\expandafter{\jobname\noexpand}}|\\
|\def\redirectjob |\textit{prefix}|#1~~~{\gdef\jobname{|\textit{dest}|#1}}|\\
|\expandafter\redirectjob\jobname~~~}\input{\jobname}|
\end{tabular}
\end{center}

In an alternative approach,
child documents can be compiled by a specific command line
without additional code or specific definitions:
%
\begin{center}
|... -jobname "|\textit{target}|" "|[\textit{flags}]%
|\includeonly{|\textit{dest}|}\input{|\textit{main}|}"|
\end{center}
%

%%%%%%%%%%%%%%%%%%%%%%%%%%%%%%%%%%%%%%%%%%%%%%%%%%%%%%%%%%%%%%%%%%%%%%%%%%%%%%%%
%%%%%%%%%%%%%%%%%%%%%%%%%%%%%%%%%%%%%%%%%%%%%%%%%%%%%%%%%%%%%%%%%%%%%%%%%%%%%%%%
\section{Information}

%%%%%%%%%%%%%%%%%%%%%%%%%%%%%%%%%%%%%%%%%%%%%%%%%%%%%%%%%%%%%%%%%%%%%%%%%%%%%%%%
\subsection{Copyright}

Copyright \copyright{} 2017--2018 Niklas Beisert

This work may be distributed and/or modified under the
conditions of the \LaTeX{} Project Public License, either version 1.3
of this license or (at your option) any later version.
The latest version of this license is in
  \url{http://www.latex-project.org/lppl.txt}
and version 1.3 or later is part of all distributions of \LaTeX{}
version 2005/12/01 or later.

This work has the LPPL maintenance status `maintained'.

The Current Maintainer of this work is Niklas Beisert.

This work consists of the files |README.txt|, |childdoc.ins| and |childdoc.dtx|
as well as the derived files |childdoc.def|, |cdocsamp.tex|
with |cdocsch1.tex|, |cdocsch2.tex|, |cdocspt3.tex|, |cdocspt4.tex|,
|cdocsdrf.tex|, |cdocsfn1.tex|, |cdocsfn2.tex|
as well as |childdoc.pdf|.

%%%%%%%%%%%%%%%%%%%%%%%%%%%%%%%%%%%%%%%%%%%%%%%%%%%%%%%%%%%%%%%%%%%%%%%%%%%%%%%%
\subsection{Files and Installation}

The package consists of the files:
%
\begin{center}
\begin{tabular}{ll}
    |README.txt|   & readme file \\
    |childdoc.ins| & installation file \\
    |childdoc.dtx| & source file \\
    |childdoc.def| & definition file \\
    |cdocsamp.tex| & sample main file \\
    |cdocsch1.tex| & sample include file \\
    |cdocsch2.tex| & sample include file \\
    |cdocspt3.tex| & sample part file \\
    |cdocspt4.tex| & sample part file \\
    |cdocsdrf.tex| & sample redirection file \\
    |cdocsfn1.tex| & sample redirection file \\
    |cdocsfn2.tex| & sample redirection file \\
    |childdoc.pdf| & manual
\end{tabular}
\end{center}
%
The distribution consists of the files
|README.txt|, |childdoc.ins| and |childdoc.dtx|.
%
\begin{itemize}
\item
Run (pdf)\LaTeX{} on |childdoc.dtx|
to compile the manual |childdoc.pdf| (this file).
\item
Run \LaTeX{} on |childdoc.ins| to create the definitions file |childdoc.def|
and the sample |cdocsamp.tex| with include files
|cdocsch1.tex|, |cdocsch2.tex|, |cdocspt3.tex|, |cdocspt4.tex|,
|cdocsdrf.tex|, |cdocsfn1.tex|, |cdocsfn2.tex|.
Then copy the file |childdoc.def| to an appropriate directory of your \LaTeX{}
distribution, e.g.\ \textit{texmf-root}|/tex/latex/childdoc|.
\end{itemize}

%%%%%%%%%%%%%%%%%%%%%%%%%%%%%%%%%%%%%%%%%%%%%%%%%%%%%%%%%%%%%%%%%%%%%%%%%%%%%%%%
\subsection{Related CTAN Packages}

There are several other packages which offer a similar functionality:
%
\begin{itemize}
\item
The packages
\href{http://ctan.org/pkg/docmute}{\textsf{docmute}},
\href{http://ctan.org/pkg/includex}{\textsf{includex}} and
\href{http://ctan.org/pkg/standalone}{\textsf{standalone}}
provide commands to include only the document body of
a child file thus allowing both files to be compiled individually.
\item
The packages \href{http://ctan.org/pkg/subdocs}{\textsf{subdocs}}
and \href{http://ctan.org/pkg/subfiles}{\textsf{subfiles}}
provide structures in which the main and child documents can be
encapsulated and allowing them to be compiled individually.
The inclusion mechanism is different from the conventional |\include|.
\item
The package \href{http://ctan.org/pkg/combine}{\textsf{combine}}
is an elaborate solution to combine several documents into one.
\end{itemize}
%
See also the CTAN topic \href{http://ctan.org/topic/subdocs}{\textsf{subdocs}}
for further related packages.
The present package differs from the above solutions in that
a document structure constructed with the conventional |\include| mechanism
just needs two extra commands at the top of every file
such that all constituent files can be compiled individually.

%%%%%%%%%%%%%%%%%%%%%%%%%%%%%%%%%%%%%%%%%%%%%%%%%%%%%%%%%%%%%%%%%%%%%%%%%%%%%%%%
%\subsection{Feature Suggestions}
%
%The following is a list of features which may be useful for future
%versions of this package:
%%
%\begin{itemize}
%\item
%\ldots
%\end{itemize}

%%%%%%%%%%%%%%%%%%%%%%%%%%%%%%%%%%%%%%%%%%%%%%%%%%%%%%%%%%%%%%%%%%%%%%%%%%%%%%%%
\subsection{Revision History}

%%%%%%%%%%%%%%%%%%%%%%%%%%%%%%%%%%%%%%%%
\paragraph{v2.0:} 2018/12/30

\begin{itemize}
\item
immediate forward processing
\item
added |\childdocby| mechanism
\item
manual restructured
\end{itemize}

%%%%%%%%%%%%%%%%%%%%%%%%%%%%%%%%%%%%%%%%
\paragraph{v1.6:} 2018/01/17

\begin{itemize}
\item
application for development of include files
\item
corrections to manual
\end{itemize}

%%%%%%%%%%%%%%%%%%%%%%%%%%%%%%%%%%%%%%%%
\paragraph{v1.5:} 2017/05/21

\begin{itemize}
\item
more complete structuring introduced
\item
|\childdocof| introduced
\item
|\childdoc| renamed to |\childdocmain|
\item
|\childredirect| renamed to |\childdocforward| and |\childdocforwardprefix|
and functionality expanded
\end{itemize}

%%%%%%%%%%%%%%%%%%%%%%%%%%%%%%%%%%%%%%%%
\paragraph{v1.0:} 2017/04/27

\begin{itemize}
\item
manual and install package
\item
first version published on CTAN
\end{itemize}

%%%%%%%%%%%%%%%%%%%%%%%%%%%%%%%%%%%%%%%%
\paragraph{v0.6:} 2017/04/26

\begin{itemize}
\item
redirection mechanism added
\end{itemize}

%%%%%%%%%%%%%%%%%%%%%%%%%%%%%%%%%%%%%%%%
\paragraph{v0.5:} 2017/04/26

\begin{itemize}
\item
functionality in definition file
\end{itemize}


%%%%%%%%%%%%%%%%%%%%%%%%%%%%%%%%%%%%%%%%%%%%%%%%%%%%%%%%%%%%%%%%%%%%%%%%%%%%%%%%
%%%%%%%%%%%%%%%%%%%%%%%%%%%%%%%%%%%%%%%%%%%%%%%%%%%%%%%%%%%%%%%%%%%%%%%%%%%%%%%%
%%%%%%%%%%%%%%%%%%%%%%%%%%%%%%%%%%%%%%%%%%%%%%%%%%%%%%%%%%%%%%%%%%%%%%%%%%%%%%%%
\appendix

\settowidth\MacroIndent{\rmfamily\scriptsize 000\ }

 \DocInput{childdoc.dtx}

\end{document}
%</driver>
% \fi
%
% %%%%%%%%%%%%%%%%%%%%%%%%%%%%%%%%%%%%%%%%%%%%%%%%%%%%%%%%%%%%%%%%%%%%%%%%%%%%%%
% %%%%%%%%%%%%%%%%%%%%%%%%%%%%%%%%%%%%%%%%%%%%%%%%%%%%%%%%%%%%%%%%%%%%%%%%%%%%%%
% \section{Sample}
%\iffalse
%<*samplemain>
%\fi
%
% The following presents a sample document
% with two chapters, two parts, a title page,
% a compile flag as well as three forwarding files to set the flag.
% It consists of eight |.tex| files:
% \begin{center}
% \begin{tabular}{ll}
% |cdocsamp.tex|&main file\\
% |cdocsch1.tex|&include file for chapter 1\\
% |cdocsch2.tex|&include file for chapter 2\\
% |cdocspt3.tex|&include file for part 3\\
% |cdocspt4.tex|&include file for part 4\\
% |cdocsdrf.tex|&forwarding file for main file in draft mode\\
% |cdocsfi1.tex|&forwarding file for final version of chapter 1\\
% |cdocsfi2.tex|&forwarding file for final version of chapter 2\\
% \end{tabular}
% \end{center}
% Each of the eight files can be compiled directly by the \LaTeX{} compiler.
%
% %%%%%%%%%%%%%%%%%%%%%%%%%%%%%%%%%%%%%%
% \paragraph{Main File.}
%
% The main file is called |cdocsamp.tex|.
%
% Load the \textsf{childdoc} definitions and
% declare the filename for the main document:
%    \begin{macrocode}
\input{childdoc.def}
\childdocmain{}
%    \end{macrocode}

% Optional override for |\version| flag:
%    \begin{macrocode}
%%\ifchilddoc\else\providecommand{\version}{draft}\fi
%    \end{macrocode}

% Define the default values for the |\version| flag
% (|final| for the main file and |draft| for childs):
%    \begin{macrocode}
\ifchilddoc
\providecommand{\version}{draft}
\else
\providecommand{\version}{final}
\fi
%    \end{macrocode}

% Load the standard document class:
%    \begin{macrocode}
\documentclass[12pt]{article}
%    \end{macrocode}

% Start the document body:
%    \begin{macrocode}
\begin{document}
%    \end{macrocode}

% Declare a title page.
% Print title, part of document being processed and version flag:
%    \begin{macrocode}
\addtocounter{page}{-1}
\begin{center}
{\LARGE\bfseries{}childdoc example\par}
\vspace{1cm}
\ifchilddoc
\ifchilddocmanual part\else chapter\fi:
`\childdocname' of `\childdocjob'\par
\else
main document: `\childdocjob'\par
\fi
version: \version\par
\end{center}
\newpage
%    \end{macrocode}

% Manually include selected file,
% otherwise process as usual:
%    \begin{macrocode}
\ifchilddocmanual
\section*{part `\childdocname'}
\input{\childdocname}
\else
%    \end{macrocode}

% Include the two chapters:
%    \begin{macrocode}
\include{cdocsch1}
\include{cdocsch2}
%    \end{macrocode}

% Include the two parts unless only chapters should be displayed:
%    \begin{macrocode}
\ifchilddoc\else
\section{part three}
\input{cdocspt3}
\section{part four}
\input{cdocspt4}
\fi
%    \end{macrocode}

% Process as usual until here:
%    \begin{macrocode}
\fi
%    \end{macrocode}

% End of document body:
%    \begin{macrocode}
\end{document}
%    \end{macrocode}
%\iffalse
%</samplemain>
%\fi
%
% %%%%%%%%%%%%%%%%%%%%%%%%%%%%%%%%%%%%%%
% \paragraph{Chapter Include Files.}
%
% The include files are called |cdocsch1.tex| and |cdocsch2.tex|.
%
%\iffalse
%<*samplechap1|samplechap2>
%\fi

% Optional override for |\version| flag:
%    \begin{macrocode}
%%\providecommand{\version}{final}
%    \end{macrocode}

% Include the main document:
%    \begin{macrocode}
\input{childdoc.def}
\childdocof{cdocsamp}
%    \end{macrocode}

%\iffalse
%</samplechap1|samplechap2>
%\fi
%
%\iffalse
%<*samplechap1>
%\fi
% Some text for chapter 1:
%    \begin{macrocode}
\section{one}
some text in chapter one
%    \end{macrocode}

%\iffalse
%</samplechap1>
%\fi
% Some text for chapter 2:
%\iffalse
%<*samplechap2>
%\fi
%    \begin{macrocode}
\section{two}
more text in chapter two
%    \end{macrocode}

%\iffalse
%</samplechap2>
%\fi
%
% %%%%%%%%%%%%%%%%%%%%%%%%%%%%%%%%%%%%%%
% \paragraph{Part Include Files.}
%
% The include files are called |cdocspt3.tex| and |cdocspt4.tex|.
%
%\iffalse
%<*samplepart3|samplepart4>
%\fi

% Optional override for |\version| flag:
%    \begin{macrocode}
%%\providecommand{\version}{final}
%    \end{macrocode}

% Include the main document:
%    \begin{macrocode}
\input{childdoc.def}
\childdocby{cdocsamp}
%    \end{macrocode}

%\iffalse
%</samplepart3|samplepart4>
%\fi
%
%\iffalse
%<*samplepart3>
%\fi
% Some text for part 3:
%    \begin{macrocode}
some text in part three
%    \end{macrocode}

%\iffalse
%</samplepart3>
%\fi
% Some text for part 4:
%\iffalse
%<*samplepart4>
%\fi
%    \begin{macrocode}
more text in part four
%    \end{macrocode}

%\iffalse
%</samplepart4>
%\fi
%
% %%%%%%%%%%%%%%%%%%%%%%%%%%%%%%%%%%%%%%
% \paragraph{Forwarding for a Complete Draft.}
%
% The following forwarding file |cdocsdrf.tex|
% compiles the main document in draft mode:
%\iffalse
%<*sampledraft>
%\fi
%    \begin{macrocode}
\def\version{draft}
\input{childdoc.def}
\childdocforward{cdocsamp}
%    \end{macrocode}

%\iffalse
%</sampledraft>
%\fi
%
% %%%%%%%%%%%%%%%%%%%%%%%%%%%%%%%%%%%%%%
% \paragraph{Forwarding for Final Version of the Chapters.}
%
% The following forwarding files |cdocsfn1.tex| and |cdocsfn2.tex|
% (with identical content)
% compile the final versions of the child documents
% |cdocsch1.tex| and |cdocsch2.tex|, respectively:
%\iffalse
%<*samplefinal>
%\fi
%    \begin{macrocode}
\def\version{final}
\input{childdoc.def}
\childdocforwardprefix[cdocsamp]{cdocsfn}{cdocsch}
%    \end{macrocode}

%\iffalse
%</samplefinal>
%\fi
%
% %%%%%%%%%%%%%%%%%%%%%%%%%%%%%%%%%%%%%%
% \paragraph{Command Line Processing.}
%
% The following three command lines generate the output files
% |cdocscld|, |cdocscl1| and |cdocscl2|
% which should be identical to
% |cdocsdrf|, |cdocsch1| and |cdocsfn2|, respectively:
% \begin{center}
% \begin{tabular}{l}
% |latex -jobname cdocscld \|\\
% |  "\def\version{draft}\input{childdoc.def}\childdocforward{cdocsamp}"|\\
% |latex -jobname cdocscl1 \|\\
% |  "\input{childdoc.def}\childdocforward[cdocsamp]{cdocsch1}"|\\
% |latex -jobname cdocscl2 \|\\
% |  "\def\version{final}\input{childdoc.def}\childdocforward{cdocsch2}"|
% \end{tabular}
% \end{center}
% Note that the trailing backslash on each first line
% merely continues the input to the second line
% (for convenient cut ant paste).
% Furthermore, the command |latex| can be replaced by any
% of its alternative versions such as |pdflatex|.
%
% %%%%%%%%%%%%%%%%%%%%%%%%%%%%%%%%%%%%%%%%%%%%%%%%%%%%%%%%%%%%%%%%%%%%%%%%%%%%%%
% %%%%%%%%%%%%%%%%%%%%%%%%%%%%%%%%%%%%%%%%%%%%%%%%%%%%%%%%%%%%%%%%%%%%%%%%%%%%%%
% \section{Implementation}
%\iffalse
%<*package>
%\fi
%
% This section describes the definitions file |childdoc.def|.

% The definitions cannot be loaded using |\usepackage| or |\RequirePackage|
% which has a mechanism to prevent loading a style file more than once.
% When loading the definitions by means of |\input|
% multiple instances have to be prevented manually:
%\iffalse
%This code needs to be before the `\ProvidesFile' directive
%which is defined at the beginning of this file.
%Therefore it is also placed there and commented out here.
%</package>
%<*discard>
%\fi
%    \begin{macrocode}
\ifdefined\childdocmain\endinput\fi
%    \end{macrocode}
%\iffalse
%</discard>
%<*package>
%\fi
%
% \macro{\ifchilddoc}
% \macro{\ifchilddocmanual}
% The conditional |\ifchilddoc| tells whether a
% child (true) or main (false) document is being compiled.
% The conditional |\ifchilddocmanual| tells whether
% the |\includeonly| mechanism is used (false) or
% the selection of child files must be performed manually (true).
% The definitions initialise to false:
%    \begin{macrocode}
\newif\ifchilddoc
\newif\ifchilddocmanual
%    \end{macrocode}

% \macro{\childdocname}
% \macro{\childdocjob}
% The macro |\childdocname| stores the name of the main document
% to be compiled. The macro |\childdocjob| stores the name of
% the document on which the \LaTeX{} compiler was originally invoked.
% The content of |\jobname| cannot be compared
% to filenames specified in the source due to different catcodes.
% The following code rescans |\jobname|, stores the result
% in |\childdocname| and saves a copy in |\childdocjob|:
%    \begin{macrocode}
\edef\childdocname{\scantokens\expandafter{\jobname\noexpand}}
\let\childdocjob\childdocname
%    \end{macrocode}

% \macro{\childdocdisable}
% The macro |\childdocdisable| prevents the main file
% from being processed more than once.
% At this stage, the main document command |\childdocmain|
% is assumed to be called once again where it should do nothing.
% Any subsequent call to it should prevent
% a secondary processing of the main document
% It overwrites the forwarding commands
% |\childdocof| and |\childdocforward|
% with empty macros to prevent further inclusions of the main document:
%    \begin{macrocode}
\newcommand{\childdocdisable}
{
  \renewcommand{\childdocmain}[1]{\renewcommand{\childdocmain}[1]{\endinput}}
  \renewcommand{\childdocof}[1]{}
  \renewcommand{\childdocby}[2][]{}
  \renewcommand{\childdocforward}[2][]{}
  \renewcommand{\childdocdisable}{}
}
%    \end{macrocode}

% \macro{\childdocmain}
% The macro |\childdocmain| is to be called at the top of the main file
% with nothing or the main filename (without extension) as argument.
% First, it breaks loops.
% If the argument is not empty and does not match |\childdocname|
% (which is set by the first inclusion of |childdoc.def|),
% |\ifchilddoc| is set to true, |\includeonly| is applied to the child file
% and |\jobname| is set to the main file
% (for proper handling of |.aux| files):
%    \begin{macrocode}
\newcommand{\childdocmain}[1]
{
  \childdocdisable\childdocmain{}
  \if?#1?\else
    \begingroup
      \def\childdoctmp{#1}
      \ifx\childdoctmp\childdocname
        \def\childdoctmp{}
      \else
        \def\childdoctmp
        {
          \childdoctrue
          \includeonly{\childdocname}
          \def\childdocjob{#1}
          \def\jobname{#1}
        }
      \fi
      \expandafter
    \endgroup
    \childdoctmp
  \fi
}
%    \end{macrocode}

% \macro{\childdocof}
% The command |\childdocof| redirects
% compilation to the main file |#1|.
%    \begin{macrocode}
\newcommand{\childdocof}[1]
{
  \childdocdisable
  \childdoctrue
  \includeonly{\childdocname}
  \def\jobname{#1}
  \def\childdocjob{#1}
  \input{#1}
}
%    \end{macrocode}

% \macro{\childdocby}
% The command |\childdocby| ....
%    \begin{macrocode}
\newcommand{\childdocby}[2][]
{
  \childdocdisable
  \childdoctrue
  \childdocmanualtrue
  \if?#1?\else
    \def\jobname{#2}
  \fi
  \def\childdocjob{#2}
  \input{#2}
  \endinput
}
%    \end{macrocode}

% \macro{\childdocforward}
% The command |\childdocforward| redirects
% compilation to the main file or
% (if the optional argument is given) a child file.
% Parameters are set as if the main file
% or a child file starting with |\childdocof| was compiled.
% Then compilation is handed over to the main file:
%    \begin{macrocode}
\newcommand{\childdocforward}[2][]
{
  \begingroup
    \if?#1?
      \def\childdoctmp
      {
        \def\childdocname{#2}
        \def\childdocjob{#2}
        \def\jobname{#2}
        \input{#2}
        \endinput
      }
    \else
      \def\childdoctmp
      {
        \childdocdisable
        \def\childdocname{#2}
        \childdoctrue
        \includeonly{#2}
        \def\childdocjob{#1}
        \def\jobname{#1}
        \input{#1}
        \endinput
      }
    \fi
    \expandafter
  \endgroup
  \childdoctmp
}
%    \end{macrocode}

% \macro{\childdocforwardprefix}
% The command |\childdocforwardprefix| redirects
% compilation to the main or a child file by means of a pattern.
% The prefix |#1| in the current filename is replaced by |#2|
% and the suffix of the current filename is kept
% (it is assumed that the filename does not contain the substring `|~~~|'
% which is used as a delimiter).
% Compilation is handed over to the new file by |\childdocforward|:
%    \begin{macrocode}
\newcommand{\childdocforwardprefix}[3][]
{
  \begingroup
    \def\childdocextract #2##1~~~{\def\childdoctmp{\childdocforward[#1]{#3##1}}}
    \expandafter\childdocextract\childdocname~~~
    \expandafter
  \endgroup
  \childdoctmp
}
%    \end{macrocode}

% \macro{\childdoc}
% The deprecated macro |\childdoc| is a legacy version of |\childdocmain|:
%    \begin{macrocode}
\newcommand{\childdoc}{\childdocmain}
%    \end{macrocode}

% \macro{\childdocredirect}
% The deprecated macro |\childdocredirect| is a legacy version
% of |\childdocforward| and |\childdocforwardprefix|:
%    \begin{macrocode}
\newcommand{\childdocredirect}[2][]
{
  \begingroup
    \if?#1?
      \def\childdoctmp{\childdocforward{#2}}
    \else
      \def\childdoctmp{\childdocforwardprefix{#1}{#2}}
    \fi
    \expandafter
  \endgroup
  \childdoctmp
}
%    \end{macrocode}

%\iffalse
%</package>
%\fi
%
\endinput

\childdocmain{}
%    \end{macrocode}

% Optional override for |\version| flag:
%    \begin{macrocode}
%%\ifchilddoc\else\providecommand{\version}{draft}\fi
%    \end{macrocode}

% Define the default values for the |\version| flag
% (|final| for the main file and |draft| for childs):
%    \begin{macrocode}
\ifchilddoc
\providecommand{\version}{draft}
\else
\providecommand{\version}{final}
\fi
%    \end{macrocode}

% Load the standard document class:
%    \begin{macrocode}
\documentclass[12pt]{article}
%    \end{macrocode}

% Start the document body:
%    \begin{macrocode}
\begin{document}
%    \end{macrocode}

% Declare a title page.
% Print title, part of document being processed and version flag:
%    \begin{macrocode}
\addtocounter{page}{-1}
\begin{center}
{\LARGE\bfseries{}childdoc example\par}
\vspace{1cm}
\ifchilddoc
\ifchilddocmanual part\else chapter\fi:
`\childdocname' of `\childdocjob'\par
\else
main document: `\childdocjob'\par
\fi
version: \version\par
\end{center}
\newpage
%    \end{macrocode}

% Manually include selected file,
% otherwise process as usual:
%    \begin{macrocode}
\ifchilddocmanual
\section*{part `\childdocname'}
\input{\childdocname}
\else
%    \end{macrocode}

% Include the two chapters:
%    \begin{macrocode}
\include{cdocsch1}
\include{cdocsch2}
%    \end{macrocode}

% Include the two parts unless only chapters should be displayed:
%    \begin{macrocode}
\ifchilddoc\else
\section{part three}
\input{cdocspt3}
\section{part four}
\input{cdocspt4}
\fi
%    \end{macrocode}

% Process as usual until here:
%    \begin{macrocode}
\fi
%    \end{macrocode}

% End of document body:
%    \begin{macrocode}
\end{document}
%    \end{macrocode}
%\iffalse
%</samplemain>
%\fi
%
% %%%%%%%%%%%%%%%%%%%%%%%%%%%%%%%%%%%%%%
% \paragraph{Chapter Include Files.}
%
% The include files are called |cdocsch1.tex| and |cdocsch2.tex|.
%
%\iffalse
%<*samplechap1|samplechap2>
%\fi

% Optional override for |\version| flag:
%    \begin{macrocode}
%%\providecommand{\version}{final}
%    \end{macrocode}

% Include the main document:
%    \begin{macrocode}
% \iffalse
%
% childdoc.dtx Copyright (C) 2017-2018 Niklas Beisert
%
% This work may be distributed and/or modified under the
% conditions of the LaTeX Project Public License, either version 1.3
% of this license or (at your option) any later version.
% The latest version of this license is in
%   http://www.latex-project.org/lppl.txt
% and version 1.3 or later is part of all distributions of LaTeX
% version 2005/12/01 or later.
%
% This work has the LPPL maintenance status `maintained'.
%
% The Current Maintainer of this work is Niklas Beisert.
%
% This work consists of the files childdoc.dtx and childdoc.ins
% and the derived files childdoc.def and cdocsamp.tex with
% cdocsch1.tex, cdocsch2.tex, cdocsdrf.tex, cdocsfn1.tex, cdocsfn2.tex.
%
%<package>\ifdefined\childdocmain\endinput\fi
%<package>\ProvidesFile{childdoc.def}[2018/12/30 v2.0 child document driver]
%<samplemain>\ProvidesFile{cdocsamp.tex}[2018/12/30 v2.0 sample for childdoc]
%<*driver>
%\ProvidesFile{childdoc.drv}[2018/12/30 v2.0 childdoc reference manual file]
\PassOptionsToClass{10pt,a4paper}{article}
\documentclass{ltxdoc}

\usepackage[margin=35mm]{geometry}
\usepackage{hyperref}
\usepackage{hyperxmp}
\usepackage[usenames]{color}

\hypersetup{colorlinks=true}
\hypersetup{pdfstartview=FitH}
\hypersetup{pdfpagemode=UseNone}
\hypersetup{pdfsource={}}
\hypersetup{pdflang={en-UK}}
\hypersetup{pdfcopyright={Copyright 2017-2018 Niklas Beisert.
  This work may be distributed and/or modified under the
  conditions of the LaTeX Project Public License, either version 1.3
  of this license or (at your option) any later version.}}
\hypersetup{pdflicenseurl={http://www.latex-project.org/lppl.txt}}
\hypersetup{pdfcontactaddress={ETH Zurich, ITP, HIT K,
  Wolfgang-Pauli-Strasse 27}}
\hypersetup{pdfcontactpostcode={8093}}
\hypersetup{pdfcontactcity={Zurich}}
\hypersetup{pdfcontactcountry={Switzerland}}
\hypersetup{pdfcontactemail={nbeisert@itp.phys.ethz.ch}}
\hypersetup{pdfcontacturl={http://people.phys.ethz.ch/\xmptilde nbeisert/}}

\newcommand{\secref}[1]{\hyperref[#1]{section \ref*{#1}}}

\parskip1ex
\parindent0pt
\let\olditemize\itemize
\def\itemize{\olditemize\parskip0pt}

\begin{document}

\title{The \textsf{childdoc} Package}
\hypersetup{pdftitle={The childdoc Package}}
\author{Niklas Beisert\\[2ex]
  Institut f\"ur Theoretische Physik\\
  Eidgen\"ossische Technische Hochschule Z\"urich\\
  Wolfgang-Pauli-Strasse 27, 8093 Z\"urich, Switzerland\\[1ex]
  \href{mailto:nbeisert@itp.phys.ethz.ch}
  {\texttt{nbeisert@itp.phys.ethz.ch}}}
\hypersetup{pdfauthor={Niklas Beisert}}
\hypersetup{pdfsubject={Manual for the LaTeX2e Package childdoc}}
\date{30 December 2018, \textsf{v2.0}}
\maketitle

\begin{abstract}\noindent
\textsf{childdoc} is a \LaTeXe{} package
that enables the direct compilation
of document sections included by |\include|
to individual files.
\end{abstract}

\begingroup
\parskip0ex
\tableofcontents
\endgroup

%%%%%%%%%%%%%%%%%%%%%%%%%%%%%%%%%%%%%%%%%%%%%%%%%%%%%%%%%%%%%%%%%%%%%%%%%%%%%%%%
%%%%%%%%%%%%%%%%%%%%%%%%%%%%%%%%%%%%%%%%%%%%%%%%%%%%%%%%%%%%%%%%%%%%%%%%%%%%%%%%
\section{Introduction}

\LaTeX{} provides a mechanism to structure a large document (such as a book)
into a main file and several child files (containing the chapters)
using the |\include| command.
This mechanism is beneficial for documents
which span hundreds of pages in order to
make the source file(s) more manageable.
Moreover, compilation can be restricted to
selected child files by means of the |\includeonly| command.
The latter feature can be used to reduce the compilation time while editing
(this was significantly more useful in the earlier days of \LaTeX{})
or to generate a smaller document which is easier to navigate.
Another application of |\includeonly| is to generate
documents consisting of selected parts of the complete document.

However, there are a few drawbacks of the plain |\include| mechanism:
\begin{itemize}
\item
The child files cannot be compiled on their own,
they can only be compiled via the main file.
A naive editing environment
(such as a text editor with an option
to have the current file processed by \LaTeX)
may require one to switch to the main file before compiling;
attempting to compile the child file produces errors.
\item
The main file must be modified (each time)
to adjust the |\includeonly| command
to the present needs. This easily leaves the main file in a messy state.
\item
The generated document will always carry the filename
of the main document. This is inconvenient if
several child files are to be compiled and
to be kept for distribution.
\end{itemize}

The present package provides a simple interface
to make child files individually compilable by \LaTeX{}.
Compiling a child file then has the same effect as compiling
the main file with an |\includeonly| command
to select the appropriate child.
Moreover the generated document will carry the name of the child
rather than the main file.
This resolves all three above issues.

This feature is meant to make the editing of books,
thesis documents and lecture notes somewhat more convenient.
However, the package can also be used efficiently for
composing a series of documents (such as exercise sheets)
which are typically distributed individually.
It then assists the author in generating the individual documents
(potentially in different versions)
as well as a document containing the collected series.
Another application is in developing style files
or other kinds of included material
where compilation of the style file could redirect
to a sample or test file.

%%%%%%%%%%%%%%%%%%%%%%%%%%%%%%%%%%%%%%%%%%%%%%%%%%%%%%%%%%%%%%%%%%%%%%%%%%%%%%%%
%%%%%%%%%%%%%%%%%%%%%%%%%%%%%%%%%%%%%%%%%%%%%%%%%%%%%%%%%%%%%%%%%%%%%%%%%%%%%%%%
\section{Usage}

First of all, the package \textsf{childdoc} is \emph{not} a standard
\LaTeXe{} |.sty| style file! Therefore it needs to be invoked in
a non-standard way.

%%%%%%%%%%%%%%%%%%%%%%%%%%%%%%%%%%%%%%%%%%%%%%%%%%%%%%%%%%%%%%%%%%%%%%%%%%%%%%%%
\subsection{Included Files}
\label{sec:include}

%%%%%%%%%%%%%%%%%%%%%%%%%%%%%%%%%%%%%%%%
\DescribeMacro{\childdocmain}
To use the package, add the commands
\begin{center}
\begin{tabular}{l}
|\input{childdoc.def}|\\
|\childdocmain{}|\\
\end{tabular}
\end{center}
at the very top of the main \LaTeX{} file,
in particular \emph{before} the |\documentclass| statement!
The argument of |\childdocmain| should be left empty
(but it must be present).

%%%%%%%%%%%%%%%%%%%%%%%%%%%%%%%%%%%%%%%%
\DescribeMacro{\childdocof}
Furthermore, add the commands
\begin{center}
\begin{tabular}{l}
|\input{childdoc.def}|\\
|\childdocof{|\textit{main}|}|\\
\end{tabular}
\end{center}
at the top of every child file \textit{child}
which is included by |\include{|\textit{child}|}|
from within the main file
(or at least for those files to be compiled individually).
The argument \textit{main} must be the filename of the main file.

There are a couple of
considerations in setting up the main and child documents:

%%%%%%%%%%%%%%%%%%%%%%%%%%%%%%%%%%%%%%%%
\paragraph{Restrictions.}

Please note the following restrictions:
\begin{itemize}
\item
|\childdocmain| must be called with one argument \textit{main}
to ensure compatibility with earlier version of the package.
It must either be empty (|\childdocmain{}|)
or precisely match the filename of the main file in which it is specified.
See \secref{sec:detection} for further information.
\item
The filename \textit{main} must be specified without the |.tex| extension.
\item
The filename \textit{main} is case sensitive
(even in case-insensitive file systems)
due to internal string comparison.
\item
The argument \textit{main} should be fully expanded, it cannot be a macro.
\item
Subdirectories and special characters should be avoided in filenames.
\item
The command |\childdocmain{|\textit{main}|}| must be followed by a whitespace.
It should not be followed immediately by another command
or by a comment mark `|%|'.
This is because the \TeX{} parser reads the token immediately following
the argument of |\childdocmain| and puts it
at the beginning of every child section;
however, a white\-space is ignored.
\end{itemize}

%%%%%%%%%%%%%%%%%%%%%%%%%%%%%%%%%%%%%%%%
\paragraph{Content of Main File.}

It is advisable to place all content in the child files included by |\include|.
Any output contained in the main file will appear in all child documents
unless suppressed manually;
it cannot be suppressed automatically by the |\includeonly| directive
and thus should normally be avoided.
A method to include some content in the main file
by means of conditional processing is described in \secref{sec:conditional}.

%%%%%%%%%%%%%%%%%%%%%%%%%%%%%%%%%%%%%%%%
\paragraph{Page Numbering.}

When only a part of the document is compiled,
the appropriate numbering of pages
(as well as other status parameters)
is determined from the |.aux| files.
The latter contain information from previous passes.
However this information needs to propagate through
all intermediate child documents.
Therefore the page numbering in child documents may well
be inconsistent until the complete document is compiled at least once.

A useful (if unconventional) way to always ensure a consistent
page numbering is to restart the numbering in each child document
and denote the pages by `\textit{child}|.|\textit{page}'
where \textit{child} represents the chapter/section number of the child file.
This can be achieved by the command
|\numberwithin{page}{|\textit{child}|}|
of the \textsf{amsmath} package
where \textit{child} can be |chapter| or |section|
depending on the chosen structuring.
Alternatively, one can modify the macro |\thepage| appropriately
and reset the counter |page| at the start of each child file.

%%%%%%%%%%%%%%%%%%%%%%%%%%%%%%%%%%%%%%%%%%%%%%%%%%%%%%%%%%%%%%%%%%%%%%%%%%%%%%%%
\subsection{Conditional Processing}
\label{sec:conditional}

The package provides a mechanism to compile different versions
of a document. To customise the versions further some conditional processing
can come in handy to distinguish which version is being compiled.
The package provides two macros to describe the compilation context:

%%%%%%%%%%%%%%%%%%%%%%%%%%%%%%%%%%%%%%%%
\DescribeMacro{\ifchilddoc}
The conditional |\ifchilddoc| distinguishes between the compilation of
child documents and the main document:
%
\begin{center}
|\ifchilddoc |\textit{child-code}| |[|\||else |\textit{main-code}]| \||fi|
\end{center}

%%%%%%%%%%%%%%%%%%%%%%%%%%%%%%%%%%%%%%%%
\DescribeMacro{\childdocname}
\DescribeMacro{\childdocjob}
The macro |\childdocname| contains the filename (without extension)
of the main or child file being processed.
Note that |\childdocjob| will always contain the name of the main file.

%%%%%%%%%%%%%%%%%%%%%%%%%%%%%%%%%%%%%%%%
\paragraph{Title Page.}

Conditional processing can be used to include a title or banner page
in the main document when proper precautions are taken.
Importantly, the code in the main file should ensure that the page counter
(as well as other status parameters which are stored in the |.aux| files)
takes the same value after the conditional processing.
Otherwise the page numbers may take divergent values
depending on which part is compiled.

For example, a title page could be declared by:
%
\begin{center}
\begin{tabular}{l}
|\ifchilddoc\||else|\\
|\addtocounter{page}{-1}|\\
\textit{code for title page}\\
|\newpage|\\
|\||fi|
\end{tabular}
\end{center}
%
A banner page for the child documents can be generated by:
%
\begin{center}
\begin{tabular}{l}
|\ifchilddoc|\\
|\addtocounter{page}{-1}|\\
\textit{code for banner page}\\
|\newpage|\\
|\||fi|
\end{tabular}
\end{center}
%
Here one could write a message such as:
\begin{center}
|This is the part \childdocname{} of \childdocjob{}.|
\end{center}

%%%%%%%%%%%%%%%%%%%%%%%%%%%%%%%%%%%%%%%%%%%%%%%%%%%%%%%%%%%%%%%%%%%%%%%%%%%%%%%%
\subsection{Flags}
\label{sec:flags}

The package makes it easy to generate different versions
of the main or child documents.
To this end compilation flags can be defined
and assigned different default values.
They will be particularly useful in conjunction
with the forwarding mechanism described in \secref{sec:forward}.

For example, it may be useful to have a flag |\version|
which can be set to |draft| or |final|.
The document source will contain some conditional code
depending on the value of |\version|.
Suppose further, the flag should default to |final| for the main file
and to |draft| for child files
which is a natural assignment for editing the document.
This is achieved by placing the following code
in the preamble of the main document
(below the |\childdocmain| directive):
%
\begin{center}
\begin{tabular}{l}
|\ifchilddoc|\\
|\providecommand{\version}{draft}|\\
|\||else|\\
|\providecommand{\version}{final}|\\
|\||fi|
\end{tabular}
\end{center}
%
The definition by |\providecommand| makes sure
that previous definitions are not overwritten.
Further statements |\providecommand{\version}{...}|
can thus be added before the above code to override it.

For the main file, one might add a line
(between |\childdocmain| and the above block)
%
\begin{center}
|%\ifchilddoc\||else\providecommand{\version}{draft}\||fi|
\end{center}
%
which can be uncommented to produce a draft version.
Likewise one can add a line to the very top of a child file
(above the |\childdocof{|\textit{main}|}| directive)
%
\begin{center}
|%\providecommand{\version}{final}|
\end{center}
%
which can be uncommented to produce the final version of this child document.

%%%%%%%%%%%%%%%%%%%%%%%%%%%%%%%%%%%%%%%%%%%%%%%%%%%%%%%%%%%%%%%%%%%%%%%%%%%%%%%%
\subsection{Forwarding}
\label{sec:forward}

Different versions of the main or child documents
using compilation flags as described in \secref{sec:flags}
can be (permanently) stored in different files
for convenient compilation, viewing and distribution.
To this end, the package defines a command
to pass on compilation to a different file:

%%%%%%%%%%%%%%%%%%%%%%%%%%%%%%%%%%%%%%%%
\DescribeMacro{\childdocforward}
The command |\childdocforward| redirects processing to
another source file:
%
\begin{center}
\begin{tabular}{l}
|\input{childdoc.def}|\\
|\childdocforward[|\textit{main}|]{|\textit{dest}|}|\\
\end{tabular}
\end{center}
%
The argument \textit{dest} is the destination file
(without extension).
It should be the main file or one of the child files.
Note that further \textsf{childdoc} directives
such as |\childdocof| and |\childdocforward|
in the indicated file will be processed in this form.
The optional argument \textit{main}
passes on directly to the main file \textit{main}
while pretending to compile the child \textit{dest}.
This form behaves as if \textit{dest}
issues |\childdocof{|\textit{main}|}| right away,
and no further \textsf{childdoc} directives will be processed.

%%%%%%%%%%%%%%%%%%%%%%%%%%%%%%%%%%%%%%%%
\DescribeMacro{\...prefix}
In the alternative form |\childdocforwardprefix|,
%
\begin{center}
\begin{tabular}{l}
|\input{childdoc.def}|\\
|\childdocforwardprefix[|\textit{main}|]{|\textit{prefix}|}{|\textit{dest}|}|
\end{tabular}
\end{center}
%
the destination file is determined by a pattern
depending on the current file:
To make this work, the current file must be called
`{\textit{prefix}\hspace{0.2em}\textit{suffix}}'
with \textit{prefix} matching precisely the argument.
Processing is then passed on to the file
`{\textit{dest}\hspace{0.2em}\textit{suffix}}'.
Surely, the same effect is achieved by
directly specifying the
argument `{\textit{dest}\hspace{0.2em}\textit{suffix}}'
in the first form.
However, that requires to set up a different file
for each child. With the alternative form of the command
all these files can have exactly the same content
which simplifies setting them up and maintaining them.

For example, the following file |draft.tex|
with a compilation flag |\version| as described in \secref{sec:flags}
compiles the main document as a draft:
%
\begin{center}
\begin{tabular}{l}
|\def\version{draft}|\\
|\input{childdoc.def}|\\
|\childdocforward{|\textit{main}|}|
\end{tabular}
\end{center}
%
Likewise, the following files |final|\textit{nn}|.tex|
compile the final version of the child document
|child|\textit{nn}|.tex|:
%
\begin{center}
\begin{tabular}{l}
|\def\version{final}|\\
|\input{childdoc.def}|\\
|\childdocforwardprefix{final}{child}|
\end{tabular}
\end{center}
%

Note that when several versions of a main file and/or of each child file
are to be generated, it may be convenient to set up a |Makefile| or
shell script to automatise the process.

%%%%%%%%%%%%%%%%%%%%%%%%%%%%%%%%%%%%%%%%%%%%%%%%%%%%%%%%%%%%%%%%%%%%%%%%%%%%%%%%
\subsection{Command Line Processing}
\label{sec:commandline}

The effect of redirection files can also be achieved by invoking
the \LaTeX{} compiler with a more elaborate command line.
Most conveniently this should be done as part
of a shell script or a |Makefile|.

When using \textsf{childdoc} in the main file, the following
command lines effectively perform a redirection
(note that depending on the shell being used,
backslashes may have to be doubled: `|\|' $\to$ `|\\|'):
%
\begin{center}
|... -jobname "|\textit{target}|" |\\|"|[\textit{flags}]%
|\input{childdoc.def}\childdocforward[|\textit{main}|]{|\textit{dest}|}"|
\end{center}
%
Here \textit{target} is the name of the output file,
\textit{main} is the name of the main file
and \textit{dest} is the name of the main or child file to be processed
(all filenames without extensions).
The optional argument \textit{main} can be omitted
if \textit{main} matches \textit{dest}.
Optionally, compilation \textit{flags} can be defined via |\def| commands.
This command line makes the \TeX{} engine believe
it is compiling the file \textit{target}
whose content is specified as the latter parameter.
The provided code then forwards the processing to
\textit{main} or \textit{dest} as described in \secref{sec:forward}.

%%%%%%%%%%%%%%%%%%%%%%%%%%%%%%%%%%%%%%%%%%%%%%%%%%%%%%%%%%%%%%%%%%%%%%%%%%%%%%%%
\subsection{Include by Input}
\label{sec:input}

Including child documents by |\include| has some restrictions by design.
Most notably, the content of a child document always occupies
its own set of pages; pages cannot be shared between child documents.
Usually, this behaviour makes perfect sense
because each child document contain an essential part of the document.
However, in some situations it may be desirable to compose
a document from a collection of parts
without having mandatory page breaks between then.
For this case, the package
provides a mechanism to include parts
by |\input| which can also be processed individually.
However, by construction this mechanism
requires manual handling of the content to be output.

%%%%%%%%%%%%%%%%%%%%%%%%%%%%%%%%%%%%%%%%
\DescribeMacro{\ifchilddocmanual}
The main file should be prepared as usual, see \secref{sec:include}.
However, the document body must make a distinction
between processing of an individual part and of the main document, e.g.:
%
\begin{center}
\begin{tabular}{l}
|\ifchilddocmanual|\\
|\input{\childdocname}|\\
|\||else|\\
\textit{document body with }|\input{|\textit{part}|}|\\
|\||fi|
\end{tabular}
\end{center}
%
The conditional |\ifchilddocmanual| is true whenever
a part to be included by |\input| is being compiled,
and the name of the part is stored in |\childdocname|.

%%%%%%%%%%%%%%%%%%%%%%%%%%%%%%%%%%%%%%%%
\DescribeMacro{\childdocby}
Each part to be included by |\input| should start with:
%
\begin{center}
\begin{tabular}{l}
|\input{childdoc.def}|\\
|\childdocby{|\textit{main}|}|\\
\end{tabular}
\end{center}
%
The directive |\childdocby| is similar to |\childdocof|
described in \secref{sec:include},
but the subsequent selection of content must be done manually.
To that end, both |\ifchilddoc| and |\ifchilddocmanual|
will be true upon processing of a part,
and the name of the part is stored in |\childdocname|.
Note that |\jobname| will be set to the filename of the current part
so that each part receives an individual |.aux| file
that does not interfere with the |.aux| file(s) of the main document.
This behaviour can be altered by the alternative form
|\childdocby[*]{|\textit{main}|}| (with a non-empty optional argument)
which uses the |.aux| file of the main document
by setting |\jobname| to \textit{main}.

%%%%%%%%%%%%%%%%%%%%%%%%%%%%%%%%%%%%%%%%%%%%%%%%%%%%%%%%%%%%%%%%%%%%%%%%%%%%%%%%
\subsection{Driver Development}
\label{sec:driver}

The \textsf{childdoc} mechanism can also be use for the development
of definition files such as \LaTeX{} styles or classes.
This case differs from the above setup with multiple parts
included by |\include| in that no |\includeonly| should be invoked.
This can be achieved by starting the include file
(before |\ProvidesPackage|) with:
%
\begin{center}
\begin{tabular}{l}
|\input{childdoc.def}|\\
|\childdocforward{|\textit{main}|}|\\
\end{tabular}
\end{center}
%
or alternatively with:
%
\begin{center}
\begin{tabular}{l}
|\input{childdoc.def}|\\
|\childdocby{|\textit{main}|}|\\
\end{tabular}
\end{center}
%
Both forms have slightly different effects as described above.
The main file is prepared as usual, see \secref{sec:include}.

%%%%%%%%%%%%%%%%%%%%%%%%%%%%%%%%%%%%%%%%%%%%%%%%%%%%%%%%%%%%%%%%%%%%%%%%%%%%%%%%
\subsection{Legacy Detection}
\label{sec:detection}

The directive |\childdocmain| in the main file can detect
whether the complete document or merely a child is to be compiled
even without using the directive |\childdocof|.
This method is deprecated because it is less robust
and there is no compelling reason to use it;
it is merely provided for backward compatibility
and it may be removed in future versions.

If the detection mechanism is to be used,
it is mandatory to correctly specify
the filename of the main file as the argument of |\childdocmain|:
%
\begin{center}
\begin{tabular}{l}
|\input{childdoc.def}|\\
|\childdocmain{|\textit{main}|}|\\
\end{tabular}
\end{center}
%
If |\jobname| does not match the argument \textit{main} of |\childdocmain|,
it is assumed that |\jobname| points to the child file to be compiled.
When using |\childdocmain| with the main file specified as argument,
it suffices to start a child file
with just |\input{|\textit{main}|}|
without loading of the package and using |\childdocof|.
If instead all processing is done
with the appropriate \textsf{childdoc} directives,
the argument of \textit{main} of |\childdocmain| can be empty.

An alternative version of the command line processing described
in \secref{sec:commandline} using the detection mechanism reads:
%
\begin{center}
|... -jobname "|\textit{target}|" "|[\textit{flags}]%
[|\def\jobname{|\textit{dest}|}|]|\input{|\textit{main}|}"|
\end{center}

%%%%%%%%%%%%%%%%%%%%%%%%%%%%%%%%%%%%%%%%%%%%%%%%%%%%%%%%%%%%%%%%%%%%%%%%%%%%%%%%
\subsection{Manual Code}
\label{sec:manual}

In case one cannot be certain whether the definitions file |childdoc.def|
is installed on the target \TeX{} distribution
and one prefers not to ship it,
it is conceivable to paste a few relevant commands into the sources.

To that end, drop all statements |\input{childdoc.def}|
and perform the replacements as outlined below.
Instead of |\childdocmain{|\textit{main}|}| add the following code
to the top of the main file:
%
\begin{center}
\begin{tabular}{l}
|\||ifdefined\childdocname\endinput\||fi\newif\ifchilddoc|\\
|\edef\childdocname{\scantokens\expandafter{\jobname\noexpand}}|\\
|\def\childdocmain{|\textit{main}|}\||ifx\childdocmain\childdocname\||else|\\
|\childdoctrue\includeonly{\childdocname}\let\jobname\childdocmain\||fi|\\
\end{tabular}
\end{center}
%
Instead of |\childdocof{|\textit{main}|}| just include the main file
at the top of each child file:
%
\begin{center}
|\input{|\textit{main}|}|
\end{center}
%
A simple redirection |\childdocforward{|\textit{dest}|}| is achieved by:
%
\begin{center}
|\def\jobname{|\textit{dest}|}\input{\jobname}|
\end{center}
%
The redirection with prefix
|\childdocforwardprefix[|\textit{prefix}|]{|\textit{dest}|}|
is accomplished by:
%
\begin{center}
\begin{tabular}{l}
|{\edef\jobname{\scantokens\expandafter{\jobname\noexpand}}|\\
|\def\redirectjob |\textit{prefix}|#1~~~{\gdef\jobname{|\textit{dest}|#1}}|\\
|\expandafter\redirectjob\jobname~~~}\input{\jobname}|
\end{tabular}
\end{center}

In an alternative approach,
child documents can be compiled by a specific command line
without additional code or specific definitions:
%
\begin{center}
|... -jobname "|\textit{target}|" "|[\textit{flags}]%
|\includeonly{|\textit{dest}|}\input{|\textit{main}|}"|
\end{center}
%

%%%%%%%%%%%%%%%%%%%%%%%%%%%%%%%%%%%%%%%%%%%%%%%%%%%%%%%%%%%%%%%%%%%%%%%%%%%%%%%%
%%%%%%%%%%%%%%%%%%%%%%%%%%%%%%%%%%%%%%%%%%%%%%%%%%%%%%%%%%%%%%%%%%%%%%%%%%%%%%%%
\section{Information}

%%%%%%%%%%%%%%%%%%%%%%%%%%%%%%%%%%%%%%%%%%%%%%%%%%%%%%%%%%%%%%%%%%%%%%%%%%%%%%%%
\subsection{Copyright}

Copyright \copyright{} 2017--2018 Niklas Beisert

This work may be distributed and/or modified under the
conditions of the \LaTeX{} Project Public License, either version 1.3
of this license or (at your option) any later version.
The latest version of this license is in
  \url{http://www.latex-project.org/lppl.txt}
and version 1.3 or later is part of all distributions of \LaTeX{}
version 2005/12/01 or later.

This work has the LPPL maintenance status `maintained'.

The Current Maintainer of this work is Niklas Beisert.

This work consists of the files |README.txt|, |childdoc.ins| and |childdoc.dtx|
as well as the derived files |childdoc.def|, |cdocsamp.tex|
with |cdocsch1.tex|, |cdocsch2.tex|, |cdocspt3.tex|, |cdocspt4.tex|,
|cdocsdrf.tex|, |cdocsfn1.tex|, |cdocsfn2.tex|
as well as |childdoc.pdf|.

%%%%%%%%%%%%%%%%%%%%%%%%%%%%%%%%%%%%%%%%%%%%%%%%%%%%%%%%%%%%%%%%%%%%%%%%%%%%%%%%
\subsection{Files and Installation}

The package consists of the files:
%
\begin{center}
\begin{tabular}{ll}
    |README.txt|   & readme file \\
    |childdoc.ins| & installation file \\
    |childdoc.dtx| & source file \\
    |childdoc.def| & definition file \\
    |cdocsamp.tex| & sample main file \\
    |cdocsch1.tex| & sample include file \\
    |cdocsch2.tex| & sample include file \\
    |cdocspt3.tex| & sample part file \\
    |cdocspt4.tex| & sample part file \\
    |cdocsdrf.tex| & sample redirection file \\
    |cdocsfn1.tex| & sample redirection file \\
    |cdocsfn2.tex| & sample redirection file \\
    |childdoc.pdf| & manual
\end{tabular}
\end{center}
%
The distribution consists of the files
|README.txt|, |childdoc.ins| and |childdoc.dtx|.
%
\begin{itemize}
\item
Run (pdf)\LaTeX{} on |childdoc.dtx|
to compile the manual |childdoc.pdf| (this file).
\item
Run \LaTeX{} on |childdoc.ins| to create the definitions file |childdoc.def|
and the sample |cdocsamp.tex| with include files
|cdocsch1.tex|, |cdocsch2.tex|, |cdocspt3.tex|, |cdocspt4.tex|,
|cdocsdrf.tex|, |cdocsfn1.tex|, |cdocsfn2.tex|.
Then copy the file |childdoc.def| to an appropriate directory of your \LaTeX{}
distribution, e.g.\ \textit{texmf-root}|/tex/latex/childdoc|.
\end{itemize}

%%%%%%%%%%%%%%%%%%%%%%%%%%%%%%%%%%%%%%%%%%%%%%%%%%%%%%%%%%%%%%%%%%%%%%%%%%%%%%%%
\subsection{Related CTAN Packages}

There are several other packages which offer a similar functionality:
%
\begin{itemize}
\item
The packages
\href{http://ctan.org/pkg/docmute}{\textsf{docmute}},
\href{http://ctan.org/pkg/includex}{\textsf{includex}} and
\href{http://ctan.org/pkg/standalone}{\textsf{standalone}}
provide commands to include only the document body of
a child file thus allowing both files to be compiled individually.
\item
The packages \href{http://ctan.org/pkg/subdocs}{\textsf{subdocs}}
and \href{http://ctan.org/pkg/subfiles}{\textsf{subfiles}}
provide structures in which the main and child documents can be
encapsulated and allowing them to be compiled individually.
The inclusion mechanism is different from the conventional |\include|.
\item
The package \href{http://ctan.org/pkg/combine}{\textsf{combine}}
is an elaborate solution to combine several documents into one.
\end{itemize}
%
See also the CTAN topic \href{http://ctan.org/topic/subdocs}{\textsf{subdocs}}
for further related packages.
The present package differs from the above solutions in that
a document structure constructed with the conventional |\include| mechanism
just needs two extra commands at the top of every file
such that all constituent files can be compiled individually.

%%%%%%%%%%%%%%%%%%%%%%%%%%%%%%%%%%%%%%%%%%%%%%%%%%%%%%%%%%%%%%%%%%%%%%%%%%%%%%%%
%\subsection{Feature Suggestions}
%
%The following is a list of features which may be useful for future
%versions of this package:
%%
%\begin{itemize}
%\item
%\ldots
%\end{itemize}

%%%%%%%%%%%%%%%%%%%%%%%%%%%%%%%%%%%%%%%%%%%%%%%%%%%%%%%%%%%%%%%%%%%%%%%%%%%%%%%%
\subsection{Revision History}

%%%%%%%%%%%%%%%%%%%%%%%%%%%%%%%%%%%%%%%%
\paragraph{v2.0:} 2018/12/30

\begin{itemize}
\item
immediate forward processing
\item
added |\childdocby| mechanism
\item
manual restructured
\end{itemize}

%%%%%%%%%%%%%%%%%%%%%%%%%%%%%%%%%%%%%%%%
\paragraph{v1.6:} 2018/01/17

\begin{itemize}
\item
application for development of include files
\item
corrections to manual
\end{itemize}

%%%%%%%%%%%%%%%%%%%%%%%%%%%%%%%%%%%%%%%%
\paragraph{v1.5:} 2017/05/21

\begin{itemize}
\item
more complete structuring introduced
\item
|\childdocof| introduced
\item
|\childdoc| renamed to |\childdocmain|
\item
|\childredirect| renamed to |\childdocforward| and |\childdocforwardprefix|
and functionality expanded
\end{itemize}

%%%%%%%%%%%%%%%%%%%%%%%%%%%%%%%%%%%%%%%%
\paragraph{v1.0:} 2017/04/27

\begin{itemize}
\item
manual and install package
\item
first version published on CTAN
\end{itemize}

%%%%%%%%%%%%%%%%%%%%%%%%%%%%%%%%%%%%%%%%
\paragraph{v0.6:} 2017/04/26

\begin{itemize}
\item
redirection mechanism added
\end{itemize}

%%%%%%%%%%%%%%%%%%%%%%%%%%%%%%%%%%%%%%%%
\paragraph{v0.5:} 2017/04/26

\begin{itemize}
\item
functionality in definition file
\end{itemize}


%%%%%%%%%%%%%%%%%%%%%%%%%%%%%%%%%%%%%%%%%%%%%%%%%%%%%%%%%%%%%%%%%%%%%%%%%%%%%%%%
%%%%%%%%%%%%%%%%%%%%%%%%%%%%%%%%%%%%%%%%%%%%%%%%%%%%%%%%%%%%%%%%%%%%%%%%%%%%%%%%
%%%%%%%%%%%%%%%%%%%%%%%%%%%%%%%%%%%%%%%%%%%%%%%%%%%%%%%%%%%%%%%%%%%%%%%%%%%%%%%%
\appendix

\settowidth\MacroIndent{\rmfamily\scriptsize 000\ }

 \DocInput{childdoc.dtx}

\end{document}
%</driver>
% \fi
%
% %%%%%%%%%%%%%%%%%%%%%%%%%%%%%%%%%%%%%%%%%%%%%%%%%%%%%%%%%%%%%%%%%%%%%%%%%%%%%%
% %%%%%%%%%%%%%%%%%%%%%%%%%%%%%%%%%%%%%%%%%%%%%%%%%%%%%%%%%%%%%%%%%%%%%%%%%%%%%%
% \section{Sample}
%\iffalse
%<*samplemain>
%\fi
%
% The following presents a sample document
% with two chapters, two parts, a title page,
% a compile flag as well as three forwarding files to set the flag.
% It consists of eight |.tex| files:
% \begin{center}
% \begin{tabular}{ll}
% |cdocsamp.tex|&main file\\
% |cdocsch1.tex|&include file for chapter 1\\
% |cdocsch2.tex|&include file for chapter 2\\
% |cdocspt3.tex|&include file for part 3\\
% |cdocspt4.tex|&include file for part 4\\
% |cdocsdrf.tex|&forwarding file for main file in draft mode\\
% |cdocsfi1.tex|&forwarding file for final version of chapter 1\\
% |cdocsfi2.tex|&forwarding file for final version of chapter 2\\
% \end{tabular}
% \end{center}
% Each of the eight files can be compiled directly by the \LaTeX{} compiler.
%
% %%%%%%%%%%%%%%%%%%%%%%%%%%%%%%%%%%%%%%
% \paragraph{Main File.}
%
% The main file is called |cdocsamp.tex|.
%
% Load the \textsf{childdoc} definitions and
% declare the filename for the main document:
%    \begin{macrocode}
\input{childdoc.def}
\childdocmain{}
%    \end{macrocode}

% Optional override for |\version| flag:
%    \begin{macrocode}
%%\ifchilddoc\else\providecommand{\version}{draft}\fi
%    \end{macrocode}

% Define the default values for the |\version| flag
% (|final| for the main file and |draft| for childs):
%    \begin{macrocode}
\ifchilddoc
\providecommand{\version}{draft}
\else
\providecommand{\version}{final}
\fi
%    \end{macrocode}

% Load the standard document class:
%    \begin{macrocode}
\documentclass[12pt]{article}
%    \end{macrocode}

% Start the document body:
%    \begin{macrocode}
\begin{document}
%    \end{macrocode}

% Declare a title page.
% Print title, part of document being processed and version flag:
%    \begin{macrocode}
\addtocounter{page}{-1}
\begin{center}
{\LARGE\bfseries{}childdoc example\par}
\vspace{1cm}
\ifchilddoc
\ifchilddocmanual part\else chapter\fi:
`\childdocname' of `\childdocjob'\par
\else
main document: `\childdocjob'\par
\fi
version: \version\par
\end{center}
\newpage
%    \end{macrocode}

% Manually include selected file,
% otherwise process as usual:
%    \begin{macrocode}
\ifchilddocmanual
\section*{part `\childdocname'}
\input{\childdocname}
\else
%    \end{macrocode}

% Include the two chapters:
%    \begin{macrocode}
\include{cdocsch1}
\include{cdocsch2}
%    \end{macrocode}

% Include the two parts unless only chapters should be displayed:
%    \begin{macrocode}
\ifchilddoc\else
\section{part three}
\input{cdocspt3}
\section{part four}
\input{cdocspt4}
\fi
%    \end{macrocode}

% Process as usual until here:
%    \begin{macrocode}
\fi
%    \end{macrocode}

% End of document body:
%    \begin{macrocode}
\end{document}
%    \end{macrocode}
%\iffalse
%</samplemain>
%\fi
%
% %%%%%%%%%%%%%%%%%%%%%%%%%%%%%%%%%%%%%%
% \paragraph{Chapter Include Files.}
%
% The include files are called |cdocsch1.tex| and |cdocsch2.tex|.
%
%\iffalse
%<*samplechap1|samplechap2>
%\fi

% Optional override for |\version| flag:
%    \begin{macrocode}
%%\providecommand{\version}{final}
%    \end{macrocode}

% Include the main document:
%    \begin{macrocode}
\input{childdoc.def}
\childdocof{cdocsamp}
%    \end{macrocode}

%\iffalse
%</samplechap1|samplechap2>
%\fi
%
%\iffalse
%<*samplechap1>
%\fi
% Some text for chapter 1:
%    \begin{macrocode}
\section{one}
some text in chapter one
%    \end{macrocode}

%\iffalse
%</samplechap1>
%\fi
% Some text for chapter 2:
%\iffalse
%<*samplechap2>
%\fi
%    \begin{macrocode}
\section{two}
more text in chapter two
%    \end{macrocode}

%\iffalse
%</samplechap2>
%\fi
%
% %%%%%%%%%%%%%%%%%%%%%%%%%%%%%%%%%%%%%%
% \paragraph{Part Include Files.}
%
% The include files are called |cdocspt3.tex| and |cdocspt4.tex|.
%
%\iffalse
%<*samplepart3|samplepart4>
%\fi

% Optional override for |\version| flag:
%    \begin{macrocode}
%%\providecommand{\version}{final}
%    \end{macrocode}

% Include the main document:
%    \begin{macrocode}
\input{childdoc.def}
\childdocby{cdocsamp}
%    \end{macrocode}

%\iffalse
%</samplepart3|samplepart4>
%\fi
%
%\iffalse
%<*samplepart3>
%\fi
% Some text for part 3:
%    \begin{macrocode}
some text in part three
%    \end{macrocode}

%\iffalse
%</samplepart3>
%\fi
% Some text for part 4:
%\iffalse
%<*samplepart4>
%\fi
%    \begin{macrocode}
more text in part four
%    \end{macrocode}

%\iffalse
%</samplepart4>
%\fi
%
% %%%%%%%%%%%%%%%%%%%%%%%%%%%%%%%%%%%%%%
% \paragraph{Forwarding for a Complete Draft.}
%
% The following forwarding file |cdocsdrf.tex|
% compiles the main document in draft mode:
%\iffalse
%<*sampledraft>
%\fi
%    \begin{macrocode}
\def\version{draft}
\input{childdoc.def}
\childdocforward{cdocsamp}
%    \end{macrocode}

%\iffalse
%</sampledraft>
%\fi
%
% %%%%%%%%%%%%%%%%%%%%%%%%%%%%%%%%%%%%%%
% \paragraph{Forwarding for Final Version of the Chapters.}
%
% The following forwarding files |cdocsfn1.tex| and |cdocsfn2.tex|
% (with identical content)
% compile the final versions of the child documents
% |cdocsch1.tex| and |cdocsch2.tex|, respectively:
%\iffalse
%<*samplefinal>
%\fi
%    \begin{macrocode}
\def\version{final}
\input{childdoc.def}
\childdocforwardprefix[cdocsamp]{cdocsfn}{cdocsch}
%    \end{macrocode}

%\iffalse
%</samplefinal>
%\fi
%
% %%%%%%%%%%%%%%%%%%%%%%%%%%%%%%%%%%%%%%
% \paragraph{Command Line Processing.}
%
% The following three command lines generate the output files
% |cdocscld|, |cdocscl1| and |cdocscl2|
% which should be identical to
% |cdocsdrf|, |cdocsch1| and |cdocsfn2|, respectively:
% \begin{center}
% \begin{tabular}{l}
% |latex -jobname cdocscld \|\\
% |  "\def\version{draft}\input{childdoc.def}\childdocforward{cdocsamp}"|\\
% |latex -jobname cdocscl1 \|\\
% |  "\input{childdoc.def}\childdocforward[cdocsamp]{cdocsch1}"|\\
% |latex -jobname cdocscl2 \|\\
% |  "\def\version{final}\input{childdoc.def}\childdocforward{cdocsch2}"|
% \end{tabular}
% \end{center}
% Note that the trailing backslash on each first line
% merely continues the input to the second line
% (for convenient cut ant paste).
% Furthermore, the command |latex| can be replaced by any
% of its alternative versions such as |pdflatex|.
%
% %%%%%%%%%%%%%%%%%%%%%%%%%%%%%%%%%%%%%%%%%%%%%%%%%%%%%%%%%%%%%%%%%%%%%%%%%%%%%%
% %%%%%%%%%%%%%%%%%%%%%%%%%%%%%%%%%%%%%%%%%%%%%%%%%%%%%%%%%%%%%%%%%%%%%%%%%%%%%%
% \section{Implementation}
%\iffalse
%<*package>
%\fi
%
% This section describes the definitions file |childdoc.def|.

% The definitions cannot be loaded using |\usepackage| or |\RequirePackage|
% which has a mechanism to prevent loading a style file more than once.
% When loading the definitions by means of |\input|
% multiple instances have to be prevented manually:
%\iffalse
%This code needs to be before the `\ProvidesFile' directive
%which is defined at the beginning of this file.
%Therefore it is also placed there and commented out here.
%</package>
%<*discard>
%\fi
%    \begin{macrocode}
\ifdefined\childdocmain\endinput\fi
%    \end{macrocode}
%\iffalse
%</discard>
%<*package>
%\fi
%
% \macro{\ifchilddoc}
% \macro{\ifchilddocmanual}
% The conditional |\ifchilddoc| tells whether a
% child (true) or main (false) document is being compiled.
% The conditional |\ifchilddocmanual| tells whether
% the |\includeonly| mechanism is used (false) or
% the selection of child files must be performed manually (true).
% The definitions initialise to false:
%    \begin{macrocode}
\newif\ifchilddoc
\newif\ifchilddocmanual
%    \end{macrocode}

% \macro{\childdocname}
% \macro{\childdocjob}
% The macro |\childdocname| stores the name of the main document
% to be compiled. The macro |\childdocjob| stores the name of
% the document on which the \LaTeX{} compiler was originally invoked.
% The content of |\jobname| cannot be compared
% to filenames specified in the source due to different catcodes.
% The following code rescans |\jobname|, stores the result
% in |\childdocname| and saves a copy in |\childdocjob|:
%    \begin{macrocode}
\edef\childdocname{\scantokens\expandafter{\jobname\noexpand}}
\let\childdocjob\childdocname
%    \end{macrocode}

% \macro{\childdocdisable}
% The macro |\childdocdisable| prevents the main file
% from being processed more than once.
% At this stage, the main document command |\childdocmain|
% is assumed to be called once again where it should do nothing.
% Any subsequent call to it should prevent
% a secondary processing of the main document
% It overwrites the forwarding commands
% |\childdocof| and |\childdocforward|
% with empty macros to prevent further inclusions of the main document:
%    \begin{macrocode}
\newcommand{\childdocdisable}
{
  \renewcommand{\childdocmain}[1]{\renewcommand{\childdocmain}[1]{\endinput}}
  \renewcommand{\childdocof}[1]{}
  \renewcommand{\childdocby}[2][]{}
  \renewcommand{\childdocforward}[2][]{}
  \renewcommand{\childdocdisable}{}
}
%    \end{macrocode}

% \macro{\childdocmain}
% The macro |\childdocmain| is to be called at the top of the main file
% with nothing or the main filename (without extension) as argument.
% First, it breaks loops.
% If the argument is not empty and does not match |\childdocname|
% (which is set by the first inclusion of |childdoc.def|),
% |\ifchilddoc| is set to true, |\includeonly| is applied to the child file
% and |\jobname| is set to the main file
% (for proper handling of |.aux| files):
%    \begin{macrocode}
\newcommand{\childdocmain}[1]
{
  \childdocdisable\childdocmain{}
  \if?#1?\else
    \begingroup
      \def\childdoctmp{#1}
      \ifx\childdoctmp\childdocname
        \def\childdoctmp{}
      \else
        \def\childdoctmp
        {
          \childdoctrue
          \includeonly{\childdocname}
          \def\childdocjob{#1}
          \def\jobname{#1}
        }
      \fi
      \expandafter
    \endgroup
    \childdoctmp
  \fi
}
%    \end{macrocode}

% \macro{\childdocof}
% The command |\childdocof| redirects
% compilation to the main file |#1|.
%    \begin{macrocode}
\newcommand{\childdocof}[1]
{
  \childdocdisable
  \childdoctrue
  \includeonly{\childdocname}
  \def\jobname{#1}
  \def\childdocjob{#1}
  \input{#1}
}
%    \end{macrocode}

% \macro{\childdocby}
% The command |\childdocby| ....
%    \begin{macrocode}
\newcommand{\childdocby}[2][]
{
  \childdocdisable
  \childdoctrue
  \childdocmanualtrue
  \if?#1?\else
    \def\jobname{#2}
  \fi
  \def\childdocjob{#2}
  \input{#2}
  \endinput
}
%    \end{macrocode}

% \macro{\childdocforward}
% The command |\childdocforward| redirects
% compilation to the main file or
% (if the optional argument is given) a child file.
% Parameters are set as if the main file
% or a child file starting with |\childdocof| was compiled.
% Then compilation is handed over to the main file:
%    \begin{macrocode}
\newcommand{\childdocforward}[2][]
{
  \begingroup
    \if?#1?
      \def\childdoctmp
      {
        \def\childdocname{#2}
        \def\childdocjob{#2}
        \def\jobname{#2}
        \input{#2}
        \endinput
      }
    \else
      \def\childdoctmp
      {
        \childdocdisable
        \def\childdocname{#2}
        \childdoctrue
        \includeonly{#2}
        \def\childdocjob{#1}
        \def\jobname{#1}
        \input{#1}
        \endinput
      }
    \fi
    \expandafter
  \endgroup
  \childdoctmp
}
%    \end{macrocode}

% \macro{\childdocforwardprefix}
% The command |\childdocforwardprefix| redirects
% compilation to the main or a child file by means of a pattern.
% The prefix |#1| in the current filename is replaced by |#2|
% and the suffix of the current filename is kept
% (it is assumed that the filename does not contain the substring `|~~~|'
% which is used as a delimiter).
% Compilation is handed over to the new file by |\childdocforward|:
%    \begin{macrocode}
\newcommand{\childdocforwardprefix}[3][]
{
  \begingroup
    \def\childdocextract #2##1~~~{\def\childdoctmp{\childdocforward[#1]{#3##1}}}
    \expandafter\childdocextract\childdocname~~~
    \expandafter
  \endgroup
  \childdoctmp
}
%    \end{macrocode}

% \macro{\childdoc}
% The deprecated macro |\childdoc| is a legacy version of |\childdocmain|:
%    \begin{macrocode}
\newcommand{\childdoc}{\childdocmain}
%    \end{macrocode}

% \macro{\childdocredirect}
% The deprecated macro |\childdocredirect| is a legacy version
% of |\childdocforward| and |\childdocforwardprefix|:
%    \begin{macrocode}
\newcommand{\childdocredirect}[2][]
{
  \begingroup
    \if?#1?
      \def\childdoctmp{\childdocforward{#2}}
    \else
      \def\childdoctmp{\childdocforwardprefix{#1}{#2}}
    \fi
    \expandafter
  \endgroup
  \childdoctmp
}
%    \end{macrocode}

%\iffalse
%</package>
%\fi
%
\endinput

\childdocof{cdocsamp}
%    \end{macrocode}

%\iffalse
%</samplechap1|samplechap2>
%\fi
%
%\iffalse
%<*samplechap1>
%\fi
% Some text for chapter 1:
%    \begin{macrocode}
\section{one}
some text in chapter one
%    \end{macrocode}

%\iffalse
%</samplechap1>
%\fi
% Some text for chapter 2:
%\iffalse
%<*samplechap2>
%\fi
%    \begin{macrocode}
\section{two}
more text in chapter two
%    \end{macrocode}

%\iffalse
%</samplechap2>
%\fi
%
% %%%%%%%%%%%%%%%%%%%%%%%%%%%%%%%%%%%%%%
% \paragraph{Part Include Files.}
%
% The include files are called |cdocspt3.tex| and |cdocspt4.tex|.
%
%\iffalse
%<*samplepart3|samplepart4>
%\fi

% Optional override for |\version| flag:
%    \begin{macrocode}
%%\providecommand{\version}{final}
%    \end{macrocode}

% Include the main document:
%    \begin{macrocode}
% \iffalse
%
% childdoc.dtx Copyright (C) 2017-2018 Niklas Beisert
%
% This work may be distributed and/or modified under the
% conditions of the LaTeX Project Public License, either version 1.3
% of this license or (at your option) any later version.
% The latest version of this license is in
%   http://www.latex-project.org/lppl.txt
% and version 1.3 or later is part of all distributions of LaTeX
% version 2005/12/01 or later.
%
% This work has the LPPL maintenance status `maintained'.
%
% The Current Maintainer of this work is Niklas Beisert.
%
% This work consists of the files childdoc.dtx and childdoc.ins
% and the derived files childdoc.def and cdocsamp.tex with
% cdocsch1.tex, cdocsch2.tex, cdocsdrf.tex, cdocsfn1.tex, cdocsfn2.tex.
%
%<package>\ifdefined\childdocmain\endinput\fi
%<package>\ProvidesFile{childdoc.def}[2018/12/30 v2.0 child document driver]
%<samplemain>\ProvidesFile{cdocsamp.tex}[2018/12/30 v2.0 sample for childdoc]
%<*driver>
%\ProvidesFile{childdoc.drv}[2018/12/30 v2.0 childdoc reference manual file]
\PassOptionsToClass{10pt,a4paper}{article}
\documentclass{ltxdoc}

\usepackage[margin=35mm]{geometry}
\usepackage{hyperref}
\usepackage{hyperxmp}
\usepackage[usenames]{color}

\hypersetup{colorlinks=true}
\hypersetup{pdfstartview=FitH}
\hypersetup{pdfpagemode=UseNone}
\hypersetup{pdfsource={}}
\hypersetup{pdflang={en-UK}}
\hypersetup{pdfcopyright={Copyright 2017-2018 Niklas Beisert.
  This work may be distributed and/or modified under the
  conditions of the LaTeX Project Public License, either version 1.3
  of this license or (at your option) any later version.}}
\hypersetup{pdflicenseurl={http://www.latex-project.org/lppl.txt}}
\hypersetup{pdfcontactaddress={ETH Zurich, ITP, HIT K,
  Wolfgang-Pauli-Strasse 27}}
\hypersetup{pdfcontactpostcode={8093}}
\hypersetup{pdfcontactcity={Zurich}}
\hypersetup{pdfcontactcountry={Switzerland}}
\hypersetup{pdfcontactemail={nbeisert@itp.phys.ethz.ch}}
\hypersetup{pdfcontacturl={http://people.phys.ethz.ch/\xmptilde nbeisert/}}

\newcommand{\secref}[1]{\hyperref[#1]{section \ref*{#1}}}

\parskip1ex
\parindent0pt
\let\olditemize\itemize
\def\itemize{\olditemize\parskip0pt}

\begin{document}

\title{The \textsf{childdoc} Package}
\hypersetup{pdftitle={The childdoc Package}}
\author{Niklas Beisert\\[2ex]
  Institut f\"ur Theoretische Physik\\
  Eidgen\"ossische Technische Hochschule Z\"urich\\
  Wolfgang-Pauli-Strasse 27, 8093 Z\"urich, Switzerland\\[1ex]
  \href{mailto:nbeisert@itp.phys.ethz.ch}
  {\texttt{nbeisert@itp.phys.ethz.ch}}}
\hypersetup{pdfauthor={Niklas Beisert}}
\hypersetup{pdfsubject={Manual for the LaTeX2e Package childdoc}}
\date{30 December 2018, \textsf{v2.0}}
\maketitle

\begin{abstract}\noindent
\textsf{childdoc} is a \LaTeXe{} package
that enables the direct compilation
of document sections included by |\include|
to individual files.
\end{abstract}

\begingroup
\parskip0ex
\tableofcontents
\endgroup

%%%%%%%%%%%%%%%%%%%%%%%%%%%%%%%%%%%%%%%%%%%%%%%%%%%%%%%%%%%%%%%%%%%%%%%%%%%%%%%%
%%%%%%%%%%%%%%%%%%%%%%%%%%%%%%%%%%%%%%%%%%%%%%%%%%%%%%%%%%%%%%%%%%%%%%%%%%%%%%%%
\section{Introduction}

\LaTeX{} provides a mechanism to structure a large document (such as a book)
into a main file and several child files (containing the chapters)
using the |\include| command.
This mechanism is beneficial for documents
which span hundreds of pages in order to
make the source file(s) more manageable.
Moreover, compilation can be restricted to
selected child files by means of the |\includeonly| command.
The latter feature can be used to reduce the compilation time while editing
(this was significantly more useful in the earlier days of \LaTeX{})
or to generate a smaller document which is easier to navigate.
Another application of |\includeonly| is to generate
documents consisting of selected parts of the complete document.

However, there are a few drawbacks of the plain |\include| mechanism:
\begin{itemize}
\item
The child files cannot be compiled on their own,
they can only be compiled via the main file.
A naive editing environment
(such as a text editor with an option
to have the current file processed by \LaTeX)
may require one to switch to the main file before compiling;
attempting to compile the child file produces errors.
\item
The main file must be modified (each time)
to adjust the |\includeonly| command
to the present needs. This easily leaves the main file in a messy state.
\item
The generated document will always carry the filename
of the main document. This is inconvenient if
several child files are to be compiled and
to be kept for distribution.
\end{itemize}

The present package provides a simple interface
to make child files individually compilable by \LaTeX{}.
Compiling a child file then has the same effect as compiling
the main file with an |\includeonly| command
to select the appropriate child.
Moreover the generated document will carry the name of the child
rather than the main file.
This resolves all three above issues.

This feature is meant to make the editing of books,
thesis documents and lecture notes somewhat more convenient.
However, the package can also be used efficiently for
composing a series of documents (such as exercise sheets)
which are typically distributed individually.
It then assists the author in generating the individual documents
(potentially in different versions)
as well as a document containing the collected series.
Another application is in developing style files
or other kinds of included material
where compilation of the style file could redirect
to a sample or test file.

%%%%%%%%%%%%%%%%%%%%%%%%%%%%%%%%%%%%%%%%%%%%%%%%%%%%%%%%%%%%%%%%%%%%%%%%%%%%%%%%
%%%%%%%%%%%%%%%%%%%%%%%%%%%%%%%%%%%%%%%%%%%%%%%%%%%%%%%%%%%%%%%%%%%%%%%%%%%%%%%%
\section{Usage}

First of all, the package \textsf{childdoc} is \emph{not} a standard
\LaTeXe{} |.sty| style file! Therefore it needs to be invoked in
a non-standard way.

%%%%%%%%%%%%%%%%%%%%%%%%%%%%%%%%%%%%%%%%%%%%%%%%%%%%%%%%%%%%%%%%%%%%%%%%%%%%%%%%
\subsection{Included Files}
\label{sec:include}

%%%%%%%%%%%%%%%%%%%%%%%%%%%%%%%%%%%%%%%%
\DescribeMacro{\childdocmain}
To use the package, add the commands
\begin{center}
\begin{tabular}{l}
|\input{childdoc.def}|\\
|\childdocmain{}|\\
\end{tabular}
\end{center}
at the very top of the main \LaTeX{} file,
in particular \emph{before} the |\documentclass| statement!
The argument of |\childdocmain| should be left empty
(but it must be present).

%%%%%%%%%%%%%%%%%%%%%%%%%%%%%%%%%%%%%%%%
\DescribeMacro{\childdocof}
Furthermore, add the commands
\begin{center}
\begin{tabular}{l}
|\input{childdoc.def}|\\
|\childdocof{|\textit{main}|}|\\
\end{tabular}
\end{center}
at the top of every child file \textit{child}
which is included by |\include{|\textit{child}|}|
from within the main file
(or at least for those files to be compiled individually).
The argument \textit{main} must be the filename of the main file.

There are a couple of
considerations in setting up the main and child documents:

%%%%%%%%%%%%%%%%%%%%%%%%%%%%%%%%%%%%%%%%
\paragraph{Restrictions.}

Please note the following restrictions:
\begin{itemize}
\item
|\childdocmain| must be called with one argument \textit{main}
to ensure compatibility with earlier version of the package.
It must either be empty (|\childdocmain{}|)
or precisely match the filename of the main file in which it is specified.
See \secref{sec:detection} for further information.
\item
The filename \textit{main} must be specified without the |.tex| extension.
\item
The filename \textit{main} is case sensitive
(even in case-insensitive file systems)
due to internal string comparison.
\item
The argument \textit{main} should be fully expanded, it cannot be a macro.
\item
Subdirectories and special characters should be avoided in filenames.
\item
The command |\childdocmain{|\textit{main}|}| must be followed by a whitespace.
It should not be followed immediately by another command
or by a comment mark `|%|'.
This is because the \TeX{} parser reads the token immediately following
the argument of |\childdocmain| and puts it
at the beginning of every child section;
however, a white\-space is ignored.
\end{itemize}

%%%%%%%%%%%%%%%%%%%%%%%%%%%%%%%%%%%%%%%%
\paragraph{Content of Main File.}

It is advisable to place all content in the child files included by |\include|.
Any output contained in the main file will appear in all child documents
unless suppressed manually;
it cannot be suppressed automatically by the |\includeonly| directive
and thus should normally be avoided.
A method to include some content in the main file
by means of conditional processing is described in \secref{sec:conditional}.

%%%%%%%%%%%%%%%%%%%%%%%%%%%%%%%%%%%%%%%%
\paragraph{Page Numbering.}

When only a part of the document is compiled,
the appropriate numbering of pages
(as well as other status parameters)
is determined from the |.aux| files.
The latter contain information from previous passes.
However this information needs to propagate through
all intermediate child documents.
Therefore the page numbering in child documents may well
be inconsistent until the complete document is compiled at least once.

A useful (if unconventional) way to always ensure a consistent
page numbering is to restart the numbering in each child document
and denote the pages by `\textit{child}|.|\textit{page}'
where \textit{child} represents the chapter/section number of the child file.
This can be achieved by the command
|\numberwithin{page}{|\textit{child}|}|
of the \textsf{amsmath} package
where \textit{child} can be |chapter| or |section|
depending on the chosen structuring.
Alternatively, one can modify the macro |\thepage| appropriately
and reset the counter |page| at the start of each child file.

%%%%%%%%%%%%%%%%%%%%%%%%%%%%%%%%%%%%%%%%%%%%%%%%%%%%%%%%%%%%%%%%%%%%%%%%%%%%%%%%
\subsection{Conditional Processing}
\label{sec:conditional}

The package provides a mechanism to compile different versions
of a document. To customise the versions further some conditional processing
can come in handy to distinguish which version is being compiled.
The package provides two macros to describe the compilation context:

%%%%%%%%%%%%%%%%%%%%%%%%%%%%%%%%%%%%%%%%
\DescribeMacro{\ifchilddoc}
The conditional |\ifchilddoc| distinguishes between the compilation of
child documents and the main document:
%
\begin{center}
|\ifchilddoc |\textit{child-code}| |[|\||else |\textit{main-code}]| \||fi|
\end{center}

%%%%%%%%%%%%%%%%%%%%%%%%%%%%%%%%%%%%%%%%
\DescribeMacro{\childdocname}
\DescribeMacro{\childdocjob}
The macro |\childdocname| contains the filename (without extension)
of the main or child file being processed.
Note that |\childdocjob| will always contain the name of the main file.

%%%%%%%%%%%%%%%%%%%%%%%%%%%%%%%%%%%%%%%%
\paragraph{Title Page.}

Conditional processing can be used to include a title or banner page
in the main document when proper precautions are taken.
Importantly, the code in the main file should ensure that the page counter
(as well as other status parameters which are stored in the |.aux| files)
takes the same value after the conditional processing.
Otherwise the page numbers may take divergent values
depending on which part is compiled.

For example, a title page could be declared by:
%
\begin{center}
\begin{tabular}{l}
|\ifchilddoc\||else|\\
|\addtocounter{page}{-1}|\\
\textit{code for title page}\\
|\newpage|\\
|\||fi|
\end{tabular}
\end{center}
%
A banner page for the child documents can be generated by:
%
\begin{center}
\begin{tabular}{l}
|\ifchilddoc|\\
|\addtocounter{page}{-1}|\\
\textit{code for banner page}\\
|\newpage|\\
|\||fi|
\end{tabular}
\end{center}
%
Here one could write a message such as:
\begin{center}
|This is the part \childdocname{} of \childdocjob{}.|
\end{center}

%%%%%%%%%%%%%%%%%%%%%%%%%%%%%%%%%%%%%%%%%%%%%%%%%%%%%%%%%%%%%%%%%%%%%%%%%%%%%%%%
\subsection{Flags}
\label{sec:flags}

The package makes it easy to generate different versions
of the main or child documents.
To this end compilation flags can be defined
and assigned different default values.
They will be particularly useful in conjunction
with the forwarding mechanism described in \secref{sec:forward}.

For example, it may be useful to have a flag |\version|
which can be set to |draft| or |final|.
The document source will contain some conditional code
depending on the value of |\version|.
Suppose further, the flag should default to |final| for the main file
and to |draft| for child files
which is a natural assignment for editing the document.
This is achieved by placing the following code
in the preamble of the main document
(below the |\childdocmain| directive):
%
\begin{center}
\begin{tabular}{l}
|\ifchilddoc|\\
|\providecommand{\version}{draft}|\\
|\||else|\\
|\providecommand{\version}{final}|\\
|\||fi|
\end{tabular}
\end{center}
%
The definition by |\providecommand| makes sure
that previous definitions are not overwritten.
Further statements |\providecommand{\version}{...}|
can thus be added before the above code to override it.

For the main file, one might add a line
(between |\childdocmain| and the above block)
%
\begin{center}
|%\ifchilddoc\||else\providecommand{\version}{draft}\||fi|
\end{center}
%
which can be uncommented to produce a draft version.
Likewise one can add a line to the very top of a child file
(above the |\childdocof{|\textit{main}|}| directive)
%
\begin{center}
|%\providecommand{\version}{final}|
\end{center}
%
which can be uncommented to produce the final version of this child document.

%%%%%%%%%%%%%%%%%%%%%%%%%%%%%%%%%%%%%%%%%%%%%%%%%%%%%%%%%%%%%%%%%%%%%%%%%%%%%%%%
\subsection{Forwarding}
\label{sec:forward}

Different versions of the main or child documents
using compilation flags as described in \secref{sec:flags}
can be (permanently) stored in different files
for convenient compilation, viewing and distribution.
To this end, the package defines a command
to pass on compilation to a different file:

%%%%%%%%%%%%%%%%%%%%%%%%%%%%%%%%%%%%%%%%
\DescribeMacro{\childdocforward}
The command |\childdocforward| redirects processing to
another source file:
%
\begin{center}
\begin{tabular}{l}
|\input{childdoc.def}|\\
|\childdocforward[|\textit{main}|]{|\textit{dest}|}|\\
\end{tabular}
\end{center}
%
The argument \textit{dest} is the destination file
(without extension).
It should be the main file or one of the child files.
Note that further \textsf{childdoc} directives
such as |\childdocof| and |\childdocforward|
in the indicated file will be processed in this form.
The optional argument \textit{main}
passes on directly to the main file \textit{main}
while pretending to compile the child \textit{dest}.
This form behaves as if \textit{dest}
issues |\childdocof{|\textit{main}|}| right away,
and no further \textsf{childdoc} directives will be processed.

%%%%%%%%%%%%%%%%%%%%%%%%%%%%%%%%%%%%%%%%
\DescribeMacro{\...prefix}
In the alternative form |\childdocforwardprefix|,
%
\begin{center}
\begin{tabular}{l}
|\input{childdoc.def}|\\
|\childdocforwardprefix[|\textit{main}|]{|\textit{prefix}|}{|\textit{dest}|}|
\end{tabular}
\end{center}
%
the destination file is determined by a pattern
depending on the current file:
To make this work, the current file must be called
`{\textit{prefix}\hspace{0.2em}\textit{suffix}}'
with \textit{prefix} matching precisely the argument.
Processing is then passed on to the file
`{\textit{dest}\hspace{0.2em}\textit{suffix}}'.
Surely, the same effect is achieved by
directly specifying the
argument `{\textit{dest}\hspace{0.2em}\textit{suffix}}'
in the first form.
However, that requires to set up a different file
for each child. With the alternative form of the command
all these files can have exactly the same content
which simplifies setting them up and maintaining them.

For example, the following file |draft.tex|
with a compilation flag |\version| as described in \secref{sec:flags}
compiles the main document as a draft:
%
\begin{center}
\begin{tabular}{l}
|\def\version{draft}|\\
|\input{childdoc.def}|\\
|\childdocforward{|\textit{main}|}|
\end{tabular}
\end{center}
%
Likewise, the following files |final|\textit{nn}|.tex|
compile the final version of the child document
|child|\textit{nn}|.tex|:
%
\begin{center}
\begin{tabular}{l}
|\def\version{final}|\\
|\input{childdoc.def}|\\
|\childdocforwardprefix{final}{child}|
\end{tabular}
\end{center}
%

Note that when several versions of a main file and/or of each child file
are to be generated, it may be convenient to set up a |Makefile| or
shell script to automatise the process.

%%%%%%%%%%%%%%%%%%%%%%%%%%%%%%%%%%%%%%%%%%%%%%%%%%%%%%%%%%%%%%%%%%%%%%%%%%%%%%%%
\subsection{Command Line Processing}
\label{sec:commandline}

The effect of redirection files can also be achieved by invoking
the \LaTeX{} compiler with a more elaborate command line.
Most conveniently this should be done as part
of a shell script or a |Makefile|.

When using \textsf{childdoc} in the main file, the following
command lines effectively perform a redirection
(note that depending on the shell being used,
backslashes may have to be doubled: `|\|' $\to$ `|\\|'):
%
\begin{center}
|... -jobname "|\textit{target}|" |\\|"|[\textit{flags}]%
|\input{childdoc.def}\childdocforward[|\textit{main}|]{|\textit{dest}|}"|
\end{center}
%
Here \textit{target} is the name of the output file,
\textit{main} is the name of the main file
and \textit{dest} is the name of the main or child file to be processed
(all filenames without extensions).
The optional argument \textit{main} can be omitted
if \textit{main} matches \textit{dest}.
Optionally, compilation \textit{flags} can be defined via |\def| commands.
This command line makes the \TeX{} engine believe
it is compiling the file \textit{target}
whose content is specified as the latter parameter.
The provided code then forwards the processing to
\textit{main} or \textit{dest} as described in \secref{sec:forward}.

%%%%%%%%%%%%%%%%%%%%%%%%%%%%%%%%%%%%%%%%%%%%%%%%%%%%%%%%%%%%%%%%%%%%%%%%%%%%%%%%
\subsection{Include by Input}
\label{sec:input}

Including child documents by |\include| has some restrictions by design.
Most notably, the content of a child document always occupies
its own set of pages; pages cannot be shared between child documents.
Usually, this behaviour makes perfect sense
because each child document contain an essential part of the document.
However, in some situations it may be desirable to compose
a document from a collection of parts
without having mandatory page breaks between then.
For this case, the package
provides a mechanism to include parts
by |\input| which can also be processed individually.
However, by construction this mechanism
requires manual handling of the content to be output.

%%%%%%%%%%%%%%%%%%%%%%%%%%%%%%%%%%%%%%%%
\DescribeMacro{\ifchilddocmanual}
The main file should be prepared as usual, see \secref{sec:include}.
However, the document body must make a distinction
between processing of an individual part and of the main document, e.g.:
%
\begin{center}
\begin{tabular}{l}
|\ifchilddocmanual|\\
|\input{\childdocname}|\\
|\||else|\\
\textit{document body with }|\input{|\textit{part}|}|\\
|\||fi|
\end{tabular}
\end{center}
%
The conditional |\ifchilddocmanual| is true whenever
a part to be included by |\input| is being compiled,
and the name of the part is stored in |\childdocname|.

%%%%%%%%%%%%%%%%%%%%%%%%%%%%%%%%%%%%%%%%
\DescribeMacro{\childdocby}
Each part to be included by |\input| should start with:
%
\begin{center}
\begin{tabular}{l}
|\input{childdoc.def}|\\
|\childdocby{|\textit{main}|}|\\
\end{tabular}
\end{center}
%
The directive |\childdocby| is similar to |\childdocof|
described in \secref{sec:include},
but the subsequent selection of content must be done manually.
To that end, both |\ifchilddoc| and |\ifchilddocmanual|
will be true upon processing of a part,
and the name of the part is stored in |\childdocname|.
Note that |\jobname| will be set to the filename of the current part
so that each part receives an individual |.aux| file
that does not interfere with the |.aux| file(s) of the main document.
This behaviour can be altered by the alternative form
|\childdocby[*]{|\textit{main}|}| (with a non-empty optional argument)
which uses the |.aux| file of the main document
by setting |\jobname| to \textit{main}.

%%%%%%%%%%%%%%%%%%%%%%%%%%%%%%%%%%%%%%%%%%%%%%%%%%%%%%%%%%%%%%%%%%%%%%%%%%%%%%%%
\subsection{Driver Development}
\label{sec:driver}

The \textsf{childdoc} mechanism can also be use for the development
of definition files such as \LaTeX{} styles or classes.
This case differs from the above setup with multiple parts
included by |\include| in that no |\includeonly| should be invoked.
This can be achieved by starting the include file
(before |\ProvidesPackage|) with:
%
\begin{center}
\begin{tabular}{l}
|\input{childdoc.def}|\\
|\childdocforward{|\textit{main}|}|\\
\end{tabular}
\end{center}
%
or alternatively with:
%
\begin{center}
\begin{tabular}{l}
|\input{childdoc.def}|\\
|\childdocby{|\textit{main}|}|\\
\end{tabular}
\end{center}
%
Both forms have slightly different effects as described above.
The main file is prepared as usual, see \secref{sec:include}.

%%%%%%%%%%%%%%%%%%%%%%%%%%%%%%%%%%%%%%%%%%%%%%%%%%%%%%%%%%%%%%%%%%%%%%%%%%%%%%%%
\subsection{Legacy Detection}
\label{sec:detection}

The directive |\childdocmain| in the main file can detect
whether the complete document or merely a child is to be compiled
even without using the directive |\childdocof|.
This method is deprecated because it is less robust
and there is no compelling reason to use it;
it is merely provided for backward compatibility
and it may be removed in future versions.

If the detection mechanism is to be used,
it is mandatory to correctly specify
the filename of the main file as the argument of |\childdocmain|:
%
\begin{center}
\begin{tabular}{l}
|\input{childdoc.def}|\\
|\childdocmain{|\textit{main}|}|\\
\end{tabular}
\end{center}
%
If |\jobname| does not match the argument \textit{main} of |\childdocmain|,
it is assumed that |\jobname| points to the child file to be compiled.
When using |\childdocmain| with the main file specified as argument,
it suffices to start a child file
with just |\input{|\textit{main}|}|
without loading of the package and using |\childdocof|.
If instead all processing is done
with the appropriate \textsf{childdoc} directives,
the argument of \textit{main} of |\childdocmain| can be empty.

An alternative version of the command line processing described
in \secref{sec:commandline} using the detection mechanism reads:
%
\begin{center}
|... -jobname "|\textit{target}|" "|[\textit{flags}]%
[|\def\jobname{|\textit{dest}|}|]|\input{|\textit{main}|}"|
\end{center}

%%%%%%%%%%%%%%%%%%%%%%%%%%%%%%%%%%%%%%%%%%%%%%%%%%%%%%%%%%%%%%%%%%%%%%%%%%%%%%%%
\subsection{Manual Code}
\label{sec:manual}

In case one cannot be certain whether the definitions file |childdoc.def|
is installed on the target \TeX{} distribution
and one prefers not to ship it,
it is conceivable to paste a few relevant commands into the sources.

To that end, drop all statements |\input{childdoc.def}|
and perform the replacements as outlined below.
Instead of |\childdocmain{|\textit{main}|}| add the following code
to the top of the main file:
%
\begin{center}
\begin{tabular}{l}
|\||ifdefined\childdocname\endinput\||fi\newif\ifchilddoc|\\
|\edef\childdocname{\scantokens\expandafter{\jobname\noexpand}}|\\
|\def\childdocmain{|\textit{main}|}\||ifx\childdocmain\childdocname\||else|\\
|\childdoctrue\includeonly{\childdocname}\let\jobname\childdocmain\||fi|\\
\end{tabular}
\end{center}
%
Instead of |\childdocof{|\textit{main}|}| just include the main file
at the top of each child file:
%
\begin{center}
|\input{|\textit{main}|}|
\end{center}
%
A simple redirection |\childdocforward{|\textit{dest}|}| is achieved by:
%
\begin{center}
|\def\jobname{|\textit{dest}|}\input{\jobname}|
\end{center}
%
The redirection with prefix
|\childdocforwardprefix[|\textit{prefix}|]{|\textit{dest}|}|
is accomplished by:
%
\begin{center}
\begin{tabular}{l}
|{\edef\jobname{\scantokens\expandafter{\jobname\noexpand}}|\\
|\def\redirectjob |\textit{prefix}|#1~~~{\gdef\jobname{|\textit{dest}|#1}}|\\
|\expandafter\redirectjob\jobname~~~}\input{\jobname}|
\end{tabular}
\end{center}

In an alternative approach,
child documents can be compiled by a specific command line
without additional code or specific definitions:
%
\begin{center}
|... -jobname "|\textit{target}|" "|[\textit{flags}]%
|\includeonly{|\textit{dest}|}\input{|\textit{main}|}"|
\end{center}
%

%%%%%%%%%%%%%%%%%%%%%%%%%%%%%%%%%%%%%%%%%%%%%%%%%%%%%%%%%%%%%%%%%%%%%%%%%%%%%%%%
%%%%%%%%%%%%%%%%%%%%%%%%%%%%%%%%%%%%%%%%%%%%%%%%%%%%%%%%%%%%%%%%%%%%%%%%%%%%%%%%
\section{Information}

%%%%%%%%%%%%%%%%%%%%%%%%%%%%%%%%%%%%%%%%%%%%%%%%%%%%%%%%%%%%%%%%%%%%%%%%%%%%%%%%
\subsection{Copyright}

Copyright \copyright{} 2017--2018 Niklas Beisert

This work may be distributed and/or modified under the
conditions of the \LaTeX{} Project Public License, either version 1.3
of this license or (at your option) any later version.
The latest version of this license is in
  \url{http://www.latex-project.org/lppl.txt}
and version 1.3 or later is part of all distributions of \LaTeX{}
version 2005/12/01 or later.

This work has the LPPL maintenance status `maintained'.

The Current Maintainer of this work is Niklas Beisert.

This work consists of the files |README.txt|, |childdoc.ins| and |childdoc.dtx|
as well as the derived files |childdoc.def|, |cdocsamp.tex|
with |cdocsch1.tex|, |cdocsch2.tex|, |cdocspt3.tex|, |cdocspt4.tex|,
|cdocsdrf.tex|, |cdocsfn1.tex|, |cdocsfn2.tex|
as well as |childdoc.pdf|.

%%%%%%%%%%%%%%%%%%%%%%%%%%%%%%%%%%%%%%%%%%%%%%%%%%%%%%%%%%%%%%%%%%%%%%%%%%%%%%%%
\subsection{Files and Installation}

The package consists of the files:
%
\begin{center}
\begin{tabular}{ll}
    |README.txt|   & readme file \\
    |childdoc.ins| & installation file \\
    |childdoc.dtx| & source file \\
    |childdoc.def| & definition file \\
    |cdocsamp.tex| & sample main file \\
    |cdocsch1.tex| & sample include file \\
    |cdocsch2.tex| & sample include file \\
    |cdocspt3.tex| & sample part file \\
    |cdocspt4.tex| & sample part file \\
    |cdocsdrf.tex| & sample redirection file \\
    |cdocsfn1.tex| & sample redirection file \\
    |cdocsfn2.tex| & sample redirection file \\
    |childdoc.pdf| & manual
\end{tabular}
\end{center}
%
The distribution consists of the files
|README.txt|, |childdoc.ins| and |childdoc.dtx|.
%
\begin{itemize}
\item
Run (pdf)\LaTeX{} on |childdoc.dtx|
to compile the manual |childdoc.pdf| (this file).
\item
Run \LaTeX{} on |childdoc.ins| to create the definitions file |childdoc.def|
and the sample |cdocsamp.tex| with include files
|cdocsch1.tex|, |cdocsch2.tex|, |cdocspt3.tex|, |cdocspt4.tex|,
|cdocsdrf.tex|, |cdocsfn1.tex|, |cdocsfn2.tex|.
Then copy the file |childdoc.def| to an appropriate directory of your \LaTeX{}
distribution, e.g.\ \textit{texmf-root}|/tex/latex/childdoc|.
\end{itemize}

%%%%%%%%%%%%%%%%%%%%%%%%%%%%%%%%%%%%%%%%%%%%%%%%%%%%%%%%%%%%%%%%%%%%%%%%%%%%%%%%
\subsection{Related CTAN Packages}

There are several other packages which offer a similar functionality:
%
\begin{itemize}
\item
The packages
\href{http://ctan.org/pkg/docmute}{\textsf{docmute}},
\href{http://ctan.org/pkg/includex}{\textsf{includex}} and
\href{http://ctan.org/pkg/standalone}{\textsf{standalone}}
provide commands to include only the document body of
a child file thus allowing both files to be compiled individually.
\item
The packages \href{http://ctan.org/pkg/subdocs}{\textsf{subdocs}}
and \href{http://ctan.org/pkg/subfiles}{\textsf{subfiles}}
provide structures in which the main and child documents can be
encapsulated and allowing them to be compiled individually.
The inclusion mechanism is different from the conventional |\include|.
\item
The package \href{http://ctan.org/pkg/combine}{\textsf{combine}}
is an elaborate solution to combine several documents into one.
\end{itemize}
%
See also the CTAN topic \href{http://ctan.org/topic/subdocs}{\textsf{subdocs}}
for further related packages.
The present package differs from the above solutions in that
a document structure constructed with the conventional |\include| mechanism
just needs two extra commands at the top of every file
such that all constituent files can be compiled individually.

%%%%%%%%%%%%%%%%%%%%%%%%%%%%%%%%%%%%%%%%%%%%%%%%%%%%%%%%%%%%%%%%%%%%%%%%%%%%%%%%
%\subsection{Feature Suggestions}
%
%The following is a list of features which may be useful for future
%versions of this package:
%%
%\begin{itemize}
%\item
%\ldots
%\end{itemize}

%%%%%%%%%%%%%%%%%%%%%%%%%%%%%%%%%%%%%%%%%%%%%%%%%%%%%%%%%%%%%%%%%%%%%%%%%%%%%%%%
\subsection{Revision History}

%%%%%%%%%%%%%%%%%%%%%%%%%%%%%%%%%%%%%%%%
\paragraph{v2.0:} 2018/12/30

\begin{itemize}
\item
immediate forward processing
\item
added |\childdocby| mechanism
\item
manual restructured
\end{itemize}

%%%%%%%%%%%%%%%%%%%%%%%%%%%%%%%%%%%%%%%%
\paragraph{v1.6:} 2018/01/17

\begin{itemize}
\item
application for development of include files
\item
corrections to manual
\end{itemize}

%%%%%%%%%%%%%%%%%%%%%%%%%%%%%%%%%%%%%%%%
\paragraph{v1.5:} 2017/05/21

\begin{itemize}
\item
more complete structuring introduced
\item
|\childdocof| introduced
\item
|\childdoc| renamed to |\childdocmain|
\item
|\childredirect| renamed to |\childdocforward| and |\childdocforwardprefix|
and functionality expanded
\end{itemize}

%%%%%%%%%%%%%%%%%%%%%%%%%%%%%%%%%%%%%%%%
\paragraph{v1.0:} 2017/04/27

\begin{itemize}
\item
manual and install package
\item
first version published on CTAN
\end{itemize}

%%%%%%%%%%%%%%%%%%%%%%%%%%%%%%%%%%%%%%%%
\paragraph{v0.6:} 2017/04/26

\begin{itemize}
\item
redirection mechanism added
\end{itemize}

%%%%%%%%%%%%%%%%%%%%%%%%%%%%%%%%%%%%%%%%
\paragraph{v0.5:} 2017/04/26

\begin{itemize}
\item
functionality in definition file
\end{itemize}


%%%%%%%%%%%%%%%%%%%%%%%%%%%%%%%%%%%%%%%%%%%%%%%%%%%%%%%%%%%%%%%%%%%%%%%%%%%%%%%%
%%%%%%%%%%%%%%%%%%%%%%%%%%%%%%%%%%%%%%%%%%%%%%%%%%%%%%%%%%%%%%%%%%%%%%%%%%%%%%%%
%%%%%%%%%%%%%%%%%%%%%%%%%%%%%%%%%%%%%%%%%%%%%%%%%%%%%%%%%%%%%%%%%%%%%%%%%%%%%%%%
\appendix

\settowidth\MacroIndent{\rmfamily\scriptsize 000\ }

 \DocInput{childdoc.dtx}

\end{document}
%</driver>
% \fi
%
% %%%%%%%%%%%%%%%%%%%%%%%%%%%%%%%%%%%%%%%%%%%%%%%%%%%%%%%%%%%%%%%%%%%%%%%%%%%%%%
% %%%%%%%%%%%%%%%%%%%%%%%%%%%%%%%%%%%%%%%%%%%%%%%%%%%%%%%%%%%%%%%%%%%%%%%%%%%%%%
% \section{Sample}
%\iffalse
%<*samplemain>
%\fi
%
% The following presents a sample document
% with two chapters, two parts, a title page,
% a compile flag as well as three forwarding files to set the flag.
% It consists of eight |.tex| files:
% \begin{center}
% \begin{tabular}{ll}
% |cdocsamp.tex|&main file\\
% |cdocsch1.tex|&include file for chapter 1\\
% |cdocsch2.tex|&include file for chapter 2\\
% |cdocspt3.tex|&include file for part 3\\
% |cdocspt4.tex|&include file for part 4\\
% |cdocsdrf.tex|&forwarding file for main file in draft mode\\
% |cdocsfi1.tex|&forwarding file for final version of chapter 1\\
% |cdocsfi2.tex|&forwarding file for final version of chapter 2\\
% \end{tabular}
% \end{center}
% Each of the eight files can be compiled directly by the \LaTeX{} compiler.
%
% %%%%%%%%%%%%%%%%%%%%%%%%%%%%%%%%%%%%%%
% \paragraph{Main File.}
%
% The main file is called |cdocsamp.tex|.
%
% Load the \textsf{childdoc} definitions and
% declare the filename for the main document:
%    \begin{macrocode}
\input{childdoc.def}
\childdocmain{}
%    \end{macrocode}

% Optional override for |\version| flag:
%    \begin{macrocode}
%%\ifchilddoc\else\providecommand{\version}{draft}\fi
%    \end{macrocode}

% Define the default values for the |\version| flag
% (|final| for the main file and |draft| for childs):
%    \begin{macrocode}
\ifchilddoc
\providecommand{\version}{draft}
\else
\providecommand{\version}{final}
\fi
%    \end{macrocode}

% Load the standard document class:
%    \begin{macrocode}
\documentclass[12pt]{article}
%    \end{macrocode}

% Start the document body:
%    \begin{macrocode}
\begin{document}
%    \end{macrocode}

% Declare a title page.
% Print title, part of document being processed and version flag:
%    \begin{macrocode}
\addtocounter{page}{-1}
\begin{center}
{\LARGE\bfseries{}childdoc example\par}
\vspace{1cm}
\ifchilddoc
\ifchilddocmanual part\else chapter\fi:
`\childdocname' of `\childdocjob'\par
\else
main document: `\childdocjob'\par
\fi
version: \version\par
\end{center}
\newpage
%    \end{macrocode}

% Manually include selected file,
% otherwise process as usual:
%    \begin{macrocode}
\ifchilddocmanual
\section*{part `\childdocname'}
\input{\childdocname}
\else
%    \end{macrocode}

% Include the two chapters:
%    \begin{macrocode}
\include{cdocsch1}
\include{cdocsch2}
%    \end{macrocode}

% Include the two parts unless only chapters should be displayed:
%    \begin{macrocode}
\ifchilddoc\else
\section{part three}
\input{cdocspt3}
\section{part four}
\input{cdocspt4}
\fi
%    \end{macrocode}

% Process as usual until here:
%    \begin{macrocode}
\fi
%    \end{macrocode}

% End of document body:
%    \begin{macrocode}
\end{document}
%    \end{macrocode}
%\iffalse
%</samplemain>
%\fi
%
% %%%%%%%%%%%%%%%%%%%%%%%%%%%%%%%%%%%%%%
% \paragraph{Chapter Include Files.}
%
% The include files are called |cdocsch1.tex| and |cdocsch2.tex|.
%
%\iffalse
%<*samplechap1|samplechap2>
%\fi

% Optional override for |\version| flag:
%    \begin{macrocode}
%%\providecommand{\version}{final}
%    \end{macrocode}

% Include the main document:
%    \begin{macrocode}
\input{childdoc.def}
\childdocof{cdocsamp}
%    \end{macrocode}

%\iffalse
%</samplechap1|samplechap2>
%\fi
%
%\iffalse
%<*samplechap1>
%\fi
% Some text for chapter 1:
%    \begin{macrocode}
\section{one}
some text in chapter one
%    \end{macrocode}

%\iffalse
%</samplechap1>
%\fi
% Some text for chapter 2:
%\iffalse
%<*samplechap2>
%\fi
%    \begin{macrocode}
\section{two}
more text in chapter two
%    \end{macrocode}

%\iffalse
%</samplechap2>
%\fi
%
% %%%%%%%%%%%%%%%%%%%%%%%%%%%%%%%%%%%%%%
% \paragraph{Part Include Files.}
%
% The include files are called |cdocspt3.tex| and |cdocspt4.tex|.
%
%\iffalse
%<*samplepart3|samplepart4>
%\fi

% Optional override for |\version| flag:
%    \begin{macrocode}
%%\providecommand{\version}{final}
%    \end{macrocode}

% Include the main document:
%    \begin{macrocode}
\input{childdoc.def}
\childdocby{cdocsamp}
%    \end{macrocode}

%\iffalse
%</samplepart3|samplepart4>
%\fi
%
%\iffalse
%<*samplepart3>
%\fi
% Some text for part 3:
%    \begin{macrocode}
some text in part three
%    \end{macrocode}

%\iffalse
%</samplepart3>
%\fi
% Some text for part 4:
%\iffalse
%<*samplepart4>
%\fi
%    \begin{macrocode}
more text in part four
%    \end{macrocode}

%\iffalse
%</samplepart4>
%\fi
%
% %%%%%%%%%%%%%%%%%%%%%%%%%%%%%%%%%%%%%%
% \paragraph{Forwarding for a Complete Draft.}
%
% The following forwarding file |cdocsdrf.tex|
% compiles the main document in draft mode:
%\iffalse
%<*sampledraft>
%\fi
%    \begin{macrocode}
\def\version{draft}
\input{childdoc.def}
\childdocforward{cdocsamp}
%    \end{macrocode}

%\iffalse
%</sampledraft>
%\fi
%
% %%%%%%%%%%%%%%%%%%%%%%%%%%%%%%%%%%%%%%
% \paragraph{Forwarding for Final Version of the Chapters.}
%
% The following forwarding files |cdocsfn1.tex| and |cdocsfn2.tex|
% (with identical content)
% compile the final versions of the child documents
% |cdocsch1.tex| and |cdocsch2.tex|, respectively:
%\iffalse
%<*samplefinal>
%\fi
%    \begin{macrocode}
\def\version{final}
\input{childdoc.def}
\childdocforwardprefix[cdocsamp]{cdocsfn}{cdocsch}
%    \end{macrocode}

%\iffalse
%</samplefinal>
%\fi
%
% %%%%%%%%%%%%%%%%%%%%%%%%%%%%%%%%%%%%%%
% \paragraph{Command Line Processing.}
%
% The following three command lines generate the output files
% |cdocscld|, |cdocscl1| and |cdocscl2|
% which should be identical to
% |cdocsdrf|, |cdocsch1| and |cdocsfn2|, respectively:
% \begin{center}
% \begin{tabular}{l}
% |latex -jobname cdocscld \|\\
% |  "\def\version{draft}\input{childdoc.def}\childdocforward{cdocsamp}"|\\
% |latex -jobname cdocscl1 \|\\
% |  "\input{childdoc.def}\childdocforward[cdocsamp]{cdocsch1}"|\\
% |latex -jobname cdocscl2 \|\\
% |  "\def\version{final}\input{childdoc.def}\childdocforward{cdocsch2}"|
% \end{tabular}
% \end{center}
% Note that the trailing backslash on each first line
% merely continues the input to the second line
% (for convenient cut ant paste).
% Furthermore, the command |latex| can be replaced by any
% of its alternative versions such as |pdflatex|.
%
% %%%%%%%%%%%%%%%%%%%%%%%%%%%%%%%%%%%%%%%%%%%%%%%%%%%%%%%%%%%%%%%%%%%%%%%%%%%%%%
% %%%%%%%%%%%%%%%%%%%%%%%%%%%%%%%%%%%%%%%%%%%%%%%%%%%%%%%%%%%%%%%%%%%%%%%%%%%%%%
% \section{Implementation}
%\iffalse
%<*package>
%\fi
%
% This section describes the definitions file |childdoc.def|.

% The definitions cannot be loaded using |\usepackage| or |\RequirePackage|
% which has a mechanism to prevent loading a style file more than once.
% When loading the definitions by means of |\input|
% multiple instances have to be prevented manually:
%\iffalse
%This code needs to be before the `\ProvidesFile' directive
%which is defined at the beginning of this file.
%Therefore it is also placed there and commented out here.
%</package>
%<*discard>
%\fi
%    \begin{macrocode}
\ifdefined\childdocmain\endinput\fi
%    \end{macrocode}
%\iffalse
%</discard>
%<*package>
%\fi
%
% \macro{\ifchilddoc}
% \macro{\ifchilddocmanual}
% The conditional |\ifchilddoc| tells whether a
% child (true) or main (false) document is being compiled.
% The conditional |\ifchilddocmanual| tells whether
% the |\includeonly| mechanism is used (false) or
% the selection of child files must be performed manually (true).
% The definitions initialise to false:
%    \begin{macrocode}
\newif\ifchilddoc
\newif\ifchilddocmanual
%    \end{macrocode}

% \macro{\childdocname}
% \macro{\childdocjob}
% The macro |\childdocname| stores the name of the main document
% to be compiled. The macro |\childdocjob| stores the name of
% the document on which the \LaTeX{} compiler was originally invoked.
% The content of |\jobname| cannot be compared
% to filenames specified in the source due to different catcodes.
% The following code rescans |\jobname|, stores the result
% in |\childdocname| and saves a copy in |\childdocjob|:
%    \begin{macrocode}
\edef\childdocname{\scantokens\expandafter{\jobname\noexpand}}
\let\childdocjob\childdocname
%    \end{macrocode}

% \macro{\childdocdisable}
% The macro |\childdocdisable| prevents the main file
% from being processed more than once.
% At this stage, the main document command |\childdocmain|
% is assumed to be called once again where it should do nothing.
% Any subsequent call to it should prevent
% a secondary processing of the main document
% It overwrites the forwarding commands
% |\childdocof| and |\childdocforward|
% with empty macros to prevent further inclusions of the main document:
%    \begin{macrocode}
\newcommand{\childdocdisable}
{
  \renewcommand{\childdocmain}[1]{\renewcommand{\childdocmain}[1]{\endinput}}
  \renewcommand{\childdocof}[1]{}
  \renewcommand{\childdocby}[2][]{}
  \renewcommand{\childdocforward}[2][]{}
  \renewcommand{\childdocdisable}{}
}
%    \end{macrocode}

% \macro{\childdocmain}
% The macro |\childdocmain| is to be called at the top of the main file
% with nothing or the main filename (without extension) as argument.
% First, it breaks loops.
% If the argument is not empty and does not match |\childdocname|
% (which is set by the first inclusion of |childdoc.def|),
% |\ifchilddoc| is set to true, |\includeonly| is applied to the child file
% and |\jobname| is set to the main file
% (for proper handling of |.aux| files):
%    \begin{macrocode}
\newcommand{\childdocmain}[1]
{
  \childdocdisable\childdocmain{}
  \if?#1?\else
    \begingroup
      \def\childdoctmp{#1}
      \ifx\childdoctmp\childdocname
        \def\childdoctmp{}
      \else
        \def\childdoctmp
        {
          \childdoctrue
          \includeonly{\childdocname}
          \def\childdocjob{#1}
          \def\jobname{#1}
        }
      \fi
      \expandafter
    \endgroup
    \childdoctmp
  \fi
}
%    \end{macrocode}

% \macro{\childdocof}
% The command |\childdocof| redirects
% compilation to the main file |#1|.
%    \begin{macrocode}
\newcommand{\childdocof}[1]
{
  \childdocdisable
  \childdoctrue
  \includeonly{\childdocname}
  \def\jobname{#1}
  \def\childdocjob{#1}
  \input{#1}
}
%    \end{macrocode}

% \macro{\childdocby}
% The command |\childdocby| ....
%    \begin{macrocode}
\newcommand{\childdocby}[2][]
{
  \childdocdisable
  \childdoctrue
  \childdocmanualtrue
  \if?#1?\else
    \def\jobname{#2}
  \fi
  \def\childdocjob{#2}
  \input{#2}
  \endinput
}
%    \end{macrocode}

% \macro{\childdocforward}
% The command |\childdocforward| redirects
% compilation to the main file or
% (if the optional argument is given) a child file.
% Parameters are set as if the main file
% or a child file starting with |\childdocof| was compiled.
% Then compilation is handed over to the main file:
%    \begin{macrocode}
\newcommand{\childdocforward}[2][]
{
  \begingroup
    \if?#1?
      \def\childdoctmp
      {
        \def\childdocname{#2}
        \def\childdocjob{#2}
        \def\jobname{#2}
        \input{#2}
        \endinput
      }
    \else
      \def\childdoctmp
      {
        \childdocdisable
        \def\childdocname{#2}
        \childdoctrue
        \includeonly{#2}
        \def\childdocjob{#1}
        \def\jobname{#1}
        \input{#1}
        \endinput
      }
    \fi
    \expandafter
  \endgroup
  \childdoctmp
}
%    \end{macrocode}

% \macro{\childdocforwardprefix}
% The command |\childdocforwardprefix| redirects
% compilation to the main or a child file by means of a pattern.
% The prefix |#1| in the current filename is replaced by |#2|
% and the suffix of the current filename is kept
% (it is assumed that the filename does not contain the substring `|~~~|'
% which is used as a delimiter).
% Compilation is handed over to the new file by |\childdocforward|:
%    \begin{macrocode}
\newcommand{\childdocforwardprefix}[3][]
{
  \begingroup
    \def\childdocextract #2##1~~~{\def\childdoctmp{\childdocforward[#1]{#3##1}}}
    \expandafter\childdocextract\childdocname~~~
    \expandafter
  \endgroup
  \childdoctmp
}
%    \end{macrocode}

% \macro{\childdoc}
% The deprecated macro |\childdoc| is a legacy version of |\childdocmain|:
%    \begin{macrocode}
\newcommand{\childdoc}{\childdocmain}
%    \end{macrocode}

% \macro{\childdocredirect}
% The deprecated macro |\childdocredirect| is a legacy version
% of |\childdocforward| and |\childdocforwardprefix|:
%    \begin{macrocode}
\newcommand{\childdocredirect}[2][]
{
  \begingroup
    \if?#1?
      \def\childdoctmp{\childdocforward{#2}}
    \else
      \def\childdoctmp{\childdocforwardprefix{#1}{#2}}
    \fi
    \expandafter
  \endgroup
  \childdoctmp
}
%    \end{macrocode}

%\iffalse
%</package>
%\fi
%
\endinput

\childdocby{cdocsamp}
%    \end{macrocode}

%\iffalse
%</samplepart3|samplepart4>
%\fi
%
%\iffalse
%<*samplepart3>
%\fi
% Some text for part 3:
%    \begin{macrocode}
some text in part three
%    \end{macrocode}

%\iffalse
%</samplepart3>
%\fi
% Some text for part 4:
%\iffalse
%<*samplepart4>
%\fi
%    \begin{macrocode}
more text in part four
%    \end{macrocode}

%\iffalse
%</samplepart4>
%\fi
%
% %%%%%%%%%%%%%%%%%%%%%%%%%%%%%%%%%%%%%%
% \paragraph{Forwarding for a Complete Draft.}
%
% The following forwarding file |cdocsdrf.tex|
% compiles the main document in draft mode:
%\iffalse
%<*sampledraft>
%\fi
%    \begin{macrocode}
\def\version{draft}
% \iffalse
%
% childdoc.dtx Copyright (C) 2017-2018 Niklas Beisert
%
% This work may be distributed and/or modified under the
% conditions of the LaTeX Project Public License, either version 1.3
% of this license or (at your option) any later version.
% The latest version of this license is in
%   http://www.latex-project.org/lppl.txt
% and version 1.3 or later is part of all distributions of LaTeX
% version 2005/12/01 or later.
%
% This work has the LPPL maintenance status `maintained'.
%
% The Current Maintainer of this work is Niklas Beisert.
%
% This work consists of the files childdoc.dtx and childdoc.ins
% and the derived files childdoc.def and cdocsamp.tex with
% cdocsch1.tex, cdocsch2.tex, cdocsdrf.tex, cdocsfn1.tex, cdocsfn2.tex.
%
%<package>\ifdefined\childdocmain\endinput\fi
%<package>\ProvidesFile{childdoc.def}[2018/12/30 v2.0 child document driver]
%<samplemain>\ProvidesFile{cdocsamp.tex}[2018/12/30 v2.0 sample for childdoc]
%<*driver>
%\ProvidesFile{childdoc.drv}[2018/12/30 v2.0 childdoc reference manual file]
\PassOptionsToClass{10pt,a4paper}{article}
\documentclass{ltxdoc}

\usepackage[margin=35mm]{geometry}
\usepackage{hyperref}
\usepackage{hyperxmp}
\usepackage[usenames]{color}

\hypersetup{colorlinks=true}
\hypersetup{pdfstartview=FitH}
\hypersetup{pdfpagemode=UseNone}
\hypersetup{pdfsource={}}
\hypersetup{pdflang={en-UK}}
\hypersetup{pdfcopyright={Copyright 2017-2018 Niklas Beisert.
  This work may be distributed and/or modified under the
  conditions of the LaTeX Project Public License, either version 1.3
  of this license or (at your option) any later version.}}
\hypersetup{pdflicenseurl={http://www.latex-project.org/lppl.txt}}
\hypersetup{pdfcontactaddress={ETH Zurich, ITP, HIT K,
  Wolfgang-Pauli-Strasse 27}}
\hypersetup{pdfcontactpostcode={8093}}
\hypersetup{pdfcontactcity={Zurich}}
\hypersetup{pdfcontactcountry={Switzerland}}
\hypersetup{pdfcontactemail={nbeisert@itp.phys.ethz.ch}}
\hypersetup{pdfcontacturl={http://people.phys.ethz.ch/\xmptilde nbeisert/}}

\newcommand{\secref}[1]{\hyperref[#1]{section \ref*{#1}}}

\parskip1ex
\parindent0pt
\let\olditemize\itemize
\def\itemize{\olditemize\parskip0pt}

\begin{document}

\title{The \textsf{childdoc} Package}
\hypersetup{pdftitle={The childdoc Package}}
\author{Niklas Beisert\\[2ex]
  Institut f\"ur Theoretische Physik\\
  Eidgen\"ossische Technische Hochschule Z\"urich\\
  Wolfgang-Pauli-Strasse 27, 8093 Z\"urich, Switzerland\\[1ex]
  \href{mailto:nbeisert@itp.phys.ethz.ch}
  {\texttt{nbeisert@itp.phys.ethz.ch}}}
\hypersetup{pdfauthor={Niklas Beisert}}
\hypersetup{pdfsubject={Manual for the LaTeX2e Package childdoc}}
\date{30 December 2018, \textsf{v2.0}}
\maketitle

\begin{abstract}\noindent
\textsf{childdoc} is a \LaTeXe{} package
that enables the direct compilation
of document sections included by |\include|
to individual files.
\end{abstract}

\begingroup
\parskip0ex
\tableofcontents
\endgroup

%%%%%%%%%%%%%%%%%%%%%%%%%%%%%%%%%%%%%%%%%%%%%%%%%%%%%%%%%%%%%%%%%%%%%%%%%%%%%%%%
%%%%%%%%%%%%%%%%%%%%%%%%%%%%%%%%%%%%%%%%%%%%%%%%%%%%%%%%%%%%%%%%%%%%%%%%%%%%%%%%
\section{Introduction}

\LaTeX{} provides a mechanism to structure a large document (such as a book)
into a main file and several child files (containing the chapters)
using the |\include| command.
This mechanism is beneficial for documents
which span hundreds of pages in order to
make the source file(s) more manageable.
Moreover, compilation can be restricted to
selected child files by means of the |\includeonly| command.
The latter feature can be used to reduce the compilation time while editing
(this was significantly more useful in the earlier days of \LaTeX{})
or to generate a smaller document which is easier to navigate.
Another application of |\includeonly| is to generate
documents consisting of selected parts of the complete document.

However, there are a few drawbacks of the plain |\include| mechanism:
\begin{itemize}
\item
The child files cannot be compiled on their own,
they can only be compiled via the main file.
A naive editing environment
(such as a text editor with an option
to have the current file processed by \LaTeX)
may require one to switch to the main file before compiling;
attempting to compile the child file produces errors.
\item
The main file must be modified (each time)
to adjust the |\includeonly| command
to the present needs. This easily leaves the main file in a messy state.
\item
The generated document will always carry the filename
of the main document. This is inconvenient if
several child files are to be compiled and
to be kept for distribution.
\end{itemize}

The present package provides a simple interface
to make child files individually compilable by \LaTeX{}.
Compiling a child file then has the same effect as compiling
the main file with an |\includeonly| command
to select the appropriate child.
Moreover the generated document will carry the name of the child
rather than the main file.
This resolves all three above issues.

This feature is meant to make the editing of books,
thesis documents and lecture notes somewhat more convenient.
However, the package can also be used efficiently for
composing a series of documents (such as exercise sheets)
which are typically distributed individually.
It then assists the author in generating the individual documents
(potentially in different versions)
as well as a document containing the collected series.
Another application is in developing style files
or other kinds of included material
where compilation of the style file could redirect
to a sample or test file.

%%%%%%%%%%%%%%%%%%%%%%%%%%%%%%%%%%%%%%%%%%%%%%%%%%%%%%%%%%%%%%%%%%%%%%%%%%%%%%%%
%%%%%%%%%%%%%%%%%%%%%%%%%%%%%%%%%%%%%%%%%%%%%%%%%%%%%%%%%%%%%%%%%%%%%%%%%%%%%%%%
\section{Usage}

First of all, the package \textsf{childdoc} is \emph{not} a standard
\LaTeXe{} |.sty| style file! Therefore it needs to be invoked in
a non-standard way.

%%%%%%%%%%%%%%%%%%%%%%%%%%%%%%%%%%%%%%%%%%%%%%%%%%%%%%%%%%%%%%%%%%%%%%%%%%%%%%%%
\subsection{Included Files}
\label{sec:include}

%%%%%%%%%%%%%%%%%%%%%%%%%%%%%%%%%%%%%%%%
\DescribeMacro{\childdocmain}
To use the package, add the commands
\begin{center}
\begin{tabular}{l}
|\input{childdoc.def}|\\
|\childdocmain{}|\\
\end{tabular}
\end{center}
at the very top of the main \LaTeX{} file,
in particular \emph{before} the |\documentclass| statement!
The argument of |\childdocmain| should be left empty
(but it must be present).

%%%%%%%%%%%%%%%%%%%%%%%%%%%%%%%%%%%%%%%%
\DescribeMacro{\childdocof}
Furthermore, add the commands
\begin{center}
\begin{tabular}{l}
|\input{childdoc.def}|\\
|\childdocof{|\textit{main}|}|\\
\end{tabular}
\end{center}
at the top of every child file \textit{child}
which is included by |\include{|\textit{child}|}|
from within the main file
(or at least for those files to be compiled individually).
The argument \textit{main} must be the filename of the main file.

There are a couple of
considerations in setting up the main and child documents:

%%%%%%%%%%%%%%%%%%%%%%%%%%%%%%%%%%%%%%%%
\paragraph{Restrictions.}

Please note the following restrictions:
\begin{itemize}
\item
|\childdocmain| must be called with one argument \textit{main}
to ensure compatibility with earlier version of the package.
It must either be empty (|\childdocmain{}|)
or precisely match the filename of the main file in which it is specified.
See \secref{sec:detection} for further information.
\item
The filename \textit{main} must be specified without the |.tex| extension.
\item
The filename \textit{main} is case sensitive
(even in case-insensitive file systems)
due to internal string comparison.
\item
The argument \textit{main} should be fully expanded, it cannot be a macro.
\item
Subdirectories and special characters should be avoided in filenames.
\item
The command |\childdocmain{|\textit{main}|}| must be followed by a whitespace.
It should not be followed immediately by another command
or by a comment mark `|%|'.
This is because the \TeX{} parser reads the token immediately following
the argument of |\childdocmain| and puts it
at the beginning of every child section;
however, a white\-space is ignored.
\end{itemize}

%%%%%%%%%%%%%%%%%%%%%%%%%%%%%%%%%%%%%%%%
\paragraph{Content of Main File.}

It is advisable to place all content in the child files included by |\include|.
Any output contained in the main file will appear in all child documents
unless suppressed manually;
it cannot be suppressed automatically by the |\includeonly| directive
and thus should normally be avoided.
A method to include some content in the main file
by means of conditional processing is described in \secref{sec:conditional}.

%%%%%%%%%%%%%%%%%%%%%%%%%%%%%%%%%%%%%%%%
\paragraph{Page Numbering.}

When only a part of the document is compiled,
the appropriate numbering of pages
(as well as other status parameters)
is determined from the |.aux| files.
The latter contain information from previous passes.
However this information needs to propagate through
all intermediate child documents.
Therefore the page numbering in child documents may well
be inconsistent until the complete document is compiled at least once.

A useful (if unconventional) way to always ensure a consistent
page numbering is to restart the numbering in each child document
and denote the pages by `\textit{child}|.|\textit{page}'
where \textit{child} represents the chapter/section number of the child file.
This can be achieved by the command
|\numberwithin{page}{|\textit{child}|}|
of the \textsf{amsmath} package
where \textit{child} can be |chapter| or |section|
depending on the chosen structuring.
Alternatively, one can modify the macro |\thepage| appropriately
and reset the counter |page| at the start of each child file.

%%%%%%%%%%%%%%%%%%%%%%%%%%%%%%%%%%%%%%%%%%%%%%%%%%%%%%%%%%%%%%%%%%%%%%%%%%%%%%%%
\subsection{Conditional Processing}
\label{sec:conditional}

The package provides a mechanism to compile different versions
of a document. To customise the versions further some conditional processing
can come in handy to distinguish which version is being compiled.
The package provides two macros to describe the compilation context:

%%%%%%%%%%%%%%%%%%%%%%%%%%%%%%%%%%%%%%%%
\DescribeMacro{\ifchilddoc}
The conditional |\ifchilddoc| distinguishes between the compilation of
child documents and the main document:
%
\begin{center}
|\ifchilddoc |\textit{child-code}| |[|\||else |\textit{main-code}]| \||fi|
\end{center}

%%%%%%%%%%%%%%%%%%%%%%%%%%%%%%%%%%%%%%%%
\DescribeMacro{\childdocname}
\DescribeMacro{\childdocjob}
The macro |\childdocname| contains the filename (without extension)
of the main or child file being processed.
Note that |\childdocjob| will always contain the name of the main file.

%%%%%%%%%%%%%%%%%%%%%%%%%%%%%%%%%%%%%%%%
\paragraph{Title Page.}

Conditional processing can be used to include a title or banner page
in the main document when proper precautions are taken.
Importantly, the code in the main file should ensure that the page counter
(as well as other status parameters which are stored in the |.aux| files)
takes the same value after the conditional processing.
Otherwise the page numbers may take divergent values
depending on which part is compiled.

For example, a title page could be declared by:
%
\begin{center}
\begin{tabular}{l}
|\ifchilddoc\||else|\\
|\addtocounter{page}{-1}|\\
\textit{code for title page}\\
|\newpage|\\
|\||fi|
\end{tabular}
\end{center}
%
A banner page for the child documents can be generated by:
%
\begin{center}
\begin{tabular}{l}
|\ifchilddoc|\\
|\addtocounter{page}{-1}|\\
\textit{code for banner page}\\
|\newpage|\\
|\||fi|
\end{tabular}
\end{center}
%
Here one could write a message such as:
\begin{center}
|This is the part \childdocname{} of \childdocjob{}.|
\end{center}

%%%%%%%%%%%%%%%%%%%%%%%%%%%%%%%%%%%%%%%%%%%%%%%%%%%%%%%%%%%%%%%%%%%%%%%%%%%%%%%%
\subsection{Flags}
\label{sec:flags}

The package makes it easy to generate different versions
of the main or child documents.
To this end compilation flags can be defined
and assigned different default values.
They will be particularly useful in conjunction
with the forwarding mechanism described in \secref{sec:forward}.

For example, it may be useful to have a flag |\version|
which can be set to |draft| or |final|.
The document source will contain some conditional code
depending on the value of |\version|.
Suppose further, the flag should default to |final| for the main file
and to |draft| for child files
which is a natural assignment for editing the document.
This is achieved by placing the following code
in the preamble of the main document
(below the |\childdocmain| directive):
%
\begin{center}
\begin{tabular}{l}
|\ifchilddoc|\\
|\providecommand{\version}{draft}|\\
|\||else|\\
|\providecommand{\version}{final}|\\
|\||fi|
\end{tabular}
\end{center}
%
The definition by |\providecommand| makes sure
that previous definitions are not overwritten.
Further statements |\providecommand{\version}{...}|
can thus be added before the above code to override it.

For the main file, one might add a line
(between |\childdocmain| and the above block)
%
\begin{center}
|%\ifchilddoc\||else\providecommand{\version}{draft}\||fi|
\end{center}
%
which can be uncommented to produce a draft version.
Likewise one can add a line to the very top of a child file
(above the |\childdocof{|\textit{main}|}| directive)
%
\begin{center}
|%\providecommand{\version}{final}|
\end{center}
%
which can be uncommented to produce the final version of this child document.

%%%%%%%%%%%%%%%%%%%%%%%%%%%%%%%%%%%%%%%%%%%%%%%%%%%%%%%%%%%%%%%%%%%%%%%%%%%%%%%%
\subsection{Forwarding}
\label{sec:forward}

Different versions of the main or child documents
using compilation flags as described in \secref{sec:flags}
can be (permanently) stored in different files
for convenient compilation, viewing and distribution.
To this end, the package defines a command
to pass on compilation to a different file:

%%%%%%%%%%%%%%%%%%%%%%%%%%%%%%%%%%%%%%%%
\DescribeMacro{\childdocforward}
The command |\childdocforward| redirects processing to
another source file:
%
\begin{center}
\begin{tabular}{l}
|\input{childdoc.def}|\\
|\childdocforward[|\textit{main}|]{|\textit{dest}|}|\\
\end{tabular}
\end{center}
%
The argument \textit{dest} is the destination file
(without extension).
It should be the main file or one of the child files.
Note that further \textsf{childdoc} directives
such as |\childdocof| and |\childdocforward|
in the indicated file will be processed in this form.
The optional argument \textit{main}
passes on directly to the main file \textit{main}
while pretending to compile the child \textit{dest}.
This form behaves as if \textit{dest}
issues |\childdocof{|\textit{main}|}| right away,
and no further \textsf{childdoc} directives will be processed.

%%%%%%%%%%%%%%%%%%%%%%%%%%%%%%%%%%%%%%%%
\DescribeMacro{\...prefix}
In the alternative form |\childdocforwardprefix|,
%
\begin{center}
\begin{tabular}{l}
|\input{childdoc.def}|\\
|\childdocforwardprefix[|\textit{main}|]{|\textit{prefix}|}{|\textit{dest}|}|
\end{tabular}
\end{center}
%
the destination file is determined by a pattern
depending on the current file:
To make this work, the current file must be called
`{\textit{prefix}\hspace{0.2em}\textit{suffix}}'
with \textit{prefix} matching precisely the argument.
Processing is then passed on to the file
`{\textit{dest}\hspace{0.2em}\textit{suffix}}'.
Surely, the same effect is achieved by
directly specifying the
argument `{\textit{dest}\hspace{0.2em}\textit{suffix}}'
in the first form.
However, that requires to set up a different file
for each child. With the alternative form of the command
all these files can have exactly the same content
which simplifies setting them up and maintaining them.

For example, the following file |draft.tex|
with a compilation flag |\version| as described in \secref{sec:flags}
compiles the main document as a draft:
%
\begin{center}
\begin{tabular}{l}
|\def\version{draft}|\\
|\input{childdoc.def}|\\
|\childdocforward{|\textit{main}|}|
\end{tabular}
\end{center}
%
Likewise, the following files |final|\textit{nn}|.tex|
compile the final version of the child document
|child|\textit{nn}|.tex|:
%
\begin{center}
\begin{tabular}{l}
|\def\version{final}|\\
|\input{childdoc.def}|\\
|\childdocforwardprefix{final}{child}|
\end{tabular}
\end{center}
%

Note that when several versions of a main file and/or of each child file
are to be generated, it may be convenient to set up a |Makefile| or
shell script to automatise the process.

%%%%%%%%%%%%%%%%%%%%%%%%%%%%%%%%%%%%%%%%%%%%%%%%%%%%%%%%%%%%%%%%%%%%%%%%%%%%%%%%
\subsection{Command Line Processing}
\label{sec:commandline}

The effect of redirection files can also be achieved by invoking
the \LaTeX{} compiler with a more elaborate command line.
Most conveniently this should be done as part
of a shell script or a |Makefile|.

When using \textsf{childdoc} in the main file, the following
command lines effectively perform a redirection
(note that depending on the shell being used,
backslashes may have to be doubled: `|\|' $\to$ `|\\|'):
%
\begin{center}
|... -jobname "|\textit{target}|" |\\|"|[\textit{flags}]%
|\input{childdoc.def}\childdocforward[|\textit{main}|]{|\textit{dest}|}"|
\end{center}
%
Here \textit{target} is the name of the output file,
\textit{main} is the name of the main file
and \textit{dest} is the name of the main or child file to be processed
(all filenames without extensions).
The optional argument \textit{main} can be omitted
if \textit{main} matches \textit{dest}.
Optionally, compilation \textit{flags} can be defined via |\def| commands.
This command line makes the \TeX{} engine believe
it is compiling the file \textit{target}
whose content is specified as the latter parameter.
The provided code then forwards the processing to
\textit{main} or \textit{dest} as described in \secref{sec:forward}.

%%%%%%%%%%%%%%%%%%%%%%%%%%%%%%%%%%%%%%%%%%%%%%%%%%%%%%%%%%%%%%%%%%%%%%%%%%%%%%%%
\subsection{Include by Input}
\label{sec:input}

Including child documents by |\include| has some restrictions by design.
Most notably, the content of a child document always occupies
its own set of pages; pages cannot be shared between child documents.
Usually, this behaviour makes perfect sense
because each child document contain an essential part of the document.
However, in some situations it may be desirable to compose
a document from a collection of parts
without having mandatory page breaks between then.
For this case, the package
provides a mechanism to include parts
by |\input| which can also be processed individually.
However, by construction this mechanism
requires manual handling of the content to be output.

%%%%%%%%%%%%%%%%%%%%%%%%%%%%%%%%%%%%%%%%
\DescribeMacro{\ifchilddocmanual}
The main file should be prepared as usual, see \secref{sec:include}.
However, the document body must make a distinction
between processing of an individual part and of the main document, e.g.:
%
\begin{center}
\begin{tabular}{l}
|\ifchilddocmanual|\\
|\input{\childdocname}|\\
|\||else|\\
\textit{document body with }|\input{|\textit{part}|}|\\
|\||fi|
\end{tabular}
\end{center}
%
The conditional |\ifchilddocmanual| is true whenever
a part to be included by |\input| is being compiled,
and the name of the part is stored in |\childdocname|.

%%%%%%%%%%%%%%%%%%%%%%%%%%%%%%%%%%%%%%%%
\DescribeMacro{\childdocby}
Each part to be included by |\input| should start with:
%
\begin{center}
\begin{tabular}{l}
|\input{childdoc.def}|\\
|\childdocby{|\textit{main}|}|\\
\end{tabular}
\end{center}
%
The directive |\childdocby| is similar to |\childdocof|
described in \secref{sec:include},
but the subsequent selection of content must be done manually.
To that end, both |\ifchilddoc| and |\ifchilddocmanual|
will be true upon processing of a part,
and the name of the part is stored in |\childdocname|.
Note that |\jobname| will be set to the filename of the current part
so that each part receives an individual |.aux| file
that does not interfere with the |.aux| file(s) of the main document.
This behaviour can be altered by the alternative form
|\childdocby[*]{|\textit{main}|}| (with a non-empty optional argument)
which uses the |.aux| file of the main document
by setting |\jobname| to \textit{main}.

%%%%%%%%%%%%%%%%%%%%%%%%%%%%%%%%%%%%%%%%%%%%%%%%%%%%%%%%%%%%%%%%%%%%%%%%%%%%%%%%
\subsection{Driver Development}
\label{sec:driver}

The \textsf{childdoc} mechanism can also be use for the development
of definition files such as \LaTeX{} styles or classes.
This case differs from the above setup with multiple parts
included by |\include| in that no |\includeonly| should be invoked.
This can be achieved by starting the include file
(before |\ProvidesPackage|) with:
%
\begin{center}
\begin{tabular}{l}
|\input{childdoc.def}|\\
|\childdocforward{|\textit{main}|}|\\
\end{tabular}
\end{center}
%
or alternatively with:
%
\begin{center}
\begin{tabular}{l}
|\input{childdoc.def}|\\
|\childdocby{|\textit{main}|}|\\
\end{tabular}
\end{center}
%
Both forms have slightly different effects as described above.
The main file is prepared as usual, see \secref{sec:include}.

%%%%%%%%%%%%%%%%%%%%%%%%%%%%%%%%%%%%%%%%%%%%%%%%%%%%%%%%%%%%%%%%%%%%%%%%%%%%%%%%
\subsection{Legacy Detection}
\label{sec:detection}

The directive |\childdocmain| in the main file can detect
whether the complete document or merely a child is to be compiled
even without using the directive |\childdocof|.
This method is deprecated because it is less robust
and there is no compelling reason to use it;
it is merely provided for backward compatibility
and it may be removed in future versions.

If the detection mechanism is to be used,
it is mandatory to correctly specify
the filename of the main file as the argument of |\childdocmain|:
%
\begin{center}
\begin{tabular}{l}
|\input{childdoc.def}|\\
|\childdocmain{|\textit{main}|}|\\
\end{tabular}
\end{center}
%
If |\jobname| does not match the argument \textit{main} of |\childdocmain|,
it is assumed that |\jobname| points to the child file to be compiled.
When using |\childdocmain| with the main file specified as argument,
it suffices to start a child file
with just |\input{|\textit{main}|}|
without loading of the package and using |\childdocof|.
If instead all processing is done
with the appropriate \textsf{childdoc} directives,
the argument of \textit{main} of |\childdocmain| can be empty.

An alternative version of the command line processing described
in \secref{sec:commandline} using the detection mechanism reads:
%
\begin{center}
|... -jobname "|\textit{target}|" "|[\textit{flags}]%
[|\def\jobname{|\textit{dest}|}|]|\input{|\textit{main}|}"|
\end{center}

%%%%%%%%%%%%%%%%%%%%%%%%%%%%%%%%%%%%%%%%%%%%%%%%%%%%%%%%%%%%%%%%%%%%%%%%%%%%%%%%
\subsection{Manual Code}
\label{sec:manual}

In case one cannot be certain whether the definitions file |childdoc.def|
is installed on the target \TeX{} distribution
and one prefers not to ship it,
it is conceivable to paste a few relevant commands into the sources.

To that end, drop all statements |\input{childdoc.def}|
and perform the replacements as outlined below.
Instead of |\childdocmain{|\textit{main}|}| add the following code
to the top of the main file:
%
\begin{center}
\begin{tabular}{l}
|\||ifdefined\childdocname\endinput\||fi\newif\ifchilddoc|\\
|\edef\childdocname{\scantokens\expandafter{\jobname\noexpand}}|\\
|\def\childdocmain{|\textit{main}|}\||ifx\childdocmain\childdocname\||else|\\
|\childdoctrue\includeonly{\childdocname}\let\jobname\childdocmain\||fi|\\
\end{tabular}
\end{center}
%
Instead of |\childdocof{|\textit{main}|}| just include the main file
at the top of each child file:
%
\begin{center}
|\input{|\textit{main}|}|
\end{center}
%
A simple redirection |\childdocforward{|\textit{dest}|}| is achieved by:
%
\begin{center}
|\def\jobname{|\textit{dest}|}\input{\jobname}|
\end{center}
%
The redirection with prefix
|\childdocforwardprefix[|\textit{prefix}|]{|\textit{dest}|}|
is accomplished by:
%
\begin{center}
\begin{tabular}{l}
|{\edef\jobname{\scantokens\expandafter{\jobname\noexpand}}|\\
|\def\redirectjob |\textit{prefix}|#1~~~{\gdef\jobname{|\textit{dest}|#1}}|\\
|\expandafter\redirectjob\jobname~~~}\input{\jobname}|
\end{tabular}
\end{center}

In an alternative approach,
child documents can be compiled by a specific command line
without additional code or specific definitions:
%
\begin{center}
|... -jobname "|\textit{target}|" "|[\textit{flags}]%
|\includeonly{|\textit{dest}|}\input{|\textit{main}|}"|
\end{center}
%

%%%%%%%%%%%%%%%%%%%%%%%%%%%%%%%%%%%%%%%%%%%%%%%%%%%%%%%%%%%%%%%%%%%%%%%%%%%%%%%%
%%%%%%%%%%%%%%%%%%%%%%%%%%%%%%%%%%%%%%%%%%%%%%%%%%%%%%%%%%%%%%%%%%%%%%%%%%%%%%%%
\section{Information}

%%%%%%%%%%%%%%%%%%%%%%%%%%%%%%%%%%%%%%%%%%%%%%%%%%%%%%%%%%%%%%%%%%%%%%%%%%%%%%%%
\subsection{Copyright}

Copyright \copyright{} 2017--2018 Niklas Beisert

This work may be distributed and/or modified under the
conditions of the \LaTeX{} Project Public License, either version 1.3
of this license or (at your option) any later version.
The latest version of this license is in
  \url{http://www.latex-project.org/lppl.txt}
and version 1.3 or later is part of all distributions of \LaTeX{}
version 2005/12/01 or later.

This work has the LPPL maintenance status `maintained'.

The Current Maintainer of this work is Niklas Beisert.

This work consists of the files |README.txt|, |childdoc.ins| and |childdoc.dtx|
as well as the derived files |childdoc.def|, |cdocsamp.tex|
with |cdocsch1.tex|, |cdocsch2.tex|, |cdocspt3.tex|, |cdocspt4.tex|,
|cdocsdrf.tex|, |cdocsfn1.tex|, |cdocsfn2.tex|
as well as |childdoc.pdf|.

%%%%%%%%%%%%%%%%%%%%%%%%%%%%%%%%%%%%%%%%%%%%%%%%%%%%%%%%%%%%%%%%%%%%%%%%%%%%%%%%
\subsection{Files and Installation}

The package consists of the files:
%
\begin{center}
\begin{tabular}{ll}
    |README.txt|   & readme file \\
    |childdoc.ins| & installation file \\
    |childdoc.dtx| & source file \\
    |childdoc.def| & definition file \\
    |cdocsamp.tex| & sample main file \\
    |cdocsch1.tex| & sample include file \\
    |cdocsch2.tex| & sample include file \\
    |cdocspt3.tex| & sample part file \\
    |cdocspt4.tex| & sample part file \\
    |cdocsdrf.tex| & sample redirection file \\
    |cdocsfn1.tex| & sample redirection file \\
    |cdocsfn2.tex| & sample redirection file \\
    |childdoc.pdf| & manual
\end{tabular}
\end{center}
%
The distribution consists of the files
|README.txt|, |childdoc.ins| and |childdoc.dtx|.
%
\begin{itemize}
\item
Run (pdf)\LaTeX{} on |childdoc.dtx|
to compile the manual |childdoc.pdf| (this file).
\item
Run \LaTeX{} on |childdoc.ins| to create the definitions file |childdoc.def|
and the sample |cdocsamp.tex| with include files
|cdocsch1.tex|, |cdocsch2.tex|, |cdocspt3.tex|, |cdocspt4.tex|,
|cdocsdrf.tex|, |cdocsfn1.tex|, |cdocsfn2.tex|.
Then copy the file |childdoc.def| to an appropriate directory of your \LaTeX{}
distribution, e.g.\ \textit{texmf-root}|/tex/latex/childdoc|.
\end{itemize}

%%%%%%%%%%%%%%%%%%%%%%%%%%%%%%%%%%%%%%%%%%%%%%%%%%%%%%%%%%%%%%%%%%%%%%%%%%%%%%%%
\subsection{Related CTAN Packages}

There are several other packages which offer a similar functionality:
%
\begin{itemize}
\item
The packages
\href{http://ctan.org/pkg/docmute}{\textsf{docmute}},
\href{http://ctan.org/pkg/includex}{\textsf{includex}} and
\href{http://ctan.org/pkg/standalone}{\textsf{standalone}}
provide commands to include only the document body of
a child file thus allowing both files to be compiled individually.
\item
The packages \href{http://ctan.org/pkg/subdocs}{\textsf{subdocs}}
and \href{http://ctan.org/pkg/subfiles}{\textsf{subfiles}}
provide structures in which the main and child documents can be
encapsulated and allowing them to be compiled individually.
The inclusion mechanism is different from the conventional |\include|.
\item
The package \href{http://ctan.org/pkg/combine}{\textsf{combine}}
is an elaborate solution to combine several documents into one.
\end{itemize}
%
See also the CTAN topic \href{http://ctan.org/topic/subdocs}{\textsf{subdocs}}
for further related packages.
The present package differs from the above solutions in that
a document structure constructed with the conventional |\include| mechanism
just needs two extra commands at the top of every file
such that all constituent files can be compiled individually.

%%%%%%%%%%%%%%%%%%%%%%%%%%%%%%%%%%%%%%%%%%%%%%%%%%%%%%%%%%%%%%%%%%%%%%%%%%%%%%%%
%\subsection{Feature Suggestions}
%
%The following is a list of features which may be useful for future
%versions of this package:
%%
%\begin{itemize}
%\item
%\ldots
%\end{itemize}

%%%%%%%%%%%%%%%%%%%%%%%%%%%%%%%%%%%%%%%%%%%%%%%%%%%%%%%%%%%%%%%%%%%%%%%%%%%%%%%%
\subsection{Revision History}

%%%%%%%%%%%%%%%%%%%%%%%%%%%%%%%%%%%%%%%%
\paragraph{v2.0:} 2018/12/30

\begin{itemize}
\item
immediate forward processing
\item
added |\childdocby| mechanism
\item
manual restructured
\end{itemize}

%%%%%%%%%%%%%%%%%%%%%%%%%%%%%%%%%%%%%%%%
\paragraph{v1.6:} 2018/01/17

\begin{itemize}
\item
application for development of include files
\item
corrections to manual
\end{itemize}

%%%%%%%%%%%%%%%%%%%%%%%%%%%%%%%%%%%%%%%%
\paragraph{v1.5:} 2017/05/21

\begin{itemize}
\item
more complete structuring introduced
\item
|\childdocof| introduced
\item
|\childdoc| renamed to |\childdocmain|
\item
|\childredirect| renamed to |\childdocforward| and |\childdocforwardprefix|
and functionality expanded
\end{itemize}

%%%%%%%%%%%%%%%%%%%%%%%%%%%%%%%%%%%%%%%%
\paragraph{v1.0:} 2017/04/27

\begin{itemize}
\item
manual and install package
\item
first version published on CTAN
\end{itemize}

%%%%%%%%%%%%%%%%%%%%%%%%%%%%%%%%%%%%%%%%
\paragraph{v0.6:} 2017/04/26

\begin{itemize}
\item
redirection mechanism added
\end{itemize}

%%%%%%%%%%%%%%%%%%%%%%%%%%%%%%%%%%%%%%%%
\paragraph{v0.5:} 2017/04/26

\begin{itemize}
\item
functionality in definition file
\end{itemize}


%%%%%%%%%%%%%%%%%%%%%%%%%%%%%%%%%%%%%%%%%%%%%%%%%%%%%%%%%%%%%%%%%%%%%%%%%%%%%%%%
%%%%%%%%%%%%%%%%%%%%%%%%%%%%%%%%%%%%%%%%%%%%%%%%%%%%%%%%%%%%%%%%%%%%%%%%%%%%%%%%
%%%%%%%%%%%%%%%%%%%%%%%%%%%%%%%%%%%%%%%%%%%%%%%%%%%%%%%%%%%%%%%%%%%%%%%%%%%%%%%%
\appendix

\settowidth\MacroIndent{\rmfamily\scriptsize 000\ }

 \DocInput{childdoc.dtx}

\end{document}
%</driver>
% \fi
%
% %%%%%%%%%%%%%%%%%%%%%%%%%%%%%%%%%%%%%%%%%%%%%%%%%%%%%%%%%%%%%%%%%%%%%%%%%%%%%%
% %%%%%%%%%%%%%%%%%%%%%%%%%%%%%%%%%%%%%%%%%%%%%%%%%%%%%%%%%%%%%%%%%%%%%%%%%%%%%%
% \section{Sample}
%\iffalse
%<*samplemain>
%\fi
%
% The following presents a sample document
% with two chapters, two parts, a title page,
% a compile flag as well as three forwarding files to set the flag.
% It consists of eight |.tex| files:
% \begin{center}
% \begin{tabular}{ll}
% |cdocsamp.tex|&main file\\
% |cdocsch1.tex|&include file for chapter 1\\
% |cdocsch2.tex|&include file for chapter 2\\
% |cdocspt3.tex|&include file for part 3\\
% |cdocspt4.tex|&include file for part 4\\
% |cdocsdrf.tex|&forwarding file for main file in draft mode\\
% |cdocsfi1.tex|&forwarding file for final version of chapter 1\\
% |cdocsfi2.tex|&forwarding file for final version of chapter 2\\
% \end{tabular}
% \end{center}
% Each of the eight files can be compiled directly by the \LaTeX{} compiler.
%
% %%%%%%%%%%%%%%%%%%%%%%%%%%%%%%%%%%%%%%
% \paragraph{Main File.}
%
% The main file is called |cdocsamp.tex|.
%
% Load the \textsf{childdoc} definitions and
% declare the filename for the main document:
%    \begin{macrocode}
\input{childdoc.def}
\childdocmain{}
%    \end{macrocode}

% Optional override for |\version| flag:
%    \begin{macrocode}
%%\ifchilddoc\else\providecommand{\version}{draft}\fi
%    \end{macrocode}

% Define the default values for the |\version| flag
% (|final| for the main file and |draft| for childs):
%    \begin{macrocode}
\ifchilddoc
\providecommand{\version}{draft}
\else
\providecommand{\version}{final}
\fi
%    \end{macrocode}

% Load the standard document class:
%    \begin{macrocode}
\documentclass[12pt]{article}
%    \end{macrocode}

% Start the document body:
%    \begin{macrocode}
\begin{document}
%    \end{macrocode}

% Declare a title page.
% Print title, part of document being processed and version flag:
%    \begin{macrocode}
\addtocounter{page}{-1}
\begin{center}
{\LARGE\bfseries{}childdoc example\par}
\vspace{1cm}
\ifchilddoc
\ifchilddocmanual part\else chapter\fi:
`\childdocname' of `\childdocjob'\par
\else
main document: `\childdocjob'\par
\fi
version: \version\par
\end{center}
\newpage
%    \end{macrocode}

% Manually include selected file,
% otherwise process as usual:
%    \begin{macrocode}
\ifchilddocmanual
\section*{part `\childdocname'}
\input{\childdocname}
\else
%    \end{macrocode}

% Include the two chapters:
%    \begin{macrocode}
\include{cdocsch1}
\include{cdocsch2}
%    \end{macrocode}

% Include the two parts unless only chapters should be displayed:
%    \begin{macrocode}
\ifchilddoc\else
\section{part three}
\input{cdocspt3}
\section{part four}
\input{cdocspt4}
\fi
%    \end{macrocode}

% Process as usual until here:
%    \begin{macrocode}
\fi
%    \end{macrocode}

% End of document body:
%    \begin{macrocode}
\end{document}
%    \end{macrocode}
%\iffalse
%</samplemain>
%\fi
%
% %%%%%%%%%%%%%%%%%%%%%%%%%%%%%%%%%%%%%%
% \paragraph{Chapter Include Files.}
%
% The include files are called |cdocsch1.tex| and |cdocsch2.tex|.
%
%\iffalse
%<*samplechap1|samplechap2>
%\fi

% Optional override for |\version| flag:
%    \begin{macrocode}
%%\providecommand{\version}{final}
%    \end{macrocode}

% Include the main document:
%    \begin{macrocode}
\input{childdoc.def}
\childdocof{cdocsamp}
%    \end{macrocode}

%\iffalse
%</samplechap1|samplechap2>
%\fi
%
%\iffalse
%<*samplechap1>
%\fi
% Some text for chapter 1:
%    \begin{macrocode}
\section{one}
some text in chapter one
%    \end{macrocode}

%\iffalse
%</samplechap1>
%\fi
% Some text for chapter 2:
%\iffalse
%<*samplechap2>
%\fi
%    \begin{macrocode}
\section{two}
more text in chapter two
%    \end{macrocode}

%\iffalse
%</samplechap2>
%\fi
%
% %%%%%%%%%%%%%%%%%%%%%%%%%%%%%%%%%%%%%%
% \paragraph{Part Include Files.}
%
% The include files are called |cdocspt3.tex| and |cdocspt4.tex|.
%
%\iffalse
%<*samplepart3|samplepart4>
%\fi

% Optional override for |\version| flag:
%    \begin{macrocode}
%%\providecommand{\version}{final}
%    \end{macrocode}

% Include the main document:
%    \begin{macrocode}
\input{childdoc.def}
\childdocby{cdocsamp}
%    \end{macrocode}

%\iffalse
%</samplepart3|samplepart4>
%\fi
%
%\iffalse
%<*samplepart3>
%\fi
% Some text for part 3:
%    \begin{macrocode}
some text in part three
%    \end{macrocode}

%\iffalse
%</samplepart3>
%\fi
% Some text for part 4:
%\iffalse
%<*samplepart4>
%\fi
%    \begin{macrocode}
more text in part four
%    \end{macrocode}

%\iffalse
%</samplepart4>
%\fi
%
% %%%%%%%%%%%%%%%%%%%%%%%%%%%%%%%%%%%%%%
% \paragraph{Forwarding for a Complete Draft.}
%
% The following forwarding file |cdocsdrf.tex|
% compiles the main document in draft mode:
%\iffalse
%<*sampledraft>
%\fi
%    \begin{macrocode}
\def\version{draft}
\input{childdoc.def}
\childdocforward{cdocsamp}
%    \end{macrocode}

%\iffalse
%</sampledraft>
%\fi
%
% %%%%%%%%%%%%%%%%%%%%%%%%%%%%%%%%%%%%%%
% \paragraph{Forwarding for Final Version of the Chapters.}
%
% The following forwarding files |cdocsfn1.tex| and |cdocsfn2.tex|
% (with identical content)
% compile the final versions of the child documents
% |cdocsch1.tex| and |cdocsch2.tex|, respectively:
%\iffalse
%<*samplefinal>
%\fi
%    \begin{macrocode}
\def\version{final}
\input{childdoc.def}
\childdocforwardprefix[cdocsamp]{cdocsfn}{cdocsch}
%    \end{macrocode}

%\iffalse
%</samplefinal>
%\fi
%
% %%%%%%%%%%%%%%%%%%%%%%%%%%%%%%%%%%%%%%
% \paragraph{Command Line Processing.}
%
% The following three command lines generate the output files
% |cdocscld|, |cdocscl1| and |cdocscl2|
% which should be identical to
% |cdocsdrf|, |cdocsch1| and |cdocsfn2|, respectively:
% \begin{center}
% \begin{tabular}{l}
% |latex -jobname cdocscld \|\\
% |  "\def\version{draft}\input{childdoc.def}\childdocforward{cdocsamp}"|\\
% |latex -jobname cdocscl1 \|\\
% |  "\input{childdoc.def}\childdocforward[cdocsamp]{cdocsch1}"|\\
% |latex -jobname cdocscl2 \|\\
% |  "\def\version{final}\input{childdoc.def}\childdocforward{cdocsch2}"|
% \end{tabular}
% \end{center}
% Note that the trailing backslash on each first line
% merely continues the input to the second line
% (for convenient cut ant paste).
% Furthermore, the command |latex| can be replaced by any
% of its alternative versions such as |pdflatex|.
%
% %%%%%%%%%%%%%%%%%%%%%%%%%%%%%%%%%%%%%%%%%%%%%%%%%%%%%%%%%%%%%%%%%%%%%%%%%%%%%%
% %%%%%%%%%%%%%%%%%%%%%%%%%%%%%%%%%%%%%%%%%%%%%%%%%%%%%%%%%%%%%%%%%%%%%%%%%%%%%%
% \section{Implementation}
%\iffalse
%<*package>
%\fi
%
% This section describes the definitions file |childdoc.def|.

% The definitions cannot be loaded using |\usepackage| or |\RequirePackage|
% which has a mechanism to prevent loading a style file more than once.
% When loading the definitions by means of |\input|
% multiple instances have to be prevented manually:
%\iffalse
%This code needs to be before the `\ProvidesFile' directive
%which is defined at the beginning of this file.
%Therefore it is also placed there and commented out here.
%</package>
%<*discard>
%\fi
%    \begin{macrocode}
\ifdefined\childdocmain\endinput\fi
%    \end{macrocode}
%\iffalse
%</discard>
%<*package>
%\fi
%
% \macro{\ifchilddoc}
% \macro{\ifchilddocmanual}
% The conditional |\ifchilddoc| tells whether a
% child (true) or main (false) document is being compiled.
% The conditional |\ifchilddocmanual| tells whether
% the |\includeonly| mechanism is used (false) or
% the selection of child files must be performed manually (true).
% The definitions initialise to false:
%    \begin{macrocode}
\newif\ifchilddoc
\newif\ifchilddocmanual
%    \end{macrocode}

% \macro{\childdocname}
% \macro{\childdocjob}
% The macro |\childdocname| stores the name of the main document
% to be compiled. The macro |\childdocjob| stores the name of
% the document on which the \LaTeX{} compiler was originally invoked.
% The content of |\jobname| cannot be compared
% to filenames specified in the source due to different catcodes.
% The following code rescans |\jobname|, stores the result
% in |\childdocname| and saves a copy in |\childdocjob|:
%    \begin{macrocode}
\edef\childdocname{\scantokens\expandafter{\jobname\noexpand}}
\let\childdocjob\childdocname
%    \end{macrocode}

% \macro{\childdocdisable}
% The macro |\childdocdisable| prevents the main file
% from being processed more than once.
% At this stage, the main document command |\childdocmain|
% is assumed to be called once again where it should do nothing.
% Any subsequent call to it should prevent
% a secondary processing of the main document
% It overwrites the forwarding commands
% |\childdocof| and |\childdocforward|
% with empty macros to prevent further inclusions of the main document:
%    \begin{macrocode}
\newcommand{\childdocdisable}
{
  \renewcommand{\childdocmain}[1]{\renewcommand{\childdocmain}[1]{\endinput}}
  \renewcommand{\childdocof}[1]{}
  \renewcommand{\childdocby}[2][]{}
  \renewcommand{\childdocforward}[2][]{}
  \renewcommand{\childdocdisable}{}
}
%    \end{macrocode}

% \macro{\childdocmain}
% The macro |\childdocmain| is to be called at the top of the main file
% with nothing or the main filename (without extension) as argument.
% First, it breaks loops.
% If the argument is not empty and does not match |\childdocname|
% (which is set by the first inclusion of |childdoc.def|),
% |\ifchilddoc| is set to true, |\includeonly| is applied to the child file
% and |\jobname| is set to the main file
% (for proper handling of |.aux| files):
%    \begin{macrocode}
\newcommand{\childdocmain}[1]
{
  \childdocdisable\childdocmain{}
  \if?#1?\else
    \begingroup
      \def\childdoctmp{#1}
      \ifx\childdoctmp\childdocname
        \def\childdoctmp{}
      \else
        \def\childdoctmp
        {
          \childdoctrue
          \includeonly{\childdocname}
          \def\childdocjob{#1}
          \def\jobname{#1}
        }
      \fi
      \expandafter
    \endgroup
    \childdoctmp
  \fi
}
%    \end{macrocode}

% \macro{\childdocof}
% The command |\childdocof| redirects
% compilation to the main file |#1|.
%    \begin{macrocode}
\newcommand{\childdocof}[1]
{
  \childdocdisable
  \childdoctrue
  \includeonly{\childdocname}
  \def\jobname{#1}
  \def\childdocjob{#1}
  \input{#1}
}
%    \end{macrocode}

% \macro{\childdocby}
% The command |\childdocby| ....
%    \begin{macrocode}
\newcommand{\childdocby}[2][]
{
  \childdocdisable
  \childdoctrue
  \childdocmanualtrue
  \if?#1?\else
    \def\jobname{#2}
  \fi
  \def\childdocjob{#2}
  \input{#2}
  \endinput
}
%    \end{macrocode}

% \macro{\childdocforward}
% The command |\childdocforward| redirects
% compilation to the main file or
% (if the optional argument is given) a child file.
% Parameters are set as if the main file
% or a child file starting with |\childdocof| was compiled.
% Then compilation is handed over to the main file:
%    \begin{macrocode}
\newcommand{\childdocforward}[2][]
{
  \begingroup
    \if?#1?
      \def\childdoctmp
      {
        \def\childdocname{#2}
        \def\childdocjob{#2}
        \def\jobname{#2}
        \input{#2}
        \endinput
      }
    \else
      \def\childdoctmp
      {
        \childdocdisable
        \def\childdocname{#2}
        \childdoctrue
        \includeonly{#2}
        \def\childdocjob{#1}
        \def\jobname{#1}
        \input{#1}
        \endinput
      }
    \fi
    \expandafter
  \endgroup
  \childdoctmp
}
%    \end{macrocode}

% \macro{\childdocforwardprefix}
% The command |\childdocforwardprefix| redirects
% compilation to the main or a child file by means of a pattern.
% The prefix |#1| in the current filename is replaced by |#2|
% and the suffix of the current filename is kept
% (it is assumed that the filename does not contain the substring `|~~~|'
% which is used as a delimiter).
% Compilation is handed over to the new file by |\childdocforward|:
%    \begin{macrocode}
\newcommand{\childdocforwardprefix}[3][]
{
  \begingroup
    \def\childdocextract #2##1~~~{\def\childdoctmp{\childdocforward[#1]{#3##1}}}
    \expandafter\childdocextract\childdocname~~~
    \expandafter
  \endgroup
  \childdoctmp
}
%    \end{macrocode}

% \macro{\childdoc}
% The deprecated macro |\childdoc| is a legacy version of |\childdocmain|:
%    \begin{macrocode}
\newcommand{\childdoc}{\childdocmain}
%    \end{macrocode}

% \macro{\childdocredirect}
% The deprecated macro |\childdocredirect| is a legacy version
% of |\childdocforward| and |\childdocforwardprefix|:
%    \begin{macrocode}
\newcommand{\childdocredirect}[2][]
{
  \begingroup
    \if?#1?
      \def\childdoctmp{\childdocforward{#2}}
    \else
      \def\childdoctmp{\childdocforwardprefix{#1}{#2}}
    \fi
    \expandafter
  \endgroup
  \childdoctmp
}
%    \end{macrocode}

%\iffalse
%</package>
%\fi
%
\endinput

\childdocforward{cdocsamp}
%    \end{macrocode}

%\iffalse
%</sampledraft>
%\fi
%
% %%%%%%%%%%%%%%%%%%%%%%%%%%%%%%%%%%%%%%
% \paragraph{Forwarding for Final Version of the Chapters.}
%
% The following forwarding files |cdocsfn1.tex| and |cdocsfn2.tex|
% (with identical content)
% compile the final versions of the child documents
% |cdocsch1.tex| and |cdocsch2.tex|, respectively:
%\iffalse
%<*samplefinal>
%\fi
%    \begin{macrocode}
\def\version{final}
% \iffalse
%
% childdoc.dtx Copyright (C) 2017-2018 Niklas Beisert
%
% This work may be distributed and/or modified under the
% conditions of the LaTeX Project Public License, either version 1.3
% of this license or (at your option) any later version.
% The latest version of this license is in
%   http://www.latex-project.org/lppl.txt
% and version 1.3 or later is part of all distributions of LaTeX
% version 2005/12/01 or later.
%
% This work has the LPPL maintenance status `maintained'.
%
% The Current Maintainer of this work is Niklas Beisert.
%
% This work consists of the files childdoc.dtx and childdoc.ins
% and the derived files childdoc.def and cdocsamp.tex with
% cdocsch1.tex, cdocsch2.tex, cdocsdrf.tex, cdocsfn1.tex, cdocsfn2.tex.
%
%<package>\ifdefined\childdocmain\endinput\fi
%<package>\ProvidesFile{childdoc.def}[2018/12/30 v2.0 child document driver]
%<samplemain>\ProvidesFile{cdocsamp.tex}[2018/12/30 v2.0 sample for childdoc]
%<*driver>
%\ProvidesFile{childdoc.drv}[2018/12/30 v2.0 childdoc reference manual file]
\PassOptionsToClass{10pt,a4paper}{article}
\documentclass{ltxdoc}

\usepackage[margin=35mm]{geometry}
\usepackage{hyperref}
\usepackage{hyperxmp}
\usepackage[usenames]{color}

\hypersetup{colorlinks=true}
\hypersetup{pdfstartview=FitH}
\hypersetup{pdfpagemode=UseNone}
\hypersetup{pdfsource={}}
\hypersetup{pdflang={en-UK}}
\hypersetup{pdfcopyright={Copyright 2017-2018 Niklas Beisert.
  This work may be distributed and/or modified under the
  conditions of the LaTeX Project Public License, either version 1.3
  of this license or (at your option) any later version.}}
\hypersetup{pdflicenseurl={http://www.latex-project.org/lppl.txt}}
\hypersetup{pdfcontactaddress={ETH Zurich, ITP, HIT K,
  Wolfgang-Pauli-Strasse 27}}
\hypersetup{pdfcontactpostcode={8093}}
\hypersetup{pdfcontactcity={Zurich}}
\hypersetup{pdfcontactcountry={Switzerland}}
\hypersetup{pdfcontactemail={nbeisert@itp.phys.ethz.ch}}
\hypersetup{pdfcontacturl={http://people.phys.ethz.ch/\xmptilde nbeisert/}}

\newcommand{\secref}[1]{\hyperref[#1]{section \ref*{#1}}}

\parskip1ex
\parindent0pt
\let\olditemize\itemize
\def\itemize{\olditemize\parskip0pt}

\begin{document}

\title{The \textsf{childdoc} Package}
\hypersetup{pdftitle={The childdoc Package}}
\author{Niklas Beisert\\[2ex]
  Institut f\"ur Theoretische Physik\\
  Eidgen\"ossische Technische Hochschule Z\"urich\\
  Wolfgang-Pauli-Strasse 27, 8093 Z\"urich, Switzerland\\[1ex]
  \href{mailto:nbeisert@itp.phys.ethz.ch}
  {\texttt{nbeisert@itp.phys.ethz.ch}}}
\hypersetup{pdfauthor={Niklas Beisert}}
\hypersetup{pdfsubject={Manual for the LaTeX2e Package childdoc}}
\date{30 December 2018, \textsf{v2.0}}
\maketitle

\begin{abstract}\noindent
\textsf{childdoc} is a \LaTeXe{} package
that enables the direct compilation
of document sections included by |\include|
to individual files.
\end{abstract}

\begingroup
\parskip0ex
\tableofcontents
\endgroup

%%%%%%%%%%%%%%%%%%%%%%%%%%%%%%%%%%%%%%%%%%%%%%%%%%%%%%%%%%%%%%%%%%%%%%%%%%%%%%%%
%%%%%%%%%%%%%%%%%%%%%%%%%%%%%%%%%%%%%%%%%%%%%%%%%%%%%%%%%%%%%%%%%%%%%%%%%%%%%%%%
\section{Introduction}

\LaTeX{} provides a mechanism to structure a large document (such as a book)
into a main file and several child files (containing the chapters)
using the |\include| command.
This mechanism is beneficial for documents
which span hundreds of pages in order to
make the source file(s) more manageable.
Moreover, compilation can be restricted to
selected child files by means of the |\includeonly| command.
The latter feature can be used to reduce the compilation time while editing
(this was significantly more useful in the earlier days of \LaTeX{})
or to generate a smaller document which is easier to navigate.
Another application of |\includeonly| is to generate
documents consisting of selected parts of the complete document.

However, there are a few drawbacks of the plain |\include| mechanism:
\begin{itemize}
\item
The child files cannot be compiled on their own,
they can only be compiled via the main file.
A naive editing environment
(such as a text editor with an option
to have the current file processed by \LaTeX)
may require one to switch to the main file before compiling;
attempting to compile the child file produces errors.
\item
The main file must be modified (each time)
to adjust the |\includeonly| command
to the present needs. This easily leaves the main file in a messy state.
\item
The generated document will always carry the filename
of the main document. This is inconvenient if
several child files are to be compiled and
to be kept for distribution.
\end{itemize}

The present package provides a simple interface
to make child files individually compilable by \LaTeX{}.
Compiling a child file then has the same effect as compiling
the main file with an |\includeonly| command
to select the appropriate child.
Moreover the generated document will carry the name of the child
rather than the main file.
This resolves all three above issues.

This feature is meant to make the editing of books,
thesis documents and lecture notes somewhat more convenient.
However, the package can also be used efficiently for
composing a series of documents (such as exercise sheets)
which are typically distributed individually.
It then assists the author in generating the individual documents
(potentially in different versions)
as well as a document containing the collected series.
Another application is in developing style files
or other kinds of included material
where compilation of the style file could redirect
to a sample or test file.

%%%%%%%%%%%%%%%%%%%%%%%%%%%%%%%%%%%%%%%%%%%%%%%%%%%%%%%%%%%%%%%%%%%%%%%%%%%%%%%%
%%%%%%%%%%%%%%%%%%%%%%%%%%%%%%%%%%%%%%%%%%%%%%%%%%%%%%%%%%%%%%%%%%%%%%%%%%%%%%%%
\section{Usage}

First of all, the package \textsf{childdoc} is \emph{not} a standard
\LaTeXe{} |.sty| style file! Therefore it needs to be invoked in
a non-standard way.

%%%%%%%%%%%%%%%%%%%%%%%%%%%%%%%%%%%%%%%%%%%%%%%%%%%%%%%%%%%%%%%%%%%%%%%%%%%%%%%%
\subsection{Included Files}
\label{sec:include}

%%%%%%%%%%%%%%%%%%%%%%%%%%%%%%%%%%%%%%%%
\DescribeMacro{\childdocmain}
To use the package, add the commands
\begin{center}
\begin{tabular}{l}
|\input{childdoc.def}|\\
|\childdocmain{}|\\
\end{tabular}
\end{center}
at the very top of the main \LaTeX{} file,
in particular \emph{before} the |\documentclass| statement!
The argument of |\childdocmain| should be left empty
(but it must be present).

%%%%%%%%%%%%%%%%%%%%%%%%%%%%%%%%%%%%%%%%
\DescribeMacro{\childdocof}
Furthermore, add the commands
\begin{center}
\begin{tabular}{l}
|\input{childdoc.def}|\\
|\childdocof{|\textit{main}|}|\\
\end{tabular}
\end{center}
at the top of every child file \textit{child}
which is included by |\include{|\textit{child}|}|
from within the main file
(or at least for those files to be compiled individually).
The argument \textit{main} must be the filename of the main file.

There are a couple of
considerations in setting up the main and child documents:

%%%%%%%%%%%%%%%%%%%%%%%%%%%%%%%%%%%%%%%%
\paragraph{Restrictions.}

Please note the following restrictions:
\begin{itemize}
\item
|\childdocmain| must be called with one argument \textit{main}
to ensure compatibility with earlier version of the package.
It must either be empty (|\childdocmain{}|)
or precisely match the filename of the main file in which it is specified.
See \secref{sec:detection} for further information.
\item
The filename \textit{main} must be specified without the |.tex| extension.
\item
The filename \textit{main} is case sensitive
(even in case-insensitive file systems)
due to internal string comparison.
\item
The argument \textit{main} should be fully expanded, it cannot be a macro.
\item
Subdirectories and special characters should be avoided in filenames.
\item
The command |\childdocmain{|\textit{main}|}| must be followed by a whitespace.
It should not be followed immediately by another command
or by a comment mark `|%|'.
This is because the \TeX{} parser reads the token immediately following
the argument of |\childdocmain| and puts it
at the beginning of every child section;
however, a white\-space is ignored.
\end{itemize}

%%%%%%%%%%%%%%%%%%%%%%%%%%%%%%%%%%%%%%%%
\paragraph{Content of Main File.}

It is advisable to place all content in the child files included by |\include|.
Any output contained in the main file will appear in all child documents
unless suppressed manually;
it cannot be suppressed automatically by the |\includeonly| directive
and thus should normally be avoided.
A method to include some content in the main file
by means of conditional processing is described in \secref{sec:conditional}.

%%%%%%%%%%%%%%%%%%%%%%%%%%%%%%%%%%%%%%%%
\paragraph{Page Numbering.}

When only a part of the document is compiled,
the appropriate numbering of pages
(as well as other status parameters)
is determined from the |.aux| files.
The latter contain information from previous passes.
However this information needs to propagate through
all intermediate child documents.
Therefore the page numbering in child documents may well
be inconsistent until the complete document is compiled at least once.

A useful (if unconventional) way to always ensure a consistent
page numbering is to restart the numbering in each child document
and denote the pages by `\textit{child}|.|\textit{page}'
where \textit{child} represents the chapter/section number of the child file.
This can be achieved by the command
|\numberwithin{page}{|\textit{child}|}|
of the \textsf{amsmath} package
where \textit{child} can be |chapter| or |section|
depending on the chosen structuring.
Alternatively, one can modify the macro |\thepage| appropriately
and reset the counter |page| at the start of each child file.

%%%%%%%%%%%%%%%%%%%%%%%%%%%%%%%%%%%%%%%%%%%%%%%%%%%%%%%%%%%%%%%%%%%%%%%%%%%%%%%%
\subsection{Conditional Processing}
\label{sec:conditional}

The package provides a mechanism to compile different versions
of a document. To customise the versions further some conditional processing
can come in handy to distinguish which version is being compiled.
The package provides two macros to describe the compilation context:

%%%%%%%%%%%%%%%%%%%%%%%%%%%%%%%%%%%%%%%%
\DescribeMacro{\ifchilddoc}
The conditional |\ifchilddoc| distinguishes between the compilation of
child documents and the main document:
%
\begin{center}
|\ifchilddoc |\textit{child-code}| |[|\||else |\textit{main-code}]| \||fi|
\end{center}

%%%%%%%%%%%%%%%%%%%%%%%%%%%%%%%%%%%%%%%%
\DescribeMacro{\childdocname}
\DescribeMacro{\childdocjob}
The macro |\childdocname| contains the filename (without extension)
of the main or child file being processed.
Note that |\childdocjob| will always contain the name of the main file.

%%%%%%%%%%%%%%%%%%%%%%%%%%%%%%%%%%%%%%%%
\paragraph{Title Page.}

Conditional processing can be used to include a title or banner page
in the main document when proper precautions are taken.
Importantly, the code in the main file should ensure that the page counter
(as well as other status parameters which are stored in the |.aux| files)
takes the same value after the conditional processing.
Otherwise the page numbers may take divergent values
depending on which part is compiled.

For example, a title page could be declared by:
%
\begin{center}
\begin{tabular}{l}
|\ifchilddoc\||else|\\
|\addtocounter{page}{-1}|\\
\textit{code for title page}\\
|\newpage|\\
|\||fi|
\end{tabular}
\end{center}
%
A banner page for the child documents can be generated by:
%
\begin{center}
\begin{tabular}{l}
|\ifchilddoc|\\
|\addtocounter{page}{-1}|\\
\textit{code for banner page}\\
|\newpage|\\
|\||fi|
\end{tabular}
\end{center}
%
Here one could write a message such as:
\begin{center}
|This is the part \childdocname{} of \childdocjob{}.|
\end{center}

%%%%%%%%%%%%%%%%%%%%%%%%%%%%%%%%%%%%%%%%%%%%%%%%%%%%%%%%%%%%%%%%%%%%%%%%%%%%%%%%
\subsection{Flags}
\label{sec:flags}

The package makes it easy to generate different versions
of the main or child documents.
To this end compilation flags can be defined
and assigned different default values.
They will be particularly useful in conjunction
with the forwarding mechanism described in \secref{sec:forward}.

For example, it may be useful to have a flag |\version|
which can be set to |draft| or |final|.
The document source will contain some conditional code
depending on the value of |\version|.
Suppose further, the flag should default to |final| for the main file
and to |draft| for child files
which is a natural assignment for editing the document.
This is achieved by placing the following code
in the preamble of the main document
(below the |\childdocmain| directive):
%
\begin{center}
\begin{tabular}{l}
|\ifchilddoc|\\
|\providecommand{\version}{draft}|\\
|\||else|\\
|\providecommand{\version}{final}|\\
|\||fi|
\end{tabular}
\end{center}
%
The definition by |\providecommand| makes sure
that previous definitions are not overwritten.
Further statements |\providecommand{\version}{...}|
can thus be added before the above code to override it.

For the main file, one might add a line
(between |\childdocmain| and the above block)
%
\begin{center}
|%\ifchilddoc\||else\providecommand{\version}{draft}\||fi|
\end{center}
%
which can be uncommented to produce a draft version.
Likewise one can add a line to the very top of a child file
(above the |\childdocof{|\textit{main}|}| directive)
%
\begin{center}
|%\providecommand{\version}{final}|
\end{center}
%
which can be uncommented to produce the final version of this child document.

%%%%%%%%%%%%%%%%%%%%%%%%%%%%%%%%%%%%%%%%%%%%%%%%%%%%%%%%%%%%%%%%%%%%%%%%%%%%%%%%
\subsection{Forwarding}
\label{sec:forward}

Different versions of the main or child documents
using compilation flags as described in \secref{sec:flags}
can be (permanently) stored in different files
for convenient compilation, viewing and distribution.
To this end, the package defines a command
to pass on compilation to a different file:

%%%%%%%%%%%%%%%%%%%%%%%%%%%%%%%%%%%%%%%%
\DescribeMacro{\childdocforward}
The command |\childdocforward| redirects processing to
another source file:
%
\begin{center}
\begin{tabular}{l}
|\input{childdoc.def}|\\
|\childdocforward[|\textit{main}|]{|\textit{dest}|}|\\
\end{tabular}
\end{center}
%
The argument \textit{dest} is the destination file
(without extension).
It should be the main file or one of the child files.
Note that further \textsf{childdoc} directives
such as |\childdocof| and |\childdocforward|
in the indicated file will be processed in this form.
The optional argument \textit{main}
passes on directly to the main file \textit{main}
while pretending to compile the child \textit{dest}.
This form behaves as if \textit{dest}
issues |\childdocof{|\textit{main}|}| right away,
and no further \textsf{childdoc} directives will be processed.

%%%%%%%%%%%%%%%%%%%%%%%%%%%%%%%%%%%%%%%%
\DescribeMacro{\...prefix}
In the alternative form |\childdocforwardprefix|,
%
\begin{center}
\begin{tabular}{l}
|\input{childdoc.def}|\\
|\childdocforwardprefix[|\textit{main}|]{|\textit{prefix}|}{|\textit{dest}|}|
\end{tabular}
\end{center}
%
the destination file is determined by a pattern
depending on the current file:
To make this work, the current file must be called
`{\textit{prefix}\hspace{0.2em}\textit{suffix}}'
with \textit{prefix} matching precisely the argument.
Processing is then passed on to the file
`{\textit{dest}\hspace{0.2em}\textit{suffix}}'.
Surely, the same effect is achieved by
directly specifying the
argument `{\textit{dest}\hspace{0.2em}\textit{suffix}}'
in the first form.
However, that requires to set up a different file
for each child. With the alternative form of the command
all these files can have exactly the same content
which simplifies setting them up and maintaining them.

For example, the following file |draft.tex|
with a compilation flag |\version| as described in \secref{sec:flags}
compiles the main document as a draft:
%
\begin{center}
\begin{tabular}{l}
|\def\version{draft}|\\
|\input{childdoc.def}|\\
|\childdocforward{|\textit{main}|}|
\end{tabular}
\end{center}
%
Likewise, the following files |final|\textit{nn}|.tex|
compile the final version of the child document
|child|\textit{nn}|.tex|:
%
\begin{center}
\begin{tabular}{l}
|\def\version{final}|\\
|\input{childdoc.def}|\\
|\childdocforwardprefix{final}{child}|
\end{tabular}
\end{center}
%

Note that when several versions of a main file and/or of each child file
are to be generated, it may be convenient to set up a |Makefile| or
shell script to automatise the process.

%%%%%%%%%%%%%%%%%%%%%%%%%%%%%%%%%%%%%%%%%%%%%%%%%%%%%%%%%%%%%%%%%%%%%%%%%%%%%%%%
\subsection{Command Line Processing}
\label{sec:commandline}

The effect of redirection files can also be achieved by invoking
the \LaTeX{} compiler with a more elaborate command line.
Most conveniently this should be done as part
of a shell script or a |Makefile|.

When using \textsf{childdoc} in the main file, the following
command lines effectively perform a redirection
(note that depending on the shell being used,
backslashes may have to be doubled: `|\|' $\to$ `|\\|'):
%
\begin{center}
|... -jobname "|\textit{target}|" |\\|"|[\textit{flags}]%
|\input{childdoc.def}\childdocforward[|\textit{main}|]{|\textit{dest}|}"|
\end{center}
%
Here \textit{target} is the name of the output file,
\textit{main} is the name of the main file
and \textit{dest} is the name of the main or child file to be processed
(all filenames without extensions).
The optional argument \textit{main} can be omitted
if \textit{main} matches \textit{dest}.
Optionally, compilation \textit{flags} can be defined via |\def| commands.
This command line makes the \TeX{} engine believe
it is compiling the file \textit{target}
whose content is specified as the latter parameter.
The provided code then forwards the processing to
\textit{main} or \textit{dest} as described in \secref{sec:forward}.

%%%%%%%%%%%%%%%%%%%%%%%%%%%%%%%%%%%%%%%%%%%%%%%%%%%%%%%%%%%%%%%%%%%%%%%%%%%%%%%%
\subsection{Include by Input}
\label{sec:input}

Including child documents by |\include| has some restrictions by design.
Most notably, the content of a child document always occupies
its own set of pages; pages cannot be shared between child documents.
Usually, this behaviour makes perfect sense
because each child document contain an essential part of the document.
However, in some situations it may be desirable to compose
a document from a collection of parts
without having mandatory page breaks between then.
For this case, the package
provides a mechanism to include parts
by |\input| which can also be processed individually.
However, by construction this mechanism
requires manual handling of the content to be output.

%%%%%%%%%%%%%%%%%%%%%%%%%%%%%%%%%%%%%%%%
\DescribeMacro{\ifchilddocmanual}
The main file should be prepared as usual, see \secref{sec:include}.
However, the document body must make a distinction
between processing of an individual part and of the main document, e.g.:
%
\begin{center}
\begin{tabular}{l}
|\ifchilddocmanual|\\
|\input{\childdocname}|\\
|\||else|\\
\textit{document body with }|\input{|\textit{part}|}|\\
|\||fi|
\end{tabular}
\end{center}
%
The conditional |\ifchilddocmanual| is true whenever
a part to be included by |\input| is being compiled,
and the name of the part is stored in |\childdocname|.

%%%%%%%%%%%%%%%%%%%%%%%%%%%%%%%%%%%%%%%%
\DescribeMacro{\childdocby}
Each part to be included by |\input| should start with:
%
\begin{center}
\begin{tabular}{l}
|\input{childdoc.def}|\\
|\childdocby{|\textit{main}|}|\\
\end{tabular}
\end{center}
%
The directive |\childdocby| is similar to |\childdocof|
described in \secref{sec:include},
but the subsequent selection of content must be done manually.
To that end, both |\ifchilddoc| and |\ifchilddocmanual|
will be true upon processing of a part,
and the name of the part is stored in |\childdocname|.
Note that |\jobname| will be set to the filename of the current part
so that each part receives an individual |.aux| file
that does not interfere with the |.aux| file(s) of the main document.
This behaviour can be altered by the alternative form
|\childdocby[*]{|\textit{main}|}| (with a non-empty optional argument)
which uses the |.aux| file of the main document
by setting |\jobname| to \textit{main}.

%%%%%%%%%%%%%%%%%%%%%%%%%%%%%%%%%%%%%%%%%%%%%%%%%%%%%%%%%%%%%%%%%%%%%%%%%%%%%%%%
\subsection{Driver Development}
\label{sec:driver}

The \textsf{childdoc} mechanism can also be use for the development
of definition files such as \LaTeX{} styles or classes.
This case differs from the above setup with multiple parts
included by |\include| in that no |\includeonly| should be invoked.
This can be achieved by starting the include file
(before |\ProvidesPackage|) with:
%
\begin{center}
\begin{tabular}{l}
|\input{childdoc.def}|\\
|\childdocforward{|\textit{main}|}|\\
\end{tabular}
\end{center}
%
or alternatively with:
%
\begin{center}
\begin{tabular}{l}
|\input{childdoc.def}|\\
|\childdocby{|\textit{main}|}|\\
\end{tabular}
\end{center}
%
Both forms have slightly different effects as described above.
The main file is prepared as usual, see \secref{sec:include}.

%%%%%%%%%%%%%%%%%%%%%%%%%%%%%%%%%%%%%%%%%%%%%%%%%%%%%%%%%%%%%%%%%%%%%%%%%%%%%%%%
\subsection{Legacy Detection}
\label{sec:detection}

The directive |\childdocmain| in the main file can detect
whether the complete document or merely a child is to be compiled
even without using the directive |\childdocof|.
This method is deprecated because it is less robust
and there is no compelling reason to use it;
it is merely provided for backward compatibility
and it may be removed in future versions.

If the detection mechanism is to be used,
it is mandatory to correctly specify
the filename of the main file as the argument of |\childdocmain|:
%
\begin{center}
\begin{tabular}{l}
|\input{childdoc.def}|\\
|\childdocmain{|\textit{main}|}|\\
\end{tabular}
\end{center}
%
If |\jobname| does not match the argument \textit{main} of |\childdocmain|,
it is assumed that |\jobname| points to the child file to be compiled.
When using |\childdocmain| with the main file specified as argument,
it suffices to start a child file
with just |\input{|\textit{main}|}|
without loading of the package and using |\childdocof|.
If instead all processing is done
with the appropriate \textsf{childdoc} directives,
the argument of \textit{main} of |\childdocmain| can be empty.

An alternative version of the command line processing described
in \secref{sec:commandline} using the detection mechanism reads:
%
\begin{center}
|... -jobname "|\textit{target}|" "|[\textit{flags}]%
[|\def\jobname{|\textit{dest}|}|]|\input{|\textit{main}|}"|
\end{center}

%%%%%%%%%%%%%%%%%%%%%%%%%%%%%%%%%%%%%%%%%%%%%%%%%%%%%%%%%%%%%%%%%%%%%%%%%%%%%%%%
\subsection{Manual Code}
\label{sec:manual}

In case one cannot be certain whether the definitions file |childdoc.def|
is installed on the target \TeX{} distribution
and one prefers not to ship it,
it is conceivable to paste a few relevant commands into the sources.

To that end, drop all statements |\input{childdoc.def}|
and perform the replacements as outlined below.
Instead of |\childdocmain{|\textit{main}|}| add the following code
to the top of the main file:
%
\begin{center}
\begin{tabular}{l}
|\||ifdefined\childdocname\endinput\||fi\newif\ifchilddoc|\\
|\edef\childdocname{\scantokens\expandafter{\jobname\noexpand}}|\\
|\def\childdocmain{|\textit{main}|}\||ifx\childdocmain\childdocname\||else|\\
|\childdoctrue\includeonly{\childdocname}\let\jobname\childdocmain\||fi|\\
\end{tabular}
\end{center}
%
Instead of |\childdocof{|\textit{main}|}| just include the main file
at the top of each child file:
%
\begin{center}
|\input{|\textit{main}|}|
\end{center}
%
A simple redirection |\childdocforward{|\textit{dest}|}| is achieved by:
%
\begin{center}
|\def\jobname{|\textit{dest}|}\input{\jobname}|
\end{center}
%
The redirection with prefix
|\childdocforwardprefix[|\textit{prefix}|]{|\textit{dest}|}|
is accomplished by:
%
\begin{center}
\begin{tabular}{l}
|{\edef\jobname{\scantokens\expandafter{\jobname\noexpand}}|\\
|\def\redirectjob |\textit{prefix}|#1~~~{\gdef\jobname{|\textit{dest}|#1}}|\\
|\expandafter\redirectjob\jobname~~~}\input{\jobname}|
\end{tabular}
\end{center}

In an alternative approach,
child documents can be compiled by a specific command line
without additional code or specific definitions:
%
\begin{center}
|... -jobname "|\textit{target}|" "|[\textit{flags}]%
|\includeonly{|\textit{dest}|}\input{|\textit{main}|}"|
\end{center}
%

%%%%%%%%%%%%%%%%%%%%%%%%%%%%%%%%%%%%%%%%%%%%%%%%%%%%%%%%%%%%%%%%%%%%%%%%%%%%%%%%
%%%%%%%%%%%%%%%%%%%%%%%%%%%%%%%%%%%%%%%%%%%%%%%%%%%%%%%%%%%%%%%%%%%%%%%%%%%%%%%%
\section{Information}

%%%%%%%%%%%%%%%%%%%%%%%%%%%%%%%%%%%%%%%%%%%%%%%%%%%%%%%%%%%%%%%%%%%%%%%%%%%%%%%%
\subsection{Copyright}

Copyright \copyright{} 2017--2018 Niklas Beisert

This work may be distributed and/or modified under the
conditions of the \LaTeX{} Project Public License, either version 1.3
of this license or (at your option) any later version.
The latest version of this license is in
  \url{http://www.latex-project.org/lppl.txt}
and version 1.3 or later is part of all distributions of \LaTeX{}
version 2005/12/01 or later.

This work has the LPPL maintenance status `maintained'.

The Current Maintainer of this work is Niklas Beisert.

This work consists of the files |README.txt|, |childdoc.ins| and |childdoc.dtx|
as well as the derived files |childdoc.def|, |cdocsamp.tex|
with |cdocsch1.tex|, |cdocsch2.tex|, |cdocspt3.tex|, |cdocspt4.tex|,
|cdocsdrf.tex|, |cdocsfn1.tex|, |cdocsfn2.tex|
as well as |childdoc.pdf|.

%%%%%%%%%%%%%%%%%%%%%%%%%%%%%%%%%%%%%%%%%%%%%%%%%%%%%%%%%%%%%%%%%%%%%%%%%%%%%%%%
\subsection{Files and Installation}

The package consists of the files:
%
\begin{center}
\begin{tabular}{ll}
    |README.txt|   & readme file \\
    |childdoc.ins| & installation file \\
    |childdoc.dtx| & source file \\
    |childdoc.def| & definition file \\
    |cdocsamp.tex| & sample main file \\
    |cdocsch1.tex| & sample include file \\
    |cdocsch2.tex| & sample include file \\
    |cdocspt3.tex| & sample part file \\
    |cdocspt4.tex| & sample part file \\
    |cdocsdrf.tex| & sample redirection file \\
    |cdocsfn1.tex| & sample redirection file \\
    |cdocsfn2.tex| & sample redirection file \\
    |childdoc.pdf| & manual
\end{tabular}
\end{center}
%
The distribution consists of the files
|README.txt|, |childdoc.ins| and |childdoc.dtx|.
%
\begin{itemize}
\item
Run (pdf)\LaTeX{} on |childdoc.dtx|
to compile the manual |childdoc.pdf| (this file).
\item
Run \LaTeX{} on |childdoc.ins| to create the definitions file |childdoc.def|
and the sample |cdocsamp.tex| with include files
|cdocsch1.tex|, |cdocsch2.tex|, |cdocspt3.tex|, |cdocspt4.tex|,
|cdocsdrf.tex|, |cdocsfn1.tex|, |cdocsfn2.tex|.
Then copy the file |childdoc.def| to an appropriate directory of your \LaTeX{}
distribution, e.g.\ \textit{texmf-root}|/tex/latex/childdoc|.
\end{itemize}

%%%%%%%%%%%%%%%%%%%%%%%%%%%%%%%%%%%%%%%%%%%%%%%%%%%%%%%%%%%%%%%%%%%%%%%%%%%%%%%%
\subsection{Related CTAN Packages}

There are several other packages which offer a similar functionality:
%
\begin{itemize}
\item
The packages
\href{http://ctan.org/pkg/docmute}{\textsf{docmute}},
\href{http://ctan.org/pkg/includex}{\textsf{includex}} and
\href{http://ctan.org/pkg/standalone}{\textsf{standalone}}
provide commands to include only the document body of
a child file thus allowing both files to be compiled individually.
\item
The packages \href{http://ctan.org/pkg/subdocs}{\textsf{subdocs}}
and \href{http://ctan.org/pkg/subfiles}{\textsf{subfiles}}
provide structures in which the main and child documents can be
encapsulated and allowing them to be compiled individually.
The inclusion mechanism is different from the conventional |\include|.
\item
The package \href{http://ctan.org/pkg/combine}{\textsf{combine}}
is an elaborate solution to combine several documents into one.
\end{itemize}
%
See also the CTAN topic \href{http://ctan.org/topic/subdocs}{\textsf{subdocs}}
for further related packages.
The present package differs from the above solutions in that
a document structure constructed with the conventional |\include| mechanism
just needs two extra commands at the top of every file
such that all constituent files can be compiled individually.

%%%%%%%%%%%%%%%%%%%%%%%%%%%%%%%%%%%%%%%%%%%%%%%%%%%%%%%%%%%%%%%%%%%%%%%%%%%%%%%%
%\subsection{Feature Suggestions}
%
%The following is a list of features which may be useful for future
%versions of this package:
%%
%\begin{itemize}
%\item
%\ldots
%\end{itemize}

%%%%%%%%%%%%%%%%%%%%%%%%%%%%%%%%%%%%%%%%%%%%%%%%%%%%%%%%%%%%%%%%%%%%%%%%%%%%%%%%
\subsection{Revision History}

%%%%%%%%%%%%%%%%%%%%%%%%%%%%%%%%%%%%%%%%
\paragraph{v2.0:} 2018/12/30

\begin{itemize}
\item
immediate forward processing
\item
added |\childdocby| mechanism
\item
manual restructured
\end{itemize}

%%%%%%%%%%%%%%%%%%%%%%%%%%%%%%%%%%%%%%%%
\paragraph{v1.6:} 2018/01/17

\begin{itemize}
\item
application for development of include files
\item
corrections to manual
\end{itemize}

%%%%%%%%%%%%%%%%%%%%%%%%%%%%%%%%%%%%%%%%
\paragraph{v1.5:} 2017/05/21

\begin{itemize}
\item
more complete structuring introduced
\item
|\childdocof| introduced
\item
|\childdoc| renamed to |\childdocmain|
\item
|\childredirect| renamed to |\childdocforward| and |\childdocforwardprefix|
and functionality expanded
\end{itemize}

%%%%%%%%%%%%%%%%%%%%%%%%%%%%%%%%%%%%%%%%
\paragraph{v1.0:} 2017/04/27

\begin{itemize}
\item
manual and install package
\item
first version published on CTAN
\end{itemize}

%%%%%%%%%%%%%%%%%%%%%%%%%%%%%%%%%%%%%%%%
\paragraph{v0.6:} 2017/04/26

\begin{itemize}
\item
redirection mechanism added
\end{itemize}

%%%%%%%%%%%%%%%%%%%%%%%%%%%%%%%%%%%%%%%%
\paragraph{v0.5:} 2017/04/26

\begin{itemize}
\item
functionality in definition file
\end{itemize}


%%%%%%%%%%%%%%%%%%%%%%%%%%%%%%%%%%%%%%%%%%%%%%%%%%%%%%%%%%%%%%%%%%%%%%%%%%%%%%%%
%%%%%%%%%%%%%%%%%%%%%%%%%%%%%%%%%%%%%%%%%%%%%%%%%%%%%%%%%%%%%%%%%%%%%%%%%%%%%%%%
%%%%%%%%%%%%%%%%%%%%%%%%%%%%%%%%%%%%%%%%%%%%%%%%%%%%%%%%%%%%%%%%%%%%%%%%%%%%%%%%
\appendix

\settowidth\MacroIndent{\rmfamily\scriptsize 000\ }

 \DocInput{childdoc.dtx}

\end{document}
%</driver>
% \fi
%
% %%%%%%%%%%%%%%%%%%%%%%%%%%%%%%%%%%%%%%%%%%%%%%%%%%%%%%%%%%%%%%%%%%%%%%%%%%%%%%
% %%%%%%%%%%%%%%%%%%%%%%%%%%%%%%%%%%%%%%%%%%%%%%%%%%%%%%%%%%%%%%%%%%%%%%%%%%%%%%
% \section{Sample}
%\iffalse
%<*samplemain>
%\fi
%
% The following presents a sample document
% with two chapters, two parts, a title page,
% a compile flag as well as three forwarding files to set the flag.
% It consists of eight |.tex| files:
% \begin{center}
% \begin{tabular}{ll}
% |cdocsamp.tex|&main file\\
% |cdocsch1.tex|&include file for chapter 1\\
% |cdocsch2.tex|&include file for chapter 2\\
% |cdocspt3.tex|&include file for part 3\\
% |cdocspt4.tex|&include file for part 4\\
% |cdocsdrf.tex|&forwarding file for main file in draft mode\\
% |cdocsfi1.tex|&forwarding file for final version of chapter 1\\
% |cdocsfi2.tex|&forwarding file for final version of chapter 2\\
% \end{tabular}
% \end{center}
% Each of the eight files can be compiled directly by the \LaTeX{} compiler.
%
% %%%%%%%%%%%%%%%%%%%%%%%%%%%%%%%%%%%%%%
% \paragraph{Main File.}
%
% The main file is called |cdocsamp.tex|.
%
% Load the \textsf{childdoc} definitions and
% declare the filename for the main document:
%    \begin{macrocode}
\input{childdoc.def}
\childdocmain{}
%    \end{macrocode}

% Optional override for |\version| flag:
%    \begin{macrocode}
%%\ifchilddoc\else\providecommand{\version}{draft}\fi
%    \end{macrocode}

% Define the default values for the |\version| flag
% (|final| for the main file and |draft| for childs):
%    \begin{macrocode}
\ifchilddoc
\providecommand{\version}{draft}
\else
\providecommand{\version}{final}
\fi
%    \end{macrocode}

% Load the standard document class:
%    \begin{macrocode}
\documentclass[12pt]{article}
%    \end{macrocode}

% Start the document body:
%    \begin{macrocode}
\begin{document}
%    \end{macrocode}

% Declare a title page.
% Print title, part of document being processed and version flag:
%    \begin{macrocode}
\addtocounter{page}{-1}
\begin{center}
{\LARGE\bfseries{}childdoc example\par}
\vspace{1cm}
\ifchilddoc
\ifchilddocmanual part\else chapter\fi:
`\childdocname' of `\childdocjob'\par
\else
main document: `\childdocjob'\par
\fi
version: \version\par
\end{center}
\newpage
%    \end{macrocode}

% Manually include selected file,
% otherwise process as usual:
%    \begin{macrocode}
\ifchilddocmanual
\section*{part `\childdocname'}
\input{\childdocname}
\else
%    \end{macrocode}

% Include the two chapters:
%    \begin{macrocode}
\include{cdocsch1}
\include{cdocsch2}
%    \end{macrocode}

% Include the two parts unless only chapters should be displayed:
%    \begin{macrocode}
\ifchilddoc\else
\section{part three}
\input{cdocspt3}
\section{part four}
\input{cdocspt4}
\fi
%    \end{macrocode}

% Process as usual until here:
%    \begin{macrocode}
\fi
%    \end{macrocode}

% End of document body:
%    \begin{macrocode}
\end{document}
%    \end{macrocode}
%\iffalse
%</samplemain>
%\fi
%
% %%%%%%%%%%%%%%%%%%%%%%%%%%%%%%%%%%%%%%
% \paragraph{Chapter Include Files.}
%
% The include files are called |cdocsch1.tex| and |cdocsch2.tex|.
%
%\iffalse
%<*samplechap1|samplechap2>
%\fi

% Optional override for |\version| flag:
%    \begin{macrocode}
%%\providecommand{\version}{final}
%    \end{macrocode}

% Include the main document:
%    \begin{macrocode}
\input{childdoc.def}
\childdocof{cdocsamp}
%    \end{macrocode}

%\iffalse
%</samplechap1|samplechap2>
%\fi
%
%\iffalse
%<*samplechap1>
%\fi
% Some text for chapter 1:
%    \begin{macrocode}
\section{one}
some text in chapter one
%    \end{macrocode}

%\iffalse
%</samplechap1>
%\fi
% Some text for chapter 2:
%\iffalse
%<*samplechap2>
%\fi
%    \begin{macrocode}
\section{two}
more text in chapter two
%    \end{macrocode}

%\iffalse
%</samplechap2>
%\fi
%
% %%%%%%%%%%%%%%%%%%%%%%%%%%%%%%%%%%%%%%
% \paragraph{Part Include Files.}
%
% The include files are called |cdocspt3.tex| and |cdocspt4.tex|.
%
%\iffalse
%<*samplepart3|samplepart4>
%\fi

% Optional override for |\version| flag:
%    \begin{macrocode}
%%\providecommand{\version}{final}
%    \end{macrocode}

% Include the main document:
%    \begin{macrocode}
\input{childdoc.def}
\childdocby{cdocsamp}
%    \end{macrocode}

%\iffalse
%</samplepart3|samplepart4>
%\fi
%
%\iffalse
%<*samplepart3>
%\fi
% Some text for part 3:
%    \begin{macrocode}
some text in part three
%    \end{macrocode}

%\iffalse
%</samplepart3>
%\fi
% Some text for part 4:
%\iffalse
%<*samplepart4>
%\fi
%    \begin{macrocode}
more text in part four
%    \end{macrocode}

%\iffalse
%</samplepart4>
%\fi
%
% %%%%%%%%%%%%%%%%%%%%%%%%%%%%%%%%%%%%%%
% \paragraph{Forwarding for a Complete Draft.}
%
% The following forwarding file |cdocsdrf.tex|
% compiles the main document in draft mode:
%\iffalse
%<*sampledraft>
%\fi
%    \begin{macrocode}
\def\version{draft}
\input{childdoc.def}
\childdocforward{cdocsamp}
%    \end{macrocode}

%\iffalse
%</sampledraft>
%\fi
%
% %%%%%%%%%%%%%%%%%%%%%%%%%%%%%%%%%%%%%%
% \paragraph{Forwarding for Final Version of the Chapters.}
%
% The following forwarding files |cdocsfn1.tex| and |cdocsfn2.tex|
% (with identical content)
% compile the final versions of the child documents
% |cdocsch1.tex| and |cdocsch2.tex|, respectively:
%\iffalse
%<*samplefinal>
%\fi
%    \begin{macrocode}
\def\version{final}
\input{childdoc.def}
\childdocforwardprefix[cdocsamp]{cdocsfn}{cdocsch}
%    \end{macrocode}

%\iffalse
%</samplefinal>
%\fi
%
% %%%%%%%%%%%%%%%%%%%%%%%%%%%%%%%%%%%%%%
% \paragraph{Command Line Processing.}
%
% The following three command lines generate the output files
% |cdocscld|, |cdocscl1| and |cdocscl2|
% which should be identical to
% |cdocsdrf|, |cdocsch1| and |cdocsfn2|, respectively:
% \begin{center}
% \begin{tabular}{l}
% |latex -jobname cdocscld \|\\
% |  "\def\version{draft}\input{childdoc.def}\childdocforward{cdocsamp}"|\\
% |latex -jobname cdocscl1 \|\\
% |  "\input{childdoc.def}\childdocforward[cdocsamp]{cdocsch1}"|\\
% |latex -jobname cdocscl2 \|\\
% |  "\def\version{final}\input{childdoc.def}\childdocforward{cdocsch2}"|
% \end{tabular}
% \end{center}
% Note that the trailing backslash on each first line
% merely continues the input to the second line
% (for convenient cut ant paste).
% Furthermore, the command |latex| can be replaced by any
% of its alternative versions such as |pdflatex|.
%
% %%%%%%%%%%%%%%%%%%%%%%%%%%%%%%%%%%%%%%%%%%%%%%%%%%%%%%%%%%%%%%%%%%%%%%%%%%%%%%
% %%%%%%%%%%%%%%%%%%%%%%%%%%%%%%%%%%%%%%%%%%%%%%%%%%%%%%%%%%%%%%%%%%%%%%%%%%%%%%
% \section{Implementation}
%\iffalse
%<*package>
%\fi
%
% This section describes the definitions file |childdoc.def|.

% The definitions cannot be loaded using |\usepackage| or |\RequirePackage|
% which has a mechanism to prevent loading a style file more than once.
% When loading the definitions by means of |\input|
% multiple instances have to be prevented manually:
%\iffalse
%This code needs to be before the `\ProvidesFile' directive
%which is defined at the beginning of this file.
%Therefore it is also placed there and commented out here.
%</package>
%<*discard>
%\fi
%    \begin{macrocode}
\ifdefined\childdocmain\endinput\fi
%    \end{macrocode}
%\iffalse
%</discard>
%<*package>
%\fi
%
% \macro{\ifchilddoc}
% \macro{\ifchilddocmanual}
% The conditional |\ifchilddoc| tells whether a
% child (true) or main (false) document is being compiled.
% The conditional |\ifchilddocmanual| tells whether
% the |\includeonly| mechanism is used (false) or
% the selection of child files must be performed manually (true).
% The definitions initialise to false:
%    \begin{macrocode}
\newif\ifchilddoc
\newif\ifchilddocmanual
%    \end{macrocode}

% \macro{\childdocname}
% \macro{\childdocjob}
% The macro |\childdocname| stores the name of the main document
% to be compiled. The macro |\childdocjob| stores the name of
% the document on which the \LaTeX{} compiler was originally invoked.
% The content of |\jobname| cannot be compared
% to filenames specified in the source due to different catcodes.
% The following code rescans |\jobname|, stores the result
% in |\childdocname| and saves a copy in |\childdocjob|:
%    \begin{macrocode}
\edef\childdocname{\scantokens\expandafter{\jobname\noexpand}}
\let\childdocjob\childdocname
%    \end{macrocode}

% \macro{\childdocdisable}
% The macro |\childdocdisable| prevents the main file
% from being processed more than once.
% At this stage, the main document command |\childdocmain|
% is assumed to be called once again where it should do nothing.
% Any subsequent call to it should prevent
% a secondary processing of the main document
% It overwrites the forwarding commands
% |\childdocof| and |\childdocforward|
% with empty macros to prevent further inclusions of the main document:
%    \begin{macrocode}
\newcommand{\childdocdisable}
{
  \renewcommand{\childdocmain}[1]{\renewcommand{\childdocmain}[1]{\endinput}}
  \renewcommand{\childdocof}[1]{}
  \renewcommand{\childdocby}[2][]{}
  \renewcommand{\childdocforward}[2][]{}
  \renewcommand{\childdocdisable}{}
}
%    \end{macrocode}

% \macro{\childdocmain}
% The macro |\childdocmain| is to be called at the top of the main file
% with nothing or the main filename (without extension) as argument.
% First, it breaks loops.
% If the argument is not empty and does not match |\childdocname|
% (which is set by the first inclusion of |childdoc.def|),
% |\ifchilddoc| is set to true, |\includeonly| is applied to the child file
% and |\jobname| is set to the main file
% (for proper handling of |.aux| files):
%    \begin{macrocode}
\newcommand{\childdocmain}[1]
{
  \childdocdisable\childdocmain{}
  \if?#1?\else
    \begingroup
      \def\childdoctmp{#1}
      \ifx\childdoctmp\childdocname
        \def\childdoctmp{}
      \else
        \def\childdoctmp
        {
          \childdoctrue
          \includeonly{\childdocname}
          \def\childdocjob{#1}
          \def\jobname{#1}
        }
      \fi
      \expandafter
    \endgroup
    \childdoctmp
  \fi
}
%    \end{macrocode}

% \macro{\childdocof}
% The command |\childdocof| redirects
% compilation to the main file |#1|.
%    \begin{macrocode}
\newcommand{\childdocof}[1]
{
  \childdocdisable
  \childdoctrue
  \includeonly{\childdocname}
  \def\jobname{#1}
  \def\childdocjob{#1}
  \input{#1}
}
%    \end{macrocode}

% \macro{\childdocby}
% The command |\childdocby| ....
%    \begin{macrocode}
\newcommand{\childdocby}[2][]
{
  \childdocdisable
  \childdoctrue
  \childdocmanualtrue
  \if?#1?\else
    \def\jobname{#2}
  \fi
  \def\childdocjob{#2}
  \input{#2}
  \endinput
}
%    \end{macrocode}

% \macro{\childdocforward}
% The command |\childdocforward| redirects
% compilation to the main file or
% (if the optional argument is given) a child file.
% Parameters are set as if the main file
% or a child file starting with |\childdocof| was compiled.
% Then compilation is handed over to the main file:
%    \begin{macrocode}
\newcommand{\childdocforward}[2][]
{
  \begingroup
    \if?#1?
      \def\childdoctmp
      {
        \def\childdocname{#2}
        \def\childdocjob{#2}
        \def\jobname{#2}
        \input{#2}
        \endinput
      }
    \else
      \def\childdoctmp
      {
        \childdocdisable
        \def\childdocname{#2}
        \childdoctrue
        \includeonly{#2}
        \def\childdocjob{#1}
        \def\jobname{#1}
        \input{#1}
        \endinput
      }
    \fi
    \expandafter
  \endgroup
  \childdoctmp
}
%    \end{macrocode}

% \macro{\childdocforwardprefix}
% The command |\childdocforwardprefix| redirects
% compilation to the main or a child file by means of a pattern.
% The prefix |#1| in the current filename is replaced by |#2|
% and the suffix of the current filename is kept
% (it is assumed that the filename does not contain the substring `|~~~|'
% which is used as a delimiter).
% Compilation is handed over to the new file by |\childdocforward|:
%    \begin{macrocode}
\newcommand{\childdocforwardprefix}[3][]
{
  \begingroup
    \def\childdocextract #2##1~~~{\def\childdoctmp{\childdocforward[#1]{#3##1}}}
    \expandafter\childdocextract\childdocname~~~
    \expandafter
  \endgroup
  \childdoctmp
}
%    \end{macrocode}

% \macro{\childdoc}
% The deprecated macro |\childdoc| is a legacy version of |\childdocmain|:
%    \begin{macrocode}
\newcommand{\childdoc}{\childdocmain}
%    \end{macrocode}

% \macro{\childdocredirect}
% The deprecated macro |\childdocredirect| is a legacy version
% of |\childdocforward| and |\childdocforwardprefix|:
%    \begin{macrocode}
\newcommand{\childdocredirect}[2][]
{
  \begingroup
    \if?#1?
      \def\childdoctmp{\childdocforward{#2}}
    \else
      \def\childdoctmp{\childdocforwardprefix{#1}{#2}}
    \fi
    \expandafter
  \endgroup
  \childdoctmp
}
%    \end{macrocode}

%\iffalse
%</package>
%\fi
%
\endinput

\childdocforwardprefix[cdocsamp]{cdocsfn}{cdocsch}
%    \end{macrocode}

%\iffalse
%</samplefinal>
%\fi
%
% %%%%%%%%%%%%%%%%%%%%%%%%%%%%%%%%%%%%%%
% \paragraph{Command Line Processing.}
%
% The following three command lines generate the output files
% |cdocscld|, |cdocscl1| and |cdocscl2|
% which should be identical to
% |cdocsdrf|, |cdocsch1| and |cdocsfn2|, respectively:
% \begin{center}
% \begin{tabular}{l}
% |latex -jobname cdocscld \|\\
% |  "\def\version{draft}% \iffalse
%
% childdoc.dtx Copyright (C) 2017-2018 Niklas Beisert
%
% This work may be distributed and/or modified under the
% conditions of the LaTeX Project Public License, either version 1.3
% of this license or (at your option) any later version.
% The latest version of this license is in
%   http://www.latex-project.org/lppl.txt
% and version 1.3 or later is part of all distributions of LaTeX
% version 2005/12/01 or later.
%
% This work has the LPPL maintenance status `maintained'.
%
% The Current Maintainer of this work is Niklas Beisert.
%
% This work consists of the files childdoc.dtx and childdoc.ins
% and the derived files childdoc.def and cdocsamp.tex with
% cdocsch1.tex, cdocsch2.tex, cdocsdrf.tex, cdocsfn1.tex, cdocsfn2.tex.
%
%<package>\ifdefined\childdocmain\endinput\fi
%<package>\ProvidesFile{childdoc.def}[2018/12/30 v2.0 child document driver]
%<samplemain>\ProvidesFile{cdocsamp.tex}[2018/12/30 v2.0 sample for childdoc]
%<*driver>
%\ProvidesFile{childdoc.drv}[2018/12/30 v2.0 childdoc reference manual file]
\PassOptionsToClass{10pt,a4paper}{article}
\documentclass{ltxdoc}

\usepackage[margin=35mm]{geometry}
\usepackage{hyperref}
\usepackage{hyperxmp}
\usepackage[usenames]{color}

\hypersetup{colorlinks=true}
\hypersetup{pdfstartview=FitH}
\hypersetup{pdfpagemode=UseNone}
\hypersetup{pdfsource={}}
\hypersetup{pdflang={en-UK}}
\hypersetup{pdfcopyright={Copyright 2017-2018 Niklas Beisert.
  This work may be distributed and/or modified under the
  conditions of the LaTeX Project Public License, either version 1.3
  of this license or (at your option) any later version.}}
\hypersetup{pdflicenseurl={http://www.latex-project.org/lppl.txt}}
\hypersetup{pdfcontactaddress={ETH Zurich, ITP, HIT K,
  Wolfgang-Pauli-Strasse 27}}
\hypersetup{pdfcontactpostcode={8093}}
\hypersetup{pdfcontactcity={Zurich}}
\hypersetup{pdfcontactcountry={Switzerland}}
\hypersetup{pdfcontactemail={nbeisert@itp.phys.ethz.ch}}
\hypersetup{pdfcontacturl={http://people.phys.ethz.ch/\xmptilde nbeisert/}}

\newcommand{\secref}[1]{\hyperref[#1]{section \ref*{#1}}}

\parskip1ex
\parindent0pt
\let\olditemize\itemize
\def\itemize{\olditemize\parskip0pt}

\begin{document}

\title{The \textsf{childdoc} Package}
\hypersetup{pdftitle={The childdoc Package}}
\author{Niklas Beisert\\[2ex]
  Institut f\"ur Theoretische Physik\\
  Eidgen\"ossische Technische Hochschule Z\"urich\\
  Wolfgang-Pauli-Strasse 27, 8093 Z\"urich, Switzerland\\[1ex]
  \href{mailto:nbeisert@itp.phys.ethz.ch}
  {\texttt{nbeisert@itp.phys.ethz.ch}}}
\hypersetup{pdfauthor={Niklas Beisert}}
\hypersetup{pdfsubject={Manual for the LaTeX2e Package childdoc}}
\date{30 December 2018, \textsf{v2.0}}
\maketitle

\begin{abstract}\noindent
\textsf{childdoc} is a \LaTeXe{} package
that enables the direct compilation
of document sections included by |\include|
to individual files.
\end{abstract}

\begingroup
\parskip0ex
\tableofcontents
\endgroup

%%%%%%%%%%%%%%%%%%%%%%%%%%%%%%%%%%%%%%%%%%%%%%%%%%%%%%%%%%%%%%%%%%%%%%%%%%%%%%%%
%%%%%%%%%%%%%%%%%%%%%%%%%%%%%%%%%%%%%%%%%%%%%%%%%%%%%%%%%%%%%%%%%%%%%%%%%%%%%%%%
\section{Introduction}

\LaTeX{} provides a mechanism to structure a large document (such as a book)
into a main file and several child files (containing the chapters)
using the |\include| command.
This mechanism is beneficial for documents
which span hundreds of pages in order to
make the source file(s) more manageable.
Moreover, compilation can be restricted to
selected child files by means of the |\includeonly| command.
The latter feature can be used to reduce the compilation time while editing
(this was significantly more useful in the earlier days of \LaTeX{})
or to generate a smaller document which is easier to navigate.
Another application of |\includeonly| is to generate
documents consisting of selected parts of the complete document.

However, there are a few drawbacks of the plain |\include| mechanism:
\begin{itemize}
\item
The child files cannot be compiled on their own,
they can only be compiled via the main file.
A naive editing environment
(such as a text editor with an option
to have the current file processed by \LaTeX)
may require one to switch to the main file before compiling;
attempting to compile the child file produces errors.
\item
The main file must be modified (each time)
to adjust the |\includeonly| command
to the present needs. This easily leaves the main file in a messy state.
\item
The generated document will always carry the filename
of the main document. This is inconvenient if
several child files are to be compiled and
to be kept for distribution.
\end{itemize}

The present package provides a simple interface
to make child files individually compilable by \LaTeX{}.
Compiling a child file then has the same effect as compiling
the main file with an |\includeonly| command
to select the appropriate child.
Moreover the generated document will carry the name of the child
rather than the main file.
This resolves all three above issues.

This feature is meant to make the editing of books,
thesis documents and lecture notes somewhat more convenient.
However, the package can also be used efficiently for
composing a series of documents (such as exercise sheets)
which are typically distributed individually.
It then assists the author in generating the individual documents
(potentially in different versions)
as well as a document containing the collected series.
Another application is in developing style files
or other kinds of included material
where compilation of the style file could redirect
to a sample or test file.

%%%%%%%%%%%%%%%%%%%%%%%%%%%%%%%%%%%%%%%%%%%%%%%%%%%%%%%%%%%%%%%%%%%%%%%%%%%%%%%%
%%%%%%%%%%%%%%%%%%%%%%%%%%%%%%%%%%%%%%%%%%%%%%%%%%%%%%%%%%%%%%%%%%%%%%%%%%%%%%%%
\section{Usage}

First of all, the package \textsf{childdoc} is \emph{not} a standard
\LaTeXe{} |.sty| style file! Therefore it needs to be invoked in
a non-standard way.

%%%%%%%%%%%%%%%%%%%%%%%%%%%%%%%%%%%%%%%%%%%%%%%%%%%%%%%%%%%%%%%%%%%%%%%%%%%%%%%%
\subsection{Included Files}
\label{sec:include}

%%%%%%%%%%%%%%%%%%%%%%%%%%%%%%%%%%%%%%%%
\DescribeMacro{\childdocmain}
To use the package, add the commands
\begin{center}
\begin{tabular}{l}
|\input{childdoc.def}|\\
|\childdocmain{}|\\
\end{tabular}
\end{center}
at the very top of the main \LaTeX{} file,
in particular \emph{before} the |\documentclass| statement!
The argument of |\childdocmain| should be left empty
(but it must be present).

%%%%%%%%%%%%%%%%%%%%%%%%%%%%%%%%%%%%%%%%
\DescribeMacro{\childdocof}
Furthermore, add the commands
\begin{center}
\begin{tabular}{l}
|\input{childdoc.def}|\\
|\childdocof{|\textit{main}|}|\\
\end{tabular}
\end{center}
at the top of every child file \textit{child}
which is included by |\include{|\textit{child}|}|
from within the main file
(or at least for those files to be compiled individually).
The argument \textit{main} must be the filename of the main file.

There are a couple of
considerations in setting up the main and child documents:

%%%%%%%%%%%%%%%%%%%%%%%%%%%%%%%%%%%%%%%%
\paragraph{Restrictions.}

Please note the following restrictions:
\begin{itemize}
\item
|\childdocmain| must be called with one argument \textit{main}
to ensure compatibility with earlier version of the package.
It must either be empty (|\childdocmain{}|)
or precisely match the filename of the main file in which it is specified.
See \secref{sec:detection} for further information.
\item
The filename \textit{main} must be specified without the |.tex| extension.
\item
The filename \textit{main} is case sensitive
(even in case-insensitive file systems)
due to internal string comparison.
\item
The argument \textit{main} should be fully expanded, it cannot be a macro.
\item
Subdirectories and special characters should be avoided in filenames.
\item
The command |\childdocmain{|\textit{main}|}| must be followed by a whitespace.
It should not be followed immediately by another command
or by a comment mark `|%|'.
This is because the \TeX{} parser reads the token immediately following
the argument of |\childdocmain| and puts it
at the beginning of every child section;
however, a white\-space is ignored.
\end{itemize}

%%%%%%%%%%%%%%%%%%%%%%%%%%%%%%%%%%%%%%%%
\paragraph{Content of Main File.}

It is advisable to place all content in the child files included by |\include|.
Any output contained in the main file will appear in all child documents
unless suppressed manually;
it cannot be suppressed automatically by the |\includeonly| directive
and thus should normally be avoided.
A method to include some content in the main file
by means of conditional processing is described in \secref{sec:conditional}.

%%%%%%%%%%%%%%%%%%%%%%%%%%%%%%%%%%%%%%%%
\paragraph{Page Numbering.}

When only a part of the document is compiled,
the appropriate numbering of pages
(as well as other status parameters)
is determined from the |.aux| files.
The latter contain information from previous passes.
However this information needs to propagate through
all intermediate child documents.
Therefore the page numbering in child documents may well
be inconsistent until the complete document is compiled at least once.

A useful (if unconventional) way to always ensure a consistent
page numbering is to restart the numbering in each child document
and denote the pages by `\textit{child}|.|\textit{page}'
where \textit{child} represents the chapter/section number of the child file.
This can be achieved by the command
|\numberwithin{page}{|\textit{child}|}|
of the \textsf{amsmath} package
where \textit{child} can be |chapter| or |section|
depending on the chosen structuring.
Alternatively, one can modify the macro |\thepage| appropriately
and reset the counter |page| at the start of each child file.

%%%%%%%%%%%%%%%%%%%%%%%%%%%%%%%%%%%%%%%%%%%%%%%%%%%%%%%%%%%%%%%%%%%%%%%%%%%%%%%%
\subsection{Conditional Processing}
\label{sec:conditional}

The package provides a mechanism to compile different versions
of a document. To customise the versions further some conditional processing
can come in handy to distinguish which version is being compiled.
The package provides two macros to describe the compilation context:

%%%%%%%%%%%%%%%%%%%%%%%%%%%%%%%%%%%%%%%%
\DescribeMacro{\ifchilddoc}
The conditional |\ifchilddoc| distinguishes between the compilation of
child documents and the main document:
%
\begin{center}
|\ifchilddoc |\textit{child-code}| |[|\||else |\textit{main-code}]| \||fi|
\end{center}

%%%%%%%%%%%%%%%%%%%%%%%%%%%%%%%%%%%%%%%%
\DescribeMacro{\childdocname}
\DescribeMacro{\childdocjob}
The macro |\childdocname| contains the filename (without extension)
of the main or child file being processed.
Note that |\childdocjob| will always contain the name of the main file.

%%%%%%%%%%%%%%%%%%%%%%%%%%%%%%%%%%%%%%%%
\paragraph{Title Page.}

Conditional processing can be used to include a title or banner page
in the main document when proper precautions are taken.
Importantly, the code in the main file should ensure that the page counter
(as well as other status parameters which are stored in the |.aux| files)
takes the same value after the conditional processing.
Otherwise the page numbers may take divergent values
depending on which part is compiled.

For example, a title page could be declared by:
%
\begin{center}
\begin{tabular}{l}
|\ifchilddoc\||else|\\
|\addtocounter{page}{-1}|\\
\textit{code for title page}\\
|\newpage|\\
|\||fi|
\end{tabular}
\end{center}
%
A banner page for the child documents can be generated by:
%
\begin{center}
\begin{tabular}{l}
|\ifchilddoc|\\
|\addtocounter{page}{-1}|\\
\textit{code for banner page}\\
|\newpage|\\
|\||fi|
\end{tabular}
\end{center}
%
Here one could write a message such as:
\begin{center}
|This is the part \childdocname{} of \childdocjob{}.|
\end{center}

%%%%%%%%%%%%%%%%%%%%%%%%%%%%%%%%%%%%%%%%%%%%%%%%%%%%%%%%%%%%%%%%%%%%%%%%%%%%%%%%
\subsection{Flags}
\label{sec:flags}

The package makes it easy to generate different versions
of the main or child documents.
To this end compilation flags can be defined
and assigned different default values.
They will be particularly useful in conjunction
with the forwarding mechanism described in \secref{sec:forward}.

For example, it may be useful to have a flag |\version|
which can be set to |draft| or |final|.
The document source will contain some conditional code
depending on the value of |\version|.
Suppose further, the flag should default to |final| for the main file
and to |draft| for child files
which is a natural assignment for editing the document.
This is achieved by placing the following code
in the preamble of the main document
(below the |\childdocmain| directive):
%
\begin{center}
\begin{tabular}{l}
|\ifchilddoc|\\
|\providecommand{\version}{draft}|\\
|\||else|\\
|\providecommand{\version}{final}|\\
|\||fi|
\end{tabular}
\end{center}
%
The definition by |\providecommand| makes sure
that previous definitions are not overwritten.
Further statements |\providecommand{\version}{...}|
can thus be added before the above code to override it.

For the main file, one might add a line
(between |\childdocmain| and the above block)
%
\begin{center}
|%\ifchilddoc\||else\providecommand{\version}{draft}\||fi|
\end{center}
%
which can be uncommented to produce a draft version.
Likewise one can add a line to the very top of a child file
(above the |\childdocof{|\textit{main}|}| directive)
%
\begin{center}
|%\providecommand{\version}{final}|
\end{center}
%
which can be uncommented to produce the final version of this child document.

%%%%%%%%%%%%%%%%%%%%%%%%%%%%%%%%%%%%%%%%%%%%%%%%%%%%%%%%%%%%%%%%%%%%%%%%%%%%%%%%
\subsection{Forwarding}
\label{sec:forward}

Different versions of the main or child documents
using compilation flags as described in \secref{sec:flags}
can be (permanently) stored in different files
for convenient compilation, viewing and distribution.
To this end, the package defines a command
to pass on compilation to a different file:

%%%%%%%%%%%%%%%%%%%%%%%%%%%%%%%%%%%%%%%%
\DescribeMacro{\childdocforward}
The command |\childdocforward| redirects processing to
another source file:
%
\begin{center}
\begin{tabular}{l}
|\input{childdoc.def}|\\
|\childdocforward[|\textit{main}|]{|\textit{dest}|}|\\
\end{tabular}
\end{center}
%
The argument \textit{dest} is the destination file
(without extension).
It should be the main file or one of the child files.
Note that further \textsf{childdoc} directives
such as |\childdocof| and |\childdocforward|
in the indicated file will be processed in this form.
The optional argument \textit{main}
passes on directly to the main file \textit{main}
while pretending to compile the child \textit{dest}.
This form behaves as if \textit{dest}
issues |\childdocof{|\textit{main}|}| right away,
and no further \textsf{childdoc} directives will be processed.

%%%%%%%%%%%%%%%%%%%%%%%%%%%%%%%%%%%%%%%%
\DescribeMacro{\...prefix}
In the alternative form |\childdocforwardprefix|,
%
\begin{center}
\begin{tabular}{l}
|\input{childdoc.def}|\\
|\childdocforwardprefix[|\textit{main}|]{|\textit{prefix}|}{|\textit{dest}|}|
\end{tabular}
\end{center}
%
the destination file is determined by a pattern
depending on the current file:
To make this work, the current file must be called
`{\textit{prefix}\hspace{0.2em}\textit{suffix}}'
with \textit{prefix} matching precisely the argument.
Processing is then passed on to the file
`{\textit{dest}\hspace{0.2em}\textit{suffix}}'.
Surely, the same effect is achieved by
directly specifying the
argument `{\textit{dest}\hspace{0.2em}\textit{suffix}}'
in the first form.
However, that requires to set up a different file
for each child. With the alternative form of the command
all these files can have exactly the same content
which simplifies setting them up and maintaining them.

For example, the following file |draft.tex|
with a compilation flag |\version| as described in \secref{sec:flags}
compiles the main document as a draft:
%
\begin{center}
\begin{tabular}{l}
|\def\version{draft}|\\
|\input{childdoc.def}|\\
|\childdocforward{|\textit{main}|}|
\end{tabular}
\end{center}
%
Likewise, the following files |final|\textit{nn}|.tex|
compile the final version of the child document
|child|\textit{nn}|.tex|:
%
\begin{center}
\begin{tabular}{l}
|\def\version{final}|\\
|\input{childdoc.def}|\\
|\childdocforwardprefix{final}{child}|
\end{tabular}
\end{center}
%

Note that when several versions of a main file and/or of each child file
are to be generated, it may be convenient to set up a |Makefile| or
shell script to automatise the process.

%%%%%%%%%%%%%%%%%%%%%%%%%%%%%%%%%%%%%%%%%%%%%%%%%%%%%%%%%%%%%%%%%%%%%%%%%%%%%%%%
\subsection{Command Line Processing}
\label{sec:commandline}

The effect of redirection files can also be achieved by invoking
the \LaTeX{} compiler with a more elaborate command line.
Most conveniently this should be done as part
of a shell script or a |Makefile|.

When using \textsf{childdoc} in the main file, the following
command lines effectively perform a redirection
(note that depending on the shell being used,
backslashes may have to be doubled: `|\|' $\to$ `|\\|'):
%
\begin{center}
|... -jobname "|\textit{target}|" |\\|"|[\textit{flags}]%
|\input{childdoc.def}\childdocforward[|\textit{main}|]{|\textit{dest}|}"|
\end{center}
%
Here \textit{target} is the name of the output file,
\textit{main} is the name of the main file
and \textit{dest} is the name of the main or child file to be processed
(all filenames without extensions).
The optional argument \textit{main} can be omitted
if \textit{main} matches \textit{dest}.
Optionally, compilation \textit{flags} can be defined via |\def| commands.
This command line makes the \TeX{} engine believe
it is compiling the file \textit{target}
whose content is specified as the latter parameter.
The provided code then forwards the processing to
\textit{main} or \textit{dest} as described in \secref{sec:forward}.

%%%%%%%%%%%%%%%%%%%%%%%%%%%%%%%%%%%%%%%%%%%%%%%%%%%%%%%%%%%%%%%%%%%%%%%%%%%%%%%%
\subsection{Include by Input}
\label{sec:input}

Including child documents by |\include| has some restrictions by design.
Most notably, the content of a child document always occupies
its own set of pages; pages cannot be shared between child documents.
Usually, this behaviour makes perfect sense
because each child document contain an essential part of the document.
However, in some situations it may be desirable to compose
a document from a collection of parts
without having mandatory page breaks between then.
For this case, the package
provides a mechanism to include parts
by |\input| which can also be processed individually.
However, by construction this mechanism
requires manual handling of the content to be output.

%%%%%%%%%%%%%%%%%%%%%%%%%%%%%%%%%%%%%%%%
\DescribeMacro{\ifchilddocmanual}
The main file should be prepared as usual, see \secref{sec:include}.
However, the document body must make a distinction
between processing of an individual part and of the main document, e.g.:
%
\begin{center}
\begin{tabular}{l}
|\ifchilddocmanual|\\
|\input{\childdocname}|\\
|\||else|\\
\textit{document body with }|\input{|\textit{part}|}|\\
|\||fi|
\end{tabular}
\end{center}
%
The conditional |\ifchilddocmanual| is true whenever
a part to be included by |\input| is being compiled,
and the name of the part is stored in |\childdocname|.

%%%%%%%%%%%%%%%%%%%%%%%%%%%%%%%%%%%%%%%%
\DescribeMacro{\childdocby}
Each part to be included by |\input| should start with:
%
\begin{center}
\begin{tabular}{l}
|\input{childdoc.def}|\\
|\childdocby{|\textit{main}|}|\\
\end{tabular}
\end{center}
%
The directive |\childdocby| is similar to |\childdocof|
described in \secref{sec:include},
but the subsequent selection of content must be done manually.
To that end, both |\ifchilddoc| and |\ifchilddocmanual|
will be true upon processing of a part,
and the name of the part is stored in |\childdocname|.
Note that |\jobname| will be set to the filename of the current part
so that each part receives an individual |.aux| file
that does not interfere with the |.aux| file(s) of the main document.
This behaviour can be altered by the alternative form
|\childdocby[*]{|\textit{main}|}| (with a non-empty optional argument)
which uses the |.aux| file of the main document
by setting |\jobname| to \textit{main}.

%%%%%%%%%%%%%%%%%%%%%%%%%%%%%%%%%%%%%%%%%%%%%%%%%%%%%%%%%%%%%%%%%%%%%%%%%%%%%%%%
\subsection{Driver Development}
\label{sec:driver}

The \textsf{childdoc} mechanism can also be use for the development
of definition files such as \LaTeX{} styles or classes.
This case differs from the above setup with multiple parts
included by |\include| in that no |\includeonly| should be invoked.
This can be achieved by starting the include file
(before |\ProvidesPackage|) with:
%
\begin{center}
\begin{tabular}{l}
|\input{childdoc.def}|\\
|\childdocforward{|\textit{main}|}|\\
\end{tabular}
\end{center}
%
or alternatively with:
%
\begin{center}
\begin{tabular}{l}
|\input{childdoc.def}|\\
|\childdocby{|\textit{main}|}|\\
\end{tabular}
\end{center}
%
Both forms have slightly different effects as described above.
The main file is prepared as usual, see \secref{sec:include}.

%%%%%%%%%%%%%%%%%%%%%%%%%%%%%%%%%%%%%%%%%%%%%%%%%%%%%%%%%%%%%%%%%%%%%%%%%%%%%%%%
\subsection{Legacy Detection}
\label{sec:detection}

The directive |\childdocmain| in the main file can detect
whether the complete document or merely a child is to be compiled
even without using the directive |\childdocof|.
This method is deprecated because it is less robust
and there is no compelling reason to use it;
it is merely provided for backward compatibility
and it may be removed in future versions.

If the detection mechanism is to be used,
it is mandatory to correctly specify
the filename of the main file as the argument of |\childdocmain|:
%
\begin{center}
\begin{tabular}{l}
|\input{childdoc.def}|\\
|\childdocmain{|\textit{main}|}|\\
\end{tabular}
\end{center}
%
If |\jobname| does not match the argument \textit{main} of |\childdocmain|,
it is assumed that |\jobname| points to the child file to be compiled.
When using |\childdocmain| with the main file specified as argument,
it suffices to start a child file
with just |\input{|\textit{main}|}|
without loading of the package and using |\childdocof|.
If instead all processing is done
with the appropriate \textsf{childdoc} directives,
the argument of \textit{main} of |\childdocmain| can be empty.

An alternative version of the command line processing described
in \secref{sec:commandline} using the detection mechanism reads:
%
\begin{center}
|... -jobname "|\textit{target}|" "|[\textit{flags}]%
[|\def\jobname{|\textit{dest}|}|]|\input{|\textit{main}|}"|
\end{center}

%%%%%%%%%%%%%%%%%%%%%%%%%%%%%%%%%%%%%%%%%%%%%%%%%%%%%%%%%%%%%%%%%%%%%%%%%%%%%%%%
\subsection{Manual Code}
\label{sec:manual}

In case one cannot be certain whether the definitions file |childdoc.def|
is installed on the target \TeX{} distribution
and one prefers not to ship it,
it is conceivable to paste a few relevant commands into the sources.

To that end, drop all statements |\input{childdoc.def}|
and perform the replacements as outlined below.
Instead of |\childdocmain{|\textit{main}|}| add the following code
to the top of the main file:
%
\begin{center}
\begin{tabular}{l}
|\||ifdefined\childdocname\endinput\||fi\newif\ifchilddoc|\\
|\edef\childdocname{\scantokens\expandafter{\jobname\noexpand}}|\\
|\def\childdocmain{|\textit{main}|}\||ifx\childdocmain\childdocname\||else|\\
|\childdoctrue\includeonly{\childdocname}\let\jobname\childdocmain\||fi|\\
\end{tabular}
\end{center}
%
Instead of |\childdocof{|\textit{main}|}| just include the main file
at the top of each child file:
%
\begin{center}
|\input{|\textit{main}|}|
\end{center}
%
A simple redirection |\childdocforward{|\textit{dest}|}| is achieved by:
%
\begin{center}
|\def\jobname{|\textit{dest}|}\input{\jobname}|
\end{center}
%
The redirection with prefix
|\childdocforwardprefix[|\textit{prefix}|]{|\textit{dest}|}|
is accomplished by:
%
\begin{center}
\begin{tabular}{l}
|{\edef\jobname{\scantokens\expandafter{\jobname\noexpand}}|\\
|\def\redirectjob |\textit{prefix}|#1~~~{\gdef\jobname{|\textit{dest}|#1}}|\\
|\expandafter\redirectjob\jobname~~~}\input{\jobname}|
\end{tabular}
\end{center}

In an alternative approach,
child documents can be compiled by a specific command line
without additional code or specific definitions:
%
\begin{center}
|... -jobname "|\textit{target}|" "|[\textit{flags}]%
|\includeonly{|\textit{dest}|}\input{|\textit{main}|}"|
\end{center}
%

%%%%%%%%%%%%%%%%%%%%%%%%%%%%%%%%%%%%%%%%%%%%%%%%%%%%%%%%%%%%%%%%%%%%%%%%%%%%%%%%
%%%%%%%%%%%%%%%%%%%%%%%%%%%%%%%%%%%%%%%%%%%%%%%%%%%%%%%%%%%%%%%%%%%%%%%%%%%%%%%%
\section{Information}

%%%%%%%%%%%%%%%%%%%%%%%%%%%%%%%%%%%%%%%%%%%%%%%%%%%%%%%%%%%%%%%%%%%%%%%%%%%%%%%%
\subsection{Copyright}

Copyright \copyright{} 2017--2018 Niklas Beisert

This work may be distributed and/or modified under the
conditions of the \LaTeX{} Project Public License, either version 1.3
of this license or (at your option) any later version.
The latest version of this license is in
  \url{http://www.latex-project.org/lppl.txt}
and version 1.3 or later is part of all distributions of \LaTeX{}
version 2005/12/01 or later.

This work has the LPPL maintenance status `maintained'.

The Current Maintainer of this work is Niklas Beisert.

This work consists of the files |README.txt|, |childdoc.ins| and |childdoc.dtx|
as well as the derived files |childdoc.def|, |cdocsamp.tex|
with |cdocsch1.tex|, |cdocsch2.tex|, |cdocspt3.tex|, |cdocspt4.tex|,
|cdocsdrf.tex|, |cdocsfn1.tex|, |cdocsfn2.tex|
as well as |childdoc.pdf|.

%%%%%%%%%%%%%%%%%%%%%%%%%%%%%%%%%%%%%%%%%%%%%%%%%%%%%%%%%%%%%%%%%%%%%%%%%%%%%%%%
\subsection{Files and Installation}

The package consists of the files:
%
\begin{center}
\begin{tabular}{ll}
    |README.txt|   & readme file \\
    |childdoc.ins| & installation file \\
    |childdoc.dtx| & source file \\
    |childdoc.def| & definition file \\
    |cdocsamp.tex| & sample main file \\
    |cdocsch1.tex| & sample include file \\
    |cdocsch2.tex| & sample include file \\
    |cdocspt3.tex| & sample part file \\
    |cdocspt4.tex| & sample part file \\
    |cdocsdrf.tex| & sample redirection file \\
    |cdocsfn1.tex| & sample redirection file \\
    |cdocsfn2.tex| & sample redirection file \\
    |childdoc.pdf| & manual
\end{tabular}
\end{center}
%
The distribution consists of the files
|README.txt|, |childdoc.ins| and |childdoc.dtx|.
%
\begin{itemize}
\item
Run (pdf)\LaTeX{} on |childdoc.dtx|
to compile the manual |childdoc.pdf| (this file).
\item
Run \LaTeX{} on |childdoc.ins| to create the definitions file |childdoc.def|
and the sample |cdocsamp.tex| with include files
|cdocsch1.tex|, |cdocsch2.tex|, |cdocspt3.tex|, |cdocspt4.tex|,
|cdocsdrf.tex|, |cdocsfn1.tex|, |cdocsfn2.tex|.
Then copy the file |childdoc.def| to an appropriate directory of your \LaTeX{}
distribution, e.g.\ \textit{texmf-root}|/tex/latex/childdoc|.
\end{itemize}

%%%%%%%%%%%%%%%%%%%%%%%%%%%%%%%%%%%%%%%%%%%%%%%%%%%%%%%%%%%%%%%%%%%%%%%%%%%%%%%%
\subsection{Related CTAN Packages}

There are several other packages which offer a similar functionality:
%
\begin{itemize}
\item
The packages
\href{http://ctan.org/pkg/docmute}{\textsf{docmute}},
\href{http://ctan.org/pkg/includex}{\textsf{includex}} and
\href{http://ctan.org/pkg/standalone}{\textsf{standalone}}
provide commands to include only the document body of
a child file thus allowing both files to be compiled individually.
\item
The packages \href{http://ctan.org/pkg/subdocs}{\textsf{subdocs}}
and \href{http://ctan.org/pkg/subfiles}{\textsf{subfiles}}
provide structures in which the main and child documents can be
encapsulated and allowing them to be compiled individually.
The inclusion mechanism is different from the conventional |\include|.
\item
The package \href{http://ctan.org/pkg/combine}{\textsf{combine}}
is an elaborate solution to combine several documents into one.
\end{itemize}
%
See also the CTAN topic \href{http://ctan.org/topic/subdocs}{\textsf{subdocs}}
for further related packages.
The present package differs from the above solutions in that
a document structure constructed with the conventional |\include| mechanism
just needs two extra commands at the top of every file
such that all constituent files can be compiled individually.

%%%%%%%%%%%%%%%%%%%%%%%%%%%%%%%%%%%%%%%%%%%%%%%%%%%%%%%%%%%%%%%%%%%%%%%%%%%%%%%%
%\subsection{Feature Suggestions}
%
%The following is a list of features which may be useful for future
%versions of this package:
%%
%\begin{itemize}
%\item
%\ldots
%\end{itemize}

%%%%%%%%%%%%%%%%%%%%%%%%%%%%%%%%%%%%%%%%%%%%%%%%%%%%%%%%%%%%%%%%%%%%%%%%%%%%%%%%
\subsection{Revision History}

%%%%%%%%%%%%%%%%%%%%%%%%%%%%%%%%%%%%%%%%
\paragraph{v2.0:} 2018/12/30

\begin{itemize}
\item
immediate forward processing
\item
added |\childdocby| mechanism
\item
manual restructured
\end{itemize}

%%%%%%%%%%%%%%%%%%%%%%%%%%%%%%%%%%%%%%%%
\paragraph{v1.6:} 2018/01/17

\begin{itemize}
\item
application for development of include files
\item
corrections to manual
\end{itemize}

%%%%%%%%%%%%%%%%%%%%%%%%%%%%%%%%%%%%%%%%
\paragraph{v1.5:} 2017/05/21

\begin{itemize}
\item
more complete structuring introduced
\item
|\childdocof| introduced
\item
|\childdoc| renamed to |\childdocmain|
\item
|\childredirect| renamed to |\childdocforward| and |\childdocforwardprefix|
and functionality expanded
\end{itemize}

%%%%%%%%%%%%%%%%%%%%%%%%%%%%%%%%%%%%%%%%
\paragraph{v1.0:} 2017/04/27

\begin{itemize}
\item
manual and install package
\item
first version published on CTAN
\end{itemize}

%%%%%%%%%%%%%%%%%%%%%%%%%%%%%%%%%%%%%%%%
\paragraph{v0.6:} 2017/04/26

\begin{itemize}
\item
redirection mechanism added
\end{itemize}

%%%%%%%%%%%%%%%%%%%%%%%%%%%%%%%%%%%%%%%%
\paragraph{v0.5:} 2017/04/26

\begin{itemize}
\item
functionality in definition file
\end{itemize}


%%%%%%%%%%%%%%%%%%%%%%%%%%%%%%%%%%%%%%%%%%%%%%%%%%%%%%%%%%%%%%%%%%%%%%%%%%%%%%%%
%%%%%%%%%%%%%%%%%%%%%%%%%%%%%%%%%%%%%%%%%%%%%%%%%%%%%%%%%%%%%%%%%%%%%%%%%%%%%%%%
%%%%%%%%%%%%%%%%%%%%%%%%%%%%%%%%%%%%%%%%%%%%%%%%%%%%%%%%%%%%%%%%%%%%%%%%%%%%%%%%
\appendix

\settowidth\MacroIndent{\rmfamily\scriptsize 000\ }

 \DocInput{childdoc.dtx}

\end{document}
%</driver>
% \fi
%
% %%%%%%%%%%%%%%%%%%%%%%%%%%%%%%%%%%%%%%%%%%%%%%%%%%%%%%%%%%%%%%%%%%%%%%%%%%%%%%
% %%%%%%%%%%%%%%%%%%%%%%%%%%%%%%%%%%%%%%%%%%%%%%%%%%%%%%%%%%%%%%%%%%%%%%%%%%%%%%
% \section{Sample}
%\iffalse
%<*samplemain>
%\fi
%
% The following presents a sample document
% with two chapters, two parts, a title page,
% a compile flag as well as three forwarding files to set the flag.
% It consists of eight |.tex| files:
% \begin{center}
% \begin{tabular}{ll}
% |cdocsamp.tex|&main file\\
% |cdocsch1.tex|&include file for chapter 1\\
% |cdocsch2.tex|&include file for chapter 2\\
% |cdocspt3.tex|&include file for part 3\\
% |cdocspt4.tex|&include file for part 4\\
% |cdocsdrf.tex|&forwarding file for main file in draft mode\\
% |cdocsfi1.tex|&forwarding file for final version of chapter 1\\
% |cdocsfi2.tex|&forwarding file for final version of chapter 2\\
% \end{tabular}
% \end{center}
% Each of the eight files can be compiled directly by the \LaTeX{} compiler.
%
% %%%%%%%%%%%%%%%%%%%%%%%%%%%%%%%%%%%%%%
% \paragraph{Main File.}
%
% The main file is called |cdocsamp.tex|.
%
% Load the \textsf{childdoc} definitions and
% declare the filename for the main document:
%    \begin{macrocode}
\input{childdoc.def}
\childdocmain{}
%    \end{macrocode}

% Optional override for |\version| flag:
%    \begin{macrocode}
%%\ifchilddoc\else\providecommand{\version}{draft}\fi
%    \end{macrocode}

% Define the default values for the |\version| flag
% (|final| for the main file and |draft| for childs):
%    \begin{macrocode}
\ifchilddoc
\providecommand{\version}{draft}
\else
\providecommand{\version}{final}
\fi
%    \end{macrocode}

% Load the standard document class:
%    \begin{macrocode}
\documentclass[12pt]{article}
%    \end{macrocode}

% Start the document body:
%    \begin{macrocode}
\begin{document}
%    \end{macrocode}

% Declare a title page.
% Print title, part of document being processed and version flag:
%    \begin{macrocode}
\addtocounter{page}{-1}
\begin{center}
{\LARGE\bfseries{}childdoc example\par}
\vspace{1cm}
\ifchilddoc
\ifchilddocmanual part\else chapter\fi:
`\childdocname' of `\childdocjob'\par
\else
main document: `\childdocjob'\par
\fi
version: \version\par
\end{center}
\newpage
%    \end{macrocode}

% Manually include selected file,
% otherwise process as usual:
%    \begin{macrocode}
\ifchilddocmanual
\section*{part `\childdocname'}
\input{\childdocname}
\else
%    \end{macrocode}

% Include the two chapters:
%    \begin{macrocode}
\include{cdocsch1}
\include{cdocsch2}
%    \end{macrocode}

% Include the two parts unless only chapters should be displayed:
%    \begin{macrocode}
\ifchilddoc\else
\section{part three}
\input{cdocspt3}
\section{part four}
\input{cdocspt4}
\fi
%    \end{macrocode}

% Process as usual until here:
%    \begin{macrocode}
\fi
%    \end{macrocode}

% End of document body:
%    \begin{macrocode}
\end{document}
%    \end{macrocode}
%\iffalse
%</samplemain>
%\fi
%
% %%%%%%%%%%%%%%%%%%%%%%%%%%%%%%%%%%%%%%
% \paragraph{Chapter Include Files.}
%
% The include files are called |cdocsch1.tex| and |cdocsch2.tex|.
%
%\iffalse
%<*samplechap1|samplechap2>
%\fi

% Optional override for |\version| flag:
%    \begin{macrocode}
%%\providecommand{\version}{final}
%    \end{macrocode}

% Include the main document:
%    \begin{macrocode}
\input{childdoc.def}
\childdocof{cdocsamp}
%    \end{macrocode}

%\iffalse
%</samplechap1|samplechap2>
%\fi
%
%\iffalse
%<*samplechap1>
%\fi
% Some text for chapter 1:
%    \begin{macrocode}
\section{one}
some text in chapter one
%    \end{macrocode}

%\iffalse
%</samplechap1>
%\fi
% Some text for chapter 2:
%\iffalse
%<*samplechap2>
%\fi
%    \begin{macrocode}
\section{two}
more text in chapter two
%    \end{macrocode}

%\iffalse
%</samplechap2>
%\fi
%
% %%%%%%%%%%%%%%%%%%%%%%%%%%%%%%%%%%%%%%
% \paragraph{Part Include Files.}
%
% The include files are called |cdocspt3.tex| and |cdocspt4.tex|.
%
%\iffalse
%<*samplepart3|samplepart4>
%\fi

% Optional override for |\version| flag:
%    \begin{macrocode}
%%\providecommand{\version}{final}
%    \end{macrocode}

% Include the main document:
%    \begin{macrocode}
\input{childdoc.def}
\childdocby{cdocsamp}
%    \end{macrocode}

%\iffalse
%</samplepart3|samplepart4>
%\fi
%
%\iffalse
%<*samplepart3>
%\fi
% Some text for part 3:
%    \begin{macrocode}
some text in part three
%    \end{macrocode}

%\iffalse
%</samplepart3>
%\fi
% Some text for part 4:
%\iffalse
%<*samplepart4>
%\fi
%    \begin{macrocode}
more text in part four
%    \end{macrocode}

%\iffalse
%</samplepart4>
%\fi
%
% %%%%%%%%%%%%%%%%%%%%%%%%%%%%%%%%%%%%%%
% \paragraph{Forwarding for a Complete Draft.}
%
% The following forwarding file |cdocsdrf.tex|
% compiles the main document in draft mode:
%\iffalse
%<*sampledraft>
%\fi
%    \begin{macrocode}
\def\version{draft}
\input{childdoc.def}
\childdocforward{cdocsamp}
%    \end{macrocode}

%\iffalse
%</sampledraft>
%\fi
%
% %%%%%%%%%%%%%%%%%%%%%%%%%%%%%%%%%%%%%%
% \paragraph{Forwarding for Final Version of the Chapters.}
%
% The following forwarding files |cdocsfn1.tex| and |cdocsfn2.tex|
% (with identical content)
% compile the final versions of the child documents
% |cdocsch1.tex| and |cdocsch2.tex|, respectively:
%\iffalse
%<*samplefinal>
%\fi
%    \begin{macrocode}
\def\version{final}
\input{childdoc.def}
\childdocforwardprefix[cdocsamp]{cdocsfn}{cdocsch}
%    \end{macrocode}

%\iffalse
%</samplefinal>
%\fi
%
% %%%%%%%%%%%%%%%%%%%%%%%%%%%%%%%%%%%%%%
% \paragraph{Command Line Processing.}
%
% The following three command lines generate the output files
% |cdocscld|, |cdocscl1| and |cdocscl2|
% which should be identical to
% |cdocsdrf|, |cdocsch1| and |cdocsfn2|, respectively:
% \begin{center}
% \begin{tabular}{l}
% |latex -jobname cdocscld \|\\
% |  "\def\version{draft}\input{childdoc.def}\childdocforward{cdocsamp}"|\\
% |latex -jobname cdocscl1 \|\\
% |  "\input{childdoc.def}\childdocforward[cdocsamp]{cdocsch1}"|\\
% |latex -jobname cdocscl2 \|\\
% |  "\def\version{final}\input{childdoc.def}\childdocforward{cdocsch2}"|
% \end{tabular}
% \end{center}
% Note that the trailing backslash on each first line
% merely continues the input to the second line
% (for convenient cut ant paste).
% Furthermore, the command |latex| can be replaced by any
% of its alternative versions such as |pdflatex|.
%
% %%%%%%%%%%%%%%%%%%%%%%%%%%%%%%%%%%%%%%%%%%%%%%%%%%%%%%%%%%%%%%%%%%%%%%%%%%%%%%
% %%%%%%%%%%%%%%%%%%%%%%%%%%%%%%%%%%%%%%%%%%%%%%%%%%%%%%%%%%%%%%%%%%%%%%%%%%%%%%
% \section{Implementation}
%\iffalse
%<*package>
%\fi
%
% This section describes the definitions file |childdoc.def|.

% The definitions cannot be loaded using |\usepackage| or |\RequirePackage|
% which has a mechanism to prevent loading a style file more than once.
% When loading the definitions by means of |\input|
% multiple instances have to be prevented manually:
%\iffalse
%This code needs to be before the `\ProvidesFile' directive
%which is defined at the beginning of this file.
%Therefore it is also placed there and commented out here.
%</package>
%<*discard>
%\fi
%    \begin{macrocode}
\ifdefined\childdocmain\endinput\fi
%    \end{macrocode}
%\iffalse
%</discard>
%<*package>
%\fi
%
% \macro{\ifchilddoc}
% \macro{\ifchilddocmanual}
% The conditional |\ifchilddoc| tells whether a
% child (true) or main (false) document is being compiled.
% The conditional |\ifchilddocmanual| tells whether
% the |\includeonly| mechanism is used (false) or
% the selection of child files must be performed manually (true).
% The definitions initialise to false:
%    \begin{macrocode}
\newif\ifchilddoc
\newif\ifchilddocmanual
%    \end{macrocode}

% \macro{\childdocname}
% \macro{\childdocjob}
% The macro |\childdocname| stores the name of the main document
% to be compiled. The macro |\childdocjob| stores the name of
% the document on which the \LaTeX{} compiler was originally invoked.
% The content of |\jobname| cannot be compared
% to filenames specified in the source due to different catcodes.
% The following code rescans |\jobname|, stores the result
% in |\childdocname| and saves a copy in |\childdocjob|:
%    \begin{macrocode}
\edef\childdocname{\scantokens\expandafter{\jobname\noexpand}}
\let\childdocjob\childdocname
%    \end{macrocode}

% \macro{\childdocdisable}
% The macro |\childdocdisable| prevents the main file
% from being processed more than once.
% At this stage, the main document command |\childdocmain|
% is assumed to be called once again where it should do nothing.
% Any subsequent call to it should prevent
% a secondary processing of the main document
% It overwrites the forwarding commands
% |\childdocof| and |\childdocforward|
% with empty macros to prevent further inclusions of the main document:
%    \begin{macrocode}
\newcommand{\childdocdisable}
{
  \renewcommand{\childdocmain}[1]{\renewcommand{\childdocmain}[1]{\endinput}}
  \renewcommand{\childdocof}[1]{}
  \renewcommand{\childdocby}[2][]{}
  \renewcommand{\childdocforward}[2][]{}
  \renewcommand{\childdocdisable}{}
}
%    \end{macrocode}

% \macro{\childdocmain}
% The macro |\childdocmain| is to be called at the top of the main file
% with nothing or the main filename (without extension) as argument.
% First, it breaks loops.
% If the argument is not empty and does not match |\childdocname|
% (which is set by the first inclusion of |childdoc.def|),
% |\ifchilddoc| is set to true, |\includeonly| is applied to the child file
% and |\jobname| is set to the main file
% (for proper handling of |.aux| files):
%    \begin{macrocode}
\newcommand{\childdocmain}[1]
{
  \childdocdisable\childdocmain{}
  \if?#1?\else
    \begingroup
      \def\childdoctmp{#1}
      \ifx\childdoctmp\childdocname
        \def\childdoctmp{}
      \else
        \def\childdoctmp
        {
          \childdoctrue
          \includeonly{\childdocname}
          \def\childdocjob{#1}
          \def\jobname{#1}
        }
      \fi
      \expandafter
    \endgroup
    \childdoctmp
  \fi
}
%    \end{macrocode}

% \macro{\childdocof}
% The command |\childdocof| redirects
% compilation to the main file |#1|.
%    \begin{macrocode}
\newcommand{\childdocof}[1]
{
  \childdocdisable
  \childdoctrue
  \includeonly{\childdocname}
  \def\jobname{#1}
  \def\childdocjob{#1}
  \input{#1}
}
%    \end{macrocode}

% \macro{\childdocby}
% The command |\childdocby| ....
%    \begin{macrocode}
\newcommand{\childdocby}[2][]
{
  \childdocdisable
  \childdoctrue
  \childdocmanualtrue
  \if?#1?\else
    \def\jobname{#2}
  \fi
  \def\childdocjob{#2}
  \input{#2}
  \endinput
}
%    \end{macrocode}

% \macro{\childdocforward}
% The command |\childdocforward| redirects
% compilation to the main file or
% (if the optional argument is given) a child file.
% Parameters are set as if the main file
% or a child file starting with |\childdocof| was compiled.
% Then compilation is handed over to the main file:
%    \begin{macrocode}
\newcommand{\childdocforward}[2][]
{
  \begingroup
    \if?#1?
      \def\childdoctmp
      {
        \def\childdocname{#2}
        \def\childdocjob{#2}
        \def\jobname{#2}
        \input{#2}
        \endinput
      }
    \else
      \def\childdoctmp
      {
        \childdocdisable
        \def\childdocname{#2}
        \childdoctrue
        \includeonly{#2}
        \def\childdocjob{#1}
        \def\jobname{#1}
        \input{#1}
        \endinput
      }
    \fi
    \expandafter
  \endgroup
  \childdoctmp
}
%    \end{macrocode}

% \macro{\childdocforwardprefix}
% The command |\childdocforwardprefix| redirects
% compilation to the main or a child file by means of a pattern.
% The prefix |#1| in the current filename is replaced by |#2|
% and the suffix of the current filename is kept
% (it is assumed that the filename does not contain the substring `|~~~|'
% which is used as a delimiter).
% Compilation is handed over to the new file by |\childdocforward|:
%    \begin{macrocode}
\newcommand{\childdocforwardprefix}[3][]
{
  \begingroup
    \def\childdocextract #2##1~~~{\def\childdoctmp{\childdocforward[#1]{#3##1}}}
    \expandafter\childdocextract\childdocname~~~
    \expandafter
  \endgroup
  \childdoctmp
}
%    \end{macrocode}

% \macro{\childdoc}
% The deprecated macro |\childdoc| is a legacy version of |\childdocmain|:
%    \begin{macrocode}
\newcommand{\childdoc}{\childdocmain}
%    \end{macrocode}

% \macro{\childdocredirect}
% The deprecated macro |\childdocredirect| is a legacy version
% of |\childdocforward| and |\childdocforwardprefix|:
%    \begin{macrocode}
\newcommand{\childdocredirect}[2][]
{
  \begingroup
    \if?#1?
      \def\childdoctmp{\childdocforward{#2}}
    \else
      \def\childdoctmp{\childdocforwardprefix{#1}{#2}}
    \fi
    \expandafter
  \endgroup
  \childdoctmp
}
%    \end{macrocode}

%\iffalse
%</package>
%\fi
%
\endinput
\childdocforward{cdocsamp}"|\\
% |latex -jobname cdocscl1 \|\\
% |  "% \iffalse
%
% childdoc.dtx Copyright (C) 2017-2018 Niklas Beisert
%
% This work may be distributed and/or modified under the
% conditions of the LaTeX Project Public License, either version 1.3
% of this license or (at your option) any later version.
% The latest version of this license is in
%   http://www.latex-project.org/lppl.txt
% and version 1.3 or later is part of all distributions of LaTeX
% version 2005/12/01 or later.
%
% This work has the LPPL maintenance status `maintained'.
%
% The Current Maintainer of this work is Niklas Beisert.
%
% This work consists of the files childdoc.dtx and childdoc.ins
% and the derived files childdoc.def and cdocsamp.tex with
% cdocsch1.tex, cdocsch2.tex, cdocsdrf.tex, cdocsfn1.tex, cdocsfn2.tex.
%
%<package>\ifdefined\childdocmain\endinput\fi
%<package>\ProvidesFile{childdoc.def}[2018/12/30 v2.0 child document driver]
%<samplemain>\ProvidesFile{cdocsamp.tex}[2018/12/30 v2.0 sample for childdoc]
%<*driver>
%\ProvidesFile{childdoc.drv}[2018/12/30 v2.0 childdoc reference manual file]
\PassOptionsToClass{10pt,a4paper}{article}
\documentclass{ltxdoc}

\usepackage[margin=35mm]{geometry}
\usepackage{hyperref}
\usepackage{hyperxmp}
\usepackage[usenames]{color}

\hypersetup{colorlinks=true}
\hypersetup{pdfstartview=FitH}
\hypersetup{pdfpagemode=UseNone}
\hypersetup{pdfsource={}}
\hypersetup{pdflang={en-UK}}
\hypersetup{pdfcopyright={Copyright 2017-2018 Niklas Beisert.
  This work may be distributed and/or modified under the
  conditions of the LaTeX Project Public License, either version 1.3
  of this license or (at your option) any later version.}}
\hypersetup{pdflicenseurl={http://www.latex-project.org/lppl.txt}}
\hypersetup{pdfcontactaddress={ETH Zurich, ITP, HIT K,
  Wolfgang-Pauli-Strasse 27}}
\hypersetup{pdfcontactpostcode={8093}}
\hypersetup{pdfcontactcity={Zurich}}
\hypersetup{pdfcontactcountry={Switzerland}}
\hypersetup{pdfcontactemail={nbeisert@itp.phys.ethz.ch}}
\hypersetup{pdfcontacturl={http://people.phys.ethz.ch/\xmptilde nbeisert/}}

\newcommand{\secref}[1]{\hyperref[#1]{section \ref*{#1}}}

\parskip1ex
\parindent0pt
\let\olditemize\itemize
\def\itemize{\olditemize\parskip0pt}

\begin{document}

\title{The \textsf{childdoc} Package}
\hypersetup{pdftitle={The childdoc Package}}
\author{Niklas Beisert\\[2ex]
  Institut f\"ur Theoretische Physik\\
  Eidgen\"ossische Technische Hochschule Z\"urich\\
  Wolfgang-Pauli-Strasse 27, 8093 Z\"urich, Switzerland\\[1ex]
  \href{mailto:nbeisert@itp.phys.ethz.ch}
  {\texttt{nbeisert@itp.phys.ethz.ch}}}
\hypersetup{pdfauthor={Niklas Beisert}}
\hypersetup{pdfsubject={Manual for the LaTeX2e Package childdoc}}
\date{30 December 2018, \textsf{v2.0}}
\maketitle

\begin{abstract}\noindent
\textsf{childdoc} is a \LaTeXe{} package
that enables the direct compilation
of document sections included by |\include|
to individual files.
\end{abstract}

\begingroup
\parskip0ex
\tableofcontents
\endgroup

%%%%%%%%%%%%%%%%%%%%%%%%%%%%%%%%%%%%%%%%%%%%%%%%%%%%%%%%%%%%%%%%%%%%%%%%%%%%%%%%
%%%%%%%%%%%%%%%%%%%%%%%%%%%%%%%%%%%%%%%%%%%%%%%%%%%%%%%%%%%%%%%%%%%%%%%%%%%%%%%%
\section{Introduction}

\LaTeX{} provides a mechanism to structure a large document (such as a book)
into a main file and several child files (containing the chapters)
using the |\include| command.
This mechanism is beneficial for documents
which span hundreds of pages in order to
make the source file(s) more manageable.
Moreover, compilation can be restricted to
selected child files by means of the |\includeonly| command.
The latter feature can be used to reduce the compilation time while editing
(this was significantly more useful in the earlier days of \LaTeX{})
or to generate a smaller document which is easier to navigate.
Another application of |\includeonly| is to generate
documents consisting of selected parts of the complete document.

However, there are a few drawbacks of the plain |\include| mechanism:
\begin{itemize}
\item
The child files cannot be compiled on their own,
they can only be compiled via the main file.
A naive editing environment
(such as a text editor with an option
to have the current file processed by \LaTeX)
may require one to switch to the main file before compiling;
attempting to compile the child file produces errors.
\item
The main file must be modified (each time)
to adjust the |\includeonly| command
to the present needs. This easily leaves the main file in a messy state.
\item
The generated document will always carry the filename
of the main document. This is inconvenient if
several child files are to be compiled and
to be kept for distribution.
\end{itemize}

The present package provides a simple interface
to make child files individually compilable by \LaTeX{}.
Compiling a child file then has the same effect as compiling
the main file with an |\includeonly| command
to select the appropriate child.
Moreover the generated document will carry the name of the child
rather than the main file.
This resolves all three above issues.

This feature is meant to make the editing of books,
thesis documents and lecture notes somewhat more convenient.
However, the package can also be used efficiently for
composing a series of documents (such as exercise sheets)
which are typically distributed individually.
It then assists the author in generating the individual documents
(potentially in different versions)
as well as a document containing the collected series.
Another application is in developing style files
or other kinds of included material
where compilation of the style file could redirect
to a sample or test file.

%%%%%%%%%%%%%%%%%%%%%%%%%%%%%%%%%%%%%%%%%%%%%%%%%%%%%%%%%%%%%%%%%%%%%%%%%%%%%%%%
%%%%%%%%%%%%%%%%%%%%%%%%%%%%%%%%%%%%%%%%%%%%%%%%%%%%%%%%%%%%%%%%%%%%%%%%%%%%%%%%
\section{Usage}

First of all, the package \textsf{childdoc} is \emph{not} a standard
\LaTeXe{} |.sty| style file! Therefore it needs to be invoked in
a non-standard way.

%%%%%%%%%%%%%%%%%%%%%%%%%%%%%%%%%%%%%%%%%%%%%%%%%%%%%%%%%%%%%%%%%%%%%%%%%%%%%%%%
\subsection{Included Files}
\label{sec:include}

%%%%%%%%%%%%%%%%%%%%%%%%%%%%%%%%%%%%%%%%
\DescribeMacro{\childdocmain}
To use the package, add the commands
\begin{center}
\begin{tabular}{l}
|\input{childdoc.def}|\\
|\childdocmain{}|\\
\end{tabular}
\end{center}
at the very top of the main \LaTeX{} file,
in particular \emph{before} the |\documentclass| statement!
The argument of |\childdocmain| should be left empty
(but it must be present).

%%%%%%%%%%%%%%%%%%%%%%%%%%%%%%%%%%%%%%%%
\DescribeMacro{\childdocof}
Furthermore, add the commands
\begin{center}
\begin{tabular}{l}
|\input{childdoc.def}|\\
|\childdocof{|\textit{main}|}|\\
\end{tabular}
\end{center}
at the top of every child file \textit{child}
which is included by |\include{|\textit{child}|}|
from within the main file
(or at least for those files to be compiled individually).
The argument \textit{main} must be the filename of the main file.

There are a couple of
considerations in setting up the main and child documents:

%%%%%%%%%%%%%%%%%%%%%%%%%%%%%%%%%%%%%%%%
\paragraph{Restrictions.}

Please note the following restrictions:
\begin{itemize}
\item
|\childdocmain| must be called with one argument \textit{main}
to ensure compatibility with earlier version of the package.
It must either be empty (|\childdocmain{}|)
or precisely match the filename of the main file in which it is specified.
See \secref{sec:detection} for further information.
\item
The filename \textit{main} must be specified without the |.tex| extension.
\item
The filename \textit{main} is case sensitive
(even in case-insensitive file systems)
due to internal string comparison.
\item
The argument \textit{main} should be fully expanded, it cannot be a macro.
\item
Subdirectories and special characters should be avoided in filenames.
\item
The command |\childdocmain{|\textit{main}|}| must be followed by a whitespace.
It should not be followed immediately by another command
or by a comment mark `|%|'.
This is because the \TeX{} parser reads the token immediately following
the argument of |\childdocmain| and puts it
at the beginning of every child section;
however, a white\-space is ignored.
\end{itemize}

%%%%%%%%%%%%%%%%%%%%%%%%%%%%%%%%%%%%%%%%
\paragraph{Content of Main File.}

It is advisable to place all content in the child files included by |\include|.
Any output contained in the main file will appear in all child documents
unless suppressed manually;
it cannot be suppressed automatically by the |\includeonly| directive
and thus should normally be avoided.
A method to include some content in the main file
by means of conditional processing is described in \secref{sec:conditional}.

%%%%%%%%%%%%%%%%%%%%%%%%%%%%%%%%%%%%%%%%
\paragraph{Page Numbering.}

When only a part of the document is compiled,
the appropriate numbering of pages
(as well as other status parameters)
is determined from the |.aux| files.
The latter contain information from previous passes.
However this information needs to propagate through
all intermediate child documents.
Therefore the page numbering in child documents may well
be inconsistent until the complete document is compiled at least once.

A useful (if unconventional) way to always ensure a consistent
page numbering is to restart the numbering in each child document
and denote the pages by `\textit{child}|.|\textit{page}'
where \textit{child} represents the chapter/section number of the child file.
This can be achieved by the command
|\numberwithin{page}{|\textit{child}|}|
of the \textsf{amsmath} package
where \textit{child} can be |chapter| or |section|
depending on the chosen structuring.
Alternatively, one can modify the macro |\thepage| appropriately
and reset the counter |page| at the start of each child file.

%%%%%%%%%%%%%%%%%%%%%%%%%%%%%%%%%%%%%%%%%%%%%%%%%%%%%%%%%%%%%%%%%%%%%%%%%%%%%%%%
\subsection{Conditional Processing}
\label{sec:conditional}

The package provides a mechanism to compile different versions
of a document. To customise the versions further some conditional processing
can come in handy to distinguish which version is being compiled.
The package provides two macros to describe the compilation context:

%%%%%%%%%%%%%%%%%%%%%%%%%%%%%%%%%%%%%%%%
\DescribeMacro{\ifchilddoc}
The conditional |\ifchilddoc| distinguishes between the compilation of
child documents and the main document:
%
\begin{center}
|\ifchilddoc |\textit{child-code}| |[|\||else |\textit{main-code}]| \||fi|
\end{center}

%%%%%%%%%%%%%%%%%%%%%%%%%%%%%%%%%%%%%%%%
\DescribeMacro{\childdocname}
\DescribeMacro{\childdocjob}
The macro |\childdocname| contains the filename (without extension)
of the main or child file being processed.
Note that |\childdocjob| will always contain the name of the main file.

%%%%%%%%%%%%%%%%%%%%%%%%%%%%%%%%%%%%%%%%
\paragraph{Title Page.}

Conditional processing can be used to include a title or banner page
in the main document when proper precautions are taken.
Importantly, the code in the main file should ensure that the page counter
(as well as other status parameters which are stored in the |.aux| files)
takes the same value after the conditional processing.
Otherwise the page numbers may take divergent values
depending on which part is compiled.

For example, a title page could be declared by:
%
\begin{center}
\begin{tabular}{l}
|\ifchilddoc\||else|\\
|\addtocounter{page}{-1}|\\
\textit{code for title page}\\
|\newpage|\\
|\||fi|
\end{tabular}
\end{center}
%
A banner page for the child documents can be generated by:
%
\begin{center}
\begin{tabular}{l}
|\ifchilddoc|\\
|\addtocounter{page}{-1}|\\
\textit{code for banner page}\\
|\newpage|\\
|\||fi|
\end{tabular}
\end{center}
%
Here one could write a message such as:
\begin{center}
|This is the part \childdocname{} of \childdocjob{}.|
\end{center}

%%%%%%%%%%%%%%%%%%%%%%%%%%%%%%%%%%%%%%%%%%%%%%%%%%%%%%%%%%%%%%%%%%%%%%%%%%%%%%%%
\subsection{Flags}
\label{sec:flags}

The package makes it easy to generate different versions
of the main or child documents.
To this end compilation flags can be defined
and assigned different default values.
They will be particularly useful in conjunction
with the forwarding mechanism described in \secref{sec:forward}.

For example, it may be useful to have a flag |\version|
which can be set to |draft| or |final|.
The document source will contain some conditional code
depending on the value of |\version|.
Suppose further, the flag should default to |final| for the main file
and to |draft| for child files
which is a natural assignment for editing the document.
This is achieved by placing the following code
in the preamble of the main document
(below the |\childdocmain| directive):
%
\begin{center}
\begin{tabular}{l}
|\ifchilddoc|\\
|\providecommand{\version}{draft}|\\
|\||else|\\
|\providecommand{\version}{final}|\\
|\||fi|
\end{tabular}
\end{center}
%
The definition by |\providecommand| makes sure
that previous definitions are not overwritten.
Further statements |\providecommand{\version}{...}|
can thus be added before the above code to override it.

For the main file, one might add a line
(between |\childdocmain| and the above block)
%
\begin{center}
|%\ifchilddoc\||else\providecommand{\version}{draft}\||fi|
\end{center}
%
which can be uncommented to produce a draft version.
Likewise one can add a line to the very top of a child file
(above the |\childdocof{|\textit{main}|}| directive)
%
\begin{center}
|%\providecommand{\version}{final}|
\end{center}
%
which can be uncommented to produce the final version of this child document.

%%%%%%%%%%%%%%%%%%%%%%%%%%%%%%%%%%%%%%%%%%%%%%%%%%%%%%%%%%%%%%%%%%%%%%%%%%%%%%%%
\subsection{Forwarding}
\label{sec:forward}

Different versions of the main or child documents
using compilation flags as described in \secref{sec:flags}
can be (permanently) stored in different files
for convenient compilation, viewing and distribution.
To this end, the package defines a command
to pass on compilation to a different file:

%%%%%%%%%%%%%%%%%%%%%%%%%%%%%%%%%%%%%%%%
\DescribeMacro{\childdocforward}
The command |\childdocforward| redirects processing to
another source file:
%
\begin{center}
\begin{tabular}{l}
|\input{childdoc.def}|\\
|\childdocforward[|\textit{main}|]{|\textit{dest}|}|\\
\end{tabular}
\end{center}
%
The argument \textit{dest} is the destination file
(without extension).
It should be the main file or one of the child files.
Note that further \textsf{childdoc} directives
such as |\childdocof| and |\childdocforward|
in the indicated file will be processed in this form.
The optional argument \textit{main}
passes on directly to the main file \textit{main}
while pretending to compile the child \textit{dest}.
This form behaves as if \textit{dest}
issues |\childdocof{|\textit{main}|}| right away,
and no further \textsf{childdoc} directives will be processed.

%%%%%%%%%%%%%%%%%%%%%%%%%%%%%%%%%%%%%%%%
\DescribeMacro{\...prefix}
In the alternative form |\childdocforwardprefix|,
%
\begin{center}
\begin{tabular}{l}
|\input{childdoc.def}|\\
|\childdocforwardprefix[|\textit{main}|]{|\textit{prefix}|}{|\textit{dest}|}|
\end{tabular}
\end{center}
%
the destination file is determined by a pattern
depending on the current file:
To make this work, the current file must be called
`{\textit{prefix}\hspace{0.2em}\textit{suffix}}'
with \textit{prefix} matching precisely the argument.
Processing is then passed on to the file
`{\textit{dest}\hspace{0.2em}\textit{suffix}}'.
Surely, the same effect is achieved by
directly specifying the
argument `{\textit{dest}\hspace{0.2em}\textit{suffix}}'
in the first form.
However, that requires to set up a different file
for each child. With the alternative form of the command
all these files can have exactly the same content
which simplifies setting them up and maintaining them.

For example, the following file |draft.tex|
with a compilation flag |\version| as described in \secref{sec:flags}
compiles the main document as a draft:
%
\begin{center}
\begin{tabular}{l}
|\def\version{draft}|\\
|\input{childdoc.def}|\\
|\childdocforward{|\textit{main}|}|
\end{tabular}
\end{center}
%
Likewise, the following files |final|\textit{nn}|.tex|
compile the final version of the child document
|child|\textit{nn}|.tex|:
%
\begin{center}
\begin{tabular}{l}
|\def\version{final}|\\
|\input{childdoc.def}|\\
|\childdocforwardprefix{final}{child}|
\end{tabular}
\end{center}
%

Note that when several versions of a main file and/or of each child file
are to be generated, it may be convenient to set up a |Makefile| or
shell script to automatise the process.

%%%%%%%%%%%%%%%%%%%%%%%%%%%%%%%%%%%%%%%%%%%%%%%%%%%%%%%%%%%%%%%%%%%%%%%%%%%%%%%%
\subsection{Command Line Processing}
\label{sec:commandline}

The effect of redirection files can also be achieved by invoking
the \LaTeX{} compiler with a more elaborate command line.
Most conveniently this should be done as part
of a shell script or a |Makefile|.

When using \textsf{childdoc} in the main file, the following
command lines effectively perform a redirection
(note that depending on the shell being used,
backslashes may have to be doubled: `|\|' $\to$ `|\\|'):
%
\begin{center}
|... -jobname "|\textit{target}|" |\\|"|[\textit{flags}]%
|\input{childdoc.def}\childdocforward[|\textit{main}|]{|\textit{dest}|}"|
\end{center}
%
Here \textit{target} is the name of the output file,
\textit{main} is the name of the main file
and \textit{dest} is the name of the main or child file to be processed
(all filenames without extensions).
The optional argument \textit{main} can be omitted
if \textit{main} matches \textit{dest}.
Optionally, compilation \textit{flags} can be defined via |\def| commands.
This command line makes the \TeX{} engine believe
it is compiling the file \textit{target}
whose content is specified as the latter parameter.
The provided code then forwards the processing to
\textit{main} or \textit{dest} as described in \secref{sec:forward}.

%%%%%%%%%%%%%%%%%%%%%%%%%%%%%%%%%%%%%%%%%%%%%%%%%%%%%%%%%%%%%%%%%%%%%%%%%%%%%%%%
\subsection{Include by Input}
\label{sec:input}

Including child documents by |\include| has some restrictions by design.
Most notably, the content of a child document always occupies
its own set of pages; pages cannot be shared between child documents.
Usually, this behaviour makes perfect sense
because each child document contain an essential part of the document.
However, in some situations it may be desirable to compose
a document from a collection of parts
without having mandatory page breaks between then.
For this case, the package
provides a mechanism to include parts
by |\input| which can also be processed individually.
However, by construction this mechanism
requires manual handling of the content to be output.

%%%%%%%%%%%%%%%%%%%%%%%%%%%%%%%%%%%%%%%%
\DescribeMacro{\ifchilddocmanual}
The main file should be prepared as usual, see \secref{sec:include}.
However, the document body must make a distinction
between processing of an individual part and of the main document, e.g.:
%
\begin{center}
\begin{tabular}{l}
|\ifchilddocmanual|\\
|\input{\childdocname}|\\
|\||else|\\
\textit{document body with }|\input{|\textit{part}|}|\\
|\||fi|
\end{tabular}
\end{center}
%
The conditional |\ifchilddocmanual| is true whenever
a part to be included by |\input| is being compiled,
and the name of the part is stored in |\childdocname|.

%%%%%%%%%%%%%%%%%%%%%%%%%%%%%%%%%%%%%%%%
\DescribeMacro{\childdocby}
Each part to be included by |\input| should start with:
%
\begin{center}
\begin{tabular}{l}
|\input{childdoc.def}|\\
|\childdocby{|\textit{main}|}|\\
\end{tabular}
\end{center}
%
The directive |\childdocby| is similar to |\childdocof|
described in \secref{sec:include},
but the subsequent selection of content must be done manually.
To that end, both |\ifchilddoc| and |\ifchilddocmanual|
will be true upon processing of a part,
and the name of the part is stored in |\childdocname|.
Note that |\jobname| will be set to the filename of the current part
so that each part receives an individual |.aux| file
that does not interfere with the |.aux| file(s) of the main document.
This behaviour can be altered by the alternative form
|\childdocby[*]{|\textit{main}|}| (with a non-empty optional argument)
which uses the |.aux| file of the main document
by setting |\jobname| to \textit{main}.

%%%%%%%%%%%%%%%%%%%%%%%%%%%%%%%%%%%%%%%%%%%%%%%%%%%%%%%%%%%%%%%%%%%%%%%%%%%%%%%%
\subsection{Driver Development}
\label{sec:driver}

The \textsf{childdoc} mechanism can also be use for the development
of definition files such as \LaTeX{} styles or classes.
This case differs from the above setup with multiple parts
included by |\include| in that no |\includeonly| should be invoked.
This can be achieved by starting the include file
(before |\ProvidesPackage|) with:
%
\begin{center}
\begin{tabular}{l}
|\input{childdoc.def}|\\
|\childdocforward{|\textit{main}|}|\\
\end{tabular}
\end{center}
%
or alternatively with:
%
\begin{center}
\begin{tabular}{l}
|\input{childdoc.def}|\\
|\childdocby{|\textit{main}|}|\\
\end{tabular}
\end{center}
%
Both forms have slightly different effects as described above.
The main file is prepared as usual, see \secref{sec:include}.

%%%%%%%%%%%%%%%%%%%%%%%%%%%%%%%%%%%%%%%%%%%%%%%%%%%%%%%%%%%%%%%%%%%%%%%%%%%%%%%%
\subsection{Legacy Detection}
\label{sec:detection}

The directive |\childdocmain| in the main file can detect
whether the complete document or merely a child is to be compiled
even without using the directive |\childdocof|.
This method is deprecated because it is less robust
and there is no compelling reason to use it;
it is merely provided for backward compatibility
and it may be removed in future versions.

If the detection mechanism is to be used,
it is mandatory to correctly specify
the filename of the main file as the argument of |\childdocmain|:
%
\begin{center}
\begin{tabular}{l}
|\input{childdoc.def}|\\
|\childdocmain{|\textit{main}|}|\\
\end{tabular}
\end{center}
%
If |\jobname| does not match the argument \textit{main} of |\childdocmain|,
it is assumed that |\jobname| points to the child file to be compiled.
When using |\childdocmain| with the main file specified as argument,
it suffices to start a child file
with just |\input{|\textit{main}|}|
without loading of the package and using |\childdocof|.
If instead all processing is done
with the appropriate \textsf{childdoc} directives,
the argument of \textit{main} of |\childdocmain| can be empty.

An alternative version of the command line processing described
in \secref{sec:commandline} using the detection mechanism reads:
%
\begin{center}
|... -jobname "|\textit{target}|" "|[\textit{flags}]%
[|\def\jobname{|\textit{dest}|}|]|\input{|\textit{main}|}"|
\end{center}

%%%%%%%%%%%%%%%%%%%%%%%%%%%%%%%%%%%%%%%%%%%%%%%%%%%%%%%%%%%%%%%%%%%%%%%%%%%%%%%%
\subsection{Manual Code}
\label{sec:manual}

In case one cannot be certain whether the definitions file |childdoc.def|
is installed on the target \TeX{} distribution
and one prefers not to ship it,
it is conceivable to paste a few relevant commands into the sources.

To that end, drop all statements |\input{childdoc.def}|
and perform the replacements as outlined below.
Instead of |\childdocmain{|\textit{main}|}| add the following code
to the top of the main file:
%
\begin{center}
\begin{tabular}{l}
|\||ifdefined\childdocname\endinput\||fi\newif\ifchilddoc|\\
|\edef\childdocname{\scantokens\expandafter{\jobname\noexpand}}|\\
|\def\childdocmain{|\textit{main}|}\||ifx\childdocmain\childdocname\||else|\\
|\childdoctrue\includeonly{\childdocname}\let\jobname\childdocmain\||fi|\\
\end{tabular}
\end{center}
%
Instead of |\childdocof{|\textit{main}|}| just include the main file
at the top of each child file:
%
\begin{center}
|\input{|\textit{main}|}|
\end{center}
%
A simple redirection |\childdocforward{|\textit{dest}|}| is achieved by:
%
\begin{center}
|\def\jobname{|\textit{dest}|}\input{\jobname}|
\end{center}
%
The redirection with prefix
|\childdocforwardprefix[|\textit{prefix}|]{|\textit{dest}|}|
is accomplished by:
%
\begin{center}
\begin{tabular}{l}
|{\edef\jobname{\scantokens\expandafter{\jobname\noexpand}}|\\
|\def\redirectjob |\textit{prefix}|#1~~~{\gdef\jobname{|\textit{dest}|#1}}|\\
|\expandafter\redirectjob\jobname~~~}\input{\jobname}|
\end{tabular}
\end{center}

In an alternative approach,
child documents can be compiled by a specific command line
without additional code or specific definitions:
%
\begin{center}
|... -jobname "|\textit{target}|" "|[\textit{flags}]%
|\includeonly{|\textit{dest}|}\input{|\textit{main}|}"|
\end{center}
%

%%%%%%%%%%%%%%%%%%%%%%%%%%%%%%%%%%%%%%%%%%%%%%%%%%%%%%%%%%%%%%%%%%%%%%%%%%%%%%%%
%%%%%%%%%%%%%%%%%%%%%%%%%%%%%%%%%%%%%%%%%%%%%%%%%%%%%%%%%%%%%%%%%%%%%%%%%%%%%%%%
\section{Information}

%%%%%%%%%%%%%%%%%%%%%%%%%%%%%%%%%%%%%%%%%%%%%%%%%%%%%%%%%%%%%%%%%%%%%%%%%%%%%%%%
\subsection{Copyright}

Copyright \copyright{} 2017--2018 Niklas Beisert

This work may be distributed and/or modified under the
conditions of the \LaTeX{} Project Public License, either version 1.3
of this license or (at your option) any later version.
The latest version of this license is in
  \url{http://www.latex-project.org/lppl.txt}
and version 1.3 or later is part of all distributions of \LaTeX{}
version 2005/12/01 or later.

This work has the LPPL maintenance status `maintained'.

The Current Maintainer of this work is Niklas Beisert.

This work consists of the files |README.txt|, |childdoc.ins| and |childdoc.dtx|
as well as the derived files |childdoc.def|, |cdocsamp.tex|
with |cdocsch1.tex|, |cdocsch2.tex|, |cdocspt3.tex|, |cdocspt4.tex|,
|cdocsdrf.tex|, |cdocsfn1.tex|, |cdocsfn2.tex|
as well as |childdoc.pdf|.

%%%%%%%%%%%%%%%%%%%%%%%%%%%%%%%%%%%%%%%%%%%%%%%%%%%%%%%%%%%%%%%%%%%%%%%%%%%%%%%%
\subsection{Files and Installation}

The package consists of the files:
%
\begin{center}
\begin{tabular}{ll}
    |README.txt|   & readme file \\
    |childdoc.ins| & installation file \\
    |childdoc.dtx| & source file \\
    |childdoc.def| & definition file \\
    |cdocsamp.tex| & sample main file \\
    |cdocsch1.tex| & sample include file \\
    |cdocsch2.tex| & sample include file \\
    |cdocspt3.tex| & sample part file \\
    |cdocspt4.tex| & sample part file \\
    |cdocsdrf.tex| & sample redirection file \\
    |cdocsfn1.tex| & sample redirection file \\
    |cdocsfn2.tex| & sample redirection file \\
    |childdoc.pdf| & manual
\end{tabular}
\end{center}
%
The distribution consists of the files
|README.txt|, |childdoc.ins| and |childdoc.dtx|.
%
\begin{itemize}
\item
Run (pdf)\LaTeX{} on |childdoc.dtx|
to compile the manual |childdoc.pdf| (this file).
\item
Run \LaTeX{} on |childdoc.ins| to create the definitions file |childdoc.def|
and the sample |cdocsamp.tex| with include files
|cdocsch1.tex|, |cdocsch2.tex|, |cdocspt3.tex|, |cdocspt4.tex|,
|cdocsdrf.tex|, |cdocsfn1.tex|, |cdocsfn2.tex|.
Then copy the file |childdoc.def| to an appropriate directory of your \LaTeX{}
distribution, e.g.\ \textit{texmf-root}|/tex/latex/childdoc|.
\end{itemize}

%%%%%%%%%%%%%%%%%%%%%%%%%%%%%%%%%%%%%%%%%%%%%%%%%%%%%%%%%%%%%%%%%%%%%%%%%%%%%%%%
\subsection{Related CTAN Packages}

There are several other packages which offer a similar functionality:
%
\begin{itemize}
\item
The packages
\href{http://ctan.org/pkg/docmute}{\textsf{docmute}},
\href{http://ctan.org/pkg/includex}{\textsf{includex}} and
\href{http://ctan.org/pkg/standalone}{\textsf{standalone}}
provide commands to include only the document body of
a child file thus allowing both files to be compiled individually.
\item
The packages \href{http://ctan.org/pkg/subdocs}{\textsf{subdocs}}
and \href{http://ctan.org/pkg/subfiles}{\textsf{subfiles}}
provide structures in which the main and child documents can be
encapsulated and allowing them to be compiled individually.
The inclusion mechanism is different from the conventional |\include|.
\item
The package \href{http://ctan.org/pkg/combine}{\textsf{combine}}
is an elaborate solution to combine several documents into one.
\end{itemize}
%
See also the CTAN topic \href{http://ctan.org/topic/subdocs}{\textsf{subdocs}}
for further related packages.
The present package differs from the above solutions in that
a document structure constructed with the conventional |\include| mechanism
just needs two extra commands at the top of every file
such that all constituent files can be compiled individually.

%%%%%%%%%%%%%%%%%%%%%%%%%%%%%%%%%%%%%%%%%%%%%%%%%%%%%%%%%%%%%%%%%%%%%%%%%%%%%%%%
%\subsection{Feature Suggestions}
%
%The following is a list of features which may be useful for future
%versions of this package:
%%
%\begin{itemize}
%\item
%\ldots
%\end{itemize}

%%%%%%%%%%%%%%%%%%%%%%%%%%%%%%%%%%%%%%%%%%%%%%%%%%%%%%%%%%%%%%%%%%%%%%%%%%%%%%%%
\subsection{Revision History}

%%%%%%%%%%%%%%%%%%%%%%%%%%%%%%%%%%%%%%%%
\paragraph{v2.0:} 2018/12/30

\begin{itemize}
\item
immediate forward processing
\item
added |\childdocby| mechanism
\item
manual restructured
\end{itemize}

%%%%%%%%%%%%%%%%%%%%%%%%%%%%%%%%%%%%%%%%
\paragraph{v1.6:} 2018/01/17

\begin{itemize}
\item
application for development of include files
\item
corrections to manual
\end{itemize}

%%%%%%%%%%%%%%%%%%%%%%%%%%%%%%%%%%%%%%%%
\paragraph{v1.5:} 2017/05/21

\begin{itemize}
\item
more complete structuring introduced
\item
|\childdocof| introduced
\item
|\childdoc| renamed to |\childdocmain|
\item
|\childredirect| renamed to |\childdocforward| and |\childdocforwardprefix|
and functionality expanded
\end{itemize}

%%%%%%%%%%%%%%%%%%%%%%%%%%%%%%%%%%%%%%%%
\paragraph{v1.0:} 2017/04/27

\begin{itemize}
\item
manual and install package
\item
first version published on CTAN
\end{itemize}

%%%%%%%%%%%%%%%%%%%%%%%%%%%%%%%%%%%%%%%%
\paragraph{v0.6:} 2017/04/26

\begin{itemize}
\item
redirection mechanism added
\end{itemize}

%%%%%%%%%%%%%%%%%%%%%%%%%%%%%%%%%%%%%%%%
\paragraph{v0.5:} 2017/04/26

\begin{itemize}
\item
functionality in definition file
\end{itemize}


%%%%%%%%%%%%%%%%%%%%%%%%%%%%%%%%%%%%%%%%%%%%%%%%%%%%%%%%%%%%%%%%%%%%%%%%%%%%%%%%
%%%%%%%%%%%%%%%%%%%%%%%%%%%%%%%%%%%%%%%%%%%%%%%%%%%%%%%%%%%%%%%%%%%%%%%%%%%%%%%%
%%%%%%%%%%%%%%%%%%%%%%%%%%%%%%%%%%%%%%%%%%%%%%%%%%%%%%%%%%%%%%%%%%%%%%%%%%%%%%%%
\appendix

\settowidth\MacroIndent{\rmfamily\scriptsize 000\ }

 \DocInput{childdoc.dtx}

\end{document}
%</driver>
% \fi
%
% %%%%%%%%%%%%%%%%%%%%%%%%%%%%%%%%%%%%%%%%%%%%%%%%%%%%%%%%%%%%%%%%%%%%%%%%%%%%%%
% %%%%%%%%%%%%%%%%%%%%%%%%%%%%%%%%%%%%%%%%%%%%%%%%%%%%%%%%%%%%%%%%%%%%%%%%%%%%%%
% \section{Sample}
%\iffalse
%<*samplemain>
%\fi
%
% The following presents a sample document
% with two chapters, two parts, a title page,
% a compile flag as well as three forwarding files to set the flag.
% It consists of eight |.tex| files:
% \begin{center}
% \begin{tabular}{ll}
% |cdocsamp.tex|&main file\\
% |cdocsch1.tex|&include file for chapter 1\\
% |cdocsch2.tex|&include file for chapter 2\\
% |cdocspt3.tex|&include file for part 3\\
% |cdocspt4.tex|&include file for part 4\\
% |cdocsdrf.tex|&forwarding file for main file in draft mode\\
% |cdocsfi1.tex|&forwarding file for final version of chapter 1\\
% |cdocsfi2.tex|&forwarding file for final version of chapter 2\\
% \end{tabular}
% \end{center}
% Each of the eight files can be compiled directly by the \LaTeX{} compiler.
%
% %%%%%%%%%%%%%%%%%%%%%%%%%%%%%%%%%%%%%%
% \paragraph{Main File.}
%
% The main file is called |cdocsamp.tex|.
%
% Load the \textsf{childdoc} definitions and
% declare the filename for the main document:
%    \begin{macrocode}
\input{childdoc.def}
\childdocmain{}
%    \end{macrocode}

% Optional override for |\version| flag:
%    \begin{macrocode}
%%\ifchilddoc\else\providecommand{\version}{draft}\fi
%    \end{macrocode}

% Define the default values for the |\version| flag
% (|final| for the main file and |draft| for childs):
%    \begin{macrocode}
\ifchilddoc
\providecommand{\version}{draft}
\else
\providecommand{\version}{final}
\fi
%    \end{macrocode}

% Load the standard document class:
%    \begin{macrocode}
\documentclass[12pt]{article}
%    \end{macrocode}

% Start the document body:
%    \begin{macrocode}
\begin{document}
%    \end{macrocode}

% Declare a title page.
% Print title, part of document being processed and version flag:
%    \begin{macrocode}
\addtocounter{page}{-1}
\begin{center}
{\LARGE\bfseries{}childdoc example\par}
\vspace{1cm}
\ifchilddoc
\ifchilddocmanual part\else chapter\fi:
`\childdocname' of `\childdocjob'\par
\else
main document: `\childdocjob'\par
\fi
version: \version\par
\end{center}
\newpage
%    \end{macrocode}

% Manually include selected file,
% otherwise process as usual:
%    \begin{macrocode}
\ifchilddocmanual
\section*{part `\childdocname'}
\input{\childdocname}
\else
%    \end{macrocode}

% Include the two chapters:
%    \begin{macrocode}
\include{cdocsch1}
\include{cdocsch2}
%    \end{macrocode}

% Include the two parts unless only chapters should be displayed:
%    \begin{macrocode}
\ifchilddoc\else
\section{part three}
\input{cdocspt3}
\section{part four}
\input{cdocspt4}
\fi
%    \end{macrocode}

% Process as usual until here:
%    \begin{macrocode}
\fi
%    \end{macrocode}

% End of document body:
%    \begin{macrocode}
\end{document}
%    \end{macrocode}
%\iffalse
%</samplemain>
%\fi
%
% %%%%%%%%%%%%%%%%%%%%%%%%%%%%%%%%%%%%%%
% \paragraph{Chapter Include Files.}
%
% The include files are called |cdocsch1.tex| and |cdocsch2.tex|.
%
%\iffalse
%<*samplechap1|samplechap2>
%\fi

% Optional override for |\version| flag:
%    \begin{macrocode}
%%\providecommand{\version}{final}
%    \end{macrocode}

% Include the main document:
%    \begin{macrocode}
\input{childdoc.def}
\childdocof{cdocsamp}
%    \end{macrocode}

%\iffalse
%</samplechap1|samplechap2>
%\fi
%
%\iffalse
%<*samplechap1>
%\fi
% Some text for chapter 1:
%    \begin{macrocode}
\section{one}
some text in chapter one
%    \end{macrocode}

%\iffalse
%</samplechap1>
%\fi
% Some text for chapter 2:
%\iffalse
%<*samplechap2>
%\fi
%    \begin{macrocode}
\section{two}
more text in chapter two
%    \end{macrocode}

%\iffalse
%</samplechap2>
%\fi
%
% %%%%%%%%%%%%%%%%%%%%%%%%%%%%%%%%%%%%%%
% \paragraph{Part Include Files.}
%
% The include files are called |cdocspt3.tex| and |cdocspt4.tex|.
%
%\iffalse
%<*samplepart3|samplepart4>
%\fi

% Optional override for |\version| flag:
%    \begin{macrocode}
%%\providecommand{\version}{final}
%    \end{macrocode}

% Include the main document:
%    \begin{macrocode}
\input{childdoc.def}
\childdocby{cdocsamp}
%    \end{macrocode}

%\iffalse
%</samplepart3|samplepart4>
%\fi
%
%\iffalse
%<*samplepart3>
%\fi
% Some text for part 3:
%    \begin{macrocode}
some text in part three
%    \end{macrocode}

%\iffalse
%</samplepart3>
%\fi
% Some text for part 4:
%\iffalse
%<*samplepart4>
%\fi
%    \begin{macrocode}
more text in part four
%    \end{macrocode}

%\iffalse
%</samplepart4>
%\fi
%
% %%%%%%%%%%%%%%%%%%%%%%%%%%%%%%%%%%%%%%
% \paragraph{Forwarding for a Complete Draft.}
%
% The following forwarding file |cdocsdrf.tex|
% compiles the main document in draft mode:
%\iffalse
%<*sampledraft>
%\fi
%    \begin{macrocode}
\def\version{draft}
\input{childdoc.def}
\childdocforward{cdocsamp}
%    \end{macrocode}

%\iffalse
%</sampledraft>
%\fi
%
% %%%%%%%%%%%%%%%%%%%%%%%%%%%%%%%%%%%%%%
% \paragraph{Forwarding for Final Version of the Chapters.}
%
% The following forwarding files |cdocsfn1.tex| and |cdocsfn2.tex|
% (with identical content)
% compile the final versions of the child documents
% |cdocsch1.tex| and |cdocsch2.tex|, respectively:
%\iffalse
%<*samplefinal>
%\fi
%    \begin{macrocode}
\def\version{final}
\input{childdoc.def}
\childdocforwardprefix[cdocsamp]{cdocsfn}{cdocsch}
%    \end{macrocode}

%\iffalse
%</samplefinal>
%\fi
%
% %%%%%%%%%%%%%%%%%%%%%%%%%%%%%%%%%%%%%%
% \paragraph{Command Line Processing.}
%
% The following three command lines generate the output files
% |cdocscld|, |cdocscl1| and |cdocscl2|
% which should be identical to
% |cdocsdrf|, |cdocsch1| and |cdocsfn2|, respectively:
% \begin{center}
% \begin{tabular}{l}
% |latex -jobname cdocscld \|\\
% |  "\def\version{draft}\input{childdoc.def}\childdocforward{cdocsamp}"|\\
% |latex -jobname cdocscl1 \|\\
% |  "\input{childdoc.def}\childdocforward[cdocsamp]{cdocsch1}"|\\
% |latex -jobname cdocscl2 \|\\
% |  "\def\version{final}\input{childdoc.def}\childdocforward{cdocsch2}"|
% \end{tabular}
% \end{center}
% Note that the trailing backslash on each first line
% merely continues the input to the second line
% (for convenient cut ant paste).
% Furthermore, the command |latex| can be replaced by any
% of its alternative versions such as |pdflatex|.
%
% %%%%%%%%%%%%%%%%%%%%%%%%%%%%%%%%%%%%%%%%%%%%%%%%%%%%%%%%%%%%%%%%%%%%%%%%%%%%%%
% %%%%%%%%%%%%%%%%%%%%%%%%%%%%%%%%%%%%%%%%%%%%%%%%%%%%%%%%%%%%%%%%%%%%%%%%%%%%%%
% \section{Implementation}
%\iffalse
%<*package>
%\fi
%
% This section describes the definitions file |childdoc.def|.

% The definitions cannot be loaded using |\usepackage| or |\RequirePackage|
% which has a mechanism to prevent loading a style file more than once.
% When loading the definitions by means of |\input|
% multiple instances have to be prevented manually:
%\iffalse
%This code needs to be before the `\ProvidesFile' directive
%which is defined at the beginning of this file.
%Therefore it is also placed there and commented out here.
%</package>
%<*discard>
%\fi
%    \begin{macrocode}
\ifdefined\childdocmain\endinput\fi
%    \end{macrocode}
%\iffalse
%</discard>
%<*package>
%\fi
%
% \macro{\ifchilddoc}
% \macro{\ifchilddocmanual}
% The conditional |\ifchilddoc| tells whether a
% child (true) or main (false) document is being compiled.
% The conditional |\ifchilddocmanual| tells whether
% the |\includeonly| mechanism is used (false) or
% the selection of child files must be performed manually (true).
% The definitions initialise to false:
%    \begin{macrocode}
\newif\ifchilddoc
\newif\ifchilddocmanual
%    \end{macrocode}

% \macro{\childdocname}
% \macro{\childdocjob}
% The macro |\childdocname| stores the name of the main document
% to be compiled. The macro |\childdocjob| stores the name of
% the document on which the \LaTeX{} compiler was originally invoked.
% The content of |\jobname| cannot be compared
% to filenames specified in the source due to different catcodes.
% The following code rescans |\jobname|, stores the result
% in |\childdocname| and saves a copy in |\childdocjob|:
%    \begin{macrocode}
\edef\childdocname{\scantokens\expandafter{\jobname\noexpand}}
\let\childdocjob\childdocname
%    \end{macrocode}

% \macro{\childdocdisable}
% The macro |\childdocdisable| prevents the main file
% from being processed more than once.
% At this stage, the main document command |\childdocmain|
% is assumed to be called once again where it should do nothing.
% Any subsequent call to it should prevent
% a secondary processing of the main document
% It overwrites the forwarding commands
% |\childdocof| and |\childdocforward|
% with empty macros to prevent further inclusions of the main document:
%    \begin{macrocode}
\newcommand{\childdocdisable}
{
  \renewcommand{\childdocmain}[1]{\renewcommand{\childdocmain}[1]{\endinput}}
  \renewcommand{\childdocof}[1]{}
  \renewcommand{\childdocby}[2][]{}
  \renewcommand{\childdocforward}[2][]{}
  \renewcommand{\childdocdisable}{}
}
%    \end{macrocode}

% \macro{\childdocmain}
% The macro |\childdocmain| is to be called at the top of the main file
% with nothing or the main filename (without extension) as argument.
% First, it breaks loops.
% If the argument is not empty and does not match |\childdocname|
% (which is set by the first inclusion of |childdoc.def|),
% |\ifchilddoc| is set to true, |\includeonly| is applied to the child file
% and |\jobname| is set to the main file
% (for proper handling of |.aux| files):
%    \begin{macrocode}
\newcommand{\childdocmain}[1]
{
  \childdocdisable\childdocmain{}
  \if?#1?\else
    \begingroup
      \def\childdoctmp{#1}
      \ifx\childdoctmp\childdocname
        \def\childdoctmp{}
      \else
        \def\childdoctmp
        {
          \childdoctrue
          \includeonly{\childdocname}
          \def\childdocjob{#1}
          \def\jobname{#1}
        }
      \fi
      \expandafter
    \endgroup
    \childdoctmp
  \fi
}
%    \end{macrocode}

% \macro{\childdocof}
% The command |\childdocof| redirects
% compilation to the main file |#1|.
%    \begin{macrocode}
\newcommand{\childdocof}[1]
{
  \childdocdisable
  \childdoctrue
  \includeonly{\childdocname}
  \def\jobname{#1}
  \def\childdocjob{#1}
  \input{#1}
}
%    \end{macrocode}

% \macro{\childdocby}
% The command |\childdocby| ....
%    \begin{macrocode}
\newcommand{\childdocby}[2][]
{
  \childdocdisable
  \childdoctrue
  \childdocmanualtrue
  \if?#1?\else
    \def\jobname{#2}
  \fi
  \def\childdocjob{#2}
  \input{#2}
  \endinput
}
%    \end{macrocode}

% \macro{\childdocforward}
% The command |\childdocforward| redirects
% compilation to the main file or
% (if the optional argument is given) a child file.
% Parameters are set as if the main file
% or a child file starting with |\childdocof| was compiled.
% Then compilation is handed over to the main file:
%    \begin{macrocode}
\newcommand{\childdocforward}[2][]
{
  \begingroup
    \if?#1?
      \def\childdoctmp
      {
        \def\childdocname{#2}
        \def\childdocjob{#2}
        \def\jobname{#2}
        \input{#2}
        \endinput
      }
    \else
      \def\childdoctmp
      {
        \childdocdisable
        \def\childdocname{#2}
        \childdoctrue
        \includeonly{#2}
        \def\childdocjob{#1}
        \def\jobname{#1}
        \input{#1}
        \endinput
      }
    \fi
    \expandafter
  \endgroup
  \childdoctmp
}
%    \end{macrocode}

% \macro{\childdocforwardprefix}
% The command |\childdocforwardprefix| redirects
% compilation to the main or a child file by means of a pattern.
% The prefix |#1| in the current filename is replaced by |#2|
% and the suffix of the current filename is kept
% (it is assumed that the filename does not contain the substring `|~~~|'
% which is used as a delimiter).
% Compilation is handed over to the new file by |\childdocforward|:
%    \begin{macrocode}
\newcommand{\childdocforwardprefix}[3][]
{
  \begingroup
    \def\childdocextract #2##1~~~{\def\childdoctmp{\childdocforward[#1]{#3##1}}}
    \expandafter\childdocextract\childdocname~~~
    \expandafter
  \endgroup
  \childdoctmp
}
%    \end{macrocode}

% \macro{\childdoc}
% The deprecated macro |\childdoc| is a legacy version of |\childdocmain|:
%    \begin{macrocode}
\newcommand{\childdoc}{\childdocmain}
%    \end{macrocode}

% \macro{\childdocredirect}
% The deprecated macro |\childdocredirect| is a legacy version
% of |\childdocforward| and |\childdocforwardprefix|:
%    \begin{macrocode}
\newcommand{\childdocredirect}[2][]
{
  \begingroup
    \if?#1?
      \def\childdoctmp{\childdocforward{#2}}
    \else
      \def\childdoctmp{\childdocforwardprefix{#1}{#2}}
    \fi
    \expandafter
  \endgroup
  \childdoctmp
}
%    \end{macrocode}

%\iffalse
%</package>
%\fi
%
\endinput
\childdocforward[cdocsamp]{cdocsch1}"|\\
% |latex -jobname cdocscl2 \|\\
% |  "\def\version{final}% \iffalse
%
% childdoc.dtx Copyright (C) 2017-2018 Niklas Beisert
%
% This work may be distributed and/or modified under the
% conditions of the LaTeX Project Public License, either version 1.3
% of this license or (at your option) any later version.
% The latest version of this license is in
%   http://www.latex-project.org/lppl.txt
% and version 1.3 or later is part of all distributions of LaTeX
% version 2005/12/01 or later.
%
% This work has the LPPL maintenance status `maintained'.
%
% The Current Maintainer of this work is Niklas Beisert.
%
% This work consists of the files childdoc.dtx and childdoc.ins
% and the derived files childdoc.def and cdocsamp.tex with
% cdocsch1.tex, cdocsch2.tex, cdocsdrf.tex, cdocsfn1.tex, cdocsfn2.tex.
%
%<package>\ifdefined\childdocmain\endinput\fi
%<package>\ProvidesFile{childdoc.def}[2018/12/30 v2.0 child document driver]
%<samplemain>\ProvidesFile{cdocsamp.tex}[2018/12/30 v2.0 sample for childdoc]
%<*driver>
%\ProvidesFile{childdoc.drv}[2018/12/30 v2.0 childdoc reference manual file]
\PassOptionsToClass{10pt,a4paper}{article}
\documentclass{ltxdoc}

\usepackage[margin=35mm]{geometry}
\usepackage{hyperref}
\usepackage{hyperxmp}
\usepackage[usenames]{color}

\hypersetup{colorlinks=true}
\hypersetup{pdfstartview=FitH}
\hypersetup{pdfpagemode=UseNone}
\hypersetup{pdfsource={}}
\hypersetup{pdflang={en-UK}}
\hypersetup{pdfcopyright={Copyright 2017-2018 Niklas Beisert.
  This work may be distributed and/or modified under the
  conditions of the LaTeX Project Public License, either version 1.3
  of this license or (at your option) any later version.}}
\hypersetup{pdflicenseurl={http://www.latex-project.org/lppl.txt}}
\hypersetup{pdfcontactaddress={ETH Zurich, ITP, HIT K,
  Wolfgang-Pauli-Strasse 27}}
\hypersetup{pdfcontactpostcode={8093}}
\hypersetup{pdfcontactcity={Zurich}}
\hypersetup{pdfcontactcountry={Switzerland}}
\hypersetup{pdfcontactemail={nbeisert@itp.phys.ethz.ch}}
\hypersetup{pdfcontacturl={http://people.phys.ethz.ch/\xmptilde nbeisert/}}

\newcommand{\secref}[1]{\hyperref[#1]{section \ref*{#1}}}

\parskip1ex
\parindent0pt
\let\olditemize\itemize
\def\itemize{\olditemize\parskip0pt}

\begin{document}

\title{The \textsf{childdoc} Package}
\hypersetup{pdftitle={The childdoc Package}}
\author{Niklas Beisert\\[2ex]
  Institut f\"ur Theoretische Physik\\
  Eidgen\"ossische Technische Hochschule Z\"urich\\
  Wolfgang-Pauli-Strasse 27, 8093 Z\"urich, Switzerland\\[1ex]
  \href{mailto:nbeisert@itp.phys.ethz.ch}
  {\texttt{nbeisert@itp.phys.ethz.ch}}}
\hypersetup{pdfauthor={Niklas Beisert}}
\hypersetup{pdfsubject={Manual for the LaTeX2e Package childdoc}}
\date{30 December 2018, \textsf{v2.0}}
\maketitle

\begin{abstract}\noindent
\textsf{childdoc} is a \LaTeXe{} package
that enables the direct compilation
of document sections included by |\include|
to individual files.
\end{abstract}

\begingroup
\parskip0ex
\tableofcontents
\endgroup

%%%%%%%%%%%%%%%%%%%%%%%%%%%%%%%%%%%%%%%%%%%%%%%%%%%%%%%%%%%%%%%%%%%%%%%%%%%%%%%%
%%%%%%%%%%%%%%%%%%%%%%%%%%%%%%%%%%%%%%%%%%%%%%%%%%%%%%%%%%%%%%%%%%%%%%%%%%%%%%%%
\section{Introduction}

\LaTeX{} provides a mechanism to structure a large document (such as a book)
into a main file and several child files (containing the chapters)
using the |\include| command.
This mechanism is beneficial for documents
which span hundreds of pages in order to
make the source file(s) more manageable.
Moreover, compilation can be restricted to
selected child files by means of the |\includeonly| command.
The latter feature can be used to reduce the compilation time while editing
(this was significantly more useful in the earlier days of \LaTeX{})
or to generate a smaller document which is easier to navigate.
Another application of |\includeonly| is to generate
documents consisting of selected parts of the complete document.

However, there are a few drawbacks of the plain |\include| mechanism:
\begin{itemize}
\item
The child files cannot be compiled on their own,
they can only be compiled via the main file.
A naive editing environment
(such as a text editor with an option
to have the current file processed by \LaTeX)
may require one to switch to the main file before compiling;
attempting to compile the child file produces errors.
\item
The main file must be modified (each time)
to adjust the |\includeonly| command
to the present needs. This easily leaves the main file in a messy state.
\item
The generated document will always carry the filename
of the main document. This is inconvenient if
several child files are to be compiled and
to be kept for distribution.
\end{itemize}

The present package provides a simple interface
to make child files individually compilable by \LaTeX{}.
Compiling a child file then has the same effect as compiling
the main file with an |\includeonly| command
to select the appropriate child.
Moreover the generated document will carry the name of the child
rather than the main file.
This resolves all three above issues.

This feature is meant to make the editing of books,
thesis documents and lecture notes somewhat more convenient.
However, the package can also be used efficiently for
composing a series of documents (such as exercise sheets)
which are typically distributed individually.
It then assists the author in generating the individual documents
(potentially in different versions)
as well as a document containing the collected series.
Another application is in developing style files
or other kinds of included material
where compilation of the style file could redirect
to a sample or test file.

%%%%%%%%%%%%%%%%%%%%%%%%%%%%%%%%%%%%%%%%%%%%%%%%%%%%%%%%%%%%%%%%%%%%%%%%%%%%%%%%
%%%%%%%%%%%%%%%%%%%%%%%%%%%%%%%%%%%%%%%%%%%%%%%%%%%%%%%%%%%%%%%%%%%%%%%%%%%%%%%%
\section{Usage}

First of all, the package \textsf{childdoc} is \emph{not} a standard
\LaTeXe{} |.sty| style file! Therefore it needs to be invoked in
a non-standard way.

%%%%%%%%%%%%%%%%%%%%%%%%%%%%%%%%%%%%%%%%%%%%%%%%%%%%%%%%%%%%%%%%%%%%%%%%%%%%%%%%
\subsection{Included Files}
\label{sec:include}

%%%%%%%%%%%%%%%%%%%%%%%%%%%%%%%%%%%%%%%%
\DescribeMacro{\childdocmain}
To use the package, add the commands
\begin{center}
\begin{tabular}{l}
|\input{childdoc.def}|\\
|\childdocmain{}|\\
\end{tabular}
\end{center}
at the very top of the main \LaTeX{} file,
in particular \emph{before} the |\documentclass| statement!
The argument of |\childdocmain| should be left empty
(but it must be present).

%%%%%%%%%%%%%%%%%%%%%%%%%%%%%%%%%%%%%%%%
\DescribeMacro{\childdocof}
Furthermore, add the commands
\begin{center}
\begin{tabular}{l}
|\input{childdoc.def}|\\
|\childdocof{|\textit{main}|}|\\
\end{tabular}
\end{center}
at the top of every child file \textit{child}
which is included by |\include{|\textit{child}|}|
from within the main file
(or at least for those files to be compiled individually).
The argument \textit{main} must be the filename of the main file.

There are a couple of
considerations in setting up the main and child documents:

%%%%%%%%%%%%%%%%%%%%%%%%%%%%%%%%%%%%%%%%
\paragraph{Restrictions.}

Please note the following restrictions:
\begin{itemize}
\item
|\childdocmain| must be called with one argument \textit{main}
to ensure compatibility with earlier version of the package.
It must either be empty (|\childdocmain{}|)
or precisely match the filename of the main file in which it is specified.
See \secref{sec:detection} for further information.
\item
The filename \textit{main} must be specified without the |.tex| extension.
\item
The filename \textit{main} is case sensitive
(even in case-insensitive file systems)
due to internal string comparison.
\item
The argument \textit{main} should be fully expanded, it cannot be a macro.
\item
Subdirectories and special characters should be avoided in filenames.
\item
The command |\childdocmain{|\textit{main}|}| must be followed by a whitespace.
It should not be followed immediately by another command
or by a comment mark `|%|'.
This is because the \TeX{} parser reads the token immediately following
the argument of |\childdocmain| and puts it
at the beginning of every child section;
however, a white\-space is ignored.
\end{itemize}

%%%%%%%%%%%%%%%%%%%%%%%%%%%%%%%%%%%%%%%%
\paragraph{Content of Main File.}

It is advisable to place all content in the child files included by |\include|.
Any output contained in the main file will appear in all child documents
unless suppressed manually;
it cannot be suppressed automatically by the |\includeonly| directive
and thus should normally be avoided.
A method to include some content in the main file
by means of conditional processing is described in \secref{sec:conditional}.

%%%%%%%%%%%%%%%%%%%%%%%%%%%%%%%%%%%%%%%%
\paragraph{Page Numbering.}

When only a part of the document is compiled,
the appropriate numbering of pages
(as well as other status parameters)
is determined from the |.aux| files.
The latter contain information from previous passes.
However this information needs to propagate through
all intermediate child documents.
Therefore the page numbering in child documents may well
be inconsistent until the complete document is compiled at least once.

A useful (if unconventional) way to always ensure a consistent
page numbering is to restart the numbering in each child document
and denote the pages by `\textit{child}|.|\textit{page}'
where \textit{child} represents the chapter/section number of the child file.
This can be achieved by the command
|\numberwithin{page}{|\textit{child}|}|
of the \textsf{amsmath} package
where \textit{child} can be |chapter| or |section|
depending on the chosen structuring.
Alternatively, one can modify the macro |\thepage| appropriately
and reset the counter |page| at the start of each child file.

%%%%%%%%%%%%%%%%%%%%%%%%%%%%%%%%%%%%%%%%%%%%%%%%%%%%%%%%%%%%%%%%%%%%%%%%%%%%%%%%
\subsection{Conditional Processing}
\label{sec:conditional}

The package provides a mechanism to compile different versions
of a document. To customise the versions further some conditional processing
can come in handy to distinguish which version is being compiled.
The package provides two macros to describe the compilation context:

%%%%%%%%%%%%%%%%%%%%%%%%%%%%%%%%%%%%%%%%
\DescribeMacro{\ifchilddoc}
The conditional |\ifchilddoc| distinguishes between the compilation of
child documents and the main document:
%
\begin{center}
|\ifchilddoc |\textit{child-code}| |[|\||else |\textit{main-code}]| \||fi|
\end{center}

%%%%%%%%%%%%%%%%%%%%%%%%%%%%%%%%%%%%%%%%
\DescribeMacro{\childdocname}
\DescribeMacro{\childdocjob}
The macro |\childdocname| contains the filename (without extension)
of the main or child file being processed.
Note that |\childdocjob| will always contain the name of the main file.

%%%%%%%%%%%%%%%%%%%%%%%%%%%%%%%%%%%%%%%%
\paragraph{Title Page.}

Conditional processing can be used to include a title or banner page
in the main document when proper precautions are taken.
Importantly, the code in the main file should ensure that the page counter
(as well as other status parameters which are stored in the |.aux| files)
takes the same value after the conditional processing.
Otherwise the page numbers may take divergent values
depending on which part is compiled.

For example, a title page could be declared by:
%
\begin{center}
\begin{tabular}{l}
|\ifchilddoc\||else|\\
|\addtocounter{page}{-1}|\\
\textit{code for title page}\\
|\newpage|\\
|\||fi|
\end{tabular}
\end{center}
%
A banner page for the child documents can be generated by:
%
\begin{center}
\begin{tabular}{l}
|\ifchilddoc|\\
|\addtocounter{page}{-1}|\\
\textit{code for banner page}\\
|\newpage|\\
|\||fi|
\end{tabular}
\end{center}
%
Here one could write a message such as:
\begin{center}
|This is the part \childdocname{} of \childdocjob{}.|
\end{center}

%%%%%%%%%%%%%%%%%%%%%%%%%%%%%%%%%%%%%%%%%%%%%%%%%%%%%%%%%%%%%%%%%%%%%%%%%%%%%%%%
\subsection{Flags}
\label{sec:flags}

The package makes it easy to generate different versions
of the main or child documents.
To this end compilation flags can be defined
and assigned different default values.
They will be particularly useful in conjunction
with the forwarding mechanism described in \secref{sec:forward}.

For example, it may be useful to have a flag |\version|
which can be set to |draft| or |final|.
The document source will contain some conditional code
depending on the value of |\version|.
Suppose further, the flag should default to |final| for the main file
and to |draft| for child files
which is a natural assignment for editing the document.
This is achieved by placing the following code
in the preamble of the main document
(below the |\childdocmain| directive):
%
\begin{center}
\begin{tabular}{l}
|\ifchilddoc|\\
|\providecommand{\version}{draft}|\\
|\||else|\\
|\providecommand{\version}{final}|\\
|\||fi|
\end{tabular}
\end{center}
%
The definition by |\providecommand| makes sure
that previous definitions are not overwritten.
Further statements |\providecommand{\version}{...}|
can thus be added before the above code to override it.

For the main file, one might add a line
(between |\childdocmain| and the above block)
%
\begin{center}
|%\ifchilddoc\||else\providecommand{\version}{draft}\||fi|
\end{center}
%
which can be uncommented to produce a draft version.
Likewise one can add a line to the very top of a child file
(above the |\childdocof{|\textit{main}|}| directive)
%
\begin{center}
|%\providecommand{\version}{final}|
\end{center}
%
which can be uncommented to produce the final version of this child document.

%%%%%%%%%%%%%%%%%%%%%%%%%%%%%%%%%%%%%%%%%%%%%%%%%%%%%%%%%%%%%%%%%%%%%%%%%%%%%%%%
\subsection{Forwarding}
\label{sec:forward}

Different versions of the main or child documents
using compilation flags as described in \secref{sec:flags}
can be (permanently) stored in different files
for convenient compilation, viewing and distribution.
To this end, the package defines a command
to pass on compilation to a different file:

%%%%%%%%%%%%%%%%%%%%%%%%%%%%%%%%%%%%%%%%
\DescribeMacro{\childdocforward}
The command |\childdocforward| redirects processing to
another source file:
%
\begin{center}
\begin{tabular}{l}
|\input{childdoc.def}|\\
|\childdocforward[|\textit{main}|]{|\textit{dest}|}|\\
\end{tabular}
\end{center}
%
The argument \textit{dest} is the destination file
(without extension).
It should be the main file or one of the child files.
Note that further \textsf{childdoc} directives
such as |\childdocof| and |\childdocforward|
in the indicated file will be processed in this form.
The optional argument \textit{main}
passes on directly to the main file \textit{main}
while pretending to compile the child \textit{dest}.
This form behaves as if \textit{dest}
issues |\childdocof{|\textit{main}|}| right away,
and no further \textsf{childdoc} directives will be processed.

%%%%%%%%%%%%%%%%%%%%%%%%%%%%%%%%%%%%%%%%
\DescribeMacro{\...prefix}
In the alternative form |\childdocforwardprefix|,
%
\begin{center}
\begin{tabular}{l}
|\input{childdoc.def}|\\
|\childdocforwardprefix[|\textit{main}|]{|\textit{prefix}|}{|\textit{dest}|}|
\end{tabular}
\end{center}
%
the destination file is determined by a pattern
depending on the current file:
To make this work, the current file must be called
`{\textit{prefix}\hspace{0.2em}\textit{suffix}}'
with \textit{prefix} matching precisely the argument.
Processing is then passed on to the file
`{\textit{dest}\hspace{0.2em}\textit{suffix}}'.
Surely, the same effect is achieved by
directly specifying the
argument `{\textit{dest}\hspace{0.2em}\textit{suffix}}'
in the first form.
However, that requires to set up a different file
for each child. With the alternative form of the command
all these files can have exactly the same content
which simplifies setting them up and maintaining them.

For example, the following file |draft.tex|
with a compilation flag |\version| as described in \secref{sec:flags}
compiles the main document as a draft:
%
\begin{center}
\begin{tabular}{l}
|\def\version{draft}|\\
|\input{childdoc.def}|\\
|\childdocforward{|\textit{main}|}|
\end{tabular}
\end{center}
%
Likewise, the following files |final|\textit{nn}|.tex|
compile the final version of the child document
|child|\textit{nn}|.tex|:
%
\begin{center}
\begin{tabular}{l}
|\def\version{final}|\\
|\input{childdoc.def}|\\
|\childdocforwardprefix{final}{child}|
\end{tabular}
\end{center}
%

Note that when several versions of a main file and/or of each child file
are to be generated, it may be convenient to set up a |Makefile| or
shell script to automatise the process.

%%%%%%%%%%%%%%%%%%%%%%%%%%%%%%%%%%%%%%%%%%%%%%%%%%%%%%%%%%%%%%%%%%%%%%%%%%%%%%%%
\subsection{Command Line Processing}
\label{sec:commandline}

The effect of redirection files can also be achieved by invoking
the \LaTeX{} compiler with a more elaborate command line.
Most conveniently this should be done as part
of a shell script or a |Makefile|.

When using \textsf{childdoc} in the main file, the following
command lines effectively perform a redirection
(note that depending on the shell being used,
backslashes may have to be doubled: `|\|' $\to$ `|\\|'):
%
\begin{center}
|... -jobname "|\textit{target}|" |\\|"|[\textit{flags}]%
|\input{childdoc.def}\childdocforward[|\textit{main}|]{|\textit{dest}|}"|
\end{center}
%
Here \textit{target} is the name of the output file,
\textit{main} is the name of the main file
and \textit{dest} is the name of the main or child file to be processed
(all filenames without extensions).
The optional argument \textit{main} can be omitted
if \textit{main} matches \textit{dest}.
Optionally, compilation \textit{flags} can be defined via |\def| commands.
This command line makes the \TeX{} engine believe
it is compiling the file \textit{target}
whose content is specified as the latter parameter.
The provided code then forwards the processing to
\textit{main} or \textit{dest} as described in \secref{sec:forward}.

%%%%%%%%%%%%%%%%%%%%%%%%%%%%%%%%%%%%%%%%%%%%%%%%%%%%%%%%%%%%%%%%%%%%%%%%%%%%%%%%
\subsection{Include by Input}
\label{sec:input}

Including child documents by |\include| has some restrictions by design.
Most notably, the content of a child document always occupies
its own set of pages; pages cannot be shared between child documents.
Usually, this behaviour makes perfect sense
because each child document contain an essential part of the document.
However, in some situations it may be desirable to compose
a document from a collection of parts
without having mandatory page breaks between then.
For this case, the package
provides a mechanism to include parts
by |\input| which can also be processed individually.
However, by construction this mechanism
requires manual handling of the content to be output.

%%%%%%%%%%%%%%%%%%%%%%%%%%%%%%%%%%%%%%%%
\DescribeMacro{\ifchilddocmanual}
The main file should be prepared as usual, see \secref{sec:include}.
However, the document body must make a distinction
between processing of an individual part and of the main document, e.g.:
%
\begin{center}
\begin{tabular}{l}
|\ifchilddocmanual|\\
|\input{\childdocname}|\\
|\||else|\\
\textit{document body with }|\input{|\textit{part}|}|\\
|\||fi|
\end{tabular}
\end{center}
%
The conditional |\ifchilddocmanual| is true whenever
a part to be included by |\input| is being compiled,
and the name of the part is stored in |\childdocname|.

%%%%%%%%%%%%%%%%%%%%%%%%%%%%%%%%%%%%%%%%
\DescribeMacro{\childdocby}
Each part to be included by |\input| should start with:
%
\begin{center}
\begin{tabular}{l}
|\input{childdoc.def}|\\
|\childdocby{|\textit{main}|}|\\
\end{tabular}
\end{center}
%
The directive |\childdocby| is similar to |\childdocof|
described in \secref{sec:include},
but the subsequent selection of content must be done manually.
To that end, both |\ifchilddoc| and |\ifchilddocmanual|
will be true upon processing of a part,
and the name of the part is stored in |\childdocname|.
Note that |\jobname| will be set to the filename of the current part
so that each part receives an individual |.aux| file
that does not interfere with the |.aux| file(s) of the main document.
This behaviour can be altered by the alternative form
|\childdocby[*]{|\textit{main}|}| (with a non-empty optional argument)
which uses the |.aux| file of the main document
by setting |\jobname| to \textit{main}.

%%%%%%%%%%%%%%%%%%%%%%%%%%%%%%%%%%%%%%%%%%%%%%%%%%%%%%%%%%%%%%%%%%%%%%%%%%%%%%%%
\subsection{Driver Development}
\label{sec:driver}

The \textsf{childdoc} mechanism can also be use for the development
of definition files such as \LaTeX{} styles or classes.
This case differs from the above setup with multiple parts
included by |\include| in that no |\includeonly| should be invoked.
This can be achieved by starting the include file
(before |\ProvidesPackage|) with:
%
\begin{center}
\begin{tabular}{l}
|\input{childdoc.def}|\\
|\childdocforward{|\textit{main}|}|\\
\end{tabular}
\end{center}
%
or alternatively with:
%
\begin{center}
\begin{tabular}{l}
|\input{childdoc.def}|\\
|\childdocby{|\textit{main}|}|\\
\end{tabular}
\end{center}
%
Both forms have slightly different effects as described above.
The main file is prepared as usual, see \secref{sec:include}.

%%%%%%%%%%%%%%%%%%%%%%%%%%%%%%%%%%%%%%%%%%%%%%%%%%%%%%%%%%%%%%%%%%%%%%%%%%%%%%%%
\subsection{Legacy Detection}
\label{sec:detection}

The directive |\childdocmain| in the main file can detect
whether the complete document or merely a child is to be compiled
even without using the directive |\childdocof|.
This method is deprecated because it is less robust
and there is no compelling reason to use it;
it is merely provided for backward compatibility
and it may be removed in future versions.

If the detection mechanism is to be used,
it is mandatory to correctly specify
the filename of the main file as the argument of |\childdocmain|:
%
\begin{center}
\begin{tabular}{l}
|\input{childdoc.def}|\\
|\childdocmain{|\textit{main}|}|\\
\end{tabular}
\end{center}
%
If |\jobname| does not match the argument \textit{main} of |\childdocmain|,
it is assumed that |\jobname| points to the child file to be compiled.
When using |\childdocmain| with the main file specified as argument,
it suffices to start a child file
with just |\input{|\textit{main}|}|
without loading of the package and using |\childdocof|.
If instead all processing is done
with the appropriate \textsf{childdoc} directives,
the argument of \textit{main} of |\childdocmain| can be empty.

An alternative version of the command line processing described
in \secref{sec:commandline} using the detection mechanism reads:
%
\begin{center}
|... -jobname "|\textit{target}|" "|[\textit{flags}]%
[|\def\jobname{|\textit{dest}|}|]|\input{|\textit{main}|}"|
\end{center}

%%%%%%%%%%%%%%%%%%%%%%%%%%%%%%%%%%%%%%%%%%%%%%%%%%%%%%%%%%%%%%%%%%%%%%%%%%%%%%%%
\subsection{Manual Code}
\label{sec:manual}

In case one cannot be certain whether the definitions file |childdoc.def|
is installed on the target \TeX{} distribution
and one prefers not to ship it,
it is conceivable to paste a few relevant commands into the sources.

To that end, drop all statements |\input{childdoc.def}|
and perform the replacements as outlined below.
Instead of |\childdocmain{|\textit{main}|}| add the following code
to the top of the main file:
%
\begin{center}
\begin{tabular}{l}
|\||ifdefined\childdocname\endinput\||fi\newif\ifchilddoc|\\
|\edef\childdocname{\scantokens\expandafter{\jobname\noexpand}}|\\
|\def\childdocmain{|\textit{main}|}\||ifx\childdocmain\childdocname\||else|\\
|\childdoctrue\includeonly{\childdocname}\let\jobname\childdocmain\||fi|\\
\end{tabular}
\end{center}
%
Instead of |\childdocof{|\textit{main}|}| just include the main file
at the top of each child file:
%
\begin{center}
|\input{|\textit{main}|}|
\end{center}
%
A simple redirection |\childdocforward{|\textit{dest}|}| is achieved by:
%
\begin{center}
|\def\jobname{|\textit{dest}|}\input{\jobname}|
\end{center}
%
The redirection with prefix
|\childdocforwardprefix[|\textit{prefix}|]{|\textit{dest}|}|
is accomplished by:
%
\begin{center}
\begin{tabular}{l}
|{\edef\jobname{\scantokens\expandafter{\jobname\noexpand}}|\\
|\def\redirectjob |\textit{prefix}|#1~~~{\gdef\jobname{|\textit{dest}|#1}}|\\
|\expandafter\redirectjob\jobname~~~}\input{\jobname}|
\end{tabular}
\end{center}

In an alternative approach,
child documents can be compiled by a specific command line
without additional code or specific definitions:
%
\begin{center}
|... -jobname "|\textit{target}|" "|[\textit{flags}]%
|\includeonly{|\textit{dest}|}\input{|\textit{main}|}"|
\end{center}
%

%%%%%%%%%%%%%%%%%%%%%%%%%%%%%%%%%%%%%%%%%%%%%%%%%%%%%%%%%%%%%%%%%%%%%%%%%%%%%%%%
%%%%%%%%%%%%%%%%%%%%%%%%%%%%%%%%%%%%%%%%%%%%%%%%%%%%%%%%%%%%%%%%%%%%%%%%%%%%%%%%
\section{Information}

%%%%%%%%%%%%%%%%%%%%%%%%%%%%%%%%%%%%%%%%%%%%%%%%%%%%%%%%%%%%%%%%%%%%%%%%%%%%%%%%
\subsection{Copyright}

Copyright \copyright{} 2017--2018 Niklas Beisert

This work may be distributed and/or modified under the
conditions of the \LaTeX{} Project Public License, either version 1.3
of this license or (at your option) any later version.
The latest version of this license is in
  \url{http://www.latex-project.org/lppl.txt}
and version 1.3 or later is part of all distributions of \LaTeX{}
version 2005/12/01 or later.

This work has the LPPL maintenance status `maintained'.

The Current Maintainer of this work is Niklas Beisert.

This work consists of the files |README.txt|, |childdoc.ins| and |childdoc.dtx|
as well as the derived files |childdoc.def|, |cdocsamp.tex|
with |cdocsch1.tex|, |cdocsch2.tex|, |cdocspt3.tex|, |cdocspt4.tex|,
|cdocsdrf.tex|, |cdocsfn1.tex|, |cdocsfn2.tex|
as well as |childdoc.pdf|.

%%%%%%%%%%%%%%%%%%%%%%%%%%%%%%%%%%%%%%%%%%%%%%%%%%%%%%%%%%%%%%%%%%%%%%%%%%%%%%%%
\subsection{Files and Installation}

The package consists of the files:
%
\begin{center}
\begin{tabular}{ll}
    |README.txt|   & readme file \\
    |childdoc.ins| & installation file \\
    |childdoc.dtx| & source file \\
    |childdoc.def| & definition file \\
    |cdocsamp.tex| & sample main file \\
    |cdocsch1.tex| & sample include file \\
    |cdocsch2.tex| & sample include file \\
    |cdocspt3.tex| & sample part file \\
    |cdocspt4.tex| & sample part file \\
    |cdocsdrf.tex| & sample redirection file \\
    |cdocsfn1.tex| & sample redirection file \\
    |cdocsfn2.tex| & sample redirection file \\
    |childdoc.pdf| & manual
\end{tabular}
\end{center}
%
The distribution consists of the files
|README.txt|, |childdoc.ins| and |childdoc.dtx|.
%
\begin{itemize}
\item
Run (pdf)\LaTeX{} on |childdoc.dtx|
to compile the manual |childdoc.pdf| (this file).
\item
Run \LaTeX{} on |childdoc.ins| to create the definitions file |childdoc.def|
and the sample |cdocsamp.tex| with include files
|cdocsch1.tex|, |cdocsch2.tex|, |cdocspt3.tex|, |cdocspt4.tex|,
|cdocsdrf.tex|, |cdocsfn1.tex|, |cdocsfn2.tex|.
Then copy the file |childdoc.def| to an appropriate directory of your \LaTeX{}
distribution, e.g.\ \textit{texmf-root}|/tex/latex/childdoc|.
\end{itemize}

%%%%%%%%%%%%%%%%%%%%%%%%%%%%%%%%%%%%%%%%%%%%%%%%%%%%%%%%%%%%%%%%%%%%%%%%%%%%%%%%
\subsection{Related CTAN Packages}

There are several other packages which offer a similar functionality:
%
\begin{itemize}
\item
The packages
\href{http://ctan.org/pkg/docmute}{\textsf{docmute}},
\href{http://ctan.org/pkg/includex}{\textsf{includex}} and
\href{http://ctan.org/pkg/standalone}{\textsf{standalone}}
provide commands to include only the document body of
a child file thus allowing both files to be compiled individually.
\item
The packages \href{http://ctan.org/pkg/subdocs}{\textsf{subdocs}}
and \href{http://ctan.org/pkg/subfiles}{\textsf{subfiles}}
provide structures in which the main and child documents can be
encapsulated and allowing them to be compiled individually.
The inclusion mechanism is different from the conventional |\include|.
\item
The package \href{http://ctan.org/pkg/combine}{\textsf{combine}}
is an elaborate solution to combine several documents into one.
\end{itemize}
%
See also the CTAN topic \href{http://ctan.org/topic/subdocs}{\textsf{subdocs}}
for further related packages.
The present package differs from the above solutions in that
a document structure constructed with the conventional |\include| mechanism
just needs two extra commands at the top of every file
such that all constituent files can be compiled individually.

%%%%%%%%%%%%%%%%%%%%%%%%%%%%%%%%%%%%%%%%%%%%%%%%%%%%%%%%%%%%%%%%%%%%%%%%%%%%%%%%
%\subsection{Feature Suggestions}
%
%The following is a list of features which may be useful for future
%versions of this package:
%%
%\begin{itemize}
%\item
%\ldots
%\end{itemize}

%%%%%%%%%%%%%%%%%%%%%%%%%%%%%%%%%%%%%%%%%%%%%%%%%%%%%%%%%%%%%%%%%%%%%%%%%%%%%%%%
\subsection{Revision History}

%%%%%%%%%%%%%%%%%%%%%%%%%%%%%%%%%%%%%%%%
\paragraph{v2.0:} 2018/12/30

\begin{itemize}
\item
immediate forward processing
\item
added |\childdocby| mechanism
\item
manual restructured
\end{itemize}

%%%%%%%%%%%%%%%%%%%%%%%%%%%%%%%%%%%%%%%%
\paragraph{v1.6:} 2018/01/17

\begin{itemize}
\item
application for development of include files
\item
corrections to manual
\end{itemize}

%%%%%%%%%%%%%%%%%%%%%%%%%%%%%%%%%%%%%%%%
\paragraph{v1.5:} 2017/05/21

\begin{itemize}
\item
more complete structuring introduced
\item
|\childdocof| introduced
\item
|\childdoc| renamed to |\childdocmain|
\item
|\childredirect| renamed to |\childdocforward| and |\childdocforwardprefix|
and functionality expanded
\end{itemize}

%%%%%%%%%%%%%%%%%%%%%%%%%%%%%%%%%%%%%%%%
\paragraph{v1.0:} 2017/04/27

\begin{itemize}
\item
manual and install package
\item
first version published on CTAN
\end{itemize}

%%%%%%%%%%%%%%%%%%%%%%%%%%%%%%%%%%%%%%%%
\paragraph{v0.6:} 2017/04/26

\begin{itemize}
\item
redirection mechanism added
\end{itemize}

%%%%%%%%%%%%%%%%%%%%%%%%%%%%%%%%%%%%%%%%
\paragraph{v0.5:} 2017/04/26

\begin{itemize}
\item
functionality in definition file
\end{itemize}


%%%%%%%%%%%%%%%%%%%%%%%%%%%%%%%%%%%%%%%%%%%%%%%%%%%%%%%%%%%%%%%%%%%%%%%%%%%%%%%%
%%%%%%%%%%%%%%%%%%%%%%%%%%%%%%%%%%%%%%%%%%%%%%%%%%%%%%%%%%%%%%%%%%%%%%%%%%%%%%%%
%%%%%%%%%%%%%%%%%%%%%%%%%%%%%%%%%%%%%%%%%%%%%%%%%%%%%%%%%%%%%%%%%%%%%%%%%%%%%%%%
\appendix

\settowidth\MacroIndent{\rmfamily\scriptsize 000\ }

 \DocInput{childdoc.dtx}

\end{document}
%</driver>
% \fi
%
% %%%%%%%%%%%%%%%%%%%%%%%%%%%%%%%%%%%%%%%%%%%%%%%%%%%%%%%%%%%%%%%%%%%%%%%%%%%%%%
% %%%%%%%%%%%%%%%%%%%%%%%%%%%%%%%%%%%%%%%%%%%%%%%%%%%%%%%%%%%%%%%%%%%%%%%%%%%%%%
% \section{Sample}
%\iffalse
%<*samplemain>
%\fi
%
% The following presents a sample document
% with two chapters, two parts, a title page,
% a compile flag as well as three forwarding files to set the flag.
% It consists of eight |.tex| files:
% \begin{center}
% \begin{tabular}{ll}
% |cdocsamp.tex|&main file\\
% |cdocsch1.tex|&include file for chapter 1\\
% |cdocsch2.tex|&include file for chapter 2\\
% |cdocspt3.tex|&include file for part 3\\
% |cdocspt4.tex|&include file for part 4\\
% |cdocsdrf.tex|&forwarding file for main file in draft mode\\
% |cdocsfi1.tex|&forwarding file for final version of chapter 1\\
% |cdocsfi2.tex|&forwarding file for final version of chapter 2\\
% \end{tabular}
% \end{center}
% Each of the eight files can be compiled directly by the \LaTeX{} compiler.
%
% %%%%%%%%%%%%%%%%%%%%%%%%%%%%%%%%%%%%%%
% \paragraph{Main File.}
%
% The main file is called |cdocsamp.tex|.
%
% Load the \textsf{childdoc} definitions and
% declare the filename for the main document:
%    \begin{macrocode}
\input{childdoc.def}
\childdocmain{}
%    \end{macrocode}

% Optional override for |\version| flag:
%    \begin{macrocode}
%%\ifchilddoc\else\providecommand{\version}{draft}\fi
%    \end{macrocode}

% Define the default values for the |\version| flag
% (|final| for the main file and |draft| for childs):
%    \begin{macrocode}
\ifchilddoc
\providecommand{\version}{draft}
\else
\providecommand{\version}{final}
\fi
%    \end{macrocode}

% Load the standard document class:
%    \begin{macrocode}
\documentclass[12pt]{article}
%    \end{macrocode}

% Start the document body:
%    \begin{macrocode}
\begin{document}
%    \end{macrocode}

% Declare a title page.
% Print title, part of document being processed and version flag:
%    \begin{macrocode}
\addtocounter{page}{-1}
\begin{center}
{\LARGE\bfseries{}childdoc example\par}
\vspace{1cm}
\ifchilddoc
\ifchilddocmanual part\else chapter\fi:
`\childdocname' of `\childdocjob'\par
\else
main document: `\childdocjob'\par
\fi
version: \version\par
\end{center}
\newpage
%    \end{macrocode}

% Manually include selected file,
% otherwise process as usual:
%    \begin{macrocode}
\ifchilddocmanual
\section*{part `\childdocname'}
\input{\childdocname}
\else
%    \end{macrocode}

% Include the two chapters:
%    \begin{macrocode}
\include{cdocsch1}
\include{cdocsch2}
%    \end{macrocode}

% Include the two parts unless only chapters should be displayed:
%    \begin{macrocode}
\ifchilddoc\else
\section{part three}
\input{cdocspt3}
\section{part four}
\input{cdocspt4}
\fi
%    \end{macrocode}

% Process as usual until here:
%    \begin{macrocode}
\fi
%    \end{macrocode}

% End of document body:
%    \begin{macrocode}
\end{document}
%    \end{macrocode}
%\iffalse
%</samplemain>
%\fi
%
% %%%%%%%%%%%%%%%%%%%%%%%%%%%%%%%%%%%%%%
% \paragraph{Chapter Include Files.}
%
% The include files are called |cdocsch1.tex| and |cdocsch2.tex|.
%
%\iffalse
%<*samplechap1|samplechap2>
%\fi

% Optional override for |\version| flag:
%    \begin{macrocode}
%%\providecommand{\version}{final}
%    \end{macrocode}

% Include the main document:
%    \begin{macrocode}
\input{childdoc.def}
\childdocof{cdocsamp}
%    \end{macrocode}

%\iffalse
%</samplechap1|samplechap2>
%\fi
%
%\iffalse
%<*samplechap1>
%\fi
% Some text for chapter 1:
%    \begin{macrocode}
\section{one}
some text in chapter one
%    \end{macrocode}

%\iffalse
%</samplechap1>
%\fi
% Some text for chapter 2:
%\iffalse
%<*samplechap2>
%\fi
%    \begin{macrocode}
\section{two}
more text in chapter two
%    \end{macrocode}

%\iffalse
%</samplechap2>
%\fi
%
% %%%%%%%%%%%%%%%%%%%%%%%%%%%%%%%%%%%%%%
% \paragraph{Part Include Files.}
%
% The include files are called |cdocspt3.tex| and |cdocspt4.tex|.
%
%\iffalse
%<*samplepart3|samplepart4>
%\fi

% Optional override for |\version| flag:
%    \begin{macrocode}
%%\providecommand{\version}{final}
%    \end{macrocode}

% Include the main document:
%    \begin{macrocode}
\input{childdoc.def}
\childdocby{cdocsamp}
%    \end{macrocode}

%\iffalse
%</samplepart3|samplepart4>
%\fi
%
%\iffalse
%<*samplepart3>
%\fi
% Some text for part 3:
%    \begin{macrocode}
some text in part three
%    \end{macrocode}

%\iffalse
%</samplepart3>
%\fi
% Some text for part 4:
%\iffalse
%<*samplepart4>
%\fi
%    \begin{macrocode}
more text in part four
%    \end{macrocode}

%\iffalse
%</samplepart4>
%\fi
%
% %%%%%%%%%%%%%%%%%%%%%%%%%%%%%%%%%%%%%%
% \paragraph{Forwarding for a Complete Draft.}
%
% The following forwarding file |cdocsdrf.tex|
% compiles the main document in draft mode:
%\iffalse
%<*sampledraft>
%\fi
%    \begin{macrocode}
\def\version{draft}
\input{childdoc.def}
\childdocforward{cdocsamp}
%    \end{macrocode}

%\iffalse
%</sampledraft>
%\fi
%
% %%%%%%%%%%%%%%%%%%%%%%%%%%%%%%%%%%%%%%
% \paragraph{Forwarding for Final Version of the Chapters.}
%
% The following forwarding files |cdocsfn1.tex| and |cdocsfn2.tex|
% (with identical content)
% compile the final versions of the child documents
% |cdocsch1.tex| and |cdocsch2.tex|, respectively:
%\iffalse
%<*samplefinal>
%\fi
%    \begin{macrocode}
\def\version{final}
\input{childdoc.def}
\childdocforwardprefix[cdocsamp]{cdocsfn}{cdocsch}
%    \end{macrocode}

%\iffalse
%</samplefinal>
%\fi
%
% %%%%%%%%%%%%%%%%%%%%%%%%%%%%%%%%%%%%%%
% \paragraph{Command Line Processing.}
%
% The following three command lines generate the output files
% |cdocscld|, |cdocscl1| and |cdocscl2|
% which should be identical to
% |cdocsdrf|, |cdocsch1| and |cdocsfn2|, respectively:
% \begin{center}
% \begin{tabular}{l}
% |latex -jobname cdocscld \|\\
% |  "\def\version{draft}\input{childdoc.def}\childdocforward{cdocsamp}"|\\
% |latex -jobname cdocscl1 \|\\
% |  "\input{childdoc.def}\childdocforward[cdocsamp]{cdocsch1}"|\\
% |latex -jobname cdocscl2 \|\\
% |  "\def\version{final}\input{childdoc.def}\childdocforward{cdocsch2}"|
% \end{tabular}
% \end{center}
% Note that the trailing backslash on each first line
% merely continues the input to the second line
% (for convenient cut ant paste).
% Furthermore, the command |latex| can be replaced by any
% of its alternative versions such as |pdflatex|.
%
% %%%%%%%%%%%%%%%%%%%%%%%%%%%%%%%%%%%%%%%%%%%%%%%%%%%%%%%%%%%%%%%%%%%%%%%%%%%%%%
% %%%%%%%%%%%%%%%%%%%%%%%%%%%%%%%%%%%%%%%%%%%%%%%%%%%%%%%%%%%%%%%%%%%%%%%%%%%%%%
% \section{Implementation}
%\iffalse
%<*package>
%\fi
%
% This section describes the definitions file |childdoc.def|.

% The definitions cannot be loaded using |\usepackage| or |\RequirePackage|
% which has a mechanism to prevent loading a style file more than once.
% When loading the definitions by means of |\input|
% multiple instances have to be prevented manually:
%\iffalse
%This code needs to be before the `\ProvidesFile' directive
%which is defined at the beginning of this file.
%Therefore it is also placed there and commented out here.
%</package>
%<*discard>
%\fi
%    \begin{macrocode}
\ifdefined\childdocmain\endinput\fi
%    \end{macrocode}
%\iffalse
%</discard>
%<*package>
%\fi
%
% \macro{\ifchilddoc}
% \macro{\ifchilddocmanual}
% The conditional |\ifchilddoc| tells whether a
% child (true) or main (false) document is being compiled.
% The conditional |\ifchilddocmanual| tells whether
% the |\includeonly| mechanism is used (false) or
% the selection of child files must be performed manually (true).
% The definitions initialise to false:
%    \begin{macrocode}
\newif\ifchilddoc
\newif\ifchilddocmanual
%    \end{macrocode}

% \macro{\childdocname}
% \macro{\childdocjob}
% The macro |\childdocname| stores the name of the main document
% to be compiled. The macro |\childdocjob| stores the name of
% the document on which the \LaTeX{} compiler was originally invoked.
% The content of |\jobname| cannot be compared
% to filenames specified in the source due to different catcodes.
% The following code rescans |\jobname|, stores the result
% in |\childdocname| and saves a copy in |\childdocjob|:
%    \begin{macrocode}
\edef\childdocname{\scantokens\expandafter{\jobname\noexpand}}
\let\childdocjob\childdocname
%    \end{macrocode}

% \macro{\childdocdisable}
% The macro |\childdocdisable| prevents the main file
% from being processed more than once.
% At this stage, the main document command |\childdocmain|
% is assumed to be called once again where it should do nothing.
% Any subsequent call to it should prevent
% a secondary processing of the main document
% It overwrites the forwarding commands
% |\childdocof| and |\childdocforward|
% with empty macros to prevent further inclusions of the main document:
%    \begin{macrocode}
\newcommand{\childdocdisable}
{
  \renewcommand{\childdocmain}[1]{\renewcommand{\childdocmain}[1]{\endinput}}
  \renewcommand{\childdocof}[1]{}
  \renewcommand{\childdocby}[2][]{}
  \renewcommand{\childdocforward}[2][]{}
  \renewcommand{\childdocdisable}{}
}
%    \end{macrocode}

% \macro{\childdocmain}
% The macro |\childdocmain| is to be called at the top of the main file
% with nothing or the main filename (without extension) as argument.
% First, it breaks loops.
% If the argument is not empty and does not match |\childdocname|
% (which is set by the first inclusion of |childdoc.def|),
% |\ifchilddoc| is set to true, |\includeonly| is applied to the child file
% and |\jobname| is set to the main file
% (for proper handling of |.aux| files):
%    \begin{macrocode}
\newcommand{\childdocmain}[1]
{
  \childdocdisable\childdocmain{}
  \if?#1?\else
    \begingroup
      \def\childdoctmp{#1}
      \ifx\childdoctmp\childdocname
        \def\childdoctmp{}
      \else
        \def\childdoctmp
        {
          \childdoctrue
          \includeonly{\childdocname}
          \def\childdocjob{#1}
          \def\jobname{#1}
        }
      \fi
      \expandafter
    \endgroup
    \childdoctmp
  \fi
}
%    \end{macrocode}

% \macro{\childdocof}
% The command |\childdocof| redirects
% compilation to the main file |#1|.
%    \begin{macrocode}
\newcommand{\childdocof}[1]
{
  \childdocdisable
  \childdoctrue
  \includeonly{\childdocname}
  \def\jobname{#1}
  \def\childdocjob{#1}
  \input{#1}
}
%    \end{macrocode}

% \macro{\childdocby}
% The command |\childdocby| ....
%    \begin{macrocode}
\newcommand{\childdocby}[2][]
{
  \childdocdisable
  \childdoctrue
  \childdocmanualtrue
  \if?#1?\else
    \def\jobname{#2}
  \fi
  \def\childdocjob{#2}
  \input{#2}
  \endinput
}
%    \end{macrocode}

% \macro{\childdocforward}
% The command |\childdocforward| redirects
% compilation to the main file or
% (if the optional argument is given) a child file.
% Parameters are set as if the main file
% or a child file starting with |\childdocof| was compiled.
% Then compilation is handed over to the main file:
%    \begin{macrocode}
\newcommand{\childdocforward}[2][]
{
  \begingroup
    \if?#1?
      \def\childdoctmp
      {
        \def\childdocname{#2}
        \def\childdocjob{#2}
        \def\jobname{#2}
        \input{#2}
        \endinput
      }
    \else
      \def\childdoctmp
      {
        \childdocdisable
        \def\childdocname{#2}
        \childdoctrue
        \includeonly{#2}
        \def\childdocjob{#1}
        \def\jobname{#1}
        \input{#1}
        \endinput
      }
    \fi
    \expandafter
  \endgroup
  \childdoctmp
}
%    \end{macrocode}

% \macro{\childdocforwardprefix}
% The command |\childdocforwardprefix| redirects
% compilation to the main or a child file by means of a pattern.
% The prefix |#1| in the current filename is replaced by |#2|
% and the suffix of the current filename is kept
% (it is assumed that the filename does not contain the substring `|~~~|'
% which is used as a delimiter).
% Compilation is handed over to the new file by |\childdocforward|:
%    \begin{macrocode}
\newcommand{\childdocforwardprefix}[3][]
{
  \begingroup
    \def\childdocextract #2##1~~~{\def\childdoctmp{\childdocforward[#1]{#3##1}}}
    \expandafter\childdocextract\childdocname~~~
    \expandafter
  \endgroup
  \childdoctmp
}
%    \end{macrocode}

% \macro{\childdoc}
% The deprecated macro |\childdoc| is a legacy version of |\childdocmain|:
%    \begin{macrocode}
\newcommand{\childdoc}{\childdocmain}
%    \end{macrocode}

% \macro{\childdocredirect}
% The deprecated macro |\childdocredirect| is a legacy version
% of |\childdocforward| and |\childdocforwardprefix|:
%    \begin{macrocode}
\newcommand{\childdocredirect}[2][]
{
  \begingroup
    \if?#1?
      \def\childdoctmp{\childdocforward{#2}}
    \else
      \def\childdoctmp{\childdocforwardprefix{#1}{#2}}
    \fi
    \expandafter
  \endgroup
  \childdoctmp
}
%    \end{macrocode}

%\iffalse
%</package>
%\fi
%
\endinput
\childdocforward{cdocsch2}"|
% \end{tabular}
% \end{center}
% Note that the trailing backslash on each first line
% merely continues the input to the second line
% (for convenient cut ant paste).
% Furthermore, the command |latex| can be replaced by any
% of its alternative versions such as |pdflatex|.
%
% %%%%%%%%%%%%%%%%%%%%%%%%%%%%%%%%%%%%%%%%%%%%%%%%%%%%%%%%%%%%%%%%%%%%%%%%%%%%%%
% %%%%%%%%%%%%%%%%%%%%%%%%%%%%%%%%%%%%%%%%%%%%%%%%%%%%%%%%%%%%%%%%%%%%%%%%%%%%%%
% \section{Implementation}
%\iffalse
%<*package>
%\fi
%
% This section describes the definitions file |childdoc.def|.

% The definitions cannot be loaded using |\usepackage| or |\RequirePackage|
% which has a mechanism to prevent loading a style file more than once.
% When loading the definitions by means of |\input|
% multiple instances have to be prevented manually:
%\iffalse
%This code needs to be before the `\ProvidesFile' directive
%which is defined at the beginning of this file.
%Therefore it is also placed there and commented out here.
%</package>
%<*discard>
%\fi
%    \begin{macrocode}
\ifdefined\childdocmain\endinput\fi
%    \end{macrocode}
%\iffalse
%</discard>
%<*package>
%\fi
%
% \macro{\ifchilddoc}
% \macro{\ifchilddocmanual}
% The conditional |\ifchilddoc| tells whether a
% child (true) or main (false) document is being compiled.
% The conditional |\ifchilddocmanual| tells whether
% the |\includeonly| mechanism is used (false) or
% the selection of child files must be performed manually (true).
% The definitions initialise to false:
%    \begin{macrocode}
\newif\ifchilddoc
\newif\ifchilddocmanual
%    \end{macrocode}

% \macro{\childdocname}
% \macro{\childdocjob}
% The macro |\childdocname| stores the name of the main document
% to be compiled. The macro |\childdocjob| stores the name of
% the document on which the \LaTeX{} compiler was originally invoked.
% The content of |\jobname| cannot be compared
% to filenames specified in the source due to different catcodes.
% The following code rescans |\jobname|, stores the result
% in |\childdocname| and saves a copy in |\childdocjob|:
%    \begin{macrocode}
\edef\childdocname{\scantokens\expandafter{\jobname\noexpand}}
\let\childdocjob\childdocname
%    \end{macrocode}

% \macro{\childdocdisable}
% The macro |\childdocdisable| prevents the main file
% from being processed more than once.
% At this stage, the main document command |\childdocmain|
% is assumed to be called once again where it should do nothing.
% Any subsequent call to it should prevent
% a secondary processing of the main document
% It overwrites the forwarding commands
% |\childdocof| and |\childdocforward|
% with empty macros to prevent further inclusions of the main document:
%    \begin{macrocode}
\newcommand{\childdocdisable}
{
  \renewcommand{\childdocmain}[1]{\renewcommand{\childdocmain}[1]{\endinput}}
  \renewcommand{\childdocof}[1]{}
  \renewcommand{\childdocby}[2][]{}
  \renewcommand{\childdocforward}[2][]{}
  \renewcommand{\childdocdisable}{}
}
%    \end{macrocode}

% \macro{\childdocmain}
% The macro |\childdocmain| is to be called at the top of the main file
% with nothing or the main filename (without extension) as argument.
% First, it breaks loops.
% If the argument is not empty and does not match |\childdocname|
% (which is set by the first inclusion of |childdoc.def|),
% |\ifchilddoc| is set to true, |\includeonly| is applied to the child file
% and |\jobname| is set to the main file
% (for proper handling of |.aux| files):
%    \begin{macrocode}
\newcommand{\childdocmain}[1]
{
  \childdocdisable\childdocmain{}
  \if?#1?\else
    \begingroup
      \def\childdoctmp{#1}
      \ifx\childdoctmp\childdocname
        \def\childdoctmp{}
      \else
        \def\childdoctmp
        {
          \childdoctrue
          \includeonly{\childdocname}
          \def\childdocjob{#1}
          \def\jobname{#1}
        }
      \fi
      \expandafter
    \endgroup
    \childdoctmp
  \fi
}
%    \end{macrocode}

% \macro{\childdocof}
% The command |\childdocof| redirects
% compilation to the main file |#1|.
%    \begin{macrocode}
\newcommand{\childdocof}[1]
{
  \childdocdisable
  \childdoctrue
  \includeonly{\childdocname}
  \def\jobname{#1}
  \def\childdocjob{#1}
  \input{#1}
}
%    \end{macrocode}

% \macro{\childdocby}
% The command |\childdocby| ....
%    \begin{macrocode}
\newcommand{\childdocby}[2][]
{
  \childdocdisable
  \childdoctrue
  \childdocmanualtrue
  \if?#1?\else
    \def\jobname{#2}
  \fi
  \def\childdocjob{#2}
  \input{#2}
  \endinput
}
%    \end{macrocode}

% \macro{\childdocforward}
% The command |\childdocforward| redirects
% compilation to the main file or
% (if the optional argument is given) a child file.
% Parameters are set as if the main file
% or a child file starting with |\childdocof| was compiled.
% Then compilation is handed over to the main file:
%    \begin{macrocode}
\newcommand{\childdocforward}[2][]
{
  \begingroup
    \if?#1?
      \def\childdoctmp
      {
        \def\childdocname{#2}
        \def\childdocjob{#2}
        \def\jobname{#2}
        \input{#2}
        \endinput
      }
    \else
      \def\childdoctmp
      {
        \childdocdisable
        \def\childdocname{#2}
        \childdoctrue
        \includeonly{#2}
        \def\childdocjob{#1}
        \def\jobname{#1}
        \input{#1}
        \endinput
      }
    \fi
    \expandafter
  \endgroup
  \childdoctmp
}
%    \end{macrocode}

% \macro{\childdocforwardprefix}
% The command |\childdocforwardprefix| redirects
% compilation to the main or a child file by means of a pattern.
% The prefix |#1| in the current filename is replaced by |#2|
% and the suffix of the current filename is kept
% (it is assumed that the filename does not contain the substring `|~~~|'
% which is used as a delimiter).
% Compilation is handed over to the new file by |\childdocforward|:
%    \begin{macrocode}
\newcommand{\childdocforwardprefix}[3][]
{
  \begingroup
    \def\childdocextract #2##1~~~{\def\childdoctmp{\childdocforward[#1]{#3##1}}}
    \expandafter\childdocextract\childdocname~~~
    \expandafter
  \endgroup
  \childdoctmp
}
%    \end{macrocode}

% \macro{\childdoc}
% The deprecated macro |\childdoc| is a legacy version of |\childdocmain|:
%    \begin{macrocode}
\newcommand{\childdoc}{\childdocmain}
%    \end{macrocode}

% \macro{\childdocredirect}
% The deprecated macro |\childdocredirect| is a legacy version
% of |\childdocforward| and |\childdocforwardprefix|:
%    \begin{macrocode}
\newcommand{\childdocredirect}[2][]
{
  \begingroup
    \if?#1?
      \def\childdoctmp{\childdocforward{#2}}
    \else
      \def\childdoctmp{\childdocforwardprefix{#1}{#2}}
    \fi
    \expandafter
  \endgroup
  \childdoctmp
}
%    \end{macrocode}

%\iffalse
%</package>
%\fi
%
\endinput
|\\
|\childdocforward[|\textit{main}|]{|\textit{dest}|}|\\
\end{tabular}
\end{center}
%
The argument \textit{dest} is the destination file
(without extension).
It should be the main file or one of the child files.
Note that further \textsf{childdoc} directives
such as |\childdocof| and |\childdocforward|
in the indicated file will be processed in this form.
The optional argument \textit{main}
passes on directly to the main file \textit{main}
while pretending to compile the child \textit{dest}.
This form behaves as if \textit{dest}
issues |\childdocof{|\textit{main}|}| right away,
and no further \textsf{childdoc} directives will be processed.

%%%%%%%%%%%%%%%%%%%%%%%%%%%%%%%%%%%%%%%%
\DescribeMacro{\...prefix}
In the alternative form |\childdocforwardprefix|,
%
\begin{center}
\begin{tabular}{l}
|% \iffalse
%
% childdoc.dtx Copyright (C) 2017-2018 Niklas Beisert
%
% This work may be distributed and/or modified under the
% conditions of the LaTeX Project Public License, either version 1.3
% of this license or (at your option) any later version.
% The latest version of this license is in
%   http://www.latex-project.org/lppl.txt
% and version 1.3 or later is part of all distributions of LaTeX
% version 2005/12/01 or later.
%
% This work has the LPPL maintenance status `maintained'.
%
% The Current Maintainer of this work is Niklas Beisert.
%
% This work consists of the files childdoc.dtx and childdoc.ins
% and the derived files childdoc.def and cdocsamp.tex with
% cdocsch1.tex, cdocsch2.tex, cdocsdrf.tex, cdocsfn1.tex, cdocsfn2.tex.
%
%<package>\ifdefined\childdocmain\endinput\fi
%<package>\ProvidesFile{childdoc.def}[2018/12/30 v2.0 child document driver]
%<samplemain>\ProvidesFile{cdocsamp.tex}[2018/12/30 v2.0 sample for childdoc]
%<*driver>
%\ProvidesFile{childdoc.drv}[2018/12/30 v2.0 childdoc reference manual file]
\PassOptionsToClass{10pt,a4paper}{article}
\documentclass{ltxdoc}

\usepackage[margin=35mm]{geometry}
\usepackage{hyperref}
\usepackage{hyperxmp}
\usepackage[usenames]{color}

\hypersetup{colorlinks=true}
\hypersetup{pdfstartview=FitH}
\hypersetup{pdfpagemode=UseNone}
\hypersetup{pdfsource={}}
\hypersetup{pdflang={en-UK}}
\hypersetup{pdfcopyright={Copyright 2017-2018 Niklas Beisert.
  This work may be distributed and/or modified under the
  conditions of the LaTeX Project Public License, either version 1.3
  of this license or (at your option) any later version.}}
\hypersetup{pdflicenseurl={http://www.latex-project.org/lppl.txt}}
\hypersetup{pdfcontactaddress={ETH Zurich, ITP, HIT K,
  Wolfgang-Pauli-Strasse 27}}
\hypersetup{pdfcontactpostcode={8093}}
\hypersetup{pdfcontactcity={Zurich}}
\hypersetup{pdfcontactcountry={Switzerland}}
\hypersetup{pdfcontactemail={nbeisert@itp.phys.ethz.ch}}
\hypersetup{pdfcontacturl={http://people.phys.ethz.ch/\xmptilde nbeisert/}}

\newcommand{\secref}[1]{\hyperref[#1]{section \ref*{#1}}}

\parskip1ex
\parindent0pt
\let\olditemize\itemize
\def\itemize{\olditemize\parskip0pt}

\begin{document}

\title{The \textsf{childdoc} Package}
\hypersetup{pdftitle={The childdoc Package}}
\author{Niklas Beisert\\[2ex]
  Institut f\"ur Theoretische Physik\\
  Eidgen\"ossische Technische Hochschule Z\"urich\\
  Wolfgang-Pauli-Strasse 27, 8093 Z\"urich, Switzerland\\[1ex]
  \href{mailto:nbeisert@itp.phys.ethz.ch}
  {\texttt{nbeisert@itp.phys.ethz.ch}}}
\hypersetup{pdfauthor={Niklas Beisert}}
\hypersetup{pdfsubject={Manual for the LaTeX2e Package childdoc}}
\date{30 December 2018, \textsf{v2.0}}
\maketitle

\begin{abstract}\noindent
\textsf{childdoc} is a \LaTeXe{} package
that enables the direct compilation
of document sections included by |\include|
to individual files.
\end{abstract}

\begingroup
\parskip0ex
\tableofcontents
\endgroup

%%%%%%%%%%%%%%%%%%%%%%%%%%%%%%%%%%%%%%%%%%%%%%%%%%%%%%%%%%%%%%%%%%%%%%%%%%%%%%%%
%%%%%%%%%%%%%%%%%%%%%%%%%%%%%%%%%%%%%%%%%%%%%%%%%%%%%%%%%%%%%%%%%%%%%%%%%%%%%%%%
\section{Introduction}

\LaTeX{} provides a mechanism to structure a large document (such as a book)
into a main file and several child files (containing the chapters)
using the |\include| command.
This mechanism is beneficial for documents
which span hundreds of pages in order to
make the source file(s) more manageable.
Moreover, compilation can be restricted to
selected child files by means of the |\includeonly| command.
The latter feature can be used to reduce the compilation time while editing
(this was significantly more useful in the earlier days of \LaTeX{})
or to generate a smaller document which is easier to navigate.
Another application of |\includeonly| is to generate
documents consisting of selected parts of the complete document.

However, there are a few drawbacks of the plain |\include| mechanism:
\begin{itemize}
\item
The child files cannot be compiled on their own,
they can only be compiled via the main file.
A naive editing environment
(such as a text editor with an option
to have the current file processed by \LaTeX)
may require one to switch to the main file before compiling;
attempting to compile the child file produces errors.
\item
The main file must be modified (each time)
to adjust the |\includeonly| command
to the present needs. This easily leaves the main file in a messy state.
\item
The generated document will always carry the filename
of the main document. This is inconvenient if
several child files are to be compiled and
to be kept for distribution.
\end{itemize}

The present package provides a simple interface
to make child files individually compilable by \LaTeX{}.
Compiling a child file then has the same effect as compiling
the main file with an |\includeonly| command
to select the appropriate child.
Moreover the generated document will carry the name of the child
rather than the main file.
This resolves all three above issues.

This feature is meant to make the editing of books,
thesis documents and lecture notes somewhat more convenient.
However, the package can also be used efficiently for
composing a series of documents (such as exercise sheets)
which are typically distributed individually.
It then assists the author in generating the individual documents
(potentially in different versions)
as well as a document containing the collected series.
Another application is in developing style files
or other kinds of included material
where compilation of the style file could redirect
to a sample or test file.

%%%%%%%%%%%%%%%%%%%%%%%%%%%%%%%%%%%%%%%%%%%%%%%%%%%%%%%%%%%%%%%%%%%%%%%%%%%%%%%%
%%%%%%%%%%%%%%%%%%%%%%%%%%%%%%%%%%%%%%%%%%%%%%%%%%%%%%%%%%%%%%%%%%%%%%%%%%%%%%%%
\section{Usage}

First of all, the package \textsf{childdoc} is \emph{not} a standard
\LaTeXe{} |.sty| style file! Therefore it needs to be invoked in
a non-standard way.

%%%%%%%%%%%%%%%%%%%%%%%%%%%%%%%%%%%%%%%%%%%%%%%%%%%%%%%%%%%%%%%%%%%%%%%%%%%%%%%%
\subsection{Included Files}
\label{sec:include}

%%%%%%%%%%%%%%%%%%%%%%%%%%%%%%%%%%%%%%%%
\DescribeMacro{\childdocmain}
To use the package, add the commands
\begin{center}
\begin{tabular}{l}
|% \iffalse
%
% childdoc.dtx Copyright (C) 2017-2018 Niklas Beisert
%
% This work may be distributed and/or modified under the
% conditions of the LaTeX Project Public License, either version 1.3
% of this license or (at your option) any later version.
% The latest version of this license is in
%   http://www.latex-project.org/lppl.txt
% and version 1.3 or later is part of all distributions of LaTeX
% version 2005/12/01 or later.
%
% This work has the LPPL maintenance status `maintained'.
%
% The Current Maintainer of this work is Niklas Beisert.
%
% This work consists of the files childdoc.dtx and childdoc.ins
% and the derived files childdoc.def and cdocsamp.tex with
% cdocsch1.tex, cdocsch2.tex, cdocsdrf.tex, cdocsfn1.tex, cdocsfn2.tex.
%
%<package>\ifdefined\childdocmain\endinput\fi
%<package>\ProvidesFile{childdoc.def}[2018/12/30 v2.0 child document driver]
%<samplemain>\ProvidesFile{cdocsamp.tex}[2018/12/30 v2.0 sample for childdoc]
%<*driver>
%\ProvidesFile{childdoc.drv}[2018/12/30 v2.0 childdoc reference manual file]
\PassOptionsToClass{10pt,a4paper}{article}
\documentclass{ltxdoc}

\usepackage[margin=35mm]{geometry}
\usepackage{hyperref}
\usepackage{hyperxmp}
\usepackage[usenames]{color}

\hypersetup{colorlinks=true}
\hypersetup{pdfstartview=FitH}
\hypersetup{pdfpagemode=UseNone}
\hypersetup{pdfsource={}}
\hypersetup{pdflang={en-UK}}
\hypersetup{pdfcopyright={Copyright 2017-2018 Niklas Beisert.
  This work may be distributed and/or modified under the
  conditions of the LaTeX Project Public License, either version 1.3
  of this license or (at your option) any later version.}}
\hypersetup{pdflicenseurl={http://www.latex-project.org/lppl.txt}}
\hypersetup{pdfcontactaddress={ETH Zurich, ITP, HIT K,
  Wolfgang-Pauli-Strasse 27}}
\hypersetup{pdfcontactpostcode={8093}}
\hypersetup{pdfcontactcity={Zurich}}
\hypersetup{pdfcontactcountry={Switzerland}}
\hypersetup{pdfcontactemail={nbeisert@itp.phys.ethz.ch}}
\hypersetup{pdfcontacturl={http://people.phys.ethz.ch/\xmptilde nbeisert/}}

\newcommand{\secref}[1]{\hyperref[#1]{section \ref*{#1}}}

\parskip1ex
\parindent0pt
\let\olditemize\itemize
\def\itemize{\olditemize\parskip0pt}

\begin{document}

\title{The \textsf{childdoc} Package}
\hypersetup{pdftitle={The childdoc Package}}
\author{Niklas Beisert\\[2ex]
  Institut f\"ur Theoretische Physik\\
  Eidgen\"ossische Technische Hochschule Z\"urich\\
  Wolfgang-Pauli-Strasse 27, 8093 Z\"urich, Switzerland\\[1ex]
  \href{mailto:nbeisert@itp.phys.ethz.ch}
  {\texttt{nbeisert@itp.phys.ethz.ch}}}
\hypersetup{pdfauthor={Niklas Beisert}}
\hypersetup{pdfsubject={Manual for the LaTeX2e Package childdoc}}
\date{30 December 2018, \textsf{v2.0}}
\maketitle

\begin{abstract}\noindent
\textsf{childdoc} is a \LaTeXe{} package
that enables the direct compilation
of document sections included by |\include|
to individual files.
\end{abstract}

\begingroup
\parskip0ex
\tableofcontents
\endgroup

%%%%%%%%%%%%%%%%%%%%%%%%%%%%%%%%%%%%%%%%%%%%%%%%%%%%%%%%%%%%%%%%%%%%%%%%%%%%%%%%
%%%%%%%%%%%%%%%%%%%%%%%%%%%%%%%%%%%%%%%%%%%%%%%%%%%%%%%%%%%%%%%%%%%%%%%%%%%%%%%%
\section{Introduction}

\LaTeX{} provides a mechanism to structure a large document (such as a book)
into a main file and several child files (containing the chapters)
using the |\include| command.
This mechanism is beneficial for documents
which span hundreds of pages in order to
make the source file(s) more manageable.
Moreover, compilation can be restricted to
selected child files by means of the |\includeonly| command.
The latter feature can be used to reduce the compilation time while editing
(this was significantly more useful in the earlier days of \LaTeX{})
or to generate a smaller document which is easier to navigate.
Another application of |\includeonly| is to generate
documents consisting of selected parts of the complete document.

However, there are a few drawbacks of the plain |\include| mechanism:
\begin{itemize}
\item
The child files cannot be compiled on their own,
they can only be compiled via the main file.
A naive editing environment
(such as a text editor with an option
to have the current file processed by \LaTeX)
may require one to switch to the main file before compiling;
attempting to compile the child file produces errors.
\item
The main file must be modified (each time)
to adjust the |\includeonly| command
to the present needs. This easily leaves the main file in a messy state.
\item
The generated document will always carry the filename
of the main document. This is inconvenient if
several child files are to be compiled and
to be kept for distribution.
\end{itemize}

The present package provides a simple interface
to make child files individually compilable by \LaTeX{}.
Compiling a child file then has the same effect as compiling
the main file with an |\includeonly| command
to select the appropriate child.
Moreover the generated document will carry the name of the child
rather than the main file.
This resolves all three above issues.

This feature is meant to make the editing of books,
thesis documents and lecture notes somewhat more convenient.
However, the package can also be used efficiently for
composing a series of documents (such as exercise sheets)
which are typically distributed individually.
It then assists the author in generating the individual documents
(potentially in different versions)
as well as a document containing the collected series.
Another application is in developing style files
or other kinds of included material
where compilation of the style file could redirect
to a sample or test file.

%%%%%%%%%%%%%%%%%%%%%%%%%%%%%%%%%%%%%%%%%%%%%%%%%%%%%%%%%%%%%%%%%%%%%%%%%%%%%%%%
%%%%%%%%%%%%%%%%%%%%%%%%%%%%%%%%%%%%%%%%%%%%%%%%%%%%%%%%%%%%%%%%%%%%%%%%%%%%%%%%
\section{Usage}

First of all, the package \textsf{childdoc} is \emph{not} a standard
\LaTeXe{} |.sty| style file! Therefore it needs to be invoked in
a non-standard way.

%%%%%%%%%%%%%%%%%%%%%%%%%%%%%%%%%%%%%%%%%%%%%%%%%%%%%%%%%%%%%%%%%%%%%%%%%%%%%%%%
\subsection{Included Files}
\label{sec:include}

%%%%%%%%%%%%%%%%%%%%%%%%%%%%%%%%%%%%%%%%
\DescribeMacro{\childdocmain}
To use the package, add the commands
\begin{center}
\begin{tabular}{l}
|\input{childdoc.def}|\\
|\childdocmain{}|\\
\end{tabular}
\end{center}
at the very top of the main \LaTeX{} file,
in particular \emph{before} the |\documentclass| statement!
The argument of |\childdocmain| should be left empty
(but it must be present).

%%%%%%%%%%%%%%%%%%%%%%%%%%%%%%%%%%%%%%%%
\DescribeMacro{\childdocof}
Furthermore, add the commands
\begin{center}
\begin{tabular}{l}
|\input{childdoc.def}|\\
|\childdocof{|\textit{main}|}|\\
\end{tabular}
\end{center}
at the top of every child file \textit{child}
which is included by |\include{|\textit{child}|}|
from within the main file
(or at least for those files to be compiled individually).
The argument \textit{main} must be the filename of the main file.

There are a couple of
considerations in setting up the main and child documents:

%%%%%%%%%%%%%%%%%%%%%%%%%%%%%%%%%%%%%%%%
\paragraph{Restrictions.}

Please note the following restrictions:
\begin{itemize}
\item
|\childdocmain| must be called with one argument \textit{main}
to ensure compatibility with earlier version of the package.
It must either be empty (|\childdocmain{}|)
or precisely match the filename of the main file in which it is specified.
See \secref{sec:detection} for further information.
\item
The filename \textit{main} must be specified without the |.tex| extension.
\item
The filename \textit{main} is case sensitive
(even in case-insensitive file systems)
due to internal string comparison.
\item
The argument \textit{main} should be fully expanded, it cannot be a macro.
\item
Subdirectories and special characters should be avoided in filenames.
\item
The command |\childdocmain{|\textit{main}|}| must be followed by a whitespace.
It should not be followed immediately by another command
or by a comment mark `|%|'.
This is because the \TeX{} parser reads the token immediately following
the argument of |\childdocmain| and puts it
at the beginning of every child section;
however, a white\-space is ignored.
\end{itemize}

%%%%%%%%%%%%%%%%%%%%%%%%%%%%%%%%%%%%%%%%
\paragraph{Content of Main File.}

It is advisable to place all content in the child files included by |\include|.
Any output contained in the main file will appear in all child documents
unless suppressed manually;
it cannot be suppressed automatically by the |\includeonly| directive
and thus should normally be avoided.
A method to include some content in the main file
by means of conditional processing is described in \secref{sec:conditional}.

%%%%%%%%%%%%%%%%%%%%%%%%%%%%%%%%%%%%%%%%
\paragraph{Page Numbering.}

When only a part of the document is compiled,
the appropriate numbering of pages
(as well as other status parameters)
is determined from the |.aux| files.
The latter contain information from previous passes.
However this information needs to propagate through
all intermediate child documents.
Therefore the page numbering in child documents may well
be inconsistent until the complete document is compiled at least once.

A useful (if unconventional) way to always ensure a consistent
page numbering is to restart the numbering in each child document
and denote the pages by `\textit{child}|.|\textit{page}'
where \textit{child} represents the chapter/section number of the child file.
This can be achieved by the command
|\numberwithin{page}{|\textit{child}|}|
of the \textsf{amsmath} package
where \textit{child} can be |chapter| or |section|
depending on the chosen structuring.
Alternatively, one can modify the macro |\thepage| appropriately
and reset the counter |page| at the start of each child file.

%%%%%%%%%%%%%%%%%%%%%%%%%%%%%%%%%%%%%%%%%%%%%%%%%%%%%%%%%%%%%%%%%%%%%%%%%%%%%%%%
\subsection{Conditional Processing}
\label{sec:conditional}

The package provides a mechanism to compile different versions
of a document. To customise the versions further some conditional processing
can come in handy to distinguish which version is being compiled.
The package provides two macros to describe the compilation context:

%%%%%%%%%%%%%%%%%%%%%%%%%%%%%%%%%%%%%%%%
\DescribeMacro{\ifchilddoc}
The conditional |\ifchilddoc| distinguishes between the compilation of
child documents and the main document:
%
\begin{center}
|\ifchilddoc |\textit{child-code}| |[|\||else |\textit{main-code}]| \||fi|
\end{center}

%%%%%%%%%%%%%%%%%%%%%%%%%%%%%%%%%%%%%%%%
\DescribeMacro{\childdocname}
\DescribeMacro{\childdocjob}
The macro |\childdocname| contains the filename (without extension)
of the main or child file being processed.
Note that |\childdocjob| will always contain the name of the main file.

%%%%%%%%%%%%%%%%%%%%%%%%%%%%%%%%%%%%%%%%
\paragraph{Title Page.}

Conditional processing can be used to include a title or banner page
in the main document when proper precautions are taken.
Importantly, the code in the main file should ensure that the page counter
(as well as other status parameters which are stored in the |.aux| files)
takes the same value after the conditional processing.
Otherwise the page numbers may take divergent values
depending on which part is compiled.

For example, a title page could be declared by:
%
\begin{center}
\begin{tabular}{l}
|\ifchilddoc\||else|\\
|\addtocounter{page}{-1}|\\
\textit{code for title page}\\
|\newpage|\\
|\||fi|
\end{tabular}
\end{center}
%
A banner page for the child documents can be generated by:
%
\begin{center}
\begin{tabular}{l}
|\ifchilddoc|\\
|\addtocounter{page}{-1}|\\
\textit{code for banner page}\\
|\newpage|\\
|\||fi|
\end{tabular}
\end{center}
%
Here one could write a message such as:
\begin{center}
|This is the part \childdocname{} of \childdocjob{}.|
\end{center}

%%%%%%%%%%%%%%%%%%%%%%%%%%%%%%%%%%%%%%%%%%%%%%%%%%%%%%%%%%%%%%%%%%%%%%%%%%%%%%%%
\subsection{Flags}
\label{sec:flags}

The package makes it easy to generate different versions
of the main or child documents.
To this end compilation flags can be defined
and assigned different default values.
They will be particularly useful in conjunction
with the forwarding mechanism described in \secref{sec:forward}.

For example, it may be useful to have a flag |\version|
which can be set to |draft| or |final|.
The document source will contain some conditional code
depending on the value of |\version|.
Suppose further, the flag should default to |final| for the main file
and to |draft| for child files
which is a natural assignment for editing the document.
This is achieved by placing the following code
in the preamble of the main document
(below the |\childdocmain| directive):
%
\begin{center}
\begin{tabular}{l}
|\ifchilddoc|\\
|\providecommand{\version}{draft}|\\
|\||else|\\
|\providecommand{\version}{final}|\\
|\||fi|
\end{tabular}
\end{center}
%
The definition by |\providecommand| makes sure
that previous definitions are not overwritten.
Further statements |\providecommand{\version}{...}|
can thus be added before the above code to override it.

For the main file, one might add a line
(between |\childdocmain| and the above block)
%
\begin{center}
|%\ifchilddoc\||else\providecommand{\version}{draft}\||fi|
\end{center}
%
which can be uncommented to produce a draft version.
Likewise one can add a line to the very top of a child file
(above the |\childdocof{|\textit{main}|}| directive)
%
\begin{center}
|%\providecommand{\version}{final}|
\end{center}
%
which can be uncommented to produce the final version of this child document.

%%%%%%%%%%%%%%%%%%%%%%%%%%%%%%%%%%%%%%%%%%%%%%%%%%%%%%%%%%%%%%%%%%%%%%%%%%%%%%%%
\subsection{Forwarding}
\label{sec:forward}

Different versions of the main or child documents
using compilation flags as described in \secref{sec:flags}
can be (permanently) stored in different files
for convenient compilation, viewing and distribution.
To this end, the package defines a command
to pass on compilation to a different file:

%%%%%%%%%%%%%%%%%%%%%%%%%%%%%%%%%%%%%%%%
\DescribeMacro{\childdocforward}
The command |\childdocforward| redirects processing to
another source file:
%
\begin{center}
\begin{tabular}{l}
|\input{childdoc.def}|\\
|\childdocforward[|\textit{main}|]{|\textit{dest}|}|\\
\end{tabular}
\end{center}
%
The argument \textit{dest} is the destination file
(without extension).
It should be the main file or one of the child files.
Note that further \textsf{childdoc} directives
such as |\childdocof| and |\childdocforward|
in the indicated file will be processed in this form.
The optional argument \textit{main}
passes on directly to the main file \textit{main}
while pretending to compile the child \textit{dest}.
This form behaves as if \textit{dest}
issues |\childdocof{|\textit{main}|}| right away,
and no further \textsf{childdoc} directives will be processed.

%%%%%%%%%%%%%%%%%%%%%%%%%%%%%%%%%%%%%%%%
\DescribeMacro{\...prefix}
In the alternative form |\childdocforwardprefix|,
%
\begin{center}
\begin{tabular}{l}
|\input{childdoc.def}|\\
|\childdocforwardprefix[|\textit{main}|]{|\textit{prefix}|}{|\textit{dest}|}|
\end{tabular}
\end{center}
%
the destination file is determined by a pattern
depending on the current file:
To make this work, the current file must be called
`{\textit{prefix}\hspace{0.2em}\textit{suffix}}'
with \textit{prefix} matching precisely the argument.
Processing is then passed on to the file
`{\textit{dest}\hspace{0.2em}\textit{suffix}}'.
Surely, the same effect is achieved by
directly specifying the
argument `{\textit{dest}\hspace{0.2em}\textit{suffix}}'
in the first form.
However, that requires to set up a different file
for each child. With the alternative form of the command
all these files can have exactly the same content
which simplifies setting them up and maintaining them.

For example, the following file |draft.tex|
with a compilation flag |\version| as described in \secref{sec:flags}
compiles the main document as a draft:
%
\begin{center}
\begin{tabular}{l}
|\def\version{draft}|\\
|\input{childdoc.def}|\\
|\childdocforward{|\textit{main}|}|
\end{tabular}
\end{center}
%
Likewise, the following files |final|\textit{nn}|.tex|
compile the final version of the child document
|child|\textit{nn}|.tex|:
%
\begin{center}
\begin{tabular}{l}
|\def\version{final}|\\
|\input{childdoc.def}|\\
|\childdocforwardprefix{final}{child}|
\end{tabular}
\end{center}
%

Note that when several versions of a main file and/or of each child file
are to be generated, it may be convenient to set up a |Makefile| or
shell script to automatise the process.

%%%%%%%%%%%%%%%%%%%%%%%%%%%%%%%%%%%%%%%%%%%%%%%%%%%%%%%%%%%%%%%%%%%%%%%%%%%%%%%%
\subsection{Command Line Processing}
\label{sec:commandline}

The effect of redirection files can also be achieved by invoking
the \LaTeX{} compiler with a more elaborate command line.
Most conveniently this should be done as part
of a shell script or a |Makefile|.

When using \textsf{childdoc} in the main file, the following
command lines effectively perform a redirection
(note that depending on the shell being used,
backslashes may have to be doubled: `|\|' $\to$ `|\\|'):
%
\begin{center}
|... -jobname "|\textit{target}|" |\\|"|[\textit{flags}]%
|\input{childdoc.def}\childdocforward[|\textit{main}|]{|\textit{dest}|}"|
\end{center}
%
Here \textit{target} is the name of the output file,
\textit{main} is the name of the main file
and \textit{dest} is the name of the main or child file to be processed
(all filenames without extensions).
The optional argument \textit{main} can be omitted
if \textit{main} matches \textit{dest}.
Optionally, compilation \textit{flags} can be defined via |\def| commands.
This command line makes the \TeX{} engine believe
it is compiling the file \textit{target}
whose content is specified as the latter parameter.
The provided code then forwards the processing to
\textit{main} or \textit{dest} as described in \secref{sec:forward}.

%%%%%%%%%%%%%%%%%%%%%%%%%%%%%%%%%%%%%%%%%%%%%%%%%%%%%%%%%%%%%%%%%%%%%%%%%%%%%%%%
\subsection{Include by Input}
\label{sec:input}

Including child documents by |\include| has some restrictions by design.
Most notably, the content of a child document always occupies
its own set of pages; pages cannot be shared between child documents.
Usually, this behaviour makes perfect sense
because each child document contain an essential part of the document.
However, in some situations it may be desirable to compose
a document from a collection of parts
without having mandatory page breaks between then.
For this case, the package
provides a mechanism to include parts
by |\input| which can also be processed individually.
However, by construction this mechanism
requires manual handling of the content to be output.

%%%%%%%%%%%%%%%%%%%%%%%%%%%%%%%%%%%%%%%%
\DescribeMacro{\ifchilddocmanual}
The main file should be prepared as usual, see \secref{sec:include}.
However, the document body must make a distinction
between processing of an individual part and of the main document, e.g.:
%
\begin{center}
\begin{tabular}{l}
|\ifchilddocmanual|\\
|\input{\childdocname}|\\
|\||else|\\
\textit{document body with }|\input{|\textit{part}|}|\\
|\||fi|
\end{tabular}
\end{center}
%
The conditional |\ifchilddocmanual| is true whenever
a part to be included by |\input| is being compiled,
and the name of the part is stored in |\childdocname|.

%%%%%%%%%%%%%%%%%%%%%%%%%%%%%%%%%%%%%%%%
\DescribeMacro{\childdocby}
Each part to be included by |\input| should start with:
%
\begin{center}
\begin{tabular}{l}
|\input{childdoc.def}|\\
|\childdocby{|\textit{main}|}|\\
\end{tabular}
\end{center}
%
The directive |\childdocby| is similar to |\childdocof|
described in \secref{sec:include},
but the subsequent selection of content must be done manually.
To that end, both |\ifchilddoc| and |\ifchilddocmanual|
will be true upon processing of a part,
and the name of the part is stored in |\childdocname|.
Note that |\jobname| will be set to the filename of the current part
so that each part receives an individual |.aux| file
that does not interfere with the |.aux| file(s) of the main document.
This behaviour can be altered by the alternative form
|\childdocby[*]{|\textit{main}|}| (with a non-empty optional argument)
which uses the |.aux| file of the main document
by setting |\jobname| to \textit{main}.

%%%%%%%%%%%%%%%%%%%%%%%%%%%%%%%%%%%%%%%%%%%%%%%%%%%%%%%%%%%%%%%%%%%%%%%%%%%%%%%%
\subsection{Driver Development}
\label{sec:driver}

The \textsf{childdoc} mechanism can also be use for the development
of definition files such as \LaTeX{} styles or classes.
This case differs from the above setup with multiple parts
included by |\include| in that no |\includeonly| should be invoked.
This can be achieved by starting the include file
(before |\ProvidesPackage|) with:
%
\begin{center}
\begin{tabular}{l}
|\input{childdoc.def}|\\
|\childdocforward{|\textit{main}|}|\\
\end{tabular}
\end{center}
%
or alternatively with:
%
\begin{center}
\begin{tabular}{l}
|\input{childdoc.def}|\\
|\childdocby{|\textit{main}|}|\\
\end{tabular}
\end{center}
%
Both forms have slightly different effects as described above.
The main file is prepared as usual, see \secref{sec:include}.

%%%%%%%%%%%%%%%%%%%%%%%%%%%%%%%%%%%%%%%%%%%%%%%%%%%%%%%%%%%%%%%%%%%%%%%%%%%%%%%%
\subsection{Legacy Detection}
\label{sec:detection}

The directive |\childdocmain| in the main file can detect
whether the complete document or merely a child is to be compiled
even without using the directive |\childdocof|.
This method is deprecated because it is less robust
and there is no compelling reason to use it;
it is merely provided for backward compatibility
and it may be removed in future versions.

If the detection mechanism is to be used,
it is mandatory to correctly specify
the filename of the main file as the argument of |\childdocmain|:
%
\begin{center}
\begin{tabular}{l}
|\input{childdoc.def}|\\
|\childdocmain{|\textit{main}|}|\\
\end{tabular}
\end{center}
%
If |\jobname| does not match the argument \textit{main} of |\childdocmain|,
it is assumed that |\jobname| points to the child file to be compiled.
When using |\childdocmain| with the main file specified as argument,
it suffices to start a child file
with just |\input{|\textit{main}|}|
without loading of the package and using |\childdocof|.
If instead all processing is done
with the appropriate \textsf{childdoc} directives,
the argument of \textit{main} of |\childdocmain| can be empty.

An alternative version of the command line processing described
in \secref{sec:commandline} using the detection mechanism reads:
%
\begin{center}
|... -jobname "|\textit{target}|" "|[\textit{flags}]%
[|\def\jobname{|\textit{dest}|}|]|\input{|\textit{main}|}"|
\end{center}

%%%%%%%%%%%%%%%%%%%%%%%%%%%%%%%%%%%%%%%%%%%%%%%%%%%%%%%%%%%%%%%%%%%%%%%%%%%%%%%%
\subsection{Manual Code}
\label{sec:manual}

In case one cannot be certain whether the definitions file |childdoc.def|
is installed on the target \TeX{} distribution
and one prefers not to ship it,
it is conceivable to paste a few relevant commands into the sources.

To that end, drop all statements |\input{childdoc.def}|
and perform the replacements as outlined below.
Instead of |\childdocmain{|\textit{main}|}| add the following code
to the top of the main file:
%
\begin{center}
\begin{tabular}{l}
|\||ifdefined\childdocname\endinput\||fi\newif\ifchilddoc|\\
|\edef\childdocname{\scantokens\expandafter{\jobname\noexpand}}|\\
|\def\childdocmain{|\textit{main}|}\||ifx\childdocmain\childdocname\||else|\\
|\childdoctrue\includeonly{\childdocname}\let\jobname\childdocmain\||fi|\\
\end{tabular}
\end{center}
%
Instead of |\childdocof{|\textit{main}|}| just include the main file
at the top of each child file:
%
\begin{center}
|\input{|\textit{main}|}|
\end{center}
%
A simple redirection |\childdocforward{|\textit{dest}|}| is achieved by:
%
\begin{center}
|\def\jobname{|\textit{dest}|}\input{\jobname}|
\end{center}
%
The redirection with prefix
|\childdocforwardprefix[|\textit{prefix}|]{|\textit{dest}|}|
is accomplished by:
%
\begin{center}
\begin{tabular}{l}
|{\edef\jobname{\scantokens\expandafter{\jobname\noexpand}}|\\
|\def\redirectjob |\textit{prefix}|#1~~~{\gdef\jobname{|\textit{dest}|#1}}|\\
|\expandafter\redirectjob\jobname~~~}\input{\jobname}|
\end{tabular}
\end{center}

In an alternative approach,
child documents can be compiled by a specific command line
without additional code or specific definitions:
%
\begin{center}
|... -jobname "|\textit{target}|" "|[\textit{flags}]%
|\includeonly{|\textit{dest}|}\input{|\textit{main}|}"|
\end{center}
%

%%%%%%%%%%%%%%%%%%%%%%%%%%%%%%%%%%%%%%%%%%%%%%%%%%%%%%%%%%%%%%%%%%%%%%%%%%%%%%%%
%%%%%%%%%%%%%%%%%%%%%%%%%%%%%%%%%%%%%%%%%%%%%%%%%%%%%%%%%%%%%%%%%%%%%%%%%%%%%%%%
\section{Information}

%%%%%%%%%%%%%%%%%%%%%%%%%%%%%%%%%%%%%%%%%%%%%%%%%%%%%%%%%%%%%%%%%%%%%%%%%%%%%%%%
\subsection{Copyright}

Copyright \copyright{} 2017--2018 Niklas Beisert

This work may be distributed and/or modified under the
conditions of the \LaTeX{} Project Public License, either version 1.3
of this license or (at your option) any later version.
The latest version of this license is in
  \url{http://www.latex-project.org/lppl.txt}
and version 1.3 or later is part of all distributions of \LaTeX{}
version 2005/12/01 or later.

This work has the LPPL maintenance status `maintained'.

The Current Maintainer of this work is Niklas Beisert.

This work consists of the files |README.txt|, |childdoc.ins| and |childdoc.dtx|
as well as the derived files |childdoc.def|, |cdocsamp.tex|
with |cdocsch1.tex|, |cdocsch2.tex|, |cdocspt3.tex|, |cdocspt4.tex|,
|cdocsdrf.tex|, |cdocsfn1.tex|, |cdocsfn2.tex|
as well as |childdoc.pdf|.

%%%%%%%%%%%%%%%%%%%%%%%%%%%%%%%%%%%%%%%%%%%%%%%%%%%%%%%%%%%%%%%%%%%%%%%%%%%%%%%%
\subsection{Files and Installation}

The package consists of the files:
%
\begin{center}
\begin{tabular}{ll}
    |README.txt|   & readme file \\
    |childdoc.ins| & installation file \\
    |childdoc.dtx| & source file \\
    |childdoc.def| & definition file \\
    |cdocsamp.tex| & sample main file \\
    |cdocsch1.tex| & sample include file \\
    |cdocsch2.tex| & sample include file \\
    |cdocspt3.tex| & sample part file \\
    |cdocspt4.tex| & sample part file \\
    |cdocsdrf.tex| & sample redirection file \\
    |cdocsfn1.tex| & sample redirection file \\
    |cdocsfn2.tex| & sample redirection file \\
    |childdoc.pdf| & manual
\end{tabular}
\end{center}
%
The distribution consists of the files
|README.txt|, |childdoc.ins| and |childdoc.dtx|.
%
\begin{itemize}
\item
Run (pdf)\LaTeX{} on |childdoc.dtx|
to compile the manual |childdoc.pdf| (this file).
\item
Run \LaTeX{} on |childdoc.ins| to create the definitions file |childdoc.def|
and the sample |cdocsamp.tex| with include files
|cdocsch1.tex|, |cdocsch2.tex|, |cdocspt3.tex|, |cdocspt4.tex|,
|cdocsdrf.tex|, |cdocsfn1.tex|, |cdocsfn2.tex|.
Then copy the file |childdoc.def| to an appropriate directory of your \LaTeX{}
distribution, e.g.\ \textit{texmf-root}|/tex/latex/childdoc|.
\end{itemize}

%%%%%%%%%%%%%%%%%%%%%%%%%%%%%%%%%%%%%%%%%%%%%%%%%%%%%%%%%%%%%%%%%%%%%%%%%%%%%%%%
\subsection{Related CTAN Packages}

There are several other packages which offer a similar functionality:
%
\begin{itemize}
\item
The packages
\href{http://ctan.org/pkg/docmute}{\textsf{docmute}},
\href{http://ctan.org/pkg/includex}{\textsf{includex}} and
\href{http://ctan.org/pkg/standalone}{\textsf{standalone}}
provide commands to include only the document body of
a child file thus allowing both files to be compiled individually.
\item
The packages \href{http://ctan.org/pkg/subdocs}{\textsf{subdocs}}
and \href{http://ctan.org/pkg/subfiles}{\textsf{subfiles}}
provide structures in which the main and child documents can be
encapsulated and allowing them to be compiled individually.
The inclusion mechanism is different from the conventional |\include|.
\item
The package \href{http://ctan.org/pkg/combine}{\textsf{combine}}
is an elaborate solution to combine several documents into one.
\end{itemize}
%
See also the CTAN topic \href{http://ctan.org/topic/subdocs}{\textsf{subdocs}}
for further related packages.
The present package differs from the above solutions in that
a document structure constructed with the conventional |\include| mechanism
just needs two extra commands at the top of every file
such that all constituent files can be compiled individually.

%%%%%%%%%%%%%%%%%%%%%%%%%%%%%%%%%%%%%%%%%%%%%%%%%%%%%%%%%%%%%%%%%%%%%%%%%%%%%%%%
%\subsection{Feature Suggestions}
%
%The following is a list of features which may be useful for future
%versions of this package:
%%
%\begin{itemize}
%\item
%\ldots
%\end{itemize}

%%%%%%%%%%%%%%%%%%%%%%%%%%%%%%%%%%%%%%%%%%%%%%%%%%%%%%%%%%%%%%%%%%%%%%%%%%%%%%%%
\subsection{Revision History}

%%%%%%%%%%%%%%%%%%%%%%%%%%%%%%%%%%%%%%%%
\paragraph{v2.0:} 2018/12/30

\begin{itemize}
\item
immediate forward processing
\item
added |\childdocby| mechanism
\item
manual restructured
\end{itemize}

%%%%%%%%%%%%%%%%%%%%%%%%%%%%%%%%%%%%%%%%
\paragraph{v1.6:} 2018/01/17

\begin{itemize}
\item
application for development of include files
\item
corrections to manual
\end{itemize}

%%%%%%%%%%%%%%%%%%%%%%%%%%%%%%%%%%%%%%%%
\paragraph{v1.5:} 2017/05/21

\begin{itemize}
\item
more complete structuring introduced
\item
|\childdocof| introduced
\item
|\childdoc| renamed to |\childdocmain|
\item
|\childredirect| renamed to |\childdocforward| and |\childdocforwardprefix|
and functionality expanded
\end{itemize}

%%%%%%%%%%%%%%%%%%%%%%%%%%%%%%%%%%%%%%%%
\paragraph{v1.0:} 2017/04/27

\begin{itemize}
\item
manual and install package
\item
first version published on CTAN
\end{itemize}

%%%%%%%%%%%%%%%%%%%%%%%%%%%%%%%%%%%%%%%%
\paragraph{v0.6:} 2017/04/26

\begin{itemize}
\item
redirection mechanism added
\end{itemize}

%%%%%%%%%%%%%%%%%%%%%%%%%%%%%%%%%%%%%%%%
\paragraph{v0.5:} 2017/04/26

\begin{itemize}
\item
functionality in definition file
\end{itemize}


%%%%%%%%%%%%%%%%%%%%%%%%%%%%%%%%%%%%%%%%%%%%%%%%%%%%%%%%%%%%%%%%%%%%%%%%%%%%%%%%
%%%%%%%%%%%%%%%%%%%%%%%%%%%%%%%%%%%%%%%%%%%%%%%%%%%%%%%%%%%%%%%%%%%%%%%%%%%%%%%%
%%%%%%%%%%%%%%%%%%%%%%%%%%%%%%%%%%%%%%%%%%%%%%%%%%%%%%%%%%%%%%%%%%%%%%%%%%%%%%%%
\appendix

\settowidth\MacroIndent{\rmfamily\scriptsize 000\ }

 \DocInput{childdoc.dtx}

\end{document}
%</driver>
% \fi
%
% %%%%%%%%%%%%%%%%%%%%%%%%%%%%%%%%%%%%%%%%%%%%%%%%%%%%%%%%%%%%%%%%%%%%%%%%%%%%%%
% %%%%%%%%%%%%%%%%%%%%%%%%%%%%%%%%%%%%%%%%%%%%%%%%%%%%%%%%%%%%%%%%%%%%%%%%%%%%%%
% \section{Sample}
%\iffalse
%<*samplemain>
%\fi
%
% The following presents a sample document
% with two chapters, two parts, a title page,
% a compile flag as well as three forwarding files to set the flag.
% It consists of eight |.tex| files:
% \begin{center}
% \begin{tabular}{ll}
% |cdocsamp.tex|&main file\\
% |cdocsch1.tex|&include file for chapter 1\\
% |cdocsch2.tex|&include file for chapter 2\\
% |cdocspt3.tex|&include file for part 3\\
% |cdocspt4.tex|&include file for part 4\\
% |cdocsdrf.tex|&forwarding file for main file in draft mode\\
% |cdocsfi1.tex|&forwarding file for final version of chapter 1\\
% |cdocsfi2.tex|&forwarding file for final version of chapter 2\\
% \end{tabular}
% \end{center}
% Each of the eight files can be compiled directly by the \LaTeX{} compiler.
%
% %%%%%%%%%%%%%%%%%%%%%%%%%%%%%%%%%%%%%%
% \paragraph{Main File.}
%
% The main file is called |cdocsamp.tex|.
%
% Load the \textsf{childdoc} definitions and
% declare the filename for the main document:
%    \begin{macrocode}
\input{childdoc.def}
\childdocmain{}
%    \end{macrocode}

% Optional override for |\version| flag:
%    \begin{macrocode}
%%\ifchilddoc\else\providecommand{\version}{draft}\fi
%    \end{macrocode}

% Define the default values for the |\version| flag
% (|final| for the main file and |draft| for childs):
%    \begin{macrocode}
\ifchilddoc
\providecommand{\version}{draft}
\else
\providecommand{\version}{final}
\fi
%    \end{macrocode}

% Load the standard document class:
%    \begin{macrocode}
\documentclass[12pt]{article}
%    \end{macrocode}

% Start the document body:
%    \begin{macrocode}
\begin{document}
%    \end{macrocode}

% Declare a title page.
% Print title, part of document being processed and version flag:
%    \begin{macrocode}
\addtocounter{page}{-1}
\begin{center}
{\LARGE\bfseries{}childdoc example\par}
\vspace{1cm}
\ifchilddoc
\ifchilddocmanual part\else chapter\fi:
`\childdocname' of `\childdocjob'\par
\else
main document: `\childdocjob'\par
\fi
version: \version\par
\end{center}
\newpage
%    \end{macrocode}

% Manually include selected file,
% otherwise process as usual:
%    \begin{macrocode}
\ifchilddocmanual
\section*{part `\childdocname'}
\input{\childdocname}
\else
%    \end{macrocode}

% Include the two chapters:
%    \begin{macrocode}
\include{cdocsch1}
\include{cdocsch2}
%    \end{macrocode}

% Include the two parts unless only chapters should be displayed:
%    \begin{macrocode}
\ifchilddoc\else
\section{part three}
\input{cdocspt3}
\section{part four}
\input{cdocspt4}
\fi
%    \end{macrocode}

% Process as usual until here:
%    \begin{macrocode}
\fi
%    \end{macrocode}

% End of document body:
%    \begin{macrocode}
\end{document}
%    \end{macrocode}
%\iffalse
%</samplemain>
%\fi
%
% %%%%%%%%%%%%%%%%%%%%%%%%%%%%%%%%%%%%%%
% \paragraph{Chapter Include Files.}
%
% The include files are called |cdocsch1.tex| and |cdocsch2.tex|.
%
%\iffalse
%<*samplechap1|samplechap2>
%\fi

% Optional override for |\version| flag:
%    \begin{macrocode}
%%\providecommand{\version}{final}
%    \end{macrocode}

% Include the main document:
%    \begin{macrocode}
\input{childdoc.def}
\childdocof{cdocsamp}
%    \end{macrocode}

%\iffalse
%</samplechap1|samplechap2>
%\fi
%
%\iffalse
%<*samplechap1>
%\fi
% Some text for chapter 1:
%    \begin{macrocode}
\section{one}
some text in chapter one
%    \end{macrocode}

%\iffalse
%</samplechap1>
%\fi
% Some text for chapter 2:
%\iffalse
%<*samplechap2>
%\fi
%    \begin{macrocode}
\section{two}
more text in chapter two
%    \end{macrocode}

%\iffalse
%</samplechap2>
%\fi
%
% %%%%%%%%%%%%%%%%%%%%%%%%%%%%%%%%%%%%%%
% \paragraph{Part Include Files.}
%
% The include files are called |cdocspt3.tex| and |cdocspt4.tex|.
%
%\iffalse
%<*samplepart3|samplepart4>
%\fi

% Optional override for |\version| flag:
%    \begin{macrocode}
%%\providecommand{\version}{final}
%    \end{macrocode}

% Include the main document:
%    \begin{macrocode}
\input{childdoc.def}
\childdocby{cdocsamp}
%    \end{macrocode}

%\iffalse
%</samplepart3|samplepart4>
%\fi
%
%\iffalse
%<*samplepart3>
%\fi
% Some text for part 3:
%    \begin{macrocode}
some text in part three
%    \end{macrocode}

%\iffalse
%</samplepart3>
%\fi
% Some text for part 4:
%\iffalse
%<*samplepart4>
%\fi
%    \begin{macrocode}
more text in part four
%    \end{macrocode}

%\iffalse
%</samplepart4>
%\fi
%
% %%%%%%%%%%%%%%%%%%%%%%%%%%%%%%%%%%%%%%
% \paragraph{Forwarding for a Complete Draft.}
%
% The following forwarding file |cdocsdrf.tex|
% compiles the main document in draft mode:
%\iffalse
%<*sampledraft>
%\fi
%    \begin{macrocode}
\def\version{draft}
\input{childdoc.def}
\childdocforward{cdocsamp}
%    \end{macrocode}

%\iffalse
%</sampledraft>
%\fi
%
% %%%%%%%%%%%%%%%%%%%%%%%%%%%%%%%%%%%%%%
% \paragraph{Forwarding for Final Version of the Chapters.}
%
% The following forwarding files |cdocsfn1.tex| and |cdocsfn2.tex|
% (with identical content)
% compile the final versions of the child documents
% |cdocsch1.tex| and |cdocsch2.tex|, respectively:
%\iffalse
%<*samplefinal>
%\fi
%    \begin{macrocode}
\def\version{final}
\input{childdoc.def}
\childdocforwardprefix[cdocsamp]{cdocsfn}{cdocsch}
%    \end{macrocode}

%\iffalse
%</samplefinal>
%\fi
%
% %%%%%%%%%%%%%%%%%%%%%%%%%%%%%%%%%%%%%%
% \paragraph{Command Line Processing.}
%
% The following three command lines generate the output files
% |cdocscld|, |cdocscl1| and |cdocscl2|
% which should be identical to
% |cdocsdrf|, |cdocsch1| and |cdocsfn2|, respectively:
% \begin{center}
% \begin{tabular}{l}
% |latex -jobname cdocscld \|\\
% |  "\def\version{draft}\input{childdoc.def}\childdocforward{cdocsamp}"|\\
% |latex -jobname cdocscl1 \|\\
% |  "\input{childdoc.def}\childdocforward[cdocsamp]{cdocsch1}"|\\
% |latex -jobname cdocscl2 \|\\
% |  "\def\version{final}\input{childdoc.def}\childdocforward{cdocsch2}"|
% \end{tabular}
% \end{center}
% Note that the trailing backslash on each first line
% merely continues the input to the second line
% (for convenient cut ant paste).
% Furthermore, the command |latex| can be replaced by any
% of its alternative versions such as |pdflatex|.
%
% %%%%%%%%%%%%%%%%%%%%%%%%%%%%%%%%%%%%%%%%%%%%%%%%%%%%%%%%%%%%%%%%%%%%%%%%%%%%%%
% %%%%%%%%%%%%%%%%%%%%%%%%%%%%%%%%%%%%%%%%%%%%%%%%%%%%%%%%%%%%%%%%%%%%%%%%%%%%%%
% \section{Implementation}
%\iffalse
%<*package>
%\fi
%
% This section describes the definitions file |childdoc.def|.

% The definitions cannot be loaded using |\usepackage| or |\RequirePackage|
% which has a mechanism to prevent loading a style file more than once.
% When loading the definitions by means of |\input|
% multiple instances have to be prevented manually:
%\iffalse
%This code needs to be before the `\ProvidesFile' directive
%which is defined at the beginning of this file.
%Therefore it is also placed there and commented out here.
%</package>
%<*discard>
%\fi
%    \begin{macrocode}
\ifdefined\childdocmain\endinput\fi
%    \end{macrocode}
%\iffalse
%</discard>
%<*package>
%\fi
%
% \macro{\ifchilddoc}
% \macro{\ifchilddocmanual}
% The conditional |\ifchilddoc| tells whether a
% child (true) or main (false) document is being compiled.
% The conditional |\ifchilddocmanual| tells whether
% the |\includeonly| mechanism is used (false) or
% the selection of child files must be performed manually (true).
% The definitions initialise to false:
%    \begin{macrocode}
\newif\ifchilddoc
\newif\ifchilddocmanual
%    \end{macrocode}

% \macro{\childdocname}
% \macro{\childdocjob}
% The macro |\childdocname| stores the name of the main document
% to be compiled. The macro |\childdocjob| stores the name of
% the document on which the \LaTeX{} compiler was originally invoked.
% The content of |\jobname| cannot be compared
% to filenames specified in the source due to different catcodes.
% The following code rescans |\jobname|, stores the result
% in |\childdocname| and saves a copy in |\childdocjob|:
%    \begin{macrocode}
\edef\childdocname{\scantokens\expandafter{\jobname\noexpand}}
\let\childdocjob\childdocname
%    \end{macrocode}

% \macro{\childdocdisable}
% The macro |\childdocdisable| prevents the main file
% from being processed more than once.
% At this stage, the main document command |\childdocmain|
% is assumed to be called once again where it should do nothing.
% Any subsequent call to it should prevent
% a secondary processing of the main document
% It overwrites the forwarding commands
% |\childdocof| and |\childdocforward|
% with empty macros to prevent further inclusions of the main document:
%    \begin{macrocode}
\newcommand{\childdocdisable}
{
  \renewcommand{\childdocmain}[1]{\renewcommand{\childdocmain}[1]{\endinput}}
  \renewcommand{\childdocof}[1]{}
  \renewcommand{\childdocby}[2][]{}
  \renewcommand{\childdocforward}[2][]{}
  \renewcommand{\childdocdisable}{}
}
%    \end{macrocode}

% \macro{\childdocmain}
% The macro |\childdocmain| is to be called at the top of the main file
% with nothing or the main filename (without extension) as argument.
% First, it breaks loops.
% If the argument is not empty and does not match |\childdocname|
% (which is set by the first inclusion of |childdoc.def|),
% |\ifchilddoc| is set to true, |\includeonly| is applied to the child file
% and |\jobname| is set to the main file
% (for proper handling of |.aux| files):
%    \begin{macrocode}
\newcommand{\childdocmain}[1]
{
  \childdocdisable\childdocmain{}
  \if?#1?\else
    \begingroup
      \def\childdoctmp{#1}
      \ifx\childdoctmp\childdocname
        \def\childdoctmp{}
      \else
        \def\childdoctmp
        {
          \childdoctrue
          \includeonly{\childdocname}
          \def\childdocjob{#1}
          \def\jobname{#1}
        }
      \fi
      \expandafter
    \endgroup
    \childdoctmp
  \fi
}
%    \end{macrocode}

% \macro{\childdocof}
% The command |\childdocof| redirects
% compilation to the main file |#1|.
%    \begin{macrocode}
\newcommand{\childdocof}[1]
{
  \childdocdisable
  \childdoctrue
  \includeonly{\childdocname}
  \def\jobname{#1}
  \def\childdocjob{#1}
  \input{#1}
}
%    \end{macrocode}

% \macro{\childdocby}
% The command |\childdocby| ....
%    \begin{macrocode}
\newcommand{\childdocby}[2][]
{
  \childdocdisable
  \childdoctrue
  \childdocmanualtrue
  \if?#1?\else
    \def\jobname{#2}
  \fi
  \def\childdocjob{#2}
  \input{#2}
  \endinput
}
%    \end{macrocode}

% \macro{\childdocforward}
% The command |\childdocforward| redirects
% compilation to the main file or
% (if the optional argument is given) a child file.
% Parameters are set as if the main file
% or a child file starting with |\childdocof| was compiled.
% Then compilation is handed over to the main file:
%    \begin{macrocode}
\newcommand{\childdocforward}[2][]
{
  \begingroup
    \if?#1?
      \def\childdoctmp
      {
        \def\childdocname{#2}
        \def\childdocjob{#2}
        \def\jobname{#2}
        \input{#2}
        \endinput
      }
    \else
      \def\childdoctmp
      {
        \childdocdisable
        \def\childdocname{#2}
        \childdoctrue
        \includeonly{#2}
        \def\childdocjob{#1}
        \def\jobname{#1}
        \input{#1}
        \endinput
      }
    \fi
    \expandafter
  \endgroup
  \childdoctmp
}
%    \end{macrocode}

% \macro{\childdocforwardprefix}
% The command |\childdocforwardprefix| redirects
% compilation to the main or a child file by means of a pattern.
% The prefix |#1| in the current filename is replaced by |#2|
% and the suffix of the current filename is kept
% (it is assumed that the filename does not contain the substring `|~~~|'
% which is used as a delimiter).
% Compilation is handed over to the new file by |\childdocforward|:
%    \begin{macrocode}
\newcommand{\childdocforwardprefix}[3][]
{
  \begingroup
    \def\childdocextract #2##1~~~{\def\childdoctmp{\childdocforward[#1]{#3##1}}}
    \expandafter\childdocextract\childdocname~~~
    \expandafter
  \endgroup
  \childdoctmp
}
%    \end{macrocode}

% \macro{\childdoc}
% The deprecated macro |\childdoc| is a legacy version of |\childdocmain|:
%    \begin{macrocode}
\newcommand{\childdoc}{\childdocmain}
%    \end{macrocode}

% \macro{\childdocredirect}
% The deprecated macro |\childdocredirect| is a legacy version
% of |\childdocforward| and |\childdocforwardprefix|:
%    \begin{macrocode}
\newcommand{\childdocredirect}[2][]
{
  \begingroup
    \if?#1?
      \def\childdoctmp{\childdocforward{#2}}
    \else
      \def\childdoctmp{\childdocforwardprefix{#1}{#2}}
    \fi
    \expandafter
  \endgroup
  \childdoctmp
}
%    \end{macrocode}

%\iffalse
%</package>
%\fi
%
\endinput
|\\
|\childdocmain{}|\\
\end{tabular}
\end{center}
at the very top of the main \LaTeX{} file,
in particular \emph{before} the |\documentclass| statement!
The argument of |\childdocmain| should be left empty
(but it must be present).

%%%%%%%%%%%%%%%%%%%%%%%%%%%%%%%%%%%%%%%%
\DescribeMacro{\childdocof}
Furthermore, add the commands
\begin{center}
\begin{tabular}{l}
|% \iffalse
%
% childdoc.dtx Copyright (C) 2017-2018 Niklas Beisert
%
% This work may be distributed and/or modified under the
% conditions of the LaTeX Project Public License, either version 1.3
% of this license or (at your option) any later version.
% The latest version of this license is in
%   http://www.latex-project.org/lppl.txt
% and version 1.3 or later is part of all distributions of LaTeX
% version 2005/12/01 or later.
%
% This work has the LPPL maintenance status `maintained'.
%
% The Current Maintainer of this work is Niklas Beisert.
%
% This work consists of the files childdoc.dtx and childdoc.ins
% and the derived files childdoc.def and cdocsamp.tex with
% cdocsch1.tex, cdocsch2.tex, cdocsdrf.tex, cdocsfn1.tex, cdocsfn2.tex.
%
%<package>\ifdefined\childdocmain\endinput\fi
%<package>\ProvidesFile{childdoc.def}[2018/12/30 v2.0 child document driver]
%<samplemain>\ProvidesFile{cdocsamp.tex}[2018/12/30 v2.0 sample for childdoc]
%<*driver>
%\ProvidesFile{childdoc.drv}[2018/12/30 v2.0 childdoc reference manual file]
\PassOptionsToClass{10pt,a4paper}{article}
\documentclass{ltxdoc}

\usepackage[margin=35mm]{geometry}
\usepackage{hyperref}
\usepackage{hyperxmp}
\usepackage[usenames]{color}

\hypersetup{colorlinks=true}
\hypersetup{pdfstartview=FitH}
\hypersetup{pdfpagemode=UseNone}
\hypersetup{pdfsource={}}
\hypersetup{pdflang={en-UK}}
\hypersetup{pdfcopyright={Copyright 2017-2018 Niklas Beisert.
  This work may be distributed and/or modified under the
  conditions of the LaTeX Project Public License, either version 1.3
  of this license or (at your option) any later version.}}
\hypersetup{pdflicenseurl={http://www.latex-project.org/lppl.txt}}
\hypersetup{pdfcontactaddress={ETH Zurich, ITP, HIT K,
  Wolfgang-Pauli-Strasse 27}}
\hypersetup{pdfcontactpostcode={8093}}
\hypersetup{pdfcontactcity={Zurich}}
\hypersetup{pdfcontactcountry={Switzerland}}
\hypersetup{pdfcontactemail={nbeisert@itp.phys.ethz.ch}}
\hypersetup{pdfcontacturl={http://people.phys.ethz.ch/\xmptilde nbeisert/}}

\newcommand{\secref}[1]{\hyperref[#1]{section \ref*{#1}}}

\parskip1ex
\parindent0pt
\let\olditemize\itemize
\def\itemize{\olditemize\parskip0pt}

\begin{document}

\title{The \textsf{childdoc} Package}
\hypersetup{pdftitle={The childdoc Package}}
\author{Niklas Beisert\\[2ex]
  Institut f\"ur Theoretische Physik\\
  Eidgen\"ossische Technische Hochschule Z\"urich\\
  Wolfgang-Pauli-Strasse 27, 8093 Z\"urich, Switzerland\\[1ex]
  \href{mailto:nbeisert@itp.phys.ethz.ch}
  {\texttt{nbeisert@itp.phys.ethz.ch}}}
\hypersetup{pdfauthor={Niklas Beisert}}
\hypersetup{pdfsubject={Manual for the LaTeX2e Package childdoc}}
\date{30 December 2018, \textsf{v2.0}}
\maketitle

\begin{abstract}\noindent
\textsf{childdoc} is a \LaTeXe{} package
that enables the direct compilation
of document sections included by |\include|
to individual files.
\end{abstract}

\begingroup
\parskip0ex
\tableofcontents
\endgroup

%%%%%%%%%%%%%%%%%%%%%%%%%%%%%%%%%%%%%%%%%%%%%%%%%%%%%%%%%%%%%%%%%%%%%%%%%%%%%%%%
%%%%%%%%%%%%%%%%%%%%%%%%%%%%%%%%%%%%%%%%%%%%%%%%%%%%%%%%%%%%%%%%%%%%%%%%%%%%%%%%
\section{Introduction}

\LaTeX{} provides a mechanism to structure a large document (such as a book)
into a main file and several child files (containing the chapters)
using the |\include| command.
This mechanism is beneficial for documents
which span hundreds of pages in order to
make the source file(s) more manageable.
Moreover, compilation can be restricted to
selected child files by means of the |\includeonly| command.
The latter feature can be used to reduce the compilation time while editing
(this was significantly more useful in the earlier days of \LaTeX{})
or to generate a smaller document which is easier to navigate.
Another application of |\includeonly| is to generate
documents consisting of selected parts of the complete document.

However, there are a few drawbacks of the plain |\include| mechanism:
\begin{itemize}
\item
The child files cannot be compiled on their own,
they can only be compiled via the main file.
A naive editing environment
(such as a text editor with an option
to have the current file processed by \LaTeX)
may require one to switch to the main file before compiling;
attempting to compile the child file produces errors.
\item
The main file must be modified (each time)
to adjust the |\includeonly| command
to the present needs. This easily leaves the main file in a messy state.
\item
The generated document will always carry the filename
of the main document. This is inconvenient if
several child files are to be compiled and
to be kept for distribution.
\end{itemize}

The present package provides a simple interface
to make child files individually compilable by \LaTeX{}.
Compiling a child file then has the same effect as compiling
the main file with an |\includeonly| command
to select the appropriate child.
Moreover the generated document will carry the name of the child
rather than the main file.
This resolves all three above issues.

This feature is meant to make the editing of books,
thesis documents and lecture notes somewhat more convenient.
However, the package can also be used efficiently for
composing a series of documents (such as exercise sheets)
which are typically distributed individually.
It then assists the author in generating the individual documents
(potentially in different versions)
as well as a document containing the collected series.
Another application is in developing style files
or other kinds of included material
where compilation of the style file could redirect
to a sample or test file.

%%%%%%%%%%%%%%%%%%%%%%%%%%%%%%%%%%%%%%%%%%%%%%%%%%%%%%%%%%%%%%%%%%%%%%%%%%%%%%%%
%%%%%%%%%%%%%%%%%%%%%%%%%%%%%%%%%%%%%%%%%%%%%%%%%%%%%%%%%%%%%%%%%%%%%%%%%%%%%%%%
\section{Usage}

First of all, the package \textsf{childdoc} is \emph{not} a standard
\LaTeXe{} |.sty| style file! Therefore it needs to be invoked in
a non-standard way.

%%%%%%%%%%%%%%%%%%%%%%%%%%%%%%%%%%%%%%%%%%%%%%%%%%%%%%%%%%%%%%%%%%%%%%%%%%%%%%%%
\subsection{Included Files}
\label{sec:include}

%%%%%%%%%%%%%%%%%%%%%%%%%%%%%%%%%%%%%%%%
\DescribeMacro{\childdocmain}
To use the package, add the commands
\begin{center}
\begin{tabular}{l}
|\input{childdoc.def}|\\
|\childdocmain{}|\\
\end{tabular}
\end{center}
at the very top of the main \LaTeX{} file,
in particular \emph{before} the |\documentclass| statement!
The argument of |\childdocmain| should be left empty
(but it must be present).

%%%%%%%%%%%%%%%%%%%%%%%%%%%%%%%%%%%%%%%%
\DescribeMacro{\childdocof}
Furthermore, add the commands
\begin{center}
\begin{tabular}{l}
|\input{childdoc.def}|\\
|\childdocof{|\textit{main}|}|\\
\end{tabular}
\end{center}
at the top of every child file \textit{child}
which is included by |\include{|\textit{child}|}|
from within the main file
(or at least for those files to be compiled individually).
The argument \textit{main} must be the filename of the main file.

There are a couple of
considerations in setting up the main and child documents:

%%%%%%%%%%%%%%%%%%%%%%%%%%%%%%%%%%%%%%%%
\paragraph{Restrictions.}

Please note the following restrictions:
\begin{itemize}
\item
|\childdocmain| must be called with one argument \textit{main}
to ensure compatibility with earlier version of the package.
It must either be empty (|\childdocmain{}|)
or precisely match the filename of the main file in which it is specified.
See \secref{sec:detection} for further information.
\item
The filename \textit{main} must be specified without the |.tex| extension.
\item
The filename \textit{main} is case sensitive
(even in case-insensitive file systems)
due to internal string comparison.
\item
The argument \textit{main} should be fully expanded, it cannot be a macro.
\item
Subdirectories and special characters should be avoided in filenames.
\item
The command |\childdocmain{|\textit{main}|}| must be followed by a whitespace.
It should not be followed immediately by another command
or by a comment mark `|%|'.
This is because the \TeX{} parser reads the token immediately following
the argument of |\childdocmain| and puts it
at the beginning of every child section;
however, a white\-space is ignored.
\end{itemize}

%%%%%%%%%%%%%%%%%%%%%%%%%%%%%%%%%%%%%%%%
\paragraph{Content of Main File.}

It is advisable to place all content in the child files included by |\include|.
Any output contained in the main file will appear in all child documents
unless suppressed manually;
it cannot be suppressed automatically by the |\includeonly| directive
and thus should normally be avoided.
A method to include some content in the main file
by means of conditional processing is described in \secref{sec:conditional}.

%%%%%%%%%%%%%%%%%%%%%%%%%%%%%%%%%%%%%%%%
\paragraph{Page Numbering.}

When only a part of the document is compiled,
the appropriate numbering of pages
(as well as other status parameters)
is determined from the |.aux| files.
The latter contain information from previous passes.
However this information needs to propagate through
all intermediate child documents.
Therefore the page numbering in child documents may well
be inconsistent until the complete document is compiled at least once.

A useful (if unconventional) way to always ensure a consistent
page numbering is to restart the numbering in each child document
and denote the pages by `\textit{child}|.|\textit{page}'
where \textit{child} represents the chapter/section number of the child file.
This can be achieved by the command
|\numberwithin{page}{|\textit{child}|}|
of the \textsf{amsmath} package
where \textit{child} can be |chapter| or |section|
depending on the chosen structuring.
Alternatively, one can modify the macro |\thepage| appropriately
and reset the counter |page| at the start of each child file.

%%%%%%%%%%%%%%%%%%%%%%%%%%%%%%%%%%%%%%%%%%%%%%%%%%%%%%%%%%%%%%%%%%%%%%%%%%%%%%%%
\subsection{Conditional Processing}
\label{sec:conditional}

The package provides a mechanism to compile different versions
of a document. To customise the versions further some conditional processing
can come in handy to distinguish which version is being compiled.
The package provides two macros to describe the compilation context:

%%%%%%%%%%%%%%%%%%%%%%%%%%%%%%%%%%%%%%%%
\DescribeMacro{\ifchilddoc}
The conditional |\ifchilddoc| distinguishes between the compilation of
child documents and the main document:
%
\begin{center}
|\ifchilddoc |\textit{child-code}| |[|\||else |\textit{main-code}]| \||fi|
\end{center}

%%%%%%%%%%%%%%%%%%%%%%%%%%%%%%%%%%%%%%%%
\DescribeMacro{\childdocname}
\DescribeMacro{\childdocjob}
The macro |\childdocname| contains the filename (without extension)
of the main or child file being processed.
Note that |\childdocjob| will always contain the name of the main file.

%%%%%%%%%%%%%%%%%%%%%%%%%%%%%%%%%%%%%%%%
\paragraph{Title Page.}

Conditional processing can be used to include a title or banner page
in the main document when proper precautions are taken.
Importantly, the code in the main file should ensure that the page counter
(as well as other status parameters which are stored in the |.aux| files)
takes the same value after the conditional processing.
Otherwise the page numbers may take divergent values
depending on which part is compiled.

For example, a title page could be declared by:
%
\begin{center}
\begin{tabular}{l}
|\ifchilddoc\||else|\\
|\addtocounter{page}{-1}|\\
\textit{code for title page}\\
|\newpage|\\
|\||fi|
\end{tabular}
\end{center}
%
A banner page for the child documents can be generated by:
%
\begin{center}
\begin{tabular}{l}
|\ifchilddoc|\\
|\addtocounter{page}{-1}|\\
\textit{code for banner page}\\
|\newpage|\\
|\||fi|
\end{tabular}
\end{center}
%
Here one could write a message such as:
\begin{center}
|This is the part \childdocname{} of \childdocjob{}.|
\end{center}

%%%%%%%%%%%%%%%%%%%%%%%%%%%%%%%%%%%%%%%%%%%%%%%%%%%%%%%%%%%%%%%%%%%%%%%%%%%%%%%%
\subsection{Flags}
\label{sec:flags}

The package makes it easy to generate different versions
of the main or child documents.
To this end compilation flags can be defined
and assigned different default values.
They will be particularly useful in conjunction
with the forwarding mechanism described in \secref{sec:forward}.

For example, it may be useful to have a flag |\version|
which can be set to |draft| or |final|.
The document source will contain some conditional code
depending on the value of |\version|.
Suppose further, the flag should default to |final| for the main file
and to |draft| for child files
which is a natural assignment for editing the document.
This is achieved by placing the following code
in the preamble of the main document
(below the |\childdocmain| directive):
%
\begin{center}
\begin{tabular}{l}
|\ifchilddoc|\\
|\providecommand{\version}{draft}|\\
|\||else|\\
|\providecommand{\version}{final}|\\
|\||fi|
\end{tabular}
\end{center}
%
The definition by |\providecommand| makes sure
that previous definitions are not overwritten.
Further statements |\providecommand{\version}{...}|
can thus be added before the above code to override it.

For the main file, one might add a line
(between |\childdocmain| and the above block)
%
\begin{center}
|%\ifchilddoc\||else\providecommand{\version}{draft}\||fi|
\end{center}
%
which can be uncommented to produce a draft version.
Likewise one can add a line to the very top of a child file
(above the |\childdocof{|\textit{main}|}| directive)
%
\begin{center}
|%\providecommand{\version}{final}|
\end{center}
%
which can be uncommented to produce the final version of this child document.

%%%%%%%%%%%%%%%%%%%%%%%%%%%%%%%%%%%%%%%%%%%%%%%%%%%%%%%%%%%%%%%%%%%%%%%%%%%%%%%%
\subsection{Forwarding}
\label{sec:forward}

Different versions of the main or child documents
using compilation flags as described in \secref{sec:flags}
can be (permanently) stored in different files
for convenient compilation, viewing and distribution.
To this end, the package defines a command
to pass on compilation to a different file:

%%%%%%%%%%%%%%%%%%%%%%%%%%%%%%%%%%%%%%%%
\DescribeMacro{\childdocforward}
The command |\childdocforward| redirects processing to
another source file:
%
\begin{center}
\begin{tabular}{l}
|\input{childdoc.def}|\\
|\childdocforward[|\textit{main}|]{|\textit{dest}|}|\\
\end{tabular}
\end{center}
%
The argument \textit{dest} is the destination file
(without extension).
It should be the main file or one of the child files.
Note that further \textsf{childdoc} directives
such as |\childdocof| and |\childdocforward|
in the indicated file will be processed in this form.
The optional argument \textit{main}
passes on directly to the main file \textit{main}
while pretending to compile the child \textit{dest}.
This form behaves as if \textit{dest}
issues |\childdocof{|\textit{main}|}| right away,
and no further \textsf{childdoc} directives will be processed.

%%%%%%%%%%%%%%%%%%%%%%%%%%%%%%%%%%%%%%%%
\DescribeMacro{\...prefix}
In the alternative form |\childdocforwardprefix|,
%
\begin{center}
\begin{tabular}{l}
|\input{childdoc.def}|\\
|\childdocforwardprefix[|\textit{main}|]{|\textit{prefix}|}{|\textit{dest}|}|
\end{tabular}
\end{center}
%
the destination file is determined by a pattern
depending on the current file:
To make this work, the current file must be called
`{\textit{prefix}\hspace{0.2em}\textit{suffix}}'
with \textit{prefix} matching precisely the argument.
Processing is then passed on to the file
`{\textit{dest}\hspace{0.2em}\textit{suffix}}'.
Surely, the same effect is achieved by
directly specifying the
argument `{\textit{dest}\hspace{0.2em}\textit{suffix}}'
in the first form.
However, that requires to set up a different file
for each child. With the alternative form of the command
all these files can have exactly the same content
which simplifies setting them up and maintaining them.

For example, the following file |draft.tex|
with a compilation flag |\version| as described in \secref{sec:flags}
compiles the main document as a draft:
%
\begin{center}
\begin{tabular}{l}
|\def\version{draft}|\\
|\input{childdoc.def}|\\
|\childdocforward{|\textit{main}|}|
\end{tabular}
\end{center}
%
Likewise, the following files |final|\textit{nn}|.tex|
compile the final version of the child document
|child|\textit{nn}|.tex|:
%
\begin{center}
\begin{tabular}{l}
|\def\version{final}|\\
|\input{childdoc.def}|\\
|\childdocforwardprefix{final}{child}|
\end{tabular}
\end{center}
%

Note that when several versions of a main file and/or of each child file
are to be generated, it may be convenient to set up a |Makefile| or
shell script to automatise the process.

%%%%%%%%%%%%%%%%%%%%%%%%%%%%%%%%%%%%%%%%%%%%%%%%%%%%%%%%%%%%%%%%%%%%%%%%%%%%%%%%
\subsection{Command Line Processing}
\label{sec:commandline}

The effect of redirection files can also be achieved by invoking
the \LaTeX{} compiler with a more elaborate command line.
Most conveniently this should be done as part
of a shell script or a |Makefile|.

When using \textsf{childdoc} in the main file, the following
command lines effectively perform a redirection
(note that depending on the shell being used,
backslashes may have to be doubled: `|\|' $\to$ `|\\|'):
%
\begin{center}
|... -jobname "|\textit{target}|" |\\|"|[\textit{flags}]%
|\input{childdoc.def}\childdocforward[|\textit{main}|]{|\textit{dest}|}"|
\end{center}
%
Here \textit{target} is the name of the output file,
\textit{main} is the name of the main file
and \textit{dest} is the name of the main or child file to be processed
(all filenames without extensions).
The optional argument \textit{main} can be omitted
if \textit{main} matches \textit{dest}.
Optionally, compilation \textit{flags} can be defined via |\def| commands.
This command line makes the \TeX{} engine believe
it is compiling the file \textit{target}
whose content is specified as the latter parameter.
The provided code then forwards the processing to
\textit{main} or \textit{dest} as described in \secref{sec:forward}.

%%%%%%%%%%%%%%%%%%%%%%%%%%%%%%%%%%%%%%%%%%%%%%%%%%%%%%%%%%%%%%%%%%%%%%%%%%%%%%%%
\subsection{Include by Input}
\label{sec:input}

Including child documents by |\include| has some restrictions by design.
Most notably, the content of a child document always occupies
its own set of pages; pages cannot be shared between child documents.
Usually, this behaviour makes perfect sense
because each child document contain an essential part of the document.
However, in some situations it may be desirable to compose
a document from a collection of parts
without having mandatory page breaks between then.
For this case, the package
provides a mechanism to include parts
by |\input| which can also be processed individually.
However, by construction this mechanism
requires manual handling of the content to be output.

%%%%%%%%%%%%%%%%%%%%%%%%%%%%%%%%%%%%%%%%
\DescribeMacro{\ifchilddocmanual}
The main file should be prepared as usual, see \secref{sec:include}.
However, the document body must make a distinction
between processing of an individual part and of the main document, e.g.:
%
\begin{center}
\begin{tabular}{l}
|\ifchilddocmanual|\\
|\input{\childdocname}|\\
|\||else|\\
\textit{document body with }|\input{|\textit{part}|}|\\
|\||fi|
\end{tabular}
\end{center}
%
The conditional |\ifchilddocmanual| is true whenever
a part to be included by |\input| is being compiled,
and the name of the part is stored in |\childdocname|.

%%%%%%%%%%%%%%%%%%%%%%%%%%%%%%%%%%%%%%%%
\DescribeMacro{\childdocby}
Each part to be included by |\input| should start with:
%
\begin{center}
\begin{tabular}{l}
|\input{childdoc.def}|\\
|\childdocby{|\textit{main}|}|\\
\end{tabular}
\end{center}
%
The directive |\childdocby| is similar to |\childdocof|
described in \secref{sec:include},
but the subsequent selection of content must be done manually.
To that end, both |\ifchilddoc| and |\ifchilddocmanual|
will be true upon processing of a part,
and the name of the part is stored in |\childdocname|.
Note that |\jobname| will be set to the filename of the current part
so that each part receives an individual |.aux| file
that does not interfere with the |.aux| file(s) of the main document.
This behaviour can be altered by the alternative form
|\childdocby[*]{|\textit{main}|}| (with a non-empty optional argument)
which uses the |.aux| file of the main document
by setting |\jobname| to \textit{main}.

%%%%%%%%%%%%%%%%%%%%%%%%%%%%%%%%%%%%%%%%%%%%%%%%%%%%%%%%%%%%%%%%%%%%%%%%%%%%%%%%
\subsection{Driver Development}
\label{sec:driver}

The \textsf{childdoc} mechanism can also be use for the development
of definition files such as \LaTeX{} styles or classes.
This case differs from the above setup with multiple parts
included by |\include| in that no |\includeonly| should be invoked.
This can be achieved by starting the include file
(before |\ProvidesPackage|) with:
%
\begin{center}
\begin{tabular}{l}
|\input{childdoc.def}|\\
|\childdocforward{|\textit{main}|}|\\
\end{tabular}
\end{center}
%
or alternatively with:
%
\begin{center}
\begin{tabular}{l}
|\input{childdoc.def}|\\
|\childdocby{|\textit{main}|}|\\
\end{tabular}
\end{center}
%
Both forms have slightly different effects as described above.
The main file is prepared as usual, see \secref{sec:include}.

%%%%%%%%%%%%%%%%%%%%%%%%%%%%%%%%%%%%%%%%%%%%%%%%%%%%%%%%%%%%%%%%%%%%%%%%%%%%%%%%
\subsection{Legacy Detection}
\label{sec:detection}

The directive |\childdocmain| in the main file can detect
whether the complete document or merely a child is to be compiled
even without using the directive |\childdocof|.
This method is deprecated because it is less robust
and there is no compelling reason to use it;
it is merely provided for backward compatibility
and it may be removed in future versions.

If the detection mechanism is to be used,
it is mandatory to correctly specify
the filename of the main file as the argument of |\childdocmain|:
%
\begin{center}
\begin{tabular}{l}
|\input{childdoc.def}|\\
|\childdocmain{|\textit{main}|}|\\
\end{tabular}
\end{center}
%
If |\jobname| does not match the argument \textit{main} of |\childdocmain|,
it is assumed that |\jobname| points to the child file to be compiled.
When using |\childdocmain| with the main file specified as argument,
it suffices to start a child file
with just |\input{|\textit{main}|}|
without loading of the package and using |\childdocof|.
If instead all processing is done
with the appropriate \textsf{childdoc} directives,
the argument of \textit{main} of |\childdocmain| can be empty.

An alternative version of the command line processing described
in \secref{sec:commandline} using the detection mechanism reads:
%
\begin{center}
|... -jobname "|\textit{target}|" "|[\textit{flags}]%
[|\def\jobname{|\textit{dest}|}|]|\input{|\textit{main}|}"|
\end{center}

%%%%%%%%%%%%%%%%%%%%%%%%%%%%%%%%%%%%%%%%%%%%%%%%%%%%%%%%%%%%%%%%%%%%%%%%%%%%%%%%
\subsection{Manual Code}
\label{sec:manual}

In case one cannot be certain whether the definitions file |childdoc.def|
is installed on the target \TeX{} distribution
and one prefers not to ship it,
it is conceivable to paste a few relevant commands into the sources.

To that end, drop all statements |\input{childdoc.def}|
and perform the replacements as outlined below.
Instead of |\childdocmain{|\textit{main}|}| add the following code
to the top of the main file:
%
\begin{center}
\begin{tabular}{l}
|\||ifdefined\childdocname\endinput\||fi\newif\ifchilddoc|\\
|\edef\childdocname{\scantokens\expandafter{\jobname\noexpand}}|\\
|\def\childdocmain{|\textit{main}|}\||ifx\childdocmain\childdocname\||else|\\
|\childdoctrue\includeonly{\childdocname}\let\jobname\childdocmain\||fi|\\
\end{tabular}
\end{center}
%
Instead of |\childdocof{|\textit{main}|}| just include the main file
at the top of each child file:
%
\begin{center}
|\input{|\textit{main}|}|
\end{center}
%
A simple redirection |\childdocforward{|\textit{dest}|}| is achieved by:
%
\begin{center}
|\def\jobname{|\textit{dest}|}\input{\jobname}|
\end{center}
%
The redirection with prefix
|\childdocforwardprefix[|\textit{prefix}|]{|\textit{dest}|}|
is accomplished by:
%
\begin{center}
\begin{tabular}{l}
|{\edef\jobname{\scantokens\expandafter{\jobname\noexpand}}|\\
|\def\redirectjob |\textit{prefix}|#1~~~{\gdef\jobname{|\textit{dest}|#1}}|\\
|\expandafter\redirectjob\jobname~~~}\input{\jobname}|
\end{tabular}
\end{center}

In an alternative approach,
child documents can be compiled by a specific command line
without additional code or specific definitions:
%
\begin{center}
|... -jobname "|\textit{target}|" "|[\textit{flags}]%
|\includeonly{|\textit{dest}|}\input{|\textit{main}|}"|
\end{center}
%

%%%%%%%%%%%%%%%%%%%%%%%%%%%%%%%%%%%%%%%%%%%%%%%%%%%%%%%%%%%%%%%%%%%%%%%%%%%%%%%%
%%%%%%%%%%%%%%%%%%%%%%%%%%%%%%%%%%%%%%%%%%%%%%%%%%%%%%%%%%%%%%%%%%%%%%%%%%%%%%%%
\section{Information}

%%%%%%%%%%%%%%%%%%%%%%%%%%%%%%%%%%%%%%%%%%%%%%%%%%%%%%%%%%%%%%%%%%%%%%%%%%%%%%%%
\subsection{Copyright}

Copyright \copyright{} 2017--2018 Niklas Beisert

This work may be distributed and/or modified under the
conditions of the \LaTeX{} Project Public License, either version 1.3
of this license or (at your option) any later version.
The latest version of this license is in
  \url{http://www.latex-project.org/lppl.txt}
and version 1.3 or later is part of all distributions of \LaTeX{}
version 2005/12/01 or later.

This work has the LPPL maintenance status `maintained'.

The Current Maintainer of this work is Niklas Beisert.

This work consists of the files |README.txt|, |childdoc.ins| and |childdoc.dtx|
as well as the derived files |childdoc.def|, |cdocsamp.tex|
with |cdocsch1.tex|, |cdocsch2.tex|, |cdocspt3.tex|, |cdocspt4.tex|,
|cdocsdrf.tex|, |cdocsfn1.tex|, |cdocsfn2.tex|
as well as |childdoc.pdf|.

%%%%%%%%%%%%%%%%%%%%%%%%%%%%%%%%%%%%%%%%%%%%%%%%%%%%%%%%%%%%%%%%%%%%%%%%%%%%%%%%
\subsection{Files and Installation}

The package consists of the files:
%
\begin{center}
\begin{tabular}{ll}
    |README.txt|   & readme file \\
    |childdoc.ins| & installation file \\
    |childdoc.dtx| & source file \\
    |childdoc.def| & definition file \\
    |cdocsamp.tex| & sample main file \\
    |cdocsch1.tex| & sample include file \\
    |cdocsch2.tex| & sample include file \\
    |cdocspt3.tex| & sample part file \\
    |cdocspt4.tex| & sample part file \\
    |cdocsdrf.tex| & sample redirection file \\
    |cdocsfn1.tex| & sample redirection file \\
    |cdocsfn2.tex| & sample redirection file \\
    |childdoc.pdf| & manual
\end{tabular}
\end{center}
%
The distribution consists of the files
|README.txt|, |childdoc.ins| and |childdoc.dtx|.
%
\begin{itemize}
\item
Run (pdf)\LaTeX{} on |childdoc.dtx|
to compile the manual |childdoc.pdf| (this file).
\item
Run \LaTeX{} on |childdoc.ins| to create the definitions file |childdoc.def|
and the sample |cdocsamp.tex| with include files
|cdocsch1.tex|, |cdocsch2.tex|, |cdocspt3.tex|, |cdocspt4.tex|,
|cdocsdrf.tex|, |cdocsfn1.tex|, |cdocsfn2.tex|.
Then copy the file |childdoc.def| to an appropriate directory of your \LaTeX{}
distribution, e.g.\ \textit{texmf-root}|/tex/latex/childdoc|.
\end{itemize}

%%%%%%%%%%%%%%%%%%%%%%%%%%%%%%%%%%%%%%%%%%%%%%%%%%%%%%%%%%%%%%%%%%%%%%%%%%%%%%%%
\subsection{Related CTAN Packages}

There are several other packages which offer a similar functionality:
%
\begin{itemize}
\item
The packages
\href{http://ctan.org/pkg/docmute}{\textsf{docmute}},
\href{http://ctan.org/pkg/includex}{\textsf{includex}} and
\href{http://ctan.org/pkg/standalone}{\textsf{standalone}}
provide commands to include only the document body of
a child file thus allowing both files to be compiled individually.
\item
The packages \href{http://ctan.org/pkg/subdocs}{\textsf{subdocs}}
and \href{http://ctan.org/pkg/subfiles}{\textsf{subfiles}}
provide structures in which the main and child documents can be
encapsulated and allowing them to be compiled individually.
The inclusion mechanism is different from the conventional |\include|.
\item
The package \href{http://ctan.org/pkg/combine}{\textsf{combine}}
is an elaborate solution to combine several documents into one.
\end{itemize}
%
See also the CTAN topic \href{http://ctan.org/topic/subdocs}{\textsf{subdocs}}
for further related packages.
The present package differs from the above solutions in that
a document structure constructed with the conventional |\include| mechanism
just needs two extra commands at the top of every file
such that all constituent files can be compiled individually.

%%%%%%%%%%%%%%%%%%%%%%%%%%%%%%%%%%%%%%%%%%%%%%%%%%%%%%%%%%%%%%%%%%%%%%%%%%%%%%%%
%\subsection{Feature Suggestions}
%
%The following is a list of features which may be useful for future
%versions of this package:
%%
%\begin{itemize}
%\item
%\ldots
%\end{itemize}

%%%%%%%%%%%%%%%%%%%%%%%%%%%%%%%%%%%%%%%%%%%%%%%%%%%%%%%%%%%%%%%%%%%%%%%%%%%%%%%%
\subsection{Revision History}

%%%%%%%%%%%%%%%%%%%%%%%%%%%%%%%%%%%%%%%%
\paragraph{v2.0:} 2018/12/30

\begin{itemize}
\item
immediate forward processing
\item
added |\childdocby| mechanism
\item
manual restructured
\end{itemize}

%%%%%%%%%%%%%%%%%%%%%%%%%%%%%%%%%%%%%%%%
\paragraph{v1.6:} 2018/01/17

\begin{itemize}
\item
application for development of include files
\item
corrections to manual
\end{itemize}

%%%%%%%%%%%%%%%%%%%%%%%%%%%%%%%%%%%%%%%%
\paragraph{v1.5:} 2017/05/21

\begin{itemize}
\item
more complete structuring introduced
\item
|\childdocof| introduced
\item
|\childdoc| renamed to |\childdocmain|
\item
|\childredirect| renamed to |\childdocforward| and |\childdocforwardprefix|
and functionality expanded
\end{itemize}

%%%%%%%%%%%%%%%%%%%%%%%%%%%%%%%%%%%%%%%%
\paragraph{v1.0:} 2017/04/27

\begin{itemize}
\item
manual and install package
\item
first version published on CTAN
\end{itemize}

%%%%%%%%%%%%%%%%%%%%%%%%%%%%%%%%%%%%%%%%
\paragraph{v0.6:} 2017/04/26

\begin{itemize}
\item
redirection mechanism added
\end{itemize}

%%%%%%%%%%%%%%%%%%%%%%%%%%%%%%%%%%%%%%%%
\paragraph{v0.5:} 2017/04/26

\begin{itemize}
\item
functionality in definition file
\end{itemize}


%%%%%%%%%%%%%%%%%%%%%%%%%%%%%%%%%%%%%%%%%%%%%%%%%%%%%%%%%%%%%%%%%%%%%%%%%%%%%%%%
%%%%%%%%%%%%%%%%%%%%%%%%%%%%%%%%%%%%%%%%%%%%%%%%%%%%%%%%%%%%%%%%%%%%%%%%%%%%%%%%
%%%%%%%%%%%%%%%%%%%%%%%%%%%%%%%%%%%%%%%%%%%%%%%%%%%%%%%%%%%%%%%%%%%%%%%%%%%%%%%%
\appendix

\settowidth\MacroIndent{\rmfamily\scriptsize 000\ }

 \DocInput{childdoc.dtx}

\end{document}
%</driver>
% \fi
%
% %%%%%%%%%%%%%%%%%%%%%%%%%%%%%%%%%%%%%%%%%%%%%%%%%%%%%%%%%%%%%%%%%%%%%%%%%%%%%%
% %%%%%%%%%%%%%%%%%%%%%%%%%%%%%%%%%%%%%%%%%%%%%%%%%%%%%%%%%%%%%%%%%%%%%%%%%%%%%%
% \section{Sample}
%\iffalse
%<*samplemain>
%\fi
%
% The following presents a sample document
% with two chapters, two parts, a title page,
% a compile flag as well as three forwarding files to set the flag.
% It consists of eight |.tex| files:
% \begin{center}
% \begin{tabular}{ll}
% |cdocsamp.tex|&main file\\
% |cdocsch1.tex|&include file for chapter 1\\
% |cdocsch2.tex|&include file for chapter 2\\
% |cdocspt3.tex|&include file for part 3\\
% |cdocspt4.tex|&include file for part 4\\
% |cdocsdrf.tex|&forwarding file for main file in draft mode\\
% |cdocsfi1.tex|&forwarding file for final version of chapter 1\\
% |cdocsfi2.tex|&forwarding file for final version of chapter 2\\
% \end{tabular}
% \end{center}
% Each of the eight files can be compiled directly by the \LaTeX{} compiler.
%
% %%%%%%%%%%%%%%%%%%%%%%%%%%%%%%%%%%%%%%
% \paragraph{Main File.}
%
% The main file is called |cdocsamp.tex|.
%
% Load the \textsf{childdoc} definitions and
% declare the filename for the main document:
%    \begin{macrocode}
\input{childdoc.def}
\childdocmain{}
%    \end{macrocode}

% Optional override for |\version| flag:
%    \begin{macrocode}
%%\ifchilddoc\else\providecommand{\version}{draft}\fi
%    \end{macrocode}

% Define the default values for the |\version| flag
% (|final| for the main file and |draft| for childs):
%    \begin{macrocode}
\ifchilddoc
\providecommand{\version}{draft}
\else
\providecommand{\version}{final}
\fi
%    \end{macrocode}

% Load the standard document class:
%    \begin{macrocode}
\documentclass[12pt]{article}
%    \end{macrocode}

% Start the document body:
%    \begin{macrocode}
\begin{document}
%    \end{macrocode}

% Declare a title page.
% Print title, part of document being processed and version flag:
%    \begin{macrocode}
\addtocounter{page}{-1}
\begin{center}
{\LARGE\bfseries{}childdoc example\par}
\vspace{1cm}
\ifchilddoc
\ifchilddocmanual part\else chapter\fi:
`\childdocname' of `\childdocjob'\par
\else
main document: `\childdocjob'\par
\fi
version: \version\par
\end{center}
\newpage
%    \end{macrocode}

% Manually include selected file,
% otherwise process as usual:
%    \begin{macrocode}
\ifchilddocmanual
\section*{part `\childdocname'}
\input{\childdocname}
\else
%    \end{macrocode}

% Include the two chapters:
%    \begin{macrocode}
\include{cdocsch1}
\include{cdocsch2}
%    \end{macrocode}

% Include the two parts unless only chapters should be displayed:
%    \begin{macrocode}
\ifchilddoc\else
\section{part three}
\input{cdocspt3}
\section{part four}
\input{cdocspt4}
\fi
%    \end{macrocode}

% Process as usual until here:
%    \begin{macrocode}
\fi
%    \end{macrocode}

% End of document body:
%    \begin{macrocode}
\end{document}
%    \end{macrocode}
%\iffalse
%</samplemain>
%\fi
%
% %%%%%%%%%%%%%%%%%%%%%%%%%%%%%%%%%%%%%%
% \paragraph{Chapter Include Files.}
%
% The include files are called |cdocsch1.tex| and |cdocsch2.tex|.
%
%\iffalse
%<*samplechap1|samplechap2>
%\fi

% Optional override for |\version| flag:
%    \begin{macrocode}
%%\providecommand{\version}{final}
%    \end{macrocode}

% Include the main document:
%    \begin{macrocode}
\input{childdoc.def}
\childdocof{cdocsamp}
%    \end{macrocode}

%\iffalse
%</samplechap1|samplechap2>
%\fi
%
%\iffalse
%<*samplechap1>
%\fi
% Some text for chapter 1:
%    \begin{macrocode}
\section{one}
some text in chapter one
%    \end{macrocode}

%\iffalse
%</samplechap1>
%\fi
% Some text for chapter 2:
%\iffalse
%<*samplechap2>
%\fi
%    \begin{macrocode}
\section{two}
more text in chapter two
%    \end{macrocode}

%\iffalse
%</samplechap2>
%\fi
%
% %%%%%%%%%%%%%%%%%%%%%%%%%%%%%%%%%%%%%%
% \paragraph{Part Include Files.}
%
% The include files are called |cdocspt3.tex| and |cdocspt4.tex|.
%
%\iffalse
%<*samplepart3|samplepart4>
%\fi

% Optional override for |\version| flag:
%    \begin{macrocode}
%%\providecommand{\version}{final}
%    \end{macrocode}

% Include the main document:
%    \begin{macrocode}
\input{childdoc.def}
\childdocby{cdocsamp}
%    \end{macrocode}

%\iffalse
%</samplepart3|samplepart4>
%\fi
%
%\iffalse
%<*samplepart3>
%\fi
% Some text for part 3:
%    \begin{macrocode}
some text in part three
%    \end{macrocode}

%\iffalse
%</samplepart3>
%\fi
% Some text for part 4:
%\iffalse
%<*samplepart4>
%\fi
%    \begin{macrocode}
more text in part four
%    \end{macrocode}

%\iffalse
%</samplepart4>
%\fi
%
% %%%%%%%%%%%%%%%%%%%%%%%%%%%%%%%%%%%%%%
% \paragraph{Forwarding for a Complete Draft.}
%
% The following forwarding file |cdocsdrf.tex|
% compiles the main document in draft mode:
%\iffalse
%<*sampledraft>
%\fi
%    \begin{macrocode}
\def\version{draft}
\input{childdoc.def}
\childdocforward{cdocsamp}
%    \end{macrocode}

%\iffalse
%</sampledraft>
%\fi
%
% %%%%%%%%%%%%%%%%%%%%%%%%%%%%%%%%%%%%%%
% \paragraph{Forwarding for Final Version of the Chapters.}
%
% The following forwarding files |cdocsfn1.tex| and |cdocsfn2.tex|
% (with identical content)
% compile the final versions of the child documents
% |cdocsch1.tex| and |cdocsch2.tex|, respectively:
%\iffalse
%<*samplefinal>
%\fi
%    \begin{macrocode}
\def\version{final}
\input{childdoc.def}
\childdocforwardprefix[cdocsamp]{cdocsfn}{cdocsch}
%    \end{macrocode}

%\iffalse
%</samplefinal>
%\fi
%
% %%%%%%%%%%%%%%%%%%%%%%%%%%%%%%%%%%%%%%
% \paragraph{Command Line Processing.}
%
% The following three command lines generate the output files
% |cdocscld|, |cdocscl1| and |cdocscl2|
% which should be identical to
% |cdocsdrf|, |cdocsch1| and |cdocsfn2|, respectively:
% \begin{center}
% \begin{tabular}{l}
% |latex -jobname cdocscld \|\\
% |  "\def\version{draft}\input{childdoc.def}\childdocforward{cdocsamp}"|\\
% |latex -jobname cdocscl1 \|\\
% |  "\input{childdoc.def}\childdocforward[cdocsamp]{cdocsch1}"|\\
% |latex -jobname cdocscl2 \|\\
% |  "\def\version{final}\input{childdoc.def}\childdocforward{cdocsch2}"|
% \end{tabular}
% \end{center}
% Note that the trailing backslash on each first line
% merely continues the input to the second line
% (for convenient cut ant paste).
% Furthermore, the command |latex| can be replaced by any
% of its alternative versions such as |pdflatex|.
%
% %%%%%%%%%%%%%%%%%%%%%%%%%%%%%%%%%%%%%%%%%%%%%%%%%%%%%%%%%%%%%%%%%%%%%%%%%%%%%%
% %%%%%%%%%%%%%%%%%%%%%%%%%%%%%%%%%%%%%%%%%%%%%%%%%%%%%%%%%%%%%%%%%%%%%%%%%%%%%%
% \section{Implementation}
%\iffalse
%<*package>
%\fi
%
% This section describes the definitions file |childdoc.def|.

% The definitions cannot be loaded using |\usepackage| or |\RequirePackage|
% which has a mechanism to prevent loading a style file more than once.
% When loading the definitions by means of |\input|
% multiple instances have to be prevented manually:
%\iffalse
%This code needs to be before the `\ProvidesFile' directive
%which is defined at the beginning of this file.
%Therefore it is also placed there and commented out here.
%</package>
%<*discard>
%\fi
%    \begin{macrocode}
\ifdefined\childdocmain\endinput\fi
%    \end{macrocode}
%\iffalse
%</discard>
%<*package>
%\fi
%
% \macro{\ifchilddoc}
% \macro{\ifchilddocmanual}
% The conditional |\ifchilddoc| tells whether a
% child (true) or main (false) document is being compiled.
% The conditional |\ifchilddocmanual| tells whether
% the |\includeonly| mechanism is used (false) or
% the selection of child files must be performed manually (true).
% The definitions initialise to false:
%    \begin{macrocode}
\newif\ifchilddoc
\newif\ifchilddocmanual
%    \end{macrocode}

% \macro{\childdocname}
% \macro{\childdocjob}
% The macro |\childdocname| stores the name of the main document
% to be compiled. The macro |\childdocjob| stores the name of
% the document on which the \LaTeX{} compiler was originally invoked.
% The content of |\jobname| cannot be compared
% to filenames specified in the source due to different catcodes.
% The following code rescans |\jobname|, stores the result
% in |\childdocname| and saves a copy in |\childdocjob|:
%    \begin{macrocode}
\edef\childdocname{\scantokens\expandafter{\jobname\noexpand}}
\let\childdocjob\childdocname
%    \end{macrocode}

% \macro{\childdocdisable}
% The macro |\childdocdisable| prevents the main file
% from being processed more than once.
% At this stage, the main document command |\childdocmain|
% is assumed to be called once again where it should do nothing.
% Any subsequent call to it should prevent
% a secondary processing of the main document
% It overwrites the forwarding commands
% |\childdocof| and |\childdocforward|
% with empty macros to prevent further inclusions of the main document:
%    \begin{macrocode}
\newcommand{\childdocdisable}
{
  \renewcommand{\childdocmain}[1]{\renewcommand{\childdocmain}[1]{\endinput}}
  \renewcommand{\childdocof}[1]{}
  \renewcommand{\childdocby}[2][]{}
  \renewcommand{\childdocforward}[2][]{}
  \renewcommand{\childdocdisable}{}
}
%    \end{macrocode}

% \macro{\childdocmain}
% The macro |\childdocmain| is to be called at the top of the main file
% with nothing or the main filename (without extension) as argument.
% First, it breaks loops.
% If the argument is not empty and does not match |\childdocname|
% (which is set by the first inclusion of |childdoc.def|),
% |\ifchilddoc| is set to true, |\includeonly| is applied to the child file
% and |\jobname| is set to the main file
% (for proper handling of |.aux| files):
%    \begin{macrocode}
\newcommand{\childdocmain}[1]
{
  \childdocdisable\childdocmain{}
  \if?#1?\else
    \begingroup
      \def\childdoctmp{#1}
      \ifx\childdoctmp\childdocname
        \def\childdoctmp{}
      \else
        \def\childdoctmp
        {
          \childdoctrue
          \includeonly{\childdocname}
          \def\childdocjob{#1}
          \def\jobname{#1}
        }
      \fi
      \expandafter
    \endgroup
    \childdoctmp
  \fi
}
%    \end{macrocode}

% \macro{\childdocof}
% The command |\childdocof| redirects
% compilation to the main file |#1|.
%    \begin{macrocode}
\newcommand{\childdocof}[1]
{
  \childdocdisable
  \childdoctrue
  \includeonly{\childdocname}
  \def\jobname{#1}
  \def\childdocjob{#1}
  \input{#1}
}
%    \end{macrocode}

% \macro{\childdocby}
% The command |\childdocby| ....
%    \begin{macrocode}
\newcommand{\childdocby}[2][]
{
  \childdocdisable
  \childdoctrue
  \childdocmanualtrue
  \if?#1?\else
    \def\jobname{#2}
  \fi
  \def\childdocjob{#2}
  \input{#2}
  \endinput
}
%    \end{macrocode}

% \macro{\childdocforward}
% The command |\childdocforward| redirects
% compilation to the main file or
% (if the optional argument is given) a child file.
% Parameters are set as if the main file
% or a child file starting with |\childdocof| was compiled.
% Then compilation is handed over to the main file:
%    \begin{macrocode}
\newcommand{\childdocforward}[2][]
{
  \begingroup
    \if?#1?
      \def\childdoctmp
      {
        \def\childdocname{#2}
        \def\childdocjob{#2}
        \def\jobname{#2}
        \input{#2}
        \endinput
      }
    \else
      \def\childdoctmp
      {
        \childdocdisable
        \def\childdocname{#2}
        \childdoctrue
        \includeonly{#2}
        \def\childdocjob{#1}
        \def\jobname{#1}
        \input{#1}
        \endinput
      }
    \fi
    \expandafter
  \endgroup
  \childdoctmp
}
%    \end{macrocode}

% \macro{\childdocforwardprefix}
% The command |\childdocforwardprefix| redirects
% compilation to the main or a child file by means of a pattern.
% The prefix |#1| in the current filename is replaced by |#2|
% and the suffix of the current filename is kept
% (it is assumed that the filename does not contain the substring `|~~~|'
% which is used as a delimiter).
% Compilation is handed over to the new file by |\childdocforward|:
%    \begin{macrocode}
\newcommand{\childdocforwardprefix}[3][]
{
  \begingroup
    \def\childdocextract #2##1~~~{\def\childdoctmp{\childdocforward[#1]{#3##1}}}
    \expandafter\childdocextract\childdocname~~~
    \expandafter
  \endgroup
  \childdoctmp
}
%    \end{macrocode}

% \macro{\childdoc}
% The deprecated macro |\childdoc| is a legacy version of |\childdocmain|:
%    \begin{macrocode}
\newcommand{\childdoc}{\childdocmain}
%    \end{macrocode}

% \macro{\childdocredirect}
% The deprecated macro |\childdocredirect| is a legacy version
% of |\childdocforward| and |\childdocforwardprefix|:
%    \begin{macrocode}
\newcommand{\childdocredirect}[2][]
{
  \begingroup
    \if?#1?
      \def\childdoctmp{\childdocforward{#2}}
    \else
      \def\childdoctmp{\childdocforwardprefix{#1}{#2}}
    \fi
    \expandafter
  \endgroup
  \childdoctmp
}
%    \end{macrocode}

%\iffalse
%</package>
%\fi
%
\endinput
|\\
|\childdocof{|\textit{main}|}|\\
\end{tabular}
\end{center}
at the top of every child file \textit{child}
which is included by |\include{|\textit{child}|}|
from within the main file
(or at least for those files to be compiled individually).
The argument \textit{main} must be the filename of the main file.

There are a couple of
considerations in setting up the main and child documents:

%%%%%%%%%%%%%%%%%%%%%%%%%%%%%%%%%%%%%%%%
\paragraph{Restrictions.}

Please note the following restrictions:
\begin{itemize}
\item
|\childdocmain| must be called with one argument \textit{main}
to ensure compatibility with earlier version of the package.
It must either be empty (|\childdocmain{}|)
or precisely match the filename of the main file in which it is specified.
See \secref{sec:detection} for further information.
\item
The filename \textit{main} must be specified without the |.tex| extension.
\item
The filename \textit{main} is case sensitive
(even in case-insensitive file systems)
due to internal string comparison.
\item
The argument \textit{main} should be fully expanded, it cannot be a macro.
\item
Subdirectories and special characters should be avoided in filenames.
\item
The command |\childdocmain{|\textit{main}|}| must be followed by a whitespace.
It should not be followed immediately by another command
or by a comment mark `|%|'.
This is because the \TeX{} parser reads the token immediately following
the argument of |\childdocmain| and puts it
at the beginning of every child section;
however, a white\-space is ignored.
\end{itemize}

%%%%%%%%%%%%%%%%%%%%%%%%%%%%%%%%%%%%%%%%
\paragraph{Content of Main File.}

It is advisable to place all content in the child files included by |\include|.
Any output contained in the main file will appear in all child documents
unless suppressed manually;
it cannot be suppressed automatically by the |\includeonly| directive
and thus should normally be avoided.
A method to include some content in the main file
by means of conditional processing is described in \secref{sec:conditional}.

%%%%%%%%%%%%%%%%%%%%%%%%%%%%%%%%%%%%%%%%
\paragraph{Page Numbering.}

When only a part of the document is compiled,
the appropriate numbering of pages
(as well as other status parameters)
is determined from the |.aux| files.
The latter contain information from previous passes.
However this information needs to propagate through
all intermediate child documents.
Therefore the page numbering in child documents may well
be inconsistent until the complete document is compiled at least once.

A useful (if unconventional) way to always ensure a consistent
page numbering is to restart the numbering in each child document
and denote the pages by `\textit{child}|.|\textit{page}'
where \textit{child} represents the chapter/section number of the child file.
This can be achieved by the command
|\numberwithin{page}{|\textit{child}|}|
of the \textsf{amsmath} package
where \textit{child} can be |chapter| or |section|
depending on the chosen structuring.
Alternatively, one can modify the macro |\thepage| appropriately
and reset the counter |page| at the start of each child file.

%%%%%%%%%%%%%%%%%%%%%%%%%%%%%%%%%%%%%%%%%%%%%%%%%%%%%%%%%%%%%%%%%%%%%%%%%%%%%%%%
\subsection{Conditional Processing}
\label{sec:conditional}

The package provides a mechanism to compile different versions
of a document. To customise the versions further some conditional processing
can come in handy to distinguish which version is being compiled.
The package provides two macros to describe the compilation context:

%%%%%%%%%%%%%%%%%%%%%%%%%%%%%%%%%%%%%%%%
\DescribeMacro{\ifchilddoc}
The conditional |\ifchilddoc| distinguishes between the compilation of
child documents and the main document:
%
\begin{center}
|\ifchilddoc |\textit{child-code}| |[|\||else |\textit{main-code}]| \||fi|
\end{center}

%%%%%%%%%%%%%%%%%%%%%%%%%%%%%%%%%%%%%%%%
\DescribeMacro{\childdocname}
\DescribeMacro{\childdocjob}
The macro |\childdocname| contains the filename (without extension)
of the main or child file being processed.
Note that |\childdocjob| will always contain the name of the main file.

%%%%%%%%%%%%%%%%%%%%%%%%%%%%%%%%%%%%%%%%
\paragraph{Title Page.}

Conditional processing can be used to include a title or banner page
in the main document when proper precautions are taken.
Importantly, the code in the main file should ensure that the page counter
(as well as other status parameters which are stored in the |.aux| files)
takes the same value after the conditional processing.
Otherwise the page numbers may take divergent values
depending on which part is compiled.

For example, a title page could be declared by:
%
\begin{center}
\begin{tabular}{l}
|\ifchilddoc\||else|\\
|\addtocounter{page}{-1}|\\
\textit{code for title page}\\
|\newpage|\\
|\||fi|
\end{tabular}
\end{center}
%
A banner page for the child documents can be generated by:
%
\begin{center}
\begin{tabular}{l}
|\ifchilddoc|\\
|\addtocounter{page}{-1}|\\
\textit{code for banner page}\\
|\newpage|\\
|\||fi|
\end{tabular}
\end{center}
%
Here one could write a message such as:
\begin{center}
|This is the part \childdocname{} of \childdocjob{}.|
\end{center}

%%%%%%%%%%%%%%%%%%%%%%%%%%%%%%%%%%%%%%%%%%%%%%%%%%%%%%%%%%%%%%%%%%%%%%%%%%%%%%%%
\subsection{Flags}
\label{sec:flags}

The package makes it easy to generate different versions
of the main or child documents.
To this end compilation flags can be defined
and assigned different default values.
They will be particularly useful in conjunction
with the forwarding mechanism described in \secref{sec:forward}.

For example, it may be useful to have a flag |\version|
which can be set to |draft| or |final|.
The document source will contain some conditional code
depending on the value of |\version|.
Suppose further, the flag should default to |final| for the main file
and to |draft| for child files
which is a natural assignment for editing the document.
This is achieved by placing the following code
in the preamble of the main document
(below the |\childdocmain| directive):
%
\begin{center}
\begin{tabular}{l}
|\ifchilddoc|\\
|\providecommand{\version}{draft}|\\
|\||else|\\
|\providecommand{\version}{final}|\\
|\||fi|
\end{tabular}
\end{center}
%
The definition by |\providecommand| makes sure
that previous definitions are not overwritten.
Further statements |\providecommand{\version}{...}|
can thus be added before the above code to override it.

For the main file, one might add a line
(between |\childdocmain| and the above block)
%
\begin{center}
|%\ifchilddoc\||else\providecommand{\version}{draft}\||fi|
\end{center}
%
which can be uncommented to produce a draft version.
Likewise one can add a line to the very top of a child file
(above the |\childdocof{|\textit{main}|}| directive)
%
\begin{center}
|%\providecommand{\version}{final}|
\end{center}
%
which can be uncommented to produce the final version of this child document.

%%%%%%%%%%%%%%%%%%%%%%%%%%%%%%%%%%%%%%%%%%%%%%%%%%%%%%%%%%%%%%%%%%%%%%%%%%%%%%%%
\subsection{Forwarding}
\label{sec:forward}

Different versions of the main or child documents
using compilation flags as described in \secref{sec:flags}
can be (permanently) stored in different files
for convenient compilation, viewing and distribution.
To this end, the package defines a command
to pass on compilation to a different file:

%%%%%%%%%%%%%%%%%%%%%%%%%%%%%%%%%%%%%%%%
\DescribeMacro{\childdocforward}
The command |\childdocforward| redirects processing to
another source file:
%
\begin{center}
\begin{tabular}{l}
|% \iffalse
%
% childdoc.dtx Copyright (C) 2017-2018 Niklas Beisert
%
% This work may be distributed and/or modified under the
% conditions of the LaTeX Project Public License, either version 1.3
% of this license or (at your option) any later version.
% The latest version of this license is in
%   http://www.latex-project.org/lppl.txt
% and version 1.3 or later is part of all distributions of LaTeX
% version 2005/12/01 or later.
%
% This work has the LPPL maintenance status `maintained'.
%
% The Current Maintainer of this work is Niklas Beisert.
%
% This work consists of the files childdoc.dtx and childdoc.ins
% and the derived files childdoc.def and cdocsamp.tex with
% cdocsch1.tex, cdocsch2.tex, cdocsdrf.tex, cdocsfn1.tex, cdocsfn2.tex.
%
%<package>\ifdefined\childdocmain\endinput\fi
%<package>\ProvidesFile{childdoc.def}[2018/12/30 v2.0 child document driver]
%<samplemain>\ProvidesFile{cdocsamp.tex}[2018/12/30 v2.0 sample for childdoc]
%<*driver>
%\ProvidesFile{childdoc.drv}[2018/12/30 v2.0 childdoc reference manual file]
\PassOptionsToClass{10pt,a4paper}{article}
\documentclass{ltxdoc}

\usepackage[margin=35mm]{geometry}
\usepackage{hyperref}
\usepackage{hyperxmp}
\usepackage[usenames]{color}

\hypersetup{colorlinks=true}
\hypersetup{pdfstartview=FitH}
\hypersetup{pdfpagemode=UseNone}
\hypersetup{pdfsource={}}
\hypersetup{pdflang={en-UK}}
\hypersetup{pdfcopyright={Copyright 2017-2018 Niklas Beisert.
  This work may be distributed and/or modified under the
  conditions of the LaTeX Project Public License, either version 1.3
  of this license or (at your option) any later version.}}
\hypersetup{pdflicenseurl={http://www.latex-project.org/lppl.txt}}
\hypersetup{pdfcontactaddress={ETH Zurich, ITP, HIT K,
  Wolfgang-Pauli-Strasse 27}}
\hypersetup{pdfcontactpostcode={8093}}
\hypersetup{pdfcontactcity={Zurich}}
\hypersetup{pdfcontactcountry={Switzerland}}
\hypersetup{pdfcontactemail={nbeisert@itp.phys.ethz.ch}}
\hypersetup{pdfcontacturl={http://people.phys.ethz.ch/\xmptilde nbeisert/}}

\newcommand{\secref}[1]{\hyperref[#1]{section \ref*{#1}}}

\parskip1ex
\parindent0pt
\let\olditemize\itemize
\def\itemize{\olditemize\parskip0pt}

\begin{document}

\title{The \textsf{childdoc} Package}
\hypersetup{pdftitle={The childdoc Package}}
\author{Niklas Beisert\\[2ex]
  Institut f\"ur Theoretische Physik\\
  Eidgen\"ossische Technische Hochschule Z\"urich\\
  Wolfgang-Pauli-Strasse 27, 8093 Z\"urich, Switzerland\\[1ex]
  \href{mailto:nbeisert@itp.phys.ethz.ch}
  {\texttt{nbeisert@itp.phys.ethz.ch}}}
\hypersetup{pdfauthor={Niklas Beisert}}
\hypersetup{pdfsubject={Manual for the LaTeX2e Package childdoc}}
\date{30 December 2018, \textsf{v2.0}}
\maketitle

\begin{abstract}\noindent
\textsf{childdoc} is a \LaTeXe{} package
that enables the direct compilation
of document sections included by |\include|
to individual files.
\end{abstract}

\begingroup
\parskip0ex
\tableofcontents
\endgroup

%%%%%%%%%%%%%%%%%%%%%%%%%%%%%%%%%%%%%%%%%%%%%%%%%%%%%%%%%%%%%%%%%%%%%%%%%%%%%%%%
%%%%%%%%%%%%%%%%%%%%%%%%%%%%%%%%%%%%%%%%%%%%%%%%%%%%%%%%%%%%%%%%%%%%%%%%%%%%%%%%
\section{Introduction}

\LaTeX{} provides a mechanism to structure a large document (such as a book)
into a main file and several child files (containing the chapters)
using the |\include| command.
This mechanism is beneficial for documents
which span hundreds of pages in order to
make the source file(s) more manageable.
Moreover, compilation can be restricted to
selected child files by means of the |\includeonly| command.
The latter feature can be used to reduce the compilation time while editing
(this was significantly more useful in the earlier days of \LaTeX{})
or to generate a smaller document which is easier to navigate.
Another application of |\includeonly| is to generate
documents consisting of selected parts of the complete document.

However, there are a few drawbacks of the plain |\include| mechanism:
\begin{itemize}
\item
The child files cannot be compiled on their own,
they can only be compiled via the main file.
A naive editing environment
(such as a text editor with an option
to have the current file processed by \LaTeX)
may require one to switch to the main file before compiling;
attempting to compile the child file produces errors.
\item
The main file must be modified (each time)
to adjust the |\includeonly| command
to the present needs. This easily leaves the main file in a messy state.
\item
The generated document will always carry the filename
of the main document. This is inconvenient if
several child files are to be compiled and
to be kept for distribution.
\end{itemize}

The present package provides a simple interface
to make child files individually compilable by \LaTeX{}.
Compiling a child file then has the same effect as compiling
the main file with an |\includeonly| command
to select the appropriate child.
Moreover the generated document will carry the name of the child
rather than the main file.
This resolves all three above issues.

This feature is meant to make the editing of books,
thesis documents and lecture notes somewhat more convenient.
However, the package can also be used efficiently for
composing a series of documents (such as exercise sheets)
which are typically distributed individually.
It then assists the author in generating the individual documents
(potentially in different versions)
as well as a document containing the collected series.
Another application is in developing style files
or other kinds of included material
where compilation of the style file could redirect
to a sample or test file.

%%%%%%%%%%%%%%%%%%%%%%%%%%%%%%%%%%%%%%%%%%%%%%%%%%%%%%%%%%%%%%%%%%%%%%%%%%%%%%%%
%%%%%%%%%%%%%%%%%%%%%%%%%%%%%%%%%%%%%%%%%%%%%%%%%%%%%%%%%%%%%%%%%%%%%%%%%%%%%%%%
\section{Usage}

First of all, the package \textsf{childdoc} is \emph{not} a standard
\LaTeXe{} |.sty| style file! Therefore it needs to be invoked in
a non-standard way.

%%%%%%%%%%%%%%%%%%%%%%%%%%%%%%%%%%%%%%%%%%%%%%%%%%%%%%%%%%%%%%%%%%%%%%%%%%%%%%%%
\subsection{Included Files}
\label{sec:include}

%%%%%%%%%%%%%%%%%%%%%%%%%%%%%%%%%%%%%%%%
\DescribeMacro{\childdocmain}
To use the package, add the commands
\begin{center}
\begin{tabular}{l}
|\input{childdoc.def}|\\
|\childdocmain{}|\\
\end{tabular}
\end{center}
at the very top of the main \LaTeX{} file,
in particular \emph{before} the |\documentclass| statement!
The argument of |\childdocmain| should be left empty
(but it must be present).

%%%%%%%%%%%%%%%%%%%%%%%%%%%%%%%%%%%%%%%%
\DescribeMacro{\childdocof}
Furthermore, add the commands
\begin{center}
\begin{tabular}{l}
|\input{childdoc.def}|\\
|\childdocof{|\textit{main}|}|\\
\end{tabular}
\end{center}
at the top of every child file \textit{child}
which is included by |\include{|\textit{child}|}|
from within the main file
(or at least for those files to be compiled individually).
The argument \textit{main} must be the filename of the main file.

There are a couple of
considerations in setting up the main and child documents:

%%%%%%%%%%%%%%%%%%%%%%%%%%%%%%%%%%%%%%%%
\paragraph{Restrictions.}

Please note the following restrictions:
\begin{itemize}
\item
|\childdocmain| must be called with one argument \textit{main}
to ensure compatibility with earlier version of the package.
It must either be empty (|\childdocmain{}|)
or precisely match the filename of the main file in which it is specified.
See \secref{sec:detection} for further information.
\item
The filename \textit{main} must be specified without the |.tex| extension.
\item
The filename \textit{main} is case sensitive
(even in case-insensitive file systems)
due to internal string comparison.
\item
The argument \textit{main} should be fully expanded, it cannot be a macro.
\item
Subdirectories and special characters should be avoided in filenames.
\item
The command |\childdocmain{|\textit{main}|}| must be followed by a whitespace.
It should not be followed immediately by another command
or by a comment mark `|%|'.
This is because the \TeX{} parser reads the token immediately following
the argument of |\childdocmain| and puts it
at the beginning of every child section;
however, a white\-space is ignored.
\end{itemize}

%%%%%%%%%%%%%%%%%%%%%%%%%%%%%%%%%%%%%%%%
\paragraph{Content of Main File.}

It is advisable to place all content in the child files included by |\include|.
Any output contained in the main file will appear in all child documents
unless suppressed manually;
it cannot be suppressed automatically by the |\includeonly| directive
and thus should normally be avoided.
A method to include some content in the main file
by means of conditional processing is described in \secref{sec:conditional}.

%%%%%%%%%%%%%%%%%%%%%%%%%%%%%%%%%%%%%%%%
\paragraph{Page Numbering.}

When only a part of the document is compiled,
the appropriate numbering of pages
(as well as other status parameters)
is determined from the |.aux| files.
The latter contain information from previous passes.
However this information needs to propagate through
all intermediate child documents.
Therefore the page numbering in child documents may well
be inconsistent until the complete document is compiled at least once.

A useful (if unconventional) way to always ensure a consistent
page numbering is to restart the numbering in each child document
and denote the pages by `\textit{child}|.|\textit{page}'
where \textit{child} represents the chapter/section number of the child file.
This can be achieved by the command
|\numberwithin{page}{|\textit{child}|}|
of the \textsf{amsmath} package
where \textit{child} can be |chapter| or |section|
depending on the chosen structuring.
Alternatively, one can modify the macro |\thepage| appropriately
and reset the counter |page| at the start of each child file.

%%%%%%%%%%%%%%%%%%%%%%%%%%%%%%%%%%%%%%%%%%%%%%%%%%%%%%%%%%%%%%%%%%%%%%%%%%%%%%%%
\subsection{Conditional Processing}
\label{sec:conditional}

The package provides a mechanism to compile different versions
of a document. To customise the versions further some conditional processing
can come in handy to distinguish which version is being compiled.
The package provides two macros to describe the compilation context:

%%%%%%%%%%%%%%%%%%%%%%%%%%%%%%%%%%%%%%%%
\DescribeMacro{\ifchilddoc}
The conditional |\ifchilddoc| distinguishes between the compilation of
child documents and the main document:
%
\begin{center}
|\ifchilddoc |\textit{child-code}| |[|\||else |\textit{main-code}]| \||fi|
\end{center}

%%%%%%%%%%%%%%%%%%%%%%%%%%%%%%%%%%%%%%%%
\DescribeMacro{\childdocname}
\DescribeMacro{\childdocjob}
The macro |\childdocname| contains the filename (without extension)
of the main or child file being processed.
Note that |\childdocjob| will always contain the name of the main file.

%%%%%%%%%%%%%%%%%%%%%%%%%%%%%%%%%%%%%%%%
\paragraph{Title Page.}

Conditional processing can be used to include a title or banner page
in the main document when proper precautions are taken.
Importantly, the code in the main file should ensure that the page counter
(as well as other status parameters which are stored in the |.aux| files)
takes the same value after the conditional processing.
Otherwise the page numbers may take divergent values
depending on which part is compiled.

For example, a title page could be declared by:
%
\begin{center}
\begin{tabular}{l}
|\ifchilddoc\||else|\\
|\addtocounter{page}{-1}|\\
\textit{code for title page}\\
|\newpage|\\
|\||fi|
\end{tabular}
\end{center}
%
A banner page for the child documents can be generated by:
%
\begin{center}
\begin{tabular}{l}
|\ifchilddoc|\\
|\addtocounter{page}{-1}|\\
\textit{code for banner page}\\
|\newpage|\\
|\||fi|
\end{tabular}
\end{center}
%
Here one could write a message such as:
\begin{center}
|This is the part \childdocname{} of \childdocjob{}.|
\end{center}

%%%%%%%%%%%%%%%%%%%%%%%%%%%%%%%%%%%%%%%%%%%%%%%%%%%%%%%%%%%%%%%%%%%%%%%%%%%%%%%%
\subsection{Flags}
\label{sec:flags}

The package makes it easy to generate different versions
of the main or child documents.
To this end compilation flags can be defined
and assigned different default values.
They will be particularly useful in conjunction
with the forwarding mechanism described in \secref{sec:forward}.

For example, it may be useful to have a flag |\version|
which can be set to |draft| or |final|.
The document source will contain some conditional code
depending on the value of |\version|.
Suppose further, the flag should default to |final| for the main file
and to |draft| for child files
which is a natural assignment for editing the document.
This is achieved by placing the following code
in the preamble of the main document
(below the |\childdocmain| directive):
%
\begin{center}
\begin{tabular}{l}
|\ifchilddoc|\\
|\providecommand{\version}{draft}|\\
|\||else|\\
|\providecommand{\version}{final}|\\
|\||fi|
\end{tabular}
\end{center}
%
The definition by |\providecommand| makes sure
that previous definitions are not overwritten.
Further statements |\providecommand{\version}{...}|
can thus be added before the above code to override it.

For the main file, one might add a line
(between |\childdocmain| and the above block)
%
\begin{center}
|%\ifchilddoc\||else\providecommand{\version}{draft}\||fi|
\end{center}
%
which can be uncommented to produce a draft version.
Likewise one can add a line to the very top of a child file
(above the |\childdocof{|\textit{main}|}| directive)
%
\begin{center}
|%\providecommand{\version}{final}|
\end{center}
%
which can be uncommented to produce the final version of this child document.

%%%%%%%%%%%%%%%%%%%%%%%%%%%%%%%%%%%%%%%%%%%%%%%%%%%%%%%%%%%%%%%%%%%%%%%%%%%%%%%%
\subsection{Forwarding}
\label{sec:forward}

Different versions of the main or child documents
using compilation flags as described in \secref{sec:flags}
can be (permanently) stored in different files
for convenient compilation, viewing and distribution.
To this end, the package defines a command
to pass on compilation to a different file:

%%%%%%%%%%%%%%%%%%%%%%%%%%%%%%%%%%%%%%%%
\DescribeMacro{\childdocforward}
The command |\childdocforward| redirects processing to
another source file:
%
\begin{center}
\begin{tabular}{l}
|\input{childdoc.def}|\\
|\childdocforward[|\textit{main}|]{|\textit{dest}|}|\\
\end{tabular}
\end{center}
%
The argument \textit{dest} is the destination file
(without extension).
It should be the main file or one of the child files.
Note that further \textsf{childdoc} directives
such as |\childdocof| and |\childdocforward|
in the indicated file will be processed in this form.
The optional argument \textit{main}
passes on directly to the main file \textit{main}
while pretending to compile the child \textit{dest}.
This form behaves as if \textit{dest}
issues |\childdocof{|\textit{main}|}| right away,
and no further \textsf{childdoc} directives will be processed.

%%%%%%%%%%%%%%%%%%%%%%%%%%%%%%%%%%%%%%%%
\DescribeMacro{\...prefix}
In the alternative form |\childdocforwardprefix|,
%
\begin{center}
\begin{tabular}{l}
|\input{childdoc.def}|\\
|\childdocforwardprefix[|\textit{main}|]{|\textit{prefix}|}{|\textit{dest}|}|
\end{tabular}
\end{center}
%
the destination file is determined by a pattern
depending on the current file:
To make this work, the current file must be called
`{\textit{prefix}\hspace{0.2em}\textit{suffix}}'
with \textit{prefix} matching precisely the argument.
Processing is then passed on to the file
`{\textit{dest}\hspace{0.2em}\textit{suffix}}'.
Surely, the same effect is achieved by
directly specifying the
argument `{\textit{dest}\hspace{0.2em}\textit{suffix}}'
in the first form.
However, that requires to set up a different file
for each child. With the alternative form of the command
all these files can have exactly the same content
which simplifies setting them up and maintaining them.

For example, the following file |draft.tex|
with a compilation flag |\version| as described in \secref{sec:flags}
compiles the main document as a draft:
%
\begin{center}
\begin{tabular}{l}
|\def\version{draft}|\\
|\input{childdoc.def}|\\
|\childdocforward{|\textit{main}|}|
\end{tabular}
\end{center}
%
Likewise, the following files |final|\textit{nn}|.tex|
compile the final version of the child document
|child|\textit{nn}|.tex|:
%
\begin{center}
\begin{tabular}{l}
|\def\version{final}|\\
|\input{childdoc.def}|\\
|\childdocforwardprefix{final}{child}|
\end{tabular}
\end{center}
%

Note that when several versions of a main file and/or of each child file
are to be generated, it may be convenient to set up a |Makefile| or
shell script to automatise the process.

%%%%%%%%%%%%%%%%%%%%%%%%%%%%%%%%%%%%%%%%%%%%%%%%%%%%%%%%%%%%%%%%%%%%%%%%%%%%%%%%
\subsection{Command Line Processing}
\label{sec:commandline}

The effect of redirection files can also be achieved by invoking
the \LaTeX{} compiler with a more elaborate command line.
Most conveniently this should be done as part
of a shell script or a |Makefile|.

When using \textsf{childdoc} in the main file, the following
command lines effectively perform a redirection
(note that depending on the shell being used,
backslashes may have to be doubled: `|\|' $\to$ `|\\|'):
%
\begin{center}
|... -jobname "|\textit{target}|" |\\|"|[\textit{flags}]%
|\input{childdoc.def}\childdocforward[|\textit{main}|]{|\textit{dest}|}"|
\end{center}
%
Here \textit{target} is the name of the output file,
\textit{main} is the name of the main file
and \textit{dest} is the name of the main or child file to be processed
(all filenames without extensions).
The optional argument \textit{main} can be omitted
if \textit{main} matches \textit{dest}.
Optionally, compilation \textit{flags} can be defined via |\def| commands.
This command line makes the \TeX{} engine believe
it is compiling the file \textit{target}
whose content is specified as the latter parameter.
The provided code then forwards the processing to
\textit{main} or \textit{dest} as described in \secref{sec:forward}.

%%%%%%%%%%%%%%%%%%%%%%%%%%%%%%%%%%%%%%%%%%%%%%%%%%%%%%%%%%%%%%%%%%%%%%%%%%%%%%%%
\subsection{Include by Input}
\label{sec:input}

Including child documents by |\include| has some restrictions by design.
Most notably, the content of a child document always occupies
its own set of pages; pages cannot be shared between child documents.
Usually, this behaviour makes perfect sense
because each child document contain an essential part of the document.
However, in some situations it may be desirable to compose
a document from a collection of parts
without having mandatory page breaks between then.
For this case, the package
provides a mechanism to include parts
by |\input| which can also be processed individually.
However, by construction this mechanism
requires manual handling of the content to be output.

%%%%%%%%%%%%%%%%%%%%%%%%%%%%%%%%%%%%%%%%
\DescribeMacro{\ifchilddocmanual}
The main file should be prepared as usual, see \secref{sec:include}.
However, the document body must make a distinction
between processing of an individual part and of the main document, e.g.:
%
\begin{center}
\begin{tabular}{l}
|\ifchilddocmanual|\\
|\input{\childdocname}|\\
|\||else|\\
\textit{document body with }|\input{|\textit{part}|}|\\
|\||fi|
\end{tabular}
\end{center}
%
The conditional |\ifchilddocmanual| is true whenever
a part to be included by |\input| is being compiled,
and the name of the part is stored in |\childdocname|.

%%%%%%%%%%%%%%%%%%%%%%%%%%%%%%%%%%%%%%%%
\DescribeMacro{\childdocby}
Each part to be included by |\input| should start with:
%
\begin{center}
\begin{tabular}{l}
|\input{childdoc.def}|\\
|\childdocby{|\textit{main}|}|\\
\end{tabular}
\end{center}
%
The directive |\childdocby| is similar to |\childdocof|
described in \secref{sec:include},
but the subsequent selection of content must be done manually.
To that end, both |\ifchilddoc| and |\ifchilddocmanual|
will be true upon processing of a part,
and the name of the part is stored in |\childdocname|.
Note that |\jobname| will be set to the filename of the current part
so that each part receives an individual |.aux| file
that does not interfere with the |.aux| file(s) of the main document.
This behaviour can be altered by the alternative form
|\childdocby[*]{|\textit{main}|}| (with a non-empty optional argument)
which uses the |.aux| file of the main document
by setting |\jobname| to \textit{main}.

%%%%%%%%%%%%%%%%%%%%%%%%%%%%%%%%%%%%%%%%%%%%%%%%%%%%%%%%%%%%%%%%%%%%%%%%%%%%%%%%
\subsection{Driver Development}
\label{sec:driver}

The \textsf{childdoc} mechanism can also be use for the development
of definition files such as \LaTeX{} styles or classes.
This case differs from the above setup with multiple parts
included by |\include| in that no |\includeonly| should be invoked.
This can be achieved by starting the include file
(before |\ProvidesPackage|) with:
%
\begin{center}
\begin{tabular}{l}
|\input{childdoc.def}|\\
|\childdocforward{|\textit{main}|}|\\
\end{tabular}
\end{center}
%
or alternatively with:
%
\begin{center}
\begin{tabular}{l}
|\input{childdoc.def}|\\
|\childdocby{|\textit{main}|}|\\
\end{tabular}
\end{center}
%
Both forms have slightly different effects as described above.
The main file is prepared as usual, see \secref{sec:include}.

%%%%%%%%%%%%%%%%%%%%%%%%%%%%%%%%%%%%%%%%%%%%%%%%%%%%%%%%%%%%%%%%%%%%%%%%%%%%%%%%
\subsection{Legacy Detection}
\label{sec:detection}

The directive |\childdocmain| in the main file can detect
whether the complete document or merely a child is to be compiled
even without using the directive |\childdocof|.
This method is deprecated because it is less robust
and there is no compelling reason to use it;
it is merely provided for backward compatibility
and it may be removed in future versions.

If the detection mechanism is to be used,
it is mandatory to correctly specify
the filename of the main file as the argument of |\childdocmain|:
%
\begin{center}
\begin{tabular}{l}
|\input{childdoc.def}|\\
|\childdocmain{|\textit{main}|}|\\
\end{tabular}
\end{center}
%
If |\jobname| does not match the argument \textit{main} of |\childdocmain|,
it is assumed that |\jobname| points to the child file to be compiled.
When using |\childdocmain| with the main file specified as argument,
it suffices to start a child file
with just |\input{|\textit{main}|}|
without loading of the package and using |\childdocof|.
If instead all processing is done
with the appropriate \textsf{childdoc} directives,
the argument of \textit{main} of |\childdocmain| can be empty.

An alternative version of the command line processing described
in \secref{sec:commandline} using the detection mechanism reads:
%
\begin{center}
|... -jobname "|\textit{target}|" "|[\textit{flags}]%
[|\def\jobname{|\textit{dest}|}|]|\input{|\textit{main}|}"|
\end{center}

%%%%%%%%%%%%%%%%%%%%%%%%%%%%%%%%%%%%%%%%%%%%%%%%%%%%%%%%%%%%%%%%%%%%%%%%%%%%%%%%
\subsection{Manual Code}
\label{sec:manual}

In case one cannot be certain whether the definitions file |childdoc.def|
is installed on the target \TeX{} distribution
and one prefers not to ship it,
it is conceivable to paste a few relevant commands into the sources.

To that end, drop all statements |\input{childdoc.def}|
and perform the replacements as outlined below.
Instead of |\childdocmain{|\textit{main}|}| add the following code
to the top of the main file:
%
\begin{center}
\begin{tabular}{l}
|\||ifdefined\childdocname\endinput\||fi\newif\ifchilddoc|\\
|\edef\childdocname{\scantokens\expandafter{\jobname\noexpand}}|\\
|\def\childdocmain{|\textit{main}|}\||ifx\childdocmain\childdocname\||else|\\
|\childdoctrue\includeonly{\childdocname}\let\jobname\childdocmain\||fi|\\
\end{tabular}
\end{center}
%
Instead of |\childdocof{|\textit{main}|}| just include the main file
at the top of each child file:
%
\begin{center}
|\input{|\textit{main}|}|
\end{center}
%
A simple redirection |\childdocforward{|\textit{dest}|}| is achieved by:
%
\begin{center}
|\def\jobname{|\textit{dest}|}\input{\jobname}|
\end{center}
%
The redirection with prefix
|\childdocforwardprefix[|\textit{prefix}|]{|\textit{dest}|}|
is accomplished by:
%
\begin{center}
\begin{tabular}{l}
|{\edef\jobname{\scantokens\expandafter{\jobname\noexpand}}|\\
|\def\redirectjob |\textit{prefix}|#1~~~{\gdef\jobname{|\textit{dest}|#1}}|\\
|\expandafter\redirectjob\jobname~~~}\input{\jobname}|
\end{tabular}
\end{center}

In an alternative approach,
child documents can be compiled by a specific command line
without additional code or specific definitions:
%
\begin{center}
|... -jobname "|\textit{target}|" "|[\textit{flags}]%
|\includeonly{|\textit{dest}|}\input{|\textit{main}|}"|
\end{center}
%

%%%%%%%%%%%%%%%%%%%%%%%%%%%%%%%%%%%%%%%%%%%%%%%%%%%%%%%%%%%%%%%%%%%%%%%%%%%%%%%%
%%%%%%%%%%%%%%%%%%%%%%%%%%%%%%%%%%%%%%%%%%%%%%%%%%%%%%%%%%%%%%%%%%%%%%%%%%%%%%%%
\section{Information}

%%%%%%%%%%%%%%%%%%%%%%%%%%%%%%%%%%%%%%%%%%%%%%%%%%%%%%%%%%%%%%%%%%%%%%%%%%%%%%%%
\subsection{Copyright}

Copyright \copyright{} 2017--2018 Niklas Beisert

This work may be distributed and/or modified under the
conditions of the \LaTeX{} Project Public License, either version 1.3
of this license or (at your option) any later version.
The latest version of this license is in
  \url{http://www.latex-project.org/lppl.txt}
and version 1.3 or later is part of all distributions of \LaTeX{}
version 2005/12/01 or later.

This work has the LPPL maintenance status `maintained'.

The Current Maintainer of this work is Niklas Beisert.

This work consists of the files |README.txt|, |childdoc.ins| and |childdoc.dtx|
as well as the derived files |childdoc.def|, |cdocsamp.tex|
with |cdocsch1.tex|, |cdocsch2.tex|, |cdocspt3.tex|, |cdocspt4.tex|,
|cdocsdrf.tex|, |cdocsfn1.tex|, |cdocsfn2.tex|
as well as |childdoc.pdf|.

%%%%%%%%%%%%%%%%%%%%%%%%%%%%%%%%%%%%%%%%%%%%%%%%%%%%%%%%%%%%%%%%%%%%%%%%%%%%%%%%
\subsection{Files and Installation}

The package consists of the files:
%
\begin{center}
\begin{tabular}{ll}
    |README.txt|   & readme file \\
    |childdoc.ins| & installation file \\
    |childdoc.dtx| & source file \\
    |childdoc.def| & definition file \\
    |cdocsamp.tex| & sample main file \\
    |cdocsch1.tex| & sample include file \\
    |cdocsch2.tex| & sample include file \\
    |cdocspt3.tex| & sample part file \\
    |cdocspt4.tex| & sample part file \\
    |cdocsdrf.tex| & sample redirection file \\
    |cdocsfn1.tex| & sample redirection file \\
    |cdocsfn2.tex| & sample redirection file \\
    |childdoc.pdf| & manual
\end{tabular}
\end{center}
%
The distribution consists of the files
|README.txt|, |childdoc.ins| and |childdoc.dtx|.
%
\begin{itemize}
\item
Run (pdf)\LaTeX{} on |childdoc.dtx|
to compile the manual |childdoc.pdf| (this file).
\item
Run \LaTeX{} on |childdoc.ins| to create the definitions file |childdoc.def|
and the sample |cdocsamp.tex| with include files
|cdocsch1.tex|, |cdocsch2.tex|, |cdocspt3.tex|, |cdocspt4.tex|,
|cdocsdrf.tex|, |cdocsfn1.tex|, |cdocsfn2.tex|.
Then copy the file |childdoc.def| to an appropriate directory of your \LaTeX{}
distribution, e.g.\ \textit{texmf-root}|/tex/latex/childdoc|.
\end{itemize}

%%%%%%%%%%%%%%%%%%%%%%%%%%%%%%%%%%%%%%%%%%%%%%%%%%%%%%%%%%%%%%%%%%%%%%%%%%%%%%%%
\subsection{Related CTAN Packages}

There are several other packages which offer a similar functionality:
%
\begin{itemize}
\item
The packages
\href{http://ctan.org/pkg/docmute}{\textsf{docmute}},
\href{http://ctan.org/pkg/includex}{\textsf{includex}} and
\href{http://ctan.org/pkg/standalone}{\textsf{standalone}}
provide commands to include only the document body of
a child file thus allowing both files to be compiled individually.
\item
The packages \href{http://ctan.org/pkg/subdocs}{\textsf{subdocs}}
and \href{http://ctan.org/pkg/subfiles}{\textsf{subfiles}}
provide structures in which the main and child documents can be
encapsulated and allowing them to be compiled individually.
The inclusion mechanism is different from the conventional |\include|.
\item
The package \href{http://ctan.org/pkg/combine}{\textsf{combine}}
is an elaborate solution to combine several documents into one.
\end{itemize}
%
See also the CTAN topic \href{http://ctan.org/topic/subdocs}{\textsf{subdocs}}
for further related packages.
The present package differs from the above solutions in that
a document structure constructed with the conventional |\include| mechanism
just needs two extra commands at the top of every file
such that all constituent files can be compiled individually.

%%%%%%%%%%%%%%%%%%%%%%%%%%%%%%%%%%%%%%%%%%%%%%%%%%%%%%%%%%%%%%%%%%%%%%%%%%%%%%%%
%\subsection{Feature Suggestions}
%
%The following is a list of features which may be useful for future
%versions of this package:
%%
%\begin{itemize}
%\item
%\ldots
%\end{itemize}

%%%%%%%%%%%%%%%%%%%%%%%%%%%%%%%%%%%%%%%%%%%%%%%%%%%%%%%%%%%%%%%%%%%%%%%%%%%%%%%%
\subsection{Revision History}

%%%%%%%%%%%%%%%%%%%%%%%%%%%%%%%%%%%%%%%%
\paragraph{v2.0:} 2018/12/30

\begin{itemize}
\item
immediate forward processing
\item
added |\childdocby| mechanism
\item
manual restructured
\end{itemize}

%%%%%%%%%%%%%%%%%%%%%%%%%%%%%%%%%%%%%%%%
\paragraph{v1.6:} 2018/01/17

\begin{itemize}
\item
application for development of include files
\item
corrections to manual
\end{itemize}

%%%%%%%%%%%%%%%%%%%%%%%%%%%%%%%%%%%%%%%%
\paragraph{v1.5:} 2017/05/21

\begin{itemize}
\item
more complete structuring introduced
\item
|\childdocof| introduced
\item
|\childdoc| renamed to |\childdocmain|
\item
|\childredirect| renamed to |\childdocforward| and |\childdocforwardprefix|
and functionality expanded
\end{itemize}

%%%%%%%%%%%%%%%%%%%%%%%%%%%%%%%%%%%%%%%%
\paragraph{v1.0:} 2017/04/27

\begin{itemize}
\item
manual and install package
\item
first version published on CTAN
\end{itemize}

%%%%%%%%%%%%%%%%%%%%%%%%%%%%%%%%%%%%%%%%
\paragraph{v0.6:} 2017/04/26

\begin{itemize}
\item
redirection mechanism added
\end{itemize}

%%%%%%%%%%%%%%%%%%%%%%%%%%%%%%%%%%%%%%%%
\paragraph{v0.5:} 2017/04/26

\begin{itemize}
\item
functionality in definition file
\end{itemize}


%%%%%%%%%%%%%%%%%%%%%%%%%%%%%%%%%%%%%%%%%%%%%%%%%%%%%%%%%%%%%%%%%%%%%%%%%%%%%%%%
%%%%%%%%%%%%%%%%%%%%%%%%%%%%%%%%%%%%%%%%%%%%%%%%%%%%%%%%%%%%%%%%%%%%%%%%%%%%%%%%
%%%%%%%%%%%%%%%%%%%%%%%%%%%%%%%%%%%%%%%%%%%%%%%%%%%%%%%%%%%%%%%%%%%%%%%%%%%%%%%%
\appendix

\settowidth\MacroIndent{\rmfamily\scriptsize 000\ }

 \DocInput{childdoc.dtx}

\end{document}
%</driver>
% \fi
%
% %%%%%%%%%%%%%%%%%%%%%%%%%%%%%%%%%%%%%%%%%%%%%%%%%%%%%%%%%%%%%%%%%%%%%%%%%%%%%%
% %%%%%%%%%%%%%%%%%%%%%%%%%%%%%%%%%%%%%%%%%%%%%%%%%%%%%%%%%%%%%%%%%%%%%%%%%%%%%%
% \section{Sample}
%\iffalse
%<*samplemain>
%\fi
%
% The following presents a sample document
% with two chapters, two parts, a title page,
% a compile flag as well as three forwarding files to set the flag.
% It consists of eight |.tex| files:
% \begin{center}
% \begin{tabular}{ll}
% |cdocsamp.tex|&main file\\
% |cdocsch1.tex|&include file for chapter 1\\
% |cdocsch2.tex|&include file for chapter 2\\
% |cdocspt3.tex|&include file for part 3\\
% |cdocspt4.tex|&include file for part 4\\
% |cdocsdrf.tex|&forwarding file for main file in draft mode\\
% |cdocsfi1.tex|&forwarding file for final version of chapter 1\\
% |cdocsfi2.tex|&forwarding file for final version of chapter 2\\
% \end{tabular}
% \end{center}
% Each of the eight files can be compiled directly by the \LaTeX{} compiler.
%
% %%%%%%%%%%%%%%%%%%%%%%%%%%%%%%%%%%%%%%
% \paragraph{Main File.}
%
% The main file is called |cdocsamp.tex|.
%
% Load the \textsf{childdoc} definitions and
% declare the filename for the main document:
%    \begin{macrocode}
\input{childdoc.def}
\childdocmain{}
%    \end{macrocode}

% Optional override for |\version| flag:
%    \begin{macrocode}
%%\ifchilddoc\else\providecommand{\version}{draft}\fi
%    \end{macrocode}

% Define the default values for the |\version| flag
% (|final| for the main file and |draft| for childs):
%    \begin{macrocode}
\ifchilddoc
\providecommand{\version}{draft}
\else
\providecommand{\version}{final}
\fi
%    \end{macrocode}

% Load the standard document class:
%    \begin{macrocode}
\documentclass[12pt]{article}
%    \end{macrocode}

% Start the document body:
%    \begin{macrocode}
\begin{document}
%    \end{macrocode}

% Declare a title page.
% Print title, part of document being processed and version flag:
%    \begin{macrocode}
\addtocounter{page}{-1}
\begin{center}
{\LARGE\bfseries{}childdoc example\par}
\vspace{1cm}
\ifchilddoc
\ifchilddocmanual part\else chapter\fi:
`\childdocname' of `\childdocjob'\par
\else
main document: `\childdocjob'\par
\fi
version: \version\par
\end{center}
\newpage
%    \end{macrocode}

% Manually include selected file,
% otherwise process as usual:
%    \begin{macrocode}
\ifchilddocmanual
\section*{part `\childdocname'}
\input{\childdocname}
\else
%    \end{macrocode}

% Include the two chapters:
%    \begin{macrocode}
\include{cdocsch1}
\include{cdocsch2}
%    \end{macrocode}

% Include the two parts unless only chapters should be displayed:
%    \begin{macrocode}
\ifchilddoc\else
\section{part three}
\input{cdocspt3}
\section{part four}
\input{cdocspt4}
\fi
%    \end{macrocode}

% Process as usual until here:
%    \begin{macrocode}
\fi
%    \end{macrocode}

% End of document body:
%    \begin{macrocode}
\end{document}
%    \end{macrocode}
%\iffalse
%</samplemain>
%\fi
%
% %%%%%%%%%%%%%%%%%%%%%%%%%%%%%%%%%%%%%%
% \paragraph{Chapter Include Files.}
%
% The include files are called |cdocsch1.tex| and |cdocsch2.tex|.
%
%\iffalse
%<*samplechap1|samplechap2>
%\fi

% Optional override for |\version| flag:
%    \begin{macrocode}
%%\providecommand{\version}{final}
%    \end{macrocode}

% Include the main document:
%    \begin{macrocode}
\input{childdoc.def}
\childdocof{cdocsamp}
%    \end{macrocode}

%\iffalse
%</samplechap1|samplechap2>
%\fi
%
%\iffalse
%<*samplechap1>
%\fi
% Some text for chapter 1:
%    \begin{macrocode}
\section{one}
some text in chapter one
%    \end{macrocode}

%\iffalse
%</samplechap1>
%\fi
% Some text for chapter 2:
%\iffalse
%<*samplechap2>
%\fi
%    \begin{macrocode}
\section{two}
more text in chapter two
%    \end{macrocode}

%\iffalse
%</samplechap2>
%\fi
%
% %%%%%%%%%%%%%%%%%%%%%%%%%%%%%%%%%%%%%%
% \paragraph{Part Include Files.}
%
% The include files are called |cdocspt3.tex| and |cdocspt4.tex|.
%
%\iffalse
%<*samplepart3|samplepart4>
%\fi

% Optional override for |\version| flag:
%    \begin{macrocode}
%%\providecommand{\version}{final}
%    \end{macrocode}

% Include the main document:
%    \begin{macrocode}
\input{childdoc.def}
\childdocby{cdocsamp}
%    \end{macrocode}

%\iffalse
%</samplepart3|samplepart4>
%\fi
%
%\iffalse
%<*samplepart3>
%\fi
% Some text for part 3:
%    \begin{macrocode}
some text in part three
%    \end{macrocode}

%\iffalse
%</samplepart3>
%\fi
% Some text for part 4:
%\iffalse
%<*samplepart4>
%\fi
%    \begin{macrocode}
more text in part four
%    \end{macrocode}

%\iffalse
%</samplepart4>
%\fi
%
% %%%%%%%%%%%%%%%%%%%%%%%%%%%%%%%%%%%%%%
% \paragraph{Forwarding for a Complete Draft.}
%
% The following forwarding file |cdocsdrf.tex|
% compiles the main document in draft mode:
%\iffalse
%<*sampledraft>
%\fi
%    \begin{macrocode}
\def\version{draft}
\input{childdoc.def}
\childdocforward{cdocsamp}
%    \end{macrocode}

%\iffalse
%</sampledraft>
%\fi
%
% %%%%%%%%%%%%%%%%%%%%%%%%%%%%%%%%%%%%%%
% \paragraph{Forwarding for Final Version of the Chapters.}
%
% The following forwarding files |cdocsfn1.tex| and |cdocsfn2.tex|
% (with identical content)
% compile the final versions of the child documents
% |cdocsch1.tex| and |cdocsch2.tex|, respectively:
%\iffalse
%<*samplefinal>
%\fi
%    \begin{macrocode}
\def\version{final}
\input{childdoc.def}
\childdocforwardprefix[cdocsamp]{cdocsfn}{cdocsch}
%    \end{macrocode}

%\iffalse
%</samplefinal>
%\fi
%
% %%%%%%%%%%%%%%%%%%%%%%%%%%%%%%%%%%%%%%
% \paragraph{Command Line Processing.}
%
% The following three command lines generate the output files
% |cdocscld|, |cdocscl1| and |cdocscl2|
% which should be identical to
% |cdocsdrf|, |cdocsch1| and |cdocsfn2|, respectively:
% \begin{center}
% \begin{tabular}{l}
% |latex -jobname cdocscld \|\\
% |  "\def\version{draft}\input{childdoc.def}\childdocforward{cdocsamp}"|\\
% |latex -jobname cdocscl1 \|\\
% |  "\input{childdoc.def}\childdocforward[cdocsamp]{cdocsch1}"|\\
% |latex -jobname cdocscl2 \|\\
% |  "\def\version{final}\input{childdoc.def}\childdocforward{cdocsch2}"|
% \end{tabular}
% \end{center}
% Note that the trailing backslash on each first line
% merely continues the input to the second line
% (for convenient cut ant paste).
% Furthermore, the command |latex| can be replaced by any
% of its alternative versions such as |pdflatex|.
%
% %%%%%%%%%%%%%%%%%%%%%%%%%%%%%%%%%%%%%%%%%%%%%%%%%%%%%%%%%%%%%%%%%%%%%%%%%%%%%%
% %%%%%%%%%%%%%%%%%%%%%%%%%%%%%%%%%%%%%%%%%%%%%%%%%%%%%%%%%%%%%%%%%%%%%%%%%%%%%%
% \section{Implementation}
%\iffalse
%<*package>
%\fi
%
% This section describes the definitions file |childdoc.def|.

% The definitions cannot be loaded using |\usepackage| or |\RequirePackage|
% which has a mechanism to prevent loading a style file more than once.
% When loading the definitions by means of |\input|
% multiple instances have to be prevented manually:
%\iffalse
%This code needs to be before the `\ProvidesFile' directive
%which is defined at the beginning of this file.
%Therefore it is also placed there and commented out here.
%</package>
%<*discard>
%\fi
%    \begin{macrocode}
\ifdefined\childdocmain\endinput\fi
%    \end{macrocode}
%\iffalse
%</discard>
%<*package>
%\fi
%
% \macro{\ifchilddoc}
% \macro{\ifchilddocmanual}
% The conditional |\ifchilddoc| tells whether a
% child (true) or main (false) document is being compiled.
% The conditional |\ifchilddocmanual| tells whether
% the |\includeonly| mechanism is used (false) or
% the selection of child files must be performed manually (true).
% The definitions initialise to false:
%    \begin{macrocode}
\newif\ifchilddoc
\newif\ifchilddocmanual
%    \end{macrocode}

% \macro{\childdocname}
% \macro{\childdocjob}
% The macro |\childdocname| stores the name of the main document
% to be compiled. The macro |\childdocjob| stores the name of
% the document on which the \LaTeX{} compiler was originally invoked.
% The content of |\jobname| cannot be compared
% to filenames specified in the source due to different catcodes.
% The following code rescans |\jobname|, stores the result
% in |\childdocname| and saves a copy in |\childdocjob|:
%    \begin{macrocode}
\edef\childdocname{\scantokens\expandafter{\jobname\noexpand}}
\let\childdocjob\childdocname
%    \end{macrocode}

% \macro{\childdocdisable}
% The macro |\childdocdisable| prevents the main file
% from being processed more than once.
% At this stage, the main document command |\childdocmain|
% is assumed to be called once again where it should do nothing.
% Any subsequent call to it should prevent
% a secondary processing of the main document
% It overwrites the forwarding commands
% |\childdocof| and |\childdocforward|
% with empty macros to prevent further inclusions of the main document:
%    \begin{macrocode}
\newcommand{\childdocdisable}
{
  \renewcommand{\childdocmain}[1]{\renewcommand{\childdocmain}[1]{\endinput}}
  \renewcommand{\childdocof}[1]{}
  \renewcommand{\childdocby}[2][]{}
  \renewcommand{\childdocforward}[2][]{}
  \renewcommand{\childdocdisable}{}
}
%    \end{macrocode}

% \macro{\childdocmain}
% The macro |\childdocmain| is to be called at the top of the main file
% with nothing or the main filename (without extension) as argument.
% First, it breaks loops.
% If the argument is not empty and does not match |\childdocname|
% (which is set by the first inclusion of |childdoc.def|),
% |\ifchilddoc| is set to true, |\includeonly| is applied to the child file
% and |\jobname| is set to the main file
% (for proper handling of |.aux| files):
%    \begin{macrocode}
\newcommand{\childdocmain}[1]
{
  \childdocdisable\childdocmain{}
  \if?#1?\else
    \begingroup
      \def\childdoctmp{#1}
      \ifx\childdoctmp\childdocname
        \def\childdoctmp{}
      \else
        \def\childdoctmp
        {
          \childdoctrue
          \includeonly{\childdocname}
          \def\childdocjob{#1}
          \def\jobname{#1}
        }
      \fi
      \expandafter
    \endgroup
    \childdoctmp
  \fi
}
%    \end{macrocode}

% \macro{\childdocof}
% The command |\childdocof| redirects
% compilation to the main file |#1|.
%    \begin{macrocode}
\newcommand{\childdocof}[1]
{
  \childdocdisable
  \childdoctrue
  \includeonly{\childdocname}
  \def\jobname{#1}
  \def\childdocjob{#1}
  \input{#1}
}
%    \end{macrocode}

% \macro{\childdocby}
% The command |\childdocby| ....
%    \begin{macrocode}
\newcommand{\childdocby}[2][]
{
  \childdocdisable
  \childdoctrue
  \childdocmanualtrue
  \if?#1?\else
    \def\jobname{#2}
  \fi
  \def\childdocjob{#2}
  \input{#2}
  \endinput
}
%    \end{macrocode}

% \macro{\childdocforward}
% The command |\childdocforward| redirects
% compilation to the main file or
% (if the optional argument is given) a child file.
% Parameters are set as if the main file
% or a child file starting with |\childdocof| was compiled.
% Then compilation is handed over to the main file:
%    \begin{macrocode}
\newcommand{\childdocforward}[2][]
{
  \begingroup
    \if?#1?
      \def\childdoctmp
      {
        \def\childdocname{#2}
        \def\childdocjob{#2}
        \def\jobname{#2}
        \input{#2}
        \endinput
      }
    \else
      \def\childdoctmp
      {
        \childdocdisable
        \def\childdocname{#2}
        \childdoctrue
        \includeonly{#2}
        \def\childdocjob{#1}
        \def\jobname{#1}
        \input{#1}
        \endinput
      }
    \fi
    \expandafter
  \endgroup
  \childdoctmp
}
%    \end{macrocode}

% \macro{\childdocforwardprefix}
% The command |\childdocforwardprefix| redirects
% compilation to the main or a child file by means of a pattern.
% The prefix |#1| in the current filename is replaced by |#2|
% and the suffix of the current filename is kept
% (it is assumed that the filename does not contain the substring `|~~~|'
% which is used as a delimiter).
% Compilation is handed over to the new file by |\childdocforward|:
%    \begin{macrocode}
\newcommand{\childdocforwardprefix}[3][]
{
  \begingroup
    \def\childdocextract #2##1~~~{\def\childdoctmp{\childdocforward[#1]{#3##1}}}
    \expandafter\childdocextract\childdocname~~~
    \expandafter
  \endgroup
  \childdoctmp
}
%    \end{macrocode}

% \macro{\childdoc}
% The deprecated macro |\childdoc| is a legacy version of |\childdocmain|:
%    \begin{macrocode}
\newcommand{\childdoc}{\childdocmain}
%    \end{macrocode}

% \macro{\childdocredirect}
% The deprecated macro |\childdocredirect| is a legacy version
% of |\childdocforward| and |\childdocforwardprefix|:
%    \begin{macrocode}
\newcommand{\childdocredirect}[2][]
{
  \begingroup
    \if?#1?
      \def\childdoctmp{\childdocforward{#2}}
    \else
      \def\childdoctmp{\childdocforwardprefix{#1}{#2}}
    \fi
    \expandafter
  \endgroup
  \childdoctmp
}
%    \end{macrocode}

%\iffalse
%</package>
%\fi
%
\endinput
|\\
|\childdocforward[|\textit{main}|]{|\textit{dest}|}|\\
\end{tabular}
\end{center}
%
The argument \textit{dest} is the destination file
(without extension).
It should be the main file or one of the child files.
Note that further \textsf{childdoc} directives
such as |\childdocof| and |\childdocforward|
in the indicated file will be processed in this form.
The optional argument \textit{main}
passes on directly to the main file \textit{main}
while pretending to compile the child \textit{dest}.
This form behaves as if \textit{dest}
issues |\childdocof{|\textit{main}|}| right away,
and no further \textsf{childdoc} directives will be processed.

%%%%%%%%%%%%%%%%%%%%%%%%%%%%%%%%%%%%%%%%
\DescribeMacro{\...prefix}
In the alternative form |\childdocforwardprefix|,
%
\begin{center}
\begin{tabular}{l}
|% \iffalse
%
% childdoc.dtx Copyright (C) 2017-2018 Niklas Beisert
%
% This work may be distributed and/or modified under the
% conditions of the LaTeX Project Public License, either version 1.3
% of this license or (at your option) any later version.
% The latest version of this license is in
%   http://www.latex-project.org/lppl.txt
% and version 1.3 or later is part of all distributions of LaTeX
% version 2005/12/01 or later.
%
% This work has the LPPL maintenance status `maintained'.
%
% The Current Maintainer of this work is Niklas Beisert.
%
% This work consists of the files childdoc.dtx and childdoc.ins
% and the derived files childdoc.def and cdocsamp.tex with
% cdocsch1.tex, cdocsch2.tex, cdocsdrf.tex, cdocsfn1.tex, cdocsfn2.tex.
%
%<package>\ifdefined\childdocmain\endinput\fi
%<package>\ProvidesFile{childdoc.def}[2018/12/30 v2.0 child document driver]
%<samplemain>\ProvidesFile{cdocsamp.tex}[2018/12/30 v2.0 sample for childdoc]
%<*driver>
%\ProvidesFile{childdoc.drv}[2018/12/30 v2.0 childdoc reference manual file]
\PassOptionsToClass{10pt,a4paper}{article}
\documentclass{ltxdoc}

\usepackage[margin=35mm]{geometry}
\usepackage{hyperref}
\usepackage{hyperxmp}
\usepackage[usenames]{color}

\hypersetup{colorlinks=true}
\hypersetup{pdfstartview=FitH}
\hypersetup{pdfpagemode=UseNone}
\hypersetup{pdfsource={}}
\hypersetup{pdflang={en-UK}}
\hypersetup{pdfcopyright={Copyright 2017-2018 Niklas Beisert.
  This work may be distributed and/or modified under the
  conditions of the LaTeX Project Public License, either version 1.3
  of this license or (at your option) any later version.}}
\hypersetup{pdflicenseurl={http://www.latex-project.org/lppl.txt}}
\hypersetup{pdfcontactaddress={ETH Zurich, ITP, HIT K,
  Wolfgang-Pauli-Strasse 27}}
\hypersetup{pdfcontactpostcode={8093}}
\hypersetup{pdfcontactcity={Zurich}}
\hypersetup{pdfcontactcountry={Switzerland}}
\hypersetup{pdfcontactemail={nbeisert@itp.phys.ethz.ch}}
\hypersetup{pdfcontacturl={http://people.phys.ethz.ch/\xmptilde nbeisert/}}

\newcommand{\secref}[1]{\hyperref[#1]{section \ref*{#1}}}

\parskip1ex
\parindent0pt
\let\olditemize\itemize
\def\itemize{\olditemize\parskip0pt}

\begin{document}

\title{The \textsf{childdoc} Package}
\hypersetup{pdftitle={The childdoc Package}}
\author{Niklas Beisert\\[2ex]
  Institut f\"ur Theoretische Physik\\
  Eidgen\"ossische Technische Hochschule Z\"urich\\
  Wolfgang-Pauli-Strasse 27, 8093 Z\"urich, Switzerland\\[1ex]
  \href{mailto:nbeisert@itp.phys.ethz.ch}
  {\texttt{nbeisert@itp.phys.ethz.ch}}}
\hypersetup{pdfauthor={Niklas Beisert}}
\hypersetup{pdfsubject={Manual for the LaTeX2e Package childdoc}}
\date{30 December 2018, \textsf{v2.0}}
\maketitle

\begin{abstract}\noindent
\textsf{childdoc} is a \LaTeXe{} package
that enables the direct compilation
of document sections included by |\include|
to individual files.
\end{abstract}

\begingroup
\parskip0ex
\tableofcontents
\endgroup

%%%%%%%%%%%%%%%%%%%%%%%%%%%%%%%%%%%%%%%%%%%%%%%%%%%%%%%%%%%%%%%%%%%%%%%%%%%%%%%%
%%%%%%%%%%%%%%%%%%%%%%%%%%%%%%%%%%%%%%%%%%%%%%%%%%%%%%%%%%%%%%%%%%%%%%%%%%%%%%%%
\section{Introduction}

\LaTeX{} provides a mechanism to structure a large document (such as a book)
into a main file and several child files (containing the chapters)
using the |\include| command.
This mechanism is beneficial for documents
which span hundreds of pages in order to
make the source file(s) more manageable.
Moreover, compilation can be restricted to
selected child files by means of the |\includeonly| command.
The latter feature can be used to reduce the compilation time while editing
(this was significantly more useful in the earlier days of \LaTeX{})
or to generate a smaller document which is easier to navigate.
Another application of |\includeonly| is to generate
documents consisting of selected parts of the complete document.

However, there are a few drawbacks of the plain |\include| mechanism:
\begin{itemize}
\item
The child files cannot be compiled on their own,
they can only be compiled via the main file.
A naive editing environment
(such as a text editor with an option
to have the current file processed by \LaTeX)
may require one to switch to the main file before compiling;
attempting to compile the child file produces errors.
\item
The main file must be modified (each time)
to adjust the |\includeonly| command
to the present needs. This easily leaves the main file in a messy state.
\item
The generated document will always carry the filename
of the main document. This is inconvenient if
several child files are to be compiled and
to be kept for distribution.
\end{itemize}

The present package provides a simple interface
to make child files individually compilable by \LaTeX{}.
Compiling a child file then has the same effect as compiling
the main file with an |\includeonly| command
to select the appropriate child.
Moreover the generated document will carry the name of the child
rather than the main file.
This resolves all three above issues.

This feature is meant to make the editing of books,
thesis documents and lecture notes somewhat more convenient.
However, the package can also be used efficiently for
composing a series of documents (such as exercise sheets)
which are typically distributed individually.
It then assists the author in generating the individual documents
(potentially in different versions)
as well as a document containing the collected series.
Another application is in developing style files
or other kinds of included material
where compilation of the style file could redirect
to a sample or test file.

%%%%%%%%%%%%%%%%%%%%%%%%%%%%%%%%%%%%%%%%%%%%%%%%%%%%%%%%%%%%%%%%%%%%%%%%%%%%%%%%
%%%%%%%%%%%%%%%%%%%%%%%%%%%%%%%%%%%%%%%%%%%%%%%%%%%%%%%%%%%%%%%%%%%%%%%%%%%%%%%%
\section{Usage}

First of all, the package \textsf{childdoc} is \emph{not} a standard
\LaTeXe{} |.sty| style file! Therefore it needs to be invoked in
a non-standard way.

%%%%%%%%%%%%%%%%%%%%%%%%%%%%%%%%%%%%%%%%%%%%%%%%%%%%%%%%%%%%%%%%%%%%%%%%%%%%%%%%
\subsection{Included Files}
\label{sec:include}

%%%%%%%%%%%%%%%%%%%%%%%%%%%%%%%%%%%%%%%%
\DescribeMacro{\childdocmain}
To use the package, add the commands
\begin{center}
\begin{tabular}{l}
|\input{childdoc.def}|\\
|\childdocmain{}|\\
\end{tabular}
\end{center}
at the very top of the main \LaTeX{} file,
in particular \emph{before} the |\documentclass| statement!
The argument of |\childdocmain| should be left empty
(but it must be present).

%%%%%%%%%%%%%%%%%%%%%%%%%%%%%%%%%%%%%%%%
\DescribeMacro{\childdocof}
Furthermore, add the commands
\begin{center}
\begin{tabular}{l}
|\input{childdoc.def}|\\
|\childdocof{|\textit{main}|}|\\
\end{tabular}
\end{center}
at the top of every child file \textit{child}
which is included by |\include{|\textit{child}|}|
from within the main file
(or at least for those files to be compiled individually).
The argument \textit{main} must be the filename of the main file.

There are a couple of
considerations in setting up the main and child documents:

%%%%%%%%%%%%%%%%%%%%%%%%%%%%%%%%%%%%%%%%
\paragraph{Restrictions.}

Please note the following restrictions:
\begin{itemize}
\item
|\childdocmain| must be called with one argument \textit{main}
to ensure compatibility with earlier version of the package.
It must either be empty (|\childdocmain{}|)
or precisely match the filename of the main file in which it is specified.
See \secref{sec:detection} for further information.
\item
The filename \textit{main} must be specified without the |.tex| extension.
\item
The filename \textit{main} is case sensitive
(even in case-insensitive file systems)
due to internal string comparison.
\item
The argument \textit{main} should be fully expanded, it cannot be a macro.
\item
Subdirectories and special characters should be avoided in filenames.
\item
The command |\childdocmain{|\textit{main}|}| must be followed by a whitespace.
It should not be followed immediately by another command
or by a comment mark `|%|'.
This is because the \TeX{} parser reads the token immediately following
the argument of |\childdocmain| and puts it
at the beginning of every child section;
however, a white\-space is ignored.
\end{itemize}

%%%%%%%%%%%%%%%%%%%%%%%%%%%%%%%%%%%%%%%%
\paragraph{Content of Main File.}

It is advisable to place all content in the child files included by |\include|.
Any output contained in the main file will appear in all child documents
unless suppressed manually;
it cannot be suppressed automatically by the |\includeonly| directive
and thus should normally be avoided.
A method to include some content in the main file
by means of conditional processing is described in \secref{sec:conditional}.

%%%%%%%%%%%%%%%%%%%%%%%%%%%%%%%%%%%%%%%%
\paragraph{Page Numbering.}

When only a part of the document is compiled,
the appropriate numbering of pages
(as well as other status parameters)
is determined from the |.aux| files.
The latter contain information from previous passes.
However this information needs to propagate through
all intermediate child documents.
Therefore the page numbering in child documents may well
be inconsistent until the complete document is compiled at least once.

A useful (if unconventional) way to always ensure a consistent
page numbering is to restart the numbering in each child document
and denote the pages by `\textit{child}|.|\textit{page}'
where \textit{child} represents the chapter/section number of the child file.
This can be achieved by the command
|\numberwithin{page}{|\textit{child}|}|
of the \textsf{amsmath} package
where \textit{child} can be |chapter| or |section|
depending on the chosen structuring.
Alternatively, one can modify the macro |\thepage| appropriately
and reset the counter |page| at the start of each child file.

%%%%%%%%%%%%%%%%%%%%%%%%%%%%%%%%%%%%%%%%%%%%%%%%%%%%%%%%%%%%%%%%%%%%%%%%%%%%%%%%
\subsection{Conditional Processing}
\label{sec:conditional}

The package provides a mechanism to compile different versions
of a document. To customise the versions further some conditional processing
can come in handy to distinguish which version is being compiled.
The package provides two macros to describe the compilation context:

%%%%%%%%%%%%%%%%%%%%%%%%%%%%%%%%%%%%%%%%
\DescribeMacro{\ifchilddoc}
The conditional |\ifchilddoc| distinguishes between the compilation of
child documents and the main document:
%
\begin{center}
|\ifchilddoc |\textit{child-code}| |[|\||else |\textit{main-code}]| \||fi|
\end{center}

%%%%%%%%%%%%%%%%%%%%%%%%%%%%%%%%%%%%%%%%
\DescribeMacro{\childdocname}
\DescribeMacro{\childdocjob}
The macro |\childdocname| contains the filename (without extension)
of the main or child file being processed.
Note that |\childdocjob| will always contain the name of the main file.

%%%%%%%%%%%%%%%%%%%%%%%%%%%%%%%%%%%%%%%%
\paragraph{Title Page.}

Conditional processing can be used to include a title or banner page
in the main document when proper precautions are taken.
Importantly, the code in the main file should ensure that the page counter
(as well as other status parameters which are stored in the |.aux| files)
takes the same value after the conditional processing.
Otherwise the page numbers may take divergent values
depending on which part is compiled.

For example, a title page could be declared by:
%
\begin{center}
\begin{tabular}{l}
|\ifchilddoc\||else|\\
|\addtocounter{page}{-1}|\\
\textit{code for title page}\\
|\newpage|\\
|\||fi|
\end{tabular}
\end{center}
%
A banner page for the child documents can be generated by:
%
\begin{center}
\begin{tabular}{l}
|\ifchilddoc|\\
|\addtocounter{page}{-1}|\\
\textit{code for banner page}\\
|\newpage|\\
|\||fi|
\end{tabular}
\end{center}
%
Here one could write a message such as:
\begin{center}
|This is the part \childdocname{} of \childdocjob{}.|
\end{center}

%%%%%%%%%%%%%%%%%%%%%%%%%%%%%%%%%%%%%%%%%%%%%%%%%%%%%%%%%%%%%%%%%%%%%%%%%%%%%%%%
\subsection{Flags}
\label{sec:flags}

The package makes it easy to generate different versions
of the main or child documents.
To this end compilation flags can be defined
and assigned different default values.
They will be particularly useful in conjunction
with the forwarding mechanism described in \secref{sec:forward}.

For example, it may be useful to have a flag |\version|
which can be set to |draft| or |final|.
The document source will contain some conditional code
depending on the value of |\version|.
Suppose further, the flag should default to |final| for the main file
and to |draft| for child files
which is a natural assignment for editing the document.
This is achieved by placing the following code
in the preamble of the main document
(below the |\childdocmain| directive):
%
\begin{center}
\begin{tabular}{l}
|\ifchilddoc|\\
|\providecommand{\version}{draft}|\\
|\||else|\\
|\providecommand{\version}{final}|\\
|\||fi|
\end{tabular}
\end{center}
%
The definition by |\providecommand| makes sure
that previous definitions are not overwritten.
Further statements |\providecommand{\version}{...}|
can thus be added before the above code to override it.

For the main file, one might add a line
(between |\childdocmain| and the above block)
%
\begin{center}
|%\ifchilddoc\||else\providecommand{\version}{draft}\||fi|
\end{center}
%
which can be uncommented to produce a draft version.
Likewise one can add a line to the very top of a child file
(above the |\childdocof{|\textit{main}|}| directive)
%
\begin{center}
|%\providecommand{\version}{final}|
\end{center}
%
which can be uncommented to produce the final version of this child document.

%%%%%%%%%%%%%%%%%%%%%%%%%%%%%%%%%%%%%%%%%%%%%%%%%%%%%%%%%%%%%%%%%%%%%%%%%%%%%%%%
\subsection{Forwarding}
\label{sec:forward}

Different versions of the main or child documents
using compilation flags as described in \secref{sec:flags}
can be (permanently) stored in different files
for convenient compilation, viewing and distribution.
To this end, the package defines a command
to pass on compilation to a different file:

%%%%%%%%%%%%%%%%%%%%%%%%%%%%%%%%%%%%%%%%
\DescribeMacro{\childdocforward}
The command |\childdocforward| redirects processing to
another source file:
%
\begin{center}
\begin{tabular}{l}
|\input{childdoc.def}|\\
|\childdocforward[|\textit{main}|]{|\textit{dest}|}|\\
\end{tabular}
\end{center}
%
The argument \textit{dest} is the destination file
(without extension).
It should be the main file or one of the child files.
Note that further \textsf{childdoc} directives
such as |\childdocof| and |\childdocforward|
in the indicated file will be processed in this form.
The optional argument \textit{main}
passes on directly to the main file \textit{main}
while pretending to compile the child \textit{dest}.
This form behaves as if \textit{dest}
issues |\childdocof{|\textit{main}|}| right away,
and no further \textsf{childdoc} directives will be processed.

%%%%%%%%%%%%%%%%%%%%%%%%%%%%%%%%%%%%%%%%
\DescribeMacro{\...prefix}
In the alternative form |\childdocforwardprefix|,
%
\begin{center}
\begin{tabular}{l}
|\input{childdoc.def}|\\
|\childdocforwardprefix[|\textit{main}|]{|\textit{prefix}|}{|\textit{dest}|}|
\end{tabular}
\end{center}
%
the destination file is determined by a pattern
depending on the current file:
To make this work, the current file must be called
`{\textit{prefix}\hspace{0.2em}\textit{suffix}}'
with \textit{prefix} matching precisely the argument.
Processing is then passed on to the file
`{\textit{dest}\hspace{0.2em}\textit{suffix}}'.
Surely, the same effect is achieved by
directly specifying the
argument `{\textit{dest}\hspace{0.2em}\textit{suffix}}'
in the first form.
However, that requires to set up a different file
for each child. With the alternative form of the command
all these files can have exactly the same content
which simplifies setting them up and maintaining them.

For example, the following file |draft.tex|
with a compilation flag |\version| as described in \secref{sec:flags}
compiles the main document as a draft:
%
\begin{center}
\begin{tabular}{l}
|\def\version{draft}|\\
|\input{childdoc.def}|\\
|\childdocforward{|\textit{main}|}|
\end{tabular}
\end{center}
%
Likewise, the following files |final|\textit{nn}|.tex|
compile the final version of the child document
|child|\textit{nn}|.tex|:
%
\begin{center}
\begin{tabular}{l}
|\def\version{final}|\\
|\input{childdoc.def}|\\
|\childdocforwardprefix{final}{child}|
\end{tabular}
\end{center}
%

Note that when several versions of a main file and/or of each child file
are to be generated, it may be convenient to set up a |Makefile| or
shell script to automatise the process.

%%%%%%%%%%%%%%%%%%%%%%%%%%%%%%%%%%%%%%%%%%%%%%%%%%%%%%%%%%%%%%%%%%%%%%%%%%%%%%%%
\subsection{Command Line Processing}
\label{sec:commandline}

The effect of redirection files can also be achieved by invoking
the \LaTeX{} compiler with a more elaborate command line.
Most conveniently this should be done as part
of a shell script or a |Makefile|.

When using \textsf{childdoc} in the main file, the following
command lines effectively perform a redirection
(note that depending on the shell being used,
backslashes may have to be doubled: `|\|' $\to$ `|\\|'):
%
\begin{center}
|... -jobname "|\textit{target}|" |\\|"|[\textit{flags}]%
|\input{childdoc.def}\childdocforward[|\textit{main}|]{|\textit{dest}|}"|
\end{center}
%
Here \textit{target} is the name of the output file,
\textit{main} is the name of the main file
and \textit{dest} is the name of the main or child file to be processed
(all filenames without extensions).
The optional argument \textit{main} can be omitted
if \textit{main} matches \textit{dest}.
Optionally, compilation \textit{flags} can be defined via |\def| commands.
This command line makes the \TeX{} engine believe
it is compiling the file \textit{target}
whose content is specified as the latter parameter.
The provided code then forwards the processing to
\textit{main} or \textit{dest} as described in \secref{sec:forward}.

%%%%%%%%%%%%%%%%%%%%%%%%%%%%%%%%%%%%%%%%%%%%%%%%%%%%%%%%%%%%%%%%%%%%%%%%%%%%%%%%
\subsection{Include by Input}
\label{sec:input}

Including child documents by |\include| has some restrictions by design.
Most notably, the content of a child document always occupies
its own set of pages; pages cannot be shared between child documents.
Usually, this behaviour makes perfect sense
because each child document contain an essential part of the document.
However, in some situations it may be desirable to compose
a document from a collection of parts
without having mandatory page breaks between then.
For this case, the package
provides a mechanism to include parts
by |\input| which can also be processed individually.
However, by construction this mechanism
requires manual handling of the content to be output.

%%%%%%%%%%%%%%%%%%%%%%%%%%%%%%%%%%%%%%%%
\DescribeMacro{\ifchilddocmanual}
The main file should be prepared as usual, see \secref{sec:include}.
However, the document body must make a distinction
between processing of an individual part and of the main document, e.g.:
%
\begin{center}
\begin{tabular}{l}
|\ifchilddocmanual|\\
|\input{\childdocname}|\\
|\||else|\\
\textit{document body with }|\input{|\textit{part}|}|\\
|\||fi|
\end{tabular}
\end{center}
%
The conditional |\ifchilddocmanual| is true whenever
a part to be included by |\input| is being compiled,
and the name of the part is stored in |\childdocname|.

%%%%%%%%%%%%%%%%%%%%%%%%%%%%%%%%%%%%%%%%
\DescribeMacro{\childdocby}
Each part to be included by |\input| should start with:
%
\begin{center}
\begin{tabular}{l}
|\input{childdoc.def}|\\
|\childdocby{|\textit{main}|}|\\
\end{tabular}
\end{center}
%
The directive |\childdocby| is similar to |\childdocof|
described in \secref{sec:include},
but the subsequent selection of content must be done manually.
To that end, both |\ifchilddoc| and |\ifchilddocmanual|
will be true upon processing of a part,
and the name of the part is stored in |\childdocname|.
Note that |\jobname| will be set to the filename of the current part
so that each part receives an individual |.aux| file
that does not interfere with the |.aux| file(s) of the main document.
This behaviour can be altered by the alternative form
|\childdocby[*]{|\textit{main}|}| (with a non-empty optional argument)
which uses the |.aux| file of the main document
by setting |\jobname| to \textit{main}.

%%%%%%%%%%%%%%%%%%%%%%%%%%%%%%%%%%%%%%%%%%%%%%%%%%%%%%%%%%%%%%%%%%%%%%%%%%%%%%%%
\subsection{Driver Development}
\label{sec:driver}

The \textsf{childdoc} mechanism can also be use for the development
of definition files such as \LaTeX{} styles or classes.
This case differs from the above setup with multiple parts
included by |\include| in that no |\includeonly| should be invoked.
This can be achieved by starting the include file
(before |\ProvidesPackage|) with:
%
\begin{center}
\begin{tabular}{l}
|\input{childdoc.def}|\\
|\childdocforward{|\textit{main}|}|\\
\end{tabular}
\end{center}
%
or alternatively with:
%
\begin{center}
\begin{tabular}{l}
|\input{childdoc.def}|\\
|\childdocby{|\textit{main}|}|\\
\end{tabular}
\end{center}
%
Both forms have slightly different effects as described above.
The main file is prepared as usual, see \secref{sec:include}.

%%%%%%%%%%%%%%%%%%%%%%%%%%%%%%%%%%%%%%%%%%%%%%%%%%%%%%%%%%%%%%%%%%%%%%%%%%%%%%%%
\subsection{Legacy Detection}
\label{sec:detection}

The directive |\childdocmain| in the main file can detect
whether the complete document or merely a child is to be compiled
even without using the directive |\childdocof|.
This method is deprecated because it is less robust
and there is no compelling reason to use it;
it is merely provided for backward compatibility
and it may be removed in future versions.

If the detection mechanism is to be used,
it is mandatory to correctly specify
the filename of the main file as the argument of |\childdocmain|:
%
\begin{center}
\begin{tabular}{l}
|\input{childdoc.def}|\\
|\childdocmain{|\textit{main}|}|\\
\end{tabular}
\end{center}
%
If |\jobname| does not match the argument \textit{main} of |\childdocmain|,
it is assumed that |\jobname| points to the child file to be compiled.
When using |\childdocmain| with the main file specified as argument,
it suffices to start a child file
with just |\input{|\textit{main}|}|
without loading of the package and using |\childdocof|.
If instead all processing is done
with the appropriate \textsf{childdoc} directives,
the argument of \textit{main} of |\childdocmain| can be empty.

An alternative version of the command line processing described
in \secref{sec:commandline} using the detection mechanism reads:
%
\begin{center}
|... -jobname "|\textit{target}|" "|[\textit{flags}]%
[|\def\jobname{|\textit{dest}|}|]|\input{|\textit{main}|}"|
\end{center}

%%%%%%%%%%%%%%%%%%%%%%%%%%%%%%%%%%%%%%%%%%%%%%%%%%%%%%%%%%%%%%%%%%%%%%%%%%%%%%%%
\subsection{Manual Code}
\label{sec:manual}

In case one cannot be certain whether the definitions file |childdoc.def|
is installed on the target \TeX{} distribution
and one prefers not to ship it,
it is conceivable to paste a few relevant commands into the sources.

To that end, drop all statements |\input{childdoc.def}|
and perform the replacements as outlined below.
Instead of |\childdocmain{|\textit{main}|}| add the following code
to the top of the main file:
%
\begin{center}
\begin{tabular}{l}
|\||ifdefined\childdocname\endinput\||fi\newif\ifchilddoc|\\
|\edef\childdocname{\scantokens\expandafter{\jobname\noexpand}}|\\
|\def\childdocmain{|\textit{main}|}\||ifx\childdocmain\childdocname\||else|\\
|\childdoctrue\includeonly{\childdocname}\let\jobname\childdocmain\||fi|\\
\end{tabular}
\end{center}
%
Instead of |\childdocof{|\textit{main}|}| just include the main file
at the top of each child file:
%
\begin{center}
|\input{|\textit{main}|}|
\end{center}
%
A simple redirection |\childdocforward{|\textit{dest}|}| is achieved by:
%
\begin{center}
|\def\jobname{|\textit{dest}|}\input{\jobname}|
\end{center}
%
The redirection with prefix
|\childdocforwardprefix[|\textit{prefix}|]{|\textit{dest}|}|
is accomplished by:
%
\begin{center}
\begin{tabular}{l}
|{\edef\jobname{\scantokens\expandafter{\jobname\noexpand}}|\\
|\def\redirectjob |\textit{prefix}|#1~~~{\gdef\jobname{|\textit{dest}|#1}}|\\
|\expandafter\redirectjob\jobname~~~}\input{\jobname}|
\end{tabular}
\end{center}

In an alternative approach,
child documents can be compiled by a specific command line
without additional code or specific definitions:
%
\begin{center}
|... -jobname "|\textit{target}|" "|[\textit{flags}]%
|\includeonly{|\textit{dest}|}\input{|\textit{main}|}"|
\end{center}
%

%%%%%%%%%%%%%%%%%%%%%%%%%%%%%%%%%%%%%%%%%%%%%%%%%%%%%%%%%%%%%%%%%%%%%%%%%%%%%%%%
%%%%%%%%%%%%%%%%%%%%%%%%%%%%%%%%%%%%%%%%%%%%%%%%%%%%%%%%%%%%%%%%%%%%%%%%%%%%%%%%
\section{Information}

%%%%%%%%%%%%%%%%%%%%%%%%%%%%%%%%%%%%%%%%%%%%%%%%%%%%%%%%%%%%%%%%%%%%%%%%%%%%%%%%
\subsection{Copyright}

Copyright \copyright{} 2017--2018 Niklas Beisert

This work may be distributed and/or modified under the
conditions of the \LaTeX{} Project Public License, either version 1.3
of this license or (at your option) any later version.
The latest version of this license is in
  \url{http://www.latex-project.org/lppl.txt}
and version 1.3 or later is part of all distributions of \LaTeX{}
version 2005/12/01 or later.

This work has the LPPL maintenance status `maintained'.

The Current Maintainer of this work is Niklas Beisert.

This work consists of the files |README.txt|, |childdoc.ins| and |childdoc.dtx|
as well as the derived files |childdoc.def|, |cdocsamp.tex|
with |cdocsch1.tex|, |cdocsch2.tex|, |cdocspt3.tex|, |cdocspt4.tex|,
|cdocsdrf.tex|, |cdocsfn1.tex|, |cdocsfn2.tex|
as well as |childdoc.pdf|.

%%%%%%%%%%%%%%%%%%%%%%%%%%%%%%%%%%%%%%%%%%%%%%%%%%%%%%%%%%%%%%%%%%%%%%%%%%%%%%%%
\subsection{Files and Installation}

The package consists of the files:
%
\begin{center}
\begin{tabular}{ll}
    |README.txt|   & readme file \\
    |childdoc.ins| & installation file \\
    |childdoc.dtx| & source file \\
    |childdoc.def| & definition file \\
    |cdocsamp.tex| & sample main file \\
    |cdocsch1.tex| & sample include file \\
    |cdocsch2.tex| & sample include file \\
    |cdocspt3.tex| & sample part file \\
    |cdocspt4.tex| & sample part file \\
    |cdocsdrf.tex| & sample redirection file \\
    |cdocsfn1.tex| & sample redirection file \\
    |cdocsfn2.tex| & sample redirection file \\
    |childdoc.pdf| & manual
\end{tabular}
\end{center}
%
The distribution consists of the files
|README.txt|, |childdoc.ins| and |childdoc.dtx|.
%
\begin{itemize}
\item
Run (pdf)\LaTeX{} on |childdoc.dtx|
to compile the manual |childdoc.pdf| (this file).
\item
Run \LaTeX{} on |childdoc.ins| to create the definitions file |childdoc.def|
and the sample |cdocsamp.tex| with include files
|cdocsch1.tex|, |cdocsch2.tex|, |cdocspt3.tex|, |cdocspt4.tex|,
|cdocsdrf.tex|, |cdocsfn1.tex|, |cdocsfn2.tex|.
Then copy the file |childdoc.def| to an appropriate directory of your \LaTeX{}
distribution, e.g.\ \textit{texmf-root}|/tex/latex/childdoc|.
\end{itemize}

%%%%%%%%%%%%%%%%%%%%%%%%%%%%%%%%%%%%%%%%%%%%%%%%%%%%%%%%%%%%%%%%%%%%%%%%%%%%%%%%
\subsection{Related CTAN Packages}

There are several other packages which offer a similar functionality:
%
\begin{itemize}
\item
The packages
\href{http://ctan.org/pkg/docmute}{\textsf{docmute}},
\href{http://ctan.org/pkg/includex}{\textsf{includex}} and
\href{http://ctan.org/pkg/standalone}{\textsf{standalone}}
provide commands to include only the document body of
a child file thus allowing both files to be compiled individually.
\item
The packages \href{http://ctan.org/pkg/subdocs}{\textsf{subdocs}}
and \href{http://ctan.org/pkg/subfiles}{\textsf{subfiles}}
provide structures in which the main and child documents can be
encapsulated and allowing them to be compiled individually.
The inclusion mechanism is different from the conventional |\include|.
\item
The package \href{http://ctan.org/pkg/combine}{\textsf{combine}}
is an elaborate solution to combine several documents into one.
\end{itemize}
%
See also the CTAN topic \href{http://ctan.org/topic/subdocs}{\textsf{subdocs}}
for further related packages.
The present package differs from the above solutions in that
a document structure constructed with the conventional |\include| mechanism
just needs two extra commands at the top of every file
such that all constituent files can be compiled individually.

%%%%%%%%%%%%%%%%%%%%%%%%%%%%%%%%%%%%%%%%%%%%%%%%%%%%%%%%%%%%%%%%%%%%%%%%%%%%%%%%
%\subsection{Feature Suggestions}
%
%The following is a list of features which may be useful for future
%versions of this package:
%%
%\begin{itemize}
%\item
%\ldots
%\end{itemize}

%%%%%%%%%%%%%%%%%%%%%%%%%%%%%%%%%%%%%%%%%%%%%%%%%%%%%%%%%%%%%%%%%%%%%%%%%%%%%%%%
\subsection{Revision History}

%%%%%%%%%%%%%%%%%%%%%%%%%%%%%%%%%%%%%%%%
\paragraph{v2.0:} 2018/12/30

\begin{itemize}
\item
immediate forward processing
\item
added |\childdocby| mechanism
\item
manual restructured
\end{itemize}

%%%%%%%%%%%%%%%%%%%%%%%%%%%%%%%%%%%%%%%%
\paragraph{v1.6:} 2018/01/17

\begin{itemize}
\item
application for development of include files
\item
corrections to manual
\end{itemize}

%%%%%%%%%%%%%%%%%%%%%%%%%%%%%%%%%%%%%%%%
\paragraph{v1.5:} 2017/05/21

\begin{itemize}
\item
more complete structuring introduced
\item
|\childdocof| introduced
\item
|\childdoc| renamed to |\childdocmain|
\item
|\childredirect| renamed to |\childdocforward| and |\childdocforwardprefix|
and functionality expanded
\end{itemize}

%%%%%%%%%%%%%%%%%%%%%%%%%%%%%%%%%%%%%%%%
\paragraph{v1.0:} 2017/04/27

\begin{itemize}
\item
manual and install package
\item
first version published on CTAN
\end{itemize}

%%%%%%%%%%%%%%%%%%%%%%%%%%%%%%%%%%%%%%%%
\paragraph{v0.6:} 2017/04/26

\begin{itemize}
\item
redirection mechanism added
\end{itemize}

%%%%%%%%%%%%%%%%%%%%%%%%%%%%%%%%%%%%%%%%
\paragraph{v0.5:} 2017/04/26

\begin{itemize}
\item
functionality in definition file
\end{itemize}


%%%%%%%%%%%%%%%%%%%%%%%%%%%%%%%%%%%%%%%%%%%%%%%%%%%%%%%%%%%%%%%%%%%%%%%%%%%%%%%%
%%%%%%%%%%%%%%%%%%%%%%%%%%%%%%%%%%%%%%%%%%%%%%%%%%%%%%%%%%%%%%%%%%%%%%%%%%%%%%%%
%%%%%%%%%%%%%%%%%%%%%%%%%%%%%%%%%%%%%%%%%%%%%%%%%%%%%%%%%%%%%%%%%%%%%%%%%%%%%%%%
\appendix

\settowidth\MacroIndent{\rmfamily\scriptsize 000\ }

 \DocInput{childdoc.dtx}

\end{document}
%</driver>
% \fi
%
% %%%%%%%%%%%%%%%%%%%%%%%%%%%%%%%%%%%%%%%%%%%%%%%%%%%%%%%%%%%%%%%%%%%%%%%%%%%%%%
% %%%%%%%%%%%%%%%%%%%%%%%%%%%%%%%%%%%%%%%%%%%%%%%%%%%%%%%%%%%%%%%%%%%%%%%%%%%%%%
% \section{Sample}
%\iffalse
%<*samplemain>
%\fi
%
% The following presents a sample document
% with two chapters, two parts, a title page,
% a compile flag as well as three forwarding files to set the flag.
% It consists of eight |.tex| files:
% \begin{center}
% \begin{tabular}{ll}
% |cdocsamp.tex|&main file\\
% |cdocsch1.tex|&include file for chapter 1\\
% |cdocsch2.tex|&include file for chapter 2\\
% |cdocspt3.tex|&include file for part 3\\
% |cdocspt4.tex|&include file for part 4\\
% |cdocsdrf.tex|&forwarding file for main file in draft mode\\
% |cdocsfi1.tex|&forwarding file for final version of chapter 1\\
% |cdocsfi2.tex|&forwarding file for final version of chapter 2\\
% \end{tabular}
% \end{center}
% Each of the eight files can be compiled directly by the \LaTeX{} compiler.
%
% %%%%%%%%%%%%%%%%%%%%%%%%%%%%%%%%%%%%%%
% \paragraph{Main File.}
%
% The main file is called |cdocsamp.tex|.
%
% Load the \textsf{childdoc} definitions and
% declare the filename for the main document:
%    \begin{macrocode}
\input{childdoc.def}
\childdocmain{}
%    \end{macrocode}

% Optional override for |\version| flag:
%    \begin{macrocode}
%%\ifchilddoc\else\providecommand{\version}{draft}\fi
%    \end{macrocode}

% Define the default values for the |\version| flag
% (|final| for the main file and |draft| for childs):
%    \begin{macrocode}
\ifchilddoc
\providecommand{\version}{draft}
\else
\providecommand{\version}{final}
\fi
%    \end{macrocode}

% Load the standard document class:
%    \begin{macrocode}
\documentclass[12pt]{article}
%    \end{macrocode}

% Start the document body:
%    \begin{macrocode}
\begin{document}
%    \end{macrocode}

% Declare a title page.
% Print title, part of document being processed and version flag:
%    \begin{macrocode}
\addtocounter{page}{-1}
\begin{center}
{\LARGE\bfseries{}childdoc example\par}
\vspace{1cm}
\ifchilddoc
\ifchilddocmanual part\else chapter\fi:
`\childdocname' of `\childdocjob'\par
\else
main document: `\childdocjob'\par
\fi
version: \version\par
\end{center}
\newpage
%    \end{macrocode}

% Manually include selected file,
% otherwise process as usual:
%    \begin{macrocode}
\ifchilddocmanual
\section*{part `\childdocname'}
\input{\childdocname}
\else
%    \end{macrocode}

% Include the two chapters:
%    \begin{macrocode}
\include{cdocsch1}
\include{cdocsch2}
%    \end{macrocode}

% Include the two parts unless only chapters should be displayed:
%    \begin{macrocode}
\ifchilddoc\else
\section{part three}
\input{cdocspt3}
\section{part four}
\input{cdocspt4}
\fi
%    \end{macrocode}

% Process as usual until here:
%    \begin{macrocode}
\fi
%    \end{macrocode}

% End of document body:
%    \begin{macrocode}
\end{document}
%    \end{macrocode}
%\iffalse
%</samplemain>
%\fi
%
% %%%%%%%%%%%%%%%%%%%%%%%%%%%%%%%%%%%%%%
% \paragraph{Chapter Include Files.}
%
% The include files are called |cdocsch1.tex| and |cdocsch2.tex|.
%
%\iffalse
%<*samplechap1|samplechap2>
%\fi

% Optional override for |\version| flag:
%    \begin{macrocode}
%%\providecommand{\version}{final}
%    \end{macrocode}

% Include the main document:
%    \begin{macrocode}
\input{childdoc.def}
\childdocof{cdocsamp}
%    \end{macrocode}

%\iffalse
%</samplechap1|samplechap2>
%\fi
%
%\iffalse
%<*samplechap1>
%\fi
% Some text for chapter 1:
%    \begin{macrocode}
\section{one}
some text in chapter one
%    \end{macrocode}

%\iffalse
%</samplechap1>
%\fi
% Some text for chapter 2:
%\iffalse
%<*samplechap2>
%\fi
%    \begin{macrocode}
\section{two}
more text in chapter two
%    \end{macrocode}

%\iffalse
%</samplechap2>
%\fi
%
% %%%%%%%%%%%%%%%%%%%%%%%%%%%%%%%%%%%%%%
% \paragraph{Part Include Files.}
%
% The include files are called |cdocspt3.tex| and |cdocspt4.tex|.
%
%\iffalse
%<*samplepart3|samplepart4>
%\fi

% Optional override for |\version| flag:
%    \begin{macrocode}
%%\providecommand{\version}{final}
%    \end{macrocode}

% Include the main document:
%    \begin{macrocode}
\input{childdoc.def}
\childdocby{cdocsamp}
%    \end{macrocode}

%\iffalse
%</samplepart3|samplepart4>
%\fi
%
%\iffalse
%<*samplepart3>
%\fi
% Some text for part 3:
%    \begin{macrocode}
some text in part three
%    \end{macrocode}

%\iffalse
%</samplepart3>
%\fi
% Some text for part 4:
%\iffalse
%<*samplepart4>
%\fi
%    \begin{macrocode}
more text in part four
%    \end{macrocode}

%\iffalse
%</samplepart4>
%\fi
%
% %%%%%%%%%%%%%%%%%%%%%%%%%%%%%%%%%%%%%%
% \paragraph{Forwarding for a Complete Draft.}
%
% The following forwarding file |cdocsdrf.tex|
% compiles the main document in draft mode:
%\iffalse
%<*sampledraft>
%\fi
%    \begin{macrocode}
\def\version{draft}
\input{childdoc.def}
\childdocforward{cdocsamp}
%    \end{macrocode}

%\iffalse
%</sampledraft>
%\fi
%
% %%%%%%%%%%%%%%%%%%%%%%%%%%%%%%%%%%%%%%
% \paragraph{Forwarding for Final Version of the Chapters.}
%
% The following forwarding files |cdocsfn1.tex| and |cdocsfn2.tex|
% (with identical content)
% compile the final versions of the child documents
% |cdocsch1.tex| and |cdocsch2.tex|, respectively:
%\iffalse
%<*samplefinal>
%\fi
%    \begin{macrocode}
\def\version{final}
\input{childdoc.def}
\childdocforwardprefix[cdocsamp]{cdocsfn}{cdocsch}
%    \end{macrocode}

%\iffalse
%</samplefinal>
%\fi
%
% %%%%%%%%%%%%%%%%%%%%%%%%%%%%%%%%%%%%%%
% \paragraph{Command Line Processing.}
%
% The following three command lines generate the output files
% |cdocscld|, |cdocscl1| and |cdocscl2|
% which should be identical to
% |cdocsdrf|, |cdocsch1| and |cdocsfn2|, respectively:
% \begin{center}
% \begin{tabular}{l}
% |latex -jobname cdocscld \|\\
% |  "\def\version{draft}\input{childdoc.def}\childdocforward{cdocsamp}"|\\
% |latex -jobname cdocscl1 \|\\
% |  "\input{childdoc.def}\childdocforward[cdocsamp]{cdocsch1}"|\\
% |latex -jobname cdocscl2 \|\\
% |  "\def\version{final}\input{childdoc.def}\childdocforward{cdocsch2}"|
% \end{tabular}
% \end{center}
% Note that the trailing backslash on each first line
% merely continues the input to the second line
% (for convenient cut ant paste).
% Furthermore, the command |latex| can be replaced by any
% of its alternative versions such as |pdflatex|.
%
% %%%%%%%%%%%%%%%%%%%%%%%%%%%%%%%%%%%%%%%%%%%%%%%%%%%%%%%%%%%%%%%%%%%%%%%%%%%%%%
% %%%%%%%%%%%%%%%%%%%%%%%%%%%%%%%%%%%%%%%%%%%%%%%%%%%%%%%%%%%%%%%%%%%%%%%%%%%%%%
% \section{Implementation}
%\iffalse
%<*package>
%\fi
%
% This section describes the definitions file |childdoc.def|.

% The definitions cannot be loaded using |\usepackage| or |\RequirePackage|
% which has a mechanism to prevent loading a style file more than once.
% When loading the definitions by means of |\input|
% multiple instances have to be prevented manually:
%\iffalse
%This code needs to be before the `\ProvidesFile' directive
%which is defined at the beginning of this file.
%Therefore it is also placed there and commented out here.
%</package>
%<*discard>
%\fi
%    \begin{macrocode}
\ifdefined\childdocmain\endinput\fi
%    \end{macrocode}
%\iffalse
%</discard>
%<*package>
%\fi
%
% \macro{\ifchilddoc}
% \macro{\ifchilddocmanual}
% The conditional |\ifchilddoc| tells whether a
% child (true) or main (false) document is being compiled.
% The conditional |\ifchilddocmanual| tells whether
% the |\includeonly| mechanism is used (false) or
% the selection of child files must be performed manually (true).
% The definitions initialise to false:
%    \begin{macrocode}
\newif\ifchilddoc
\newif\ifchilddocmanual
%    \end{macrocode}

% \macro{\childdocname}
% \macro{\childdocjob}
% The macro |\childdocname| stores the name of the main document
% to be compiled. The macro |\childdocjob| stores the name of
% the document on which the \LaTeX{} compiler was originally invoked.
% The content of |\jobname| cannot be compared
% to filenames specified in the source due to different catcodes.
% The following code rescans |\jobname|, stores the result
% in |\childdocname| and saves a copy in |\childdocjob|:
%    \begin{macrocode}
\edef\childdocname{\scantokens\expandafter{\jobname\noexpand}}
\let\childdocjob\childdocname
%    \end{macrocode}

% \macro{\childdocdisable}
% The macro |\childdocdisable| prevents the main file
% from being processed more than once.
% At this stage, the main document command |\childdocmain|
% is assumed to be called once again where it should do nothing.
% Any subsequent call to it should prevent
% a secondary processing of the main document
% It overwrites the forwarding commands
% |\childdocof| and |\childdocforward|
% with empty macros to prevent further inclusions of the main document:
%    \begin{macrocode}
\newcommand{\childdocdisable}
{
  \renewcommand{\childdocmain}[1]{\renewcommand{\childdocmain}[1]{\endinput}}
  \renewcommand{\childdocof}[1]{}
  \renewcommand{\childdocby}[2][]{}
  \renewcommand{\childdocforward}[2][]{}
  \renewcommand{\childdocdisable}{}
}
%    \end{macrocode}

% \macro{\childdocmain}
% The macro |\childdocmain| is to be called at the top of the main file
% with nothing or the main filename (without extension) as argument.
% First, it breaks loops.
% If the argument is not empty and does not match |\childdocname|
% (which is set by the first inclusion of |childdoc.def|),
% |\ifchilddoc| is set to true, |\includeonly| is applied to the child file
% and |\jobname| is set to the main file
% (for proper handling of |.aux| files):
%    \begin{macrocode}
\newcommand{\childdocmain}[1]
{
  \childdocdisable\childdocmain{}
  \if?#1?\else
    \begingroup
      \def\childdoctmp{#1}
      \ifx\childdoctmp\childdocname
        \def\childdoctmp{}
      \else
        \def\childdoctmp
        {
          \childdoctrue
          \includeonly{\childdocname}
          \def\childdocjob{#1}
          \def\jobname{#1}
        }
      \fi
      \expandafter
    \endgroup
    \childdoctmp
  \fi
}
%    \end{macrocode}

% \macro{\childdocof}
% The command |\childdocof| redirects
% compilation to the main file |#1|.
%    \begin{macrocode}
\newcommand{\childdocof}[1]
{
  \childdocdisable
  \childdoctrue
  \includeonly{\childdocname}
  \def\jobname{#1}
  \def\childdocjob{#1}
  \input{#1}
}
%    \end{macrocode}

% \macro{\childdocby}
% The command |\childdocby| ....
%    \begin{macrocode}
\newcommand{\childdocby}[2][]
{
  \childdocdisable
  \childdoctrue
  \childdocmanualtrue
  \if?#1?\else
    \def\jobname{#2}
  \fi
  \def\childdocjob{#2}
  \input{#2}
  \endinput
}
%    \end{macrocode}

% \macro{\childdocforward}
% The command |\childdocforward| redirects
% compilation to the main file or
% (if the optional argument is given) a child file.
% Parameters are set as if the main file
% or a child file starting with |\childdocof| was compiled.
% Then compilation is handed over to the main file:
%    \begin{macrocode}
\newcommand{\childdocforward}[2][]
{
  \begingroup
    \if?#1?
      \def\childdoctmp
      {
        \def\childdocname{#2}
        \def\childdocjob{#2}
        \def\jobname{#2}
        \input{#2}
        \endinput
      }
    \else
      \def\childdoctmp
      {
        \childdocdisable
        \def\childdocname{#2}
        \childdoctrue
        \includeonly{#2}
        \def\childdocjob{#1}
        \def\jobname{#1}
        \input{#1}
        \endinput
      }
    \fi
    \expandafter
  \endgroup
  \childdoctmp
}
%    \end{macrocode}

% \macro{\childdocforwardprefix}
% The command |\childdocforwardprefix| redirects
% compilation to the main or a child file by means of a pattern.
% The prefix |#1| in the current filename is replaced by |#2|
% and the suffix of the current filename is kept
% (it is assumed that the filename does not contain the substring `|~~~|'
% which is used as a delimiter).
% Compilation is handed over to the new file by |\childdocforward|:
%    \begin{macrocode}
\newcommand{\childdocforwardprefix}[3][]
{
  \begingroup
    \def\childdocextract #2##1~~~{\def\childdoctmp{\childdocforward[#1]{#3##1}}}
    \expandafter\childdocextract\childdocname~~~
    \expandafter
  \endgroup
  \childdoctmp
}
%    \end{macrocode}

% \macro{\childdoc}
% The deprecated macro |\childdoc| is a legacy version of |\childdocmain|:
%    \begin{macrocode}
\newcommand{\childdoc}{\childdocmain}
%    \end{macrocode}

% \macro{\childdocredirect}
% The deprecated macro |\childdocredirect| is a legacy version
% of |\childdocforward| and |\childdocforwardprefix|:
%    \begin{macrocode}
\newcommand{\childdocredirect}[2][]
{
  \begingroup
    \if?#1?
      \def\childdoctmp{\childdocforward{#2}}
    \else
      \def\childdoctmp{\childdocforwardprefix{#1}{#2}}
    \fi
    \expandafter
  \endgroup
  \childdoctmp
}
%    \end{macrocode}

%\iffalse
%</package>
%\fi
%
\endinput
|\\
|\childdocforwardprefix[|\textit{main}|]{|\textit{prefix}|}{|\textit{dest}|}|
\end{tabular}
\end{center}
%
the destination file is determined by a pattern
depending on the current file:
To make this work, the current file must be called
`{\textit{prefix}\hspace{0.2em}\textit{suffix}}'
with \textit{prefix} matching precisely the argument.
Processing is then passed on to the file
`{\textit{dest}\hspace{0.2em}\textit{suffix}}'.
Surely, the same effect is achieved by
directly specifying the
argument `{\textit{dest}\hspace{0.2em}\textit{suffix}}'
in the first form.
However, that requires to set up a different file
for each child. With the alternative form of the command
all these files can have exactly the same content
which simplifies setting them up and maintaining them.

For example, the following file |draft.tex|
with a compilation flag |\version| as described in \secref{sec:flags}
compiles the main document as a draft:
%
\begin{center}
\begin{tabular}{l}
|\def\version{draft}|\\
|% \iffalse
%
% childdoc.dtx Copyright (C) 2017-2018 Niklas Beisert
%
% This work may be distributed and/or modified under the
% conditions of the LaTeX Project Public License, either version 1.3
% of this license or (at your option) any later version.
% The latest version of this license is in
%   http://www.latex-project.org/lppl.txt
% and version 1.3 or later is part of all distributions of LaTeX
% version 2005/12/01 or later.
%
% This work has the LPPL maintenance status `maintained'.
%
% The Current Maintainer of this work is Niklas Beisert.
%
% This work consists of the files childdoc.dtx and childdoc.ins
% and the derived files childdoc.def and cdocsamp.tex with
% cdocsch1.tex, cdocsch2.tex, cdocsdrf.tex, cdocsfn1.tex, cdocsfn2.tex.
%
%<package>\ifdefined\childdocmain\endinput\fi
%<package>\ProvidesFile{childdoc.def}[2018/12/30 v2.0 child document driver]
%<samplemain>\ProvidesFile{cdocsamp.tex}[2018/12/30 v2.0 sample for childdoc]
%<*driver>
%\ProvidesFile{childdoc.drv}[2018/12/30 v2.0 childdoc reference manual file]
\PassOptionsToClass{10pt,a4paper}{article}
\documentclass{ltxdoc}

\usepackage[margin=35mm]{geometry}
\usepackage{hyperref}
\usepackage{hyperxmp}
\usepackage[usenames]{color}

\hypersetup{colorlinks=true}
\hypersetup{pdfstartview=FitH}
\hypersetup{pdfpagemode=UseNone}
\hypersetup{pdfsource={}}
\hypersetup{pdflang={en-UK}}
\hypersetup{pdfcopyright={Copyright 2017-2018 Niklas Beisert.
  This work may be distributed and/or modified under the
  conditions of the LaTeX Project Public License, either version 1.3
  of this license or (at your option) any later version.}}
\hypersetup{pdflicenseurl={http://www.latex-project.org/lppl.txt}}
\hypersetup{pdfcontactaddress={ETH Zurich, ITP, HIT K,
  Wolfgang-Pauli-Strasse 27}}
\hypersetup{pdfcontactpostcode={8093}}
\hypersetup{pdfcontactcity={Zurich}}
\hypersetup{pdfcontactcountry={Switzerland}}
\hypersetup{pdfcontactemail={nbeisert@itp.phys.ethz.ch}}
\hypersetup{pdfcontacturl={http://people.phys.ethz.ch/\xmptilde nbeisert/}}

\newcommand{\secref}[1]{\hyperref[#1]{section \ref*{#1}}}

\parskip1ex
\parindent0pt
\let\olditemize\itemize
\def\itemize{\olditemize\parskip0pt}

\begin{document}

\title{The \textsf{childdoc} Package}
\hypersetup{pdftitle={The childdoc Package}}
\author{Niklas Beisert\\[2ex]
  Institut f\"ur Theoretische Physik\\
  Eidgen\"ossische Technische Hochschule Z\"urich\\
  Wolfgang-Pauli-Strasse 27, 8093 Z\"urich, Switzerland\\[1ex]
  \href{mailto:nbeisert@itp.phys.ethz.ch}
  {\texttt{nbeisert@itp.phys.ethz.ch}}}
\hypersetup{pdfauthor={Niklas Beisert}}
\hypersetup{pdfsubject={Manual for the LaTeX2e Package childdoc}}
\date{30 December 2018, \textsf{v2.0}}
\maketitle

\begin{abstract}\noindent
\textsf{childdoc} is a \LaTeXe{} package
that enables the direct compilation
of document sections included by |\include|
to individual files.
\end{abstract}

\begingroup
\parskip0ex
\tableofcontents
\endgroup

%%%%%%%%%%%%%%%%%%%%%%%%%%%%%%%%%%%%%%%%%%%%%%%%%%%%%%%%%%%%%%%%%%%%%%%%%%%%%%%%
%%%%%%%%%%%%%%%%%%%%%%%%%%%%%%%%%%%%%%%%%%%%%%%%%%%%%%%%%%%%%%%%%%%%%%%%%%%%%%%%
\section{Introduction}

\LaTeX{} provides a mechanism to structure a large document (such as a book)
into a main file and several child files (containing the chapters)
using the |\include| command.
This mechanism is beneficial for documents
which span hundreds of pages in order to
make the source file(s) more manageable.
Moreover, compilation can be restricted to
selected child files by means of the |\includeonly| command.
The latter feature can be used to reduce the compilation time while editing
(this was significantly more useful in the earlier days of \LaTeX{})
or to generate a smaller document which is easier to navigate.
Another application of |\includeonly| is to generate
documents consisting of selected parts of the complete document.

However, there are a few drawbacks of the plain |\include| mechanism:
\begin{itemize}
\item
The child files cannot be compiled on their own,
they can only be compiled via the main file.
A naive editing environment
(such as a text editor with an option
to have the current file processed by \LaTeX)
may require one to switch to the main file before compiling;
attempting to compile the child file produces errors.
\item
The main file must be modified (each time)
to adjust the |\includeonly| command
to the present needs. This easily leaves the main file in a messy state.
\item
The generated document will always carry the filename
of the main document. This is inconvenient if
several child files are to be compiled and
to be kept for distribution.
\end{itemize}

The present package provides a simple interface
to make child files individually compilable by \LaTeX{}.
Compiling a child file then has the same effect as compiling
the main file with an |\includeonly| command
to select the appropriate child.
Moreover the generated document will carry the name of the child
rather than the main file.
This resolves all three above issues.

This feature is meant to make the editing of books,
thesis documents and lecture notes somewhat more convenient.
However, the package can also be used efficiently for
composing a series of documents (such as exercise sheets)
which are typically distributed individually.
It then assists the author in generating the individual documents
(potentially in different versions)
as well as a document containing the collected series.
Another application is in developing style files
or other kinds of included material
where compilation of the style file could redirect
to a sample or test file.

%%%%%%%%%%%%%%%%%%%%%%%%%%%%%%%%%%%%%%%%%%%%%%%%%%%%%%%%%%%%%%%%%%%%%%%%%%%%%%%%
%%%%%%%%%%%%%%%%%%%%%%%%%%%%%%%%%%%%%%%%%%%%%%%%%%%%%%%%%%%%%%%%%%%%%%%%%%%%%%%%
\section{Usage}

First of all, the package \textsf{childdoc} is \emph{not} a standard
\LaTeXe{} |.sty| style file! Therefore it needs to be invoked in
a non-standard way.

%%%%%%%%%%%%%%%%%%%%%%%%%%%%%%%%%%%%%%%%%%%%%%%%%%%%%%%%%%%%%%%%%%%%%%%%%%%%%%%%
\subsection{Included Files}
\label{sec:include}

%%%%%%%%%%%%%%%%%%%%%%%%%%%%%%%%%%%%%%%%
\DescribeMacro{\childdocmain}
To use the package, add the commands
\begin{center}
\begin{tabular}{l}
|\input{childdoc.def}|\\
|\childdocmain{}|\\
\end{tabular}
\end{center}
at the very top of the main \LaTeX{} file,
in particular \emph{before} the |\documentclass| statement!
The argument of |\childdocmain| should be left empty
(but it must be present).

%%%%%%%%%%%%%%%%%%%%%%%%%%%%%%%%%%%%%%%%
\DescribeMacro{\childdocof}
Furthermore, add the commands
\begin{center}
\begin{tabular}{l}
|\input{childdoc.def}|\\
|\childdocof{|\textit{main}|}|\\
\end{tabular}
\end{center}
at the top of every child file \textit{child}
which is included by |\include{|\textit{child}|}|
from within the main file
(or at least for those files to be compiled individually).
The argument \textit{main} must be the filename of the main file.

There are a couple of
considerations in setting up the main and child documents:

%%%%%%%%%%%%%%%%%%%%%%%%%%%%%%%%%%%%%%%%
\paragraph{Restrictions.}

Please note the following restrictions:
\begin{itemize}
\item
|\childdocmain| must be called with one argument \textit{main}
to ensure compatibility with earlier version of the package.
It must either be empty (|\childdocmain{}|)
or precisely match the filename of the main file in which it is specified.
See \secref{sec:detection} for further information.
\item
The filename \textit{main} must be specified without the |.tex| extension.
\item
The filename \textit{main} is case sensitive
(even in case-insensitive file systems)
due to internal string comparison.
\item
The argument \textit{main} should be fully expanded, it cannot be a macro.
\item
Subdirectories and special characters should be avoided in filenames.
\item
The command |\childdocmain{|\textit{main}|}| must be followed by a whitespace.
It should not be followed immediately by another command
or by a comment mark `|%|'.
This is because the \TeX{} parser reads the token immediately following
the argument of |\childdocmain| and puts it
at the beginning of every child section;
however, a white\-space is ignored.
\end{itemize}

%%%%%%%%%%%%%%%%%%%%%%%%%%%%%%%%%%%%%%%%
\paragraph{Content of Main File.}

It is advisable to place all content in the child files included by |\include|.
Any output contained in the main file will appear in all child documents
unless suppressed manually;
it cannot be suppressed automatically by the |\includeonly| directive
and thus should normally be avoided.
A method to include some content in the main file
by means of conditional processing is described in \secref{sec:conditional}.

%%%%%%%%%%%%%%%%%%%%%%%%%%%%%%%%%%%%%%%%
\paragraph{Page Numbering.}

When only a part of the document is compiled,
the appropriate numbering of pages
(as well as other status parameters)
is determined from the |.aux| files.
The latter contain information from previous passes.
However this information needs to propagate through
all intermediate child documents.
Therefore the page numbering in child documents may well
be inconsistent until the complete document is compiled at least once.

A useful (if unconventional) way to always ensure a consistent
page numbering is to restart the numbering in each child document
and denote the pages by `\textit{child}|.|\textit{page}'
where \textit{child} represents the chapter/section number of the child file.
This can be achieved by the command
|\numberwithin{page}{|\textit{child}|}|
of the \textsf{amsmath} package
where \textit{child} can be |chapter| or |section|
depending on the chosen structuring.
Alternatively, one can modify the macro |\thepage| appropriately
and reset the counter |page| at the start of each child file.

%%%%%%%%%%%%%%%%%%%%%%%%%%%%%%%%%%%%%%%%%%%%%%%%%%%%%%%%%%%%%%%%%%%%%%%%%%%%%%%%
\subsection{Conditional Processing}
\label{sec:conditional}

The package provides a mechanism to compile different versions
of a document. To customise the versions further some conditional processing
can come in handy to distinguish which version is being compiled.
The package provides two macros to describe the compilation context:

%%%%%%%%%%%%%%%%%%%%%%%%%%%%%%%%%%%%%%%%
\DescribeMacro{\ifchilddoc}
The conditional |\ifchilddoc| distinguishes between the compilation of
child documents and the main document:
%
\begin{center}
|\ifchilddoc |\textit{child-code}| |[|\||else |\textit{main-code}]| \||fi|
\end{center}

%%%%%%%%%%%%%%%%%%%%%%%%%%%%%%%%%%%%%%%%
\DescribeMacro{\childdocname}
\DescribeMacro{\childdocjob}
The macro |\childdocname| contains the filename (without extension)
of the main or child file being processed.
Note that |\childdocjob| will always contain the name of the main file.

%%%%%%%%%%%%%%%%%%%%%%%%%%%%%%%%%%%%%%%%
\paragraph{Title Page.}

Conditional processing can be used to include a title or banner page
in the main document when proper precautions are taken.
Importantly, the code in the main file should ensure that the page counter
(as well as other status parameters which are stored in the |.aux| files)
takes the same value after the conditional processing.
Otherwise the page numbers may take divergent values
depending on which part is compiled.

For example, a title page could be declared by:
%
\begin{center}
\begin{tabular}{l}
|\ifchilddoc\||else|\\
|\addtocounter{page}{-1}|\\
\textit{code for title page}\\
|\newpage|\\
|\||fi|
\end{tabular}
\end{center}
%
A banner page for the child documents can be generated by:
%
\begin{center}
\begin{tabular}{l}
|\ifchilddoc|\\
|\addtocounter{page}{-1}|\\
\textit{code for banner page}\\
|\newpage|\\
|\||fi|
\end{tabular}
\end{center}
%
Here one could write a message such as:
\begin{center}
|This is the part \childdocname{} of \childdocjob{}.|
\end{center}

%%%%%%%%%%%%%%%%%%%%%%%%%%%%%%%%%%%%%%%%%%%%%%%%%%%%%%%%%%%%%%%%%%%%%%%%%%%%%%%%
\subsection{Flags}
\label{sec:flags}

The package makes it easy to generate different versions
of the main or child documents.
To this end compilation flags can be defined
and assigned different default values.
They will be particularly useful in conjunction
with the forwarding mechanism described in \secref{sec:forward}.

For example, it may be useful to have a flag |\version|
which can be set to |draft| or |final|.
The document source will contain some conditional code
depending on the value of |\version|.
Suppose further, the flag should default to |final| for the main file
and to |draft| for child files
which is a natural assignment for editing the document.
This is achieved by placing the following code
in the preamble of the main document
(below the |\childdocmain| directive):
%
\begin{center}
\begin{tabular}{l}
|\ifchilddoc|\\
|\providecommand{\version}{draft}|\\
|\||else|\\
|\providecommand{\version}{final}|\\
|\||fi|
\end{tabular}
\end{center}
%
The definition by |\providecommand| makes sure
that previous definitions are not overwritten.
Further statements |\providecommand{\version}{...}|
can thus be added before the above code to override it.

For the main file, one might add a line
(between |\childdocmain| and the above block)
%
\begin{center}
|%\ifchilddoc\||else\providecommand{\version}{draft}\||fi|
\end{center}
%
which can be uncommented to produce a draft version.
Likewise one can add a line to the very top of a child file
(above the |\childdocof{|\textit{main}|}| directive)
%
\begin{center}
|%\providecommand{\version}{final}|
\end{center}
%
which can be uncommented to produce the final version of this child document.

%%%%%%%%%%%%%%%%%%%%%%%%%%%%%%%%%%%%%%%%%%%%%%%%%%%%%%%%%%%%%%%%%%%%%%%%%%%%%%%%
\subsection{Forwarding}
\label{sec:forward}

Different versions of the main or child documents
using compilation flags as described in \secref{sec:flags}
can be (permanently) stored in different files
for convenient compilation, viewing and distribution.
To this end, the package defines a command
to pass on compilation to a different file:

%%%%%%%%%%%%%%%%%%%%%%%%%%%%%%%%%%%%%%%%
\DescribeMacro{\childdocforward}
The command |\childdocforward| redirects processing to
another source file:
%
\begin{center}
\begin{tabular}{l}
|\input{childdoc.def}|\\
|\childdocforward[|\textit{main}|]{|\textit{dest}|}|\\
\end{tabular}
\end{center}
%
The argument \textit{dest} is the destination file
(without extension).
It should be the main file or one of the child files.
Note that further \textsf{childdoc} directives
such as |\childdocof| and |\childdocforward|
in the indicated file will be processed in this form.
The optional argument \textit{main}
passes on directly to the main file \textit{main}
while pretending to compile the child \textit{dest}.
This form behaves as if \textit{dest}
issues |\childdocof{|\textit{main}|}| right away,
and no further \textsf{childdoc} directives will be processed.

%%%%%%%%%%%%%%%%%%%%%%%%%%%%%%%%%%%%%%%%
\DescribeMacro{\...prefix}
In the alternative form |\childdocforwardprefix|,
%
\begin{center}
\begin{tabular}{l}
|\input{childdoc.def}|\\
|\childdocforwardprefix[|\textit{main}|]{|\textit{prefix}|}{|\textit{dest}|}|
\end{tabular}
\end{center}
%
the destination file is determined by a pattern
depending on the current file:
To make this work, the current file must be called
`{\textit{prefix}\hspace{0.2em}\textit{suffix}}'
with \textit{prefix} matching precisely the argument.
Processing is then passed on to the file
`{\textit{dest}\hspace{0.2em}\textit{suffix}}'.
Surely, the same effect is achieved by
directly specifying the
argument `{\textit{dest}\hspace{0.2em}\textit{suffix}}'
in the first form.
However, that requires to set up a different file
for each child. With the alternative form of the command
all these files can have exactly the same content
which simplifies setting them up and maintaining them.

For example, the following file |draft.tex|
with a compilation flag |\version| as described in \secref{sec:flags}
compiles the main document as a draft:
%
\begin{center}
\begin{tabular}{l}
|\def\version{draft}|\\
|\input{childdoc.def}|\\
|\childdocforward{|\textit{main}|}|
\end{tabular}
\end{center}
%
Likewise, the following files |final|\textit{nn}|.tex|
compile the final version of the child document
|child|\textit{nn}|.tex|:
%
\begin{center}
\begin{tabular}{l}
|\def\version{final}|\\
|\input{childdoc.def}|\\
|\childdocforwardprefix{final}{child}|
\end{tabular}
\end{center}
%

Note that when several versions of a main file and/or of each child file
are to be generated, it may be convenient to set up a |Makefile| or
shell script to automatise the process.

%%%%%%%%%%%%%%%%%%%%%%%%%%%%%%%%%%%%%%%%%%%%%%%%%%%%%%%%%%%%%%%%%%%%%%%%%%%%%%%%
\subsection{Command Line Processing}
\label{sec:commandline}

The effect of redirection files can also be achieved by invoking
the \LaTeX{} compiler with a more elaborate command line.
Most conveniently this should be done as part
of a shell script or a |Makefile|.

When using \textsf{childdoc} in the main file, the following
command lines effectively perform a redirection
(note that depending on the shell being used,
backslashes may have to be doubled: `|\|' $\to$ `|\\|'):
%
\begin{center}
|... -jobname "|\textit{target}|" |\\|"|[\textit{flags}]%
|\input{childdoc.def}\childdocforward[|\textit{main}|]{|\textit{dest}|}"|
\end{center}
%
Here \textit{target} is the name of the output file,
\textit{main} is the name of the main file
and \textit{dest} is the name of the main or child file to be processed
(all filenames without extensions).
The optional argument \textit{main} can be omitted
if \textit{main} matches \textit{dest}.
Optionally, compilation \textit{flags} can be defined via |\def| commands.
This command line makes the \TeX{} engine believe
it is compiling the file \textit{target}
whose content is specified as the latter parameter.
The provided code then forwards the processing to
\textit{main} or \textit{dest} as described in \secref{sec:forward}.

%%%%%%%%%%%%%%%%%%%%%%%%%%%%%%%%%%%%%%%%%%%%%%%%%%%%%%%%%%%%%%%%%%%%%%%%%%%%%%%%
\subsection{Include by Input}
\label{sec:input}

Including child documents by |\include| has some restrictions by design.
Most notably, the content of a child document always occupies
its own set of pages; pages cannot be shared between child documents.
Usually, this behaviour makes perfect sense
because each child document contain an essential part of the document.
However, in some situations it may be desirable to compose
a document from a collection of parts
without having mandatory page breaks between then.
For this case, the package
provides a mechanism to include parts
by |\input| which can also be processed individually.
However, by construction this mechanism
requires manual handling of the content to be output.

%%%%%%%%%%%%%%%%%%%%%%%%%%%%%%%%%%%%%%%%
\DescribeMacro{\ifchilddocmanual}
The main file should be prepared as usual, see \secref{sec:include}.
However, the document body must make a distinction
between processing of an individual part and of the main document, e.g.:
%
\begin{center}
\begin{tabular}{l}
|\ifchilddocmanual|\\
|\input{\childdocname}|\\
|\||else|\\
\textit{document body with }|\input{|\textit{part}|}|\\
|\||fi|
\end{tabular}
\end{center}
%
The conditional |\ifchilddocmanual| is true whenever
a part to be included by |\input| is being compiled,
and the name of the part is stored in |\childdocname|.

%%%%%%%%%%%%%%%%%%%%%%%%%%%%%%%%%%%%%%%%
\DescribeMacro{\childdocby}
Each part to be included by |\input| should start with:
%
\begin{center}
\begin{tabular}{l}
|\input{childdoc.def}|\\
|\childdocby{|\textit{main}|}|\\
\end{tabular}
\end{center}
%
The directive |\childdocby| is similar to |\childdocof|
described in \secref{sec:include},
but the subsequent selection of content must be done manually.
To that end, both |\ifchilddoc| and |\ifchilddocmanual|
will be true upon processing of a part,
and the name of the part is stored in |\childdocname|.
Note that |\jobname| will be set to the filename of the current part
so that each part receives an individual |.aux| file
that does not interfere with the |.aux| file(s) of the main document.
This behaviour can be altered by the alternative form
|\childdocby[*]{|\textit{main}|}| (with a non-empty optional argument)
which uses the |.aux| file of the main document
by setting |\jobname| to \textit{main}.

%%%%%%%%%%%%%%%%%%%%%%%%%%%%%%%%%%%%%%%%%%%%%%%%%%%%%%%%%%%%%%%%%%%%%%%%%%%%%%%%
\subsection{Driver Development}
\label{sec:driver}

The \textsf{childdoc} mechanism can also be use for the development
of definition files such as \LaTeX{} styles or classes.
This case differs from the above setup with multiple parts
included by |\include| in that no |\includeonly| should be invoked.
This can be achieved by starting the include file
(before |\ProvidesPackage|) with:
%
\begin{center}
\begin{tabular}{l}
|\input{childdoc.def}|\\
|\childdocforward{|\textit{main}|}|\\
\end{tabular}
\end{center}
%
or alternatively with:
%
\begin{center}
\begin{tabular}{l}
|\input{childdoc.def}|\\
|\childdocby{|\textit{main}|}|\\
\end{tabular}
\end{center}
%
Both forms have slightly different effects as described above.
The main file is prepared as usual, see \secref{sec:include}.

%%%%%%%%%%%%%%%%%%%%%%%%%%%%%%%%%%%%%%%%%%%%%%%%%%%%%%%%%%%%%%%%%%%%%%%%%%%%%%%%
\subsection{Legacy Detection}
\label{sec:detection}

The directive |\childdocmain| in the main file can detect
whether the complete document or merely a child is to be compiled
even without using the directive |\childdocof|.
This method is deprecated because it is less robust
and there is no compelling reason to use it;
it is merely provided for backward compatibility
and it may be removed in future versions.

If the detection mechanism is to be used,
it is mandatory to correctly specify
the filename of the main file as the argument of |\childdocmain|:
%
\begin{center}
\begin{tabular}{l}
|\input{childdoc.def}|\\
|\childdocmain{|\textit{main}|}|\\
\end{tabular}
\end{center}
%
If |\jobname| does not match the argument \textit{main} of |\childdocmain|,
it is assumed that |\jobname| points to the child file to be compiled.
When using |\childdocmain| with the main file specified as argument,
it suffices to start a child file
with just |\input{|\textit{main}|}|
without loading of the package and using |\childdocof|.
If instead all processing is done
with the appropriate \textsf{childdoc} directives,
the argument of \textit{main} of |\childdocmain| can be empty.

An alternative version of the command line processing described
in \secref{sec:commandline} using the detection mechanism reads:
%
\begin{center}
|... -jobname "|\textit{target}|" "|[\textit{flags}]%
[|\def\jobname{|\textit{dest}|}|]|\input{|\textit{main}|}"|
\end{center}

%%%%%%%%%%%%%%%%%%%%%%%%%%%%%%%%%%%%%%%%%%%%%%%%%%%%%%%%%%%%%%%%%%%%%%%%%%%%%%%%
\subsection{Manual Code}
\label{sec:manual}

In case one cannot be certain whether the definitions file |childdoc.def|
is installed on the target \TeX{} distribution
and one prefers not to ship it,
it is conceivable to paste a few relevant commands into the sources.

To that end, drop all statements |\input{childdoc.def}|
and perform the replacements as outlined below.
Instead of |\childdocmain{|\textit{main}|}| add the following code
to the top of the main file:
%
\begin{center}
\begin{tabular}{l}
|\||ifdefined\childdocname\endinput\||fi\newif\ifchilddoc|\\
|\edef\childdocname{\scantokens\expandafter{\jobname\noexpand}}|\\
|\def\childdocmain{|\textit{main}|}\||ifx\childdocmain\childdocname\||else|\\
|\childdoctrue\includeonly{\childdocname}\let\jobname\childdocmain\||fi|\\
\end{tabular}
\end{center}
%
Instead of |\childdocof{|\textit{main}|}| just include the main file
at the top of each child file:
%
\begin{center}
|\input{|\textit{main}|}|
\end{center}
%
A simple redirection |\childdocforward{|\textit{dest}|}| is achieved by:
%
\begin{center}
|\def\jobname{|\textit{dest}|}\input{\jobname}|
\end{center}
%
The redirection with prefix
|\childdocforwardprefix[|\textit{prefix}|]{|\textit{dest}|}|
is accomplished by:
%
\begin{center}
\begin{tabular}{l}
|{\edef\jobname{\scantokens\expandafter{\jobname\noexpand}}|\\
|\def\redirectjob |\textit{prefix}|#1~~~{\gdef\jobname{|\textit{dest}|#1}}|\\
|\expandafter\redirectjob\jobname~~~}\input{\jobname}|
\end{tabular}
\end{center}

In an alternative approach,
child documents can be compiled by a specific command line
without additional code or specific definitions:
%
\begin{center}
|... -jobname "|\textit{target}|" "|[\textit{flags}]%
|\includeonly{|\textit{dest}|}\input{|\textit{main}|}"|
\end{center}
%

%%%%%%%%%%%%%%%%%%%%%%%%%%%%%%%%%%%%%%%%%%%%%%%%%%%%%%%%%%%%%%%%%%%%%%%%%%%%%%%%
%%%%%%%%%%%%%%%%%%%%%%%%%%%%%%%%%%%%%%%%%%%%%%%%%%%%%%%%%%%%%%%%%%%%%%%%%%%%%%%%
\section{Information}

%%%%%%%%%%%%%%%%%%%%%%%%%%%%%%%%%%%%%%%%%%%%%%%%%%%%%%%%%%%%%%%%%%%%%%%%%%%%%%%%
\subsection{Copyright}

Copyright \copyright{} 2017--2018 Niklas Beisert

This work may be distributed and/or modified under the
conditions of the \LaTeX{} Project Public License, either version 1.3
of this license or (at your option) any later version.
The latest version of this license is in
  \url{http://www.latex-project.org/lppl.txt}
and version 1.3 or later is part of all distributions of \LaTeX{}
version 2005/12/01 or later.

This work has the LPPL maintenance status `maintained'.

The Current Maintainer of this work is Niklas Beisert.

This work consists of the files |README.txt|, |childdoc.ins| and |childdoc.dtx|
as well as the derived files |childdoc.def|, |cdocsamp.tex|
with |cdocsch1.tex|, |cdocsch2.tex|, |cdocspt3.tex|, |cdocspt4.tex|,
|cdocsdrf.tex|, |cdocsfn1.tex|, |cdocsfn2.tex|
as well as |childdoc.pdf|.

%%%%%%%%%%%%%%%%%%%%%%%%%%%%%%%%%%%%%%%%%%%%%%%%%%%%%%%%%%%%%%%%%%%%%%%%%%%%%%%%
\subsection{Files and Installation}

The package consists of the files:
%
\begin{center}
\begin{tabular}{ll}
    |README.txt|   & readme file \\
    |childdoc.ins| & installation file \\
    |childdoc.dtx| & source file \\
    |childdoc.def| & definition file \\
    |cdocsamp.tex| & sample main file \\
    |cdocsch1.tex| & sample include file \\
    |cdocsch2.tex| & sample include file \\
    |cdocspt3.tex| & sample part file \\
    |cdocspt4.tex| & sample part file \\
    |cdocsdrf.tex| & sample redirection file \\
    |cdocsfn1.tex| & sample redirection file \\
    |cdocsfn2.tex| & sample redirection file \\
    |childdoc.pdf| & manual
\end{tabular}
\end{center}
%
The distribution consists of the files
|README.txt|, |childdoc.ins| and |childdoc.dtx|.
%
\begin{itemize}
\item
Run (pdf)\LaTeX{} on |childdoc.dtx|
to compile the manual |childdoc.pdf| (this file).
\item
Run \LaTeX{} on |childdoc.ins| to create the definitions file |childdoc.def|
and the sample |cdocsamp.tex| with include files
|cdocsch1.tex|, |cdocsch2.tex|, |cdocspt3.tex|, |cdocspt4.tex|,
|cdocsdrf.tex|, |cdocsfn1.tex|, |cdocsfn2.tex|.
Then copy the file |childdoc.def| to an appropriate directory of your \LaTeX{}
distribution, e.g.\ \textit{texmf-root}|/tex/latex/childdoc|.
\end{itemize}

%%%%%%%%%%%%%%%%%%%%%%%%%%%%%%%%%%%%%%%%%%%%%%%%%%%%%%%%%%%%%%%%%%%%%%%%%%%%%%%%
\subsection{Related CTAN Packages}

There are several other packages which offer a similar functionality:
%
\begin{itemize}
\item
The packages
\href{http://ctan.org/pkg/docmute}{\textsf{docmute}},
\href{http://ctan.org/pkg/includex}{\textsf{includex}} and
\href{http://ctan.org/pkg/standalone}{\textsf{standalone}}
provide commands to include only the document body of
a child file thus allowing both files to be compiled individually.
\item
The packages \href{http://ctan.org/pkg/subdocs}{\textsf{subdocs}}
and \href{http://ctan.org/pkg/subfiles}{\textsf{subfiles}}
provide structures in which the main and child documents can be
encapsulated and allowing them to be compiled individually.
The inclusion mechanism is different from the conventional |\include|.
\item
The package \href{http://ctan.org/pkg/combine}{\textsf{combine}}
is an elaborate solution to combine several documents into one.
\end{itemize}
%
See also the CTAN topic \href{http://ctan.org/topic/subdocs}{\textsf{subdocs}}
for further related packages.
The present package differs from the above solutions in that
a document structure constructed with the conventional |\include| mechanism
just needs two extra commands at the top of every file
such that all constituent files can be compiled individually.

%%%%%%%%%%%%%%%%%%%%%%%%%%%%%%%%%%%%%%%%%%%%%%%%%%%%%%%%%%%%%%%%%%%%%%%%%%%%%%%%
%\subsection{Feature Suggestions}
%
%The following is a list of features which may be useful for future
%versions of this package:
%%
%\begin{itemize}
%\item
%\ldots
%\end{itemize}

%%%%%%%%%%%%%%%%%%%%%%%%%%%%%%%%%%%%%%%%%%%%%%%%%%%%%%%%%%%%%%%%%%%%%%%%%%%%%%%%
\subsection{Revision History}

%%%%%%%%%%%%%%%%%%%%%%%%%%%%%%%%%%%%%%%%
\paragraph{v2.0:} 2018/12/30

\begin{itemize}
\item
immediate forward processing
\item
added |\childdocby| mechanism
\item
manual restructured
\end{itemize}

%%%%%%%%%%%%%%%%%%%%%%%%%%%%%%%%%%%%%%%%
\paragraph{v1.6:} 2018/01/17

\begin{itemize}
\item
application for development of include files
\item
corrections to manual
\end{itemize}

%%%%%%%%%%%%%%%%%%%%%%%%%%%%%%%%%%%%%%%%
\paragraph{v1.5:} 2017/05/21

\begin{itemize}
\item
more complete structuring introduced
\item
|\childdocof| introduced
\item
|\childdoc| renamed to |\childdocmain|
\item
|\childredirect| renamed to |\childdocforward| and |\childdocforwardprefix|
and functionality expanded
\end{itemize}

%%%%%%%%%%%%%%%%%%%%%%%%%%%%%%%%%%%%%%%%
\paragraph{v1.0:} 2017/04/27

\begin{itemize}
\item
manual and install package
\item
first version published on CTAN
\end{itemize}

%%%%%%%%%%%%%%%%%%%%%%%%%%%%%%%%%%%%%%%%
\paragraph{v0.6:} 2017/04/26

\begin{itemize}
\item
redirection mechanism added
\end{itemize}

%%%%%%%%%%%%%%%%%%%%%%%%%%%%%%%%%%%%%%%%
\paragraph{v0.5:} 2017/04/26

\begin{itemize}
\item
functionality in definition file
\end{itemize}


%%%%%%%%%%%%%%%%%%%%%%%%%%%%%%%%%%%%%%%%%%%%%%%%%%%%%%%%%%%%%%%%%%%%%%%%%%%%%%%%
%%%%%%%%%%%%%%%%%%%%%%%%%%%%%%%%%%%%%%%%%%%%%%%%%%%%%%%%%%%%%%%%%%%%%%%%%%%%%%%%
%%%%%%%%%%%%%%%%%%%%%%%%%%%%%%%%%%%%%%%%%%%%%%%%%%%%%%%%%%%%%%%%%%%%%%%%%%%%%%%%
\appendix

\settowidth\MacroIndent{\rmfamily\scriptsize 000\ }

 \DocInput{childdoc.dtx}

\end{document}
%</driver>
% \fi
%
% %%%%%%%%%%%%%%%%%%%%%%%%%%%%%%%%%%%%%%%%%%%%%%%%%%%%%%%%%%%%%%%%%%%%%%%%%%%%%%
% %%%%%%%%%%%%%%%%%%%%%%%%%%%%%%%%%%%%%%%%%%%%%%%%%%%%%%%%%%%%%%%%%%%%%%%%%%%%%%
% \section{Sample}
%\iffalse
%<*samplemain>
%\fi
%
% The following presents a sample document
% with two chapters, two parts, a title page,
% a compile flag as well as three forwarding files to set the flag.
% It consists of eight |.tex| files:
% \begin{center}
% \begin{tabular}{ll}
% |cdocsamp.tex|&main file\\
% |cdocsch1.tex|&include file for chapter 1\\
% |cdocsch2.tex|&include file for chapter 2\\
% |cdocspt3.tex|&include file for part 3\\
% |cdocspt4.tex|&include file for part 4\\
% |cdocsdrf.tex|&forwarding file for main file in draft mode\\
% |cdocsfi1.tex|&forwarding file for final version of chapter 1\\
% |cdocsfi2.tex|&forwarding file for final version of chapter 2\\
% \end{tabular}
% \end{center}
% Each of the eight files can be compiled directly by the \LaTeX{} compiler.
%
% %%%%%%%%%%%%%%%%%%%%%%%%%%%%%%%%%%%%%%
% \paragraph{Main File.}
%
% The main file is called |cdocsamp.tex|.
%
% Load the \textsf{childdoc} definitions and
% declare the filename for the main document:
%    \begin{macrocode}
\input{childdoc.def}
\childdocmain{}
%    \end{macrocode}

% Optional override for |\version| flag:
%    \begin{macrocode}
%%\ifchilddoc\else\providecommand{\version}{draft}\fi
%    \end{macrocode}

% Define the default values for the |\version| flag
% (|final| for the main file and |draft| for childs):
%    \begin{macrocode}
\ifchilddoc
\providecommand{\version}{draft}
\else
\providecommand{\version}{final}
\fi
%    \end{macrocode}

% Load the standard document class:
%    \begin{macrocode}
\documentclass[12pt]{article}
%    \end{macrocode}

% Start the document body:
%    \begin{macrocode}
\begin{document}
%    \end{macrocode}

% Declare a title page.
% Print title, part of document being processed and version flag:
%    \begin{macrocode}
\addtocounter{page}{-1}
\begin{center}
{\LARGE\bfseries{}childdoc example\par}
\vspace{1cm}
\ifchilddoc
\ifchilddocmanual part\else chapter\fi:
`\childdocname' of `\childdocjob'\par
\else
main document: `\childdocjob'\par
\fi
version: \version\par
\end{center}
\newpage
%    \end{macrocode}

% Manually include selected file,
% otherwise process as usual:
%    \begin{macrocode}
\ifchilddocmanual
\section*{part `\childdocname'}
\input{\childdocname}
\else
%    \end{macrocode}

% Include the two chapters:
%    \begin{macrocode}
\include{cdocsch1}
\include{cdocsch2}
%    \end{macrocode}

% Include the two parts unless only chapters should be displayed:
%    \begin{macrocode}
\ifchilddoc\else
\section{part three}
\input{cdocspt3}
\section{part four}
\input{cdocspt4}
\fi
%    \end{macrocode}

% Process as usual until here:
%    \begin{macrocode}
\fi
%    \end{macrocode}

% End of document body:
%    \begin{macrocode}
\end{document}
%    \end{macrocode}
%\iffalse
%</samplemain>
%\fi
%
% %%%%%%%%%%%%%%%%%%%%%%%%%%%%%%%%%%%%%%
% \paragraph{Chapter Include Files.}
%
% The include files are called |cdocsch1.tex| and |cdocsch2.tex|.
%
%\iffalse
%<*samplechap1|samplechap2>
%\fi

% Optional override for |\version| flag:
%    \begin{macrocode}
%%\providecommand{\version}{final}
%    \end{macrocode}

% Include the main document:
%    \begin{macrocode}
\input{childdoc.def}
\childdocof{cdocsamp}
%    \end{macrocode}

%\iffalse
%</samplechap1|samplechap2>
%\fi
%
%\iffalse
%<*samplechap1>
%\fi
% Some text for chapter 1:
%    \begin{macrocode}
\section{one}
some text in chapter one
%    \end{macrocode}

%\iffalse
%</samplechap1>
%\fi
% Some text for chapter 2:
%\iffalse
%<*samplechap2>
%\fi
%    \begin{macrocode}
\section{two}
more text in chapter two
%    \end{macrocode}

%\iffalse
%</samplechap2>
%\fi
%
% %%%%%%%%%%%%%%%%%%%%%%%%%%%%%%%%%%%%%%
% \paragraph{Part Include Files.}
%
% The include files are called |cdocspt3.tex| and |cdocspt4.tex|.
%
%\iffalse
%<*samplepart3|samplepart4>
%\fi

% Optional override for |\version| flag:
%    \begin{macrocode}
%%\providecommand{\version}{final}
%    \end{macrocode}

% Include the main document:
%    \begin{macrocode}
\input{childdoc.def}
\childdocby{cdocsamp}
%    \end{macrocode}

%\iffalse
%</samplepart3|samplepart4>
%\fi
%
%\iffalse
%<*samplepart3>
%\fi
% Some text for part 3:
%    \begin{macrocode}
some text in part three
%    \end{macrocode}

%\iffalse
%</samplepart3>
%\fi
% Some text for part 4:
%\iffalse
%<*samplepart4>
%\fi
%    \begin{macrocode}
more text in part four
%    \end{macrocode}

%\iffalse
%</samplepart4>
%\fi
%
% %%%%%%%%%%%%%%%%%%%%%%%%%%%%%%%%%%%%%%
% \paragraph{Forwarding for a Complete Draft.}
%
% The following forwarding file |cdocsdrf.tex|
% compiles the main document in draft mode:
%\iffalse
%<*sampledraft>
%\fi
%    \begin{macrocode}
\def\version{draft}
\input{childdoc.def}
\childdocforward{cdocsamp}
%    \end{macrocode}

%\iffalse
%</sampledraft>
%\fi
%
% %%%%%%%%%%%%%%%%%%%%%%%%%%%%%%%%%%%%%%
% \paragraph{Forwarding for Final Version of the Chapters.}
%
% The following forwarding files |cdocsfn1.tex| and |cdocsfn2.tex|
% (with identical content)
% compile the final versions of the child documents
% |cdocsch1.tex| and |cdocsch2.tex|, respectively:
%\iffalse
%<*samplefinal>
%\fi
%    \begin{macrocode}
\def\version{final}
\input{childdoc.def}
\childdocforwardprefix[cdocsamp]{cdocsfn}{cdocsch}
%    \end{macrocode}

%\iffalse
%</samplefinal>
%\fi
%
% %%%%%%%%%%%%%%%%%%%%%%%%%%%%%%%%%%%%%%
% \paragraph{Command Line Processing.}
%
% The following three command lines generate the output files
% |cdocscld|, |cdocscl1| and |cdocscl2|
% which should be identical to
% |cdocsdrf|, |cdocsch1| and |cdocsfn2|, respectively:
% \begin{center}
% \begin{tabular}{l}
% |latex -jobname cdocscld \|\\
% |  "\def\version{draft}\input{childdoc.def}\childdocforward{cdocsamp}"|\\
% |latex -jobname cdocscl1 \|\\
% |  "\input{childdoc.def}\childdocforward[cdocsamp]{cdocsch1}"|\\
% |latex -jobname cdocscl2 \|\\
% |  "\def\version{final}\input{childdoc.def}\childdocforward{cdocsch2}"|
% \end{tabular}
% \end{center}
% Note that the trailing backslash on each first line
% merely continues the input to the second line
% (for convenient cut ant paste).
% Furthermore, the command |latex| can be replaced by any
% of its alternative versions such as |pdflatex|.
%
% %%%%%%%%%%%%%%%%%%%%%%%%%%%%%%%%%%%%%%%%%%%%%%%%%%%%%%%%%%%%%%%%%%%%%%%%%%%%%%
% %%%%%%%%%%%%%%%%%%%%%%%%%%%%%%%%%%%%%%%%%%%%%%%%%%%%%%%%%%%%%%%%%%%%%%%%%%%%%%
% \section{Implementation}
%\iffalse
%<*package>
%\fi
%
% This section describes the definitions file |childdoc.def|.

% The definitions cannot be loaded using |\usepackage| or |\RequirePackage|
% which has a mechanism to prevent loading a style file more than once.
% When loading the definitions by means of |\input|
% multiple instances have to be prevented manually:
%\iffalse
%This code needs to be before the `\ProvidesFile' directive
%which is defined at the beginning of this file.
%Therefore it is also placed there and commented out here.
%</package>
%<*discard>
%\fi
%    \begin{macrocode}
\ifdefined\childdocmain\endinput\fi
%    \end{macrocode}
%\iffalse
%</discard>
%<*package>
%\fi
%
% \macro{\ifchilddoc}
% \macro{\ifchilddocmanual}
% The conditional |\ifchilddoc| tells whether a
% child (true) or main (false) document is being compiled.
% The conditional |\ifchilddocmanual| tells whether
% the |\includeonly| mechanism is used (false) or
% the selection of child files must be performed manually (true).
% The definitions initialise to false:
%    \begin{macrocode}
\newif\ifchilddoc
\newif\ifchilddocmanual
%    \end{macrocode}

% \macro{\childdocname}
% \macro{\childdocjob}
% The macro |\childdocname| stores the name of the main document
% to be compiled. The macro |\childdocjob| stores the name of
% the document on which the \LaTeX{} compiler was originally invoked.
% The content of |\jobname| cannot be compared
% to filenames specified in the source due to different catcodes.
% The following code rescans |\jobname|, stores the result
% in |\childdocname| and saves a copy in |\childdocjob|:
%    \begin{macrocode}
\edef\childdocname{\scantokens\expandafter{\jobname\noexpand}}
\let\childdocjob\childdocname
%    \end{macrocode}

% \macro{\childdocdisable}
% The macro |\childdocdisable| prevents the main file
% from being processed more than once.
% At this stage, the main document command |\childdocmain|
% is assumed to be called once again where it should do nothing.
% Any subsequent call to it should prevent
% a secondary processing of the main document
% It overwrites the forwarding commands
% |\childdocof| and |\childdocforward|
% with empty macros to prevent further inclusions of the main document:
%    \begin{macrocode}
\newcommand{\childdocdisable}
{
  \renewcommand{\childdocmain}[1]{\renewcommand{\childdocmain}[1]{\endinput}}
  \renewcommand{\childdocof}[1]{}
  \renewcommand{\childdocby}[2][]{}
  \renewcommand{\childdocforward}[2][]{}
  \renewcommand{\childdocdisable}{}
}
%    \end{macrocode}

% \macro{\childdocmain}
% The macro |\childdocmain| is to be called at the top of the main file
% with nothing or the main filename (without extension) as argument.
% First, it breaks loops.
% If the argument is not empty and does not match |\childdocname|
% (which is set by the first inclusion of |childdoc.def|),
% |\ifchilddoc| is set to true, |\includeonly| is applied to the child file
% and |\jobname| is set to the main file
% (for proper handling of |.aux| files):
%    \begin{macrocode}
\newcommand{\childdocmain}[1]
{
  \childdocdisable\childdocmain{}
  \if?#1?\else
    \begingroup
      \def\childdoctmp{#1}
      \ifx\childdoctmp\childdocname
        \def\childdoctmp{}
      \else
        \def\childdoctmp
        {
          \childdoctrue
          \includeonly{\childdocname}
          \def\childdocjob{#1}
          \def\jobname{#1}
        }
      \fi
      \expandafter
    \endgroup
    \childdoctmp
  \fi
}
%    \end{macrocode}

% \macro{\childdocof}
% The command |\childdocof| redirects
% compilation to the main file |#1|.
%    \begin{macrocode}
\newcommand{\childdocof}[1]
{
  \childdocdisable
  \childdoctrue
  \includeonly{\childdocname}
  \def\jobname{#1}
  \def\childdocjob{#1}
  \input{#1}
}
%    \end{macrocode}

% \macro{\childdocby}
% The command |\childdocby| ....
%    \begin{macrocode}
\newcommand{\childdocby}[2][]
{
  \childdocdisable
  \childdoctrue
  \childdocmanualtrue
  \if?#1?\else
    \def\jobname{#2}
  \fi
  \def\childdocjob{#2}
  \input{#2}
  \endinput
}
%    \end{macrocode}

% \macro{\childdocforward}
% The command |\childdocforward| redirects
% compilation to the main file or
% (if the optional argument is given) a child file.
% Parameters are set as if the main file
% or a child file starting with |\childdocof| was compiled.
% Then compilation is handed over to the main file:
%    \begin{macrocode}
\newcommand{\childdocforward}[2][]
{
  \begingroup
    \if?#1?
      \def\childdoctmp
      {
        \def\childdocname{#2}
        \def\childdocjob{#2}
        \def\jobname{#2}
        \input{#2}
        \endinput
      }
    \else
      \def\childdoctmp
      {
        \childdocdisable
        \def\childdocname{#2}
        \childdoctrue
        \includeonly{#2}
        \def\childdocjob{#1}
        \def\jobname{#1}
        \input{#1}
        \endinput
      }
    \fi
    \expandafter
  \endgroup
  \childdoctmp
}
%    \end{macrocode}

% \macro{\childdocforwardprefix}
% The command |\childdocforwardprefix| redirects
% compilation to the main or a child file by means of a pattern.
% The prefix |#1| in the current filename is replaced by |#2|
% and the suffix of the current filename is kept
% (it is assumed that the filename does not contain the substring `|~~~|'
% which is used as a delimiter).
% Compilation is handed over to the new file by |\childdocforward|:
%    \begin{macrocode}
\newcommand{\childdocforwardprefix}[3][]
{
  \begingroup
    \def\childdocextract #2##1~~~{\def\childdoctmp{\childdocforward[#1]{#3##1}}}
    \expandafter\childdocextract\childdocname~~~
    \expandafter
  \endgroup
  \childdoctmp
}
%    \end{macrocode}

% \macro{\childdoc}
% The deprecated macro |\childdoc| is a legacy version of |\childdocmain|:
%    \begin{macrocode}
\newcommand{\childdoc}{\childdocmain}
%    \end{macrocode}

% \macro{\childdocredirect}
% The deprecated macro |\childdocredirect| is a legacy version
% of |\childdocforward| and |\childdocforwardprefix|:
%    \begin{macrocode}
\newcommand{\childdocredirect}[2][]
{
  \begingroup
    \if?#1?
      \def\childdoctmp{\childdocforward{#2}}
    \else
      \def\childdoctmp{\childdocforwardprefix{#1}{#2}}
    \fi
    \expandafter
  \endgroup
  \childdoctmp
}
%    \end{macrocode}

%\iffalse
%</package>
%\fi
%
\endinput
|\\
|\childdocforward{|\textit{main}|}|
\end{tabular}
\end{center}
%
Likewise, the following files |final|\textit{nn}|.tex|
compile the final version of the child document
|child|\textit{nn}|.tex|:
%
\begin{center}
\begin{tabular}{l}
|\def\version{final}|\\
|% \iffalse
%
% childdoc.dtx Copyright (C) 2017-2018 Niklas Beisert
%
% This work may be distributed and/or modified under the
% conditions of the LaTeX Project Public License, either version 1.3
% of this license or (at your option) any later version.
% The latest version of this license is in
%   http://www.latex-project.org/lppl.txt
% and version 1.3 or later is part of all distributions of LaTeX
% version 2005/12/01 or later.
%
% This work has the LPPL maintenance status `maintained'.
%
% The Current Maintainer of this work is Niklas Beisert.
%
% This work consists of the files childdoc.dtx and childdoc.ins
% and the derived files childdoc.def and cdocsamp.tex with
% cdocsch1.tex, cdocsch2.tex, cdocsdrf.tex, cdocsfn1.tex, cdocsfn2.tex.
%
%<package>\ifdefined\childdocmain\endinput\fi
%<package>\ProvidesFile{childdoc.def}[2018/12/30 v2.0 child document driver]
%<samplemain>\ProvidesFile{cdocsamp.tex}[2018/12/30 v2.0 sample for childdoc]
%<*driver>
%\ProvidesFile{childdoc.drv}[2018/12/30 v2.0 childdoc reference manual file]
\PassOptionsToClass{10pt,a4paper}{article}
\documentclass{ltxdoc}

\usepackage[margin=35mm]{geometry}
\usepackage{hyperref}
\usepackage{hyperxmp}
\usepackage[usenames]{color}

\hypersetup{colorlinks=true}
\hypersetup{pdfstartview=FitH}
\hypersetup{pdfpagemode=UseNone}
\hypersetup{pdfsource={}}
\hypersetup{pdflang={en-UK}}
\hypersetup{pdfcopyright={Copyright 2017-2018 Niklas Beisert.
  This work may be distributed and/or modified under the
  conditions of the LaTeX Project Public License, either version 1.3
  of this license or (at your option) any later version.}}
\hypersetup{pdflicenseurl={http://www.latex-project.org/lppl.txt}}
\hypersetup{pdfcontactaddress={ETH Zurich, ITP, HIT K,
  Wolfgang-Pauli-Strasse 27}}
\hypersetup{pdfcontactpostcode={8093}}
\hypersetup{pdfcontactcity={Zurich}}
\hypersetup{pdfcontactcountry={Switzerland}}
\hypersetup{pdfcontactemail={nbeisert@itp.phys.ethz.ch}}
\hypersetup{pdfcontacturl={http://people.phys.ethz.ch/\xmptilde nbeisert/}}

\newcommand{\secref}[1]{\hyperref[#1]{section \ref*{#1}}}

\parskip1ex
\parindent0pt
\let\olditemize\itemize
\def\itemize{\olditemize\parskip0pt}

\begin{document}

\title{The \textsf{childdoc} Package}
\hypersetup{pdftitle={The childdoc Package}}
\author{Niklas Beisert\\[2ex]
  Institut f\"ur Theoretische Physik\\
  Eidgen\"ossische Technische Hochschule Z\"urich\\
  Wolfgang-Pauli-Strasse 27, 8093 Z\"urich, Switzerland\\[1ex]
  \href{mailto:nbeisert@itp.phys.ethz.ch}
  {\texttt{nbeisert@itp.phys.ethz.ch}}}
\hypersetup{pdfauthor={Niklas Beisert}}
\hypersetup{pdfsubject={Manual for the LaTeX2e Package childdoc}}
\date{30 December 2018, \textsf{v2.0}}
\maketitle

\begin{abstract}\noindent
\textsf{childdoc} is a \LaTeXe{} package
that enables the direct compilation
of document sections included by |\include|
to individual files.
\end{abstract}

\begingroup
\parskip0ex
\tableofcontents
\endgroup

%%%%%%%%%%%%%%%%%%%%%%%%%%%%%%%%%%%%%%%%%%%%%%%%%%%%%%%%%%%%%%%%%%%%%%%%%%%%%%%%
%%%%%%%%%%%%%%%%%%%%%%%%%%%%%%%%%%%%%%%%%%%%%%%%%%%%%%%%%%%%%%%%%%%%%%%%%%%%%%%%
\section{Introduction}

\LaTeX{} provides a mechanism to structure a large document (such as a book)
into a main file and several child files (containing the chapters)
using the |\include| command.
This mechanism is beneficial for documents
which span hundreds of pages in order to
make the source file(s) more manageable.
Moreover, compilation can be restricted to
selected child files by means of the |\includeonly| command.
The latter feature can be used to reduce the compilation time while editing
(this was significantly more useful in the earlier days of \LaTeX{})
or to generate a smaller document which is easier to navigate.
Another application of |\includeonly| is to generate
documents consisting of selected parts of the complete document.

However, there are a few drawbacks of the plain |\include| mechanism:
\begin{itemize}
\item
The child files cannot be compiled on their own,
they can only be compiled via the main file.
A naive editing environment
(such as a text editor with an option
to have the current file processed by \LaTeX)
may require one to switch to the main file before compiling;
attempting to compile the child file produces errors.
\item
The main file must be modified (each time)
to adjust the |\includeonly| command
to the present needs. This easily leaves the main file in a messy state.
\item
The generated document will always carry the filename
of the main document. This is inconvenient if
several child files are to be compiled and
to be kept for distribution.
\end{itemize}

The present package provides a simple interface
to make child files individually compilable by \LaTeX{}.
Compiling a child file then has the same effect as compiling
the main file with an |\includeonly| command
to select the appropriate child.
Moreover the generated document will carry the name of the child
rather than the main file.
This resolves all three above issues.

This feature is meant to make the editing of books,
thesis documents and lecture notes somewhat more convenient.
However, the package can also be used efficiently for
composing a series of documents (such as exercise sheets)
which are typically distributed individually.
It then assists the author in generating the individual documents
(potentially in different versions)
as well as a document containing the collected series.
Another application is in developing style files
or other kinds of included material
where compilation of the style file could redirect
to a sample or test file.

%%%%%%%%%%%%%%%%%%%%%%%%%%%%%%%%%%%%%%%%%%%%%%%%%%%%%%%%%%%%%%%%%%%%%%%%%%%%%%%%
%%%%%%%%%%%%%%%%%%%%%%%%%%%%%%%%%%%%%%%%%%%%%%%%%%%%%%%%%%%%%%%%%%%%%%%%%%%%%%%%
\section{Usage}

First of all, the package \textsf{childdoc} is \emph{not} a standard
\LaTeXe{} |.sty| style file! Therefore it needs to be invoked in
a non-standard way.

%%%%%%%%%%%%%%%%%%%%%%%%%%%%%%%%%%%%%%%%%%%%%%%%%%%%%%%%%%%%%%%%%%%%%%%%%%%%%%%%
\subsection{Included Files}
\label{sec:include}

%%%%%%%%%%%%%%%%%%%%%%%%%%%%%%%%%%%%%%%%
\DescribeMacro{\childdocmain}
To use the package, add the commands
\begin{center}
\begin{tabular}{l}
|\input{childdoc.def}|\\
|\childdocmain{}|\\
\end{tabular}
\end{center}
at the very top of the main \LaTeX{} file,
in particular \emph{before} the |\documentclass| statement!
The argument of |\childdocmain| should be left empty
(but it must be present).

%%%%%%%%%%%%%%%%%%%%%%%%%%%%%%%%%%%%%%%%
\DescribeMacro{\childdocof}
Furthermore, add the commands
\begin{center}
\begin{tabular}{l}
|\input{childdoc.def}|\\
|\childdocof{|\textit{main}|}|\\
\end{tabular}
\end{center}
at the top of every child file \textit{child}
which is included by |\include{|\textit{child}|}|
from within the main file
(or at least for those files to be compiled individually).
The argument \textit{main} must be the filename of the main file.

There are a couple of
considerations in setting up the main and child documents:

%%%%%%%%%%%%%%%%%%%%%%%%%%%%%%%%%%%%%%%%
\paragraph{Restrictions.}

Please note the following restrictions:
\begin{itemize}
\item
|\childdocmain| must be called with one argument \textit{main}
to ensure compatibility with earlier version of the package.
It must either be empty (|\childdocmain{}|)
or precisely match the filename of the main file in which it is specified.
See \secref{sec:detection} for further information.
\item
The filename \textit{main} must be specified without the |.tex| extension.
\item
The filename \textit{main} is case sensitive
(even in case-insensitive file systems)
due to internal string comparison.
\item
The argument \textit{main} should be fully expanded, it cannot be a macro.
\item
Subdirectories and special characters should be avoided in filenames.
\item
The command |\childdocmain{|\textit{main}|}| must be followed by a whitespace.
It should not be followed immediately by another command
or by a comment mark `|%|'.
This is because the \TeX{} parser reads the token immediately following
the argument of |\childdocmain| and puts it
at the beginning of every child section;
however, a white\-space is ignored.
\end{itemize}

%%%%%%%%%%%%%%%%%%%%%%%%%%%%%%%%%%%%%%%%
\paragraph{Content of Main File.}

It is advisable to place all content in the child files included by |\include|.
Any output contained in the main file will appear in all child documents
unless suppressed manually;
it cannot be suppressed automatically by the |\includeonly| directive
and thus should normally be avoided.
A method to include some content in the main file
by means of conditional processing is described in \secref{sec:conditional}.

%%%%%%%%%%%%%%%%%%%%%%%%%%%%%%%%%%%%%%%%
\paragraph{Page Numbering.}

When only a part of the document is compiled,
the appropriate numbering of pages
(as well as other status parameters)
is determined from the |.aux| files.
The latter contain information from previous passes.
However this information needs to propagate through
all intermediate child documents.
Therefore the page numbering in child documents may well
be inconsistent until the complete document is compiled at least once.

A useful (if unconventional) way to always ensure a consistent
page numbering is to restart the numbering in each child document
and denote the pages by `\textit{child}|.|\textit{page}'
where \textit{child} represents the chapter/section number of the child file.
This can be achieved by the command
|\numberwithin{page}{|\textit{child}|}|
of the \textsf{amsmath} package
where \textit{child} can be |chapter| or |section|
depending on the chosen structuring.
Alternatively, one can modify the macro |\thepage| appropriately
and reset the counter |page| at the start of each child file.

%%%%%%%%%%%%%%%%%%%%%%%%%%%%%%%%%%%%%%%%%%%%%%%%%%%%%%%%%%%%%%%%%%%%%%%%%%%%%%%%
\subsection{Conditional Processing}
\label{sec:conditional}

The package provides a mechanism to compile different versions
of a document. To customise the versions further some conditional processing
can come in handy to distinguish which version is being compiled.
The package provides two macros to describe the compilation context:

%%%%%%%%%%%%%%%%%%%%%%%%%%%%%%%%%%%%%%%%
\DescribeMacro{\ifchilddoc}
The conditional |\ifchilddoc| distinguishes between the compilation of
child documents and the main document:
%
\begin{center}
|\ifchilddoc |\textit{child-code}| |[|\||else |\textit{main-code}]| \||fi|
\end{center}

%%%%%%%%%%%%%%%%%%%%%%%%%%%%%%%%%%%%%%%%
\DescribeMacro{\childdocname}
\DescribeMacro{\childdocjob}
The macro |\childdocname| contains the filename (without extension)
of the main or child file being processed.
Note that |\childdocjob| will always contain the name of the main file.

%%%%%%%%%%%%%%%%%%%%%%%%%%%%%%%%%%%%%%%%
\paragraph{Title Page.}

Conditional processing can be used to include a title or banner page
in the main document when proper precautions are taken.
Importantly, the code in the main file should ensure that the page counter
(as well as other status parameters which are stored in the |.aux| files)
takes the same value after the conditional processing.
Otherwise the page numbers may take divergent values
depending on which part is compiled.

For example, a title page could be declared by:
%
\begin{center}
\begin{tabular}{l}
|\ifchilddoc\||else|\\
|\addtocounter{page}{-1}|\\
\textit{code for title page}\\
|\newpage|\\
|\||fi|
\end{tabular}
\end{center}
%
A banner page for the child documents can be generated by:
%
\begin{center}
\begin{tabular}{l}
|\ifchilddoc|\\
|\addtocounter{page}{-1}|\\
\textit{code for banner page}\\
|\newpage|\\
|\||fi|
\end{tabular}
\end{center}
%
Here one could write a message such as:
\begin{center}
|This is the part \childdocname{} of \childdocjob{}.|
\end{center}

%%%%%%%%%%%%%%%%%%%%%%%%%%%%%%%%%%%%%%%%%%%%%%%%%%%%%%%%%%%%%%%%%%%%%%%%%%%%%%%%
\subsection{Flags}
\label{sec:flags}

The package makes it easy to generate different versions
of the main or child documents.
To this end compilation flags can be defined
and assigned different default values.
They will be particularly useful in conjunction
with the forwarding mechanism described in \secref{sec:forward}.

For example, it may be useful to have a flag |\version|
which can be set to |draft| or |final|.
The document source will contain some conditional code
depending on the value of |\version|.
Suppose further, the flag should default to |final| for the main file
and to |draft| for child files
which is a natural assignment for editing the document.
This is achieved by placing the following code
in the preamble of the main document
(below the |\childdocmain| directive):
%
\begin{center}
\begin{tabular}{l}
|\ifchilddoc|\\
|\providecommand{\version}{draft}|\\
|\||else|\\
|\providecommand{\version}{final}|\\
|\||fi|
\end{tabular}
\end{center}
%
The definition by |\providecommand| makes sure
that previous definitions are not overwritten.
Further statements |\providecommand{\version}{...}|
can thus be added before the above code to override it.

For the main file, one might add a line
(between |\childdocmain| and the above block)
%
\begin{center}
|%\ifchilddoc\||else\providecommand{\version}{draft}\||fi|
\end{center}
%
which can be uncommented to produce a draft version.
Likewise one can add a line to the very top of a child file
(above the |\childdocof{|\textit{main}|}| directive)
%
\begin{center}
|%\providecommand{\version}{final}|
\end{center}
%
which can be uncommented to produce the final version of this child document.

%%%%%%%%%%%%%%%%%%%%%%%%%%%%%%%%%%%%%%%%%%%%%%%%%%%%%%%%%%%%%%%%%%%%%%%%%%%%%%%%
\subsection{Forwarding}
\label{sec:forward}

Different versions of the main or child documents
using compilation flags as described in \secref{sec:flags}
can be (permanently) stored in different files
for convenient compilation, viewing and distribution.
To this end, the package defines a command
to pass on compilation to a different file:

%%%%%%%%%%%%%%%%%%%%%%%%%%%%%%%%%%%%%%%%
\DescribeMacro{\childdocforward}
The command |\childdocforward| redirects processing to
another source file:
%
\begin{center}
\begin{tabular}{l}
|\input{childdoc.def}|\\
|\childdocforward[|\textit{main}|]{|\textit{dest}|}|\\
\end{tabular}
\end{center}
%
The argument \textit{dest} is the destination file
(without extension).
It should be the main file or one of the child files.
Note that further \textsf{childdoc} directives
such as |\childdocof| and |\childdocforward|
in the indicated file will be processed in this form.
The optional argument \textit{main}
passes on directly to the main file \textit{main}
while pretending to compile the child \textit{dest}.
This form behaves as if \textit{dest}
issues |\childdocof{|\textit{main}|}| right away,
and no further \textsf{childdoc} directives will be processed.

%%%%%%%%%%%%%%%%%%%%%%%%%%%%%%%%%%%%%%%%
\DescribeMacro{\...prefix}
In the alternative form |\childdocforwardprefix|,
%
\begin{center}
\begin{tabular}{l}
|\input{childdoc.def}|\\
|\childdocforwardprefix[|\textit{main}|]{|\textit{prefix}|}{|\textit{dest}|}|
\end{tabular}
\end{center}
%
the destination file is determined by a pattern
depending on the current file:
To make this work, the current file must be called
`{\textit{prefix}\hspace{0.2em}\textit{suffix}}'
with \textit{prefix} matching precisely the argument.
Processing is then passed on to the file
`{\textit{dest}\hspace{0.2em}\textit{suffix}}'.
Surely, the same effect is achieved by
directly specifying the
argument `{\textit{dest}\hspace{0.2em}\textit{suffix}}'
in the first form.
However, that requires to set up a different file
for each child. With the alternative form of the command
all these files can have exactly the same content
which simplifies setting them up and maintaining them.

For example, the following file |draft.tex|
with a compilation flag |\version| as described in \secref{sec:flags}
compiles the main document as a draft:
%
\begin{center}
\begin{tabular}{l}
|\def\version{draft}|\\
|\input{childdoc.def}|\\
|\childdocforward{|\textit{main}|}|
\end{tabular}
\end{center}
%
Likewise, the following files |final|\textit{nn}|.tex|
compile the final version of the child document
|child|\textit{nn}|.tex|:
%
\begin{center}
\begin{tabular}{l}
|\def\version{final}|\\
|\input{childdoc.def}|\\
|\childdocforwardprefix{final}{child}|
\end{tabular}
\end{center}
%

Note that when several versions of a main file and/or of each child file
are to be generated, it may be convenient to set up a |Makefile| or
shell script to automatise the process.

%%%%%%%%%%%%%%%%%%%%%%%%%%%%%%%%%%%%%%%%%%%%%%%%%%%%%%%%%%%%%%%%%%%%%%%%%%%%%%%%
\subsection{Command Line Processing}
\label{sec:commandline}

The effect of redirection files can also be achieved by invoking
the \LaTeX{} compiler with a more elaborate command line.
Most conveniently this should be done as part
of a shell script or a |Makefile|.

When using \textsf{childdoc} in the main file, the following
command lines effectively perform a redirection
(note that depending on the shell being used,
backslashes may have to be doubled: `|\|' $\to$ `|\\|'):
%
\begin{center}
|... -jobname "|\textit{target}|" |\\|"|[\textit{flags}]%
|\input{childdoc.def}\childdocforward[|\textit{main}|]{|\textit{dest}|}"|
\end{center}
%
Here \textit{target} is the name of the output file,
\textit{main} is the name of the main file
and \textit{dest} is the name of the main or child file to be processed
(all filenames without extensions).
The optional argument \textit{main} can be omitted
if \textit{main} matches \textit{dest}.
Optionally, compilation \textit{flags} can be defined via |\def| commands.
This command line makes the \TeX{} engine believe
it is compiling the file \textit{target}
whose content is specified as the latter parameter.
The provided code then forwards the processing to
\textit{main} or \textit{dest} as described in \secref{sec:forward}.

%%%%%%%%%%%%%%%%%%%%%%%%%%%%%%%%%%%%%%%%%%%%%%%%%%%%%%%%%%%%%%%%%%%%%%%%%%%%%%%%
\subsection{Include by Input}
\label{sec:input}

Including child documents by |\include| has some restrictions by design.
Most notably, the content of a child document always occupies
its own set of pages; pages cannot be shared between child documents.
Usually, this behaviour makes perfect sense
because each child document contain an essential part of the document.
However, in some situations it may be desirable to compose
a document from a collection of parts
without having mandatory page breaks between then.
For this case, the package
provides a mechanism to include parts
by |\input| which can also be processed individually.
However, by construction this mechanism
requires manual handling of the content to be output.

%%%%%%%%%%%%%%%%%%%%%%%%%%%%%%%%%%%%%%%%
\DescribeMacro{\ifchilddocmanual}
The main file should be prepared as usual, see \secref{sec:include}.
However, the document body must make a distinction
between processing of an individual part and of the main document, e.g.:
%
\begin{center}
\begin{tabular}{l}
|\ifchilddocmanual|\\
|\input{\childdocname}|\\
|\||else|\\
\textit{document body with }|\input{|\textit{part}|}|\\
|\||fi|
\end{tabular}
\end{center}
%
The conditional |\ifchilddocmanual| is true whenever
a part to be included by |\input| is being compiled,
and the name of the part is stored in |\childdocname|.

%%%%%%%%%%%%%%%%%%%%%%%%%%%%%%%%%%%%%%%%
\DescribeMacro{\childdocby}
Each part to be included by |\input| should start with:
%
\begin{center}
\begin{tabular}{l}
|\input{childdoc.def}|\\
|\childdocby{|\textit{main}|}|\\
\end{tabular}
\end{center}
%
The directive |\childdocby| is similar to |\childdocof|
described in \secref{sec:include},
but the subsequent selection of content must be done manually.
To that end, both |\ifchilddoc| and |\ifchilddocmanual|
will be true upon processing of a part,
and the name of the part is stored in |\childdocname|.
Note that |\jobname| will be set to the filename of the current part
so that each part receives an individual |.aux| file
that does not interfere with the |.aux| file(s) of the main document.
This behaviour can be altered by the alternative form
|\childdocby[*]{|\textit{main}|}| (with a non-empty optional argument)
which uses the |.aux| file of the main document
by setting |\jobname| to \textit{main}.

%%%%%%%%%%%%%%%%%%%%%%%%%%%%%%%%%%%%%%%%%%%%%%%%%%%%%%%%%%%%%%%%%%%%%%%%%%%%%%%%
\subsection{Driver Development}
\label{sec:driver}

The \textsf{childdoc} mechanism can also be use for the development
of definition files such as \LaTeX{} styles or classes.
This case differs from the above setup with multiple parts
included by |\include| in that no |\includeonly| should be invoked.
This can be achieved by starting the include file
(before |\ProvidesPackage|) with:
%
\begin{center}
\begin{tabular}{l}
|\input{childdoc.def}|\\
|\childdocforward{|\textit{main}|}|\\
\end{tabular}
\end{center}
%
or alternatively with:
%
\begin{center}
\begin{tabular}{l}
|\input{childdoc.def}|\\
|\childdocby{|\textit{main}|}|\\
\end{tabular}
\end{center}
%
Both forms have slightly different effects as described above.
The main file is prepared as usual, see \secref{sec:include}.

%%%%%%%%%%%%%%%%%%%%%%%%%%%%%%%%%%%%%%%%%%%%%%%%%%%%%%%%%%%%%%%%%%%%%%%%%%%%%%%%
\subsection{Legacy Detection}
\label{sec:detection}

The directive |\childdocmain| in the main file can detect
whether the complete document or merely a child is to be compiled
even without using the directive |\childdocof|.
This method is deprecated because it is less robust
and there is no compelling reason to use it;
it is merely provided for backward compatibility
and it may be removed in future versions.

If the detection mechanism is to be used,
it is mandatory to correctly specify
the filename of the main file as the argument of |\childdocmain|:
%
\begin{center}
\begin{tabular}{l}
|\input{childdoc.def}|\\
|\childdocmain{|\textit{main}|}|\\
\end{tabular}
\end{center}
%
If |\jobname| does not match the argument \textit{main} of |\childdocmain|,
it is assumed that |\jobname| points to the child file to be compiled.
When using |\childdocmain| with the main file specified as argument,
it suffices to start a child file
with just |\input{|\textit{main}|}|
without loading of the package and using |\childdocof|.
If instead all processing is done
with the appropriate \textsf{childdoc} directives,
the argument of \textit{main} of |\childdocmain| can be empty.

An alternative version of the command line processing described
in \secref{sec:commandline} using the detection mechanism reads:
%
\begin{center}
|... -jobname "|\textit{target}|" "|[\textit{flags}]%
[|\def\jobname{|\textit{dest}|}|]|\input{|\textit{main}|}"|
\end{center}

%%%%%%%%%%%%%%%%%%%%%%%%%%%%%%%%%%%%%%%%%%%%%%%%%%%%%%%%%%%%%%%%%%%%%%%%%%%%%%%%
\subsection{Manual Code}
\label{sec:manual}

In case one cannot be certain whether the definitions file |childdoc.def|
is installed on the target \TeX{} distribution
and one prefers not to ship it,
it is conceivable to paste a few relevant commands into the sources.

To that end, drop all statements |\input{childdoc.def}|
and perform the replacements as outlined below.
Instead of |\childdocmain{|\textit{main}|}| add the following code
to the top of the main file:
%
\begin{center}
\begin{tabular}{l}
|\||ifdefined\childdocname\endinput\||fi\newif\ifchilddoc|\\
|\edef\childdocname{\scantokens\expandafter{\jobname\noexpand}}|\\
|\def\childdocmain{|\textit{main}|}\||ifx\childdocmain\childdocname\||else|\\
|\childdoctrue\includeonly{\childdocname}\let\jobname\childdocmain\||fi|\\
\end{tabular}
\end{center}
%
Instead of |\childdocof{|\textit{main}|}| just include the main file
at the top of each child file:
%
\begin{center}
|\input{|\textit{main}|}|
\end{center}
%
A simple redirection |\childdocforward{|\textit{dest}|}| is achieved by:
%
\begin{center}
|\def\jobname{|\textit{dest}|}\input{\jobname}|
\end{center}
%
The redirection with prefix
|\childdocforwardprefix[|\textit{prefix}|]{|\textit{dest}|}|
is accomplished by:
%
\begin{center}
\begin{tabular}{l}
|{\edef\jobname{\scantokens\expandafter{\jobname\noexpand}}|\\
|\def\redirectjob |\textit{prefix}|#1~~~{\gdef\jobname{|\textit{dest}|#1}}|\\
|\expandafter\redirectjob\jobname~~~}\input{\jobname}|
\end{tabular}
\end{center}

In an alternative approach,
child documents can be compiled by a specific command line
without additional code or specific definitions:
%
\begin{center}
|... -jobname "|\textit{target}|" "|[\textit{flags}]%
|\includeonly{|\textit{dest}|}\input{|\textit{main}|}"|
\end{center}
%

%%%%%%%%%%%%%%%%%%%%%%%%%%%%%%%%%%%%%%%%%%%%%%%%%%%%%%%%%%%%%%%%%%%%%%%%%%%%%%%%
%%%%%%%%%%%%%%%%%%%%%%%%%%%%%%%%%%%%%%%%%%%%%%%%%%%%%%%%%%%%%%%%%%%%%%%%%%%%%%%%
\section{Information}

%%%%%%%%%%%%%%%%%%%%%%%%%%%%%%%%%%%%%%%%%%%%%%%%%%%%%%%%%%%%%%%%%%%%%%%%%%%%%%%%
\subsection{Copyright}

Copyright \copyright{} 2017--2018 Niklas Beisert

This work may be distributed and/or modified under the
conditions of the \LaTeX{} Project Public License, either version 1.3
of this license or (at your option) any later version.
The latest version of this license is in
  \url{http://www.latex-project.org/lppl.txt}
and version 1.3 or later is part of all distributions of \LaTeX{}
version 2005/12/01 or later.

This work has the LPPL maintenance status `maintained'.

The Current Maintainer of this work is Niklas Beisert.

This work consists of the files |README.txt|, |childdoc.ins| and |childdoc.dtx|
as well as the derived files |childdoc.def|, |cdocsamp.tex|
with |cdocsch1.tex|, |cdocsch2.tex|, |cdocspt3.tex|, |cdocspt4.tex|,
|cdocsdrf.tex|, |cdocsfn1.tex|, |cdocsfn2.tex|
as well as |childdoc.pdf|.

%%%%%%%%%%%%%%%%%%%%%%%%%%%%%%%%%%%%%%%%%%%%%%%%%%%%%%%%%%%%%%%%%%%%%%%%%%%%%%%%
\subsection{Files and Installation}

The package consists of the files:
%
\begin{center}
\begin{tabular}{ll}
    |README.txt|   & readme file \\
    |childdoc.ins| & installation file \\
    |childdoc.dtx| & source file \\
    |childdoc.def| & definition file \\
    |cdocsamp.tex| & sample main file \\
    |cdocsch1.tex| & sample include file \\
    |cdocsch2.tex| & sample include file \\
    |cdocspt3.tex| & sample part file \\
    |cdocspt4.tex| & sample part file \\
    |cdocsdrf.tex| & sample redirection file \\
    |cdocsfn1.tex| & sample redirection file \\
    |cdocsfn2.tex| & sample redirection file \\
    |childdoc.pdf| & manual
\end{tabular}
\end{center}
%
The distribution consists of the files
|README.txt|, |childdoc.ins| and |childdoc.dtx|.
%
\begin{itemize}
\item
Run (pdf)\LaTeX{} on |childdoc.dtx|
to compile the manual |childdoc.pdf| (this file).
\item
Run \LaTeX{} on |childdoc.ins| to create the definitions file |childdoc.def|
and the sample |cdocsamp.tex| with include files
|cdocsch1.tex|, |cdocsch2.tex|, |cdocspt3.tex|, |cdocspt4.tex|,
|cdocsdrf.tex|, |cdocsfn1.tex|, |cdocsfn2.tex|.
Then copy the file |childdoc.def| to an appropriate directory of your \LaTeX{}
distribution, e.g.\ \textit{texmf-root}|/tex/latex/childdoc|.
\end{itemize}

%%%%%%%%%%%%%%%%%%%%%%%%%%%%%%%%%%%%%%%%%%%%%%%%%%%%%%%%%%%%%%%%%%%%%%%%%%%%%%%%
\subsection{Related CTAN Packages}

There are several other packages which offer a similar functionality:
%
\begin{itemize}
\item
The packages
\href{http://ctan.org/pkg/docmute}{\textsf{docmute}},
\href{http://ctan.org/pkg/includex}{\textsf{includex}} and
\href{http://ctan.org/pkg/standalone}{\textsf{standalone}}
provide commands to include only the document body of
a child file thus allowing both files to be compiled individually.
\item
The packages \href{http://ctan.org/pkg/subdocs}{\textsf{subdocs}}
and \href{http://ctan.org/pkg/subfiles}{\textsf{subfiles}}
provide structures in which the main and child documents can be
encapsulated and allowing them to be compiled individually.
The inclusion mechanism is different from the conventional |\include|.
\item
The package \href{http://ctan.org/pkg/combine}{\textsf{combine}}
is an elaborate solution to combine several documents into one.
\end{itemize}
%
See also the CTAN topic \href{http://ctan.org/topic/subdocs}{\textsf{subdocs}}
for further related packages.
The present package differs from the above solutions in that
a document structure constructed with the conventional |\include| mechanism
just needs two extra commands at the top of every file
such that all constituent files can be compiled individually.

%%%%%%%%%%%%%%%%%%%%%%%%%%%%%%%%%%%%%%%%%%%%%%%%%%%%%%%%%%%%%%%%%%%%%%%%%%%%%%%%
%\subsection{Feature Suggestions}
%
%The following is a list of features which may be useful for future
%versions of this package:
%%
%\begin{itemize}
%\item
%\ldots
%\end{itemize}

%%%%%%%%%%%%%%%%%%%%%%%%%%%%%%%%%%%%%%%%%%%%%%%%%%%%%%%%%%%%%%%%%%%%%%%%%%%%%%%%
\subsection{Revision History}

%%%%%%%%%%%%%%%%%%%%%%%%%%%%%%%%%%%%%%%%
\paragraph{v2.0:} 2018/12/30

\begin{itemize}
\item
immediate forward processing
\item
added |\childdocby| mechanism
\item
manual restructured
\end{itemize}

%%%%%%%%%%%%%%%%%%%%%%%%%%%%%%%%%%%%%%%%
\paragraph{v1.6:} 2018/01/17

\begin{itemize}
\item
application for development of include files
\item
corrections to manual
\end{itemize}

%%%%%%%%%%%%%%%%%%%%%%%%%%%%%%%%%%%%%%%%
\paragraph{v1.5:} 2017/05/21

\begin{itemize}
\item
more complete structuring introduced
\item
|\childdocof| introduced
\item
|\childdoc| renamed to |\childdocmain|
\item
|\childredirect| renamed to |\childdocforward| and |\childdocforwardprefix|
and functionality expanded
\end{itemize}

%%%%%%%%%%%%%%%%%%%%%%%%%%%%%%%%%%%%%%%%
\paragraph{v1.0:} 2017/04/27

\begin{itemize}
\item
manual and install package
\item
first version published on CTAN
\end{itemize}

%%%%%%%%%%%%%%%%%%%%%%%%%%%%%%%%%%%%%%%%
\paragraph{v0.6:} 2017/04/26

\begin{itemize}
\item
redirection mechanism added
\end{itemize}

%%%%%%%%%%%%%%%%%%%%%%%%%%%%%%%%%%%%%%%%
\paragraph{v0.5:} 2017/04/26

\begin{itemize}
\item
functionality in definition file
\end{itemize}


%%%%%%%%%%%%%%%%%%%%%%%%%%%%%%%%%%%%%%%%%%%%%%%%%%%%%%%%%%%%%%%%%%%%%%%%%%%%%%%%
%%%%%%%%%%%%%%%%%%%%%%%%%%%%%%%%%%%%%%%%%%%%%%%%%%%%%%%%%%%%%%%%%%%%%%%%%%%%%%%%
%%%%%%%%%%%%%%%%%%%%%%%%%%%%%%%%%%%%%%%%%%%%%%%%%%%%%%%%%%%%%%%%%%%%%%%%%%%%%%%%
\appendix

\settowidth\MacroIndent{\rmfamily\scriptsize 000\ }

 \DocInput{childdoc.dtx}

\end{document}
%</driver>
% \fi
%
% %%%%%%%%%%%%%%%%%%%%%%%%%%%%%%%%%%%%%%%%%%%%%%%%%%%%%%%%%%%%%%%%%%%%%%%%%%%%%%
% %%%%%%%%%%%%%%%%%%%%%%%%%%%%%%%%%%%%%%%%%%%%%%%%%%%%%%%%%%%%%%%%%%%%%%%%%%%%%%
% \section{Sample}
%\iffalse
%<*samplemain>
%\fi
%
% The following presents a sample document
% with two chapters, two parts, a title page,
% a compile flag as well as three forwarding files to set the flag.
% It consists of eight |.tex| files:
% \begin{center}
% \begin{tabular}{ll}
% |cdocsamp.tex|&main file\\
% |cdocsch1.tex|&include file for chapter 1\\
% |cdocsch2.tex|&include file for chapter 2\\
% |cdocspt3.tex|&include file for part 3\\
% |cdocspt4.tex|&include file for part 4\\
% |cdocsdrf.tex|&forwarding file for main file in draft mode\\
% |cdocsfi1.tex|&forwarding file for final version of chapter 1\\
% |cdocsfi2.tex|&forwarding file for final version of chapter 2\\
% \end{tabular}
% \end{center}
% Each of the eight files can be compiled directly by the \LaTeX{} compiler.
%
% %%%%%%%%%%%%%%%%%%%%%%%%%%%%%%%%%%%%%%
% \paragraph{Main File.}
%
% The main file is called |cdocsamp.tex|.
%
% Load the \textsf{childdoc} definitions and
% declare the filename for the main document:
%    \begin{macrocode}
\input{childdoc.def}
\childdocmain{}
%    \end{macrocode}

% Optional override for |\version| flag:
%    \begin{macrocode}
%%\ifchilddoc\else\providecommand{\version}{draft}\fi
%    \end{macrocode}

% Define the default values for the |\version| flag
% (|final| for the main file and |draft| for childs):
%    \begin{macrocode}
\ifchilddoc
\providecommand{\version}{draft}
\else
\providecommand{\version}{final}
\fi
%    \end{macrocode}

% Load the standard document class:
%    \begin{macrocode}
\documentclass[12pt]{article}
%    \end{macrocode}

% Start the document body:
%    \begin{macrocode}
\begin{document}
%    \end{macrocode}

% Declare a title page.
% Print title, part of document being processed and version flag:
%    \begin{macrocode}
\addtocounter{page}{-1}
\begin{center}
{\LARGE\bfseries{}childdoc example\par}
\vspace{1cm}
\ifchilddoc
\ifchilddocmanual part\else chapter\fi:
`\childdocname' of `\childdocjob'\par
\else
main document: `\childdocjob'\par
\fi
version: \version\par
\end{center}
\newpage
%    \end{macrocode}

% Manually include selected file,
% otherwise process as usual:
%    \begin{macrocode}
\ifchilddocmanual
\section*{part `\childdocname'}
\input{\childdocname}
\else
%    \end{macrocode}

% Include the two chapters:
%    \begin{macrocode}
\include{cdocsch1}
\include{cdocsch2}
%    \end{macrocode}

% Include the two parts unless only chapters should be displayed:
%    \begin{macrocode}
\ifchilddoc\else
\section{part three}
\input{cdocspt3}
\section{part four}
\input{cdocspt4}
\fi
%    \end{macrocode}

% Process as usual until here:
%    \begin{macrocode}
\fi
%    \end{macrocode}

% End of document body:
%    \begin{macrocode}
\end{document}
%    \end{macrocode}
%\iffalse
%</samplemain>
%\fi
%
% %%%%%%%%%%%%%%%%%%%%%%%%%%%%%%%%%%%%%%
% \paragraph{Chapter Include Files.}
%
% The include files are called |cdocsch1.tex| and |cdocsch2.tex|.
%
%\iffalse
%<*samplechap1|samplechap2>
%\fi

% Optional override for |\version| flag:
%    \begin{macrocode}
%%\providecommand{\version}{final}
%    \end{macrocode}

% Include the main document:
%    \begin{macrocode}
\input{childdoc.def}
\childdocof{cdocsamp}
%    \end{macrocode}

%\iffalse
%</samplechap1|samplechap2>
%\fi
%
%\iffalse
%<*samplechap1>
%\fi
% Some text for chapter 1:
%    \begin{macrocode}
\section{one}
some text in chapter one
%    \end{macrocode}

%\iffalse
%</samplechap1>
%\fi
% Some text for chapter 2:
%\iffalse
%<*samplechap2>
%\fi
%    \begin{macrocode}
\section{two}
more text in chapter two
%    \end{macrocode}

%\iffalse
%</samplechap2>
%\fi
%
% %%%%%%%%%%%%%%%%%%%%%%%%%%%%%%%%%%%%%%
% \paragraph{Part Include Files.}
%
% The include files are called |cdocspt3.tex| and |cdocspt4.tex|.
%
%\iffalse
%<*samplepart3|samplepart4>
%\fi

% Optional override for |\version| flag:
%    \begin{macrocode}
%%\providecommand{\version}{final}
%    \end{macrocode}

% Include the main document:
%    \begin{macrocode}
\input{childdoc.def}
\childdocby{cdocsamp}
%    \end{macrocode}

%\iffalse
%</samplepart3|samplepart4>
%\fi
%
%\iffalse
%<*samplepart3>
%\fi
% Some text for part 3:
%    \begin{macrocode}
some text in part three
%    \end{macrocode}

%\iffalse
%</samplepart3>
%\fi
% Some text for part 4:
%\iffalse
%<*samplepart4>
%\fi
%    \begin{macrocode}
more text in part four
%    \end{macrocode}

%\iffalse
%</samplepart4>
%\fi
%
% %%%%%%%%%%%%%%%%%%%%%%%%%%%%%%%%%%%%%%
% \paragraph{Forwarding for a Complete Draft.}
%
% The following forwarding file |cdocsdrf.tex|
% compiles the main document in draft mode:
%\iffalse
%<*sampledraft>
%\fi
%    \begin{macrocode}
\def\version{draft}
\input{childdoc.def}
\childdocforward{cdocsamp}
%    \end{macrocode}

%\iffalse
%</sampledraft>
%\fi
%
% %%%%%%%%%%%%%%%%%%%%%%%%%%%%%%%%%%%%%%
% \paragraph{Forwarding for Final Version of the Chapters.}
%
% The following forwarding files |cdocsfn1.tex| and |cdocsfn2.tex|
% (with identical content)
% compile the final versions of the child documents
% |cdocsch1.tex| and |cdocsch2.tex|, respectively:
%\iffalse
%<*samplefinal>
%\fi
%    \begin{macrocode}
\def\version{final}
\input{childdoc.def}
\childdocforwardprefix[cdocsamp]{cdocsfn}{cdocsch}
%    \end{macrocode}

%\iffalse
%</samplefinal>
%\fi
%
% %%%%%%%%%%%%%%%%%%%%%%%%%%%%%%%%%%%%%%
% \paragraph{Command Line Processing.}
%
% The following three command lines generate the output files
% |cdocscld|, |cdocscl1| and |cdocscl2|
% which should be identical to
% |cdocsdrf|, |cdocsch1| and |cdocsfn2|, respectively:
% \begin{center}
% \begin{tabular}{l}
% |latex -jobname cdocscld \|\\
% |  "\def\version{draft}\input{childdoc.def}\childdocforward{cdocsamp}"|\\
% |latex -jobname cdocscl1 \|\\
% |  "\input{childdoc.def}\childdocforward[cdocsamp]{cdocsch1}"|\\
% |latex -jobname cdocscl2 \|\\
% |  "\def\version{final}\input{childdoc.def}\childdocforward{cdocsch2}"|
% \end{tabular}
% \end{center}
% Note that the trailing backslash on each first line
% merely continues the input to the second line
% (for convenient cut ant paste).
% Furthermore, the command |latex| can be replaced by any
% of its alternative versions such as |pdflatex|.
%
% %%%%%%%%%%%%%%%%%%%%%%%%%%%%%%%%%%%%%%%%%%%%%%%%%%%%%%%%%%%%%%%%%%%%%%%%%%%%%%
% %%%%%%%%%%%%%%%%%%%%%%%%%%%%%%%%%%%%%%%%%%%%%%%%%%%%%%%%%%%%%%%%%%%%%%%%%%%%%%
% \section{Implementation}
%\iffalse
%<*package>
%\fi
%
% This section describes the definitions file |childdoc.def|.

% The definitions cannot be loaded using |\usepackage| or |\RequirePackage|
% which has a mechanism to prevent loading a style file more than once.
% When loading the definitions by means of |\input|
% multiple instances have to be prevented manually:
%\iffalse
%This code needs to be before the `\ProvidesFile' directive
%which is defined at the beginning of this file.
%Therefore it is also placed there and commented out here.
%</package>
%<*discard>
%\fi
%    \begin{macrocode}
\ifdefined\childdocmain\endinput\fi
%    \end{macrocode}
%\iffalse
%</discard>
%<*package>
%\fi
%
% \macro{\ifchilddoc}
% \macro{\ifchilddocmanual}
% The conditional |\ifchilddoc| tells whether a
% child (true) or main (false) document is being compiled.
% The conditional |\ifchilddocmanual| tells whether
% the |\includeonly| mechanism is used (false) or
% the selection of child files must be performed manually (true).
% The definitions initialise to false:
%    \begin{macrocode}
\newif\ifchilddoc
\newif\ifchilddocmanual
%    \end{macrocode}

% \macro{\childdocname}
% \macro{\childdocjob}
% The macro |\childdocname| stores the name of the main document
% to be compiled. The macro |\childdocjob| stores the name of
% the document on which the \LaTeX{} compiler was originally invoked.
% The content of |\jobname| cannot be compared
% to filenames specified in the source due to different catcodes.
% The following code rescans |\jobname|, stores the result
% in |\childdocname| and saves a copy in |\childdocjob|:
%    \begin{macrocode}
\edef\childdocname{\scantokens\expandafter{\jobname\noexpand}}
\let\childdocjob\childdocname
%    \end{macrocode}

% \macro{\childdocdisable}
% The macro |\childdocdisable| prevents the main file
% from being processed more than once.
% At this stage, the main document command |\childdocmain|
% is assumed to be called once again where it should do nothing.
% Any subsequent call to it should prevent
% a secondary processing of the main document
% It overwrites the forwarding commands
% |\childdocof| and |\childdocforward|
% with empty macros to prevent further inclusions of the main document:
%    \begin{macrocode}
\newcommand{\childdocdisable}
{
  \renewcommand{\childdocmain}[1]{\renewcommand{\childdocmain}[1]{\endinput}}
  \renewcommand{\childdocof}[1]{}
  \renewcommand{\childdocby}[2][]{}
  \renewcommand{\childdocforward}[2][]{}
  \renewcommand{\childdocdisable}{}
}
%    \end{macrocode}

% \macro{\childdocmain}
% The macro |\childdocmain| is to be called at the top of the main file
% with nothing or the main filename (without extension) as argument.
% First, it breaks loops.
% If the argument is not empty and does not match |\childdocname|
% (which is set by the first inclusion of |childdoc.def|),
% |\ifchilddoc| is set to true, |\includeonly| is applied to the child file
% and |\jobname| is set to the main file
% (for proper handling of |.aux| files):
%    \begin{macrocode}
\newcommand{\childdocmain}[1]
{
  \childdocdisable\childdocmain{}
  \if?#1?\else
    \begingroup
      \def\childdoctmp{#1}
      \ifx\childdoctmp\childdocname
        \def\childdoctmp{}
      \else
        \def\childdoctmp
        {
          \childdoctrue
          \includeonly{\childdocname}
          \def\childdocjob{#1}
          \def\jobname{#1}
        }
      \fi
      \expandafter
    \endgroup
    \childdoctmp
  \fi
}
%    \end{macrocode}

% \macro{\childdocof}
% The command |\childdocof| redirects
% compilation to the main file |#1|.
%    \begin{macrocode}
\newcommand{\childdocof}[1]
{
  \childdocdisable
  \childdoctrue
  \includeonly{\childdocname}
  \def\jobname{#1}
  \def\childdocjob{#1}
  \input{#1}
}
%    \end{macrocode}

% \macro{\childdocby}
% The command |\childdocby| ....
%    \begin{macrocode}
\newcommand{\childdocby}[2][]
{
  \childdocdisable
  \childdoctrue
  \childdocmanualtrue
  \if?#1?\else
    \def\jobname{#2}
  \fi
  \def\childdocjob{#2}
  \input{#2}
  \endinput
}
%    \end{macrocode}

% \macro{\childdocforward}
% The command |\childdocforward| redirects
% compilation to the main file or
% (if the optional argument is given) a child file.
% Parameters are set as if the main file
% or a child file starting with |\childdocof| was compiled.
% Then compilation is handed over to the main file:
%    \begin{macrocode}
\newcommand{\childdocforward}[2][]
{
  \begingroup
    \if?#1?
      \def\childdoctmp
      {
        \def\childdocname{#2}
        \def\childdocjob{#2}
        \def\jobname{#2}
        \input{#2}
        \endinput
      }
    \else
      \def\childdoctmp
      {
        \childdocdisable
        \def\childdocname{#2}
        \childdoctrue
        \includeonly{#2}
        \def\childdocjob{#1}
        \def\jobname{#1}
        \input{#1}
        \endinput
      }
    \fi
    \expandafter
  \endgroup
  \childdoctmp
}
%    \end{macrocode}

% \macro{\childdocforwardprefix}
% The command |\childdocforwardprefix| redirects
% compilation to the main or a child file by means of a pattern.
% The prefix |#1| in the current filename is replaced by |#2|
% and the suffix of the current filename is kept
% (it is assumed that the filename does not contain the substring `|~~~|'
% which is used as a delimiter).
% Compilation is handed over to the new file by |\childdocforward|:
%    \begin{macrocode}
\newcommand{\childdocforwardprefix}[3][]
{
  \begingroup
    \def\childdocextract #2##1~~~{\def\childdoctmp{\childdocforward[#1]{#3##1}}}
    \expandafter\childdocextract\childdocname~~~
    \expandafter
  \endgroup
  \childdoctmp
}
%    \end{macrocode}

% \macro{\childdoc}
% The deprecated macro |\childdoc| is a legacy version of |\childdocmain|:
%    \begin{macrocode}
\newcommand{\childdoc}{\childdocmain}
%    \end{macrocode}

% \macro{\childdocredirect}
% The deprecated macro |\childdocredirect| is a legacy version
% of |\childdocforward| and |\childdocforwardprefix|:
%    \begin{macrocode}
\newcommand{\childdocredirect}[2][]
{
  \begingroup
    \if?#1?
      \def\childdoctmp{\childdocforward{#2}}
    \else
      \def\childdoctmp{\childdocforwardprefix{#1}{#2}}
    \fi
    \expandafter
  \endgroup
  \childdoctmp
}
%    \end{macrocode}

%\iffalse
%</package>
%\fi
%
\endinput
|\\
|\childdocforwardprefix{final}{child}|
\end{tabular}
\end{center}
%

Note that when several versions of a main file and/or of each child file
are to be generated, it may be convenient to set up a |Makefile| or
shell script to automatise the process.

%%%%%%%%%%%%%%%%%%%%%%%%%%%%%%%%%%%%%%%%%%%%%%%%%%%%%%%%%%%%%%%%%%%%%%%%%%%%%%%%
\subsection{Command Line Processing}
\label{sec:commandline}

The effect of redirection files can also be achieved by invoking
the \LaTeX{} compiler with a more elaborate command line.
Most conveniently this should be done as part
of a shell script or a |Makefile|.

When using \textsf{childdoc} in the main file, the following
command lines effectively perform a redirection
(note that depending on the shell being used,
backslashes may have to be doubled: `|\|' $\to$ `|\\|'):
%
\begin{center}
|... -jobname "|\textit{target}|" |\\|"|[\textit{flags}]%
|% \iffalse
%
% childdoc.dtx Copyright (C) 2017-2018 Niklas Beisert
%
% This work may be distributed and/or modified under the
% conditions of the LaTeX Project Public License, either version 1.3
% of this license or (at your option) any later version.
% The latest version of this license is in
%   http://www.latex-project.org/lppl.txt
% and version 1.3 or later is part of all distributions of LaTeX
% version 2005/12/01 or later.
%
% This work has the LPPL maintenance status `maintained'.
%
% The Current Maintainer of this work is Niklas Beisert.
%
% This work consists of the files childdoc.dtx and childdoc.ins
% and the derived files childdoc.def and cdocsamp.tex with
% cdocsch1.tex, cdocsch2.tex, cdocsdrf.tex, cdocsfn1.tex, cdocsfn2.tex.
%
%<package>\ifdefined\childdocmain\endinput\fi
%<package>\ProvidesFile{childdoc.def}[2018/12/30 v2.0 child document driver]
%<samplemain>\ProvidesFile{cdocsamp.tex}[2018/12/30 v2.0 sample for childdoc]
%<*driver>
%\ProvidesFile{childdoc.drv}[2018/12/30 v2.0 childdoc reference manual file]
\PassOptionsToClass{10pt,a4paper}{article}
\documentclass{ltxdoc}

\usepackage[margin=35mm]{geometry}
\usepackage{hyperref}
\usepackage{hyperxmp}
\usepackage[usenames]{color}

\hypersetup{colorlinks=true}
\hypersetup{pdfstartview=FitH}
\hypersetup{pdfpagemode=UseNone}
\hypersetup{pdfsource={}}
\hypersetup{pdflang={en-UK}}
\hypersetup{pdfcopyright={Copyright 2017-2018 Niklas Beisert.
  This work may be distributed and/or modified under the
  conditions of the LaTeX Project Public License, either version 1.3
  of this license or (at your option) any later version.}}
\hypersetup{pdflicenseurl={http://www.latex-project.org/lppl.txt}}
\hypersetup{pdfcontactaddress={ETH Zurich, ITP, HIT K,
  Wolfgang-Pauli-Strasse 27}}
\hypersetup{pdfcontactpostcode={8093}}
\hypersetup{pdfcontactcity={Zurich}}
\hypersetup{pdfcontactcountry={Switzerland}}
\hypersetup{pdfcontactemail={nbeisert@itp.phys.ethz.ch}}
\hypersetup{pdfcontacturl={http://people.phys.ethz.ch/\xmptilde nbeisert/}}

\newcommand{\secref}[1]{\hyperref[#1]{section \ref*{#1}}}

\parskip1ex
\parindent0pt
\let\olditemize\itemize
\def\itemize{\olditemize\parskip0pt}

\begin{document}

\title{The \textsf{childdoc} Package}
\hypersetup{pdftitle={The childdoc Package}}
\author{Niklas Beisert\\[2ex]
  Institut f\"ur Theoretische Physik\\
  Eidgen\"ossische Technische Hochschule Z\"urich\\
  Wolfgang-Pauli-Strasse 27, 8093 Z\"urich, Switzerland\\[1ex]
  \href{mailto:nbeisert@itp.phys.ethz.ch}
  {\texttt{nbeisert@itp.phys.ethz.ch}}}
\hypersetup{pdfauthor={Niklas Beisert}}
\hypersetup{pdfsubject={Manual for the LaTeX2e Package childdoc}}
\date{30 December 2018, \textsf{v2.0}}
\maketitle

\begin{abstract}\noindent
\textsf{childdoc} is a \LaTeXe{} package
that enables the direct compilation
of document sections included by |\include|
to individual files.
\end{abstract}

\begingroup
\parskip0ex
\tableofcontents
\endgroup

%%%%%%%%%%%%%%%%%%%%%%%%%%%%%%%%%%%%%%%%%%%%%%%%%%%%%%%%%%%%%%%%%%%%%%%%%%%%%%%%
%%%%%%%%%%%%%%%%%%%%%%%%%%%%%%%%%%%%%%%%%%%%%%%%%%%%%%%%%%%%%%%%%%%%%%%%%%%%%%%%
\section{Introduction}

\LaTeX{} provides a mechanism to structure a large document (such as a book)
into a main file and several child files (containing the chapters)
using the |\include| command.
This mechanism is beneficial for documents
which span hundreds of pages in order to
make the source file(s) more manageable.
Moreover, compilation can be restricted to
selected child files by means of the |\includeonly| command.
The latter feature can be used to reduce the compilation time while editing
(this was significantly more useful in the earlier days of \LaTeX{})
or to generate a smaller document which is easier to navigate.
Another application of |\includeonly| is to generate
documents consisting of selected parts of the complete document.

However, there are a few drawbacks of the plain |\include| mechanism:
\begin{itemize}
\item
The child files cannot be compiled on their own,
they can only be compiled via the main file.
A naive editing environment
(such as a text editor with an option
to have the current file processed by \LaTeX)
may require one to switch to the main file before compiling;
attempting to compile the child file produces errors.
\item
The main file must be modified (each time)
to adjust the |\includeonly| command
to the present needs. This easily leaves the main file in a messy state.
\item
The generated document will always carry the filename
of the main document. This is inconvenient if
several child files are to be compiled and
to be kept for distribution.
\end{itemize}

The present package provides a simple interface
to make child files individually compilable by \LaTeX{}.
Compiling a child file then has the same effect as compiling
the main file with an |\includeonly| command
to select the appropriate child.
Moreover the generated document will carry the name of the child
rather than the main file.
This resolves all three above issues.

This feature is meant to make the editing of books,
thesis documents and lecture notes somewhat more convenient.
However, the package can also be used efficiently for
composing a series of documents (such as exercise sheets)
which are typically distributed individually.
It then assists the author in generating the individual documents
(potentially in different versions)
as well as a document containing the collected series.
Another application is in developing style files
or other kinds of included material
where compilation of the style file could redirect
to a sample or test file.

%%%%%%%%%%%%%%%%%%%%%%%%%%%%%%%%%%%%%%%%%%%%%%%%%%%%%%%%%%%%%%%%%%%%%%%%%%%%%%%%
%%%%%%%%%%%%%%%%%%%%%%%%%%%%%%%%%%%%%%%%%%%%%%%%%%%%%%%%%%%%%%%%%%%%%%%%%%%%%%%%
\section{Usage}

First of all, the package \textsf{childdoc} is \emph{not} a standard
\LaTeXe{} |.sty| style file! Therefore it needs to be invoked in
a non-standard way.

%%%%%%%%%%%%%%%%%%%%%%%%%%%%%%%%%%%%%%%%%%%%%%%%%%%%%%%%%%%%%%%%%%%%%%%%%%%%%%%%
\subsection{Included Files}
\label{sec:include}

%%%%%%%%%%%%%%%%%%%%%%%%%%%%%%%%%%%%%%%%
\DescribeMacro{\childdocmain}
To use the package, add the commands
\begin{center}
\begin{tabular}{l}
|\input{childdoc.def}|\\
|\childdocmain{}|\\
\end{tabular}
\end{center}
at the very top of the main \LaTeX{} file,
in particular \emph{before} the |\documentclass| statement!
The argument of |\childdocmain| should be left empty
(but it must be present).

%%%%%%%%%%%%%%%%%%%%%%%%%%%%%%%%%%%%%%%%
\DescribeMacro{\childdocof}
Furthermore, add the commands
\begin{center}
\begin{tabular}{l}
|\input{childdoc.def}|\\
|\childdocof{|\textit{main}|}|\\
\end{tabular}
\end{center}
at the top of every child file \textit{child}
which is included by |\include{|\textit{child}|}|
from within the main file
(or at least for those files to be compiled individually).
The argument \textit{main} must be the filename of the main file.

There are a couple of
considerations in setting up the main and child documents:

%%%%%%%%%%%%%%%%%%%%%%%%%%%%%%%%%%%%%%%%
\paragraph{Restrictions.}

Please note the following restrictions:
\begin{itemize}
\item
|\childdocmain| must be called with one argument \textit{main}
to ensure compatibility with earlier version of the package.
It must either be empty (|\childdocmain{}|)
or precisely match the filename of the main file in which it is specified.
See \secref{sec:detection} for further information.
\item
The filename \textit{main} must be specified without the |.tex| extension.
\item
The filename \textit{main} is case sensitive
(even in case-insensitive file systems)
due to internal string comparison.
\item
The argument \textit{main} should be fully expanded, it cannot be a macro.
\item
Subdirectories and special characters should be avoided in filenames.
\item
The command |\childdocmain{|\textit{main}|}| must be followed by a whitespace.
It should not be followed immediately by another command
or by a comment mark `|%|'.
This is because the \TeX{} parser reads the token immediately following
the argument of |\childdocmain| and puts it
at the beginning of every child section;
however, a white\-space is ignored.
\end{itemize}

%%%%%%%%%%%%%%%%%%%%%%%%%%%%%%%%%%%%%%%%
\paragraph{Content of Main File.}

It is advisable to place all content in the child files included by |\include|.
Any output contained in the main file will appear in all child documents
unless suppressed manually;
it cannot be suppressed automatically by the |\includeonly| directive
and thus should normally be avoided.
A method to include some content in the main file
by means of conditional processing is described in \secref{sec:conditional}.

%%%%%%%%%%%%%%%%%%%%%%%%%%%%%%%%%%%%%%%%
\paragraph{Page Numbering.}

When only a part of the document is compiled,
the appropriate numbering of pages
(as well as other status parameters)
is determined from the |.aux| files.
The latter contain information from previous passes.
However this information needs to propagate through
all intermediate child documents.
Therefore the page numbering in child documents may well
be inconsistent until the complete document is compiled at least once.

A useful (if unconventional) way to always ensure a consistent
page numbering is to restart the numbering in each child document
and denote the pages by `\textit{child}|.|\textit{page}'
where \textit{child} represents the chapter/section number of the child file.
This can be achieved by the command
|\numberwithin{page}{|\textit{child}|}|
of the \textsf{amsmath} package
where \textit{child} can be |chapter| or |section|
depending on the chosen structuring.
Alternatively, one can modify the macro |\thepage| appropriately
and reset the counter |page| at the start of each child file.

%%%%%%%%%%%%%%%%%%%%%%%%%%%%%%%%%%%%%%%%%%%%%%%%%%%%%%%%%%%%%%%%%%%%%%%%%%%%%%%%
\subsection{Conditional Processing}
\label{sec:conditional}

The package provides a mechanism to compile different versions
of a document. To customise the versions further some conditional processing
can come in handy to distinguish which version is being compiled.
The package provides two macros to describe the compilation context:

%%%%%%%%%%%%%%%%%%%%%%%%%%%%%%%%%%%%%%%%
\DescribeMacro{\ifchilddoc}
The conditional |\ifchilddoc| distinguishes between the compilation of
child documents and the main document:
%
\begin{center}
|\ifchilddoc |\textit{child-code}| |[|\||else |\textit{main-code}]| \||fi|
\end{center}

%%%%%%%%%%%%%%%%%%%%%%%%%%%%%%%%%%%%%%%%
\DescribeMacro{\childdocname}
\DescribeMacro{\childdocjob}
The macro |\childdocname| contains the filename (without extension)
of the main or child file being processed.
Note that |\childdocjob| will always contain the name of the main file.

%%%%%%%%%%%%%%%%%%%%%%%%%%%%%%%%%%%%%%%%
\paragraph{Title Page.}

Conditional processing can be used to include a title or banner page
in the main document when proper precautions are taken.
Importantly, the code in the main file should ensure that the page counter
(as well as other status parameters which are stored in the |.aux| files)
takes the same value after the conditional processing.
Otherwise the page numbers may take divergent values
depending on which part is compiled.

For example, a title page could be declared by:
%
\begin{center}
\begin{tabular}{l}
|\ifchilddoc\||else|\\
|\addtocounter{page}{-1}|\\
\textit{code for title page}\\
|\newpage|\\
|\||fi|
\end{tabular}
\end{center}
%
A banner page for the child documents can be generated by:
%
\begin{center}
\begin{tabular}{l}
|\ifchilddoc|\\
|\addtocounter{page}{-1}|\\
\textit{code for banner page}\\
|\newpage|\\
|\||fi|
\end{tabular}
\end{center}
%
Here one could write a message such as:
\begin{center}
|This is the part \childdocname{} of \childdocjob{}.|
\end{center}

%%%%%%%%%%%%%%%%%%%%%%%%%%%%%%%%%%%%%%%%%%%%%%%%%%%%%%%%%%%%%%%%%%%%%%%%%%%%%%%%
\subsection{Flags}
\label{sec:flags}

The package makes it easy to generate different versions
of the main or child documents.
To this end compilation flags can be defined
and assigned different default values.
They will be particularly useful in conjunction
with the forwarding mechanism described in \secref{sec:forward}.

For example, it may be useful to have a flag |\version|
which can be set to |draft| or |final|.
The document source will contain some conditional code
depending on the value of |\version|.
Suppose further, the flag should default to |final| for the main file
and to |draft| for child files
which is a natural assignment for editing the document.
This is achieved by placing the following code
in the preamble of the main document
(below the |\childdocmain| directive):
%
\begin{center}
\begin{tabular}{l}
|\ifchilddoc|\\
|\providecommand{\version}{draft}|\\
|\||else|\\
|\providecommand{\version}{final}|\\
|\||fi|
\end{tabular}
\end{center}
%
The definition by |\providecommand| makes sure
that previous definitions are not overwritten.
Further statements |\providecommand{\version}{...}|
can thus be added before the above code to override it.

For the main file, one might add a line
(between |\childdocmain| and the above block)
%
\begin{center}
|%\ifchilddoc\||else\providecommand{\version}{draft}\||fi|
\end{center}
%
which can be uncommented to produce a draft version.
Likewise one can add a line to the very top of a child file
(above the |\childdocof{|\textit{main}|}| directive)
%
\begin{center}
|%\providecommand{\version}{final}|
\end{center}
%
which can be uncommented to produce the final version of this child document.

%%%%%%%%%%%%%%%%%%%%%%%%%%%%%%%%%%%%%%%%%%%%%%%%%%%%%%%%%%%%%%%%%%%%%%%%%%%%%%%%
\subsection{Forwarding}
\label{sec:forward}

Different versions of the main or child documents
using compilation flags as described in \secref{sec:flags}
can be (permanently) stored in different files
for convenient compilation, viewing and distribution.
To this end, the package defines a command
to pass on compilation to a different file:

%%%%%%%%%%%%%%%%%%%%%%%%%%%%%%%%%%%%%%%%
\DescribeMacro{\childdocforward}
The command |\childdocforward| redirects processing to
another source file:
%
\begin{center}
\begin{tabular}{l}
|\input{childdoc.def}|\\
|\childdocforward[|\textit{main}|]{|\textit{dest}|}|\\
\end{tabular}
\end{center}
%
The argument \textit{dest} is the destination file
(without extension).
It should be the main file or one of the child files.
Note that further \textsf{childdoc} directives
such as |\childdocof| and |\childdocforward|
in the indicated file will be processed in this form.
The optional argument \textit{main}
passes on directly to the main file \textit{main}
while pretending to compile the child \textit{dest}.
This form behaves as if \textit{dest}
issues |\childdocof{|\textit{main}|}| right away,
and no further \textsf{childdoc} directives will be processed.

%%%%%%%%%%%%%%%%%%%%%%%%%%%%%%%%%%%%%%%%
\DescribeMacro{\...prefix}
In the alternative form |\childdocforwardprefix|,
%
\begin{center}
\begin{tabular}{l}
|\input{childdoc.def}|\\
|\childdocforwardprefix[|\textit{main}|]{|\textit{prefix}|}{|\textit{dest}|}|
\end{tabular}
\end{center}
%
the destination file is determined by a pattern
depending on the current file:
To make this work, the current file must be called
`{\textit{prefix}\hspace{0.2em}\textit{suffix}}'
with \textit{prefix} matching precisely the argument.
Processing is then passed on to the file
`{\textit{dest}\hspace{0.2em}\textit{suffix}}'.
Surely, the same effect is achieved by
directly specifying the
argument `{\textit{dest}\hspace{0.2em}\textit{suffix}}'
in the first form.
However, that requires to set up a different file
for each child. With the alternative form of the command
all these files can have exactly the same content
which simplifies setting them up and maintaining them.

For example, the following file |draft.tex|
with a compilation flag |\version| as described in \secref{sec:flags}
compiles the main document as a draft:
%
\begin{center}
\begin{tabular}{l}
|\def\version{draft}|\\
|\input{childdoc.def}|\\
|\childdocforward{|\textit{main}|}|
\end{tabular}
\end{center}
%
Likewise, the following files |final|\textit{nn}|.tex|
compile the final version of the child document
|child|\textit{nn}|.tex|:
%
\begin{center}
\begin{tabular}{l}
|\def\version{final}|\\
|\input{childdoc.def}|\\
|\childdocforwardprefix{final}{child}|
\end{tabular}
\end{center}
%

Note that when several versions of a main file and/or of each child file
are to be generated, it may be convenient to set up a |Makefile| or
shell script to automatise the process.

%%%%%%%%%%%%%%%%%%%%%%%%%%%%%%%%%%%%%%%%%%%%%%%%%%%%%%%%%%%%%%%%%%%%%%%%%%%%%%%%
\subsection{Command Line Processing}
\label{sec:commandline}

The effect of redirection files can also be achieved by invoking
the \LaTeX{} compiler with a more elaborate command line.
Most conveniently this should be done as part
of a shell script or a |Makefile|.

When using \textsf{childdoc} in the main file, the following
command lines effectively perform a redirection
(note that depending on the shell being used,
backslashes may have to be doubled: `|\|' $\to$ `|\\|'):
%
\begin{center}
|... -jobname "|\textit{target}|" |\\|"|[\textit{flags}]%
|\input{childdoc.def}\childdocforward[|\textit{main}|]{|\textit{dest}|}"|
\end{center}
%
Here \textit{target} is the name of the output file,
\textit{main} is the name of the main file
and \textit{dest} is the name of the main or child file to be processed
(all filenames without extensions).
The optional argument \textit{main} can be omitted
if \textit{main} matches \textit{dest}.
Optionally, compilation \textit{flags} can be defined via |\def| commands.
This command line makes the \TeX{} engine believe
it is compiling the file \textit{target}
whose content is specified as the latter parameter.
The provided code then forwards the processing to
\textit{main} or \textit{dest} as described in \secref{sec:forward}.

%%%%%%%%%%%%%%%%%%%%%%%%%%%%%%%%%%%%%%%%%%%%%%%%%%%%%%%%%%%%%%%%%%%%%%%%%%%%%%%%
\subsection{Include by Input}
\label{sec:input}

Including child documents by |\include| has some restrictions by design.
Most notably, the content of a child document always occupies
its own set of pages; pages cannot be shared between child documents.
Usually, this behaviour makes perfect sense
because each child document contain an essential part of the document.
However, in some situations it may be desirable to compose
a document from a collection of parts
without having mandatory page breaks between then.
For this case, the package
provides a mechanism to include parts
by |\input| which can also be processed individually.
However, by construction this mechanism
requires manual handling of the content to be output.

%%%%%%%%%%%%%%%%%%%%%%%%%%%%%%%%%%%%%%%%
\DescribeMacro{\ifchilddocmanual}
The main file should be prepared as usual, see \secref{sec:include}.
However, the document body must make a distinction
between processing of an individual part and of the main document, e.g.:
%
\begin{center}
\begin{tabular}{l}
|\ifchilddocmanual|\\
|\input{\childdocname}|\\
|\||else|\\
\textit{document body with }|\input{|\textit{part}|}|\\
|\||fi|
\end{tabular}
\end{center}
%
The conditional |\ifchilddocmanual| is true whenever
a part to be included by |\input| is being compiled,
and the name of the part is stored in |\childdocname|.

%%%%%%%%%%%%%%%%%%%%%%%%%%%%%%%%%%%%%%%%
\DescribeMacro{\childdocby}
Each part to be included by |\input| should start with:
%
\begin{center}
\begin{tabular}{l}
|\input{childdoc.def}|\\
|\childdocby{|\textit{main}|}|\\
\end{tabular}
\end{center}
%
The directive |\childdocby| is similar to |\childdocof|
described in \secref{sec:include},
but the subsequent selection of content must be done manually.
To that end, both |\ifchilddoc| and |\ifchilddocmanual|
will be true upon processing of a part,
and the name of the part is stored in |\childdocname|.
Note that |\jobname| will be set to the filename of the current part
so that each part receives an individual |.aux| file
that does not interfere with the |.aux| file(s) of the main document.
This behaviour can be altered by the alternative form
|\childdocby[*]{|\textit{main}|}| (with a non-empty optional argument)
which uses the |.aux| file of the main document
by setting |\jobname| to \textit{main}.

%%%%%%%%%%%%%%%%%%%%%%%%%%%%%%%%%%%%%%%%%%%%%%%%%%%%%%%%%%%%%%%%%%%%%%%%%%%%%%%%
\subsection{Driver Development}
\label{sec:driver}

The \textsf{childdoc} mechanism can also be use for the development
of definition files such as \LaTeX{} styles or classes.
This case differs from the above setup with multiple parts
included by |\include| in that no |\includeonly| should be invoked.
This can be achieved by starting the include file
(before |\ProvidesPackage|) with:
%
\begin{center}
\begin{tabular}{l}
|\input{childdoc.def}|\\
|\childdocforward{|\textit{main}|}|\\
\end{tabular}
\end{center}
%
or alternatively with:
%
\begin{center}
\begin{tabular}{l}
|\input{childdoc.def}|\\
|\childdocby{|\textit{main}|}|\\
\end{tabular}
\end{center}
%
Both forms have slightly different effects as described above.
The main file is prepared as usual, see \secref{sec:include}.

%%%%%%%%%%%%%%%%%%%%%%%%%%%%%%%%%%%%%%%%%%%%%%%%%%%%%%%%%%%%%%%%%%%%%%%%%%%%%%%%
\subsection{Legacy Detection}
\label{sec:detection}

The directive |\childdocmain| in the main file can detect
whether the complete document or merely a child is to be compiled
even without using the directive |\childdocof|.
This method is deprecated because it is less robust
and there is no compelling reason to use it;
it is merely provided for backward compatibility
and it may be removed in future versions.

If the detection mechanism is to be used,
it is mandatory to correctly specify
the filename of the main file as the argument of |\childdocmain|:
%
\begin{center}
\begin{tabular}{l}
|\input{childdoc.def}|\\
|\childdocmain{|\textit{main}|}|\\
\end{tabular}
\end{center}
%
If |\jobname| does not match the argument \textit{main} of |\childdocmain|,
it is assumed that |\jobname| points to the child file to be compiled.
When using |\childdocmain| with the main file specified as argument,
it suffices to start a child file
with just |\input{|\textit{main}|}|
without loading of the package and using |\childdocof|.
If instead all processing is done
with the appropriate \textsf{childdoc} directives,
the argument of \textit{main} of |\childdocmain| can be empty.

An alternative version of the command line processing described
in \secref{sec:commandline} using the detection mechanism reads:
%
\begin{center}
|... -jobname "|\textit{target}|" "|[\textit{flags}]%
[|\def\jobname{|\textit{dest}|}|]|\input{|\textit{main}|}"|
\end{center}

%%%%%%%%%%%%%%%%%%%%%%%%%%%%%%%%%%%%%%%%%%%%%%%%%%%%%%%%%%%%%%%%%%%%%%%%%%%%%%%%
\subsection{Manual Code}
\label{sec:manual}

In case one cannot be certain whether the definitions file |childdoc.def|
is installed on the target \TeX{} distribution
and one prefers not to ship it,
it is conceivable to paste a few relevant commands into the sources.

To that end, drop all statements |\input{childdoc.def}|
and perform the replacements as outlined below.
Instead of |\childdocmain{|\textit{main}|}| add the following code
to the top of the main file:
%
\begin{center}
\begin{tabular}{l}
|\||ifdefined\childdocname\endinput\||fi\newif\ifchilddoc|\\
|\edef\childdocname{\scantokens\expandafter{\jobname\noexpand}}|\\
|\def\childdocmain{|\textit{main}|}\||ifx\childdocmain\childdocname\||else|\\
|\childdoctrue\includeonly{\childdocname}\let\jobname\childdocmain\||fi|\\
\end{tabular}
\end{center}
%
Instead of |\childdocof{|\textit{main}|}| just include the main file
at the top of each child file:
%
\begin{center}
|\input{|\textit{main}|}|
\end{center}
%
A simple redirection |\childdocforward{|\textit{dest}|}| is achieved by:
%
\begin{center}
|\def\jobname{|\textit{dest}|}\input{\jobname}|
\end{center}
%
The redirection with prefix
|\childdocforwardprefix[|\textit{prefix}|]{|\textit{dest}|}|
is accomplished by:
%
\begin{center}
\begin{tabular}{l}
|{\edef\jobname{\scantokens\expandafter{\jobname\noexpand}}|\\
|\def\redirectjob |\textit{prefix}|#1~~~{\gdef\jobname{|\textit{dest}|#1}}|\\
|\expandafter\redirectjob\jobname~~~}\input{\jobname}|
\end{tabular}
\end{center}

In an alternative approach,
child documents can be compiled by a specific command line
without additional code or specific definitions:
%
\begin{center}
|... -jobname "|\textit{target}|" "|[\textit{flags}]%
|\includeonly{|\textit{dest}|}\input{|\textit{main}|}"|
\end{center}
%

%%%%%%%%%%%%%%%%%%%%%%%%%%%%%%%%%%%%%%%%%%%%%%%%%%%%%%%%%%%%%%%%%%%%%%%%%%%%%%%%
%%%%%%%%%%%%%%%%%%%%%%%%%%%%%%%%%%%%%%%%%%%%%%%%%%%%%%%%%%%%%%%%%%%%%%%%%%%%%%%%
\section{Information}

%%%%%%%%%%%%%%%%%%%%%%%%%%%%%%%%%%%%%%%%%%%%%%%%%%%%%%%%%%%%%%%%%%%%%%%%%%%%%%%%
\subsection{Copyright}

Copyright \copyright{} 2017--2018 Niklas Beisert

This work may be distributed and/or modified under the
conditions of the \LaTeX{} Project Public License, either version 1.3
of this license or (at your option) any later version.
The latest version of this license is in
  \url{http://www.latex-project.org/lppl.txt}
and version 1.3 or later is part of all distributions of \LaTeX{}
version 2005/12/01 or later.

This work has the LPPL maintenance status `maintained'.

The Current Maintainer of this work is Niklas Beisert.

This work consists of the files |README.txt|, |childdoc.ins| and |childdoc.dtx|
as well as the derived files |childdoc.def|, |cdocsamp.tex|
with |cdocsch1.tex|, |cdocsch2.tex|, |cdocspt3.tex|, |cdocspt4.tex|,
|cdocsdrf.tex|, |cdocsfn1.tex|, |cdocsfn2.tex|
as well as |childdoc.pdf|.

%%%%%%%%%%%%%%%%%%%%%%%%%%%%%%%%%%%%%%%%%%%%%%%%%%%%%%%%%%%%%%%%%%%%%%%%%%%%%%%%
\subsection{Files and Installation}

The package consists of the files:
%
\begin{center}
\begin{tabular}{ll}
    |README.txt|   & readme file \\
    |childdoc.ins| & installation file \\
    |childdoc.dtx| & source file \\
    |childdoc.def| & definition file \\
    |cdocsamp.tex| & sample main file \\
    |cdocsch1.tex| & sample include file \\
    |cdocsch2.tex| & sample include file \\
    |cdocspt3.tex| & sample part file \\
    |cdocspt4.tex| & sample part file \\
    |cdocsdrf.tex| & sample redirection file \\
    |cdocsfn1.tex| & sample redirection file \\
    |cdocsfn2.tex| & sample redirection file \\
    |childdoc.pdf| & manual
\end{tabular}
\end{center}
%
The distribution consists of the files
|README.txt|, |childdoc.ins| and |childdoc.dtx|.
%
\begin{itemize}
\item
Run (pdf)\LaTeX{} on |childdoc.dtx|
to compile the manual |childdoc.pdf| (this file).
\item
Run \LaTeX{} on |childdoc.ins| to create the definitions file |childdoc.def|
and the sample |cdocsamp.tex| with include files
|cdocsch1.tex|, |cdocsch2.tex|, |cdocspt3.tex|, |cdocspt4.tex|,
|cdocsdrf.tex|, |cdocsfn1.tex|, |cdocsfn2.tex|.
Then copy the file |childdoc.def| to an appropriate directory of your \LaTeX{}
distribution, e.g.\ \textit{texmf-root}|/tex/latex/childdoc|.
\end{itemize}

%%%%%%%%%%%%%%%%%%%%%%%%%%%%%%%%%%%%%%%%%%%%%%%%%%%%%%%%%%%%%%%%%%%%%%%%%%%%%%%%
\subsection{Related CTAN Packages}

There are several other packages which offer a similar functionality:
%
\begin{itemize}
\item
The packages
\href{http://ctan.org/pkg/docmute}{\textsf{docmute}},
\href{http://ctan.org/pkg/includex}{\textsf{includex}} and
\href{http://ctan.org/pkg/standalone}{\textsf{standalone}}
provide commands to include only the document body of
a child file thus allowing both files to be compiled individually.
\item
The packages \href{http://ctan.org/pkg/subdocs}{\textsf{subdocs}}
and \href{http://ctan.org/pkg/subfiles}{\textsf{subfiles}}
provide structures in which the main and child documents can be
encapsulated and allowing them to be compiled individually.
The inclusion mechanism is different from the conventional |\include|.
\item
The package \href{http://ctan.org/pkg/combine}{\textsf{combine}}
is an elaborate solution to combine several documents into one.
\end{itemize}
%
See also the CTAN topic \href{http://ctan.org/topic/subdocs}{\textsf{subdocs}}
for further related packages.
The present package differs from the above solutions in that
a document structure constructed with the conventional |\include| mechanism
just needs two extra commands at the top of every file
such that all constituent files can be compiled individually.

%%%%%%%%%%%%%%%%%%%%%%%%%%%%%%%%%%%%%%%%%%%%%%%%%%%%%%%%%%%%%%%%%%%%%%%%%%%%%%%%
%\subsection{Feature Suggestions}
%
%The following is a list of features which may be useful for future
%versions of this package:
%%
%\begin{itemize}
%\item
%\ldots
%\end{itemize}

%%%%%%%%%%%%%%%%%%%%%%%%%%%%%%%%%%%%%%%%%%%%%%%%%%%%%%%%%%%%%%%%%%%%%%%%%%%%%%%%
\subsection{Revision History}

%%%%%%%%%%%%%%%%%%%%%%%%%%%%%%%%%%%%%%%%
\paragraph{v2.0:} 2018/12/30

\begin{itemize}
\item
immediate forward processing
\item
added |\childdocby| mechanism
\item
manual restructured
\end{itemize}

%%%%%%%%%%%%%%%%%%%%%%%%%%%%%%%%%%%%%%%%
\paragraph{v1.6:} 2018/01/17

\begin{itemize}
\item
application for development of include files
\item
corrections to manual
\end{itemize}

%%%%%%%%%%%%%%%%%%%%%%%%%%%%%%%%%%%%%%%%
\paragraph{v1.5:} 2017/05/21

\begin{itemize}
\item
more complete structuring introduced
\item
|\childdocof| introduced
\item
|\childdoc| renamed to |\childdocmain|
\item
|\childredirect| renamed to |\childdocforward| and |\childdocforwardprefix|
and functionality expanded
\end{itemize}

%%%%%%%%%%%%%%%%%%%%%%%%%%%%%%%%%%%%%%%%
\paragraph{v1.0:} 2017/04/27

\begin{itemize}
\item
manual and install package
\item
first version published on CTAN
\end{itemize}

%%%%%%%%%%%%%%%%%%%%%%%%%%%%%%%%%%%%%%%%
\paragraph{v0.6:} 2017/04/26

\begin{itemize}
\item
redirection mechanism added
\end{itemize}

%%%%%%%%%%%%%%%%%%%%%%%%%%%%%%%%%%%%%%%%
\paragraph{v0.5:} 2017/04/26

\begin{itemize}
\item
functionality in definition file
\end{itemize}


%%%%%%%%%%%%%%%%%%%%%%%%%%%%%%%%%%%%%%%%%%%%%%%%%%%%%%%%%%%%%%%%%%%%%%%%%%%%%%%%
%%%%%%%%%%%%%%%%%%%%%%%%%%%%%%%%%%%%%%%%%%%%%%%%%%%%%%%%%%%%%%%%%%%%%%%%%%%%%%%%
%%%%%%%%%%%%%%%%%%%%%%%%%%%%%%%%%%%%%%%%%%%%%%%%%%%%%%%%%%%%%%%%%%%%%%%%%%%%%%%%
\appendix

\settowidth\MacroIndent{\rmfamily\scriptsize 000\ }

 \DocInput{childdoc.dtx}

\end{document}
%</driver>
% \fi
%
% %%%%%%%%%%%%%%%%%%%%%%%%%%%%%%%%%%%%%%%%%%%%%%%%%%%%%%%%%%%%%%%%%%%%%%%%%%%%%%
% %%%%%%%%%%%%%%%%%%%%%%%%%%%%%%%%%%%%%%%%%%%%%%%%%%%%%%%%%%%%%%%%%%%%%%%%%%%%%%
% \section{Sample}
%\iffalse
%<*samplemain>
%\fi
%
% The following presents a sample document
% with two chapters, two parts, a title page,
% a compile flag as well as three forwarding files to set the flag.
% It consists of eight |.tex| files:
% \begin{center}
% \begin{tabular}{ll}
% |cdocsamp.tex|&main file\\
% |cdocsch1.tex|&include file for chapter 1\\
% |cdocsch2.tex|&include file for chapter 2\\
% |cdocspt3.tex|&include file for part 3\\
% |cdocspt4.tex|&include file for part 4\\
% |cdocsdrf.tex|&forwarding file for main file in draft mode\\
% |cdocsfi1.tex|&forwarding file for final version of chapter 1\\
% |cdocsfi2.tex|&forwarding file for final version of chapter 2\\
% \end{tabular}
% \end{center}
% Each of the eight files can be compiled directly by the \LaTeX{} compiler.
%
% %%%%%%%%%%%%%%%%%%%%%%%%%%%%%%%%%%%%%%
% \paragraph{Main File.}
%
% The main file is called |cdocsamp.tex|.
%
% Load the \textsf{childdoc} definitions and
% declare the filename for the main document:
%    \begin{macrocode}
\input{childdoc.def}
\childdocmain{}
%    \end{macrocode}

% Optional override for |\version| flag:
%    \begin{macrocode}
%%\ifchilddoc\else\providecommand{\version}{draft}\fi
%    \end{macrocode}

% Define the default values for the |\version| flag
% (|final| for the main file and |draft| for childs):
%    \begin{macrocode}
\ifchilddoc
\providecommand{\version}{draft}
\else
\providecommand{\version}{final}
\fi
%    \end{macrocode}

% Load the standard document class:
%    \begin{macrocode}
\documentclass[12pt]{article}
%    \end{macrocode}

% Start the document body:
%    \begin{macrocode}
\begin{document}
%    \end{macrocode}

% Declare a title page.
% Print title, part of document being processed and version flag:
%    \begin{macrocode}
\addtocounter{page}{-1}
\begin{center}
{\LARGE\bfseries{}childdoc example\par}
\vspace{1cm}
\ifchilddoc
\ifchilddocmanual part\else chapter\fi:
`\childdocname' of `\childdocjob'\par
\else
main document: `\childdocjob'\par
\fi
version: \version\par
\end{center}
\newpage
%    \end{macrocode}

% Manually include selected file,
% otherwise process as usual:
%    \begin{macrocode}
\ifchilddocmanual
\section*{part `\childdocname'}
\input{\childdocname}
\else
%    \end{macrocode}

% Include the two chapters:
%    \begin{macrocode}
\include{cdocsch1}
\include{cdocsch2}
%    \end{macrocode}

% Include the two parts unless only chapters should be displayed:
%    \begin{macrocode}
\ifchilddoc\else
\section{part three}
\input{cdocspt3}
\section{part four}
\input{cdocspt4}
\fi
%    \end{macrocode}

% Process as usual until here:
%    \begin{macrocode}
\fi
%    \end{macrocode}

% End of document body:
%    \begin{macrocode}
\end{document}
%    \end{macrocode}
%\iffalse
%</samplemain>
%\fi
%
% %%%%%%%%%%%%%%%%%%%%%%%%%%%%%%%%%%%%%%
% \paragraph{Chapter Include Files.}
%
% The include files are called |cdocsch1.tex| and |cdocsch2.tex|.
%
%\iffalse
%<*samplechap1|samplechap2>
%\fi

% Optional override for |\version| flag:
%    \begin{macrocode}
%%\providecommand{\version}{final}
%    \end{macrocode}

% Include the main document:
%    \begin{macrocode}
\input{childdoc.def}
\childdocof{cdocsamp}
%    \end{macrocode}

%\iffalse
%</samplechap1|samplechap2>
%\fi
%
%\iffalse
%<*samplechap1>
%\fi
% Some text for chapter 1:
%    \begin{macrocode}
\section{one}
some text in chapter one
%    \end{macrocode}

%\iffalse
%</samplechap1>
%\fi
% Some text for chapter 2:
%\iffalse
%<*samplechap2>
%\fi
%    \begin{macrocode}
\section{two}
more text in chapter two
%    \end{macrocode}

%\iffalse
%</samplechap2>
%\fi
%
% %%%%%%%%%%%%%%%%%%%%%%%%%%%%%%%%%%%%%%
% \paragraph{Part Include Files.}
%
% The include files are called |cdocspt3.tex| and |cdocspt4.tex|.
%
%\iffalse
%<*samplepart3|samplepart4>
%\fi

% Optional override for |\version| flag:
%    \begin{macrocode}
%%\providecommand{\version}{final}
%    \end{macrocode}

% Include the main document:
%    \begin{macrocode}
\input{childdoc.def}
\childdocby{cdocsamp}
%    \end{macrocode}

%\iffalse
%</samplepart3|samplepart4>
%\fi
%
%\iffalse
%<*samplepart3>
%\fi
% Some text for part 3:
%    \begin{macrocode}
some text in part three
%    \end{macrocode}

%\iffalse
%</samplepart3>
%\fi
% Some text for part 4:
%\iffalse
%<*samplepart4>
%\fi
%    \begin{macrocode}
more text in part four
%    \end{macrocode}

%\iffalse
%</samplepart4>
%\fi
%
% %%%%%%%%%%%%%%%%%%%%%%%%%%%%%%%%%%%%%%
% \paragraph{Forwarding for a Complete Draft.}
%
% The following forwarding file |cdocsdrf.tex|
% compiles the main document in draft mode:
%\iffalse
%<*sampledraft>
%\fi
%    \begin{macrocode}
\def\version{draft}
\input{childdoc.def}
\childdocforward{cdocsamp}
%    \end{macrocode}

%\iffalse
%</sampledraft>
%\fi
%
% %%%%%%%%%%%%%%%%%%%%%%%%%%%%%%%%%%%%%%
% \paragraph{Forwarding for Final Version of the Chapters.}
%
% The following forwarding files |cdocsfn1.tex| and |cdocsfn2.tex|
% (with identical content)
% compile the final versions of the child documents
% |cdocsch1.tex| and |cdocsch2.tex|, respectively:
%\iffalse
%<*samplefinal>
%\fi
%    \begin{macrocode}
\def\version{final}
\input{childdoc.def}
\childdocforwardprefix[cdocsamp]{cdocsfn}{cdocsch}
%    \end{macrocode}

%\iffalse
%</samplefinal>
%\fi
%
% %%%%%%%%%%%%%%%%%%%%%%%%%%%%%%%%%%%%%%
% \paragraph{Command Line Processing.}
%
% The following three command lines generate the output files
% |cdocscld|, |cdocscl1| and |cdocscl2|
% which should be identical to
% |cdocsdrf|, |cdocsch1| and |cdocsfn2|, respectively:
% \begin{center}
% \begin{tabular}{l}
% |latex -jobname cdocscld \|\\
% |  "\def\version{draft}\input{childdoc.def}\childdocforward{cdocsamp}"|\\
% |latex -jobname cdocscl1 \|\\
% |  "\input{childdoc.def}\childdocforward[cdocsamp]{cdocsch1}"|\\
% |latex -jobname cdocscl2 \|\\
% |  "\def\version{final}\input{childdoc.def}\childdocforward{cdocsch2}"|
% \end{tabular}
% \end{center}
% Note that the trailing backslash on each first line
% merely continues the input to the second line
% (for convenient cut ant paste).
% Furthermore, the command |latex| can be replaced by any
% of its alternative versions such as |pdflatex|.
%
% %%%%%%%%%%%%%%%%%%%%%%%%%%%%%%%%%%%%%%%%%%%%%%%%%%%%%%%%%%%%%%%%%%%%%%%%%%%%%%
% %%%%%%%%%%%%%%%%%%%%%%%%%%%%%%%%%%%%%%%%%%%%%%%%%%%%%%%%%%%%%%%%%%%%%%%%%%%%%%
% \section{Implementation}
%\iffalse
%<*package>
%\fi
%
% This section describes the definitions file |childdoc.def|.

% The definitions cannot be loaded using |\usepackage| or |\RequirePackage|
% which has a mechanism to prevent loading a style file more than once.
% When loading the definitions by means of |\input|
% multiple instances have to be prevented manually:
%\iffalse
%This code needs to be before the `\ProvidesFile' directive
%which is defined at the beginning of this file.
%Therefore it is also placed there and commented out here.
%</package>
%<*discard>
%\fi
%    \begin{macrocode}
\ifdefined\childdocmain\endinput\fi
%    \end{macrocode}
%\iffalse
%</discard>
%<*package>
%\fi
%
% \macro{\ifchilddoc}
% \macro{\ifchilddocmanual}
% The conditional |\ifchilddoc| tells whether a
% child (true) or main (false) document is being compiled.
% The conditional |\ifchilddocmanual| tells whether
% the |\includeonly| mechanism is used (false) or
% the selection of child files must be performed manually (true).
% The definitions initialise to false:
%    \begin{macrocode}
\newif\ifchilddoc
\newif\ifchilddocmanual
%    \end{macrocode}

% \macro{\childdocname}
% \macro{\childdocjob}
% The macro |\childdocname| stores the name of the main document
% to be compiled. The macro |\childdocjob| stores the name of
% the document on which the \LaTeX{} compiler was originally invoked.
% The content of |\jobname| cannot be compared
% to filenames specified in the source due to different catcodes.
% The following code rescans |\jobname|, stores the result
% in |\childdocname| and saves a copy in |\childdocjob|:
%    \begin{macrocode}
\edef\childdocname{\scantokens\expandafter{\jobname\noexpand}}
\let\childdocjob\childdocname
%    \end{macrocode}

% \macro{\childdocdisable}
% The macro |\childdocdisable| prevents the main file
% from being processed more than once.
% At this stage, the main document command |\childdocmain|
% is assumed to be called once again where it should do nothing.
% Any subsequent call to it should prevent
% a secondary processing of the main document
% It overwrites the forwarding commands
% |\childdocof| and |\childdocforward|
% with empty macros to prevent further inclusions of the main document:
%    \begin{macrocode}
\newcommand{\childdocdisable}
{
  \renewcommand{\childdocmain}[1]{\renewcommand{\childdocmain}[1]{\endinput}}
  \renewcommand{\childdocof}[1]{}
  \renewcommand{\childdocby}[2][]{}
  \renewcommand{\childdocforward}[2][]{}
  \renewcommand{\childdocdisable}{}
}
%    \end{macrocode}

% \macro{\childdocmain}
% The macro |\childdocmain| is to be called at the top of the main file
% with nothing or the main filename (without extension) as argument.
% First, it breaks loops.
% If the argument is not empty and does not match |\childdocname|
% (which is set by the first inclusion of |childdoc.def|),
% |\ifchilddoc| is set to true, |\includeonly| is applied to the child file
% and |\jobname| is set to the main file
% (for proper handling of |.aux| files):
%    \begin{macrocode}
\newcommand{\childdocmain}[1]
{
  \childdocdisable\childdocmain{}
  \if?#1?\else
    \begingroup
      \def\childdoctmp{#1}
      \ifx\childdoctmp\childdocname
        \def\childdoctmp{}
      \else
        \def\childdoctmp
        {
          \childdoctrue
          \includeonly{\childdocname}
          \def\childdocjob{#1}
          \def\jobname{#1}
        }
      \fi
      \expandafter
    \endgroup
    \childdoctmp
  \fi
}
%    \end{macrocode}

% \macro{\childdocof}
% The command |\childdocof| redirects
% compilation to the main file |#1|.
%    \begin{macrocode}
\newcommand{\childdocof}[1]
{
  \childdocdisable
  \childdoctrue
  \includeonly{\childdocname}
  \def\jobname{#1}
  \def\childdocjob{#1}
  \input{#1}
}
%    \end{macrocode}

% \macro{\childdocby}
% The command |\childdocby| ....
%    \begin{macrocode}
\newcommand{\childdocby}[2][]
{
  \childdocdisable
  \childdoctrue
  \childdocmanualtrue
  \if?#1?\else
    \def\jobname{#2}
  \fi
  \def\childdocjob{#2}
  \input{#2}
  \endinput
}
%    \end{macrocode}

% \macro{\childdocforward}
% The command |\childdocforward| redirects
% compilation to the main file or
% (if the optional argument is given) a child file.
% Parameters are set as if the main file
% or a child file starting with |\childdocof| was compiled.
% Then compilation is handed over to the main file:
%    \begin{macrocode}
\newcommand{\childdocforward}[2][]
{
  \begingroup
    \if?#1?
      \def\childdoctmp
      {
        \def\childdocname{#2}
        \def\childdocjob{#2}
        \def\jobname{#2}
        \input{#2}
        \endinput
      }
    \else
      \def\childdoctmp
      {
        \childdocdisable
        \def\childdocname{#2}
        \childdoctrue
        \includeonly{#2}
        \def\childdocjob{#1}
        \def\jobname{#1}
        \input{#1}
        \endinput
      }
    \fi
    \expandafter
  \endgroup
  \childdoctmp
}
%    \end{macrocode}

% \macro{\childdocforwardprefix}
% The command |\childdocforwardprefix| redirects
% compilation to the main or a child file by means of a pattern.
% The prefix |#1| in the current filename is replaced by |#2|
% and the suffix of the current filename is kept
% (it is assumed that the filename does not contain the substring `|~~~|'
% which is used as a delimiter).
% Compilation is handed over to the new file by |\childdocforward|:
%    \begin{macrocode}
\newcommand{\childdocforwardprefix}[3][]
{
  \begingroup
    \def\childdocextract #2##1~~~{\def\childdoctmp{\childdocforward[#1]{#3##1}}}
    \expandafter\childdocextract\childdocname~~~
    \expandafter
  \endgroup
  \childdoctmp
}
%    \end{macrocode}

% \macro{\childdoc}
% The deprecated macro |\childdoc| is a legacy version of |\childdocmain|:
%    \begin{macrocode}
\newcommand{\childdoc}{\childdocmain}
%    \end{macrocode}

% \macro{\childdocredirect}
% The deprecated macro |\childdocredirect| is a legacy version
% of |\childdocforward| and |\childdocforwardprefix|:
%    \begin{macrocode}
\newcommand{\childdocredirect}[2][]
{
  \begingroup
    \if?#1?
      \def\childdoctmp{\childdocforward{#2}}
    \else
      \def\childdoctmp{\childdocforwardprefix{#1}{#2}}
    \fi
    \expandafter
  \endgroup
  \childdoctmp
}
%    \end{macrocode}

%\iffalse
%</package>
%\fi
%
\endinput
\childdocforward[|\textit{main}|]{|\textit{dest}|}"|
\end{center}
%
Here \textit{target} is the name of the output file,
\textit{main} is the name of the main file
and \textit{dest} is the name of the main or child file to be processed
(all filenames without extensions).
The optional argument \textit{main} can be omitted
if \textit{main} matches \textit{dest}.
Optionally, compilation \textit{flags} can be defined via |\def| commands.
This command line makes the \TeX{} engine believe
it is compiling the file \textit{target}
whose content is specified as the latter parameter.
The provided code then forwards the processing to
\textit{main} or \textit{dest} as described in \secref{sec:forward}.

%%%%%%%%%%%%%%%%%%%%%%%%%%%%%%%%%%%%%%%%%%%%%%%%%%%%%%%%%%%%%%%%%%%%%%%%%%%%%%%%
\subsection{Include by Input}
\label{sec:input}

Including child documents by |\include| has some restrictions by design.
Most notably, the content of a child document always occupies
its own set of pages; pages cannot be shared between child documents.
Usually, this behaviour makes perfect sense
because each child document contain an essential part of the document.
However, in some situations it may be desirable to compose
a document from a collection of parts
without having mandatory page breaks between then.
For this case, the package
provides a mechanism to include parts
by |\input| which can also be processed individually.
However, by construction this mechanism
requires manual handling of the content to be output.

%%%%%%%%%%%%%%%%%%%%%%%%%%%%%%%%%%%%%%%%
\DescribeMacro{\ifchilddocmanual}
The main file should be prepared as usual, see \secref{sec:include}.
However, the document body must make a distinction
between processing of an individual part and of the main document, e.g.:
%
\begin{center}
\begin{tabular}{l}
|\ifchilddocmanual|\\
|\input{\childdocname}|\\
|\||else|\\
\textit{document body with }|\input{|\textit{part}|}|\\
|\||fi|
\end{tabular}
\end{center}
%
The conditional |\ifchilddocmanual| is true whenever
a part to be included by |\input| is being compiled,
and the name of the part is stored in |\childdocname|.

%%%%%%%%%%%%%%%%%%%%%%%%%%%%%%%%%%%%%%%%
\DescribeMacro{\childdocby}
Each part to be included by |\input| should start with:
%
\begin{center}
\begin{tabular}{l}
|% \iffalse
%
% childdoc.dtx Copyright (C) 2017-2018 Niklas Beisert
%
% This work may be distributed and/or modified under the
% conditions of the LaTeX Project Public License, either version 1.3
% of this license or (at your option) any later version.
% The latest version of this license is in
%   http://www.latex-project.org/lppl.txt
% and version 1.3 or later is part of all distributions of LaTeX
% version 2005/12/01 or later.
%
% This work has the LPPL maintenance status `maintained'.
%
% The Current Maintainer of this work is Niklas Beisert.
%
% This work consists of the files childdoc.dtx and childdoc.ins
% and the derived files childdoc.def and cdocsamp.tex with
% cdocsch1.tex, cdocsch2.tex, cdocsdrf.tex, cdocsfn1.tex, cdocsfn2.tex.
%
%<package>\ifdefined\childdocmain\endinput\fi
%<package>\ProvidesFile{childdoc.def}[2018/12/30 v2.0 child document driver]
%<samplemain>\ProvidesFile{cdocsamp.tex}[2018/12/30 v2.0 sample for childdoc]
%<*driver>
%\ProvidesFile{childdoc.drv}[2018/12/30 v2.0 childdoc reference manual file]
\PassOptionsToClass{10pt,a4paper}{article}
\documentclass{ltxdoc}

\usepackage[margin=35mm]{geometry}
\usepackage{hyperref}
\usepackage{hyperxmp}
\usepackage[usenames]{color}

\hypersetup{colorlinks=true}
\hypersetup{pdfstartview=FitH}
\hypersetup{pdfpagemode=UseNone}
\hypersetup{pdfsource={}}
\hypersetup{pdflang={en-UK}}
\hypersetup{pdfcopyright={Copyright 2017-2018 Niklas Beisert.
  This work may be distributed and/or modified under the
  conditions of the LaTeX Project Public License, either version 1.3
  of this license or (at your option) any later version.}}
\hypersetup{pdflicenseurl={http://www.latex-project.org/lppl.txt}}
\hypersetup{pdfcontactaddress={ETH Zurich, ITP, HIT K,
  Wolfgang-Pauli-Strasse 27}}
\hypersetup{pdfcontactpostcode={8093}}
\hypersetup{pdfcontactcity={Zurich}}
\hypersetup{pdfcontactcountry={Switzerland}}
\hypersetup{pdfcontactemail={nbeisert@itp.phys.ethz.ch}}
\hypersetup{pdfcontacturl={http://people.phys.ethz.ch/\xmptilde nbeisert/}}

\newcommand{\secref}[1]{\hyperref[#1]{section \ref*{#1}}}

\parskip1ex
\parindent0pt
\let\olditemize\itemize
\def\itemize{\olditemize\parskip0pt}

\begin{document}

\title{The \textsf{childdoc} Package}
\hypersetup{pdftitle={The childdoc Package}}
\author{Niklas Beisert\\[2ex]
  Institut f\"ur Theoretische Physik\\
  Eidgen\"ossische Technische Hochschule Z\"urich\\
  Wolfgang-Pauli-Strasse 27, 8093 Z\"urich, Switzerland\\[1ex]
  \href{mailto:nbeisert@itp.phys.ethz.ch}
  {\texttt{nbeisert@itp.phys.ethz.ch}}}
\hypersetup{pdfauthor={Niklas Beisert}}
\hypersetup{pdfsubject={Manual for the LaTeX2e Package childdoc}}
\date{30 December 2018, \textsf{v2.0}}
\maketitle

\begin{abstract}\noindent
\textsf{childdoc} is a \LaTeXe{} package
that enables the direct compilation
of document sections included by |\include|
to individual files.
\end{abstract}

\begingroup
\parskip0ex
\tableofcontents
\endgroup

%%%%%%%%%%%%%%%%%%%%%%%%%%%%%%%%%%%%%%%%%%%%%%%%%%%%%%%%%%%%%%%%%%%%%%%%%%%%%%%%
%%%%%%%%%%%%%%%%%%%%%%%%%%%%%%%%%%%%%%%%%%%%%%%%%%%%%%%%%%%%%%%%%%%%%%%%%%%%%%%%
\section{Introduction}

\LaTeX{} provides a mechanism to structure a large document (such as a book)
into a main file and several child files (containing the chapters)
using the |\include| command.
This mechanism is beneficial for documents
which span hundreds of pages in order to
make the source file(s) more manageable.
Moreover, compilation can be restricted to
selected child files by means of the |\includeonly| command.
The latter feature can be used to reduce the compilation time while editing
(this was significantly more useful in the earlier days of \LaTeX{})
or to generate a smaller document which is easier to navigate.
Another application of |\includeonly| is to generate
documents consisting of selected parts of the complete document.

However, there are a few drawbacks of the plain |\include| mechanism:
\begin{itemize}
\item
The child files cannot be compiled on their own,
they can only be compiled via the main file.
A naive editing environment
(such as a text editor with an option
to have the current file processed by \LaTeX)
may require one to switch to the main file before compiling;
attempting to compile the child file produces errors.
\item
The main file must be modified (each time)
to adjust the |\includeonly| command
to the present needs. This easily leaves the main file in a messy state.
\item
The generated document will always carry the filename
of the main document. This is inconvenient if
several child files are to be compiled and
to be kept for distribution.
\end{itemize}

The present package provides a simple interface
to make child files individually compilable by \LaTeX{}.
Compiling a child file then has the same effect as compiling
the main file with an |\includeonly| command
to select the appropriate child.
Moreover the generated document will carry the name of the child
rather than the main file.
This resolves all three above issues.

This feature is meant to make the editing of books,
thesis documents and lecture notes somewhat more convenient.
However, the package can also be used efficiently for
composing a series of documents (such as exercise sheets)
which are typically distributed individually.
It then assists the author in generating the individual documents
(potentially in different versions)
as well as a document containing the collected series.
Another application is in developing style files
or other kinds of included material
where compilation of the style file could redirect
to a sample or test file.

%%%%%%%%%%%%%%%%%%%%%%%%%%%%%%%%%%%%%%%%%%%%%%%%%%%%%%%%%%%%%%%%%%%%%%%%%%%%%%%%
%%%%%%%%%%%%%%%%%%%%%%%%%%%%%%%%%%%%%%%%%%%%%%%%%%%%%%%%%%%%%%%%%%%%%%%%%%%%%%%%
\section{Usage}

First of all, the package \textsf{childdoc} is \emph{not} a standard
\LaTeXe{} |.sty| style file! Therefore it needs to be invoked in
a non-standard way.

%%%%%%%%%%%%%%%%%%%%%%%%%%%%%%%%%%%%%%%%%%%%%%%%%%%%%%%%%%%%%%%%%%%%%%%%%%%%%%%%
\subsection{Included Files}
\label{sec:include}

%%%%%%%%%%%%%%%%%%%%%%%%%%%%%%%%%%%%%%%%
\DescribeMacro{\childdocmain}
To use the package, add the commands
\begin{center}
\begin{tabular}{l}
|\input{childdoc.def}|\\
|\childdocmain{}|\\
\end{tabular}
\end{center}
at the very top of the main \LaTeX{} file,
in particular \emph{before} the |\documentclass| statement!
The argument of |\childdocmain| should be left empty
(but it must be present).

%%%%%%%%%%%%%%%%%%%%%%%%%%%%%%%%%%%%%%%%
\DescribeMacro{\childdocof}
Furthermore, add the commands
\begin{center}
\begin{tabular}{l}
|\input{childdoc.def}|\\
|\childdocof{|\textit{main}|}|\\
\end{tabular}
\end{center}
at the top of every child file \textit{child}
which is included by |\include{|\textit{child}|}|
from within the main file
(or at least for those files to be compiled individually).
The argument \textit{main} must be the filename of the main file.

There are a couple of
considerations in setting up the main and child documents:

%%%%%%%%%%%%%%%%%%%%%%%%%%%%%%%%%%%%%%%%
\paragraph{Restrictions.}

Please note the following restrictions:
\begin{itemize}
\item
|\childdocmain| must be called with one argument \textit{main}
to ensure compatibility with earlier version of the package.
It must either be empty (|\childdocmain{}|)
or precisely match the filename of the main file in which it is specified.
See \secref{sec:detection} for further information.
\item
The filename \textit{main} must be specified without the |.tex| extension.
\item
The filename \textit{main} is case sensitive
(even in case-insensitive file systems)
due to internal string comparison.
\item
The argument \textit{main} should be fully expanded, it cannot be a macro.
\item
Subdirectories and special characters should be avoided in filenames.
\item
The command |\childdocmain{|\textit{main}|}| must be followed by a whitespace.
It should not be followed immediately by another command
or by a comment mark `|%|'.
This is because the \TeX{} parser reads the token immediately following
the argument of |\childdocmain| and puts it
at the beginning of every child section;
however, a white\-space is ignored.
\end{itemize}

%%%%%%%%%%%%%%%%%%%%%%%%%%%%%%%%%%%%%%%%
\paragraph{Content of Main File.}

It is advisable to place all content in the child files included by |\include|.
Any output contained in the main file will appear in all child documents
unless suppressed manually;
it cannot be suppressed automatically by the |\includeonly| directive
and thus should normally be avoided.
A method to include some content in the main file
by means of conditional processing is described in \secref{sec:conditional}.

%%%%%%%%%%%%%%%%%%%%%%%%%%%%%%%%%%%%%%%%
\paragraph{Page Numbering.}

When only a part of the document is compiled,
the appropriate numbering of pages
(as well as other status parameters)
is determined from the |.aux| files.
The latter contain information from previous passes.
However this information needs to propagate through
all intermediate child documents.
Therefore the page numbering in child documents may well
be inconsistent until the complete document is compiled at least once.

A useful (if unconventional) way to always ensure a consistent
page numbering is to restart the numbering in each child document
and denote the pages by `\textit{child}|.|\textit{page}'
where \textit{child} represents the chapter/section number of the child file.
This can be achieved by the command
|\numberwithin{page}{|\textit{child}|}|
of the \textsf{amsmath} package
where \textit{child} can be |chapter| or |section|
depending on the chosen structuring.
Alternatively, one can modify the macro |\thepage| appropriately
and reset the counter |page| at the start of each child file.

%%%%%%%%%%%%%%%%%%%%%%%%%%%%%%%%%%%%%%%%%%%%%%%%%%%%%%%%%%%%%%%%%%%%%%%%%%%%%%%%
\subsection{Conditional Processing}
\label{sec:conditional}

The package provides a mechanism to compile different versions
of a document. To customise the versions further some conditional processing
can come in handy to distinguish which version is being compiled.
The package provides two macros to describe the compilation context:

%%%%%%%%%%%%%%%%%%%%%%%%%%%%%%%%%%%%%%%%
\DescribeMacro{\ifchilddoc}
The conditional |\ifchilddoc| distinguishes between the compilation of
child documents and the main document:
%
\begin{center}
|\ifchilddoc |\textit{child-code}| |[|\||else |\textit{main-code}]| \||fi|
\end{center}

%%%%%%%%%%%%%%%%%%%%%%%%%%%%%%%%%%%%%%%%
\DescribeMacro{\childdocname}
\DescribeMacro{\childdocjob}
The macro |\childdocname| contains the filename (without extension)
of the main or child file being processed.
Note that |\childdocjob| will always contain the name of the main file.

%%%%%%%%%%%%%%%%%%%%%%%%%%%%%%%%%%%%%%%%
\paragraph{Title Page.}

Conditional processing can be used to include a title or banner page
in the main document when proper precautions are taken.
Importantly, the code in the main file should ensure that the page counter
(as well as other status parameters which are stored in the |.aux| files)
takes the same value after the conditional processing.
Otherwise the page numbers may take divergent values
depending on which part is compiled.

For example, a title page could be declared by:
%
\begin{center}
\begin{tabular}{l}
|\ifchilddoc\||else|\\
|\addtocounter{page}{-1}|\\
\textit{code for title page}\\
|\newpage|\\
|\||fi|
\end{tabular}
\end{center}
%
A banner page for the child documents can be generated by:
%
\begin{center}
\begin{tabular}{l}
|\ifchilddoc|\\
|\addtocounter{page}{-1}|\\
\textit{code for banner page}\\
|\newpage|\\
|\||fi|
\end{tabular}
\end{center}
%
Here one could write a message such as:
\begin{center}
|This is the part \childdocname{} of \childdocjob{}.|
\end{center}

%%%%%%%%%%%%%%%%%%%%%%%%%%%%%%%%%%%%%%%%%%%%%%%%%%%%%%%%%%%%%%%%%%%%%%%%%%%%%%%%
\subsection{Flags}
\label{sec:flags}

The package makes it easy to generate different versions
of the main or child documents.
To this end compilation flags can be defined
and assigned different default values.
They will be particularly useful in conjunction
with the forwarding mechanism described in \secref{sec:forward}.

For example, it may be useful to have a flag |\version|
which can be set to |draft| or |final|.
The document source will contain some conditional code
depending on the value of |\version|.
Suppose further, the flag should default to |final| for the main file
and to |draft| for child files
which is a natural assignment for editing the document.
This is achieved by placing the following code
in the preamble of the main document
(below the |\childdocmain| directive):
%
\begin{center}
\begin{tabular}{l}
|\ifchilddoc|\\
|\providecommand{\version}{draft}|\\
|\||else|\\
|\providecommand{\version}{final}|\\
|\||fi|
\end{tabular}
\end{center}
%
The definition by |\providecommand| makes sure
that previous definitions are not overwritten.
Further statements |\providecommand{\version}{...}|
can thus be added before the above code to override it.

For the main file, one might add a line
(between |\childdocmain| and the above block)
%
\begin{center}
|%\ifchilddoc\||else\providecommand{\version}{draft}\||fi|
\end{center}
%
which can be uncommented to produce a draft version.
Likewise one can add a line to the very top of a child file
(above the |\childdocof{|\textit{main}|}| directive)
%
\begin{center}
|%\providecommand{\version}{final}|
\end{center}
%
which can be uncommented to produce the final version of this child document.

%%%%%%%%%%%%%%%%%%%%%%%%%%%%%%%%%%%%%%%%%%%%%%%%%%%%%%%%%%%%%%%%%%%%%%%%%%%%%%%%
\subsection{Forwarding}
\label{sec:forward}

Different versions of the main or child documents
using compilation flags as described in \secref{sec:flags}
can be (permanently) stored in different files
for convenient compilation, viewing and distribution.
To this end, the package defines a command
to pass on compilation to a different file:

%%%%%%%%%%%%%%%%%%%%%%%%%%%%%%%%%%%%%%%%
\DescribeMacro{\childdocforward}
The command |\childdocforward| redirects processing to
another source file:
%
\begin{center}
\begin{tabular}{l}
|\input{childdoc.def}|\\
|\childdocforward[|\textit{main}|]{|\textit{dest}|}|\\
\end{tabular}
\end{center}
%
The argument \textit{dest} is the destination file
(without extension).
It should be the main file or one of the child files.
Note that further \textsf{childdoc} directives
such as |\childdocof| and |\childdocforward|
in the indicated file will be processed in this form.
The optional argument \textit{main}
passes on directly to the main file \textit{main}
while pretending to compile the child \textit{dest}.
This form behaves as if \textit{dest}
issues |\childdocof{|\textit{main}|}| right away,
and no further \textsf{childdoc} directives will be processed.

%%%%%%%%%%%%%%%%%%%%%%%%%%%%%%%%%%%%%%%%
\DescribeMacro{\...prefix}
In the alternative form |\childdocforwardprefix|,
%
\begin{center}
\begin{tabular}{l}
|\input{childdoc.def}|\\
|\childdocforwardprefix[|\textit{main}|]{|\textit{prefix}|}{|\textit{dest}|}|
\end{tabular}
\end{center}
%
the destination file is determined by a pattern
depending on the current file:
To make this work, the current file must be called
`{\textit{prefix}\hspace{0.2em}\textit{suffix}}'
with \textit{prefix} matching precisely the argument.
Processing is then passed on to the file
`{\textit{dest}\hspace{0.2em}\textit{suffix}}'.
Surely, the same effect is achieved by
directly specifying the
argument `{\textit{dest}\hspace{0.2em}\textit{suffix}}'
in the first form.
However, that requires to set up a different file
for each child. With the alternative form of the command
all these files can have exactly the same content
which simplifies setting them up and maintaining them.

For example, the following file |draft.tex|
with a compilation flag |\version| as described in \secref{sec:flags}
compiles the main document as a draft:
%
\begin{center}
\begin{tabular}{l}
|\def\version{draft}|\\
|\input{childdoc.def}|\\
|\childdocforward{|\textit{main}|}|
\end{tabular}
\end{center}
%
Likewise, the following files |final|\textit{nn}|.tex|
compile the final version of the child document
|child|\textit{nn}|.tex|:
%
\begin{center}
\begin{tabular}{l}
|\def\version{final}|\\
|\input{childdoc.def}|\\
|\childdocforwardprefix{final}{child}|
\end{tabular}
\end{center}
%

Note that when several versions of a main file and/or of each child file
are to be generated, it may be convenient to set up a |Makefile| or
shell script to automatise the process.

%%%%%%%%%%%%%%%%%%%%%%%%%%%%%%%%%%%%%%%%%%%%%%%%%%%%%%%%%%%%%%%%%%%%%%%%%%%%%%%%
\subsection{Command Line Processing}
\label{sec:commandline}

The effect of redirection files can also be achieved by invoking
the \LaTeX{} compiler with a more elaborate command line.
Most conveniently this should be done as part
of a shell script or a |Makefile|.

When using \textsf{childdoc} in the main file, the following
command lines effectively perform a redirection
(note that depending on the shell being used,
backslashes may have to be doubled: `|\|' $\to$ `|\\|'):
%
\begin{center}
|... -jobname "|\textit{target}|" |\\|"|[\textit{flags}]%
|\input{childdoc.def}\childdocforward[|\textit{main}|]{|\textit{dest}|}"|
\end{center}
%
Here \textit{target} is the name of the output file,
\textit{main} is the name of the main file
and \textit{dest} is the name of the main or child file to be processed
(all filenames without extensions).
The optional argument \textit{main} can be omitted
if \textit{main} matches \textit{dest}.
Optionally, compilation \textit{flags} can be defined via |\def| commands.
This command line makes the \TeX{} engine believe
it is compiling the file \textit{target}
whose content is specified as the latter parameter.
The provided code then forwards the processing to
\textit{main} or \textit{dest} as described in \secref{sec:forward}.

%%%%%%%%%%%%%%%%%%%%%%%%%%%%%%%%%%%%%%%%%%%%%%%%%%%%%%%%%%%%%%%%%%%%%%%%%%%%%%%%
\subsection{Include by Input}
\label{sec:input}

Including child documents by |\include| has some restrictions by design.
Most notably, the content of a child document always occupies
its own set of pages; pages cannot be shared between child documents.
Usually, this behaviour makes perfect sense
because each child document contain an essential part of the document.
However, in some situations it may be desirable to compose
a document from a collection of parts
without having mandatory page breaks between then.
For this case, the package
provides a mechanism to include parts
by |\input| which can also be processed individually.
However, by construction this mechanism
requires manual handling of the content to be output.

%%%%%%%%%%%%%%%%%%%%%%%%%%%%%%%%%%%%%%%%
\DescribeMacro{\ifchilddocmanual}
The main file should be prepared as usual, see \secref{sec:include}.
However, the document body must make a distinction
between processing of an individual part and of the main document, e.g.:
%
\begin{center}
\begin{tabular}{l}
|\ifchilddocmanual|\\
|\input{\childdocname}|\\
|\||else|\\
\textit{document body with }|\input{|\textit{part}|}|\\
|\||fi|
\end{tabular}
\end{center}
%
The conditional |\ifchilddocmanual| is true whenever
a part to be included by |\input| is being compiled,
and the name of the part is stored in |\childdocname|.

%%%%%%%%%%%%%%%%%%%%%%%%%%%%%%%%%%%%%%%%
\DescribeMacro{\childdocby}
Each part to be included by |\input| should start with:
%
\begin{center}
\begin{tabular}{l}
|\input{childdoc.def}|\\
|\childdocby{|\textit{main}|}|\\
\end{tabular}
\end{center}
%
The directive |\childdocby| is similar to |\childdocof|
described in \secref{sec:include},
but the subsequent selection of content must be done manually.
To that end, both |\ifchilddoc| and |\ifchilddocmanual|
will be true upon processing of a part,
and the name of the part is stored in |\childdocname|.
Note that |\jobname| will be set to the filename of the current part
so that each part receives an individual |.aux| file
that does not interfere with the |.aux| file(s) of the main document.
This behaviour can be altered by the alternative form
|\childdocby[*]{|\textit{main}|}| (with a non-empty optional argument)
which uses the |.aux| file of the main document
by setting |\jobname| to \textit{main}.

%%%%%%%%%%%%%%%%%%%%%%%%%%%%%%%%%%%%%%%%%%%%%%%%%%%%%%%%%%%%%%%%%%%%%%%%%%%%%%%%
\subsection{Driver Development}
\label{sec:driver}

The \textsf{childdoc} mechanism can also be use for the development
of definition files such as \LaTeX{} styles or classes.
This case differs from the above setup with multiple parts
included by |\include| in that no |\includeonly| should be invoked.
This can be achieved by starting the include file
(before |\ProvidesPackage|) with:
%
\begin{center}
\begin{tabular}{l}
|\input{childdoc.def}|\\
|\childdocforward{|\textit{main}|}|\\
\end{tabular}
\end{center}
%
or alternatively with:
%
\begin{center}
\begin{tabular}{l}
|\input{childdoc.def}|\\
|\childdocby{|\textit{main}|}|\\
\end{tabular}
\end{center}
%
Both forms have slightly different effects as described above.
The main file is prepared as usual, see \secref{sec:include}.

%%%%%%%%%%%%%%%%%%%%%%%%%%%%%%%%%%%%%%%%%%%%%%%%%%%%%%%%%%%%%%%%%%%%%%%%%%%%%%%%
\subsection{Legacy Detection}
\label{sec:detection}

The directive |\childdocmain| in the main file can detect
whether the complete document or merely a child is to be compiled
even without using the directive |\childdocof|.
This method is deprecated because it is less robust
and there is no compelling reason to use it;
it is merely provided for backward compatibility
and it may be removed in future versions.

If the detection mechanism is to be used,
it is mandatory to correctly specify
the filename of the main file as the argument of |\childdocmain|:
%
\begin{center}
\begin{tabular}{l}
|\input{childdoc.def}|\\
|\childdocmain{|\textit{main}|}|\\
\end{tabular}
\end{center}
%
If |\jobname| does not match the argument \textit{main} of |\childdocmain|,
it is assumed that |\jobname| points to the child file to be compiled.
When using |\childdocmain| with the main file specified as argument,
it suffices to start a child file
with just |\input{|\textit{main}|}|
without loading of the package and using |\childdocof|.
If instead all processing is done
with the appropriate \textsf{childdoc} directives,
the argument of \textit{main} of |\childdocmain| can be empty.

An alternative version of the command line processing described
in \secref{sec:commandline} using the detection mechanism reads:
%
\begin{center}
|... -jobname "|\textit{target}|" "|[\textit{flags}]%
[|\def\jobname{|\textit{dest}|}|]|\input{|\textit{main}|}"|
\end{center}

%%%%%%%%%%%%%%%%%%%%%%%%%%%%%%%%%%%%%%%%%%%%%%%%%%%%%%%%%%%%%%%%%%%%%%%%%%%%%%%%
\subsection{Manual Code}
\label{sec:manual}

In case one cannot be certain whether the definitions file |childdoc.def|
is installed on the target \TeX{} distribution
and one prefers not to ship it,
it is conceivable to paste a few relevant commands into the sources.

To that end, drop all statements |\input{childdoc.def}|
and perform the replacements as outlined below.
Instead of |\childdocmain{|\textit{main}|}| add the following code
to the top of the main file:
%
\begin{center}
\begin{tabular}{l}
|\||ifdefined\childdocname\endinput\||fi\newif\ifchilddoc|\\
|\edef\childdocname{\scantokens\expandafter{\jobname\noexpand}}|\\
|\def\childdocmain{|\textit{main}|}\||ifx\childdocmain\childdocname\||else|\\
|\childdoctrue\includeonly{\childdocname}\let\jobname\childdocmain\||fi|\\
\end{tabular}
\end{center}
%
Instead of |\childdocof{|\textit{main}|}| just include the main file
at the top of each child file:
%
\begin{center}
|\input{|\textit{main}|}|
\end{center}
%
A simple redirection |\childdocforward{|\textit{dest}|}| is achieved by:
%
\begin{center}
|\def\jobname{|\textit{dest}|}\input{\jobname}|
\end{center}
%
The redirection with prefix
|\childdocforwardprefix[|\textit{prefix}|]{|\textit{dest}|}|
is accomplished by:
%
\begin{center}
\begin{tabular}{l}
|{\edef\jobname{\scantokens\expandafter{\jobname\noexpand}}|\\
|\def\redirectjob |\textit{prefix}|#1~~~{\gdef\jobname{|\textit{dest}|#1}}|\\
|\expandafter\redirectjob\jobname~~~}\input{\jobname}|
\end{tabular}
\end{center}

In an alternative approach,
child documents can be compiled by a specific command line
without additional code or specific definitions:
%
\begin{center}
|... -jobname "|\textit{target}|" "|[\textit{flags}]%
|\includeonly{|\textit{dest}|}\input{|\textit{main}|}"|
\end{center}
%

%%%%%%%%%%%%%%%%%%%%%%%%%%%%%%%%%%%%%%%%%%%%%%%%%%%%%%%%%%%%%%%%%%%%%%%%%%%%%%%%
%%%%%%%%%%%%%%%%%%%%%%%%%%%%%%%%%%%%%%%%%%%%%%%%%%%%%%%%%%%%%%%%%%%%%%%%%%%%%%%%
\section{Information}

%%%%%%%%%%%%%%%%%%%%%%%%%%%%%%%%%%%%%%%%%%%%%%%%%%%%%%%%%%%%%%%%%%%%%%%%%%%%%%%%
\subsection{Copyright}

Copyright \copyright{} 2017--2018 Niklas Beisert

This work may be distributed and/or modified under the
conditions of the \LaTeX{} Project Public License, either version 1.3
of this license or (at your option) any later version.
The latest version of this license is in
  \url{http://www.latex-project.org/lppl.txt}
and version 1.3 or later is part of all distributions of \LaTeX{}
version 2005/12/01 or later.

This work has the LPPL maintenance status `maintained'.

The Current Maintainer of this work is Niklas Beisert.

This work consists of the files |README.txt|, |childdoc.ins| and |childdoc.dtx|
as well as the derived files |childdoc.def|, |cdocsamp.tex|
with |cdocsch1.tex|, |cdocsch2.tex|, |cdocspt3.tex|, |cdocspt4.tex|,
|cdocsdrf.tex|, |cdocsfn1.tex|, |cdocsfn2.tex|
as well as |childdoc.pdf|.

%%%%%%%%%%%%%%%%%%%%%%%%%%%%%%%%%%%%%%%%%%%%%%%%%%%%%%%%%%%%%%%%%%%%%%%%%%%%%%%%
\subsection{Files and Installation}

The package consists of the files:
%
\begin{center}
\begin{tabular}{ll}
    |README.txt|   & readme file \\
    |childdoc.ins| & installation file \\
    |childdoc.dtx| & source file \\
    |childdoc.def| & definition file \\
    |cdocsamp.tex| & sample main file \\
    |cdocsch1.tex| & sample include file \\
    |cdocsch2.tex| & sample include file \\
    |cdocspt3.tex| & sample part file \\
    |cdocspt4.tex| & sample part file \\
    |cdocsdrf.tex| & sample redirection file \\
    |cdocsfn1.tex| & sample redirection file \\
    |cdocsfn2.tex| & sample redirection file \\
    |childdoc.pdf| & manual
\end{tabular}
\end{center}
%
The distribution consists of the files
|README.txt|, |childdoc.ins| and |childdoc.dtx|.
%
\begin{itemize}
\item
Run (pdf)\LaTeX{} on |childdoc.dtx|
to compile the manual |childdoc.pdf| (this file).
\item
Run \LaTeX{} on |childdoc.ins| to create the definitions file |childdoc.def|
and the sample |cdocsamp.tex| with include files
|cdocsch1.tex|, |cdocsch2.tex|, |cdocspt3.tex|, |cdocspt4.tex|,
|cdocsdrf.tex|, |cdocsfn1.tex|, |cdocsfn2.tex|.
Then copy the file |childdoc.def| to an appropriate directory of your \LaTeX{}
distribution, e.g.\ \textit{texmf-root}|/tex/latex/childdoc|.
\end{itemize}

%%%%%%%%%%%%%%%%%%%%%%%%%%%%%%%%%%%%%%%%%%%%%%%%%%%%%%%%%%%%%%%%%%%%%%%%%%%%%%%%
\subsection{Related CTAN Packages}

There are several other packages which offer a similar functionality:
%
\begin{itemize}
\item
The packages
\href{http://ctan.org/pkg/docmute}{\textsf{docmute}},
\href{http://ctan.org/pkg/includex}{\textsf{includex}} and
\href{http://ctan.org/pkg/standalone}{\textsf{standalone}}
provide commands to include only the document body of
a child file thus allowing both files to be compiled individually.
\item
The packages \href{http://ctan.org/pkg/subdocs}{\textsf{subdocs}}
and \href{http://ctan.org/pkg/subfiles}{\textsf{subfiles}}
provide structures in which the main and child documents can be
encapsulated and allowing them to be compiled individually.
The inclusion mechanism is different from the conventional |\include|.
\item
The package \href{http://ctan.org/pkg/combine}{\textsf{combine}}
is an elaborate solution to combine several documents into one.
\end{itemize}
%
See also the CTAN topic \href{http://ctan.org/topic/subdocs}{\textsf{subdocs}}
for further related packages.
The present package differs from the above solutions in that
a document structure constructed with the conventional |\include| mechanism
just needs two extra commands at the top of every file
such that all constituent files can be compiled individually.

%%%%%%%%%%%%%%%%%%%%%%%%%%%%%%%%%%%%%%%%%%%%%%%%%%%%%%%%%%%%%%%%%%%%%%%%%%%%%%%%
%\subsection{Feature Suggestions}
%
%The following is a list of features which may be useful for future
%versions of this package:
%%
%\begin{itemize}
%\item
%\ldots
%\end{itemize}

%%%%%%%%%%%%%%%%%%%%%%%%%%%%%%%%%%%%%%%%%%%%%%%%%%%%%%%%%%%%%%%%%%%%%%%%%%%%%%%%
\subsection{Revision History}

%%%%%%%%%%%%%%%%%%%%%%%%%%%%%%%%%%%%%%%%
\paragraph{v2.0:} 2018/12/30

\begin{itemize}
\item
immediate forward processing
\item
added |\childdocby| mechanism
\item
manual restructured
\end{itemize}

%%%%%%%%%%%%%%%%%%%%%%%%%%%%%%%%%%%%%%%%
\paragraph{v1.6:} 2018/01/17

\begin{itemize}
\item
application for development of include files
\item
corrections to manual
\end{itemize}

%%%%%%%%%%%%%%%%%%%%%%%%%%%%%%%%%%%%%%%%
\paragraph{v1.5:} 2017/05/21

\begin{itemize}
\item
more complete structuring introduced
\item
|\childdocof| introduced
\item
|\childdoc| renamed to |\childdocmain|
\item
|\childredirect| renamed to |\childdocforward| and |\childdocforwardprefix|
and functionality expanded
\end{itemize}

%%%%%%%%%%%%%%%%%%%%%%%%%%%%%%%%%%%%%%%%
\paragraph{v1.0:} 2017/04/27

\begin{itemize}
\item
manual and install package
\item
first version published on CTAN
\end{itemize}

%%%%%%%%%%%%%%%%%%%%%%%%%%%%%%%%%%%%%%%%
\paragraph{v0.6:} 2017/04/26

\begin{itemize}
\item
redirection mechanism added
\end{itemize}

%%%%%%%%%%%%%%%%%%%%%%%%%%%%%%%%%%%%%%%%
\paragraph{v0.5:} 2017/04/26

\begin{itemize}
\item
functionality in definition file
\end{itemize}


%%%%%%%%%%%%%%%%%%%%%%%%%%%%%%%%%%%%%%%%%%%%%%%%%%%%%%%%%%%%%%%%%%%%%%%%%%%%%%%%
%%%%%%%%%%%%%%%%%%%%%%%%%%%%%%%%%%%%%%%%%%%%%%%%%%%%%%%%%%%%%%%%%%%%%%%%%%%%%%%%
%%%%%%%%%%%%%%%%%%%%%%%%%%%%%%%%%%%%%%%%%%%%%%%%%%%%%%%%%%%%%%%%%%%%%%%%%%%%%%%%
\appendix

\settowidth\MacroIndent{\rmfamily\scriptsize 000\ }

 \DocInput{childdoc.dtx}

\end{document}
%</driver>
% \fi
%
% %%%%%%%%%%%%%%%%%%%%%%%%%%%%%%%%%%%%%%%%%%%%%%%%%%%%%%%%%%%%%%%%%%%%%%%%%%%%%%
% %%%%%%%%%%%%%%%%%%%%%%%%%%%%%%%%%%%%%%%%%%%%%%%%%%%%%%%%%%%%%%%%%%%%%%%%%%%%%%
% \section{Sample}
%\iffalse
%<*samplemain>
%\fi
%
% The following presents a sample document
% with two chapters, two parts, a title page,
% a compile flag as well as three forwarding files to set the flag.
% It consists of eight |.tex| files:
% \begin{center}
% \begin{tabular}{ll}
% |cdocsamp.tex|&main file\\
% |cdocsch1.tex|&include file for chapter 1\\
% |cdocsch2.tex|&include file for chapter 2\\
% |cdocspt3.tex|&include file for part 3\\
% |cdocspt4.tex|&include file for part 4\\
% |cdocsdrf.tex|&forwarding file for main file in draft mode\\
% |cdocsfi1.tex|&forwarding file for final version of chapter 1\\
% |cdocsfi2.tex|&forwarding file for final version of chapter 2\\
% \end{tabular}
% \end{center}
% Each of the eight files can be compiled directly by the \LaTeX{} compiler.
%
% %%%%%%%%%%%%%%%%%%%%%%%%%%%%%%%%%%%%%%
% \paragraph{Main File.}
%
% The main file is called |cdocsamp.tex|.
%
% Load the \textsf{childdoc} definitions and
% declare the filename for the main document:
%    \begin{macrocode}
\input{childdoc.def}
\childdocmain{}
%    \end{macrocode}

% Optional override for |\version| flag:
%    \begin{macrocode}
%%\ifchilddoc\else\providecommand{\version}{draft}\fi
%    \end{macrocode}

% Define the default values for the |\version| flag
% (|final| for the main file and |draft| for childs):
%    \begin{macrocode}
\ifchilddoc
\providecommand{\version}{draft}
\else
\providecommand{\version}{final}
\fi
%    \end{macrocode}

% Load the standard document class:
%    \begin{macrocode}
\documentclass[12pt]{article}
%    \end{macrocode}

% Start the document body:
%    \begin{macrocode}
\begin{document}
%    \end{macrocode}

% Declare a title page.
% Print title, part of document being processed and version flag:
%    \begin{macrocode}
\addtocounter{page}{-1}
\begin{center}
{\LARGE\bfseries{}childdoc example\par}
\vspace{1cm}
\ifchilddoc
\ifchilddocmanual part\else chapter\fi:
`\childdocname' of `\childdocjob'\par
\else
main document: `\childdocjob'\par
\fi
version: \version\par
\end{center}
\newpage
%    \end{macrocode}

% Manually include selected file,
% otherwise process as usual:
%    \begin{macrocode}
\ifchilddocmanual
\section*{part `\childdocname'}
\input{\childdocname}
\else
%    \end{macrocode}

% Include the two chapters:
%    \begin{macrocode}
\include{cdocsch1}
\include{cdocsch2}
%    \end{macrocode}

% Include the two parts unless only chapters should be displayed:
%    \begin{macrocode}
\ifchilddoc\else
\section{part three}
\input{cdocspt3}
\section{part four}
\input{cdocspt4}
\fi
%    \end{macrocode}

% Process as usual until here:
%    \begin{macrocode}
\fi
%    \end{macrocode}

% End of document body:
%    \begin{macrocode}
\end{document}
%    \end{macrocode}
%\iffalse
%</samplemain>
%\fi
%
% %%%%%%%%%%%%%%%%%%%%%%%%%%%%%%%%%%%%%%
% \paragraph{Chapter Include Files.}
%
% The include files are called |cdocsch1.tex| and |cdocsch2.tex|.
%
%\iffalse
%<*samplechap1|samplechap2>
%\fi

% Optional override for |\version| flag:
%    \begin{macrocode}
%%\providecommand{\version}{final}
%    \end{macrocode}

% Include the main document:
%    \begin{macrocode}
\input{childdoc.def}
\childdocof{cdocsamp}
%    \end{macrocode}

%\iffalse
%</samplechap1|samplechap2>
%\fi
%
%\iffalse
%<*samplechap1>
%\fi
% Some text for chapter 1:
%    \begin{macrocode}
\section{one}
some text in chapter one
%    \end{macrocode}

%\iffalse
%</samplechap1>
%\fi
% Some text for chapter 2:
%\iffalse
%<*samplechap2>
%\fi
%    \begin{macrocode}
\section{two}
more text in chapter two
%    \end{macrocode}

%\iffalse
%</samplechap2>
%\fi
%
% %%%%%%%%%%%%%%%%%%%%%%%%%%%%%%%%%%%%%%
% \paragraph{Part Include Files.}
%
% The include files are called |cdocspt3.tex| and |cdocspt4.tex|.
%
%\iffalse
%<*samplepart3|samplepart4>
%\fi

% Optional override for |\version| flag:
%    \begin{macrocode}
%%\providecommand{\version}{final}
%    \end{macrocode}

% Include the main document:
%    \begin{macrocode}
\input{childdoc.def}
\childdocby{cdocsamp}
%    \end{macrocode}

%\iffalse
%</samplepart3|samplepart4>
%\fi
%
%\iffalse
%<*samplepart3>
%\fi
% Some text for part 3:
%    \begin{macrocode}
some text in part three
%    \end{macrocode}

%\iffalse
%</samplepart3>
%\fi
% Some text for part 4:
%\iffalse
%<*samplepart4>
%\fi
%    \begin{macrocode}
more text in part four
%    \end{macrocode}

%\iffalse
%</samplepart4>
%\fi
%
% %%%%%%%%%%%%%%%%%%%%%%%%%%%%%%%%%%%%%%
% \paragraph{Forwarding for a Complete Draft.}
%
% The following forwarding file |cdocsdrf.tex|
% compiles the main document in draft mode:
%\iffalse
%<*sampledraft>
%\fi
%    \begin{macrocode}
\def\version{draft}
\input{childdoc.def}
\childdocforward{cdocsamp}
%    \end{macrocode}

%\iffalse
%</sampledraft>
%\fi
%
% %%%%%%%%%%%%%%%%%%%%%%%%%%%%%%%%%%%%%%
% \paragraph{Forwarding for Final Version of the Chapters.}
%
% The following forwarding files |cdocsfn1.tex| and |cdocsfn2.tex|
% (with identical content)
% compile the final versions of the child documents
% |cdocsch1.tex| and |cdocsch2.tex|, respectively:
%\iffalse
%<*samplefinal>
%\fi
%    \begin{macrocode}
\def\version{final}
\input{childdoc.def}
\childdocforwardprefix[cdocsamp]{cdocsfn}{cdocsch}
%    \end{macrocode}

%\iffalse
%</samplefinal>
%\fi
%
% %%%%%%%%%%%%%%%%%%%%%%%%%%%%%%%%%%%%%%
% \paragraph{Command Line Processing.}
%
% The following three command lines generate the output files
% |cdocscld|, |cdocscl1| and |cdocscl2|
% which should be identical to
% |cdocsdrf|, |cdocsch1| and |cdocsfn2|, respectively:
% \begin{center}
% \begin{tabular}{l}
% |latex -jobname cdocscld \|\\
% |  "\def\version{draft}\input{childdoc.def}\childdocforward{cdocsamp}"|\\
% |latex -jobname cdocscl1 \|\\
% |  "\input{childdoc.def}\childdocforward[cdocsamp]{cdocsch1}"|\\
% |latex -jobname cdocscl2 \|\\
% |  "\def\version{final}\input{childdoc.def}\childdocforward{cdocsch2}"|
% \end{tabular}
% \end{center}
% Note that the trailing backslash on each first line
% merely continues the input to the second line
% (for convenient cut ant paste).
% Furthermore, the command |latex| can be replaced by any
% of its alternative versions such as |pdflatex|.
%
% %%%%%%%%%%%%%%%%%%%%%%%%%%%%%%%%%%%%%%%%%%%%%%%%%%%%%%%%%%%%%%%%%%%%%%%%%%%%%%
% %%%%%%%%%%%%%%%%%%%%%%%%%%%%%%%%%%%%%%%%%%%%%%%%%%%%%%%%%%%%%%%%%%%%%%%%%%%%%%
% \section{Implementation}
%\iffalse
%<*package>
%\fi
%
% This section describes the definitions file |childdoc.def|.

% The definitions cannot be loaded using |\usepackage| or |\RequirePackage|
% which has a mechanism to prevent loading a style file more than once.
% When loading the definitions by means of |\input|
% multiple instances have to be prevented manually:
%\iffalse
%This code needs to be before the `\ProvidesFile' directive
%which is defined at the beginning of this file.
%Therefore it is also placed there and commented out here.
%</package>
%<*discard>
%\fi
%    \begin{macrocode}
\ifdefined\childdocmain\endinput\fi
%    \end{macrocode}
%\iffalse
%</discard>
%<*package>
%\fi
%
% \macro{\ifchilddoc}
% \macro{\ifchilddocmanual}
% The conditional |\ifchilddoc| tells whether a
% child (true) or main (false) document is being compiled.
% The conditional |\ifchilddocmanual| tells whether
% the |\includeonly| mechanism is used (false) or
% the selection of child files must be performed manually (true).
% The definitions initialise to false:
%    \begin{macrocode}
\newif\ifchilddoc
\newif\ifchilddocmanual
%    \end{macrocode}

% \macro{\childdocname}
% \macro{\childdocjob}
% The macro |\childdocname| stores the name of the main document
% to be compiled. The macro |\childdocjob| stores the name of
% the document on which the \LaTeX{} compiler was originally invoked.
% The content of |\jobname| cannot be compared
% to filenames specified in the source due to different catcodes.
% The following code rescans |\jobname|, stores the result
% in |\childdocname| and saves a copy in |\childdocjob|:
%    \begin{macrocode}
\edef\childdocname{\scantokens\expandafter{\jobname\noexpand}}
\let\childdocjob\childdocname
%    \end{macrocode}

% \macro{\childdocdisable}
% The macro |\childdocdisable| prevents the main file
% from being processed more than once.
% At this stage, the main document command |\childdocmain|
% is assumed to be called once again where it should do nothing.
% Any subsequent call to it should prevent
% a secondary processing of the main document
% It overwrites the forwarding commands
% |\childdocof| and |\childdocforward|
% with empty macros to prevent further inclusions of the main document:
%    \begin{macrocode}
\newcommand{\childdocdisable}
{
  \renewcommand{\childdocmain}[1]{\renewcommand{\childdocmain}[1]{\endinput}}
  \renewcommand{\childdocof}[1]{}
  \renewcommand{\childdocby}[2][]{}
  \renewcommand{\childdocforward}[2][]{}
  \renewcommand{\childdocdisable}{}
}
%    \end{macrocode}

% \macro{\childdocmain}
% The macro |\childdocmain| is to be called at the top of the main file
% with nothing or the main filename (without extension) as argument.
% First, it breaks loops.
% If the argument is not empty and does not match |\childdocname|
% (which is set by the first inclusion of |childdoc.def|),
% |\ifchilddoc| is set to true, |\includeonly| is applied to the child file
% and |\jobname| is set to the main file
% (for proper handling of |.aux| files):
%    \begin{macrocode}
\newcommand{\childdocmain}[1]
{
  \childdocdisable\childdocmain{}
  \if?#1?\else
    \begingroup
      \def\childdoctmp{#1}
      \ifx\childdoctmp\childdocname
        \def\childdoctmp{}
      \else
        \def\childdoctmp
        {
          \childdoctrue
          \includeonly{\childdocname}
          \def\childdocjob{#1}
          \def\jobname{#1}
        }
      \fi
      \expandafter
    \endgroup
    \childdoctmp
  \fi
}
%    \end{macrocode}

% \macro{\childdocof}
% The command |\childdocof| redirects
% compilation to the main file |#1|.
%    \begin{macrocode}
\newcommand{\childdocof}[1]
{
  \childdocdisable
  \childdoctrue
  \includeonly{\childdocname}
  \def\jobname{#1}
  \def\childdocjob{#1}
  \input{#1}
}
%    \end{macrocode}

% \macro{\childdocby}
% The command |\childdocby| ....
%    \begin{macrocode}
\newcommand{\childdocby}[2][]
{
  \childdocdisable
  \childdoctrue
  \childdocmanualtrue
  \if?#1?\else
    \def\jobname{#2}
  \fi
  \def\childdocjob{#2}
  \input{#2}
  \endinput
}
%    \end{macrocode}

% \macro{\childdocforward}
% The command |\childdocforward| redirects
% compilation to the main file or
% (if the optional argument is given) a child file.
% Parameters are set as if the main file
% or a child file starting with |\childdocof| was compiled.
% Then compilation is handed over to the main file:
%    \begin{macrocode}
\newcommand{\childdocforward}[2][]
{
  \begingroup
    \if?#1?
      \def\childdoctmp
      {
        \def\childdocname{#2}
        \def\childdocjob{#2}
        \def\jobname{#2}
        \input{#2}
        \endinput
      }
    \else
      \def\childdoctmp
      {
        \childdocdisable
        \def\childdocname{#2}
        \childdoctrue
        \includeonly{#2}
        \def\childdocjob{#1}
        \def\jobname{#1}
        \input{#1}
        \endinput
      }
    \fi
    \expandafter
  \endgroup
  \childdoctmp
}
%    \end{macrocode}

% \macro{\childdocforwardprefix}
% The command |\childdocforwardprefix| redirects
% compilation to the main or a child file by means of a pattern.
% The prefix |#1| in the current filename is replaced by |#2|
% and the suffix of the current filename is kept
% (it is assumed that the filename does not contain the substring `|~~~|'
% which is used as a delimiter).
% Compilation is handed over to the new file by |\childdocforward|:
%    \begin{macrocode}
\newcommand{\childdocforwardprefix}[3][]
{
  \begingroup
    \def\childdocextract #2##1~~~{\def\childdoctmp{\childdocforward[#1]{#3##1}}}
    \expandafter\childdocextract\childdocname~~~
    \expandafter
  \endgroup
  \childdoctmp
}
%    \end{macrocode}

% \macro{\childdoc}
% The deprecated macro |\childdoc| is a legacy version of |\childdocmain|:
%    \begin{macrocode}
\newcommand{\childdoc}{\childdocmain}
%    \end{macrocode}

% \macro{\childdocredirect}
% The deprecated macro |\childdocredirect| is a legacy version
% of |\childdocforward| and |\childdocforwardprefix|:
%    \begin{macrocode}
\newcommand{\childdocredirect}[2][]
{
  \begingroup
    \if?#1?
      \def\childdoctmp{\childdocforward{#2}}
    \else
      \def\childdoctmp{\childdocforwardprefix{#1}{#2}}
    \fi
    \expandafter
  \endgroup
  \childdoctmp
}
%    \end{macrocode}

%\iffalse
%</package>
%\fi
%
\endinput
|\\
|\childdocby{|\textit{main}|}|\\
\end{tabular}
\end{center}
%
The directive |\childdocby| is similar to |\childdocof|
described in \secref{sec:include},
but the subsequent selection of content must be done manually.
To that end, both |\ifchilddoc| and |\ifchilddocmanual|
will be true upon processing of a part,
and the name of the part is stored in |\childdocname|.
Note that |\jobname| will be set to the filename of the current part
so that each part receives an individual |.aux| file
that does not interfere with the |.aux| file(s) of the main document.
This behaviour can be altered by the alternative form
|\childdocby[*]{|\textit{main}|}| (with a non-empty optional argument)
which uses the |.aux| file of the main document
by setting |\jobname| to \textit{main}.

%%%%%%%%%%%%%%%%%%%%%%%%%%%%%%%%%%%%%%%%%%%%%%%%%%%%%%%%%%%%%%%%%%%%%%%%%%%%%%%%
\subsection{Driver Development}
\label{sec:driver}

The \textsf{childdoc} mechanism can also be use for the development
of definition files such as \LaTeX{} styles or classes.
This case differs from the above setup with multiple parts
included by |\include| in that no |\includeonly| should be invoked.
This can be achieved by starting the include file
(before |\ProvidesPackage|) with:
%
\begin{center}
\begin{tabular}{l}
|% \iffalse
%
% childdoc.dtx Copyright (C) 2017-2018 Niklas Beisert
%
% This work may be distributed and/or modified under the
% conditions of the LaTeX Project Public License, either version 1.3
% of this license or (at your option) any later version.
% The latest version of this license is in
%   http://www.latex-project.org/lppl.txt
% and version 1.3 or later is part of all distributions of LaTeX
% version 2005/12/01 or later.
%
% This work has the LPPL maintenance status `maintained'.
%
% The Current Maintainer of this work is Niklas Beisert.
%
% This work consists of the files childdoc.dtx and childdoc.ins
% and the derived files childdoc.def and cdocsamp.tex with
% cdocsch1.tex, cdocsch2.tex, cdocsdrf.tex, cdocsfn1.tex, cdocsfn2.tex.
%
%<package>\ifdefined\childdocmain\endinput\fi
%<package>\ProvidesFile{childdoc.def}[2018/12/30 v2.0 child document driver]
%<samplemain>\ProvidesFile{cdocsamp.tex}[2018/12/30 v2.0 sample for childdoc]
%<*driver>
%\ProvidesFile{childdoc.drv}[2018/12/30 v2.0 childdoc reference manual file]
\PassOptionsToClass{10pt,a4paper}{article}
\documentclass{ltxdoc}

\usepackage[margin=35mm]{geometry}
\usepackage{hyperref}
\usepackage{hyperxmp}
\usepackage[usenames]{color}

\hypersetup{colorlinks=true}
\hypersetup{pdfstartview=FitH}
\hypersetup{pdfpagemode=UseNone}
\hypersetup{pdfsource={}}
\hypersetup{pdflang={en-UK}}
\hypersetup{pdfcopyright={Copyright 2017-2018 Niklas Beisert.
  This work may be distributed and/or modified under the
  conditions of the LaTeX Project Public License, either version 1.3
  of this license or (at your option) any later version.}}
\hypersetup{pdflicenseurl={http://www.latex-project.org/lppl.txt}}
\hypersetup{pdfcontactaddress={ETH Zurich, ITP, HIT K,
  Wolfgang-Pauli-Strasse 27}}
\hypersetup{pdfcontactpostcode={8093}}
\hypersetup{pdfcontactcity={Zurich}}
\hypersetup{pdfcontactcountry={Switzerland}}
\hypersetup{pdfcontactemail={nbeisert@itp.phys.ethz.ch}}
\hypersetup{pdfcontacturl={http://people.phys.ethz.ch/\xmptilde nbeisert/}}

\newcommand{\secref}[1]{\hyperref[#1]{section \ref*{#1}}}

\parskip1ex
\parindent0pt
\let\olditemize\itemize
\def\itemize{\olditemize\parskip0pt}

\begin{document}

\title{The \textsf{childdoc} Package}
\hypersetup{pdftitle={The childdoc Package}}
\author{Niklas Beisert\\[2ex]
  Institut f\"ur Theoretische Physik\\
  Eidgen\"ossische Technische Hochschule Z\"urich\\
  Wolfgang-Pauli-Strasse 27, 8093 Z\"urich, Switzerland\\[1ex]
  \href{mailto:nbeisert@itp.phys.ethz.ch}
  {\texttt{nbeisert@itp.phys.ethz.ch}}}
\hypersetup{pdfauthor={Niklas Beisert}}
\hypersetup{pdfsubject={Manual for the LaTeX2e Package childdoc}}
\date{30 December 2018, \textsf{v2.0}}
\maketitle

\begin{abstract}\noindent
\textsf{childdoc} is a \LaTeXe{} package
that enables the direct compilation
of document sections included by |\include|
to individual files.
\end{abstract}

\begingroup
\parskip0ex
\tableofcontents
\endgroup

%%%%%%%%%%%%%%%%%%%%%%%%%%%%%%%%%%%%%%%%%%%%%%%%%%%%%%%%%%%%%%%%%%%%%%%%%%%%%%%%
%%%%%%%%%%%%%%%%%%%%%%%%%%%%%%%%%%%%%%%%%%%%%%%%%%%%%%%%%%%%%%%%%%%%%%%%%%%%%%%%
\section{Introduction}

\LaTeX{} provides a mechanism to structure a large document (such as a book)
into a main file and several child files (containing the chapters)
using the |\include| command.
This mechanism is beneficial for documents
which span hundreds of pages in order to
make the source file(s) more manageable.
Moreover, compilation can be restricted to
selected child files by means of the |\includeonly| command.
The latter feature can be used to reduce the compilation time while editing
(this was significantly more useful in the earlier days of \LaTeX{})
or to generate a smaller document which is easier to navigate.
Another application of |\includeonly| is to generate
documents consisting of selected parts of the complete document.

However, there are a few drawbacks of the plain |\include| mechanism:
\begin{itemize}
\item
The child files cannot be compiled on their own,
they can only be compiled via the main file.
A naive editing environment
(such as a text editor with an option
to have the current file processed by \LaTeX)
may require one to switch to the main file before compiling;
attempting to compile the child file produces errors.
\item
The main file must be modified (each time)
to adjust the |\includeonly| command
to the present needs. This easily leaves the main file in a messy state.
\item
The generated document will always carry the filename
of the main document. This is inconvenient if
several child files are to be compiled and
to be kept for distribution.
\end{itemize}

The present package provides a simple interface
to make child files individually compilable by \LaTeX{}.
Compiling a child file then has the same effect as compiling
the main file with an |\includeonly| command
to select the appropriate child.
Moreover the generated document will carry the name of the child
rather than the main file.
This resolves all three above issues.

This feature is meant to make the editing of books,
thesis documents and lecture notes somewhat more convenient.
However, the package can also be used efficiently for
composing a series of documents (such as exercise sheets)
which are typically distributed individually.
It then assists the author in generating the individual documents
(potentially in different versions)
as well as a document containing the collected series.
Another application is in developing style files
or other kinds of included material
where compilation of the style file could redirect
to a sample or test file.

%%%%%%%%%%%%%%%%%%%%%%%%%%%%%%%%%%%%%%%%%%%%%%%%%%%%%%%%%%%%%%%%%%%%%%%%%%%%%%%%
%%%%%%%%%%%%%%%%%%%%%%%%%%%%%%%%%%%%%%%%%%%%%%%%%%%%%%%%%%%%%%%%%%%%%%%%%%%%%%%%
\section{Usage}

First of all, the package \textsf{childdoc} is \emph{not} a standard
\LaTeXe{} |.sty| style file! Therefore it needs to be invoked in
a non-standard way.

%%%%%%%%%%%%%%%%%%%%%%%%%%%%%%%%%%%%%%%%%%%%%%%%%%%%%%%%%%%%%%%%%%%%%%%%%%%%%%%%
\subsection{Included Files}
\label{sec:include}

%%%%%%%%%%%%%%%%%%%%%%%%%%%%%%%%%%%%%%%%
\DescribeMacro{\childdocmain}
To use the package, add the commands
\begin{center}
\begin{tabular}{l}
|\input{childdoc.def}|\\
|\childdocmain{}|\\
\end{tabular}
\end{center}
at the very top of the main \LaTeX{} file,
in particular \emph{before} the |\documentclass| statement!
The argument of |\childdocmain| should be left empty
(but it must be present).

%%%%%%%%%%%%%%%%%%%%%%%%%%%%%%%%%%%%%%%%
\DescribeMacro{\childdocof}
Furthermore, add the commands
\begin{center}
\begin{tabular}{l}
|\input{childdoc.def}|\\
|\childdocof{|\textit{main}|}|\\
\end{tabular}
\end{center}
at the top of every child file \textit{child}
which is included by |\include{|\textit{child}|}|
from within the main file
(or at least for those files to be compiled individually).
The argument \textit{main} must be the filename of the main file.

There are a couple of
considerations in setting up the main and child documents:

%%%%%%%%%%%%%%%%%%%%%%%%%%%%%%%%%%%%%%%%
\paragraph{Restrictions.}

Please note the following restrictions:
\begin{itemize}
\item
|\childdocmain| must be called with one argument \textit{main}
to ensure compatibility with earlier version of the package.
It must either be empty (|\childdocmain{}|)
or precisely match the filename of the main file in which it is specified.
See \secref{sec:detection} for further information.
\item
The filename \textit{main} must be specified without the |.tex| extension.
\item
The filename \textit{main} is case sensitive
(even in case-insensitive file systems)
due to internal string comparison.
\item
The argument \textit{main} should be fully expanded, it cannot be a macro.
\item
Subdirectories and special characters should be avoided in filenames.
\item
The command |\childdocmain{|\textit{main}|}| must be followed by a whitespace.
It should not be followed immediately by another command
or by a comment mark `|%|'.
This is because the \TeX{} parser reads the token immediately following
the argument of |\childdocmain| and puts it
at the beginning of every child section;
however, a white\-space is ignored.
\end{itemize}

%%%%%%%%%%%%%%%%%%%%%%%%%%%%%%%%%%%%%%%%
\paragraph{Content of Main File.}

It is advisable to place all content in the child files included by |\include|.
Any output contained in the main file will appear in all child documents
unless suppressed manually;
it cannot be suppressed automatically by the |\includeonly| directive
and thus should normally be avoided.
A method to include some content in the main file
by means of conditional processing is described in \secref{sec:conditional}.

%%%%%%%%%%%%%%%%%%%%%%%%%%%%%%%%%%%%%%%%
\paragraph{Page Numbering.}

When only a part of the document is compiled,
the appropriate numbering of pages
(as well as other status parameters)
is determined from the |.aux| files.
The latter contain information from previous passes.
However this information needs to propagate through
all intermediate child documents.
Therefore the page numbering in child documents may well
be inconsistent until the complete document is compiled at least once.

A useful (if unconventional) way to always ensure a consistent
page numbering is to restart the numbering in each child document
and denote the pages by `\textit{child}|.|\textit{page}'
where \textit{child} represents the chapter/section number of the child file.
This can be achieved by the command
|\numberwithin{page}{|\textit{child}|}|
of the \textsf{amsmath} package
where \textit{child} can be |chapter| or |section|
depending on the chosen structuring.
Alternatively, one can modify the macro |\thepage| appropriately
and reset the counter |page| at the start of each child file.

%%%%%%%%%%%%%%%%%%%%%%%%%%%%%%%%%%%%%%%%%%%%%%%%%%%%%%%%%%%%%%%%%%%%%%%%%%%%%%%%
\subsection{Conditional Processing}
\label{sec:conditional}

The package provides a mechanism to compile different versions
of a document. To customise the versions further some conditional processing
can come in handy to distinguish which version is being compiled.
The package provides two macros to describe the compilation context:

%%%%%%%%%%%%%%%%%%%%%%%%%%%%%%%%%%%%%%%%
\DescribeMacro{\ifchilddoc}
The conditional |\ifchilddoc| distinguishes between the compilation of
child documents and the main document:
%
\begin{center}
|\ifchilddoc |\textit{child-code}| |[|\||else |\textit{main-code}]| \||fi|
\end{center}

%%%%%%%%%%%%%%%%%%%%%%%%%%%%%%%%%%%%%%%%
\DescribeMacro{\childdocname}
\DescribeMacro{\childdocjob}
The macro |\childdocname| contains the filename (without extension)
of the main or child file being processed.
Note that |\childdocjob| will always contain the name of the main file.

%%%%%%%%%%%%%%%%%%%%%%%%%%%%%%%%%%%%%%%%
\paragraph{Title Page.}

Conditional processing can be used to include a title or banner page
in the main document when proper precautions are taken.
Importantly, the code in the main file should ensure that the page counter
(as well as other status parameters which are stored in the |.aux| files)
takes the same value after the conditional processing.
Otherwise the page numbers may take divergent values
depending on which part is compiled.

For example, a title page could be declared by:
%
\begin{center}
\begin{tabular}{l}
|\ifchilddoc\||else|\\
|\addtocounter{page}{-1}|\\
\textit{code for title page}\\
|\newpage|\\
|\||fi|
\end{tabular}
\end{center}
%
A banner page for the child documents can be generated by:
%
\begin{center}
\begin{tabular}{l}
|\ifchilddoc|\\
|\addtocounter{page}{-1}|\\
\textit{code for banner page}\\
|\newpage|\\
|\||fi|
\end{tabular}
\end{center}
%
Here one could write a message such as:
\begin{center}
|This is the part \childdocname{} of \childdocjob{}.|
\end{center}

%%%%%%%%%%%%%%%%%%%%%%%%%%%%%%%%%%%%%%%%%%%%%%%%%%%%%%%%%%%%%%%%%%%%%%%%%%%%%%%%
\subsection{Flags}
\label{sec:flags}

The package makes it easy to generate different versions
of the main or child documents.
To this end compilation flags can be defined
and assigned different default values.
They will be particularly useful in conjunction
with the forwarding mechanism described in \secref{sec:forward}.

For example, it may be useful to have a flag |\version|
which can be set to |draft| or |final|.
The document source will contain some conditional code
depending on the value of |\version|.
Suppose further, the flag should default to |final| for the main file
and to |draft| for child files
which is a natural assignment for editing the document.
This is achieved by placing the following code
in the preamble of the main document
(below the |\childdocmain| directive):
%
\begin{center}
\begin{tabular}{l}
|\ifchilddoc|\\
|\providecommand{\version}{draft}|\\
|\||else|\\
|\providecommand{\version}{final}|\\
|\||fi|
\end{tabular}
\end{center}
%
The definition by |\providecommand| makes sure
that previous definitions are not overwritten.
Further statements |\providecommand{\version}{...}|
can thus be added before the above code to override it.

For the main file, one might add a line
(between |\childdocmain| and the above block)
%
\begin{center}
|%\ifchilddoc\||else\providecommand{\version}{draft}\||fi|
\end{center}
%
which can be uncommented to produce a draft version.
Likewise one can add a line to the very top of a child file
(above the |\childdocof{|\textit{main}|}| directive)
%
\begin{center}
|%\providecommand{\version}{final}|
\end{center}
%
which can be uncommented to produce the final version of this child document.

%%%%%%%%%%%%%%%%%%%%%%%%%%%%%%%%%%%%%%%%%%%%%%%%%%%%%%%%%%%%%%%%%%%%%%%%%%%%%%%%
\subsection{Forwarding}
\label{sec:forward}

Different versions of the main or child documents
using compilation flags as described in \secref{sec:flags}
can be (permanently) stored in different files
for convenient compilation, viewing and distribution.
To this end, the package defines a command
to pass on compilation to a different file:

%%%%%%%%%%%%%%%%%%%%%%%%%%%%%%%%%%%%%%%%
\DescribeMacro{\childdocforward}
The command |\childdocforward| redirects processing to
another source file:
%
\begin{center}
\begin{tabular}{l}
|\input{childdoc.def}|\\
|\childdocforward[|\textit{main}|]{|\textit{dest}|}|\\
\end{tabular}
\end{center}
%
The argument \textit{dest} is the destination file
(without extension).
It should be the main file or one of the child files.
Note that further \textsf{childdoc} directives
such as |\childdocof| and |\childdocforward|
in the indicated file will be processed in this form.
The optional argument \textit{main}
passes on directly to the main file \textit{main}
while pretending to compile the child \textit{dest}.
This form behaves as if \textit{dest}
issues |\childdocof{|\textit{main}|}| right away,
and no further \textsf{childdoc} directives will be processed.

%%%%%%%%%%%%%%%%%%%%%%%%%%%%%%%%%%%%%%%%
\DescribeMacro{\...prefix}
In the alternative form |\childdocforwardprefix|,
%
\begin{center}
\begin{tabular}{l}
|\input{childdoc.def}|\\
|\childdocforwardprefix[|\textit{main}|]{|\textit{prefix}|}{|\textit{dest}|}|
\end{tabular}
\end{center}
%
the destination file is determined by a pattern
depending on the current file:
To make this work, the current file must be called
`{\textit{prefix}\hspace{0.2em}\textit{suffix}}'
with \textit{prefix} matching precisely the argument.
Processing is then passed on to the file
`{\textit{dest}\hspace{0.2em}\textit{suffix}}'.
Surely, the same effect is achieved by
directly specifying the
argument `{\textit{dest}\hspace{0.2em}\textit{suffix}}'
in the first form.
However, that requires to set up a different file
for each child. With the alternative form of the command
all these files can have exactly the same content
which simplifies setting them up and maintaining them.

For example, the following file |draft.tex|
with a compilation flag |\version| as described in \secref{sec:flags}
compiles the main document as a draft:
%
\begin{center}
\begin{tabular}{l}
|\def\version{draft}|\\
|\input{childdoc.def}|\\
|\childdocforward{|\textit{main}|}|
\end{tabular}
\end{center}
%
Likewise, the following files |final|\textit{nn}|.tex|
compile the final version of the child document
|child|\textit{nn}|.tex|:
%
\begin{center}
\begin{tabular}{l}
|\def\version{final}|\\
|\input{childdoc.def}|\\
|\childdocforwardprefix{final}{child}|
\end{tabular}
\end{center}
%

Note that when several versions of a main file and/or of each child file
are to be generated, it may be convenient to set up a |Makefile| or
shell script to automatise the process.

%%%%%%%%%%%%%%%%%%%%%%%%%%%%%%%%%%%%%%%%%%%%%%%%%%%%%%%%%%%%%%%%%%%%%%%%%%%%%%%%
\subsection{Command Line Processing}
\label{sec:commandline}

The effect of redirection files can also be achieved by invoking
the \LaTeX{} compiler with a more elaborate command line.
Most conveniently this should be done as part
of a shell script or a |Makefile|.

When using \textsf{childdoc} in the main file, the following
command lines effectively perform a redirection
(note that depending on the shell being used,
backslashes may have to be doubled: `|\|' $\to$ `|\\|'):
%
\begin{center}
|... -jobname "|\textit{target}|" |\\|"|[\textit{flags}]%
|\input{childdoc.def}\childdocforward[|\textit{main}|]{|\textit{dest}|}"|
\end{center}
%
Here \textit{target} is the name of the output file,
\textit{main} is the name of the main file
and \textit{dest} is the name of the main or child file to be processed
(all filenames without extensions).
The optional argument \textit{main} can be omitted
if \textit{main} matches \textit{dest}.
Optionally, compilation \textit{flags} can be defined via |\def| commands.
This command line makes the \TeX{} engine believe
it is compiling the file \textit{target}
whose content is specified as the latter parameter.
The provided code then forwards the processing to
\textit{main} or \textit{dest} as described in \secref{sec:forward}.

%%%%%%%%%%%%%%%%%%%%%%%%%%%%%%%%%%%%%%%%%%%%%%%%%%%%%%%%%%%%%%%%%%%%%%%%%%%%%%%%
\subsection{Include by Input}
\label{sec:input}

Including child documents by |\include| has some restrictions by design.
Most notably, the content of a child document always occupies
its own set of pages; pages cannot be shared between child documents.
Usually, this behaviour makes perfect sense
because each child document contain an essential part of the document.
However, in some situations it may be desirable to compose
a document from a collection of parts
without having mandatory page breaks between then.
For this case, the package
provides a mechanism to include parts
by |\input| which can also be processed individually.
However, by construction this mechanism
requires manual handling of the content to be output.

%%%%%%%%%%%%%%%%%%%%%%%%%%%%%%%%%%%%%%%%
\DescribeMacro{\ifchilddocmanual}
The main file should be prepared as usual, see \secref{sec:include}.
However, the document body must make a distinction
between processing of an individual part and of the main document, e.g.:
%
\begin{center}
\begin{tabular}{l}
|\ifchilddocmanual|\\
|\input{\childdocname}|\\
|\||else|\\
\textit{document body with }|\input{|\textit{part}|}|\\
|\||fi|
\end{tabular}
\end{center}
%
The conditional |\ifchilddocmanual| is true whenever
a part to be included by |\input| is being compiled,
and the name of the part is stored in |\childdocname|.

%%%%%%%%%%%%%%%%%%%%%%%%%%%%%%%%%%%%%%%%
\DescribeMacro{\childdocby}
Each part to be included by |\input| should start with:
%
\begin{center}
\begin{tabular}{l}
|\input{childdoc.def}|\\
|\childdocby{|\textit{main}|}|\\
\end{tabular}
\end{center}
%
The directive |\childdocby| is similar to |\childdocof|
described in \secref{sec:include},
but the subsequent selection of content must be done manually.
To that end, both |\ifchilddoc| and |\ifchilddocmanual|
will be true upon processing of a part,
and the name of the part is stored in |\childdocname|.
Note that |\jobname| will be set to the filename of the current part
so that each part receives an individual |.aux| file
that does not interfere with the |.aux| file(s) of the main document.
This behaviour can be altered by the alternative form
|\childdocby[*]{|\textit{main}|}| (with a non-empty optional argument)
which uses the |.aux| file of the main document
by setting |\jobname| to \textit{main}.

%%%%%%%%%%%%%%%%%%%%%%%%%%%%%%%%%%%%%%%%%%%%%%%%%%%%%%%%%%%%%%%%%%%%%%%%%%%%%%%%
\subsection{Driver Development}
\label{sec:driver}

The \textsf{childdoc} mechanism can also be use for the development
of definition files such as \LaTeX{} styles or classes.
This case differs from the above setup with multiple parts
included by |\include| in that no |\includeonly| should be invoked.
This can be achieved by starting the include file
(before |\ProvidesPackage|) with:
%
\begin{center}
\begin{tabular}{l}
|\input{childdoc.def}|\\
|\childdocforward{|\textit{main}|}|\\
\end{tabular}
\end{center}
%
or alternatively with:
%
\begin{center}
\begin{tabular}{l}
|\input{childdoc.def}|\\
|\childdocby{|\textit{main}|}|\\
\end{tabular}
\end{center}
%
Both forms have slightly different effects as described above.
The main file is prepared as usual, see \secref{sec:include}.

%%%%%%%%%%%%%%%%%%%%%%%%%%%%%%%%%%%%%%%%%%%%%%%%%%%%%%%%%%%%%%%%%%%%%%%%%%%%%%%%
\subsection{Legacy Detection}
\label{sec:detection}

The directive |\childdocmain| in the main file can detect
whether the complete document or merely a child is to be compiled
even without using the directive |\childdocof|.
This method is deprecated because it is less robust
and there is no compelling reason to use it;
it is merely provided for backward compatibility
and it may be removed in future versions.

If the detection mechanism is to be used,
it is mandatory to correctly specify
the filename of the main file as the argument of |\childdocmain|:
%
\begin{center}
\begin{tabular}{l}
|\input{childdoc.def}|\\
|\childdocmain{|\textit{main}|}|\\
\end{tabular}
\end{center}
%
If |\jobname| does not match the argument \textit{main} of |\childdocmain|,
it is assumed that |\jobname| points to the child file to be compiled.
When using |\childdocmain| with the main file specified as argument,
it suffices to start a child file
with just |\input{|\textit{main}|}|
without loading of the package and using |\childdocof|.
If instead all processing is done
with the appropriate \textsf{childdoc} directives,
the argument of \textit{main} of |\childdocmain| can be empty.

An alternative version of the command line processing described
in \secref{sec:commandline} using the detection mechanism reads:
%
\begin{center}
|... -jobname "|\textit{target}|" "|[\textit{flags}]%
[|\def\jobname{|\textit{dest}|}|]|\input{|\textit{main}|}"|
\end{center}

%%%%%%%%%%%%%%%%%%%%%%%%%%%%%%%%%%%%%%%%%%%%%%%%%%%%%%%%%%%%%%%%%%%%%%%%%%%%%%%%
\subsection{Manual Code}
\label{sec:manual}

In case one cannot be certain whether the definitions file |childdoc.def|
is installed on the target \TeX{} distribution
and one prefers not to ship it,
it is conceivable to paste a few relevant commands into the sources.

To that end, drop all statements |\input{childdoc.def}|
and perform the replacements as outlined below.
Instead of |\childdocmain{|\textit{main}|}| add the following code
to the top of the main file:
%
\begin{center}
\begin{tabular}{l}
|\||ifdefined\childdocname\endinput\||fi\newif\ifchilddoc|\\
|\edef\childdocname{\scantokens\expandafter{\jobname\noexpand}}|\\
|\def\childdocmain{|\textit{main}|}\||ifx\childdocmain\childdocname\||else|\\
|\childdoctrue\includeonly{\childdocname}\let\jobname\childdocmain\||fi|\\
\end{tabular}
\end{center}
%
Instead of |\childdocof{|\textit{main}|}| just include the main file
at the top of each child file:
%
\begin{center}
|\input{|\textit{main}|}|
\end{center}
%
A simple redirection |\childdocforward{|\textit{dest}|}| is achieved by:
%
\begin{center}
|\def\jobname{|\textit{dest}|}\input{\jobname}|
\end{center}
%
The redirection with prefix
|\childdocforwardprefix[|\textit{prefix}|]{|\textit{dest}|}|
is accomplished by:
%
\begin{center}
\begin{tabular}{l}
|{\edef\jobname{\scantokens\expandafter{\jobname\noexpand}}|\\
|\def\redirectjob |\textit{prefix}|#1~~~{\gdef\jobname{|\textit{dest}|#1}}|\\
|\expandafter\redirectjob\jobname~~~}\input{\jobname}|
\end{tabular}
\end{center}

In an alternative approach,
child documents can be compiled by a specific command line
without additional code or specific definitions:
%
\begin{center}
|... -jobname "|\textit{target}|" "|[\textit{flags}]%
|\includeonly{|\textit{dest}|}\input{|\textit{main}|}"|
\end{center}
%

%%%%%%%%%%%%%%%%%%%%%%%%%%%%%%%%%%%%%%%%%%%%%%%%%%%%%%%%%%%%%%%%%%%%%%%%%%%%%%%%
%%%%%%%%%%%%%%%%%%%%%%%%%%%%%%%%%%%%%%%%%%%%%%%%%%%%%%%%%%%%%%%%%%%%%%%%%%%%%%%%
\section{Information}

%%%%%%%%%%%%%%%%%%%%%%%%%%%%%%%%%%%%%%%%%%%%%%%%%%%%%%%%%%%%%%%%%%%%%%%%%%%%%%%%
\subsection{Copyright}

Copyright \copyright{} 2017--2018 Niklas Beisert

This work may be distributed and/or modified under the
conditions of the \LaTeX{} Project Public License, either version 1.3
of this license or (at your option) any later version.
The latest version of this license is in
  \url{http://www.latex-project.org/lppl.txt}
and version 1.3 or later is part of all distributions of \LaTeX{}
version 2005/12/01 or later.

This work has the LPPL maintenance status `maintained'.

The Current Maintainer of this work is Niklas Beisert.

This work consists of the files |README.txt|, |childdoc.ins| and |childdoc.dtx|
as well as the derived files |childdoc.def|, |cdocsamp.tex|
with |cdocsch1.tex|, |cdocsch2.tex|, |cdocspt3.tex|, |cdocspt4.tex|,
|cdocsdrf.tex|, |cdocsfn1.tex|, |cdocsfn2.tex|
as well as |childdoc.pdf|.

%%%%%%%%%%%%%%%%%%%%%%%%%%%%%%%%%%%%%%%%%%%%%%%%%%%%%%%%%%%%%%%%%%%%%%%%%%%%%%%%
\subsection{Files and Installation}

The package consists of the files:
%
\begin{center}
\begin{tabular}{ll}
    |README.txt|   & readme file \\
    |childdoc.ins| & installation file \\
    |childdoc.dtx| & source file \\
    |childdoc.def| & definition file \\
    |cdocsamp.tex| & sample main file \\
    |cdocsch1.tex| & sample include file \\
    |cdocsch2.tex| & sample include file \\
    |cdocspt3.tex| & sample part file \\
    |cdocspt4.tex| & sample part file \\
    |cdocsdrf.tex| & sample redirection file \\
    |cdocsfn1.tex| & sample redirection file \\
    |cdocsfn2.tex| & sample redirection file \\
    |childdoc.pdf| & manual
\end{tabular}
\end{center}
%
The distribution consists of the files
|README.txt|, |childdoc.ins| and |childdoc.dtx|.
%
\begin{itemize}
\item
Run (pdf)\LaTeX{} on |childdoc.dtx|
to compile the manual |childdoc.pdf| (this file).
\item
Run \LaTeX{} on |childdoc.ins| to create the definitions file |childdoc.def|
and the sample |cdocsamp.tex| with include files
|cdocsch1.tex|, |cdocsch2.tex|, |cdocspt3.tex|, |cdocspt4.tex|,
|cdocsdrf.tex|, |cdocsfn1.tex|, |cdocsfn2.tex|.
Then copy the file |childdoc.def| to an appropriate directory of your \LaTeX{}
distribution, e.g.\ \textit{texmf-root}|/tex/latex/childdoc|.
\end{itemize}

%%%%%%%%%%%%%%%%%%%%%%%%%%%%%%%%%%%%%%%%%%%%%%%%%%%%%%%%%%%%%%%%%%%%%%%%%%%%%%%%
\subsection{Related CTAN Packages}

There are several other packages which offer a similar functionality:
%
\begin{itemize}
\item
The packages
\href{http://ctan.org/pkg/docmute}{\textsf{docmute}},
\href{http://ctan.org/pkg/includex}{\textsf{includex}} and
\href{http://ctan.org/pkg/standalone}{\textsf{standalone}}
provide commands to include only the document body of
a child file thus allowing both files to be compiled individually.
\item
The packages \href{http://ctan.org/pkg/subdocs}{\textsf{subdocs}}
and \href{http://ctan.org/pkg/subfiles}{\textsf{subfiles}}
provide structures in which the main and child documents can be
encapsulated and allowing them to be compiled individually.
The inclusion mechanism is different from the conventional |\include|.
\item
The package \href{http://ctan.org/pkg/combine}{\textsf{combine}}
is an elaborate solution to combine several documents into one.
\end{itemize}
%
See also the CTAN topic \href{http://ctan.org/topic/subdocs}{\textsf{subdocs}}
for further related packages.
The present package differs from the above solutions in that
a document structure constructed with the conventional |\include| mechanism
just needs two extra commands at the top of every file
such that all constituent files can be compiled individually.

%%%%%%%%%%%%%%%%%%%%%%%%%%%%%%%%%%%%%%%%%%%%%%%%%%%%%%%%%%%%%%%%%%%%%%%%%%%%%%%%
%\subsection{Feature Suggestions}
%
%The following is a list of features which may be useful for future
%versions of this package:
%%
%\begin{itemize}
%\item
%\ldots
%\end{itemize}

%%%%%%%%%%%%%%%%%%%%%%%%%%%%%%%%%%%%%%%%%%%%%%%%%%%%%%%%%%%%%%%%%%%%%%%%%%%%%%%%
\subsection{Revision History}

%%%%%%%%%%%%%%%%%%%%%%%%%%%%%%%%%%%%%%%%
\paragraph{v2.0:} 2018/12/30

\begin{itemize}
\item
immediate forward processing
\item
added |\childdocby| mechanism
\item
manual restructured
\end{itemize}

%%%%%%%%%%%%%%%%%%%%%%%%%%%%%%%%%%%%%%%%
\paragraph{v1.6:} 2018/01/17

\begin{itemize}
\item
application for development of include files
\item
corrections to manual
\end{itemize}

%%%%%%%%%%%%%%%%%%%%%%%%%%%%%%%%%%%%%%%%
\paragraph{v1.5:} 2017/05/21

\begin{itemize}
\item
more complete structuring introduced
\item
|\childdocof| introduced
\item
|\childdoc| renamed to |\childdocmain|
\item
|\childredirect| renamed to |\childdocforward| and |\childdocforwardprefix|
and functionality expanded
\end{itemize}

%%%%%%%%%%%%%%%%%%%%%%%%%%%%%%%%%%%%%%%%
\paragraph{v1.0:} 2017/04/27

\begin{itemize}
\item
manual and install package
\item
first version published on CTAN
\end{itemize}

%%%%%%%%%%%%%%%%%%%%%%%%%%%%%%%%%%%%%%%%
\paragraph{v0.6:} 2017/04/26

\begin{itemize}
\item
redirection mechanism added
\end{itemize}

%%%%%%%%%%%%%%%%%%%%%%%%%%%%%%%%%%%%%%%%
\paragraph{v0.5:} 2017/04/26

\begin{itemize}
\item
functionality in definition file
\end{itemize}


%%%%%%%%%%%%%%%%%%%%%%%%%%%%%%%%%%%%%%%%%%%%%%%%%%%%%%%%%%%%%%%%%%%%%%%%%%%%%%%%
%%%%%%%%%%%%%%%%%%%%%%%%%%%%%%%%%%%%%%%%%%%%%%%%%%%%%%%%%%%%%%%%%%%%%%%%%%%%%%%%
%%%%%%%%%%%%%%%%%%%%%%%%%%%%%%%%%%%%%%%%%%%%%%%%%%%%%%%%%%%%%%%%%%%%%%%%%%%%%%%%
\appendix

\settowidth\MacroIndent{\rmfamily\scriptsize 000\ }

 \DocInput{childdoc.dtx}

\end{document}
%</driver>
% \fi
%
% %%%%%%%%%%%%%%%%%%%%%%%%%%%%%%%%%%%%%%%%%%%%%%%%%%%%%%%%%%%%%%%%%%%%%%%%%%%%%%
% %%%%%%%%%%%%%%%%%%%%%%%%%%%%%%%%%%%%%%%%%%%%%%%%%%%%%%%%%%%%%%%%%%%%%%%%%%%%%%
% \section{Sample}
%\iffalse
%<*samplemain>
%\fi
%
% The following presents a sample document
% with two chapters, two parts, a title page,
% a compile flag as well as three forwarding files to set the flag.
% It consists of eight |.tex| files:
% \begin{center}
% \begin{tabular}{ll}
% |cdocsamp.tex|&main file\\
% |cdocsch1.tex|&include file for chapter 1\\
% |cdocsch2.tex|&include file for chapter 2\\
% |cdocspt3.tex|&include file for part 3\\
% |cdocspt4.tex|&include file for part 4\\
% |cdocsdrf.tex|&forwarding file for main file in draft mode\\
% |cdocsfi1.tex|&forwarding file for final version of chapter 1\\
% |cdocsfi2.tex|&forwarding file for final version of chapter 2\\
% \end{tabular}
% \end{center}
% Each of the eight files can be compiled directly by the \LaTeX{} compiler.
%
% %%%%%%%%%%%%%%%%%%%%%%%%%%%%%%%%%%%%%%
% \paragraph{Main File.}
%
% The main file is called |cdocsamp.tex|.
%
% Load the \textsf{childdoc} definitions and
% declare the filename for the main document:
%    \begin{macrocode}
\input{childdoc.def}
\childdocmain{}
%    \end{macrocode}

% Optional override for |\version| flag:
%    \begin{macrocode}
%%\ifchilddoc\else\providecommand{\version}{draft}\fi
%    \end{macrocode}

% Define the default values for the |\version| flag
% (|final| for the main file and |draft| for childs):
%    \begin{macrocode}
\ifchilddoc
\providecommand{\version}{draft}
\else
\providecommand{\version}{final}
\fi
%    \end{macrocode}

% Load the standard document class:
%    \begin{macrocode}
\documentclass[12pt]{article}
%    \end{macrocode}

% Start the document body:
%    \begin{macrocode}
\begin{document}
%    \end{macrocode}

% Declare a title page.
% Print title, part of document being processed and version flag:
%    \begin{macrocode}
\addtocounter{page}{-1}
\begin{center}
{\LARGE\bfseries{}childdoc example\par}
\vspace{1cm}
\ifchilddoc
\ifchilddocmanual part\else chapter\fi:
`\childdocname' of `\childdocjob'\par
\else
main document: `\childdocjob'\par
\fi
version: \version\par
\end{center}
\newpage
%    \end{macrocode}

% Manually include selected file,
% otherwise process as usual:
%    \begin{macrocode}
\ifchilddocmanual
\section*{part `\childdocname'}
\input{\childdocname}
\else
%    \end{macrocode}

% Include the two chapters:
%    \begin{macrocode}
\include{cdocsch1}
\include{cdocsch2}
%    \end{macrocode}

% Include the two parts unless only chapters should be displayed:
%    \begin{macrocode}
\ifchilddoc\else
\section{part three}
\input{cdocspt3}
\section{part four}
\input{cdocspt4}
\fi
%    \end{macrocode}

% Process as usual until here:
%    \begin{macrocode}
\fi
%    \end{macrocode}

% End of document body:
%    \begin{macrocode}
\end{document}
%    \end{macrocode}
%\iffalse
%</samplemain>
%\fi
%
% %%%%%%%%%%%%%%%%%%%%%%%%%%%%%%%%%%%%%%
% \paragraph{Chapter Include Files.}
%
% The include files are called |cdocsch1.tex| and |cdocsch2.tex|.
%
%\iffalse
%<*samplechap1|samplechap2>
%\fi

% Optional override for |\version| flag:
%    \begin{macrocode}
%%\providecommand{\version}{final}
%    \end{macrocode}

% Include the main document:
%    \begin{macrocode}
\input{childdoc.def}
\childdocof{cdocsamp}
%    \end{macrocode}

%\iffalse
%</samplechap1|samplechap2>
%\fi
%
%\iffalse
%<*samplechap1>
%\fi
% Some text for chapter 1:
%    \begin{macrocode}
\section{one}
some text in chapter one
%    \end{macrocode}

%\iffalse
%</samplechap1>
%\fi
% Some text for chapter 2:
%\iffalse
%<*samplechap2>
%\fi
%    \begin{macrocode}
\section{two}
more text in chapter two
%    \end{macrocode}

%\iffalse
%</samplechap2>
%\fi
%
% %%%%%%%%%%%%%%%%%%%%%%%%%%%%%%%%%%%%%%
% \paragraph{Part Include Files.}
%
% The include files are called |cdocspt3.tex| and |cdocspt4.tex|.
%
%\iffalse
%<*samplepart3|samplepart4>
%\fi

% Optional override for |\version| flag:
%    \begin{macrocode}
%%\providecommand{\version}{final}
%    \end{macrocode}

% Include the main document:
%    \begin{macrocode}
\input{childdoc.def}
\childdocby{cdocsamp}
%    \end{macrocode}

%\iffalse
%</samplepart3|samplepart4>
%\fi
%
%\iffalse
%<*samplepart3>
%\fi
% Some text for part 3:
%    \begin{macrocode}
some text in part three
%    \end{macrocode}

%\iffalse
%</samplepart3>
%\fi
% Some text for part 4:
%\iffalse
%<*samplepart4>
%\fi
%    \begin{macrocode}
more text in part four
%    \end{macrocode}

%\iffalse
%</samplepart4>
%\fi
%
% %%%%%%%%%%%%%%%%%%%%%%%%%%%%%%%%%%%%%%
% \paragraph{Forwarding for a Complete Draft.}
%
% The following forwarding file |cdocsdrf.tex|
% compiles the main document in draft mode:
%\iffalse
%<*sampledraft>
%\fi
%    \begin{macrocode}
\def\version{draft}
\input{childdoc.def}
\childdocforward{cdocsamp}
%    \end{macrocode}

%\iffalse
%</sampledraft>
%\fi
%
% %%%%%%%%%%%%%%%%%%%%%%%%%%%%%%%%%%%%%%
% \paragraph{Forwarding for Final Version of the Chapters.}
%
% The following forwarding files |cdocsfn1.tex| and |cdocsfn2.tex|
% (with identical content)
% compile the final versions of the child documents
% |cdocsch1.tex| and |cdocsch2.tex|, respectively:
%\iffalse
%<*samplefinal>
%\fi
%    \begin{macrocode}
\def\version{final}
\input{childdoc.def}
\childdocforwardprefix[cdocsamp]{cdocsfn}{cdocsch}
%    \end{macrocode}

%\iffalse
%</samplefinal>
%\fi
%
% %%%%%%%%%%%%%%%%%%%%%%%%%%%%%%%%%%%%%%
% \paragraph{Command Line Processing.}
%
% The following three command lines generate the output files
% |cdocscld|, |cdocscl1| and |cdocscl2|
% which should be identical to
% |cdocsdrf|, |cdocsch1| and |cdocsfn2|, respectively:
% \begin{center}
% \begin{tabular}{l}
% |latex -jobname cdocscld \|\\
% |  "\def\version{draft}\input{childdoc.def}\childdocforward{cdocsamp}"|\\
% |latex -jobname cdocscl1 \|\\
% |  "\input{childdoc.def}\childdocforward[cdocsamp]{cdocsch1}"|\\
% |latex -jobname cdocscl2 \|\\
% |  "\def\version{final}\input{childdoc.def}\childdocforward{cdocsch2}"|
% \end{tabular}
% \end{center}
% Note that the trailing backslash on each first line
% merely continues the input to the second line
% (for convenient cut ant paste).
% Furthermore, the command |latex| can be replaced by any
% of its alternative versions such as |pdflatex|.
%
% %%%%%%%%%%%%%%%%%%%%%%%%%%%%%%%%%%%%%%%%%%%%%%%%%%%%%%%%%%%%%%%%%%%%%%%%%%%%%%
% %%%%%%%%%%%%%%%%%%%%%%%%%%%%%%%%%%%%%%%%%%%%%%%%%%%%%%%%%%%%%%%%%%%%%%%%%%%%%%
% \section{Implementation}
%\iffalse
%<*package>
%\fi
%
% This section describes the definitions file |childdoc.def|.

% The definitions cannot be loaded using |\usepackage| or |\RequirePackage|
% which has a mechanism to prevent loading a style file more than once.
% When loading the definitions by means of |\input|
% multiple instances have to be prevented manually:
%\iffalse
%This code needs to be before the `\ProvidesFile' directive
%which is defined at the beginning of this file.
%Therefore it is also placed there and commented out here.
%</package>
%<*discard>
%\fi
%    \begin{macrocode}
\ifdefined\childdocmain\endinput\fi
%    \end{macrocode}
%\iffalse
%</discard>
%<*package>
%\fi
%
% \macro{\ifchilddoc}
% \macro{\ifchilddocmanual}
% The conditional |\ifchilddoc| tells whether a
% child (true) or main (false) document is being compiled.
% The conditional |\ifchilddocmanual| tells whether
% the |\includeonly| mechanism is used (false) or
% the selection of child files must be performed manually (true).
% The definitions initialise to false:
%    \begin{macrocode}
\newif\ifchilddoc
\newif\ifchilddocmanual
%    \end{macrocode}

% \macro{\childdocname}
% \macro{\childdocjob}
% The macro |\childdocname| stores the name of the main document
% to be compiled. The macro |\childdocjob| stores the name of
% the document on which the \LaTeX{} compiler was originally invoked.
% The content of |\jobname| cannot be compared
% to filenames specified in the source due to different catcodes.
% The following code rescans |\jobname|, stores the result
% in |\childdocname| and saves a copy in |\childdocjob|:
%    \begin{macrocode}
\edef\childdocname{\scantokens\expandafter{\jobname\noexpand}}
\let\childdocjob\childdocname
%    \end{macrocode}

% \macro{\childdocdisable}
% The macro |\childdocdisable| prevents the main file
% from being processed more than once.
% At this stage, the main document command |\childdocmain|
% is assumed to be called once again where it should do nothing.
% Any subsequent call to it should prevent
% a secondary processing of the main document
% It overwrites the forwarding commands
% |\childdocof| and |\childdocforward|
% with empty macros to prevent further inclusions of the main document:
%    \begin{macrocode}
\newcommand{\childdocdisable}
{
  \renewcommand{\childdocmain}[1]{\renewcommand{\childdocmain}[1]{\endinput}}
  \renewcommand{\childdocof}[1]{}
  \renewcommand{\childdocby}[2][]{}
  \renewcommand{\childdocforward}[2][]{}
  \renewcommand{\childdocdisable}{}
}
%    \end{macrocode}

% \macro{\childdocmain}
% The macro |\childdocmain| is to be called at the top of the main file
% with nothing or the main filename (without extension) as argument.
% First, it breaks loops.
% If the argument is not empty and does not match |\childdocname|
% (which is set by the first inclusion of |childdoc.def|),
% |\ifchilddoc| is set to true, |\includeonly| is applied to the child file
% and |\jobname| is set to the main file
% (for proper handling of |.aux| files):
%    \begin{macrocode}
\newcommand{\childdocmain}[1]
{
  \childdocdisable\childdocmain{}
  \if?#1?\else
    \begingroup
      \def\childdoctmp{#1}
      \ifx\childdoctmp\childdocname
        \def\childdoctmp{}
      \else
        \def\childdoctmp
        {
          \childdoctrue
          \includeonly{\childdocname}
          \def\childdocjob{#1}
          \def\jobname{#1}
        }
      \fi
      \expandafter
    \endgroup
    \childdoctmp
  \fi
}
%    \end{macrocode}

% \macro{\childdocof}
% The command |\childdocof| redirects
% compilation to the main file |#1|.
%    \begin{macrocode}
\newcommand{\childdocof}[1]
{
  \childdocdisable
  \childdoctrue
  \includeonly{\childdocname}
  \def\jobname{#1}
  \def\childdocjob{#1}
  \input{#1}
}
%    \end{macrocode}

% \macro{\childdocby}
% The command |\childdocby| ....
%    \begin{macrocode}
\newcommand{\childdocby}[2][]
{
  \childdocdisable
  \childdoctrue
  \childdocmanualtrue
  \if?#1?\else
    \def\jobname{#2}
  \fi
  \def\childdocjob{#2}
  \input{#2}
  \endinput
}
%    \end{macrocode}

% \macro{\childdocforward}
% The command |\childdocforward| redirects
% compilation to the main file or
% (if the optional argument is given) a child file.
% Parameters are set as if the main file
% or a child file starting with |\childdocof| was compiled.
% Then compilation is handed over to the main file:
%    \begin{macrocode}
\newcommand{\childdocforward}[2][]
{
  \begingroup
    \if?#1?
      \def\childdoctmp
      {
        \def\childdocname{#2}
        \def\childdocjob{#2}
        \def\jobname{#2}
        \input{#2}
        \endinput
      }
    \else
      \def\childdoctmp
      {
        \childdocdisable
        \def\childdocname{#2}
        \childdoctrue
        \includeonly{#2}
        \def\childdocjob{#1}
        \def\jobname{#1}
        \input{#1}
        \endinput
      }
    \fi
    \expandafter
  \endgroup
  \childdoctmp
}
%    \end{macrocode}

% \macro{\childdocforwardprefix}
% The command |\childdocforwardprefix| redirects
% compilation to the main or a child file by means of a pattern.
% The prefix |#1| in the current filename is replaced by |#2|
% and the suffix of the current filename is kept
% (it is assumed that the filename does not contain the substring `|~~~|'
% which is used as a delimiter).
% Compilation is handed over to the new file by |\childdocforward|:
%    \begin{macrocode}
\newcommand{\childdocforwardprefix}[3][]
{
  \begingroup
    \def\childdocextract #2##1~~~{\def\childdoctmp{\childdocforward[#1]{#3##1}}}
    \expandafter\childdocextract\childdocname~~~
    \expandafter
  \endgroup
  \childdoctmp
}
%    \end{macrocode}

% \macro{\childdoc}
% The deprecated macro |\childdoc| is a legacy version of |\childdocmain|:
%    \begin{macrocode}
\newcommand{\childdoc}{\childdocmain}
%    \end{macrocode}

% \macro{\childdocredirect}
% The deprecated macro |\childdocredirect| is a legacy version
% of |\childdocforward| and |\childdocforwardprefix|:
%    \begin{macrocode}
\newcommand{\childdocredirect}[2][]
{
  \begingroup
    \if?#1?
      \def\childdoctmp{\childdocforward{#2}}
    \else
      \def\childdoctmp{\childdocforwardprefix{#1}{#2}}
    \fi
    \expandafter
  \endgroup
  \childdoctmp
}
%    \end{macrocode}

%\iffalse
%</package>
%\fi
%
\endinput
|\\
|\childdocforward{|\textit{main}|}|\\
\end{tabular}
\end{center}
%
or alternatively with:
%
\begin{center}
\begin{tabular}{l}
|% \iffalse
%
% childdoc.dtx Copyright (C) 2017-2018 Niklas Beisert
%
% This work may be distributed and/or modified under the
% conditions of the LaTeX Project Public License, either version 1.3
% of this license or (at your option) any later version.
% The latest version of this license is in
%   http://www.latex-project.org/lppl.txt
% and version 1.3 or later is part of all distributions of LaTeX
% version 2005/12/01 or later.
%
% This work has the LPPL maintenance status `maintained'.
%
% The Current Maintainer of this work is Niklas Beisert.
%
% This work consists of the files childdoc.dtx and childdoc.ins
% and the derived files childdoc.def and cdocsamp.tex with
% cdocsch1.tex, cdocsch2.tex, cdocsdrf.tex, cdocsfn1.tex, cdocsfn2.tex.
%
%<package>\ifdefined\childdocmain\endinput\fi
%<package>\ProvidesFile{childdoc.def}[2018/12/30 v2.0 child document driver]
%<samplemain>\ProvidesFile{cdocsamp.tex}[2018/12/30 v2.0 sample for childdoc]
%<*driver>
%\ProvidesFile{childdoc.drv}[2018/12/30 v2.0 childdoc reference manual file]
\PassOptionsToClass{10pt,a4paper}{article}
\documentclass{ltxdoc}

\usepackage[margin=35mm]{geometry}
\usepackage{hyperref}
\usepackage{hyperxmp}
\usepackage[usenames]{color}

\hypersetup{colorlinks=true}
\hypersetup{pdfstartview=FitH}
\hypersetup{pdfpagemode=UseNone}
\hypersetup{pdfsource={}}
\hypersetup{pdflang={en-UK}}
\hypersetup{pdfcopyright={Copyright 2017-2018 Niklas Beisert.
  This work may be distributed and/or modified under the
  conditions of the LaTeX Project Public License, either version 1.3
  of this license or (at your option) any later version.}}
\hypersetup{pdflicenseurl={http://www.latex-project.org/lppl.txt}}
\hypersetup{pdfcontactaddress={ETH Zurich, ITP, HIT K,
  Wolfgang-Pauli-Strasse 27}}
\hypersetup{pdfcontactpostcode={8093}}
\hypersetup{pdfcontactcity={Zurich}}
\hypersetup{pdfcontactcountry={Switzerland}}
\hypersetup{pdfcontactemail={nbeisert@itp.phys.ethz.ch}}
\hypersetup{pdfcontacturl={http://people.phys.ethz.ch/\xmptilde nbeisert/}}

\newcommand{\secref}[1]{\hyperref[#1]{section \ref*{#1}}}

\parskip1ex
\parindent0pt
\let\olditemize\itemize
\def\itemize{\olditemize\parskip0pt}

\begin{document}

\title{The \textsf{childdoc} Package}
\hypersetup{pdftitle={The childdoc Package}}
\author{Niklas Beisert\\[2ex]
  Institut f\"ur Theoretische Physik\\
  Eidgen\"ossische Technische Hochschule Z\"urich\\
  Wolfgang-Pauli-Strasse 27, 8093 Z\"urich, Switzerland\\[1ex]
  \href{mailto:nbeisert@itp.phys.ethz.ch}
  {\texttt{nbeisert@itp.phys.ethz.ch}}}
\hypersetup{pdfauthor={Niklas Beisert}}
\hypersetup{pdfsubject={Manual for the LaTeX2e Package childdoc}}
\date{30 December 2018, \textsf{v2.0}}
\maketitle

\begin{abstract}\noindent
\textsf{childdoc} is a \LaTeXe{} package
that enables the direct compilation
of document sections included by |\include|
to individual files.
\end{abstract}

\begingroup
\parskip0ex
\tableofcontents
\endgroup

%%%%%%%%%%%%%%%%%%%%%%%%%%%%%%%%%%%%%%%%%%%%%%%%%%%%%%%%%%%%%%%%%%%%%%%%%%%%%%%%
%%%%%%%%%%%%%%%%%%%%%%%%%%%%%%%%%%%%%%%%%%%%%%%%%%%%%%%%%%%%%%%%%%%%%%%%%%%%%%%%
\section{Introduction}

\LaTeX{} provides a mechanism to structure a large document (such as a book)
into a main file and several child files (containing the chapters)
using the |\include| command.
This mechanism is beneficial for documents
which span hundreds of pages in order to
make the source file(s) more manageable.
Moreover, compilation can be restricted to
selected child files by means of the |\includeonly| command.
The latter feature can be used to reduce the compilation time while editing
(this was significantly more useful in the earlier days of \LaTeX{})
or to generate a smaller document which is easier to navigate.
Another application of |\includeonly| is to generate
documents consisting of selected parts of the complete document.

However, there are a few drawbacks of the plain |\include| mechanism:
\begin{itemize}
\item
The child files cannot be compiled on their own,
they can only be compiled via the main file.
A naive editing environment
(such as a text editor with an option
to have the current file processed by \LaTeX)
may require one to switch to the main file before compiling;
attempting to compile the child file produces errors.
\item
The main file must be modified (each time)
to adjust the |\includeonly| command
to the present needs. This easily leaves the main file in a messy state.
\item
The generated document will always carry the filename
of the main document. This is inconvenient if
several child files are to be compiled and
to be kept for distribution.
\end{itemize}

The present package provides a simple interface
to make child files individually compilable by \LaTeX{}.
Compiling a child file then has the same effect as compiling
the main file with an |\includeonly| command
to select the appropriate child.
Moreover the generated document will carry the name of the child
rather than the main file.
This resolves all three above issues.

This feature is meant to make the editing of books,
thesis documents and lecture notes somewhat more convenient.
However, the package can also be used efficiently for
composing a series of documents (such as exercise sheets)
which are typically distributed individually.
It then assists the author in generating the individual documents
(potentially in different versions)
as well as a document containing the collected series.
Another application is in developing style files
or other kinds of included material
where compilation of the style file could redirect
to a sample or test file.

%%%%%%%%%%%%%%%%%%%%%%%%%%%%%%%%%%%%%%%%%%%%%%%%%%%%%%%%%%%%%%%%%%%%%%%%%%%%%%%%
%%%%%%%%%%%%%%%%%%%%%%%%%%%%%%%%%%%%%%%%%%%%%%%%%%%%%%%%%%%%%%%%%%%%%%%%%%%%%%%%
\section{Usage}

First of all, the package \textsf{childdoc} is \emph{not} a standard
\LaTeXe{} |.sty| style file! Therefore it needs to be invoked in
a non-standard way.

%%%%%%%%%%%%%%%%%%%%%%%%%%%%%%%%%%%%%%%%%%%%%%%%%%%%%%%%%%%%%%%%%%%%%%%%%%%%%%%%
\subsection{Included Files}
\label{sec:include}

%%%%%%%%%%%%%%%%%%%%%%%%%%%%%%%%%%%%%%%%
\DescribeMacro{\childdocmain}
To use the package, add the commands
\begin{center}
\begin{tabular}{l}
|\input{childdoc.def}|\\
|\childdocmain{}|\\
\end{tabular}
\end{center}
at the very top of the main \LaTeX{} file,
in particular \emph{before} the |\documentclass| statement!
The argument of |\childdocmain| should be left empty
(but it must be present).

%%%%%%%%%%%%%%%%%%%%%%%%%%%%%%%%%%%%%%%%
\DescribeMacro{\childdocof}
Furthermore, add the commands
\begin{center}
\begin{tabular}{l}
|\input{childdoc.def}|\\
|\childdocof{|\textit{main}|}|\\
\end{tabular}
\end{center}
at the top of every child file \textit{child}
which is included by |\include{|\textit{child}|}|
from within the main file
(or at least for those files to be compiled individually).
The argument \textit{main} must be the filename of the main file.

There are a couple of
considerations in setting up the main and child documents:

%%%%%%%%%%%%%%%%%%%%%%%%%%%%%%%%%%%%%%%%
\paragraph{Restrictions.}

Please note the following restrictions:
\begin{itemize}
\item
|\childdocmain| must be called with one argument \textit{main}
to ensure compatibility with earlier version of the package.
It must either be empty (|\childdocmain{}|)
or precisely match the filename of the main file in which it is specified.
See \secref{sec:detection} for further information.
\item
The filename \textit{main} must be specified without the |.tex| extension.
\item
The filename \textit{main} is case sensitive
(even in case-insensitive file systems)
due to internal string comparison.
\item
The argument \textit{main} should be fully expanded, it cannot be a macro.
\item
Subdirectories and special characters should be avoided in filenames.
\item
The command |\childdocmain{|\textit{main}|}| must be followed by a whitespace.
It should not be followed immediately by another command
or by a comment mark `|%|'.
This is because the \TeX{} parser reads the token immediately following
the argument of |\childdocmain| and puts it
at the beginning of every child section;
however, a white\-space is ignored.
\end{itemize}

%%%%%%%%%%%%%%%%%%%%%%%%%%%%%%%%%%%%%%%%
\paragraph{Content of Main File.}

It is advisable to place all content in the child files included by |\include|.
Any output contained in the main file will appear in all child documents
unless suppressed manually;
it cannot be suppressed automatically by the |\includeonly| directive
and thus should normally be avoided.
A method to include some content in the main file
by means of conditional processing is described in \secref{sec:conditional}.

%%%%%%%%%%%%%%%%%%%%%%%%%%%%%%%%%%%%%%%%
\paragraph{Page Numbering.}

When only a part of the document is compiled,
the appropriate numbering of pages
(as well as other status parameters)
is determined from the |.aux| files.
The latter contain information from previous passes.
However this information needs to propagate through
all intermediate child documents.
Therefore the page numbering in child documents may well
be inconsistent until the complete document is compiled at least once.

A useful (if unconventional) way to always ensure a consistent
page numbering is to restart the numbering in each child document
and denote the pages by `\textit{child}|.|\textit{page}'
where \textit{child} represents the chapter/section number of the child file.
This can be achieved by the command
|\numberwithin{page}{|\textit{child}|}|
of the \textsf{amsmath} package
where \textit{child} can be |chapter| or |section|
depending on the chosen structuring.
Alternatively, one can modify the macro |\thepage| appropriately
and reset the counter |page| at the start of each child file.

%%%%%%%%%%%%%%%%%%%%%%%%%%%%%%%%%%%%%%%%%%%%%%%%%%%%%%%%%%%%%%%%%%%%%%%%%%%%%%%%
\subsection{Conditional Processing}
\label{sec:conditional}

The package provides a mechanism to compile different versions
of a document. To customise the versions further some conditional processing
can come in handy to distinguish which version is being compiled.
The package provides two macros to describe the compilation context:

%%%%%%%%%%%%%%%%%%%%%%%%%%%%%%%%%%%%%%%%
\DescribeMacro{\ifchilddoc}
The conditional |\ifchilddoc| distinguishes between the compilation of
child documents and the main document:
%
\begin{center}
|\ifchilddoc |\textit{child-code}| |[|\||else |\textit{main-code}]| \||fi|
\end{center}

%%%%%%%%%%%%%%%%%%%%%%%%%%%%%%%%%%%%%%%%
\DescribeMacro{\childdocname}
\DescribeMacro{\childdocjob}
The macro |\childdocname| contains the filename (without extension)
of the main or child file being processed.
Note that |\childdocjob| will always contain the name of the main file.

%%%%%%%%%%%%%%%%%%%%%%%%%%%%%%%%%%%%%%%%
\paragraph{Title Page.}

Conditional processing can be used to include a title or banner page
in the main document when proper precautions are taken.
Importantly, the code in the main file should ensure that the page counter
(as well as other status parameters which are stored in the |.aux| files)
takes the same value after the conditional processing.
Otherwise the page numbers may take divergent values
depending on which part is compiled.

For example, a title page could be declared by:
%
\begin{center}
\begin{tabular}{l}
|\ifchilddoc\||else|\\
|\addtocounter{page}{-1}|\\
\textit{code for title page}\\
|\newpage|\\
|\||fi|
\end{tabular}
\end{center}
%
A banner page for the child documents can be generated by:
%
\begin{center}
\begin{tabular}{l}
|\ifchilddoc|\\
|\addtocounter{page}{-1}|\\
\textit{code for banner page}\\
|\newpage|\\
|\||fi|
\end{tabular}
\end{center}
%
Here one could write a message such as:
\begin{center}
|This is the part \childdocname{} of \childdocjob{}.|
\end{center}

%%%%%%%%%%%%%%%%%%%%%%%%%%%%%%%%%%%%%%%%%%%%%%%%%%%%%%%%%%%%%%%%%%%%%%%%%%%%%%%%
\subsection{Flags}
\label{sec:flags}

The package makes it easy to generate different versions
of the main or child documents.
To this end compilation flags can be defined
and assigned different default values.
They will be particularly useful in conjunction
with the forwarding mechanism described in \secref{sec:forward}.

For example, it may be useful to have a flag |\version|
which can be set to |draft| or |final|.
The document source will contain some conditional code
depending on the value of |\version|.
Suppose further, the flag should default to |final| for the main file
and to |draft| for child files
which is a natural assignment for editing the document.
This is achieved by placing the following code
in the preamble of the main document
(below the |\childdocmain| directive):
%
\begin{center}
\begin{tabular}{l}
|\ifchilddoc|\\
|\providecommand{\version}{draft}|\\
|\||else|\\
|\providecommand{\version}{final}|\\
|\||fi|
\end{tabular}
\end{center}
%
The definition by |\providecommand| makes sure
that previous definitions are not overwritten.
Further statements |\providecommand{\version}{...}|
can thus be added before the above code to override it.

For the main file, one might add a line
(between |\childdocmain| and the above block)
%
\begin{center}
|%\ifchilddoc\||else\providecommand{\version}{draft}\||fi|
\end{center}
%
which can be uncommented to produce a draft version.
Likewise one can add a line to the very top of a child file
(above the |\childdocof{|\textit{main}|}| directive)
%
\begin{center}
|%\providecommand{\version}{final}|
\end{center}
%
which can be uncommented to produce the final version of this child document.

%%%%%%%%%%%%%%%%%%%%%%%%%%%%%%%%%%%%%%%%%%%%%%%%%%%%%%%%%%%%%%%%%%%%%%%%%%%%%%%%
\subsection{Forwarding}
\label{sec:forward}

Different versions of the main or child documents
using compilation flags as described in \secref{sec:flags}
can be (permanently) stored in different files
for convenient compilation, viewing and distribution.
To this end, the package defines a command
to pass on compilation to a different file:

%%%%%%%%%%%%%%%%%%%%%%%%%%%%%%%%%%%%%%%%
\DescribeMacro{\childdocforward}
The command |\childdocforward| redirects processing to
another source file:
%
\begin{center}
\begin{tabular}{l}
|\input{childdoc.def}|\\
|\childdocforward[|\textit{main}|]{|\textit{dest}|}|\\
\end{tabular}
\end{center}
%
The argument \textit{dest} is the destination file
(without extension).
It should be the main file or one of the child files.
Note that further \textsf{childdoc} directives
such as |\childdocof| and |\childdocforward|
in the indicated file will be processed in this form.
The optional argument \textit{main}
passes on directly to the main file \textit{main}
while pretending to compile the child \textit{dest}.
This form behaves as if \textit{dest}
issues |\childdocof{|\textit{main}|}| right away,
and no further \textsf{childdoc} directives will be processed.

%%%%%%%%%%%%%%%%%%%%%%%%%%%%%%%%%%%%%%%%
\DescribeMacro{\...prefix}
In the alternative form |\childdocforwardprefix|,
%
\begin{center}
\begin{tabular}{l}
|\input{childdoc.def}|\\
|\childdocforwardprefix[|\textit{main}|]{|\textit{prefix}|}{|\textit{dest}|}|
\end{tabular}
\end{center}
%
the destination file is determined by a pattern
depending on the current file:
To make this work, the current file must be called
`{\textit{prefix}\hspace{0.2em}\textit{suffix}}'
with \textit{prefix} matching precisely the argument.
Processing is then passed on to the file
`{\textit{dest}\hspace{0.2em}\textit{suffix}}'.
Surely, the same effect is achieved by
directly specifying the
argument `{\textit{dest}\hspace{0.2em}\textit{suffix}}'
in the first form.
However, that requires to set up a different file
for each child. With the alternative form of the command
all these files can have exactly the same content
which simplifies setting them up and maintaining them.

For example, the following file |draft.tex|
with a compilation flag |\version| as described in \secref{sec:flags}
compiles the main document as a draft:
%
\begin{center}
\begin{tabular}{l}
|\def\version{draft}|\\
|\input{childdoc.def}|\\
|\childdocforward{|\textit{main}|}|
\end{tabular}
\end{center}
%
Likewise, the following files |final|\textit{nn}|.tex|
compile the final version of the child document
|child|\textit{nn}|.tex|:
%
\begin{center}
\begin{tabular}{l}
|\def\version{final}|\\
|\input{childdoc.def}|\\
|\childdocforwardprefix{final}{child}|
\end{tabular}
\end{center}
%

Note that when several versions of a main file and/or of each child file
are to be generated, it may be convenient to set up a |Makefile| or
shell script to automatise the process.

%%%%%%%%%%%%%%%%%%%%%%%%%%%%%%%%%%%%%%%%%%%%%%%%%%%%%%%%%%%%%%%%%%%%%%%%%%%%%%%%
\subsection{Command Line Processing}
\label{sec:commandline}

The effect of redirection files can also be achieved by invoking
the \LaTeX{} compiler with a more elaborate command line.
Most conveniently this should be done as part
of a shell script or a |Makefile|.

When using \textsf{childdoc} in the main file, the following
command lines effectively perform a redirection
(note that depending on the shell being used,
backslashes may have to be doubled: `|\|' $\to$ `|\\|'):
%
\begin{center}
|... -jobname "|\textit{target}|" |\\|"|[\textit{flags}]%
|\input{childdoc.def}\childdocforward[|\textit{main}|]{|\textit{dest}|}"|
\end{center}
%
Here \textit{target} is the name of the output file,
\textit{main} is the name of the main file
and \textit{dest} is the name of the main or child file to be processed
(all filenames without extensions).
The optional argument \textit{main} can be omitted
if \textit{main} matches \textit{dest}.
Optionally, compilation \textit{flags} can be defined via |\def| commands.
This command line makes the \TeX{} engine believe
it is compiling the file \textit{target}
whose content is specified as the latter parameter.
The provided code then forwards the processing to
\textit{main} or \textit{dest} as described in \secref{sec:forward}.

%%%%%%%%%%%%%%%%%%%%%%%%%%%%%%%%%%%%%%%%%%%%%%%%%%%%%%%%%%%%%%%%%%%%%%%%%%%%%%%%
\subsection{Include by Input}
\label{sec:input}

Including child documents by |\include| has some restrictions by design.
Most notably, the content of a child document always occupies
its own set of pages; pages cannot be shared between child documents.
Usually, this behaviour makes perfect sense
because each child document contain an essential part of the document.
However, in some situations it may be desirable to compose
a document from a collection of parts
without having mandatory page breaks between then.
For this case, the package
provides a mechanism to include parts
by |\input| which can also be processed individually.
However, by construction this mechanism
requires manual handling of the content to be output.

%%%%%%%%%%%%%%%%%%%%%%%%%%%%%%%%%%%%%%%%
\DescribeMacro{\ifchilddocmanual}
The main file should be prepared as usual, see \secref{sec:include}.
However, the document body must make a distinction
between processing of an individual part and of the main document, e.g.:
%
\begin{center}
\begin{tabular}{l}
|\ifchilddocmanual|\\
|\input{\childdocname}|\\
|\||else|\\
\textit{document body with }|\input{|\textit{part}|}|\\
|\||fi|
\end{tabular}
\end{center}
%
The conditional |\ifchilddocmanual| is true whenever
a part to be included by |\input| is being compiled,
and the name of the part is stored in |\childdocname|.

%%%%%%%%%%%%%%%%%%%%%%%%%%%%%%%%%%%%%%%%
\DescribeMacro{\childdocby}
Each part to be included by |\input| should start with:
%
\begin{center}
\begin{tabular}{l}
|\input{childdoc.def}|\\
|\childdocby{|\textit{main}|}|\\
\end{tabular}
\end{center}
%
The directive |\childdocby| is similar to |\childdocof|
described in \secref{sec:include},
but the subsequent selection of content must be done manually.
To that end, both |\ifchilddoc| and |\ifchilddocmanual|
will be true upon processing of a part,
and the name of the part is stored in |\childdocname|.
Note that |\jobname| will be set to the filename of the current part
so that each part receives an individual |.aux| file
that does not interfere with the |.aux| file(s) of the main document.
This behaviour can be altered by the alternative form
|\childdocby[*]{|\textit{main}|}| (with a non-empty optional argument)
which uses the |.aux| file of the main document
by setting |\jobname| to \textit{main}.

%%%%%%%%%%%%%%%%%%%%%%%%%%%%%%%%%%%%%%%%%%%%%%%%%%%%%%%%%%%%%%%%%%%%%%%%%%%%%%%%
\subsection{Driver Development}
\label{sec:driver}

The \textsf{childdoc} mechanism can also be use for the development
of definition files such as \LaTeX{} styles or classes.
This case differs from the above setup with multiple parts
included by |\include| in that no |\includeonly| should be invoked.
This can be achieved by starting the include file
(before |\ProvidesPackage|) with:
%
\begin{center}
\begin{tabular}{l}
|\input{childdoc.def}|\\
|\childdocforward{|\textit{main}|}|\\
\end{tabular}
\end{center}
%
or alternatively with:
%
\begin{center}
\begin{tabular}{l}
|\input{childdoc.def}|\\
|\childdocby{|\textit{main}|}|\\
\end{tabular}
\end{center}
%
Both forms have slightly different effects as described above.
The main file is prepared as usual, see \secref{sec:include}.

%%%%%%%%%%%%%%%%%%%%%%%%%%%%%%%%%%%%%%%%%%%%%%%%%%%%%%%%%%%%%%%%%%%%%%%%%%%%%%%%
\subsection{Legacy Detection}
\label{sec:detection}

The directive |\childdocmain| in the main file can detect
whether the complete document or merely a child is to be compiled
even without using the directive |\childdocof|.
This method is deprecated because it is less robust
and there is no compelling reason to use it;
it is merely provided for backward compatibility
and it may be removed in future versions.

If the detection mechanism is to be used,
it is mandatory to correctly specify
the filename of the main file as the argument of |\childdocmain|:
%
\begin{center}
\begin{tabular}{l}
|\input{childdoc.def}|\\
|\childdocmain{|\textit{main}|}|\\
\end{tabular}
\end{center}
%
If |\jobname| does not match the argument \textit{main} of |\childdocmain|,
it is assumed that |\jobname| points to the child file to be compiled.
When using |\childdocmain| with the main file specified as argument,
it suffices to start a child file
with just |\input{|\textit{main}|}|
without loading of the package and using |\childdocof|.
If instead all processing is done
with the appropriate \textsf{childdoc} directives,
the argument of \textit{main} of |\childdocmain| can be empty.

An alternative version of the command line processing described
in \secref{sec:commandline} using the detection mechanism reads:
%
\begin{center}
|... -jobname "|\textit{target}|" "|[\textit{flags}]%
[|\def\jobname{|\textit{dest}|}|]|\input{|\textit{main}|}"|
\end{center}

%%%%%%%%%%%%%%%%%%%%%%%%%%%%%%%%%%%%%%%%%%%%%%%%%%%%%%%%%%%%%%%%%%%%%%%%%%%%%%%%
\subsection{Manual Code}
\label{sec:manual}

In case one cannot be certain whether the definitions file |childdoc.def|
is installed on the target \TeX{} distribution
and one prefers not to ship it,
it is conceivable to paste a few relevant commands into the sources.

To that end, drop all statements |\input{childdoc.def}|
and perform the replacements as outlined below.
Instead of |\childdocmain{|\textit{main}|}| add the following code
to the top of the main file:
%
\begin{center}
\begin{tabular}{l}
|\||ifdefined\childdocname\endinput\||fi\newif\ifchilddoc|\\
|\edef\childdocname{\scantokens\expandafter{\jobname\noexpand}}|\\
|\def\childdocmain{|\textit{main}|}\||ifx\childdocmain\childdocname\||else|\\
|\childdoctrue\includeonly{\childdocname}\let\jobname\childdocmain\||fi|\\
\end{tabular}
\end{center}
%
Instead of |\childdocof{|\textit{main}|}| just include the main file
at the top of each child file:
%
\begin{center}
|\input{|\textit{main}|}|
\end{center}
%
A simple redirection |\childdocforward{|\textit{dest}|}| is achieved by:
%
\begin{center}
|\def\jobname{|\textit{dest}|}\input{\jobname}|
\end{center}
%
The redirection with prefix
|\childdocforwardprefix[|\textit{prefix}|]{|\textit{dest}|}|
is accomplished by:
%
\begin{center}
\begin{tabular}{l}
|{\edef\jobname{\scantokens\expandafter{\jobname\noexpand}}|\\
|\def\redirectjob |\textit{prefix}|#1~~~{\gdef\jobname{|\textit{dest}|#1}}|\\
|\expandafter\redirectjob\jobname~~~}\input{\jobname}|
\end{tabular}
\end{center}

In an alternative approach,
child documents can be compiled by a specific command line
without additional code or specific definitions:
%
\begin{center}
|... -jobname "|\textit{target}|" "|[\textit{flags}]%
|\includeonly{|\textit{dest}|}\input{|\textit{main}|}"|
\end{center}
%

%%%%%%%%%%%%%%%%%%%%%%%%%%%%%%%%%%%%%%%%%%%%%%%%%%%%%%%%%%%%%%%%%%%%%%%%%%%%%%%%
%%%%%%%%%%%%%%%%%%%%%%%%%%%%%%%%%%%%%%%%%%%%%%%%%%%%%%%%%%%%%%%%%%%%%%%%%%%%%%%%
\section{Information}

%%%%%%%%%%%%%%%%%%%%%%%%%%%%%%%%%%%%%%%%%%%%%%%%%%%%%%%%%%%%%%%%%%%%%%%%%%%%%%%%
\subsection{Copyright}

Copyright \copyright{} 2017--2018 Niklas Beisert

This work may be distributed and/or modified under the
conditions of the \LaTeX{} Project Public License, either version 1.3
of this license or (at your option) any later version.
The latest version of this license is in
  \url{http://www.latex-project.org/lppl.txt}
and version 1.3 or later is part of all distributions of \LaTeX{}
version 2005/12/01 or later.

This work has the LPPL maintenance status `maintained'.

The Current Maintainer of this work is Niklas Beisert.

This work consists of the files |README.txt|, |childdoc.ins| and |childdoc.dtx|
as well as the derived files |childdoc.def|, |cdocsamp.tex|
with |cdocsch1.tex|, |cdocsch2.tex|, |cdocspt3.tex|, |cdocspt4.tex|,
|cdocsdrf.tex|, |cdocsfn1.tex|, |cdocsfn2.tex|
as well as |childdoc.pdf|.

%%%%%%%%%%%%%%%%%%%%%%%%%%%%%%%%%%%%%%%%%%%%%%%%%%%%%%%%%%%%%%%%%%%%%%%%%%%%%%%%
\subsection{Files and Installation}

The package consists of the files:
%
\begin{center}
\begin{tabular}{ll}
    |README.txt|   & readme file \\
    |childdoc.ins| & installation file \\
    |childdoc.dtx| & source file \\
    |childdoc.def| & definition file \\
    |cdocsamp.tex| & sample main file \\
    |cdocsch1.tex| & sample include file \\
    |cdocsch2.tex| & sample include file \\
    |cdocspt3.tex| & sample part file \\
    |cdocspt4.tex| & sample part file \\
    |cdocsdrf.tex| & sample redirection file \\
    |cdocsfn1.tex| & sample redirection file \\
    |cdocsfn2.tex| & sample redirection file \\
    |childdoc.pdf| & manual
\end{tabular}
\end{center}
%
The distribution consists of the files
|README.txt|, |childdoc.ins| and |childdoc.dtx|.
%
\begin{itemize}
\item
Run (pdf)\LaTeX{} on |childdoc.dtx|
to compile the manual |childdoc.pdf| (this file).
\item
Run \LaTeX{} on |childdoc.ins| to create the definitions file |childdoc.def|
and the sample |cdocsamp.tex| with include files
|cdocsch1.tex|, |cdocsch2.tex|, |cdocspt3.tex|, |cdocspt4.tex|,
|cdocsdrf.tex|, |cdocsfn1.tex|, |cdocsfn2.tex|.
Then copy the file |childdoc.def| to an appropriate directory of your \LaTeX{}
distribution, e.g.\ \textit{texmf-root}|/tex/latex/childdoc|.
\end{itemize}

%%%%%%%%%%%%%%%%%%%%%%%%%%%%%%%%%%%%%%%%%%%%%%%%%%%%%%%%%%%%%%%%%%%%%%%%%%%%%%%%
\subsection{Related CTAN Packages}

There are several other packages which offer a similar functionality:
%
\begin{itemize}
\item
The packages
\href{http://ctan.org/pkg/docmute}{\textsf{docmute}},
\href{http://ctan.org/pkg/includex}{\textsf{includex}} and
\href{http://ctan.org/pkg/standalone}{\textsf{standalone}}
provide commands to include only the document body of
a child file thus allowing both files to be compiled individually.
\item
The packages \href{http://ctan.org/pkg/subdocs}{\textsf{subdocs}}
and \href{http://ctan.org/pkg/subfiles}{\textsf{subfiles}}
provide structures in which the main and child documents can be
encapsulated and allowing them to be compiled individually.
The inclusion mechanism is different from the conventional |\include|.
\item
The package \href{http://ctan.org/pkg/combine}{\textsf{combine}}
is an elaborate solution to combine several documents into one.
\end{itemize}
%
See also the CTAN topic \href{http://ctan.org/topic/subdocs}{\textsf{subdocs}}
for further related packages.
The present package differs from the above solutions in that
a document structure constructed with the conventional |\include| mechanism
just needs two extra commands at the top of every file
such that all constituent files can be compiled individually.

%%%%%%%%%%%%%%%%%%%%%%%%%%%%%%%%%%%%%%%%%%%%%%%%%%%%%%%%%%%%%%%%%%%%%%%%%%%%%%%%
%\subsection{Feature Suggestions}
%
%The following is a list of features which may be useful for future
%versions of this package:
%%
%\begin{itemize}
%\item
%\ldots
%\end{itemize}

%%%%%%%%%%%%%%%%%%%%%%%%%%%%%%%%%%%%%%%%%%%%%%%%%%%%%%%%%%%%%%%%%%%%%%%%%%%%%%%%
\subsection{Revision History}

%%%%%%%%%%%%%%%%%%%%%%%%%%%%%%%%%%%%%%%%
\paragraph{v2.0:} 2018/12/30

\begin{itemize}
\item
immediate forward processing
\item
added |\childdocby| mechanism
\item
manual restructured
\end{itemize}

%%%%%%%%%%%%%%%%%%%%%%%%%%%%%%%%%%%%%%%%
\paragraph{v1.6:} 2018/01/17

\begin{itemize}
\item
application for development of include files
\item
corrections to manual
\end{itemize}

%%%%%%%%%%%%%%%%%%%%%%%%%%%%%%%%%%%%%%%%
\paragraph{v1.5:} 2017/05/21

\begin{itemize}
\item
more complete structuring introduced
\item
|\childdocof| introduced
\item
|\childdoc| renamed to |\childdocmain|
\item
|\childredirect| renamed to |\childdocforward| and |\childdocforwardprefix|
and functionality expanded
\end{itemize}

%%%%%%%%%%%%%%%%%%%%%%%%%%%%%%%%%%%%%%%%
\paragraph{v1.0:} 2017/04/27

\begin{itemize}
\item
manual and install package
\item
first version published on CTAN
\end{itemize}

%%%%%%%%%%%%%%%%%%%%%%%%%%%%%%%%%%%%%%%%
\paragraph{v0.6:} 2017/04/26

\begin{itemize}
\item
redirection mechanism added
\end{itemize}

%%%%%%%%%%%%%%%%%%%%%%%%%%%%%%%%%%%%%%%%
\paragraph{v0.5:} 2017/04/26

\begin{itemize}
\item
functionality in definition file
\end{itemize}


%%%%%%%%%%%%%%%%%%%%%%%%%%%%%%%%%%%%%%%%%%%%%%%%%%%%%%%%%%%%%%%%%%%%%%%%%%%%%%%%
%%%%%%%%%%%%%%%%%%%%%%%%%%%%%%%%%%%%%%%%%%%%%%%%%%%%%%%%%%%%%%%%%%%%%%%%%%%%%%%%
%%%%%%%%%%%%%%%%%%%%%%%%%%%%%%%%%%%%%%%%%%%%%%%%%%%%%%%%%%%%%%%%%%%%%%%%%%%%%%%%
\appendix

\settowidth\MacroIndent{\rmfamily\scriptsize 000\ }

 \DocInput{childdoc.dtx}

\end{document}
%</driver>
% \fi
%
% %%%%%%%%%%%%%%%%%%%%%%%%%%%%%%%%%%%%%%%%%%%%%%%%%%%%%%%%%%%%%%%%%%%%%%%%%%%%%%
% %%%%%%%%%%%%%%%%%%%%%%%%%%%%%%%%%%%%%%%%%%%%%%%%%%%%%%%%%%%%%%%%%%%%%%%%%%%%%%
% \section{Sample}
%\iffalse
%<*samplemain>
%\fi
%
% The following presents a sample document
% with two chapters, two parts, a title page,
% a compile flag as well as three forwarding files to set the flag.
% It consists of eight |.tex| files:
% \begin{center}
% \begin{tabular}{ll}
% |cdocsamp.tex|&main file\\
% |cdocsch1.tex|&include file for chapter 1\\
% |cdocsch2.tex|&include file for chapter 2\\
% |cdocspt3.tex|&include file for part 3\\
% |cdocspt4.tex|&include file for part 4\\
% |cdocsdrf.tex|&forwarding file for main file in draft mode\\
% |cdocsfi1.tex|&forwarding file for final version of chapter 1\\
% |cdocsfi2.tex|&forwarding file for final version of chapter 2\\
% \end{tabular}
% \end{center}
% Each of the eight files can be compiled directly by the \LaTeX{} compiler.
%
% %%%%%%%%%%%%%%%%%%%%%%%%%%%%%%%%%%%%%%
% \paragraph{Main File.}
%
% The main file is called |cdocsamp.tex|.
%
% Load the \textsf{childdoc} definitions and
% declare the filename for the main document:
%    \begin{macrocode}
\input{childdoc.def}
\childdocmain{}
%    \end{macrocode}

% Optional override for |\version| flag:
%    \begin{macrocode}
%%\ifchilddoc\else\providecommand{\version}{draft}\fi
%    \end{macrocode}

% Define the default values for the |\version| flag
% (|final| for the main file and |draft| for childs):
%    \begin{macrocode}
\ifchilddoc
\providecommand{\version}{draft}
\else
\providecommand{\version}{final}
\fi
%    \end{macrocode}

% Load the standard document class:
%    \begin{macrocode}
\documentclass[12pt]{article}
%    \end{macrocode}

% Start the document body:
%    \begin{macrocode}
\begin{document}
%    \end{macrocode}

% Declare a title page.
% Print title, part of document being processed and version flag:
%    \begin{macrocode}
\addtocounter{page}{-1}
\begin{center}
{\LARGE\bfseries{}childdoc example\par}
\vspace{1cm}
\ifchilddoc
\ifchilddocmanual part\else chapter\fi:
`\childdocname' of `\childdocjob'\par
\else
main document: `\childdocjob'\par
\fi
version: \version\par
\end{center}
\newpage
%    \end{macrocode}

% Manually include selected file,
% otherwise process as usual:
%    \begin{macrocode}
\ifchilddocmanual
\section*{part `\childdocname'}
\input{\childdocname}
\else
%    \end{macrocode}

% Include the two chapters:
%    \begin{macrocode}
\include{cdocsch1}
\include{cdocsch2}
%    \end{macrocode}

% Include the two parts unless only chapters should be displayed:
%    \begin{macrocode}
\ifchilddoc\else
\section{part three}
\input{cdocspt3}
\section{part four}
\input{cdocspt4}
\fi
%    \end{macrocode}

% Process as usual until here:
%    \begin{macrocode}
\fi
%    \end{macrocode}

% End of document body:
%    \begin{macrocode}
\end{document}
%    \end{macrocode}
%\iffalse
%</samplemain>
%\fi
%
% %%%%%%%%%%%%%%%%%%%%%%%%%%%%%%%%%%%%%%
% \paragraph{Chapter Include Files.}
%
% The include files are called |cdocsch1.tex| and |cdocsch2.tex|.
%
%\iffalse
%<*samplechap1|samplechap2>
%\fi

% Optional override for |\version| flag:
%    \begin{macrocode}
%%\providecommand{\version}{final}
%    \end{macrocode}

% Include the main document:
%    \begin{macrocode}
\input{childdoc.def}
\childdocof{cdocsamp}
%    \end{macrocode}

%\iffalse
%</samplechap1|samplechap2>
%\fi
%
%\iffalse
%<*samplechap1>
%\fi
% Some text for chapter 1:
%    \begin{macrocode}
\section{one}
some text in chapter one
%    \end{macrocode}

%\iffalse
%</samplechap1>
%\fi
% Some text for chapter 2:
%\iffalse
%<*samplechap2>
%\fi
%    \begin{macrocode}
\section{two}
more text in chapter two
%    \end{macrocode}

%\iffalse
%</samplechap2>
%\fi
%
% %%%%%%%%%%%%%%%%%%%%%%%%%%%%%%%%%%%%%%
% \paragraph{Part Include Files.}
%
% The include files are called |cdocspt3.tex| and |cdocspt4.tex|.
%
%\iffalse
%<*samplepart3|samplepart4>
%\fi

% Optional override for |\version| flag:
%    \begin{macrocode}
%%\providecommand{\version}{final}
%    \end{macrocode}

% Include the main document:
%    \begin{macrocode}
\input{childdoc.def}
\childdocby{cdocsamp}
%    \end{macrocode}

%\iffalse
%</samplepart3|samplepart4>
%\fi
%
%\iffalse
%<*samplepart3>
%\fi
% Some text for part 3:
%    \begin{macrocode}
some text in part three
%    \end{macrocode}

%\iffalse
%</samplepart3>
%\fi
% Some text for part 4:
%\iffalse
%<*samplepart4>
%\fi
%    \begin{macrocode}
more text in part four
%    \end{macrocode}

%\iffalse
%</samplepart4>
%\fi
%
% %%%%%%%%%%%%%%%%%%%%%%%%%%%%%%%%%%%%%%
% \paragraph{Forwarding for a Complete Draft.}
%
% The following forwarding file |cdocsdrf.tex|
% compiles the main document in draft mode:
%\iffalse
%<*sampledraft>
%\fi
%    \begin{macrocode}
\def\version{draft}
\input{childdoc.def}
\childdocforward{cdocsamp}
%    \end{macrocode}

%\iffalse
%</sampledraft>
%\fi
%
% %%%%%%%%%%%%%%%%%%%%%%%%%%%%%%%%%%%%%%
% \paragraph{Forwarding for Final Version of the Chapters.}
%
% The following forwarding files |cdocsfn1.tex| and |cdocsfn2.tex|
% (with identical content)
% compile the final versions of the child documents
% |cdocsch1.tex| and |cdocsch2.tex|, respectively:
%\iffalse
%<*samplefinal>
%\fi
%    \begin{macrocode}
\def\version{final}
\input{childdoc.def}
\childdocforwardprefix[cdocsamp]{cdocsfn}{cdocsch}
%    \end{macrocode}

%\iffalse
%</samplefinal>
%\fi
%
% %%%%%%%%%%%%%%%%%%%%%%%%%%%%%%%%%%%%%%
% \paragraph{Command Line Processing.}
%
% The following three command lines generate the output files
% |cdocscld|, |cdocscl1| and |cdocscl2|
% which should be identical to
% |cdocsdrf|, |cdocsch1| and |cdocsfn2|, respectively:
% \begin{center}
% \begin{tabular}{l}
% |latex -jobname cdocscld \|\\
% |  "\def\version{draft}\input{childdoc.def}\childdocforward{cdocsamp}"|\\
% |latex -jobname cdocscl1 \|\\
% |  "\input{childdoc.def}\childdocforward[cdocsamp]{cdocsch1}"|\\
% |latex -jobname cdocscl2 \|\\
% |  "\def\version{final}\input{childdoc.def}\childdocforward{cdocsch2}"|
% \end{tabular}
% \end{center}
% Note that the trailing backslash on each first line
% merely continues the input to the second line
% (for convenient cut ant paste).
% Furthermore, the command |latex| can be replaced by any
% of its alternative versions such as |pdflatex|.
%
% %%%%%%%%%%%%%%%%%%%%%%%%%%%%%%%%%%%%%%%%%%%%%%%%%%%%%%%%%%%%%%%%%%%%%%%%%%%%%%
% %%%%%%%%%%%%%%%%%%%%%%%%%%%%%%%%%%%%%%%%%%%%%%%%%%%%%%%%%%%%%%%%%%%%%%%%%%%%%%
% \section{Implementation}
%\iffalse
%<*package>
%\fi
%
% This section describes the definitions file |childdoc.def|.

% The definitions cannot be loaded using |\usepackage| or |\RequirePackage|
% which has a mechanism to prevent loading a style file more than once.
% When loading the definitions by means of |\input|
% multiple instances have to be prevented manually:
%\iffalse
%This code needs to be before the `\ProvidesFile' directive
%which is defined at the beginning of this file.
%Therefore it is also placed there and commented out here.
%</package>
%<*discard>
%\fi
%    \begin{macrocode}
\ifdefined\childdocmain\endinput\fi
%    \end{macrocode}
%\iffalse
%</discard>
%<*package>
%\fi
%
% \macro{\ifchilddoc}
% \macro{\ifchilddocmanual}
% The conditional |\ifchilddoc| tells whether a
% child (true) or main (false) document is being compiled.
% The conditional |\ifchilddocmanual| tells whether
% the |\includeonly| mechanism is used (false) or
% the selection of child files must be performed manually (true).
% The definitions initialise to false:
%    \begin{macrocode}
\newif\ifchilddoc
\newif\ifchilddocmanual
%    \end{macrocode}

% \macro{\childdocname}
% \macro{\childdocjob}
% The macro |\childdocname| stores the name of the main document
% to be compiled. The macro |\childdocjob| stores the name of
% the document on which the \LaTeX{} compiler was originally invoked.
% The content of |\jobname| cannot be compared
% to filenames specified in the source due to different catcodes.
% The following code rescans |\jobname|, stores the result
% in |\childdocname| and saves a copy in |\childdocjob|:
%    \begin{macrocode}
\edef\childdocname{\scantokens\expandafter{\jobname\noexpand}}
\let\childdocjob\childdocname
%    \end{macrocode}

% \macro{\childdocdisable}
% The macro |\childdocdisable| prevents the main file
% from being processed more than once.
% At this stage, the main document command |\childdocmain|
% is assumed to be called once again where it should do nothing.
% Any subsequent call to it should prevent
% a secondary processing of the main document
% It overwrites the forwarding commands
% |\childdocof| and |\childdocforward|
% with empty macros to prevent further inclusions of the main document:
%    \begin{macrocode}
\newcommand{\childdocdisable}
{
  \renewcommand{\childdocmain}[1]{\renewcommand{\childdocmain}[1]{\endinput}}
  \renewcommand{\childdocof}[1]{}
  \renewcommand{\childdocby}[2][]{}
  \renewcommand{\childdocforward}[2][]{}
  \renewcommand{\childdocdisable}{}
}
%    \end{macrocode}

% \macro{\childdocmain}
% The macro |\childdocmain| is to be called at the top of the main file
% with nothing or the main filename (without extension) as argument.
% First, it breaks loops.
% If the argument is not empty and does not match |\childdocname|
% (which is set by the first inclusion of |childdoc.def|),
% |\ifchilddoc| is set to true, |\includeonly| is applied to the child file
% and |\jobname| is set to the main file
% (for proper handling of |.aux| files):
%    \begin{macrocode}
\newcommand{\childdocmain}[1]
{
  \childdocdisable\childdocmain{}
  \if?#1?\else
    \begingroup
      \def\childdoctmp{#1}
      \ifx\childdoctmp\childdocname
        \def\childdoctmp{}
      \else
        \def\childdoctmp
        {
          \childdoctrue
          \includeonly{\childdocname}
          \def\childdocjob{#1}
          \def\jobname{#1}
        }
      \fi
      \expandafter
    \endgroup
    \childdoctmp
  \fi
}
%    \end{macrocode}

% \macro{\childdocof}
% The command |\childdocof| redirects
% compilation to the main file |#1|.
%    \begin{macrocode}
\newcommand{\childdocof}[1]
{
  \childdocdisable
  \childdoctrue
  \includeonly{\childdocname}
  \def\jobname{#1}
  \def\childdocjob{#1}
  \input{#1}
}
%    \end{macrocode}

% \macro{\childdocby}
% The command |\childdocby| ....
%    \begin{macrocode}
\newcommand{\childdocby}[2][]
{
  \childdocdisable
  \childdoctrue
  \childdocmanualtrue
  \if?#1?\else
    \def\jobname{#2}
  \fi
  \def\childdocjob{#2}
  \input{#2}
  \endinput
}
%    \end{macrocode}

% \macro{\childdocforward}
% The command |\childdocforward| redirects
% compilation to the main file or
% (if the optional argument is given) a child file.
% Parameters are set as if the main file
% or a child file starting with |\childdocof| was compiled.
% Then compilation is handed over to the main file:
%    \begin{macrocode}
\newcommand{\childdocforward}[2][]
{
  \begingroup
    \if?#1?
      \def\childdoctmp
      {
        \def\childdocname{#2}
        \def\childdocjob{#2}
        \def\jobname{#2}
        \input{#2}
        \endinput
      }
    \else
      \def\childdoctmp
      {
        \childdocdisable
        \def\childdocname{#2}
        \childdoctrue
        \includeonly{#2}
        \def\childdocjob{#1}
        \def\jobname{#1}
        \input{#1}
        \endinput
      }
    \fi
    \expandafter
  \endgroup
  \childdoctmp
}
%    \end{macrocode}

% \macro{\childdocforwardprefix}
% The command |\childdocforwardprefix| redirects
% compilation to the main or a child file by means of a pattern.
% The prefix |#1| in the current filename is replaced by |#2|
% and the suffix of the current filename is kept
% (it is assumed that the filename does not contain the substring `|~~~|'
% which is used as a delimiter).
% Compilation is handed over to the new file by |\childdocforward|:
%    \begin{macrocode}
\newcommand{\childdocforwardprefix}[3][]
{
  \begingroup
    \def\childdocextract #2##1~~~{\def\childdoctmp{\childdocforward[#1]{#3##1}}}
    \expandafter\childdocextract\childdocname~~~
    \expandafter
  \endgroup
  \childdoctmp
}
%    \end{macrocode}

% \macro{\childdoc}
% The deprecated macro |\childdoc| is a legacy version of |\childdocmain|:
%    \begin{macrocode}
\newcommand{\childdoc}{\childdocmain}
%    \end{macrocode}

% \macro{\childdocredirect}
% The deprecated macro |\childdocredirect| is a legacy version
% of |\childdocforward| and |\childdocforwardprefix|:
%    \begin{macrocode}
\newcommand{\childdocredirect}[2][]
{
  \begingroup
    \if?#1?
      \def\childdoctmp{\childdocforward{#2}}
    \else
      \def\childdoctmp{\childdocforwardprefix{#1}{#2}}
    \fi
    \expandafter
  \endgroup
  \childdoctmp
}
%    \end{macrocode}

%\iffalse
%</package>
%\fi
%
\endinput
|\\
|\childdocby{|\textit{main}|}|\\
\end{tabular}
\end{center}
%
Both forms have slightly different effects as described above.
The main file is prepared as usual, see \secref{sec:include}.

%%%%%%%%%%%%%%%%%%%%%%%%%%%%%%%%%%%%%%%%%%%%%%%%%%%%%%%%%%%%%%%%%%%%%%%%%%%%%%%%
\subsection{Legacy Detection}
\label{sec:detection}

The directive |\childdocmain| in the main file can detect
whether the complete document or merely a child is to be compiled
even without using the directive |\childdocof|.
This method is deprecated because it is less robust
and there is no compelling reason to use it;
it is merely provided for backward compatibility
and it may be removed in future versions.

If the detection mechanism is to be used,
it is mandatory to correctly specify
the filename of the main file as the argument of |\childdocmain|:
%
\begin{center}
\begin{tabular}{l}
|% \iffalse
%
% childdoc.dtx Copyright (C) 2017-2018 Niklas Beisert
%
% This work may be distributed and/or modified under the
% conditions of the LaTeX Project Public License, either version 1.3
% of this license or (at your option) any later version.
% The latest version of this license is in
%   http://www.latex-project.org/lppl.txt
% and version 1.3 or later is part of all distributions of LaTeX
% version 2005/12/01 or later.
%
% This work has the LPPL maintenance status `maintained'.
%
% The Current Maintainer of this work is Niklas Beisert.
%
% This work consists of the files childdoc.dtx and childdoc.ins
% and the derived files childdoc.def and cdocsamp.tex with
% cdocsch1.tex, cdocsch2.tex, cdocsdrf.tex, cdocsfn1.tex, cdocsfn2.tex.
%
%<package>\ifdefined\childdocmain\endinput\fi
%<package>\ProvidesFile{childdoc.def}[2018/12/30 v2.0 child document driver]
%<samplemain>\ProvidesFile{cdocsamp.tex}[2018/12/30 v2.0 sample for childdoc]
%<*driver>
%\ProvidesFile{childdoc.drv}[2018/12/30 v2.0 childdoc reference manual file]
\PassOptionsToClass{10pt,a4paper}{article}
\documentclass{ltxdoc}

\usepackage[margin=35mm]{geometry}
\usepackage{hyperref}
\usepackage{hyperxmp}
\usepackage[usenames]{color}

\hypersetup{colorlinks=true}
\hypersetup{pdfstartview=FitH}
\hypersetup{pdfpagemode=UseNone}
\hypersetup{pdfsource={}}
\hypersetup{pdflang={en-UK}}
\hypersetup{pdfcopyright={Copyright 2017-2018 Niklas Beisert.
  This work may be distributed and/or modified under the
  conditions of the LaTeX Project Public License, either version 1.3
  of this license or (at your option) any later version.}}
\hypersetup{pdflicenseurl={http://www.latex-project.org/lppl.txt}}
\hypersetup{pdfcontactaddress={ETH Zurich, ITP, HIT K,
  Wolfgang-Pauli-Strasse 27}}
\hypersetup{pdfcontactpostcode={8093}}
\hypersetup{pdfcontactcity={Zurich}}
\hypersetup{pdfcontactcountry={Switzerland}}
\hypersetup{pdfcontactemail={nbeisert@itp.phys.ethz.ch}}
\hypersetup{pdfcontacturl={http://people.phys.ethz.ch/\xmptilde nbeisert/}}

\newcommand{\secref}[1]{\hyperref[#1]{section \ref*{#1}}}

\parskip1ex
\parindent0pt
\let\olditemize\itemize
\def\itemize{\olditemize\parskip0pt}

\begin{document}

\title{The \textsf{childdoc} Package}
\hypersetup{pdftitle={The childdoc Package}}
\author{Niklas Beisert\\[2ex]
  Institut f\"ur Theoretische Physik\\
  Eidgen\"ossische Technische Hochschule Z\"urich\\
  Wolfgang-Pauli-Strasse 27, 8093 Z\"urich, Switzerland\\[1ex]
  \href{mailto:nbeisert@itp.phys.ethz.ch}
  {\texttt{nbeisert@itp.phys.ethz.ch}}}
\hypersetup{pdfauthor={Niklas Beisert}}
\hypersetup{pdfsubject={Manual for the LaTeX2e Package childdoc}}
\date{30 December 2018, \textsf{v2.0}}
\maketitle

\begin{abstract}\noindent
\textsf{childdoc} is a \LaTeXe{} package
that enables the direct compilation
of document sections included by |\include|
to individual files.
\end{abstract}

\begingroup
\parskip0ex
\tableofcontents
\endgroup

%%%%%%%%%%%%%%%%%%%%%%%%%%%%%%%%%%%%%%%%%%%%%%%%%%%%%%%%%%%%%%%%%%%%%%%%%%%%%%%%
%%%%%%%%%%%%%%%%%%%%%%%%%%%%%%%%%%%%%%%%%%%%%%%%%%%%%%%%%%%%%%%%%%%%%%%%%%%%%%%%
\section{Introduction}

\LaTeX{} provides a mechanism to structure a large document (such as a book)
into a main file and several child files (containing the chapters)
using the |\include| command.
This mechanism is beneficial for documents
which span hundreds of pages in order to
make the source file(s) more manageable.
Moreover, compilation can be restricted to
selected child files by means of the |\includeonly| command.
The latter feature can be used to reduce the compilation time while editing
(this was significantly more useful in the earlier days of \LaTeX{})
or to generate a smaller document which is easier to navigate.
Another application of |\includeonly| is to generate
documents consisting of selected parts of the complete document.

However, there are a few drawbacks of the plain |\include| mechanism:
\begin{itemize}
\item
The child files cannot be compiled on their own,
they can only be compiled via the main file.
A naive editing environment
(such as a text editor with an option
to have the current file processed by \LaTeX)
may require one to switch to the main file before compiling;
attempting to compile the child file produces errors.
\item
The main file must be modified (each time)
to adjust the |\includeonly| command
to the present needs. This easily leaves the main file in a messy state.
\item
The generated document will always carry the filename
of the main document. This is inconvenient if
several child files are to be compiled and
to be kept for distribution.
\end{itemize}

The present package provides a simple interface
to make child files individually compilable by \LaTeX{}.
Compiling a child file then has the same effect as compiling
the main file with an |\includeonly| command
to select the appropriate child.
Moreover the generated document will carry the name of the child
rather than the main file.
This resolves all three above issues.

This feature is meant to make the editing of books,
thesis documents and lecture notes somewhat more convenient.
However, the package can also be used efficiently for
composing a series of documents (such as exercise sheets)
which are typically distributed individually.
It then assists the author in generating the individual documents
(potentially in different versions)
as well as a document containing the collected series.
Another application is in developing style files
or other kinds of included material
where compilation of the style file could redirect
to a sample or test file.

%%%%%%%%%%%%%%%%%%%%%%%%%%%%%%%%%%%%%%%%%%%%%%%%%%%%%%%%%%%%%%%%%%%%%%%%%%%%%%%%
%%%%%%%%%%%%%%%%%%%%%%%%%%%%%%%%%%%%%%%%%%%%%%%%%%%%%%%%%%%%%%%%%%%%%%%%%%%%%%%%
\section{Usage}

First of all, the package \textsf{childdoc} is \emph{not} a standard
\LaTeXe{} |.sty| style file! Therefore it needs to be invoked in
a non-standard way.

%%%%%%%%%%%%%%%%%%%%%%%%%%%%%%%%%%%%%%%%%%%%%%%%%%%%%%%%%%%%%%%%%%%%%%%%%%%%%%%%
\subsection{Included Files}
\label{sec:include}

%%%%%%%%%%%%%%%%%%%%%%%%%%%%%%%%%%%%%%%%
\DescribeMacro{\childdocmain}
To use the package, add the commands
\begin{center}
\begin{tabular}{l}
|\input{childdoc.def}|\\
|\childdocmain{}|\\
\end{tabular}
\end{center}
at the very top of the main \LaTeX{} file,
in particular \emph{before} the |\documentclass| statement!
The argument of |\childdocmain| should be left empty
(but it must be present).

%%%%%%%%%%%%%%%%%%%%%%%%%%%%%%%%%%%%%%%%
\DescribeMacro{\childdocof}
Furthermore, add the commands
\begin{center}
\begin{tabular}{l}
|\input{childdoc.def}|\\
|\childdocof{|\textit{main}|}|\\
\end{tabular}
\end{center}
at the top of every child file \textit{child}
which is included by |\include{|\textit{child}|}|
from within the main file
(or at least for those files to be compiled individually).
The argument \textit{main} must be the filename of the main file.

There are a couple of
considerations in setting up the main and child documents:

%%%%%%%%%%%%%%%%%%%%%%%%%%%%%%%%%%%%%%%%
\paragraph{Restrictions.}

Please note the following restrictions:
\begin{itemize}
\item
|\childdocmain| must be called with one argument \textit{main}
to ensure compatibility with earlier version of the package.
It must either be empty (|\childdocmain{}|)
or precisely match the filename of the main file in which it is specified.
See \secref{sec:detection} for further information.
\item
The filename \textit{main} must be specified without the |.tex| extension.
\item
The filename \textit{main} is case sensitive
(even in case-insensitive file systems)
due to internal string comparison.
\item
The argument \textit{main} should be fully expanded, it cannot be a macro.
\item
Subdirectories and special characters should be avoided in filenames.
\item
The command |\childdocmain{|\textit{main}|}| must be followed by a whitespace.
It should not be followed immediately by another command
or by a comment mark `|%|'.
This is because the \TeX{} parser reads the token immediately following
the argument of |\childdocmain| and puts it
at the beginning of every child section;
however, a white\-space is ignored.
\end{itemize}

%%%%%%%%%%%%%%%%%%%%%%%%%%%%%%%%%%%%%%%%
\paragraph{Content of Main File.}

It is advisable to place all content in the child files included by |\include|.
Any output contained in the main file will appear in all child documents
unless suppressed manually;
it cannot be suppressed automatically by the |\includeonly| directive
and thus should normally be avoided.
A method to include some content in the main file
by means of conditional processing is described in \secref{sec:conditional}.

%%%%%%%%%%%%%%%%%%%%%%%%%%%%%%%%%%%%%%%%
\paragraph{Page Numbering.}

When only a part of the document is compiled,
the appropriate numbering of pages
(as well as other status parameters)
is determined from the |.aux| files.
The latter contain information from previous passes.
However this information needs to propagate through
all intermediate child documents.
Therefore the page numbering in child documents may well
be inconsistent until the complete document is compiled at least once.

A useful (if unconventional) way to always ensure a consistent
page numbering is to restart the numbering in each child document
and denote the pages by `\textit{child}|.|\textit{page}'
where \textit{child} represents the chapter/section number of the child file.
This can be achieved by the command
|\numberwithin{page}{|\textit{child}|}|
of the \textsf{amsmath} package
where \textit{child} can be |chapter| or |section|
depending on the chosen structuring.
Alternatively, one can modify the macro |\thepage| appropriately
and reset the counter |page| at the start of each child file.

%%%%%%%%%%%%%%%%%%%%%%%%%%%%%%%%%%%%%%%%%%%%%%%%%%%%%%%%%%%%%%%%%%%%%%%%%%%%%%%%
\subsection{Conditional Processing}
\label{sec:conditional}

The package provides a mechanism to compile different versions
of a document. To customise the versions further some conditional processing
can come in handy to distinguish which version is being compiled.
The package provides two macros to describe the compilation context:

%%%%%%%%%%%%%%%%%%%%%%%%%%%%%%%%%%%%%%%%
\DescribeMacro{\ifchilddoc}
The conditional |\ifchilddoc| distinguishes between the compilation of
child documents and the main document:
%
\begin{center}
|\ifchilddoc |\textit{child-code}| |[|\||else |\textit{main-code}]| \||fi|
\end{center}

%%%%%%%%%%%%%%%%%%%%%%%%%%%%%%%%%%%%%%%%
\DescribeMacro{\childdocname}
\DescribeMacro{\childdocjob}
The macro |\childdocname| contains the filename (without extension)
of the main or child file being processed.
Note that |\childdocjob| will always contain the name of the main file.

%%%%%%%%%%%%%%%%%%%%%%%%%%%%%%%%%%%%%%%%
\paragraph{Title Page.}

Conditional processing can be used to include a title or banner page
in the main document when proper precautions are taken.
Importantly, the code in the main file should ensure that the page counter
(as well as other status parameters which are stored in the |.aux| files)
takes the same value after the conditional processing.
Otherwise the page numbers may take divergent values
depending on which part is compiled.

For example, a title page could be declared by:
%
\begin{center}
\begin{tabular}{l}
|\ifchilddoc\||else|\\
|\addtocounter{page}{-1}|\\
\textit{code for title page}\\
|\newpage|\\
|\||fi|
\end{tabular}
\end{center}
%
A banner page for the child documents can be generated by:
%
\begin{center}
\begin{tabular}{l}
|\ifchilddoc|\\
|\addtocounter{page}{-1}|\\
\textit{code for banner page}\\
|\newpage|\\
|\||fi|
\end{tabular}
\end{center}
%
Here one could write a message such as:
\begin{center}
|This is the part \childdocname{} of \childdocjob{}.|
\end{center}

%%%%%%%%%%%%%%%%%%%%%%%%%%%%%%%%%%%%%%%%%%%%%%%%%%%%%%%%%%%%%%%%%%%%%%%%%%%%%%%%
\subsection{Flags}
\label{sec:flags}

The package makes it easy to generate different versions
of the main or child documents.
To this end compilation flags can be defined
and assigned different default values.
They will be particularly useful in conjunction
with the forwarding mechanism described in \secref{sec:forward}.

For example, it may be useful to have a flag |\version|
which can be set to |draft| or |final|.
The document source will contain some conditional code
depending on the value of |\version|.
Suppose further, the flag should default to |final| for the main file
and to |draft| for child files
which is a natural assignment for editing the document.
This is achieved by placing the following code
in the preamble of the main document
(below the |\childdocmain| directive):
%
\begin{center}
\begin{tabular}{l}
|\ifchilddoc|\\
|\providecommand{\version}{draft}|\\
|\||else|\\
|\providecommand{\version}{final}|\\
|\||fi|
\end{tabular}
\end{center}
%
The definition by |\providecommand| makes sure
that previous definitions are not overwritten.
Further statements |\providecommand{\version}{...}|
can thus be added before the above code to override it.

For the main file, one might add a line
(between |\childdocmain| and the above block)
%
\begin{center}
|%\ifchilddoc\||else\providecommand{\version}{draft}\||fi|
\end{center}
%
which can be uncommented to produce a draft version.
Likewise one can add a line to the very top of a child file
(above the |\childdocof{|\textit{main}|}| directive)
%
\begin{center}
|%\providecommand{\version}{final}|
\end{center}
%
which can be uncommented to produce the final version of this child document.

%%%%%%%%%%%%%%%%%%%%%%%%%%%%%%%%%%%%%%%%%%%%%%%%%%%%%%%%%%%%%%%%%%%%%%%%%%%%%%%%
\subsection{Forwarding}
\label{sec:forward}

Different versions of the main or child documents
using compilation flags as described in \secref{sec:flags}
can be (permanently) stored in different files
for convenient compilation, viewing and distribution.
To this end, the package defines a command
to pass on compilation to a different file:

%%%%%%%%%%%%%%%%%%%%%%%%%%%%%%%%%%%%%%%%
\DescribeMacro{\childdocforward}
The command |\childdocforward| redirects processing to
another source file:
%
\begin{center}
\begin{tabular}{l}
|\input{childdoc.def}|\\
|\childdocforward[|\textit{main}|]{|\textit{dest}|}|\\
\end{tabular}
\end{center}
%
The argument \textit{dest} is the destination file
(without extension).
It should be the main file or one of the child files.
Note that further \textsf{childdoc} directives
such as |\childdocof| and |\childdocforward|
in the indicated file will be processed in this form.
The optional argument \textit{main}
passes on directly to the main file \textit{main}
while pretending to compile the child \textit{dest}.
This form behaves as if \textit{dest}
issues |\childdocof{|\textit{main}|}| right away,
and no further \textsf{childdoc} directives will be processed.

%%%%%%%%%%%%%%%%%%%%%%%%%%%%%%%%%%%%%%%%
\DescribeMacro{\...prefix}
In the alternative form |\childdocforwardprefix|,
%
\begin{center}
\begin{tabular}{l}
|\input{childdoc.def}|\\
|\childdocforwardprefix[|\textit{main}|]{|\textit{prefix}|}{|\textit{dest}|}|
\end{tabular}
\end{center}
%
the destination file is determined by a pattern
depending on the current file:
To make this work, the current file must be called
`{\textit{prefix}\hspace{0.2em}\textit{suffix}}'
with \textit{prefix} matching precisely the argument.
Processing is then passed on to the file
`{\textit{dest}\hspace{0.2em}\textit{suffix}}'.
Surely, the same effect is achieved by
directly specifying the
argument `{\textit{dest}\hspace{0.2em}\textit{suffix}}'
in the first form.
However, that requires to set up a different file
for each child. With the alternative form of the command
all these files can have exactly the same content
which simplifies setting them up and maintaining them.

For example, the following file |draft.tex|
with a compilation flag |\version| as described in \secref{sec:flags}
compiles the main document as a draft:
%
\begin{center}
\begin{tabular}{l}
|\def\version{draft}|\\
|\input{childdoc.def}|\\
|\childdocforward{|\textit{main}|}|
\end{tabular}
\end{center}
%
Likewise, the following files |final|\textit{nn}|.tex|
compile the final version of the child document
|child|\textit{nn}|.tex|:
%
\begin{center}
\begin{tabular}{l}
|\def\version{final}|\\
|\input{childdoc.def}|\\
|\childdocforwardprefix{final}{child}|
\end{tabular}
\end{center}
%

Note that when several versions of a main file and/or of each child file
are to be generated, it may be convenient to set up a |Makefile| or
shell script to automatise the process.

%%%%%%%%%%%%%%%%%%%%%%%%%%%%%%%%%%%%%%%%%%%%%%%%%%%%%%%%%%%%%%%%%%%%%%%%%%%%%%%%
\subsection{Command Line Processing}
\label{sec:commandline}

The effect of redirection files can also be achieved by invoking
the \LaTeX{} compiler with a more elaborate command line.
Most conveniently this should be done as part
of a shell script or a |Makefile|.

When using \textsf{childdoc} in the main file, the following
command lines effectively perform a redirection
(note that depending on the shell being used,
backslashes may have to be doubled: `|\|' $\to$ `|\\|'):
%
\begin{center}
|... -jobname "|\textit{target}|" |\\|"|[\textit{flags}]%
|\input{childdoc.def}\childdocforward[|\textit{main}|]{|\textit{dest}|}"|
\end{center}
%
Here \textit{target} is the name of the output file,
\textit{main} is the name of the main file
and \textit{dest} is the name of the main or child file to be processed
(all filenames without extensions).
The optional argument \textit{main} can be omitted
if \textit{main} matches \textit{dest}.
Optionally, compilation \textit{flags} can be defined via |\def| commands.
This command line makes the \TeX{} engine believe
it is compiling the file \textit{target}
whose content is specified as the latter parameter.
The provided code then forwards the processing to
\textit{main} or \textit{dest} as described in \secref{sec:forward}.

%%%%%%%%%%%%%%%%%%%%%%%%%%%%%%%%%%%%%%%%%%%%%%%%%%%%%%%%%%%%%%%%%%%%%%%%%%%%%%%%
\subsection{Include by Input}
\label{sec:input}

Including child documents by |\include| has some restrictions by design.
Most notably, the content of a child document always occupies
its own set of pages; pages cannot be shared between child documents.
Usually, this behaviour makes perfect sense
because each child document contain an essential part of the document.
However, in some situations it may be desirable to compose
a document from a collection of parts
without having mandatory page breaks between then.
For this case, the package
provides a mechanism to include parts
by |\input| which can also be processed individually.
However, by construction this mechanism
requires manual handling of the content to be output.

%%%%%%%%%%%%%%%%%%%%%%%%%%%%%%%%%%%%%%%%
\DescribeMacro{\ifchilddocmanual}
The main file should be prepared as usual, see \secref{sec:include}.
However, the document body must make a distinction
between processing of an individual part and of the main document, e.g.:
%
\begin{center}
\begin{tabular}{l}
|\ifchilddocmanual|\\
|\input{\childdocname}|\\
|\||else|\\
\textit{document body with }|\input{|\textit{part}|}|\\
|\||fi|
\end{tabular}
\end{center}
%
The conditional |\ifchilddocmanual| is true whenever
a part to be included by |\input| is being compiled,
and the name of the part is stored in |\childdocname|.

%%%%%%%%%%%%%%%%%%%%%%%%%%%%%%%%%%%%%%%%
\DescribeMacro{\childdocby}
Each part to be included by |\input| should start with:
%
\begin{center}
\begin{tabular}{l}
|\input{childdoc.def}|\\
|\childdocby{|\textit{main}|}|\\
\end{tabular}
\end{center}
%
The directive |\childdocby| is similar to |\childdocof|
described in \secref{sec:include},
but the subsequent selection of content must be done manually.
To that end, both |\ifchilddoc| and |\ifchilddocmanual|
will be true upon processing of a part,
and the name of the part is stored in |\childdocname|.
Note that |\jobname| will be set to the filename of the current part
so that each part receives an individual |.aux| file
that does not interfere with the |.aux| file(s) of the main document.
This behaviour can be altered by the alternative form
|\childdocby[*]{|\textit{main}|}| (with a non-empty optional argument)
which uses the |.aux| file of the main document
by setting |\jobname| to \textit{main}.

%%%%%%%%%%%%%%%%%%%%%%%%%%%%%%%%%%%%%%%%%%%%%%%%%%%%%%%%%%%%%%%%%%%%%%%%%%%%%%%%
\subsection{Driver Development}
\label{sec:driver}

The \textsf{childdoc} mechanism can also be use for the development
of definition files such as \LaTeX{} styles or classes.
This case differs from the above setup with multiple parts
included by |\include| in that no |\includeonly| should be invoked.
This can be achieved by starting the include file
(before |\ProvidesPackage|) with:
%
\begin{center}
\begin{tabular}{l}
|\input{childdoc.def}|\\
|\childdocforward{|\textit{main}|}|\\
\end{tabular}
\end{center}
%
or alternatively with:
%
\begin{center}
\begin{tabular}{l}
|\input{childdoc.def}|\\
|\childdocby{|\textit{main}|}|\\
\end{tabular}
\end{center}
%
Both forms have slightly different effects as described above.
The main file is prepared as usual, see \secref{sec:include}.

%%%%%%%%%%%%%%%%%%%%%%%%%%%%%%%%%%%%%%%%%%%%%%%%%%%%%%%%%%%%%%%%%%%%%%%%%%%%%%%%
\subsection{Legacy Detection}
\label{sec:detection}

The directive |\childdocmain| in the main file can detect
whether the complete document or merely a child is to be compiled
even without using the directive |\childdocof|.
This method is deprecated because it is less robust
and there is no compelling reason to use it;
it is merely provided for backward compatibility
and it may be removed in future versions.

If the detection mechanism is to be used,
it is mandatory to correctly specify
the filename of the main file as the argument of |\childdocmain|:
%
\begin{center}
\begin{tabular}{l}
|\input{childdoc.def}|\\
|\childdocmain{|\textit{main}|}|\\
\end{tabular}
\end{center}
%
If |\jobname| does not match the argument \textit{main} of |\childdocmain|,
it is assumed that |\jobname| points to the child file to be compiled.
When using |\childdocmain| with the main file specified as argument,
it suffices to start a child file
with just |\input{|\textit{main}|}|
without loading of the package and using |\childdocof|.
If instead all processing is done
with the appropriate \textsf{childdoc} directives,
the argument of \textit{main} of |\childdocmain| can be empty.

An alternative version of the command line processing described
in \secref{sec:commandline} using the detection mechanism reads:
%
\begin{center}
|... -jobname "|\textit{target}|" "|[\textit{flags}]%
[|\def\jobname{|\textit{dest}|}|]|\input{|\textit{main}|}"|
\end{center}

%%%%%%%%%%%%%%%%%%%%%%%%%%%%%%%%%%%%%%%%%%%%%%%%%%%%%%%%%%%%%%%%%%%%%%%%%%%%%%%%
\subsection{Manual Code}
\label{sec:manual}

In case one cannot be certain whether the definitions file |childdoc.def|
is installed on the target \TeX{} distribution
and one prefers not to ship it,
it is conceivable to paste a few relevant commands into the sources.

To that end, drop all statements |\input{childdoc.def}|
and perform the replacements as outlined below.
Instead of |\childdocmain{|\textit{main}|}| add the following code
to the top of the main file:
%
\begin{center}
\begin{tabular}{l}
|\||ifdefined\childdocname\endinput\||fi\newif\ifchilddoc|\\
|\edef\childdocname{\scantokens\expandafter{\jobname\noexpand}}|\\
|\def\childdocmain{|\textit{main}|}\||ifx\childdocmain\childdocname\||else|\\
|\childdoctrue\includeonly{\childdocname}\let\jobname\childdocmain\||fi|\\
\end{tabular}
\end{center}
%
Instead of |\childdocof{|\textit{main}|}| just include the main file
at the top of each child file:
%
\begin{center}
|\input{|\textit{main}|}|
\end{center}
%
A simple redirection |\childdocforward{|\textit{dest}|}| is achieved by:
%
\begin{center}
|\def\jobname{|\textit{dest}|}\input{\jobname}|
\end{center}
%
The redirection with prefix
|\childdocforwardprefix[|\textit{prefix}|]{|\textit{dest}|}|
is accomplished by:
%
\begin{center}
\begin{tabular}{l}
|{\edef\jobname{\scantokens\expandafter{\jobname\noexpand}}|\\
|\def\redirectjob |\textit{prefix}|#1~~~{\gdef\jobname{|\textit{dest}|#1}}|\\
|\expandafter\redirectjob\jobname~~~}\input{\jobname}|
\end{tabular}
\end{center}

In an alternative approach,
child documents can be compiled by a specific command line
without additional code or specific definitions:
%
\begin{center}
|... -jobname "|\textit{target}|" "|[\textit{flags}]%
|\includeonly{|\textit{dest}|}\input{|\textit{main}|}"|
\end{center}
%

%%%%%%%%%%%%%%%%%%%%%%%%%%%%%%%%%%%%%%%%%%%%%%%%%%%%%%%%%%%%%%%%%%%%%%%%%%%%%%%%
%%%%%%%%%%%%%%%%%%%%%%%%%%%%%%%%%%%%%%%%%%%%%%%%%%%%%%%%%%%%%%%%%%%%%%%%%%%%%%%%
\section{Information}

%%%%%%%%%%%%%%%%%%%%%%%%%%%%%%%%%%%%%%%%%%%%%%%%%%%%%%%%%%%%%%%%%%%%%%%%%%%%%%%%
\subsection{Copyright}

Copyright \copyright{} 2017--2018 Niklas Beisert

This work may be distributed and/or modified under the
conditions of the \LaTeX{} Project Public License, either version 1.3
of this license or (at your option) any later version.
The latest version of this license is in
  \url{http://www.latex-project.org/lppl.txt}
and version 1.3 or later is part of all distributions of \LaTeX{}
version 2005/12/01 or later.

This work has the LPPL maintenance status `maintained'.

The Current Maintainer of this work is Niklas Beisert.

This work consists of the files |README.txt|, |childdoc.ins| and |childdoc.dtx|
as well as the derived files |childdoc.def|, |cdocsamp.tex|
with |cdocsch1.tex|, |cdocsch2.tex|, |cdocspt3.tex|, |cdocspt4.tex|,
|cdocsdrf.tex|, |cdocsfn1.tex|, |cdocsfn2.tex|
as well as |childdoc.pdf|.

%%%%%%%%%%%%%%%%%%%%%%%%%%%%%%%%%%%%%%%%%%%%%%%%%%%%%%%%%%%%%%%%%%%%%%%%%%%%%%%%
\subsection{Files and Installation}

The package consists of the files:
%
\begin{center}
\begin{tabular}{ll}
    |README.txt|   & readme file \\
    |childdoc.ins| & installation file \\
    |childdoc.dtx| & source file \\
    |childdoc.def| & definition file \\
    |cdocsamp.tex| & sample main file \\
    |cdocsch1.tex| & sample include file \\
    |cdocsch2.tex| & sample include file \\
    |cdocspt3.tex| & sample part file \\
    |cdocspt4.tex| & sample part file \\
    |cdocsdrf.tex| & sample redirection file \\
    |cdocsfn1.tex| & sample redirection file \\
    |cdocsfn2.tex| & sample redirection file \\
    |childdoc.pdf| & manual
\end{tabular}
\end{center}
%
The distribution consists of the files
|README.txt|, |childdoc.ins| and |childdoc.dtx|.
%
\begin{itemize}
\item
Run (pdf)\LaTeX{} on |childdoc.dtx|
to compile the manual |childdoc.pdf| (this file).
\item
Run \LaTeX{} on |childdoc.ins| to create the definitions file |childdoc.def|
and the sample |cdocsamp.tex| with include files
|cdocsch1.tex|, |cdocsch2.tex|, |cdocspt3.tex|, |cdocspt4.tex|,
|cdocsdrf.tex|, |cdocsfn1.tex|, |cdocsfn2.tex|.
Then copy the file |childdoc.def| to an appropriate directory of your \LaTeX{}
distribution, e.g.\ \textit{texmf-root}|/tex/latex/childdoc|.
\end{itemize}

%%%%%%%%%%%%%%%%%%%%%%%%%%%%%%%%%%%%%%%%%%%%%%%%%%%%%%%%%%%%%%%%%%%%%%%%%%%%%%%%
\subsection{Related CTAN Packages}

There are several other packages which offer a similar functionality:
%
\begin{itemize}
\item
The packages
\href{http://ctan.org/pkg/docmute}{\textsf{docmute}},
\href{http://ctan.org/pkg/includex}{\textsf{includex}} and
\href{http://ctan.org/pkg/standalone}{\textsf{standalone}}
provide commands to include only the document body of
a child file thus allowing both files to be compiled individually.
\item
The packages \href{http://ctan.org/pkg/subdocs}{\textsf{subdocs}}
and \href{http://ctan.org/pkg/subfiles}{\textsf{subfiles}}
provide structures in which the main and child documents can be
encapsulated and allowing them to be compiled individually.
The inclusion mechanism is different from the conventional |\include|.
\item
The package \href{http://ctan.org/pkg/combine}{\textsf{combine}}
is an elaborate solution to combine several documents into one.
\end{itemize}
%
See also the CTAN topic \href{http://ctan.org/topic/subdocs}{\textsf{subdocs}}
for further related packages.
The present package differs from the above solutions in that
a document structure constructed with the conventional |\include| mechanism
just needs two extra commands at the top of every file
such that all constituent files can be compiled individually.

%%%%%%%%%%%%%%%%%%%%%%%%%%%%%%%%%%%%%%%%%%%%%%%%%%%%%%%%%%%%%%%%%%%%%%%%%%%%%%%%
%\subsection{Feature Suggestions}
%
%The following is a list of features which may be useful for future
%versions of this package:
%%
%\begin{itemize}
%\item
%\ldots
%\end{itemize}

%%%%%%%%%%%%%%%%%%%%%%%%%%%%%%%%%%%%%%%%%%%%%%%%%%%%%%%%%%%%%%%%%%%%%%%%%%%%%%%%
\subsection{Revision History}

%%%%%%%%%%%%%%%%%%%%%%%%%%%%%%%%%%%%%%%%
\paragraph{v2.0:} 2018/12/30

\begin{itemize}
\item
immediate forward processing
\item
added |\childdocby| mechanism
\item
manual restructured
\end{itemize}

%%%%%%%%%%%%%%%%%%%%%%%%%%%%%%%%%%%%%%%%
\paragraph{v1.6:} 2018/01/17

\begin{itemize}
\item
application for development of include files
\item
corrections to manual
\end{itemize}

%%%%%%%%%%%%%%%%%%%%%%%%%%%%%%%%%%%%%%%%
\paragraph{v1.5:} 2017/05/21

\begin{itemize}
\item
more complete structuring introduced
\item
|\childdocof| introduced
\item
|\childdoc| renamed to |\childdocmain|
\item
|\childredirect| renamed to |\childdocforward| and |\childdocforwardprefix|
and functionality expanded
\end{itemize}

%%%%%%%%%%%%%%%%%%%%%%%%%%%%%%%%%%%%%%%%
\paragraph{v1.0:} 2017/04/27

\begin{itemize}
\item
manual and install package
\item
first version published on CTAN
\end{itemize}

%%%%%%%%%%%%%%%%%%%%%%%%%%%%%%%%%%%%%%%%
\paragraph{v0.6:} 2017/04/26

\begin{itemize}
\item
redirection mechanism added
\end{itemize}

%%%%%%%%%%%%%%%%%%%%%%%%%%%%%%%%%%%%%%%%
\paragraph{v0.5:} 2017/04/26

\begin{itemize}
\item
functionality in definition file
\end{itemize}


%%%%%%%%%%%%%%%%%%%%%%%%%%%%%%%%%%%%%%%%%%%%%%%%%%%%%%%%%%%%%%%%%%%%%%%%%%%%%%%%
%%%%%%%%%%%%%%%%%%%%%%%%%%%%%%%%%%%%%%%%%%%%%%%%%%%%%%%%%%%%%%%%%%%%%%%%%%%%%%%%
%%%%%%%%%%%%%%%%%%%%%%%%%%%%%%%%%%%%%%%%%%%%%%%%%%%%%%%%%%%%%%%%%%%%%%%%%%%%%%%%
\appendix

\settowidth\MacroIndent{\rmfamily\scriptsize 000\ }

 \DocInput{childdoc.dtx}

\end{document}
%</driver>
% \fi
%
% %%%%%%%%%%%%%%%%%%%%%%%%%%%%%%%%%%%%%%%%%%%%%%%%%%%%%%%%%%%%%%%%%%%%%%%%%%%%%%
% %%%%%%%%%%%%%%%%%%%%%%%%%%%%%%%%%%%%%%%%%%%%%%%%%%%%%%%%%%%%%%%%%%%%%%%%%%%%%%
% \section{Sample}
%\iffalse
%<*samplemain>
%\fi
%
% The following presents a sample document
% with two chapters, two parts, a title page,
% a compile flag as well as three forwarding files to set the flag.
% It consists of eight |.tex| files:
% \begin{center}
% \begin{tabular}{ll}
% |cdocsamp.tex|&main file\\
% |cdocsch1.tex|&include file for chapter 1\\
% |cdocsch2.tex|&include file for chapter 2\\
% |cdocspt3.tex|&include file for part 3\\
% |cdocspt4.tex|&include file for part 4\\
% |cdocsdrf.tex|&forwarding file for main file in draft mode\\
% |cdocsfi1.tex|&forwarding file for final version of chapter 1\\
% |cdocsfi2.tex|&forwarding file for final version of chapter 2\\
% \end{tabular}
% \end{center}
% Each of the eight files can be compiled directly by the \LaTeX{} compiler.
%
% %%%%%%%%%%%%%%%%%%%%%%%%%%%%%%%%%%%%%%
% \paragraph{Main File.}
%
% The main file is called |cdocsamp.tex|.
%
% Load the \textsf{childdoc} definitions and
% declare the filename for the main document:
%    \begin{macrocode}
\input{childdoc.def}
\childdocmain{}
%    \end{macrocode}

% Optional override for |\version| flag:
%    \begin{macrocode}
%%\ifchilddoc\else\providecommand{\version}{draft}\fi
%    \end{macrocode}

% Define the default values for the |\version| flag
% (|final| for the main file and |draft| for childs):
%    \begin{macrocode}
\ifchilddoc
\providecommand{\version}{draft}
\else
\providecommand{\version}{final}
\fi
%    \end{macrocode}

% Load the standard document class:
%    \begin{macrocode}
\documentclass[12pt]{article}
%    \end{macrocode}

% Start the document body:
%    \begin{macrocode}
\begin{document}
%    \end{macrocode}

% Declare a title page.
% Print title, part of document being processed and version flag:
%    \begin{macrocode}
\addtocounter{page}{-1}
\begin{center}
{\LARGE\bfseries{}childdoc example\par}
\vspace{1cm}
\ifchilddoc
\ifchilddocmanual part\else chapter\fi:
`\childdocname' of `\childdocjob'\par
\else
main document: `\childdocjob'\par
\fi
version: \version\par
\end{center}
\newpage
%    \end{macrocode}

% Manually include selected file,
% otherwise process as usual:
%    \begin{macrocode}
\ifchilddocmanual
\section*{part `\childdocname'}
\input{\childdocname}
\else
%    \end{macrocode}

% Include the two chapters:
%    \begin{macrocode}
\include{cdocsch1}
\include{cdocsch2}
%    \end{macrocode}

% Include the two parts unless only chapters should be displayed:
%    \begin{macrocode}
\ifchilddoc\else
\section{part three}
\input{cdocspt3}
\section{part four}
\input{cdocspt4}
\fi
%    \end{macrocode}

% Process as usual until here:
%    \begin{macrocode}
\fi
%    \end{macrocode}

% End of document body:
%    \begin{macrocode}
\end{document}
%    \end{macrocode}
%\iffalse
%</samplemain>
%\fi
%
% %%%%%%%%%%%%%%%%%%%%%%%%%%%%%%%%%%%%%%
% \paragraph{Chapter Include Files.}
%
% The include files are called |cdocsch1.tex| and |cdocsch2.tex|.
%
%\iffalse
%<*samplechap1|samplechap2>
%\fi

% Optional override for |\version| flag:
%    \begin{macrocode}
%%\providecommand{\version}{final}
%    \end{macrocode}

% Include the main document:
%    \begin{macrocode}
\input{childdoc.def}
\childdocof{cdocsamp}
%    \end{macrocode}

%\iffalse
%</samplechap1|samplechap2>
%\fi
%
%\iffalse
%<*samplechap1>
%\fi
% Some text for chapter 1:
%    \begin{macrocode}
\section{one}
some text in chapter one
%    \end{macrocode}

%\iffalse
%</samplechap1>
%\fi
% Some text for chapter 2:
%\iffalse
%<*samplechap2>
%\fi
%    \begin{macrocode}
\section{two}
more text in chapter two
%    \end{macrocode}

%\iffalse
%</samplechap2>
%\fi
%
% %%%%%%%%%%%%%%%%%%%%%%%%%%%%%%%%%%%%%%
% \paragraph{Part Include Files.}
%
% The include files are called |cdocspt3.tex| and |cdocspt4.tex|.
%
%\iffalse
%<*samplepart3|samplepart4>
%\fi

% Optional override for |\version| flag:
%    \begin{macrocode}
%%\providecommand{\version}{final}
%    \end{macrocode}

% Include the main document:
%    \begin{macrocode}
\input{childdoc.def}
\childdocby{cdocsamp}
%    \end{macrocode}

%\iffalse
%</samplepart3|samplepart4>
%\fi
%
%\iffalse
%<*samplepart3>
%\fi
% Some text for part 3:
%    \begin{macrocode}
some text in part three
%    \end{macrocode}

%\iffalse
%</samplepart3>
%\fi
% Some text for part 4:
%\iffalse
%<*samplepart4>
%\fi
%    \begin{macrocode}
more text in part four
%    \end{macrocode}

%\iffalse
%</samplepart4>
%\fi
%
% %%%%%%%%%%%%%%%%%%%%%%%%%%%%%%%%%%%%%%
% \paragraph{Forwarding for a Complete Draft.}
%
% The following forwarding file |cdocsdrf.tex|
% compiles the main document in draft mode:
%\iffalse
%<*sampledraft>
%\fi
%    \begin{macrocode}
\def\version{draft}
\input{childdoc.def}
\childdocforward{cdocsamp}
%    \end{macrocode}

%\iffalse
%</sampledraft>
%\fi
%
% %%%%%%%%%%%%%%%%%%%%%%%%%%%%%%%%%%%%%%
% \paragraph{Forwarding for Final Version of the Chapters.}
%
% The following forwarding files |cdocsfn1.tex| and |cdocsfn2.tex|
% (with identical content)
% compile the final versions of the child documents
% |cdocsch1.tex| and |cdocsch2.tex|, respectively:
%\iffalse
%<*samplefinal>
%\fi
%    \begin{macrocode}
\def\version{final}
\input{childdoc.def}
\childdocforwardprefix[cdocsamp]{cdocsfn}{cdocsch}
%    \end{macrocode}

%\iffalse
%</samplefinal>
%\fi
%
% %%%%%%%%%%%%%%%%%%%%%%%%%%%%%%%%%%%%%%
% \paragraph{Command Line Processing.}
%
% The following three command lines generate the output files
% |cdocscld|, |cdocscl1| and |cdocscl2|
% which should be identical to
% |cdocsdrf|, |cdocsch1| and |cdocsfn2|, respectively:
% \begin{center}
% \begin{tabular}{l}
% |latex -jobname cdocscld \|\\
% |  "\def\version{draft}\input{childdoc.def}\childdocforward{cdocsamp}"|\\
% |latex -jobname cdocscl1 \|\\
% |  "\input{childdoc.def}\childdocforward[cdocsamp]{cdocsch1}"|\\
% |latex -jobname cdocscl2 \|\\
% |  "\def\version{final}\input{childdoc.def}\childdocforward{cdocsch2}"|
% \end{tabular}
% \end{center}
% Note that the trailing backslash on each first line
% merely continues the input to the second line
% (for convenient cut ant paste).
% Furthermore, the command |latex| can be replaced by any
% of its alternative versions such as |pdflatex|.
%
% %%%%%%%%%%%%%%%%%%%%%%%%%%%%%%%%%%%%%%%%%%%%%%%%%%%%%%%%%%%%%%%%%%%%%%%%%%%%%%
% %%%%%%%%%%%%%%%%%%%%%%%%%%%%%%%%%%%%%%%%%%%%%%%%%%%%%%%%%%%%%%%%%%%%%%%%%%%%%%
% \section{Implementation}
%\iffalse
%<*package>
%\fi
%
% This section describes the definitions file |childdoc.def|.

% The definitions cannot be loaded using |\usepackage| or |\RequirePackage|
% which has a mechanism to prevent loading a style file more than once.
% When loading the definitions by means of |\input|
% multiple instances have to be prevented manually:
%\iffalse
%This code needs to be before the `\ProvidesFile' directive
%which is defined at the beginning of this file.
%Therefore it is also placed there and commented out here.
%</package>
%<*discard>
%\fi
%    \begin{macrocode}
\ifdefined\childdocmain\endinput\fi
%    \end{macrocode}
%\iffalse
%</discard>
%<*package>
%\fi
%
% \macro{\ifchilddoc}
% \macro{\ifchilddocmanual}
% The conditional |\ifchilddoc| tells whether a
% child (true) or main (false) document is being compiled.
% The conditional |\ifchilddocmanual| tells whether
% the |\includeonly| mechanism is used (false) or
% the selection of child files must be performed manually (true).
% The definitions initialise to false:
%    \begin{macrocode}
\newif\ifchilddoc
\newif\ifchilddocmanual
%    \end{macrocode}

% \macro{\childdocname}
% \macro{\childdocjob}
% The macro |\childdocname| stores the name of the main document
% to be compiled. The macro |\childdocjob| stores the name of
% the document on which the \LaTeX{} compiler was originally invoked.
% The content of |\jobname| cannot be compared
% to filenames specified in the source due to different catcodes.
% The following code rescans |\jobname|, stores the result
% in |\childdocname| and saves a copy in |\childdocjob|:
%    \begin{macrocode}
\edef\childdocname{\scantokens\expandafter{\jobname\noexpand}}
\let\childdocjob\childdocname
%    \end{macrocode}

% \macro{\childdocdisable}
% The macro |\childdocdisable| prevents the main file
% from being processed more than once.
% At this stage, the main document command |\childdocmain|
% is assumed to be called once again where it should do nothing.
% Any subsequent call to it should prevent
% a secondary processing of the main document
% It overwrites the forwarding commands
% |\childdocof| and |\childdocforward|
% with empty macros to prevent further inclusions of the main document:
%    \begin{macrocode}
\newcommand{\childdocdisable}
{
  \renewcommand{\childdocmain}[1]{\renewcommand{\childdocmain}[1]{\endinput}}
  \renewcommand{\childdocof}[1]{}
  \renewcommand{\childdocby}[2][]{}
  \renewcommand{\childdocforward}[2][]{}
  \renewcommand{\childdocdisable}{}
}
%    \end{macrocode}

% \macro{\childdocmain}
% The macro |\childdocmain| is to be called at the top of the main file
% with nothing or the main filename (without extension) as argument.
% First, it breaks loops.
% If the argument is not empty and does not match |\childdocname|
% (which is set by the first inclusion of |childdoc.def|),
% |\ifchilddoc| is set to true, |\includeonly| is applied to the child file
% and |\jobname| is set to the main file
% (for proper handling of |.aux| files):
%    \begin{macrocode}
\newcommand{\childdocmain}[1]
{
  \childdocdisable\childdocmain{}
  \if?#1?\else
    \begingroup
      \def\childdoctmp{#1}
      \ifx\childdoctmp\childdocname
        \def\childdoctmp{}
      \else
        \def\childdoctmp
        {
          \childdoctrue
          \includeonly{\childdocname}
          \def\childdocjob{#1}
          \def\jobname{#1}
        }
      \fi
      \expandafter
    \endgroup
    \childdoctmp
  \fi
}
%    \end{macrocode}

% \macro{\childdocof}
% The command |\childdocof| redirects
% compilation to the main file |#1|.
%    \begin{macrocode}
\newcommand{\childdocof}[1]
{
  \childdocdisable
  \childdoctrue
  \includeonly{\childdocname}
  \def\jobname{#1}
  \def\childdocjob{#1}
  \input{#1}
}
%    \end{macrocode}

% \macro{\childdocby}
% The command |\childdocby| ....
%    \begin{macrocode}
\newcommand{\childdocby}[2][]
{
  \childdocdisable
  \childdoctrue
  \childdocmanualtrue
  \if?#1?\else
    \def\jobname{#2}
  \fi
  \def\childdocjob{#2}
  \input{#2}
  \endinput
}
%    \end{macrocode}

% \macro{\childdocforward}
% The command |\childdocforward| redirects
% compilation to the main file or
% (if the optional argument is given) a child file.
% Parameters are set as if the main file
% or a child file starting with |\childdocof| was compiled.
% Then compilation is handed over to the main file:
%    \begin{macrocode}
\newcommand{\childdocforward}[2][]
{
  \begingroup
    \if?#1?
      \def\childdoctmp
      {
        \def\childdocname{#2}
        \def\childdocjob{#2}
        \def\jobname{#2}
        \input{#2}
        \endinput
      }
    \else
      \def\childdoctmp
      {
        \childdocdisable
        \def\childdocname{#2}
        \childdoctrue
        \includeonly{#2}
        \def\childdocjob{#1}
        \def\jobname{#1}
        \input{#1}
        \endinput
      }
    \fi
    \expandafter
  \endgroup
  \childdoctmp
}
%    \end{macrocode}

% \macro{\childdocforwardprefix}
% The command |\childdocforwardprefix| redirects
% compilation to the main or a child file by means of a pattern.
% The prefix |#1| in the current filename is replaced by |#2|
% and the suffix of the current filename is kept
% (it is assumed that the filename does not contain the substring `|~~~|'
% which is used as a delimiter).
% Compilation is handed over to the new file by |\childdocforward|:
%    \begin{macrocode}
\newcommand{\childdocforwardprefix}[3][]
{
  \begingroup
    \def\childdocextract #2##1~~~{\def\childdoctmp{\childdocforward[#1]{#3##1}}}
    \expandafter\childdocextract\childdocname~~~
    \expandafter
  \endgroup
  \childdoctmp
}
%    \end{macrocode}

% \macro{\childdoc}
% The deprecated macro |\childdoc| is a legacy version of |\childdocmain|:
%    \begin{macrocode}
\newcommand{\childdoc}{\childdocmain}
%    \end{macrocode}

% \macro{\childdocredirect}
% The deprecated macro |\childdocredirect| is a legacy version
% of |\childdocforward| and |\childdocforwardprefix|:
%    \begin{macrocode}
\newcommand{\childdocredirect}[2][]
{
  \begingroup
    \if?#1?
      \def\childdoctmp{\childdocforward{#2}}
    \else
      \def\childdoctmp{\childdocforwardprefix{#1}{#2}}
    \fi
    \expandafter
  \endgroup
  \childdoctmp
}
%    \end{macrocode}

%\iffalse
%</package>
%\fi
%
\endinput
|\\
|\childdocmain{|\textit{main}|}|\\
\end{tabular}
\end{center}
%
If |\jobname| does not match the argument \textit{main} of |\childdocmain|,
it is assumed that |\jobname| points to the child file to be compiled.
When using |\childdocmain| with the main file specified as argument,
it suffices to start a child file
with just |\input{|\textit{main}|}|
without loading of the package and using |\childdocof|.
If instead all processing is done
with the appropriate \textsf{childdoc} directives,
the argument of \textit{main} of |\childdocmain| can be empty.

An alternative version of the command line processing described
in \secref{sec:commandline} using the detection mechanism reads:
%
\begin{center}
|... -jobname "|\textit{target}|" "|[\textit{flags}]%
[|\def\jobname{|\textit{dest}|}|]|\input{|\textit{main}|}"|
\end{center}

%%%%%%%%%%%%%%%%%%%%%%%%%%%%%%%%%%%%%%%%%%%%%%%%%%%%%%%%%%%%%%%%%%%%%%%%%%%%%%%%
\subsection{Manual Code}
\label{sec:manual}

In case one cannot be certain whether the definitions file |childdoc.def|
is installed on the target \TeX{} distribution
and one prefers not to ship it,
it is conceivable to paste a few relevant commands into the sources.

To that end, drop all statements |% \iffalse
%
% childdoc.dtx Copyright (C) 2017-2018 Niklas Beisert
%
% This work may be distributed and/or modified under the
% conditions of the LaTeX Project Public License, either version 1.3
% of this license or (at your option) any later version.
% The latest version of this license is in
%   http://www.latex-project.org/lppl.txt
% and version 1.3 or later is part of all distributions of LaTeX
% version 2005/12/01 or later.
%
% This work has the LPPL maintenance status `maintained'.
%
% The Current Maintainer of this work is Niklas Beisert.
%
% This work consists of the files childdoc.dtx and childdoc.ins
% and the derived files childdoc.def and cdocsamp.tex with
% cdocsch1.tex, cdocsch2.tex, cdocsdrf.tex, cdocsfn1.tex, cdocsfn2.tex.
%
%<package>\ifdefined\childdocmain\endinput\fi
%<package>\ProvidesFile{childdoc.def}[2018/12/30 v2.0 child document driver]
%<samplemain>\ProvidesFile{cdocsamp.tex}[2018/12/30 v2.0 sample for childdoc]
%<*driver>
%\ProvidesFile{childdoc.drv}[2018/12/30 v2.0 childdoc reference manual file]
\PassOptionsToClass{10pt,a4paper}{article}
\documentclass{ltxdoc}

\usepackage[margin=35mm]{geometry}
\usepackage{hyperref}
\usepackage{hyperxmp}
\usepackage[usenames]{color}

\hypersetup{colorlinks=true}
\hypersetup{pdfstartview=FitH}
\hypersetup{pdfpagemode=UseNone}
\hypersetup{pdfsource={}}
\hypersetup{pdflang={en-UK}}
\hypersetup{pdfcopyright={Copyright 2017-2018 Niklas Beisert.
  This work may be distributed and/or modified under the
  conditions of the LaTeX Project Public License, either version 1.3
  of this license or (at your option) any later version.}}
\hypersetup{pdflicenseurl={http://www.latex-project.org/lppl.txt}}
\hypersetup{pdfcontactaddress={ETH Zurich, ITP, HIT K,
  Wolfgang-Pauli-Strasse 27}}
\hypersetup{pdfcontactpostcode={8093}}
\hypersetup{pdfcontactcity={Zurich}}
\hypersetup{pdfcontactcountry={Switzerland}}
\hypersetup{pdfcontactemail={nbeisert@itp.phys.ethz.ch}}
\hypersetup{pdfcontacturl={http://people.phys.ethz.ch/\xmptilde nbeisert/}}

\newcommand{\secref}[1]{\hyperref[#1]{section \ref*{#1}}}

\parskip1ex
\parindent0pt
\let\olditemize\itemize
\def\itemize{\olditemize\parskip0pt}

\begin{document}

\title{The \textsf{childdoc} Package}
\hypersetup{pdftitle={The childdoc Package}}
\author{Niklas Beisert\\[2ex]
  Institut f\"ur Theoretische Physik\\
  Eidgen\"ossische Technische Hochschule Z\"urich\\
  Wolfgang-Pauli-Strasse 27, 8093 Z\"urich, Switzerland\\[1ex]
  \href{mailto:nbeisert@itp.phys.ethz.ch}
  {\texttt{nbeisert@itp.phys.ethz.ch}}}
\hypersetup{pdfauthor={Niklas Beisert}}
\hypersetup{pdfsubject={Manual for the LaTeX2e Package childdoc}}
\date{30 December 2018, \textsf{v2.0}}
\maketitle

\begin{abstract}\noindent
\textsf{childdoc} is a \LaTeXe{} package
that enables the direct compilation
of document sections included by |\include|
to individual files.
\end{abstract}

\begingroup
\parskip0ex
\tableofcontents
\endgroup

%%%%%%%%%%%%%%%%%%%%%%%%%%%%%%%%%%%%%%%%%%%%%%%%%%%%%%%%%%%%%%%%%%%%%%%%%%%%%%%%
%%%%%%%%%%%%%%%%%%%%%%%%%%%%%%%%%%%%%%%%%%%%%%%%%%%%%%%%%%%%%%%%%%%%%%%%%%%%%%%%
\section{Introduction}

\LaTeX{} provides a mechanism to structure a large document (such as a book)
into a main file and several child files (containing the chapters)
using the |\include| command.
This mechanism is beneficial for documents
which span hundreds of pages in order to
make the source file(s) more manageable.
Moreover, compilation can be restricted to
selected child files by means of the |\includeonly| command.
The latter feature can be used to reduce the compilation time while editing
(this was significantly more useful in the earlier days of \LaTeX{})
or to generate a smaller document which is easier to navigate.
Another application of |\includeonly| is to generate
documents consisting of selected parts of the complete document.

However, there are a few drawbacks of the plain |\include| mechanism:
\begin{itemize}
\item
The child files cannot be compiled on their own,
they can only be compiled via the main file.
A naive editing environment
(such as a text editor with an option
to have the current file processed by \LaTeX)
may require one to switch to the main file before compiling;
attempting to compile the child file produces errors.
\item
The main file must be modified (each time)
to adjust the |\includeonly| command
to the present needs. This easily leaves the main file in a messy state.
\item
The generated document will always carry the filename
of the main document. This is inconvenient if
several child files are to be compiled and
to be kept for distribution.
\end{itemize}

The present package provides a simple interface
to make child files individually compilable by \LaTeX{}.
Compiling a child file then has the same effect as compiling
the main file with an |\includeonly| command
to select the appropriate child.
Moreover the generated document will carry the name of the child
rather than the main file.
This resolves all three above issues.

This feature is meant to make the editing of books,
thesis documents and lecture notes somewhat more convenient.
However, the package can also be used efficiently for
composing a series of documents (such as exercise sheets)
which are typically distributed individually.
It then assists the author in generating the individual documents
(potentially in different versions)
as well as a document containing the collected series.
Another application is in developing style files
or other kinds of included material
where compilation of the style file could redirect
to a sample or test file.

%%%%%%%%%%%%%%%%%%%%%%%%%%%%%%%%%%%%%%%%%%%%%%%%%%%%%%%%%%%%%%%%%%%%%%%%%%%%%%%%
%%%%%%%%%%%%%%%%%%%%%%%%%%%%%%%%%%%%%%%%%%%%%%%%%%%%%%%%%%%%%%%%%%%%%%%%%%%%%%%%
\section{Usage}

First of all, the package \textsf{childdoc} is \emph{not} a standard
\LaTeXe{} |.sty| style file! Therefore it needs to be invoked in
a non-standard way.

%%%%%%%%%%%%%%%%%%%%%%%%%%%%%%%%%%%%%%%%%%%%%%%%%%%%%%%%%%%%%%%%%%%%%%%%%%%%%%%%
\subsection{Included Files}
\label{sec:include}

%%%%%%%%%%%%%%%%%%%%%%%%%%%%%%%%%%%%%%%%
\DescribeMacro{\childdocmain}
To use the package, add the commands
\begin{center}
\begin{tabular}{l}
|\input{childdoc.def}|\\
|\childdocmain{}|\\
\end{tabular}
\end{center}
at the very top of the main \LaTeX{} file,
in particular \emph{before} the |\documentclass| statement!
The argument of |\childdocmain| should be left empty
(but it must be present).

%%%%%%%%%%%%%%%%%%%%%%%%%%%%%%%%%%%%%%%%
\DescribeMacro{\childdocof}
Furthermore, add the commands
\begin{center}
\begin{tabular}{l}
|\input{childdoc.def}|\\
|\childdocof{|\textit{main}|}|\\
\end{tabular}
\end{center}
at the top of every child file \textit{child}
which is included by |\include{|\textit{child}|}|
from within the main file
(or at least for those files to be compiled individually).
The argument \textit{main} must be the filename of the main file.

There are a couple of
considerations in setting up the main and child documents:

%%%%%%%%%%%%%%%%%%%%%%%%%%%%%%%%%%%%%%%%
\paragraph{Restrictions.}

Please note the following restrictions:
\begin{itemize}
\item
|\childdocmain| must be called with one argument \textit{main}
to ensure compatibility with earlier version of the package.
It must either be empty (|\childdocmain{}|)
or precisely match the filename of the main file in which it is specified.
See \secref{sec:detection} for further information.
\item
The filename \textit{main} must be specified without the |.tex| extension.
\item
The filename \textit{main} is case sensitive
(even in case-insensitive file systems)
due to internal string comparison.
\item
The argument \textit{main} should be fully expanded, it cannot be a macro.
\item
Subdirectories and special characters should be avoided in filenames.
\item
The command |\childdocmain{|\textit{main}|}| must be followed by a whitespace.
It should not be followed immediately by another command
or by a comment mark `|%|'.
This is because the \TeX{} parser reads the token immediately following
the argument of |\childdocmain| and puts it
at the beginning of every child section;
however, a white\-space is ignored.
\end{itemize}

%%%%%%%%%%%%%%%%%%%%%%%%%%%%%%%%%%%%%%%%
\paragraph{Content of Main File.}

It is advisable to place all content in the child files included by |\include|.
Any output contained in the main file will appear in all child documents
unless suppressed manually;
it cannot be suppressed automatically by the |\includeonly| directive
and thus should normally be avoided.
A method to include some content in the main file
by means of conditional processing is described in \secref{sec:conditional}.

%%%%%%%%%%%%%%%%%%%%%%%%%%%%%%%%%%%%%%%%
\paragraph{Page Numbering.}

When only a part of the document is compiled,
the appropriate numbering of pages
(as well as other status parameters)
is determined from the |.aux| files.
The latter contain information from previous passes.
However this information needs to propagate through
all intermediate child documents.
Therefore the page numbering in child documents may well
be inconsistent until the complete document is compiled at least once.

A useful (if unconventional) way to always ensure a consistent
page numbering is to restart the numbering in each child document
and denote the pages by `\textit{child}|.|\textit{page}'
where \textit{child} represents the chapter/section number of the child file.
This can be achieved by the command
|\numberwithin{page}{|\textit{child}|}|
of the \textsf{amsmath} package
where \textit{child} can be |chapter| or |section|
depending on the chosen structuring.
Alternatively, one can modify the macro |\thepage| appropriately
and reset the counter |page| at the start of each child file.

%%%%%%%%%%%%%%%%%%%%%%%%%%%%%%%%%%%%%%%%%%%%%%%%%%%%%%%%%%%%%%%%%%%%%%%%%%%%%%%%
\subsection{Conditional Processing}
\label{sec:conditional}

The package provides a mechanism to compile different versions
of a document. To customise the versions further some conditional processing
can come in handy to distinguish which version is being compiled.
The package provides two macros to describe the compilation context:

%%%%%%%%%%%%%%%%%%%%%%%%%%%%%%%%%%%%%%%%
\DescribeMacro{\ifchilddoc}
The conditional |\ifchilddoc| distinguishes between the compilation of
child documents and the main document:
%
\begin{center}
|\ifchilddoc |\textit{child-code}| |[|\||else |\textit{main-code}]| \||fi|
\end{center}

%%%%%%%%%%%%%%%%%%%%%%%%%%%%%%%%%%%%%%%%
\DescribeMacro{\childdocname}
\DescribeMacro{\childdocjob}
The macro |\childdocname| contains the filename (without extension)
of the main or child file being processed.
Note that |\childdocjob| will always contain the name of the main file.

%%%%%%%%%%%%%%%%%%%%%%%%%%%%%%%%%%%%%%%%
\paragraph{Title Page.}

Conditional processing can be used to include a title or banner page
in the main document when proper precautions are taken.
Importantly, the code in the main file should ensure that the page counter
(as well as other status parameters which are stored in the |.aux| files)
takes the same value after the conditional processing.
Otherwise the page numbers may take divergent values
depending on which part is compiled.

For example, a title page could be declared by:
%
\begin{center}
\begin{tabular}{l}
|\ifchilddoc\||else|\\
|\addtocounter{page}{-1}|\\
\textit{code for title page}\\
|\newpage|\\
|\||fi|
\end{tabular}
\end{center}
%
A banner page for the child documents can be generated by:
%
\begin{center}
\begin{tabular}{l}
|\ifchilddoc|\\
|\addtocounter{page}{-1}|\\
\textit{code for banner page}\\
|\newpage|\\
|\||fi|
\end{tabular}
\end{center}
%
Here one could write a message such as:
\begin{center}
|This is the part \childdocname{} of \childdocjob{}.|
\end{center}

%%%%%%%%%%%%%%%%%%%%%%%%%%%%%%%%%%%%%%%%%%%%%%%%%%%%%%%%%%%%%%%%%%%%%%%%%%%%%%%%
\subsection{Flags}
\label{sec:flags}

The package makes it easy to generate different versions
of the main or child documents.
To this end compilation flags can be defined
and assigned different default values.
They will be particularly useful in conjunction
with the forwarding mechanism described in \secref{sec:forward}.

For example, it may be useful to have a flag |\version|
which can be set to |draft| or |final|.
The document source will contain some conditional code
depending on the value of |\version|.
Suppose further, the flag should default to |final| for the main file
and to |draft| for child files
which is a natural assignment for editing the document.
This is achieved by placing the following code
in the preamble of the main document
(below the |\childdocmain| directive):
%
\begin{center}
\begin{tabular}{l}
|\ifchilddoc|\\
|\providecommand{\version}{draft}|\\
|\||else|\\
|\providecommand{\version}{final}|\\
|\||fi|
\end{tabular}
\end{center}
%
The definition by |\providecommand| makes sure
that previous definitions are not overwritten.
Further statements |\providecommand{\version}{...}|
can thus be added before the above code to override it.

For the main file, one might add a line
(between |\childdocmain| and the above block)
%
\begin{center}
|%\ifchilddoc\||else\providecommand{\version}{draft}\||fi|
\end{center}
%
which can be uncommented to produce a draft version.
Likewise one can add a line to the very top of a child file
(above the |\childdocof{|\textit{main}|}| directive)
%
\begin{center}
|%\providecommand{\version}{final}|
\end{center}
%
which can be uncommented to produce the final version of this child document.

%%%%%%%%%%%%%%%%%%%%%%%%%%%%%%%%%%%%%%%%%%%%%%%%%%%%%%%%%%%%%%%%%%%%%%%%%%%%%%%%
\subsection{Forwarding}
\label{sec:forward}

Different versions of the main or child documents
using compilation flags as described in \secref{sec:flags}
can be (permanently) stored in different files
for convenient compilation, viewing and distribution.
To this end, the package defines a command
to pass on compilation to a different file:

%%%%%%%%%%%%%%%%%%%%%%%%%%%%%%%%%%%%%%%%
\DescribeMacro{\childdocforward}
The command |\childdocforward| redirects processing to
another source file:
%
\begin{center}
\begin{tabular}{l}
|\input{childdoc.def}|\\
|\childdocforward[|\textit{main}|]{|\textit{dest}|}|\\
\end{tabular}
\end{center}
%
The argument \textit{dest} is the destination file
(without extension).
It should be the main file or one of the child files.
Note that further \textsf{childdoc} directives
such as |\childdocof| and |\childdocforward|
in the indicated file will be processed in this form.
The optional argument \textit{main}
passes on directly to the main file \textit{main}
while pretending to compile the child \textit{dest}.
This form behaves as if \textit{dest}
issues |\childdocof{|\textit{main}|}| right away,
and no further \textsf{childdoc} directives will be processed.

%%%%%%%%%%%%%%%%%%%%%%%%%%%%%%%%%%%%%%%%
\DescribeMacro{\...prefix}
In the alternative form |\childdocforwardprefix|,
%
\begin{center}
\begin{tabular}{l}
|\input{childdoc.def}|\\
|\childdocforwardprefix[|\textit{main}|]{|\textit{prefix}|}{|\textit{dest}|}|
\end{tabular}
\end{center}
%
the destination file is determined by a pattern
depending on the current file:
To make this work, the current file must be called
`{\textit{prefix}\hspace{0.2em}\textit{suffix}}'
with \textit{prefix} matching precisely the argument.
Processing is then passed on to the file
`{\textit{dest}\hspace{0.2em}\textit{suffix}}'.
Surely, the same effect is achieved by
directly specifying the
argument `{\textit{dest}\hspace{0.2em}\textit{suffix}}'
in the first form.
However, that requires to set up a different file
for each child. With the alternative form of the command
all these files can have exactly the same content
which simplifies setting them up and maintaining them.

For example, the following file |draft.tex|
with a compilation flag |\version| as described in \secref{sec:flags}
compiles the main document as a draft:
%
\begin{center}
\begin{tabular}{l}
|\def\version{draft}|\\
|\input{childdoc.def}|\\
|\childdocforward{|\textit{main}|}|
\end{tabular}
\end{center}
%
Likewise, the following files |final|\textit{nn}|.tex|
compile the final version of the child document
|child|\textit{nn}|.tex|:
%
\begin{center}
\begin{tabular}{l}
|\def\version{final}|\\
|\input{childdoc.def}|\\
|\childdocforwardprefix{final}{child}|
\end{tabular}
\end{center}
%

Note that when several versions of a main file and/or of each child file
are to be generated, it may be convenient to set up a |Makefile| or
shell script to automatise the process.

%%%%%%%%%%%%%%%%%%%%%%%%%%%%%%%%%%%%%%%%%%%%%%%%%%%%%%%%%%%%%%%%%%%%%%%%%%%%%%%%
\subsection{Command Line Processing}
\label{sec:commandline}

The effect of redirection files can also be achieved by invoking
the \LaTeX{} compiler with a more elaborate command line.
Most conveniently this should be done as part
of a shell script or a |Makefile|.

When using \textsf{childdoc} in the main file, the following
command lines effectively perform a redirection
(note that depending on the shell being used,
backslashes may have to be doubled: `|\|' $\to$ `|\\|'):
%
\begin{center}
|... -jobname "|\textit{target}|" |\\|"|[\textit{flags}]%
|\input{childdoc.def}\childdocforward[|\textit{main}|]{|\textit{dest}|}"|
\end{center}
%
Here \textit{target} is the name of the output file,
\textit{main} is the name of the main file
and \textit{dest} is the name of the main or child file to be processed
(all filenames without extensions).
The optional argument \textit{main} can be omitted
if \textit{main} matches \textit{dest}.
Optionally, compilation \textit{flags} can be defined via |\def| commands.
This command line makes the \TeX{} engine believe
it is compiling the file \textit{target}
whose content is specified as the latter parameter.
The provided code then forwards the processing to
\textit{main} or \textit{dest} as described in \secref{sec:forward}.

%%%%%%%%%%%%%%%%%%%%%%%%%%%%%%%%%%%%%%%%%%%%%%%%%%%%%%%%%%%%%%%%%%%%%%%%%%%%%%%%
\subsection{Include by Input}
\label{sec:input}

Including child documents by |\include| has some restrictions by design.
Most notably, the content of a child document always occupies
its own set of pages; pages cannot be shared between child documents.
Usually, this behaviour makes perfect sense
because each child document contain an essential part of the document.
However, in some situations it may be desirable to compose
a document from a collection of parts
without having mandatory page breaks between then.
For this case, the package
provides a mechanism to include parts
by |\input| which can also be processed individually.
However, by construction this mechanism
requires manual handling of the content to be output.

%%%%%%%%%%%%%%%%%%%%%%%%%%%%%%%%%%%%%%%%
\DescribeMacro{\ifchilddocmanual}
The main file should be prepared as usual, see \secref{sec:include}.
However, the document body must make a distinction
between processing of an individual part and of the main document, e.g.:
%
\begin{center}
\begin{tabular}{l}
|\ifchilddocmanual|\\
|\input{\childdocname}|\\
|\||else|\\
\textit{document body with }|\input{|\textit{part}|}|\\
|\||fi|
\end{tabular}
\end{center}
%
The conditional |\ifchilddocmanual| is true whenever
a part to be included by |\input| is being compiled,
and the name of the part is stored in |\childdocname|.

%%%%%%%%%%%%%%%%%%%%%%%%%%%%%%%%%%%%%%%%
\DescribeMacro{\childdocby}
Each part to be included by |\input| should start with:
%
\begin{center}
\begin{tabular}{l}
|\input{childdoc.def}|\\
|\childdocby{|\textit{main}|}|\\
\end{tabular}
\end{center}
%
The directive |\childdocby| is similar to |\childdocof|
described in \secref{sec:include},
but the subsequent selection of content must be done manually.
To that end, both |\ifchilddoc| and |\ifchilddocmanual|
will be true upon processing of a part,
and the name of the part is stored in |\childdocname|.
Note that |\jobname| will be set to the filename of the current part
so that each part receives an individual |.aux| file
that does not interfere with the |.aux| file(s) of the main document.
This behaviour can be altered by the alternative form
|\childdocby[*]{|\textit{main}|}| (with a non-empty optional argument)
which uses the |.aux| file of the main document
by setting |\jobname| to \textit{main}.

%%%%%%%%%%%%%%%%%%%%%%%%%%%%%%%%%%%%%%%%%%%%%%%%%%%%%%%%%%%%%%%%%%%%%%%%%%%%%%%%
\subsection{Driver Development}
\label{sec:driver}

The \textsf{childdoc} mechanism can also be use for the development
of definition files such as \LaTeX{} styles or classes.
This case differs from the above setup with multiple parts
included by |\include| in that no |\includeonly| should be invoked.
This can be achieved by starting the include file
(before |\ProvidesPackage|) with:
%
\begin{center}
\begin{tabular}{l}
|\input{childdoc.def}|\\
|\childdocforward{|\textit{main}|}|\\
\end{tabular}
\end{center}
%
or alternatively with:
%
\begin{center}
\begin{tabular}{l}
|\input{childdoc.def}|\\
|\childdocby{|\textit{main}|}|\\
\end{tabular}
\end{center}
%
Both forms have slightly different effects as described above.
The main file is prepared as usual, see \secref{sec:include}.

%%%%%%%%%%%%%%%%%%%%%%%%%%%%%%%%%%%%%%%%%%%%%%%%%%%%%%%%%%%%%%%%%%%%%%%%%%%%%%%%
\subsection{Legacy Detection}
\label{sec:detection}

The directive |\childdocmain| in the main file can detect
whether the complete document or merely a child is to be compiled
even without using the directive |\childdocof|.
This method is deprecated because it is less robust
and there is no compelling reason to use it;
it is merely provided for backward compatibility
and it may be removed in future versions.

If the detection mechanism is to be used,
it is mandatory to correctly specify
the filename of the main file as the argument of |\childdocmain|:
%
\begin{center}
\begin{tabular}{l}
|\input{childdoc.def}|\\
|\childdocmain{|\textit{main}|}|\\
\end{tabular}
\end{center}
%
If |\jobname| does not match the argument \textit{main} of |\childdocmain|,
it is assumed that |\jobname| points to the child file to be compiled.
When using |\childdocmain| with the main file specified as argument,
it suffices to start a child file
with just |\input{|\textit{main}|}|
without loading of the package and using |\childdocof|.
If instead all processing is done
with the appropriate \textsf{childdoc} directives,
the argument of \textit{main} of |\childdocmain| can be empty.

An alternative version of the command line processing described
in \secref{sec:commandline} using the detection mechanism reads:
%
\begin{center}
|... -jobname "|\textit{target}|" "|[\textit{flags}]%
[|\def\jobname{|\textit{dest}|}|]|\input{|\textit{main}|}"|
\end{center}

%%%%%%%%%%%%%%%%%%%%%%%%%%%%%%%%%%%%%%%%%%%%%%%%%%%%%%%%%%%%%%%%%%%%%%%%%%%%%%%%
\subsection{Manual Code}
\label{sec:manual}

In case one cannot be certain whether the definitions file |childdoc.def|
is installed on the target \TeX{} distribution
and one prefers not to ship it,
it is conceivable to paste a few relevant commands into the sources.

To that end, drop all statements |\input{childdoc.def}|
and perform the replacements as outlined below.
Instead of |\childdocmain{|\textit{main}|}| add the following code
to the top of the main file:
%
\begin{center}
\begin{tabular}{l}
|\||ifdefined\childdocname\endinput\||fi\newif\ifchilddoc|\\
|\edef\childdocname{\scantokens\expandafter{\jobname\noexpand}}|\\
|\def\childdocmain{|\textit{main}|}\||ifx\childdocmain\childdocname\||else|\\
|\childdoctrue\includeonly{\childdocname}\let\jobname\childdocmain\||fi|\\
\end{tabular}
\end{center}
%
Instead of |\childdocof{|\textit{main}|}| just include the main file
at the top of each child file:
%
\begin{center}
|\input{|\textit{main}|}|
\end{center}
%
A simple redirection |\childdocforward{|\textit{dest}|}| is achieved by:
%
\begin{center}
|\def\jobname{|\textit{dest}|}\input{\jobname}|
\end{center}
%
The redirection with prefix
|\childdocforwardprefix[|\textit{prefix}|]{|\textit{dest}|}|
is accomplished by:
%
\begin{center}
\begin{tabular}{l}
|{\edef\jobname{\scantokens\expandafter{\jobname\noexpand}}|\\
|\def\redirectjob |\textit{prefix}|#1~~~{\gdef\jobname{|\textit{dest}|#1}}|\\
|\expandafter\redirectjob\jobname~~~}\input{\jobname}|
\end{tabular}
\end{center}

In an alternative approach,
child documents can be compiled by a specific command line
without additional code or specific definitions:
%
\begin{center}
|... -jobname "|\textit{target}|" "|[\textit{flags}]%
|\includeonly{|\textit{dest}|}\input{|\textit{main}|}"|
\end{center}
%

%%%%%%%%%%%%%%%%%%%%%%%%%%%%%%%%%%%%%%%%%%%%%%%%%%%%%%%%%%%%%%%%%%%%%%%%%%%%%%%%
%%%%%%%%%%%%%%%%%%%%%%%%%%%%%%%%%%%%%%%%%%%%%%%%%%%%%%%%%%%%%%%%%%%%%%%%%%%%%%%%
\section{Information}

%%%%%%%%%%%%%%%%%%%%%%%%%%%%%%%%%%%%%%%%%%%%%%%%%%%%%%%%%%%%%%%%%%%%%%%%%%%%%%%%
\subsection{Copyright}

Copyright \copyright{} 2017--2018 Niklas Beisert

This work may be distributed and/or modified under the
conditions of the \LaTeX{} Project Public License, either version 1.3
of this license or (at your option) any later version.
The latest version of this license is in
  \url{http://www.latex-project.org/lppl.txt}
and version 1.3 or later is part of all distributions of \LaTeX{}
version 2005/12/01 or later.

This work has the LPPL maintenance status `maintained'.

The Current Maintainer of this work is Niklas Beisert.

This work consists of the files |README.txt|, |childdoc.ins| and |childdoc.dtx|
as well as the derived files |childdoc.def|, |cdocsamp.tex|
with |cdocsch1.tex|, |cdocsch2.tex|, |cdocspt3.tex|, |cdocspt4.tex|,
|cdocsdrf.tex|, |cdocsfn1.tex|, |cdocsfn2.tex|
as well as |childdoc.pdf|.

%%%%%%%%%%%%%%%%%%%%%%%%%%%%%%%%%%%%%%%%%%%%%%%%%%%%%%%%%%%%%%%%%%%%%%%%%%%%%%%%
\subsection{Files and Installation}

The package consists of the files:
%
\begin{center}
\begin{tabular}{ll}
    |README.txt|   & readme file \\
    |childdoc.ins| & installation file \\
    |childdoc.dtx| & source file \\
    |childdoc.def| & definition file \\
    |cdocsamp.tex| & sample main file \\
    |cdocsch1.tex| & sample include file \\
    |cdocsch2.tex| & sample include file \\
    |cdocspt3.tex| & sample part file \\
    |cdocspt4.tex| & sample part file \\
    |cdocsdrf.tex| & sample redirection file \\
    |cdocsfn1.tex| & sample redirection file \\
    |cdocsfn2.tex| & sample redirection file \\
    |childdoc.pdf| & manual
\end{tabular}
\end{center}
%
The distribution consists of the files
|README.txt|, |childdoc.ins| and |childdoc.dtx|.
%
\begin{itemize}
\item
Run (pdf)\LaTeX{} on |childdoc.dtx|
to compile the manual |childdoc.pdf| (this file).
\item
Run \LaTeX{} on |childdoc.ins| to create the definitions file |childdoc.def|
and the sample |cdocsamp.tex| with include files
|cdocsch1.tex|, |cdocsch2.tex|, |cdocspt3.tex|, |cdocspt4.tex|,
|cdocsdrf.tex|, |cdocsfn1.tex|, |cdocsfn2.tex|.
Then copy the file |childdoc.def| to an appropriate directory of your \LaTeX{}
distribution, e.g.\ \textit{texmf-root}|/tex/latex/childdoc|.
\end{itemize}

%%%%%%%%%%%%%%%%%%%%%%%%%%%%%%%%%%%%%%%%%%%%%%%%%%%%%%%%%%%%%%%%%%%%%%%%%%%%%%%%
\subsection{Related CTAN Packages}

There are several other packages which offer a similar functionality:
%
\begin{itemize}
\item
The packages
\href{http://ctan.org/pkg/docmute}{\textsf{docmute}},
\href{http://ctan.org/pkg/includex}{\textsf{includex}} and
\href{http://ctan.org/pkg/standalone}{\textsf{standalone}}
provide commands to include only the document body of
a child file thus allowing both files to be compiled individually.
\item
The packages \href{http://ctan.org/pkg/subdocs}{\textsf{subdocs}}
and \href{http://ctan.org/pkg/subfiles}{\textsf{subfiles}}
provide structures in which the main and child documents can be
encapsulated and allowing them to be compiled individually.
The inclusion mechanism is different from the conventional |\include|.
\item
The package \href{http://ctan.org/pkg/combine}{\textsf{combine}}
is an elaborate solution to combine several documents into one.
\end{itemize}
%
See also the CTAN topic \href{http://ctan.org/topic/subdocs}{\textsf{subdocs}}
for further related packages.
The present package differs from the above solutions in that
a document structure constructed with the conventional |\include| mechanism
just needs two extra commands at the top of every file
such that all constituent files can be compiled individually.

%%%%%%%%%%%%%%%%%%%%%%%%%%%%%%%%%%%%%%%%%%%%%%%%%%%%%%%%%%%%%%%%%%%%%%%%%%%%%%%%
%\subsection{Feature Suggestions}
%
%The following is a list of features which may be useful for future
%versions of this package:
%%
%\begin{itemize}
%\item
%\ldots
%\end{itemize}

%%%%%%%%%%%%%%%%%%%%%%%%%%%%%%%%%%%%%%%%%%%%%%%%%%%%%%%%%%%%%%%%%%%%%%%%%%%%%%%%
\subsection{Revision History}

%%%%%%%%%%%%%%%%%%%%%%%%%%%%%%%%%%%%%%%%
\paragraph{v2.0:} 2018/12/30

\begin{itemize}
\item
immediate forward processing
\item
added |\childdocby| mechanism
\item
manual restructured
\end{itemize}

%%%%%%%%%%%%%%%%%%%%%%%%%%%%%%%%%%%%%%%%
\paragraph{v1.6:} 2018/01/17

\begin{itemize}
\item
application for development of include files
\item
corrections to manual
\end{itemize}

%%%%%%%%%%%%%%%%%%%%%%%%%%%%%%%%%%%%%%%%
\paragraph{v1.5:} 2017/05/21

\begin{itemize}
\item
more complete structuring introduced
\item
|\childdocof| introduced
\item
|\childdoc| renamed to |\childdocmain|
\item
|\childredirect| renamed to |\childdocforward| and |\childdocforwardprefix|
and functionality expanded
\end{itemize}

%%%%%%%%%%%%%%%%%%%%%%%%%%%%%%%%%%%%%%%%
\paragraph{v1.0:} 2017/04/27

\begin{itemize}
\item
manual and install package
\item
first version published on CTAN
\end{itemize}

%%%%%%%%%%%%%%%%%%%%%%%%%%%%%%%%%%%%%%%%
\paragraph{v0.6:} 2017/04/26

\begin{itemize}
\item
redirection mechanism added
\end{itemize}

%%%%%%%%%%%%%%%%%%%%%%%%%%%%%%%%%%%%%%%%
\paragraph{v0.5:} 2017/04/26

\begin{itemize}
\item
functionality in definition file
\end{itemize}


%%%%%%%%%%%%%%%%%%%%%%%%%%%%%%%%%%%%%%%%%%%%%%%%%%%%%%%%%%%%%%%%%%%%%%%%%%%%%%%%
%%%%%%%%%%%%%%%%%%%%%%%%%%%%%%%%%%%%%%%%%%%%%%%%%%%%%%%%%%%%%%%%%%%%%%%%%%%%%%%%
%%%%%%%%%%%%%%%%%%%%%%%%%%%%%%%%%%%%%%%%%%%%%%%%%%%%%%%%%%%%%%%%%%%%%%%%%%%%%%%%
\appendix

\settowidth\MacroIndent{\rmfamily\scriptsize 000\ }

 \DocInput{childdoc.dtx}

\end{document}
%</driver>
% \fi
%
% %%%%%%%%%%%%%%%%%%%%%%%%%%%%%%%%%%%%%%%%%%%%%%%%%%%%%%%%%%%%%%%%%%%%%%%%%%%%%%
% %%%%%%%%%%%%%%%%%%%%%%%%%%%%%%%%%%%%%%%%%%%%%%%%%%%%%%%%%%%%%%%%%%%%%%%%%%%%%%
% \section{Sample}
%\iffalse
%<*samplemain>
%\fi
%
% The following presents a sample document
% with two chapters, two parts, a title page,
% a compile flag as well as three forwarding files to set the flag.
% It consists of eight |.tex| files:
% \begin{center}
% \begin{tabular}{ll}
% |cdocsamp.tex|&main file\\
% |cdocsch1.tex|&include file for chapter 1\\
% |cdocsch2.tex|&include file for chapter 2\\
% |cdocspt3.tex|&include file for part 3\\
% |cdocspt4.tex|&include file for part 4\\
% |cdocsdrf.tex|&forwarding file for main file in draft mode\\
% |cdocsfi1.tex|&forwarding file for final version of chapter 1\\
% |cdocsfi2.tex|&forwarding file for final version of chapter 2\\
% \end{tabular}
% \end{center}
% Each of the eight files can be compiled directly by the \LaTeX{} compiler.
%
% %%%%%%%%%%%%%%%%%%%%%%%%%%%%%%%%%%%%%%
% \paragraph{Main File.}
%
% The main file is called |cdocsamp.tex|.
%
% Load the \textsf{childdoc} definitions and
% declare the filename for the main document:
%    \begin{macrocode}
\input{childdoc.def}
\childdocmain{}
%    \end{macrocode}

% Optional override for |\version| flag:
%    \begin{macrocode}
%%\ifchilddoc\else\providecommand{\version}{draft}\fi
%    \end{macrocode}

% Define the default values for the |\version| flag
% (|final| for the main file and |draft| for childs):
%    \begin{macrocode}
\ifchilddoc
\providecommand{\version}{draft}
\else
\providecommand{\version}{final}
\fi
%    \end{macrocode}

% Load the standard document class:
%    \begin{macrocode}
\documentclass[12pt]{article}
%    \end{macrocode}

% Start the document body:
%    \begin{macrocode}
\begin{document}
%    \end{macrocode}

% Declare a title page.
% Print title, part of document being processed and version flag:
%    \begin{macrocode}
\addtocounter{page}{-1}
\begin{center}
{\LARGE\bfseries{}childdoc example\par}
\vspace{1cm}
\ifchilddoc
\ifchilddocmanual part\else chapter\fi:
`\childdocname' of `\childdocjob'\par
\else
main document: `\childdocjob'\par
\fi
version: \version\par
\end{center}
\newpage
%    \end{macrocode}

% Manually include selected file,
% otherwise process as usual:
%    \begin{macrocode}
\ifchilddocmanual
\section*{part `\childdocname'}
\input{\childdocname}
\else
%    \end{macrocode}

% Include the two chapters:
%    \begin{macrocode}
\include{cdocsch1}
\include{cdocsch2}
%    \end{macrocode}

% Include the two parts unless only chapters should be displayed:
%    \begin{macrocode}
\ifchilddoc\else
\section{part three}
\input{cdocspt3}
\section{part four}
\input{cdocspt4}
\fi
%    \end{macrocode}

% Process as usual until here:
%    \begin{macrocode}
\fi
%    \end{macrocode}

% End of document body:
%    \begin{macrocode}
\end{document}
%    \end{macrocode}
%\iffalse
%</samplemain>
%\fi
%
% %%%%%%%%%%%%%%%%%%%%%%%%%%%%%%%%%%%%%%
% \paragraph{Chapter Include Files.}
%
% The include files are called |cdocsch1.tex| and |cdocsch2.tex|.
%
%\iffalse
%<*samplechap1|samplechap2>
%\fi

% Optional override for |\version| flag:
%    \begin{macrocode}
%%\providecommand{\version}{final}
%    \end{macrocode}

% Include the main document:
%    \begin{macrocode}
\input{childdoc.def}
\childdocof{cdocsamp}
%    \end{macrocode}

%\iffalse
%</samplechap1|samplechap2>
%\fi
%
%\iffalse
%<*samplechap1>
%\fi
% Some text for chapter 1:
%    \begin{macrocode}
\section{one}
some text in chapter one
%    \end{macrocode}

%\iffalse
%</samplechap1>
%\fi
% Some text for chapter 2:
%\iffalse
%<*samplechap2>
%\fi
%    \begin{macrocode}
\section{two}
more text in chapter two
%    \end{macrocode}

%\iffalse
%</samplechap2>
%\fi
%
% %%%%%%%%%%%%%%%%%%%%%%%%%%%%%%%%%%%%%%
% \paragraph{Part Include Files.}
%
% The include files are called |cdocspt3.tex| and |cdocspt4.tex|.
%
%\iffalse
%<*samplepart3|samplepart4>
%\fi

% Optional override for |\version| flag:
%    \begin{macrocode}
%%\providecommand{\version}{final}
%    \end{macrocode}

% Include the main document:
%    \begin{macrocode}
\input{childdoc.def}
\childdocby{cdocsamp}
%    \end{macrocode}

%\iffalse
%</samplepart3|samplepart4>
%\fi
%
%\iffalse
%<*samplepart3>
%\fi
% Some text for part 3:
%    \begin{macrocode}
some text in part three
%    \end{macrocode}

%\iffalse
%</samplepart3>
%\fi
% Some text for part 4:
%\iffalse
%<*samplepart4>
%\fi
%    \begin{macrocode}
more text in part four
%    \end{macrocode}

%\iffalse
%</samplepart4>
%\fi
%
% %%%%%%%%%%%%%%%%%%%%%%%%%%%%%%%%%%%%%%
% \paragraph{Forwarding for a Complete Draft.}
%
% The following forwarding file |cdocsdrf.tex|
% compiles the main document in draft mode:
%\iffalse
%<*sampledraft>
%\fi
%    \begin{macrocode}
\def\version{draft}
\input{childdoc.def}
\childdocforward{cdocsamp}
%    \end{macrocode}

%\iffalse
%</sampledraft>
%\fi
%
% %%%%%%%%%%%%%%%%%%%%%%%%%%%%%%%%%%%%%%
% \paragraph{Forwarding for Final Version of the Chapters.}
%
% The following forwarding files |cdocsfn1.tex| and |cdocsfn2.tex|
% (with identical content)
% compile the final versions of the child documents
% |cdocsch1.tex| and |cdocsch2.tex|, respectively:
%\iffalse
%<*samplefinal>
%\fi
%    \begin{macrocode}
\def\version{final}
\input{childdoc.def}
\childdocforwardprefix[cdocsamp]{cdocsfn}{cdocsch}
%    \end{macrocode}

%\iffalse
%</samplefinal>
%\fi
%
% %%%%%%%%%%%%%%%%%%%%%%%%%%%%%%%%%%%%%%
% \paragraph{Command Line Processing.}
%
% The following three command lines generate the output files
% |cdocscld|, |cdocscl1| and |cdocscl2|
% which should be identical to
% |cdocsdrf|, |cdocsch1| and |cdocsfn2|, respectively:
% \begin{center}
% \begin{tabular}{l}
% |latex -jobname cdocscld \|\\
% |  "\def\version{draft}\input{childdoc.def}\childdocforward{cdocsamp}"|\\
% |latex -jobname cdocscl1 \|\\
% |  "\input{childdoc.def}\childdocforward[cdocsamp]{cdocsch1}"|\\
% |latex -jobname cdocscl2 \|\\
% |  "\def\version{final}\input{childdoc.def}\childdocforward{cdocsch2}"|
% \end{tabular}
% \end{center}
% Note that the trailing backslash on each first line
% merely continues the input to the second line
% (for convenient cut ant paste).
% Furthermore, the command |latex| can be replaced by any
% of its alternative versions such as |pdflatex|.
%
% %%%%%%%%%%%%%%%%%%%%%%%%%%%%%%%%%%%%%%%%%%%%%%%%%%%%%%%%%%%%%%%%%%%%%%%%%%%%%%
% %%%%%%%%%%%%%%%%%%%%%%%%%%%%%%%%%%%%%%%%%%%%%%%%%%%%%%%%%%%%%%%%%%%%%%%%%%%%%%
% \section{Implementation}
%\iffalse
%<*package>
%\fi
%
% This section describes the definitions file |childdoc.def|.

% The definitions cannot be loaded using |\usepackage| or |\RequirePackage|
% which has a mechanism to prevent loading a style file more than once.
% When loading the definitions by means of |\input|
% multiple instances have to be prevented manually:
%\iffalse
%This code needs to be before the `\ProvidesFile' directive
%which is defined at the beginning of this file.
%Therefore it is also placed there and commented out here.
%</package>
%<*discard>
%\fi
%    \begin{macrocode}
\ifdefined\childdocmain\endinput\fi
%    \end{macrocode}
%\iffalse
%</discard>
%<*package>
%\fi
%
% \macro{\ifchilddoc}
% \macro{\ifchilddocmanual}
% The conditional |\ifchilddoc| tells whether a
% child (true) or main (false) document is being compiled.
% The conditional |\ifchilddocmanual| tells whether
% the |\includeonly| mechanism is used (false) or
% the selection of child files must be performed manually (true).
% The definitions initialise to false:
%    \begin{macrocode}
\newif\ifchilddoc
\newif\ifchilddocmanual
%    \end{macrocode}

% \macro{\childdocname}
% \macro{\childdocjob}
% The macro |\childdocname| stores the name of the main document
% to be compiled. The macro |\childdocjob| stores the name of
% the document on which the \LaTeX{} compiler was originally invoked.
% The content of |\jobname| cannot be compared
% to filenames specified in the source due to different catcodes.
% The following code rescans |\jobname|, stores the result
% in |\childdocname| and saves a copy in |\childdocjob|:
%    \begin{macrocode}
\edef\childdocname{\scantokens\expandafter{\jobname\noexpand}}
\let\childdocjob\childdocname
%    \end{macrocode}

% \macro{\childdocdisable}
% The macro |\childdocdisable| prevents the main file
% from being processed more than once.
% At this stage, the main document command |\childdocmain|
% is assumed to be called once again where it should do nothing.
% Any subsequent call to it should prevent
% a secondary processing of the main document
% It overwrites the forwarding commands
% |\childdocof| and |\childdocforward|
% with empty macros to prevent further inclusions of the main document:
%    \begin{macrocode}
\newcommand{\childdocdisable}
{
  \renewcommand{\childdocmain}[1]{\renewcommand{\childdocmain}[1]{\endinput}}
  \renewcommand{\childdocof}[1]{}
  \renewcommand{\childdocby}[2][]{}
  \renewcommand{\childdocforward}[2][]{}
  \renewcommand{\childdocdisable}{}
}
%    \end{macrocode}

% \macro{\childdocmain}
% The macro |\childdocmain| is to be called at the top of the main file
% with nothing or the main filename (without extension) as argument.
% First, it breaks loops.
% If the argument is not empty and does not match |\childdocname|
% (which is set by the first inclusion of |childdoc.def|),
% |\ifchilddoc| is set to true, |\includeonly| is applied to the child file
% and |\jobname| is set to the main file
% (for proper handling of |.aux| files):
%    \begin{macrocode}
\newcommand{\childdocmain}[1]
{
  \childdocdisable\childdocmain{}
  \if?#1?\else
    \begingroup
      \def\childdoctmp{#1}
      \ifx\childdoctmp\childdocname
        \def\childdoctmp{}
      \else
        \def\childdoctmp
        {
          \childdoctrue
          \includeonly{\childdocname}
          \def\childdocjob{#1}
          \def\jobname{#1}
        }
      \fi
      \expandafter
    \endgroup
    \childdoctmp
  \fi
}
%    \end{macrocode}

% \macro{\childdocof}
% The command |\childdocof| redirects
% compilation to the main file |#1|.
%    \begin{macrocode}
\newcommand{\childdocof}[1]
{
  \childdocdisable
  \childdoctrue
  \includeonly{\childdocname}
  \def\jobname{#1}
  \def\childdocjob{#1}
  \input{#1}
}
%    \end{macrocode}

% \macro{\childdocby}
% The command |\childdocby| ....
%    \begin{macrocode}
\newcommand{\childdocby}[2][]
{
  \childdocdisable
  \childdoctrue
  \childdocmanualtrue
  \if?#1?\else
    \def\jobname{#2}
  \fi
  \def\childdocjob{#2}
  \input{#2}
  \endinput
}
%    \end{macrocode}

% \macro{\childdocforward}
% The command |\childdocforward| redirects
% compilation to the main file or
% (if the optional argument is given) a child file.
% Parameters are set as if the main file
% or a child file starting with |\childdocof| was compiled.
% Then compilation is handed over to the main file:
%    \begin{macrocode}
\newcommand{\childdocforward}[2][]
{
  \begingroup
    \if?#1?
      \def\childdoctmp
      {
        \def\childdocname{#2}
        \def\childdocjob{#2}
        \def\jobname{#2}
        \input{#2}
        \endinput
      }
    \else
      \def\childdoctmp
      {
        \childdocdisable
        \def\childdocname{#2}
        \childdoctrue
        \includeonly{#2}
        \def\childdocjob{#1}
        \def\jobname{#1}
        \input{#1}
        \endinput
      }
    \fi
    \expandafter
  \endgroup
  \childdoctmp
}
%    \end{macrocode}

% \macro{\childdocforwardprefix}
% The command |\childdocforwardprefix| redirects
% compilation to the main or a child file by means of a pattern.
% The prefix |#1| in the current filename is replaced by |#2|
% and the suffix of the current filename is kept
% (it is assumed that the filename does not contain the substring `|~~~|'
% which is used as a delimiter).
% Compilation is handed over to the new file by |\childdocforward|:
%    \begin{macrocode}
\newcommand{\childdocforwardprefix}[3][]
{
  \begingroup
    \def\childdocextract #2##1~~~{\def\childdoctmp{\childdocforward[#1]{#3##1}}}
    \expandafter\childdocextract\childdocname~~~
    \expandafter
  \endgroup
  \childdoctmp
}
%    \end{macrocode}

% \macro{\childdoc}
% The deprecated macro |\childdoc| is a legacy version of |\childdocmain|:
%    \begin{macrocode}
\newcommand{\childdoc}{\childdocmain}
%    \end{macrocode}

% \macro{\childdocredirect}
% The deprecated macro |\childdocredirect| is a legacy version
% of |\childdocforward| and |\childdocforwardprefix|:
%    \begin{macrocode}
\newcommand{\childdocredirect}[2][]
{
  \begingroup
    \if?#1?
      \def\childdoctmp{\childdocforward{#2}}
    \else
      \def\childdoctmp{\childdocforwardprefix{#1}{#2}}
    \fi
    \expandafter
  \endgroup
  \childdoctmp
}
%    \end{macrocode}

%\iffalse
%</package>
%\fi
%
\endinput
|
and perform the replacements as outlined below.
Instead of |\childdocmain{|\textit{main}|}| add the following code
to the top of the main file:
%
\begin{center}
\begin{tabular}{l}
|\||ifdefined\childdocname\endinput\||fi\newif\ifchilddoc|\\
|\edef\childdocname{\scantokens\expandafter{\jobname\noexpand}}|\\
|\def\childdocmain{|\textit{main}|}\||ifx\childdocmain\childdocname\||else|\\
|\childdoctrue\includeonly{\childdocname}\let\jobname\childdocmain\||fi|\\
\end{tabular}
\end{center}
%
Instead of |\childdocof{|\textit{main}|}| just include the main file
at the top of each child file:
%
\begin{center}
|\input{|\textit{main}|}|
\end{center}
%
A simple redirection |\childdocforward{|\textit{dest}|}| is achieved by:
%
\begin{center}
|\def\jobname{|\textit{dest}|}\input{\jobname}|
\end{center}
%
The redirection with prefix
|\childdocforwardprefix[|\textit{prefix}|]{|\textit{dest}|}|
is accomplished by:
%
\begin{center}
\begin{tabular}{l}
|{\edef\jobname{\scantokens\expandafter{\jobname\noexpand}}|\\
|\def\redirectjob |\textit{prefix}|#1~~~{\gdef\jobname{|\textit{dest}|#1}}|\\
|\expandafter\redirectjob\jobname~~~}\input{\jobname}|
\end{tabular}
\end{center}

In an alternative approach,
child documents can be compiled by a specific command line
without additional code or specific definitions:
%
\begin{center}
|... -jobname "|\textit{target}|" "|[\textit{flags}]%
|\includeonly{|\textit{dest}|}\input{|\textit{main}|}"|
\end{center}
%

%%%%%%%%%%%%%%%%%%%%%%%%%%%%%%%%%%%%%%%%%%%%%%%%%%%%%%%%%%%%%%%%%%%%%%%%%%%%%%%%
%%%%%%%%%%%%%%%%%%%%%%%%%%%%%%%%%%%%%%%%%%%%%%%%%%%%%%%%%%%%%%%%%%%%%%%%%%%%%%%%
\section{Information}

%%%%%%%%%%%%%%%%%%%%%%%%%%%%%%%%%%%%%%%%%%%%%%%%%%%%%%%%%%%%%%%%%%%%%%%%%%%%%%%%
\subsection{Copyright}

Copyright \copyright{} 2017--2018 Niklas Beisert

This work may be distributed and/or modified under the
conditions of the \LaTeX{} Project Public License, either version 1.3
of this license or (at your option) any later version.
The latest version of this license is in
  \url{http://www.latex-project.org/lppl.txt}
and version 1.3 or later is part of all distributions of \LaTeX{}
version 2005/12/01 or later.

This work has the LPPL maintenance status `maintained'.

The Current Maintainer of this work is Niklas Beisert.

This work consists of the files |README.txt|, |childdoc.ins| and |childdoc.dtx|
as well as the derived files |childdoc.def|, |cdocsamp.tex|
with |cdocsch1.tex|, |cdocsch2.tex|, |cdocspt3.tex|, |cdocspt4.tex|,
|cdocsdrf.tex|, |cdocsfn1.tex|, |cdocsfn2.tex|
as well as |childdoc.pdf|.

%%%%%%%%%%%%%%%%%%%%%%%%%%%%%%%%%%%%%%%%%%%%%%%%%%%%%%%%%%%%%%%%%%%%%%%%%%%%%%%%
\subsection{Files and Installation}

The package consists of the files:
%
\begin{center}
\begin{tabular}{ll}
    |README.txt|   & readme file \\
    |childdoc.ins| & installation file \\
    |childdoc.dtx| & source file \\
    |childdoc.def| & definition file \\
    |cdocsamp.tex| & sample main file \\
    |cdocsch1.tex| & sample include file \\
    |cdocsch2.tex| & sample include file \\
    |cdocspt3.tex| & sample part file \\
    |cdocspt4.tex| & sample part file \\
    |cdocsdrf.tex| & sample redirection file \\
    |cdocsfn1.tex| & sample redirection file \\
    |cdocsfn2.tex| & sample redirection file \\
    |childdoc.pdf| & manual
\end{tabular}
\end{center}
%
The distribution consists of the files
|README.txt|, |childdoc.ins| and |childdoc.dtx|.
%
\begin{itemize}
\item
Run (pdf)\LaTeX{} on |childdoc.dtx|
to compile the manual |childdoc.pdf| (this file).
\item
Run \LaTeX{} on |childdoc.ins| to create the definitions file |childdoc.def|
and the sample |cdocsamp.tex| with include files
|cdocsch1.tex|, |cdocsch2.tex|, |cdocspt3.tex|, |cdocspt4.tex|,
|cdocsdrf.tex|, |cdocsfn1.tex|, |cdocsfn2.tex|.
Then copy the file |childdoc.def| to an appropriate directory of your \LaTeX{}
distribution, e.g.\ \textit{texmf-root}|/tex/latex/childdoc|.
\end{itemize}

%%%%%%%%%%%%%%%%%%%%%%%%%%%%%%%%%%%%%%%%%%%%%%%%%%%%%%%%%%%%%%%%%%%%%%%%%%%%%%%%
\subsection{Related CTAN Packages}

There are several other packages which offer a similar functionality:
%
\begin{itemize}
\item
The packages
\href{http://ctan.org/pkg/docmute}{\textsf{docmute}},
\href{http://ctan.org/pkg/includex}{\textsf{includex}} and
\href{http://ctan.org/pkg/standalone}{\textsf{standalone}}
provide commands to include only the document body of
a child file thus allowing both files to be compiled individually.
\item
The packages \href{http://ctan.org/pkg/subdocs}{\textsf{subdocs}}
and \href{http://ctan.org/pkg/subfiles}{\textsf{subfiles}}
provide structures in which the main and child documents can be
encapsulated and allowing them to be compiled individually.
The inclusion mechanism is different from the conventional |\include|.
\item
The package \href{http://ctan.org/pkg/combine}{\textsf{combine}}
is an elaborate solution to combine several documents into one.
\end{itemize}
%
See also the CTAN topic \href{http://ctan.org/topic/subdocs}{\textsf{subdocs}}
for further related packages.
The present package differs from the above solutions in that
a document structure constructed with the conventional |\include| mechanism
just needs two extra commands at the top of every file
such that all constituent files can be compiled individually.

%%%%%%%%%%%%%%%%%%%%%%%%%%%%%%%%%%%%%%%%%%%%%%%%%%%%%%%%%%%%%%%%%%%%%%%%%%%%%%%%
%\subsection{Feature Suggestions}
%
%The following is a list of features which may be useful for future
%versions of this package:
%%
%\begin{itemize}
%\item
%\ldots
%\end{itemize}

%%%%%%%%%%%%%%%%%%%%%%%%%%%%%%%%%%%%%%%%%%%%%%%%%%%%%%%%%%%%%%%%%%%%%%%%%%%%%%%%
\subsection{Revision History}

%%%%%%%%%%%%%%%%%%%%%%%%%%%%%%%%%%%%%%%%
\paragraph{v2.0:} 2018/12/30

\begin{itemize}
\item
immediate forward processing
\item
added |\childdocby| mechanism
\item
manual restructured
\end{itemize}

%%%%%%%%%%%%%%%%%%%%%%%%%%%%%%%%%%%%%%%%
\paragraph{v1.6:} 2018/01/17

\begin{itemize}
\item
application for development of include files
\item
corrections to manual
\end{itemize}

%%%%%%%%%%%%%%%%%%%%%%%%%%%%%%%%%%%%%%%%
\paragraph{v1.5:} 2017/05/21

\begin{itemize}
\item
more complete structuring introduced
\item
|\childdocof| introduced
\item
|\childdoc| renamed to |\childdocmain|
\item
|\childredirect| renamed to |\childdocforward| and |\childdocforwardprefix|
and functionality expanded
\end{itemize}

%%%%%%%%%%%%%%%%%%%%%%%%%%%%%%%%%%%%%%%%
\paragraph{v1.0:} 2017/04/27

\begin{itemize}
\item
manual and install package
\item
first version published on CTAN
\end{itemize}

%%%%%%%%%%%%%%%%%%%%%%%%%%%%%%%%%%%%%%%%
\paragraph{v0.6:} 2017/04/26

\begin{itemize}
\item
redirection mechanism added
\end{itemize}

%%%%%%%%%%%%%%%%%%%%%%%%%%%%%%%%%%%%%%%%
\paragraph{v0.5:} 2017/04/26

\begin{itemize}
\item
functionality in definition file
\end{itemize}


%%%%%%%%%%%%%%%%%%%%%%%%%%%%%%%%%%%%%%%%%%%%%%%%%%%%%%%%%%%%%%%%%%%%%%%%%%%%%%%%
%%%%%%%%%%%%%%%%%%%%%%%%%%%%%%%%%%%%%%%%%%%%%%%%%%%%%%%%%%%%%%%%%%%%%%%%%%%%%%%%
%%%%%%%%%%%%%%%%%%%%%%%%%%%%%%%%%%%%%%%%%%%%%%%%%%%%%%%%%%%%%%%%%%%%%%%%%%%%%%%%
\appendix

\settowidth\MacroIndent{\rmfamily\scriptsize 000\ }

 \DocInput{childdoc.dtx}

\end{document}
%</driver>
% \fi
%
% %%%%%%%%%%%%%%%%%%%%%%%%%%%%%%%%%%%%%%%%%%%%%%%%%%%%%%%%%%%%%%%%%%%%%%%%%%%%%%
% %%%%%%%%%%%%%%%%%%%%%%%%%%%%%%%%%%%%%%%%%%%%%%%%%%%%%%%%%%%%%%%%%%%%%%%%%%%%%%
% \section{Sample}
%\iffalse
%<*samplemain>
%\fi
%
% The following presents a sample document
% with two chapters, two parts, a title page,
% a compile flag as well as three forwarding files to set the flag.
% It consists of eight |.tex| files:
% \begin{center}
% \begin{tabular}{ll}
% |cdocsamp.tex|&main file\\
% |cdocsch1.tex|&include file for chapter 1\\
% |cdocsch2.tex|&include file for chapter 2\\
% |cdocspt3.tex|&include file for part 3\\
% |cdocspt4.tex|&include file for part 4\\
% |cdocsdrf.tex|&forwarding file for main file in draft mode\\
% |cdocsfi1.tex|&forwarding file for final version of chapter 1\\
% |cdocsfi2.tex|&forwarding file for final version of chapter 2\\
% \end{tabular}
% \end{center}
% Each of the eight files can be compiled directly by the \LaTeX{} compiler.
%
% %%%%%%%%%%%%%%%%%%%%%%%%%%%%%%%%%%%%%%
% \paragraph{Main File.}
%
% The main file is called |cdocsamp.tex|.
%
% Load the \textsf{childdoc} definitions and
% declare the filename for the main document:
%    \begin{macrocode}
% \iffalse
%
% childdoc.dtx Copyright (C) 2017-2018 Niklas Beisert
%
% This work may be distributed and/or modified under the
% conditions of the LaTeX Project Public License, either version 1.3
% of this license or (at your option) any later version.
% The latest version of this license is in
%   http://www.latex-project.org/lppl.txt
% and version 1.3 or later is part of all distributions of LaTeX
% version 2005/12/01 or later.
%
% This work has the LPPL maintenance status `maintained'.
%
% The Current Maintainer of this work is Niklas Beisert.
%
% This work consists of the files childdoc.dtx and childdoc.ins
% and the derived files childdoc.def and cdocsamp.tex with
% cdocsch1.tex, cdocsch2.tex, cdocsdrf.tex, cdocsfn1.tex, cdocsfn2.tex.
%
%<package>\ifdefined\childdocmain\endinput\fi
%<package>\ProvidesFile{childdoc.def}[2018/12/30 v2.0 child document driver]
%<samplemain>\ProvidesFile{cdocsamp.tex}[2018/12/30 v2.0 sample for childdoc]
%<*driver>
%\ProvidesFile{childdoc.drv}[2018/12/30 v2.0 childdoc reference manual file]
\PassOptionsToClass{10pt,a4paper}{article}
\documentclass{ltxdoc}

\usepackage[margin=35mm]{geometry}
\usepackage{hyperref}
\usepackage{hyperxmp}
\usepackage[usenames]{color}

\hypersetup{colorlinks=true}
\hypersetup{pdfstartview=FitH}
\hypersetup{pdfpagemode=UseNone}
\hypersetup{pdfsource={}}
\hypersetup{pdflang={en-UK}}
\hypersetup{pdfcopyright={Copyright 2017-2018 Niklas Beisert.
  This work may be distributed and/or modified under the
  conditions of the LaTeX Project Public License, either version 1.3
  of this license or (at your option) any later version.}}
\hypersetup{pdflicenseurl={http://www.latex-project.org/lppl.txt}}
\hypersetup{pdfcontactaddress={ETH Zurich, ITP, HIT K,
  Wolfgang-Pauli-Strasse 27}}
\hypersetup{pdfcontactpostcode={8093}}
\hypersetup{pdfcontactcity={Zurich}}
\hypersetup{pdfcontactcountry={Switzerland}}
\hypersetup{pdfcontactemail={nbeisert@itp.phys.ethz.ch}}
\hypersetup{pdfcontacturl={http://people.phys.ethz.ch/\xmptilde nbeisert/}}

\newcommand{\secref}[1]{\hyperref[#1]{section \ref*{#1}}}

\parskip1ex
\parindent0pt
\let\olditemize\itemize
\def\itemize{\olditemize\parskip0pt}

\begin{document}

\title{The \textsf{childdoc} Package}
\hypersetup{pdftitle={The childdoc Package}}
\author{Niklas Beisert\\[2ex]
  Institut f\"ur Theoretische Physik\\
  Eidgen\"ossische Technische Hochschule Z\"urich\\
  Wolfgang-Pauli-Strasse 27, 8093 Z\"urich, Switzerland\\[1ex]
  \href{mailto:nbeisert@itp.phys.ethz.ch}
  {\texttt{nbeisert@itp.phys.ethz.ch}}}
\hypersetup{pdfauthor={Niklas Beisert}}
\hypersetup{pdfsubject={Manual for the LaTeX2e Package childdoc}}
\date{30 December 2018, \textsf{v2.0}}
\maketitle

\begin{abstract}\noindent
\textsf{childdoc} is a \LaTeXe{} package
that enables the direct compilation
of document sections included by |\include|
to individual files.
\end{abstract}

\begingroup
\parskip0ex
\tableofcontents
\endgroup

%%%%%%%%%%%%%%%%%%%%%%%%%%%%%%%%%%%%%%%%%%%%%%%%%%%%%%%%%%%%%%%%%%%%%%%%%%%%%%%%
%%%%%%%%%%%%%%%%%%%%%%%%%%%%%%%%%%%%%%%%%%%%%%%%%%%%%%%%%%%%%%%%%%%%%%%%%%%%%%%%
\section{Introduction}

\LaTeX{} provides a mechanism to structure a large document (such as a book)
into a main file and several child files (containing the chapters)
using the |\include| command.
This mechanism is beneficial for documents
which span hundreds of pages in order to
make the source file(s) more manageable.
Moreover, compilation can be restricted to
selected child files by means of the |\includeonly| command.
The latter feature can be used to reduce the compilation time while editing
(this was significantly more useful in the earlier days of \LaTeX{})
or to generate a smaller document which is easier to navigate.
Another application of |\includeonly| is to generate
documents consisting of selected parts of the complete document.

However, there are a few drawbacks of the plain |\include| mechanism:
\begin{itemize}
\item
The child files cannot be compiled on their own,
they can only be compiled via the main file.
A naive editing environment
(such as a text editor with an option
to have the current file processed by \LaTeX)
may require one to switch to the main file before compiling;
attempting to compile the child file produces errors.
\item
The main file must be modified (each time)
to adjust the |\includeonly| command
to the present needs. This easily leaves the main file in a messy state.
\item
The generated document will always carry the filename
of the main document. This is inconvenient if
several child files are to be compiled and
to be kept for distribution.
\end{itemize}

The present package provides a simple interface
to make child files individually compilable by \LaTeX{}.
Compiling a child file then has the same effect as compiling
the main file with an |\includeonly| command
to select the appropriate child.
Moreover the generated document will carry the name of the child
rather than the main file.
This resolves all three above issues.

This feature is meant to make the editing of books,
thesis documents and lecture notes somewhat more convenient.
However, the package can also be used efficiently for
composing a series of documents (such as exercise sheets)
which are typically distributed individually.
It then assists the author in generating the individual documents
(potentially in different versions)
as well as a document containing the collected series.
Another application is in developing style files
or other kinds of included material
where compilation of the style file could redirect
to a sample or test file.

%%%%%%%%%%%%%%%%%%%%%%%%%%%%%%%%%%%%%%%%%%%%%%%%%%%%%%%%%%%%%%%%%%%%%%%%%%%%%%%%
%%%%%%%%%%%%%%%%%%%%%%%%%%%%%%%%%%%%%%%%%%%%%%%%%%%%%%%%%%%%%%%%%%%%%%%%%%%%%%%%
\section{Usage}

First of all, the package \textsf{childdoc} is \emph{not} a standard
\LaTeXe{} |.sty| style file! Therefore it needs to be invoked in
a non-standard way.

%%%%%%%%%%%%%%%%%%%%%%%%%%%%%%%%%%%%%%%%%%%%%%%%%%%%%%%%%%%%%%%%%%%%%%%%%%%%%%%%
\subsection{Included Files}
\label{sec:include}

%%%%%%%%%%%%%%%%%%%%%%%%%%%%%%%%%%%%%%%%
\DescribeMacro{\childdocmain}
To use the package, add the commands
\begin{center}
\begin{tabular}{l}
|\input{childdoc.def}|\\
|\childdocmain{}|\\
\end{tabular}
\end{center}
at the very top of the main \LaTeX{} file,
in particular \emph{before} the |\documentclass| statement!
The argument of |\childdocmain| should be left empty
(but it must be present).

%%%%%%%%%%%%%%%%%%%%%%%%%%%%%%%%%%%%%%%%
\DescribeMacro{\childdocof}
Furthermore, add the commands
\begin{center}
\begin{tabular}{l}
|\input{childdoc.def}|\\
|\childdocof{|\textit{main}|}|\\
\end{tabular}
\end{center}
at the top of every child file \textit{child}
which is included by |\include{|\textit{child}|}|
from within the main file
(or at least for those files to be compiled individually).
The argument \textit{main} must be the filename of the main file.

There are a couple of
considerations in setting up the main and child documents:

%%%%%%%%%%%%%%%%%%%%%%%%%%%%%%%%%%%%%%%%
\paragraph{Restrictions.}

Please note the following restrictions:
\begin{itemize}
\item
|\childdocmain| must be called with one argument \textit{main}
to ensure compatibility with earlier version of the package.
It must either be empty (|\childdocmain{}|)
or precisely match the filename of the main file in which it is specified.
See \secref{sec:detection} for further information.
\item
The filename \textit{main} must be specified without the |.tex| extension.
\item
The filename \textit{main} is case sensitive
(even in case-insensitive file systems)
due to internal string comparison.
\item
The argument \textit{main} should be fully expanded, it cannot be a macro.
\item
Subdirectories and special characters should be avoided in filenames.
\item
The command |\childdocmain{|\textit{main}|}| must be followed by a whitespace.
It should not be followed immediately by another command
or by a comment mark `|%|'.
This is because the \TeX{} parser reads the token immediately following
the argument of |\childdocmain| and puts it
at the beginning of every child section;
however, a white\-space is ignored.
\end{itemize}

%%%%%%%%%%%%%%%%%%%%%%%%%%%%%%%%%%%%%%%%
\paragraph{Content of Main File.}

It is advisable to place all content in the child files included by |\include|.
Any output contained in the main file will appear in all child documents
unless suppressed manually;
it cannot be suppressed automatically by the |\includeonly| directive
and thus should normally be avoided.
A method to include some content in the main file
by means of conditional processing is described in \secref{sec:conditional}.

%%%%%%%%%%%%%%%%%%%%%%%%%%%%%%%%%%%%%%%%
\paragraph{Page Numbering.}

When only a part of the document is compiled,
the appropriate numbering of pages
(as well as other status parameters)
is determined from the |.aux| files.
The latter contain information from previous passes.
However this information needs to propagate through
all intermediate child documents.
Therefore the page numbering in child documents may well
be inconsistent until the complete document is compiled at least once.

A useful (if unconventional) way to always ensure a consistent
page numbering is to restart the numbering in each child document
and denote the pages by `\textit{child}|.|\textit{page}'
where \textit{child} represents the chapter/section number of the child file.
This can be achieved by the command
|\numberwithin{page}{|\textit{child}|}|
of the \textsf{amsmath} package
where \textit{child} can be |chapter| or |section|
depending on the chosen structuring.
Alternatively, one can modify the macro |\thepage| appropriately
and reset the counter |page| at the start of each child file.

%%%%%%%%%%%%%%%%%%%%%%%%%%%%%%%%%%%%%%%%%%%%%%%%%%%%%%%%%%%%%%%%%%%%%%%%%%%%%%%%
\subsection{Conditional Processing}
\label{sec:conditional}

The package provides a mechanism to compile different versions
of a document. To customise the versions further some conditional processing
can come in handy to distinguish which version is being compiled.
The package provides two macros to describe the compilation context:

%%%%%%%%%%%%%%%%%%%%%%%%%%%%%%%%%%%%%%%%
\DescribeMacro{\ifchilddoc}
The conditional |\ifchilddoc| distinguishes between the compilation of
child documents and the main document:
%
\begin{center}
|\ifchilddoc |\textit{child-code}| |[|\||else |\textit{main-code}]| \||fi|
\end{center}

%%%%%%%%%%%%%%%%%%%%%%%%%%%%%%%%%%%%%%%%
\DescribeMacro{\childdocname}
\DescribeMacro{\childdocjob}
The macro |\childdocname| contains the filename (without extension)
of the main or child file being processed.
Note that |\childdocjob| will always contain the name of the main file.

%%%%%%%%%%%%%%%%%%%%%%%%%%%%%%%%%%%%%%%%
\paragraph{Title Page.}

Conditional processing can be used to include a title or banner page
in the main document when proper precautions are taken.
Importantly, the code in the main file should ensure that the page counter
(as well as other status parameters which are stored in the |.aux| files)
takes the same value after the conditional processing.
Otherwise the page numbers may take divergent values
depending on which part is compiled.

For example, a title page could be declared by:
%
\begin{center}
\begin{tabular}{l}
|\ifchilddoc\||else|\\
|\addtocounter{page}{-1}|\\
\textit{code for title page}\\
|\newpage|\\
|\||fi|
\end{tabular}
\end{center}
%
A banner page for the child documents can be generated by:
%
\begin{center}
\begin{tabular}{l}
|\ifchilddoc|\\
|\addtocounter{page}{-1}|\\
\textit{code for banner page}\\
|\newpage|\\
|\||fi|
\end{tabular}
\end{center}
%
Here one could write a message such as:
\begin{center}
|This is the part \childdocname{} of \childdocjob{}.|
\end{center}

%%%%%%%%%%%%%%%%%%%%%%%%%%%%%%%%%%%%%%%%%%%%%%%%%%%%%%%%%%%%%%%%%%%%%%%%%%%%%%%%
\subsection{Flags}
\label{sec:flags}

The package makes it easy to generate different versions
of the main or child documents.
To this end compilation flags can be defined
and assigned different default values.
They will be particularly useful in conjunction
with the forwarding mechanism described in \secref{sec:forward}.

For example, it may be useful to have a flag |\version|
which can be set to |draft| or |final|.
The document source will contain some conditional code
depending on the value of |\version|.
Suppose further, the flag should default to |final| for the main file
and to |draft| for child files
which is a natural assignment for editing the document.
This is achieved by placing the following code
in the preamble of the main document
(below the |\childdocmain| directive):
%
\begin{center}
\begin{tabular}{l}
|\ifchilddoc|\\
|\providecommand{\version}{draft}|\\
|\||else|\\
|\providecommand{\version}{final}|\\
|\||fi|
\end{tabular}
\end{center}
%
The definition by |\providecommand| makes sure
that previous definitions are not overwritten.
Further statements |\providecommand{\version}{...}|
can thus be added before the above code to override it.

For the main file, one might add a line
(between |\childdocmain| and the above block)
%
\begin{center}
|%\ifchilddoc\||else\providecommand{\version}{draft}\||fi|
\end{center}
%
which can be uncommented to produce a draft version.
Likewise one can add a line to the very top of a child file
(above the |\childdocof{|\textit{main}|}| directive)
%
\begin{center}
|%\providecommand{\version}{final}|
\end{center}
%
which can be uncommented to produce the final version of this child document.

%%%%%%%%%%%%%%%%%%%%%%%%%%%%%%%%%%%%%%%%%%%%%%%%%%%%%%%%%%%%%%%%%%%%%%%%%%%%%%%%
\subsection{Forwarding}
\label{sec:forward}

Different versions of the main or child documents
using compilation flags as described in \secref{sec:flags}
can be (permanently) stored in different files
for convenient compilation, viewing and distribution.
To this end, the package defines a command
to pass on compilation to a different file:

%%%%%%%%%%%%%%%%%%%%%%%%%%%%%%%%%%%%%%%%
\DescribeMacro{\childdocforward}
The command |\childdocforward| redirects processing to
another source file:
%
\begin{center}
\begin{tabular}{l}
|\input{childdoc.def}|\\
|\childdocforward[|\textit{main}|]{|\textit{dest}|}|\\
\end{tabular}
\end{center}
%
The argument \textit{dest} is the destination file
(without extension).
It should be the main file or one of the child files.
Note that further \textsf{childdoc} directives
such as |\childdocof| and |\childdocforward|
in the indicated file will be processed in this form.
The optional argument \textit{main}
passes on directly to the main file \textit{main}
while pretending to compile the child \textit{dest}.
This form behaves as if \textit{dest}
issues |\childdocof{|\textit{main}|}| right away,
and no further \textsf{childdoc} directives will be processed.

%%%%%%%%%%%%%%%%%%%%%%%%%%%%%%%%%%%%%%%%
\DescribeMacro{\...prefix}
In the alternative form |\childdocforwardprefix|,
%
\begin{center}
\begin{tabular}{l}
|\input{childdoc.def}|\\
|\childdocforwardprefix[|\textit{main}|]{|\textit{prefix}|}{|\textit{dest}|}|
\end{tabular}
\end{center}
%
the destination file is determined by a pattern
depending on the current file:
To make this work, the current file must be called
`{\textit{prefix}\hspace{0.2em}\textit{suffix}}'
with \textit{prefix} matching precisely the argument.
Processing is then passed on to the file
`{\textit{dest}\hspace{0.2em}\textit{suffix}}'.
Surely, the same effect is achieved by
directly specifying the
argument `{\textit{dest}\hspace{0.2em}\textit{suffix}}'
in the first form.
However, that requires to set up a different file
for each child. With the alternative form of the command
all these files can have exactly the same content
which simplifies setting them up and maintaining them.

For example, the following file |draft.tex|
with a compilation flag |\version| as described in \secref{sec:flags}
compiles the main document as a draft:
%
\begin{center}
\begin{tabular}{l}
|\def\version{draft}|\\
|\input{childdoc.def}|\\
|\childdocforward{|\textit{main}|}|
\end{tabular}
\end{center}
%
Likewise, the following files |final|\textit{nn}|.tex|
compile the final version of the child document
|child|\textit{nn}|.tex|:
%
\begin{center}
\begin{tabular}{l}
|\def\version{final}|\\
|\input{childdoc.def}|\\
|\childdocforwardprefix{final}{child}|
\end{tabular}
\end{center}
%

Note that when several versions of a main file and/or of each child file
are to be generated, it may be convenient to set up a |Makefile| or
shell script to automatise the process.

%%%%%%%%%%%%%%%%%%%%%%%%%%%%%%%%%%%%%%%%%%%%%%%%%%%%%%%%%%%%%%%%%%%%%%%%%%%%%%%%
\subsection{Command Line Processing}
\label{sec:commandline}

The effect of redirection files can also be achieved by invoking
the \LaTeX{} compiler with a more elaborate command line.
Most conveniently this should be done as part
of a shell script or a |Makefile|.

When using \textsf{childdoc} in the main file, the following
command lines effectively perform a redirection
(note that depending on the shell being used,
backslashes may have to be doubled: `|\|' $\to$ `|\\|'):
%
\begin{center}
|... -jobname "|\textit{target}|" |\\|"|[\textit{flags}]%
|\input{childdoc.def}\childdocforward[|\textit{main}|]{|\textit{dest}|}"|
\end{center}
%
Here \textit{target} is the name of the output file,
\textit{main} is the name of the main file
and \textit{dest} is the name of the main or child file to be processed
(all filenames without extensions).
The optional argument \textit{main} can be omitted
if \textit{main} matches \textit{dest}.
Optionally, compilation \textit{flags} can be defined via |\def| commands.
This command line makes the \TeX{} engine believe
it is compiling the file \textit{target}
whose content is specified as the latter parameter.
The provided code then forwards the processing to
\textit{main} or \textit{dest} as described in \secref{sec:forward}.

%%%%%%%%%%%%%%%%%%%%%%%%%%%%%%%%%%%%%%%%%%%%%%%%%%%%%%%%%%%%%%%%%%%%%%%%%%%%%%%%
\subsection{Include by Input}
\label{sec:input}

Including child documents by |\include| has some restrictions by design.
Most notably, the content of a child document always occupies
its own set of pages; pages cannot be shared between child documents.
Usually, this behaviour makes perfect sense
because each child document contain an essential part of the document.
However, in some situations it may be desirable to compose
a document from a collection of parts
without having mandatory page breaks between then.
For this case, the package
provides a mechanism to include parts
by |\input| which can also be processed individually.
However, by construction this mechanism
requires manual handling of the content to be output.

%%%%%%%%%%%%%%%%%%%%%%%%%%%%%%%%%%%%%%%%
\DescribeMacro{\ifchilddocmanual}
The main file should be prepared as usual, see \secref{sec:include}.
However, the document body must make a distinction
between processing of an individual part and of the main document, e.g.:
%
\begin{center}
\begin{tabular}{l}
|\ifchilddocmanual|\\
|\input{\childdocname}|\\
|\||else|\\
\textit{document body with }|\input{|\textit{part}|}|\\
|\||fi|
\end{tabular}
\end{center}
%
The conditional |\ifchilddocmanual| is true whenever
a part to be included by |\input| is being compiled,
and the name of the part is stored in |\childdocname|.

%%%%%%%%%%%%%%%%%%%%%%%%%%%%%%%%%%%%%%%%
\DescribeMacro{\childdocby}
Each part to be included by |\input| should start with:
%
\begin{center}
\begin{tabular}{l}
|\input{childdoc.def}|\\
|\childdocby{|\textit{main}|}|\\
\end{tabular}
\end{center}
%
The directive |\childdocby| is similar to |\childdocof|
described in \secref{sec:include},
but the subsequent selection of content must be done manually.
To that end, both |\ifchilddoc| and |\ifchilddocmanual|
will be true upon processing of a part,
and the name of the part is stored in |\childdocname|.
Note that |\jobname| will be set to the filename of the current part
so that each part receives an individual |.aux| file
that does not interfere with the |.aux| file(s) of the main document.
This behaviour can be altered by the alternative form
|\childdocby[*]{|\textit{main}|}| (with a non-empty optional argument)
which uses the |.aux| file of the main document
by setting |\jobname| to \textit{main}.

%%%%%%%%%%%%%%%%%%%%%%%%%%%%%%%%%%%%%%%%%%%%%%%%%%%%%%%%%%%%%%%%%%%%%%%%%%%%%%%%
\subsection{Driver Development}
\label{sec:driver}

The \textsf{childdoc} mechanism can also be use for the development
of definition files such as \LaTeX{} styles or classes.
This case differs from the above setup with multiple parts
included by |\include| in that no |\includeonly| should be invoked.
This can be achieved by starting the include file
(before |\ProvidesPackage|) with:
%
\begin{center}
\begin{tabular}{l}
|\input{childdoc.def}|\\
|\childdocforward{|\textit{main}|}|\\
\end{tabular}
\end{center}
%
or alternatively with:
%
\begin{center}
\begin{tabular}{l}
|\input{childdoc.def}|\\
|\childdocby{|\textit{main}|}|\\
\end{tabular}
\end{center}
%
Both forms have slightly different effects as described above.
The main file is prepared as usual, see \secref{sec:include}.

%%%%%%%%%%%%%%%%%%%%%%%%%%%%%%%%%%%%%%%%%%%%%%%%%%%%%%%%%%%%%%%%%%%%%%%%%%%%%%%%
\subsection{Legacy Detection}
\label{sec:detection}

The directive |\childdocmain| in the main file can detect
whether the complete document or merely a child is to be compiled
even without using the directive |\childdocof|.
This method is deprecated because it is less robust
and there is no compelling reason to use it;
it is merely provided for backward compatibility
and it may be removed in future versions.

If the detection mechanism is to be used,
it is mandatory to correctly specify
the filename of the main file as the argument of |\childdocmain|:
%
\begin{center}
\begin{tabular}{l}
|\input{childdoc.def}|\\
|\childdocmain{|\textit{main}|}|\\
\end{tabular}
\end{center}
%
If |\jobname| does not match the argument \textit{main} of |\childdocmain|,
it is assumed that |\jobname| points to the child file to be compiled.
When using |\childdocmain| with the main file specified as argument,
it suffices to start a child file
with just |\input{|\textit{main}|}|
without loading of the package and using |\childdocof|.
If instead all processing is done
with the appropriate \textsf{childdoc} directives,
the argument of \textit{main} of |\childdocmain| can be empty.

An alternative version of the command line processing described
in \secref{sec:commandline} using the detection mechanism reads:
%
\begin{center}
|... -jobname "|\textit{target}|" "|[\textit{flags}]%
[|\def\jobname{|\textit{dest}|}|]|\input{|\textit{main}|}"|
\end{center}

%%%%%%%%%%%%%%%%%%%%%%%%%%%%%%%%%%%%%%%%%%%%%%%%%%%%%%%%%%%%%%%%%%%%%%%%%%%%%%%%
\subsection{Manual Code}
\label{sec:manual}

In case one cannot be certain whether the definitions file |childdoc.def|
is installed on the target \TeX{} distribution
and one prefers not to ship it,
it is conceivable to paste a few relevant commands into the sources.

To that end, drop all statements |\input{childdoc.def}|
and perform the replacements as outlined below.
Instead of |\childdocmain{|\textit{main}|}| add the following code
to the top of the main file:
%
\begin{center}
\begin{tabular}{l}
|\||ifdefined\childdocname\endinput\||fi\newif\ifchilddoc|\\
|\edef\childdocname{\scantokens\expandafter{\jobname\noexpand}}|\\
|\def\childdocmain{|\textit{main}|}\||ifx\childdocmain\childdocname\||else|\\
|\childdoctrue\includeonly{\childdocname}\let\jobname\childdocmain\||fi|\\
\end{tabular}
\end{center}
%
Instead of |\childdocof{|\textit{main}|}| just include the main file
at the top of each child file:
%
\begin{center}
|\input{|\textit{main}|}|
\end{center}
%
A simple redirection |\childdocforward{|\textit{dest}|}| is achieved by:
%
\begin{center}
|\def\jobname{|\textit{dest}|}\input{\jobname}|
\end{center}
%
The redirection with prefix
|\childdocforwardprefix[|\textit{prefix}|]{|\textit{dest}|}|
is accomplished by:
%
\begin{center}
\begin{tabular}{l}
|{\edef\jobname{\scantokens\expandafter{\jobname\noexpand}}|\\
|\def\redirectjob |\textit{prefix}|#1~~~{\gdef\jobname{|\textit{dest}|#1}}|\\
|\expandafter\redirectjob\jobname~~~}\input{\jobname}|
\end{tabular}
\end{center}

In an alternative approach,
child documents can be compiled by a specific command line
without additional code or specific definitions:
%
\begin{center}
|... -jobname "|\textit{target}|" "|[\textit{flags}]%
|\includeonly{|\textit{dest}|}\input{|\textit{main}|}"|
\end{center}
%

%%%%%%%%%%%%%%%%%%%%%%%%%%%%%%%%%%%%%%%%%%%%%%%%%%%%%%%%%%%%%%%%%%%%%%%%%%%%%%%%
%%%%%%%%%%%%%%%%%%%%%%%%%%%%%%%%%%%%%%%%%%%%%%%%%%%%%%%%%%%%%%%%%%%%%%%%%%%%%%%%
\section{Information}

%%%%%%%%%%%%%%%%%%%%%%%%%%%%%%%%%%%%%%%%%%%%%%%%%%%%%%%%%%%%%%%%%%%%%%%%%%%%%%%%
\subsection{Copyright}

Copyright \copyright{} 2017--2018 Niklas Beisert

This work may be distributed and/or modified under the
conditions of the \LaTeX{} Project Public License, either version 1.3
of this license or (at your option) any later version.
The latest version of this license is in
  \url{http://www.latex-project.org/lppl.txt}
and version 1.3 or later is part of all distributions of \LaTeX{}
version 2005/12/01 or later.

This work has the LPPL maintenance status `maintained'.

The Current Maintainer of this work is Niklas Beisert.

This work consists of the files |README.txt|, |childdoc.ins| and |childdoc.dtx|
as well as the derived files |childdoc.def|, |cdocsamp.tex|
with |cdocsch1.tex|, |cdocsch2.tex|, |cdocspt3.tex|, |cdocspt4.tex|,
|cdocsdrf.tex|, |cdocsfn1.tex|, |cdocsfn2.tex|
as well as |childdoc.pdf|.

%%%%%%%%%%%%%%%%%%%%%%%%%%%%%%%%%%%%%%%%%%%%%%%%%%%%%%%%%%%%%%%%%%%%%%%%%%%%%%%%
\subsection{Files and Installation}

The package consists of the files:
%
\begin{center}
\begin{tabular}{ll}
    |README.txt|   & readme file \\
    |childdoc.ins| & installation file \\
    |childdoc.dtx| & source file \\
    |childdoc.def| & definition file \\
    |cdocsamp.tex| & sample main file \\
    |cdocsch1.tex| & sample include file \\
    |cdocsch2.tex| & sample include file \\
    |cdocspt3.tex| & sample part file \\
    |cdocspt4.tex| & sample part file \\
    |cdocsdrf.tex| & sample redirection file \\
    |cdocsfn1.tex| & sample redirection file \\
    |cdocsfn2.tex| & sample redirection file \\
    |childdoc.pdf| & manual
\end{tabular}
\end{center}
%
The distribution consists of the files
|README.txt|, |childdoc.ins| and |childdoc.dtx|.
%
\begin{itemize}
\item
Run (pdf)\LaTeX{} on |childdoc.dtx|
to compile the manual |childdoc.pdf| (this file).
\item
Run \LaTeX{} on |childdoc.ins| to create the definitions file |childdoc.def|
and the sample |cdocsamp.tex| with include files
|cdocsch1.tex|, |cdocsch2.tex|, |cdocspt3.tex|, |cdocspt4.tex|,
|cdocsdrf.tex|, |cdocsfn1.tex|, |cdocsfn2.tex|.
Then copy the file |childdoc.def| to an appropriate directory of your \LaTeX{}
distribution, e.g.\ \textit{texmf-root}|/tex/latex/childdoc|.
\end{itemize}

%%%%%%%%%%%%%%%%%%%%%%%%%%%%%%%%%%%%%%%%%%%%%%%%%%%%%%%%%%%%%%%%%%%%%%%%%%%%%%%%
\subsection{Related CTAN Packages}

There are several other packages which offer a similar functionality:
%
\begin{itemize}
\item
The packages
\href{http://ctan.org/pkg/docmute}{\textsf{docmute}},
\href{http://ctan.org/pkg/includex}{\textsf{includex}} and
\href{http://ctan.org/pkg/standalone}{\textsf{standalone}}
provide commands to include only the document body of
a child file thus allowing both files to be compiled individually.
\item
The packages \href{http://ctan.org/pkg/subdocs}{\textsf{subdocs}}
and \href{http://ctan.org/pkg/subfiles}{\textsf{subfiles}}
provide structures in which the main and child documents can be
encapsulated and allowing them to be compiled individually.
The inclusion mechanism is different from the conventional |\include|.
\item
The package \href{http://ctan.org/pkg/combine}{\textsf{combine}}
is an elaborate solution to combine several documents into one.
\end{itemize}
%
See also the CTAN topic \href{http://ctan.org/topic/subdocs}{\textsf{subdocs}}
for further related packages.
The present package differs from the above solutions in that
a document structure constructed with the conventional |\include| mechanism
just needs two extra commands at the top of every file
such that all constituent files can be compiled individually.

%%%%%%%%%%%%%%%%%%%%%%%%%%%%%%%%%%%%%%%%%%%%%%%%%%%%%%%%%%%%%%%%%%%%%%%%%%%%%%%%
%\subsection{Feature Suggestions}
%
%The following is a list of features which may be useful for future
%versions of this package:
%%
%\begin{itemize}
%\item
%\ldots
%\end{itemize}

%%%%%%%%%%%%%%%%%%%%%%%%%%%%%%%%%%%%%%%%%%%%%%%%%%%%%%%%%%%%%%%%%%%%%%%%%%%%%%%%
\subsection{Revision History}

%%%%%%%%%%%%%%%%%%%%%%%%%%%%%%%%%%%%%%%%
\paragraph{v2.0:} 2018/12/30

\begin{itemize}
\item
immediate forward processing
\item
added |\childdocby| mechanism
\item
manual restructured
\end{itemize}

%%%%%%%%%%%%%%%%%%%%%%%%%%%%%%%%%%%%%%%%
\paragraph{v1.6:} 2018/01/17

\begin{itemize}
\item
application for development of include files
\item
corrections to manual
\end{itemize}

%%%%%%%%%%%%%%%%%%%%%%%%%%%%%%%%%%%%%%%%
\paragraph{v1.5:} 2017/05/21

\begin{itemize}
\item
more complete structuring introduced
\item
|\childdocof| introduced
\item
|\childdoc| renamed to |\childdocmain|
\item
|\childredirect| renamed to |\childdocforward| and |\childdocforwardprefix|
and functionality expanded
\end{itemize}

%%%%%%%%%%%%%%%%%%%%%%%%%%%%%%%%%%%%%%%%
\paragraph{v1.0:} 2017/04/27

\begin{itemize}
\item
manual and install package
\item
first version published on CTAN
\end{itemize}

%%%%%%%%%%%%%%%%%%%%%%%%%%%%%%%%%%%%%%%%
\paragraph{v0.6:} 2017/04/26

\begin{itemize}
\item
redirection mechanism added
\end{itemize}

%%%%%%%%%%%%%%%%%%%%%%%%%%%%%%%%%%%%%%%%
\paragraph{v0.5:} 2017/04/26

\begin{itemize}
\item
functionality in definition file
\end{itemize}


%%%%%%%%%%%%%%%%%%%%%%%%%%%%%%%%%%%%%%%%%%%%%%%%%%%%%%%%%%%%%%%%%%%%%%%%%%%%%%%%
%%%%%%%%%%%%%%%%%%%%%%%%%%%%%%%%%%%%%%%%%%%%%%%%%%%%%%%%%%%%%%%%%%%%%%%%%%%%%%%%
%%%%%%%%%%%%%%%%%%%%%%%%%%%%%%%%%%%%%%%%%%%%%%%%%%%%%%%%%%%%%%%%%%%%%%%%%%%%%%%%
\appendix

\settowidth\MacroIndent{\rmfamily\scriptsize 000\ }

 \DocInput{childdoc.dtx}

\end{document}
%</driver>
% \fi
%
% %%%%%%%%%%%%%%%%%%%%%%%%%%%%%%%%%%%%%%%%%%%%%%%%%%%%%%%%%%%%%%%%%%%%%%%%%%%%%%
% %%%%%%%%%%%%%%%%%%%%%%%%%%%%%%%%%%%%%%%%%%%%%%%%%%%%%%%%%%%%%%%%%%%%%%%%%%%%%%
% \section{Sample}
%\iffalse
%<*samplemain>
%\fi
%
% The following presents a sample document
% with two chapters, two parts, a title page,
% a compile flag as well as three forwarding files to set the flag.
% It consists of eight |.tex| files:
% \begin{center}
% \begin{tabular}{ll}
% |cdocsamp.tex|&main file\\
% |cdocsch1.tex|&include file for chapter 1\\
% |cdocsch2.tex|&include file for chapter 2\\
% |cdocspt3.tex|&include file for part 3\\
% |cdocspt4.tex|&include file for part 4\\
% |cdocsdrf.tex|&forwarding file for main file in draft mode\\
% |cdocsfi1.tex|&forwarding file for final version of chapter 1\\
% |cdocsfi2.tex|&forwarding file for final version of chapter 2\\
% \end{tabular}
% \end{center}
% Each of the eight files can be compiled directly by the \LaTeX{} compiler.
%
% %%%%%%%%%%%%%%%%%%%%%%%%%%%%%%%%%%%%%%
% \paragraph{Main File.}
%
% The main file is called |cdocsamp.tex|.
%
% Load the \textsf{childdoc} definitions and
% declare the filename for the main document:
%    \begin{macrocode}
\input{childdoc.def}
\childdocmain{}
%    \end{macrocode}

% Optional override for |\version| flag:
%    \begin{macrocode}
%%\ifchilddoc\else\providecommand{\version}{draft}\fi
%    \end{macrocode}

% Define the default values for the |\version| flag
% (|final| for the main file and |draft| for childs):
%    \begin{macrocode}
\ifchilddoc
\providecommand{\version}{draft}
\else
\providecommand{\version}{final}
\fi
%    \end{macrocode}

% Load the standard document class:
%    \begin{macrocode}
\documentclass[12pt]{article}
%    \end{macrocode}

% Start the document body:
%    \begin{macrocode}
\begin{document}
%    \end{macrocode}

% Declare a title page.
% Print title, part of document being processed and version flag:
%    \begin{macrocode}
\addtocounter{page}{-1}
\begin{center}
{\LARGE\bfseries{}childdoc example\par}
\vspace{1cm}
\ifchilddoc
\ifchilddocmanual part\else chapter\fi:
`\childdocname' of `\childdocjob'\par
\else
main document: `\childdocjob'\par
\fi
version: \version\par
\end{center}
\newpage
%    \end{macrocode}

% Manually include selected file,
% otherwise process as usual:
%    \begin{macrocode}
\ifchilddocmanual
\section*{part `\childdocname'}
\input{\childdocname}
\else
%    \end{macrocode}

% Include the two chapters:
%    \begin{macrocode}
\include{cdocsch1}
\include{cdocsch2}
%    \end{macrocode}

% Include the two parts unless only chapters should be displayed:
%    \begin{macrocode}
\ifchilddoc\else
\section{part three}
\input{cdocspt3}
\section{part four}
\input{cdocspt4}
\fi
%    \end{macrocode}

% Process as usual until here:
%    \begin{macrocode}
\fi
%    \end{macrocode}

% End of document body:
%    \begin{macrocode}
\end{document}
%    \end{macrocode}
%\iffalse
%</samplemain>
%\fi
%
% %%%%%%%%%%%%%%%%%%%%%%%%%%%%%%%%%%%%%%
% \paragraph{Chapter Include Files.}
%
% The include files are called |cdocsch1.tex| and |cdocsch2.tex|.
%
%\iffalse
%<*samplechap1|samplechap2>
%\fi

% Optional override for |\version| flag:
%    \begin{macrocode}
%%\providecommand{\version}{final}
%    \end{macrocode}

% Include the main document:
%    \begin{macrocode}
\input{childdoc.def}
\childdocof{cdocsamp}
%    \end{macrocode}

%\iffalse
%</samplechap1|samplechap2>
%\fi
%
%\iffalse
%<*samplechap1>
%\fi
% Some text for chapter 1:
%    \begin{macrocode}
\section{one}
some text in chapter one
%    \end{macrocode}

%\iffalse
%</samplechap1>
%\fi
% Some text for chapter 2:
%\iffalse
%<*samplechap2>
%\fi
%    \begin{macrocode}
\section{two}
more text in chapter two
%    \end{macrocode}

%\iffalse
%</samplechap2>
%\fi
%
% %%%%%%%%%%%%%%%%%%%%%%%%%%%%%%%%%%%%%%
% \paragraph{Part Include Files.}
%
% The include files are called |cdocspt3.tex| and |cdocspt4.tex|.
%
%\iffalse
%<*samplepart3|samplepart4>
%\fi

% Optional override for |\version| flag:
%    \begin{macrocode}
%%\providecommand{\version}{final}
%    \end{macrocode}

% Include the main document:
%    \begin{macrocode}
\input{childdoc.def}
\childdocby{cdocsamp}
%    \end{macrocode}

%\iffalse
%</samplepart3|samplepart4>
%\fi
%
%\iffalse
%<*samplepart3>
%\fi
% Some text for part 3:
%    \begin{macrocode}
some text in part three
%    \end{macrocode}

%\iffalse
%</samplepart3>
%\fi
% Some text for part 4:
%\iffalse
%<*samplepart4>
%\fi
%    \begin{macrocode}
more text in part four
%    \end{macrocode}

%\iffalse
%</samplepart4>
%\fi
%
% %%%%%%%%%%%%%%%%%%%%%%%%%%%%%%%%%%%%%%
% \paragraph{Forwarding for a Complete Draft.}
%
% The following forwarding file |cdocsdrf.tex|
% compiles the main document in draft mode:
%\iffalse
%<*sampledraft>
%\fi
%    \begin{macrocode}
\def\version{draft}
\input{childdoc.def}
\childdocforward{cdocsamp}
%    \end{macrocode}

%\iffalse
%</sampledraft>
%\fi
%
% %%%%%%%%%%%%%%%%%%%%%%%%%%%%%%%%%%%%%%
% \paragraph{Forwarding for Final Version of the Chapters.}
%
% The following forwarding files |cdocsfn1.tex| and |cdocsfn2.tex|
% (with identical content)
% compile the final versions of the child documents
% |cdocsch1.tex| and |cdocsch2.tex|, respectively:
%\iffalse
%<*samplefinal>
%\fi
%    \begin{macrocode}
\def\version{final}
\input{childdoc.def}
\childdocforwardprefix[cdocsamp]{cdocsfn}{cdocsch}
%    \end{macrocode}

%\iffalse
%</samplefinal>
%\fi
%
% %%%%%%%%%%%%%%%%%%%%%%%%%%%%%%%%%%%%%%
% \paragraph{Command Line Processing.}
%
% The following three command lines generate the output files
% |cdocscld|, |cdocscl1| and |cdocscl2|
% which should be identical to
% |cdocsdrf|, |cdocsch1| and |cdocsfn2|, respectively:
% \begin{center}
% \begin{tabular}{l}
% |latex -jobname cdocscld \|\\
% |  "\def\version{draft}\input{childdoc.def}\childdocforward{cdocsamp}"|\\
% |latex -jobname cdocscl1 \|\\
% |  "\input{childdoc.def}\childdocforward[cdocsamp]{cdocsch1}"|\\
% |latex -jobname cdocscl2 \|\\
% |  "\def\version{final}\input{childdoc.def}\childdocforward{cdocsch2}"|
% \end{tabular}
% \end{center}
% Note that the trailing backslash on each first line
% merely continues the input to the second line
% (for convenient cut ant paste).
% Furthermore, the command |latex| can be replaced by any
% of its alternative versions such as |pdflatex|.
%
% %%%%%%%%%%%%%%%%%%%%%%%%%%%%%%%%%%%%%%%%%%%%%%%%%%%%%%%%%%%%%%%%%%%%%%%%%%%%%%
% %%%%%%%%%%%%%%%%%%%%%%%%%%%%%%%%%%%%%%%%%%%%%%%%%%%%%%%%%%%%%%%%%%%%%%%%%%%%%%
% \section{Implementation}
%\iffalse
%<*package>
%\fi
%
% This section describes the definitions file |childdoc.def|.

% The definitions cannot be loaded using |\usepackage| or |\RequirePackage|
% which has a mechanism to prevent loading a style file more than once.
% When loading the definitions by means of |\input|
% multiple instances have to be prevented manually:
%\iffalse
%This code needs to be before the `\ProvidesFile' directive
%which is defined at the beginning of this file.
%Therefore it is also placed there and commented out here.
%</package>
%<*discard>
%\fi
%    \begin{macrocode}
\ifdefined\childdocmain\endinput\fi
%    \end{macrocode}
%\iffalse
%</discard>
%<*package>
%\fi
%
% \macro{\ifchilddoc}
% \macro{\ifchilddocmanual}
% The conditional |\ifchilddoc| tells whether a
% child (true) or main (false) document is being compiled.
% The conditional |\ifchilddocmanual| tells whether
% the |\includeonly| mechanism is used (false) or
% the selection of child files must be performed manually (true).
% The definitions initialise to false:
%    \begin{macrocode}
\newif\ifchilddoc
\newif\ifchilddocmanual
%    \end{macrocode}

% \macro{\childdocname}
% \macro{\childdocjob}
% The macro |\childdocname| stores the name of the main document
% to be compiled. The macro |\childdocjob| stores the name of
% the document on which the \LaTeX{} compiler was originally invoked.
% The content of |\jobname| cannot be compared
% to filenames specified in the source due to different catcodes.
% The following code rescans |\jobname|, stores the result
% in |\childdocname| and saves a copy in |\childdocjob|:
%    \begin{macrocode}
\edef\childdocname{\scantokens\expandafter{\jobname\noexpand}}
\let\childdocjob\childdocname
%    \end{macrocode}

% \macro{\childdocdisable}
% The macro |\childdocdisable| prevents the main file
% from being processed more than once.
% At this stage, the main document command |\childdocmain|
% is assumed to be called once again where it should do nothing.
% Any subsequent call to it should prevent
% a secondary processing of the main document
% It overwrites the forwarding commands
% |\childdocof| and |\childdocforward|
% with empty macros to prevent further inclusions of the main document:
%    \begin{macrocode}
\newcommand{\childdocdisable}
{
  \renewcommand{\childdocmain}[1]{\renewcommand{\childdocmain}[1]{\endinput}}
  \renewcommand{\childdocof}[1]{}
  \renewcommand{\childdocby}[2][]{}
  \renewcommand{\childdocforward}[2][]{}
  \renewcommand{\childdocdisable}{}
}
%    \end{macrocode}

% \macro{\childdocmain}
% The macro |\childdocmain| is to be called at the top of the main file
% with nothing or the main filename (without extension) as argument.
% First, it breaks loops.
% If the argument is not empty and does not match |\childdocname|
% (which is set by the first inclusion of |childdoc.def|),
% |\ifchilddoc| is set to true, |\includeonly| is applied to the child file
% and |\jobname| is set to the main file
% (for proper handling of |.aux| files):
%    \begin{macrocode}
\newcommand{\childdocmain}[1]
{
  \childdocdisable\childdocmain{}
  \if?#1?\else
    \begingroup
      \def\childdoctmp{#1}
      \ifx\childdoctmp\childdocname
        \def\childdoctmp{}
      \else
        \def\childdoctmp
        {
          \childdoctrue
          \includeonly{\childdocname}
          \def\childdocjob{#1}
          \def\jobname{#1}
        }
      \fi
      \expandafter
    \endgroup
    \childdoctmp
  \fi
}
%    \end{macrocode}

% \macro{\childdocof}
% The command |\childdocof| redirects
% compilation to the main file |#1|.
%    \begin{macrocode}
\newcommand{\childdocof}[1]
{
  \childdocdisable
  \childdoctrue
  \includeonly{\childdocname}
  \def\jobname{#1}
  \def\childdocjob{#1}
  \input{#1}
}
%    \end{macrocode}

% \macro{\childdocby}
% The command |\childdocby| ....
%    \begin{macrocode}
\newcommand{\childdocby}[2][]
{
  \childdocdisable
  \childdoctrue
  \childdocmanualtrue
  \if?#1?\else
    \def\jobname{#2}
  \fi
  \def\childdocjob{#2}
  \input{#2}
  \endinput
}
%    \end{macrocode}

% \macro{\childdocforward}
% The command |\childdocforward| redirects
% compilation to the main file or
% (if the optional argument is given) a child file.
% Parameters are set as if the main file
% or a child file starting with |\childdocof| was compiled.
% Then compilation is handed over to the main file:
%    \begin{macrocode}
\newcommand{\childdocforward}[2][]
{
  \begingroup
    \if?#1?
      \def\childdoctmp
      {
        \def\childdocname{#2}
        \def\childdocjob{#2}
        \def\jobname{#2}
        \input{#2}
        \endinput
      }
    \else
      \def\childdoctmp
      {
        \childdocdisable
        \def\childdocname{#2}
        \childdoctrue
        \includeonly{#2}
        \def\childdocjob{#1}
        \def\jobname{#1}
        \input{#1}
        \endinput
      }
    \fi
    \expandafter
  \endgroup
  \childdoctmp
}
%    \end{macrocode}

% \macro{\childdocforwardprefix}
% The command |\childdocforwardprefix| redirects
% compilation to the main or a child file by means of a pattern.
% The prefix |#1| in the current filename is replaced by |#2|
% and the suffix of the current filename is kept
% (it is assumed that the filename does not contain the substring `|~~~|'
% which is used as a delimiter).
% Compilation is handed over to the new file by |\childdocforward|:
%    \begin{macrocode}
\newcommand{\childdocforwardprefix}[3][]
{
  \begingroup
    \def\childdocextract #2##1~~~{\def\childdoctmp{\childdocforward[#1]{#3##1}}}
    \expandafter\childdocextract\childdocname~~~
    \expandafter
  \endgroup
  \childdoctmp
}
%    \end{macrocode}

% \macro{\childdoc}
% The deprecated macro |\childdoc| is a legacy version of |\childdocmain|:
%    \begin{macrocode}
\newcommand{\childdoc}{\childdocmain}
%    \end{macrocode}

% \macro{\childdocredirect}
% The deprecated macro |\childdocredirect| is a legacy version
% of |\childdocforward| and |\childdocforwardprefix|:
%    \begin{macrocode}
\newcommand{\childdocredirect}[2][]
{
  \begingroup
    \if?#1?
      \def\childdoctmp{\childdocforward{#2}}
    \else
      \def\childdoctmp{\childdocforwardprefix{#1}{#2}}
    \fi
    \expandafter
  \endgroup
  \childdoctmp
}
%    \end{macrocode}

%\iffalse
%</package>
%\fi
%
\endinput

\childdocmain{}
%    \end{macrocode}

% Optional override for |\version| flag:
%    \begin{macrocode}
%%\ifchilddoc\else\providecommand{\version}{draft}\fi
%    \end{macrocode}

% Define the default values for the |\version| flag
% (|final| for the main file and |draft| for childs):
%    \begin{macrocode}
\ifchilddoc
\providecommand{\version}{draft}
\else
\providecommand{\version}{final}
\fi
%    \end{macrocode}

% Load the standard document class:
%    \begin{macrocode}
\documentclass[12pt]{article}
%    \end{macrocode}

% Start the document body:
%    \begin{macrocode}
\begin{document}
%    \end{macrocode}

% Declare a title page.
% Print title, part of document being processed and version flag:
%    \begin{macrocode}
\addtocounter{page}{-1}
\begin{center}
{\LARGE\bfseries{}childdoc example\par}
\vspace{1cm}
\ifchilddoc
\ifchilddocmanual part\else chapter\fi:
`\childdocname' of `\childdocjob'\par
\else
main document: `\childdocjob'\par
\fi
version: \version\par
\end{center}
\newpage
%    \end{macrocode}

% Manually include selected file,
% otherwise process as usual:
%    \begin{macrocode}
\ifchilddocmanual
\section*{part `\childdocname'}
\input{\childdocname}
\else
%    \end{macrocode}

% Include the two chapters:
%    \begin{macrocode}
\include{cdocsch1}
\include{cdocsch2}
%    \end{macrocode}

% Include the two parts unless only chapters should be displayed:
%    \begin{macrocode}
\ifchilddoc\else
\section{part three}
\input{cdocspt3}
\section{part four}
\input{cdocspt4}
\fi
%    \end{macrocode}

% Process as usual until here:
%    \begin{macrocode}
\fi
%    \end{macrocode}

% End of document body:
%    \begin{macrocode}
\end{document}
%    \end{macrocode}
%\iffalse
%</samplemain>
%\fi
%
% %%%%%%%%%%%%%%%%%%%%%%%%%%%%%%%%%%%%%%
% \paragraph{Chapter Include Files.}
%
% The include files are called |cdocsch1.tex| and |cdocsch2.tex|.
%
%\iffalse
%<*samplechap1|samplechap2>
%\fi

% Optional override for |\version| flag:
%    \begin{macrocode}
%%\providecommand{\version}{final}
%    \end{macrocode}

% Include the main document:
%    \begin{macrocode}
% \iffalse
%
% childdoc.dtx Copyright (C) 2017-2018 Niklas Beisert
%
% This work may be distributed and/or modified under the
% conditions of the LaTeX Project Public License, either version 1.3
% of this license or (at your option) any later version.
% The latest version of this license is in
%   http://www.latex-project.org/lppl.txt
% and version 1.3 or later is part of all distributions of LaTeX
% version 2005/12/01 or later.
%
% This work has the LPPL maintenance status `maintained'.
%
% The Current Maintainer of this work is Niklas Beisert.
%
% This work consists of the files childdoc.dtx and childdoc.ins
% and the derived files childdoc.def and cdocsamp.tex with
% cdocsch1.tex, cdocsch2.tex, cdocsdrf.tex, cdocsfn1.tex, cdocsfn2.tex.
%
%<package>\ifdefined\childdocmain\endinput\fi
%<package>\ProvidesFile{childdoc.def}[2018/12/30 v2.0 child document driver]
%<samplemain>\ProvidesFile{cdocsamp.tex}[2018/12/30 v2.0 sample for childdoc]
%<*driver>
%\ProvidesFile{childdoc.drv}[2018/12/30 v2.0 childdoc reference manual file]
\PassOptionsToClass{10pt,a4paper}{article}
\documentclass{ltxdoc}

\usepackage[margin=35mm]{geometry}
\usepackage{hyperref}
\usepackage{hyperxmp}
\usepackage[usenames]{color}

\hypersetup{colorlinks=true}
\hypersetup{pdfstartview=FitH}
\hypersetup{pdfpagemode=UseNone}
\hypersetup{pdfsource={}}
\hypersetup{pdflang={en-UK}}
\hypersetup{pdfcopyright={Copyright 2017-2018 Niklas Beisert.
  This work may be distributed and/or modified under the
  conditions of the LaTeX Project Public License, either version 1.3
  of this license or (at your option) any later version.}}
\hypersetup{pdflicenseurl={http://www.latex-project.org/lppl.txt}}
\hypersetup{pdfcontactaddress={ETH Zurich, ITP, HIT K,
  Wolfgang-Pauli-Strasse 27}}
\hypersetup{pdfcontactpostcode={8093}}
\hypersetup{pdfcontactcity={Zurich}}
\hypersetup{pdfcontactcountry={Switzerland}}
\hypersetup{pdfcontactemail={nbeisert@itp.phys.ethz.ch}}
\hypersetup{pdfcontacturl={http://people.phys.ethz.ch/\xmptilde nbeisert/}}

\newcommand{\secref}[1]{\hyperref[#1]{section \ref*{#1}}}

\parskip1ex
\parindent0pt
\let\olditemize\itemize
\def\itemize{\olditemize\parskip0pt}

\begin{document}

\title{The \textsf{childdoc} Package}
\hypersetup{pdftitle={The childdoc Package}}
\author{Niklas Beisert\\[2ex]
  Institut f\"ur Theoretische Physik\\
  Eidgen\"ossische Technische Hochschule Z\"urich\\
  Wolfgang-Pauli-Strasse 27, 8093 Z\"urich, Switzerland\\[1ex]
  \href{mailto:nbeisert@itp.phys.ethz.ch}
  {\texttt{nbeisert@itp.phys.ethz.ch}}}
\hypersetup{pdfauthor={Niklas Beisert}}
\hypersetup{pdfsubject={Manual for the LaTeX2e Package childdoc}}
\date{30 December 2018, \textsf{v2.0}}
\maketitle

\begin{abstract}\noindent
\textsf{childdoc} is a \LaTeXe{} package
that enables the direct compilation
of document sections included by |\include|
to individual files.
\end{abstract}

\begingroup
\parskip0ex
\tableofcontents
\endgroup

%%%%%%%%%%%%%%%%%%%%%%%%%%%%%%%%%%%%%%%%%%%%%%%%%%%%%%%%%%%%%%%%%%%%%%%%%%%%%%%%
%%%%%%%%%%%%%%%%%%%%%%%%%%%%%%%%%%%%%%%%%%%%%%%%%%%%%%%%%%%%%%%%%%%%%%%%%%%%%%%%
\section{Introduction}

\LaTeX{} provides a mechanism to structure a large document (such as a book)
into a main file and several child files (containing the chapters)
using the |\include| command.
This mechanism is beneficial for documents
which span hundreds of pages in order to
make the source file(s) more manageable.
Moreover, compilation can be restricted to
selected child files by means of the |\includeonly| command.
The latter feature can be used to reduce the compilation time while editing
(this was significantly more useful in the earlier days of \LaTeX{})
or to generate a smaller document which is easier to navigate.
Another application of |\includeonly| is to generate
documents consisting of selected parts of the complete document.

However, there are a few drawbacks of the plain |\include| mechanism:
\begin{itemize}
\item
The child files cannot be compiled on their own,
they can only be compiled via the main file.
A naive editing environment
(such as a text editor with an option
to have the current file processed by \LaTeX)
may require one to switch to the main file before compiling;
attempting to compile the child file produces errors.
\item
The main file must be modified (each time)
to adjust the |\includeonly| command
to the present needs. This easily leaves the main file in a messy state.
\item
The generated document will always carry the filename
of the main document. This is inconvenient if
several child files are to be compiled and
to be kept for distribution.
\end{itemize}

The present package provides a simple interface
to make child files individually compilable by \LaTeX{}.
Compiling a child file then has the same effect as compiling
the main file with an |\includeonly| command
to select the appropriate child.
Moreover the generated document will carry the name of the child
rather than the main file.
This resolves all three above issues.

This feature is meant to make the editing of books,
thesis documents and lecture notes somewhat more convenient.
However, the package can also be used efficiently for
composing a series of documents (such as exercise sheets)
which are typically distributed individually.
It then assists the author in generating the individual documents
(potentially in different versions)
as well as a document containing the collected series.
Another application is in developing style files
or other kinds of included material
where compilation of the style file could redirect
to a sample or test file.

%%%%%%%%%%%%%%%%%%%%%%%%%%%%%%%%%%%%%%%%%%%%%%%%%%%%%%%%%%%%%%%%%%%%%%%%%%%%%%%%
%%%%%%%%%%%%%%%%%%%%%%%%%%%%%%%%%%%%%%%%%%%%%%%%%%%%%%%%%%%%%%%%%%%%%%%%%%%%%%%%
\section{Usage}

First of all, the package \textsf{childdoc} is \emph{not} a standard
\LaTeXe{} |.sty| style file! Therefore it needs to be invoked in
a non-standard way.

%%%%%%%%%%%%%%%%%%%%%%%%%%%%%%%%%%%%%%%%%%%%%%%%%%%%%%%%%%%%%%%%%%%%%%%%%%%%%%%%
\subsection{Included Files}
\label{sec:include}

%%%%%%%%%%%%%%%%%%%%%%%%%%%%%%%%%%%%%%%%
\DescribeMacro{\childdocmain}
To use the package, add the commands
\begin{center}
\begin{tabular}{l}
|\input{childdoc.def}|\\
|\childdocmain{}|\\
\end{tabular}
\end{center}
at the very top of the main \LaTeX{} file,
in particular \emph{before} the |\documentclass| statement!
The argument of |\childdocmain| should be left empty
(but it must be present).

%%%%%%%%%%%%%%%%%%%%%%%%%%%%%%%%%%%%%%%%
\DescribeMacro{\childdocof}
Furthermore, add the commands
\begin{center}
\begin{tabular}{l}
|\input{childdoc.def}|\\
|\childdocof{|\textit{main}|}|\\
\end{tabular}
\end{center}
at the top of every child file \textit{child}
which is included by |\include{|\textit{child}|}|
from within the main file
(or at least for those files to be compiled individually).
The argument \textit{main} must be the filename of the main file.

There are a couple of
considerations in setting up the main and child documents:

%%%%%%%%%%%%%%%%%%%%%%%%%%%%%%%%%%%%%%%%
\paragraph{Restrictions.}

Please note the following restrictions:
\begin{itemize}
\item
|\childdocmain| must be called with one argument \textit{main}
to ensure compatibility with earlier version of the package.
It must either be empty (|\childdocmain{}|)
or precisely match the filename of the main file in which it is specified.
See \secref{sec:detection} for further information.
\item
The filename \textit{main} must be specified without the |.tex| extension.
\item
The filename \textit{main} is case sensitive
(even in case-insensitive file systems)
due to internal string comparison.
\item
The argument \textit{main} should be fully expanded, it cannot be a macro.
\item
Subdirectories and special characters should be avoided in filenames.
\item
The command |\childdocmain{|\textit{main}|}| must be followed by a whitespace.
It should not be followed immediately by another command
or by a comment mark `|%|'.
This is because the \TeX{} parser reads the token immediately following
the argument of |\childdocmain| and puts it
at the beginning of every child section;
however, a white\-space is ignored.
\end{itemize}

%%%%%%%%%%%%%%%%%%%%%%%%%%%%%%%%%%%%%%%%
\paragraph{Content of Main File.}

It is advisable to place all content in the child files included by |\include|.
Any output contained in the main file will appear in all child documents
unless suppressed manually;
it cannot be suppressed automatically by the |\includeonly| directive
and thus should normally be avoided.
A method to include some content in the main file
by means of conditional processing is described in \secref{sec:conditional}.

%%%%%%%%%%%%%%%%%%%%%%%%%%%%%%%%%%%%%%%%
\paragraph{Page Numbering.}

When only a part of the document is compiled,
the appropriate numbering of pages
(as well as other status parameters)
is determined from the |.aux| files.
The latter contain information from previous passes.
However this information needs to propagate through
all intermediate child documents.
Therefore the page numbering in child documents may well
be inconsistent until the complete document is compiled at least once.

A useful (if unconventional) way to always ensure a consistent
page numbering is to restart the numbering in each child document
and denote the pages by `\textit{child}|.|\textit{page}'
where \textit{child} represents the chapter/section number of the child file.
This can be achieved by the command
|\numberwithin{page}{|\textit{child}|}|
of the \textsf{amsmath} package
where \textit{child} can be |chapter| or |section|
depending on the chosen structuring.
Alternatively, one can modify the macro |\thepage| appropriately
and reset the counter |page| at the start of each child file.

%%%%%%%%%%%%%%%%%%%%%%%%%%%%%%%%%%%%%%%%%%%%%%%%%%%%%%%%%%%%%%%%%%%%%%%%%%%%%%%%
\subsection{Conditional Processing}
\label{sec:conditional}

The package provides a mechanism to compile different versions
of a document. To customise the versions further some conditional processing
can come in handy to distinguish which version is being compiled.
The package provides two macros to describe the compilation context:

%%%%%%%%%%%%%%%%%%%%%%%%%%%%%%%%%%%%%%%%
\DescribeMacro{\ifchilddoc}
The conditional |\ifchilddoc| distinguishes between the compilation of
child documents and the main document:
%
\begin{center}
|\ifchilddoc |\textit{child-code}| |[|\||else |\textit{main-code}]| \||fi|
\end{center}

%%%%%%%%%%%%%%%%%%%%%%%%%%%%%%%%%%%%%%%%
\DescribeMacro{\childdocname}
\DescribeMacro{\childdocjob}
The macro |\childdocname| contains the filename (without extension)
of the main or child file being processed.
Note that |\childdocjob| will always contain the name of the main file.

%%%%%%%%%%%%%%%%%%%%%%%%%%%%%%%%%%%%%%%%
\paragraph{Title Page.}

Conditional processing can be used to include a title or banner page
in the main document when proper precautions are taken.
Importantly, the code in the main file should ensure that the page counter
(as well as other status parameters which are stored in the |.aux| files)
takes the same value after the conditional processing.
Otherwise the page numbers may take divergent values
depending on which part is compiled.

For example, a title page could be declared by:
%
\begin{center}
\begin{tabular}{l}
|\ifchilddoc\||else|\\
|\addtocounter{page}{-1}|\\
\textit{code for title page}\\
|\newpage|\\
|\||fi|
\end{tabular}
\end{center}
%
A banner page for the child documents can be generated by:
%
\begin{center}
\begin{tabular}{l}
|\ifchilddoc|\\
|\addtocounter{page}{-1}|\\
\textit{code for banner page}\\
|\newpage|\\
|\||fi|
\end{tabular}
\end{center}
%
Here one could write a message such as:
\begin{center}
|This is the part \childdocname{} of \childdocjob{}.|
\end{center}

%%%%%%%%%%%%%%%%%%%%%%%%%%%%%%%%%%%%%%%%%%%%%%%%%%%%%%%%%%%%%%%%%%%%%%%%%%%%%%%%
\subsection{Flags}
\label{sec:flags}

The package makes it easy to generate different versions
of the main or child documents.
To this end compilation flags can be defined
and assigned different default values.
They will be particularly useful in conjunction
with the forwarding mechanism described in \secref{sec:forward}.

For example, it may be useful to have a flag |\version|
which can be set to |draft| or |final|.
The document source will contain some conditional code
depending on the value of |\version|.
Suppose further, the flag should default to |final| for the main file
and to |draft| for child files
which is a natural assignment for editing the document.
This is achieved by placing the following code
in the preamble of the main document
(below the |\childdocmain| directive):
%
\begin{center}
\begin{tabular}{l}
|\ifchilddoc|\\
|\providecommand{\version}{draft}|\\
|\||else|\\
|\providecommand{\version}{final}|\\
|\||fi|
\end{tabular}
\end{center}
%
The definition by |\providecommand| makes sure
that previous definitions are not overwritten.
Further statements |\providecommand{\version}{...}|
can thus be added before the above code to override it.

For the main file, one might add a line
(between |\childdocmain| and the above block)
%
\begin{center}
|%\ifchilddoc\||else\providecommand{\version}{draft}\||fi|
\end{center}
%
which can be uncommented to produce a draft version.
Likewise one can add a line to the very top of a child file
(above the |\childdocof{|\textit{main}|}| directive)
%
\begin{center}
|%\providecommand{\version}{final}|
\end{center}
%
which can be uncommented to produce the final version of this child document.

%%%%%%%%%%%%%%%%%%%%%%%%%%%%%%%%%%%%%%%%%%%%%%%%%%%%%%%%%%%%%%%%%%%%%%%%%%%%%%%%
\subsection{Forwarding}
\label{sec:forward}

Different versions of the main or child documents
using compilation flags as described in \secref{sec:flags}
can be (permanently) stored in different files
for convenient compilation, viewing and distribution.
To this end, the package defines a command
to pass on compilation to a different file:

%%%%%%%%%%%%%%%%%%%%%%%%%%%%%%%%%%%%%%%%
\DescribeMacro{\childdocforward}
The command |\childdocforward| redirects processing to
another source file:
%
\begin{center}
\begin{tabular}{l}
|\input{childdoc.def}|\\
|\childdocforward[|\textit{main}|]{|\textit{dest}|}|\\
\end{tabular}
\end{center}
%
The argument \textit{dest} is the destination file
(without extension).
It should be the main file or one of the child files.
Note that further \textsf{childdoc} directives
such as |\childdocof| and |\childdocforward|
in the indicated file will be processed in this form.
The optional argument \textit{main}
passes on directly to the main file \textit{main}
while pretending to compile the child \textit{dest}.
This form behaves as if \textit{dest}
issues |\childdocof{|\textit{main}|}| right away,
and no further \textsf{childdoc} directives will be processed.

%%%%%%%%%%%%%%%%%%%%%%%%%%%%%%%%%%%%%%%%
\DescribeMacro{\...prefix}
In the alternative form |\childdocforwardprefix|,
%
\begin{center}
\begin{tabular}{l}
|\input{childdoc.def}|\\
|\childdocforwardprefix[|\textit{main}|]{|\textit{prefix}|}{|\textit{dest}|}|
\end{tabular}
\end{center}
%
the destination file is determined by a pattern
depending on the current file:
To make this work, the current file must be called
`{\textit{prefix}\hspace{0.2em}\textit{suffix}}'
with \textit{prefix} matching precisely the argument.
Processing is then passed on to the file
`{\textit{dest}\hspace{0.2em}\textit{suffix}}'.
Surely, the same effect is achieved by
directly specifying the
argument `{\textit{dest}\hspace{0.2em}\textit{suffix}}'
in the first form.
However, that requires to set up a different file
for each child. With the alternative form of the command
all these files can have exactly the same content
which simplifies setting them up and maintaining them.

For example, the following file |draft.tex|
with a compilation flag |\version| as described in \secref{sec:flags}
compiles the main document as a draft:
%
\begin{center}
\begin{tabular}{l}
|\def\version{draft}|\\
|\input{childdoc.def}|\\
|\childdocforward{|\textit{main}|}|
\end{tabular}
\end{center}
%
Likewise, the following files |final|\textit{nn}|.tex|
compile the final version of the child document
|child|\textit{nn}|.tex|:
%
\begin{center}
\begin{tabular}{l}
|\def\version{final}|\\
|\input{childdoc.def}|\\
|\childdocforwardprefix{final}{child}|
\end{tabular}
\end{center}
%

Note that when several versions of a main file and/or of each child file
are to be generated, it may be convenient to set up a |Makefile| or
shell script to automatise the process.

%%%%%%%%%%%%%%%%%%%%%%%%%%%%%%%%%%%%%%%%%%%%%%%%%%%%%%%%%%%%%%%%%%%%%%%%%%%%%%%%
\subsection{Command Line Processing}
\label{sec:commandline}

The effect of redirection files can also be achieved by invoking
the \LaTeX{} compiler with a more elaborate command line.
Most conveniently this should be done as part
of a shell script or a |Makefile|.

When using \textsf{childdoc} in the main file, the following
command lines effectively perform a redirection
(note that depending on the shell being used,
backslashes may have to be doubled: `|\|' $\to$ `|\\|'):
%
\begin{center}
|... -jobname "|\textit{target}|" |\\|"|[\textit{flags}]%
|\input{childdoc.def}\childdocforward[|\textit{main}|]{|\textit{dest}|}"|
\end{center}
%
Here \textit{target} is the name of the output file,
\textit{main} is the name of the main file
and \textit{dest} is the name of the main or child file to be processed
(all filenames without extensions).
The optional argument \textit{main} can be omitted
if \textit{main} matches \textit{dest}.
Optionally, compilation \textit{flags} can be defined via |\def| commands.
This command line makes the \TeX{} engine believe
it is compiling the file \textit{target}
whose content is specified as the latter parameter.
The provided code then forwards the processing to
\textit{main} or \textit{dest} as described in \secref{sec:forward}.

%%%%%%%%%%%%%%%%%%%%%%%%%%%%%%%%%%%%%%%%%%%%%%%%%%%%%%%%%%%%%%%%%%%%%%%%%%%%%%%%
\subsection{Include by Input}
\label{sec:input}

Including child documents by |\include| has some restrictions by design.
Most notably, the content of a child document always occupies
its own set of pages; pages cannot be shared between child documents.
Usually, this behaviour makes perfect sense
because each child document contain an essential part of the document.
However, in some situations it may be desirable to compose
a document from a collection of parts
without having mandatory page breaks between then.
For this case, the package
provides a mechanism to include parts
by |\input| which can also be processed individually.
However, by construction this mechanism
requires manual handling of the content to be output.

%%%%%%%%%%%%%%%%%%%%%%%%%%%%%%%%%%%%%%%%
\DescribeMacro{\ifchilddocmanual}
The main file should be prepared as usual, see \secref{sec:include}.
However, the document body must make a distinction
between processing of an individual part and of the main document, e.g.:
%
\begin{center}
\begin{tabular}{l}
|\ifchilddocmanual|\\
|\input{\childdocname}|\\
|\||else|\\
\textit{document body with }|\input{|\textit{part}|}|\\
|\||fi|
\end{tabular}
\end{center}
%
The conditional |\ifchilddocmanual| is true whenever
a part to be included by |\input| is being compiled,
and the name of the part is stored in |\childdocname|.

%%%%%%%%%%%%%%%%%%%%%%%%%%%%%%%%%%%%%%%%
\DescribeMacro{\childdocby}
Each part to be included by |\input| should start with:
%
\begin{center}
\begin{tabular}{l}
|\input{childdoc.def}|\\
|\childdocby{|\textit{main}|}|\\
\end{tabular}
\end{center}
%
The directive |\childdocby| is similar to |\childdocof|
described in \secref{sec:include},
but the subsequent selection of content must be done manually.
To that end, both |\ifchilddoc| and |\ifchilddocmanual|
will be true upon processing of a part,
and the name of the part is stored in |\childdocname|.
Note that |\jobname| will be set to the filename of the current part
so that each part receives an individual |.aux| file
that does not interfere with the |.aux| file(s) of the main document.
This behaviour can be altered by the alternative form
|\childdocby[*]{|\textit{main}|}| (with a non-empty optional argument)
which uses the |.aux| file of the main document
by setting |\jobname| to \textit{main}.

%%%%%%%%%%%%%%%%%%%%%%%%%%%%%%%%%%%%%%%%%%%%%%%%%%%%%%%%%%%%%%%%%%%%%%%%%%%%%%%%
\subsection{Driver Development}
\label{sec:driver}

The \textsf{childdoc} mechanism can also be use for the development
of definition files such as \LaTeX{} styles or classes.
This case differs from the above setup with multiple parts
included by |\include| in that no |\includeonly| should be invoked.
This can be achieved by starting the include file
(before |\ProvidesPackage|) with:
%
\begin{center}
\begin{tabular}{l}
|\input{childdoc.def}|\\
|\childdocforward{|\textit{main}|}|\\
\end{tabular}
\end{center}
%
or alternatively with:
%
\begin{center}
\begin{tabular}{l}
|\input{childdoc.def}|\\
|\childdocby{|\textit{main}|}|\\
\end{tabular}
\end{center}
%
Both forms have slightly different effects as described above.
The main file is prepared as usual, see \secref{sec:include}.

%%%%%%%%%%%%%%%%%%%%%%%%%%%%%%%%%%%%%%%%%%%%%%%%%%%%%%%%%%%%%%%%%%%%%%%%%%%%%%%%
\subsection{Legacy Detection}
\label{sec:detection}

The directive |\childdocmain| in the main file can detect
whether the complete document or merely a child is to be compiled
even without using the directive |\childdocof|.
This method is deprecated because it is less robust
and there is no compelling reason to use it;
it is merely provided for backward compatibility
and it may be removed in future versions.

If the detection mechanism is to be used,
it is mandatory to correctly specify
the filename of the main file as the argument of |\childdocmain|:
%
\begin{center}
\begin{tabular}{l}
|\input{childdoc.def}|\\
|\childdocmain{|\textit{main}|}|\\
\end{tabular}
\end{center}
%
If |\jobname| does not match the argument \textit{main} of |\childdocmain|,
it is assumed that |\jobname| points to the child file to be compiled.
When using |\childdocmain| with the main file specified as argument,
it suffices to start a child file
with just |\input{|\textit{main}|}|
without loading of the package and using |\childdocof|.
If instead all processing is done
with the appropriate \textsf{childdoc} directives,
the argument of \textit{main} of |\childdocmain| can be empty.

An alternative version of the command line processing described
in \secref{sec:commandline} using the detection mechanism reads:
%
\begin{center}
|... -jobname "|\textit{target}|" "|[\textit{flags}]%
[|\def\jobname{|\textit{dest}|}|]|\input{|\textit{main}|}"|
\end{center}

%%%%%%%%%%%%%%%%%%%%%%%%%%%%%%%%%%%%%%%%%%%%%%%%%%%%%%%%%%%%%%%%%%%%%%%%%%%%%%%%
\subsection{Manual Code}
\label{sec:manual}

In case one cannot be certain whether the definitions file |childdoc.def|
is installed on the target \TeX{} distribution
and one prefers not to ship it,
it is conceivable to paste a few relevant commands into the sources.

To that end, drop all statements |\input{childdoc.def}|
and perform the replacements as outlined below.
Instead of |\childdocmain{|\textit{main}|}| add the following code
to the top of the main file:
%
\begin{center}
\begin{tabular}{l}
|\||ifdefined\childdocname\endinput\||fi\newif\ifchilddoc|\\
|\edef\childdocname{\scantokens\expandafter{\jobname\noexpand}}|\\
|\def\childdocmain{|\textit{main}|}\||ifx\childdocmain\childdocname\||else|\\
|\childdoctrue\includeonly{\childdocname}\let\jobname\childdocmain\||fi|\\
\end{tabular}
\end{center}
%
Instead of |\childdocof{|\textit{main}|}| just include the main file
at the top of each child file:
%
\begin{center}
|\input{|\textit{main}|}|
\end{center}
%
A simple redirection |\childdocforward{|\textit{dest}|}| is achieved by:
%
\begin{center}
|\def\jobname{|\textit{dest}|}\input{\jobname}|
\end{center}
%
The redirection with prefix
|\childdocforwardprefix[|\textit{prefix}|]{|\textit{dest}|}|
is accomplished by:
%
\begin{center}
\begin{tabular}{l}
|{\edef\jobname{\scantokens\expandafter{\jobname\noexpand}}|\\
|\def\redirectjob |\textit{prefix}|#1~~~{\gdef\jobname{|\textit{dest}|#1}}|\\
|\expandafter\redirectjob\jobname~~~}\input{\jobname}|
\end{tabular}
\end{center}

In an alternative approach,
child documents can be compiled by a specific command line
without additional code or specific definitions:
%
\begin{center}
|... -jobname "|\textit{target}|" "|[\textit{flags}]%
|\includeonly{|\textit{dest}|}\input{|\textit{main}|}"|
\end{center}
%

%%%%%%%%%%%%%%%%%%%%%%%%%%%%%%%%%%%%%%%%%%%%%%%%%%%%%%%%%%%%%%%%%%%%%%%%%%%%%%%%
%%%%%%%%%%%%%%%%%%%%%%%%%%%%%%%%%%%%%%%%%%%%%%%%%%%%%%%%%%%%%%%%%%%%%%%%%%%%%%%%
\section{Information}

%%%%%%%%%%%%%%%%%%%%%%%%%%%%%%%%%%%%%%%%%%%%%%%%%%%%%%%%%%%%%%%%%%%%%%%%%%%%%%%%
\subsection{Copyright}

Copyright \copyright{} 2017--2018 Niklas Beisert

This work may be distributed and/or modified under the
conditions of the \LaTeX{} Project Public License, either version 1.3
of this license or (at your option) any later version.
The latest version of this license is in
  \url{http://www.latex-project.org/lppl.txt}
and version 1.3 or later is part of all distributions of \LaTeX{}
version 2005/12/01 or later.

This work has the LPPL maintenance status `maintained'.

The Current Maintainer of this work is Niklas Beisert.

This work consists of the files |README.txt|, |childdoc.ins| and |childdoc.dtx|
as well as the derived files |childdoc.def|, |cdocsamp.tex|
with |cdocsch1.tex|, |cdocsch2.tex|, |cdocspt3.tex|, |cdocspt4.tex|,
|cdocsdrf.tex|, |cdocsfn1.tex|, |cdocsfn2.tex|
as well as |childdoc.pdf|.

%%%%%%%%%%%%%%%%%%%%%%%%%%%%%%%%%%%%%%%%%%%%%%%%%%%%%%%%%%%%%%%%%%%%%%%%%%%%%%%%
\subsection{Files and Installation}

The package consists of the files:
%
\begin{center}
\begin{tabular}{ll}
    |README.txt|   & readme file \\
    |childdoc.ins| & installation file \\
    |childdoc.dtx| & source file \\
    |childdoc.def| & definition file \\
    |cdocsamp.tex| & sample main file \\
    |cdocsch1.tex| & sample include file \\
    |cdocsch2.tex| & sample include file \\
    |cdocspt3.tex| & sample part file \\
    |cdocspt4.tex| & sample part file \\
    |cdocsdrf.tex| & sample redirection file \\
    |cdocsfn1.tex| & sample redirection file \\
    |cdocsfn2.tex| & sample redirection file \\
    |childdoc.pdf| & manual
\end{tabular}
\end{center}
%
The distribution consists of the files
|README.txt|, |childdoc.ins| and |childdoc.dtx|.
%
\begin{itemize}
\item
Run (pdf)\LaTeX{} on |childdoc.dtx|
to compile the manual |childdoc.pdf| (this file).
\item
Run \LaTeX{} on |childdoc.ins| to create the definitions file |childdoc.def|
and the sample |cdocsamp.tex| with include files
|cdocsch1.tex|, |cdocsch2.tex|, |cdocspt3.tex|, |cdocspt4.tex|,
|cdocsdrf.tex|, |cdocsfn1.tex|, |cdocsfn2.tex|.
Then copy the file |childdoc.def| to an appropriate directory of your \LaTeX{}
distribution, e.g.\ \textit{texmf-root}|/tex/latex/childdoc|.
\end{itemize}

%%%%%%%%%%%%%%%%%%%%%%%%%%%%%%%%%%%%%%%%%%%%%%%%%%%%%%%%%%%%%%%%%%%%%%%%%%%%%%%%
\subsection{Related CTAN Packages}

There are several other packages which offer a similar functionality:
%
\begin{itemize}
\item
The packages
\href{http://ctan.org/pkg/docmute}{\textsf{docmute}},
\href{http://ctan.org/pkg/includex}{\textsf{includex}} and
\href{http://ctan.org/pkg/standalone}{\textsf{standalone}}
provide commands to include only the document body of
a child file thus allowing both files to be compiled individually.
\item
The packages \href{http://ctan.org/pkg/subdocs}{\textsf{subdocs}}
and \href{http://ctan.org/pkg/subfiles}{\textsf{subfiles}}
provide structures in which the main and child documents can be
encapsulated and allowing them to be compiled individually.
The inclusion mechanism is different from the conventional |\include|.
\item
The package \href{http://ctan.org/pkg/combine}{\textsf{combine}}
is an elaborate solution to combine several documents into one.
\end{itemize}
%
See also the CTAN topic \href{http://ctan.org/topic/subdocs}{\textsf{subdocs}}
for further related packages.
The present package differs from the above solutions in that
a document structure constructed with the conventional |\include| mechanism
just needs two extra commands at the top of every file
such that all constituent files can be compiled individually.

%%%%%%%%%%%%%%%%%%%%%%%%%%%%%%%%%%%%%%%%%%%%%%%%%%%%%%%%%%%%%%%%%%%%%%%%%%%%%%%%
%\subsection{Feature Suggestions}
%
%The following is a list of features which may be useful for future
%versions of this package:
%%
%\begin{itemize}
%\item
%\ldots
%\end{itemize}

%%%%%%%%%%%%%%%%%%%%%%%%%%%%%%%%%%%%%%%%%%%%%%%%%%%%%%%%%%%%%%%%%%%%%%%%%%%%%%%%
\subsection{Revision History}

%%%%%%%%%%%%%%%%%%%%%%%%%%%%%%%%%%%%%%%%
\paragraph{v2.0:} 2018/12/30

\begin{itemize}
\item
immediate forward processing
\item
added |\childdocby| mechanism
\item
manual restructured
\end{itemize}

%%%%%%%%%%%%%%%%%%%%%%%%%%%%%%%%%%%%%%%%
\paragraph{v1.6:} 2018/01/17

\begin{itemize}
\item
application for development of include files
\item
corrections to manual
\end{itemize}

%%%%%%%%%%%%%%%%%%%%%%%%%%%%%%%%%%%%%%%%
\paragraph{v1.5:} 2017/05/21

\begin{itemize}
\item
more complete structuring introduced
\item
|\childdocof| introduced
\item
|\childdoc| renamed to |\childdocmain|
\item
|\childredirect| renamed to |\childdocforward| and |\childdocforwardprefix|
and functionality expanded
\end{itemize}

%%%%%%%%%%%%%%%%%%%%%%%%%%%%%%%%%%%%%%%%
\paragraph{v1.0:} 2017/04/27

\begin{itemize}
\item
manual and install package
\item
first version published on CTAN
\end{itemize}

%%%%%%%%%%%%%%%%%%%%%%%%%%%%%%%%%%%%%%%%
\paragraph{v0.6:} 2017/04/26

\begin{itemize}
\item
redirection mechanism added
\end{itemize}

%%%%%%%%%%%%%%%%%%%%%%%%%%%%%%%%%%%%%%%%
\paragraph{v0.5:} 2017/04/26

\begin{itemize}
\item
functionality in definition file
\end{itemize}


%%%%%%%%%%%%%%%%%%%%%%%%%%%%%%%%%%%%%%%%%%%%%%%%%%%%%%%%%%%%%%%%%%%%%%%%%%%%%%%%
%%%%%%%%%%%%%%%%%%%%%%%%%%%%%%%%%%%%%%%%%%%%%%%%%%%%%%%%%%%%%%%%%%%%%%%%%%%%%%%%
%%%%%%%%%%%%%%%%%%%%%%%%%%%%%%%%%%%%%%%%%%%%%%%%%%%%%%%%%%%%%%%%%%%%%%%%%%%%%%%%
\appendix

\settowidth\MacroIndent{\rmfamily\scriptsize 000\ }

 \DocInput{childdoc.dtx}

\end{document}
%</driver>
% \fi
%
% %%%%%%%%%%%%%%%%%%%%%%%%%%%%%%%%%%%%%%%%%%%%%%%%%%%%%%%%%%%%%%%%%%%%%%%%%%%%%%
% %%%%%%%%%%%%%%%%%%%%%%%%%%%%%%%%%%%%%%%%%%%%%%%%%%%%%%%%%%%%%%%%%%%%%%%%%%%%%%
% \section{Sample}
%\iffalse
%<*samplemain>
%\fi
%
% The following presents a sample document
% with two chapters, two parts, a title page,
% a compile flag as well as three forwarding files to set the flag.
% It consists of eight |.tex| files:
% \begin{center}
% \begin{tabular}{ll}
% |cdocsamp.tex|&main file\\
% |cdocsch1.tex|&include file for chapter 1\\
% |cdocsch2.tex|&include file for chapter 2\\
% |cdocspt3.tex|&include file for part 3\\
% |cdocspt4.tex|&include file for part 4\\
% |cdocsdrf.tex|&forwarding file for main file in draft mode\\
% |cdocsfi1.tex|&forwarding file for final version of chapter 1\\
% |cdocsfi2.tex|&forwarding file for final version of chapter 2\\
% \end{tabular}
% \end{center}
% Each of the eight files can be compiled directly by the \LaTeX{} compiler.
%
% %%%%%%%%%%%%%%%%%%%%%%%%%%%%%%%%%%%%%%
% \paragraph{Main File.}
%
% The main file is called |cdocsamp.tex|.
%
% Load the \textsf{childdoc} definitions and
% declare the filename for the main document:
%    \begin{macrocode}
\input{childdoc.def}
\childdocmain{}
%    \end{macrocode}

% Optional override for |\version| flag:
%    \begin{macrocode}
%%\ifchilddoc\else\providecommand{\version}{draft}\fi
%    \end{macrocode}

% Define the default values for the |\version| flag
% (|final| for the main file and |draft| for childs):
%    \begin{macrocode}
\ifchilddoc
\providecommand{\version}{draft}
\else
\providecommand{\version}{final}
\fi
%    \end{macrocode}

% Load the standard document class:
%    \begin{macrocode}
\documentclass[12pt]{article}
%    \end{macrocode}

% Start the document body:
%    \begin{macrocode}
\begin{document}
%    \end{macrocode}

% Declare a title page.
% Print title, part of document being processed and version flag:
%    \begin{macrocode}
\addtocounter{page}{-1}
\begin{center}
{\LARGE\bfseries{}childdoc example\par}
\vspace{1cm}
\ifchilddoc
\ifchilddocmanual part\else chapter\fi:
`\childdocname' of `\childdocjob'\par
\else
main document: `\childdocjob'\par
\fi
version: \version\par
\end{center}
\newpage
%    \end{macrocode}

% Manually include selected file,
% otherwise process as usual:
%    \begin{macrocode}
\ifchilddocmanual
\section*{part `\childdocname'}
\input{\childdocname}
\else
%    \end{macrocode}

% Include the two chapters:
%    \begin{macrocode}
\include{cdocsch1}
\include{cdocsch2}
%    \end{macrocode}

% Include the two parts unless only chapters should be displayed:
%    \begin{macrocode}
\ifchilddoc\else
\section{part three}
\input{cdocspt3}
\section{part four}
\input{cdocspt4}
\fi
%    \end{macrocode}

% Process as usual until here:
%    \begin{macrocode}
\fi
%    \end{macrocode}

% End of document body:
%    \begin{macrocode}
\end{document}
%    \end{macrocode}
%\iffalse
%</samplemain>
%\fi
%
% %%%%%%%%%%%%%%%%%%%%%%%%%%%%%%%%%%%%%%
% \paragraph{Chapter Include Files.}
%
% The include files are called |cdocsch1.tex| and |cdocsch2.tex|.
%
%\iffalse
%<*samplechap1|samplechap2>
%\fi

% Optional override for |\version| flag:
%    \begin{macrocode}
%%\providecommand{\version}{final}
%    \end{macrocode}

% Include the main document:
%    \begin{macrocode}
\input{childdoc.def}
\childdocof{cdocsamp}
%    \end{macrocode}

%\iffalse
%</samplechap1|samplechap2>
%\fi
%
%\iffalse
%<*samplechap1>
%\fi
% Some text for chapter 1:
%    \begin{macrocode}
\section{one}
some text in chapter one
%    \end{macrocode}

%\iffalse
%</samplechap1>
%\fi
% Some text for chapter 2:
%\iffalse
%<*samplechap2>
%\fi
%    \begin{macrocode}
\section{two}
more text in chapter two
%    \end{macrocode}

%\iffalse
%</samplechap2>
%\fi
%
% %%%%%%%%%%%%%%%%%%%%%%%%%%%%%%%%%%%%%%
% \paragraph{Part Include Files.}
%
% The include files are called |cdocspt3.tex| and |cdocspt4.tex|.
%
%\iffalse
%<*samplepart3|samplepart4>
%\fi

% Optional override for |\version| flag:
%    \begin{macrocode}
%%\providecommand{\version}{final}
%    \end{macrocode}

% Include the main document:
%    \begin{macrocode}
\input{childdoc.def}
\childdocby{cdocsamp}
%    \end{macrocode}

%\iffalse
%</samplepart3|samplepart4>
%\fi
%
%\iffalse
%<*samplepart3>
%\fi
% Some text for part 3:
%    \begin{macrocode}
some text in part three
%    \end{macrocode}

%\iffalse
%</samplepart3>
%\fi
% Some text for part 4:
%\iffalse
%<*samplepart4>
%\fi
%    \begin{macrocode}
more text in part four
%    \end{macrocode}

%\iffalse
%</samplepart4>
%\fi
%
% %%%%%%%%%%%%%%%%%%%%%%%%%%%%%%%%%%%%%%
% \paragraph{Forwarding for a Complete Draft.}
%
% The following forwarding file |cdocsdrf.tex|
% compiles the main document in draft mode:
%\iffalse
%<*sampledraft>
%\fi
%    \begin{macrocode}
\def\version{draft}
\input{childdoc.def}
\childdocforward{cdocsamp}
%    \end{macrocode}

%\iffalse
%</sampledraft>
%\fi
%
% %%%%%%%%%%%%%%%%%%%%%%%%%%%%%%%%%%%%%%
% \paragraph{Forwarding for Final Version of the Chapters.}
%
% The following forwarding files |cdocsfn1.tex| and |cdocsfn2.tex|
% (with identical content)
% compile the final versions of the child documents
% |cdocsch1.tex| and |cdocsch2.tex|, respectively:
%\iffalse
%<*samplefinal>
%\fi
%    \begin{macrocode}
\def\version{final}
\input{childdoc.def}
\childdocforwardprefix[cdocsamp]{cdocsfn}{cdocsch}
%    \end{macrocode}

%\iffalse
%</samplefinal>
%\fi
%
% %%%%%%%%%%%%%%%%%%%%%%%%%%%%%%%%%%%%%%
% \paragraph{Command Line Processing.}
%
% The following three command lines generate the output files
% |cdocscld|, |cdocscl1| and |cdocscl2|
% which should be identical to
% |cdocsdrf|, |cdocsch1| and |cdocsfn2|, respectively:
% \begin{center}
% \begin{tabular}{l}
% |latex -jobname cdocscld \|\\
% |  "\def\version{draft}\input{childdoc.def}\childdocforward{cdocsamp}"|\\
% |latex -jobname cdocscl1 \|\\
% |  "\input{childdoc.def}\childdocforward[cdocsamp]{cdocsch1}"|\\
% |latex -jobname cdocscl2 \|\\
% |  "\def\version{final}\input{childdoc.def}\childdocforward{cdocsch2}"|
% \end{tabular}
% \end{center}
% Note that the trailing backslash on each first line
% merely continues the input to the second line
% (for convenient cut ant paste).
% Furthermore, the command |latex| can be replaced by any
% of its alternative versions such as |pdflatex|.
%
% %%%%%%%%%%%%%%%%%%%%%%%%%%%%%%%%%%%%%%%%%%%%%%%%%%%%%%%%%%%%%%%%%%%%%%%%%%%%%%
% %%%%%%%%%%%%%%%%%%%%%%%%%%%%%%%%%%%%%%%%%%%%%%%%%%%%%%%%%%%%%%%%%%%%%%%%%%%%%%
% \section{Implementation}
%\iffalse
%<*package>
%\fi
%
% This section describes the definitions file |childdoc.def|.

% The definitions cannot be loaded using |\usepackage| or |\RequirePackage|
% which has a mechanism to prevent loading a style file more than once.
% When loading the definitions by means of |\input|
% multiple instances have to be prevented manually:
%\iffalse
%This code needs to be before the `\ProvidesFile' directive
%which is defined at the beginning of this file.
%Therefore it is also placed there and commented out here.
%</package>
%<*discard>
%\fi
%    \begin{macrocode}
\ifdefined\childdocmain\endinput\fi
%    \end{macrocode}
%\iffalse
%</discard>
%<*package>
%\fi
%
% \macro{\ifchilddoc}
% \macro{\ifchilddocmanual}
% The conditional |\ifchilddoc| tells whether a
% child (true) or main (false) document is being compiled.
% The conditional |\ifchilddocmanual| tells whether
% the |\includeonly| mechanism is used (false) or
% the selection of child files must be performed manually (true).
% The definitions initialise to false:
%    \begin{macrocode}
\newif\ifchilddoc
\newif\ifchilddocmanual
%    \end{macrocode}

% \macro{\childdocname}
% \macro{\childdocjob}
% The macro |\childdocname| stores the name of the main document
% to be compiled. The macro |\childdocjob| stores the name of
% the document on which the \LaTeX{} compiler was originally invoked.
% The content of |\jobname| cannot be compared
% to filenames specified in the source due to different catcodes.
% The following code rescans |\jobname|, stores the result
% in |\childdocname| and saves a copy in |\childdocjob|:
%    \begin{macrocode}
\edef\childdocname{\scantokens\expandafter{\jobname\noexpand}}
\let\childdocjob\childdocname
%    \end{macrocode}

% \macro{\childdocdisable}
% The macro |\childdocdisable| prevents the main file
% from being processed more than once.
% At this stage, the main document command |\childdocmain|
% is assumed to be called once again where it should do nothing.
% Any subsequent call to it should prevent
% a secondary processing of the main document
% It overwrites the forwarding commands
% |\childdocof| and |\childdocforward|
% with empty macros to prevent further inclusions of the main document:
%    \begin{macrocode}
\newcommand{\childdocdisable}
{
  \renewcommand{\childdocmain}[1]{\renewcommand{\childdocmain}[1]{\endinput}}
  \renewcommand{\childdocof}[1]{}
  \renewcommand{\childdocby}[2][]{}
  \renewcommand{\childdocforward}[2][]{}
  \renewcommand{\childdocdisable}{}
}
%    \end{macrocode}

% \macro{\childdocmain}
% The macro |\childdocmain| is to be called at the top of the main file
% with nothing or the main filename (without extension) as argument.
% First, it breaks loops.
% If the argument is not empty and does not match |\childdocname|
% (which is set by the first inclusion of |childdoc.def|),
% |\ifchilddoc| is set to true, |\includeonly| is applied to the child file
% and |\jobname| is set to the main file
% (for proper handling of |.aux| files):
%    \begin{macrocode}
\newcommand{\childdocmain}[1]
{
  \childdocdisable\childdocmain{}
  \if?#1?\else
    \begingroup
      \def\childdoctmp{#1}
      \ifx\childdoctmp\childdocname
        \def\childdoctmp{}
      \else
        \def\childdoctmp
        {
          \childdoctrue
          \includeonly{\childdocname}
          \def\childdocjob{#1}
          \def\jobname{#1}
        }
      \fi
      \expandafter
    \endgroup
    \childdoctmp
  \fi
}
%    \end{macrocode}

% \macro{\childdocof}
% The command |\childdocof| redirects
% compilation to the main file |#1|.
%    \begin{macrocode}
\newcommand{\childdocof}[1]
{
  \childdocdisable
  \childdoctrue
  \includeonly{\childdocname}
  \def\jobname{#1}
  \def\childdocjob{#1}
  \input{#1}
}
%    \end{macrocode}

% \macro{\childdocby}
% The command |\childdocby| ....
%    \begin{macrocode}
\newcommand{\childdocby}[2][]
{
  \childdocdisable
  \childdoctrue
  \childdocmanualtrue
  \if?#1?\else
    \def\jobname{#2}
  \fi
  \def\childdocjob{#2}
  \input{#2}
  \endinput
}
%    \end{macrocode}

% \macro{\childdocforward}
% The command |\childdocforward| redirects
% compilation to the main file or
% (if the optional argument is given) a child file.
% Parameters are set as if the main file
% or a child file starting with |\childdocof| was compiled.
% Then compilation is handed over to the main file:
%    \begin{macrocode}
\newcommand{\childdocforward}[2][]
{
  \begingroup
    \if?#1?
      \def\childdoctmp
      {
        \def\childdocname{#2}
        \def\childdocjob{#2}
        \def\jobname{#2}
        \input{#2}
        \endinput
      }
    \else
      \def\childdoctmp
      {
        \childdocdisable
        \def\childdocname{#2}
        \childdoctrue
        \includeonly{#2}
        \def\childdocjob{#1}
        \def\jobname{#1}
        \input{#1}
        \endinput
      }
    \fi
    \expandafter
  \endgroup
  \childdoctmp
}
%    \end{macrocode}

% \macro{\childdocforwardprefix}
% The command |\childdocforwardprefix| redirects
% compilation to the main or a child file by means of a pattern.
% The prefix |#1| in the current filename is replaced by |#2|
% and the suffix of the current filename is kept
% (it is assumed that the filename does not contain the substring `|~~~|'
% which is used as a delimiter).
% Compilation is handed over to the new file by |\childdocforward|:
%    \begin{macrocode}
\newcommand{\childdocforwardprefix}[3][]
{
  \begingroup
    \def\childdocextract #2##1~~~{\def\childdoctmp{\childdocforward[#1]{#3##1}}}
    \expandafter\childdocextract\childdocname~~~
    \expandafter
  \endgroup
  \childdoctmp
}
%    \end{macrocode}

% \macro{\childdoc}
% The deprecated macro |\childdoc| is a legacy version of |\childdocmain|:
%    \begin{macrocode}
\newcommand{\childdoc}{\childdocmain}
%    \end{macrocode}

% \macro{\childdocredirect}
% The deprecated macro |\childdocredirect| is a legacy version
% of |\childdocforward| and |\childdocforwardprefix|:
%    \begin{macrocode}
\newcommand{\childdocredirect}[2][]
{
  \begingroup
    \if?#1?
      \def\childdoctmp{\childdocforward{#2}}
    \else
      \def\childdoctmp{\childdocforwardprefix{#1}{#2}}
    \fi
    \expandafter
  \endgroup
  \childdoctmp
}
%    \end{macrocode}

%\iffalse
%</package>
%\fi
%
\endinput

\childdocof{cdocsamp}
%    \end{macrocode}

%\iffalse
%</samplechap1|samplechap2>
%\fi
%
%\iffalse
%<*samplechap1>
%\fi
% Some text for chapter 1:
%    \begin{macrocode}
\section{one}
some text in chapter one
%    \end{macrocode}

%\iffalse
%</samplechap1>
%\fi
% Some text for chapter 2:
%\iffalse
%<*samplechap2>
%\fi
%    \begin{macrocode}
\section{two}
more text in chapter two
%    \end{macrocode}

%\iffalse
%</samplechap2>
%\fi
%
% %%%%%%%%%%%%%%%%%%%%%%%%%%%%%%%%%%%%%%
% \paragraph{Part Include Files.}
%
% The include files are called |cdocspt3.tex| and |cdocspt4.tex|.
%
%\iffalse
%<*samplepart3|samplepart4>
%\fi

% Optional override for |\version| flag:
%    \begin{macrocode}
%%\providecommand{\version}{final}
%    \end{macrocode}

% Include the main document:
%    \begin{macrocode}
% \iffalse
%
% childdoc.dtx Copyright (C) 2017-2018 Niklas Beisert
%
% This work may be distributed and/or modified under the
% conditions of the LaTeX Project Public License, either version 1.3
% of this license or (at your option) any later version.
% The latest version of this license is in
%   http://www.latex-project.org/lppl.txt
% and version 1.3 or later is part of all distributions of LaTeX
% version 2005/12/01 or later.
%
% This work has the LPPL maintenance status `maintained'.
%
% The Current Maintainer of this work is Niklas Beisert.
%
% This work consists of the files childdoc.dtx and childdoc.ins
% and the derived files childdoc.def and cdocsamp.tex with
% cdocsch1.tex, cdocsch2.tex, cdocsdrf.tex, cdocsfn1.tex, cdocsfn2.tex.
%
%<package>\ifdefined\childdocmain\endinput\fi
%<package>\ProvidesFile{childdoc.def}[2018/12/30 v2.0 child document driver]
%<samplemain>\ProvidesFile{cdocsamp.tex}[2018/12/30 v2.0 sample for childdoc]
%<*driver>
%\ProvidesFile{childdoc.drv}[2018/12/30 v2.0 childdoc reference manual file]
\PassOptionsToClass{10pt,a4paper}{article}
\documentclass{ltxdoc}

\usepackage[margin=35mm]{geometry}
\usepackage{hyperref}
\usepackage{hyperxmp}
\usepackage[usenames]{color}

\hypersetup{colorlinks=true}
\hypersetup{pdfstartview=FitH}
\hypersetup{pdfpagemode=UseNone}
\hypersetup{pdfsource={}}
\hypersetup{pdflang={en-UK}}
\hypersetup{pdfcopyright={Copyright 2017-2018 Niklas Beisert.
  This work may be distributed and/or modified under the
  conditions of the LaTeX Project Public License, either version 1.3
  of this license or (at your option) any later version.}}
\hypersetup{pdflicenseurl={http://www.latex-project.org/lppl.txt}}
\hypersetup{pdfcontactaddress={ETH Zurich, ITP, HIT K,
  Wolfgang-Pauli-Strasse 27}}
\hypersetup{pdfcontactpostcode={8093}}
\hypersetup{pdfcontactcity={Zurich}}
\hypersetup{pdfcontactcountry={Switzerland}}
\hypersetup{pdfcontactemail={nbeisert@itp.phys.ethz.ch}}
\hypersetup{pdfcontacturl={http://people.phys.ethz.ch/\xmptilde nbeisert/}}

\newcommand{\secref}[1]{\hyperref[#1]{section \ref*{#1}}}

\parskip1ex
\parindent0pt
\let\olditemize\itemize
\def\itemize{\olditemize\parskip0pt}

\begin{document}

\title{The \textsf{childdoc} Package}
\hypersetup{pdftitle={The childdoc Package}}
\author{Niklas Beisert\\[2ex]
  Institut f\"ur Theoretische Physik\\
  Eidgen\"ossische Technische Hochschule Z\"urich\\
  Wolfgang-Pauli-Strasse 27, 8093 Z\"urich, Switzerland\\[1ex]
  \href{mailto:nbeisert@itp.phys.ethz.ch}
  {\texttt{nbeisert@itp.phys.ethz.ch}}}
\hypersetup{pdfauthor={Niklas Beisert}}
\hypersetup{pdfsubject={Manual for the LaTeX2e Package childdoc}}
\date{30 December 2018, \textsf{v2.0}}
\maketitle

\begin{abstract}\noindent
\textsf{childdoc} is a \LaTeXe{} package
that enables the direct compilation
of document sections included by |\include|
to individual files.
\end{abstract}

\begingroup
\parskip0ex
\tableofcontents
\endgroup

%%%%%%%%%%%%%%%%%%%%%%%%%%%%%%%%%%%%%%%%%%%%%%%%%%%%%%%%%%%%%%%%%%%%%%%%%%%%%%%%
%%%%%%%%%%%%%%%%%%%%%%%%%%%%%%%%%%%%%%%%%%%%%%%%%%%%%%%%%%%%%%%%%%%%%%%%%%%%%%%%
\section{Introduction}

\LaTeX{} provides a mechanism to structure a large document (such as a book)
into a main file and several child files (containing the chapters)
using the |\include| command.
This mechanism is beneficial for documents
which span hundreds of pages in order to
make the source file(s) more manageable.
Moreover, compilation can be restricted to
selected child files by means of the |\includeonly| command.
The latter feature can be used to reduce the compilation time while editing
(this was significantly more useful in the earlier days of \LaTeX{})
or to generate a smaller document which is easier to navigate.
Another application of |\includeonly| is to generate
documents consisting of selected parts of the complete document.

However, there are a few drawbacks of the plain |\include| mechanism:
\begin{itemize}
\item
The child files cannot be compiled on their own,
they can only be compiled via the main file.
A naive editing environment
(such as a text editor with an option
to have the current file processed by \LaTeX)
may require one to switch to the main file before compiling;
attempting to compile the child file produces errors.
\item
The main file must be modified (each time)
to adjust the |\includeonly| command
to the present needs. This easily leaves the main file in a messy state.
\item
The generated document will always carry the filename
of the main document. This is inconvenient if
several child files are to be compiled and
to be kept for distribution.
\end{itemize}

The present package provides a simple interface
to make child files individually compilable by \LaTeX{}.
Compiling a child file then has the same effect as compiling
the main file with an |\includeonly| command
to select the appropriate child.
Moreover the generated document will carry the name of the child
rather than the main file.
This resolves all three above issues.

This feature is meant to make the editing of books,
thesis documents and lecture notes somewhat more convenient.
However, the package can also be used efficiently for
composing a series of documents (such as exercise sheets)
which are typically distributed individually.
It then assists the author in generating the individual documents
(potentially in different versions)
as well as a document containing the collected series.
Another application is in developing style files
or other kinds of included material
where compilation of the style file could redirect
to a sample or test file.

%%%%%%%%%%%%%%%%%%%%%%%%%%%%%%%%%%%%%%%%%%%%%%%%%%%%%%%%%%%%%%%%%%%%%%%%%%%%%%%%
%%%%%%%%%%%%%%%%%%%%%%%%%%%%%%%%%%%%%%%%%%%%%%%%%%%%%%%%%%%%%%%%%%%%%%%%%%%%%%%%
\section{Usage}

First of all, the package \textsf{childdoc} is \emph{not} a standard
\LaTeXe{} |.sty| style file! Therefore it needs to be invoked in
a non-standard way.

%%%%%%%%%%%%%%%%%%%%%%%%%%%%%%%%%%%%%%%%%%%%%%%%%%%%%%%%%%%%%%%%%%%%%%%%%%%%%%%%
\subsection{Included Files}
\label{sec:include}

%%%%%%%%%%%%%%%%%%%%%%%%%%%%%%%%%%%%%%%%
\DescribeMacro{\childdocmain}
To use the package, add the commands
\begin{center}
\begin{tabular}{l}
|\input{childdoc.def}|\\
|\childdocmain{}|\\
\end{tabular}
\end{center}
at the very top of the main \LaTeX{} file,
in particular \emph{before} the |\documentclass| statement!
The argument of |\childdocmain| should be left empty
(but it must be present).

%%%%%%%%%%%%%%%%%%%%%%%%%%%%%%%%%%%%%%%%
\DescribeMacro{\childdocof}
Furthermore, add the commands
\begin{center}
\begin{tabular}{l}
|\input{childdoc.def}|\\
|\childdocof{|\textit{main}|}|\\
\end{tabular}
\end{center}
at the top of every child file \textit{child}
which is included by |\include{|\textit{child}|}|
from within the main file
(or at least for those files to be compiled individually).
The argument \textit{main} must be the filename of the main file.

There are a couple of
considerations in setting up the main and child documents:

%%%%%%%%%%%%%%%%%%%%%%%%%%%%%%%%%%%%%%%%
\paragraph{Restrictions.}

Please note the following restrictions:
\begin{itemize}
\item
|\childdocmain| must be called with one argument \textit{main}
to ensure compatibility with earlier version of the package.
It must either be empty (|\childdocmain{}|)
or precisely match the filename of the main file in which it is specified.
See \secref{sec:detection} for further information.
\item
The filename \textit{main} must be specified without the |.tex| extension.
\item
The filename \textit{main} is case sensitive
(even in case-insensitive file systems)
due to internal string comparison.
\item
The argument \textit{main} should be fully expanded, it cannot be a macro.
\item
Subdirectories and special characters should be avoided in filenames.
\item
The command |\childdocmain{|\textit{main}|}| must be followed by a whitespace.
It should not be followed immediately by another command
or by a comment mark `|%|'.
This is because the \TeX{} parser reads the token immediately following
the argument of |\childdocmain| and puts it
at the beginning of every child section;
however, a white\-space is ignored.
\end{itemize}

%%%%%%%%%%%%%%%%%%%%%%%%%%%%%%%%%%%%%%%%
\paragraph{Content of Main File.}

It is advisable to place all content in the child files included by |\include|.
Any output contained in the main file will appear in all child documents
unless suppressed manually;
it cannot be suppressed automatically by the |\includeonly| directive
and thus should normally be avoided.
A method to include some content in the main file
by means of conditional processing is described in \secref{sec:conditional}.

%%%%%%%%%%%%%%%%%%%%%%%%%%%%%%%%%%%%%%%%
\paragraph{Page Numbering.}

When only a part of the document is compiled,
the appropriate numbering of pages
(as well as other status parameters)
is determined from the |.aux| files.
The latter contain information from previous passes.
However this information needs to propagate through
all intermediate child documents.
Therefore the page numbering in child documents may well
be inconsistent until the complete document is compiled at least once.

A useful (if unconventional) way to always ensure a consistent
page numbering is to restart the numbering in each child document
and denote the pages by `\textit{child}|.|\textit{page}'
where \textit{child} represents the chapter/section number of the child file.
This can be achieved by the command
|\numberwithin{page}{|\textit{child}|}|
of the \textsf{amsmath} package
where \textit{child} can be |chapter| or |section|
depending on the chosen structuring.
Alternatively, one can modify the macro |\thepage| appropriately
and reset the counter |page| at the start of each child file.

%%%%%%%%%%%%%%%%%%%%%%%%%%%%%%%%%%%%%%%%%%%%%%%%%%%%%%%%%%%%%%%%%%%%%%%%%%%%%%%%
\subsection{Conditional Processing}
\label{sec:conditional}

The package provides a mechanism to compile different versions
of a document. To customise the versions further some conditional processing
can come in handy to distinguish which version is being compiled.
The package provides two macros to describe the compilation context:

%%%%%%%%%%%%%%%%%%%%%%%%%%%%%%%%%%%%%%%%
\DescribeMacro{\ifchilddoc}
The conditional |\ifchilddoc| distinguishes between the compilation of
child documents and the main document:
%
\begin{center}
|\ifchilddoc |\textit{child-code}| |[|\||else |\textit{main-code}]| \||fi|
\end{center}

%%%%%%%%%%%%%%%%%%%%%%%%%%%%%%%%%%%%%%%%
\DescribeMacro{\childdocname}
\DescribeMacro{\childdocjob}
The macro |\childdocname| contains the filename (without extension)
of the main or child file being processed.
Note that |\childdocjob| will always contain the name of the main file.

%%%%%%%%%%%%%%%%%%%%%%%%%%%%%%%%%%%%%%%%
\paragraph{Title Page.}

Conditional processing can be used to include a title or banner page
in the main document when proper precautions are taken.
Importantly, the code in the main file should ensure that the page counter
(as well as other status parameters which are stored in the |.aux| files)
takes the same value after the conditional processing.
Otherwise the page numbers may take divergent values
depending on which part is compiled.

For example, a title page could be declared by:
%
\begin{center}
\begin{tabular}{l}
|\ifchilddoc\||else|\\
|\addtocounter{page}{-1}|\\
\textit{code for title page}\\
|\newpage|\\
|\||fi|
\end{tabular}
\end{center}
%
A banner page for the child documents can be generated by:
%
\begin{center}
\begin{tabular}{l}
|\ifchilddoc|\\
|\addtocounter{page}{-1}|\\
\textit{code for banner page}\\
|\newpage|\\
|\||fi|
\end{tabular}
\end{center}
%
Here one could write a message such as:
\begin{center}
|This is the part \childdocname{} of \childdocjob{}.|
\end{center}

%%%%%%%%%%%%%%%%%%%%%%%%%%%%%%%%%%%%%%%%%%%%%%%%%%%%%%%%%%%%%%%%%%%%%%%%%%%%%%%%
\subsection{Flags}
\label{sec:flags}

The package makes it easy to generate different versions
of the main or child documents.
To this end compilation flags can be defined
and assigned different default values.
They will be particularly useful in conjunction
with the forwarding mechanism described in \secref{sec:forward}.

For example, it may be useful to have a flag |\version|
which can be set to |draft| or |final|.
The document source will contain some conditional code
depending on the value of |\version|.
Suppose further, the flag should default to |final| for the main file
and to |draft| for child files
which is a natural assignment for editing the document.
This is achieved by placing the following code
in the preamble of the main document
(below the |\childdocmain| directive):
%
\begin{center}
\begin{tabular}{l}
|\ifchilddoc|\\
|\providecommand{\version}{draft}|\\
|\||else|\\
|\providecommand{\version}{final}|\\
|\||fi|
\end{tabular}
\end{center}
%
The definition by |\providecommand| makes sure
that previous definitions are not overwritten.
Further statements |\providecommand{\version}{...}|
can thus be added before the above code to override it.

For the main file, one might add a line
(between |\childdocmain| and the above block)
%
\begin{center}
|%\ifchilddoc\||else\providecommand{\version}{draft}\||fi|
\end{center}
%
which can be uncommented to produce a draft version.
Likewise one can add a line to the very top of a child file
(above the |\childdocof{|\textit{main}|}| directive)
%
\begin{center}
|%\providecommand{\version}{final}|
\end{center}
%
which can be uncommented to produce the final version of this child document.

%%%%%%%%%%%%%%%%%%%%%%%%%%%%%%%%%%%%%%%%%%%%%%%%%%%%%%%%%%%%%%%%%%%%%%%%%%%%%%%%
\subsection{Forwarding}
\label{sec:forward}

Different versions of the main or child documents
using compilation flags as described in \secref{sec:flags}
can be (permanently) stored in different files
for convenient compilation, viewing and distribution.
To this end, the package defines a command
to pass on compilation to a different file:

%%%%%%%%%%%%%%%%%%%%%%%%%%%%%%%%%%%%%%%%
\DescribeMacro{\childdocforward}
The command |\childdocforward| redirects processing to
another source file:
%
\begin{center}
\begin{tabular}{l}
|\input{childdoc.def}|\\
|\childdocforward[|\textit{main}|]{|\textit{dest}|}|\\
\end{tabular}
\end{center}
%
The argument \textit{dest} is the destination file
(without extension).
It should be the main file or one of the child files.
Note that further \textsf{childdoc} directives
such as |\childdocof| and |\childdocforward|
in the indicated file will be processed in this form.
The optional argument \textit{main}
passes on directly to the main file \textit{main}
while pretending to compile the child \textit{dest}.
This form behaves as if \textit{dest}
issues |\childdocof{|\textit{main}|}| right away,
and no further \textsf{childdoc} directives will be processed.

%%%%%%%%%%%%%%%%%%%%%%%%%%%%%%%%%%%%%%%%
\DescribeMacro{\...prefix}
In the alternative form |\childdocforwardprefix|,
%
\begin{center}
\begin{tabular}{l}
|\input{childdoc.def}|\\
|\childdocforwardprefix[|\textit{main}|]{|\textit{prefix}|}{|\textit{dest}|}|
\end{tabular}
\end{center}
%
the destination file is determined by a pattern
depending on the current file:
To make this work, the current file must be called
`{\textit{prefix}\hspace{0.2em}\textit{suffix}}'
with \textit{prefix} matching precisely the argument.
Processing is then passed on to the file
`{\textit{dest}\hspace{0.2em}\textit{suffix}}'.
Surely, the same effect is achieved by
directly specifying the
argument `{\textit{dest}\hspace{0.2em}\textit{suffix}}'
in the first form.
However, that requires to set up a different file
for each child. With the alternative form of the command
all these files can have exactly the same content
which simplifies setting them up and maintaining them.

For example, the following file |draft.tex|
with a compilation flag |\version| as described in \secref{sec:flags}
compiles the main document as a draft:
%
\begin{center}
\begin{tabular}{l}
|\def\version{draft}|\\
|\input{childdoc.def}|\\
|\childdocforward{|\textit{main}|}|
\end{tabular}
\end{center}
%
Likewise, the following files |final|\textit{nn}|.tex|
compile the final version of the child document
|child|\textit{nn}|.tex|:
%
\begin{center}
\begin{tabular}{l}
|\def\version{final}|\\
|\input{childdoc.def}|\\
|\childdocforwardprefix{final}{child}|
\end{tabular}
\end{center}
%

Note that when several versions of a main file and/or of each child file
are to be generated, it may be convenient to set up a |Makefile| or
shell script to automatise the process.

%%%%%%%%%%%%%%%%%%%%%%%%%%%%%%%%%%%%%%%%%%%%%%%%%%%%%%%%%%%%%%%%%%%%%%%%%%%%%%%%
\subsection{Command Line Processing}
\label{sec:commandline}

The effect of redirection files can also be achieved by invoking
the \LaTeX{} compiler with a more elaborate command line.
Most conveniently this should be done as part
of a shell script or a |Makefile|.

When using \textsf{childdoc} in the main file, the following
command lines effectively perform a redirection
(note that depending on the shell being used,
backslashes may have to be doubled: `|\|' $\to$ `|\\|'):
%
\begin{center}
|... -jobname "|\textit{target}|" |\\|"|[\textit{flags}]%
|\input{childdoc.def}\childdocforward[|\textit{main}|]{|\textit{dest}|}"|
\end{center}
%
Here \textit{target} is the name of the output file,
\textit{main} is the name of the main file
and \textit{dest} is the name of the main or child file to be processed
(all filenames without extensions).
The optional argument \textit{main} can be omitted
if \textit{main} matches \textit{dest}.
Optionally, compilation \textit{flags} can be defined via |\def| commands.
This command line makes the \TeX{} engine believe
it is compiling the file \textit{target}
whose content is specified as the latter parameter.
The provided code then forwards the processing to
\textit{main} or \textit{dest} as described in \secref{sec:forward}.

%%%%%%%%%%%%%%%%%%%%%%%%%%%%%%%%%%%%%%%%%%%%%%%%%%%%%%%%%%%%%%%%%%%%%%%%%%%%%%%%
\subsection{Include by Input}
\label{sec:input}

Including child documents by |\include| has some restrictions by design.
Most notably, the content of a child document always occupies
its own set of pages; pages cannot be shared between child documents.
Usually, this behaviour makes perfect sense
because each child document contain an essential part of the document.
However, in some situations it may be desirable to compose
a document from a collection of parts
without having mandatory page breaks between then.
For this case, the package
provides a mechanism to include parts
by |\input| which can also be processed individually.
However, by construction this mechanism
requires manual handling of the content to be output.

%%%%%%%%%%%%%%%%%%%%%%%%%%%%%%%%%%%%%%%%
\DescribeMacro{\ifchilddocmanual}
The main file should be prepared as usual, see \secref{sec:include}.
However, the document body must make a distinction
between processing of an individual part and of the main document, e.g.:
%
\begin{center}
\begin{tabular}{l}
|\ifchilddocmanual|\\
|\input{\childdocname}|\\
|\||else|\\
\textit{document body with }|\input{|\textit{part}|}|\\
|\||fi|
\end{tabular}
\end{center}
%
The conditional |\ifchilddocmanual| is true whenever
a part to be included by |\input| is being compiled,
and the name of the part is stored in |\childdocname|.

%%%%%%%%%%%%%%%%%%%%%%%%%%%%%%%%%%%%%%%%
\DescribeMacro{\childdocby}
Each part to be included by |\input| should start with:
%
\begin{center}
\begin{tabular}{l}
|\input{childdoc.def}|\\
|\childdocby{|\textit{main}|}|\\
\end{tabular}
\end{center}
%
The directive |\childdocby| is similar to |\childdocof|
described in \secref{sec:include},
but the subsequent selection of content must be done manually.
To that end, both |\ifchilddoc| and |\ifchilddocmanual|
will be true upon processing of a part,
and the name of the part is stored in |\childdocname|.
Note that |\jobname| will be set to the filename of the current part
so that each part receives an individual |.aux| file
that does not interfere with the |.aux| file(s) of the main document.
This behaviour can be altered by the alternative form
|\childdocby[*]{|\textit{main}|}| (with a non-empty optional argument)
which uses the |.aux| file of the main document
by setting |\jobname| to \textit{main}.

%%%%%%%%%%%%%%%%%%%%%%%%%%%%%%%%%%%%%%%%%%%%%%%%%%%%%%%%%%%%%%%%%%%%%%%%%%%%%%%%
\subsection{Driver Development}
\label{sec:driver}

The \textsf{childdoc} mechanism can also be use for the development
of definition files such as \LaTeX{} styles or classes.
This case differs from the above setup with multiple parts
included by |\include| in that no |\includeonly| should be invoked.
This can be achieved by starting the include file
(before |\ProvidesPackage|) with:
%
\begin{center}
\begin{tabular}{l}
|\input{childdoc.def}|\\
|\childdocforward{|\textit{main}|}|\\
\end{tabular}
\end{center}
%
or alternatively with:
%
\begin{center}
\begin{tabular}{l}
|\input{childdoc.def}|\\
|\childdocby{|\textit{main}|}|\\
\end{tabular}
\end{center}
%
Both forms have slightly different effects as described above.
The main file is prepared as usual, see \secref{sec:include}.

%%%%%%%%%%%%%%%%%%%%%%%%%%%%%%%%%%%%%%%%%%%%%%%%%%%%%%%%%%%%%%%%%%%%%%%%%%%%%%%%
\subsection{Legacy Detection}
\label{sec:detection}

The directive |\childdocmain| in the main file can detect
whether the complete document or merely a child is to be compiled
even without using the directive |\childdocof|.
This method is deprecated because it is less robust
and there is no compelling reason to use it;
it is merely provided for backward compatibility
and it may be removed in future versions.

If the detection mechanism is to be used,
it is mandatory to correctly specify
the filename of the main file as the argument of |\childdocmain|:
%
\begin{center}
\begin{tabular}{l}
|\input{childdoc.def}|\\
|\childdocmain{|\textit{main}|}|\\
\end{tabular}
\end{center}
%
If |\jobname| does not match the argument \textit{main} of |\childdocmain|,
it is assumed that |\jobname| points to the child file to be compiled.
When using |\childdocmain| with the main file specified as argument,
it suffices to start a child file
with just |\input{|\textit{main}|}|
without loading of the package and using |\childdocof|.
If instead all processing is done
with the appropriate \textsf{childdoc} directives,
the argument of \textit{main} of |\childdocmain| can be empty.

An alternative version of the command line processing described
in \secref{sec:commandline} using the detection mechanism reads:
%
\begin{center}
|... -jobname "|\textit{target}|" "|[\textit{flags}]%
[|\def\jobname{|\textit{dest}|}|]|\input{|\textit{main}|}"|
\end{center}

%%%%%%%%%%%%%%%%%%%%%%%%%%%%%%%%%%%%%%%%%%%%%%%%%%%%%%%%%%%%%%%%%%%%%%%%%%%%%%%%
\subsection{Manual Code}
\label{sec:manual}

In case one cannot be certain whether the definitions file |childdoc.def|
is installed on the target \TeX{} distribution
and one prefers not to ship it,
it is conceivable to paste a few relevant commands into the sources.

To that end, drop all statements |\input{childdoc.def}|
and perform the replacements as outlined below.
Instead of |\childdocmain{|\textit{main}|}| add the following code
to the top of the main file:
%
\begin{center}
\begin{tabular}{l}
|\||ifdefined\childdocname\endinput\||fi\newif\ifchilddoc|\\
|\edef\childdocname{\scantokens\expandafter{\jobname\noexpand}}|\\
|\def\childdocmain{|\textit{main}|}\||ifx\childdocmain\childdocname\||else|\\
|\childdoctrue\includeonly{\childdocname}\let\jobname\childdocmain\||fi|\\
\end{tabular}
\end{center}
%
Instead of |\childdocof{|\textit{main}|}| just include the main file
at the top of each child file:
%
\begin{center}
|\input{|\textit{main}|}|
\end{center}
%
A simple redirection |\childdocforward{|\textit{dest}|}| is achieved by:
%
\begin{center}
|\def\jobname{|\textit{dest}|}\input{\jobname}|
\end{center}
%
The redirection with prefix
|\childdocforwardprefix[|\textit{prefix}|]{|\textit{dest}|}|
is accomplished by:
%
\begin{center}
\begin{tabular}{l}
|{\edef\jobname{\scantokens\expandafter{\jobname\noexpand}}|\\
|\def\redirectjob |\textit{prefix}|#1~~~{\gdef\jobname{|\textit{dest}|#1}}|\\
|\expandafter\redirectjob\jobname~~~}\input{\jobname}|
\end{tabular}
\end{center}

In an alternative approach,
child documents can be compiled by a specific command line
without additional code or specific definitions:
%
\begin{center}
|... -jobname "|\textit{target}|" "|[\textit{flags}]%
|\includeonly{|\textit{dest}|}\input{|\textit{main}|}"|
\end{center}
%

%%%%%%%%%%%%%%%%%%%%%%%%%%%%%%%%%%%%%%%%%%%%%%%%%%%%%%%%%%%%%%%%%%%%%%%%%%%%%%%%
%%%%%%%%%%%%%%%%%%%%%%%%%%%%%%%%%%%%%%%%%%%%%%%%%%%%%%%%%%%%%%%%%%%%%%%%%%%%%%%%
\section{Information}

%%%%%%%%%%%%%%%%%%%%%%%%%%%%%%%%%%%%%%%%%%%%%%%%%%%%%%%%%%%%%%%%%%%%%%%%%%%%%%%%
\subsection{Copyright}

Copyright \copyright{} 2017--2018 Niklas Beisert

This work may be distributed and/or modified under the
conditions of the \LaTeX{} Project Public License, either version 1.3
of this license or (at your option) any later version.
The latest version of this license is in
  \url{http://www.latex-project.org/lppl.txt}
and version 1.3 or later is part of all distributions of \LaTeX{}
version 2005/12/01 or later.

This work has the LPPL maintenance status `maintained'.

The Current Maintainer of this work is Niklas Beisert.

This work consists of the files |README.txt|, |childdoc.ins| and |childdoc.dtx|
as well as the derived files |childdoc.def|, |cdocsamp.tex|
with |cdocsch1.tex|, |cdocsch2.tex|, |cdocspt3.tex|, |cdocspt4.tex|,
|cdocsdrf.tex|, |cdocsfn1.tex|, |cdocsfn2.tex|
as well as |childdoc.pdf|.

%%%%%%%%%%%%%%%%%%%%%%%%%%%%%%%%%%%%%%%%%%%%%%%%%%%%%%%%%%%%%%%%%%%%%%%%%%%%%%%%
\subsection{Files and Installation}

The package consists of the files:
%
\begin{center}
\begin{tabular}{ll}
    |README.txt|   & readme file \\
    |childdoc.ins| & installation file \\
    |childdoc.dtx| & source file \\
    |childdoc.def| & definition file \\
    |cdocsamp.tex| & sample main file \\
    |cdocsch1.tex| & sample include file \\
    |cdocsch2.tex| & sample include file \\
    |cdocspt3.tex| & sample part file \\
    |cdocspt4.tex| & sample part file \\
    |cdocsdrf.tex| & sample redirection file \\
    |cdocsfn1.tex| & sample redirection file \\
    |cdocsfn2.tex| & sample redirection file \\
    |childdoc.pdf| & manual
\end{tabular}
\end{center}
%
The distribution consists of the files
|README.txt|, |childdoc.ins| and |childdoc.dtx|.
%
\begin{itemize}
\item
Run (pdf)\LaTeX{} on |childdoc.dtx|
to compile the manual |childdoc.pdf| (this file).
\item
Run \LaTeX{} on |childdoc.ins| to create the definitions file |childdoc.def|
and the sample |cdocsamp.tex| with include files
|cdocsch1.tex|, |cdocsch2.tex|, |cdocspt3.tex|, |cdocspt4.tex|,
|cdocsdrf.tex|, |cdocsfn1.tex|, |cdocsfn2.tex|.
Then copy the file |childdoc.def| to an appropriate directory of your \LaTeX{}
distribution, e.g.\ \textit{texmf-root}|/tex/latex/childdoc|.
\end{itemize}

%%%%%%%%%%%%%%%%%%%%%%%%%%%%%%%%%%%%%%%%%%%%%%%%%%%%%%%%%%%%%%%%%%%%%%%%%%%%%%%%
\subsection{Related CTAN Packages}

There are several other packages which offer a similar functionality:
%
\begin{itemize}
\item
The packages
\href{http://ctan.org/pkg/docmute}{\textsf{docmute}},
\href{http://ctan.org/pkg/includex}{\textsf{includex}} and
\href{http://ctan.org/pkg/standalone}{\textsf{standalone}}
provide commands to include only the document body of
a child file thus allowing both files to be compiled individually.
\item
The packages \href{http://ctan.org/pkg/subdocs}{\textsf{subdocs}}
and \href{http://ctan.org/pkg/subfiles}{\textsf{subfiles}}
provide structures in which the main and child documents can be
encapsulated and allowing them to be compiled individually.
The inclusion mechanism is different from the conventional |\include|.
\item
The package \href{http://ctan.org/pkg/combine}{\textsf{combine}}
is an elaborate solution to combine several documents into one.
\end{itemize}
%
See also the CTAN topic \href{http://ctan.org/topic/subdocs}{\textsf{subdocs}}
for further related packages.
The present package differs from the above solutions in that
a document structure constructed with the conventional |\include| mechanism
just needs two extra commands at the top of every file
such that all constituent files can be compiled individually.

%%%%%%%%%%%%%%%%%%%%%%%%%%%%%%%%%%%%%%%%%%%%%%%%%%%%%%%%%%%%%%%%%%%%%%%%%%%%%%%%
%\subsection{Feature Suggestions}
%
%The following is a list of features which may be useful for future
%versions of this package:
%%
%\begin{itemize}
%\item
%\ldots
%\end{itemize}

%%%%%%%%%%%%%%%%%%%%%%%%%%%%%%%%%%%%%%%%%%%%%%%%%%%%%%%%%%%%%%%%%%%%%%%%%%%%%%%%
\subsection{Revision History}

%%%%%%%%%%%%%%%%%%%%%%%%%%%%%%%%%%%%%%%%
\paragraph{v2.0:} 2018/12/30

\begin{itemize}
\item
immediate forward processing
\item
added |\childdocby| mechanism
\item
manual restructured
\end{itemize}

%%%%%%%%%%%%%%%%%%%%%%%%%%%%%%%%%%%%%%%%
\paragraph{v1.6:} 2018/01/17

\begin{itemize}
\item
application for development of include files
\item
corrections to manual
\end{itemize}

%%%%%%%%%%%%%%%%%%%%%%%%%%%%%%%%%%%%%%%%
\paragraph{v1.5:} 2017/05/21

\begin{itemize}
\item
more complete structuring introduced
\item
|\childdocof| introduced
\item
|\childdoc| renamed to |\childdocmain|
\item
|\childredirect| renamed to |\childdocforward| and |\childdocforwardprefix|
and functionality expanded
\end{itemize}

%%%%%%%%%%%%%%%%%%%%%%%%%%%%%%%%%%%%%%%%
\paragraph{v1.0:} 2017/04/27

\begin{itemize}
\item
manual and install package
\item
first version published on CTAN
\end{itemize}

%%%%%%%%%%%%%%%%%%%%%%%%%%%%%%%%%%%%%%%%
\paragraph{v0.6:} 2017/04/26

\begin{itemize}
\item
redirection mechanism added
\end{itemize}

%%%%%%%%%%%%%%%%%%%%%%%%%%%%%%%%%%%%%%%%
\paragraph{v0.5:} 2017/04/26

\begin{itemize}
\item
functionality in definition file
\end{itemize}


%%%%%%%%%%%%%%%%%%%%%%%%%%%%%%%%%%%%%%%%%%%%%%%%%%%%%%%%%%%%%%%%%%%%%%%%%%%%%%%%
%%%%%%%%%%%%%%%%%%%%%%%%%%%%%%%%%%%%%%%%%%%%%%%%%%%%%%%%%%%%%%%%%%%%%%%%%%%%%%%%
%%%%%%%%%%%%%%%%%%%%%%%%%%%%%%%%%%%%%%%%%%%%%%%%%%%%%%%%%%%%%%%%%%%%%%%%%%%%%%%%
\appendix

\settowidth\MacroIndent{\rmfamily\scriptsize 000\ }

 \DocInput{childdoc.dtx}

\end{document}
%</driver>
% \fi
%
% %%%%%%%%%%%%%%%%%%%%%%%%%%%%%%%%%%%%%%%%%%%%%%%%%%%%%%%%%%%%%%%%%%%%%%%%%%%%%%
% %%%%%%%%%%%%%%%%%%%%%%%%%%%%%%%%%%%%%%%%%%%%%%%%%%%%%%%%%%%%%%%%%%%%%%%%%%%%%%
% \section{Sample}
%\iffalse
%<*samplemain>
%\fi
%
% The following presents a sample document
% with two chapters, two parts, a title page,
% a compile flag as well as three forwarding files to set the flag.
% It consists of eight |.tex| files:
% \begin{center}
% \begin{tabular}{ll}
% |cdocsamp.tex|&main file\\
% |cdocsch1.tex|&include file for chapter 1\\
% |cdocsch2.tex|&include file for chapter 2\\
% |cdocspt3.tex|&include file for part 3\\
% |cdocspt4.tex|&include file for part 4\\
% |cdocsdrf.tex|&forwarding file for main file in draft mode\\
% |cdocsfi1.tex|&forwarding file for final version of chapter 1\\
% |cdocsfi2.tex|&forwarding file for final version of chapter 2\\
% \end{tabular}
% \end{center}
% Each of the eight files can be compiled directly by the \LaTeX{} compiler.
%
% %%%%%%%%%%%%%%%%%%%%%%%%%%%%%%%%%%%%%%
% \paragraph{Main File.}
%
% The main file is called |cdocsamp.tex|.
%
% Load the \textsf{childdoc} definitions and
% declare the filename for the main document:
%    \begin{macrocode}
\input{childdoc.def}
\childdocmain{}
%    \end{macrocode}

% Optional override for |\version| flag:
%    \begin{macrocode}
%%\ifchilddoc\else\providecommand{\version}{draft}\fi
%    \end{macrocode}

% Define the default values for the |\version| flag
% (|final| for the main file and |draft| for childs):
%    \begin{macrocode}
\ifchilddoc
\providecommand{\version}{draft}
\else
\providecommand{\version}{final}
\fi
%    \end{macrocode}

% Load the standard document class:
%    \begin{macrocode}
\documentclass[12pt]{article}
%    \end{macrocode}

% Start the document body:
%    \begin{macrocode}
\begin{document}
%    \end{macrocode}

% Declare a title page.
% Print title, part of document being processed and version flag:
%    \begin{macrocode}
\addtocounter{page}{-1}
\begin{center}
{\LARGE\bfseries{}childdoc example\par}
\vspace{1cm}
\ifchilddoc
\ifchilddocmanual part\else chapter\fi:
`\childdocname' of `\childdocjob'\par
\else
main document: `\childdocjob'\par
\fi
version: \version\par
\end{center}
\newpage
%    \end{macrocode}

% Manually include selected file,
% otherwise process as usual:
%    \begin{macrocode}
\ifchilddocmanual
\section*{part `\childdocname'}
\input{\childdocname}
\else
%    \end{macrocode}

% Include the two chapters:
%    \begin{macrocode}
\include{cdocsch1}
\include{cdocsch2}
%    \end{macrocode}

% Include the two parts unless only chapters should be displayed:
%    \begin{macrocode}
\ifchilddoc\else
\section{part three}
\input{cdocspt3}
\section{part four}
\input{cdocspt4}
\fi
%    \end{macrocode}

% Process as usual until here:
%    \begin{macrocode}
\fi
%    \end{macrocode}

% End of document body:
%    \begin{macrocode}
\end{document}
%    \end{macrocode}
%\iffalse
%</samplemain>
%\fi
%
% %%%%%%%%%%%%%%%%%%%%%%%%%%%%%%%%%%%%%%
% \paragraph{Chapter Include Files.}
%
% The include files are called |cdocsch1.tex| and |cdocsch2.tex|.
%
%\iffalse
%<*samplechap1|samplechap2>
%\fi

% Optional override for |\version| flag:
%    \begin{macrocode}
%%\providecommand{\version}{final}
%    \end{macrocode}

% Include the main document:
%    \begin{macrocode}
\input{childdoc.def}
\childdocof{cdocsamp}
%    \end{macrocode}

%\iffalse
%</samplechap1|samplechap2>
%\fi
%
%\iffalse
%<*samplechap1>
%\fi
% Some text for chapter 1:
%    \begin{macrocode}
\section{one}
some text in chapter one
%    \end{macrocode}

%\iffalse
%</samplechap1>
%\fi
% Some text for chapter 2:
%\iffalse
%<*samplechap2>
%\fi
%    \begin{macrocode}
\section{two}
more text in chapter two
%    \end{macrocode}

%\iffalse
%</samplechap2>
%\fi
%
% %%%%%%%%%%%%%%%%%%%%%%%%%%%%%%%%%%%%%%
% \paragraph{Part Include Files.}
%
% The include files are called |cdocspt3.tex| and |cdocspt4.tex|.
%
%\iffalse
%<*samplepart3|samplepart4>
%\fi

% Optional override for |\version| flag:
%    \begin{macrocode}
%%\providecommand{\version}{final}
%    \end{macrocode}

% Include the main document:
%    \begin{macrocode}
\input{childdoc.def}
\childdocby{cdocsamp}
%    \end{macrocode}

%\iffalse
%</samplepart3|samplepart4>
%\fi
%
%\iffalse
%<*samplepart3>
%\fi
% Some text for part 3:
%    \begin{macrocode}
some text in part three
%    \end{macrocode}

%\iffalse
%</samplepart3>
%\fi
% Some text for part 4:
%\iffalse
%<*samplepart4>
%\fi
%    \begin{macrocode}
more text in part four
%    \end{macrocode}

%\iffalse
%</samplepart4>
%\fi
%
% %%%%%%%%%%%%%%%%%%%%%%%%%%%%%%%%%%%%%%
% \paragraph{Forwarding for a Complete Draft.}
%
% The following forwarding file |cdocsdrf.tex|
% compiles the main document in draft mode:
%\iffalse
%<*sampledraft>
%\fi
%    \begin{macrocode}
\def\version{draft}
\input{childdoc.def}
\childdocforward{cdocsamp}
%    \end{macrocode}

%\iffalse
%</sampledraft>
%\fi
%
% %%%%%%%%%%%%%%%%%%%%%%%%%%%%%%%%%%%%%%
% \paragraph{Forwarding for Final Version of the Chapters.}
%
% The following forwarding files |cdocsfn1.tex| and |cdocsfn2.tex|
% (with identical content)
% compile the final versions of the child documents
% |cdocsch1.tex| and |cdocsch2.tex|, respectively:
%\iffalse
%<*samplefinal>
%\fi
%    \begin{macrocode}
\def\version{final}
\input{childdoc.def}
\childdocforwardprefix[cdocsamp]{cdocsfn}{cdocsch}
%    \end{macrocode}

%\iffalse
%</samplefinal>
%\fi
%
% %%%%%%%%%%%%%%%%%%%%%%%%%%%%%%%%%%%%%%
% \paragraph{Command Line Processing.}
%
% The following three command lines generate the output files
% |cdocscld|, |cdocscl1| and |cdocscl2|
% which should be identical to
% |cdocsdrf|, |cdocsch1| and |cdocsfn2|, respectively:
% \begin{center}
% \begin{tabular}{l}
% |latex -jobname cdocscld \|\\
% |  "\def\version{draft}\input{childdoc.def}\childdocforward{cdocsamp}"|\\
% |latex -jobname cdocscl1 \|\\
% |  "\input{childdoc.def}\childdocforward[cdocsamp]{cdocsch1}"|\\
% |latex -jobname cdocscl2 \|\\
% |  "\def\version{final}\input{childdoc.def}\childdocforward{cdocsch2}"|
% \end{tabular}
% \end{center}
% Note that the trailing backslash on each first line
% merely continues the input to the second line
% (for convenient cut ant paste).
% Furthermore, the command |latex| can be replaced by any
% of its alternative versions such as |pdflatex|.
%
% %%%%%%%%%%%%%%%%%%%%%%%%%%%%%%%%%%%%%%%%%%%%%%%%%%%%%%%%%%%%%%%%%%%%%%%%%%%%%%
% %%%%%%%%%%%%%%%%%%%%%%%%%%%%%%%%%%%%%%%%%%%%%%%%%%%%%%%%%%%%%%%%%%%%%%%%%%%%%%
% \section{Implementation}
%\iffalse
%<*package>
%\fi
%
% This section describes the definitions file |childdoc.def|.

% The definitions cannot be loaded using |\usepackage| or |\RequirePackage|
% which has a mechanism to prevent loading a style file more than once.
% When loading the definitions by means of |\input|
% multiple instances have to be prevented manually:
%\iffalse
%This code needs to be before the `\ProvidesFile' directive
%which is defined at the beginning of this file.
%Therefore it is also placed there and commented out here.
%</package>
%<*discard>
%\fi
%    \begin{macrocode}
\ifdefined\childdocmain\endinput\fi
%    \end{macrocode}
%\iffalse
%</discard>
%<*package>
%\fi
%
% \macro{\ifchilddoc}
% \macro{\ifchilddocmanual}
% The conditional |\ifchilddoc| tells whether a
% child (true) or main (false) document is being compiled.
% The conditional |\ifchilddocmanual| tells whether
% the |\includeonly| mechanism is used (false) or
% the selection of child files must be performed manually (true).
% The definitions initialise to false:
%    \begin{macrocode}
\newif\ifchilddoc
\newif\ifchilddocmanual
%    \end{macrocode}

% \macro{\childdocname}
% \macro{\childdocjob}
% The macro |\childdocname| stores the name of the main document
% to be compiled. The macro |\childdocjob| stores the name of
% the document on which the \LaTeX{} compiler was originally invoked.
% The content of |\jobname| cannot be compared
% to filenames specified in the source due to different catcodes.
% The following code rescans |\jobname|, stores the result
% in |\childdocname| and saves a copy in |\childdocjob|:
%    \begin{macrocode}
\edef\childdocname{\scantokens\expandafter{\jobname\noexpand}}
\let\childdocjob\childdocname
%    \end{macrocode}

% \macro{\childdocdisable}
% The macro |\childdocdisable| prevents the main file
% from being processed more than once.
% At this stage, the main document command |\childdocmain|
% is assumed to be called once again where it should do nothing.
% Any subsequent call to it should prevent
% a secondary processing of the main document
% It overwrites the forwarding commands
% |\childdocof| and |\childdocforward|
% with empty macros to prevent further inclusions of the main document:
%    \begin{macrocode}
\newcommand{\childdocdisable}
{
  \renewcommand{\childdocmain}[1]{\renewcommand{\childdocmain}[1]{\endinput}}
  \renewcommand{\childdocof}[1]{}
  \renewcommand{\childdocby}[2][]{}
  \renewcommand{\childdocforward}[2][]{}
  \renewcommand{\childdocdisable}{}
}
%    \end{macrocode}

% \macro{\childdocmain}
% The macro |\childdocmain| is to be called at the top of the main file
% with nothing or the main filename (without extension) as argument.
% First, it breaks loops.
% If the argument is not empty and does not match |\childdocname|
% (which is set by the first inclusion of |childdoc.def|),
% |\ifchilddoc| is set to true, |\includeonly| is applied to the child file
% and |\jobname| is set to the main file
% (for proper handling of |.aux| files):
%    \begin{macrocode}
\newcommand{\childdocmain}[1]
{
  \childdocdisable\childdocmain{}
  \if?#1?\else
    \begingroup
      \def\childdoctmp{#1}
      \ifx\childdoctmp\childdocname
        \def\childdoctmp{}
      \else
        \def\childdoctmp
        {
          \childdoctrue
          \includeonly{\childdocname}
          \def\childdocjob{#1}
          \def\jobname{#1}
        }
      \fi
      \expandafter
    \endgroup
    \childdoctmp
  \fi
}
%    \end{macrocode}

% \macro{\childdocof}
% The command |\childdocof| redirects
% compilation to the main file |#1|.
%    \begin{macrocode}
\newcommand{\childdocof}[1]
{
  \childdocdisable
  \childdoctrue
  \includeonly{\childdocname}
  \def\jobname{#1}
  \def\childdocjob{#1}
  \input{#1}
}
%    \end{macrocode}

% \macro{\childdocby}
% The command |\childdocby| ....
%    \begin{macrocode}
\newcommand{\childdocby}[2][]
{
  \childdocdisable
  \childdoctrue
  \childdocmanualtrue
  \if?#1?\else
    \def\jobname{#2}
  \fi
  \def\childdocjob{#2}
  \input{#2}
  \endinput
}
%    \end{macrocode}

% \macro{\childdocforward}
% The command |\childdocforward| redirects
% compilation to the main file or
% (if the optional argument is given) a child file.
% Parameters are set as if the main file
% or a child file starting with |\childdocof| was compiled.
% Then compilation is handed over to the main file:
%    \begin{macrocode}
\newcommand{\childdocforward}[2][]
{
  \begingroup
    \if?#1?
      \def\childdoctmp
      {
        \def\childdocname{#2}
        \def\childdocjob{#2}
        \def\jobname{#2}
        \input{#2}
        \endinput
      }
    \else
      \def\childdoctmp
      {
        \childdocdisable
        \def\childdocname{#2}
        \childdoctrue
        \includeonly{#2}
        \def\childdocjob{#1}
        \def\jobname{#1}
        \input{#1}
        \endinput
      }
    \fi
    \expandafter
  \endgroup
  \childdoctmp
}
%    \end{macrocode}

% \macro{\childdocforwardprefix}
% The command |\childdocforwardprefix| redirects
% compilation to the main or a child file by means of a pattern.
% The prefix |#1| in the current filename is replaced by |#2|
% and the suffix of the current filename is kept
% (it is assumed that the filename does not contain the substring `|~~~|'
% which is used as a delimiter).
% Compilation is handed over to the new file by |\childdocforward|:
%    \begin{macrocode}
\newcommand{\childdocforwardprefix}[3][]
{
  \begingroup
    \def\childdocextract #2##1~~~{\def\childdoctmp{\childdocforward[#1]{#3##1}}}
    \expandafter\childdocextract\childdocname~~~
    \expandafter
  \endgroup
  \childdoctmp
}
%    \end{macrocode}

% \macro{\childdoc}
% The deprecated macro |\childdoc| is a legacy version of |\childdocmain|:
%    \begin{macrocode}
\newcommand{\childdoc}{\childdocmain}
%    \end{macrocode}

% \macro{\childdocredirect}
% The deprecated macro |\childdocredirect| is a legacy version
% of |\childdocforward| and |\childdocforwardprefix|:
%    \begin{macrocode}
\newcommand{\childdocredirect}[2][]
{
  \begingroup
    \if?#1?
      \def\childdoctmp{\childdocforward{#2}}
    \else
      \def\childdoctmp{\childdocforwardprefix{#1}{#2}}
    \fi
    \expandafter
  \endgroup
  \childdoctmp
}
%    \end{macrocode}

%\iffalse
%</package>
%\fi
%
\endinput

\childdocby{cdocsamp}
%    \end{macrocode}

%\iffalse
%</samplepart3|samplepart4>
%\fi
%
%\iffalse
%<*samplepart3>
%\fi
% Some text for part 3:
%    \begin{macrocode}
some text in part three
%    \end{macrocode}

%\iffalse
%</samplepart3>
%\fi
% Some text for part 4:
%\iffalse
%<*samplepart4>
%\fi
%    \begin{macrocode}
more text in part four
%    \end{macrocode}

%\iffalse
%</samplepart4>
%\fi
%
% %%%%%%%%%%%%%%%%%%%%%%%%%%%%%%%%%%%%%%
% \paragraph{Forwarding for a Complete Draft.}
%
% The following forwarding file |cdocsdrf.tex|
% compiles the main document in draft mode:
%\iffalse
%<*sampledraft>
%\fi
%    \begin{macrocode}
\def\version{draft}
% \iffalse
%
% childdoc.dtx Copyright (C) 2017-2018 Niklas Beisert
%
% This work may be distributed and/or modified under the
% conditions of the LaTeX Project Public License, either version 1.3
% of this license or (at your option) any later version.
% The latest version of this license is in
%   http://www.latex-project.org/lppl.txt
% and version 1.3 or later is part of all distributions of LaTeX
% version 2005/12/01 or later.
%
% This work has the LPPL maintenance status `maintained'.
%
% The Current Maintainer of this work is Niklas Beisert.
%
% This work consists of the files childdoc.dtx and childdoc.ins
% and the derived files childdoc.def and cdocsamp.tex with
% cdocsch1.tex, cdocsch2.tex, cdocsdrf.tex, cdocsfn1.tex, cdocsfn2.tex.
%
%<package>\ifdefined\childdocmain\endinput\fi
%<package>\ProvidesFile{childdoc.def}[2018/12/30 v2.0 child document driver]
%<samplemain>\ProvidesFile{cdocsamp.tex}[2018/12/30 v2.0 sample for childdoc]
%<*driver>
%\ProvidesFile{childdoc.drv}[2018/12/30 v2.0 childdoc reference manual file]
\PassOptionsToClass{10pt,a4paper}{article}
\documentclass{ltxdoc}

\usepackage[margin=35mm]{geometry}
\usepackage{hyperref}
\usepackage{hyperxmp}
\usepackage[usenames]{color}

\hypersetup{colorlinks=true}
\hypersetup{pdfstartview=FitH}
\hypersetup{pdfpagemode=UseNone}
\hypersetup{pdfsource={}}
\hypersetup{pdflang={en-UK}}
\hypersetup{pdfcopyright={Copyright 2017-2018 Niklas Beisert.
  This work may be distributed and/or modified under the
  conditions of the LaTeX Project Public License, either version 1.3
  of this license or (at your option) any later version.}}
\hypersetup{pdflicenseurl={http://www.latex-project.org/lppl.txt}}
\hypersetup{pdfcontactaddress={ETH Zurich, ITP, HIT K,
  Wolfgang-Pauli-Strasse 27}}
\hypersetup{pdfcontactpostcode={8093}}
\hypersetup{pdfcontactcity={Zurich}}
\hypersetup{pdfcontactcountry={Switzerland}}
\hypersetup{pdfcontactemail={nbeisert@itp.phys.ethz.ch}}
\hypersetup{pdfcontacturl={http://people.phys.ethz.ch/\xmptilde nbeisert/}}

\newcommand{\secref}[1]{\hyperref[#1]{section \ref*{#1}}}

\parskip1ex
\parindent0pt
\let\olditemize\itemize
\def\itemize{\olditemize\parskip0pt}

\begin{document}

\title{The \textsf{childdoc} Package}
\hypersetup{pdftitle={The childdoc Package}}
\author{Niklas Beisert\\[2ex]
  Institut f\"ur Theoretische Physik\\
  Eidgen\"ossische Technische Hochschule Z\"urich\\
  Wolfgang-Pauli-Strasse 27, 8093 Z\"urich, Switzerland\\[1ex]
  \href{mailto:nbeisert@itp.phys.ethz.ch}
  {\texttt{nbeisert@itp.phys.ethz.ch}}}
\hypersetup{pdfauthor={Niklas Beisert}}
\hypersetup{pdfsubject={Manual for the LaTeX2e Package childdoc}}
\date{30 December 2018, \textsf{v2.0}}
\maketitle

\begin{abstract}\noindent
\textsf{childdoc} is a \LaTeXe{} package
that enables the direct compilation
of document sections included by |\include|
to individual files.
\end{abstract}

\begingroup
\parskip0ex
\tableofcontents
\endgroup

%%%%%%%%%%%%%%%%%%%%%%%%%%%%%%%%%%%%%%%%%%%%%%%%%%%%%%%%%%%%%%%%%%%%%%%%%%%%%%%%
%%%%%%%%%%%%%%%%%%%%%%%%%%%%%%%%%%%%%%%%%%%%%%%%%%%%%%%%%%%%%%%%%%%%%%%%%%%%%%%%
\section{Introduction}

\LaTeX{} provides a mechanism to structure a large document (such as a book)
into a main file and several child files (containing the chapters)
using the |\include| command.
This mechanism is beneficial for documents
which span hundreds of pages in order to
make the source file(s) more manageable.
Moreover, compilation can be restricted to
selected child files by means of the |\includeonly| command.
The latter feature can be used to reduce the compilation time while editing
(this was significantly more useful in the earlier days of \LaTeX{})
or to generate a smaller document which is easier to navigate.
Another application of |\includeonly| is to generate
documents consisting of selected parts of the complete document.

However, there are a few drawbacks of the plain |\include| mechanism:
\begin{itemize}
\item
The child files cannot be compiled on their own,
they can only be compiled via the main file.
A naive editing environment
(such as a text editor with an option
to have the current file processed by \LaTeX)
may require one to switch to the main file before compiling;
attempting to compile the child file produces errors.
\item
The main file must be modified (each time)
to adjust the |\includeonly| command
to the present needs. This easily leaves the main file in a messy state.
\item
The generated document will always carry the filename
of the main document. This is inconvenient if
several child files are to be compiled and
to be kept for distribution.
\end{itemize}

The present package provides a simple interface
to make child files individually compilable by \LaTeX{}.
Compiling a child file then has the same effect as compiling
the main file with an |\includeonly| command
to select the appropriate child.
Moreover the generated document will carry the name of the child
rather than the main file.
This resolves all three above issues.

This feature is meant to make the editing of books,
thesis documents and lecture notes somewhat more convenient.
However, the package can also be used efficiently for
composing a series of documents (such as exercise sheets)
which are typically distributed individually.
It then assists the author in generating the individual documents
(potentially in different versions)
as well as a document containing the collected series.
Another application is in developing style files
or other kinds of included material
where compilation of the style file could redirect
to a sample or test file.

%%%%%%%%%%%%%%%%%%%%%%%%%%%%%%%%%%%%%%%%%%%%%%%%%%%%%%%%%%%%%%%%%%%%%%%%%%%%%%%%
%%%%%%%%%%%%%%%%%%%%%%%%%%%%%%%%%%%%%%%%%%%%%%%%%%%%%%%%%%%%%%%%%%%%%%%%%%%%%%%%
\section{Usage}

First of all, the package \textsf{childdoc} is \emph{not} a standard
\LaTeXe{} |.sty| style file! Therefore it needs to be invoked in
a non-standard way.

%%%%%%%%%%%%%%%%%%%%%%%%%%%%%%%%%%%%%%%%%%%%%%%%%%%%%%%%%%%%%%%%%%%%%%%%%%%%%%%%
\subsection{Included Files}
\label{sec:include}

%%%%%%%%%%%%%%%%%%%%%%%%%%%%%%%%%%%%%%%%
\DescribeMacro{\childdocmain}
To use the package, add the commands
\begin{center}
\begin{tabular}{l}
|\input{childdoc.def}|\\
|\childdocmain{}|\\
\end{tabular}
\end{center}
at the very top of the main \LaTeX{} file,
in particular \emph{before} the |\documentclass| statement!
The argument of |\childdocmain| should be left empty
(but it must be present).

%%%%%%%%%%%%%%%%%%%%%%%%%%%%%%%%%%%%%%%%
\DescribeMacro{\childdocof}
Furthermore, add the commands
\begin{center}
\begin{tabular}{l}
|\input{childdoc.def}|\\
|\childdocof{|\textit{main}|}|\\
\end{tabular}
\end{center}
at the top of every child file \textit{child}
which is included by |\include{|\textit{child}|}|
from within the main file
(or at least for those files to be compiled individually).
The argument \textit{main} must be the filename of the main file.

There are a couple of
considerations in setting up the main and child documents:

%%%%%%%%%%%%%%%%%%%%%%%%%%%%%%%%%%%%%%%%
\paragraph{Restrictions.}

Please note the following restrictions:
\begin{itemize}
\item
|\childdocmain| must be called with one argument \textit{main}
to ensure compatibility with earlier version of the package.
It must either be empty (|\childdocmain{}|)
or precisely match the filename of the main file in which it is specified.
See \secref{sec:detection} for further information.
\item
The filename \textit{main} must be specified without the |.tex| extension.
\item
The filename \textit{main} is case sensitive
(even in case-insensitive file systems)
due to internal string comparison.
\item
The argument \textit{main} should be fully expanded, it cannot be a macro.
\item
Subdirectories and special characters should be avoided in filenames.
\item
The command |\childdocmain{|\textit{main}|}| must be followed by a whitespace.
It should not be followed immediately by another command
or by a comment mark `|%|'.
This is because the \TeX{} parser reads the token immediately following
the argument of |\childdocmain| and puts it
at the beginning of every child section;
however, a white\-space is ignored.
\end{itemize}

%%%%%%%%%%%%%%%%%%%%%%%%%%%%%%%%%%%%%%%%
\paragraph{Content of Main File.}

It is advisable to place all content in the child files included by |\include|.
Any output contained in the main file will appear in all child documents
unless suppressed manually;
it cannot be suppressed automatically by the |\includeonly| directive
and thus should normally be avoided.
A method to include some content in the main file
by means of conditional processing is described in \secref{sec:conditional}.

%%%%%%%%%%%%%%%%%%%%%%%%%%%%%%%%%%%%%%%%
\paragraph{Page Numbering.}

When only a part of the document is compiled,
the appropriate numbering of pages
(as well as other status parameters)
is determined from the |.aux| files.
The latter contain information from previous passes.
However this information needs to propagate through
all intermediate child documents.
Therefore the page numbering in child documents may well
be inconsistent until the complete document is compiled at least once.

A useful (if unconventional) way to always ensure a consistent
page numbering is to restart the numbering in each child document
and denote the pages by `\textit{child}|.|\textit{page}'
where \textit{child} represents the chapter/section number of the child file.
This can be achieved by the command
|\numberwithin{page}{|\textit{child}|}|
of the \textsf{amsmath} package
where \textit{child} can be |chapter| or |section|
depending on the chosen structuring.
Alternatively, one can modify the macro |\thepage| appropriately
and reset the counter |page| at the start of each child file.

%%%%%%%%%%%%%%%%%%%%%%%%%%%%%%%%%%%%%%%%%%%%%%%%%%%%%%%%%%%%%%%%%%%%%%%%%%%%%%%%
\subsection{Conditional Processing}
\label{sec:conditional}

The package provides a mechanism to compile different versions
of a document. To customise the versions further some conditional processing
can come in handy to distinguish which version is being compiled.
The package provides two macros to describe the compilation context:

%%%%%%%%%%%%%%%%%%%%%%%%%%%%%%%%%%%%%%%%
\DescribeMacro{\ifchilddoc}
The conditional |\ifchilddoc| distinguishes between the compilation of
child documents and the main document:
%
\begin{center}
|\ifchilddoc |\textit{child-code}| |[|\||else |\textit{main-code}]| \||fi|
\end{center}

%%%%%%%%%%%%%%%%%%%%%%%%%%%%%%%%%%%%%%%%
\DescribeMacro{\childdocname}
\DescribeMacro{\childdocjob}
The macro |\childdocname| contains the filename (without extension)
of the main or child file being processed.
Note that |\childdocjob| will always contain the name of the main file.

%%%%%%%%%%%%%%%%%%%%%%%%%%%%%%%%%%%%%%%%
\paragraph{Title Page.}

Conditional processing can be used to include a title or banner page
in the main document when proper precautions are taken.
Importantly, the code in the main file should ensure that the page counter
(as well as other status parameters which are stored in the |.aux| files)
takes the same value after the conditional processing.
Otherwise the page numbers may take divergent values
depending on which part is compiled.

For example, a title page could be declared by:
%
\begin{center}
\begin{tabular}{l}
|\ifchilddoc\||else|\\
|\addtocounter{page}{-1}|\\
\textit{code for title page}\\
|\newpage|\\
|\||fi|
\end{tabular}
\end{center}
%
A banner page for the child documents can be generated by:
%
\begin{center}
\begin{tabular}{l}
|\ifchilddoc|\\
|\addtocounter{page}{-1}|\\
\textit{code for banner page}\\
|\newpage|\\
|\||fi|
\end{tabular}
\end{center}
%
Here one could write a message such as:
\begin{center}
|This is the part \childdocname{} of \childdocjob{}.|
\end{center}

%%%%%%%%%%%%%%%%%%%%%%%%%%%%%%%%%%%%%%%%%%%%%%%%%%%%%%%%%%%%%%%%%%%%%%%%%%%%%%%%
\subsection{Flags}
\label{sec:flags}

The package makes it easy to generate different versions
of the main or child documents.
To this end compilation flags can be defined
and assigned different default values.
They will be particularly useful in conjunction
with the forwarding mechanism described in \secref{sec:forward}.

For example, it may be useful to have a flag |\version|
which can be set to |draft| or |final|.
The document source will contain some conditional code
depending on the value of |\version|.
Suppose further, the flag should default to |final| for the main file
and to |draft| for child files
which is a natural assignment for editing the document.
This is achieved by placing the following code
in the preamble of the main document
(below the |\childdocmain| directive):
%
\begin{center}
\begin{tabular}{l}
|\ifchilddoc|\\
|\providecommand{\version}{draft}|\\
|\||else|\\
|\providecommand{\version}{final}|\\
|\||fi|
\end{tabular}
\end{center}
%
The definition by |\providecommand| makes sure
that previous definitions are not overwritten.
Further statements |\providecommand{\version}{...}|
can thus be added before the above code to override it.

For the main file, one might add a line
(between |\childdocmain| and the above block)
%
\begin{center}
|%\ifchilddoc\||else\providecommand{\version}{draft}\||fi|
\end{center}
%
which can be uncommented to produce a draft version.
Likewise one can add a line to the very top of a child file
(above the |\childdocof{|\textit{main}|}| directive)
%
\begin{center}
|%\providecommand{\version}{final}|
\end{center}
%
which can be uncommented to produce the final version of this child document.

%%%%%%%%%%%%%%%%%%%%%%%%%%%%%%%%%%%%%%%%%%%%%%%%%%%%%%%%%%%%%%%%%%%%%%%%%%%%%%%%
\subsection{Forwarding}
\label{sec:forward}

Different versions of the main or child documents
using compilation flags as described in \secref{sec:flags}
can be (permanently) stored in different files
for convenient compilation, viewing and distribution.
To this end, the package defines a command
to pass on compilation to a different file:

%%%%%%%%%%%%%%%%%%%%%%%%%%%%%%%%%%%%%%%%
\DescribeMacro{\childdocforward}
The command |\childdocforward| redirects processing to
another source file:
%
\begin{center}
\begin{tabular}{l}
|\input{childdoc.def}|\\
|\childdocforward[|\textit{main}|]{|\textit{dest}|}|\\
\end{tabular}
\end{center}
%
The argument \textit{dest} is the destination file
(without extension).
It should be the main file or one of the child files.
Note that further \textsf{childdoc} directives
such as |\childdocof| and |\childdocforward|
in the indicated file will be processed in this form.
The optional argument \textit{main}
passes on directly to the main file \textit{main}
while pretending to compile the child \textit{dest}.
This form behaves as if \textit{dest}
issues |\childdocof{|\textit{main}|}| right away,
and no further \textsf{childdoc} directives will be processed.

%%%%%%%%%%%%%%%%%%%%%%%%%%%%%%%%%%%%%%%%
\DescribeMacro{\...prefix}
In the alternative form |\childdocforwardprefix|,
%
\begin{center}
\begin{tabular}{l}
|\input{childdoc.def}|\\
|\childdocforwardprefix[|\textit{main}|]{|\textit{prefix}|}{|\textit{dest}|}|
\end{tabular}
\end{center}
%
the destination file is determined by a pattern
depending on the current file:
To make this work, the current file must be called
`{\textit{prefix}\hspace{0.2em}\textit{suffix}}'
with \textit{prefix} matching precisely the argument.
Processing is then passed on to the file
`{\textit{dest}\hspace{0.2em}\textit{suffix}}'.
Surely, the same effect is achieved by
directly specifying the
argument `{\textit{dest}\hspace{0.2em}\textit{suffix}}'
in the first form.
However, that requires to set up a different file
for each child. With the alternative form of the command
all these files can have exactly the same content
which simplifies setting them up and maintaining them.

For example, the following file |draft.tex|
with a compilation flag |\version| as described in \secref{sec:flags}
compiles the main document as a draft:
%
\begin{center}
\begin{tabular}{l}
|\def\version{draft}|\\
|\input{childdoc.def}|\\
|\childdocforward{|\textit{main}|}|
\end{tabular}
\end{center}
%
Likewise, the following files |final|\textit{nn}|.tex|
compile the final version of the child document
|child|\textit{nn}|.tex|:
%
\begin{center}
\begin{tabular}{l}
|\def\version{final}|\\
|\input{childdoc.def}|\\
|\childdocforwardprefix{final}{child}|
\end{tabular}
\end{center}
%

Note that when several versions of a main file and/or of each child file
are to be generated, it may be convenient to set up a |Makefile| or
shell script to automatise the process.

%%%%%%%%%%%%%%%%%%%%%%%%%%%%%%%%%%%%%%%%%%%%%%%%%%%%%%%%%%%%%%%%%%%%%%%%%%%%%%%%
\subsection{Command Line Processing}
\label{sec:commandline}

The effect of redirection files can also be achieved by invoking
the \LaTeX{} compiler with a more elaborate command line.
Most conveniently this should be done as part
of a shell script or a |Makefile|.

When using \textsf{childdoc} in the main file, the following
command lines effectively perform a redirection
(note that depending on the shell being used,
backslashes may have to be doubled: `|\|' $\to$ `|\\|'):
%
\begin{center}
|... -jobname "|\textit{target}|" |\\|"|[\textit{flags}]%
|\input{childdoc.def}\childdocforward[|\textit{main}|]{|\textit{dest}|}"|
\end{center}
%
Here \textit{target} is the name of the output file,
\textit{main} is the name of the main file
and \textit{dest} is the name of the main or child file to be processed
(all filenames without extensions).
The optional argument \textit{main} can be omitted
if \textit{main} matches \textit{dest}.
Optionally, compilation \textit{flags} can be defined via |\def| commands.
This command line makes the \TeX{} engine believe
it is compiling the file \textit{target}
whose content is specified as the latter parameter.
The provided code then forwards the processing to
\textit{main} or \textit{dest} as described in \secref{sec:forward}.

%%%%%%%%%%%%%%%%%%%%%%%%%%%%%%%%%%%%%%%%%%%%%%%%%%%%%%%%%%%%%%%%%%%%%%%%%%%%%%%%
\subsection{Include by Input}
\label{sec:input}

Including child documents by |\include| has some restrictions by design.
Most notably, the content of a child document always occupies
its own set of pages; pages cannot be shared between child documents.
Usually, this behaviour makes perfect sense
because each child document contain an essential part of the document.
However, in some situations it may be desirable to compose
a document from a collection of parts
without having mandatory page breaks between then.
For this case, the package
provides a mechanism to include parts
by |\input| which can also be processed individually.
However, by construction this mechanism
requires manual handling of the content to be output.

%%%%%%%%%%%%%%%%%%%%%%%%%%%%%%%%%%%%%%%%
\DescribeMacro{\ifchilddocmanual}
The main file should be prepared as usual, see \secref{sec:include}.
However, the document body must make a distinction
between processing of an individual part and of the main document, e.g.:
%
\begin{center}
\begin{tabular}{l}
|\ifchilddocmanual|\\
|\input{\childdocname}|\\
|\||else|\\
\textit{document body with }|\input{|\textit{part}|}|\\
|\||fi|
\end{tabular}
\end{center}
%
The conditional |\ifchilddocmanual| is true whenever
a part to be included by |\input| is being compiled,
and the name of the part is stored in |\childdocname|.

%%%%%%%%%%%%%%%%%%%%%%%%%%%%%%%%%%%%%%%%
\DescribeMacro{\childdocby}
Each part to be included by |\input| should start with:
%
\begin{center}
\begin{tabular}{l}
|\input{childdoc.def}|\\
|\childdocby{|\textit{main}|}|\\
\end{tabular}
\end{center}
%
The directive |\childdocby| is similar to |\childdocof|
described in \secref{sec:include},
but the subsequent selection of content must be done manually.
To that end, both |\ifchilddoc| and |\ifchilddocmanual|
will be true upon processing of a part,
and the name of the part is stored in |\childdocname|.
Note that |\jobname| will be set to the filename of the current part
so that each part receives an individual |.aux| file
that does not interfere with the |.aux| file(s) of the main document.
This behaviour can be altered by the alternative form
|\childdocby[*]{|\textit{main}|}| (with a non-empty optional argument)
which uses the |.aux| file of the main document
by setting |\jobname| to \textit{main}.

%%%%%%%%%%%%%%%%%%%%%%%%%%%%%%%%%%%%%%%%%%%%%%%%%%%%%%%%%%%%%%%%%%%%%%%%%%%%%%%%
\subsection{Driver Development}
\label{sec:driver}

The \textsf{childdoc} mechanism can also be use for the development
of definition files such as \LaTeX{} styles or classes.
This case differs from the above setup with multiple parts
included by |\include| in that no |\includeonly| should be invoked.
This can be achieved by starting the include file
(before |\ProvidesPackage|) with:
%
\begin{center}
\begin{tabular}{l}
|\input{childdoc.def}|\\
|\childdocforward{|\textit{main}|}|\\
\end{tabular}
\end{center}
%
or alternatively with:
%
\begin{center}
\begin{tabular}{l}
|\input{childdoc.def}|\\
|\childdocby{|\textit{main}|}|\\
\end{tabular}
\end{center}
%
Both forms have slightly different effects as described above.
The main file is prepared as usual, see \secref{sec:include}.

%%%%%%%%%%%%%%%%%%%%%%%%%%%%%%%%%%%%%%%%%%%%%%%%%%%%%%%%%%%%%%%%%%%%%%%%%%%%%%%%
\subsection{Legacy Detection}
\label{sec:detection}

The directive |\childdocmain| in the main file can detect
whether the complete document or merely a child is to be compiled
even without using the directive |\childdocof|.
This method is deprecated because it is less robust
and there is no compelling reason to use it;
it is merely provided for backward compatibility
and it may be removed in future versions.

If the detection mechanism is to be used,
it is mandatory to correctly specify
the filename of the main file as the argument of |\childdocmain|:
%
\begin{center}
\begin{tabular}{l}
|\input{childdoc.def}|\\
|\childdocmain{|\textit{main}|}|\\
\end{tabular}
\end{center}
%
If |\jobname| does not match the argument \textit{main} of |\childdocmain|,
it is assumed that |\jobname| points to the child file to be compiled.
When using |\childdocmain| with the main file specified as argument,
it suffices to start a child file
with just |\input{|\textit{main}|}|
without loading of the package and using |\childdocof|.
If instead all processing is done
with the appropriate \textsf{childdoc} directives,
the argument of \textit{main} of |\childdocmain| can be empty.

An alternative version of the command line processing described
in \secref{sec:commandline} using the detection mechanism reads:
%
\begin{center}
|... -jobname "|\textit{target}|" "|[\textit{flags}]%
[|\def\jobname{|\textit{dest}|}|]|\input{|\textit{main}|}"|
\end{center}

%%%%%%%%%%%%%%%%%%%%%%%%%%%%%%%%%%%%%%%%%%%%%%%%%%%%%%%%%%%%%%%%%%%%%%%%%%%%%%%%
\subsection{Manual Code}
\label{sec:manual}

In case one cannot be certain whether the definitions file |childdoc.def|
is installed on the target \TeX{} distribution
and one prefers not to ship it,
it is conceivable to paste a few relevant commands into the sources.

To that end, drop all statements |\input{childdoc.def}|
and perform the replacements as outlined below.
Instead of |\childdocmain{|\textit{main}|}| add the following code
to the top of the main file:
%
\begin{center}
\begin{tabular}{l}
|\||ifdefined\childdocname\endinput\||fi\newif\ifchilddoc|\\
|\edef\childdocname{\scantokens\expandafter{\jobname\noexpand}}|\\
|\def\childdocmain{|\textit{main}|}\||ifx\childdocmain\childdocname\||else|\\
|\childdoctrue\includeonly{\childdocname}\let\jobname\childdocmain\||fi|\\
\end{tabular}
\end{center}
%
Instead of |\childdocof{|\textit{main}|}| just include the main file
at the top of each child file:
%
\begin{center}
|\input{|\textit{main}|}|
\end{center}
%
A simple redirection |\childdocforward{|\textit{dest}|}| is achieved by:
%
\begin{center}
|\def\jobname{|\textit{dest}|}\input{\jobname}|
\end{center}
%
The redirection with prefix
|\childdocforwardprefix[|\textit{prefix}|]{|\textit{dest}|}|
is accomplished by:
%
\begin{center}
\begin{tabular}{l}
|{\edef\jobname{\scantokens\expandafter{\jobname\noexpand}}|\\
|\def\redirectjob |\textit{prefix}|#1~~~{\gdef\jobname{|\textit{dest}|#1}}|\\
|\expandafter\redirectjob\jobname~~~}\input{\jobname}|
\end{tabular}
\end{center}

In an alternative approach,
child documents can be compiled by a specific command line
without additional code or specific definitions:
%
\begin{center}
|... -jobname "|\textit{target}|" "|[\textit{flags}]%
|\includeonly{|\textit{dest}|}\input{|\textit{main}|}"|
\end{center}
%

%%%%%%%%%%%%%%%%%%%%%%%%%%%%%%%%%%%%%%%%%%%%%%%%%%%%%%%%%%%%%%%%%%%%%%%%%%%%%%%%
%%%%%%%%%%%%%%%%%%%%%%%%%%%%%%%%%%%%%%%%%%%%%%%%%%%%%%%%%%%%%%%%%%%%%%%%%%%%%%%%
\section{Information}

%%%%%%%%%%%%%%%%%%%%%%%%%%%%%%%%%%%%%%%%%%%%%%%%%%%%%%%%%%%%%%%%%%%%%%%%%%%%%%%%
\subsection{Copyright}

Copyright \copyright{} 2017--2018 Niklas Beisert

This work may be distributed and/or modified under the
conditions of the \LaTeX{} Project Public License, either version 1.3
of this license or (at your option) any later version.
The latest version of this license is in
  \url{http://www.latex-project.org/lppl.txt}
and version 1.3 or later is part of all distributions of \LaTeX{}
version 2005/12/01 or later.

This work has the LPPL maintenance status `maintained'.

The Current Maintainer of this work is Niklas Beisert.

This work consists of the files |README.txt|, |childdoc.ins| and |childdoc.dtx|
as well as the derived files |childdoc.def|, |cdocsamp.tex|
with |cdocsch1.tex|, |cdocsch2.tex|, |cdocspt3.tex|, |cdocspt4.tex|,
|cdocsdrf.tex|, |cdocsfn1.tex|, |cdocsfn2.tex|
as well as |childdoc.pdf|.

%%%%%%%%%%%%%%%%%%%%%%%%%%%%%%%%%%%%%%%%%%%%%%%%%%%%%%%%%%%%%%%%%%%%%%%%%%%%%%%%
\subsection{Files and Installation}

The package consists of the files:
%
\begin{center}
\begin{tabular}{ll}
    |README.txt|   & readme file \\
    |childdoc.ins| & installation file \\
    |childdoc.dtx| & source file \\
    |childdoc.def| & definition file \\
    |cdocsamp.tex| & sample main file \\
    |cdocsch1.tex| & sample include file \\
    |cdocsch2.tex| & sample include file \\
    |cdocspt3.tex| & sample part file \\
    |cdocspt4.tex| & sample part file \\
    |cdocsdrf.tex| & sample redirection file \\
    |cdocsfn1.tex| & sample redirection file \\
    |cdocsfn2.tex| & sample redirection file \\
    |childdoc.pdf| & manual
\end{tabular}
\end{center}
%
The distribution consists of the files
|README.txt|, |childdoc.ins| and |childdoc.dtx|.
%
\begin{itemize}
\item
Run (pdf)\LaTeX{} on |childdoc.dtx|
to compile the manual |childdoc.pdf| (this file).
\item
Run \LaTeX{} on |childdoc.ins| to create the definitions file |childdoc.def|
and the sample |cdocsamp.tex| with include files
|cdocsch1.tex|, |cdocsch2.tex|, |cdocspt3.tex|, |cdocspt4.tex|,
|cdocsdrf.tex|, |cdocsfn1.tex|, |cdocsfn2.tex|.
Then copy the file |childdoc.def| to an appropriate directory of your \LaTeX{}
distribution, e.g.\ \textit{texmf-root}|/tex/latex/childdoc|.
\end{itemize}

%%%%%%%%%%%%%%%%%%%%%%%%%%%%%%%%%%%%%%%%%%%%%%%%%%%%%%%%%%%%%%%%%%%%%%%%%%%%%%%%
\subsection{Related CTAN Packages}

There are several other packages which offer a similar functionality:
%
\begin{itemize}
\item
The packages
\href{http://ctan.org/pkg/docmute}{\textsf{docmute}},
\href{http://ctan.org/pkg/includex}{\textsf{includex}} and
\href{http://ctan.org/pkg/standalone}{\textsf{standalone}}
provide commands to include only the document body of
a child file thus allowing both files to be compiled individually.
\item
The packages \href{http://ctan.org/pkg/subdocs}{\textsf{subdocs}}
and \href{http://ctan.org/pkg/subfiles}{\textsf{subfiles}}
provide structures in which the main and child documents can be
encapsulated and allowing them to be compiled individually.
The inclusion mechanism is different from the conventional |\include|.
\item
The package \href{http://ctan.org/pkg/combine}{\textsf{combine}}
is an elaborate solution to combine several documents into one.
\end{itemize}
%
See also the CTAN topic \href{http://ctan.org/topic/subdocs}{\textsf{subdocs}}
for further related packages.
The present package differs from the above solutions in that
a document structure constructed with the conventional |\include| mechanism
just needs two extra commands at the top of every file
such that all constituent files can be compiled individually.

%%%%%%%%%%%%%%%%%%%%%%%%%%%%%%%%%%%%%%%%%%%%%%%%%%%%%%%%%%%%%%%%%%%%%%%%%%%%%%%%
%\subsection{Feature Suggestions}
%
%The following is a list of features which may be useful for future
%versions of this package:
%%
%\begin{itemize}
%\item
%\ldots
%\end{itemize}

%%%%%%%%%%%%%%%%%%%%%%%%%%%%%%%%%%%%%%%%%%%%%%%%%%%%%%%%%%%%%%%%%%%%%%%%%%%%%%%%
\subsection{Revision History}

%%%%%%%%%%%%%%%%%%%%%%%%%%%%%%%%%%%%%%%%
\paragraph{v2.0:} 2018/12/30

\begin{itemize}
\item
immediate forward processing
\item
added |\childdocby| mechanism
\item
manual restructured
\end{itemize}

%%%%%%%%%%%%%%%%%%%%%%%%%%%%%%%%%%%%%%%%
\paragraph{v1.6:} 2018/01/17

\begin{itemize}
\item
application for development of include files
\item
corrections to manual
\end{itemize}

%%%%%%%%%%%%%%%%%%%%%%%%%%%%%%%%%%%%%%%%
\paragraph{v1.5:} 2017/05/21

\begin{itemize}
\item
more complete structuring introduced
\item
|\childdocof| introduced
\item
|\childdoc| renamed to |\childdocmain|
\item
|\childredirect| renamed to |\childdocforward| and |\childdocforwardprefix|
and functionality expanded
\end{itemize}

%%%%%%%%%%%%%%%%%%%%%%%%%%%%%%%%%%%%%%%%
\paragraph{v1.0:} 2017/04/27

\begin{itemize}
\item
manual and install package
\item
first version published on CTAN
\end{itemize}

%%%%%%%%%%%%%%%%%%%%%%%%%%%%%%%%%%%%%%%%
\paragraph{v0.6:} 2017/04/26

\begin{itemize}
\item
redirection mechanism added
\end{itemize}

%%%%%%%%%%%%%%%%%%%%%%%%%%%%%%%%%%%%%%%%
\paragraph{v0.5:} 2017/04/26

\begin{itemize}
\item
functionality in definition file
\end{itemize}


%%%%%%%%%%%%%%%%%%%%%%%%%%%%%%%%%%%%%%%%%%%%%%%%%%%%%%%%%%%%%%%%%%%%%%%%%%%%%%%%
%%%%%%%%%%%%%%%%%%%%%%%%%%%%%%%%%%%%%%%%%%%%%%%%%%%%%%%%%%%%%%%%%%%%%%%%%%%%%%%%
%%%%%%%%%%%%%%%%%%%%%%%%%%%%%%%%%%%%%%%%%%%%%%%%%%%%%%%%%%%%%%%%%%%%%%%%%%%%%%%%
\appendix

\settowidth\MacroIndent{\rmfamily\scriptsize 000\ }

 \DocInput{childdoc.dtx}

\end{document}
%</driver>
% \fi
%
% %%%%%%%%%%%%%%%%%%%%%%%%%%%%%%%%%%%%%%%%%%%%%%%%%%%%%%%%%%%%%%%%%%%%%%%%%%%%%%
% %%%%%%%%%%%%%%%%%%%%%%%%%%%%%%%%%%%%%%%%%%%%%%%%%%%%%%%%%%%%%%%%%%%%%%%%%%%%%%
% \section{Sample}
%\iffalse
%<*samplemain>
%\fi
%
% The following presents a sample document
% with two chapters, two parts, a title page,
% a compile flag as well as three forwarding files to set the flag.
% It consists of eight |.tex| files:
% \begin{center}
% \begin{tabular}{ll}
% |cdocsamp.tex|&main file\\
% |cdocsch1.tex|&include file for chapter 1\\
% |cdocsch2.tex|&include file for chapter 2\\
% |cdocspt3.tex|&include file for part 3\\
% |cdocspt4.tex|&include file for part 4\\
% |cdocsdrf.tex|&forwarding file for main file in draft mode\\
% |cdocsfi1.tex|&forwarding file for final version of chapter 1\\
% |cdocsfi2.tex|&forwarding file for final version of chapter 2\\
% \end{tabular}
% \end{center}
% Each of the eight files can be compiled directly by the \LaTeX{} compiler.
%
% %%%%%%%%%%%%%%%%%%%%%%%%%%%%%%%%%%%%%%
% \paragraph{Main File.}
%
% The main file is called |cdocsamp.tex|.
%
% Load the \textsf{childdoc} definitions and
% declare the filename for the main document:
%    \begin{macrocode}
\input{childdoc.def}
\childdocmain{}
%    \end{macrocode}

% Optional override for |\version| flag:
%    \begin{macrocode}
%%\ifchilddoc\else\providecommand{\version}{draft}\fi
%    \end{macrocode}

% Define the default values for the |\version| flag
% (|final| for the main file and |draft| for childs):
%    \begin{macrocode}
\ifchilddoc
\providecommand{\version}{draft}
\else
\providecommand{\version}{final}
\fi
%    \end{macrocode}

% Load the standard document class:
%    \begin{macrocode}
\documentclass[12pt]{article}
%    \end{macrocode}

% Start the document body:
%    \begin{macrocode}
\begin{document}
%    \end{macrocode}

% Declare a title page.
% Print title, part of document being processed and version flag:
%    \begin{macrocode}
\addtocounter{page}{-1}
\begin{center}
{\LARGE\bfseries{}childdoc example\par}
\vspace{1cm}
\ifchilddoc
\ifchilddocmanual part\else chapter\fi:
`\childdocname' of `\childdocjob'\par
\else
main document: `\childdocjob'\par
\fi
version: \version\par
\end{center}
\newpage
%    \end{macrocode}

% Manually include selected file,
% otherwise process as usual:
%    \begin{macrocode}
\ifchilddocmanual
\section*{part `\childdocname'}
\input{\childdocname}
\else
%    \end{macrocode}

% Include the two chapters:
%    \begin{macrocode}
\include{cdocsch1}
\include{cdocsch2}
%    \end{macrocode}

% Include the two parts unless only chapters should be displayed:
%    \begin{macrocode}
\ifchilddoc\else
\section{part three}
\input{cdocspt3}
\section{part four}
\input{cdocspt4}
\fi
%    \end{macrocode}

% Process as usual until here:
%    \begin{macrocode}
\fi
%    \end{macrocode}

% End of document body:
%    \begin{macrocode}
\end{document}
%    \end{macrocode}
%\iffalse
%</samplemain>
%\fi
%
% %%%%%%%%%%%%%%%%%%%%%%%%%%%%%%%%%%%%%%
% \paragraph{Chapter Include Files.}
%
% The include files are called |cdocsch1.tex| and |cdocsch2.tex|.
%
%\iffalse
%<*samplechap1|samplechap2>
%\fi

% Optional override for |\version| flag:
%    \begin{macrocode}
%%\providecommand{\version}{final}
%    \end{macrocode}

% Include the main document:
%    \begin{macrocode}
\input{childdoc.def}
\childdocof{cdocsamp}
%    \end{macrocode}

%\iffalse
%</samplechap1|samplechap2>
%\fi
%
%\iffalse
%<*samplechap1>
%\fi
% Some text for chapter 1:
%    \begin{macrocode}
\section{one}
some text in chapter one
%    \end{macrocode}

%\iffalse
%</samplechap1>
%\fi
% Some text for chapter 2:
%\iffalse
%<*samplechap2>
%\fi
%    \begin{macrocode}
\section{two}
more text in chapter two
%    \end{macrocode}

%\iffalse
%</samplechap2>
%\fi
%
% %%%%%%%%%%%%%%%%%%%%%%%%%%%%%%%%%%%%%%
% \paragraph{Part Include Files.}
%
% The include files are called |cdocspt3.tex| and |cdocspt4.tex|.
%
%\iffalse
%<*samplepart3|samplepart4>
%\fi

% Optional override for |\version| flag:
%    \begin{macrocode}
%%\providecommand{\version}{final}
%    \end{macrocode}

% Include the main document:
%    \begin{macrocode}
\input{childdoc.def}
\childdocby{cdocsamp}
%    \end{macrocode}

%\iffalse
%</samplepart3|samplepart4>
%\fi
%
%\iffalse
%<*samplepart3>
%\fi
% Some text for part 3:
%    \begin{macrocode}
some text in part three
%    \end{macrocode}

%\iffalse
%</samplepart3>
%\fi
% Some text for part 4:
%\iffalse
%<*samplepart4>
%\fi
%    \begin{macrocode}
more text in part four
%    \end{macrocode}

%\iffalse
%</samplepart4>
%\fi
%
% %%%%%%%%%%%%%%%%%%%%%%%%%%%%%%%%%%%%%%
% \paragraph{Forwarding for a Complete Draft.}
%
% The following forwarding file |cdocsdrf.tex|
% compiles the main document in draft mode:
%\iffalse
%<*sampledraft>
%\fi
%    \begin{macrocode}
\def\version{draft}
\input{childdoc.def}
\childdocforward{cdocsamp}
%    \end{macrocode}

%\iffalse
%</sampledraft>
%\fi
%
% %%%%%%%%%%%%%%%%%%%%%%%%%%%%%%%%%%%%%%
% \paragraph{Forwarding for Final Version of the Chapters.}
%
% The following forwarding files |cdocsfn1.tex| and |cdocsfn2.tex|
% (with identical content)
% compile the final versions of the child documents
% |cdocsch1.tex| and |cdocsch2.tex|, respectively:
%\iffalse
%<*samplefinal>
%\fi
%    \begin{macrocode}
\def\version{final}
\input{childdoc.def}
\childdocforwardprefix[cdocsamp]{cdocsfn}{cdocsch}
%    \end{macrocode}

%\iffalse
%</samplefinal>
%\fi
%
% %%%%%%%%%%%%%%%%%%%%%%%%%%%%%%%%%%%%%%
% \paragraph{Command Line Processing.}
%
% The following three command lines generate the output files
% |cdocscld|, |cdocscl1| and |cdocscl2|
% which should be identical to
% |cdocsdrf|, |cdocsch1| and |cdocsfn2|, respectively:
% \begin{center}
% \begin{tabular}{l}
% |latex -jobname cdocscld \|\\
% |  "\def\version{draft}\input{childdoc.def}\childdocforward{cdocsamp}"|\\
% |latex -jobname cdocscl1 \|\\
% |  "\input{childdoc.def}\childdocforward[cdocsamp]{cdocsch1}"|\\
% |latex -jobname cdocscl2 \|\\
% |  "\def\version{final}\input{childdoc.def}\childdocforward{cdocsch2}"|
% \end{tabular}
% \end{center}
% Note that the trailing backslash on each first line
% merely continues the input to the second line
% (for convenient cut ant paste).
% Furthermore, the command |latex| can be replaced by any
% of its alternative versions such as |pdflatex|.
%
% %%%%%%%%%%%%%%%%%%%%%%%%%%%%%%%%%%%%%%%%%%%%%%%%%%%%%%%%%%%%%%%%%%%%%%%%%%%%%%
% %%%%%%%%%%%%%%%%%%%%%%%%%%%%%%%%%%%%%%%%%%%%%%%%%%%%%%%%%%%%%%%%%%%%%%%%%%%%%%
% \section{Implementation}
%\iffalse
%<*package>
%\fi
%
% This section describes the definitions file |childdoc.def|.

% The definitions cannot be loaded using |\usepackage| or |\RequirePackage|
% which has a mechanism to prevent loading a style file more than once.
% When loading the definitions by means of |\input|
% multiple instances have to be prevented manually:
%\iffalse
%This code needs to be before the `\ProvidesFile' directive
%which is defined at the beginning of this file.
%Therefore it is also placed there and commented out here.
%</package>
%<*discard>
%\fi
%    \begin{macrocode}
\ifdefined\childdocmain\endinput\fi
%    \end{macrocode}
%\iffalse
%</discard>
%<*package>
%\fi
%
% \macro{\ifchilddoc}
% \macro{\ifchilddocmanual}
% The conditional |\ifchilddoc| tells whether a
% child (true) or main (false) document is being compiled.
% The conditional |\ifchilddocmanual| tells whether
% the |\includeonly| mechanism is used (false) or
% the selection of child files must be performed manually (true).
% The definitions initialise to false:
%    \begin{macrocode}
\newif\ifchilddoc
\newif\ifchilddocmanual
%    \end{macrocode}

% \macro{\childdocname}
% \macro{\childdocjob}
% The macro |\childdocname| stores the name of the main document
% to be compiled. The macro |\childdocjob| stores the name of
% the document on which the \LaTeX{} compiler was originally invoked.
% The content of |\jobname| cannot be compared
% to filenames specified in the source due to different catcodes.
% The following code rescans |\jobname|, stores the result
% in |\childdocname| and saves a copy in |\childdocjob|:
%    \begin{macrocode}
\edef\childdocname{\scantokens\expandafter{\jobname\noexpand}}
\let\childdocjob\childdocname
%    \end{macrocode}

% \macro{\childdocdisable}
% The macro |\childdocdisable| prevents the main file
% from being processed more than once.
% At this stage, the main document command |\childdocmain|
% is assumed to be called once again where it should do nothing.
% Any subsequent call to it should prevent
% a secondary processing of the main document
% It overwrites the forwarding commands
% |\childdocof| and |\childdocforward|
% with empty macros to prevent further inclusions of the main document:
%    \begin{macrocode}
\newcommand{\childdocdisable}
{
  \renewcommand{\childdocmain}[1]{\renewcommand{\childdocmain}[1]{\endinput}}
  \renewcommand{\childdocof}[1]{}
  \renewcommand{\childdocby}[2][]{}
  \renewcommand{\childdocforward}[2][]{}
  \renewcommand{\childdocdisable}{}
}
%    \end{macrocode}

% \macro{\childdocmain}
% The macro |\childdocmain| is to be called at the top of the main file
% with nothing or the main filename (without extension) as argument.
% First, it breaks loops.
% If the argument is not empty and does not match |\childdocname|
% (which is set by the first inclusion of |childdoc.def|),
% |\ifchilddoc| is set to true, |\includeonly| is applied to the child file
% and |\jobname| is set to the main file
% (for proper handling of |.aux| files):
%    \begin{macrocode}
\newcommand{\childdocmain}[1]
{
  \childdocdisable\childdocmain{}
  \if?#1?\else
    \begingroup
      \def\childdoctmp{#1}
      \ifx\childdoctmp\childdocname
        \def\childdoctmp{}
      \else
        \def\childdoctmp
        {
          \childdoctrue
          \includeonly{\childdocname}
          \def\childdocjob{#1}
          \def\jobname{#1}
        }
      \fi
      \expandafter
    \endgroup
    \childdoctmp
  \fi
}
%    \end{macrocode}

% \macro{\childdocof}
% The command |\childdocof| redirects
% compilation to the main file |#1|.
%    \begin{macrocode}
\newcommand{\childdocof}[1]
{
  \childdocdisable
  \childdoctrue
  \includeonly{\childdocname}
  \def\jobname{#1}
  \def\childdocjob{#1}
  \input{#1}
}
%    \end{macrocode}

% \macro{\childdocby}
% The command |\childdocby| ....
%    \begin{macrocode}
\newcommand{\childdocby}[2][]
{
  \childdocdisable
  \childdoctrue
  \childdocmanualtrue
  \if?#1?\else
    \def\jobname{#2}
  \fi
  \def\childdocjob{#2}
  \input{#2}
  \endinput
}
%    \end{macrocode}

% \macro{\childdocforward}
% The command |\childdocforward| redirects
% compilation to the main file or
% (if the optional argument is given) a child file.
% Parameters are set as if the main file
% or a child file starting with |\childdocof| was compiled.
% Then compilation is handed over to the main file:
%    \begin{macrocode}
\newcommand{\childdocforward}[2][]
{
  \begingroup
    \if?#1?
      \def\childdoctmp
      {
        \def\childdocname{#2}
        \def\childdocjob{#2}
        \def\jobname{#2}
        \input{#2}
        \endinput
      }
    \else
      \def\childdoctmp
      {
        \childdocdisable
        \def\childdocname{#2}
        \childdoctrue
        \includeonly{#2}
        \def\childdocjob{#1}
        \def\jobname{#1}
        \input{#1}
        \endinput
      }
    \fi
    \expandafter
  \endgroup
  \childdoctmp
}
%    \end{macrocode}

% \macro{\childdocforwardprefix}
% The command |\childdocforwardprefix| redirects
% compilation to the main or a child file by means of a pattern.
% The prefix |#1| in the current filename is replaced by |#2|
% and the suffix of the current filename is kept
% (it is assumed that the filename does not contain the substring `|~~~|'
% which is used as a delimiter).
% Compilation is handed over to the new file by |\childdocforward|:
%    \begin{macrocode}
\newcommand{\childdocforwardprefix}[3][]
{
  \begingroup
    \def\childdocextract #2##1~~~{\def\childdoctmp{\childdocforward[#1]{#3##1}}}
    \expandafter\childdocextract\childdocname~~~
    \expandafter
  \endgroup
  \childdoctmp
}
%    \end{macrocode}

% \macro{\childdoc}
% The deprecated macro |\childdoc| is a legacy version of |\childdocmain|:
%    \begin{macrocode}
\newcommand{\childdoc}{\childdocmain}
%    \end{macrocode}

% \macro{\childdocredirect}
% The deprecated macro |\childdocredirect| is a legacy version
% of |\childdocforward| and |\childdocforwardprefix|:
%    \begin{macrocode}
\newcommand{\childdocredirect}[2][]
{
  \begingroup
    \if?#1?
      \def\childdoctmp{\childdocforward{#2}}
    \else
      \def\childdoctmp{\childdocforwardprefix{#1}{#2}}
    \fi
    \expandafter
  \endgroup
  \childdoctmp
}
%    \end{macrocode}

%\iffalse
%</package>
%\fi
%
\endinput

\childdocforward{cdocsamp}
%    \end{macrocode}

%\iffalse
%</sampledraft>
%\fi
%
% %%%%%%%%%%%%%%%%%%%%%%%%%%%%%%%%%%%%%%
% \paragraph{Forwarding for Final Version of the Chapters.}
%
% The following forwarding files |cdocsfn1.tex| and |cdocsfn2.tex|
% (with identical content)
% compile the final versions of the child documents
% |cdocsch1.tex| and |cdocsch2.tex|, respectively:
%\iffalse
%<*samplefinal>
%\fi
%    \begin{macrocode}
\def\version{final}
% \iffalse
%
% childdoc.dtx Copyright (C) 2017-2018 Niklas Beisert
%
% This work may be distributed and/or modified under the
% conditions of the LaTeX Project Public License, either version 1.3
% of this license or (at your option) any later version.
% The latest version of this license is in
%   http://www.latex-project.org/lppl.txt
% and version 1.3 or later is part of all distributions of LaTeX
% version 2005/12/01 or later.
%
% This work has the LPPL maintenance status `maintained'.
%
% The Current Maintainer of this work is Niklas Beisert.
%
% This work consists of the files childdoc.dtx and childdoc.ins
% and the derived files childdoc.def and cdocsamp.tex with
% cdocsch1.tex, cdocsch2.tex, cdocsdrf.tex, cdocsfn1.tex, cdocsfn2.tex.
%
%<package>\ifdefined\childdocmain\endinput\fi
%<package>\ProvidesFile{childdoc.def}[2018/12/30 v2.0 child document driver]
%<samplemain>\ProvidesFile{cdocsamp.tex}[2018/12/30 v2.0 sample for childdoc]
%<*driver>
%\ProvidesFile{childdoc.drv}[2018/12/30 v2.0 childdoc reference manual file]
\PassOptionsToClass{10pt,a4paper}{article}
\documentclass{ltxdoc}

\usepackage[margin=35mm]{geometry}
\usepackage{hyperref}
\usepackage{hyperxmp}
\usepackage[usenames]{color}

\hypersetup{colorlinks=true}
\hypersetup{pdfstartview=FitH}
\hypersetup{pdfpagemode=UseNone}
\hypersetup{pdfsource={}}
\hypersetup{pdflang={en-UK}}
\hypersetup{pdfcopyright={Copyright 2017-2018 Niklas Beisert.
  This work may be distributed and/or modified under the
  conditions of the LaTeX Project Public License, either version 1.3
  of this license or (at your option) any later version.}}
\hypersetup{pdflicenseurl={http://www.latex-project.org/lppl.txt}}
\hypersetup{pdfcontactaddress={ETH Zurich, ITP, HIT K,
  Wolfgang-Pauli-Strasse 27}}
\hypersetup{pdfcontactpostcode={8093}}
\hypersetup{pdfcontactcity={Zurich}}
\hypersetup{pdfcontactcountry={Switzerland}}
\hypersetup{pdfcontactemail={nbeisert@itp.phys.ethz.ch}}
\hypersetup{pdfcontacturl={http://people.phys.ethz.ch/\xmptilde nbeisert/}}

\newcommand{\secref}[1]{\hyperref[#1]{section \ref*{#1}}}

\parskip1ex
\parindent0pt
\let\olditemize\itemize
\def\itemize{\olditemize\parskip0pt}

\begin{document}

\title{The \textsf{childdoc} Package}
\hypersetup{pdftitle={The childdoc Package}}
\author{Niklas Beisert\\[2ex]
  Institut f\"ur Theoretische Physik\\
  Eidgen\"ossische Technische Hochschule Z\"urich\\
  Wolfgang-Pauli-Strasse 27, 8093 Z\"urich, Switzerland\\[1ex]
  \href{mailto:nbeisert@itp.phys.ethz.ch}
  {\texttt{nbeisert@itp.phys.ethz.ch}}}
\hypersetup{pdfauthor={Niklas Beisert}}
\hypersetup{pdfsubject={Manual for the LaTeX2e Package childdoc}}
\date{30 December 2018, \textsf{v2.0}}
\maketitle

\begin{abstract}\noindent
\textsf{childdoc} is a \LaTeXe{} package
that enables the direct compilation
of document sections included by |\include|
to individual files.
\end{abstract}

\begingroup
\parskip0ex
\tableofcontents
\endgroup

%%%%%%%%%%%%%%%%%%%%%%%%%%%%%%%%%%%%%%%%%%%%%%%%%%%%%%%%%%%%%%%%%%%%%%%%%%%%%%%%
%%%%%%%%%%%%%%%%%%%%%%%%%%%%%%%%%%%%%%%%%%%%%%%%%%%%%%%%%%%%%%%%%%%%%%%%%%%%%%%%
\section{Introduction}

\LaTeX{} provides a mechanism to structure a large document (such as a book)
into a main file and several child files (containing the chapters)
using the |\include| command.
This mechanism is beneficial for documents
which span hundreds of pages in order to
make the source file(s) more manageable.
Moreover, compilation can be restricted to
selected child files by means of the |\includeonly| command.
The latter feature can be used to reduce the compilation time while editing
(this was significantly more useful in the earlier days of \LaTeX{})
or to generate a smaller document which is easier to navigate.
Another application of |\includeonly| is to generate
documents consisting of selected parts of the complete document.

However, there are a few drawbacks of the plain |\include| mechanism:
\begin{itemize}
\item
The child files cannot be compiled on their own,
they can only be compiled via the main file.
A naive editing environment
(such as a text editor with an option
to have the current file processed by \LaTeX)
may require one to switch to the main file before compiling;
attempting to compile the child file produces errors.
\item
The main file must be modified (each time)
to adjust the |\includeonly| command
to the present needs. This easily leaves the main file in a messy state.
\item
The generated document will always carry the filename
of the main document. This is inconvenient if
several child files are to be compiled and
to be kept for distribution.
\end{itemize}

The present package provides a simple interface
to make child files individually compilable by \LaTeX{}.
Compiling a child file then has the same effect as compiling
the main file with an |\includeonly| command
to select the appropriate child.
Moreover the generated document will carry the name of the child
rather than the main file.
This resolves all three above issues.

This feature is meant to make the editing of books,
thesis documents and lecture notes somewhat more convenient.
However, the package can also be used efficiently for
composing a series of documents (such as exercise sheets)
which are typically distributed individually.
It then assists the author in generating the individual documents
(potentially in different versions)
as well as a document containing the collected series.
Another application is in developing style files
or other kinds of included material
where compilation of the style file could redirect
to a sample or test file.

%%%%%%%%%%%%%%%%%%%%%%%%%%%%%%%%%%%%%%%%%%%%%%%%%%%%%%%%%%%%%%%%%%%%%%%%%%%%%%%%
%%%%%%%%%%%%%%%%%%%%%%%%%%%%%%%%%%%%%%%%%%%%%%%%%%%%%%%%%%%%%%%%%%%%%%%%%%%%%%%%
\section{Usage}

First of all, the package \textsf{childdoc} is \emph{not} a standard
\LaTeXe{} |.sty| style file! Therefore it needs to be invoked in
a non-standard way.

%%%%%%%%%%%%%%%%%%%%%%%%%%%%%%%%%%%%%%%%%%%%%%%%%%%%%%%%%%%%%%%%%%%%%%%%%%%%%%%%
\subsection{Included Files}
\label{sec:include}

%%%%%%%%%%%%%%%%%%%%%%%%%%%%%%%%%%%%%%%%
\DescribeMacro{\childdocmain}
To use the package, add the commands
\begin{center}
\begin{tabular}{l}
|\input{childdoc.def}|\\
|\childdocmain{}|\\
\end{tabular}
\end{center}
at the very top of the main \LaTeX{} file,
in particular \emph{before} the |\documentclass| statement!
The argument of |\childdocmain| should be left empty
(but it must be present).

%%%%%%%%%%%%%%%%%%%%%%%%%%%%%%%%%%%%%%%%
\DescribeMacro{\childdocof}
Furthermore, add the commands
\begin{center}
\begin{tabular}{l}
|\input{childdoc.def}|\\
|\childdocof{|\textit{main}|}|\\
\end{tabular}
\end{center}
at the top of every child file \textit{child}
which is included by |\include{|\textit{child}|}|
from within the main file
(or at least for those files to be compiled individually).
The argument \textit{main} must be the filename of the main file.

There are a couple of
considerations in setting up the main and child documents:

%%%%%%%%%%%%%%%%%%%%%%%%%%%%%%%%%%%%%%%%
\paragraph{Restrictions.}

Please note the following restrictions:
\begin{itemize}
\item
|\childdocmain| must be called with one argument \textit{main}
to ensure compatibility with earlier version of the package.
It must either be empty (|\childdocmain{}|)
or precisely match the filename of the main file in which it is specified.
See \secref{sec:detection} for further information.
\item
The filename \textit{main} must be specified without the |.tex| extension.
\item
The filename \textit{main} is case sensitive
(even in case-insensitive file systems)
due to internal string comparison.
\item
The argument \textit{main} should be fully expanded, it cannot be a macro.
\item
Subdirectories and special characters should be avoided in filenames.
\item
The command |\childdocmain{|\textit{main}|}| must be followed by a whitespace.
It should not be followed immediately by another command
or by a comment mark `|%|'.
This is because the \TeX{} parser reads the token immediately following
the argument of |\childdocmain| and puts it
at the beginning of every child section;
however, a white\-space is ignored.
\end{itemize}

%%%%%%%%%%%%%%%%%%%%%%%%%%%%%%%%%%%%%%%%
\paragraph{Content of Main File.}

It is advisable to place all content in the child files included by |\include|.
Any output contained in the main file will appear in all child documents
unless suppressed manually;
it cannot be suppressed automatically by the |\includeonly| directive
and thus should normally be avoided.
A method to include some content in the main file
by means of conditional processing is described in \secref{sec:conditional}.

%%%%%%%%%%%%%%%%%%%%%%%%%%%%%%%%%%%%%%%%
\paragraph{Page Numbering.}

When only a part of the document is compiled,
the appropriate numbering of pages
(as well as other status parameters)
is determined from the |.aux| files.
The latter contain information from previous passes.
However this information needs to propagate through
all intermediate child documents.
Therefore the page numbering in child documents may well
be inconsistent until the complete document is compiled at least once.

A useful (if unconventional) way to always ensure a consistent
page numbering is to restart the numbering in each child document
and denote the pages by `\textit{child}|.|\textit{page}'
where \textit{child} represents the chapter/section number of the child file.
This can be achieved by the command
|\numberwithin{page}{|\textit{child}|}|
of the \textsf{amsmath} package
where \textit{child} can be |chapter| or |section|
depending on the chosen structuring.
Alternatively, one can modify the macro |\thepage| appropriately
and reset the counter |page| at the start of each child file.

%%%%%%%%%%%%%%%%%%%%%%%%%%%%%%%%%%%%%%%%%%%%%%%%%%%%%%%%%%%%%%%%%%%%%%%%%%%%%%%%
\subsection{Conditional Processing}
\label{sec:conditional}

The package provides a mechanism to compile different versions
of a document. To customise the versions further some conditional processing
can come in handy to distinguish which version is being compiled.
The package provides two macros to describe the compilation context:

%%%%%%%%%%%%%%%%%%%%%%%%%%%%%%%%%%%%%%%%
\DescribeMacro{\ifchilddoc}
The conditional |\ifchilddoc| distinguishes between the compilation of
child documents and the main document:
%
\begin{center}
|\ifchilddoc |\textit{child-code}| |[|\||else |\textit{main-code}]| \||fi|
\end{center}

%%%%%%%%%%%%%%%%%%%%%%%%%%%%%%%%%%%%%%%%
\DescribeMacro{\childdocname}
\DescribeMacro{\childdocjob}
The macro |\childdocname| contains the filename (without extension)
of the main or child file being processed.
Note that |\childdocjob| will always contain the name of the main file.

%%%%%%%%%%%%%%%%%%%%%%%%%%%%%%%%%%%%%%%%
\paragraph{Title Page.}

Conditional processing can be used to include a title or banner page
in the main document when proper precautions are taken.
Importantly, the code in the main file should ensure that the page counter
(as well as other status parameters which are stored in the |.aux| files)
takes the same value after the conditional processing.
Otherwise the page numbers may take divergent values
depending on which part is compiled.

For example, a title page could be declared by:
%
\begin{center}
\begin{tabular}{l}
|\ifchilddoc\||else|\\
|\addtocounter{page}{-1}|\\
\textit{code for title page}\\
|\newpage|\\
|\||fi|
\end{tabular}
\end{center}
%
A banner page for the child documents can be generated by:
%
\begin{center}
\begin{tabular}{l}
|\ifchilddoc|\\
|\addtocounter{page}{-1}|\\
\textit{code for banner page}\\
|\newpage|\\
|\||fi|
\end{tabular}
\end{center}
%
Here one could write a message such as:
\begin{center}
|This is the part \childdocname{} of \childdocjob{}.|
\end{center}

%%%%%%%%%%%%%%%%%%%%%%%%%%%%%%%%%%%%%%%%%%%%%%%%%%%%%%%%%%%%%%%%%%%%%%%%%%%%%%%%
\subsection{Flags}
\label{sec:flags}

The package makes it easy to generate different versions
of the main or child documents.
To this end compilation flags can be defined
and assigned different default values.
They will be particularly useful in conjunction
with the forwarding mechanism described in \secref{sec:forward}.

For example, it may be useful to have a flag |\version|
which can be set to |draft| or |final|.
The document source will contain some conditional code
depending on the value of |\version|.
Suppose further, the flag should default to |final| for the main file
and to |draft| for child files
which is a natural assignment for editing the document.
This is achieved by placing the following code
in the preamble of the main document
(below the |\childdocmain| directive):
%
\begin{center}
\begin{tabular}{l}
|\ifchilddoc|\\
|\providecommand{\version}{draft}|\\
|\||else|\\
|\providecommand{\version}{final}|\\
|\||fi|
\end{tabular}
\end{center}
%
The definition by |\providecommand| makes sure
that previous definitions are not overwritten.
Further statements |\providecommand{\version}{...}|
can thus be added before the above code to override it.

For the main file, one might add a line
(between |\childdocmain| and the above block)
%
\begin{center}
|%\ifchilddoc\||else\providecommand{\version}{draft}\||fi|
\end{center}
%
which can be uncommented to produce a draft version.
Likewise one can add a line to the very top of a child file
(above the |\childdocof{|\textit{main}|}| directive)
%
\begin{center}
|%\providecommand{\version}{final}|
\end{center}
%
which can be uncommented to produce the final version of this child document.

%%%%%%%%%%%%%%%%%%%%%%%%%%%%%%%%%%%%%%%%%%%%%%%%%%%%%%%%%%%%%%%%%%%%%%%%%%%%%%%%
\subsection{Forwarding}
\label{sec:forward}

Different versions of the main or child documents
using compilation flags as described in \secref{sec:flags}
can be (permanently) stored in different files
for convenient compilation, viewing and distribution.
To this end, the package defines a command
to pass on compilation to a different file:

%%%%%%%%%%%%%%%%%%%%%%%%%%%%%%%%%%%%%%%%
\DescribeMacro{\childdocforward}
The command |\childdocforward| redirects processing to
another source file:
%
\begin{center}
\begin{tabular}{l}
|\input{childdoc.def}|\\
|\childdocforward[|\textit{main}|]{|\textit{dest}|}|\\
\end{tabular}
\end{center}
%
The argument \textit{dest} is the destination file
(without extension).
It should be the main file or one of the child files.
Note that further \textsf{childdoc} directives
such as |\childdocof| and |\childdocforward|
in the indicated file will be processed in this form.
The optional argument \textit{main}
passes on directly to the main file \textit{main}
while pretending to compile the child \textit{dest}.
This form behaves as if \textit{dest}
issues |\childdocof{|\textit{main}|}| right away,
and no further \textsf{childdoc} directives will be processed.

%%%%%%%%%%%%%%%%%%%%%%%%%%%%%%%%%%%%%%%%
\DescribeMacro{\...prefix}
In the alternative form |\childdocforwardprefix|,
%
\begin{center}
\begin{tabular}{l}
|\input{childdoc.def}|\\
|\childdocforwardprefix[|\textit{main}|]{|\textit{prefix}|}{|\textit{dest}|}|
\end{tabular}
\end{center}
%
the destination file is determined by a pattern
depending on the current file:
To make this work, the current file must be called
`{\textit{prefix}\hspace{0.2em}\textit{suffix}}'
with \textit{prefix} matching precisely the argument.
Processing is then passed on to the file
`{\textit{dest}\hspace{0.2em}\textit{suffix}}'.
Surely, the same effect is achieved by
directly specifying the
argument `{\textit{dest}\hspace{0.2em}\textit{suffix}}'
in the first form.
However, that requires to set up a different file
for each child. With the alternative form of the command
all these files can have exactly the same content
which simplifies setting them up and maintaining them.

For example, the following file |draft.tex|
with a compilation flag |\version| as described in \secref{sec:flags}
compiles the main document as a draft:
%
\begin{center}
\begin{tabular}{l}
|\def\version{draft}|\\
|\input{childdoc.def}|\\
|\childdocforward{|\textit{main}|}|
\end{tabular}
\end{center}
%
Likewise, the following files |final|\textit{nn}|.tex|
compile the final version of the child document
|child|\textit{nn}|.tex|:
%
\begin{center}
\begin{tabular}{l}
|\def\version{final}|\\
|\input{childdoc.def}|\\
|\childdocforwardprefix{final}{child}|
\end{tabular}
\end{center}
%

Note that when several versions of a main file and/or of each child file
are to be generated, it may be convenient to set up a |Makefile| or
shell script to automatise the process.

%%%%%%%%%%%%%%%%%%%%%%%%%%%%%%%%%%%%%%%%%%%%%%%%%%%%%%%%%%%%%%%%%%%%%%%%%%%%%%%%
\subsection{Command Line Processing}
\label{sec:commandline}

The effect of redirection files can also be achieved by invoking
the \LaTeX{} compiler with a more elaborate command line.
Most conveniently this should be done as part
of a shell script or a |Makefile|.

When using \textsf{childdoc} in the main file, the following
command lines effectively perform a redirection
(note that depending on the shell being used,
backslashes may have to be doubled: `|\|' $\to$ `|\\|'):
%
\begin{center}
|... -jobname "|\textit{target}|" |\\|"|[\textit{flags}]%
|\input{childdoc.def}\childdocforward[|\textit{main}|]{|\textit{dest}|}"|
\end{center}
%
Here \textit{target} is the name of the output file,
\textit{main} is the name of the main file
and \textit{dest} is the name of the main or child file to be processed
(all filenames without extensions).
The optional argument \textit{main} can be omitted
if \textit{main} matches \textit{dest}.
Optionally, compilation \textit{flags} can be defined via |\def| commands.
This command line makes the \TeX{} engine believe
it is compiling the file \textit{target}
whose content is specified as the latter parameter.
The provided code then forwards the processing to
\textit{main} or \textit{dest} as described in \secref{sec:forward}.

%%%%%%%%%%%%%%%%%%%%%%%%%%%%%%%%%%%%%%%%%%%%%%%%%%%%%%%%%%%%%%%%%%%%%%%%%%%%%%%%
\subsection{Include by Input}
\label{sec:input}

Including child documents by |\include| has some restrictions by design.
Most notably, the content of a child document always occupies
its own set of pages; pages cannot be shared between child documents.
Usually, this behaviour makes perfect sense
because each child document contain an essential part of the document.
However, in some situations it may be desirable to compose
a document from a collection of parts
without having mandatory page breaks between then.
For this case, the package
provides a mechanism to include parts
by |\input| which can also be processed individually.
However, by construction this mechanism
requires manual handling of the content to be output.

%%%%%%%%%%%%%%%%%%%%%%%%%%%%%%%%%%%%%%%%
\DescribeMacro{\ifchilddocmanual}
The main file should be prepared as usual, see \secref{sec:include}.
However, the document body must make a distinction
between processing of an individual part and of the main document, e.g.:
%
\begin{center}
\begin{tabular}{l}
|\ifchilddocmanual|\\
|\input{\childdocname}|\\
|\||else|\\
\textit{document body with }|\input{|\textit{part}|}|\\
|\||fi|
\end{tabular}
\end{center}
%
The conditional |\ifchilddocmanual| is true whenever
a part to be included by |\input| is being compiled,
and the name of the part is stored in |\childdocname|.

%%%%%%%%%%%%%%%%%%%%%%%%%%%%%%%%%%%%%%%%
\DescribeMacro{\childdocby}
Each part to be included by |\input| should start with:
%
\begin{center}
\begin{tabular}{l}
|\input{childdoc.def}|\\
|\childdocby{|\textit{main}|}|\\
\end{tabular}
\end{center}
%
The directive |\childdocby| is similar to |\childdocof|
described in \secref{sec:include},
but the subsequent selection of content must be done manually.
To that end, both |\ifchilddoc| and |\ifchilddocmanual|
will be true upon processing of a part,
and the name of the part is stored in |\childdocname|.
Note that |\jobname| will be set to the filename of the current part
so that each part receives an individual |.aux| file
that does not interfere with the |.aux| file(s) of the main document.
This behaviour can be altered by the alternative form
|\childdocby[*]{|\textit{main}|}| (with a non-empty optional argument)
which uses the |.aux| file of the main document
by setting |\jobname| to \textit{main}.

%%%%%%%%%%%%%%%%%%%%%%%%%%%%%%%%%%%%%%%%%%%%%%%%%%%%%%%%%%%%%%%%%%%%%%%%%%%%%%%%
\subsection{Driver Development}
\label{sec:driver}

The \textsf{childdoc} mechanism can also be use for the development
of definition files such as \LaTeX{} styles or classes.
This case differs from the above setup with multiple parts
included by |\include| in that no |\includeonly| should be invoked.
This can be achieved by starting the include file
(before |\ProvidesPackage|) with:
%
\begin{center}
\begin{tabular}{l}
|\input{childdoc.def}|\\
|\childdocforward{|\textit{main}|}|\\
\end{tabular}
\end{center}
%
or alternatively with:
%
\begin{center}
\begin{tabular}{l}
|\input{childdoc.def}|\\
|\childdocby{|\textit{main}|}|\\
\end{tabular}
\end{center}
%
Both forms have slightly different effects as described above.
The main file is prepared as usual, see \secref{sec:include}.

%%%%%%%%%%%%%%%%%%%%%%%%%%%%%%%%%%%%%%%%%%%%%%%%%%%%%%%%%%%%%%%%%%%%%%%%%%%%%%%%
\subsection{Legacy Detection}
\label{sec:detection}

The directive |\childdocmain| in the main file can detect
whether the complete document or merely a child is to be compiled
even without using the directive |\childdocof|.
This method is deprecated because it is less robust
and there is no compelling reason to use it;
it is merely provided for backward compatibility
and it may be removed in future versions.

If the detection mechanism is to be used,
it is mandatory to correctly specify
the filename of the main file as the argument of |\childdocmain|:
%
\begin{center}
\begin{tabular}{l}
|\input{childdoc.def}|\\
|\childdocmain{|\textit{main}|}|\\
\end{tabular}
\end{center}
%
If |\jobname| does not match the argument \textit{main} of |\childdocmain|,
it is assumed that |\jobname| points to the child file to be compiled.
When using |\childdocmain| with the main file specified as argument,
it suffices to start a child file
with just |\input{|\textit{main}|}|
without loading of the package and using |\childdocof|.
If instead all processing is done
with the appropriate \textsf{childdoc} directives,
the argument of \textit{main} of |\childdocmain| can be empty.

An alternative version of the command line processing described
in \secref{sec:commandline} using the detection mechanism reads:
%
\begin{center}
|... -jobname "|\textit{target}|" "|[\textit{flags}]%
[|\def\jobname{|\textit{dest}|}|]|\input{|\textit{main}|}"|
\end{center}

%%%%%%%%%%%%%%%%%%%%%%%%%%%%%%%%%%%%%%%%%%%%%%%%%%%%%%%%%%%%%%%%%%%%%%%%%%%%%%%%
\subsection{Manual Code}
\label{sec:manual}

In case one cannot be certain whether the definitions file |childdoc.def|
is installed on the target \TeX{} distribution
and one prefers not to ship it,
it is conceivable to paste a few relevant commands into the sources.

To that end, drop all statements |\input{childdoc.def}|
and perform the replacements as outlined below.
Instead of |\childdocmain{|\textit{main}|}| add the following code
to the top of the main file:
%
\begin{center}
\begin{tabular}{l}
|\||ifdefined\childdocname\endinput\||fi\newif\ifchilddoc|\\
|\edef\childdocname{\scantokens\expandafter{\jobname\noexpand}}|\\
|\def\childdocmain{|\textit{main}|}\||ifx\childdocmain\childdocname\||else|\\
|\childdoctrue\includeonly{\childdocname}\let\jobname\childdocmain\||fi|\\
\end{tabular}
\end{center}
%
Instead of |\childdocof{|\textit{main}|}| just include the main file
at the top of each child file:
%
\begin{center}
|\input{|\textit{main}|}|
\end{center}
%
A simple redirection |\childdocforward{|\textit{dest}|}| is achieved by:
%
\begin{center}
|\def\jobname{|\textit{dest}|}\input{\jobname}|
\end{center}
%
The redirection with prefix
|\childdocforwardprefix[|\textit{prefix}|]{|\textit{dest}|}|
is accomplished by:
%
\begin{center}
\begin{tabular}{l}
|{\edef\jobname{\scantokens\expandafter{\jobname\noexpand}}|\\
|\def\redirectjob |\textit{prefix}|#1~~~{\gdef\jobname{|\textit{dest}|#1}}|\\
|\expandafter\redirectjob\jobname~~~}\input{\jobname}|
\end{tabular}
\end{center}

In an alternative approach,
child documents can be compiled by a specific command line
without additional code or specific definitions:
%
\begin{center}
|... -jobname "|\textit{target}|" "|[\textit{flags}]%
|\includeonly{|\textit{dest}|}\input{|\textit{main}|}"|
\end{center}
%

%%%%%%%%%%%%%%%%%%%%%%%%%%%%%%%%%%%%%%%%%%%%%%%%%%%%%%%%%%%%%%%%%%%%%%%%%%%%%%%%
%%%%%%%%%%%%%%%%%%%%%%%%%%%%%%%%%%%%%%%%%%%%%%%%%%%%%%%%%%%%%%%%%%%%%%%%%%%%%%%%
\section{Information}

%%%%%%%%%%%%%%%%%%%%%%%%%%%%%%%%%%%%%%%%%%%%%%%%%%%%%%%%%%%%%%%%%%%%%%%%%%%%%%%%
\subsection{Copyright}

Copyright \copyright{} 2017--2018 Niklas Beisert

This work may be distributed and/or modified under the
conditions of the \LaTeX{} Project Public License, either version 1.3
of this license or (at your option) any later version.
The latest version of this license is in
  \url{http://www.latex-project.org/lppl.txt}
and version 1.3 or later is part of all distributions of \LaTeX{}
version 2005/12/01 or later.

This work has the LPPL maintenance status `maintained'.

The Current Maintainer of this work is Niklas Beisert.

This work consists of the files |README.txt|, |childdoc.ins| and |childdoc.dtx|
as well as the derived files |childdoc.def|, |cdocsamp.tex|
with |cdocsch1.tex|, |cdocsch2.tex|, |cdocspt3.tex|, |cdocspt4.tex|,
|cdocsdrf.tex|, |cdocsfn1.tex|, |cdocsfn2.tex|
as well as |childdoc.pdf|.

%%%%%%%%%%%%%%%%%%%%%%%%%%%%%%%%%%%%%%%%%%%%%%%%%%%%%%%%%%%%%%%%%%%%%%%%%%%%%%%%
\subsection{Files and Installation}

The package consists of the files:
%
\begin{center}
\begin{tabular}{ll}
    |README.txt|   & readme file \\
    |childdoc.ins| & installation file \\
    |childdoc.dtx| & source file \\
    |childdoc.def| & definition file \\
    |cdocsamp.tex| & sample main file \\
    |cdocsch1.tex| & sample include file \\
    |cdocsch2.tex| & sample include file \\
    |cdocspt3.tex| & sample part file \\
    |cdocspt4.tex| & sample part file \\
    |cdocsdrf.tex| & sample redirection file \\
    |cdocsfn1.tex| & sample redirection file \\
    |cdocsfn2.tex| & sample redirection file \\
    |childdoc.pdf| & manual
\end{tabular}
\end{center}
%
The distribution consists of the files
|README.txt|, |childdoc.ins| and |childdoc.dtx|.
%
\begin{itemize}
\item
Run (pdf)\LaTeX{} on |childdoc.dtx|
to compile the manual |childdoc.pdf| (this file).
\item
Run \LaTeX{} on |childdoc.ins| to create the definitions file |childdoc.def|
and the sample |cdocsamp.tex| with include files
|cdocsch1.tex|, |cdocsch2.tex|, |cdocspt3.tex|, |cdocspt4.tex|,
|cdocsdrf.tex|, |cdocsfn1.tex|, |cdocsfn2.tex|.
Then copy the file |childdoc.def| to an appropriate directory of your \LaTeX{}
distribution, e.g.\ \textit{texmf-root}|/tex/latex/childdoc|.
\end{itemize}

%%%%%%%%%%%%%%%%%%%%%%%%%%%%%%%%%%%%%%%%%%%%%%%%%%%%%%%%%%%%%%%%%%%%%%%%%%%%%%%%
\subsection{Related CTAN Packages}

There are several other packages which offer a similar functionality:
%
\begin{itemize}
\item
The packages
\href{http://ctan.org/pkg/docmute}{\textsf{docmute}},
\href{http://ctan.org/pkg/includex}{\textsf{includex}} and
\href{http://ctan.org/pkg/standalone}{\textsf{standalone}}
provide commands to include only the document body of
a child file thus allowing both files to be compiled individually.
\item
The packages \href{http://ctan.org/pkg/subdocs}{\textsf{subdocs}}
and \href{http://ctan.org/pkg/subfiles}{\textsf{subfiles}}
provide structures in which the main and child documents can be
encapsulated and allowing them to be compiled individually.
The inclusion mechanism is different from the conventional |\include|.
\item
The package \href{http://ctan.org/pkg/combine}{\textsf{combine}}
is an elaborate solution to combine several documents into one.
\end{itemize}
%
See also the CTAN topic \href{http://ctan.org/topic/subdocs}{\textsf{subdocs}}
for further related packages.
The present package differs from the above solutions in that
a document structure constructed with the conventional |\include| mechanism
just needs two extra commands at the top of every file
such that all constituent files can be compiled individually.

%%%%%%%%%%%%%%%%%%%%%%%%%%%%%%%%%%%%%%%%%%%%%%%%%%%%%%%%%%%%%%%%%%%%%%%%%%%%%%%%
%\subsection{Feature Suggestions}
%
%The following is a list of features which may be useful for future
%versions of this package:
%%
%\begin{itemize}
%\item
%\ldots
%\end{itemize}

%%%%%%%%%%%%%%%%%%%%%%%%%%%%%%%%%%%%%%%%%%%%%%%%%%%%%%%%%%%%%%%%%%%%%%%%%%%%%%%%
\subsection{Revision History}

%%%%%%%%%%%%%%%%%%%%%%%%%%%%%%%%%%%%%%%%
\paragraph{v2.0:} 2018/12/30

\begin{itemize}
\item
immediate forward processing
\item
added |\childdocby| mechanism
\item
manual restructured
\end{itemize}

%%%%%%%%%%%%%%%%%%%%%%%%%%%%%%%%%%%%%%%%
\paragraph{v1.6:} 2018/01/17

\begin{itemize}
\item
application for development of include files
\item
corrections to manual
\end{itemize}

%%%%%%%%%%%%%%%%%%%%%%%%%%%%%%%%%%%%%%%%
\paragraph{v1.5:} 2017/05/21

\begin{itemize}
\item
more complete structuring introduced
\item
|\childdocof| introduced
\item
|\childdoc| renamed to |\childdocmain|
\item
|\childredirect| renamed to |\childdocforward| and |\childdocforwardprefix|
and functionality expanded
\end{itemize}

%%%%%%%%%%%%%%%%%%%%%%%%%%%%%%%%%%%%%%%%
\paragraph{v1.0:} 2017/04/27

\begin{itemize}
\item
manual and install package
\item
first version published on CTAN
\end{itemize}

%%%%%%%%%%%%%%%%%%%%%%%%%%%%%%%%%%%%%%%%
\paragraph{v0.6:} 2017/04/26

\begin{itemize}
\item
redirection mechanism added
\end{itemize}

%%%%%%%%%%%%%%%%%%%%%%%%%%%%%%%%%%%%%%%%
\paragraph{v0.5:} 2017/04/26

\begin{itemize}
\item
functionality in definition file
\end{itemize}


%%%%%%%%%%%%%%%%%%%%%%%%%%%%%%%%%%%%%%%%%%%%%%%%%%%%%%%%%%%%%%%%%%%%%%%%%%%%%%%%
%%%%%%%%%%%%%%%%%%%%%%%%%%%%%%%%%%%%%%%%%%%%%%%%%%%%%%%%%%%%%%%%%%%%%%%%%%%%%%%%
%%%%%%%%%%%%%%%%%%%%%%%%%%%%%%%%%%%%%%%%%%%%%%%%%%%%%%%%%%%%%%%%%%%%%%%%%%%%%%%%
\appendix

\settowidth\MacroIndent{\rmfamily\scriptsize 000\ }

 \DocInput{childdoc.dtx}

\end{document}
%</driver>
% \fi
%
% %%%%%%%%%%%%%%%%%%%%%%%%%%%%%%%%%%%%%%%%%%%%%%%%%%%%%%%%%%%%%%%%%%%%%%%%%%%%%%
% %%%%%%%%%%%%%%%%%%%%%%%%%%%%%%%%%%%%%%%%%%%%%%%%%%%%%%%%%%%%%%%%%%%%%%%%%%%%%%
% \section{Sample}
%\iffalse
%<*samplemain>
%\fi
%
% The following presents a sample document
% with two chapters, two parts, a title page,
% a compile flag as well as three forwarding files to set the flag.
% It consists of eight |.tex| files:
% \begin{center}
% \begin{tabular}{ll}
% |cdocsamp.tex|&main file\\
% |cdocsch1.tex|&include file for chapter 1\\
% |cdocsch2.tex|&include file for chapter 2\\
% |cdocspt3.tex|&include file for part 3\\
% |cdocspt4.tex|&include file for part 4\\
% |cdocsdrf.tex|&forwarding file for main file in draft mode\\
% |cdocsfi1.tex|&forwarding file for final version of chapter 1\\
% |cdocsfi2.tex|&forwarding file for final version of chapter 2\\
% \end{tabular}
% \end{center}
% Each of the eight files can be compiled directly by the \LaTeX{} compiler.
%
% %%%%%%%%%%%%%%%%%%%%%%%%%%%%%%%%%%%%%%
% \paragraph{Main File.}
%
% The main file is called |cdocsamp.tex|.
%
% Load the \textsf{childdoc} definitions and
% declare the filename for the main document:
%    \begin{macrocode}
\input{childdoc.def}
\childdocmain{}
%    \end{macrocode}

% Optional override for |\version| flag:
%    \begin{macrocode}
%%\ifchilddoc\else\providecommand{\version}{draft}\fi
%    \end{macrocode}

% Define the default values for the |\version| flag
% (|final| for the main file and |draft| for childs):
%    \begin{macrocode}
\ifchilddoc
\providecommand{\version}{draft}
\else
\providecommand{\version}{final}
\fi
%    \end{macrocode}

% Load the standard document class:
%    \begin{macrocode}
\documentclass[12pt]{article}
%    \end{macrocode}

% Start the document body:
%    \begin{macrocode}
\begin{document}
%    \end{macrocode}

% Declare a title page.
% Print title, part of document being processed and version flag:
%    \begin{macrocode}
\addtocounter{page}{-1}
\begin{center}
{\LARGE\bfseries{}childdoc example\par}
\vspace{1cm}
\ifchilddoc
\ifchilddocmanual part\else chapter\fi:
`\childdocname' of `\childdocjob'\par
\else
main document: `\childdocjob'\par
\fi
version: \version\par
\end{center}
\newpage
%    \end{macrocode}

% Manually include selected file,
% otherwise process as usual:
%    \begin{macrocode}
\ifchilddocmanual
\section*{part `\childdocname'}
\input{\childdocname}
\else
%    \end{macrocode}

% Include the two chapters:
%    \begin{macrocode}
\include{cdocsch1}
\include{cdocsch2}
%    \end{macrocode}

% Include the two parts unless only chapters should be displayed:
%    \begin{macrocode}
\ifchilddoc\else
\section{part three}
\input{cdocspt3}
\section{part four}
\input{cdocspt4}
\fi
%    \end{macrocode}

% Process as usual until here:
%    \begin{macrocode}
\fi
%    \end{macrocode}

% End of document body:
%    \begin{macrocode}
\end{document}
%    \end{macrocode}
%\iffalse
%</samplemain>
%\fi
%
% %%%%%%%%%%%%%%%%%%%%%%%%%%%%%%%%%%%%%%
% \paragraph{Chapter Include Files.}
%
% The include files are called |cdocsch1.tex| and |cdocsch2.tex|.
%
%\iffalse
%<*samplechap1|samplechap2>
%\fi

% Optional override for |\version| flag:
%    \begin{macrocode}
%%\providecommand{\version}{final}
%    \end{macrocode}

% Include the main document:
%    \begin{macrocode}
\input{childdoc.def}
\childdocof{cdocsamp}
%    \end{macrocode}

%\iffalse
%</samplechap1|samplechap2>
%\fi
%
%\iffalse
%<*samplechap1>
%\fi
% Some text for chapter 1:
%    \begin{macrocode}
\section{one}
some text in chapter one
%    \end{macrocode}

%\iffalse
%</samplechap1>
%\fi
% Some text for chapter 2:
%\iffalse
%<*samplechap2>
%\fi
%    \begin{macrocode}
\section{two}
more text in chapter two
%    \end{macrocode}

%\iffalse
%</samplechap2>
%\fi
%
% %%%%%%%%%%%%%%%%%%%%%%%%%%%%%%%%%%%%%%
% \paragraph{Part Include Files.}
%
% The include files are called |cdocspt3.tex| and |cdocspt4.tex|.
%
%\iffalse
%<*samplepart3|samplepart4>
%\fi

% Optional override for |\version| flag:
%    \begin{macrocode}
%%\providecommand{\version}{final}
%    \end{macrocode}

% Include the main document:
%    \begin{macrocode}
\input{childdoc.def}
\childdocby{cdocsamp}
%    \end{macrocode}

%\iffalse
%</samplepart3|samplepart4>
%\fi
%
%\iffalse
%<*samplepart3>
%\fi
% Some text for part 3:
%    \begin{macrocode}
some text in part three
%    \end{macrocode}

%\iffalse
%</samplepart3>
%\fi
% Some text for part 4:
%\iffalse
%<*samplepart4>
%\fi
%    \begin{macrocode}
more text in part four
%    \end{macrocode}

%\iffalse
%</samplepart4>
%\fi
%
% %%%%%%%%%%%%%%%%%%%%%%%%%%%%%%%%%%%%%%
% \paragraph{Forwarding for a Complete Draft.}
%
% The following forwarding file |cdocsdrf.tex|
% compiles the main document in draft mode:
%\iffalse
%<*sampledraft>
%\fi
%    \begin{macrocode}
\def\version{draft}
\input{childdoc.def}
\childdocforward{cdocsamp}
%    \end{macrocode}

%\iffalse
%</sampledraft>
%\fi
%
% %%%%%%%%%%%%%%%%%%%%%%%%%%%%%%%%%%%%%%
% \paragraph{Forwarding for Final Version of the Chapters.}
%
% The following forwarding files |cdocsfn1.tex| and |cdocsfn2.tex|
% (with identical content)
% compile the final versions of the child documents
% |cdocsch1.tex| and |cdocsch2.tex|, respectively:
%\iffalse
%<*samplefinal>
%\fi
%    \begin{macrocode}
\def\version{final}
\input{childdoc.def}
\childdocforwardprefix[cdocsamp]{cdocsfn}{cdocsch}
%    \end{macrocode}

%\iffalse
%</samplefinal>
%\fi
%
% %%%%%%%%%%%%%%%%%%%%%%%%%%%%%%%%%%%%%%
% \paragraph{Command Line Processing.}
%
% The following three command lines generate the output files
% |cdocscld|, |cdocscl1| and |cdocscl2|
% which should be identical to
% |cdocsdrf|, |cdocsch1| and |cdocsfn2|, respectively:
% \begin{center}
% \begin{tabular}{l}
% |latex -jobname cdocscld \|\\
% |  "\def\version{draft}\input{childdoc.def}\childdocforward{cdocsamp}"|\\
% |latex -jobname cdocscl1 \|\\
% |  "\input{childdoc.def}\childdocforward[cdocsamp]{cdocsch1}"|\\
% |latex -jobname cdocscl2 \|\\
% |  "\def\version{final}\input{childdoc.def}\childdocforward{cdocsch2}"|
% \end{tabular}
% \end{center}
% Note that the trailing backslash on each first line
% merely continues the input to the second line
% (for convenient cut ant paste).
% Furthermore, the command |latex| can be replaced by any
% of its alternative versions such as |pdflatex|.
%
% %%%%%%%%%%%%%%%%%%%%%%%%%%%%%%%%%%%%%%%%%%%%%%%%%%%%%%%%%%%%%%%%%%%%%%%%%%%%%%
% %%%%%%%%%%%%%%%%%%%%%%%%%%%%%%%%%%%%%%%%%%%%%%%%%%%%%%%%%%%%%%%%%%%%%%%%%%%%%%
% \section{Implementation}
%\iffalse
%<*package>
%\fi
%
% This section describes the definitions file |childdoc.def|.

% The definitions cannot be loaded using |\usepackage| or |\RequirePackage|
% which has a mechanism to prevent loading a style file more than once.
% When loading the definitions by means of |\input|
% multiple instances have to be prevented manually:
%\iffalse
%This code needs to be before the `\ProvidesFile' directive
%which is defined at the beginning of this file.
%Therefore it is also placed there and commented out here.
%</package>
%<*discard>
%\fi
%    \begin{macrocode}
\ifdefined\childdocmain\endinput\fi
%    \end{macrocode}
%\iffalse
%</discard>
%<*package>
%\fi
%
% \macro{\ifchilddoc}
% \macro{\ifchilddocmanual}
% The conditional |\ifchilddoc| tells whether a
% child (true) or main (false) document is being compiled.
% The conditional |\ifchilddocmanual| tells whether
% the |\includeonly| mechanism is used (false) or
% the selection of child files must be performed manually (true).
% The definitions initialise to false:
%    \begin{macrocode}
\newif\ifchilddoc
\newif\ifchilddocmanual
%    \end{macrocode}

% \macro{\childdocname}
% \macro{\childdocjob}
% The macro |\childdocname| stores the name of the main document
% to be compiled. The macro |\childdocjob| stores the name of
% the document on which the \LaTeX{} compiler was originally invoked.
% The content of |\jobname| cannot be compared
% to filenames specified in the source due to different catcodes.
% The following code rescans |\jobname|, stores the result
% in |\childdocname| and saves a copy in |\childdocjob|:
%    \begin{macrocode}
\edef\childdocname{\scantokens\expandafter{\jobname\noexpand}}
\let\childdocjob\childdocname
%    \end{macrocode}

% \macro{\childdocdisable}
% The macro |\childdocdisable| prevents the main file
% from being processed more than once.
% At this stage, the main document command |\childdocmain|
% is assumed to be called once again where it should do nothing.
% Any subsequent call to it should prevent
% a secondary processing of the main document
% It overwrites the forwarding commands
% |\childdocof| and |\childdocforward|
% with empty macros to prevent further inclusions of the main document:
%    \begin{macrocode}
\newcommand{\childdocdisable}
{
  \renewcommand{\childdocmain}[1]{\renewcommand{\childdocmain}[1]{\endinput}}
  \renewcommand{\childdocof}[1]{}
  \renewcommand{\childdocby}[2][]{}
  \renewcommand{\childdocforward}[2][]{}
  \renewcommand{\childdocdisable}{}
}
%    \end{macrocode}

% \macro{\childdocmain}
% The macro |\childdocmain| is to be called at the top of the main file
% with nothing or the main filename (without extension) as argument.
% First, it breaks loops.
% If the argument is not empty and does not match |\childdocname|
% (which is set by the first inclusion of |childdoc.def|),
% |\ifchilddoc| is set to true, |\includeonly| is applied to the child file
% and |\jobname| is set to the main file
% (for proper handling of |.aux| files):
%    \begin{macrocode}
\newcommand{\childdocmain}[1]
{
  \childdocdisable\childdocmain{}
  \if?#1?\else
    \begingroup
      \def\childdoctmp{#1}
      \ifx\childdoctmp\childdocname
        \def\childdoctmp{}
      \else
        \def\childdoctmp
        {
          \childdoctrue
          \includeonly{\childdocname}
          \def\childdocjob{#1}
          \def\jobname{#1}
        }
      \fi
      \expandafter
    \endgroup
    \childdoctmp
  \fi
}
%    \end{macrocode}

% \macro{\childdocof}
% The command |\childdocof| redirects
% compilation to the main file |#1|.
%    \begin{macrocode}
\newcommand{\childdocof}[1]
{
  \childdocdisable
  \childdoctrue
  \includeonly{\childdocname}
  \def\jobname{#1}
  \def\childdocjob{#1}
  \input{#1}
}
%    \end{macrocode}

% \macro{\childdocby}
% The command |\childdocby| ....
%    \begin{macrocode}
\newcommand{\childdocby}[2][]
{
  \childdocdisable
  \childdoctrue
  \childdocmanualtrue
  \if?#1?\else
    \def\jobname{#2}
  \fi
  \def\childdocjob{#2}
  \input{#2}
  \endinput
}
%    \end{macrocode}

% \macro{\childdocforward}
% The command |\childdocforward| redirects
% compilation to the main file or
% (if the optional argument is given) a child file.
% Parameters are set as if the main file
% or a child file starting with |\childdocof| was compiled.
% Then compilation is handed over to the main file:
%    \begin{macrocode}
\newcommand{\childdocforward}[2][]
{
  \begingroup
    \if?#1?
      \def\childdoctmp
      {
        \def\childdocname{#2}
        \def\childdocjob{#2}
        \def\jobname{#2}
        \input{#2}
        \endinput
      }
    \else
      \def\childdoctmp
      {
        \childdocdisable
        \def\childdocname{#2}
        \childdoctrue
        \includeonly{#2}
        \def\childdocjob{#1}
        \def\jobname{#1}
        \input{#1}
        \endinput
      }
    \fi
    \expandafter
  \endgroup
  \childdoctmp
}
%    \end{macrocode}

% \macro{\childdocforwardprefix}
% The command |\childdocforwardprefix| redirects
% compilation to the main or a child file by means of a pattern.
% The prefix |#1| in the current filename is replaced by |#2|
% and the suffix of the current filename is kept
% (it is assumed that the filename does not contain the substring `|~~~|'
% which is used as a delimiter).
% Compilation is handed over to the new file by |\childdocforward|:
%    \begin{macrocode}
\newcommand{\childdocforwardprefix}[3][]
{
  \begingroup
    \def\childdocextract #2##1~~~{\def\childdoctmp{\childdocforward[#1]{#3##1}}}
    \expandafter\childdocextract\childdocname~~~
    \expandafter
  \endgroup
  \childdoctmp
}
%    \end{macrocode}

% \macro{\childdoc}
% The deprecated macro |\childdoc| is a legacy version of |\childdocmain|:
%    \begin{macrocode}
\newcommand{\childdoc}{\childdocmain}
%    \end{macrocode}

% \macro{\childdocredirect}
% The deprecated macro |\childdocredirect| is a legacy version
% of |\childdocforward| and |\childdocforwardprefix|:
%    \begin{macrocode}
\newcommand{\childdocredirect}[2][]
{
  \begingroup
    \if?#1?
      \def\childdoctmp{\childdocforward{#2}}
    \else
      \def\childdoctmp{\childdocforwardprefix{#1}{#2}}
    \fi
    \expandafter
  \endgroup
  \childdoctmp
}
%    \end{macrocode}

%\iffalse
%</package>
%\fi
%
\endinput

\childdocforwardprefix[cdocsamp]{cdocsfn}{cdocsch}
%    \end{macrocode}

%\iffalse
%</samplefinal>
%\fi
%
% %%%%%%%%%%%%%%%%%%%%%%%%%%%%%%%%%%%%%%
% \paragraph{Command Line Processing.}
%
% The following three command lines generate the output files
% |cdocscld|, |cdocscl1| and |cdocscl2|
% which should be identical to
% |cdocsdrf|, |cdocsch1| and |cdocsfn2|, respectively:
% \begin{center}
% \begin{tabular}{l}
% |latex -jobname cdocscld \|\\
% |  "\def\version{draft}% \iffalse
%
% childdoc.dtx Copyright (C) 2017-2018 Niklas Beisert
%
% This work may be distributed and/or modified under the
% conditions of the LaTeX Project Public License, either version 1.3
% of this license or (at your option) any later version.
% The latest version of this license is in
%   http://www.latex-project.org/lppl.txt
% and version 1.3 or later is part of all distributions of LaTeX
% version 2005/12/01 or later.
%
% This work has the LPPL maintenance status `maintained'.
%
% The Current Maintainer of this work is Niklas Beisert.
%
% This work consists of the files childdoc.dtx and childdoc.ins
% and the derived files childdoc.def and cdocsamp.tex with
% cdocsch1.tex, cdocsch2.tex, cdocsdrf.tex, cdocsfn1.tex, cdocsfn2.tex.
%
%<package>\ifdefined\childdocmain\endinput\fi
%<package>\ProvidesFile{childdoc.def}[2018/12/30 v2.0 child document driver]
%<samplemain>\ProvidesFile{cdocsamp.tex}[2018/12/30 v2.0 sample for childdoc]
%<*driver>
%\ProvidesFile{childdoc.drv}[2018/12/30 v2.0 childdoc reference manual file]
\PassOptionsToClass{10pt,a4paper}{article}
\documentclass{ltxdoc}

\usepackage[margin=35mm]{geometry}
\usepackage{hyperref}
\usepackage{hyperxmp}
\usepackage[usenames]{color}

\hypersetup{colorlinks=true}
\hypersetup{pdfstartview=FitH}
\hypersetup{pdfpagemode=UseNone}
\hypersetup{pdfsource={}}
\hypersetup{pdflang={en-UK}}
\hypersetup{pdfcopyright={Copyright 2017-2018 Niklas Beisert.
  This work may be distributed and/or modified under the
  conditions of the LaTeX Project Public License, either version 1.3
  of this license or (at your option) any later version.}}
\hypersetup{pdflicenseurl={http://www.latex-project.org/lppl.txt}}
\hypersetup{pdfcontactaddress={ETH Zurich, ITP, HIT K,
  Wolfgang-Pauli-Strasse 27}}
\hypersetup{pdfcontactpostcode={8093}}
\hypersetup{pdfcontactcity={Zurich}}
\hypersetup{pdfcontactcountry={Switzerland}}
\hypersetup{pdfcontactemail={nbeisert@itp.phys.ethz.ch}}
\hypersetup{pdfcontacturl={http://people.phys.ethz.ch/\xmptilde nbeisert/}}

\newcommand{\secref}[1]{\hyperref[#1]{section \ref*{#1}}}

\parskip1ex
\parindent0pt
\let\olditemize\itemize
\def\itemize{\olditemize\parskip0pt}

\begin{document}

\title{The \textsf{childdoc} Package}
\hypersetup{pdftitle={The childdoc Package}}
\author{Niklas Beisert\\[2ex]
  Institut f\"ur Theoretische Physik\\
  Eidgen\"ossische Technische Hochschule Z\"urich\\
  Wolfgang-Pauli-Strasse 27, 8093 Z\"urich, Switzerland\\[1ex]
  \href{mailto:nbeisert@itp.phys.ethz.ch}
  {\texttt{nbeisert@itp.phys.ethz.ch}}}
\hypersetup{pdfauthor={Niklas Beisert}}
\hypersetup{pdfsubject={Manual for the LaTeX2e Package childdoc}}
\date{30 December 2018, \textsf{v2.0}}
\maketitle

\begin{abstract}\noindent
\textsf{childdoc} is a \LaTeXe{} package
that enables the direct compilation
of document sections included by |\include|
to individual files.
\end{abstract}

\begingroup
\parskip0ex
\tableofcontents
\endgroup

%%%%%%%%%%%%%%%%%%%%%%%%%%%%%%%%%%%%%%%%%%%%%%%%%%%%%%%%%%%%%%%%%%%%%%%%%%%%%%%%
%%%%%%%%%%%%%%%%%%%%%%%%%%%%%%%%%%%%%%%%%%%%%%%%%%%%%%%%%%%%%%%%%%%%%%%%%%%%%%%%
\section{Introduction}

\LaTeX{} provides a mechanism to structure a large document (such as a book)
into a main file and several child files (containing the chapters)
using the |\include| command.
This mechanism is beneficial for documents
which span hundreds of pages in order to
make the source file(s) more manageable.
Moreover, compilation can be restricted to
selected child files by means of the |\includeonly| command.
The latter feature can be used to reduce the compilation time while editing
(this was significantly more useful in the earlier days of \LaTeX{})
or to generate a smaller document which is easier to navigate.
Another application of |\includeonly| is to generate
documents consisting of selected parts of the complete document.

However, there are a few drawbacks of the plain |\include| mechanism:
\begin{itemize}
\item
The child files cannot be compiled on their own,
they can only be compiled via the main file.
A naive editing environment
(such as a text editor with an option
to have the current file processed by \LaTeX)
may require one to switch to the main file before compiling;
attempting to compile the child file produces errors.
\item
The main file must be modified (each time)
to adjust the |\includeonly| command
to the present needs. This easily leaves the main file in a messy state.
\item
The generated document will always carry the filename
of the main document. This is inconvenient if
several child files are to be compiled and
to be kept for distribution.
\end{itemize}

The present package provides a simple interface
to make child files individually compilable by \LaTeX{}.
Compiling a child file then has the same effect as compiling
the main file with an |\includeonly| command
to select the appropriate child.
Moreover the generated document will carry the name of the child
rather than the main file.
This resolves all three above issues.

This feature is meant to make the editing of books,
thesis documents and lecture notes somewhat more convenient.
However, the package can also be used efficiently for
composing a series of documents (such as exercise sheets)
which are typically distributed individually.
It then assists the author in generating the individual documents
(potentially in different versions)
as well as a document containing the collected series.
Another application is in developing style files
or other kinds of included material
where compilation of the style file could redirect
to a sample or test file.

%%%%%%%%%%%%%%%%%%%%%%%%%%%%%%%%%%%%%%%%%%%%%%%%%%%%%%%%%%%%%%%%%%%%%%%%%%%%%%%%
%%%%%%%%%%%%%%%%%%%%%%%%%%%%%%%%%%%%%%%%%%%%%%%%%%%%%%%%%%%%%%%%%%%%%%%%%%%%%%%%
\section{Usage}

First of all, the package \textsf{childdoc} is \emph{not} a standard
\LaTeXe{} |.sty| style file! Therefore it needs to be invoked in
a non-standard way.

%%%%%%%%%%%%%%%%%%%%%%%%%%%%%%%%%%%%%%%%%%%%%%%%%%%%%%%%%%%%%%%%%%%%%%%%%%%%%%%%
\subsection{Included Files}
\label{sec:include}

%%%%%%%%%%%%%%%%%%%%%%%%%%%%%%%%%%%%%%%%
\DescribeMacro{\childdocmain}
To use the package, add the commands
\begin{center}
\begin{tabular}{l}
|\input{childdoc.def}|\\
|\childdocmain{}|\\
\end{tabular}
\end{center}
at the very top of the main \LaTeX{} file,
in particular \emph{before} the |\documentclass| statement!
The argument of |\childdocmain| should be left empty
(but it must be present).

%%%%%%%%%%%%%%%%%%%%%%%%%%%%%%%%%%%%%%%%
\DescribeMacro{\childdocof}
Furthermore, add the commands
\begin{center}
\begin{tabular}{l}
|\input{childdoc.def}|\\
|\childdocof{|\textit{main}|}|\\
\end{tabular}
\end{center}
at the top of every child file \textit{child}
which is included by |\include{|\textit{child}|}|
from within the main file
(or at least for those files to be compiled individually).
The argument \textit{main} must be the filename of the main file.

There are a couple of
considerations in setting up the main and child documents:

%%%%%%%%%%%%%%%%%%%%%%%%%%%%%%%%%%%%%%%%
\paragraph{Restrictions.}

Please note the following restrictions:
\begin{itemize}
\item
|\childdocmain| must be called with one argument \textit{main}
to ensure compatibility with earlier version of the package.
It must either be empty (|\childdocmain{}|)
or precisely match the filename of the main file in which it is specified.
See \secref{sec:detection} for further information.
\item
The filename \textit{main} must be specified without the |.tex| extension.
\item
The filename \textit{main} is case sensitive
(even in case-insensitive file systems)
due to internal string comparison.
\item
The argument \textit{main} should be fully expanded, it cannot be a macro.
\item
Subdirectories and special characters should be avoided in filenames.
\item
The command |\childdocmain{|\textit{main}|}| must be followed by a whitespace.
It should not be followed immediately by another command
or by a comment mark `|%|'.
This is because the \TeX{} parser reads the token immediately following
the argument of |\childdocmain| and puts it
at the beginning of every child section;
however, a white\-space is ignored.
\end{itemize}

%%%%%%%%%%%%%%%%%%%%%%%%%%%%%%%%%%%%%%%%
\paragraph{Content of Main File.}

It is advisable to place all content in the child files included by |\include|.
Any output contained in the main file will appear in all child documents
unless suppressed manually;
it cannot be suppressed automatically by the |\includeonly| directive
and thus should normally be avoided.
A method to include some content in the main file
by means of conditional processing is described in \secref{sec:conditional}.

%%%%%%%%%%%%%%%%%%%%%%%%%%%%%%%%%%%%%%%%
\paragraph{Page Numbering.}

When only a part of the document is compiled,
the appropriate numbering of pages
(as well as other status parameters)
is determined from the |.aux| files.
The latter contain information from previous passes.
However this information needs to propagate through
all intermediate child documents.
Therefore the page numbering in child documents may well
be inconsistent until the complete document is compiled at least once.

A useful (if unconventional) way to always ensure a consistent
page numbering is to restart the numbering in each child document
and denote the pages by `\textit{child}|.|\textit{page}'
where \textit{child} represents the chapter/section number of the child file.
This can be achieved by the command
|\numberwithin{page}{|\textit{child}|}|
of the \textsf{amsmath} package
where \textit{child} can be |chapter| or |section|
depending on the chosen structuring.
Alternatively, one can modify the macro |\thepage| appropriately
and reset the counter |page| at the start of each child file.

%%%%%%%%%%%%%%%%%%%%%%%%%%%%%%%%%%%%%%%%%%%%%%%%%%%%%%%%%%%%%%%%%%%%%%%%%%%%%%%%
\subsection{Conditional Processing}
\label{sec:conditional}

The package provides a mechanism to compile different versions
of a document. To customise the versions further some conditional processing
can come in handy to distinguish which version is being compiled.
The package provides two macros to describe the compilation context:

%%%%%%%%%%%%%%%%%%%%%%%%%%%%%%%%%%%%%%%%
\DescribeMacro{\ifchilddoc}
The conditional |\ifchilddoc| distinguishes between the compilation of
child documents and the main document:
%
\begin{center}
|\ifchilddoc |\textit{child-code}| |[|\||else |\textit{main-code}]| \||fi|
\end{center}

%%%%%%%%%%%%%%%%%%%%%%%%%%%%%%%%%%%%%%%%
\DescribeMacro{\childdocname}
\DescribeMacro{\childdocjob}
The macro |\childdocname| contains the filename (without extension)
of the main or child file being processed.
Note that |\childdocjob| will always contain the name of the main file.

%%%%%%%%%%%%%%%%%%%%%%%%%%%%%%%%%%%%%%%%
\paragraph{Title Page.}

Conditional processing can be used to include a title or banner page
in the main document when proper precautions are taken.
Importantly, the code in the main file should ensure that the page counter
(as well as other status parameters which are stored in the |.aux| files)
takes the same value after the conditional processing.
Otherwise the page numbers may take divergent values
depending on which part is compiled.

For example, a title page could be declared by:
%
\begin{center}
\begin{tabular}{l}
|\ifchilddoc\||else|\\
|\addtocounter{page}{-1}|\\
\textit{code for title page}\\
|\newpage|\\
|\||fi|
\end{tabular}
\end{center}
%
A banner page for the child documents can be generated by:
%
\begin{center}
\begin{tabular}{l}
|\ifchilddoc|\\
|\addtocounter{page}{-1}|\\
\textit{code for banner page}\\
|\newpage|\\
|\||fi|
\end{tabular}
\end{center}
%
Here one could write a message such as:
\begin{center}
|This is the part \childdocname{} of \childdocjob{}.|
\end{center}

%%%%%%%%%%%%%%%%%%%%%%%%%%%%%%%%%%%%%%%%%%%%%%%%%%%%%%%%%%%%%%%%%%%%%%%%%%%%%%%%
\subsection{Flags}
\label{sec:flags}

The package makes it easy to generate different versions
of the main or child documents.
To this end compilation flags can be defined
and assigned different default values.
They will be particularly useful in conjunction
with the forwarding mechanism described in \secref{sec:forward}.

For example, it may be useful to have a flag |\version|
which can be set to |draft| or |final|.
The document source will contain some conditional code
depending on the value of |\version|.
Suppose further, the flag should default to |final| for the main file
and to |draft| for child files
which is a natural assignment for editing the document.
This is achieved by placing the following code
in the preamble of the main document
(below the |\childdocmain| directive):
%
\begin{center}
\begin{tabular}{l}
|\ifchilddoc|\\
|\providecommand{\version}{draft}|\\
|\||else|\\
|\providecommand{\version}{final}|\\
|\||fi|
\end{tabular}
\end{center}
%
The definition by |\providecommand| makes sure
that previous definitions are not overwritten.
Further statements |\providecommand{\version}{...}|
can thus be added before the above code to override it.

For the main file, one might add a line
(between |\childdocmain| and the above block)
%
\begin{center}
|%\ifchilddoc\||else\providecommand{\version}{draft}\||fi|
\end{center}
%
which can be uncommented to produce a draft version.
Likewise one can add a line to the very top of a child file
(above the |\childdocof{|\textit{main}|}| directive)
%
\begin{center}
|%\providecommand{\version}{final}|
\end{center}
%
which can be uncommented to produce the final version of this child document.

%%%%%%%%%%%%%%%%%%%%%%%%%%%%%%%%%%%%%%%%%%%%%%%%%%%%%%%%%%%%%%%%%%%%%%%%%%%%%%%%
\subsection{Forwarding}
\label{sec:forward}

Different versions of the main or child documents
using compilation flags as described in \secref{sec:flags}
can be (permanently) stored in different files
for convenient compilation, viewing and distribution.
To this end, the package defines a command
to pass on compilation to a different file:

%%%%%%%%%%%%%%%%%%%%%%%%%%%%%%%%%%%%%%%%
\DescribeMacro{\childdocforward}
The command |\childdocforward| redirects processing to
another source file:
%
\begin{center}
\begin{tabular}{l}
|\input{childdoc.def}|\\
|\childdocforward[|\textit{main}|]{|\textit{dest}|}|\\
\end{tabular}
\end{center}
%
The argument \textit{dest} is the destination file
(without extension).
It should be the main file or one of the child files.
Note that further \textsf{childdoc} directives
such as |\childdocof| and |\childdocforward|
in the indicated file will be processed in this form.
The optional argument \textit{main}
passes on directly to the main file \textit{main}
while pretending to compile the child \textit{dest}.
This form behaves as if \textit{dest}
issues |\childdocof{|\textit{main}|}| right away,
and no further \textsf{childdoc} directives will be processed.

%%%%%%%%%%%%%%%%%%%%%%%%%%%%%%%%%%%%%%%%
\DescribeMacro{\...prefix}
In the alternative form |\childdocforwardprefix|,
%
\begin{center}
\begin{tabular}{l}
|\input{childdoc.def}|\\
|\childdocforwardprefix[|\textit{main}|]{|\textit{prefix}|}{|\textit{dest}|}|
\end{tabular}
\end{center}
%
the destination file is determined by a pattern
depending on the current file:
To make this work, the current file must be called
`{\textit{prefix}\hspace{0.2em}\textit{suffix}}'
with \textit{prefix} matching precisely the argument.
Processing is then passed on to the file
`{\textit{dest}\hspace{0.2em}\textit{suffix}}'.
Surely, the same effect is achieved by
directly specifying the
argument `{\textit{dest}\hspace{0.2em}\textit{suffix}}'
in the first form.
However, that requires to set up a different file
for each child. With the alternative form of the command
all these files can have exactly the same content
which simplifies setting them up and maintaining them.

For example, the following file |draft.tex|
with a compilation flag |\version| as described in \secref{sec:flags}
compiles the main document as a draft:
%
\begin{center}
\begin{tabular}{l}
|\def\version{draft}|\\
|\input{childdoc.def}|\\
|\childdocforward{|\textit{main}|}|
\end{tabular}
\end{center}
%
Likewise, the following files |final|\textit{nn}|.tex|
compile the final version of the child document
|child|\textit{nn}|.tex|:
%
\begin{center}
\begin{tabular}{l}
|\def\version{final}|\\
|\input{childdoc.def}|\\
|\childdocforwardprefix{final}{child}|
\end{tabular}
\end{center}
%

Note that when several versions of a main file and/or of each child file
are to be generated, it may be convenient to set up a |Makefile| or
shell script to automatise the process.

%%%%%%%%%%%%%%%%%%%%%%%%%%%%%%%%%%%%%%%%%%%%%%%%%%%%%%%%%%%%%%%%%%%%%%%%%%%%%%%%
\subsection{Command Line Processing}
\label{sec:commandline}

The effect of redirection files can also be achieved by invoking
the \LaTeX{} compiler with a more elaborate command line.
Most conveniently this should be done as part
of a shell script or a |Makefile|.

When using \textsf{childdoc} in the main file, the following
command lines effectively perform a redirection
(note that depending on the shell being used,
backslashes may have to be doubled: `|\|' $\to$ `|\\|'):
%
\begin{center}
|... -jobname "|\textit{target}|" |\\|"|[\textit{flags}]%
|\input{childdoc.def}\childdocforward[|\textit{main}|]{|\textit{dest}|}"|
\end{center}
%
Here \textit{target} is the name of the output file,
\textit{main} is the name of the main file
and \textit{dest} is the name of the main or child file to be processed
(all filenames without extensions).
The optional argument \textit{main} can be omitted
if \textit{main} matches \textit{dest}.
Optionally, compilation \textit{flags} can be defined via |\def| commands.
This command line makes the \TeX{} engine believe
it is compiling the file \textit{target}
whose content is specified as the latter parameter.
The provided code then forwards the processing to
\textit{main} or \textit{dest} as described in \secref{sec:forward}.

%%%%%%%%%%%%%%%%%%%%%%%%%%%%%%%%%%%%%%%%%%%%%%%%%%%%%%%%%%%%%%%%%%%%%%%%%%%%%%%%
\subsection{Include by Input}
\label{sec:input}

Including child documents by |\include| has some restrictions by design.
Most notably, the content of a child document always occupies
its own set of pages; pages cannot be shared between child documents.
Usually, this behaviour makes perfect sense
because each child document contain an essential part of the document.
However, in some situations it may be desirable to compose
a document from a collection of parts
without having mandatory page breaks between then.
For this case, the package
provides a mechanism to include parts
by |\input| which can also be processed individually.
However, by construction this mechanism
requires manual handling of the content to be output.

%%%%%%%%%%%%%%%%%%%%%%%%%%%%%%%%%%%%%%%%
\DescribeMacro{\ifchilddocmanual}
The main file should be prepared as usual, see \secref{sec:include}.
However, the document body must make a distinction
between processing of an individual part and of the main document, e.g.:
%
\begin{center}
\begin{tabular}{l}
|\ifchilddocmanual|\\
|\input{\childdocname}|\\
|\||else|\\
\textit{document body with }|\input{|\textit{part}|}|\\
|\||fi|
\end{tabular}
\end{center}
%
The conditional |\ifchilddocmanual| is true whenever
a part to be included by |\input| is being compiled,
and the name of the part is stored in |\childdocname|.

%%%%%%%%%%%%%%%%%%%%%%%%%%%%%%%%%%%%%%%%
\DescribeMacro{\childdocby}
Each part to be included by |\input| should start with:
%
\begin{center}
\begin{tabular}{l}
|\input{childdoc.def}|\\
|\childdocby{|\textit{main}|}|\\
\end{tabular}
\end{center}
%
The directive |\childdocby| is similar to |\childdocof|
described in \secref{sec:include},
but the subsequent selection of content must be done manually.
To that end, both |\ifchilddoc| and |\ifchilddocmanual|
will be true upon processing of a part,
and the name of the part is stored in |\childdocname|.
Note that |\jobname| will be set to the filename of the current part
so that each part receives an individual |.aux| file
that does not interfere with the |.aux| file(s) of the main document.
This behaviour can be altered by the alternative form
|\childdocby[*]{|\textit{main}|}| (with a non-empty optional argument)
which uses the |.aux| file of the main document
by setting |\jobname| to \textit{main}.

%%%%%%%%%%%%%%%%%%%%%%%%%%%%%%%%%%%%%%%%%%%%%%%%%%%%%%%%%%%%%%%%%%%%%%%%%%%%%%%%
\subsection{Driver Development}
\label{sec:driver}

The \textsf{childdoc} mechanism can also be use for the development
of definition files such as \LaTeX{} styles or classes.
This case differs from the above setup with multiple parts
included by |\include| in that no |\includeonly| should be invoked.
This can be achieved by starting the include file
(before |\ProvidesPackage|) with:
%
\begin{center}
\begin{tabular}{l}
|\input{childdoc.def}|\\
|\childdocforward{|\textit{main}|}|\\
\end{tabular}
\end{center}
%
or alternatively with:
%
\begin{center}
\begin{tabular}{l}
|\input{childdoc.def}|\\
|\childdocby{|\textit{main}|}|\\
\end{tabular}
\end{center}
%
Both forms have slightly different effects as described above.
The main file is prepared as usual, see \secref{sec:include}.

%%%%%%%%%%%%%%%%%%%%%%%%%%%%%%%%%%%%%%%%%%%%%%%%%%%%%%%%%%%%%%%%%%%%%%%%%%%%%%%%
\subsection{Legacy Detection}
\label{sec:detection}

The directive |\childdocmain| in the main file can detect
whether the complete document or merely a child is to be compiled
even without using the directive |\childdocof|.
This method is deprecated because it is less robust
and there is no compelling reason to use it;
it is merely provided for backward compatibility
and it may be removed in future versions.

If the detection mechanism is to be used,
it is mandatory to correctly specify
the filename of the main file as the argument of |\childdocmain|:
%
\begin{center}
\begin{tabular}{l}
|\input{childdoc.def}|\\
|\childdocmain{|\textit{main}|}|\\
\end{tabular}
\end{center}
%
If |\jobname| does not match the argument \textit{main} of |\childdocmain|,
it is assumed that |\jobname| points to the child file to be compiled.
When using |\childdocmain| with the main file specified as argument,
it suffices to start a child file
with just |\input{|\textit{main}|}|
without loading of the package and using |\childdocof|.
If instead all processing is done
with the appropriate \textsf{childdoc} directives,
the argument of \textit{main} of |\childdocmain| can be empty.

An alternative version of the command line processing described
in \secref{sec:commandline} using the detection mechanism reads:
%
\begin{center}
|... -jobname "|\textit{target}|" "|[\textit{flags}]%
[|\def\jobname{|\textit{dest}|}|]|\input{|\textit{main}|}"|
\end{center}

%%%%%%%%%%%%%%%%%%%%%%%%%%%%%%%%%%%%%%%%%%%%%%%%%%%%%%%%%%%%%%%%%%%%%%%%%%%%%%%%
\subsection{Manual Code}
\label{sec:manual}

In case one cannot be certain whether the definitions file |childdoc.def|
is installed on the target \TeX{} distribution
and one prefers not to ship it,
it is conceivable to paste a few relevant commands into the sources.

To that end, drop all statements |\input{childdoc.def}|
and perform the replacements as outlined below.
Instead of |\childdocmain{|\textit{main}|}| add the following code
to the top of the main file:
%
\begin{center}
\begin{tabular}{l}
|\||ifdefined\childdocname\endinput\||fi\newif\ifchilddoc|\\
|\edef\childdocname{\scantokens\expandafter{\jobname\noexpand}}|\\
|\def\childdocmain{|\textit{main}|}\||ifx\childdocmain\childdocname\||else|\\
|\childdoctrue\includeonly{\childdocname}\let\jobname\childdocmain\||fi|\\
\end{tabular}
\end{center}
%
Instead of |\childdocof{|\textit{main}|}| just include the main file
at the top of each child file:
%
\begin{center}
|\input{|\textit{main}|}|
\end{center}
%
A simple redirection |\childdocforward{|\textit{dest}|}| is achieved by:
%
\begin{center}
|\def\jobname{|\textit{dest}|}\input{\jobname}|
\end{center}
%
The redirection with prefix
|\childdocforwardprefix[|\textit{prefix}|]{|\textit{dest}|}|
is accomplished by:
%
\begin{center}
\begin{tabular}{l}
|{\edef\jobname{\scantokens\expandafter{\jobname\noexpand}}|\\
|\def\redirectjob |\textit{prefix}|#1~~~{\gdef\jobname{|\textit{dest}|#1}}|\\
|\expandafter\redirectjob\jobname~~~}\input{\jobname}|
\end{tabular}
\end{center}

In an alternative approach,
child documents can be compiled by a specific command line
without additional code or specific definitions:
%
\begin{center}
|... -jobname "|\textit{target}|" "|[\textit{flags}]%
|\includeonly{|\textit{dest}|}\input{|\textit{main}|}"|
\end{center}
%

%%%%%%%%%%%%%%%%%%%%%%%%%%%%%%%%%%%%%%%%%%%%%%%%%%%%%%%%%%%%%%%%%%%%%%%%%%%%%%%%
%%%%%%%%%%%%%%%%%%%%%%%%%%%%%%%%%%%%%%%%%%%%%%%%%%%%%%%%%%%%%%%%%%%%%%%%%%%%%%%%
\section{Information}

%%%%%%%%%%%%%%%%%%%%%%%%%%%%%%%%%%%%%%%%%%%%%%%%%%%%%%%%%%%%%%%%%%%%%%%%%%%%%%%%
\subsection{Copyright}

Copyright \copyright{} 2017--2018 Niklas Beisert

This work may be distributed and/or modified under the
conditions of the \LaTeX{} Project Public License, either version 1.3
of this license or (at your option) any later version.
The latest version of this license is in
  \url{http://www.latex-project.org/lppl.txt}
and version 1.3 or later is part of all distributions of \LaTeX{}
version 2005/12/01 or later.

This work has the LPPL maintenance status `maintained'.

The Current Maintainer of this work is Niklas Beisert.

This work consists of the files |README.txt|, |childdoc.ins| and |childdoc.dtx|
as well as the derived files |childdoc.def|, |cdocsamp.tex|
with |cdocsch1.tex|, |cdocsch2.tex|, |cdocspt3.tex|, |cdocspt4.tex|,
|cdocsdrf.tex|, |cdocsfn1.tex|, |cdocsfn2.tex|
as well as |childdoc.pdf|.

%%%%%%%%%%%%%%%%%%%%%%%%%%%%%%%%%%%%%%%%%%%%%%%%%%%%%%%%%%%%%%%%%%%%%%%%%%%%%%%%
\subsection{Files and Installation}

The package consists of the files:
%
\begin{center}
\begin{tabular}{ll}
    |README.txt|   & readme file \\
    |childdoc.ins| & installation file \\
    |childdoc.dtx| & source file \\
    |childdoc.def| & definition file \\
    |cdocsamp.tex| & sample main file \\
    |cdocsch1.tex| & sample include file \\
    |cdocsch2.tex| & sample include file \\
    |cdocspt3.tex| & sample part file \\
    |cdocspt4.tex| & sample part file \\
    |cdocsdrf.tex| & sample redirection file \\
    |cdocsfn1.tex| & sample redirection file \\
    |cdocsfn2.tex| & sample redirection file \\
    |childdoc.pdf| & manual
\end{tabular}
\end{center}
%
The distribution consists of the files
|README.txt|, |childdoc.ins| and |childdoc.dtx|.
%
\begin{itemize}
\item
Run (pdf)\LaTeX{} on |childdoc.dtx|
to compile the manual |childdoc.pdf| (this file).
\item
Run \LaTeX{} on |childdoc.ins| to create the definitions file |childdoc.def|
and the sample |cdocsamp.tex| with include files
|cdocsch1.tex|, |cdocsch2.tex|, |cdocspt3.tex|, |cdocspt4.tex|,
|cdocsdrf.tex|, |cdocsfn1.tex|, |cdocsfn2.tex|.
Then copy the file |childdoc.def| to an appropriate directory of your \LaTeX{}
distribution, e.g.\ \textit{texmf-root}|/tex/latex/childdoc|.
\end{itemize}

%%%%%%%%%%%%%%%%%%%%%%%%%%%%%%%%%%%%%%%%%%%%%%%%%%%%%%%%%%%%%%%%%%%%%%%%%%%%%%%%
\subsection{Related CTAN Packages}

There are several other packages which offer a similar functionality:
%
\begin{itemize}
\item
The packages
\href{http://ctan.org/pkg/docmute}{\textsf{docmute}},
\href{http://ctan.org/pkg/includex}{\textsf{includex}} and
\href{http://ctan.org/pkg/standalone}{\textsf{standalone}}
provide commands to include only the document body of
a child file thus allowing both files to be compiled individually.
\item
The packages \href{http://ctan.org/pkg/subdocs}{\textsf{subdocs}}
and \href{http://ctan.org/pkg/subfiles}{\textsf{subfiles}}
provide structures in which the main and child documents can be
encapsulated and allowing them to be compiled individually.
The inclusion mechanism is different from the conventional |\include|.
\item
The package \href{http://ctan.org/pkg/combine}{\textsf{combine}}
is an elaborate solution to combine several documents into one.
\end{itemize}
%
See also the CTAN topic \href{http://ctan.org/topic/subdocs}{\textsf{subdocs}}
for further related packages.
The present package differs from the above solutions in that
a document structure constructed with the conventional |\include| mechanism
just needs two extra commands at the top of every file
such that all constituent files can be compiled individually.

%%%%%%%%%%%%%%%%%%%%%%%%%%%%%%%%%%%%%%%%%%%%%%%%%%%%%%%%%%%%%%%%%%%%%%%%%%%%%%%%
%\subsection{Feature Suggestions}
%
%The following is a list of features which may be useful for future
%versions of this package:
%%
%\begin{itemize}
%\item
%\ldots
%\end{itemize}

%%%%%%%%%%%%%%%%%%%%%%%%%%%%%%%%%%%%%%%%%%%%%%%%%%%%%%%%%%%%%%%%%%%%%%%%%%%%%%%%
\subsection{Revision History}

%%%%%%%%%%%%%%%%%%%%%%%%%%%%%%%%%%%%%%%%
\paragraph{v2.0:} 2018/12/30

\begin{itemize}
\item
immediate forward processing
\item
added |\childdocby| mechanism
\item
manual restructured
\end{itemize}

%%%%%%%%%%%%%%%%%%%%%%%%%%%%%%%%%%%%%%%%
\paragraph{v1.6:} 2018/01/17

\begin{itemize}
\item
application for development of include files
\item
corrections to manual
\end{itemize}

%%%%%%%%%%%%%%%%%%%%%%%%%%%%%%%%%%%%%%%%
\paragraph{v1.5:} 2017/05/21

\begin{itemize}
\item
more complete structuring introduced
\item
|\childdocof| introduced
\item
|\childdoc| renamed to |\childdocmain|
\item
|\childredirect| renamed to |\childdocforward| and |\childdocforwardprefix|
and functionality expanded
\end{itemize}

%%%%%%%%%%%%%%%%%%%%%%%%%%%%%%%%%%%%%%%%
\paragraph{v1.0:} 2017/04/27

\begin{itemize}
\item
manual and install package
\item
first version published on CTAN
\end{itemize}

%%%%%%%%%%%%%%%%%%%%%%%%%%%%%%%%%%%%%%%%
\paragraph{v0.6:} 2017/04/26

\begin{itemize}
\item
redirection mechanism added
\end{itemize}

%%%%%%%%%%%%%%%%%%%%%%%%%%%%%%%%%%%%%%%%
\paragraph{v0.5:} 2017/04/26

\begin{itemize}
\item
functionality in definition file
\end{itemize}


%%%%%%%%%%%%%%%%%%%%%%%%%%%%%%%%%%%%%%%%%%%%%%%%%%%%%%%%%%%%%%%%%%%%%%%%%%%%%%%%
%%%%%%%%%%%%%%%%%%%%%%%%%%%%%%%%%%%%%%%%%%%%%%%%%%%%%%%%%%%%%%%%%%%%%%%%%%%%%%%%
%%%%%%%%%%%%%%%%%%%%%%%%%%%%%%%%%%%%%%%%%%%%%%%%%%%%%%%%%%%%%%%%%%%%%%%%%%%%%%%%
\appendix

\settowidth\MacroIndent{\rmfamily\scriptsize 000\ }

 \DocInput{childdoc.dtx}

\end{document}
%</driver>
% \fi
%
% %%%%%%%%%%%%%%%%%%%%%%%%%%%%%%%%%%%%%%%%%%%%%%%%%%%%%%%%%%%%%%%%%%%%%%%%%%%%%%
% %%%%%%%%%%%%%%%%%%%%%%%%%%%%%%%%%%%%%%%%%%%%%%%%%%%%%%%%%%%%%%%%%%%%%%%%%%%%%%
% \section{Sample}
%\iffalse
%<*samplemain>
%\fi
%
% The following presents a sample document
% with two chapters, two parts, a title page,
% a compile flag as well as three forwarding files to set the flag.
% It consists of eight |.tex| files:
% \begin{center}
% \begin{tabular}{ll}
% |cdocsamp.tex|&main file\\
% |cdocsch1.tex|&include file for chapter 1\\
% |cdocsch2.tex|&include file for chapter 2\\
% |cdocspt3.tex|&include file for part 3\\
% |cdocspt4.tex|&include file for part 4\\
% |cdocsdrf.tex|&forwarding file for main file in draft mode\\
% |cdocsfi1.tex|&forwarding file for final version of chapter 1\\
% |cdocsfi2.tex|&forwarding file for final version of chapter 2\\
% \end{tabular}
% \end{center}
% Each of the eight files can be compiled directly by the \LaTeX{} compiler.
%
% %%%%%%%%%%%%%%%%%%%%%%%%%%%%%%%%%%%%%%
% \paragraph{Main File.}
%
% The main file is called |cdocsamp.tex|.
%
% Load the \textsf{childdoc} definitions and
% declare the filename for the main document:
%    \begin{macrocode}
\input{childdoc.def}
\childdocmain{}
%    \end{macrocode}

% Optional override for |\version| flag:
%    \begin{macrocode}
%%\ifchilddoc\else\providecommand{\version}{draft}\fi
%    \end{macrocode}

% Define the default values for the |\version| flag
% (|final| for the main file and |draft| for childs):
%    \begin{macrocode}
\ifchilddoc
\providecommand{\version}{draft}
\else
\providecommand{\version}{final}
\fi
%    \end{macrocode}

% Load the standard document class:
%    \begin{macrocode}
\documentclass[12pt]{article}
%    \end{macrocode}

% Start the document body:
%    \begin{macrocode}
\begin{document}
%    \end{macrocode}

% Declare a title page.
% Print title, part of document being processed and version flag:
%    \begin{macrocode}
\addtocounter{page}{-1}
\begin{center}
{\LARGE\bfseries{}childdoc example\par}
\vspace{1cm}
\ifchilddoc
\ifchilddocmanual part\else chapter\fi:
`\childdocname' of `\childdocjob'\par
\else
main document: `\childdocjob'\par
\fi
version: \version\par
\end{center}
\newpage
%    \end{macrocode}

% Manually include selected file,
% otherwise process as usual:
%    \begin{macrocode}
\ifchilddocmanual
\section*{part `\childdocname'}
\input{\childdocname}
\else
%    \end{macrocode}

% Include the two chapters:
%    \begin{macrocode}
\include{cdocsch1}
\include{cdocsch2}
%    \end{macrocode}

% Include the two parts unless only chapters should be displayed:
%    \begin{macrocode}
\ifchilddoc\else
\section{part three}
\input{cdocspt3}
\section{part four}
\input{cdocspt4}
\fi
%    \end{macrocode}

% Process as usual until here:
%    \begin{macrocode}
\fi
%    \end{macrocode}

% End of document body:
%    \begin{macrocode}
\end{document}
%    \end{macrocode}
%\iffalse
%</samplemain>
%\fi
%
% %%%%%%%%%%%%%%%%%%%%%%%%%%%%%%%%%%%%%%
% \paragraph{Chapter Include Files.}
%
% The include files are called |cdocsch1.tex| and |cdocsch2.tex|.
%
%\iffalse
%<*samplechap1|samplechap2>
%\fi

% Optional override for |\version| flag:
%    \begin{macrocode}
%%\providecommand{\version}{final}
%    \end{macrocode}

% Include the main document:
%    \begin{macrocode}
\input{childdoc.def}
\childdocof{cdocsamp}
%    \end{macrocode}

%\iffalse
%</samplechap1|samplechap2>
%\fi
%
%\iffalse
%<*samplechap1>
%\fi
% Some text for chapter 1:
%    \begin{macrocode}
\section{one}
some text in chapter one
%    \end{macrocode}

%\iffalse
%</samplechap1>
%\fi
% Some text for chapter 2:
%\iffalse
%<*samplechap2>
%\fi
%    \begin{macrocode}
\section{two}
more text in chapter two
%    \end{macrocode}

%\iffalse
%</samplechap2>
%\fi
%
% %%%%%%%%%%%%%%%%%%%%%%%%%%%%%%%%%%%%%%
% \paragraph{Part Include Files.}
%
% The include files are called |cdocspt3.tex| and |cdocspt4.tex|.
%
%\iffalse
%<*samplepart3|samplepart4>
%\fi

% Optional override for |\version| flag:
%    \begin{macrocode}
%%\providecommand{\version}{final}
%    \end{macrocode}

% Include the main document:
%    \begin{macrocode}
\input{childdoc.def}
\childdocby{cdocsamp}
%    \end{macrocode}

%\iffalse
%</samplepart3|samplepart4>
%\fi
%
%\iffalse
%<*samplepart3>
%\fi
% Some text for part 3:
%    \begin{macrocode}
some text in part three
%    \end{macrocode}

%\iffalse
%</samplepart3>
%\fi
% Some text for part 4:
%\iffalse
%<*samplepart4>
%\fi
%    \begin{macrocode}
more text in part four
%    \end{macrocode}

%\iffalse
%</samplepart4>
%\fi
%
% %%%%%%%%%%%%%%%%%%%%%%%%%%%%%%%%%%%%%%
% \paragraph{Forwarding for a Complete Draft.}
%
% The following forwarding file |cdocsdrf.tex|
% compiles the main document in draft mode:
%\iffalse
%<*sampledraft>
%\fi
%    \begin{macrocode}
\def\version{draft}
\input{childdoc.def}
\childdocforward{cdocsamp}
%    \end{macrocode}

%\iffalse
%</sampledraft>
%\fi
%
% %%%%%%%%%%%%%%%%%%%%%%%%%%%%%%%%%%%%%%
% \paragraph{Forwarding for Final Version of the Chapters.}
%
% The following forwarding files |cdocsfn1.tex| and |cdocsfn2.tex|
% (with identical content)
% compile the final versions of the child documents
% |cdocsch1.tex| and |cdocsch2.tex|, respectively:
%\iffalse
%<*samplefinal>
%\fi
%    \begin{macrocode}
\def\version{final}
\input{childdoc.def}
\childdocforwardprefix[cdocsamp]{cdocsfn}{cdocsch}
%    \end{macrocode}

%\iffalse
%</samplefinal>
%\fi
%
% %%%%%%%%%%%%%%%%%%%%%%%%%%%%%%%%%%%%%%
% \paragraph{Command Line Processing.}
%
% The following three command lines generate the output files
% |cdocscld|, |cdocscl1| and |cdocscl2|
% which should be identical to
% |cdocsdrf|, |cdocsch1| and |cdocsfn2|, respectively:
% \begin{center}
% \begin{tabular}{l}
% |latex -jobname cdocscld \|\\
% |  "\def\version{draft}\input{childdoc.def}\childdocforward{cdocsamp}"|\\
% |latex -jobname cdocscl1 \|\\
% |  "\input{childdoc.def}\childdocforward[cdocsamp]{cdocsch1}"|\\
% |latex -jobname cdocscl2 \|\\
% |  "\def\version{final}\input{childdoc.def}\childdocforward{cdocsch2}"|
% \end{tabular}
% \end{center}
% Note that the trailing backslash on each first line
% merely continues the input to the second line
% (for convenient cut ant paste).
% Furthermore, the command |latex| can be replaced by any
% of its alternative versions such as |pdflatex|.
%
% %%%%%%%%%%%%%%%%%%%%%%%%%%%%%%%%%%%%%%%%%%%%%%%%%%%%%%%%%%%%%%%%%%%%%%%%%%%%%%
% %%%%%%%%%%%%%%%%%%%%%%%%%%%%%%%%%%%%%%%%%%%%%%%%%%%%%%%%%%%%%%%%%%%%%%%%%%%%%%
% \section{Implementation}
%\iffalse
%<*package>
%\fi
%
% This section describes the definitions file |childdoc.def|.

% The definitions cannot be loaded using |\usepackage| or |\RequirePackage|
% which has a mechanism to prevent loading a style file more than once.
% When loading the definitions by means of |\input|
% multiple instances have to be prevented manually:
%\iffalse
%This code needs to be before the `\ProvidesFile' directive
%which is defined at the beginning of this file.
%Therefore it is also placed there and commented out here.
%</package>
%<*discard>
%\fi
%    \begin{macrocode}
\ifdefined\childdocmain\endinput\fi
%    \end{macrocode}
%\iffalse
%</discard>
%<*package>
%\fi
%
% \macro{\ifchilddoc}
% \macro{\ifchilddocmanual}
% The conditional |\ifchilddoc| tells whether a
% child (true) or main (false) document is being compiled.
% The conditional |\ifchilddocmanual| tells whether
% the |\includeonly| mechanism is used (false) or
% the selection of child files must be performed manually (true).
% The definitions initialise to false:
%    \begin{macrocode}
\newif\ifchilddoc
\newif\ifchilddocmanual
%    \end{macrocode}

% \macro{\childdocname}
% \macro{\childdocjob}
% The macro |\childdocname| stores the name of the main document
% to be compiled. The macro |\childdocjob| stores the name of
% the document on which the \LaTeX{} compiler was originally invoked.
% The content of |\jobname| cannot be compared
% to filenames specified in the source due to different catcodes.
% The following code rescans |\jobname|, stores the result
% in |\childdocname| and saves a copy in |\childdocjob|:
%    \begin{macrocode}
\edef\childdocname{\scantokens\expandafter{\jobname\noexpand}}
\let\childdocjob\childdocname
%    \end{macrocode}

% \macro{\childdocdisable}
% The macro |\childdocdisable| prevents the main file
% from being processed more than once.
% At this stage, the main document command |\childdocmain|
% is assumed to be called once again where it should do nothing.
% Any subsequent call to it should prevent
% a secondary processing of the main document
% It overwrites the forwarding commands
% |\childdocof| and |\childdocforward|
% with empty macros to prevent further inclusions of the main document:
%    \begin{macrocode}
\newcommand{\childdocdisable}
{
  \renewcommand{\childdocmain}[1]{\renewcommand{\childdocmain}[1]{\endinput}}
  \renewcommand{\childdocof}[1]{}
  \renewcommand{\childdocby}[2][]{}
  \renewcommand{\childdocforward}[2][]{}
  \renewcommand{\childdocdisable}{}
}
%    \end{macrocode}

% \macro{\childdocmain}
% The macro |\childdocmain| is to be called at the top of the main file
% with nothing or the main filename (without extension) as argument.
% First, it breaks loops.
% If the argument is not empty and does not match |\childdocname|
% (which is set by the first inclusion of |childdoc.def|),
% |\ifchilddoc| is set to true, |\includeonly| is applied to the child file
% and |\jobname| is set to the main file
% (for proper handling of |.aux| files):
%    \begin{macrocode}
\newcommand{\childdocmain}[1]
{
  \childdocdisable\childdocmain{}
  \if?#1?\else
    \begingroup
      \def\childdoctmp{#1}
      \ifx\childdoctmp\childdocname
        \def\childdoctmp{}
      \else
        \def\childdoctmp
        {
          \childdoctrue
          \includeonly{\childdocname}
          \def\childdocjob{#1}
          \def\jobname{#1}
        }
      \fi
      \expandafter
    \endgroup
    \childdoctmp
  \fi
}
%    \end{macrocode}

% \macro{\childdocof}
% The command |\childdocof| redirects
% compilation to the main file |#1|.
%    \begin{macrocode}
\newcommand{\childdocof}[1]
{
  \childdocdisable
  \childdoctrue
  \includeonly{\childdocname}
  \def\jobname{#1}
  \def\childdocjob{#1}
  \input{#1}
}
%    \end{macrocode}

% \macro{\childdocby}
% The command |\childdocby| ....
%    \begin{macrocode}
\newcommand{\childdocby}[2][]
{
  \childdocdisable
  \childdoctrue
  \childdocmanualtrue
  \if?#1?\else
    \def\jobname{#2}
  \fi
  \def\childdocjob{#2}
  \input{#2}
  \endinput
}
%    \end{macrocode}

% \macro{\childdocforward}
% The command |\childdocforward| redirects
% compilation to the main file or
% (if the optional argument is given) a child file.
% Parameters are set as if the main file
% or a child file starting with |\childdocof| was compiled.
% Then compilation is handed over to the main file:
%    \begin{macrocode}
\newcommand{\childdocforward}[2][]
{
  \begingroup
    \if?#1?
      \def\childdoctmp
      {
        \def\childdocname{#2}
        \def\childdocjob{#2}
        \def\jobname{#2}
        \input{#2}
        \endinput
      }
    \else
      \def\childdoctmp
      {
        \childdocdisable
        \def\childdocname{#2}
        \childdoctrue
        \includeonly{#2}
        \def\childdocjob{#1}
        \def\jobname{#1}
        \input{#1}
        \endinput
      }
    \fi
    \expandafter
  \endgroup
  \childdoctmp
}
%    \end{macrocode}

% \macro{\childdocforwardprefix}
% The command |\childdocforwardprefix| redirects
% compilation to the main or a child file by means of a pattern.
% The prefix |#1| in the current filename is replaced by |#2|
% and the suffix of the current filename is kept
% (it is assumed that the filename does not contain the substring `|~~~|'
% which is used as a delimiter).
% Compilation is handed over to the new file by |\childdocforward|:
%    \begin{macrocode}
\newcommand{\childdocforwardprefix}[3][]
{
  \begingroup
    \def\childdocextract #2##1~~~{\def\childdoctmp{\childdocforward[#1]{#3##1}}}
    \expandafter\childdocextract\childdocname~~~
    \expandafter
  \endgroup
  \childdoctmp
}
%    \end{macrocode}

% \macro{\childdoc}
% The deprecated macro |\childdoc| is a legacy version of |\childdocmain|:
%    \begin{macrocode}
\newcommand{\childdoc}{\childdocmain}
%    \end{macrocode}

% \macro{\childdocredirect}
% The deprecated macro |\childdocredirect| is a legacy version
% of |\childdocforward| and |\childdocforwardprefix|:
%    \begin{macrocode}
\newcommand{\childdocredirect}[2][]
{
  \begingroup
    \if?#1?
      \def\childdoctmp{\childdocforward{#2}}
    \else
      \def\childdoctmp{\childdocforwardprefix{#1}{#2}}
    \fi
    \expandafter
  \endgroup
  \childdoctmp
}
%    \end{macrocode}

%\iffalse
%</package>
%\fi
%
\endinput
\childdocforward{cdocsamp}"|\\
% |latex -jobname cdocscl1 \|\\
% |  "% \iffalse
%
% childdoc.dtx Copyright (C) 2017-2018 Niklas Beisert
%
% This work may be distributed and/or modified under the
% conditions of the LaTeX Project Public License, either version 1.3
% of this license or (at your option) any later version.
% The latest version of this license is in
%   http://www.latex-project.org/lppl.txt
% and version 1.3 or later is part of all distributions of LaTeX
% version 2005/12/01 or later.
%
% This work has the LPPL maintenance status `maintained'.
%
% The Current Maintainer of this work is Niklas Beisert.
%
% This work consists of the files childdoc.dtx and childdoc.ins
% and the derived files childdoc.def and cdocsamp.tex with
% cdocsch1.tex, cdocsch2.tex, cdocsdrf.tex, cdocsfn1.tex, cdocsfn2.tex.
%
%<package>\ifdefined\childdocmain\endinput\fi
%<package>\ProvidesFile{childdoc.def}[2018/12/30 v2.0 child document driver]
%<samplemain>\ProvidesFile{cdocsamp.tex}[2018/12/30 v2.0 sample for childdoc]
%<*driver>
%\ProvidesFile{childdoc.drv}[2018/12/30 v2.0 childdoc reference manual file]
\PassOptionsToClass{10pt,a4paper}{article}
\documentclass{ltxdoc}

\usepackage[margin=35mm]{geometry}
\usepackage{hyperref}
\usepackage{hyperxmp}
\usepackage[usenames]{color}

\hypersetup{colorlinks=true}
\hypersetup{pdfstartview=FitH}
\hypersetup{pdfpagemode=UseNone}
\hypersetup{pdfsource={}}
\hypersetup{pdflang={en-UK}}
\hypersetup{pdfcopyright={Copyright 2017-2018 Niklas Beisert.
  This work may be distributed and/or modified under the
  conditions of the LaTeX Project Public License, either version 1.3
  of this license or (at your option) any later version.}}
\hypersetup{pdflicenseurl={http://www.latex-project.org/lppl.txt}}
\hypersetup{pdfcontactaddress={ETH Zurich, ITP, HIT K,
  Wolfgang-Pauli-Strasse 27}}
\hypersetup{pdfcontactpostcode={8093}}
\hypersetup{pdfcontactcity={Zurich}}
\hypersetup{pdfcontactcountry={Switzerland}}
\hypersetup{pdfcontactemail={nbeisert@itp.phys.ethz.ch}}
\hypersetup{pdfcontacturl={http://people.phys.ethz.ch/\xmptilde nbeisert/}}

\newcommand{\secref}[1]{\hyperref[#1]{section \ref*{#1}}}

\parskip1ex
\parindent0pt
\let\olditemize\itemize
\def\itemize{\olditemize\parskip0pt}

\begin{document}

\title{The \textsf{childdoc} Package}
\hypersetup{pdftitle={The childdoc Package}}
\author{Niklas Beisert\\[2ex]
  Institut f\"ur Theoretische Physik\\
  Eidgen\"ossische Technische Hochschule Z\"urich\\
  Wolfgang-Pauli-Strasse 27, 8093 Z\"urich, Switzerland\\[1ex]
  \href{mailto:nbeisert@itp.phys.ethz.ch}
  {\texttt{nbeisert@itp.phys.ethz.ch}}}
\hypersetup{pdfauthor={Niklas Beisert}}
\hypersetup{pdfsubject={Manual for the LaTeX2e Package childdoc}}
\date{30 December 2018, \textsf{v2.0}}
\maketitle

\begin{abstract}\noindent
\textsf{childdoc} is a \LaTeXe{} package
that enables the direct compilation
of document sections included by |\include|
to individual files.
\end{abstract}

\begingroup
\parskip0ex
\tableofcontents
\endgroup

%%%%%%%%%%%%%%%%%%%%%%%%%%%%%%%%%%%%%%%%%%%%%%%%%%%%%%%%%%%%%%%%%%%%%%%%%%%%%%%%
%%%%%%%%%%%%%%%%%%%%%%%%%%%%%%%%%%%%%%%%%%%%%%%%%%%%%%%%%%%%%%%%%%%%%%%%%%%%%%%%
\section{Introduction}

\LaTeX{} provides a mechanism to structure a large document (such as a book)
into a main file and several child files (containing the chapters)
using the |\include| command.
This mechanism is beneficial for documents
which span hundreds of pages in order to
make the source file(s) more manageable.
Moreover, compilation can be restricted to
selected child files by means of the |\includeonly| command.
The latter feature can be used to reduce the compilation time while editing
(this was significantly more useful in the earlier days of \LaTeX{})
or to generate a smaller document which is easier to navigate.
Another application of |\includeonly| is to generate
documents consisting of selected parts of the complete document.

However, there are a few drawbacks of the plain |\include| mechanism:
\begin{itemize}
\item
The child files cannot be compiled on their own,
they can only be compiled via the main file.
A naive editing environment
(such as a text editor with an option
to have the current file processed by \LaTeX)
may require one to switch to the main file before compiling;
attempting to compile the child file produces errors.
\item
The main file must be modified (each time)
to adjust the |\includeonly| command
to the present needs. This easily leaves the main file in a messy state.
\item
The generated document will always carry the filename
of the main document. This is inconvenient if
several child files are to be compiled and
to be kept for distribution.
\end{itemize}

The present package provides a simple interface
to make child files individually compilable by \LaTeX{}.
Compiling a child file then has the same effect as compiling
the main file with an |\includeonly| command
to select the appropriate child.
Moreover the generated document will carry the name of the child
rather than the main file.
This resolves all three above issues.

This feature is meant to make the editing of books,
thesis documents and lecture notes somewhat more convenient.
However, the package can also be used efficiently for
composing a series of documents (such as exercise sheets)
which are typically distributed individually.
It then assists the author in generating the individual documents
(potentially in different versions)
as well as a document containing the collected series.
Another application is in developing style files
or other kinds of included material
where compilation of the style file could redirect
to a sample or test file.

%%%%%%%%%%%%%%%%%%%%%%%%%%%%%%%%%%%%%%%%%%%%%%%%%%%%%%%%%%%%%%%%%%%%%%%%%%%%%%%%
%%%%%%%%%%%%%%%%%%%%%%%%%%%%%%%%%%%%%%%%%%%%%%%%%%%%%%%%%%%%%%%%%%%%%%%%%%%%%%%%
\section{Usage}

First of all, the package \textsf{childdoc} is \emph{not} a standard
\LaTeXe{} |.sty| style file! Therefore it needs to be invoked in
a non-standard way.

%%%%%%%%%%%%%%%%%%%%%%%%%%%%%%%%%%%%%%%%%%%%%%%%%%%%%%%%%%%%%%%%%%%%%%%%%%%%%%%%
\subsection{Included Files}
\label{sec:include}

%%%%%%%%%%%%%%%%%%%%%%%%%%%%%%%%%%%%%%%%
\DescribeMacro{\childdocmain}
To use the package, add the commands
\begin{center}
\begin{tabular}{l}
|\input{childdoc.def}|\\
|\childdocmain{}|\\
\end{tabular}
\end{center}
at the very top of the main \LaTeX{} file,
in particular \emph{before} the |\documentclass| statement!
The argument of |\childdocmain| should be left empty
(but it must be present).

%%%%%%%%%%%%%%%%%%%%%%%%%%%%%%%%%%%%%%%%
\DescribeMacro{\childdocof}
Furthermore, add the commands
\begin{center}
\begin{tabular}{l}
|\input{childdoc.def}|\\
|\childdocof{|\textit{main}|}|\\
\end{tabular}
\end{center}
at the top of every child file \textit{child}
which is included by |\include{|\textit{child}|}|
from within the main file
(or at least for those files to be compiled individually).
The argument \textit{main} must be the filename of the main file.

There are a couple of
considerations in setting up the main and child documents:

%%%%%%%%%%%%%%%%%%%%%%%%%%%%%%%%%%%%%%%%
\paragraph{Restrictions.}

Please note the following restrictions:
\begin{itemize}
\item
|\childdocmain| must be called with one argument \textit{main}
to ensure compatibility with earlier version of the package.
It must either be empty (|\childdocmain{}|)
or precisely match the filename of the main file in which it is specified.
See \secref{sec:detection} for further information.
\item
The filename \textit{main} must be specified without the |.tex| extension.
\item
The filename \textit{main} is case sensitive
(even in case-insensitive file systems)
due to internal string comparison.
\item
The argument \textit{main} should be fully expanded, it cannot be a macro.
\item
Subdirectories and special characters should be avoided in filenames.
\item
The command |\childdocmain{|\textit{main}|}| must be followed by a whitespace.
It should not be followed immediately by another command
or by a comment mark `|%|'.
This is because the \TeX{} parser reads the token immediately following
the argument of |\childdocmain| and puts it
at the beginning of every child section;
however, a white\-space is ignored.
\end{itemize}

%%%%%%%%%%%%%%%%%%%%%%%%%%%%%%%%%%%%%%%%
\paragraph{Content of Main File.}

It is advisable to place all content in the child files included by |\include|.
Any output contained in the main file will appear in all child documents
unless suppressed manually;
it cannot be suppressed automatically by the |\includeonly| directive
and thus should normally be avoided.
A method to include some content in the main file
by means of conditional processing is described in \secref{sec:conditional}.

%%%%%%%%%%%%%%%%%%%%%%%%%%%%%%%%%%%%%%%%
\paragraph{Page Numbering.}

When only a part of the document is compiled,
the appropriate numbering of pages
(as well as other status parameters)
is determined from the |.aux| files.
The latter contain information from previous passes.
However this information needs to propagate through
all intermediate child documents.
Therefore the page numbering in child documents may well
be inconsistent until the complete document is compiled at least once.

A useful (if unconventional) way to always ensure a consistent
page numbering is to restart the numbering in each child document
and denote the pages by `\textit{child}|.|\textit{page}'
where \textit{child} represents the chapter/section number of the child file.
This can be achieved by the command
|\numberwithin{page}{|\textit{child}|}|
of the \textsf{amsmath} package
where \textit{child} can be |chapter| or |section|
depending on the chosen structuring.
Alternatively, one can modify the macro |\thepage| appropriately
and reset the counter |page| at the start of each child file.

%%%%%%%%%%%%%%%%%%%%%%%%%%%%%%%%%%%%%%%%%%%%%%%%%%%%%%%%%%%%%%%%%%%%%%%%%%%%%%%%
\subsection{Conditional Processing}
\label{sec:conditional}

The package provides a mechanism to compile different versions
of a document. To customise the versions further some conditional processing
can come in handy to distinguish which version is being compiled.
The package provides two macros to describe the compilation context:

%%%%%%%%%%%%%%%%%%%%%%%%%%%%%%%%%%%%%%%%
\DescribeMacro{\ifchilddoc}
The conditional |\ifchilddoc| distinguishes between the compilation of
child documents and the main document:
%
\begin{center}
|\ifchilddoc |\textit{child-code}| |[|\||else |\textit{main-code}]| \||fi|
\end{center}

%%%%%%%%%%%%%%%%%%%%%%%%%%%%%%%%%%%%%%%%
\DescribeMacro{\childdocname}
\DescribeMacro{\childdocjob}
The macro |\childdocname| contains the filename (without extension)
of the main or child file being processed.
Note that |\childdocjob| will always contain the name of the main file.

%%%%%%%%%%%%%%%%%%%%%%%%%%%%%%%%%%%%%%%%
\paragraph{Title Page.}

Conditional processing can be used to include a title or banner page
in the main document when proper precautions are taken.
Importantly, the code in the main file should ensure that the page counter
(as well as other status parameters which are stored in the |.aux| files)
takes the same value after the conditional processing.
Otherwise the page numbers may take divergent values
depending on which part is compiled.

For example, a title page could be declared by:
%
\begin{center}
\begin{tabular}{l}
|\ifchilddoc\||else|\\
|\addtocounter{page}{-1}|\\
\textit{code for title page}\\
|\newpage|\\
|\||fi|
\end{tabular}
\end{center}
%
A banner page for the child documents can be generated by:
%
\begin{center}
\begin{tabular}{l}
|\ifchilddoc|\\
|\addtocounter{page}{-1}|\\
\textit{code for banner page}\\
|\newpage|\\
|\||fi|
\end{tabular}
\end{center}
%
Here one could write a message such as:
\begin{center}
|This is the part \childdocname{} of \childdocjob{}.|
\end{center}

%%%%%%%%%%%%%%%%%%%%%%%%%%%%%%%%%%%%%%%%%%%%%%%%%%%%%%%%%%%%%%%%%%%%%%%%%%%%%%%%
\subsection{Flags}
\label{sec:flags}

The package makes it easy to generate different versions
of the main or child documents.
To this end compilation flags can be defined
and assigned different default values.
They will be particularly useful in conjunction
with the forwarding mechanism described in \secref{sec:forward}.

For example, it may be useful to have a flag |\version|
which can be set to |draft| or |final|.
The document source will contain some conditional code
depending on the value of |\version|.
Suppose further, the flag should default to |final| for the main file
and to |draft| for child files
which is a natural assignment for editing the document.
This is achieved by placing the following code
in the preamble of the main document
(below the |\childdocmain| directive):
%
\begin{center}
\begin{tabular}{l}
|\ifchilddoc|\\
|\providecommand{\version}{draft}|\\
|\||else|\\
|\providecommand{\version}{final}|\\
|\||fi|
\end{tabular}
\end{center}
%
The definition by |\providecommand| makes sure
that previous definitions are not overwritten.
Further statements |\providecommand{\version}{...}|
can thus be added before the above code to override it.

For the main file, one might add a line
(between |\childdocmain| and the above block)
%
\begin{center}
|%\ifchilddoc\||else\providecommand{\version}{draft}\||fi|
\end{center}
%
which can be uncommented to produce a draft version.
Likewise one can add a line to the very top of a child file
(above the |\childdocof{|\textit{main}|}| directive)
%
\begin{center}
|%\providecommand{\version}{final}|
\end{center}
%
which can be uncommented to produce the final version of this child document.

%%%%%%%%%%%%%%%%%%%%%%%%%%%%%%%%%%%%%%%%%%%%%%%%%%%%%%%%%%%%%%%%%%%%%%%%%%%%%%%%
\subsection{Forwarding}
\label{sec:forward}

Different versions of the main or child documents
using compilation flags as described in \secref{sec:flags}
can be (permanently) stored in different files
for convenient compilation, viewing and distribution.
To this end, the package defines a command
to pass on compilation to a different file:

%%%%%%%%%%%%%%%%%%%%%%%%%%%%%%%%%%%%%%%%
\DescribeMacro{\childdocforward}
The command |\childdocforward| redirects processing to
another source file:
%
\begin{center}
\begin{tabular}{l}
|\input{childdoc.def}|\\
|\childdocforward[|\textit{main}|]{|\textit{dest}|}|\\
\end{tabular}
\end{center}
%
The argument \textit{dest} is the destination file
(without extension).
It should be the main file or one of the child files.
Note that further \textsf{childdoc} directives
such as |\childdocof| and |\childdocforward|
in the indicated file will be processed in this form.
The optional argument \textit{main}
passes on directly to the main file \textit{main}
while pretending to compile the child \textit{dest}.
This form behaves as if \textit{dest}
issues |\childdocof{|\textit{main}|}| right away,
and no further \textsf{childdoc} directives will be processed.

%%%%%%%%%%%%%%%%%%%%%%%%%%%%%%%%%%%%%%%%
\DescribeMacro{\...prefix}
In the alternative form |\childdocforwardprefix|,
%
\begin{center}
\begin{tabular}{l}
|\input{childdoc.def}|\\
|\childdocforwardprefix[|\textit{main}|]{|\textit{prefix}|}{|\textit{dest}|}|
\end{tabular}
\end{center}
%
the destination file is determined by a pattern
depending on the current file:
To make this work, the current file must be called
`{\textit{prefix}\hspace{0.2em}\textit{suffix}}'
with \textit{prefix} matching precisely the argument.
Processing is then passed on to the file
`{\textit{dest}\hspace{0.2em}\textit{suffix}}'.
Surely, the same effect is achieved by
directly specifying the
argument `{\textit{dest}\hspace{0.2em}\textit{suffix}}'
in the first form.
However, that requires to set up a different file
for each child. With the alternative form of the command
all these files can have exactly the same content
which simplifies setting them up and maintaining them.

For example, the following file |draft.tex|
with a compilation flag |\version| as described in \secref{sec:flags}
compiles the main document as a draft:
%
\begin{center}
\begin{tabular}{l}
|\def\version{draft}|\\
|\input{childdoc.def}|\\
|\childdocforward{|\textit{main}|}|
\end{tabular}
\end{center}
%
Likewise, the following files |final|\textit{nn}|.tex|
compile the final version of the child document
|child|\textit{nn}|.tex|:
%
\begin{center}
\begin{tabular}{l}
|\def\version{final}|\\
|\input{childdoc.def}|\\
|\childdocforwardprefix{final}{child}|
\end{tabular}
\end{center}
%

Note that when several versions of a main file and/or of each child file
are to be generated, it may be convenient to set up a |Makefile| or
shell script to automatise the process.

%%%%%%%%%%%%%%%%%%%%%%%%%%%%%%%%%%%%%%%%%%%%%%%%%%%%%%%%%%%%%%%%%%%%%%%%%%%%%%%%
\subsection{Command Line Processing}
\label{sec:commandline}

The effect of redirection files can also be achieved by invoking
the \LaTeX{} compiler with a more elaborate command line.
Most conveniently this should be done as part
of a shell script or a |Makefile|.

When using \textsf{childdoc} in the main file, the following
command lines effectively perform a redirection
(note that depending on the shell being used,
backslashes may have to be doubled: `|\|' $\to$ `|\\|'):
%
\begin{center}
|... -jobname "|\textit{target}|" |\\|"|[\textit{flags}]%
|\input{childdoc.def}\childdocforward[|\textit{main}|]{|\textit{dest}|}"|
\end{center}
%
Here \textit{target} is the name of the output file,
\textit{main} is the name of the main file
and \textit{dest} is the name of the main or child file to be processed
(all filenames without extensions).
The optional argument \textit{main} can be omitted
if \textit{main} matches \textit{dest}.
Optionally, compilation \textit{flags} can be defined via |\def| commands.
This command line makes the \TeX{} engine believe
it is compiling the file \textit{target}
whose content is specified as the latter parameter.
The provided code then forwards the processing to
\textit{main} or \textit{dest} as described in \secref{sec:forward}.

%%%%%%%%%%%%%%%%%%%%%%%%%%%%%%%%%%%%%%%%%%%%%%%%%%%%%%%%%%%%%%%%%%%%%%%%%%%%%%%%
\subsection{Include by Input}
\label{sec:input}

Including child documents by |\include| has some restrictions by design.
Most notably, the content of a child document always occupies
its own set of pages; pages cannot be shared between child documents.
Usually, this behaviour makes perfect sense
because each child document contain an essential part of the document.
However, in some situations it may be desirable to compose
a document from a collection of parts
without having mandatory page breaks between then.
For this case, the package
provides a mechanism to include parts
by |\input| which can also be processed individually.
However, by construction this mechanism
requires manual handling of the content to be output.

%%%%%%%%%%%%%%%%%%%%%%%%%%%%%%%%%%%%%%%%
\DescribeMacro{\ifchilddocmanual}
The main file should be prepared as usual, see \secref{sec:include}.
However, the document body must make a distinction
between processing of an individual part and of the main document, e.g.:
%
\begin{center}
\begin{tabular}{l}
|\ifchilddocmanual|\\
|\input{\childdocname}|\\
|\||else|\\
\textit{document body with }|\input{|\textit{part}|}|\\
|\||fi|
\end{tabular}
\end{center}
%
The conditional |\ifchilddocmanual| is true whenever
a part to be included by |\input| is being compiled,
and the name of the part is stored in |\childdocname|.

%%%%%%%%%%%%%%%%%%%%%%%%%%%%%%%%%%%%%%%%
\DescribeMacro{\childdocby}
Each part to be included by |\input| should start with:
%
\begin{center}
\begin{tabular}{l}
|\input{childdoc.def}|\\
|\childdocby{|\textit{main}|}|\\
\end{tabular}
\end{center}
%
The directive |\childdocby| is similar to |\childdocof|
described in \secref{sec:include},
but the subsequent selection of content must be done manually.
To that end, both |\ifchilddoc| and |\ifchilddocmanual|
will be true upon processing of a part,
and the name of the part is stored in |\childdocname|.
Note that |\jobname| will be set to the filename of the current part
so that each part receives an individual |.aux| file
that does not interfere with the |.aux| file(s) of the main document.
This behaviour can be altered by the alternative form
|\childdocby[*]{|\textit{main}|}| (with a non-empty optional argument)
which uses the |.aux| file of the main document
by setting |\jobname| to \textit{main}.

%%%%%%%%%%%%%%%%%%%%%%%%%%%%%%%%%%%%%%%%%%%%%%%%%%%%%%%%%%%%%%%%%%%%%%%%%%%%%%%%
\subsection{Driver Development}
\label{sec:driver}

The \textsf{childdoc} mechanism can also be use for the development
of definition files such as \LaTeX{} styles or classes.
This case differs from the above setup with multiple parts
included by |\include| in that no |\includeonly| should be invoked.
This can be achieved by starting the include file
(before |\ProvidesPackage|) with:
%
\begin{center}
\begin{tabular}{l}
|\input{childdoc.def}|\\
|\childdocforward{|\textit{main}|}|\\
\end{tabular}
\end{center}
%
or alternatively with:
%
\begin{center}
\begin{tabular}{l}
|\input{childdoc.def}|\\
|\childdocby{|\textit{main}|}|\\
\end{tabular}
\end{center}
%
Both forms have slightly different effects as described above.
The main file is prepared as usual, see \secref{sec:include}.

%%%%%%%%%%%%%%%%%%%%%%%%%%%%%%%%%%%%%%%%%%%%%%%%%%%%%%%%%%%%%%%%%%%%%%%%%%%%%%%%
\subsection{Legacy Detection}
\label{sec:detection}

The directive |\childdocmain| in the main file can detect
whether the complete document or merely a child is to be compiled
even without using the directive |\childdocof|.
This method is deprecated because it is less robust
and there is no compelling reason to use it;
it is merely provided for backward compatibility
and it may be removed in future versions.

If the detection mechanism is to be used,
it is mandatory to correctly specify
the filename of the main file as the argument of |\childdocmain|:
%
\begin{center}
\begin{tabular}{l}
|\input{childdoc.def}|\\
|\childdocmain{|\textit{main}|}|\\
\end{tabular}
\end{center}
%
If |\jobname| does not match the argument \textit{main} of |\childdocmain|,
it is assumed that |\jobname| points to the child file to be compiled.
When using |\childdocmain| with the main file specified as argument,
it suffices to start a child file
with just |\input{|\textit{main}|}|
without loading of the package and using |\childdocof|.
If instead all processing is done
with the appropriate \textsf{childdoc} directives,
the argument of \textit{main} of |\childdocmain| can be empty.

An alternative version of the command line processing described
in \secref{sec:commandline} using the detection mechanism reads:
%
\begin{center}
|... -jobname "|\textit{target}|" "|[\textit{flags}]%
[|\def\jobname{|\textit{dest}|}|]|\input{|\textit{main}|}"|
\end{center}

%%%%%%%%%%%%%%%%%%%%%%%%%%%%%%%%%%%%%%%%%%%%%%%%%%%%%%%%%%%%%%%%%%%%%%%%%%%%%%%%
\subsection{Manual Code}
\label{sec:manual}

In case one cannot be certain whether the definitions file |childdoc.def|
is installed on the target \TeX{} distribution
and one prefers not to ship it,
it is conceivable to paste a few relevant commands into the sources.

To that end, drop all statements |\input{childdoc.def}|
and perform the replacements as outlined below.
Instead of |\childdocmain{|\textit{main}|}| add the following code
to the top of the main file:
%
\begin{center}
\begin{tabular}{l}
|\||ifdefined\childdocname\endinput\||fi\newif\ifchilddoc|\\
|\edef\childdocname{\scantokens\expandafter{\jobname\noexpand}}|\\
|\def\childdocmain{|\textit{main}|}\||ifx\childdocmain\childdocname\||else|\\
|\childdoctrue\includeonly{\childdocname}\let\jobname\childdocmain\||fi|\\
\end{tabular}
\end{center}
%
Instead of |\childdocof{|\textit{main}|}| just include the main file
at the top of each child file:
%
\begin{center}
|\input{|\textit{main}|}|
\end{center}
%
A simple redirection |\childdocforward{|\textit{dest}|}| is achieved by:
%
\begin{center}
|\def\jobname{|\textit{dest}|}\input{\jobname}|
\end{center}
%
The redirection with prefix
|\childdocforwardprefix[|\textit{prefix}|]{|\textit{dest}|}|
is accomplished by:
%
\begin{center}
\begin{tabular}{l}
|{\edef\jobname{\scantokens\expandafter{\jobname\noexpand}}|\\
|\def\redirectjob |\textit{prefix}|#1~~~{\gdef\jobname{|\textit{dest}|#1}}|\\
|\expandafter\redirectjob\jobname~~~}\input{\jobname}|
\end{tabular}
\end{center}

In an alternative approach,
child documents can be compiled by a specific command line
without additional code or specific definitions:
%
\begin{center}
|... -jobname "|\textit{target}|" "|[\textit{flags}]%
|\includeonly{|\textit{dest}|}\input{|\textit{main}|}"|
\end{center}
%

%%%%%%%%%%%%%%%%%%%%%%%%%%%%%%%%%%%%%%%%%%%%%%%%%%%%%%%%%%%%%%%%%%%%%%%%%%%%%%%%
%%%%%%%%%%%%%%%%%%%%%%%%%%%%%%%%%%%%%%%%%%%%%%%%%%%%%%%%%%%%%%%%%%%%%%%%%%%%%%%%
\section{Information}

%%%%%%%%%%%%%%%%%%%%%%%%%%%%%%%%%%%%%%%%%%%%%%%%%%%%%%%%%%%%%%%%%%%%%%%%%%%%%%%%
\subsection{Copyright}

Copyright \copyright{} 2017--2018 Niklas Beisert

This work may be distributed and/or modified under the
conditions of the \LaTeX{} Project Public License, either version 1.3
of this license or (at your option) any later version.
The latest version of this license is in
  \url{http://www.latex-project.org/lppl.txt}
and version 1.3 or later is part of all distributions of \LaTeX{}
version 2005/12/01 or later.

This work has the LPPL maintenance status `maintained'.

The Current Maintainer of this work is Niklas Beisert.

This work consists of the files |README.txt|, |childdoc.ins| and |childdoc.dtx|
as well as the derived files |childdoc.def|, |cdocsamp.tex|
with |cdocsch1.tex|, |cdocsch2.tex|, |cdocspt3.tex|, |cdocspt4.tex|,
|cdocsdrf.tex|, |cdocsfn1.tex|, |cdocsfn2.tex|
as well as |childdoc.pdf|.

%%%%%%%%%%%%%%%%%%%%%%%%%%%%%%%%%%%%%%%%%%%%%%%%%%%%%%%%%%%%%%%%%%%%%%%%%%%%%%%%
\subsection{Files and Installation}

The package consists of the files:
%
\begin{center}
\begin{tabular}{ll}
    |README.txt|   & readme file \\
    |childdoc.ins| & installation file \\
    |childdoc.dtx| & source file \\
    |childdoc.def| & definition file \\
    |cdocsamp.tex| & sample main file \\
    |cdocsch1.tex| & sample include file \\
    |cdocsch2.tex| & sample include file \\
    |cdocspt3.tex| & sample part file \\
    |cdocspt4.tex| & sample part file \\
    |cdocsdrf.tex| & sample redirection file \\
    |cdocsfn1.tex| & sample redirection file \\
    |cdocsfn2.tex| & sample redirection file \\
    |childdoc.pdf| & manual
\end{tabular}
\end{center}
%
The distribution consists of the files
|README.txt|, |childdoc.ins| and |childdoc.dtx|.
%
\begin{itemize}
\item
Run (pdf)\LaTeX{} on |childdoc.dtx|
to compile the manual |childdoc.pdf| (this file).
\item
Run \LaTeX{} on |childdoc.ins| to create the definitions file |childdoc.def|
and the sample |cdocsamp.tex| with include files
|cdocsch1.tex|, |cdocsch2.tex|, |cdocspt3.tex|, |cdocspt4.tex|,
|cdocsdrf.tex|, |cdocsfn1.tex|, |cdocsfn2.tex|.
Then copy the file |childdoc.def| to an appropriate directory of your \LaTeX{}
distribution, e.g.\ \textit{texmf-root}|/tex/latex/childdoc|.
\end{itemize}

%%%%%%%%%%%%%%%%%%%%%%%%%%%%%%%%%%%%%%%%%%%%%%%%%%%%%%%%%%%%%%%%%%%%%%%%%%%%%%%%
\subsection{Related CTAN Packages}

There are several other packages which offer a similar functionality:
%
\begin{itemize}
\item
The packages
\href{http://ctan.org/pkg/docmute}{\textsf{docmute}},
\href{http://ctan.org/pkg/includex}{\textsf{includex}} and
\href{http://ctan.org/pkg/standalone}{\textsf{standalone}}
provide commands to include only the document body of
a child file thus allowing both files to be compiled individually.
\item
The packages \href{http://ctan.org/pkg/subdocs}{\textsf{subdocs}}
and \href{http://ctan.org/pkg/subfiles}{\textsf{subfiles}}
provide structures in which the main and child documents can be
encapsulated and allowing them to be compiled individually.
The inclusion mechanism is different from the conventional |\include|.
\item
The package \href{http://ctan.org/pkg/combine}{\textsf{combine}}
is an elaborate solution to combine several documents into one.
\end{itemize}
%
See also the CTAN topic \href{http://ctan.org/topic/subdocs}{\textsf{subdocs}}
for further related packages.
The present package differs from the above solutions in that
a document structure constructed with the conventional |\include| mechanism
just needs two extra commands at the top of every file
such that all constituent files can be compiled individually.

%%%%%%%%%%%%%%%%%%%%%%%%%%%%%%%%%%%%%%%%%%%%%%%%%%%%%%%%%%%%%%%%%%%%%%%%%%%%%%%%
%\subsection{Feature Suggestions}
%
%The following is a list of features which may be useful for future
%versions of this package:
%%
%\begin{itemize}
%\item
%\ldots
%\end{itemize}

%%%%%%%%%%%%%%%%%%%%%%%%%%%%%%%%%%%%%%%%%%%%%%%%%%%%%%%%%%%%%%%%%%%%%%%%%%%%%%%%
\subsection{Revision History}

%%%%%%%%%%%%%%%%%%%%%%%%%%%%%%%%%%%%%%%%
\paragraph{v2.0:} 2018/12/30

\begin{itemize}
\item
immediate forward processing
\item
added |\childdocby| mechanism
\item
manual restructured
\end{itemize}

%%%%%%%%%%%%%%%%%%%%%%%%%%%%%%%%%%%%%%%%
\paragraph{v1.6:} 2018/01/17

\begin{itemize}
\item
application for development of include files
\item
corrections to manual
\end{itemize}

%%%%%%%%%%%%%%%%%%%%%%%%%%%%%%%%%%%%%%%%
\paragraph{v1.5:} 2017/05/21

\begin{itemize}
\item
more complete structuring introduced
\item
|\childdocof| introduced
\item
|\childdoc| renamed to |\childdocmain|
\item
|\childredirect| renamed to |\childdocforward| and |\childdocforwardprefix|
and functionality expanded
\end{itemize}

%%%%%%%%%%%%%%%%%%%%%%%%%%%%%%%%%%%%%%%%
\paragraph{v1.0:} 2017/04/27

\begin{itemize}
\item
manual and install package
\item
first version published on CTAN
\end{itemize}

%%%%%%%%%%%%%%%%%%%%%%%%%%%%%%%%%%%%%%%%
\paragraph{v0.6:} 2017/04/26

\begin{itemize}
\item
redirection mechanism added
\end{itemize}

%%%%%%%%%%%%%%%%%%%%%%%%%%%%%%%%%%%%%%%%
\paragraph{v0.5:} 2017/04/26

\begin{itemize}
\item
functionality in definition file
\end{itemize}


%%%%%%%%%%%%%%%%%%%%%%%%%%%%%%%%%%%%%%%%%%%%%%%%%%%%%%%%%%%%%%%%%%%%%%%%%%%%%%%%
%%%%%%%%%%%%%%%%%%%%%%%%%%%%%%%%%%%%%%%%%%%%%%%%%%%%%%%%%%%%%%%%%%%%%%%%%%%%%%%%
%%%%%%%%%%%%%%%%%%%%%%%%%%%%%%%%%%%%%%%%%%%%%%%%%%%%%%%%%%%%%%%%%%%%%%%%%%%%%%%%
\appendix

\settowidth\MacroIndent{\rmfamily\scriptsize 000\ }

 \DocInput{childdoc.dtx}

\end{document}
%</driver>
% \fi
%
% %%%%%%%%%%%%%%%%%%%%%%%%%%%%%%%%%%%%%%%%%%%%%%%%%%%%%%%%%%%%%%%%%%%%%%%%%%%%%%
% %%%%%%%%%%%%%%%%%%%%%%%%%%%%%%%%%%%%%%%%%%%%%%%%%%%%%%%%%%%%%%%%%%%%%%%%%%%%%%
% \section{Sample}
%\iffalse
%<*samplemain>
%\fi
%
% The following presents a sample document
% with two chapters, two parts, a title page,
% a compile flag as well as three forwarding files to set the flag.
% It consists of eight |.tex| files:
% \begin{center}
% \begin{tabular}{ll}
% |cdocsamp.tex|&main file\\
% |cdocsch1.tex|&include file for chapter 1\\
% |cdocsch2.tex|&include file for chapter 2\\
% |cdocspt3.tex|&include file for part 3\\
% |cdocspt4.tex|&include file for part 4\\
% |cdocsdrf.tex|&forwarding file for main file in draft mode\\
% |cdocsfi1.tex|&forwarding file for final version of chapter 1\\
% |cdocsfi2.tex|&forwarding file for final version of chapter 2\\
% \end{tabular}
% \end{center}
% Each of the eight files can be compiled directly by the \LaTeX{} compiler.
%
% %%%%%%%%%%%%%%%%%%%%%%%%%%%%%%%%%%%%%%
% \paragraph{Main File.}
%
% The main file is called |cdocsamp.tex|.
%
% Load the \textsf{childdoc} definitions and
% declare the filename for the main document:
%    \begin{macrocode}
\input{childdoc.def}
\childdocmain{}
%    \end{macrocode}

% Optional override for |\version| flag:
%    \begin{macrocode}
%%\ifchilddoc\else\providecommand{\version}{draft}\fi
%    \end{macrocode}

% Define the default values for the |\version| flag
% (|final| for the main file and |draft| for childs):
%    \begin{macrocode}
\ifchilddoc
\providecommand{\version}{draft}
\else
\providecommand{\version}{final}
\fi
%    \end{macrocode}

% Load the standard document class:
%    \begin{macrocode}
\documentclass[12pt]{article}
%    \end{macrocode}

% Start the document body:
%    \begin{macrocode}
\begin{document}
%    \end{macrocode}

% Declare a title page.
% Print title, part of document being processed and version flag:
%    \begin{macrocode}
\addtocounter{page}{-1}
\begin{center}
{\LARGE\bfseries{}childdoc example\par}
\vspace{1cm}
\ifchilddoc
\ifchilddocmanual part\else chapter\fi:
`\childdocname' of `\childdocjob'\par
\else
main document: `\childdocjob'\par
\fi
version: \version\par
\end{center}
\newpage
%    \end{macrocode}

% Manually include selected file,
% otherwise process as usual:
%    \begin{macrocode}
\ifchilddocmanual
\section*{part `\childdocname'}
\input{\childdocname}
\else
%    \end{macrocode}

% Include the two chapters:
%    \begin{macrocode}
\include{cdocsch1}
\include{cdocsch2}
%    \end{macrocode}

% Include the two parts unless only chapters should be displayed:
%    \begin{macrocode}
\ifchilddoc\else
\section{part three}
\input{cdocspt3}
\section{part four}
\input{cdocspt4}
\fi
%    \end{macrocode}

% Process as usual until here:
%    \begin{macrocode}
\fi
%    \end{macrocode}

% End of document body:
%    \begin{macrocode}
\end{document}
%    \end{macrocode}
%\iffalse
%</samplemain>
%\fi
%
% %%%%%%%%%%%%%%%%%%%%%%%%%%%%%%%%%%%%%%
% \paragraph{Chapter Include Files.}
%
% The include files are called |cdocsch1.tex| and |cdocsch2.tex|.
%
%\iffalse
%<*samplechap1|samplechap2>
%\fi

% Optional override for |\version| flag:
%    \begin{macrocode}
%%\providecommand{\version}{final}
%    \end{macrocode}

% Include the main document:
%    \begin{macrocode}
\input{childdoc.def}
\childdocof{cdocsamp}
%    \end{macrocode}

%\iffalse
%</samplechap1|samplechap2>
%\fi
%
%\iffalse
%<*samplechap1>
%\fi
% Some text for chapter 1:
%    \begin{macrocode}
\section{one}
some text in chapter one
%    \end{macrocode}

%\iffalse
%</samplechap1>
%\fi
% Some text for chapter 2:
%\iffalse
%<*samplechap2>
%\fi
%    \begin{macrocode}
\section{two}
more text in chapter two
%    \end{macrocode}

%\iffalse
%</samplechap2>
%\fi
%
% %%%%%%%%%%%%%%%%%%%%%%%%%%%%%%%%%%%%%%
% \paragraph{Part Include Files.}
%
% The include files are called |cdocspt3.tex| and |cdocspt4.tex|.
%
%\iffalse
%<*samplepart3|samplepart4>
%\fi

% Optional override for |\version| flag:
%    \begin{macrocode}
%%\providecommand{\version}{final}
%    \end{macrocode}

% Include the main document:
%    \begin{macrocode}
\input{childdoc.def}
\childdocby{cdocsamp}
%    \end{macrocode}

%\iffalse
%</samplepart3|samplepart4>
%\fi
%
%\iffalse
%<*samplepart3>
%\fi
% Some text for part 3:
%    \begin{macrocode}
some text in part three
%    \end{macrocode}

%\iffalse
%</samplepart3>
%\fi
% Some text for part 4:
%\iffalse
%<*samplepart4>
%\fi
%    \begin{macrocode}
more text in part four
%    \end{macrocode}

%\iffalse
%</samplepart4>
%\fi
%
% %%%%%%%%%%%%%%%%%%%%%%%%%%%%%%%%%%%%%%
% \paragraph{Forwarding for a Complete Draft.}
%
% The following forwarding file |cdocsdrf.tex|
% compiles the main document in draft mode:
%\iffalse
%<*sampledraft>
%\fi
%    \begin{macrocode}
\def\version{draft}
\input{childdoc.def}
\childdocforward{cdocsamp}
%    \end{macrocode}

%\iffalse
%</sampledraft>
%\fi
%
% %%%%%%%%%%%%%%%%%%%%%%%%%%%%%%%%%%%%%%
% \paragraph{Forwarding for Final Version of the Chapters.}
%
% The following forwarding files |cdocsfn1.tex| and |cdocsfn2.tex|
% (with identical content)
% compile the final versions of the child documents
% |cdocsch1.tex| and |cdocsch2.tex|, respectively:
%\iffalse
%<*samplefinal>
%\fi
%    \begin{macrocode}
\def\version{final}
\input{childdoc.def}
\childdocforwardprefix[cdocsamp]{cdocsfn}{cdocsch}
%    \end{macrocode}

%\iffalse
%</samplefinal>
%\fi
%
% %%%%%%%%%%%%%%%%%%%%%%%%%%%%%%%%%%%%%%
% \paragraph{Command Line Processing.}
%
% The following three command lines generate the output files
% |cdocscld|, |cdocscl1| and |cdocscl2|
% which should be identical to
% |cdocsdrf|, |cdocsch1| and |cdocsfn2|, respectively:
% \begin{center}
% \begin{tabular}{l}
% |latex -jobname cdocscld \|\\
% |  "\def\version{draft}\input{childdoc.def}\childdocforward{cdocsamp}"|\\
% |latex -jobname cdocscl1 \|\\
% |  "\input{childdoc.def}\childdocforward[cdocsamp]{cdocsch1}"|\\
% |latex -jobname cdocscl2 \|\\
% |  "\def\version{final}\input{childdoc.def}\childdocforward{cdocsch2}"|
% \end{tabular}
% \end{center}
% Note that the trailing backslash on each first line
% merely continues the input to the second line
% (for convenient cut ant paste).
% Furthermore, the command |latex| can be replaced by any
% of its alternative versions such as |pdflatex|.
%
% %%%%%%%%%%%%%%%%%%%%%%%%%%%%%%%%%%%%%%%%%%%%%%%%%%%%%%%%%%%%%%%%%%%%%%%%%%%%%%
% %%%%%%%%%%%%%%%%%%%%%%%%%%%%%%%%%%%%%%%%%%%%%%%%%%%%%%%%%%%%%%%%%%%%%%%%%%%%%%
% \section{Implementation}
%\iffalse
%<*package>
%\fi
%
% This section describes the definitions file |childdoc.def|.

% The definitions cannot be loaded using |\usepackage| or |\RequirePackage|
% which has a mechanism to prevent loading a style file more than once.
% When loading the definitions by means of |\input|
% multiple instances have to be prevented manually:
%\iffalse
%This code needs to be before the `\ProvidesFile' directive
%which is defined at the beginning of this file.
%Therefore it is also placed there and commented out here.
%</package>
%<*discard>
%\fi
%    \begin{macrocode}
\ifdefined\childdocmain\endinput\fi
%    \end{macrocode}
%\iffalse
%</discard>
%<*package>
%\fi
%
% \macro{\ifchilddoc}
% \macro{\ifchilddocmanual}
% The conditional |\ifchilddoc| tells whether a
% child (true) or main (false) document is being compiled.
% The conditional |\ifchilddocmanual| tells whether
% the |\includeonly| mechanism is used (false) or
% the selection of child files must be performed manually (true).
% The definitions initialise to false:
%    \begin{macrocode}
\newif\ifchilddoc
\newif\ifchilddocmanual
%    \end{macrocode}

% \macro{\childdocname}
% \macro{\childdocjob}
% The macro |\childdocname| stores the name of the main document
% to be compiled. The macro |\childdocjob| stores the name of
% the document on which the \LaTeX{} compiler was originally invoked.
% The content of |\jobname| cannot be compared
% to filenames specified in the source due to different catcodes.
% The following code rescans |\jobname|, stores the result
% in |\childdocname| and saves a copy in |\childdocjob|:
%    \begin{macrocode}
\edef\childdocname{\scantokens\expandafter{\jobname\noexpand}}
\let\childdocjob\childdocname
%    \end{macrocode}

% \macro{\childdocdisable}
% The macro |\childdocdisable| prevents the main file
% from being processed more than once.
% At this stage, the main document command |\childdocmain|
% is assumed to be called once again where it should do nothing.
% Any subsequent call to it should prevent
% a secondary processing of the main document
% It overwrites the forwarding commands
% |\childdocof| and |\childdocforward|
% with empty macros to prevent further inclusions of the main document:
%    \begin{macrocode}
\newcommand{\childdocdisable}
{
  \renewcommand{\childdocmain}[1]{\renewcommand{\childdocmain}[1]{\endinput}}
  \renewcommand{\childdocof}[1]{}
  \renewcommand{\childdocby}[2][]{}
  \renewcommand{\childdocforward}[2][]{}
  \renewcommand{\childdocdisable}{}
}
%    \end{macrocode}

% \macro{\childdocmain}
% The macro |\childdocmain| is to be called at the top of the main file
% with nothing or the main filename (without extension) as argument.
% First, it breaks loops.
% If the argument is not empty and does not match |\childdocname|
% (which is set by the first inclusion of |childdoc.def|),
% |\ifchilddoc| is set to true, |\includeonly| is applied to the child file
% and |\jobname| is set to the main file
% (for proper handling of |.aux| files):
%    \begin{macrocode}
\newcommand{\childdocmain}[1]
{
  \childdocdisable\childdocmain{}
  \if?#1?\else
    \begingroup
      \def\childdoctmp{#1}
      \ifx\childdoctmp\childdocname
        \def\childdoctmp{}
      \else
        \def\childdoctmp
        {
          \childdoctrue
          \includeonly{\childdocname}
          \def\childdocjob{#1}
          \def\jobname{#1}
        }
      \fi
      \expandafter
    \endgroup
    \childdoctmp
  \fi
}
%    \end{macrocode}

% \macro{\childdocof}
% The command |\childdocof| redirects
% compilation to the main file |#1|.
%    \begin{macrocode}
\newcommand{\childdocof}[1]
{
  \childdocdisable
  \childdoctrue
  \includeonly{\childdocname}
  \def\jobname{#1}
  \def\childdocjob{#1}
  \input{#1}
}
%    \end{macrocode}

% \macro{\childdocby}
% The command |\childdocby| ....
%    \begin{macrocode}
\newcommand{\childdocby}[2][]
{
  \childdocdisable
  \childdoctrue
  \childdocmanualtrue
  \if?#1?\else
    \def\jobname{#2}
  \fi
  \def\childdocjob{#2}
  \input{#2}
  \endinput
}
%    \end{macrocode}

% \macro{\childdocforward}
% The command |\childdocforward| redirects
% compilation to the main file or
% (if the optional argument is given) a child file.
% Parameters are set as if the main file
% or a child file starting with |\childdocof| was compiled.
% Then compilation is handed over to the main file:
%    \begin{macrocode}
\newcommand{\childdocforward}[2][]
{
  \begingroup
    \if?#1?
      \def\childdoctmp
      {
        \def\childdocname{#2}
        \def\childdocjob{#2}
        \def\jobname{#2}
        \input{#2}
        \endinput
      }
    \else
      \def\childdoctmp
      {
        \childdocdisable
        \def\childdocname{#2}
        \childdoctrue
        \includeonly{#2}
        \def\childdocjob{#1}
        \def\jobname{#1}
        \input{#1}
        \endinput
      }
    \fi
    \expandafter
  \endgroup
  \childdoctmp
}
%    \end{macrocode}

% \macro{\childdocforwardprefix}
% The command |\childdocforwardprefix| redirects
% compilation to the main or a child file by means of a pattern.
% The prefix |#1| in the current filename is replaced by |#2|
% and the suffix of the current filename is kept
% (it is assumed that the filename does not contain the substring `|~~~|'
% which is used as a delimiter).
% Compilation is handed over to the new file by |\childdocforward|:
%    \begin{macrocode}
\newcommand{\childdocforwardprefix}[3][]
{
  \begingroup
    \def\childdocextract #2##1~~~{\def\childdoctmp{\childdocforward[#1]{#3##1}}}
    \expandafter\childdocextract\childdocname~~~
    \expandafter
  \endgroup
  \childdoctmp
}
%    \end{macrocode}

% \macro{\childdoc}
% The deprecated macro |\childdoc| is a legacy version of |\childdocmain|:
%    \begin{macrocode}
\newcommand{\childdoc}{\childdocmain}
%    \end{macrocode}

% \macro{\childdocredirect}
% The deprecated macro |\childdocredirect| is a legacy version
% of |\childdocforward| and |\childdocforwardprefix|:
%    \begin{macrocode}
\newcommand{\childdocredirect}[2][]
{
  \begingroup
    \if?#1?
      \def\childdoctmp{\childdocforward{#2}}
    \else
      \def\childdoctmp{\childdocforwardprefix{#1}{#2}}
    \fi
    \expandafter
  \endgroup
  \childdoctmp
}
%    \end{macrocode}

%\iffalse
%</package>
%\fi
%
\endinput
\childdocforward[cdocsamp]{cdocsch1}"|\\
% |latex -jobname cdocscl2 \|\\
% |  "\def\version{final}% \iffalse
%
% childdoc.dtx Copyright (C) 2017-2018 Niklas Beisert
%
% This work may be distributed and/or modified under the
% conditions of the LaTeX Project Public License, either version 1.3
% of this license or (at your option) any later version.
% The latest version of this license is in
%   http://www.latex-project.org/lppl.txt
% and version 1.3 or later is part of all distributions of LaTeX
% version 2005/12/01 or later.
%
% This work has the LPPL maintenance status `maintained'.
%
% The Current Maintainer of this work is Niklas Beisert.
%
% This work consists of the files childdoc.dtx and childdoc.ins
% and the derived files childdoc.def and cdocsamp.tex with
% cdocsch1.tex, cdocsch2.tex, cdocsdrf.tex, cdocsfn1.tex, cdocsfn2.tex.
%
%<package>\ifdefined\childdocmain\endinput\fi
%<package>\ProvidesFile{childdoc.def}[2018/12/30 v2.0 child document driver]
%<samplemain>\ProvidesFile{cdocsamp.tex}[2018/12/30 v2.0 sample for childdoc]
%<*driver>
%\ProvidesFile{childdoc.drv}[2018/12/30 v2.0 childdoc reference manual file]
\PassOptionsToClass{10pt,a4paper}{article}
\documentclass{ltxdoc}

\usepackage[margin=35mm]{geometry}
\usepackage{hyperref}
\usepackage{hyperxmp}
\usepackage[usenames]{color}

\hypersetup{colorlinks=true}
\hypersetup{pdfstartview=FitH}
\hypersetup{pdfpagemode=UseNone}
\hypersetup{pdfsource={}}
\hypersetup{pdflang={en-UK}}
\hypersetup{pdfcopyright={Copyright 2017-2018 Niklas Beisert.
  This work may be distributed and/or modified under the
  conditions of the LaTeX Project Public License, either version 1.3
  of this license or (at your option) any later version.}}
\hypersetup{pdflicenseurl={http://www.latex-project.org/lppl.txt}}
\hypersetup{pdfcontactaddress={ETH Zurich, ITP, HIT K,
  Wolfgang-Pauli-Strasse 27}}
\hypersetup{pdfcontactpostcode={8093}}
\hypersetup{pdfcontactcity={Zurich}}
\hypersetup{pdfcontactcountry={Switzerland}}
\hypersetup{pdfcontactemail={nbeisert@itp.phys.ethz.ch}}
\hypersetup{pdfcontacturl={http://people.phys.ethz.ch/\xmptilde nbeisert/}}

\newcommand{\secref}[1]{\hyperref[#1]{section \ref*{#1}}}

\parskip1ex
\parindent0pt
\let\olditemize\itemize
\def\itemize{\olditemize\parskip0pt}

\begin{document}

\title{The \textsf{childdoc} Package}
\hypersetup{pdftitle={The childdoc Package}}
\author{Niklas Beisert\\[2ex]
  Institut f\"ur Theoretische Physik\\
  Eidgen\"ossische Technische Hochschule Z\"urich\\
  Wolfgang-Pauli-Strasse 27, 8093 Z\"urich, Switzerland\\[1ex]
  \href{mailto:nbeisert@itp.phys.ethz.ch}
  {\texttt{nbeisert@itp.phys.ethz.ch}}}
\hypersetup{pdfauthor={Niklas Beisert}}
\hypersetup{pdfsubject={Manual for the LaTeX2e Package childdoc}}
\date{30 December 2018, \textsf{v2.0}}
\maketitle

\begin{abstract}\noindent
\textsf{childdoc} is a \LaTeXe{} package
that enables the direct compilation
of document sections included by |\include|
to individual files.
\end{abstract}

\begingroup
\parskip0ex
\tableofcontents
\endgroup

%%%%%%%%%%%%%%%%%%%%%%%%%%%%%%%%%%%%%%%%%%%%%%%%%%%%%%%%%%%%%%%%%%%%%%%%%%%%%%%%
%%%%%%%%%%%%%%%%%%%%%%%%%%%%%%%%%%%%%%%%%%%%%%%%%%%%%%%%%%%%%%%%%%%%%%%%%%%%%%%%
\section{Introduction}

\LaTeX{} provides a mechanism to structure a large document (such as a book)
into a main file and several child files (containing the chapters)
using the |\include| command.
This mechanism is beneficial for documents
which span hundreds of pages in order to
make the source file(s) more manageable.
Moreover, compilation can be restricted to
selected child files by means of the |\includeonly| command.
The latter feature can be used to reduce the compilation time while editing
(this was significantly more useful in the earlier days of \LaTeX{})
or to generate a smaller document which is easier to navigate.
Another application of |\includeonly| is to generate
documents consisting of selected parts of the complete document.

However, there are a few drawbacks of the plain |\include| mechanism:
\begin{itemize}
\item
The child files cannot be compiled on their own,
they can only be compiled via the main file.
A naive editing environment
(such as a text editor with an option
to have the current file processed by \LaTeX)
may require one to switch to the main file before compiling;
attempting to compile the child file produces errors.
\item
The main file must be modified (each time)
to adjust the |\includeonly| command
to the present needs. This easily leaves the main file in a messy state.
\item
The generated document will always carry the filename
of the main document. This is inconvenient if
several child files are to be compiled and
to be kept for distribution.
\end{itemize}

The present package provides a simple interface
to make child files individually compilable by \LaTeX{}.
Compiling a child file then has the same effect as compiling
the main file with an |\includeonly| command
to select the appropriate child.
Moreover the generated document will carry the name of the child
rather than the main file.
This resolves all three above issues.

This feature is meant to make the editing of books,
thesis documents and lecture notes somewhat more convenient.
However, the package can also be used efficiently for
composing a series of documents (such as exercise sheets)
which are typically distributed individually.
It then assists the author in generating the individual documents
(potentially in different versions)
as well as a document containing the collected series.
Another application is in developing style files
or other kinds of included material
where compilation of the style file could redirect
to a sample or test file.

%%%%%%%%%%%%%%%%%%%%%%%%%%%%%%%%%%%%%%%%%%%%%%%%%%%%%%%%%%%%%%%%%%%%%%%%%%%%%%%%
%%%%%%%%%%%%%%%%%%%%%%%%%%%%%%%%%%%%%%%%%%%%%%%%%%%%%%%%%%%%%%%%%%%%%%%%%%%%%%%%
\section{Usage}

First of all, the package \textsf{childdoc} is \emph{not} a standard
\LaTeXe{} |.sty| style file! Therefore it needs to be invoked in
a non-standard way.

%%%%%%%%%%%%%%%%%%%%%%%%%%%%%%%%%%%%%%%%%%%%%%%%%%%%%%%%%%%%%%%%%%%%%%%%%%%%%%%%
\subsection{Included Files}
\label{sec:include}

%%%%%%%%%%%%%%%%%%%%%%%%%%%%%%%%%%%%%%%%
\DescribeMacro{\childdocmain}
To use the package, add the commands
\begin{center}
\begin{tabular}{l}
|\input{childdoc.def}|\\
|\childdocmain{}|\\
\end{tabular}
\end{center}
at the very top of the main \LaTeX{} file,
in particular \emph{before} the |\documentclass| statement!
The argument of |\childdocmain| should be left empty
(but it must be present).

%%%%%%%%%%%%%%%%%%%%%%%%%%%%%%%%%%%%%%%%
\DescribeMacro{\childdocof}
Furthermore, add the commands
\begin{center}
\begin{tabular}{l}
|\input{childdoc.def}|\\
|\childdocof{|\textit{main}|}|\\
\end{tabular}
\end{center}
at the top of every child file \textit{child}
which is included by |\include{|\textit{child}|}|
from within the main file
(or at least for those files to be compiled individually).
The argument \textit{main} must be the filename of the main file.

There are a couple of
considerations in setting up the main and child documents:

%%%%%%%%%%%%%%%%%%%%%%%%%%%%%%%%%%%%%%%%
\paragraph{Restrictions.}

Please note the following restrictions:
\begin{itemize}
\item
|\childdocmain| must be called with one argument \textit{main}
to ensure compatibility with earlier version of the package.
It must either be empty (|\childdocmain{}|)
or precisely match the filename of the main file in which it is specified.
See \secref{sec:detection} for further information.
\item
The filename \textit{main} must be specified without the |.tex| extension.
\item
The filename \textit{main} is case sensitive
(even in case-insensitive file systems)
due to internal string comparison.
\item
The argument \textit{main} should be fully expanded, it cannot be a macro.
\item
Subdirectories and special characters should be avoided in filenames.
\item
The command |\childdocmain{|\textit{main}|}| must be followed by a whitespace.
It should not be followed immediately by another command
or by a comment mark `|%|'.
This is because the \TeX{} parser reads the token immediately following
the argument of |\childdocmain| and puts it
at the beginning of every child section;
however, a white\-space is ignored.
\end{itemize}

%%%%%%%%%%%%%%%%%%%%%%%%%%%%%%%%%%%%%%%%
\paragraph{Content of Main File.}

It is advisable to place all content in the child files included by |\include|.
Any output contained in the main file will appear in all child documents
unless suppressed manually;
it cannot be suppressed automatically by the |\includeonly| directive
and thus should normally be avoided.
A method to include some content in the main file
by means of conditional processing is described in \secref{sec:conditional}.

%%%%%%%%%%%%%%%%%%%%%%%%%%%%%%%%%%%%%%%%
\paragraph{Page Numbering.}

When only a part of the document is compiled,
the appropriate numbering of pages
(as well as other status parameters)
is determined from the |.aux| files.
The latter contain information from previous passes.
However this information needs to propagate through
all intermediate child documents.
Therefore the page numbering in child documents may well
be inconsistent until the complete document is compiled at least once.

A useful (if unconventional) way to always ensure a consistent
page numbering is to restart the numbering in each child document
and denote the pages by `\textit{child}|.|\textit{page}'
where \textit{child} represents the chapter/section number of the child file.
This can be achieved by the command
|\numberwithin{page}{|\textit{child}|}|
of the \textsf{amsmath} package
where \textit{child} can be |chapter| or |section|
depending on the chosen structuring.
Alternatively, one can modify the macro |\thepage| appropriately
and reset the counter |page| at the start of each child file.

%%%%%%%%%%%%%%%%%%%%%%%%%%%%%%%%%%%%%%%%%%%%%%%%%%%%%%%%%%%%%%%%%%%%%%%%%%%%%%%%
\subsection{Conditional Processing}
\label{sec:conditional}

The package provides a mechanism to compile different versions
of a document. To customise the versions further some conditional processing
can come in handy to distinguish which version is being compiled.
The package provides two macros to describe the compilation context:

%%%%%%%%%%%%%%%%%%%%%%%%%%%%%%%%%%%%%%%%
\DescribeMacro{\ifchilddoc}
The conditional |\ifchilddoc| distinguishes between the compilation of
child documents and the main document:
%
\begin{center}
|\ifchilddoc |\textit{child-code}| |[|\||else |\textit{main-code}]| \||fi|
\end{center}

%%%%%%%%%%%%%%%%%%%%%%%%%%%%%%%%%%%%%%%%
\DescribeMacro{\childdocname}
\DescribeMacro{\childdocjob}
The macro |\childdocname| contains the filename (without extension)
of the main or child file being processed.
Note that |\childdocjob| will always contain the name of the main file.

%%%%%%%%%%%%%%%%%%%%%%%%%%%%%%%%%%%%%%%%
\paragraph{Title Page.}

Conditional processing can be used to include a title or banner page
in the main document when proper precautions are taken.
Importantly, the code in the main file should ensure that the page counter
(as well as other status parameters which are stored in the |.aux| files)
takes the same value after the conditional processing.
Otherwise the page numbers may take divergent values
depending on which part is compiled.

For example, a title page could be declared by:
%
\begin{center}
\begin{tabular}{l}
|\ifchilddoc\||else|\\
|\addtocounter{page}{-1}|\\
\textit{code for title page}\\
|\newpage|\\
|\||fi|
\end{tabular}
\end{center}
%
A banner page for the child documents can be generated by:
%
\begin{center}
\begin{tabular}{l}
|\ifchilddoc|\\
|\addtocounter{page}{-1}|\\
\textit{code for banner page}\\
|\newpage|\\
|\||fi|
\end{tabular}
\end{center}
%
Here one could write a message such as:
\begin{center}
|This is the part \childdocname{} of \childdocjob{}.|
\end{center}

%%%%%%%%%%%%%%%%%%%%%%%%%%%%%%%%%%%%%%%%%%%%%%%%%%%%%%%%%%%%%%%%%%%%%%%%%%%%%%%%
\subsection{Flags}
\label{sec:flags}

The package makes it easy to generate different versions
of the main or child documents.
To this end compilation flags can be defined
and assigned different default values.
They will be particularly useful in conjunction
with the forwarding mechanism described in \secref{sec:forward}.

For example, it may be useful to have a flag |\version|
which can be set to |draft| or |final|.
The document source will contain some conditional code
depending on the value of |\version|.
Suppose further, the flag should default to |final| for the main file
and to |draft| for child files
which is a natural assignment for editing the document.
This is achieved by placing the following code
in the preamble of the main document
(below the |\childdocmain| directive):
%
\begin{center}
\begin{tabular}{l}
|\ifchilddoc|\\
|\providecommand{\version}{draft}|\\
|\||else|\\
|\providecommand{\version}{final}|\\
|\||fi|
\end{tabular}
\end{center}
%
The definition by |\providecommand| makes sure
that previous definitions are not overwritten.
Further statements |\providecommand{\version}{...}|
can thus be added before the above code to override it.

For the main file, one might add a line
(between |\childdocmain| and the above block)
%
\begin{center}
|%\ifchilddoc\||else\providecommand{\version}{draft}\||fi|
\end{center}
%
which can be uncommented to produce a draft version.
Likewise one can add a line to the very top of a child file
(above the |\childdocof{|\textit{main}|}| directive)
%
\begin{center}
|%\providecommand{\version}{final}|
\end{center}
%
which can be uncommented to produce the final version of this child document.

%%%%%%%%%%%%%%%%%%%%%%%%%%%%%%%%%%%%%%%%%%%%%%%%%%%%%%%%%%%%%%%%%%%%%%%%%%%%%%%%
\subsection{Forwarding}
\label{sec:forward}

Different versions of the main or child documents
using compilation flags as described in \secref{sec:flags}
can be (permanently) stored in different files
for convenient compilation, viewing and distribution.
To this end, the package defines a command
to pass on compilation to a different file:

%%%%%%%%%%%%%%%%%%%%%%%%%%%%%%%%%%%%%%%%
\DescribeMacro{\childdocforward}
The command |\childdocforward| redirects processing to
another source file:
%
\begin{center}
\begin{tabular}{l}
|\input{childdoc.def}|\\
|\childdocforward[|\textit{main}|]{|\textit{dest}|}|\\
\end{tabular}
\end{center}
%
The argument \textit{dest} is the destination file
(without extension).
It should be the main file or one of the child files.
Note that further \textsf{childdoc} directives
such as |\childdocof| and |\childdocforward|
in the indicated file will be processed in this form.
The optional argument \textit{main}
passes on directly to the main file \textit{main}
while pretending to compile the child \textit{dest}.
This form behaves as if \textit{dest}
issues |\childdocof{|\textit{main}|}| right away,
and no further \textsf{childdoc} directives will be processed.

%%%%%%%%%%%%%%%%%%%%%%%%%%%%%%%%%%%%%%%%
\DescribeMacro{\...prefix}
In the alternative form |\childdocforwardprefix|,
%
\begin{center}
\begin{tabular}{l}
|\input{childdoc.def}|\\
|\childdocforwardprefix[|\textit{main}|]{|\textit{prefix}|}{|\textit{dest}|}|
\end{tabular}
\end{center}
%
the destination file is determined by a pattern
depending on the current file:
To make this work, the current file must be called
`{\textit{prefix}\hspace{0.2em}\textit{suffix}}'
with \textit{prefix} matching precisely the argument.
Processing is then passed on to the file
`{\textit{dest}\hspace{0.2em}\textit{suffix}}'.
Surely, the same effect is achieved by
directly specifying the
argument `{\textit{dest}\hspace{0.2em}\textit{suffix}}'
in the first form.
However, that requires to set up a different file
for each child. With the alternative form of the command
all these files can have exactly the same content
which simplifies setting them up and maintaining them.

For example, the following file |draft.tex|
with a compilation flag |\version| as described in \secref{sec:flags}
compiles the main document as a draft:
%
\begin{center}
\begin{tabular}{l}
|\def\version{draft}|\\
|\input{childdoc.def}|\\
|\childdocforward{|\textit{main}|}|
\end{tabular}
\end{center}
%
Likewise, the following files |final|\textit{nn}|.tex|
compile the final version of the child document
|child|\textit{nn}|.tex|:
%
\begin{center}
\begin{tabular}{l}
|\def\version{final}|\\
|\input{childdoc.def}|\\
|\childdocforwardprefix{final}{child}|
\end{tabular}
\end{center}
%

Note that when several versions of a main file and/or of each child file
are to be generated, it may be convenient to set up a |Makefile| or
shell script to automatise the process.

%%%%%%%%%%%%%%%%%%%%%%%%%%%%%%%%%%%%%%%%%%%%%%%%%%%%%%%%%%%%%%%%%%%%%%%%%%%%%%%%
\subsection{Command Line Processing}
\label{sec:commandline}

The effect of redirection files can also be achieved by invoking
the \LaTeX{} compiler with a more elaborate command line.
Most conveniently this should be done as part
of a shell script or a |Makefile|.

When using \textsf{childdoc} in the main file, the following
command lines effectively perform a redirection
(note that depending on the shell being used,
backslashes may have to be doubled: `|\|' $\to$ `|\\|'):
%
\begin{center}
|... -jobname "|\textit{target}|" |\\|"|[\textit{flags}]%
|\input{childdoc.def}\childdocforward[|\textit{main}|]{|\textit{dest}|}"|
\end{center}
%
Here \textit{target} is the name of the output file,
\textit{main} is the name of the main file
and \textit{dest} is the name of the main or child file to be processed
(all filenames without extensions).
The optional argument \textit{main} can be omitted
if \textit{main} matches \textit{dest}.
Optionally, compilation \textit{flags} can be defined via |\def| commands.
This command line makes the \TeX{} engine believe
it is compiling the file \textit{target}
whose content is specified as the latter parameter.
The provided code then forwards the processing to
\textit{main} or \textit{dest} as described in \secref{sec:forward}.

%%%%%%%%%%%%%%%%%%%%%%%%%%%%%%%%%%%%%%%%%%%%%%%%%%%%%%%%%%%%%%%%%%%%%%%%%%%%%%%%
\subsection{Include by Input}
\label{sec:input}

Including child documents by |\include| has some restrictions by design.
Most notably, the content of a child document always occupies
its own set of pages; pages cannot be shared between child documents.
Usually, this behaviour makes perfect sense
because each child document contain an essential part of the document.
However, in some situations it may be desirable to compose
a document from a collection of parts
without having mandatory page breaks between then.
For this case, the package
provides a mechanism to include parts
by |\input| which can also be processed individually.
However, by construction this mechanism
requires manual handling of the content to be output.

%%%%%%%%%%%%%%%%%%%%%%%%%%%%%%%%%%%%%%%%
\DescribeMacro{\ifchilddocmanual}
The main file should be prepared as usual, see \secref{sec:include}.
However, the document body must make a distinction
between processing of an individual part and of the main document, e.g.:
%
\begin{center}
\begin{tabular}{l}
|\ifchilddocmanual|\\
|\input{\childdocname}|\\
|\||else|\\
\textit{document body with }|\input{|\textit{part}|}|\\
|\||fi|
\end{tabular}
\end{center}
%
The conditional |\ifchilddocmanual| is true whenever
a part to be included by |\input| is being compiled,
and the name of the part is stored in |\childdocname|.

%%%%%%%%%%%%%%%%%%%%%%%%%%%%%%%%%%%%%%%%
\DescribeMacro{\childdocby}
Each part to be included by |\input| should start with:
%
\begin{center}
\begin{tabular}{l}
|\input{childdoc.def}|\\
|\childdocby{|\textit{main}|}|\\
\end{tabular}
\end{center}
%
The directive |\childdocby| is similar to |\childdocof|
described in \secref{sec:include},
but the subsequent selection of content must be done manually.
To that end, both |\ifchilddoc| and |\ifchilddocmanual|
will be true upon processing of a part,
and the name of the part is stored in |\childdocname|.
Note that |\jobname| will be set to the filename of the current part
so that each part receives an individual |.aux| file
that does not interfere with the |.aux| file(s) of the main document.
This behaviour can be altered by the alternative form
|\childdocby[*]{|\textit{main}|}| (with a non-empty optional argument)
which uses the |.aux| file of the main document
by setting |\jobname| to \textit{main}.

%%%%%%%%%%%%%%%%%%%%%%%%%%%%%%%%%%%%%%%%%%%%%%%%%%%%%%%%%%%%%%%%%%%%%%%%%%%%%%%%
\subsection{Driver Development}
\label{sec:driver}

The \textsf{childdoc} mechanism can also be use for the development
of definition files such as \LaTeX{} styles or classes.
This case differs from the above setup with multiple parts
included by |\include| in that no |\includeonly| should be invoked.
This can be achieved by starting the include file
(before |\ProvidesPackage|) with:
%
\begin{center}
\begin{tabular}{l}
|\input{childdoc.def}|\\
|\childdocforward{|\textit{main}|}|\\
\end{tabular}
\end{center}
%
or alternatively with:
%
\begin{center}
\begin{tabular}{l}
|\input{childdoc.def}|\\
|\childdocby{|\textit{main}|}|\\
\end{tabular}
\end{center}
%
Both forms have slightly different effects as described above.
The main file is prepared as usual, see \secref{sec:include}.

%%%%%%%%%%%%%%%%%%%%%%%%%%%%%%%%%%%%%%%%%%%%%%%%%%%%%%%%%%%%%%%%%%%%%%%%%%%%%%%%
\subsection{Legacy Detection}
\label{sec:detection}

The directive |\childdocmain| in the main file can detect
whether the complete document or merely a child is to be compiled
even without using the directive |\childdocof|.
This method is deprecated because it is less robust
and there is no compelling reason to use it;
it is merely provided for backward compatibility
and it may be removed in future versions.

If the detection mechanism is to be used,
it is mandatory to correctly specify
the filename of the main file as the argument of |\childdocmain|:
%
\begin{center}
\begin{tabular}{l}
|\input{childdoc.def}|\\
|\childdocmain{|\textit{main}|}|\\
\end{tabular}
\end{center}
%
If |\jobname| does not match the argument \textit{main} of |\childdocmain|,
it is assumed that |\jobname| points to the child file to be compiled.
When using |\childdocmain| with the main file specified as argument,
it suffices to start a child file
with just |\input{|\textit{main}|}|
without loading of the package and using |\childdocof|.
If instead all processing is done
with the appropriate \textsf{childdoc} directives,
the argument of \textit{main} of |\childdocmain| can be empty.

An alternative version of the command line processing described
in \secref{sec:commandline} using the detection mechanism reads:
%
\begin{center}
|... -jobname "|\textit{target}|" "|[\textit{flags}]%
[|\def\jobname{|\textit{dest}|}|]|\input{|\textit{main}|}"|
\end{center}

%%%%%%%%%%%%%%%%%%%%%%%%%%%%%%%%%%%%%%%%%%%%%%%%%%%%%%%%%%%%%%%%%%%%%%%%%%%%%%%%
\subsection{Manual Code}
\label{sec:manual}

In case one cannot be certain whether the definitions file |childdoc.def|
is installed on the target \TeX{} distribution
and one prefers not to ship it,
it is conceivable to paste a few relevant commands into the sources.

To that end, drop all statements |\input{childdoc.def}|
and perform the replacements as outlined below.
Instead of |\childdocmain{|\textit{main}|}| add the following code
to the top of the main file:
%
\begin{center}
\begin{tabular}{l}
|\||ifdefined\childdocname\endinput\||fi\newif\ifchilddoc|\\
|\edef\childdocname{\scantokens\expandafter{\jobname\noexpand}}|\\
|\def\childdocmain{|\textit{main}|}\||ifx\childdocmain\childdocname\||else|\\
|\childdoctrue\includeonly{\childdocname}\let\jobname\childdocmain\||fi|\\
\end{tabular}
\end{center}
%
Instead of |\childdocof{|\textit{main}|}| just include the main file
at the top of each child file:
%
\begin{center}
|\input{|\textit{main}|}|
\end{center}
%
A simple redirection |\childdocforward{|\textit{dest}|}| is achieved by:
%
\begin{center}
|\def\jobname{|\textit{dest}|}\input{\jobname}|
\end{center}
%
The redirection with prefix
|\childdocforwardprefix[|\textit{prefix}|]{|\textit{dest}|}|
is accomplished by:
%
\begin{center}
\begin{tabular}{l}
|{\edef\jobname{\scantokens\expandafter{\jobname\noexpand}}|\\
|\def\redirectjob |\textit{prefix}|#1~~~{\gdef\jobname{|\textit{dest}|#1}}|\\
|\expandafter\redirectjob\jobname~~~}\input{\jobname}|
\end{tabular}
\end{center}

In an alternative approach,
child documents can be compiled by a specific command line
without additional code or specific definitions:
%
\begin{center}
|... -jobname "|\textit{target}|" "|[\textit{flags}]%
|\includeonly{|\textit{dest}|}\input{|\textit{main}|}"|
\end{center}
%

%%%%%%%%%%%%%%%%%%%%%%%%%%%%%%%%%%%%%%%%%%%%%%%%%%%%%%%%%%%%%%%%%%%%%%%%%%%%%%%%
%%%%%%%%%%%%%%%%%%%%%%%%%%%%%%%%%%%%%%%%%%%%%%%%%%%%%%%%%%%%%%%%%%%%%%%%%%%%%%%%
\section{Information}

%%%%%%%%%%%%%%%%%%%%%%%%%%%%%%%%%%%%%%%%%%%%%%%%%%%%%%%%%%%%%%%%%%%%%%%%%%%%%%%%
\subsection{Copyright}

Copyright \copyright{} 2017--2018 Niklas Beisert

This work may be distributed and/or modified under the
conditions of the \LaTeX{} Project Public License, either version 1.3
of this license or (at your option) any later version.
The latest version of this license is in
  \url{http://www.latex-project.org/lppl.txt}
and version 1.3 or later is part of all distributions of \LaTeX{}
version 2005/12/01 or later.

This work has the LPPL maintenance status `maintained'.

The Current Maintainer of this work is Niklas Beisert.

This work consists of the files |README.txt|, |childdoc.ins| and |childdoc.dtx|
as well as the derived files |childdoc.def|, |cdocsamp.tex|
with |cdocsch1.tex|, |cdocsch2.tex|, |cdocspt3.tex|, |cdocspt4.tex|,
|cdocsdrf.tex|, |cdocsfn1.tex|, |cdocsfn2.tex|
as well as |childdoc.pdf|.

%%%%%%%%%%%%%%%%%%%%%%%%%%%%%%%%%%%%%%%%%%%%%%%%%%%%%%%%%%%%%%%%%%%%%%%%%%%%%%%%
\subsection{Files and Installation}

The package consists of the files:
%
\begin{center}
\begin{tabular}{ll}
    |README.txt|   & readme file \\
    |childdoc.ins| & installation file \\
    |childdoc.dtx| & source file \\
    |childdoc.def| & definition file \\
    |cdocsamp.tex| & sample main file \\
    |cdocsch1.tex| & sample include file \\
    |cdocsch2.tex| & sample include file \\
    |cdocspt3.tex| & sample part file \\
    |cdocspt4.tex| & sample part file \\
    |cdocsdrf.tex| & sample redirection file \\
    |cdocsfn1.tex| & sample redirection file \\
    |cdocsfn2.tex| & sample redirection file \\
    |childdoc.pdf| & manual
\end{tabular}
\end{center}
%
The distribution consists of the files
|README.txt|, |childdoc.ins| and |childdoc.dtx|.
%
\begin{itemize}
\item
Run (pdf)\LaTeX{} on |childdoc.dtx|
to compile the manual |childdoc.pdf| (this file).
\item
Run \LaTeX{} on |childdoc.ins| to create the definitions file |childdoc.def|
and the sample |cdocsamp.tex| with include files
|cdocsch1.tex|, |cdocsch2.tex|, |cdocspt3.tex|, |cdocspt4.tex|,
|cdocsdrf.tex|, |cdocsfn1.tex|, |cdocsfn2.tex|.
Then copy the file |childdoc.def| to an appropriate directory of your \LaTeX{}
distribution, e.g.\ \textit{texmf-root}|/tex/latex/childdoc|.
\end{itemize}

%%%%%%%%%%%%%%%%%%%%%%%%%%%%%%%%%%%%%%%%%%%%%%%%%%%%%%%%%%%%%%%%%%%%%%%%%%%%%%%%
\subsection{Related CTAN Packages}

There are several other packages which offer a similar functionality:
%
\begin{itemize}
\item
The packages
\href{http://ctan.org/pkg/docmute}{\textsf{docmute}},
\href{http://ctan.org/pkg/includex}{\textsf{includex}} and
\href{http://ctan.org/pkg/standalone}{\textsf{standalone}}
provide commands to include only the document body of
a child file thus allowing both files to be compiled individually.
\item
The packages \href{http://ctan.org/pkg/subdocs}{\textsf{subdocs}}
and \href{http://ctan.org/pkg/subfiles}{\textsf{subfiles}}
provide structures in which the main and child documents can be
encapsulated and allowing them to be compiled individually.
The inclusion mechanism is different from the conventional |\include|.
\item
The package \href{http://ctan.org/pkg/combine}{\textsf{combine}}
is an elaborate solution to combine several documents into one.
\end{itemize}
%
See also the CTAN topic \href{http://ctan.org/topic/subdocs}{\textsf{subdocs}}
for further related packages.
The present package differs from the above solutions in that
a document structure constructed with the conventional |\include| mechanism
just needs two extra commands at the top of every file
such that all constituent files can be compiled individually.

%%%%%%%%%%%%%%%%%%%%%%%%%%%%%%%%%%%%%%%%%%%%%%%%%%%%%%%%%%%%%%%%%%%%%%%%%%%%%%%%
%\subsection{Feature Suggestions}
%
%The following is a list of features which may be useful for future
%versions of this package:
%%
%\begin{itemize}
%\item
%\ldots
%\end{itemize}

%%%%%%%%%%%%%%%%%%%%%%%%%%%%%%%%%%%%%%%%%%%%%%%%%%%%%%%%%%%%%%%%%%%%%%%%%%%%%%%%
\subsection{Revision History}

%%%%%%%%%%%%%%%%%%%%%%%%%%%%%%%%%%%%%%%%
\paragraph{v2.0:} 2018/12/30

\begin{itemize}
\item
immediate forward processing
\item
added |\childdocby| mechanism
\item
manual restructured
\end{itemize}

%%%%%%%%%%%%%%%%%%%%%%%%%%%%%%%%%%%%%%%%
\paragraph{v1.6:} 2018/01/17

\begin{itemize}
\item
application for development of include files
\item
corrections to manual
\end{itemize}

%%%%%%%%%%%%%%%%%%%%%%%%%%%%%%%%%%%%%%%%
\paragraph{v1.5:} 2017/05/21

\begin{itemize}
\item
more complete structuring introduced
\item
|\childdocof| introduced
\item
|\childdoc| renamed to |\childdocmain|
\item
|\childredirect| renamed to |\childdocforward| and |\childdocforwardprefix|
and functionality expanded
\end{itemize}

%%%%%%%%%%%%%%%%%%%%%%%%%%%%%%%%%%%%%%%%
\paragraph{v1.0:} 2017/04/27

\begin{itemize}
\item
manual and install package
\item
first version published on CTAN
\end{itemize}

%%%%%%%%%%%%%%%%%%%%%%%%%%%%%%%%%%%%%%%%
\paragraph{v0.6:} 2017/04/26

\begin{itemize}
\item
redirection mechanism added
\end{itemize}

%%%%%%%%%%%%%%%%%%%%%%%%%%%%%%%%%%%%%%%%
\paragraph{v0.5:} 2017/04/26

\begin{itemize}
\item
functionality in definition file
\end{itemize}


%%%%%%%%%%%%%%%%%%%%%%%%%%%%%%%%%%%%%%%%%%%%%%%%%%%%%%%%%%%%%%%%%%%%%%%%%%%%%%%%
%%%%%%%%%%%%%%%%%%%%%%%%%%%%%%%%%%%%%%%%%%%%%%%%%%%%%%%%%%%%%%%%%%%%%%%%%%%%%%%%
%%%%%%%%%%%%%%%%%%%%%%%%%%%%%%%%%%%%%%%%%%%%%%%%%%%%%%%%%%%%%%%%%%%%%%%%%%%%%%%%
\appendix

\settowidth\MacroIndent{\rmfamily\scriptsize 000\ }

 \DocInput{childdoc.dtx}

\end{document}
%</driver>
% \fi
%
% %%%%%%%%%%%%%%%%%%%%%%%%%%%%%%%%%%%%%%%%%%%%%%%%%%%%%%%%%%%%%%%%%%%%%%%%%%%%%%
% %%%%%%%%%%%%%%%%%%%%%%%%%%%%%%%%%%%%%%%%%%%%%%%%%%%%%%%%%%%%%%%%%%%%%%%%%%%%%%
% \section{Sample}
%\iffalse
%<*samplemain>
%\fi
%
% The following presents a sample document
% with two chapters, two parts, a title page,
% a compile flag as well as three forwarding files to set the flag.
% It consists of eight |.tex| files:
% \begin{center}
% \begin{tabular}{ll}
% |cdocsamp.tex|&main file\\
% |cdocsch1.tex|&include file for chapter 1\\
% |cdocsch2.tex|&include file for chapter 2\\
% |cdocspt3.tex|&include file for part 3\\
% |cdocspt4.tex|&include file for part 4\\
% |cdocsdrf.tex|&forwarding file for main file in draft mode\\
% |cdocsfi1.tex|&forwarding file for final version of chapter 1\\
% |cdocsfi2.tex|&forwarding file for final version of chapter 2\\
% \end{tabular}
% \end{center}
% Each of the eight files can be compiled directly by the \LaTeX{} compiler.
%
% %%%%%%%%%%%%%%%%%%%%%%%%%%%%%%%%%%%%%%
% \paragraph{Main File.}
%
% The main file is called |cdocsamp.tex|.
%
% Load the \textsf{childdoc} definitions and
% declare the filename for the main document:
%    \begin{macrocode}
\input{childdoc.def}
\childdocmain{}
%    \end{macrocode}

% Optional override for |\version| flag:
%    \begin{macrocode}
%%\ifchilddoc\else\providecommand{\version}{draft}\fi
%    \end{macrocode}

% Define the default values for the |\version| flag
% (|final| for the main file and |draft| for childs):
%    \begin{macrocode}
\ifchilddoc
\providecommand{\version}{draft}
\else
\providecommand{\version}{final}
\fi
%    \end{macrocode}

% Load the standard document class:
%    \begin{macrocode}
\documentclass[12pt]{article}
%    \end{macrocode}

% Start the document body:
%    \begin{macrocode}
\begin{document}
%    \end{macrocode}

% Declare a title page.
% Print title, part of document being processed and version flag:
%    \begin{macrocode}
\addtocounter{page}{-1}
\begin{center}
{\LARGE\bfseries{}childdoc example\par}
\vspace{1cm}
\ifchilddoc
\ifchilddocmanual part\else chapter\fi:
`\childdocname' of `\childdocjob'\par
\else
main document: `\childdocjob'\par
\fi
version: \version\par
\end{center}
\newpage
%    \end{macrocode}

% Manually include selected file,
% otherwise process as usual:
%    \begin{macrocode}
\ifchilddocmanual
\section*{part `\childdocname'}
\input{\childdocname}
\else
%    \end{macrocode}

% Include the two chapters:
%    \begin{macrocode}
\include{cdocsch1}
\include{cdocsch2}
%    \end{macrocode}

% Include the two parts unless only chapters should be displayed:
%    \begin{macrocode}
\ifchilddoc\else
\section{part three}
\input{cdocspt3}
\section{part four}
\input{cdocspt4}
\fi
%    \end{macrocode}

% Process as usual until here:
%    \begin{macrocode}
\fi
%    \end{macrocode}

% End of document body:
%    \begin{macrocode}
\end{document}
%    \end{macrocode}
%\iffalse
%</samplemain>
%\fi
%
% %%%%%%%%%%%%%%%%%%%%%%%%%%%%%%%%%%%%%%
% \paragraph{Chapter Include Files.}
%
% The include files are called |cdocsch1.tex| and |cdocsch2.tex|.
%
%\iffalse
%<*samplechap1|samplechap2>
%\fi

% Optional override for |\version| flag:
%    \begin{macrocode}
%%\providecommand{\version}{final}
%    \end{macrocode}

% Include the main document:
%    \begin{macrocode}
\input{childdoc.def}
\childdocof{cdocsamp}
%    \end{macrocode}

%\iffalse
%</samplechap1|samplechap2>
%\fi
%
%\iffalse
%<*samplechap1>
%\fi
% Some text for chapter 1:
%    \begin{macrocode}
\section{one}
some text in chapter one
%    \end{macrocode}

%\iffalse
%</samplechap1>
%\fi
% Some text for chapter 2:
%\iffalse
%<*samplechap2>
%\fi
%    \begin{macrocode}
\section{two}
more text in chapter two
%    \end{macrocode}

%\iffalse
%</samplechap2>
%\fi
%
% %%%%%%%%%%%%%%%%%%%%%%%%%%%%%%%%%%%%%%
% \paragraph{Part Include Files.}
%
% The include files are called |cdocspt3.tex| and |cdocspt4.tex|.
%
%\iffalse
%<*samplepart3|samplepart4>
%\fi

% Optional override for |\version| flag:
%    \begin{macrocode}
%%\providecommand{\version}{final}
%    \end{macrocode}

% Include the main document:
%    \begin{macrocode}
\input{childdoc.def}
\childdocby{cdocsamp}
%    \end{macrocode}

%\iffalse
%</samplepart3|samplepart4>
%\fi
%
%\iffalse
%<*samplepart3>
%\fi
% Some text for part 3:
%    \begin{macrocode}
some text in part three
%    \end{macrocode}

%\iffalse
%</samplepart3>
%\fi
% Some text for part 4:
%\iffalse
%<*samplepart4>
%\fi
%    \begin{macrocode}
more text in part four
%    \end{macrocode}

%\iffalse
%</samplepart4>
%\fi
%
% %%%%%%%%%%%%%%%%%%%%%%%%%%%%%%%%%%%%%%
% \paragraph{Forwarding for a Complete Draft.}
%
% The following forwarding file |cdocsdrf.tex|
% compiles the main document in draft mode:
%\iffalse
%<*sampledraft>
%\fi
%    \begin{macrocode}
\def\version{draft}
\input{childdoc.def}
\childdocforward{cdocsamp}
%    \end{macrocode}

%\iffalse
%</sampledraft>
%\fi
%
% %%%%%%%%%%%%%%%%%%%%%%%%%%%%%%%%%%%%%%
% \paragraph{Forwarding for Final Version of the Chapters.}
%
% The following forwarding files |cdocsfn1.tex| and |cdocsfn2.tex|
% (with identical content)
% compile the final versions of the child documents
% |cdocsch1.tex| and |cdocsch2.tex|, respectively:
%\iffalse
%<*samplefinal>
%\fi
%    \begin{macrocode}
\def\version{final}
\input{childdoc.def}
\childdocforwardprefix[cdocsamp]{cdocsfn}{cdocsch}
%    \end{macrocode}

%\iffalse
%</samplefinal>
%\fi
%
% %%%%%%%%%%%%%%%%%%%%%%%%%%%%%%%%%%%%%%
% \paragraph{Command Line Processing.}
%
% The following three command lines generate the output files
% |cdocscld|, |cdocscl1| and |cdocscl2|
% which should be identical to
% |cdocsdrf|, |cdocsch1| and |cdocsfn2|, respectively:
% \begin{center}
% \begin{tabular}{l}
% |latex -jobname cdocscld \|\\
% |  "\def\version{draft}\input{childdoc.def}\childdocforward{cdocsamp}"|\\
% |latex -jobname cdocscl1 \|\\
% |  "\input{childdoc.def}\childdocforward[cdocsamp]{cdocsch1}"|\\
% |latex -jobname cdocscl2 \|\\
% |  "\def\version{final}\input{childdoc.def}\childdocforward{cdocsch2}"|
% \end{tabular}
% \end{center}
% Note that the trailing backslash on each first line
% merely continues the input to the second line
% (for convenient cut ant paste).
% Furthermore, the command |latex| can be replaced by any
% of its alternative versions such as |pdflatex|.
%
% %%%%%%%%%%%%%%%%%%%%%%%%%%%%%%%%%%%%%%%%%%%%%%%%%%%%%%%%%%%%%%%%%%%%%%%%%%%%%%
% %%%%%%%%%%%%%%%%%%%%%%%%%%%%%%%%%%%%%%%%%%%%%%%%%%%%%%%%%%%%%%%%%%%%%%%%%%%%%%
% \section{Implementation}
%\iffalse
%<*package>
%\fi
%
% This section describes the definitions file |childdoc.def|.

% The definitions cannot be loaded using |\usepackage| or |\RequirePackage|
% which has a mechanism to prevent loading a style file more than once.
% When loading the definitions by means of |\input|
% multiple instances have to be prevented manually:
%\iffalse
%This code needs to be before the `\ProvidesFile' directive
%which is defined at the beginning of this file.
%Therefore it is also placed there and commented out here.
%</package>
%<*discard>
%\fi
%    \begin{macrocode}
\ifdefined\childdocmain\endinput\fi
%    \end{macrocode}
%\iffalse
%</discard>
%<*package>
%\fi
%
% \macro{\ifchilddoc}
% \macro{\ifchilddocmanual}
% The conditional |\ifchilddoc| tells whether a
% child (true) or main (false) document is being compiled.
% The conditional |\ifchilddocmanual| tells whether
% the |\includeonly| mechanism is used (false) or
% the selection of child files must be performed manually (true).
% The definitions initialise to false:
%    \begin{macrocode}
\newif\ifchilddoc
\newif\ifchilddocmanual
%    \end{macrocode}

% \macro{\childdocname}
% \macro{\childdocjob}
% The macro |\childdocname| stores the name of the main document
% to be compiled. The macro |\childdocjob| stores the name of
% the document on which the \LaTeX{} compiler was originally invoked.
% The content of |\jobname| cannot be compared
% to filenames specified in the source due to different catcodes.
% The following code rescans |\jobname|, stores the result
% in |\childdocname| and saves a copy in |\childdocjob|:
%    \begin{macrocode}
\edef\childdocname{\scantokens\expandafter{\jobname\noexpand}}
\let\childdocjob\childdocname
%    \end{macrocode}

% \macro{\childdocdisable}
% The macro |\childdocdisable| prevents the main file
% from being processed more than once.
% At this stage, the main document command |\childdocmain|
% is assumed to be called once again where it should do nothing.
% Any subsequent call to it should prevent
% a secondary processing of the main document
% It overwrites the forwarding commands
% |\childdocof| and |\childdocforward|
% with empty macros to prevent further inclusions of the main document:
%    \begin{macrocode}
\newcommand{\childdocdisable}
{
  \renewcommand{\childdocmain}[1]{\renewcommand{\childdocmain}[1]{\endinput}}
  \renewcommand{\childdocof}[1]{}
  \renewcommand{\childdocby}[2][]{}
  \renewcommand{\childdocforward}[2][]{}
  \renewcommand{\childdocdisable}{}
}
%    \end{macrocode}

% \macro{\childdocmain}
% The macro |\childdocmain| is to be called at the top of the main file
% with nothing or the main filename (without extension) as argument.
% First, it breaks loops.
% If the argument is not empty and does not match |\childdocname|
% (which is set by the first inclusion of |childdoc.def|),
% |\ifchilddoc| is set to true, |\includeonly| is applied to the child file
% and |\jobname| is set to the main file
% (for proper handling of |.aux| files):
%    \begin{macrocode}
\newcommand{\childdocmain}[1]
{
  \childdocdisable\childdocmain{}
  \if?#1?\else
    \begingroup
      \def\childdoctmp{#1}
      \ifx\childdoctmp\childdocname
        \def\childdoctmp{}
      \else
        \def\childdoctmp
        {
          \childdoctrue
          \includeonly{\childdocname}
          \def\childdocjob{#1}
          \def\jobname{#1}
        }
      \fi
      \expandafter
    \endgroup
    \childdoctmp
  \fi
}
%    \end{macrocode}

% \macro{\childdocof}
% The command |\childdocof| redirects
% compilation to the main file |#1|.
%    \begin{macrocode}
\newcommand{\childdocof}[1]
{
  \childdocdisable
  \childdoctrue
  \includeonly{\childdocname}
  \def\jobname{#1}
  \def\childdocjob{#1}
  \input{#1}
}
%    \end{macrocode}

% \macro{\childdocby}
% The command |\childdocby| ....
%    \begin{macrocode}
\newcommand{\childdocby}[2][]
{
  \childdocdisable
  \childdoctrue
  \childdocmanualtrue
  \if?#1?\else
    \def\jobname{#2}
  \fi
  \def\childdocjob{#2}
  \input{#2}
  \endinput
}
%    \end{macrocode}

% \macro{\childdocforward}
% The command |\childdocforward| redirects
% compilation to the main file or
% (if the optional argument is given) a child file.
% Parameters are set as if the main file
% or a child file starting with |\childdocof| was compiled.
% Then compilation is handed over to the main file:
%    \begin{macrocode}
\newcommand{\childdocforward}[2][]
{
  \begingroup
    \if?#1?
      \def\childdoctmp
      {
        \def\childdocname{#2}
        \def\childdocjob{#2}
        \def\jobname{#2}
        \input{#2}
        \endinput
      }
    \else
      \def\childdoctmp
      {
        \childdocdisable
        \def\childdocname{#2}
        \childdoctrue
        \includeonly{#2}
        \def\childdocjob{#1}
        \def\jobname{#1}
        \input{#1}
        \endinput
      }
    \fi
    \expandafter
  \endgroup
  \childdoctmp
}
%    \end{macrocode}

% \macro{\childdocforwardprefix}
% The command |\childdocforwardprefix| redirects
% compilation to the main or a child file by means of a pattern.
% The prefix |#1| in the current filename is replaced by |#2|
% and the suffix of the current filename is kept
% (it is assumed that the filename does not contain the substring `|~~~|'
% which is used as a delimiter).
% Compilation is handed over to the new file by |\childdocforward|:
%    \begin{macrocode}
\newcommand{\childdocforwardprefix}[3][]
{
  \begingroup
    \def\childdocextract #2##1~~~{\def\childdoctmp{\childdocforward[#1]{#3##1}}}
    \expandafter\childdocextract\childdocname~~~
    \expandafter
  \endgroup
  \childdoctmp
}
%    \end{macrocode}

% \macro{\childdoc}
% The deprecated macro |\childdoc| is a legacy version of |\childdocmain|:
%    \begin{macrocode}
\newcommand{\childdoc}{\childdocmain}
%    \end{macrocode}

% \macro{\childdocredirect}
% The deprecated macro |\childdocredirect| is a legacy version
% of |\childdocforward| and |\childdocforwardprefix|:
%    \begin{macrocode}
\newcommand{\childdocredirect}[2][]
{
  \begingroup
    \if?#1?
      \def\childdoctmp{\childdocforward{#2}}
    \else
      \def\childdoctmp{\childdocforwardprefix{#1}{#2}}
    \fi
    \expandafter
  \endgroup
  \childdoctmp
}
%    \end{macrocode}

%\iffalse
%</package>
%\fi
%
\endinput
\childdocforward{cdocsch2}"|
% \end{tabular}
% \end{center}
% Note that the trailing backslash on each first line
% merely continues the input to the second line
% (for convenient cut ant paste).
% Furthermore, the command |latex| can be replaced by any
% of its alternative versions such as |pdflatex|.
%
% %%%%%%%%%%%%%%%%%%%%%%%%%%%%%%%%%%%%%%%%%%%%%%%%%%%%%%%%%%%%%%%%%%%%%%%%%%%%%%
% %%%%%%%%%%%%%%%%%%%%%%%%%%%%%%%%%%%%%%%%%%%%%%%%%%%%%%%%%%%%%%%%%%%%%%%%%%%%%%
% \section{Implementation}
%\iffalse
%<*package>
%\fi
%
% This section describes the definitions file |childdoc.def|.

% The definitions cannot be loaded using |\usepackage| or |\RequirePackage|
% which has a mechanism to prevent loading a style file more than once.
% When loading the definitions by means of |\input|
% multiple instances have to be prevented manually:
%\iffalse
%This code needs to be before the `\ProvidesFile' directive
%which is defined at the beginning of this file.
%Therefore it is also placed there and commented out here.
%</package>
%<*discard>
%\fi
%    \begin{macrocode}
\ifdefined\childdocmain\endinput\fi
%    \end{macrocode}
%\iffalse
%</discard>
%<*package>
%\fi
%
% \macro{\ifchilddoc}
% \macro{\ifchilddocmanual}
% The conditional |\ifchilddoc| tells whether a
% child (true) or main (false) document is being compiled.
% The conditional |\ifchilddocmanual| tells whether
% the |\includeonly| mechanism is used (false) or
% the selection of child files must be performed manually (true).
% The definitions initialise to false:
%    \begin{macrocode}
\newif\ifchilddoc
\newif\ifchilddocmanual
%    \end{macrocode}

% \macro{\childdocname}
% \macro{\childdocjob}
% The macro |\childdocname| stores the name of the main document
% to be compiled. The macro |\childdocjob| stores the name of
% the document on which the \LaTeX{} compiler was originally invoked.
% The content of |\jobname| cannot be compared
% to filenames specified in the source due to different catcodes.
% The following code rescans |\jobname|, stores the result
% in |\childdocname| and saves a copy in |\childdocjob|:
%    \begin{macrocode}
\edef\childdocname{\scantokens\expandafter{\jobname\noexpand}}
\let\childdocjob\childdocname
%    \end{macrocode}

% \macro{\childdocdisable}
% The macro |\childdocdisable| prevents the main file
% from being processed more than once.
% At this stage, the main document command |\childdocmain|
% is assumed to be called once again where it should do nothing.
% Any subsequent call to it should prevent
% a secondary processing of the main document
% It overwrites the forwarding commands
% |\childdocof| and |\childdocforward|
% with empty macros to prevent further inclusions of the main document:
%    \begin{macrocode}
\newcommand{\childdocdisable}
{
  \renewcommand{\childdocmain}[1]{\renewcommand{\childdocmain}[1]{\endinput}}
  \renewcommand{\childdocof}[1]{}
  \renewcommand{\childdocby}[2][]{}
  \renewcommand{\childdocforward}[2][]{}
  \renewcommand{\childdocdisable}{}
}
%    \end{macrocode}

% \macro{\childdocmain}
% The macro |\childdocmain| is to be called at the top of the main file
% with nothing or the main filename (without extension) as argument.
% First, it breaks loops.
% If the argument is not empty and does not match |\childdocname|
% (which is set by the first inclusion of |childdoc.def|),
% |\ifchilddoc| is set to true, |\includeonly| is applied to the child file
% and |\jobname| is set to the main file
% (for proper handling of |.aux| files):
%    \begin{macrocode}
\newcommand{\childdocmain}[1]
{
  \childdocdisable\childdocmain{}
  \if?#1?\else
    \begingroup
      \def\childdoctmp{#1}
      \ifx\childdoctmp\childdocname
        \def\childdoctmp{}
      \else
        \def\childdoctmp
        {
          \childdoctrue
          \includeonly{\childdocname}
          \def\childdocjob{#1}
          \def\jobname{#1}
        }
      \fi
      \expandafter
    \endgroup
    \childdoctmp
  \fi
}
%    \end{macrocode}

% \macro{\childdocof}
% The command |\childdocof| redirects
% compilation to the main file |#1|.
%    \begin{macrocode}
\newcommand{\childdocof}[1]
{
  \childdocdisable
  \childdoctrue
  \includeonly{\childdocname}
  \def\jobname{#1}
  \def\childdocjob{#1}
  \input{#1}
}
%    \end{macrocode}

% \macro{\childdocby}
% The command |\childdocby| ....
%    \begin{macrocode}
\newcommand{\childdocby}[2][]
{
  \childdocdisable
  \childdoctrue
  \childdocmanualtrue
  \if?#1?\else
    \def\jobname{#2}
  \fi
  \def\childdocjob{#2}
  \input{#2}
  \endinput
}
%    \end{macrocode}

% \macro{\childdocforward}
% The command |\childdocforward| redirects
% compilation to the main file or
% (if the optional argument is given) a child file.
% Parameters are set as if the main file
% or a child file starting with |\childdocof| was compiled.
% Then compilation is handed over to the main file:
%    \begin{macrocode}
\newcommand{\childdocforward}[2][]
{
  \begingroup
    \if?#1?
      \def\childdoctmp
      {
        \def\childdocname{#2}
        \def\childdocjob{#2}
        \def\jobname{#2}
        \input{#2}
        \endinput
      }
    \else
      \def\childdoctmp
      {
        \childdocdisable
        \def\childdocname{#2}
        \childdoctrue
        \includeonly{#2}
        \def\childdocjob{#1}
        \def\jobname{#1}
        \input{#1}
        \endinput
      }
    \fi
    \expandafter
  \endgroup
  \childdoctmp
}
%    \end{macrocode}

% \macro{\childdocforwardprefix}
% The command |\childdocforwardprefix| redirects
% compilation to the main or a child file by means of a pattern.
% The prefix |#1| in the current filename is replaced by |#2|
% and the suffix of the current filename is kept
% (it is assumed that the filename does not contain the substring `|~~~|'
% which is used as a delimiter).
% Compilation is handed over to the new file by |\childdocforward|:
%    \begin{macrocode}
\newcommand{\childdocforwardprefix}[3][]
{
  \begingroup
    \def\childdocextract #2##1~~~{\def\childdoctmp{\childdocforward[#1]{#3##1}}}
    \expandafter\childdocextract\childdocname~~~
    \expandafter
  \endgroup
  \childdoctmp
}
%    \end{macrocode}

% \macro{\childdoc}
% The deprecated macro |\childdoc| is a legacy version of |\childdocmain|:
%    \begin{macrocode}
\newcommand{\childdoc}{\childdocmain}
%    \end{macrocode}

% \macro{\childdocredirect}
% The deprecated macro |\childdocredirect| is a legacy version
% of |\childdocforward| and |\childdocforwardprefix|:
%    \begin{macrocode}
\newcommand{\childdocredirect}[2][]
{
  \begingroup
    \if?#1?
      \def\childdoctmp{\childdocforward{#2}}
    \else
      \def\childdoctmp{\childdocforwardprefix{#1}{#2}}
    \fi
    \expandafter
  \endgroup
  \childdoctmp
}
%    \end{macrocode}

%\iffalse
%</package>
%\fi
%
\endinput
|\\
|\childdocforwardprefix[|\textit{main}|]{|\textit{prefix}|}{|\textit{dest}|}|
\end{tabular}
\end{center}
%
the destination file is determined by a pattern
depending on the current file:
To make this work, the current file must be called
`{\textit{prefix}\hspace{0.2em}\textit{suffix}}'
with \textit{prefix} matching precisely the argument.
Processing is then passed on to the file
`{\textit{dest}\hspace{0.2em}\textit{suffix}}'.
Surely, the same effect is achieved by
directly specifying the
argument `{\textit{dest}\hspace{0.2em}\textit{suffix}}'
in the first form.
However, that requires to set up a different file
for each child. With the alternative form of the command
all these files can have exactly the same content
which simplifies setting them up and maintaining them.

For example, the following file |draft.tex|
with a compilation flag |\version| as described in \secref{sec:flags}
compiles the main document as a draft:
%
\begin{center}
\begin{tabular}{l}
|\def\version{draft}|\\
|% \iffalse
%
% childdoc.dtx Copyright (C) 2017-2018 Niklas Beisert
%
% This work may be distributed and/or modified under the
% conditions of the LaTeX Project Public License, either version 1.3
% of this license or (at your option) any later version.
% The latest version of this license is in
%   http://www.latex-project.org/lppl.txt
% and version 1.3 or later is part of all distributions of LaTeX
% version 2005/12/01 or later.
%
% This work has the LPPL maintenance status `maintained'.
%
% The Current Maintainer of this work is Niklas Beisert.
%
% This work consists of the files childdoc.dtx and childdoc.ins
% and the derived files childdoc.def and cdocsamp.tex with
% cdocsch1.tex, cdocsch2.tex, cdocsdrf.tex, cdocsfn1.tex, cdocsfn2.tex.
%
%<package>\ifdefined\childdocmain\endinput\fi
%<package>\ProvidesFile{childdoc.def}[2018/12/30 v2.0 child document driver]
%<samplemain>\ProvidesFile{cdocsamp.tex}[2018/12/30 v2.0 sample for childdoc]
%<*driver>
%\ProvidesFile{childdoc.drv}[2018/12/30 v2.0 childdoc reference manual file]
\PassOptionsToClass{10pt,a4paper}{article}
\documentclass{ltxdoc}

\usepackage[margin=35mm]{geometry}
\usepackage{hyperref}
\usepackage{hyperxmp}
\usepackage[usenames]{color}

\hypersetup{colorlinks=true}
\hypersetup{pdfstartview=FitH}
\hypersetup{pdfpagemode=UseNone}
\hypersetup{pdfsource={}}
\hypersetup{pdflang={en-UK}}
\hypersetup{pdfcopyright={Copyright 2017-2018 Niklas Beisert.
  This work may be distributed and/or modified under the
  conditions of the LaTeX Project Public License, either version 1.3
  of this license or (at your option) any later version.}}
\hypersetup{pdflicenseurl={http://www.latex-project.org/lppl.txt}}
\hypersetup{pdfcontactaddress={ETH Zurich, ITP, HIT K,
  Wolfgang-Pauli-Strasse 27}}
\hypersetup{pdfcontactpostcode={8093}}
\hypersetup{pdfcontactcity={Zurich}}
\hypersetup{pdfcontactcountry={Switzerland}}
\hypersetup{pdfcontactemail={nbeisert@itp.phys.ethz.ch}}
\hypersetup{pdfcontacturl={http://people.phys.ethz.ch/\xmptilde nbeisert/}}

\newcommand{\secref}[1]{\hyperref[#1]{section \ref*{#1}}}

\parskip1ex
\parindent0pt
\let\olditemize\itemize
\def\itemize{\olditemize\parskip0pt}

\begin{document}

\title{The \textsf{childdoc} Package}
\hypersetup{pdftitle={The childdoc Package}}
\author{Niklas Beisert\\[2ex]
  Institut f\"ur Theoretische Physik\\
  Eidgen\"ossische Technische Hochschule Z\"urich\\
  Wolfgang-Pauli-Strasse 27, 8093 Z\"urich, Switzerland\\[1ex]
  \href{mailto:nbeisert@itp.phys.ethz.ch}
  {\texttt{nbeisert@itp.phys.ethz.ch}}}
\hypersetup{pdfauthor={Niklas Beisert}}
\hypersetup{pdfsubject={Manual for the LaTeX2e Package childdoc}}
\date{30 December 2018, \textsf{v2.0}}
\maketitle

\begin{abstract}\noindent
\textsf{childdoc} is a \LaTeXe{} package
that enables the direct compilation
of document sections included by |\include|
to individual files.
\end{abstract}

\begingroup
\parskip0ex
\tableofcontents
\endgroup

%%%%%%%%%%%%%%%%%%%%%%%%%%%%%%%%%%%%%%%%%%%%%%%%%%%%%%%%%%%%%%%%%%%%%%%%%%%%%%%%
%%%%%%%%%%%%%%%%%%%%%%%%%%%%%%%%%%%%%%%%%%%%%%%%%%%%%%%%%%%%%%%%%%%%%%%%%%%%%%%%
\section{Introduction}

\LaTeX{} provides a mechanism to structure a large document (such as a book)
into a main file and several child files (containing the chapters)
using the |\include| command.
This mechanism is beneficial for documents
which span hundreds of pages in order to
make the source file(s) more manageable.
Moreover, compilation can be restricted to
selected child files by means of the |\includeonly| command.
The latter feature can be used to reduce the compilation time while editing
(this was significantly more useful in the earlier days of \LaTeX{})
or to generate a smaller document which is easier to navigate.
Another application of |\includeonly| is to generate
documents consisting of selected parts of the complete document.

However, there are a few drawbacks of the plain |\include| mechanism:
\begin{itemize}
\item
The child files cannot be compiled on their own,
they can only be compiled via the main file.
A naive editing environment
(such as a text editor with an option
to have the current file processed by \LaTeX)
may require one to switch to the main file before compiling;
attempting to compile the child file produces errors.
\item
The main file must be modified (each time)
to adjust the |\includeonly| command
to the present needs. This easily leaves the main file in a messy state.
\item
The generated document will always carry the filename
of the main document. This is inconvenient if
several child files are to be compiled and
to be kept for distribution.
\end{itemize}

The present package provides a simple interface
to make child files individually compilable by \LaTeX{}.
Compiling a child file then has the same effect as compiling
the main file with an |\includeonly| command
to select the appropriate child.
Moreover the generated document will carry the name of the child
rather than the main file.
This resolves all three above issues.

This feature is meant to make the editing of books,
thesis documents and lecture notes somewhat more convenient.
However, the package can also be used efficiently for
composing a series of documents (such as exercise sheets)
which are typically distributed individually.
It then assists the author in generating the individual documents
(potentially in different versions)
as well as a document containing the collected series.
Another application is in developing style files
or other kinds of included material
where compilation of the style file could redirect
to a sample or test file.

%%%%%%%%%%%%%%%%%%%%%%%%%%%%%%%%%%%%%%%%%%%%%%%%%%%%%%%%%%%%%%%%%%%%%%%%%%%%%%%%
%%%%%%%%%%%%%%%%%%%%%%%%%%%%%%%%%%%%%%%%%%%%%%%%%%%%%%%%%%%%%%%%%%%%%%%%%%%%%%%%
\section{Usage}

First of all, the package \textsf{childdoc} is \emph{not} a standard
\LaTeXe{} |.sty| style file! Therefore it needs to be invoked in
a non-standard way.

%%%%%%%%%%%%%%%%%%%%%%%%%%%%%%%%%%%%%%%%%%%%%%%%%%%%%%%%%%%%%%%%%%%%%%%%%%%%%%%%
\subsection{Included Files}
\label{sec:include}

%%%%%%%%%%%%%%%%%%%%%%%%%%%%%%%%%%%%%%%%
\DescribeMacro{\childdocmain}
To use the package, add the commands
\begin{center}
\begin{tabular}{l}
|% \iffalse
%
% childdoc.dtx Copyright (C) 2017-2018 Niklas Beisert
%
% This work may be distributed and/or modified under the
% conditions of the LaTeX Project Public License, either version 1.3
% of this license or (at your option) any later version.
% The latest version of this license is in
%   http://www.latex-project.org/lppl.txt
% and version 1.3 or later is part of all distributions of LaTeX
% version 2005/12/01 or later.
%
% This work has the LPPL maintenance status `maintained'.
%
% The Current Maintainer of this work is Niklas Beisert.
%
% This work consists of the files childdoc.dtx and childdoc.ins
% and the derived files childdoc.def and cdocsamp.tex with
% cdocsch1.tex, cdocsch2.tex, cdocsdrf.tex, cdocsfn1.tex, cdocsfn2.tex.
%
%<package>\ifdefined\childdocmain\endinput\fi
%<package>\ProvidesFile{childdoc.def}[2018/12/30 v2.0 child document driver]
%<samplemain>\ProvidesFile{cdocsamp.tex}[2018/12/30 v2.0 sample for childdoc]
%<*driver>
%\ProvidesFile{childdoc.drv}[2018/12/30 v2.0 childdoc reference manual file]
\PassOptionsToClass{10pt,a4paper}{article}
\documentclass{ltxdoc}

\usepackage[margin=35mm]{geometry}
\usepackage{hyperref}
\usepackage{hyperxmp}
\usepackage[usenames]{color}

\hypersetup{colorlinks=true}
\hypersetup{pdfstartview=FitH}
\hypersetup{pdfpagemode=UseNone}
\hypersetup{pdfsource={}}
\hypersetup{pdflang={en-UK}}
\hypersetup{pdfcopyright={Copyright 2017-2018 Niklas Beisert.
  This work may be distributed and/or modified under the
  conditions of the LaTeX Project Public License, either version 1.3
  of this license or (at your option) any later version.}}
\hypersetup{pdflicenseurl={http://www.latex-project.org/lppl.txt}}
\hypersetup{pdfcontactaddress={ETH Zurich, ITP, HIT K,
  Wolfgang-Pauli-Strasse 27}}
\hypersetup{pdfcontactpostcode={8093}}
\hypersetup{pdfcontactcity={Zurich}}
\hypersetup{pdfcontactcountry={Switzerland}}
\hypersetup{pdfcontactemail={nbeisert@itp.phys.ethz.ch}}
\hypersetup{pdfcontacturl={http://people.phys.ethz.ch/\xmptilde nbeisert/}}

\newcommand{\secref}[1]{\hyperref[#1]{section \ref*{#1}}}

\parskip1ex
\parindent0pt
\let\olditemize\itemize
\def\itemize{\olditemize\parskip0pt}

\begin{document}

\title{The \textsf{childdoc} Package}
\hypersetup{pdftitle={The childdoc Package}}
\author{Niklas Beisert\\[2ex]
  Institut f\"ur Theoretische Physik\\
  Eidgen\"ossische Technische Hochschule Z\"urich\\
  Wolfgang-Pauli-Strasse 27, 8093 Z\"urich, Switzerland\\[1ex]
  \href{mailto:nbeisert@itp.phys.ethz.ch}
  {\texttt{nbeisert@itp.phys.ethz.ch}}}
\hypersetup{pdfauthor={Niklas Beisert}}
\hypersetup{pdfsubject={Manual for the LaTeX2e Package childdoc}}
\date{30 December 2018, \textsf{v2.0}}
\maketitle

\begin{abstract}\noindent
\textsf{childdoc} is a \LaTeXe{} package
that enables the direct compilation
of document sections included by |\include|
to individual files.
\end{abstract}

\begingroup
\parskip0ex
\tableofcontents
\endgroup

%%%%%%%%%%%%%%%%%%%%%%%%%%%%%%%%%%%%%%%%%%%%%%%%%%%%%%%%%%%%%%%%%%%%%%%%%%%%%%%%
%%%%%%%%%%%%%%%%%%%%%%%%%%%%%%%%%%%%%%%%%%%%%%%%%%%%%%%%%%%%%%%%%%%%%%%%%%%%%%%%
\section{Introduction}

\LaTeX{} provides a mechanism to structure a large document (such as a book)
into a main file and several child files (containing the chapters)
using the |\include| command.
This mechanism is beneficial for documents
which span hundreds of pages in order to
make the source file(s) more manageable.
Moreover, compilation can be restricted to
selected child files by means of the |\includeonly| command.
The latter feature can be used to reduce the compilation time while editing
(this was significantly more useful in the earlier days of \LaTeX{})
or to generate a smaller document which is easier to navigate.
Another application of |\includeonly| is to generate
documents consisting of selected parts of the complete document.

However, there are a few drawbacks of the plain |\include| mechanism:
\begin{itemize}
\item
The child files cannot be compiled on their own,
they can only be compiled via the main file.
A naive editing environment
(such as a text editor with an option
to have the current file processed by \LaTeX)
may require one to switch to the main file before compiling;
attempting to compile the child file produces errors.
\item
The main file must be modified (each time)
to adjust the |\includeonly| command
to the present needs. This easily leaves the main file in a messy state.
\item
The generated document will always carry the filename
of the main document. This is inconvenient if
several child files are to be compiled and
to be kept for distribution.
\end{itemize}

The present package provides a simple interface
to make child files individually compilable by \LaTeX{}.
Compiling a child file then has the same effect as compiling
the main file with an |\includeonly| command
to select the appropriate child.
Moreover the generated document will carry the name of the child
rather than the main file.
This resolves all three above issues.

This feature is meant to make the editing of books,
thesis documents and lecture notes somewhat more convenient.
However, the package can also be used efficiently for
composing a series of documents (such as exercise sheets)
which are typically distributed individually.
It then assists the author in generating the individual documents
(potentially in different versions)
as well as a document containing the collected series.
Another application is in developing style files
or other kinds of included material
where compilation of the style file could redirect
to a sample or test file.

%%%%%%%%%%%%%%%%%%%%%%%%%%%%%%%%%%%%%%%%%%%%%%%%%%%%%%%%%%%%%%%%%%%%%%%%%%%%%%%%
%%%%%%%%%%%%%%%%%%%%%%%%%%%%%%%%%%%%%%%%%%%%%%%%%%%%%%%%%%%%%%%%%%%%%%%%%%%%%%%%
\section{Usage}

First of all, the package \textsf{childdoc} is \emph{not} a standard
\LaTeXe{} |.sty| style file! Therefore it needs to be invoked in
a non-standard way.

%%%%%%%%%%%%%%%%%%%%%%%%%%%%%%%%%%%%%%%%%%%%%%%%%%%%%%%%%%%%%%%%%%%%%%%%%%%%%%%%
\subsection{Included Files}
\label{sec:include}

%%%%%%%%%%%%%%%%%%%%%%%%%%%%%%%%%%%%%%%%
\DescribeMacro{\childdocmain}
To use the package, add the commands
\begin{center}
\begin{tabular}{l}
|\input{childdoc.def}|\\
|\childdocmain{}|\\
\end{tabular}
\end{center}
at the very top of the main \LaTeX{} file,
in particular \emph{before} the |\documentclass| statement!
The argument of |\childdocmain| should be left empty
(but it must be present).

%%%%%%%%%%%%%%%%%%%%%%%%%%%%%%%%%%%%%%%%
\DescribeMacro{\childdocof}
Furthermore, add the commands
\begin{center}
\begin{tabular}{l}
|\input{childdoc.def}|\\
|\childdocof{|\textit{main}|}|\\
\end{tabular}
\end{center}
at the top of every child file \textit{child}
which is included by |\include{|\textit{child}|}|
from within the main file
(or at least for those files to be compiled individually).
The argument \textit{main} must be the filename of the main file.

There are a couple of
considerations in setting up the main and child documents:

%%%%%%%%%%%%%%%%%%%%%%%%%%%%%%%%%%%%%%%%
\paragraph{Restrictions.}

Please note the following restrictions:
\begin{itemize}
\item
|\childdocmain| must be called with one argument \textit{main}
to ensure compatibility with earlier version of the package.
It must either be empty (|\childdocmain{}|)
or precisely match the filename of the main file in which it is specified.
See \secref{sec:detection} for further information.
\item
The filename \textit{main} must be specified without the |.tex| extension.
\item
The filename \textit{main} is case sensitive
(even in case-insensitive file systems)
due to internal string comparison.
\item
The argument \textit{main} should be fully expanded, it cannot be a macro.
\item
Subdirectories and special characters should be avoided in filenames.
\item
The command |\childdocmain{|\textit{main}|}| must be followed by a whitespace.
It should not be followed immediately by another command
or by a comment mark `|%|'.
This is because the \TeX{} parser reads the token immediately following
the argument of |\childdocmain| and puts it
at the beginning of every child section;
however, a white\-space is ignored.
\end{itemize}

%%%%%%%%%%%%%%%%%%%%%%%%%%%%%%%%%%%%%%%%
\paragraph{Content of Main File.}

It is advisable to place all content in the child files included by |\include|.
Any output contained in the main file will appear in all child documents
unless suppressed manually;
it cannot be suppressed automatically by the |\includeonly| directive
and thus should normally be avoided.
A method to include some content in the main file
by means of conditional processing is described in \secref{sec:conditional}.

%%%%%%%%%%%%%%%%%%%%%%%%%%%%%%%%%%%%%%%%
\paragraph{Page Numbering.}

When only a part of the document is compiled,
the appropriate numbering of pages
(as well as other status parameters)
is determined from the |.aux| files.
The latter contain information from previous passes.
However this information needs to propagate through
all intermediate child documents.
Therefore the page numbering in child documents may well
be inconsistent until the complete document is compiled at least once.

A useful (if unconventional) way to always ensure a consistent
page numbering is to restart the numbering in each child document
and denote the pages by `\textit{child}|.|\textit{page}'
where \textit{child} represents the chapter/section number of the child file.
This can be achieved by the command
|\numberwithin{page}{|\textit{child}|}|
of the \textsf{amsmath} package
where \textit{child} can be |chapter| or |section|
depending on the chosen structuring.
Alternatively, one can modify the macro |\thepage| appropriately
and reset the counter |page| at the start of each child file.

%%%%%%%%%%%%%%%%%%%%%%%%%%%%%%%%%%%%%%%%%%%%%%%%%%%%%%%%%%%%%%%%%%%%%%%%%%%%%%%%
\subsection{Conditional Processing}
\label{sec:conditional}

The package provides a mechanism to compile different versions
of a document. To customise the versions further some conditional processing
can come in handy to distinguish which version is being compiled.
The package provides two macros to describe the compilation context:

%%%%%%%%%%%%%%%%%%%%%%%%%%%%%%%%%%%%%%%%
\DescribeMacro{\ifchilddoc}
The conditional |\ifchilddoc| distinguishes between the compilation of
child documents and the main document:
%
\begin{center}
|\ifchilddoc |\textit{child-code}| |[|\||else |\textit{main-code}]| \||fi|
\end{center}

%%%%%%%%%%%%%%%%%%%%%%%%%%%%%%%%%%%%%%%%
\DescribeMacro{\childdocname}
\DescribeMacro{\childdocjob}
The macro |\childdocname| contains the filename (without extension)
of the main or child file being processed.
Note that |\childdocjob| will always contain the name of the main file.

%%%%%%%%%%%%%%%%%%%%%%%%%%%%%%%%%%%%%%%%
\paragraph{Title Page.}

Conditional processing can be used to include a title or banner page
in the main document when proper precautions are taken.
Importantly, the code in the main file should ensure that the page counter
(as well as other status parameters which are stored in the |.aux| files)
takes the same value after the conditional processing.
Otherwise the page numbers may take divergent values
depending on which part is compiled.

For example, a title page could be declared by:
%
\begin{center}
\begin{tabular}{l}
|\ifchilddoc\||else|\\
|\addtocounter{page}{-1}|\\
\textit{code for title page}\\
|\newpage|\\
|\||fi|
\end{tabular}
\end{center}
%
A banner page for the child documents can be generated by:
%
\begin{center}
\begin{tabular}{l}
|\ifchilddoc|\\
|\addtocounter{page}{-1}|\\
\textit{code for banner page}\\
|\newpage|\\
|\||fi|
\end{tabular}
\end{center}
%
Here one could write a message such as:
\begin{center}
|This is the part \childdocname{} of \childdocjob{}.|
\end{center}

%%%%%%%%%%%%%%%%%%%%%%%%%%%%%%%%%%%%%%%%%%%%%%%%%%%%%%%%%%%%%%%%%%%%%%%%%%%%%%%%
\subsection{Flags}
\label{sec:flags}

The package makes it easy to generate different versions
of the main or child documents.
To this end compilation flags can be defined
and assigned different default values.
They will be particularly useful in conjunction
with the forwarding mechanism described in \secref{sec:forward}.

For example, it may be useful to have a flag |\version|
which can be set to |draft| or |final|.
The document source will contain some conditional code
depending on the value of |\version|.
Suppose further, the flag should default to |final| for the main file
and to |draft| for child files
which is a natural assignment for editing the document.
This is achieved by placing the following code
in the preamble of the main document
(below the |\childdocmain| directive):
%
\begin{center}
\begin{tabular}{l}
|\ifchilddoc|\\
|\providecommand{\version}{draft}|\\
|\||else|\\
|\providecommand{\version}{final}|\\
|\||fi|
\end{tabular}
\end{center}
%
The definition by |\providecommand| makes sure
that previous definitions are not overwritten.
Further statements |\providecommand{\version}{...}|
can thus be added before the above code to override it.

For the main file, one might add a line
(between |\childdocmain| and the above block)
%
\begin{center}
|%\ifchilddoc\||else\providecommand{\version}{draft}\||fi|
\end{center}
%
which can be uncommented to produce a draft version.
Likewise one can add a line to the very top of a child file
(above the |\childdocof{|\textit{main}|}| directive)
%
\begin{center}
|%\providecommand{\version}{final}|
\end{center}
%
which can be uncommented to produce the final version of this child document.

%%%%%%%%%%%%%%%%%%%%%%%%%%%%%%%%%%%%%%%%%%%%%%%%%%%%%%%%%%%%%%%%%%%%%%%%%%%%%%%%
\subsection{Forwarding}
\label{sec:forward}

Different versions of the main or child documents
using compilation flags as described in \secref{sec:flags}
can be (permanently) stored in different files
for convenient compilation, viewing and distribution.
To this end, the package defines a command
to pass on compilation to a different file:

%%%%%%%%%%%%%%%%%%%%%%%%%%%%%%%%%%%%%%%%
\DescribeMacro{\childdocforward}
The command |\childdocforward| redirects processing to
another source file:
%
\begin{center}
\begin{tabular}{l}
|\input{childdoc.def}|\\
|\childdocforward[|\textit{main}|]{|\textit{dest}|}|\\
\end{tabular}
\end{center}
%
The argument \textit{dest} is the destination file
(without extension).
It should be the main file or one of the child files.
Note that further \textsf{childdoc} directives
such as |\childdocof| and |\childdocforward|
in the indicated file will be processed in this form.
The optional argument \textit{main}
passes on directly to the main file \textit{main}
while pretending to compile the child \textit{dest}.
This form behaves as if \textit{dest}
issues |\childdocof{|\textit{main}|}| right away,
and no further \textsf{childdoc} directives will be processed.

%%%%%%%%%%%%%%%%%%%%%%%%%%%%%%%%%%%%%%%%
\DescribeMacro{\...prefix}
In the alternative form |\childdocforwardprefix|,
%
\begin{center}
\begin{tabular}{l}
|\input{childdoc.def}|\\
|\childdocforwardprefix[|\textit{main}|]{|\textit{prefix}|}{|\textit{dest}|}|
\end{tabular}
\end{center}
%
the destination file is determined by a pattern
depending on the current file:
To make this work, the current file must be called
`{\textit{prefix}\hspace{0.2em}\textit{suffix}}'
with \textit{prefix} matching precisely the argument.
Processing is then passed on to the file
`{\textit{dest}\hspace{0.2em}\textit{suffix}}'.
Surely, the same effect is achieved by
directly specifying the
argument `{\textit{dest}\hspace{0.2em}\textit{suffix}}'
in the first form.
However, that requires to set up a different file
for each child. With the alternative form of the command
all these files can have exactly the same content
which simplifies setting them up and maintaining them.

For example, the following file |draft.tex|
with a compilation flag |\version| as described in \secref{sec:flags}
compiles the main document as a draft:
%
\begin{center}
\begin{tabular}{l}
|\def\version{draft}|\\
|\input{childdoc.def}|\\
|\childdocforward{|\textit{main}|}|
\end{tabular}
\end{center}
%
Likewise, the following files |final|\textit{nn}|.tex|
compile the final version of the child document
|child|\textit{nn}|.tex|:
%
\begin{center}
\begin{tabular}{l}
|\def\version{final}|\\
|\input{childdoc.def}|\\
|\childdocforwardprefix{final}{child}|
\end{tabular}
\end{center}
%

Note that when several versions of a main file and/or of each child file
are to be generated, it may be convenient to set up a |Makefile| or
shell script to automatise the process.

%%%%%%%%%%%%%%%%%%%%%%%%%%%%%%%%%%%%%%%%%%%%%%%%%%%%%%%%%%%%%%%%%%%%%%%%%%%%%%%%
\subsection{Command Line Processing}
\label{sec:commandline}

The effect of redirection files can also be achieved by invoking
the \LaTeX{} compiler with a more elaborate command line.
Most conveniently this should be done as part
of a shell script or a |Makefile|.

When using \textsf{childdoc} in the main file, the following
command lines effectively perform a redirection
(note that depending on the shell being used,
backslashes may have to be doubled: `|\|' $\to$ `|\\|'):
%
\begin{center}
|... -jobname "|\textit{target}|" |\\|"|[\textit{flags}]%
|\input{childdoc.def}\childdocforward[|\textit{main}|]{|\textit{dest}|}"|
\end{center}
%
Here \textit{target} is the name of the output file,
\textit{main} is the name of the main file
and \textit{dest} is the name of the main or child file to be processed
(all filenames without extensions).
The optional argument \textit{main} can be omitted
if \textit{main} matches \textit{dest}.
Optionally, compilation \textit{flags} can be defined via |\def| commands.
This command line makes the \TeX{} engine believe
it is compiling the file \textit{target}
whose content is specified as the latter parameter.
The provided code then forwards the processing to
\textit{main} or \textit{dest} as described in \secref{sec:forward}.

%%%%%%%%%%%%%%%%%%%%%%%%%%%%%%%%%%%%%%%%%%%%%%%%%%%%%%%%%%%%%%%%%%%%%%%%%%%%%%%%
\subsection{Include by Input}
\label{sec:input}

Including child documents by |\include| has some restrictions by design.
Most notably, the content of a child document always occupies
its own set of pages; pages cannot be shared between child documents.
Usually, this behaviour makes perfect sense
because each child document contain an essential part of the document.
However, in some situations it may be desirable to compose
a document from a collection of parts
without having mandatory page breaks between then.
For this case, the package
provides a mechanism to include parts
by |\input| which can also be processed individually.
However, by construction this mechanism
requires manual handling of the content to be output.

%%%%%%%%%%%%%%%%%%%%%%%%%%%%%%%%%%%%%%%%
\DescribeMacro{\ifchilddocmanual}
The main file should be prepared as usual, see \secref{sec:include}.
However, the document body must make a distinction
between processing of an individual part and of the main document, e.g.:
%
\begin{center}
\begin{tabular}{l}
|\ifchilddocmanual|\\
|\input{\childdocname}|\\
|\||else|\\
\textit{document body with }|\input{|\textit{part}|}|\\
|\||fi|
\end{tabular}
\end{center}
%
The conditional |\ifchilddocmanual| is true whenever
a part to be included by |\input| is being compiled,
and the name of the part is stored in |\childdocname|.

%%%%%%%%%%%%%%%%%%%%%%%%%%%%%%%%%%%%%%%%
\DescribeMacro{\childdocby}
Each part to be included by |\input| should start with:
%
\begin{center}
\begin{tabular}{l}
|\input{childdoc.def}|\\
|\childdocby{|\textit{main}|}|\\
\end{tabular}
\end{center}
%
The directive |\childdocby| is similar to |\childdocof|
described in \secref{sec:include},
but the subsequent selection of content must be done manually.
To that end, both |\ifchilddoc| and |\ifchilddocmanual|
will be true upon processing of a part,
and the name of the part is stored in |\childdocname|.
Note that |\jobname| will be set to the filename of the current part
so that each part receives an individual |.aux| file
that does not interfere with the |.aux| file(s) of the main document.
This behaviour can be altered by the alternative form
|\childdocby[*]{|\textit{main}|}| (with a non-empty optional argument)
which uses the |.aux| file of the main document
by setting |\jobname| to \textit{main}.

%%%%%%%%%%%%%%%%%%%%%%%%%%%%%%%%%%%%%%%%%%%%%%%%%%%%%%%%%%%%%%%%%%%%%%%%%%%%%%%%
\subsection{Driver Development}
\label{sec:driver}

The \textsf{childdoc} mechanism can also be use for the development
of definition files such as \LaTeX{} styles or classes.
This case differs from the above setup with multiple parts
included by |\include| in that no |\includeonly| should be invoked.
This can be achieved by starting the include file
(before |\ProvidesPackage|) with:
%
\begin{center}
\begin{tabular}{l}
|\input{childdoc.def}|\\
|\childdocforward{|\textit{main}|}|\\
\end{tabular}
\end{center}
%
or alternatively with:
%
\begin{center}
\begin{tabular}{l}
|\input{childdoc.def}|\\
|\childdocby{|\textit{main}|}|\\
\end{tabular}
\end{center}
%
Both forms have slightly different effects as described above.
The main file is prepared as usual, see \secref{sec:include}.

%%%%%%%%%%%%%%%%%%%%%%%%%%%%%%%%%%%%%%%%%%%%%%%%%%%%%%%%%%%%%%%%%%%%%%%%%%%%%%%%
\subsection{Legacy Detection}
\label{sec:detection}

The directive |\childdocmain| in the main file can detect
whether the complete document or merely a child is to be compiled
even without using the directive |\childdocof|.
This method is deprecated because it is less robust
and there is no compelling reason to use it;
it is merely provided for backward compatibility
and it may be removed in future versions.

If the detection mechanism is to be used,
it is mandatory to correctly specify
the filename of the main file as the argument of |\childdocmain|:
%
\begin{center}
\begin{tabular}{l}
|\input{childdoc.def}|\\
|\childdocmain{|\textit{main}|}|\\
\end{tabular}
\end{center}
%
If |\jobname| does not match the argument \textit{main} of |\childdocmain|,
it is assumed that |\jobname| points to the child file to be compiled.
When using |\childdocmain| with the main file specified as argument,
it suffices to start a child file
with just |\input{|\textit{main}|}|
without loading of the package and using |\childdocof|.
If instead all processing is done
with the appropriate \textsf{childdoc} directives,
the argument of \textit{main} of |\childdocmain| can be empty.

An alternative version of the command line processing described
in \secref{sec:commandline} using the detection mechanism reads:
%
\begin{center}
|... -jobname "|\textit{target}|" "|[\textit{flags}]%
[|\def\jobname{|\textit{dest}|}|]|\input{|\textit{main}|}"|
\end{center}

%%%%%%%%%%%%%%%%%%%%%%%%%%%%%%%%%%%%%%%%%%%%%%%%%%%%%%%%%%%%%%%%%%%%%%%%%%%%%%%%
\subsection{Manual Code}
\label{sec:manual}

In case one cannot be certain whether the definitions file |childdoc.def|
is installed on the target \TeX{} distribution
and one prefers not to ship it,
it is conceivable to paste a few relevant commands into the sources.

To that end, drop all statements |\input{childdoc.def}|
and perform the replacements as outlined below.
Instead of |\childdocmain{|\textit{main}|}| add the following code
to the top of the main file:
%
\begin{center}
\begin{tabular}{l}
|\||ifdefined\childdocname\endinput\||fi\newif\ifchilddoc|\\
|\edef\childdocname{\scantokens\expandafter{\jobname\noexpand}}|\\
|\def\childdocmain{|\textit{main}|}\||ifx\childdocmain\childdocname\||else|\\
|\childdoctrue\includeonly{\childdocname}\let\jobname\childdocmain\||fi|\\
\end{tabular}
\end{center}
%
Instead of |\childdocof{|\textit{main}|}| just include the main file
at the top of each child file:
%
\begin{center}
|\input{|\textit{main}|}|
\end{center}
%
A simple redirection |\childdocforward{|\textit{dest}|}| is achieved by:
%
\begin{center}
|\def\jobname{|\textit{dest}|}\input{\jobname}|
\end{center}
%
The redirection with prefix
|\childdocforwardprefix[|\textit{prefix}|]{|\textit{dest}|}|
is accomplished by:
%
\begin{center}
\begin{tabular}{l}
|{\edef\jobname{\scantokens\expandafter{\jobname\noexpand}}|\\
|\def\redirectjob |\textit{prefix}|#1~~~{\gdef\jobname{|\textit{dest}|#1}}|\\
|\expandafter\redirectjob\jobname~~~}\input{\jobname}|
\end{tabular}
\end{center}

In an alternative approach,
child documents can be compiled by a specific command line
without additional code or specific definitions:
%
\begin{center}
|... -jobname "|\textit{target}|" "|[\textit{flags}]%
|\includeonly{|\textit{dest}|}\input{|\textit{main}|}"|
\end{center}
%

%%%%%%%%%%%%%%%%%%%%%%%%%%%%%%%%%%%%%%%%%%%%%%%%%%%%%%%%%%%%%%%%%%%%%%%%%%%%%%%%
%%%%%%%%%%%%%%%%%%%%%%%%%%%%%%%%%%%%%%%%%%%%%%%%%%%%%%%%%%%%%%%%%%%%%%%%%%%%%%%%
\section{Information}

%%%%%%%%%%%%%%%%%%%%%%%%%%%%%%%%%%%%%%%%%%%%%%%%%%%%%%%%%%%%%%%%%%%%%%%%%%%%%%%%
\subsection{Copyright}

Copyright \copyright{} 2017--2018 Niklas Beisert

This work may be distributed and/or modified under the
conditions of the \LaTeX{} Project Public License, either version 1.3
of this license or (at your option) any later version.
The latest version of this license is in
  \url{http://www.latex-project.org/lppl.txt}
and version 1.3 or later is part of all distributions of \LaTeX{}
version 2005/12/01 or later.

This work has the LPPL maintenance status `maintained'.

The Current Maintainer of this work is Niklas Beisert.

This work consists of the files |README.txt|, |childdoc.ins| and |childdoc.dtx|
as well as the derived files |childdoc.def|, |cdocsamp.tex|
with |cdocsch1.tex|, |cdocsch2.tex|, |cdocspt3.tex|, |cdocspt4.tex|,
|cdocsdrf.tex|, |cdocsfn1.tex|, |cdocsfn2.tex|
as well as |childdoc.pdf|.

%%%%%%%%%%%%%%%%%%%%%%%%%%%%%%%%%%%%%%%%%%%%%%%%%%%%%%%%%%%%%%%%%%%%%%%%%%%%%%%%
\subsection{Files and Installation}

The package consists of the files:
%
\begin{center}
\begin{tabular}{ll}
    |README.txt|   & readme file \\
    |childdoc.ins| & installation file \\
    |childdoc.dtx| & source file \\
    |childdoc.def| & definition file \\
    |cdocsamp.tex| & sample main file \\
    |cdocsch1.tex| & sample include file \\
    |cdocsch2.tex| & sample include file \\
    |cdocspt3.tex| & sample part file \\
    |cdocspt4.tex| & sample part file \\
    |cdocsdrf.tex| & sample redirection file \\
    |cdocsfn1.tex| & sample redirection file \\
    |cdocsfn2.tex| & sample redirection file \\
    |childdoc.pdf| & manual
\end{tabular}
\end{center}
%
The distribution consists of the files
|README.txt|, |childdoc.ins| and |childdoc.dtx|.
%
\begin{itemize}
\item
Run (pdf)\LaTeX{} on |childdoc.dtx|
to compile the manual |childdoc.pdf| (this file).
\item
Run \LaTeX{} on |childdoc.ins| to create the definitions file |childdoc.def|
and the sample |cdocsamp.tex| with include files
|cdocsch1.tex|, |cdocsch2.tex|, |cdocspt3.tex|, |cdocspt4.tex|,
|cdocsdrf.tex|, |cdocsfn1.tex|, |cdocsfn2.tex|.
Then copy the file |childdoc.def| to an appropriate directory of your \LaTeX{}
distribution, e.g.\ \textit{texmf-root}|/tex/latex/childdoc|.
\end{itemize}

%%%%%%%%%%%%%%%%%%%%%%%%%%%%%%%%%%%%%%%%%%%%%%%%%%%%%%%%%%%%%%%%%%%%%%%%%%%%%%%%
\subsection{Related CTAN Packages}

There are several other packages which offer a similar functionality:
%
\begin{itemize}
\item
The packages
\href{http://ctan.org/pkg/docmute}{\textsf{docmute}},
\href{http://ctan.org/pkg/includex}{\textsf{includex}} and
\href{http://ctan.org/pkg/standalone}{\textsf{standalone}}
provide commands to include only the document body of
a child file thus allowing both files to be compiled individually.
\item
The packages \href{http://ctan.org/pkg/subdocs}{\textsf{subdocs}}
and \href{http://ctan.org/pkg/subfiles}{\textsf{subfiles}}
provide structures in which the main and child documents can be
encapsulated and allowing them to be compiled individually.
The inclusion mechanism is different from the conventional |\include|.
\item
The package \href{http://ctan.org/pkg/combine}{\textsf{combine}}
is an elaborate solution to combine several documents into one.
\end{itemize}
%
See also the CTAN topic \href{http://ctan.org/topic/subdocs}{\textsf{subdocs}}
for further related packages.
The present package differs from the above solutions in that
a document structure constructed with the conventional |\include| mechanism
just needs two extra commands at the top of every file
such that all constituent files can be compiled individually.

%%%%%%%%%%%%%%%%%%%%%%%%%%%%%%%%%%%%%%%%%%%%%%%%%%%%%%%%%%%%%%%%%%%%%%%%%%%%%%%%
%\subsection{Feature Suggestions}
%
%The following is a list of features which may be useful for future
%versions of this package:
%%
%\begin{itemize}
%\item
%\ldots
%\end{itemize}

%%%%%%%%%%%%%%%%%%%%%%%%%%%%%%%%%%%%%%%%%%%%%%%%%%%%%%%%%%%%%%%%%%%%%%%%%%%%%%%%
\subsection{Revision History}

%%%%%%%%%%%%%%%%%%%%%%%%%%%%%%%%%%%%%%%%
\paragraph{v2.0:} 2018/12/30

\begin{itemize}
\item
immediate forward processing
\item
added |\childdocby| mechanism
\item
manual restructured
\end{itemize}

%%%%%%%%%%%%%%%%%%%%%%%%%%%%%%%%%%%%%%%%
\paragraph{v1.6:} 2018/01/17

\begin{itemize}
\item
application for development of include files
\item
corrections to manual
\end{itemize}

%%%%%%%%%%%%%%%%%%%%%%%%%%%%%%%%%%%%%%%%
\paragraph{v1.5:} 2017/05/21

\begin{itemize}
\item
more complete structuring introduced
\item
|\childdocof| introduced
\item
|\childdoc| renamed to |\childdocmain|
\item
|\childredirect| renamed to |\childdocforward| and |\childdocforwardprefix|
and functionality expanded
\end{itemize}

%%%%%%%%%%%%%%%%%%%%%%%%%%%%%%%%%%%%%%%%
\paragraph{v1.0:} 2017/04/27

\begin{itemize}
\item
manual and install package
\item
first version published on CTAN
\end{itemize}

%%%%%%%%%%%%%%%%%%%%%%%%%%%%%%%%%%%%%%%%
\paragraph{v0.6:} 2017/04/26

\begin{itemize}
\item
redirection mechanism added
\end{itemize}

%%%%%%%%%%%%%%%%%%%%%%%%%%%%%%%%%%%%%%%%
\paragraph{v0.5:} 2017/04/26

\begin{itemize}
\item
functionality in definition file
\end{itemize}


%%%%%%%%%%%%%%%%%%%%%%%%%%%%%%%%%%%%%%%%%%%%%%%%%%%%%%%%%%%%%%%%%%%%%%%%%%%%%%%%
%%%%%%%%%%%%%%%%%%%%%%%%%%%%%%%%%%%%%%%%%%%%%%%%%%%%%%%%%%%%%%%%%%%%%%%%%%%%%%%%
%%%%%%%%%%%%%%%%%%%%%%%%%%%%%%%%%%%%%%%%%%%%%%%%%%%%%%%%%%%%%%%%%%%%%%%%%%%%%%%%
\appendix

\settowidth\MacroIndent{\rmfamily\scriptsize 000\ }

 \DocInput{childdoc.dtx}

\end{document}
%</driver>
% \fi
%
% %%%%%%%%%%%%%%%%%%%%%%%%%%%%%%%%%%%%%%%%%%%%%%%%%%%%%%%%%%%%%%%%%%%%%%%%%%%%%%
% %%%%%%%%%%%%%%%%%%%%%%%%%%%%%%%%%%%%%%%%%%%%%%%%%%%%%%%%%%%%%%%%%%%%%%%%%%%%%%
% \section{Sample}
%\iffalse
%<*samplemain>
%\fi
%
% The following presents a sample document
% with two chapters, two parts, a title page,
% a compile flag as well as three forwarding files to set the flag.
% It consists of eight |.tex| files:
% \begin{center}
% \begin{tabular}{ll}
% |cdocsamp.tex|&main file\\
% |cdocsch1.tex|&include file for chapter 1\\
% |cdocsch2.tex|&include file for chapter 2\\
% |cdocspt3.tex|&include file for part 3\\
% |cdocspt4.tex|&include file for part 4\\
% |cdocsdrf.tex|&forwarding file for main file in draft mode\\
% |cdocsfi1.tex|&forwarding file for final version of chapter 1\\
% |cdocsfi2.tex|&forwarding file for final version of chapter 2\\
% \end{tabular}
% \end{center}
% Each of the eight files can be compiled directly by the \LaTeX{} compiler.
%
% %%%%%%%%%%%%%%%%%%%%%%%%%%%%%%%%%%%%%%
% \paragraph{Main File.}
%
% The main file is called |cdocsamp.tex|.
%
% Load the \textsf{childdoc} definitions and
% declare the filename for the main document:
%    \begin{macrocode}
\input{childdoc.def}
\childdocmain{}
%    \end{macrocode}

% Optional override for |\version| flag:
%    \begin{macrocode}
%%\ifchilddoc\else\providecommand{\version}{draft}\fi
%    \end{macrocode}

% Define the default values for the |\version| flag
% (|final| for the main file and |draft| for childs):
%    \begin{macrocode}
\ifchilddoc
\providecommand{\version}{draft}
\else
\providecommand{\version}{final}
\fi
%    \end{macrocode}

% Load the standard document class:
%    \begin{macrocode}
\documentclass[12pt]{article}
%    \end{macrocode}

% Start the document body:
%    \begin{macrocode}
\begin{document}
%    \end{macrocode}

% Declare a title page.
% Print title, part of document being processed and version flag:
%    \begin{macrocode}
\addtocounter{page}{-1}
\begin{center}
{\LARGE\bfseries{}childdoc example\par}
\vspace{1cm}
\ifchilddoc
\ifchilddocmanual part\else chapter\fi:
`\childdocname' of `\childdocjob'\par
\else
main document: `\childdocjob'\par
\fi
version: \version\par
\end{center}
\newpage
%    \end{macrocode}

% Manually include selected file,
% otherwise process as usual:
%    \begin{macrocode}
\ifchilddocmanual
\section*{part `\childdocname'}
\input{\childdocname}
\else
%    \end{macrocode}

% Include the two chapters:
%    \begin{macrocode}
\include{cdocsch1}
\include{cdocsch2}
%    \end{macrocode}

% Include the two parts unless only chapters should be displayed:
%    \begin{macrocode}
\ifchilddoc\else
\section{part three}
\input{cdocspt3}
\section{part four}
\input{cdocspt4}
\fi
%    \end{macrocode}

% Process as usual until here:
%    \begin{macrocode}
\fi
%    \end{macrocode}

% End of document body:
%    \begin{macrocode}
\end{document}
%    \end{macrocode}
%\iffalse
%</samplemain>
%\fi
%
% %%%%%%%%%%%%%%%%%%%%%%%%%%%%%%%%%%%%%%
% \paragraph{Chapter Include Files.}
%
% The include files are called |cdocsch1.tex| and |cdocsch2.tex|.
%
%\iffalse
%<*samplechap1|samplechap2>
%\fi

% Optional override for |\version| flag:
%    \begin{macrocode}
%%\providecommand{\version}{final}
%    \end{macrocode}

% Include the main document:
%    \begin{macrocode}
\input{childdoc.def}
\childdocof{cdocsamp}
%    \end{macrocode}

%\iffalse
%</samplechap1|samplechap2>
%\fi
%
%\iffalse
%<*samplechap1>
%\fi
% Some text for chapter 1:
%    \begin{macrocode}
\section{one}
some text in chapter one
%    \end{macrocode}

%\iffalse
%</samplechap1>
%\fi
% Some text for chapter 2:
%\iffalse
%<*samplechap2>
%\fi
%    \begin{macrocode}
\section{two}
more text in chapter two
%    \end{macrocode}

%\iffalse
%</samplechap2>
%\fi
%
% %%%%%%%%%%%%%%%%%%%%%%%%%%%%%%%%%%%%%%
% \paragraph{Part Include Files.}
%
% The include files are called |cdocspt3.tex| and |cdocspt4.tex|.
%
%\iffalse
%<*samplepart3|samplepart4>
%\fi

% Optional override for |\version| flag:
%    \begin{macrocode}
%%\providecommand{\version}{final}
%    \end{macrocode}

% Include the main document:
%    \begin{macrocode}
\input{childdoc.def}
\childdocby{cdocsamp}
%    \end{macrocode}

%\iffalse
%</samplepart3|samplepart4>
%\fi
%
%\iffalse
%<*samplepart3>
%\fi
% Some text for part 3:
%    \begin{macrocode}
some text in part three
%    \end{macrocode}

%\iffalse
%</samplepart3>
%\fi
% Some text for part 4:
%\iffalse
%<*samplepart4>
%\fi
%    \begin{macrocode}
more text in part four
%    \end{macrocode}

%\iffalse
%</samplepart4>
%\fi
%
% %%%%%%%%%%%%%%%%%%%%%%%%%%%%%%%%%%%%%%
% \paragraph{Forwarding for a Complete Draft.}
%
% The following forwarding file |cdocsdrf.tex|
% compiles the main document in draft mode:
%\iffalse
%<*sampledraft>
%\fi
%    \begin{macrocode}
\def\version{draft}
\input{childdoc.def}
\childdocforward{cdocsamp}
%    \end{macrocode}

%\iffalse
%</sampledraft>
%\fi
%
% %%%%%%%%%%%%%%%%%%%%%%%%%%%%%%%%%%%%%%
% \paragraph{Forwarding for Final Version of the Chapters.}
%
% The following forwarding files |cdocsfn1.tex| and |cdocsfn2.tex|
% (with identical content)
% compile the final versions of the child documents
% |cdocsch1.tex| and |cdocsch2.tex|, respectively:
%\iffalse
%<*samplefinal>
%\fi
%    \begin{macrocode}
\def\version{final}
\input{childdoc.def}
\childdocforwardprefix[cdocsamp]{cdocsfn}{cdocsch}
%    \end{macrocode}

%\iffalse
%</samplefinal>
%\fi
%
% %%%%%%%%%%%%%%%%%%%%%%%%%%%%%%%%%%%%%%
% \paragraph{Command Line Processing.}
%
% The following three command lines generate the output files
% |cdocscld|, |cdocscl1| and |cdocscl2|
% which should be identical to
% |cdocsdrf|, |cdocsch1| and |cdocsfn2|, respectively:
% \begin{center}
% \begin{tabular}{l}
% |latex -jobname cdocscld \|\\
% |  "\def\version{draft}\input{childdoc.def}\childdocforward{cdocsamp}"|\\
% |latex -jobname cdocscl1 \|\\
% |  "\input{childdoc.def}\childdocforward[cdocsamp]{cdocsch1}"|\\
% |latex -jobname cdocscl2 \|\\
% |  "\def\version{final}\input{childdoc.def}\childdocforward{cdocsch2}"|
% \end{tabular}
% \end{center}
% Note that the trailing backslash on each first line
% merely continues the input to the second line
% (for convenient cut ant paste).
% Furthermore, the command |latex| can be replaced by any
% of its alternative versions such as |pdflatex|.
%
% %%%%%%%%%%%%%%%%%%%%%%%%%%%%%%%%%%%%%%%%%%%%%%%%%%%%%%%%%%%%%%%%%%%%%%%%%%%%%%
% %%%%%%%%%%%%%%%%%%%%%%%%%%%%%%%%%%%%%%%%%%%%%%%%%%%%%%%%%%%%%%%%%%%%%%%%%%%%%%
% \section{Implementation}
%\iffalse
%<*package>
%\fi
%
% This section describes the definitions file |childdoc.def|.

% The definitions cannot be loaded using |\usepackage| or |\RequirePackage|
% which has a mechanism to prevent loading a style file more than once.
% When loading the definitions by means of |\input|
% multiple instances have to be prevented manually:
%\iffalse
%This code needs to be before the `\ProvidesFile' directive
%which is defined at the beginning of this file.
%Therefore it is also placed there and commented out here.
%</package>
%<*discard>
%\fi
%    \begin{macrocode}
\ifdefined\childdocmain\endinput\fi
%    \end{macrocode}
%\iffalse
%</discard>
%<*package>
%\fi
%
% \macro{\ifchilddoc}
% \macro{\ifchilddocmanual}
% The conditional |\ifchilddoc| tells whether a
% child (true) or main (false) document is being compiled.
% The conditional |\ifchilddocmanual| tells whether
% the |\includeonly| mechanism is used (false) or
% the selection of child files must be performed manually (true).
% The definitions initialise to false:
%    \begin{macrocode}
\newif\ifchilddoc
\newif\ifchilddocmanual
%    \end{macrocode}

% \macro{\childdocname}
% \macro{\childdocjob}
% The macro |\childdocname| stores the name of the main document
% to be compiled. The macro |\childdocjob| stores the name of
% the document on which the \LaTeX{} compiler was originally invoked.
% The content of |\jobname| cannot be compared
% to filenames specified in the source due to different catcodes.
% The following code rescans |\jobname|, stores the result
% in |\childdocname| and saves a copy in |\childdocjob|:
%    \begin{macrocode}
\edef\childdocname{\scantokens\expandafter{\jobname\noexpand}}
\let\childdocjob\childdocname
%    \end{macrocode}

% \macro{\childdocdisable}
% The macro |\childdocdisable| prevents the main file
% from being processed more than once.
% At this stage, the main document command |\childdocmain|
% is assumed to be called once again where it should do nothing.
% Any subsequent call to it should prevent
% a secondary processing of the main document
% It overwrites the forwarding commands
% |\childdocof| and |\childdocforward|
% with empty macros to prevent further inclusions of the main document:
%    \begin{macrocode}
\newcommand{\childdocdisable}
{
  \renewcommand{\childdocmain}[1]{\renewcommand{\childdocmain}[1]{\endinput}}
  \renewcommand{\childdocof}[1]{}
  \renewcommand{\childdocby}[2][]{}
  \renewcommand{\childdocforward}[2][]{}
  \renewcommand{\childdocdisable}{}
}
%    \end{macrocode}

% \macro{\childdocmain}
% The macro |\childdocmain| is to be called at the top of the main file
% with nothing or the main filename (without extension) as argument.
% First, it breaks loops.
% If the argument is not empty and does not match |\childdocname|
% (which is set by the first inclusion of |childdoc.def|),
% |\ifchilddoc| is set to true, |\includeonly| is applied to the child file
% and |\jobname| is set to the main file
% (for proper handling of |.aux| files):
%    \begin{macrocode}
\newcommand{\childdocmain}[1]
{
  \childdocdisable\childdocmain{}
  \if?#1?\else
    \begingroup
      \def\childdoctmp{#1}
      \ifx\childdoctmp\childdocname
        \def\childdoctmp{}
      \else
        \def\childdoctmp
        {
          \childdoctrue
          \includeonly{\childdocname}
          \def\childdocjob{#1}
          \def\jobname{#1}
        }
      \fi
      \expandafter
    \endgroup
    \childdoctmp
  \fi
}
%    \end{macrocode}

% \macro{\childdocof}
% The command |\childdocof| redirects
% compilation to the main file |#1|.
%    \begin{macrocode}
\newcommand{\childdocof}[1]
{
  \childdocdisable
  \childdoctrue
  \includeonly{\childdocname}
  \def\jobname{#1}
  \def\childdocjob{#1}
  \input{#1}
}
%    \end{macrocode}

% \macro{\childdocby}
% The command |\childdocby| ....
%    \begin{macrocode}
\newcommand{\childdocby}[2][]
{
  \childdocdisable
  \childdoctrue
  \childdocmanualtrue
  \if?#1?\else
    \def\jobname{#2}
  \fi
  \def\childdocjob{#2}
  \input{#2}
  \endinput
}
%    \end{macrocode}

% \macro{\childdocforward}
% The command |\childdocforward| redirects
% compilation to the main file or
% (if the optional argument is given) a child file.
% Parameters are set as if the main file
% or a child file starting with |\childdocof| was compiled.
% Then compilation is handed over to the main file:
%    \begin{macrocode}
\newcommand{\childdocforward}[2][]
{
  \begingroup
    \if?#1?
      \def\childdoctmp
      {
        \def\childdocname{#2}
        \def\childdocjob{#2}
        \def\jobname{#2}
        \input{#2}
        \endinput
      }
    \else
      \def\childdoctmp
      {
        \childdocdisable
        \def\childdocname{#2}
        \childdoctrue
        \includeonly{#2}
        \def\childdocjob{#1}
        \def\jobname{#1}
        \input{#1}
        \endinput
      }
    \fi
    \expandafter
  \endgroup
  \childdoctmp
}
%    \end{macrocode}

% \macro{\childdocforwardprefix}
% The command |\childdocforwardprefix| redirects
% compilation to the main or a child file by means of a pattern.
% The prefix |#1| in the current filename is replaced by |#2|
% and the suffix of the current filename is kept
% (it is assumed that the filename does not contain the substring `|~~~|'
% which is used as a delimiter).
% Compilation is handed over to the new file by |\childdocforward|:
%    \begin{macrocode}
\newcommand{\childdocforwardprefix}[3][]
{
  \begingroup
    \def\childdocextract #2##1~~~{\def\childdoctmp{\childdocforward[#1]{#3##1}}}
    \expandafter\childdocextract\childdocname~~~
    \expandafter
  \endgroup
  \childdoctmp
}
%    \end{macrocode}

% \macro{\childdoc}
% The deprecated macro |\childdoc| is a legacy version of |\childdocmain|:
%    \begin{macrocode}
\newcommand{\childdoc}{\childdocmain}
%    \end{macrocode}

% \macro{\childdocredirect}
% The deprecated macro |\childdocredirect| is a legacy version
% of |\childdocforward| and |\childdocforwardprefix|:
%    \begin{macrocode}
\newcommand{\childdocredirect}[2][]
{
  \begingroup
    \if?#1?
      \def\childdoctmp{\childdocforward{#2}}
    \else
      \def\childdoctmp{\childdocforwardprefix{#1}{#2}}
    \fi
    \expandafter
  \endgroup
  \childdoctmp
}
%    \end{macrocode}

%\iffalse
%</package>
%\fi
%
\endinput
|\\
|\childdocmain{}|\\
\end{tabular}
\end{center}
at the very top of the main \LaTeX{} file,
in particular \emph{before} the |\documentclass| statement!
The argument of |\childdocmain| should be left empty
(but it must be present).

%%%%%%%%%%%%%%%%%%%%%%%%%%%%%%%%%%%%%%%%
\DescribeMacro{\childdocof}
Furthermore, add the commands
\begin{center}
\begin{tabular}{l}
|% \iffalse
%
% childdoc.dtx Copyright (C) 2017-2018 Niklas Beisert
%
% This work may be distributed and/or modified under the
% conditions of the LaTeX Project Public License, either version 1.3
% of this license or (at your option) any later version.
% The latest version of this license is in
%   http://www.latex-project.org/lppl.txt
% and version 1.3 or later is part of all distributions of LaTeX
% version 2005/12/01 or later.
%
% This work has the LPPL maintenance status `maintained'.
%
% The Current Maintainer of this work is Niklas Beisert.
%
% This work consists of the files childdoc.dtx and childdoc.ins
% and the derived files childdoc.def and cdocsamp.tex with
% cdocsch1.tex, cdocsch2.tex, cdocsdrf.tex, cdocsfn1.tex, cdocsfn2.tex.
%
%<package>\ifdefined\childdocmain\endinput\fi
%<package>\ProvidesFile{childdoc.def}[2018/12/30 v2.0 child document driver]
%<samplemain>\ProvidesFile{cdocsamp.tex}[2018/12/30 v2.0 sample for childdoc]
%<*driver>
%\ProvidesFile{childdoc.drv}[2018/12/30 v2.0 childdoc reference manual file]
\PassOptionsToClass{10pt,a4paper}{article}
\documentclass{ltxdoc}

\usepackage[margin=35mm]{geometry}
\usepackage{hyperref}
\usepackage{hyperxmp}
\usepackage[usenames]{color}

\hypersetup{colorlinks=true}
\hypersetup{pdfstartview=FitH}
\hypersetup{pdfpagemode=UseNone}
\hypersetup{pdfsource={}}
\hypersetup{pdflang={en-UK}}
\hypersetup{pdfcopyright={Copyright 2017-2018 Niklas Beisert.
  This work may be distributed and/or modified under the
  conditions of the LaTeX Project Public License, either version 1.3
  of this license or (at your option) any later version.}}
\hypersetup{pdflicenseurl={http://www.latex-project.org/lppl.txt}}
\hypersetup{pdfcontactaddress={ETH Zurich, ITP, HIT K,
  Wolfgang-Pauli-Strasse 27}}
\hypersetup{pdfcontactpostcode={8093}}
\hypersetup{pdfcontactcity={Zurich}}
\hypersetup{pdfcontactcountry={Switzerland}}
\hypersetup{pdfcontactemail={nbeisert@itp.phys.ethz.ch}}
\hypersetup{pdfcontacturl={http://people.phys.ethz.ch/\xmptilde nbeisert/}}

\newcommand{\secref}[1]{\hyperref[#1]{section \ref*{#1}}}

\parskip1ex
\parindent0pt
\let\olditemize\itemize
\def\itemize{\olditemize\parskip0pt}

\begin{document}

\title{The \textsf{childdoc} Package}
\hypersetup{pdftitle={The childdoc Package}}
\author{Niklas Beisert\\[2ex]
  Institut f\"ur Theoretische Physik\\
  Eidgen\"ossische Technische Hochschule Z\"urich\\
  Wolfgang-Pauli-Strasse 27, 8093 Z\"urich, Switzerland\\[1ex]
  \href{mailto:nbeisert@itp.phys.ethz.ch}
  {\texttt{nbeisert@itp.phys.ethz.ch}}}
\hypersetup{pdfauthor={Niklas Beisert}}
\hypersetup{pdfsubject={Manual for the LaTeX2e Package childdoc}}
\date{30 December 2018, \textsf{v2.0}}
\maketitle

\begin{abstract}\noindent
\textsf{childdoc} is a \LaTeXe{} package
that enables the direct compilation
of document sections included by |\include|
to individual files.
\end{abstract}

\begingroup
\parskip0ex
\tableofcontents
\endgroup

%%%%%%%%%%%%%%%%%%%%%%%%%%%%%%%%%%%%%%%%%%%%%%%%%%%%%%%%%%%%%%%%%%%%%%%%%%%%%%%%
%%%%%%%%%%%%%%%%%%%%%%%%%%%%%%%%%%%%%%%%%%%%%%%%%%%%%%%%%%%%%%%%%%%%%%%%%%%%%%%%
\section{Introduction}

\LaTeX{} provides a mechanism to structure a large document (such as a book)
into a main file and several child files (containing the chapters)
using the |\include| command.
This mechanism is beneficial for documents
which span hundreds of pages in order to
make the source file(s) more manageable.
Moreover, compilation can be restricted to
selected child files by means of the |\includeonly| command.
The latter feature can be used to reduce the compilation time while editing
(this was significantly more useful in the earlier days of \LaTeX{})
or to generate a smaller document which is easier to navigate.
Another application of |\includeonly| is to generate
documents consisting of selected parts of the complete document.

However, there are a few drawbacks of the plain |\include| mechanism:
\begin{itemize}
\item
The child files cannot be compiled on their own,
they can only be compiled via the main file.
A naive editing environment
(such as a text editor with an option
to have the current file processed by \LaTeX)
may require one to switch to the main file before compiling;
attempting to compile the child file produces errors.
\item
The main file must be modified (each time)
to adjust the |\includeonly| command
to the present needs. This easily leaves the main file in a messy state.
\item
The generated document will always carry the filename
of the main document. This is inconvenient if
several child files are to be compiled and
to be kept for distribution.
\end{itemize}

The present package provides a simple interface
to make child files individually compilable by \LaTeX{}.
Compiling a child file then has the same effect as compiling
the main file with an |\includeonly| command
to select the appropriate child.
Moreover the generated document will carry the name of the child
rather than the main file.
This resolves all three above issues.

This feature is meant to make the editing of books,
thesis documents and lecture notes somewhat more convenient.
However, the package can also be used efficiently for
composing a series of documents (such as exercise sheets)
which are typically distributed individually.
It then assists the author in generating the individual documents
(potentially in different versions)
as well as a document containing the collected series.
Another application is in developing style files
or other kinds of included material
where compilation of the style file could redirect
to a sample or test file.

%%%%%%%%%%%%%%%%%%%%%%%%%%%%%%%%%%%%%%%%%%%%%%%%%%%%%%%%%%%%%%%%%%%%%%%%%%%%%%%%
%%%%%%%%%%%%%%%%%%%%%%%%%%%%%%%%%%%%%%%%%%%%%%%%%%%%%%%%%%%%%%%%%%%%%%%%%%%%%%%%
\section{Usage}

First of all, the package \textsf{childdoc} is \emph{not} a standard
\LaTeXe{} |.sty| style file! Therefore it needs to be invoked in
a non-standard way.

%%%%%%%%%%%%%%%%%%%%%%%%%%%%%%%%%%%%%%%%%%%%%%%%%%%%%%%%%%%%%%%%%%%%%%%%%%%%%%%%
\subsection{Included Files}
\label{sec:include}

%%%%%%%%%%%%%%%%%%%%%%%%%%%%%%%%%%%%%%%%
\DescribeMacro{\childdocmain}
To use the package, add the commands
\begin{center}
\begin{tabular}{l}
|\input{childdoc.def}|\\
|\childdocmain{}|\\
\end{tabular}
\end{center}
at the very top of the main \LaTeX{} file,
in particular \emph{before} the |\documentclass| statement!
The argument of |\childdocmain| should be left empty
(but it must be present).

%%%%%%%%%%%%%%%%%%%%%%%%%%%%%%%%%%%%%%%%
\DescribeMacro{\childdocof}
Furthermore, add the commands
\begin{center}
\begin{tabular}{l}
|\input{childdoc.def}|\\
|\childdocof{|\textit{main}|}|\\
\end{tabular}
\end{center}
at the top of every child file \textit{child}
which is included by |\include{|\textit{child}|}|
from within the main file
(or at least for those files to be compiled individually).
The argument \textit{main} must be the filename of the main file.

There are a couple of
considerations in setting up the main and child documents:

%%%%%%%%%%%%%%%%%%%%%%%%%%%%%%%%%%%%%%%%
\paragraph{Restrictions.}

Please note the following restrictions:
\begin{itemize}
\item
|\childdocmain| must be called with one argument \textit{main}
to ensure compatibility with earlier version of the package.
It must either be empty (|\childdocmain{}|)
or precisely match the filename of the main file in which it is specified.
See \secref{sec:detection} for further information.
\item
The filename \textit{main} must be specified without the |.tex| extension.
\item
The filename \textit{main} is case sensitive
(even in case-insensitive file systems)
due to internal string comparison.
\item
The argument \textit{main} should be fully expanded, it cannot be a macro.
\item
Subdirectories and special characters should be avoided in filenames.
\item
The command |\childdocmain{|\textit{main}|}| must be followed by a whitespace.
It should not be followed immediately by another command
or by a comment mark `|%|'.
This is because the \TeX{} parser reads the token immediately following
the argument of |\childdocmain| and puts it
at the beginning of every child section;
however, a white\-space is ignored.
\end{itemize}

%%%%%%%%%%%%%%%%%%%%%%%%%%%%%%%%%%%%%%%%
\paragraph{Content of Main File.}

It is advisable to place all content in the child files included by |\include|.
Any output contained in the main file will appear in all child documents
unless suppressed manually;
it cannot be suppressed automatically by the |\includeonly| directive
and thus should normally be avoided.
A method to include some content in the main file
by means of conditional processing is described in \secref{sec:conditional}.

%%%%%%%%%%%%%%%%%%%%%%%%%%%%%%%%%%%%%%%%
\paragraph{Page Numbering.}

When only a part of the document is compiled,
the appropriate numbering of pages
(as well as other status parameters)
is determined from the |.aux| files.
The latter contain information from previous passes.
However this information needs to propagate through
all intermediate child documents.
Therefore the page numbering in child documents may well
be inconsistent until the complete document is compiled at least once.

A useful (if unconventional) way to always ensure a consistent
page numbering is to restart the numbering in each child document
and denote the pages by `\textit{child}|.|\textit{page}'
where \textit{child} represents the chapter/section number of the child file.
This can be achieved by the command
|\numberwithin{page}{|\textit{child}|}|
of the \textsf{amsmath} package
where \textit{child} can be |chapter| or |section|
depending on the chosen structuring.
Alternatively, one can modify the macro |\thepage| appropriately
and reset the counter |page| at the start of each child file.

%%%%%%%%%%%%%%%%%%%%%%%%%%%%%%%%%%%%%%%%%%%%%%%%%%%%%%%%%%%%%%%%%%%%%%%%%%%%%%%%
\subsection{Conditional Processing}
\label{sec:conditional}

The package provides a mechanism to compile different versions
of a document. To customise the versions further some conditional processing
can come in handy to distinguish which version is being compiled.
The package provides two macros to describe the compilation context:

%%%%%%%%%%%%%%%%%%%%%%%%%%%%%%%%%%%%%%%%
\DescribeMacro{\ifchilddoc}
The conditional |\ifchilddoc| distinguishes between the compilation of
child documents and the main document:
%
\begin{center}
|\ifchilddoc |\textit{child-code}| |[|\||else |\textit{main-code}]| \||fi|
\end{center}

%%%%%%%%%%%%%%%%%%%%%%%%%%%%%%%%%%%%%%%%
\DescribeMacro{\childdocname}
\DescribeMacro{\childdocjob}
The macro |\childdocname| contains the filename (without extension)
of the main or child file being processed.
Note that |\childdocjob| will always contain the name of the main file.

%%%%%%%%%%%%%%%%%%%%%%%%%%%%%%%%%%%%%%%%
\paragraph{Title Page.}

Conditional processing can be used to include a title or banner page
in the main document when proper precautions are taken.
Importantly, the code in the main file should ensure that the page counter
(as well as other status parameters which are stored in the |.aux| files)
takes the same value after the conditional processing.
Otherwise the page numbers may take divergent values
depending on which part is compiled.

For example, a title page could be declared by:
%
\begin{center}
\begin{tabular}{l}
|\ifchilddoc\||else|\\
|\addtocounter{page}{-1}|\\
\textit{code for title page}\\
|\newpage|\\
|\||fi|
\end{tabular}
\end{center}
%
A banner page for the child documents can be generated by:
%
\begin{center}
\begin{tabular}{l}
|\ifchilddoc|\\
|\addtocounter{page}{-1}|\\
\textit{code for banner page}\\
|\newpage|\\
|\||fi|
\end{tabular}
\end{center}
%
Here one could write a message such as:
\begin{center}
|This is the part \childdocname{} of \childdocjob{}.|
\end{center}

%%%%%%%%%%%%%%%%%%%%%%%%%%%%%%%%%%%%%%%%%%%%%%%%%%%%%%%%%%%%%%%%%%%%%%%%%%%%%%%%
\subsection{Flags}
\label{sec:flags}

The package makes it easy to generate different versions
of the main or child documents.
To this end compilation flags can be defined
and assigned different default values.
They will be particularly useful in conjunction
with the forwarding mechanism described in \secref{sec:forward}.

For example, it may be useful to have a flag |\version|
which can be set to |draft| or |final|.
The document source will contain some conditional code
depending on the value of |\version|.
Suppose further, the flag should default to |final| for the main file
and to |draft| for child files
which is a natural assignment for editing the document.
This is achieved by placing the following code
in the preamble of the main document
(below the |\childdocmain| directive):
%
\begin{center}
\begin{tabular}{l}
|\ifchilddoc|\\
|\providecommand{\version}{draft}|\\
|\||else|\\
|\providecommand{\version}{final}|\\
|\||fi|
\end{tabular}
\end{center}
%
The definition by |\providecommand| makes sure
that previous definitions are not overwritten.
Further statements |\providecommand{\version}{...}|
can thus be added before the above code to override it.

For the main file, one might add a line
(between |\childdocmain| and the above block)
%
\begin{center}
|%\ifchilddoc\||else\providecommand{\version}{draft}\||fi|
\end{center}
%
which can be uncommented to produce a draft version.
Likewise one can add a line to the very top of a child file
(above the |\childdocof{|\textit{main}|}| directive)
%
\begin{center}
|%\providecommand{\version}{final}|
\end{center}
%
which can be uncommented to produce the final version of this child document.

%%%%%%%%%%%%%%%%%%%%%%%%%%%%%%%%%%%%%%%%%%%%%%%%%%%%%%%%%%%%%%%%%%%%%%%%%%%%%%%%
\subsection{Forwarding}
\label{sec:forward}

Different versions of the main or child documents
using compilation flags as described in \secref{sec:flags}
can be (permanently) stored in different files
for convenient compilation, viewing and distribution.
To this end, the package defines a command
to pass on compilation to a different file:

%%%%%%%%%%%%%%%%%%%%%%%%%%%%%%%%%%%%%%%%
\DescribeMacro{\childdocforward}
The command |\childdocforward| redirects processing to
another source file:
%
\begin{center}
\begin{tabular}{l}
|\input{childdoc.def}|\\
|\childdocforward[|\textit{main}|]{|\textit{dest}|}|\\
\end{tabular}
\end{center}
%
The argument \textit{dest} is the destination file
(without extension).
It should be the main file or one of the child files.
Note that further \textsf{childdoc} directives
such as |\childdocof| and |\childdocforward|
in the indicated file will be processed in this form.
The optional argument \textit{main}
passes on directly to the main file \textit{main}
while pretending to compile the child \textit{dest}.
This form behaves as if \textit{dest}
issues |\childdocof{|\textit{main}|}| right away,
and no further \textsf{childdoc} directives will be processed.

%%%%%%%%%%%%%%%%%%%%%%%%%%%%%%%%%%%%%%%%
\DescribeMacro{\...prefix}
In the alternative form |\childdocforwardprefix|,
%
\begin{center}
\begin{tabular}{l}
|\input{childdoc.def}|\\
|\childdocforwardprefix[|\textit{main}|]{|\textit{prefix}|}{|\textit{dest}|}|
\end{tabular}
\end{center}
%
the destination file is determined by a pattern
depending on the current file:
To make this work, the current file must be called
`{\textit{prefix}\hspace{0.2em}\textit{suffix}}'
with \textit{prefix} matching precisely the argument.
Processing is then passed on to the file
`{\textit{dest}\hspace{0.2em}\textit{suffix}}'.
Surely, the same effect is achieved by
directly specifying the
argument `{\textit{dest}\hspace{0.2em}\textit{suffix}}'
in the first form.
However, that requires to set up a different file
for each child. With the alternative form of the command
all these files can have exactly the same content
which simplifies setting them up and maintaining them.

For example, the following file |draft.tex|
with a compilation flag |\version| as described in \secref{sec:flags}
compiles the main document as a draft:
%
\begin{center}
\begin{tabular}{l}
|\def\version{draft}|\\
|\input{childdoc.def}|\\
|\childdocforward{|\textit{main}|}|
\end{tabular}
\end{center}
%
Likewise, the following files |final|\textit{nn}|.tex|
compile the final version of the child document
|child|\textit{nn}|.tex|:
%
\begin{center}
\begin{tabular}{l}
|\def\version{final}|\\
|\input{childdoc.def}|\\
|\childdocforwardprefix{final}{child}|
\end{tabular}
\end{center}
%

Note that when several versions of a main file and/or of each child file
are to be generated, it may be convenient to set up a |Makefile| or
shell script to automatise the process.

%%%%%%%%%%%%%%%%%%%%%%%%%%%%%%%%%%%%%%%%%%%%%%%%%%%%%%%%%%%%%%%%%%%%%%%%%%%%%%%%
\subsection{Command Line Processing}
\label{sec:commandline}

The effect of redirection files can also be achieved by invoking
the \LaTeX{} compiler with a more elaborate command line.
Most conveniently this should be done as part
of a shell script or a |Makefile|.

When using \textsf{childdoc} in the main file, the following
command lines effectively perform a redirection
(note that depending on the shell being used,
backslashes may have to be doubled: `|\|' $\to$ `|\\|'):
%
\begin{center}
|... -jobname "|\textit{target}|" |\\|"|[\textit{flags}]%
|\input{childdoc.def}\childdocforward[|\textit{main}|]{|\textit{dest}|}"|
\end{center}
%
Here \textit{target} is the name of the output file,
\textit{main} is the name of the main file
and \textit{dest} is the name of the main or child file to be processed
(all filenames without extensions).
The optional argument \textit{main} can be omitted
if \textit{main} matches \textit{dest}.
Optionally, compilation \textit{flags} can be defined via |\def| commands.
This command line makes the \TeX{} engine believe
it is compiling the file \textit{target}
whose content is specified as the latter parameter.
The provided code then forwards the processing to
\textit{main} or \textit{dest} as described in \secref{sec:forward}.

%%%%%%%%%%%%%%%%%%%%%%%%%%%%%%%%%%%%%%%%%%%%%%%%%%%%%%%%%%%%%%%%%%%%%%%%%%%%%%%%
\subsection{Include by Input}
\label{sec:input}

Including child documents by |\include| has some restrictions by design.
Most notably, the content of a child document always occupies
its own set of pages; pages cannot be shared between child documents.
Usually, this behaviour makes perfect sense
because each child document contain an essential part of the document.
However, in some situations it may be desirable to compose
a document from a collection of parts
without having mandatory page breaks between then.
For this case, the package
provides a mechanism to include parts
by |\input| which can also be processed individually.
However, by construction this mechanism
requires manual handling of the content to be output.

%%%%%%%%%%%%%%%%%%%%%%%%%%%%%%%%%%%%%%%%
\DescribeMacro{\ifchilddocmanual}
The main file should be prepared as usual, see \secref{sec:include}.
However, the document body must make a distinction
between processing of an individual part and of the main document, e.g.:
%
\begin{center}
\begin{tabular}{l}
|\ifchilddocmanual|\\
|\input{\childdocname}|\\
|\||else|\\
\textit{document body with }|\input{|\textit{part}|}|\\
|\||fi|
\end{tabular}
\end{center}
%
The conditional |\ifchilddocmanual| is true whenever
a part to be included by |\input| is being compiled,
and the name of the part is stored in |\childdocname|.

%%%%%%%%%%%%%%%%%%%%%%%%%%%%%%%%%%%%%%%%
\DescribeMacro{\childdocby}
Each part to be included by |\input| should start with:
%
\begin{center}
\begin{tabular}{l}
|\input{childdoc.def}|\\
|\childdocby{|\textit{main}|}|\\
\end{tabular}
\end{center}
%
The directive |\childdocby| is similar to |\childdocof|
described in \secref{sec:include},
but the subsequent selection of content must be done manually.
To that end, both |\ifchilddoc| and |\ifchilddocmanual|
will be true upon processing of a part,
and the name of the part is stored in |\childdocname|.
Note that |\jobname| will be set to the filename of the current part
so that each part receives an individual |.aux| file
that does not interfere with the |.aux| file(s) of the main document.
This behaviour can be altered by the alternative form
|\childdocby[*]{|\textit{main}|}| (with a non-empty optional argument)
which uses the |.aux| file of the main document
by setting |\jobname| to \textit{main}.

%%%%%%%%%%%%%%%%%%%%%%%%%%%%%%%%%%%%%%%%%%%%%%%%%%%%%%%%%%%%%%%%%%%%%%%%%%%%%%%%
\subsection{Driver Development}
\label{sec:driver}

The \textsf{childdoc} mechanism can also be use for the development
of definition files such as \LaTeX{} styles or classes.
This case differs from the above setup with multiple parts
included by |\include| in that no |\includeonly| should be invoked.
This can be achieved by starting the include file
(before |\ProvidesPackage|) with:
%
\begin{center}
\begin{tabular}{l}
|\input{childdoc.def}|\\
|\childdocforward{|\textit{main}|}|\\
\end{tabular}
\end{center}
%
or alternatively with:
%
\begin{center}
\begin{tabular}{l}
|\input{childdoc.def}|\\
|\childdocby{|\textit{main}|}|\\
\end{tabular}
\end{center}
%
Both forms have slightly different effects as described above.
The main file is prepared as usual, see \secref{sec:include}.

%%%%%%%%%%%%%%%%%%%%%%%%%%%%%%%%%%%%%%%%%%%%%%%%%%%%%%%%%%%%%%%%%%%%%%%%%%%%%%%%
\subsection{Legacy Detection}
\label{sec:detection}

The directive |\childdocmain| in the main file can detect
whether the complete document or merely a child is to be compiled
even without using the directive |\childdocof|.
This method is deprecated because it is less robust
and there is no compelling reason to use it;
it is merely provided for backward compatibility
and it may be removed in future versions.

If the detection mechanism is to be used,
it is mandatory to correctly specify
the filename of the main file as the argument of |\childdocmain|:
%
\begin{center}
\begin{tabular}{l}
|\input{childdoc.def}|\\
|\childdocmain{|\textit{main}|}|\\
\end{tabular}
\end{center}
%
If |\jobname| does not match the argument \textit{main} of |\childdocmain|,
it is assumed that |\jobname| points to the child file to be compiled.
When using |\childdocmain| with the main file specified as argument,
it suffices to start a child file
with just |\input{|\textit{main}|}|
without loading of the package and using |\childdocof|.
If instead all processing is done
with the appropriate \textsf{childdoc} directives,
the argument of \textit{main} of |\childdocmain| can be empty.

An alternative version of the command line processing described
in \secref{sec:commandline} using the detection mechanism reads:
%
\begin{center}
|... -jobname "|\textit{target}|" "|[\textit{flags}]%
[|\def\jobname{|\textit{dest}|}|]|\input{|\textit{main}|}"|
\end{center}

%%%%%%%%%%%%%%%%%%%%%%%%%%%%%%%%%%%%%%%%%%%%%%%%%%%%%%%%%%%%%%%%%%%%%%%%%%%%%%%%
\subsection{Manual Code}
\label{sec:manual}

In case one cannot be certain whether the definitions file |childdoc.def|
is installed on the target \TeX{} distribution
and one prefers not to ship it,
it is conceivable to paste a few relevant commands into the sources.

To that end, drop all statements |\input{childdoc.def}|
and perform the replacements as outlined below.
Instead of |\childdocmain{|\textit{main}|}| add the following code
to the top of the main file:
%
\begin{center}
\begin{tabular}{l}
|\||ifdefined\childdocname\endinput\||fi\newif\ifchilddoc|\\
|\edef\childdocname{\scantokens\expandafter{\jobname\noexpand}}|\\
|\def\childdocmain{|\textit{main}|}\||ifx\childdocmain\childdocname\||else|\\
|\childdoctrue\includeonly{\childdocname}\let\jobname\childdocmain\||fi|\\
\end{tabular}
\end{center}
%
Instead of |\childdocof{|\textit{main}|}| just include the main file
at the top of each child file:
%
\begin{center}
|\input{|\textit{main}|}|
\end{center}
%
A simple redirection |\childdocforward{|\textit{dest}|}| is achieved by:
%
\begin{center}
|\def\jobname{|\textit{dest}|}\input{\jobname}|
\end{center}
%
The redirection with prefix
|\childdocforwardprefix[|\textit{prefix}|]{|\textit{dest}|}|
is accomplished by:
%
\begin{center}
\begin{tabular}{l}
|{\edef\jobname{\scantokens\expandafter{\jobname\noexpand}}|\\
|\def\redirectjob |\textit{prefix}|#1~~~{\gdef\jobname{|\textit{dest}|#1}}|\\
|\expandafter\redirectjob\jobname~~~}\input{\jobname}|
\end{tabular}
\end{center}

In an alternative approach,
child documents can be compiled by a specific command line
without additional code or specific definitions:
%
\begin{center}
|... -jobname "|\textit{target}|" "|[\textit{flags}]%
|\includeonly{|\textit{dest}|}\input{|\textit{main}|}"|
\end{center}
%

%%%%%%%%%%%%%%%%%%%%%%%%%%%%%%%%%%%%%%%%%%%%%%%%%%%%%%%%%%%%%%%%%%%%%%%%%%%%%%%%
%%%%%%%%%%%%%%%%%%%%%%%%%%%%%%%%%%%%%%%%%%%%%%%%%%%%%%%%%%%%%%%%%%%%%%%%%%%%%%%%
\section{Information}

%%%%%%%%%%%%%%%%%%%%%%%%%%%%%%%%%%%%%%%%%%%%%%%%%%%%%%%%%%%%%%%%%%%%%%%%%%%%%%%%
\subsection{Copyright}

Copyright \copyright{} 2017--2018 Niklas Beisert

This work may be distributed and/or modified under the
conditions of the \LaTeX{} Project Public License, either version 1.3
of this license or (at your option) any later version.
The latest version of this license is in
  \url{http://www.latex-project.org/lppl.txt}
and version 1.3 or later is part of all distributions of \LaTeX{}
version 2005/12/01 or later.

This work has the LPPL maintenance status `maintained'.

The Current Maintainer of this work is Niklas Beisert.

This work consists of the files |README.txt|, |childdoc.ins| and |childdoc.dtx|
as well as the derived files |childdoc.def|, |cdocsamp.tex|
with |cdocsch1.tex|, |cdocsch2.tex|, |cdocspt3.tex|, |cdocspt4.tex|,
|cdocsdrf.tex|, |cdocsfn1.tex|, |cdocsfn2.tex|
as well as |childdoc.pdf|.

%%%%%%%%%%%%%%%%%%%%%%%%%%%%%%%%%%%%%%%%%%%%%%%%%%%%%%%%%%%%%%%%%%%%%%%%%%%%%%%%
\subsection{Files and Installation}

The package consists of the files:
%
\begin{center}
\begin{tabular}{ll}
    |README.txt|   & readme file \\
    |childdoc.ins| & installation file \\
    |childdoc.dtx| & source file \\
    |childdoc.def| & definition file \\
    |cdocsamp.tex| & sample main file \\
    |cdocsch1.tex| & sample include file \\
    |cdocsch2.tex| & sample include file \\
    |cdocspt3.tex| & sample part file \\
    |cdocspt4.tex| & sample part file \\
    |cdocsdrf.tex| & sample redirection file \\
    |cdocsfn1.tex| & sample redirection file \\
    |cdocsfn2.tex| & sample redirection file \\
    |childdoc.pdf| & manual
\end{tabular}
\end{center}
%
The distribution consists of the files
|README.txt|, |childdoc.ins| and |childdoc.dtx|.
%
\begin{itemize}
\item
Run (pdf)\LaTeX{} on |childdoc.dtx|
to compile the manual |childdoc.pdf| (this file).
\item
Run \LaTeX{} on |childdoc.ins| to create the definitions file |childdoc.def|
and the sample |cdocsamp.tex| with include files
|cdocsch1.tex|, |cdocsch2.tex|, |cdocspt3.tex|, |cdocspt4.tex|,
|cdocsdrf.tex|, |cdocsfn1.tex|, |cdocsfn2.tex|.
Then copy the file |childdoc.def| to an appropriate directory of your \LaTeX{}
distribution, e.g.\ \textit{texmf-root}|/tex/latex/childdoc|.
\end{itemize}

%%%%%%%%%%%%%%%%%%%%%%%%%%%%%%%%%%%%%%%%%%%%%%%%%%%%%%%%%%%%%%%%%%%%%%%%%%%%%%%%
\subsection{Related CTAN Packages}

There are several other packages which offer a similar functionality:
%
\begin{itemize}
\item
The packages
\href{http://ctan.org/pkg/docmute}{\textsf{docmute}},
\href{http://ctan.org/pkg/includex}{\textsf{includex}} and
\href{http://ctan.org/pkg/standalone}{\textsf{standalone}}
provide commands to include only the document body of
a child file thus allowing both files to be compiled individually.
\item
The packages \href{http://ctan.org/pkg/subdocs}{\textsf{subdocs}}
and \href{http://ctan.org/pkg/subfiles}{\textsf{subfiles}}
provide structures in which the main and child documents can be
encapsulated and allowing them to be compiled individually.
The inclusion mechanism is different from the conventional |\include|.
\item
The package \href{http://ctan.org/pkg/combine}{\textsf{combine}}
is an elaborate solution to combine several documents into one.
\end{itemize}
%
See also the CTAN topic \href{http://ctan.org/topic/subdocs}{\textsf{subdocs}}
for further related packages.
The present package differs from the above solutions in that
a document structure constructed with the conventional |\include| mechanism
just needs two extra commands at the top of every file
such that all constituent files can be compiled individually.

%%%%%%%%%%%%%%%%%%%%%%%%%%%%%%%%%%%%%%%%%%%%%%%%%%%%%%%%%%%%%%%%%%%%%%%%%%%%%%%%
%\subsection{Feature Suggestions}
%
%The following is a list of features which may be useful for future
%versions of this package:
%%
%\begin{itemize}
%\item
%\ldots
%\end{itemize}

%%%%%%%%%%%%%%%%%%%%%%%%%%%%%%%%%%%%%%%%%%%%%%%%%%%%%%%%%%%%%%%%%%%%%%%%%%%%%%%%
\subsection{Revision History}

%%%%%%%%%%%%%%%%%%%%%%%%%%%%%%%%%%%%%%%%
\paragraph{v2.0:} 2018/12/30

\begin{itemize}
\item
immediate forward processing
\item
added |\childdocby| mechanism
\item
manual restructured
\end{itemize}

%%%%%%%%%%%%%%%%%%%%%%%%%%%%%%%%%%%%%%%%
\paragraph{v1.6:} 2018/01/17

\begin{itemize}
\item
application for development of include files
\item
corrections to manual
\end{itemize}

%%%%%%%%%%%%%%%%%%%%%%%%%%%%%%%%%%%%%%%%
\paragraph{v1.5:} 2017/05/21

\begin{itemize}
\item
more complete structuring introduced
\item
|\childdocof| introduced
\item
|\childdoc| renamed to |\childdocmain|
\item
|\childredirect| renamed to |\childdocforward| and |\childdocforwardprefix|
and functionality expanded
\end{itemize}

%%%%%%%%%%%%%%%%%%%%%%%%%%%%%%%%%%%%%%%%
\paragraph{v1.0:} 2017/04/27

\begin{itemize}
\item
manual and install package
\item
first version published on CTAN
\end{itemize}

%%%%%%%%%%%%%%%%%%%%%%%%%%%%%%%%%%%%%%%%
\paragraph{v0.6:} 2017/04/26

\begin{itemize}
\item
redirection mechanism added
\end{itemize}

%%%%%%%%%%%%%%%%%%%%%%%%%%%%%%%%%%%%%%%%
\paragraph{v0.5:} 2017/04/26

\begin{itemize}
\item
functionality in definition file
\end{itemize}


%%%%%%%%%%%%%%%%%%%%%%%%%%%%%%%%%%%%%%%%%%%%%%%%%%%%%%%%%%%%%%%%%%%%%%%%%%%%%%%%
%%%%%%%%%%%%%%%%%%%%%%%%%%%%%%%%%%%%%%%%%%%%%%%%%%%%%%%%%%%%%%%%%%%%%%%%%%%%%%%%
%%%%%%%%%%%%%%%%%%%%%%%%%%%%%%%%%%%%%%%%%%%%%%%%%%%%%%%%%%%%%%%%%%%%%%%%%%%%%%%%
\appendix

\settowidth\MacroIndent{\rmfamily\scriptsize 000\ }

 \DocInput{childdoc.dtx}

\end{document}
%</driver>
% \fi
%
% %%%%%%%%%%%%%%%%%%%%%%%%%%%%%%%%%%%%%%%%%%%%%%%%%%%%%%%%%%%%%%%%%%%%%%%%%%%%%%
% %%%%%%%%%%%%%%%%%%%%%%%%%%%%%%%%%%%%%%%%%%%%%%%%%%%%%%%%%%%%%%%%%%%%%%%%%%%%%%
% \section{Sample}
%\iffalse
%<*samplemain>
%\fi
%
% The following presents a sample document
% with two chapters, two parts, a title page,
% a compile flag as well as three forwarding files to set the flag.
% It consists of eight |.tex| files:
% \begin{center}
% \begin{tabular}{ll}
% |cdocsamp.tex|&main file\\
% |cdocsch1.tex|&include file for chapter 1\\
% |cdocsch2.tex|&include file for chapter 2\\
% |cdocspt3.tex|&include file for part 3\\
% |cdocspt4.tex|&include file for part 4\\
% |cdocsdrf.tex|&forwarding file for main file in draft mode\\
% |cdocsfi1.tex|&forwarding file for final version of chapter 1\\
% |cdocsfi2.tex|&forwarding file for final version of chapter 2\\
% \end{tabular}
% \end{center}
% Each of the eight files can be compiled directly by the \LaTeX{} compiler.
%
% %%%%%%%%%%%%%%%%%%%%%%%%%%%%%%%%%%%%%%
% \paragraph{Main File.}
%
% The main file is called |cdocsamp.tex|.
%
% Load the \textsf{childdoc} definitions and
% declare the filename for the main document:
%    \begin{macrocode}
\input{childdoc.def}
\childdocmain{}
%    \end{macrocode}

% Optional override for |\version| flag:
%    \begin{macrocode}
%%\ifchilddoc\else\providecommand{\version}{draft}\fi
%    \end{macrocode}

% Define the default values for the |\version| flag
% (|final| for the main file and |draft| for childs):
%    \begin{macrocode}
\ifchilddoc
\providecommand{\version}{draft}
\else
\providecommand{\version}{final}
\fi
%    \end{macrocode}

% Load the standard document class:
%    \begin{macrocode}
\documentclass[12pt]{article}
%    \end{macrocode}

% Start the document body:
%    \begin{macrocode}
\begin{document}
%    \end{macrocode}

% Declare a title page.
% Print title, part of document being processed and version flag:
%    \begin{macrocode}
\addtocounter{page}{-1}
\begin{center}
{\LARGE\bfseries{}childdoc example\par}
\vspace{1cm}
\ifchilddoc
\ifchilddocmanual part\else chapter\fi:
`\childdocname' of `\childdocjob'\par
\else
main document: `\childdocjob'\par
\fi
version: \version\par
\end{center}
\newpage
%    \end{macrocode}

% Manually include selected file,
% otherwise process as usual:
%    \begin{macrocode}
\ifchilddocmanual
\section*{part `\childdocname'}
\input{\childdocname}
\else
%    \end{macrocode}

% Include the two chapters:
%    \begin{macrocode}
\include{cdocsch1}
\include{cdocsch2}
%    \end{macrocode}

% Include the two parts unless only chapters should be displayed:
%    \begin{macrocode}
\ifchilddoc\else
\section{part three}
\input{cdocspt3}
\section{part four}
\input{cdocspt4}
\fi
%    \end{macrocode}

% Process as usual until here:
%    \begin{macrocode}
\fi
%    \end{macrocode}

% End of document body:
%    \begin{macrocode}
\end{document}
%    \end{macrocode}
%\iffalse
%</samplemain>
%\fi
%
% %%%%%%%%%%%%%%%%%%%%%%%%%%%%%%%%%%%%%%
% \paragraph{Chapter Include Files.}
%
% The include files are called |cdocsch1.tex| and |cdocsch2.tex|.
%
%\iffalse
%<*samplechap1|samplechap2>
%\fi

% Optional override for |\version| flag:
%    \begin{macrocode}
%%\providecommand{\version}{final}
%    \end{macrocode}

% Include the main document:
%    \begin{macrocode}
\input{childdoc.def}
\childdocof{cdocsamp}
%    \end{macrocode}

%\iffalse
%</samplechap1|samplechap2>
%\fi
%
%\iffalse
%<*samplechap1>
%\fi
% Some text for chapter 1:
%    \begin{macrocode}
\section{one}
some text in chapter one
%    \end{macrocode}

%\iffalse
%</samplechap1>
%\fi
% Some text for chapter 2:
%\iffalse
%<*samplechap2>
%\fi
%    \begin{macrocode}
\section{two}
more text in chapter two
%    \end{macrocode}

%\iffalse
%</samplechap2>
%\fi
%
% %%%%%%%%%%%%%%%%%%%%%%%%%%%%%%%%%%%%%%
% \paragraph{Part Include Files.}
%
% The include files are called |cdocspt3.tex| and |cdocspt4.tex|.
%
%\iffalse
%<*samplepart3|samplepart4>
%\fi

% Optional override for |\version| flag:
%    \begin{macrocode}
%%\providecommand{\version}{final}
%    \end{macrocode}

% Include the main document:
%    \begin{macrocode}
\input{childdoc.def}
\childdocby{cdocsamp}
%    \end{macrocode}

%\iffalse
%</samplepart3|samplepart4>
%\fi
%
%\iffalse
%<*samplepart3>
%\fi
% Some text for part 3:
%    \begin{macrocode}
some text in part three
%    \end{macrocode}

%\iffalse
%</samplepart3>
%\fi
% Some text for part 4:
%\iffalse
%<*samplepart4>
%\fi
%    \begin{macrocode}
more text in part four
%    \end{macrocode}

%\iffalse
%</samplepart4>
%\fi
%
% %%%%%%%%%%%%%%%%%%%%%%%%%%%%%%%%%%%%%%
% \paragraph{Forwarding for a Complete Draft.}
%
% The following forwarding file |cdocsdrf.tex|
% compiles the main document in draft mode:
%\iffalse
%<*sampledraft>
%\fi
%    \begin{macrocode}
\def\version{draft}
\input{childdoc.def}
\childdocforward{cdocsamp}
%    \end{macrocode}

%\iffalse
%</sampledraft>
%\fi
%
% %%%%%%%%%%%%%%%%%%%%%%%%%%%%%%%%%%%%%%
% \paragraph{Forwarding for Final Version of the Chapters.}
%
% The following forwarding files |cdocsfn1.tex| and |cdocsfn2.tex|
% (with identical content)
% compile the final versions of the child documents
% |cdocsch1.tex| and |cdocsch2.tex|, respectively:
%\iffalse
%<*samplefinal>
%\fi
%    \begin{macrocode}
\def\version{final}
\input{childdoc.def}
\childdocforwardprefix[cdocsamp]{cdocsfn}{cdocsch}
%    \end{macrocode}

%\iffalse
%</samplefinal>
%\fi
%
% %%%%%%%%%%%%%%%%%%%%%%%%%%%%%%%%%%%%%%
% \paragraph{Command Line Processing.}
%
% The following three command lines generate the output files
% |cdocscld|, |cdocscl1| and |cdocscl2|
% which should be identical to
% |cdocsdrf|, |cdocsch1| and |cdocsfn2|, respectively:
% \begin{center}
% \begin{tabular}{l}
% |latex -jobname cdocscld \|\\
% |  "\def\version{draft}\input{childdoc.def}\childdocforward{cdocsamp}"|\\
% |latex -jobname cdocscl1 \|\\
% |  "\input{childdoc.def}\childdocforward[cdocsamp]{cdocsch1}"|\\
% |latex -jobname cdocscl2 \|\\
% |  "\def\version{final}\input{childdoc.def}\childdocforward{cdocsch2}"|
% \end{tabular}
% \end{center}
% Note that the trailing backslash on each first line
% merely continues the input to the second line
% (for convenient cut ant paste).
% Furthermore, the command |latex| can be replaced by any
% of its alternative versions such as |pdflatex|.
%
% %%%%%%%%%%%%%%%%%%%%%%%%%%%%%%%%%%%%%%%%%%%%%%%%%%%%%%%%%%%%%%%%%%%%%%%%%%%%%%
% %%%%%%%%%%%%%%%%%%%%%%%%%%%%%%%%%%%%%%%%%%%%%%%%%%%%%%%%%%%%%%%%%%%%%%%%%%%%%%
% \section{Implementation}
%\iffalse
%<*package>
%\fi
%
% This section describes the definitions file |childdoc.def|.

% The definitions cannot be loaded using |\usepackage| or |\RequirePackage|
% which has a mechanism to prevent loading a style file more than once.
% When loading the definitions by means of |\input|
% multiple instances have to be prevented manually:
%\iffalse
%This code needs to be before the `\ProvidesFile' directive
%which is defined at the beginning of this file.
%Therefore it is also placed there and commented out here.
%</package>
%<*discard>
%\fi
%    \begin{macrocode}
\ifdefined\childdocmain\endinput\fi
%    \end{macrocode}
%\iffalse
%</discard>
%<*package>
%\fi
%
% \macro{\ifchilddoc}
% \macro{\ifchilddocmanual}
% The conditional |\ifchilddoc| tells whether a
% child (true) or main (false) document is being compiled.
% The conditional |\ifchilddocmanual| tells whether
% the |\includeonly| mechanism is used (false) or
% the selection of child files must be performed manually (true).
% The definitions initialise to false:
%    \begin{macrocode}
\newif\ifchilddoc
\newif\ifchilddocmanual
%    \end{macrocode}

% \macro{\childdocname}
% \macro{\childdocjob}
% The macro |\childdocname| stores the name of the main document
% to be compiled. The macro |\childdocjob| stores the name of
% the document on which the \LaTeX{} compiler was originally invoked.
% The content of |\jobname| cannot be compared
% to filenames specified in the source due to different catcodes.
% The following code rescans |\jobname|, stores the result
% in |\childdocname| and saves a copy in |\childdocjob|:
%    \begin{macrocode}
\edef\childdocname{\scantokens\expandafter{\jobname\noexpand}}
\let\childdocjob\childdocname
%    \end{macrocode}

% \macro{\childdocdisable}
% The macro |\childdocdisable| prevents the main file
% from being processed more than once.
% At this stage, the main document command |\childdocmain|
% is assumed to be called once again where it should do nothing.
% Any subsequent call to it should prevent
% a secondary processing of the main document
% It overwrites the forwarding commands
% |\childdocof| and |\childdocforward|
% with empty macros to prevent further inclusions of the main document:
%    \begin{macrocode}
\newcommand{\childdocdisable}
{
  \renewcommand{\childdocmain}[1]{\renewcommand{\childdocmain}[1]{\endinput}}
  \renewcommand{\childdocof}[1]{}
  \renewcommand{\childdocby}[2][]{}
  \renewcommand{\childdocforward}[2][]{}
  \renewcommand{\childdocdisable}{}
}
%    \end{macrocode}

% \macro{\childdocmain}
% The macro |\childdocmain| is to be called at the top of the main file
% with nothing or the main filename (without extension) as argument.
% First, it breaks loops.
% If the argument is not empty and does not match |\childdocname|
% (which is set by the first inclusion of |childdoc.def|),
% |\ifchilddoc| is set to true, |\includeonly| is applied to the child file
% and |\jobname| is set to the main file
% (for proper handling of |.aux| files):
%    \begin{macrocode}
\newcommand{\childdocmain}[1]
{
  \childdocdisable\childdocmain{}
  \if?#1?\else
    \begingroup
      \def\childdoctmp{#1}
      \ifx\childdoctmp\childdocname
        \def\childdoctmp{}
      \else
        \def\childdoctmp
        {
          \childdoctrue
          \includeonly{\childdocname}
          \def\childdocjob{#1}
          \def\jobname{#1}
        }
      \fi
      \expandafter
    \endgroup
    \childdoctmp
  \fi
}
%    \end{macrocode}

% \macro{\childdocof}
% The command |\childdocof| redirects
% compilation to the main file |#1|.
%    \begin{macrocode}
\newcommand{\childdocof}[1]
{
  \childdocdisable
  \childdoctrue
  \includeonly{\childdocname}
  \def\jobname{#1}
  \def\childdocjob{#1}
  \input{#1}
}
%    \end{macrocode}

% \macro{\childdocby}
% The command |\childdocby| ....
%    \begin{macrocode}
\newcommand{\childdocby}[2][]
{
  \childdocdisable
  \childdoctrue
  \childdocmanualtrue
  \if?#1?\else
    \def\jobname{#2}
  \fi
  \def\childdocjob{#2}
  \input{#2}
  \endinput
}
%    \end{macrocode}

% \macro{\childdocforward}
% The command |\childdocforward| redirects
% compilation to the main file or
% (if the optional argument is given) a child file.
% Parameters are set as if the main file
% or a child file starting with |\childdocof| was compiled.
% Then compilation is handed over to the main file:
%    \begin{macrocode}
\newcommand{\childdocforward}[2][]
{
  \begingroup
    \if?#1?
      \def\childdoctmp
      {
        \def\childdocname{#2}
        \def\childdocjob{#2}
        \def\jobname{#2}
        \input{#2}
        \endinput
      }
    \else
      \def\childdoctmp
      {
        \childdocdisable
        \def\childdocname{#2}
        \childdoctrue
        \includeonly{#2}
        \def\childdocjob{#1}
        \def\jobname{#1}
        \input{#1}
        \endinput
      }
    \fi
    \expandafter
  \endgroup
  \childdoctmp
}
%    \end{macrocode}

% \macro{\childdocforwardprefix}
% The command |\childdocforwardprefix| redirects
% compilation to the main or a child file by means of a pattern.
% The prefix |#1| in the current filename is replaced by |#2|
% and the suffix of the current filename is kept
% (it is assumed that the filename does not contain the substring `|~~~|'
% which is used as a delimiter).
% Compilation is handed over to the new file by |\childdocforward|:
%    \begin{macrocode}
\newcommand{\childdocforwardprefix}[3][]
{
  \begingroup
    \def\childdocextract #2##1~~~{\def\childdoctmp{\childdocforward[#1]{#3##1}}}
    \expandafter\childdocextract\childdocname~~~
    \expandafter
  \endgroup
  \childdoctmp
}
%    \end{macrocode}

% \macro{\childdoc}
% The deprecated macro |\childdoc| is a legacy version of |\childdocmain|:
%    \begin{macrocode}
\newcommand{\childdoc}{\childdocmain}
%    \end{macrocode}

% \macro{\childdocredirect}
% The deprecated macro |\childdocredirect| is a legacy version
% of |\childdocforward| and |\childdocforwardprefix|:
%    \begin{macrocode}
\newcommand{\childdocredirect}[2][]
{
  \begingroup
    \if?#1?
      \def\childdoctmp{\childdocforward{#2}}
    \else
      \def\childdoctmp{\childdocforwardprefix{#1}{#2}}
    \fi
    \expandafter
  \endgroup
  \childdoctmp
}
%    \end{macrocode}

%\iffalse
%</package>
%\fi
%
\endinput
|\\
|\childdocof{|\textit{main}|}|\\
\end{tabular}
\end{center}
at the top of every child file \textit{child}
which is included by |\include{|\textit{child}|}|
from within the main file
(or at least for those files to be compiled individually).
The argument \textit{main} must be the filename of the main file.

There are a couple of
considerations in setting up the main and child documents:

%%%%%%%%%%%%%%%%%%%%%%%%%%%%%%%%%%%%%%%%
\paragraph{Restrictions.}

Please note the following restrictions:
\begin{itemize}
\item
|\childdocmain| must be called with one argument \textit{main}
to ensure compatibility with earlier version of the package.
It must either be empty (|\childdocmain{}|)
or precisely match the filename of the main file in which it is specified.
See \secref{sec:detection} for further information.
\item
The filename \textit{main} must be specified without the |.tex| extension.
\item
The filename \textit{main} is case sensitive
(even in case-insensitive file systems)
due to internal string comparison.
\item
The argument \textit{main} should be fully expanded, it cannot be a macro.
\item
Subdirectories and special characters should be avoided in filenames.
\item
The command |\childdocmain{|\textit{main}|}| must be followed by a whitespace.
It should not be followed immediately by another command
or by a comment mark `|%|'.
This is because the \TeX{} parser reads the token immediately following
the argument of |\childdocmain| and puts it
at the beginning of every child section;
however, a white\-space is ignored.
\end{itemize}

%%%%%%%%%%%%%%%%%%%%%%%%%%%%%%%%%%%%%%%%
\paragraph{Content of Main File.}

It is advisable to place all content in the child files included by |\include|.
Any output contained in the main file will appear in all child documents
unless suppressed manually;
it cannot be suppressed automatically by the |\includeonly| directive
and thus should normally be avoided.
A method to include some content in the main file
by means of conditional processing is described in \secref{sec:conditional}.

%%%%%%%%%%%%%%%%%%%%%%%%%%%%%%%%%%%%%%%%
\paragraph{Page Numbering.}

When only a part of the document is compiled,
the appropriate numbering of pages
(as well as other status parameters)
is determined from the |.aux| files.
The latter contain information from previous passes.
However this information needs to propagate through
all intermediate child documents.
Therefore the page numbering in child documents may well
be inconsistent until the complete document is compiled at least once.

A useful (if unconventional) way to always ensure a consistent
page numbering is to restart the numbering in each child document
and denote the pages by `\textit{child}|.|\textit{page}'
where \textit{child} represents the chapter/section number of the child file.
This can be achieved by the command
|\numberwithin{page}{|\textit{child}|}|
of the \textsf{amsmath} package
where \textit{child} can be |chapter| or |section|
depending on the chosen structuring.
Alternatively, one can modify the macro |\thepage| appropriately
and reset the counter |page| at the start of each child file.

%%%%%%%%%%%%%%%%%%%%%%%%%%%%%%%%%%%%%%%%%%%%%%%%%%%%%%%%%%%%%%%%%%%%%%%%%%%%%%%%
\subsection{Conditional Processing}
\label{sec:conditional}

The package provides a mechanism to compile different versions
of a document. To customise the versions further some conditional processing
can come in handy to distinguish which version is being compiled.
The package provides two macros to describe the compilation context:

%%%%%%%%%%%%%%%%%%%%%%%%%%%%%%%%%%%%%%%%
\DescribeMacro{\ifchilddoc}
The conditional |\ifchilddoc| distinguishes between the compilation of
child documents and the main document:
%
\begin{center}
|\ifchilddoc |\textit{child-code}| |[|\||else |\textit{main-code}]| \||fi|
\end{center}

%%%%%%%%%%%%%%%%%%%%%%%%%%%%%%%%%%%%%%%%
\DescribeMacro{\childdocname}
\DescribeMacro{\childdocjob}
The macro |\childdocname| contains the filename (without extension)
of the main or child file being processed.
Note that |\childdocjob| will always contain the name of the main file.

%%%%%%%%%%%%%%%%%%%%%%%%%%%%%%%%%%%%%%%%
\paragraph{Title Page.}

Conditional processing can be used to include a title or banner page
in the main document when proper precautions are taken.
Importantly, the code in the main file should ensure that the page counter
(as well as other status parameters which are stored in the |.aux| files)
takes the same value after the conditional processing.
Otherwise the page numbers may take divergent values
depending on which part is compiled.

For example, a title page could be declared by:
%
\begin{center}
\begin{tabular}{l}
|\ifchilddoc\||else|\\
|\addtocounter{page}{-1}|\\
\textit{code for title page}\\
|\newpage|\\
|\||fi|
\end{tabular}
\end{center}
%
A banner page for the child documents can be generated by:
%
\begin{center}
\begin{tabular}{l}
|\ifchilddoc|\\
|\addtocounter{page}{-1}|\\
\textit{code for banner page}\\
|\newpage|\\
|\||fi|
\end{tabular}
\end{center}
%
Here one could write a message such as:
\begin{center}
|This is the part \childdocname{} of \childdocjob{}.|
\end{center}

%%%%%%%%%%%%%%%%%%%%%%%%%%%%%%%%%%%%%%%%%%%%%%%%%%%%%%%%%%%%%%%%%%%%%%%%%%%%%%%%
\subsection{Flags}
\label{sec:flags}

The package makes it easy to generate different versions
of the main or child documents.
To this end compilation flags can be defined
and assigned different default values.
They will be particularly useful in conjunction
with the forwarding mechanism described in \secref{sec:forward}.

For example, it may be useful to have a flag |\version|
which can be set to |draft| or |final|.
The document source will contain some conditional code
depending on the value of |\version|.
Suppose further, the flag should default to |final| for the main file
and to |draft| for child files
which is a natural assignment for editing the document.
This is achieved by placing the following code
in the preamble of the main document
(below the |\childdocmain| directive):
%
\begin{center}
\begin{tabular}{l}
|\ifchilddoc|\\
|\providecommand{\version}{draft}|\\
|\||else|\\
|\providecommand{\version}{final}|\\
|\||fi|
\end{tabular}
\end{center}
%
The definition by |\providecommand| makes sure
that previous definitions are not overwritten.
Further statements |\providecommand{\version}{...}|
can thus be added before the above code to override it.

For the main file, one might add a line
(between |\childdocmain| and the above block)
%
\begin{center}
|%\ifchilddoc\||else\providecommand{\version}{draft}\||fi|
\end{center}
%
which can be uncommented to produce a draft version.
Likewise one can add a line to the very top of a child file
(above the |\childdocof{|\textit{main}|}| directive)
%
\begin{center}
|%\providecommand{\version}{final}|
\end{center}
%
which can be uncommented to produce the final version of this child document.

%%%%%%%%%%%%%%%%%%%%%%%%%%%%%%%%%%%%%%%%%%%%%%%%%%%%%%%%%%%%%%%%%%%%%%%%%%%%%%%%
\subsection{Forwarding}
\label{sec:forward}

Different versions of the main or child documents
using compilation flags as described in \secref{sec:flags}
can be (permanently) stored in different files
for convenient compilation, viewing and distribution.
To this end, the package defines a command
to pass on compilation to a different file:

%%%%%%%%%%%%%%%%%%%%%%%%%%%%%%%%%%%%%%%%
\DescribeMacro{\childdocforward}
The command |\childdocforward| redirects processing to
another source file:
%
\begin{center}
\begin{tabular}{l}
|% \iffalse
%
% childdoc.dtx Copyright (C) 2017-2018 Niklas Beisert
%
% This work may be distributed and/or modified under the
% conditions of the LaTeX Project Public License, either version 1.3
% of this license or (at your option) any later version.
% The latest version of this license is in
%   http://www.latex-project.org/lppl.txt
% and version 1.3 or later is part of all distributions of LaTeX
% version 2005/12/01 or later.
%
% This work has the LPPL maintenance status `maintained'.
%
% The Current Maintainer of this work is Niklas Beisert.
%
% This work consists of the files childdoc.dtx and childdoc.ins
% and the derived files childdoc.def and cdocsamp.tex with
% cdocsch1.tex, cdocsch2.tex, cdocsdrf.tex, cdocsfn1.tex, cdocsfn2.tex.
%
%<package>\ifdefined\childdocmain\endinput\fi
%<package>\ProvidesFile{childdoc.def}[2018/12/30 v2.0 child document driver]
%<samplemain>\ProvidesFile{cdocsamp.tex}[2018/12/30 v2.0 sample for childdoc]
%<*driver>
%\ProvidesFile{childdoc.drv}[2018/12/30 v2.0 childdoc reference manual file]
\PassOptionsToClass{10pt,a4paper}{article}
\documentclass{ltxdoc}

\usepackage[margin=35mm]{geometry}
\usepackage{hyperref}
\usepackage{hyperxmp}
\usepackage[usenames]{color}

\hypersetup{colorlinks=true}
\hypersetup{pdfstartview=FitH}
\hypersetup{pdfpagemode=UseNone}
\hypersetup{pdfsource={}}
\hypersetup{pdflang={en-UK}}
\hypersetup{pdfcopyright={Copyright 2017-2018 Niklas Beisert.
  This work may be distributed and/or modified under the
  conditions of the LaTeX Project Public License, either version 1.3
  of this license or (at your option) any later version.}}
\hypersetup{pdflicenseurl={http://www.latex-project.org/lppl.txt}}
\hypersetup{pdfcontactaddress={ETH Zurich, ITP, HIT K,
  Wolfgang-Pauli-Strasse 27}}
\hypersetup{pdfcontactpostcode={8093}}
\hypersetup{pdfcontactcity={Zurich}}
\hypersetup{pdfcontactcountry={Switzerland}}
\hypersetup{pdfcontactemail={nbeisert@itp.phys.ethz.ch}}
\hypersetup{pdfcontacturl={http://people.phys.ethz.ch/\xmptilde nbeisert/}}

\newcommand{\secref}[1]{\hyperref[#1]{section \ref*{#1}}}

\parskip1ex
\parindent0pt
\let\olditemize\itemize
\def\itemize{\olditemize\parskip0pt}

\begin{document}

\title{The \textsf{childdoc} Package}
\hypersetup{pdftitle={The childdoc Package}}
\author{Niklas Beisert\\[2ex]
  Institut f\"ur Theoretische Physik\\
  Eidgen\"ossische Technische Hochschule Z\"urich\\
  Wolfgang-Pauli-Strasse 27, 8093 Z\"urich, Switzerland\\[1ex]
  \href{mailto:nbeisert@itp.phys.ethz.ch}
  {\texttt{nbeisert@itp.phys.ethz.ch}}}
\hypersetup{pdfauthor={Niklas Beisert}}
\hypersetup{pdfsubject={Manual for the LaTeX2e Package childdoc}}
\date{30 December 2018, \textsf{v2.0}}
\maketitle

\begin{abstract}\noindent
\textsf{childdoc} is a \LaTeXe{} package
that enables the direct compilation
of document sections included by |\include|
to individual files.
\end{abstract}

\begingroup
\parskip0ex
\tableofcontents
\endgroup

%%%%%%%%%%%%%%%%%%%%%%%%%%%%%%%%%%%%%%%%%%%%%%%%%%%%%%%%%%%%%%%%%%%%%%%%%%%%%%%%
%%%%%%%%%%%%%%%%%%%%%%%%%%%%%%%%%%%%%%%%%%%%%%%%%%%%%%%%%%%%%%%%%%%%%%%%%%%%%%%%
\section{Introduction}

\LaTeX{} provides a mechanism to structure a large document (such as a book)
into a main file and several child files (containing the chapters)
using the |\include| command.
This mechanism is beneficial for documents
which span hundreds of pages in order to
make the source file(s) more manageable.
Moreover, compilation can be restricted to
selected child files by means of the |\includeonly| command.
The latter feature can be used to reduce the compilation time while editing
(this was significantly more useful in the earlier days of \LaTeX{})
or to generate a smaller document which is easier to navigate.
Another application of |\includeonly| is to generate
documents consisting of selected parts of the complete document.

However, there are a few drawbacks of the plain |\include| mechanism:
\begin{itemize}
\item
The child files cannot be compiled on their own,
they can only be compiled via the main file.
A naive editing environment
(such as a text editor with an option
to have the current file processed by \LaTeX)
may require one to switch to the main file before compiling;
attempting to compile the child file produces errors.
\item
The main file must be modified (each time)
to adjust the |\includeonly| command
to the present needs. This easily leaves the main file in a messy state.
\item
The generated document will always carry the filename
of the main document. This is inconvenient if
several child files are to be compiled and
to be kept for distribution.
\end{itemize}

The present package provides a simple interface
to make child files individually compilable by \LaTeX{}.
Compiling a child file then has the same effect as compiling
the main file with an |\includeonly| command
to select the appropriate child.
Moreover the generated document will carry the name of the child
rather than the main file.
This resolves all three above issues.

This feature is meant to make the editing of books,
thesis documents and lecture notes somewhat more convenient.
However, the package can also be used efficiently for
composing a series of documents (such as exercise sheets)
which are typically distributed individually.
It then assists the author in generating the individual documents
(potentially in different versions)
as well as a document containing the collected series.
Another application is in developing style files
or other kinds of included material
where compilation of the style file could redirect
to a sample or test file.

%%%%%%%%%%%%%%%%%%%%%%%%%%%%%%%%%%%%%%%%%%%%%%%%%%%%%%%%%%%%%%%%%%%%%%%%%%%%%%%%
%%%%%%%%%%%%%%%%%%%%%%%%%%%%%%%%%%%%%%%%%%%%%%%%%%%%%%%%%%%%%%%%%%%%%%%%%%%%%%%%
\section{Usage}

First of all, the package \textsf{childdoc} is \emph{not} a standard
\LaTeXe{} |.sty| style file! Therefore it needs to be invoked in
a non-standard way.

%%%%%%%%%%%%%%%%%%%%%%%%%%%%%%%%%%%%%%%%%%%%%%%%%%%%%%%%%%%%%%%%%%%%%%%%%%%%%%%%
\subsection{Included Files}
\label{sec:include}

%%%%%%%%%%%%%%%%%%%%%%%%%%%%%%%%%%%%%%%%
\DescribeMacro{\childdocmain}
To use the package, add the commands
\begin{center}
\begin{tabular}{l}
|\input{childdoc.def}|\\
|\childdocmain{}|\\
\end{tabular}
\end{center}
at the very top of the main \LaTeX{} file,
in particular \emph{before} the |\documentclass| statement!
The argument of |\childdocmain| should be left empty
(but it must be present).

%%%%%%%%%%%%%%%%%%%%%%%%%%%%%%%%%%%%%%%%
\DescribeMacro{\childdocof}
Furthermore, add the commands
\begin{center}
\begin{tabular}{l}
|\input{childdoc.def}|\\
|\childdocof{|\textit{main}|}|\\
\end{tabular}
\end{center}
at the top of every child file \textit{child}
which is included by |\include{|\textit{child}|}|
from within the main file
(or at least for those files to be compiled individually).
The argument \textit{main} must be the filename of the main file.

There are a couple of
considerations in setting up the main and child documents:

%%%%%%%%%%%%%%%%%%%%%%%%%%%%%%%%%%%%%%%%
\paragraph{Restrictions.}

Please note the following restrictions:
\begin{itemize}
\item
|\childdocmain| must be called with one argument \textit{main}
to ensure compatibility with earlier version of the package.
It must either be empty (|\childdocmain{}|)
or precisely match the filename of the main file in which it is specified.
See \secref{sec:detection} for further information.
\item
The filename \textit{main} must be specified without the |.tex| extension.
\item
The filename \textit{main} is case sensitive
(even in case-insensitive file systems)
due to internal string comparison.
\item
The argument \textit{main} should be fully expanded, it cannot be a macro.
\item
Subdirectories and special characters should be avoided in filenames.
\item
The command |\childdocmain{|\textit{main}|}| must be followed by a whitespace.
It should not be followed immediately by another command
or by a comment mark `|%|'.
This is because the \TeX{} parser reads the token immediately following
the argument of |\childdocmain| and puts it
at the beginning of every child section;
however, a white\-space is ignored.
\end{itemize}

%%%%%%%%%%%%%%%%%%%%%%%%%%%%%%%%%%%%%%%%
\paragraph{Content of Main File.}

It is advisable to place all content in the child files included by |\include|.
Any output contained in the main file will appear in all child documents
unless suppressed manually;
it cannot be suppressed automatically by the |\includeonly| directive
and thus should normally be avoided.
A method to include some content in the main file
by means of conditional processing is described in \secref{sec:conditional}.

%%%%%%%%%%%%%%%%%%%%%%%%%%%%%%%%%%%%%%%%
\paragraph{Page Numbering.}

When only a part of the document is compiled,
the appropriate numbering of pages
(as well as other status parameters)
is determined from the |.aux| files.
The latter contain information from previous passes.
However this information needs to propagate through
all intermediate child documents.
Therefore the page numbering in child documents may well
be inconsistent until the complete document is compiled at least once.

A useful (if unconventional) way to always ensure a consistent
page numbering is to restart the numbering in each child document
and denote the pages by `\textit{child}|.|\textit{page}'
where \textit{child} represents the chapter/section number of the child file.
This can be achieved by the command
|\numberwithin{page}{|\textit{child}|}|
of the \textsf{amsmath} package
where \textit{child} can be |chapter| or |section|
depending on the chosen structuring.
Alternatively, one can modify the macro |\thepage| appropriately
and reset the counter |page| at the start of each child file.

%%%%%%%%%%%%%%%%%%%%%%%%%%%%%%%%%%%%%%%%%%%%%%%%%%%%%%%%%%%%%%%%%%%%%%%%%%%%%%%%
\subsection{Conditional Processing}
\label{sec:conditional}

The package provides a mechanism to compile different versions
of a document. To customise the versions further some conditional processing
can come in handy to distinguish which version is being compiled.
The package provides two macros to describe the compilation context:

%%%%%%%%%%%%%%%%%%%%%%%%%%%%%%%%%%%%%%%%
\DescribeMacro{\ifchilddoc}
The conditional |\ifchilddoc| distinguishes between the compilation of
child documents and the main document:
%
\begin{center}
|\ifchilddoc |\textit{child-code}| |[|\||else |\textit{main-code}]| \||fi|
\end{center}

%%%%%%%%%%%%%%%%%%%%%%%%%%%%%%%%%%%%%%%%
\DescribeMacro{\childdocname}
\DescribeMacro{\childdocjob}
The macro |\childdocname| contains the filename (without extension)
of the main or child file being processed.
Note that |\childdocjob| will always contain the name of the main file.

%%%%%%%%%%%%%%%%%%%%%%%%%%%%%%%%%%%%%%%%
\paragraph{Title Page.}

Conditional processing can be used to include a title or banner page
in the main document when proper precautions are taken.
Importantly, the code in the main file should ensure that the page counter
(as well as other status parameters which are stored in the |.aux| files)
takes the same value after the conditional processing.
Otherwise the page numbers may take divergent values
depending on which part is compiled.

For example, a title page could be declared by:
%
\begin{center}
\begin{tabular}{l}
|\ifchilddoc\||else|\\
|\addtocounter{page}{-1}|\\
\textit{code for title page}\\
|\newpage|\\
|\||fi|
\end{tabular}
\end{center}
%
A banner page for the child documents can be generated by:
%
\begin{center}
\begin{tabular}{l}
|\ifchilddoc|\\
|\addtocounter{page}{-1}|\\
\textit{code for banner page}\\
|\newpage|\\
|\||fi|
\end{tabular}
\end{center}
%
Here one could write a message such as:
\begin{center}
|This is the part \childdocname{} of \childdocjob{}.|
\end{center}

%%%%%%%%%%%%%%%%%%%%%%%%%%%%%%%%%%%%%%%%%%%%%%%%%%%%%%%%%%%%%%%%%%%%%%%%%%%%%%%%
\subsection{Flags}
\label{sec:flags}

The package makes it easy to generate different versions
of the main or child documents.
To this end compilation flags can be defined
and assigned different default values.
They will be particularly useful in conjunction
with the forwarding mechanism described in \secref{sec:forward}.

For example, it may be useful to have a flag |\version|
which can be set to |draft| or |final|.
The document source will contain some conditional code
depending on the value of |\version|.
Suppose further, the flag should default to |final| for the main file
and to |draft| for child files
which is a natural assignment for editing the document.
This is achieved by placing the following code
in the preamble of the main document
(below the |\childdocmain| directive):
%
\begin{center}
\begin{tabular}{l}
|\ifchilddoc|\\
|\providecommand{\version}{draft}|\\
|\||else|\\
|\providecommand{\version}{final}|\\
|\||fi|
\end{tabular}
\end{center}
%
The definition by |\providecommand| makes sure
that previous definitions are not overwritten.
Further statements |\providecommand{\version}{...}|
can thus be added before the above code to override it.

For the main file, one might add a line
(between |\childdocmain| and the above block)
%
\begin{center}
|%\ifchilddoc\||else\providecommand{\version}{draft}\||fi|
\end{center}
%
which can be uncommented to produce a draft version.
Likewise one can add a line to the very top of a child file
(above the |\childdocof{|\textit{main}|}| directive)
%
\begin{center}
|%\providecommand{\version}{final}|
\end{center}
%
which can be uncommented to produce the final version of this child document.

%%%%%%%%%%%%%%%%%%%%%%%%%%%%%%%%%%%%%%%%%%%%%%%%%%%%%%%%%%%%%%%%%%%%%%%%%%%%%%%%
\subsection{Forwarding}
\label{sec:forward}

Different versions of the main or child documents
using compilation flags as described in \secref{sec:flags}
can be (permanently) stored in different files
for convenient compilation, viewing and distribution.
To this end, the package defines a command
to pass on compilation to a different file:

%%%%%%%%%%%%%%%%%%%%%%%%%%%%%%%%%%%%%%%%
\DescribeMacro{\childdocforward}
The command |\childdocforward| redirects processing to
another source file:
%
\begin{center}
\begin{tabular}{l}
|\input{childdoc.def}|\\
|\childdocforward[|\textit{main}|]{|\textit{dest}|}|\\
\end{tabular}
\end{center}
%
The argument \textit{dest} is the destination file
(without extension).
It should be the main file or one of the child files.
Note that further \textsf{childdoc} directives
such as |\childdocof| and |\childdocforward|
in the indicated file will be processed in this form.
The optional argument \textit{main}
passes on directly to the main file \textit{main}
while pretending to compile the child \textit{dest}.
This form behaves as if \textit{dest}
issues |\childdocof{|\textit{main}|}| right away,
and no further \textsf{childdoc} directives will be processed.

%%%%%%%%%%%%%%%%%%%%%%%%%%%%%%%%%%%%%%%%
\DescribeMacro{\...prefix}
In the alternative form |\childdocforwardprefix|,
%
\begin{center}
\begin{tabular}{l}
|\input{childdoc.def}|\\
|\childdocforwardprefix[|\textit{main}|]{|\textit{prefix}|}{|\textit{dest}|}|
\end{tabular}
\end{center}
%
the destination file is determined by a pattern
depending on the current file:
To make this work, the current file must be called
`{\textit{prefix}\hspace{0.2em}\textit{suffix}}'
with \textit{prefix} matching precisely the argument.
Processing is then passed on to the file
`{\textit{dest}\hspace{0.2em}\textit{suffix}}'.
Surely, the same effect is achieved by
directly specifying the
argument `{\textit{dest}\hspace{0.2em}\textit{suffix}}'
in the first form.
However, that requires to set up a different file
for each child. With the alternative form of the command
all these files can have exactly the same content
which simplifies setting them up and maintaining them.

For example, the following file |draft.tex|
with a compilation flag |\version| as described in \secref{sec:flags}
compiles the main document as a draft:
%
\begin{center}
\begin{tabular}{l}
|\def\version{draft}|\\
|\input{childdoc.def}|\\
|\childdocforward{|\textit{main}|}|
\end{tabular}
\end{center}
%
Likewise, the following files |final|\textit{nn}|.tex|
compile the final version of the child document
|child|\textit{nn}|.tex|:
%
\begin{center}
\begin{tabular}{l}
|\def\version{final}|\\
|\input{childdoc.def}|\\
|\childdocforwardprefix{final}{child}|
\end{tabular}
\end{center}
%

Note that when several versions of a main file and/or of each child file
are to be generated, it may be convenient to set up a |Makefile| or
shell script to automatise the process.

%%%%%%%%%%%%%%%%%%%%%%%%%%%%%%%%%%%%%%%%%%%%%%%%%%%%%%%%%%%%%%%%%%%%%%%%%%%%%%%%
\subsection{Command Line Processing}
\label{sec:commandline}

The effect of redirection files can also be achieved by invoking
the \LaTeX{} compiler with a more elaborate command line.
Most conveniently this should be done as part
of a shell script or a |Makefile|.

When using \textsf{childdoc} in the main file, the following
command lines effectively perform a redirection
(note that depending on the shell being used,
backslashes may have to be doubled: `|\|' $\to$ `|\\|'):
%
\begin{center}
|... -jobname "|\textit{target}|" |\\|"|[\textit{flags}]%
|\input{childdoc.def}\childdocforward[|\textit{main}|]{|\textit{dest}|}"|
\end{center}
%
Here \textit{target} is the name of the output file,
\textit{main} is the name of the main file
and \textit{dest} is the name of the main or child file to be processed
(all filenames without extensions).
The optional argument \textit{main} can be omitted
if \textit{main} matches \textit{dest}.
Optionally, compilation \textit{flags} can be defined via |\def| commands.
This command line makes the \TeX{} engine believe
it is compiling the file \textit{target}
whose content is specified as the latter parameter.
The provided code then forwards the processing to
\textit{main} or \textit{dest} as described in \secref{sec:forward}.

%%%%%%%%%%%%%%%%%%%%%%%%%%%%%%%%%%%%%%%%%%%%%%%%%%%%%%%%%%%%%%%%%%%%%%%%%%%%%%%%
\subsection{Include by Input}
\label{sec:input}

Including child documents by |\include| has some restrictions by design.
Most notably, the content of a child document always occupies
its own set of pages; pages cannot be shared between child documents.
Usually, this behaviour makes perfect sense
because each child document contain an essential part of the document.
However, in some situations it may be desirable to compose
a document from a collection of parts
without having mandatory page breaks between then.
For this case, the package
provides a mechanism to include parts
by |\input| which can also be processed individually.
However, by construction this mechanism
requires manual handling of the content to be output.

%%%%%%%%%%%%%%%%%%%%%%%%%%%%%%%%%%%%%%%%
\DescribeMacro{\ifchilddocmanual}
The main file should be prepared as usual, see \secref{sec:include}.
However, the document body must make a distinction
between processing of an individual part and of the main document, e.g.:
%
\begin{center}
\begin{tabular}{l}
|\ifchilddocmanual|\\
|\input{\childdocname}|\\
|\||else|\\
\textit{document body with }|\input{|\textit{part}|}|\\
|\||fi|
\end{tabular}
\end{center}
%
The conditional |\ifchilddocmanual| is true whenever
a part to be included by |\input| is being compiled,
and the name of the part is stored in |\childdocname|.

%%%%%%%%%%%%%%%%%%%%%%%%%%%%%%%%%%%%%%%%
\DescribeMacro{\childdocby}
Each part to be included by |\input| should start with:
%
\begin{center}
\begin{tabular}{l}
|\input{childdoc.def}|\\
|\childdocby{|\textit{main}|}|\\
\end{tabular}
\end{center}
%
The directive |\childdocby| is similar to |\childdocof|
described in \secref{sec:include},
but the subsequent selection of content must be done manually.
To that end, both |\ifchilddoc| and |\ifchilddocmanual|
will be true upon processing of a part,
and the name of the part is stored in |\childdocname|.
Note that |\jobname| will be set to the filename of the current part
so that each part receives an individual |.aux| file
that does not interfere with the |.aux| file(s) of the main document.
This behaviour can be altered by the alternative form
|\childdocby[*]{|\textit{main}|}| (with a non-empty optional argument)
which uses the |.aux| file of the main document
by setting |\jobname| to \textit{main}.

%%%%%%%%%%%%%%%%%%%%%%%%%%%%%%%%%%%%%%%%%%%%%%%%%%%%%%%%%%%%%%%%%%%%%%%%%%%%%%%%
\subsection{Driver Development}
\label{sec:driver}

The \textsf{childdoc} mechanism can also be use for the development
of definition files such as \LaTeX{} styles or classes.
This case differs from the above setup with multiple parts
included by |\include| in that no |\includeonly| should be invoked.
This can be achieved by starting the include file
(before |\ProvidesPackage|) with:
%
\begin{center}
\begin{tabular}{l}
|\input{childdoc.def}|\\
|\childdocforward{|\textit{main}|}|\\
\end{tabular}
\end{center}
%
or alternatively with:
%
\begin{center}
\begin{tabular}{l}
|\input{childdoc.def}|\\
|\childdocby{|\textit{main}|}|\\
\end{tabular}
\end{center}
%
Both forms have slightly different effects as described above.
The main file is prepared as usual, see \secref{sec:include}.

%%%%%%%%%%%%%%%%%%%%%%%%%%%%%%%%%%%%%%%%%%%%%%%%%%%%%%%%%%%%%%%%%%%%%%%%%%%%%%%%
\subsection{Legacy Detection}
\label{sec:detection}

The directive |\childdocmain| in the main file can detect
whether the complete document or merely a child is to be compiled
even without using the directive |\childdocof|.
This method is deprecated because it is less robust
and there is no compelling reason to use it;
it is merely provided for backward compatibility
and it may be removed in future versions.

If the detection mechanism is to be used,
it is mandatory to correctly specify
the filename of the main file as the argument of |\childdocmain|:
%
\begin{center}
\begin{tabular}{l}
|\input{childdoc.def}|\\
|\childdocmain{|\textit{main}|}|\\
\end{tabular}
\end{center}
%
If |\jobname| does not match the argument \textit{main} of |\childdocmain|,
it is assumed that |\jobname| points to the child file to be compiled.
When using |\childdocmain| with the main file specified as argument,
it suffices to start a child file
with just |\input{|\textit{main}|}|
without loading of the package and using |\childdocof|.
If instead all processing is done
with the appropriate \textsf{childdoc} directives,
the argument of \textit{main} of |\childdocmain| can be empty.

An alternative version of the command line processing described
in \secref{sec:commandline} using the detection mechanism reads:
%
\begin{center}
|... -jobname "|\textit{target}|" "|[\textit{flags}]%
[|\def\jobname{|\textit{dest}|}|]|\input{|\textit{main}|}"|
\end{center}

%%%%%%%%%%%%%%%%%%%%%%%%%%%%%%%%%%%%%%%%%%%%%%%%%%%%%%%%%%%%%%%%%%%%%%%%%%%%%%%%
\subsection{Manual Code}
\label{sec:manual}

In case one cannot be certain whether the definitions file |childdoc.def|
is installed on the target \TeX{} distribution
and one prefers not to ship it,
it is conceivable to paste a few relevant commands into the sources.

To that end, drop all statements |\input{childdoc.def}|
and perform the replacements as outlined below.
Instead of |\childdocmain{|\textit{main}|}| add the following code
to the top of the main file:
%
\begin{center}
\begin{tabular}{l}
|\||ifdefined\childdocname\endinput\||fi\newif\ifchilddoc|\\
|\edef\childdocname{\scantokens\expandafter{\jobname\noexpand}}|\\
|\def\childdocmain{|\textit{main}|}\||ifx\childdocmain\childdocname\||else|\\
|\childdoctrue\includeonly{\childdocname}\let\jobname\childdocmain\||fi|\\
\end{tabular}
\end{center}
%
Instead of |\childdocof{|\textit{main}|}| just include the main file
at the top of each child file:
%
\begin{center}
|\input{|\textit{main}|}|
\end{center}
%
A simple redirection |\childdocforward{|\textit{dest}|}| is achieved by:
%
\begin{center}
|\def\jobname{|\textit{dest}|}\input{\jobname}|
\end{center}
%
The redirection with prefix
|\childdocforwardprefix[|\textit{prefix}|]{|\textit{dest}|}|
is accomplished by:
%
\begin{center}
\begin{tabular}{l}
|{\edef\jobname{\scantokens\expandafter{\jobname\noexpand}}|\\
|\def\redirectjob |\textit{prefix}|#1~~~{\gdef\jobname{|\textit{dest}|#1}}|\\
|\expandafter\redirectjob\jobname~~~}\input{\jobname}|
\end{tabular}
\end{center}

In an alternative approach,
child documents can be compiled by a specific command line
without additional code or specific definitions:
%
\begin{center}
|... -jobname "|\textit{target}|" "|[\textit{flags}]%
|\includeonly{|\textit{dest}|}\input{|\textit{main}|}"|
\end{center}
%

%%%%%%%%%%%%%%%%%%%%%%%%%%%%%%%%%%%%%%%%%%%%%%%%%%%%%%%%%%%%%%%%%%%%%%%%%%%%%%%%
%%%%%%%%%%%%%%%%%%%%%%%%%%%%%%%%%%%%%%%%%%%%%%%%%%%%%%%%%%%%%%%%%%%%%%%%%%%%%%%%
\section{Information}

%%%%%%%%%%%%%%%%%%%%%%%%%%%%%%%%%%%%%%%%%%%%%%%%%%%%%%%%%%%%%%%%%%%%%%%%%%%%%%%%
\subsection{Copyright}

Copyright \copyright{} 2017--2018 Niklas Beisert

This work may be distributed and/or modified under the
conditions of the \LaTeX{} Project Public License, either version 1.3
of this license or (at your option) any later version.
The latest version of this license is in
  \url{http://www.latex-project.org/lppl.txt}
and version 1.3 or later is part of all distributions of \LaTeX{}
version 2005/12/01 or later.

This work has the LPPL maintenance status `maintained'.

The Current Maintainer of this work is Niklas Beisert.

This work consists of the files |README.txt|, |childdoc.ins| and |childdoc.dtx|
as well as the derived files |childdoc.def|, |cdocsamp.tex|
with |cdocsch1.tex|, |cdocsch2.tex|, |cdocspt3.tex|, |cdocspt4.tex|,
|cdocsdrf.tex|, |cdocsfn1.tex|, |cdocsfn2.tex|
as well as |childdoc.pdf|.

%%%%%%%%%%%%%%%%%%%%%%%%%%%%%%%%%%%%%%%%%%%%%%%%%%%%%%%%%%%%%%%%%%%%%%%%%%%%%%%%
\subsection{Files and Installation}

The package consists of the files:
%
\begin{center}
\begin{tabular}{ll}
    |README.txt|   & readme file \\
    |childdoc.ins| & installation file \\
    |childdoc.dtx| & source file \\
    |childdoc.def| & definition file \\
    |cdocsamp.tex| & sample main file \\
    |cdocsch1.tex| & sample include file \\
    |cdocsch2.tex| & sample include file \\
    |cdocspt3.tex| & sample part file \\
    |cdocspt4.tex| & sample part file \\
    |cdocsdrf.tex| & sample redirection file \\
    |cdocsfn1.tex| & sample redirection file \\
    |cdocsfn2.tex| & sample redirection file \\
    |childdoc.pdf| & manual
\end{tabular}
\end{center}
%
The distribution consists of the files
|README.txt|, |childdoc.ins| and |childdoc.dtx|.
%
\begin{itemize}
\item
Run (pdf)\LaTeX{} on |childdoc.dtx|
to compile the manual |childdoc.pdf| (this file).
\item
Run \LaTeX{} on |childdoc.ins| to create the definitions file |childdoc.def|
and the sample |cdocsamp.tex| with include files
|cdocsch1.tex|, |cdocsch2.tex|, |cdocspt3.tex|, |cdocspt4.tex|,
|cdocsdrf.tex|, |cdocsfn1.tex|, |cdocsfn2.tex|.
Then copy the file |childdoc.def| to an appropriate directory of your \LaTeX{}
distribution, e.g.\ \textit{texmf-root}|/tex/latex/childdoc|.
\end{itemize}

%%%%%%%%%%%%%%%%%%%%%%%%%%%%%%%%%%%%%%%%%%%%%%%%%%%%%%%%%%%%%%%%%%%%%%%%%%%%%%%%
\subsection{Related CTAN Packages}

There are several other packages which offer a similar functionality:
%
\begin{itemize}
\item
The packages
\href{http://ctan.org/pkg/docmute}{\textsf{docmute}},
\href{http://ctan.org/pkg/includex}{\textsf{includex}} and
\href{http://ctan.org/pkg/standalone}{\textsf{standalone}}
provide commands to include only the document body of
a child file thus allowing both files to be compiled individually.
\item
The packages \href{http://ctan.org/pkg/subdocs}{\textsf{subdocs}}
and \href{http://ctan.org/pkg/subfiles}{\textsf{subfiles}}
provide structures in which the main and child documents can be
encapsulated and allowing them to be compiled individually.
The inclusion mechanism is different from the conventional |\include|.
\item
The package \href{http://ctan.org/pkg/combine}{\textsf{combine}}
is an elaborate solution to combine several documents into one.
\end{itemize}
%
See also the CTAN topic \href{http://ctan.org/topic/subdocs}{\textsf{subdocs}}
for further related packages.
The present package differs from the above solutions in that
a document structure constructed with the conventional |\include| mechanism
just needs two extra commands at the top of every file
such that all constituent files can be compiled individually.

%%%%%%%%%%%%%%%%%%%%%%%%%%%%%%%%%%%%%%%%%%%%%%%%%%%%%%%%%%%%%%%%%%%%%%%%%%%%%%%%
%\subsection{Feature Suggestions}
%
%The following is a list of features which may be useful for future
%versions of this package:
%%
%\begin{itemize}
%\item
%\ldots
%\end{itemize}

%%%%%%%%%%%%%%%%%%%%%%%%%%%%%%%%%%%%%%%%%%%%%%%%%%%%%%%%%%%%%%%%%%%%%%%%%%%%%%%%
\subsection{Revision History}

%%%%%%%%%%%%%%%%%%%%%%%%%%%%%%%%%%%%%%%%
\paragraph{v2.0:} 2018/12/30

\begin{itemize}
\item
immediate forward processing
\item
added |\childdocby| mechanism
\item
manual restructured
\end{itemize}

%%%%%%%%%%%%%%%%%%%%%%%%%%%%%%%%%%%%%%%%
\paragraph{v1.6:} 2018/01/17

\begin{itemize}
\item
application for development of include files
\item
corrections to manual
\end{itemize}

%%%%%%%%%%%%%%%%%%%%%%%%%%%%%%%%%%%%%%%%
\paragraph{v1.5:} 2017/05/21

\begin{itemize}
\item
more complete structuring introduced
\item
|\childdocof| introduced
\item
|\childdoc| renamed to |\childdocmain|
\item
|\childredirect| renamed to |\childdocforward| and |\childdocforwardprefix|
and functionality expanded
\end{itemize}

%%%%%%%%%%%%%%%%%%%%%%%%%%%%%%%%%%%%%%%%
\paragraph{v1.0:} 2017/04/27

\begin{itemize}
\item
manual and install package
\item
first version published on CTAN
\end{itemize}

%%%%%%%%%%%%%%%%%%%%%%%%%%%%%%%%%%%%%%%%
\paragraph{v0.6:} 2017/04/26

\begin{itemize}
\item
redirection mechanism added
\end{itemize}

%%%%%%%%%%%%%%%%%%%%%%%%%%%%%%%%%%%%%%%%
\paragraph{v0.5:} 2017/04/26

\begin{itemize}
\item
functionality in definition file
\end{itemize}


%%%%%%%%%%%%%%%%%%%%%%%%%%%%%%%%%%%%%%%%%%%%%%%%%%%%%%%%%%%%%%%%%%%%%%%%%%%%%%%%
%%%%%%%%%%%%%%%%%%%%%%%%%%%%%%%%%%%%%%%%%%%%%%%%%%%%%%%%%%%%%%%%%%%%%%%%%%%%%%%%
%%%%%%%%%%%%%%%%%%%%%%%%%%%%%%%%%%%%%%%%%%%%%%%%%%%%%%%%%%%%%%%%%%%%%%%%%%%%%%%%
\appendix

\settowidth\MacroIndent{\rmfamily\scriptsize 000\ }

 \DocInput{childdoc.dtx}

\end{document}
%</driver>
% \fi
%
% %%%%%%%%%%%%%%%%%%%%%%%%%%%%%%%%%%%%%%%%%%%%%%%%%%%%%%%%%%%%%%%%%%%%%%%%%%%%%%
% %%%%%%%%%%%%%%%%%%%%%%%%%%%%%%%%%%%%%%%%%%%%%%%%%%%%%%%%%%%%%%%%%%%%%%%%%%%%%%
% \section{Sample}
%\iffalse
%<*samplemain>
%\fi
%
% The following presents a sample document
% with two chapters, two parts, a title page,
% a compile flag as well as three forwarding files to set the flag.
% It consists of eight |.tex| files:
% \begin{center}
% \begin{tabular}{ll}
% |cdocsamp.tex|&main file\\
% |cdocsch1.tex|&include file for chapter 1\\
% |cdocsch2.tex|&include file for chapter 2\\
% |cdocspt3.tex|&include file for part 3\\
% |cdocspt4.tex|&include file for part 4\\
% |cdocsdrf.tex|&forwarding file for main file in draft mode\\
% |cdocsfi1.tex|&forwarding file for final version of chapter 1\\
% |cdocsfi2.tex|&forwarding file for final version of chapter 2\\
% \end{tabular}
% \end{center}
% Each of the eight files can be compiled directly by the \LaTeX{} compiler.
%
% %%%%%%%%%%%%%%%%%%%%%%%%%%%%%%%%%%%%%%
% \paragraph{Main File.}
%
% The main file is called |cdocsamp.tex|.
%
% Load the \textsf{childdoc} definitions and
% declare the filename for the main document:
%    \begin{macrocode}
\input{childdoc.def}
\childdocmain{}
%    \end{macrocode}

% Optional override for |\version| flag:
%    \begin{macrocode}
%%\ifchilddoc\else\providecommand{\version}{draft}\fi
%    \end{macrocode}

% Define the default values for the |\version| flag
% (|final| for the main file and |draft| for childs):
%    \begin{macrocode}
\ifchilddoc
\providecommand{\version}{draft}
\else
\providecommand{\version}{final}
\fi
%    \end{macrocode}

% Load the standard document class:
%    \begin{macrocode}
\documentclass[12pt]{article}
%    \end{macrocode}

% Start the document body:
%    \begin{macrocode}
\begin{document}
%    \end{macrocode}

% Declare a title page.
% Print title, part of document being processed and version flag:
%    \begin{macrocode}
\addtocounter{page}{-1}
\begin{center}
{\LARGE\bfseries{}childdoc example\par}
\vspace{1cm}
\ifchilddoc
\ifchilddocmanual part\else chapter\fi:
`\childdocname' of `\childdocjob'\par
\else
main document: `\childdocjob'\par
\fi
version: \version\par
\end{center}
\newpage
%    \end{macrocode}

% Manually include selected file,
% otherwise process as usual:
%    \begin{macrocode}
\ifchilddocmanual
\section*{part `\childdocname'}
\input{\childdocname}
\else
%    \end{macrocode}

% Include the two chapters:
%    \begin{macrocode}
\include{cdocsch1}
\include{cdocsch2}
%    \end{macrocode}

% Include the two parts unless only chapters should be displayed:
%    \begin{macrocode}
\ifchilddoc\else
\section{part three}
\input{cdocspt3}
\section{part four}
\input{cdocspt4}
\fi
%    \end{macrocode}

% Process as usual until here:
%    \begin{macrocode}
\fi
%    \end{macrocode}

% End of document body:
%    \begin{macrocode}
\end{document}
%    \end{macrocode}
%\iffalse
%</samplemain>
%\fi
%
% %%%%%%%%%%%%%%%%%%%%%%%%%%%%%%%%%%%%%%
% \paragraph{Chapter Include Files.}
%
% The include files are called |cdocsch1.tex| and |cdocsch2.tex|.
%
%\iffalse
%<*samplechap1|samplechap2>
%\fi

% Optional override for |\version| flag:
%    \begin{macrocode}
%%\providecommand{\version}{final}
%    \end{macrocode}

% Include the main document:
%    \begin{macrocode}
\input{childdoc.def}
\childdocof{cdocsamp}
%    \end{macrocode}

%\iffalse
%</samplechap1|samplechap2>
%\fi
%
%\iffalse
%<*samplechap1>
%\fi
% Some text for chapter 1:
%    \begin{macrocode}
\section{one}
some text in chapter one
%    \end{macrocode}

%\iffalse
%</samplechap1>
%\fi
% Some text for chapter 2:
%\iffalse
%<*samplechap2>
%\fi
%    \begin{macrocode}
\section{two}
more text in chapter two
%    \end{macrocode}

%\iffalse
%</samplechap2>
%\fi
%
% %%%%%%%%%%%%%%%%%%%%%%%%%%%%%%%%%%%%%%
% \paragraph{Part Include Files.}
%
% The include files are called |cdocspt3.tex| and |cdocspt4.tex|.
%
%\iffalse
%<*samplepart3|samplepart4>
%\fi

% Optional override for |\version| flag:
%    \begin{macrocode}
%%\providecommand{\version}{final}
%    \end{macrocode}

% Include the main document:
%    \begin{macrocode}
\input{childdoc.def}
\childdocby{cdocsamp}
%    \end{macrocode}

%\iffalse
%</samplepart3|samplepart4>
%\fi
%
%\iffalse
%<*samplepart3>
%\fi
% Some text for part 3:
%    \begin{macrocode}
some text in part three
%    \end{macrocode}

%\iffalse
%</samplepart3>
%\fi
% Some text for part 4:
%\iffalse
%<*samplepart4>
%\fi
%    \begin{macrocode}
more text in part four
%    \end{macrocode}

%\iffalse
%</samplepart4>
%\fi
%
% %%%%%%%%%%%%%%%%%%%%%%%%%%%%%%%%%%%%%%
% \paragraph{Forwarding for a Complete Draft.}
%
% The following forwarding file |cdocsdrf.tex|
% compiles the main document in draft mode:
%\iffalse
%<*sampledraft>
%\fi
%    \begin{macrocode}
\def\version{draft}
\input{childdoc.def}
\childdocforward{cdocsamp}
%    \end{macrocode}

%\iffalse
%</sampledraft>
%\fi
%
% %%%%%%%%%%%%%%%%%%%%%%%%%%%%%%%%%%%%%%
% \paragraph{Forwarding for Final Version of the Chapters.}
%
% The following forwarding files |cdocsfn1.tex| and |cdocsfn2.tex|
% (with identical content)
% compile the final versions of the child documents
% |cdocsch1.tex| and |cdocsch2.tex|, respectively:
%\iffalse
%<*samplefinal>
%\fi
%    \begin{macrocode}
\def\version{final}
\input{childdoc.def}
\childdocforwardprefix[cdocsamp]{cdocsfn}{cdocsch}
%    \end{macrocode}

%\iffalse
%</samplefinal>
%\fi
%
% %%%%%%%%%%%%%%%%%%%%%%%%%%%%%%%%%%%%%%
% \paragraph{Command Line Processing.}
%
% The following three command lines generate the output files
% |cdocscld|, |cdocscl1| and |cdocscl2|
% which should be identical to
% |cdocsdrf|, |cdocsch1| and |cdocsfn2|, respectively:
% \begin{center}
% \begin{tabular}{l}
% |latex -jobname cdocscld \|\\
% |  "\def\version{draft}\input{childdoc.def}\childdocforward{cdocsamp}"|\\
% |latex -jobname cdocscl1 \|\\
% |  "\input{childdoc.def}\childdocforward[cdocsamp]{cdocsch1}"|\\
% |latex -jobname cdocscl2 \|\\
% |  "\def\version{final}\input{childdoc.def}\childdocforward{cdocsch2}"|
% \end{tabular}
% \end{center}
% Note that the trailing backslash on each first line
% merely continues the input to the second line
% (for convenient cut ant paste).
% Furthermore, the command |latex| can be replaced by any
% of its alternative versions such as |pdflatex|.
%
% %%%%%%%%%%%%%%%%%%%%%%%%%%%%%%%%%%%%%%%%%%%%%%%%%%%%%%%%%%%%%%%%%%%%%%%%%%%%%%
% %%%%%%%%%%%%%%%%%%%%%%%%%%%%%%%%%%%%%%%%%%%%%%%%%%%%%%%%%%%%%%%%%%%%%%%%%%%%%%
% \section{Implementation}
%\iffalse
%<*package>
%\fi
%
% This section describes the definitions file |childdoc.def|.

% The definitions cannot be loaded using |\usepackage| or |\RequirePackage|
% which has a mechanism to prevent loading a style file more than once.
% When loading the definitions by means of |\input|
% multiple instances have to be prevented manually:
%\iffalse
%This code needs to be before the `\ProvidesFile' directive
%which is defined at the beginning of this file.
%Therefore it is also placed there and commented out here.
%</package>
%<*discard>
%\fi
%    \begin{macrocode}
\ifdefined\childdocmain\endinput\fi
%    \end{macrocode}
%\iffalse
%</discard>
%<*package>
%\fi
%
% \macro{\ifchilddoc}
% \macro{\ifchilddocmanual}
% The conditional |\ifchilddoc| tells whether a
% child (true) or main (false) document is being compiled.
% The conditional |\ifchilddocmanual| tells whether
% the |\includeonly| mechanism is used (false) or
% the selection of child files must be performed manually (true).
% The definitions initialise to false:
%    \begin{macrocode}
\newif\ifchilddoc
\newif\ifchilddocmanual
%    \end{macrocode}

% \macro{\childdocname}
% \macro{\childdocjob}
% The macro |\childdocname| stores the name of the main document
% to be compiled. The macro |\childdocjob| stores the name of
% the document on which the \LaTeX{} compiler was originally invoked.
% The content of |\jobname| cannot be compared
% to filenames specified in the source due to different catcodes.
% The following code rescans |\jobname|, stores the result
% in |\childdocname| and saves a copy in |\childdocjob|:
%    \begin{macrocode}
\edef\childdocname{\scantokens\expandafter{\jobname\noexpand}}
\let\childdocjob\childdocname
%    \end{macrocode}

% \macro{\childdocdisable}
% The macro |\childdocdisable| prevents the main file
% from being processed more than once.
% At this stage, the main document command |\childdocmain|
% is assumed to be called once again where it should do nothing.
% Any subsequent call to it should prevent
% a secondary processing of the main document
% It overwrites the forwarding commands
% |\childdocof| and |\childdocforward|
% with empty macros to prevent further inclusions of the main document:
%    \begin{macrocode}
\newcommand{\childdocdisable}
{
  \renewcommand{\childdocmain}[1]{\renewcommand{\childdocmain}[1]{\endinput}}
  \renewcommand{\childdocof}[1]{}
  \renewcommand{\childdocby}[2][]{}
  \renewcommand{\childdocforward}[2][]{}
  \renewcommand{\childdocdisable}{}
}
%    \end{macrocode}

% \macro{\childdocmain}
% The macro |\childdocmain| is to be called at the top of the main file
% with nothing or the main filename (without extension) as argument.
% First, it breaks loops.
% If the argument is not empty and does not match |\childdocname|
% (which is set by the first inclusion of |childdoc.def|),
% |\ifchilddoc| is set to true, |\includeonly| is applied to the child file
% and |\jobname| is set to the main file
% (for proper handling of |.aux| files):
%    \begin{macrocode}
\newcommand{\childdocmain}[1]
{
  \childdocdisable\childdocmain{}
  \if?#1?\else
    \begingroup
      \def\childdoctmp{#1}
      \ifx\childdoctmp\childdocname
        \def\childdoctmp{}
      \else
        \def\childdoctmp
        {
          \childdoctrue
          \includeonly{\childdocname}
          \def\childdocjob{#1}
          \def\jobname{#1}
        }
      \fi
      \expandafter
    \endgroup
    \childdoctmp
  \fi
}
%    \end{macrocode}

% \macro{\childdocof}
% The command |\childdocof| redirects
% compilation to the main file |#1|.
%    \begin{macrocode}
\newcommand{\childdocof}[1]
{
  \childdocdisable
  \childdoctrue
  \includeonly{\childdocname}
  \def\jobname{#1}
  \def\childdocjob{#1}
  \input{#1}
}
%    \end{macrocode}

% \macro{\childdocby}
% The command |\childdocby| ....
%    \begin{macrocode}
\newcommand{\childdocby}[2][]
{
  \childdocdisable
  \childdoctrue
  \childdocmanualtrue
  \if?#1?\else
    \def\jobname{#2}
  \fi
  \def\childdocjob{#2}
  \input{#2}
  \endinput
}
%    \end{macrocode}

% \macro{\childdocforward}
% The command |\childdocforward| redirects
% compilation to the main file or
% (if the optional argument is given) a child file.
% Parameters are set as if the main file
% or a child file starting with |\childdocof| was compiled.
% Then compilation is handed over to the main file:
%    \begin{macrocode}
\newcommand{\childdocforward}[2][]
{
  \begingroup
    \if?#1?
      \def\childdoctmp
      {
        \def\childdocname{#2}
        \def\childdocjob{#2}
        \def\jobname{#2}
        \input{#2}
        \endinput
      }
    \else
      \def\childdoctmp
      {
        \childdocdisable
        \def\childdocname{#2}
        \childdoctrue
        \includeonly{#2}
        \def\childdocjob{#1}
        \def\jobname{#1}
        \input{#1}
        \endinput
      }
    \fi
    \expandafter
  \endgroup
  \childdoctmp
}
%    \end{macrocode}

% \macro{\childdocforwardprefix}
% The command |\childdocforwardprefix| redirects
% compilation to the main or a child file by means of a pattern.
% The prefix |#1| in the current filename is replaced by |#2|
% and the suffix of the current filename is kept
% (it is assumed that the filename does not contain the substring `|~~~|'
% which is used as a delimiter).
% Compilation is handed over to the new file by |\childdocforward|:
%    \begin{macrocode}
\newcommand{\childdocforwardprefix}[3][]
{
  \begingroup
    \def\childdocextract #2##1~~~{\def\childdoctmp{\childdocforward[#1]{#3##1}}}
    \expandafter\childdocextract\childdocname~~~
    \expandafter
  \endgroup
  \childdoctmp
}
%    \end{macrocode}

% \macro{\childdoc}
% The deprecated macro |\childdoc| is a legacy version of |\childdocmain|:
%    \begin{macrocode}
\newcommand{\childdoc}{\childdocmain}
%    \end{macrocode}

% \macro{\childdocredirect}
% The deprecated macro |\childdocredirect| is a legacy version
% of |\childdocforward| and |\childdocforwardprefix|:
%    \begin{macrocode}
\newcommand{\childdocredirect}[2][]
{
  \begingroup
    \if?#1?
      \def\childdoctmp{\childdocforward{#2}}
    \else
      \def\childdoctmp{\childdocforwardprefix{#1}{#2}}
    \fi
    \expandafter
  \endgroup
  \childdoctmp
}
%    \end{macrocode}

%\iffalse
%</package>
%\fi
%
\endinput
|\\
|\childdocforward[|\textit{main}|]{|\textit{dest}|}|\\
\end{tabular}
\end{center}
%
The argument \textit{dest} is the destination file
(without extension).
It should be the main file or one of the child files.
Note that further \textsf{childdoc} directives
such as |\childdocof| and |\childdocforward|
in the indicated file will be processed in this form.
The optional argument \textit{main}
passes on directly to the main file \textit{main}
while pretending to compile the child \textit{dest}.
This form behaves as if \textit{dest}
issues |\childdocof{|\textit{main}|}| right away,
and no further \textsf{childdoc} directives will be processed.

%%%%%%%%%%%%%%%%%%%%%%%%%%%%%%%%%%%%%%%%
\DescribeMacro{\...prefix}
In the alternative form |\childdocforwardprefix|,
%
\begin{center}
\begin{tabular}{l}
|% \iffalse
%
% childdoc.dtx Copyright (C) 2017-2018 Niklas Beisert
%
% This work may be distributed and/or modified under the
% conditions of the LaTeX Project Public License, either version 1.3
% of this license or (at your option) any later version.
% The latest version of this license is in
%   http://www.latex-project.org/lppl.txt
% and version 1.3 or later is part of all distributions of LaTeX
% version 2005/12/01 or later.
%
% This work has the LPPL maintenance status `maintained'.
%
% The Current Maintainer of this work is Niklas Beisert.
%
% This work consists of the files childdoc.dtx and childdoc.ins
% and the derived files childdoc.def and cdocsamp.tex with
% cdocsch1.tex, cdocsch2.tex, cdocsdrf.tex, cdocsfn1.tex, cdocsfn2.tex.
%
%<package>\ifdefined\childdocmain\endinput\fi
%<package>\ProvidesFile{childdoc.def}[2018/12/30 v2.0 child document driver]
%<samplemain>\ProvidesFile{cdocsamp.tex}[2018/12/30 v2.0 sample for childdoc]
%<*driver>
%\ProvidesFile{childdoc.drv}[2018/12/30 v2.0 childdoc reference manual file]
\PassOptionsToClass{10pt,a4paper}{article}
\documentclass{ltxdoc}

\usepackage[margin=35mm]{geometry}
\usepackage{hyperref}
\usepackage{hyperxmp}
\usepackage[usenames]{color}

\hypersetup{colorlinks=true}
\hypersetup{pdfstartview=FitH}
\hypersetup{pdfpagemode=UseNone}
\hypersetup{pdfsource={}}
\hypersetup{pdflang={en-UK}}
\hypersetup{pdfcopyright={Copyright 2017-2018 Niklas Beisert.
  This work may be distributed and/or modified under the
  conditions of the LaTeX Project Public License, either version 1.3
  of this license or (at your option) any later version.}}
\hypersetup{pdflicenseurl={http://www.latex-project.org/lppl.txt}}
\hypersetup{pdfcontactaddress={ETH Zurich, ITP, HIT K,
  Wolfgang-Pauli-Strasse 27}}
\hypersetup{pdfcontactpostcode={8093}}
\hypersetup{pdfcontactcity={Zurich}}
\hypersetup{pdfcontactcountry={Switzerland}}
\hypersetup{pdfcontactemail={nbeisert@itp.phys.ethz.ch}}
\hypersetup{pdfcontacturl={http://people.phys.ethz.ch/\xmptilde nbeisert/}}

\newcommand{\secref}[1]{\hyperref[#1]{section \ref*{#1}}}

\parskip1ex
\parindent0pt
\let\olditemize\itemize
\def\itemize{\olditemize\parskip0pt}

\begin{document}

\title{The \textsf{childdoc} Package}
\hypersetup{pdftitle={The childdoc Package}}
\author{Niklas Beisert\\[2ex]
  Institut f\"ur Theoretische Physik\\
  Eidgen\"ossische Technische Hochschule Z\"urich\\
  Wolfgang-Pauli-Strasse 27, 8093 Z\"urich, Switzerland\\[1ex]
  \href{mailto:nbeisert@itp.phys.ethz.ch}
  {\texttt{nbeisert@itp.phys.ethz.ch}}}
\hypersetup{pdfauthor={Niklas Beisert}}
\hypersetup{pdfsubject={Manual for the LaTeX2e Package childdoc}}
\date{30 December 2018, \textsf{v2.0}}
\maketitle

\begin{abstract}\noindent
\textsf{childdoc} is a \LaTeXe{} package
that enables the direct compilation
of document sections included by |\include|
to individual files.
\end{abstract}

\begingroup
\parskip0ex
\tableofcontents
\endgroup

%%%%%%%%%%%%%%%%%%%%%%%%%%%%%%%%%%%%%%%%%%%%%%%%%%%%%%%%%%%%%%%%%%%%%%%%%%%%%%%%
%%%%%%%%%%%%%%%%%%%%%%%%%%%%%%%%%%%%%%%%%%%%%%%%%%%%%%%%%%%%%%%%%%%%%%%%%%%%%%%%
\section{Introduction}

\LaTeX{} provides a mechanism to structure a large document (such as a book)
into a main file and several child files (containing the chapters)
using the |\include| command.
This mechanism is beneficial for documents
which span hundreds of pages in order to
make the source file(s) more manageable.
Moreover, compilation can be restricted to
selected child files by means of the |\includeonly| command.
The latter feature can be used to reduce the compilation time while editing
(this was significantly more useful in the earlier days of \LaTeX{})
or to generate a smaller document which is easier to navigate.
Another application of |\includeonly| is to generate
documents consisting of selected parts of the complete document.

However, there are a few drawbacks of the plain |\include| mechanism:
\begin{itemize}
\item
The child files cannot be compiled on their own,
they can only be compiled via the main file.
A naive editing environment
(such as a text editor with an option
to have the current file processed by \LaTeX)
may require one to switch to the main file before compiling;
attempting to compile the child file produces errors.
\item
The main file must be modified (each time)
to adjust the |\includeonly| command
to the present needs. This easily leaves the main file in a messy state.
\item
The generated document will always carry the filename
of the main document. This is inconvenient if
several child files are to be compiled and
to be kept for distribution.
\end{itemize}

The present package provides a simple interface
to make child files individually compilable by \LaTeX{}.
Compiling a child file then has the same effect as compiling
the main file with an |\includeonly| command
to select the appropriate child.
Moreover the generated document will carry the name of the child
rather than the main file.
This resolves all three above issues.

This feature is meant to make the editing of books,
thesis documents and lecture notes somewhat more convenient.
However, the package can also be used efficiently for
composing a series of documents (such as exercise sheets)
which are typically distributed individually.
It then assists the author in generating the individual documents
(potentially in different versions)
as well as a document containing the collected series.
Another application is in developing style files
or other kinds of included material
where compilation of the style file could redirect
to a sample or test file.

%%%%%%%%%%%%%%%%%%%%%%%%%%%%%%%%%%%%%%%%%%%%%%%%%%%%%%%%%%%%%%%%%%%%%%%%%%%%%%%%
%%%%%%%%%%%%%%%%%%%%%%%%%%%%%%%%%%%%%%%%%%%%%%%%%%%%%%%%%%%%%%%%%%%%%%%%%%%%%%%%
\section{Usage}

First of all, the package \textsf{childdoc} is \emph{not} a standard
\LaTeXe{} |.sty| style file! Therefore it needs to be invoked in
a non-standard way.

%%%%%%%%%%%%%%%%%%%%%%%%%%%%%%%%%%%%%%%%%%%%%%%%%%%%%%%%%%%%%%%%%%%%%%%%%%%%%%%%
\subsection{Included Files}
\label{sec:include}

%%%%%%%%%%%%%%%%%%%%%%%%%%%%%%%%%%%%%%%%
\DescribeMacro{\childdocmain}
To use the package, add the commands
\begin{center}
\begin{tabular}{l}
|\input{childdoc.def}|\\
|\childdocmain{}|\\
\end{tabular}
\end{center}
at the very top of the main \LaTeX{} file,
in particular \emph{before} the |\documentclass| statement!
The argument of |\childdocmain| should be left empty
(but it must be present).

%%%%%%%%%%%%%%%%%%%%%%%%%%%%%%%%%%%%%%%%
\DescribeMacro{\childdocof}
Furthermore, add the commands
\begin{center}
\begin{tabular}{l}
|\input{childdoc.def}|\\
|\childdocof{|\textit{main}|}|\\
\end{tabular}
\end{center}
at the top of every child file \textit{child}
which is included by |\include{|\textit{child}|}|
from within the main file
(or at least for those files to be compiled individually).
The argument \textit{main} must be the filename of the main file.

There are a couple of
considerations in setting up the main and child documents:

%%%%%%%%%%%%%%%%%%%%%%%%%%%%%%%%%%%%%%%%
\paragraph{Restrictions.}

Please note the following restrictions:
\begin{itemize}
\item
|\childdocmain| must be called with one argument \textit{main}
to ensure compatibility with earlier version of the package.
It must either be empty (|\childdocmain{}|)
or precisely match the filename of the main file in which it is specified.
See \secref{sec:detection} for further information.
\item
The filename \textit{main} must be specified without the |.tex| extension.
\item
The filename \textit{main} is case sensitive
(even in case-insensitive file systems)
due to internal string comparison.
\item
The argument \textit{main} should be fully expanded, it cannot be a macro.
\item
Subdirectories and special characters should be avoided in filenames.
\item
The command |\childdocmain{|\textit{main}|}| must be followed by a whitespace.
It should not be followed immediately by another command
or by a comment mark `|%|'.
This is because the \TeX{} parser reads the token immediately following
the argument of |\childdocmain| and puts it
at the beginning of every child section;
however, a white\-space is ignored.
\end{itemize}

%%%%%%%%%%%%%%%%%%%%%%%%%%%%%%%%%%%%%%%%
\paragraph{Content of Main File.}

It is advisable to place all content in the child files included by |\include|.
Any output contained in the main file will appear in all child documents
unless suppressed manually;
it cannot be suppressed automatically by the |\includeonly| directive
and thus should normally be avoided.
A method to include some content in the main file
by means of conditional processing is described in \secref{sec:conditional}.

%%%%%%%%%%%%%%%%%%%%%%%%%%%%%%%%%%%%%%%%
\paragraph{Page Numbering.}

When only a part of the document is compiled,
the appropriate numbering of pages
(as well as other status parameters)
is determined from the |.aux| files.
The latter contain information from previous passes.
However this information needs to propagate through
all intermediate child documents.
Therefore the page numbering in child documents may well
be inconsistent until the complete document is compiled at least once.

A useful (if unconventional) way to always ensure a consistent
page numbering is to restart the numbering in each child document
and denote the pages by `\textit{child}|.|\textit{page}'
where \textit{child} represents the chapter/section number of the child file.
This can be achieved by the command
|\numberwithin{page}{|\textit{child}|}|
of the \textsf{amsmath} package
where \textit{child} can be |chapter| or |section|
depending on the chosen structuring.
Alternatively, one can modify the macro |\thepage| appropriately
and reset the counter |page| at the start of each child file.

%%%%%%%%%%%%%%%%%%%%%%%%%%%%%%%%%%%%%%%%%%%%%%%%%%%%%%%%%%%%%%%%%%%%%%%%%%%%%%%%
\subsection{Conditional Processing}
\label{sec:conditional}

The package provides a mechanism to compile different versions
of a document. To customise the versions further some conditional processing
can come in handy to distinguish which version is being compiled.
The package provides two macros to describe the compilation context:

%%%%%%%%%%%%%%%%%%%%%%%%%%%%%%%%%%%%%%%%
\DescribeMacro{\ifchilddoc}
The conditional |\ifchilddoc| distinguishes between the compilation of
child documents and the main document:
%
\begin{center}
|\ifchilddoc |\textit{child-code}| |[|\||else |\textit{main-code}]| \||fi|
\end{center}

%%%%%%%%%%%%%%%%%%%%%%%%%%%%%%%%%%%%%%%%
\DescribeMacro{\childdocname}
\DescribeMacro{\childdocjob}
The macro |\childdocname| contains the filename (without extension)
of the main or child file being processed.
Note that |\childdocjob| will always contain the name of the main file.

%%%%%%%%%%%%%%%%%%%%%%%%%%%%%%%%%%%%%%%%
\paragraph{Title Page.}

Conditional processing can be used to include a title or banner page
in the main document when proper precautions are taken.
Importantly, the code in the main file should ensure that the page counter
(as well as other status parameters which are stored in the |.aux| files)
takes the same value after the conditional processing.
Otherwise the page numbers may take divergent values
depending on which part is compiled.

For example, a title page could be declared by:
%
\begin{center}
\begin{tabular}{l}
|\ifchilddoc\||else|\\
|\addtocounter{page}{-1}|\\
\textit{code for title page}\\
|\newpage|\\
|\||fi|
\end{tabular}
\end{center}
%
A banner page for the child documents can be generated by:
%
\begin{center}
\begin{tabular}{l}
|\ifchilddoc|\\
|\addtocounter{page}{-1}|\\
\textit{code for banner page}\\
|\newpage|\\
|\||fi|
\end{tabular}
\end{center}
%
Here one could write a message such as:
\begin{center}
|This is the part \childdocname{} of \childdocjob{}.|
\end{center}

%%%%%%%%%%%%%%%%%%%%%%%%%%%%%%%%%%%%%%%%%%%%%%%%%%%%%%%%%%%%%%%%%%%%%%%%%%%%%%%%
\subsection{Flags}
\label{sec:flags}

The package makes it easy to generate different versions
of the main or child documents.
To this end compilation flags can be defined
and assigned different default values.
They will be particularly useful in conjunction
with the forwarding mechanism described in \secref{sec:forward}.

For example, it may be useful to have a flag |\version|
which can be set to |draft| or |final|.
The document source will contain some conditional code
depending on the value of |\version|.
Suppose further, the flag should default to |final| for the main file
and to |draft| for child files
which is a natural assignment for editing the document.
This is achieved by placing the following code
in the preamble of the main document
(below the |\childdocmain| directive):
%
\begin{center}
\begin{tabular}{l}
|\ifchilddoc|\\
|\providecommand{\version}{draft}|\\
|\||else|\\
|\providecommand{\version}{final}|\\
|\||fi|
\end{tabular}
\end{center}
%
The definition by |\providecommand| makes sure
that previous definitions are not overwritten.
Further statements |\providecommand{\version}{...}|
can thus be added before the above code to override it.

For the main file, one might add a line
(between |\childdocmain| and the above block)
%
\begin{center}
|%\ifchilddoc\||else\providecommand{\version}{draft}\||fi|
\end{center}
%
which can be uncommented to produce a draft version.
Likewise one can add a line to the very top of a child file
(above the |\childdocof{|\textit{main}|}| directive)
%
\begin{center}
|%\providecommand{\version}{final}|
\end{center}
%
which can be uncommented to produce the final version of this child document.

%%%%%%%%%%%%%%%%%%%%%%%%%%%%%%%%%%%%%%%%%%%%%%%%%%%%%%%%%%%%%%%%%%%%%%%%%%%%%%%%
\subsection{Forwarding}
\label{sec:forward}

Different versions of the main or child documents
using compilation flags as described in \secref{sec:flags}
can be (permanently) stored in different files
for convenient compilation, viewing and distribution.
To this end, the package defines a command
to pass on compilation to a different file:

%%%%%%%%%%%%%%%%%%%%%%%%%%%%%%%%%%%%%%%%
\DescribeMacro{\childdocforward}
The command |\childdocforward| redirects processing to
another source file:
%
\begin{center}
\begin{tabular}{l}
|\input{childdoc.def}|\\
|\childdocforward[|\textit{main}|]{|\textit{dest}|}|\\
\end{tabular}
\end{center}
%
The argument \textit{dest} is the destination file
(without extension).
It should be the main file or one of the child files.
Note that further \textsf{childdoc} directives
such as |\childdocof| and |\childdocforward|
in the indicated file will be processed in this form.
The optional argument \textit{main}
passes on directly to the main file \textit{main}
while pretending to compile the child \textit{dest}.
This form behaves as if \textit{dest}
issues |\childdocof{|\textit{main}|}| right away,
and no further \textsf{childdoc} directives will be processed.

%%%%%%%%%%%%%%%%%%%%%%%%%%%%%%%%%%%%%%%%
\DescribeMacro{\...prefix}
In the alternative form |\childdocforwardprefix|,
%
\begin{center}
\begin{tabular}{l}
|\input{childdoc.def}|\\
|\childdocforwardprefix[|\textit{main}|]{|\textit{prefix}|}{|\textit{dest}|}|
\end{tabular}
\end{center}
%
the destination file is determined by a pattern
depending on the current file:
To make this work, the current file must be called
`{\textit{prefix}\hspace{0.2em}\textit{suffix}}'
with \textit{prefix} matching precisely the argument.
Processing is then passed on to the file
`{\textit{dest}\hspace{0.2em}\textit{suffix}}'.
Surely, the same effect is achieved by
directly specifying the
argument `{\textit{dest}\hspace{0.2em}\textit{suffix}}'
in the first form.
However, that requires to set up a different file
for each child. With the alternative form of the command
all these files can have exactly the same content
which simplifies setting them up and maintaining them.

For example, the following file |draft.tex|
with a compilation flag |\version| as described in \secref{sec:flags}
compiles the main document as a draft:
%
\begin{center}
\begin{tabular}{l}
|\def\version{draft}|\\
|\input{childdoc.def}|\\
|\childdocforward{|\textit{main}|}|
\end{tabular}
\end{center}
%
Likewise, the following files |final|\textit{nn}|.tex|
compile the final version of the child document
|child|\textit{nn}|.tex|:
%
\begin{center}
\begin{tabular}{l}
|\def\version{final}|\\
|\input{childdoc.def}|\\
|\childdocforwardprefix{final}{child}|
\end{tabular}
\end{center}
%

Note that when several versions of a main file and/or of each child file
are to be generated, it may be convenient to set up a |Makefile| or
shell script to automatise the process.

%%%%%%%%%%%%%%%%%%%%%%%%%%%%%%%%%%%%%%%%%%%%%%%%%%%%%%%%%%%%%%%%%%%%%%%%%%%%%%%%
\subsection{Command Line Processing}
\label{sec:commandline}

The effect of redirection files can also be achieved by invoking
the \LaTeX{} compiler with a more elaborate command line.
Most conveniently this should be done as part
of a shell script or a |Makefile|.

When using \textsf{childdoc} in the main file, the following
command lines effectively perform a redirection
(note that depending on the shell being used,
backslashes may have to be doubled: `|\|' $\to$ `|\\|'):
%
\begin{center}
|... -jobname "|\textit{target}|" |\\|"|[\textit{flags}]%
|\input{childdoc.def}\childdocforward[|\textit{main}|]{|\textit{dest}|}"|
\end{center}
%
Here \textit{target} is the name of the output file,
\textit{main} is the name of the main file
and \textit{dest} is the name of the main or child file to be processed
(all filenames without extensions).
The optional argument \textit{main} can be omitted
if \textit{main} matches \textit{dest}.
Optionally, compilation \textit{flags} can be defined via |\def| commands.
This command line makes the \TeX{} engine believe
it is compiling the file \textit{target}
whose content is specified as the latter parameter.
The provided code then forwards the processing to
\textit{main} or \textit{dest} as described in \secref{sec:forward}.

%%%%%%%%%%%%%%%%%%%%%%%%%%%%%%%%%%%%%%%%%%%%%%%%%%%%%%%%%%%%%%%%%%%%%%%%%%%%%%%%
\subsection{Include by Input}
\label{sec:input}

Including child documents by |\include| has some restrictions by design.
Most notably, the content of a child document always occupies
its own set of pages; pages cannot be shared between child documents.
Usually, this behaviour makes perfect sense
because each child document contain an essential part of the document.
However, in some situations it may be desirable to compose
a document from a collection of parts
without having mandatory page breaks between then.
For this case, the package
provides a mechanism to include parts
by |\input| which can also be processed individually.
However, by construction this mechanism
requires manual handling of the content to be output.

%%%%%%%%%%%%%%%%%%%%%%%%%%%%%%%%%%%%%%%%
\DescribeMacro{\ifchilddocmanual}
The main file should be prepared as usual, see \secref{sec:include}.
However, the document body must make a distinction
between processing of an individual part and of the main document, e.g.:
%
\begin{center}
\begin{tabular}{l}
|\ifchilddocmanual|\\
|\input{\childdocname}|\\
|\||else|\\
\textit{document body with }|\input{|\textit{part}|}|\\
|\||fi|
\end{tabular}
\end{center}
%
The conditional |\ifchilddocmanual| is true whenever
a part to be included by |\input| is being compiled,
and the name of the part is stored in |\childdocname|.

%%%%%%%%%%%%%%%%%%%%%%%%%%%%%%%%%%%%%%%%
\DescribeMacro{\childdocby}
Each part to be included by |\input| should start with:
%
\begin{center}
\begin{tabular}{l}
|\input{childdoc.def}|\\
|\childdocby{|\textit{main}|}|\\
\end{tabular}
\end{center}
%
The directive |\childdocby| is similar to |\childdocof|
described in \secref{sec:include},
but the subsequent selection of content must be done manually.
To that end, both |\ifchilddoc| and |\ifchilddocmanual|
will be true upon processing of a part,
and the name of the part is stored in |\childdocname|.
Note that |\jobname| will be set to the filename of the current part
so that each part receives an individual |.aux| file
that does not interfere with the |.aux| file(s) of the main document.
This behaviour can be altered by the alternative form
|\childdocby[*]{|\textit{main}|}| (with a non-empty optional argument)
which uses the |.aux| file of the main document
by setting |\jobname| to \textit{main}.

%%%%%%%%%%%%%%%%%%%%%%%%%%%%%%%%%%%%%%%%%%%%%%%%%%%%%%%%%%%%%%%%%%%%%%%%%%%%%%%%
\subsection{Driver Development}
\label{sec:driver}

The \textsf{childdoc} mechanism can also be use for the development
of definition files such as \LaTeX{} styles or classes.
This case differs from the above setup with multiple parts
included by |\include| in that no |\includeonly| should be invoked.
This can be achieved by starting the include file
(before |\ProvidesPackage|) with:
%
\begin{center}
\begin{tabular}{l}
|\input{childdoc.def}|\\
|\childdocforward{|\textit{main}|}|\\
\end{tabular}
\end{center}
%
or alternatively with:
%
\begin{center}
\begin{tabular}{l}
|\input{childdoc.def}|\\
|\childdocby{|\textit{main}|}|\\
\end{tabular}
\end{center}
%
Both forms have slightly different effects as described above.
The main file is prepared as usual, see \secref{sec:include}.

%%%%%%%%%%%%%%%%%%%%%%%%%%%%%%%%%%%%%%%%%%%%%%%%%%%%%%%%%%%%%%%%%%%%%%%%%%%%%%%%
\subsection{Legacy Detection}
\label{sec:detection}

The directive |\childdocmain| in the main file can detect
whether the complete document or merely a child is to be compiled
even without using the directive |\childdocof|.
This method is deprecated because it is less robust
and there is no compelling reason to use it;
it is merely provided for backward compatibility
and it may be removed in future versions.

If the detection mechanism is to be used,
it is mandatory to correctly specify
the filename of the main file as the argument of |\childdocmain|:
%
\begin{center}
\begin{tabular}{l}
|\input{childdoc.def}|\\
|\childdocmain{|\textit{main}|}|\\
\end{tabular}
\end{center}
%
If |\jobname| does not match the argument \textit{main} of |\childdocmain|,
it is assumed that |\jobname| points to the child file to be compiled.
When using |\childdocmain| with the main file specified as argument,
it suffices to start a child file
with just |\input{|\textit{main}|}|
without loading of the package and using |\childdocof|.
If instead all processing is done
with the appropriate \textsf{childdoc} directives,
the argument of \textit{main} of |\childdocmain| can be empty.

An alternative version of the command line processing described
in \secref{sec:commandline} using the detection mechanism reads:
%
\begin{center}
|... -jobname "|\textit{target}|" "|[\textit{flags}]%
[|\def\jobname{|\textit{dest}|}|]|\input{|\textit{main}|}"|
\end{center}

%%%%%%%%%%%%%%%%%%%%%%%%%%%%%%%%%%%%%%%%%%%%%%%%%%%%%%%%%%%%%%%%%%%%%%%%%%%%%%%%
\subsection{Manual Code}
\label{sec:manual}

In case one cannot be certain whether the definitions file |childdoc.def|
is installed on the target \TeX{} distribution
and one prefers not to ship it,
it is conceivable to paste a few relevant commands into the sources.

To that end, drop all statements |\input{childdoc.def}|
and perform the replacements as outlined below.
Instead of |\childdocmain{|\textit{main}|}| add the following code
to the top of the main file:
%
\begin{center}
\begin{tabular}{l}
|\||ifdefined\childdocname\endinput\||fi\newif\ifchilddoc|\\
|\edef\childdocname{\scantokens\expandafter{\jobname\noexpand}}|\\
|\def\childdocmain{|\textit{main}|}\||ifx\childdocmain\childdocname\||else|\\
|\childdoctrue\includeonly{\childdocname}\let\jobname\childdocmain\||fi|\\
\end{tabular}
\end{center}
%
Instead of |\childdocof{|\textit{main}|}| just include the main file
at the top of each child file:
%
\begin{center}
|\input{|\textit{main}|}|
\end{center}
%
A simple redirection |\childdocforward{|\textit{dest}|}| is achieved by:
%
\begin{center}
|\def\jobname{|\textit{dest}|}\input{\jobname}|
\end{center}
%
The redirection with prefix
|\childdocforwardprefix[|\textit{prefix}|]{|\textit{dest}|}|
is accomplished by:
%
\begin{center}
\begin{tabular}{l}
|{\edef\jobname{\scantokens\expandafter{\jobname\noexpand}}|\\
|\def\redirectjob |\textit{prefix}|#1~~~{\gdef\jobname{|\textit{dest}|#1}}|\\
|\expandafter\redirectjob\jobname~~~}\input{\jobname}|
\end{tabular}
\end{center}

In an alternative approach,
child documents can be compiled by a specific command line
without additional code or specific definitions:
%
\begin{center}
|... -jobname "|\textit{target}|" "|[\textit{flags}]%
|\includeonly{|\textit{dest}|}\input{|\textit{main}|}"|
\end{center}
%

%%%%%%%%%%%%%%%%%%%%%%%%%%%%%%%%%%%%%%%%%%%%%%%%%%%%%%%%%%%%%%%%%%%%%%%%%%%%%%%%
%%%%%%%%%%%%%%%%%%%%%%%%%%%%%%%%%%%%%%%%%%%%%%%%%%%%%%%%%%%%%%%%%%%%%%%%%%%%%%%%
\section{Information}

%%%%%%%%%%%%%%%%%%%%%%%%%%%%%%%%%%%%%%%%%%%%%%%%%%%%%%%%%%%%%%%%%%%%%%%%%%%%%%%%
\subsection{Copyright}

Copyright \copyright{} 2017--2018 Niklas Beisert

This work may be distributed and/or modified under the
conditions of the \LaTeX{} Project Public License, either version 1.3
of this license or (at your option) any later version.
The latest version of this license is in
  \url{http://www.latex-project.org/lppl.txt}
and version 1.3 or later is part of all distributions of \LaTeX{}
version 2005/12/01 or later.

This work has the LPPL maintenance status `maintained'.

The Current Maintainer of this work is Niklas Beisert.

This work consists of the files |README.txt|, |childdoc.ins| and |childdoc.dtx|
as well as the derived files |childdoc.def|, |cdocsamp.tex|
with |cdocsch1.tex|, |cdocsch2.tex|, |cdocspt3.tex|, |cdocspt4.tex|,
|cdocsdrf.tex|, |cdocsfn1.tex|, |cdocsfn2.tex|
as well as |childdoc.pdf|.

%%%%%%%%%%%%%%%%%%%%%%%%%%%%%%%%%%%%%%%%%%%%%%%%%%%%%%%%%%%%%%%%%%%%%%%%%%%%%%%%
\subsection{Files and Installation}

The package consists of the files:
%
\begin{center}
\begin{tabular}{ll}
    |README.txt|   & readme file \\
    |childdoc.ins| & installation file \\
    |childdoc.dtx| & source file \\
    |childdoc.def| & definition file \\
    |cdocsamp.tex| & sample main file \\
    |cdocsch1.tex| & sample include file \\
    |cdocsch2.tex| & sample include file \\
    |cdocspt3.tex| & sample part file \\
    |cdocspt4.tex| & sample part file \\
    |cdocsdrf.tex| & sample redirection file \\
    |cdocsfn1.tex| & sample redirection file \\
    |cdocsfn2.tex| & sample redirection file \\
    |childdoc.pdf| & manual
\end{tabular}
\end{center}
%
The distribution consists of the files
|README.txt|, |childdoc.ins| and |childdoc.dtx|.
%
\begin{itemize}
\item
Run (pdf)\LaTeX{} on |childdoc.dtx|
to compile the manual |childdoc.pdf| (this file).
\item
Run \LaTeX{} on |childdoc.ins| to create the definitions file |childdoc.def|
and the sample |cdocsamp.tex| with include files
|cdocsch1.tex|, |cdocsch2.tex|, |cdocspt3.tex|, |cdocspt4.tex|,
|cdocsdrf.tex|, |cdocsfn1.tex|, |cdocsfn2.tex|.
Then copy the file |childdoc.def| to an appropriate directory of your \LaTeX{}
distribution, e.g.\ \textit{texmf-root}|/tex/latex/childdoc|.
\end{itemize}

%%%%%%%%%%%%%%%%%%%%%%%%%%%%%%%%%%%%%%%%%%%%%%%%%%%%%%%%%%%%%%%%%%%%%%%%%%%%%%%%
\subsection{Related CTAN Packages}

There are several other packages which offer a similar functionality:
%
\begin{itemize}
\item
The packages
\href{http://ctan.org/pkg/docmute}{\textsf{docmute}},
\href{http://ctan.org/pkg/includex}{\textsf{includex}} and
\href{http://ctan.org/pkg/standalone}{\textsf{standalone}}
provide commands to include only the document body of
a child file thus allowing both files to be compiled individually.
\item
The packages \href{http://ctan.org/pkg/subdocs}{\textsf{subdocs}}
and \href{http://ctan.org/pkg/subfiles}{\textsf{subfiles}}
provide structures in which the main and child documents can be
encapsulated and allowing them to be compiled individually.
The inclusion mechanism is different from the conventional |\include|.
\item
The package \href{http://ctan.org/pkg/combine}{\textsf{combine}}
is an elaborate solution to combine several documents into one.
\end{itemize}
%
See also the CTAN topic \href{http://ctan.org/topic/subdocs}{\textsf{subdocs}}
for further related packages.
The present package differs from the above solutions in that
a document structure constructed with the conventional |\include| mechanism
just needs two extra commands at the top of every file
such that all constituent files can be compiled individually.

%%%%%%%%%%%%%%%%%%%%%%%%%%%%%%%%%%%%%%%%%%%%%%%%%%%%%%%%%%%%%%%%%%%%%%%%%%%%%%%%
%\subsection{Feature Suggestions}
%
%The following is a list of features which may be useful for future
%versions of this package:
%%
%\begin{itemize}
%\item
%\ldots
%\end{itemize}

%%%%%%%%%%%%%%%%%%%%%%%%%%%%%%%%%%%%%%%%%%%%%%%%%%%%%%%%%%%%%%%%%%%%%%%%%%%%%%%%
\subsection{Revision History}

%%%%%%%%%%%%%%%%%%%%%%%%%%%%%%%%%%%%%%%%
\paragraph{v2.0:} 2018/12/30

\begin{itemize}
\item
immediate forward processing
\item
added |\childdocby| mechanism
\item
manual restructured
\end{itemize}

%%%%%%%%%%%%%%%%%%%%%%%%%%%%%%%%%%%%%%%%
\paragraph{v1.6:} 2018/01/17

\begin{itemize}
\item
application for development of include files
\item
corrections to manual
\end{itemize}

%%%%%%%%%%%%%%%%%%%%%%%%%%%%%%%%%%%%%%%%
\paragraph{v1.5:} 2017/05/21

\begin{itemize}
\item
more complete structuring introduced
\item
|\childdocof| introduced
\item
|\childdoc| renamed to |\childdocmain|
\item
|\childredirect| renamed to |\childdocforward| and |\childdocforwardprefix|
and functionality expanded
\end{itemize}

%%%%%%%%%%%%%%%%%%%%%%%%%%%%%%%%%%%%%%%%
\paragraph{v1.0:} 2017/04/27

\begin{itemize}
\item
manual and install package
\item
first version published on CTAN
\end{itemize}

%%%%%%%%%%%%%%%%%%%%%%%%%%%%%%%%%%%%%%%%
\paragraph{v0.6:} 2017/04/26

\begin{itemize}
\item
redirection mechanism added
\end{itemize}

%%%%%%%%%%%%%%%%%%%%%%%%%%%%%%%%%%%%%%%%
\paragraph{v0.5:} 2017/04/26

\begin{itemize}
\item
functionality in definition file
\end{itemize}


%%%%%%%%%%%%%%%%%%%%%%%%%%%%%%%%%%%%%%%%%%%%%%%%%%%%%%%%%%%%%%%%%%%%%%%%%%%%%%%%
%%%%%%%%%%%%%%%%%%%%%%%%%%%%%%%%%%%%%%%%%%%%%%%%%%%%%%%%%%%%%%%%%%%%%%%%%%%%%%%%
%%%%%%%%%%%%%%%%%%%%%%%%%%%%%%%%%%%%%%%%%%%%%%%%%%%%%%%%%%%%%%%%%%%%%%%%%%%%%%%%
\appendix

\settowidth\MacroIndent{\rmfamily\scriptsize 000\ }

 \DocInput{childdoc.dtx}

\end{document}
%</driver>
% \fi
%
% %%%%%%%%%%%%%%%%%%%%%%%%%%%%%%%%%%%%%%%%%%%%%%%%%%%%%%%%%%%%%%%%%%%%%%%%%%%%%%
% %%%%%%%%%%%%%%%%%%%%%%%%%%%%%%%%%%%%%%%%%%%%%%%%%%%%%%%%%%%%%%%%%%%%%%%%%%%%%%
% \section{Sample}
%\iffalse
%<*samplemain>
%\fi
%
% The following presents a sample document
% with two chapters, two parts, a title page,
% a compile flag as well as three forwarding files to set the flag.
% It consists of eight |.tex| files:
% \begin{center}
% \begin{tabular}{ll}
% |cdocsamp.tex|&main file\\
% |cdocsch1.tex|&include file for chapter 1\\
% |cdocsch2.tex|&include file for chapter 2\\
% |cdocspt3.tex|&include file for part 3\\
% |cdocspt4.tex|&include file for part 4\\
% |cdocsdrf.tex|&forwarding file for main file in draft mode\\
% |cdocsfi1.tex|&forwarding file for final version of chapter 1\\
% |cdocsfi2.tex|&forwarding file for final version of chapter 2\\
% \end{tabular}
% \end{center}
% Each of the eight files can be compiled directly by the \LaTeX{} compiler.
%
% %%%%%%%%%%%%%%%%%%%%%%%%%%%%%%%%%%%%%%
% \paragraph{Main File.}
%
% The main file is called |cdocsamp.tex|.
%
% Load the \textsf{childdoc} definitions and
% declare the filename for the main document:
%    \begin{macrocode}
\input{childdoc.def}
\childdocmain{}
%    \end{macrocode}

% Optional override for |\version| flag:
%    \begin{macrocode}
%%\ifchilddoc\else\providecommand{\version}{draft}\fi
%    \end{macrocode}

% Define the default values for the |\version| flag
% (|final| for the main file and |draft| for childs):
%    \begin{macrocode}
\ifchilddoc
\providecommand{\version}{draft}
\else
\providecommand{\version}{final}
\fi
%    \end{macrocode}

% Load the standard document class:
%    \begin{macrocode}
\documentclass[12pt]{article}
%    \end{macrocode}

% Start the document body:
%    \begin{macrocode}
\begin{document}
%    \end{macrocode}

% Declare a title page.
% Print title, part of document being processed and version flag:
%    \begin{macrocode}
\addtocounter{page}{-1}
\begin{center}
{\LARGE\bfseries{}childdoc example\par}
\vspace{1cm}
\ifchilddoc
\ifchilddocmanual part\else chapter\fi:
`\childdocname' of `\childdocjob'\par
\else
main document: `\childdocjob'\par
\fi
version: \version\par
\end{center}
\newpage
%    \end{macrocode}

% Manually include selected file,
% otherwise process as usual:
%    \begin{macrocode}
\ifchilddocmanual
\section*{part `\childdocname'}
\input{\childdocname}
\else
%    \end{macrocode}

% Include the two chapters:
%    \begin{macrocode}
\include{cdocsch1}
\include{cdocsch2}
%    \end{macrocode}

% Include the two parts unless only chapters should be displayed:
%    \begin{macrocode}
\ifchilddoc\else
\section{part three}
\input{cdocspt3}
\section{part four}
\input{cdocspt4}
\fi
%    \end{macrocode}

% Process as usual until here:
%    \begin{macrocode}
\fi
%    \end{macrocode}

% End of document body:
%    \begin{macrocode}
\end{document}
%    \end{macrocode}
%\iffalse
%</samplemain>
%\fi
%
% %%%%%%%%%%%%%%%%%%%%%%%%%%%%%%%%%%%%%%
% \paragraph{Chapter Include Files.}
%
% The include files are called |cdocsch1.tex| and |cdocsch2.tex|.
%
%\iffalse
%<*samplechap1|samplechap2>
%\fi

% Optional override for |\version| flag:
%    \begin{macrocode}
%%\providecommand{\version}{final}
%    \end{macrocode}

% Include the main document:
%    \begin{macrocode}
\input{childdoc.def}
\childdocof{cdocsamp}
%    \end{macrocode}

%\iffalse
%</samplechap1|samplechap2>
%\fi
%
%\iffalse
%<*samplechap1>
%\fi
% Some text for chapter 1:
%    \begin{macrocode}
\section{one}
some text in chapter one
%    \end{macrocode}

%\iffalse
%</samplechap1>
%\fi
% Some text for chapter 2:
%\iffalse
%<*samplechap2>
%\fi
%    \begin{macrocode}
\section{two}
more text in chapter two
%    \end{macrocode}

%\iffalse
%</samplechap2>
%\fi
%
% %%%%%%%%%%%%%%%%%%%%%%%%%%%%%%%%%%%%%%
% \paragraph{Part Include Files.}
%
% The include files are called |cdocspt3.tex| and |cdocspt4.tex|.
%
%\iffalse
%<*samplepart3|samplepart4>
%\fi

% Optional override for |\version| flag:
%    \begin{macrocode}
%%\providecommand{\version}{final}
%    \end{macrocode}

% Include the main document:
%    \begin{macrocode}
\input{childdoc.def}
\childdocby{cdocsamp}
%    \end{macrocode}

%\iffalse
%</samplepart3|samplepart4>
%\fi
%
%\iffalse
%<*samplepart3>
%\fi
% Some text for part 3:
%    \begin{macrocode}
some text in part three
%    \end{macrocode}

%\iffalse
%</samplepart3>
%\fi
% Some text for part 4:
%\iffalse
%<*samplepart4>
%\fi
%    \begin{macrocode}
more text in part four
%    \end{macrocode}

%\iffalse
%</samplepart4>
%\fi
%
% %%%%%%%%%%%%%%%%%%%%%%%%%%%%%%%%%%%%%%
% \paragraph{Forwarding for a Complete Draft.}
%
% The following forwarding file |cdocsdrf.tex|
% compiles the main document in draft mode:
%\iffalse
%<*sampledraft>
%\fi
%    \begin{macrocode}
\def\version{draft}
\input{childdoc.def}
\childdocforward{cdocsamp}
%    \end{macrocode}

%\iffalse
%</sampledraft>
%\fi
%
% %%%%%%%%%%%%%%%%%%%%%%%%%%%%%%%%%%%%%%
% \paragraph{Forwarding for Final Version of the Chapters.}
%
% The following forwarding files |cdocsfn1.tex| and |cdocsfn2.tex|
% (with identical content)
% compile the final versions of the child documents
% |cdocsch1.tex| and |cdocsch2.tex|, respectively:
%\iffalse
%<*samplefinal>
%\fi
%    \begin{macrocode}
\def\version{final}
\input{childdoc.def}
\childdocforwardprefix[cdocsamp]{cdocsfn}{cdocsch}
%    \end{macrocode}

%\iffalse
%</samplefinal>
%\fi
%
% %%%%%%%%%%%%%%%%%%%%%%%%%%%%%%%%%%%%%%
% \paragraph{Command Line Processing.}
%
% The following three command lines generate the output files
% |cdocscld|, |cdocscl1| and |cdocscl2|
% which should be identical to
% |cdocsdrf|, |cdocsch1| and |cdocsfn2|, respectively:
% \begin{center}
% \begin{tabular}{l}
% |latex -jobname cdocscld \|\\
% |  "\def\version{draft}\input{childdoc.def}\childdocforward{cdocsamp}"|\\
% |latex -jobname cdocscl1 \|\\
% |  "\input{childdoc.def}\childdocforward[cdocsamp]{cdocsch1}"|\\
% |latex -jobname cdocscl2 \|\\
% |  "\def\version{final}\input{childdoc.def}\childdocforward{cdocsch2}"|
% \end{tabular}
% \end{center}
% Note that the trailing backslash on each first line
% merely continues the input to the second line
% (for convenient cut ant paste).
% Furthermore, the command |latex| can be replaced by any
% of its alternative versions such as |pdflatex|.
%
% %%%%%%%%%%%%%%%%%%%%%%%%%%%%%%%%%%%%%%%%%%%%%%%%%%%%%%%%%%%%%%%%%%%%%%%%%%%%%%
% %%%%%%%%%%%%%%%%%%%%%%%%%%%%%%%%%%%%%%%%%%%%%%%%%%%%%%%%%%%%%%%%%%%%%%%%%%%%%%
% \section{Implementation}
%\iffalse
%<*package>
%\fi
%
% This section describes the definitions file |childdoc.def|.

% The definitions cannot be loaded using |\usepackage| or |\RequirePackage|
% which has a mechanism to prevent loading a style file more than once.
% When loading the definitions by means of |\input|
% multiple instances have to be prevented manually:
%\iffalse
%This code needs to be before the `\ProvidesFile' directive
%which is defined at the beginning of this file.
%Therefore it is also placed there and commented out here.
%</package>
%<*discard>
%\fi
%    \begin{macrocode}
\ifdefined\childdocmain\endinput\fi
%    \end{macrocode}
%\iffalse
%</discard>
%<*package>
%\fi
%
% \macro{\ifchilddoc}
% \macro{\ifchilddocmanual}
% The conditional |\ifchilddoc| tells whether a
% child (true) or main (false) document is being compiled.
% The conditional |\ifchilddocmanual| tells whether
% the |\includeonly| mechanism is used (false) or
% the selection of child files must be performed manually (true).
% The definitions initialise to false:
%    \begin{macrocode}
\newif\ifchilddoc
\newif\ifchilddocmanual
%    \end{macrocode}

% \macro{\childdocname}
% \macro{\childdocjob}
% The macro |\childdocname| stores the name of the main document
% to be compiled. The macro |\childdocjob| stores the name of
% the document on which the \LaTeX{} compiler was originally invoked.
% The content of |\jobname| cannot be compared
% to filenames specified in the source due to different catcodes.
% The following code rescans |\jobname|, stores the result
% in |\childdocname| and saves a copy in |\childdocjob|:
%    \begin{macrocode}
\edef\childdocname{\scantokens\expandafter{\jobname\noexpand}}
\let\childdocjob\childdocname
%    \end{macrocode}

% \macro{\childdocdisable}
% The macro |\childdocdisable| prevents the main file
% from being processed more than once.
% At this stage, the main document command |\childdocmain|
% is assumed to be called once again where it should do nothing.
% Any subsequent call to it should prevent
% a secondary processing of the main document
% It overwrites the forwarding commands
% |\childdocof| and |\childdocforward|
% with empty macros to prevent further inclusions of the main document:
%    \begin{macrocode}
\newcommand{\childdocdisable}
{
  \renewcommand{\childdocmain}[1]{\renewcommand{\childdocmain}[1]{\endinput}}
  \renewcommand{\childdocof}[1]{}
  \renewcommand{\childdocby}[2][]{}
  \renewcommand{\childdocforward}[2][]{}
  \renewcommand{\childdocdisable}{}
}
%    \end{macrocode}

% \macro{\childdocmain}
% The macro |\childdocmain| is to be called at the top of the main file
% with nothing or the main filename (without extension) as argument.
% First, it breaks loops.
% If the argument is not empty and does not match |\childdocname|
% (which is set by the first inclusion of |childdoc.def|),
% |\ifchilddoc| is set to true, |\includeonly| is applied to the child file
% and |\jobname| is set to the main file
% (for proper handling of |.aux| files):
%    \begin{macrocode}
\newcommand{\childdocmain}[1]
{
  \childdocdisable\childdocmain{}
  \if?#1?\else
    \begingroup
      \def\childdoctmp{#1}
      \ifx\childdoctmp\childdocname
        \def\childdoctmp{}
      \else
        \def\childdoctmp
        {
          \childdoctrue
          \includeonly{\childdocname}
          \def\childdocjob{#1}
          \def\jobname{#1}
        }
      \fi
      \expandafter
    \endgroup
    \childdoctmp
  \fi
}
%    \end{macrocode}

% \macro{\childdocof}
% The command |\childdocof| redirects
% compilation to the main file |#1|.
%    \begin{macrocode}
\newcommand{\childdocof}[1]
{
  \childdocdisable
  \childdoctrue
  \includeonly{\childdocname}
  \def\jobname{#1}
  \def\childdocjob{#1}
  \input{#1}
}
%    \end{macrocode}

% \macro{\childdocby}
% The command |\childdocby| ....
%    \begin{macrocode}
\newcommand{\childdocby}[2][]
{
  \childdocdisable
  \childdoctrue
  \childdocmanualtrue
  \if?#1?\else
    \def\jobname{#2}
  \fi
  \def\childdocjob{#2}
  \input{#2}
  \endinput
}
%    \end{macrocode}

% \macro{\childdocforward}
% The command |\childdocforward| redirects
% compilation to the main file or
% (if the optional argument is given) a child file.
% Parameters are set as if the main file
% or a child file starting with |\childdocof| was compiled.
% Then compilation is handed over to the main file:
%    \begin{macrocode}
\newcommand{\childdocforward}[2][]
{
  \begingroup
    \if?#1?
      \def\childdoctmp
      {
        \def\childdocname{#2}
        \def\childdocjob{#2}
        \def\jobname{#2}
        \input{#2}
        \endinput
      }
    \else
      \def\childdoctmp
      {
        \childdocdisable
        \def\childdocname{#2}
        \childdoctrue
        \includeonly{#2}
        \def\childdocjob{#1}
        \def\jobname{#1}
        \input{#1}
        \endinput
      }
    \fi
    \expandafter
  \endgroup
  \childdoctmp
}
%    \end{macrocode}

% \macro{\childdocforwardprefix}
% The command |\childdocforwardprefix| redirects
% compilation to the main or a child file by means of a pattern.
% The prefix |#1| in the current filename is replaced by |#2|
% and the suffix of the current filename is kept
% (it is assumed that the filename does not contain the substring `|~~~|'
% which is used as a delimiter).
% Compilation is handed over to the new file by |\childdocforward|:
%    \begin{macrocode}
\newcommand{\childdocforwardprefix}[3][]
{
  \begingroup
    \def\childdocextract #2##1~~~{\def\childdoctmp{\childdocforward[#1]{#3##1}}}
    \expandafter\childdocextract\childdocname~~~
    \expandafter
  \endgroup
  \childdoctmp
}
%    \end{macrocode}

% \macro{\childdoc}
% The deprecated macro |\childdoc| is a legacy version of |\childdocmain|:
%    \begin{macrocode}
\newcommand{\childdoc}{\childdocmain}
%    \end{macrocode}

% \macro{\childdocredirect}
% The deprecated macro |\childdocredirect| is a legacy version
% of |\childdocforward| and |\childdocforwardprefix|:
%    \begin{macrocode}
\newcommand{\childdocredirect}[2][]
{
  \begingroup
    \if?#1?
      \def\childdoctmp{\childdocforward{#2}}
    \else
      \def\childdoctmp{\childdocforwardprefix{#1}{#2}}
    \fi
    \expandafter
  \endgroup
  \childdoctmp
}
%    \end{macrocode}

%\iffalse
%</package>
%\fi
%
\endinput
|\\
|\childdocforwardprefix[|\textit{main}|]{|\textit{prefix}|}{|\textit{dest}|}|
\end{tabular}
\end{center}
%
the destination file is determined by a pattern
depending on the current file:
To make this work, the current file must be called
`{\textit{prefix}\hspace{0.2em}\textit{suffix}}'
with \textit{prefix} matching precisely the argument.
Processing is then passed on to the file
`{\textit{dest}\hspace{0.2em}\textit{suffix}}'.
Surely, the same effect is achieved by
directly specifying the
argument `{\textit{dest}\hspace{0.2em}\textit{suffix}}'
in the first form.
However, that requires to set up a different file
for each child. With the alternative form of the command
all these files can have exactly the same content
which simplifies setting them up and maintaining them.

For example, the following file |draft.tex|
with a compilation flag |\version| as described in \secref{sec:flags}
compiles the main document as a draft:
%
\begin{center}
\begin{tabular}{l}
|\def\version{draft}|\\
|% \iffalse
%
% childdoc.dtx Copyright (C) 2017-2018 Niklas Beisert
%
% This work may be distributed and/or modified under the
% conditions of the LaTeX Project Public License, either version 1.3
% of this license or (at your option) any later version.
% The latest version of this license is in
%   http://www.latex-project.org/lppl.txt
% and version 1.3 or later is part of all distributions of LaTeX
% version 2005/12/01 or later.
%
% This work has the LPPL maintenance status `maintained'.
%
% The Current Maintainer of this work is Niklas Beisert.
%
% This work consists of the files childdoc.dtx and childdoc.ins
% and the derived files childdoc.def and cdocsamp.tex with
% cdocsch1.tex, cdocsch2.tex, cdocsdrf.tex, cdocsfn1.tex, cdocsfn2.tex.
%
%<package>\ifdefined\childdocmain\endinput\fi
%<package>\ProvidesFile{childdoc.def}[2018/12/30 v2.0 child document driver]
%<samplemain>\ProvidesFile{cdocsamp.tex}[2018/12/30 v2.0 sample for childdoc]
%<*driver>
%\ProvidesFile{childdoc.drv}[2018/12/30 v2.0 childdoc reference manual file]
\PassOptionsToClass{10pt,a4paper}{article}
\documentclass{ltxdoc}

\usepackage[margin=35mm]{geometry}
\usepackage{hyperref}
\usepackage{hyperxmp}
\usepackage[usenames]{color}

\hypersetup{colorlinks=true}
\hypersetup{pdfstartview=FitH}
\hypersetup{pdfpagemode=UseNone}
\hypersetup{pdfsource={}}
\hypersetup{pdflang={en-UK}}
\hypersetup{pdfcopyright={Copyright 2017-2018 Niklas Beisert.
  This work may be distributed and/or modified under the
  conditions of the LaTeX Project Public License, either version 1.3
  of this license or (at your option) any later version.}}
\hypersetup{pdflicenseurl={http://www.latex-project.org/lppl.txt}}
\hypersetup{pdfcontactaddress={ETH Zurich, ITP, HIT K,
  Wolfgang-Pauli-Strasse 27}}
\hypersetup{pdfcontactpostcode={8093}}
\hypersetup{pdfcontactcity={Zurich}}
\hypersetup{pdfcontactcountry={Switzerland}}
\hypersetup{pdfcontactemail={nbeisert@itp.phys.ethz.ch}}
\hypersetup{pdfcontacturl={http://people.phys.ethz.ch/\xmptilde nbeisert/}}

\newcommand{\secref}[1]{\hyperref[#1]{section \ref*{#1}}}

\parskip1ex
\parindent0pt
\let\olditemize\itemize
\def\itemize{\olditemize\parskip0pt}

\begin{document}

\title{The \textsf{childdoc} Package}
\hypersetup{pdftitle={The childdoc Package}}
\author{Niklas Beisert\\[2ex]
  Institut f\"ur Theoretische Physik\\
  Eidgen\"ossische Technische Hochschule Z\"urich\\
  Wolfgang-Pauli-Strasse 27, 8093 Z\"urich, Switzerland\\[1ex]
  \href{mailto:nbeisert@itp.phys.ethz.ch}
  {\texttt{nbeisert@itp.phys.ethz.ch}}}
\hypersetup{pdfauthor={Niklas Beisert}}
\hypersetup{pdfsubject={Manual for the LaTeX2e Package childdoc}}
\date{30 December 2018, \textsf{v2.0}}
\maketitle

\begin{abstract}\noindent
\textsf{childdoc} is a \LaTeXe{} package
that enables the direct compilation
of document sections included by |\include|
to individual files.
\end{abstract}

\begingroup
\parskip0ex
\tableofcontents
\endgroup

%%%%%%%%%%%%%%%%%%%%%%%%%%%%%%%%%%%%%%%%%%%%%%%%%%%%%%%%%%%%%%%%%%%%%%%%%%%%%%%%
%%%%%%%%%%%%%%%%%%%%%%%%%%%%%%%%%%%%%%%%%%%%%%%%%%%%%%%%%%%%%%%%%%%%%%%%%%%%%%%%
\section{Introduction}

\LaTeX{} provides a mechanism to structure a large document (such as a book)
into a main file and several child files (containing the chapters)
using the |\include| command.
This mechanism is beneficial for documents
which span hundreds of pages in order to
make the source file(s) more manageable.
Moreover, compilation can be restricted to
selected child files by means of the |\includeonly| command.
The latter feature can be used to reduce the compilation time while editing
(this was significantly more useful in the earlier days of \LaTeX{})
or to generate a smaller document which is easier to navigate.
Another application of |\includeonly| is to generate
documents consisting of selected parts of the complete document.

However, there are a few drawbacks of the plain |\include| mechanism:
\begin{itemize}
\item
The child files cannot be compiled on their own,
they can only be compiled via the main file.
A naive editing environment
(such as a text editor with an option
to have the current file processed by \LaTeX)
may require one to switch to the main file before compiling;
attempting to compile the child file produces errors.
\item
The main file must be modified (each time)
to adjust the |\includeonly| command
to the present needs. This easily leaves the main file in a messy state.
\item
The generated document will always carry the filename
of the main document. This is inconvenient if
several child files are to be compiled and
to be kept for distribution.
\end{itemize}

The present package provides a simple interface
to make child files individually compilable by \LaTeX{}.
Compiling a child file then has the same effect as compiling
the main file with an |\includeonly| command
to select the appropriate child.
Moreover the generated document will carry the name of the child
rather than the main file.
This resolves all three above issues.

This feature is meant to make the editing of books,
thesis documents and lecture notes somewhat more convenient.
However, the package can also be used efficiently for
composing a series of documents (such as exercise sheets)
which are typically distributed individually.
It then assists the author in generating the individual documents
(potentially in different versions)
as well as a document containing the collected series.
Another application is in developing style files
or other kinds of included material
where compilation of the style file could redirect
to a sample or test file.

%%%%%%%%%%%%%%%%%%%%%%%%%%%%%%%%%%%%%%%%%%%%%%%%%%%%%%%%%%%%%%%%%%%%%%%%%%%%%%%%
%%%%%%%%%%%%%%%%%%%%%%%%%%%%%%%%%%%%%%%%%%%%%%%%%%%%%%%%%%%%%%%%%%%%%%%%%%%%%%%%
\section{Usage}

First of all, the package \textsf{childdoc} is \emph{not} a standard
\LaTeXe{} |.sty| style file! Therefore it needs to be invoked in
a non-standard way.

%%%%%%%%%%%%%%%%%%%%%%%%%%%%%%%%%%%%%%%%%%%%%%%%%%%%%%%%%%%%%%%%%%%%%%%%%%%%%%%%
\subsection{Included Files}
\label{sec:include}

%%%%%%%%%%%%%%%%%%%%%%%%%%%%%%%%%%%%%%%%
\DescribeMacro{\childdocmain}
To use the package, add the commands
\begin{center}
\begin{tabular}{l}
|\input{childdoc.def}|\\
|\childdocmain{}|\\
\end{tabular}
\end{center}
at the very top of the main \LaTeX{} file,
in particular \emph{before} the |\documentclass| statement!
The argument of |\childdocmain| should be left empty
(but it must be present).

%%%%%%%%%%%%%%%%%%%%%%%%%%%%%%%%%%%%%%%%
\DescribeMacro{\childdocof}
Furthermore, add the commands
\begin{center}
\begin{tabular}{l}
|\input{childdoc.def}|\\
|\childdocof{|\textit{main}|}|\\
\end{tabular}
\end{center}
at the top of every child file \textit{child}
which is included by |\include{|\textit{child}|}|
from within the main file
(or at least for those files to be compiled individually).
The argument \textit{main} must be the filename of the main file.

There are a couple of
considerations in setting up the main and child documents:

%%%%%%%%%%%%%%%%%%%%%%%%%%%%%%%%%%%%%%%%
\paragraph{Restrictions.}

Please note the following restrictions:
\begin{itemize}
\item
|\childdocmain| must be called with one argument \textit{main}
to ensure compatibility with earlier version of the package.
It must either be empty (|\childdocmain{}|)
or precisely match the filename of the main file in which it is specified.
See \secref{sec:detection} for further information.
\item
The filename \textit{main} must be specified without the |.tex| extension.
\item
The filename \textit{main} is case sensitive
(even in case-insensitive file systems)
due to internal string comparison.
\item
The argument \textit{main} should be fully expanded, it cannot be a macro.
\item
Subdirectories and special characters should be avoided in filenames.
\item
The command |\childdocmain{|\textit{main}|}| must be followed by a whitespace.
It should not be followed immediately by another command
or by a comment mark `|%|'.
This is because the \TeX{} parser reads the token immediately following
the argument of |\childdocmain| and puts it
at the beginning of every child section;
however, a white\-space is ignored.
\end{itemize}

%%%%%%%%%%%%%%%%%%%%%%%%%%%%%%%%%%%%%%%%
\paragraph{Content of Main File.}

It is advisable to place all content in the child files included by |\include|.
Any output contained in the main file will appear in all child documents
unless suppressed manually;
it cannot be suppressed automatically by the |\includeonly| directive
and thus should normally be avoided.
A method to include some content in the main file
by means of conditional processing is described in \secref{sec:conditional}.

%%%%%%%%%%%%%%%%%%%%%%%%%%%%%%%%%%%%%%%%
\paragraph{Page Numbering.}

When only a part of the document is compiled,
the appropriate numbering of pages
(as well as other status parameters)
is determined from the |.aux| files.
The latter contain information from previous passes.
However this information needs to propagate through
all intermediate child documents.
Therefore the page numbering in child documents may well
be inconsistent until the complete document is compiled at least once.

A useful (if unconventional) way to always ensure a consistent
page numbering is to restart the numbering in each child document
and denote the pages by `\textit{child}|.|\textit{page}'
where \textit{child} represents the chapter/section number of the child file.
This can be achieved by the command
|\numberwithin{page}{|\textit{child}|}|
of the \textsf{amsmath} package
where \textit{child} can be |chapter| or |section|
depending on the chosen structuring.
Alternatively, one can modify the macro |\thepage| appropriately
and reset the counter |page| at the start of each child file.

%%%%%%%%%%%%%%%%%%%%%%%%%%%%%%%%%%%%%%%%%%%%%%%%%%%%%%%%%%%%%%%%%%%%%%%%%%%%%%%%
\subsection{Conditional Processing}
\label{sec:conditional}

The package provides a mechanism to compile different versions
of a document. To customise the versions further some conditional processing
can come in handy to distinguish which version is being compiled.
The package provides two macros to describe the compilation context:

%%%%%%%%%%%%%%%%%%%%%%%%%%%%%%%%%%%%%%%%
\DescribeMacro{\ifchilddoc}
The conditional |\ifchilddoc| distinguishes between the compilation of
child documents and the main document:
%
\begin{center}
|\ifchilddoc |\textit{child-code}| |[|\||else |\textit{main-code}]| \||fi|
\end{center}

%%%%%%%%%%%%%%%%%%%%%%%%%%%%%%%%%%%%%%%%
\DescribeMacro{\childdocname}
\DescribeMacro{\childdocjob}
The macro |\childdocname| contains the filename (without extension)
of the main or child file being processed.
Note that |\childdocjob| will always contain the name of the main file.

%%%%%%%%%%%%%%%%%%%%%%%%%%%%%%%%%%%%%%%%
\paragraph{Title Page.}

Conditional processing can be used to include a title or banner page
in the main document when proper precautions are taken.
Importantly, the code in the main file should ensure that the page counter
(as well as other status parameters which are stored in the |.aux| files)
takes the same value after the conditional processing.
Otherwise the page numbers may take divergent values
depending on which part is compiled.

For example, a title page could be declared by:
%
\begin{center}
\begin{tabular}{l}
|\ifchilddoc\||else|\\
|\addtocounter{page}{-1}|\\
\textit{code for title page}\\
|\newpage|\\
|\||fi|
\end{tabular}
\end{center}
%
A banner page for the child documents can be generated by:
%
\begin{center}
\begin{tabular}{l}
|\ifchilddoc|\\
|\addtocounter{page}{-1}|\\
\textit{code for banner page}\\
|\newpage|\\
|\||fi|
\end{tabular}
\end{center}
%
Here one could write a message such as:
\begin{center}
|This is the part \childdocname{} of \childdocjob{}.|
\end{center}

%%%%%%%%%%%%%%%%%%%%%%%%%%%%%%%%%%%%%%%%%%%%%%%%%%%%%%%%%%%%%%%%%%%%%%%%%%%%%%%%
\subsection{Flags}
\label{sec:flags}

The package makes it easy to generate different versions
of the main or child documents.
To this end compilation flags can be defined
and assigned different default values.
They will be particularly useful in conjunction
with the forwarding mechanism described in \secref{sec:forward}.

For example, it may be useful to have a flag |\version|
which can be set to |draft| or |final|.
The document source will contain some conditional code
depending on the value of |\version|.
Suppose further, the flag should default to |final| for the main file
and to |draft| for child files
which is a natural assignment for editing the document.
This is achieved by placing the following code
in the preamble of the main document
(below the |\childdocmain| directive):
%
\begin{center}
\begin{tabular}{l}
|\ifchilddoc|\\
|\providecommand{\version}{draft}|\\
|\||else|\\
|\providecommand{\version}{final}|\\
|\||fi|
\end{tabular}
\end{center}
%
The definition by |\providecommand| makes sure
that previous definitions are not overwritten.
Further statements |\providecommand{\version}{...}|
can thus be added before the above code to override it.

For the main file, one might add a line
(between |\childdocmain| and the above block)
%
\begin{center}
|%\ifchilddoc\||else\providecommand{\version}{draft}\||fi|
\end{center}
%
which can be uncommented to produce a draft version.
Likewise one can add a line to the very top of a child file
(above the |\childdocof{|\textit{main}|}| directive)
%
\begin{center}
|%\providecommand{\version}{final}|
\end{center}
%
which can be uncommented to produce the final version of this child document.

%%%%%%%%%%%%%%%%%%%%%%%%%%%%%%%%%%%%%%%%%%%%%%%%%%%%%%%%%%%%%%%%%%%%%%%%%%%%%%%%
\subsection{Forwarding}
\label{sec:forward}

Different versions of the main or child documents
using compilation flags as described in \secref{sec:flags}
can be (permanently) stored in different files
for convenient compilation, viewing and distribution.
To this end, the package defines a command
to pass on compilation to a different file:

%%%%%%%%%%%%%%%%%%%%%%%%%%%%%%%%%%%%%%%%
\DescribeMacro{\childdocforward}
The command |\childdocforward| redirects processing to
another source file:
%
\begin{center}
\begin{tabular}{l}
|\input{childdoc.def}|\\
|\childdocforward[|\textit{main}|]{|\textit{dest}|}|\\
\end{tabular}
\end{center}
%
The argument \textit{dest} is the destination file
(without extension).
It should be the main file or one of the child files.
Note that further \textsf{childdoc} directives
such as |\childdocof| and |\childdocforward|
in the indicated file will be processed in this form.
The optional argument \textit{main}
passes on directly to the main file \textit{main}
while pretending to compile the child \textit{dest}.
This form behaves as if \textit{dest}
issues |\childdocof{|\textit{main}|}| right away,
and no further \textsf{childdoc} directives will be processed.

%%%%%%%%%%%%%%%%%%%%%%%%%%%%%%%%%%%%%%%%
\DescribeMacro{\...prefix}
In the alternative form |\childdocforwardprefix|,
%
\begin{center}
\begin{tabular}{l}
|\input{childdoc.def}|\\
|\childdocforwardprefix[|\textit{main}|]{|\textit{prefix}|}{|\textit{dest}|}|
\end{tabular}
\end{center}
%
the destination file is determined by a pattern
depending on the current file:
To make this work, the current file must be called
`{\textit{prefix}\hspace{0.2em}\textit{suffix}}'
with \textit{prefix} matching precisely the argument.
Processing is then passed on to the file
`{\textit{dest}\hspace{0.2em}\textit{suffix}}'.
Surely, the same effect is achieved by
directly specifying the
argument `{\textit{dest}\hspace{0.2em}\textit{suffix}}'
in the first form.
However, that requires to set up a different file
for each child. With the alternative form of the command
all these files can have exactly the same content
which simplifies setting them up and maintaining them.

For example, the following file |draft.tex|
with a compilation flag |\version| as described in \secref{sec:flags}
compiles the main document as a draft:
%
\begin{center}
\begin{tabular}{l}
|\def\version{draft}|\\
|\input{childdoc.def}|\\
|\childdocforward{|\textit{main}|}|
\end{tabular}
\end{center}
%
Likewise, the following files |final|\textit{nn}|.tex|
compile the final version of the child document
|child|\textit{nn}|.tex|:
%
\begin{center}
\begin{tabular}{l}
|\def\version{final}|\\
|\input{childdoc.def}|\\
|\childdocforwardprefix{final}{child}|
\end{tabular}
\end{center}
%

Note that when several versions of a main file and/or of each child file
are to be generated, it may be convenient to set up a |Makefile| or
shell script to automatise the process.

%%%%%%%%%%%%%%%%%%%%%%%%%%%%%%%%%%%%%%%%%%%%%%%%%%%%%%%%%%%%%%%%%%%%%%%%%%%%%%%%
\subsection{Command Line Processing}
\label{sec:commandline}

The effect of redirection files can also be achieved by invoking
the \LaTeX{} compiler with a more elaborate command line.
Most conveniently this should be done as part
of a shell script or a |Makefile|.

When using \textsf{childdoc} in the main file, the following
command lines effectively perform a redirection
(note that depending on the shell being used,
backslashes may have to be doubled: `|\|' $\to$ `|\\|'):
%
\begin{center}
|... -jobname "|\textit{target}|" |\\|"|[\textit{flags}]%
|\input{childdoc.def}\childdocforward[|\textit{main}|]{|\textit{dest}|}"|
\end{center}
%
Here \textit{target} is the name of the output file,
\textit{main} is the name of the main file
and \textit{dest} is the name of the main or child file to be processed
(all filenames without extensions).
The optional argument \textit{main} can be omitted
if \textit{main} matches \textit{dest}.
Optionally, compilation \textit{flags} can be defined via |\def| commands.
This command line makes the \TeX{} engine believe
it is compiling the file \textit{target}
whose content is specified as the latter parameter.
The provided code then forwards the processing to
\textit{main} or \textit{dest} as described in \secref{sec:forward}.

%%%%%%%%%%%%%%%%%%%%%%%%%%%%%%%%%%%%%%%%%%%%%%%%%%%%%%%%%%%%%%%%%%%%%%%%%%%%%%%%
\subsection{Include by Input}
\label{sec:input}

Including child documents by |\include| has some restrictions by design.
Most notably, the content of a child document always occupies
its own set of pages; pages cannot be shared between child documents.
Usually, this behaviour makes perfect sense
because each child document contain an essential part of the document.
However, in some situations it may be desirable to compose
a document from a collection of parts
without having mandatory page breaks between then.
For this case, the package
provides a mechanism to include parts
by |\input| which can also be processed individually.
However, by construction this mechanism
requires manual handling of the content to be output.

%%%%%%%%%%%%%%%%%%%%%%%%%%%%%%%%%%%%%%%%
\DescribeMacro{\ifchilddocmanual}
The main file should be prepared as usual, see \secref{sec:include}.
However, the document body must make a distinction
between processing of an individual part and of the main document, e.g.:
%
\begin{center}
\begin{tabular}{l}
|\ifchilddocmanual|\\
|\input{\childdocname}|\\
|\||else|\\
\textit{document body with }|\input{|\textit{part}|}|\\
|\||fi|
\end{tabular}
\end{center}
%
The conditional |\ifchilddocmanual| is true whenever
a part to be included by |\input| is being compiled,
and the name of the part is stored in |\childdocname|.

%%%%%%%%%%%%%%%%%%%%%%%%%%%%%%%%%%%%%%%%
\DescribeMacro{\childdocby}
Each part to be included by |\input| should start with:
%
\begin{center}
\begin{tabular}{l}
|\input{childdoc.def}|\\
|\childdocby{|\textit{main}|}|\\
\end{tabular}
\end{center}
%
The directive |\childdocby| is similar to |\childdocof|
described in \secref{sec:include},
but the subsequent selection of content must be done manually.
To that end, both |\ifchilddoc| and |\ifchilddocmanual|
will be true upon processing of a part,
and the name of the part is stored in |\childdocname|.
Note that |\jobname| will be set to the filename of the current part
so that each part receives an individual |.aux| file
that does not interfere with the |.aux| file(s) of the main document.
This behaviour can be altered by the alternative form
|\childdocby[*]{|\textit{main}|}| (with a non-empty optional argument)
which uses the |.aux| file of the main document
by setting |\jobname| to \textit{main}.

%%%%%%%%%%%%%%%%%%%%%%%%%%%%%%%%%%%%%%%%%%%%%%%%%%%%%%%%%%%%%%%%%%%%%%%%%%%%%%%%
\subsection{Driver Development}
\label{sec:driver}

The \textsf{childdoc} mechanism can also be use for the development
of definition files such as \LaTeX{} styles or classes.
This case differs from the above setup with multiple parts
included by |\include| in that no |\includeonly| should be invoked.
This can be achieved by starting the include file
(before |\ProvidesPackage|) with:
%
\begin{center}
\begin{tabular}{l}
|\input{childdoc.def}|\\
|\childdocforward{|\textit{main}|}|\\
\end{tabular}
\end{center}
%
or alternatively with:
%
\begin{center}
\begin{tabular}{l}
|\input{childdoc.def}|\\
|\childdocby{|\textit{main}|}|\\
\end{tabular}
\end{center}
%
Both forms have slightly different effects as described above.
The main file is prepared as usual, see \secref{sec:include}.

%%%%%%%%%%%%%%%%%%%%%%%%%%%%%%%%%%%%%%%%%%%%%%%%%%%%%%%%%%%%%%%%%%%%%%%%%%%%%%%%
\subsection{Legacy Detection}
\label{sec:detection}

The directive |\childdocmain| in the main file can detect
whether the complete document or merely a child is to be compiled
even without using the directive |\childdocof|.
This method is deprecated because it is less robust
and there is no compelling reason to use it;
it is merely provided for backward compatibility
and it may be removed in future versions.

If the detection mechanism is to be used,
it is mandatory to correctly specify
the filename of the main file as the argument of |\childdocmain|:
%
\begin{center}
\begin{tabular}{l}
|\input{childdoc.def}|\\
|\childdocmain{|\textit{main}|}|\\
\end{tabular}
\end{center}
%
If |\jobname| does not match the argument \textit{main} of |\childdocmain|,
it is assumed that |\jobname| points to the child file to be compiled.
When using |\childdocmain| with the main file specified as argument,
it suffices to start a child file
with just |\input{|\textit{main}|}|
without loading of the package and using |\childdocof|.
If instead all processing is done
with the appropriate \textsf{childdoc} directives,
the argument of \textit{main} of |\childdocmain| can be empty.

An alternative version of the command line processing described
in \secref{sec:commandline} using the detection mechanism reads:
%
\begin{center}
|... -jobname "|\textit{target}|" "|[\textit{flags}]%
[|\def\jobname{|\textit{dest}|}|]|\input{|\textit{main}|}"|
\end{center}

%%%%%%%%%%%%%%%%%%%%%%%%%%%%%%%%%%%%%%%%%%%%%%%%%%%%%%%%%%%%%%%%%%%%%%%%%%%%%%%%
\subsection{Manual Code}
\label{sec:manual}

In case one cannot be certain whether the definitions file |childdoc.def|
is installed on the target \TeX{} distribution
and one prefers not to ship it,
it is conceivable to paste a few relevant commands into the sources.

To that end, drop all statements |\input{childdoc.def}|
and perform the replacements as outlined below.
Instead of |\childdocmain{|\textit{main}|}| add the following code
to the top of the main file:
%
\begin{center}
\begin{tabular}{l}
|\||ifdefined\childdocname\endinput\||fi\newif\ifchilddoc|\\
|\edef\childdocname{\scantokens\expandafter{\jobname\noexpand}}|\\
|\def\childdocmain{|\textit{main}|}\||ifx\childdocmain\childdocname\||else|\\
|\childdoctrue\includeonly{\childdocname}\let\jobname\childdocmain\||fi|\\
\end{tabular}
\end{center}
%
Instead of |\childdocof{|\textit{main}|}| just include the main file
at the top of each child file:
%
\begin{center}
|\input{|\textit{main}|}|
\end{center}
%
A simple redirection |\childdocforward{|\textit{dest}|}| is achieved by:
%
\begin{center}
|\def\jobname{|\textit{dest}|}\input{\jobname}|
\end{center}
%
The redirection with prefix
|\childdocforwardprefix[|\textit{prefix}|]{|\textit{dest}|}|
is accomplished by:
%
\begin{center}
\begin{tabular}{l}
|{\edef\jobname{\scantokens\expandafter{\jobname\noexpand}}|\\
|\def\redirectjob |\textit{prefix}|#1~~~{\gdef\jobname{|\textit{dest}|#1}}|\\
|\expandafter\redirectjob\jobname~~~}\input{\jobname}|
\end{tabular}
\end{center}

In an alternative approach,
child documents can be compiled by a specific command line
without additional code or specific definitions:
%
\begin{center}
|... -jobname "|\textit{target}|" "|[\textit{flags}]%
|\includeonly{|\textit{dest}|}\input{|\textit{main}|}"|
\end{center}
%

%%%%%%%%%%%%%%%%%%%%%%%%%%%%%%%%%%%%%%%%%%%%%%%%%%%%%%%%%%%%%%%%%%%%%%%%%%%%%%%%
%%%%%%%%%%%%%%%%%%%%%%%%%%%%%%%%%%%%%%%%%%%%%%%%%%%%%%%%%%%%%%%%%%%%%%%%%%%%%%%%
\section{Information}

%%%%%%%%%%%%%%%%%%%%%%%%%%%%%%%%%%%%%%%%%%%%%%%%%%%%%%%%%%%%%%%%%%%%%%%%%%%%%%%%
\subsection{Copyright}

Copyright \copyright{} 2017--2018 Niklas Beisert

This work may be distributed and/or modified under the
conditions of the \LaTeX{} Project Public License, either version 1.3
of this license or (at your option) any later version.
The latest version of this license is in
  \url{http://www.latex-project.org/lppl.txt}
and version 1.3 or later is part of all distributions of \LaTeX{}
version 2005/12/01 or later.

This work has the LPPL maintenance status `maintained'.

The Current Maintainer of this work is Niklas Beisert.

This work consists of the files |README.txt|, |childdoc.ins| and |childdoc.dtx|
as well as the derived files |childdoc.def|, |cdocsamp.tex|
with |cdocsch1.tex|, |cdocsch2.tex|, |cdocspt3.tex|, |cdocspt4.tex|,
|cdocsdrf.tex|, |cdocsfn1.tex|, |cdocsfn2.tex|
as well as |childdoc.pdf|.

%%%%%%%%%%%%%%%%%%%%%%%%%%%%%%%%%%%%%%%%%%%%%%%%%%%%%%%%%%%%%%%%%%%%%%%%%%%%%%%%
\subsection{Files and Installation}

The package consists of the files:
%
\begin{center}
\begin{tabular}{ll}
    |README.txt|   & readme file \\
    |childdoc.ins| & installation file \\
    |childdoc.dtx| & source file \\
    |childdoc.def| & definition file \\
    |cdocsamp.tex| & sample main file \\
    |cdocsch1.tex| & sample include file \\
    |cdocsch2.tex| & sample include file \\
    |cdocspt3.tex| & sample part file \\
    |cdocspt4.tex| & sample part file \\
    |cdocsdrf.tex| & sample redirection file \\
    |cdocsfn1.tex| & sample redirection file \\
    |cdocsfn2.tex| & sample redirection file \\
    |childdoc.pdf| & manual
\end{tabular}
\end{center}
%
The distribution consists of the files
|README.txt|, |childdoc.ins| and |childdoc.dtx|.
%
\begin{itemize}
\item
Run (pdf)\LaTeX{} on |childdoc.dtx|
to compile the manual |childdoc.pdf| (this file).
\item
Run \LaTeX{} on |childdoc.ins| to create the definitions file |childdoc.def|
and the sample |cdocsamp.tex| with include files
|cdocsch1.tex|, |cdocsch2.tex|, |cdocspt3.tex|, |cdocspt4.tex|,
|cdocsdrf.tex|, |cdocsfn1.tex|, |cdocsfn2.tex|.
Then copy the file |childdoc.def| to an appropriate directory of your \LaTeX{}
distribution, e.g.\ \textit{texmf-root}|/tex/latex/childdoc|.
\end{itemize}

%%%%%%%%%%%%%%%%%%%%%%%%%%%%%%%%%%%%%%%%%%%%%%%%%%%%%%%%%%%%%%%%%%%%%%%%%%%%%%%%
\subsection{Related CTAN Packages}

There are several other packages which offer a similar functionality:
%
\begin{itemize}
\item
The packages
\href{http://ctan.org/pkg/docmute}{\textsf{docmute}},
\href{http://ctan.org/pkg/includex}{\textsf{includex}} and
\href{http://ctan.org/pkg/standalone}{\textsf{standalone}}
provide commands to include only the document body of
a child file thus allowing both files to be compiled individually.
\item
The packages \href{http://ctan.org/pkg/subdocs}{\textsf{subdocs}}
and \href{http://ctan.org/pkg/subfiles}{\textsf{subfiles}}
provide structures in which the main and child documents can be
encapsulated and allowing them to be compiled individually.
The inclusion mechanism is different from the conventional |\include|.
\item
The package \href{http://ctan.org/pkg/combine}{\textsf{combine}}
is an elaborate solution to combine several documents into one.
\end{itemize}
%
See also the CTAN topic \href{http://ctan.org/topic/subdocs}{\textsf{subdocs}}
for further related packages.
The present package differs from the above solutions in that
a document structure constructed with the conventional |\include| mechanism
just needs two extra commands at the top of every file
such that all constituent files can be compiled individually.

%%%%%%%%%%%%%%%%%%%%%%%%%%%%%%%%%%%%%%%%%%%%%%%%%%%%%%%%%%%%%%%%%%%%%%%%%%%%%%%%
%\subsection{Feature Suggestions}
%
%The following is a list of features which may be useful for future
%versions of this package:
%%
%\begin{itemize}
%\item
%\ldots
%\end{itemize}

%%%%%%%%%%%%%%%%%%%%%%%%%%%%%%%%%%%%%%%%%%%%%%%%%%%%%%%%%%%%%%%%%%%%%%%%%%%%%%%%
\subsection{Revision History}

%%%%%%%%%%%%%%%%%%%%%%%%%%%%%%%%%%%%%%%%
\paragraph{v2.0:} 2018/12/30

\begin{itemize}
\item
immediate forward processing
\item
added |\childdocby| mechanism
\item
manual restructured
\end{itemize}

%%%%%%%%%%%%%%%%%%%%%%%%%%%%%%%%%%%%%%%%
\paragraph{v1.6:} 2018/01/17

\begin{itemize}
\item
application for development of include files
\item
corrections to manual
\end{itemize}

%%%%%%%%%%%%%%%%%%%%%%%%%%%%%%%%%%%%%%%%
\paragraph{v1.5:} 2017/05/21

\begin{itemize}
\item
more complete structuring introduced
\item
|\childdocof| introduced
\item
|\childdoc| renamed to |\childdocmain|
\item
|\childredirect| renamed to |\childdocforward| and |\childdocforwardprefix|
and functionality expanded
\end{itemize}

%%%%%%%%%%%%%%%%%%%%%%%%%%%%%%%%%%%%%%%%
\paragraph{v1.0:} 2017/04/27

\begin{itemize}
\item
manual and install package
\item
first version published on CTAN
\end{itemize}

%%%%%%%%%%%%%%%%%%%%%%%%%%%%%%%%%%%%%%%%
\paragraph{v0.6:} 2017/04/26

\begin{itemize}
\item
redirection mechanism added
\end{itemize}

%%%%%%%%%%%%%%%%%%%%%%%%%%%%%%%%%%%%%%%%
\paragraph{v0.5:} 2017/04/26

\begin{itemize}
\item
functionality in definition file
\end{itemize}


%%%%%%%%%%%%%%%%%%%%%%%%%%%%%%%%%%%%%%%%%%%%%%%%%%%%%%%%%%%%%%%%%%%%%%%%%%%%%%%%
%%%%%%%%%%%%%%%%%%%%%%%%%%%%%%%%%%%%%%%%%%%%%%%%%%%%%%%%%%%%%%%%%%%%%%%%%%%%%%%%
%%%%%%%%%%%%%%%%%%%%%%%%%%%%%%%%%%%%%%%%%%%%%%%%%%%%%%%%%%%%%%%%%%%%%%%%%%%%%%%%
\appendix

\settowidth\MacroIndent{\rmfamily\scriptsize 000\ }

 \DocInput{childdoc.dtx}

\end{document}
%</driver>
% \fi
%
% %%%%%%%%%%%%%%%%%%%%%%%%%%%%%%%%%%%%%%%%%%%%%%%%%%%%%%%%%%%%%%%%%%%%%%%%%%%%%%
% %%%%%%%%%%%%%%%%%%%%%%%%%%%%%%%%%%%%%%%%%%%%%%%%%%%%%%%%%%%%%%%%%%%%%%%%%%%%%%
% \section{Sample}
%\iffalse
%<*samplemain>
%\fi
%
% The following presents a sample document
% with two chapters, two parts, a title page,
% a compile flag as well as three forwarding files to set the flag.
% It consists of eight |.tex| files:
% \begin{center}
% \begin{tabular}{ll}
% |cdocsamp.tex|&main file\\
% |cdocsch1.tex|&include file for chapter 1\\
% |cdocsch2.tex|&include file for chapter 2\\
% |cdocspt3.tex|&include file for part 3\\
% |cdocspt4.tex|&include file for part 4\\
% |cdocsdrf.tex|&forwarding file for main file in draft mode\\
% |cdocsfi1.tex|&forwarding file for final version of chapter 1\\
% |cdocsfi2.tex|&forwarding file for final version of chapter 2\\
% \end{tabular}
% \end{center}
% Each of the eight files can be compiled directly by the \LaTeX{} compiler.
%
% %%%%%%%%%%%%%%%%%%%%%%%%%%%%%%%%%%%%%%
% \paragraph{Main File.}
%
% The main file is called |cdocsamp.tex|.
%
% Load the \textsf{childdoc} definitions and
% declare the filename for the main document:
%    \begin{macrocode}
\input{childdoc.def}
\childdocmain{}
%    \end{macrocode}

% Optional override for |\version| flag:
%    \begin{macrocode}
%%\ifchilddoc\else\providecommand{\version}{draft}\fi
%    \end{macrocode}

% Define the default values for the |\version| flag
% (|final| for the main file and |draft| for childs):
%    \begin{macrocode}
\ifchilddoc
\providecommand{\version}{draft}
\else
\providecommand{\version}{final}
\fi
%    \end{macrocode}

% Load the standard document class:
%    \begin{macrocode}
\documentclass[12pt]{article}
%    \end{macrocode}

% Start the document body:
%    \begin{macrocode}
\begin{document}
%    \end{macrocode}

% Declare a title page.
% Print title, part of document being processed and version flag:
%    \begin{macrocode}
\addtocounter{page}{-1}
\begin{center}
{\LARGE\bfseries{}childdoc example\par}
\vspace{1cm}
\ifchilddoc
\ifchilddocmanual part\else chapter\fi:
`\childdocname' of `\childdocjob'\par
\else
main document: `\childdocjob'\par
\fi
version: \version\par
\end{center}
\newpage
%    \end{macrocode}

% Manually include selected file,
% otherwise process as usual:
%    \begin{macrocode}
\ifchilddocmanual
\section*{part `\childdocname'}
\input{\childdocname}
\else
%    \end{macrocode}

% Include the two chapters:
%    \begin{macrocode}
\include{cdocsch1}
\include{cdocsch2}
%    \end{macrocode}

% Include the two parts unless only chapters should be displayed:
%    \begin{macrocode}
\ifchilddoc\else
\section{part three}
\input{cdocspt3}
\section{part four}
\input{cdocspt4}
\fi
%    \end{macrocode}

% Process as usual until here:
%    \begin{macrocode}
\fi
%    \end{macrocode}

% End of document body:
%    \begin{macrocode}
\end{document}
%    \end{macrocode}
%\iffalse
%</samplemain>
%\fi
%
% %%%%%%%%%%%%%%%%%%%%%%%%%%%%%%%%%%%%%%
% \paragraph{Chapter Include Files.}
%
% The include files are called |cdocsch1.tex| and |cdocsch2.tex|.
%
%\iffalse
%<*samplechap1|samplechap2>
%\fi

% Optional override for |\version| flag:
%    \begin{macrocode}
%%\providecommand{\version}{final}
%    \end{macrocode}

% Include the main document:
%    \begin{macrocode}
\input{childdoc.def}
\childdocof{cdocsamp}
%    \end{macrocode}

%\iffalse
%</samplechap1|samplechap2>
%\fi
%
%\iffalse
%<*samplechap1>
%\fi
% Some text for chapter 1:
%    \begin{macrocode}
\section{one}
some text in chapter one
%    \end{macrocode}

%\iffalse
%</samplechap1>
%\fi
% Some text for chapter 2:
%\iffalse
%<*samplechap2>
%\fi
%    \begin{macrocode}
\section{two}
more text in chapter two
%    \end{macrocode}

%\iffalse
%</samplechap2>
%\fi
%
% %%%%%%%%%%%%%%%%%%%%%%%%%%%%%%%%%%%%%%
% \paragraph{Part Include Files.}
%
% The include files are called |cdocspt3.tex| and |cdocspt4.tex|.
%
%\iffalse
%<*samplepart3|samplepart4>
%\fi

% Optional override for |\version| flag:
%    \begin{macrocode}
%%\providecommand{\version}{final}
%    \end{macrocode}

% Include the main document:
%    \begin{macrocode}
\input{childdoc.def}
\childdocby{cdocsamp}
%    \end{macrocode}

%\iffalse
%</samplepart3|samplepart4>
%\fi
%
%\iffalse
%<*samplepart3>
%\fi
% Some text for part 3:
%    \begin{macrocode}
some text in part three
%    \end{macrocode}

%\iffalse
%</samplepart3>
%\fi
% Some text for part 4:
%\iffalse
%<*samplepart4>
%\fi
%    \begin{macrocode}
more text in part four
%    \end{macrocode}

%\iffalse
%</samplepart4>
%\fi
%
% %%%%%%%%%%%%%%%%%%%%%%%%%%%%%%%%%%%%%%
% \paragraph{Forwarding for a Complete Draft.}
%
% The following forwarding file |cdocsdrf.tex|
% compiles the main document in draft mode:
%\iffalse
%<*sampledraft>
%\fi
%    \begin{macrocode}
\def\version{draft}
\input{childdoc.def}
\childdocforward{cdocsamp}
%    \end{macrocode}

%\iffalse
%</sampledraft>
%\fi
%
% %%%%%%%%%%%%%%%%%%%%%%%%%%%%%%%%%%%%%%
% \paragraph{Forwarding for Final Version of the Chapters.}
%
% The following forwarding files |cdocsfn1.tex| and |cdocsfn2.tex|
% (with identical content)
% compile the final versions of the child documents
% |cdocsch1.tex| and |cdocsch2.tex|, respectively:
%\iffalse
%<*samplefinal>
%\fi
%    \begin{macrocode}
\def\version{final}
\input{childdoc.def}
\childdocforwardprefix[cdocsamp]{cdocsfn}{cdocsch}
%    \end{macrocode}

%\iffalse
%</samplefinal>
%\fi
%
% %%%%%%%%%%%%%%%%%%%%%%%%%%%%%%%%%%%%%%
% \paragraph{Command Line Processing.}
%
% The following three command lines generate the output files
% |cdocscld|, |cdocscl1| and |cdocscl2|
% which should be identical to
% |cdocsdrf|, |cdocsch1| and |cdocsfn2|, respectively:
% \begin{center}
% \begin{tabular}{l}
% |latex -jobname cdocscld \|\\
% |  "\def\version{draft}\input{childdoc.def}\childdocforward{cdocsamp}"|\\
% |latex -jobname cdocscl1 \|\\
% |  "\input{childdoc.def}\childdocforward[cdocsamp]{cdocsch1}"|\\
% |latex -jobname cdocscl2 \|\\
% |  "\def\version{final}\input{childdoc.def}\childdocforward{cdocsch2}"|
% \end{tabular}
% \end{center}
% Note that the trailing backslash on each first line
% merely continues the input to the second line
% (for convenient cut ant paste).
% Furthermore, the command |latex| can be replaced by any
% of its alternative versions such as |pdflatex|.
%
% %%%%%%%%%%%%%%%%%%%%%%%%%%%%%%%%%%%%%%%%%%%%%%%%%%%%%%%%%%%%%%%%%%%%%%%%%%%%%%
% %%%%%%%%%%%%%%%%%%%%%%%%%%%%%%%%%%%%%%%%%%%%%%%%%%%%%%%%%%%%%%%%%%%%%%%%%%%%%%
% \section{Implementation}
%\iffalse
%<*package>
%\fi
%
% This section describes the definitions file |childdoc.def|.

% The definitions cannot be loaded using |\usepackage| or |\RequirePackage|
% which has a mechanism to prevent loading a style file more than once.
% When loading the definitions by means of |\input|
% multiple instances have to be prevented manually:
%\iffalse
%This code needs to be before the `\ProvidesFile' directive
%which is defined at the beginning of this file.
%Therefore it is also placed there and commented out here.
%</package>
%<*discard>
%\fi
%    \begin{macrocode}
\ifdefined\childdocmain\endinput\fi
%    \end{macrocode}
%\iffalse
%</discard>
%<*package>
%\fi
%
% \macro{\ifchilddoc}
% \macro{\ifchilddocmanual}
% The conditional |\ifchilddoc| tells whether a
% child (true) or main (false) document is being compiled.
% The conditional |\ifchilddocmanual| tells whether
% the |\includeonly| mechanism is used (false) or
% the selection of child files must be performed manually (true).
% The definitions initialise to false:
%    \begin{macrocode}
\newif\ifchilddoc
\newif\ifchilddocmanual
%    \end{macrocode}

% \macro{\childdocname}
% \macro{\childdocjob}
% The macro |\childdocname| stores the name of the main document
% to be compiled. The macro |\childdocjob| stores the name of
% the document on which the \LaTeX{} compiler was originally invoked.
% The content of |\jobname| cannot be compared
% to filenames specified in the source due to different catcodes.
% The following code rescans |\jobname|, stores the result
% in |\childdocname| and saves a copy in |\childdocjob|:
%    \begin{macrocode}
\edef\childdocname{\scantokens\expandafter{\jobname\noexpand}}
\let\childdocjob\childdocname
%    \end{macrocode}

% \macro{\childdocdisable}
% The macro |\childdocdisable| prevents the main file
% from being processed more than once.
% At this stage, the main document command |\childdocmain|
% is assumed to be called once again where it should do nothing.
% Any subsequent call to it should prevent
% a secondary processing of the main document
% It overwrites the forwarding commands
% |\childdocof| and |\childdocforward|
% with empty macros to prevent further inclusions of the main document:
%    \begin{macrocode}
\newcommand{\childdocdisable}
{
  \renewcommand{\childdocmain}[1]{\renewcommand{\childdocmain}[1]{\endinput}}
  \renewcommand{\childdocof}[1]{}
  \renewcommand{\childdocby}[2][]{}
  \renewcommand{\childdocforward}[2][]{}
  \renewcommand{\childdocdisable}{}
}
%    \end{macrocode}

% \macro{\childdocmain}
% The macro |\childdocmain| is to be called at the top of the main file
% with nothing or the main filename (without extension) as argument.
% First, it breaks loops.
% If the argument is not empty and does not match |\childdocname|
% (which is set by the first inclusion of |childdoc.def|),
% |\ifchilddoc| is set to true, |\includeonly| is applied to the child file
% and |\jobname| is set to the main file
% (for proper handling of |.aux| files):
%    \begin{macrocode}
\newcommand{\childdocmain}[1]
{
  \childdocdisable\childdocmain{}
  \if?#1?\else
    \begingroup
      \def\childdoctmp{#1}
      \ifx\childdoctmp\childdocname
        \def\childdoctmp{}
      \else
        \def\childdoctmp
        {
          \childdoctrue
          \includeonly{\childdocname}
          \def\childdocjob{#1}
          \def\jobname{#1}
        }
      \fi
      \expandafter
    \endgroup
    \childdoctmp
  \fi
}
%    \end{macrocode}

% \macro{\childdocof}
% The command |\childdocof| redirects
% compilation to the main file |#1|.
%    \begin{macrocode}
\newcommand{\childdocof}[1]
{
  \childdocdisable
  \childdoctrue
  \includeonly{\childdocname}
  \def\jobname{#1}
  \def\childdocjob{#1}
  \input{#1}
}
%    \end{macrocode}

% \macro{\childdocby}
% The command |\childdocby| ....
%    \begin{macrocode}
\newcommand{\childdocby}[2][]
{
  \childdocdisable
  \childdoctrue
  \childdocmanualtrue
  \if?#1?\else
    \def\jobname{#2}
  \fi
  \def\childdocjob{#2}
  \input{#2}
  \endinput
}
%    \end{macrocode}

% \macro{\childdocforward}
% The command |\childdocforward| redirects
% compilation to the main file or
% (if the optional argument is given) a child file.
% Parameters are set as if the main file
% or a child file starting with |\childdocof| was compiled.
% Then compilation is handed over to the main file:
%    \begin{macrocode}
\newcommand{\childdocforward}[2][]
{
  \begingroup
    \if?#1?
      \def\childdoctmp
      {
        \def\childdocname{#2}
        \def\childdocjob{#2}
        \def\jobname{#2}
        \input{#2}
        \endinput
      }
    \else
      \def\childdoctmp
      {
        \childdocdisable
        \def\childdocname{#2}
        \childdoctrue
        \includeonly{#2}
        \def\childdocjob{#1}
        \def\jobname{#1}
        \input{#1}
        \endinput
      }
    \fi
    \expandafter
  \endgroup
  \childdoctmp
}
%    \end{macrocode}

% \macro{\childdocforwardprefix}
% The command |\childdocforwardprefix| redirects
% compilation to the main or a child file by means of a pattern.
% The prefix |#1| in the current filename is replaced by |#2|
% and the suffix of the current filename is kept
% (it is assumed that the filename does not contain the substring `|~~~|'
% which is used as a delimiter).
% Compilation is handed over to the new file by |\childdocforward|:
%    \begin{macrocode}
\newcommand{\childdocforwardprefix}[3][]
{
  \begingroup
    \def\childdocextract #2##1~~~{\def\childdoctmp{\childdocforward[#1]{#3##1}}}
    \expandafter\childdocextract\childdocname~~~
    \expandafter
  \endgroup
  \childdoctmp
}
%    \end{macrocode}

% \macro{\childdoc}
% The deprecated macro |\childdoc| is a legacy version of |\childdocmain|:
%    \begin{macrocode}
\newcommand{\childdoc}{\childdocmain}
%    \end{macrocode}

% \macro{\childdocredirect}
% The deprecated macro |\childdocredirect| is a legacy version
% of |\childdocforward| and |\childdocforwardprefix|:
%    \begin{macrocode}
\newcommand{\childdocredirect}[2][]
{
  \begingroup
    \if?#1?
      \def\childdoctmp{\childdocforward{#2}}
    \else
      \def\childdoctmp{\childdocforwardprefix{#1}{#2}}
    \fi
    \expandafter
  \endgroup
  \childdoctmp
}
%    \end{macrocode}

%\iffalse
%</package>
%\fi
%
\endinput
|\\
|\childdocforward{|\textit{main}|}|
\end{tabular}
\end{center}
%
Likewise, the following files |final|\textit{nn}|.tex|
compile the final version of the child document
|child|\textit{nn}|.tex|:
%
\begin{center}
\begin{tabular}{l}
|\def\version{final}|\\
|% \iffalse
%
% childdoc.dtx Copyright (C) 2017-2018 Niklas Beisert
%
% This work may be distributed and/or modified under the
% conditions of the LaTeX Project Public License, either version 1.3
% of this license or (at your option) any later version.
% The latest version of this license is in
%   http://www.latex-project.org/lppl.txt
% and version 1.3 or later is part of all distributions of LaTeX
% version 2005/12/01 or later.
%
% This work has the LPPL maintenance status `maintained'.
%
% The Current Maintainer of this work is Niklas Beisert.
%
% This work consists of the files childdoc.dtx and childdoc.ins
% and the derived files childdoc.def and cdocsamp.tex with
% cdocsch1.tex, cdocsch2.tex, cdocsdrf.tex, cdocsfn1.tex, cdocsfn2.tex.
%
%<package>\ifdefined\childdocmain\endinput\fi
%<package>\ProvidesFile{childdoc.def}[2018/12/30 v2.0 child document driver]
%<samplemain>\ProvidesFile{cdocsamp.tex}[2018/12/30 v2.0 sample for childdoc]
%<*driver>
%\ProvidesFile{childdoc.drv}[2018/12/30 v2.0 childdoc reference manual file]
\PassOptionsToClass{10pt,a4paper}{article}
\documentclass{ltxdoc}

\usepackage[margin=35mm]{geometry}
\usepackage{hyperref}
\usepackage{hyperxmp}
\usepackage[usenames]{color}

\hypersetup{colorlinks=true}
\hypersetup{pdfstartview=FitH}
\hypersetup{pdfpagemode=UseNone}
\hypersetup{pdfsource={}}
\hypersetup{pdflang={en-UK}}
\hypersetup{pdfcopyright={Copyright 2017-2018 Niklas Beisert.
  This work may be distributed and/or modified under the
  conditions of the LaTeX Project Public License, either version 1.3
  of this license or (at your option) any later version.}}
\hypersetup{pdflicenseurl={http://www.latex-project.org/lppl.txt}}
\hypersetup{pdfcontactaddress={ETH Zurich, ITP, HIT K,
  Wolfgang-Pauli-Strasse 27}}
\hypersetup{pdfcontactpostcode={8093}}
\hypersetup{pdfcontactcity={Zurich}}
\hypersetup{pdfcontactcountry={Switzerland}}
\hypersetup{pdfcontactemail={nbeisert@itp.phys.ethz.ch}}
\hypersetup{pdfcontacturl={http://people.phys.ethz.ch/\xmptilde nbeisert/}}

\newcommand{\secref}[1]{\hyperref[#1]{section \ref*{#1}}}

\parskip1ex
\parindent0pt
\let\olditemize\itemize
\def\itemize{\olditemize\parskip0pt}

\begin{document}

\title{The \textsf{childdoc} Package}
\hypersetup{pdftitle={The childdoc Package}}
\author{Niklas Beisert\\[2ex]
  Institut f\"ur Theoretische Physik\\
  Eidgen\"ossische Technische Hochschule Z\"urich\\
  Wolfgang-Pauli-Strasse 27, 8093 Z\"urich, Switzerland\\[1ex]
  \href{mailto:nbeisert@itp.phys.ethz.ch}
  {\texttt{nbeisert@itp.phys.ethz.ch}}}
\hypersetup{pdfauthor={Niklas Beisert}}
\hypersetup{pdfsubject={Manual for the LaTeX2e Package childdoc}}
\date{30 December 2018, \textsf{v2.0}}
\maketitle

\begin{abstract}\noindent
\textsf{childdoc} is a \LaTeXe{} package
that enables the direct compilation
of document sections included by |\include|
to individual files.
\end{abstract}

\begingroup
\parskip0ex
\tableofcontents
\endgroup

%%%%%%%%%%%%%%%%%%%%%%%%%%%%%%%%%%%%%%%%%%%%%%%%%%%%%%%%%%%%%%%%%%%%%%%%%%%%%%%%
%%%%%%%%%%%%%%%%%%%%%%%%%%%%%%%%%%%%%%%%%%%%%%%%%%%%%%%%%%%%%%%%%%%%%%%%%%%%%%%%
\section{Introduction}

\LaTeX{} provides a mechanism to structure a large document (such as a book)
into a main file and several child files (containing the chapters)
using the |\include| command.
This mechanism is beneficial for documents
which span hundreds of pages in order to
make the source file(s) more manageable.
Moreover, compilation can be restricted to
selected child files by means of the |\includeonly| command.
The latter feature can be used to reduce the compilation time while editing
(this was significantly more useful in the earlier days of \LaTeX{})
or to generate a smaller document which is easier to navigate.
Another application of |\includeonly| is to generate
documents consisting of selected parts of the complete document.

However, there are a few drawbacks of the plain |\include| mechanism:
\begin{itemize}
\item
The child files cannot be compiled on their own,
they can only be compiled via the main file.
A naive editing environment
(such as a text editor with an option
to have the current file processed by \LaTeX)
may require one to switch to the main file before compiling;
attempting to compile the child file produces errors.
\item
The main file must be modified (each time)
to adjust the |\includeonly| command
to the present needs. This easily leaves the main file in a messy state.
\item
The generated document will always carry the filename
of the main document. This is inconvenient if
several child files are to be compiled and
to be kept for distribution.
\end{itemize}

The present package provides a simple interface
to make child files individually compilable by \LaTeX{}.
Compiling a child file then has the same effect as compiling
the main file with an |\includeonly| command
to select the appropriate child.
Moreover the generated document will carry the name of the child
rather than the main file.
This resolves all three above issues.

This feature is meant to make the editing of books,
thesis documents and lecture notes somewhat more convenient.
However, the package can also be used efficiently for
composing a series of documents (such as exercise sheets)
which are typically distributed individually.
It then assists the author in generating the individual documents
(potentially in different versions)
as well as a document containing the collected series.
Another application is in developing style files
or other kinds of included material
where compilation of the style file could redirect
to a sample or test file.

%%%%%%%%%%%%%%%%%%%%%%%%%%%%%%%%%%%%%%%%%%%%%%%%%%%%%%%%%%%%%%%%%%%%%%%%%%%%%%%%
%%%%%%%%%%%%%%%%%%%%%%%%%%%%%%%%%%%%%%%%%%%%%%%%%%%%%%%%%%%%%%%%%%%%%%%%%%%%%%%%
\section{Usage}

First of all, the package \textsf{childdoc} is \emph{not} a standard
\LaTeXe{} |.sty| style file! Therefore it needs to be invoked in
a non-standard way.

%%%%%%%%%%%%%%%%%%%%%%%%%%%%%%%%%%%%%%%%%%%%%%%%%%%%%%%%%%%%%%%%%%%%%%%%%%%%%%%%
\subsection{Included Files}
\label{sec:include}

%%%%%%%%%%%%%%%%%%%%%%%%%%%%%%%%%%%%%%%%
\DescribeMacro{\childdocmain}
To use the package, add the commands
\begin{center}
\begin{tabular}{l}
|\input{childdoc.def}|\\
|\childdocmain{}|\\
\end{tabular}
\end{center}
at the very top of the main \LaTeX{} file,
in particular \emph{before} the |\documentclass| statement!
The argument of |\childdocmain| should be left empty
(but it must be present).

%%%%%%%%%%%%%%%%%%%%%%%%%%%%%%%%%%%%%%%%
\DescribeMacro{\childdocof}
Furthermore, add the commands
\begin{center}
\begin{tabular}{l}
|\input{childdoc.def}|\\
|\childdocof{|\textit{main}|}|\\
\end{tabular}
\end{center}
at the top of every child file \textit{child}
which is included by |\include{|\textit{child}|}|
from within the main file
(or at least for those files to be compiled individually).
The argument \textit{main} must be the filename of the main file.

There are a couple of
considerations in setting up the main and child documents:

%%%%%%%%%%%%%%%%%%%%%%%%%%%%%%%%%%%%%%%%
\paragraph{Restrictions.}

Please note the following restrictions:
\begin{itemize}
\item
|\childdocmain| must be called with one argument \textit{main}
to ensure compatibility with earlier version of the package.
It must either be empty (|\childdocmain{}|)
or precisely match the filename of the main file in which it is specified.
See \secref{sec:detection} for further information.
\item
The filename \textit{main} must be specified without the |.tex| extension.
\item
The filename \textit{main} is case sensitive
(even in case-insensitive file systems)
due to internal string comparison.
\item
The argument \textit{main} should be fully expanded, it cannot be a macro.
\item
Subdirectories and special characters should be avoided in filenames.
\item
The command |\childdocmain{|\textit{main}|}| must be followed by a whitespace.
It should not be followed immediately by another command
or by a comment mark `|%|'.
This is because the \TeX{} parser reads the token immediately following
the argument of |\childdocmain| and puts it
at the beginning of every child section;
however, a white\-space is ignored.
\end{itemize}

%%%%%%%%%%%%%%%%%%%%%%%%%%%%%%%%%%%%%%%%
\paragraph{Content of Main File.}

It is advisable to place all content in the child files included by |\include|.
Any output contained in the main file will appear in all child documents
unless suppressed manually;
it cannot be suppressed automatically by the |\includeonly| directive
and thus should normally be avoided.
A method to include some content in the main file
by means of conditional processing is described in \secref{sec:conditional}.

%%%%%%%%%%%%%%%%%%%%%%%%%%%%%%%%%%%%%%%%
\paragraph{Page Numbering.}

When only a part of the document is compiled,
the appropriate numbering of pages
(as well as other status parameters)
is determined from the |.aux| files.
The latter contain information from previous passes.
However this information needs to propagate through
all intermediate child documents.
Therefore the page numbering in child documents may well
be inconsistent until the complete document is compiled at least once.

A useful (if unconventional) way to always ensure a consistent
page numbering is to restart the numbering in each child document
and denote the pages by `\textit{child}|.|\textit{page}'
where \textit{child} represents the chapter/section number of the child file.
This can be achieved by the command
|\numberwithin{page}{|\textit{child}|}|
of the \textsf{amsmath} package
where \textit{child} can be |chapter| or |section|
depending on the chosen structuring.
Alternatively, one can modify the macro |\thepage| appropriately
and reset the counter |page| at the start of each child file.

%%%%%%%%%%%%%%%%%%%%%%%%%%%%%%%%%%%%%%%%%%%%%%%%%%%%%%%%%%%%%%%%%%%%%%%%%%%%%%%%
\subsection{Conditional Processing}
\label{sec:conditional}

The package provides a mechanism to compile different versions
of a document. To customise the versions further some conditional processing
can come in handy to distinguish which version is being compiled.
The package provides two macros to describe the compilation context:

%%%%%%%%%%%%%%%%%%%%%%%%%%%%%%%%%%%%%%%%
\DescribeMacro{\ifchilddoc}
The conditional |\ifchilddoc| distinguishes between the compilation of
child documents and the main document:
%
\begin{center}
|\ifchilddoc |\textit{child-code}| |[|\||else |\textit{main-code}]| \||fi|
\end{center}

%%%%%%%%%%%%%%%%%%%%%%%%%%%%%%%%%%%%%%%%
\DescribeMacro{\childdocname}
\DescribeMacro{\childdocjob}
The macro |\childdocname| contains the filename (without extension)
of the main or child file being processed.
Note that |\childdocjob| will always contain the name of the main file.

%%%%%%%%%%%%%%%%%%%%%%%%%%%%%%%%%%%%%%%%
\paragraph{Title Page.}

Conditional processing can be used to include a title or banner page
in the main document when proper precautions are taken.
Importantly, the code in the main file should ensure that the page counter
(as well as other status parameters which are stored in the |.aux| files)
takes the same value after the conditional processing.
Otherwise the page numbers may take divergent values
depending on which part is compiled.

For example, a title page could be declared by:
%
\begin{center}
\begin{tabular}{l}
|\ifchilddoc\||else|\\
|\addtocounter{page}{-1}|\\
\textit{code for title page}\\
|\newpage|\\
|\||fi|
\end{tabular}
\end{center}
%
A banner page for the child documents can be generated by:
%
\begin{center}
\begin{tabular}{l}
|\ifchilddoc|\\
|\addtocounter{page}{-1}|\\
\textit{code for banner page}\\
|\newpage|\\
|\||fi|
\end{tabular}
\end{center}
%
Here one could write a message such as:
\begin{center}
|This is the part \childdocname{} of \childdocjob{}.|
\end{center}

%%%%%%%%%%%%%%%%%%%%%%%%%%%%%%%%%%%%%%%%%%%%%%%%%%%%%%%%%%%%%%%%%%%%%%%%%%%%%%%%
\subsection{Flags}
\label{sec:flags}

The package makes it easy to generate different versions
of the main or child documents.
To this end compilation flags can be defined
and assigned different default values.
They will be particularly useful in conjunction
with the forwarding mechanism described in \secref{sec:forward}.

For example, it may be useful to have a flag |\version|
which can be set to |draft| or |final|.
The document source will contain some conditional code
depending on the value of |\version|.
Suppose further, the flag should default to |final| for the main file
and to |draft| for child files
which is a natural assignment for editing the document.
This is achieved by placing the following code
in the preamble of the main document
(below the |\childdocmain| directive):
%
\begin{center}
\begin{tabular}{l}
|\ifchilddoc|\\
|\providecommand{\version}{draft}|\\
|\||else|\\
|\providecommand{\version}{final}|\\
|\||fi|
\end{tabular}
\end{center}
%
The definition by |\providecommand| makes sure
that previous definitions are not overwritten.
Further statements |\providecommand{\version}{...}|
can thus be added before the above code to override it.

For the main file, one might add a line
(between |\childdocmain| and the above block)
%
\begin{center}
|%\ifchilddoc\||else\providecommand{\version}{draft}\||fi|
\end{center}
%
which can be uncommented to produce a draft version.
Likewise one can add a line to the very top of a child file
(above the |\childdocof{|\textit{main}|}| directive)
%
\begin{center}
|%\providecommand{\version}{final}|
\end{center}
%
which can be uncommented to produce the final version of this child document.

%%%%%%%%%%%%%%%%%%%%%%%%%%%%%%%%%%%%%%%%%%%%%%%%%%%%%%%%%%%%%%%%%%%%%%%%%%%%%%%%
\subsection{Forwarding}
\label{sec:forward}

Different versions of the main or child documents
using compilation flags as described in \secref{sec:flags}
can be (permanently) stored in different files
for convenient compilation, viewing and distribution.
To this end, the package defines a command
to pass on compilation to a different file:

%%%%%%%%%%%%%%%%%%%%%%%%%%%%%%%%%%%%%%%%
\DescribeMacro{\childdocforward}
The command |\childdocforward| redirects processing to
another source file:
%
\begin{center}
\begin{tabular}{l}
|\input{childdoc.def}|\\
|\childdocforward[|\textit{main}|]{|\textit{dest}|}|\\
\end{tabular}
\end{center}
%
The argument \textit{dest} is the destination file
(without extension).
It should be the main file or one of the child files.
Note that further \textsf{childdoc} directives
such as |\childdocof| and |\childdocforward|
in the indicated file will be processed in this form.
The optional argument \textit{main}
passes on directly to the main file \textit{main}
while pretending to compile the child \textit{dest}.
This form behaves as if \textit{dest}
issues |\childdocof{|\textit{main}|}| right away,
and no further \textsf{childdoc} directives will be processed.

%%%%%%%%%%%%%%%%%%%%%%%%%%%%%%%%%%%%%%%%
\DescribeMacro{\...prefix}
In the alternative form |\childdocforwardprefix|,
%
\begin{center}
\begin{tabular}{l}
|\input{childdoc.def}|\\
|\childdocforwardprefix[|\textit{main}|]{|\textit{prefix}|}{|\textit{dest}|}|
\end{tabular}
\end{center}
%
the destination file is determined by a pattern
depending on the current file:
To make this work, the current file must be called
`{\textit{prefix}\hspace{0.2em}\textit{suffix}}'
with \textit{prefix} matching precisely the argument.
Processing is then passed on to the file
`{\textit{dest}\hspace{0.2em}\textit{suffix}}'.
Surely, the same effect is achieved by
directly specifying the
argument `{\textit{dest}\hspace{0.2em}\textit{suffix}}'
in the first form.
However, that requires to set up a different file
for each child. With the alternative form of the command
all these files can have exactly the same content
which simplifies setting them up and maintaining them.

For example, the following file |draft.tex|
with a compilation flag |\version| as described in \secref{sec:flags}
compiles the main document as a draft:
%
\begin{center}
\begin{tabular}{l}
|\def\version{draft}|\\
|\input{childdoc.def}|\\
|\childdocforward{|\textit{main}|}|
\end{tabular}
\end{center}
%
Likewise, the following files |final|\textit{nn}|.tex|
compile the final version of the child document
|child|\textit{nn}|.tex|:
%
\begin{center}
\begin{tabular}{l}
|\def\version{final}|\\
|\input{childdoc.def}|\\
|\childdocforwardprefix{final}{child}|
\end{tabular}
\end{center}
%

Note that when several versions of a main file and/or of each child file
are to be generated, it may be convenient to set up a |Makefile| or
shell script to automatise the process.

%%%%%%%%%%%%%%%%%%%%%%%%%%%%%%%%%%%%%%%%%%%%%%%%%%%%%%%%%%%%%%%%%%%%%%%%%%%%%%%%
\subsection{Command Line Processing}
\label{sec:commandline}

The effect of redirection files can also be achieved by invoking
the \LaTeX{} compiler with a more elaborate command line.
Most conveniently this should be done as part
of a shell script or a |Makefile|.

When using \textsf{childdoc} in the main file, the following
command lines effectively perform a redirection
(note that depending on the shell being used,
backslashes may have to be doubled: `|\|' $\to$ `|\\|'):
%
\begin{center}
|... -jobname "|\textit{target}|" |\\|"|[\textit{flags}]%
|\input{childdoc.def}\childdocforward[|\textit{main}|]{|\textit{dest}|}"|
\end{center}
%
Here \textit{target} is the name of the output file,
\textit{main} is the name of the main file
and \textit{dest} is the name of the main or child file to be processed
(all filenames without extensions).
The optional argument \textit{main} can be omitted
if \textit{main} matches \textit{dest}.
Optionally, compilation \textit{flags} can be defined via |\def| commands.
This command line makes the \TeX{} engine believe
it is compiling the file \textit{target}
whose content is specified as the latter parameter.
The provided code then forwards the processing to
\textit{main} or \textit{dest} as described in \secref{sec:forward}.

%%%%%%%%%%%%%%%%%%%%%%%%%%%%%%%%%%%%%%%%%%%%%%%%%%%%%%%%%%%%%%%%%%%%%%%%%%%%%%%%
\subsection{Include by Input}
\label{sec:input}

Including child documents by |\include| has some restrictions by design.
Most notably, the content of a child document always occupies
its own set of pages; pages cannot be shared between child documents.
Usually, this behaviour makes perfect sense
because each child document contain an essential part of the document.
However, in some situations it may be desirable to compose
a document from a collection of parts
without having mandatory page breaks between then.
For this case, the package
provides a mechanism to include parts
by |\input| which can also be processed individually.
However, by construction this mechanism
requires manual handling of the content to be output.

%%%%%%%%%%%%%%%%%%%%%%%%%%%%%%%%%%%%%%%%
\DescribeMacro{\ifchilddocmanual}
The main file should be prepared as usual, see \secref{sec:include}.
However, the document body must make a distinction
between processing of an individual part and of the main document, e.g.:
%
\begin{center}
\begin{tabular}{l}
|\ifchilddocmanual|\\
|\input{\childdocname}|\\
|\||else|\\
\textit{document body with }|\input{|\textit{part}|}|\\
|\||fi|
\end{tabular}
\end{center}
%
The conditional |\ifchilddocmanual| is true whenever
a part to be included by |\input| is being compiled,
and the name of the part is stored in |\childdocname|.

%%%%%%%%%%%%%%%%%%%%%%%%%%%%%%%%%%%%%%%%
\DescribeMacro{\childdocby}
Each part to be included by |\input| should start with:
%
\begin{center}
\begin{tabular}{l}
|\input{childdoc.def}|\\
|\childdocby{|\textit{main}|}|\\
\end{tabular}
\end{center}
%
The directive |\childdocby| is similar to |\childdocof|
described in \secref{sec:include},
but the subsequent selection of content must be done manually.
To that end, both |\ifchilddoc| and |\ifchilddocmanual|
will be true upon processing of a part,
and the name of the part is stored in |\childdocname|.
Note that |\jobname| will be set to the filename of the current part
so that each part receives an individual |.aux| file
that does not interfere with the |.aux| file(s) of the main document.
This behaviour can be altered by the alternative form
|\childdocby[*]{|\textit{main}|}| (with a non-empty optional argument)
which uses the |.aux| file of the main document
by setting |\jobname| to \textit{main}.

%%%%%%%%%%%%%%%%%%%%%%%%%%%%%%%%%%%%%%%%%%%%%%%%%%%%%%%%%%%%%%%%%%%%%%%%%%%%%%%%
\subsection{Driver Development}
\label{sec:driver}

The \textsf{childdoc} mechanism can also be use for the development
of definition files such as \LaTeX{} styles or classes.
This case differs from the above setup with multiple parts
included by |\include| in that no |\includeonly| should be invoked.
This can be achieved by starting the include file
(before |\ProvidesPackage|) with:
%
\begin{center}
\begin{tabular}{l}
|\input{childdoc.def}|\\
|\childdocforward{|\textit{main}|}|\\
\end{tabular}
\end{center}
%
or alternatively with:
%
\begin{center}
\begin{tabular}{l}
|\input{childdoc.def}|\\
|\childdocby{|\textit{main}|}|\\
\end{tabular}
\end{center}
%
Both forms have slightly different effects as described above.
The main file is prepared as usual, see \secref{sec:include}.

%%%%%%%%%%%%%%%%%%%%%%%%%%%%%%%%%%%%%%%%%%%%%%%%%%%%%%%%%%%%%%%%%%%%%%%%%%%%%%%%
\subsection{Legacy Detection}
\label{sec:detection}

The directive |\childdocmain| in the main file can detect
whether the complete document or merely a child is to be compiled
even without using the directive |\childdocof|.
This method is deprecated because it is less robust
and there is no compelling reason to use it;
it is merely provided for backward compatibility
and it may be removed in future versions.

If the detection mechanism is to be used,
it is mandatory to correctly specify
the filename of the main file as the argument of |\childdocmain|:
%
\begin{center}
\begin{tabular}{l}
|\input{childdoc.def}|\\
|\childdocmain{|\textit{main}|}|\\
\end{tabular}
\end{center}
%
If |\jobname| does not match the argument \textit{main} of |\childdocmain|,
it is assumed that |\jobname| points to the child file to be compiled.
When using |\childdocmain| with the main file specified as argument,
it suffices to start a child file
with just |\input{|\textit{main}|}|
without loading of the package and using |\childdocof|.
If instead all processing is done
with the appropriate \textsf{childdoc} directives,
the argument of \textit{main} of |\childdocmain| can be empty.

An alternative version of the command line processing described
in \secref{sec:commandline} using the detection mechanism reads:
%
\begin{center}
|... -jobname "|\textit{target}|" "|[\textit{flags}]%
[|\def\jobname{|\textit{dest}|}|]|\input{|\textit{main}|}"|
\end{center}

%%%%%%%%%%%%%%%%%%%%%%%%%%%%%%%%%%%%%%%%%%%%%%%%%%%%%%%%%%%%%%%%%%%%%%%%%%%%%%%%
\subsection{Manual Code}
\label{sec:manual}

In case one cannot be certain whether the definitions file |childdoc.def|
is installed on the target \TeX{} distribution
and one prefers not to ship it,
it is conceivable to paste a few relevant commands into the sources.

To that end, drop all statements |\input{childdoc.def}|
and perform the replacements as outlined below.
Instead of |\childdocmain{|\textit{main}|}| add the following code
to the top of the main file:
%
\begin{center}
\begin{tabular}{l}
|\||ifdefined\childdocname\endinput\||fi\newif\ifchilddoc|\\
|\edef\childdocname{\scantokens\expandafter{\jobname\noexpand}}|\\
|\def\childdocmain{|\textit{main}|}\||ifx\childdocmain\childdocname\||else|\\
|\childdoctrue\includeonly{\childdocname}\let\jobname\childdocmain\||fi|\\
\end{tabular}
\end{center}
%
Instead of |\childdocof{|\textit{main}|}| just include the main file
at the top of each child file:
%
\begin{center}
|\input{|\textit{main}|}|
\end{center}
%
A simple redirection |\childdocforward{|\textit{dest}|}| is achieved by:
%
\begin{center}
|\def\jobname{|\textit{dest}|}\input{\jobname}|
\end{center}
%
The redirection with prefix
|\childdocforwardprefix[|\textit{prefix}|]{|\textit{dest}|}|
is accomplished by:
%
\begin{center}
\begin{tabular}{l}
|{\edef\jobname{\scantokens\expandafter{\jobname\noexpand}}|\\
|\def\redirectjob |\textit{prefix}|#1~~~{\gdef\jobname{|\textit{dest}|#1}}|\\
|\expandafter\redirectjob\jobname~~~}\input{\jobname}|
\end{tabular}
\end{center}

In an alternative approach,
child documents can be compiled by a specific command line
without additional code or specific definitions:
%
\begin{center}
|... -jobname "|\textit{target}|" "|[\textit{flags}]%
|\includeonly{|\textit{dest}|}\input{|\textit{main}|}"|
\end{center}
%

%%%%%%%%%%%%%%%%%%%%%%%%%%%%%%%%%%%%%%%%%%%%%%%%%%%%%%%%%%%%%%%%%%%%%%%%%%%%%%%%
%%%%%%%%%%%%%%%%%%%%%%%%%%%%%%%%%%%%%%%%%%%%%%%%%%%%%%%%%%%%%%%%%%%%%%%%%%%%%%%%
\section{Information}

%%%%%%%%%%%%%%%%%%%%%%%%%%%%%%%%%%%%%%%%%%%%%%%%%%%%%%%%%%%%%%%%%%%%%%%%%%%%%%%%
\subsection{Copyright}

Copyright \copyright{} 2017--2018 Niklas Beisert

This work may be distributed and/or modified under the
conditions of the \LaTeX{} Project Public License, either version 1.3
of this license or (at your option) any later version.
The latest version of this license is in
  \url{http://www.latex-project.org/lppl.txt}
and version 1.3 or later is part of all distributions of \LaTeX{}
version 2005/12/01 or later.

This work has the LPPL maintenance status `maintained'.

The Current Maintainer of this work is Niklas Beisert.

This work consists of the files |README.txt|, |childdoc.ins| and |childdoc.dtx|
as well as the derived files |childdoc.def|, |cdocsamp.tex|
with |cdocsch1.tex|, |cdocsch2.tex|, |cdocspt3.tex|, |cdocspt4.tex|,
|cdocsdrf.tex|, |cdocsfn1.tex|, |cdocsfn2.tex|
as well as |childdoc.pdf|.

%%%%%%%%%%%%%%%%%%%%%%%%%%%%%%%%%%%%%%%%%%%%%%%%%%%%%%%%%%%%%%%%%%%%%%%%%%%%%%%%
\subsection{Files and Installation}

The package consists of the files:
%
\begin{center}
\begin{tabular}{ll}
    |README.txt|   & readme file \\
    |childdoc.ins| & installation file \\
    |childdoc.dtx| & source file \\
    |childdoc.def| & definition file \\
    |cdocsamp.tex| & sample main file \\
    |cdocsch1.tex| & sample include file \\
    |cdocsch2.tex| & sample include file \\
    |cdocspt3.tex| & sample part file \\
    |cdocspt4.tex| & sample part file \\
    |cdocsdrf.tex| & sample redirection file \\
    |cdocsfn1.tex| & sample redirection file \\
    |cdocsfn2.tex| & sample redirection file \\
    |childdoc.pdf| & manual
\end{tabular}
\end{center}
%
The distribution consists of the files
|README.txt|, |childdoc.ins| and |childdoc.dtx|.
%
\begin{itemize}
\item
Run (pdf)\LaTeX{} on |childdoc.dtx|
to compile the manual |childdoc.pdf| (this file).
\item
Run \LaTeX{} on |childdoc.ins| to create the definitions file |childdoc.def|
and the sample |cdocsamp.tex| with include files
|cdocsch1.tex|, |cdocsch2.tex|, |cdocspt3.tex|, |cdocspt4.tex|,
|cdocsdrf.tex|, |cdocsfn1.tex|, |cdocsfn2.tex|.
Then copy the file |childdoc.def| to an appropriate directory of your \LaTeX{}
distribution, e.g.\ \textit{texmf-root}|/tex/latex/childdoc|.
\end{itemize}

%%%%%%%%%%%%%%%%%%%%%%%%%%%%%%%%%%%%%%%%%%%%%%%%%%%%%%%%%%%%%%%%%%%%%%%%%%%%%%%%
\subsection{Related CTAN Packages}

There are several other packages which offer a similar functionality:
%
\begin{itemize}
\item
The packages
\href{http://ctan.org/pkg/docmute}{\textsf{docmute}},
\href{http://ctan.org/pkg/includex}{\textsf{includex}} and
\href{http://ctan.org/pkg/standalone}{\textsf{standalone}}
provide commands to include only the document body of
a child file thus allowing both files to be compiled individually.
\item
The packages \href{http://ctan.org/pkg/subdocs}{\textsf{subdocs}}
and \href{http://ctan.org/pkg/subfiles}{\textsf{subfiles}}
provide structures in which the main and child documents can be
encapsulated and allowing them to be compiled individually.
The inclusion mechanism is different from the conventional |\include|.
\item
The package \href{http://ctan.org/pkg/combine}{\textsf{combine}}
is an elaborate solution to combine several documents into one.
\end{itemize}
%
See also the CTAN topic \href{http://ctan.org/topic/subdocs}{\textsf{subdocs}}
for further related packages.
The present package differs from the above solutions in that
a document structure constructed with the conventional |\include| mechanism
just needs two extra commands at the top of every file
such that all constituent files can be compiled individually.

%%%%%%%%%%%%%%%%%%%%%%%%%%%%%%%%%%%%%%%%%%%%%%%%%%%%%%%%%%%%%%%%%%%%%%%%%%%%%%%%
%\subsection{Feature Suggestions}
%
%The following is a list of features which may be useful for future
%versions of this package:
%%
%\begin{itemize}
%\item
%\ldots
%\end{itemize}

%%%%%%%%%%%%%%%%%%%%%%%%%%%%%%%%%%%%%%%%%%%%%%%%%%%%%%%%%%%%%%%%%%%%%%%%%%%%%%%%
\subsection{Revision History}

%%%%%%%%%%%%%%%%%%%%%%%%%%%%%%%%%%%%%%%%
\paragraph{v2.0:} 2018/12/30

\begin{itemize}
\item
immediate forward processing
\item
added |\childdocby| mechanism
\item
manual restructured
\end{itemize}

%%%%%%%%%%%%%%%%%%%%%%%%%%%%%%%%%%%%%%%%
\paragraph{v1.6:} 2018/01/17

\begin{itemize}
\item
application for development of include files
\item
corrections to manual
\end{itemize}

%%%%%%%%%%%%%%%%%%%%%%%%%%%%%%%%%%%%%%%%
\paragraph{v1.5:} 2017/05/21

\begin{itemize}
\item
more complete structuring introduced
\item
|\childdocof| introduced
\item
|\childdoc| renamed to |\childdocmain|
\item
|\childredirect| renamed to |\childdocforward| and |\childdocforwardprefix|
and functionality expanded
\end{itemize}

%%%%%%%%%%%%%%%%%%%%%%%%%%%%%%%%%%%%%%%%
\paragraph{v1.0:} 2017/04/27

\begin{itemize}
\item
manual and install package
\item
first version published on CTAN
\end{itemize}

%%%%%%%%%%%%%%%%%%%%%%%%%%%%%%%%%%%%%%%%
\paragraph{v0.6:} 2017/04/26

\begin{itemize}
\item
redirection mechanism added
\end{itemize}

%%%%%%%%%%%%%%%%%%%%%%%%%%%%%%%%%%%%%%%%
\paragraph{v0.5:} 2017/04/26

\begin{itemize}
\item
functionality in definition file
\end{itemize}


%%%%%%%%%%%%%%%%%%%%%%%%%%%%%%%%%%%%%%%%%%%%%%%%%%%%%%%%%%%%%%%%%%%%%%%%%%%%%%%%
%%%%%%%%%%%%%%%%%%%%%%%%%%%%%%%%%%%%%%%%%%%%%%%%%%%%%%%%%%%%%%%%%%%%%%%%%%%%%%%%
%%%%%%%%%%%%%%%%%%%%%%%%%%%%%%%%%%%%%%%%%%%%%%%%%%%%%%%%%%%%%%%%%%%%%%%%%%%%%%%%
\appendix

\settowidth\MacroIndent{\rmfamily\scriptsize 000\ }

 \DocInput{childdoc.dtx}

\end{document}
%</driver>
% \fi
%
% %%%%%%%%%%%%%%%%%%%%%%%%%%%%%%%%%%%%%%%%%%%%%%%%%%%%%%%%%%%%%%%%%%%%%%%%%%%%%%
% %%%%%%%%%%%%%%%%%%%%%%%%%%%%%%%%%%%%%%%%%%%%%%%%%%%%%%%%%%%%%%%%%%%%%%%%%%%%%%
% \section{Sample}
%\iffalse
%<*samplemain>
%\fi
%
% The following presents a sample document
% with two chapters, two parts, a title page,
% a compile flag as well as three forwarding files to set the flag.
% It consists of eight |.tex| files:
% \begin{center}
% \begin{tabular}{ll}
% |cdocsamp.tex|&main file\\
% |cdocsch1.tex|&include file for chapter 1\\
% |cdocsch2.tex|&include file for chapter 2\\
% |cdocspt3.tex|&include file for part 3\\
% |cdocspt4.tex|&include file for part 4\\
% |cdocsdrf.tex|&forwarding file for main file in draft mode\\
% |cdocsfi1.tex|&forwarding file for final version of chapter 1\\
% |cdocsfi2.tex|&forwarding file for final version of chapter 2\\
% \end{tabular}
% \end{center}
% Each of the eight files can be compiled directly by the \LaTeX{} compiler.
%
% %%%%%%%%%%%%%%%%%%%%%%%%%%%%%%%%%%%%%%
% \paragraph{Main File.}
%
% The main file is called |cdocsamp.tex|.
%
% Load the \textsf{childdoc} definitions and
% declare the filename for the main document:
%    \begin{macrocode}
\input{childdoc.def}
\childdocmain{}
%    \end{macrocode}

% Optional override for |\version| flag:
%    \begin{macrocode}
%%\ifchilddoc\else\providecommand{\version}{draft}\fi
%    \end{macrocode}

% Define the default values for the |\version| flag
% (|final| for the main file and |draft| for childs):
%    \begin{macrocode}
\ifchilddoc
\providecommand{\version}{draft}
\else
\providecommand{\version}{final}
\fi
%    \end{macrocode}

% Load the standard document class:
%    \begin{macrocode}
\documentclass[12pt]{article}
%    \end{macrocode}

% Start the document body:
%    \begin{macrocode}
\begin{document}
%    \end{macrocode}

% Declare a title page.
% Print title, part of document being processed and version flag:
%    \begin{macrocode}
\addtocounter{page}{-1}
\begin{center}
{\LARGE\bfseries{}childdoc example\par}
\vspace{1cm}
\ifchilddoc
\ifchilddocmanual part\else chapter\fi:
`\childdocname' of `\childdocjob'\par
\else
main document: `\childdocjob'\par
\fi
version: \version\par
\end{center}
\newpage
%    \end{macrocode}

% Manually include selected file,
% otherwise process as usual:
%    \begin{macrocode}
\ifchilddocmanual
\section*{part `\childdocname'}
\input{\childdocname}
\else
%    \end{macrocode}

% Include the two chapters:
%    \begin{macrocode}
\include{cdocsch1}
\include{cdocsch2}
%    \end{macrocode}

% Include the two parts unless only chapters should be displayed:
%    \begin{macrocode}
\ifchilddoc\else
\section{part three}
\input{cdocspt3}
\section{part four}
\input{cdocspt4}
\fi
%    \end{macrocode}

% Process as usual until here:
%    \begin{macrocode}
\fi
%    \end{macrocode}

% End of document body:
%    \begin{macrocode}
\end{document}
%    \end{macrocode}
%\iffalse
%</samplemain>
%\fi
%
% %%%%%%%%%%%%%%%%%%%%%%%%%%%%%%%%%%%%%%
% \paragraph{Chapter Include Files.}
%
% The include files are called |cdocsch1.tex| and |cdocsch2.tex|.
%
%\iffalse
%<*samplechap1|samplechap2>
%\fi

% Optional override for |\version| flag:
%    \begin{macrocode}
%%\providecommand{\version}{final}
%    \end{macrocode}

% Include the main document:
%    \begin{macrocode}
\input{childdoc.def}
\childdocof{cdocsamp}
%    \end{macrocode}

%\iffalse
%</samplechap1|samplechap2>
%\fi
%
%\iffalse
%<*samplechap1>
%\fi
% Some text for chapter 1:
%    \begin{macrocode}
\section{one}
some text in chapter one
%    \end{macrocode}

%\iffalse
%</samplechap1>
%\fi
% Some text for chapter 2:
%\iffalse
%<*samplechap2>
%\fi
%    \begin{macrocode}
\section{two}
more text in chapter two
%    \end{macrocode}

%\iffalse
%</samplechap2>
%\fi
%
% %%%%%%%%%%%%%%%%%%%%%%%%%%%%%%%%%%%%%%
% \paragraph{Part Include Files.}
%
% The include files are called |cdocspt3.tex| and |cdocspt4.tex|.
%
%\iffalse
%<*samplepart3|samplepart4>
%\fi

% Optional override for |\version| flag:
%    \begin{macrocode}
%%\providecommand{\version}{final}
%    \end{macrocode}

% Include the main document:
%    \begin{macrocode}
\input{childdoc.def}
\childdocby{cdocsamp}
%    \end{macrocode}

%\iffalse
%</samplepart3|samplepart4>
%\fi
%
%\iffalse
%<*samplepart3>
%\fi
% Some text for part 3:
%    \begin{macrocode}
some text in part three
%    \end{macrocode}

%\iffalse
%</samplepart3>
%\fi
% Some text for part 4:
%\iffalse
%<*samplepart4>
%\fi
%    \begin{macrocode}
more text in part four
%    \end{macrocode}

%\iffalse
%</samplepart4>
%\fi
%
% %%%%%%%%%%%%%%%%%%%%%%%%%%%%%%%%%%%%%%
% \paragraph{Forwarding for a Complete Draft.}
%
% The following forwarding file |cdocsdrf.tex|
% compiles the main document in draft mode:
%\iffalse
%<*sampledraft>
%\fi
%    \begin{macrocode}
\def\version{draft}
\input{childdoc.def}
\childdocforward{cdocsamp}
%    \end{macrocode}

%\iffalse
%</sampledraft>
%\fi
%
% %%%%%%%%%%%%%%%%%%%%%%%%%%%%%%%%%%%%%%
% \paragraph{Forwarding for Final Version of the Chapters.}
%
% The following forwarding files |cdocsfn1.tex| and |cdocsfn2.tex|
% (with identical content)
% compile the final versions of the child documents
% |cdocsch1.tex| and |cdocsch2.tex|, respectively:
%\iffalse
%<*samplefinal>
%\fi
%    \begin{macrocode}
\def\version{final}
\input{childdoc.def}
\childdocforwardprefix[cdocsamp]{cdocsfn}{cdocsch}
%    \end{macrocode}

%\iffalse
%</samplefinal>
%\fi
%
% %%%%%%%%%%%%%%%%%%%%%%%%%%%%%%%%%%%%%%
% \paragraph{Command Line Processing.}
%
% The following three command lines generate the output files
% |cdocscld|, |cdocscl1| and |cdocscl2|
% which should be identical to
% |cdocsdrf|, |cdocsch1| and |cdocsfn2|, respectively:
% \begin{center}
% \begin{tabular}{l}
% |latex -jobname cdocscld \|\\
% |  "\def\version{draft}\input{childdoc.def}\childdocforward{cdocsamp}"|\\
% |latex -jobname cdocscl1 \|\\
% |  "\input{childdoc.def}\childdocforward[cdocsamp]{cdocsch1}"|\\
% |latex -jobname cdocscl2 \|\\
% |  "\def\version{final}\input{childdoc.def}\childdocforward{cdocsch2}"|
% \end{tabular}
% \end{center}
% Note that the trailing backslash on each first line
% merely continues the input to the second line
% (for convenient cut ant paste).
% Furthermore, the command |latex| can be replaced by any
% of its alternative versions such as |pdflatex|.
%
% %%%%%%%%%%%%%%%%%%%%%%%%%%%%%%%%%%%%%%%%%%%%%%%%%%%%%%%%%%%%%%%%%%%%%%%%%%%%%%
% %%%%%%%%%%%%%%%%%%%%%%%%%%%%%%%%%%%%%%%%%%%%%%%%%%%%%%%%%%%%%%%%%%%%%%%%%%%%%%
% \section{Implementation}
%\iffalse
%<*package>
%\fi
%
% This section describes the definitions file |childdoc.def|.

% The definitions cannot be loaded using |\usepackage| or |\RequirePackage|
% which has a mechanism to prevent loading a style file more than once.
% When loading the definitions by means of |\input|
% multiple instances have to be prevented manually:
%\iffalse
%This code needs to be before the `\ProvidesFile' directive
%which is defined at the beginning of this file.
%Therefore it is also placed there and commented out here.
%</package>
%<*discard>
%\fi
%    \begin{macrocode}
\ifdefined\childdocmain\endinput\fi
%    \end{macrocode}
%\iffalse
%</discard>
%<*package>
%\fi
%
% \macro{\ifchilddoc}
% \macro{\ifchilddocmanual}
% The conditional |\ifchilddoc| tells whether a
% child (true) or main (false) document is being compiled.
% The conditional |\ifchilddocmanual| tells whether
% the |\includeonly| mechanism is used (false) or
% the selection of child files must be performed manually (true).
% The definitions initialise to false:
%    \begin{macrocode}
\newif\ifchilddoc
\newif\ifchilddocmanual
%    \end{macrocode}

% \macro{\childdocname}
% \macro{\childdocjob}
% The macro |\childdocname| stores the name of the main document
% to be compiled. The macro |\childdocjob| stores the name of
% the document on which the \LaTeX{} compiler was originally invoked.
% The content of |\jobname| cannot be compared
% to filenames specified in the source due to different catcodes.
% The following code rescans |\jobname|, stores the result
% in |\childdocname| and saves a copy in |\childdocjob|:
%    \begin{macrocode}
\edef\childdocname{\scantokens\expandafter{\jobname\noexpand}}
\let\childdocjob\childdocname
%    \end{macrocode}

% \macro{\childdocdisable}
% The macro |\childdocdisable| prevents the main file
% from being processed more than once.
% At this stage, the main document command |\childdocmain|
% is assumed to be called once again where it should do nothing.
% Any subsequent call to it should prevent
% a secondary processing of the main document
% It overwrites the forwarding commands
% |\childdocof| and |\childdocforward|
% with empty macros to prevent further inclusions of the main document:
%    \begin{macrocode}
\newcommand{\childdocdisable}
{
  \renewcommand{\childdocmain}[1]{\renewcommand{\childdocmain}[1]{\endinput}}
  \renewcommand{\childdocof}[1]{}
  \renewcommand{\childdocby}[2][]{}
  \renewcommand{\childdocforward}[2][]{}
  \renewcommand{\childdocdisable}{}
}
%    \end{macrocode}

% \macro{\childdocmain}
% The macro |\childdocmain| is to be called at the top of the main file
% with nothing or the main filename (without extension) as argument.
% First, it breaks loops.
% If the argument is not empty and does not match |\childdocname|
% (which is set by the first inclusion of |childdoc.def|),
% |\ifchilddoc| is set to true, |\includeonly| is applied to the child file
% and |\jobname| is set to the main file
% (for proper handling of |.aux| files):
%    \begin{macrocode}
\newcommand{\childdocmain}[1]
{
  \childdocdisable\childdocmain{}
  \if?#1?\else
    \begingroup
      \def\childdoctmp{#1}
      \ifx\childdoctmp\childdocname
        \def\childdoctmp{}
      \else
        \def\childdoctmp
        {
          \childdoctrue
          \includeonly{\childdocname}
          \def\childdocjob{#1}
          \def\jobname{#1}
        }
      \fi
      \expandafter
    \endgroup
    \childdoctmp
  \fi
}
%    \end{macrocode}

% \macro{\childdocof}
% The command |\childdocof| redirects
% compilation to the main file |#1|.
%    \begin{macrocode}
\newcommand{\childdocof}[1]
{
  \childdocdisable
  \childdoctrue
  \includeonly{\childdocname}
  \def\jobname{#1}
  \def\childdocjob{#1}
  \input{#1}
}
%    \end{macrocode}

% \macro{\childdocby}
% The command |\childdocby| ....
%    \begin{macrocode}
\newcommand{\childdocby}[2][]
{
  \childdocdisable
  \childdoctrue
  \childdocmanualtrue
  \if?#1?\else
    \def\jobname{#2}
  \fi
  \def\childdocjob{#2}
  \input{#2}
  \endinput
}
%    \end{macrocode}

% \macro{\childdocforward}
% The command |\childdocforward| redirects
% compilation to the main file or
% (if the optional argument is given) a child file.
% Parameters are set as if the main file
% or a child file starting with |\childdocof| was compiled.
% Then compilation is handed over to the main file:
%    \begin{macrocode}
\newcommand{\childdocforward}[2][]
{
  \begingroup
    \if?#1?
      \def\childdoctmp
      {
        \def\childdocname{#2}
        \def\childdocjob{#2}
        \def\jobname{#2}
        \input{#2}
        \endinput
      }
    \else
      \def\childdoctmp
      {
        \childdocdisable
        \def\childdocname{#2}
        \childdoctrue
        \includeonly{#2}
        \def\childdocjob{#1}
        \def\jobname{#1}
        \input{#1}
        \endinput
      }
    \fi
    \expandafter
  \endgroup
  \childdoctmp
}
%    \end{macrocode}

% \macro{\childdocforwardprefix}
% The command |\childdocforwardprefix| redirects
% compilation to the main or a child file by means of a pattern.
% The prefix |#1| in the current filename is replaced by |#2|
% and the suffix of the current filename is kept
% (it is assumed that the filename does not contain the substring `|~~~|'
% which is used as a delimiter).
% Compilation is handed over to the new file by |\childdocforward|:
%    \begin{macrocode}
\newcommand{\childdocforwardprefix}[3][]
{
  \begingroup
    \def\childdocextract #2##1~~~{\def\childdoctmp{\childdocforward[#1]{#3##1}}}
    \expandafter\childdocextract\childdocname~~~
    \expandafter
  \endgroup
  \childdoctmp
}
%    \end{macrocode}

% \macro{\childdoc}
% The deprecated macro |\childdoc| is a legacy version of |\childdocmain|:
%    \begin{macrocode}
\newcommand{\childdoc}{\childdocmain}
%    \end{macrocode}

% \macro{\childdocredirect}
% The deprecated macro |\childdocredirect| is a legacy version
% of |\childdocforward| and |\childdocforwardprefix|:
%    \begin{macrocode}
\newcommand{\childdocredirect}[2][]
{
  \begingroup
    \if?#1?
      \def\childdoctmp{\childdocforward{#2}}
    \else
      \def\childdoctmp{\childdocforwardprefix{#1}{#2}}
    \fi
    \expandafter
  \endgroup
  \childdoctmp
}
%    \end{macrocode}

%\iffalse
%</package>
%\fi
%
\endinput
|\\
|\childdocforwardprefix{final}{child}|
\end{tabular}
\end{center}
%

Note that when several versions of a main file and/or of each child file
are to be generated, it may be convenient to set up a |Makefile| or
shell script to automatise the process.

%%%%%%%%%%%%%%%%%%%%%%%%%%%%%%%%%%%%%%%%%%%%%%%%%%%%%%%%%%%%%%%%%%%%%%%%%%%%%%%%
\subsection{Command Line Processing}
\label{sec:commandline}

The effect of redirection files can also be achieved by invoking
the \LaTeX{} compiler with a more elaborate command line.
Most conveniently this should be done as part
of a shell script or a |Makefile|.

When using \textsf{childdoc} in the main file, the following
command lines effectively perform a redirection
(note that depending on the shell being used,
backslashes may have to be doubled: `|\|' $\to$ `|\\|'):
%
\begin{center}
|... -jobname "|\textit{target}|" |\\|"|[\textit{flags}]%
|% \iffalse
%
% childdoc.dtx Copyright (C) 2017-2018 Niklas Beisert
%
% This work may be distributed and/or modified under the
% conditions of the LaTeX Project Public License, either version 1.3
% of this license or (at your option) any later version.
% The latest version of this license is in
%   http://www.latex-project.org/lppl.txt
% and version 1.3 or later is part of all distributions of LaTeX
% version 2005/12/01 or later.
%
% This work has the LPPL maintenance status `maintained'.
%
% The Current Maintainer of this work is Niklas Beisert.
%
% This work consists of the files childdoc.dtx and childdoc.ins
% and the derived files childdoc.def and cdocsamp.tex with
% cdocsch1.tex, cdocsch2.tex, cdocsdrf.tex, cdocsfn1.tex, cdocsfn2.tex.
%
%<package>\ifdefined\childdocmain\endinput\fi
%<package>\ProvidesFile{childdoc.def}[2018/12/30 v2.0 child document driver]
%<samplemain>\ProvidesFile{cdocsamp.tex}[2018/12/30 v2.0 sample for childdoc]
%<*driver>
%\ProvidesFile{childdoc.drv}[2018/12/30 v2.0 childdoc reference manual file]
\PassOptionsToClass{10pt,a4paper}{article}
\documentclass{ltxdoc}

\usepackage[margin=35mm]{geometry}
\usepackage{hyperref}
\usepackage{hyperxmp}
\usepackage[usenames]{color}

\hypersetup{colorlinks=true}
\hypersetup{pdfstartview=FitH}
\hypersetup{pdfpagemode=UseNone}
\hypersetup{pdfsource={}}
\hypersetup{pdflang={en-UK}}
\hypersetup{pdfcopyright={Copyright 2017-2018 Niklas Beisert.
  This work may be distributed and/or modified under the
  conditions of the LaTeX Project Public License, either version 1.3
  of this license or (at your option) any later version.}}
\hypersetup{pdflicenseurl={http://www.latex-project.org/lppl.txt}}
\hypersetup{pdfcontactaddress={ETH Zurich, ITP, HIT K,
  Wolfgang-Pauli-Strasse 27}}
\hypersetup{pdfcontactpostcode={8093}}
\hypersetup{pdfcontactcity={Zurich}}
\hypersetup{pdfcontactcountry={Switzerland}}
\hypersetup{pdfcontactemail={nbeisert@itp.phys.ethz.ch}}
\hypersetup{pdfcontacturl={http://people.phys.ethz.ch/\xmptilde nbeisert/}}

\newcommand{\secref}[1]{\hyperref[#1]{section \ref*{#1}}}

\parskip1ex
\parindent0pt
\let\olditemize\itemize
\def\itemize{\olditemize\parskip0pt}

\begin{document}

\title{The \textsf{childdoc} Package}
\hypersetup{pdftitle={The childdoc Package}}
\author{Niklas Beisert\\[2ex]
  Institut f\"ur Theoretische Physik\\
  Eidgen\"ossische Technische Hochschule Z\"urich\\
  Wolfgang-Pauli-Strasse 27, 8093 Z\"urich, Switzerland\\[1ex]
  \href{mailto:nbeisert@itp.phys.ethz.ch}
  {\texttt{nbeisert@itp.phys.ethz.ch}}}
\hypersetup{pdfauthor={Niklas Beisert}}
\hypersetup{pdfsubject={Manual for the LaTeX2e Package childdoc}}
\date{30 December 2018, \textsf{v2.0}}
\maketitle

\begin{abstract}\noindent
\textsf{childdoc} is a \LaTeXe{} package
that enables the direct compilation
of document sections included by |\include|
to individual files.
\end{abstract}

\begingroup
\parskip0ex
\tableofcontents
\endgroup

%%%%%%%%%%%%%%%%%%%%%%%%%%%%%%%%%%%%%%%%%%%%%%%%%%%%%%%%%%%%%%%%%%%%%%%%%%%%%%%%
%%%%%%%%%%%%%%%%%%%%%%%%%%%%%%%%%%%%%%%%%%%%%%%%%%%%%%%%%%%%%%%%%%%%%%%%%%%%%%%%
\section{Introduction}

\LaTeX{} provides a mechanism to structure a large document (such as a book)
into a main file and several child files (containing the chapters)
using the |\include| command.
This mechanism is beneficial for documents
which span hundreds of pages in order to
make the source file(s) more manageable.
Moreover, compilation can be restricted to
selected child files by means of the |\includeonly| command.
The latter feature can be used to reduce the compilation time while editing
(this was significantly more useful in the earlier days of \LaTeX{})
or to generate a smaller document which is easier to navigate.
Another application of |\includeonly| is to generate
documents consisting of selected parts of the complete document.

However, there are a few drawbacks of the plain |\include| mechanism:
\begin{itemize}
\item
The child files cannot be compiled on their own,
they can only be compiled via the main file.
A naive editing environment
(such as a text editor with an option
to have the current file processed by \LaTeX)
may require one to switch to the main file before compiling;
attempting to compile the child file produces errors.
\item
The main file must be modified (each time)
to adjust the |\includeonly| command
to the present needs. This easily leaves the main file in a messy state.
\item
The generated document will always carry the filename
of the main document. This is inconvenient if
several child files are to be compiled and
to be kept for distribution.
\end{itemize}

The present package provides a simple interface
to make child files individually compilable by \LaTeX{}.
Compiling a child file then has the same effect as compiling
the main file with an |\includeonly| command
to select the appropriate child.
Moreover the generated document will carry the name of the child
rather than the main file.
This resolves all three above issues.

This feature is meant to make the editing of books,
thesis documents and lecture notes somewhat more convenient.
However, the package can also be used efficiently for
composing a series of documents (such as exercise sheets)
which are typically distributed individually.
It then assists the author in generating the individual documents
(potentially in different versions)
as well as a document containing the collected series.
Another application is in developing style files
or other kinds of included material
where compilation of the style file could redirect
to a sample or test file.

%%%%%%%%%%%%%%%%%%%%%%%%%%%%%%%%%%%%%%%%%%%%%%%%%%%%%%%%%%%%%%%%%%%%%%%%%%%%%%%%
%%%%%%%%%%%%%%%%%%%%%%%%%%%%%%%%%%%%%%%%%%%%%%%%%%%%%%%%%%%%%%%%%%%%%%%%%%%%%%%%
\section{Usage}

First of all, the package \textsf{childdoc} is \emph{not} a standard
\LaTeXe{} |.sty| style file! Therefore it needs to be invoked in
a non-standard way.

%%%%%%%%%%%%%%%%%%%%%%%%%%%%%%%%%%%%%%%%%%%%%%%%%%%%%%%%%%%%%%%%%%%%%%%%%%%%%%%%
\subsection{Included Files}
\label{sec:include}

%%%%%%%%%%%%%%%%%%%%%%%%%%%%%%%%%%%%%%%%
\DescribeMacro{\childdocmain}
To use the package, add the commands
\begin{center}
\begin{tabular}{l}
|\input{childdoc.def}|\\
|\childdocmain{}|\\
\end{tabular}
\end{center}
at the very top of the main \LaTeX{} file,
in particular \emph{before} the |\documentclass| statement!
The argument of |\childdocmain| should be left empty
(but it must be present).

%%%%%%%%%%%%%%%%%%%%%%%%%%%%%%%%%%%%%%%%
\DescribeMacro{\childdocof}
Furthermore, add the commands
\begin{center}
\begin{tabular}{l}
|\input{childdoc.def}|\\
|\childdocof{|\textit{main}|}|\\
\end{tabular}
\end{center}
at the top of every child file \textit{child}
which is included by |\include{|\textit{child}|}|
from within the main file
(or at least for those files to be compiled individually).
The argument \textit{main} must be the filename of the main file.

There are a couple of
considerations in setting up the main and child documents:

%%%%%%%%%%%%%%%%%%%%%%%%%%%%%%%%%%%%%%%%
\paragraph{Restrictions.}

Please note the following restrictions:
\begin{itemize}
\item
|\childdocmain| must be called with one argument \textit{main}
to ensure compatibility with earlier version of the package.
It must either be empty (|\childdocmain{}|)
or precisely match the filename of the main file in which it is specified.
See \secref{sec:detection} for further information.
\item
The filename \textit{main} must be specified without the |.tex| extension.
\item
The filename \textit{main} is case sensitive
(even in case-insensitive file systems)
due to internal string comparison.
\item
The argument \textit{main} should be fully expanded, it cannot be a macro.
\item
Subdirectories and special characters should be avoided in filenames.
\item
The command |\childdocmain{|\textit{main}|}| must be followed by a whitespace.
It should not be followed immediately by another command
or by a comment mark `|%|'.
This is because the \TeX{} parser reads the token immediately following
the argument of |\childdocmain| and puts it
at the beginning of every child section;
however, a white\-space is ignored.
\end{itemize}

%%%%%%%%%%%%%%%%%%%%%%%%%%%%%%%%%%%%%%%%
\paragraph{Content of Main File.}

It is advisable to place all content in the child files included by |\include|.
Any output contained in the main file will appear in all child documents
unless suppressed manually;
it cannot be suppressed automatically by the |\includeonly| directive
and thus should normally be avoided.
A method to include some content in the main file
by means of conditional processing is described in \secref{sec:conditional}.

%%%%%%%%%%%%%%%%%%%%%%%%%%%%%%%%%%%%%%%%
\paragraph{Page Numbering.}

When only a part of the document is compiled,
the appropriate numbering of pages
(as well as other status parameters)
is determined from the |.aux| files.
The latter contain information from previous passes.
However this information needs to propagate through
all intermediate child documents.
Therefore the page numbering in child documents may well
be inconsistent until the complete document is compiled at least once.

A useful (if unconventional) way to always ensure a consistent
page numbering is to restart the numbering in each child document
and denote the pages by `\textit{child}|.|\textit{page}'
where \textit{child} represents the chapter/section number of the child file.
This can be achieved by the command
|\numberwithin{page}{|\textit{child}|}|
of the \textsf{amsmath} package
where \textit{child} can be |chapter| or |section|
depending on the chosen structuring.
Alternatively, one can modify the macro |\thepage| appropriately
and reset the counter |page| at the start of each child file.

%%%%%%%%%%%%%%%%%%%%%%%%%%%%%%%%%%%%%%%%%%%%%%%%%%%%%%%%%%%%%%%%%%%%%%%%%%%%%%%%
\subsection{Conditional Processing}
\label{sec:conditional}

The package provides a mechanism to compile different versions
of a document. To customise the versions further some conditional processing
can come in handy to distinguish which version is being compiled.
The package provides two macros to describe the compilation context:

%%%%%%%%%%%%%%%%%%%%%%%%%%%%%%%%%%%%%%%%
\DescribeMacro{\ifchilddoc}
The conditional |\ifchilddoc| distinguishes between the compilation of
child documents and the main document:
%
\begin{center}
|\ifchilddoc |\textit{child-code}| |[|\||else |\textit{main-code}]| \||fi|
\end{center}

%%%%%%%%%%%%%%%%%%%%%%%%%%%%%%%%%%%%%%%%
\DescribeMacro{\childdocname}
\DescribeMacro{\childdocjob}
The macro |\childdocname| contains the filename (without extension)
of the main or child file being processed.
Note that |\childdocjob| will always contain the name of the main file.

%%%%%%%%%%%%%%%%%%%%%%%%%%%%%%%%%%%%%%%%
\paragraph{Title Page.}

Conditional processing can be used to include a title or banner page
in the main document when proper precautions are taken.
Importantly, the code in the main file should ensure that the page counter
(as well as other status parameters which are stored in the |.aux| files)
takes the same value after the conditional processing.
Otherwise the page numbers may take divergent values
depending on which part is compiled.

For example, a title page could be declared by:
%
\begin{center}
\begin{tabular}{l}
|\ifchilddoc\||else|\\
|\addtocounter{page}{-1}|\\
\textit{code for title page}\\
|\newpage|\\
|\||fi|
\end{tabular}
\end{center}
%
A banner page for the child documents can be generated by:
%
\begin{center}
\begin{tabular}{l}
|\ifchilddoc|\\
|\addtocounter{page}{-1}|\\
\textit{code for banner page}\\
|\newpage|\\
|\||fi|
\end{tabular}
\end{center}
%
Here one could write a message such as:
\begin{center}
|This is the part \childdocname{} of \childdocjob{}.|
\end{center}

%%%%%%%%%%%%%%%%%%%%%%%%%%%%%%%%%%%%%%%%%%%%%%%%%%%%%%%%%%%%%%%%%%%%%%%%%%%%%%%%
\subsection{Flags}
\label{sec:flags}

The package makes it easy to generate different versions
of the main or child documents.
To this end compilation flags can be defined
and assigned different default values.
They will be particularly useful in conjunction
with the forwarding mechanism described in \secref{sec:forward}.

For example, it may be useful to have a flag |\version|
which can be set to |draft| or |final|.
The document source will contain some conditional code
depending on the value of |\version|.
Suppose further, the flag should default to |final| for the main file
and to |draft| for child files
which is a natural assignment for editing the document.
This is achieved by placing the following code
in the preamble of the main document
(below the |\childdocmain| directive):
%
\begin{center}
\begin{tabular}{l}
|\ifchilddoc|\\
|\providecommand{\version}{draft}|\\
|\||else|\\
|\providecommand{\version}{final}|\\
|\||fi|
\end{tabular}
\end{center}
%
The definition by |\providecommand| makes sure
that previous definitions are not overwritten.
Further statements |\providecommand{\version}{...}|
can thus be added before the above code to override it.

For the main file, one might add a line
(between |\childdocmain| and the above block)
%
\begin{center}
|%\ifchilddoc\||else\providecommand{\version}{draft}\||fi|
\end{center}
%
which can be uncommented to produce a draft version.
Likewise one can add a line to the very top of a child file
(above the |\childdocof{|\textit{main}|}| directive)
%
\begin{center}
|%\providecommand{\version}{final}|
\end{center}
%
which can be uncommented to produce the final version of this child document.

%%%%%%%%%%%%%%%%%%%%%%%%%%%%%%%%%%%%%%%%%%%%%%%%%%%%%%%%%%%%%%%%%%%%%%%%%%%%%%%%
\subsection{Forwarding}
\label{sec:forward}

Different versions of the main or child documents
using compilation flags as described in \secref{sec:flags}
can be (permanently) stored in different files
for convenient compilation, viewing and distribution.
To this end, the package defines a command
to pass on compilation to a different file:

%%%%%%%%%%%%%%%%%%%%%%%%%%%%%%%%%%%%%%%%
\DescribeMacro{\childdocforward}
The command |\childdocforward| redirects processing to
another source file:
%
\begin{center}
\begin{tabular}{l}
|\input{childdoc.def}|\\
|\childdocforward[|\textit{main}|]{|\textit{dest}|}|\\
\end{tabular}
\end{center}
%
The argument \textit{dest} is the destination file
(without extension).
It should be the main file or one of the child files.
Note that further \textsf{childdoc} directives
such as |\childdocof| and |\childdocforward|
in the indicated file will be processed in this form.
The optional argument \textit{main}
passes on directly to the main file \textit{main}
while pretending to compile the child \textit{dest}.
This form behaves as if \textit{dest}
issues |\childdocof{|\textit{main}|}| right away,
and no further \textsf{childdoc} directives will be processed.

%%%%%%%%%%%%%%%%%%%%%%%%%%%%%%%%%%%%%%%%
\DescribeMacro{\...prefix}
In the alternative form |\childdocforwardprefix|,
%
\begin{center}
\begin{tabular}{l}
|\input{childdoc.def}|\\
|\childdocforwardprefix[|\textit{main}|]{|\textit{prefix}|}{|\textit{dest}|}|
\end{tabular}
\end{center}
%
the destination file is determined by a pattern
depending on the current file:
To make this work, the current file must be called
`{\textit{prefix}\hspace{0.2em}\textit{suffix}}'
with \textit{prefix} matching precisely the argument.
Processing is then passed on to the file
`{\textit{dest}\hspace{0.2em}\textit{suffix}}'.
Surely, the same effect is achieved by
directly specifying the
argument `{\textit{dest}\hspace{0.2em}\textit{suffix}}'
in the first form.
However, that requires to set up a different file
for each child. With the alternative form of the command
all these files can have exactly the same content
which simplifies setting them up and maintaining them.

For example, the following file |draft.tex|
with a compilation flag |\version| as described in \secref{sec:flags}
compiles the main document as a draft:
%
\begin{center}
\begin{tabular}{l}
|\def\version{draft}|\\
|\input{childdoc.def}|\\
|\childdocforward{|\textit{main}|}|
\end{tabular}
\end{center}
%
Likewise, the following files |final|\textit{nn}|.tex|
compile the final version of the child document
|child|\textit{nn}|.tex|:
%
\begin{center}
\begin{tabular}{l}
|\def\version{final}|\\
|\input{childdoc.def}|\\
|\childdocforwardprefix{final}{child}|
\end{tabular}
\end{center}
%

Note that when several versions of a main file and/or of each child file
are to be generated, it may be convenient to set up a |Makefile| or
shell script to automatise the process.

%%%%%%%%%%%%%%%%%%%%%%%%%%%%%%%%%%%%%%%%%%%%%%%%%%%%%%%%%%%%%%%%%%%%%%%%%%%%%%%%
\subsection{Command Line Processing}
\label{sec:commandline}

The effect of redirection files can also be achieved by invoking
the \LaTeX{} compiler with a more elaborate command line.
Most conveniently this should be done as part
of a shell script or a |Makefile|.

When using \textsf{childdoc} in the main file, the following
command lines effectively perform a redirection
(note that depending on the shell being used,
backslashes may have to be doubled: `|\|' $\to$ `|\\|'):
%
\begin{center}
|... -jobname "|\textit{target}|" |\\|"|[\textit{flags}]%
|\input{childdoc.def}\childdocforward[|\textit{main}|]{|\textit{dest}|}"|
\end{center}
%
Here \textit{target} is the name of the output file,
\textit{main} is the name of the main file
and \textit{dest} is the name of the main or child file to be processed
(all filenames without extensions).
The optional argument \textit{main} can be omitted
if \textit{main} matches \textit{dest}.
Optionally, compilation \textit{flags} can be defined via |\def| commands.
This command line makes the \TeX{} engine believe
it is compiling the file \textit{target}
whose content is specified as the latter parameter.
The provided code then forwards the processing to
\textit{main} or \textit{dest} as described in \secref{sec:forward}.

%%%%%%%%%%%%%%%%%%%%%%%%%%%%%%%%%%%%%%%%%%%%%%%%%%%%%%%%%%%%%%%%%%%%%%%%%%%%%%%%
\subsection{Include by Input}
\label{sec:input}

Including child documents by |\include| has some restrictions by design.
Most notably, the content of a child document always occupies
its own set of pages; pages cannot be shared between child documents.
Usually, this behaviour makes perfect sense
because each child document contain an essential part of the document.
However, in some situations it may be desirable to compose
a document from a collection of parts
without having mandatory page breaks between then.
For this case, the package
provides a mechanism to include parts
by |\input| which can also be processed individually.
However, by construction this mechanism
requires manual handling of the content to be output.

%%%%%%%%%%%%%%%%%%%%%%%%%%%%%%%%%%%%%%%%
\DescribeMacro{\ifchilddocmanual}
The main file should be prepared as usual, see \secref{sec:include}.
However, the document body must make a distinction
between processing of an individual part and of the main document, e.g.:
%
\begin{center}
\begin{tabular}{l}
|\ifchilddocmanual|\\
|\input{\childdocname}|\\
|\||else|\\
\textit{document body with }|\input{|\textit{part}|}|\\
|\||fi|
\end{tabular}
\end{center}
%
The conditional |\ifchilddocmanual| is true whenever
a part to be included by |\input| is being compiled,
and the name of the part is stored in |\childdocname|.

%%%%%%%%%%%%%%%%%%%%%%%%%%%%%%%%%%%%%%%%
\DescribeMacro{\childdocby}
Each part to be included by |\input| should start with:
%
\begin{center}
\begin{tabular}{l}
|\input{childdoc.def}|\\
|\childdocby{|\textit{main}|}|\\
\end{tabular}
\end{center}
%
The directive |\childdocby| is similar to |\childdocof|
described in \secref{sec:include},
but the subsequent selection of content must be done manually.
To that end, both |\ifchilddoc| and |\ifchilddocmanual|
will be true upon processing of a part,
and the name of the part is stored in |\childdocname|.
Note that |\jobname| will be set to the filename of the current part
so that each part receives an individual |.aux| file
that does not interfere with the |.aux| file(s) of the main document.
This behaviour can be altered by the alternative form
|\childdocby[*]{|\textit{main}|}| (with a non-empty optional argument)
which uses the |.aux| file of the main document
by setting |\jobname| to \textit{main}.

%%%%%%%%%%%%%%%%%%%%%%%%%%%%%%%%%%%%%%%%%%%%%%%%%%%%%%%%%%%%%%%%%%%%%%%%%%%%%%%%
\subsection{Driver Development}
\label{sec:driver}

The \textsf{childdoc} mechanism can also be use for the development
of definition files such as \LaTeX{} styles or classes.
This case differs from the above setup with multiple parts
included by |\include| in that no |\includeonly| should be invoked.
This can be achieved by starting the include file
(before |\ProvidesPackage|) with:
%
\begin{center}
\begin{tabular}{l}
|\input{childdoc.def}|\\
|\childdocforward{|\textit{main}|}|\\
\end{tabular}
\end{center}
%
or alternatively with:
%
\begin{center}
\begin{tabular}{l}
|\input{childdoc.def}|\\
|\childdocby{|\textit{main}|}|\\
\end{tabular}
\end{center}
%
Both forms have slightly different effects as described above.
The main file is prepared as usual, see \secref{sec:include}.

%%%%%%%%%%%%%%%%%%%%%%%%%%%%%%%%%%%%%%%%%%%%%%%%%%%%%%%%%%%%%%%%%%%%%%%%%%%%%%%%
\subsection{Legacy Detection}
\label{sec:detection}

The directive |\childdocmain| in the main file can detect
whether the complete document or merely a child is to be compiled
even without using the directive |\childdocof|.
This method is deprecated because it is less robust
and there is no compelling reason to use it;
it is merely provided for backward compatibility
and it may be removed in future versions.

If the detection mechanism is to be used,
it is mandatory to correctly specify
the filename of the main file as the argument of |\childdocmain|:
%
\begin{center}
\begin{tabular}{l}
|\input{childdoc.def}|\\
|\childdocmain{|\textit{main}|}|\\
\end{tabular}
\end{center}
%
If |\jobname| does not match the argument \textit{main} of |\childdocmain|,
it is assumed that |\jobname| points to the child file to be compiled.
When using |\childdocmain| with the main file specified as argument,
it suffices to start a child file
with just |\input{|\textit{main}|}|
without loading of the package and using |\childdocof|.
If instead all processing is done
with the appropriate \textsf{childdoc} directives,
the argument of \textit{main} of |\childdocmain| can be empty.

An alternative version of the command line processing described
in \secref{sec:commandline} using the detection mechanism reads:
%
\begin{center}
|... -jobname "|\textit{target}|" "|[\textit{flags}]%
[|\def\jobname{|\textit{dest}|}|]|\input{|\textit{main}|}"|
\end{center}

%%%%%%%%%%%%%%%%%%%%%%%%%%%%%%%%%%%%%%%%%%%%%%%%%%%%%%%%%%%%%%%%%%%%%%%%%%%%%%%%
\subsection{Manual Code}
\label{sec:manual}

In case one cannot be certain whether the definitions file |childdoc.def|
is installed on the target \TeX{} distribution
and one prefers not to ship it,
it is conceivable to paste a few relevant commands into the sources.

To that end, drop all statements |\input{childdoc.def}|
and perform the replacements as outlined below.
Instead of |\childdocmain{|\textit{main}|}| add the following code
to the top of the main file:
%
\begin{center}
\begin{tabular}{l}
|\||ifdefined\childdocname\endinput\||fi\newif\ifchilddoc|\\
|\edef\childdocname{\scantokens\expandafter{\jobname\noexpand}}|\\
|\def\childdocmain{|\textit{main}|}\||ifx\childdocmain\childdocname\||else|\\
|\childdoctrue\includeonly{\childdocname}\let\jobname\childdocmain\||fi|\\
\end{tabular}
\end{center}
%
Instead of |\childdocof{|\textit{main}|}| just include the main file
at the top of each child file:
%
\begin{center}
|\input{|\textit{main}|}|
\end{center}
%
A simple redirection |\childdocforward{|\textit{dest}|}| is achieved by:
%
\begin{center}
|\def\jobname{|\textit{dest}|}\input{\jobname}|
\end{center}
%
The redirection with prefix
|\childdocforwardprefix[|\textit{prefix}|]{|\textit{dest}|}|
is accomplished by:
%
\begin{center}
\begin{tabular}{l}
|{\edef\jobname{\scantokens\expandafter{\jobname\noexpand}}|\\
|\def\redirectjob |\textit{prefix}|#1~~~{\gdef\jobname{|\textit{dest}|#1}}|\\
|\expandafter\redirectjob\jobname~~~}\input{\jobname}|
\end{tabular}
\end{center}

In an alternative approach,
child documents can be compiled by a specific command line
without additional code or specific definitions:
%
\begin{center}
|... -jobname "|\textit{target}|" "|[\textit{flags}]%
|\includeonly{|\textit{dest}|}\input{|\textit{main}|}"|
\end{center}
%

%%%%%%%%%%%%%%%%%%%%%%%%%%%%%%%%%%%%%%%%%%%%%%%%%%%%%%%%%%%%%%%%%%%%%%%%%%%%%%%%
%%%%%%%%%%%%%%%%%%%%%%%%%%%%%%%%%%%%%%%%%%%%%%%%%%%%%%%%%%%%%%%%%%%%%%%%%%%%%%%%
\section{Information}

%%%%%%%%%%%%%%%%%%%%%%%%%%%%%%%%%%%%%%%%%%%%%%%%%%%%%%%%%%%%%%%%%%%%%%%%%%%%%%%%
\subsection{Copyright}

Copyright \copyright{} 2017--2018 Niklas Beisert

This work may be distributed and/or modified under the
conditions of the \LaTeX{} Project Public License, either version 1.3
of this license or (at your option) any later version.
The latest version of this license is in
  \url{http://www.latex-project.org/lppl.txt}
and version 1.3 or later is part of all distributions of \LaTeX{}
version 2005/12/01 or later.

This work has the LPPL maintenance status `maintained'.

The Current Maintainer of this work is Niklas Beisert.

This work consists of the files |README.txt|, |childdoc.ins| and |childdoc.dtx|
as well as the derived files |childdoc.def|, |cdocsamp.tex|
with |cdocsch1.tex|, |cdocsch2.tex|, |cdocspt3.tex|, |cdocspt4.tex|,
|cdocsdrf.tex|, |cdocsfn1.tex|, |cdocsfn2.tex|
as well as |childdoc.pdf|.

%%%%%%%%%%%%%%%%%%%%%%%%%%%%%%%%%%%%%%%%%%%%%%%%%%%%%%%%%%%%%%%%%%%%%%%%%%%%%%%%
\subsection{Files and Installation}

The package consists of the files:
%
\begin{center}
\begin{tabular}{ll}
    |README.txt|   & readme file \\
    |childdoc.ins| & installation file \\
    |childdoc.dtx| & source file \\
    |childdoc.def| & definition file \\
    |cdocsamp.tex| & sample main file \\
    |cdocsch1.tex| & sample include file \\
    |cdocsch2.tex| & sample include file \\
    |cdocspt3.tex| & sample part file \\
    |cdocspt4.tex| & sample part file \\
    |cdocsdrf.tex| & sample redirection file \\
    |cdocsfn1.tex| & sample redirection file \\
    |cdocsfn2.tex| & sample redirection file \\
    |childdoc.pdf| & manual
\end{tabular}
\end{center}
%
The distribution consists of the files
|README.txt|, |childdoc.ins| and |childdoc.dtx|.
%
\begin{itemize}
\item
Run (pdf)\LaTeX{} on |childdoc.dtx|
to compile the manual |childdoc.pdf| (this file).
\item
Run \LaTeX{} on |childdoc.ins| to create the definitions file |childdoc.def|
and the sample |cdocsamp.tex| with include files
|cdocsch1.tex|, |cdocsch2.tex|, |cdocspt3.tex|, |cdocspt4.tex|,
|cdocsdrf.tex|, |cdocsfn1.tex|, |cdocsfn2.tex|.
Then copy the file |childdoc.def| to an appropriate directory of your \LaTeX{}
distribution, e.g.\ \textit{texmf-root}|/tex/latex/childdoc|.
\end{itemize}

%%%%%%%%%%%%%%%%%%%%%%%%%%%%%%%%%%%%%%%%%%%%%%%%%%%%%%%%%%%%%%%%%%%%%%%%%%%%%%%%
\subsection{Related CTAN Packages}

There are several other packages which offer a similar functionality:
%
\begin{itemize}
\item
The packages
\href{http://ctan.org/pkg/docmute}{\textsf{docmute}},
\href{http://ctan.org/pkg/includex}{\textsf{includex}} and
\href{http://ctan.org/pkg/standalone}{\textsf{standalone}}
provide commands to include only the document body of
a child file thus allowing both files to be compiled individually.
\item
The packages \href{http://ctan.org/pkg/subdocs}{\textsf{subdocs}}
and \href{http://ctan.org/pkg/subfiles}{\textsf{subfiles}}
provide structures in which the main and child documents can be
encapsulated and allowing them to be compiled individually.
The inclusion mechanism is different from the conventional |\include|.
\item
The package \href{http://ctan.org/pkg/combine}{\textsf{combine}}
is an elaborate solution to combine several documents into one.
\end{itemize}
%
See also the CTAN topic \href{http://ctan.org/topic/subdocs}{\textsf{subdocs}}
for further related packages.
The present package differs from the above solutions in that
a document structure constructed with the conventional |\include| mechanism
just needs two extra commands at the top of every file
such that all constituent files can be compiled individually.

%%%%%%%%%%%%%%%%%%%%%%%%%%%%%%%%%%%%%%%%%%%%%%%%%%%%%%%%%%%%%%%%%%%%%%%%%%%%%%%%
%\subsection{Feature Suggestions}
%
%The following is a list of features which may be useful for future
%versions of this package:
%%
%\begin{itemize}
%\item
%\ldots
%\end{itemize}

%%%%%%%%%%%%%%%%%%%%%%%%%%%%%%%%%%%%%%%%%%%%%%%%%%%%%%%%%%%%%%%%%%%%%%%%%%%%%%%%
\subsection{Revision History}

%%%%%%%%%%%%%%%%%%%%%%%%%%%%%%%%%%%%%%%%
\paragraph{v2.0:} 2018/12/30

\begin{itemize}
\item
immediate forward processing
\item
added |\childdocby| mechanism
\item
manual restructured
\end{itemize}

%%%%%%%%%%%%%%%%%%%%%%%%%%%%%%%%%%%%%%%%
\paragraph{v1.6:} 2018/01/17

\begin{itemize}
\item
application for development of include files
\item
corrections to manual
\end{itemize}

%%%%%%%%%%%%%%%%%%%%%%%%%%%%%%%%%%%%%%%%
\paragraph{v1.5:} 2017/05/21

\begin{itemize}
\item
more complete structuring introduced
\item
|\childdocof| introduced
\item
|\childdoc| renamed to |\childdocmain|
\item
|\childredirect| renamed to |\childdocforward| and |\childdocforwardprefix|
and functionality expanded
\end{itemize}

%%%%%%%%%%%%%%%%%%%%%%%%%%%%%%%%%%%%%%%%
\paragraph{v1.0:} 2017/04/27

\begin{itemize}
\item
manual and install package
\item
first version published on CTAN
\end{itemize}

%%%%%%%%%%%%%%%%%%%%%%%%%%%%%%%%%%%%%%%%
\paragraph{v0.6:} 2017/04/26

\begin{itemize}
\item
redirection mechanism added
\end{itemize}

%%%%%%%%%%%%%%%%%%%%%%%%%%%%%%%%%%%%%%%%
\paragraph{v0.5:} 2017/04/26

\begin{itemize}
\item
functionality in definition file
\end{itemize}


%%%%%%%%%%%%%%%%%%%%%%%%%%%%%%%%%%%%%%%%%%%%%%%%%%%%%%%%%%%%%%%%%%%%%%%%%%%%%%%%
%%%%%%%%%%%%%%%%%%%%%%%%%%%%%%%%%%%%%%%%%%%%%%%%%%%%%%%%%%%%%%%%%%%%%%%%%%%%%%%%
%%%%%%%%%%%%%%%%%%%%%%%%%%%%%%%%%%%%%%%%%%%%%%%%%%%%%%%%%%%%%%%%%%%%%%%%%%%%%%%%
\appendix

\settowidth\MacroIndent{\rmfamily\scriptsize 000\ }

 \DocInput{childdoc.dtx}

\end{document}
%</driver>
% \fi
%
% %%%%%%%%%%%%%%%%%%%%%%%%%%%%%%%%%%%%%%%%%%%%%%%%%%%%%%%%%%%%%%%%%%%%%%%%%%%%%%
% %%%%%%%%%%%%%%%%%%%%%%%%%%%%%%%%%%%%%%%%%%%%%%%%%%%%%%%%%%%%%%%%%%%%%%%%%%%%%%
% \section{Sample}
%\iffalse
%<*samplemain>
%\fi
%
% The following presents a sample document
% with two chapters, two parts, a title page,
% a compile flag as well as three forwarding files to set the flag.
% It consists of eight |.tex| files:
% \begin{center}
% \begin{tabular}{ll}
% |cdocsamp.tex|&main file\\
% |cdocsch1.tex|&include file for chapter 1\\
% |cdocsch2.tex|&include file for chapter 2\\
% |cdocspt3.tex|&include file for part 3\\
% |cdocspt4.tex|&include file for part 4\\
% |cdocsdrf.tex|&forwarding file for main file in draft mode\\
% |cdocsfi1.tex|&forwarding file for final version of chapter 1\\
% |cdocsfi2.tex|&forwarding file for final version of chapter 2\\
% \end{tabular}
% \end{center}
% Each of the eight files can be compiled directly by the \LaTeX{} compiler.
%
% %%%%%%%%%%%%%%%%%%%%%%%%%%%%%%%%%%%%%%
% \paragraph{Main File.}
%
% The main file is called |cdocsamp.tex|.
%
% Load the \textsf{childdoc} definitions and
% declare the filename for the main document:
%    \begin{macrocode}
\input{childdoc.def}
\childdocmain{}
%    \end{macrocode}

% Optional override for |\version| flag:
%    \begin{macrocode}
%%\ifchilddoc\else\providecommand{\version}{draft}\fi
%    \end{macrocode}

% Define the default values for the |\version| flag
% (|final| for the main file and |draft| for childs):
%    \begin{macrocode}
\ifchilddoc
\providecommand{\version}{draft}
\else
\providecommand{\version}{final}
\fi
%    \end{macrocode}

% Load the standard document class:
%    \begin{macrocode}
\documentclass[12pt]{article}
%    \end{macrocode}

% Start the document body:
%    \begin{macrocode}
\begin{document}
%    \end{macrocode}

% Declare a title page.
% Print title, part of document being processed and version flag:
%    \begin{macrocode}
\addtocounter{page}{-1}
\begin{center}
{\LARGE\bfseries{}childdoc example\par}
\vspace{1cm}
\ifchilddoc
\ifchilddocmanual part\else chapter\fi:
`\childdocname' of `\childdocjob'\par
\else
main document: `\childdocjob'\par
\fi
version: \version\par
\end{center}
\newpage
%    \end{macrocode}

% Manually include selected file,
% otherwise process as usual:
%    \begin{macrocode}
\ifchilddocmanual
\section*{part `\childdocname'}
\input{\childdocname}
\else
%    \end{macrocode}

% Include the two chapters:
%    \begin{macrocode}
\include{cdocsch1}
\include{cdocsch2}
%    \end{macrocode}

% Include the two parts unless only chapters should be displayed:
%    \begin{macrocode}
\ifchilddoc\else
\section{part three}
\input{cdocspt3}
\section{part four}
\input{cdocspt4}
\fi
%    \end{macrocode}

% Process as usual until here:
%    \begin{macrocode}
\fi
%    \end{macrocode}

% End of document body:
%    \begin{macrocode}
\end{document}
%    \end{macrocode}
%\iffalse
%</samplemain>
%\fi
%
% %%%%%%%%%%%%%%%%%%%%%%%%%%%%%%%%%%%%%%
% \paragraph{Chapter Include Files.}
%
% The include files are called |cdocsch1.tex| and |cdocsch2.tex|.
%
%\iffalse
%<*samplechap1|samplechap2>
%\fi

% Optional override for |\version| flag:
%    \begin{macrocode}
%%\providecommand{\version}{final}
%    \end{macrocode}

% Include the main document:
%    \begin{macrocode}
\input{childdoc.def}
\childdocof{cdocsamp}
%    \end{macrocode}

%\iffalse
%</samplechap1|samplechap2>
%\fi
%
%\iffalse
%<*samplechap1>
%\fi
% Some text for chapter 1:
%    \begin{macrocode}
\section{one}
some text in chapter one
%    \end{macrocode}

%\iffalse
%</samplechap1>
%\fi
% Some text for chapter 2:
%\iffalse
%<*samplechap2>
%\fi
%    \begin{macrocode}
\section{two}
more text in chapter two
%    \end{macrocode}

%\iffalse
%</samplechap2>
%\fi
%
% %%%%%%%%%%%%%%%%%%%%%%%%%%%%%%%%%%%%%%
% \paragraph{Part Include Files.}
%
% The include files are called |cdocspt3.tex| and |cdocspt4.tex|.
%
%\iffalse
%<*samplepart3|samplepart4>
%\fi

% Optional override for |\version| flag:
%    \begin{macrocode}
%%\providecommand{\version}{final}
%    \end{macrocode}

% Include the main document:
%    \begin{macrocode}
\input{childdoc.def}
\childdocby{cdocsamp}
%    \end{macrocode}

%\iffalse
%</samplepart3|samplepart4>
%\fi
%
%\iffalse
%<*samplepart3>
%\fi
% Some text for part 3:
%    \begin{macrocode}
some text in part three
%    \end{macrocode}

%\iffalse
%</samplepart3>
%\fi
% Some text for part 4:
%\iffalse
%<*samplepart4>
%\fi
%    \begin{macrocode}
more text in part four
%    \end{macrocode}

%\iffalse
%</samplepart4>
%\fi
%
% %%%%%%%%%%%%%%%%%%%%%%%%%%%%%%%%%%%%%%
% \paragraph{Forwarding for a Complete Draft.}
%
% The following forwarding file |cdocsdrf.tex|
% compiles the main document in draft mode:
%\iffalse
%<*sampledraft>
%\fi
%    \begin{macrocode}
\def\version{draft}
\input{childdoc.def}
\childdocforward{cdocsamp}
%    \end{macrocode}

%\iffalse
%</sampledraft>
%\fi
%
% %%%%%%%%%%%%%%%%%%%%%%%%%%%%%%%%%%%%%%
% \paragraph{Forwarding for Final Version of the Chapters.}
%
% The following forwarding files |cdocsfn1.tex| and |cdocsfn2.tex|
% (with identical content)
% compile the final versions of the child documents
% |cdocsch1.tex| and |cdocsch2.tex|, respectively:
%\iffalse
%<*samplefinal>
%\fi
%    \begin{macrocode}
\def\version{final}
\input{childdoc.def}
\childdocforwardprefix[cdocsamp]{cdocsfn}{cdocsch}
%    \end{macrocode}

%\iffalse
%</samplefinal>
%\fi
%
% %%%%%%%%%%%%%%%%%%%%%%%%%%%%%%%%%%%%%%
% \paragraph{Command Line Processing.}
%
% The following three command lines generate the output files
% |cdocscld|, |cdocscl1| and |cdocscl2|
% which should be identical to
% |cdocsdrf|, |cdocsch1| and |cdocsfn2|, respectively:
% \begin{center}
% \begin{tabular}{l}
% |latex -jobname cdocscld \|\\
% |  "\def\version{draft}\input{childdoc.def}\childdocforward{cdocsamp}"|\\
% |latex -jobname cdocscl1 \|\\
% |  "\input{childdoc.def}\childdocforward[cdocsamp]{cdocsch1}"|\\
% |latex -jobname cdocscl2 \|\\
% |  "\def\version{final}\input{childdoc.def}\childdocforward{cdocsch2}"|
% \end{tabular}
% \end{center}
% Note that the trailing backslash on each first line
% merely continues the input to the second line
% (for convenient cut ant paste).
% Furthermore, the command |latex| can be replaced by any
% of its alternative versions such as |pdflatex|.
%
% %%%%%%%%%%%%%%%%%%%%%%%%%%%%%%%%%%%%%%%%%%%%%%%%%%%%%%%%%%%%%%%%%%%%%%%%%%%%%%
% %%%%%%%%%%%%%%%%%%%%%%%%%%%%%%%%%%%%%%%%%%%%%%%%%%%%%%%%%%%%%%%%%%%%%%%%%%%%%%
% \section{Implementation}
%\iffalse
%<*package>
%\fi
%
% This section describes the definitions file |childdoc.def|.

% The definitions cannot be loaded using |\usepackage| or |\RequirePackage|
% which has a mechanism to prevent loading a style file more than once.
% When loading the definitions by means of |\input|
% multiple instances have to be prevented manually:
%\iffalse
%This code needs to be before the `\ProvidesFile' directive
%which is defined at the beginning of this file.
%Therefore it is also placed there and commented out here.
%</package>
%<*discard>
%\fi
%    \begin{macrocode}
\ifdefined\childdocmain\endinput\fi
%    \end{macrocode}
%\iffalse
%</discard>
%<*package>
%\fi
%
% \macro{\ifchilddoc}
% \macro{\ifchilddocmanual}
% The conditional |\ifchilddoc| tells whether a
% child (true) or main (false) document is being compiled.
% The conditional |\ifchilddocmanual| tells whether
% the |\includeonly| mechanism is used (false) or
% the selection of child files must be performed manually (true).
% The definitions initialise to false:
%    \begin{macrocode}
\newif\ifchilddoc
\newif\ifchilddocmanual
%    \end{macrocode}

% \macro{\childdocname}
% \macro{\childdocjob}
% The macro |\childdocname| stores the name of the main document
% to be compiled. The macro |\childdocjob| stores the name of
% the document on which the \LaTeX{} compiler was originally invoked.
% The content of |\jobname| cannot be compared
% to filenames specified in the source due to different catcodes.
% The following code rescans |\jobname|, stores the result
% in |\childdocname| and saves a copy in |\childdocjob|:
%    \begin{macrocode}
\edef\childdocname{\scantokens\expandafter{\jobname\noexpand}}
\let\childdocjob\childdocname
%    \end{macrocode}

% \macro{\childdocdisable}
% The macro |\childdocdisable| prevents the main file
% from being processed more than once.
% At this stage, the main document command |\childdocmain|
% is assumed to be called once again where it should do nothing.
% Any subsequent call to it should prevent
% a secondary processing of the main document
% It overwrites the forwarding commands
% |\childdocof| and |\childdocforward|
% with empty macros to prevent further inclusions of the main document:
%    \begin{macrocode}
\newcommand{\childdocdisable}
{
  \renewcommand{\childdocmain}[1]{\renewcommand{\childdocmain}[1]{\endinput}}
  \renewcommand{\childdocof}[1]{}
  \renewcommand{\childdocby}[2][]{}
  \renewcommand{\childdocforward}[2][]{}
  \renewcommand{\childdocdisable}{}
}
%    \end{macrocode}

% \macro{\childdocmain}
% The macro |\childdocmain| is to be called at the top of the main file
% with nothing or the main filename (without extension) as argument.
% First, it breaks loops.
% If the argument is not empty and does not match |\childdocname|
% (which is set by the first inclusion of |childdoc.def|),
% |\ifchilddoc| is set to true, |\includeonly| is applied to the child file
% and |\jobname| is set to the main file
% (for proper handling of |.aux| files):
%    \begin{macrocode}
\newcommand{\childdocmain}[1]
{
  \childdocdisable\childdocmain{}
  \if?#1?\else
    \begingroup
      \def\childdoctmp{#1}
      \ifx\childdoctmp\childdocname
        \def\childdoctmp{}
      \else
        \def\childdoctmp
        {
          \childdoctrue
          \includeonly{\childdocname}
          \def\childdocjob{#1}
          \def\jobname{#1}
        }
      \fi
      \expandafter
    \endgroup
    \childdoctmp
  \fi
}
%    \end{macrocode}

% \macro{\childdocof}
% The command |\childdocof| redirects
% compilation to the main file |#1|.
%    \begin{macrocode}
\newcommand{\childdocof}[1]
{
  \childdocdisable
  \childdoctrue
  \includeonly{\childdocname}
  \def\jobname{#1}
  \def\childdocjob{#1}
  \input{#1}
}
%    \end{macrocode}

% \macro{\childdocby}
% The command |\childdocby| ....
%    \begin{macrocode}
\newcommand{\childdocby}[2][]
{
  \childdocdisable
  \childdoctrue
  \childdocmanualtrue
  \if?#1?\else
    \def\jobname{#2}
  \fi
  \def\childdocjob{#2}
  \input{#2}
  \endinput
}
%    \end{macrocode}

% \macro{\childdocforward}
% The command |\childdocforward| redirects
% compilation to the main file or
% (if the optional argument is given) a child file.
% Parameters are set as if the main file
% or a child file starting with |\childdocof| was compiled.
% Then compilation is handed over to the main file:
%    \begin{macrocode}
\newcommand{\childdocforward}[2][]
{
  \begingroup
    \if?#1?
      \def\childdoctmp
      {
        \def\childdocname{#2}
        \def\childdocjob{#2}
        \def\jobname{#2}
        \input{#2}
        \endinput
      }
    \else
      \def\childdoctmp
      {
        \childdocdisable
        \def\childdocname{#2}
        \childdoctrue
        \includeonly{#2}
        \def\childdocjob{#1}
        \def\jobname{#1}
        \input{#1}
        \endinput
      }
    \fi
    \expandafter
  \endgroup
  \childdoctmp
}
%    \end{macrocode}

% \macro{\childdocforwardprefix}
% The command |\childdocforwardprefix| redirects
% compilation to the main or a child file by means of a pattern.
% The prefix |#1| in the current filename is replaced by |#2|
% and the suffix of the current filename is kept
% (it is assumed that the filename does not contain the substring `|~~~|'
% which is used as a delimiter).
% Compilation is handed over to the new file by |\childdocforward|:
%    \begin{macrocode}
\newcommand{\childdocforwardprefix}[3][]
{
  \begingroup
    \def\childdocextract #2##1~~~{\def\childdoctmp{\childdocforward[#1]{#3##1}}}
    \expandafter\childdocextract\childdocname~~~
    \expandafter
  \endgroup
  \childdoctmp
}
%    \end{macrocode}

% \macro{\childdoc}
% The deprecated macro |\childdoc| is a legacy version of |\childdocmain|:
%    \begin{macrocode}
\newcommand{\childdoc}{\childdocmain}
%    \end{macrocode}

% \macro{\childdocredirect}
% The deprecated macro |\childdocredirect| is a legacy version
% of |\childdocforward| and |\childdocforwardprefix|:
%    \begin{macrocode}
\newcommand{\childdocredirect}[2][]
{
  \begingroup
    \if?#1?
      \def\childdoctmp{\childdocforward{#2}}
    \else
      \def\childdoctmp{\childdocforwardprefix{#1}{#2}}
    \fi
    \expandafter
  \endgroup
  \childdoctmp
}
%    \end{macrocode}

%\iffalse
%</package>
%\fi
%
\endinput
\childdocforward[|\textit{main}|]{|\textit{dest}|}"|
\end{center}
%
Here \textit{target} is the name of the output file,
\textit{main} is the name of the main file
and \textit{dest} is the name of the main or child file to be processed
(all filenames without extensions).
The optional argument \textit{main} can be omitted
if \textit{main} matches \textit{dest}.
Optionally, compilation \textit{flags} can be defined via |\def| commands.
This command line makes the \TeX{} engine believe
it is compiling the file \textit{target}
whose content is specified as the latter parameter.
The provided code then forwards the processing to
\textit{main} or \textit{dest} as described in \secref{sec:forward}.

%%%%%%%%%%%%%%%%%%%%%%%%%%%%%%%%%%%%%%%%%%%%%%%%%%%%%%%%%%%%%%%%%%%%%%%%%%%%%%%%
\subsection{Include by Input}
\label{sec:input}

Including child documents by |\include| has some restrictions by design.
Most notably, the content of a child document always occupies
its own set of pages; pages cannot be shared between child documents.
Usually, this behaviour makes perfect sense
because each child document contain an essential part of the document.
However, in some situations it may be desirable to compose
a document from a collection of parts
without having mandatory page breaks between then.
For this case, the package
provides a mechanism to include parts
by |\input| which can also be processed individually.
However, by construction this mechanism
requires manual handling of the content to be output.

%%%%%%%%%%%%%%%%%%%%%%%%%%%%%%%%%%%%%%%%
\DescribeMacro{\ifchilddocmanual}
The main file should be prepared as usual, see \secref{sec:include}.
However, the document body must make a distinction
between processing of an individual part and of the main document, e.g.:
%
\begin{center}
\begin{tabular}{l}
|\ifchilddocmanual|\\
|\input{\childdocname}|\\
|\||else|\\
\textit{document body with }|\input{|\textit{part}|}|\\
|\||fi|
\end{tabular}
\end{center}
%
The conditional |\ifchilddocmanual| is true whenever
a part to be included by |\input| is being compiled,
and the name of the part is stored in |\childdocname|.

%%%%%%%%%%%%%%%%%%%%%%%%%%%%%%%%%%%%%%%%
\DescribeMacro{\childdocby}
Each part to be included by |\input| should start with:
%
\begin{center}
\begin{tabular}{l}
|% \iffalse
%
% childdoc.dtx Copyright (C) 2017-2018 Niklas Beisert
%
% This work may be distributed and/or modified under the
% conditions of the LaTeX Project Public License, either version 1.3
% of this license or (at your option) any later version.
% The latest version of this license is in
%   http://www.latex-project.org/lppl.txt
% and version 1.3 or later is part of all distributions of LaTeX
% version 2005/12/01 or later.
%
% This work has the LPPL maintenance status `maintained'.
%
% The Current Maintainer of this work is Niklas Beisert.
%
% This work consists of the files childdoc.dtx and childdoc.ins
% and the derived files childdoc.def and cdocsamp.tex with
% cdocsch1.tex, cdocsch2.tex, cdocsdrf.tex, cdocsfn1.tex, cdocsfn2.tex.
%
%<package>\ifdefined\childdocmain\endinput\fi
%<package>\ProvidesFile{childdoc.def}[2018/12/30 v2.0 child document driver]
%<samplemain>\ProvidesFile{cdocsamp.tex}[2018/12/30 v2.0 sample for childdoc]
%<*driver>
%\ProvidesFile{childdoc.drv}[2018/12/30 v2.0 childdoc reference manual file]
\PassOptionsToClass{10pt,a4paper}{article}
\documentclass{ltxdoc}

\usepackage[margin=35mm]{geometry}
\usepackage{hyperref}
\usepackage{hyperxmp}
\usepackage[usenames]{color}

\hypersetup{colorlinks=true}
\hypersetup{pdfstartview=FitH}
\hypersetup{pdfpagemode=UseNone}
\hypersetup{pdfsource={}}
\hypersetup{pdflang={en-UK}}
\hypersetup{pdfcopyright={Copyright 2017-2018 Niklas Beisert.
  This work may be distributed and/or modified under the
  conditions of the LaTeX Project Public License, either version 1.3
  of this license or (at your option) any later version.}}
\hypersetup{pdflicenseurl={http://www.latex-project.org/lppl.txt}}
\hypersetup{pdfcontactaddress={ETH Zurich, ITP, HIT K,
  Wolfgang-Pauli-Strasse 27}}
\hypersetup{pdfcontactpostcode={8093}}
\hypersetup{pdfcontactcity={Zurich}}
\hypersetup{pdfcontactcountry={Switzerland}}
\hypersetup{pdfcontactemail={nbeisert@itp.phys.ethz.ch}}
\hypersetup{pdfcontacturl={http://people.phys.ethz.ch/\xmptilde nbeisert/}}

\newcommand{\secref}[1]{\hyperref[#1]{section \ref*{#1}}}

\parskip1ex
\parindent0pt
\let\olditemize\itemize
\def\itemize{\olditemize\parskip0pt}

\begin{document}

\title{The \textsf{childdoc} Package}
\hypersetup{pdftitle={The childdoc Package}}
\author{Niklas Beisert\\[2ex]
  Institut f\"ur Theoretische Physik\\
  Eidgen\"ossische Technische Hochschule Z\"urich\\
  Wolfgang-Pauli-Strasse 27, 8093 Z\"urich, Switzerland\\[1ex]
  \href{mailto:nbeisert@itp.phys.ethz.ch}
  {\texttt{nbeisert@itp.phys.ethz.ch}}}
\hypersetup{pdfauthor={Niklas Beisert}}
\hypersetup{pdfsubject={Manual for the LaTeX2e Package childdoc}}
\date{30 December 2018, \textsf{v2.0}}
\maketitle

\begin{abstract}\noindent
\textsf{childdoc} is a \LaTeXe{} package
that enables the direct compilation
of document sections included by |\include|
to individual files.
\end{abstract}

\begingroup
\parskip0ex
\tableofcontents
\endgroup

%%%%%%%%%%%%%%%%%%%%%%%%%%%%%%%%%%%%%%%%%%%%%%%%%%%%%%%%%%%%%%%%%%%%%%%%%%%%%%%%
%%%%%%%%%%%%%%%%%%%%%%%%%%%%%%%%%%%%%%%%%%%%%%%%%%%%%%%%%%%%%%%%%%%%%%%%%%%%%%%%
\section{Introduction}

\LaTeX{} provides a mechanism to structure a large document (such as a book)
into a main file and several child files (containing the chapters)
using the |\include| command.
This mechanism is beneficial for documents
which span hundreds of pages in order to
make the source file(s) more manageable.
Moreover, compilation can be restricted to
selected child files by means of the |\includeonly| command.
The latter feature can be used to reduce the compilation time while editing
(this was significantly more useful in the earlier days of \LaTeX{})
or to generate a smaller document which is easier to navigate.
Another application of |\includeonly| is to generate
documents consisting of selected parts of the complete document.

However, there are a few drawbacks of the plain |\include| mechanism:
\begin{itemize}
\item
The child files cannot be compiled on their own,
they can only be compiled via the main file.
A naive editing environment
(such as a text editor with an option
to have the current file processed by \LaTeX)
may require one to switch to the main file before compiling;
attempting to compile the child file produces errors.
\item
The main file must be modified (each time)
to adjust the |\includeonly| command
to the present needs. This easily leaves the main file in a messy state.
\item
The generated document will always carry the filename
of the main document. This is inconvenient if
several child files are to be compiled and
to be kept for distribution.
\end{itemize}

The present package provides a simple interface
to make child files individually compilable by \LaTeX{}.
Compiling a child file then has the same effect as compiling
the main file with an |\includeonly| command
to select the appropriate child.
Moreover the generated document will carry the name of the child
rather than the main file.
This resolves all three above issues.

This feature is meant to make the editing of books,
thesis documents and lecture notes somewhat more convenient.
However, the package can also be used efficiently for
composing a series of documents (such as exercise sheets)
which are typically distributed individually.
It then assists the author in generating the individual documents
(potentially in different versions)
as well as a document containing the collected series.
Another application is in developing style files
or other kinds of included material
where compilation of the style file could redirect
to a sample or test file.

%%%%%%%%%%%%%%%%%%%%%%%%%%%%%%%%%%%%%%%%%%%%%%%%%%%%%%%%%%%%%%%%%%%%%%%%%%%%%%%%
%%%%%%%%%%%%%%%%%%%%%%%%%%%%%%%%%%%%%%%%%%%%%%%%%%%%%%%%%%%%%%%%%%%%%%%%%%%%%%%%
\section{Usage}

First of all, the package \textsf{childdoc} is \emph{not} a standard
\LaTeXe{} |.sty| style file! Therefore it needs to be invoked in
a non-standard way.

%%%%%%%%%%%%%%%%%%%%%%%%%%%%%%%%%%%%%%%%%%%%%%%%%%%%%%%%%%%%%%%%%%%%%%%%%%%%%%%%
\subsection{Included Files}
\label{sec:include}

%%%%%%%%%%%%%%%%%%%%%%%%%%%%%%%%%%%%%%%%
\DescribeMacro{\childdocmain}
To use the package, add the commands
\begin{center}
\begin{tabular}{l}
|\input{childdoc.def}|\\
|\childdocmain{}|\\
\end{tabular}
\end{center}
at the very top of the main \LaTeX{} file,
in particular \emph{before} the |\documentclass| statement!
The argument of |\childdocmain| should be left empty
(but it must be present).

%%%%%%%%%%%%%%%%%%%%%%%%%%%%%%%%%%%%%%%%
\DescribeMacro{\childdocof}
Furthermore, add the commands
\begin{center}
\begin{tabular}{l}
|\input{childdoc.def}|\\
|\childdocof{|\textit{main}|}|\\
\end{tabular}
\end{center}
at the top of every child file \textit{child}
which is included by |\include{|\textit{child}|}|
from within the main file
(or at least for those files to be compiled individually).
The argument \textit{main} must be the filename of the main file.

There are a couple of
considerations in setting up the main and child documents:

%%%%%%%%%%%%%%%%%%%%%%%%%%%%%%%%%%%%%%%%
\paragraph{Restrictions.}

Please note the following restrictions:
\begin{itemize}
\item
|\childdocmain| must be called with one argument \textit{main}
to ensure compatibility with earlier version of the package.
It must either be empty (|\childdocmain{}|)
or precisely match the filename of the main file in which it is specified.
See \secref{sec:detection} for further information.
\item
The filename \textit{main} must be specified without the |.tex| extension.
\item
The filename \textit{main} is case sensitive
(even in case-insensitive file systems)
due to internal string comparison.
\item
The argument \textit{main} should be fully expanded, it cannot be a macro.
\item
Subdirectories and special characters should be avoided in filenames.
\item
The command |\childdocmain{|\textit{main}|}| must be followed by a whitespace.
It should not be followed immediately by another command
or by a comment mark `|%|'.
This is because the \TeX{} parser reads the token immediately following
the argument of |\childdocmain| and puts it
at the beginning of every child section;
however, a white\-space is ignored.
\end{itemize}

%%%%%%%%%%%%%%%%%%%%%%%%%%%%%%%%%%%%%%%%
\paragraph{Content of Main File.}

It is advisable to place all content in the child files included by |\include|.
Any output contained in the main file will appear in all child documents
unless suppressed manually;
it cannot be suppressed automatically by the |\includeonly| directive
and thus should normally be avoided.
A method to include some content in the main file
by means of conditional processing is described in \secref{sec:conditional}.

%%%%%%%%%%%%%%%%%%%%%%%%%%%%%%%%%%%%%%%%
\paragraph{Page Numbering.}

When only a part of the document is compiled,
the appropriate numbering of pages
(as well as other status parameters)
is determined from the |.aux| files.
The latter contain information from previous passes.
However this information needs to propagate through
all intermediate child documents.
Therefore the page numbering in child documents may well
be inconsistent until the complete document is compiled at least once.

A useful (if unconventional) way to always ensure a consistent
page numbering is to restart the numbering in each child document
and denote the pages by `\textit{child}|.|\textit{page}'
where \textit{child} represents the chapter/section number of the child file.
This can be achieved by the command
|\numberwithin{page}{|\textit{child}|}|
of the \textsf{amsmath} package
where \textit{child} can be |chapter| or |section|
depending on the chosen structuring.
Alternatively, one can modify the macro |\thepage| appropriately
and reset the counter |page| at the start of each child file.

%%%%%%%%%%%%%%%%%%%%%%%%%%%%%%%%%%%%%%%%%%%%%%%%%%%%%%%%%%%%%%%%%%%%%%%%%%%%%%%%
\subsection{Conditional Processing}
\label{sec:conditional}

The package provides a mechanism to compile different versions
of a document. To customise the versions further some conditional processing
can come in handy to distinguish which version is being compiled.
The package provides two macros to describe the compilation context:

%%%%%%%%%%%%%%%%%%%%%%%%%%%%%%%%%%%%%%%%
\DescribeMacro{\ifchilddoc}
The conditional |\ifchilddoc| distinguishes between the compilation of
child documents and the main document:
%
\begin{center}
|\ifchilddoc |\textit{child-code}| |[|\||else |\textit{main-code}]| \||fi|
\end{center}

%%%%%%%%%%%%%%%%%%%%%%%%%%%%%%%%%%%%%%%%
\DescribeMacro{\childdocname}
\DescribeMacro{\childdocjob}
The macro |\childdocname| contains the filename (without extension)
of the main or child file being processed.
Note that |\childdocjob| will always contain the name of the main file.

%%%%%%%%%%%%%%%%%%%%%%%%%%%%%%%%%%%%%%%%
\paragraph{Title Page.}

Conditional processing can be used to include a title or banner page
in the main document when proper precautions are taken.
Importantly, the code in the main file should ensure that the page counter
(as well as other status parameters which are stored in the |.aux| files)
takes the same value after the conditional processing.
Otherwise the page numbers may take divergent values
depending on which part is compiled.

For example, a title page could be declared by:
%
\begin{center}
\begin{tabular}{l}
|\ifchilddoc\||else|\\
|\addtocounter{page}{-1}|\\
\textit{code for title page}\\
|\newpage|\\
|\||fi|
\end{tabular}
\end{center}
%
A banner page for the child documents can be generated by:
%
\begin{center}
\begin{tabular}{l}
|\ifchilddoc|\\
|\addtocounter{page}{-1}|\\
\textit{code for banner page}\\
|\newpage|\\
|\||fi|
\end{tabular}
\end{center}
%
Here one could write a message such as:
\begin{center}
|This is the part \childdocname{} of \childdocjob{}.|
\end{center}

%%%%%%%%%%%%%%%%%%%%%%%%%%%%%%%%%%%%%%%%%%%%%%%%%%%%%%%%%%%%%%%%%%%%%%%%%%%%%%%%
\subsection{Flags}
\label{sec:flags}

The package makes it easy to generate different versions
of the main or child documents.
To this end compilation flags can be defined
and assigned different default values.
They will be particularly useful in conjunction
with the forwarding mechanism described in \secref{sec:forward}.

For example, it may be useful to have a flag |\version|
which can be set to |draft| or |final|.
The document source will contain some conditional code
depending on the value of |\version|.
Suppose further, the flag should default to |final| for the main file
and to |draft| for child files
which is a natural assignment for editing the document.
This is achieved by placing the following code
in the preamble of the main document
(below the |\childdocmain| directive):
%
\begin{center}
\begin{tabular}{l}
|\ifchilddoc|\\
|\providecommand{\version}{draft}|\\
|\||else|\\
|\providecommand{\version}{final}|\\
|\||fi|
\end{tabular}
\end{center}
%
The definition by |\providecommand| makes sure
that previous definitions are not overwritten.
Further statements |\providecommand{\version}{...}|
can thus be added before the above code to override it.

For the main file, one might add a line
(between |\childdocmain| and the above block)
%
\begin{center}
|%\ifchilddoc\||else\providecommand{\version}{draft}\||fi|
\end{center}
%
which can be uncommented to produce a draft version.
Likewise one can add a line to the very top of a child file
(above the |\childdocof{|\textit{main}|}| directive)
%
\begin{center}
|%\providecommand{\version}{final}|
\end{center}
%
which can be uncommented to produce the final version of this child document.

%%%%%%%%%%%%%%%%%%%%%%%%%%%%%%%%%%%%%%%%%%%%%%%%%%%%%%%%%%%%%%%%%%%%%%%%%%%%%%%%
\subsection{Forwarding}
\label{sec:forward}

Different versions of the main or child documents
using compilation flags as described in \secref{sec:flags}
can be (permanently) stored in different files
for convenient compilation, viewing and distribution.
To this end, the package defines a command
to pass on compilation to a different file:

%%%%%%%%%%%%%%%%%%%%%%%%%%%%%%%%%%%%%%%%
\DescribeMacro{\childdocforward}
The command |\childdocforward| redirects processing to
another source file:
%
\begin{center}
\begin{tabular}{l}
|\input{childdoc.def}|\\
|\childdocforward[|\textit{main}|]{|\textit{dest}|}|\\
\end{tabular}
\end{center}
%
The argument \textit{dest} is the destination file
(without extension).
It should be the main file or one of the child files.
Note that further \textsf{childdoc} directives
such as |\childdocof| and |\childdocforward|
in the indicated file will be processed in this form.
The optional argument \textit{main}
passes on directly to the main file \textit{main}
while pretending to compile the child \textit{dest}.
This form behaves as if \textit{dest}
issues |\childdocof{|\textit{main}|}| right away,
and no further \textsf{childdoc} directives will be processed.

%%%%%%%%%%%%%%%%%%%%%%%%%%%%%%%%%%%%%%%%
\DescribeMacro{\...prefix}
In the alternative form |\childdocforwardprefix|,
%
\begin{center}
\begin{tabular}{l}
|\input{childdoc.def}|\\
|\childdocforwardprefix[|\textit{main}|]{|\textit{prefix}|}{|\textit{dest}|}|
\end{tabular}
\end{center}
%
the destination file is determined by a pattern
depending on the current file:
To make this work, the current file must be called
`{\textit{prefix}\hspace{0.2em}\textit{suffix}}'
with \textit{prefix} matching precisely the argument.
Processing is then passed on to the file
`{\textit{dest}\hspace{0.2em}\textit{suffix}}'.
Surely, the same effect is achieved by
directly specifying the
argument `{\textit{dest}\hspace{0.2em}\textit{suffix}}'
in the first form.
However, that requires to set up a different file
for each child. With the alternative form of the command
all these files can have exactly the same content
which simplifies setting them up and maintaining them.

For example, the following file |draft.tex|
with a compilation flag |\version| as described in \secref{sec:flags}
compiles the main document as a draft:
%
\begin{center}
\begin{tabular}{l}
|\def\version{draft}|\\
|\input{childdoc.def}|\\
|\childdocforward{|\textit{main}|}|
\end{tabular}
\end{center}
%
Likewise, the following files |final|\textit{nn}|.tex|
compile the final version of the child document
|child|\textit{nn}|.tex|:
%
\begin{center}
\begin{tabular}{l}
|\def\version{final}|\\
|\input{childdoc.def}|\\
|\childdocforwardprefix{final}{child}|
\end{tabular}
\end{center}
%

Note that when several versions of a main file and/or of each child file
are to be generated, it may be convenient to set up a |Makefile| or
shell script to automatise the process.

%%%%%%%%%%%%%%%%%%%%%%%%%%%%%%%%%%%%%%%%%%%%%%%%%%%%%%%%%%%%%%%%%%%%%%%%%%%%%%%%
\subsection{Command Line Processing}
\label{sec:commandline}

The effect of redirection files can also be achieved by invoking
the \LaTeX{} compiler with a more elaborate command line.
Most conveniently this should be done as part
of a shell script or a |Makefile|.

When using \textsf{childdoc} in the main file, the following
command lines effectively perform a redirection
(note that depending on the shell being used,
backslashes may have to be doubled: `|\|' $\to$ `|\\|'):
%
\begin{center}
|... -jobname "|\textit{target}|" |\\|"|[\textit{flags}]%
|\input{childdoc.def}\childdocforward[|\textit{main}|]{|\textit{dest}|}"|
\end{center}
%
Here \textit{target} is the name of the output file,
\textit{main} is the name of the main file
and \textit{dest} is the name of the main or child file to be processed
(all filenames without extensions).
The optional argument \textit{main} can be omitted
if \textit{main} matches \textit{dest}.
Optionally, compilation \textit{flags} can be defined via |\def| commands.
This command line makes the \TeX{} engine believe
it is compiling the file \textit{target}
whose content is specified as the latter parameter.
The provided code then forwards the processing to
\textit{main} or \textit{dest} as described in \secref{sec:forward}.

%%%%%%%%%%%%%%%%%%%%%%%%%%%%%%%%%%%%%%%%%%%%%%%%%%%%%%%%%%%%%%%%%%%%%%%%%%%%%%%%
\subsection{Include by Input}
\label{sec:input}

Including child documents by |\include| has some restrictions by design.
Most notably, the content of a child document always occupies
its own set of pages; pages cannot be shared between child documents.
Usually, this behaviour makes perfect sense
because each child document contain an essential part of the document.
However, in some situations it may be desirable to compose
a document from a collection of parts
without having mandatory page breaks between then.
For this case, the package
provides a mechanism to include parts
by |\input| which can also be processed individually.
However, by construction this mechanism
requires manual handling of the content to be output.

%%%%%%%%%%%%%%%%%%%%%%%%%%%%%%%%%%%%%%%%
\DescribeMacro{\ifchilddocmanual}
The main file should be prepared as usual, see \secref{sec:include}.
However, the document body must make a distinction
between processing of an individual part and of the main document, e.g.:
%
\begin{center}
\begin{tabular}{l}
|\ifchilddocmanual|\\
|\input{\childdocname}|\\
|\||else|\\
\textit{document body with }|\input{|\textit{part}|}|\\
|\||fi|
\end{tabular}
\end{center}
%
The conditional |\ifchilddocmanual| is true whenever
a part to be included by |\input| is being compiled,
and the name of the part is stored in |\childdocname|.

%%%%%%%%%%%%%%%%%%%%%%%%%%%%%%%%%%%%%%%%
\DescribeMacro{\childdocby}
Each part to be included by |\input| should start with:
%
\begin{center}
\begin{tabular}{l}
|\input{childdoc.def}|\\
|\childdocby{|\textit{main}|}|\\
\end{tabular}
\end{center}
%
The directive |\childdocby| is similar to |\childdocof|
described in \secref{sec:include},
but the subsequent selection of content must be done manually.
To that end, both |\ifchilddoc| and |\ifchilddocmanual|
will be true upon processing of a part,
and the name of the part is stored in |\childdocname|.
Note that |\jobname| will be set to the filename of the current part
so that each part receives an individual |.aux| file
that does not interfere with the |.aux| file(s) of the main document.
This behaviour can be altered by the alternative form
|\childdocby[*]{|\textit{main}|}| (with a non-empty optional argument)
which uses the |.aux| file of the main document
by setting |\jobname| to \textit{main}.

%%%%%%%%%%%%%%%%%%%%%%%%%%%%%%%%%%%%%%%%%%%%%%%%%%%%%%%%%%%%%%%%%%%%%%%%%%%%%%%%
\subsection{Driver Development}
\label{sec:driver}

The \textsf{childdoc} mechanism can also be use for the development
of definition files such as \LaTeX{} styles or classes.
This case differs from the above setup with multiple parts
included by |\include| in that no |\includeonly| should be invoked.
This can be achieved by starting the include file
(before |\ProvidesPackage|) with:
%
\begin{center}
\begin{tabular}{l}
|\input{childdoc.def}|\\
|\childdocforward{|\textit{main}|}|\\
\end{tabular}
\end{center}
%
or alternatively with:
%
\begin{center}
\begin{tabular}{l}
|\input{childdoc.def}|\\
|\childdocby{|\textit{main}|}|\\
\end{tabular}
\end{center}
%
Both forms have slightly different effects as described above.
The main file is prepared as usual, see \secref{sec:include}.

%%%%%%%%%%%%%%%%%%%%%%%%%%%%%%%%%%%%%%%%%%%%%%%%%%%%%%%%%%%%%%%%%%%%%%%%%%%%%%%%
\subsection{Legacy Detection}
\label{sec:detection}

The directive |\childdocmain| in the main file can detect
whether the complete document or merely a child is to be compiled
even without using the directive |\childdocof|.
This method is deprecated because it is less robust
and there is no compelling reason to use it;
it is merely provided for backward compatibility
and it may be removed in future versions.

If the detection mechanism is to be used,
it is mandatory to correctly specify
the filename of the main file as the argument of |\childdocmain|:
%
\begin{center}
\begin{tabular}{l}
|\input{childdoc.def}|\\
|\childdocmain{|\textit{main}|}|\\
\end{tabular}
\end{center}
%
If |\jobname| does not match the argument \textit{main} of |\childdocmain|,
it is assumed that |\jobname| points to the child file to be compiled.
When using |\childdocmain| with the main file specified as argument,
it suffices to start a child file
with just |\input{|\textit{main}|}|
without loading of the package and using |\childdocof|.
If instead all processing is done
with the appropriate \textsf{childdoc} directives,
the argument of \textit{main} of |\childdocmain| can be empty.

An alternative version of the command line processing described
in \secref{sec:commandline} using the detection mechanism reads:
%
\begin{center}
|... -jobname "|\textit{target}|" "|[\textit{flags}]%
[|\def\jobname{|\textit{dest}|}|]|\input{|\textit{main}|}"|
\end{center}

%%%%%%%%%%%%%%%%%%%%%%%%%%%%%%%%%%%%%%%%%%%%%%%%%%%%%%%%%%%%%%%%%%%%%%%%%%%%%%%%
\subsection{Manual Code}
\label{sec:manual}

In case one cannot be certain whether the definitions file |childdoc.def|
is installed on the target \TeX{} distribution
and one prefers not to ship it,
it is conceivable to paste a few relevant commands into the sources.

To that end, drop all statements |\input{childdoc.def}|
and perform the replacements as outlined below.
Instead of |\childdocmain{|\textit{main}|}| add the following code
to the top of the main file:
%
\begin{center}
\begin{tabular}{l}
|\||ifdefined\childdocname\endinput\||fi\newif\ifchilddoc|\\
|\edef\childdocname{\scantokens\expandafter{\jobname\noexpand}}|\\
|\def\childdocmain{|\textit{main}|}\||ifx\childdocmain\childdocname\||else|\\
|\childdoctrue\includeonly{\childdocname}\let\jobname\childdocmain\||fi|\\
\end{tabular}
\end{center}
%
Instead of |\childdocof{|\textit{main}|}| just include the main file
at the top of each child file:
%
\begin{center}
|\input{|\textit{main}|}|
\end{center}
%
A simple redirection |\childdocforward{|\textit{dest}|}| is achieved by:
%
\begin{center}
|\def\jobname{|\textit{dest}|}\input{\jobname}|
\end{center}
%
The redirection with prefix
|\childdocforwardprefix[|\textit{prefix}|]{|\textit{dest}|}|
is accomplished by:
%
\begin{center}
\begin{tabular}{l}
|{\edef\jobname{\scantokens\expandafter{\jobname\noexpand}}|\\
|\def\redirectjob |\textit{prefix}|#1~~~{\gdef\jobname{|\textit{dest}|#1}}|\\
|\expandafter\redirectjob\jobname~~~}\input{\jobname}|
\end{tabular}
\end{center}

In an alternative approach,
child documents can be compiled by a specific command line
without additional code or specific definitions:
%
\begin{center}
|... -jobname "|\textit{target}|" "|[\textit{flags}]%
|\includeonly{|\textit{dest}|}\input{|\textit{main}|}"|
\end{center}
%

%%%%%%%%%%%%%%%%%%%%%%%%%%%%%%%%%%%%%%%%%%%%%%%%%%%%%%%%%%%%%%%%%%%%%%%%%%%%%%%%
%%%%%%%%%%%%%%%%%%%%%%%%%%%%%%%%%%%%%%%%%%%%%%%%%%%%%%%%%%%%%%%%%%%%%%%%%%%%%%%%
\section{Information}

%%%%%%%%%%%%%%%%%%%%%%%%%%%%%%%%%%%%%%%%%%%%%%%%%%%%%%%%%%%%%%%%%%%%%%%%%%%%%%%%
\subsection{Copyright}

Copyright \copyright{} 2017--2018 Niklas Beisert

This work may be distributed and/or modified under the
conditions of the \LaTeX{} Project Public License, either version 1.3
of this license or (at your option) any later version.
The latest version of this license is in
  \url{http://www.latex-project.org/lppl.txt}
and version 1.3 or later is part of all distributions of \LaTeX{}
version 2005/12/01 or later.

This work has the LPPL maintenance status `maintained'.

The Current Maintainer of this work is Niklas Beisert.

This work consists of the files |README.txt|, |childdoc.ins| and |childdoc.dtx|
as well as the derived files |childdoc.def|, |cdocsamp.tex|
with |cdocsch1.tex|, |cdocsch2.tex|, |cdocspt3.tex|, |cdocspt4.tex|,
|cdocsdrf.tex|, |cdocsfn1.tex|, |cdocsfn2.tex|
as well as |childdoc.pdf|.

%%%%%%%%%%%%%%%%%%%%%%%%%%%%%%%%%%%%%%%%%%%%%%%%%%%%%%%%%%%%%%%%%%%%%%%%%%%%%%%%
\subsection{Files and Installation}

The package consists of the files:
%
\begin{center}
\begin{tabular}{ll}
    |README.txt|   & readme file \\
    |childdoc.ins| & installation file \\
    |childdoc.dtx| & source file \\
    |childdoc.def| & definition file \\
    |cdocsamp.tex| & sample main file \\
    |cdocsch1.tex| & sample include file \\
    |cdocsch2.tex| & sample include file \\
    |cdocspt3.tex| & sample part file \\
    |cdocspt4.tex| & sample part file \\
    |cdocsdrf.tex| & sample redirection file \\
    |cdocsfn1.tex| & sample redirection file \\
    |cdocsfn2.tex| & sample redirection file \\
    |childdoc.pdf| & manual
\end{tabular}
\end{center}
%
The distribution consists of the files
|README.txt|, |childdoc.ins| and |childdoc.dtx|.
%
\begin{itemize}
\item
Run (pdf)\LaTeX{} on |childdoc.dtx|
to compile the manual |childdoc.pdf| (this file).
\item
Run \LaTeX{} on |childdoc.ins| to create the definitions file |childdoc.def|
and the sample |cdocsamp.tex| with include files
|cdocsch1.tex|, |cdocsch2.tex|, |cdocspt3.tex|, |cdocspt4.tex|,
|cdocsdrf.tex|, |cdocsfn1.tex|, |cdocsfn2.tex|.
Then copy the file |childdoc.def| to an appropriate directory of your \LaTeX{}
distribution, e.g.\ \textit{texmf-root}|/tex/latex/childdoc|.
\end{itemize}

%%%%%%%%%%%%%%%%%%%%%%%%%%%%%%%%%%%%%%%%%%%%%%%%%%%%%%%%%%%%%%%%%%%%%%%%%%%%%%%%
\subsection{Related CTAN Packages}

There are several other packages which offer a similar functionality:
%
\begin{itemize}
\item
The packages
\href{http://ctan.org/pkg/docmute}{\textsf{docmute}},
\href{http://ctan.org/pkg/includex}{\textsf{includex}} and
\href{http://ctan.org/pkg/standalone}{\textsf{standalone}}
provide commands to include only the document body of
a child file thus allowing both files to be compiled individually.
\item
The packages \href{http://ctan.org/pkg/subdocs}{\textsf{subdocs}}
and \href{http://ctan.org/pkg/subfiles}{\textsf{subfiles}}
provide structures in which the main and child documents can be
encapsulated and allowing them to be compiled individually.
The inclusion mechanism is different from the conventional |\include|.
\item
The package \href{http://ctan.org/pkg/combine}{\textsf{combine}}
is an elaborate solution to combine several documents into one.
\end{itemize}
%
See also the CTAN topic \href{http://ctan.org/topic/subdocs}{\textsf{subdocs}}
for further related packages.
The present package differs from the above solutions in that
a document structure constructed with the conventional |\include| mechanism
just needs two extra commands at the top of every file
such that all constituent files can be compiled individually.

%%%%%%%%%%%%%%%%%%%%%%%%%%%%%%%%%%%%%%%%%%%%%%%%%%%%%%%%%%%%%%%%%%%%%%%%%%%%%%%%
%\subsection{Feature Suggestions}
%
%The following is a list of features which may be useful for future
%versions of this package:
%%
%\begin{itemize}
%\item
%\ldots
%\end{itemize}

%%%%%%%%%%%%%%%%%%%%%%%%%%%%%%%%%%%%%%%%%%%%%%%%%%%%%%%%%%%%%%%%%%%%%%%%%%%%%%%%
\subsection{Revision History}

%%%%%%%%%%%%%%%%%%%%%%%%%%%%%%%%%%%%%%%%
\paragraph{v2.0:} 2018/12/30

\begin{itemize}
\item
immediate forward processing
\item
added |\childdocby| mechanism
\item
manual restructured
\end{itemize}

%%%%%%%%%%%%%%%%%%%%%%%%%%%%%%%%%%%%%%%%
\paragraph{v1.6:} 2018/01/17

\begin{itemize}
\item
application for development of include files
\item
corrections to manual
\end{itemize}

%%%%%%%%%%%%%%%%%%%%%%%%%%%%%%%%%%%%%%%%
\paragraph{v1.5:} 2017/05/21

\begin{itemize}
\item
more complete structuring introduced
\item
|\childdocof| introduced
\item
|\childdoc| renamed to |\childdocmain|
\item
|\childredirect| renamed to |\childdocforward| and |\childdocforwardprefix|
and functionality expanded
\end{itemize}

%%%%%%%%%%%%%%%%%%%%%%%%%%%%%%%%%%%%%%%%
\paragraph{v1.0:} 2017/04/27

\begin{itemize}
\item
manual and install package
\item
first version published on CTAN
\end{itemize}

%%%%%%%%%%%%%%%%%%%%%%%%%%%%%%%%%%%%%%%%
\paragraph{v0.6:} 2017/04/26

\begin{itemize}
\item
redirection mechanism added
\end{itemize}

%%%%%%%%%%%%%%%%%%%%%%%%%%%%%%%%%%%%%%%%
\paragraph{v0.5:} 2017/04/26

\begin{itemize}
\item
functionality in definition file
\end{itemize}


%%%%%%%%%%%%%%%%%%%%%%%%%%%%%%%%%%%%%%%%%%%%%%%%%%%%%%%%%%%%%%%%%%%%%%%%%%%%%%%%
%%%%%%%%%%%%%%%%%%%%%%%%%%%%%%%%%%%%%%%%%%%%%%%%%%%%%%%%%%%%%%%%%%%%%%%%%%%%%%%%
%%%%%%%%%%%%%%%%%%%%%%%%%%%%%%%%%%%%%%%%%%%%%%%%%%%%%%%%%%%%%%%%%%%%%%%%%%%%%%%%
\appendix

\settowidth\MacroIndent{\rmfamily\scriptsize 000\ }

 \DocInput{childdoc.dtx}

\end{document}
%</driver>
% \fi
%
% %%%%%%%%%%%%%%%%%%%%%%%%%%%%%%%%%%%%%%%%%%%%%%%%%%%%%%%%%%%%%%%%%%%%%%%%%%%%%%
% %%%%%%%%%%%%%%%%%%%%%%%%%%%%%%%%%%%%%%%%%%%%%%%%%%%%%%%%%%%%%%%%%%%%%%%%%%%%%%
% \section{Sample}
%\iffalse
%<*samplemain>
%\fi
%
% The following presents a sample document
% with two chapters, two parts, a title page,
% a compile flag as well as three forwarding files to set the flag.
% It consists of eight |.tex| files:
% \begin{center}
% \begin{tabular}{ll}
% |cdocsamp.tex|&main file\\
% |cdocsch1.tex|&include file for chapter 1\\
% |cdocsch2.tex|&include file for chapter 2\\
% |cdocspt3.tex|&include file for part 3\\
% |cdocspt4.tex|&include file for part 4\\
% |cdocsdrf.tex|&forwarding file for main file in draft mode\\
% |cdocsfi1.tex|&forwarding file for final version of chapter 1\\
% |cdocsfi2.tex|&forwarding file for final version of chapter 2\\
% \end{tabular}
% \end{center}
% Each of the eight files can be compiled directly by the \LaTeX{} compiler.
%
% %%%%%%%%%%%%%%%%%%%%%%%%%%%%%%%%%%%%%%
% \paragraph{Main File.}
%
% The main file is called |cdocsamp.tex|.
%
% Load the \textsf{childdoc} definitions and
% declare the filename for the main document:
%    \begin{macrocode}
\input{childdoc.def}
\childdocmain{}
%    \end{macrocode}

% Optional override for |\version| flag:
%    \begin{macrocode}
%%\ifchilddoc\else\providecommand{\version}{draft}\fi
%    \end{macrocode}

% Define the default values for the |\version| flag
% (|final| for the main file and |draft| for childs):
%    \begin{macrocode}
\ifchilddoc
\providecommand{\version}{draft}
\else
\providecommand{\version}{final}
\fi
%    \end{macrocode}

% Load the standard document class:
%    \begin{macrocode}
\documentclass[12pt]{article}
%    \end{macrocode}

% Start the document body:
%    \begin{macrocode}
\begin{document}
%    \end{macrocode}

% Declare a title page.
% Print title, part of document being processed and version flag:
%    \begin{macrocode}
\addtocounter{page}{-1}
\begin{center}
{\LARGE\bfseries{}childdoc example\par}
\vspace{1cm}
\ifchilddoc
\ifchilddocmanual part\else chapter\fi:
`\childdocname' of `\childdocjob'\par
\else
main document: `\childdocjob'\par
\fi
version: \version\par
\end{center}
\newpage
%    \end{macrocode}

% Manually include selected file,
% otherwise process as usual:
%    \begin{macrocode}
\ifchilddocmanual
\section*{part `\childdocname'}
\input{\childdocname}
\else
%    \end{macrocode}

% Include the two chapters:
%    \begin{macrocode}
\include{cdocsch1}
\include{cdocsch2}
%    \end{macrocode}

% Include the two parts unless only chapters should be displayed:
%    \begin{macrocode}
\ifchilddoc\else
\section{part three}
\input{cdocspt3}
\section{part four}
\input{cdocspt4}
\fi
%    \end{macrocode}

% Process as usual until here:
%    \begin{macrocode}
\fi
%    \end{macrocode}

% End of document body:
%    \begin{macrocode}
\end{document}
%    \end{macrocode}
%\iffalse
%</samplemain>
%\fi
%
% %%%%%%%%%%%%%%%%%%%%%%%%%%%%%%%%%%%%%%
% \paragraph{Chapter Include Files.}
%
% The include files are called |cdocsch1.tex| and |cdocsch2.tex|.
%
%\iffalse
%<*samplechap1|samplechap2>
%\fi

% Optional override for |\version| flag:
%    \begin{macrocode}
%%\providecommand{\version}{final}
%    \end{macrocode}

% Include the main document:
%    \begin{macrocode}
\input{childdoc.def}
\childdocof{cdocsamp}
%    \end{macrocode}

%\iffalse
%</samplechap1|samplechap2>
%\fi
%
%\iffalse
%<*samplechap1>
%\fi
% Some text for chapter 1:
%    \begin{macrocode}
\section{one}
some text in chapter one
%    \end{macrocode}

%\iffalse
%</samplechap1>
%\fi
% Some text for chapter 2:
%\iffalse
%<*samplechap2>
%\fi
%    \begin{macrocode}
\section{two}
more text in chapter two
%    \end{macrocode}

%\iffalse
%</samplechap2>
%\fi
%
% %%%%%%%%%%%%%%%%%%%%%%%%%%%%%%%%%%%%%%
% \paragraph{Part Include Files.}
%
% The include files are called |cdocspt3.tex| and |cdocspt4.tex|.
%
%\iffalse
%<*samplepart3|samplepart4>
%\fi

% Optional override for |\version| flag:
%    \begin{macrocode}
%%\providecommand{\version}{final}
%    \end{macrocode}

% Include the main document:
%    \begin{macrocode}
\input{childdoc.def}
\childdocby{cdocsamp}
%    \end{macrocode}

%\iffalse
%</samplepart3|samplepart4>
%\fi
%
%\iffalse
%<*samplepart3>
%\fi
% Some text for part 3:
%    \begin{macrocode}
some text in part three
%    \end{macrocode}

%\iffalse
%</samplepart3>
%\fi
% Some text for part 4:
%\iffalse
%<*samplepart4>
%\fi
%    \begin{macrocode}
more text in part four
%    \end{macrocode}

%\iffalse
%</samplepart4>
%\fi
%
% %%%%%%%%%%%%%%%%%%%%%%%%%%%%%%%%%%%%%%
% \paragraph{Forwarding for a Complete Draft.}
%
% The following forwarding file |cdocsdrf.tex|
% compiles the main document in draft mode:
%\iffalse
%<*sampledraft>
%\fi
%    \begin{macrocode}
\def\version{draft}
\input{childdoc.def}
\childdocforward{cdocsamp}
%    \end{macrocode}

%\iffalse
%</sampledraft>
%\fi
%
% %%%%%%%%%%%%%%%%%%%%%%%%%%%%%%%%%%%%%%
% \paragraph{Forwarding for Final Version of the Chapters.}
%
% The following forwarding files |cdocsfn1.tex| and |cdocsfn2.tex|
% (with identical content)
% compile the final versions of the child documents
% |cdocsch1.tex| and |cdocsch2.tex|, respectively:
%\iffalse
%<*samplefinal>
%\fi
%    \begin{macrocode}
\def\version{final}
\input{childdoc.def}
\childdocforwardprefix[cdocsamp]{cdocsfn}{cdocsch}
%    \end{macrocode}

%\iffalse
%</samplefinal>
%\fi
%
% %%%%%%%%%%%%%%%%%%%%%%%%%%%%%%%%%%%%%%
% \paragraph{Command Line Processing.}
%
% The following three command lines generate the output files
% |cdocscld|, |cdocscl1| and |cdocscl2|
% which should be identical to
% |cdocsdrf|, |cdocsch1| and |cdocsfn2|, respectively:
% \begin{center}
% \begin{tabular}{l}
% |latex -jobname cdocscld \|\\
% |  "\def\version{draft}\input{childdoc.def}\childdocforward{cdocsamp}"|\\
% |latex -jobname cdocscl1 \|\\
% |  "\input{childdoc.def}\childdocforward[cdocsamp]{cdocsch1}"|\\
% |latex -jobname cdocscl2 \|\\
% |  "\def\version{final}\input{childdoc.def}\childdocforward{cdocsch2}"|
% \end{tabular}
% \end{center}
% Note that the trailing backslash on each first line
% merely continues the input to the second line
% (for convenient cut ant paste).
% Furthermore, the command |latex| can be replaced by any
% of its alternative versions such as |pdflatex|.
%
% %%%%%%%%%%%%%%%%%%%%%%%%%%%%%%%%%%%%%%%%%%%%%%%%%%%%%%%%%%%%%%%%%%%%%%%%%%%%%%
% %%%%%%%%%%%%%%%%%%%%%%%%%%%%%%%%%%%%%%%%%%%%%%%%%%%%%%%%%%%%%%%%%%%%%%%%%%%%%%
% \section{Implementation}
%\iffalse
%<*package>
%\fi
%
% This section describes the definitions file |childdoc.def|.

% The definitions cannot be loaded using |\usepackage| or |\RequirePackage|
% which has a mechanism to prevent loading a style file more than once.
% When loading the definitions by means of |\input|
% multiple instances have to be prevented manually:
%\iffalse
%This code needs to be before the `\ProvidesFile' directive
%which is defined at the beginning of this file.
%Therefore it is also placed there and commented out here.
%</package>
%<*discard>
%\fi
%    \begin{macrocode}
\ifdefined\childdocmain\endinput\fi
%    \end{macrocode}
%\iffalse
%</discard>
%<*package>
%\fi
%
% \macro{\ifchilddoc}
% \macro{\ifchilddocmanual}
% The conditional |\ifchilddoc| tells whether a
% child (true) or main (false) document is being compiled.
% The conditional |\ifchilddocmanual| tells whether
% the |\includeonly| mechanism is used (false) or
% the selection of child files must be performed manually (true).
% The definitions initialise to false:
%    \begin{macrocode}
\newif\ifchilddoc
\newif\ifchilddocmanual
%    \end{macrocode}

% \macro{\childdocname}
% \macro{\childdocjob}
% The macro |\childdocname| stores the name of the main document
% to be compiled. The macro |\childdocjob| stores the name of
% the document on which the \LaTeX{} compiler was originally invoked.
% The content of |\jobname| cannot be compared
% to filenames specified in the source due to different catcodes.
% The following code rescans |\jobname|, stores the result
% in |\childdocname| and saves a copy in |\childdocjob|:
%    \begin{macrocode}
\edef\childdocname{\scantokens\expandafter{\jobname\noexpand}}
\let\childdocjob\childdocname
%    \end{macrocode}

% \macro{\childdocdisable}
% The macro |\childdocdisable| prevents the main file
% from being processed more than once.
% At this stage, the main document command |\childdocmain|
% is assumed to be called once again where it should do nothing.
% Any subsequent call to it should prevent
% a secondary processing of the main document
% It overwrites the forwarding commands
% |\childdocof| and |\childdocforward|
% with empty macros to prevent further inclusions of the main document:
%    \begin{macrocode}
\newcommand{\childdocdisable}
{
  \renewcommand{\childdocmain}[1]{\renewcommand{\childdocmain}[1]{\endinput}}
  \renewcommand{\childdocof}[1]{}
  \renewcommand{\childdocby}[2][]{}
  \renewcommand{\childdocforward}[2][]{}
  \renewcommand{\childdocdisable}{}
}
%    \end{macrocode}

% \macro{\childdocmain}
% The macro |\childdocmain| is to be called at the top of the main file
% with nothing or the main filename (without extension) as argument.
% First, it breaks loops.
% If the argument is not empty and does not match |\childdocname|
% (which is set by the first inclusion of |childdoc.def|),
% |\ifchilddoc| is set to true, |\includeonly| is applied to the child file
% and |\jobname| is set to the main file
% (for proper handling of |.aux| files):
%    \begin{macrocode}
\newcommand{\childdocmain}[1]
{
  \childdocdisable\childdocmain{}
  \if?#1?\else
    \begingroup
      \def\childdoctmp{#1}
      \ifx\childdoctmp\childdocname
        \def\childdoctmp{}
      \else
        \def\childdoctmp
        {
          \childdoctrue
          \includeonly{\childdocname}
          \def\childdocjob{#1}
          \def\jobname{#1}
        }
      \fi
      \expandafter
    \endgroup
    \childdoctmp
  \fi
}
%    \end{macrocode}

% \macro{\childdocof}
% The command |\childdocof| redirects
% compilation to the main file |#1|.
%    \begin{macrocode}
\newcommand{\childdocof}[1]
{
  \childdocdisable
  \childdoctrue
  \includeonly{\childdocname}
  \def\jobname{#1}
  \def\childdocjob{#1}
  \input{#1}
}
%    \end{macrocode}

% \macro{\childdocby}
% The command |\childdocby| ....
%    \begin{macrocode}
\newcommand{\childdocby}[2][]
{
  \childdocdisable
  \childdoctrue
  \childdocmanualtrue
  \if?#1?\else
    \def\jobname{#2}
  \fi
  \def\childdocjob{#2}
  \input{#2}
  \endinput
}
%    \end{macrocode}

% \macro{\childdocforward}
% The command |\childdocforward| redirects
% compilation to the main file or
% (if the optional argument is given) a child file.
% Parameters are set as if the main file
% or a child file starting with |\childdocof| was compiled.
% Then compilation is handed over to the main file:
%    \begin{macrocode}
\newcommand{\childdocforward}[2][]
{
  \begingroup
    \if?#1?
      \def\childdoctmp
      {
        \def\childdocname{#2}
        \def\childdocjob{#2}
        \def\jobname{#2}
        \input{#2}
        \endinput
      }
    \else
      \def\childdoctmp
      {
        \childdocdisable
        \def\childdocname{#2}
        \childdoctrue
        \includeonly{#2}
        \def\childdocjob{#1}
        \def\jobname{#1}
        \input{#1}
        \endinput
      }
    \fi
    \expandafter
  \endgroup
  \childdoctmp
}
%    \end{macrocode}

% \macro{\childdocforwardprefix}
% The command |\childdocforwardprefix| redirects
% compilation to the main or a child file by means of a pattern.
% The prefix |#1| in the current filename is replaced by |#2|
% and the suffix of the current filename is kept
% (it is assumed that the filename does not contain the substring `|~~~|'
% which is used as a delimiter).
% Compilation is handed over to the new file by |\childdocforward|:
%    \begin{macrocode}
\newcommand{\childdocforwardprefix}[3][]
{
  \begingroup
    \def\childdocextract #2##1~~~{\def\childdoctmp{\childdocforward[#1]{#3##1}}}
    \expandafter\childdocextract\childdocname~~~
    \expandafter
  \endgroup
  \childdoctmp
}
%    \end{macrocode}

% \macro{\childdoc}
% The deprecated macro |\childdoc| is a legacy version of |\childdocmain|:
%    \begin{macrocode}
\newcommand{\childdoc}{\childdocmain}
%    \end{macrocode}

% \macro{\childdocredirect}
% The deprecated macro |\childdocredirect| is a legacy version
% of |\childdocforward| and |\childdocforwardprefix|:
%    \begin{macrocode}
\newcommand{\childdocredirect}[2][]
{
  \begingroup
    \if?#1?
      \def\childdoctmp{\childdocforward{#2}}
    \else
      \def\childdoctmp{\childdocforwardprefix{#1}{#2}}
    \fi
    \expandafter
  \endgroup
  \childdoctmp
}
%    \end{macrocode}

%\iffalse
%</package>
%\fi
%
\endinput
|\\
|\childdocby{|\textit{main}|}|\\
\end{tabular}
\end{center}
%
The directive |\childdocby| is similar to |\childdocof|
described in \secref{sec:include},
but the subsequent selection of content must be done manually.
To that end, both |\ifchilddoc| and |\ifchilddocmanual|
will be true upon processing of a part,
and the name of the part is stored in |\childdocname|.
Note that |\jobname| will be set to the filename of the current part
so that each part receives an individual |.aux| file
that does not interfere with the |.aux| file(s) of the main document.
This behaviour can be altered by the alternative form
|\childdocby[*]{|\textit{main}|}| (with a non-empty optional argument)
which uses the |.aux| file of the main document
by setting |\jobname| to \textit{main}.

%%%%%%%%%%%%%%%%%%%%%%%%%%%%%%%%%%%%%%%%%%%%%%%%%%%%%%%%%%%%%%%%%%%%%%%%%%%%%%%%
\subsection{Driver Development}
\label{sec:driver}

The \textsf{childdoc} mechanism can also be use for the development
of definition files such as \LaTeX{} styles or classes.
This case differs from the above setup with multiple parts
included by |\include| in that no |\includeonly| should be invoked.
This can be achieved by starting the include file
(before |\ProvidesPackage|) with:
%
\begin{center}
\begin{tabular}{l}
|% \iffalse
%
% childdoc.dtx Copyright (C) 2017-2018 Niklas Beisert
%
% This work may be distributed and/or modified under the
% conditions of the LaTeX Project Public License, either version 1.3
% of this license or (at your option) any later version.
% The latest version of this license is in
%   http://www.latex-project.org/lppl.txt
% and version 1.3 or later is part of all distributions of LaTeX
% version 2005/12/01 or later.
%
% This work has the LPPL maintenance status `maintained'.
%
% The Current Maintainer of this work is Niklas Beisert.
%
% This work consists of the files childdoc.dtx and childdoc.ins
% and the derived files childdoc.def and cdocsamp.tex with
% cdocsch1.tex, cdocsch2.tex, cdocsdrf.tex, cdocsfn1.tex, cdocsfn2.tex.
%
%<package>\ifdefined\childdocmain\endinput\fi
%<package>\ProvidesFile{childdoc.def}[2018/12/30 v2.0 child document driver]
%<samplemain>\ProvidesFile{cdocsamp.tex}[2018/12/30 v2.0 sample for childdoc]
%<*driver>
%\ProvidesFile{childdoc.drv}[2018/12/30 v2.0 childdoc reference manual file]
\PassOptionsToClass{10pt,a4paper}{article}
\documentclass{ltxdoc}

\usepackage[margin=35mm]{geometry}
\usepackage{hyperref}
\usepackage{hyperxmp}
\usepackage[usenames]{color}

\hypersetup{colorlinks=true}
\hypersetup{pdfstartview=FitH}
\hypersetup{pdfpagemode=UseNone}
\hypersetup{pdfsource={}}
\hypersetup{pdflang={en-UK}}
\hypersetup{pdfcopyright={Copyright 2017-2018 Niklas Beisert.
  This work may be distributed and/or modified under the
  conditions of the LaTeX Project Public License, either version 1.3
  of this license or (at your option) any later version.}}
\hypersetup{pdflicenseurl={http://www.latex-project.org/lppl.txt}}
\hypersetup{pdfcontactaddress={ETH Zurich, ITP, HIT K,
  Wolfgang-Pauli-Strasse 27}}
\hypersetup{pdfcontactpostcode={8093}}
\hypersetup{pdfcontactcity={Zurich}}
\hypersetup{pdfcontactcountry={Switzerland}}
\hypersetup{pdfcontactemail={nbeisert@itp.phys.ethz.ch}}
\hypersetup{pdfcontacturl={http://people.phys.ethz.ch/\xmptilde nbeisert/}}

\newcommand{\secref}[1]{\hyperref[#1]{section \ref*{#1}}}

\parskip1ex
\parindent0pt
\let\olditemize\itemize
\def\itemize{\olditemize\parskip0pt}

\begin{document}

\title{The \textsf{childdoc} Package}
\hypersetup{pdftitle={The childdoc Package}}
\author{Niklas Beisert\\[2ex]
  Institut f\"ur Theoretische Physik\\
  Eidgen\"ossische Technische Hochschule Z\"urich\\
  Wolfgang-Pauli-Strasse 27, 8093 Z\"urich, Switzerland\\[1ex]
  \href{mailto:nbeisert@itp.phys.ethz.ch}
  {\texttt{nbeisert@itp.phys.ethz.ch}}}
\hypersetup{pdfauthor={Niklas Beisert}}
\hypersetup{pdfsubject={Manual for the LaTeX2e Package childdoc}}
\date{30 December 2018, \textsf{v2.0}}
\maketitle

\begin{abstract}\noindent
\textsf{childdoc} is a \LaTeXe{} package
that enables the direct compilation
of document sections included by |\include|
to individual files.
\end{abstract}

\begingroup
\parskip0ex
\tableofcontents
\endgroup

%%%%%%%%%%%%%%%%%%%%%%%%%%%%%%%%%%%%%%%%%%%%%%%%%%%%%%%%%%%%%%%%%%%%%%%%%%%%%%%%
%%%%%%%%%%%%%%%%%%%%%%%%%%%%%%%%%%%%%%%%%%%%%%%%%%%%%%%%%%%%%%%%%%%%%%%%%%%%%%%%
\section{Introduction}

\LaTeX{} provides a mechanism to structure a large document (such as a book)
into a main file and several child files (containing the chapters)
using the |\include| command.
This mechanism is beneficial for documents
which span hundreds of pages in order to
make the source file(s) more manageable.
Moreover, compilation can be restricted to
selected child files by means of the |\includeonly| command.
The latter feature can be used to reduce the compilation time while editing
(this was significantly more useful in the earlier days of \LaTeX{})
or to generate a smaller document which is easier to navigate.
Another application of |\includeonly| is to generate
documents consisting of selected parts of the complete document.

However, there are a few drawbacks of the plain |\include| mechanism:
\begin{itemize}
\item
The child files cannot be compiled on their own,
they can only be compiled via the main file.
A naive editing environment
(such as a text editor with an option
to have the current file processed by \LaTeX)
may require one to switch to the main file before compiling;
attempting to compile the child file produces errors.
\item
The main file must be modified (each time)
to adjust the |\includeonly| command
to the present needs. This easily leaves the main file in a messy state.
\item
The generated document will always carry the filename
of the main document. This is inconvenient if
several child files are to be compiled and
to be kept for distribution.
\end{itemize}

The present package provides a simple interface
to make child files individually compilable by \LaTeX{}.
Compiling a child file then has the same effect as compiling
the main file with an |\includeonly| command
to select the appropriate child.
Moreover the generated document will carry the name of the child
rather than the main file.
This resolves all three above issues.

This feature is meant to make the editing of books,
thesis documents and lecture notes somewhat more convenient.
However, the package can also be used efficiently for
composing a series of documents (such as exercise sheets)
which are typically distributed individually.
It then assists the author in generating the individual documents
(potentially in different versions)
as well as a document containing the collected series.
Another application is in developing style files
or other kinds of included material
where compilation of the style file could redirect
to a sample or test file.

%%%%%%%%%%%%%%%%%%%%%%%%%%%%%%%%%%%%%%%%%%%%%%%%%%%%%%%%%%%%%%%%%%%%%%%%%%%%%%%%
%%%%%%%%%%%%%%%%%%%%%%%%%%%%%%%%%%%%%%%%%%%%%%%%%%%%%%%%%%%%%%%%%%%%%%%%%%%%%%%%
\section{Usage}

First of all, the package \textsf{childdoc} is \emph{not} a standard
\LaTeXe{} |.sty| style file! Therefore it needs to be invoked in
a non-standard way.

%%%%%%%%%%%%%%%%%%%%%%%%%%%%%%%%%%%%%%%%%%%%%%%%%%%%%%%%%%%%%%%%%%%%%%%%%%%%%%%%
\subsection{Included Files}
\label{sec:include}

%%%%%%%%%%%%%%%%%%%%%%%%%%%%%%%%%%%%%%%%
\DescribeMacro{\childdocmain}
To use the package, add the commands
\begin{center}
\begin{tabular}{l}
|\input{childdoc.def}|\\
|\childdocmain{}|\\
\end{tabular}
\end{center}
at the very top of the main \LaTeX{} file,
in particular \emph{before} the |\documentclass| statement!
The argument of |\childdocmain| should be left empty
(but it must be present).

%%%%%%%%%%%%%%%%%%%%%%%%%%%%%%%%%%%%%%%%
\DescribeMacro{\childdocof}
Furthermore, add the commands
\begin{center}
\begin{tabular}{l}
|\input{childdoc.def}|\\
|\childdocof{|\textit{main}|}|\\
\end{tabular}
\end{center}
at the top of every child file \textit{child}
which is included by |\include{|\textit{child}|}|
from within the main file
(or at least for those files to be compiled individually).
The argument \textit{main} must be the filename of the main file.

There are a couple of
considerations in setting up the main and child documents:

%%%%%%%%%%%%%%%%%%%%%%%%%%%%%%%%%%%%%%%%
\paragraph{Restrictions.}

Please note the following restrictions:
\begin{itemize}
\item
|\childdocmain| must be called with one argument \textit{main}
to ensure compatibility with earlier version of the package.
It must either be empty (|\childdocmain{}|)
or precisely match the filename of the main file in which it is specified.
See \secref{sec:detection} for further information.
\item
The filename \textit{main} must be specified without the |.tex| extension.
\item
The filename \textit{main} is case sensitive
(even in case-insensitive file systems)
due to internal string comparison.
\item
The argument \textit{main} should be fully expanded, it cannot be a macro.
\item
Subdirectories and special characters should be avoided in filenames.
\item
The command |\childdocmain{|\textit{main}|}| must be followed by a whitespace.
It should not be followed immediately by another command
or by a comment mark `|%|'.
This is because the \TeX{} parser reads the token immediately following
the argument of |\childdocmain| and puts it
at the beginning of every child section;
however, a white\-space is ignored.
\end{itemize}

%%%%%%%%%%%%%%%%%%%%%%%%%%%%%%%%%%%%%%%%
\paragraph{Content of Main File.}

It is advisable to place all content in the child files included by |\include|.
Any output contained in the main file will appear in all child documents
unless suppressed manually;
it cannot be suppressed automatically by the |\includeonly| directive
and thus should normally be avoided.
A method to include some content in the main file
by means of conditional processing is described in \secref{sec:conditional}.

%%%%%%%%%%%%%%%%%%%%%%%%%%%%%%%%%%%%%%%%
\paragraph{Page Numbering.}

When only a part of the document is compiled,
the appropriate numbering of pages
(as well as other status parameters)
is determined from the |.aux| files.
The latter contain information from previous passes.
However this information needs to propagate through
all intermediate child documents.
Therefore the page numbering in child documents may well
be inconsistent until the complete document is compiled at least once.

A useful (if unconventional) way to always ensure a consistent
page numbering is to restart the numbering in each child document
and denote the pages by `\textit{child}|.|\textit{page}'
where \textit{child} represents the chapter/section number of the child file.
This can be achieved by the command
|\numberwithin{page}{|\textit{child}|}|
of the \textsf{amsmath} package
where \textit{child} can be |chapter| or |section|
depending on the chosen structuring.
Alternatively, one can modify the macro |\thepage| appropriately
and reset the counter |page| at the start of each child file.

%%%%%%%%%%%%%%%%%%%%%%%%%%%%%%%%%%%%%%%%%%%%%%%%%%%%%%%%%%%%%%%%%%%%%%%%%%%%%%%%
\subsection{Conditional Processing}
\label{sec:conditional}

The package provides a mechanism to compile different versions
of a document. To customise the versions further some conditional processing
can come in handy to distinguish which version is being compiled.
The package provides two macros to describe the compilation context:

%%%%%%%%%%%%%%%%%%%%%%%%%%%%%%%%%%%%%%%%
\DescribeMacro{\ifchilddoc}
The conditional |\ifchilddoc| distinguishes between the compilation of
child documents and the main document:
%
\begin{center}
|\ifchilddoc |\textit{child-code}| |[|\||else |\textit{main-code}]| \||fi|
\end{center}

%%%%%%%%%%%%%%%%%%%%%%%%%%%%%%%%%%%%%%%%
\DescribeMacro{\childdocname}
\DescribeMacro{\childdocjob}
The macro |\childdocname| contains the filename (without extension)
of the main or child file being processed.
Note that |\childdocjob| will always contain the name of the main file.

%%%%%%%%%%%%%%%%%%%%%%%%%%%%%%%%%%%%%%%%
\paragraph{Title Page.}

Conditional processing can be used to include a title or banner page
in the main document when proper precautions are taken.
Importantly, the code in the main file should ensure that the page counter
(as well as other status parameters which are stored in the |.aux| files)
takes the same value after the conditional processing.
Otherwise the page numbers may take divergent values
depending on which part is compiled.

For example, a title page could be declared by:
%
\begin{center}
\begin{tabular}{l}
|\ifchilddoc\||else|\\
|\addtocounter{page}{-1}|\\
\textit{code for title page}\\
|\newpage|\\
|\||fi|
\end{tabular}
\end{center}
%
A banner page for the child documents can be generated by:
%
\begin{center}
\begin{tabular}{l}
|\ifchilddoc|\\
|\addtocounter{page}{-1}|\\
\textit{code for banner page}\\
|\newpage|\\
|\||fi|
\end{tabular}
\end{center}
%
Here one could write a message such as:
\begin{center}
|This is the part \childdocname{} of \childdocjob{}.|
\end{center}

%%%%%%%%%%%%%%%%%%%%%%%%%%%%%%%%%%%%%%%%%%%%%%%%%%%%%%%%%%%%%%%%%%%%%%%%%%%%%%%%
\subsection{Flags}
\label{sec:flags}

The package makes it easy to generate different versions
of the main or child documents.
To this end compilation flags can be defined
and assigned different default values.
They will be particularly useful in conjunction
with the forwarding mechanism described in \secref{sec:forward}.

For example, it may be useful to have a flag |\version|
which can be set to |draft| or |final|.
The document source will contain some conditional code
depending on the value of |\version|.
Suppose further, the flag should default to |final| for the main file
and to |draft| for child files
which is a natural assignment for editing the document.
This is achieved by placing the following code
in the preamble of the main document
(below the |\childdocmain| directive):
%
\begin{center}
\begin{tabular}{l}
|\ifchilddoc|\\
|\providecommand{\version}{draft}|\\
|\||else|\\
|\providecommand{\version}{final}|\\
|\||fi|
\end{tabular}
\end{center}
%
The definition by |\providecommand| makes sure
that previous definitions are not overwritten.
Further statements |\providecommand{\version}{...}|
can thus be added before the above code to override it.

For the main file, one might add a line
(between |\childdocmain| and the above block)
%
\begin{center}
|%\ifchilddoc\||else\providecommand{\version}{draft}\||fi|
\end{center}
%
which can be uncommented to produce a draft version.
Likewise one can add a line to the very top of a child file
(above the |\childdocof{|\textit{main}|}| directive)
%
\begin{center}
|%\providecommand{\version}{final}|
\end{center}
%
which can be uncommented to produce the final version of this child document.

%%%%%%%%%%%%%%%%%%%%%%%%%%%%%%%%%%%%%%%%%%%%%%%%%%%%%%%%%%%%%%%%%%%%%%%%%%%%%%%%
\subsection{Forwarding}
\label{sec:forward}

Different versions of the main or child documents
using compilation flags as described in \secref{sec:flags}
can be (permanently) stored in different files
for convenient compilation, viewing and distribution.
To this end, the package defines a command
to pass on compilation to a different file:

%%%%%%%%%%%%%%%%%%%%%%%%%%%%%%%%%%%%%%%%
\DescribeMacro{\childdocforward}
The command |\childdocforward| redirects processing to
another source file:
%
\begin{center}
\begin{tabular}{l}
|\input{childdoc.def}|\\
|\childdocforward[|\textit{main}|]{|\textit{dest}|}|\\
\end{tabular}
\end{center}
%
The argument \textit{dest} is the destination file
(without extension).
It should be the main file or one of the child files.
Note that further \textsf{childdoc} directives
such as |\childdocof| and |\childdocforward|
in the indicated file will be processed in this form.
The optional argument \textit{main}
passes on directly to the main file \textit{main}
while pretending to compile the child \textit{dest}.
This form behaves as if \textit{dest}
issues |\childdocof{|\textit{main}|}| right away,
and no further \textsf{childdoc} directives will be processed.

%%%%%%%%%%%%%%%%%%%%%%%%%%%%%%%%%%%%%%%%
\DescribeMacro{\...prefix}
In the alternative form |\childdocforwardprefix|,
%
\begin{center}
\begin{tabular}{l}
|\input{childdoc.def}|\\
|\childdocforwardprefix[|\textit{main}|]{|\textit{prefix}|}{|\textit{dest}|}|
\end{tabular}
\end{center}
%
the destination file is determined by a pattern
depending on the current file:
To make this work, the current file must be called
`{\textit{prefix}\hspace{0.2em}\textit{suffix}}'
with \textit{prefix} matching precisely the argument.
Processing is then passed on to the file
`{\textit{dest}\hspace{0.2em}\textit{suffix}}'.
Surely, the same effect is achieved by
directly specifying the
argument `{\textit{dest}\hspace{0.2em}\textit{suffix}}'
in the first form.
However, that requires to set up a different file
for each child. With the alternative form of the command
all these files can have exactly the same content
which simplifies setting them up and maintaining them.

For example, the following file |draft.tex|
with a compilation flag |\version| as described in \secref{sec:flags}
compiles the main document as a draft:
%
\begin{center}
\begin{tabular}{l}
|\def\version{draft}|\\
|\input{childdoc.def}|\\
|\childdocforward{|\textit{main}|}|
\end{tabular}
\end{center}
%
Likewise, the following files |final|\textit{nn}|.tex|
compile the final version of the child document
|child|\textit{nn}|.tex|:
%
\begin{center}
\begin{tabular}{l}
|\def\version{final}|\\
|\input{childdoc.def}|\\
|\childdocforwardprefix{final}{child}|
\end{tabular}
\end{center}
%

Note that when several versions of a main file and/or of each child file
are to be generated, it may be convenient to set up a |Makefile| or
shell script to automatise the process.

%%%%%%%%%%%%%%%%%%%%%%%%%%%%%%%%%%%%%%%%%%%%%%%%%%%%%%%%%%%%%%%%%%%%%%%%%%%%%%%%
\subsection{Command Line Processing}
\label{sec:commandline}

The effect of redirection files can also be achieved by invoking
the \LaTeX{} compiler with a more elaborate command line.
Most conveniently this should be done as part
of a shell script or a |Makefile|.

When using \textsf{childdoc} in the main file, the following
command lines effectively perform a redirection
(note that depending on the shell being used,
backslashes may have to be doubled: `|\|' $\to$ `|\\|'):
%
\begin{center}
|... -jobname "|\textit{target}|" |\\|"|[\textit{flags}]%
|\input{childdoc.def}\childdocforward[|\textit{main}|]{|\textit{dest}|}"|
\end{center}
%
Here \textit{target} is the name of the output file,
\textit{main} is the name of the main file
and \textit{dest} is the name of the main or child file to be processed
(all filenames without extensions).
The optional argument \textit{main} can be omitted
if \textit{main} matches \textit{dest}.
Optionally, compilation \textit{flags} can be defined via |\def| commands.
This command line makes the \TeX{} engine believe
it is compiling the file \textit{target}
whose content is specified as the latter parameter.
The provided code then forwards the processing to
\textit{main} or \textit{dest} as described in \secref{sec:forward}.

%%%%%%%%%%%%%%%%%%%%%%%%%%%%%%%%%%%%%%%%%%%%%%%%%%%%%%%%%%%%%%%%%%%%%%%%%%%%%%%%
\subsection{Include by Input}
\label{sec:input}

Including child documents by |\include| has some restrictions by design.
Most notably, the content of a child document always occupies
its own set of pages; pages cannot be shared between child documents.
Usually, this behaviour makes perfect sense
because each child document contain an essential part of the document.
However, in some situations it may be desirable to compose
a document from a collection of parts
without having mandatory page breaks between then.
For this case, the package
provides a mechanism to include parts
by |\input| which can also be processed individually.
However, by construction this mechanism
requires manual handling of the content to be output.

%%%%%%%%%%%%%%%%%%%%%%%%%%%%%%%%%%%%%%%%
\DescribeMacro{\ifchilddocmanual}
The main file should be prepared as usual, see \secref{sec:include}.
However, the document body must make a distinction
between processing of an individual part and of the main document, e.g.:
%
\begin{center}
\begin{tabular}{l}
|\ifchilddocmanual|\\
|\input{\childdocname}|\\
|\||else|\\
\textit{document body with }|\input{|\textit{part}|}|\\
|\||fi|
\end{tabular}
\end{center}
%
The conditional |\ifchilddocmanual| is true whenever
a part to be included by |\input| is being compiled,
and the name of the part is stored in |\childdocname|.

%%%%%%%%%%%%%%%%%%%%%%%%%%%%%%%%%%%%%%%%
\DescribeMacro{\childdocby}
Each part to be included by |\input| should start with:
%
\begin{center}
\begin{tabular}{l}
|\input{childdoc.def}|\\
|\childdocby{|\textit{main}|}|\\
\end{tabular}
\end{center}
%
The directive |\childdocby| is similar to |\childdocof|
described in \secref{sec:include},
but the subsequent selection of content must be done manually.
To that end, both |\ifchilddoc| and |\ifchilddocmanual|
will be true upon processing of a part,
and the name of the part is stored in |\childdocname|.
Note that |\jobname| will be set to the filename of the current part
so that each part receives an individual |.aux| file
that does not interfere with the |.aux| file(s) of the main document.
This behaviour can be altered by the alternative form
|\childdocby[*]{|\textit{main}|}| (with a non-empty optional argument)
which uses the |.aux| file of the main document
by setting |\jobname| to \textit{main}.

%%%%%%%%%%%%%%%%%%%%%%%%%%%%%%%%%%%%%%%%%%%%%%%%%%%%%%%%%%%%%%%%%%%%%%%%%%%%%%%%
\subsection{Driver Development}
\label{sec:driver}

The \textsf{childdoc} mechanism can also be use for the development
of definition files such as \LaTeX{} styles or classes.
This case differs from the above setup with multiple parts
included by |\include| in that no |\includeonly| should be invoked.
This can be achieved by starting the include file
(before |\ProvidesPackage|) with:
%
\begin{center}
\begin{tabular}{l}
|\input{childdoc.def}|\\
|\childdocforward{|\textit{main}|}|\\
\end{tabular}
\end{center}
%
or alternatively with:
%
\begin{center}
\begin{tabular}{l}
|\input{childdoc.def}|\\
|\childdocby{|\textit{main}|}|\\
\end{tabular}
\end{center}
%
Both forms have slightly different effects as described above.
The main file is prepared as usual, see \secref{sec:include}.

%%%%%%%%%%%%%%%%%%%%%%%%%%%%%%%%%%%%%%%%%%%%%%%%%%%%%%%%%%%%%%%%%%%%%%%%%%%%%%%%
\subsection{Legacy Detection}
\label{sec:detection}

The directive |\childdocmain| in the main file can detect
whether the complete document or merely a child is to be compiled
even without using the directive |\childdocof|.
This method is deprecated because it is less robust
and there is no compelling reason to use it;
it is merely provided for backward compatibility
and it may be removed in future versions.

If the detection mechanism is to be used,
it is mandatory to correctly specify
the filename of the main file as the argument of |\childdocmain|:
%
\begin{center}
\begin{tabular}{l}
|\input{childdoc.def}|\\
|\childdocmain{|\textit{main}|}|\\
\end{tabular}
\end{center}
%
If |\jobname| does not match the argument \textit{main} of |\childdocmain|,
it is assumed that |\jobname| points to the child file to be compiled.
When using |\childdocmain| with the main file specified as argument,
it suffices to start a child file
with just |\input{|\textit{main}|}|
without loading of the package and using |\childdocof|.
If instead all processing is done
with the appropriate \textsf{childdoc} directives,
the argument of \textit{main} of |\childdocmain| can be empty.

An alternative version of the command line processing described
in \secref{sec:commandline} using the detection mechanism reads:
%
\begin{center}
|... -jobname "|\textit{target}|" "|[\textit{flags}]%
[|\def\jobname{|\textit{dest}|}|]|\input{|\textit{main}|}"|
\end{center}

%%%%%%%%%%%%%%%%%%%%%%%%%%%%%%%%%%%%%%%%%%%%%%%%%%%%%%%%%%%%%%%%%%%%%%%%%%%%%%%%
\subsection{Manual Code}
\label{sec:manual}

In case one cannot be certain whether the definitions file |childdoc.def|
is installed on the target \TeX{} distribution
and one prefers not to ship it,
it is conceivable to paste a few relevant commands into the sources.

To that end, drop all statements |\input{childdoc.def}|
and perform the replacements as outlined below.
Instead of |\childdocmain{|\textit{main}|}| add the following code
to the top of the main file:
%
\begin{center}
\begin{tabular}{l}
|\||ifdefined\childdocname\endinput\||fi\newif\ifchilddoc|\\
|\edef\childdocname{\scantokens\expandafter{\jobname\noexpand}}|\\
|\def\childdocmain{|\textit{main}|}\||ifx\childdocmain\childdocname\||else|\\
|\childdoctrue\includeonly{\childdocname}\let\jobname\childdocmain\||fi|\\
\end{tabular}
\end{center}
%
Instead of |\childdocof{|\textit{main}|}| just include the main file
at the top of each child file:
%
\begin{center}
|\input{|\textit{main}|}|
\end{center}
%
A simple redirection |\childdocforward{|\textit{dest}|}| is achieved by:
%
\begin{center}
|\def\jobname{|\textit{dest}|}\input{\jobname}|
\end{center}
%
The redirection with prefix
|\childdocforwardprefix[|\textit{prefix}|]{|\textit{dest}|}|
is accomplished by:
%
\begin{center}
\begin{tabular}{l}
|{\edef\jobname{\scantokens\expandafter{\jobname\noexpand}}|\\
|\def\redirectjob |\textit{prefix}|#1~~~{\gdef\jobname{|\textit{dest}|#1}}|\\
|\expandafter\redirectjob\jobname~~~}\input{\jobname}|
\end{tabular}
\end{center}

In an alternative approach,
child documents can be compiled by a specific command line
without additional code or specific definitions:
%
\begin{center}
|... -jobname "|\textit{target}|" "|[\textit{flags}]%
|\includeonly{|\textit{dest}|}\input{|\textit{main}|}"|
\end{center}
%

%%%%%%%%%%%%%%%%%%%%%%%%%%%%%%%%%%%%%%%%%%%%%%%%%%%%%%%%%%%%%%%%%%%%%%%%%%%%%%%%
%%%%%%%%%%%%%%%%%%%%%%%%%%%%%%%%%%%%%%%%%%%%%%%%%%%%%%%%%%%%%%%%%%%%%%%%%%%%%%%%
\section{Information}

%%%%%%%%%%%%%%%%%%%%%%%%%%%%%%%%%%%%%%%%%%%%%%%%%%%%%%%%%%%%%%%%%%%%%%%%%%%%%%%%
\subsection{Copyright}

Copyright \copyright{} 2017--2018 Niklas Beisert

This work may be distributed and/or modified under the
conditions of the \LaTeX{} Project Public License, either version 1.3
of this license or (at your option) any later version.
The latest version of this license is in
  \url{http://www.latex-project.org/lppl.txt}
and version 1.3 or later is part of all distributions of \LaTeX{}
version 2005/12/01 or later.

This work has the LPPL maintenance status `maintained'.

The Current Maintainer of this work is Niklas Beisert.

This work consists of the files |README.txt|, |childdoc.ins| and |childdoc.dtx|
as well as the derived files |childdoc.def|, |cdocsamp.tex|
with |cdocsch1.tex|, |cdocsch2.tex|, |cdocspt3.tex|, |cdocspt4.tex|,
|cdocsdrf.tex|, |cdocsfn1.tex|, |cdocsfn2.tex|
as well as |childdoc.pdf|.

%%%%%%%%%%%%%%%%%%%%%%%%%%%%%%%%%%%%%%%%%%%%%%%%%%%%%%%%%%%%%%%%%%%%%%%%%%%%%%%%
\subsection{Files and Installation}

The package consists of the files:
%
\begin{center}
\begin{tabular}{ll}
    |README.txt|   & readme file \\
    |childdoc.ins| & installation file \\
    |childdoc.dtx| & source file \\
    |childdoc.def| & definition file \\
    |cdocsamp.tex| & sample main file \\
    |cdocsch1.tex| & sample include file \\
    |cdocsch2.tex| & sample include file \\
    |cdocspt3.tex| & sample part file \\
    |cdocspt4.tex| & sample part file \\
    |cdocsdrf.tex| & sample redirection file \\
    |cdocsfn1.tex| & sample redirection file \\
    |cdocsfn2.tex| & sample redirection file \\
    |childdoc.pdf| & manual
\end{tabular}
\end{center}
%
The distribution consists of the files
|README.txt|, |childdoc.ins| and |childdoc.dtx|.
%
\begin{itemize}
\item
Run (pdf)\LaTeX{} on |childdoc.dtx|
to compile the manual |childdoc.pdf| (this file).
\item
Run \LaTeX{} on |childdoc.ins| to create the definitions file |childdoc.def|
and the sample |cdocsamp.tex| with include files
|cdocsch1.tex|, |cdocsch2.tex|, |cdocspt3.tex|, |cdocspt4.tex|,
|cdocsdrf.tex|, |cdocsfn1.tex|, |cdocsfn2.tex|.
Then copy the file |childdoc.def| to an appropriate directory of your \LaTeX{}
distribution, e.g.\ \textit{texmf-root}|/tex/latex/childdoc|.
\end{itemize}

%%%%%%%%%%%%%%%%%%%%%%%%%%%%%%%%%%%%%%%%%%%%%%%%%%%%%%%%%%%%%%%%%%%%%%%%%%%%%%%%
\subsection{Related CTAN Packages}

There are several other packages which offer a similar functionality:
%
\begin{itemize}
\item
The packages
\href{http://ctan.org/pkg/docmute}{\textsf{docmute}},
\href{http://ctan.org/pkg/includex}{\textsf{includex}} and
\href{http://ctan.org/pkg/standalone}{\textsf{standalone}}
provide commands to include only the document body of
a child file thus allowing both files to be compiled individually.
\item
The packages \href{http://ctan.org/pkg/subdocs}{\textsf{subdocs}}
and \href{http://ctan.org/pkg/subfiles}{\textsf{subfiles}}
provide structures in which the main and child documents can be
encapsulated and allowing them to be compiled individually.
The inclusion mechanism is different from the conventional |\include|.
\item
The package \href{http://ctan.org/pkg/combine}{\textsf{combine}}
is an elaborate solution to combine several documents into one.
\end{itemize}
%
See also the CTAN topic \href{http://ctan.org/topic/subdocs}{\textsf{subdocs}}
for further related packages.
The present package differs from the above solutions in that
a document structure constructed with the conventional |\include| mechanism
just needs two extra commands at the top of every file
such that all constituent files can be compiled individually.

%%%%%%%%%%%%%%%%%%%%%%%%%%%%%%%%%%%%%%%%%%%%%%%%%%%%%%%%%%%%%%%%%%%%%%%%%%%%%%%%
%\subsection{Feature Suggestions}
%
%The following is a list of features which may be useful for future
%versions of this package:
%%
%\begin{itemize}
%\item
%\ldots
%\end{itemize}

%%%%%%%%%%%%%%%%%%%%%%%%%%%%%%%%%%%%%%%%%%%%%%%%%%%%%%%%%%%%%%%%%%%%%%%%%%%%%%%%
\subsection{Revision History}

%%%%%%%%%%%%%%%%%%%%%%%%%%%%%%%%%%%%%%%%
\paragraph{v2.0:} 2018/12/30

\begin{itemize}
\item
immediate forward processing
\item
added |\childdocby| mechanism
\item
manual restructured
\end{itemize}

%%%%%%%%%%%%%%%%%%%%%%%%%%%%%%%%%%%%%%%%
\paragraph{v1.6:} 2018/01/17

\begin{itemize}
\item
application for development of include files
\item
corrections to manual
\end{itemize}

%%%%%%%%%%%%%%%%%%%%%%%%%%%%%%%%%%%%%%%%
\paragraph{v1.5:} 2017/05/21

\begin{itemize}
\item
more complete structuring introduced
\item
|\childdocof| introduced
\item
|\childdoc| renamed to |\childdocmain|
\item
|\childredirect| renamed to |\childdocforward| and |\childdocforwardprefix|
and functionality expanded
\end{itemize}

%%%%%%%%%%%%%%%%%%%%%%%%%%%%%%%%%%%%%%%%
\paragraph{v1.0:} 2017/04/27

\begin{itemize}
\item
manual and install package
\item
first version published on CTAN
\end{itemize}

%%%%%%%%%%%%%%%%%%%%%%%%%%%%%%%%%%%%%%%%
\paragraph{v0.6:} 2017/04/26

\begin{itemize}
\item
redirection mechanism added
\end{itemize}

%%%%%%%%%%%%%%%%%%%%%%%%%%%%%%%%%%%%%%%%
\paragraph{v0.5:} 2017/04/26

\begin{itemize}
\item
functionality in definition file
\end{itemize}


%%%%%%%%%%%%%%%%%%%%%%%%%%%%%%%%%%%%%%%%%%%%%%%%%%%%%%%%%%%%%%%%%%%%%%%%%%%%%%%%
%%%%%%%%%%%%%%%%%%%%%%%%%%%%%%%%%%%%%%%%%%%%%%%%%%%%%%%%%%%%%%%%%%%%%%%%%%%%%%%%
%%%%%%%%%%%%%%%%%%%%%%%%%%%%%%%%%%%%%%%%%%%%%%%%%%%%%%%%%%%%%%%%%%%%%%%%%%%%%%%%
\appendix

\settowidth\MacroIndent{\rmfamily\scriptsize 000\ }

 \DocInput{childdoc.dtx}

\end{document}
%</driver>
% \fi
%
% %%%%%%%%%%%%%%%%%%%%%%%%%%%%%%%%%%%%%%%%%%%%%%%%%%%%%%%%%%%%%%%%%%%%%%%%%%%%%%
% %%%%%%%%%%%%%%%%%%%%%%%%%%%%%%%%%%%%%%%%%%%%%%%%%%%%%%%%%%%%%%%%%%%%%%%%%%%%%%
% \section{Sample}
%\iffalse
%<*samplemain>
%\fi
%
% The following presents a sample document
% with two chapters, two parts, a title page,
% a compile flag as well as three forwarding files to set the flag.
% It consists of eight |.tex| files:
% \begin{center}
% \begin{tabular}{ll}
% |cdocsamp.tex|&main file\\
% |cdocsch1.tex|&include file for chapter 1\\
% |cdocsch2.tex|&include file for chapter 2\\
% |cdocspt3.tex|&include file for part 3\\
% |cdocspt4.tex|&include file for part 4\\
% |cdocsdrf.tex|&forwarding file for main file in draft mode\\
% |cdocsfi1.tex|&forwarding file for final version of chapter 1\\
% |cdocsfi2.tex|&forwarding file for final version of chapter 2\\
% \end{tabular}
% \end{center}
% Each of the eight files can be compiled directly by the \LaTeX{} compiler.
%
% %%%%%%%%%%%%%%%%%%%%%%%%%%%%%%%%%%%%%%
% \paragraph{Main File.}
%
% The main file is called |cdocsamp.tex|.
%
% Load the \textsf{childdoc} definitions and
% declare the filename for the main document:
%    \begin{macrocode}
\input{childdoc.def}
\childdocmain{}
%    \end{macrocode}

% Optional override for |\version| flag:
%    \begin{macrocode}
%%\ifchilddoc\else\providecommand{\version}{draft}\fi
%    \end{macrocode}

% Define the default values for the |\version| flag
% (|final| for the main file and |draft| for childs):
%    \begin{macrocode}
\ifchilddoc
\providecommand{\version}{draft}
\else
\providecommand{\version}{final}
\fi
%    \end{macrocode}

% Load the standard document class:
%    \begin{macrocode}
\documentclass[12pt]{article}
%    \end{macrocode}

% Start the document body:
%    \begin{macrocode}
\begin{document}
%    \end{macrocode}

% Declare a title page.
% Print title, part of document being processed and version flag:
%    \begin{macrocode}
\addtocounter{page}{-1}
\begin{center}
{\LARGE\bfseries{}childdoc example\par}
\vspace{1cm}
\ifchilddoc
\ifchilddocmanual part\else chapter\fi:
`\childdocname' of `\childdocjob'\par
\else
main document: `\childdocjob'\par
\fi
version: \version\par
\end{center}
\newpage
%    \end{macrocode}

% Manually include selected file,
% otherwise process as usual:
%    \begin{macrocode}
\ifchilddocmanual
\section*{part `\childdocname'}
\input{\childdocname}
\else
%    \end{macrocode}

% Include the two chapters:
%    \begin{macrocode}
\include{cdocsch1}
\include{cdocsch2}
%    \end{macrocode}

% Include the two parts unless only chapters should be displayed:
%    \begin{macrocode}
\ifchilddoc\else
\section{part three}
\input{cdocspt3}
\section{part four}
\input{cdocspt4}
\fi
%    \end{macrocode}

% Process as usual until here:
%    \begin{macrocode}
\fi
%    \end{macrocode}

% End of document body:
%    \begin{macrocode}
\end{document}
%    \end{macrocode}
%\iffalse
%</samplemain>
%\fi
%
% %%%%%%%%%%%%%%%%%%%%%%%%%%%%%%%%%%%%%%
% \paragraph{Chapter Include Files.}
%
% The include files are called |cdocsch1.tex| and |cdocsch2.tex|.
%
%\iffalse
%<*samplechap1|samplechap2>
%\fi

% Optional override for |\version| flag:
%    \begin{macrocode}
%%\providecommand{\version}{final}
%    \end{macrocode}

% Include the main document:
%    \begin{macrocode}
\input{childdoc.def}
\childdocof{cdocsamp}
%    \end{macrocode}

%\iffalse
%</samplechap1|samplechap2>
%\fi
%
%\iffalse
%<*samplechap1>
%\fi
% Some text for chapter 1:
%    \begin{macrocode}
\section{one}
some text in chapter one
%    \end{macrocode}

%\iffalse
%</samplechap1>
%\fi
% Some text for chapter 2:
%\iffalse
%<*samplechap2>
%\fi
%    \begin{macrocode}
\section{two}
more text in chapter two
%    \end{macrocode}

%\iffalse
%</samplechap2>
%\fi
%
% %%%%%%%%%%%%%%%%%%%%%%%%%%%%%%%%%%%%%%
% \paragraph{Part Include Files.}
%
% The include files are called |cdocspt3.tex| and |cdocspt4.tex|.
%
%\iffalse
%<*samplepart3|samplepart4>
%\fi

% Optional override for |\version| flag:
%    \begin{macrocode}
%%\providecommand{\version}{final}
%    \end{macrocode}

% Include the main document:
%    \begin{macrocode}
\input{childdoc.def}
\childdocby{cdocsamp}
%    \end{macrocode}

%\iffalse
%</samplepart3|samplepart4>
%\fi
%
%\iffalse
%<*samplepart3>
%\fi
% Some text for part 3:
%    \begin{macrocode}
some text in part three
%    \end{macrocode}

%\iffalse
%</samplepart3>
%\fi
% Some text for part 4:
%\iffalse
%<*samplepart4>
%\fi
%    \begin{macrocode}
more text in part four
%    \end{macrocode}

%\iffalse
%</samplepart4>
%\fi
%
% %%%%%%%%%%%%%%%%%%%%%%%%%%%%%%%%%%%%%%
% \paragraph{Forwarding for a Complete Draft.}
%
% The following forwarding file |cdocsdrf.tex|
% compiles the main document in draft mode:
%\iffalse
%<*sampledraft>
%\fi
%    \begin{macrocode}
\def\version{draft}
\input{childdoc.def}
\childdocforward{cdocsamp}
%    \end{macrocode}

%\iffalse
%</sampledraft>
%\fi
%
% %%%%%%%%%%%%%%%%%%%%%%%%%%%%%%%%%%%%%%
% \paragraph{Forwarding for Final Version of the Chapters.}
%
% The following forwarding files |cdocsfn1.tex| and |cdocsfn2.tex|
% (with identical content)
% compile the final versions of the child documents
% |cdocsch1.tex| and |cdocsch2.tex|, respectively:
%\iffalse
%<*samplefinal>
%\fi
%    \begin{macrocode}
\def\version{final}
\input{childdoc.def}
\childdocforwardprefix[cdocsamp]{cdocsfn}{cdocsch}
%    \end{macrocode}

%\iffalse
%</samplefinal>
%\fi
%
% %%%%%%%%%%%%%%%%%%%%%%%%%%%%%%%%%%%%%%
% \paragraph{Command Line Processing.}
%
% The following three command lines generate the output files
% |cdocscld|, |cdocscl1| and |cdocscl2|
% which should be identical to
% |cdocsdrf|, |cdocsch1| and |cdocsfn2|, respectively:
% \begin{center}
% \begin{tabular}{l}
% |latex -jobname cdocscld \|\\
% |  "\def\version{draft}\input{childdoc.def}\childdocforward{cdocsamp}"|\\
% |latex -jobname cdocscl1 \|\\
% |  "\input{childdoc.def}\childdocforward[cdocsamp]{cdocsch1}"|\\
% |latex -jobname cdocscl2 \|\\
% |  "\def\version{final}\input{childdoc.def}\childdocforward{cdocsch2}"|
% \end{tabular}
% \end{center}
% Note that the trailing backslash on each first line
% merely continues the input to the second line
% (for convenient cut ant paste).
% Furthermore, the command |latex| can be replaced by any
% of its alternative versions such as |pdflatex|.
%
% %%%%%%%%%%%%%%%%%%%%%%%%%%%%%%%%%%%%%%%%%%%%%%%%%%%%%%%%%%%%%%%%%%%%%%%%%%%%%%
% %%%%%%%%%%%%%%%%%%%%%%%%%%%%%%%%%%%%%%%%%%%%%%%%%%%%%%%%%%%%%%%%%%%%%%%%%%%%%%
% \section{Implementation}
%\iffalse
%<*package>
%\fi
%
% This section describes the definitions file |childdoc.def|.

% The definitions cannot be loaded using |\usepackage| or |\RequirePackage|
% which has a mechanism to prevent loading a style file more than once.
% When loading the definitions by means of |\input|
% multiple instances have to be prevented manually:
%\iffalse
%This code needs to be before the `\ProvidesFile' directive
%which is defined at the beginning of this file.
%Therefore it is also placed there and commented out here.
%</package>
%<*discard>
%\fi
%    \begin{macrocode}
\ifdefined\childdocmain\endinput\fi
%    \end{macrocode}
%\iffalse
%</discard>
%<*package>
%\fi
%
% \macro{\ifchilddoc}
% \macro{\ifchilddocmanual}
% The conditional |\ifchilddoc| tells whether a
% child (true) or main (false) document is being compiled.
% The conditional |\ifchilddocmanual| tells whether
% the |\includeonly| mechanism is used (false) or
% the selection of child files must be performed manually (true).
% The definitions initialise to false:
%    \begin{macrocode}
\newif\ifchilddoc
\newif\ifchilddocmanual
%    \end{macrocode}

% \macro{\childdocname}
% \macro{\childdocjob}
% The macro |\childdocname| stores the name of the main document
% to be compiled. The macro |\childdocjob| stores the name of
% the document on which the \LaTeX{} compiler was originally invoked.
% The content of |\jobname| cannot be compared
% to filenames specified in the source due to different catcodes.
% The following code rescans |\jobname|, stores the result
% in |\childdocname| and saves a copy in |\childdocjob|:
%    \begin{macrocode}
\edef\childdocname{\scantokens\expandafter{\jobname\noexpand}}
\let\childdocjob\childdocname
%    \end{macrocode}

% \macro{\childdocdisable}
% The macro |\childdocdisable| prevents the main file
% from being processed more than once.
% At this stage, the main document command |\childdocmain|
% is assumed to be called once again where it should do nothing.
% Any subsequent call to it should prevent
% a secondary processing of the main document
% It overwrites the forwarding commands
% |\childdocof| and |\childdocforward|
% with empty macros to prevent further inclusions of the main document:
%    \begin{macrocode}
\newcommand{\childdocdisable}
{
  \renewcommand{\childdocmain}[1]{\renewcommand{\childdocmain}[1]{\endinput}}
  \renewcommand{\childdocof}[1]{}
  \renewcommand{\childdocby}[2][]{}
  \renewcommand{\childdocforward}[2][]{}
  \renewcommand{\childdocdisable}{}
}
%    \end{macrocode}

% \macro{\childdocmain}
% The macro |\childdocmain| is to be called at the top of the main file
% with nothing or the main filename (without extension) as argument.
% First, it breaks loops.
% If the argument is not empty and does not match |\childdocname|
% (which is set by the first inclusion of |childdoc.def|),
% |\ifchilddoc| is set to true, |\includeonly| is applied to the child file
% and |\jobname| is set to the main file
% (for proper handling of |.aux| files):
%    \begin{macrocode}
\newcommand{\childdocmain}[1]
{
  \childdocdisable\childdocmain{}
  \if?#1?\else
    \begingroup
      \def\childdoctmp{#1}
      \ifx\childdoctmp\childdocname
        \def\childdoctmp{}
      \else
        \def\childdoctmp
        {
          \childdoctrue
          \includeonly{\childdocname}
          \def\childdocjob{#1}
          \def\jobname{#1}
        }
      \fi
      \expandafter
    \endgroup
    \childdoctmp
  \fi
}
%    \end{macrocode}

% \macro{\childdocof}
% The command |\childdocof| redirects
% compilation to the main file |#1|.
%    \begin{macrocode}
\newcommand{\childdocof}[1]
{
  \childdocdisable
  \childdoctrue
  \includeonly{\childdocname}
  \def\jobname{#1}
  \def\childdocjob{#1}
  \input{#1}
}
%    \end{macrocode}

% \macro{\childdocby}
% The command |\childdocby| ....
%    \begin{macrocode}
\newcommand{\childdocby}[2][]
{
  \childdocdisable
  \childdoctrue
  \childdocmanualtrue
  \if?#1?\else
    \def\jobname{#2}
  \fi
  \def\childdocjob{#2}
  \input{#2}
  \endinput
}
%    \end{macrocode}

% \macro{\childdocforward}
% The command |\childdocforward| redirects
% compilation to the main file or
% (if the optional argument is given) a child file.
% Parameters are set as if the main file
% or a child file starting with |\childdocof| was compiled.
% Then compilation is handed over to the main file:
%    \begin{macrocode}
\newcommand{\childdocforward}[2][]
{
  \begingroup
    \if?#1?
      \def\childdoctmp
      {
        \def\childdocname{#2}
        \def\childdocjob{#2}
        \def\jobname{#2}
        \input{#2}
        \endinput
      }
    \else
      \def\childdoctmp
      {
        \childdocdisable
        \def\childdocname{#2}
        \childdoctrue
        \includeonly{#2}
        \def\childdocjob{#1}
        \def\jobname{#1}
        \input{#1}
        \endinput
      }
    \fi
    \expandafter
  \endgroup
  \childdoctmp
}
%    \end{macrocode}

% \macro{\childdocforwardprefix}
% The command |\childdocforwardprefix| redirects
% compilation to the main or a child file by means of a pattern.
% The prefix |#1| in the current filename is replaced by |#2|
% and the suffix of the current filename is kept
% (it is assumed that the filename does not contain the substring `|~~~|'
% which is used as a delimiter).
% Compilation is handed over to the new file by |\childdocforward|:
%    \begin{macrocode}
\newcommand{\childdocforwardprefix}[3][]
{
  \begingroup
    \def\childdocextract #2##1~~~{\def\childdoctmp{\childdocforward[#1]{#3##1}}}
    \expandafter\childdocextract\childdocname~~~
    \expandafter
  \endgroup
  \childdoctmp
}
%    \end{macrocode}

% \macro{\childdoc}
% The deprecated macro |\childdoc| is a legacy version of |\childdocmain|:
%    \begin{macrocode}
\newcommand{\childdoc}{\childdocmain}
%    \end{macrocode}

% \macro{\childdocredirect}
% The deprecated macro |\childdocredirect| is a legacy version
% of |\childdocforward| and |\childdocforwardprefix|:
%    \begin{macrocode}
\newcommand{\childdocredirect}[2][]
{
  \begingroup
    \if?#1?
      \def\childdoctmp{\childdocforward{#2}}
    \else
      \def\childdoctmp{\childdocforwardprefix{#1}{#2}}
    \fi
    \expandafter
  \endgroup
  \childdoctmp
}
%    \end{macrocode}

%\iffalse
%</package>
%\fi
%
\endinput
|\\
|\childdocforward{|\textit{main}|}|\\
\end{tabular}
\end{center}
%
or alternatively with:
%
\begin{center}
\begin{tabular}{l}
|% \iffalse
%
% childdoc.dtx Copyright (C) 2017-2018 Niklas Beisert
%
% This work may be distributed and/or modified under the
% conditions of the LaTeX Project Public License, either version 1.3
% of this license or (at your option) any later version.
% The latest version of this license is in
%   http://www.latex-project.org/lppl.txt
% and version 1.3 or later is part of all distributions of LaTeX
% version 2005/12/01 or later.
%
% This work has the LPPL maintenance status `maintained'.
%
% The Current Maintainer of this work is Niklas Beisert.
%
% This work consists of the files childdoc.dtx and childdoc.ins
% and the derived files childdoc.def and cdocsamp.tex with
% cdocsch1.tex, cdocsch2.tex, cdocsdrf.tex, cdocsfn1.tex, cdocsfn2.tex.
%
%<package>\ifdefined\childdocmain\endinput\fi
%<package>\ProvidesFile{childdoc.def}[2018/12/30 v2.0 child document driver]
%<samplemain>\ProvidesFile{cdocsamp.tex}[2018/12/30 v2.0 sample for childdoc]
%<*driver>
%\ProvidesFile{childdoc.drv}[2018/12/30 v2.0 childdoc reference manual file]
\PassOptionsToClass{10pt,a4paper}{article}
\documentclass{ltxdoc}

\usepackage[margin=35mm]{geometry}
\usepackage{hyperref}
\usepackage{hyperxmp}
\usepackage[usenames]{color}

\hypersetup{colorlinks=true}
\hypersetup{pdfstartview=FitH}
\hypersetup{pdfpagemode=UseNone}
\hypersetup{pdfsource={}}
\hypersetup{pdflang={en-UK}}
\hypersetup{pdfcopyright={Copyright 2017-2018 Niklas Beisert.
  This work may be distributed and/or modified under the
  conditions of the LaTeX Project Public License, either version 1.3
  of this license or (at your option) any later version.}}
\hypersetup{pdflicenseurl={http://www.latex-project.org/lppl.txt}}
\hypersetup{pdfcontactaddress={ETH Zurich, ITP, HIT K,
  Wolfgang-Pauli-Strasse 27}}
\hypersetup{pdfcontactpostcode={8093}}
\hypersetup{pdfcontactcity={Zurich}}
\hypersetup{pdfcontactcountry={Switzerland}}
\hypersetup{pdfcontactemail={nbeisert@itp.phys.ethz.ch}}
\hypersetup{pdfcontacturl={http://people.phys.ethz.ch/\xmptilde nbeisert/}}

\newcommand{\secref}[1]{\hyperref[#1]{section \ref*{#1}}}

\parskip1ex
\parindent0pt
\let\olditemize\itemize
\def\itemize{\olditemize\parskip0pt}

\begin{document}

\title{The \textsf{childdoc} Package}
\hypersetup{pdftitle={The childdoc Package}}
\author{Niklas Beisert\\[2ex]
  Institut f\"ur Theoretische Physik\\
  Eidgen\"ossische Technische Hochschule Z\"urich\\
  Wolfgang-Pauli-Strasse 27, 8093 Z\"urich, Switzerland\\[1ex]
  \href{mailto:nbeisert@itp.phys.ethz.ch}
  {\texttt{nbeisert@itp.phys.ethz.ch}}}
\hypersetup{pdfauthor={Niklas Beisert}}
\hypersetup{pdfsubject={Manual for the LaTeX2e Package childdoc}}
\date{30 December 2018, \textsf{v2.0}}
\maketitle

\begin{abstract}\noindent
\textsf{childdoc} is a \LaTeXe{} package
that enables the direct compilation
of document sections included by |\include|
to individual files.
\end{abstract}

\begingroup
\parskip0ex
\tableofcontents
\endgroup

%%%%%%%%%%%%%%%%%%%%%%%%%%%%%%%%%%%%%%%%%%%%%%%%%%%%%%%%%%%%%%%%%%%%%%%%%%%%%%%%
%%%%%%%%%%%%%%%%%%%%%%%%%%%%%%%%%%%%%%%%%%%%%%%%%%%%%%%%%%%%%%%%%%%%%%%%%%%%%%%%
\section{Introduction}

\LaTeX{} provides a mechanism to structure a large document (such as a book)
into a main file and several child files (containing the chapters)
using the |\include| command.
This mechanism is beneficial for documents
which span hundreds of pages in order to
make the source file(s) more manageable.
Moreover, compilation can be restricted to
selected child files by means of the |\includeonly| command.
The latter feature can be used to reduce the compilation time while editing
(this was significantly more useful in the earlier days of \LaTeX{})
or to generate a smaller document which is easier to navigate.
Another application of |\includeonly| is to generate
documents consisting of selected parts of the complete document.

However, there are a few drawbacks of the plain |\include| mechanism:
\begin{itemize}
\item
The child files cannot be compiled on their own,
they can only be compiled via the main file.
A naive editing environment
(such as a text editor with an option
to have the current file processed by \LaTeX)
may require one to switch to the main file before compiling;
attempting to compile the child file produces errors.
\item
The main file must be modified (each time)
to adjust the |\includeonly| command
to the present needs. This easily leaves the main file in a messy state.
\item
The generated document will always carry the filename
of the main document. This is inconvenient if
several child files are to be compiled and
to be kept for distribution.
\end{itemize}

The present package provides a simple interface
to make child files individually compilable by \LaTeX{}.
Compiling a child file then has the same effect as compiling
the main file with an |\includeonly| command
to select the appropriate child.
Moreover the generated document will carry the name of the child
rather than the main file.
This resolves all three above issues.

This feature is meant to make the editing of books,
thesis documents and lecture notes somewhat more convenient.
However, the package can also be used efficiently for
composing a series of documents (such as exercise sheets)
which are typically distributed individually.
It then assists the author in generating the individual documents
(potentially in different versions)
as well as a document containing the collected series.
Another application is in developing style files
or other kinds of included material
where compilation of the style file could redirect
to a sample or test file.

%%%%%%%%%%%%%%%%%%%%%%%%%%%%%%%%%%%%%%%%%%%%%%%%%%%%%%%%%%%%%%%%%%%%%%%%%%%%%%%%
%%%%%%%%%%%%%%%%%%%%%%%%%%%%%%%%%%%%%%%%%%%%%%%%%%%%%%%%%%%%%%%%%%%%%%%%%%%%%%%%
\section{Usage}

First of all, the package \textsf{childdoc} is \emph{not} a standard
\LaTeXe{} |.sty| style file! Therefore it needs to be invoked in
a non-standard way.

%%%%%%%%%%%%%%%%%%%%%%%%%%%%%%%%%%%%%%%%%%%%%%%%%%%%%%%%%%%%%%%%%%%%%%%%%%%%%%%%
\subsection{Included Files}
\label{sec:include}

%%%%%%%%%%%%%%%%%%%%%%%%%%%%%%%%%%%%%%%%
\DescribeMacro{\childdocmain}
To use the package, add the commands
\begin{center}
\begin{tabular}{l}
|\input{childdoc.def}|\\
|\childdocmain{}|\\
\end{tabular}
\end{center}
at the very top of the main \LaTeX{} file,
in particular \emph{before} the |\documentclass| statement!
The argument of |\childdocmain| should be left empty
(but it must be present).

%%%%%%%%%%%%%%%%%%%%%%%%%%%%%%%%%%%%%%%%
\DescribeMacro{\childdocof}
Furthermore, add the commands
\begin{center}
\begin{tabular}{l}
|\input{childdoc.def}|\\
|\childdocof{|\textit{main}|}|\\
\end{tabular}
\end{center}
at the top of every child file \textit{child}
which is included by |\include{|\textit{child}|}|
from within the main file
(or at least for those files to be compiled individually).
The argument \textit{main} must be the filename of the main file.

There are a couple of
considerations in setting up the main and child documents:

%%%%%%%%%%%%%%%%%%%%%%%%%%%%%%%%%%%%%%%%
\paragraph{Restrictions.}

Please note the following restrictions:
\begin{itemize}
\item
|\childdocmain| must be called with one argument \textit{main}
to ensure compatibility with earlier version of the package.
It must either be empty (|\childdocmain{}|)
or precisely match the filename of the main file in which it is specified.
See \secref{sec:detection} for further information.
\item
The filename \textit{main} must be specified without the |.tex| extension.
\item
The filename \textit{main} is case sensitive
(even in case-insensitive file systems)
due to internal string comparison.
\item
The argument \textit{main} should be fully expanded, it cannot be a macro.
\item
Subdirectories and special characters should be avoided in filenames.
\item
The command |\childdocmain{|\textit{main}|}| must be followed by a whitespace.
It should not be followed immediately by another command
or by a comment mark `|%|'.
This is because the \TeX{} parser reads the token immediately following
the argument of |\childdocmain| and puts it
at the beginning of every child section;
however, a white\-space is ignored.
\end{itemize}

%%%%%%%%%%%%%%%%%%%%%%%%%%%%%%%%%%%%%%%%
\paragraph{Content of Main File.}

It is advisable to place all content in the child files included by |\include|.
Any output contained in the main file will appear in all child documents
unless suppressed manually;
it cannot be suppressed automatically by the |\includeonly| directive
and thus should normally be avoided.
A method to include some content in the main file
by means of conditional processing is described in \secref{sec:conditional}.

%%%%%%%%%%%%%%%%%%%%%%%%%%%%%%%%%%%%%%%%
\paragraph{Page Numbering.}

When only a part of the document is compiled,
the appropriate numbering of pages
(as well as other status parameters)
is determined from the |.aux| files.
The latter contain information from previous passes.
However this information needs to propagate through
all intermediate child documents.
Therefore the page numbering in child documents may well
be inconsistent until the complete document is compiled at least once.

A useful (if unconventional) way to always ensure a consistent
page numbering is to restart the numbering in each child document
and denote the pages by `\textit{child}|.|\textit{page}'
where \textit{child} represents the chapter/section number of the child file.
This can be achieved by the command
|\numberwithin{page}{|\textit{child}|}|
of the \textsf{amsmath} package
where \textit{child} can be |chapter| or |section|
depending on the chosen structuring.
Alternatively, one can modify the macro |\thepage| appropriately
and reset the counter |page| at the start of each child file.

%%%%%%%%%%%%%%%%%%%%%%%%%%%%%%%%%%%%%%%%%%%%%%%%%%%%%%%%%%%%%%%%%%%%%%%%%%%%%%%%
\subsection{Conditional Processing}
\label{sec:conditional}

The package provides a mechanism to compile different versions
of a document. To customise the versions further some conditional processing
can come in handy to distinguish which version is being compiled.
The package provides two macros to describe the compilation context:

%%%%%%%%%%%%%%%%%%%%%%%%%%%%%%%%%%%%%%%%
\DescribeMacro{\ifchilddoc}
The conditional |\ifchilddoc| distinguishes between the compilation of
child documents and the main document:
%
\begin{center}
|\ifchilddoc |\textit{child-code}| |[|\||else |\textit{main-code}]| \||fi|
\end{center}

%%%%%%%%%%%%%%%%%%%%%%%%%%%%%%%%%%%%%%%%
\DescribeMacro{\childdocname}
\DescribeMacro{\childdocjob}
The macro |\childdocname| contains the filename (without extension)
of the main or child file being processed.
Note that |\childdocjob| will always contain the name of the main file.

%%%%%%%%%%%%%%%%%%%%%%%%%%%%%%%%%%%%%%%%
\paragraph{Title Page.}

Conditional processing can be used to include a title or banner page
in the main document when proper precautions are taken.
Importantly, the code in the main file should ensure that the page counter
(as well as other status parameters which are stored in the |.aux| files)
takes the same value after the conditional processing.
Otherwise the page numbers may take divergent values
depending on which part is compiled.

For example, a title page could be declared by:
%
\begin{center}
\begin{tabular}{l}
|\ifchilddoc\||else|\\
|\addtocounter{page}{-1}|\\
\textit{code for title page}\\
|\newpage|\\
|\||fi|
\end{tabular}
\end{center}
%
A banner page for the child documents can be generated by:
%
\begin{center}
\begin{tabular}{l}
|\ifchilddoc|\\
|\addtocounter{page}{-1}|\\
\textit{code for banner page}\\
|\newpage|\\
|\||fi|
\end{tabular}
\end{center}
%
Here one could write a message such as:
\begin{center}
|This is the part \childdocname{} of \childdocjob{}.|
\end{center}

%%%%%%%%%%%%%%%%%%%%%%%%%%%%%%%%%%%%%%%%%%%%%%%%%%%%%%%%%%%%%%%%%%%%%%%%%%%%%%%%
\subsection{Flags}
\label{sec:flags}

The package makes it easy to generate different versions
of the main or child documents.
To this end compilation flags can be defined
and assigned different default values.
They will be particularly useful in conjunction
with the forwarding mechanism described in \secref{sec:forward}.

For example, it may be useful to have a flag |\version|
which can be set to |draft| or |final|.
The document source will contain some conditional code
depending on the value of |\version|.
Suppose further, the flag should default to |final| for the main file
and to |draft| for child files
which is a natural assignment for editing the document.
This is achieved by placing the following code
in the preamble of the main document
(below the |\childdocmain| directive):
%
\begin{center}
\begin{tabular}{l}
|\ifchilddoc|\\
|\providecommand{\version}{draft}|\\
|\||else|\\
|\providecommand{\version}{final}|\\
|\||fi|
\end{tabular}
\end{center}
%
The definition by |\providecommand| makes sure
that previous definitions are not overwritten.
Further statements |\providecommand{\version}{...}|
can thus be added before the above code to override it.

For the main file, one might add a line
(between |\childdocmain| and the above block)
%
\begin{center}
|%\ifchilddoc\||else\providecommand{\version}{draft}\||fi|
\end{center}
%
which can be uncommented to produce a draft version.
Likewise one can add a line to the very top of a child file
(above the |\childdocof{|\textit{main}|}| directive)
%
\begin{center}
|%\providecommand{\version}{final}|
\end{center}
%
which can be uncommented to produce the final version of this child document.

%%%%%%%%%%%%%%%%%%%%%%%%%%%%%%%%%%%%%%%%%%%%%%%%%%%%%%%%%%%%%%%%%%%%%%%%%%%%%%%%
\subsection{Forwarding}
\label{sec:forward}

Different versions of the main or child documents
using compilation flags as described in \secref{sec:flags}
can be (permanently) stored in different files
for convenient compilation, viewing and distribution.
To this end, the package defines a command
to pass on compilation to a different file:

%%%%%%%%%%%%%%%%%%%%%%%%%%%%%%%%%%%%%%%%
\DescribeMacro{\childdocforward}
The command |\childdocforward| redirects processing to
another source file:
%
\begin{center}
\begin{tabular}{l}
|\input{childdoc.def}|\\
|\childdocforward[|\textit{main}|]{|\textit{dest}|}|\\
\end{tabular}
\end{center}
%
The argument \textit{dest} is the destination file
(without extension).
It should be the main file or one of the child files.
Note that further \textsf{childdoc} directives
such as |\childdocof| and |\childdocforward|
in the indicated file will be processed in this form.
The optional argument \textit{main}
passes on directly to the main file \textit{main}
while pretending to compile the child \textit{dest}.
This form behaves as if \textit{dest}
issues |\childdocof{|\textit{main}|}| right away,
and no further \textsf{childdoc} directives will be processed.

%%%%%%%%%%%%%%%%%%%%%%%%%%%%%%%%%%%%%%%%
\DescribeMacro{\...prefix}
In the alternative form |\childdocforwardprefix|,
%
\begin{center}
\begin{tabular}{l}
|\input{childdoc.def}|\\
|\childdocforwardprefix[|\textit{main}|]{|\textit{prefix}|}{|\textit{dest}|}|
\end{tabular}
\end{center}
%
the destination file is determined by a pattern
depending on the current file:
To make this work, the current file must be called
`{\textit{prefix}\hspace{0.2em}\textit{suffix}}'
with \textit{prefix} matching precisely the argument.
Processing is then passed on to the file
`{\textit{dest}\hspace{0.2em}\textit{suffix}}'.
Surely, the same effect is achieved by
directly specifying the
argument `{\textit{dest}\hspace{0.2em}\textit{suffix}}'
in the first form.
However, that requires to set up a different file
for each child. With the alternative form of the command
all these files can have exactly the same content
which simplifies setting them up and maintaining them.

For example, the following file |draft.tex|
with a compilation flag |\version| as described in \secref{sec:flags}
compiles the main document as a draft:
%
\begin{center}
\begin{tabular}{l}
|\def\version{draft}|\\
|\input{childdoc.def}|\\
|\childdocforward{|\textit{main}|}|
\end{tabular}
\end{center}
%
Likewise, the following files |final|\textit{nn}|.tex|
compile the final version of the child document
|child|\textit{nn}|.tex|:
%
\begin{center}
\begin{tabular}{l}
|\def\version{final}|\\
|\input{childdoc.def}|\\
|\childdocforwardprefix{final}{child}|
\end{tabular}
\end{center}
%

Note that when several versions of a main file and/or of each child file
are to be generated, it may be convenient to set up a |Makefile| or
shell script to automatise the process.

%%%%%%%%%%%%%%%%%%%%%%%%%%%%%%%%%%%%%%%%%%%%%%%%%%%%%%%%%%%%%%%%%%%%%%%%%%%%%%%%
\subsection{Command Line Processing}
\label{sec:commandline}

The effect of redirection files can also be achieved by invoking
the \LaTeX{} compiler with a more elaborate command line.
Most conveniently this should be done as part
of a shell script or a |Makefile|.

When using \textsf{childdoc} in the main file, the following
command lines effectively perform a redirection
(note that depending on the shell being used,
backslashes may have to be doubled: `|\|' $\to$ `|\\|'):
%
\begin{center}
|... -jobname "|\textit{target}|" |\\|"|[\textit{flags}]%
|\input{childdoc.def}\childdocforward[|\textit{main}|]{|\textit{dest}|}"|
\end{center}
%
Here \textit{target} is the name of the output file,
\textit{main} is the name of the main file
and \textit{dest} is the name of the main or child file to be processed
(all filenames without extensions).
The optional argument \textit{main} can be omitted
if \textit{main} matches \textit{dest}.
Optionally, compilation \textit{flags} can be defined via |\def| commands.
This command line makes the \TeX{} engine believe
it is compiling the file \textit{target}
whose content is specified as the latter parameter.
The provided code then forwards the processing to
\textit{main} or \textit{dest} as described in \secref{sec:forward}.

%%%%%%%%%%%%%%%%%%%%%%%%%%%%%%%%%%%%%%%%%%%%%%%%%%%%%%%%%%%%%%%%%%%%%%%%%%%%%%%%
\subsection{Include by Input}
\label{sec:input}

Including child documents by |\include| has some restrictions by design.
Most notably, the content of a child document always occupies
its own set of pages; pages cannot be shared between child documents.
Usually, this behaviour makes perfect sense
because each child document contain an essential part of the document.
However, in some situations it may be desirable to compose
a document from a collection of parts
without having mandatory page breaks between then.
For this case, the package
provides a mechanism to include parts
by |\input| which can also be processed individually.
However, by construction this mechanism
requires manual handling of the content to be output.

%%%%%%%%%%%%%%%%%%%%%%%%%%%%%%%%%%%%%%%%
\DescribeMacro{\ifchilddocmanual}
The main file should be prepared as usual, see \secref{sec:include}.
However, the document body must make a distinction
between processing of an individual part and of the main document, e.g.:
%
\begin{center}
\begin{tabular}{l}
|\ifchilddocmanual|\\
|\input{\childdocname}|\\
|\||else|\\
\textit{document body with }|\input{|\textit{part}|}|\\
|\||fi|
\end{tabular}
\end{center}
%
The conditional |\ifchilddocmanual| is true whenever
a part to be included by |\input| is being compiled,
and the name of the part is stored in |\childdocname|.

%%%%%%%%%%%%%%%%%%%%%%%%%%%%%%%%%%%%%%%%
\DescribeMacro{\childdocby}
Each part to be included by |\input| should start with:
%
\begin{center}
\begin{tabular}{l}
|\input{childdoc.def}|\\
|\childdocby{|\textit{main}|}|\\
\end{tabular}
\end{center}
%
The directive |\childdocby| is similar to |\childdocof|
described in \secref{sec:include},
but the subsequent selection of content must be done manually.
To that end, both |\ifchilddoc| and |\ifchilddocmanual|
will be true upon processing of a part,
and the name of the part is stored in |\childdocname|.
Note that |\jobname| will be set to the filename of the current part
so that each part receives an individual |.aux| file
that does not interfere with the |.aux| file(s) of the main document.
This behaviour can be altered by the alternative form
|\childdocby[*]{|\textit{main}|}| (with a non-empty optional argument)
which uses the |.aux| file of the main document
by setting |\jobname| to \textit{main}.

%%%%%%%%%%%%%%%%%%%%%%%%%%%%%%%%%%%%%%%%%%%%%%%%%%%%%%%%%%%%%%%%%%%%%%%%%%%%%%%%
\subsection{Driver Development}
\label{sec:driver}

The \textsf{childdoc} mechanism can also be use for the development
of definition files such as \LaTeX{} styles or classes.
This case differs from the above setup with multiple parts
included by |\include| in that no |\includeonly| should be invoked.
This can be achieved by starting the include file
(before |\ProvidesPackage|) with:
%
\begin{center}
\begin{tabular}{l}
|\input{childdoc.def}|\\
|\childdocforward{|\textit{main}|}|\\
\end{tabular}
\end{center}
%
or alternatively with:
%
\begin{center}
\begin{tabular}{l}
|\input{childdoc.def}|\\
|\childdocby{|\textit{main}|}|\\
\end{tabular}
\end{center}
%
Both forms have slightly different effects as described above.
The main file is prepared as usual, see \secref{sec:include}.

%%%%%%%%%%%%%%%%%%%%%%%%%%%%%%%%%%%%%%%%%%%%%%%%%%%%%%%%%%%%%%%%%%%%%%%%%%%%%%%%
\subsection{Legacy Detection}
\label{sec:detection}

The directive |\childdocmain| in the main file can detect
whether the complete document or merely a child is to be compiled
even without using the directive |\childdocof|.
This method is deprecated because it is less robust
and there is no compelling reason to use it;
it is merely provided for backward compatibility
and it may be removed in future versions.

If the detection mechanism is to be used,
it is mandatory to correctly specify
the filename of the main file as the argument of |\childdocmain|:
%
\begin{center}
\begin{tabular}{l}
|\input{childdoc.def}|\\
|\childdocmain{|\textit{main}|}|\\
\end{tabular}
\end{center}
%
If |\jobname| does not match the argument \textit{main} of |\childdocmain|,
it is assumed that |\jobname| points to the child file to be compiled.
When using |\childdocmain| with the main file specified as argument,
it suffices to start a child file
with just |\input{|\textit{main}|}|
without loading of the package and using |\childdocof|.
If instead all processing is done
with the appropriate \textsf{childdoc} directives,
the argument of \textit{main} of |\childdocmain| can be empty.

An alternative version of the command line processing described
in \secref{sec:commandline} using the detection mechanism reads:
%
\begin{center}
|... -jobname "|\textit{target}|" "|[\textit{flags}]%
[|\def\jobname{|\textit{dest}|}|]|\input{|\textit{main}|}"|
\end{center}

%%%%%%%%%%%%%%%%%%%%%%%%%%%%%%%%%%%%%%%%%%%%%%%%%%%%%%%%%%%%%%%%%%%%%%%%%%%%%%%%
\subsection{Manual Code}
\label{sec:manual}

In case one cannot be certain whether the definitions file |childdoc.def|
is installed on the target \TeX{} distribution
and one prefers not to ship it,
it is conceivable to paste a few relevant commands into the sources.

To that end, drop all statements |\input{childdoc.def}|
and perform the replacements as outlined below.
Instead of |\childdocmain{|\textit{main}|}| add the following code
to the top of the main file:
%
\begin{center}
\begin{tabular}{l}
|\||ifdefined\childdocname\endinput\||fi\newif\ifchilddoc|\\
|\edef\childdocname{\scantokens\expandafter{\jobname\noexpand}}|\\
|\def\childdocmain{|\textit{main}|}\||ifx\childdocmain\childdocname\||else|\\
|\childdoctrue\includeonly{\childdocname}\let\jobname\childdocmain\||fi|\\
\end{tabular}
\end{center}
%
Instead of |\childdocof{|\textit{main}|}| just include the main file
at the top of each child file:
%
\begin{center}
|\input{|\textit{main}|}|
\end{center}
%
A simple redirection |\childdocforward{|\textit{dest}|}| is achieved by:
%
\begin{center}
|\def\jobname{|\textit{dest}|}\input{\jobname}|
\end{center}
%
The redirection with prefix
|\childdocforwardprefix[|\textit{prefix}|]{|\textit{dest}|}|
is accomplished by:
%
\begin{center}
\begin{tabular}{l}
|{\edef\jobname{\scantokens\expandafter{\jobname\noexpand}}|\\
|\def\redirectjob |\textit{prefix}|#1~~~{\gdef\jobname{|\textit{dest}|#1}}|\\
|\expandafter\redirectjob\jobname~~~}\input{\jobname}|
\end{tabular}
\end{center}

In an alternative approach,
child documents can be compiled by a specific command line
without additional code or specific definitions:
%
\begin{center}
|... -jobname "|\textit{target}|" "|[\textit{flags}]%
|\includeonly{|\textit{dest}|}\input{|\textit{main}|}"|
\end{center}
%

%%%%%%%%%%%%%%%%%%%%%%%%%%%%%%%%%%%%%%%%%%%%%%%%%%%%%%%%%%%%%%%%%%%%%%%%%%%%%%%%
%%%%%%%%%%%%%%%%%%%%%%%%%%%%%%%%%%%%%%%%%%%%%%%%%%%%%%%%%%%%%%%%%%%%%%%%%%%%%%%%
\section{Information}

%%%%%%%%%%%%%%%%%%%%%%%%%%%%%%%%%%%%%%%%%%%%%%%%%%%%%%%%%%%%%%%%%%%%%%%%%%%%%%%%
\subsection{Copyright}

Copyright \copyright{} 2017--2018 Niklas Beisert

This work may be distributed and/or modified under the
conditions of the \LaTeX{} Project Public License, either version 1.3
of this license or (at your option) any later version.
The latest version of this license is in
  \url{http://www.latex-project.org/lppl.txt}
and version 1.3 or later is part of all distributions of \LaTeX{}
version 2005/12/01 or later.

This work has the LPPL maintenance status `maintained'.

The Current Maintainer of this work is Niklas Beisert.

This work consists of the files |README.txt|, |childdoc.ins| and |childdoc.dtx|
as well as the derived files |childdoc.def|, |cdocsamp.tex|
with |cdocsch1.tex|, |cdocsch2.tex|, |cdocspt3.tex|, |cdocspt4.tex|,
|cdocsdrf.tex|, |cdocsfn1.tex|, |cdocsfn2.tex|
as well as |childdoc.pdf|.

%%%%%%%%%%%%%%%%%%%%%%%%%%%%%%%%%%%%%%%%%%%%%%%%%%%%%%%%%%%%%%%%%%%%%%%%%%%%%%%%
\subsection{Files and Installation}

The package consists of the files:
%
\begin{center}
\begin{tabular}{ll}
    |README.txt|   & readme file \\
    |childdoc.ins| & installation file \\
    |childdoc.dtx| & source file \\
    |childdoc.def| & definition file \\
    |cdocsamp.tex| & sample main file \\
    |cdocsch1.tex| & sample include file \\
    |cdocsch2.tex| & sample include file \\
    |cdocspt3.tex| & sample part file \\
    |cdocspt4.tex| & sample part file \\
    |cdocsdrf.tex| & sample redirection file \\
    |cdocsfn1.tex| & sample redirection file \\
    |cdocsfn2.tex| & sample redirection file \\
    |childdoc.pdf| & manual
\end{tabular}
\end{center}
%
The distribution consists of the files
|README.txt|, |childdoc.ins| and |childdoc.dtx|.
%
\begin{itemize}
\item
Run (pdf)\LaTeX{} on |childdoc.dtx|
to compile the manual |childdoc.pdf| (this file).
\item
Run \LaTeX{} on |childdoc.ins| to create the definitions file |childdoc.def|
and the sample |cdocsamp.tex| with include files
|cdocsch1.tex|, |cdocsch2.tex|, |cdocspt3.tex|, |cdocspt4.tex|,
|cdocsdrf.tex|, |cdocsfn1.tex|, |cdocsfn2.tex|.
Then copy the file |childdoc.def| to an appropriate directory of your \LaTeX{}
distribution, e.g.\ \textit{texmf-root}|/tex/latex/childdoc|.
\end{itemize}

%%%%%%%%%%%%%%%%%%%%%%%%%%%%%%%%%%%%%%%%%%%%%%%%%%%%%%%%%%%%%%%%%%%%%%%%%%%%%%%%
\subsection{Related CTAN Packages}

There are several other packages which offer a similar functionality:
%
\begin{itemize}
\item
The packages
\href{http://ctan.org/pkg/docmute}{\textsf{docmute}},
\href{http://ctan.org/pkg/includex}{\textsf{includex}} and
\href{http://ctan.org/pkg/standalone}{\textsf{standalone}}
provide commands to include only the document body of
a child file thus allowing both files to be compiled individually.
\item
The packages \href{http://ctan.org/pkg/subdocs}{\textsf{subdocs}}
and \href{http://ctan.org/pkg/subfiles}{\textsf{subfiles}}
provide structures in which the main and child documents can be
encapsulated and allowing them to be compiled individually.
The inclusion mechanism is different from the conventional |\include|.
\item
The package \href{http://ctan.org/pkg/combine}{\textsf{combine}}
is an elaborate solution to combine several documents into one.
\end{itemize}
%
See also the CTAN topic \href{http://ctan.org/topic/subdocs}{\textsf{subdocs}}
for further related packages.
The present package differs from the above solutions in that
a document structure constructed with the conventional |\include| mechanism
just needs two extra commands at the top of every file
such that all constituent files can be compiled individually.

%%%%%%%%%%%%%%%%%%%%%%%%%%%%%%%%%%%%%%%%%%%%%%%%%%%%%%%%%%%%%%%%%%%%%%%%%%%%%%%%
%\subsection{Feature Suggestions}
%
%The following is a list of features which may be useful for future
%versions of this package:
%%
%\begin{itemize}
%\item
%\ldots
%\end{itemize}

%%%%%%%%%%%%%%%%%%%%%%%%%%%%%%%%%%%%%%%%%%%%%%%%%%%%%%%%%%%%%%%%%%%%%%%%%%%%%%%%
\subsection{Revision History}

%%%%%%%%%%%%%%%%%%%%%%%%%%%%%%%%%%%%%%%%
\paragraph{v2.0:} 2018/12/30

\begin{itemize}
\item
immediate forward processing
\item
added |\childdocby| mechanism
\item
manual restructured
\end{itemize}

%%%%%%%%%%%%%%%%%%%%%%%%%%%%%%%%%%%%%%%%
\paragraph{v1.6:} 2018/01/17

\begin{itemize}
\item
application for development of include files
\item
corrections to manual
\end{itemize}

%%%%%%%%%%%%%%%%%%%%%%%%%%%%%%%%%%%%%%%%
\paragraph{v1.5:} 2017/05/21

\begin{itemize}
\item
more complete structuring introduced
\item
|\childdocof| introduced
\item
|\childdoc| renamed to |\childdocmain|
\item
|\childredirect| renamed to |\childdocforward| and |\childdocforwardprefix|
and functionality expanded
\end{itemize}

%%%%%%%%%%%%%%%%%%%%%%%%%%%%%%%%%%%%%%%%
\paragraph{v1.0:} 2017/04/27

\begin{itemize}
\item
manual and install package
\item
first version published on CTAN
\end{itemize}

%%%%%%%%%%%%%%%%%%%%%%%%%%%%%%%%%%%%%%%%
\paragraph{v0.6:} 2017/04/26

\begin{itemize}
\item
redirection mechanism added
\end{itemize}

%%%%%%%%%%%%%%%%%%%%%%%%%%%%%%%%%%%%%%%%
\paragraph{v0.5:} 2017/04/26

\begin{itemize}
\item
functionality in definition file
\end{itemize}


%%%%%%%%%%%%%%%%%%%%%%%%%%%%%%%%%%%%%%%%%%%%%%%%%%%%%%%%%%%%%%%%%%%%%%%%%%%%%%%%
%%%%%%%%%%%%%%%%%%%%%%%%%%%%%%%%%%%%%%%%%%%%%%%%%%%%%%%%%%%%%%%%%%%%%%%%%%%%%%%%
%%%%%%%%%%%%%%%%%%%%%%%%%%%%%%%%%%%%%%%%%%%%%%%%%%%%%%%%%%%%%%%%%%%%%%%%%%%%%%%%
\appendix

\settowidth\MacroIndent{\rmfamily\scriptsize 000\ }

 \DocInput{childdoc.dtx}

\end{document}
%</driver>
% \fi
%
% %%%%%%%%%%%%%%%%%%%%%%%%%%%%%%%%%%%%%%%%%%%%%%%%%%%%%%%%%%%%%%%%%%%%%%%%%%%%%%
% %%%%%%%%%%%%%%%%%%%%%%%%%%%%%%%%%%%%%%%%%%%%%%%%%%%%%%%%%%%%%%%%%%%%%%%%%%%%%%
% \section{Sample}
%\iffalse
%<*samplemain>
%\fi
%
% The following presents a sample document
% with two chapters, two parts, a title page,
% a compile flag as well as three forwarding files to set the flag.
% It consists of eight |.tex| files:
% \begin{center}
% \begin{tabular}{ll}
% |cdocsamp.tex|&main file\\
% |cdocsch1.tex|&include file for chapter 1\\
% |cdocsch2.tex|&include file for chapter 2\\
% |cdocspt3.tex|&include file for part 3\\
% |cdocspt4.tex|&include file for part 4\\
% |cdocsdrf.tex|&forwarding file for main file in draft mode\\
% |cdocsfi1.tex|&forwarding file for final version of chapter 1\\
% |cdocsfi2.tex|&forwarding file for final version of chapter 2\\
% \end{tabular}
% \end{center}
% Each of the eight files can be compiled directly by the \LaTeX{} compiler.
%
% %%%%%%%%%%%%%%%%%%%%%%%%%%%%%%%%%%%%%%
% \paragraph{Main File.}
%
% The main file is called |cdocsamp.tex|.
%
% Load the \textsf{childdoc} definitions and
% declare the filename for the main document:
%    \begin{macrocode}
\input{childdoc.def}
\childdocmain{}
%    \end{macrocode}

% Optional override for |\version| flag:
%    \begin{macrocode}
%%\ifchilddoc\else\providecommand{\version}{draft}\fi
%    \end{macrocode}

% Define the default values for the |\version| flag
% (|final| for the main file and |draft| for childs):
%    \begin{macrocode}
\ifchilddoc
\providecommand{\version}{draft}
\else
\providecommand{\version}{final}
\fi
%    \end{macrocode}

% Load the standard document class:
%    \begin{macrocode}
\documentclass[12pt]{article}
%    \end{macrocode}

% Start the document body:
%    \begin{macrocode}
\begin{document}
%    \end{macrocode}

% Declare a title page.
% Print title, part of document being processed and version flag:
%    \begin{macrocode}
\addtocounter{page}{-1}
\begin{center}
{\LARGE\bfseries{}childdoc example\par}
\vspace{1cm}
\ifchilddoc
\ifchilddocmanual part\else chapter\fi:
`\childdocname' of `\childdocjob'\par
\else
main document: `\childdocjob'\par
\fi
version: \version\par
\end{center}
\newpage
%    \end{macrocode}

% Manually include selected file,
% otherwise process as usual:
%    \begin{macrocode}
\ifchilddocmanual
\section*{part `\childdocname'}
\input{\childdocname}
\else
%    \end{macrocode}

% Include the two chapters:
%    \begin{macrocode}
\include{cdocsch1}
\include{cdocsch2}
%    \end{macrocode}

% Include the two parts unless only chapters should be displayed:
%    \begin{macrocode}
\ifchilddoc\else
\section{part three}
\input{cdocspt3}
\section{part four}
\input{cdocspt4}
\fi
%    \end{macrocode}

% Process as usual until here:
%    \begin{macrocode}
\fi
%    \end{macrocode}

% End of document body:
%    \begin{macrocode}
\end{document}
%    \end{macrocode}
%\iffalse
%</samplemain>
%\fi
%
% %%%%%%%%%%%%%%%%%%%%%%%%%%%%%%%%%%%%%%
% \paragraph{Chapter Include Files.}
%
% The include files are called |cdocsch1.tex| and |cdocsch2.tex|.
%
%\iffalse
%<*samplechap1|samplechap2>
%\fi

% Optional override for |\version| flag:
%    \begin{macrocode}
%%\providecommand{\version}{final}
%    \end{macrocode}

% Include the main document:
%    \begin{macrocode}
\input{childdoc.def}
\childdocof{cdocsamp}
%    \end{macrocode}

%\iffalse
%</samplechap1|samplechap2>
%\fi
%
%\iffalse
%<*samplechap1>
%\fi
% Some text for chapter 1:
%    \begin{macrocode}
\section{one}
some text in chapter one
%    \end{macrocode}

%\iffalse
%</samplechap1>
%\fi
% Some text for chapter 2:
%\iffalse
%<*samplechap2>
%\fi
%    \begin{macrocode}
\section{two}
more text in chapter two
%    \end{macrocode}

%\iffalse
%</samplechap2>
%\fi
%
% %%%%%%%%%%%%%%%%%%%%%%%%%%%%%%%%%%%%%%
% \paragraph{Part Include Files.}
%
% The include files are called |cdocspt3.tex| and |cdocspt4.tex|.
%
%\iffalse
%<*samplepart3|samplepart4>
%\fi

% Optional override for |\version| flag:
%    \begin{macrocode}
%%\providecommand{\version}{final}
%    \end{macrocode}

% Include the main document:
%    \begin{macrocode}
\input{childdoc.def}
\childdocby{cdocsamp}
%    \end{macrocode}

%\iffalse
%</samplepart3|samplepart4>
%\fi
%
%\iffalse
%<*samplepart3>
%\fi
% Some text for part 3:
%    \begin{macrocode}
some text in part three
%    \end{macrocode}

%\iffalse
%</samplepart3>
%\fi
% Some text for part 4:
%\iffalse
%<*samplepart4>
%\fi
%    \begin{macrocode}
more text in part four
%    \end{macrocode}

%\iffalse
%</samplepart4>
%\fi
%
% %%%%%%%%%%%%%%%%%%%%%%%%%%%%%%%%%%%%%%
% \paragraph{Forwarding for a Complete Draft.}
%
% The following forwarding file |cdocsdrf.tex|
% compiles the main document in draft mode:
%\iffalse
%<*sampledraft>
%\fi
%    \begin{macrocode}
\def\version{draft}
\input{childdoc.def}
\childdocforward{cdocsamp}
%    \end{macrocode}

%\iffalse
%</sampledraft>
%\fi
%
% %%%%%%%%%%%%%%%%%%%%%%%%%%%%%%%%%%%%%%
% \paragraph{Forwarding for Final Version of the Chapters.}
%
% The following forwarding files |cdocsfn1.tex| and |cdocsfn2.tex|
% (with identical content)
% compile the final versions of the child documents
% |cdocsch1.tex| and |cdocsch2.tex|, respectively:
%\iffalse
%<*samplefinal>
%\fi
%    \begin{macrocode}
\def\version{final}
\input{childdoc.def}
\childdocforwardprefix[cdocsamp]{cdocsfn}{cdocsch}
%    \end{macrocode}

%\iffalse
%</samplefinal>
%\fi
%
% %%%%%%%%%%%%%%%%%%%%%%%%%%%%%%%%%%%%%%
% \paragraph{Command Line Processing.}
%
% The following three command lines generate the output files
% |cdocscld|, |cdocscl1| and |cdocscl2|
% which should be identical to
% |cdocsdrf|, |cdocsch1| and |cdocsfn2|, respectively:
% \begin{center}
% \begin{tabular}{l}
% |latex -jobname cdocscld \|\\
% |  "\def\version{draft}\input{childdoc.def}\childdocforward{cdocsamp}"|\\
% |latex -jobname cdocscl1 \|\\
% |  "\input{childdoc.def}\childdocforward[cdocsamp]{cdocsch1}"|\\
% |latex -jobname cdocscl2 \|\\
% |  "\def\version{final}\input{childdoc.def}\childdocforward{cdocsch2}"|
% \end{tabular}
% \end{center}
% Note that the trailing backslash on each first line
% merely continues the input to the second line
% (for convenient cut ant paste).
% Furthermore, the command |latex| can be replaced by any
% of its alternative versions such as |pdflatex|.
%
% %%%%%%%%%%%%%%%%%%%%%%%%%%%%%%%%%%%%%%%%%%%%%%%%%%%%%%%%%%%%%%%%%%%%%%%%%%%%%%
% %%%%%%%%%%%%%%%%%%%%%%%%%%%%%%%%%%%%%%%%%%%%%%%%%%%%%%%%%%%%%%%%%%%%%%%%%%%%%%
% \section{Implementation}
%\iffalse
%<*package>
%\fi
%
% This section describes the definitions file |childdoc.def|.

% The definitions cannot be loaded using |\usepackage| or |\RequirePackage|
% which has a mechanism to prevent loading a style file more than once.
% When loading the definitions by means of |\input|
% multiple instances have to be prevented manually:
%\iffalse
%This code needs to be before the `\ProvidesFile' directive
%which is defined at the beginning of this file.
%Therefore it is also placed there and commented out here.
%</package>
%<*discard>
%\fi
%    \begin{macrocode}
\ifdefined\childdocmain\endinput\fi
%    \end{macrocode}
%\iffalse
%</discard>
%<*package>
%\fi
%
% \macro{\ifchilddoc}
% \macro{\ifchilddocmanual}
% The conditional |\ifchilddoc| tells whether a
% child (true) or main (false) document is being compiled.
% The conditional |\ifchilddocmanual| tells whether
% the |\includeonly| mechanism is used (false) or
% the selection of child files must be performed manually (true).
% The definitions initialise to false:
%    \begin{macrocode}
\newif\ifchilddoc
\newif\ifchilddocmanual
%    \end{macrocode}

% \macro{\childdocname}
% \macro{\childdocjob}
% The macro |\childdocname| stores the name of the main document
% to be compiled. The macro |\childdocjob| stores the name of
% the document on which the \LaTeX{} compiler was originally invoked.
% The content of |\jobname| cannot be compared
% to filenames specified in the source due to different catcodes.
% The following code rescans |\jobname|, stores the result
% in |\childdocname| and saves a copy in |\childdocjob|:
%    \begin{macrocode}
\edef\childdocname{\scantokens\expandafter{\jobname\noexpand}}
\let\childdocjob\childdocname
%    \end{macrocode}

% \macro{\childdocdisable}
% The macro |\childdocdisable| prevents the main file
% from being processed more than once.
% At this stage, the main document command |\childdocmain|
% is assumed to be called once again where it should do nothing.
% Any subsequent call to it should prevent
% a secondary processing of the main document
% It overwrites the forwarding commands
% |\childdocof| and |\childdocforward|
% with empty macros to prevent further inclusions of the main document:
%    \begin{macrocode}
\newcommand{\childdocdisable}
{
  \renewcommand{\childdocmain}[1]{\renewcommand{\childdocmain}[1]{\endinput}}
  \renewcommand{\childdocof}[1]{}
  \renewcommand{\childdocby}[2][]{}
  \renewcommand{\childdocforward}[2][]{}
  \renewcommand{\childdocdisable}{}
}
%    \end{macrocode}

% \macro{\childdocmain}
% The macro |\childdocmain| is to be called at the top of the main file
% with nothing or the main filename (without extension) as argument.
% First, it breaks loops.
% If the argument is not empty and does not match |\childdocname|
% (which is set by the first inclusion of |childdoc.def|),
% |\ifchilddoc| is set to true, |\includeonly| is applied to the child file
% and |\jobname| is set to the main file
% (for proper handling of |.aux| files):
%    \begin{macrocode}
\newcommand{\childdocmain}[1]
{
  \childdocdisable\childdocmain{}
  \if?#1?\else
    \begingroup
      \def\childdoctmp{#1}
      \ifx\childdoctmp\childdocname
        \def\childdoctmp{}
      \else
        \def\childdoctmp
        {
          \childdoctrue
          \includeonly{\childdocname}
          \def\childdocjob{#1}
          \def\jobname{#1}
        }
      \fi
      \expandafter
    \endgroup
    \childdoctmp
  \fi
}
%    \end{macrocode}

% \macro{\childdocof}
% The command |\childdocof| redirects
% compilation to the main file |#1|.
%    \begin{macrocode}
\newcommand{\childdocof}[1]
{
  \childdocdisable
  \childdoctrue
  \includeonly{\childdocname}
  \def\jobname{#1}
  \def\childdocjob{#1}
  \input{#1}
}
%    \end{macrocode}

% \macro{\childdocby}
% The command |\childdocby| ....
%    \begin{macrocode}
\newcommand{\childdocby}[2][]
{
  \childdocdisable
  \childdoctrue
  \childdocmanualtrue
  \if?#1?\else
    \def\jobname{#2}
  \fi
  \def\childdocjob{#2}
  \input{#2}
  \endinput
}
%    \end{macrocode}

% \macro{\childdocforward}
% The command |\childdocforward| redirects
% compilation to the main file or
% (if the optional argument is given) a child file.
% Parameters are set as if the main file
% or a child file starting with |\childdocof| was compiled.
% Then compilation is handed over to the main file:
%    \begin{macrocode}
\newcommand{\childdocforward}[2][]
{
  \begingroup
    \if?#1?
      \def\childdoctmp
      {
        \def\childdocname{#2}
        \def\childdocjob{#2}
        \def\jobname{#2}
        \input{#2}
        \endinput
      }
    \else
      \def\childdoctmp
      {
        \childdocdisable
        \def\childdocname{#2}
        \childdoctrue
        \includeonly{#2}
        \def\childdocjob{#1}
        \def\jobname{#1}
        \input{#1}
        \endinput
      }
    \fi
    \expandafter
  \endgroup
  \childdoctmp
}
%    \end{macrocode}

% \macro{\childdocforwardprefix}
% The command |\childdocforwardprefix| redirects
% compilation to the main or a child file by means of a pattern.
% The prefix |#1| in the current filename is replaced by |#2|
% and the suffix of the current filename is kept
% (it is assumed that the filename does not contain the substring `|~~~|'
% which is used as a delimiter).
% Compilation is handed over to the new file by |\childdocforward|:
%    \begin{macrocode}
\newcommand{\childdocforwardprefix}[3][]
{
  \begingroup
    \def\childdocextract #2##1~~~{\def\childdoctmp{\childdocforward[#1]{#3##1}}}
    \expandafter\childdocextract\childdocname~~~
    \expandafter
  \endgroup
  \childdoctmp
}
%    \end{macrocode}

% \macro{\childdoc}
% The deprecated macro |\childdoc| is a legacy version of |\childdocmain|:
%    \begin{macrocode}
\newcommand{\childdoc}{\childdocmain}
%    \end{macrocode}

% \macro{\childdocredirect}
% The deprecated macro |\childdocredirect| is a legacy version
% of |\childdocforward| and |\childdocforwardprefix|:
%    \begin{macrocode}
\newcommand{\childdocredirect}[2][]
{
  \begingroup
    \if?#1?
      \def\childdoctmp{\childdocforward{#2}}
    \else
      \def\childdoctmp{\childdocforwardprefix{#1}{#2}}
    \fi
    \expandafter
  \endgroup
  \childdoctmp
}
%    \end{macrocode}

%\iffalse
%</package>
%\fi
%
\endinput
|\\
|\childdocby{|\textit{main}|}|\\
\end{tabular}
\end{center}
%
Both forms have slightly different effects as described above.
The main file is prepared as usual, see \secref{sec:include}.

%%%%%%%%%%%%%%%%%%%%%%%%%%%%%%%%%%%%%%%%%%%%%%%%%%%%%%%%%%%%%%%%%%%%%%%%%%%%%%%%
\subsection{Legacy Detection}
\label{sec:detection}

The directive |\childdocmain| in the main file can detect
whether the complete document or merely a child is to be compiled
even without using the directive |\childdocof|.
This method is deprecated because it is less robust
and there is no compelling reason to use it;
it is merely provided for backward compatibility
and it may be removed in future versions.

If the detection mechanism is to be used,
it is mandatory to correctly specify
the filename of the main file as the argument of |\childdocmain|:
%
\begin{center}
\begin{tabular}{l}
|% \iffalse
%
% childdoc.dtx Copyright (C) 2017-2018 Niklas Beisert
%
% This work may be distributed and/or modified under the
% conditions of the LaTeX Project Public License, either version 1.3
% of this license or (at your option) any later version.
% The latest version of this license is in
%   http://www.latex-project.org/lppl.txt
% and version 1.3 or later is part of all distributions of LaTeX
% version 2005/12/01 or later.
%
% This work has the LPPL maintenance status `maintained'.
%
% The Current Maintainer of this work is Niklas Beisert.
%
% This work consists of the files childdoc.dtx and childdoc.ins
% and the derived files childdoc.def and cdocsamp.tex with
% cdocsch1.tex, cdocsch2.tex, cdocsdrf.tex, cdocsfn1.tex, cdocsfn2.tex.
%
%<package>\ifdefined\childdocmain\endinput\fi
%<package>\ProvidesFile{childdoc.def}[2018/12/30 v2.0 child document driver]
%<samplemain>\ProvidesFile{cdocsamp.tex}[2018/12/30 v2.0 sample for childdoc]
%<*driver>
%\ProvidesFile{childdoc.drv}[2018/12/30 v2.0 childdoc reference manual file]
\PassOptionsToClass{10pt,a4paper}{article}
\documentclass{ltxdoc}

\usepackage[margin=35mm]{geometry}
\usepackage{hyperref}
\usepackage{hyperxmp}
\usepackage[usenames]{color}

\hypersetup{colorlinks=true}
\hypersetup{pdfstartview=FitH}
\hypersetup{pdfpagemode=UseNone}
\hypersetup{pdfsource={}}
\hypersetup{pdflang={en-UK}}
\hypersetup{pdfcopyright={Copyright 2017-2018 Niklas Beisert.
  This work may be distributed and/or modified under the
  conditions of the LaTeX Project Public License, either version 1.3
  of this license or (at your option) any later version.}}
\hypersetup{pdflicenseurl={http://www.latex-project.org/lppl.txt}}
\hypersetup{pdfcontactaddress={ETH Zurich, ITP, HIT K,
  Wolfgang-Pauli-Strasse 27}}
\hypersetup{pdfcontactpostcode={8093}}
\hypersetup{pdfcontactcity={Zurich}}
\hypersetup{pdfcontactcountry={Switzerland}}
\hypersetup{pdfcontactemail={nbeisert@itp.phys.ethz.ch}}
\hypersetup{pdfcontacturl={http://people.phys.ethz.ch/\xmptilde nbeisert/}}

\newcommand{\secref}[1]{\hyperref[#1]{section \ref*{#1}}}

\parskip1ex
\parindent0pt
\let\olditemize\itemize
\def\itemize{\olditemize\parskip0pt}

\begin{document}

\title{The \textsf{childdoc} Package}
\hypersetup{pdftitle={The childdoc Package}}
\author{Niklas Beisert\\[2ex]
  Institut f\"ur Theoretische Physik\\
  Eidgen\"ossische Technische Hochschule Z\"urich\\
  Wolfgang-Pauli-Strasse 27, 8093 Z\"urich, Switzerland\\[1ex]
  \href{mailto:nbeisert@itp.phys.ethz.ch}
  {\texttt{nbeisert@itp.phys.ethz.ch}}}
\hypersetup{pdfauthor={Niklas Beisert}}
\hypersetup{pdfsubject={Manual for the LaTeX2e Package childdoc}}
\date{30 December 2018, \textsf{v2.0}}
\maketitle

\begin{abstract}\noindent
\textsf{childdoc} is a \LaTeXe{} package
that enables the direct compilation
of document sections included by |\include|
to individual files.
\end{abstract}

\begingroup
\parskip0ex
\tableofcontents
\endgroup

%%%%%%%%%%%%%%%%%%%%%%%%%%%%%%%%%%%%%%%%%%%%%%%%%%%%%%%%%%%%%%%%%%%%%%%%%%%%%%%%
%%%%%%%%%%%%%%%%%%%%%%%%%%%%%%%%%%%%%%%%%%%%%%%%%%%%%%%%%%%%%%%%%%%%%%%%%%%%%%%%
\section{Introduction}

\LaTeX{} provides a mechanism to structure a large document (such as a book)
into a main file and several child files (containing the chapters)
using the |\include| command.
This mechanism is beneficial for documents
which span hundreds of pages in order to
make the source file(s) more manageable.
Moreover, compilation can be restricted to
selected child files by means of the |\includeonly| command.
The latter feature can be used to reduce the compilation time while editing
(this was significantly more useful in the earlier days of \LaTeX{})
or to generate a smaller document which is easier to navigate.
Another application of |\includeonly| is to generate
documents consisting of selected parts of the complete document.

However, there are a few drawbacks of the plain |\include| mechanism:
\begin{itemize}
\item
The child files cannot be compiled on their own,
they can only be compiled via the main file.
A naive editing environment
(such as a text editor with an option
to have the current file processed by \LaTeX)
may require one to switch to the main file before compiling;
attempting to compile the child file produces errors.
\item
The main file must be modified (each time)
to adjust the |\includeonly| command
to the present needs. This easily leaves the main file in a messy state.
\item
The generated document will always carry the filename
of the main document. This is inconvenient if
several child files are to be compiled and
to be kept for distribution.
\end{itemize}

The present package provides a simple interface
to make child files individually compilable by \LaTeX{}.
Compiling a child file then has the same effect as compiling
the main file with an |\includeonly| command
to select the appropriate child.
Moreover the generated document will carry the name of the child
rather than the main file.
This resolves all three above issues.

This feature is meant to make the editing of books,
thesis documents and lecture notes somewhat more convenient.
However, the package can also be used efficiently for
composing a series of documents (such as exercise sheets)
which are typically distributed individually.
It then assists the author in generating the individual documents
(potentially in different versions)
as well as a document containing the collected series.
Another application is in developing style files
or other kinds of included material
where compilation of the style file could redirect
to a sample or test file.

%%%%%%%%%%%%%%%%%%%%%%%%%%%%%%%%%%%%%%%%%%%%%%%%%%%%%%%%%%%%%%%%%%%%%%%%%%%%%%%%
%%%%%%%%%%%%%%%%%%%%%%%%%%%%%%%%%%%%%%%%%%%%%%%%%%%%%%%%%%%%%%%%%%%%%%%%%%%%%%%%
\section{Usage}

First of all, the package \textsf{childdoc} is \emph{not} a standard
\LaTeXe{} |.sty| style file! Therefore it needs to be invoked in
a non-standard way.

%%%%%%%%%%%%%%%%%%%%%%%%%%%%%%%%%%%%%%%%%%%%%%%%%%%%%%%%%%%%%%%%%%%%%%%%%%%%%%%%
\subsection{Included Files}
\label{sec:include}

%%%%%%%%%%%%%%%%%%%%%%%%%%%%%%%%%%%%%%%%
\DescribeMacro{\childdocmain}
To use the package, add the commands
\begin{center}
\begin{tabular}{l}
|\input{childdoc.def}|\\
|\childdocmain{}|\\
\end{tabular}
\end{center}
at the very top of the main \LaTeX{} file,
in particular \emph{before} the |\documentclass| statement!
The argument of |\childdocmain| should be left empty
(but it must be present).

%%%%%%%%%%%%%%%%%%%%%%%%%%%%%%%%%%%%%%%%
\DescribeMacro{\childdocof}
Furthermore, add the commands
\begin{center}
\begin{tabular}{l}
|\input{childdoc.def}|\\
|\childdocof{|\textit{main}|}|\\
\end{tabular}
\end{center}
at the top of every child file \textit{child}
which is included by |\include{|\textit{child}|}|
from within the main file
(or at least for those files to be compiled individually).
The argument \textit{main} must be the filename of the main file.

There are a couple of
considerations in setting up the main and child documents:

%%%%%%%%%%%%%%%%%%%%%%%%%%%%%%%%%%%%%%%%
\paragraph{Restrictions.}

Please note the following restrictions:
\begin{itemize}
\item
|\childdocmain| must be called with one argument \textit{main}
to ensure compatibility with earlier version of the package.
It must either be empty (|\childdocmain{}|)
or precisely match the filename of the main file in which it is specified.
See \secref{sec:detection} for further information.
\item
The filename \textit{main} must be specified without the |.tex| extension.
\item
The filename \textit{main} is case sensitive
(even in case-insensitive file systems)
due to internal string comparison.
\item
The argument \textit{main} should be fully expanded, it cannot be a macro.
\item
Subdirectories and special characters should be avoided in filenames.
\item
The command |\childdocmain{|\textit{main}|}| must be followed by a whitespace.
It should not be followed immediately by another command
or by a comment mark `|%|'.
This is because the \TeX{} parser reads the token immediately following
the argument of |\childdocmain| and puts it
at the beginning of every child section;
however, a white\-space is ignored.
\end{itemize}

%%%%%%%%%%%%%%%%%%%%%%%%%%%%%%%%%%%%%%%%
\paragraph{Content of Main File.}

It is advisable to place all content in the child files included by |\include|.
Any output contained in the main file will appear in all child documents
unless suppressed manually;
it cannot be suppressed automatically by the |\includeonly| directive
and thus should normally be avoided.
A method to include some content in the main file
by means of conditional processing is described in \secref{sec:conditional}.

%%%%%%%%%%%%%%%%%%%%%%%%%%%%%%%%%%%%%%%%
\paragraph{Page Numbering.}

When only a part of the document is compiled,
the appropriate numbering of pages
(as well as other status parameters)
is determined from the |.aux| files.
The latter contain information from previous passes.
However this information needs to propagate through
all intermediate child documents.
Therefore the page numbering in child documents may well
be inconsistent until the complete document is compiled at least once.

A useful (if unconventional) way to always ensure a consistent
page numbering is to restart the numbering in each child document
and denote the pages by `\textit{child}|.|\textit{page}'
where \textit{child} represents the chapter/section number of the child file.
This can be achieved by the command
|\numberwithin{page}{|\textit{child}|}|
of the \textsf{amsmath} package
where \textit{child} can be |chapter| or |section|
depending on the chosen structuring.
Alternatively, one can modify the macro |\thepage| appropriately
and reset the counter |page| at the start of each child file.

%%%%%%%%%%%%%%%%%%%%%%%%%%%%%%%%%%%%%%%%%%%%%%%%%%%%%%%%%%%%%%%%%%%%%%%%%%%%%%%%
\subsection{Conditional Processing}
\label{sec:conditional}

The package provides a mechanism to compile different versions
of a document. To customise the versions further some conditional processing
can come in handy to distinguish which version is being compiled.
The package provides two macros to describe the compilation context:

%%%%%%%%%%%%%%%%%%%%%%%%%%%%%%%%%%%%%%%%
\DescribeMacro{\ifchilddoc}
The conditional |\ifchilddoc| distinguishes between the compilation of
child documents and the main document:
%
\begin{center}
|\ifchilddoc |\textit{child-code}| |[|\||else |\textit{main-code}]| \||fi|
\end{center}

%%%%%%%%%%%%%%%%%%%%%%%%%%%%%%%%%%%%%%%%
\DescribeMacro{\childdocname}
\DescribeMacro{\childdocjob}
The macro |\childdocname| contains the filename (without extension)
of the main or child file being processed.
Note that |\childdocjob| will always contain the name of the main file.

%%%%%%%%%%%%%%%%%%%%%%%%%%%%%%%%%%%%%%%%
\paragraph{Title Page.}

Conditional processing can be used to include a title or banner page
in the main document when proper precautions are taken.
Importantly, the code in the main file should ensure that the page counter
(as well as other status parameters which are stored in the |.aux| files)
takes the same value after the conditional processing.
Otherwise the page numbers may take divergent values
depending on which part is compiled.

For example, a title page could be declared by:
%
\begin{center}
\begin{tabular}{l}
|\ifchilddoc\||else|\\
|\addtocounter{page}{-1}|\\
\textit{code for title page}\\
|\newpage|\\
|\||fi|
\end{tabular}
\end{center}
%
A banner page for the child documents can be generated by:
%
\begin{center}
\begin{tabular}{l}
|\ifchilddoc|\\
|\addtocounter{page}{-1}|\\
\textit{code for banner page}\\
|\newpage|\\
|\||fi|
\end{tabular}
\end{center}
%
Here one could write a message such as:
\begin{center}
|This is the part \childdocname{} of \childdocjob{}.|
\end{center}

%%%%%%%%%%%%%%%%%%%%%%%%%%%%%%%%%%%%%%%%%%%%%%%%%%%%%%%%%%%%%%%%%%%%%%%%%%%%%%%%
\subsection{Flags}
\label{sec:flags}

The package makes it easy to generate different versions
of the main or child documents.
To this end compilation flags can be defined
and assigned different default values.
They will be particularly useful in conjunction
with the forwarding mechanism described in \secref{sec:forward}.

For example, it may be useful to have a flag |\version|
which can be set to |draft| or |final|.
The document source will contain some conditional code
depending on the value of |\version|.
Suppose further, the flag should default to |final| for the main file
and to |draft| for child files
which is a natural assignment for editing the document.
This is achieved by placing the following code
in the preamble of the main document
(below the |\childdocmain| directive):
%
\begin{center}
\begin{tabular}{l}
|\ifchilddoc|\\
|\providecommand{\version}{draft}|\\
|\||else|\\
|\providecommand{\version}{final}|\\
|\||fi|
\end{tabular}
\end{center}
%
The definition by |\providecommand| makes sure
that previous definitions are not overwritten.
Further statements |\providecommand{\version}{...}|
can thus be added before the above code to override it.

For the main file, one might add a line
(between |\childdocmain| and the above block)
%
\begin{center}
|%\ifchilddoc\||else\providecommand{\version}{draft}\||fi|
\end{center}
%
which can be uncommented to produce a draft version.
Likewise one can add a line to the very top of a child file
(above the |\childdocof{|\textit{main}|}| directive)
%
\begin{center}
|%\providecommand{\version}{final}|
\end{center}
%
which can be uncommented to produce the final version of this child document.

%%%%%%%%%%%%%%%%%%%%%%%%%%%%%%%%%%%%%%%%%%%%%%%%%%%%%%%%%%%%%%%%%%%%%%%%%%%%%%%%
\subsection{Forwarding}
\label{sec:forward}

Different versions of the main or child documents
using compilation flags as described in \secref{sec:flags}
can be (permanently) stored in different files
for convenient compilation, viewing and distribution.
To this end, the package defines a command
to pass on compilation to a different file:

%%%%%%%%%%%%%%%%%%%%%%%%%%%%%%%%%%%%%%%%
\DescribeMacro{\childdocforward}
The command |\childdocforward| redirects processing to
another source file:
%
\begin{center}
\begin{tabular}{l}
|\input{childdoc.def}|\\
|\childdocforward[|\textit{main}|]{|\textit{dest}|}|\\
\end{tabular}
\end{center}
%
The argument \textit{dest} is the destination file
(without extension).
It should be the main file or one of the child files.
Note that further \textsf{childdoc} directives
such as |\childdocof| and |\childdocforward|
in the indicated file will be processed in this form.
The optional argument \textit{main}
passes on directly to the main file \textit{main}
while pretending to compile the child \textit{dest}.
This form behaves as if \textit{dest}
issues |\childdocof{|\textit{main}|}| right away,
and no further \textsf{childdoc} directives will be processed.

%%%%%%%%%%%%%%%%%%%%%%%%%%%%%%%%%%%%%%%%
\DescribeMacro{\...prefix}
In the alternative form |\childdocforwardprefix|,
%
\begin{center}
\begin{tabular}{l}
|\input{childdoc.def}|\\
|\childdocforwardprefix[|\textit{main}|]{|\textit{prefix}|}{|\textit{dest}|}|
\end{tabular}
\end{center}
%
the destination file is determined by a pattern
depending on the current file:
To make this work, the current file must be called
`{\textit{prefix}\hspace{0.2em}\textit{suffix}}'
with \textit{prefix} matching precisely the argument.
Processing is then passed on to the file
`{\textit{dest}\hspace{0.2em}\textit{suffix}}'.
Surely, the same effect is achieved by
directly specifying the
argument `{\textit{dest}\hspace{0.2em}\textit{suffix}}'
in the first form.
However, that requires to set up a different file
for each child. With the alternative form of the command
all these files can have exactly the same content
which simplifies setting them up and maintaining them.

For example, the following file |draft.tex|
with a compilation flag |\version| as described in \secref{sec:flags}
compiles the main document as a draft:
%
\begin{center}
\begin{tabular}{l}
|\def\version{draft}|\\
|\input{childdoc.def}|\\
|\childdocforward{|\textit{main}|}|
\end{tabular}
\end{center}
%
Likewise, the following files |final|\textit{nn}|.tex|
compile the final version of the child document
|child|\textit{nn}|.tex|:
%
\begin{center}
\begin{tabular}{l}
|\def\version{final}|\\
|\input{childdoc.def}|\\
|\childdocforwardprefix{final}{child}|
\end{tabular}
\end{center}
%

Note that when several versions of a main file and/or of each child file
are to be generated, it may be convenient to set up a |Makefile| or
shell script to automatise the process.

%%%%%%%%%%%%%%%%%%%%%%%%%%%%%%%%%%%%%%%%%%%%%%%%%%%%%%%%%%%%%%%%%%%%%%%%%%%%%%%%
\subsection{Command Line Processing}
\label{sec:commandline}

The effect of redirection files can also be achieved by invoking
the \LaTeX{} compiler with a more elaborate command line.
Most conveniently this should be done as part
of a shell script or a |Makefile|.

When using \textsf{childdoc} in the main file, the following
command lines effectively perform a redirection
(note that depending on the shell being used,
backslashes may have to be doubled: `|\|' $\to$ `|\\|'):
%
\begin{center}
|... -jobname "|\textit{target}|" |\\|"|[\textit{flags}]%
|\input{childdoc.def}\childdocforward[|\textit{main}|]{|\textit{dest}|}"|
\end{center}
%
Here \textit{target} is the name of the output file,
\textit{main} is the name of the main file
and \textit{dest} is the name of the main or child file to be processed
(all filenames without extensions).
The optional argument \textit{main} can be omitted
if \textit{main} matches \textit{dest}.
Optionally, compilation \textit{flags} can be defined via |\def| commands.
This command line makes the \TeX{} engine believe
it is compiling the file \textit{target}
whose content is specified as the latter parameter.
The provided code then forwards the processing to
\textit{main} or \textit{dest} as described in \secref{sec:forward}.

%%%%%%%%%%%%%%%%%%%%%%%%%%%%%%%%%%%%%%%%%%%%%%%%%%%%%%%%%%%%%%%%%%%%%%%%%%%%%%%%
\subsection{Include by Input}
\label{sec:input}

Including child documents by |\include| has some restrictions by design.
Most notably, the content of a child document always occupies
its own set of pages; pages cannot be shared between child documents.
Usually, this behaviour makes perfect sense
because each child document contain an essential part of the document.
However, in some situations it may be desirable to compose
a document from a collection of parts
without having mandatory page breaks between then.
For this case, the package
provides a mechanism to include parts
by |\input| which can also be processed individually.
However, by construction this mechanism
requires manual handling of the content to be output.

%%%%%%%%%%%%%%%%%%%%%%%%%%%%%%%%%%%%%%%%
\DescribeMacro{\ifchilddocmanual}
The main file should be prepared as usual, see \secref{sec:include}.
However, the document body must make a distinction
between processing of an individual part and of the main document, e.g.:
%
\begin{center}
\begin{tabular}{l}
|\ifchilddocmanual|\\
|\input{\childdocname}|\\
|\||else|\\
\textit{document body with }|\input{|\textit{part}|}|\\
|\||fi|
\end{tabular}
\end{center}
%
The conditional |\ifchilddocmanual| is true whenever
a part to be included by |\input| is being compiled,
and the name of the part is stored in |\childdocname|.

%%%%%%%%%%%%%%%%%%%%%%%%%%%%%%%%%%%%%%%%
\DescribeMacro{\childdocby}
Each part to be included by |\input| should start with:
%
\begin{center}
\begin{tabular}{l}
|\input{childdoc.def}|\\
|\childdocby{|\textit{main}|}|\\
\end{tabular}
\end{center}
%
The directive |\childdocby| is similar to |\childdocof|
described in \secref{sec:include},
but the subsequent selection of content must be done manually.
To that end, both |\ifchilddoc| and |\ifchilddocmanual|
will be true upon processing of a part,
and the name of the part is stored in |\childdocname|.
Note that |\jobname| will be set to the filename of the current part
so that each part receives an individual |.aux| file
that does not interfere with the |.aux| file(s) of the main document.
This behaviour can be altered by the alternative form
|\childdocby[*]{|\textit{main}|}| (with a non-empty optional argument)
which uses the |.aux| file of the main document
by setting |\jobname| to \textit{main}.

%%%%%%%%%%%%%%%%%%%%%%%%%%%%%%%%%%%%%%%%%%%%%%%%%%%%%%%%%%%%%%%%%%%%%%%%%%%%%%%%
\subsection{Driver Development}
\label{sec:driver}

The \textsf{childdoc} mechanism can also be use for the development
of definition files such as \LaTeX{} styles or classes.
This case differs from the above setup with multiple parts
included by |\include| in that no |\includeonly| should be invoked.
This can be achieved by starting the include file
(before |\ProvidesPackage|) with:
%
\begin{center}
\begin{tabular}{l}
|\input{childdoc.def}|\\
|\childdocforward{|\textit{main}|}|\\
\end{tabular}
\end{center}
%
or alternatively with:
%
\begin{center}
\begin{tabular}{l}
|\input{childdoc.def}|\\
|\childdocby{|\textit{main}|}|\\
\end{tabular}
\end{center}
%
Both forms have slightly different effects as described above.
The main file is prepared as usual, see \secref{sec:include}.

%%%%%%%%%%%%%%%%%%%%%%%%%%%%%%%%%%%%%%%%%%%%%%%%%%%%%%%%%%%%%%%%%%%%%%%%%%%%%%%%
\subsection{Legacy Detection}
\label{sec:detection}

The directive |\childdocmain| in the main file can detect
whether the complete document or merely a child is to be compiled
even without using the directive |\childdocof|.
This method is deprecated because it is less robust
and there is no compelling reason to use it;
it is merely provided for backward compatibility
and it may be removed in future versions.

If the detection mechanism is to be used,
it is mandatory to correctly specify
the filename of the main file as the argument of |\childdocmain|:
%
\begin{center}
\begin{tabular}{l}
|\input{childdoc.def}|\\
|\childdocmain{|\textit{main}|}|\\
\end{tabular}
\end{center}
%
If |\jobname| does not match the argument \textit{main} of |\childdocmain|,
it is assumed that |\jobname| points to the child file to be compiled.
When using |\childdocmain| with the main file specified as argument,
it suffices to start a child file
with just |\input{|\textit{main}|}|
without loading of the package and using |\childdocof|.
If instead all processing is done
with the appropriate \textsf{childdoc} directives,
the argument of \textit{main} of |\childdocmain| can be empty.

An alternative version of the command line processing described
in \secref{sec:commandline} using the detection mechanism reads:
%
\begin{center}
|... -jobname "|\textit{target}|" "|[\textit{flags}]%
[|\def\jobname{|\textit{dest}|}|]|\input{|\textit{main}|}"|
\end{center}

%%%%%%%%%%%%%%%%%%%%%%%%%%%%%%%%%%%%%%%%%%%%%%%%%%%%%%%%%%%%%%%%%%%%%%%%%%%%%%%%
\subsection{Manual Code}
\label{sec:manual}

In case one cannot be certain whether the definitions file |childdoc.def|
is installed on the target \TeX{} distribution
and one prefers not to ship it,
it is conceivable to paste a few relevant commands into the sources.

To that end, drop all statements |\input{childdoc.def}|
and perform the replacements as outlined below.
Instead of |\childdocmain{|\textit{main}|}| add the following code
to the top of the main file:
%
\begin{center}
\begin{tabular}{l}
|\||ifdefined\childdocname\endinput\||fi\newif\ifchilddoc|\\
|\edef\childdocname{\scantokens\expandafter{\jobname\noexpand}}|\\
|\def\childdocmain{|\textit{main}|}\||ifx\childdocmain\childdocname\||else|\\
|\childdoctrue\includeonly{\childdocname}\let\jobname\childdocmain\||fi|\\
\end{tabular}
\end{center}
%
Instead of |\childdocof{|\textit{main}|}| just include the main file
at the top of each child file:
%
\begin{center}
|\input{|\textit{main}|}|
\end{center}
%
A simple redirection |\childdocforward{|\textit{dest}|}| is achieved by:
%
\begin{center}
|\def\jobname{|\textit{dest}|}\input{\jobname}|
\end{center}
%
The redirection with prefix
|\childdocforwardprefix[|\textit{prefix}|]{|\textit{dest}|}|
is accomplished by:
%
\begin{center}
\begin{tabular}{l}
|{\edef\jobname{\scantokens\expandafter{\jobname\noexpand}}|\\
|\def\redirectjob |\textit{prefix}|#1~~~{\gdef\jobname{|\textit{dest}|#1}}|\\
|\expandafter\redirectjob\jobname~~~}\input{\jobname}|
\end{tabular}
\end{center}

In an alternative approach,
child documents can be compiled by a specific command line
without additional code or specific definitions:
%
\begin{center}
|... -jobname "|\textit{target}|" "|[\textit{flags}]%
|\includeonly{|\textit{dest}|}\input{|\textit{main}|}"|
\end{center}
%

%%%%%%%%%%%%%%%%%%%%%%%%%%%%%%%%%%%%%%%%%%%%%%%%%%%%%%%%%%%%%%%%%%%%%%%%%%%%%%%%
%%%%%%%%%%%%%%%%%%%%%%%%%%%%%%%%%%%%%%%%%%%%%%%%%%%%%%%%%%%%%%%%%%%%%%%%%%%%%%%%
\section{Information}

%%%%%%%%%%%%%%%%%%%%%%%%%%%%%%%%%%%%%%%%%%%%%%%%%%%%%%%%%%%%%%%%%%%%%%%%%%%%%%%%
\subsection{Copyright}

Copyright \copyright{} 2017--2018 Niklas Beisert

This work may be distributed and/or modified under the
conditions of the \LaTeX{} Project Public License, either version 1.3
of this license or (at your option) any later version.
The latest version of this license is in
  \url{http://www.latex-project.org/lppl.txt}
and version 1.3 or later is part of all distributions of \LaTeX{}
version 2005/12/01 or later.

This work has the LPPL maintenance status `maintained'.

The Current Maintainer of this work is Niklas Beisert.

This work consists of the files |README.txt|, |childdoc.ins| and |childdoc.dtx|
as well as the derived files |childdoc.def|, |cdocsamp.tex|
with |cdocsch1.tex|, |cdocsch2.tex|, |cdocspt3.tex|, |cdocspt4.tex|,
|cdocsdrf.tex|, |cdocsfn1.tex|, |cdocsfn2.tex|
as well as |childdoc.pdf|.

%%%%%%%%%%%%%%%%%%%%%%%%%%%%%%%%%%%%%%%%%%%%%%%%%%%%%%%%%%%%%%%%%%%%%%%%%%%%%%%%
\subsection{Files and Installation}

The package consists of the files:
%
\begin{center}
\begin{tabular}{ll}
    |README.txt|   & readme file \\
    |childdoc.ins| & installation file \\
    |childdoc.dtx| & source file \\
    |childdoc.def| & definition file \\
    |cdocsamp.tex| & sample main file \\
    |cdocsch1.tex| & sample include file \\
    |cdocsch2.tex| & sample include file \\
    |cdocspt3.tex| & sample part file \\
    |cdocspt4.tex| & sample part file \\
    |cdocsdrf.tex| & sample redirection file \\
    |cdocsfn1.tex| & sample redirection file \\
    |cdocsfn2.tex| & sample redirection file \\
    |childdoc.pdf| & manual
\end{tabular}
\end{center}
%
The distribution consists of the files
|README.txt|, |childdoc.ins| and |childdoc.dtx|.
%
\begin{itemize}
\item
Run (pdf)\LaTeX{} on |childdoc.dtx|
to compile the manual |childdoc.pdf| (this file).
\item
Run \LaTeX{} on |childdoc.ins| to create the definitions file |childdoc.def|
and the sample |cdocsamp.tex| with include files
|cdocsch1.tex|, |cdocsch2.tex|, |cdocspt3.tex|, |cdocspt4.tex|,
|cdocsdrf.tex|, |cdocsfn1.tex|, |cdocsfn2.tex|.
Then copy the file |childdoc.def| to an appropriate directory of your \LaTeX{}
distribution, e.g.\ \textit{texmf-root}|/tex/latex/childdoc|.
\end{itemize}

%%%%%%%%%%%%%%%%%%%%%%%%%%%%%%%%%%%%%%%%%%%%%%%%%%%%%%%%%%%%%%%%%%%%%%%%%%%%%%%%
\subsection{Related CTAN Packages}

There are several other packages which offer a similar functionality:
%
\begin{itemize}
\item
The packages
\href{http://ctan.org/pkg/docmute}{\textsf{docmute}},
\href{http://ctan.org/pkg/includex}{\textsf{includex}} and
\href{http://ctan.org/pkg/standalone}{\textsf{standalone}}
provide commands to include only the document body of
a child file thus allowing both files to be compiled individually.
\item
The packages \href{http://ctan.org/pkg/subdocs}{\textsf{subdocs}}
and \href{http://ctan.org/pkg/subfiles}{\textsf{subfiles}}
provide structures in which the main and child documents can be
encapsulated and allowing them to be compiled individually.
The inclusion mechanism is different from the conventional |\include|.
\item
The package \href{http://ctan.org/pkg/combine}{\textsf{combine}}
is an elaborate solution to combine several documents into one.
\end{itemize}
%
See also the CTAN topic \href{http://ctan.org/topic/subdocs}{\textsf{subdocs}}
for further related packages.
The present package differs from the above solutions in that
a document structure constructed with the conventional |\include| mechanism
just needs two extra commands at the top of every file
such that all constituent files can be compiled individually.

%%%%%%%%%%%%%%%%%%%%%%%%%%%%%%%%%%%%%%%%%%%%%%%%%%%%%%%%%%%%%%%%%%%%%%%%%%%%%%%%
%\subsection{Feature Suggestions}
%
%The following is a list of features which may be useful for future
%versions of this package:
%%
%\begin{itemize}
%\item
%\ldots
%\end{itemize}

%%%%%%%%%%%%%%%%%%%%%%%%%%%%%%%%%%%%%%%%%%%%%%%%%%%%%%%%%%%%%%%%%%%%%%%%%%%%%%%%
\subsection{Revision History}

%%%%%%%%%%%%%%%%%%%%%%%%%%%%%%%%%%%%%%%%
\paragraph{v2.0:} 2018/12/30

\begin{itemize}
\item
immediate forward processing
\item
added |\childdocby| mechanism
\item
manual restructured
\end{itemize}

%%%%%%%%%%%%%%%%%%%%%%%%%%%%%%%%%%%%%%%%
\paragraph{v1.6:} 2018/01/17

\begin{itemize}
\item
application for development of include files
\item
corrections to manual
\end{itemize}

%%%%%%%%%%%%%%%%%%%%%%%%%%%%%%%%%%%%%%%%
\paragraph{v1.5:} 2017/05/21

\begin{itemize}
\item
more complete structuring introduced
\item
|\childdocof| introduced
\item
|\childdoc| renamed to |\childdocmain|
\item
|\childredirect| renamed to |\childdocforward| and |\childdocforwardprefix|
and functionality expanded
\end{itemize}

%%%%%%%%%%%%%%%%%%%%%%%%%%%%%%%%%%%%%%%%
\paragraph{v1.0:} 2017/04/27

\begin{itemize}
\item
manual and install package
\item
first version published on CTAN
\end{itemize}

%%%%%%%%%%%%%%%%%%%%%%%%%%%%%%%%%%%%%%%%
\paragraph{v0.6:} 2017/04/26

\begin{itemize}
\item
redirection mechanism added
\end{itemize}

%%%%%%%%%%%%%%%%%%%%%%%%%%%%%%%%%%%%%%%%
\paragraph{v0.5:} 2017/04/26

\begin{itemize}
\item
functionality in definition file
\end{itemize}


%%%%%%%%%%%%%%%%%%%%%%%%%%%%%%%%%%%%%%%%%%%%%%%%%%%%%%%%%%%%%%%%%%%%%%%%%%%%%%%%
%%%%%%%%%%%%%%%%%%%%%%%%%%%%%%%%%%%%%%%%%%%%%%%%%%%%%%%%%%%%%%%%%%%%%%%%%%%%%%%%
%%%%%%%%%%%%%%%%%%%%%%%%%%%%%%%%%%%%%%%%%%%%%%%%%%%%%%%%%%%%%%%%%%%%%%%%%%%%%%%%
\appendix

\settowidth\MacroIndent{\rmfamily\scriptsize 000\ }

 \DocInput{childdoc.dtx}

\end{document}
%</driver>
% \fi
%
% %%%%%%%%%%%%%%%%%%%%%%%%%%%%%%%%%%%%%%%%%%%%%%%%%%%%%%%%%%%%%%%%%%%%%%%%%%%%%%
% %%%%%%%%%%%%%%%%%%%%%%%%%%%%%%%%%%%%%%%%%%%%%%%%%%%%%%%%%%%%%%%%%%%%%%%%%%%%%%
% \section{Sample}
%\iffalse
%<*samplemain>
%\fi
%
% The following presents a sample document
% with two chapters, two parts, a title page,
% a compile flag as well as three forwarding files to set the flag.
% It consists of eight |.tex| files:
% \begin{center}
% \begin{tabular}{ll}
% |cdocsamp.tex|&main file\\
% |cdocsch1.tex|&include file for chapter 1\\
% |cdocsch2.tex|&include file for chapter 2\\
% |cdocspt3.tex|&include file for part 3\\
% |cdocspt4.tex|&include file for part 4\\
% |cdocsdrf.tex|&forwarding file for main file in draft mode\\
% |cdocsfi1.tex|&forwarding file for final version of chapter 1\\
% |cdocsfi2.tex|&forwarding file for final version of chapter 2\\
% \end{tabular}
% \end{center}
% Each of the eight files can be compiled directly by the \LaTeX{} compiler.
%
% %%%%%%%%%%%%%%%%%%%%%%%%%%%%%%%%%%%%%%
% \paragraph{Main File.}
%
% The main file is called |cdocsamp.tex|.
%
% Load the \textsf{childdoc} definitions and
% declare the filename for the main document:
%    \begin{macrocode}
\input{childdoc.def}
\childdocmain{}
%    \end{macrocode}

% Optional override for |\version| flag:
%    \begin{macrocode}
%%\ifchilddoc\else\providecommand{\version}{draft}\fi
%    \end{macrocode}

% Define the default values for the |\version| flag
% (|final| for the main file and |draft| for childs):
%    \begin{macrocode}
\ifchilddoc
\providecommand{\version}{draft}
\else
\providecommand{\version}{final}
\fi
%    \end{macrocode}

% Load the standard document class:
%    \begin{macrocode}
\documentclass[12pt]{article}
%    \end{macrocode}

% Start the document body:
%    \begin{macrocode}
\begin{document}
%    \end{macrocode}

% Declare a title page.
% Print title, part of document being processed and version flag:
%    \begin{macrocode}
\addtocounter{page}{-1}
\begin{center}
{\LARGE\bfseries{}childdoc example\par}
\vspace{1cm}
\ifchilddoc
\ifchilddocmanual part\else chapter\fi:
`\childdocname' of `\childdocjob'\par
\else
main document: `\childdocjob'\par
\fi
version: \version\par
\end{center}
\newpage
%    \end{macrocode}

% Manually include selected file,
% otherwise process as usual:
%    \begin{macrocode}
\ifchilddocmanual
\section*{part `\childdocname'}
\input{\childdocname}
\else
%    \end{macrocode}

% Include the two chapters:
%    \begin{macrocode}
\include{cdocsch1}
\include{cdocsch2}
%    \end{macrocode}

% Include the two parts unless only chapters should be displayed:
%    \begin{macrocode}
\ifchilddoc\else
\section{part three}
\input{cdocspt3}
\section{part four}
\input{cdocspt4}
\fi
%    \end{macrocode}

% Process as usual until here:
%    \begin{macrocode}
\fi
%    \end{macrocode}

% End of document body:
%    \begin{macrocode}
\end{document}
%    \end{macrocode}
%\iffalse
%</samplemain>
%\fi
%
% %%%%%%%%%%%%%%%%%%%%%%%%%%%%%%%%%%%%%%
% \paragraph{Chapter Include Files.}
%
% The include files are called |cdocsch1.tex| and |cdocsch2.tex|.
%
%\iffalse
%<*samplechap1|samplechap2>
%\fi

% Optional override for |\version| flag:
%    \begin{macrocode}
%%\providecommand{\version}{final}
%    \end{macrocode}

% Include the main document:
%    \begin{macrocode}
\input{childdoc.def}
\childdocof{cdocsamp}
%    \end{macrocode}

%\iffalse
%</samplechap1|samplechap2>
%\fi
%
%\iffalse
%<*samplechap1>
%\fi
% Some text for chapter 1:
%    \begin{macrocode}
\section{one}
some text in chapter one
%    \end{macrocode}

%\iffalse
%</samplechap1>
%\fi
% Some text for chapter 2:
%\iffalse
%<*samplechap2>
%\fi
%    \begin{macrocode}
\section{two}
more text in chapter two
%    \end{macrocode}

%\iffalse
%</samplechap2>
%\fi
%
% %%%%%%%%%%%%%%%%%%%%%%%%%%%%%%%%%%%%%%
% \paragraph{Part Include Files.}
%
% The include files are called |cdocspt3.tex| and |cdocspt4.tex|.
%
%\iffalse
%<*samplepart3|samplepart4>
%\fi

% Optional override for |\version| flag:
%    \begin{macrocode}
%%\providecommand{\version}{final}
%    \end{macrocode}

% Include the main document:
%    \begin{macrocode}
\input{childdoc.def}
\childdocby{cdocsamp}
%    \end{macrocode}

%\iffalse
%</samplepart3|samplepart4>
%\fi
%
%\iffalse
%<*samplepart3>
%\fi
% Some text for part 3:
%    \begin{macrocode}
some text in part three
%    \end{macrocode}

%\iffalse
%</samplepart3>
%\fi
% Some text for part 4:
%\iffalse
%<*samplepart4>
%\fi
%    \begin{macrocode}
more text in part four
%    \end{macrocode}

%\iffalse
%</samplepart4>
%\fi
%
% %%%%%%%%%%%%%%%%%%%%%%%%%%%%%%%%%%%%%%
% \paragraph{Forwarding for a Complete Draft.}
%
% The following forwarding file |cdocsdrf.tex|
% compiles the main document in draft mode:
%\iffalse
%<*sampledraft>
%\fi
%    \begin{macrocode}
\def\version{draft}
\input{childdoc.def}
\childdocforward{cdocsamp}
%    \end{macrocode}

%\iffalse
%</sampledraft>
%\fi
%
% %%%%%%%%%%%%%%%%%%%%%%%%%%%%%%%%%%%%%%
% \paragraph{Forwarding for Final Version of the Chapters.}
%
% The following forwarding files |cdocsfn1.tex| and |cdocsfn2.tex|
% (with identical content)
% compile the final versions of the child documents
% |cdocsch1.tex| and |cdocsch2.tex|, respectively:
%\iffalse
%<*samplefinal>
%\fi
%    \begin{macrocode}
\def\version{final}
\input{childdoc.def}
\childdocforwardprefix[cdocsamp]{cdocsfn}{cdocsch}
%    \end{macrocode}

%\iffalse
%</samplefinal>
%\fi
%
% %%%%%%%%%%%%%%%%%%%%%%%%%%%%%%%%%%%%%%
% \paragraph{Command Line Processing.}
%
% The following three command lines generate the output files
% |cdocscld|, |cdocscl1| and |cdocscl2|
% which should be identical to
% |cdocsdrf|, |cdocsch1| and |cdocsfn2|, respectively:
% \begin{center}
% \begin{tabular}{l}
% |latex -jobname cdocscld \|\\
% |  "\def\version{draft}\input{childdoc.def}\childdocforward{cdocsamp}"|\\
% |latex -jobname cdocscl1 \|\\
% |  "\input{childdoc.def}\childdocforward[cdocsamp]{cdocsch1}"|\\
% |latex -jobname cdocscl2 \|\\
% |  "\def\version{final}\input{childdoc.def}\childdocforward{cdocsch2}"|
% \end{tabular}
% \end{center}
% Note that the trailing backslash on each first line
% merely continues the input to the second line
% (for convenient cut ant paste).
% Furthermore, the command |latex| can be replaced by any
% of its alternative versions such as |pdflatex|.
%
% %%%%%%%%%%%%%%%%%%%%%%%%%%%%%%%%%%%%%%%%%%%%%%%%%%%%%%%%%%%%%%%%%%%%%%%%%%%%%%
% %%%%%%%%%%%%%%%%%%%%%%%%%%%%%%%%%%%%%%%%%%%%%%%%%%%%%%%%%%%%%%%%%%%%%%%%%%%%%%
% \section{Implementation}
%\iffalse
%<*package>
%\fi
%
% This section describes the definitions file |childdoc.def|.

% The definitions cannot be loaded using |\usepackage| or |\RequirePackage|
% which has a mechanism to prevent loading a style file more than once.
% When loading the definitions by means of |\input|
% multiple instances have to be prevented manually:
%\iffalse
%This code needs to be before the `\ProvidesFile' directive
%which is defined at the beginning of this file.
%Therefore it is also placed there and commented out here.
%</package>
%<*discard>
%\fi
%    \begin{macrocode}
\ifdefined\childdocmain\endinput\fi
%    \end{macrocode}
%\iffalse
%</discard>
%<*package>
%\fi
%
% \macro{\ifchilddoc}
% \macro{\ifchilddocmanual}
% The conditional |\ifchilddoc| tells whether a
% child (true) or main (false) document is being compiled.
% The conditional |\ifchilddocmanual| tells whether
% the |\includeonly| mechanism is used (false) or
% the selection of child files must be performed manually (true).
% The definitions initialise to false:
%    \begin{macrocode}
\newif\ifchilddoc
\newif\ifchilddocmanual
%    \end{macrocode}

% \macro{\childdocname}
% \macro{\childdocjob}
% The macro |\childdocname| stores the name of the main document
% to be compiled. The macro |\childdocjob| stores the name of
% the document on which the \LaTeX{} compiler was originally invoked.
% The content of |\jobname| cannot be compared
% to filenames specified in the source due to different catcodes.
% The following code rescans |\jobname|, stores the result
% in |\childdocname| and saves a copy in |\childdocjob|:
%    \begin{macrocode}
\edef\childdocname{\scantokens\expandafter{\jobname\noexpand}}
\let\childdocjob\childdocname
%    \end{macrocode}

% \macro{\childdocdisable}
% The macro |\childdocdisable| prevents the main file
% from being processed more than once.
% At this stage, the main document command |\childdocmain|
% is assumed to be called once again where it should do nothing.
% Any subsequent call to it should prevent
% a secondary processing of the main document
% It overwrites the forwarding commands
% |\childdocof| and |\childdocforward|
% with empty macros to prevent further inclusions of the main document:
%    \begin{macrocode}
\newcommand{\childdocdisable}
{
  \renewcommand{\childdocmain}[1]{\renewcommand{\childdocmain}[1]{\endinput}}
  \renewcommand{\childdocof}[1]{}
  \renewcommand{\childdocby}[2][]{}
  \renewcommand{\childdocforward}[2][]{}
  \renewcommand{\childdocdisable}{}
}
%    \end{macrocode}

% \macro{\childdocmain}
% The macro |\childdocmain| is to be called at the top of the main file
% with nothing or the main filename (without extension) as argument.
% First, it breaks loops.
% If the argument is not empty and does not match |\childdocname|
% (which is set by the first inclusion of |childdoc.def|),
% |\ifchilddoc| is set to true, |\includeonly| is applied to the child file
% and |\jobname| is set to the main file
% (for proper handling of |.aux| files):
%    \begin{macrocode}
\newcommand{\childdocmain}[1]
{
  \childdocdisable\childdocmain{}
  \if?#1?\else
    \begingroup
      \def\childdoctmp{#1}
      \ifx\childdoctmp\childdocname
        \def\childdoctmp{}
      \else
        \def\childdoctmp
        {
          \childdoctrue
          \includeonly{\childdocname}
          \def\childdocjob{#1}
          \def\jobname{#1}
        }
      \fi
      \expandafter
    \endgroup
    \childdoctmp
  \fi
}
%    \end{macrocode}

% \macro{\childdocof}
% The command |\childdocof| redirects
% compilation to the main file |#1|.
%    \begin{macrocode}
\newcommand{\childdocof}[1]
{
  \childdocdisable
  \childdoctrue
  \includeonly{\childdocname}
  \def\jobname{#1}
  \def\childdocjob{#1}
  \input{#1}
}
%    \end{macrocode}

% \macro{\childdocby}
% The command |\childdocby| ....
%    \begin{macrocode}
\newcommand{\childdocby}[2][]
{
  \childdocdisable
  \childdoctrue
  \childdocmanualtrue
  \if?#1?\else
    \def\jobname{#2}
  \fi
  \def\childdocjob{#2}
  \input{#2}
  \endinput
}
%    \end{macrocode}

% \macro{\childdocforward}
% The command |\childdocforward| redirects
% compilation to the main file or
% (if the optional argument is given) a child file.
% Parameters are set as if the main file
% or a child file starting with |\childdocof| was compiled.
% Then compilation is handed over to the main file:
%    \begin{macrocode}
\newcommand{\childdocforward}[2][]
{
  \begingroup
    \if?#1?
      \def\childdoctmp
      {
        \def\childdocname{#2}
        \def\childdocjob{#2}
        \def\jobname{#2}
        \input{#2}
        \endinput
      }
    \else
      \def\childdoctmp
      {
        \childdocdisable
        \def\childdocname{#2}
        \childdoctrue
        \includeonly{#2}
        \def\childdocjob{#1}
        \def\jobname{#1}
        \input{#1}
        \endinput
      }
    \fi
    \expandafter
  \endgroup
  \childdoctmp
}
%    \end{macrocode}

% \macro{\childdocforwardprefix}
% The command |\childdocforwardprefix| redirects
% compilation to the main or a child file by means of a pattern.
% The prefix |#1| in the current filename is replaced by |#2|
% and the suffix of the current filename is kept
% (it is assumed that the filename does not contain the substring `|~~~|'
% which is used as a delimiter).
% Compilation is handed over to the new file by |\childdocforward|:
%    \begin{macrocode}
\newcommand{\childdocforwardprefix}[3][]
{
  \begingroup
    \def\childdocextract #2##1~~~{\def\childdoctmp{\childdocforward[#1]{#3##1}}}
    \expandafter\childdocextract\childdocname~~~
    \expandafter
  \endgroup
  \childdoctmp
}
%    \end{macrocode}

% \macro{\childdoc}
% The deprecated macro |\childdoc| is a legacy version of |\childdocmain|:
%    \begin{macrocode}
\newcommand{\childdoc}{\childdocmain}
%    \end{macrocode}

% \macro{\childdocredirect}
% The deprecated macro |\childdocredirect| is a legacy version
% of |\childdocforward| and |\childdocforwardprefix|:
%    \begin{macrocode}
\newcommand{\childdocredirect}[2][]
{
  \begingroup
    \if?#1?
      \def\childdoctmp{\childdocforward{#2}}
    \else
      \def\childdoctmp{\childdocforwardprefix{#1}{#2}}
    \fi
    \expandafter
  \endgroup
  \childdoctmp
}
%    \end{macrocode}

%\iffalse
%</package>
%\fi
%
\endinput
|\\
|\childdocmain{|\textit{main}|}|\\
\end{tabular}
\end{center}
%
If |\jobname| does not match the argument \textit{main} of |\childdocmain|,
it is assumed that |\jobname| points to the child file to be compiled.
When using |\childdocmain| with the main file specified as argument,
it suffices to start a child file
with just |\input{|\textit{main}|}|
without loading of the package and using |\childdocof|.
If instead all processing is done
with the appropriate \textsf{childdoc} directives,
the argument of \textit{main} of |\childdocmain| can be empty.

An alternative version of the command line processing described
in \secref{sec:commandline} using the detection mechanism reads:
%
\begin{center}
|... -jobname "|\textit{target}|" "|[\textit{flags}]%
[|\def\jobname{|\textit{dest}|}|]|\input{|\textit{main}|}"|
\end{center}

%%%%%%%%%%%%%%%%%%%%%%%%%%%%%%%%%%%%%%%%%%%%%%%%%%%%%%%%%%%%%%%%%%%%%%%%%%%%%%%%
\subsection{Manual Code}
\label{sec:manual}

In case one cannot be certain whether the definitions file |childdoc.def|
is installed on the target \TeX{} distribution
and one prefers not to ship it,
it is conceivable to paste a few relevant commands into the sources.

To that end, drop all statements |% \iffalse
%
% childdoc.dtx Copyright (C) 2017-2018 Niklas Beisert
%
% This work may be distributed and/or modified under the
% conditions of the LaTeX Project Public License, either version 1.3
% of this license or (at your option) any later version.
% The latest version of this license is in
%   http://www.latex-project.org/lppl.txt
% and version 1.3 or later is part of all distributions of LaTeX
% version 2005/12/01 or later.
%
% This work has the LPPL maintenance status `maintained'.
%
% The Current Maintainer of this work is Niklas Beisert.
%
% This work consists of the files childdoc.dtx and childdoc.ins
% and the derived files childdoc.def and cdocsamp.tex with
% cdocsch1.tex, cdocsch2.tex, cdocsdrf.tex, cdocsfn1.tex, cdocsfn2.tex.
%
%<package>\ifdefined\childdocmain\endinput\fi
%<package>\ProvidesFile{childdoc.def}[2018/12/30 v2.0 child document driver]
%<samplemain>\ProvidesFile{cdocsamp.tex}[2018/12/30 v2.0 sample for childdoc]
%<*driver>
%\ProvidesFile{childdoc.drv}[2018/12/30 v2.0 childdoc reference manual file]
\PassOptionsToClass{10pt,a4paper}{article}
\documentclass{ltxdoc}

\usepackage[margin=35mm]{geometry}
\usepackage{hyperref}
\usepackage{hyperxmp}
\usepackage[usenames]{color}

\hypersetup{colorlinks=true}
\hypersetup{pdfstartview=FitH}
\hypersetup{pdfpagemode=UseNone}
\hypersetup{pdfsource={}}
\hypersetup{pdflang={en-UK}}
\hypersetup{pdfcopyright={Copyright 2017-2018 Niklas Beisert.
  This work may be distributed and/or modified under the
  conditions of the LaTeX Project Public License, either version 1.3
  of this license or (at your option) any later version.}}
\hypersetup{pdflicenseurl={http://www.latex-project.org/lppl.txt}}
\hypersetup{pdfcontactaddress={ETH Zurich, ITP, HIT K,
  Wolfgang-Pauli-Strasse 27}}
\hypersetup{pdfcontactpostcode={8093}}
\hypersetup{pdfcontactcity={Zurich}}
\hypersetup{pdfcontactcountry={Switzerland}}
\hypersetup{pdfcontactemail={nbeisert@itp.phys.ethz.ch}}
\hypersetup{pdfcontacturl={http://people.phys.ethz.ch/\xmptilde nbeisert/}}

\newcommand{\secref}[1]{\hyperref[#1]{section \ref*{#1}}}

\parskip1ex
\parindent0pt
\let\olditemize\itemize
\def\itemize{\olditemize\parskip0pt}

\begin{document}

\title{The \textsf{childdoc} Package}
\hypersetup{pdftitle={The childdoc Package}}
\author{Niklas Beisert\\[2ex]
  Institut f\"ur Theoretische Physik\\
  Eidgen\"ossische Technische Hochschule Z\"urich\\
  Wolfgang-Pauli-Strasse 27, 8093 Z\"urich, Switzerland\\[1ex]
  \href{mailto:nbeisert@itp.phys.ethz.ch}
  {\texttt{nbeisert@itp.phys.ethz.ch}}}
\hypersetup{pdfauthor={Niklas Beisert}}
\hypersetup{pdfsubject={Manual for the LaTeX2e Package childdoc}}
\date{30 December 2018, \textsf{v2.0}}
\maketitle

\begin{abstract}\noindent
\textsf{childdoc} is a \LaTeXe{} package
that enables the direct compilation
of document sections included by |\include|
to individual files.
\end{abstract}

\begingroup
\parskip0ex
\tableofcontents
\endgroup

%%%%%%%%%%%%%%%%%%%%%%%%%%%%%%%%%%%%%%%%%%%%%%%%%%%%%%%%%%%%%%%%%%%%%%%%%%%%%%%%
%%%%%%%%%%%%%%%%%%%%%%%%%%%%%%%%%%%%%%%%%%%%%%%%%%%%%%%%%%%%%%%%%%%%%%%%%%%%%%%%
\section{Introduction}

\LaTeX{} provides a mechanism to structure a large document (such as a book)
into a main file and several child files (containing the chapters)
using the |\include| command.
This mechanism is beneficial for documents
which span hundreds of pages in order to
make the source file(s) more manageable.
Moreover, compilation can be restricted to
selected child files by means of the |\includeonly| command.
The latter feature can be used to reduce the compilation time while editing
(this was significantly more useful in the earlier days of \LaTeX{})
or to generate a smaller document which is easier to navigate.
Another application of |\includeonly| is to generate
documents consisting of selected parts of the complete document.

However, there are a few drawbacks of the plain |\include| mechanism:
\begin{itemize}
\item
The child files cannot be compiled on their own,
they can only be compiled via the main file.
A naive editing environment
(such as a text editor with an option
to have the current file processed by \LaTeX)
may require one to switch to the main file before compiling;
attempting to compile the child file produces errors.
\item
The main file must be modified (each time)
to adjust the |\includeonly| command
to the present needs. This easily leaves the main file in a messy state.
\item
The generated document will always carry the filename
of the main document. This is inconvenient if
several child files are to be compiled and
to be kept for distribution.
\end{itemize}

The present package provides a simple interface
to make child files individually compilable by \LaTeX{}.
Compiling a child file then has the same effect as compiling
the main file with an |\includeonly| command
to select the appropriate child.
Moreover the generated document will carry the name of the child
rather than the main file.
This resolves all three above issues.

This feature is meant to make the editing of books,
thesis documents and lecture notes somewhat more convenient.
However, the package can also be used efficiently for
composing a series of documents (such as exercise sheets)
which are typically distributed individually.
It then assists the author in generating the individual documents
(potentially in different versions)
as well as a document containing the collected series.
Another application is in developing style files
or other kinds of included material
where compilation of the style file could redirect
to a sample or test file.

%%%%%%%%%%%%%%%%%%%%%%%%%%%%%%%%%%%%%%%%%%%%%%%%%%%%%%%%%%%%%%%%%%%%%%%%%%%%%%%%
%%%%%%%%%%%%%%%%%%%%%%%%%%%%%%%%%%%%%%%%%%%%%%%%%%%%%%%%%%%%%%%%%%%%%%%%%%%%%%%%
\section{Usage}

First of all, the package \textsf{childdoc} is \emph{not} a standard
\LaTeXe{} |.sty| style file! Therefore it needs to be invoked in
a non-standard way.

%%%%%%%%%%%%%%%%%%%%%%%%%%%%%%%%%%%%%%%%%%%%%%%%%%%%%%%%%%%%%%%%%%%%%%%%%%%%%%%%
\subsection{Included Files}
\label{sec:include}

%%%%%%%%%%%%%%%%%%%%%%%%%%%%%%%%%%%%%%%%
\DescribeMacro{\childdocmain}
To use the package, add the commands
\begin{center}
\begin{tabular}{l}
|\input{childdoc.def}|\\
|\childdocmain{}|\\
\end{tabular}
\end{center}
at the very top of the main \LaTeX{} file,
in particular \emph{before} the |\documentclass| statement!
The argument of |\childdocmain| should be left empty
(but it must be present).

%%%%%%%%%%%%%%%%%%%%%%%%%%%%%%%%%%%%%%%%
\DescribeMacro{\childdocof}
Furthermore, add the commands
\begin{center}
\begin{tabular}{l}
|\input{childdoc.def}|\\
|\childdocof{|\textit{main}|}|\\
\end{tabular}
\end{center}
at the top of every child file \textit{child}
which is included by |\include{|\textit{child}|}|
from within the main file
(or at least for those files to be compiled individually).
The argument \textit{main} must be the filename of the main file.

There are a couple of
considerations in setting up the main and child documents:

%%%%%%%%%%%%%%%%%%%%%%%%%%%%%%%%%%%%%%%%
\paragraph{Restrictions.}

Please note the following restrictions:
\begin{itemize}
\item
|\childdocmain| must be called with one argument \textit{main}
to ensure compatibility with earlier version of the package.
It must either be empty (|\childdocmain{}|)
or precisely match the filename of the main file in which it is specified.
See \secref{sec:detection} for further information.
\item
The filename \textit{main} must be specified without the |.tex| extension.
\item
The filename \textit{main} is case sensitive
(even in case-insensitive file systems)
due to internal string comparison.
\item
The argument \textit{main} should be fully expanded, it cannot be a macro.
\item
Subdirectories and special characters should be avoided in filenames.
\item
The command |\childdocmain{|\textit{main}|}| must be followed by a whitespace.
It should not be followed immediately by another command
or by a comment mark `|%|'.
This is because the \TeX{} parser reads the token immediately following
the argument of |\childdocmain| and puts it
at the beginning of every child section;
however, a white\-space is ignored.
\end{itemize}

%%%%%%%%%%%%%%%%%%%%%%%%%%%%%%%%%%%%%%%%
\paragraph{Content of Main File.}

It is advisable to place all content in the child files included by |\include|.
Any output contained in the main file will appear in all child documents
unless suppressed manually;
it cannot be suppressed automatically by the |\includeonly| directive
and thus should normally be avoided.
A method to include some content in the main file
by means of conditional processing is described in \secref{sec:conditional}.

%%%%%%%%%%%%%%%%%%%%%%%%%%%%%%%%%%%%%%%%
\paragraph{Page Numbering.}

When only a part of the document is compiled,
the appropriate numbering of pages
(as well as other status parameters)
is determined from the |.aux| files.
The latter contain information from previous passes.
However this information needs to propagate through
all intermediate child documents.
Therefore the page numbering in child documents may well
be inconsistent until the complete document is compiled at least once.

A useful (if unconventional) way to always ensure a consistent
page numbering is to restart the numbering in each child document
and denote the pages by `\textit{child}|.|\textit{page}'
where \textit{child} represents the chapter/section number of the child file.
This can be achieved by the command
|\numberwithin{page}{|\textit{child}|}|
of the \textsf{amsmath} package
where \textit{child} can be |chapter| or |section|
depending on the chosen structuring.
Alternatively, one can modify the macro |\thepage| appropriately
and reset the counter |page| at the start of each child file.

%%%%%%%%%%%%%%%%%%%%%%%%%%%%%%%%%%%%%%%%%%%%%%%%%%%%%%%%%%%%%%%%%%%%%%%%%%%%%%%%
\subsection{Conditional Processing}
\label{sec:conditional}

The package provides a mechanism to compile different versions
of a document. To customise the versions further some conditional processing
can come in handy to distinguish which version is being compiled.
The package provides two macros to describe the compilation context:

%%%%%%%%%%%%%%%%%%%%%%%%%%%%%%%%%%%%%%%%
\DescribeMacro{\ifchilddoc}
The conditional |\ifchilddoc| distinguishes between the compilation of
child documents and the main document:
%
\begin{center}
|\ifchilddoc |\textit{child-code}| |[|\||else |\textit{main-code}]| \||fi|
\end{center}

%%%%%%%%%%%%%%%%%%%%%%%%%%%%%%%%%%%%%%%%
\DescribeMacro{\childdocname}
\DescribeMacro{\childdocjob}
The macro |\childdocname| contains the filename (without extension)
of the main or child file being processed.
Note that |\childdocjob| will always contain the name of the main file.

%%%%%%%%%%%%%%%%%%%%%%%%%%%%%%%%%%%%%%%%
\paragraph{Title Page.}

Conditional processing can be used to include a title or banner page
in the main document when proper precautions are taken.
Importantly, the code in the main file should ensure that the page counter
(as well as other status parameters which are stored in the |.aux| files)
takes the same value after the conditional processing.
Otherwise the page numbers may take divergent values
depending on which part is compiled.

For example, a title page could be declared by:
%
\begin{center}
\begin{tabular}{l}
|\ifchilddoc\||else|\\
|\addtocounter{page}{-1}|\\
\textit{code for title page}\\
|\newpage|\\
|\||fi|
\end{tabular}
\end{center}
%
A banner page for the child documents can be generated by:
%
\begin{center}
\begin{tabular}{l}
|\ifchilddoc|\\
|\addtocounter{page}{-1}|\\
\textit{code for banner page}\\
|\newpage|\\
|\||fi|
\end{tabular}
\end{center}
%
Here one could write a message such as:
\begin{center}
|This is the part \childdocname{} of \childdocjob{}.|
\end{center}

%%%%%%%%%%%%%%%%%%%%%%%%%%%%%%%%%%%%%%%%%%%%%%%%%%%%%%%%%%%%%%%%%%%%%%%%%%%%%%%%
\subsection{Flags}
\label{sec:flags}

The package makes it easy to generate different versions
of the main or child documents.
To this end compilation flags can be defined
and assigned different default values.
They will be particularly useful in conjunction
with the forwarding mechanism described in \secref{sec:forward}.

For example, it may be useful to have a flag |\version|
which can be set to |draft| or |final|.
The document source will contain some conditional code
depending on the value of |\version|.
Suppose further, the flag should default to |final| for the main file
and to |draft| for child files
which is a natural assignment for editing the document.
This is achieved by placing the following code
in the preamble of the main document
(below the |\childdocmain| directive):
%
\begin{center}
\begin{tabular}{l}
|\ifchilddoc|\\
|\providecommand{\version}{draft}|\\
|\||else|\\
|\providecommand{\version}{final}|\\
|\||fi|
\end{tabular}
\end{center}
%
The definition by |\providecommand| makes sure
that previous definitions are not overwritten.
Further statements |\providecommand{\version}{...}|
can thus be added before the above code to override it.

For the main file, one might add a line
(between |\childdocmain| and the above block)
%
\begin{center}
|%\ifchilddoc\||else\providecommand{\version}{draft}\||fi|
\end{center}
%
which can be uncommented to produce a draft version.
Likewise one can add a line to the very top of a child file
(above the |\childdocof{|\textit{main}|}| directive)
%
\begin{center}
|%\providecommand{\version}{final}|
\end{center}
%
which can be uncommented to produce the final version of this child document.

%%%%%%%%%%%%%%%%%%%%%%%%%%%%%%%%%%%%%%%%%%%%%%%%%%%%%%%%%%%%%%%%%%%%%%%%%%%%%%%%
\subsection{Forwarding}
\label{sec:forward}

Different versions of the main or child documents
using compilation flags as described in \secref{sec:flags}
can be (permanently) stored in different files
for convenient compilation, viewing and distribution.
To this end, the package defines a command
to pass on compilation to a different file:

%%%%%%%%%%%%%%%%%%%%%%%%%%%%%%%%%%%%%%%%
\DescribeMacro{\childdocforward}
The command |\childdocforward| redirects processing to
another source file:
%
\begin{center}
\begin{tabular}{l}
|\input{childdoc.def}|\\
|\childdocforward[|\textit{main}|]{|\textit{dest}|}|\\
\end{tabular}
\end{center}
%
The argument \textit{dest} is the destination file
(without extension).
It should be the main file or one of the child files.
Note that further \textsf{childdoc} directives
such as |\childdocof| and |\childdocforward|
in the indicated file will be processed in this form.
The optional argument \textit{main}
passes on directly to the main file \textit{main}
while pretending to compile the child \textit{dest}.
This form behaves as if \textit{dest}
issues |\childdocof{|\textit{main}|}| right away,
and no further \textsf{childdoc} directives will be processed.

%%%%%%%%%%%%%%%%%%%%%%%%%%%%%%%%%%%%%%%%
\DescribeMacro{\...prefix}
In the alternative form |\childdocforwardprefix|,
%
\begin{center}
\begin{tabular}{l}
|\input{childdoc.def}|\\
|\childdocforwardprefix[|\textit{main}|]{|\textit{prefix}|}{|\textit{dest}|}|
\end{tabular}
\end{center}
%
the destination file is determined by a pattern
depending on the current file:
To make this work, the current file must be called
`{\textit{prefix}\hspace{0.2em}\textit{suffix}}'
with \textit{prefix} matching precisely the argument.
Processing is then passed on to the file
`{\textit{dest}\hspace{0.2em}\textit{suffix}}'.
Surely, the same effect is achieved by
directly specifying the
argument `{\textit{dest}\hspace{0.2em}\textit{suffix}}'
in the first form.
However, that requires to set up a different file
for each child. With the alternative form of the command
all these files can have exactly the same content
which simplifies setting them up and maintaining them.

For example, the following file |draft.tex|
with a compilation flag |\version| as described in \secref{sec:flags}
compiles the main document as a draft:
%
\begin{center}
\begin{tabular}{l}
|\def\version{draft}|\\
|\input{childdoc.def}|\\
|\childdocforward{|\textit{main}|}|
\end{tabular}
\end{center}
%
Likewise, the following files |final|\textit{nn}|.tex|
compile the final version of the child document
|child|\textit{nn}|.tex|:
%
\begin{center}
\begin{tabular}{l}
|\def\version{final}|\\
|\input{childdoc.def}|\\
|\childdocforwardprefix{final}{child}|
\end{tabular}
\end{center}
%

Note that when several versions of a main file and/or of each child file
are to be generated, it may be convenient to set up a |Makefile| or
shell script to automatise the process.

%%%%%%%%%%%%%%%%%%%%%%%%%%%%%%%%%%%%%%%%%%%%%%%%%%%%%%%%%%%%%%%%%%%%%%%%%%%%%%%%
\subsection{Command Line Processing}
\label{sec:commandline}

The effect of redirection files can also be achieved by invoking
the \LaTeX{} compiler with a more elaborate command line.
Most conveniently this should be done as part
of a shell script or a |Makefile|.

When using \textsf{childdoc} in the main file, the following
command lines effectively perform a redirection
(note that depending on the shell being used,
backslashes may have to be doubled: `|\|' $\to$ `|\\|'):
%
\begin{center}
|... -jobname "|\textit{target}|" |\\|"|[\textit{flags}]%
|\input{childdoc.def}\childdocforward[|\textit{main}|]{|\textit{dest}|}"|
\end{center}
%
Here \textit{target} is the name of the output file,
\textit{main} is the name of the main file
and \textit{dest} is the name of the main or child file to be processed
(all filenames without extensions).
The optional argument \textit{main} can be omitted
if \textit{main} matches \textit{dest}.
Optionally, compilation \textit{flags} can be defined via |\def| commands.
This command line makes the \TeX{} engine believe
it is compiling the file \textit{target}
whose content is specified as the latter parameter.
The provided code then forwards the processing to
\textit{main} or \textit{dest} as described in \secref{sec:forward}.

%%%%%%%%%%%%%%%%%%%%%%%%%%%%%%%%%%%%%%%%%%%%%%%%%%%%%%%%%%%%%%%%%%%%%%%%%%%%%%%%
\subsection{Include by Input}
\label{sec:input}

Including child documents by |\include| has some restrictions by design.
Most notably, the content of a child document always occupies
its own set of pages; pages cannot be shared between child documents.
Usually, this behaviour makes perfect sense
because each child document contain an essential part of the document.
However, in some situations it may be desirable to compose
a document from a collection of parts
without having mandatory page breaks between then.
For this case, the package
provides a mechanism to include parts
by |\input| which can also be processed individually.
However, by construction this mechanism
requires manual handling of the content to be output.

%%%%%%%%%%%%%%%%%%%%%%%%%%%%%%%%%%%%%%%%
\DescribeMacro{\ifchilddocmanual}
The main file should be prepared as usual, see \secref{sec:include}.
However, the document body must make a distinction
between processing of an individual part and of the main document, e.g.:
%
\begin{center}
\begin{tabular}{l}
|\ifchilddocmanual|\\
|\input{\childdocname}|\\
|\||else|\\
\textit{document body with }|\input{|\textit{part}|}|\\
|\||fi|
\end{tabular}
\end{center}
%
The conditional |\ifchilddocmanual| is true whenever
a part to be included by |\input| is being compiled,
and the name of the part is stored in |\childdocname|.

%%%%%%%%%%%%%%%%%%%%%%%%%%%%%%%%%%%%%%%%
\DescribeMacro{\childdocby}
Each part to be included by |\input| should start with:
%
\begin{center}
\begin{tabular}{l}
|\input{childdoc.def}|\\
|\childdocby{|\textit{main}|}|\\
\end{tabular}
\end{center}
%
The directive |\childdocby| is similar to |\childdocof|
described in \secref{sec:include},
but the subsequent selection of content must be done manually.
To that end, both |\ifchilddoc| and |\ifchilddocmanual|
will be true upon processing of a part,
and the name of the part is stored in |\childdocname|.
Note that |\jobname| will be set to the filename of the current part
so that each part receives an individual |.aux| file
that does not interfere with the |.aux| file(s) of the main document.
This behaviour can be altered by the alternative form
|\childdocby[*]{|\textit{main}|}| (with a non-empty optional argument)
which uses the |.aux| file of the main document
by setting |\jobname| to \textit{main}.

%%%%%%%%%%%%%%%%%%%%%%%%%%%%%%%%%%%%%%%%%%%%%%%%%%%%%%%%%%%%%%%%%%%%%%%%%%%%%%%%
\subsection{Driver Development}
\label{sec:driver}

The \textsf{childdoc} mechanism can also be use for the development
of definition files such as \LaTeX{} styles or classes.
This case differs from the above setup with multiple parts
included by |\include| in that no |\includeonly| should be invoked.
This can be achieved by starting the include file
(before |\ProvidesPackage|) with:
%
\begin{center}
\begin{tabular}{l}
|\input{childdoc.def}|\\
|\childdocforward{|\textit{main}|}|\\
\end{tabular}
\end{center}
%
or alternatively with:
%
\begin{center}
\begin{tabular}{l}
|\input{childdoc.def}|\\
|\childdocby{|\textit{main}|}|\\
\end{tabular}
\end{center}
%
Both forms have slightly different effects as described above.
The main file is prepared as usual, see \secref{sec:include}.

%%%%%%%%%%%%%%%%%%%%%%%%%%%%%%%%%%%%%%%%%%%%%%%%%%%%%%%%%%%%%%%%%%%%%%%%%%%%%%%%
\subsection{Legacy Detection}
\label{sec:detection}

The directive |\childdocmain| in the main file can detect
whether the complete document or merely a child is to be compiled
even without using the directive |\childdocof|.
This method is deprecated because it is less robust
and there is no compelling reason to use it;
it is merely provided for backward compatibility
and it may be removed in future versions.

If the detection mechanism is to be used,
it is mandatory to correctly specify
the filename of the main file as the argument of |\childdocmain|:
%
\begin{center}
\begin{tabular}{l}
|\input{childdoc.def}|\\
|\childdocmain{|\textit{main}|}|\\
\end{tabular}
\end{center}
%
If |\jobname| does not match the argument \textit{main} of |\childdocmain|,
it is assumed that |\jobname| points to the child file to be compiled.
When using |\childdocmain| with the main file specified as argument,
it suffices to start a child file
with just |\input{|\textit{main}|}|
without loading of the package and using |\childdocof|.
If instead all processing is done
with the appropriate \textsf{childdoc} directives,
the argument of \textit{main} of |\childdocmain| can be empty.

An alternative version of the command line processing described
in \secref{sec:commandline} using the detection mechanism reads:
%
\begin{center}
|... -jobname "|\textit{target}|" "|[\textit{flags}]%
[|\def\jobname{|\textit{dest}|}|]|\input{|\textit{main}|}"|
\end{center}

%%%%%%%%%%%%%%%%%%%%%%%%%%%%%%%%%%%%%%%%%%%%%%%%%%%%%%%%%%%%%%%%%%%%%%%%%%%%%%%%
\subsection{Manual Code}
\label{sec:manual}

In case one cannot be certain whether the definitions file |childdoc.def|
is installed on the target \TeX{} distribution
and one prefers not to ship it,
it is conceivable to paste a few relevant commands into the sources.

To that end, drop all statements |\input{childdoc.def}|
and perform the replacements as outlined below.
Instead of |\childdocmain{|\textit{main}|}| add the following code
to the top of the main file:
%
\begin{center}
\begin{tabular}{l}
|\||ifdefined\childdocname\endinput\||fi\newif\ifchilddoc|\\
|\edef\childdocname{\scantokens\expandafter{\jobname\noexpand}}|\\
|\def\childdocmain{|\textit{main}|}\||ifx\childdocmain\childdocname\||else|\\
|\childdoctrue\includeonly{\childdocname}\let\jobname\childdocmain\||fi|\\
\end{tabular}
\end{center}
%
Instead of |\childdocof{|\textit{main}|}| just include the main file
at the top of each child file:
%
\begin{center}
|\input{|\textit{main}|}|
\end{center}
%
A simple redirection |\childdocforward{|\textit{dest}|}| is achieved by:
%
\begin{center}
|\def\jobname{|\textit{dest}|}\input{\jobname}|
\end{center}
%
The redirection with prefix
|\childdocforwardprefix[|\textit{prefix}|]{|\textit{dest}|}|
is accomplished by:
%
\begin{center}
\begin{tabular}{l}
|{\edef\jobname{\scantokens\expandafter{\jobname\noexpand}}|\\
|\def\redirectjob |\textit{prefix}|#1~~~{\gdef\jobname{|\textit{dest}|#1}}|\\
|\expandafter\redirectjob\jobname~~~}\input{\jobname}|
\end{tabular}
\end{center}

In an alternative approach,
child documents can be compiled by a specific command line
without additional code or specific definitions:
%
\begin{center}
|... -jobname "|\textit{target}|" "|[\textit{flags}]%
|\includeonly{|\textit{dest}|}\input{|\textit{main}|}"|
\end{center}
%

%%%%%%%%%%%%%%%%%%%%%%%%%%%%%%%%%%%%%%%%%%%%%%%%%%%%%%%%%%%%%%%%%%%%%%%%%%%%%%%%
%%%%%%%%%%%%%%%%%%%%%%%%%%%%%%%%%%%%%%%%%%%%%%%%%%%%%%%%%%%%%%%%%%%%%%%%%%%%%%%%
\section{Information}

%%%%%%%%%%%%%%%%%%%%%%%%%%%%%%%%%%%%%%%%%%%%%%%%%%%%%%%%%%%%%%%%%%%%%%%%%%%%%%%%
\subsection{Copyright}

Copyright \copyright{} 2017--2018 Niklas Beisert

This work may be distributed and/or modified under the
conditions of the \LaTeX{} Project Public License, either version 1.3
of this license or (at your option) any later version.
The latest version of this license is in
  \url{http://www.latex-project.org/lppl.txt}
and version 1.3 or later is part of all distributions of \LaTeX{}
version 2005/12/01 or later.

This work has the LPPL maintenance status `maintained'.

The Current Maintainer of this work is Niklas Beisert.

This work consists of the files |README.txt|, |childdoc.ins| and |childdoc.dtx|
as well as the derived files |childdoc.def|, |cdocsamp.tex|
with |cdocsch1.tex|, |cdocsch2.tex|, |cdocspt3.tex|, |cdocspt4.tex|,
|cdocsdrf.tex|, |cdocsfn1.tex|, |cdocsfn2.tex|
as well as |childdoc.pdf|.

%%%%%%%%%%%%%%%%%%%%%%%%%%%%%%%%%%%%%%%%%%%%%%%%%%%%%%%%%%%%%%%%%%%%%%%%%%%%%%%%
\subsection{Files and Installation}

The package consists of the files:
%
\begin{center}
\begin{tabular}{ll}
    |README.txt|   & readme file \\
    |childdoc.ins| & installation file \\
    |childdoc.dtx| & source file \\
    |childdoc.def| & definition file \\
    |cdocsamp.tex| & sample main file \\
    |cdocsch1.tex| & sample include file \\
    |cdocsch2.tex| & sample include file \\
    |cdocspt3.tex| & sample part file \\
    |cdocspt4.tex| & sample part file \\
    |cdocsdrf.tex| & sample redirection file \\
    |cdocsfn1.tex| & sample redirection file \\
    |cdocsfn2.tex| & sample redirection file \\
    |childdoc.pdf| & manual
\end{tabular}
\end{center}
%
The distribution consists of the files
|README.txt|, |childdoc.ins| and |childdoc.dtx|.
%
\begin{itemize}
\item
Run (pdf)\LaTeX{} on |childdoc.dtx|
to compile the manual |childdoc.pdf| (this file).
\item
Run \LaTeX{} on |childdoc.ins| to create the definitions file |childdoc.def|
and the sample |cdocsamp.tex| with include files
|cdocsch1.tex|, |cdocsch2.tex|, |cdocspt3.tex|, |cdocspt4.tex|,
|cdocsdrf.tex|, |cdocsfn1.tex|, |cdocsfn2.tex|.
Then copy the file |childdoc.def| to an appropriate directory of your \LaTeX{}
distribution, e.g.\ \textit{texmf-root}|/tex/latex/childdoc|.
\end{itemize}

%%%%%%%%%%%%%%%%%%%%%%%%%%%%%%%%%%%%%%%%%%%%%%%%%%%%%%%%%%%%%%%%%%%%%%%%%%%%%%%%
\subsection{Related CTAN Packages}

There are several other packages which offer a similar functionality:
%
\begin{itemize}
\item
The packages
\href{http://ctan.org/pkg/docmute}{\textsf{docmute}},
\href{http://ctan.org/pkg/includex}{\textsf{includex}} and
\href{http://ctan.org/pkg/standalone}{\textsf{standalone}}
provide commands to include only the document body of
a child file thus allowing both files to be compiled individually.
\item
The packages \href{http://ctan.org/pkg/subdocs}{\textsf{subdocs}}
and \href{http://ctan.org/pkg/subfiles}{\textsf{subfiles}}
provide structures in which the main and child documents can be
encapsulated and allowing them to be compiled individually.
The inclusion mechanism is different from the conventional |\include|.
\item
The package \href{http://ctan.org/pkg/combine}{\textsf{combine}}
is an elaborate solution to combine several documents into one.
\end{itemize}
%
See also the CTAN topic \href{http://ctan.org/topic/subdocs}{\textsf{subdocs}}
for further related packages.
The present package differs from the above solutions in that
a document structure constructed with the conventional |\include| mechanism
just needs two extra commands at the top of every file
such that all constituent files can be compiled individually.

%%%%%%%%%%%%%%%%%%%%%%%%%%%%%%%%%%%%%%%%%%%%%%%%%%%%%%%%%%%%%%%%%%%%%%%%%%%%%%%%
%\subsection{Feature Suggestions}
%
%The following is a list of features which may be useful for future
%versions of this package:
%%
%\begin{itemize}
%\item
%\ldots
%\end{itemize}

%%%%%%%%%%%%%%%%%%%%%%%%%%%%%%%%%%%%%%%%%%%%%%%%%%%%%%%%%%%%%%%%%%%%%%%%%%%%%%%%
\subsection{Revision History}

%%%%%%%%%%%%%%%%%%%%%%%%%%%%%%%%%%%%%%%%
\paragraph{v2.0:} 2018/12/30

\begin{itemize}
\item
immediate forward processing
\item
added |\childdocby| mechanism
\item
manual restructured
\end{itemize}

%%%%%%%%%%%%%%%%%%%%%%%%%%%%%%%%%%%%%%%%
\paragraph{v1.6:} 2018/01/17

\begin{itemize}
\item
application for development of include files
\item
corrections to manual
\end{itemize}

%%%%%%%%%%%%%%%%%%%%%%%%%%%%%%%%%%%%%%%%
\paragraph{v1.5:} 2017/05/21

\begin{itemize}
\item
more complete structuring introduced
\item
|\childdocof| introduced
\item
|\childdoc| renamed to |\childdocmain|
\item
|\childredirect| renamed to |\childdocforward| and |\childdocforwardprefix|
and functionality expanded
\end{itemize}

%%%%%%%%%%%%%%%%%%%%%%%%%%%%%%%%%%%%%%%%
\paragraph{v1.0:} 2017/04/27

\begin{itemize}
\item
manual and install package
\item
first version published on CTAN
\end{itemize}

%%%%%%%%%%%%%%%%%%%%%%%%%%%%%%%%%%%%%%%%
\paragraph{v0.6:} 2017/04/26

\begin{itemize}
\item
redirection mechanism added
\end{itemize}

%%%%%%%%%%%%%%%%%%%%%%%%%%%%%%%%%%%%%%%%
\paragraph{v0.5:} 2017/04/26

\begin{itemize}
\item
functionality in definition file
\end{itemize}


%%%%%%%%%%%%%%%%%%%%%%%%%%%%%%%%%%%%%%%%%%%%%%%%%%%%%%%%%%%%%%%%%%%%%%%%%%%%%%%%
%%%%%%%%%%%%%%%%%%%%%%%%%%%%%%%%%%%%%%%%%%%%%%%%%%%%%%%%%%%%%%%%%%%%%%%%%%%%%%%%
%%%%%%%%%%%%%%%%%%%%%%%%%%%%%%%%%%%%%%%%%%%%%%%%%%%%%%%%%%%%%%%%%%%%%%%%%%%%%%%%
\appendix

\settowidth\MacroIndent{\rmfamily\scriptsize 000\ }

 \DocInput{childdoc.dtx}

\end{document}
%</driver>
% \fi
%
% %%%%%%%%%%%%%%%%%%%%%%%%%%%%%%%%%%%%%%%%%%%%%%%%%%%%%%%%%%%%%%%%%%%%%%%%%%%%%%
% %%%%%%%%%%%%%%%%%%%%%%%%%%%%%%%%%%%%%%%%%%%%%%%%%%%%%%%%%%%%%%%%%%%%%%%%%%%%%%
% \section{Sample}
%\iffalse
%<*samplemain>
%\fi
%
% The following presents a sample document
% with two chapters, two parts, a title page,
% a compile flag as well as three forwarding files to set the flag.
% It consists of eight |.tex| files:
% \begin{center}
% \begin{tabular}{ll}
% |cdocsamp.tex|&main file\\
% |cdocsch1.tex|&include file for chapter 1\\
% |cdocsch2.tex|&include file for chapter 2\\
% |cdocspt3.tex|&include file for part 3\\
% |cdocspt4.tex|&include file for part 4\\
% |cdocsdrf.tex|&forwarding file for main file in draft mode\\
% |cdocsfi1.tex|&forwarding file for final version of chapter 1\\
% |cdocsfi2.tex|&forwarding file for final version of chapter 2\\
% \end{tabular}
% \end{center}
% Each of the eight files can be compiled directly by the \LaTeX{} compiler.
%
% %%%%%%%%%%%%%%%%%%%%%%%%%%%%%%%%%%%%%%
% \paragraph{Main File.}
%
% The main file is called |cdocsamp.tex|.
%
% Load the \textsf{childdoc} definitions and
% declare the filename for the main document:
%    \begin{macrocode}
\input{childdoc.def}
\childdocmain{}
%    \end{macrocode}

% Optional override for |\version| flag:
%    \begin{macrocode}
%%\ifchilddoc\else\providecommand{\version}{draft}\fi
%    \end{macrocode}

% Define the default values for the |\version| flag
% (|final| for the main file and |draft| for childs):
%    \begin{macrocode}
\ifchilddoc
\providecommand{\version}{draft}
\else
\providecommand{\version}{final}
\fi
%    \end{macrocode}

% Load the standard document class:
%    \begin{macrocode}
\documentclass[12pt]{article}
%    \end{macrocode}

% Start the document body:
%    \begin{macrocode}
\begin{document}
%    \end{macrocode}

% Declare a title page.
% Print title, part of document being processed and version flag:
%    \begin{macrocode}
\addtocounter{page}{-1}
\begin{center}
{\LARGE\bfseries{}childdoc example\par}
\vspace{1cm}
\ifchilddoc
\ifchilddocmanual part\else chapter\fi:
`\childdocname' of `\childdocjob'\par
\else
main document: `\childdocjob'\par
\fi
version: \version\par
\end{center}
\newpage
%    \end{macrocode}

% Manually include selected file,
% otherwise process as usual:
%    \begin{macrocode}
\ifchilddocmanual
\section*{part `\childdocname'}
\input{\childdocname}
\else
%    \end{macrocode}

% Include the two chapters:
%    \begin{macrocode}
\include{cdocsch1}
\include{cdocsch2}
%    \end{macrocode}

% Include the two parts unless only chapters should be displayed:
%    \begin{macrocode}
\ifchilddoc\else
\section{part three}
\input{cdocspt3}
\section{part four}
\input{cdocspt4}
\fi
%    \end{macrocode}

% Process as usual until here:
%    \begin{macrocode}
\fi
%    \end{macrocode}

% End of document body:
%    \begin{macrocode}
\end{document}
%    \end{macrocode}
%\iffalse
%</samplemain>
%\fi
%
% %%%%%%%%%%%%%%%%%%%%%%%%%%%%%%%%%%%%%%
% \paragraph{Chapter Include Files.}
%
% The include files are called |cdocsch1.tex| and |cdocsch2.tex|.
%
%\iffalse
%<*samplechap1|samplechap2>
%\fi

% Optional override for |\version| flag:
%    \begin{macrocode}
%%\providecommand{\version}{final}
%    \end{macrocode}

% Include the main document:
%    \begin{macrocode}
\input{childdoc.def}
\childdocof{cdocsamp}
%    \end{macrocode}

%\iffalse
%</samplechap1|samplechap2>
%\fi
%
%\iffalse
%<*samplechap1>
%\fi
% Some text for chapter 1:
%    \begin{macrocode}
\section{one}
some text in chapter one
%    \end{macrocode}

%\iffalse
%</samplechap1>
%\fi
% Some text for chapter 2:
%\iffalse
%<*samplechap2>
%\fi
%    \begin{macrocode}
\section{two}
more text in chapter two
%    \end{macrocode}

%\iffalse
%</samplechap2>
%\fi
%
% %%%%%%%%%%%%%%%%%%%%%%%%%%%%%%%%%%%%%%
% \paragraph{Part Include Files.}
%
% The include files are called |cdocspt3.tex| and |cdocspt4.tex|.
%
%\iffalse
%<*samplepart3|samplepart4>
%\fi

% Optional override for |\version| flag:
%    \begin{macrocode}
%%\providecommand{\version}{final}
%    \end{macrocode}

% Include the main document:
%    \begin{macrocode}
\input{childdoc.def}
\childdocby{cdocsamp}
%    \end{macrocode}

%\iffalse
%</samplepart3|samplepart4>
%\fi
%
%\iffalse
%<*samplepart3>
%\fi
% Some text for part 3:
%    \begin{macrocode}
some text in part three
%    \end{macrocode}

%\iffalse
%</samplepart3>
%\fi
% Some text for part 4:
%\iffalse
%<*samplepart4>
%\fi
%    \begin{macrocode}
more text in part four
%    \end{macrocode}

%\iffalse
%</samplepart4>
%\fi
%
% %%%%%%%%%%%%%%%%%%%%%%%%%%%%%%%%%%%%%%
% \paragraph{Forwarding for a Complete Draft.}
%
% The following forwarding file |cdocsdrf.tex|
% compiles the main document in draft mode:
%\iffalse
%<*sampledraft>
%\fi
%    \begin{macrocode}
\def\version{draft}
\input{childdoc.def}
\childdocforward{cdocsamp}
%    \end{macrocode}

%\iffalse
%</sampledraft>
%\fi
%
% %%%%%%%%%%%%%%%%%%%%%%%%%%%%%%%%%%%%%%
% \paragraph{Forwarding for Final Version of the Chapters.}
%
% The following forwarding files |cdocsfn1.tex| and |cdocsfn2.tex|
% (with identical content)
% compile the final versions of the child documents
% |cdocsch1.tex| and |cdocsch2.tex|, respectively:
%\iffalse
%<*samplefinal>
%\fi
%    \begin{macrocode}
\def\version{final}
\input{childdoc.def}
\childdocforwardprefix[cdocsamp]{cdocsfn}{cdocsch}
%    \end{macrocode}

%\iffalse
%</samplefinal>
%\fi
%
% %%%%%%%%%%%%%%%%%%%%%%%%%%%%%%%%%%%%%%
% \paragraph{Command Line Processing.}
%
% The following three command lines generate the output files
% |cdocscld|, |cdocscl1| and |cdocscl2|
% which should be identical to
% |cdocsdrf|, |cdocsch1| and |cdocsfn2|, respectively:
% \begin{center}
% \begin{tabular}{l}
% |latex -jobname cdocscld \|\\
% |  "\def\version{draft}\input{childdoc.def}\childdocforward{cdocsamp}"|\\
% |latex -jobname cdocscl1 \|\\
% |  "\input{childdoc.def}\childdocforward[cdocsamp]{cdocsch1}"|\\
% |latex -jobname cdocscl2 \|\\
% |  "\def\version{final}\input{childdoc.def}\childdocforward{cdocsch2}"|
% \end{tabular}
% \end{center}
% Note that the trailing backslash on each first line
% merely continues the input to the second line
% (for convenient cut ant paste).
% Furthermore, the command |latex| can be replaced by any
% of its alternative versions such as |pdflatex|.
%
% %%%%%%%%%%%%%%%%%%%%%%%%%%%%%%%%%%%%%%%%%%%%%%%%%%%%%%%%%%%%%%%%%%%%%%%%%%%%%%
% %%%%%%%%%%%%%%%%%%%%%%%%%%%%%%%%%%%%%%%%%%%%%%%%%%%%%%%%%%%%%%%%%%%%%%%%%%%%%%
% \section{Implementation}
%\iffalse
%<*package>
%\fi
%
% This section describes the definitions file |childdoc.def|.

% The definitions cannot be loaded using |\usepackage| or |\RequirePackage|
% which has a mechanism to prevent loading a style file more than once.
% When loading the definitions by means of |\input|
% multiple instances have to be prevented manually:
%\iffalse
%This code needs to be before the `\ProvidesFile' directive
%which is defined at the beginning of this file.
%Therefore it is also placed there and commented out here.
%</package>
%<*discard>
%\fi
%    \begin{macrocode}
\ifdefined\childdocmain\endinput\fi
%    \end{macrocode}
%\iffalse
%</discard>
%<*package>
%\fi
%
% \macro{\ifchilddoc}
% \macro{\ifchilddocmanual}
% The conditional |\ifchilddoc| tells whether a
% child (true) or main (false) document is being compiled.
% The conditional |\ifchilddocmanual| tells whether
% the |\includeonly| mechanism is used (false) or
% the selection of child files must be performed manually (true).
% The definitions initialise to false:
%    \begin{macrocode}
\newif\ifchilddoc
\newif\ifchilddocmanual
%    \end{macrocode}

% \macro{\childdocname}
% \macro{\childdocjob}
% The macro |\childdocname| stores the name of the main document
% to be compiled. The macro |\childdocjob| stores the name of
% the document on which the \LaTeX{} compiler was originally invoked.
% The content of |\jobname| cannot be compared
% to filenames specified in the source due to different catcodes.
% The following code rescans |\jobname|, stores the result
% in |\childdocname| and saves a copy in |\childdocjob|:
%    \begin{macrocode}
\edef\childdocname{\scantokens\expandafter{\jobname\noexpand}}
\let\childdocjob\childdocname
%    \end{macrocode}

% \macro{\childdocdisable}
% The macro |\childdocdisable| prevents the main file
% from being processed more than once.
% At this stage, the main document command |\childdocmain|
% is assumed to be called once again where it should do nothing.
% Any subsequent call to it should prevent
% a secondary processing of the main document
% It overwrites the forwarding commands
% |\childdocof| and |\childdocforward|
% with empty macros to prevent further inclusions of the main document:
%    \begin{macrocode}
\newcommand{\childdocdisable}
{
  \renewcommand{\childdocmain}[1]{\renewcommand{\childdocmain}[1]{\endinput}}
  \renewcommand{\childdocof}[1]{}
  \renewcommand{\childdocby}[2][]{}
  \renewcommand{\childdocforward}[2][]{}
  \renewcommand{\childdocdisable}{}
}
%    \end{macrocode}

% \macro{\childdocmain}
% The macro |\childdocmain| is to be called at the top of the main file
% with nothing or the main filename (without extension) as argument.
% First, it breaks loops.
% If the argument is not empty and does not match |\childdocname|
% (which is set by the first inclusion of |childdoc.def|),
% |\ifchilddoc| is set to true, |\includeonly| is applied to the child file
% and |\jobname| is set to the main file
% (for proper handling of |.aux| files):
%    \begin{macrocode}
\newcommand{\childdocmain}[1]
{
  \childdocdisable\childdocmain{}
  \if?#1?\else
    \begingroup
      \def\childdoctmp{#1}
      \ifx\childdoctmp\childdocname
        \def\childdoctmp{}
      \else
        \def\childdoctmp
        {
          \childdoctrue
          \includeonly{\childdocname}
          \def\childdocjob{#1}
          \def\jobname{#1}
        }
      \fi
      \expandafter
    \endgroup
    \childdoctmp
  \fi
}
%    \end{macrocode}

% \macro{\childdocof}
% The command |\childdocof| redirects
% compilation to the main file |#1|.
%    \begin{macrocode}
\newcommand{\childdocof}[1]
{
  \childdocdisable
  \childdoctrue
  \includeonly{\childdocname}
  \def\jobname{#1}
  \def\childdocjob{#1}
  \input{#1}
}
%    \end{macrocode}

% \macro{\childdocby}
% The command |\childdocby| ....
%    \begin{macrocode}
\newcommand{\childdocby}[2][]
{
  \childdocdisable
  \childdoctrue
  \childdocmanualtrue
  \if?#1?\else
    \def\jobname{#2}
  \fi
  \def\childdocjob{#2}
  \input{#2}
  \endinput
}
%    \end{macrocode}

% \macro{\childdocforward}
% The command |\childdocforward| redirects
% compilation to the main file or
% (if the optional argument is given) a child file.
% Parameters are set as if the main file
% or a child file starting with |\childdocof| was compiled.
% Then compilation is handed over to the main file:
%    \begin{macrocode}
\newcommand{\childdocforward}[2][]
{
  \begingroup
    \if?#1?
      \def\childdoctmp
      {
        \def\childdocname{#2}
        \def\childdocjob{#2}
        \def\jobname{#2}
        \input{#2}
        \endinput
      }
    \else
      \def\childdoctmp
      {
        \childdocdisable
        \def\childdocname{#2}
        \childdoctrue
        \includeonly{#2}
        \def\childdocjob{#1}
        \def\jobname{#1}
        \input{#1}
        \endinput
      }
    \fi
    \expandafter
  \endgroup
  \childdoctmp
}
%    \end{macrocode}

% \macro{\childdocforwardprefix}
% The command |\childdocforwardprefix| redirects
% compilation to the main or a child file by means of a pattern.
% The prefix |#1| in the current filename is replaced by |#2|
% and the suffix of the current filename is kept
% (it is assumed that the filename does not contain the substring `|~~~|'
% which is used as a delimiter).
% Compilation is handed over to the new file by |\childdocforward|:
%    \begin{macrocode}
\newcommand{\childdocforwardprefix}[3][]
{
  \begingroup
    \def\childdocextract #2##1~~~{\def\childdoctmp{\childdocforward[#1]{#3##1}}}
    \expandafter\childdocextract\childdocname~~~
    \expandafter
  \endgroup
  \childdoctmp
}
%    \end{macrocode}

% \macro{\childdoc}
% The deprecated macro |\childdoc| is a legacy version of |\childdocmain|:
%    \begin{macrocode}
\newcommand{\childdoc}{\childdocmain}
%    \end{macrocode}

% \macro{\childdocredirect}
% The deprecated macro |\childdocredirect| is a legacy version
% of |\childdocforward| and |\childdocforwardprefix|:
%    \begin{macrocode}
\newcommand{\childdocredirect}[2][]
{
  \begingroup
    \if?#1?
      \def\childdoctmp{\childdocforward{#2}}
    \else
      \def\childdoctmp{\childdocforwardprefix{#1}{#2}}
    \fi
    \expandafter
  \endgroup
  \childdoctmp
}
%    \end{macrocode}

%\iffalse
%</package>
%\fi
%
\endinput
|
and perform the replacements as outlined below.
Instead of |\childdocmain{|\textit{main}|}| add the following code
to the top of the main file:
%
\begin{center}
\begin{tabular}{l}
|\||ifdefined\childdocname\endinput\||fi\newif\ifchilddoc|\\
|\edef\childdocname{\scantokens\expandafter{\jobname\noexpand}}|\\
|\def\childdocmain{|\textit{main}|}\||ifx\childdocmain\childdocname\||else|\\
|\childdoctrue\includeonly{\childdocname}\let\jobname\childdocmain\||fi|\\
\end{tabular}
\end{center}
%
Instead of |\childdocof{|\textit{main}|}| just include the main file
at the top of each child file:
%
\begin{center}
|\input{|\textit{main}|}|
\end{center}
%
A simple redirection |\childdocforward{|\textit{dest}|}| is achieved by:
%
\begin{center}
|\def\jobname{|\textit{dest}|}\input{\jobname}|
\end{center}
%
The redirection with prefix
|\childdocforwardprefix[|\textit{prefix}|]{|\textit{dest}|}|
is accomplished by:
%
\begin{center}
\begin{tabular}{l}
|{\edef\jobname{\scantokens\expandafter{\jobname\noexpand}}|\\
|\def\redirectjob |\textit{prefix}|#1~~~{\gdef\jobname{|\textit{dest}|#1}}|\\
|\expandafter\redirectjob\jobname~~~}\input{\jobname}|
\end{tabular}
\end{center}

In an alternative approach,
child documents can be compiled by a specific command line
without additional code or specific definitions:
%
\begin{center}
|... -jobname "|\textit{target}|" "|[\textit{flags}]%
|\includeonly{|\textit{dest}|}\input{|\textit{main}|}"|
\end{center}
%

%%%%%%%%%%%%%%%%%%%%%%%%%%%%%%%%%%%%%%%%%%%%%%%%%%%%%%%%%%%%%%%%%%%%%%%%%%%%%%%%
%%%%%%%%%%%%%%%%%%%%%%%%%%%%%%%%%%%%%%%%%%%%%%%%%%%%%%%%%%%%%%%%%%%%%%%%%%%%%%%%
\section{Information}

%%%%%%%%%%%%%%%%%%%%%%%%%%%%%%%%%%%%%%%%%%%%%%%%%%%%%%%%%%%%%%%%%%%%%%%%%%%%%%%%
\subsection{Copyright}

Copyright \copyright{} 2017--2018 Niklas Beisert

This work may be distributed and/or modified under the
conditions of the \LaTeX{} Project Public License, either version 1.3
of this license or (at your option) any later version.
The latest version of this license is in
  \url{http://www.latex-project.org/lppl.txt}
and version 1.3 or later is part of all distributions of \LaTeX{}
version 2005/12/01 or later.

This work has the LPPL maintenance status `maintained'.

The Current Maintainer of this work is Niklas Beisert.

This work consists of the files |README.txt|, |childdoc.ins| and |childdoc.dtx|
as well as the derived files |childdoc.def|, |cdocsamp.tex|
with |cdocsch1.tex|, |cdocsch2.tex|, |cdocspt3.tex|, |cdocspt4.tex|,
|cdocsdrf.tex|, |cdocsfn1.tex|, |cdocsfn2.tex|
as well as |childdoc.pdf|.

%%%%%%%%%%%%%%%%%%%%%%%%%%%%%%%%%%%%%%%%%%%%%%%%%%%%%%%%%%%%%%%%%%%%%%%%%%%%%%%%
\subsection{Files and Installation}

The package consists of the files:
%
\begin{center}
\begin{tabular}{ll}
    |README.txt|   & readme file \\
    |childdoc.ins| & installation file \\
    |childdoc.dtx| & source file \\
    |childdoc.def| & definition file \\
    |cdocsamp.tex| & sample main file \\
    |cdocsch1.tex| & sample include file \\
    |cdocsch2.tex| & sample include file \\
    |cdocspt3.tex| & sample part file \\
    |cdocspt4.tex| & sample part file \\
    |cdocsdrf.tex| & sample redirection file \\
    |cdocsfn1.tex| & sample redirection file \\
    |cdocsfn2.tex| & sample redirection file \\
    |childdoc.pdf| & manual
\end{tabular}
\end{center}
%
The distribution consists of the files
|README.txt|, |childdoc.ins| and |childdoc.dtx|.
%
\begin{itemize}
\item
Run (pdf)\LaTeX{} on |childdoc.dtx|
to compile the manual |childdoc.pdf| (this file).
\item
Run \LaTeX{} on |childdoc.ins| to create the definitions file |childdoc.def|
and the sample |cdocsamp.tex| with include files
|cdocsch1.tex|, |cdocsch2.tex|, |cdocspt3.tex|, |cdocspt4.tex|,
|cdocsdrf.tex|, |cdocsfn1.tex|, |cdocsfn2.tex|.
Then copy the file |childdoc.def| to an appropriate directory of your \LaTeX{}
distribution, e.g.\ \textit{texmf-root}|/tex/latex/childdoc|.
\end{itemize}

%%%%%%%%%%%%%%%%%%%%%%%%%%%%%%%%%%%%%%%%%%%%%%%%%%%%%%%%%%%%%%%%%%%%%%%%%%%%%%%%
\subsection{Related CTAN Packages}

There are several other packages which offer a similar functionality:
%
\begin{itemize}
\item
The packages
\href{http://ctan.org/pkg/docmute}{\textsf{docmute}},
\href{http://ctan.org/pkg/includex}{\textsf{includex}} and
\href{http://ctan.org/pkg/standalone}{\textsf{standalone}}
provide commands to include only the document body of
a child file thus allowing both files to be compiled individually.
\item
The packages \href{http://ctan.org/pkg/subdocs}{\textsf{subdocs}}
and \href{http://ctan.org/pkg/subfiles}{\textsf{subfiles}}
provide structures in which the main and child documents can be
encapsulated and allowing them to be compiled individually.
The inclusion mechanism is different from the conventional |\include|.
\item
The package \href{http://ctan.org/pkg/combine}{\textsf{combine}}
is an elaborate solution to combine several documents into one.
\end{itemize}
%
See also the CTAN topic \href{http://ctan.org/topic/subdocs}{\textsf{subdocs}}
for further related packages.
The present package differs from the above solutions in that
a document structure constructed with the conventional |\include| mechanism
just needs two extra commands at the top of every file
such that all constituent files can be compiled individually.

%%%%%%%%%%%%%%%%%%%%%%%%%%%%%%%%%%%%%%%%%%%%%%%%%%%%%%%%%%%%%%%%%%%%%%%%%%%%%%%%
%\subsection{Feature Suggestions}
%
%The following is a list of features which may be useful for future
%versions of this package:
%%
%\begin{itemize}
%\item
%\ldots
%\end{itemize}

%%%%%%%%%%%%%%%%%%%%%%%%%%%%%%%%%%%%%%%%%%%%%%%%%%%%%%%%%%%%%%%%%%%%%%%%%%%%%%%%
\subsection{Revision History}

%%%%%%%%%%%%%%%%%%%%%%%%%%%%%%%%%%%%%%%%
\paragraph{v2.0:} 2018/12/30

\begin{itemize}
\item
immediate forward processing
\item
added |\childdocby| mechanism
\item
manual restructured
\end{itemize}

%%%%%%%%%%%%%%%%%%%%%%%%%%%%%%%%%%%%%%%%
\paragraph{v1.6:} 2018/01/17

\begin{itemize}
\item
application for development of include files
\item
corrections to manual
\end{itemize}

%%%%%%%%%%%%%%%%%%%%%%%%%%%%%%%%%%%%%%%%
\paragraph{v1.5:} 2017/05/21

\begin{itemize}
\item
more complete structuring introduced
\item
|\childdocof| introduced
\item
|\childdoc| renamed to |\childdocmain|
\item
|\childredirect| renamed to |\childdocforward| and |\childdocforwardprefix|
and functionality expanded
\end{itemize}

%%%%%%%%%%%%%%%%%%%%%%%%%%%%%%%%%%%%%%%%
\paragraph{v1.0:} 2017/04/27

\begin{itemize}
\item
manual and install package
\item
first version published on CTAN
\end{itemize}

%%%%%%%%%%%%%%%%%%%%%%%%%%%%%%%%%%%%%%%%
\paragraph{v0.6:} 2017/04/26

\begin{itemize}
\item
redirection mechanism added
\end{itemize}

%%%%%%%%%%%%%%%%%%%%%%%%%%%%%%%%%%%%%%%%
\paragraph{v0.5:} 2017/04/26

\begin{itemize}
\item
functionality in definition file
\end{itemize}


%%%%%%%%%%%%%%%%%%%%%%%%%%%%%%%%%%%%%%%%%%%%%%%%%%%%%%%%%%%%%%%%%%%%%%%%%%%%%%%%
%%%%%%%%%%%%%%%%%%%%%%%%%%%%%%%%%%%%%%%%%%%%%%%%%%%%%%%%%%%%%%%%%%%%%%%%%%%%%%%%
%%%%%%%%%%%%%%%%%%%%%%%%%%%%%%%%%%%%%%%%%%%%%%%%%%%%%%%%%%%%%%%%%%%%%%%%%%%%%%%%
\appendix

\settowidth\MacroIndent{\rmfamily\scriptsize 000\ }

 \DocInput{childdoc.dtx}

\end{document}
%</driver>
% \fi
%
% %%%%%%%%%%%%%%%%%%%%%%%%%%%%%%%%%%%%%%%%%%%%%%%%%%%%%%%%%%%%%%%%%%%%%%%%%%%%%%
% %%%%%%%%%%%%%%%%%%%%%%%%%%%%%%%%%%%%%%%%%%%%%%%%%%%%%%%%%%%%%%%%%%%%%%%%%%%%%%
% \section{Sample}
%\iffalse
%<*samplemain>
%\fi
%
% The following presents a sample document
% with two chapters, two parts, a title page,
% a compile flag as well as three forwarding files to set the flag.
% It consists of eight |.tex| files:
% \begin{center}
% \begin{tabular}{ll}
% |cdocsamp.tex|&main file\\
% |cdocsch1.tex|&include file for chapter 1\\
% |cdocsch2.tex|&include file for chapter 2\\
% |cdocspt3.tex|&include file for part 3\\
% |cdocspt4.tex|&include file for part 4\\
% |cdocsdrf.tex|&forwarding file for main file in draft mode\\
% |cdocsfi1.tex|&forwarding file for final version of chapter 1\\
% |cdocsfi2.tex|&forwarding file for final version of chapter 2\\
% \end{tabular}
% \end{center}
% Each of the eight files can be compiled directly by the \LaTeX{} compiler.
%
% %%%%%%%%%%%%%%%%%%%%%%%%%%%%%%%%%%%%%%
% \paragraph{Main File.}
%
% The main file is called |cdocsamp.tex|.
%
% Load the \textsf{childdoc} definitions and
% declare the filename for the main document:
%    \begin{macrocode}
% \iffalse
%
% childdoc.dtx Copyright (C) 2017-2018 Niklas Beisert
%
% This work may be distributed and/or modified under the
% conditions of the LaTeX Project Public License, either version 1.3
% of this license or (at your option) any later version.
% The latest version of this license is in
%   http://www.latex-project.org/lppl.txt
% and version 1.3 or later is part of all distributions of LaTeX
% version 2005/12/01 or later.
%
% This work has the LPPL maintenance status `maintained'.
%
% The Current Maintainer of this work is Niklas Beisert.
%
% This work consists of the files childdoc.dtx and childdoc.ins
% and the derived files childdoc.def and cdocsamp.tex with
% cdocsch1.tex, cdocsch2.tex, cdocsdrf.tex, cdocsfn1.tex, cdocsfn2.tex.
%
%<package>\ifdefined\childdocmain\endinput\fi
%<package>\ProvidesFile{childdoc.def}[2018/12/30 v2.0 child document driver]
%<samplemain>\ProvidesFile{cdocsamp.tex}[2018/12/30 v2.0 sample for childdoc]
%<*driver>
%\ProvidesFile{childdoc.drv}[2018/12/30 v2.0 childdoc reference manual file]
\PassOptionsToClass{10pt,a4paper}{article}
\documentclass{ltxdoc}

\usepackage[margin=35mm]{geometry}
\usepackage{hyperref}
\usepackage{hyperxmp}
\usepackage[usenames]{color}

\hypersetup{colorlinks=true}
\hypersetup{pdfstartview=FitH}
\hypersetup{pdfpagemode=UseNone}
\hypersetup{pdfsource={}}
\hypersetup{pdflang={en-UK}}
\hypersetup{pdfcopyright={Copyright 2017-2018 Niklas Beisert.
  This work may be distributed and/or modified under the
  conditions of the LaTeX Project Public License, either version 1.3
  of this license or (at your option) any later version.}}
\hypersetup{pdflicenseurl={http://www.latex-project.org/lppl.txt}}
\hypersetup{pdfcontactaddress={ETH Zurich, ITP, HIT K,
  Wolfgang-Pauli-Strasse 27}}
\hypersetup{pdfcontactpostcode={8093}}
\hypersetup{pdfcontactcity={Zurich}}
\hypersetup{pdfcontactcountry={Switzerland}}
\hypersetup{pdfcontactemail={nbeisert@itp.phys.ethz.ch}}
\hypersetup{pdfcontacturl={http://people.phys.ethz.ch/\xmptilde nbeisert/}}

\newcommand{\secref}[1]{\hyperref[#1]{section \ref*{#1}}}

\parskip1ex
\parindent0pt
\let\olditemize\itemize
\def\itemize{\olditemize\parskip0pt}

\begin{document}

\title{The \textsf{childdoc} Package}
\hypersetup{pdftitle={The childdoc Package}}
\author{Niklas Beisert\\[2ex]
  Institut f\"ur Theoretische Physik\\
  Eidgen\"ossische Technische Hochschule Z\"urich\\
  Wolfgang-Pauli-Strasse 27, 8093 Z\"urich, Switzerland\\[1ex]
  \href{mailto:nbeisert@itp.phys.ethz.ch}
  {\texttt{nbeisert@itp.phys.ethz.ch}}}
\hypersetup{pdfauthor={Niklas Beisert}}
\hypersetup{pdfsubject={Manual for the LaTeX2e Package childdoc}}
\date{30 December 2018, \textsf{v2.0}}
\maketitle

\begin{abstract}\noindent
\textsf{childdoc} is a \LaTeXe{} package
that enables the direct compilation
of document sections included by |\include|
to individual files.
\end{abstract}

\begingroup
\parskip0ex
\tableofcontents
\endgroup

%%%%%%%%%%%%%%%%%%%%%%%%%%%%%%%%%%%%%%%%%%%%%%%%%%%%%%%%%%%%%%%%%%%%%%%%%%%%%%%%
%%%%%%%%%%%%%%%%%%%%%%%%%%%%%%%%%%%%%%%%%%%%%%%%%%%%%%%%%%%%%%%%%%%%%%%%%%%%%%%%
\section{Introduction}

\LaTeX{} provides a mechanism to structure a large document (such as a book)
into a main file and several child files (containing the chapters)
using the |\include| command.
This mechanism is beneficial for documents
which span hundreds of pages in order to
make the source file(s) more manageable.
Moreover, compilation can be restricted to
selected child files by means of the |\includeonly| command.
The latter feature can be used to reduce the compilation time while editing
(this was significantly more useful in the earlier days of \LaTeX{})
or to generate a smaller document which is easier to navigate.
Another application of |\includeonly| is to generate
documents consisting of selected parts of the complete document.

However, there are a few drawbacks of the plain |\include| mechanism:
\begin{itemize}
\item
The child files cannot be compiled on their own,
they can only be compiled via the main file.
A naive editing environment
(such as a text editor with an option
to have the current file processed by \LaTeX)
may require one to switch to the main file before compiling;
attempting to compile the child file produces errors.
\item
The main file must be modified (each time)
to adjust the |\includeonly| command
to the present needs. This easily leaves the main file in a messy state.
\item
The generated document will always carry the filename
of the main document. This is inconvenient if
several child files are to be compiled and
to be kept for distribution.
\end{itemize}

The present package provides a simple interface
to make child files individually compilable by \LaTeX{}.
Compiling a child file then has the same effect as compiling
the main file with an |\includeonly| command
to select the appropriate child.
Moreover the generated document will carry the name of the child
rather than the main file.
This resolves all three above issues.

This feature is meant to make the editing of books,
thesis documents and lecture notes somewhat more convenient.
However, the package can also be used efficiently for
composing a series of documents (such as exercise sheets)
which are typically distributed individually.
It then assists the author in generating the individual documents
(potentially in different versions)
as well as a document containing the collected series.
Another application is in developing style files
or other kinds of included material
where compilation of the style file could redirect
to a sample or test file.

%%%%%%%%%%%%%%%%%%%%%%%%%%%%%%%%%%%%%%%%%%%%%%%%%%%%%%%%%%%%%%%%%%%%%%%%%%%%%%%%
%%%%%%%%%%%%%%%%%%%%%%%%%%%%%%%%%%%%%%%%%%%%%%%%%%%%%%%%%%%%%%%%%%%%%%%%%%%%%%%%
\section{Usage}

First of all, the package \textsf{childdoc} is \emph{not} a standard
\LaTeXe{} |.sty| style file! Therefore it needs to be invoked in
a non-standard way.

%%%%%%%%%%%%%%%%%%%%%%%%%%%%%%%%%%%%%%%%%%%%%%%%%%%%%%%%%%%%%%%%%%%%%%%%%%%%%%%%
\subsection{Included Files}
\label{sec:include}

%%%%%%%%%%%%%%%%%%%%%%%%%%%%%%%%%%%%%%%%
\DescribeMacro{\childdocmain}
To use the package, add the commands
\begin{center}
\begin{tabular}{l}
|\input{childdoc.def}|\\
|\childdocmain{}|\\
\end{tabular}
\end{center}
at the very top of the main \LaTeX{} file,
in particular \emph{before} the |\documentclass| statement!
The argument of |\childdocmain| should be left empty
(but it must be present).

%%%%%%%%%%%%%%%%%%%%%%%%%%%%%%%%%%%%%%%%
\DescribeMacro{\childdocof}
Furthermore, add the commands
\begin{center}
\begin{tabular}{l}
|\input{childdoc.def}|\\
|\childdocof{|\textit{main}|}|\\
\end{tabular}
\end{center}
at the top of every child file \textit{child}
which is included by |\include{|\textit{child}|}|
from within the main file
(or at least for those files to be compiled individually).
The argument \textit{main} must be the filename of the main file.

There are a couple of
considerations in setting up the main and child documents:

%%%%%%%%%%%%%%%%%%%%%%%%%%%%%%%%%%%%%%%%
\paragraph{Restrictions.}

Please note the following restrictions:
\begin{itemize}
\item
|\childdocmain| must be called with one argument \textit{main}
to ensure compatibility with earlier version of the package.
It must either be empty (|\childdocmain{}|)
or precisely match the filename of the main file in which it is specified.
See \secref{sec:detection} for further information.
\item
The filename \textit{main} must be specified without the |.tex| extension.
\item
The filename \textit{main} is case sensitive
(even in case-insensitive file systems)
due to internal string comparison.
\item
The argument \textit{main} should be fully expanded, it cannot be a macro.
\item
Subdirectories and special characters should be avoided in filenames.
\item
The command |\childdocmain{|\textit{main}|}| must be followed by a whitespace.
It should not be followed immediately by another command
or by a comment mark `|%|'.
This is because the \TeX{} parser reads the token immediately following
the argument of |\childdocmain| and puts it
at the beginning of every child section;
however, a white\-space is ignored.
\end{itemize}

%%%%%%%%%%%%%%%%%%%%%%%%%%%%%%%%%%%%%%%%
\paragraph{Content of Main File.}

It is advisable to place all content in the child files included by |\include|.
Any output contained in the main file will appear in all child documents
unless suppressed manually;
it cannot be suppressed automatically by the |\includeonly| directive
and thus should normally be avoided.
A method to include some content in the main file
by means of conditional processing is described in \secref{sec:conditional}.

%%%%%%%%%%%%%%%%%%%%%%%%%%%%%%%%%%%%%%%%
\paragraph{Page Numbering.}

When only a part of the document is compiled,
the appropriate numbering of pages
(as well as other status parameters)
is determined from the |.aux| files.
The latter contain information from previous passes.
However this information needs to propagate through
all intermediate child documents.
Therefore the page numbering in child documents may well
be inconsistent until the complete document is compiled at least once.

A useful (if unconventional) way to always ensure a consistent
page numbering is to restart the numbering in each child document
and denote the pages by `\textit{child}|.|\textit{page}'
where \textit{child} represents the chapter/section number of the child file.
This can be achieved by the command
|\numberwithin{page}{|\textit{child}|}|
of the \textsf{amsmath} package
where \textit{child} can be |chapter| or |section|
depending on the chosen structuring.
Alternatively, one can modify the macro |\thepage| appropriately
and reset the counter |page| at the start of each child file.

%%%%%%%%%%%%%%%%%%%%%%%%%%%%%%%%%%%%%%%%%%%%%%%%%%%%%%%%%%%%%%%%%%%%%%%%%%%%%%%%
\subsection{Conditional Processing}
\label{sec:conditional}

The package provides a mechanism to compile different versions
of a document. To customise the versions further some conditional processing
can come in handy to distinguish which version is being compiled.
The package provides two macros to describe the compilation context:

%%%%%%%%%%%%%%%%%%%%%%%%%%%%%%%%%%%%%%%%
\DescribeMacro{\ifchilddoc}
The conditional |\ifchilddoc| distinguishes between the compilation of
child documents and the main document:
%
\begin{center}
|\ifchilddoc |\textit{child-code}| |[|\||else |\textit{main-code}]| \||fi|
\end{center}

%%%%%%%%%%%%%%%%%%%%%%%%%%%%%%%%%%%%%%%%
\DescribeMacro{\childdocname}
\DescribeMacro{\childdocjob}
The macro |\childdocname| contains the filename (without extension)
of the main or child file being processed.
Note that |\childdocjob| will always contain the name of the main file.

%%%%%%%%%%%%%%%%%%%%%%%%%%%%%%%%%%%%%%%%
\paragraph{Title Page.}

Conditional processing can be used to include a title or banner page
in the main document when proper precautions are taken.
Importantly, the code in the main file should ensure that the page counter
(as well as other status parameters which are stored in the |.aux| files)
takes the same value after the conditional processing.
Otherwise the page numbers may take divergent values
depending on which part is compiled.

For example, a title page could be declared by:
%
\begin{center}
\begin{tabular}{l}
|\ifchilddoc\||else|\\
|\addtocounter{page}{-1}|\\
\textit{code for title page}\\
|\newpage|\\
|\||fi|
\end{tabular}
\end{center}
%
A banner page for the child documents can be generated by:
%
\begin{center}
\begin{tabular}{l}
|\ifchilddoc|\\
|\addtocounter{page}{-1}|\\
\textit{code for banner page}\\
|\newpage|\\
|\||fi|
\end{tabular}
\end{center}
%
Here one could write a message such as:
\begin{center}
|This is the part \childdocname{} of \childdocjob{}.|
\end{center}

%%%%%%%%%%%%%%%%%%%%%%%%%%%%%%%%%%%%%%%%%%%%%%%%%%%%%%%%%%%%%%%%%%%%%%%%%%%%%%%%
\subsection{Flags}
\label{sec:flags}

The package makes it easy to generate different versions
of the main or child documents.
To this end compilation flags can be defined
and assigned different default values.
They will be particularly useful in conjunction
with the forwarding mechanism described in \secref{sec:forward}.

For example, it may be useful to have a flag |\version|
which can be set to |draft| or |final|.
The document source will contain some conditional code
depending on the value of |\version|.
Suppose further, the flag should default to |final| for the main file
and to |draft| for child files
which is a natural assignment for editing the document.
This is achieved by placing the following code
in the preamble of the main document
(below the |\childdocmain| directive):
%
\begin{center}
\begin{tabular}{l}
|\ifchilddoc|\\
|\providecommand{\version}{draft}|\\
|\||else|\\
|\providecommand{\version}{final}|\\
|\||fi|
\end{tabular}
\end{center}
%
The definition by |\providecommand| makes sure
that previous definitions are not overwritten.
Further statements |\providecommand{\version}{...}|
can thus be added before the above code to override it.

For the main file, one might add a line
(between |\childdocmain| and the above block)
%
\begin{center}
|%\ifchilddoc\||else\providecommand{\version}{draft}\||fi|
\end{center}
%
which can be uncommented to produce a draft version.
Likewise one can add a line to the very top of a child file
(above the |\childdocof{|\textit{main}|}| directive)
%
\begin{center}
|%\providecommand{\version}{final}|
\end{center}
%
which can be uncommented to produce the final version of this child document.

%%%%%%%%%%%%%%%%%%%%%%%%%%%%%%%%%%%%%%%%%%%%%%%%%%%%%%%%%%%%%%%%%%%%%%%%%%%%%%%%
\subsection{Forwarding}
\label{sec:forward}

Different versions of the main or child documents
using compilation flags as described in \secref{sec:flags}
can be (permanently) stored in different files
for convenient compilation, viewing and distribution.
To this end, the package defines a command
to pass on compilation to a different file:

%%%%%%%%%%%%%%%%%%%%%%%%%%%%%%%%%%%%%%%%
\DescribeMacro{\childdocforward}
The command |\childdocforward| redirects processing to
another source file:
%
\begin{center}
\begin{tabular}{l}
|\input{childdoc.def}|\\
|\childdocforward[|\textit{main}|]{|\textit{dest}|}|\\
\end{tabular}
\end{center}
%
The argument \textit{dest} is the destination file
(without extension).
It should be the main file or one of the child files.
Note that further \textsf{childdoc} directives
such as |\childdocof| and |\childdocforward|
in the indicated file will be processed in this form.
The optional argument \textit{main}
passes on directly to the main file \textit{main}
while pretending to compile the child \textit{dest}.
This form behaves as if \textit{dest}
issues |\childdocof{|\textit{main}|}| right away,
and no further \textsf{childdoc} directives will be processed.

%%%%%%%%%%%%%%%%%%%%%%%%%%%%%%%%%%%%%%%%
\DescribeMacro{\...prefix}
In the alternative form |\childdocforwardprefix|,
%
\begin{center}
\begin{tabular}{l}
|\input{childdoc.def}|\\
|\childdocforwardprefix[|\textit{main}|]{|\textit{prefix}|}{|\textit{dest}|}|
\end{tabular}
\end{center}
%
the destination file is determined by a pattern
depending on the current file:
To make this work, the current file must be called
`{\textit{prefix}\hspace{0.2em}\textit{suffix}}'
with \textit{prefix} matching precisely the argument.
Processing is then passed on to the file
`{\textit{dest}\hspace{0.2em}\textit{suffix}}'.
Surely, the same effect is achieved by
directly specifying the
argument `{\textit{dest}\hspace{0.2em}\textit{suffix}}'
in the first form.
However, that requires to set up a different file
for each child. With the alternative form of the command
all these files can have exactly the same content
which simplifies setting them up and maintaining them.

For example, the following file |draft.tex|
with a compilation flag |\version| as described in \secref{sec:flags}
compiles the main document as a draft:
%
\begin{center}
\begin{tabular}{l}
|\def\version{draft}|\\
|\input{childdoc.def}|\\
|\childdocforward{|\textit{main}|}|
\end{tabular}
\end{center}
%
Likewise, the following files |final|\textit{nn}|.tex|
compile the final version of the child document
|child|\textit{nn}|.tex|:
%
\begin{center}
\begin{tabular}{l}
|\def\version{final}|\\
|\input{childdoc.def}|\\
|\childdocforwardprefix{final}{child}|
\end{tabular}
\end{center}
%

Note that when several versions of a main file and/or of each child file
are to be generated, it may be convenient to set up a |Makefile| or
shell script to automatise the process.

%%%%%%%%%%%%%%%%%%%%%%%%%%%%%%%%%%%%%%%%%%%%%%%%%%%%%%%%%%%%%%%%%%%%%%%%%%%%%%%%
\subsection{Command Line Processing}
\label{sec:commandline}

The effect of redirection files can also be achieved by invoking
the \LaTeX{} compiler with a more elaborate command line.
Most conveniently this should be done as part
of a shell script or a |Makefile|.

When using \textsf{childdoc} in the main file, the following
command lines effectively perform a redirection
(note that depending on the shell being used,
backslashes may have to be doubled: `|\|' $\to$ `|\\|'):
%
\begin{center}
|... -jobname "|\textit{target}|" |\\|"|[\textit{flags}]%
|\input{childdoc.def}\childdocforward[|\textit{main}|]{|\textit{dest}|}"|
\end{center}
%
Here \textit{target} is the name of the output file,
\textit{main} is the name of the main file
and \textit{dest} is the name of the main or child file to be processed
(all filenames without extensions).
The optional argument \textit{main} can be omitted
if \textit{main} matches \textit{dest}.
Optionally, compilation \textit{flags} can be defined via |\def| commands.
This command line makes the \TeX{} engine believe
it is compiling the file \textit{target}
whose content is specified as the latter parameter.
The provided code then forwards the processing to
\textit{main} or \textit{dest} as described in \secref{sec:forward}.

%%%%%%%%%%%%%%%%%%%%%%%%%%%%%%%%%%%%%%%%%%%%%%%%%%%%%%%%%%%%%%%%%%%%%%%%%%%%%%%%
\subsection{Include by Input}
\label{sec:input}

Including child documents by |\include| has some restrictions by design.
Most notably, the content of a child document always occupies
its own set of pages; pages cannot be shared between child documents.
Usually, this behaviour makes perfect sense
because each child document contain an essential part of the document.
However, in some situations it may be desirable to compose
a document from a collection of parts
without having mandatory page breaks between then.
For this case, the package
provides a mechanism to include parts
by |\input| which can also be processed individually.
However, by construction this mechanism
requires manual handling of the content to be output.

%%%%%%%%%%%%%%%%%%%%%%%%%%%%%%%%%%%%%%%%
\DescribeMacro{\ifchilddocmanual}
The main file should be prepared as usual, see \secref{sec:include}.
However, the document body must make a distinction
between processing of an individual part and of the main document, e.g.:
%
\begin{center}
\begin{tabular}{l}
|\ifchilddocmanual|\\
|\input{\childdocname}|\\
|\||else|\\
\textit{document body with }|\input{|\textit{part}|}|\\
|\||fi|
\end{tabular}
\end{center}
%
The conditional |\ifchilddocmanual| is true whenever
a part to be included by |\input| is being compiled,
and the name of the part is stored in |\childdocname|.

%%%%%%%%%%%%%%%%%%%%%%%%%%%%%%%%%%%%%%%%
\DescribeMacro{\childdocby}
Each part to be included by |\input| should start with:
%
\begin{center}
\begin{tabular}{l}
|\input{childdoc.def}|\\
|\childdocby{|\textit{main}|}|\\
\end{tabular}
\end{center}
%
The directive |\childdocby| is similar to |\childdocof|
described in \secref{sec:include},
but the subsequent selection of content must be done manually.
To that end, both |\ifchilddoc| and |\ifchilddocmanual|
will be true upon processing of a part,
and the name of the part is stored in |\childdocname|.
Note that |\jobname| will be set to the filename of the current part
so that each part receives an individual |.aux| file
that does not interfere with the |.aux| file(s) of the main document.
This behaviour can be altered by the alternative form
|\childdocby[*]{|\textit{main}|}| (with a non-empty optional argument)
which uses the |.aux| file of the main document
by setting |\jobname| to \textit{main}.

%%%%%%%%%%%%%%%%%%%%%%%%%%%%%%%%%%%%%%%%%%%%%%%%%%%%%%%%%%%%%%%%%%%%%%%%%%%%%%%%
\subsection{Driver Development}
\label{sec:driver}

The \textsf{childdoc} mechanism can also be use for the development
of definition files such as \LaTeX{} styles or classes.
This case differs from the above setup with multiple parts
included by |\include| in that no |\includeonly| should be invoked.
This can be achieved by starting the include file
(before |\ProvidesPackage|) with:
%
\begin{center}
\begin{tabular}{l}
|\input{childdoc.def}|\\
|\childdocforward{|\textit{main}|}|\\
\end{tabular}
\end{center}
%
or alternatively with:
%
\begin{center}
\begin{tabular}{l}
|\input{childdoc.def}|\\
|\childdocby{|\textit{main}|}|\\
\end{tabular}
\end{center}
%
Both forms have slightly different effects as described above.
The main file is prepared as usual, see \secref{sec:include}.

%%%%%%%%%%%%%%%%%%%%%%%%%%%%%%%%%%%%%%%%%%%%%%%%%%%%%%%%%%%%%%%%%%%%%%%%%%%%%%%%
\subsection{Legacy Detection}
\label{sec:detection}

The directive |\childdocmain| in the main file can detect
whether the complete document or merely a child is to be compiled
even without using the directive |\childdocof|.
This method is deprecated because it is less robust
and there is no compelling reason to use it;
it is merely provided for backward compatibility
and it may be removed in future versions.

If the detection mechanism is to be used,
it is mandatory to correctly specify
the filename of the main file as the argument of |\childdocmain|:
%
\begin{center}
\begin{tabular}{l}
|\input{childdoc.def}|\\
|\childdocmain{|\textit{main}|}|\\
\end{tabular}
\end{center}
%
If |\jobname| does not match the argument \textit{main} of |\childdocmain|,
it is assumed that |\jobname| points to the child file to be compiled.
When using |\childdocmain| with the main file specified as argument,
it suffices to start a child file
with just |\input{|\textit{main}|}|
without loading of the package and using |\childdocof|.
If instead all processing is done
with the appropriate \textsf{childdoc} directives,
the argument of \textit{main} of |\childdocmain| can be empty.

An alternative version of the command line processing described
in \secref{sec:commandline} using the detection mechanism reads:
%
\begin{center}
|... -jobname "|\textit{target}|" "|[\textit{flags}]%
[|\def\jobname{|\textit{dest}|}|]|\input{|\textit{main}|}"|
\end{center}

%%%%%%%%%%%%%%%%%%%%%%%%%%%%%%%%%%%%%%%%%%%%%%%%%%%%%%%%%%%%%%%%%%%%%%%%%%%%%%%%
\subsection{Manual Code}
\label{sec:manual}

In case one cannot be certain whether the definitions file |childdoc.def|
is installed on the target \TeX{} distribution
and one prefers not to ship it,
it is conceivable to paste a few relevant commands into the sources.

To that end, drop all statements |\input{childdoc.def}|
and perform the replacements as outlined below.
Instead of |\childdocmain{|\textit{main}|}| add the following code
to the top of the main file:
%
\begin{center}
\begin{tabular}{l}
|\||ifdefined\childdocname\endinput\||fi\newif\ifchilddoc|\\
|\edef\childdocname{\scantokens\expandafter{\jobname\noexpand}}|\\
|\def\childdocmain{|\textit{main}|}\||ifx\childdocmain\childdocname\||else|\\
|\childdoctrue\includeonly{\childdocname}\let\jobname\childdocmain\||fi|\\
\end{tabular}
\end{center}
%
Instead of |\childdocof{|\textit{main}|}| just include the main file
at the top of each child file:
%
\begin{center}
|\input{|\textit{main}|}|
\end{center}
%
A simple redirection |\childdocforward{|\textit{dest}|}| is achieved by:
%
\begin{center}
|\def\jobname{|\textit{dest}|}\input{\jobname}|
\end{center}
%
The redirection with prefix
|\childdocforwardprefix[|\textit{prefix}|]{|\textit{dest}|}|
is accomplished by:
%
\begin{center}
\begin{tabular}{l}
|{\edef\jobname{\scantokens\expandafter{\jobname\noexpand}}|\\
|\def\redirectjob |\textit{prefix}|#1~~~{\gdef\jobname{|\textit{dest}|#1}}|\\
|\expandafter\redirectjob\jobname~~~}\input{\jobname}|
\end{tabular}
\end{center}

In an alternative approach,
child documents can be compiled by a specific command line
without additional code or specific definitions:
%
\begin{center}
|... -jobname "|\textit{target}|" "|[\textit{flags}]%
|\includeonly{|\textit{dest}|}\input{|\textit{main}|}"|
\end{center}
%

%%%%%%%%%%%%%%%%%%%%%%%%%%%%%%%%%%%%%%%%%%%%%%%%%%%%%%%%%%%%%%%%%%%%%%%%%%%%%%%%
%%%%%%%%%%%%%%%%%%%%%%%%%%%%%%%%%%%%%%%%%%%%%%%%%%%%%%%%%%%%%%%%%%%%%%%%%%%%%%%%
\section{Information}

%%%%%%%%%%%%%%%%%%%%%%%%%%%%%%%%%%%%%%%%%%%%%%%%%%%%%%%%%%%%%%%%%%%%%%%%%%%%%%%%
\subsection{Copyright}

Copyright \copyright{} 2017--2018 Niklas Beisert

This work may be distributed and/or modified under the
conditions of the \LaTeX{} Project Public License, either version 1.3
of this license or (at your option) any later version.
The latest version of this license is in
  \url{http://www.latex-project.org/lppl.txt}
and version 1.3 or later is part of all distributions of \LaTeX{}
version 2005/12/01 or later.

This work has the LPPL maintenance status `maintained'.

The Current Maintainer of this work is Niklas Beisert.

This work consists of the files |README.txt|, |childdoc.ins| and |childdoc.dtx|
as well as the derived files |childdoc.def|, |cdocsamp.tex|
with |cdocsch1.tex|, |cdocsch2.tex|, |cdocspt3.tex|, |cdocspt4.tex|,
|cdocsdrf.tex|, |cdocsfn1.tex|, |cdocsfn2.tex|
as well as |childdoc.pdf|.

%%%%%%%%%%%%%%%%%%%%%%%%%%%%%%%%%%%%%%%%%%%%%%%%%%%%%%%%%%%%%%%%%%%%%%%%%%%%%%%%
\subsection{Files and Installation}

The package consists of the files:
%
\begin{center}
\begin{tabular}{ll}
    |README.txt|   & readme file \\
    |childdoc.ins| & installation file \\
    |childdoc.dtx| & source file \\
    |childdoc.def| & definition file \\
    |cdocsamp.tex| & sample main file \\
    |cdocsch1.tex| & sample include file \\
    |cdocsch2.tex| & sample include file \\
    |cdocspt3.tex| & sample part file \\
    |cdocspt4.tex| & sample part file \\
    |cdocsdrf.tex| & sample redirection file \\
    |cdocsfn1.tex| & sample redirection file \\
    |cdocsfn2.tex| & sample redirection file \\
    |childdoc.pdf| & manual
\end{tabular}
\end{center}
%
The distribution consists of the files
|README.txt|, |childdoc.ins| and |childdoc.dtx|.
%
\begin{itemize}
\item
Run (pdf)\LaTeX{} on |childdoc.dtx|
to compile the manual |childdoc.pdf| (this file).
\item
Run \LaTeX{} on |childdoc.ins| to create the definitions file |childdoc.def|
and the sample |cdocsamp.tex| with include files
|cdocsch1.tex|, |cdocsch2.tex|, |cdocspt3.tex|, |cdocspt4.tex|,
|cdocsdrf.tex|, |cdocsfn1.tex|, |cdocsfn2.tex|.
Then copy the file |childdoc.def| to an appropriate directory of your \LaTeX{}
distribution, e.g.\ \textit{texmf-root}|/tex/latex/childdoc|.
\end{itemize}

%%%%%%%%%%%%%%%%%%%%%%%%%%%%%%%%%%%%%%%%%%%%%%%%%%%%%%%%%%%%%%%%%%%%%%%%%%%%%%%%
\subsection{Related CTAN Packages}

There are several other packages which offer a similar functionality:
%
\begin{itemize}
\item
The packages
\href{http://ctan.org/pkg/docmute}{\textsf{docmute}},
\href{http://ctan.org/pkg/includex}{\textsf{includex}} and
\href{http://ctan.org/pkg/standalone}{\textsf{standalone}}
provide commands to include only the document body of
a child file thus allowing both files to be compiled individually.
\item
The packages \href{http://ctan.org/pkg/subdocs}{\textsf{subdocs}}
and \href{http://ctan.org/pkg/subfiles}{\textsf{subfiles}}
provide structures in which the main and child documents can be
encapsulated and allowing them to be compiled individually.
The inclusion mechanism is different from the conventional |\include|.
\item
The package \href{http://ctan.org/pkg/combine}{\textsf{combine}}
is an elaborate solution to combine several documents into one.
\end{itemize}
%
See also the CTAN topic \href{http://ctan.org/topic/subdocs}{\textsf{subdocs}}
for further related packages.
The present package differs from the above solutions in that
a document structure constructed with the conventional |\include| mechanism
just needs two extra commands at the top of every file
such that all constituent files can be compiled individually.

%%%%%%%%%%%%%%%%%%%%%%%%%%%%%%%%%%%%%%%%%%%%%%%%%%%%%%%%%%%%%%%%%%%%%%%%%%%%%%%%
%\subsection{Feature Suggestions}
%
%The following is a list of features which may be useful for future
%versions of this package:
%%
%\begin{itemize}
%\item
%\ldots
%\end{itemize}

%%%%%%%%%%%%%%%%%%%%%%%%%%%%%%%%%%%%%%%%%%%%%%%%%%%%%%%%%%%%%%%%%%%%%%%%%%%%%%%%
\subsection{Revision History}

%%%%%%%%%%%%%%%%%%%%%%%%%%%%%%%%%%%%%%%%
\paragraph{v2.0:} 2018/12/30

\begin{itemize}
\item
immediate forward processing
\item
added |\childdocby| mechanism
\item
manual restructured
\end{itemize}

%%%%%%%%%%%%%%%%%%%%%%%%%%%%%%%%%%%%%%%%
\paragraph{v1.6:} 2018/01/17

\begin{itemize}
\item
application for development of include files
\item
corrections to manual
\end{itemize}

%%%%%%%%%%%%%%%%%%%%%%%%%%%%%%%%%%%%%%%%
\paragraph{v1.5:} 2017/05/21

\begin{itemize}
\item
more complete structuring introduced
\item
|\childdocof| introduced
\item
|\childdoc| renamed to |\childdocmain|
\item
|\childredirect| renamed to |\childdocforward| and |\childdocforwardprefix|
and functionality expanded
\end{itemize}

%%%%%%%%%%%%%%%%%%%%%%%%%%%%%%%%%%%%%%%%
\paragraph{v1.0:} 2017/04/27

\begin{itemize}
\item
manual and install package
\item
first version published on CTAN
\end{itemize}

%%%%%%%%%%%%%%%%%%%%%%%%%%%%%%%%%%%%%%%%
\paragraph{v0.6:} 2017/04/26

\begin{itemize}
\item
redirection mechanism added
\end{itemize}

%%%%%%%%%%%%%%%%%%%%%%%%%%%%%%%%%%%%%%%%
\paragraph{v0.5:} 2017/04/26

\begin{itemize}
\item
functionality in definition file
\end{itemize}


%%%%%%%%%%%%%%%%%%%%%%%%%%%%%%%%%%%%%%%%%%%%%%%%%%%%%%%%%%%%%%%%%%%%%%%%%%%%%%%%
%%%%%%%%%%%%%%%%%%%%%%%%%%%%%%%%%%%%%%%%%%%%%%%%%%%%%%%%%%%%%%%%%%%%%%%%%%%%%%%%
%%%%%%%%%%%%%%%%%%%%%%%%%%%%%%%%%%%%%%%%%%%%%%%%%%%%%%%%%%%%%%%%%%%%%%%%%%%%%%%%
\appendix

\settowidth\MacroIndent{\rmfamily\scriptsize 000\ }

 \DocInput{childdoc.dtx}

\end{document}
%</driver>
% \fi
%
% %%%%%%%%%%%%%%%%%%%%%%%%%%%%%%%%%%%%%%%%%%%%%%%%%%%%%%%%%%%%%%%%%%%%%%%%%%%%%%
% %%%%%%%%%%%%%%%%%%%%%%%%%%%%%%%%%%%%%%%%%%%%%%%%%%%%%%%%%%%%%%%%%%%%%%%%%%%%%%
% \section{Sample}
%\iffalse
%<*samplemain>
%\fi
%
% The following presents a sample document
% with two chapters, two parts, a title page,
% a compile flag as well as three forwarding files to set the flag.
% It consists of eight |.tex| files:
% \begin{center}
% \begin{tabular}{ll}
% |cdocsamp.tex|&main file\\
% |cdocsch1.tex|&include file for chapter 1\\
% |cdocsch2.tex|&include file for chapter 2\\
% |cdocspt3.tex|&include file for part 3\\
% |cdocspt4.tex|&include file for part 4\\
% |cdocsdrf.tex|&forwarding file for main file in draft mode\\
% |cdocsfi1.tex|&forwarding file for final version of chapter 1\\
% |cdocsfi2.tex|&forwarding file for final version of chapter 2\\
% \end{tabular}
% \end{center}
% Each of the eight files can be compiled directly by the \LaTeX{} compiler.
%
% %%%%%%%%%%%%%%%%%%%%%%%%%%%%%%%%%%%%%%
% \paragraph{Main File.}
%
% The main file is called |cdocsamp.tex|.
%
% Load the \textsf{childdoc} definitions and
% declare the filename for the main document:
%    \begin{macrocode}
\input{childdoc.def}
\childdocmain{}
%    \end{macrocode}

% Optional override for |\version| flag:
%    \begin{macrocode}
%%\ifchilddoc\else\providecommand{\version}{draft}\fi
%    \end{macrocode}

% Define the default values for the |\version| flag
% (|final| for the main file and |draft| for childs):
%    \begin{macrocode}
\ifchilddoc
\providecommand{\version}{draft}
\else
\providecommand{\version}{final}
\fi
%    \end{macrocode}

% Load the standard document class:
%    \begin{macrocode}
\documentclass[12pt]{article}
%    \end{macrocode}

% Start the document body:
%    \begin{macrocode}
\begin{document}
%    \end{macrocode}

% Declare a title page.
% Print title, part of document being processed and version flag:
%    \begin{macrocode}
\addtocounter{page}{-1}
\begin{center}
{\LARGE\bfseries{}childdoc example\par}
\vspace{1cm}
\ifchilddoc
\ifchilddocmanual part\else chapter\fi:
`\childdocname' of `\childdocjob'\par
\else
main document: `\childdocjob'\par
\fi
version: \version\par
\end{center}
\newpage
%    \end{macrocode}

% Manually include selected file,
% otherwise process as usual:
%    \begin{macrocode}
\ifchilddocmanual
\section*{part `\childdocname'}
\input{\childdocname}
\else
%    \end{macrocode}

% Include the two chapters:
%    \begin{macrocode}
\include{cdocsch1}
\include{cdocsch2}
%    \end{macrocode}

% Include the two parts unless only chapters should be displayed:
%    \begin{macrocode}
\ifchilddoc\else
\section{part three}
\input{cdocspt3}
\section{part four}
\input{cdocspt4}
\fi
%    \end{macrocode}

% Process as usual until here:
%    \begin{macrocode}
\fi
%    \end{macrocode}

% End of document body:
%    \begin{macrocode}
\end{document}
%    \end{macrocode}
%\iffalse
%</samplemain>
%\fi
%
% %%%%%%%%%%%%%%%%%%%%%%%%%%%%%%%%%%%%%%
% \paragraph{Chapter Include Files.}
%
% The include files are called |cdocsch1.tex| and |cdocsch2.tex|.
%
%\iffalse
%<*samplechap1|samplechap2>
%\fi

% Optional override for |\version| flag:
%    \begin{macrocode}
%%\providecommand{\version}{final}
%    \end{macrocode}

% Include the main document:
%    \begin{macrocode}
\input{childdoc.def}
\childdocof{cdocsamp}
%    \end{macrocode}

%\iffalse
%</samplechap1|samplechap2>
%\fi
%
%\iffalse
%<*samplechap1>
%\fi
% Some text for chapter 1:
%    \begin{macrocode}
\section{one}
some text in chapter one
%    \end{macrocode}

%\iffalse
%</samplechap1>
%\fi
% Some text for chapter 2:
%\iffalse
%<*samplechap2>
%\fi
%    \begin{macrocode}
\section{two}
more text in chapter two
%    \end{macrocode}

%\iffalse
%</samplechap2>
%\fi
%
% %%%%%%%%%%%%%%%%%%%%%%%%%%%%%%%%%%%%%%
% \paragraph{Part Include Files.}
%
% The include files are called |cdocspt3.tex| and |cdocspt4.tex|.
%
%\iffalse
%<*samplepart3|samplepart4>
%\fi

% Optional override for |\version| flag:
%    \begin{macrocode}
%%\providecommand{\version}{final}
%    \end{macrocode}

% Include the main document:
%    \begin{macrocode}
\input{childdoc.def}
\childdocby{cdocsamp}
%    \end{macrocode}

%\iffalse
%</samplepart3|samplepart4>
%\fi
%
%\iffalse
%<*samplepart3>
%\fi
% Some text for part 3:
%    \begin{macrocode}
some text in part three
%    \end{macrocode}

%\iffalse
%</samplepart3>
%\fi
% Some text for part 4:
%\iffalse
%<*samplepart4>
%\fi
%    \begin{macrocode}
more text in part four
%    \end{macrocode}

%\iffalse
%</samplepart4>
%\fi
%
% %%%%%%%%%%%%%%%%%%%%%%%%%%%%%%%%%%%%%%
% \paragraph{Forwarding for a Complete Draft.}
%
% The following forwarding file |cdocsdrf.tex|
% compiles the main document in draft mode:
%\iffalse
%<*sampledraft>
%\fi
%    \begin{macrocode}
\def\version{draft}
\input{childdoc.def}
\childdocforward{cdocsamp}
%    \end{macrocode}

%\iffalse
%</sampledraft>
%\fi
%
% %%%%%%%%%%%%%%%%%%%%%%%%%%%%%%%%%%%%%%
% \paragraph{Forwarding for Final Version of the Chapters.}
%
% The following forwarding files |cdocsfn1.tex| and |cdocsfn2.tex|
% (with identical content)
% compile the final versions of the child documents
% |cdocsch1.tex| and |cdocsch2.tex|, respectively:
%\iffalse
%<*samplefinal>
%\fi
%    \begin{macrocode}
\def\version{final}
\input{childdoc.def}
\childdocforwardprefix[cdocsamp]{cdocsfn}{cdocsch}
%    \end{macrocode}

%\iffalse
%</samplefinal>
%\fi
%
% %%%%%%%%%%%%%%%%%%%%%%%%%%%%%%%%%%%%%%
% \paragraph{Command Line Processing.}
%
% The following three command lines generate the output files
% |cdocscld|, |cdocscl1| and |cdocscl2|
% which should be identical to
% |cdocsdrf|, |cdocsch1| and |cdocsfn2|, respectively:
% \begin{center}
% \begin{tabular}{l}
% |latex -jobname cdocscld \|\\
% |  "\def\version{draft}\input{childdoc.def}\childdocforward{cdocsamp}"|\\
% |latex -jobname cdocscl1 \|\\
% |  "\input{childdoc.def}\childdocforward[cdocsamp]{cdocsch1}"|\\
% |latex -jobname cdocscl2 \|\\
% |  "\def\version{final}\input{childdoc.def}\childdocforward{cdocsch2}"|
% \end{tabular}
% \end{center}
% Note that the trailing backslash on each first line
% merely continues the input to the second line
% (for convenient cut ant paste).
% Furthermore, the command |latex| can be replaced by any
% of its alternative versions such as |pdflatex|.
%
% %%%%%%%%%%%%%%%%%%%%%%%%%%%%%%%%%%%%%%%%%%%%%%%%%%%%%%%%%%%%%%%%%%%%%%%%%%%%%%
% %%%%%%%%%%%%%%%%%%%%%%%%%%%%%%%%%%%%%%%%%%%%%%%%%%%%%%%%%%%%%%%%%%%%%%%%%%%%%%
% \section{Implementation}
%\iffalse
%<*package>
%\fi
%
% This section describes the definitions file |childdoc.def|.

% The definitions cannot be loaded using |\usepackage| or |\RequirePackage|
% which has a mechanism to prevent loading a style file more than once.
% When loading the definitions by means of |\input|
% multiple instances have to be prevented manually:
%\iffalse
%This code needs to be before the `\ProvidesFile' directive
%which is defined at the beginning of this file.
%Therefore it is also placed there and commented out here.
%</package>
%<*discard>
%\fi
%    \begin{macrocode}
\ifdefined\childdocmain\endinput\fi
%    \end{macrocode}
%\iffalse
%</discard>
%<*package>
%\fi
%
% \macro{\ifchilddoc}
% \macro{\ifchilddocmanual}
% The conditional |\ifchilddoc| tells whether a
% child (true) or main (false) document is being compiled.
% The conditional |\ifchilddocmanual| tells whether
% the |\includeonly| mechanism is used (false) or
% the selection of child files must be performed manually (true).
% The definitions initialise to false:
%    \begin{macrocode}
\newif\ifchilddoc
\newif\ifchilddocmanual
%    \end{macrocode}

% \macro{\childdocname}
% \macro{\childdocjob}
% The macro |\childdocname| stores the name of the main document
% to be compiled. The macro |\childdocjob| stores the name of
% the document on which the \LaTeX{} compiler was originally invoked.
% The content of |\jobname| cannot be compared
% to filenames specified in the source due to different catcodes.
% The following code rescans |\jobname|, stores the result
% in |\childdocname| and saves a copy in |\childdocjob|:
%    \begin{macrocode}
\edef\childdocname{\scantokens\expandafter{\jobname\noexpand}}
\let\childdocjob\childdocname
%    \end{macrocode}

% \macro{\childdocdisable}
% The macro |\childdocdisable| prevents the main file
% from being processed more than once.
% At this stage, the main document command |\childdocmain|
% is assumed to be called once again where it should do nothing.
% Any subsequent call to it should prevent
% a secondary processing of the main document
% It overwrites the forwarding commands
% |\childdocof| and |\childdocforward|
% with empty macros to prevent further inclusions of the main document:
%    \begin{macrocode}
\newcommand{\childdocdisable}
{
  \renewcommand{\childdocmain}[1]{\renewcommand{\childdocmain}[1]{\endinput}}
  \renewcommand{\childdocof}[1]{}
  \renewcommand{\childdocby}[2][]{}
  \renewcommand{\childdocforward}[2][]{}
  \renewcommand{\childdocdisable}{}
}
%    \end{macrocode}

% \macro{\childdocmain}
% The macro |\childdocmain| is to be called at the top of the main file
% with nothing or the main filename (without extension) as argument.
% First, it breaks loops.
% If the argument is not empty and does not match |\childdocname|
% (which is set by the first inclusion of |childdoc.def|),
% |\ifchilddoc| is set to true, |\includeonly| is applied to the child file
% and |\jobname| is set to the main file
% (for proper handling of |.aux| files):
%    \begin{macrocode}
\newcommand{\childdocmain}[1]
{
  \childdocdisable\childdocmain{}
  \if?#1?\else
    \begingroup
      \def\childdoctmp{#1}
      \ifx\childdoctmp\childdocname
        \def\childdoctmp{}
      \else
        \def\childdoctmp
        {
          \childdoctrue
          \includeonly{\childdocname}
          \def\childdocjob{#1}
          \def\jobname{#1}
        }
      \fi
      \expandafter
    \endgroup
    \childdoctmp
  \fi
}
%    \end{macrocode}

% \macro{\childdocof}
% The command |\childdocof| redirects
% compilation to the main file |#1|.
%    \begin{macrocode}
\newcommand{\childdocof}[1]
{
  \childdocdisable
  \childdoctrue
  \includeonly{\childdocname}
  \def\jobname{#1}
  \def\childdocjob{#1}
  \input{#1}
}
%    \end{macrocode}

% \macro{\childdocby}
% The command |\childdocby| ....
%    \begin{macrocode}
\newcommand{\childdocby}[2][]
{
  \childdocdisable
  \childdoctrue
  \childdocmanualtrue
  \if?#1?\else
    \def\jobname{#2}
  \fi
  \def\childdocjob{#2}
  \input{#2}
  \endinput
}
%    \end{macrocode}

% \macro{\childdocforward}
% The command |\childdocforward| redirects
% compilation to the main file or
% (if the optional argument is given) a child file.
% Parameters are set as if the main file
% or a child file starting with |\childdocof| was compiled.
% Then compilation is handed over to the main file:
%    \begin{macrocode}
\newcommand{\childdocforward}[2][]
{
  \begingroup
    \if?#1?
      \def\childdoctmp
      {
        \def\childdocname{#2}
        \def\childdocjob{#2}
        \def\jobname{#2}
        \input{#2}
        \endinput
      }
    \else
      \def\childdoctmp
      {
        \childdocdisable
        \def\childdocname{#2}
        \childdoctrue
        \includeonly{#2}
        \def\childdocjob{#1}
        \def\jobname{#1}
        \input{#1}
        \endinput
      }
    \fi
    \expandafter
  \endgroup
  \childdoctmp
}
%    \end{macrocode}

% \macro{\childdocforwardprefix}
% The command |\childdocforwardprefix| redirects
% compilation to the main or a child file by means of a pattern.
% The prefix |#1| in the current filename is replaced by |#2|
% and the suffix of the current filename is kept
% (it is assumed that the filename does not contain the substring `|~~~|'
% which is used as a delimiter).
% Compilation is handed over to the new file by |\childdocforward|:
%    \begin{macrocode}
\newcommand{\childdocforwardprefix}[3][]
{
  \begingroup
    \def\childdocextract #2##1~~~{\def\childdoctmp{\childdocforward[#1]{#3##1}}}
    \expandafter\childdocextract\childdocname~~~
    \expandafter
  \endgroup
  \childdoctmp
}
%    \end{macrocode}

% \macro{\childdoc}
% The deprecated macro |\childdoc| is a legacy version of |\childdocmain|:
%    \begin{macrocode}
\newcommand{\childdoc}{\childdocmain}
%    \end{macrocode}

% \macro{\childdocredirect}
% The deprecated macro |\childdocredirect| is a legacy version
% of |\childdocforward| and |\childdocforwardprefix|:
%    \begin{macrocode}
\newcommand{\childdocredirect}[2][]
{
  \begingroup
    \if?#1?
      \def\childdoctmp{\childdocforward{#2}}
    \else
      \def\childdoctmp{\childdocforwardprefix{#1}{#2}}
    \fi
    \expandafter
  \endgroup
  \childdoctmp
}
%    \end{macrocode}

%\iffalse
%</package>
%\fi
%
\endinput

\childdocmain{}
%    \end{macrocode}

% Optional override for |\version| flag:
%    \begin{macrocode}
%%\ifchilddoc\else\providecommand{\version}{draft}\fi
%    \end{macrocode}

% Define the default values for the |\version| flag
% (|final| for the main file and |draft| for childs):
%    \begin{macrocode}
\ifchilddoc
\providecommand{\version}{draft}
\else
\providecommand{\version}{final}
\fi
%    \end{macrocode}

% Load the standard document class:
%    \begin{macrocode}
\documentclass[12pt]{article}
%    \end{macrocode}

% Start the document body:
%    \begin{macrocode}
\begin{document}
%    \end{macrocode}

% Declare a title page.
% Print title, part of document being processed and version flag:
%    \begin{macrocode}
\addtocounter{page}{-1}
\begin{center}
{\LARGE\bfseries{}childdoc example\par}
\vspace{1cm}
\ifchilddoc
\ifchilddocmanual part\else chapter\fi:
`\childdocname' of `\childdocjob'\par
\else
main document: `\childdocjob'\par
\fi
version: \version\par
\end{center}
\newpage
%    \end{macrocode}

% Manually include selected file,
% otherwise process as usual:
%    \begin{macrocode}
\ifchilddocmanual
\section*{part `\childdocname'}
\input{\childdocname}
\else
%    \end{macrocode}

% Include the two chapters:
%    \begin{macrocode}
\include{cdocsch1}
\include{cdocsch2}
%    \end{macrocode}

% Include the two parts unless only chapters should be displayed:
%    \begin{macrocode}
\ifchilddoc\else
\section{part three}
\input{cdocspt3}
\section{part four}
\input{cdocspt4}
\fi
%    \end{macrocode}

% Process as usual until here:
%    \begin{macrocode}
\fi
%    \end{macrocode}

% End of document body:
%    \begin{macrocode}
\end{document}
%    \end{macrocode}
%\iffalse
%</samplemain>
%\fi
%
% %%%%%%%%%%%%%%%%%%%%%%%%%%%%%%%%%%%%%%
% \paragraph{Chapter Include Files.}
%
% The include files are called |cdocsch1.tex| and |cdocsch2.tex|.
%
%\iffalse
%<*samplechap1|samplechap2>
%\fi

% Optional override for |\version| flag:
%    \begin{macrocode}
%%\providecommand{\version}{final}
%    \end{macrocode}

% Include the main document:
%    \begin{macrocode}
% \iffalse
%
% childdoc.dtx Copyright (C) 2017-2018 Niklas Beisert
%
% This work may be distributed and/or modified under the
% conditions of the LaTeX Project Public License, either version 1.3
% of this license or (at your option) any later version.
% The latest version of this license is in
%   http://www.latex-project.org/lppl.txt
% and version 1.3 or later is part of all distributions of LaTeX
% version 2005/12/01 or later.
%
% This work has the LPPL maintenance status `maintained'.
%
% The Current Maintainer of this work is Niklas Beisert.
%
% This work consists of the files childdoc.dtx and childdoc.ins
% and the derived files childdoc.def and cdocsamp.tex with
% cdocsch1.tex, cdocsch2.tex, cdocsdrf.tex, cdocsfn1.tex, cdocsfn2.tex.
%
%<package>\ifdefined\childdocmain\endinput\fi
%<package>\ProvidesFile{childdoc.def}[2018/12/30 v2.0 child document driver]
%<samplemain>\ProvidesFile{cdocsamp.tex}[2018/12/30 v2.0 sample for childdoc]
%<*driver>
%\ProvidesFile{childdoc.drv}[2018/12/30 v2.0 childdoc reference manual file]
\PassOptionsToClass{10pt,a4paper}{article}
\documentclass{ltxdoc}

\usepackage[margin=35mm]{geometry}
\usepackage{hyperref}
\usepackage{hyperxmp}
\usepackage[usenames]{color}

\hypersetup{colorlinks=true}
\hypersetup{pdfstartview=FitH}
\hypersetup{pdfpagemode=UseNone}
\hypersetup{pdfsource={}}
\hypersetup{pdflang={en-UK}}
\hypersetup{pdfcopyright={Copyright 2017-2018 Niklas Beisert.
  This work may be distributed and/or modified under the
  conditions of the LaTeX Project Public License, either version 1.3
  of this license or (at your option) any later version.}}
\hypersetup{pdflicenseurl={http://www.latex-project.org/lppl.txt}}
\hypersetup{pdfcontactaddress={ETH Zurich, ITP, HIT K,
  Wolfgang-Pauli-Strasse 27}}
\hypersetup{pdfcontactpostcode={8093}}
\hypersetup{pdfcontactcity={Zurich}}
\hypersetup{pdfcontactcountry={Switzerland}}
\hypersetup{pdfcontactemail={nbeisert@itp.phys.ethz.ch}}
\hypersetup{pdfcontacturl={http://people.phys.ethz.ch/\xmptilde nbeisert/}}

\newcommand{\secref}[1]{\hyperref[#1]{section \ref*{#1}}}

\parskip1ex
\parindent0pt
\let\olditemize\itemize
\def\itemize{\olditemize\parskip0pt}

\begin{document}

\title{The \textsf{childdoc} Package}
\hypersetup{pdftitle={The childdoc Package}}
\author{Niklas Beisert\\[2ex]
  Institut f\"ur Theoretische Physik\\
  Eidgen\"ossische Technische Hochschule Z\"urich\\
  Wolfgang-Pauli-Strasse 27, 8093 Z\"urich, Switzerland\\[1ex]
  \href{mailto:nbeisert@itp.phys.ethz.ch}
  {\texttt{nbeisert@itp.phys.ethz.ch}}}
\hypersetup{pdfauthor={Niklas Beisert}}
\hypersetup{pdfsubject={Manual for the LaTeX2e Package childdoc}}
\date{30 December 2018, \textsf{v2.0}}
\maketitle

\begin{abstract}\noindent
\textsf{childdoc} is a \LaTeXe{} package
that enables the direct compilation
of document sections included by |\include|
to individual files.
\end{abstract}

\begingroup
\parskip0ex
\tableofcontents
\endgroup

%%%%%%%%%%%%%%%%%%%%%%%%%%%%%%%%%%%%%%%%%%%%%%%%%%%%%%%%%%%%%%%%%%%%%%%%%%%%%%%%
%%%%%%%%%%%%%%%%%%%%%%%%%%%%%%%%%%%%%%%%%%%%%%%%%%%%%%%%%%%%%%%%%%%%%%%%%%%%%%%%
\section{Introduction}

\LaTeX{} provides a mechanism to structure a large document (such as a book)
into a main file and several child files (containing the chapters)
using the |\include| command.
This mechanism is beneficial for documents
which span hundreds of pages in order to
make the source file(s) more manageable.
Moreover, compilation can be restricted to
selected child files by means of the |\includeonly| command.
The latter feature can be used to reduce the compilation time while editing
(this was significantly more useful in the earlier days of \LaTeX{})
or to generate a smaller document which is easier to navigate.
Another application of |\includeonly| is to generate
documents consisting of selected parts of the complete document.

However, there are a few drawbacks of the plain |\include| mechanism:
\begin{itemize}
\item
The child files cannot be compiled on their own,
they can only be compiled via the main file.
A naive editing environment
(such as a text editor with an option
to have the current file processed by \LaTeX)
may require one to switch to the main file before compiling;
attempting to compile the child file produces errors.
\item
The main file must be modified (each time)
to adjust the |\includeonly| command
to the present needs. This easily leaves the main file in a messy state.
\item
The generated document will always carry the filename
of the main document. This is inconvenient if
several child files are to be compiled and
to be kept for distribution.
\end{itemize}

The present package provides a simple interface
to make child files individually compilable by \LaTeX{}.
Compiling a child file then has the same effect as compiling
the main file with an |\includeonly| command
to select the appropriate child.
Moreover the generated document will carry the name of the child
rather than the main file.
This resolves all three above issues.

This feature is meant to make the editing of books,
thesis documents and lecture notes somewhat more convenient.
However, the package can also be used efficiently for
composing a series of documents (such as exercise sheets)
which are typically distributed individually.
It then assists the author in generating the individual documents
(potentially in different versions)
as well as a document containing the collected series.
Another application is in developing style files
or other kinds of included material
where compilation of the style file could redirect
to a sample or test file.

%%%%%%%%%%%%%%%%%%%%%%%%%%%%%%%%%%%%%%%%%%%%%%%%%%%%%%%%%%%%%%%%%%%%%%%%%%%%%%%%
%%%%%%%%%%%%%%%%%%%%%%%%%%%%%%%%%%%%%%%%%%%%%%%%%%%%%%%%%%%%%%%%%%%%%%%%%%%%%%%%
\section{Usage}

First of all, the package \textsf{childdoc} is \emph{not} a standard
\LaTeXe{} |.sty| style file! Therefore it needs to be invoked in
a non-standard way.

%%%%%%%%%%%%%%%%%%%%%%%%%%%%%%%%%%%%%%%%%%%%%%%%%%%%%%%%%%%%%%%%%%%%%%%%%%%%%%%%
\subsection{Included Files}
\label{sec:include}

%%%%%%%%%%%%%%%%%%%%%%%%%%%%%%%%%%%%%%%%
\DescribeMacro{\childdocmain}
To use the package, add the commands
\begin{center}
\begin{tabular}{l}
|\input{childdoc.def}|\\
|\childdocmain{}|\\
\end{tabular}
\end{center}
at the very top of the main \LaTeX{} file,
in particular \emph{before} the |\documentclass| statement!
The argument of |\childdocmain| should be left empty
(but it must be present).

%%%%%%%%%%%%%%%%%%%%%%%%%%%%%%%%%%%%%%%%
\DescribeMacro{\childdocof}
Furthermore, add the commands
\begin{center}
\begin{tabular}{l}
|\input{childdoc.def}|\\
|\childdocof{|\textit{main}|}|\\
\end{tabular}
\end{center}
at the top of every child file \textit{child}
which is included by |\include{|\textit{child}|}|
from within the main file
(or at least for those files to be compiled individually).
The argument \textit{main} must be the filename of the main file.

There are a couple of
considerations in setting up the main and child documents:

%%%%%%%%%%%%%%%%%%%%%%%%%%%%%%%%%%%%%%%%
\paragraph{Restrictions.}

Please note the following restrictions:
\begin{itemize}
\item
|\childdocmain| must be called with one argument \textit{main}
to ensure compatibility with earlier version of the package.
It must either be empty (|\childdocmain{}|)
or precisely match the filename of the main file in which it is specified.
See \secref{sec:detection} for further information.
\item
The filename \textit{main} must be specified without the |.tex| extension.
\item
The filename \textit{main} is case sensitive
(even in case-insensitive file systems)
due to internal string comparison.
\item
The argument \textit{main} should be fully expanded, it cannot be a macro.
\item
Subdirectories and special characters should be avoided in filenames.
\item
The command |\childdocmain{|\textit{main}|}| must be followed by a whitespace.
It should not be followed immediately by another command
or by a comment mark `|%|'.
This is because the \TeX{} parser reads the token immediately following
the argument of |\childdocmain| and puts it
at the beginning of every child section;
however, a white\-space is ignored.
\end{itemize}

%%%%%%%%%%%%%%%%%%%%%%%%%%%%%%%%%%%%%%%%
\paragraph{Content of Main File.}

It is advisable to place all content in the child files included by |\include|.
Any output contained in the main file will appear in all child documents
unless suppressed manually;
it cannot be suppressed automatically by the |\includeonly| directive
and thus should normally be avoided.
A method to include some content in the main file
by means of conditional processing is described in \secref{sec:conditional}.

%%%%%%%%%%%%%%%%%%%%%%%%%%%%%%%%%%%%%%%%
\paragraph{Page Numbering.}

When only a part of the document is compiled,
the appropriate numbering of pages
(as well as other status parameters)
is determined from the |.aux| files.
The latter contain information from previous passes.
However this information needs to propagate through
all intermediate child documents.
Therefore the page numbering in child documents may well
be inconsistent until the complete document is compiled at least once.

A useful (if unconventional) way to always ensure a consistent
page numbering is to restart the numbering in each child document
and denote the pages by `\textit{child}|.|\textit{page}'
where \textit{child} represents the chapter/section number of the child file.
This can be achieved by the command
|\numberwithin{page}{|\textit{child}|}|
of the \textsf{amsmath} package
where \textit{child} can be |chapter| or |section|
depending on the chosen structuring.
Alternatively, one can modify the macro |\thepage| appropriately
and reset the counter |page| at the start of each child file.

%%%%%%%%%%%%%%%%%%%%%%%%%%%%%%%%%%%%%%%%%%%%%%%%%%%%%%%%%%%%%%%%%%%%%%%%%%%%%%%%
\subsection{Conditional Processing}
\label{sec:conditional}

The package provides a mechanism to compile different versions
of a document. To customise the versions further some conditional processing
can come in handy to distinguish which version is being compiled.
The package provides two macros to describe the compilation context:

%%%%%%%%%%%%%%%%%%%%%%%%%%%%%%%%%%%%%%%%
\DescribeMacro{\ifchilddoc}
The conditional |\ifchilddoc| distinguishes between the compilation of
child documents and the main document:
%
\begin{center}
|\ifchilddoc |\textit{child-code}| |[|\||else |\textit{main-code}]| \||fi|
\end{center}

%%%%%%%%%%%%%%%%%%%%%%%%%%%%%%%%%%%%%%%%
\DescribeMacro{\childdocname}
\DescribeMacro{\childdocjob}
The macro |\childdocname| contains the filename (without extension)
of the main or child file being processed.
Note that |\childdocjob| will always contain the name of the main file.

%%%%%%%%%%%%%%%%%%%%%%%%%%%%%%%%%%%%%%%%
\paragraph{Title Page.}

Conditional processing can be used to include a title or banner page
in the main document when proper precautions are taken.
Importantly, the code in the main file should ensure that the page counter
(as well as other status parameters which are stored in the |.aux| files)
takes the same value after the conditional processing.
Otherwise the page numbers may take divergent values
depending on which part is compiled.

For example, a title page could be declared by:
%
\begin{center}
\begin{tabular}{l}
|\ifchilddoc\||else|\\
|\addtocounter{page}{-1}|\\
\textit{code for title page}\\
|\newpage|\\
|\||fi|
\end{tabular}
\end{center}
%
A banner page for the child documents can be generated by:
%
\begin{center}
\begin{tabular}{l}
|\ifchilddoc|\\
|\addtocounter{page}{-1}|\\
\textit{code for banner page}\\
|\newpage|\\
|\||fi|
\end{tabular}
\end{center}
%
Here one could write a message such as:
\begin{center}
|This is the part \childdocname{} of \childdocjob{}.|
\end{center}

%%%%%%%%%%%%%%%%%%%%%%%%%%%%%%%%%%%%%%%%%%%%%%%%%%%%%%%%%%%%%%%%%%%%%%%%%%%%%%%%
\subsection{Flags}
\label{sec:flags}

The package makes it easy to generate different versions
of the main or child documents.
To this end compilation flags can be defined
and assigned different default values.
They will be particularly useful in conjunction
with the forwarding mechanism described in \secref{sec:forward}.

For example, it may be useful to have a flag |\version|
which can be set to |draft| or |final|.
The document source will contain some conditional code
depending on the value of |\version|.
Suppose further, the flag should default to |final| for the main file
and to |draft| for child files
which is a natural assignment for editing the document.
This is achieved by placing the following code
in the preamble of the main document
(below the |\childdocmain| directive):
%
\begin{center}
\begin{tabular}{l}
|\ifchilddoc|\\
|\providecommand{\version}{draft}|\\
|\||else|\\
|\providecommand{\version}{final}|\\
|\||fi|
\end{tabular}
\end{center}
%
The definition by |\providecommand| makes sure
that previous definitions are not overwritten.
Further statements |\providecommand{\version}{...}|
can thus be added before the above code to override it.

For the main file, one might add a line
(between |\childdocmain| and the above block)
%
\begin{center}
|%\ifchilddoc\||else\providecommand{\version}{draft}\||fi|
\end{center}
%
which can be uncommented to produce a draft version.
Likewise one can add a line to the very top of a child file
(above the |\childdocof{|\textit{main}|}| directive)
%
\begin{center}
|%\providecommand{\version}{final}|
\end{center}
%
which can be uncommented to produce the final version of this child document.

%%%%%%%%%%%%%%%%%%%%%%%%%%%%%%%%%%%%%%%%%%%%%%%%%%%%%%%%%%%%%%%%%%%%%%%%%%%%%%%%
\subsection{Forwarding}
\label{sec:forward}

Different versions of the main or child documents
using compilation flags as described in \secref{sec:flags}
can be (permanently) stored in different files
for convenient compilation, viewing and distribution.
To this end, the package defines a command
to pass on compilation to a different file:

%%%%%%%%%%%%%%%%%%%%%%%%%%%%%%%%%%%%%%%%
\DescribeMacro{\childdocforward}
The command |\childdocforward| redirects processing to
another source file:
%
\begin{center}
\begin{tabular}{l}
|\input{childdoc.def}|\\
|\childdocforward[|\textit{main}|]{|\textit{dest}|}|\\
\end{tabular}
\end{center}
%
The argument \textit{dest} is the destination file
(without extension).
It should be the main file or one of the child files.
Note that further \textsf{childdoc} directives
such as |\childdocof| and |\childdocforward|
in the indicated file will be processed in this form.
The optional argument \textit{main}
passes on directly to the main file \textit{main}
while pretending to compile the child \textit{dest}.
This form behaves as if \textit{dest}
issues |\childdocof{|\textit{main}|}| right away,
and no further \textsf{childdoc} directives will be processed.

%%%%%%%%%%%%%%%%%%%%%%%%%%%%%%%%%%%%%%%%
\DescribeMacro{\...prefix}
In the alternative form |\childdocforwardprefix|,
%
\begin{center}
\begin{tabular}{l}
|\input{childdoc.def}|\\
|\childdocforwardprefix[|\textit{main}|]{|\textit{prefix}|}{|\textit{dest}|}|
\end{tabular}
\end{center}
%
the destination file is determined by a pattern
depending on the current file:
To make this work, the current file must be called
`{\textit{prefix}\hspace{0.2em}\textit{suffix}}'
with \textit{prefix} matching precisely the argument.
Processing is then passed on to the file
`{\textit{dest}\hspace{0.2em}\textit{suffix}}'.
Surely, the same effect is achieved by
directly specifying the
argument `{\textit{dest}\hspace{0.2em}\textit{suffix}}'
in the first form.
However, that requires to set up a different file
for each child. With the alternative form of the command
all these files can have exactly the same content
which simplifies setting them up and maintaining them.

For example, the following file |draft.tex|
with a compilation flag |\version| as described in \secref{sec:flags}
compiles the main document as a draft:
%
\begin{center}
\begin{tabular}{l}
|\def\version{draft}|\\
|\input{childdoc.def}|\\
|\childdocforward{|\textit{main}|}|
\end{tabular}
\end{center}
%
Likewise, the following files |final|\textit{nn}|.tex|
compile the final version of the child document
|child|\textit{nn}|.tex|:
%
\begin{center}
\begin{tabular}{l}
|\def\version{final}|\\
|\input{childdoc.def}|\\
|\childdocforwardprefix{final}{child}|
\end{tabular}
\end{center}
%

Note that when several versions of a main file and/or of each child file
are to be generated, it may be convenient to set up a |Makefile| or
shell script to automatise the process.

%%%%%%%%%%%%%%%%%%%%%%%%%%%%%%%%%%%%%%%%%%%%%%%%%%%%%%%%%%%%%%%%%%%%%%%%%%%%%%%%
\subsection{Command Line Processing}
\label{sec:commandline}

The effect of redirection files can also be achieved by invoking
the \LaTeX{} compiler with a more elaborate command line.
Most conveniently this should be done as part
of a shell script or a |Makefile|.

When using \textsf{childdoc} in the main file, the following
command lines effectively perform a redirection
(note that depending on the shell being used,
backslashes may have to be doubled: `|\|' $\to$ `|\\|'):
%
\begin{center}
|... -jobname "|\textit{target}|" |\\|"|[\textit{flags}]%
|\input{childdoc.def}\childdocforward[|\textit{main}|]{|\textit{dest}|}"|
\end{center}
%
Here \textit{target} is the name of the output file,
\textit{main} is the name of the main file
and \textit{dest} is the name of the main or child file to be processed
(all filenames without extensions).
The optional argument \textit{main} can be omitted
if \textit{main} matches \textit{dest}.
Optionally, compilation \textit{flags} can be defined via |\def| commands.
This command line makes the \TeX{} engine believe
it is compiling the file \textit{target}
whose content is specified as the latter parameter.
The provided code then forwards the processing to
\textit{main} or \textit{dest} as described in \secref{sec:forward}.

%%%%%%%%%%%%%%%%%%%%%%%%%%%%%%%%%%%%%%%%%%%%%%%%%%%%%%%%%%%%%%%%%%%%%%%%%%%%%%%%
\subsection{Include by Input}
\label{sec:input}

Including child documents by |\include| has some restrictions by design.
Most notably, the content of a child document always occupies
its own set of pages; pages cannot be shared between child documents.
Usually, this behaviour makes perfect sense
because each child document contain an essential part of the document.
However, in some situations it may be desirable to compose
a document from a collection of parts
without having mandatory page breaks between then.
For this case, the package
provides a mechanism to include parts
by |\input| which can also be processed individually.
However, by construction this mechanism
requires manual handling of the content to be output.

%%%%%%%%%%%%%%%%%%%%%%%%%%%%%%%%%%%%%%%%
\DescribeMacro{\ifchilddocmanual}
The main file should be prepared as usual, see \secref{sec:include}.
However, the document body must make a distinction
between processing of an individual part and of the main document, e.g.:
%
\begin{center}
\begin{tabular}{l}
|\ifchilddocmanual|\\
|\input{\childdocname}|\\
|\||else|\\
\textit{document body with }|\input{|\textit{part}|}|\\
|\||fi|
\end{tabular}
\end{center}
%
The conditional |\ifchilddocmanual| is true whenever
a part to be included by |\input| is being compiled,
and the name of the part is stored in |\childdocname|.

%%%%%%%%%%%%%%%%%%%%%%%%%%%%%%%%%%%%%%%%
\DescribeMacro{\childdocby}
Each part to be included by |\input| should start with:
%
\begin{center}
\begin{tabular}{l}
|\input{childdoc.def}|\\
|\childdocby{|\textit{main}|}|\\
\end{tabular}
\end{center}
%
The directive |\childdocby| is similar to |\childdocof|
described in \secref{sec:include},
but the subsequent selection of content must be done manually.
To that end, both |\ifchilddoc| and |\ifchilddocmanual|
will be true upon processing of a part,
and the name of the part is stored in |\childdocname|.
Note that |\jobname| will be set to the filename of the current part
so that each part receives an individual |.aux| file
that does not interfere with the |.aux| file(s) of the main document.
This behaviour can be altered by the alternative form
|\childdocby[*]{|\textit{main}|}| (with a non-empty optional argument)
which uses the |.aux| file of the main document
by setting |\jobname| to \textit{main}.

%%%%%%%%%%%%%%%%%%%%%%%%%%%%%%%%%%%%%%%%%%%%%%%%%%%%%%%%%%%%%%%%%%%%%%%%%%%%%%%%
\subsection{Driver Development}
\label{sec:driver}

The \textsf{childdoc} mechanism can also be use for the development
of definition files such as \LaTeX{} styles or classes.
This case differs from the above setup with multiple parts
included by |\include| in that no |\includeonly| should be invoked.
This can be achieved by starting the include file
(before |\ProvidesPackage|) with:
%
\begin{center}
\begin{tabular}{l}
|\input{childdoc.def}|\\
|\childdocforward{|\textit{main}|}|\\
\end{tabular}
\end{center}
%
or alternatively with:
%
\begin{center}
\begin{tabular}{l}
|\input{childdoc.def}|\\
|\childdocby{|\textit{main}|}|\\
\end{tabular}
\end{center}
%
Both forms have slightly different effects as described above.
The main file is prepared as usual, see \secref{sec:include}.

%%%%%%%%%%%%%%%%%%%%%%%%%%%%%%%%%%%%%%%%%%%%%%%%%%%%%%%%%%%%%%%%%%%%%%%%%%%%%%%%
\subsection{Legacy Detection}
\label{sec:detection}

The directive |\childdocmain| in the main file can detect
whether the complete document or merely a child is to be compiled
even without using the directive |\childdocof|.
This method is deprecated because it is less robust
and there is no compelling reason to use it;
it is merely provided for backward compatibility
and it may be removed in future versions.

If the detection mechanism is to be used,
it is mandatory to correctly specify
the filename of the main file as the argument of |\childdocmain|:
%
\begin{center}
\begin{tabular}{l}
|\input{childdoc.def}|\\
|\childdocmain{|\textit{main}|}|\\
\end{tabular}
\end{center}
%
If |\jobname| does not match the argument \textit{main} of |\childdocmain|,
it is assumed that |\jobname| points to the child file to be compiled.
When using |\childdocmain| with the main file specified as argument,
it suffices to start a child file
with just |\input{|\textit{main}|}|
without loading of the package and using |\childdocof|.
If instead all processing is done
with the appropriate \textsf{childdoc} directives,
the argument of \textit{main} of |\childdocmain| can be empty.

An alternative version of the command line processing described
in \secref{sec:commandline} using the detection mechanism reads:
%
\begin{center}
|... -jobname "|\textit{target}|" "|[\textit{flags}]%
[|\def\jobname{|\textit{dest}|}|]|\input{|\textit{main}|}"|
\end{center}

%%%%%%%%%%%%%%%%%%%%%%%%%%%%%%%%%%%%%%%%%%%%%%%%%%%%%%%%%%%%%%%%%%%%%%%%%%%%%%%%
\subsection{Manual Code}
\label{sec:manual}

In case one cannot be certain whether the definitions file |childdoc.def|
is installed on the target \TeX{} distribution
and one prefers not to ship it,
it is conceivable to paste a few relevant commands into the sources.

To that end, drop all statements |\input{childdoc.def}|
and perform the replacements as outlined below.
Instead of |\childdocmain{|\textit{main}|}| add the following code
to the top of the main file:
%
\begin{center}
\begin{tabular}{l}
|\||ifdefined\childdocname\endinput\||fi\newif\ifchilddoc|\\
|\edef\childdocname{\scantokens\expandafter{\jobname\noexpand}}|\\
|\def\childdocmain{|\textit{main}|}\||ifx\childdocmain\childdocname\||else|\\
|\childdoctrue\includeonly{\childdocname}\let\jobname\childdocmain\||fi|\\
\end{tabular}
\end{center}
%
Instead of |\childdocof{|\textit{main}|}| just include the main file
at the top of each child file:
%
\begin{center}
|\input{|\textit{main}|}|
\end{center}
%
A simple redirection |\childdocforward{|\textit{dest}|}| is achieved by:
%
\begin{center}
|\def\jobname{|\textit{dest}|}\input{\jobname}|
\end{center}
%
The redirection with prefix
|\childdocforwardprefix[|\textit{prefix}|]{|\textit{dest}|}|
is accomplished by:
%
\begin{center}
\begin{tabular}{l}
|{\edef\jobname{\scantokens\expandafter{\jobname\noexpand}}|\\
|\def\redirectjob |\textit{prefix}|#1~~~{\gdef\jobname{|\textit{dest}|#1}}|\\
|\expandafter\redirectjob\jobname~~~}\input{\jobname}|
\end{tabular}
\end{center}

In an alternative approach,
child documents can be compiled by a specific command line
without additional code or specific definitions:
%
\begin{center}
|... -jobname "|\textit{target}|" "|[\textit{flags}]%
|\includeonly{|\textit{dest}|}\input{|\textit{main}|}"|
\end{center}
%

%%%%%%%%%%%%%%%%%%%%%%%%%%%%%%%%%%%%%%%%%%%%%%%%%%%%%%%%%%%%%%%%%%%%%%%%%%%%%%%%
%%%%%%%%%%%%%%%%%%%%%%%%%%%%%%%%%%%%%%%%%%%%%%%%%%%%%%%%%%%%%%%%%%%%%%%%%%%%%%%%
\section{Information}

%%%%%%%%%%%%%%%%%%%%%%%%%%%%%%%%%%%%%%%%%%%%%%%%%%%%%%%%%%%%%%%%%%%%%%%%%%%%%%%%
\subsection{Copyright}

Copyright \copyright{} 2017--2018 Niklas Beisert

This work may be distributed and/or modified under the
conditions of the \LaTeX{} Project Public License, either version 1.3
of this license or (at your option) any later version.
The latest version of this license is in
  \url{http://www.latex-project.org/lppl.txt}
and version 1.3 or later is part of all distributions of \LaTeX{}
version 2005/12/01 or later.

This work has the LPPL maintenance status `maintained'.

The Current Maintainer of this work is Niklas Beisert.

This work consists of the files |README.txt|, |childdoc.ins| and |childdoc.dtx|
as well as the derived files |childdoc.def|, |cdocsamp.tex|
with |cdocsch1.tex|, |cdocsch2.tex|, |cdocspt3.tex|, |cdocspt4.tex|,
|cdocsdrf.tex|, |cdocsfn1.tex|, |cdocsfn2.tex|
as well as |childdoc.pdf|.

%%%%%%%%%%%%%%%%%%%%%%%%%%%%%%%%%%%%%%%%%%%%%%%%%%%%%%%%%%%%%%%%%%%%%%%%%%%%%%%%
\subsection{Files and Installation}

The package consists of the files:
%
\begin{center}
\begin{tabular}{ll}
    |README.txt|   & readme file \\
    |childdoc.ins| & installation file \\
    |childdoc.dtx| & source file \\
    |childdoc.def| & definition file \\
    |cdocsamp.tex| & sample main file \\
    |cdocsch1.tex| & sample include file \\
    |cdocsch2.tex| & sample include file \\
    |cdocspt3.tex| & sample part file \\
    |cdocspt4.tex| & sample part file \\
    |cdocsdrf.tex| & sample redirection file \\
    |cdocsfn1.tex| & sample redirection file \\
    |cdocsfn2.tex| & sample redirection file \\
    |childdoc.pdf| & manual
\end{tabular}
\end{center}
%
The distribution consists of the files
|README.txt|, |childdoc.ins| and |childdoc.dtx|.
%
\begin{itemize}
\item
Run (pdf)\LaTeX{} on |childdoc.dtx|
to compile the manual |childdoc.pdf| (this file).
\item
Run \LaTeX{} on |childdoc.ins| to create the definitions file |childdoc.def|
and the sample |cdocsamp.tex| with include files
|cdocsch1.tex|, |cdocsch2.tex|, |cdocspt3.tex|, |cdocspt4.tex|,
|cdocsdrf.tex|, |cdocsfn1.tex|, |cdocsfn2.tex|.
Then copy the file |childdoc.def| to an appropriate directory of your \LaTeX{}
distribution, e.g.\ \textit{texmf-root}|/tex/latex/childdoc|.
\end{itemize}

%%%%%%%%%%%%%%%%%%%%%%%%%%%%%%%%%%%%%%%%%%%%%%%%%%%%%%%%%%%%%%%%%%%%%%%%%%%%%%%%
\subsection{Related CTAN Packages}

There are several other packages which offer a similar functionality:
%
\begin{itemize}
\item
The packages
\href{http://ctan.org/pkg/docmute}{\textsf{docmute}},
\href{http://ctan.org/pkg/includex}{\textsf{includex}} and
\href{http://ctan.org/pkg/standalone}{\textsf{standalone}}
provide commands to include only the document body of
a child file thus allowing both files to be compiled individually.
\item
The packages \href{http://ctan.org/pkg/subdocs}{\textsf{subdocs}}
and \href{http://ctan.org/pkg/subfiles}{\textsf{subfiles}}
provide structures in which the main and child documents can be
encapsulated and allowing them to be compiled individually.
The inclusion mechanism is different from the conventional |\include|.
\item
The package \href{http://ctan.org/pkg/combine}{\textsf{combine}}
is an elaborate solution to combine several documents into one.
\end{itemize}
%
See also the CTAN topic \href{http://ctan.org/topic/subdocs}{\textsf{subdocs}}
for further related packages.
The present package differs from the above solutions in that
a document structure constructed with the conventional |\include| mechanism
just needs two extra commands at the top of every file
such that all constituent files can be compiled individually.

%%%%%%%%%%%%%%%%%%%%%%%%%%%%%%%%%%%%%%%%%%%%%%%%%%%%%%%%%%%%%%%%%%%%%%%%%%%%%%%%
%\subsection{Feature Suggestions}
%
%The following is a list of features which may be useful for future
%versions of this package:
%%
%\begin{itemize}
%\item
%\ldots
%\end{itemize}

%%%%%%%%%%%%%%%%%%%%%%%%%%%%%%%%%%%%%%%%%%%%%%%%%%%%%%%%%%%%%%%%%%%%%%%%%%%%%%%%
\subsection{Revision History}

%%%%%%%%%%%%%%%%%%%%%%%%%%%%%%%%%%%%%%%%
\paragraph{v2.0:} 2018/12/30

\begin{itemize}
\item
immediate forward processing
\item
added |\childdocby| mechanism
\item
manual restructured
\end{itemize}

%%%%%%%%%%%%%%%%%%%%%%%%%%%%%%%%%%%%%%%%
\paragraph{v1.6:} 2018/01/17

\begin{itemize}
\item
application for development of include files
\item
corrections to manual
\end{itemize}

%%%%%%%%%%%%%%%%%%%%%%%%%%%%%%%%%%%%%%%%
\paragraph{v1.5:} 2017/05/21

\begin{itemize}
\item
more complete structuring introduced
\item
|\childdocof| introduced
\item
|\childdoc| renamed to |\childdocmain|
\item
|\childredirect| renamed to |\childdocforward| and |\childdocforwardprefix|
and functionality expanded
\end{itemize}

%%%%%%%%%%%%%%%%%%%%%%%%%%%%%%%%%%%%%%%%
\paragraph{v1.0:} 2017/04/27

\begin{itemize}
\item
manual and install package
\item
first version published on CTAN
\end{itemize}

%%%%%%%%%%%%%%%%%%%%%%%%%%%%%%%%%%%%%%%%
\paragraph{v0.6:} 2017/04/26

\begin{itemize}
\item
redirection mechanism added
\end{itemize}

%%%%%%%%%%%%%%%%%%%%%%%%%%%%%%%%%%%%%%%%
\paragraph{v0.5:} 2017/04/26

\begin{itemize}
\item
functionality in definition file
\end{itemize}


%%%%%%%%%%%%%%%%%%%%%%%%%%%%%%%%%%%%%%%%%%%%%%%%%%%%%%%%%%%%%%%%%%%%%%%%%%%%%%%%
%%%%%%%%%%%%%%%%%%%%%%%%%%%%%%%%%%%%%%%%%%%%%%%%%%%%%%%%%%%%%%%%%%%%%%%%%%%%%%%%
%%%%%%%%%%%%%%%%%%%%%%%%%%%%%%%%%%%%%%%%%%%%%%%%%%%%%%%%%%%%%%%%%%%%%%%%%%%%%%%%
\appendix

\settowidth\MacroIndent{\rmfamily\scriptsize 000\ }

 \DocInput{childdoc.dtx}

\end{document}
%</driver>
% \fi
%
% %%%%%%%%%%%%%%%%%%%%%%%%%%%%%%%%%%%%%%%%%%%%%%%%%%%%%%%%%%%%%%%%%%%%%%%%%%%%%%
% %%%%%%%%%%%%%%%%%%%%%%%%%%%%%%%%%%%%%%%%%%%%%%%%%%%%%%%%%%%%%%%%%%%%%%%%%%%%%%
% \section{Sample}
%\iffalse
%<*samplemain>
%\fi
%
% The following presents a sample document
% with two chapters, two parts, a title page,
% a compile flag as well as three forwarding files to set the flag.
% It consists of eight |.tex| files:
% \begin{center}
% \begin{tabular}{ll}
% |cdocsamp.tex|&main file\\
% |cdocsch1.tex|&include file for chapter 1\\
% |cdocsch2.tex|&include file for chapter 2\\
% |cdocspt3.tex|&include file for part 3\\
% |cdocspt4.tex|&include file for part 4\\
% |cdocsdrf.tex|&forwarding file for main file in draft mode\\
% |cdocsfi1.tex|&forwarding file for final version of chapter 1\\
% |cdocsfi2.tex|&forwarding file for final version of chapter 2\\
% \end{tabular}
% \end{center}
% Each of the eight files can be compiled directly by the \LaTeX{} compiler.
%
% %%%%%%%%%%%%%%%%%%%%%%%%%%%%%%%%%%%%%%
% \paragraph{Main File.}
%
% The main file is called |cdocsamp.tex|.
%
% Load the \textsf{childdoc} definitions and
% declare the filename for the main document:
%    \begin{macrocode}
\input{childdoc.def}
\childdocmain{}
%    \end{macrocode}

% Optional override for |\version| flag:
%    \begin{macrocode}
%%\ifchilddoc\else\providecommand{\version}{draft}\fi
%    \end{macrocode}

% Define the default values for the |\version| flag
% (|final| for the main file and |draft| for childs):
%    \begin{macrocode}
\ifchilddoc
\providecommand{\version}{draft}
\else
\providecommand{\version}{final}
\fi
%    \end{macrocode}

% Load the standard document class:
%    \begin{macrocode}
\documentclass[12pt]{article}
%    \end{macrocode}

% Start the document body:
%    \begin{macrocode}
\begin{document}
%    \end{macrocode}

% Declare a title page.
% Print title, part of document being processed and version flag:
%    \begin{macrocode}
\addtocounter{page}{-1}
\begin{center}
{\LARGE\bfseries{}childdoc example\par}
\vspace{1cm}
\ifchilddoc
\ifchilddocmanual part\else chapter\fi:
`\childdocname' of `\childdocjob'\par
\else
main document: `\childdocjob'\par
\fi
version: \version\par
\end{center}
\newpage
%    \end{macrocode}

% Manually include selected file,
% otherwise process as usual:
%    \begin{macrocode}
\ifchilddocmanual
\section*{part `\childdocname'}
\input{\childdocname}
\else
%    \end{macrocode}

% Include the two chapters:
%    \begin{macrocode}
\include{cdocsch1}
\include{cdocsch2}
%    \end{macrocode}

% Include the two parts unless only chapters should be displayed:
%    \begin{macrocode}
\ifchilddoc\else
\section{part three}
\input{cdocspt3}
\section{part four}
\input{cdocspt4}
\fi
%    \end{macrocode}

% Process as usual until here:
%    \begin{macrocode}
\fi
%    \end{macrocode}

% End of document body:
%    \begin{macrocode}
\end{document}
%    \end{macrocode}
%\iffalse
%</samplemain>
%\fi
%
% %%%%%%%%%%%%%%%%%%%%%%%%%%%%%%%%%%%%%%
% \paragraph{Chapter Include Files.}
%
% The include files are called |cdocsch1.tex| and |cdocsch2.tex|.
%
%\iffalse
%<*samplechap1|samplechap2>
%\fi

% Optional override for |\version| flag:
%    \begin{macrocode}
%%\providecommand{\version}{final}
%    \end{macrocode}

% Include the main document:
%    \begin{macrocode}
\input{childdoc.def}
\childdocof{cdocsamp}
%    \end{macrocode}

%\iffalse
%</samplechap1|samplechap2>
%\fi
%
%\iffalse
%<*samplechap1>
%\fi
% Some text for chapter 1:
%    \begin{macrocode}
\section{one}
some text in chapter one
%    \end{macrocode}

%\iffalse
%</samplechap1>
%\fi
% Some text for chapter 2:
%\iffalse
%<*samplechap2>
%\fi
%    \begin{macrocode}
\section{two}
more text in chapter two
%    \end{macrocode}

%\iffalse
%</samplechap2>
%\fi
%
% %%%%%%%%%%%%%%%%%%%%%%%%%%%%%%%%%%%%%%
% \paragraph{Part Include Files.}
%
% The include files are called |cdocspt3.tex| and |cdocspt4.tex|.
%
%\iffalse
%<*samplepart3|samplepart4>
%\fi

% Optional override for |\version| flag:
%    \begin{macrocode}
%%\providecommand{\version}{final}
%    \end{macrocode}

% Include the main document:
%    \begin{macrocode}
\input{childdoc.def}
\childdocby{cdocsamp}
%    \end{macrocode}

%\iffalse
%</samplepart3|samplepart4>
%\fi
%
%\iffalse
%<*samplepart3>
%\fi
% Some text for part 3:
%    \begin{macrocode}
some text in part three
%    \end{macrocode}

%\iffalse
%</samplepart3>
%\fi
% Some text for part 4:
%\iffalse
%<*samplepart4>
%\fi
%    \begin{macrocode}
more text in part four
%    \end{macrocode}

%\iffalse
%</samplepart4>
%\fi
%
% %%%%%%%%%%%%%%%%%%%%%%%%%%%%%%%%%%%%%%
% \paragraph{Forwarding for a Complete Draft.}
%
% The following forwarding file |cdocsdrf.tex|
% compiles the main document in draft mode:
%\iffalse
%<*sampledraft>
%\fi
%    \begin{macrocode}
\def\version{draft}
\input{childdoc.def}
\childdocforward{cdocsamp}
%    \end{macrocode}

%\iffalse
%</sampledraft>
%\fi
%
% %%%%%%%%%%%%%%%%%%%%%%%%%%%%%%%%%%%%%%
% \paragraph{Forwarding for Final Version of the Chapters.}
%
% The following forwarding files |cdocsfn1.tex| and |cdocsfn2.tex|
% (with identical content)
% compile the final versions of the child documents
% |cdocsch1.tex| and |cdocsch2.tex|, respectively:
%\iffalse
%<*samplefinal>
%\fi
%    \begin{macrocode}
\def\version{final}
\input{childdoc.def}
\childdocforwardprefix[cdocsamp]{cdocsfn}{cdocsch}
%    \end{macrocode}

%\iffalse
%</samplefinal>
%\fi
%
% %%%%%%%%%%%%%%%%%%%%%%%%%%%%%%%%%%%%%%
% \paragraph{Command Line Processing.}
%
% The following three command lines generate the output files
% |cdocscld|, |cdocscl1| and |cdocscl2|
% which should be identical to
% |cdocsdrf|, |cdocsch1| and |cdocsfn2|, respectively:
% \begin{center}
% \begin{tabular}{l}
% |latex -jobname cdocscld \|\\
% |  "\def\version{draft}\input{childdoc.def}\childdocforward{cdocsamp}"|\\
% |latex -jobname cdocscl1 \|\\
% |  "\input{childdoc.def}\childdocforward[cdocsamp]{cdocsch1}"|\\
% |latex -jobname cdocscl2 \|\\
% |  "\def\version{final}\input{childdoc.def}\childdocforward{cdocsch2}"|
% \end{tabular}
% \end{center}
% Note that the trailing backslash on each first line
% merely continues the input to the second line
% (for convenient cut ant paste).
% Furthermore, the command |latex| can be replaced by any
% of its alternative versions such as |pdflatex|.
%
% %%%%%%%%%%%%%%%%%%%%%%%%%%%%%%%%%%%%%%%%%%%%%%%%%%%%%%%%%%%%%%%%%%%%%%%%%%%%%%
% %%%%%%%%%%%%%%%%%%%%%%%%%%%%%%%%%%%%%%%%%%%%%%%%%%%%%%%%%%%%%%%%%%%%%%%%%%%%%%
% \section{Implementation}
%\iffalse
%<*package>
%\fi
%
% This section describes the definitions file |childdoc.def|.

% The definitions cannot be loaded using |\usepackage| or |\RequirePackage|
% which has a mechanism to prevent loading a style file more than once.
% When loading the definitions by means of |\input|
% multiple instances have to be prevented manually:
%\iffalse
%This code needs to be before the `\ProvidesFile' directive
%which is defined at the beginning of this file.
%Therefore it is also placed there and commented out here.
%</package>
%<*discard>
%\fi
%    \begin{macrocode}
\ifdefined\childdocmain\endinput\fi
%    \end{macrocode}
%\iffalse
%</discard>
%<*package>
%\fi
%
% \macro{\ifchilddoc}
% \macro{\ifchilddocmanual}
% The conditional |\ifchilddoc| tells whether a
% child (true) or main (false) document is being compiled.
% The conditional |\ifchilddocmanual| tells whether
% the |\includeonly| mechanism is used (false) or
% the selection of child files must be performed manually (true).
% The definitions initialise to false:
%    \begin{macrocode}
\newif\ifchilddoc
\newif\ifchilddocmanual
%    \end{macrocode}

% \macro{\childdocname}
% \macro{\childdocjob}
% The macro |\childdocname| stores the name of the main document
% to be compiled. The macro |\childdocjob| stores the name of
% the document on which the \LaTeX{} compiler was originally invoked.
% The content of |\jobname| cannot be compared
% to filenames specified in the source due to different catcodes.
% The following code rescans |\jobname|, stores the result
% in |\childdocname| and saves a copy in |\childdocjob|:
%    \begin{macrocode}
\edef\childdocname{\scantokens\expandafter{\jobname\noexpand}}
\let\childdocjob\childdocname
%    \end{macrocode}

% \macro{\childdocdisable}
% The macro |\childdocdisable| prevents the main file
% from being processed more than once.
% At this stage, the main document command |\childdocmain|
% is assumed to be called once again where it should do nothing.
% Any subsequent call to it should prevent
% a secondary processing of the main document
% It overwrites the forwarding commands
% |\childdocof| and |\childdocforward|
% with empty macros to prevent further inclusions of the main document:
%    \begin{macrocode}
\newcommand{\childdocdisable}
{
  \renewcommand{\childdocmain}[1]{\renewcommand{\childdocmain}[1]{\endinput}}
  \renewcommand{\childdocof}[1]{}
  \renewcommand{\childdocby}[2][]{}
  \renewcommand{\childdocforward}[2][]{}
  \renewcommand{\childdocdisable}{}
}
%    \end{macrocode}

% \macro{\childdocmain}
% The macro |\childdocmain| is to be called at the top of the main file
% with nothing or the main filename (without extension) as argument.
% First, it breaks loops.
% If the argument is not empty and does not match |\childdocname|
% (which is set by the first inclusion of |childdoc.def|),
% |\ifchilddoc| is set to true, |\includeonly| is applied to the child file
% and |\jobname| is set to the main file
% (for proper handling of |.aux| files):
%    \begin{macrocode}
\newcommand{\childdocmain}[1]
{
  \childdocdisable\childdocmain{}
  \if?#1?\else
    \begingroup
      \def\childdoctmp{#1}
      \ifx\childdoctmp\childdocname
        \def\childdoctmp{}
      \else
        \def\childdoctmp
        {
          \childdoctrue
          \includeonly{\childdocname}
          \def\childdocjob{#1}
          \def\jobname{#1}
        }
      \fi
      \expandafter
    \endgroup
    \childdoctmp
  \fi
}
%    \end{macrocode}

% \macro{\childdocof}
% The command |\childdocof| redirects
% compilation to the main file |#1|.
%    \begin{macrocode}
\newcommand{\childdocof}[1]
{
  \childdocdisable
  \childdoctrue
  \includeonly{\childdocname}
  \def\jobname{#1}
  \def\childdocjob{#1}
  \input{#1}
}
%    \end{macrocode}

% \macro{\childdocby}
% The command |\childdocby| ....
%    \begin{macrocode}
\newcommand{\childdocby}[2][]
{
  \childdocdisable
  \childdoctrue
  \childdocmanualtrue
  \if?#1?\else
    \def\jobname{#2}
  \fi
  \def\childdocjob{#2}
  \input{#2}
  \endinput
}
%    \end{macrocode}

% \macro{\childdocforward}
% The command |\childdocforward| redirects
% compilation to the main file or
% (if the optional argument is given) a child file.
% Parameters are set as if the main file
% or a child file starting with |\childdocof| was compiled.
% Then compilation is handed over to the main file:
%    \begin{macrocode}
\newcommand{\childdocforward}[2][]
{
  \begingroup
    \if?#1?
      \def\childdoctmp
      {
        \def\childdocname{#2}
        \def\childdocjob{#2}
        \def\jobname{#2}
        \input{#2}
        \endinput
      }
    \else
      \def\childdoctmp
      {
        \childdocdisable
        \def\childdocname{#2}
        \childdoctrue
        \includeonly{#2}
        \def\childdocjob{#1}
        \def\jobname{#1}
        \input{#1}
        \endinput
      }
    \fi
    \expandafter
  \endgroup
  \childdoctmp
}
%    \end{macrocode}

% \macro{\childdocforwardprefix}
% The command |\childdocforwardprefix| redirects
% compilation to the main or a child file by means of a pattern.
% The prefix |#1| in the current filename is replaced by |#2|
% and the suffix of the current filename is kept
% (it is assumed that the filename does not contain the substring `|~~~|'
% which is used as a delimiter).
% Compilation is handed over to the new file by |\childdocforward|:
%    \begin{macrocode}
\newcommand{\childdocforwardprefix}[3][]
{
  \begingroup
    \def\childdocextract #2##1~~~{\def\childdoctmp{\childdocforward[#1]{#3##1}}}
    \expandafter\childdocextract\childdocname~~~
    \expandafter
  \endgroup
  \childdoctmp
}
%    \end{macrocode}

% \macro{\childdoc}
% The deprecated macro |\childdoc| is a legacy version of |\childdocmain|:
%    \begin{macrocode}
\newcommand{\childdoc}{\childdocmain}
%    \end{macrocode}

% \macro{\childdocredirect}
% The deprecated macro |\childdocredirect| is a legacy version
% of |\childdocforward| and |\childdocforwardprefix|:
%    \begin{macrocode}
\newcommand{\childdocredirect}[2][]
{
  \begingroup
    \if?#1?
      \def\childdoctmp{\childdocforward{#2}}
    \else
      \def\childdoctmp{\childdocforwardprefix{#1}{#2}}
    \fi
    \expandafter
  \endgroup
  \childdoctmp
}
%    \end{macrocode}

%\iffalse
%</package>
%\fi
%
\endinput

\childdocof{cdocsamp}
%    \end{macrocode}

%\iffalse
%</samplechap1|samplechap2>
%\fi
%
%\iffalse
%<*samplechap1>
%\fi
% Some text for chapter 1:
%    \begin{macrocode}
\section{one}
some text in chapter one
%    \end{macrocode}

%\iffalse
%</samplechap1>
%\fi
% Some text for chapter 2:
%\iffalse
%<*samplechap2>
%\fi
%    \begin{macrocode}
\section{two}
more text in chapter two
%    \end{macrocode}

%\iffalse
%</samplechap2>
%\fi
%
% %%%%%%%%%%%%%%%%%%%%%%%%%%%%%%%%%%%%%%
% \paragraph{Part Include Files.}
%
% The include files are called |cdocspt3.tex| and |cdocspt4.tex|.
%
%\iffalse
%<*samplepart3|samplepart4>
%\fi

% Optional override for |\version| flag:
%    \begin{macrocode}
%%\providecommand{\version}{final}
%    \end{macrocode}

% Include the main document:
%    \begin{macrocode}
% \iffalse
%
% childdoc.dtx Copyright (C) 2017-2018 Niklas Beisert
%
% This work may be distributed and/or modified under the
% conditions of the LaTeX Project Public License, either version 1.3
% of this license or (at your option) any later version.
% The latest version of this license is in
%   http://www.latex-project.org/lppl.txt
% and version 1.3 or later is part of all distributions of LaTeX
% version 2005/12/01 or later.
%
% This work has the LPPL maintenance status `maintained'.
%
% The Current Maintainer of this work is Niklas Beisert.
%
% This work consists of the files childdoc.dtx and childdoc.ins
% and the derived files childdoc.def and cdocsamp.tex with
% cdocsch1.tex, cdocsch2.tex, cdocsdrf.tex, cdocsfn1.tex, cdocsfn2.tex.
%
%<package>\ifdefined\childdocmain\endinput\fi
%<package>\ProvidesFile{childdoc.def}[2018/12/30 v2.0 child document driver]
%<samplemain>\ProvidesFile{cdocsamp.tex}[2018/12/30 v2.0 sample for childdoc]
%<*driver>
%\ProvidesFile{childdoc.drv}[2018/12/30 v2.0 childdoc reference manual file]
\PassOptionsToClass{10pt,a4paper}{article}
\documentclass{ltxdoc}

\usepackage[margin=35mm]{geometry}
\usepackage{hyperref}
\usepackage{hyperxmp}
\usepackage[usenames]{color}

\hypersetup{colorlinks=true}
\hypersetup{pdfstartview=FitH}
\hypersetup{pdfpagemode=UseNone}
\hypersetup{pdfsource={}}
\hypersetup{pdflang={en-UK}}
\hypersetup{pdfcopyright={Copyright 2017-2018 Niklas Beisert.
  This work may be distributed and/or modified under the
  conditions of the LaTeX Project Public License, either version 1.3
  of this license or (at your option) any later version.}}
\hypersetup{pdflicenseurl={http://www.latex-project.org/lppl.txt}}
\hypersetup{pdfcontactaddress={ETH Zurich, ITP, HIT K,
  Wolfgang-Pauli-Strasse 27}}
\hypersetup{pdfcontactpostcode={8093}}
\hypersetup{pdfcontactcity={Zurich}}
\hypersetup{pdfcontactcountry={Switzerland}}
\hypersetup{pdfcontactemail={nbeisert@itp.phys.ethz.ch}}
\hypersetup{pdfcontacturl={http://people.phys.ethz.ch/\xmptilde nbeisert/}}

\newcommand{\secref}[1]{\hyperref[#1]{section \ref*{#1}}}

\parskip1ex
\parindent0pt
\let\olditemize\itemize
\def\itemize{\olditemize\parskip0pt}

\begin{document}

\title{The \textsf{childdoc} Package}
\hypersetup{pdftitle={The childdoc Package}}
\author{Niklas Beisert\\[2ex]
  Institut f\"ur Theoretische Physik\\
  Eidgen\"ossische Technische Hochschule Z\"urich\\
  Wolfgang-Pauli-Strasse 27, 8093 Z\"urich, Switzerland\\[1ex]
  \href{mailto:nbeisert@itp.phys.ethz.ch}
  {\texttt{nbeisert@itp.phys.ethz.ch}}}
\hypersetup{pdfauthor={Niklas Beisert}}
\hypersetup{pdfsubject={Manual for the LaTeX2e Package childdoc}}
\date{30 December 2018, \textsf{v2.0}}
\maketitle

\begin{abstract}\noindent
\textsf{childdoc} is a \LaTeXe{} package
that enables the direct compilation
of document sections included by |\include|
to individual files.
\end{abstract}

\begingroup
\parskip0ex
\tableofcontents
\endgroup

%%%%%%%%%%%%%%%%%%%%%%%%%%%%%%%%%%%%%%%%%%%%%%%%%%%%%%%%%%%%%%%%%%%%%%%%%%%%%%%%
%%%%%%%%%%%%%%%%%%%%%%%%%%%%%%%%%%%%%%%%%%%%%%%%%%%%%%%%%%%%%%%%%%%%%%%%%%%%%%%%
\section{Introduction}

\LaTeX{} provides a mechanism to structure a large document (such as a book)
into a main file and several child files (containing the chapters)
using the |\include| command.
This mechanism is beneficial for documents
which span hundreds of pages in order to
make the source file(s) more manageable.
Moreover, compilation can be restricted to
selected child files by means of the |\includeonly| command.
The latter feature can be used to reduce the compilation time while editing
(this was significantly more useful in the earlier days of \LaTeX{})
or to generate a smaller document which is easier to navigate.
Another application of |\includeonly| is to generate
documents consisting of selected parts of the complete document.

However, there are a few drawbacks of the plain |\include| mechanism:
\begin{itemize}
\item
The child files cannot be compiled on their own,
they can only be compiled via the main file.
A naive editing environment
(such as a text editor with an option
to have the current file processed by \LaTeX)
may require one to switch to the main file before compiling;
attempting to compile the child file produces errors.
\item
The main file must be modified (each time)
to adjust the |\includeonly| command
to the present needs. This easily leaves the main file in a messy state.
\item
The generated document will always carry the filename
of the main document. This is inconvenient if
several child files are to be compiled and
to be kept for distribution.
\end{itemize}

The present package provides a simple interface
to make child files individually compilable by \LaTeX{}.
Compiling a child file then has the same effect as compiling
the main file with an |\includeonly| command
to select the appropriate child.
Moreover the generated document will carry the name of the child
rather than the main file.
This resolves all three above issues.

This feature is meant to make the editing of books,
thesis documents and lecture notes somewhat more convenient.
However, the package can also be used efficiently for
composing a series of documents (such as exercise sheets)
which are typically distributed individually.
It then assists the author in generating the individual documents
(potentially in different versions)
as well as a document containing the collected series.
Another application is in developing style files
or other kinds of included material
where compilation of the style file could redirect
to a sample or test file.

%%%%%%%%%%%%%%%%%%%%%%%%%%%%%%%%%%%%%%%%%%%%%%%%%%%%%%%%%%%%%%%%%%%%%%%%%%%%%%%%
%%%%%%%%%%%%%%%%%%%%%%%%%%%%%%%%%%%%%%%%%%%%%%%%%%%%%%%%%%%%%%%%%%%%%%%%%%%%%%%%
\section{Usage}

First of all, the package \textsf{childdoc} is \emph{not} a standard
\LaTeXe{} |.sty| style file! Therefore it needs to be invoked in
a non-standard way.

%%%%%%%%%%%%%%%%%%%%%%%%%%%%%%%%%%%%%%%%%%%%%%%%%%%%%%%%%%%%%%%%%%%%%%%%%%%%%%%%
\subsection{Included Files}
\label{sec:include}

%%%%%%%%%%%%%%%%%%%%%%%%%%%%%%%%%%%%%%%%
\DescribeMacro{\childdocmain}
To use the package, add the commands
\begin{center}
\begin{tabular}{l}
|\input{childdoc.def}|\\
|\childdocmain{}|\\
\end{tabular}
\end{center}
at the very top of the main \LaTeX{} file,
in particular \emph{before} the |\documentclass| statement!
The argument of |\childdocmain| should be left empty
(but it must be present).

%%%%%%%%%%%%%%%%%%%%%%%%%%%%%%%%%%%%%%%%
\DescribeMacro{\childdocof}
Furthermore, add the commands
\begin{center}
\begin{tabular}{l}
|\input{childdoc.def}|\\
|\childdocof{|\textit{main}|}|\\
\end{tabular}
\end{center}
at the top of every child file \textit{child}
which is included by |\include{|\textit{child}|}|
from within the main file
(or at least for those files to be compiled individually).
The argument \textit{main} must be the filename of the main file.

There are a couple of
considerations in setting up the main and child documents:

%%%%%%%%%%%%%%%%%%%%%%%%%%%%%%%%%%%%%%%%
\paragraph{Restrictions.}

Please note the following restrictions:
\begin{itemize}
\item
|\childdocmain| must be called with one argument \textit{main}
to ensure compatibility with earlier version of the package.
It must either be empty (|\childdocmain{}|)
or precisely match the filename of the main file in which it is specified.
See \secref{sec:detection} for further information.
\item
The filename \textit{main} must be specified without the |.tex| extension.
\item
The filename \textit{main} is case sensitive
(even in case-insensitive file systems)
due to internal string comparison.
\item
The argument \textit{main} should be fully expanded, it cannot be a macro.
\item
Subdirectories and special characters should be avoided in filenames.
\item
The command |\childdocmain{|\textit{main}|}| must be followed by a whitespace.
It should not be followed immediately by another command
or by a comment mark `|%|'.
This is because the \TeX{} parser reads the token immediately following
the argument of |\childdocmain| and puts it
at the beginning of every child section;
however, a white\-space is ignored.
\end{itemize}

%%%%%%%%%%%%%%%%%%%%%%%%%%%%%%%%%%%%%%%%
\paragraph{Content of Main File.}

It is advisable to place all content in the child files included by |\include|.
Any output contained in the main file will appear in all child documents
unless suppressed manually;
it cannot be suppressed automatically by the |\includeonly| directive
and thus should normally be avoided.
A method to include some content in the main file
by means of conditional processing is described in \secref{sec:conditional}.

%%%%%%%%%%%%%%%%%%%%%%%%%%%%%%%%%%%%%%%%
\paragraph{Page Numbering.}

When only a part of the document is compiled,
the appropriate numbering of pages
(as well as other status parameters)
is determined from the |.aux| files.
The latter contain information from previous passes.
However this information needs to propagate through
all intermediate child documents.
Therefore the page numbering in child documents may well
be inconsistent until the complete document is compiled at least once.

A useful (if unconventional) way to always ensure a consistent
page numbering is to restart the numbering in each child document
and denote the pages by `\textit{child}|.|\textit{page}'
where \textit{child} represents the chapter/section number of the child file.
This can be achieved by the command
|\numberwithin{page}{|\textit{child}|}|
of the \textsf{amsmath} package
where \textit{child} can be |chapter| or |section|
depending on the chosen structuring.
Alternatively, one can modify the macro |\thepage| appropriately
and reset the counter |page| at the start of each child file.

%%%%%%%%%%%%%%%%%%%%%%%%%%%%%%%%%%%%%%%%%%%%%%%%%%%%%%%%%%%%%%%%%%%%%%%%%%%%%%%%
\subsection{Conditional Processing}
\label{sec:conditional}

The package provides a mechanism to compile different versions
of a document. To customise the versions further some conditional processing
can come in handy to distinguish which version is being compiled.
The package provides two macros to describe the compilation context:

%%%%%%%%%%%%%%%%%%%%%%%%%%%%%%%%%%%%%%%%
\DescribeMacro{\ifchilddoc}
The conditional |\ifchilddoc| distinguishes between the compilation of
child documents and the main document:
%
\begin{center}
|\ifchilddoc |\textit{child-code}| |[|\||else |\textit{main-code}]| \||fi|
\end{center}

%%%%%%%%%%%%%%%%%%%%%%%%%%%%%%%%%%%%%%%%
\DescribeMacro{\childdocname}
\DescribeMacro{\childdocjob}
The macro |\childdocname| contains the filename (without extension)
of the main or child file being processed.
Note that |\childdocjob| will always contain the name of the main file.

%%%%%%%%%%%%%%%%%%%%%%%%%%%%%%%%%%%%%%%%
\paragraph{Title Page.}

Conditional processing can be used to include a title or banner page
in the main document when proper precautions are taken.
Importantly, the code in the main file should ensure that the page counter
(as well as other status parameters which are stored in the |.aux| files)
takes the same value after the conditional processing.
Otherwise the page numbers may take divergent values
depending on which part is compiled.

For example, a title page could be declared by:
%
\begin{center}
\begin{tabular}{l}
|\ifchilddoc\||else|\\
|\addtocounter{page}{-1}|\\
\textit{code for title page}\\
|\newpage|\\
|\||fi|
\end{tabular}
\end{center}
%
A banner page for the child documents can be generated by:
%
\begin{center}
\begin{tabular}{l}
|\ifchilddoc|\\
|\addtocounter{page}{-1}|\\
\textit{code for banner page}\\
|\newpage|\\
|\||fi|
\end{tabular}
\end{center}
%
Here one could write a message such as:
\begin{center}
|This is the part \childdocname{} of \childdocjob{}.|
\end{center}

%%%%%%%%%%%%%%%%%%%%%%%%%%%%%%%%%%%%%%%%%%%%%%%%%%%%%%%%%%%%%%%%%%%%%%%%%%%%%%%%
\subsection{Flags}
\label{sec:flags}

The package makes it easy to generate different versions
of the main or child documents.
To this end compilation flags can be defined
and assigned different default values.
They will be particularly useful in conjunction
with the forwarding mechanism described in \secref{sec:forward}.

For example, it may be useful to have a flag |\version|
which can be set to |draft| or |final|.
The document source will contain some conditional code
depending on the value of |\version|.
Suppose further, the flag should default to |final| for the main file
and to |draft| for child files
which is a natural assignment for editing the document.
This is achieved by placing the following code
in the preamble of the main document
(below the |\childdocmain| directive):
%
\begin{center}
\begin{tabular}{l}
|\ifchilddoc|\\
|\providecommand{\version}{draft}|\\
|\||else|\\
|\providecommand{\version}{final}|\\
|\||fi|
\end{tabular}
\end{center}
%
The definition by |\providecommand| makes sure
that previous definitions are not overwritten.
Further statements |\providecommand{\version}{...}|
can thus be added before the above code to override it.

For the main file, one might add a line
(between |\childdocmain| and the above block)
%
\begin{center}
|%\ifchilddoc\||else\providecommand{\version}{draft}\||fi|
\end{center}
%
which can be uncommented to produce a draft version.
Likewise one can add a line to the very top of a child file
(above the |\childdocof{|\textit{main}|}| directive)
%
\begin{center}
|%\providecommand{\version}{final}|
\end{center}
%
which can be uncommented to produce the final version of this child document.

%%%%%%%%%%%%%%%%%%%%%%%%%%%%%%%%%%%%%%%%%%%%%%%%%%%%%%%%%%%%%%%%%%%%%%%%%%%%%%%%
\subsection{Forwarding}
\label{sec:forward}

Different versions of the main or child documents
using compilation flags as described in \secref{sec:flags}
can be (permanently) stored in different files
for convenient compilation, viewing and distribution.
To this end, the package defines a command
to pass on compilation to a different file:

%%%%%%%%%%%%%%%%%%%%%%%%%%%%%%%%%%%%%%%%
\DescribeMacro{\childdocforward}
The command |\childdocforward| redirects processing to
another source file:
%
\begin{center}
\begin{tabular}{l}
|\input{childdoc.def}|\\
|\childdocforward[|\textit{main}|]{|\textit{dest}|}|\\
\end{tabular}
\end{center}
%
The argument \textit{dest} is the destination file
(without extension).
It should be the main file or one of the child files.
Note that further \textsf{childdoc} directives
such as |\childdocof| and |\childdocforward|
in the indicated file will be processed in this form.
The optional argument \textit{main}
passes on directly to the main file \textit{main}
while pretending to compile the child \textit{dest}.
This form behaves as if \textit{dest}
issues |\childdocof{|\textit{main}|}| right away,
and no further \textsf{childdoc} directives will be processed.

%%%%%%%%%%%%%%%%%%%%%%%%%%%%%%%%%%%%%%%%
\DescribeMacro{\...prefix}
In the alternative form |\childdocforwardprefix|,
%
\begin{center}
\begin{tabular}{l}
|\input{childdoc.def}|\\
|\childdocforwardprefix[|\textit{main}|]{|\textit{prefix}|}{|\textit{dest}|}|
\end{tabular}
\end{center}
%
the destination file is determined by a pattern
depending on the current file:
To make this work, the current file must be called
`{\textit{prefix}\hspace{0.2em}\textit{suffix}}'
with \textit{prefix} matching precisely the argument.
Processing is then passed on to the file
`{\textit{dest}\hspace{0.2em}\textit{suffix}}'.
Surely, the same effect is achieved by
directly specifying the
argument `{\textit{dest}\hspace{0.2em}\textit{suffix}}'
in the first form.
However, that requires to set up a different file
for each child. With the alternative form of the command
all these files can have exactly the same content
which simplifies setting them up and maintaining them.

For example, the following file |draft.tex|
with a compilation flag |\version| as described in \secref{sec:flags}
compiles the main document as a draft:
%
\begin{center}
\begin{tabular}{l}
|\def\version{draft}|\\
|\input{childdoc.def}|\\
|\childdocforward{|\textit{main}|}|
\end{tabular}
\end{center}
%
Likewise, the following files |final|\textit{nn}|.tex|
compile the final version of the child document
|child|\textit{nn}|.tex|:
%
\begin{center}
\begin{tabular}{l}
|\def\version{final}|\\
|\input{childdoc.def}|\\
|\childdocforwardprefix{final}{child}|
\end{tabular}
\end{center}
%

Note that when several versions of a main file and/or of each child file
are to be generated, it may be convenient to set up a |Makefile| or
shell script to automatise the process.

%%%%%%%%%%%%%%%%%%%%%%%%%%%%%%%%%%%%%%%%%%%%%%%%%%%%%%%%%%%%%%%%%%%%%%%%%%%%%%%%
\subsection{Command Line Processing}
\label{sec:commandline}

The effect of redirection files can also be achieved by invoking
the \LaTeX{} compiler with a more elaborate command line.
Most conveniently this should be done as part
of a shell script or a |Makefile|.

When using \textsf{childdoc} in the main file, the following
command lines effectively perform a redirection
(note that depending on the shell being used,
backslashes may have to be doubled: `|\|' $\to$ `|\\|'):
%
\begin{center}
|... -jobname "|\textit{target}|" |\\|"|[\textit{flags}]%
|\input{childdoc.def}\childdocforward[|\textit{main}|]{|\textit{dest}|}"|
\end{center}
%
Here \textit{target} is the name of the output file,
\textit{main} is the name of the main file
and \textit{dest} is the name of the main or child file to be processed
(all filenames without extensions).
The optional argument \textit{main} can be omitted
if \textit{main} matches \textit{dest}.
Optionally, compilation \textit{flags} can be defined via |\def| commands.
This command line makes the \TeX{} engine believe
it is compiling the file \textit{target}
whose content is specified as the latter parameter.
The provided code then forwards the processing to
\textit{main} or \textit{dest} as described in \secref{sec:forward}.

%%%%%%%%%%%%%%%%%%%%%%%%%%%%%%%%%%%%%%%%%%%%%%%%%%%%%%%%%%%%%%%%%%%%%%%%%%%%%%%%
\subsection{Include by Input}
\label{sec:input}

Including child documents by |\include| has some restrictions by design.
Most notably, the content of a child document always occupies
its own set of pages; pages cannot be shared between child documents.
Usually, this behaviour makes perfect sense
because each child document contain an essential part of the document.
However, in some situations it may be desirable to compose
a document from a collection of parts
without having mandatory page breaks between then.
For this case, the package
provides a mechanism to include parts
by |\input| which can also be processed individually.
However, by construction this mechanism
requires manual handling of the content to be output.

%%%%%%%%%%%%%%%%%%%%%%%%%%%%%%%%%%%%%%%%
\DescribeMacro{\ifchilddocmanual}
The main file should be prepared as usual, see \secref{sec:include}.
However, the document body must make a distinction
between processing of an individual part and of the main document, e.g.:
%
\begin{center}
\begin{tabular}{l}
|\ifchilddocmanual|\\
|\input{\childdocname}|\\
|\||else|\\
\textit{document body with }|\input{|\textit{part}|}|\\
|\||fi|
\end{tabular}
\end{center}
%
The conditional |\ifchilddocmanual| is true whenever
a part to be included by |\input| is being compiled,
and the name of the part is stored in |\childdocname|.

%%%%%%%%%%%%%%%%%%%%%%%%%%%%%%%%%%%%%%%%
\DescribeMacro{\childdocby}
Each part to be included by |\input| should start with:
%
\begin{center}
\begin{tabular}{l}
|\input{childdoc.def}|\\
|\childdocby{|\textit{main}|}|\\
\end{tabular}
\end{center}
%
The directive |\childdocby| is similar to |\childdocof|
described in \secref{sec:include},
but the subsequent selection of content must be done manually.
To that end, both |\ifchilddoc| and |\ifchilddocmanual|
will be true upon processing of a part,
and the name of the part is stored in |\childdocname|.
Note that |\jobname| will be set to the filename of the current part
so that each part receives an individual |.aux| file
that does not interfere with the |.aux| file(s) of the main document.
This behaviour can be altered by the alternative form
|\childdocby[*]{|\textit{main}|}| (with a non-empty optional argument)
which uses the |.aux| file of the main document
by setting |\jobname| to \textit{main}.

%%%%%%%%%%%%%%%%%%%%%%%%%%%%%%%%%%%%%%%%%%%%%%%%%%%%%%%%%%%%%%%%%%%%%%%%%%%%%%%%
\subsection{Driver Development}
\label{sec:driver}

The \textsf{childdoc} mechanism can also be use for the development
of definition files such as \LaTeX{} styles or classes.
This case differs from the above setup with multiple parts
included by |\include| in that no |\includeonly| should be invoked.
This can be achieved by starting the include file
(before |\ProvidesPackage|) with:
%
\begin{center}
\begin{tabular}{l}
|\input{childdoc.def}|\\
|\childdocforward{|\textit{main}|}|\\
\end{tabular}
\end{center}
%
or alternatively with:
%
\begin{center}
\begin{tabular}{l}
|\input{childdoc.def}|\\
|\childdocby{|\textit{main}|}|\\
\end{tabular}
\end{center}
%
Both forms have slightly different effects as described above.
The main file is prepared as usual, see \secref{sec:include}.

%%%%%%%%%%%%%%%%%%%%%%%%%%%%%%%%%%%%%%%%%%%%%%%%%%%%%%%%%%%%%%%%%%%%%%%%%%%%%%%%
\subsection{Legacy Detection}
\label{sec:detection}

The directive |\childdocmain| in the main file can detect
whether the complete document or merely a child is to be compiled
even without using the directive |\childdocof|.
This method is deprecated because it is less robust
and there is no compelling reason to use it;
it is merely provided for backward compatibility
and it may be removed in future versions.

If the detection mechanism is to be used,
it is mandatory to correctly specify
the filename of the main file as the argument of |\childdocmain|:
%
\begin{center}
\begin{tabular}{l}
|\input{childdoc.def}|\\
|\childdocmain{|\textit{main}|}|\\
\end{tabular}
\end{center}
%
If |\jobname| does not match the argument \textit{main} of |\childdocmain|,
it is assumed that |\jobname| points to the child file to be compiled.
When using |\childdocmain| with the main file specified as argument,
it suffices to start a child file
with just |\input{|\textit{main}|}|
without loading of the package and using |\childdocof|.
If instead all processing is done
with the appropriate \textsf{childdoc} directives,
the argument of \textit{main} of |\childdocmain| can be empty.

An alternative version of the command line processing described
in \secref{sec:commandline} using the detection mechanism reads:
%
\begin{center}
|... -jobname "|\textit{target}|" "|[\textit{flags}]%
[|\def\jobname{|\textit{dest}|}|]|\input{|\textit{main}|}"|
\end{center}

%%%%%%%%%%%%%%%%%%%%%%%%%%%%%%%%%%%%%%%%%%%%%%%%%%%%%%%%%%%%%%%%%%%%%%%%%%%%%%%%
\subsection{Manual Code}
\label{sec:manual}

In case one cannot be certain whether the definitions file |childdoc.def|
is installed on the target \TeX{} distribution
and one prefers not to ship it,
it is conceivable to paste a few relevant commands into the sources.

To that end, drop all statements |\input{childdoc.def}|
and perform the replacements as outlined below.
Instead of |\childdocmain{|\textit{main}|}| add the following code
to the top of the main file:
%
\begin{center}
\begin{tabular}{l}
|\||ifdefined\childdocname\endinput\||fi\newif\ifchilddoc|\\
|\edef\childdocname{\scantokens\expandafter{\jobname\noexpand}}|\\
|\def\childdocmain{|\textit{main}|}\||ifx\childdocmain\childdocname\||else|\\
|\childdoctrue\includeonly{\childdocname}\let\jobname\childdocmain\||fi|\\
\end{tabular}
\end{center}
%
Instead of |\childdocof{|\textit{main}|}| just include the main file
at the top of each child file:
%
\begin{center}
|\input{|\textit{main}|}|
\end{center}
%
A simple redirection |\childdocforward{|\textit{dest}|}| is achieved by:
%
\begin{center}
|\def\jobname{|\textit{dest}|}\input{\jobname}|
\end{center}
%
The redirection with prefix
|\childdocforwardprefix[|\textit{prefix}|]{|\textit{dest}|}|
is accomplished by:
%
\begin{center}
\begin{tabular}{l}
|{\edef\jobname{\scantokens\expandafter{\jobname\noexpand}}|\\
|\def\redirectjob |\textit{prefix}|#1~~~{\gdef\jobname{|\textit{dest}|#1}}|\\
|\expandafter\redirectjob\jobname~~~}\input{\jobname}|
\end{tabular}
\end{center}

In an alternative approach,
child documents can be compiled by a specific command line
without additional code or specific definitions:
%
\begin{center}
|... -jobname "|\textit{target}|" "|[\textit{flags}]%
|\includeonly{|\textit{dest}|}\input{|\textit{main}|}"|
\end{center}
%

%%%%%%%%%%%%%%%%%%%%%%%%%%%%%%%%%%%%%%%%%%%%%%%%%%%%%%%%%%%%%%%%%%%%%%%%%%%%%%%%
%%%%%%%%%%%%%%%%%%%%%%%%%%%%%%%%%%%%%%%%%%%%%%%%%%%%%%%%%%%%%%%%%%%%%%%%%%%%%%%%
\section{Information}

%%%%%%%%%%%%%%%%%%%%%%%%%%%%%%%%%%%%%%%%%%%%%%%%%%%%%%%%%%%%%%%%%%%%%%%%%%%%%%%%
\subsection{Copyright}

Copyright \copyright{} 2017--2018 Niklas Beisert

This work may be distributed and/or modified under the
conditions of the \LaTeX{} Project Public License, either version 1.3
of this license or (at your option) any later version.
The latest version of this license is in
  \url{http://www.latex-project.org/lppl.txt}
and version 1.3 or later is part of all distributions of \LaTeX{}
version 2005/12/01 or later.

This work has the LPPL maintenance status `maintained'.

The Current Maintainer of this work is Niklas Beisert.

This work consists of the files |README.txt|, |childdoc.ins| and |childdoc.dtx|
as well as the derived files |childdoc.def|, |cdocsamp.tex|
with |cdocsch1.tex|, |cdocsch2.tex|, |cdocspt3.tex|, |cdocspt4.tex|,
|cdocsdrf.tex|, |cdocsfn1.tex|, |cdocsfn2.tex|
as well as |childdoc.pdf|.

%%%%%%%%%%%%%%%%%%%%%%%%%%%%%%%%%%%%%%%%%%%%%%%%%%%%%%%%%%%%%%%%%%%%%%%%%%%%%%%%
\subsection{Files and Installation}

The package consists of the files:
%
\begin{center}
\begin{tabular}{ll}
    |README.txt|   & readme file \\
    |childdoc.ins| & installation file \\
    |childdoc.dtx| & source file \\
    |childdoc.def| & definition file \\
    |cdocsamp.tex| & sample main file \\
    |cdocsch1.tex| & sample include file \\
    |cdocsch2.tex| & sample include file \\
    |cdocspt3.tex| & sample part file \\
    |cdocspt4.tex| & sample part file \\
    |cdocsdrf.tex| & sample redirection file \\
    |cdocsfn1.tex| & sample redirection file \\
    |cdocsfn2.tex| & sample redirection file \\
    |childdoc.pdf| & manual
\end{tabular}
\end{center}
%
The distribution consists of the files
|README.txt|, |childdoc.ins| and |childdoc.dtx|.
%
\begin{itemize}
\item
Run (pdf)\LaTeX{} on |childdoc.dtx|
to compile the manual |childdoc.pdf| (this file).
\item
Run \LaTeX{} on |childdoc.ins| to create the definitions file |childdoc.def|
and the sample |cdocsamp.tex| with include files
|cdocsch1.tex|, |cdocsch2.tex|, |cdocspt3.tex|, |cdocspt4.tex|,
|cdocsdrf.tex|, |cdocsfn1.tex|, |cdocsfn2.tex|.
Then copy the file |childdoc.def| to an appropriate directory of your \LaTeX{}
distribution, e.g.\ \textit{texmf-root}|/tex/latex/childdoc|.
\end{itemize}

%%%%%%%%%%%%%%%%%%%%%%%%%%%%%%%%%%%%%%%%%%%%%%%%%%%%%%%%%%%%%%%%%%%%%%%%%%%%%%%%
\subsection{Related CTAN Packages}

There are several other packages which offer a similar functionality:
%
\begin{itemize}
\item
The packages
\href{http://ctan.org/pkg/docmute}{\textsf{docmute}},
\href{http://ctan.org/pkg/includex}{\textsf{includex}} and
\href{http://ctan.org/pkg/standalone}{\textsf{standalone}}
provide commands to include only the document body of
a child file thus allowing both files to be compiled individually.
\item
The packages \href{http://ctan.org/pkg/subdocs}{\textsf{subdocs}}
and \href{http://ctan.org/pkg/subfiles}{\textsf{subfiles}}
provide structures in which the main and child documents can be
encapsulated and allowing them to be compiled individually.
The inclusion mechanism is different from the conventional |\include|.
\item
The package \href{http://ctan.org/pkg/combine}{\textsf{combine}}
is an elaborate solution to combine several documents into one.
\end{itemize}
%
See also the CTAN topic \href{http://ctan.org/topic/subdocs}{\textsf{subdocs}}
for further related packages.
The present package differs from the above solutions in that
a document structure constructed with the conventional |\include| mechanism
just needs two extra commands at the top of every file
such that all constituent files can be compiled individually.

%%%%%%%%%%%%%%%%%%%%%%%%%%%%%%%%%%%%%%%%%%%%%%%%%%%%%%%%%%%%%%%%%%%%%%%%%%%%%%%%
%\subsection{Feature Suggestions}
%
%The following is a list of features which may be useful for future
%versions of this package:
%%
%\begin{itemize}
%\item
%\ldots
%\end{itemize}

%%%%%%%%%%%%%%%%%%%%%%%%%%%%%%%%%%%%%%%%%%%%%%%%%%%%%%%%%%%%%%%%%%%%%%%%%%%%%%%%
\subsection{Revision History}

%%%%%%%%%%%%%%%%%%%%%%%%%%%%%%%%%%%%%%%%
\paragraph{v2.0:} 2018/12/30

\begin{itemize}
\item
immediate forward processing
\item
added |\childdocby| mechanism
\item
manual restructured
\end{itemize}

%%%%%%%%%%%%%%%%%%%%%%%%%%%%%%%%%%%%%%%%
\paragraph{v1.6:} 2018/01/17

\begin{itemize}
\item
application for development of include files
\item
corrections to manual
\end{itemize}

%%%%%%%%%%%%%%%%%%%%%%%%%%%%%%%%%%%%%%%%
\paragraph{v1.5:} 2017/05/21

\begin{itemize}
\item
more complete structuring introduced
\item
|\childdocof| introduced
\item
|\childdoc| renamed to |\childdocmain|
\item
|\childredirect| renamed to |\childdocforward| and |\childdocforwardprefix|
and functionality expanded
\end{itemize}

%%%%%%%%%%%%%%%%%%%%%%%%%%%%%%%%%%%%%%%%
\paragraph{v1.0:} 2017/04/27

\begin{itemize}
\item
manual and install package
\item
first version published on CTAN
\end{itemize}

%%%%%%%%%%%%%%%%%%%%%%%%%%%%%%%%%%%%%%%%
\paragraph{v0.6:} 2017/04/26

\begin{itemize}
\item
redirection mechanism added
\end{itemize}

%%%%%%%%%%%%%%%%%%%%%%%%%%%%%%%%%%%%%%%%
\paragraph{v0.5:} 2017/04/26

\begin{itemize}
\item
functionality in definition file
\end{itemize}


%%%%%%%%%%%%%%%%%%%%%%%%%%%%%%%%%%%%%%%%%%%%%%%%%%%%%%%%%%%%%%%%%%%%%%%%%%%%%%%%
%%%%%%%%%%%%%%%%%%%%%%%%%%%%%%%%%%%%%%%%%%%%%%%%%%%%%%%%%%%%%%%%%%%%%%%%%%%%%%%%
%%%%%%%%%%%%%%%%%%%%%%%%%%%%%%%%%%%%%%%%%%%%%%%%%%%%%%%%%%%%%%%%%%%%%%%%%%%%%%%%
\appendix

\settowidth\MacroIndent{\rmfamily\scriptsize 000\ }

 \DocInput{childdoc.dtx}

\end{document}
%</driver>
% \fi
%
% %%%%%%%%%%%%%%%%%%%%%%%%%%%%%%%%%%%%%%%%%%%%%%%%%%%%%%%%%%%%%%%%%%%%%%%%%%%%%%
% %%%%%%%%%%%%%%%%%%%%%%%%%%%%%%%%%%%%%%%%%%%%%%%%%%%%%%%%%%%%%%%%%%%%%%%%%%%%%%
% \section{Sample}
%\iffalse
%<*samplemain>
%\fi
%
% The following presents a sample document
% with two chapters, two parts, a title page,
% a compile flag as well as three forwarding files to set the flag.
% It consists of eight |.tex| files:
% \begin{center}
% \begin{tabular}{ll}
% |cdocsamp.tex|&main file\\
% |cdocsch1.tex|&include file for chapter 1\\
% |cdocsch2.tex|&include file for chapter 2\\
% |cdocspt3.tex|&include file for part 3\\
% |cdocspt4.tex|&include file for part 4\\
% |cdocsdrf.tex|&forwarding file for main file in draft mode\\
% |cdocsfi1.tex|&forwarding file for final version of chapter 1\\
% |cdocsfi2.tex|&forwarding file for final version of chapter 2\\
% \end{tabular}
% \end{center}
% Each of the eight files can be compiled directly by the \LaTeX{} compiler.
%
% %%%%%%%%%%%%%%%%%%%%%%%%%%%%%%%%%%%%%%
% \paragraph{Main File.}
%
% The main file is called |cdocsamp.tex|.
%
% Load the \textsf{childdoc} definitions and
% declare the filename for the main document:
%    \begin{macrocode}
\input{childdoc.def}
\childdocmain{}
%    \end{macrocode}

% Optional override for |\version| flag:
%    \begin{macrocode}
%%\ifchilddoc\else\providecommand{\version}{draft}\fi
%    \end{macrocode}

% Define the default values for the |\version| flag
% (|final| for the main file and |draft| for childs):
%    \begin{macrocode}
\ifchilddoc
\providecommand{\version}{draft}
\else
\providecommand{\version}{final}
\fi
%    \end{macrocode}

% Load the standard document class:
%    \begin{macrocode}
\documentclass[12pt]{article}
%    \end{macrocode}

% Start the document body:
%    \begin{macrocode}
\begin{document}
%    \end{macrocode}

% Declare a title page.
% Print title, part of document being processed and version flag:
%    \begin{macrocode}
\addtocounter{page}{-1}
\begin{center}
{\LARGE\bfseries{}childdoc example\par}
\vspace{1cm}
\ifchilddoc
\ifchilddocmanual part\else chapter\fi:
`\childdocname' of `\childdocjob'\par
\else
main document: `\childdocjob'\par
\fi
version: \version\par
\end{center}
\newpage
%    \end{macrocode}

% Manually include selected file,
% otherwise process as usual:
%    \begin{macrocode}
\ifchilddocmanual
\section*{part `\childdocname'}
\input{\childdocname}
\else
%    \end{macrocode}

% Include the two chapters:
%    \begin{macrocode}
\include{cdocsch1}
\include{cdocsch2}
%    \end{macrocode}

% Include the two parts unless only chapters should be displayed:
%    \begin{macrocode}
\ifchilddoc\else
\section{part three}
\input{cdocspt3}
\section{part four}
\input{cdocspt4}
\fi
%    \end{macrocode}

% Process as usual until here:
%    \begin{macrocode}
\fi
%    \end{macrocode}

% End of document body:
%    \begin{macrocode}
\end{document}
%    \end{macrocode}
%\iffalse
%</samplemain>
%\fi
%
% %%%%%%%%%%%%%%%%%%%%%%%%%%%%%%%%%%%%%%
% \paragraph{Chapter Include Files.}
%
% The include files are called |cdocsch1.tex| and |cdocsch2.tex|.
%
%\iffalse
%<*samplechap1|samplechap2>
%\fi

% Optional override for |\version| flag:
%    \begin{macrocode}
%%\providecommand{\version}{final}
%    \end{macrocode}

% Include the main document:
%    \begin{macrocode}
\input{childdoc.def}
\childdocof{cdocsamp}
%    \end{macrocode}

%\iffalse
%</samplechap1|samplechap2>
%\fi
%
%\iffalse
%<*samplechap1>
%\fi
% Some text for chapter 1:
%    \begin{macrocode}
\section{one}
some text in chapter one
%    \end{macrocode}

%\iffalse
%</samplechap1>
%\fi
% Some text for chapter 2:
%\iffalse
%<*samplechap2>
%\fi
%    \begin{macrocode}
\section{two}
more text in chapter two
%    \end{macrocode}

%\iffalse
%</samplechap2>
%\fi
%
% %%%%%%%%%%%%%%%%%%%%%%%%%%%%%%%%%%%%%%
% \paragraph{Part Include Files.}
%
% The include files are called |cdocspt3.tex| and |cdocspt4.tex|.
%
%\iffalse
%<*samplepart3|samplepart4>
%\fi

% Optional override for |\version| flag:
%    \begin{macrocode}
%%\providecommand{\version}{final}
%    \end{macrocode}

% Include the main document:
%    \begin{macrocode}
\input{childdoc.def}
\childdocby{cdocsamp}
%    \end{macrocode}

%\iffalse
%</samplepart3|samplepart4>
%\fi
%
%\iffalse
%<*samplepart3>
%\fi
% Some text for part 3:
%    \begin{macrocode}
some text in part three
%    \end{macrocode}

%\iffalse
%</samplepart3>
%\fi
% Some text for part 4:
%\iffalse
%<*samplepart4>
%\fi
%    \begin{macrocode}
more text in part four
%    \end{macrocode}

%\iffalse
%</samplepart4>
%\fi
%
% %%%%%%%%%%%%%%%%%%%%%%%%%%%%%%%%%%%%%%
% \paragraph{Forwarding for a Complete Draft.}
%
% The following forwarding file |cdocsdrf.tex|
% compiles the main document in draft mode:
%\iffalse
%<*sampledraft>
%\fi
%    \begin{macrocode}
\def\version{draft}
\input{childdoc.def}
\childdocforward{cdocsamp}
%    \end{macrocode}

%\iffalse
%</sampledraft>
%\fi
%
% %%%%%%%%%%%%%%%%%%%%%%%%%%%%%%%%%%%%%%
% \paragraph{Forwarding for Final Version of the Chapters.}
%
% The following forwarding files |cdocsfn1.tex| and |cdocsfn2.tex|
% (with identical content)
% compile the final versions of the child documents
% |cdocsch1.tex| and |cdocsch2.tex|, respectively:
%\iffalse
%<*samplefinal>
%\fi
%    \begin{macrocode}
\def\version{final}
\input{childdoc.def}
\childdocforwardprefix[cdocsamp]{cdocsfn}{cdocsch}
%    \end{macrocode}

%\iffalse
%</samplefinal>
%\fi
%
% %%%%%%%%%%%%%%%%%%%%%%%%%%%%%%%%%%%%%%
% \paragraph{Command Line Processing.}
%
% The following three command lines generate the output files
% |cdocscld|, |cdocscl1| and |cdocscl2|
% which should be identical to
% |cdocsdrf|, |cdocsch1| and |cdocsfn2|, respectively:
% \begin{center}
% \begin{tabular}{l}
% |latex -jobname cdocscld \|\\
% |  "\def\version{draft}\input{childdoc.def}\childdocforward{cdocsamp}"|\\
% |latex -jobname cdocscl1 \|\\
% |  "\input{childdoc.def}\childdocforward[cdocsamp]{cdocsch1}"|\\
% |latex -jobname cdocscl2 \|\\
% |  "\def\version{final}\input{childdoc.def}\childdocforward{cdocsch2}"|
% \end{tabular}
% \end{center}
% Note that the trailing backslash on each first line
% merely continues the input to the second line
% (for convenient cut ant paste).
% Furthermore, the command |latex| can be replaced by any
% of its alternative versions such as |pdflatex|.
%
% %%%%%%%%%%%%%%%%%%%%%%%%%%%%%%%%%%%%%%%%%%%%%%%%%%%%%%%%%%%%%%%%%%%%%%%%%%%%%%
% %%%%%%%%%%%%%%%%%%%%%%%%%%%%%%%%%%%%%%%%%%%%%%%%%%%%%%%%%%%%%%%%%%%%%%%%%%%%%%
% \section{Implementation}
%\iffalse
%<*package>
%\fi
%
% This section describes the definitions file |childdoc.def|.

% The definitions cannot be loaded using |\usepackage| or |\RequirePackage|
% which has a mechanism to prevent loading a style file more than once.
% When loading the definitions by means of |\input|
% multiple instances have to be prevented manually:
%\iffalse
%This code needs to be before the `\ProvidesFile' directive
%which is defined at the beginning of this file.
%Therefore it is also placed there and commented out here.
%</package>
%<*discard>
%\fi
%    \begin{macrocode}
\ifdefined\childdocmain\endinput\fi
%    \end{macrocode}
%\iffalse
%</discard>
%<*package>
%\fi
%
% \macro{\ifchilddoc}
% \macro{\ifchilddocmanual}
% The conditional |\ifchilddoc| tells whether a
% child (true) or main (false) document is being compiled.
% The conditional |\ifchilddocmanual| tells whether
% the |\includeonly| mechanism is used (false) or
% the selection of child files must be performed manually (true).
% The definitions initialise to false:
%    \begin{macrocode}
\newif\ifchilddoc
\newif\ifchilddocmanual
%    \end{macrocode}

% \macro{\childdocname}
% \macro{\childdocjob}
% The macro |\childdocname| stores the name of the main document
% to be compiled. The macro |\childdocjob| stores the name of
% the document on which the \LaTeX{} compiler was originally invoked.
% The content of |\jobname| cannot be compared
% to filenames specified in the source due to different catcodes.
% The following code rescans |\jobname|, stores the result
% in |\childdocname| and saves a copy in |\childdocjob|:
%    \begin{macrocode}
\edef\childdocname{\scantokens\expandafter{\jobname\noexpand}}
\let\childdocjob\childdocname
%    \end{macrocode}

% \macro{\childdocdisable}
% The macro |\childdocdisable| prevents the main file
% from being processed more than once.
% At this stage, the main document command |\childdocmain|
% is assumed to be called once again where it should do nothing.
% Any subsequent call to it should prevent
% a secondary processing of the main document
% It overwrites the forwarding commands
% |\childdocof| and |\childdocforward|
% with empty macros to prevent further inclusions of the main document:
%    \begin{macrocode}
\newcommand{\childdocdisable}
{
  \renewcommand{\childdocmain}[1]{\renewcommand{\childdocmain}[1]{\endinput}}
  \renewcommand{\childdocof}[1]{}
  \renewcommand{\childdocby}[2][]{}
  \renewcommand{\childdocforward}[2][]{}
  \renewcommand{\childdocdisable}{}
}
%    \end{macrocode}

% \macro{\childdocmain}
% The macro |\childdocmain| is to be called at the top of the main file
% with nothing or the main filename (without extension) as argument.
% First, it breaks loops.
% If the argument is not empty and does not match |\childdocname|
% (which is set by the first inclusion of |childdoc.def|),
% |\ifchilddoc| is set to true, |\includeonly| is applied to the child file
% and |\jobname| is set to the main file
% (for proper handling of |.aux| files):
%    \begin{macrocode}
\newcommand{\childdocmain}[1]
{
  \childdocdisable\childdocmain{}
  \if?#1?\else
    \begingroup
      \def\childdoctmp{#1}
      \ifx\childdoctmp\childdocname
        \def\childdoctmp{}
      \else
        \def\childdoctmp
        {
          \childdoctrue
          \includeonly{\childdocname}
          \def\childdocjob{#1}
          \def\jobname{#1}
        }
      \fi
      \expandafter
    \endgroup
    \childdoctmp
  \fi
}
%    \end{macrocode}

% \macro{\childdocof}
% The command |\childdocof| redirects
% compilation to the main file |#1|.
%    \begin{macrocode}
\newcommand{\childdocof}[1]
{
  \childdocdisable
  \childdoctrue
  \includeonly{\childdocname}
  \def\jobname{#1}
  \def\childdocjob{#1}
  \input{#1}
}
%    \end{macrocode}

% \macro{\childdocby}
% The command |\childdocby| ....
%    \begin{macrocode}
\newcommand{\childdocby}[2][]
{
  \childdocdisable
  \childdoctrue
  \childdocmanualtrue
  \if?#1?\else
    \def\jobname{#2}
  \fi
  \def\childdocjob{#2}
  \input{#2}
  \endinput
}
%    \end{macrocode}

% \macro{\childdocforward}
% The command |\childdocforward| redirects
% compilation to the main file or
% (if the optional argument is given) a child file.
% Parameters are set as if the main file
% or a child file starting with |\childdocof| was compiled.
% Then compilation is handed over to the main file:
%    \begin{macrocode}
\newcommand{\childdocforward}[2][]
{
  \begingroup
    \if?#1?
      \def\childdoctmp
      {
        \def\childdocname{#2}
        \def\childdocjob{#2}
        \def\jobname{#2}
        \input{#2}
        \endinput
      }
    \else
      \def\childdoctmp
      {
        \childdocdisable
        \def\childdocname{#2}
        \childdoctrue
        \includeonly{#2}
        \def\childdocjob{#1}
        \def\jobname{#1}
        \input{#1}
        \endinput
      }
    \fi
    \expandafter
  \endgroup
  \childdoctmp
}
%    \end{macrocode}

% \macro{\childdocforwardprefix}
% The command |\childdocforwardprefix| redirects
% compilation to the main or a child file by means of a pattern.
% The prefix |#1| in the current filename is replaced by |#2|
% and the suffix of the current filename is kept
% (it is assumed that the filename does not contain the substring `|~~~|'
% which is used as a delimiter).
% Compilation is handed over to the new file by |\childdocforward|:
%    \begin{macrocode}
\newcommand{\childdocforwardprefix}[3][]
{
  \begingroup
    \def\childdocextract #2##1~~~{\def\childdoctmp{\childdocforward[#1]{#3##1}}}
    \expandafter\childdocextract\childdocname~~~
    \expandafter
  \endgroup
  \childdoctmp
}
%    \end{macrocode}

% \macro{\childdoc}
% The deprecated macro |\childdoc| is a legacy version of |\childdocmain|:
%    \begin{macrocode}
\newcommand{\childdoc}{\childdocmain}
%    \end{macrocode}

% \macro{\childdocredirect}
% The deprecated macro |\childdocredirect| is a legacy version
% of |\childdocforward| and |\childdocforwardprefix|:
%    \begin{macrocode}
\newcommand{\childdocredirect}[2][]
{
  \begingroup
    \if?#1?
      \def\childdoctmp{\childdocforward{#2}}
    \else
      \def\childdoctmp{\childdocforwardprefix{#1}{#2}}
    \fi
    \expandafter
  \endgroup
  \childdoctmp
}
%    \end{macrocode}

%\iffalse
%</package>
%\fi
%
\endinput

\childdocby{cdocsamp}
%    \end{macrocode}

%\iffalse
%</samplepart3|samplepart4>
%\fi
%
%\iffalse
%<*samplepart3>
%\fi
% Some text for part 3:
%    \begin{macrocode}
some text in part three
%    \end{macrocode}

%\iffalse
%</samplepart3>
%\fi
% Some text for part 4:
%\iffalse
%<*samplepart4>
%\fi
%    \begin{macrocode}
more text in part four
%    \end{macrocode}

%\iffalse
%</samplepart4>
%\fi
%
% %%%%%%%%%%%%%%%%%%%%%%%%%%%%%%%%%%%%%%
% \paragraph{Forwarding for a Complete Draft.}
%
% The following forwarding file |cdocsdrf.tex|
% compiles the main document in draft mode:
%\iffalse
%<*sampledraft>
%\fi
%    \begin{macrocode}
\def\version{draft}
% \iffalse
%
% childdoc.dtx Copyright (C) 2017-2018 Niklas Beisert
%
% This work may be distributed and/or modified under the
% conditions of the LaTeX Project Public License, either version 1.3
% of this license or (at your option) any later version.
% The latest version of this license is in
%   http://www.latex-project.org/lppl.txt
% and version 1.3 or later is part of all distributions of LaTeX
% version 2005/12/01 or later.
%
% This work has the LPPL maintenance status `maintained'.
%
% The Current Maintainer of this work is Niklas Beisert.
%
% This work consists of the files childdoc.dtx and childdoc.ins
% and the derived files childdoc.def and cdocsamp.tex with
% cdocsch1.tex, cdocsch2.tex, cdocsdrf.tex, cdocsfn1.tex, cdocsfn2.tex.
%
%<package>\ifdefined\childdocmain\endinput\fi
%<package>\ProvidesFile{childdoc.def}[2018/12/30 v2.0 child document driver]
%<samplemain>\ProvidesFile{cdocsamp.tex}[2018/12/30 v2.0 sample for childdoc]
%<*driver>
%\ProvidesFile{childdoc.drv}[2018/12/30 v2.0 childdoc reference manual file]
\PassOptionsToClass{10pt,a4paper}{article}
\documentclass{ltxdoc}

\usepackage[margin=35mm]{geometry}
\usepackage{hyperref}
\usepackage{hyperxmp}
\usepackage[usenames]{color}

\hypersetup{colorlinks=true}
\hypersetup{pdfstartview=FitH}
\hypersetup{pdfpagemode=UseNone}
\hypersetup{pdfsource={}}
\hypersetup{pdflang={en-UK}}
\hypersetup{pdfcopyright={Copyright 2017-2018 Niklas Beisert.
  This work may be distributed and/or modified under the
  conditions of the LaTeX Project Public License, either version 1.3
  of this license or (at your option) any later version.}}
\hypersetup{pdflicenseurl={http://www.latex-project.org/lppl.txt}}
\hypersetup{pdfcontactaddress={ETH Zurich, ITP, HIT K,
  Wolfgang-Pauli-Strasse 27}}
\hypersetup{pdfcontactpostcode={8093}}
\hypersetup{pdfcontactcity={Zurich}}
\hypersetup{pdfcontactcountry={Switzerland}}
\hypersetup{pdfcontactemail={nbeisert@itp.phys.ethz.ch}}
\hypersetup{pdfcontacturl={http://people.phys.ethz.ch/\xmptilde nbeisert/}}

\newcommand{\secref}[1]{\hyperref[#1]{section \ref*{#1}}}

\parskip1ex
\parindent0pt
\let\olditemize\itemize
\def\itemize{\olditemize\parskip0pt}

\begin{document}

\title{The \textsf{childdoc} Package}
\hypersetup{pdftitle={The childdoc Package}}
\author{Niklas Beisert\\[2ex]
  Institut f\"ur Theoretische Physik\\
  Eidgen\"ossische Technische Hochschule Z\"urich\\
  Wolfgang-Pauli-Strasse 27, 8093 Z\"urich, Switzerland\\[1ex]
  \href{mailto:nbeisert@itp.phys.ethz.ch}
  {\texttt{nbeisert@itp.phys.ethz.ch}}}
\hypersetup{pdfauthor={Niklas Beisert}}
\hypersetup{pdfsubject={Manual for the LaTeX2e Package childdoc}}
\date{30 December 2018, \textsf{v2.0}}
\maketitle

\begin{abstract}\noindent
\textsf{childdoc} is a \LaTeXe{} package
that enables the direct compilation
of document sections included by |\include|
to individual files.
\end{abstract}

\begingroup
\parskip0ex
\tableofcontents
\endgroup

%%%%%%%%%%%%%%%%%%%%%%%%%%%%%%%%%%%%%%%%%%%%%%%%%%%%%%%%%%%%%%%%%%%%%%%%%%%%%%%%
%%%%%%%%%%%%%%%%%%%%%%%%%%%%%%%%%%%%%%%%%%%%%%%%%%%%%%%%%%%%%%%%%%%%%%%%%%%%%%%%
\section{Introduction}

\LaTeX{} provides a mechanism to structure a large document (such as a book)
into a main file and several child files (containing the chapters)
using the |\include| command.
This mechanism is beneficial for documents
which span hundreds of pages in order to
make the source file(s) more manageable.
Moreover, compilation can be restricted to
selected child files by means of the |\includeonly| command.
The latter feature can be used to reduce the compilation time while editing
(this was significantly more useful in the earlier days of \LaTeX{})
or to generate a smaller document which is easier to navigate.
Another application of |\includeonly| is to generate
documents consisting of selected parts of the complete document.

However, there are a few drawbacks of the plain |\include| mechanism:
\begin{itemize}
\item
The child files cannot be compiled on their own,
they can only be compiled via the main file.
A naive editing environment
(such as a text editor with an option
to have the current file processed by \LaTeX)
may require one to switch to the main file before compiling;
attempting to compile the child file produces errors.
\item
The main file must be modified (each time)
to adjust the |\includeonly| command
to the present needs. This easily leaves the main file in a messy state.
\item
The generated document will always carry the filename
of the main document. This is inconvenient if
several child files are to be compiled and
to be kept for distribution.
\end{itemize}

The present package provides a simple interface
to make child files individually compilable by \LaTeX{}.
Compiling a child file then has the same effect as compiling
the main file with an |\includeonly| command
to select the appropriate child.
Moreover the generated document will carry the name of the child
rather than the main file.
This resolves all three above issues.

This feature is meant to make the editing of books,
thesis documents and lecture notes somewhat more convenient.
However, the package can also be used efficiently for
composing a series of documents (such as exercise sheets)
which are typically distributed individually.
It then assists the author in generating the individual documents
(potentially in different versions)
as well as a document containing the collected series.
Another application is in developing style files
or other kinds of included material
where compilation of the style file could redirect
to a sample or test file.

%%%%%%%%%%%%%%%%%%%%%%%%%%%%%%%%%%%%%%%%%%%%%%%%%%%%%%%%%%%%%%%%%%%%%%%%%%%%%%%%
%%%%%%%%%%%%%%%%%%%%%%%%%%%%%%%%%%%%%%%%%%%%%%%%%%%%%%%%%%%%%%%%%%%%%%%%%%%%%%%%
\section{Usage}

First of all, the package \textsf{childdoc} is \emph{not} a standard
\LaTeXe{} |.sty| style file! Therefore it needs to be invoked in
a non-standard way.

%%%%%%%%%%%%%%%%%%%%%%%%%%%%%%%%%%%%%%%%%%%%%%%%%%%%%%%%%%%%%%%%%%%%%%%%%%%%%%%%
\subsection{Included Files}
\label{sec:include}

%%%%%%%%%%%%%%%%%%%%%%%%%%%%%%%%%%%%%%%%
\DescribeMacro{\childdocmain}
To use the package, add the commands
\begin{center}
\begin{tabular}{l}
|\input{childdoc.def}|\\
|\childdocmain{}|\\
\end{tabular}
\end{center}
at the very top of the main \LaTeX{} file,
in particular \emph{before} the |\documentclass| statement!
The argument of |\childdocmain| should be left empty
(but it must be present).

%%%%%%%%%%%%%%%%%%%%%%%%%%%%%%%%%%%%%%%%
\DescribeMacro{\childdocof}
Furthermore, add the commands
\begin{center}
\begin{tabular}{l}
|\input{childdoc.def}|\\
|\childdocof{|\textit{main}|}|\\
\end{tabular}
\end{center}
at the top of every child file \textit{child}
which is included by |\include{|\textit{child}|}|
from within the main file
(or at least for those files to be compiled individually).
The argument \textit{main} must be the filename of the main file.

There are a couple of
considerations in setting up the main and child documents:

%%%%%%%%%%%%%%%%%%%%%%%%%%%%%%%%%%%%%%%%
\paragraph{Restrictions.}

Please note the following restrictions:
\begin{itemize}
\item
|\childdocmain| must be called with one argument \textit{main}
to ensure compatibility with earlier version of the package.
It must either be empty (|\childdocmain{}|)
or precisely match the filename of the main file in which it is specified.
See \secref{sec:detection} for further information.
\item
The filename \textit{main} must be specified without the |.tex| extension.
\item
The filename \textit{main} is case sensitive
(even in case-insensitive file systems)
due to internal string comparison.
\item
The argument \textit{main} should be fully expanded, it cannot be a macro.
\item
Subdirectories and special characters should be avoided in filenames.
\item
The command |\childdocmain{|\textit{main}|}| must be followed by a whitespace.
It should not be followed immediately by another command
or by a comment mark `|%|'.
This is because the \TeX{} parser reads the token immediately following
the argument of |\childdocmain| and puts it
at the beginning of every child section;
however, a white\-space is ignored.
\end{itemize}

%%%%%%%%%%%%%%%%%%%%%%%%%%%%%%%%%%%%%%%%
\paragraph{Content of Main File.}

It is advisable to place all content in the child files included by |\include|.
Any output contained in the main file will appear in all child documents
unless suppressed manually;
it cannot be suppressed automatically by the |\includeonly| directive
and thus should normally be avoided.
A method to include some content in the main file
by means of conditional processing is described in \secref{sec:conditional}.

%%%%%%%%%%%%%%%%%%%%%%%%%%%%%%%%%%%%%%%%
\paragraph{Page Numbering.}

When only a part of the document is compiled,
the appropriate numbering of pages
(as well as other status parameters)
is determined from the |.aux| files.
The latter contain information from previous passes.
However this information needs to propagate through
all intermediate child documents.
Therefore the page numbering in child documents may well
be inconsistent until the complete document is compiled at least once.

A useful (if unconventional) way to always ensure a consistent
page numbering is to restart the numbering in each child document
and denote the pages by `\textit{child}|.|\textit{page}'
where \textit{child} represents the chapter/section number of the child file.
This can be achieved by the command
|\numberwithin{page}{|\textit{child}|}|
of the \textsf{amsmath} package
where \textit{child} can be |chapter| or |section|
depending on the chosen structuring.
Alternatively, one can modify the macro |\thepage| appropriately
and reset the counter |page| at the start of each child file.

%%%%%%%%%%%%%%%%%%%%%%%%%%%%%%%%%%%%%%%%%%%%%%%%%%%%%%%%%%%%%%%%%%%%%%%%%%%%%%%%
\subsection{Conditional Processing}
\label{sec:conditional}

The package provides a mechanism to compile different versions
of a document. To customise the versions further some conditional processing
can come in handy to distinguish which version is being compiled.
The package provides two macros to describe the compilation context:

%%%%%%%%%%%%%%%%%%%%%%%%%%%%%%%%%%%%%%%%
\DescribeMacro{\ifchilddoc}
The conditional |\ifchilddoc| distinguishes between the compilation of
child documents and the main document:
%
\begin{center}
|\ifchilddoc |\textit{child-code}| |[|\||else |\textit{main-code}]| \||fi|
\end{center}

%%%%%%%%%%%%%%%%%%%%%%%%%%%%%%%%%%%%%%%%
\DescribeMacro{\childdocname}
\DescribeMacro{\childdocjob}
The macro |\childdocname| contains the filename (without extension)
of the main or child file being processed.
Note that |\childdocjob| will always contain the name of the main file.

%%%%%%%%%%%%%%%%%%%%%%%%%%%%%%%%%%%%%%%%
\paragraph{Title Page.}

Conditional processing can be used to include a title or banner page
in the main document when proper precautions are taken.
Importantly, the code in the main file should ensure that the page counter
(as well as other status parameters which are stored in the |.aux| files)
takes the same value after the conditional processing.
Otherwise the page numbers may take divergent values
depending on which part is compiled.

For example, a title page could be declared by:
%
\begin{center}
\begin{tabular}{l}
|\ifchilddoc\||else|\\
|\addtocounter{page}{-1}|\\
\textit{code for title page}\\
|\newpage|\\
|\||fi|
\end{tabular}
\end{center}
%
A banner page for the child documents can be generated by:
%
\begin{center}
\begin{tabular}{l}
|\ifchilddoc|\\
|\addtocounter{page}{-1}|\\
\textit{code for banner page}\\
|\newpage|\\
|\||fi|
\end{tabular}
\end{center}
%
Here one could write a message such as:
\begin{center}
|This is the part \childdocname{} of \childdocjob{}.|
\end{center}

%%%%%%%%%%%%%%%%%%%%%%%%%%%%%%%%%%%%%%%%%%%%%%%%%%%%%%%%%%%%%%%%%%%%%%%%%%%%%%%%
\subsection{Flags}
\label{sec:flags}

The package makes it easy to generate different versions
of the main or child documents.
To this end compilation flags can be defined
and assigned different default values.
They will be particularly useful in conjunction
with the forwarding mechanism described in \secref{sec:forward}.

For example, it may be useful to have a flag |\version|
which can be set to |draft| or |final|.
The document source will contain some conditional code
depending on the value of |\version|.
Suppose further, the flag should default to |final| for the main file
and to |draft| for child files
which is a natural assignment for editing the document.
This is achieved by placing the following code
in the preamble of the main document
(below the |\childdocmain| directive):
%
\begin{center}
\begin{tabular}{l}
|\ifchilddoc|\\
|\providecommand{\version}{draft}|\\
|\||else|\\
|\providecommand{\version}{final}|\\
|\||fi|
\end{tabular}
\end{center}
%
The definition by |\providecommand| makes sure
that previous definitions are not overwritten.
Further statements |\providecommand{\version}{...}|
can thus be added before the above code to override it.

For the main file, one might add a line
(between |\childdocmain| and the above block)
%
\begin{center}
|%\ifchilddoc\||else\providecommand{\version}{draft}\||fi|
\end{center}
%
which can be uncommented to produce a draft version.
Likewise one can add a line to the very top of a child file
(above the |\childdocof{|\textit{main}|}| directive)
%
\begin{center}
|%\providecommand{\version}{final}|
\end{center}
%
which can be uncommented to produce the final version of this child document.

%%%%%%%%%%%%%%%%%%%%%%%%%%%%%%%%%%%%%%%%%%%%%%%%%%%%%%%%%%%%%%%%%%%%%%%%%%%%%%%%
\subsection{Forwarding}
\label{sec:forward}

Different versions of the main or child documents
using compilation flags as described in \secref{sec:flags}
can be (permanently) stored in different files
for convenient compilation, viewing and distribution.
To this end, the package defines a command
to pass on compilation to a different file:

%%%%%%%%%%%%%%%%%%%%%%%%%%%%%%%%%%%%%%%%
\DescribeMacro{\childdocforward}
The command |\childdocforward| redirects processing to
another source file:
%
\begin{center}
\begin{tabular}{l}
|\input{childdoc.def}|\\
|\childdocforward[|\textit{main}|]{|\textit{dest}|}|\\
\end{tabular}
\end{center}
%
The argument \textit{dest} is the destination file
(without extension).
It should be the main file or one of the child files.
Note that further \textsf{childdoc} directives
such as |\childdocof| and |\childdocforward|
in the indicated file will be processed in this form.
The optional argument \textit{main}
passes on directly to the main file \textit{main}
while pretending to compile the child \textit{dest}.
This form behaves as if \textit{dest}
issues |\childdocof{|\textit{main}|}| right away,
and no further \textsf{childdoc} directives will be processed.

%%%%%%%%%%%%%%%%%%%%%%%%%%%%%%%%%%%%%%%%
\DescribeMacro{\...prefix}
In the alternative form |\childdocforwardprefix|,
%
\begin{center}
\begin{tabular}{l}
|\input{childdoc.def}|\\
|\childdocforwardprefix[|\textit{main}|]{|\textit{prefix}|}{|\textit{dest}|}|
\end{tabular}
\end{center}
%
the destination file is determined by a pattern
depending on the current file:
To make this work, the current file must be called
`{\textit{prefix}\hspace{0.2em}\textit{suffix}}'
with \textit{prefix} matching precisely the argument.
Processing is then passed on to the file
`{\textit{dest}\hspace{0.2em}\textit{suffix}}'.
Surely, the same effect is achieved by
directly specifying the
argument `{\textit{dest}\hspace{0.2em}\textit{suffix}}'
in the first form.
However, that requires to set up a different file
for each child. With the alternative form of the command
all these files can have exactly the same content
which simplifies setting them up and maintaining them.

For example, the following file |draft.tex|
with a compilation flag |\version| as described in \secref{sec:flags}
compiles the main document as a draft:
%
\begin{center}
\begin{tabular}{l}
|\def\version{draft}|\\
|\input{childdoc.def}|\\
|\childdocforward{|\textit{main}|}|
\end{tabular}
\end{center}
%
Likewise, the following files |final|\textit{nn}|.tex|
compile the final version of the child document
|child|\textit{nn}|.tex|:
%
\begin{center}
\begin{tabular}{l}
|\def\version{final}|\\
|\input{childdoc.def}|\\
|\childdocforwardprefix{final}{child}|
\end{tabular}
\end{center}
%

Note that when several versions of a main file and/or of each child file
are to be generated, it may be convenient to set up a |Makefile| or
shell script to automatise the process.

%%%%%%%%%%%%%%%%%%%%%%%%%%%%%%%%%%%%%%%%%%%%%%%%%%%%%%%%%%%%%%%%%%%%%%%%%%%%%%%%
\subsection{Command Line Processing}
\label{sec:commandline}

The effect of redirection files can also be achieved by invoking
the \LaTeX{} compiler with a more elaborate command line.
Most conveniently this should be done as part
of a shell script or a |Makefile|.

When using \textsf{childdoc} in the main file, the following
command lines effectively perform a redirection
(note that depending on the shell being used,
backslashes may have to be doubled: `|\|' $\to$ `|\\|'):
%
\begin{center}
|... -jobname "|\textit{target}|" |\\|"|[\textit{flags}]%
|\input{childdoc.def}\childdocforward[|\textit{main}|]{|\textit{dest}|}"|
\end{center}
%
Here \textit{target} is the name of the output file,
\textit{main} is the name of the main file
and \textit{dest} is the name of the main or child file to be processed
(all filenames without extensions).
The optional argument \textit{main} can be omitted
if \textit{main} matches \textit{dest}.
Optionally, compilation \textit{flags} can be defined via |\def| commands.
This command line makes the \TeX{} engine believe
it is compiling the file \textit{target}
whose content is specified as the latter parameter.
The provided code then forwards the processing to
\textit{main} or \textit{dest} as described in \secref{sec:forward}.

%%%%%%%%%%%%%%%%%%%%%%%%%%%%%%%%%%%%%%%%%%%%%%%%%%%%%%%%%%%%%%%%%%%%%%%%%%%%%%%%
\subsection{Include by Input}
\label{sec:input}

Including child documents by |\include| has some restrictions by design.
Most notably, the content of a child document always occupies
its own set of pages; pages cannot be shared between child documents.
Usually, this behaviour makes perfect sense
because each child document contain an essential part of the document.
However, in some situations it may be desirable to compose
a document from a collection of parts
without having mandatory page breaks between then.
For this case, the package
provides a mechanism to include parts
by |\input| which can also be processed individually.
However, by construction this mechanism
requires manual handling of the content to be output.

%%%%%%%%%%%%%%%%%%%%%%%%%%%%%%%%%%%%%%%%
\DescribeMacro{\ifchilddocmanual}
The main file should be prepared as usual, see \secref{sec:include}.
However, the document body must make a distinction
between processing of an individual part and of the main document, e.g.:
%
\begin{center}
\begin{tabular}{l}
|\ifchilddocmanual|\\
|\input{\childdocname}|\\
|\||else|\\
\textit{document body with }|\input{|\textit{part}|}|\\
|\||fi|
\end{tabular}
\end{center}
%
The conditional |\ifchilddocmanual| is true whenever
a part to be included by |\input| is being compiled,
and the name of the part is stored in |\childdocname|.

%%%%%%%%%%%%%%%%%%%%%%%%%%%%%%%%%%%%%%%%
\DescribeMacro{\childdocby}
Each part to be included by |\input| should start with:
%
\begin{center}
\begin{tabular}{l}
|\input{childdoc.def}|\\
|\childdocby{|\textit{main}|}|\\
\end{tabular}
\end{center}
%
The directive |\childdocby| is similar to |\childdocof|
described in \secref{sec:include},
but the subsequent selection of content must be done manually.
To that end, both |\ifchilddoc| and |\ifchilddocmanual|
will be true upon processing of a part,
and the name of the part is stored in |\childdocname|.
Note that |\jobname| will be set to the filename of the current part
so that each part receives an individual |.aux| file
that does not interfere with the |.aux| file(s) of the main document.
This behaviour can be altered by the alternative form
|\childdocby[*]{|\textit{main}|}| (with a non-empty optional argument)
which uses the |.aux| file of the main document
by setting |\jobname| to \textit{main}.

%%%%%%%%%%%%%%%%%%%%%%%%%%%%%%%%%%%%%%%%%%%%%%%%%%%%%%%%%%%%%%%%%%%%%%%%%%%%%%%%
\subsection{Driver Development}
\label{sec:driver}

The \textsf{childdoc} mechanism can also be use for the development
of definition files such as \LaTeX{} styles or classes.
This case differs from the above setup with multiple parts
included by |\include| in that no |\includeonly| should be invoked.
This can be achieved by starting the include file
(before |\ProvidesPackage|) with:
%
\begin{center}
\begin{tabular}{l}
|\input{childdoc.def}|\\
|\childdocforward{|\textit{main}|}|\\
\end{tabular}
\end{center}
%
or alternatively with:
%
\begin{center}
\begin{tabular}{l}
|\input{childdoc.def}|\\
|\childdocby{|\textit{main}|}|\\
\end{tabular}
\end{center}
%
Both forms have slightly different effects as described above.
The main file is prepared as usual, see \secref{sec:include}.

%%%%%%%%%%%%%%%%%%%%%%%%%%%%%%%%%%%%%%%%%%%%%%%%%%%%%%%%%%%%%%%%%%%%%%%%%%%%%%%%
\subsection{Legacy Detection}
\label{sec:detection}

The directive |\childdocmain| in the main file can detect
whether the complete document or merely a child is to be compiled
even without using the directive |\childdocof|.
This method is deprecated because it is less robust
and there is no compelling reason to use it;
it is merely provided for backward compatibility
and it may be removed in future versions.

If the detection mechanism is to be used,
it is mandatory to correctly specify
the filename of the main file as the argument of |\childdocmain|:
%
\begin{center}
\begin{tabular}{l}
|\input{childdoc.def}|\\
|\childdocmain{|\textit{main}|}|\\
\end{tabular}
\end{center}
%
If |\jobname| does not match the argument \textit{main} of |\childdocmain|,
it is assumed that |\jobname| points to the child file to be compiled.
When using |\childdocmain| with the main file specified as argument,
it suffices to start a child file
with just |\input{|\textit{main}|}|
without loading of the package and using |\childdocof|.
If instead all processing is done
with the appropriate \textsf{childdoc} directives,
the argument of \textit{main} of |\childdocmain| can be empty.

An alternative version of the command line processing described
in \secref{sec:commandline} using the detection mechanism reads:
%
\begin{center}
|... -jobname "|\textit{target}|" "|[\textit{flags}]%
[|\def\jobname{|\textit{dest}|}|]|\input{|\textit{main}|}"|
\end{center}

%%%%%%%%%%%%%%%%%%%%%%%%%%%%%%%%%%%%%%%%%%%%%%%%%%%%%%%%%%%%%%%%%%%%%%%%%%%%%%%%
\subsection{Manual Code}
\label{sec:manual}

In case one cannot be certain whether the definitions file |childdoc.def|
is installed on the target \TeX{} distribution
and one prefers not to ship it,
it is conceivable to paste a few relevant commands into the sources.

To that end, drop all statements |\input{childdoc.def}|
and perform the replacements as outlined below.
Instead of |\childdocmain{|\textit{main}|}| add the following code
to the top of the main file:
%
\begin{center}
\begin{tabular}{l}
|\||ifdefined\childdocname\endinput\||fi\newif\ifchilddoc|\\
|\edef\childdocname{\scantokens\expandafter{\jobname\noexpand}}|\\
|\def\childdocmain{|\textit{main}|}\||ifx\childdocmain\childdocname\||else|\\
|\childdoctrue\includeonly{\childdocname}\let\jobname\childdocmain\||fi|\\
\end{tabular}
\end{center}
%
Instead of |\childdocof{|\textit{main}|}| just include the main file
at the top of each child file:
%
\begin{center}
|\input{|\textit{main}|}|
\end{center}
%
A simple redirection |\childdocforward{|\textit{dest}|}| is achieved by:
%
\begin{center}
|\def\jobname{|\textit{dest}|}\input{\jobname}|
\end{center}
%
The redirection with prefix
|\childdocforwardprefix[|\textit{prefix}|]{|\textit{dest}|}|
is accomplished by:
%
\begin{center}
\begin{tabular}{l}
|{\edef\jobname{\scantokens\expandafter{\jobname\noexpand}}|\\
|\def\redirectjob |\textit{prefix}|#1~~~{\gdef\jobname{|\textit{dest}|#1}}|\\
|\expandafter\redirectjob\jobname~~~}\input{\jobname}|
\end{tabular}
\end{center}

In an alternative approach,
child documents can be compiled by a specific command line
without additional code or specific definitions:
%
\begin{center}
|... -jobname "|\textit{target}|" "|[\textit{flags}]%
|\includeonly{|\textit{dest}|}\input{|\textit{main}|}"|
\end{center}
%

%%%%%%%%%%%%%%%%%%%%%%%%%%%%%%%%%%%%%%%%%%%%%%%%%%%%%%%%%%%%%%%%%%%%%%%%%%%%%%%%
%%%%%%%%%%%%%%%%%%%%%%%%%%%%%%%%%%%%%%%%%%%%%%%%%%%%%%%%%%%%%%%%%%%%%%%%%%%%%%%%
\section{Information}

%%%%%%%%%%%%%%%%%%%%%%%%%%%%%%%%%%%%%%%%%%%%%%%%%%%%%%%%%%%%%%%%%%%%%%%%%%%%%%%%
\subsection{Copyright}

Copyright \copyright{} 2017--2018 Niklas Beisert

This work may be distributed and/or modified under the
conditions of the \LaTeX{} Project Public License, either version 1.3
of this license or (at your option) any later version.
The latest version of this license is in
  \url{http://www.latex-project.org/lppl.txt}
and version 1.3 or later is part of all distributions of \LaTeX{}
version 2005/12/01 or later.

This work has the LPPL maintenance status `maintained'.

The Current Maintainer of this work is Niklas Beisert.

This work consists of the files |README.txt|, |childdoc.ins| and |childdoc.dtx|
as well as the derived files |childdoc.def|, |cdocsamp.tex|
with |cdocsch1.tex|, |cdocsch2.tex|, |cdocspt3.tex|, |cdocspt4.tex|,
|cdocsdrf.tex|, |cdocsfn1.tex|, |cdocsfn2.tex|
as well as |childdoc.pdf|.

%%%%%%%%%%%%%%%%%%%%%%%%%%%%%%%%%%%%%%%%%%%%%%%%%%%%%%%%%%%%%%%%%%%%%%%%%%%%%%%%
\subsection{Files and Installation}

The package consists of the files:
%
\begin{center}
\begin{tabular}{ll}
    |README.txt|   & readme file \\
    |childdoc.ins| & installation file \\
    |childdoc.dtx| & source file \\
    |childdoc.def| & definition file \\
    |cdocsamp.tex| & sample main file \\
    |cdocsch1.tex| & sample include file \\
    |cdocsch2.tex| & sample include file \\
    |cdocspt3.tex| & sample part file \\
    |cdocspt4.tex| & sample part file \\
    |cdocsdrf.tex| & sample redirection file \\
    |cdocsfn1.tex| & sample redirection file \\
    |cdocsfn2.tex| & sample redirection file \\
    |childdoc.pdf| & manual
\end{tabular}
\end{center}
%
The distribution consists of the files
|README.txt|, |childdoc.ins| and |childdoc.dtx|.
%
\begin{itemize}
\item
Run (pdf)\LaTeX{} on |childdoc.dtx|
to compile the manual |childdoc.pdf| (this file).
\item
Run \LaTeX{} on |childdoc.ins| to create the definitions file |childdoc.def|
and the sample |cdocsamp.tex| with include files
|cdocsch1.tex|, |cdocsch2.tex|, |cdocspt3.tex|, |cdocspt4.tex|,
|cdocsdrf.tex|, |cdocsfn1.tex|, |cdocsfn2.tex|.
Then copy the file |childdoc.def| to an appropriate directory of your \LaTeX{}
distribution, e.g.\ \textit{texmf-root}|/tex/latex/childdoc|.
\end{itemize}

%%%%%%%%%%%%%%%%%%%%%%%%%%%%%%%%%%%%%%%%%%%%%%%%%%%%%%%%%%%%%%%%%%%%%%%%%%%%%%%%
\subsection{Related CTAN Packages}

There are several other packages which offer a similar functionality:
%
\begin{itemize}
\item
The packages
\href{http://ctan.org/pkg/docmute}{\textsf{docmute}},
\href{http://ctan.org/pkg/includex}{\textsf{includex}} and
\href{http://ctan.org/pkg/standalone}{\textsf{standalone}}
provide commands to include only the document body of
a child file thus allowing both files to be compiled individually.
\item
The packages \href{http://ctan.org/pkg/subdocs}{\textsf{subdocs}}
and \href{http://ctan.org/pkg/subfiles}{\textsf{subfiles}}
provide structures in which the main and child documents can be
encapsulated and allowing them to be compiled individually.
The inclusion mechanism is different from the conventional |\include|.
\item
The package \href{http://ctan.org/pkg/combine}{\textsf{combine}}
is an elaborate solution to combine several documents into one.
\end{itemize}
%
See also the CTAN topic \href{http://ctan.org/topic/subdocs}{\textsf{subdocs}}
for further related packages.
The present package differs from the above solutions in that
a document structure constructed with the conventional |\include| mechanism
just needs two extra commands at the top of every file
such that all constituent files can be compiled individually.

%%%%%%%%%%%%%%%%%%%%%%%%%%%%%%%%%%%%%%%%%%%%%%%%%%%%%%%%%%%%%%%%%%%%%%%%%%%%%%%%
%\subsection{Feature Suggestions}
%
%The following is a list of features which may be useful for future
%versions of this package:
%%
%\begin{itemize}
%\item
%\ldots
%\end{itemize}

%%%%%%%%%%%%%%%%%%%%%%%%%%%%%%%%%%%%%%%%%%%%%%%%%%%%%%%%%%%%%%%%%%%%%%%%%%%%%%%%
\subsection{Revision History}

%%%%%%%%%%%%%%%%%%%%%%%%%%%%%%%%%%%%%%%%
\paragraph{v2.0:} 2018/12/30

\begin{itemize}
\item
immediate forward processing
\item
added |\childdocby| mechanism
\item
manual restructured
\end{itemize}

%%%%%%%%%%%%%%%%%%%%%%%%%%%%%%%%%%%%%%%%
\paragraph{v1.6:} 2018/01/17

\begin{itemize}
\item
application for development of include files
\item
corrections to manual
\end{itemize}

%%%%%%%%%%%%%%%%%%%%%%%%%%%%%%%%%%%%%%%%
\paragraph{v1.5:} 2017/05/21

\begin{itemize}
\item
more complete structuring introduced
\item
|\childdocof| introduced
\item
|\childdoc| renamed to |\childdocmain|
\item
|\childredirect| renamed to |\childdocforward| and |\childdocforwardprefix|
and functionality expanded
\end{itemize}

%%%%%%%%%%%%%%%%%%%%%%%%%%%%%%%%%%%%%%%%
\paragraph{v1.0:} 2017/04/27

\begin{itemize}
\item
manual and install package
\item
first version published on CTAN
\end{itemize}

%%%%%%%%%%%%%%%%%%%%%%%%%%%%%%%%%%%%%%%%
\paragraph{v0.6:} 2017/04/26

\begin{itemize}
\item
redirection mechanism added
\end{itemize}

%%%%%%%%%%%%%%%%%%%%%%%%%%%%%%%%%%%%%%%%
\paragraph{v0.5:} 2017/04/26

\begin{itemize}
\item
functionality in definition file
\end{itemize}


%%%%%%%%%%%%%%%%%%%%%%%%%%%%%%%%%%%%%%%%%%%%%%%%%%%%%%%%%%%%%%%%%%%%%%%%%%%%%%%%
%%%%%%%%%%%%%%%%%%%%%%%%%%%%%%%%%%%%%%%%%%%%%%%%%%%%%%%%%%%%%%%%%%%%%%%%%%%%%%%%
%%%%%%%%%%%%%%%%%%%%%%%%%%%%%%%%%%%%%%%%%%%%%%%%%%%%%%%%%%%%%%%%%%%%%%%%%%%%%%%%
\appendix

\settowidth\MacroIndent{\rmfamily\scriptsize 000\ }

 \DocInput{childdoc.dtx}

\end{document}
%</driver>
% \fi
%
% %%%%%%%%%%%%%%%%%%%%%%%%%%%%%%%%%%%%%%%%%%%%%%%%%%%%%%%%%%%%%%%%%%%%%%%%%%%%%%
% %%%%%%%%%%%%%%%%%%%%%%%%%%%%%%%%%%%%%%%%%%%%%%%%%%%%%%%%%%%%%%%%%%%%%%%%%%%%%%
% \section{Sample}
%\iffalse
%<*samplemain>
%\fi
%
% The following presents a sample document
% with two chapters, two parts, a title page,
% a compile flag as well as three forwarding files to set the flag.
% It consists of eight |.tex| files:
% \begin{center}
% \begin{tabular}{ll}
% |cdocsamp.tex|&main file\\
% |cdocsch1.tex|&include file for chapter 1\\
% |cdocsch2.tex|&include file for chapter 2\\
% |cdocspt3.tex|&include file for part 3\\
% |cdocspt4.tex|&include file for part 4\\
% |cdocsdrf.tex|&forwarding file for main file in draft mode\\
% |cdocsfi1.tex|&forwarding file for final version of chapter 1\\
% |cdocsfi2.tex|&forwarding file for final version of chapter 2\\
% \end{tabular}
% \end{center}
% Each of the eight files can be compiled directly by the \LaTeX{} compiler.
%
% %%%%%%%%%%%%%%%%%%%%%%%%%%%%%%%%%%%%%%
% \paragraph{Main File.}
%
% The main file is called |cdocsamp.tex|.
%
% Load the \textsf{childdoc} definitions and
% declare the filename for the main document:
%    \begin{macrocode}
\input{childdoc.def}
\childdocmain{}
%    \end{macrocode}

% Optional override for |\version| flag:
%    \begin{macrocode}
%%\ifchilddoc\else\providecommand{\version}{draft}\fi
%    \end{macrocode}

% Define the default values for the |\version| flag
% (|final| for the main file and |draft| for childs):
%    \begin{macrocode}
\ifchilddoc
\providecommand{\version}{draft}
\else
\providecommand{\version}{final}
\fi
%    \end{macrocode}

% Load the standard document class:
%    \begin{macrocode}
\documentclass[12pt]{article}
%    \end{macrocode}

% Start the document body:
%    \begin{macrocode}
\begin{document}
%    \end{macrocode}

% Declare a title page.
% Print title, part of document being processed and version flag:
%    \begin{macrocode}
\addtocounter{page}{-1}
\begin{center}
{\LARGE\bfseries{}childdoc example\par}
\vspace{1cm}
\ifchilddoc
\ifchilddocmanual part\else chapter\fi:
`\childdocname' of `\childdocjob'\par
\else
main document: `\childdocjob'\par
\fi
version: \version\par
\end{center}
\newpage
%    \end{macrocode}

% Manually include selected file,
% otherwise process as usual:
%    \begin{macrocode}
\ifchilddocmanual
\section*{part `\childdocname'}
\input{\childdocname}
\else
%    \end{macrocode}

% Include the two chapters:
%    \begin{macrocode}
\include{cdocsch1}
\include{cdocsch2}
%    \end{macrocode}

% Include the two parts unless only chapters should be displayed:
%    \begin{macrocode}
\ifchilddoc\else
\section{part three}
\input{cdocspt3}
\section{part four}
\input{cdocspt4}
\fi
%    \end{macrocode}

% Process as usual until here:
%    \begin{macrocode}
\fi
%    \end{macrocode}

% End of document body:
%    \begin{macrocode}
\end{document}
%    \end{macrocode}
%\iffalse
%</samplemain>
%\fi
%
% %%%%%%%%%%%%%%%%%%%%%%%%%%%%%%%%%%%%%%
% \paragraph{Chapter Include Files.}
%
% The include files are called |cdocsch1.tex| and |cdocsch2.tex|.
%
%\iffalse
%<*samplechap1|samplechap2>
%\fi

% Optional override for |\version| flag:
%    \begin{macrocode}
%%\providecommand{\version}{final}
%    \end{macrocode}

% Include the main document:
%    \begin{macrocode}
\input{childdoc.def}
\childdocof{cdocsamp}
%    \end{macrocode}

%\iffalse
%</samplechap1|samplechap2>
%\fi
%
%\iffalse
%<*samplechap1>
%\fi
% Some text for chapter 1:
%    \begin{macrocode}
\section{one}
some text in chapter one
%    \end{macrocode}

%\iffalse
%</samplechap1>
%\fi
% Some text for chapter 2:
%\iffalse
%<*samplechap2>
%\fi
%    \begin{macrocode}
\section{two}
more text in chapter two
%    \end{macrocode}

%\iffalse
%</samplechap2>
%\fi
%
% %%%%%%%%%%%%%%%%%%%%%%%%%%%%%%%%%%%%%%
% \paragraph{Part Include Files.}
%
% The include files are called |cdocspt3.tex| and |cdocspt4.tex|.
%
%\iffalse
%<*samplepart3|samplepart4>
%\fi

% Optional override for |\version| flag:
%    \begin{macrocode}
%%\providecommand{\version}{final}
%    \end{macrocode}

% Include the main document:
%    \begin{macrocode}
\input{childdoc.def}
\childdocby{cdocsamp}
%    \end{macrocode}

%\iffalse
%</samplepart3|samplepart4>
%\fi
%
%\iffalse
%<*samplepart3>
%\fi
% Some text for part 3:
%    \begin{macrocode}
some text in part three
%    \end{macrocode}

%\iffalse
%</samplepart3>
%\fi
% Some text for part 4:
%\iffalse
%<*samplepart4>
%\fi
%    \begin{macrocode}
more text in part four
%    \end{macrocode}

%\iffalse
%</samplepart4>
%\fi
%
% %%%%%%%%%%%%%%%%%%%%%%%%%%%%%%%%%%%%%%
% \paragraph{Forwarding for a Complete Draft.}
%
% The following forwarding file |cdocsdrf.tex|
% compiles the main document in draft mode:
%\iffalse
%<*sampledraft>
%\fi
%    \begin{macrocode}
\def\version{draft}
\input{childdoc.def}
\childdocforward{cdocsamp}
%    \end{macrocode}

%\iffalse
%</sampledraft>
%\fi
%
% %%%%%%%%%%%%%%%%%%%%%%%%%%%%%%%%%%%%%%
% \paragraph{Forwarding for Final Version of the Chapters.}
%
% The following forwarding files |cdocsfn1.tex| and |cdocsfn2.tex|
% (with identical content)
% compile the final versions of the child documents
% |cdocsch1.tex| and |cdocsch2.tex|, respectively:
%\iffalse
%<*samplefinal>
%\fi
%    \begin{macrocode}
\def\version{final}
\input{childdoc.def}
\childdocforwardprefix[cdocsamp]{cdocsfn}{cdocsch}
%    \end{macrocode}

%\iffalse
%</samplefinal>
%\fi
%
% %%%%%%%%%%%%%%%%%%%%%%%%%%%%%%%%%%%%%%
% \paragraph{Command Line Processing.}
%
% The following three command lines generate the output files
% |cdocscld|, |cdocscl1| and |cdocscl2|
% which should be identical to
% |cdocsdrf|, |cdocsch1| and |cdocsfn2|, respectively:
% \begin{center}
% \begin{tabular}{l}
% |latex -jobname cdocscld \|\\
% |  "\def\version{draft}\input{childdoc.def}\childdocforward{cdocsamp}"|\\
% |latex -jobname cdocscl1 \|\\
% |  "\input{childdoc.def}\childdocforward[cdocsamp]{cdocsch1}"|\\
% |latex -jobname cdocscl2 \|\\
% |  "\def\version{final}\input{childdoc.def}\childdocforward{cdocsch2}"|
% \end{tabular}
% \end{center}
% Note that the trailing backslash on each first line
% merely continues the input to the second line
% (for convenient cut ant paste).
% Furthermore, the command |latex| can be replaced by any
% of its alternative versions such as |pdflatex|.
%
% %%%%%%%%%%%%%%%%%%%%%%%%%%%%%%%%%%%%%%%%%%%%%%%%%%%%%%%%%%%%%%%%%%%%%%%%%%%%%%
% %%%%%%%%%%%%%%%%%%%%%%%%%%%%%%%%%%%%%%%%%%%%%%%%%%%%%%%%%%%%%%%%%%%%%%%%%%%%%%
% \section{Implementation}
%\iffalse
%<*package>
%\fi
%
% This section describes the definitions file |childdoc.def|.

% The definitions cannot be loaded using |\usepackage| or |\RequirePackage|
% which has a mechanism to prevent loading a style file more than once.
% When loading the definitions by means of |\input|
% multiple instances have to be prevented manually:
%\iffalse
%This code needs to be before the `\ProvidesFile' directive
%which is defined at the beginning of this file.
%Therefore it is also placed there and commented out here.
%</package>
%<*discard>
%\fi
%    \begin{macrocode}
\ifdefined\childdocmain\endinput\fi
%    \end{macrocode}
%\iffalse
%</discard>
%<*package>
%\fi
%
% \macro{\ifchilddoc}
% \macro{\ifchilddocmanual}
% The conditional |\ifchilddoc| tells whether a
% child (true) or main (false) document is being compiled.
% The conditional |\ifchilddocmanual| tells whether
% the |\includeonly| mechanism is used (false) or
% the selection of child files must be performed manually (true).
% The definitions initialise to false:
%    \begin{macrocode}
\newif\ifchilddoc
\newif\ifchilddocmanual
%    \end{macrocode}

% \macro{\childdocname}
% \macro{\childdocjob}
% The macro |\childdocname| stores the name of the main document
% to be compiled. The macro |\childdocjob| stores the name of
% the document on which the \LaTeX{} compiler was originally invoked.
% The content of |\jobname| cannot be compared
% to filenames specified in the source due to different catcodes.
% The following code rescans |\jobname|, stores the result
% in |\childdocname| and saves a copy in |\childdocjob|:
%    \begin{macrocode}
\edef\childdocname{\scantokens\expandafter{\jobname\noexpand}}
\let\childdocjob\childdocname
%    \end{macrocode}

% \macro{\childdocdisable}
% The macro |\childdocdisable| prevents the main file
% from being processed more than once.
% At this stage, the main document command |\childdocmain|
% is assumed to be called once again where it should do nothing.
% Any subsequent call to it should prevent
% a secondary processing of the main document
% It overwrites the forwarding commands
% |\childdocof| and |\childdocforward|
% with empty macros to prevent further inclusions of the main document:
%    \begin{macrocode}
\newcommand{\childdocdisable}
{
  \renewcommand{\childdocmain}[1]{\renewcommand{\childdocmain}[1]{\endinput}}
  \renewcommand{\childdocof}[1]{}
  \renewcommand{\childdocby}[2][]{}
  \renewcommand{\childdocforward}[2][]{}
  \renewcommand{\childdocdisable}{}
}
%    \end{macrocode}

% \macro{\childdocmain}
% The macro |\childdocmain| is to be called at the top of the main file
% with nothing or the main filename (without extension) as argument.
% First, it breaks loops.
% If the argument is not empty and does not match |\childdocname|
% (which is set by the first inclusion of |childdoc.def|),
% |\ifchilddoc| is set to true, |\includeonly| is applied to the child file
% and |\jobname| is set to the main file
% (for proper handling of |.aux| files):
%    \begin{macrocode}
\newcommand{\childdocmain}[1]
{
  \childdocdisable\childdocmain{}
  \if?#1?\else
    \begingroup
      \def\childdoctmp{#1}
      \ifx\childdoctmp\childdocname
        \def\childdoctmp{}
      \else
        \def\childdoctmp
        {
          \childdoctrue
          \includeonly{\childdocname}
          \def\childdocjob{#1}
          \def\jobname{#1}
        }
      \fi
      \expandafter
    \endgroup
    \childdoctmp
  \fi
}
%    \end{macrocode}

% \macro{\childdocof}
% The command |\childdocof| redirects
% compilation to the main file |#1|.
%    \begin{macrocode}
\newcommand{\childdocof}[1]
{
  \childdocdisable
  \childdoctrue
  \includeonly{\childdocname}
  \def\jobname{#1}
  \def\childdocjob{#1}
  \input{#1}
}
%    \end{macrocode}

% \macro{\childdocby}
% The command |\childdocby| ....
%    \begin{macrocode}
\newcommand{\childdocby}[2][]
{
  \childdocdisable
  \childdoctrue
  \childdocmanualtrue
  \if?#1?\else
    \def\jobname{#2}
  \fi
  \def\childdocjob{#2}
  \input{#2}
  \endinput
}
%    \end{macrocode}

% \macro{\childdocforward}
% The command |\childdocforward| redirects
% compilation to the main file or
% (if the optional argument is given) a child file.
% Parameters are set as if the main file
% or a child file starting with |\childdocof| was compiled.
% Then compilation is handed over to the main file:
%    \begin{macrocode}
\newcommand{\childdocforward}[2][]
{
  \begingroup
    \if?#1?
      \def\childdoctmp
      {
        \def\childdocname{#2}
        \def\childdocjob{#2}
        \def\jobname{#2}
        \input{#2}
        \endinput
      }
    \else
      \def\childdoctmp
      {
        \childdocdisable
        \def\childdocname{#2}
        \childdoctrue
        \includeonly{#2}
        \def\childdocjob{#1}
        \def\jobname{#1}
        \input{#1}
        \endinput
      }
    \fi
    \expandafter
  \endgroup
  \childdoctmp
}
%    \end{macrocode}

% \macro{\childdocforwardprefix}
% The command |\childdocforwardprefix| redirects
% compilation to the main or a child file by means of a pattern.
% The prefix |#1| in the current filename is replaced by |#2|
% and the suffix of the current filename is kept
% (it is assumed that the filename does not contain the substring `|~~~|'
% which is used as a delimiter).
% Compilation is handed over to the new file by |\childdocforward|:
%    \begin{macrocode}
\newcommand{\childdocforwardprefix}[3][]
{
  \begingroup
    \def\childdocextract #2##1~~~{\def\childdoctmp{\childdocforward[#1]{#3##1}}}
    \expandafter\childdocextract\childdocname~~~
    \expandafter
  \endgroup
  \childdoctmp
}
%    \end{macrocode}

% \macro{\childdoc}
% The deprecated macro |\childdoc| is a legacy version of |\childdocmain|:
%    \begin{macrocode}
\newcommand{\childdoc}{\childdocmain}
%    \end{macrocode}

% \macro{\childdocredirect}
% The deprecated macro |\childdocredirect| is a legacy version
% of |\childdocforward| and |\childdocforwardprefix|:
%    \begin{macrocode}
\newcommand{\childdocredirect}[2][]
{
  \begingroup
    \if?#1?
      \def\childdoctmp{\childdocforward{#2}}
    \else
      \def\childdoctmp{\childdocforwardprefix{#1}{#2}}
    \fi
    \expandafter
  \endgroup
  \childdoctmp
}
%    \end{macrocode}

%\iffalse
%</package>
%\fi
%
\endinput

\childdocforward{cdocsamp}
%    \end{macrocode}

%\iffalse
%</sampledraft>
%\fi
%
% %%%%%%%%%%%%%%%%%%%%%%%%%%%%%%%%%%%%%%
% \paragraph{Forwarding for Final Version of the Chapters.}
%
% The following forwarding files |cdocsfn1.tex| and |cdocsfn2.tex|
% (with identical content)
% compile the final versions of the child documents
% |cdocsch1.tex| and |cdocsch2.tex|, respectively:
%\iffalse
%<*samplefinal>
%\fi
%    \begin{macrocode}
\def\version{final}
% \iffalse
%
% childdoc.dtx Copyright (C) 2017-2018 Niklas Beisert
%
% This work may be distributed and/or modified under the
% conditions of the LaTeX Project Public License, either version 1.3
% of this license or (at your option) any later version.
% The latest version of this license is in
%   http://www.latex-project.org/lppl.txt
% and version 1.3 or later is part of all distributions of LaTeX
% version 2005/12/01 or later.
%
% This work has the LPPL maintenance status `maintained'.
%
% The Current Maintainer of this work is Niklas Beisert.
%
% This work consists of the files childdoc.dtx and childdoc.ins
% and the derived files childdoc.def and cdocsamp.tex with
% cdocsch1.tex, cdocsch2.tex, cdocsdrf.tex, cdocsfn1.tex, cdocsfn2.tex.
%
%<package>\ifdefined\childdocmain\endinput\fi
%<package>\ProvidesFile{childdoc.def}[2018/12/30 v2.0 child document driver]
%<samplemain>\ProvidesFile{cdocsamp.tex}[2018/12/30 v2.0 sample for childdoc]
%<*driver>
%\ProvidesFile{childdoc.drv}[2018/12/30 v2.0 childdoc reference manual file]
\PassOptionsToClass{10pt,a4paper}{article}
\documentclass{ltxdoc}

\usepackage[margin=35mm]{geometry}
\usepackage{hyperref}
\usepackage{hyperxmp}
\usepackage[usenames]{color}

\hypersetup{colorlinks=true}
\hypersetup{pdfstartview=FitH}
\hypersetup{pdfpagemode=UseNone}
\hypersetup{pdfsource={}}
\hypersetup{pdflang={en-UK}}
\hypersetup{pdfcopyright={Copyright 2017-2018 Niklas Beisert.
  This work may be distributed and/or modified under the
  conditions of the LaTeX Project Public License, either version 1.3
  of this license or (at your option) any later version.}}
\hypersetup{pdflicenseurl={http://www.latex-project.org/lppl.txt}}
\hypersetup{pdfcontactaddress={ETH Zurich, ITP, HIT K,
  Wolfgang-Pauli-Strasse 27}}
\hypersetup{pdfcontactpostcode={8093}}
\hypersetup{pdfcontactcity={Zurich}}
\hypersetup{pdfcontactcountry={Switzerland}}
\hypersetup{pdfcontactemail={nbeisert@itp.phys.ethz.ch}}
\hypersetup{pdfcontacturl={http://people.phys.ethz.ch/\xmptilde nbeisert/}}

\newcommand{\secref}[1]{\hyperref[#1]{section \ref*{#1}}}

\parskip1ex
\parindent0pt
\let\olditemize\itemize
\def\itemize{\olditemize\parskip0pt}

\begin{document}

\title{The \textsf{childdoc} Package}
\hypersetup{pdftitle={The childdoc Package}}
\author{Niklas Beisert\\[2ex]
  Institut f\"ur Theoretische Physik\\
  Eidgen\"ossische Technische Hochschule Z\"urich\\
  Wolfgang-Pauli-Strasse 27, 8093 Z\"urich, Switzerland\\[1ex]
  \href{mailto:nbeisert@itp.phys.ethz.ch}
  {\texttt{nbeisert@itp.phys.ethz.ch}}}
\hypersetup{pdfauthor={Niklas Beisert}}
\hypersetup{pdfsubject={Manual for the LaTeX2e Package childdoc}}
\date{30 December 2018, \textsf{v2.0}}
\maketitle

\begin{abstract}\noindent
\textsf{childdoc} is a \LaTeXe{} package
that enables the direct compilation
of document sections included by |\include|
to individual files.
\end{abstract}

\begingroup
\parskip0ex
\tableofcontents
\endgroup

%%%%%%%%%%%%%%%%%%%%%%%%%%%%%%%%%%%%%%%%%%%%%%%%%%%%%%%%%%%%%%%%%%%%%%%%%%%%%%%%
%%%%%%%%%%%%%%%%%%%%%%%%%%%%%%%%%%%%%%%%%%%%%%%%%%%%%%%%%%%%%%%%%%%%%%%%%%%%%%%%
\section{Introduction}

\LaTeX{} provides a mechanism to structure a large document (such as a book)
into a main file and several child files (containing the chapters)
using the |\include| command.
This mechanism is beneficial for documents
which span hundreds of pages in order to
make the source file(s) more manageable.
Moreover, compilation can be restricted to
selected child files by means of the |\includeonly| command.
The latter feature can be used to reduce the compilation time while editing
(this was significantly more useful in the earlier days of \LaTeX{})
or to generate a smaller document which is easier to navigate.
Another application of |\includeonly| is to generate
documents consisting of selected parts of the complete document.

However, there are a few drawbacks of the plain |\include| mechanism:
\begin{itemize}
\item
The child files cannot be compiled on their own,
they can only be compiled via the main file.
A naive editing environment
(such as a text editor with an option
to have the current file processed by \LaTeX)
may require one to switch to the main file before compiling;
attempting to compile the child file produces errors.
\item
The main file must be modified (each time)
to adjust the |\includeonly| command
to the present needs. This easily leaves the main file in a messy state.
\item
The generated document will always carry the filename
of the main document. This is inconvenient if
several child files are to be compiled and
to be kept for distribution.
\end{itemize}

The present package provides a simple interface
to make child files individually compilable by \LaTeX{}.
Compiling a child file then has the same effect as compiling
the main file with an |\includeonly| command
to select the appropriate child.
Moreover the generated document will carry the name of the child
rather than the main file.
This resolves all three above issues.

This feature is meant to make the editing of books,
thesis documents and lecture notes somewhat more convenient.
However, the package can also be used efficiently for
composing a series of documents (such as exercise sheets)
which are typically distributed individually.
It then assists the author in generating the individual documents
(potentially in different versions)
as well as a document containing the collected series.
Another application is in developing style files
or other kinds of included material
where compilation of the style file could redirect
to a sample or test file.

%%%%%%%%%%%%%%%%%%%%%%%%%%%%%%%%%%%%%%%%%%%%%%%%%%%%%%%%%%%%%%%%%%%%%%%%%%%%%%%%
%%%%%%%%%%%%%%%%%%%%%%%%%%%%%%%%%%%%%%%%%%%%%%%%%%%%%%%%%%%%%%%%%%%%%%%%%%%%%%%%
\section{Usage}

First of all, the package \textsf{childdoc} is \emph{not} a standard
\LaTeXe{} |.sty| style file! Therefore it needs to be invoked in
a non-standard way.

%%%%%%%%%%%%%%%%%%%%%%%%%%%%%%%%%%%%%%%%%%%%%%%%%%%%%%%%%%%%%%%%%%%%%%%%%%%%%%%%
\subsection{Included Files}
\label{sec:include}

%%%%%%%%%%%%%%%%%%%%%%%%%%%%%%%%%%%%%%%%
\DescribeMacro{\childdocmain}
To use the package, add the commands
\begin{center}
\begin{tabular}{l}
|\input{childdoc.def}|\\
|\childdocmain{}|\\
\end{tabular}
\end{center}
at the very top of the main \LaTeX{} file,
in particular \emph{before} the |\documentclass| statement!
The argument of |\childdocmain| should be left empty
(but it must be present).

%%%%%%%%%%%%%%%%%%%%%%%%%%%%%%%%%%%%%%%%
\DescribeMacro{\childdocof}
Furthermore, add the commands
\begin{center}
\begin{tabular}{l}
|\input{childdoc.def}|\\
|\childdocof{|\textit{main}|}|\\
\end{tabular}
\end{center}
at the top of every child file \textit{child}
which is included by |\include{|\textit{child}|}|
from within the main file
(or at least for those files to be compiled individually).
The argument \textit{main} must be the filename of the main file.

There are a couple of
considerations in setting up the main and child documents:

%%%%%%%%%%%%%%%%%%%%%%%%%%%%%%%%%%%%%%%%
\paragraph{Restrictions.}

Please note the following restrictions:
\begin{itemize}
\item
|\childdocmain| must be called with one argument \textit{main}
to ensure compatibility with earlier version of the package.
It must either be empty (|\childdocmain{}|)
or precisely match the filename of the main file in which it is specified.
See \secref{sec:detection} for further information.
\item
The filename \textit{main} must be specified without the |.tex| extension.
\item
The filename \textit{main} is case sensitive
(even in case-insensitive file systems)
due to internal string comparison.
\item
The argument \textit{main} should be fully expanded, it cannot be a macro.
\item
Subdirectories and special characters should be avoided in filenames.
\item
The command |\childdocmain{|\textit{main}|}| must be followed by a whitespace.
It should not be followed immediately by another command
or by a comment mark `|%|'.
This is because the \TeX{} parser reads the token immediately following
the argument of |\childdocmain| and puts it
at the beginning of every child section;
however, a white\-space is ignored.
\end{itemize}

%%%%%%%%%%%%%%%%%%%%%%%%%%%%%%%%%%%%%%%%
\paragraph{Content of Main File.}

It is advisable to place all content in the child files included by |\include|.
Any output contained in the main file will appear in all child documents
unless suppressed manually;
it cannot be suppressed automatically by the |\includeonly| directive
and thus should normally be avoided.
A method to include some content in the main file
by means of conditional processing is described in \secref{sec:conditional}.

%%%%%%%%%%%%%%%%%%%%%%%%%%%%%%%%%%%%%%%%
\paragraph{Page Numbering.}

When only a part of the document is compiled,
the appropriate numbering of pages
(as well as other status parameters)
is determined from the |.aux| files.
The latter contain information from previous passes.
However this information needs to propagate through
all intermediate child documents.
Therefore the page numbering in child documents may well
be inconsistent until the complete document is compiled at least once.

A useful (if unconventional) way to always ensure a consistent
page numbering is to restart the numbering in each child document
and denote the pages by `\textit{child}|.|\textit{page}'
where \textit{child} represents the chapter/section number of the child file.
This can be achieved by the command
|\numberwithin{page}{|\textit{child}|}|
of the \textsf{amsmath} package
where \textit{child} can be |chapter| or |section|
depending on the chosen structuring.
Alternatively, one can modify the macro |\thepage| appropriately
and reset the counter |page| at the start of each child file.

%%%%%%%%%%%%%%%%%%%%%%%%%%%%%%%%%%%%%%%%%%%%%%%%%%%%%%%%%%%%%%%%%%%%%%%%%%%%%%%%
\subsection{Conditional Processing}
\label{sec:conditional}

The package provides a mechanism to compile different versions
of a document. To customise the versions further some conditional processing
can come in handy to distinguish which version is being compiled.
The package provides two macros to describe the compilation context:

%%%%%%%%%%%%%%%%%%%%%%%%%%%%%%%%%%%%%%%%
\DescribeMacro{\ifchilddoc}
The conditional |\ifchilddoc| distinguishes between the compilation of
child documents and the main document:
%
\begin{center}
|\ifchilddoc |\textit{child-code}| |[|\||else |\textit{main-code}]| \||fi|
\end{center}

%%%%%%%%%%%%%%%%%%%%%%%%%%%%%%%%%%%%%%%%
\DescribeMacro{\childdocname}
\DescribeMacro{\childdocjob}
The macro |\childdocname| contains the filename (without extension)
of the main or child file being processed.
Note that |\childdocjob| will always contain the name of the main file.

%%%%%%%%%%%%%%%%%%%%%%%%%%%%%%%%%%%%%%%%
\paragraph{Title Page.}

Conditional processing can be used to include a title or banner page
in the main document when proper precautions are taken.
Importantly, the code in the main file should ensure that the page counter
(as well as other status parameters which are stored in the |.aux| files)
takes the same value after the conditional processing.
Otherwise the page numbers may take divergent values
depending on which part is compiled.

For example, a title page could be declared by:
%
\begin{center}
\begin{tabular}{l}
|\ifchilddoc\||else|\\
|\addtocounter{page}{-1}|\\
\textit{code for title page}\\
|\newpage|\\
|\||fi|
\end{tabular}
\end{center}
%
A banner page for the child documents can be generated by:
%
\begin{center}
\begin{tabular}{l}
|\ifchilddoc|\\
|\addtocounter{page}{-1}|\\
\textit{code for banner page}\\
|\newpage|\\
|\||fi|
\end{tabular}
\end{center}
%
Here one could write a message such as:
\begin{center}
|This is the part \childdocname{} of \childdocjob{}.|
\end{center}

%%%%%%%%%%%%%%%%%%%%%%%%%%%%%%%%%%%%%%%%%%%%%%%%%%%%%%%%%%%%%%%%%%%%%%%%%%%%%%%%
\subsection{Flags}
\label{sec:flags}

The package makes it easy to generate different versions
of the main or child documents.
To this end compilation flags can be defined
and assigned different default values.
They will be particularly useful in conjunction
with the forwarding mechanism described in \secref{sec:forward}.

For example, it may be useful to have a flag |\version|
which can be set to |draft| or |final|.
The document source will contain some conditional code
depending on the value of |\version|.
Suppose further, the flag should default to |final| for the main file
and to |draft| for child files
which is a natural assignment for editing the document.
This is achieved by placing the following code
in the preamble of the main document
(below the |\childdocmain| directive):
%
\begin{center}
\begin{tabular}{l}
|\ifchilddoc|\\
|\providecommand{\version}{draft}|\\
|\||else|\\
|\providecommand{\version}{final}|\\
|\||fi|
\end{tabular}
\end{center}
%
The definition by |\providecommand| makes sure
that previous definitions are not overwritten.
Further statements |\providecommand{\version}{...}|
can thus be added before the above code to override it.

For the main file, one might add a line
(between |\childdocmain| and the above block)
%
\begin{center}
|%\ifchilddoc\||else\providecommand{\version}{draft}\||fi|
\end{center}
%
which can be uncommented to produce a draft version.
Likewise one can add a line to the very top of a child file
(above the |\childdocof{|\textit{main}|}| directive)
%
\begin{center}
|%\providecommand{\version}{final}|
\end{center}
%
which can be uncommented to produce the final version of this child document.

%%%%%%%%%%%%%%%%%%%%%%%%%%%%%%%%%%%%%%%%%%%%%%%%%%%%%%%%%%%%%%%%%%%%%%%%%%%%%%%%
\subsection{Forwarding}
\label{sec:forward}

Different versions of the main or child documents
using compilation flags as described in \secref{sec:flags}
can be (permanently) stored in different files
for convenient compilation, viewing and distribution.
To this end, the package defines a command
to pass on compilation to a different file:

%%%%%%%%%%%%%%%%%%%%%%%%%%%%%%%%%%%%%%%%
\DescribeMacro{\childdocforward}
The command |\childdocforward| redirects processing to
another source file:
%
\begin{center}
\begin{tabular}{l}
|\input{childdoc.def}|\\
|\childdocforward[|\textit{main}|]{|\textit{dest}|}|\\
\end{tabular}
\end{center}
%
The argument \textit{dest} is the destination file
(without extension).
It should be the main file or one of the child files.
Note that further \textsf{childdoc} directives
such as |\childdocof| and |\childdocforward|
in the indicated file will be processed in this form.
The optional argument \textit{main}
passes on directly to the main file \textit{main}
while pretending to compile the child \textit{dest}.
This form behaves as if \textit{dest}
issues |\childdocof{|\textit{main}|}| right away,
and no further \textsf{childdoc} directives will be processed.

%%%%%%%%%%%%%%%%%%%%%%%%%%%%%%%%%%%%%%%%
\DescribeMacro{\...prefix}
In the alternative form |\childdocforwardprefix|,
%
\begin{center}
\begin{tabular}{l}
|\input{childdoc.def}|\\
|\childdocforwardprefix[|\textit{main}|]{|\textit{prefix}|}{|\textit{dest}|}|
\end{tabular}
\end{center}
%
the destination file is determined by a pattern
depending on the current file:
To make this work, the current file must be called
`{\textit{prefix}\hspace{0.2em}\textit{suffix}}'
with \textit{prefix} matching precisely the argument.
Processing is then passed on to the file
`{\textit{dest}\hspace{0.2em}\textit{suffix}}'.
Surely, the same effect is achieved by
directly specifying the
argument `{\textit{dest}\hspace{0.2em}\textit{suffix}}'
in the first form.
However, that requires to set up a different file
for each child. With the alternative form of the command
all these files can have exactly the same content
which simplifies setting them up and maintaining them.

For example, the following file |draft.tex|
with a compilation flag |\version| as described in \secref{sec:flags}
compiles the main document as a draft:
%
\begin{center}
\begin{tabular}{l}
|\def\version{draft}|\\
|\input{childdoc.def}|\\
|\childdocforward{|\textit{main}|}|
\end{tabular}
\end{center}
%
Likewise, the following files |final|\textit{nn}|.tex|
compile the final version of the child document
|child|\textit{nn}|.tex|:
%
\begin{center}
\begin{tabular}{l}
|\def\version{final}|\\
|\input{childdoc.def}|\\
|\childdocforwardprefix{final}{child}|
\end{tabular}
\end{center}
%

Note that when several versions of a main file and/or of each child file
are to be generated, it may be convenient to set up a |Makefile| or
shell script to automatise the process.

%%%%%%%%%%%%%%%%%%%%%%%%%%%%%%%%%%%%%%%%%%%%%%%%%%%%%%%%%%%%%%%%%%%%%%%%%%%%%%%%
\subsection{Command Line Processing}
\label{sec:commandline}

The effect of redirection files can also be achieved by invoking
the \LaTeX{} compiler with a more elaborate command line.
Most conveniently this should be done as part
of a shell script or a |Makefile|.

When using \textsf{childdoc} in the main file, the following
command lines effectively perform a redirection
(note that depending on the shell being used,
backslashes may have to be doubled: `|\|' $\to$ `|\\|'):
%
\begin{center}
|... -jobname "|\textit{target}|" |\\|"|[\textit{flags}]%
|\input{childdoc.def}\childdocforward[|\textit{main}|]{|\textit{dest}|}"|
\end{center}
%
Here \textit{target} is the name of the output file,
\textit{main} is the name of the main file
and \textit{dest} is the name of the main or child file to be processed
(all filenames without extensions).
The optional argument \textit{main} can be omitted
if \textit{main} matches \textit{dest}.
Optionally, compilation \textit{flags} can be defined via |\def| commands.
This command line makes the \TeX{} engine believe
it is compiling the file \textit{target}
whose content is specified as the latter parameter.
The provided code then forwards the processing to
\textit{main} or \textit{dest} as described in \secref{sec:forward}.

%%%%%%%%%%%%%%%%%%%%%%%%%%%%%%%%%%%%%%%%%%%%%%%%%%%%%%%%%%%%%%%%%%%%%%%%%%%%%%%%
\subsection{Include by Input}
\label{sec:input}

Including child documents by |\include| has some restrictions by design.
Most notably, the content of a child document always occupies
its own set of pages; pages cannot be shared between child documents.
Usually, this behaviour makes perfect sense
because each child document contain an essential part of the document.
However, in some situations it may be desirable to compose
a document from a collection of parts
without having mandatory page breaks between then.
For this case, the package
provides a mechanism to include parts
by |\input| which can also be processed individually.
However, by construction this mechanism
requires manual handling of the content to be output.

%%%%%%%%%%%%%%%%%%%%%%%%%%%%%%%%%%%%%%%%
\DescribeMacro{\ifchilddocmanual}
The main file should be prepared as usual, see \secref{sec:include}.
However, the document body must make a distinction
between processing of an individual part and of the main document, e.g.:
%
\begin{center}
\begin{tabular}{l}
|\ifchilddocmanual|\\
|\input{\childdocname}|\\
|\||else|\\
\textit{document body with }|\input{|\textit{part}|}|\\
|\||fi|
\end{tabular}
\end{center}
%
The conditional |\ifchilddocmanual| is true whenever
a part to be included by |\input| is being compiled,
and the name of the part is stored in |\childdocname|.

%%%%%%%%%%%%%%%%%%%%%%%%%%%%%%%%%%%%%%%%
\DescribeMacro{\childdocby}
Each part to be included by |\input| should start with:
%
\begin{center}
\begin{tabular}{l}
|\input{childdoc.def}|\\
|\childdocby{|\textit{main}|}|\\
\end{tabular}
\end{center}
%
The directive |\childdocby| is similar to |\childdocof|
described in \secref{sec:include},
but the subsequent selection of content must be done manually.
To that end, both |\ifchilddoc| and |\ifchilddocmanual|
will be true upon processing of a part,
and the name of the part is stored in |\childdocname|.
Note that |\jobname| will be set to the filename of the current part
so that each part receives an individual |.aux| file
that does not interfere with the |.aux| file(s) of the main document.
This behaviour can be altered by the alternative form
|\childdocby[*]{|\textit{main}|}| (with a non-empty optional argument)
which uses the |.aux| file of the main document
by setting |\jobname| to \textit{main}.

%%%%%%%%%%%%%%%%%%%%%%%%%%%%%%%%%%%%%%%%%%%%%%%%%%%%%%%%%%%%%%%%%%%%%%%%%%%%%%%%
\subsection{Driver Development}
\label{sec:driver}

The \textsf{childdoc} mechanism can also be use for the development
of definition files such as \LaTeX{} styles or classes.
This case differs from the above setup with multiple parts
included by |\include| in that no |\includeonly| should be invoked.
This can be achieved by starting the include file
(before |\ProvidesPackage|) with:
%
\begin{center}
\begin{tabular}{l}
|\input{childdoc.def}|\\
|\childdocforward{|\textit{main}|}|\\
\end{tabular}
\end{center}
%
or alternatively with:
%
\begin{center}
\begin{tabular}{l}
|\input{childdoc.def}|\\
|\childdocby{|\textit{main}|}|\\
\end{tabular}
\end{center}
%
Both forms have slightly different effects as described above.
The main file is prepared as usual, see \secref{sec:include}.

%%%%%%%%%%%%%%%%%%%%%%%%%%%%%%%%%%%%%%%%%%%%%%%%%%%%%%%%%%%%%%%%%%%%%%%%%%%%%%%%
\subsection{Legacy Detection}
\label{sec:detection}

The directive |\childdocmain| in the main file can detect
whether the complete document or merely a child is to be compiled
even without using the directive |\childdocof|.
This method is deprecated because it is less robust
and there is no compelling reason to use it;
it is merely provided for backward compatibility
and it may be removed in future versions.

If the detection mechanism is to be used,
it is mandatory to correctly specify
the filename of the main file as the argument of |\childdocmain|:
%
\begin{center}
\begin{tabular}{l}
|\input{childdoc.def}|\\
|\childdocmain{|\textit{main}|}|\\
\end{tabular}
\end{center}
%
If |\jobname| does not match the argument \textit{main} of |\childdocmain|,
it is assumed that |\jobname| points to the child file to be compiled.
When using |\childdocmain| with the main file specified as argument,
it suffices to start a child file
with just |\input{|\textit{main}|}|
without loading of the package and using |\childdocof|.
If instead all processing is done
with the appropriate \textsf{childdoc} directives,
the argument of \textit{main} of |\childdocmain| can be empty.

An alternative version of the command line processing described
in \secref{sec:commandline} using the detection mechanism reads:
%
\begin{center}
|... -jobname "|\textit{target}|" "|[\textit{flags}]%
[|\def\jobname{|\textit{dest}|}|]|\input{|\textit{main}|}"|
\end{center}

%%%%%%%%%%%%%%%%%%%%%%%%%%%%%%%%%%%%%%%%%%%%%%%%%%%%%%%%%%%%%%%%%%%%%%%%%%%%%%%%
\subsection{Manual Code}
\label{sec:manual}

In case one cannot be certain whether the definitions file |childdoc.def|
is installed on the target \TeX{} distribution
and one prefers not to ship it,
it is conceivable to paste a few relevant commands into the sources.

To that end, drop all statements |\input{childdoc.def}|
and perform the replacements as outlined below.
Instead of |\childdocmain{|\textit{main}|}| add the following code
to the top of the main file:
%
\begin{center}
\begin{tabular}{l}
|\||ifdefined\childdocname\endinput\||fi\newif\ifchilddoc|\\
|\edef\childdocname{\scantokens\expandafter{\jobname\noexpand}}|\\
|\def\childdocmain{|\textit{main}|}\||ifx\childdocmain\childdocname\||else|\\
|\childdoctrue\includeonly{\childdocname}\let\jobname\childdocmain\||fi|\\
\end{tabular}
\end{center}
%
Instead of |\childdocof{|\textit{main}|}| just include the main file
at the top of each child file:
%
\begin{center}
|\input{|\textit{main}|}|
\end{center}
%
A simple redirection |\childdocforward{|\textit{dest}|}| is achieved by:
%
\begin{center}
|\def\jobname{|\textit{dest}|}\input{\jobname}|
\end{center}
%
The redirection with prefix
|\childdocforwardprefix[|\textit{prefix}|]{|\textit{dest}|}|
is accomplished by:
%
\begin{center}
\begin{tabular}{l}
|{\edef\jobname{\scantokens\expandafter{\jobname\noexpand}}|\\
|\def\redirectjob |\textit{prefix}|#1~~~{\gdef\jobname{|\textit{dest}|#1}}|\\
|\expandafter\redirectjob\jobname~~~}\input{\jobname}|
\end{tabular}
\end{center}

In an alternative approach,
child documents can be compiled by a specific command line
without additional code or specific definitions:
%
\begin{center}
|... -jobname "|\textit{target}|" "|[\textit{flags}]%
|\includeonly{|\textit{dest}|}\input{|\textit{main}|}"|
\end{center}
%

%%%%%%%%%%%%%%%%%%%%%%%%%%%%%%%%%%%%%%%%%%%%%%%%%%%%%%%%%%%%%%%%%%%%%%%%%%%%%%%%
%%%%%%%%%%%%%%%%%%%%%%%%%%%%%%%%%%%%%%%%%%%%%%%%%%%%%%%%%%%%%%%%%%%%%%%%%%%%%%%%
\section{Information}

%%%%%%%%%%%%%%%%%%%%%%%%%%%%%%%%%%%%%%%%%%%%%%%%%%%%%%%%%%%%%%%%%%%%%%%%%%%%%%%%
\subsection{Copyright}

Copyright \copyright{} 2017--2018 Niklas Beisert

This work may be distributed and/or modified under the
conditions of the \LaTeX{} Project Public License, either version 1.3
of this license or (at your option) any later version.
The latest version of this license is in
  \url{http://www.latex-project.org/lppl.txt}
and version 1.3 or later is part of all distributions of \LaTeX{}
version 2005/12/01 or later.

This work has the LPPL maintenance status `maintained'.

The Current Maintainer of this work is Niklas Beisert.

This work consists of the files |README.txt|, |childdoc.ins| and |childdoc.dtx|
as well as the derived files |childdoc.def|, |cdocsamp.tex|
with |cdocsch1.tex|, |cdocsch2.tex|, |cdocspt3.tex|, |cdocspt4.tex|,
|cdocsdrf.tex|, |cdocsfn1.tex|, |cdocsfn2.tex|
as well as |childdoc.pdf|.

%%%%%%%%%%%%%%%%%%%%%%%%%%%%%%%%%%%%%%%%%%%%%%%%%%%%%%%%%%%%%%%%%%%%%%%%%%%%%%%%
\subsection{Files and Installation}

The package consists of the files:
%
\begin{center}
\begin{tabular}{ll}
    |README.txt|   & readme file \\
    |childdoc.ins| & installation file \\
    |childdoc.dtx| & source file \\
    |childdoc.def| & definition file \\
    |cdocsamp.tex| & sample main file \\
    |cdocsch1.tex| & sample include file \\
    |cdocsch2.tex| & sample include file \\
    |cdocspt3.tex| & sample part file \\
    |cdocspt4.tex| & sample part file \\
    |cdocsdrf.tex| & sample redirection file \\
    |cdocsfn1.tex| & sample redirection file \\
    |cdocsfn2.tex| & sample redirection file \\
    |childdoc.pdf| & manual
\end{tabular}
\end{center}
%
The distribution consists of the files
|README.txt|, |childdoc.ins| and |childdoc.dtx|.
%
\begin{itemize}
\item
Run (pdf)\LaTeX{} on |childdoc.dtx|
to compile the manual |childdoc.pdf| (this file).
\item
Run \LaTeX{} on |childdoc.ins| to create the definitions file |childdoc.def|
and the sample |cdocsamp.tex| with include files
|cdocsch1.tex|, |cdocsch2.tex|, |cdocspt3.tex|, |cdocspt4.tex|,
|cdocsdrf.tex|, |cdocsfn1.tex|, |cdocsfn2.tex|.
Then copy the file |childdoc.def| to an appropriate directory of your \LaTeX{}
distribution, e.g.\ \textit{texmf-root}|/tex/latex/childdoc|.
\end{itemize}

%%%%%%%%%%%%%%%%%%%%%%%%%%%%%%%%%%%%%%%%%%%%%%%%%%%%%%%%%%%%%%%%%%%%%%%%%%%%%%%%
\subsection{Related CTAN Packages}

There are several other packages which offer a similar functionality:
%
\begin{itemize}
\item
The packages
\href{http://ctan.org/pkg/docmute}{\textsf{docmute}},
\href{http://ctan.org/pkg/includex}{\textsf{includex}} and
\href{http://ctan.org/pkg/standalone}{\textsf{standalone}}
provide commands to include only the document body of
a child file thus allowing both files to be compiled individually.
\item
The packages \href{http://ctan.org/pkg/subdocs}{\textsf{subdocs}}
and \href{http://ctan.org/pkg/subfiles}{\textsf{subfiles}}
provide structures in which the main and child documents can be
encapsulated and allowing them to be compiled individually.
The inclusion mechanism is different from the conventional |\include|.
\item
The package \href{http://ctan.org/pkg/combine}{\textsf{combine}}
is an elaborate solution to combine several documents into one.
\end{itemize}
%
See also the CTAN topic \href{http://ctan.org/topic/subdocs}{\textsf{subdocs}}
for further related packages.
The present package differs from the above solutions in that
a document structure constructed with the conventional |\include| mechanism
just needs two extra commands at the top of every file
such that all constituent files can be compiled individually.

%%%%%%%%%%%%%%%%%%%%%%%%%%%%%%%%%%%%%%%%%%%%%%%%%%%%%%%%%%%%%%%%%%%%%%%%%%%%%%%%
%\subsection{Feature Suggestions}
%
%The following is a list of features which may be useful for future
%versions of this package:
%%
%\begin{itemize}
%\item
%\ldots
%\end{itemize}

%%%%%%%%%%%%%%%%%%%%%%%%%%%%%%%%%%%%%%%%%%%%%%%%%%%%%%%%%%%%%%%%%%%%%%%%%%%%%%%%
\subsection{Revision History}

%%%%%%%%%%%%%%%%%%%%%%%%%%%%%%%%%%%%%%%%
\paragraph{v2.0:} 2018/12/30

\begin{itemize}
\item
immediate forward processing
\item
added |\childdocby| mechanism
\item
manual restructured
\end{itemize}

%%%%%%%%%%%%%%%%%%%%%%%%%%%%%%%%%%%%%%%%
\paragraph{v1.6:} 2018/01/17

\begin{itemize}
\item
application for development of include files
\item
corrections to manual
\end{itemize}

%%%%%%%%%%%%%%%%%%%%%%%%%%%%%%%%%%%%%%%%
\paragraph{v1.5:} 2017/05/21

\begin{itemize}
\item
more complete structuring introduced
\item
|\childdocof| introduced
\item
|\childdoc| renamed to |\childdocmain|
\item
|\childredirect| renamed to |\childdocforward| and |\childdocforwardprefix|
and functionality expanded
\end{itemize}

%%%%%%%%%%%%%%%%%%%%%%%%%%%%%%%%%%%%%%%%
\paragraph{v1.0:} 2017/04/27

\begin{itemize}
\item
manual and install package
\item
first version published on CTAN
\end{itemize}

%%%%%%%%%%%%%%%%%%%%%%%%%%%%%%%%%%%%%%%%
\paragraph{v0.6:} 2017/04/26

\begin{itemize}
\item
redirection mechanism added
\end{itemize}

%%%%%%%%%%%%%%%%%%%%%%%%%%%%%%%%%%%%%%%%
\paragraph{v0.5:} 2017/04/26

\begin{itemize}
\item
functionality in definition file
\end{itemize}


%%%%%%%%%%%%%%%%%%%%%%%%%%%%%%%%%%%%%%%%%%%%%%%%%%%%%%%%%%%%%%%%%%%%%%%%%%%%%%%%
%%%%%%%%%%%%%%%%%%%%%%%%%%%%%%%%%%%%%%%%%%%%%%%%%%%%%%%%%%%%%%%%%%%%%%%%%%%%%%%%
%%%%%%%%%%%%%%%%%%%%%%%%%%%%%%%%%%%%%%%%%%%%%%%%%%%%%%%%%%%%%%%%%%%%%%%%%%%%%%%%
\appendix

\settowidth\MacroIndent{\rmfamily\scriptsize 000\ }

 \DocInput{childdoc.dtx}

\end{document}
%</driver>
% \fi
%
% %%%%%%%%%%%%%%%%%%%%%%%%%%%%%%%%%%%%%%%%%%%%%%%%%%%%%%%%%%%%%%%%%%%%%%%%%%%%%%
% %%%%%%%%%%%%%%%%%%%%%%%%%%%%%%%%%%%%%%%%%%%%%%%%%%%%%%%%%%%%%%%%%%%%%%%%%%%%%%
% \section{Sample}
%\iffalse
%<*samplemain>
%\fi
%
% The following presents a sample document
% with two chapters, two parts, a title page,
% a compile flag as well as three forwarding files to set the flag.
% It consists of eight |.tex| files:
% \begin{center}
% \begin{tabular}{ll}
% |cdocsamp.tex|&main file\\
% |cdocsch1.tex|&include file for chapter 1\\
% |cdocsch2.tex|&include file for chapter 2\\
% |cdocspt3.tex|&include file for part 3\\
% |cdocspt4.tex|&include file for part 4\\
% |cdocsdrf.tex|&forwarding file for main file in draft mode\\
% |cdocsfi1.tex|&forwarding file for final version of chapter 1\\
% |cdocsfi2.tex|&forwarding file for final version of chapter 2\\
% \end{tabular}
% \end{center}
% Each of the eight files can be compiled directly by the \LaTeX{} compiler.
%
% %%%%%%%%%%%%%%%%%%%%%%%%%%%%%%%%%%%%%%
% \paragraph{Main File.}
%
% The main file is called |cdocsamp.tex|.
%
% Load the \textsf{childdoc} definitions and
% declare the filename for the main document:
%    \begin{macrocode}
\input{childdoc.def}
\childdocmain{}
%    \end{macrocode}

% Optional override for |\version| flag:
%    \begin{macrocode}
%%\ifchilddoc\else\providecommand{\version}{draft}\fi
%    \end{macrocode}

% Define the default values for the |\version| flag
% (|final| for the main file and |draft| for childs):
%    \begin{macrocode}
\ifchilddoc
\providecommand{\version}{draft}
\else
\providecommand{\version}{final}
\fi
%    \end{macrocode}

% Load the standard document class:
%    \begin{macrocode}
\documentclass[12pt]{article}
%    \end{macrocode}

% Start the document body:
%    \begin{macrocode}
\begin{document}
%    \end{macrocode}

% Declare a title page.
% Print title, part of document being processed and version flag:
%    \begin{macrocode}
\addtocounter{page}{-1}
\begin{center}
{\LARGE\bfseries{}childdoc example\par}
\vspace{1cm}
\ifchilddoc
\ifchilddocmanual part\else chapter\fi:
`\childdocname' of `\childdocjob'\par
\else
main document: `\childdocjob'\par
\fi
version: \version\par
\end{center}
\newpage
%    \end{macrocode}

% Manually include selected file,
% otherwise process as usual:
%    \begin{macrocode}
\ifchilddocmanual
\section*{part `\childdocname'}
\input{\childdocname}
\else
%    \end{macrocode}

% Include the two chapters:
%    \begin{macrocode}
\include{cdocsch1}
\include{cdocsch2}
%    \end{macrocode}

% Include the two parts unless only chapters should be displayed:
%    \begin{macrocode}
\ifchilddoc\else
\section{part three}
\input{cdocspt3}
\section{part four}
\input{cdocspt4}
\fi
%    \end{macrocode}

% Process as usual until here:
%    \begin{macrocode}
\fi
%    \end{macrocode}

% End of document body:
%    \begin{macrocode}
\end{document}
%    \end{macrocode}
%\iffalse
%</samplemain>
%\fi
%
% %%%%%%%%%%%%%%%%%%%%%%%%%%%%%%%%%%%%%%
% \paragraph{Chapter Include Files.}
%
% The include files are called |cdocsch1.tex| and |cdocsch2.tex|.
%
%\iffalse
%<*samplechap1|samplechap2>
%\fi

% Optional override for |\version| flag:
%    \begin{macrocode}
%%\providecommand{\version}{final}
%    \end{macrocode}

% Include the main document:
%    \begin{macrocode}
\input{childdoc.def}
\childdocof{cdocsamp}
%    \end{macrocode}

%\iffalse
%</samplechap1|samplechap2>
%\fi
%
%\iffalse
%<*samplechap1>
%\fi
% Some text for chapter 1:
%    \begin{macrocode}
\section{one}
some text in chapter one
%    \end{macrocode}

%\iffalse
%</samplechap1>
%\fi
% Some text for chapter 2:
%\iffalse
%<*samplechap2>
%\fi
%    \begin{macrocode}
\section{two}
more text in chapter two
%    \end{macrocode}

%\iffalse
%</samplechap2>
%\fi
%
% %%%%%%%%%%%%%%%%%%%%%%%%%%%%%%%%%%%%%%
% \paragraph{Part Include Files.}
%
% The include files are called |cdocspt3.tex| and |cdocspt4.tex|.
%
%\iffalse
%<*samplepart3|samplepart4>
%\fi

% Optional override for |\version| flag:
%    \begin{macrocode}
%%\providecommand{\version}{final}
%    \end{macrocode}

% Include the main document:
%    \begin{macrocode}
\input{childdoc.def}
\childdocby{cdocsamp}
%    \end{macrocode}

%\iffalse
%</samplepart3|samplepart4>
%\fi
%
%\iffalse
%<*samplepart3>
%\fi
% Some text for part 3:
%    \begin{macrocode}
some text in part three
%    \end{macrocode}

%\iffalse
%</samplepart3>
%\fi
% Some text for part 4:
%\iffalse
%<*samplepart4>
%\fi
%    \begin{macrocode}
more text in part four
%    \end{macrocode}

%\iffalse
%</samplepart4>
%\fi
%
% %%%%%%%%%%%%%%%%%%%%%%%%%%%%%%%%%%%%%%
% \paragraph{Forwarding for a Complete Draft.}
%
% The following forwarding file |cdocsdrf.tex|
% compiles the main document in draft mode:
%\iffalse
%<*sampledraft>
%\fi
%    \begin{macrocode}
\def\version{draft}
\input{childdoc.def}
\childdocforward{cdocsamp}
%    \end{macrocode}

%\iffalse
%</sampledraft>
%\fi
%
% %%%%%%%%%%%%%%%%%%%%%%%%%%%%%%%%%%%%%%
% \paragraph{Forwarding for Final Version of the Chapters.}
%
% The following forwarding files |cdocsfn1.tex| and |cdocsfn2.tex|
% (with identical content)
% compile the final versions of the child documents
% |cdocsch1.tex| and |cdocsch2.tex|, respectively:
%\iffalse
%<*samplefinal>
%\fi
%    \begin{macrocode}
\def\version{final}
\input{childdoc.def}
\childdocforwardprefix[cdocsamp]{cdocsfn}{cdocsch}
%    \end{macrocode}

%\iffalse
%</samplefinal>
%\fi
%
% %%%%%%%%%%%%%%%%%%%%%%%%%%%%%%%%%%%%%%
% \paragraph{Command Line Processing.}
%
% The following three command lines generate the output files
% |cdocscld|, |cdocscl1| and |cdocscl2|
% which should be identical to
% |cdocsdrf|, |cdocsch1| and |cdocsfn2|, respectively:
% \begin{center}
% \begin{tabular}{l}
% |latex -jobname cdocscld \|\\
% |  "\def\version{draft}\input{childdoc.def}\childdocforward{cdocsamp}"|\\
% |latex -jobname cdocscl1 \|\\
% |  "\input{childdoc.def}\childdocforward[cdocsamp]{cdocsch1}"|\\
% |latex -jobname cdocscl2 \|\\
% |  "\def\version{final}\input{childdoc.def}\childdocforward{cdocsch2}"|
% \end{tabular}
% \end{center}
% Note that the trailing backslash on each first line
% merely continues the input to the second line
% (for convenient cut ant paste).
% Furthermore, the command |latex| can be replaced by any
% of its alternative versions such as |pdflatex|.
%
% %%%%%%%%%%%%%%%%%%%%%%%%%%%%%%%%%%%%%%%%%%%%%%%%%%%%%%%%%%%%%%%%%%%%%%%%%%%%%%
% %%%%%%%%%%%%%%%%%%%%%%%%%%%%%%%%%%%%%%%%%%%%%%%%%%%%%%%%%%%%%%%%%%%%%%%%%%%%%%
% \section{Implementation}
%\iffalse
%<*package>
%\fi
%
% This section describes the definitions file |childdoc.def|.

% The definitions cannot be loaded using |\usepackage| or |\RequirePackage|
% which has a mechanism to prevent loading a style file more than once.
% When loading the definitions by means of |\input|
% multiple instances have to be prevented manually:
%\iffalse
%This code needs to be before the `\ProvidesFile' directive
%which is defined at the beginning of this file.
%Therefore it is also placed there and commented out here.
%</package>
%<*discard>
%\fi
%    \begin{macrocode}
\ifdefined\childdocmain\endinput\fi
%    \end{macrocode}
%\iffalse
%</discard>
%<*package>
%\fi
%
% \macro{\ifchilddoc}
% \macro{\ifchilddocmanual}
% The conditional |\ifchilddoc| tells whether a
% child (true) or main (false) document is being compiled.
% The conditional |\ifchilddocmanual| tells whether
% the |\includeonly| mechanism is used (false) or
% the selection of child files must be performed manually (true).
% The definitions initialise to false:
%    \begin{macrocode}
\newif\ifchilddoc
\newif\ifchilddocmanual
%    \end{macrocode}

% \macro{\childdocname}
% \macro{\childdocjob}
% The macro |\childdocname| stores the name of the main document
% to be compiled. The macro |\childdocjob| stores the name of
% the document on which the \LaTeX{} compiler was originally invoked.
% The content of |\jobname| cannot be compared
% to filenames specified in the source due to different catcodes.
% The following code rescans |\jobname|, stores the result
% in |\childdocname| and saves a copy in |\childdocjob|:
%    \begin{macrocode}
\edef\childdocname{\scantokens\expandafter{\jobname\noexpand}}
\let\childdocjob\childdocname
%    \end{macrocode}

% \macro{\childdocdisable}
% The macro |\childdocdisable| prevents the main file
% from being processed more than once.
% At this stage, the main document command |\childdocmain|
% is assumed to be called once again where it should do nothing.
% Any subsequent call to it should prevent
% a secondary processing of the main document
% It overwrites the forwarding commands
% |\childdocof| and |\childdocforward|
% with empty macros to prevent further inclusions of the main document:
%    \begin{macrocode}
\newcommand{\childdocdisable}
{
  \renewcommand{\childdocmain}[1]{\renewcommand{\childdocmain}[1]{\endinput}}
  \renewcommand{\childdocof}[1]{}
  \renewcommand{\childdocby}[2][]{}
  \renewcommand{\childdocforward}[2][]{}
  \renewcommand{\childdocdisable}{}
}
%    \end{macrocode}

% \macro{\childdocmain}
% The macro |\childdocmain| is to be called at the top of the main file
% with nothing or the main filename (without extension) as argument.
% First, it breaks loops.
% If the argument is not empty and does not match |\childdocname|
% (which is set by the first inclusion of |childdoc.def|),
% |\ifchilddoc| is set to true, |\includeonly| is applied to the child file
% and |\jobname| is set to the main file
% (for proper handling of |.aux| files):
%    \begin{macrocode}
\newcommand{\childdocmain}[1]
{
  \childdocdisable\childdocmain{}
  \if?#1?\else
    \begingroup
      \def\childdoctmp{#1}
      \ifx\childdoctmp\childdocname
        \def\childdoctmp{}
      \else
        \def\childdoctmp
        {
          \childdoctrue
          \includeonly{\childdocname}
          \def\childdocjob{#1}
          \def\jobname{#1}
        }
      \fi
      \expandafter
    \endgroup
    \childdoctmp
  \fi
}
%    \end{macrocode}

% \macro{\childdocof}
% The command |\childdocof| redirects
% compilation to the main file |#1|.
%    \begin{macrocode}
\newcommand{\childdocof}[1]
{
  \childdocdisable
  \childdoctrue
  \includeonly{\childdocname}
  \def\jobname{#1}
  \def\childdocjob{#1}
  \input{#1}
}
%    \end{macrocode}

% \macro{\childdocby}
% The command |\childdocby| ....
%    \begin{macrocode}
\newcommand{\childdocby}[2][]
{
  \childdocdisable
  \childdoctrue
  \childdocmanualtrue
  \if?#1?\else
    \def\jobname{#2}
  \fi
  \def\childdocjob{#2}
  \input{#2}
  \endinput
}
%    \end{macrocode}

% \macro{\childdocforward}
% The command |\childdocforward| redirects
% compilation to the main file or
% (if the optional argument is given) a child file.
% Parameters are set as if the main file
% or a child file starting with |\childdocof| was compiled.
% Then compilation is handed over to the main file:
%    \begin{macrocode}
\newcommand{\childdocforward}[2][]
{
  \begingroup
    \if?#1?
      \def\childdoctmp
      {
        \def\childdocname{#2}
        \def\childdocjob{#2}
        \def\jobname{#2}
        \input{#2}
        \endinput
      }
    \else
      \def\childdoctmp
      {
        \childdocdisable
        \def\childdocname{#2}
        \childdoctrue
        \includeonly{#2}
        \def\childdocjob{#1}
        \def\jobname{#1}
        \input{#1}
        \endinput
      }
    \fi
    \expandafter
  \endgroup
  \childdoctmp
}
%    \end{macrocode}

% \macro{\childdocforwardprefix}
% The command |\childdocforwardprefix| redirects
% compilation to the main or a child file by means of a pattern.
% The prefix |#1| in the current filename is replaced by |#2|
% and the suffix of the current filename is kept
% (it is assumed that the filename does not contain the substring `|~~~|'
% which is used as a delimiter).
% Compilation is handed over to the new file by |\childdocforward|:
%    \begin{macrocode}
\newcommand{\childdocforwardprefix}[3][]
{
  \begingroup
    \def\childdocextract #2##1~~~{\def\childdoctmp{\childdocforward[#1]{#3##1}}}
    \expandafter\childdocextract\childdocname~~~
    \expandafter
  \endgroup
  \childdoctmp
}
%    \end{macrocode}

% \macro{\childdoc}
% The deprecated macro |\childdoc| is a legacy version of |\childdocmain|:
%    \begin{macrocode}
\newcommand{\childdoc}{\childdocmain}
%    \end{macrocode}

% \macro{\childdocredirect}
% The deprecated macro |\childdocredirect| is a legacy version
% of |\childdocforward| and |\childdocforwardprefix|:
%    \begin{macrocode}
\newcommand{\childdocredirect}[2][]
{
  \begingroup
    \if?#1?
      \def\childdoctmp{\childdocforward{#2}}
    \else
      \def\childdoctmp{\childdocforwardprefix{#1}{#2}}
    \fi
    \expandafter
  \endgroup
  \childdoctmp
}
%    \end{macrocode}

%\iffalse
%</package>
%\fi
%
\endinput

\childdocforwardprefix[cdocsamp]{cdocsfn}{cdocsch}
%    \end{macrocode}

%\iffalse
%</samplefinal>
%\fi
%
% %%%%%%%%%%%%%%%%%%%%%%%%%%%%%%%%%%%%%%
% \paragraph{Command Line Processing.}
%
% The following three command lines generate the output files
% |cdocscld|, |cdocscl1| and |cdocscl2|
% which should be identical to
% |cdocsdrf|, |cdocsch1| and |cdocsfn2|, respectively:
% \begin{center}
% \begin{tabular}{l}
% |latex -jobname cdocscld \|\\
% |  "\def\version{draft}% \iffalse
%
% childdoc.dtx Copyright (C) 2017-2018 Niklas Beisert
%
% This work may be distributed and/or modified under the
% conditions of the LaTeX Project Public License, either version 1.3
% of this license or (at your option) any later version.
% The latest version of this license is in
%   http://www.latex-project.org/lppl.txt
% and version 1.3 or later is part of all distributions of LaTeX
% version 2005/12/01 or later.
%
% This work has the LPPL maintenance status `maintained'.
%
% The Current Maintainer of this work is Niklas Beisert.
%
% This work consists of the files childdoc.dtx and childdoc.ins
% and the derived files childdoc.def and cdocsamp.tex with
% cdocsch1.tex, cdocsch2.tex, cdocsdrf.tex, cdocsfn1.tex, cdocsfn2.tex.
%
%<package>\ifdefined\childdocmain\endinput\fi
%<package>\ProvidesFile{childdoc.def}[2018/12/30 v2.0 child document driver]
%<samplemain>\ProvidesFile{cdocsamp.tex}[2018/12/30 v2.0 sample for childdoc]
%<*driver>
%\ProvidesFile{childdoc.drv}[2018/12/30 v2.0 childdoc reference manual file]
\PassOptionsToClass{10pt,a4paper}{article}
\documentclass{ltxdoc}

\usepackage[margin=35mm]{geometry}
\usepackage{hyperref}
\usepackage{hyperxmp}
\usepackage[usenames]{color}

\hypersetup{colorlinks=true}
\hypersetup{pdfstartview=FitH}
\hypersetup{pdfpagemode=UseNone}
\hypersetup{pdfsource={}}
\hypersetup{pdflang={en-UK}}
\hypersetup{pdfcopyright={Copyright 2017-2018 Niklas Beisert.
  This work may be distributed and/or modified under the
  conditions of the LaTeX Project Public License, either version 1.3
  of this license or (at your option) any later version.}}
\hypersetup{pdflicenseurl={http://www.latex-project.org/lppl.txt}}
\hypersetup{pdfcontactaddress={ETH Zurich, ITP, HIT K,
  Wolfgang-Pauli-Strasse 27}}
\hypersetup{pdfcontactpostcode={8093}}
\hypersetup{pdfcontactcity={Zurich}}
\hypersetup{pdfcontactcountry={Switzerland}}
\hypersetup{pdfcontactemail={nbeisert@itp.phys.ethz.ch}}
\hypersetup{pdfcontacturl={http://people.phys.ethz.ch/\xmptilde nbeisert/}}

\newcommand{\secref}[1]{\hyperref[#1]{section \ref*{#1}}}

\parskip1ex
\parindent0pt
\let\olditemize\itemize
\def\itemize{\olditemize\parskip0pt}

\begin{document}

\title{The \textsf{childdoc} Package}
\hypersetup{pdftitle={The childdoc Package}}
\author{Niklas Beisert\\[2ex]
  Institut f\"ur Theoretische Physik\\
  Eidgen\"ossische Technische Hochschule Z\"urich\\
  Wolfgang-Pauli-Strasse 27, 8093 Z\"urich, Switzerland\\[1ex]
  \href{mailto:nbeisert@itp.phys.ethz.ch}
  {\texttt{nbeisert@itp.phys.ethz.ch}}}
\hypersetup{pdfauthor={Niklas Beisert}}
\hypersetup{pdfsubject={Manual for the LaTeX2e Package childdoc}}
\date{30 December 2018, \textsf{v2.0}}
\maketitle

\begin{abstract}\noindent
\textsf{childdoc} is a \LaTeXe{} package
that enables the direct compilation
of document sections included by |\include|
to individual files.
\end{abstract}

\begingroup
\parskip0ex
\tableofcontents
\endgroup

%%%%%%%%%%%%%%%%%%%%%%%%%%%%%%%%%%%%%%%%%%%%%%%%%%%%%%%%%%%%%%%%%%%%%%%%%%%%%%%%
%%%%%%%%%%%%%%%%%%%%%%%%%%%%%%%%%%%%%%%%%%%%%%%%%%%%%%%%%%%%%%%%%%%%%%%%%%%%%%%%
\section{Introduction}

\LaTeX{} provides a mechanism to structure a large document (such as a book)
into a main file and several child files (containing the chapters)
using the |\include| command.
This mechanism is beneficial for documents
which span hundreds of pages in order to
make the source file(s) more manageable.
Moreover, compilation can be restricted to
selected child files by means of the |\includeonly| command.
The latter feature can be used to reduce the compilation time while editing
(this was significantly more useful in the earlier days of \LaTeX{})
or to generate a smaller document which is easier to navigate.
Another application of |\includeonly| is to generate
documents consisting of selected parts of the complete document.

However, there are a few drawbacks of the plain |\include| mechanism:
\begin{itemize}
\item
The child files cannot be compiled on their own,
they can only be compiled via the main file.
A naive editing environment
(such as a text editor with an option
to have the current file processed by \LaTeX)
may require one to switch to the main file before compiling;
attempting to compile the child file produces errors.
\item
The main file must be modified (each time)
to adjust the |\includeonly| command
to the present needs. This easily leaves the main file in a messy state.
\item
The generated document will always carry the filename
of the main document. This is inconvenient if
several child files are to be compiled and
to be kept for distribution.
\end{itemize}

The present package provides a simple interface
to make child files individually compilable by \LaTeX{}.
Compiling a child file then has the same effect as compiling
the main file with an |\includeonly| command
to select the appropriate child.
Moreover the generated document will carry the name of the child
rather than the main file.
This resolves all three above issues.

This feature is meant to make the editing of books,
thesis documents and lecture notes somewhat more convenient.
However, the package can also be used efficiently for
composing a series of documents (such as exercise sheets)
which are typically distributed individually.
It then assists the author in generating the individual documents
(potentially in different versions)
as well as a document containing the collected series.
Another application is in developing style files
or other kinds of included material
where compilation of the style file could redirect
to a sample or test file.

%%%%%%%%%%%%%%%%%%%%%%%%%%%%%%%%%%%%%%%%%%%%%%%%%%%%%%%%%%%%%%%%%%%%%%%%%%%%%%%%
%%%%%%%%%%%%%%%%%%%%%%%%%%%%%%%%%%%%%%%%%%%%%%%%%%%%%%%%%%%%%%%%%%%%%%%%%%%%%%%%
\section{Usage}

First of all, the package \textsf{childdoc} is \emph{not} a standard
\LaTeXe{} |.sty| style file! Therefore it needs to be invoked in
a non-standard way.

%%%%%%%%%%%%%%%%%%%%%%%%%%%%%%%%%%%%%%%%%%%%%%%%%%%%%%%%%%%%%%%%%%%%%%%%%%%%%%%%
\subsection{Included Files}
\label{sec:include}

%%%%%%%%%%%%%%%%%%%%%%%%%%%%%%%%%%%%%%%%
\DescribeMacro{\childdocmain}
To use the package, add the commands
\begin{center}
\begin{tabular}{l}
|\input{childdoc.def}|\\
|\childdocmain{}|\\
\end{tabular}
\end{center}
at the very top of the main \LaTeX{} file,
in particular \emph{before} the |\documentclass| statement!
The argument of |\childdocmain| should be left empty
(but it must be present).

%%%%%%%%%%%%%%%%%%%%%%%%%%%%%%%%%%%%%%%%
\DescribeMacro{\childdocof}
Furthermore, add the commands
\begin{center}
\begin{tabular}{l}
|\input{childdoc.def}|\\
|\childdocof{|\textit{main}|}|\\
\end{tabular}
\end{center}
at the top of every child file \textit{child}
which is included by |\include{|\textit{child}|}|
from within the main file
(or at least for those files to be compiled individually).
The argument \textit{main} must be the filename of the main file.

There are a couple of
considerations in setting up the main and child documents:

%%%%%%%%%%%%%%%%%%%%%%%%%%%%%%%%%%%%%%%%
\paragraph{Restrictions.}

Please note the following restrictions:
\begin{itemize}
\item
|\childdocmain| must be called with one argument \textit{main}
to ensure compatibility with earlier version of the package.
It must either be empty (|\childdocmain{}|)
or precisely match the filename of the main file in which it is specified.
See \secref{sec:detection} for further information.
\item
The filename \textit{main} must be specified without the |.tex| extension.
\item
The filename \textit{main} is case sensitive
(even in case-insensitive file systems)
due to internal string comparison.
\item
The argument \textit{main} should be fully expanded, it cannot be a macro.
\item
Subdirectories and special characters should be avoided in filenames.
\item
The command |\childdocmain{|\textit{main}|}| must be followed by a whitespace.
It should not be followed immediately by another command
or by a comment mark `|%|'.
This is because the \TeX{} parser reads the token immediately following
the argument of |\childdocmain| and puts it
at the beginning of every child section;
however, a white\-space is ignored.
\end{itemize}

%%%%%%%%%%%%%%%%%%%%%%%%%%%%%%%%%%%%%%%%
\paragraph{Content of Main File.}

It is advisable to place all content in the child files included by |\include|.
Any output contained in the main file will appear in all child documents
unless suppressed manually;
it cannot be suppressed automatically by the |\includeonly| directive
and thus should normally be avoided.
A method to include some content in the main file
by means of conditional processing is described in \secref{sec:conditional}.

%%%%%%%%%%%%%%%%%%%%%%%%%%%%%%%%%%%%%%%%
\paragraph{Page Numbering.}

When only a part of the document is compiled,
the appropriate numbering of pages
(as well as other status parameters)
is determined from the |.aux| files.
The latter contain information from previous passes.
However this information needs to propagate through
all intermediate child documents.
Therefore the page numbering in child documents may well
be inconsistent until the complete document is compiled at least once.

A useful (if unconventional) way to always ensure a consistent
page numbering is to restart the numbering in each child document
and denote the pages by `\textit{child}|.|\textit{page}'
where \textit{child} represents the chapter/section number of the child file.
This can be achieved by the command
|\numberwithin{page}{|\textit{child}|}|
of the \textsf{amsmath} package
where \textit{child} can be |chapter| or |section|
depending on the chosen structuring.
Alternatively, one can modify the macro |\thepage| appropriately
and reset the counter |page| at the start of each child file.

%%%%%%%%%%%%%%%%%%%%%%%%%%%%%%%%%%%%%%%%%%%%%%%%%%%%%%%%%%%%%%%%%%%%%%%%%%%%%%%%
\subsection{Conditional Processing}
\label{sec:conditional}

The package provides a mechanism to compile different versions
of a document. To customise the versions further some conditional processing
can come in handy to distinguish which version is being compiled.
The package provides two macros to describe the compilation context:

%%%%%%%%%%%%%%%%%%%%%%%%%%%%%%%%%%%%%%%%
\DescribeMacro{\ifchilddoc}
The conditional |\ifchilddoc| distinguishes between the compilation of
child documents and the main document:
%
\begin{center}
|\ifchilddoc |\textit{child-code}| |[|\||else |\textit{main-code}]| \||fi|
\end{center}

%%%%%%%%%%%%%%%%%%%%%%%%%%%%%%%%%%%%%%%%
\DescribeMacro{\childdocname}
\DescribeMacro{\childdocjob}
The macro |\childdocname| contains the filename (without extension)
of the main or child file being processed.
Note that |\childdocjob| will always contain the name of the main file.

%%%%%%%%%%%%%%%%%%%%%%%%%%%%%%%%%%%%%%%%
\paragraph{Title Page.}

Conditional processing can be used to include a title or banner page
in the main document when proper precautions are taken.
Importantly, the code in the main file should ensure that the page counter
(as well as other status parameters which are stored in the |.aux| files)
takes the same value after the conditional processing.
Otherwise the page numbers may take divergent values
depending on which part is compiled.

For example, a title page could be declared by:
%
\begin{center}
\begin{tabular}{l}
|\ifchilddoc\||else|\\
|\addtocounter{page}{-1}|\\
\textit{code for title page}\\
|\newpage|\\
|\||fi|
\end{tabular}
\end{center}
%
A banner page for the child documents can be generated by:
%
\begin{center}
\begin{tabular}{l}
|\ifchilddoc|\\
|\addtocounter{page}{-1}|\\
\textit{code for banner page}\\
|\newpage|\\
|\||fi|
\end{tabular}
\end{center}
%
Here one could write a message such as:
\begin{center}
|This is the part \childdocname{} of \childdocjob{}.|
\end{center}

%%%%%%%%%%%%%%%%%%%%%%%%%%%%%%%%%%%%%%%%%%%%%%%%%%%%%%%%%%%%%%%%%%%%%%%%%%%%%%%%
\subsection{Flags}
\label{sec:flags}

The package makes it easy to generate different versions
of the main or child documents.
To this end compilation flags can be defined
and assigned different default values.
They will be particularly useful in conjunction
with the forwarding mechanism described in \secref{sec:forward}.

For example, it may be useful to have a flag |\version|
which can be set to |draft| or |final|.
The document source will contain some conditional code
depending on the value of |\version|.
Suppose further, the flag should default to |final| for the main file
and to |draft| for child files
which is a natural assignment for editing the document.
This is achieved by placing the following code
in the preamble of the main document
(below the |\childdocmain| directive):
%
\begin{center}
\begin{tabular}{l}
|\ifchilddoc|\\
|\providecommand{\version}{draft}|\\
|\||else|\\
|\providecommand{\version}{final}|\\
|\||fi|
\end{tabular}
\end{center}
%
The definition by |\providecommand| makes sure
that previous definitions are not overwritten.
Further statements |\providecommand{\version}{...}|
can thus be added before the above code to override it.

For the main file, one might add a line
(between |\childdocmain| and the above block)
%
\begin{center}
|%\ifchilddoc\||else\providecommand{\version}{draft}\||fi|
\end{center}
%
which can be uncommented to produce a draft version.
Likewise one can add a line to the very top of a child file
(above the |\childdocof{|\textit{main}|}| directive)
%
\begin{center}
|%\providecommand{\version}{final}|
\end{center}
%
which can be uncommented to produce the final version of this child document.

%%%%%%%%%%%%%%%%%%%%%%%%%%%%%%%%%%%%%%%%%%%%%%%%%%%%%%%%%%%%%%%%%%%%%%%%%%%%%%%%
\subsection{Forwarding}
\label{sec:forward}

Different versions of the main or child documents
using compilation flags as described in \secref{sec:flags}
can be (permanently) stored in different files
for convenient compilation, viewing and distribution.
To this end, the package defines a command
to pass on compilation to a different file:

%%%%%%%%%%%%%%%%%%%%%%%%%%%%%%%%%%%%%%%%
\DescribeMacro{\childdocforward}
The command |\childdocforward| redirects processing to
another source file:
%
\begin{center}
\begin{tabular}{l}
|\input{childdoc.def}|\\
|\childdocforward[|\textit{main}|]{|\textit{dest}|}|\\
\end{tabular}
\end{center}
%
The argument \textit{dest} is the destination file
(without extension).
It should be the main file or one of the child files.
Note that further \textsf{childdoc} directives
such as |\childdocof| and |\childdocforward|
in the indicated file will be processed in this form.
The optional argument \textit{main}
passes on directly to the main file \textit{main}
while pretending to compile the child \textit{dest}.
This form behaves as if \textit{dest}
issues |\childdocof{|\textit{main}|}| right away,
and no further \textsf{childdoc} directives will be processed.

%%%%%%%%%%%%%%%%%%%%%%%%%%%%%%%%%%%%%%%%
\DescribeMacro{\...prefix}
In the alternative form |\childdocforwardprefix|,
%
\begin{center}
\begin{tabular}{l}
|\input{childdoc.def}|\\
|\childdocforwardprefix[|\textit{main}|]{|\textit{prefix}|}{|\textit{dest}|}|
\end{tabular}
\end{center}
%
the destination file is determined by a pattern
depending on the current file:
To make this work, the current file must be called
`{\textit{prefix}\hspace{0.2em}\textit{suffix}}'
with \textit{prefix} matching precisely the argument.
Processing is then passed on to the file
`{\textit{dest}\hspace{0.2em}\textit{suffix}}'.
Surely, the same effect is achieved by
directly specifying the
argument `{\textit{dest}\hspace{0.2em}\textit{suffix}}'
in the first form.
However, that requires to set up a different file
for each child. With the alternative form of the command
all these files can have exactly the same content
which simplifies setting them up and maintaining them.

For example, the following file |draft.tex|
with a compilation flag |\version| as described in \secref{sec:flags}
compiles the main document as a draft:
%
\begin{center}
\begin{tabular}{l}
|\def\version{draft}|\\
|\input{childdoc.def}|\\
|\childdocforward{|\textit{main}|}|
\end{tabular}
\end{center}
%
Likewise, the following files |final|\textit{nn}|.tex|
compile the final version of the child document
|child|\textit{nn}|.tex|:
%
\begin{center}
\begin{tabular}{l}
|\def\version{final}|\\
|\input{childdoc.def}|\\
|\childdocforwardprefix{final}{child}|
\end{tabular}
\end{center}
%

Note that when several versions of a main file and/or of each child file
are to be generated, it may be convenient to set up a |Makefile| or
shell script to automatise the process.

%%%%%%%%%%%%%%%%%%%%%%%%%%%%%%%%%%%%%%%%%%%%%%%%%%%%%%%%%%%%%%%%%%%%%%%%%%%%%%%%
\subsection{Command Line Processing}
\label{sec:commandline}

The effect of redirection files can also be achieved by invoking
the \LaTeX{} compiler with a more elaborate command line.
Most conveniently this should be done as part
of a shell script or a |Makefile|.

When using \textsf{childdoc} in the main file, the following
command lines effectively perform a redirection
(note that depending on the shell being used,
backslashes may have to be doubled: `|\|' $\to$ `|\\|'):
%
\begin{center}
|... -jobname "|\textit{target}|" |\\|"|[\textit{flags}]%
|\input{childdoc.def}\childdocforward[|\textit{main}|]{|\textit{dest}|}"|
\end{center}
%
Here \textit{target} is the name of the output file,
\textit{main} is the name of the main file
and \textit{dest} is the name of the main or child file to be processed
(all filenames without extensions).
The optional argument \textit{main} can be omitted
if \textit{main} matches \textit{dest}.
Optionally, compilation \textit{flags} can be defined via |\def| commands.
This command line makes the \TeX{} engine believe
it is compiling the file \textit{target}
whose content is specified as the latter parameter.
The provided code then forwards the processing to
\textit{main} or \textit{dest} as described in \secref{sec:forward}.

%%%%%%%%%%%%%%%%%%%%%%%%%%%%%%%%%%%%%%%%%%%%%%%%%%%%%%%%%%%%%%%%%%%%%%%%%%%%%%%%
\subsection{Include by Input}
\label{sec:input}

Including child documents by |\include| has some restrictions by design.
Most notably, the content of a child document always occupies
its own set of pages; pages cannot be shared between child documents.
Usually, this behaviour makes perfect sense
because each child document contain an essential part of the document.
However, in some situations it may be desirable to compose
a document from a collection of parts
without having mandatory page breaks between then.
For this case, the package
provides a mechanism to include parts
by |\input| which can also be processed individually.
However, by construction this mechanism
requires manual handling of the content to be output.

%%%%%%%%%%%%%%%%%%%%%%%%%%%%%%%%%%%%%%%%
\DescribeMacro{\ifchilddocmanual}
The main file should be prepared as usual, see \secref{sec:include}.
However, the document body must make a distinction
between processing of an individual part and of the main document, e.g.:
%
\begin{center}
\begin{tabular}{l}
|\ifchilddocmanual|\\
|\input{\childdocname}|\\
|\||else|\\
\textit{document body with }|\input{|\textit{part}|}|\\
|\||fi|
\end{tabular}
\end{center}
%
The conditional |\ifchilddocmanual| is true whenever
a part to be included by |\input| is being compiled,
and the name of the part is stored in |\childdocname|.

%%%%%%%%%%%%%%%%%%%%%%%%%%%%%%%%%%%%%%%%
\DescribeMacro{\childdocby}
Each part to be included by |\input| should start with:
%
\begin{center}
\begin{tabular}{l}
|\input{childdoc.def}|\\
|\childdocby{|\textit{main}|}|\\
\end{tabular}
\end{center}
%
The directive |\childdocby| is similar to |\childdocof|
described in \secref{sec:include},
but the subsequent selection of content must be done manually.
To that end, both |\ifchilddoc| and |\ifchilddocmanual|
will be true upon processing of a part,
and the name of the part is stored in |\childdocname|.
Note that |\jobname| will be set to the filename of the current part
so that each part receives an individual |.aux| file
that does not interfere with the |.aux| file(s) of the main document.
This behaviour can be altered by the alternative form
|\childdocby[*]{|\textit{main}|}| (with a non-empty optional argument)
which uses the |.aux| file of the main document
by setting |\jobname| to \textit{main}.

%%%%%%%%%%%%%%%%%%%%%%%%%%%%%%%%%%%%%%%%%%%%%%%%%%%%%%%%%%%%%%%%%%%%%%%%%%%%%%%%
\subsection{Driver Development}
\label{sec:driver}

The \textsf{childdoc} mechanism can also be use for the development
of definition files such as \LaTeX{} styles or classes.
This case differs from the above setup with multiple parts
included by |\include| in that no |\includeonly| should be invoked.
This can be achieved by starting the include file
(before |\ProvidesPackage|) with:
%
\begin{center}
\begin{tabular}{l}
|\input{childdoc.def}|\\
|\childdocforward{|\textit{main}|}|\\
\end{tabular}
\end{center}
%
or alternatively with:
%
\begin{center}
\begin{tabular}{l}
|\input{childdoc.def}|\\
|\childdocby{|\textit{main}|}|\\
\end{tabular}
\end{center}
%
Both forms have slightly different effects as described above.
The main file is prepared as usual, see \secref{sec:include}.

%%%%%%%%%%%%%%%%%%%%%%%%%%%%%%%%%%%%%%%%%%%%%%%%%%%%%%%%%%%%%%%%%%%%%%%%%%%%%%%%
\subsection{Legacy Detection}
\label{sec:detection}

The directive |\childdocmain| in the main file can detect
whether the complete document or merely a child is to be compiled
even without using the directive |\childdocof|.
This method is deprecated because it is less robust
and there is no compelling reason to use it;
it is merely provided for backward compatibility
and it may be removed in future versions.

If the detection mechanism is to be used,
it is mandatory to correctly specify
the filename of the main file as the argument of |\childdocmain|:
%
\begin{center}
\begin{tabular}{l}
|\input{childdoc.def}|\\
|\childdocmain{|\textit{main}|}|\\
\end{tabular}
\end{center}
%
If |\jobname| does not match the argument \textit{main} of |\childdocmain|,
it is assumed that |\jobname| points to the child file to be compiled.
When using |\childdocmain| with the main file specified as argument,
it suffices to start a child file
with just |\input{|\textit{main}|}|
without loading of the package and using |\childdocof|.
If instead all processing is done
with the appropriate \textsf{childdoc} directives,
the argument of \textit{main} of |\childdocmain| can be empty.

An alternative version of the command line processing described
in \secref{sec:commandline} using the detection mechanism reads:
%
\begin{center}
|... -jobname "|\textit{target}|" "|[\textit{flags}]%
[|\def\jobname{|\textit{dest}|}|]|\input{|\textit{main}|}"|
\end{center}

%%%%%%%%%%%%%%%%%%%%%%%%%%%%%%%%%%%%%%%%%%%%%%%%%%%%%%%%%%%%%%%%%%%%%%%%%%%%%%%%
\subsection{Manual Code}
\label{sec:manual}

In case one cannot be certain whether the definitions file |childdoc.def|
is installed on the target \TeX{} distribution
and one prefers not to ship it,
it is conceivable to paste a few relevant commands into the sources.

To that end, drop all statements |\input{childdoc.def}|
and perform the replacements as outlined below.
Instead of |\childdocmain{|\textit{main}|}| add the following code
to the top of the main file:
%
\begin{center}
\begin{tabular}{l}
|\||ifdefined\childdocname\endinput\||fi\newif\ifchilddoc|\\
|\edef\childdocname{\scantokens\expandafter{\jobname\noexpand}}|\\
|\def\childdocmain{|\textit{main}|}\||ifx\childdocmain\childdocname\||else|\\
|\childdoctrue\includeonly{\childdocname}\let\jobname\childdocmain\||fi|\\
\end{tabular}
\end{center}
%
Instead of |\childdocof{|\textit{main}|}| just include the main file
at the top of each child file:
%
\begin{center}
|\input{|\textit{main}|}|
\end{center}
%
A simple redirection |\childdocforward{|\textit{dest}|}| is achieved by:
%
\begin{center}
|\def\jobname{|\textit{dest}|}\input{\jobname}|
\end{center}
%
The redirection with prefix
|\childdocforwardprefix[|\textit{prefix}|]{|\textit{dest}|}|
is accomplished by:
%
\begin{center}
\begin{tabular}{l}
|{\edef\jobname{\scantokens\expandafter{\jobname\noexpand}}|\\
|\def\redirectjob |\textit{prefix}|#1~~~{\gdef\jobname{|\textit{dest}|#1}}|\\
|\expandafter\redirectjob\jobname~~~}\input{\jobname}|
\end{tabular}
\end{center}

In an alternative approach,
child documents can be compiled by a specific command line
without additional code or specific definitions:
%
\begin{center}
|... -jobname "|\textit{target}|" "|[\textit{flags}]%
|\includeonly{|\textit{dest}|}\input{|\textit{main}|}"|
\end{center}
%

%%%%%%%%%%%%%%%%%%%%%%%%%%%%%%%%%%%%%%%%%%%%%%%%%%%%%%%%%%%%%%%%%%%%%%%%%%%%%%%%
%%%%%%%%%%%%%%%%%%%%%%%%%%%%%%%%%%%%%%%%%%%%%%%%%%%%%%%%%%%%%%%%%%%%%%%%%%%%%%%%
\section{Information}

%%%%%%%%%%%%%%%%%%%%%%%%%%%%%%%%%%%%%%%%%%%%%%%%%%%%%%%%%%%%%%%%%%%%%%%%%%%%%%%%
\subsection{Copyright}

Copyright \copyright{} 2017--2018 Niklas Beisert

This work may be distributed and/or modified under the
conditions of the \LaTeX{} Project Public License, either version 1.3
of this license or (at your option) any later version.
The latest version of this license is in
  \url{http://www.latex-project.org/lppl.txt}
and version 1.3 or later is part of all distributions of \LaTeX{}
version 2005/12/01 or later.

This work has the LPPL maintenance status `maintained'.

The Current Maintainer of this work is Niklas Beisert.

This work consists of the files |README.txt|, |childdoc.ins| and |childdoc.dtx|
as well as the derived files |childdoc.def|, |cdocsamp.tex|
with |cdocsch1.tex|, |cdocsch2.tex|, |cdocspt3.tex|, |cdocspt4.tex|,
|cdocsdrf.tex|, |cdocsfn1.tex|, |cdocsfn2.tex|
as well as |childdoc.pdf|.

%%%%%%%%%%%%%%%%%%%%%%%%%%%%%%%%%%%%%%%%%%%%%%%%%%%%%%%%%%%%%%%%%%%%%%%%%%%%%%%%
\subsection{Files and Installation}

The package consists of the files:
%
\begin{center}
\begin{tabular}{ll}
    |README.txt|   & readme file \\
    |childdoc.ins| & installation file \\
    |childdoc.dtx| & source file \\
    |childdoc.def| & definition file \\
    |cdocsamp.tex| & sample main file \\
    |cdocsch1.tex| & sample include file \\
    |cdocsch2.tex| & sample include file \\
    |cdocspt3.tex| & sample part file \\
    |cdocspt4.tex| & sample part file \\
    |cdocsdrf.tex| & sample redirection file \\
    |cdocsfn1.tex| & sample redirection file \\
    |cdocsfn2.tex| & sample redirection file \\
    |childdoc.pdf| & manual
\end{tabular}
\end{center}
%
The distribution consists of the files
|README.txt|, |childdoc.ins| and |childdoc.dtx|.
%
\begin{itemize}
\item
Run (pdf)\LaTeX{} on |childdoc.dtx|
to compile the manual |childdoc.pdf| (this file).
\item
Run \LaTeX{} on |childdoc.ins| to create the definitions file |childdoc.def|
and the sample |cdocsamp.tex| with include files
|cdocsch1.tex|, |cdocsch2.tex|, |cdocspt3.tex|, |cdocspt4.tex|,
|cdocsdrf.tex|, |cdocsfn1.tex|, |cdocsfn2.tex|.
Then copy the file |childdoc.def| to an appropriate directory of your \LaTeX{}
distribution, e.g.\ \textit{texmf-root}|/tex/latex/childdoc|.
\end{itemize}

%%%%%%%%%%%%%%%%%%%%%%%%%%%%%%%%%%%%%%%%%%%%%%%%%%%%%%%%%%%%%%%%%%%%%%%%%%%%%%%%
\subsection{Related CTAN Packages}

There are several other packages which offer a similar functionality:
%
\begin{itemize}
\item
The packages
\href{http://ctan.org/pkg/docmute}{\textsf{docmute}},
\href{http://ctan.org/pkg/includex}{\textsf{includex}} and
\href{http://ctan.org/pkg/standalone}{\textsf{standalone}}
provide commands to include only the document body of
a child file thus allowing both files to be compiled individually.
\item
The packages \href{http://ctan.org/pkg/subdocs}{\textsf{subdocs}}
and \href{http://ctan.org/pkg/subfiles}{\textsf{subfiles}}
provide structures in which the main and child documents can be
encapsulated and allowing them to be compiled individually.
The inclusion mechanism is different from the conventional |\include|.
\item
The package \href{http://ctan.org/pkg/combine}{\textsf{combine}}
is an elaborate solution to combine several documents into one.
\end{itemize}
%
See also the CTAN topic \href{http://ctan.org/topic/subdocs}{\textsf{subdocs}}
for further related packages.
The present package differs from the above solutions in that
a document structure constructed with the conventional |\include| mechanism
just needs two extra commands at the top of every file
such that all constituent files can be compiled individually.

%%%%%%%%%%%%%%%%%%%%%%%%%%%%%%%%%%%%%%%%%%%%%%%%%%%%%%%%%%%%%%%%%%%%%%%%%%%%%%%%
%\subsection{Feature Suggestions}
%
%The following is a list of features which may be useful for future
%versions of this package:
%%
%\begin{itemize}
%\item
%\ldots
%\end{itemize}

%%%%%%%%%%%%%%%%%%%%%%%%%%%%%%%%%%%%%%%%%%%%%%%%%%%%%%%%%%%%%%%%%%%%%%%%%%%%%%%%
\subsection{Revision History}

%%%%%%%%%%%%%%%%%%%%%%%%%%%%%%%%%%%%%%%%
\paragraph{v2.0:} 2018/12/30

\begin{itemize}
\item
immediate forward processing
\item
added |\childdocby| mechanism
\item
manual restructured
\end{itemize}

%%%%%%%%%%%%%%%%%%%%%%%%%%%%%%%%%%%%%%%%
\paragraph{v1.6:} 2018/01/17

\begin{itemize}
\item
application for development of include files
\item
corrections to manual
\end{itemize}

%%%%%%%%%%%%%%%%%%%%%%%%%%%%%%%%%%%%%%%%
\paragraph{v1.5:} 2017/05/21

\begin{itemize}
\item
more complete structuring introduced
\item
|\childdocof| introduced
\item
|\childdoc| renamed to |\childdocmain|
\item
|\childredirect| renamed to |\childdocforward| and |\childdocforwardprefix|
and functionality expanded
\end{itemize}

%%%%%%%%%%%%%%%%%%%%%%%%%%%%%%%%%%%%%%%%
\paragraph{v1.0:} 2017/04/27

\begin{itemize}
\item
manual and install package
\item
first version published on CTAN
\end{itemize}

%%%%%%%%%%%%%%%%%%%%%%%%%%%%%%%%%%%%%%%%
\paragraph{v0.6:} 2017/04/26

\begin{itemize}
\item
redirection mechanism added
\end{itemize}

%%%%%%%%%%%%%%%%%%%%%%%%%%%%%%%%%%%%%%%%
\paragraph{v0.5:} 2017/04/26

\begin{itemize}
\item
functionality in definition file
\end{itemize}


%%%%%%%%%%%%%%%%%%%%%%%%%%%%%%%%%%%%%%%%%%%%%%%%%%%%%%%%%%%%%%%%%%%%%%%%%%%%%%%%
%%%%%%%%%%%%%%%%%%%%%%%%%%%%%%%%%%%%%%%%%%%%%%%%%%%%%%%%%%%%%%%%%%%%%%%%%%%%%%%%
%%%%%%%%%%%%%%%%%%%%%%%%%%%%%%%%%%%%%%%%%%%%%%%%%%%%%%%%%%%%%%%%%%%%%%%%%%%%%%%%
\appendix

\settowidth\MacroIndent{\rmfamily\scriptsize 000\ }

 \DocInput{childdoc.dtx}

\end{document}
%</driver>
% \fi
%
% %%%%%%%%%%%%%%%%%%%%%%%%%%%%%%%%%%%%%%%%%%%%%%%%%%%%%%%%%%%%%%%%%%%%%%%%%%%%%%
% %%%%%%%%%%%%%%%%%%%%%%%%%%%%%%%%%%%%%%%%%%%%%%%%%%%%%%%%%%%%%%%%%%%%%%%%%%%%%%
% \section{Sample}
%\iffalse
%<*samplemain>
%\fi
%
% The following presents a sample document
% with two chapters, two parts, a title page,
% a compile flag as well as three forwarding files to set the flag.
% It consists of eight |.tex| files:
% \begin{center}
% \begin{tabular}{ll}
% |cdocsamp.tex|&main file\\
% |cdocsch1.tex|&include file for chapter 1\\
% |cdocsch2.tex|&include file for chapter 2\\
% |cdocspt3.tex|&include file for part 3\\
% |cdocspt4.tex|&include file for part 4\\
% |cdocsdrf.tex|&forwarding file for main file in draft mode\\
% |cdocsfi1.tex|&forwarding file for final version of chapter 1\\
% |cdocsfi2.tex|&forwarding file for final version of chapter 2\\
% \end{tabular}
% \end{center}
% Each of the eight files can be compiled directly by the \LaTeX{} compiler.
%
% %%%%%%%%%%%%%%%%%%%%%%%%%%%%%%%%%%%%%%
% \paragraph{Main File.}
%
% The main file is called |cdocsamp.tex|.
%
% Load the \textsf{childdoc} definitions and
% declare the filename for the main document:
%    \begin{macrocode}
\input{childdoc.def}
\childdocmain{}
%    \end{macrocode}

% Optional override for |\version| flag:
%    \begin{macrocode}
%%\ifchilddoc\else\providecommand{\version}{draft}\fi
%    \end{macrocode}

% Define the default values for the |\version| flag
% (|final| for the main file and |draft| for childs):
%    \begin{macrocode}
\ifchilddoc
\providecommand{\version}{draft}
\else
\providecommand{\version}{final}
\fi
%    \end{macrocode}

% Load the standard document class:
%    \begin{macrocode}
\documentclass[12pt]{article}
%    \end{macrocode}

% Start the document body:
%    \begin{macrocode}
\begin{document}
%    \end{macrocode}

% Declare a title page.
% Print title, part of document being processed and version flag:
%    \begin{macrocode}
\addtocounter{page}{-1}
\begin{center}
{\LARGE\bfseries{}childdoc example\par}
\vspace{1cm}
\ifchilddoc
\ifchilddocmanual part\else chapter\fi:
`\childdocname' of `\childdocjob'\par
\else
main document: `\childdocjob'\par
\fi
version: \version\par
\end{center}
\newpage
%    \end{macrocode}

% Manually include selected file,
% otherwise process as usual:
%    \begin{macrocode}
\ifchilddocmanual
\section*{part `\childdocname'}
\input{\childdocname}
\else
%    \end{macrocode}

% Include the two chapters:
%    \begin{macrocode}
\include{cdocsch1}
\include{cdocsch2}
%    \end{macrocode}

% Include the two parts unless only chapters should be displayed:
%    \begin{macrocode}
\ifchilddoc\else
\section{part three}
\input{cdocspt3}
\section{part four}
\input{cdocspt4}
\fi
%    \end{macrocode}

% Process as usual until here:
%    \begin{macrocode}
\fi
%    \end{macrocode}

% End of document body:
%    \begin{macrocode}
\end{document}
%    \end{macrocode}
%\iffalse
%</samplemain>
%\fi
%
% %%%%%%%%%%%%%%%%%%%%%%%%%%%%%%%%%%%%%%
% \paragraph{Chapter Include Files.}
%
% The include files are called |cdocsch1.tex| and |cdocsch2.tex|.
%
%\iffalse
%<*samplechap1|samplechap2>
%\fi

% Optional override for |\version| flag:
%    \begin{macrocode}
%%\providecommand{\version}{final}
%    \end{macrocode}

% Include the main document:
%    \begin{macrocode}
\input{childdoc.def}
\childdocof{cdocsamp}
%    \end{macrocode}

%\iffalse
%</samplechap1|samplechap2>
%\fi
%
%\iffalse
%<*samplechap1>
%\fi
% Some text for chapter 1:
%    \begin{macrocode}
\section{one}
some text in chapter one
%    \end{macrocode}

%\iffalse
%</samplechap1>
%\fi
% Some text for chapter 2:
%\iffalse
%<*samplechap2>
%\fi
%    \begin{macrocode}
\section{two}
more text in chapter two
%    \end{macrocode}

%\iffalse
%</samplechap2>
%\fi
%
% %%%%%%%%%%%%%%%%%%%%%%%%%%%%%%%%%%%%%%
% \paragraph{Part Include Files.}
%
% The include files are called |cdocspt3.tex| and |cdocspt4.tex|.
%
%\iffalse
%<*samplepart3|samplepart4>
%\fi

% Optional override for |\version| flag:
%    \begin{macrocode}
%%\providecommand{\version}{final}
%    \end{macrocode}

% Include the main document:
%    \begin{macrocode}
\input{childdoc.def}
\childdocby{cdocsamp}
%    \end{macrocode}

%\iffalse
%</samplepart3|samplepart4>
%\fi
%
%\iffalse
%<*samplepart3>
%\fi
% Some text for part 3:
%    \begin{macrocode}
some text in part three
%    \end{macrocode}

%\iffalse
%</samplepart3>
%\fi
% Some text for part 4:
%\iffalse
%<*samplepart4>
%\fi
%    \begin{macrocode}
more text in part four
%    \end{macrocode}

%\iffalse
%</samplepart4>
%\fi
%
% %%%%%%%%%%%%%%%%%%%%%%%%%%%%%%%%%%%%%%
% \paragraph{Forwarding for a Complete Draft.}
%
% The following forwarding file |cdocsdrf.tex|
% compiles the main document in draft mode:
%\iffalse
%<*sampledraft>
%\fi
%    \begin{macrocode}
\def\version{draft}
\input{childdoc.def}
\childdocforward{cdocsamp}
%    \end{macrocode}

%\iffalse
%</sampledraft>
%\fi
%
% %%%%%%%%%%%%%%%%%%%%%%%%%%%%%%%%%%%%%%
% \paragraph{Forwarding for Final Version of the Chapters.}
%
% The following forwarding files |cdocsfn1.tex| and |cdocsfn2.tex|
% (with identical content)
% compile the final versions of the child documents
% |cdocsch1.tex| and |cdocsch2.tex|, respectively:
%\iffalse
%<*samplefinal>
%\fi
%    \begin{macrocode}
\def\version{final}
\input{childdoc.def}
\childdocforwardprefix[cdocsamp]{cdocsfn}{cdocsch}
%    \end{macrocode}

%\iffalse
%</samplefinal>
%\fi
%
% %%%%%%%%%%%%%%%%%%%%%%%%%%%%%%%%%%%%%%
% \paragraph{Command Line Processing.}
%
% The following three command lines generate the output files
% |cdocscld|, |cdocscl1| and |cdocscl2|
% which should be identical to
% |cdocsdrf|, |cdocsch1| and |cdocsfn2|, respectively:
% \begin{center}
% \begin{tabular}{l}
% |latex -jobname cdocscld \|\\
% |  "\def\version{draft}\input{childdoc.def}\childdocforward{cdocsamp}"|\\
% |latex -jobname cdocscl1 \|\\
% |  "\input{childdoc.def}\childdocforward[cdocsamp]{cdocsch1}"|\\
% |latex -jobname cdocscl2 \|\\
% |  "\def\version{final}\input{childdoc.def}\childdocforward{cdocsch2}"|
% \end{tabular}
% \end{center}
% Note that the trailing backslash on each first line
% merely continues the input to the second line
% (for convenient cut ant paste).
% Furthermore, the command |latex| can be replaced by any
% of its alternative versions such as |pdflatex|.
%
% %%%%%%%%%%%%%%%%%%%%%%%%%%%%%%%%%%%%%%%%%%%%%%%%%%%%%%%%%%%%%%%%%%%%%%%%%%%%%%
% %%%%%%%%%%%%%%%%%%%%%%%%%%%%%%%%%%%%%%%%%%%%%%%%%%%%%%%%%%%%%%%%%%%%%%%%%%%%%%
% \section{Implementation}
%\iffalse
%<*package>
%\fi
%
% This section describes the definitions file |childdoc.def|.

% The definitions cannot be loaded using |\usepackage| or |\RequirePackage|
% which has a mechanism to prevent loading a style file more than once.
% When loading the definitions by means of |\input|
% multiple instances have to be prevented manually:
%\iffalse
%This code needs to be before the `\ProvidesFile' directive
%which is defined at the beginning of this file.
%Therefore it is also placed there and commented out here.
%</package>
%<*discard>
%\fi
%    \begin{macrocode}
\ifdefined\childdocmain\endinput\fi
%    \end{macrocode}
%\iffalse
%</discard>
%<*package>
%\fi
%
% \macro{\ifchilddoc}
% \macro{\ifchilddocmanual}
% The conditional |\ifchilddoc| tells whether a
% child (true) or main (false) document is being compiled.
% The conditional |\ifchilddocmanual| tells whether
% the |\includeonly| mechanism is used (false) or
% the selection of child files must be performed manually (true).
% The definitions initialise to false:
%    \begin{macrocode}
\newif\ifchilddoc
\newif\ifchilddocmanual
%    \end{macrocode}

% \macro{\childdocname}
% \macro{\childdocjob}
% The macro |\childdocname| stores the name of the main document
% to be compiled. The macro |\childdocjob| stores the name of
% the document on which the \LaTeX{} compiler was originally invoked.
% The content of |\jobname| cannot be compared
% to filenames specified in the source due to different catcodes.
% The following code rescans |\jobname|, stores the result
% in |\childdocname| and saves a copy in |\childdocjob|:
%    \begin{macrocode}
\edef\childdocname{\scantokens\expandafter{\jobname\noexpand}}
\let\childdocjob\childdocname
%    \end{macrocode}

% \macro{\childdocdisable}
% The macro |\childdocdisable| prevents the main file
% from being processed more than once.
% At this stage, the main document command |\childdocmain|
% is assumed to be called once again where it should do nothing.
% Any subsequent call to it should prevent
% a secondary processing of the main document
% It overwrites the forwarding commands
% |\childdocof| and |\childdocforward|
% with empty macros to prevent further inclusions of the main document:
%    \begin{macrocode}
\newcommand{\childdocdisable}
{
  \renewcommand{\childdocmain}[1]{\renewcommand{\childdocmain}[1]{\endinput}}
  \renewcommand{\childdocof}[1]{}
  \renewcommand{\childdocby}[2][]{}
  \renewcommand{\childdocforward}[2][]{}
  \renewcommand{\childdocdisable}{}
}
%    \end{macrocode}

% \macro{\childdocmain}
% The macro |\childdocmain| is to be called at the top of the main file
% with nothing or the main filename (without extension) as argument.
% First, it breaks loops.
% If the argument is not empty and does not match |\childdocname|
% (which is set by the first inclusion of |childdoc.def|),
% |\ifchilddoc| is set to true, |\includeonly| is applied to the child file
% and |\jobname| is set to the main file
% (for proper handling of |.aux| files):
%    \begin{macrocode}
\newcommand{\childdocmain}[1]
{
  \childdocdisable\childdocmain{}
  \if?#1?\else
    \begingroup
      \def\childdoctmp{#1}
      \ifx\childdoctmp\childdocname
        \def\childdoctmp{}
      \else
        \def\childdoctmp
        {
          \childdoctrue
          \includeonly{\childdocname}
          \def\childdocjob{#1}
          \def\jobname{#1}
        }
      \fi
      \expandafter
    \endgroup
    \childdoctmp
  \fi
}
%    \end{macrocode}

% \macro{\childdocof}
% The command |\childdocof| redirects
% compilation to the main file |#1|.
%    \begin{macrocode}
\newcommand{\childdocof}[1]
{
  \childdocdisable
  \childdoctrue
  \includeonly{\childdocname}
  \def\jobname{#1}
  \def\childdocjob{#1}
  \input{#1}
}
%    \end{macrocode}

% \macro{\childdocby}
% The command |\childdocby| ....
%    \begin{macrocode}
\newcommand{\childdocby}[2][]
{
  \childdocdisable
  \childdoctrue
  \childdocmanualtrue
  \if?#1?\else
    \def\jobname{#2}
  \fi
  \def\childdocjob{#2}
  \input{#2}
  \endinput
}
%    \end{macrocode}

% \macro{\childdocforward}
% The command |\childdocforward| redirects
% compilation to the main file or
% (if the optional argument is given) a child file.
% Parameters are set as if the main file
% or a child file starting with |\childdocof| was compiled.
% Then compilation is handed over to the main file:
%    \begin{macrocode}
\newcommand{\childdocforward}[2][]
{
  \begingroup
    \if?#1?
      \def\childdoctmp
      {
        \def\childdocname{#2}
        \def\childdocjob{#2}
        \def\jobname{#2}
        \input{#2}
        \endinput
      }
    \else
      \def\childdoctmp
      {
        \childdocdisable
        \def\childdocname{#2}
        \childdoctrue
        \includeonly{#2}
        \def\childdocjob{#1}
        \def\jobname{#1}
        \input{#1}
        \endinput
      }
    \fi
    \expandafter
  \endgroup
  \childdoctmp
}
%    \end{macrocode}

% \macro{\childdocforwardprefix}
% The command |\childdocforwardprefix| redirects
% compilation to the main or a child file by means of a pattern.
% The prefix |#1| in the current filename is replaced by |#2|
% and the suffix of the current filename is kept
% (it is assumed that the filename does not contain the substring `|~~~|'
% which is used as a delimiter).
% Compilation is handed over to the new file by |\childdocforward|:
%    \begin{macrocode}
\newcommand{\childdocforwardprefix}[3][]
{
  \begingroup
    \def\childdocextract #2##1~~~{\def\childdoctmp{\childdocforward[#1]{#3##1}}}
    \expandafter\childdocextract\childdocname~~~
    \expandafter
  \endgroup
  \childdoctmp
}
%    \end{macrocode}

% \macro{\childdoc}
% The deprecated macro |\childdoc| is a legacy version of |\childdocmain|:
%    \begin{macrocode}
\newcommand{\childdoc}{\childdocmain}
%    \end{macrocode}

% \macro{\childdocredirect}
% The deprecated macro |\childdocredirect| is a legacy version
% of |\childdocforward| and |\childdocforwardprefix|:
%    \begin{macrocode}
\newcommand{\childdocredirect}[2][]
{
  \begingroup
    \if?#1?
      \def\childdoctmp{\childdocforward{#2}}
    \else
      \def\childdoctmp{\childdocforwardprefix{#1}{#2}}
    \fi
    \expandafter
  \endgroup
  \childdoctmp
}
%    \end{macrocode}

%\iffalse
%</package>
%\fi
%
\endinput
\childdocforward{cdocsamp}"|\\
% |latex -jobname cdocscl1 \|\\
% |  "% \iffalse
%
% childdoc.dtx Copyright (C) 2017-2018 Niklas Beisert
%
% This work may be distributed and/or modified under the
% conditions of the LaTeX Project Public License, either version 1.3
% of this license or (at your option) any later version.
% The latest version of this license is in
%   http://www.latex-project.org/lppl.txt
% and version 1.3 or later is part of all distributions of LaTeX
% version 2005/12/01 or later.
%
% This work has the LPPL maintenance status `maintained'.
%
% The Current Maintainer of this work is Niklas Beisert.
%
% This work consists of the files childdoc.dtx and childdoc.ins
% and the derived files childdoc.def and cdocsamp.tex with
% cdocsch1.tex, cdocsch2.tex, cdocsdrf.tex, cdocsfn1.tex, cdocsfn2.tex.
%
%<package>\ifdefined\childdocmain\endinput\fi
%<package>\ProvidesFile{childdoc.def}[2018/12/30 v2.0 child document driver]
%<samplemain>\ProvidesFile{cdocsamp.tex}[2018/12/30 v2.0 sample for childdoc]
%<*driver>
%\ProvidesFile{childdoc.drv}[2018/12/30 v2.0 childdoc reference manual file]
\PassOptionsToClass{10pt,a4paper}{article}
\documentclass{ltxdoc}

\usepackage[margin=35mm]{geometry}
\usepackage{hyperref}
\usepackage{hyperxmp}
\usepackage[usenames]{color}

\hypersetup{colorlinks=true}
\hypersetup{pdfstartview=FitH}
\hypersetup{pdfpagemode=UseNone}
\hypersetup{pdfsource={}}
\hypersetup{pdflang={en-UK}}
\hypersetup{pdfcopyright={Copyright 2017-2018 Niklas Beisert.
  This work may be distributed and/or modified under the
  conditions of the LaTeX Project Public License, either version 1.3
  of this license or (at your option) any later version.}}
\hypersetup{pdflicenseurl={http://www.latex-project.org/lppl.txt}}
\hypersetup{pdfcontactaddress={ETH Zurich, ITP, HIT K,
  Wolfgang-Pauli-Strasse 27}}
\hypersetup{pdfcontactpostcode={8093}}
\hypersetup{pdfcontactcity={Zurich}}
\hypersetup{pdfcontactcountry={Switzerland}}
\hypersetup{pdfcontactemail={nbeisert@itp.phys.ethz.ch}}
\hypersetup{pdfcontacturl={http://people.phys.ethz.ch/\xmptilde nbeisert/}}

\newcommand{\secref}[1]{\hyperref[#1]{section \ref*{#1}}}

\parskip1ex
\parindent0pt
\let\olditemize\itemize
\def\itemize{\olditemize\parskip0pt}

\begin{document}

\title{The \textsf{childdoc} Package}
\hypersetup{pdftitle={The childdoc Package}}
\author{Niklas Beisert\\[2ex]
  Institut f\"ur Theoretische Physik\\
  Eidgen\"ossische Technische Hochschule Z\"urich\\
  Wolfgang-Pauli-Strasse 27, 8093 Z\"urich, Switzerland\\[1ex]
  \href{mailto:nbeisert@itp.phys.ethz.ch}
  {\texttt{nbeisert@itp.phys.ethz.ch}}}
\hypersetup{pdfauthor={Niklas Beisert}}
\hypersetup{pdfsubject={Manual for the LaTeX2e Package childdoc}}
\date{30 December 2018, \textsf{v2.0}}
\maketitle

\begin{abstract}\noindent
\textsf{childdoc} is a \LaTeXe{} package
that enables the direct compilation
of document sections included by |\include|
to individual files.
\end{abstract}

\begingroup
\parskip0ex
\tableofcontents
\endgroup

%%%%%%%%%%%%%%%%%%%%%%%%%%%%%%%%%%%%%%%%%%%%%%%%%%%%%%%%%%%%%%%%%%%%%%%%%%%%%%%%
%%%%%%%%%%%%%%%%%%%%%%%%%%%%%%%%%%%%%%%%%%%%%%%%%%%%%%%%%%%%%%%%%%%%%%%%%%%%%%%%
\section{Introduction}

\LaTeX{} provides a mechanism to structure a large document (such as a book)
into a main file and several child files (containing the chapters)
using the |\include| command.
This mechanism is beneficial for documents
which span hundreds of pages in order to
make the source file(s) more manageable.
Moreover, compilation can be restricted to
selected child files by means of the |\includeonly| command.
The latter feature can be used to reduce the compilation time while editing
(this was significantly more useful in the earlier days of \LaTeX{})
or to generate a smaller document which is easier to navigate.
Another application of |\includeonly| is to generate
documents consisting of selected parts of the complete document.

However, there are a few drawbacks of the plain |\include| mechanism:
\begin{itemize}
\item
The child files cannot be compiled on their own,
they can only be compiled via the main file.
A naive editing environment
(such as a text editor with an option
to have the current file processed by \LaTeX)
may require one to switch to the main file before compiling;
attempting to compile the child file produces errors.
\item
The main file must be modified (each time)
to adjust the |\includeonly| command
to the present needs. This easily leaves the main file in a messy state.
\item
The generated document will always carry the filename
of the main document. This is inconvenient if
several child files are to be compiled and
to be kept for distribution.
\end{itemize}

The present package provides a simple interface
to make child files individually compilable by \LaTeX{}.
Compiling a child file then has the same effect as compiling
the main file with an |\includeonly| command
to select the appropriate child.
Moreover the generated document will carry the name of the child
rather than the main file.
This resolves all three above issues.

This feature is meant to make the editing of books,
thesis documents and lecture notes somewhat more convenient.
However, the package can also be used efficiently for
composing a series of documents (such as exercise sheets)
which are typically distributed individually.
It then assists the author in generating the individual documents
(potentially in different versions)
as well as a document containing the collected series.
Another application is in developing style files
or other kinds of included material
where compilation of the style file could redirect
to a sample or test file.

%%%%%%%%%%%%%%%%%%%%%%%%%%%%%%%%%%%%%%%%%%%%%%%%%%%%%%%%%%%%%%%%%%%%%%%%%%%%%%%%
%%%%%%%%%%%%%%%%%%%%%%%%%%%%%%%%%%%%%%%%%%%%%%%%%%%%%%%%%%%%%%%%%%%%%%%%%%%%%%%%
\section{Usage}

First of all, the package \textsf{childdoc} is \emph{not} a standard
\LaTeXe{} |.sty| style file! Therefore it needs to be invoked in
a non-standard way.

%%%%%%%%%%%%%%%%%%%%%%%%%%%%%%%%%%%%%%%%%%%%%%%%%%%%%%%%%%%%%%%%%%%%%%%%%%%%%%%%
\subsection{Included Files}
\label{sec:include}

%%%%%%%%%%%%%%%%%%%%%%%%%%%%%%%%%%%%%%%%
\DescribeMacro{\childdocmain}
To use the package, add the commands
\begin{center}
\begin{tabular}{l}
|\input{childdoc.def}|\\
|\childdocmain{}|\\
\end{tabular}
\end{center}
at the very top of the main \LaTeX{} file,
in particular \emph{before} the |\documentclass| statement!
The argument of |\childdocmain| should be left empty
(but it must be present).

%%%%%%%%%%%%%%%%%%%%%%%%%%%%%%%%%%%%%%%%
\DescribeMacro{\childdocof}
Furthermore, add the commands
\begin{center}
\begin{tabular}{l}
|\input{childdoc.def}|\\
|\childdocof{|\textit{main}|}|\\
\end{tabular}
\end{center}
at the top of every child file \textit{child}
which is included by |\include{|\textit{child}|}|
from within the main file
(or at least for those files to be compiled individually).
The argument \textit{main} must be the filename of the main file.

There are a couple of
considerations in setting up the main and child documents:

%%%%%%%%%%%%%%%%%%%%%%%%%%%%%%%%%%%%%%%%
\paragraph{Restrictions.}

Please note the following restrictions:
\begin{itemize}
\item
|\childdocmain| must be called with one argument \textit{main}
to ensure compatibility with earlier version of the package.
It must either be empty (|\childdocmain{}|)
or precisely match the filename of the main file in which it is specified.
See \secref{sec:detection} for further information.
\item
The filename \textit{main} must be specified without the |.tex| extension.
\item
The filename \textit{main} is case sensitive
(even in case-insensitive file systems)
due to internal string comparison.
\item
The argument \textit{main} should be fully expanded, it cannot be a macro.
\item
Subdirectories and special characters should be avoided in filenames.
\item
The command |\childdocmain{|\textit{main}|}| must be followed by a whitespace.
It should not be followed immediately by another command
or by a comment mark `|%|'.
This is because the \TeX{} parser reads the token immediately following
the argument of |\childdocmain| and puts it
at the beginning of every child section;
however, a white\-space is ignored.
\end{itemize}

%%%%%%%%%%%%%%%%%%%%%%%%%%%%%%%%%%%%%%%%
\paragraph{Content of Main File.}

It is advisable to place all content in the child files included by |\include|.
Any output contained in the main file will appear in all child documents
unless suppressed manually;
it cannot be suppressed automatically by the |\includeonly| directive
and thus should normally be avoided.
A method to include some content in the main file
by means of conditional processing is described in \secref{sec:conditional}.

%%%%%%%%%%%%%%%%%%%%%%%%%%%%%%%%%%%%%%%%
\paragraph{Page Numbering.}

When only a part of the document is compiled,
the appropriate numbering of pages
(as well as other status parameters)
is determined from the |.aux| files.
The latter contain information from previous passes.
However this information needs to propagate through
all intermediate child documents.
Therefore the page numbering in child documents may well
be inconsistent until the complete document is compiled at least once.

A useful (if unconventional) way to always ensure a consistent
page numbering is to restart the numbering in each child document
and denote the pages by `\textit{child}|.|\textit{page}'
where \textit{child} represents the chapter/section number of the child file.
This can be achieved by the command
|\numberwithin{page}{|\textit{child}|}|
of the \textsf{amsmath} package
where \textit{child} can be |chapter| or |section|
depending on the chosen structuring.
Alternatively, one can modify the macro |\thepage| appropriately
and reset the counter |page| at the start of each child file.

%%%%%%%%%%%%%%%%%%%%%%%%%%%%%%%%%%%%%%%%%%%%%%%%%%%%%%%%%%%%%%%%%%%%%%%%%%%%%%%%
\subsection{Conditional Processing}
\label{sec:conditional}

The package provides a mechanism to compile different versions
of a document. To customise the versions further some conditional processing
can come in handy to distinguish which version is being compiled.
The package provides two macros to describe the compilation context:

%%%%%%%%%%%%%%%%%%%%%%%%%%%%%%%%%%%%%%%%
\DescribeMacro{\ifchilddoc}
The conditional |\ifchilddoc| distinguishes between the compilation of
child documents and the main document:
%
\begin{center}
|\ifchilddoc |\textit{child-code}| |[|\||else |\textit{main-code}]| \||fi|
\end{center}

%%%%%%%%%%%%%%%%%%%%%%%%%%%%%%%%%%%%%%%%
\DescribeMacro{\childdocname}
\DescribeMacro{\childdocjob}
The macro |\childdocname| contains the filename (without extension)
of the main or child file being processed.
Note that |\childdocjob| will always contain the name of the main file.

%%%%%%%%%%%%%%%%%%%%%%%%%%%%%%%%%%%%%%%%
\paragraph{Title Page.}

Conditional processing can be used to include a title or banner page
in the main document when proper precautions are taken.
Importantly, the code in the main file should ensure that the page counter
(as well as other status parameters which are stored in the |.aux| files)
takes the same value after the conditional processing.
Otherwise the page numbers may take divergent values
depending on which part is compiled.

For example, a title page could be declared by:
%
\begin{center}
\begin{tabular}{l}
|\ifchilddoc\||else|\\
|\addtocounter{page}{-1}|\\
\textit{code for title page}\\
|\newpage|\\
|\||fi|
\end{tabular}
\end{center}
%
A banner page for the child documents can be generated by:
%
\begin{center}
\begin{tabular}{l}
|\ifchilddoc|\\
|\addtocounter{page}{-1}|\\
\textit{code for banner page}\\
|\newpage|\\
|\||fi|
\end{tabular}
\end{center}
%
Here one could write a message such as:
\begin{center}
|This is the part \childdocname{} of \childdocjob{}.|
\end{center}

%%%%%%%%%%%%%%%%%%%%%%%%%%%%%%%%%%%%%%%%%%%%%%%%%%%%%%%%%%%%%%%%%%%%%%%%%%%%%%%%
\subsection{Flags}
\label{sec:flags}

The package makes it easy to generate different versions
of the main or child documents.
To this end compilation flags can be defined
and assigned different default values.
They will be particularly useful in conjunction
with the forwarding mechanism described in \secref{sec:forward}.

For example, it may be useful to have a flag |\version|
which can be set to |draft| or |final|.
The document source will contain some conditional code
depending on the value of |\version|.
Suppose further, the flag should default to |final| for the main file
and to |draft| for child files
which is a natural assignment for editing the document.
This is achieved by placing the following code
in the preamble of the main document
(below the |\childdocmain| directive):
%
\begin{center}
\begin{tabular}{l}
|\ifchilddoc|\\
|\providecommand{\version}{draft}|\\
|\||else|\\
|\providecommand{\version}{final}|\\
|\||fi|
\end{tabular}
\end{center}
%
The definition by |\providecommand| makes sure
that previous definitions are not overwritten.
Further statements |\providecommand{\version}{...}|
can thus be added before the above code to override it.

For the main file, one might add a line
(between |\childdocmain| and the above block)
%
\begin{center}
|%\ifchilddoc\||else\providecommand{\version}{draft}\||fi|
\end{center}
%
which can be uncommented to produce a draft version.
Likewise one can add a line to the very top of a child file
(above the |\childdocof{|\textit{main}|}| directive)
%
\begin{center}
|%\providecommand{\version}{final}|
\end{center}
%
which can be uncommented to produce the final version of this child document.

%%%%%%%%%%%%%%%%%%%%%%%%%%%%%%%%%%%%%%%%%%%%%%%%%%%%%%%%%%%%%%%%%%%%%%%%%%%%%%%%
\subsection{Forwarding}
\label{sec:forward}

Different versions of the main or child documents
using compilation flags as described in \secref{sec:flags}
can be (permanently) stored in different files
for convenient compilation, viewing and distribution.
To this end, the package defines a command
to pass on compilation to a different file:

%%%%%%%%%%%%%%%%%%%%%%%%%%%%%%%%%%%%%%%%
\DescribeMacro{\childdocforward}
The command |\childdocforward| redirects processing to
another source file:
%
\begin{center}
\begin{tabular}{l}
|\input{childdoc.def}|\\
|\childdocforward[|\textit{main}|]{|\textit{dest}|}|\\
\end{tabular}
\end{center}
%
The argument \textit{dest} is the destination file
(without extension).
It should be the main file or one of the child files.
Note that further \textsf{childdoc} directives
such as |\childdocof| and |\childdocforward|
in the indicated file will be processed in this form.
The optional argument \textit{main}
passes on directly to the main file \textit{main}
while pretending to compile the child \textit{dest}.
This form behaves as if \textit{dest}
issues |\childdocof{|\textit{main}|}| right away,
and no further \textsf{childdoc} directives will be processed.

%%%%%%%%%%%%%%%%%%%%%%%%%%%%%%%%%%%%%%%%
\DescribeMacro{\...prefix}
In the alternative form |\childdocforwardprefix|,
%
\begin{center}
\begin{tabular}{l}
|\input{childdoc.def}|\\
|\childdocforwardprefix[|\textit{main}|]{|\textit{prefix}|}{|\textit{dest}|}|
\end{tabular}
\end{center}
%
the destination file is determined by a pattern
depending on the current file:
To make this work, the current file must be called
`{\textit{prefix}\hspace{0.2em}\textit{suffix}}'
with \textit{prefix} matching precisely the argument.
Processing is then passed on to the file
`{\textit{dest}\hspace{0.2em}\textit{suffix}}'.
Surely, the same effect is achieved by
directly specifying the
argument `{\textit{dest}\hspace{0.2em}\textit{suffix}}'
in the first form.
However, that requires to set up a different file
for each child. With the alternative form of the command
all these files can have exactly the same content
which simplifies setting them up and maintaining them.

For example, the following file |draft.tex|
with a compilation flag |\version| as described in \secref{sec:flags}
compiles the main document as a draft:
%
\begin{center}
\begin{tabular}{l}
|\def\version{draft}|\\
|\input{childdoc.def}|\\
|\childdocforward{|\textit{main}|}|
\end{tabular}
\end{center}
%
Likewise, the following files |final|\textit{nn}|.tex|
compile the final version of the child document
|child|\textit{nn}|.tex|:
%
\begin{center}
\begin{tabular}{l}
|\def\version{final}|\\
|\input{childdoc.def}|\\
|\childdocforwardprefix{final}{child}|
\end{tabular}
\end{center}
%

Note that when several versions of a main file and/or of each child file
are to be generated, it may be convenient to set up a |Makefile| or
shell script to automatise the process.

%%%%%%%%%%%%%%%%%%%%%%%%%%%%%%%%%%%%%%%%%%%%%%%%%%%%%%%%%%%%%%%%%%%%%%%%%%%%%%%%
\subsection{Command Line Processing}
\label{sec:commandline}

The effect of redirection files can also be achieved by invoking
the \LaTeX{} compiler with a more elaborate command line.
Most conveniently this should be done as part
of a shell script or a |Makefile|.

When using \textsf{childdoc} in the main file, the following
command lines effectively perform a redirection
(note that depending on the shell being used,
backslashes may have to be doubled: `|\|' $\to$ `|\\|'):
%
\begin{center}
|... -jobname "|\textit{target}|" |\\|"|[\textit{flags}]%
|\input{childdoc.def}\childdocforward[|\textit{main}|]{|\textit{dest}|}"|
\end{center}
%
Here \textit{target} is the name of the output file,
\textit{main} is the name of the main file
and \textit{dest} is the name of the main or child file to be processed
(all filenames without extensions).
The optional argument \textit{main} can be omitted
if \textit{main} matches \textit{dest}.
Optionally, compilation \textit{flags} can be defined via |\def| commands.
This command line makes the \TeX{} engine believe
it is compiling the file \textit{target}
whose content is specified as the latter parameter.
The provided code then forwards the processing to
\textit{main} or \textit{dest} as described in \secref{sec:forward}.

%%%%%%%%%%%%%%%%%%%%%%%%%%%%%%%%%%%%%%%%%%%%%%%%%%%%%%%%%%%%%%%%%%%%%%%%%%%%%%%%
\subsection{Include by Input}
\label{sec:input}

Including child documents by |\include| has some restrictions by design.
Most notably, the content of a child document always occupies
its own set of pages; pages cannot be shared between child documents.
Usually, this behaviour makes perfect sense
because each child document contain an essential part of the document.
However, in some situations it may be desirable to compose
a document from a collection of parts
without having mandatory page breaks between then.
For this case, the package
provides a mechanism to include parts
by |\input| which can also be processed individually.
However, by construction this mechanism
requires manual handling of the content to be output.

%%%%%%%%%%%%%%%%%%%%%%%%%%%%%%%%%%%%%%%%
\DescribeMacro{\ifchilddocmanual}
The main file should be prepared as usual, see \secref{sec:include}.
However, the document body must make a distinction
between processing of an individual part and of the main document, e.g.:
%
\begin{center}
\begin{tabular}{l}
|\ifchilddocmanual|\\
|\input{\childdocname}|\\
|\||else|\\
\textit{document body with }|\input{|\textit{part}|}|\\
|\||fi|
\end{tabular}
\end{center}
%
The conditional |\ifchilddocmanual| is true whenever
a part to be included by |\input| is being compiled,
and the name of the part is stored in |\childdocname|.

%%%%%%%%%%%%%%%%%%%%%%%%%%%%%%%%%%%%%%%%
\DescribeMacro{\childdocby}
Each part to be included by |\input| should start with:
%
\begin{center}
\begin{tabular}{l}
|\input{childdoc.def}|\\
|\childdocby{|\textit{main}|}|\\
\end{tabular}
\end{center}
%
The directive |\childdocby| is similar to |\childdocof|
described in \secref{sec:include},
but the subsequent selection of content must be done manually.
To that end, both |\ifchilddoc| and |\ifchilddocmanual|
will be true upon processing of a part,
and the name of the part is stored in |\childdocname|.
Note that |\jobname| will be set to the filename of the current part
so that each part receives an individual |.aux| file
that does not interfere with the |.aux| file(s) of the main document.
This behaviour can be altered by the alternative form
|\childdocby[*]{|\textit{main}|}| (with a non-empty optional argument)
which uses the |.aux| file of the main document
by setting |\jobname| to \textit{main}.

%%%%%%%%%%%%%%%%%%%%%%%%%%%%%%%%%%%%%%%%%%%%%%%%%%%%%%%%%%%%%%%%%%%%%%%%%%%%%%%%
\subsection{Driver Development}
\label{sec:driver}

The \textsf{childdoc} mechanism can also be use for the development
of definition files such as \LaTeX{} styles or classes.
This case differs from the above setup with multiple parts
included by |\include| in that no |\includeonly| should be invoked.
This can be achieved by starting the include file
(before |\ProvidesPackage|) with:
%
\begin{center}
\begin{tabular}{l}
|\input{childdoc.def}|\\
|\childdocforward{|\textit{main}|}|\\
\end{tabular}
\end{center}
%
or alternatively with:
%
\begin{center}
\begin{tabular}{l}
|\input{childdoc.def}|\\
|\childdocby{|\textit{main}|}|\\
\end{tabular}
\end{center}
%
Both forms have slightly different effects as described above.
The main file is prepared as usual, see \secref{sec:include}.

%%%%%%%%%%%%%%%%%%%%%%%%%%%%%%%%%%%%%%%%%%%%%%%%%%%%%%%%%%%%%%%%%%%%%%%%%%%%%%%%
\subsection{Legacy Detection}
\label{sec:detection}

The directive |\childdocmain| in the main file can detect
whether the complete document or merely a child is to be compiled
even without using the directive |\childdocof|.
This method is deprecated because it is less robust
and there is no compelling reason to use it;
it is merely provided for backward compatibility
and it may be removed in future versions.

If the detection mechanism is to be used,
it is mandatory to correctly specify
the filename of the main file as the argument of |\childdocmain|:
%
\begin{center}
\begin{tabular}{l}
|\input{childdoc.def}|\\
|\childdocmain{|\textit{main}|}|\\
\end{tabular}
\end{center}
%
If |\jobname| does not match the argument \textit{main} of |\childdocmain|,
it is assumed that |\jobname| points to the child file to be compiled.
When using |\childdocmain| with the main file specified as argument,
it suffices to start a child file
with just |\input{|\textit{main}|}|
without loading of the package and using |\childdocof|.
If instead all processing is done
with the appropriate \textsf{childdoc} directives,
the argument of \textit{main} of |\childdocmain| can be empty.

An alternative version of the command line processing described
in \secref{sec:commandline} using the detection mechanism reads:
%
\begin{center}
|... -jobname "|\textit{target}|" "|[\textit{flags}]%
[|\def\jobname{|\textit{dest}|}|]|\input{|\textit{main}|}"|
\end{center}

%%%%%%%%%%%%%%%%%%%%%%%%%%%%%%%%%%%%%%%%%%%%%%%%%%%%%%%%%%%%%%%%%%%%%%%%%%%%%%%%
\subsection{Manual Code}
\label{sec:manual}

In case one cannot be certain whether the definitions file |childdoc.def|
is installed on the target \TeX{} distribution
and one prefers not to ship it,
it is conceivable to paste a few relevant commands into the sources.

To that end, drop all statements |\input{childdoc.def}|
and perform the replacements as outlined below.
Instead of |\childdocmain{|\textit{main}|}| add the following code
to the top of the main file:
%
\begin{center}
\begin{tabular}{l}
|\||ifdefined\childdocname\endinput\||fi\newif\ifchilddoc|\\
|\edef\childdocname{\scantokens\expandafter{\jobname\noexpand}}|\\
|\def\childdocmain{|\textit{main}|}\||ifx\childdocmain\childdocname\||else|\\
|\childdoctrue\includeonly{\childdocname}\let\jobname\childdocmain\||fi|\\
\end{tabular}
\end{center}
%
Instead of |\childdocof{|\textit{main}|}| just include the main file
at the top of each child file:
%
\begin{center}
|\input{|\textit{main}|}|
\end{center}
%
A simple redirection |\childdocforward{|\textit{dest}|}| is achieved by:
%
\begin{center}
|\def\jobname{|\textit{dest}|}\input{\jobname}|
\end{center}
%
The redirection with prefix
|\childdocforwardprefix[|\textit{prefix}|]{|\textit{dest}|}|
is accomplished by:
%
\begin{center}
\begin{tabular}{l}
|{\edef\jobname{\scantokens\expandafter{\jobname\noexpand}}|\\
|\def\redirectjob |\textit{prefix}|#1~~~{\gdef\jobname{|\textit{dest}|#1}}|\\
|\expandafter\redirectjob\jobname~~~}\input{\jobname}|
\end{tabular}
\end{center}

In an alternative approach,
child documents can be compiled by a specific command line
without additional code or specific definitions:
%
\begin{center}
|... -jobname "|\textit{target}|" "|[\textit{flags}]%
|\includeonly{|\textit{dest}|}\input{|\textit{main}|}"|
\end{center}
%

%%%%%%%%%%%%%%%%%%%%%%%%%%%%%%%%%%%%%%%%%%%%%%%%%%%%%%%%%%%%%%%%%%%%%%%%%%%%%%%%
%%%%%%%%%%%%%%%%%%%%%%%%%%%%%%%%%%%%%%%%%%%%%%%%%%%%%%%%%%%%%%%%%%%%%%%%%%%%%%%%
\section{Information}

%%%%%%%%%%%%%%%%%%%%%%%%%%%%%%%%%%%%%%%%%%%%%%%%%%%%%%%%%%%%%%%%%%%%%%%%%%%%%%%%
\subsection{Copyright}

Copyright \copyright{} 2017--2018 Niklas Beisert

This work may be distributed and/or modified under the
conditions of the \LaTeX{} Project Public License, either version 1.3
of this license or (at your option) any later version.
The latest version of this license is in
  \url{http://www.latex-project.org/lppl.txt}
and version 1.3 or later is part of all distributions of \LaTeX{}
version 2005/12/01 or later.

This work has the LPPL maintenance status `maintained'.

The Current Maintainer of this work is Niklas Beisert.

This work consists of the files |README.txt|, |childdoc.ins| and |childdoc.dtx|
as well as the derived files |childdoc.def|, |cdocsamp.tex|
with |cdocsch1.tex|, |cdocsch2.tex|, |cdocspt3.tex|, |cdocspt4.tex|,
|cdocsdrf.tex|, |cdocsfn1.tex|, |cdocsfn2.tex|
as well as |childdoc.pdf|.

%%%%%%%%%%%%%%%%%%%%%%%%%%%%%%%%%%%%%%%%%%%%%%%%%%%%%%%%%%%%%%%%%%%%%%%%%%%%%%%%
\subsection{Files and Installation}

The package consists of the files:
%
\begin{center}
\begin{tabular}{ll}
    |README.txt|   & readme file \\
    |childdoc.ins| & installation file \\
    |childdoc.dtx| & source file \\
    |childdoc.def| & definition file \\
    |cdocsamp.tex| & sample main file \\
    |cdocsch1.tex| & sample include file \\
    |cdocsch2.tex| & sample include file \\
    |cdocspt3.tex| & sample part file \\
    |cdocspt4.tex| & sample part file \\
    |cdocsdrf.tex| & sample redirection file \\
    |cdocsfn1.tex| & sample redirection file \\
    |cdocsfn2.tex| & sample redirection file \\
    |childdoc.pdf| & manual
\end{tabular}
\end{center}
%
The distribution consists of the files
|README.txt|, |childdoc.ins| and |childdoc.dtx|.
%
\begin{itemize}
\item
Run (pdf)\LaTeX{} on |childdoc.dtx|
to compile the manual |childdoc.pdf| (this file).
\item
Run \LaTeX{} on |childdoc.ins| to create the definitions file |childdoc.def|
and the sample |cdocsamp.tex| with include files
|cdocsch1.tex|, |cdocsch2.tex|, |cdocspt3.tex|, |cdocspt4.tex|,
|cdocsdrf.tex|, |cdocsfn1.tex|, |cdocsfn2.tex|.
Then copy the file |childdoc.def| to an appropriate directory of your \LaTeX{}
distribution, e.g.\ \textit{texmf-root}|/tex/latex/childdoc|.
\end{itemize}

%%%%%%%%%%%%%%%%%%%%%%%%%%%%%%%%%%%%%%%%%%%%%%%%%%%%%%%%%%%%%%%%%%%%%%%%%%%%%%%%
\subsection{Related CTAN Packages}

There are several other packages which offer a similar functionality:
%
\begin{itemize}
\item
The packages
\href{http://ctan.org/pkg/docmute}{\textsf{docmute}},
\href{http://ctan.org/pkg/includex}{\textsf{includex}} and
\href{http://ctan.org/pkg/standalone}{\textsf{standalone}}
provide commands to include only the document body of
a child file thus allowing both files to be compiled individually.
\item
The packages \href{http://ctan.org/pkg/subdocs}{\textsf{subdocs}}
and \href{http://ctan.org/pkg/subfiles}{\textsf{subfiles}}
provide structures in which the main and child documents can be
encapsulated and allowing them to be compiled individually.
The inclusion mechanism is different from the conventional |\include|.
\item
The package \href{http://ctan.org/pkg/combine}{\textsf{combine}}
is an elaborate solution to combine several documents into one.
\end{itemize}
%
See also the CTAN topic \href{http://ctan.org/topic/subdocs}{\textsf{subdocs}}
for further related packages.
The present package differs from the above solutions in that
a document structure constructed with the conventional |\include| mechanism
just needs two extra commands at the top of every file
such that all constituent files can be compiled individually.

%%%%%%%%%%%%%%%%%%%%%%%%%%%%%%%%%%%%%%%%%%%%%%%%%%%%%%%%%%%%%%%%%%%%%%%%%%%%%%%%
%\subsection{Feature Suggestions}
%
%The following is a list of features which may be useful for future
%versions of this package:
%%
%\begin{itemize}
%\item
%\ldots
%\end{itemize}

%%%%%%%%%%%%%%%%%%%%%%%%%%%%%%%%%%%%%%%%%%%%%%%%%%%%%%%%%%%%%%%%%%%%%%%%%%%%%%%%
\subsection{Revision History}

%%%%%%%%%%%%%%%%%%%%%%%%%%%%%%%%%%%%%%%%
\paragraph{v2.0:} 2018/12/30

\begin{itemize}
\item
immediate forward processing
\item
added |\childdocby| mechanism
\item
manual restructured
\end{itemize}

%%%%%%%%%%%%%%%%%%%%%%%%%%%%%%%%%%%%%%%%
\paragraph{v1.6:} 2018/01/17

\begin{itemize}
\item
application for development of include files
\item
corrections to manual
\end{itemize}

%%%%%%%%%%%%%%%%%%%%%%%%%%%%%%%%%%%%%%%%
\paragraph{v1.5:} 2017/05/21

\begin{itemize}
\item
more complete structuring introduced
\item
|\childdocof| introduced
\item
|\childdoc| renamed to |\childdocmain|
\item
|\childredirect| renamed to |\childdocforward| and |\childdocforwardprefix|
and functionality expanded
\end{itemize}

%%%%%%%%%%%%%%%%%%%%%%%%%%%%%%%%%%%%%%%%
\paragraph{v1.0:} 2017/04/27

\begin{itemize}
\item
manual and install package
\item
first version published on CTAN
\end{itemize}

%%%%%%%%%%%%%%%%%%%%%%%%%%%%%%%%%%%%%%%%
\paragraph{v0.6:} 2017/04/26

\begin{itemize}
\item
redirection mechanism added
\end{itemize}

%%%%%%%%%%%%%%%%%%%%%%%%%%%%%%%%%%%%%%%%
\paragraph{v0.5:} 2017/04/26

\begin{itemize}
\item
functionality in definition file
\end{itemize}


%%%%%%%%%%%%%%%%%%%%%%%%%%%%%%%%%%%%%%%%%%%%%%%%%%%%%%%%%%%%%%%%%%%%%%%%%%%%%%%%
%%%%%%%%%%%%%%%%%%%%%%%%%%%%%%%%%%%%%%%%%%%%%%%%%%%%%%%%%%%%%%%%%%%%%%%%%%%%%%%%
%%%%%%%%%%%%%%%%%%%%%%%%%%%%%%%%%%%%%%%%%%%%%%%%%%%%%%%%%%%%%%%%%%%%%%%%%%%%%%%%
\appendix

\settowidth\MacroIndent{\rmfamily\scriptsize 000\ }

 \DocInput{childdoc.dtx}

\end{document}
%</driver>
% \fi
%
% %%%%%%%%%%%%%%%%%%%%%%%%%%%%%%%%%%%%%%%%%%%%%%%%%%%%%%%%%%%%%%%%%%%%%%%%%%%%%%
% %%%%%%%%%%%%%%%%%%%%%%%%%%%%%%%%%%%%%%%%%%%%%%%%%%%%%%%%%%%%%%%%%%%%%%%%%%%%%%
% \section{Sample}
%\iffalse
%<*samplemain>
%\fi
%
% The following presents a sample document
% with two chapters, two parts, a title page,
% a compile flag as well as three forwarding files to set the flag.
% It consists of eight |.tex| files:
% \begin{center}
% \begin{tabular}{ll}
% |cdocsamp.tex|&main file\\
% |cdocsch1.tex|&include file for chapter 1\\
% |cdocsch2.tex|&include file for chapter 2\\
% |cdocspt3.tex|&include file for part 3\\
% |cdocspt4.tex|&include file for part 4\\
% |cdocsdrf.tex|&forwarding file for main file in draft mode\\
% |cdocsfi1.tex|&forwarding file for final version of chapter 1\\
% |cdocsfi2.tex|&forwarding file for final version of chapter 2\\
% \end{tabular}
% \end{center}
% Each of the eight files can be compiled directly by the \LaTeX{} compiler.
%
% %%%%%%%%%%%%%%%%%%%%%%%%%%%%%%%%%%%%%%
% \paragraph{Main File.}
%
% The main file is called |cdocsamp.tex|.
%
% Load the \textsf{childdoc} definitions and
% declare the filename for the main document:
%    \begin{macrocode}
\input{childdoc.def}
\childdocmain{}
%    \end{macrocode}

% Optional override for |\version| flag:
%    \begin{macrocode}
%%\ifchilddoc\else\providecommand{\version}{draft}\fi
%    \end{macrocode}

% Define the default values for the |\version| flag
% (|final| for the main file and |draft| for childs):
%    \begin{macrocode}
\ifchilddoc
\providecommand{\version}{draft}
\else
\providecommand{\version}{final}
\fi
%    \end{macrocode}

% Load the standard document class:
%    \begin{macrocode}
\documentclass[12pt]{article}
%    \end{macrocode}

% Start the document body:
%    \begin{macrocode}
\begin{document}
%    \end{macrocode}

% Declare a title page.
% Print title, part of document being processed and version flag:
%    \begin{macrocode}
\addtocounter{page}{-1}
\begin{center}
{\LARGE\bfseries{}childdoc example\par}
\vspace{1cm}
\ifchilddoc
\ifchilddocmanual part\else chapter\fi:
`\childdocname' of `\childdocjob'\par
\else
main document: `\childdocjob'\par
\fi
version: \version\par
\end{center}
\newpage
%    \end{macrocode}

% Manually include selected file,
% otherwise process as usual:
%    \begin{macrocode}
\ifchilddocmanual
\section*{part `\childdocname'}
\input{\childdocname}
\else
%    \end{macrocode}

% Include the two chapters:
%    \begin{macrocode}
\include{cdocsch1}
\include{cdocsch2}
%    \end{macrocode}

% Include the two parts unless only chapters should be displayed:
%    \begin{macrocode}
\ifchilddoc\else
\section{part three}
\input{cdocspt3}
\section{part four}
\input{cdocspt4}
\fi
%    \end{macrocode}

% Process as usual until here:
%    \begin{macrocode}
\fi
%    \end{macrocode}

% End of document body:
%    \begin{macrocode}
\end{document}
%    \end{macrocode}
%\iffalse
%</samplemain>
%\fi
%
% %%%%%%%%%%%%%%%%%%%%%%%%%%%%%%%%%%%%%%
% \paragraph{Chapter Include Files.}
%
% The include files are called |cdocsch1.tex| and |cdocsch2.tex|.
%
%\iffalse
%<*samplechap1|samplechap2>
%\fi

% Optional override for |\version| flag:
%    \begin{macrocode}
%%\providecommand{\version}{final}
%    \end{macrocode}

% Include the main document:
%    \begin{macrocode}
\input{childdoc.def}
\childdocof{cdocsamp}
%    \end{macrocode}

%\iffalse
%</samplechap1|samplechap2>
%\fi
%
%\iffalse
%<*samplechap1>
%\fi
% Some text for chapter 1:
%    \begin{macrocode}
\section{one}
some text in chapter one
%    \end{macrocode}

%\iffalse
%</samplechap1>
%\fi
% Some text for chapter 2:
%\iffalse
%<*samplechap2>
%\fi
%    \begin{macrocode}
\section{two}
more text in chapter two
%    \end{macrocode}

%\iffalse
%</samplechap2>
%\fi
%
% %%%%%%%%%%%%%%%%%%%%%%%%%%%%%%%%%%%%%%
% \paragraph{Part Include Files.}
%
% The include files are called |cdocspt3.tex| and |cdocspt4.tex|.
%
%\iffalse
%<*samplepart3|samplepart4>
%\fi

% Optional override for |\version| flag:
%    \begin{macrocode}
%%\providecommand{\version}{final}
%    \end{macrocode}

% Include the main document:
%    \begin{macrocode}
\input{childdoc.def}
\childdocby{cdocsamp}
%    \end{macrocode}

%\iffalse
%</samplepart3|samplepart4>
%\fi
%
%\iffalse
%<*samplepart3>
%\fi
% Some text for part 3:
%    \begin{macrocode}
some text in part three
%    \end{macrocode}

%\iffalse
%</samplepart3>
%\fi
% Some text for part 4:
%\iffalse
%<*samplepart4>
%\fi
%    \begin{macrocode}
more text in part four
%    \end{macrocode}

%\iffalse
%</samplepart4>
%\fi
%
% %%%%%%%%%%%%%%%%%%%%%%%%%%%%%%%%%%%%%%
% \paragraph{Forwarding for a Complete Draft.}
%
% The following forwarding file |cdocsdrf.tex|
% compiles the main document in draft mode:
%\iffalse
%<*sampledraft>
%\fi
%    \begin{macrocode}
\def\version{draft}
\input{childdoc.def}
\childdocforward{cdocsamp}
%    \end{macrocode}

%\iffalse
%</sampledraft>
%\fi
%
% %%%%%%%%%%%%%%%%%%%%%%%%%%%%%%%%%%%%%%
% \paragraph{Forwarding for Final Version of the Chapters.}
%
% The following forwarding files |cdocsfn1.tex| and |cdocsfn2.tex|
% (with identical content)
% compile the final versions of the child documents
% |cdocsch1.tex| and |cdocsch2.tex|, respectively:
%\iffalse
%<*samplefinal>
%\fi
%    \begin{macrocode}
\def\version{final}
\input{childdoc.def}
\childdocforwardprefix[cdocsamp]{cdocsfn}{cdocsch}
%    \end{macrocode}

%\iffalse
%</samplefinal>
%\fi
%
% %%%%%%%%%%%%%%%%%%%%%%%%%%%%%%%%%%%%%%
% \paragraph{Command Line Processing.}
%
% The following three command lines generate the output files
% |cdocscld|, |cdocscl1| and |cdocscl2|
% which should be identical to
% |cdocsdrf|, |cdocsch1| and |cdocsfn2|, respectively:
% \begin{center}
% \begin{tabular}{l}
% |latex -jobname cdocscld \|\\
% |  "\def\version{draft}\input{childdoc.def}\childdocforward{cdocsamp}"|\\
% |latex -jobname cdocscl1 \|\\
% |  "\input{childdoc.def}\childdocforward[cdocsamp]{cdocsch1}"|\\
% |latex -jobname cdocscl2 \|\\
% |  "\def\version{final}\input{childdoc.def}\childdocforward{cdocsch2}"|
% \end{tabular}
% \end{center}
% Note that the trailing backslash on each first line
% merely continues the input to the second line
% (for convenient cut ant paste).
% Furthermore, the command |latex| can be replaced by any
% of its alternative versions such as |pdflatex|.
%
% %%%%%%%%%%%%%%%%%%%%%%%%%%%%%%%%%%%%%%%%%%%%%%%%%%%%%%%%%%%%%%%%%%%%%%%%%%%%%%
% %%%%%%%%%%%%%%%%%%%%%%%%%%%%%%%%%%%%%%%%%%%%%%%%%%%%%%%%%%%%%%%%%%%%%%%%%%%%%%
% \section{Implementation}
%\iffalse
%<*package>
%\fi
%
% This section describes the definitions file |childdoc.def|.

% The definitions cannot be loaded using |\usepackage| or |\RequirePackage|
% which has a mechanism to prevent loading a style file more than once.
% When loading the definitions by means of |\input|
% multiple instances have to be prevented manually:
%\iffalse
%This code needs to be before the `\ProvidesFile' directive
%which is defined at the beginning of this file.
%Therefore it is also placed there and commented out here.
%</package>
%<*discard>
%\fi
%    \begin{macrocode}
\ifdefined\childdocmain\endinput\fi
%    \end{macrocode}
%\iffalse
%</discard>
%<*package>
%\fi
%
% \macro{\ifchilddoc}
% \macro{\ifchilddocmanual}
% The conditional |\ifchilddoc| tells whether a
% child (true) or main (false) document is being compiled.
% The conditional |\ifchilddocmanual| tells whether
% the |\includeonly| mechanism is used (false) or
% the selection of child files must be performed manually (true).
% The definitions initialise to false:
%    \begin{macrocode}
\newif\ifchilddoc
\newif\ifchilddocmanual
%    \end{macrocode}

% \macro{\childdocname}
% \macro{\childdocjob}
% The macro |\childdocname| stores the name of the main document
% to be compiled. The macro |\childdocjob| stores the name of
% the document on which the \LaTeX{} compiler was originally invoked.
% The content of |\jobname| cannot be compared
% to filenames specified in the source due to different catcodes.
% The following code rescans |\jobname|, stores the result
% in |\childdocname| and saves a copy in |\childdocjob|:
%    \begin{macrocode}
\edef\childdocname{\scantokens\expandafter{\jobname\noexpand}}
\let\childdocjob\childdocname
%    \end{macrocode}

% \macro{\childdocdisable}
% The macro |\childdocdisable| prevents the main file
% from being processed more than once.
% At this stage, the main document command |\childdocmain|
% is assumed to be called once again where it should do nothing.
% Any subsequent call to it should prevent
% a secondary processing of the main document
% It overwrites the forwarding commands
% |\childdocof| and |\childdocforward|
% with empty macros to prevent further inclusions of the main document:
%    \begin{macrocode}
\newcommand{\childdocdisable}
{
  \renewcommand{\childdocmain}[1]{\renewcommand{\childdocmain}[1]{\endinput}}
  \renewcommand{\childdocof}[1]{}
  \renewcommand{\childdocby}[2][]{}
  \renewcommand{\childdocforward}[2][]{}
  \renewcommand{\childdocdisable}{}
}
%    \end{macrocode}

% \macro{\childdocmain}
% The macro |\childdocmain| is to be called at the top of the main file
% with nothing or the main filename (without extension) as argument.
% First, it breaks loops.
% If the argument is not empty and does not match |\childdocname|
% (which is set by the first inclusion of |childdoc.def|),
% |\ifchilddoc| is set to true, |\includeonly| is applied to the child file
% and |\jobname| is set to the main file
% (for proper handling of |.aux| files):
%    \begin{macrocode}
\newcommand{\childdocmain}[1]
{
  \childdocdisable\childdocmain{}
  \if?#1?\else
    \begingroup
      \def\childdoctmp{#1}
      \ifx\childdoctmp\childdocname
        \def\childdoctmp{}
      \else
        \def\childdoctmp
        {
          \childdoctrue
          \includeonly{\childdocname}
          \def\childdocjob{#1}
          \def\jobname{#1}
        }
      \fi
      \expandafter
    \endgroup
    \childdoctmp
  \fi
}
%    \end{macrocode}

% \macro{\childdocof}
% The command |\childdocof| redirects
% compilation to the main file |#1|.
%    \begin{macrocode}
\newcommand{\childdocof}[1]
{
  \childdocdisable
  \childdoctrue
  \includeonly{\childdocname}
  \def\jobname{#1}
  \def\childdocjob{#1}
  \input{#1}
}
%    \end{macrocode}

% \macro{\childdocby}
% The command |\childdocby| ....
%    \begin{macrocode}
\newcommand{\childdocby}[2][]
{
  \childdocdisable
  \childdoctrue
  \childdocmanualtrue
  \if?#1?\else
    \def\jobname{#2}
  \fi
  \def\childdocjob{#2}
  \input{#2}
  \endinput
}
%    \end{macrocode}

% \macro{\childdocforward}
% The command |\childdocforward| redirects
% compilation to the main file or
% (if the optional argument is given) a child file.
% Parameters are set as if the main file
% or a child file starting with |\childdocof| was compiled.
% Then compilation is handed over to the main file:
%    \begin{macrocode}
\newcommand{\childdocforward}[2][]
{
  \begingroup
    \if?#1?
      \def\childdoctmp
      {
        \def\childdocname{#2}
        \def\childdocjob{#2}
        \def\jobname{#2}
        \input{#2}
        \endinput
      }
    \else
      \def\childdoctmp
      {
        \childdocdisable
        \def\childdocname{#2}
        \childdoctrue
        \includeonly{#2}
        \def\childdocjob{#1}
        \def\jobname{#1}
        \input{#1}
        \endinput
      }
    \fi
    \expandafter
  \endgroup
  \childdoctmp
}
%    \end{macrocode}

% \macro{\childdocforwardprefix}
% The command |\childdocforwardprefix| redirects
% compilation to the main or a child file by means of a pattern.
% The prefix |#1| in the current filename is replaced by |#2|
% and the suffix of the current filename is kept
% (it is assumed that the filename does not contain the substring `|~~~|'
% which is used as a delimiter).
% Compilation is handed over to the new file by |\childdocforward|:
%    \begin{macrocode}
\newcommand{\childdocforwardprefix}[3][]
{
  \begingroup
    \def\childdocextract #2##1~~~{\def\childdoctmp{\childdocforward[#1]{#3##1}}}
    \expandafter\childdocextract\childdocname~~~
    \expandafter
  \endgroup
  \childdoctmp
}
%    \end{macrocode}

% \macro{\childdoc}
% The deprecated macro |\childdoc| is a legacy version of |\childdocmain|:
%    \begin{macrocode}
\newcommand{\childdoc}{\childdocmain}
%    \end{macrocode}

% \macro{\childdocredirect}
% The deprecated macro |\childdocredirect| is a legacy version
% of |\childdocforward| and |\childdocforwardprefix|:
%    \begin{macrocode}
\newcommand{\childdocredirect}[2][]
{
  \begingroup
    \if?#1?
      \def\childdoctmp{\childdocforward{#2}}
    \else
      \def\childdoctmp{\childdocforwardprefix{#1}{#2}}
    \fi
    \expandafter
  \endgroup
  \childdoctmp
}
%    \end{macrocode}

%\iffalse
%</package>
%\fi
%
\endinput
\childdocforward[cdocsamp]{cdocsch1}"|\\
% |latex -jobname cdocscl2 \|\\
% |  "\def\version{final}% \iffalse
%
% childdoc.dtx Copyright (C) 2017-2018 Niklas Beisert
%
% This work may be distributed and/or modified under the
% conditions of the LaTeX Project Public License, either version 1.3
% of this license or (at your option) any later version.
% The latest version of this license is in
%   http://www.latex-project.org/lppl.txt
% and version 1.3 or later is part of all distributions of LaTeX
% version 2005/12/01 or later.
%
% This work has the LPPL maintenance status `maintained'.
%
% The Current Maintainer of this work is Niklas Beisert.
%
% This work consists of the files childdoc.dtx and childdoc.ins
% and the derived files childdoc.def and cdocsamp.tex with
% cdocsch1.tex, cdocsch2.tex, cdocsdrf.tex, cdocsfn1.tex, cdocsfn2.tex.
%
%<package>\ifdefined\childdocmain\endinput\fi
%<package>\ProvidesFile{childdoc.def}[2018/12/30 v2.0 child document driver]
%<samplemain>\ProvidesFile{cdocsamp.tex}[2018/12/30 v2.0 sample for childdoc]
%<*driver>
%\ProvidesFile{childdoc.drv}[2018/12/30 v2.0 childdoc reference manual file]
\PassOptionsToClass{10pt,a4paper}{article}
\documentclass{ltxdoc}

\usepackage[margin=35mm]{geometry}
\usepackage{hyperref}
\usepackage{hyperxmp}
\usepackage[usenames]{color}

\hypersetup{colorlinks=true}
\hypersetup{pdfstartview=FitH}
\hypersetup{pdfpagemode=UseNone}
\hypersetup{pdfsource={}}
\hypersetup{pdflang={en-UK}}
\hypersetup{pdfcopyright={Copyright 2017-2018 Niklas Beisert.
  This work may be distributed and/or modified under the
  conditions of the LaTeX Project Public License, either version 1.3
  of this license or (at your option) any later version.}}
\hypersetup{pdflicenseurl={http://www.latex-project.org/lppl.txt}}
\hypersetup{pdfcontactaddress={ETH Zurich, ITP, HIT K,
  Wolfgang-Pauli-Strasse 27}}
\hypersetup{pdfcontactpostcode={8093}}
\hypersetup{pdfcontactcity={Zurich}}
\hypersetup{pdfcontactcountry={Switzerland}}
\hypersetup{pdfcontactemail={nbeisert@itp.phys.ethz.ch}}
\hypersetup{pdfcontacturl={http://people.phys.ethz.ch/\xmptilde nbeisert/}}

\newcommand{\secref}[1]{\hyperref[#1]{section \ref*{#1}}}

\parskip1ex
\parindent0pt
\let\olditemize\itemize
\def\itemize{\olditemize\parskip0pt}

\begin{document}

\title{The \textsf{childdoc} Package}
\hypersetup{pdftitle={The childdoc Package}}
\author{Niklas Beisert\\[2ex]
  Institut f\"ur Theoretische Physik\\
  Eidgen\"ossische Technische Hochschule Z\"urich\\
  Wolfgang-Pauli-Strasse 27, 8093 Z\"urich, Switzerland\\[1ex]
  \href{mailto:nbeisert@itp.phys.ethz.ch}
  {\texttt{nbeisert@itp.phys.ethz.ch}}}
\hypersetup{pdfauthor={Niklas Beisert}}
\hypersetup{pdfsubject={Manual for the LaTeX2e Package childdoc}}
\date{30 December 2018, \textsf{v2.0}}
\maketitle

\begin{abstract}\noindent
\textsf{childdoc} is a \LaTeXe{} package
that enables the direct compilation
of document sections included by |\include|
to individual files.
\end{abstract}

\begingroup
\parskip0ex
\tableofcontents
\endgroup

%%%%%%%%%%%%%%%%%%%%%%%%%%%%%%%%%%%%%%%%%%%%%%%%%%%%%%%%%%%%%%%%%%%%%%%%%%%%%%%%
%%%%%%%%%%%%%%%%%%%%%%%%%%%%%%%%%%%%%%%%%%%%%%%%%%%%%%%%%%%%%%%%%%%%%%%%%%%%%%%%
\section{Introduction}

\LaTeX{} provides a mechanism to structure a large document (such as a book)
into a main file and several child files (containing the chapters)
using the |\include| command.
This mechanism is beneficial for documents
which span hundreds of pages in order to
make the source file(s) more manageable.
Moreover, compilation can be restricted to
selected child files by means of the |\includeonly| command.
The latter feature can be used to reduce the compilation time while editing
(this was significantly more useful in the earlier days of \LaTeX{})
or to generate a smaller document which is easier to navigate.
Another application of |\includeonly| is to generate
documents consisting of selected parts of the complete document.

However, there are a few drawbacks of the plain |\include| mechanism:
\begin{itemize}
\item
The child files cannot be compiled on their own,
they can only be compiled via the main file.
A naive editing environment
(such as a text editor with an option
to have the current file processed by \LaTeX)
may require one to switch to the main file before compiling;
attempting to compile the child file produces errors.
\item
The main file must be modified (each time)
to adjust the |\includeonly| command
to the present needs. This easily leaves the main file in a messy state.
\item
The generated document will always carry the filename
of the main document. This is inconvenient if
several child files are to be compiled and
to be kept for distribution.
\end{itemize}

The present package provides a simple interface
to make child files individually compilable by \LaTeX{}.
Compiling a child file then has the same effect as compiling
the main file with an |\includeonly| command
to select the appropriate child.
Moreover the generated document will carry the name of the child
rather than the main file.
This resolves all three above issues.

This feature is meant to make the editing of books,
thesis documents and lecture notes somewhat more convenient.
However, the package can also be used efficiently for
composing a series of documents (such as exercise sheets)
which are typically distributed individually.
It then assists the author in generating the individual documents
(potentially in different versions)
as well as a document containing the collected series.
Another application is in developing style files
or other kinds of included material
where compilation of the style file could redirect
to a sample or test file.

%%%%%%%%%%%%%%%%%%%%%%%%%%%%%%%%%%%%%%%%%%%%%%%%%%%%%%%%%%%%%%%%%%%%%%%%%%%%%%%%
%%%%%%%%%%%%%%%%%%%%%%%%%%%%%%%%%%%%%%%%%%%%%%%%%%%%%%%%%%%%%%%%%%%%%%%%%%%%%%%%
\section{Usage}

First of all, the package \textsf{childdoc} is \emph{not} a standard
\LaTeXe{} |.sty| style file! Therefore it needs to be invoked in
a non-standard way.

%%%%%%%%%%%%%%%%%%%%%%%%%%%%%%%%%%%%%%%%%%%%%%%%%%%%%%%%%%%%%%%%%%%%%%%%%%%%%%%%
\subsection{Included Files}
\label{sec:include}

%%%%%%%%%%%%%%%%%%%%%%%%%%%%%%%%%%%%%%%%
\DescribeMacro{\childdocmain}
To use the package, add the commands
\begin{center}
\begin{tabular}{l}
|\input{childdoc.def}|\\
|\childdocmain{}|\\
\end{tabular}
\end{center}
at the very top of the main \LaTeX{} file,
in particular \emph{before} the |\documentclass| statement!
The argument of |\childdocmain| should be left empty
(but it must be present).

%%%%%%%%%%%%%%%%%%%%%%%%%%%%%%%%%%%%%%%%
\DescribeMacro{\childdocof}
Furthermore, add the commands
\begin{center}
\begin{tabular}{l}
|\input{childdoc.def}|\\
|\childdocof{|\textit{main}|}|\\
\end{tabular}
\end{center}
at the top of every child file \textit{child}
which is included by |\include{|\textit{child}|}|
from within the main file
(or at least for those files to be compiled individually).
The argument \textit{main} must be the filename of the main file.

There are a couple of
considerations in setting up the main and child documents:

%%%%%%%%%%%%%%%%%%%%%%%%%%%%%%%%%%%%%%%%
\paragraph{Restrictions.}

Please note the following restrictions:
\begin{itemize}
\item
|\childdocmain| must be called with one argument \textit{main}
to ensure compatibility with earlier version of the package.
It must either be empty (|\childdocmain{}|)
or precisely match the filename of the main file in which it is specified.
See \secref{sec:detection} for further information.
\item
The filename \textit{main} must be specified without the |.tex| extension.
\item
The filename \textit{main} is case sensitive
(even in case-insensitive file systems)
due to internal string comparison.
\item
The argument \textit{main} should be fully expanded, it cannot be a macro.
\item
Subdirectories and special characters should be avoided in filenames.
\item
The command |\childdocmain{|\textit{main}|}| must be followed by a whitespace.
It should not be followed immediately by another command
or by a comment mark `|%|'.
This is because the \TeX{} parser reads the token immediately following
the argument of |\childdocmain| and puts it
at the beginning of every child section;
however, a white\-space is ignored.
\end{itemize}

%%%%%%%%%%%%%%%%%%%%%%%%%%%%%%%%%%%%%%%%
\paragraph{Content of Main File.}

It is advisable to place all content in the child files included by |\include|.
Any output contained in the main file will appear in all child documents
unless suppressed manually;
it cannot be suppressed automatically by the |\includeonly| directive
and thus should normally be avoided.
A method to include some content in the main file
by means of conditional processing is described in \secref{sec:conditional}.

%%%%%%%%%%%%%%%%%%%%%%%%%%%%%%%%%%%%%%%%
\paragraph{Page Numbering.}

When only a part of the document is compiled,
the appropriate numbering of pages
(as well as other status parameters)
is determined from the |.aux| files.
The latter contain information from previous passes.
However this information needs to propagate through
all intermediate child documents.
Therefore the page numbering in child documents may well
be inconsistent until the complete document is compiled at least once.

A useful (if unconventional) way to always ensure a consistent
page numbering is to restart the numbering in each child document
and denote the pages by `\textit{child}|.|\textit{page}'
where \textit{child} represents the chapter/section number of the child file.
This can be achieved by the command
|\numberwithin{page}{|\textit{child}|}|
of the \textsf{amsmath} package
where \textit{child} can be |chapter| or |section|
depending on the chosen structuring.
Alternatively, one can modify the macro |\thepage| appropriately
and reset the counter |page| at the start of each child file.

%%%%%%%%%%%%%%%%%%%%%%%%%%%%%%%%%%%%%%%%%%%%%%%%%%%%%%%%%%%%%%%%%%%%%%%%%%%%%%%%
\subsection{Conditional Processing}
\label{sec:conditional}

The package provides a mechanism to compile different versions
of a document. To customise the versions further some conditional processing
can come in handy to distinguish which version is being compiled.
The package provides two macros to describe the compilation context:

%%%%%%%%%%%%%%%%%%%%%%%%%%%%%%%%%%%%%%%%
\DescribeMacro{\ifchilddoc}
The conditional |\ifchilddoc| distinguishes between the compilation of
child documents and the main document:
%
\begin{center}
|\ifchilddoc |\textit{child-code}| |[|\||else |\textit{main-code}]| \||fi|
\end{center}

%%%%%%%%%%%%%%%%%%%%%%%%%%%%%%%%%%%%%%%%
\DescribeMacro{\childdocname}
\DescribeMacro{\childdocjob}
The macro |\childdocname| contains the filename (without extension)
of the main or child file being processed.
Note that |\childdocjob| will always contain the name of the main file.

%%%%%%%%%%%%%%%%%%%%%%%%%%%%%%%%%%%%%%%%
\paragraph{Title Page.}

Conditional processing can be used to include a title or banner page
in the main document when proper precautions are taken.
Importantly, the code in the main file should ensure that the page counter
(as well as other status parameters which are stored in the |.aux| files)
takes the same value after the conditional processing.
Otherwise the page numbers may take divergent values
depending on which part is compiled.

For example, a title page could be declared by:
%
\begin{center}
\begin{tabular}{l}
|\ifchilddoc\||else|\\
|\addtocounter{page}{-1}|\\
\textit{code for title page}\\
|\newpage|\\
|\||fi|
\end{tabular}
\end{center}
%
A banner page for the child documents can be generated by:
%
\begin{center}
\begin{tabular}{l}
|\ifchilddoc|\\
|\addtocounter{page}{-1}|\\
\textit{code for banner page}\\
|\newpage|\\
|\||fi|
\end{tabular}
\end{center}
%
Here one could write a message such as:
\begin{center}
|This is the part \childdocname{} of \childdocjob{}.|
\end{center}

%%%%%%%%%%%%%%%%%%%%%%%%%%%%%%%%%%%%%%%%%%%%%%%%%%%%%%%%%%%%%%%%%%%%%%%%%%%%%%%%
\subsection{Flags}
\label{sec:flags}

The package makes it easy to generate different versions
of the main or child documents.
To this end compilation flags can be defined
and assigned different default values.
They will be particularly useful in conjunction
with the forwarding mechanism described in \secref{sec:forward}.

For example, it may be useful to have a flag |\version|
which can be set to |draft| or |final|.
The document source will contain some conditional code
depending on the value of |\version|.
Suppose further, the flag should default to |final| for the main file
and to |draft| for child files
which is a natural assignment for editing the document.
This is achieved by placing the following code
in the preamble of the main document
(below the |\childdocmain| directive):
%
\begin{center}
\begin{tabular}{l}
|\ifchilddoc|\\
|\providecommand{\version}{draft}|\\
|\||else|\\
|\providecommand{\version}{final}|\\
|\||fi|
\end{tabular}
\end{center}
%
The definition by |\providecommand| makes sure
that previous definitions are not overwritten.
Further statements |\providecommand{\version}{...}|
can thus be added before the above code to override it.

For the main file, one might add a line
(between |\childdocmain| and the above block)
%
\begin{center}
|%\ifchilddoc\||else\providecommand{\version}{draft}\||fi|
\end{center}
%
which can be uncommented to produce a draft version.
Likewise one can add a line to the very top of a child file
(above the |\childdocof{|\textit{main}|}| directive)
%
\begin{center}
|%\providecommand{\version}{final}|
\end{center}
%
which can be uncommented to produce the final version of this child document.

%%%%%%%%%%%%%%%%%%%%%%%%%%%%%%%%%%%%%%%%%%%%%%%%%%%%%%%%%%%%%%%%%%%%%%%%%%%%%%%%
\subsection{Forwarding}
\label{sec:forward}

Different versions of the main or child documents
using compilation flags as described in \secref{sec:flags}
can be (permanently) stored in different files
for convenient compilation, viewing and distribution.
To this end, the package defines a command
to pass on compilation to a different file:

%%%%%%%%%%%%%%%%%%%%%%%%%%%%%%%%%%%%%%%%
\DescribeMacro{\childdocforward}
The command |\childdocforward| redirects processing to
another source file:
%
\begin{center}
\begin{tabular}{l}
|\input{childdoc.def}|\\
|\childdocforward[|\textit{main}|]{|\textit{dest}|}|\\
\end{tabular}
\end{center}
%
The argument \textit{dest} is the destination file
(without extension).
It should be the main file or one of the child files.
Note that further \textsf{childdoc} directives
such as |\childdocof| and |\childdocforward|
in the indicated file will be processed in this form.
The optional argument \textit{main}
passes on directly to the main file \textit{main}
while pretending to compile the child \textit{dest}.
This form behaves as if \textit{dest}
issues |\childdocof{|\textit{main}|}| right away,
and no further \textsf{childdoc} directives will be processed.

%%%%%%%%%%%%%%%%%%%%%%%%%%%%%%%%%%%%%%%%
\DescribeMacro{\...prefix}
In the alternative form |\childdocforwardprefix|,
%
\begin{center}
\begin{tabular}{l}
|\input{childdoc.def}|\\
|\childdocforwardprefix[|\textit{main}|]{|\textit{prefix}|}{|\textit{dest}|}|
\end{tabular}
\end{center}
%
the destination file is determined by a pattern
depending on the current file:
To make this work, the current file must be called
`{\textit{prefix}\hspace{0.2em}\textit{suffix}}'
with \textit{prefix} matching precisely the argument.
Processing is then passed on to the file
`{\textit{dest}\hspace{0.2em}\textit{suffix}}'.
Surely, the same effect is achieved by
directly specifying the
argument `{\textit{dest}\hspace{0.2em}\textit{suffix}}'
in the first form.
However, that requires to set up a different file
for each child. With the alternative form of the command
all these files can have exactly the same content
which simplifies setting them up and maintaining them.

For example, the following file |draft.tex|
with a compilation flag |\version| as described in \secref{sec:flags}
compiles the main document as a draft:
%
\begin{center}
\begin{tabular}{l}
|\def\version{draft}|\\
|\input{childdoc.def}|\\
|\childdocforward{|\textit{main}|}|
\end{tabular}
\end{center}
%
Likewise, the following files |final|\textit{nn}|.tex|
compile the final version of the child document
|child|\textit{nn}|.tex|:
%
\begin{center}
\begin{tabular}{l}
|\def\version{final}|\\
|\input{childdoc.def}|\\
|\childdocforwardprefix{final}{child}|
\end{tabular}
\end{center}
%

Note that when several versions of a main file and/or of each child file
are to be generated, it may be convenient to set up a |Makefile| or
shell script to automatise the process.

%%%%%%%%%%%%%%%%%%%%%%%%%%%%%%%%%%%%%%%%%%%%%%%%%%%%%%%%%%%%%%%%%%%%%%%%%%%%%%%%
\subsection{Command Line Processing}
\label{sec:commandline}

The effect of redirection files can also be achieved by invoking
the \LaTeX{} compiler with a more elaborate command line.
Most conveniently this should be done as part
of a shell script or a |Makefile|.

When using \textsf{childdoc} in the main file, the following
command lines effectively perform a redirection
(note that depending on the shell being used,
backslashes may have to be doubled: `|\|' $\to$ `|\\|'):
%
\begin{center}
|... -jobname "|\textit{target}|" |\\|"|[\textit{flags}]%
|\input{childdoc.def}\childdocforward[|\textit{main}|]{|\textit{dest}|}"|
\end{center}
%
Here \textit{target} is the name of the output file,
\textit{main} is the name of the main file
and \textit{dest} is the name of the main or child file to be processed
(all filenames without extensions).
The optional argument \textit{main} can be omitted
if \textit{main} matches \textit{dest}.
Optionally, compilation \textit{flags} can be defined via |\def| commands.
This command line makes the \TeX{} engine believe
it is compiling the file \textit{target}
whose content is specified as the latter parameter.
The provided code then forwards the processing to
\textit{main} or \textit{dest} as described in \secref{sec:forward}.

%%%%%%%%%%%%%%%%%%%%%%%%%%%%%%%%%%%%%%%%%%%%%%%%%%%%%%%%%%%%%%%%%%%%%%%%%%%%%%%%
\subsection{Include by Input}
\label{sec:input}

Including child documents by |\include| has some restrictions by design.
Most notably, the content of a child document always occupies
its own set of pages; pages cannot be shared between child documents.
Usually, this behaviour makes perfect sense
because each child document contain an essential part of the document.
However, in some situations it may be desirable to compose
a document from a collection of parts
without having mandatory page breaks between then.
For this case, the package
provides a mechanism to include parts
by |\input| which can also be processed individually.
However, by construction this mechanism
requires manual handling of the content to be output.

%%%%%%%%%%%%%%%%%%%%%%%%%%%%%%%%%%%%%%%%
\DescribeMacro{\ifchilddocmanual}
The main file should be prepared as usual, see \secref{sec:include}.
However, the document body must make a distinction
between processing of an individual part and of the main document, e.g.:
%
\begin{center}
\begin{tabular}{l}
|\ifchilddocmanual|\\
|\input{\childdocname}|\\
|\||else|\\
\textit{document body with }|\input{|\textit{part}|}|\\
|\||fi|
\end{tabular}
\end{center}
%
The conditional |\ifchilddocmanual| is true whenever
a part to be included by |\input| is being compiled,
and the name of the part is stored in |\childdocname|.

%%%%%%%%%%%%%%%%%%%%%%%%%%%%%%%%%%%%%%%%
\DescribeMacro{\childdocby}
Each part to be included by |\input| should start with:
%
\begin{center}
\begin{tabular}{l}
|\input{childdoc.def}|\\
|\childdocby{|\textit{main}|}|\\
\end{tabular}
\end{center}
%
The directive |\childdocby| is similar to |\childdocof|
described in \secref{sec:include},
but the subsequent selection of content must be done manually.
To that end, both |\ifchilddoc| and |\ifchilddocmanual|
will be true upon processing of a part,
and the name of the part is stored in |\childdocname|.
Note that |\jobname| will be set to the filename of the current part
so that each part receives an individual |.aux| file
that does not interfere with the |.aux| file(s) of the main document.
This behaviour can be altered by the alternative form
|\childdocby[*]{|\textit{main}|}| (with a non-empty optional argument)
which uses the |.aux| file of the main document
by setting |\jobname| to \textit{main}.

%%%%%%%%%%%%%%%%%%%%%%%%%%%%%%%%%%%%%%%%%%%%%%%%%%%%%%%%%%%%%%%%%%%%%%%%%%%%%%%%
\subsection{Driver Development}
\label{sec:driver}

The \textsf{childdoc} mechanism can also be use for the development
of definition files such as \LaTeX{} styles or classes.
This case differs from the above setup with multiple parts
included by |\include| in that no |\includeonly| should be invoked.
This can be achieved by starting the include file
(before |\ProvidesPackage|) with:
%
\begin{center}
\begin{tabular}{l}
|\input{childdoc.def}|\\
|\childdocforward{|\textit{main}|}|\\
\end{tabular}
\end{center}
%
or alternatively with:
%
\begin{center}
\begin{tabular}{l}
|\input{childdoc.def}|\\
|\childdocby{|\textit{main}|}|\\
\end{tabular}
\end{center}
%
Both forms have slightly different effects as described above.
The main file is prepared as usual, see \secref{sec:include}.

%%%%%%%%%%%%%%%%%%%%%%%%%%%%%%%%%%%%%%%%%%%%%%%%%%%%%%%%%%%%%%%%%%%%%%%%%%%%%%%%
\subsection{Legacy Detection}
\label{sec:detection}

The directive |\childdocmain| in the main file can detect
whether the complete document or merely a child is to be compiled
even without using the directive |\childdocof|.
This method is deprecated because it is less robust
and there is no compelling reason to use it;
it is merely provided for backward compatibility
and it may be removed in future versions.

If the detection mechanism is to be used,
it is mandatory to correctly specify
the filename of the main file as the argument of |\childdocmain|:
%
\begin{center}
\begin{tabular}{l}
|\input{childdoc.def}|\\
|\childdocmain{|\textit{main}|}|\\
\end{tabular}
\end{center}
%
If |\jobname| does not match the argument \textit{main} of |\childdocmain|,
it is assumed that |\jobname| points to the child file to be compiled.
When using |\childdocmain| with the main file specified as argument,
it suffices to start a child file
with just |\input{|\textit{main}|}|
without loading of the package and using |\childdocof|.
If instead all processing is done
with the appropriate \textsf{childdoc} directives,
the argument of \textit{main} of |\childdocmain| can be empty.

An alternative version of the command line processing described
in \secref{sec:commandline} using the detection mechanism reads:
%
\begin{center}
|... -jobname "|\textit{target}|" "|[\textit{flags}]%
[|\def\jobname{|\textit{dest}|}|]|\input{|\textit{main}|}"|
\end{center}

%%%%%%%%%%%%%%%%%%%%%%%%%%%%%%%%%%%%%%%%%%%%%%%%%%%%%%%%%%%%%%%%%%%%%%%%%%%%%%%%
\subsection{Manual Code}
\label{sec:manual}

In case one cannot be certain whether the definitions file |childdoc.def|
is installed on the target \TeX{} distribution
and one prefers not to ship it,
it is conceivable to paste a few relevant commands into the sources.

To that end, drop all statements |\input{childdoc.def}|
and perform the replacements as outlined below.
Instead of |\childdocmain{|\textit{main}|}| add the following code
to the top of the main file:
%
\begin{center}
\begin{tabular}{l}
|\||ifdefined\childdocname\endinput\||fi\newif\ifchilddoc|\\
|\edef\childdocname{\scantokens\expandafter{\jobname\noexpand}}|\\
|\def\childdocmain{|\textit{main}|}\||ifx\childdocmain\childdocname\||else|\\
|\childdoctrue\includeonly{\childdocname}\let\jobname\childdocmain\||fi|\\
\end{tabular}
\end{center}
%
Instead of |\childdocof{|\textit{main}|}| just include the main file
at the top of each child file:
%
\begin{center}
|\input{|\textit{main}|}|
\end{center}
%
A simple redirection |\childdocforward{|\textit{dest}|}| is achieved by:
%
\begin{center}
|\def\jobname{|\textit{dest}|}\input{\jobname}|
\end{center}
%
The redirection with prefix
|\childdocforwardprefix[|\textit{prefix}|]{|\textit{dest}|}|
is accomplished by:
%
\begin{center}
\begin{tabular}{l}
|{\edef\jobname{\scantokens\expandafter{\jobname\noexpand}}|\\
|\def\redirectjob |\textit{prefix}|#1~~~{\gdef\jobname{|\textit{dest}|#1}}|\\
|\expandafter\redirectjob\jobname~~~}\input{\jobname}|
\end{tabular}
\end{center}

In an alternative approach,
child documents can be compiled by a specific command line
without additional code or specific definitions:
%
\begin{center}
|... -jobname "|\textit{target}|" "|[\textit{flags}]%
|\includeonly{|\textit{dest}|}\input{|\textit{main}|}"|
\end{center}
%

%%%%%%%%%%%%%%%%%%%%%%%%%%%%%%%%%%%%%%%%%%%%%%%%%%%%%%%%%%%%%%%%%%%%%%%%%%%%%%%%
%%%%%%%%%%%%%%%%%%%%%%%%%%%%%%%%%%%%%%%%%%%%%%%%%%%%%%%%%%%%%%%%%%%%%%%%%%%%%%%%
\section{Information}

%%%%%%%%%%%%%%%%%%%%%%%%%%%%%%%%%%%%%%%%%%%%%%%%%%%%%%%%%%%%%%%%%%%%%%%%%%%%%%%%
\subsection{Copyright}

Copyright \copyright{} 2017--2018 Niklas Beisert

This work may be distributed and/or modified under the
conditions of the \LaTeX{} Project Public License, either version 1.3
of this license or (at your option) any later version.
The latest version of this license is in
  \url{http://www.latex-project.org/lppl.txt}
and version 1.3 or later is part of all distributions of \LaTeX{}
version 2005/12/01 or later.

This work has the LPPL maintenance status `maintained'.

The Current Maintainer of this work is Niklas Beisert.

This work consists of the files |README.txt|, |childdoc.ins| and |childdoc.dtx|
as well as the derived files |childdoc.def|, |cdocsamp.tex|
with |cdocsch1.tex|, |cdocsch2.tex|, |cdocspt3.tex|, |cdocspt4.tex|,
|cdocsdrf.tex|, |cdocsfn1.tex|, |cdocsfn2.tex|
as well as |childdoc.pdf|.

%%%%%%%%%%%%%%%%%%%%%%%%%%%%%%%%%%%%%%%%%%%%%%%%%%%%%%%%%%%%%%%%%%%%%%%%%%%%%%%%
\subsection{Files and Installation}

The package consists of the files:
%
\begin{center}
\begin{tabular}{ll}
    |README.txt|   & readme file \\
    |childdoc.ins| & installation file \\
    |childdoc.dtx| & source file \\
    |childdoc.def| & definition file \\
    |cdocsamp.tex| & sample main file \\
    |cdocsch1.tex| & sample include file \\
    |cdocsch2.tex| & sample include file \\
    |cdocspt3.tex| & sample part file \\
    |cdocspt4.tex| & sample part file \\
    |cdocsdrf.tex| & sample redirection file \\
    |cdocsfn1.tex| & sample redirection file \\
    |cdocsfn2.tex| & sample redirection file \\
    |childdoc.pdf| & manual
\end{tabular}
\end{center}
%
The distribution consists of the files
|README.txt|, |childdoc.ins| and |childdoc.dtx|.
%
\begin{itemize}
\item
Run (pdf)\LaTeX{} on |childdoc.dtx|
to compile the manual |childdoc.pdf| (this file).
\item
Run \LaTeX{} on |childdoc.ins| to create the definitions file |childdoc.def|
and the sample |cdocsamp.tex| with include files
|cdocsch1.tex|, |cdocsch2.tex|, |cdocspt3.tex|, |cdocspt4.tex|,
|cdocsdrf.tex|, |cdocsfn1.tex|, |cdocsfn2.tex|.
Then copy the file |childdoc.def| to an appropriate directory of your \LaTeX{}
distribution, e.g.\ \textit{texmf-root}|/tex/latex/childdoc|.
\end{itemize}

%%%%%%%%%%%%%%%%%%%%%%%%%%%%%%%%%%%%%%%%%%%%%%%%%%%%%%%%%%%%%%%%%%%%%%%%%%%%%%%%
\subsection{Related CTAN Packages}

There are several other packages which offer a similar functionality:
%
\begin{itemize}
\item
The packages
\href{http://ctan.org/pkg/docmute}{\textsf{docmute}},
\href{http://ctan.org/pkg/includex}{\textsf{includex}} and
\href{http://ctan.org/pkg/standalone}{\textsf{standalone}}
provide commands to include only the document body of
a child file thus allowing both files to be compiled individually.
\item
The packages \href{http://ctan.org/pkg/subdocs}{\textsf{subdocs}}
and \href{http://ctan.org/pkg/subfiles}{\textsf{subfiles}}
provide structures in which the main and child documents can be
encapsulated and allowing them to be compiled individually.
The inclusion mechanism is different from the conventional |\include|.
\item
The package \href{http://ctan.org/pkg/combine}{\textsf{combine}}
is an elaborate solution to combine several documents into one.
\end{itemize}
%
See also the CTAN topic \href{http://ctan.org/topic/subdocs}{\textsf{subdocs}}
for further related packages.
The present package differs from the above solutions in that
a document structure constructed with the conventional |\include| mechanism
just needs two extra commands at the top of every file
such that all constituent files can be compiled individually.

%%%%%%%%%%%%%%%%%%%%%%%%%%%%%%%%%%%%%%%%%%%%%%%%%%%%%%%%%%%%%%%%%%%%%%%%%%%%%%%%
%\subsection{Feature Suggestions}
%
%The following is a list of features which may be useful for future
%versions of this package:
%%
%\begin{itemize}
%\item
%\ldots
%\end{itemize}

%%%%%%%%%%%%%%%%%%%%%%%%%%%%%%%%%%%%%%%%%%%%%%%%%%%%%%%%%%%%%%%%%%%%%%%%%%%%%%%%
\subsection{Revision History}

%%%%%%%%%%%%%%%%%%%%%%%%%%%%%%%%%%%%%%%%
\paragraph{v2.0:} 2018/12/30

\begin{itemize}
\item
immediate forward processing
\item
added |\childdocby| mechanism
\item
manual restructured
\end{itemize}

%%%%%%%%%%%%%%%%%%%%%%%%%%%%%%%%%%%%%%%%
\paragraph{v1.6:} 2018/01/17

\begin{itemize}
\item
application for development of include files
\item
corrections to manual
\end{itemize}

%%%%%%%%%%%%%%%%%%%%%%%%%%%%%%%%%%%%%%%%
\paragraph{v1.5:} 2017/05/21

\begin{itemize}
\item
more complete structuring introduced
\item
|\childdocof| introduced
\item
|\childdoc| renamed to |\childdocmain|
\item
|\childredirect| renamed to |\childdocforward| and |\childdocforwardprefix|
and functionality expanded
\end{itemize}

%%%%%%%%%%%%%%%%%%%%%%%%%%%%%%%%%%%%%%%%
\paragraph{v1.0:} 2017/04/27

\begin{itemize}
\item
manual and install package
\item
first version published on CTAN
\end{itemize}

%%%%%%%%%%%%%%%%%%%%%%%%%%%%%%%%%%%%%%%%
\paragraph{v0.6:} 2017/04/26

\begin{itemize}
\item
redirection mechanism added
\end{itemize}

%%%%%%%%%%%%%%%%%%%%%%%%%%%%%%%%%%%%%%%%
\paragraph{v0.5:} 2017/04/26

\begin{itemize}
\item
functionality in definition file
\end{itemize}


%%%%%%%%%%%%%%%%%%%%%%%%%%%%%%%%%%%%%%%%%%%%%%%%%%%%%%%%%%%%%%%%%%%%%%%%%%%%%%%%
%%%%%%%%%%%%%%%%%%%%%%%%%%%%%%%%%%%%%%%%%%%%%%%%%%%%%%%%%%%%%%%%%%%%%%%%%%%%%%%%
%%%%%%%%%%%%%%%%%%%%%%%%%%%%%%%%%%%%%%%%%%%%%%%%%%%%%%%%%%%%%%%%%%%%%%%%%%%%%%%%
\appendix

\settowidth\MacroIndent{\rmfamily\scriptsize 000\ }

 \DocInput{childdoc.dtx}

\end{document}
%</driver>
% \fi
%
% %%%%%%%%%%%%%%%%%%%%%%%%%%%%%%%%%%%%%%%%%%%%%%%%%%%%%%%%%%%%%%%%%%%%%%%%%%%%%%
% %%%%%%%%%%%%%%%%%%%%%%%%%%%%%%%%%%%%%%%%%%%%%%%%%%%%%%%%%%%%%%%%%%%%%%%%%%%%%%
% \section{Sample}
%\iffalse
%<*samplemain>
%\fi
%
% The following presents a sample document
% with two chapters, two parts, a title page,
% a compile flag as well as three forwarding files to set the flag.
% It consists of eight |.tex| files:
% \begin{center}
% \begin{tabular}{ll}
% |cdocsamp.tex|&main file\\
% |cdocsch1.tex|&include file for chapter 1\\
% |cdocsch2.tex|&include file for chapter 2\\
% |cdocspt3.tex|&include file for part 3\\
% |cdocspt4.tex|&include file for part 4\\
% |cdocsdrf.tex|&forwarding file for main file in draft mode\\
% |cdocsfi1.tex|&forwarding file for final version of chapter 1\\
% |cdocsfi2.tex|&forwarding file for final version of chapter 2\\
% \end{tabular}
% \end{center}
% Each of the eight files can be compiled directly by the \LaTeX{} compiler.
%
% %%%%%%%%%%%%%%%%%%%%%%%%%%%%%%%%%%%%%%
% \paragraph{Main File.}
%
% The main file is called |cdocsamp.tex|.
%
% Load the \textsf{childdoc} definitions and
% declare the filename for the main document:
%    \begin{macrocode}
\input{childdoc.def}
\childdocmain{}
%    \end{macrocode}

% Optional override for |\version| flag:
%    \begin{macrocode}
%%\ifchilddoc\else\providecommand{\version}{draft}\fi
%    \end{macrocode}

% Define the default values for the |\version| flag
% (|final| for the main file and |draft| for childs):
%    \begin{macrocode}
\ifchilddoc
\providecommand{\version}{draft}
\else
\providecommand{\version}{final}
\fi
%    \end{macrocode}

% Load the standard document class:
%    \begin{macrocode}
\documentclass[12pt]{article}
%    \end{macrocode}

% Start the document body:
%    \begin{macrocode}
\begin{document}
%    \end{macrocode}

% Declare a title page.
% Print title, part of document being processed and version flag:
%    \begin{macrocode}
\addtocounter{page}{-1}
\begin{center}
{\LARGE\bfseries{}childdoc example\par}
\vspace{1cm}
\ifchilddoc
\ifchilddocmanual part\else chapter\fi:
`\childdocname' of `\childdocjob'\par
\else
main document: `\childdocjob'\par
\fi
version: \version\par
\end{center}
\newpage
%    \end{macrocode}

% Manually include selected file,
% otherwise process as usual:
%    \begin{macrocode}
\ifchilddocmanual
\section*{part `\childdocname'}
\input{\childdocname}
\else
%    \end{macrocode}

% Include the two chapters:
%    \begin{macrocode}
\include{cdocsch1}
\include{cdocsch2}
%    \end{macrocode}

% Include the two parts unless only chapters should be displayed:
%    \begin{macrocode}
\ifchilddoc\else
\section{part three}
\input{cdocspt3}
\section{part four}
\input{cdocspt4}
\fi
%    \end{macrocode}

% Process as usual until here:
%    \begin{macrocode}
\fi
%    \end{macrocode}

% End of document body:
%    \begin{macrocode}
\end{document}
%    \end{macrocode}
%\iffalse
%</samplemain>
%\fi
%
% %%%%%%%%%%%%%%%%%%%%%%%%%%%%%%%%%%%%%%
% \paragraph{Chapter Include Files.}
%
% The include files are called |cdocsch1.tex| and |cdocsch2.tex|.
%
%\iffalse
%<*samplechap1|samplechap2>
%\fi

% Optional override for |\version| flag:
%    \begin{macrocode}
%%\providecommand{\version}{final}
%    \end{macrocode}

% Include the main document:
%    \begin{macrocode}
\input{childdoc.def}
\childdocof{cdocsamp}
%    \end{macrocode}

%\iffalse
%</samplechap1|samplechap2>
%\fi
%
%\iffalse
%<*samplechap1>
%\fi
% Some text for chapter 1:
%    \begin{macrocode}
\section{one}
some text in chapter one
%    \end{macrocode}

%\iffalse
%</samplechap1>
%\fi
% Some text for chapter 2:
%\iffalse
%<*samplechap2>
%\fi
%    \begin{macrocode}
\section{two}
more text in chapter two
%    \end{macrocode}

%\iffalse
%</samplechap2>
%\fi
%
% %%%%%%%%%%%%%%%%%%%%%%%%%%%%%%%%%%%%%%
% \paragraph{Part Include Files.}
%
% The include files are called |cdocspt3.tex| and |cdocspt4.tex|.
%
%\iffalse
%<*samplepart3|samplepart4>
%\fi

% Optional override for |\version| flag:
%    \begin{macrocode}
%%\providecommand{\version}{final}
%    \end{macrocode}

% Include the main document:
%    \begin{macrocode}
\input{childdoc.def}
\childdocby{cdocsamp}
%    \end{macrocode}

%\iffalse
%</samplepart3|samplepart4>
%\fi
%
%\iffalse
%<*samplepart3>
%\fi
% Some text for part 3:
%    \begin{macrocode}
some text in part three
%    \end{macrocode}

%\iffalse
%</samplepart3>
%\fi
% Some text for part 4:
%\iffalse
%<*samplepart4>
%\fi
%    \begin{macrocode}
more text in part four
%    \end{macrocode}

%\iffalse
%</samplepart4>
%\fi
%
% %%%%%%%%%%%%%%%%%%%%%%%%%%%%%%%%%%%%%%
% \paragraph{Forwarding for a Complete Draft.}
%
% The following forwarding file |cdocsdrf.tex|
% compiles the main document in draft mode:
%\iffalse
%<*sampledraft>
%\fi
%    \begin{macrocode}
\def\version{draft}
\input{childdoc.def}
\childdocforward{cdocsamp}
%    \end{macrocode}

%\iffalse
%</sampledraft>
%\fi
%
% %%%%%%%%%%%%%%%%%%%%%%%%%%%%%%%%%%%%%%
% \paragraph{Forwarding for Final Version of the Chapters.}
%
% The following forwarding files |cdocsfn1.tex| and |cdocsfn2.tex|
% (with identical content)
% compile the final versions of the child documents
% |cdocsch1.tex| and |cdocsch2.tex|, respectively:
%\iffalse
%<*samplefinal>
%\fi
%    \begin{macrocode}
\def\version{final}
\input{childdoc.def}
\childdocforwardprefix[cdocsamp]{cdocsfn}{cdocsch}
%    \end{macrocode}

%\iffalse
%</samplefinal>
%\fi
%
% %%%%%%%%%%%%%%%%%%%%%%%%%%%%%%%%%%%%%%
% \paragraph{Command Line Processing.}
%
% The following three command lines generate the output files
% |cdocscld|, |cdocscl1| and |cdocscl2|
% which should be identical to
% |cdocsdrf|, |cdocsch1| and |cdocsfn2|, respectively:
% \begin{center}
% \begin{tabular}{l}
% |latex -jobname cdocscld \|\\
% |  "\def\version{draft}\input{childdoc.def}\childdocforward{cdocsamp}"|\\
% |latex -jobname cdocscl1 \|\\
% |  "\input{childdoc.def}\childdocforward[cdocsamp]{cdocsch1}"|\\
% |latex -jobname cdocscl2 \|\\
% |  "\def\version{final}\input{childdoc.def}\childdocforward{cdocsch2}"|
% \end{tabular}
% \end{center}
% Note that the trailing backslash on each first line
% merely continues the input to the second line
% (for convenient cut ant paste).
% Furthermore, the command |latex| can be replaced by any
% of its alternative versions such as |pdflatex|.
%
% %%%%%%%%%%%%%%%%%%%%%%%%%%%%%%%%%%%%%%%%%%%%%%%%%%%%%%%%%%%%%%%%%%%%%%%%%%%%%%
% %%%%%%%%%%%%%%%%%%%%%%%%%%%%%%%%%%%%%%%%%%%%%%%%%%%%%%%%%%%%%%%%%%%%%%%%%%%%%%
% \section{Implementation}
%\iffalse
%<*package>
%\fi
%
% This section describes the definitions file |childdoc.def|.

% The definitions cannot be loaded using |\usepackage| or |\RequirePackage|
% which has a mechanism to prevent loading a style file more than once.
% When loading the definitions by means of |\input|
% multiple instances have to be prevented manually:
%\iffalse
%This code needs to be before the `\ProvidesFile' directive
%which is defined at the beginning of this file.
%Therefore it is also placed there and commented out here.
%</package>
%<*discard>
%\fi
%    \begin{macrocode}
\ifdefined\childdocmain\endinput\fi
%    \end{macrocode}
%\iffalse
%</discard>
%<*package>
%\fi
%
% \macro{\ifchilddoc}
% \macro{\ifchilddocmanual}
% The conditional |\ifchilddoc| tells whether a
% child (true) or main (false) document is being compiled.
% The conditional |\ifchilddocmanual| tells whether
% the |\includeonly| mechanism is used (false) or
% the selection of child files must be performed manually (true).
% The definitions initialise to false:
%    \begin{macrocode}
\newif\ifchilddoc
\newif\ifchilddocmanual
%    \end{macrocode}

% \macro{\childdocname}
% \macro{\childdocjob}
% The macro |\childdocname| stores the name of the main document
% to be compiled. The macro |\childdocjob| stores the name of
% the document on which the \LaTeX{} compiler was originally invoked.
% The content of |\jobname| cannot be compared
% to filenames specified in the source due to different catcodes.
% The following code rescans |\jobname|, stores the result
% in |\childdocname| and saves a copy in |\childdocjob|:
%    \begin{macrocode}
\edef\childdocname{\scantokens\expandafter{\jobname\noexpand}}
\let\childdocjob\childdocname
%    \end{macrocode}

% \macro{\childdocdisable}
% The macro |\childdocdisable| prevents the main file
% from being processed more than once.
% At this stage, the main document command |\childdocmain|
% is assumed to be called once again where it should do nothing.
% Any subsequent call to it should prevent
% a secondary processing of the main document
% It overwrites the forwarding commands
% |\childdocof| and |\childdocforward|
% with empty macros to prevent further inclusions of the main document:
%    \begin{macrocode}
\newcommand{\childdocdisable}
{
  \renewcommand{\childdocmain}[1]{\renewcommand{\childdocmain}[1]{\endinput}}
  \renewcommand{\childdocof}[1]{}
  \renewcommand{\childdocby}[2][]{}
  \renewcommand{\childdocforward}[2][]{}
  \renewcommand{\childdocdisable}{}
}
%    \end{macrocode}

% \macro{\childdocmain}
% The macro |\childdocmain| is to be called at the top of the main file
% with nothing or the main filename (without extension) as argument.
% First, it breaks loops.
% If the argument is not empty and does not match |\childdocname|
% (which is set by the first inclusion of |childdoc.def|),
% |\ifchilddoc| is set to true, |\includeonly| is applied to the child file
% and |\jobname| is set to the main file
% (for proper handling of |.aux| files):
%    \begin{macrocode}
\newcommand{\childdocmain}[1]
{
  \childdocdisable\childdocmain{}
  \if?#1?\else
    \begingroup
      \def\childdoctmp{#1}
      \ifx\childdoctmp\childdocname
        \def\childdoctmp{}
      \else
        \def\childdoctmp
        {
          \childdoctrue
          \includeonly{\childdocname}
          \def\childdocjob{#1}
          \def\jobname{#1}
        }
      \fi
      \expandafter
    \endgroup
    \childdoctmp
  \fi
}
%    \end{macrocode}

% \macro{\childdocof}
% The command |\childdocof| redirects
% compilation to the main file |#1|.
%    \begin{macrocode}
\newcommand{\childdocof}[1]
{
  \childdocdisable
  \childdoctrue
  \includeonly{\childdocname}
  \def\jobname{#1}
  \def\childdocjob{#1}
  \input{#1}
}
%    \end{macrocode}

% \macro{\childdocby}
% The command |\childdocby| ....
%    \begin{macrocode}
\newcommand{\childdocby}[2][]
{
  \childdocdisable
  \childdoctrue
  \childdocmanualtrue
  \if?#1?\else
    \def\jobname{#2}
  \fi
  \def\childdocjob{#2}
  \input{#2}
  \endinput
}
%    \end{macrocode}

% \macro{\childdocforward}
% The command |\childdocforward| redirects
% compilation to the main file or
% (if the optional argument is given) a child file.
% Parameters are set as if the main file
% or a child file starting with |\childdocof| was compiled.
% Then compilation is handed over to the main file:
%    \begin{macrocode}
\newcommand{\childdocforward}[2][]
{
  \begingroup
    \if?#1?
      \def\childdoctmp
      {
        \def\childdocname{#2}
        \def\childdocjob{#2}
        \def\jobname{#2}
        \input{#2}
        \endinput
      }
    \else
      \def\childdoctmp
      {
        \childdocdisable
        \def\childdocname{#2}
        \childdoctrue
        \includeonly{#2}
        \def\childdocjob{#1}
        \def\jobname{#1}
        \input{#1}
        \endinput
      }
    \fi
    \expandafter
  \endgroup
  \childdoctmp
}
%    \end{macrocode}

% \macro{\childdocforwardprefix}
% The command |\childdocforwardprefix| redirects
% compilation to the main or a child file by means of a pattern.
% The prefix |#1| in the current filename is replaced by |#2|
% and the suffix of the current filename is kept
% (it is assumed that the filename does not contain the substring `|~~~|'
% which is used as a delimiter).
% Compilation is handed over to the new file by |\childdocforward|:
%    \begin{macrocode}
\newcommand{\childdocforwardprefix}[3][]
{
  \begingroup
    \def\childdocextract #2##1~~~{\def\childdoctmp{\childdocforward[#1]{#3##1}}}
    \expandafter\childdocextract\childdocname~~~
    \expandafter
  \endgroup
  \childdoctmp
}
%    \end{macrocode}

% \macro{\childdoc}
% The deprecated macro |\childdoc| is a legacy version of |\childdocmain|:
%    \begin{macrocode}
\newcommand{\childdoc}{\childdocmain}
%    \end{macrocode}

% \macro{\childdocredirect}
% The deprecated macro |\childdocredirect| is a legacy version
% of |\childdocforward| and |\childdocforwardprefix|:
%    \begin{macrocode}
\newcommand{\childdocredirect}[2][]
{
  \begingroup
    \if?#1?
      \def\childdoctmp{\childdocforward{#2}}
    \else
      \def\childdoctmp{\childdocforwardprefix{#1}{#2}}
    \fi
    \expandafter
  \endgroup
  \childdoctmp
}
%    \end{macrocode}

%\iffalse
%</package>
%\fi
%
\endinput
\childdocforward{cdocsch2}"|
% \end{tabular}
% \end{center}
% Note that the trailing backslash on each first line
% merely continues the input to the second line
% (for convenient cut ant paste).
% Furthermore, the command |latex| can be replaced by any
% of its alternative versions such as |pdflatex|.
%
% %%%%%%%%%%%%%%%%%%%%%%%%%%%%%%%%%%%%%%%%%%%%%%%%%%%%%%%%%%%%%%%%%%%%%%%%%%%%%%
% %%%%%%%%%%%%%%%%%%%%%%%%%%%%%%%%%%%%%%%%%%%%%%%%%%%%%%%%%%%%%%%%%%%%%%%%%%%%%%
% \section{Implementation}
%\iffalse
%<*package>
%\fi
%
% This section describes the definitions file |childdoc.def|.

% The definitions cannot be loaded using |\usepackage| or |\RequirePackage|
% which has a mechanism to prevent loading a style file more than once.
% When loading the definitions by means of |\input|
% multiple instances have to be prevented manually:
%\iffalse
%This code needs to be before the `\ProvidesFile' directive
%which is defined at the beginning of this file.
%Therefore it is also placed there and commented out here.
%</package>
%<*discard>
%\fi
%    \begin{macrocode}
\ifdefined\childdocmain\endinput\fi
%    \end{macrocode}
%\iffalse
%</discard>
%<*package>
%\fi
%
% \macro{\ifchilddoc}
% \macro{\ifchilddocmanual}
% The conditional |\ifchilddoc| tells whether a
% child (true) or main (false) document is being compiled.
% The conditional |\ifchilddocmanual| tells whether
% the |\includeonly| mechanism is used (false) or
% the selection of child files must be performed manually (true).
% The definitions initialise to false:
%    \begin{macrocode}
\newif\ifchilddoc
\newif\ifchilddocmanual
%    \end{macrocode}

% \macro{\childdocname}
% \macro{\childdocjob}
% The macro |\childdocname| stores the name of the main document
% to be compiled. The macro |\childdocjob| stores the name of
% the document on which the \LaTeX{} compiler was originally invoked.
% The content of |\jobname| cannot be compared
% to filenames specified in the source due to different catcodes.
% The following code rescans |\jobname|, stores the result
% in |\childdocname| and saves a copy in |\childdocjob|:
%    \begin{macrocode}
\edef\childdocname{\scantokens\expandafter{\jobname\noexpand}}
\let\childdocjob\childdocname
%    \end{macrocode}

% \macro{\childdocdisable}
% The macro |\childdocdisable| prevents the main file
% from being processed more than once.
% At this stage, the main document command |\childdocmain|
% is assumed to be called once again where it should do nothing.
% Any subsequent call to it should prevent
% a secondary processing of the main document
% It overwrites the forwarding commands
% |\childdocof| and |\childdocforward|
% with empty macros to prevent further inclusions of the main document:
%    \begin{macrocode}
\newcommand{\childdocdisable}
{
  \renewcommand{\childdocmain}[1]{\renewcommand{\childdocmain}[1]{\endinput}}
  \renewcommand{\childdocof}[1]{}
  \renewcommand{\childdocby}[2][]{}
  \renewcommand{\childdocforward}[2][]{}
  \renewcommand{\childdocdisable}{}
}
%    \end{macrocode}

% \macro{\childdocmain}
% The macro |\childdocmain| is to be called at the top of the main file
% with nothing or the main filename (without extension) as argument.
% First, it breaks loops.
% If the argument is not empty and does not match |\childdocname|
% (which is set by the first inclusion of |childdoc.def|),
% |\ifchilddoc| is set to true, |\includeonly| is applied to the child file
% and |\jobname| is set to the main file
% (for proper handling of |.aux| files):
%    \begin{macrocode}
\newcommand{\childdocmain}[1]
{
  \childdocdisable\childdocmain{}
  \if?#1?\else
    \begingroup
      \def\childdoctmp{#1}
      \ifx\childdoctmp\childdocname
        \def\childdoctmp{}
      \else
        \def\childdoctmp
        {
          \childdoctrue
          \includeonly{\childdocname}
          \def\childdocjob{#1}
          \def\jobname{#1}
        }
      \fi
      \expandafter
    \endgroup
    \childdoctmp
  \fi
}
%    \end{macrocode}

% \macro{\childdocof}
% The command |\childdocof| redirects
% compilation to the main file |#1|.
%    \begin{macrocode}
\newcommand{\childdocof}[1]
{
  \childdocdisable
  \childdoctrue
  \includeonly{\childdocname}
  \def\jobname{#1}
  \def\childdocjob{#1}
  \input{#1}
}
%    \end{macrocode}

% \macro{\childdocby}
% The command |\childdocby| ....
%    \begin{macrocode}
\newcommand{\childdocby}[2][]
{
  \childdocdisable
  \childdoctrue
  \childdocmanualtrue
  \if?#1?\else
    \def\jobname{#2}
  \fi
  \def\childdocjob{#2}
  \input{#2}
  \endinput
}
%    \end{macrocode}

% \macro{\childdocforward}
% The command |\childdocforward| redirects
% compilation to the main file or
% (if the optional argument is given) a child file.
% Parameters are set as if the main file
% or a child file starting with |\childdocof| was compiled.
% Then compilation is handed over to the main file:
%    \begin{macrocode}
\newcommand{\childdocforward}[2][]
{
  \begingroup
    \if?#1?
      \def\childdoctmp
      {
        \def\childdocname{#2}
        \def\childdocjob{#2}
        \def\jobname{#2}
        \input{#2}
        \endinput
      }
    \else
      \def\childdoctmp
      {
        \childdocdisable
        \def\childdocname{#2}
        \childdoctrue
        \includeonly{#2}
        \def\childdocjob{#1}
        \def\jobname{#1}
        \input{#1}
        \endinput
      }
    \fi
    \expandafter
  \endgroup
  \childdoctmp
}
%    \end{macrocode}

% \macro{\childdocforwardprefix}
% The command |\childdocforwardprefix| redirects
% compilation to the main or a child file by means of a pattern.
% The prefix |#1| in the current filename is replaced by |#2|
% and the suffix of the current filename is kept
% (it is assumed that the filename does not contain the substring `|~~~|'
% which is used as a delimiter).
% Compilation is handed over to the new file by |\childdocforward|:
%    \begin{macrocode}
\newcommand{\childdocforwardprefix}[3][]
{
  \begingroup
    \def\childdocextract #2##1~~~{\def\childdoctmp{\childdocforward[#1]{#3##1}}}
    \expandafter\childdocextract\childdocname~~~
    \expandafter
  \endgroup
  \childdoctmp
}
%    \end{macrocode}

% \macro{\childdoc}
% The deprecated macro |\childdoc| is a legacy version of |\childdocmain|:
%    \begin{macrocode}
\newcommand{\childdoc}{\childdocmain}
%    \end{macrocode}

% \macro{\childdocredirect}
% The deprecated macro |\childdocredirect| is a legacy version
% of |\childdocforward| and |\childdocforwardprefix|:
%    \begin{macrocode}
\newcommand{\childdocredirect}[2][]
{
  \begingroup
    \if?#1?
      \def\childdoctmp{\childdocforward{#2}}
    \else
      \def\childdoctmp{\childdocforwardprefix{#1}{#2}}
    \fi
    \expandafter
  \endgroup
  \childdoctmp
}
%    \end{macrocode}

%\iffalse
%</package>
%\fi
%
\endinput
|\\
|\childdocforward{|\textit{main}|}|
\end{tabular}
\end{center}
%
Likewise, the following files |final|\textit{nn}|.tex|
compile the final version of the child document
|child|\textit{nn}|.tex|:
%
\begin{center}
\begin{tabular}{l}
|\def\version{final}|\\
|% \iffalse
%
% childdoc.dtx Copyright (C) 2017-2018 Niklas Beisert
%
% This work may be distributed and/or modified under the
% conditions of the LaTeX Project Public License, either version 1.3
% of this license or (at your option) any later version.
% The latest version of this license is in
%   http://www.latex-project.org/lppl.txt
% and version 1.3 or later is part of all distributions of LaTeX
% version 2005/12/01 or later.
%
% This work has the LPPL maintenance status `maintained'.
%
% The Current Maintainer of this work is Niklas Beisert.
%
% This work consists of the files childdoc.dtx and childdoc.ins
% and the derived files childdoc.def and cdocsamp.tex with
% cdocsch1.tex, cdocsch2.tex, cdocsdrf.tex, cdocsfn1.tex, cdocsfn2.tex.
%
%<package>\ifdefined\childdocmain\endinput\fi
%<package>\ProvidesFile{childdoc.def}[2018/12/30 v2.0 child document driver]
%<samplemain>\ProvidesFile{cdocsamp.tex}[2018/12/30 v2.0 sample for childdoc]
%<*driver>
%\ProvidesFile{childdoc.drv}[2018/12/30 v2.0 childdoc reference manual file]
\PassOptionsToClass{10pt,a4paper}{article}
\documentclass{ltxdoc}

\usepackage[margin=35mm]{geometry}
\usepackage{hyperref}
\usepackage{hyperxmp}
\usepackage[usenames]{color}

\hypersetup{colorlinks=true}
\hypersetup{pdfstartview=FitH}
\hypersetup{pdfpagemode=UseNone}
\hypersetup{pdfsource={}}
\hypersetup{pdflang={en-UK}}
\hypersetup{pdfcopyright={Copyright 2017-2018 Niklas Beisert.
  This work may be distributed and/or modified under the
  conditions of the LaTeX Project Public License, either version 1.3
  of this license or (at your option) any later version.}}
\hypersetup{pdflicenseurl={http://www.latex-project.org/lppl.txt}}
\hypersetup{pdfcontactaddress={ETH Zurich, ITP, HIT K,
  Wolfgang-Pauli-Strasse 27}}
\hypersetup{pdfcontactpostcode={8093}}
\hypersetup{pdfcontactcity={Zurich}}
\hypersetup{pdfcontactcountry={Switzerland}}
\hypersetup{pdfcontactemail={nbeisert@itp.phys.ethz.ch}}
\hypersetup{pdfcontacturl={http://people.phys.ethz.ch/\xmptilde nbeisert/}}

\newcommand{\secref}[1]{\hyperref[#1]{section \ref*{#1}}}

\parskip1ex
\parindent0pt
\let\olditemize\itemize
\def\itemize{\olditemize\parskip0pt}

\begin{document}

\title{The \textsf{childdoc} Package}
\hypersetup{pdftitle={The childdoc Package}}
\author{Niklas Beisert\\[2ex]
  Institut f\"ur Theoretische Physik\\
  Eidgen\"ossische Technische Hochschule Z\"urich\\
  Wolfgang-Pauli-Strasse 27, 8093 Z\"urich, Switzerland\\[1ex]
  \href{mailto:nbeisert@itp.phys.ethz.ch}
  {\texttt{nbeisert@itp.phys.ethz.ch}}}
\hypersetup{pdfauthor={Niklas Beisert}}
\hypersetup{pdfsubject={Manual for the LaTeX2e Package childdoc}}
\date{30 December 2018, \textsf{v2.0}}
\maketitle

\begin{abstract}\noindent
\textsf{childdoc} is a \LaTeXe{} package
that enables the direct compilation
of document sections included by |\include|
to individual files.
\end{abstract}

\begingroup
\parskip0ex
\tableofcontents
\endgroup

%%%%%%%%%%%%%%%%%%%%%%%%%%%%%%%%%%%%%%%%%%%%%%%%%%%%%%%%%%%%%%%%%%%%%%%%%%%%%%%%
%%%%%%%%%%%%%%%%%%%%%%%%%%%%%%%%%%%%%%%%%%%%%%%%%%%%%%%%%%%%%%%%%%%%%%%%%%%%%%%%
\section{Introduction}

\LaTeX{} provides a mechanism to structure a large document (such as a book)
into a main file and several child files (containing the chapters)
using the |\include| command.
This mechanism is beneficial for documents
which span hundreds of pages in order to
make the source file(s) more manageable.
Moreover, compilation can be restricted to
selected child files by means of the |\includeonly| command.
The latter feature can be used to reduce the compilation time while editing
(this was significantly more useful in the earlier days of \LaTeX{})
or to generate a smaller document which is easier to navigate.
Another application of |\includeonly| is to generate
documents consisting of selected parts of the complete document.

However, there are a few drawbacks of the plain |\include| mechanism:
\begin{itemize}
\item
The child files cannot be compiled on their own,
they can only be compiled via the main file.
A naive editing environment
(such as a text editor with an option
to have the current file processed by \LaTeX)
may require one to switch to the main file before compiling;
attempting to compile the child file produces errors.
\item
The main file must be modified (each time)
to adjust the |\includeonly| command
to the present needs. This easily leaves the main file in a messy state.
\item
The generated document will always carry the filename
of the main document. This is inconvenient if
several child files are to be compiled and
to be kept for distribution.
\end{itemize}

The present package provides a simple interface
to make child files individually compilable by \LaTeX{}.
Compiling a child file then has the same effect as compiling
the main file with an |\includeonly| command
to select the appropriate child.
Moreover the generated document will carry the name of the child
rather than the main file.
This resolves all three above issues.

This feature is meant to make the editing of books,
thesis documents and lecture notes somewhat more convenient.
However, the package can also be used efficiently for
composing a series of documents (such as exercise sheets)
which are typically distributed individually.
It then assists the author in generating the individual documents
(potentially in different versions)
as well as a document containing the collected series.
Another application is in developing style files
or other kinds of included material
where compilation of the style file could redirect
to a sample or test file.

%%%%%%%%%%%%%%%%%%%%%%%%%%%%%%%%%%%%%%%%%%%%%%%%%%%%%%%%%%%%%%%%%%%%%%%%%%%%%%%%
%%%%%%%%%%%%%%%%%%%%%%%%%%%%%%%%%%%%%%%%%%%%%%%%%%%%%%%%%%%%%%%%%%%%%%%%%%%%%%%%
\section{Usage}

First of all, the package \textsf{childdoc} is \emph{not} a standard
\LaTeXe{} |.sty| style file! Therefore it needs to be invoked in
a non-standard way.

%%%%%%%%%%%%%%%%%%%%%%%%%%%%%%%%%%%%%%%%%%%%%%%%%%%%%%%%%%%%%%%%%%%%%%%%%%%%%%%%
\subsection{Included Files}
\label{sec:include}

%%%%%%%%%%%%%%%%%%%%%%%%%%%%%%%%%%%%%%%%
\DescribeMacro{\childdocmain}
To use the package, add the commands
\begin{center}
\begin{tabular}{l}
|% \iffalse
%
% childdoc.dtx Copyright (C) 2017-2018 Niklas Beisert
%
% This work may be distributed and/or modified under the
% conditions of the LaTeX Project Public License, either version 1.3
% of this license or (at your option) any later version.
% The latest version of this license is in
%   http://www.latex-project.org/lppl.txt
% and version 1.3 or later is part of all distributions of LaTeX
% version 2005/12/01 or later.
%
% This work has the LPPL maintenance status `maintained'.
%
% The Current Maintainer of this work is Niklas Beisert.
%
% This work consists of the files childdoc.dtx and childdoc.ins
% and the derived files childdoc.def and cdocsamp.tex with
% cdocsch1.tex, cdocsch2.tex, cdocsdrf.tex, cdocsfn1.tex, cdocsfn2.tex.
%
%<package>\ifdefined\childdocmain\endinput\fi
%<package>\ProvidesFile{childdoc.def}[2018/12/30 v2.0 child document driver]
%<samplemain>\ProvidesFile{cdocsamp.tex}[2018/12/30 v2.0 sample for childdoc]
%<*driver>
%\ProvidesFile{childdoc.drv}[2018/12/30 v2.0 childdoc reference manual file]
\PassOptionsToClass{10pt,a4paper}{article}
\documentclass{ltxdoc}

\usepackage[margin=35mm]{geometry}
\usepackage{hyperref}
\usepackage{hyperxmp}
\usepackage[usenames]{color}

\hypersetup{colorlinks=true}
\hypersetup{pdfstartview=FitH}
\hypersetup{pdfpagemode=UseNone}
\hypersetup{pdfsource={}}
\hypersetup{pdflang={en-UK}}
\hypersetup{pdfcopyright={Copyright 2017-2018 Niklas Beisert.
  This work may be distributed and/or modified under the
  conditions of the LaTeX Project Public License, either version 1.3
  of this license or (at your option) any later version.}}
\hypersetup{pdflicenseurl={http://www.latex-project.org/lppl.txt}}
\hypersetup{pdfcontactaddress={ETH Zurich, ITP, HIT K,
  Wolfgang-Pauli-Strasse 27}}
\hypersetup{pdfcontactpostcode={8093}}
\hypersetup{pdfcontactcity={Zurich}}
\hypersetup{pdfcontactcountry={Switzerland}}
\hypersetup{pdfcontactemail={nbeisert@itp.phys.ethz.ch}}
\hypersetup{pdfcontacturl={http://people.phys.ethz.ch/\xmptilde nbeisert/}}

\newcommand{\secref}[1]{\hyperref[#1]{section \ref*{#1}}}

\parskip1ex
\parindent0pt
\let\olditemize\itemize
\def\itemize{\olditemize\parskip0pt}

\begin{document}

\title{The \textsf{childdoc} Package}
\hypersetup{pdftitle={The childdoc Package}}
\author{Niklas Beisert\\[2ex]
  Institut f\"ur Theoretische Physik\\
  Eidgen\"ossische Technische Hochschule Z\"urich\\
  Wolfgang-Pauli-Strasse 27, 8093 Z\"urich, Switzerland\\[1ex]
  \href{mailto:nbeisert@itp.phys.ethz.ch}
  {\texttt{nbeisert@itp.phys.ethz.ch}}}
\hypersetup{pdfauthor={Niklas Beisert}}
\hypersetup{pdfsubject={Manual for the LaTeX2e Package childdoc}}
\date{30 December 2018, \textsf{v2.0}}
\maketitle

\begin{abstract}\noindent
\textsf{childdoc} is a \LaTeXe{} package
that enables the direct compilation
of document sections included by |\include|
to individual files.
\end{abstract}

\begingroup
\parskip0ex
\tableofcontents
\endgroup

%%%%%%%%%%%%%%%%%%%%%%%%%%%%%%%%%%%%%%%%%%%%%%%%%%%%%%%%%%%%%%%%%%%%%%%%%%%%%%%%
%%%%%%%%%%%%%%%%%%%%%%%%%%%%%%%%%%%%%%%%%%%%%%%%%%%%%%%%%%%%%%%%%%%%%%%%%%%%%%%%
\section{Introduction}

\LaTeX{} provides a mechanism to structure a large document (such as a book)
into a main file and several child files (containing the chapters)
using the |\include| command.
This mechanism is beneficial for documents
which span hundreds of pages in order to
make the source file(s) more manageable.
Moreover, compilation can be restricted to
selected child files by means of the |\includeonly| command.
The latter feature can be used to reduce the compilation time while editing
(this was significantly more useful in the earlier days of \LaTeX{})
or to generate a smaller document which is easier to navigate.
Another application of |\includeonly| is to generate
documents consisting of selected parts of the complete document.

However, there are a few drawbacks of the plain |\include| mechanism:
\begin{itemize}
\item
The child files cannot be compiled on their own,
they can only be compiled via the main file.
A naive editing environment
(such as a text editor with an option
to have the current file processed by \LaTeX)
may require one to switch to the main file before compiling;
attempting to compile the child file produces errors.
\item
The main file must be modified (each time)
to adjust the |\includeonly| command
to the present needs. This easily leaves the main file in a messy state.
\item
The generated document will always carry the filename
of the main document. This is inconvenient if
several child files are to be compiled and
to be kept for distribution.
\end{itemize}

The present package provides a simple interface
to make child files individually compilable by \LaTeX{}.
Compiling a child file then has the same effect as compiling
the main file with an |\includeonly| command
to select the appropriate child.
Moreover the generated document will carry the name of the child
rather than the main file.
This resolves all three above issues.

This feature is meant to make the editing of books,
thesis documents and lecture notes somewhat more convenient.
However, the package can also be used efficiently for
composing a series of documents (such as exercise sheets)
which are typically distributed individually.
It then assists the author in generating the individual documents
(potentially in different versions)
as well as a document containing the collected series.
Another application is in developing style files
or other kinds of included material
where compilation of the style file could redirect
to a sample or test file.

%%%%%%%%%%%%%%%%%%%%%%%%%%%%%%%%%%%%%%%%%%%%%%%%%%%%%%%%%%%%%%%%%%%%%%%%%%%%%%%%
%%%%%%%%%%%%%%%%%%%%%%%%%%%%%%%%%%%%%%%%%%%%%%%%%%%%%%%%%%%%%%%%%%%%%%%%%%%%%%%%
\section{Usage}

First of all, the package \textsf{childdoc} is \emph{not} a standard
\LaTeXe{} |.sty| style file! Therefore it needs to be invoked in
a non-standard way.

%%%%%%%%%%%%%%%%%%%%%%%%%%%%%%%%%%%%%%%%%%%%%%%%%%%%%%%%%%%%%%%%%%%%%%%%%%%%%%%%
\subsection{Included Files}
\label{sec:include}

%%%%%%%%%%%%%%%%%%%%%%%%%%%%%%%%%%%%%%%%
\DescribeMacro{\childdocmain}
To use the package, add the commands
\begin{center}
\begin{tabular}{l}
|\input{childdoc.def}|\\
|\childdocmain{}|\\
\end{tabular}
\end{center}
at the very top of the main \LaTeX{} file,
in particular \emph{before} the |\documentclass| statement!
The argument of |\childdocmain| should be left empty
(but it must be present).

%%%%%%%%%%%%%%%%%%%%%%%%%%%%%%%%%%%%%%%%
\DescribeMacro{\childdocof}
Furthermore, add the commands
\begin{center}
\begin{tabular}{l}
|\input{childdoc.def}|\\
|\childdocof{|\textit{main}|}|\\
\end{tabular}
\end{center}
at the top of every child file \textit{child}
which is included by |\include{|\textit{child}|}|
from within the main file
(or at least for those files to be compiled individually).
The argument \textit{main} must be the filename of the main file.

There are a couple of
considerations in setting up the main and child documents:

%%%%%%%%%%%%%%%%%%%%%%%%%%%%%%%%%%%%%%%%
\paragraph{Restrictions.}

Please note the following restrictions:
\begin{itemize}
\item
|\childdocmain| must be called with one argument \textit{main}
to ensure compatibility with earlier version of the package.
It must either be empty (|\childdocmain{}|)
or precisely match the filename of the main file in which it is specified.
See \secref{sec:detection} for further information.
\item
The filename \textit{main} must be specified without the |.tex| extension.
\item
The filename \textit{main} is case sensitive
(even in case-insensitive file systems)
due to internal string comparison.
\item
The argument \textit{main} should be fully expanded, it cannot be a macro.
\item
Subdirectories and special characters should be avoided in filenames.
\item
The command |\childdocmain{|\textit{main}|}| must be followed by a whitespace.
It should not be followed immediately by another command
or by a comment mark `|%|'.
This is because the \TeX{} parser reads the token immediately following
the argument of |\childdocmain| and puts it
at the beginning of every child section;
however, a white\-space is ignored.
\end{itemize}

%%%%%%%%%%%%%%%%%%%%%%%%%%%%%%%%%%%%%%%%
\paragraph{Content of Main File.}

It is advisable to place all content in the child files included by |\include|.
Any output contained in the main file will appear in all child documents
unless suppressed manually;
it cannot be suppressed automatically by the |\includeonly| directive
and thus should normally be avoided.
A method to include some content in the main file
by means of conditional processing is described in \secref{sec:conditional}.

%%%%%%%%%%%%%%%%%%%%%%%%%%%%%%%%%%%%%%%%
\paragraph{Page Numbering.}

When only a part of the document is compiled,
the appropriate numbering of pages
(as well as other status parameters)
is determined from the |.aux| files.
The latter contain information from previous passes.
However this information needs to propagate through
all intermediate child documents.
Therefore the page numbering in child documents may well
be inconsistent until the complete document is compiled at least once.

A useful (if unconventional) way to always ensure a consistent
page numbering is to restart the numbering in each child document
and denote the pages by `\textit{child}|.|\textit{page}'
where \textit{child} represents the chapter/section number of the child file.
This can be achieved by the command
|\numberwithin{page}{|\textit{child}|}|
of the \textsf{amsmath} package
where \textit{child} can be |chapter| or |section|
depending on the chosen structuring.
Alternatively, one can modify the macro |\thepage| appropriately
and reset the counter |page| at the start of each child file.

%%%%%%%%%%%%%%%%%%%%%%%%%%%%%%%%%%%%%%%%%%%%%%%%%%%%%%%%%%%%%%%%%%%%%%%%%%%%%%%%
\subsection{Conditional Processing}
\label{sec:conditional}

The package provides a mechanism to compile different versions
of a document. To customise the versions further some conditional processing
can come in handy to distinguish which version is being compiled.
The package provides two macros to describe the compilation context:

%%%%%%%%%%%%%%%%%%%%%%%%%%%%%%%%%%%%%%%%
\DescribeMacro{\ifchilddoc}
The conditional |\ifchilddoc| distinguishes between the compilation of
child documents and the main document:
%
\begin{center}
|\ifchilddoc |\textit{child-code}| |[|\||else |\textit{main-code}]| \||fi|
\end{center}

%%%%%%%%%%%%%%%%%%%%%%%%%%%%%%%%%%%%%%%%
\DescribeMacro{\childdocname}
\DescribeMacro{\childdocjob}
The macro |\childdocname| contains the filename (without extension)
of the main or child file being processed.
Note that |\childdocjob| will always contain the name of the main file.

%%%%%%%%%%%%%%%%%%%%%%%%%%%%%%%%%%%%%%%%
\paragraph{Title Page.}

Conditional processing can be used to include a title or banner page
in the main document when proper precautions are taken.
Importantly, the code in the main file should ensure that the page counter
(as well as other status parameters which are stored in the |.aux| files)
takes the same value after the conditional processing.
Otherwise the page numbers may take divergent values
depending on which part is compiled.

For example, a title page could be declared by:
%
\begin{center}
\begin{tabular}{l}
|\ifchilddoc\||else|\\
|\addtocounter{page}{-1}|\\
\textit{code for title page}\\
|\newpage|\\
|\||fi|
\end{tabular}
\end{center}
%
A banner page for the child documents can be generated by:
%
\begin{center}
\begin{tabular}{l}
|\ifchilddoc|\\
|\addtocounter{page}{-1}|\\
\textit{code for banner page}\\
|\newpage|\\
|\||fi|
\end{tabular}
\end{center}
%
Here one could write a message such as:
\begin{center}
|This is the part \childdocname{} of \childdocjob{}.|
\end{center}

%%%%%%%%%%%%%%%%%%%%%%%%%%%%%%%%%%%%%%%%%%%%%%%%%%%%%%%%%%%%%%%%%%%%%%%%%%%%%%%%
\subsection{Flags}
\label{sec:flags}

The package makes it easy to generate different versions
of the main or child documents.
To this end compilation flags can be defined
and assigned different default values.
They will be particularly useful in conjunction
with the forwarding mechanism described in \secref{sec:forward}.

For example, it may be useful to have a flag |\version|
which can be set to |draft| or |final|.
The document source will contain some conditional code
depending on the value of |\version|.
Suppose further, the flag should default to |final| for the main file
and to |draft| for child files
which is a natural assignment for editing the document.
This is achieved by placing the following code
in the preamble of the main document
(below the |\childdocmain| directive):
%
\begin{center}
\begin{tabular}{l}
|\ifchilddoc|\\
|\providecommand{\version}{draft}|\\
|\||else|\\
|\providecommand{\version}{final}|\\
|\||fi|
\end{tabular}
\end{center}
%
The definition by |\providecommand| makes sure
that previous definitions are not overwritten.
Further statements |\providecommand{\version}{...}|
can thus be added before the above code to override it.

For the main file, one might add a line
(between |\childdocmain| and the above block)
%
\begin{center}
|%\ifchilddoc\||else\providecommand{\version}{draft}\||fi|
\end{center}
%
which can be uncommented to produce a draft version.
Likewise one can add a line to the very top of a child file
(above the |\childdocof{|\textit{main}|}| directive)
%
\begin{center}
|%\providecommand{\version}{final}|
\end{center}
%
which can be uncommented to produce the final version of this child document.

%%%%%%%%%%%%%%%%%%%%%%%%%%%%%%%%%%%%%%%%%%%%%%%%%%%%%%%%%%%%%%%%%%%%%%%%%%%%%%%%
\subsection{Forwarding}
\label{sec:forward}

Different versions of the main or child documents
using compilation flags as described in \secref{sec:flags}
can be (permanently) stored in different files
for convenient compilation, viewing and distribution.
To this end, the package defines a command
to pass on compilation to a different file:

%%%%%%%%%%%%%%%%%%%%%%%%%%%%%%%%%%%%%%%%
\DescribeMacro{\childdocforward}
The command |\childdocforward| redirects processing to
another source file:
%
\begin{center}
\begin{tabular}{l}
|\input{childdoc.def}|\\
|\childdocforward[|\textit{main}|]{|\textit{dest}|}|\\
\end{tabular}
\end{center}
%
The argument \textit{dest} is the destination file
(without extension).
It should be the main file or one of the child files.
Note that further \textsf{childdoc} directives
such as |\childdocof| and |\childdocforward|
in the indicated file will be processed in this form.
The optional argument \textit{main}
passes on directly to the main file \textit{main}
while pretending to compile the child \textit{dest}.
This form behaves as if \textit{dest}
issues |\childdocof{|\textit{main}|}| right away,
and no further \textsf{childdoc} directives will be processed.

%%%%%%%%%%%%%%%%%%%%%%%%%%%%%%%%%%%%%%%%
\DescribeMacro{\...prefix}
In the alternative form |\childdocforwardprefix|,
%
\begin{center}
\begin{tabular}{l}
|\input{childdoc.def}|\\
|\childdocforwardprefix[|\textit{main}|]{|\textit{prefix}|}{|\textit{dest}|}|
\end{tabular}
\end{center}
%
the destination file is determined by a pattern
depending on the current file:
To make this work, the current file must be called
`{\textit{prefix}\hspace{0.2em}\textit{suffix}}'
with \textit{prefix} matching precisely the argument.
Processing is then passed on to the file
`{\textit{dest}\hspace{0.2em}\textit{suffix}}'.
Surely, the same effect is achieved by
directly specifying the
argument `{\textit{dest}\hspace{0.2em}\textit{suffix}}'
in the first form.
However, that requires to set up a different file
for each child. With the alternative form of the command
all these files can have exactly the same content
which simplifies setting them up and maintaining them.

For example, the following file |draft.tex|
with a compilation flag |\version| as described in \secref{sec:flags}
compiles the main document as a draft:
%
\begin{center}
\begin{tabular}{l}
|\def\version{draft}|\\
|\input{childdoc.def}|\\
|\childdocforward{|\textit{main}|}|
\end{tabular}
\end{center}
%
Likewise, the following files |final|\textit{nn}|.tex|
compile the final version of the child document
|child|\textit{nn}|.tex|:
%
\begin{center}
\begin{tabular}{l}
|\def\version{final}|\\
|\input{childdoc.def}|\\
|\childdocforwardprefix{final}{child}|
\end{tabular}
\end{center}
%

Note that when several versions of a main file and/or of each child file
are to be generated, it may be convenient to set up a |Makefile| or
shell script to automatise the process.

%%%%%%%%%%%%%%%%%%%%%%%%%%%%%%%%%%%%%%%%%%%%%%%%%%%%%%%%%%%%%%%%%%%%%%%%%%%%%%%%
\subsection{Command Line Processing}
\label{sec:commandline}

The effect of redirection files can also be achieved by invoking
the \LaTeX{} compiler with a more elaborate command line.
Most conveniently this should be done as part
of a shell script or a |Makefile|.

When using \textsf{childdoc} in the main file, the following
command lines effectively perform a redirection
(note that depending on the shell being used,
backslashes may have to be doubled: `|\|' $\to$ `|\\|'):
%
\begin{center}
|... -jobname "|\textit{target}|" |\\|"|[\textit{flags}]%
|\input{childdoc.def}\childdocforward[|\textit{main}|]{|\textit{dest}|}"|
\end{center}
%
Here \textit{target} is the name of the output file,
\textit{main} is the name of the main file
and \textit{dest} is the name of the main or child file to be processed
(all filenames without extensions).
The optional argument \textit{main} can be omitted
if \textit{main} matches \textit{dest}.
Optionally, compilation \textit{flags} can be defined via |\def| commands.
This command line makes the \TeX{} engine believe
it is compiling the file \textit{target}
whose content is specified as the latter parameter.
The provided code then forwards the processing to
\textit{main} or \textit{dest} as described in \secref{sec:forward}.

%%%%%%%%%%%%%%%%%%%%%%%%%%%%%%%%%%%%%%%%%%%%%%%%%%%%%%%%%%%%%%%%%%%%%%%%%%%%%%%%
\subsection{Include by Input}
\label{sec:input}

Including child documents by |\include| has some restrictions by design.
Most notably, the content of a child document always occupies
its own set of pages; pages cannot be shared between child documents.
Usually, this behaviour makes perfect sense
because each child document contain an essential part of the document.
However, in some situations it may be desirable to compose
a document from a collection of parts
without having mandatory page breaks between then.
For this case, the package
provides a mechanism to include parts
by |\input| which can also be processed individually.
However, by construction this mechanism
requires manual handling of the content to be output.

%%%%%%%%%%%%%%%%%%%%%%%%%%%%%%%%%%%%%%%%
\DescribeMacro{\ifchilddocmanual}
The main file should be prepared as usual, see \secref{sec:include}.
However, the document body must make a distinction
between processing of an individual part and of the main document, e.g.:
%
\begin{center}
\begin{tabular}{l}
|\ifchilddocmanual|\\
|\input{\childdocname}|\\
|\||else|\\
\textit{document body with }|\input{|\textit{part}|}|\\
|\||fi|
\end{tabular}
\end{center}
%
The conditional |\ifchilddocmanual| is true whenever
a part to be included by |\input| is being compiled,
and the name of the part is stored in |\childdocname|.

%%%%%%%%%%%%%%%%%%%%%%%%%%%%%%%%%%%%%%%%
\DescribeMacro{\childdocby}
Each part to be included by |\input| should start with:
%
\begin{center}
\begin{tabular}{l}
|\input{childdoc.def}|\\
|\childdocby{|\textit{main}|}|\\
\end{tabular}
\end{center}
%
The directive |\childdocby| is similar to |\childdocof|
described in \secref{sec:include},
but the subsequent selection of content must be done manually.
To that end, both |\ifchilddoc| and |\ifchilddocmanual|
will be true upon processing of a part,
and the name of the part is stored in |\childdocname|.
Note that |\jobname| will be set to the filename of the current part
so that each part receives an individual |.aux| file
that does not interfere with the |.aux| file(s) of the main document.
This behaviour can be altered by the alternative form
|\childdocby[*]{|\textit{main}|}| (with a non-empty optional argument)
which uses the |.aux| file of the main document
by setting |\jobname| to \textit{main}.

%%%%%%%%%%%%%%%%%%%%%%%%%%%%%%%%%%%%%%%%%%%%%%%%%%%%%%%%%%%%%%%%%%%%%%%%%%%%%%%%
\subsection{Driver Development}
\label{sec:driver}

The \textsf{childdoc} mechanism can also be use for the development
of definition files such as \LaTeX{} styles or classes.
This case differs from the above setup with multiple parts
included by |\include| in that no |\includeonly| should be invoked.
This can be achieved by starting the include file
(before |\ProvidesPackage|) with:
%
\begin{center}
\begin{tabular}{l}
|\input{childdoc.def}|\\
|\childdocforward{|\textit{main}|}|\\
\end{tabular}
\end{center}
%
or alternatively with:
%
\begin{center}
\begin{tabular}{l}
|\input{childdoc.def}|\\
|\childdocby{|\textit{main}|}|\\
\end{tabular}
\end{center}
%
Both forms have slightly different effects as described above.
The main file is prepared as usual, see \secref{sec:include}.

%%%%%%%%%%%%%%%%%%%%%%%%%%%%%%%%%%%%%%%%%%%%%%%%%%%%%%%%%%%%%%%%%%%%%%%%%%%%%%%%
\subsection{Legacy Detection}
\label{sec:detection}

The directive |\childdocmain| in the main file can detect
whether the complete document or merely a child is to be compiled
even without using the directive |\childdocof|.
This method is deprecated because it is less robust
and there is no compelling reason to use it;
it is merely provided for backward compatibility
and it may be removed in future versions.

If the detection mechanism is to be used,
it is mandatory to correctly specify
the filename of the main file as the argument of |\childdocmain|:
%
\begin{center}
\begin{tabular}{l}
|\input{childdoc.def}|\\
|\childdocmain{|\textit{main}|}|\\
\end{tabular}
\end{center}
%
If |\jobname| does not match the argument \textit{main} of |\childdocmain|,
it is assumed that |\jobname| points to the child file to be compiled.
When using |\childdocmain| with the main file specified as argument,
it suffices to start a child file
with just |\input{|\textit{main}|}|
without loading of the package and using |\childdocof|.
If instead all processing is done
with the appropriate \textsf{childdoc} directives,
the argument of \textit{main} of |\childdocmain| can be empty.

An alternative version of the command line processing described
in \secref{sec:commandline} using the detection mechanism reads:
%
\begin{center}
|... -jobname "|\textit{target}|" "|[\textit{flags}]%
[|\def\jobname{|\textit{dest}|}|]|\input{|\textit{main}|}"|
\end{center}

%%%%%%%%%%%%%%%%%%%%%%%%%%%%%%%%%%%%%%%%%%%%%%%%%%%%%%%%%%%%%%%%%%%%%%%%%%%%%%%%
\subsection{Manual Code}
\label{sec:manual}

In case one cannot be certain whether the definitions file |childdoc.def|
is installed on the target \TeX{} distribution
and one prefers not to ship it,
it is conceivable to paste a few relevant commands into the sources.

To that end, drop all statements |\input{childdoc.def}|
and perform the replacements as outlined below.
Instead of |\childdocmain{|\textit{main}|}| add the following code
to the top of the main file:
%
\begin{center}
\begin{tabular}{l}
|\||ifdefined\childdocname\endinput\||fi\newif\ifchilddoc|\\
|\edef\childdocname{\scantokens\expandafter{\jobname\noexpand}}|\\
|\def\childdocmain{|\textit{main}|}\||ifx\childdocmain\childdocname\||else|\\
|\childdoctrue\includeonly{\childdocname}\let\jobname\childdocmain\||fi|\\
\end{tabular}
\end{center}
%
Instead of |\childdocof{|\textit{main}|}| just include the main file
at the top of each child file:
%
\begin{center}
|\input{|\textit{main}|}|
\end{center}
%
A simple redirection |\childdocforward{|\textit{dest}|}| is achieved by:
%
\begin{center}
|\def\jobname{|\textit{dest}|}\input{\jobname}|
\end{center}
%
The redirection with prefix
|\childdocforwardprefix[|\textit{prefix}|]{|\textit{dest}|}|
is accomplished by:
%
\begin{center}
\begin{tabular}{l}
|{\edef\jobname{\scantokens\expandafter{\jobname\noexpand}}|\\
|\def\redirectjob |\textit{prefix}|#1~~~{\gdef\jobname{|\textit{dest}|#1}}|\\
|\expandafter\redirectjob\jobname~~~}\input{\jobname}|
\end{tabular}
\end{center}

In an alternative approach,
child documents can be compiled by a specific command line
without additional code or specific definitions:
%
\begin{center}
|... -jobname "|\textit{target}|" "|[\textit{flags}]%
|\includeonly{|\textit{dest}|}\input{|\textit{main}|}"|
\end{center}
%

%%%%%%%%%%%%%%%%%%%%%%%%%%%%%%%%%%%%%%%%%%%%%%%%%%%%%%%%%%%%%%%%%%%%%%%%%%%%%%%%
%%%%%%%%%%%%%%%%%%%%%%%%%%%%%%%%%%%%%%%%%%%%%%%%%%%%%%%%%%%%%%%%%%%%%%%%%%%%%%%%
\section{Information}

%%%%%%%%%%%%%%%%%%%%%%%%%%%%%%%%%%%%%%%%%%%%%%%%%%%%%%%%%%%%%%%%%%%%%%%%%%%%%%%%
\subsection{Copyright}

Copyright \copyright{} 2017--2018 Niklas Beisert

This work may be distributed and/or modified under the
conditions of the \LaTeX{} Project Public License, either version 1.3
of this license or (at your option) any later version.
The latest version of this license is in
  \url{http://www.latex-project.org/lppl.txt}
and version 1.3 or later is part of all distributions of \LaTeX{}
version 2005/12/01 or later.

This work has the LPPL maintenance status `maintained'.

The Current Maintainer of this work is Niklas Beisert.

This work consists of the files |README.txt|, |childdoc.ins| and |childdoc.dtx|
as well as the derived files |childdoc.def|, |cdocsamp.tex|
with |cdocsch1.tex|, |cdocsch2.tex|, |cdocspt3.tex|, |cdocspt4.tex|,
|cdocsdrf.tex|, |cdocsfn1.tex|, |cdocsfn2.tex|
as well as |childdoc.pdf|.

%%%%%%%%%%%%%%%%%%%%%%%%%%%%%%%%%%%%%%%%%%%%%%%%%%%%%%%%%%%%%%%%%%%%%%%%%%%%%%%%
\subsection{Files and Installation}

The package consists of the files:
%
\begin{center}
\begin{tabular}{ll}
    |README.txt|   & readme file \\
    |childdoc.ins| & installation file \\
    |childdoc.dtx| & source file \\
    |childdoc.def| & definition file \\
    |cdocsamp.tex| & sample main file \\
    |cdocsch1.tex| & sample include file \\
    |cdocsch2.tex| & sample include file \\
    |cdocspt3.tex| & sample part file \\
    |cdocspt4.tex| & sample part file \\
    |cdocsdrf.tex| & sample redirection file \\
    |cdocsfn1.tex| & sample redirection file \\
    |cdocsfn2.tex| & sample redirection file \\
    |childdoc.pdf| & manual
\end{tabular}
\end{center}
%
The distribution consists of the files
|README.txt|, |childdoc.ins| and |childdoc.dtx|.
%
\begin{itemize}
\item
Run (pdf)\LaTeX{} on |childdoc.dtx|
to compile the manual |childdoc.pdf| (this file).
\item
Run \LaTeX{} on |childdoc.ins| to create the definitions file |childdoc.def|
and the sample |cdocsamp.tex| with include files
|cdocsch1.tex|, |cdocsch2.tex|, |cdocspt3.tex|, |cdocspt4.tex|,
|cdocsdrf.tex|, |cdocsfn1.tex|, |cdocsfn2.tex|.
Then copy the file |childdoc.def| to an appropriate directory of your \LaTeX{}
distribution, e.g.\ \textit{texmf-root}|/tex/latex/childdoc|.
\end{itemize}

%%%%%%%%%%%%%%%%%%%%%%%%%%%%%%%%%%%%%%%%%%%%%%%%%%%%%%%%%%%%%%%%%%%%%%%%%%%%%%%%
\subsection{Related CTAN Packages}

There are several other packages which offer a similar functionality:
%
\begin{itemize}
\item
The packages
\href{http://ctan.org/pkg/docmute}{\textsf{docmute}},
\href{http://ctan.org/pkg/includex}{\textsf{includex}} and
\href{http://ctan.org/pkg/standalone}{\textsf{standalone}}
provide commands to include only the document body of
a child file thus allowing both files to be compiled individually.
\item
The packages \href{http://ctan.org/pkg/subdocs}{\textsf{subdocs}}
and \href{http://ctan.org/pkg/subfiles}{\textsf{subfiles}}
provide structures in which the main and child documents can be
encapsulated and allowing them to be compiled individually.
The inclusion mechanism is different from the conventional |\include|.
\item
The package \href{http://ctan.org/pkg/combine}{\textsf{combine}}
is an elaborate solution to combine several documents into one.
\end{itemize}
%
See also the CTAN topic \href{http://ctan.org/topic/subdocs}{\textsf{subdocs}}
for further related packages.
The present package differs from the above solutions in that
a document structure constructed with the conventional |\include| mechanism
just needs two extra commands at the top of every file
such that all constituent files can be compiled individually.

%%%%%%%%%%%%%%%%%%%%%%%%%%%%%%%%%%%%%%%%%%%%%%%%%%%%%%%%%%%%%%%%%%%%%%%%%%%%%%%%
%\subsection{Feature Suggestions}
%
%The following is a list of features which may be useful for future
%versions of this package:
%%
%\begin{itemize}
%\item
%\ldots
%\end{itemize}

%%%%%%%%%%%%%%%%%%%%%%%%%%%%%%%%%%%%%%%%%%%%%%%%%%%%%%%%%%%%%%%%%%%%%%%%%%%%%%%%
\subsection{Revision History}

%%%%%%%%%%%%%%%%%%%%%%%%%%%%%%%%%%%%%%%%
\paragraph{v2.0:} 2018/12/30

\begin{itemize}
\item
immediate forward processing
\item
added |\childdocby| mechanism
\item
manual restructured
\end{itemize}

%%%%%%%%%%%%%%%%%%%%%%%%%%%%%%%%%%%%%%%%
\paragraph{v1.6:} 2018/01/17

\begin{itemize}
\item
application for development of include files
\item
corrections to manual
\end{itemize}

%%%%%%%%%%%%%%%%%%%%%%%%%%%%%%%%%%%%%%%%
\paragraph{v1.5:} 2017/05/21

\begin{itemize}
\item
more complete structuring introduced
\item
|\childdocof| introduced
\item
|\childdoc| renamed to |\childdocmain|
\item
|\childredirect| renamed to |\childdocforward| and |\childdocforwardprefix|
and functionality expanded
\end{itemize}

%%%%%%%%%%%%%%%%%%%%%%%%%%%%%%%%%%%%%%%%
\paragraph{v1.0:} 2017/04/27

\begin{itemize}
\item
manual and install package
\item
first version published on CTAN
\end{itemize}

%%%%%%%%%%%%%%%%%%%%%%%%%%%%%%%%%%%%%%%%
\paragraph{v0.6:} 2017/04/26

\begin{itemize}
\item
redirection mechanism added
\end{itemize}

%%%%%%%%%%%%%%%%%%%%%%%%%%%%%%%%%%%%%%%%
\paragraph{v0.5:} 2017/04/26

\begin{itemize}
\item
functionality in definition file
\end{itemize}


%%%%%%%%%%%%%%%%%%%%%%%%%%%%%%%%%%%%%%%%%%%%%%%%%%%%%%%%%%%%%%%%%%%%%%%%%%%%%%%%
%%%%%%%%%%%%%%%%%%%%%%%%%%%%%%%%%%%%%%%%%%%%%%%%%%%%%%%%%%%%%%%%%%%%%%%%%%%%%%%%
%%%%%%%%%%%%%%%%%%%%%%%%%%%%%%%%%%%%%%%%%%%%%%%%%%%%%%%%%%%%%%%%%%%%%%%%%%%%%%%%
\appendix

\settowidth\MacroIndent{\rmfamily\scriptsize 000\ }

 \DocInput{childdoc.dtx}

\end{document}
%</driver>
% \fi
%
% %%%%%%%%%%%%%%%%%%%%%%%%%%%%%%%%%%%%%%%%%%%%%%%%%%%%%%%%%%%%%%%%%%%%%%%%%%%%%%
% %%%%%%%%%%%%%%%%%%%%%%%%%%%%%%%%%%%%%%%%%%%%%%%%%%%%%%%%%%%%%%%%%%%%%%%%%%%%%%
% \section{Sample}
%\iffalse
%<*samplemain>
%\fi
%
% The following presents a sample document
% with two chapters, two parts, a title page,
% a compile flag as well as three forwarding files to set the flag.
% It consists of eight |.tex| files:
% \begin{center}
% \begin{tabular}{ll}
% |cdocsamp.tex|&main file\\
% |cdocsch1.tex|&include file for chapter 1\\
% |cdocsch2.tex|&include file for chapter 2\\
% |cdocspt3.tex|&include file for part 3\\
% |cdocspt4.tex|&include file for part 4\\
% |cdocsdrf.tex|&forwarding file for main file in draft mode\\
% |cdocsfi1.tex|&forwarding file for final version of chapter 1\\
% |cdocsfi2.tex|&forwarding file for final version of chapter 2\\
% \end{tabular}
% \end{center}
% Each of the eight files can be compiled directly by the \LaTeX{} compiler.
%
% %%%%%%%%%%%%%%%%%%%%%%%%%%%%%%%%%%%%%%
% \paragraph{Main File.}
%
% The main file is called |cdocsamp.tex|.
%
% Load the \textsf{childdoc} definitions and
% declare the filename for the main document:
%    \begin{macrocode}
\input{childdoc.def}
\childdocmain{}
%    \end{macrocode}

% Optional override for |\version| flag:
%    \begin{macrocode}
%%\ifchilddoc\else\providecommand{\version}{draft}\fi
%    \end{macrocode}

% Define the default values for the |\version| flag
% (|final| for the main file and |draft| for childs):
%    \begin{macrocode}
\ifchilddoc
\providecommand{\version}{draft}
\else
\providecommand{\version}{final}
\fi
%    \end{macrocode}

% Load the standard document class:
%    \begin{macrocode}
\documentclass[12pt]{article}
%    \end{macrocode}

% Start the document body:
%    \begin{macrocode}
\begin{document}
%    \end{macrocode}

% Declare a title page.
% Print title, part of document being processed and version flag:
%    \begin{macrocode}
\addtocounter{page}{-1}
\begin{center}
{\LARGE\bfseries{}childdoc example\par}
\vspace{1cm}
\ifchilddoc
\ifchilddocmanual part\else chapter\fi:
`\childdocname' of `\childdocjob'\par
\else
main document: `\childdocjob'\par
\fi
version: \version\par
\end{center}
\newpage
%    \end{macrocode}

% Manually include selected file,
% otherwise process as usual:
%    \begin{macrocode}
\ifchilddocmanual
\section*{part `\childdocname'}
\input{\childdocname}
\else
%    \end{macrocode}

% Include the two chapters:
%    \begin{macrocode}
\include{cdocsch1}
\include{cdocsch2}
%    \end{macrocode}

% Include the two parts unless only chapters should be displayed:
%    \begin{macrocode}
\ifchilddoc\else
\section{part three}
\input{cdocspt3}
\section{part four}
\input{cdocspt4}
\fi
%    \end{macrocode}

% Process as usual until here:
%    \begin{macrocode}
\fi
%    \end{macrocode}

% End of document body:
%    \begin{macrocode}
\end{document}
%    \end{macrocode}
%\iffalse
%</samplemain>
%\fi
%
% %%%%%%%%%%%%%%%%%%%%%%%%%%%%%%%%%%%%%%
% \paragraph{Chapter Include Files.}
%
% The include files are called |cdocsch1.tex| and |cdocsch2.tex|.
%
%\iffalse
%<*samplechap1|samplechap2>
%\fi

% Optional override for |\version| flag:
%    \begin{macrocode}
%%\providecommand{\version}{final}
%    \end{macrocode}

% Include the main document:
%    \begin{macrocode}
\input{childdoc.def}
\childdocof{cdocsamp}
%    \end{macrocode}

%\iffalse
%</samplechap1|samplechap2>
%\fi
%
%\iffalse
%<*samplechap1>
%\fi
% Some text for chapter 1:
%    \begin{macrocode}
\section{one}
some text in chapter one
%    \end{macrocode}

%\iffalse
%</samplechap1>
%\fi
% Some text for chapter 2:
%\iffalse
%<*samplechap2>
%\fi
%    \begin{macrocode}
\section{two}
more text in chapter two
%    \end{macrocode}

%\iffalse
%</samplechap2>
%\fi
%
% %%%%%%%%%%%%%%%%%%%%%%%%%%%%%%%%%%%%%%
% \paragraph{Part Include Files.}
%
% The include files are called |cdocspt3.tex| and |cdocspt4.tex|.
%
%\iffalse
%<*samplepart3|samplepart4>
%\fi

% Optional override for |\version| flag:
%    \begin{macrocode}
%%\providecommand{\version}{final}
%    \end{macrocode}

% Include the main document:
%    \begin{macrocode}
\input{childdoc.def}
\childdocby{cdocsamp}
%    \end{macrocode}

%\iffalse
%</samplepart3|samplepart4>
%\fi
%
%\iffalse
%<*samplepart3>
%\fi
% Some text for part 3:
%    \begin{macrocode}
some text in part three
%    \end{macrocode}

%\iffalse
%</samplepart3>
%\fi
% Some text for part 4:
%\iffalse
%<*samplepart4>
%\fi
%    \begin{macrocode}
more text in part four
%    \end{macrocode}

%\iffalse
%</samplepart4>
%\fi
%
% %%%%%%%%%%%%%%%%%%%%%%%%%%%%%%%%%%%%%%
% \paragraph{Forwarding for a Complete Draft.}
%
% The following forwarding file |cdocsdrf.tex|
% compiles the main document in draft mode:
%\iffalse
%<*sampledraft>
%\fi
%    \begin{macrocode}
\def\version{draft}
\input{childdoc.def}
\childdocforward{cdocsamp}
%    \end{macrocode}

%\iffalse
%</sampledraft>
%\fi
%
% %%%%%%%%%%%%%%%%%%%%%%%%%%%%%%%%%%%%%%
% \paragraph{Forwarding for Final Version of the Chapters.}
%
% The following forwarding files |cdocsfn1.tex| and |cdocsfn2.tex|
% (with identical content)
% compile the final versions of the child documents
% |cdocsch1.tex| and |cdocsch2.tex|, respectively:
%\iffalse
%<*samplefinal>
%\fi
%    \begin{macrocode}
\def\version{final}
\input{childdoc.def}
\childdocforwardprefix[cdocsamp]{cdocsfn}{cdocsch}
%    \end{macrocode}

%\iffalse
%</samplefinal>
%\fi
%
% %%%%%%%%%%%%%%%%%%%%%%%%%%%%%%%%%%%%%%
% \paragraph{Command Line Processing.}
%
% The following three command lines generate the output files
% |cdocscld|, |cdocscl1| and |cdocscl2|
% which should be identical to
% |cdocsdrf|, |cdocsch1| and |cdocsfn2|, respectively:
% \begin{center}
% \begin{tabular}{l}
% |latex -jobname cdocscld \|\\
% |  "\def\version{draft}\input{childdoc.def}\childdocforward{cdocsamp}"|\\
% |latex -jobname cdocscl1 \|\\
% |  "\input{childdoc.def}\childdocforward[cdocsamp]{cdocsch1}"|\\
% |latex -jobname cdocscl2 \|\\
% |  "\def\version{final}\input{childdoc.def}\childdocforward{cdocsch2}"|
% \end{tabular}
% \end{center}
% Note that the trailing backslash on each first line
% merely continues the input to the second line
% (for convenient cut ant paste).
% Furthermore, the command |latex| can be replaced by any
% of its alternative versions such as |pdflatex|.
%
% %%%%%%%%%%%%%%%%%%%%%%%%%%%%%%%%%%%%%%%%%%%%%%%%%%%%%%%%%%%%%%%%%%%%%%%%%%%%%%
% %%%%%%%%%%%%%%%%%%%%%%%%%%%%%%%%%%%%%%%%%%%%%%%%%%%%%%%%%%%%%%%%%%%%%%%%%%%%%%
% \section{Implementation}
%\iffalse
%<*package>
%\fi
%
% This section describes the definitions file |childdoc.def|.

% The definitions cannot be loaded using |\usepackage| or |\RequirePackage|
% which has a mechanism to prevent loading a style file more than once.
% When loading the definitions by means of |\input|
% multiple instances have to be prevented manually:
%\iffalse
%This code needs to be before the `\ProvidesFile' directive
%which is defined at the beginning of this file.
%Therefore it is also placed there and commented out here.
%</package>
%<*discard>
%\fi
%    \begin{macrocode}
\ifdefined\childdocmain\endinput\fi
%    \end{macrocode}
%\iffalse
%</discard>
%<*package>
%\fi
%
% \macro{\ifchilddoc}
% \macro{\ifchilddocmanual}
% The conditional |\ifchilddoc| tells whether a
% child (true) or main (false) document is being compiled.
% The conditional |\ifchilddocmanual| tells whether
% the |\includeonly| mechanism is used (false) or
% the selection of child files must be performed manually (true).
% The definitions initialise to false:
%    \begin{macrocode}
\newif\ifchilddoc
\newif\ifchilddocmanual
%    \end{macrocode}

% \macro{\childdocname}
% \macro{\childdocjob}
% The macro |\childdocname| stores the name of the main document
% to be compiled. The macro |\childdocjob| stores the name of
% the document on which the \LaTeX{} compiler was originally invoked.
% The content of |\jobname| cannot be compared
% to filenames specified in the source due to different catcodes.
% The following code rescans |\jobname|, stores the result
% in |\childdocname| and saves a copy in |\childdocjob|:
%    \begin{macrocode}
\edef\childdocname{\scantokens\expandafter{\jobname\noexpand}}
\let\childdocjob\childdocname
%    \end{macrocode}

% \macro{\childdocdisable}
% The macro |\childdocdisable| prevents the main file
% from being processed more than once.
% At this stage, the main document command |\childdocmain|
% is assumed to be called once again where it should do nothing.
% Any subsequent call to it should prevent
% a secondary processing of the main document
% It overwrites the forwarding commands
% |\childdocof| and |\childdocforward|
% with empty macros to prevent further inclusions of the main document:
%    \begin{macrocode}
\newcommand{\childdocdisable}
{
  \renewcommand{\childdocmain}[1]{\renewcommand{\childdocmain}[1]{\endinput}}
  \renewcommand{\childdocof}[1]{}
  \renewcommand{\childdocby}[2][]{}
  \renewcommand{\childdocforward}[2][]{}
  \renewcommand{\childdocdisable}{}
}
%    \end{macrocode}

% \macro{\childdocmain}
% The macro |\childdocmain| is to be called at the top of the main file
% with nothing or the main filename (without extension) as argument.
% First, it breaks loops.
% If the argument is not empty and does not match |\childdocname|
% (which is set by the first inclusion of |childdoc.def|),
% |\ifchilddoc| is set to true, |\includeonly| is applied to the child file
% and |\jobname| is set to the main file
% (for proper handling of |.aux| files):
%    \begin{macrocode}
\newcommand{\childdocmain}[1]
{
  \childdocdisable\childdocmain{}
  \if?#1?\else
    \begingroup
      \def\childdoctmp{#1}
      \ifx\childdoctmp\childdocname
        \def\childdoctmp{}
      \else
        \def\childdoctmp
        {
          \childdoctrue
          \includeonly{\childdocname}
          \def\childdocjob{#1}
          \def\jobname{#1}
        }
      \fi
      \expandafter
    \endgroup
    \childdoctmp
  \fi
}
%    \end{macrocode}

% \macro{\childdocof}
% The command |\childdocof| redirects
% compilation to the main file |#1|.
%    \begin{macrocode}
\newcommand{\childdocof}[1]
{
  \childdocdisable
  \childdoctrue
  \includeonly{\childdocname}
  \def\jobname{#1}
  \def\childdocjob{#1}
  \input{#1}
}
%    \end{macrocode}

% \macro{\childdocby}
% The command |\childdocby| ....
%    \begin{macrocode}
\newcommand{\childdocby}[2][]
{
  \childdocdisable
  \childdoctrue
  \childdocmanualtrue
  \if?#1?\else
    \def\jobname{#2}
  \fi
  \def\childdocjob{#2}
  \input{#2}
  \endinput
}
%    \end{macrocode}

% \macro{\childdocforward}
% The command |\childdocforward| redirects
% compilation to the main file or
% (if the optional argument is given) a child file.
% Parameters are set as if the main file
% or a child file starting with |\childdocof| was compiled.
% Then compilation is handed over to the main file:
%    \begin{macrocode}
\newcommand{\childdocforward}[2][]
{
  \begingroup
    \if?#1?
      \def\childdoctmp
      {
        \def\childdocname{#2}
        \def\childdocjob{#2}
        \def\jobname{#2}
        \input{#2}
        \endinput
      }
    \else
      \def\childdoctmp
      {
        \childdocdisable
        \def\childdocname{#2}
        \childdoctrue
        \includeonly{#2}
        \def\childdocjob{#1}
        \def\jobname{#1}
        \input{#1}
        \endinput
      }
    \fi
    \expandafter
  \endgroup
  \childdoctmp
}
%    \end{macrocode}

% \macro{\childdocforwardprefix}
% The command |\childdocforwardprefix| redirects
% compilation to the main or a child file by means of a pattern.
% The prefix |#1| in the current filename is replaced by |#2|
% and the suffix of the current filename is kept
% (it is assumed that the filename does not contain the substring `|~~~|'
% which is used as a delimiter).
% Compilation is handed over to the new file by |\childdocforward|:
%    \begin{macrocode}
\newcommand{\childdocforwardprefix}[3][]
{
  \begingroup
    \def\childdocextract #2##1~~~{\def\childdoctmp{\childdocforward[#1]{#3##1}}}
    \expandafter\childdocextract\childdocname~~~
    \expandafter
  \endgroup
  \childdoctmp
}
%    \end{macrocode}

% \macro{\childdoc}
% The deprecated macro |\childdoc| is a legacy version of |\childdocmain|:
%    \begin{macrocode}
\newcommand{\childdoc}{\childdocmain}
%    \end{macrocode}

% \macro{\childdocredirect}
% The deprecated macro |\childdocredirect| is a legacy version
% of |\childdocforward| and |\childdocforwardprefix|:
%    \begin{macrocode}
\newcommand{\childdocredirect}[2][]
{
  \begingroup
    \if?#1?
      \def\childdoctmp{\childdocforward{#2}}
    \else
      \def\childdoctmp{\childdocforwardprefix{#1}{#2}}
    \fi
    \expandafter
  \endgroup
  \childdoctmp
}
%    \end{macrocode}

%\iffalse
%</package>
%\fi
%
\endinput
|\\
|\childdocmain{}|\\
\end{tabular}
\end{center}
at the very top of the main \LaTeX{} file,
in particular \emph{before} the |\documentclass| statement!
The argument of |\childdocmain| should be left empty
(but it must be present).

%%%%%%%%%%%%%%%%%%%%%%%%%%%%%%%%%%%%%%%%
\DescribeMacro{\childdocof}
Furthermore, add the commands
\begin{center}
\begin{tabular}{l}
|% \iffalse
%
% childdoc.dtx Copyright (C) 2017-2018 Niklas Beisert
%
% This work may be distributed and/or modified under the
% conditions of the LaTeX Project Public License, either version 1.3
% of this license or (at your option) any later version.
% The latest version of this license is in
%   http://www.latex-project.org/lppl.txt
% and version 1.3 or later is part of all distributions of LaTeX
% version 2005/12/01 or later.
%
% This work has the LPPL maintenance status `maintained'.
%
% The Current Maintainer of this work is Niklas Beisert.
%
% This work consists of the files childdoc.dtx and childdoc.ins
% and the derived files childdoc.def and cdocsamp.tex with
% cdocsch1.tex, cdocsch2.tex, cdocsdrf.tex, cdocsfn1.tex, cdocsfn2.tex.
%
%<package>\ifdefined\childdocmain\endinput\fi
%<package>\ProvidesFile{childdoc.def}[2018/12/30 v2.0 child document driver]
%<samplemain>\ProvidesFile{cdocsamp.tex}[2018/12/30 v2.0 sample for childdoc]
%<*driver>
%\ProvidesFile{childdoc.drv}[2018/12/30 v2.0 childdoc reference manual file]
\PassOptionsToClass{10pt,a4paper}{article}
\documentclass{ltxdoc}

\usepackage[margin=35mm]{geometry}
\usepackage{hyperref}
\usepackage{hyperxmp}
\usepackage[usenames]{color}

\hypersetup{colorlinks=true}
\hypersetup{pdfstartview=FitH}
\hypersetup{pdfpagemode=UseNone}
\hypersetup{pdfsource={}}
\hypersetup{pdflang={en-UK}}
\hypersetup{pdfcopyright={Copyright 2017-2018 Niklas Beisert.
  This work may be distributed and/or modified under the
  conditions of the LaTeX Project Public License, either version 1.3
  of this license or (at your option) any later version.}}
\hypersetup{pdflicenseurl={http://www.latex-project.org/lppl.txt}}
\hypersetup{pdfcontactaddress={ETH Zurich, ITP, HIT K,
  Wolfgang-Pauli-Strasse 27}}
\hypersetup{pdfcontactpostcode={8093}}
\hypersetup{pdfcontactcity={Zurich}}
\hypersetup{pdfcontactcountry={Switzerland}}
\hypersetup{pdfcontactemail={nbeisert@itp.phys.ethz.ch}}
\hypersetup{pdfcontacturl={http://people.phys.ethz.ch/\xmptilde nbeisert/}}

\newcommand{\secref}[1]{\hyperref[#1]{section \ref*{#1}}}

\parskip1ex
\parindent0pt
\let\olditemize\itemize
\def\itemize{\olditemize\parskip0pt}

\begin{document}

\title{The \textsf{childdoc} Package}
\hypersetup{pdftitle={The childdoc Package}}
\author{Niklas Beisert\\[2ex]
  Institut f\"ur Theoretische Physik\\
  Eidgen\"ossische Technische Hochschule Z\"urich\\
  Wolfgang-Pauli-Strasse 27, 8093 Z\"urich, Switzerland\\[1ex]
  \href{mailto:nbeisert@itp.phys.ethz.ch}
  {\texttt{nbeisert@itp.phys.ethz.ch}}}
\hypersetup{pdfauthor={Niklas Beisert}}
\hypersetup{pdfsubject={Manual for the LaTeX2e Package childdoc}}
\date{30 December 2018, \textsf{v2.0}}
\maketitle

\begin{abstract}\noindent
\textsf{childdoc} is a \LaTeXe{} package
that enables the direct compilation
of document sections included by |\include|
to individual files.
\end{abstract}

\begingroup
\parskip0ex
\tableofcontents
\endgroup

%%%%%%%%%%%%%%%%%%%%%%%%%%%%%%%%%%%%%%%%%%%%%%%%%%%%%%%%%%%%%%%%%%%%%%%%%%%%%%%%
%%%%%%%%%%%%%%%%%%%%%%%%%%%%%%%%%%%%%%%%%%%%%%%%%%%%%%%%%%%%%%%%%%%%%%%%%%%%%%%%
\section{Introduction}

\LaTeX{} provides a mechanism to structure a large document (such as a book)
into a main file and several child files (containing the chapters)
using the |\include| command.
This mechanism is beneficial for documents
which span hundreds of pages in order to
make the source file(s) more manageable.
Moreover, compilation can be restricted to
selected child files by means of the |\includeonly| command.
The latter feature can be used to reduce the compilation time while editing
(this was significantly more useful in the earlier days of \LaTeX{})
or to generate a smaller document which is easier to navigate.
Another application of |\includeonly| is to generate
documents consisting of selected parts of the complete document.

However, there are a few drawbacks of the plain |\include| mechanism:
\begin{itemize}
\item
The child files cannot be compiled on their own,
they can only be compiled via the main file.
A naive editing environment
(such as a text editor with an option
to have the current file processed by \LaTeX)
may require one to switch to the main file before compiling;
attempting to compile the child file produces errors.
\item
The main file must be modified (each time)
to adjust the |\includeonly| command
to the present needs. This easily leaves the main file in a messy state.
\item
The generated document will always carry the filename
of the main document. This is inconvenient if
several child files are to be compiled and
to be kept for distribution.
\end{itemize}

The present package provides a simple interface
to make child files individually compilable by \LaTeX{}.
Compiling a child file then has the same effect as compiling
the main file with an |\includeonly| command
to select the appropriate child.
Moreover the generated document will carry the name of the child
rather than the main file.
This resolves all three above issues.

This feature is meant to make the editing of books,
thesis documents and lecture notes somewhat more convenient.
However, the package can also be used efficiently for
composing a series of documents (such as exercise sheets)
which are typically distributed individually.
It then assists the author in generating the individual documents
(potentially in different versions)
as well as a document containing the collected series.
Another application is in developing style files
or other kinds of included material
where compilation of the style file could redirect
to a sample or test file.

%%%%%%%%%%%%%%%%%%%%%%%%%%%%%%%%%%%%%%%%%%%%%%%%%%%%%%%%%%%%%%%%%%%%%%%%%%%%%%%%
%%%%%%%%%%%%%%%%%%%%%%%%%%%%%%%%%%%%%%%%%%%%%%%%%%%%%%%%%%%%%%%%%%%%%%%%%%%%%%%%
\section{Usage}

First of all, the package \textsf{childdoc} is \emph{not} a standard
\LaTeXe{} |.sty| style file! Therefore it needs to be invoked in
a non-standard way.

%%%%%%%%%%%%%%%%%%%%%%%%%%%%%%%%%%%%%%%%%%%%%%%%%%%%%%%%%%%%%%%%%%%%%%%%%%%%%%%%
\subsection{Included Files}
\label{sec:include}

%%%%%%%%%%%%%%%%%%%%%%%%%%%%%%%%%%%%%%%%
\DescribeMacro{\childdocmain}
To use the package, add the commands
\begin{center}
\begin{tabular}{l}
|\input{childdoc.def}|\\
|\childdocmain{}|\\
\end{tabular}
\end{center}
at the very top of the main \LaTeX{} file,
in particular \emph{before} the |\documentclass| statement!
The argument of |\childdocmain| should be left empty
(but it must be present).

%%%%%%%%%%%%%%%%%%%%%%%%%%%%%%%%%%%%%%%%
\DescribeMacro{\childdocof}
Furthermore, add the commands
\begin{center}
\begin{tabular}{l}
|\input{childdoc.def}|\\
|\childdocof{|\textit{main}|}|\\
\end{tabular}
\end{center}
at the top of every child file \textit{child}
which is included by |\include{|\textit{child}|}|
from within the main file
(or at least for those files to be compiled individually).
The argument \textit{main} must be the filename of the main file.

There are a couple of
considerations in setting up the main and child documents:

%%%%%%%%%%%%%%%%%%%%%%%%%%%%%%%%%%%%%%%%
\paragraph{Restrictions.}

Please note the following restrictions:
\begin{itemize}
\item
|\childdocmain| must be called with one argument \textit{main}
to ensure compatibility with earlier version of the package.
It must either be empty (|\childdocmain{}|)
or precisely match the filename of the main file in which it is specified.
See \secref{sec:detection} for further information.
\item
The filename \textit{main} must be specified without the |.tex| extension.
\item
The filename \textit{main} is case sensitive
(even in case-insensitive file systems)
due to internal string comparison.
\item
The argument \textit{main} should be fully expanded, it cannot be a macro.
\item
Subdirectories and special characters should be avoided in filenames.
\item
The command |\childdocmain{|\textit{main}|}| must be followed by a whitespace.
It should not be followed immediately by another command
or by a comment mark `|%|'.
This is because the \TeX{} parser reads the token immediately following
the argument of |\childdocmain| and puts it
at the beginning of every child section;
however, a white\-space is ignored.
\end{itemize}

%%%%%%%%%%%%%%%%%%%%%%%%%%%%%%%%%%%%%%%%
\paragraph{Content of Main File.}

It is advisable to place all content in the child files included by |\include|.
Any output contained in the main file will appear in all child documents
unless suppressed manually;
it cannot be suppressed automatically by the |\includeonly| directive
and thus should normally be avoided.
A method to include some content in the main file
by means of conditional processing is described in \secref{sec:conditional}.

%%%%%%%%%%%%%%%%%%%%%%%%%%%%%%%%%%%%%%%%
\paragraph{Page Numbering.}

When only a part of the document is compiled,
the appropriate numbering of pages
(as well as other status parameters)
is determined from the |.aux| files.
The latter contain information from previous passes.
However this information needs to propagate through
all intermediate child documents.
Therefore the page numbering in child documents may well
be inconsistent until the complete document is compiled at least once.

A useful (if unconventional) way to always ensure a consistent
page numbering is to restart the numbering in each child document
and denote the pages by `\textit{child}|.|\textit{page}'
where \textit{child} represents the chapter/section number of the child file.
This can be achieved by the command
|\numberwithin{page}{|\textit{child}|}|
of the \textsf{amsmath} package
where \textit{child} can be |chapter| or |section|
depending on the chosen structuring.
Alternatively, one can modify the macro |\thepage| appropriately
and reset the counter |page| at the start of each child file.

%%%%%%%%%%%%%%%%%%%%%%%%%%%%%%%%%%%%%%%%%%%%%%%%%%%%%%%%%%%%%%%%%%%%%%%%%%%%%%%%
\subsection{Conditional Processing}
\label{sec:conditional}

The package provides a mechanism to compile different versions
of a document. To customise the versions further some conditional processing
can come in handy to distinguish which version is being compiled.
The package provides two macros to describe the compilation context:

%%%%%%%%%%%%%%%%%%%%%%%%%%%%%%%%%%%%%%%%
\DescribeMacro{\ifchilddoc}
The conditional |\ifchilddoc| distinguishes between the compilation of
child documents and the main document:
%
\begin{center}
|\ifchilddoc |\textit{child-code}| |[|\||else |\textit{main-code}]| \||fi|
\end{center}

%%%%%%%%%%%%%%%%%%%%%%%%%%%%%%%%%%%%%%%%
\DescribeMacro{\childdocname}
\DescribeMacro{\childdocjob}
The macro |\childdocname| contains the filename (without extension)
of the main or child file being processed.
Note that |\childdocjob| will always contain the name of the main file.

%%%%%%%%%%%%%%%%%%%%%%%%%%%%%%%%%%%%%%%%
\paragraph{Title Page.}

Conditional processing can be used to include a title or banner page
in the main document when proper precautions are taken.
Importantly, the code in the main file should ensure that the page counter
(as well as other status parameters which are stored in the |.aux| files)
takes the same value after the conditional processing.
Otherwise the page numbers may take divergent values
depending on which part is compiled.

For example, a title page could be declared by:
%
\begin{center}
\begin{tabular}{l}
|\ifchilddoc\||else|\\
|\addtocounter{page}{-1}|\\
\textit{code for title page}\\
|\newpage|\\
|\||fi|
\end{tabular}
\end{center}
%
A banner page for the child documents can be generated by:
%
\begin{center}
\begin{tabular}{l}
|\ifchilddoc|\\
|\addtocounter{page}{-1}|\\
\textit{code for banner page}\\
|\newpage|\\
|\||fi|
\end{tabular}
\end{center}
%
Here one could write a message such as:
\begin{center}
|This is the part \childdocname{} of \childdocjob{}.|
\end{center}

%%%%%%%%%%%%%%%%%%%%%%%%%%%%%%%%%%%%%%%%%%%%%%%%%%%%%%%%%%%%%%%%%%%%%%%%%%%%%%%%
\subsection{Flags}
\label{sec:flags}

The package makes it easy to generate different versions
of the main or child documents.
To this end compilation flags can be defined
and assigned different default values.
They will be particularly useful in conjunction
with the forwarding mechanism described in \secref{sec:forward}.

For example, it may be useful to have a flag |\version|
which can be set to |draft| or |final|.
The document source will contain some conditional code
depending on the value of |\version|.
Suppose further, the flag should default to |final| for the main file
and to |draft| for child files
which is a natural assignment for editing the document.
This is achieved by placing the following code
in the preamble of the main document
(below the |\childdocmain| directive):
%
\begin{center}
\begin{tabular}{l}
|\ifchilddoc|\\
|\providecommand{\version}{draft}|\\
|\||else|\\
|\providecommand{\version}{final}|\\
|\||fi|
\end{tabular}
\end{center}
%
The definition by |\providecommand| makes sure
that previous definitions are not overwritten.
Further statements |\providecommand{\version}{...}|
can thus be added before the above code to override it.

For the main file, one might add a line
(between |\childdocmain| and the above block)
%
\begin{center}
|%\ifchilddoc\||else\providecommand{\version}{draft}\||fi|
\end{center}
%
which can be uncommented to produce a draft version.
Likewise one can add a line to the very top of a child file
(above the |\childdocof{|\textit{main}|}| directive)
%
\begin{center}
|%\providecommand{\version}{final}|
\end{center}
%
which can be uncommented to produce the final version of this child document.

%%%%%%%%%%%%%%%%%%%%%%%%%%%%%%%%%%%%%%%%%%%%%%%%%%%%%%%%%%%%%%%%%%%%%%%%%%%%%%%%
\subsection{Forwarding}
\label{sec:forward}

Different versions of the main or child documents
using compilation flags as described in \secref{sec:flags}
can be (permanently) stored in different files
for convenient compilation, viewing and distribution.
To this end, the package defines a command
to pass on compilation to a different file:

%%%%%%%%%%%%%%%%%%%%%%%%%%%%%%%%%%%%%%%%
\DescribeMacro{\childdocforward}
The command |\childdocforward| redirects processing to
another source file:
%
\begin{center}
\begin{tabular}{l}
|\input{childdoc.def}|\\
|\childdocforward[|\textit{main}|]{|\textit{dest}|}|\\
\end{tabular}
\end{center}
%
The argument \textit{dest} is the destination file
(without extension).
It should be the main file or one of the child files.
Note that further \textsf{childdoc} directives
such as |\childdocof| and |\childdocforward|
in the indicated file will be processed in this form.
The optional argument \textit{main}
passes on directly to the main file \textit{main}
while pretending to compile the child \textit{dest}.
This form behaves as if \textit{dest}
issues |\childdocof{|\textit{main}|}| right away,
and no further \textsf{childdoc} directives will be processed.

%%%%%%%%%%%%%%%%%%%%%%%%%%%%%%%%%%%%%%%%
\DescribeMacro{\...prefix}
In the alternative form |\childdocforwardprefix|,
%
\begin{center}
\begin{tabular}{l}
|\input{childdoc.def}|\\
|\childdocforwardprefix[|\textit{main}|]{|\textit{prefix}|}{|\textit{dest}|}|
\end{tabular}
\end{center}
%
the destination file is determined by a pattern
depending on the current file:
To make this work, the current file must be called
`{\textit{prefix}\hspace{0.2em}\textit{suffix}}'
with \textit{prefix} matching precisely the argument.
Processing is then passed on to the file
`{\textit{dest}\hspace{0.2em}\textit{suffix}}'.
Surely, the same effect is achieved by
directly specifying the
argument `{\textit{dest}\hspace{0.2em}\textit{suffix}}'
in the first form.
However, that requires to set up a different file
for each child. With the alternative form of the command
all these files can have exactly the same content
which simplifies setting them up and maintaining them.

For example, the following file |draft.tex|
with a compilation flag |\version| as described in \secref{sec:flags}
compiles the main document as a draft:
%
\begin{center}
\begin{tabular}{l}
|\def\version{draft}|\\
|\input{childdoc.def}|\\
|\childdocforward{|\textit{main}|}|
\end{tabular}
\end{center}
%
Likewise, the following files |final|\textit{nn}|.tex|
compile the final version of the child document
|child|\textit{nn}|.tex|:
%
\begin{center}
\begin{tabular}{l}
|\def\version{final}|\\
|\input{childdoc.def}|\\
|\childdocforwardprefix{final}{child}|
\end{tabular}
\end{center}
%

Note that when several versions of a main file and/or of each child file
are to be generated, it may be convenient to set up a |Makefile| or
shell script to automatise the process.

%%%%%%%%%%%%%%%%%%%%%%%%%%%%%%%%%%%%%%%%%%%%%%%%%%%%%%%%%%%%%%%%%%%%%%%%%%%%%%%%
\subsection{Command Line Processing}
\label{sec:commandline}

The effect of redirection files can also be achieved by invoking
the \LaTeX{} compiler with a more elaborate command line.
Most conveniently this should be done as part
of a shell script or a |Makefile|.

When using \textsf{childdoc} in the main file, the following
command lines effectively perform a redirection
(note that depending on the shell being used,
backslashes may have to be doubled: `|\|' $\to$ `|\\|'):
%
\begin{center}
|... -jobname "|\textit{target}|" |\\|"|[\textit{flags}]%
|\input{childdoc.def}\childdocforward[|\textit{main}|]{|\textit{dest}|}"|
\end{center}
%
Here \textit{target} is the name of the output file,
\textit{main} is the name of the main file
and \textit{dest} is the name of the main or child file to be processed
(all filenames without extensions).
The optional argument \textit{main} can be omitted
if \textit{main} matches \textit{dest}.
Optionally, compilation \textit{flags} can be defined via |\def| commands.
This command line makes the \TeX{} engine believe
it is compiling the file \textit{target}
whose content is specified as the latter parameter.
The provided code then forwards the processing to
\textit{main} or \textit{dest} as described in \secref{sec:forward}.

%%%%%%%%%%%%%%%%%%%%%%%%%%%%%%%%%%%%%%%%%%%%%%%%%%%%%%%%%%%%%%%%%%%%%%%%%%%%%%%%
\subsection{Include by Input}
\label{sec:input}

Including child documents by |\include| has some restrictions by design.
Most notably, the content of a child document always occupies
its own set of pages; pages cannot be shared between child documents.
Usually, this behaviour makes perfect sense
because each child document contain an essential part of the document.
However, in some situations it may be desirable to compose
a document from a collection of parts
without having mandatory page breaks between then.
For this case, the package
provides a mechanism to include parts
by |\input| which can also be processed individually.
However, by construction this mechanism
requires manual handling of the content to be output.

%%%%%%%%%%%%%%%%%%%%%%%%%%%%%%%%%%%%%%%%
\DescribeMacro{\ifchilddocmanual}
The main file should be prepared as usual, see \secref{sec:include}.
However, the document body must make a distinction
between processing of an individual part and of the main document, e.g.:
%
\begin{center}
\begin{tabular}{l}
|\ifchilddocmanual|\\
|\input{\childdocname}|\\
|\||else|\\
\textit{document body with }|\input{|\textit{part}|}|\\
|\||fi|
\end{tabular}
\end{center}
%
The conditional |\ifchilddocmanual| is true whenever
a part to be included by |\input| is being compiled,
and the name of the part is stored in |\childdocname|.

%%%%%%%%%%%%%%%%%%%%%%%%%%%%%%%%%%%%%%%%
\DescribeMacro{\childdocby}
Each part to be included by |\input| should start with:
%
\begin{center}
\begin{tabular}{l}
|\input{childdoc.def}|\\
|\childdocby{|\textit{main}|}|\\
\end{tabular}
\end{center}
%
The directive |\childdocby| is similar to |\childdocof|
described in \secref{sec:include},
but the subsequent selection of content must be done manually.
To that end, both |\ifchilddoc| and |\ifchilddocmanual|
will be true upon processing of a part,
and the name of the part is stored in |\childdocname|.
Note that |\jobname| will be set to the filename of the current part
so that each part receives an individual |.aux| file
that does not interfere with the |.aux| file(s) of the main document.
This behaviour can be altered by the alternative form
|\childdocby[*]{|\textit{main}|}| (with a non-empty optional argument)
which uses the |.aux| file of the main document
by setting |\jobname| to \textit{main}.

%%%%%%%%%%%%%%%%%%%%%%%%%%%%%%%%%%%%%%%%%%%%%%%%%%%%%%%%%%%%%%%%%%%%%%%%%%%%%%%%
\subsection{Driver Development}
\label{sec:driver}

The \textsf{childdoc} mechanism can also be use for the development
of definition files such as \LaTeX{} styles or classes.
This case differs from the above setup with multiple parts
included by |\include| in that no |\includeonly| should be invoked.
This can be achieved by starting the include file
(before |\ProvidesPackage|) with:
%
\begin{center}
\begin{tabular}{l}
|\input{childdoc.def}|\\
|\childdocforward{|\textit{main}|}|\\
\end{tabular}
\end{center}
%
or alternatively with:
%
\begin{center}
\begin{tabular}{l}
|\input{childdoc.def}|\\
|\childdocby{|\textit{main}|}|\\
\end{tabular}
\end{center}
%
Both forms have slightly different effects as described above.
The main file is prepared as usual, see \secref{sec:include}.

%%%%%%%%%%%%%%%%%%%%%%%%%%%%%%%%%%%%%%%%%%%%%%%%%%%%%%%%%%%%%%%%%%%%%%%%%%%%%%%%
\subsection{Legacy Detection}
\label{sec:detection}

The directive |\childdocmain| in the main file can detect
whether the complete document or merely a child is to be compiled
even without using the directive |\childdocof|.
This method is deprecated because it is less robust
and there is no compelling reason to use it;
it is merely provided for backward compatibility
and it may be removed in future versions.

If the detection mechanism is to be used,
it is mandatory to correctly specify
the filename of the main file as the argument of |\childdocmain|:
%
\begin{center}
\begin{tabular}{l}
|\input{childdoc.def}|\\
|\childdocmain{|\textit{main}|}|\\
\end{tabular}
\end{center}
%
If |\jobname| does not match the argument \textit{main} of |\childdocmain|,
it is assumed that |\jobname| points to the child file to be compiled.
When using |\childdocmain| with the main file specified as argument,
it suffices to start a child file
with just |\input{|\textit{main}|}|
without loading of the package and using |\childdocof|.
If instead all processing is done
with the appropriate \textsf{childdoc} directives,
the argument of \textit{main} of |\childdocmain| can be empty.

An alternative version of the command line processing described
in \secref{sec:commandline} using the detection mechanism reads:
%
\begin{center}
|... -jobname "|\textit{target}|" "|[\textit{flags}]%
[|\def\jobname{|\textit{dest}|}|]|\input{|\textit{main}|}"|
\end{center}

%%%%%%%%%%%%%%%%%%%%%%%%%%%%%%%%%%%%%%%%%%%%%%%%%%%%%%%%%%%%%%%%%%%%%%%%%%%%%%%%
\subsection{Manual Code}
\label{sec:manual}

In case one cannot be certain whether the definitions file |childdoc.def|
is installed on the target \TeX{} distribution
and one prefers not to ship it,
it is conceivable to paste a few relevant commands into the sources.

To that end, drop all statements |\input{childdoc.def}|
and perform the replacements as outlined below.
Instead of |\childdocmain{|\textit{main}|}| add the following code
to the top of the main file:
%
\begin{center}
\begin{tabular}{l}
|\||ifdefined\childdocname\endinput\||fi\newif\ifchilddoc|\\
|\edef\childdocname{\scantokens\expandafter{\jobname\noexpand}}|\\
|\def\childdocmain{|\textit{main}|}\||ifx\childdocmain\childdocname\||else|\\
|\childdoctrue\includeonly{\childdocname}\let\jobname\childdocmain\||fi|\\
\end{tabular}
\end{center}
%
Instead of |\childdocof{|\textit{main}|}| just include the main file
at the top of each child file:
%
\begin{center}
|\input{|\textit{main}|}|
\end{center}
%
A simple redirection |\childdocforward{|\textit{dest}|}| is achieved by:
%
\begin{center}
|\def\jobname{|\textit{dest}|}\input{\jobname}|
\end{center}
%
The redirection with prefix
|\childdocforwardprefix[|\textit{prefix}|]{|\textit{dest}|}|
is accomplished by:
%
\begin{center}
\begin{tabular}{l}
|{\edef\jobname{\scantokens\expandafter{\jobname\noexpand}}|\\
|\def\redirectjob |\textit{prefix}|#1~~~{\gdef\jobname{|\textit{dest}|#1}}|\\
|\expandafter\redirectjob\jobname~~~}\input{\jobname}|
\end{tabular}
\end{center}

In an alternative approach,
child documents can be compiled by a specific command line
without additional code or specific definitions:
%
\begin{center}
|... -jobname "|\textit{target}|" "|[\textit{flags}]%
|\includeonly{|\textit{dest}|}\input{|\textit{main}|}"|
\end{center}
%

%%%%%%%%%%%%%%%%%%%%%%%%%%%%%%%%%%%%%%%%%%%%%%%%%%%%%%%%%%%%%%%%%%%%%%%%%%%%%%%%
%%%%%%%%%%%%%%%%%%%%%%%%%%%%%%%%%%%%%%%%%%%%%%%%%%%%%%%%%%%%%%%%%%%%%%%%%%%%%%%%
\section{Information}

%%%%%%%%%%%%%%%%%%%%%%%%%%%%%%%%%%%%%%%%%%%%%%%%%%%%%%%%%%%%%%%%%%%%%%%%%%%%%%%%
\subsection{Copyright}

Copyright \copyright{} 2017--2018 Niklas Beisert

This work may be distributed and/or modified under the
conditions of the \LaTeX{} Project Public License, either version 1.3
of this license or (at your option) any later version.
The latest version of this license is in
  \url{http://www.latex-project.org/lppl.txt}
and version 1.3 or later is part of all distributions of \LaTeX{}
version 2005/12/01 or later.

This work has the LPPL maintenance status `maintained'.

The Current Maintainer of this work is Niklas Beisert.

This work consists of the files |README.txt|, |childdoc.ins| and |childdoc.dtx|
as well as the derived files |childdoc.def|, |cdocsamp.tex|
with |cdocsch1.tex|, |cdocsch2.tex|, |cdocspt3.tex|, |cdocspt4.tex|,
|cdocsdrf.tex|, |cdocsfn1.tex|, |cdocsfn2.tex|
as well as |childdoc.pdf|.

%%%%%%%%%%%%%%%%%%%%%%%%%%%%%%%%%%%%%%%%%%%%%%%%%%%%%%%%%%%%%%%%%%%%%%%%%%%%%%%%
\subsection{Files and Installation}

The package consists of the files:
%
\begin{center}
\begin{tabular}{ll}
    |README.txt|   & readme file \\
    |childdoc.ins| & installation file \\
    |childdoc.dtx| & source file \\
    |childdoc.def| & definition file \\
    |cdocsamp.tex| & sample main file \\
    |cdocsch1.tex| & sample include file \\
    |cdocsch2.tex| & sample include file \\
    |cdocspt3.tex| & sample part file \\
    |cdocspt4.tex| & sample part file \\
    |cdocsdrf.tex| & sample redirection file \\
    |cdocsfn1.tex| & sample redirection file \\
    |cdocsfn2.tex| & sample redirection file \\
    |childdoc.pdf| & manual
\end{tabular}
\end{center}
%
The distribution consists of the files
|README.txt|, |childdoc.ins| and |childdoc.dtx|.
%
\begin{itemize}
\item
Run (pdf)\LaTeX{} on |childdoc.dtx|
to compile the manual |childdoc.pdf| (this file).
\item
Run \LaTeX{} on |childdoc.ins| to create the definitions file |childdoc.def|
and the sample |cdocsamp.tex| with include files
|cdocsch1.tex|, |cdocsch2.tex|, |cdocspt3.tex|, |cdocspt4.tex|,
|cdocsdrf.tex|, |cdocsfn1.tex|, |cdocsfn2.tex|.
Then copy the file |childdoc.def| to an appropriate directory of your \LaTeX{}
distribution, e.g.\ \textit{texmf-root}|/tex/latex/childdoc|.
\end{itemize}

%%%%%%%%%%%%%%%%%%%%%%%%%%%%%%%%%%%%%%%%%%%%%%%%%%%%%%%%%%%%%%%%%%%%%%%%%%%%%%%%
\subsection{Related CTAN Packages}

There are several other packages which offer a similar functionality:
%
\begin{itemize}
\item
The packages
\href{http://ctan.org/pkg/docmute}{\textsf{docmute}},
\href{http://ctan.org/pkg/includex}{\textsf{includex}} and
\href{http://ctan.org/pkg/standalone}{\textsf{standalone}}
provide commands to include only the document body of
a child file thus allowing both files to be compiled individually.
\item
The packages \href{http://ctan.org/pkg/subdocs}{\textsf{subdocs}}
and \href{http://ctan.org/pkg/subfiles}{\textsf{subfiles}}
provide structures in which the main and child documents can be
encapsulated and allowing them to be compiled individually.
The inclusion mechanism is different from the conventional |\include|.
\item
The package \href{http://ctan.org/pkg/combine}{\textsf{combine}}
is an elaborate solution to combine several documents into one.
\end{itemize}
%
See also the CTAN topic \href{http://ctan.org/topic/subdocs}{\textsf{subdocs}}
for further related packages.
The present package differs from the above solutions in that
a document structure constructed with the conventional |\include| mechanism
just needs two extra commands at the top of every file
such that all constituent files can be compiled individually.

%%%%%%%%%%%%%%%%%%%%%%%%%%%%%%%%%%%%%%%%%%%%%%%%%%%%%%%%%%%%%%%%%%%%%%%%%%%%%%%%
%\subsection{Feature Suggestions}
%
%The following is a list of features which may be useful for future
%versions of this package:
%%
%\begin{itemize}
%\item
%\ldots
%\end{itemize}

%%%%%%%%%%%%%%%%%%%%%%%%%%%%%%%%%%%%%%%%%%%%%%%%%%%%%%%%%%%%%%%%%%%%%%%%%%%%%%%%
\subsection{Revision History}

%%%%%%%%%%%%%%%%%%%%%%%%%%%%%%%%%%%%%%%%
\paragraph{v2.0:} 2018/12/30

\begin{itemize}
\item
immediate forward processing
\item
added |\childdocby| mechanism
\item
manual restructured
\end{itemize}

%%%%%%%%%%%%%%%%%%%%%%%%%%%%%%%%%%%%%%%%
\paragraph{v1.6:} 2018/01/17

\begin{itemize}
\item
application for development of include files
\item
corrections to manual
\end{itemize}

%%%%%%%%%%%%%%%%%%%%%%%%%%%%%%%%%%%%%%%%
\paragraph{v1.5:} 2017/05/21

\begin{itemize}
\item
more complete structuring introduced
\item
|\childdocof| introduced
\item
|\childdoc| renamed to |\childdocmain|
\item
|\childredirect| renamed to |\childdocforward| and |\childdocforwardprefix|
and functionality expanded
\end{itemize}

%%%%%%%%%%%%%%%%%%%%%%%%%%%%%%%%%%%%%%%%
\paragraph{v1.0:} 2017/04/27

\begin{itemize}
\item
manual and install package
\item
first version published on CTAN
\end{itemize}

%%%%%%%%%%%%%%%%%%%%%%%%%%%%%%%%%%%%%%%%
\paragraph{v0.6:} 2017/04/26

\begin{itemize}
\item
redirection mechanism added
\end{itemize}

%%%%%%%%%%%%%%%%%%%%%%%%%%%%%%%%%%%%%%%%
\paragraph{v0.5:} 2017/04/26

\begin{itemize}
\item
functionality in definition file
\end{itemize}


%%%%%%%%%%%%%%%%%%%%%%%%%%%%%%%%%%%%%%%%%%%%%%%%%%%%%%%%%%%%%%%%%%%%%%%%%%%%%%%%
%%%%%%%%%%%%%%%%%%%%%%%%%%%%%%%%%%%%%%%%%%%%%%%%%%%%%%%%%%%%%%%%%%%%%%%%%%%%%%%%
%%%%%%%%%%%%%%%%%%%%%%%%%%%%%%%%%%%%%%%%%%%%%%%%%%%%%%%%%%%%%%%%%%%%%%%%%%%%%%%%
\appendix

\settowidth\MacroIndent{\rmfamily\scriptsize 000\ }

 \DocInput{childdoc.dtx}

\end{document}
%</driver>
% \fi
%
% %%%%%%%%%%%%%%%%%%%%%%%%%%%%%%%%%%%%%%%%%%%%%%%%%%%%%%%%%%%%%%%%%%%%%%%%%%%%%%
% %%%%%%%%%%%%%%%%%%%%%%%%%%%%%%%%%%%%%%%%%%%%%%%%%%%%%%%%%%%%%%%%%%%%%%%%%%%%%%
% \section{Sample}
%\iffalse
%<*samplemain>
%\fi
%
% The following presents a sample document
% with two chapters, two parts, a title page,
% a compile flag as well as three forwarding files to set the flag.
% It consists of eight |.tex| files:
% \begin{center}
% \begin{tabular}{ll}
% |cdocsamp.tex|&main file\\
% |cdocsch1.tex|&include file for chapter 1\\
% |cdocsch2.tex|&include file for chapter 2\\
% |cdocspt3.tex|&include file for part 3\\
% |cdocspt4.tex|&include file for part 4\\
% |cdocsdrf.tex|&forwarding file for main file in draft mode\\
% |cdocsfi1.tex|&forwarding file for final version of chapter 1\\
% |cdocsfi2.tex|&forwarding file for final version of chapter 2\\
% \end{tabular}
% \end{center}
% Each of the eight files can be compiled directly by the \LaTeX{} compiler.
%
% %%%%%%%%%%%%%%%%%%%%%%%%%%%%%%%%%%%%%%
% \paragraph{Main File.}
%
% The main file is called |cdocsamp.tex|.
%
% Load the \textsf{childdoc} definitions and
% declare the filename for the main document:
%    \begin{macrocode}
\input{childdoc.def}
\childdocmain{}
%    \end{macrocode}

% Optional override for |\version| flag:
%    \begin{macrocode}
%%\ifchilddoc\else\providecommand{\version}{draft}\fi
%    \end{macrocode}

% Define the default values for the |\version| flag
% (|final| for the main file and |draft| for childs):
%    \begin{macrocode}
\ifchilddoc
\providecommand{\version}{draft}
\else
\providecommand{\version}{final}
\fi
%    \end{macrocode}

% Load the standard document class:
%    \begin{macrocode}
\documentclass[12pt]{article}
%    \end{macrocode}

% Start the document body:
%    \begin{macrocode}
\begin{document}
%    \end{macrocode}

% Declare a title page.
% Print title, part of document being processed and version flag:
%    \begin{macrocode}
\addtocounter{page}{-1}
\begin{center}
{\LARGE\bfseries{}childdoc example\par}
\vspace{1cm}
\ifchilddoc
\ifchilddocmanual part\else chapter\fi:
`\childdocname' of `\childdocjob'\par
\else
main document: `\childdocjob'\par
\fi
version: \version\par
\end{center}
\newpage
%    \end{macrocode}

% Manually include selected file,
% otherwise process as usual:
%    \begin{macrocode}
\ifchilddocmanual
\section*{part `\childdocname'}
\input{\childdocname}
\else
%    \end{macrocode}

% Include the two chapters:
%    \begin{macrocode}
\include{cdocsch1}
\include{cdocsch2}
%    \end{macrocode}

% Include the two parts unless only chapters should be displayed:
%    \begin{macrocode}
\ifchilddoc\else
\section{part three}
\input{cdocspt3}
\section{part four}
\input{cdocspt4}
\fi
%    \end{macrocode}

% Process as usual until here:
%    \begin{macrocode}
\fi
%    \end{macrocode}

% End of document body:
%    \begin{macrocode}
\end{document}
%    \end{macrocode}
%\iffalse
%</samplemain>
%\fi
%
% %%%%%%%%%%%%%%%%%%%%%%%%%%%%%%%%%%%%%%
% \paragraph{Chapter Include Files.}
%
% The include files are called |cdocsch1.tex| and |cdocsch2.tex|.
%
%\iffalse
%<*samplechap1|samplechap2>
%\fi

% Optional override for |\version| flag:
%    \begin{macrocode}
%%\providecommand{\version}{final}
%    \end{macrocode}

% Include the main document:
%    \begin{macrocode}
\input{childdoc.def}
\childdocof{cdocsamp}
%    \end{macrocode}

%\iffalse
%</samplechap1|samplechap2>
%\fi
%
%\iffalse
%<*samplechap1>
%\fi
% Some text for chapter 1:
%    \begin{macrocode}
\section{one}
some text in chapter one
%    \end{macrocode}

%\iffalse
%</samplechap1>
%\fi
% Some text for chapter 2:
%\iffalse
%<*samplechap2>
%\fi
%    \begin{macrocode}
\section{two}
more text in chapter two
%    \end{macrocode}

%\iffalse
%</samplechap2>
%\fi
%
% %%%%%%%%%%%%%%%%%%%%%%%%%%%%%%%%%%%%%%
% \paragraph{Part Include Files.}
%
% The include files are called |cdocspt3.tex| and |cdocspt4.tex|.
%
%\iffalse
%<*samplepart3|samplepart4>
%\fi

% Optional override for |\version| flag:
%    \begin{macrocode}
%%\providecommand{\version}{final}
%    \end{macrocode}

% Include the main document:
%    \begin{macrocode}
\input{childdoc.def}
\childdocby{cdocsamp}
%    \end{macrocode}

%\iffalse
%</samplepart3|samplepart4>
%\fi
%
%\iffalse
%<*samplepart3>
%\fi
% Some text for part 3:
%    \begin{macrocode}
some text in part three
%    \end{macrocode}

%\iffalse
%</samplepart3>
%\fi
% Some text for part 4:
%\iffalse
%<*samplepart4>
%\fi
%    \begin{macrocode}
more text in part four
%    \end{macrocode}

%\iffalse
%</samplepart4>
%\fi
%
% %%%%%%%%%%%%%%%%%%%%%%%%%%%%%%%%%%%%%%
% \paragraph{Forwarding for a Complete Draft.}
%
% The following forwarding file |cdocsdrf.tex|
% compiles the main document in draft mode:
%\iffalse
%<*sampledraft>
%\fi
%    \begin{macrocode}
\def\version{draft}
\input{childdoc.def}
\childdocforward{cdocsamp}
%    \end{macrocode}

%\iffalse
%</sampledraft>
%\fi
%
% %%%%%%%%%%%%%%%%%%%%%%%%%%%%%%%%%%%%%%
% \paragraph{Forwarding for Final Version of the Chapters.}
%
% The following forwarding files |cdocsfn1.tex| and |cdocsfn2.tex|
% (with identical content)
% compile the final versions of the child documents
% |cdocsch1.tex| and |cdocsch2.tex|, respectively:
%\iffalse
%<*samplefinal>
%\fi
%    \begin{macrocode}
\def\version{final}
\input{childdoc.def}
\childdocforwardprefix[cdocsamp]{cdocsfn}{cdocsch}
%    \end{macrocode}

%\iffalse
%</samplefinal>
%\fi
%
% %%%%%%%%%%%%%%%%%%%%%%%%%%%%%%%%%%%%%%
% \paragraph{Command Line Processing.}
%
% The following three command lines generate the output files
% |cdocscld|, |cdocscl1| and |cdocscl2|
% which should be identical to
% |cdocsdrf|, |cdocsch1| and |cdocsfn2|, respectively:
% \begin{center}
% \begin{tabular}{l}
% |latex -jobname cdocscld \|\\
% |  "\def\version{draft}\input{childdoc.def}\childdocforward{cdocsamp}"|\\
% |latex -jobname cdocscl1 \|\\
% |  "\input{childdoc.def}\childdocforward[cdocsamp]{cdocsch1}"|\\
% |latex -jobname cdocscl2 \|\\
% |  "\def\version{final}\input{childdoc.def}\childdocforward{cdocsch2}"|
% \end{tabular}
% \end{center}
% Note that the trailing backslash on each first line
% merely continues the input to the second line
% (for convenient cut ant paste).
% Furthermore, the command |latex| can be replaced by any
% of its alternative versions such as |pdflatex|.
%
% %%%%%%%%%%%%%%%%%%%%%%%%%%%%%%%%%%%%%%%%%%%%%%%%%%%%%%%%%%%%%%%%%%%%%%%%%%%%%%
% %%%%%%%%%%%%%%%%%%%%%%%%%%%%%%%%%%%%%%%%%%%%%%%%%%%%%%%%%%%%%%%%%%%%%%%%%%%%%%
% \section{Implementation}
%\iffalse
%<*package>
%\fi
%
% This section describes the definitions file |childdoc.def|.

% The definitions cannot be loaded using |\usepackage| or |\RequirePackage|
% which has a mechanism to prevent loading a style file more than once.
% When loading the definitions by means of |\input|
% multiple instances have to be prevented manually:
%\iffalse
%This code needs to be before the `\ProvidesFile' directive
%which is defined at the beginning of this file.
%Therefore it is also placed there and commented out here.
%</package>
%<*discard>
%\fi
%    \begin{macrocode}
\ifdefined\childdocmain\endinput\fi
%    \end{macrocode}
%\iffalse
%</discard>
%<*package>
%\fi
%
% \macro{\ifchilddoc}
% \macro{\ifchilddocmanual}
% The conditional |\ifchilddoc| tells whether a
% child (true) or main (false) document is being compiled.
% The conditional |\ifchilddocmanual| tells whether
% the |\includeonly| mechanism is used (false) or
% the selection of child files must be performed manually (true).
% The definitions initialise to false:
%    \begin{macrocode}
\newif\ifchilddoc
\newif\ifchilddocmanual
%    \end{macrocode}

% \macro{\childdocname}
% \macro{\childdocjob}
% The macro |\childdocname| stores the name of the main document
% to be compiled. The macro |\childdocjob| stores the name of
% the document on which the \LaTeX{} compiler was originally invoked.
% The content of |\jobname| cannot be compared
% to filenames specified in the source due to different catcodes.
% The following code rescans |\jobname|, stores the result
% in |\childdocname| and saves a copy in |\childdocjob|:
%    \begin{macrocode}
\edef\childdocname{\scantokens\expandafter{\jobname\noexpand}}
\let\childdocjob\childdocname
%    \end{macrocode}

% \macro{\childdocdisable}
% The macro |\childdocdisable| prevents the main file
% from being processed more than once.
% At this stage, the main document command |\childdocmain|
% is assumed to be called once again where it should do nothing.
% Any subsequent call to it should prevent
% a secondary processing of the main document
% It overwrites the forwarding commands
% |\childdocof| and |\childdocforward|
% with empty macros to prevent further inclusions of the main document:
%    \begin{macrocode}
\newcommand{\childdocdisable}
{
  \renewcommand{\childdocmain}[1]{\renewcommand{\childdocmain}[1]{\endinput}}
  \renewcommand{\childdocof}[1]{}
  \renewcommand{\childdocby}[2][]{}
  \renewcommand{\childdocforward}[2][]{}
  \renewcommand{\childdocdisable}{}
}
%    \end{macrocode}

% \macro{\childdocmain}
% The macro |\childdocmain| is to be called at the top of the main file
% with nothing or the main filename (without extension) as argument.
% First, it breaks loops.
% If the argument is not empty and does not match |\childdocname|
% (which is set by the first inclusion of |childdoc.def|),
% |\ifchilddoc| is set to true, |\includeonly| is applied to the child file
% and |\jobname| is set to the main file
% (for proper handling of |.aux| files):
%    \begin{macrocode}
\newcommand{\childdocmain}[1]
{
  \childdocdisable\childdocmain{}
  \if?#1?\else
    \begingroup
      \def\childdoctmp{#1}
      \ifx\childdoctmp\childdocname
        \def\childdoctmp{}
      \else
        \def\childdoctmp
        {
          \childdoctrue
          \includeonly{\childdocname}
          \def\childdocjob{#1}
          \def\jobname{#1}
        }
      \fi
      \expandafter
    \endgroup
    \childdoctmp
  \fi
}
%    \end{macrocode}

% \macro{\childdocof}
% The command |\childdocof| redirects
% compilation to the main file |#1|.
%    \begin{macrocode}
\newcommand{\childdocof}[1]
{
  \childdocdisable
  \childdoctrue
  \includeonly{\childdocname}
  \def\jobname{#1}
  \def\childdocjob{#1}
  \input{#1}
}
%    \end{macrocode}

% \macro{\childdocby}
% The command |\childdocby| ....
%    \begin{macrocode}
\newcommand{\childdocby}[2][]
{
  \childdocdisable
  \childdoctrue
  \childdocmanualtrue
  \if?#1?\else
    \def\jobname{#2}
  \fi
  \def\childdocjob{#2}
  \input{#2}
  \endinput
}
%    \end{macrocode}

% \macro{\childdocforward}
% The command |\childdocforward| redirects
% compilation to the main file or
% (if the optional argument is given) a child file.
% Parameters are set as if the main file
% or a child file starting with |\childdocof| was compiled.
% Then compilation is handed over to the main file:
%    \begin{macrocode}
\newcommand{\childdocforward}[2][]
{
  \begingroup
    \if?#1?
      \def\childdoctmp
      {
        \def\childdocname{#2}
        \def\childdocjob{#2}
        \def\jobname{#2}
        \input{#2}
        \endinput
      }
    \else
      \def\childdoctmp
      {
        \childdocdisable
        \def\childdocname{#2}
        \childdoctrue
        \includeonly{#2}
        \def\childdocjob{#1}
        \def\jobname{#1}
        \input{#1}
        \endinput
      }
    \fi
    \expandafter
  \endgroup
  \childdoctmp
}
%    \end{macrocode}

% \macro{\childdocforwardprefix}
% The command |\childdocforwardprefix| redirects
% compilation to the main or a child file by means of a pattern.
% The prefix |#1| in the current filename is replaced by |#2|
% and the suffix of the current filename is kept
% (it is assumed that the filename does not contain the substring `|~~~|'
% which is used as a delimiter).
% Compilation is handed over to the new file by |\childdocforward|:
%    \begin{macrocode}
\newcommand{\childdocforwardprefix}[3][]
{
  \begingroup
    \def\childdocextract #2##1~~~{\def\childdoctmp{\childdocforward[#1]{#3##1}}}
    \expandafter\childdocextract\childdocname~~~
    \expandafter
  \endgroup
  \childdoctmp
}
%    \end{macrocode}

% \macro{\childdoc}
% The deprecated macro |\childdoc| is a legacy version of |\childdocmain|:
%    \begin{macrocode}
\newcommand{\childdoc}{\childdocmain}
%    \end{macrocode}

% \macro{\childdocredirect}
% The deprecated macro |\childdocredirect| is a legacy version
% of |\childdocforward| and |\childdocforwardprefix|:
%    \begin{macrocode}
\newcommand{\childdocredirect}[2][]
{
  \begingroup
    \if?#1?
      \def\childdoctmp{\childdocforward{#2}}
    \else
      \def\childdoctmp{\childdocforwardprefix{#1}{#2}}
    \fi
    \expandafter
  \endgroup
  \childdoctmp
}
%    \end{macrocode}

%\iffalse
%</package>
%\fi
%
\endinput
|\\
|\childdocof{|\textit{main}|}|\\
\end{tabular}
\end{center}
at the top of every child file \textit{child}
which is included by |\include{|\textit{child}|}|
from within the main file
(or at least for those files to be compiled individually).
The argument \textit{main} must be the filename of the main file.

There are a couple of
considerations in setting up the main and child documents:

%%%%%%%%%%%%%%%%%%%%%%%%%%%%%%%%%%%%%%%%
\paragraph{Restrictions.}

Please note the following restrictions:
\begin{itemize}
\item
|\childdocmain| must be called with one argument \textit{main}
to ensure compatibility with earlier version of the package.
It must either be empty (|\childdocmain{}|)
or precisely match the filename of the main file in which it is specified.
See \secref{sec:detection} for further information.
\item
The filename \textit{main} must be specified without the |.tex| extension.
\item
The filename \textit{main} is case sensitive
(even in case-insensitive file systems)
due to internal string comparison.
\item
The argument \textit{main} should be fully expanded, it cannot be a macro.
\item
Subdirectories and special characters should be avoided in filenames.
\item
The command |\childdocmain{|\textit{main}|}| must be followed by a whitespace.
It should not be followed immediately by another command
or by a comment mark `|%|'.
This is because the \TeX{} parser reads the token immediately following
the argument of |\childdocmain| and puts it
at the beginning of every child section;
however, a white\-space is ignored.
\end{itemize}

%%%%%%%%%%%%%%%%%%%%%%%%%%%%%%%%%%%%%%%%
\paragraph{Content of Main File.}

It is advisable to place all content in the child files included by |\include|.
Any output contained in the main file will appear in all child documents
unless suppressed manually;
it cannot be suppressed automatically by the |\includeonly| directive
and thus should normally be avoided.
A method to include some content in the main file
by means of conditional processing is described in \secref{sec:conditional}.

%%%%%%%%%%%%%%%%%%%%%%%%%%%%%%%%%%%%%%%%
\paragraph{Page Numbering.}

When only a part of the document is compiled,
the appropriate numbering of pages
(as well as other status parameters)
is determined from the |.aux| files.
The latter contain information from previous passes.
However this information needs to propagate through
all intermediate child documents.
Therefore the page numbering in child documents may well
be inconsistent until the complete document is compiled at least once.

A useful (if unconventional) way to always ensure a consistent
page numbering is to restart the numbering in each child document
and denote the pages by `\textit{child}|.|\textit{page}'
where \textit{child} represents the chapter/section number of the child file.
This can be achieved by the command
|\numberwithin{page}{|\textit{child}|}|
of the \textsf{amsmath} package
where \textit{child} can be |chapter| or |section|
depending on the chosen structuring.
Alternatively, one can modify the macro |\thepage| appropriately
and reset the counter |page| at the start of each child file.

%%%%%%%%%%%%%%%%%%%%%%%%%%%%%%%%%%%%%%%%%%%%%%%%%%%%%%%%%%%%%%%%%%%%%%%%%%%%%%%%
\subsection{Conditional Processing}
\label{sec:conditional}

The package provides a mechanism to compile different versions
of a document. To customise the versions further some conditional processing
can come in handy to distinguish which version is being compiled.
The package provides two macros to describe the compilation context:

%%%%%%%%%%%%%%%%%%%%%%%%%%%%%%%%%%%%%%%%
\DescribeMacro{\ifchilddoc}
The conditional |\ifchilddoc| distinguishes between the compilation of
child documents and the main document:
%
\begin{center}
|\ifchilddoc |\textit{child-code}| |[|\||else |\textit{main-code}]| \||fi|
\end{center}

%%%%%%%%%%%%%%%%%%%%%%%%%%%%%%%%%%%%%%%%
\DescribeMacro{\childdocname}
\DescribeMacro{\childdocjob}
The macro |\childdocname| contains the filename (without extension)
of the main or child file being processed.
Note that |\childdocjob| will always contain the name of the main file.

%%%%%%%%%%%%%%%%%%%%%%%%%%%%%%%%%%%%%%%%
\paragraph{Title Page.}

Conditional processing can be used to include a title or banner page
in the main document when proper precautions are taken.
Importantly, the code in the main file should ensure that the page counter
(as well as other status parameters which are stored in the |.aux| files)
takes the same value after the conditional processing.
Otherwise the page numbers may take divergent values
depending on which part is compiled.

For example, a title page could be declared by:
%
\begin{center}
\begin{tabular}{l}
|\ifchilddoc\||else|\\
|\addtocounter{page}{-1}|\\
\textit{code for title page}\\
|\newpage|\\
|\||fi|
\end{tabular}
\end{center}
%
A banner page for the child documents can be generated by:
%
\begin{center}
\begin{tabular}{l}
|\ifchilddoc|\\
|\addtocounter{page}{-1}|\\
\textit{code for banner page}\\
|\newpage|\\
|\||fi|
\end{tabular}
\end{center}
%
Here one could write a message such as:
\begin{center}
|This is the part \childdocname{} of \childdocjob{}.|
\end{center}

%%%%%%%%%%%%%%%%%%%%%%%%%%%%%%%%%%%%%%%%%%%%%%%%%%%%%%%%%%%%%%%%%%%%%%%%%%%%%%%%
\subsection{Flags}
\label{sec:flags}

The package makes it easy to generate different versions
of the main or child documents.
To this end compilation flags can be defined
and assigned different default values.
They will be particularly useful in conjunction
with the forwarding mechanism described in \secref{sec:forward}.

For example, it may be useful to have a flag |\version|
which can be set to |draft| or |final|.
The document source will contain some conditional code
depending on the value of |\version|.
Suppose further, the flag should default to |final| for the main file
and to |draft| for child files
which is a natural assignment for editing the document.
This is achieved by placing the following code
in the preamble of the main document
(below the |\childdocmain| directive):
%
\begin{center}
\begin{tabular}{l}
|\ifchilddoc|\\
|\providecommand{\version}{draft}|\\
|\||else|\\
|\providecommand{\version}{final}|\\
|\||fi|
\end{tabular}
\end{center}
%
The definition by |\providecommand| makes sure
that previous definitions are not overwritten.
Further statements |\providecommand{\version}{...}|
can thus be added before the above code to override it.

For the main file, one might add a line
(between |\childdocmain| and the above block)
%
\begin{center}
|%\ifchilddoc\||else\providecommand{\version}{draft}\||fi|
\end{center}
%
which can be uncommented to produce a draft version.
Likewise one can add a line to the very top of a child file
(above the |\childdocof{|\textit{main}|}| directive)
%
\begin{center}
|%\providecommand{\version}{final}|
\end{center}
%
which can be uncommented to produce the final version of this child document.

%%%%%%%%%%%%%%%%%%%%%%%%%%%%%%%%%%%%%%%%%%%%%%%%%%%%%%%%%%%%%%%%%%%%%%%%%%%%%%%%
\subsection{Forwarding}
\label{sec:forward}

Different versions of the main or child documents
using compilation flags as described in \secref{sec:flags}
can be (permanently) stored in different files
for convenient compilation, viewing and distribution.
To this end, the package defines a command
to pass on compilation to a different file:

%%%%%%%%%%%%%%%%%%%%%%%%%%%%%%%%%%%%%%%%
\DescribeMacro{\childdocforward}
The command |\childdocforward| redirects processing to
another source file:
%
\begin{center}
\begin{tabular}{l}
|% \iffalse
%
% childdoc.dtx Copyright (C) 2017-2018 Niklas Beisert
%
% This work may be distributed and/or modified under the
% conditions of the LaTeX Project Public License, either version 1.3
% of this license or (at your option) any later version.
% The latest version of this license is in
%   http://www.latex-project.org/lppl.txt
% and version 1.3 or later is part of all distributions of LaTeX
% version 2005/12/01 or later.
%
% This work has the LPPL maintenance status `maintained'.
%
% The Current Maintainer of this work is Niklas Beisert.
%
% This work consists of the files childdoc.dtx and childdoc.ins
% and the derived files childdoc.def and cdocsamp.tex with
% cdocsch1.tex, cdocsch2.tex, cdocsdrf.tex, cdocsfn1.tex, cdocsfn2.tex.
%
%<package>\ifdefined\childdocmain\endinput\fi
%<package>\ProvidesFile{childdoc.def}[2018/12/30 v2.0 child document driver]
%<samplemain>\ProvidesFile{cdocsamp.tex}[2018/12/30 v2.0 sample for childdoc]
%<*driver>
%\ProvidesFile{childdoc.drv}[2018/12/30 v2.0 childdoc reference manual file]
\PassOptionsToClass{10pt,a4paper}{article}
\documentclass{ltxdoc}

\usepackage[margin=35mm]{geometry}
\usepackage{hyperref}
\usepackage{hyperxmp}
\usepackage[usenames]{color}

\hypersetup{colorlinks=true}
\hypersetup{pdfstartview=FitH}
\hypersetup{pdfpagemode=UseNone}
\hypersetup{pdfsource={}}
\hypersetup{pdflang={en-UK}}
\hypersetup{pdfcopyright={Copyright 2017-2018 Niklas Beisert.
  This work may be distributed and/or modified under the
  conditions of the LaTeX Project Public License, either version 1.3
  of this license or (at your option) any later version.}}
\hypersetup{pdflicenseurl={http://www.latex-project.org/lppl.txt}}
\hypersetup{pdfcontactaddress={ETH Zurich, ITP, HIT K,
  Wolfgang-Pauli-Strasse 27}}
\hypersetup{pdfcontactpostcode={8093}}
\hypersetup{pdfcontactcity={Zurich}}
\hypersetup{pdfcontactcountry={Switzerland}}
\hypersetup{pdfcontactemail={nbeisert@itp.phys.ethz.ch}}
\hypersetup{pdfcontacturl={http://people.phys.ethz.ch/\xmptilde nbeisert/}}

\newcommand{\secref}[1]{\hyperref[#1]{section \ref*{#1}}}

\parskip1ex
\parindent0pt
\let\olditemize\itemize
\def\itemize{\olditemize\parskip0pt}

\begin{document}

\title{The \textsf{childdoc} Package}
\hypersetup{pdftitle={The childdoc Package}}
\author{Niklas Beisert\\[2ex]
  Institut f\"ur Theoretische Physik\\
  Eidgen\"ossische Technische Hochschule Z\"urich\\
  Wolfgang-Pauli-Strasse 27, 8093 Z\"urich, Switzerland\\[1ex]
  \href{mailto:nbeisert@itp.phys.ethz.ch}
  {\texttt{nbeisert@itp.phys.ethz.ch}}}
\hypersetup{pdfauthor={Niklas Beisert}}
\hypersetup{pdfsubject={Manual for the LaTeX2e Package childdoc}}
\date{30 December 2018, \textsf{v2.0}}
\maketitle

\begin{abstract}\noindent
\textsf{childdoc} is a \LaTeXe{} package
that enables the direct compilation
of document sections included by |\include|
to individual files.
\end{abstract}

\begingroup
\parskip0ex
\tableofcontents
\endgroup

%%%%%%%%%%%%%%%%%%%%%%%%%%%%%%%%%%%%%%%%%%%%%%%%%%%%%%%%%%%%%%%%%%%%%%%%%%%%%%%%
%%%%%%%%%%%%%%%%%%%%%%%%%%%%%%%%%%%%%%%%%%%%%%%%%%%%%%%%%%%%%%%%%%%%%%%%%%%%%%%%
\section{Introduction}

\LaTeX{} provides a mechanism to structure a large document (such as a book)
into a main file and several child files (containing the chapters)
using the |\include| command.
This mechanism is beneficial for documents
which span hundreds of pages in order to
make the source file(s) more manageable.
Moreover, compilation can be restricted to
selected child files by means of the |\includeonly| command.
The latter feature can be used to reduce the compilation time while editing
(this was significantly more useful in the earlier days of \LaTeX{})
or to generate a smaller document which is easier to navigate.
Another application of |\includeonly| is to generate
documents consisting of selected parts of the complete document.

However, there are a few drawbacks of the plain |\include| mechanism:
\begin{itemize}
\item
The child files cannot be compiled on their own,
they can only be compiled via the main file.
A naive editing environment
(such as a text editor with an option
to have the current file processed by \LaTeX)
may require one to switch to the main file before compiling;
attempting to compile the child file produces errors.
\item
The main file must be modified (each time)
to adjust the |\includeonly| command
to the present needs. This easily leaves the main file in a messy state.
\item
The generated document will always carry the filename
of the main document. This is inconvenient if
several child files are to be compiled and
to be kept for distribution.
\end{itemize}

The present package provides a simple interface
to make child files individually compilable by \LaTeX{}.
Compiling a child file then has the same effect as compiling
the main file with an |\includeonly| command
to select the appropriate child.
Moreover the generated document will carry the name of the child
rather than the main file.
This resolves all three above issues.

This feature is meant to make the editing of books,
thesis documents and lecture notes somewhat more convenient.
However, the package can also be used efficiently for
composing a series of documents (such as exercise sheets)
which are typically distributed individually.
It then assists the author in generating the individual documents
(potentially in different versions)
as well as a document containing the collected series.
Another application is in developing style files
or other kinds of included material
where compilation of the style file could redirect
to a sample or test file.

%%%%%%%%%%%%%%%%%%%%%%%%%%%%%%%%%%%%%%%%%%%%%%%%%%%%%%%%%%%%%%%%%%%%%%%%%%%%%%%%
%%%%%%%%%%%%%%%%%%%%%%%%%%%%%%%%%%%%%%%%%%%%%%%%%%%%%%%%%%%%%%%%%%%%%%%%%%%%%%%%
\section{Usage}

First of all, the package \textsf{childdoc} is \emph{not} a standard
\LaTeXe{} |.sty| style file! Therefore it needs to be invoked in
a non-standard way.

%%%%%%%%%%%%%%%%%%%%%%%%%%%%%%%%%%%%%%%%%%%%%%%%%%%%%%%%%%%%%%%%%%%%%%%%%%%%%%%%
\subsection{Included Files}
\label{sec:include}

%%%%%%%%%%%%%%%%%%%%%%%%%%%%%%%%%%%%%%%%
\DescribeMacro{\childdocmain}
To use the package, add the commands
\begin{center}
\begin{tabular}{l}
|\input{childdoc.def}|\\
|\childdocmain{}|\\
\end{tabular}
\end{center}
at the very top of the main \LaTeX{} file,
in particular \emph{before} the |\documentclass| statement!
The argument of |\childdocmain| should be left empty
(but it must be present).

%%%%%%%%%%%%%%%%%%%%%%%%%%%%%%%%%%%%%%%%
\DescribeMacro{\childdocof}
Furthermore, add the commands
\begin{center}
\begin{tabular}{l}
|\input{childdoc.def}|\\
|\childdocof{|\textit{main}|}|\\
\end{tabular}
\end{center}
at the top of every child file \textit{child}
which is included by |\include{|\textit{child}|}|
from within the main file
(or at least for those files to be compiled individually).
The argument \textit{main} must be the filename of the main file.

There are a couple of
considerations in setting up the main and child documents:

%%%%%%%%%%%%%%%%%%%%%%%%%%%%%%%%%%%%%%%%
\paragraph{Restrictions.}

Please note the following restrictions:
\begin{itemize}
\item
|\childdocmain| must be called with one argument \textit{main}
to ensure compatibility with earlier version of the package.
It must either be empty (|\childdocmain{}|)
or precisely match the filename of the main file in which it is specified.
See \secref{sec:detection} for further information.
\item
The filename \textit{main} must be specified without the |.tex| extension.
\item
The filename \textit{main} is case sensitive
(even in case-insensitive file systems)
due to internal string comparison.
\item
The argument \textit{main} should be fully expanded, it cannot be a macro.
\item
Subdirectories and special characters should be avoided in filenames.
\item
The command |\childdocmain{|\textit{main}|}| must be followed by a whitespace.
It should not be followed immediately by another command
or by a comment mark `|%|'.
This is because the \TeX{} parser reads the token immediately following
the argument of |\childdocmain| and puts it
at the beginning of every child section;
however, a white\-space is ignored.
\end{itemize}

%%%%%%%%%%%%%%%%%%%%%%%%%%%%%%%%%%%%%%%%
\paragraph{Content of Main File.}

It is advisable to place all content in the child files included by |\include|.
Any output contained in the main file will appear in all child documents
unless suppressed manually;
it cannot be suppressed automatically by the |\includeonly| directive
and thus should normally be avoided.
A method to include some content in the main file
by means of conditional processing is described in \secref{sec:conditional}.

%%%%%%%%%%%%%%%%%%%%%%%%%%%%%%%%%%%%%%%%
\paragraph{Page Numbering.}

When only a part of the document is compiled,
the appropriate numbering of pages
(as well as other status parameters)
is determined from the |.aux| files.
The latter contain information from previous passes.
However this information needs to propagate through
all intermediate child documents.
Therefore the page numbering in child documents may well
be inconsistent until the complete document is compiled at least once.

A useful (if unconventional) way to always ensure a consistent
page numbering is to restart the numbering in each child document
and denote the pages by `\textit{child}|.|\textit{page}'
where \textit{child} represents the chapter/section number of the child file.
This can be achieved by the command
|\numberwithin{page}{|\textit{child}|}|
of the \textsf{amsmath} package
where \textit{child} can be |chapter| or |section|
depending on the chosen structuring.
Alternatively, one can modify the macro |\thepage| appropriately
and reset the counter |page| at the start of each child file.

%%%%%%%%%%%%%%%%%%%%%%%%%%%%%%%%%%%%%%%%%%%%%%%%%%%%%%%%%%%%%%%%%%%%%%%%%%%%%%%%
\subsection{Conditional Processing}
\label{sec:conditional}

The package provides a mechanism to compile different versions
of a document. To customise the versions further some conditional processing
can come in handy to distinguish which version is being compiled.
The package provides two macros to describe the compilation context:

%%%%%%%%%%%%%%%%%%%%%%%%%%%%%%%%%%%%%%%%
\DescribeMacro{\ifchilddoc}
The conditional |\ifchilddoc| distinguishes between the compilation of
child documents and the main document:
%
\begin{center}
|\ifchilddoc |\textit{child-code}| |[|\||else |\textit{main-code}]| \||fi|
\end{center}

%%%%%%%%%%%%%%%%%%%%%%%%%%%%%%%%%%%%%%%%
\DescribeMacro{\childdocname}
\DescribeMacro{\childdocjob}
The macro |\childdocname| contains the filename (without extension)
of the main or child file being processed.
Note that |\childdocjob| will always contain the name of the main file.

%%%%%%%%%%%%%%%%%%%%%%%%%%%%%%%%%%%%%%%%
\paragraph{Title Page.}

Conditional processing can be used to include a title or banner page
in the main document when proper precautions are taken.
Importantly, the code in the main file should ensure that the page counter
(as well as other status parameters which are stored in the |.aux| files)
takes the same value after the conditional processing.
Otherwise the page numbers may take divergent values
depending on which part is compiled.

For example, a title page could be declared by:
%
\begin{center}
\begin{tabular}{l}
|\ifchilddoc\||else|\\
|\addtocounter{page}{-1}|\\
\textit{code for title page}\\
|\newpage|\\
|\||fi|
\end{tabular}
\end{center}
%
A banner page for the child documents can be generated by:
%
\begin{center}
\begin{tabular}{l}
|\ifchilddoc|\\
|\addtocounter{page}{-1}|\\
\textit{code for banner page}\\
|\newpage|\\
|\||fi|
\end{tabular}
\end{center}
%
Here one could write a message such as:
\begin{center}
|This is the part \childdocname{} of \childdocjob{}.|
\end{center}

%%%%%%%%%%%%%%%%%%%%%%%%%%%%%%%%%%%%%%%%%%%%%%%%%%%%%%%%%%%%%%%%%%%%%%%%%%%%%%%%
\subsection{Flags}
\label{sec:flags}

The package makes it easy to generate different versions
of the main or child documents.
To this end compilation flags can be defined
and assigned different default values.
They will be particularly useful in conjunction
with the forwarding mechanism described in \secref{sec:forward}.

For example, it may be useful to have a flag |\version|
which can be set to |draft| or |final|.
The document source will contain some conditional code
depending on the value of |\version|.
Suppose further, the flag should default to |final| for the main file
and to |draft| for child files
which is a natural assignment for editing the document.
This is achieved by placing the following code
in the preamble of the main document
(below the |\childdocmain| directive):
%
\begin{center}
\begin{tabular}{l}
|\ifchilddoc|\\
|\providecommand{\version}{draft}|\\
|\||else|\\
|\providecommand{\version}{final}|\\
|\||fi|
\end{tabular}
\end{center}
%
The definition by |\providecommand| makes sure
that previous definitions are not overwritten.
Further statements |\providecommand{\version}{...}|
can thus be added before the above code to override it.

For the main file, one might add a line
(between |\childdocmain| and the above block)
%
\begin{center}
|%\ifchilddoc\||else\providecommand{\version}{draft}\||fi|
\end{center}
%
which can be uncommented to produce a draft version.
Likewise one can add a line to the very top of a child file
(above the |\childdocof{|\textit{main}|}| directive)
%
\begin{center}
|%\providecommand{\version}{final}|
\end{center}
%
which can be uncommented to produce the final version of this child document.

%%%%%%%%%%%%%%%%%%%%%%%%%%%%%%%%%%%%%%%%%%%%%%%%%%%%%%%%%%%%%%%%%%%%%%%%%%%%%%%%
\subsection{Forwarding}
\label{sec:forward}

Different versions of the main or child documents
using compilation flags as described in \secref{sec:flags}
can be (permanently) stored in different files
for convenient compilation, viewing and distribution.
To this end, the package defines a command
to pass on compilation to a different file:

%%%%%%%%%%%%%%%%%%%%%%%%%%%%%%%%%%%%%%%%
\DescribeMacro{\childdocforward}
The command |\childdocforward| redirects processing to
another source file:
%
\begin{center}
\begin{tabular}{l}
|\input{childdoc.def}|\\
|\childdocforward[|\textit{main}|]{|\textit{dest}|}|\\
\end{tabular}
\end{center}
%
The argument \textit{dest} is the destination file
(without extension).
It should be the main file or one of the child files.
Note that further \textsf{childdoc} directives
such as |\childdocof| and |\childdocforward|
in the indicated file will be processed in this form.
The optional argument \textit{main}
passes on directly to the main file \textit{main}
while pretending to compile the child \textit{dest}.
This form behaves as if \textit{dest}
issues |\childdocof{|\textit{main}|}| right away,
and no further \textsf{childdoc} directives will be processed.

%%%%%%%%%%%%%%%%%%%%%%%%%%%%%%%%%%%%%%%%
\DescribeMacro{\...prefix}
In the alternative form |\childdocforwardprefix|,
%
\begin{center}
\begin{tabular}{l}
|\input{childdoc.def}|\\
|\childdocforwardprefix[|\textit{main}|]{|\textit{prefix}|}{|\textit{dest}|}|
\end{tabular}
\end{center}
%
the destination file is determined by a pattern
depending on the current file:
To make this work, the current file must be called
`{\textit{prefix}\hspace{0.2em}\textit{suffix}}'
with \textit{prefix} matching precisely the argument.
Processing is then passed on to the file
`{\textit{dest}\hspace{0.2em}\textit{suffix}}'.
Surely, the same effect is achieved by
directly specifying the
argument `{\textit{dest}\hspace{0.2em}\textit{suffix}}'
in the first form.
However, that requires to set up a different file
for each child. With the alternative form of the command
all these files can have exactly the same content
which simplifies setting them up and maintaining them.

For example, the following file |draft.tex|
with a compilation flag |\version| as described in \secref{sec:flags}
compiles the main document as a draft:
%
\begin{center}
\begin{tabular}{l}
|\def\version{draft}|\\
|\input{childdoc.def}|\\
|\childdocforward{|\textit{main}|}|
\end{tabular}
\end{center}
%
Likewise, the following files |final|\textit{nn}|.tex|
compile the final version of the child document
|child|\textit{nn}|.tex|:
%
\begin{center}
\begin{tabular}{l}
|\def\version{final}|\\
|\input{childdoc.def}|\\
|\childdocforwardprefix{final}{child}|
\end{tabular}
\end{center}
%

Note that when several versions of a main file and/or of each child file
are to be generated, it may be convenient to set up a |Makefile| or
shell script to automatise the process.

%%%%%%%%%%%%%%%%%%%%%%%%%%%%%%%%%%%%%%%%%%%%%%%%%%%%%%%%%%%%%%%%%%%%%%%%%%%%%%%%
\subsection{Command Line Processing}
\label{sec:commandline}

The effect of redirection files can also be achieved by invoking
the \LaTeX{} compiler with a more elaborate command line.
Most conveniently this should be done as part
of a shell script or a |Makefile|.

When using \textsf{childdoc} in the main file, the following
command lines effectively perform a redirection
(note that depending on the shell being used,
backslashes may have to be doubled: `|\|' $\to$ `|\\|'):
%
\begin{center}
|... -jobname "|\textit{target}|" |\\|"|[\textit{flags}]%
|\input{childdoc.def}\childdocforward[|\textit{main}|]{|\textit{dest}|}"|
\end{center}
%
Here \textit{target} is the name of the output file,
\textit{main} is the name of the main file
and \textit{dest} is the name of the main or child file to be processed
(all filenames without extensions).
The optional argument \textit{main} can be omitted
if \textit{main} matches \textit{dest}.
Optionally, compilation \textit{flags} can be defined via |\def| commands.
This command line makes the \TeX{} engine believe
it is compiling the file \textit{target}
whose content is specified as the latter parameter.
The provided code then forwards the processing to
\textit{main} or \textit{dest} as described in \secref{sec:forward}.

%%%%%%%%%%%%%%%%%%%%%%%%%%%%%%%%%%%%%%%%%%%%%%%%%%%%%%%%%%%%%%%%%%%%%%%%%%%%%%%%
\subsection{Include by Input}
\label{sec:input}

Including child documents by |\include| has some restrictions by design.
Most notably, the content of a child document always occupies
its own set of pages; pages cannot be shared between child documents.
Usually, this behaviour makes perfect sense
because each child document contain an essential part of the document.
However, in some situations it may be desirable to compose
a document from a collection of parts
without having mandatory page breaks between then.
For this case, the package
provides a mechanism to include parts
by |\input| which can also be processed individually.
However, by construction this mechanism
requires manual handling of the content to be output.

%%%%%%%%%%%%%%%%%%%%%%%%%%%%%%%%%%%%%%%%
\DescribeMacro{\ifchilddocmanual}
The main file should be prepared as usual, see \secref{sec:include}.
However, the document body must make a distinction
between processing of an individual part and of the main document, e.g.:
%
\begin{center}
\begin{tabular}{l}
|\ifchilddocmanual|\\
|\input{\childdocname}|\\
|\||else|\\
\textit{document body with }|\input{|\textit{part}|}|\\
|\||fi|
\end{tabular}
\end{center}
%
The conditional |\ifchilddocmanual| is true whenever
a part to be included by |\input| is being compiled,
and the name of the part is stored in |\childdocname|.

%%%%%%%%%%%%%%%%%%%%%%%%%%%%%%%%%%%%%%%%
\DescribeMacro{\childdocby}
Each part to be included by |\input| should start with:
%
\begin{center}
\begin{tabular}{l}
|\input{childdoc.def}|\\
|\childdocby{|\textit{main}|}|\\
\end{tabular}
\end{center}
%
The directive |\childdocby| is similar to |\childdocof|
described in \secref{sec:include},
but the subsequent selection of content must be done manually.
To that end, both |\ifchilddoc| and |\ifchilddocmanual|
will be true upon processing of a part,
and the name of the part is stored in |\childdocname|.
Note that |\jobname| will be set to the filename of the current part
so that each part receives an individual |.aux| file
that does not interfere with the |.aux| file(s) of the main document.
This behaviour can be altered by the alternative form
|\childdocby[*]{|\textit{main}|}| (with a non-empty optional argument)
which uses the |.aux| file of the main document
by setting |\jobname| to \textit{main}.

%%%%%%%%%%%%%%%%%%%%%%%%%%%%%%%%%%%%%%%%%%%%%%%%%%%%%%%%%%%%%%%%%%%%%%%%%%%%%%%%
\subsection{Driver Development}
\label{sec:driver}

The \textsf{childdoc} mechanism can also be use for the development
of definition files such as \LaTeX{} styles or classes.
This case differs from the above setup with multiple parts
included by |\include| in that no |\includeonly| should be invoked.
This can be achieved by starting the include file
(before |\ProvidesPackage|) with:
%
\begin{center}
\begin{tabular}{l}
|\input{childdoc.def}|\\
|\childdocforward{|\textit{main}|}|\\
\end{tabular}
\end{center}
%
or alternatively with:
%
\begin{center}
\begin{tabular}{l}
|\input{childdoc.def}|\\
|\childdocby{|\textit{main}|}|\\
\end{tabular}
\end{center}
%
Both forms have slightly different effects as described above.
The main file is prepared as usual, see \secref{sec:include}.

%%%%%%%%%%%%%%%%%%%%%%%%%%%%%%%%%%%%%%%%%%%%%%%%%%%%%%%%%%%%%%%%%%%%%%%%%%%%%%%%
\subsection{Legacy Detection}
\label{sec:detection}

The directive |\childdocmain| in the main file can detect
whether the complete document or merely a child is to be compiled
even without using the directive |\childdocof|.
This method is deprecated because it is less robust
and there is no compelling reason to use it;
it is merely provided for backward compatibility
and it may be removed in future versions.

If the detection mechanism is to be used,
it is mandatory to correctly specify
the filename of the main file as the argument of |\childdocmain|:
%
\begin{center}
\begin{tabular}{l}
|\input{childdoc.def}|\\
|\childdocmain{|\textit{main}|}|\\
\end{tabular}
\end{center}
%
If |\jobname| does not match the argument \textit{main} of |\childdocmain|,
it is assumed that |\jobname| points to the child file to be compiled.
When using |\childdocmain| with the main file specified as argument,
it suffices to start a child file
with just |\input{|\textit{main}|}|
without loading of the package and using |\childdocof|.
If instead all processing is done
with the appropriate \textsf{childdoc} directives,
the argument of \textit{main} of |\childdocmain| can be empty.

An alternative version of the command line processing described
in \secref{sec:commandline} using the detection mechanism reads:
%
\begin{center}
|... -jobname "|\textit{target}|" "|[\textit{flags}]%
[|\def\jobname{|\textit{dest}|}|]|\input{|\textit{main}|}"|
\end{center}

%%%%%%%%%%%%%%%%%%%%%%%%%%%%%%%%%%%%%%%%%%%%%%%%%%%%%%%%%%%%%%%%%%%%%%%%%%%%%%%%
\subsection{Manual Code}
\label{sec:manual}

In case one cannot be certain whether the definitions file |childdoc.def|
is installed on the target \TeX{} distribution
and one prefers not to ship it,
it is conceivable to paste a few relevant commands into the sources.

To that end, drop all statements |\input{childdoc.def}|
and perform the replacements as outlined below.
Instead of |\childdocmain{|\textit{main}|}| add the following code
to the top of the main file:
%
\begin{center}
\begin{tabular}{l}
|\||ifdefined\childdocname\endinput\||fi\newif\ifchilddoc|\\
|\edef\childdocname{\scantokens\expandafter{\jobname\noexpand}}|\\
|\def\childdocmain{|\textit{main}|}\||ifx\childdocmain\childdocname\||else|\\
|\childdoctrue\includeonly{\childdocname}\let\jobname\childdocmain\||fi|\\
\end{tabular}
\end{center}
%
Instead of |\childdocof{|\textit{main}|}| just include the main file
at the top of each child file:
%
\begin{center}
|\input{|\textit{main}|}|
\end{center}
%
A simple redirection |\childdocforward{|\textit{dest}|}| is achieved by:
%
\begin{center}
|\def\jobname{|\textit{dest}|}\input{\jobname}|
\end{center}
%
The redirection with prefix
|\childdocforwardprefix[|\textit{prefix}|]{|\textit{dest}|}|
is accomplished by:
%
\begin{center}
\begin{tabular}{l}
|{\edef\jobname{\scantokens\expandafter{\jobname\noexpand}}|\\
|\def\redirectjob |\textit{prefix}|#1~~~{\gdef\jobname{|\textit{dest}|#1}}|\\
|\expandafter\redirectjob\jobname~~~}\input{\jobname}|
\end{tabular}
\end{center}

In an alternative approach,
child documents can be compiled by a specific command line
without additional code or specific definitions:
%
\begin{center}
|... -jobname "|\textit{target}|" "|[\textit{flags}]%
|\includeonly{|\textit{dest}|}\input{|\textit{main}|}"|
\end{center}
%

%%%%%%%%%%%%%%%%%%%%%%%%%%%%%%%%%%%%%%%%%%%%%%%%%%%%%%%%%%%%%%%%%%%%%%%%%%%%%%%%
%%%%%%%%%%%%%%%%%%%%%%%%%%%%%%%%%%%%%%%%%%%%%%%%%%%%%%%%%%%%%%%%%%%%%%%%%%%%%%%%
\section{Information}

%%%%%%%%%%%%%%%%%%%%%%%%%%%%%%%%%%%%%%%%%%%%%%%%%%%%%%%%%%%%%%%%%%%%%%%%%%%%%%%%
\subsection{Copyright}

Copyright \copyright{} 2017--2018 Niklas Beisert

This work may be distributed and/or modified under the
conditions of the \LaTeX{} Project Public License, either version 1.3
of this license or (at your option) any later version.
The latest version of this license is in
  \url{http://www.latex-project.org/lppl.txt}
and version 1.3 or later is part of all distributions of \LaTeX{}
version 2005/12/01 or later.

This work has the LPPL maintenance status `maintained'.

The Current Maintainer of this work is Niklas Beisert.

This work consists of the files |README.txt|, |childdoc.ins| and |childdoc.dtx|
as well as the derived files |childdoc.def|, |cdocsamp.tex|
with |cdocsch1.tex|, |cdocsch2.tex|, |cdocspt3.tex|, |cdocspt4.tex|,
|cdocsdrf.tex|, |cdocsfn1.tex|, |cdocsfn2.tex|
as well as |childdoc.pdf|.

%%%%%%%%%%%%%%%%%%%%%%%%%%%%%%%%%%%%%%%%%%%%%%%%%%%%%%%%%%%%%%%%%%%%%%%%%%%%%%%%
\subsection{Files and Installation}

The package consists of the files:
%
\begin{center}
\begin{tabular}{ll}
    |README.txt|   & readme file \\
    |childdoc.ins| & installation file \\
    |childdoc.dtx| & source file \\
    |childdoc.def| & definition file \\
    |cdocsamp.tex| & sample main file \\
    |cdocsch1.tex| & sample include file \\
    |cdocsch2.tex| & sample include file \\
    |cdocspt3.tex| & sample part file \\
    |cdocspt4.tex| & sample part file \\
    |cdocsdrf.tex| & sample redirection file \\
    |cdocsfn1.tex| & sample redirection file \\
    |cdocsfn2.tex| & sample redirection file \\
    |childdoc.pdf| & manual
\end{tabular}
\end{center}
%
The distribution consists of the files
|README.txt|, |childdoc.ins| and |childdoc.dtx|.
%
\begin{itemize}
\item
Run (pdf)\LaTeX{} on |childdoc.dtx|
to compile the manual |childdoc.pdf| (this file).
\item
Run \LaTeX{} on |childdoc.ins| to create the definitions file |childdoc.def|
and the sample |cdocsamp.tex| with include files
|cdocsch1.tex|, |cdocsch2.tex|, |cdocspt3.tex|, |cdocspt4.tex|,
|cdocsdrf.tex|, |cdocsfn1.tex|, |cdocsfn2.tex|.
Then copy the file |childdoc.def| to an appropriate directory of your \LaTeX{}
distribution, e.g.\ \textit{texmf-root}|/tex/latex/childdoc|.
\end{itemize}

%%%%%%%%%%%%%%%%%%%%%%%%%%%%%%%%%%%%%%%%%%%%%%%%%%%%%%%%%%%%%%%%%%%%%%%%%%%%%%%%
\subsection{Related CTAN Packages}

There are several other packages which offer a similar functionality:
%
\begin{itemize}
\item
The packages
\href{http://ctan.org/pkg/docmute}{\textsf{docmute}},
\href{http://ctan.org/pkg/includex}{\textsf{includex}} and
\href{http://ctan.org/pkg/standalone}{\textsf{standalone}}
provide commands to include only the document body of
a child file thus allowing both files to be compiled individually.
\item
The packages \href{http://ctan.org/pkg/subdocs}{\textsf{subdocs}}
and \href{http://ctan.org/pkg/subfiles}{\textsf{subfiles}}
provide structures in which the main and child documents can be
encapsulated and allowing them to be compiled individually.
The inclusion mechanism is different from the conventional |\include|.
\item
The package \href{http://ctan.org/pkg/combine}{\textsf{combine}}
is an elaborate solution to combine several documents into one.
\end{itemize}
%
See also the CTAN topic \href{http://ctan.org/topic/subdocs}{\textsf{subdocs}}
for further related packages.
The present package differs from the above solutions in that
a document structure constructed with the conventional |\include| mechanism
just needs two extra commands at the top of every file
such that all constituent files can be compiled individually.

%%%%%%%%%%%%%%%%%%%%%%%%%%%%%%%%%%%%%%%%%%%%%%%%%%%%%%%%%%%%%%%%%%%%%%%%%%%%%%%%
%\subsection{Feature Suggestions}
%
%The following is a list of features which may be useful for future
%versions of this package:
%%
%\begin{itemize}
%\item
%\ldots
%\end{itemize}

%%%%%%%%%%%%%%%%%%%%%%%%%%%%%%%%%%%%%%%%%%%%%%%%%%%%%%%%%%%%%%%%%%%%%%%%%%%%%%%%
\subsection{Revision History}

%%%%%%%%%%%%%%%%%%%%%%%%%%%%%%%%%%%%%%%%
\paragraph{v2.0:} 2018/12/30

\begin{itemize}
\item
immediate forward processing
\item
added |\childdocby| mechanism
\item
manual restructured
\end{itemize}

%%%%%%%%%%%%%%%%%%%%%%%%%%%%%%%%%%%%%%%%
\paragraph{v1.6:} 2018/01/17

\begin{itemize}
\item
application for development of include files
\item
corrections to manual
\end{itemize}

%%%%%%%%%%%%%%%%%%%%%%%%%%%%%%%%%%%%%%%%
\paragraph{v1.5:} 2017/05/21

\begin{itemize}
\item
more complete structuring introduced
\item
|\childdocof| introduced
\item
|\childdoc| renamed to |\childdocmain|
\item
|\childredirect| renamed to |\childdocforward| and |\childdocforwardprefix|
and functionality expanded
\end{itemize}

%%%%%%%%%%%%%%%%%%%%%%%%%%%%%%%%%%%%%%%%
\paragraph{v1.0:} 2017/04/27

\begin{itemize}
\item
manual and install package
\item
first version published on CTAN
\end{itemize}

%%%%%%%%%%%%%%%%%%%%%%%%%%%%%%%%%%%%%%%%
\paragraph{v0.6:} 2017/04/26

\begin{itemize}
\item
redirection mechanism added
\end{itemize}

%%%%%%%%%%%%%%%%%%%%%%%%%%%%%%%%%%%%%%%%
\paragraph{v0.5:} 2017/04/26

\begin{itemize}
\item
functionality in definition file
\end{itemize}


%%%%%%%%%%%%%%%%%%%%%%%%%%%%%%%%%%%%%%%%%%%%%%%%%%%%%%%%%%%%%%%%%%%%%%%%%%%%%%%%
%%%%%%%%%%%%%%%%%%%%%%%%%%%%%%%%%%%%%%%%%%%%%%%%%%%%%%%%%%%%%%%%%%%%%%%%%%%%%%%%
%%%%%%%%%%%%%%%%%%%%%%%%%%%%%%%%%%%%%%%%%%%%%%%%%%%%%%%%%%%%%%%%%%%%%%%%%%%%%%%%
\appendix

\settowidth\MacroIndent{\rmfamily\scriptsize 000\ }

 \DocInput{childdoc.dtx}

\end{document}
%</driver>
% \fi
%
% %%%%%%%%%%%%%%%%%%%%%%%%%%%%%%%%%%%%%%%%%%%%%%%%%%%%%%%%%%%%%%%%%%%%%%%%%%%%%%
% %%%%%%%%%%%%%%%%%%%%%%%%%%%%%%%%%%%%%%%%%%%%%%%%%%%%%%%%%%%%%%%%%%%%%%%%%%%%%%
% \section{Sample}
%\iffalse
%<*samplemain>
%\fi
%
% The following presents a sample document
% with two chapters, two parts, a title page,
% a compile flag as well as three forwarding files to set the flag.
% It consists of eight |.tex| files:
% \begin{center}
% \begin{tabular}{ll}
% |cdocsamp.tex|&main file\\
% |cdocsch1.tex|&include file for chapter 1\\
% |cdocsch2.tex|&include file for chapter 2\\
% |cdocspt3.tex|&include file for part 3\\
% |cdocspt4.tex|&include file for part 4\\
% |cdocsdrf.tex|&forwarding file for main file in draft mode\\
% |cdocsfi1.tex|&forwarding file for final version of chapter 1\\
% |cdocsfi2.tex|&forwarding file for final version of chapter 2\\
% \end{tabular}
% \end{center}
% Each of the eight files can be compiled directly by the \LaTeX{} compiler.
%
% %%%%%%%%%%%%%%%%%%%%%%%%%%%%%%%%%%%%%%
% \paragraph{Main File.}
%
% The main file is called |cdocsamp.tex|.
%
% Load the \textsf{childdoc} definitions and
% declare the filename for the main document:
%    \begin{macrocode}
\input{childdoc.def}
\childdocmain{}
%    \end{macrocode}

% Optional override for |\version| flag:
%    \begin{macrocode}
%%\ifchilddoc\else\providecommand{\version}{draft}\fi
%    \end{macrocode}

% Define the default values for the |\version| flag
% (|final| for the main file and |draft| for childs):
%    \begin{macrocode}
\ifchilddoc
\providecommand{\version}{draft}
\else
\providecommand{\version}{final}
\fi
%    \end{macrocode}

% Load the standard document class:
%    \begin{macrocode}
\documentclass[12pt]{article}
%    \end{macrocode}

% Start the document body:
%    \begin{macrocode}
\begin{document}
%    \end{macrocode}

% Declare a title page.
% Print title, part of document being processed and version flag:
%    \begin{macrocode}
\addtocounter{page}{-1}
\begin{center}
{\LARGE\bfseries{}childdoc example\par}
\vspace{1cm}
\ifchilddoc
\ifchilddocmanual part\else chapter\fi:
`\childdocname' of `\childdocjob'\par
\else
main document: `\childdocjob'\par
\fi
version: \version\par
\end{center}
\newpage
%    \end{macrocode}

% Manually include selected file,
% otherwise process as usual:
%    \begin{macrocode}
\ifchilddocmanual
\section*{part `\childdocname'}
\input{\childdocname}
\else
%    \end{macrocode}

% Include the two chapters:
%    \begin{macrocode}
\include{cdocsch1}
\include{cdocsch2}
%    \end{macrocode}

% Include the two parts unless only chapters should be displayed:
%    \begin{macrocode}
\ifchilddoc\else
\section{part three}
\input{cdocspt3}
\section{part four}
\input{cdocspt4}
\fi
%    \end{macrocode}

% Process as usual until here:
%    \begin{macrocode}
\fi
%    \end{macrocode}

% End of document body:
%    \begin{macrocode}
\end{document}
%    \end{macrocode}
%\iffalse
%</samplemain>
%\fi
%
% %%%%%%%%%%%%%%%%%%%%%%%%%%%%%%%%%%%%%%
% \paragraph{Chapter Include Files.}
%
% The include files are called |cdocsch1.tex| and |cdocsch2.tex|.
%
%\iffalse
%<*samplechap1|samplechap2>
%\fi

% Optional override for |\version| flag:
%    \begin{macrocode}
%%\providecommand{\version}{final}
%    \end{macrocode}

% Include the main document:
%    \begin{macrocode}
\input{childdoc.def}
\childdocof{cdocsamp}
%    \end{macrocode}

%\iffalse
%</samplechap1|samplechap2>
%\fi
%
%\iffalse
%<*samplechap1>
%\fi
% Some text for chapter 1:
%    \begin{macrocode}
\section{one}
some text in chapter one
%    \end{macrocode}

%\iffalse
%</samplechap1>
%\fi
% Some text for chapter 2:
%\iffalse
%<*samplechap2>
%\fi
%    \begin{macrocode}
\section{two}
more text in chapter two
%    \end{macrocode}

%\iffalse
%</samplechap2>
%\fi
%
% %%%%%%%%%%%%%%%%%%%%%%%%%%%%%%%%%%%%%%
% \paragraph{Part Include Files.}
%
% The include files are called |cdocspt3.tex| and |cdocspt4.tex|.
%
%\iffalse
%<*samplepart3|samplepart4>
%\fi

% Optional override for |\version| flag:
%    \begin{macrocode}
%%\providecommand{\version}{final}
%    \end{macrocode}

% Include the main document:
%    \begin{macrocode}
\input{childdoc.def}
\childdocby{cdocsamp}
%    \end{macrocode}

%\iffalse
%</samplepart3|samplepart4>
%\fi
%
%\iffalse
%<*samplepart3>
%\fi
% Some text for part 3:
%    \begin{macrocode}
some text in part three
%    \end{macrocode}

%\iffalse
%</samplepart3>
%\fi
% Some text for part 4:
%\iffalse
%<*samplepart4>
%\fi
%    \begin{macrocode}
more text in part four
%    \end{macrocode}

%\iffalse
%</samplepart4>
%\fi
%
% %%%%%%%%%%%%%%%%%%%%%%%%%%%%%%%%%%%%%%
% \paragraph{Forwarding for a Complete Draft.}
%
% The following forwarding file |cdocsdrf.tex|
% compiles the main document in draft mode:
%\iffalse
%<*sampledraft>
%\fi
%    \begin{macrocode}
\def\version{draft}
\input{childdoc.def}
\childdocforward{cdocsamp}
%    \end{macrocode}

%\iffalse
%</sampledraft>
%\fi
%
% %%%%%%%%%%%%%%%%%%%%%%%%%%%%%%%%%%%%%%
% \paragraph{Forwarding for Final Version of the Chapters.}
%
% The following forwarding files |cdocsfn1.tex| and |cdocsfn2.tex|
% (with identical content)
% compile the final versions of the child documents
% |cdocsch1.tex| and |cdocsch2.tex|, respectively:
%\iffalse
%<*samplefinal>
%\fi
%    \begin{macrocode}
\def\version{final}
\input{childdoc.def}
\childdocforwardprefix[cdocsamp]{cdocsfn}{cdocsch}
%    \end{macrocode}

%\iffalse
%</samplefinal>
%\fi
%
% %%%%%%%%%%%%%%%%%%%%%%%%%%%%%%%%%%%%%%
% \paragraph{Command Line Processing.}
%
% The following three command lines generate the output files
% |cdocscld|, |cdocscl1| and |cdocscl2|
% which should be identical to
% |cdocsdrf|, |cdocsch1| and |cdocsfn2|, respectively:
% \begin{center}
% \begin{tabular}{l}
% |latex -jobname cdocscld \|\\
% |  "\def\version{draft}\input{childdoc.def}\childdocforward{cdocsamp}"|\\
% |latex -jobname cdocscl1 \|\\
% |  "\input{childdoc.def}\childdocforward[cdocsamp]{cdocsch1}"|\\
% |latex -jobname cdocscl2 \|\\
% |  "\def\version{final}\input{childdoc.def}\childdocforward{cdocsch2}"|
% \end{tabular}
% \end{center}
% Note that the trailing backslash on each first line
% merely continues the input to the second line
% (for convenient cut ant paste).
% Furthermore, the command |latex| can be replaced by any
% of its alternative versions such as |pdflatex|.
%
% %%%%%%%%%%%%%%%%%%%%%%%%%%%%%%%%%%%%%%%%%%%%%%%%%%%%%%%%%%%%%%%%%%%%%%%%%%%%%%
% %%%%%%%%%%%%%%%%%%%%%%%%%%%%%%%%%%%%%%%%%%%%%%%%%%%%%%%%%%%%%%%%%%%%%%%%%%%%%%
% \section{Implementation}
%\iffalse
%<*package>
%\fi
%
% This section describes the definitions file |childdoc.def|.

% The definitions cannot be loaded using |\usepackage| or |\RequirePackage|
% which has a mechanism to prevent loading a style file more than once.
% When loading the definitions by means of |\input|
% multiple instances have to be prevented manually:
%\iffalse
%This code needs to be before the `\ProvidesFile' directive
%which is defined at the beginning of this file.
%Therefore it is also placed there and commented out here.
%</package>
%<*discard>
%\fi
%    \begin{macrocode}
\ifdefined\childdocmain\endinput\fi
%    \end{macrocode}
%\iffalse
%</discard>
%<*package>
%\fi
%
% \macro{\ifchilddoc}
% \macro{\ifchilddocmanual}
% The conditional |\ifchilddoc| tells whether a
% child (true) or main (false) document is being compiled.
% The conditional |\ifchilddocmanual| tells whether
% the |\includeonly| mechanism is used (false) or
% the selection of child files must be performed manually (true).
% The definitions initialise to false:
%    \begin{macrocode}
\newif\ifchilddoc
\newif\ifchilddocmanual
%    \end{macrocode}

% \macro{\childdocname}
% \macro{\childdocjob}
% The macro |\childdocname| stores the name of the main document
% to be compiled. The macro |\childdocjob| stores the name of
% the document on which the \LaTeX{} compiler was originally invoked.
% The content of |\jobname| cannot be compared
% to filenames specified in the source due to different catcodes.
% The following code rescans |\jobname|, stores the result
% in |\childdocname| and saves a copy in |\childdocjob|:
%    \begin{macrocode}
\edef\childdocname{\scantokens\expandafter{\jobname\noexpand}}
\let\childdocjob\childdocname
%    \end{macrocode}

% \macro{\childdocdisable}
% The macro |\childdocdisable| prevents the main file
% from being processed more than once.
% At this stage, the main document command |\childdocmain|
% is assumed to be called once again where it should do nothing.
% Any subsequent call to it should prevent
% a secondary processing of the main document
% It overwrites the forwarding commands
% |\childdocof| and |\childdocforward|
% with empty macros to prevent further inclusions of the main document:
%    \begin{macrocode}
\newcommand{\childdocdisable}
{
  \renewcommand{\childdocmain}[1]{\renewcommand{\childdocmain}[1]{\endinput}}
  \renewcommand{\childdocof}[1]{}
  \renewcommand{\childdocby}[2][]{}
  \renewcommand{\childdocforward}[2][]{}
  \renewcommand{\childdocdisable}{}
}
%    \end{macrocode}

% \macro{\childdocmain}
% The macro |\childdocmain| is to be called at the top of the main file
% with nothing or the main filename (without extension) as argument.
% First, it breaks loops.
% If the argument is not empty and does not match |\childdocname|
% (which is set by the first inclusion of |childdoc.def|),
% |\ifchilddoc| is set to true, |\includeonly| is applied to the child file
% and |\jobname| is set to the main file
% (for proper handling of |.aux| files):
%    \begin{macrocode}
\newcommand{\childdocmain}[1]
{
  \childdocdisable\childdocmain{}
  \if?#1?\else
    \begingroup
      \def\childdoctmp{#1}
      \ifx\childdoctmp\childdocname
        \def\childdoctmp{}
      \else
        \def\childdoctmp
        {
          \childdoctrue
          \includeonly{\childdocname}
          \def\childdocjob{#1}
          \def\jobname{#1}
        }
      \fi
      \expandafter
    \endgroup
    \childdoctmp
  \fi
}
%    \end{macrocode}

% \macro{\childdocof}
% The command |\childdocof| redirects
% compilation to the main file |#1|.
%    \begin{macrocode}
\newcommand{\childdocof}[1]
{
  \childdocdisable
  \childdoctrue
  \includeonly{\childdocname}
  \def\jobname{#1}
  \def\childdocjob{#1}
  \input{#1}
}
%    \end{macrocode}

% \macro{\childdocby}
% The command |\childdocby| ....
%    \begin{macrocode}
\newcommand{\childdocby}[2][]
{
  \childdocdisable
  \childdoctrue
  \childdocmanualtrue
  \if?#1?\else
    \def\jobname{#2}
  \fi
  \def\childdocjob{#2}
  \input{#2}
  \endinput
}
%    \end{macrocode}

% \macro{\childdocforward}
% The command |\childdocforward| redirects
% compilation to the main file or
% (if the optional argument is given) a child file.
% Parameters are set as if the main file
% or a child file starting with |\childdocof| was compiled.
% Then compilation is handed over to the main file:
%    \begin{macrocode}
\newcommand{\childdocforward}[2][]
{
  \begingroup
    \if?#1?
      \def\childdoctmp
      {
        \def\childdocname{#2}
        \def\childdocjob{#2}
        \def\jobname{#2}
        \input{#2}
        \endinput
      }
    \else
      \def\childdoctmp
      {
        \childdocdisable
        \def\childdocname{#2}
        \childdoctrue
        \includeonly{#2}
        \def\childdocjob{#1}
        \def\jobname{#1}
        \input{#1}
        \endinput
      }
    \fi
    \expandafter
  \endgroup
  \childdoctmp
}
%    \end{macrocode}

% \macro{\childdocforwardprefix}
% The command |\childdocforwardprefix| redirects
% compilation to the main or a child file by means of a pattern.
% The prefix |#1| in the current filename is replaced by |#2|
% and the suffix of the current filename is kept
% (it is assumed that the filename does not contain the substring `|~~~|'
% which is used as a delimiter).
% Compilation is handed over to the new file by |\childdocforward|:
%    \begin{macrocode}
\newcommand{\childdocforwardprefix}[3][]
{
  \begingroup
    \def\childdocextract #2##1~~~{\def\childdoctmp{\childdocforward[#1]{#3##1}}}
    \expandafter\childdocextract\childdocname~~~
    \expandafter
  \endgroup
  \childdoctmp
}
%    \end{macrocode}

% \macro{\childdoc}
% The deprecated macro |\childdoc| is a legacy version of |\childdocmain|:
%    \begin{macrocode}
\newcommand{\childdoc}{\childdocmain}
%    \end{macrocode}

% \macro{\childdocredirect}
% The deprecated macro |\childdocredirect| is a legacy version
% of |\childdocforward| and |\childdocforwardprefix|:
%    \begin{macrocode}
\newcommand{\childdocredirect}[2][]
{
  \begingroup
    \if?#1?
      \def\childdoctmp{\childdocforward{#2}}
    \else
      \def\childdoctmp{\childdocforwardprefix{#1}{#2}}
    \fi
    \expandafter
  \endgroup
  \childdoctmp
}
%    \end{macrocode}

%\iffalse
%</package>
%\fi
%
\endinput
|\\
|\childdocforward[|\textit{main}|]{|\textit{dest}|}|\\
\end{tabular}
\end{center}
%
The argument \textit{dest} is the destination file
(without extension).
It should be the main file or one of the child files.
Note that further \textsf{childdoc} directives
such as |\childdocof| and |\childdocforward|
in the indicated file will be processed in this form.
The optional argument \textit{main}
passes on directly to the main file \textit{main}
while pretending to compile the child \textit{dest}.
This form behaves as if \textit{dest}
issues |\childdocof{|\textit{main}|}| right away,
and no further \textsf{childdoc} directives will be processed.

%%%%%%%%%%%%%%%%%%%%%%%%%%%%%%%%%%%%%%%%
\DescribeMacro{\...prefix}
In the alternative form |\childdocforwardprefix|,
%
\begin{center}
\begin{tabular}{l}
|% \iffalse
%
% childdoc.dtx Copyright (C) 2017-2018 Niklas Beisert
%
% This work may be distributed and/or modified under the
% conditions of the LaTeX Project Public License, either version 1.3
% of this license or (at your option) any later version.
% The latest version of this license is in
%   http://www.latex-project.org/lppl.txt
% and version 1.3 or later is part of all distributions of LaTeX
% version 2005/12/01 or later.
%
% This work has the LPPL maintenance status `maintained'.
%
% The Current Maintainer of this work is Niklas Beisert.
%
% This work consists of the files childdoc.dtx and childdoc.ins
% and the derived files childdoc.def and cdocsamp.tex with
% cdocsch1.tex, cdocsch2.tex, cdocsdrf.tex, cdocsfn1.tex, cdocsfn2.tex.
%
%<package>\ifdefined\childdocmain\endinput\fi
%<package>\ProvidesFile{childdoc.def}[2018/12/30 v2.0 child document driver]
%<samplemain>\ProvidesFile{cdocsamp.tex}[2018/12/30 v2.0 sample for childdoc]
%<*driver>
%\ProvidesFile{childdoc.drv}[2018/12/30 v2.0 childdoc reference manual file]
\PassOptionsToClass{10pt,a4paper}{article}
\documentclass{ltxdoc}

\usepackage[margin=35mm]{geometry}
\usepackage{hyperref}
\usepackage{hyperxmp}
\usepackage[usenames]{color}

\hypersetup{colorlinks=true}
\hypersetup{pdfstartview=FitH}
\hypersetup{pdfpagemode=UseNone}
\hypersetup{pdfsource={}}
\hypersetup{pdflang={en-UK}}
\hypersetup{pdfcopyright={Copyright 2017-2018 Niklas Beisert.
  This work may be distributed and/or modified under the
  conditions of the LaTeX Project Public License, either version 1.3
  of this license or (at your option) any later version.}}
\hypersetup{pdflicenseurl={http://www.latex-project.org/lppl.txt}}
\hypersetup{pdfcontactaddress={ETH Zurich, ITP, HIT K,
  Wolfgang-Pauli-Strasse 27}}
\hypersetup{pdfcontactpostcode={8093}}
\hypersetup{pdfcontactcity={Zurich}}
\hypersetup{pdfcontactcountry={Switzerland}}
\hypersetup{pdfcontactemail={nbeisert@itp.phys.ethz.ch}}
\hypersetup{pdfcontacturl={http://people.phys.ethz.ch/\xmptilde nbeisert/}}

\newcommand{\secref}[1]{\hyperref[#1]{section \ref*{#1}}}

\parskip1ex
\parindent0pt
\let\olditemize\itemize
\def\itemize{\olditemize\parskip0pt}

\begin{document}

\title{The \textsf{childdoc} Package}
\hypersetup{pdftitle={The childdoc Package}}
\author{Niklas Beisert\\[2ex]
  Institut f\"ur Theoretische Physik\\
  Eidgen\"ossische Technische Hochschule Z\"urich\\
  Wolfgang-Pauli-Strasse 27, 8093 Z\"urich, Switzerland\\[1ex]
  \href{mailto:nbeisert@itp.phys.ethz.ch}
  {\texttt{nbeisert@itp.phys.ethz.ch}}}
\hypersetup{pdfauthor={Niklas Beisert}}
\hypersetup{pdfsubject={Manual for the LaTeX2e Package childdoc}}
\date{30 December 2018, \textsf{v2.0}}
\maketitle

\begin{abstract}\noindent
\textsf{childdoc} is a \LaTeXe{} package
that enables the direct compilation
of document sections included by |\include|
to individual files.
\end{abstract}

\begingroup
\parskip0ex
\tableofcontents
\endgroup

%%%%%%%%%%%%%%%%%%%%%%%%%%%%%%%%%%%%%%%%%%%%%%%%%%%%%%%%%%%%%%%%%%%%%%%%%%%%%%%%
%%%%%%%%%%%%%%%%%%%%%%%%%%%%%%%%%%%%%%%%%%%%%%%%%%%%%%%%%%%%%%%%%%%%%%%%%%%%%%%%
\section{Introduction}

\LaTeX{} provides a mechanism to structure a large document (such as a book)
into a main file and several child files (containing the chapters)
using the |\include| command.
This mechanism is beneficial for documents
which span hundreds of pages in order to
make the source file(s) more manageable.
Moreover, compilation can be restricted to
selected child files by means of the |\includeonly| command.
The latter feature can be used to reduce the compilation time while editing
(this was significantly more useful in the earlier days of \LaTeX{})
or to generate a smaller document which is easier to navigate.
Another application of |\includeonly| is to generate
documents consisting of selected parts of the complete document.

However, there are a few drawbacks of the plain |\include| mechanism:
\begin{itemize}
\item
The child files cannot be compiled on their own,
they can only be compiled via the main file.
A naive editing environment
(such as a text editor with an option
to have the current file processed by \LaTeX)
may require one to switch to the main file before compiling;
attempting to compile the child file produces errors.
\item
The main file must be modified (each time)
to adjust the |\includeonly| command
to the present needs. This easily leaves the main file in a messy state.
\item
The generated document will always carry the filename
of the main document. This is inconvenient if
several child files are to be compiled and
to be kept for distribution.
\end{itemize}

The present package provides a simple interface
to make child files individually compilable by \LaTeX{}.
Compiling a child file then has the same effect as compiling
the main file with an |\includeonly| command
to select the appropriate child.
Moreover the generated document will carry the name of the child
rather than the main file.
This resolves all three above issues.

This feature is meant to make the editing of books,
thesis documents and lecture notes somewhat more convenient.
However, the package can also be used efficiently for
composing a series of documents (such as exercise sheets)
which are typically distributed individually.
It then assists the author in generating the individual documents
(potentially in different versions)
as well as a document containing the collected series.
Another application is in developing style files
or other kinds of included material
where compilation of the style file could redirect
to a sample or test file.

%%%%%%%%%%%%%%%%%%%%%%%%%%%%%%%%%%%%%%%%%%%%%%%%%%%%%%%%%%%%%%%%%%%%%%%%%%%%%%%%
%%%%%%%%%%%%%%%%%%%%%%%%%%%%%%%%%%%%%%%%%%%%%%%%%%%%%%%%%%%%%%%%%%%%%%%%%%%%%%%%
\section{Usage}

First of all, the package \textsf{childdoc} is \emph{not} a standard
\LaTeXe{} |.sty| style file! Therefore it needs to be invoked in
a non-standard way.

%%%%%%%%%%%%%%%%%%%%%%%%%%%%%%%%%%%%%%%%%%%%%%%%%%%%%%%%%%%%%%%%%%%%%%%%%%%%%%%%
\subsection{Included Files}
\label{sec:include}

%%%%%%%%%%%%%%%%%%%%%%%%%%%%%%%%%%%%%%%%
\DescribeMacro{\childdocmain}
To use the package, add the commands
\begin{center}
\begin{tabular}{l}
|\input{childdoc.def}|\\
|\childdocmain{}|\\
\end{tabular}
\end{center}
at the very top of the main \LaTeX{} file,
in particular \emph{before} the |\documentclass| statement!
The argument of |\childdocmain| should be left empty
(but it must be present).

%%%%%%%%%%%%%%%%%%%%%%%%%%%%%%%%%%%%%%%%
\DescribeMacro{\childdocof}
Furthermore, add the commands
\begin{center}
\begin{tabular}{l}
|\input{childdoc.def}|\\
|\childdocof{|\textit{main}|}|\\
\end{tabular}
\end{center}
at the top of every child file \textit{child}
which is included by |\include{|\textit{child}|}|
from within the main file
(or at least for those files to be compiled individually).
The argument \textit{main} must be the filename of the main file.

There are a couple of
considerations in setting up the main and child documents:

%%%%%%%%%%%%%%%%%%%%%%%%%%%%%%%%%%%%%%%%
\paragraph{Restrictions.}

Please note the following restrictions:
\begin{itemize}
\item
|\childdocmain| must be called with one argument \textit{main}
to ensure compatibility with earlier version of the package.
It must either be empty (|\childdocmain{}|)
or precisely match the filename of the main file in which it is specified.
See \secref{sec:detection} for further information.
\item
The filename \textit{main} must be specified without the |.tex| extension.
\item
The filename \textit{main} is case sensitive
(even in case-insensitive file systems)
due to internal string comparison.
\item
The argument \textit{main} should be fully expanded, it cannot be a macro.
\item
Subdirectories and special characters should be avoided in filenames.
\item
The command |\childdocmain{|\textit{main}|}| must be followed by a whitespace.
It should not be followed immediately by another command
or by a comment mark `|%|'.
This is because the \TeX{} parser reads the token immediately following
the argument of |\childdocmain| and puts it
at the beginning of every child section;
however, a white\-space is ignored.
\end{itemize}

%%%%%%%%%%%%%%%%%%%%%%%%%%%%%%%%%%%%%%%%
\paragraph{Content of Main File.}

It is advisable to place all content in the child files included by |\include|.
Any output contained in the main file will appear in all child documents
unless suppressed manually;
it cannot be suppressed automatically by the |\includeonly| directive
and thus should normally be avoided.
A method to include some content in the main file
by means of conditional processing is described in \secref{sec:conditional}.

%%%%%%%%%%%%%%%%%%%%%%%%%%%%%%%%%%%%%%%%
\paragraph{Page Numbering.}

When only a part of the document is compiled,
the appropriate numbering of pages
(as well as other status parameters)
is determined from the |.aux| files.
The latter contain information from previous passes.
However this information needs to propagate through
all intermediate child documents.
Therefore the page numbering in child documents may well
be inconsistent until the complete document is compiled at least once.

A useful (if unconventional) way to always ensure a consistent
page numbering is to restart the numbering in each child document
and denote the pages by `\textit{child}|.|\textit{page}'
where \textit{child} represents the chapter/section number of the child file.
This can be achieved by the command
|\numberwithin{page}{|\textit{child}|}|
of the \textsf{amsmath} package
where \textit{child} can be |chapter| or |section|
depending on the chosen structuring.
Alternatively, one can modify the macro |\thepage| appropriately
and reset the counter |page| at the start of each child file.

%%%%%%%%%%%%%%%%%%%%%%%%%%%%%%%%%%%%%%%%%%%%%%%%%%%%%%%%%%%%%%%%%%%%%%%%%%%%%%%%
\subsection{Conditional Processing}
\label{sec:conditional}

The package provides a mechanism to compile different versions
of a document. To customise the versions further some conditional processing
can come in handy to distinguish which version is being compiled.
The package provides two macros to describe the compilation context:

%%%%%%%%%%%%%%%%%%%%%%%%%%%%%%%%%%%%%%%%
\DescribeMacro{\ifchilddoc}
The conditional |\ifchilddoc| distinguishes between the compilation of
child documents and the main document:
%
\begin{center}
|\ifchilddoc |\textit{child-code}| |[|\||else |\textit{main-code}]| \||fi|
\end{center}

%%%%%%%%%%%%%%%%%%%%%%%%%%%%%%%%%%%%%%%%
\DescribeMacro{\childdocname}
\DescribeMacro{\childdocjob}
The macro |\childdocname| contains the filename (without extension)
of the main or child file being processed.
Note that |\childdocjob| will always contain the name of the main file.

%%%%%%%%%%%%%%%%%%%%%%%%%%%%%%%%%%%%%%%%
\paragraph{Title Page.}

Conditional processing can be used to include a title or banner page
in the main document when proper precautions are taken.
Importantly, the code in the main file should ensure that the page counter
(as well as other status parameters which are stored in the |.aux| files)
takes the same value after the conditional processing.
Otherwise the page numbers may take divergent values
depending on which part is compiled.

For example, a title page could be declared by:
%
\begin{center}
\begin{tabular}{l}
|\ifchilddoc\||else|\\
|\addtocounter{page}{-1}|\\
\textit{code for title page}\\
|\newpage|\\
|\||fi|
\end{tabular}
\end{center}
%
A banner page for the child documents can be generated by:
%
\begin{center}
\begin{tabular}{l}
|\ifchilddoc|\\
|\addtocounter{page}{-1}|\\
\textit{code for banner page}\\
|\newpage|\\
|\||fi|
\end{tabular}
\end{center}
%
Here one could write a message such as:
\begin{center}
|This is the part \childdocname{} of \childdocjob{}.|
\end{center}

%%%%%%%%%%%%%%%%%%%%%%%%%%%%%%%%%%%%%%%%%%%%%%%%%%%%%%%%%%%%%%%%%%%%%%%%%%%%%%%%
\subsection{Flags}
\label{sec:flags}

The package makes it easy to generate different versions
of the main or child documents.
To this end compilation flags can be defined
and assigned different default values.
They will be particularly useful in conjunction
with the forwarding mechanism described in \secref{sec:forward}.

For example, it may be useful to have a flag |\version|
which can be set to |draft| or |final|.
The document source will contain some conditional code
depending on the value of |\version|.
Suppose further, the flag should default to |final| for the main file
and to |draft| for child files
which is a natural assignment for editing the document.
This is achieved by placing the following code
in the preamble of the main document
(below the |\childdocmain| directive):
%
\begin{center}
\begin{tabular}{l}
|\ifchilddoc|\\
|\providecommand{\version}{draft}|\\
|\||else|\\
|\providecommand{\version}{final}|\\
|\||fi|
\end{tabular}
\end{center}
%
The definition by |\providecommand| makes sure
that previous definitions are not overwritten.
Further statements |\providecommand{\version}{...}|
can thus be added before the above code to override it.

For the main file, one might add a line
(between |\childdocmain| and the above block)
%
\begin{center}
|%\ifchilddoc\||else\providecommand{\version}{draft}\||fi|
\end{center}
%
which can be uncommented to produce a draft version.
Likewise one can add a line to the very top of a child file
(above the |\childdocof{|\textit{main}|}| directive)
%
\begin{center}
|%\providecommand{\version}{final}|
\end{center}
%
which can be uncommented to produce the final version of this child document.

%%%%%%%%%%%%%%%%%%%%%%%%%%%%%%%%%%%%%%%%%%%%%%%%%%%%%%%%%%%%%%%%%%%%%%%%%%%%%%%%
\subsection{Forwarding}
\label{sec:forward}

Different versions of the main or child documents
using compilation flags as described in \secref{sec:flags}
can be (permanently) stored in different files
for convenient compilation, viewing and distribution.
To this end, the package defines a command
to pass on compilation to a different file:

%%%%%%%%%%%%%%%%%%%%%%%%%%%%%%%%%%%%%%%%
\DescribeMacro{\childdocforward}
The command |\childdocforward| redirects processing to
another source file:
%
\begin{center}
\begin{tabular}{l}
|\input{childdoc.def}|\\
|\childdocforward[|\textit{main}|]{|\textit{dest}|}|\\
\end{tabular}
\end{center}
%
The argument \textit{dest} is the destination file
(without extension).
It should be the main file or one of the child files.
Note that further \textsf{childdoc} directives
such as |\childdocof| and |\childdocforward|
in the indicated file will be processed in this form.
The optional argument \textit{main}
passes on directly to the main file \textit{main}
while pretending to compile the child \textit{dest}.
This form behaves as if \textit{dest}
issues |\childdocof{|\textit{main}|}| right away,
and no further \textsf{childdoc} directives will be processed.

%%%%%%%%%%%%%%%%%%%%%%%%%%%%%%%%%%%%%%%%
\DescribeMacro{\...prefix}
In the alternative form |\childdocforwardprefix|,
%
\begin{center}
\begin{tabular}{l}
|\input{childdoc.def}|\\
|\childdocforwardprefix[|\textit{main}|]{|\textit{prefix}|}{|\textit{dest}|}|
\end{tabular}
\end{center}
%
the destination file is determined by a pattern
depending on the current file:
To make this work, the current file must be called
`{\textit{prefix}\hspace{0.2em}\textit{suffix}}'
with \textit{prefix} matching precisely the argument.
Processing is then passed on to the file
`{\textit{dest}\hspace{0.2em}\textit{suffix}}'.
Surely, the same effect is achieved by
directly specifying the
argument `{\textit{dest}\hspace{0.2em}\textit{suffix}}'
in the first form.
However, that requires to set up a different file
for each child. With the alternative form of the command
all these files can have exactly the same content
which simplifies setting them up and maintaining them.

For example, the following file |draft.tex|
with a compilation flag |\version| as described in \secref{sec:flags}
compiles the main document as a draft:
%
\begin{center}
\begin{tabular}{l}
|\def\version{draft}|\\
|\input{childdoc.def}|\\
|\childdocforward{|\textit{main}|}|
\end{tabular}
\end{center}
%
Likewise, the following files |final|\textit{nn}|.tex|
compile the final version of the child document
|child|\textit{nn}|.tex|:
%
\begin{center}
\begin{tabular}{l}
|\def\version{final}|\\
|\input{childdoc.def}|\\
|\childdocforwardprefix{final}{child}|
\end{tabular}
\end{center}
%

Note that when several versions of a main file and/or of each child file
are to be generated, it may be convenient to set up a |Makefile| or
shell script to automatise the process.

%%%%%%%%%%%%%%%%%%%%%%%%%%%%%%%%%%%%%%%%%%%%%%%%%%%%%%%%%%%%%%%%%%%%%%%%%%%%%%%%
\subsection{Command Line Processing}
\label{sec:commandline}

The effect of redirection files can also be achieved by invoking
the \LaTeX{} compiler with a more elaborate command line.
Most conveniently this should be done as part
of a shell script or a |Makefile|.

When using \textsf{childdoc} in the main file, the following
command lines effectively perform a redirection
(note that depending on the shell being used,
backslashes may have to be doubled: `|\|' $\to$ `|\\|'):
%
\begin{center}
|... -jobname "|\textit{target}|" |\\|"|[\textit{flags}]%
|\input{childdoc.def}\childdocforward[|\textit{main}|]{|\textit{dest}|}"|
\end{center}
%
Here \textit{target} is the name of the output file,
\textit{main} is the name of the main file
and \textit{dest} is the name of the main or child file to be processed
(all filenames without extensions).
The optional argument \textit{main} can be omitted
if \textit{main} matches \textit{dest}.
Optionally, compilation \textit{flags} can be defined via |\def| commands.
This command line makes the \TeX{} engine believe
it is compiling the file \textit{target}
whose content is specified as the latter parameter.
The provided code then forwards the processing to
\textit{main} or \textit{dest} as described in \secref{sec:forward}.

%%%%%%%%%%%%%%%%%%%%%%%%%%%%%%%%%%%%%%%%%%%%%%%%%%%%%%%%%%%%%%%%%%%%%%%%%%%%%%%%
\subsection{Include by Input}
\label{sec:input}

Including child documents by |\include| has some restrictions by design.
Most notably, the content of a child document always occupies
its own set of pages; pages cannot be shared between child documents.
Usually, this behaviour makes perfect sense
because each child document contain an essential part of the document.
However, in some situations it may be desirable to compose
a document from a collection of parts
without having mandatory page breaks between then.
For this case, the package
provides a mechanism to include parts
by |\input| which can also be processed individually.
However, by construction this mechanism
requires manual handling of the content to be output.

%%%%%%%%%%%%%%%%%%%%%%%%%%%%%%%%%%%%%%%%
\DescribeMacro{\ifchilddocmanual}
The main file should be prepared as usual, see \secref{sec:include}.
However, the document body must make a distinction
between processing of an individual part and of the main document, e.g.:
%
\begin{center}
\begin{tabular}{l}
|\ifchilddocmanual|\\
|\input{\childdocname}|\\
|\||else|\\
\textit{document body with }|\input{|\textit{part}|}|\\
|\||fi|
\end{tabular}
\end{center}
%
The conditional |\ifchilddocmanual| is true whenever
a part to be included by |\input| is being compiled,
and the name of the part is stored in |\childdocname|.

%%%%%%%%%%%%%%%%%%%%%%%%%%%%%%%%%%%%%%%%
\DescribeMacro{\childdocby}
Each part to be included by |\input| should start with:
%
\begin{center}
\begin{tabular}{l}
|\input{childdoc.def}|\\
|\childdocby{|\textit{main}|}|\\
\end{tabular}
\end{center}
%
The directive |\childdocby| is similar to |\childdocof|
described in \secref{sec:include},
but the subsequent selection of content must be done manually.
To that end, both |\ifchilddoc| and |\ifchilddocmanual|
will be true upon processing of a part,
and the name of the part is stored in |\childdocname|.
Note that |\jobname| will be set to the filename of the current part
so that each part receives an individual |.aux| file
that does not interfere with the |.aux| file(s) of the main document.
This behaviour can be altered by the alternative form
|\childdocby[*]{|\textit{main}|}| (with a non-empty optional argument)
which uses the |.aux| file of the main document
by setting |\jobname| to \textit{main}.

%%%%%%%%%%%%%%%%%%%%%%%%%%%%%%%%%%%%%%%%%%%%%%%%%%%%%%%%%%%%%%%%%%%%%%%%%%%%%%%%
\subsection{Driver Development}
\label{sec:driver}

The \textsf{childdoc} mechanism can also be use for the development
of definition files such as \LaTeX{} styles or classes.
This case differs from the above setup with multiple parts
included by |\include| in that no |\includeonly| should be invoked.
This can be achieved by starting the include file
(before |\ProvidesPackage|) with:
%
\begin{center}
\begin{tabular}{l}
|\input{childdoc.def}|\\
|\childdocforward{|\textit{main}|}|\\
\end{tabular}
\end{center}
%
or alternatively with:
%
\begin{center}
\begin{tabular}{l}
|\input{childdoc.def}|\\
|\childdocby{|\textit{main}|}|\\
\end{tabular}
\end{center}
%
Both forms have slightly different effects as described above.
The main file is prepared as usual, see \secref{sec:include}.

%%%%%%%%%%%%%%%%%%%%%%%%%%%%%%%%%%%%%%%%%%%%%%%%%%%%%%%%%%%%%%%%%%%%%%%%%%%%%%%%
\subsection{Legacy Detection}
\label{sec:detection}

The directive |\childdocmain| in the main file can detect
whether the complete document or merely a child is to be compiled
even without using the directive |\childdocof|.
This method is deprecated because it is less robust
and there is no compelling reason to use it;
it is merely provided for backward compatibility
and it may be removed in future versions.

If the detection mechanism is to be used,
it is mandatory to correctly specify
the filename of the main file as the argument of |\childdocmain|:
%
\begin{center}
\begin{tabular}{l}
|\input{childdoc.def}|\\
|\childdocmain{|\textit{main}|}|\\
\end{tabular}
\end{center}
%
If |\jobname| does not match the argument \textit{main} of |\childdocmain|,
it is assumed that |\jobname| points to the child file to be compiled.
When using |\childdocmain| with the main file specified as argument,
it suffices to start a child file
with just |\input{|\textit{main}|}|
without loading of the package and using |\childdocof|.
If instead all processing is done
with the appropriate \textsf{childdoc} directives,
the argument of \textit{main} of |\childdocmain| can be empty.

An alternative version of the command line processing described
in \secref{sec:commandline} using the detection mechanism reads:
%
\begin{center}
|... -jobname "|\textit{target}|" "|[\textit{flags}]%
[|\def\jobname{|\textit{dest}|}|]|\input{|\textit{main}|}"|
\end{center}

%%%%%%%%%%%%%%%%%%%%%%%%%%%%%%%%%%%%%%%%%%%%%%%%%%%%%%%%%%%%%%%%%%%%%%%%%%%%%%%%
\subsection{Manual Code}
\label{sec:manual}

In case one cannot be certain whether the definitions file |childdoc.def|
is installed on the target \TeX{} distribution
and one prefers not to ship it,
it is conceivable to paste a few relevant commands into the sources.

To that end, drop all statements |\input{childdoc.def}|
and perform the replacements as outlined below.
Instead of |\childdocmain{|\textit{main}|}| add the following code
to the top of the main file:
%
\begin{center}
\begin{tabular}{l}
|\||ifdefined\childdocname\endinput\||fi\newif\ifchilddoc|\\
|\edef\childdocname{\scantokens\expandafter{\jobname\noexpand}}|\\
|\def\childdocmain{|\textit{main}|}\||ifx\childdocmain\childdocname\||else|\\
|\childdoctrue\includeonly{\childdocname}\let\jobname\childdocmain\||fi|\\
\end{tabular}
\end{center}
%
Instead of |\childdocof{|\textit{main}|}| just include the main file
at the top of each child file:
%
\begin{center}
|\input{|\textit{main}|}|
\end{center}
%
A simple redirection |\childdocforward{|\textit{dest}|}| is achieved by:
%
\begin{center}
|\def\jobname{|\textit{dest}|}\input{\jobname}|
\end{center}
%
The redirection with prefix
|\childdocforwardprefix[|\textit{prefix}|]{|\textit{dest}|}|
is accomplished by:
%
\begin{center}
\begin{tabular}{l}
|{\edef\jobname{\scantokens\expandafter{\jobname\noexpand}}|\\
|\def\redirectjob |\textit{prefix}|#1~~~{\gdef\jobname{|\textit{dest}|#1}}|\\
|\expandafter\redirectjob\jobname~~~}\input{\jobname}|
\end{tabular}
\end{center}

In an alternative approach,
child documents can be compiled by a specific command line
without additional code or specific definitions:
%
\begin{center}
|... -jobname "|\textit{target}|" "|[\textit{flags}]%
|\includeonly{|\textit{dest}|}\input{|\textit{main}|}"|
\end{center}
%

%%%%%%%%%%%%%%%%%%%%%%%%%%%%%%%%%%%%%%%%%%%%%%%%%%%%%%%%%%%%%%%%%%%%%%%%%%%%%%%%
%%%%%%%%%%%%%%%%%%%%%%%%%%%%%%%%%%%%%%%%%%%%%%%%%%%%%%%%%%%%%%%%%%%%%%%%%%%%%%%%
\section{Information}

%%%%%%%%%%%%%%%%%%%%%%%%%%%%%%%%%%%%%%%%%%%%%%%%%%%%%%%%%%%%%%%%%%%%%%%%%%%%%%%%
\subsection{Copyright}

Copyright \copyright{} 2017--2018 Niklas Beisert

This work may be distributed and/or modified under the
conditions of the \LaTeX{} Project Public License, either version 1.3
of this license or (at your option) any later version.
The latest version of this license is in
  \url{http://www.latex-project.org/lppl.txt}
and version 1.3 or later is part of all distributions of \LaTeX{}
version 2005/12/01 or later.

This work has the LPPL maintenance status `maintained'.

The Current Maintainer of this work is Niklas Beisert.

This work consists of the files |README.txt|, |childdoc.ins| and |childdoc.dtx|
as well as the derived files |childdoc.def|, |cdocsamp.tex|
with |cdocsch1.tex|, |cdocsch2.tex|, |cdocspt3.tex|, |cdocspt4.tex|,
|cdocsdrf.tex|, |cdocsfn1.tex|, |cdocsfn2.tex|
as well as |childdoc.pdf|.

%%%%%%%%%%%%%%%%%%%%%%%%%%%%%%%%%%%%%%%%%%%%%%%%%%%%%%%%%%%%%%%%%%%%%%%%%%%%%%%%
\subsection{Files and Installation}

The package consists of the files:
%
\begin{center}
\begin{tabular}{ll}
    |README.txt|   & readme file \\
    |childdoc.ins| & installation file \\
    |childdoc.dtx| & source file \\
    |childdoc.def| & definition file \\
    |cdocsamp.tex| & sample main file \\
    |cdocsch1.tex| & sample include file \\
    |cdocsch2.tex| & sample include file \\
    |cdocspt3.tex| & sample part file \\
    |cdocspt4.tex| & sample part file \\
    |cdocsdrf.tex| & sample redirection file \\
    |cdocsfn1.tex| & sample redirection file \\
    |cdocsfn2.tex| & sample redirection file \\
    |childdoc.pdf| & manual
\end{tabular}
\end{center}
%
The distribution consists of the files
|README.txt|, |childdoc.ins| and |childdoc.dtx|.
%
\begin{itemize}
\item
Run (pdf)\LaTeX{} on |childdoc.dtx|
to compile the manual |childdoc.pdf| (this file).
\item
Run \LaTeX{} on |childdoc.ins| to create the definitions file |childdoc.def|
and the sample |cdocsamp.tex| with include files
|cdocsch1.tex|, |cdocsch2.tex|, |cdocspt3.tex|, |cdocspt4.tex|,
|cdocsdrf.tex|, |cdocsfn1.tex|, |cdocsfn2.tex|.
Then copy the file |childdoc.def| to an appropriate directory of your \LaTeX{}
distribution, e.g.\ \textit{texmf-root}|/tex/latex/childdoc|.
\end{itemize}

%%%%%%%%%%%%%%%%%%%%%%%%%%%%%%%%%%%%%%%%%%%%%%%%%%%%%%%%%%%%%%%%%%%%%%%%%%%%%%%%
\subsection{Related CTAN Packages}

There are several other packages which offer a similar functionality:
%
\begin{itemize}
\item
The packages
\href{http://ctan.org/pkg/docmute}{\textsf{docmute}},
\href{http://ctan.org/pkg/includex}{\textsf{includex}} and
\href{http://ctan.org/pkg/standalone}{\textsf{standalone}}
provide commands to include only the document body of
a child file thus allowing both files to be compiled individually.
\item
The packages \href{http://ctan.org/pkg/subdocs}{\textsf{subdocs}}
and \href{http://ctan.org/pkg/subfiles}{\textsf{subfiles}}
provide structures in which the main and child documents can be
encapsulated and allowing them to be compiled individually.
The inclusion mechanism is different from the conventional |\include|.
\item
The package \href{http://ctan.org/pkg/combine}{\textsf{combine}}
is an elaborate solution to combine several documents into one.
\end{itemize}
%
See also the CTAN topic \href{http://ctan.org/topic/subdocs}{\textsf{subdocs}}
for further related packages.
The present package differs from the above solutions in that
a document structure constructed with the conventional |\include| mechanism
just needs two extra commands at the top of every file
such that all constituent files can be compiled individually.

%%%%%%%%%%%%%%%%%%%%%%%%%%%%%%%%%%%%%%%%%%%%%%%%%%%%%%%%%%%%%%%%%%%%%%%%%%%%%%%%
%\subsection{Feature Suggestions}
%
%The following is a list of features which may be useful for future
%versions of this package:
%%
%\begin{itemize}
%\item
%\ldots
%\end{itemize}

%%%%%%%%%%%%%%%%%%%%%%%%%%%%%%%%%%%%%%%%%%%%%%%%%%%%%%%%%%%%%%%%%%%%%%%%%%%%%%%%
\subsection{Revision History}

%%%%%%%%%%%%%%%%%%%%%%%%%%%%%%%%%%%%%%%%
\paragraph{v2.0:} 2018/12/30

\begin{itemize}
\item
immediate forward processing
\item
added |\childdocby| mechanism
\item
manual restructured
\end{itemize}

%%%%%%%%%%%%%%%%%%%%%%%%%%%%%%%%%%%%%%%%
\paragraph{v1.6:} 2018/01/17

\begin{itemize}
\item
application for development of include files
\item
corrections to manual
\end{itemize}

%%%%%%%%%%%%%%%%%%%%%%%%%%%%%%%%%%%%%%%%
\paragraph{v1.5:} 2017/05/21

\begin{itemize}
\item
more complete structuring introduced
\item
|\childdocof| introduced
\item
|\childdoc| renamed to |\childdocmain|
\item
|\childredirect| renamed to |\childdocforward| and |\childdocforwardprefix|
and functionality expanded
\end{itemize}

%%%%%%%%%%%%%%%%%%%%%%%%%%%%%%%%%%%%%%%%
\paragraph{v1.0:} 2017/04/27

\begin{itemize}
\item
manual and install package
\item
first version published on CTAN
\end{itemize}

%%%%%%%%%%%%%%%%%%%%%%%%%%%%%%%%%%%%%%%%
\paragraph{v0.6:} 2017/04/26

\begin{itemize}
\item
redirection mechanism added
\end{itemize}

%%%%%%%%%%%%%%%%%%%%%%%%%%%%%%%%%%%%%%%%
\paragraph{v0.5:} 2017/04/26

\begin{itemize}
\item
functionality in definition file
\end{itemize}


%%%%%%%%%%%%%%%%%%%%%%%%%%%%%%%%%%%%%%%%%%%%%%%%%%%%%%%%%%%%%%%%%%%%%%%%%%%%%%%%
%%%%%%%%%%%%%%%%%%%%%%%%%%%%%%%%%%%%%%%%%%%%%%%%%%%%%%%%%%%%%%%%%%%%%%%%%%%%%%%%
%%%%%%%%%%%%%%%%%%%%%%%%%%%%%%%%%%%%%%%%%%%%%%%%%%%%%%%%%%%%%%%%%%%%%%%%%%%%%%%%
\appendix

\settowidth\MacroIndent{\rmfamily\scriptsize 000\ }

 \DocInput{childdoc.dtx}

\end{document}
%</driver>
% \fi
%
% %%%%%%%%%%%%%%%%%%%%%%%%%%%%%%%%%%%%%%%%%%%%%%%%%%%%%%%%%%%%%%%%%%%%%%%%%%%%%%
% %%%%%%%%%%%%%%%%%%%%%%%%%%%%%%%%%%%%%%%%%%%%%%%%%%%%%%%%%%%%%%%%%%%%%%%%%%%%%%
% \section{Sample}
%\iffalse
%<*samplemain>
%\fi
%
% The following presents a sample document
% with two chapters, two parts, a title page,
% a compile flag as well as three forwarding files to set the flag.
% It consists of eight |.tex| files:
% \begin{center}
% \begin{tabular}{ll}
% |cdocsamp.tex|&main file\\
% |cdocsch1.tex|&include file for chapter 1\\
% |cdocsch2.tex|&include file for chapter 2\\
% |cdocspt3.tex|&include file for part 3\\
% |cdocspt4.tex|&include file for part 4\\
% |cdocsdrf.tex|&forwarding file for main file in draft mode\\
% |cdocsfi1.tex|&forwarding file for final version of chapter 1\\
% |cdocsfi2.tex|&forwarding file for final version of chapter 2\\
% \end{tabular}
% \end{center}
% Each of the eight files can be compiled directly by the \LaTeX{} compiler.
%
% %%%%%%%%%%%%%%%%%%%%%%%%%%%%%%%%%%%%%%
% \paragraph{Main File.}
%
% The main file is called |cdocsamp.tex|.
%
% Load the \textsf{childdoc} definitions and
% declare the filename for the main document:
%    \begin{macrocode}
\input{childdoc.def}
\childdocmain{}
%    \end{macrocode}

% Optional override for |\version| flag:
%    \begin{macrocode}
%%\ifchilddoc\else\providecommand{\version}{draft}\fi
%    \end{macrocode}

% Define the default values for the |\version| flag
% (|final| for the main file and |draft| for childs):
%    \begin{macrocode}
\ifchilddoc
\providecommand{\version}{draft}
\else
\providecommand{\version}{final}
\fi
%    \end{macrocode}

% Load the standard document class:
%    \begin{macrocode}
\documentclass[12pt]{article}
%    \end{macrocode}

% Start the document body:
%    \begin{macrocode}
\begin{document}
%    \end{macrocode}

% Declare a title page.
% Print title, part of document being processed and version flag:
%    \begin{macrocode}
\addtocounter{page}{-1}
\begin{center}
{\LARGE\bfseries{}childdoc example\par}
\vspace{1cm}
\ifchilddoc
\ifchilddocmanual part\else chapter\fi:
`\childdocname' of `\childdocjob'\par
\else
main document: `\childdocjob'\par
\fi
version: \version\par
\end{center}
\newpage
%    \end{macrocode}

% Manually include selected file,
% otherwise process as usual:
%    \begin{macrocode}
\ifchilddocmanual
\section*{part `\childdocname'}
\input{\childdocname}
\else
%    \end{macrocode}

% Include the two chapters:
%    \begin{macrocode}
\include{cdocsch1}
\include{cdocsch2}
%    \end{macrocode}

% Include the two parts unless only chapters should be displayed:
%    \begin{macrocode}
\ifchilddoc\else
\section{part three}
\input{cdocspt3}
\section{part four}
\input{cdocspt4}
\fi
%    \end{macrocode}

% Process as usual until here:
%    \begin{macrocode}
\fi
%    \end{macrocode}

% End of document body:
%    \begin{macrocode}
\end{document}
%    \end{macrocode}
%\iffalse
%</samplemain>
%\fi
%
% %%%%%%%%%%%%%%%%%%%%%%%%%%%%%%%%%%%%%%
% \paragraph{Chapter Include Files.}
%
% The include files are called |cdocsch1.tex| and |cdocsch2.tex|.
%
%\iffalse
%<*samplechap1|samplechap2>
%\fi

% Optional override for |\version| flag:
%    \begin{macrocode}
%%\providecommand{\version}{final}
%    \end{macrocode}

% Include the main document:
%    \begin{macrocode}
\input{childdoc.def}
\childdocof{cdocsamp}
%    \end{macrocode}

%\iffalse
%</samplechap1|samplechap2>
%\fi
%
%\iffalse
%<*samplechap1>
%\fi
% Some text for chapter 1:
%    \begin{macrocode}
\section{one}
some text in chapter one
%    \end{macrocode}

%\iffalse
%</samplechap1>
%\fi
% Some text for chapter 2:
%\iffalse
%<*samplechap2>
%\fi
%    \begin{macrocode}
\section{two}
more text in chapter two
%    \end{macrocode}

%\iffalse
%</samplechap2>
%\fi
%
% %%%%%%%%%%%%%%%%%%%%%%%%%%%%%%%%%%%%%%
% \paragraph{Part Include Files.}
%
% The include files are called |cdocspt3.tex| and |cdocspt4.tex|.
%
%\iffalse
%<*samplepart3|samplepart4>
%\fi

% Optional override for |\version| flag:
%    \begin{macrocode}
%%\providecommand{\version}{final}
%    \end{macrocode}

% Include the main document:
%    \begin{macrocode}
\input{childdoc.def}
\childdocby{cdocsamp}
%    \end{macrocode}

%\iffalse
%</samplepart3|samplepart4>
%\fi
%
%\iffalse
%<*samplepart3>
%\fi
% Some text for part 3:
%    \begin{macrocode}
some text in part three
%    \end{macrocode}

%\iffalse
%</samplepart3>
%\fi
% Some text for part 4:
%\iffalse
%<*samplepart4>
%\fi
%    \begin{macrocode}
more text in part four
%    \end{macrocode}

%\iffalse
%</samplepart4>
%\fi
%
% %%%%%%%%%%%%%%%%%%%%%%%%%%%%%%%%%%%%%%
% \paragraph{Forwarding for a Complete Draft.}
%
% The following forwarding file |cdocsdrf.tex|
% compiles the main document in draft mode:
%\iffalse
%<*sampledraft>
%\fi
%    \begin{macrocode}
\def\version{draft}
\input{childdoc.def}
\childdocforward{cdocsamp}
%    \end{macrocode}

%\iffalse
%</sampledraft>
%\fi
%
% %%%%%%%%%%%%%%%%%%%%%%%%%%%%%%%%%%%%%%
% \paragraph{Forwarding for Final Version of the Chapters.}
%
% The following forwarding files |cdocsfn1.tex| and |cdocsfn2.tex|
% (with identical content)
% compile the final versions of the child documents
% |cdocsch1.tex| and |cdocsch2.tex|, respectively:
%\iffalse
%<*samplefinal>
%\fi
%    \begin{macrocode}
\def\version{final}
\input{childdoc.def}
\childdocforwardprefix[cdocsamp]{cdocsfn}{cdocsch}
%    \end{macrocode}

%\iffalse
%</samplefinal>
%\fi
%
% %%%%%%%%%%%%%%%%%%%%%%%%%%%%%%%%%%%%%%
% \paragraph{Command Line Processing.}
%
% The following three command lines generate the output files
% |cdocscld|, |cdocscl1| and |cdocscl2|
% which should be identical to
% |cdocsdrf|, |cdocsch1| and |cdocsfn2|, respectively:
% \begin{center}
% \begin{tabular}{l}
% |latex -jobname cdocscld \|\\
% |  "\def\version{draft}\input{childdoc.def}\childdocforward{cdocsamp}"|\\
% |latex -jobname cdocscl1 \|\\
% |  "\input{childdoc.def}\childdocforward[cdocsamp]{cdocsch1}"|\\
% |latex -jobname cdocscl2 \|\\
% |  "\def\version{final}\input{childdoc.def}\childdocforward{cdocsch2}"|
% \end{tabular}
% \end{center}
% Note that the trailing backslash on each first line
% merely continues the input to the second line
% (for convenient cut ant paste).
% Furthermore, the command |latex| can be replaced by any
% of its alternative versions such as |pdflatex|.
%
% %%%%%%%%%%%%%%%%%%%%%%%%%%%%%%%%%%%%%%%%%%%%%%%%%%%%%%%%%%%%%%%%%%%%%%%%%%%%%%
% %%%%%%%%%%%%%%%%%%%%%%%%%%%%%%%%%%%%%%%%%%%%%%%%%%%%%%%%%%%%%%%%%%%%%%%%%%%%%%
% \section{Implementation}
%\iffalse
%<*package>
%\fi
%
% This section describes the definitions file |childdoc.def|.

% The definitions cannot be loaded using |\usepackage| or |\RequirePackage|
% which has a mechanism to prevent loading a style file more than once.
% When loading the definitions by means of |\input|
% multiple instances have to be prevented manually:
%\iffalse
%This code needs to be before the `\ProvidesFile' directive
%which is defined at the beginning of this file.
%Therefore it is also placed there and commented out here.
%</package>
%<*discard>
%\fi
%    \begin{macrocode}
\ifdefined\childdocmain\endinput\fi
%    \end{macrocode}
%\iffalse
%</discard>
%<*package>
%\fi
%
% \macro{\ifchilddoc}
% \macro{\ifchilddocmanual}
% The conditional |\ifchilddoc| tells whether a
% child (true) or main (false) document is being compiled.
% The conditional |\ifchilddocmanual| tells whether
% the |\includeonly| mechanism is used (false) or
% the selection of child files must be performed manually (true).
% The definitions initialise to false:
%    \begin{macrocode}
\newif\ifchilddoc
\newif\ifchilddocmanual
%    \end{macrocode}

% \macro{\childdocname}
% \macro{\childdocjob}
% The macro |\childdocname| stores the name of the main document
% to be compiled. The macro |\childdocjob| stores the name of
% the document on which the \LaTeX{} compiler was originally invoked.
% The content of |\jobname| cannot be compared
% to filenames specified in the source due to different catcodes.
% The following code rescans |\jobname|, stores the result
% in |\childdocname| and saves a copy in |\childdocjob|:
%    \begin{macrocode}
\edef\childdocname{\scantokens\expandafter{\jobname\noexpand}}
\let\childdocjob\childdocname
%    \end{macrocode}

% \macro{\childdocdisable}
% The macro |\childdocdisable| prevents the main file
% from being processed more than once.
% At this stage, the main document command |\childdocmain|
% is assumed to be called once again where it should do nothing.
% Any subsequent call to it should prevent
% a secondary processing of the main document
% It overwrites the forwarding commands
% |\childdocof| and |\childdocforward|
% with empty macros to prevent further inclusions of the main document:
%    \begin{macrocode}
\newcommand{\childdocdisable}
{
  \renewcommand{\childdocmain}[1]{\renewcommand{\childdocmain}[1]{\endinput}}
  \renewcommand{\childdocof}[1]{}
  \renewcommand{\childdocby}[2][]{}
  \renewcommand{\childdocforward}[2][]{}
  \renewcommand{\childdocdisable}{}
}
%    \end{macrocode}

% \macro{\childdocmain}
% The macro |\childdocmain| is to be called at the top of the main file
% with nothing or the main filename (without extension) as argument.
% First, it breaks loops.
% If the argument is not empty and does not match |\childdocname|
% (which is set by the first inclusion of |childdoc.def|),
% |\ifchilddoc| is set to true, |\includeonly| is applied to the child file
% and |\jobname| is set to the main file
% (for proper handling of |.aux| files):
%    \begin{macrocode}
\newcommand{\childdocmain}[1]
{
  \childdocdisable\childdocmain{}
  \if?#1?\else
    \begingroup
      \def\childdoctmp{#1}
      \ifx\childdoctmp\childdocname
        \def\childdoctmp{}
      \else
        \def\childdoctmp
        {
          \childdoctrue
          \includeonly{\childdocname}
          \def\childdocjob{#1}
          \def\jobname{#1}
        }
      \fi
      \expandafter
    \endgroup
    \childdoctmp
  \fi
}
%    \end{macrocode}

% \macro{\childdocof}
% The command |\childdocof| redirects
% compilation to the main file |#1|.
%    \begin{macrocode}
\newcommand{\childdocof}[1]
{
  \childdocdisable
  \childdoctrue
  \includeonly{\childdocname}
  \def\jobname{#1}
  \def\childdocjob{#1}
  \input{#1}
}
%    \end{macrocode}

% \macro{\childdocby}
% The command |\childdocby| ....
%    \begin{macrocode}
\newcommand{\childdocby}[2][]
{
  \childdocdisable
  \childdoctrue
  \childdocmanualtrue
  \if?#1?\else
    \def\jobname{#2}
  \fi
  \def\childdocjob{#2}
  \input{#2}
  \endinput
}
%    \end{macrocode}

% \macro{\childdocforward}
% The command |\childdocforward| redirects
% compilation to the main file or
% (if the optional argument is given) a child file.
% Parameters are set as if the main file
% or a child file starting with |\childdocof| was compiled.
% Then compilation is handed over to the main file:
%    \begin{macrocode}
\newcommand{\childdocforward}[2][]
{
  \begingroup
    \if?#1?
      \def\childdoctmp
      {
        \def\childdocname{#2}
        \def\childdocjob{#2}
        \def\jobname{#2}
        \input{#2}
        \endinput
      }
    \else
      \def\childdoctmp
      {
        \childdocdisable
        \def\childdocname{#2}
        \childdoctrue
        \includeonly{#2}
        \def\childdocjob{#1}
        \def\jobname{#1}
        \input{#1}
        \endinput
      }
    \fi
    \expandafter
  \endgroup
  \childdoctmp
}
%    \end{macrocode}

% \macro{\childdocforwardprefix}
% The command |\childdocforwardprefix| redirects
% compilation to the main or a child file by means of a pattern.
% The prefix |#1| in the current filename is replaced by |#2|
% and the suffix of the current filename is kept
% (it is assumed that the filename does not contain the substring `|~~~|'
% which is used as a delimiter).
% Compilation is handed over to the new file by |\childdocforward|:
%    \begin{macrocode}
\newcommand{\childdocforwardprefix}[3][]
{
  \begingroup
    \def\childdocextract #2##1~~~{\def\childdoctmp{\childdocforward[#1]{#3##1}}}
    \expandafter\childdocextract\childdocname~~~
    \expandafter
  \endgroup
  \childdoctmp
}
%    \end{macrocode}

% \macro{\childdoc}
% The deprecated macro |\childdoc| is a legacy version of |\childdocmain|:
%    \begin{macrocode}
\newcommand{\childdoc}{\childdocmain}
%    \end{macrocode}

% \macro{\childdocredirect}
% The deprecated macro |\childdocredirect| is a legacy version
% of |\childdocforward| and |\childdocforwardprefix|:
%    \begin{macrocode}
\newcommand{\childdocredirect}[2][]
{
  \begingroup
    \if?#1?
      \def\childdoctmp{\childdocforward{#2}}
    \else
      \def\childdoctmp{\childdocforwardprefix{#1}{#2}}
    \fi
    \expandafter
  \endgroup
  \childdoctmp
}
%    \end{macrocode}

%\iffalse
%</package>
%\fi
%
\endinput
|\\
|\childdocforwardprefix[|\textit{main}|]{|\textit{prefix}|}{|\textit{dest}|}|
\end{tabular}
\end{center}
%
the destination file is determined by a pattern
depending on the current file:
To make this work, the current file must be called
`{\textit{prefix}\hspace{0.2em}\textit{suffix}}'
with \textit{prefix} matching precisely the argument.
Processing is then passed on to the file
`{\textit{dest}\hspace{0.2em}\textit{suffix}}'.
Surely, the same effect is achieved by
directly specifying the
argument `{\textit{dest}\hspace{0.2em}\textit{suffix}}'
in the first form.
However, that requires to set up a different file
for each child. With the alternative form of the command
all these files can have exactly the same content
which simplifies setting them up and maintaining them.

For example, the following file |draft.tex|
with a compilation flag |\version| as described in \secref{sec:flags}
compiles the main document as a draft:
%
\begin{center}
\begin{tabular}{l}
|\def\version{draft}|\\
|% \iffalse
%
% childdoc.dtx Copyright (C) 2017-2018 Niklas Beisert
%
% This work may be distributed and/or modified under the
% conditions of the LaTeX Project Public License, either version 1.3
% of this license or (at your option) any later version.
% The latest version of this license is in
%   http://www.latex-project.org/lppl.txt
% and version 1.3 or later is part of all distributions of LaTeX
% version 2005/12/01 or later.
%
% This work has the LPPL maintenance status `maintained'.
%
% The Current Maintainer of this work is Niklas Beisert.
%
% This work consists of the files childdoc.dtx and childdoc.ins
% and the derived files childdoc.def and cdocsamp.tex with
% cdocsch1.tex, cdocsch2.tex, cdocsdrf.tex, cdocsfn1.tex, cdocsfn2.tex.
%
%<package>\ifdefined\childdocmain\endinput\fi
%<package>\ProvidesFile{childdoc.def}[2018/12/30 v2.0 child document driver]
%<samplemain>\ProvidesFile{cdocsamp.tex}[2018/12/30 v2.0 sample for childdoc]
%<*driver>
%\ProvidesFile{childdoc.drv}[2018/12/30 v2.0 childdoc reference manual file]
\PassOptionsToClass{10pt,a4paper}{article}
\documentclass{ltxdoc}

\usepackage[margin=35mm]{geometry}
\usepackage{hyperref}
\usepackage{hyperxmp}
\usepackage[usenames]{color}

\hypersetup{colorlinks=true}
\hypersetup{pdfstartview=FitH}
\hypersetup{pdfpagemode=UseNone}
\hypersetup{pdfsource={}}
\hypersetup{pdflang={en-UK}}
\hypersetup{pdfcopyright={Copyright 2017-2018 Niklas Beisert.
  This work may be distributed and/or modified under the
  conditions of the LaTeX Project Public License, either version 1.3
  of this license or (at your option) any later version.}}
\hypersetup{pdflicenseurl={http://www.latex-project.org/lppl.txt}}
\hypersetup{pdfcontactaddress={ETH Zurich, ITP, HIT K,
  Wolfgang-Pauli-Strasse 27}}
\hypersetup{pdfcontactpostcode={8093}}
\hypersetup{pdfcontactcity={Zurich}}
\hypersetup{pdfcontactcountry={Switzerland}}
\hypersetup{pdfcontactemail={nbeisert@itp.phys.ethz.ch}}
\hypersetup{pdfcontacturl={http://people.phys.ethz.ch/\xmptilde nbeisert/}}

\newcommand{\secref}[1]{\hyperref[#1]{section \ref*{#1}}}

\parskip1ex
\parindent0pt
\let\olditemize\itemize
\def\itemize{\olditemize\parskip0pt}

\begin{document}

\title{The \textsf{childdoc} Package}
\hypersetup{pdftitle={The childdoc Package}}
\author{Niklas Beisert\\[2ex]
  Institut f\"ur Theoretische Physik\\
  Eidgen\"ossische Technische Hochschule Z\"urich\\
  Wolfgang-Pauli-Strasse 27, 8093 Z\"urich, Switzerland\\[1ex]
  \href{mailto:nbeisert@itp.phys.ethz.ch}
  {\texttt{nbeisert@itp.phys.ethz.ch}}}
\hypersetup{pdfauthor={Niklas Beisert}}
\hypersetup{pdfsubject={Manual for the LaTeX2e Package childdoc}}
\date{30 December 2018, \textsf{v2.0}}
\maketitle

\begin{abstract}\noindent
\textsf{childdoc} is a \LaTeXe{} package
that enables the direct compilation
of document sections included by |\include|
to individual files.
\end{abstract}

\begingroup
\parskip0ex
\tableofcontents
\endgroup

%%%%%%%%%%%%%%%%%%%%%%%%%%%%%%%%%%%%%%%%%%%%%%%%%%%%%%%%%%%%%%%%%%%%%%%%%%%%%%%%
%%%%%%%%%%%%%%%%%%%%%%%%%%%%%%%%%%%%%%%%%%%%%%%%%%%%%%%%%%%%%%%%%%%%%%%%%%%%%%%%
\section{Introduction}

\LaTeX{} provides a mechanism to structure a large document (such as a book)
into a main file and several child files (containing the chapters)
using the |\include| command.
This mechanism is beneficial for documents
which span hundreds of pages in order to
make the source file(s) more manageable.
Moreover, compilation can be restricted to
selected child files by means of the |\includeonly| command.
The latter feature can be used to reduce the compilation time while editing
(this was significantly more useful in the earlier days of \LaTeX{})
or to generate a smaller document which is easier to navigate.
Another application of |\includeonly| is to generate
documents consisting of selected parts of the complete document.

However, there are a few drawbacks of the plain |\include| mechanism:
\begin{itemize}
\item
The child files cannot be compiled on their own,
they can only be compiled via the main file.
A naive editing environment
(such as a text editor with an option
to have the current file processed by \LaTeX)
may require one to switch to the main file before compiling;
attempting to compile the child file produces errors.
\item
The main file must be modified (each time)
to adjust the |\includeonly| command
to the present needs. This easily leaves the main file in a messy state.
\item
The generated document will always carry the filename
of the main document. This is inconvenient if
several child files are to be compiled and
to be kept for distribution.
\end{itemize}

The present package provides a simple interface
to make child files individually compilable by \LaTeX{}.
Compiling a child file then has the same effect as compiling
the main file with an |\includeonly| command
to select the appropriate child.
Moreover the generated document will carry the name of the child
rather than the main file.
This resolves all three above issues.

This feature is meant to make the editing of books,
thesis documents and lecture notes somewhat more convenient.
However, the package can also be used efficiently for
composing a series of documents (such as exercise sheets)
which are typically distributed individually.
It then assists the author in generating the individual documents
(potentially in different versions)
as well as a document containing the collected series.
Another application is in developing style files
or other kinds of included material
where compilation of the style file could redirect
to a sample or test file.

%%%%%%%%%%%%%%%%%%%%%%%%%%%%%%%%%%%%%%%%%%%%%%%%%%%%%%%%%%%%%%%%%%%%%%%%%%%%%%%%
%%%%%%%%%%%%%%%%%%%%%%%%%%%%%%%%%%%%%%%%%%%%%%%%%%%%%%%%%%%%%%%%%%%%%%%%%%%%%%%%
\section{Usage}

First of all, the package \textsf{childdoc} is \emph{not} a standard
\LaTeXe{} |.sty| style file! Therefore it needs to be invoked in
a non-standard way.

%%%%%%%%%%%%%%%%%%%%%%%%%%%%%%%%%%%%%%%%%%%%%%%%%%%%%%%%%%%%%%%%%%%%%%%%%%%%%%%%
\subsection{Included Files}
\label{sec:include}

%%%%%%%%%%%%%%%%%%%%%%%%%%%%%%%%%%%%%%%%
\DescribeMacro{\childdocmain}
To use the package, add the commands
\begin{center}
\begin{tabular}{l}
|\input{childdoc.def}|\\
|\childdocmain{}|\\
\end{tabular}
\end{center}
at the very top of the main \LaTeX{} file,
in particular \emph{before} the |\documentclass| statement!
The argument of |\childdocmain| should be left empty
(but it must be present).

%%%%%%%%%%%%%%%%%%%%%%%%%%%%%%%%%%%%%%%%
\DescribeMacro{\childdocof}
Furthermore, add the commands
\begin{center}
\begin{tabular}{l}
|\input{childdoc.def}|\\
|\childdocof{|\textit{main}|}|\\
\end{tabular}
\end{center}
at the top of every child file \textit{child}
which is included by |\include{|\textit{child}|}|
from within the main file
(or at least for those files to be compiled individually).
The argument \textit{main} must be the filename of the main file.

There are a couple of
considerations in setting up the main and child documents:

%%%%%%%%%%%%%%%%%%%%%%%%%%%%%%%%%%%%%%%%
\paragraph{Restrictions.}

Please note the following restrictions:
\begin{itemize}
\item
|\childdocmain| must be called with one argument \textit{main}
to ensure compatibility with earlier version of the package.
It must either be empty (|\childdocmain{}|)
or precisely match the filename of the main file in which it is specified.
See \secref{sec:detection} for further information.
\item
The filename \textit{main} must be specified without the |.tex| extension.
\item
The filename \textit{main} is case sensitive
(even in case-insensitive file systems)
due to internal string comparison.
\item
The argument \textit{main} should be fully expanded, it cannot be a macro.
\item
Subdirectories and special characters should be avoided in filenames.
\item
The command |\childdocmain{|\textit{main}|}| must be followed by a whitespace.
It should not be followed immediately by another command
or by a comment mark `|%|'.
This is because the \TeX{} parser reads the token immediately following
the argument of |\childdocmain| and puts it
at the beginning of every child section;
however, a white\-space is ignored.
\end{itemize}

%%%%%%%%%%%%%%%%%%%%%%%%%%%%%%%%%%%%%%%%
\paragraph{Content of Main File.}

It is advisable to place all content in the child files included by |\include|.
Any output contained in the main file will appear in all child documents
unless suppressed manually;
it cannot be suppressed automatically by the |\includeonly| directive
and thus should normally be avoided.
A method to include some content in the main file
by means of conditional processing is described in \secref{sec:conditional}.

%%%%%%%%%%%%%%%%%%%%%%%%%%%%%%%%%%%%%%%%
\paragraph{Page Numbering.}

When only a part of the document is compiled,
the appropriate numbering of pages
(as well as other status parameters)
is determined from the |.aux| files.
The latter contain information from previous passes.
However this information needs to propagate through
all intermediate child documents.
Therefore the page numbering in child documents may well
be inconsistent until the complete document is compiled at least once.

A useful (if unconventional) way to always ensure a consistent
page numbering is to restart the numbering in each child document
and denote the pages by `\textit{child}|.|\textit{page}'
where \textit{child} represents the chapter/section number of the child file.
This can be achieved by the command
|\numberwithin{page}{|\textit{child}|}|
of the \textsf{amsmath} package
where \textit{child} can be |chapter| or |section|
depending on the chosen structuring.
Alternatively, one can modify the macro |\thepage| appropriately
and reset the counter |page| at the start of each child file.

%%%%%%%%%%%%%%%%%%%%%%%%%%%%%%%%%%%%%%%%%%%%%%%%%%%%%%%%%%%%%%%%%%%%%%%%%%%%%%%%
\subsection{Conditional Processing}
\label{sec:conditional}

The package provides a mechanism to compile different versions
of a document. To customise the versions further some conditional processing
can come in handy to distinguish which version is being compiled.
The package provides two macros to describe the compilation context:

%%%%%%%%%%%%%%%%%%%%%%%%%%%%%%%%%%%%%%%%
\DescribeMacro{\ifchilddoc}
The conditional |\ifchilddoc| distinguishes between the compilation of
child documents and the main document:
%
\begin{center}
|\ifchilddoc |\textit{child-code}| |[|\||else |\textit{main-code}]| \||fi|
\end{center}

%%%%%%%%%%%%%%%%%%%%%%%%%%%%%%%%%%%%%%%%
\DescribeMacro{\childdocname}
\DescribeMacro{\childdocjob}
The macro |\childdocname| contains the filename (without extension)
of the main or child file being processed.
Note that |\childdocjob| will always contain the name of the main file.

%%%%%%%%%%%%%%%%%%%%%%%%%%%%%%%%%%%%%%%%
\paragraph{Title Page.}

Conditional processing can be used to include a title or banner page
in the main document when proper precautions are taken.
Importantly, the code in the main file should ensure that the page counter
(as well as other status parameters which are stored in the |.aux| files)
takes the same value after the conditional processing.
Otherwise the page numbers may take divergent values
depending on which part is compiled.

For example, a title page could be declared by:
%
\begin{center}
\begin{tabular}{l}
|\ifchilddoc\||else|\\
|\addtocounter{page}{-1}|\\
\textit{code for title page}\\
|\newpage|\\
|\||fi|
\end{tabular}
\end{center}
%
A banner page for the child documents can be generated by:
%
\begin{center}
\begin{tabular}{l}
|\ifchilddoc|\\
|\addtocounter{page}{-1}|\\
\textit{code for banner page}\\
|\newpage|\\
|\||fi|
\end{tabular}
\end{center}
%
Here one could write a message such as:
\begin{center}
|This is the part \childdocname{} of \childdocjob{}.|
\end{center}

%%%%%%%%%%%%%%%%%%%%%%%%%%%%%%%%%%%%%%%%%%%%%%%%%%%%%%%%%%%%%%%%%%%%%%%%%%%%%%%%
\subsection{Flags}
\label{sec:flags}

The package makes it easy to generate different versions
of the main or child documents.
To this end compilation flags can be defined
and assigned different default values.
They will be particularly useful in conjunction
with the forwarding mechanism described in \secref{sec:forward}.

For example, it may be useful to have a flag |\version|
which can be set to |draft| or |final|.
The document source will contain some conditional code
depending on the value of |\version|.
Suppose further, the flag should default to |final| for the main file
and to |draft| for child files
which is a natural assignment for editing the document.
This is achieved by placing the following code
in the preamble of the main document
(below the |\childdocmain| directive):
%
\begin{center}
\begin{tabular}{l}
|\ifchilddoc|\\
|\providecommand{\version}{draft}|\\
|\||else|\\
|\providecommand{\version}{final}|\\
|\||fi|
\end{tabular}
\end{center}
%
The definition by |\providecommand| makes sure
that previous definitions are not overwritten.
Further statements |\providecommand{\version}{...}|
can thus be added before the above code to override it.

For the main file, one might add a line
(between |\childdocmain| and the above block)
%
\begin{center}
|%\ifchilddoc\||else\providecommand{\version}{draft}\||fi|
\end{center}
%
which can be uncommented to produce a draft version.
Likewise one can add a line to the very top of a child file
(above the |\childdocof{|\textit{main}|}| directive)
%
\begin{center}
|%\providecommand{\version}{final}|
\end{center}
%
which can be uncommented to produce the final version of this child document.

%%%%%%%%%%%%%%%%%%%%%%%%%%%%%%%%%%%%%%%%%%%%%%%%%%%%%%%%%%%%%%%%%%%%%%%%%%%%%%%%
\subsection{Forwarding}
\label{sec:forward}

Different versions of the main or child documents
using compilation flags as described in \secref{sec:flags}
can be (permanently) stored in different files
for convenient compilation, viewing and distribution.
To this end, the package defines a command
to pass on compilation to a different file:

%%%%%%%%%%%%%%%%%%%%%%%%%%%%%%%%%%%%%%%%
\DescribeMacro{\childdocforward}
The command |\childdocforward| redirects processing to
another source file:
%
\begin{center}
\begin{tabular}{l}
|\input{childdoc.def}|\\
|\childdocforward[|\textit{main}|]{|\textit{dest}|}|\\
\end{tabular}
\end{center}
%
The argument \textit{dest} is the destination file
(without extension).
It should be the main file or one of the child files.
Note that further \textsf{childdoc} directives
such as |\childdocof| and |\childdocforward|
in the indicated file will be processed in this form.
The optional argument \textit{main}
passes on directly to the main file \textit{main}
while pretending to compile the child \textit{dest}.
This form behaves as if \textit{dest}
issues |\childdocof{|\textit{main}|}| right away,
and no further \textsf{childdoc} directives will be processed.

%%%%%%%%%%%%%%%%%%%%%%%%%%%%%%%%%%%%%%%%
\DescribeMacro{\...prefix}
In the alternative form |\childdocforwardprefix|,
%
\begin{center}
\begin{tabular}{l}
|\input{childdoc.def}|\\
|\childdocforwardprefix[|\textit{main}|]{|\textit{prefix}|}{|\textit{dest}|}|
\end{tabular}
\end{center}
%
the destination file is determined by a pattern
depending on the current file:
To make this work, the current file must be called
`{\textit{prefix}\hspace{0.2em}\textit{suffix}}'
with \textit{prefix} matching precisely the argument.
Processing is then passed on to the file
`{\textit{dest}\hspace{0.2em}\textit{suffix}}'.
Surely, the same effect is achieved by
directly specifying the
argument `{\textit{dest}\hspace{0.2em}\textit{suffix}}'
in the first form.
However, that requires to set up a different file
for each child. With the alternative form of the command
all these files can have exactly the same content
which simplifies setting them up and maintaining them.

For example, the following file |draft.tex|
with a compilation flag |\version| as described in \secref{sec:flags}
compiles the main document as a draft:
%
\begin{center}
\begin{tabular}{l}
|\def\version{draft}|\\
|\input{childdoc.def}|\\
|\childdocforward{|\textit{main}|}|
\end{tabular}
\end{center}
%
Likewise, the following files |final|\textit{nn}|.tex|
compile the final version of the child document
|child|\textit{nn}|.tex|:
%
\begin{center}
\begin{tabular}{l}
|\def\version{final}|\\
|\input{childdoc.def}|\\
|\childdocforwardprefix{final}{child}|
\end{tabular}
\end{center}
%

Note that when several versions of a main file and/or of each child file
are to be generated, it may be convenient to set up a |Makefile| or
shell script to automatise the process.

%%%%%%%%%%%%%%%%%%%%%%%%%%%%%%%%%%%%%%%%%%%%%%%%%%%%%%%%%%%%%%%%%%%%%%%%%%%%%%%%
\subsection{Command Line Processing}
\label{sec:commandline}

The effect of redirection files can also be achieved by invoking
the \LaTeX{} compiler with a more elaborate command line.
Most conveniently this should be done as part
of a shell script or a |Makefile|.

When using \textsf{childdoc} in the main file, the following
command lines effectively perform a redirection
(note that depending on the shell being used,
backslashes may have to be doubled: `|\|' $\to$ `|\\|'):
%
\begin{center}
|... -jobname "|\textit{target}|" |\\|"|[\textit{flags}]%
|\input{childdoc.def}\childdocforward[|\textit{main}|]{|\textit{dest}|}"|
\end{center}
%
Here \textit{target} is the name of the output file,
\textit{main} is the name of the main file
and \textit{dest} is the name of the main or child file to be processed
(all filenames without extensions).
The optional argument \textit{main} can be omitted
if \textit{main} matches \textit{dest}.
Optionally, compilation \textit{flags} can be defined via |\def| commands.
This command line makes the \TeX{} engine believe
it is compiling the file \textit{target}
whose content is specified as the latter parameter.
The provided code then forwards the processing to
\textit{main} or \textit{dest} as described in \secref{sec:forward}.

%%%%%%%%%%%%%%%%%%%%%%%%%%%%%%%%%%%%%%%%%%%%%%%%%%%%%%%%%%%%%%%%%%%%%%%%%%%%%%%%
\subsection{Include by Input}
\label{sec:input}

Including child documents by |\include| has some restrictions by design.
Most notably, the content of a child document always occupies
its own set of pages; pages cannot be shared between child documents.
Usually, this behaviour makes perfect sense
because each child document contain an essential part of the document.
However, in some situations it may be desirable to compose
a document from a collection of parts
without having mandatory page breaks between then.
For this case, the package
provides a mechanism to include parts
by |\input| which can also be processed individually.
However, by construction this mechanism
requires manual handling of the content to be output.

%%%%%%%%%%%%%%%%%%%%%%%%%%%%%%%%%%%%%%%%
\DescribeMacro{\ifchilddocmanual}
The main file should be prepared as usual, see \secref{sec:include}.
However, the document body must make a distinction
between processing of an individual part and of the main document, e.g.:
%
\begin{center}
\begin{tabular}{l}
|\ifchilddocmanual|\\
|\input{\childdocname}|\\
|\||else|\\
\textit{document body with }|\input{|\textit{part}|}|\\
|\||fi|
\end{tabular}
\end{center}
%
The conditional |\ifchilddocmanual| is true whenever
a part to be included by |\input| is being compiled,
and the name of the part is stored in |\childdocname|.

%%%%%%%%%%%%%%%%%%%%%%%%%%%%%%%%%%%%%%%%
\DescribeMacro{\childdocby}
Each part to be included by |\input| should start with:
%
\begin{center}
\begin{tabular}{l}
|\input{childdoc.def}|\\
|\childdocby{|\textit{main}|}|\\
\end{tabular}
\end{center}
%
The directive |\childdocby| is similar to |\childdocof|
described in \secref{sec:include},
but the subsequent selection of content must be done manually.
To that end, both |\ifchilddoc| and |\ifchilddocmanual|
will be true upon processing of a part,
and the name of the part is stored in |\childdocname|.
Note that |\jobname| will be set to the filename of the current part
so that each part receives an individual |.aux| file
that does not interfere with the |.aux| file(s) of the main document.
This behaviour can be altered by the alternative form
|\childdocby[*]{|\textit{main}|}| (with a non-empty optional argument)
which uses the |.aux| file of the main document
by setting |\jobname| to \textit{main}.

%%%%%%%%%%%%%%%%%%%%%%%%%%%%%%%%%%%%%%%%%%%%%%%%%%%%%%%%%%%%%%%%%%%%%%%%%%%%%%%%
\subsection{Driver Development}
\label{sec:driver}

The \textsf{childdoc} mechanism can also be use for the development
of definition files such as \LaTeX{} styles or classes.
This case differs from the above setup with multiple parts
included by |\include| in that no |\includeonly| should be invoked.
This can be achieved by starting the include file
(before |\ProvidesPackage|) with:
%
\begin{center}
\begin{tabular}{l}
|\input{childdoc.def}|\\
|\childdocforward{|\textit{main}|}|\\
\end{tabular}
\end{center}
%
or alternatively with:
%
\begin{center}
\begin{tabular}{l}
|\input{childdoc.def}|\\
|\childdocby{|\textit{main}|}|\\
\end{tabular}
\end{center}
%
Both forms have slightly different effects as described above.
The main file is prepared as usual, see \secref{sec:include}.

%%%%%%%%%%%%%%%%%%%%%%%%%%%%%%%%%%%%%%%%%%%%%%%%%%%%%%%%%%%%%%%%%%%%%%%%%%%%%%%%
\subsection{Legacy Detection}
\label{sec:detection}

The directive |\childdocmain| in the main file can detect
whether the complete document or merely a child is to be compiled
even without using the directive |\childdocof|.
This method is deprecated because it is less robust
and there is no compelling reason to use it;
it is merely provided for backward compatibility
and it may be removed in future versions.

If the detection mechanism is to be used,
it is mandatory to correctly specify
the filename of the main file as the argument of |\childdocmain|:
%
\begin{center}
\begin{tabular}{l}
|\input{childdoc.def}|\\
|\childdocmain{|\textit{main}|}|\\
\end{tabular}
\end{center}
%
If |\jobname| does not match the argument \textit{main} of |\childdocmain|,
it is assumed that |\jobname| points to the child file to be compiled.
When using |\childdocmain| with the main file specified as argument,
it suffices to start a child file
with just |\input{|\textit{main}|}|
without loading of the package and using |\childdocof|.
If instead all processing is done
with the appropriate \textsf{childdoc} directives,
the argument of \textit{main} of |\childdocmain| can be empty.

An alternative version of the command line processing described
in \secref{sec:commandline} using the detection mechanism reads:
%
\begin{center}
|... -jobname "|\textit{target}|" "|[\textit{flags}]%
[|\def\jobname{|\textit{dest}|}|]|\input{|\textit{main}|}"|
\end{center}

%%%%%%%%%%%%%%%%%%%%%%%%%%%%%%%%%%%%%%%%%%%%%%%%%%%%%%%%%%%%%%%%%%%%%%%%%%%%%%%%
\subsection{Manual Code}
\label{sec:manual}

In case one cannot be certain whether the definitions file |childdoc.def|
is installed on the target \TeX{} distribution
and one prefers not to ship it,
it is conceivable to paste a few relevant commands into the sources.

To that end, drop all statements |\input{childdoc.def}|
and perform the replacements as outlined below.
Instead of |\childdocmain{|\textit{main}|}| add the following code
to the top of the main file:
%
\begin{center}
\begin{tabular}{l}
|\||ifdefined\childdocname\endinput\||fi\newif\ifchilddoc|\\
|\edef\childdocname{\scantokens\expandafter{\jobname\noexpand}}|\\
|\def\childdocmain{|\textit{main}|}\||ifx\childdocmain\childdocname\||else|\\
|\childdoctrue\includeonly{\childdocname}\let\jobname\childdocmain\||fi|\\
\end{tabular}
\end{center}
%
Instead of |\childdocof{|\textit{main}|}| just include the main file
at the top of each child file:
%
\begin{center}
|\input{|\textit{main}|}|
\end{center}
%
A simple redirection |\childdocforward{|\textit{dest}|}| is achieved by:
%
\begin{center}
|\def\jobname{|\textit{dest}|}\input{\jobname}|
\end{center}
%
The redirection with prefix
|\childdocforwardprefix[|\textit{prefix}|]{|\textit{dest}|}|
is accomplished by:
%
\begin{center}
\begin{tabular}{l}
|{\edef\jobname{\scantokens\expandafter{\jobname\noexpand}}|\\
|\def\redirectjob |\textit{prefix}|#1~~~{\gdef\jobname{|\textit{dest}|#1}}|\\
|\expandafter\redirectjob\jobname~~~}\input{\jobname}|
\end{tabular}
\end{center}

In an alternative approach,
child documents can be compiled by a specific command line
without additional code or specific definitions:
%
\begin{center}
|... -jobname "|\textit{target}|" "|[\textit{flags}]%
|\includeonly{|\textit{dest}|}\input{|\textit{main}|}"|
\end{center}
%

%%%%%%%%%%%%%%%%%%%%%%%%%%%%%%%%%%%%%%%%%%%%%%%%%%%%%%%%%%%%%%%%%%%%%%%%%%%%%%%%
%%%%%%%%%%%%%%%%%%%%%%%%%%%%%%%%%%%%%%%%%%%%%%%%%%%%%%%%%%%%%%%%%%%%%%%%%%%%%%%%
\section{Information}

%%%%%%%%%%%%%%%%%%%%%%%%%%%%%%%%%%%%%%%%%%%%%%%%%%%%%%%%%%%%%%%%%%%%%%%%%%%%%%%%
\subsection{Copyright}

Copyright \copyright{} 2017--2018 Niklas Beisert

This work may be distributed and/or modified under the
conditions of the \LaTeX{} Project Public License, either version 1.3
of this license or (at your option) any later version.
The latest version of this license is in
  \url{http://www.latex-project.org/lppl.txt}
and version 1.3 or later is part of all distributions of \LaTeX{}
version 2005/12/01 or later.

This work has the LPPL maintenance status `maintained'.

The Current Maintainer of this work is Niklas Beisert.

This work consists of the files |README.txt|, |childdoc.ins| and |childdoc.dtx|
as well as the derived files |childdoc.def|, |cdocsamp.tex|
with |cdocsch1.tex|, |cdocsch2.tex|, |cdocspt3.tex|, |cdocspt4.tex|,
|cdocsdrf.tex|, |cdocsfn1.tex|, |cdocsfn2.tex|
as well as |childdoc.pdf|.

%%%%%%%%%%%%%%%%%%%%%%%%%%%%%%%%%%%%%%%%%%%%%%%%%%%%%%%%%%%%%%%%%%%%%%%%%%%%%%%%
\subsection{Files and Installation}

The package consists of the files:
%
\begin{center}
\begin{tabular}{ll}
    |README.txt|   & readme file \\
    |childdoc.ins| & installation file \\
    |childdoc.dtx| & source file \\
    |childdoc.def| & definition file \\
    |cdocsamp.tex| & sample main file \\
    |cdocsch1.tex| & sample include file \\
    |cdocsch2.tex| & sample include file \\
    |cdocspt3.tex| & sample part file \\
    |cdocspt4.tex| & sample part file \\
    |cdocsdrf.tex| & sample redirection file \\
    |cdocsfn1.tex| & sample redirection file \\
    |cdocsfn2.tex| & sample redirection file \\
    |childdoc.pdf| & manual
\end{tabular}
\end{center}
%
The distribution consists of the files
|README.txt|, |childdoc.ins| and |childdoc.dtx|.
%
\begin{itemize}
\item
Run (pdf)\LaTeX{} on |childdoc.dtx|
to compile the manual |childdoc.pdf| (this file).
\item
Run \LaTeX{} on |childdoc.ins| to create the definitions file |childdoc.def|
and the sample |cdocsamp.tex| with include files
|cdocsch1.tex|, |cdocsch2.tex|, |cdocspt3.tex|, |cdocspt4.tex|,
|cdocsdrf.tex|, |cdocsfn1.tex|, |cdocsfn2.tex|.
Then copy the file |childdoc.def| to an appropriate directory of your \LaTeX{}
distribution, e.g.\ \textit{texmf-root}|/tex/latex/childdoc|.
\end{itemize}

%%%%%%%%%%%%%%%%%%%%%%%%%%%%%%%%%%%%%%%%%%%%%%%%%%%%%%%%%%%%%%%%%%%%%%%%%%%%%%%%
\subsection{Related CTAN Packages}

There are several other packages which offer a similar functionality:
%
\begin{itemize}
\item
The packages
\href{http://ctan.org/pkg/docmute}{\textsf{docmute}},
\href{http://ctan.org/pkg/includex}{\textsf{includex}} and
\href{http://ctan.org/pkg/standalone}{\textsf{standalone}}
provide commands to include only the document body of
a child file thus allowing both files to be compiled individually.
\item
The packages \href{http://ctan.org/pkg/subdocs}{\textsf{subdocs}}
and \href{http://ctan.org/pkg/subfiles}{\textsf{subfiles}}
provide structures in which the main and child documents can be
encapsulated and allowing them to be compiled individually.
The inclusion mechanism is different from the conventional |\include|.
\item
The package \href{http://ctan.org/pkg/combine}{\textsf{combine}}
is an elaborate solution to combine several documents into one.
\end{itemize}
%
See also the CTAN topic \href{http://ctan.org/topic/subdocs}{\textsf{subdocs}}
for further related packages.
The present package differs from the above solutions in that
a document structure constructed with the conventional |\include| mechanism
just needs two extra commands at the top of every file
such that all constituent files can be compiled individually.

%%%%%%%%%%%%%%%%%%%%%%%%%%%%%%%%%%%%%%%%%%%%%%%%%%%%%%%%%%%%%%%%%%%%%%%%%%%%%%%%
%\subsection{Feature Suggestions}
%
%The following is a list of features which may be useful for future
%versions of this package:
%%
%\begin{itemize}
%\item
%\ldots
%\end{itemize}

%%%%%%%%%%%%%%%%%%%%%%%%%%%%%%%%%%%%%%%%%%%%%%%%%%%%%%%%%%%%%%%%%%%%%%%%%%%%%%%%
\subsection{Revision History}

%%%%%%%%%%%%%%%%%%%%%%%%%%%%%%%%%%%%%%%%
\paragraph{v2.0:} 2018/12/30

\begin{itemize}
\item
immediate forward processing
\item
added |\childdocby| mechanism
\item
manual restructured
\end{itemize}

%%%%%%%%%%%%%%%%%%%%%%%%%%%%%%%%%%%%%%%%
\paragraph{v1.6:} 2018/01/17

\begin{itemize}
\item
application for development of include files
\item
corrections to manual
\end{itemize}

%%%%%%%%%%%%%%%%%%%%%%%%%%%%%%%%%%%%%%%%
\paragraph{v1.5:} 2017/05/21

\begin{itemize}
\item
more complete structuring introduced
\item
|\childdocof| introduced
\item
|\childdoc| renamed to |\childdocmain|
\item
|\childredirect| renamed to |\childdocforward| and |\childdocforwardprefix|
and functionality expanded
\end{itemize}

%%%%%%%%%%%%%%%%%%%%%%%%%%%%%%%%%%%%%%%%
\paragraph{v1.0:} 2017/04/27

\begin{itemize}
\item
manual and install package
\item
first version published on CTAN
\end{itemize}

%%%%%%%%%%%%%%%%%%%%%%%%%%%%%%%%%%%%%%%%
\paragraph{v0.6:} 2017/04/26

\begin{itemize}
\item
redirection mechanism added
\end{itemize}

%%%%%%%%%%%%%%%%%%%%%%%%%%%%%%%%%%%%%%%%
\paragraph{v0.5:} 2017/04/26

\begin{itemize}
\item
functionality in definition file
\end{itemize}


%%%%%%%%%%%%%%%%%%%%%%%%%%%%%%%%%%%%%%%%%%%%%%%%%%%%%%%%%%%%%%%%%%%%%%%%%%%%%%%%
%%%%%%%%%%%%%%%%%%%%%%%%%%%%%%%%%%%%%%%%%%%%%%%%%%%%%%%%%%%%%%%%%%%%%%%%%%%%%%%%
%%%%%%%%%%%%%%%%%%%%%%%%%%%%%%%%%%%%%%%%%%%%%%%%%%%%%%%%%%%%%%%%%%%%%%%%%%%%%%%%
\appendix

\settowidth\MacroIndent{\rmfamily\scriptsize 000\ }

 \DocInput{childdoc.dtx}

\end{document}
%</driver>
% \fi
%
% %%%%%%%%%%%%%%%%%%%%%%%%%%%%%%%%%%%%%%%%%%%%%%%%%%%%%%%%%%%%%%%%%%%%%%%%%%%%%%
% %%%%%%%%%%%%%%%%%%%%%%%%%%%%%%%%%%%%%%%%%%%%%%%%%%%%%%%%%%%%%%%%%%%%%%%%%%%%%%
% \section{Sample}
%\iffalse
%<*samplemain>
%\fi
%
% The following presents a sample document
% with two chapters, two parts, a title page,
% a compile flag as well as three forwarding files to set the flag.
% It consists of eight |.tex| files:
% \begin{center}
% \begin{tabular}{ll}
% |cdocsamp.tex|&main file\\
% |cdocsch1.tex|&include file for chapter 1\\
% |cdocsch2.tex|&include file for chapter 2\\
% |cdocspt3.tex|&include file for part 3\\
% |cdocspt4.tex|&include file for part 4\\
% |cdocsdrf.tex|&forwarding file for main file in draft mode\\
% |cdocsfi1.tex|&forwarding file for final version of chapter 1\\
% |cdocsfi2.tex|&forwarding file for final version of chapter 2\\
% \end{tabular}
% \end{center}
% Each of the eight files can be compiled directly by the \LaTeX{} compiler.
%
% %%%%%%%%%%%%%%%%%%%%%%%%%%%%%%%%%%%%%%
% \paragraph{Main File.}
%
% The main file is called |cdocsamp.tex|.
%
% Load the \textsf{childdoc} definitions and
% declare the filename for the main document:
%    \begin{macrocode}
\input{childdoc.def}
\childdocmain{}
%    \end{macrocode}

% Optional override for |\version| flag:
%    \begin{macrocode}
%%\ifchilddoc\else\providecommand{\version}{draft}\fi
%    \end{macrocode}

% Define the default values for the |\version| flag
% (|final| for the main file and |draft| for childs):
%    \begin{macrocode}
\ifchilddoc
\providecommand{\version}{draft}
\else
\providecommand{\version}{final}
\fi
%    \end{macrocode}

% Load the standard document class:
%    \begin{macrocode}
\documentclass[12pt]{article}
%    \end{macrocode}

% Start the document body:
%    \begin{macrocode}
\begin{document}
%    \end{macrocode}

% Declare a title page.
% Print title, part of document being processed and version flag:
%    \begin{macrocode}
\addtocounter{page}{-1}
\begin{center}
{\LARGE\bfseries{}childdoc example\par}
\vspace{1cm}
\ifchilddoc
\ifchilddocmanual part\else chapter\fi:
`\childdocname' of `\childdocjob'\par
\else
main document: `\childdocjob'\par
\fi
version: \version\par
\end{center}
\newpage
%    \end{macrocode}

% Manually include selected file,
% otherwise process as usual:
%    \begin{macrocode}
\ifchilddocmanual
\section*{part `\childdocname'}
\input{\childdocname}
\else
%    \end{macrocode}

% Include the two chapters:
%    \begin{macrocode}
\include{cdocsch1}
\include{cdocsch2}
%    \end{macrocode}

% Include the two parts unless only chapters should be displayed:
%    \begin{macrocode}
\ifchilddoc\else
\section{part three}
\input{cdocspt3}
\section{part four}
\input{cdocspt4}
\fi
%    \end{macrocode}

% Process as usual until here:
%    \begin{macrocode}
\fi
%    \end{macrocode}

% End of document body:
%    \begin{macrocode}
\end{document}
%    \end{macrocode}
%\iffalse
%</samplemain>
%\fi
%
% %%%%%%%%%%%%%%%%%%%%%%%%%%%%%%%%%%%%%%
% \paragraph{Chapter Include Files.}
%
% The include files are called |cdocsch1.tex| and |cdocsch2.tex|.
%
%\iffalse
%<*samplechap1|samplechap2>
%\fi

% Optional override for |\version| flag:
%    \begin{macrocode}
%%\providecommand{\version}{final}
%    \end{macrocode}

% Include the main document:
%    \begin{macrocode}
\input{childdoc.def}
\childdocof{cdocsamp}
%    \end{macrocode}

%\iffalse
%</samplechap1|samplechap2>
%\fi
%
%\iffalse
%<*samplechap1>
%\fi
% Some text for chapter 1:
%    \begin{macrocode}
\section{one}
some text in chapter one
%    \end{macrocode}

%\iffalse
%</samplechap1>
%\fi
% Some text for chapter 2:
%\iffalse
%<*samplechap2>
%\fi
%    \begin{macrocode}
\section{two}
more text in chapter two
%    \end{macrocode}

%\iffalse
%</samplechap2>
%\fi
%
% %%%%%%%%%%%%%%%%%%%%%%%%%%%%%%%%%%%%%%
% \paragraph{Part Include Files.}
%
% The include files are called |cdocspt3.tex| and |cdocspt4.tex|.
%
%\iffalse
%<*samplepart3|samplepart4>
%\fi

% Optional override for |\version| flag:
%    \begin{macrocode}
%%\providecommand{\version}{final}
%    \end{macrocode}

% Include the main document:
%    \begin{macrocode}
\input{childdoc.def}
\childdocby{cdocsamp}
%    \end{macrocode}

%\iffalse
%</samplepart3|samplepart4>
%\fi
%
%\iffalse
%<*samplepart3>
%\fi
% Some text for part 3:
%    \begin{macrocode}
some text in part three
%    \end{macrocode}

%\iffalse
%</samplepart3>
%\fi
% Some text for part 4:
%\iffalse
%<*samplepart4>
%\fi
%    \begin{macrocode}
more text in part four
%    \end{macrocode}

%\iffalse
%</samplepart4>
%\fi
%
% %%%%%%%%%%%%%%%%%%%%%%%%%%%%%%%%%%%%%%
% \paragraph{Forwarding for a Complete Draft.}
%
% The following forwarding file |cdocsdrf.tex|
% compiles the main document in draft mode:
%\iffalse
%<*sampledraft>
%\fi
%    \begin{macrocode}
\def\version{draft}
\input{childdoc.def}
\childdocforward{cdocsamp}
%    \end{macrocode}

%\iffalse
%</sampledraft>
%\fi
%
% %%%%%%%%%%%%%%%%%%%%%%%%%%%%%%%%%%%%%%
% \paragraph{Forwarding for Final Version of the Chapters.}
%
% The following forwarding files |cdocsfn1.tex| and |cdocsfn2.tex|
% (with identical content)
% compile the final versions of the child documents
% |cdocsch1.tex| and |cdocsch2.tex|, respectively:
%\iffalse
%<*samplefinal>
%\fi
%    \begin{macrocode}
\def\version{final}
\input{childdoc.def}
\childdocforwardprefix[cdocsamp]{cdocsfn}{cdocsch}
%    \end{macrocode}

%\iffalse
%</samplefinal>
%\fi
%
% %%%%%%%%%%%%%%%%%%%%%%%%%%%%%%%%%%%%%%
% \paragraph{Command Line Processing.}
%
% The following three command lines generate the output files
% |cdocscld|, |cdocscl1| and |cdocscl2|
% which should be identical to
% |cdocsdrf|, |cdocsch1| and |cdocsfn2|, respectively:
% \begin{center}
% \begin{tabular}{l}
% |latex -jobname cdocscld \|\\
% |  "\def\version{draft}\input{childdoc.def}\childdocforward{cdocsamp}"|\\
% |latex -jobname cdocscl1 \|\\
% |  "\input{childdoc.def}\childdocforward[cdocsamp]{cdocsch1}"|\\
% |latex -jobname cdocscl2 \|\\
% |  "\def\version{final}\input{childdoc.def}\childdocforward{cdocsch2}"|
% \end{tabular}
% \end{center}
% Note that the trailing backslash on each first line
% merely continues the input to the second line
% (for convenient cut ant paste).
% Furthermore, the command |latex| can be replaced by any
% of its alternative versions such as |pdflatex|.
%
% %%%%%%%%%%%%%%%%%%%%%%%%%%%%%%%%%%%%%%%%%%%%%%%%%%%%%%%%%%%%%%%%%%%%%%%%%%%%%%
% %%%%%%%%%%%%%%%%%%%%%%%%%%%%%%%%%%%%%%%%%%%%%%%%%%%%%%%%%%%%%%%%%%%%%%%%%%%%%%
% \section{Implementation}
%\iffalse
%<*package>
%\fi
%
% This section describes the definitions file |childdoc.def|.

% The definitions cannot be loaded using |\usepackage| or |\RequirePackage|
% which has a mechanism to prevent loading a style file more than once.
% When loading the definitions by means of |\input|
% multiple instances have to be prevented manually:
%\iffalse
%This code needs to be before the `\ProvidesFile' directive
%which is defined at the beginning of this file.
%Therefore it is also placed there and commented out here.
%</package>
%<*discard>
%\fi
%    \begin{macrocode}
\ifdefined\childdocmain\endinput\fi
%    \end{macrocode}
%\iffalse
%</discard>
%<*package>
%\fi
%
% \macro{\ifchilddoc}
% \macro{\ifchilddocmanual}
% The conditional |\ifchilddoc| tells whether a
% child (true) or main (false) document is being compiled.
% The conditional |\ifchilddocmanual| tells whether
% the |\includeonly| mechanism is used (false) or
% the selection of child files must be performed manually (true).
% The definitions initialise to false:
%    \begin{macrocode}
\newif\ifchilddoc
\newif\ifchilddocmanual
%    \end{macrocode}

% \macro{\childdocname}
% \macro{\childdocjob}
% The macro |\childdocname| stores the name of the main document
% to be compiled. The macro |\childdocjob| stores the name of
% the document on which the \LaTeX{} compiler was originally invoked.
% The content of |\jobname| cannot be compared
% to filenames specified in the source due to different catcodes.
% The following code rescans |\jobname|, stores the result
% in |\childdocname| and saves a copy in |\childdocjob|:
%    \begin{macrocode}
\edef\childdocname{\scantokens\expandafter{\jobname\noexpand}}
\let\childdocjob\childdocname
%    \end{macrocode}

% \macro{\childdocdisable}
% The macro |\childdocdisable| prevents the main file
% from being processed more than once.
% At this stage, the main document command |\childdocmain|
% is assumed to be called once again where it should do nothing.
% Any subsequent call to it should prevent
% a secondary processing of the main document
% It overwrites the forwarding commands
% |\childdocof| and |\childdocforward|
% with empty macros to prevent further inclusions of the main document:
%    \begin{macrocode}
\newcommand{\childdocdisable}
{
  \renewcommand{\childdocmain}[1]{\renewcommand{\childdocmain}[1]{\endinput}}
  \renewcommand{\childdocof}[1]{}
  \renewcommand{\childdocby}[2][]{}
  \renewcommand{\childdocforward}[2][]{}
  \renewcommand{\childdocdisable}{}
}
%    \end{macrocode}

% \macro{\childdocmain}
% The macro |\childdocmain| is to be called at the top of the main file
% with nothing or the main filename (without extension) as argument.
% First, it breaks loops.
% If the argument is not empty and does not match |\childdocname|
% (which is set by the first inclusion of |childdoc.def|),
% |\ifchilddoc| is set to true, |\includeonly| is applied to the child file
% and |\jobname| is set to the main file
% (for proper handling of |.aux| files):
%    \begin{macrocode}
\newcommand{\childdocmain}[1]
{
  \childdocdisable\childdocmain{}
  \if?#1?\else
    \begingroup
      \def\childdoctmp{#1}
      \ifx\childdoctmp\childdocname
        \def\childdoctmp{}
      \else
        \def\childdoctmp
        {
          \childdoctrue
          \includeonly{\childdocname}
          \def\childdocjob{#1}
          \def\jobname{#1}
        }
      \fi
      \expandafter
    \endgroup
    \childdoctmp
  \fi
}
%    \end{macrocode}

% \macro{\childdocof}
% The command |\childdocof| redirects
% compilation to the main file |#1|.
%    \begin{macrocode}
\newcommand{\childdocof}[1]
{
  \childdocdisable
  \childdoctrue
  \includeonly{\childdocname}
  \def\jobname{#1}
  \def\childdocjob{#1}
  \input{#1}
}
%    \end{macrocode}

% \macro{\childdocby}
% The command |\childdocby| ....
%    \begin{macrocode}
\newcommand{\childdocby}[2][]
{
  \childdocdisable
  \childdoctrue
  \childdocmanualtrue
  \if?#1?\else
    \def\jobname{#2}
  \fi
  \def\childdocjob{#2}
  \input{#2}
  \endinput
}
%    \end{macrocode}

% \macro{\childdocforward}
% The command |\childdocforward| redirects
% compilation to the main file or
% (if the optional argument is given) a child file.
% Parameters are set as if the main file
% or a child file starting with |\childdocof| was compiled.
% Then compilation is handed over to the main file:
%    \begin{macrocode}
\newcommand{\childdocforward}[2][]
{
  \begingroup
    \if?#1?
      \def\childdoctmp
      {
        \def\childdocname{#2}
        \def\childdocjob{#2}
        \def\jobname{#2}
        \input{#2}
        \endinput
      }
    \else
      \def\childdoctmp
      {
        \childdocdisable
        \def\childdocname{#2}
        \childdoctrue
        \includeonly{#2}
        \def\childdocjob{#1}
        \def\jobname{#1}
        \input{#1}
        \endinput
      }
    \fi
    \expandafter
  \endgroup
  \childdoctmp
}
%    \end{macrocode}

% \macro{\childdocforwardprefix}
% The command |\childdocforwardprefix| redirects
% compilation to the main or a child file by means of a pattern.
% The prefix |#1| in the current filename is replaced by |#2|
% and the suffix of the current filename is kept
% (it is assumed that the filename does not contain the substring `|~~~|'
% which is used as a delimiter).
% Compilation is handed over to the new file by |\childdocforward|:
%    \begin{macrocode}
\newcommand{\childdocforwardprefix}[3][]
{
  \begingroup
    \def\childdocextract #2##1~~~{\def\childdoctmp{\childdocforward[#1]{#3##1}}}
    \expandafter\childdocextract\childdocname~~~
    \expandafter
  \endgroup
  \childdoctmp
}
%    \end{macrocode}

% \macro{\childdoc}
% The deprecated macro |\childdoc| is a legacy version of |\childdocmain|:
%    \begin{macrocode}
\newcommand{\childdoc}{\childdocmain}
%    \end{macrocode}

% \macro{\childdocredirect}
% The deprecated macro |\childdocredirect| is a legacy version
% of |\childdocforward| and |\childdocforwardprefix|:
%    \begin{macrocode}
\newcommand{\childdocredirect}[2][]
{
  \begingroup
    \if?#1?
      \def\childdoctmp{\childdocforward{#2}}
    \else
      \def\childdoctmp{\childdocforwardprefix{#1}{#2}}
    \fi
    \expandafter
  \endgroup
  \childdoctmp
}
%    \end{macrocode}

%\iffalse
%</package>
%\fi
%
\endinput
|\\
|\childdocforward{|\textit{main}|}|
\end{tabular}
\end{center}
%
Likewise, the following files |final|\textit{nn}|.tex|
compile the final version of the child document
|child|\textit{nn}|.tex|:
%
\begin{center}
\begin{tabular}{l}
|\def\version{final}|\\
|% \iffalse
%
% childdoc.dtx Copyright (C) 2017-2018 Niklas Beisert
%
% This work may be distributed and/or modified under the
% conditions of the LaTeX Project Public License, either version 1.3
% of this license or (at your option) any later version.
% The latest version of this license is in
%   http://www.latex-project.org/lppl.txt
% and version 1.3 or later is part of all distributions of LaTeX
% version 2005/12/01 or later.
%
% This work has the LPPL maintenance status `maintained'.
%
% The Current Maintainer of this work is Niklas Beisert.
%
% This work consists of the files childdoc.dtx and childdoc.ins
% and the derived files childdoc.def and cdocsamp.tex with
% cdocsch1.tex, cdocsch2.tex, cdocsdrf.tex, cdocsfn1.tex, cdocsfn2.tex.
%
%<package>\ifdefined\childdocmain\endinput\fi
%<package>\ProvidesFile{childdoc.def}[2018/12/30 v2.0 child document driver]
%<samplemain>\ProvidesFile{cdocsamp.tex}[2018/12/30 v2.0 sample for childdoc]
%<*driver>
%\ProvidesFile{childdoc.drv}[2018/12/30 v2.0 childdoc reference manual file]
\PassOptionsToClass{10pt,a4paper}{article}
\documentclass{ltxdoc}

\usepackage[margin=35mm]{geometry}
\usepackage{hyperref}
\usepackage{hyperxmp}
\usepackage[usenames]{color}

\hypersetup{colorlinks=true}
\hypersetup{pdfstartview=FitH}
\hypersetup{pdfpagemode=UseNone}
\hypersetup{pdfsource={}}
\hypersetup{pdflang={en-UK}}
\hypersetup{pdfcopyright={Copyright 2017-2018 Niklas Beisert.
  This work may be distributed and/or modified under the
  conditions of the LaTeX Project Public License, either version 1.3
  of this license or (at your option) any later version.}}
\hypersetup{pdflicenseurl={http://www.latex-project.org/lppl.txt}}
\hypersetup{pdfcontactaddress={ETH Zurich, ITP, HIT K,
  Wolfgang-Pauli-Strasse 27}}
\hypersetup{pdfcontactpostcode={8093}}
\hypersetup{pdfcontactcity={Zurich}}
\hypersetup{pdfcontactcountry={Switzerland}}
\hypersetup{pdfcontactemail={nbeisert@itp.phys.ethz.ch}}
\hypersetup{pdfcontacturl={http://people.phys.ethz.ch/\xmptilde nbeisert/}}

\newcommand{\secref}[1]{\hyperref[#1]{section \ref*{#1}}}

\parskip1ex
\parindent0pt
\let\olditemize\itemize
\def\itemize{\olditemize\parskip0pt}

\begin{document}

\title{The \textsf{childdoc} Package}
\hypersetup{pdftitle={The childdoc Package}}
\author{Niklas Beisert\\[2ex]
  Institut f\"ur Theoretische Physik\\
  Eidgen\"ossische Technische Hochschule Z\"urich\\
  Wolfgang-Pauli-Strasse 27, 8093 Z\"urich, Switzerland\\[1ex]
  \href{mailto:nbeisert@itp.phys.ethz.ch}
  {\texttt{nbeisert@itp.phys.ethz.ch}}}
\hypersetup{pdfauthor={Niklas Beisert}}
\hypersetup{pdfsubject={Manual for the LaTeX2e Package childdoc}}
\date{30 December 2018, \textsf{v2.0}}
\maketitle

\begin{abstract}\noindent
\textsf{childdoc} is a \LaTeXe{} package
that enables the direct compilation
of document sections included by |\include|
to individual files.
\end{abstract}

\begingroup
\parskip0ex
\tableofcontents
\endgroup

%%%%%%%%%%%%%%%%%%%%%%%%%%%%%%%%%%%%%%%%%%%%%%%%%%%%%%%%%%%%%%%%%%%%%%%%%%%%%%%%
%%%%%%%%%%%%%%%%%%%%%%%%%%%%%%%%%%%%%%%%%%%%%%%%%%%%%%%%%%%%%%%%%%%%%%%%%%%%%%%%
\section{Introduction}

\LaTeX{} provides a mechanism to structure a large document (such as a book)
into a main file and several child files (containing the chapters)
using the |\include| command.
This mechanism is beneficial for documents
which span hundreds of pages in order to
make the source file(s) more manageable.
Moreover, compilation can be restricted to
selected child files by means of the |\includeonly| command.
The latter feature can be used to reduce the compilation time while editing
(this was significantly more useful in the earlier days of \LaTeX{})
or to generate a smaller document which is easier to navigate.
Another application of |\includeonly| is to generate
documents consisting of selected parts of the complete document.

However, there are a few drawbacks of the plain |\include| mechanism:
\begin{itemize}
\item
The child files cannot be compiled on their own,
they can only be compiled via the main file.
A naive editing environment
(such as a text editor with an option
to have the current file processed by \LaTeX)
may require one to switch to the main file before compiling;
attempting to compile the child file produces errors.
\item
The main file must be modified (each time)
to adjust the |\includeonly| command
to the present needs. This easily leaves the main file in a messy state.
\item
The generated document will always carry the filename
of the main document. This is inconvenient if
several child files are to be compiled and
to be kept for distribution.
\end{itemize}

The present package provides a simple interface
to make child files individually compilable by \LaTeX{}.
Compiling a child file then has the same effect as compiling
the main file with an |\includeonly| command
to select the appropriate child.
Moreover the generated document will carry the name of the child
rather than the main file.
This resolves all three above issues.

This feature is meant to make the editing of books,
thesis documents and lecture notes somewhat more convenient.
However, the package can also be used efficiently for
composing a series of documents (such as exercise sheets)
which are typically distributed individually.
It then assists the author in generating the individual documents
(potentially in different versions)
as well as a document containing the collected series.
Another application is in developing style files
or other kinds of included material
where compilation of the style file could redirect
to a sample or test file.

%%%%%%%%%%%%%%%%%%%%%%%%%%%%%%%%%%%%%%%%%%%%%%%%%%%%%%%%%%%%%%%%%%%%%%%%%%%%%%%%
%%%%%%%%%%%%%%%%%%%%%%%%%%%%%%%%%%%%%%%%%%%%%%%%%%%%%%%%%%%%%%%%%%%%%%%%%%%%%%%%
\section{Usage}

First of all, the package \textsf{childdoc} is \emph{not} a standard
\LaTeXe{} |.sty| style file! Therefore it needs to be invoked in
a non-standard way.

%%%%%%%%%%%%%%%%%%%%%%%%%%%%%%%%%%%%%%%%%%%%%%%%%%%%%%%%%%%%%%%%%%%%%%%%%%%%%%%%
\subsection{Included Files}
\label{sec:include}

%%%%%%%%%%%%%%%%%%%%%%%%%%%%%%%%%%%%%%%%
\DescribeMacro{\childdocmain}
To use the package, add the commands
\begin{center}
\begin{tabular}{l}
|\input{childdoc.def}|\\
|\childdocmain{}|\\
\end{tabular}
\end{center}
at the very top of the main \LaTeX{} file,
in particular \emph{before} the |\documentclass| statement!
The argument of |\childdocmain| should be left empty
(but it must be present).

%%%%%%%%%%%%%%%%%%%%%%%%%%%%%%%%%%%%%%%%
\DescribeMacro{\childdocof}
Furthermore, add the commands
\begin{center}
\begin{tabular}{l}
|\input{childdoc.def}|\\
|\childdocof{|\textit{main}|}|\\
\end{tabular}
\end{center}
at the top of every child file \textit{child}
which is included by |\include{|\textit{child}|}|
from within the main file
(or at least for those files to be compiled individually).
The argument \textit{main} must be the filename of the main file.

There are a couple of
considerations in setting up the main and child documents:

%%%%%%%%%%%%%%%%%%%%%%%%%%%%%%%%%%%%%%%%
\paragraph{Restrictions.}

Please note the following restrictions:
\begin{itemize}
\item
|\childdocmain| must be called with one argument \textit{main}
to ensure compatibility with earlier version of the package.
It must either be empty (|\childdocmain{}|)
or precisely match the filename of the main file in which it is specified.
See \secref{sec:detection} for further information.
\item
The filename \textit{main} must be specified without the |.tex| extension.
\item
The filename \textit{main} is case sensitive
(even in case-insensitive file systems)
due to internal string comparison.
\item
The argument \textit{main} should be fully expanded, it cannot be a macro.
\item
Subdirectories and special characters should be avoided in filenames.
\item
The command |\childdocmain{|\textit{main}|}| must be followed by a whitespace.
It should not be followed immediately by another command
or by a comment mark `|%|'.
This is because the \TeX{} parser reads the token immediately following
the argument of |\childdocmain| and puts it
at the beginning of every child section;
however, a white\-space is ignored.
\end{itemize}

%%%%%%%%%%%%%%%%%%%%%%%%%%%%%%%%%%%%%%%%
\paragraph{Content of Main File.}

It is advisable to place all content in the child files included by |\include|.
Any output contained in the main file will appear in all child documents
unless suppressed manually;
it cannot be suppressed automatically by the |\includeonly| directive
and thus should normally be avoided.
A method to include some content in the main file
by means of conditional processing is described in \secref{sec:conditional}.

%%%%%%%%%%%%%%%%%%%%%%%%%%%%%%%%%%%%%%%%
\paragraph{Page Numbering.}

When only a part of the document is compiled,
the appropriate numbering of pages
(as well as other status parameters)
is determined from the |.aux| files.
The latter contain information from previous passes.
However this information needs to propagate through
all intermediate child documents.
Therefore the page numbering in child documents may well
be inconsistent until the complete document is compiled at least once.

A useful (if unconventional) way to always ensure a consistent
page numbering is to restart the numbering in each child document
and denote the pages by `\textit{child}|.|\textit{page}'
where \textit{child} represents the chapter/section number of the child file.
This can be achieved by the command
|\numberwithin{page}{|\textit{child}|}|
of the \textsf{amsmath} package
where \textit{child} can be |chapter| or |section|
depending on the chosen structuring.
Alternatively, one can modify the macro |\thepage| appropriately
and reset the counter |page| at the start of each child file.

%%%%%%%%%%%%%%%%%%%%%%%%%%%%%%%%%%%%%%%%%%%%%%%%%%%%%%%%%%%%%%%%%%%%%%%%%%%%%%%%
\subsection{Conditional Processing}
\label{sec:conditional}

The package provides a mechanism to compile different versions
of a document. To customise the versions further some conditional processing
can come in handy to distinguish which version is being compiled.
The package provides two macros to describe the compilation context:

%%%%%%%%%%%%%%%%%%%%%%%%%%%%%%%%%%%%%%%%
\DescribeMacro{\ifchilddoc}
The conditional |\ifchilddoc| distinguishes between the compilation of
child documents and the main document:
%
\begin{center}
|\ifchilddoc |\textit{child-code}| |[|\||else |\textit{main-code}]| \||fi|
\end{center}

%%%%%%%%%%%%%%%%%%%%%%%%%%%%%%%%%%%%%%%%
\DescribeMacro{\childdocname}
\DescribeMacro{\childdocjob}
The macro |\childdocname| contains the filename (without extension)
of the main or child file being processed.
Note that |\childdocjob| will always contain the name of the main file.

%%%%%%%%%%%%%%%%%%%%%%%%%%%%%%%%%%%%%%%%
\paragraph{Title Page.}

Conditional processing can be used to include a title or banner page
in the main document when proper precautions are taken.
Importantly, the code in the main file should ensure that the page counter
(as well as other status parameters which are stored in the |.aux| files)
takes the same value after the conditional processing.
Otherwise the page numbers may take divergent values
depending on which part is compiled.

For example, a title page could be declared by:
%
\begin{center}
\begin{tabular}{l}
|\ifchilddoc\||else|\\
|\addtocounter{page}{-1}|\\
\textit{code for title page}\\
|\newpage|\\
|\||fi|
\end{tabular}
\end{center}
%
A banner page for the child documents can be generated by:
%
\begin{center}
\begin{tabular}{l}
|\ifchilddoc|\\
|\addtocounter{page}{-1}|\\
\textit{code for banner page}\\
|\newpage|\\
|\||fi|
\end{tabular}
\end{center}
%
Here one could write a message such as:
\begin{center}
|This is the part \childdocname{} of \childdocjob{}.|
\end{center}

%%%%%%%%%%%%%%%%%%%%%%%%%%%%%%%%%%%%%%%%%%%%%%%%%%%%%%%%%%%%%%%%%%%%%%%%%%%%%%%%
\subsection{Flags}
\label{sec:flags}

The package makes it easy to generate different versions
of the main or child documents.
To this end compilation flags can be defined
and assigned different default values.
They will be particularly useful in conjunction
with the forwarding mechanism described in \secref{sec:forward}.

For example, it may be useful to have a flag |\version|
which can be set to |draft| or |final|.
The document source will contain some conditional code
depending on the value of |\version|.
Suppose further, the flag should default to |final| for the main file
and to |draft| for child files
which is a natural assignment for editing the document.
This is achieved by placing the following code
in the preamble of the main document
(below the |\childdocmain| directive):
%
\begin{center}
\begin{tabular}{l}
|\ifchilddoc|\\
|\providecommand{\version}{draft}|\\
|\||else|\\
|\providecommand{\version}{final}|\\
|\||fi|
\end{tabular}
\end{center}
%
The definition by |\providecommand| makes sure
that previous definitions are not overwritten.
Further statements |\providecommand{\version}{...}|
can thus be added before the above code to override it.

For the main file, one might add a line
(between |\childdocmain| and the above block)
%
\begin{center}
|%\ifchilddoc\||else\providecommand{\version}{draft}\||fi|
\end{center}
%
which can be uncommented to produce a draft version.
Likewise one can add a line to the very top of a child file
(above the |\childdocof{|\textit{main}|}| directive)
%
\begin{center}
|%\providecommand{\version}{final}|
\end{center}
%
which can be uncommented to produce the final version of this child document.

%%%%%%%%%%%%%%%%%%%%%%%%%%%%%%%%%%%%%%%%%%%%%%%%%%%%%%%%%%%%%%%%%%%%%%%%%%%%%%%%
\subsection{Forwarding}
\label{sec:forward}

Different versions of the main or child documents
using compilation flags as described in \secref{sec:flags}
can be (permanently) stored in different files
for convenient compilation, viewing and distribution.
To this end, the package defines a command
to pass on compilation to a different file:

%%%%%%%%%%%%%%%%%%%%%%%%%%%%%%%%%%%%%%%%
\DescribeMacro{\childdocforward}
The command |\childdocforward| redirects processing to
another source file:
%
\begin{center}
\begin{tabular}{l}
|\input{childdoc.def}|\\
|\childdocforward[|\textit{main}|]{|\textit{dest}|}|\\
\end{tabular}
\end{center}
%
The argument \textit{dest} is the destination file
(without extension).
It should be the main file or one of the child files.
Note that further \textsf{childdoc} directives
such as |\childdocof| and |\childdocforward|
in the indicated file will be processed in this form.
The optional argument \textit{main}
passes on directly to the main file \textit{main}
while pretending to compile the child \textit{dest}.
This form behaves as if \textit{dest}
issues |\childdocof{|\textit{main}|}| right away,
and no further \textsf{childdoc} directives will be processed.

%%%%%%%%%%%%%%%%%%%%%%%%%%%%%%%%%%%%%%%%
\DescribeMacro{\...prefix}
In the alternative form |\childdocforwardprefix|,
%
\begin{center}
\begin{tabular}{l}
|\input{childdoc.def}|\\
|\childdocforwardprefix[|\textit{main}|]{|\textit{prefix}|}{|\textit{dest}|}|
\end{tabular}
\end{center}
%
the destination file is determined by a pattern
depending on the current file:
To make this work, the current file must be called
`{\textit{prefix}\hspace{0.2em}\textit{suffix}}'
with \textit{prefix} matching precisely the argument.
Processing is then passed on to the file
`{\textit{dest}\hspace{0.2em}\textit{suffix}}'.
Surely, the same effect is achieved by
directly specifying the
argument `{\textit{dest}\hspace{0.2em}\textit{suffix}}'
in the first form.
However, that requires to set up a different file
for each child. With the alternative form of the command
all these files can have exactly the same content
which simplifies setting them up and maintaining them.

For example, the following file |draft.tex|
with a compilation flag |\version| as described in \secref{sec:flags}
compiles the main document as a draft:
%
\begin{center}
\begin{tabular}{l}
|\def\version{draft}|\\
|\input{childdoc.def}|\\
|\childdocforward{|\textit{main}|}|
\end{tabular}
\end{center}
%
Likewise, the following files |final|\textit{nn}|.tex|
compile the final version of the child document
|child|\textit{nn}|.tex|:
%
\begin{center}
\begin{tabular}{l}
|\def\version{final}|\\
|\input{childdoc.def}|\\
|\childdocforwardprefix{final}{child}|
\end{tabular}
\end{center}
%

Note that when several versions of a main file and/or of each child file
are to be generated, it may be convenient to set up a |Makefile| or
shell script to automatise the process.

%%%%%%%%%%%%%%%%%%%%%%%%%%%%%%%%%%%%%%%%%%%%%%%%%%%%%%%%%%%%%%%%%%%%%%%%%%%%%%%%
\subsection{Command Line Processing}
\label{sec:commandline}

The effect of redirection files can also be achieved by invoking
the \LaTeX{} compiler with a more elaborate command line.
Most conveniently this should be done as part
of a shell script or a |Makefile|.

When using \textsf{childdoc} in the main file, the following
command lines effectively perform a redirection
(note that depending on the shell being used,
backslashes may have to be doubled: `|\|' $\to$ `|\\|'):
%
\begin{center}
|... -jobname "|\textit{target}|" |\\|"|[\textit{flags}]%
|\input{childdoc.def}\childdocforward[|\textit{main}|]{|\textit{dest}|}"|
\end{center}
%
Here \textit{target} is the name of the output file,
\textit{main} is the name of the main file
and \textit{dest} is the name of the main or child file to be processed
(all filenames without extensions).
The optional argument \textit{main} can be omitted
if \textit{main} matches \textit{dest}.
Optionally, compilation \textit{flags} can be defined via |\def| commands.
This command line makes the \TeX{} engine believe
it is compiling the file \textit{target}
whose content is specified as the latter parameter.
The provided code then forwards the processing to
\textit{main} or \textit{dest} as described in \secref{sec:forward}.

%%%%%%%%%%%%%%%%%%%%%%%%%%%%%%%%%%%%%%%%%%%%%%%%%%%%%%%%%%%%%%%%%%%%%%%%%%%%%%%%
\subsection{Include by Input}
\label{sec:input}

Including child documents by |\include| has some restrictions by design.
Most notably, the content of a child document always occupies
its own set of pages; pages cannot be shared between child documents.
Usually, this behaviour makes perfect sense
because each child document contain an essential part of the document.
However, in some situations it may be desirable to compose
a document from a collection of parts
without having mandatory page breaks between then.
For this case, the package
provides a mechanism to include parts
by |\input| which can also be processed individually.
However, by construction this mechanism
requires manual handling of the content to be output.

%%%%%%%%%%%%%%%%%%%%%%%%%%%%%%%%%%%%%%%%
\DescribeMacro{\ifchilddocmanual}
The main file should be prepared as usual, see \secref{sec:include}.
However, the document body must make a distinction
between processing of an individual part and of the main document, e.g.:
%
\begin{center}
\begin{tabular}{l}
|\ifchilddocmanual|\\
|\input{\childdocname}|\\
|\||else|\\
\textit{document body with }|\input{|\textit{part}|}|\\
|\||fi|
\end{tabular}
\end{center}
%
The conditional |\ifchilddocmanual| is true whenever
a part to be included by |\input| is being compiled,
and the name of the part is stored in |\childdocname|.

%%%%%%%%%%%%%%%%%%%%%%%%%%%%%%%%%%%%%%%%
\DescribeMacro{\childdocby}
Each part to be included by |\input| should start with:
%
\begin{center}
\begin{tabular}{l}
|\input{childdoc.def}|\\
|\childdocby{|\textit{main}|}|\\
\end{tabular}
\end{center}
%
The directive |\childdocby| is similar to |\childdocof|
described in \secref{sec:include},
but the subsequent selection of content must be done manually.
To that end, both |\ifchilddoc| and |\ifchilddocmanual|
will be true upon processing of a part,
and the name of the part is stored in |\childdocname|.
Note that |\jobname| will be set to the filename of the current part
so that each part receives an individual |.aux| file
that does not interfere with the |.aux| file(s) of the main document.
This behaviour can be altered by the alternative form
|\childdocby[*]{|\textit{main}|}| (with a non-empty optional argument)
which uses the |.aux| file of the main document
by setting |\jobname| to \textit{main}.

%%%%%%%%%%%%%%%%%%%%%%%%%%%%%%%%%%%%%%%%%%%%%%%%%%%%%%%%%%%%%%%%%%%%%%%%%%%%%%%%
\subsection{Driver Development}
\label{sec:driver}

The \textsf{childdoc} mechanism can also be use for the development
of definition files such as \LaTeX{} styles or classes.
This case differs from the above setup with multiple parts
included by |\include| in that no |\includeonly| should be invoked.
This can be achieved by starting the include file
(before |\ProvidesPackage|) with:
%
\begin{center}
\begin{tabular}{l}
|\input{childdoc.def}|\\
|\childdocforward{|\textit{main}|}|\\
\end{tabular}
\end{center}
%
or alternatively with:
%
\begin{center}
\begin{tabular}{l}
|\input{childdoc.def}|\\
|\childdocby{|\textit{main}|}|\\
\end{tabular}
\end{center}
%
Both forms have slightly different effects as described above.
The main file is prepared as usual, see \secref{sec:include}.

%%%%%%%%%%%%%%%%%%%%%%%%%%%%%%%%%%%%%%%%%%%%%%%%%%%%%%%%%%%%%%%%%%%%%%%%%%%%%%%%
\subsection{Legacy Detection}
\label{sec:detection}

The directive |\childdocmain| in the main file can detect
whether the complete document or merely a child is to be compiled
even without using the directive |\childdocof|.
This method is deprecated because it is less robust
and there is no compelling reason to use it;
it is merely provided for backward compatibility
and it may be removed in future versions.

If the detection mechanism is to be used,
it is mandatory to correctly specify
the filename of the main file as the argument of |\childdocmain|:
%
\begin{center}
\begin{tabular}{l}
|\input{childdoc.def}|\\
|\childdocmain{|\textit{main}|}|\\
\end{tabular}
\end{center}
%
If |\jobname| does not match the argument \textit{main} of |\childdocmain|,
it is assumed that |\jobname| points to the child file to be compiled.
When using |\childdocmain| with the main file specified as argument,
it suffices to start a child file
with just |\input{|\textit{main}|}|
without loading of the package and using |\childdocof|.
If instead all processing is done
with the appropriate \textsf{childdoc} directives,
the argument of \textit{main} of |\childdocmain| can be empty.

An alternative version of the command line processing described
in \secref{sec:commandline} using the detection mechanism reads:
%
\begin{center}
|... -jobname "|\textit{target}|" "|[\textit{flags}]%
[|\def\jobname{|\textit{dest}|}|]|\input{|\textit{main}|}"|
\end{center}

%%%%%%%%%%%%%%%%%%%%%%%%%%%%%%%%%%%%%%%%%%%%%%%%%%%%%%%%%%%%%%%%%%%%%%%%%%%%%%%%
\subsection{Manual Code}
\label{sec:manual}

In case one cannot be certain whether the definitions file |childdoc.def|
is installed on the target \TeX{} distribution
and one prefers not to ship it,
it is conceivable to paste a few relevant commands into the sources.

To that end, drop all statements |\input{childdoc.def}|
and perform the replacements as outlined below.
Instead of |\childdocmain{|\textit{main}|}| add the following code
to the top of the main file:
%
\begin{center}
\begin{tabular}{l}
|\||ifdefined\childdocname\endinput\||fi\newif\ifchilddoc|\\
|\edef\childdocname{\scantokens\expandafter{\jobname\noexpand}}|\\
|\def\childdocmain{|\textit{main}|}\||ifx\childdocmain\childdocname\||else|\\
|\childdoctrue\includeonly{\childdocname}\let\jobname\childdocmain\||fi|\\
\end{tabular}
\end{center}
%
Instead of |\childdocof{|\textit{main}|}| just include the main file
at the top of each child file:
%
\begin{center}
|\input{|\textit{main}|}|
\end{center}
%
A simple redirection |\childdocforward{|\textit{dest}|}| is achieved by:
%
\begin{center}
|\def\jobname{|\textit{dest}|}\input{\jobname}|
\end{center}
%
The redirection with prefix
|\childdocforwardprefix[|\textit{prefix}|]{|\textit{dest}|}|
is accomplished by:
%
\begin{center}
\begin{tabular}{l}
|{\edef\jobname{\scantokens\expandafter{\jobname\noexpand}}|\\
|\def\redirectjob |\textit{prefix}|#1~~~{\gdef\jobname{|\textit{dest}|#1}}|\\
|\expandafter\redirectjob\jobname~~~}\input{\jobname}|
\end{tabular}
\end{center}

In an alternative approach,
child documents can be compiled by a specific command line
without additional code or specific definitions:
%
\begin{center}
|... -jobname "|\textit{target}|" "|[\textit{flags}]%
|\includeonly{|\textit{dest}|}\input{|\textit{main}|}"|
\end{center}
%

%%%%%%%%%%%%%%%%%%%%%%%%%%%%%%%%%%%%%%%%%%%%%%%%%%%%%%%%%%%%%%%%%%%%%%%%%%%%%%%%
%%%%%%%%%%%%%%%%%%%%%%%%%%%%%%%%%%%%%%%%%%%%%%%%%%%%%%%%%%%%%%%%%%%%%%%%%%%%%%%%
\section{Information}

%%%%%%%%%%%%%%%%%%%%%%%%%%%%%%%%%%%%%%%%%%%%%%%%%%%%%%%%%%%%%%%%%%%%%%%%%%%%%%%%
\subsection{Copyright}

Copyright \copyright{} 2017--2018 Niklas Beisert

This work may be distributed and/or modified under the
conditions of the \LaTeX{} Project Public License, either version 1.3
of this license or (at your option) any later version.
The latest version of this license is in
  \url{http://www.latex-project.org/lppl.txt}
and version 1.3 or later is part of all distributions of \LaTeX{}
version 2005/12/01 or later.

This work has the LPPL maintenance status `maintained'.

The Current Maintainer of this work is Niklas Beisert.

This work consists of the files |README.txt|, |childdoc.ins| and |childdoc.dtx|
as well as the derived files |childdoc.def|, |cdocsamp.tex|
with |cdocsch1.tex|, |cdocsch2.tex|, |cdocspt3.tex|, |cdocspt4.tex|,
|cdocsdrf.tex|, |cdocsfn1.tex|, |cdocsfn2.tex|
as well as |childdoc.pdf|.

%%%%%%%%%%%%%%%%%%%%%%%%%%%%%%%%%%%%%%%%%%%%%%%%%%%%%%%%%%%%%%%%%%%%%%%%%%%%%%%%
\subsection{Files and Installation}

The package consists of the files:
%
\begin{center}
\begin{tabular}{ll}
    |README.txt|   & readme file \\
    |childdoc.ins| & installation file \\
    |childdoc.dtx| & source file \\
    |childdoc.def| & definition file \\
    |cdocsamp.tex| & sample main file \\
    |cdocsch1.tex| & sample include file \\
    |cdocsch2.tex| & sample include file \\
    |cdocspt3.tex| & sample part file \\
    |cdocspt4.tex| & sample part file \\
    |cdocsdrf.tex| & sample redirection file \\
    |cdocsfn1.tex| & sample redirection file \\
    |cdocsfn2.tex| & sample redirection file \\
    |childdoc.pdf| & manual
\end{tabular}
\end{center}
%
The distribution consists of the files
|README.txt|, |childdoc.ins| and |childdoc.dtx|.
%
\begin{itemize}
\item
Run (pdf)\LaTeX{} on |childdoc.dtx|
to compile the manual |childdoc.pdf| (this file).
\item
Run \LaTeX{} on |childdoc.ins| to create the definitions file |childdoc.def|
and the sample |cdocsamp.tex| with include files
|cdocsch1.tex|, |cdocsch2.tex|, |cdocspt3.tex|, |cdocspt4.tex|,
|cdocsdrf.tex|, |cdocsfn1.tex|, |cdocsfn2.tex|.
Then copy the file |childdoc.def| to an appropriate directory of your \LaTeX{}
distribution, e.g.\ \textit{texmf-root}|/tex/latex/childdoc|.
\end{itemize}

%%%%%%%%%%%%%%%%%%%%%%%%%%%%%%%%%%%%%%%%%%%%%%%%%%%%%%%%%%%%%%%%%%%%%%%%%%%%%%%%
\subsection{Related CTAN Packages}

There are several other packages which offer a similar functionality:
%
\begin{itemize}
\item
The packages
\href{http://ctan.org/pkg/docmute}{\textsf{docmute}},
\href{http://ctan.org/pkg/includex}{\textsf{includex}} and
\href{http://ctan.org/pkg/standalone}{\textsf{standalone}}
provide commands to include only the document body of
a child file thus allowing both files to be compiled individually.
\item
The packages \href{http://ctan.org/pkg/subdocs}{\textsf{subdocs}}
and \href{http://ctan.org/pkg/subfiles}{\textsf{subfiles}}
provide structures in which the main and child documents can be
encapsulated and allowing them to be compiled individually.
The inclusion mechanism is different from the conventional |\include|.
\item
The package \href{http://ctan.org/pkg/combine}{\textsf{combine}}
is an elaborate solution to combine several documents into one.
\end{itemize}
%
See also the CTAN topic \href{http://ctan.org/topic/subdocs}{\textsf{subdocs}}
for further related packages.
The present package differs from the above solutions in that
a document structure constructed with the conventional |\include| mechanism
just needs two extra commands at the top of every file
such that all constituent files can be compiled individually.

%%%%%%%%%%%%%%%%%%%%%%%%%%%%%%%%%%%%%%%%%%%%%%%%%%%%%%%%%%%%%%%%%%%%%%%%%%%%%%%%
%\subsection{Feature Suggestions}
%
%The following is a list of features which may be useful for future
%versions of this package:
%%
%\begin{itemize}
%\item
%\ldots
%\end{itemize}

%%%%%%%%%%%%%%%%%%%%%%%%%%%%%%%%%%%%%%%%%%%%%%%%%%%%%%%%%%%%%%%%%%%%%%%%%%%%%%%%
\subsection{Revision History}

%%%%%%%%%%%%%%%%%%%%%%%%%%%%%%%%%%%%%%%%
\paragraph{v2.0:} 2018/12/30

\begin{itemize}
\item
immediate forward processing
\item
added |\childdocby| mechanism
\item
manual restructured
\end{itemize}

%%%%%%%%%%%%%%%%%%%%%%%%%%%%%%%%%%%%%%%%
\paragraph{v1.6:} 2018/01/17

\begin{itemize}
\item
application for development of include files
\item
corrections to manual
\end{itemize}

%%%%%%%%%%%%%%%%%%%%%%%%%%%%%%%%%%%%%%%%
\paragraph{v1.5:} 2017/05/21

\begin{itemize}
\item
more complete structuring introduced
\item
|\childdocof| introduced
\item
|\childdoc| renamed to |\childdocmain|
\item
|\childredirect| renamed to |\childdocforward| and |\childdocforwardprefix|
and functionality expanded
\end{itemize}

%%%%%%%%%%%%%%%%%%%%%%%%%%%%%%%%%%%%%%%%
\paragraph{v1.0:} 2017/04/27

\begin{itemize}
\item
manual and install package
\item
first version published on CTAN
\end{itemize}

%%%%%%%%%%%%%%%%%%%%%%%%%%%%%%%%%%%%%%%%
\paragraph{v0.6:} 2017/04/26

\begin{itemize}
\item
redirection mechanism added
\end{itemize}

%%%%%%%%%%%%%%%%%%%%%%%%%%%%%%%%%%%%%%%%
\paragraph{v0.5:} 2017/04/26

\begin{itemize}
\item
functionality in definition file
\end{itemize}


%%%%%%%%%%%%%%%%%%%%%%%%%%%%%%%%%%%%%%%%%%%%%%%%%%%%%%%%%%%%%%%%%%%%%%%%%%%%%%%%
%%%%%%%%%%%%%%%%%%%%%%%%%%%%%%%%%%%%%%%%%%%%%%%%%%%%%%%%%%%%%%%%%%%%%%%%%%%%%%%%
%%%%%%%%%%%%%%%%%%%%%%%%%%%%%%%%%%%%%%%%%%%%%%%%%%%%%%%%%%%%%%%%%%%%%%%%%%%%%%%%
\appendix

\settowidth\MacroIndent{\rmfamily\scriptsize 000\ }

 \DocInput{childdoc.dtx}

\end{document}
%</driver>
% \fi
%
% %%%%%%%%%%%%%%%%%%%%%%%%%%%%%%%%%%%%%%%%%%%%%%%%%%%%%%%%%%%%%%%%%%%%%%%%%%%%%%
% %%%%%%%%%%%%%%%%%%%%%%%%%%%%%%%%%%%%%%%%%%%%%%%%%%%%%%%%%%%%%%%%%%%%%%%%%%%%%%
% \section{Sample}
%\iffalse
%<*samplemain>
%\fi
%
% The following presents a sample document
% with two chapters, two parts, a title page,
% a compile flag as well as three forwarding files to set the flag.
% It consists of eight |.tex| files:
% \begin{center}
% \begin{tabular}{ll}
% |cdocsamp.tex|&main file\\
% |cdocsch1.tex|&include file for chapter 1\\
% |cdocsch2.tex|&include file for chapter 2\\
% |cdocspt3.tex|&include file for part 3\\
% |cdocspt4.tex|&include file for part 4\\
% |cdocsdrf.tex|&forwarding file for main file in draft mode\\
% |cdocsfi1.tex|&forwarding file for final version of chapter 1\\
% |cdocsfi2.tex|&forwarding file for final version of chapter 2\\
% \end{tabular}
% \end{center}
% Each of the eight files can be compiled directly by the \LaTeX{} compiler.
%
% %%%%%%%%%%%%%%%%%%%%%%%%%%%%%%%%%%%%%%
% \paragraph{Main File.}
%
% The main file is called |cdocsamp.tex|.
%
% Load the \textsf{childdoc} definitions and
% declare the filename for the main document:
%    \begin{macrocode}
\input{childdoc.def}
\childdocmain{}
%    \end{macrocode}

% Optional override for |\version| flag:
%    \begin{macrocode}
%%\ifchilddoc\else\providecommand{\version}{draft}\fi
%    \end{macrocode}

% Define the default values for the |\version| flag
% (|final| for the main file and |draft| for childs):
%    \begin{macrocode}
\ifchilddoc
\providecommand{\version}{draft}
\else
\providecommand{\version}{final}
\fi
%    \end{macrocode}

% Load the standard document class:
%    \begin{macrocode}
\documentclass[12pt]{article}
%    \end{macrocode}

% Start the document body:
%    \begin{macrocode}
\begin{document}
%    \end{macrocode}

% Declare a title page.
% Print title, part of document being processed and version flag:
%    \begin{macrocode}
\addtocounter{page}{-1}
\begin{center}
{\LARGE\bfseries{}childdoc example\par}
\vspace{1cm}
\ifchilddoc
\ifchilddocmanual part\else chapter\fi:
`\childdocname' of `\childdocjob'\par
\else
main document: `\childdocjob'\par
\fi
version: \version\par
\end{center}
\newpage
%    \end{macrocode}

% Manually include selected file,
% otherwise process as usual:
%    \begin{macrocode}
\ifchilddocmanual
\section*{part `\childdocname'}
\input{\childdocname}
\else
%    \end{macrocode}

% Include the two chapters:
%    \begin{macrocode}
\include{cdocsch1}
\include{cdocsch2}
%    \end{macrocode}

% Include the two parts unless only chapters should be displayed:
%    \begin{macrocode}
\ifchilddoc\else
\section{part three}
\input{cdocspt3}
\section{part four}
\input{cdocspt4}
\fi
%    \end{macrocode}

% Process as usual until here:
%    \begin{macrocode}
\fi
%    \end{macrocode}

% End of document body:
%    \begin{macrocode}
\end{document}
%    \end{macrocode}
%\iffalse
%</samplemain>
%\fi
%
% %%%%%%%%%%%%%%%%%%%%%%%%%%%%%%%%%%%%%%
% \paragraph{Chapter Include Files.}
%
% The include files are called |cdocsch1.tex| and |cdocsch2.tex|.
%
%\iffalse
%<*samplechap1|samplechap2>
%\fi

% Optional override for |\version| flag:
%    \begin{macrocode}
%%\providecommand{\version}{final}
%    \end{macrocode}

% Include the main document:
%    \begin{macrocode}
\input{childdoc.def}
\childdocof{cdocsamp}
%    \end{macrocode}

%\iffalse
%</samplechap1|samplechap2>
%\fi
%
%\iffalse
%<*samplechap1>
%\fi
% Some text for chapter 1:
%    \begin{macrocode}
\section{one}
some text in chapter one
%    \end{macrocode}

%\iffalse
%</samplechap1>
%\fi
% Some text for chapter 2:
%\iffalse
%<*samplechap2>
%\fi
%    \begin{macrocode}
\section{two}
more text in chapter two
%    \end{macrocode}

%\iffalse
%</samplechap2>
%\fi
%
% %%%%%%%%%%%%%%%%%%%%%%%%%%%%%%%%%%%%%%
% \paragraph{Part Include Files.}
%
% The include files are called |cdocspt3.tex| and |cdocspt4.tex|.
%
%\iffalse
%<*samplepart3|samplepart4>
%\fi

% Optional override for |\version| flag:
%    \begin{macrocode}
%%\providecommand{\version}{final}
%    \end{macrocode}

% Include the main document:
%    \begin{macrocode}
\input{childdoc.def}
\childdocby{cdocsamp}
%    \end{macrocode}

%\iffalse
%</samplepart3|samplepart4>
%\fi
%
%\iffalse
%<*samplepart3>
%\fi
% Some text for part 3:
%    \begin{macrocode}
some text in part three
%    \end{macrocode}

%\iffalse
%</samplepart3>
%\fi
% Some text for part 4:
%\iffalse
%<*samplepart4>
%\fi
%    \begin{macrocode}
more text in part four
%    \end{macrocode}

%\iffalse
%</samplepart4>
%\fi
%
% %%%%%%%%%%%%%%%%%%%%%%%%%%%%%%%%%%%%%%
% \paragraph{Forwarding for a Complete Draft.}
%
% The following forwarding file |cdocsdrf.tex|
% compiles the main document in draft mode:
%\iffalse
%<*sampledraft>
%\fi
%    \begin{macrocode}
\def\version{draft}
\input{childdoc.def}
\childdocforward{cdocsamp}
%    \end{macrocode}

%\iffalse
%</sampledraft>
%\fi
%
% %%%%%%%%%%%%%%%%%%%%%%%%%%%%%%%%%%%%%%
% \paragraph{Forwarding for Final Version of the Chapters.}
%
% The following forwarding files |cdocsfn1.tex| and |cdocsfn2.tex|
% (with identical content)
% compile the final versions of the child documents
% |cdocsch1.tex| and |cdocsch2.tex|, respectively:
%\iffalse
%<*samplefinal>
%\fi
%    \begin{macrocode}
\def\version{final}
\input{childdoc.def}
\childdocforwardprefix[cdocsamp]{cdocsfn}{cdocsch}
%    \end{macrocode}

%\iffalse
%</samplefinal>
%\fi
%
% %%%%%%%%%%%%%%%%%%%%%%%%%%%%%%%%%%%%%%
% \paragraph{Command Line Processing.}
%
% The following three command lines generate the output files
% |cdocscld|, |cdocscl1| and |cdocscl2|
% which should be identical to
% |cdocsdrf|, |cdocsch1| and |cdocsfn2|, respectively:
% \begin{center}
% \begin{tabular}{l}
% |latex -jobname cdocscld \|\\
% |  "\def\version{draft}\input{childdoc.def}\childdocforward{cdocsamp}"|\\
% |latex -jobname cdocscl1 \|\\
% |  "\input{childdoc.def}\childdocforward[cdocsamp]{cdocsch1}"|\\
% |latex -jobname cdocscl2 \|\\
% |  "\def\version{final}\input{childdoc.def}\childdocforward{cdocsch2}"|
% \end{tabular}
% \end{center}
% Note that the trailing backslash on each first line
% merely continues the input to the second line
% (for convenient cut ant paste).
% Furthermore, the command |latex| can be replaced by any
% of its alternative versions such as |pdflatex|.
%
% %%%%%%%%%%%%%%%%%%%%%%%%%%%%%%%%%%%%%%%%%%%%%%%%%%%%%%%%%%%%%%%%%%%%%%%%%%%%%%
% %%%%%%%%%%%%%%%%%%%%%%%%%%%%%%%%%%%%%%%%%%%%%%%%%%%%%%%%%%%%%%%%%%%%%%%%%%%%%%
% \section{Implementation}
%\iffalse
%<*package>
%\fi
%
% This section describes the definitions file |childdoc.def|.

% The definitions cannot be loaded using |\usepackage| or |\RequirePackage|
% which has a mechanism to prevent loading a style file more than once.
% When loading the definitions by means of |\input|
% multiple instances have to be prevented manually:
%\iffalse
%This code needs to be before the `\ProvidesFile' directive
%which is defined at the beginning of this file.
%Therefore it is also placed there and commented out here.
%</package>
%<*discard>
%\fi
%    \begin{macrocode}
\ifdefined\childdocmain\endinput\fi
%    \end{macrocode}
%\iffalse
%</discard>
%<*package>
%\fi
%
% \macro{\ifchilddoc}
% \macro{\ifchilddocmanual}
% The conditional |\ifchilddoc| tells whether a
% child (true) or main (false) document is being compiled.
% The conditional |\ifchilddocmanual| tells whether
% the |\includeonly| mechanism is used (false) or
% the selection of child files must be performed manually (true).
% The definitions initialise to false:
%    \begin{macrocode}
\newif\ifchilddoc
\newif\ifchilddocmanual
%    \end{macrocode}

% \macro{\childdocname}
% \macro{\childdocjob}
% The macro |\childdocname| stores the name of the main document
% to be compiled. The macro |\childdocjob| stores the name of
% the document on which the \LaTeX{} compiler was originally invoked.
% The content of |\jobname| cannot be compared
% to filenames specified in the source due to different catcodes.
% The following code rescans |\jobname|, stores the result
% in |\childdocname| and saves a copy in |\childdocjob|:
%    \begin{macrocode}
\edef\childdocname{\scantokens\expandafter{\jobname\noexpand}}
\let\childdocjob\childdocname
%    \end{macrocode}

% \macro{\childdocdisable}
% The macro |\childdocdisable| prevents the main file
% from being processed more than once.
% At this stage, the main document command |\childdocmain|
% is assumed to be called once again where it should do nothing.
% Any subsequent call to it should prevent
% a secondary processing of the main document
% It overwrites the forwarding commands
% |\childdocof| and |\childdocforward|
% with empty macros to prevent further inclusions of the main document:
%    \begin{macrocode}
\newcommand{\childdocdisable}
{
  \renewcommand{\childdocmain}[1]{\renewcommand{\childdocmain}[1]{\endinput}}
  \renewcommand{\childdocof}[1]{}
  \renewcommand{\childdocby}[2][]{}
  \renewcommand{\childdocforward}[2][]{}
  \renewcommand{\childdocdisable}{}
}
%    \end{macrocode}

% \macro{\childdocmain}
% The macro |\childdocmain| is to be called at the top of the main file
% with nothing or the main filename (without extension) as argument.
% First, it breaks loops.
% If the argument is not empty and does not match |\childdocname|
% (which is set by the first inclusion of |childdoc.def|),
% |\ifchilddoc| is set to true, |\includeonly| is applied to the child file
% and |\jobname| is set to the main file
% (for proper handling of |.aux| files):
%    \begin{macrocode}
\newcommand{\childdocmain}[1]
{
  \childdocdisable\childdocmain{}
  \if?#1?\else
    \begingroup
      \def\childdoctmp{#1}
      \ifx\childdoctmp\childdocname
        \def\childdoctmp{}
      \else
        \def\childdoctmp
        {
          \childdoctrue
          \includeonly{\childdocname}
          \def\childdocjob{#1}
          \def\jobname{#1}
        }
      \fi
      \expandafter
    \endgroup
    \childdoctmp
  \fi
}
%    \end{macrocode}

% \macro{\childdocof}
% The command |\childdocof| redirects
% compilation to the main file |#1|.
%    \begin{macrocode}
\newcommand{\childdocof}[1]
{
  \childdocdisable
  \childdoctrue
  \includeonly{\childdocname}
  \def\jobname{#1}
  \def\childdocjob{#1}
  \input{#1}
}
%    \end{macrocode}

% \macro{\childdocby}
% The command |\childdocby| ....
%    \begin{macrocode}
\newcommand{\childdocby}[2][]
{
  \childdocdisable
  \childdoctrue
  \childdocmanualtrue
  \if?#1?\else
    \def\jobname{#2}
  \fi
  \def\childdocjob{#2}
  \input{#2}
  \endinput
}
%    \end{macrocode}

% \macro{\childdocforward}
% The command |\childdocforward| redirects
% compilation to the main file or
% (if the optional argument is given) a child file.
% Parameters are set as if the main file
% or a child file starting with |\childdocof| was compiled.
% Then compilation is handed over to the main file:
%    \begin{macrocode}
\newcommand{\childdocforward}[2][]
{
  \begingroup
    \if?#1?
      \def\childdoctmp
      {
        \def\childdocname{#2}
        \def\childdocjob{#2}
        \def\jobname{#2}
        \input{#2}
        \endinput
      }
    \else
      \def\childdoctmp
      {
        \childdocdisable
        \def\childdocname{#2}
        \childdoctrue
        \includeonly{#2}
        \def\childdocjob{#1}
        \def\jobname{#1}
        \input{#1}
        \endinput
      }
    \fi
    \expandafter
  \endgroup
  \childdoctmp
}
%    \end{macrocode}

% \macro{\childdocforwardprefix}
% The command |\childdocforwardprefix| redirects
% compilation to the main or a child file by means of a pattern.
% The prefix |#1| in the current filename is replaced by |#2|
% and the suffix of the current filename is kept
% (it is assumed that the filename does not contain the substring `|~~~|'
% which is used as a delimiter).
% Compilation is handed over to the new file by |\childdocforward|:
%    \begin{macrocode}
\newcommand{\childdocforwardprefix}[3][]
{
  \begingroup
    \def\childdocextract #2##1~~~{\def\childdoctmp{\childdocforward[#1]{#3##1}}}
    \expandafter\childdocextract\childdocname~~~
    \expandafter
  \endgroup
  \childdoctmp
}
%    \end{macrocode}

% \macro{\childdoc}
% The deprecated macro |\childdoc| is a legacy version of |\childdocmain|:
%    \begin{macrocode}
\newcommand{\childdoc}{\childdocmain}
%    \end{macrocode}

% \macro{\childdocredirect}
% The deprecated macro |\childdocredirect| is a legacy version
% of |\childdocforward| and |\childdocforwardprefix|:
%    \begin{macrocode}
\newcommand{\childdocredirect}[2][]
{
  \begingroup
    \if?#1?
      \def\childdoctmp{\childdocforward{#2}}
    \else
      \def\childdoctmp{\childdocforwardprefix{#1}{#2}}
    \fi
    \expandafter
  \endgroup
  \childdoctmp
}
%    \end{macrocode}

%\iffalse
%</package>
%\fi
%
\endinput
|\\
|\childdocforwardprefix{final}{child}|
\end{tabular}
\end{center}
%

Note that when several versions of a main file and/or of each child file
are to be generated, it may be convenient to set up a |Makefile| or
shell script to automatise the process.

%%%%%%%%%%%%%%%%%%%%%%%%%%%%%%%%%%%%%%%%%%%%%%%%%%%%%%%%%%%%%%%%%%%%%%%%%%%%%%%%
\subsection{Command Line Processing}
\label{sec:commandline}

The effect of redirection files can also be achieved by invoking
the \LaTeX{} compiler with a more elaborate command line.
Most conveniently this should be done as part
of a shell script or a |Makefile|.

When using \textsf{childdoc} in the main file, the following
command lines effectively perform a redirection
(note that depending on the shell being used,
backslashes may have to be doubled: `|\|' $\to$ `|\\|'):
%
\begin{center}
|... -jobname "|\textit{target}|" |\\|"|[\textit{flags}]%
|% \iffalse
%
% childdoc.dtx Copyright (C) 2017-2018 Niklas Beisert
%
% This work may be distributed and/or modified under the
% conditions of the LaTeX Project Public License, either version 1.3
% of this license or (at your option) any later version.
% The latest version of this license is in
%   http://www.latex-project.org/lppl.txt
% and version 1.3 or later is part of all distributions of LaTeX
% version 2005/12/01 or later.
%
% This work has the LPPL maintenance status `maintained'.
%
% The Current Maintainer of this work is Niklas Beisert.
%
% This work consists of the files childdoc.dtx and childdoc.ins
% and the derived files childdoc.def and cdocsamp.tex with
% cdocsch1.tex, cdocsch2.tex, cdocsdrf.tex, cdocsfn1.tex, cdocsfn2.tex.
%
%<package>\ifdefined\childdocmain\endinput\fi
%<package>\ProvidesFile{childdoc.def}[2018/12/30 v2.0 child document driver]
%<samplemain>\ProvidesFile{cdocsamp.tex}[2018/12/30 v2.0 sample for childdoc]
%<*driver>
%\ProvidesFile{childdoc.drv}[2018/12/30 v2.0 childdoc reference manual file]
\PassOptionsToClass{10pt,a4paper}{article}
\documentclass{ltxdoc}

\usepackage[margin=35mm]{geometry}
\usepackage{hyperref}
\usepackage{hyperxmp}
\usepackage[usenames]{color}

\hypersetup{colorlinks=true}
\hypersetup{pdfstartview=FitH}
\hypersetup{pdfpagemode=UseNone}
\hypersetup{pdfsource={}}
\hypersetup{pdflang={en-UK}}
\hypersetup{pdfcopyright={Copyright 2017-2018 Niklas Beisert.
  This work may be distributed and/or modified under the
  conditions of the LaTeX Project Public License, either version 1.3
  of this license or (at your option) any later version.}}
\hypersetup{pdflicenseurl={http://www.latex-project.org/lppl.txt}}
\hypersetup{pdfcontactaddress={ETH Zurich, ITP, HIT K,
  Wolfgang-Pauli-Strasse 27}}
\hypersetup{pdfcontactpostcode={8093}}
\hypersetup{pdfcontactcity={Zurich}}
\hypersetup{pdfcontactcountry={Switzerland}}
\hypersetup{pdfcontactemail={nbeisert@itp.phys.ethz.ch}}
\hypersetup{pdfcontacturl={http://people.phys.ethz.ch/\xmptilde nbeisert/}}

\newcommand{\secref}[1]{\hyperref[#1]{section \ref*{#1}}}

\parskip1ex
\parindent0pt
\let\olditemize\itemize
\def\itemize{\olditemize\parskip0pt}

\begin{document}

\title{The \textsf{childdoc} Package}
\hypersetup{pdftitle={The childdoc Package}}
\author{Niklas Beisert\\[2ex]
  Institut f\"ur Theoretische Physik\\
  Eidgen\"ossische Technische Hochschule Z\"urich\\
  Wolfgang-Pauli-Strasse 27, 8093 Z\"urich, Switzerland\\[1ex]
  \href{mailto:nbeisert@itp.phys.ethz.ch}
  {\texttt{nbeisert@itp.phys.ethz.ch}}}
\hypersetup{pdfauthor={Niklas Beisert}}
\hypersetup{pdfsubject={Manual for the LaTeX2e Package childdoc}}
\date{30 December 2018, \textsf{v2.0}}
\maketitle

\begin{abstract}\noindent
\textsf{childdoc} is a \LaTeXe{} package
that enables the direct compilation
of document sections included by |\include|
to individual files.
\end{abstract}

\begingroup
\parskip0ex
\tableofcontents
\endgroup

%%%%%%%%%%%%%%%%%%%%%%%%%%%%%%%%%%%%%%%%%%%%%%%%%%%%%%%%%%%%%%%%%%%%%%%%%%%%%%%%
%%%%%%%%%%%%%%%%%%%%%%%%%%%%%%%%%%%%%%%%%%%%%%%%%%%%%%%%%%%%%%%%%%%%%%%%%%%%%%%%
\section{Introduction}

\LaTeX{} provides a mechanism to structure a large document (such as a book)
into a main file and several child files (containing the chapters)
using the |\include| command.
This mechanism is beneficial for documents
which span hundreds of pages in order to
make the source file(s) more manageable.
Moreover, compilation can be restricted to
selected child files by means of the |\includeonly| command.
The latter feature can be used to reduce the compilation time while editing
(this was significantly more useful in the earlier days of \LaTeX{})
or to generate a smaller document which is easier to navigate.
Another application of |\includeonly| is to generate
documents consisting of selected parts of the complete document.

However, there are a few drawbacks of the plain |\include| mechanism:
\begin{itemize}
\item
The child files cannot be compiled on their own,
they can only be compiled via the main file.
A naive editing environment
(such as a text editor with an option
to have the current file processed by \LaTeX)
may require one to switch to the main file before compiling;
attempting to compile the child file produces errors.
\item
The main file must be modified (each time)
to adjust the |\includeonly| command
to the present needs. This easily leaves the main file in a messy state.
\item
The generated document will always carry the filename
of the main document. This is inconvenient if
several child files are to be compiled and
to be kept for distribution.
\end{itemize}

The present package provides a simple interface
to make child files individually compilable by \LaTeX{}.
Compiling a child file then has the same effect as compiling
the main file with an |\includeonly| command
to select the appropriate child.
Moreover the generated document will carry the name of the child
rather than the main file.
This resolves all three above issues.

This feature is meant to make the editing of books,
thesis documents and lecture notes somewhat more convenient.
However, the package can also be used efficiently for
composing a series of documents (such as exercise sheets)
which are typically distributed individually.
It then assists the author in generating the individual documents
(potentially in different versions)
as well as a document containing the collected series.
Another application is in developing style files
or other kinds of included material
where compilation of the style file could redirect
to a sample or test file.

%%%%%%%%%%%%%%%%%%%%%%%%%%%%%%%%%%%%%%%%%%%%%%%%%%%%%%%%%%%%%%%%%%%%%%%%%%%%%%%%
%%%%%%%%%%%%%%%%%%%%%%%%%%%%%%%%%%%%%%%%%%%%%%%%%%%%%%%%%%%%%%%%%%%%%%%%%%%%%%%%
\section{Usage}

First of all, the package \textsf{childdoc} is \emph{not} a standard
\LaTeXe{} |.sty| style file! Therefore it needs to be invoked in
a non-standard way.

%%%%%%%%%%%%%%%%%%%%%%%%%%%%%%%%%%%%%%%%%%%%%%%%%%%%%%%%%%%%%%%%%%%%%%%%%%%%%%%%
\subsection{Included Files}
\label{sec:include}

%%%%%%%%%%%%%%%%%%%%%%%%%%%%%%%%%%%%%%%%
\DescribeMacro{\childdocmain}
To use the package, add the commands
\begin{center}
\begin{tabular}{l}
|\input{childdoc.def}|\\
|\childdocmain{}|\\
\end{tabular}
\end{center}
at the very top of the main \LaTeX{} file,
in particular \emph{before} the |\documentclass| statement!
The argument of |\childdocmain| should be left empty
(but it must be present).

%%%%%%%%%%%%%%%%%%%%%%%%%%%%%%%%%%%%%%%%
\DescribeMacro{\childdocof}
Furthermore, add the commands
\begin{center}
\begin{tabular}{l}
|\input{childdoc.def}|\\
|\childdocof{|\textit{main}|}|\\
\end{tabular}
\end{center}
at the top of every child file \textit{child}
which is included by |\include{|\textit{child}|}|
from within the main file
(or at least for those files to be compiled individually).
The argument \textit{main} must be the filename of the main file.

There are a couple of
considerations in setting up the main and child documents:

%%%%%%%%%%%%%%%%%%%%%%%%%%%%%%%%%%%%%%%%
\paragraph{Restrictions.}

Please note the following restrictions:
\begin{itemize}
\item
|\childdocmain| must be called with one argument \textit{main}
to ensure compatibility with earlier version of the package.
It must either be empty (|\childdocmain{}|)
or precisely match the filename of the main file in which it is specified.
See \secref{sec:detection} for further information.
\item
The filename \textit{main} must be specified without the |.tex| extension.
\item
The filename \textit{main} is case sensitive
(even in case-insensitive file systems)
due to internal string comparison.
\item
The argument \textit{main} should be fully expanded, it cannot be a macro.
\item
Subdirectories and special characters should be avoided in filenames.
\item
The command |\childdocmain{|\textit{main}|}| must be followed by a whitespace.
It should not be followed immediately by another command
or by a comment mark `|%|'.
This is because the \TeX{} parser reads the token immediately following
the argument of |\childdocmain| and puts it
at the beginning of every child section;
however, a white\-space is ignored.
\end{itemize}

%%%%%%%%%%%%%%%%%%%%%%%%%%%%%%%%%%%%%%%%
\paragraph{Content of Main File.}

It is advisable to place all content in the child files included by |\include|.
Any output contained in the main file will appear in all child documents
unless suppressed manually;
it cannot be suppressed automatically by the |\includeonly| directive
and thus should normally be avoided.
A method to include some content in the main file
by means of conditional processing is described in \secref{sec:conditional}.

%%%%%%%%%%%%%%%%%%%%%%%%%%%%%%%%%%%%%%%%
\paragraph{Page Numbering.}

When only a part of the document is compiled,
the appropriate numbering of pages
(as well as other status parameters)
is determined from the |.aux| files.
The latter contain information from previous passes.
However this information needs to propagate through
all intermediate child documents.
Therefore the page numbering in child documents may well
be inconsistent until the complete document is compiled at least once.

A useful (if unconventional) way to always ensure a consistent
page numbering is to restart the numbering in each child document
and denote the pages by `\textit{child}|.|\textit{page}'
where \textit{child} represents the chapter/section number of the child file.
This can be achieved by the command
|\numberwithin{page}{|\textit{child}|}|
of the \textsf{amsmath} package
where \textit{child} can be |chapter| or |section|
depending on the chosen structuring.
Alternatively, one can modify the macro |\thepage| appropriately
and reset the counter |page| at the start of each child file.

%%%%%%%%%%%%%%%%%%%%%%%%%%%%%%%%%%%%%%%%%%%%%%%%%%%%%%%%%%%%%%%%%%%%%%%%%%%%%%%%
\subsection{Conditional Processing}
\label{sec:conditional}

The package provides a mechanism to compile different versions
of a document. To customise the versions further some conditional processing
can come in handy to distinguish which version is being compiled.
The package provides two macros to describe the compilation context:

%%%%%%%%%%%%%%%%%%%%%%%%%%%%%%%%%%%%%%%%
\DescribeMacro{\ifchilddoc}
The conditional |\ifchilddoc| distinguishes between the compilation of
child documents and the main document:
%
\begin{center}
|\ifchilddoc |\textit{child-code}| |[|\||else |\textit{main-code}]| \||fi|
\end{center}

%%%%%%%%%%%%%%%%%%%%%%%%%%%%%%%%%%%%%%%%
\DescribeMacro{\childdocname}
\DescribeMacro{\childdocjob}
The macro |\childdocname| contains the filename (without extension)
of the main or child file being processed.
Note that |\childdocjob| will always contain the name of the main file.

%%%%%%%%%%%%%%%%%%%%%%%%%%%%%%%%%%%%%%%%
\paragraph{Title Page.}

Conditional processing can be used to include a title or banner page
in the main document when proper precautions are taken.
Importantly, the code in the main file should ensure that the page counter
(as well as other status parameters which are stored in the |.aux| files)
takes the same value after the conditional processing.
Otherwise the page numbers may take divergent values
depending on which part is compiled.

For example, a title page could be declared by:
%
\begin{center}
\begin{tabular}{l}
|\ifchilddoc\||else|\\
|\addtocounter{page}{-1}|\\
\textit{code for title page}\\
|\newpage|\\
|\||fi|
\end{tabular}
\end{center}
%
A banner page for the child documents can be generated by:
%
\begin{center}
\begin{tabular}{l}
|\ifchilddoc|\\
|\addtocounter{page}{-1}|\\
\textit{code for banner page}\\
|\newpage|\\
|\||fi|
\end{tabular}
\end{center}
%
Here one could write a message such as:
\begin{center}
|This is the part \childdocname{} of \childdocjob{}.|
\end{center}

%%%%%%%%%%%%%%%%%%%%%%%%%%%%%%%%%%%%%%%%%%%%%%%%%%%%%%%%%%%%%%%%%%%%%%%%%%%%%%%%
\subsection{Flags}
\label{sec:flags}

The package makes it easy to generate different versions
of the main or child documents.
To this end compilation flags can be defined
and assigned different default values.
They will be particularly useful in conjunction
with the forwarding mechanism described in \secref{sec:forward}.

For example, it may be useful to have a flag |\version|
which can be set to |draft| or |final|.
The document source will contain some conditional code
depending on the value of |\version|.
Suppose further, the flag should default to |final| for the main file
and to |draft| for child files
which is a natural assignment for editing the document.
This is achieved by placing the following code
in the preamble of the main document
(below the |\childdocmain| directive):
%
\begin{center}
\begin{tabular}{l}
|\ifchilddoc|\\
|\providecommand{\version}{draft}|\\
|\||else|\\
|\providecommand{\version}{final}|\\
|\||fi|
\end{tabular}
\end{center}
%
The definition by |\providecommand| makes sure
that previous definitions are not overwritten.
Further statements |\providecommand{\version}{...}|
can thus be added before the above code to override it.

For the main file, one might add a line
(between |\childdocmain| and the above block)
%
\begin{center}
|%\ifchilddoc\||else\providecommand{\version}{draft}\||fi|
\end{center}
%
which can be uncommented to produce a draft version.
Likewise one can add a line to the very top of a child file
(above the |\childdocof{|\textit{main}|}| directive)
%
\begin{center}
|%\providecommand{\version}{final}|
\end{center}
%
which can be uncommented to produce the final version of this child document.

%%%%%%%%%%%%%%%%%%%%%%%%%%%%%%%%%%%%%%%%%%%%%%%%%%%%%%%%%%%%%%%%%%%%%%%%%%%%%%%%
\subsection{Forwarding}
\label{sec:forward}

Different versions of the main or child documents
using compilation flags as described in \secref{sec:flags}
can be (permanently) stored in different files
for convenient compilation, viewing and distribution.
To this end, the package defines a command
to pass on compilation to a different file:

%%%%%%%%%%%%%%%%%%%%%%%%%%%%%%%%%%%%%%%%
\DescribeMacro{\childdocforward}
The command |\childdocforward| redirects processing to
another source file:
%
\begin{center}
\begin{tabular}{l}
|\input{childdoc.def}|\\
|\childdocforward[|\textit{main}|]{|\textit{dest}|}|\\
\end{tabular}
\end{center}
%
The argument \textit{dest} is the destination file
(without extension).
It should be the main file or one of the child files.
Note that further \textsf{childdoc} directives
such as |\childdocof| and |\childdocforward|
in the indicated file will be processed in this form.
The optional argument \textit{main}
passes on directly to the main file \textit{main}
while pretending to compile the child \textit{dest}.
This form behaves as if \textit{dest}
issues |\childdocof{|\textit{main}|}| right away,
and no further \textsf{childdoc} directives will be processed.

%%%%%%%%%%%%%%%%%%%%%%%%%%%%%%%%%%%%%%%%
\DescribeMacro{\...prefix}
In the alternative form |\childdocforwardprefix|,
%
\begin{center}
\begin{tabular}{l}
|\input{childdoc.def}|\\
|\childdocforwardprefix[|\textit{main}|]{|\textit{prefix}|}{|\textit{dest}|}|
\end{tabular}
\end{center}
%
the destination file is determined by a pattern
depending on the current file:
To make this work, the current file must be called
`{\textit{prefix}\hspace{0.2em}\textit{suffix}}'
with \textit{prefix} matching precisely the argument.
Processing is then passed on to the file
`{\textit{dest}\hspace{0.2em}\textit{suffix}}'.
Surely, the same effect is achieved by
directly specifying the
argument `{\textit{dest}\hspace{0.2em}\textit{suffix}}'
in the first form.
However, that requires to set up a different file
for each child. With the alternative form of the command
all these files can have exactly the same content
which simplifies setting them up and maintaining them.

For example, the following file |draft.tex|
with a compilation flag |\version| as described in \secref{sec:flags}
compiles the main document as a draft:
%
\begin{center}
\begin{tabular}{l}
|\def\version{draft}|\\
|\input{childdoc.def}|\\
|\childdocforward{|\textit{main}|}|
\end{tabular}
\end{center}
%
Likewise, the following files |final|\textit{nn}|.tex|
compile the final version of the child document
|child|\textit{nn}|.tex|:
%
\begin{center}
\begin{tabular}{l}
|\def\version{final}|\\
|\input{childdoc.def}|\\
|\childdocforwardprefix{final}{child}|
\end{tabular}
\end{center}
%

Note that when several versions of a main file and/or of each child file
are to be generated, it may be convenient to set up a |Makefile| or
shell script to automatise the process.

%%%%%%%%%%%%%%%%%%%%%%%%%%%%%%%%%%%%%%%%%%%%%%%%%%%%%%%%%%%%%%%%%%%%%%%%%%%%%%%%
\subsection{Command Line Processing}
\label{sec:commandline}

The effect of redirection files can also be achieved by invoking
the \LaTeX{} compiler with a more elaborate command line.
Most conveniently this should be done as part
of a shell script or a |Makefile|.

When using \textsf{childdoc} in the main file, the following
command lines effectively perform a redirection
(note that depending on the shell being used,
backslashes may have to be doubled: `|\|' $\to$ `|\\|'):
%
\begin{center}
|... -jobname "|\textit{target}|" |\\|"|[\textit{flags}]%
|\input{childdoc.def}\childdocforward[|\textit{main}|]{|\textit{dest}|}"|
\end{center}
%
Here \textit{target} is the name of the output file,
\textit{main} is the name of the main file
and \textit{dest} is the name of the main or child file to be processed
(all filenames without extensions).
The optional argument \textit{main} can be omitted
if \textit{main} matches \textit{dest}.
Optionally, compilation \textit{flags} can be defined via |\def| commands.
This command line makes the \TeX{} engine believe
it is compiling the file \textit{target}
whose content is specified as the latter parameter.
The provided code then forwards the processing to
\textit{main} or \textit{dest} as described in \secref{sec:forward}.

%%%%%%%%%%%%%%%%%%%%%%%%%%%%%%%%%%%%%%%%%%%%%%%%%%%%%%%%%%%%%%%%%%%%%%%%%%%%%%%%
\subsection{Include by Input}
\label{sec:input}

Including child documents by |\include| has some restrictions by design.
Most notably, the content of a child document always occupies
its own set of pages; pages cannot be shared between child documents.
Usually, this behaviour makes perfect sense
because each child document contain an essential part of the document.
However, in some situations it may be desirable to compose
a document from a collection of parts
without having mandatory page breaks between then.
For this case, the package
provides a mechanism to include parts
by |\input| which can also be processed individually.
However, by construction this mechanism
requires manual handling of the content to be output.

%%%%%%%%%%%%%%%%%%%%%%%%%%%%%%%%%%%%%%%%
\DescribeMacro{\ifchilddocmanual}
The main file should be prepared as usual, see \secref{sec:include}.
However, the document body must make a distinction
between processing of an individual part and of the main document, e.g.:
%
\begin{center}
\begin{tabular}{l}
|\ifchilddocmanual|\\
|\input{\childdocname}|\\
|\||else|\\
\textit{document body with }|\input{|\textit{part}|}|\\
|\||fi|
\end{tabular}
\end{center}
%
The conditional |\ifchilddocmanual| is true whenever
a part to be included by |\input| is being compiled,
and the name of the part is stored in |\childdocname|.

%%%%%%%%%%%%%%%%%%%%%%%%%%%%%%%%%%%%%%%%
\DescribeMacro{\childdocby}
Each part to be included by |\input| should start with:
%
\begin{center}
\begin{tabular}{l}
|\input{childdoc.def}|\\
|\childdocby{|\textit{main}|}|\\
\end{tabular}
\end{center}
%
The directive |\childdocby| is similar to |\childdocof|
described in \secref{sec:include},
but the subsequent selection of content must be done manually.
To that end, both |\ifchilddoc| and |\ifchilddocmanual|
will be true upon processing of a part,
and the name of the part is stored in |\childdocname|.
Note that |\jobname| will be set to the filename of the current part
so that each part receives an individual |.aux| file
that does not interfere with the |.aux| file(s) of the main document.
This behaviour can be altered by the alternative form
|\childdocby[*]{|\textit{main}|}| (with a non-empty optional argument)
which uses the |.aux| file of the main document
by setting |\jobname| to \textit{main}.

%%%%%%%%%%%%%%%%%%%%%%%%%%%%%%%%%%%%%%%%%%%%%%%%%%%%%%%%%%%%%%%%%%%%%%%%%%%%%%%%
\subsection{Driver Development}
\label{sec:driver}

The \textsf{childdoc} mechanism can also be use for the development
of definition files such as \LaTeX{} styles or classes.
This case differs from the above setup with multiple parts
included by |\include| in that no |\includeonly| should be invoked.
This can be achieved by starting the include file
(before |\ProvidesPackage|) with:
%
\begin{center}
\begin{tabular}{l}
|\input{childdoc.def}|\\
|\childdocforward{|\textit{main}|}|\\
\end{tabular}
\end{center}
%
or alternatively with:
%
\begin{center}
\begin{tabular}{l}
|\input{childdoc.def}|\\
|\childdocby{|\textit{main}|}|\\
\end{tabular}
\end{center}
%
Both forms have slightly different effects as described above.
The main file is prepared as usual, see \secref{sec:include}.

%%%%%%%%%%%%%%%%%%%%%%%%%%%%%%%%%%%%%%%%%%%%%%%%%%%%%%%%%%%%%%%%%%%%%%%%%%%%%%%%
\subsection{Legacy Detection}
\label{sec:detection}

The directive |\childdocmain| in the main file can detect
whether the complete document or merely a child is to be compiled
even without using the directive |\childdocof|.
This method is deprecated because it is less robust
and there is no compelling reason to use it;
it is merely provided for backward compatibility
and it may be removed in future versions.

If the detection mechanism is to be used,
it is mandatory to correctly specify
the filename of the main file as the argument of |\childdocmain|:
%
\begin{center}
\begin{tabular}{l}
|\input{childdoc.def}|\\
|\childdocmain{|\textit{main}|}|\\
\end{tabular}
\end{center}
%
If |\jobname| does not match the argument \textit{main} of |\childdocmain|,
it is assumed that |\jobname| points to the child file to be compiled.
When using |\childdocmain| with the main file specified as argument,
it suffices to start a child file
with just |\input{|\textit{main}|}|
without loading of the package and using |\childdocof|.
If instead all processing is done
with the appropriate \textsf{childdoc} directives,
the argument of \textit{main} of |\childdocmain| can be empty.

An alternative version of the command line processing described
in \secref{sec:commandline} using the detection mechanism reads:
%
\begin{center}
|... -jobname "|\textit{target}|" "|[\textit{flags}]%
[|\def\jobname{|\textit{dest}|}|]|\input{|\textit{main}|}"|
\end{center}

%%%%%%%%%%%%%%%%%%%%%%%%%%%%%%%%%%%%%%%%%%%%%%%%%%%%%%%%%%%%%%%%%%%%%%%%%%%%%%%%
\subsection{Manual Code}
\label{sec:manual}

In case one cannot be certain whether the definitions file |childdoc.def|
is installed on the target \TeX{} distribution
and one prefers not to ship it,
it is conceivable to paste a few relevant commands into the sources.

To that end, drop all statements |\input{childdoc.def}|
and perform the replacements as outlined below.
Instead of |\childdocmain{|\textit{main}|}| add the following code
to the top of the main file:
%
\begin{center}
\begin{tabular}{l}
|\||ifdefined\childdocname\endinput\||fi\newif\ifchilddoc|\\
|\edef\childdocname{\scantokens\expandafter{\jobname\noexpand}}|\\
|\def\childdocmain{|\textit{main}|}\||ifx\childdocmain\childdocname\||else|\\
|\childdoctrue\includeonly{\childdocname}\let\jobname\childdocmain\||fi|\\
\end{tabular}
\end{center}
%
Instead of |\childdocof{|\textit{main}|}| just include the main file
at the top of each child file:
%
\begin{center}
|\input{|\textit{main}|}|
\end{center}
%
A simple redirection |\childdocforward{|\textit{dest}|}| is achieved by:
%
\begin{center}
|\def\jobname{|\textit{dest}|}\input{\jobname}|
\end{center}
%
The redirection with prefix
|\childdocforwardprefix[|\textit{prefix}|]{|\textit{dest}|}|
is accomplished by:
%
\begin{center}
\begin{tabular}{l}
|{\edef\jobname{\scantokens\expandafter{\jobname\noexpand}}|\\
|\def\redirectjob |\textit{prefix}|#1~~~{\gdef\jobname{|\textit{dest}|#1}}|\\
|\expandafter\redirectjob\jobname~~~}\input{\jobname}|
\end{tabular}
\end{center}

In an alternative approach,
child documents can be compiled by a specific command line
without additional code or specific definitions:
%
\begin{center}
|... -jobname "|\textit{target}|" "|[\textit{flags}]%
|\includeonly{|\textit{dest}|}\input{|\textit{main}|}"|
\end{center}
%

%%%%%%%%%%%%%%%%%%%%%%%%%%%%%%%%%%%%%%%%%%%%%%%%%%%%%%%%%%%%%%%%%%%%%%%%%%%%%%%%
%%%%%%%%%%%%%%%%%%%%%%%%%%%%%%%%%%%%%%%%%%%%%%%%%%%%%%%%%%%%%%%%%%%%%%%%%%%%%%%%
\section{Information}

%%%%%%%%%%%%%%%%%%%%%%%%%%%%%%%%%%%%%%%%%%%%%%%%%%%%%%%%%%%%%%%%%%%%%%%%%%%%%%%%
\subsection{Copyright}

Copyright \copyright{} 2017--2018 Niklas Beisert

This work may be distributed and/or modified under the
conditions of the \LaTeX{} Project Public License, either version 1.3
of this license or (at your option) any later version.
The latest version of this license is in
  \url{http://www.latex-project.org/lppl.txt}
and version 1.3 or later is part of all distributions of \LaTeX{}
version 2005/12/01 or later.

This work has the LPPL maintenance status `maintained'.

The Current Maintainer of this work is Niklas Beisert.

This work consists of the files |README.txt|, |childdoc.ins| and |childdoc.dtx|
as well as the derived files |childdoc.def|, |cdocsamp.tex|
with |cdocsch1.tex|, |cdocsch2.tex|, |cdocspt3.tex|, |cdocspt4.tex|,
|cdocsdrf.tex|, |cdocsfn1.tex|, |cdocsfn2.tex|
as well as |childdoc.pdf|.

%%%%%%%%%%%%%%%%%%%%%%%%%%%%%%%%%%%%%%%%%%%%%%%%%%%%%%%%%%%%%%%%%%%%%%%%%%%%%%%%
\subsection{Files and Installation}

The package consists of the files:
%
\begin{center}
\begin{tabular}{ll}
    |README.txt|   & readme file \\
    |childdoc.ins| & installation file \\
    |childdoc.dtx| & source file \\
    |childdoc.def| & definition file \\
    |cdocsamp.tex| & sample main file \\
    |cdocsch1.tex| & sample include file \\
    |cdocsch2.tex| & sample include file \\
    |cdocspt3.tex| & sample part file \\
    |cdocspt4.tex| & sample part file \\
    |cdocsdrf.tex| & sample redirection file \\
    |cdocsfn1.tex| & sample redirection file \\
    |cdocsfn2.tex| & sample redirection file \\
    |childdoc.pdf| & manual
\end{tabular}
\end{center}
%
The distribution consists of the files
|README.txt|, |childdoc.ins| and |childdoc.dtx|.
%
\begin{itemize}
\item
Run (pdf)\LaTeX{} on |childdoc.dtx|
to compile the manual |childdoc.pdf| (this file).
\item
Run \LaTeX{} on |childdoc.ins| to create the definitions file |childdoc.def|
and the sample |cdocsamp.tex| with include files
|cdocsch1.tex|, |cdocsch2.tex|, |cdocspt3.tex|, |cdocspt4.tex|,
|cdocsdrf.tex|, |cdocsfn1.tex|, |cdocsfn2.tex|.
Then copy the file |childdoc.def| to an appropriate directory of your \LaTeX{}
distribution, e.g.\ \textit{texmf-root}|/tex/latex/childdoc|.
\end{itemize}

%%%%%%%%%%%%%%%%%%%%%%%%%%%%%%%%%%%%%%%%%%%%%%%%%%%%%%%%%%%%%%%%%%%%%%%%%%%%%%%%
\subsection{Related CTAN Packages}

There are several other packages which offer a similar functionality:
%
\begin{itemize}
\item
The packages
\href{http://ctan.org/pkg/docmute}{\textsf{docmute}},
\href{http://ctan.org/pkg/includex}{\textsf{includex}} and
\href{http://ctan.org/pkg/standalone}{\textsf{standalone}}
provide commands to include only the document body of
a child file thus allowing both files to be compiled individually.
\item
The packages \href{http://ctan.org/pkg/subdocs}{\textsf{subdocs}}
and \href{http://ctan.org/pkg/subfiles}{\textsf{subfiles}}
provide structures in which the main and child documents can be
encapsulated and allowing them to be compiled individually.
The inclusion mechanism is different from the conventional |\include|.
\item
The package \href{http://ctan.org/pkg/combine}{\textsf{combine}}
is an elaborate solution to combine several documents into one.
\end{itemize}
%
See also the CTAN topic \href{http://ctan.org/topic/subdocs}{\textsf{subdocs}}
for further related packages.
The present package differs from the above solutions in that
a document structure constructed with the conventional |\include| mechanism
just needs two extra commands at the top of every file
such that all constituent files can be compiled individually.

%%%%%%%%%%%%%%%%%%%%%%%%%%%%%%%%%%%%%%%%%%%%%%%%%%%%%%%%%%%%%%%%%%%%%%%%%%%%%%%%
%\subsection{Feature Suggestions}
%
%The following is a list of features which may be useful for future
%versions of this package:
%%
%\begin{itemize}
%\item
%\ldots
%\end{itemize}

%%%%%%%%%%%%%%%%%%%%%%%%%%%%%%%%%%%%%%%%%%%%%%%%%%%%%%%%%%%%%%%%%%%%%%%%%%%%%%%%
\subsection{Revision History}

%%%%%%%%%%%%%%%%%%%%%%%%%%%%%%%%%%%%%%%%
\paragraph{v2.0:} 2018/12/30

\begin{itemize}
\item
immediate forward processing
\item
added |\childdocby| mechanism
\item
manual restructured
\end{itemize}

%%%%%%%%%%%%%%%%%%%%%%%%%%%%%%%%%%%%%%%%
\paragraph{v1.6:} 2018/01/17

\begin{itemize}
\item
application for development of include files
\item
corrections to manual
\end{itemize}

%%%%%%%%%%%%%%%%%%%%%%%%%%%%%%%%%%%%%%%%
\paragraph{v1.5:} 2017/05/21

\begin{itemize}
\item
more complete structuring introduced
\item
|\childdocof| introduced
\item
|\childdoc| renamed to |\childdocmain|
\item
|\childredirect| renamed to |\childdocforward| and |\childdocforwardprefix|
and functionality expanded
\end{itemize}

%%%%%%%%%%%%%%%%%%%%%%%%%%%%%%%%%%%%%%%%
\paragraph{v1.0:} 2017/04/27

\begin{itemize}
\item
manual and install package
\item
first version published on CTAN
\end{itemize}

%%%%%%%%%%%%%%%%%%%%%%%%%%%%%%%%%%%%%%%%
\paragraph{v0.6:} 2017/04/26

\begin{itemize}
\item
redirection mechanism added
\end{itemize}

%%%%%%%%%%%%%%%%%%%%%%%%%%%%%%%%%%%%%%%%
\paragraph{v0.5:} 2017/04/26

\begin{itemize}
\item
functionality in definition file
\end{itemize}


%%%%%%%%%%%%%%%%%%%%%%%%%%%%%%%%%%%%%%%%%%%%%%%%%%%%%%%%%%%%%%%%%%%%%%%%%%%%%%%%
%%%%%%%%%%%%%%%%%%%%%%%%%%%%%%%%%%%%%%%%%%%%%%%%%%%%%%%%%%%%%%%%%%%%%%%%%%%%%%%%
%%%%%%%%%%%%%%%%%%%%%%%%%%%%%%%%%%%%%%%%%%%%%%%%%%%%%%%%%%%%%%%%%%%%%%%%%%%%%%%%
\appendix

\settowidth\MacroIndent{\rmfamily\scriptsize 000\ }

 \DocInput{childdoc.dtx}

\end{document}
%</driver>
% \fi
%
% %%%%%%%%%%%%%%%%%%%%%%%%%%%%%%%%%%%%%%%%%%%%%%%%%%%%%%%%%%%%%%%%%%%%%%%%%%%%%%
% %%%%%%%%%%%%%%%%%%%%%%%%%%%%%%%%%%%%%%%%%%%%%%%%%%%%%%%%%%%%%%%%%%%%%%%%%%%%%%
% \section{Sample}
%\iffalse
%<*samplemain>
%\fi
%
% The following presents a sample document
% with two chapters, two parts, a title page,
% a compile flag as well as three forwarding files to set the flag.
% It consists of eight |.tex| files:
% \begin{center}
% \begin{tabular}{ll}
% |cdocsamp.tex|&main file\\
% |cdocsch1.tex|&include file for chapter 1\\
% |cdocsch2.tex|&include file for chapter 2\\
% |cdocspt3.tex|&include file for part 3\\
% |cdocspt4.tex|&include file for part 4\\
% |cdocsdrf.tex|&forwarding file for main file in draft mode\\
% |cdocsfi1.tex|&forwarding file for final version of chapter 1\\
% |cdocsfi2.tex|&forwarding file for final version of chapter 2\\
% \end{tabular}
% \end{center}
% Each of the eight files can be compiled directly by the \LaTeX{} compiler.
%
% %%%%%%%%%%%%%%%%%%%%%%%%%%%%%%%%%%%%%%
% \paragraph{Main File.}
%
% The main file is called |cdocsamp.tex|.
%
% Load the \textsf{childdoc} definitions and
% declare the filename for the main document:
%    \begin{macrocode}
\input{childdoc.def}
\childdocmain{}
%    \end{macrocode}

% Optional override for |\version| flag:
%    \begin{macrocode}
%%\ifchilddoc\else\providecommand{\version}{draft}\fi
%    \end{macrocode}

% Define the default values for the |\version| flag
% (|final| for the main file and |draft| for childs):
%    \begin{macrocode}
\ifchilddoc
\providecommand{\version}{draft}
\else
\providecommand{\version}{final}
\fi
%    \end{macrocode}

% Load the standard document class:
%    \begin{macrocode}
\documentclass[12pt]{article}
%    \end{macrocode}

% Start the document body:
%    \begin{macrocode}
\begin{document}
%    \end{macrocode}

% Declare a title page.
% Print title, part of document being processed and version flag:
%    \begin{macrocode}
\addtocounter{page}{-1}
\begin{center}
{\LARGE\bfseries{}childdoc example\par}
\vspace{1cm}
\ifchilddoc
\ifchilddocmanual part\else chapter\fi:
`\childdocname' of `\childdocjob'\par
\else
main document: `\childdocjob'\par
\fi
version: \version\par
\end{center}
\newpage
%    \end{macrocode}

% Manually include selected file,
% otherwise process as usual:
%    \begin{macrocode}
\ifchilddocmanual
\section*{part `\childdocname'}
\input{\childdocname}
\else
%    \end{macrocode}

% Include the two chapters:
%    \begin{macrocode}
\include{cdocsch1}
\include{cdocsch2}
%    \end{macrocode}

% Include the two parts unless only chapters should be displayed:
%    \begin{macrocode}
\ifchilddoc\else
\section{part three}
\input{cdocspt3}
\section{part four}
\input{cdocspt4}
\fi
%    \end{macrocode}

% Process as usual until here:
%    \begin{macrocode}
\fi
%    \end{macrocode}

% End of document body:
%    \begin{macrocode}
\end{document}
%    \end{macrocode}
%\iffalse
%</samplemain>
%\fi
%
% %%%%%%%%%%%%%%%%%%%%%%%%%%%%%%%%%%%%%%
% \paragraph{Chapter Include Files.}
%
% The include files are called |cdocsch1.tex| and |cdocsch2.tex|.
%
%\iffalse
%<*samplechap1|samplechap2>
%\fi

% Optional override for |\version| flag:
%    \begin{macrocode}
%%\providecommand{\version}{final}
%    \end{macrocode}

% Include the main document:
%    \begin{macrocode}
\input{childdoc.def}
\childdocof{cdocsamp}
%    \end{macrocode}

%\iffalse
%</samplechap1|samplechap2>
%\fi
%
%\iffalse
%<*samplechap1>
%\fi
% Some text for chapter 1:
%    \begin{macrocode}
\section{one}
some text in chapter one
%    \end{macrocode}

%\iffalse
%</samplechap1>
%\fi
% Some text for chapter 2:
%\iffalse
%<*samplechap2>
%\fi
%    \begin{macrocode}
\section{two}
more text in chapter two
%    \end{macrocode}

%\iffalse
%</samplechap2>
%\fi
%
% %%%%%%%%%%%%%%%%%%%%%%%%%%%%%%%%%%%%%%
% \paragraph{Part Include Files.}
%
% The include files are called |cdocspt3.tex| and |cdocspt4.tex|.
%
%\iffalse
%<*samplepart3|samplepart4>
%\fi

% Optional override for |\version| flag:
%    \begin{macrocode}
%%\providecommand{\version}{final}
%    \end{macrocode}

% Include the main document:
%    \begin{macrocode}
\input{childdoc.def}
\childdocby{cdocsamp}
%    \end{macrocode}

%\iffalse
%</samplepart3|samplepart4>
%\fi
%
%\iffalse
%<*samplepart3>
%\fi
% Some text for part 3:
%    \begin{macrocode}
some text in part three
%    \end{macrocode}

%\iffalse
%</samplepart3>
%\fi
% Some text for part 4:
%\iffalse
%<*samplepart4>
%\fi
%    \begin{macrocode}
more text in part four
%    \end{macrocode}

%\iffalse
%</samplepart4>
%\fi
%
% %%%%%%%%%%%%%%%%%%%%%%%%%%%%%%%%%%%%%%
% \paragraph{Forwarding for a Complete Draft.}
%
% The following forwarding file |cdocsdrf.tex|
% compiles the main document in draft mode:
%\iffalse
%<*sampledraft>
%\fi
%    \begin{macrocode}
\def\version{draft}
\input{childdoc.def}
\childdocforward{cdocsamp}
%    \end{macrocode}

%\iffalse
%</sampledraft>
%\fi
%
% %%%%%%%%%%%%%%%%%%%%%%%%%%%%%%%%%%%%%%
% \paragraph{Forwarding for Final Version of the Chapters.}
%
% The following forwarding files |cdocsfn1.tex| and |cdocsfn2.tex|
% (with identical content)
% compile the final versions of the child documents
% |cdocsch1.tex| and |cdocsch2.tex|, respectively:
%\iffalse
%<*samplefinal>
%\fi
%    \begin{macrocode}
\def\version{final}
\input{childdoc.def}
\childdocforwardprefix[cdocsamp]{cdocsfn}{cdocsch}
%    \end{macrocode}

%\iffalse
%</samplefinal>
%\fi
%
% %%%%%%%%%%%%%%%%%%%%%%%%%%%%%%%%%%%%%%
% \paragraph{Command Line Processing.}
%
% The following three command lines generate the output files
% |cdocscld|, |cdocscl1| and |cdocscl2|
% which should be identical to
% |cdocsdrf|, |cdocsch1| and |cdocsfn2|, respectively:
% \begin{center}
% \begin{tabular}{l}
% |latex -jobname cdocscld \|\\
% |  "\def\version{draft}\input{childdoc.def}\childdocforward{cdocsamp}"|\\
% |latex -jobname cdocscl1 \|\\
% |  "\input{childdoc.def}\childdocforward[cdocsamp]{cdocsch1}"|\\
% |latex -jobname cdocscl2 \|\\
% |  "\def\version{final}\input{childdoc.def}\childdocforward{cdocsch2}"|
% \end{tabular}
% \end{center}
% Note that the trailing backslash on each first line
% merely continues the input to the second line
% (for convenient cut ant paste).
% Furthermore, the command |latex| can be replaced by any
% of its alternative versions such as |pdflatex|.
%
% %%%%%%%%%%%%%%%%%%%%%%%%%%%%%%%%%%%%%%%%%%%%%%%%%%%%%%%%%%%%%%%%%%%%%%%%%%%%%%
% %%%%%%%%%%%%%%%%%%%%%%%%%%%%%%%%%%%%%%%%%%%%%%%%%%%%%%%%%%%%%%%%%%%%%%%%%%%%%%
% \section{Implementation}
%\iffalse
%<*package>
%\fi
%
% This section describes the definitions file |childdoc.def|.

% The definitions cannot be loaded using |\usepackage| or |\RequirePackage|
% which has a mechanism to prevent loading a style file more than once.
% When loading the definitions by means of |\input|
% multiple instances have to be prevented manually:
%\iffalse
%This code needs to be before the `\ProvidesFile' directive
%which is defined at the beginning of this file.
%Therefore it is also placed there and commented out here.
%</package>
%<*discard>
%\fi
%    \begin{macrocode}
\ifdefined\childdocmain\endinput\fi
%    \end{macrocode}
%\iffalse
%</discard>
%<*package>
%\fi
%
% \macro{\ifchilddoc}
% \macro{\ifchilddocmanual}
% The conditional |\ifchilddoc| tells whether a
% child (true) or main (false) document is being compiled.
% The conditional |\ifchilddocmanual| tells whether
% the |\includeonly| mechanism is used (false) or
% the selection of child files must be performed manually (true).
% The definitions initialise to false:
%    \begin{macrocode}
\newif\ifchilddoc
\newif\ifchilddocmanual
%    \end{macrocode}

% \macro{\childdocname}
% \macro{\childdocjob}
% The macro |\childdocname| stores the name of the main document
% to be compiled. The macro |\childdocjob| stores the name of
% the document on which the \LaTeX{} compiler was originally invoked.
% The content of |\jobname| cannot be compared
% to filenames specified in the source due to different catcodes.
% The following code rescans |\jobname|, stores the result
% in |\childdocname| and saves a copy in |\childdocjob|:
%    \begin{macrocode}
\edef\childdocname{\scantokens\expandafter{\jobname\noexpand}}
\let\childdocjob\childdocname
%    \end{macrocode}

% \macro{\childdocdisable}
% The macro |\childdocdisable| prevents the main file
% from being processed more than once.
% At this stage, the main document command |\childdocmain|
% is assumed to be called once again where it should do nothing.
% Any subsequent call to it should prevent
% a secondary processing of the main document
% It overwrites the forwarding commands
% |\childdocof| and |\childdocforward|
% with empty macros to prevent further inclusions of the main document:
%    \begin{macrocode}
\newcommand{\childdocdisable}
{
  \renewcommand{\childdocmain}[1]{\renewcommand{\childdocmain}[1]{\endinput}}
  \renewcommand{\childdocof}[1]{}
  \renewcommand{\childdocby}[2][]{}
  \renewcommand{\childdocforward}[2][]{}
  \renewcommand{\childdocdisable}{}
}
%    \end{macrocode}

% \macro{\childdocmain}
% The macro |\childdocmain| is to be called at the top of the main file
% with nothing or the main filename (without extension) as argument.
% First, it breaks loops.
% If the argument is not empty and does not match |\childdocname|
% (which is set by the first inclusion of |childdoc.def|),
% |\ifchilddoc| is set to true, |\includeonly| is applied to the child file
% and |\jobname| is set to the main file
% (for proper handling of |.aux| files):
%    \begin{macrocode}
\newcommand{\childdocmain}[1]
{
  \childdocdisable\childdocmain{}
  \if?#1?\else
    \begingroup
      \def\childdoctmp{#1}
      \ifx\childdoctmp\childdocname
        \def\childdoctmp{}
      \else
        \def\childdoctmp
        {
          \childdoctrue
          \includeonly{\childdocname}
          \def\childdocjob{#1}
          \def\jobname{#1}
        }
      \fi
      \expandafter
    \endgroup
    \childdoctmp
  \fi
}
%    \end{macrocode}

% \macro{\childdocof}
% The command |\childdocof| redirects
% compilation to the main file |#1|.
%    \begin{macrocode}
\newcommand{\childdocof}[1]
{
  \childdocdisable
  \childdoctrue
  \includeonly{\childdocname}
  \def\jobname{#1}
  \def\childdocjob{#1}
  \input{#1}
}
%    \end{macrocode}

% \macro{\childdocby}
% The command |\childdocby| ....
%    \begin{macrocode}
\newcommand{\childdocby}[2][]
{
  \childdocdisable
  \childdoctrue
  \childdocmanualtrue
  \if?#1?\else
    \def\jobname{#2}
  \fi
  \def\childdocjob{#2}
  \input{#2}
  \endinput
}
%    \end{macrocode}

% \macro{\childdocforward}
% The command |\childdocforward| redirects
% compilation to the main file or
% (if the optional argument is given) a child file.
% Parameters are set as if the main file
% or a child file starting with |\childdocof| was compiled.
% Then compilation is handed over to the main file:
%    \begin{macrocode}
\newcommand{\childdocforward}[2][]
{
  \begingroup
    \if?#1?
      \def\childdoctmp
      {
        \def\childdocname{#2}
        \def\childdocjob{#2}
        \def\jobname{#2}
        \input{#2}
        \endinput
      }
    \else
      \def\childdoctmp
      {
        \childdocdisable
        \def\childdocname{#2}
        \childdoctrue
        \includeonly{#2}
        \def\childdocjob{#1}
        \def\jobname{#1}
        \input{#1}
        \endinput
      }
    \fi
    \expandafter
  \endgroup
  \childdoctmp
}
%    \end{macrocode}

% \macro{\childdocforwardprefix}
% The command |\childdocforwardprefix| redirects
% compilation to the main or a child file by means of a pattern.
% The prefix |#1| in the current filename is replaced by |#2|
% and the suffix of the current filename is kept
% (it is assumed that the filename does not contain the substring `|~~~|'
% which is used as a delimiter).
% Compilation is handed over to the new file by |\childdocforward|:
%    \begin{macrocode}
\newcommand{\childdocforwardprefix}[3][]
{
  \begingroup
    \def\childdocextract #2##1~~~{\def\childdoctmp{\childdocforward[#1]{#3##1}}}
    \expandafter\childdocextract\childdocname~~~
    \expandafter
  \endgroup
  \childdoctmp
}
%    \end{macrocode}

% \macro{\childdoc}
% The deprecated macro |\childdoc| is a legacy version of |\childdocmain|:
%    \begin{macrocode}
\newcommand{\childdoc}{\childdocmain}
%    \end{macrocode}

% \macro{\childdocredirect}
% The deprecated macro |\childdocredirect| is a legacy version
% of |\childdocforward| and |\childdocforwardprefix|:
%    \begin{macrocode}
\newcommand{\childdocredirect}[2][]
{
  \begingroup
    \if?#1?
      \def\childdoctmp{\childdocforward{#2}}
    \else
      \def\childdoctmp{\childdocforwardprefix{#1}{#2}}
    \fi
    \expandafter
  \endgroup
  \childdoctmp
}
%    \end{macrocode}

%\iffalse
%</package>
%\fi
%
\endinput
\childdocforward[|\textit{main}|]{|\textit{dest}|}"|
\end{center}
%
Here \textit{target} is the name of the output file,
\textit{main} is the name of the main file
and \textit{dest} is the name of the main or child file to be processed
(all filenames without extensions).
The optional argument \textit{main} can be omitted
if \textit{main} matches \textit{dest}.
Optionally, compilation \textit{flags} can be defined via |\def| commands.
This command line makes the \TeX{} engine believe
it is compiling the file \textit{target}
whose content is specified as the latter parameter.
The provided code then forwards the processing to
\textit{main} or \textit{dest} as described in \secref{sec:forward}.

%%%%%%%%%%%%%%%%%%%%%%%%%%%%%%%%%%%%%%%%%%%%%%%%%%%%%%%%%%%%%%%%%%%%%%%%%%%%%%%%
\subsection{Include by Input}
\label{sec:input}

Including child documents by |\include| has some restrictions by design.
Most notably, the content of a child document always occupies
its own set of pages; pages cannot be shared between child documents.
Usually, this behaviour makes perfect sense
because each child document contain an essential part of the document.
However, in some situations it may be desirable to compose
a document from a collection of parts
without having mandatory page breaks between then.
For this case, the package
provides a mechanism to include parts
by |\input| which can also be processed individually.
However, by construction this mechanism
requires manual handling of the content to be output.

%%%%%%%%%%%%%%%%%%%%%%%%%%%%%%%%%%%%%%%%
\DescribeMacro{\ifchilddocmanual}
The main file should be prepared as usual, see \secref{sec:include}.
However, the document body must make a distinction
between processing of an individual part and of the main document, e.g.:
%
\begin{center}
\begin{tabular}{l}
|\ifchilddocmanual|\\
|\input{\childdocname}|\\
|\||else|\\
\textit{document body with }|\input{|\textit{part}|}|\\
|\||fi|
\end{tabular}
\end{center}
%
The conditional |\ifchilddocmanual| is true whenever
a part to be included by |\input| is being compiled,
and the name of the part is stored in |\childdocname|.

%%%%%%%%%%%%%%%%%%%%%%%%%%%%%%%%%%%%%%%%
\DescribeMacro{\childdocby}
Each part to be included by |\input| should start with:
%
\begin{center}
\begin{tabular}{l}
|% \iffalse
%
% childdoc.dtx Copyright (C) 2017-2018 Niklas Beisert
%
% This work may be distributed and/or modified under the
% conditions of the LaTeX Project Public License, either version 1.3
% of this license or (at your option) any later version.
% The latest version of this license is in
%   http://www.latex-project.org/lppl.txt
% and version 1.3 or later is part of all distributions of LaTeX
% version 2005/12/01 or later.
%
% This work has the LPPL maintenance status `maintained'.
%
% The Current Maintainer of this work is Niklas Beisert.
%
% This work consists of the files childdoc.dtx and childdoc.ins
% and the derived files childdoc.def and cdocsamp.tex with
% cdocsch1.tex, cdocsch2.tex, cdocsdrf.tex, cdocsfn1.tex, cdocsfn2.tex.
%
%<package>\ifdefined\childdocmain\endinput\fi
%<package>\ProvidesFile{childdoc.def}[2018/12/30 v2.0 child document driver]
%<samplemain>\ProvidesFile{cdocsamp.tex}[2018/12/30 v2.0 sample for childdoc]
%<*driver>
%\ProvidesFile{childdoc.drv}[2018/12/30 v2.0 childdoc reference manual file]
\PassOptionsToClass{10pt,a4paper}{article}
\documentclass{ltxdoc}

\usepackage[margin=35mm]{geometry}
\usepackage{hyperref}
\usepackage{hyperxmp}
\usepackage[usenames]{color}

\hypersetup{colorlinks=true}
\hypersetup{pdfstartview=FitH}
\hypersetup{pdfpagemode=UseNone}
\hypersetup{pdfsource={}}
\hypersetup{pdflang={en-UK}}
\hypersetup{pdfcopyright={Copyright 2017-2018 Niklas Beisert.
  This work may be distributed and/or modified under the
  conditions of the LaTeX Project Public License, either version 1.3
  of this license or (at your option) any later version.}}
\hypersetup{pdflicenseurl={http://www.latex-project.org/lppl.txt}}
\hypersetup{pdfcontactaddress={ETH Zurich, ITP, HIT K,
  Wolfgang-Pauli-Strasse 27}}
\hypersetup{pdfcontactpostcode={8093}}
\hypersetup{pdfcontactcity={Zurich}}
\hypersetup{pdfcontactcountry={Switzerland}}
\hypersetup{pdfcontactemail={nbeisert@itp.phys.ethz.ch}}
\hypersetup{pdfcontacturl={http://people.phys.ethz.ch/\xmptilde nbeisert/}}

\newcommand{\secref}[1]{\hyperref[#1]{section \ref*{#1}}}

\parskip1ex
\parindent0pt
\let\olditemize\itemize
\def\itemize{\olditemize\parskip0pt}

\begin{document}

\title{The \textsf{childdoc} Package}
\hypersetup{pdftitle={The childdoc Package}}
\author{Niklas Beisert\\[2ex]
  Institut f\"ur Theoretische Physik\\
  Eidgen\"ossische Technische Hochschule Z\"urich\\
  Wolfgang-Pauli-Strasse 27, 8093 Z\"urich, Switzerland\\[1ex]
  \href{mailto:nbeisert@itp.phys.ethz.ch}
  {\texttt{nbeisert@itp.phys.ethz.ch}}}
\hypersetup{pdfauthor={Niklas Beisert}}
\hypersetup{pdfsubject={Manual for the LaTeX2e Package childdoc}}
\date{30 December 2018, \textsf{v2.0}}
\maketitle

\begin{abstract}\noindent
\textsf{childdoc} is a \LaTeXe{} package
that enables the direct compilation
of document sections included by |\include|
to individual files.
\end{abstract}

\begingroup
\parskip0ex
\tableofcontents
\endgroup

%%%%%%%%%%%%%%%%%%%%%%%%%%%%%%%%%%%%%%%%%%%%%%%%%%%%%%%%%%%%%%%%%%%%%%%%%%%%%%%%
%%%%%%%%%%%%%%%%%%%%%%%%%%%%%%%%%%%%%%%%%%%%%%%%%%%%%%%%%%%%%%%%%%%%%%%%%%%%%%%%
\section{Introduction}

\LaTeX{} provides a mechanism to structure a large document (such as a book)
into a main file and several child files (containing the chapters)
using the |\include| command.
This mechanism is beneficial for documents
which span hundreds of pages in order to
make the source file(s) more manageable.
Moreover, compilation can be restricted to
selected child files by means of the |\includeonly| command.
The latter feature can be used to reduce the compilation time while editing
(this was significantly more useful in the earlier days of \LaTeX{})
or to generate a smaller document which is easier to navigate.
Another application of |\includeonly| is to generate
documents consisting of selected parts of the complete document.

However, there are a few drawbacks of the plain |\include| mechanism:
\begin{itemize}
\item
The child files cannot be compiled on their own,
they can only be compiled via the main file.
A naive editing environment
(such as a text editor with an option
to have the current file processed by \LaTeX)
may require one to switch to the main file before compiling;
attempting to compile the child file produces errors.
\item
The main file must be modified (each time)
to adjust the |\includeonly| command
to the present needs. This easily leaves the main file in a messy state.
\item
The generated document will always carry the filename
of the main document. This is inconvenient if
several child files are to be compiled and
to be kept for distribution.
\end{itemize}

The present package provides a simple interface
to make child files individually compilable by \LaTeX{}.
Compiling a child file then has the same effect as compiling
the main file with an |\includeonly| command
to select the appropriate child.
Moreover the generated document will carry the name of the child
rather than the main file.
This resolves all three above issues.

This feature is meant to make the editing of books,
thesis documents and lecture notes somewhat more convenient.
However, the package can also be used efficiently for
composing a series of documents (such as exercise sheets)
which are typically distributed individually.
It then assists the author in generating the individual documents
(potentially in different versions)
as well as a document containing the collected series.
Another application is in developing style files
or other kinds of included material
where compilation of the style file could redirect
to a sample or test file.

%%%%%%%%%%%%%%%%%%%%%%%%%%%%%%%%%%%%%%%%%%%%%%%%%%%%%%%%%%%%%%%%%%%%%%%%%%%%%%%%
%%%%%%%%%%%%%%%%%%%%%%%%%%%%%%%%%%%%%%%%%%%%%%%%%%%%%%%%%%%%%%%%%%%%%%%%%%%%%%%%
\section{Usage}

First of all, the package \textsf{childdoc} is \emph{not} a standard
\LaTeXe{} |.sty| style file! Therefore it needs to be invoked in
a non-standard way.

%%%%%%%%%%%%%%%%%%%%%%%%%%%%%%%%%%%%%%%%%%%%%%%%%%%%%%%%%%%%%%%%%%%%%%%%%%%%%%%%
\subsection{Included Files}
\label{sec:include}

%%%%%%%%%%%%%%%%%%%%%%%%%%%%%%%%%%%%%%%%
\DescribeMacro{\childdocmain}
To use the package, add the commands
\begin{center}
\begin{tabular}{l}
|\input{childdoc.def}|\\
|\childdocmain{}|\\
\end{tabular}
\end{center}
at the very top of the main \LaTeX{} file,
in particular \emph{before} the |\documentclass| statement!
The argument of |\childdocmain| should be left empty
(but it must be present).

%%%%%%%%%%%%%%%%%%%%%%%%%%%%%%%%%%%%%%%%
\DescribeMacro{\childdocof}
Furthermore, add the commands
\begin{center}
\begin{tabular}{l}
|\input{childdoc.def}|\\
|\childdocof{|\textit{main}|}|\\
\end{tabular}
\end{center}
at the top of every child file \textit{child}
which is included by |\include{|\textit{child}|}|
from within the main file
(or at least for those files to be compiled individually).
The argument \textit{main} must be the filename of the main file.

There are a couple of
considerations in setting up the main and child documents:

%%%%%%%%%%%%%%%%%%%%%%%%%%%%%%%%%%%%%%%%
\paragraph{Restrictions.}

Please note the following restrictions:
\begin{itemize}
\item
|\childdocmain| must be called with one argument \textit{main}
to ensure compatibility with earlier version of the package.
It must either be empty (|\childdocmain{}|)
or precisely match the filename of the main file in which it is specified.
See \secref{sec:detection} for further information.
\item
The filename \textit{main} must be specified without the |.tex| extension.
\item
The filename \textit{main} is case sensitive
(even in case-insensitive file systems)
due to internal string comparison.
\item
The argument \textit{main} should be fully expanded, it cannot be a macro.
\item
Subdirectories and special characters should be avoided in filenames.
\item
The command |\childdocmain{|\textit{main}|}| must be followed by a whitespace.
It should not be followed immediately by another command
or by a comment mark `|%|'.
This is because the \TeX{} parser reads the token immediately following
the argument of |\childdocmain| and puts it
at the beginning of every child section;
however, a white\-space is ignored.
\end{itemize}

%%%%%%%%%%%%%%%%%%%%%%%%%%%%%%%%%%%%%%%%
\paragraph{Content of Main File.}

It is advisable to place all content in the child files included by |\include|.
Any output contained in the main file will appear in all child documents
unless suppressed manually;
it cannot be suppressed automatically by the |\includeonly| directive
and thus should normally be avoided.
A method to include some content in the main file
by means of conditional processing is described in \secref{sec:conditional}.

%%%%%%%%%%%%%%%%%%%%%%%%%%%%%%%%%%%%%%%%
\paragraph{Page Numbering.}

When only a part of the document is compiled,
the appropriate numbering of pages
(as well as other status parameters)
is determined from the |.aux| files.
The latter contain information from previous passes.
However this information needs to propagate through
all intermediate child documents.
Therefore the page numbering in child documents may well
be inconsistent until the complete document is compiled at least once.

A useful (if unconventional) way to always ensure a consistent
page numbering is to restart the numbering in each child document
and denote the pages by `\textit{child}|.|\textit{page}'
where \textit{child} represents the chapter/section number of the child file.
This can be achieved by the command
|\numberwithin{page}{|\textit{child}|}|
of the \textsf{amsmath} package
where \textit{child} can be |chapter| or |section|
depending on the chosen structuring.
Alternatively, one can modify the macro |\thepage| appropriately
and reset the counter |page| at the start of each child file.

%%%%%%%%%%%%%%%%%%%%%%%%%%%%%%%%%%%%%%%%%%%%%%%%%%%%%%%%%%%%%%%%%%%%%%%%%%%%%%%%
\subsection{Conditional Processing}
\label{sec:conditional}

The package provides a mechanism to compile different versions
of a document. To customise the versions further some conditional processing
can come in handy to distinguish which version is being compiled.
The package provides two macros to describe the compilation context:

%%%%%%%%%%%%%%%%%%%%%%%%%%%%%%%%%%%%%%%%
\DescribeMacro{\ifchilddoc}
The conditional |\ifchilddoc| distinguishes between the compilation of
child documents and the main document:
%
\begin{center}
|\ifchilddoc |\textit{child-code}| |[|\||else |\textit{main-code}]| \||fi|
\end{center}

%%%%%%%%%%%%%%%%%%%%%%%%%%%%%%%%%%%%%%%%
\DescribeMacro{\childdocname}
\DescribeMacro{\childdocjob}
The macro |\childdocname| contains the filename (without extension)
of the main or child file being processed.
Note that |\childdocjob| will always contain the name of the main file.

%%%%%%%%%%%%%%%%%%%%%%%%%%%%%%%%%%%%%%%%
\paragraph{Title Page.}

Conditional processing can be used to include a title or banner page
in the main document when proper precautions are taken.
Importantly, the code in the main file should ensure that the page counter
(as well as other status parameters which are stored in the |.aux| files)
takes the same value after the conditional processing.
Otherwise the page numbers may take divergent values
depending on which part is compiled.

For example, a title page could be declared by:
%
\begin{center}
\begin{tabular}{l}
|\ifchilddoc\||else|\\
|\addtocounter{page}{-1}|\\
\textit{code for title page}\\
|\newpage|\\
|\||fi|
\end{tabular}
\end{center}
%
A banner page for the child documents can be generated by:
%
\begin{center}
\begin{tabular}{l}
|\ifchilddoc|\\
|\addtocounter{page}{-1}|\\
\textit{code for banner page}\\
|\newpage|\\
|\||fi|
\end{tabular}
\end{center}
%
Here one could write a message such as:
\begin{center}
|This is the part \childdocname{} of \childdocjob{}.|
\end{center}

%%%%%%%%%%%%%%%%%%%%%%%%%%%%%%%%%%%%%%%%%%%%%%%%%%%%%%%%%%%%%%%%%%%%%%%%%%%%%%%%
\subsection{Flags}
\label{sec:flags}

The package makes it easy to generate different versions
of the main or child documents.
To this end compilation flags can be defined
and assigned different default values.
They will be particularly useful in conjunction
with the forwarding mechanism described in \secref{sec:forward}.

For example, it may be useful to have a flag |\version|
which can be set to |draft| or |final|.
The document source will contain some conditional code
depending on the value of |\version|.
Suppose further, the flag should default to |final| for the main file
and to |draft| for child files
which is a natural assignment for editing the document.
This is achieved by placing the following code
in the preamble of the main document
(below the |\childdocmain| directive):
%
\begin{center}
\begin{tabular}{l}
|\ifchilddoc|\\
|\providecommand{\version}{draft}|\\
|\||else|\\
|\providecommand{\version}{final}|\\
|\||fi|
\end{tabular}
\end{center}
%
The definition by |\providecommand| makes sure
that previous definitions are not overwritten.
Further statements |\providecommand{\version}{...}|
can thus be added before the above code to override it.

For the main file, one might add a line
(between |\childdocmain| and the above block)
%
\begin{center}
|%\ifchilddoc\||else\providecommand{\version}{draft}\||fi|
\end{center}
%
which can be uncommented to produce a draft version.
Likewise one can add a line to the very top of a child file
(above the |\childdocof{|\textit{main}|}| directive)
%
\begin{center}
|%\providecommand{\version}{final}|
\end{center}
%
which can be uncommented to produce the final version of this child document.

%%%%%%%%%%%%%%%%%%%%%%%%%%%%%%%%%%%%%%%%%%%%%%%%%%%%%%%%%%%%%%%%%%%%%%%%%%%%%%%%
\subsection{Forwarding}
\label{sec:forward}

Different versions of the main or child documents
using compilation flags as described in \secref{sec:flags}
can be (permanently) stored in different files
for convenient compilation, viewing and distribution.
To this end, the package defines a command
to pass on compilation to a different file:

%%%%%%%%%%%%%%%%%%%%%%%%%%%%%%%%%%%%%%%%
\DescribeMacro{\childdocforward}
The command |\childdocforward| redirects processing to
another source file:
%
\begin{center}
\begin{tabular}{l}
|\input{childdoc.def}|\\
|\childdocforward[|\textit{main}|]{|\textit{dest}|}|\\
\end{tabular}
\end{center}
%
The argument \textit{dest} is the destination file
(without extension).
It should be the main file or one of the child files.
Note that further \textsf{childdoc} directives
such as |\childdocof| and |\childdocforward|
in the indicated file will be processed in this form.
The optional argument \textit{main}
passes on directly to the main file \textit{main}
while pretending to compile the child \textit{dest}.
This form behaves as if \textit{dest}
issues |\childdocof{|\textit{main}|}| right away,
and no further \textsf{childdoc} directives will be processed.

%%%%%%%%%%%%%%%%%%%%%%%%%%%%%%%%%%%%%%%%
\DescribeMacro{\...prefix}
In the alternative form |\childdocforwardprefix|,
%
\begin{center}
\begin{tabular}{l}
|\input{childdoc.def}|\\
|\childdocforwardprefix[|\textit{main}|]{|\textit{prefix}|}{|\textit{dest}|}|
\end{tabular}
\end{center}
%
the destination file is determined by a pattern
depending on the current file:
To make this work, the current file must be called
`{\textit{prefix}\hspace{0.2em}\textit{suffix}}'
with \textit{prefix} matching precisely the argument.
Processing is then passed on to the file
`{\textit{dest}\hspace{0.2em}\textit{suffix}}'.
Surely, the same effect is achieved by
directly specifying the
argument `{\textit{dest}\hspace{0.2em}\textit{suffix}}'
in the first form.
However, that requires to set up a different file
for each child. With the alternative form of the command
all these files can have exactly the same content
which simplifies setting them up and maintaining them.

For example, the following file |draft.tex|
with a compilation flag |\version| as described in \secref{sec:flags}
compiles the main document as a draft:
%
\begin{center}
\begin{tabular}{l}
|\def\version{draft}|\\
|\input{childdoc.def}|\\
|\childdocforward{|\textit{main}|}|
\end{tabular}
\end{center}
%
Likewise, the following files |final|\textit{nn}|.tex|
compile the final version of the child document
|child|\textit{nn}|.tex|:
%
\begin{center}
\begin{tabular}{l}
|\def\version{final}|\\
|\input{childdoc.def}|\\
|\childdocforwardprefix{final}{child}|
\end{tabular}
\end{center}
%

Note that when several versions of a main file and/or of each child file
are to be generated, it may be convenient to set up a |Makefile| or
shell script to automatise the process.

%%%%%%%%%%%%%%%%%%%%%%%%%%%%%%%%%%%%%%%%%%%%%%%%%%%%%%%%%%%%%%%%%%%%%%%%%%%%%%%%
\subsection{Command Line Processing}
\label{sec:commandline}

The effect of redirection files can also be achieved by invoking
the \LaTeX{} compiler with a more elaborate command line.
Most conveniently this should be done as part
of a shell script or a |Makefile|.

When using \textsf{childdoc} in the main file, the following
command lines effectively perform a redirection
(note that depending on the shell being used,
backslashes may have to be doubled: `|\|' $\to$ `|\\|'):
%
\begin{center}
|... -jobname "|\textit{target}|" |\\|"|[\textit{flags}]%
|\input{childdoc.def}\childdocforward[|\textit{main}|]{|\textit{dest}|}"|
\end{center}
%
Here \textit{target} is the name of the output file,
\textit{main} is the name of the main file
and \textit{dest} is the name of the main or child file to be processed
(all filenames without extensions).
The optional argument \textit{main} can be omitted
if \textit{main} matches \textit{dest}.
Optionally, compilation \textit{flags} can be defined via |\def| commands.
This command line makes the \TeX{} engine believe
it is compiling the file \textit{target}
whose content is specified as the latter parameter.
The provided code then forwards the processing to
\textit{main} or \textit{dest} as described in \secref{sec:forward}.

%%%%%%%%%%%%%%%%%%%%%%%%%%%%%%%%%%%%%%%%%%%%%%%%%%%%%%%%%%%%%%%%%%%%%%%%%%%%%%%%
\subsection{Include by Input}
\label{sec:input}

Including child documents by |\include| has some restrictions by design.
Most notably, the content of a child document always occupies
its own set of pages; pages cannot be shared between child documents.
Usually, this behaviour makes perfect sense
because each child document contain an essential part of the document.
However, in some situations it may be desirable to compose
a document from a collection of parts
without having mandatory page breaks between then.
For this case, the package
provides a mechanism to include parts
by |\input| which can also be processed individually.
However, by construction this mechanism
requires manual handling of the content to be output.

%%%%%%%%%%%%%%%%%%%%%%%%%%%%%%%%%%%%%%%%
\DescribeMacro{\ifchilddocmanual}
The main file should be prepared as usual, see \secref{sec:include}.
However, the document body must make a distinction
between processing of an individual part and of the main document, e.g.:
%
\begin{center}
\begin{tabular}{l}
|\ifchilddocmanual|\\
|\input{\childdocname}|\\
|\||else|\\
\textit{document body with }|\input{|\textit{part}|}|\\
|\||fi|
\end{tabular}
\end{center}
%
The conditional |\ifchilddocmanual| is true whenever
a part to be included by |\input| is being compiled,
and the name of the part is stored in |\childdocname|.

%%%%%%%%%%%%%%%%%%%%%%%%%%%%%%%%%%%%%%%%
\DescribeMacro{\childdocby}
Each part to be included by |\input| should start with:
%
\begin{center}
\begin{tabular}{l}
|\input{childdoc.def}|\\
|\childdocby{|\textit{main}|}|\\
\end{tabular}
\end{center}
%
The directive |\childdocby| is similar to |\childdocof|
described in \secref{sec:include},
but the subsequent selection of content must be done manually.
To that end, both |\ifchilddoc| and |\ifchilddocmanual|
will be true upon processing of a part,
and the name of the part is stored in |\childdocname|.
Note that |\jobname| will be set to the filename of the current part
so that each part receives an individual |.aux| file
that does not interfere with the |.aux| file(s) of the main document.
This behaviour can be altered by the alternative form
|\childdocby[*]{|\textit{main}|}| (with a non-empty optional argument)
which uses the |.aux| file of the main document
by setting |\jobname| to \textit{main}.

%%%%%%%%%%%%%%%%%%%%%%%%%%%%%%%%%%%%%%%%%%%%%%%%%%%%%%%%%%%%%%%%%%%%%%%%%%%%%%%%
\subsection{Driver Development}
\label{sec:driver}

The \textsf{childdoc} mechanism can also be use for the development
of definition files such as \LaTeX{} styles or classes.
This case differs from the above setup with multiple parts
included by |\include| in that no |\includeonly| should be invoked.
This can be achieved by starting the include file
(before |\ProvidesPackage|) with:
%
\begin{center}
\begin{tabular}{l}
|\input{childdoc.def}|\\
|\childdocforward{|\textit{main}|}|\\
\end{tabular}
\end{center}
%
or alternatively with:
%
\begin{center}
\begin{tabular}{l}
|\input{childdoc.def}|\\
|\childdocby{|\textit{main}|}|\\
\end{tabular}
\end{center}
%
Both forms have slightly different effects as described above.
The main file is prepared as usual, see \secref{sec:include}.

%%%%%%%%%%%%%%%%%%%%%%%%%%%%%%%%%%%%%%%%%%%%%%%%%%%%%%%%%%%%%%%%%%%%%%%%%%%%%%%%
\subsection{Legacy Detection}
\label{sec:detection}

The directive |\childdocmain| in the main file can detect
whether the complete document or merely a child is to be compiled
even without using the directive |\childdocof|.
This method is deprecated because it is less robust
and there is no compelling reason to use it;
it is merely provided for backward compatibility
and it may be removed in future versions.

If the detection mechanism is to be used,
it is mandatory to correctly specify
the filename of the main file as the argument of |\childdocmain|:
%
\begin{center}
\begin{tabular}{l}
|\input{childdoc.def}|\\
|\childdocmain{|\textit{main}|}|\\
\end{tabular}
\end{center}
%
If |\jobname| does not match the argument \textit{main} of |\childdocmain|,
it is assumed that |\jobname| points to the child file to be compiled.
When using |\childdocmain| with the main file specified as argument,
it suffices to start a child file
with just |\input{|\textit{main}|}|
without loading of the package and using |\childdocof|.
If instead all processing is done
with the appropriate \textsf{childdoc} directives,
the argument of \textit{main} of |\childdocmain| can be empty.

An alternative version of the command line processing described
in \secref{sec:commandline} using the detection mechanism reads:
%
\begin{center}
|... -jobname "|\textit{target}|" "|[\textit{flags}]%
[|\def\jobname{|\textit{dest}|}|]|\input{|\textit{main}|}"|
\end{center}

%%%%%%%%%%%%%%%%%%%%%%%%%%%%%%%%%%%%%%%%%%%%%%%%%%%%%%%%%%%%%%%%%%%%%%%%%%%%%%%%
\subsection{Manual Code}
\label{sec:manual}

In case one cannot be certain whether the definitions file |childdoc.def|
is installed on the target \TeX{} distribution
and one prefers not to ship it,
it is conceivable to paste a few relevant commands into the sources.

To that end, drop all statements |\input{childdoc.def}|
and perform the replacements as outlined below.
Instead of |\childdocmain{|\textit{main}|}| add the following code
to the top of the main file:
%
\begin{center}
\begin{tabular}{l}
|\||ifdefined\childdocname\endinput\||fi\newif\ifchilddoc|\\
|\edef\childdocname{\scantokens\expandafter{\jobname\noexpand}}|\\
|\def\childdocmain{|\textit{main}|}\||ifx\childdocmain\childdocname\||else|\\
|\childdoctrue\includeonly{\childdocname}\let\jobname\childdocmain\||fi|\\
\end{tabular}
\end{center}
%
Instead of |\childdocof{|\textit{main}|}| just include the main file
at the top of each child file:
%
\begin{center}
|\input{|\textit{main}|}|
\end{center}
%
A simple redirection |\childdocforward{|\textit{dest}|}| is achieved by:
%
\begin{center}
|\def\jobname{|\textit{dest}|}\input{\jobname}|
\end{center}
%
The redirection with prefix
|\childdocforwardprefix[|\textit{prefix}|]{|\textit{dest}|}|
is accomplished by:
%
\begin{center}
\begin{tabular}{l}
|{\edef\jobname{\scantokens\expandafter{\jobname\noexpand}}|\\
|\def\redirectjob |\textit{prefix}|#1~~~{\gdef\jobname{|\textit{dest}|#1}}|\\
|\expandafter\redirectjob\jobname~~~}\input{\jobname}|
\end{tabular}
\end{center}

In an alternative approach,
child documents can be compiled by a specific command line
without additional code or specific definitions:
%
\begin{center}
|... -jobname "|\textit{target}|" "|[\textit{flags}]%
|\includeonly{|\textit{dest}|}\input{|\textit{main}|}"|
\end{center}
%

%%%%%%%%%%%%%%%%%%%%%%%%%%%%%%%%%%%%%%%%%%%%%%%%%%%%%%%%%%%%%%%%%%%%%%%%%%%%%%%%
%%%%%%%%%%%%%%%%%%%%%%%%%%%%%%%%%%%%%%%%%%%%%%%%%%%%%%%%%%%%%%%%%%%%%%%%%%%%%%%%
\section{Information}

%%%%%%%%%%%%%%%%%%%%%%%%%%%%%%%%%%%%%%%%%%%%%%%%%%%%%%%%%%%%%%%%%%%%%%%%%%%%%%%%
\subsection{Copyright}

Copyright \copyright{} 2017--2018 Niklas Beisert

This work may be distributed and/or modified under the
conditions of the \LaTeX{} Project Public License, either version 1.3
of this license or (at your option) any later version.
The latest version of this license is in
  \url{http://www.latex-project.org/lppl.txt}
and version 1.3 or later is part of all distributions of \LaTeX{}
version 2005/12/01 or later.

This work has the LPPL maintenance status `maintained'.

The Current Maintainer of this work is Niklas Beisert.

This work consists of the files |README.txt|, |childdoc.ins| and |childdoc.dtx|
as well as the derived files |childdoc.def|, |cdocsamp.tex|
with |cdocsch1.tex|, |cdocsch2.tex|, |cdocspt3.tex|, |cdocspt4.tex|,
|cdocsdrf.tex|, |cdocsfn1.tex|, |cdocsfn2.tex|
as well as |childdoc.pdf|.

%%%%%%%%%%%%%%%%%%%%%%%%%%%%%%%%%%%%%%%%%%%%%%%%%%%%%%%%%%%%%%%%%%%%%%%%%%%%%%%%
\subsection{Files and Installation}

The package consists of the files:
%
\begin{center}
\begin{tabular}{ll}
    |README.txt|   & readme file \\
    |childdoc.ins| & installation file \\
    |childdoc.dtx| & source file \\
    |childdoc.def| & definition file \\
    |cdocsamp.tex| & sample main file \\
    |cdocsch1.tex| & sample include file \\
    |cdocsch2.tex| & sample include file \\
    |cdocspt3.tex| & sample part file \\
    |cdocspt4.tex| & sample part file \\
    |cdocsdrf.tex| & sample redirection file \\
    |cdocsfn1.tex| & sample redirection file \\
    |cdocsfn2.tex| & sample redirection file \\
    |childdoc.pdf| & manual
\end{tabular}
\end{center}
%
The distribution consists of the files
|README.txt|, |childdoc.ins| and |childdoc.dtx|.
%
\begin{itemize}
\item
Run (pdf)\LaTeX{} on |childdoc.dtx|
to compile the manual |childdoc.pdf| (this file).
\item
Run \LaTeX{} on |childdoc.ins| to create the definitions file |childdoc.def|
and the sample |cdocsamp.tex| with include files
|cdocsch1.tex|, |cdocsch2.tex|, |cdocspt3.tex|, |cdocspt4.tex|,
|cdocsdrf.tex|, |cdocsfn1.tex|, |cdocsfn2.tex|.
Then copy the file |childdoc.def| to an appropriate directory of your \LaTeX{}
distribution, e.g.\ \textit{texmf-root}|/tex/latex/childdoc|.
\end{itemize}

%%%%%%%%%%%%%%%%%%%%%%%%%%%%%%%%%%%%%%%%%%%%%%%%%%%%%%%%%%%%%%%%%%%%%%%%%%%%%%%%
\subsection{Related CTAN Packages}

There are several other packages which offer a similar functionality:
%
\begin{itemize}
\item
The packages
\href{http://ctan.org/pkg/docmute}{\textsf{docmute}},
\href{http://ctan.org/pkg/includex}{\textsf{includex}} and
\href{http://ctan.org/pkg/standalone}{\textsf{standalone}}
provide commands to include only the document body of
a child file thus allowing both files to be compiled individually.
\item
The packages \href{http://ctan.org/pkg/subdocs}{\textsf{subdocs}}
and \href{http://ctan.org/pkg/subfiles}{\textsf{subfiles}}
provide structures in which the main and child documents can be
encapsulated and allowing them to be compiled individually.
The inclusion mechanism is different from the conventional |\include|.
\item
The package \href{http://ctan.org/pkg/combine}{\textsf{combine}}
is an elaborate solution to combine several documents into one.
\end{itemize}
%
See also the CTAN topic \href{http://ctan.org/topic/subdocs}{\textsf{subdocs}}
for further related packages.
The present package differs from the above solutions in that
a document structure constructed with the conventional |\include| mechanism
just needs two extra commands at the top of every file
such that all constituent files can be compiled individually.

%%%%%%%%%%%%%%%%%%%%%%%%%%%%%%%%%%%%%%%%%%%%%%%%%%%%%%%%%%%%%%%%%%%%%%%%%%%%%%%%
%\subsection{Feature Suggestions}
%
%The following is a list of features which may be useful for future
%versions of this package:
%%
%\begin{itemize}
%\item
%\ldots
%\end{itemize}

%%%%%%%%%%%%%%%%%%%%%%%%%%%%%%%%%%%%%%%%%%%%%%%%%%%%%%%%%%%%%%%%%%%%%%%%%%%%%%%%
\subsection{Revision History}

%%%%%%%%%%%%%%%%%%%%%%%%%%%%%%%%%%%%%%%%
\paragraph{v2.0:} 2018/12/30

\begin{itemize}
\item
immediate forward processing
\item
added |\childdocby| mechanism
\item
manual restructured
\end{itemize}

%%%%%%%%%%%%%%%%%%%%%%%%%%%%%%%%%%%%%%%%
\paragraph{v1.6:} 2018/01/17

\begin{itemize}
\item
application for development of include files
\item
corrections to manual
\end{itemize}

%%%%%%%%%%%%%%%%%%%%%%%%%%%%%%%%%%%%%%%%
\paragraph{v1.5:} 2017/05/21

\begin{itemize}
\item
more complete structuring introduced
\item
|\childdocof| introduced
\item
|\childdoc| renamed to |\childdocmain|
\item
|\childredirect| renamed to |\childdocforward| and |\childdocforwardprefix|
and functionality expanded
\end{itemize}

%%%%%%%%%%%%%%%%%%%%%%%%%%%%%%%%%%%%%%%%
\paragraph{v1.0:} 2017/04/27

\begin{itemize}
\item
manual and install package
\item
first version published on CTAN
\end{itemize}

%%%%%%%%%%%%%%%%%%%%%%%%%%%%%%%%%%%%%%%%
\paragraph{v0.6:} 2017/04/26

\begin{itemize}
\item
redirection mechanism added
\end{itemize}

%%%%%%%%%%%%%%%%%%%%%%%%%%%%%%%%%%%%%%%%
\paragraph{v0.5:} 2017/04/26

\begin{itemize}
\item
functionality in definition file
\end{itemize}


%%%%%%%%%%%%%%%%%%%%%%%%%%%%%%%%%%%%%%%%%%%%%%%%%%%%%%%%%%%%%%%%%%%%%%%%%%%%%%%%
%%%%%%%%%%%%%%%%%%%%%%%%%%%%%%%%%%%%%%%%%%%%%%%%%%%%%%%%%%%%%%%%%%%%%%%%%%%%%%%%
%%%%%%%%%%%%%%%%%%%%%%%%%%%%%%%%%%%%%%%%%%%%%%%%%%%%%%%%%%%%%%%%%%%%%%%%%%%%%%%%
\appendix

\settowidth\MacroIndent{\rmfamily\scriptsize 000\ }

 \DocInput{childdoc.dtx}

\end{document}
%</driver>
% \fi
%
% %%%%%%%%%%%%%%%%%%%%%%%%%%%%%%%%%%%%%%%%%%%%%%%%%%%%%%%%%%%%%%%%%%%%%%%%%%%%%%
% %%%%%%%%%%%%%%%%%%%%%%%%%%%%%%%%%%%%%%%%%%%%%%%%%%%%%%%%%%%%%%%%%%%%%%%%%%%%%%
% \section{Sample}
%\iffalse
%<*samplemain>
%\fi
%
% The following presents a sample document
% with two chapters, two parts, a title page,
% a compile flag as well as three forwarding files to set the flag.
% It consists of eight |.tex| files:
% \begin{center}
% \begin{tabular}{ll}
% |cdocsamp.tex|&main file\\
% |cdocsch1.tex|&include file for chapter 1\\
% |cdocsch2.tex|&include file for chapter 2\\
% |cdocspt3.tex|&include file for part 3\\
% |cdocspt4.tex|&include file for part 4\\
% |cdocsdrf.tex|&forwarding file for main file in draft mode\\
% |cdocsfi1.tex|&forwarding file for final version of chapter 1\\
% |cdocsfi2.tex|&forwarding file for final version of chapter 2\\
% \end{tabular}
% \end{center}
% Each of the eight files can be compiled directly by the \LaTeX{} compiler.
%
% %%%%%%%%%%%%%%%%%%%%%%%%%%%%%%%%%%%%%%
% \paragraph{Main File.}
%
% The main file is called |cdocsamp.tex|.
%
% Load the \textsf{childdoc} definitions and
% declare the filename for the main document:
%    \begin{macrocode}
\input{childdoc.def}
\childdocmain{}
%    \end{macrocode}

% Optional override for |\version| flag:
%    \begin{macrocode}
%%\ifchilddoc\else\providecommand{\version}{draft}\fi
%    \end{macrocode}

% Define the default values for the |\version| flag
% (|final| for the main file and |draft| for childs):
%    \begin{macrocode}
\ifchilddoc
\providecommand{\version}{draft}
\else
\providecommand{\version}{final}
\fi
%    \end{macrocode}

% Load the standard document class:
%    \begin{macrocode}
\documentclass[12pt]{article}
%    \end{macrocode}

% Start the document body:
%    \begin{macrocode}
\begin{document}
%    \end{macrocode}

% Declare a title page.
% Print title, part of document being processed and version flag:
%    \begin{macrocode}
\addtocounter{page}{-1}
\begin{center}
{\LARGE\bfseries{}childdoc example\par}
\vspace{1cm}
\ifchilddoc
\ifchilddocmanual part\else chapter\fi:
`\childdocname' of `\childdocjob'\par
\else
main document: `\childdocjob'\par
\fi
version: \version\par
\end{center}
\newpage
%    \end{macrocode}

% Manually include selected file,
% otherwise process as usual:
%    \begin{macrocode}
\ifchilddocmanual
\section*{part `\childdocname'}
\input{\childdocname}
\else
%    \end{macrocode}

% Include the two chapters:
%    \begin{macrocode}
\include{cdocsch1}
\include{cdocsch2}
%    \end{macrocode}

% Include the two parts unless only chapters should be displayed:
%    \begin{macrocode}
\ifchilddoc\else
\section{part three}
\input{cdocspt3}
\section{part four}
\input{cdocspt4}
\fi
%    \end{macrocode}

% Process as usual until here:
%    \begin{macrocode}
\fi
%    \end{macrocode}

% End of document body:
%    \begin{macrocode}
\end{document}
%    \end{macrocode}
%\iffalse
%</samplemain>
%\fi
%
% %%%%%%%%%%%%%%%%%%%%%%%%%%%%%%%%%%%%%%
% \paragraph{Chapter Include Files.}
%
% The include files are called |cdocsch1.tex| and |cdocsch2.tex|.
%
%\iffalse
%<*samplechap1|samplechap2>
%\fi

% Optional override for |\version| flag:
%    \begin{macrocode}
%%\providecommand{\version}{final}
%    \end{macrocode}

% Include the main document:
%    \begin{macrocode}
\input{childdoc.def}
\childdocof{cdocsamp}
%    \end{macrocode}

%\iffalse
%</samplechap1|samplechap2>
%\fi
%
%\iffalse
%<*samplechap1>
%\fi
% Some text for chapter 1:
%    \begin{macrocode}
\section{one}
some text in chapter one
%    \end{macrocode}

%\iffalse
%</samplechap1>
%\fi
% Some text for chapter 2:
%\iffalse
%<*samplechap2>
%\fi
%    \begin{macrocode}
\section{two}
more text in chapter two
%    \end{macrocode}

%\iffalse
%</samplechap2>
%\fi
%
% %%%%%%%%%%%%%%%%%%%%%%%%%%%%%%%%%%%%%%
% \paragraph{Part Include Files.}
%
% The include files are called |cdocspt3.tex| and |cdocspt4.tex|.
%
%\iffalse
%<*samplepart3|samplepart4>
%\fi

% Optional override for |\version| flag:
%    \begin{macrocode}
%%\providecommand{\version}{final}
%    \end{macrocode}

% Include the main document:
%    \begin{macrocode}
\input{childdoc.def}
\childdocby{cdocsamp}
%    \end{macrocode}

%\iffalse
%</samplepart3|samplepart4>
%\fi
%
%\iffalse
%<*samplepart3>
%\fi
% Some text for part 3:
%    \begin{macrocode}
some text in part three
%    \end{macrocode}

%\iffalse
%</samplepart3>
%\fi
% Some text for part 4:
%\iffalse
%<*samplepart4>
%\fi
%    \begin{macrocode}
more text in part four
%    \end{macrocode}

%\iffalse
%</samplepart4>
%\fi
%
% %%%%%%%%%%%%%%%%%%%%%%%%%%%%%%%%%%%%%%
% \paragraph{Forwarding for a Complete Draft.}
%
% The following forwarding file |cdocsdrf.tex|
% compiles the main document in draft mode:
%\iffalse
%<*sampledraft>
%\fi
%    \begin{macrocode}
\def\version{draft}
\input{childdoc.def}
\childdocforward{cdocsamp}
%    \end{macrocode}

%\iffalse
%</sampledraft>
%\fi
%
% %%%%%%%%%%%%%%%%%%%%%%%%%%%%%%%%%%%%%%
% \paragraph{Forwarding for Final Version of the Chapters.}
%
% The following forwarding files |cdocsfn1.tex| and |cdocsfn2.tex|
% (with identical content)
% compile the final versions of the child documents
% |cdocsch1.tex| and |cdocsch2.tex|, respectively:
%\iffalse
%<*samplefinal>
%\fi
%    \begin{macrocode}
\def\version{final}
\input{childdoc.def}
\childdocforwardprefix[cdocsamp]{cdocsfn}{cdocsch}
%    \end{macrocode}

%\iffalse
%</samplefinal>
%\fi
%
% %%%%%%%%%%%%%%%%%%%%%%%%%%%%%%%%%%%%%%
% \paragraph{Command Line Processing.}
%
% The following three command lines generate the output files
% |cdocscld|, |cdocscl1| and |cdocscl2|
% which should be identical to
% |cdocsdrf|, |cdocsch1| and |cdocsfn2|, respectively:
% \begin{center}
% \begin{tabular}{l}
% |latex -jobname cdocscld \|\\
% |  "\def\version{draft}\input{childdoc.def}\childdocforward{cdocsamp}"|\\
% |latex -jobname cdocscl1 \|\\
% |  "\input{childdoc.def}\childdocforward[cdocsamp]{cdocsch1}"|\\
% |latex -jobname cdocscl2 \|\\
% |  "\def\version{final}\input{childdoc.def}\childdocforward{cdocsch2}"|
% \end{tabular}
% \end{center}
% Note that the trailing backslash on each first line
% merely continues the input to the second line
% (for convenient cut ant paste).
% Furthermore, the command |latex| can be replaced by any
% of its alternative versions such as |pdflatex|.
%
% %%%%%%%%%%%%%%%%%%%%%%%%%%%%%%%%%%%%%%%%%%%%%%%%%%%%%%%%%%%%%%%%%%%%%%%%%%%%%%
% %%%%%%%%%%%%%%%%%%%%%%%%%%%%%%%%%%%%%%%%%%%%%%%%%%%%%%%%%%%%%%%%%%%%%%%%%%%%%%
% \section{Implementation}
%\iffalse
%<*package>
%\fi
%
% This section describes the definitions file |childdoc.def|.

% The definitions cannot be loaded using |\usepackage| or |\RequirePackage|
% which has a mechanism to prevent loading a style file more than once.
% When loading the definitions by means of |\input|
% multiple instances have to be prevented manually:
%\iffalse
%This code needs to be before the `\ProvidesFile' directive
%which is defined at the beginning of this file.
%Therefore it is also placed there and commented out here.
%</package>
%<*discard>
%\fi
%    \begin{macrocode}
\ifdefined\childdocmain\endinput\fi
%    \end{macrocode}
%\iffalse
%</discard>
%<*package>
%\fi
%
% \macro{\ifchilddoc}
% \macro{\ifchilddocmanual}
% The conditional |\ifchilddoc| tells whether a
% child (true) or main (false) document is being compiled.
% The conditional |\ifchilddocmanual| tells whether
% the |\includeonly| mechanism is used (false) or
% the selection of child files must be performed manually (true).
% The definitions initialise to false:
%    \begin{macrocode}
\newif\ifchilddoc
\newif\ifchilddocmanual
%    \end{macrocode}

% \macro{\childdocname}
% \macro{\childdocjob}
% The macro |\childdocname| stores the name of the main document
% to be compiled. The macro |\childdocjob| stores the name of
% the document on which the \LaTeX{} compiler was originally invoked.
% The content of |\jobname| cannot be compared
% to filenames specified in the source due to different catcodes.
% The following code rescans |\jobname|, stores the result
% in |\childdocname| and saves a copy in |\childdocjob|:
%    \begin{macrocode}
\edef\childdocname{\scantokens\expandafter{\jobname\noexpand}}
\let\childdocjob\childdocname
%    \end{macrocode}

% \macro{\childdocdisable}
% The macro |\childdocdisable| prevents the main file
% from being processed more than once.
% At this stage, the main document command |\childdocmain|
% is assumed to be called once again where it should do nothing.
% Any subsequent call to it should prevent
% a secondary processing of the main document
% It overwrites the forwarding commands
% |\childdocof| and |\childdocforward|
% with empty macros to prevent further inclusions of the main document:
%    \begin{macrocode}
\newcommand{\childdocdisable}
{
  \renewcommand{\childdocmain}[1]{\renewcommand{\childdocmain}[1]{\endinput}}
  \renewcommand{\childdocof}[1]{}
  \renewcommand{\childdocby}[2][]{}
  \renewcommand{\childdocforward}[2][]{}
  \renewcommand{\childdocdisable}{}
}
%    \end{macrocode}

% \macro{\childdocmain}
% The macro |\childdocmain| is to be called at the top of the main file
% with nothing or the main filename (without extension) as argument.
% First, it breaks loops.
% If the argument is not empty and does not match |\childdocname|
% (which is set by the first inclusion of |childdoc.def|),
% |\ifchilddoc| is set to true, |\includeonly| is applied to the child file
% and |\jobname| is set to the main file
% (for proper handling of |.aux| files):
%    \begin{macrocode}
\newcommand{\childdocmain}[1]
{
  \childdocdisable\childdocmain{}
  \if?#1?\else
    \begingroup
      \def\childdoctmp{#1}
      \ifx\childdoctmp\childdocname
        \def\childdoctmp{}
      \else
        \def\childdoctmp
        {
          \childdoctrue
          \includeonly{\childdocname}
          \def\childdocjob{#1}
          \def\jobname{#1}
        }
      \fi
      \expandafter
    \endgroup
    \childdoctmp
  \fi
}
%    \end{macrocode}

% \macro{\childdocof}
% The command |\childdocof| redirects
% compilation to the main file |#1|.
%    \begin{macrocode}
\newcommand{\childdocof}[1]
{
  \childdocdisable
  \childdoctrue
  \includeonly{\childdocname}
  \def\jobname{#1}
  \def\childdocjob{#1}
  \input{#1}
}
%    \end{macrocode}

% \macro{\childdocby}
% The command |\childdocby| ....
%    \begin{macrocode}
\newcommand{\childdocby}[2][]
{
  \childdocdisable
  \childdoctrue
  \childdocmanualtrue
  \if?#1?\else
    \def\jobname{#2}
  \fi
  \def\childdocjob{#2}
  \input{#2}
  \endinput
}
%    \end{macrocode}

% \macro{\childdocforward}
% The command |\childdocforward| redirects
% compilation to the main file or
% (if the optional argument is given) a child file.
% Parameters are set as if the main file
% or a child file starting with |\childdocof| was compiled.
% Then compilation is handed over to the main file:
%    \begin{macrocode}
\newcommand{\childdocforward}[2][]
{
  \begingroup
    \if?#1?
      \def\childdoctmp
      {
        \def\childdocname{#2}
        \def\childdocjob{#2}
        \def\jobname{#2}
        \input{#2}
        \endinput
      }
    \else
      \def\childdoctmp
      {
        \childdocdisable
        \def\childdocname{#2}
        \childdoctrue
        \includeonly{#2}
        \def\childdocjob{#1}
        \def\jobname{#1}
        \input{#1}
        \endinput
      }
    \fi
    \expandafter
  \endgroup
  \childdoctmp
}
%    \end{macrocode}

% \macro{\childdocforwardprefix}
% The command |\childdocforwardprefix| redirects
% compilation to the main or a child file by means of a pattern.
% The prefix |#1| in the current filename is replaced by |#2|
% and the suffix of the current filename is kept
% (it is assumed that the filename does not contain the substring `|~~~|'
% which is used as a delimiter).
% Compilation is handed over to the new file by |\childdocforward|:
%    \begin{macrocode}
\newcommand{\childdocforwardprefix}[3][]
{
  \begingroup
    \def\childdocextract #2##1~~~{\def\childdoctmp{\childdocforward[#1]{#3##1}}}
    \expandafter\childdocextract\childdocname~~~
    \expandafter
  \endgroup
  \childdoctmp
}
%    \end{macrocode}

% \macro{\childdoc}
% The deprecated macro |\childdoc| is a legacy version of |\childdocmain|:
%    \begin{macrocode}
\newcommand{\childdoc}{\childdocmain}
%    \end{macrocode}

% \macro{\childdocredirect}
% The deprecated macro |\childdocredirect| is a legacy version
% of |\childdocforward| and |\childdocforwardprefix|:
%    \begin{macrocode}
\newcommand{\childdocredirect}[2][]
{
  \begingroup
    \if?#1?
      \def\childdoctmp{\childdocforward{#2}}
    \else
      \def\childdoctmp{\childdocforwardprefix{#1}{#2}}
    \fi
    \expandafter
  \endgroup
  \childdoctmp
}
%    \end{macrocode}

%\iffalse
%</package>
%\fi
%
\endinput
|\\
|\childdocby{|\textit{main}|}|\\
\end{tabular}
\end{center}
%
The directive |\childdocby| is similar to |\childdocof|
described in \secref{sec:include},
but the subsequent selection of content must be done manually.
To that end, both |\ifchilddoc| and |\ifchilddocmanual|
will be true upon processing of a part,
and the name of the part is stored in |\childdocname|.
Note that |\jobname| will be set to the filename of the current part
so that each part receives an individual |.aux| file
that does not interfere with the |.aux| file(s) of the main document.
This behaviour can be altered by the alternative form
|\childdocby[*]{|\textit{main}|}| (with a non-empty optional argument)
which uses the |.aux| file of the main document
by setting |\jobname| to \textit{main}.

%%%%%%%%%%%%%%%%%%%%%%%%%%%%%%%%%%%%%%%%%%%%%%%%%%%%%%%%%%%%%%%%%%%%%%%%%%%%%%%%
\subsection{Driver Development}
\label{sec:driver}

The \textsf{childdoc} mechanism can also be use for the development
of definition files such as \LaTeX{} styles or classes.
This case differs from the above setup with multiple parts
included by |\include| in that no |\includeonly| should be invoked.
This can be achieved by starting the include file
(before |\ProvidesPackage|) with:
%
\begin{center}
\begin{tabular}{l}
|% \iffalse
%
% childdoc.dtx Copyright (C) 2017-2018 Niklas Beisert
%
% This work may be distributed and/or modified under the
% conditions of the LaTeX Project Public License, either version 1.3
% of this license or (at your option) any later version.
% The latest version of this license is in
%   http://www.latex-project.org/lppl.txt
% and version 1.3 or later is part of all distributions of LaTeX
% version 2005/12/01 or later.
%
% This work has the LPPL maintenance status `maintained'.
%
% The Current Maintainer of this work is Niklas Beisert.
%
% This work consists of the files childdoc.dtx and childdoc.ins
% and the derived files childdoc.def and cdocsamp.tex with
% cdocsch1.tex, cdocsch2.tex, cdocsdrf.tex, cdocsfn1.tex, cdocsfn2.tex.
%
%<package>\ifdefined\childdocmain\endinput\fi
%<package>\ProvidesFile{childdoc.def}[2018/12/30 v2.0 child document driver]
%<samplemain>\ProvidesFile{cdocsamp.tex}[2018/12/30 v2.0 sample for childdoc]
%<*driver>
%\ProvidesFile{childdoc.drv}[2018/12/30 v2.0 childdoc reference manual file]
\PassOptionsToClass{10pt,a4paper}{article}
\documentclass{ltxdoc}

\usepackage[margin=35mm]{geometry}
\usepackage{hyperref}
\usepackage{hyperxmp}
\usepackage[usenames]{color}

\hypersetup{colorlinks=true}
\hypersetup{pdfstartview=FitH}
\hypersetup{pdfpagemode=UseNone}
\hypersetup{pdfsource={}}
\hypersetup{pdflang={en-UK}}
\hypersetup{pdfcopyright={Copyright 2017-2018 Niklas Beisert.
  This work may be distributed and/or modified under the
  conditions of the LaTeX Project Public License, either version 1.3
  of this license or (at your option) any later version.}}
\hypersetup{pdflicenseurl={http://www.latex-project.org/lppl.txt}}
\hypersetup{pdfcontactaddress={ETH Zurich, ITP, HIT K,
  Wolfgang-Pauli-Strasse 27}}
\hypersetup{pdfcontactpostcode={8093}}
\hypersetup{pdfcontactcity={Zurich}}
\hypersetup{pdfcontactcountry={Switzerland}}
\hypersetup{pdfcontactemail={nbeisert@itp.phys.ethz.ch}}
\hypersetup{pdfcontacturl={http://people.phys.ethz.ch/\xmptilde nbeisert/}}

\newcommand{\secref}[1]{\hyperref[#1]{section \ref*{#1}}}

\parskip1ex
\parindent0pt
\let\olditemize\itemize
\def\itemize{\olditemize\parskip0pt}

\begin{document}

\title{The \textsf{childdoc} Package}
\hypersetup{pdftitle={The childdoc Package}}
\author{Niklas Beisert\\[2ex]
  Institut f\"ur Theoretische Physik\\
  Eidgen\"ossische Technische Hochschule Z\"urich\\
  Wolfgang-Pauli-Strasse 27, 8093 Z\"urich, Switzerland\\[1ex]
  \href{mailto:nbeisert@itp.phys.ethz.ch}
  {\texttt{nbeisert@itp.phys.ethz.ch}}}
\hypersetup{pdfauthor={Niklas Beisert}}
\hypersetup{pdfsubject={Manual for the LaTeX2e Package childdoc}}
\date{30 December 2018, \textsf{v2.0}}
\maketitle

\begin{abstract}\noindent
\textsf{childdoc} is a \LaTeXe{} package
that enables the direct compilation
of document sections included by |\include|
to individual files.
\end{abstract}

\begingroup
\parskip0ex
\tableofcontents
\endgroup

%%%%%%%%%%%%%%%%%%%%%%%%%%%%%%%%%%%%%%%%%%%%%%%%%%%%%%%%%%%%%%%%%%%%%%%%%%%%%%%%
%%%%%%%%%%%%%%%%%%%%%%%%%%%%%%%%%%%%%%%%%%%%%%%%%%%%%%%%%%%%%%%%%%%%%%%%%%%%%%%%
\section{Introduction}

\LaTeX{} provides a mechanism to structure a large document (such as a book)
into a main file and several child files (containing the chapters)
using the |\include| command.
This mechanism is beneficial for documents
which span hundreds of pages in order to
make the source file(s) more manageable.
Moreover, compilation can be restricted to
selected child files by means of the |\includeonly| command.
The latter feature can be used to reduce the compilation time while editing
(this was significantly more useful in the earlier days of \LaTeX{})
or to generate a smaller document which is easier to navigate.
Another application of |\includeonly| is to generate
documents consisting of selected parts of the complete document.

However, there are a few drawbacks of the plain |\include| mechanism:
\begin{itemize}
\item
The child files cannot be compiled on their own,
they can only be compiled via the main file.
A naive editing environment
(such as a text editor with an option
to have the current file processed by \LaTeX)
may require one to switch to the main file before compiling;
attempting to compile the child file produces errors.
\item
The main file must be modified (each time)
to adjust the |\includeonly| command
to the present needs. This easily leaves the main file in a messy state.
\item
The generated document will always carry the filename
of the main document. This is inconvenient if
several child files are to be compiled and
to be kept for distribution.
\end{itemize}

The present package provides a simple interface
to make child files individually compilable by \LaTeX{}.
Compiling a child file then has the same effect as compiling
the main file with an |\includeonly| command
to select the appropriate child.
Moreover the generated document will carry the name of the child
rather than the main file.
This resolves all three above issues.

This feature is meant to make the editing of books,
thesis documents and lecture notes somewhat more convenient.
However, the package can also be used efficiently for
composing a series of documents (such as exercise sheets)
which are typically distributed individually.
It then assists the author in generating the individual documents
(potentially in different versions)
as well as a document containing the collected series.
Another application is in developing style files
or other kinds of included material
where compilation of the style file could redirect
to a sample or test file.

%%%%%%%%%%%%%%%%%%%%%%%%%%%%%%%%%%%%%%%%%%%%%%%%%%%%%%%%%%%%%%%%%%%%%%%%%%%%%%%%
%%%%%%%%%%%%%%%%%%%%%%%%%%%%%%%%%%%%%%%%%%%%%%%%%%%%%%%%%%%%%%%%%%%%%%%%%%%%%%%%
\section{Usage}

First of all, the package \textsf{childdoc} is \emph{not} a standard
\LaTeXe{} |.sty| style file! Therefore it needs to be invoked in
a non-standard way.

%%%%%%%%%%%%%%%%%%%%%%%%%%%%%%%%%%%%%%%%%%%%%%%%%%%%%%%%%%%%%%%%%%%%%%%%%%%%%%%%
\subsection{Included Files}
\label{sec:include}

%%%%%%%%%%%%%%%%%%%%%%%%%%%%%%%%%%%%%%%%
\DescribeMacro{\childdocmain}
To use the package, add the commands
\begin{center}
\begin{tabular}{l}
|\input{childdoc.def}|\\
|\childdocmain{}|\\
\end{tabular}
\end{center}
at the very top of the main \LaTeX{} file,
in particular \emph{before} the |\documentclass| statement!
The argument of |\childdocmain| should be left empty
(but it must be present).

%%%%%%%%%%%%%%%%%%%%%%%%%%%%%%%%%%%%%%%%
\DescribeMacro{\childdocof}
Furthermore, add the commands
\begin{center}
\begin{tabular}{l}
|\input{childdoc.def}|\\
|\childdocof{|\textit{main}|}|\\
\end{tabular}
\end{center}
at the top of every child file \textit{child}
which is included by |\include{|\textit{child}|}|
from within the main file
(or at least for those files to be compiled individually).
The argument \textit{main} must be the filename of the main file.

There are a couple of
considerations in setting up the main and child documents:

%%%%%%%%%%%%%%%%%%%%%%%%%%%%%%%%%%%%%%%%
\paragraph{Restrictions.}

Please note the following restrictions:
\begin{itemize}
\item
|\childdocmain| must be called with one argument \textit{main}
to ensure compatibility with earlier version of the package.
It must either be empty (|\childdocmain{}|)
or precisely match the filename of the main file in which it is specified.
See \secref{sec:detection} for further information.
\item
The filename \textit{main} must be specified without the |.tex| extension.
\item
The filename \textit{main} is case sensitive
(even in case-insensitive file systems)
due to internal string comparison.
\item
The argument \textit{main} should be fully expanded, it cannot be a macro.
\item
Subdirectories and special characters should be avoided in filenames.
\item
The command |\childdocmain{|\textit{main}|}| must be followed by a whitespace.
It should not be followed immediately by another command
or by a comment mark `|%|'.
This is because the \TeX{} parser reads the token immediately following
the argument of |\childdocmain| and puts it
at the beginning of every child section;
however, a white\-space is ignored.
\end{itemize}

%%%%%%%%%%%%%%%%%%%%%%%%%%%%%%%%%%%%%%%%
\paragraph{Content of Main File.}

It is advisable to place all content in the child files included by |\include|.
Any output contained in the main file will appear in all child documents
unless suppressed manually;
it cannot be suppressed automatically by the |\includeonly| directive
and thus should normally be avoided.
A method to include some content in the main file
by means of conditional processing is described in \secref{sec:conditional}.

%%%%%%%%%%%%%%%%%%%%%%%%%%%%%%%%%%%%%%%%
\paragraph{Page Numbering.}

When only a part of the document is compiled,
the appropriate numbering of pages
(as well as other status parameters)
is determined from the |.aux| files.
The latter contain information from previous passes.
However this information needs to propagate through
all intermediate child documents.
Therefore the page numbering in child documents may well
be inconsistent until the complete document is compiled at least once.

A useful (if unconventional) way to always ensure a consistent
page numbering is to restart the numbering in each child document
and denote the pages by `\textit{child}|.|\textit{page}'
where \textit{child} represents the chapter/section number of the child file.
This can be achieved by the command
|\numberwithin{page}{|\textit{child}|}|
of the \textsf{amsmath} package
where \textit{child} can be |chapter| or |section|
depending on the chosen structuring.
Alternatively, one can modify the macro |\thepage| appropriately
and reset the counter |page| at the start of each child file.

%%%%%%%%%%%%%%%%%%%%%%%%%%%%%%%%%%%%%%%%%%%%%%%%%%%%%%%%%%%%%%%%%%%%%%%%%%%%%%%%
\subsection{Conditional Processing}
\label{sec:conditional}

The package provides a mechanism to compile different versions
of a document. To customise the versions further some conditional processing
can come in handy to distinguish which version is being compiled.
The package provides two macros to describe the compilation context:

%%%%%%%%%%%%%%%%%%%%%%%%%%%%%%%%%%%%%%%%
\DescribeMacro{\ifchilddoc}
The conditional |\ifchilddoc| distinguishes between the compilation of
child documents and the main document:
%
\begin{center}
|\ifchilddoc |\textit{child-code}| |[|\||else |\textit{main-code}]| \||fi|
\end{center}

%%%%%%%%%%%%%%%%%%%%%%%%%%%%%%%%%%%%%%%%
\DescribeMacro{\childdocname}
\DescribeMacro{\childdocjob}
The macro |\childdocname| contains the filename (without extension)
of the main or child file being processed.
Note that |\childdocjob| will always contain the name of the main file.

%%%%%%%%%%%%%%%%%%%%%%%%%%%%%%%%%%%%%%%%
\paragraph{Title Page.}

Conditional processing can be used to include a title or banner page
in the main document when proper precautions are taken.
Importantly, the code in the main file should ensure that the page counter
(as well as other status parameters which are stored in the |.aux| files)
takes the same value after the conditional processing.
Otherwise the page numbers may take divergent values
depending on which part is compiled.

For example, a title page could be declared by:
%
\begin{center}
\begin{tabular}{l}
|\ifchilddoc\||else|\\
|\addtocounter{page}{-1}|\\
\textit{code for title page}\\
|\newpage|\\
|\||fi|
\end{tabular}
\end{center}
%
A banner page for the child documents can be generated by:
%
\begin{center}
\begin{tabular}{l}
|\ifchilddoc|\\
|\addtocounter{page}{-1}|\\
\textit{code for banner page}\\
|\newpage|\\
|\||fi|
\end{tabular}
\end{center}
%
Here one could write a message such as:
\begin{center}
|This is the part \childdocname{} of \childdocjob{}.|
\end{center}

%%%%%%%%%%%%%%%%%%%%%%%%%%%%%%%%%%%%%%%%%%%%%%%%%%%%%%%%%%%%%%%%%%%%%%%%%%%%%%%%
\subsection{Flags}
\label{sec:flags}

The package makes it easy to generate different versions
of the main or child documents.
To this end compilation flags can be defined
and assigned different default values.
They will be particularly useful in conjunction
with the forwarding mechanism described in \secref{sec:forward}.

For example, it may be useful to have a flag |\version|
which can be set to |draft| or |final|.
The document source will contain some conditional code
depending on the value of |\version|.
Suppose further, the flag should default to |final| for the main file
and to |draft| for child files
which is a natural assignment for editing the document.
This is achieved by placing the following code
in the preamble of the main document
(below the |\childdocmain| directive):
%
\begin{center}
\begin{tabular}{l}
|\ifchilddoc|\\
|\providecommand{\version}{draft}|\\
|\||else|\\
|\providecommand{\version}{final}|\\
|\||fi|
\end{tabular}
\end{center}
%
The definition by |\providecommand| makes sure
that previous definitions are not overwritten.
Further statements |\providecommand{\version}{...}|
can thus be added before the above code to override it.

For the main file, one might add a line
(between |\childdocmain| and the above block)
%
\begin{center}
|%\ifchilddoc\||else\providecommand{\version}{draft}\||fi|
\end{center}
%
which can be uncommented to produce a draft version.
Likewise one can add a line to the very top of a child file
(above the |\childdocof{|\textit{main}|}| directive)
%
\begin{center}
|%\providecommand{\version}{final}|
\end{center}
%
which can be uncommented to produce the final version of this child document.

%%%%%%%%%%%%%%%%%%%%%%%%%%%%%%%%%%%%%%%%%%%%%%%%%%%%%%%%%%%%%%%%%%%%%%%%%%%%%%%%
\subsection{Forwarding}
\label{sec:forward}

Different versions of the main or child documents
using compilation flags as described in \secref{sec:flags}
can be (permanently) stored in different files
for convenient compilation, viewing and distribution.
To this end, the package defines a command
to pass on compilation to a different file:

%%%%%%%%%%%%%%%%%%%%%%%%%%%%%%%%%%%%%%%%
\DescribeMacro{\childdocforward}
The command |\childdocforward| redirects processing to
another source file:
%
\begin{center}
\begin{tabular}{l}
|\input{childdoc.def}|\\
|\childdocforward[|\textit{main}|]{|\textit{dest}|}|\\
\end{tabular}
\end{center}
%
The argument \textit{dest} is the destination file
(without extension).
It should be the main file or one of the child files.
Note that further \textsf{childdoc} directives
such as |\childdocof| and |\childdocforward|
in the indicated file will be processed in this form.
The optional argument \textit{main}
passes on directly to the main file \textit{main}
while pretending to compile the child \textit{dest}.
This form behaves as if \textit{dest}
issues |\childdocof{|\textit{main}|}| right away,
and no further \textsf{childdoc} directives will be processed.

%%%%%%%%%%%%%%%%%%%%%%%%%%%%%%%%%%%%%%%%
\DescribeMacro{\...prefix}
In the alternative form |\childdocforwardprefix|,
%
\begin{center}
\begin{tabular}{l}
|\input{childdoc.def}|\\
|\childdocforwardprefix[|\textit{main}|]{|\textit{prefix}|}{|\textit{dest}|}|
\end{tabular}
\end{center}
%
the destination file is determined by a pattern
depending on the current file:
To make this work, the current file must be called
`{\textit{prefix}\hspace{0.2em}\textit{suffix}}'
with \textit{prefix} matching precisely the argument.
Processing is then passed on to the file
`{\textit{dest}\hspace{0.2em}\textit{suffix}}'.
Surely, the same effect is achieved by
directly specifying the
argument `{\textit{dest}\hspace{0.2em}\textit{suffix}}'
in the first form.
However, that requires to set up a different file
for each child. With the alternative form of the command
all these files can have exactly the same content
which simplifies setting them up and maintaining them.

For example, the following file |draft.tex|
with a compilation flag |\version| as described in \secref{sec:flags}
compiles the main document as a draft:
%
\begin{center}
\begin{tabular}{l}
|\def\version{draft}|\\
|\input{childdoc.def}|\\
|\childdocforward{|\textit{main}|}|
\end{tabular}
\end{center}
%
Likewise, the following files |final|\textit{nn}|.tex|
compile the final version of the child document
|child|\textit{nn}|.tex|:
%
\begin{center}
\begin{tabular}{l}
|\def\version{final}|\\
|\input{childdoc.def}|\\
|\childdocforwardprefix{final}{child}|
\end{tabular}
\end{center}
%

Note that when several versions of a main file and/or of each child file
are to be generated, it may be convenient to set up a |Makefile| or
shell script to automatise the process.

%%%%%%%%%%%%%%%%%%%%%%%%%%%%%%%%%%%%%%%%%%%%%%%%%%%%%%%%%%%%%%%%%%%%%%%%%%%%%%%%
\subsection{Command Line Processing}
\label{sec:commandline}

The effect of redirection files can also be achieved by invoking
the \LaTeX{} compiler with a more elaborate command line.
Most conveniently this should be done as part
of a shell script or a |Makefile|.

When using \textsf{childdoc} in the main file, the following
command lines effectively perform a redirection
(note that depending on the shell being used,
backslashes may have to be doubled: `|\|' $\to$ `|\\|'):
%
\begin{center}
|... -jobname "|\textit{target}|" |\\|"|[\textit{flags}]%
|\input{childdoc.def}\childdocforward[|\textit{main}|]{|\textit{dest}|}"|
\end{center}
%
Here \textit{target} is the name of the output file,
\textit{main} is the name of the main file
and \textit{dest} is the name of the main or child file to be processed
(all filenames without extensions).
The optional argument \textit{main} can be omitted
if \textit{main} matches \textit{dest}.
Optionally, compilation \textit{flags} can be defined via |\def| commands.
This command line makes the \TeX{} engine believe
it is compiling the file \textit{target}
whose content is specified as the latter parameter.
The provided code then forwards the processing to
\textit{main} or \textit{dest} as described in \secref{sec:forward}.

%%%%%%%%%%%%%%%%%%%%%%%%%%%%%%%%%%%%%%%%%%%%%%%%%%%%%%%%%%%%%%%%%%%%%%%%%%%%%%%%
\subsection{Include by Input}
\label{sec:input}

Including child documents by |\include| has some restrictions by design.
Most notably, the content of a child document always occupies
its own set of pages; pages cannot be shared between child documents.
Usually, this behaviour makes perfect sense
because each child document contain an essential part of the document.
However, in some situations it may be desirable to compose
a document from a collection of parts
without having mandatory page breaks between then.
For this case, the package
provides a mechanism to include parts
by |\input| which can also be processed individually.
However, by construction this mechanism
requires manual handling of the content to be output.

%%%%%%%%%%%%%%%%%%%%%%%%%%%%%%%%%%%%%%%%
\DescribeMacro{\ifchilddocmanual}
The main file should be prepared as usual, see \secref{sec:include}.
However, the document body must make a distinction
between processing of an individual part and of the main document, e.g.:
%
\begin{center}
\begin{tabular}{l}
|\ifchilddocmanual|\\
|\input{\childdocname}|\\
|\||else|\\
\textit{document body with }|\input{|\textit{part}|}|\\
|\||fi|
\end{tabular}
\end{center}
%
The conditional |\ifchilddocmanual| is true whenever
a part to be included by |\input| is being compiled,
and the name of the part is stored in |\childdocname|.

%%%%%%%%%%%%%%%%%%%%%%%%%%%%%%%%%%%%%%%%
\DescribeMacro{\childdocby}
Each part to be included by |\input| should start with:
%
\begin{center}
\begin{tabular}{l}
|\input{childdoc.def}|\\
|\childdocby{|\textit{main}|}|\\
\end{tabular}
\end{center}
%
The directive |\childdocby| is similar to |\childdocof|
described in \secref{sec:include},
but the subsequent selection of content must be done manually.
To that end, both |\ifchilddoc| and |\ifchilddocmanual|
will be true upon processing of a part,
and the name of the part is stored in |\childdocname|.
Note that |\jobname| will be set to the filename of the current part
so that each part receives an individual |.aux| file
that does not interfere with the |.aux| file(s) of the main document.
This behaviour can be altered by the alternative form
|\childdocby[*]{|\textit{main}|}| (with a non-empty optional argument)
which uses the |.aux| file of the main document
by setting |\jobname| to \textit{main}.

%%%%%%%%%%%%%%%%%%%%%%%%%%%%%%%%%%%%%%%%%%%%%%%%%%%%%%%%%%%%%%%%%%%%%%%%%%%%%%%%
\subsection{Driver Development}
\label{sec:driver}

The \textsf{childdoc} mechanism can also be use for the development
of definition files such as \LaTeX{} styles or classes.
This case differs from the above setup with multiple parts
included by |\include| in that no |\includeonly| should be invoked.
This can be achieved by starting the include file
(before |\ProvidesPackage|) with:
%
\begin{center}
\begin{tabular}{l}
|\input{childdoc.def}|\\
|\childdocforward{|\textit{main}|}|\\
\end{tabular}
\end{center}
%
or alternatively with:
%
\begin{center}
\begin{tabular}{l}
|\input{childdoc.def}|\\
|\childdocby{|\textit{main}|}|\\
\end{tabular}
\end{center}
%
Both forms have slightly different effects as described above.
The main file is prepared as usual, see \secref{sec:include}.

%%%%%%%%%%%%%%%%%%%%%%%%%%%%%%%%%%%%%%%%%%%%%%%%%%%%%%%%%%%%%%%%%%%%%%%%%%%%%%%%
\subsection{Legacy Detection}
\label{sec:detection}

The directive |\childdocmain| in the main file can detect
whether the complete document or merely a child is to be compiled
even without using the directive |\childdocof|.
This method is deprecated because it is less robust
and there is no compelling reason to use it;
it is merely provided for backward compatibility
and it may be removed in future versions.

If the detection mechanism is to be used,
it is mandatory to correctly specify
the filename of the main file as the argument of |\childdocmain|:
%
\begin{center}
\begin{tabular}{l}
|\input{childdoc.def}|\\
|\childdocmain{|\textit{main}|}|\\
\end{tabular}
\end{center}
%
If |\jobname| does not match the argument \textit{main} of |\childdocmain|,
it is assumed that |\jobname| points to the child file to be compiled.
When using |\childdocmain| with the main file specified as argument,
it suffices to start a child file
with just |\input{|\textit{main}|}|
without loading of the package and using |\childdocof|.
If instead all processing is done
with the appropriate \textsf{childdoc} directives,
the argument of \textit{main} of |\childdocmain| can be empty.

An alternative version of the command line processing described
in \secref{sec:commandline} using the detection mechanism reads:
%
\begin{center}
|... -jobname "|\textit{target}|" "|[\textit{flags}]%
[|\def\jobname{|\textit{dest}|}|]|\input{|\textit{main}|}"|
\end{center}

%%%%%%%%%%%%%%%%%%%%%%%%%%%%%%%%%%%%%%%%%%%%%%%%%%%%%%%%%%%%%%%%%%%%%%%%%%%%%%%%
\subsection{Manual Code}
\label{sec:manual}

In case one cannot be certain whether the definitions file |childdoc.def|
is installed on the target \TeX{} distribution
and one prefers not to ship it,
it is conceivable to paste a few relevant commands into the sources.

To that end, drop all statements |\input{childdoc.def}|
and perform the replacements as outlined below.
Instead of |\childdocmain{|\textit{main}|}| add the following code
to the top of the main file:
%
\begin{center}
\begin{tabular}{l}
|\||ifdefined\childdocname\endinput\||fi\newif\ifchilddoc|\\
|\edef\childdocname{\scantokens\expandafter{\jobname\noexpand}}|\\
|\def\childdocmain{|\textit{main}|}\||ifx\childdocmain\childdocname\||else|\\
|\childdoctrue\includeonly{\childdocname}\let\jobname\childdocmain\||fi|\\
\end{tabular}
\end{center}
%
Instead of |\childdocof{|\textit{main}|}| just include the main file
at the top of each child file:
%
\begin{center}
|\input{|\textit{main}|}|
\end{center}
%
A simple redirection |\childdocforward{|\textit{dest}|}| is achieved by:
%
\begin{center}
|\def\jobname{|\textit{dest}|}\input{\jobname}|
\end{center}
%
The redirection with prefix
|\childdocforwardprefix[|\textit{prefix}|]{|\textit{dest}|}|
is accomplished by:
%
\begin{center}
\begin{tabular}{l}
|{\edef\jobname{\scantokens\expandafter{\jobname\noexpand}}|\\
|\def\redirectjob |\textit{prefix}|#1~~~{\gdef\jobname{|\textit{dest}|#1}}|\\
|\expandafter\redirectjob\jobname~~~}\input{\jobname}|
\end{tabular}
\end{center}

In an alternative approach,
child documents can be compiled by a specific command line
without additional code or specific definitions:
%
\begin{center}
|... -jobname "|\textit{target}|" "|[\textit{flags}]%
|\includeonly{|\textit{dest}|}\input{|\textit{main}|}"|
\end{center}
%

%%%%%%%%%%%%%%%%%%%%%%%%%%%%%%%%%%%%%%%%%%%%%%%%%%%%%%%%%%%%%%%%%%%%%%%%%%%%%%%%
%%%%%%%%%%%%%%%%%%%%%%%%%%%%%%%%%%%%%%%%%%%%%%%%%%%%%%%%%%%%%%%%%%%%%%%%%%%%%%%%
\section{Information}

%%%%%%%%%%%%%%%%%%%%%%%%%%%%%%%%%%%%%%%%%%%%%%%%%%%%%%%%%%%%%%%%%%%%%%%%%%%%%%%%
\subsection{Copyright}

Copyright \copyright{} 2017--2018 Niklas Beisert

This work may be distributed and/or modified under the
conditions of the \LaTeX{} Project Public License, either version 1.3
of this license or (at your option) any later version.
The latest version of this license is in
  \url{http://www.latex-project.org/lppl.txt}
and version 1.3 or later is part of all distributions of \LaTeX{}
version 2005/12/01 or later.

This work has the LPPL maintenance status `maintained'.

The Current Maintainer of this work is Niklas Beisert.

This work consists of the files |README.txt|, |childdoc.ins| and |childdoc.dtx|
as well as the derived files |childdoc.def|, |cdocsamp.tex|
with |cdocsch1.tex|, |cdocsch2.tex|, |cdocspt3.tex|, |cdocspt4.tex|,
|cdocsdrf.tex|, |cdocsfn1.tex|, |cdocsfn2.tex|
as well as |childdoc.pdf|.

%%%%%%%%%%%%%%%%%%%%%%%%%%%%%%%%%%%%%%%%%%%%%%%%%%%%%%%%%%%%%%%%%%%%%%%%%%%%%%%%
\subsection{Files and Installation}

The package consists of the files:
%
\begin{center}
\begin{tabular}{ll}
    |README.txt|   & readme file \\
    |childdoc.ins| & installation file \\
    |childdoc.dtx| & source file \\
    |childdoc.def| & definition file \\
    |cdocsamp.tex| & sample main file \\
    |cdocsch1.tex| & sample include file \\
    |cdocsch2.tex| & sample include file \\
    |cdocspt3.tex| & sample part file \\
    |cdocspt4.tex| & sample part file \\
    |cdocsdrf.tex| & sample redirection file \\
    |cdocsfn1.tex| & sample redirection file \\
    |cdocsfn2.tex| & sample redirection file \\
    |childdoc.pdf| & manual
\end{tabular}
\end{center}
%
The distribution consists of the files
|README.txt|, |childdoc.ins| and |childdoc.dtx|.
%
\begin{itemize}
\item
Run (pdf)\LaTeX{} on |childdoc.dtx|
to compile the manual |childdoc.pdf| (this file).
\item
Run \LaTeX{} on |childdoc.ins| to create the definitions file |childdoc.def|
and the sample |cdocsamp.tex| with include files
|cdocsch1.tex|, |cdocsch2.tex|, |cdocspt3.tex|, |cdocspt4.tex|,
|cdocsdrf.tex|, |cdocsfn1.tex|, |cdocsfn2.tex|.
Then copy the file |childdoc.def| to an appropriate directory of your \LaTeX{}
distribution, e.g.\ \textit{texmf-root}|/tex/latex/childdoc|.
\end{itemize}

%%%%%%%%%%%%%%%%%%%%%%%%%%%%%%%%%%%%%%%%%%%%%%%%%%%%%%%%%%%%%%%%%%%%%%%%%%%%%%%%
\subsection{Related CTAN Packages}

There are several other packages which offer a similar functionality:
%
\begin{itemize}
\item
The packages
\href{http://ctan.org/pkg/docmute}{\textsf{docmute}},
\href{http://ctan.org/pkg/includex}{\textsf{includex}} and
\href{http://ctan.org/pkg/standalone}{\textsf{standalone}}
provide commands to include only the document body of
a child file thus allowing both files to be compiled individually.
\item
The packages \href{http://ctan.org/pkg/subdocs}{\textsf{subdocs}}
and \href{http://ctan.org/pkg/subfiles}{\textsf{subfiles}}
provide structures in which the main and child documents can be
encapsulated and allowing them to be compiled individually.
The inclusion mechanism is different from the conventional |\include|.
\item
The package \href{http://ctan.org/pkg/combine}{\textsf{combine}}
is an elaborate solution to combine several documents into one.
\end{itemize}
%
See also the CTAN topic \href{http://ctan.org/topic/subdocs}{\textsf{subdocs}}
for further related packages.
The present package differs from the above solutions in that
a document structure constructed with the conventional |\include| mechanism
just needs two extra commands at the top of every file
such that all constituent files can be compiled individually.

%%%%%%%%%%%%%%%%%%%%%%%%%%%%%%%%%%%%%%%%%%%%%%%%%%%%%%%%%%%%%%%%%%%%%%%%%%%%%%%%
%\subsection{Feature Suggestions}
%
%The following is a list of features which may be useful for future
%versions of this package:
%%
%\begin{itemize}
%\item
%\ldots
%\end{itemize}

%%%%%%%%%%%%%%%%%%%%%%%%%%%%%%%%%%%%%%%%%%%%%%%%%%%%%%%%%%%%%%%%%%%%%%%%%%%%%%%%
\subsection{Revision History}

%%%%%%%%%%%%%%%%%%%%%%%%%%%%%%%%%%%%%%%%
\paragraph{v2.0:} 2018/12/30

\begin{itemize}
\item
immediate forward processing
\item
added |\childdocby| mechanism
\item
manual restructured
\end{itemize}

%%%%%%%%%%%%%%%%%%%%%%%%%%%%%%%%%%%%%%%%
\paragraph{v1.6:} 2018/01/17

\begin{itemize}
\item
application for development of include files
\item
corrections to manual
\end{itemize}

%%%%%%%%%%%%%%%%%%%%%%%%%%%%%%%%%%%%%%%%
\paragraph{v1.5:} 2017/05/21

\begin{itemize}
\item
more complete structuring introduced
\item
|\childdocof| introduced
\item
|\childdoc| renamed to |\childdocmain|
\item
|\childredirect| renamed to |\childdocforward| and |\childdocforwardprefix|
and functionality expanded
\end{itemize}

%%%%%%%%%%%%%%%%%%%%%%%%%%%%%%%%%%%%%%%%
\paragraph{v1.0:} 2017/04/27

\begin{itemize}
\item
manual and install package
\item
first version published on CTAN
\end{itemize}

%%%%%%%%%%%%%%%%%%%%%%%%%%%%%%%%%%%%%%%%
\paragraph{v0.6:} 2017/04/26

\begin{itemize}
\item
redirection mechanism added
\end{itemize}

%%%%%%%%%%%%%%%%%%%%%%%%%%%%%%%%%%%%%%%%
\paragraph{v0.5:} 2017/04/26

\begin{itemize}
\item
functionality in definition file
\end{itemize}


%%%%%%%%%%%%%%%%%%%%%%%%%%%%%%%%%%%%%%%%%%%%%%%%%%%%%%%%%%%%%%%%%%%%%%%%%%%%%%%%
%%%%%%%%%%%%%%%%%%%%%%%%%%%%%%%%%%%%%%%%%%%%%%%%%%%%%%%%%%%%%%%%%%%%%%%%%%%%%%%%
%%%%%%%%%%%%%%%%%%%%%%%%%%%%%%%%%%%%%%%%%%%%%%%%%%%%%%%%%%%%%%%%%%%%%%%%%%%%%%%%
\appendix

\settowidth\MacroIndent{\rmfamily\scriptsize 000\ }

 \DocInput{childdoc.dtx}

\end{document}
%</driver>
% \fi
%
% %%%%%%%%%%%%%%%%%%%%%%%%%%%%%%%%%%%%%%%%%%%%%%%%%%%%%%%%%%%%%%%%%%%%%%%%%%%%%%
% %%%%%%%%%%%%%%%%%%%%%%%%%%%%%%%%%%%%%%%%%%%%%%%%%%%%%%%%%%%%%%%%%%%%%%%%%%%%%%
% \section{Sample}
%\iffalse
%<*samplemain>
%\fi
%
% The following presents a sample document
% with two chapters, two parts, a title page,
% a compile flag as well as three forwarding files to set the flag.
% It consists of eight |.tex| files:
% \begin{center}
% \begin{tabular}{ll}
% |cdocsamp.tex|&main file\\
% |cdocsch1.tex|&include file for chapter 1\\
% |cdocsch2.tex|&include file for chapter 2\\
% |cdocspt3.tex|&include file for part 3\\
% |cdocspt4.tex|&include file for part 4\\
% |cdocsdrf.tex|&forwarding file for main file in draft mode\\
% |cdocsfi1.tex|&forwarding file for final version of chapter 1\\
% |cdocsfi2.tex|&forwarding file for final version of chapter 2\\
% \end{tabular}
% \end{center}
% Each of the eight files can be compiled directly by the \LaTeX{} compiler.
%
% %%%%%%%%%%%%%%%%%%%%%%%%%%%%%%%%%%%%%%
% \paragraph{Main File.}
%
% The main file is called |cdocsamp.tex|.
%
% Load the \textsf{childdoc} definitions and
% declare the filename for the main document:
%    \begin{macrocode}
\input{childdoc.def}
\childdocmain{}
%    \end{macrocode}

% Optional override for |\version| flag:
%    \begin{macrocode}
%%\ifchilddoc\else\providecommand{\version}{draft}\fi
%    \end{macrocode}

% Define the default values for the |\version| flag
% (|final| for the main file and |draft| for childs):
%    \begin{macrocode}
\ifchilddoc
\providecommand{\version}{draft}
\else
\providecommand{\version}{final}
\fi
%    \end{macrocode}

% Load the standard document class:
%    \begin{macrocode}
\documentclass[12pt]{article}
%    \end{macrocode}

% Start the document body:
%    \begin{macrocode}
\begin{document}
%    \end{macrocode}

% Declare a title page.
% Print title, part of document being processed and version flag:
%    \begin{macrocode}
\addtocounter{page}{-1}
\begin{center}
{\LARGE\bfseries{}childdoc example\par}
\vspace{1cm}
\ifchilddoc
\ifchilddocmanual part\else chapter\fi:
`\childdocname' of `\childdocjob'\par
\else
main document: `\childdocjob'\par
\fi
version: \version\par
\end{center}
\newpage
%    \end{macrocode}

% Manually include selected file,
% otherwise process as usual:
%    \begin{macrocode}
\ifchilddocmanual
\section*{part `\childdocname'}
\input{\childdocname}
\else
%    \end{macrocode}

% Include the two chapters:
%    \begin{macrocode}
\include{cdocsch1}
\include{cdocsch2}
%    \end{macrocode}

% Include the two parts unless only chapters should be displayed:
%    \begin{macrocode}
\ifchilddoc\else
\section{part three}
\input{cdocspt3}
\section{part four}
\input{cdocspt4}
\fi
%    \end{macrocode}

% Process as usual until here:
%    \begin{macrocode}
\fi
%    \end{macrocode}

% End of document body:
%    \begin{macrocode}
\end{document}
%    \end{macrocode}
%\iffalse
%</samplemain>
%\fi
%
% %%%%%%%%%%%%%%%%%%%%%%%%%%%%%%%%%%%%%%
% \paragraph{Chapter Include Files.}
%
% The include files are called |cdocsch1.tex| and |cdocsch2.tex|.
%
%\iffalse
%<*samplechap1|samplechap2>
%\fi

% Optional override for |\version| flag:
%    \begin{macrocode}
%%\providecommand{\version}{final}
%    \end{macrocode}

% Include the main document:
%    \begin{macrocode}
\input{childdoc.def}
\childdocof{cdocsamp}
%    \end{macrocode}

%\iffalse
%</samplechap1|samplechap2>
%\fi
%
%\iffalse
%<*samplechap1>
%\fi
% Some text for chapter 1:
%    \begin{macrocode}
\section{one}
some text in chapter one
%    \end{macrocode}

%\iffalse
%</samplechap1>
%\fi
% Some text for chapter 2:
%\iffalse
%<*samplechap2>
%\fi
%    \begin{macrocode}
\section{two}
more text in chapter two
%    \end{macrocode}

%\iffalse
%</samplechap2>
%\fi
%
% %%%%%%%%%%%%%%%%%%%%%%%%%%%%%%%%%%%%%%
% \paragraph{Part Include Files.}
%
% The include files are called |cdocspt3.tex| and |cdocspt4.tex|.
%
%\iffalse
%<*samplepart3|samplepart4>
%\fi

% Optional override for |\version| flag:
%    \begin{macrocode}
%%\providecommand{\version}{final}
%    \end{macrocode}

% Include the main document:
%    \begin{macrocode}
\input{childdoc.def}
\childdocby{cdocsamp}
%    \end{macrocode}

%\iffalse
%</samplepart3|samplepart4>
%\fi
%
%\iffalse
%<*samplepart3>
%\fi
% Some text for part 3:
%    \begin{macrocode}
some text in part three
%    \end{macrocode}

%\iffalse
%</samplepart3>
%\fi
% Some text for part 4:
%\iffalse
%<*samplepart4>
%\fi
%    \begin{macrocode}
more text in part four
%    \end{macrocode}

%\iffalse
%</samplepart4>
%\fi
%
% %%%%%%%%%%%%%%%%%%%%%%%%%%%%%%%%%%%%%%
% \paragraph{Forwarding for a Complete Draft.}
%
% The following forwarding file |cdocsdrf.tex|
% compiles the main document in draft mode:
%\iffalse
%<*sampledraft>
%\fi
%    \begin{macrocode}
\def\version{draft}
\input{childdoc.def}
\childdocforward{cdocsamp}
%    \end{macrocode}

%\iffalse
%</sampledraft>
%\fi
%
% %%%%%%%%%%%%%%%%%%%%%%%%%%%%%%%%%%%%%%
% \paragraph{Forwarding for Final Version of the Chapters.}
%
% The following forwarding files |cdocsfn1.tex| and |cdocsfn2.tex|
% (with identical content)
% compile the final versions of the child documents
% |cdocsch1.tex| and |cdocsch2.tex|, respectively:
%\iffalse
%<*samplefinal>
%\fi
%    \begin{macrocode}
\def\version{final}
\input{childdoc.def}
\childdocforwardprefix[cdocsamp]{cdocsfn}{cdocsch}
%    \end{macrocode}

%\iffalse
%</samplefinal>
%\fi
%
% %%%%%%%%%%%%%%%%%%%%%%%%%%%%%%%%%%%%%%
% \paragraph{Command Line Processing.}
%
% The following three command lines generate the output files
% |cdocscld|, |cdocscl1| and |cdocscl2|
% which should be identical to
% |cdocsdrf|, |cdocsch1| and |cdocsfn2|, respectively:
% \begin{center}
% \begin{tabular}{l}
% |latex -jobname cdocscld \|\\
% |  "\def\version{draft}\input{childdoc.def}\childdocforward{cdocsamp}"|\\
% |latex -jobname cdocscl1 \|\\
% |  "\input{childdoc.def}\childdocforward[cdocsamp]{cdocsch1}"|\\
% |latex -jobname cdocscl2 \|\\
% |  "\def\version{final}\input{childdoc.def}\childdocforward{cdocsch2}"|
% \end{tabular}
% \end{center}
% Note that the trailing backslash on each first line
% merely continues the input to the second line
% (for convenient cut ant paste).
% Furthermore, the command |latex| can be replaced by any
% of its alternative versions such as |pdflatex|.
%
% %%%%%%%%%%%%%%%%%%%%%%%%%%%%%%%%%%%%%%%%%%%%%%%%%%%%%%%%%%%%%%%%%%%%%%%%%%%%%%
% %%%%%%%%%%%%%%%%%%%%%%%%%%%%%%%%%%%%%%%%%%%%%%%%%%%%%%%%%%%%%%%%%%%%%%%%%%%%%%
% \section{Implementation}
%\iffalse
%<*package>
%\fi
%
% This section describes the definitions file |childdoc.def|.

% The definitions cannot be loaded using |\usepackage| or |\RequirePackage|
% which has a mechanism to prevent loading a style file more than once.
% When loading the definitions by means of |\input|
% multiple instances have to be prevented manually:
%\iffalse
%This code needs to be before the `\ProvidesFile' directive
%which is defined at the beginning of this file.
%Therefore it is also placed there and commented out here.
%</package>
%<*discard>
%\fi
%    \begin{macrocode}
\ifdefined\childdocmain\endinput\fi
%    \end{macrocode}
%\iffalse
%</discard>
%<*package>
%\fi
%
% \macro{\ifchilddoc}
% \macro{\ifchilddocmanual}
% The conditional |\ifchilddoc| tells whether a
% child (true) or main (false) document is being compiled.
% The conditional |\ifchilddocmanual| tells whether
% the |\includeonly| mechanism is used (false) or
% the selection of child files must be performed manually (true).
% The definitions initialise to false:
%    \begin{macrocode}
\newif\ifchilddoc
\newif\ifchilddocmanual
%    \end{macrocode}

% \macro{\childdocname}
% \macro{\childdocjob}
% The macro |\childdocname| stores the name of the main document
% to be compiled. The macro |\childdocjob| stores the name of
% the document on which the \LaTeX{} compiler was originally invoked.
% The content of |\jobname| cannot be compared
% to filenames specified in the source due to different catcodes.
% The following code rescans |\jobname|, stores the result
% in |\childdocname| and saves a copy in |\childdocjob|:
%    \begin{macrocode}
\edef\childdocname{\scantokens\expandafter{\jobname\noexpand}}
\let\childdocjob\childdocname
%    \end{macrocode}

% \macro{\childdocdisable}
% The macro |\childdocdisable| prevents the main file
% from being processed more than once.
% At this stage, the main document command |\childdocmain|
% is assumed to be called once again where it should do nothing.
% Any subsequent call to it should prevent
% a secondary processing of the main document
% It overwrites the forwarding commands
% |\childdocof| and |\childdocforward|
% with empty macros to prevent further inclusions of the main document:
%    \begin{macrocode}
\newcommand{\childdocdisable}
{
  \renewcommand{\childdocmain}[1]{\renewcommand{\childdocmain}[1]{\endinput}}
  \renewcommand{\childdocof}[1]{}
  \renewcommand{\childdocby}[2][]{}
  \renewcommand{\childdocforward}[2][]{}
  \renewcommand{\childdocdisable}{}
}
%    \end{macrocode}

% \macro{\childdocmain}
% The macro |\childdocmain| is to be called at the top of the main file
% with nothing or the main filename (without extension) as argument.
% First, it breaks loops.
% If the argument is not empty and does not match |\childdocname|
% (which is set by the first inclusion of |childdoc.def|),
% |\ifchilddoc| is set to true, |\includeonly| is applied to the child file
% and |\jobname| is set to the main file
% (for proper handling of |.aux| files):
%    \begin{macrocode}
\newcommand{\childdocmain}[1]
{
  \childdocdisable\childdocmain{}
  \if?#1?\else
    \begingroup
      \def\childdoctmp{#1}
      \ifx\childdoctmp\childdocname
        \def\childdoctmp{}
      \else
        \def\childdoctmp
        {
          \childdoctrue
          \includeonly{\childdocname}
          \def\childdocjob{#1}
          \def\jobname{#1}
        }
      \fi
      \expandafter
    \endgroup
    \childdoctmp
  \fi
}
%    \end{macrocode}

% \macro{\childdocof}
% The command |\childdocof| redirects
% compilation to the main file |#1|.
%    \begin{macrocode}
\newcommand{\childdocof}[1]
{
  \childdocdisable
  \childdoctrue
  \includeonly{\childdocname}
  \def\jobname{#1}
  \def\childdocjob{#1}
  \input{#1}
}
%    \end{macrocode}

% \macro{\childdocby}
% The command |\childdocby| ....
%    \begin{macrocode}
\newcommand{\childdocby}[2][]
{
  \childdocdisable
  \childdoctrue
  \childdocmanualtrue
  \if?#1?\else
    \def\jobname{#2}
  \fi
  \def\childdocjob{#2}
  \input{#2}
  \endinput
}
%    \end{macrocode}

% \macro{\childdocforward}
% The command |\childdocforward| redirects
% compilation to the main file or
% (if the optional argument is given) a child file.
% Parameters are set as if the main file
% or a child file starting with |\childdocof| was compiled.
% Then compilation is handed over to the main file:
%    \begin{macrocode}
\newcommand{\childdocforward}[2][]
{
  \begingroup
    \if?#1?
      \def\childdoctmp
      {
        \def\childdocname{#2}
        \def\childdocjob{#2}
        \def\jobname{#2}
        \input{#2}
        \endinput
      }
    \else
      \def\childdoctmp
      {
        \childdocdisable
        \def\childdocname{#2}
        \childdoctrue
        \includeonly{#2}
        \def\childdocjob{#1}
        \def\jobname{#1}
        \input{#1}
        \endinput
      }
    \fi
    \expandafter
  \endgroup
  \childdoctmp
}
%    \end{macrocode}

% \macro{\childdocforwardprefix}
% The command |\childdocforwardprefix| redirects
% compilation to the main or a child file by means of a pattern.
% The prefix |#1| in the current filename is replaced by |#2|
% and the suffix of the current filename is kept
% (it is assumed that the filename does not contain the substring `|~~~|'
% which is used as a delimiter).
% Compilation is handed over to the new file by |\childdocforward|:
%    \begin{macrocode}
\newcommand{\childdocforwardprefix}[3][]
{
  \begingroup
    \def\childdocextract #2##1~~~{\def\childdoctmp{\childdocforward[#1]{#3##1}}}
    \expandafter\childdocextract\childdocname~~~
    \expandafter
  \endgroup
  \childdoctmp
}
%    \end{macrocode}

% \macro{\childdoc}
% The deprecated macro |\childdoc| is a legacy version of |\childdocmain|:
%    \begin{macrocode}
\newcommand{\childdoc}{\childdocmain}
%    \end{macrocode}

% \macro{\childdocredirect}
% The deprecated macro |\childdocredirect| is a legacy version
% of |\childdocforward| and |\childdocforwardprefix|:
%    \begin{macrocode}
\newcommand{\childdocredirect}[2][]
{
  \begingroup
    \if?#1?
      \def\childdoctmp{\childdocforward{#2}}
    \else
      \def\childdoctmp{\childdocforwardprefix{#1}{#2}}
    \fi
    \expandafter
  \endgroup
  \childdoctmp
}
%    \end{macrocode}

%\iffalse
%</package>
%\fi
%
\endinput
|\\
|\childdocforward{|\textit{main}|}|\\
\end{tabular}
\end{center}
%
or alternatively with:
%
\begin{center}
\begin{tabular}{l}
|% \iffalse
%
% childdoc.dtx Copyright (C) 2017-2018 Niklas Beisert
%
% This work may be distributed and/or modified under the
% conditions of the LaTeX Project Public License, either version 1.3
% of this license or (at your option) any later version.
% The latest version of this license is in
%   http://www.latex-project.org/lppl.txt
% and version 1.3 or later is part of all distributions of LaTeX
% version 2005/12/01 or later.
%
% This work has the LPPL maintenance status `maintained'.
%
% The Current Maintainer of this work is Niklas Beisert.
%
% This work consists of the files childdoc.dtx and childdoc.ins
% and the derived files childdoc.def and cdocsamp.tex with
% cdocsch1.tex, cdocsch2.tex, cdocsdrf.tex, cdocsfn1.tex, cdocsfn2.tex.
%
%<package>\ifdefined\childdocmain\endinput\fi
%<package>\ProvidesFile{childdoc.def}[2018/12/30 v2.0 child document driver]
%<samplemain>\ProvidesFile{cdocsamp.tex}[2018/12/30 v2.0 sample for childdoc]
%<*driver>
%\ProvidesFile{childdoc.drv}[2018/12/30 v2.0 childdoc reference manual file]
\PassOptionsToClass{10pt,a4paper}{article}
\documentclass{ltxdoc}

\usepackage[margin=35mm]{geometry}
\usepackage{hyperref}
\usepackage{hyperxmp}
\usepackage[usenames]{color}

\hypersetup{colorlinks=true}
\hypersetup{pdfstartview=FitH}
\hypersetup{pdfpagemode=UseNone}
\hypersetup{pdfsource={}}
\hypersetup{pdflang={en-UK}}
\hypersetup{pdfcopyright={Copyright 2017-2018 Niklas Beisert.
  This work may be distributed and/or modified under the
  conditions of the LaTeX Project Public License, either version 1.3
  of this license or (at your option) any later version.}}
\hypersetup{pdflicenseurl={http://www.latex-project.org/lppl.txt}}
\hypersetup{pdfcontactaddress={ETH Zurich, ITP, HIT K,
  Wolfgang-Pauli-Strasse 27}}
\hypersetup{pdfcontactpostcode={8093}}
\hypersetup{pdfcontactcity={Zurich}}
\hypersetup{pdfcontactcountry={Switzerland}}
\hypersetup{pdfcontactemail={nbeisert@itp.phys.ethz.ch}}
\hypersetup{pdfcontacturl={http://people.phys.ethz.ch/\xmptilde nbeisert/}}

\newcommand{\secref}[1]{\hyperref[#1]{section \ref*{#1}}}

\parskip1ex
\parindent0pt
\let\olditemize\itemize
\def\itemize{\olditemize\parskip0pt}

\begin{document}

\title{The \textsf{childdoc} Package}
\hypersetup{pdftitle={The childdoc Package}}
\author{Niklas Beisert\\[2ex]
  Institut f\"ur Theoretische Physik\\
  Eidgen\"ossische Technische Hochschule Z\"urich\\
  Wolfgang-Pauli-Strasse 27, 8093 Z\"urich, Switzerland\\[1ex]
  \href{mailto:nbeisert@itp.phys.ethz.ch}
  {\texttt{nbeisert@itp.phys.ethz.ch}}}
\hypersetup{pdfauthor={Niklas Beisert}}
\hypersetup{pdfsubject={Manual for the LaTeX2e Package childdoc}}
\date{30 December 2018, \textsf{v2.0}}
\maketitle

\begin{abstract}\noindent
\textsf{childdoc} is a \LaTeXe{} package
that enables the direct compilation
of document sections included by |\include|
to individual files.
\end{abstract}

\begingroup
\parskip0ex
\tableofcontents
\endgroup

%%%%%%%%%%%%%%%%%%%%%%%%%%%%%%%%%%%%%%%%%%%%%%%%%%%%%%%%%%%%%%%%%%%%%%%%%%%%%%%%
%%%%%%%%%%%%%%%%%%%%%%%%%%%%%%%%%%%%%%%%%%%%%%%%%%%%%%%%%%%%%%%%%%%%%%%%%%%%%%%%
\section{Introduction}

\LaTeX{} provides a mechanism to structure a large document (such as a book)
into a main file and several child files (containing the chapters)
using the |\include| command.
This mechanism is beneficial for documents
which span hundreds of pages in order to
make the source file(s) more manageable.
Moreover, compilation can be restricted to
selected child files by means of the |\includeonly| command.
The latter feature can be used to reduce the compilation time while editing
(this was significantly more useful in the earlier days of \LaTeX{})
or to generate a smaller document which is easier to navigate.
Another application of |\includeonly| is to generate
documents consisting of selected parts of the complete document.

However, there are a few drawbacks of the plain |\include| mechanism:
\begin{itemize}
\item
The child files cannot be compiled on their own,
they can only be compiled via the main file.
A naive editing environment
(such as a text editor with an option
to have the current file processed by \LaTeX)
may require one to switch to the main file before compiling;
attempting to compile the child file produces errors.
\item
The main file must be modified (each time)
to adjust the |\includeonly| command
to the present needs. This easily leaves the main file in a messy state.
\item
The generated document will always carry the filename
of the main document. This is inconvenient if
several child files are to be compiled and
to be kept for distribution.
\end{itemize}

The present package provides a simple interface
to make child files individually compilable by \LaTeX{}.
Compiling a child file then has the same effect as compiling
the main file with an |\includeonly| command
to select the appropriate child.
Moreover the generated document will carry the name of the child
rather than the main file.
This resolves all three above issues.

This feature is meant to make the editing of books,
thesis documents and lecture notes somewhat more convenient.
However, the package can also be used efficiently for
composing a series of documents (such as exercise sheets)
which are typically distributed individually.
It then assists the author in generating the individual documents
(potentially in different versions)
as well as a document containing the collected series.
Another application is in developing style files
or other kinds of included material
where compilation of the style file could redirect
to a sample or test file.

%%%%%%%%%%%%%%%%%%%%%%%%%%%%%%%%%%%%%%%%%%%%%%%%%%%%%%%%%%%%%%%%%%%%%%%%%%%%%%%%
%%%%%%%%%%%%%%%%%%%%%%%%%%%%%%%%%%%%%%%%%%%%%%%%%%%%%%%%%%%%%%%%%%%%%%%%%%%%%%%%
\section{Usage}

First of all, the package \textsf{childdoc} is \emph{not} a standard
\LaTeXe{} |.sty| style file! Therefore it needs to be invoked in
a non-standard way.

%%%%%%%%%%%%%%%%%%%%%%%%%%%%%%%%%%%%%%%%%%%%%%%%%%%%%%%%%%%%%%%%%%%%%%%%%%%%%%%%
\subsection{Included Files}
\label{sec:include}

%%%%%%%%%%%%%%%%%%%%%%%%%%%%%%%%%%%%%%%%
\DescribeMacro{\childdocmain}
To use the package, add the commands
\begin{center}
\begin{tabular}{l}
|\input{childdoc.def}|\\
|\childdocmain{}|\\
\end{tabular}
\end{center}
at the very top of the main \LaTeX{} file,
in particular \emph{before} the |\documentclass| statement!
The argument of |\childdocmain| should be left empty
(but it must be present).

%%%%%%%%%%%%%%%%%%%%%%%%%%%%%%%%%%%%%%%%
\DescribeMacro{\childdocof}
Furthermore, add the commands
\begin{center}
\begin{tabular}{l}
|\input{childdoc.def}|\\
|\childdocof{|\textit{main}|}|\\
\end{tabular}
\end{center}
at the top of every child file \textit{child}
which is included by |\include{|\textit{child}|}|
from within the main file
(or at least for those files to be compiled individually).
The argument \textit{main} must be the filename of the main file.

There are a couple of
considerations in setting up the main and child documents:

%%%%%%%%%%%%%%%%%%%%%%%%%%%%%%%%%%%%%%%%
\paragraph{Restrictions.}

Please note the following restrictions:
\begin{itemize}
\item
|\childdocmain| must be called with one argument \textit{main}
to ensure compatibility with earlier version of the package.
It must either be empty (|\childdocmain{}|)
or precisely match the filename of the main file in which it is specified.
See \secref{sec:detection} for further information.
\item
The filename \textit{main} must be specified without the |.tex| extension.
\item
The filename \textit{main} is case sensitive
(even in case-insensitive file systems)
due to internal string comparison.
\item
The argument \textit{main} should be fully expanded, it cannot be a macro.
\item
Subdirectories and special characters should be avoided in filenames.
\item
The command |\childdocmain{|\textit{main}|}| must be followed by a whitespace.
It should not be followed immediately by another command
or by a comment mark `|%|'.
This is because the \TeX{} parser reads the token immediately following
the argument of |\childdocmain| and puts it
at the beginning of every child section;
however, a white\-space is ignored.
\end{itemize}

%%%%%%%%%%%%%%%%%%%%%%%%%%%%%%%%%%%%%%%%
\paragraph{Content of Main File.}

It is advisable to place all content in the child files included by |\include|.
Any output contained in the main file will appear in all child documents
unless suppressed manually;
it cannot be suppressed automatically by the |\includeonly| directive
and thus should normally be avoided.
A method to include some content in the main file
by means of conditional processing is described in \secref{sec:conditional}.

%%%%%%%%%%%%%%%%%%%%%%%%%%%%%%%%%%%%%%%%
\paragraph{Page Numbering.}

When only a part of the document is compiled,
the appropriate numbering of pages
(as well as other status parameters)
is determined from the |.aux| files.
The latter contain information from previous passes.
However this information needs to propagate through
all intermediate child documents.
Therefore the page numbering in child documents may well
be inconsistent until the complete document is compiled at least once.

A useful (if unconventional) way to always ensure a consistent
page numbering is to restart the numbering in each child document
and denote the pages by `\textit{child}|.|\textit{page}'
where \textit{child} represents the chapter/section number of the child file.
This can be achieved by the command
|\numberwithin{page}{|\textit{child}|}|
of the \textsf{amsmath} package
where \textit{child} can be |chapter| or |section|
depending on the chosen structuring.
Alternatively, one can modify the macro |\thepage| appropriately
and reset the counter |page| at the start of each child file.

%%%%%%%%%%%%%%%%%%%%%%%%%%%%%%%%%%%%%%%%%%%%%%%%%%%%%%%%%%%%%%%%%%%%%%%%%%%%%%%%
\subsection{Conditional Processing}
\label{sec:conditional}

The package provides a mechanism to compile different versions
of a document. To customise the versions further some conditional processing
can come in handy to distinguish which version is being compiled.
The package provides two macros to describe the compilation context:

%%%%%%%%%%%%%%%%%%%%%%%%%%%%%%%%%%%%%%%%
\DescribeMacro{\ifchilddoc}
The conditional |\ifchilddoc| distinguishes between the compilation of
child documents and the main document:
%
\begin{center}
|\ifchilddoc |\textit{child-code}| |[|\||else |\textit{main-code}]| \||fi|
\end{center}

%%%%%%%%%%%%%%%%%%%%%%%%%%%%%%%%%%%%%%%%
\DescribeMacro{\childdocname}
\DescribeMacro{\childdocjob}
The macro |\childdocname| contains the filename (without extension)
of the main or child file being processed.
Note that |\childdocjob| will always contain the name of the main file.

%%%%%%%%%%%%%%%%%%%%%%%%%%%%%%%%%%%%%%%%
\paragraph{Title Page.}

Conditional processing can be used to include a title or banner page
in the main document when proper precautions are taken.
Importantly, the code in the main file should ensure that the page counter
(as well as other status parameters which are stored in the |.aux| files)
takes the same value after the conditional processing.
Otherwise the page numbers may take divergent values
depending on which part is compiled.

For example, a title page could be declared by:
%
\begin{center}
\begin{tabular}{l}
|\ifchilddoc\||else|\\
|\addtocounter{page}{-1}|\\
\textit{code for title page}\\
|\newpage|\\
|\||fi|
\end{tabular}
\end{center}
%
A banner page for the child documents can be generated by:
%
\begin{center}
\begin{tabular}{l}
|\ifchilddoc|\\
|\addtocounter{page}{-1}|\\
\textit{code for banner page}\\
|\newpage|\\
|\||fi|
\end{tabular}
\end{center}
%
Here one could write a message such as:
\begin{center}
|This is the part \childdocname{} of \childdocjob{}.|
\end{center}

%%%%%%%%%%%%%%%%%%%%%%%%%%%%%%%%%%%%%%%%%%%%%%%%%%%%%%%%%%%%%%%%%%%%%%%%%%%%%%%%
\subsection{Flags}
\label{sec:flags}

The package makes it easy to generate different versions
of the main or child documents.
To this end compilation flags can be defined
and assigned different default values.
They will be particularly useful in conjunction
with the forwarding mechanism described in \secref{sec:forward}.

For example, it may be useful to have a flag |\version|
which can be set to |draft| or |final|.
The document source will contain some conditional code
depending on the value of |\version|.
Suppose further, the flag should default to |final| for the main file
and to |draft| for child files
which is a natural assignment for editing the document.
This is achieved by placing the following code
in the preamble of the main document
(below the |\childdocmain| directive):
%
\begin{center}
\begin{tabular}{l}
|\ifchilddoc|\\
|\providecommand{\version}{draft}|\\
|\||else|\\
|\providecommand{\version}{final}|\\
|\||fi|
\end{tabular}
\end{center}
%
The definition by |\providecommand| makes sure
that previous definitions are not overwritten.
Further statements |\providecommand{\version}{...}|
can thus be added before the above code to override it.

For the main file, one might add a line
(between |\childdocmain| and the above block)
%
\begin{center}
|%\ifchilddoc\||else\providecommand{\version}{draft}\||fi|
\end{center}
%
which can be uncommented to produce a draft version.
Likewise one can add a line to the very top of a child file
(above the |\childdocof{|\textit{main}|}| directive)
%
\begin{center}
|%\providecommand{\version}{final}|
\end{center}
%
which can be uncommented to produce the final version of this child document.

%%%%%%%%%%%%%%%%%%%%%%%%%%%%%%%%%%%%%%%%%%%%%%%%%%%%%%%%%%%%%%%%%%%%%%%%%%%%%%%%
\subsection{Forwarding}
\label{sec:forward}

Different versions of the main or child documents
using compilation flags as described in \secref{sec:flags}
can be (permanently) stored in different files
for convenient compilation, viewing and distribution.
To this end, the package defines a command
to pass on compilation to a different file:

%%%%%%%%%%%%%%%%%%%%%%%%%%%%%%%%%%%%%%%%
\DescribeMacro{\childdocforward}
The command |\childdocforward| redirects processing to
another source file:
%
\begin{center}
\begin{tabular}{l}
|\input{childdoc.def}|\\
|\childdocforward[|\textit{main}|]{|\textit{dest}|}|\\
\end{tabular}
\end{center}
%
The argument \textit{dest} is the destination file
(without extension).
It should be the main file or one of the child files.
Note that further \textsf{childdoc} directives
such as |\childdocof| and |\childdocforward|
in the indicated file will be processed in this form.
The optional argument \textit{main}
passes on directly to the main file \textit{main}
while pretending to compile the child \textit{dest}.
This form behaves as if \textit{dest}
issues |\childdocof{|\textit{main}|}| right away,
and no further \textsf{childdoc} directives will be processed.

%%%%%%%%%%%%%%%%%%%%%%%%%%%%%%%%%%%%%%%%
\DescribeMacro{\...prefix}
In the alternative form |\childdocforwardprefix|,
%
\begin{center}
\begin{tabular}{l}
|\input{childdoc.def}|\\
|\childdocforwardprefix[|\textit{main}|]{|\textit{prefix}|}{|\textit{dest}|}|
\end{tabular}
\end{center}
%
the destination file is determined by a pattern
depending on the current file:
To make this work, the current file must be called
`{\textit{prefix}\hspace{0.2em}\textit{suffix}}'
with \textit{prefix} matching precisely the argument.
Processing is then passed on to the file
`{\textit{dest}\hspace{0.2em}\textit{suffix}}'.
Surely, the same effect is achieved by
directly specifying the
argument `{\textit{dest}\hspace{0.2em}\textit{suffix}}'
in the first form.
However, that requires to set up a different file
for each child. With the alternative form of the command
all these files can have exactly the same content
which simplifies setting them up and maintaining them.

For example, the following file |draft.tex|
with a compilation flag |\version| as described in \secref{sec:flags}
compiles the main document as a draft:
%
\begin{center}
\begin{tabular}{l}
|\def\version{draft}|\\
|\input{childdoc.def}|\\
|\childdocforward{|\textit{main}|}|
\end{tabular}
\end{center}
%
Likewise, the following files |final|\textit{nn}|.tex|
compile the final version of the child document
|child|\textit{nn}|.tex|:
%
\begin{center}
\begin{tabular}{l}
|\def\version{final}|\\
|\input{childdoc.def}|\\
|\childdocforwardprefix{final}{child}|
\end{tabular}
\end{center}
%

Note that when several versions of a main file and/or of each child file
are to be generated, it may be convenient to set up a |Makefile| or
shell script to automatise the process.

%%%%%%%%%%%%%%%%%%%%%%%%%%%%%%%%%%%%%%%%%%%%%%%%%%%%%%%%%%%%%%%%%%%%%%%%%%%%%%%%
\subsection{Command Line Processing}
\label{sec:commandline}

The effect of redirection files can also be achieved by invoking
the \LaTeX{} compiler with a more elaborate command line.
Most conveniently this should be done as part
of a shell script or a |Makefile|.

When using \textsf{childdoc} in the main file, the following
command lines effectively perform a redirection
(note that depending on the shell being used,
backslashes may have to be doubled: `|\|' $\to$ `|\\|'):
%
\begin{center}
|... -jobname "|\textit{target}|" |\\|"|[\textit{flags}]%
|\input{childdoc.def}\childdocforward[|\textit{main}|]{|\textit{dest}|}"|
\end{center}
%
Here \textit{target} is the name of the output file,
\textit{main} is the name of the main file
and \textit{dest} is the name of the main or child file to be processed
(all filenames without extensions).
The optional argument \textit{main} can be omitted
if \textit{main} matches \textit{dest}.
Optionally, compilation \textit{flags} can be defined via |\def| commands.
This command line makes the \TeX{} engine believe
it is compiling the file \textit{target}
whose content is specified as the latter parameter.
The provided code then forwards the processing to
\textit{main} or \textit{dest} as described in \secref{sec:forward}.

%%%%%%%%%%%%%%%%%%%%%%%%%%%%%%%%%%%%%%%%%%%%%%%%%%%%%%%%%%%%%%%%%%%%%%%%%%%%%%%%
\subsection{Include by Input}
\label{sec:input}

Including child documents by |\include| has some restrictions by design.
Most notably, the content of a child document always occupies
its own set of pages; pages cannot be shared between child documents.
Usually, this behaviour makes perfect sense
because each child document contain an essential part of the document.
However, in some situations it may be desirable to compose
a document from a collection of parts
without having mandatory page breaks between then.
For this case, the package
provides a mechanism to include parts
by |\input| which can also be processed individually.
However, by construction this mechanism
requires manual handling of the content to be output.

%%%%%%%%%%%%%%%%%%%%%%%%%%%%%%%%%%%%%%%%
\DescribeMacro{\ifchilddocmanual}
The main file should be prepared as usual, see \secref{sec:include}.
However, the document body must make a distinction
between processing of an individual part and of the main document, e.g.:
%
\begin{center}
\begin{tabular}{l}
|\ifchilddocmanual|\\
|\input{\childdocname}|\\
|\||else|\\
\textit{document body with }|\input{|\textit{part}|}|\\
|\||fi|
\end{tabular}
\end{center}
%
The conditional |\ifchilddocmanual| is true whenever
a part to be included by |\input| is being compiled,
and the name of the part is stored in |\childdocname|.

%%%%%%%%%%%%%%%%%%%%%%%%%%%%%%%%%%%%%%%%
\DescribeMacro{\childdocby}
Each part to be included by |\input| should start with:
%
\begin{center}
\begin{tabular}{l}
|\input{childdoc.def}|\\
|\childdocby{|\textit{main}|}|\\
\end{tabular}
\end{center}
%
The directive |\childdocby| is similar to |\childdocof|
described in \secref{sec:include},
but the subsequent selection of content must be done manually.
To that end, both |\ifchilddoc| and |\ifchilddocmanual|
will be true upon processing of a part,
and the name of the part is stored in |\childdocname|.
Note that |\jobname| will be set to the filename of the current part
so that each part receives an individual |.aux| file
that does not interfere with the |.aux| file(s) of the main document.
This behaviour can be altered by the alternative form
|\childdocby[*]{|\textit{main}|}| (with a non-empty optional argument)
which uses the |.aux| file of the main document
by setting |\jobname| to \textit{main}.

%%%%%%%%%%%%%%%%%%%%%%%%%%%%%%%%%%%%%%%%%%%%%%%%%%%%%%%%%%%%%%%%%%%%%%%%%%%%%%%%
\subsection{Driver Development}
\label{sec:driver}

The \textsf{childdoc} mechanism can also be use for the development
of definition files such as \LaTeX{} styles or classes.
This case differs from the above setup with multiple parts
included by |\include| in that no |\includeonly| should be invoked.
This can be achieved by starting the include file
(before |\ProvidesPackage|) with:
%
\begin{center}
\begin{tabular}{l}
|\input{childdoc.def}|\\
|\childdocforward{|\textit{main}|}|\\
\end{tabular}
\end{center}
%
or alternatively with:
%
\begin{center}
\begin{tabular}{l}
|\input{childdoc.def}|\\
|\childdocby{|\textit{main}|}|\\
\end{tabular}
\end{center}
%
Both forms have slightly different effects as described above.
The main file is prepared as usual, see \secref{sec:include}.

%%%%%%%%%%%%%%%%%%%%%%%%%%%%%%%%%%%%%%%%%%%%%%%%%%%%%%%%%%%%%%%%%%%%%%%%%%%%%%%%
\subsection{Legacy Detection}
\label{sec:detection}

The directive |\childdocmain| in the main file can detect
whether the complete document or merely a child is to be compiled
even without using the directive |\childdocof|.
This method is deprecated because it is less robust
and there is no compelling reason to use it;
it is merely provided for backward compatibility
and it may be removed in future versions.

If the detection mechanism is to be used,
it is mandatory to correctly specify
the filename of the main file as the argument of |\childdocmain|:
%
\begin{center}
\begin{tabular}{l}
|\input{childdoc.def}|\\
|\childdocmain{|\textit{main}|}|\\
\end{tabular}
\end{center}
%
If |\jobname| does not match the argument \textit{main} of |\childdocmain|,
it is assumed that |\jobname| points to the child file to be compiled.
When using |\childdocmain| with the main file specified as argument,
it suffices to start a child file
with just |\input{|\textit{main}|}|
without loading of the package and using |\childdocof|.
If instead all processing is done
with the appropriate \textsf{childdoc} directives,
the argument of \textit{main} of |\childdocmain| can be empty.

An alternative version of the command line processing described
in \secref{sec:commandline} using the detection mechanism reads:
%
\begin{center}
|... -jobname "|\textit{target}|" "|[\textit{flags}]%
[|\def\jobname{|\textit{dest}|}|]|\input{|\textit{main}|}"|
\end{center}

%%%%%%%%%%%%%%%%%%%%%%%%%%%%%%%%%%%%%%%%%%%%%%%%%%%%%%%%%%%%%%%%%%%%%%%%%%%%%%%%
\subsection{Manual Code}
\label{sec:manual}

In case one cannot be certain whether the definitions file |childdoc.def|
is installed on the target \TeX{} distribution
and one prefers not to ship it,
it is conceivable to paste a few relevant commands into the sources.

To that end, drop all statements |\input{childdoc.def}|
and perform the replacements as outlined below.
Instead of |\childdocmain{|\textit{main}|}| add the following code
to the top of the main file:
%
\begin{center}
\begin{tabular}{l}
|\||ifdefined\childdocname\endinput\||fi\newif\ifchilddoc|\\
|\edef\childdocname{\scantokens\expandafter{\jobname\noexpand}}|\\
|\def\childdocmain{|\textit{main}|}\||ifx\childdocmain\childdocname\||else|\\
|\childdoctrue\includeonly{\childdocname}\let\jobname\childdocmain\||fi|\\
\end{tabular}
\end{center}
%
Instead of |\childdocof{|\textit{main}|}| just include the main file
at the top of each child file:
%
\begin{center}
|\input{|\textit{main}|}|
\end{center}
%
A simple redirection |\childdocforward{|\textit{dest}|}| is achieved by:
%
\begin{center}
|\def\jobname{|\textit{dest}|}\input{\jobname}|
\end{center}
%
The redirection with prefix
|\childdocforwardprefix[|\textit{prefix}|]{|\textit{dest}|}|
is accomplished by:
%
\begin{center}
\begin{tabular}{l}
|{\edef\jobname{\scantokens\expandafter{\jobname\noexpand}}|\\
|\def\redirectjob |\textit{prefix}|#1~~~{\gdef\jobname{|\textit{dest}|#1}}|\\
|\expandafter\redirectjob\jobname~~~}\input{\jobname}|
\end{tabular}
\end{center}

In an alternative approach,
child documents can be compiled by a specific command line
without additional code or specific definitions:
%
\begin{center}
|... -jobname "|\textit{target}|" "|[\textit{flags}]%
|\includeonly{|\textit{dest}|}\input{|\textit{main}|}"|
\end{center}
%

%%%%%%%%%%%%%%%%%%%%%%%%%%%%%%%%%%%%%%%%%%%%%%%%%%%%%%%%%%%%%%%%%%%%%%%%%%%%%%%%
%%%%%%%%%%%%%%%%%%%%%%%%%%%%%%%%%%%%%%%%%%%%%%%%%%%%%%%%%%%%%%%%%%%%%%%%%%%%%%%%
\section{Information}

%%%%%%%%%%%%%%%%%%%%%%%%%%%%%%%%%%%%%%%%%%%%%%%%%%%%%%%%%%%%%%%%%%%%%%%%%%%%%%%%
\subsection{Copyright}

Copyright \copyright{} 2017--2018 Niklas Beisert

This work may be distributed and/or modified under the
conditions of the \LaTeX{} Project Public License, either version 1.3
of this license or (at your option) any later version.
The latest version of this license is in
  \url{http://www.latex-project.org/lppl.txt}
and version 1.3 or later is part of all distributions of \LaTeX{}
version 2005/12/01 or later.

This work has the LPPL maintenance status `maintained'.

The Current Maintainer of this work is Niklas Beisert.

This work consists of the files |README.txt|, |childdoc.ins| and |childdoc.dtx|
as well as the derived files |childdoc.def|, |cdocsamp.tex|
with |cdocsch1.tex|, |cdocsch2.tex|, |cdocspt3.tex|, |cdocspt4.tex|,
|cdocsdrf.tex|, |cdocsfn1.tex|, |cdocsfn2.tex|
as well as |childdoc.pdf|.

%%%%%%%%%%%%%%%%%%%%%%%%%%%%%%%%%%%%%%%%%%%%%%%%%%%%%%%%%%%%%%%%%%%%%%%%%%%%%%%%
\subsection{Files and Installation}

The package consists of the files:
%
\begin{center}
\begin{tabular}{ll}
    |README.txt|   & readme file \\
    |childdoc.ins| & installation file \\
    |childdoc.dtx| & source file \\
    |childdoc.def| & definition file \\
    |cdocsamp.tex| & sample main file \\
    |cdocsch1.tex| & sample include file \\
    |cdocsch2.tex| & sample include file \\
    |cdocspt3.tex| & sample part file \\
    |cdocspt4.tex| & sample part file \\
    |cdocsdrf.tex| & sample redirection file \\
    |cdocsfn1.tex| & sample redirection file \\
    |cdocsfn2.tex| & sample redirection file \\
    |childdoc.pdf| & manual
\end{tabular}
\end{center}
%
The distribution consists of the files
|README.txt|, |childdoc.ins| and |childdoc.dtx|.
%
\begin{itemize}
\item
Run (pdf)\LaTeX{} on |childdoc.dtx|
to compile the manual |childdoc.pdf| (this file).
\item
Run \LaTeX{} on |childdoc.ins| to create the definitions file |childdoc.def|
and the sample |cdocsamp.tex| with include files
|cdocsch1.tex|, |cdocsch2.tex|, |cdocspt3.tex|, |cdocspt4.tex|,
|cdocsdrf.tex|, |cdocsfn1.tex|, |cdocsfn2.tex|.
Then copy the file |childdoc.def| to an appropriate directory of your \LaTeX{}
distribution, e.g.\ \textit{texmf-root}|/tex/latex/childdoc|.
\end{itemize}

%%%%%%%%%%%%%%%%%%%%%%%%%%%%%%%%%%%%%%%%%%%%%%%%%%%%%%%%%%%%%%%%%%%%%%%%%%%%%%%%
\subsection{Related CTAN Packages}

There are several other packages which offer a similar functionality:
%
\begin{itemize}
\item
The packages
\href{http://ctan.org/pkg/docmute}{\textsf{docmute}},
\href{http://ctan.org/pkg/includex}{\textsf{includex}} and
\href{http://ctan.org/pkg/standalone}{\textsf{standalone}}
provide commands to include only the document body of
a child file thus allowing both files to be compiled individually.
\item
The packages \href{http://ctan.org/pkg/subdocs}{\textsf{subdocs}}
and \href{http://ctan.org/pkg/subfiles}{\textsf{subfiles}}
provide structures in which the main and child documents can be
encapsulated and allowing them to be compiled individually.
The inclusion mechanism is different from the conventional |\include|.
\item
The package \href{http://ctan.org/pkg/combine}{\textsf{combine}}
is an elaborate solution to combine several documents into one.
\end{itemize}
%
See also the CTAN topic \href{http://ctan.org/topic/subdocs}{\textsf{subdocs}}
for further related packages.
The present package differs from the above solutions in that
a document structure constructed with the conventional |\include| mechanism
just needs two extra commands at the top of every file
such that all constituent files can be compiled individually.

%%%%%%%%%%%%%%%%%%%%%%%%%%%%%%%%%%%%%%%%%%%%%%%%%%%%%%%%%%%%%%%%%%%%%%%%%%%%%%%%
%\subsection{Feature Suggestions}
%
%The following is a list of features which may be useful for future
%versions of this package:
%%
%\begin{itemize}
%\item
%\ldots
%\end{itemize}

%%%%%%%%%%%%%%%%%%%%%%%%%%%%%%%%%%%%%%%%%%%%%%%%%%%%%%%%%%%%%%%%%%%%%%%%%%%%%%%%
\subsection{Revision History}

%%%%%%%%%%%%%%%%%%%%%%%%%%%%%%%%%%%%%%%%
\paragraph{v2.0:} 2018/12/30

\begin{itemize}
\item
immediate forward processing
\item
added |\childdocby| mechanism
\item
manual restructured
\end{itemize}

%%%%%%%%%%%%%%%%%%%%%%%%%%%%%%%%%%%%%%%%
\paragraph{v1.6:} 2018/01/17

\begin{itemize}
\item
application for development of include files
\item
corrections to manual
\end{itemize}

%%%%%%%%%%%%%%%%%%%%%%%%%%%%%%%%%%%%%%%%
\paragraph{v1.5:} 2017/05/21

\begin{itemize}
\item
more complete structuring introduced
\item
|\childdocof| introduced
\item
|\childdoc| renamed to |\childdocmain|
\item
|\childredirect| renamed to |\childdocforward| and |\childdocforwardprefix|
and functionality expanded
\end{itemize}

%%%%%%%%%%%%%%%%%%%%%%%%%%%%%%%%%%%%%%%%
\paragraph{v1.0:} 2017/04/27

\begin{itemize}
\item
manual and install package
\item
first version published on CTAN
\end{itemize}

%%%%%%%%%%%%%%%%%%%%%%%%%%%%%%%%%%%%%%%%
\paragraph{v0.6:} 2017/04/26

\begin{itemize}
\item
redirection mechanism added
\end{itemize}

%%%%%%%%%%%%%%%%%%%%%%%%%%%%%%%%%%%%%%%%
\paragraph{v0.5:} 2017/04/26

\begin{itemize}
\item
functionality in definition file
\end{itemize}


%%%%%%%%%%%%%%%%%%%%%%%%%%%%%%%%%%%%%%%%%%%%%%%%%%%%%%%%%%%%%%%%%%%%%%%%%%%%%%%%
%%%%%%%%%%%%%%%%%%%%%%%%%%%%%%%%%%%%%%%%%%%%%%%%%%%%%%%%%%%%%%%%%%%%%%%%%%%%%%%%
%%%%%%%%%%%%%%%%%%%%%%%%%%%%%%%%%%%%%%%%%%%%%%%%%%%%%%%%%%%%%%%%%%%%%%%%%%%%%%%%
\appendix

\settowidth\MacroIndent{\rmfamily\scriptsize 000\ }

 \DocInput{childdoc.dtx}

\end{document}
%</driver>
% \fi
%
% %%%%%%%%%%%%%%%%%%%%%%%%%%%%%%%%%%%%%%%%%%%%%%%%%%%%%%%%%%%%%%%%%%%%%%%%%%%%%%
% %%%%%%%%%%%%%%%%%%%%%%%%%%%%%%%%%%%%%%%%%%%%%%%%%%%%%%%%%%%%%%%%%%%%%%%%%%%%%%
% \section{Sample}
%\iffalse
%<*samplemain>
%\fi
%
% The following presents a sample document
% with two chapters, two parts, a title page,
% a compile flag as well as three forwarding files to set the flag.
% It consists of eight |.tex| files:
% \begin{center}
% \begin{tabular}{ll}
% |cdocsamp.tex|&main file\\
% |cdocsch1.tex|&include file for chapter 1\\
% |cdocsch2.tex|&include file for chapter 2\\
% |cdocspt3.tex|&include file for part 3\\
% |cdocspt4.tex|&include file for part 4\\
% |cdocsdrf.tex|&forwarding file for main file in draft mode\\
% |cdocsfi1.tex|&forwarding file for final version of chapter 1\\
% |cdocsfi2.tex|&forwarding file for final version of chapter 2\\
% \end{tabular}
% \end{center}
% Each of the eight files can be compiled directly by the \LaTeX{} compiler.
%
% %%%%%%%%%%%%%%%%%%%%%%%%%%%%%%%%%%%%%%
% \paragraph{Main File.}
%
% The main file is called |cdocsamp.tex|.
%
% Load the \textsf{childdoc} definitions and
% declare the filename for the main document:
%    \begin{macrocode}
\input{childdoc.def}
\childdocmain{}
%    \end{macrocode}

% Optional override for |\version| flag:
%    \begin{macrocode}
%%\ifchilddoc\else\providecommand{\version}{draft}\fi
%    \end{macrocode}

% Define the default values for the |\version| flag
% (|final| for the main file and |draft| for childs):
%    \begin{macrocode}
\ifchilddoc
\providecommand{\version}{draft}
\else
\providecommand{\version}{final}
\fi
%    \end{macrocode}

% Load the standard document class:
%    \begin{macrocode}
\documentclass[12pt]{article}
%    \end{macrocode}

% Start the document body:
%    \begin{macrocode}
\begin{document}
%    \end{macrocode}

% Declare a title page.
% Print title, part of document being processed and version flag:
%    \begin{macrocode}
\addtocounter{page}{-1}
\begin{center}
{\LARGE\bfseries{}childdoc example\par}
\vspace{1cm}
\ifchilddoc
\ifchilddocmanual part\else chapter\fi:
`\childdocname' of `\childdocjob'\par
\else
main document: `\childdocjob'\par
\fi
version: \version\par
\end{center}
\newpage
%    \end{macrocode}

% Manually include selected file,
% otherwise process as usual:
%    \begin{macrocode}
\ifchilddocmanual
\section*{part `\childdocname'}
\input{\childdocname}
\else
%    \end{macrocode}

% Include the two chapters:
%    \begin{macrocode}
\include{cdocsch1}
\include{cdocsch2}
%    \end{macrocode}

% Include the two parts unless only chapters should be displayed:
%    \begin{macrocode}
\ifchilddoc\else
\section{part three}
\input{cdocspt3}
\section{part four}
\input{cdocspt4}
\fi
%    \end{macrocode}

% Process as usual until here:
%    \begin{macrocode}
\fi
%    \end{macrocode}

% End of document body:
%    \begin{macrocode}
\end{document}
%    \end{macrocode}
%\iffalse
%</samplemain>
%\fi
%
% %%%%%%%%%%%%%%%%%%%%%%%%%%%%%%%%%%%%%%
% \paragraph{Chapter Include Files.}
%
% The include files are called |cdocsch1.tex| and |cdocsch2.tex|.
%
%\iffalse
%<*samplechap1|samplechap2>
%\fi

% Optional override for |\version| flag:
%    \begin{macrocode}
%%\providecommand{\version}{final}
%    \end{macrocode}

% Include the main document:
%    \begin{macrocode}
\input{childdoc.def}
\childdocof{cdocsamp}
%    \end{macrocode}

%\iffalse
%</samplechap1|samplechap2>
%\fi
%
%\iffalse
%<*samplechap1>
%\fi
% Some text for chapter 1:
%    \begin{macrocode}
\section{one}
some text in chapter one
%    \end{macrocode}

%\iffalse
%</samplechap1>
%\fi
% Some text for chapter 2:
%\iffalse
%<*samplechap2>
%\fi
%    \begin{macrocode}
\section{two}
more text in chapter two
%    \end{macrocode}

%\iffalse
%</samplechap2>
%\fi
%
% %%%%%%%%%%%%%%%%%%%%%%%%%%%%%%%%%%%%%%
% \paragraph{Part Include Files.}
%
% The include files are called |cdocspt3.tex| and |cdocspt4.tex|.
%
%\iffalse
%<*samplepart3|samplepart4>
%\fi

% Optional override for |\version| flag:
%    \begin{macrocode}
%%\providecommand{\version}{final}
%    \end{macrocode}

% Include the main document:
%    \begin{macrocode}
\input{childdoc.def}
\childdocby{cdocsamp}
%    \end{macrocode}

%\iffalse
%</samplepart3|samplepart4>
%\fi
%
%\iffalse
%<*samplepart3>
%\fi
% Some text for part 3:
%    \begin{macrocode}
some text in part three
%    \end{macrocode}

%\iffalse
%</samplepart3>
%\fi
% Some text for part 4:
%\iffalse
%<*samplepart4>
%\fi
%    \begin{macrocode}
more text in part four
%    \end{macrocode}

%\iffalse
%</samplepart4>
%\fi
%
% %%%%%%%%%%%%%%%%%%%%%%%%%%%%%%%%%%%%%%
% \paragraph{Forwarding for a Complete Draft.}
%
% The following forwarding file |cdocsdrf.tex|
% compiles the main document in draft mode:
%\iffalse
%<*sampledraft>
%\fi
%    \begin{macrocode}
\def\version{draft}
\input{childdoc.def}
\childdocforward{cdocsamp}
%    \end{macrocode}

%\iffalse
%</sampledraft>
%\fi
%
% %%%%%%%%%%%%%%%%%%%%%%%%%%%%%%%%%%%%%%
% \paragraph{Forwarding for Final Version of the Chapters.}
%
% The following forwarding files |cdocsfn1.tex| and |cdocsfn2.tex|
% (with identical content)
% compile the final versions of the child documents
% |cdocsch1.tex| and |cdocsch2.tex|, respectively:
%\iffalse
%<*samplefinal>
%\fi
%    \begin{macrocode}
\def\version{final}
\input{childdoc.def}
\childdocforwardprefix[cdocsamp]{cdocsfn}{cdocsch}
%    \end{macrocode}

%\iffalse
%</samplefinal>
%\fi
%
% %%%%%%%%%%%%%%%%%%%%%%%%%%%%%%%%%%%%%%
% \paragraph{Command Line Processing.}
%
% The following three command lines generate the output files
% |cdocscld|, |cdocscl1| and |cdocscl2|
% which should be identical to
% |cdocsdrf|, |cdocsch1| and |cdocsfn2|, respectively:
% \begin{center}
% \begin{tabular}{l}
% |latex -jobname cdocscld \|\\
% |  "\def\version{draft}\input{childdoc.def}\childdocforward{cdocsamp}"|\\
% |latex -jobname cdocscl1 \|\\
% |  "\input{childdoc.def}\childdocforward[cdocsamp]{cdocsch1}"|\\
% |latex -jobname cdocscl2 \|\\
% |  "\def\version{final}\input{childdoc.def}\childdocforward{cdocsch2}"|
% \end{tabular}
% \end{center}
% Note that the trailing backslash on each first line
% merely continues the input to the second line
% (for convenient cut ant paste).
% Furthermore, the command |latex| can be replaced by any
% of its alternative versions such as |pdflatex|.
%
% %%%%%%%%%%%%%%%%%%%%%%%%%%%%%%%%%%%%%%%%%%%%%%%%%%%%%%%%%%%%%%%%%%%%%%%%%%%%%%
% %%%%%%%%%%%%%%%%%%%%%%%%%%%%%%%%%%%%%%%%%%%%%%%%%%%%%%%%%%%%%%%%%%%%%%%%%%%%%%
% \section{Implementation}
%\iffalse
%<*package>
%\fi
%
% This section describes the definitions file |childdoc.def|.

% The definitions cannot be loaded using |\usepackage| or |\RequirePackage|
% which has a mechanism to prevent loading a style file more than once.
% When loading the definitions by means of |\input|
% multiple instances have to be prevented manually:
%\iffalse
%This code needs to be before the `\ProvidesFile' directive
%which is defined at the beginning of this file.
%Therefore it is also placed there and commented out here.
%</package>
%<*discard>
%\fi
%    \begin{macrocode}
\ifdefined\childdocmain\endinput\fi
%    \end{macrocode}
%\iffalse
%</discard>
%<*package>
%\fi
%
% \macro{\ifchilddoc}
% \macro{\ifchilddocmanual}
% The conditional |\ifchilddoc| tells whether a
% child (true) or main (false) document is being compiled.
% The conditional |\ifchilddocmanual| tells whether
% the |\includeonly| mechanism is used (false) or
% the selection of child files must be performed manually (true).
% The definitions initialise to false:
%    \begin{macrocode}
\newif\ifchilddoc
\newif\ifchilddocmanual
%    \end{macrocode}

% \macro{\childdocname}
% \macro{\childdocjob}
% The macro |\childdocname| stores the name of the main document
% to be compiled. The macro |\childdocjob| stores the name of
% the document on which the \LaTeX{} compiler was originally invoked.
% The content of |\jobname| cannot be compared
% to filenames specified in the source due to different catcodes.
% The following code rescans |\jobname|, stores the result
% in |\childdocname| and saves a copy in |\childdocjob|:
%    \begin{macrocode}
\edef\childdocname{\scantokens\expandafter{\jobname\noexpand}}
\let\childdocjob\childdocname
%    \end{macrocode}

% \macro{\childdocdisable}
% The macro |\childdocdisable| prevents the main file
% from being processed more than once.
% At this stage, the main document command |\childdocmain|
% is assumed to be called once again where it should do nothing.
% Any subsequent call to it should prevent
% a secondary processing of the main document
% It overwrites the forwarding commands
% |\childdocof| and |\childdocforward|
% with empty macros to prevent further inclusions of the main document:
%    \begin{macrocode}
\newcommand{\childdocdisable}
{
  \renewcommand{\childdocmain}[1]{\renewcommand{\childdocmain}[1]{\endinput}}
  \renewcommand{\childdocof}[1]{}
  \renewcommand{\childdocby}[2][]{}
  \renewcommand{\childdocforward}[2][]{}
  \renewcommand{\childdocdisable}{}
}
%    \end{macrocode}

% \macro{\childdocmain}
% The macro |\childdocmain| is to be called at the top of the main file
% with nothing or the main filename (without extension) as argument.
% First, it breaks loops.
% If the argument is not empty and does not match |\childdocname|
% (which is set by the first inclusion of |childdoc.def|),
% |\ifchilddoc| is set to true, |\includeonly| is applied to the child file
% and |\jobname| is set to the main file
% (for proper handling of |.aux| files):
%    \begin{macrocode}
\newcommand{\childdocmain}[1]
{
  \childdocdisable\childdocmain{}
  \if?#1?\else
    \begingroup
      \def\childdoctmp{#1}
      \ifx\childdoctmp\childdocname
        \def\childdoctmp{}
      \else
        \def\childdoctmp
        {
          \childdoctrue
          \includeonly{\childdocname}
          \def\childdocjob{#1}
          \def\jobname{#1}
        }
      \fi
      \expandafter
    \endgroup
    \childdoctmp
  \fi
}
%    \end{macrocode}

% \macro{\childdocof}
% The command |\childdocof| redirects
% compilation to the main file |#1|.
%    \begin{macrocode}
\newcommand{\childdocof}[1]
{
  \childdocdisable
  \childdoctrue
  \includeonly{\childdocname}
  \def\jobname{#1}
  \def\childdocjob{#1}
  \input{#1}
}
%    \end{macrocode}

% \macro{\childdocby}
% The command |\childdocby| ....
%    \begin{macrocode}
\newcommand{\childdocby}[2][]
{
  \childdocdisable
  \childdoctrue
  \childdocmanualtrue
  \if?#1?\else
    \def\jobname{#2}
  \fi
  \def\childdocjob{#2}
  \input{#2}
  \endinput
}
%    \end{macrocode}

% \macro{\childdocforward}
% The command |\childdocforward| redirects
% compilation to the main file or
% (if the optional argument is given) a child file.
% Parameters are set as if the main file
% or a child file starting with |\childdocof| was compiled.
% Then compilation is handed over to the main file:
%    \begin{macrocode}
\newcommand{\childdocforward}[2][]
{
  \begingroup
    \if?#1?
      \def\childdoctmp
      {
        \def\childdocname{#2}
        \def\childdocjob{#2}
        \def\jobname{#2}
        \input{#2}
        \endinput
      }
    \else
      \def\childdoctmp
      {
        \childdocdisable
        \def\childdocname{#2}
        \childdoctrue
        \includeonly{#2}
        \def\childdocjob{#1}
        \def\jobname{#1}
        \input{#1}
        \endinput
      }
    \fi
    \expandafter
  \endgroup
  \childdoctmp
}
%    \end{macrocode}

% \macro{\childdocforwardprefix}
% The command |\childdocforwardprefix| redirects
% compilation to the main or a child file by means of a pattern.
% The prefix |#1| in the current filename is replaced by |#2|
% and the suffix of the current filename is kept
% (it is assumed that the filename does not contain the substring `|~~~|'
% which is used as a delimiter).
% Compilation is handed over to the new file by |\childdocforward|:
%    \begin{macrocode}
\newcommand{\childdocforwardprefix}[3][]
{
  \begingroup
    \def\childdocextract #2##1~~~{\def\childdoctmp{\childdocforward[#1]{#3##1}}}
    \expandafter\childdocextract\childdocname~~~
    \expandafter
  \endgroup
  \childdoctmp
}
%    \end{macrocode}

% \macro{\childdoc}
% The deprecated macro |\childdoc| is a legacy version of |\childdocmain|:
%    \begin{macrocode}
\newcommand{\childdoc}{\childdocmain}
%    \end{macrocode}

% \macro{\childdocredirect}
% The deprecated macro |\childdocredirect| is a legacy version
% of |\childdocforward| and |\childdocforwardprefix|:
%    \begin{macrocode}
\newcommand{\childdocredirect}[2][]
{
  \begingroup
    \if?#1?
      \def\childdoctmp{\childdocforward{#2}}
    \else
      \def\childdoctmp{\childdocforwardprefix{#1}{#2}}
    \fi
    \expandafter
  \endgroup
  \childdoctmp
}
%    \end{macrocode}

%\iffalse
%</package>
%\fi
%
\endinput
|\\
|\childdocby{|\textit{main}|}|\\
\end{tabular}
\end{center}
%
Both forms have slightly different effects as described above.
The main file is prepared as usual, see \secref{sec:include}.

%%%%%%%%%%%%%%%%%%%%%%%%%%%%%%%%%%%%%%%%%%%%%%%%%%%%%%%%%%%%%%%%%%%%%%%%%%%%%%%%
\subsection{Legacy Detection}
\label{sec:detection}

The directive |\childdocmain| in the main file can detect
whether the complete document or merely a child is to be compiled
even without using the directive |\childdocof|.
This method is deprecated because it is less robust
and there is no compelling reason to use it;
it is merely provided for backward compatibility
and it may be removed in future versions.

If the detection mechanism is to be used,
it is mandatory to correctly specify
the filename of the main file as the argument of |\childdocmain|:
%
\begin{center}
\begin{tabular}{l}
|% \iffalse
%
% childdoc.dtx Copyright (C) 2017-2018 Niklas Beisert
%
% This work may be distributed and/or modified under the
% conditions of the LaTeX Project Public License, either version 1.3
% of this license or (at your option) any later version.
% The latest version of this license is in
%   http://www.latex-project.org/lppl.txt
% and version 1.3 or later is part of all distributions of LaTeX
% version 2005/12/01 or later.
%
% This work has the LPPL maintenance status `maintained'.
%
% The Current Maintainer of this work is Niklas Beisert.
%
% This work consists of the files childdoc.dtx and childdoc.ins
% and the derived files childdoc.def and cdocsamp.tex with
% cdocsch1.tex, cdocsch2.tex, cdocsdrf.tex, cdocsfn1.tex, cdocsfn2.tex.
%
%<package>\ifdefined\childdocmain\endinput\fi
%<package>\ProvidesFile{childdoc.def}[2018/12/30 v2.0 child document driver]
%<samplemain>\ProvidesFile{cdocsamp.tex}[2018/12/30 v2.0 sample for childdoc]
%<*driver>
%\ProvidesFile{childdoc.drv}[2018/12/30 v2.0 childdoc reference manual file]
\PassOptionsToClass{10pt,a4paper}{article}
\documentclass{ltxdoc}

\usepackage[margin=35mm]{geometry}
\usepackage{hyperref}
\usepackage{hyperxmp}
\usepackage[usenames]{color}

\hypersetup{colorlinks=true}
\hypersetup{pdfstartview=FitH}
\hypersetup{pdfpagemode=UseNone}
\hypersetup{pdfsource={}}
\hypersetup{pdflang={en-UK}}
\hypersetup{pdfcopyright={Copyright 2017-2018 Niklas Beisert.
  This work may be distributed and/or modified under the
  conditions of the LaTeX Project Public License, either version 1.3
  of this license or (at your option) any later version.}}
\hypersetup{pdflicenseurl={http://www.latex-project.org/lppl.txt}}
\hypersetup{pdfcontactaddress={ETH Zurich, ITP, HIT K,
  Wolfgang-Pauli-Strasse 27}}
\hypersetup{pdfcontactpostcode={8093}}
\hypersetup{pdfcontactcity={Zurich}}
\hypersetup{pdfcontactcountry={Switzerland}}
\hypersetup{pdfcontactemail={nbeisert@itp.phys.ethz.ch}}
\hypersetup{pdfcontacturl={http://people.phys.ethz.ch/\xmptilde nbeisert/}}

\newcommand{\secref}[1]{\hyperref[#1]{section \ref*{#1}}}

\parskip1ex
\parindent0pt
\let\olditemize\itemize
\def\itemize{\olditemize\parskip0pt}

\begin{document}

\title{The \textsf{childdoc} Package}
\hypersetup{pdftitle={The childdoc Package}}
\author{Niklas Beisert\\[2ex]
  Institut f\"ur Theoretische Physik\\
  Eidgen\"ossische Technische Hochschule Z\"urich\\
  Wolfgang-Pauli-Strasse 27, 8093 Z\"urich, Switzerland\\[1ex]
  \href{mailto:nbeisert@itp.phys.ethz.ch}
  {\texttt{nbeisert@itp.phys.ethz.ch}}}
\hypersetup{pdfauthor={Niklas Beisert}}
\hypersetup{pdfsubject={Manual for the LaTeX2e Package childdoc}}
\date{30 December 2018, \textsf{v2.0}}
\maketitle

\begin{abstract}\noindent
\textsf{childdoc} is a \LaTeXe{} package
that enables the direct compilation
of document sections included by |\include|
to individual files.
\end{abstract}

\begingroup
\parskip0ex
\tableofcontents
\endgroup

%%%%%%%%%%%%%%%%%%%%%%%%%%%%%%%%%%%%%%%%%%%%%%%%%%%%%%%%%%%%%%%%%%%%%%%%%%%%%%%%
%%%%%%%%%%%%%%%%%%%%%%%%%%%%%%%%%%%%%%%%%%%%%%%%%%%%%%%%%%%%%%%%%%%%%%%%%%%%%%%%
\section{Introduction}

\LaTeX{} provides a mechanism to structure a large document (such as a book)
into a main file and several child files (containing the chapters)
using the |\include| command.
This mechanism is beneficial for documents
which span hundreds of pages in order to
make the source file(s) more manageable.
Moreover, compilation can be restricted to
selected child files by means of the |\includeonly| command.
The latter feature can be used to reduce the compilation time while editing
(this was significantly more useful in the earlier days of \LaTeX{})
or to generate a smaller document which is easier to navigate.
Another application of |\includeonly| is to generate
documents consisting of selected parts of the complete document.

However, there are a few drawbacks of the plain |\include| mechanism:
\begin{itemize}
\item
The child files cannot be compiled on their own,
they can only be compiled via the main file.
A naive editing environment
(such as a text editor with an option
to have the current file processed by \LaTeX)
may require one to switch to the main file before compiling;
attempting to compile the child file produces errors.
\item
The main file must be modified (each time)
to adjust the |\includeonly| command
to the present needs. This easily leaves the main file in a messy state.
\item
The generated document will always carry the filename
of the main document. This is inconvenient if
several child files are to be compiled and
to be kept for distribution.
\end{itemize}

The present package provides a simple interface
to make child files individually compilable by \LaTeX{}.
Compiling a child file then has the same effect as compiling
the main file with an |\includeonly| command
to select the appropriate child.
Moreover the generated document will carry the name of the child
rather than the main file.
This resolves all three above issues.

This feature is meant to make the editing of books,
thesis documents and lecture notes somewhat more convenient.
However, the package can also be used efficiently for
composing a series of documents (such as exercise sheets)
which are typically distributed individually.
It then assists the author in generating the individual documents
(potentially in different versions)
as well as a document containing the collected series.
Another application is in developing style files
or other kinds of included material
where compilation of the style file could redirect
to a sample or test file.

%%%%%%%%%%%%%%%%%%%%%%%%%%%%%%%%%%%%%%%%%%%%%%%%%%%%%%%%%%%%%%%%%%%%%%%%%%%%%%%%
%%%%%%%%%%%%%%%%%%%%%%%%%%%%%%%%%%%%%%%%%%%%%%%%%%%%%%%%%%%%%%%%%%%%%%%%%%%%%%%%
\section{Usage}

First of all, the package \textsf{childdoc} is \emph{not} a standard
\LaTeXe{} |.sty| style file! Therefore it needs to be invoked in
a non-standard way.

%%%%%%%%%%%%%%%%%%%%%%%%%%%%%%%%%%%%%%%%%%%%%%%%%%%%%%%%%%%%%%%%%%%%%%%%%%%%%%%%
\subsection{Included Files}
\label{sec:include}

%%%%%%%%%%%%%%%%%%%%%%%%%%%%%%%%%%%%%%%%
\DescribeMacro{\childdocmain}
To use the package, add the commands
\begin{center}
\begin{tabular}{l}
|\input{childdoc.def}|\\
|\childdocmain{}|\\
\end{tabular}
\end{center}
at the very top of the main \LaTeX{} file,
in particular \emph{before} the |\documentclass| statement!
The argument of |\childdocmain| should be left empty
(but it must be present).

%%%%%%%%%%%%%%%%%%%%%%%%%%%%%%%%%%%%%%%%
\DescribeMacro{\childdocof}
Furthermore, add the commands
\begin{center}
\begin{tabular}{l}
|\input{childdoc.def}|\\
|\childdocof{|\textit{main}|}|\\
\end{tabular}
\end{center}
at the top of every child file \textit{child}
which is included by |\include{|\textit{child}|}|
from within the main file
(or at least for those files to be compiled individually).
The argument \textit{main} must be the filename of the main file.

There are a couple of
considerations in setting up the main and child documents:

%%%%%%%%%%%%%%%%%%%%%%%%%%%%%%%%%%%%%%%%
\paragraph{Restrictions.}

Please note the following restrictions:
\begin{itemize}
\item
|\childdocmain| must be called with one argument \textit{main}
to ensure compatibility with earlier version of the package.
It must either be empty (|\childdocmain{}|)
or precisely match the filename of the main file in which it is specified.
See \secref{sec:detection} for further information.
\item
The filename \textit{main} must be specified without the |.tex| extension.
\item
The filename \textit{main} is case sensitive
(even in case-insensitive file systems)
due to internal string comparison.
\item
The argument \textit{main} should be fully expanded, it cannot be a macro.
\item
Subdirectories and special characters should be avoided in filenames.
\item
The command |\childdocmain{|\textit{main}|}| must be followed by a whitespace.
It should not be followed immediately by another command
or by a comment mark `|%|'.
This is because the \TeX{} parser reads the token immediately following
the argument of |\childdocmain| and puts it
at the beginning of every child section;
however, a white\-space is ignored.
\end{itemize}

%%%%%%%%%%%%%%%%%%%%%%%%%%%%%%%%%%%%%%%%
\paragraph{Content of Main File.}

It is advisable to place all content in the child files included by |\include|.
Any output contained in the main file will appear in all child documents
unless suppressed manually;
it cannot be suppressed automatically by the |\includeonly| directive
and thus should normally be avoided.
A method to include some content in the main file
by means of conditional processing is described in \secref{sec:conditional}.

%%%%%%%%%%%%%%%%%%%%%%%%%%%%%%%%%%%%%%%%
\paragraph{Page Numbering.}

When only a part of the document is compiled,
the appropriate numbering of pages
(as well as other status parameters)
is determined from the |.aux| files.
The latter contain information from previous passes.
However this information needs to propagate through
all intermediate child documents.
Therefore the page numbering in child documents may well
be inconsistent until the complete document is compiled at least once.

A useful (if unconventional) way to always ensure a consistent
page numbering is to restart the numbering in each child document
and denote the pages by `\textit{child}|.|\textit{page}'
where \textit{child} represents the chapter/section number of the child file.
This can be achieved by the command
|\numberwithin{page}{|\textit{child}|}|
of the \textsf{amsmath} package
where \textit{child} can be |chapter| or |section|
depending on the chosen structuring.
Alternatively, one can modify the macro |\thepage| appropriately
and reset the counter |page| at the start of each child file.

%%%%%%%%%%%%%%%%%%%%%%%%%%%%%%%%%%%%%%%%%%%%%%%%%%%%%%%%%%%%%%%%%%%%%%%%%%%%%%%%
\subsection{Conditional Processing}
\label{sec:conditional}

The package provides a mechanism to compile different versions
of a document. To customise the versions further some conditional processing
can come in handy to distinguish which version is being compiled.
The package provides two macros to describe the compilation context:

%%%%%%%%%%%%%%%%%%%%%%%%%%%%%%%%%%%%%%%%
\DescribeMacro{\ifchilddoc}
The conditional |\ifchilddoc| distinguishes between the compilation of
child documents and the main document:
%
\begin{center}
|\ifchilddoc |\textit{child-code}| |[|\||else |\textit{main-code}]| \||fi|
\end{center}

%%%%%%%%%%%%%%%%%%%%%%%%%%%%%%%%%%%%%%%%
\DescribeMacro{\childdocname}
\DescribeMacro{\childdocjob}
The macro |\childdocname| contains the filename (without extension)
of the main or child file being processed.
Note that |\childdocjob| will always contain the name of the main file.

%%%%%%%%%%%%%%%%%%%%%%%%%%%%%%%%%%%%%%%%
\paragraph{Title Page.}

Conditional processing can be used to include a title or banner page
in the main document when proper precautions are taken.
Importantly, the code in the main file should ensure that the page counter
(as well as other status parameters which are stored in the |.aux| files)
takes the same value after the conditional processing.
Otherwise the page numbers may take divergent values
depending on which part is compiled.

For example, a title page could be declared by:
%
\begin{center}
\begin{tabular}{l}
|\ifchilddoc\||else|\\
|\addtocounter{page}{-1}|\\
\textit{code for title page}\\
|\newpage|\\
|\||fi|
\end{tabular}
\end{center}
%
A banner page for the child documents can be generated by:
%
\begin{center}
\begin{tabular}{l}
|\ifchilddoc|\\
|\addtocounter{page}{-1}|\\
\textit{code for banner page}\\
|\newpage|\\
|\||fi|
\end{tabular}
\end{center}
%
Here one could write a message such as:
\begin{center}
|This is the part \childdocname{} of \childdocjob{}.|
\end{center}

%%%%%%%%%%%%%%%%%%%%%%%%%%%%%%%%%%%%%%%%%%%%%%%%%%%%%%%%%%%%%%%%%%%%%%%%%%%%%%%%
\subsection{Flags}
\label{sec:flags}

The package makes it easy to generate different versions
of the main or child documents.
To this end compilation flags can be defined
and assigned different default values.
They will be particularly useful in conjunction
with the forwarding mechanism described in \secref{sec:forward}.

For example, it may be useful to have a flag |\version|
which can be set to |draft| or |final|.
The document source will contain some conditional code
depending on the value of |\version|.
Suppose further, the flag should default to |final| for the main file
and to |draft| for child files
which is a natural assignment for editing the document.
This is achieved by placing the following code
in the preamble of the main document
(below the |\childdocmain| directive):
%
\begin{center}
\begin{tabular}{l}
|\ifchilddoc|\\
|\providecommand{\version}{draft}|\\
|\||else|\\
|\providecommand{\version}{final}|\\
|\||fi|
\end{tabular}
\end{center}
%
The definition by |\providecommand| makes sure
that previous definitions are not overwritten.
Further statements |\providecommand{\version}{...}|
can thus be added before the above code to override it.

For the main file, one might add a line
(between |\childdocmain| and the above block)
%
\begin{center}
|%\ifchilddoc\||else\providecommand{\version}{draft}\||fi|
\end{center}
%
which can be uncommented to produce a draft version.
Likewise one can add a line to the very top of a child file
(above the |\childdocof{|\textit{main}|}| directive)
%
\begin{center}
|%\providecommand{\version}{final}|
\end{center}
%
which can be uncommented to produce the final version of this child document.

%%%%%%%%%%%%%%%%%%%%%%%%%%%%%%%%%%%%%%%%%%%%%%%%%%%%%%%%%%%%%%%%%%%%%%%%%%%%%%%%
\subsection{Forwarding}
\label{sec:forward}

Different versions of the main or child documents
using compilation flags as described in \secref{sec:flags}
can be (permanently) stored in different files
for convenient compilation, viewing and distribution.
To this end, the package defines a command
to pass on compilation to a different file:

%%%%%%%%%%%%%%%%%%%%%%%%%%%%%%%%%%%%%%%%
\DescribeMacro{\childdocforward}
The command |\childdocforward| redirects processing to
another source file:
%
\begin{center}
\begin{tabular}{l}
|\input{childdoc.def}|\\
|\childdocforward[|\textit{main}|]{|\textit{dest}|}|\\
\end{tabular}
\end{center}
%
The argument \textit{dest} is the destination file
(without extension).
It should be the main file or one of the child files.
Note that further \textsf{childdoc} directives
such as |\childdocof| and |\childdocforward|
in the indicated file will be processed in this form.
The optional argument \textit{main}
passes on directly to the main file \textit{main}
while pretending to compile the child \textit{dest}.
This form behaves as if \textit{dest}
issues |\childdocof{|\textit{main}|}| right away,
and no further \textsf{childdoc} directives will be processed.

%%%%%%%%%%%%%%%%%%%%%%%%%%%%%%%%%%%%%%%%
\DescribeMacro{\...prefix}
In the alternative form |\childdocforwardprefix|,
%
\begin{center}
\begin{tabular}{l}
|\input{childdoc.def}|\\
|\childdocforwardprefix[|\textit{main}|]{|\textit{prefix}|}{|\textit{dest}|}|
\end{tabular}
\end{center}
%
the destination file is determined by a pattern
depending on the current file:
To make this work, the current file must be called
`{\textit{prefix}\hspace{0.2em}\textit{suffix}}'
with \textit{prefix} matching precisely the argument.
Processing is then passed on to the file
`{\textit{dest}\hspace{0.2em}\textit{suffix}}'.
Surely, the same effect is achieved by
directly specifying the
argument `{\textit{dest}\hspace{0.2em}\textit{suffix}}'
in the first form.
However, that requires to set up a different file
for each child. With the alternative form of the command
all these files can have exactly the same content
which simplifies setting them up and maintaining them.

For example, the following file |draft.tex|
with a compilation flag |\version| as described in \secref{sec:flags}
compiles the main document as a draft:
%
\begin{center}
\begin{tabular}{l}
|\def\version{draft}|\\
|\input{childdoc.def}|\\
|\childdocforward{|\textit{main}|}|
\end{tabular}
\end{center}
%
Likewise, the following files |final|\textit{nn}|.tex|
compile the final version of the child document
|child|\textit{nn}|.tex|:
%
\begin{center}
\begin{tabular}{l}
|\def\version{final}|\\
|\input{childdoc.def}|\\
|\childdocforwardprefix{final}{child}|
\end{tabular}
\end{center}
%

Note that when several versions of a main file and/or of each child file
are to be generated, it may be convenient to set up a |Makefile| or
shell script to automatise the process.

%%%%%%%%%%%%%%%%%%%%%%%%%%%%%%%%%%%%%%%%%%%%%%%%%%%%%%%%%%%%%%%%%%%%%%%%%%%%%%%%
\subsection{Command Line Processing}
\label{sec:commandline}

The effect of redirection files can also be achieved by invoking
the \LaTeX{} compiler with a more elaborate command line.
Most conveniently this should be done as part
of a shell script or a |Makefile|.

When using \textsf{childdoc} in the main file, the following
command lines effectively perform a redirection
(note that depending on the shell being used,
backslashes may have to be doubled: `|\|' $\to$ `|\\|'):
%
\begin{center}
|... -jobname "|\textit{target}|" |\\|"|[\textit{flags}]%
|\input{childdoc.def}\childdocforward[|\textit{main}|]{|\textit{dest}|}"|
\end{center}
%
Here \textit{target} is the name of the output file,
\textit{main} is the name of the main file
and \textit{dest} is the name of the main or child file to be processed
(all filenames without extensions).
The optional argument \textit{main} can be omitted
if \textit{main} matches \textit{dest}.
Optionally, compilation \textit{flags} can be defined via |\def| commands.
This command line makes the \TeX{} engine believe
it is compiling the file \textit{target}
whose content is specified as the latter parameter.
The provided code then forwards the processing to
\textit{main} or \textit{dest} as described in \secref{sec:forward}.

%%%%%%%%%%%%%%%%%%%%%%%%%%%%%%%%%%%%%%%%%%%%%%%%%%%%%%%%%%%%%%%%%%%%%%%%%%%%%%%%
\subsection{Include by Input}
\label{sec:input}

Including child documents by |\include| has some restrictions by design.
Most notably, the content of a child document always occupies
its own set of pages; pages cannot be shared between child documents.
Usually, this behaviour makes perfect sense
because each child document contain an essential part of the document.
However, in some situations it may be desirable to compose
a document from a collection of parts
without having mandatory page breaks between then.
For this case, the package
provides a mechanism to include parts
by |\input| which can also be processed individually.
However, by construction this mechanism
requires manual handling of the content to be output.

%%%%%%%%%%%%%%%%%%%%%%%%%%%%%%%%%%%%%%%%
\DescribeMacro{\ifchilddocmanual}
The main file should be prepared as usual, see \secref{sec:include}.
However, the document body must make a distinction
between processing of an individual part and of the main document, e.g.:
%
\begin{center}
\begin{tabular}{l}
|\ifchilddocmanual|\\
|\input{\childdocname}|\\
|\||else|\\
\textit{document body with }|\input{|\textit{part}|}|\\
|\||fi|
\end{tabular}
\end{center}
%
The conditional |\ifchilddocmanual| is true whenever
a part to be included by |\input| is being compiled,
and the name of the part is stored in |\childdocname|.

%%%%%%%%%%%%%%%%%%%%%%%%%%%%%%%%%%%%%%%%
\DescribeMacro{\childdocby}
Each part to be included by |\input| should start with:
%
\begin{center}
\begin{tabular}{l}
|\input{childdoc.def}|\\
|\childdocby{|\textit{main}|}|\\
\end{tabular}
\end{center}
%
The directive |\childdocby| is similar to |\childdocof|
described in \secref{sec:include},
but the subsequent selection of content must be done manually.
To that end, both |\ifchilddoc| and |\ifchilddocmanual|
will be true upon processing of a part,
and the name of the part is stored in |\childdocname|.
Note that |\jobname| will be set to the filename of the current part
so that each part receives an individual |.aux| file
that does not interfere with the |.aux| file(s) of the main document.
This behaviour can be altered by the alternative form
|\childdocby[*]{|\textit{main}|}| (with a non-empty optional argument)
which uses the |.aux| file of the main document
by setting |\jobname| to \textit{main}.

%%%%%%%%%%%%%%%%%%%%%%%%%%%%%%%%%%%%%%%%%%%%%%%%%%%%%%%%%%%%%%%%%%%%%%%%%%%%%%%%
\subsection{Driver Development}
\label{sec:driver}

The \textsf{childdoc} mechanism can also be use for the development
of definition files such as \LaTeX{} styles or classes.
This case differs from the above setup with multiple parts
included by |\include| in that no |\includeonly| should be invoked.
This can be achieved by starting the include file
(before |\ProvidesPackage|) with:
%
\begin{center}
\begin{tabular}{l}
|\input{childdoc.def}|\\
|\childdocforward{|\textit{main}|}|\\
\end{tabular}
\end{center}
%
or alternatively with:
%
\begin{center}
\begin{tabular}{l}
|\input{childdoc.def}|\\
|\childdocby{|\textit{main}|}|\\
\end{tabular}
\end{center}
%
Both forms have slightly different effects as described above.
The main file is prepared as usual, see \secref{sec:include}.

%%%%%%%%%%%%%%%%%%%%%%%%%%%%%%%%%%%%%%%%%%%%%%%%%%%%%%%%%%%%%%%%%%%%%%%%%%%%%%%%
\subsection{Legacy Detection}
\label{sec:detection}

The directive |\childdocmain| in the main file can detect
whether the complete document or merely a child is to be compiled
even without using the directive |\childdocof|.
This method is deprecated because it is less robust
and there is no compelling reason to use it;
it is merely provided for backward compatibility
and it may be removed in future versions.

If the detection mechanism is to be used,
it is mandatory to correctly specify
the filename of the main file as the argument of |\childdocmain|:
%
\begin{center}
\begin{tabular}{l}
|\input{childdoc.def}|\\
|\childdocmain{|\textit{main}|}|\\
\end{tabular}
\end{center}
%
If |\jobname| does not match the argument \textit{main} of |\childdocmain|,
it is assumed that |\jobname| points to the child file to be compiled.
When using |\childdocmain| with the main file specified as argument,
it suffices to start a child file
with just |\input{|\textit{main}|}|
without loading of the package and using |\childdocof|.
If instead all processing is done
with the appropriate \textsf{childdoc} directives,
the argument of \textit{main} of |\childdocmain| can be empty.

An alternative version of the command line processing described
in \secref{sec:commandline} using the detection mechanism reads:
%
\begin{center}
|... -jobname "|\textit{target}|" "|[\textit{flags}]%
[|\def\jobname{|\textit{dest}|}|]|\input{|\textit{main}|}"|
\end{center}

%%%%%%%%%%%%%%%%%%%%%%%%%%%%%%%%%%%%%%%%%%%%%%%%%%%%%%%%%%%%%%%%%%%%%%%%%%%%%%%%
\subsection{Manual Code}
\label{sec:manual}

In case one cannot be certain whether the definitions file |childdoc.def|
is installed on the target \TeX{} distribution
and one prefers not to ship it,
it is conceivable to paste a few relevant commands into the sources.

To that end, drop all statements |\input{childdoc.def}|
and perform the replacements as outlined below.
Instead of |\childdocmain{|\textit{main}|}| add the following code
to the top of the main file:
%
\begin{center}
\begin{tabular}{l}
|\||ifdefined\childdocname\endinput\||fi\newif\ifchilddoc|\\
|\edef\childdocname{\scantokens\expandafter{\jobname\noexpand}}|\\
|\def\childdocmain{|\textit{main}|}\||ifx\childdocmain\childdocname\||else|\\
|\childdoctrue\includeonly{\childdocname}\let\jobname\childdocmain\||fi|\\
\end{tabular}
\end{center}
%
Instead of |\childdocof{|\textit{main}|}| just include the main file
at the top of each child file:
%
\begin{center}
|\input{|\textit{main}|}|
\end{center}
%
A simple redirection |\childdocforward{|\textit{dest}|}| is achieved by:
%
\begin{center}
|\def\jobname{|\textit{dest}|}\input{\jobname}|
\end{center}
%
The redirection with prefix
|\childdocforwardprefix[|\textit{prefix}|]{|\textit{dest}|}|
is accomplished by:
%
\begin{center}
\begin{tabular}{l}
|{\edef\jobname{\scantokens\expandafter{\jobname\noexpand}}|\\
|\def\redirectjob |\textit{prefix}|#1~~~{\gdef\jobname{|\textit{dest}|#1}}|\\
|\expandafter\redirectjob\jobname~~~}\input{\jobname}|
\end{tabular}
\end{center}

In an alternative approach,
child documents can be compiled by a specific command line
without additional code or specific definitions:
%
\begin{center}
|... -jobname "|\textit{target}|" "|[\textit{flags}]%
|\includeonly{|\textit{dest}|}\input{|\textit{main}|}"|
\end{center}
%

%%%%%%%%%%%%%%%%%%%%%%%%%%%%%%%%%%%%%%%%%%%%%%%%%%%%%%%%%%%%%%%%%%%%%%%%%%%%%%%%
%%%%%%%%%%%%%%%%%%%%%%%%%%%%%%%%%%%%%%%%%%%%%%%%%%%%%%%%%%%%%%%%%%%%%%%%%%%%%%%%
\section{Information}

%%%%%%%%%%%%%%%%%%%%%%%%%%%%%%%%%%%%%%%%%%%%%%%%%%%%%%%%%%%%%%%%%%%%%%%%%%%%%%%%
\subsection{Copyright}

Copyright \copyright{} 2017--2018 Niklas Beisert

This work may be distributed and/or modified under the
conditions of the \LaTeX{} Project Public License, either version 1.3
of this license or (at your option) any later version.
The latest version of this license is in
  \url{http://www.latex-project.org/lppl.txt}
and version 1.3 or later is part of all distributions of \LaTeX{}
version 2005/12/01 or later.

This work has the LPPL maintenance status `maintained'.

The Current Maintainer of this work is Niklas Beisert.

This work consists of the files |README.txt|, |childdoc.ins| and |childdoc.dtx|
as well as the derived files |childdoc.def|, |cdocsamp.tex|
with |cdocsch1.tex|, |cdocsch2.tex|, |cdocspt3.tex|, |cdocspt4.tex|,
|cdocsdrf.tex|, |cdocsfn1.tex|, |cdocsfn2.tex|
as well as |childdoc.pdf|.

%%%%%%%%%%%%%%%%%%%%%%%%%%%%%%%%%%%%%%%%%%%%%%%%%%%%%%%%%%%%%%%%%%%%%%%%%%%%%%%%
\subsection{Files and Installation}

The package consists of the files:
%
\begin{center}
\begin{tabular}{ll}
    |README.txt|   & readme file \\
    |childdoc.ins| & installation file \\
    |childdoc.dtx| & source file \\
    |childdoc.def| & definition file \\
    |cdocsamp.tex| & sample main file \\
    |cdocsch1.tex| & sample include file \\
    |cdocsch2.tex| & sample include file \\
    |cdocspt3.tex| & sample part file \\
    |cdocspt4.tex| & sample part file \\
    |cdocsdrf.tex| & sample redirection file \\
    |cdocsfn1.tex| & sample redirection file \\
    |cdocsfn2.tex| & sample redirection file \\
    |childdoc.pdf| & manual
\end{tabular}
\end{center}
%
The distribution consists of the files
|README.txt|, |childdoc.ins| and |childdoc.dtx|.
%
\begin{itemize}
\item
Run (pdf)\LaTeX{} on |childdoc.dtx|
to compile the manual |childdoc.pdf| (this file).
\item
Run \LaTeX{} on |childdoc.ins| to create the definitions file |childdoc.def|
and the sample |cdocsamp.tex| with include files
|cdocsch1.tex|, |cdocsch2.tex|, |cdocspt3.tex|, |cdocspt4.tex|,
|cdocsdrf.tex|, |cdocsfn1.tex|, |cdocsfn2.tex|.
Then copy the file |childdoc.def| to an appropriate directory of your \LaTeX{}
distribution, e.g.\ \textit{texmf-root}|/tex/latex/childdoc|.
\end{itemize}

%%%%%%%%%%%%%%%%%%%%%%%%%%%%%%%%%%%%%%%%%%%%%%%%%%%%%%%%%%%%%%%%%%%%%%%%%%%%%%%%
\subsection{Related CTAN Packages}

There are several other packages which offer a similar functionality:
%
\begin{itemize}
\item
The packages
\href{http://ctan.org/pkg/docmute}{\textsf{docmute}},
\href{http://ctan.org/pkg/includex}{\textsf{includex}} and
\href{http://ctan.org/pkg/standalone}{\textsf{standalone}}
provide commands to include only the document body of
a child file thus allowing both files to be compiled individually.
\item
The packages \href{http://ctan.org/pkg/subdocs}{\textsf{subdocs}}
and \href{http://ctan.org/pkg/subfiles}{\textsf{subfiles}}
provide structures in which the main and child documents can be
encapsulated and allowing them to be compiled individually.
The inclusion mechanism is different from the conventional |\include|.
\item
The package \href{http://ctan.org/pkg/combine}{\textsf{combine}}
is an elaborate solution to combine several documents into one.
\end{itemize}
%
See also the CTAN topic \href{http://ctan.org/topic/subdocs}{\textsf{subdocs}}
for further related packages.
The present package differs from the above solutions in that
a document structure constructed with the conventional |\include| mechanism
just needs two extra commands at the top of every file
such that all constituent files can be compiled individually.

%%%%%%%%%%%%%%%%%%%%%%%%%%%%%%%%%%%%%%%%%%%%%%%%%%%%%%%%%%%%%%%%%%%%%%%%%%%%%%%%
%\subsection{Feature Suggestions}
%
%The following is a list of features which may be useful for future
%versions of this package:
%%
%\begin{itemize}
%\item
%\ldots
%\end{itemize}

%%%%%%%%%%%%%%%%%%%%%%%%%%%%%%%%%%%%%%%%%%%%%%%%%%%%%%%%%%%%%%%%%%%%%%%%%%%%%%%%
\subsection{Revision History}

%%%%%%%%%%%%%%%%%%%%%%%%%%%%%%%%%%%%%%%%
\paragraph{v2.0:} 2018/12/30

\begin{itemize}
\item
immediate forward processing
\item
added |\childdocby| mechanism
\item
manual restructured
\end{itemize}

%%%%%%%%%%%%%%%%%%%%%%%%%%%%%%%%%%%%%%%%
\paragraph{v1.6:} 2018/01/17

\begin{itemize}
\item
application for development of include files
\item
corrections to manual
\end{itemize}

%%%%%%%%%%%%%%%%%%%%%%%%%%%%%%%%%%%%%%%%
\paragraph{v1.5:} 2017/05/21

\begin{itemize}
\item
more complete structuring introduced
\item
|\childdocof| introduced
\item
|\childdoc| renamed to |\childdocmain|
\item
|\childredirect| renamed to |\childdocforward| and |\childdocforwardprefix|
and functionality expanded
\end{itemize}

%%%%%%%%%%%%%%%%%%%%%%%%%%%%%%%%%%%%%%%%
\paragraph{v1.0:} 2017/04/27

\begin{itemize}
\item
manual and install package
\item
first version published on CTAN
\end{itemize}

%%%%%%%%%%%%%%%%%%%%%%%%%%%%%%%%%%%%%%%%
\paragraph{v0.6:} 2017/04/26

\begin{itemize}
\item
redirection mechanism added
\end{itemize}

%%%%%%%%%%%%%%%%%%%%%%%%%%%%%%%%%%%%%%%%
\paragraph{v0.5:} 2017/04/26

\begin{itemize}
\item
functionality in definition file
\end{itemize}


%%%%%%%%%%%%%%%%%%%%%%%%%%%%%%%%%%%%%%%%%%%%%%%%%%%%%%%%%%%%%%%%%%%%%%%%%%%%%%%%
%%%%%%%%%%%%%%%%%%%%%%%%%%%%%%%%%%%%%%%%%%%%%%%%%%%%%%%%%%%%%%%%%%%%%%%%%%%%%%%%
%%%%%%%%%%%%%%%%%%%%%%%%%%%%%%%%%%%%%%%%%%%%%%%%%%%%%%%%%%%%%%%%%%%%%%%%%%%%%%%%
\appendix

\settowidth\MacroIndent{\rmfamily\scriptsize 000\ }

 \DocInput{childdoc.dtx}

\end{document}
%</driver>
% \fi
%
% %%%%%%%%%%%%%%%%%%%%%%%%%%%%%%%%%%%%%%%%%%%%%%%%%%%%%%%%%%%%%%%%%%%%%%%%%%%%%%
% %%%%%%%%%%%%%%%%%%%%%%%%%%%%%%%%%%%%%%%%%%%%%%%%%%%%%%%%%%%%%%%%%%%%%%%%%%%%%%
% \section{Sample}
%\iffalse
%<*samplemain>
%\fi
%
% The following presents a sample document
% with two chapters, two parts, a title page,
% a compile flag as well as three forwarding files to set the flag.
% It consists of eight |.tex| files:
% \begin{center}
% \begin{tabular}{ll}
% |cdocsamp.tex|&main file\\
% |cdocsch1.tex|&include file for chapter 1\\
% |cdocsch2.tex|&include file for chapter 2\\
% |cdocspt3.tex|&include file for part 3\\
% |cdocspt4.tex|&include file for part 4\\
% |cdocsdrf.tex|&forwarding file for main file in draft mode\\
% |cdocsfi1.tex|&forwarding file for final version of chapter 1\\
% |cdocsfi2.tex|&forwarding file for final version of chapter 2\\
% \end{tabular}
% \end{center}
% Each of the eight files can be compiled directly by the \LaTeX{} compiler.
%
% %%%%%%%%%%%%%%%%%%%%%%%%%%%%%%%%%%%%%%
% \paragraph{Main File.}
%
% The main file is called |cdocsamp.tex|.
%
% Load the \textsf{childdoc} definitions and
% declare the filename for the main document:
%    \begin{macrocode}
\input{childdoc.def}
\childdocmain{}
%    \end{macrocode}

% Optional override for |\version| flag:
%    \begin{macrocode}
%%\ifchilddoc\else\providecommand{\version}{draft}\fi
%    \end{macrocode}

% Define the default values for the |\version| flag
% (|final| for the main file and |draft| for childs):
%    \begin{macrocode}
\ifchilddoc
\providecommand{\version}{draft}
\else
\providecommand{\version}{final}
\fi
%    \end{macrocode}

% Load the standard document class:
%    \begin{macrocode}
\documentclass[12pt]{article}
%    \end{macrocode}

% Start the document body:
%    \begin{macrocode}
\begin{document}
%    \end{macrocode}

% Declare a title page.
% Print title, part of document being processed and version flag:
%    \begin{macrocode}
\addtocounter{page}{-1}
\begin{center}
{\LARGE\bfseries{}childdoc example\par}
\vspace{1cm}
\ifchilddoc
\ifchilddocmanual part\else chapter\fi:
`\childdocname' of `\childdocjob'\par
\else
main document: `\childdocjob'\par
\fi
version: \version\par
\end{center}
\newpage
%    \end{macrocode}

% Manually include selected file,
% otherwise process as usual:
%    \begin{macrocode}
\ifchilddocmanual
\section*{part `\childdocname'}
\input{\childdocname}
\else
%    \end{macrocode}

% Include the two chapters:
%    \begin{macrocode}
\include{cdocsch1}
\include{cdocsch2}
%    \end{macrocode}

% Include the two parts unless only chapters should be displayed:
%    \begin{macrocode}
\ifchilddoc\else
\section{part three}
\input{cdocspt3}
\section{part four}
\input{cdocspt4}
\fi
%    \end{macrocode}

% Process as usual until here:
%    \begin{macrocode}
\fi
%    \end{macrocode}

% End of document body:
%    \begin{macrocode}
\end{document}
%    \end{macrocode}
%\iffalse
%</samplemain>
%\fi
%
% %%%%%%%%%%%%%%%%%%%%%%%%%%%%%%%%%%%%%%
% \paragraph{Chapter Include Files.}
%
% The include files are called |cdocsch1.tex| and |cdocsch2.tex|.
%
%\iffalse
%<*samplechap1|samplechap2>
%\fi

% Optional override for |\version| flag:
%    \begin{macrocode}
%%\providecommand{\version}{final}
%    \end{macrocode}

% Include the main document:
%    \begin{macrocode}
\input{childdoc.def}
\childdocof{cdocsamp}
%    \end{macrocode}

%\iffalse
%</samplechap1|samplechap2>
%\fi
%
%\iffalse
%<*samplechap1>
%\fi
% Some text for chapter 1:
%    \begin{macrocode}
\section{one}
some text in chapter one
%    \end{macrocode}

%\iffalse
%</samplechap1>
%\fi
% Some text for chapter 2:
%\iffalse
%<*samplechap2>
%\fi
%    \begin{macrocode}
\section{two}
more text in chapter two
%    \end{macrocode}

%\iffalse
%</samplechap2>
%\fi
%
% %%%%%%%%%%%%%%%%%%%%%%%%%%%%%%%%%%%%%%
% \paragraph{Part Include Files.}
%
% The include files are called |cdocspt3.tex| and |cdocspt4.tex|.
%
%\iffalse
%<*samplepart3|samplepart4>
%\fi

% Optional override for |\version| flag:
%    \begin{macrocode}
%%\providecommand{\version}{final}
%    \end{macrocode}

% Include the main document:
%    \begin{macrocode}
\input{childdoc.def}
\childdocby{cdocsamp}
%    \end{macrocode}

%\iffalse
%</samplepart3|samplepart4>
%\fi
%
%\iffalse
%<*samplepart3>
%\fi
% Some text for part 3:
%    \begin{macrocode}
some text in part three
%    \end{macrocode}

%\iffalse
%</samplepart3>
%\fi
% Some text for part 4:
%\iffalse
%<*samplepart4>
%\fi
%    \begin{macrocode}
more text in part four
%    \end{macrocode}

%\iffalse
%</samplepart4>
%\fi
%
% %%%%%%%%%%%%%%%%%%%%%%%%%%%%%%%%%%%%%%
% \paragraph{Forwarding for a Complete Draft.}
%
% The following forwarding file |cdocsdrf.tex|
% compiles the main document in draft mode:
%\iffalse
%<*sampledraft>
%\fi
%    \begin{macrocode}
\def\version{draft}
\input{childdoc.def}
\childdocforward{cdocsamp}
%    \end{macrocode}

%\iffalse
%</sampledraft>
%\fi
%
% %%%%%%%%%%%%%%%%%%%%%%%%%%%%%%%%%%%%%%
% \paragraph{Forwarding for Final Version of the Chapters.}
%
% The following forwarding files |cdocsfn1.tex| and |cdocsfn2.tex|
% (with identical content)
% compile the final versions of the child documents
% |cdocsch1.tex| and |cdocsch2.tex|, respectively:
%\iffalse
%<*samplefinal>
%\fi
%    \begin{macrocode}
\def\version{final}
\input{childdoc.def}
\childdocforwardprefix[cdocsamp]{cdocsfn}{cdocsch}
%    \end{macrocode}

%\iffalse
%</samplefinal>
%\fi
%
% %%%%%%%%%%%%%%%%%%%%%%%%%%%%%%%%%%%%%%
% \paragraph{Command Line Processing.}
%
% The following three command lines generate the output files
% |cdocscld|, |cdocscl1| and |cdocscl2|
% which should be identical to
% |cdocsdrf|, |cdocsch1| and |cdocsfn2|, respectively:
% \begin{center}
% \begin{tabular}{l}
% |latex -jobname cdocscld \|\\
% |  "\def\version{draft}\input{childdoc.def}\childdocforward{cdocsamp}"|\\
% |latex -jobname cdocscl1 \|\\
% |  "\input{childdoc.def}\childdocforward[cdocsamp]{cdocsch1}"|\\
% |latex -jobname cdocscl2 \|\\
% |  "\def\version{final}\input{childdoc.def}\childdocforward{cdocsch2}"|
% \end{tabular}
% \end{center}
% Note that the trailing backslash on each first line
% merely continues the input to the second line
% (for convenient cut ant paste).
% Furthermore, the command |latex| can be replaced by any
% of its alternative versions such as |pdflatex|.
%
% %%%%%%%%%%%%%%%%%%%%%%%%%%%%%%%%%%%%%%%%%%%%%%%%%%%%%%%%%%%%%%%%%%%%%%%%%%%%%%
% %%%%%%%%%%%%%%%%%%%%%%%%%%%%%%%%%%%%%%%%%%%%%%%%%%%%%%%%%%%%%%%%%%%%%%%%%%%%%%
% \section{Implementation}
%\iffalse
%<*package>
%\fi
%
% This section describes the definitions file |childdoc.def|.

% The definitions cannot be loaded using |\usepackage| or |\RequirePackage|
% which has a mechanism to prevent loading a style file more than once.
% When loading the definitions by means of |\input|
% multiple instances have to be prevented manually:
%\iffalse
%This code needs to be before the `\ProvidesFile' directive
%which is defined at the beginning of this file.
%Therefore it is also placed there and commented out here.
%</package>
%<*discard>
%\fi
%    \begin{macrocode}
\ifdefined\childdocmain\endinput\fi
%    \end{macrocode}
%\iffalse
%</discard>
%<*package>
%\fi
%
% \macro{\ifchilddoc}
% \macro{\ifchilddocmanual}
% The conditional |\ifchilddoc| tells whether a
% child (true) or main (false) document is being compiled.
% The conditional |\ifchilddocmanual| tells whether
% the |\includeonly| mechanism is used (false) or
% the selection of child files must be performed manually (true).
% The definitions initialise to false:
%    \begin{macrocode}
\newif\ifchilddoc
\newif\ifchilddocmanual
%    \end{macrocode}

% \macro{\childdocname}
% \macro{\childdocjob}
% The macro |\childdocname| stores the name of the main document
% to be compiled. The macro |\childdocjob| stores the name of
% the document on which the \LaTeX{} compiler was originally invoked.
% The content of |\jobname| cannot be compared
% to filenames specified in the source due to different catcodes.
% The following code rescans |\jobname|, stores the result
% in |\childdocname| and saves a copy in |\childdocjob|:
%    \begin{macrocode}
\edef\childdocname{\scantokens\expandafter{\jobname\noexpand}}
\let\childdocjob\childdocname
%    \end{macrocode}

% \macro{\childdocdisable}
% The macro |\childdocdisable| prevents the main file
% from being processed more than once.
% At this stage, the main document command |\childdocmain|
% is assumed to be called once again where it should do nothing.
% Any subsequent call to it should prevent
% a secondary processing of the main document
% It overwrites the forwarding commands
% |\childdocof| and |\childdocforward|
% with empty macros to prevent further inclusions of the main document:
%    \begin{macrocode}
\newcommand{\childdocdisable}
{
  \renewcommand{\childdocmain}[1]{\renewcommand{\childdocmain}[1]{\endinput}}
  \renewcommand{\childdocof}[1]{}
  \renewcommand{\childdocby}[2][]{}
  \renewcommand{\childdocforward}[2][]{}
  \renewcommand{\childdocdisable}{}
}
%    \end{macrocode}

% \macro{\childdocmain}
% The macro |\childdocmain| is to be called at the top of the main file
% with nothing or the main filename (without extension) as argument.
% First, it breaks loops.
% If the argument is not empty and does not match |\childdocname|
% (which is set by the first inclusion of |childdoc.def|),
% |\ifchilddoc| is set to true, |\includeonly| is applied to the child file
% and |\jobname| is set to the main file
% (for proper handling of |.aux| files):
%    \begin{macrocode}
\newcommand{\childdocmain}[1]
{
  \childdocdisable\childdocmain{}
  \if?#1?\else
    \begingroup
      \def\childdoctmp{#1}
      \ifx\childdoctmp\childdocname
        \def\childdoctmp{}
      \else
        \def\childdoctmp
        {
          \childdoctrue
          \includeonly{\childdocname}
          \def\childdocjob{#1}
          \def\jobname{#1}
        }
      \fi
      \expandafter
    \endgroup
    \childdoctmp
  \fi
}
%    \end{macrocode}

% \macro{\childdocof}
% The command |\childdocof| redirects
% compilation to the main file |#1|.
%    \begin{macrocode}
\newcommand{\childdocof}[1]
{
  \childdocdisable
  \childdoctrue
  \includeonly{\childdocname}
  \def\jobname{#1}
  \def\childdocjob{#1}
  \input{#1}
}
%    \end{macrocode}

% \macro{\childdocby}
% The command |\childdocby| ....
%    \begin{macrocode}
\newcommand{\childdocby}[2][]
{
  \childdocdisable
  \childdoctrue
  \childdocmanualtrue
  \if?#1?\else
    \def\jobname{#2}
  \fi
  \def\childdocjob{#2}
  \input{#2}
  \endinput
}
%    \end{macrocode}

% \macro{\childdocforward}
% The command |\childdocforward| redirects
% compilation to the main file or
% (if the optional argument is given) a child file.
% Parameters are set as if the main file
% or a child file starting with |\childdocof| was compiled.
% Then compilation is handed over to the main file:
%    \begin{macrocode}
\newcommand{\childdocforward}[2][]
{
  \begingroup
    \if?#1?
      \def\childdoctmp
      {
        \def\childdocname{#2}
        \def\childdocjob{#2}
        \def\jobname{#2}
        \input{#2}
        \endinput
      }
    \else
      \def\childdoctmp
      {
        \childdocdisable
        \def\childdocname{#2}
        \childdoctrue
        \includeonly{#2}
        \def\childdocjob{#1}
        \def\jobname{#1}
        \input{#1}
        \endinput
      }
    \fi
    \expandafter
  \endgroup
  \childdoctmp
}
%    \end{macrocode}

% \macro{\childdocforwardprefix}
% The command |\childdocforwardprefix| redirects
% compilation to the main or a child file by means of a pattern.
% The prefix |#1| in the current filename is replaced by |#2|
% and the suffix of the current filename is kept
% (it is assumed that the filename does not contain the substring `|~~~|'
% which is used as a delimiter).
% Compilation is handed over to the new file by |\childdocforward|:
%    \begin{macrocode}
\newcommand{\childdocforwardprefix}[3][]
{
  \begingroup
    \def\childdocextract #2##1~~~{\def\childdoctmp{\childdocforward[#1]{#3##1}}}
    \expandafter\childdocextract\childdocname~~~
    \expandafter
  \endgroup
  \childdoctmp
}
%    \end{macrocode}

% \macro{\childdoc}
% The deprecated macro |\childdoc| is a legacy version of |\childdocmain|:
%    \begin{macrocode}
\newcommand{\childdoc}{\childdocmain}
%    \end{macrocode}

% \macro{\childdocredirect}
% The deprecated macro |\childdocredirect| is a legacy version
% of |\childdocforward| and |\childdocforwardprefix|:
%    \begin{macrocode}
\newcommand{\childdocredirect}[2][]
{
  \begingroup
    \if?#1?
      \def\childdoctmp{\childdocforward{#2}}
    \else
      \def\childdoctmp{\childdocforwardprefix{#1}{#2}}
    \fi
    \expandafter
  \endgroup
  \childdoctmp
}
%    \end{macrocode}

%\iffalse
%</package>
%\fi
%
\endinput
|\\
|\childdocmain{|\textit{main}|}|\\
\end{tabular}
\end{center}
%
If |\jobname| does not match the argument \textit{main} of |\childdocmain|,
it is assumed that |\jobname| points to the child file to be compiled.
When using |\childdocmain| with the main file specified as argument,
it suffices to start a child file
with just |\input{|\textit{main}|}|
without loading of the package and using |\childdocof|.
If instead all processing is done
with the appropriate \textsf{childdoc} directives,
the argument of \textit{main} of |\childdocmain| can be empty.

An alternative version of the command line processing described
in \secref{sec:commandline} using the detection mechanism reads:
%
\begin{center}
|... -jobname "|\textit{target}|" "|[\textit{flags}]%
[|\def\jobname{|\textit{dest}|}|]|\input{|\textit{main}|}"|
\end{center}

%%%%%%%%%%%%%%%%%%%%%%%%%%%%%%%%%%%%%%%%%%%%%%%%%%%%%%%%%%%%%%%%%%%%%%%%%%%%%%%%
\subsection{Manual Code}
\label{sec:manual}

In case one cannot be certain whether the definitions file |childdoc.def|
is installed on the target \TeX{} distribution
and one prefers not to ship it,
it is conceivable to paste a few relevant commands into the sources.

To that end, drop all statements |% \iffalse
%
% childdoc.dtx Copyright (C) 2017-2018 Niklas Beisert
%
% This work may be distributed and/or modified under the
% conditions of the LaTeX Project Public License, either version 1.3
% of this license or (at your option) any later version.
% The latest version of this license is in
%   http://www.latex-project.org/lppl.txt
% and version 1.3 or later is part of all distributions of LaTeX
% version 2005/12/01 or later.
%
% This work has the LPPL maintenance status `maintained'.
%
% The Current Maintainer of this work is Niklas Beisert.
%
% This work consists of the files childdoc.dtx and childdoc.ins
% and the derived files childdoc.def and cdocsamp.tex with
% cdocsch1.tex, cdocsch2.tex, cdocsdrf.tex, cdocsfn1.tex, cdocsfn2.tex.
%
%<package>\ifdefined\childdocmain\endinput\fi
%<package>\ProvidesFile{childdoc.def}[2018/12/30 v2.0 child document driver]
%<samplemain>\ProvidesFile{cdocsamp.tex}[2018/12/30 v2.0 sample for childdoc]
%<*driver>
%\ProvidesFile{childdoc.drv}[2018/12/30 v2.0 childdoc reference manual file]
\PassOptionsToClass{10pt,a4paper}{article}
\documentclass{ltxdoc}

\usepackage[margin=35mm]{geometry}
\usepackage{hyperref}
\usepackage{hyperxmp}
\usepackage[usenames]{color}

\hypersetup{colorlinks=true}
\hypersetup{pdfstartview=FitH}
\hypersetup{pdfpagemode=UseNone}
\hypersetup{pdfsource={}}
\hypersetup{pdflang={en-UK}}
\hypersetup{pdfcopyright={Copyright 2017-2018 Niklas Beisert.
  This work may be distributed and/or modified under the
  conditions of the LaTeX Project Public License, either version 1.3
  of this license or (at your option) any later version.}}
\hypersetup{pdflicenseurl={http://www.latex-project.org/lppl.txt}}
\hypersetup{pdfcontactaddress={ETH Zurich, ITP, HIT K,
  Wolfgang-Pauli-Strasse 27}}
\hypersetup{pdfcontactpostcode={8093}}
\hypersetup{pdfcontactcity={Zurich}}
\hypersetup{pdfcontactcountry={Switzerland}}
\hypersetup{pdfcontactemail={nbeisert@itp.phys.ethz.ch}}
\hypersetup{pdfcontacturl={http://people.phys.ethz.ch/\xmptilde nbeisert/}}

\newcommand{\secref}[1]{\hyperref[#1]{section \ref*{#1}}}

\parskip1ex
\parindent0pt
\let\olditemize\itemize
\def\itemize{\olditemize\parskip0pt}

\begin{document}

\title{The \textsf{childdoc} Package}
\hypersetup{pdftitle={The childdoc Package}}
\author{Niklas Beisert\\[2ex]
  Institut f\"ur Theoretische Physik\\
  Eidgen\"ossische Technische Hochschule Z\"urich\\
  Wolfgang-Pauli-Strasse 27, 8093 Z\"urich, Switzerland\\[1ex]
  \href{mailto:nbeisert@itp.phys.ethz.ch}
  {\texttt{nbeisert@itp.phys.ethz.ch}}}
\hypersetup{pdfauthor={Niklas Beisert}}
\hypersetup{pdfsubject={Manual for the LaTeX2e Package childdoc}}
\date{30 December 2018, \textsf{v2.0}}
\maketitle

\begin{abstract}\noindent
\textsf{childdoc} is a \LaTeXe{} package
that enables the direct compilation
of document sections included by |\include|
to individual files.
\end{abstract}

\begingroup
\parskip0ex
\tableofcontents
\endgroup

%%%%%%%%%%%%%%%%%%%%%%%%%%%%%%%%%%%%%%%%%%%%%%%%%%%%%%%%%%%%%%%%%%%%%%%%%%%%%%%%
%%%%%%%%%%%%%%%%%%%%%%%%%%%%%%%%%%%%%%%%%%%%%%%%%%%%%%%%%%%%%%%%%%%%%%%%%%%%%%%%
\section{Introduction}

\LaTeX{} provides a mechanism to structure a large document (such as a book)
into a main file and several child files (containing the chapters)
using the |\include| command.
This mechanism is beneficial for documents
which span hundreds of pages in order to
make the source file(s) more manageable.
Moreover, compilation can be restricted to
selected child files by means of the |\includeonly| command.
The latter feature can be used to reduce the compilation time while editing
(this was significantly more useful in the earlier days of \LaTeX{})
or to generate a smaller document which is easier to navigate.
Another application of |\includeonly| is to generate
documents consisting of selected parts of the complete document.

However, there are a few drawbacks of the plain |\include| mechanism:
\begin{itemize}
\item
The child files cannot be compiled on their own,
they can only be compiled via the main file.
A naive editing environment
(such as a text editor with an option
to have the current file processed by \LaTeX)
may require one to switch to the main file before compiling;
attempting to compile the child file produces errors.
\item
The main file must be modified (each time)
to adjust the |\includeonly| command
to the present needs. This easily leaves the main file in a messy state.
\item
The generated document will always carry the filename
of the main document. This is inconvenient if
several child files are to be compiled and
to be kept for distribution.
\end{itemize}

The present package provides a simple interface
to make child files individually compilable by \LaTeX{}.
Compiling a child file then has the same effect as compiling
the main file with an |\includeonly| command
to select the appropriate child.
Moreover the generated document will carry the name of the child
rather than the main file.
This resolves all three above issues.

This feature is meant to make the editing of books,
thesis documents and lecture notes somewhat more convenient.
However, the package can also be used efficiently for
composing a series of documents (such as exercise sheets)
which are typically distributed individually.
It then assists the author in generating the individual documents
(potentially in different versions)
as well as a document containing the collected series.
Another application is in developing style files
or other kinds of included material
where compilation of the style file could redirect
to a sample or test file.

%%%%%%%%%%%%%%%%%%%%%%%%%%%%%%%%%%%%%%%%%%%%%%%%%%%%%%%%%%%%%%%%%%%%%%%%%%%%%%%%
%%%%%%%%%%%%%%%%%%%%%%%%%%%%%%%%%%%%%%%%%%%%%%%%%%%%%%%%%%%%%%%%%%%%%%%%%%%%%%%%
\section{Usage}

First of all, the package \textsf{childdoc} is \emph{not} a standard
\LaTeXe{} |.sty| style file! Therefore it needs to be invoked in
a non-standard way.

%%%%%%%%%%%%%%%%%%%%%%%%%%%%%%%%%%%%%%%%%%%%%%%%%%%%%%%%%%%%%%%%%%%%%%%%%%%%%%%%
\subsection{Included Files}
\label{sec:include}

%%%%%%%%%%%%%%%%%%%%%%%%%%%%%%%%%%%%%%%%
\DescribeMacro{\childdocmain}
To use the package, add the commands
\begin{center}
\begin{tabular}{l}
|\input{childdoc.def}|\\
|\childdocmain{}|\\
\end{tabular}
\end{center}
at the very top of the main \LaTeX{} file,
in particular \emph{before} the |\documentclass| statement!
The argument of |\childdocmain| should be left empty
(but it must be present).

%%%%%%%%%%%%%%%%%%%%%%%%%%%%%%%%%%%%%%%%
\DescribeMacro{\childdocof}
Furthermore, add the commands
\begin{center}
\begin{tabular}{l}
|\input{childdoc.def}|\\
|\childdocof{|\textit{main}|}|\\
\end{tabular}
\end{center}
at the top of every child file \textit{child}
which is included by |\include{|\textit{child}|}|
from within the main file
(or at least for those files to be compiled individually).
The argument \textit{main} must be the filename of the main file.

There are a couple of
considerations in setting up the main and child documents:

%%%%%%%%%%%%%%%%%%%%%%%%%%%%%%%%%%%%%%%%
\paragraph{Restrictions.}

Please note the following restrictions:
\begin{itemize}
\item
|\childdocmain| must be called with one argument \textit{main}
to ensure compatibility with earlier version of the package.
It must either be empty (|\childdocmain{}|)
or precisely match the filename of the main file in which it is specified.
See \secref{sec:detection} for further information.
\item
The filename \textit{main} must be specified without the |.tex| extension.
\item
The filename \textit{main} is case sensitive
(even in case-insensitive file systems)
due to internal string comparison.
\item
The argument \textit{main} should be fully expanded, it cannot be a macro.
\item
Subdirectories and special characters should be avoided in filenames.
\item
The command |\childdocmain{|\textit{main}|}| must be followed by a whitespace.
It should not be followed immediately by another command
or by a comment mark `|%|'.
This is because the \TeX{} parser reads the token immediately following
the argument of |\childdocmain| and puts it
at the beginning of every child section;
however, a white\-space is ignored.
\end{itemize}

%%%%%%%%%%%%%%%%%%%%%%%%%%%%%%%%%%%%%%%%
\paragraph{Content of Main File.}

It is advisable to place all content in the child files included by |\include|.
Any output contained in the main file will appear in all child documents
unless suppressed manually;
it cannot be suppressed automatically by the |\includeonly| directive
and thus should normally be avoided.
A method to include some content in the main file
by means of conditional processing is described in \secref{sec:conditional}.

%%%%%%%%%%%%%%%%%%%%%%%%%%%%%%%%%%%%%%%%
\paragraph{Page Numbering.}

When only a part of the document is compiled,
the appropriate numbering of pages
(as well as other status parameters)
is determined from the |.aux| files.
The latter contain information from previous passes.
However this information needs to propagate through
all intermediate child documents.
Therefore the page numbering in child documents may well
be inconsistent until the complete document is compiled at least once.

A useful (if unconventional) way to always ensure a consistent
page numbering is to restart the numbering in each child document
and denote the pages by `\textit{child}|.|\textit{page}'
where \textit{child} represents the chapter/section number of the child file.
This can be achieved by the command
|\numberwithin{page}{|\textit{child}|}|
of the \textsf{amsmath} package
where \textit{child} can be |chapter| or |section|
depending on the chosen structuring.
Alternatively, one can modify the macro |\thepage| appropriately
and reset the counter |page| at the start of each child file.

%%%%%%%%%%%%%%%%%%%%%%%%%%%%%%%%%%%%%%%%%%%%%%%%%%%%%%%%%%%%%%%%%%%%%%%%%%%%%%%%
\subsection{Conditional Processing}
\label{sec:conditional}

The package provides a mechanism to compile different versions
of a document. To customise the versions further some conditional processing
can come in handy to distinguish which version is being compiled.
The package provides two macros to describe the compilation context:

%%%%%%%%%%%%%%%%%%%%%%%%%%%%%%%%%%%%%%%%
\DescribeMacro{\ifchilddoc}
The conditional |\ifchilddoc| distinguishes between the compilation of
child documents and the main document:
%
\begin{center}
|\ifchilddoc |\textit{child-code}| |[|\||else |\textit{main-code}]| \||fi|
\end{center}

%%%%%%%%%%%%%%%%%%%%%%%%%%%%%%%%%%%%%%%%
\DescribeMacro{\childdocname}
\DescribeMacro{\childdocjob}
The macro |\childdocname| contains the filename (without extension)
of the main or child file being processed.
Note that |\childdocjob| will always contain the name of the main file.

%%%%%%%%%%%%%%%%%%%%%%%%%%%%%%%%%%%%%%%%
\paragraph{Title Page.}

Conditional processing can be used to include a title or banner page
in the main document when proper precautions are taken.
Importantly, the code in the main file should ensure that the page counter
(as well as other status parameters which are stored in the |.aux| files)
takes the same value after the conditional processing.
Otherwise the page numbers may take divergent values
depending on which part is compiled.

For example, a title page could be declared by:
%
\begin{center}
\begin{tabular}{l}
|\ifchilddoc\||else|\\
|\addtocounter{page}{-1}|\\
\textit{code for title page}\\
|\newpage|\\
|\||fi|
\end{tabular}
\end{center}
%
A banner page for the child documents can be generated by:
%
\begin{center}
\begin{tabular}{l}
|\ifchilddoc|\\
|\addtocounter{page}{-1}|\\
\textit{code for banner page}\\
|\newpage|\\
|\||fi|
\end{tabular}
\end{center}
%
Here one could write a message such as:
\begin{center}
|This is the part \childdocname{} of \childdocjob{}.|
\end{center}

%%%%%%%%%%%%%%%%%%%%%%%%%%%%%%%%%%%%%%%%%%%%%%%%%%%%%%%%%%%%%%%%%%%%%%%%%%%%%%%%
\subsection{Flags}
\label{sec:flags}

The package makes it easy to generate different versions
of the main or child documents.
To this end compilation flags can be defined
and assigned different default values.
They will be particularly useful in conjunction
with the forwarding mechanism described in \secref{sec:forward}.

For example, it may be useful to have a flag |\version|
which can be set to |draft| or |final|.
The document source will contain some conditional code
depending on the value of |\version|.
Suppose further, the flag should default to |final| for the main file
and to |draft| for child files
which is a natural assignment for editing the document.
This is achieved by placing the following code
in the preamble of the main document
(below the |\childdocmain| directive):
%
\begin{center}
\begin{tabular}{l}
|\ifchilddoc|\\
|\providecommand{\version}{draft}|\\
|\||else|\\
|\providecommand{\version}{final}|\\
|\||fi|
\end{tabular}
\end{center}
%
The definition by |\providecommand| makes sure
that previous definitions are not overwritten.
Further statements |\providecommand{\version}{...}|
can thus be added before the above code to override it.

For the main file, one might add a line
(between |\childdocmain| and the above block)
%
\begin{center}
|%\ifchilddoc\||else\providecommand{\version}{draft}\||fi|
\end{center}
%
which can be uncommented to produce a draft version.
Likewise one can add a line to the very top of a child file
(above the |\childdocof{|\textit{main}|}| directive)
%
\begin{center}
|%\providecommand{\version}{final}|
\end{center}
%
which can be uncommented to produce the final version of this child document.

%%%%%%%%%%%%%%%%%%%%%%%%%%%%%%%%%%%%%%%%%%%%%%%%%%%%%%%%%%%%%%%%%%%%%%%%%%%%%%%%
\subsection{Forwarding}
\label{sec:forward}

Different versions of the main or child documents
using compilation flags as described in \secref{sec:flags}
can be (permanently) stored in different files
for convenient compilation, viewing and distribution.
To this end, the package defines a command
to pass on compilation to a different file:

%%%%%%%%%%%%%%%%%%%%%%%%%%%%%%%%%%%%%%%%
\DescribeMacro{\childdocforward}
The command |\childdocforward| redirects processing to
another source file:
%
\begin{center}
\begin{tabular}{l}
|\input{childdoc.def}|\\
|\childdocforward[|\textit{main}|]{|\textit{dest}|}|\\
\end{tabular}
\end{center}
%
The argument \textit{dest} is the destination file
(without extension).
It should be the main file or one of the child files.
Note that further \textsf{childdoc} directives
such as |\childdocof| and |\childdocforward|
in the indicated file will be processed in this form.
The optional argument \textit{main}
passes on directly to the main file \textit{main}
while pretending to compile the child \textit{dest}.
This form behaves as if \textit{dest}
issues |\childdocof{|\textit{main}|}| right away,
and no further \textsf{childdoc} directives will be processed.

%%%%%%%%%%%%%%%%%%%%%%%%%%%%%%%%%%%%%%%%
\DescribeMacro{\...prefix}
In the alternative form |\childdocforwardprefix|,
%
\begin{center}
\begin{tabular}{l}
|\input{childdoc.def}|\\
|\childdocforwardprefix[|\textit{main}|]{|\textit{prefix}|}{|\textit{dest}|}|
\end{tabular}
\end{center}
%
the destination file is determined by a pattern
depending on the current file:
To make this work, the current file must be called
`{\textit{prefix}\hspace{0.2em}\textit{suffix}}'
with \textit{prefix} matching precisely the argument.
Processing is then passed on to the file
`{\textit{dest}\hspace{0.2em}\textit{suffix}}'.
Surely, the same effect is achieved by
directly specifying the
argument `{\textit{dest}\hspace{0.2em}\textit{suffix}}'
in the first form.
However, that requires to set up a different file
for each child. With the alternative form of the command
all these files can have exactly the same content
which simplifies setting them up and maintaining them.

For example, the following file |draft.tex|
with a compilation flag |\version| as described in \secref{sec:flags}
compiles the main document as a draft:
%
\begin{center}
\begin{tabular}{l}
|\def\version{draft}|\\
|\input{childdoc.def}|\\
|\childdocforward{|\textit{main}|}|
\end{tabular}
\end{center}
%
Likewise, the following files |final|\textit{nn}|.tex|
compile the final version of the child document
|child|\textit{nn}|.tex|:
%
\begin{center}
\begin{tabular}{l}
|\def\version{final}|\\
|\input{childdoc.def}|\\
|\childdocforwardprefix{final}{child}|
\end{tabular}
\end{center}
%

Note that when several versions of a main file and/or of each child file
are to be generated, it may be convenient to set up a |Makefile| or
shell script to automatise the process.

%%%%%%%%%%%%%%%%%%%%%%%%%%%%%%%%%%%%%%%%%%%%%%%%%%%%%%%%%%%%%%%%%%%%%%%%%%%%%%%%
\subsection{Command Line Processing}
\label{sec:commandline}

The effect of redirection files can also be achieved by invoking
the \LaTeX{} compiler with a more elaborate command line.
Most conveniently this should be done as part
of a shell script or a |Makefile|.

When using \textsf{childdoc} in the main file, the following
command lines effectively perform a redirection
(note that depending on the shell being used,
backslashes may have to be doubled: `|\|' $\to$ `|\\|'):
%
\begin{center}
|... -jobname "|\textit{target}|" |\\|"|[\textit{flags}]%
|\input{childdoc.def}\childdocforward[|\textit{main}|]{|\textit{dest}|}"|
\end{center}
%
Here \textit{target} is the name of the output file,
\textit{main} is the name of the main file
and \textit{dest} is the name of the main or child file to be processed
(all filenames without extensions).
The optional argument \textit{main} can be omitted
if \textit{main} matches \textit{dest}.
Optionally, compilation \textit{flags} can be defined via |\def| commands.
This command line makes the \TeX{} engine believe
it is compiling the file \textit{target}
whose content is specified as the latter parameter.
The provided code then forwards the processing to
\textit{main} or \textit{dest} as described in \secref{sec:forward}.

%%%%%%%%%%%%%%%%%%%%%%%%%%%%%%%%%%%%%%%%%%%%%%%%%%%%%%%%%%%%%%%%%%%%%%%%%%%%%%%%
\subsection{Include by Input}
\label{sec:input}

Including child documents by |\include| has some restrictions by design.
Most notably, the content of a child document always occupies
its own set of pages; pages cannot be shared between child documents.
Usually, this behaviour makes perfect sense
because each child document contain an essential part of the document.
However, in some situations it may be desirable to compose
a document from a collection of parts
without having mandatory page breaks between then.
For this case, the package
provides a mechanism to include parts
by |\input| which can also be processed individually.
However, by construction this mechanism
requires manual handling of the content to be output.

%%%%%%%%%%%%%%%%%%%%%%%%%%%%%%%%%%%%%%%%
\DescribeMacro{\ifchilddocmanual}
The main file should be prepared as usual, see \secref{sec:include}.
However, the document body must make a distinction
between processing of an individual part and of the main document, e.g.:
%
\begin{center}
\begin{tabular}{l}
|\ifchilddocmanual|\\
|\input{\childdocname}|\\
|\||else|\\
\textit{document body with }|\input{|\textit{part}|}|\\
|\||fi|
\end{tabular}
\end{center}
%
The conditional |\ifchilddocmanual| is true whenever
a part to be included by |\input| is being compiled,
and the name of the part is stored in |\childdocname|.

%%%%%%%%%%%%%%%%%%%%%%%%%%%%%%%%%%%%%%%%
\DescribeMacro{\childdocby}
Each part to be included by |\input| should start with:
%
\begin{center}
\begin{tabular}{l}
|\input{childdoc.def}|\\
|\childdocby{|\textit{main}|}|\\
\end{tabular}
\end{center}
%
The directive |\childdocby| is similar to |\childdocof|
described in \secref{sec:include},
but the subsequent selection of content must be done manually.
To that end, both |\ifchilddoc| and |\ifchilddocmanual|
will be true upon processing of a part,
and the name of the part is stored in |\childdocname|.
Note that |\jobname| will be set to the filename of the current part
so that each part receives an individual |.aux| file
that does not interfere with the |.aux| file(s) of the main document.
This behaviour can be altered by the alternative form
|\childdocby[*]{|\textit{main}|}| (with a non-empty optional argument)
which uses the |.aux| file of the main document
by setting |\jobname| to \textit{main}.

%%%%%%%%%%%%%%%%%%%%%%%%%%%%%%%%%%%%%%%%%%%%%%%%%%%%%%%%%%%%%%%%%%%%%%%%%%%%%%%%
\subsection{Driver Development}
\label{sec:driver}

The \textsf{childdoc} mechanism can also be use for the development
of definition files such as \LaTeX{} styles or classes.
This case differs from the above setup with multiple parts
included by |\include| in that no |\includeonly| should be invoked.
This can be achieved by starting the include file
(before |\ProvidesPackage|) with:
%
\begin{center}
\begin{tabular}{l}
|\input{childdoc.def}|\\
|\childdocforward{|\textit{main}|}|\\
\end{tabular}
\end{center}
%
or alternatively with:
%
\begin{center}
\begin{tabular}{l}
|\input{childdoc.def}|\\
|\childdocby{|\textit{main}|}|\\
\end{tabular}
\end{center}
%
Both forms have slightly different effects as described above.
The main file is prepared as usual, see \secref{sec:include}.

%%%%%%%%%%%%%%%%%%%%%%%%%%%%%%%%%%%%%%%%%%%%%%%%%%%%%%%%%%%%%%%%%%%%%%%%%%%%%%%%
\subsection{Legacy Detection}
\label{sec:detection}

The directive |\childdocmain| in the main file can detect
whether the complete document or merely a child is to be compiled
even without using the directive |\childdocof|.
This method is deprecated because it is less robust
and there is no compelling reason to use it;
it is merely provided for backward compatibility
and it may be removed in future versions.

If the detection mechanism is to be used,
it is mandatory to correctly specify
the filename of the main file as the argument of |\childdocmain|:
%
\begin{center}
\begin{tabular}{l}
|\input{childdoc.def}|\\
|\childdocmain{|\textit{main}|}|\\
\end{tabular}
\end{center}
%
If |\jobname| does not match the argument \textit{main} of |\childdocmain|,
it is assumed that |\jobname| points to the child file to be compiled.
When using |\childdocmain| with the main file specified as argument,
it suffices to start a child file
with just |\input{|\textit{main}|}|
without loading of the package and using |\childdocof|.
If instead all processing is done
with the appropriate \textsf{childdoc} directives,
the argument of \textit{main} of |\childdocmain| can be empty.

An alternative version of the command line processing described
in \secref{sec:commandline} using the detection mechanism reads:
%
\begin{center}
|... -jobname "|\textit{target}|" "|[\textit{flags}]%
[|\def\jobname{|\textit{dest}|}|]|\input{|\textit{main}|}"|
\end{center}

%%%%%%%%%%%%%%%%%%%%%%%%%%%%%%%%%%%%%%%%%%%%%%%%%%%%%%%%%%%%%%%%%%%%%%%%%%%%%%%%
\subsection{Manual Code}
\label{sec:manual}

In case one cannot be certain whether the definitions file |childdoc.def|
is installed on the target \TeX{} distribution
and one prefers not to ship it,
it is conceivable to paste a few relevant commands into the sources.

To that end, drop all statements |\input{childdoc.def}|
and perform the replacements as outlined below.
Instead of |\childdocmain{|\textit{main}|}| add the following code
to the top of the main file:
%
\begin{center}
\begin{tabular}{l}
|\||ifdefined\childdocname\endinput\||fi\newif\ifchilddoc|\\
|\edef\childdocname{\scantokens\expandafter{\jobname\noexpand}}|\\
|\def\childdocmain{|\textit{main}|}\||ifx\childdocmain\childdocname\||else|\\
|\childdoctrue\includeonly{\childdocname}\let\jobname\childdocmain\||fi|\\
\end{tabular}
\end{center}
%
Instead of |\childdocof{|\textit{main}|}| just include the main file
at the top of each child file:
%
\begin{center}
|\input{|\textit{main}|}|
\end{center}
%
A simple redirection |\childdocforward{|\textit{dest}|}| is achieved by:
%
\begin{center}
|\def\jobname{|\textit{dest}|}\input{\jobname}|
\end{center}
%
The redirection with prefix
|\childdocforwardprefix[|\textit{prefix}|]{|\textit{dest}|}|
is accomplished by:
%
\begin{center}
\begin{tabular}{l}
|{\edef\jobname{\scantokens\expandafter{\jobname\noexpand}}|\\
|\def\redirectjob |\textit{prefix}|#1~~~{\gdef\jobname{|\textit{dest}|#1}}|\\
|\expandafter\redirectjob\jobname~~~}\input{\jobname}|
\end{tabular}
\end{center}

In an alternative approach,
child documents can be compiled by a specific command line
without additional code or specific definitions:
%
\begin{center}
|... -jobname "|\textit{target}|" "|[\textit{flags}]%
|\includeonly{|\textit{dest}|}\input{|\textit{main}|}"|
\end{center}
%

%%%%%%%%%%%%%%%%%%%%%%%%%%%%%%%%%%%%%%%%%%%%%%%%%%%%%%%%%%%%%%%%%%%%%%%%%%%%%%%%
%%%%%%%%%%%%%%%%%%%%%%%%%%%%%%%%%%%%%%%%%%%%%%%%%%%%%%%%%%%%%%%%%%%%%%%%%%%%%%%%
\section{Information}

%%%%%%%%%%%%%%%%%%%%%%%%%%%%%%%%%%%%%%%%%%%%%%%%%%%%%%%%%%%%%%%%%%%%%%%%%%%%%%%%
\subsection{Copyright}

Copyright \copyright{} 2017--2018 Niklas Beisert

This work may be distributed and/or modified under the
conditions of the \LaTeX{} Project Public License, either version 1.3
of this license or (at your option) any later version.
The latest version of this license is in
  \url{http://www.latex-project.org/lppl.txt}
and version 1.3 or later is part of all distributions of \LaTeX{}
version 2005/12/01 or later.

This work has the LPPL maintenance status `maintained'.

The Current Maintainer of this work is Niklas Beisert.

This work consists of the files |README.txt|, |childdoc.ins| and |childdoc.dtx|
as well as the derived files |childdoc.def|, |cdocsamp.tex|
with |cdocsch1.tex|, |cdocsch2.tex|, |cdocspt3.tex|, |cdocspt4.tex|,
|cdocsdrf.tex|, |cdocsfn1.tex|, |cdocsfn2.tex|
as well as |childdoc.pdf|.

%%%%%%%%%%%%%%%%%%%%%%%%%%%%%%%%%%%%%%%%%%%%%%%%%%%%%%%%%%%%%%%%%%%%%%%%%%%%%%%%
\subsection{Files and Installation}

The package consists of the files:
%
\begin{center}
\begin{tabular}{ll}
    |README.txt|   & readme file \\
    |childdoc.ins| & installation file \\
    |childdoc.dtx| & source file \\
    |childdoc.def| & definition file \\
    |cdocsamp.tex| & sample main file \\
    |cdocsch1.tex| & sample include file \\
    |cdocsch2.tex| & sample include file \\
    |cdocspt3.tex| & sample part file \\
    |cdocspt4.tex| & sample part file \\
    |cdocsdrf.tex| & sample redirection file \\
    |cdocsfn1.tex| & sample redirection file \\
    |cdocsfn2.tex| & sample redirection file \\
    |childdoc.pdf| & manual
\end{tabular}
\end{center}
%
The distribution consists of the files
|README.txt|, |childdoc.ins| and |childdoc.dtx|.
%
\begin{itemize}
\item
Run (pdf)\LaTeX{} on |childdoc.dtx|
to compile the manual |childdoc.pdf| (this file).
\item
Run \LaTeX{} on |childdoc.ins| to create the definitions file |childdoc.def|
and the sample |cdocsamp.tex| with include files
|cdocsch1.tex|, |cdocsch2.tex|, |cdocspt3.tex|, |cdocspt4.tex|,
|cdocsdrf.tex|, |cdocsfn1.tex|, |cdocsfn2.tex|.
Then copy the file |childdoc.def| to an appropriate directory of your \LaTeX{}
distribution, e.g.\ \textit{texmf-root}|/tex/latex/childdoc|.
\end{itemize}

%%%%%%%%%%%%%%%%%%%%%%%%%%%%%%%%%%%%%%%%%%%%%%%%%%%%%%%%%%%%%%%%%%%%%%%%%%%%%%%%
\subsection{Related CTAN Packages}

There are several other packages which offer a similar functionality:
%
\begin{itemize}
\item
The packages
\href{http://ctan.org/pkg/docmute}{\textsf{docmute}},
\href{http://ctan.org/pkg/includex}{\textsf{includex}} and
\href{http://ctan.org/pkg/standalone}{\textsf{standalone}}
provide commands to include only the document body of
a child file thus allowing both files to be compiled individually.
\item
The packages \href{http://ctan.org/pkg/subdocs}{\textsf{subdocs}}
and \href{http://ctan.org/pkg/subfiles}{\textsf{subfiles}}
provide structures in which the main and child documents can be
encapsulated and allowing them to be compiled individually.
The inclusion mechanism is different from the conventional |\include|.
\item
The package \href{http://ctan.org/pkg/combine}{\textsf{combine}}
is an elaborate solution to combine several documents into one.
\end{itemize}
%
See also the CTAN topic \href{http://ctan.org/topic/subdocs}{\textsf{subdocs}}
for further related packages.
The present package differs from the above solutions in that
a document structure constructed with the conventional |\include| mechanism
just needs two extra commands at the top of every file
such that all constituent files can be compiled individually.

%%%%%%%%%%%%%%%%%%%%%%%%%%%%%%%%%%%%%%%%%%%%%%%%%%%%%%%%%%%%%%%%%%%%%%%%%%%%%%%%
%\subsection{Feature Suggestions}
%
%The following is a list of features which may be useful for future
%versions of this package:
%%
%\begin{itemize}
%\item
%\ldots
%\end{itemize}

%%%%%%%%%%%%%%%%%%%%%%%%%%%%%%%%%%%%%%%%%%%%%%%%%%%%%%%%%%%%%%%%%%%%%%%%%%%%%%%%
\subsection{Revision History}

%%%%%%%%%%%%%%%%%%%%%%%%%%%%%%%%%%%%%%%%
\paragraph{v2.0:} 2018/12/30

\begin{itemize}
\item
immediate forward processing
\item
added |\childdocby| mechanism
\item
manual restructured
\end{itemize}

%%%%%%%%%%%%%%%%%%%%%%%%%%%%%%%%%%%%%%%%
\paragraph{v1.6:} 2018/01/17

\begin{itemize}
\item
application for development of include files
\item
corrections to manual
\end{itemize}

%%%%%%%%%%%%%%%%%%%%%%%%%%%%%%%%%%%%%%%%
\paragraph{v1.5:} 2017/05/21

\begin{itemize}
\item
more complete structuring introduced
\item
|\childdocof| introduced
\item
|\childdoc| renamed to |\childdocmain|
\item
|\childredirect| renamed to |\childdocforward| and |\childdocforwardprefix|
and functionality expanded
\end{itemize}

%%%%%%%%%%%%%%%%%%%%%%%%%%%%%%%%%%%%%%%%
\paragraph{v1.0:} 2017/04/27

\begin{itemize}
\item
manual and install package
\item
first version published on CTAN
\end{itemize}

%%%%%%%%%%%%%%%%%%%%%%%%%%%%%%%%%%%%%%%%
\paragraph{v0.6:} 2017/04/26

\begin{itemize}
\item
redirection mechanism added
\end{itemize}

%%%%%%%%%%%%%%%%%%%%%%%%%%%%%%%%%%%%%%%%
\paragraph{v0.5:} 2017/04/26

\begin{itemize}
\item
functionality in definition file
\end{itemize}


%%%%%%%%%%%%%%%%%%%%%%%%%%%%%%%%%%%%%%%%%%%%%%%%%%%%%%%%%%%%%%%%%%%%%%%%%%%%%%%%
%%%%%%%%%%%%%%%%%%%%%%%%%%%%%%%%%%%%%%%%%%%%%%%%%%%%%%%%%%%%%%%%%%%%%%%%%%%%%%%%
%%%%%%%%%%%%%%%%%%%%%%%%%%%%%%%%%%%%%%%%%%%%%%%%%%%%%%%%%%%%%%%%%%%%%%%%%%%%%%%%
\appendix

\settowidth\MacroIndent{\rmfamily\scriptsize 000\ }

 \DocInput{childdoc.dtx}

\end{document}
%</driver>
% \fi
%
% %%%%%%%%%%%%%%%%%%%%%%%%%%%%%%%%%%%%%%%%%%%%%%%%%%%%%%%%%%%%%%%%%%%%%%%%%%%%%%
% %%%%%%%%%%%%%%%%%%%%%%%%%%%%%%%%%%%%%%%%%%%%%%%%%%%%%%%%%%%%%%%%%%%%%%%%%%%%%%
% \section{Sample}
%\iffalse
%<*samplemain>
%\fi
%
% The following presents a sample document
% with two chapters, two parts, a title page,
% a compile flag as well as three forwarding files to set the flag.
% It consists of eight |.tex| files:
% \begin{center}
% \begin{tabular}{ll}
% |cdocsamp.tex|&main file\\
% |cdocsch1.tex|&include file for chapter 1\\
% |cdocsch2.tex|&include file for chapter 2\\
% |cdocspt3.tex|&include file for part 3\\
% |cdocspt4.tex|&include file for part 4\\
% |cdocsdrf.tex|&forwarding file for main file in draft mode\\
% |cdocsfi1.tex|&forwarding file for final version of chapter 1\\
% |cdocsfi2.tex|&forwarding file for final version of chapter 2\\
% \end{tabular}
% \end{center}
% Each of the eight files can be compiled directly by the \LaTeX{} compiler.
%
% %%%%%%%%%%%%%%%%%%%%%%%%%%%%%%%%%%%%%%
% \paragraph{Main File.}
%
% The main file is called |cdocsamp.tex|.
%
% Load the \textsf{childdoc} definitions and
% declare the filename for the main document:
%    \begin{macrocode}
\input{childdoc.def}
\childdocmain{}
%    \end{macrocode}

% Optional override for |\version| flag:
%    \begin{macrocode}
%%\ifchilddoc\else\providecommand{\version}{draft}\fi
%    \end{macrocode}

% Define the default values for the |\version| flag
% (|final| for the main file and |draft| for childs):
%    \begin{macrocode}
\ifchilddoc
\providecommand{\version}{draft}
\else
\providecommand{\version}{final}
\fi
%    \end{macrocode}

% Load the standard document class:
%    \begin{macrocode}
\documentclass[12pt]{article}
%    \end{macrocode}

% Start the document body:
%    \begin{macrocode}
\begin{document}
%    \end{macrocode}

% Declare a title page.
% Print title, part of document being processed and version flag:
%    \begin{macrocode}
\addtocounter{page}{-1}
\begin{center}
{\LARGE\bfseries{}childdoc example\par}
\vspace{1cm}
\ifchilddoc
\ifchilddocmanual part\else chapter\fi:
`\childdocname' of `\childdocjob'\par
\else
main document: `\childdocjob'\par
\fi
version: \version\par
\end{center}
\newpage
%    \end{macrocode}

% Manually include selected file,
% otherwise process as usual:
%    \begin{macrocode}
\ifchilddocmanual
\section*{part `\childdocname'}
\input{\childdocname}
\else
%    \end{macrocode}

% Include the two chapters:
%    \begin{macrocode}
\include{cdocsch1}
\include{cdocsch2}
%    \end{macrocode}

% Include the two parts unless only chapters should be displayed:
%    \begin{macrocode}
\ifchilddoc\else
\section{part three}
\input{cdocspt3}
\section{part four}
\input{cdocspt4}
\fi
%    \end{macrocode}

% Process as usual until here:
%    \begin{macrocode}
\fi
%    \end{macrocode}

% End of document body:
%    \begin{macrocode}
\end{document}
%    \end{macrocode}
%\iffalse
%</samplemain>
%\fi
%
% %%%%%%%%%%%%%%%%%%%%%%%%%%%%%%%%%%%%%%
% \paragraph{Chapter Include Files.}
%
% The include files are called |cdocsch1.tex| and |cdocsch2.tex|.
%
%\iffalse
%<*samplechap1|samplechap2>
%\fi

% Optional override for |\version| flag:
%    \begin{macrocode}
%%\providecommand{\version}{final}
%    \end{macrocode}

% Include the main document:
%    \begin{macrocode}
\input{childdoc.def}
\childdocof{cdocsamp}
%    \end{macrocode}

%\iffalse
%</samplechap1|samplechap2>
%\fi
%
%\iffalse
%<*samplechap1>
%\fi
% Some text for chapter 1:
%    \begin{macrocode}
\section{one}
some text in chapter one
%    \end{macrocode}

%\iffalse
%</samplechap1>
%\fi
% Some text for chapter 2:
%\iffalse
%<*samplechap2>
%\fi
%    \begin{macrocode}
\section{two}
more text in chapter two
%    \end{macrocode}

%\iffalse
%</samplechap2>
%\fi
%
% %%%%%%%%%%%%%%%%%%%%%%%%%%%%%%%%%%%%%%
% \paragraph{Part Include Files.}
%
% The include files are called |cdocspt3.tex| and |cdocspt4.tex|.
%
%\iffalse
%<*samplepart3|samplepart4>
%\fi

% Optional override for |\version| flag:
%    \begin{macrocode}
%%\providecommand{\version}{final}
%    \end{macrocode}

% Include the main document:
%    \begin{macrocode}
\input{childdoc.def}
\childdocby{cdocsamp}
%    \end{macrocode}

%\iffalse
%</samplepart3|samplepart4>
%\fi
%
%\iffalse
%<*samplepart3>
%\fi
% Some text for part 3:
%    \begin{macrocode}
some text in part three
%    \end{macrocode}

%\iffalse
%</samplepart3>
%\fi
% Some text for part 4:
%\iffalse
%<*samplepart4>
%\fi
%    \begin{macrocode}
more text in part four
%    \end{macrocode}

%\iffalse
%</samplepart4>
%\fi
%
% %%%%%%%%%%%%%%%%%%%%%%%%%%%%%%%%%%%%%%
% \paragraph{Forwarding for a Complete Draft.}
%
% The following forwarding file |cdocsdrf.tex|
% compiles the main document in draft mode:
%\iffalse
%<*sampledraft>
%\fi
%    \begin{macrocode}
\def\version{draft}
\input{childdoc.def}
\childdocforward{cdocsamp}
%    \end{macrocode}

%\iffalse
%</sampledraft>
%\fi
%
% %%%%%%%%%%%%%%%%%%%%%%%%%%%%%%%%%%%%%%
% \paragraph{Forwarding for Final Version of the Chapters.}
%
% The following forwarding files |cdocsfn1.tex| and |cdocsfn2.tex|
% (with identical content)
% compile the final versions of the child documents
% |cdocsch1.tex| and |cdocsch2.tex|, respectively:
%\iffalse
%<*samplefinal>
%\fi
%    \begin{macrocode}
\def\version{final}
\input{childdoc.def}
\childdocforwardprefix[cdocsamp]{cdocsfn}{cdocsch}
%    \end{macrocode}

%\iffalse
%</samplefinal>
%\fi
%
% %%%%%%%%%%%%%%%%%%%%%%%%%%%%%%%%%%%%%%
% \paragraph{Command Line Processing.}
%
% The following three command lines generate the output files
% |cdocscld|, |cdocscl1| and |cdocscl2|
% which should be identical to
% |cdocsdrf|, |cdocsch1| and |cdocsfn2|, respectively:
% \begin{center}
% \begin{tabular}{l}
% |latex -jobname cdocscld \|\\
% |  "\def\version{draft}\input{childdoc.def}\childdocforward{cdocsamp}"|\\
% |latex -jobname cdocscl1 \|\\
% |  "\input{childdoc.def}\childdocforward[cdocsamp]{cdocsch1}"|\\
% |latex -jobname cdocscl2 \|\\
% |  "\def\version{final}\input{childdoc.def}\childdocforward{cdocsch2}"|
% \end{tabular}
% \end{center}
% Note that the trailing backslash on each first line
% merely continues the input to the second line
% (for convenient cut ant paste).
% Furthermore, the command |latex| can be replaced by any
% of its alternative versions such as |pdflatex|.
%
% %%%%%%%%%%%%%%%%%%%%%%%%%%%%%%%%%%%%%%%%%%%%%%%%%%%%%%%%%%%%%%%%%%%%%%%%%%%%%%
% %%%%%%%%%%%%%%%%%%%%%%%%%%%%%%%%%%%%%%%%%%%%%%%%%%%%%%%%%%%%%%%%%%%%%%%%%%%%%%
% \section{Implementation}
%\iffalse
%<*package>
%\fi
%
% This section describes the definitions file |childdoc.def|.

% The definitions cannot be loaded using |\usepackage| or |\RequirePackage|
% which has a mechanism to prevent loading a style file more than once.
% When loading the definitions by means of |\input|
% multiple instances have to be prevented manually:
%\iffalse
%This code needs to be before the `\ProvidesFile' directive
%which is defined at the beginning of this file.
%Therefore it is also placed there and commented out here.
%</package>
%<*discard>
%\fi
%    \begin{macrocode}
\ifdefined\childdocmain\endinput\fi
%    \end{macrocode}
%\iffalse
%</discard>
%<*package>
%\fi
%
% \macro{\ifchilddoc}
% \macro{\ifchilddocmanual}
% The conditional |\ifchilddoc| tells whether a
% child (true) or main (false) document is being compiled.
% The conditional |\ifchilddocmanual| tells whether
% the |\includeonly| mechanism is used (false) or
% the selection of child files must be performed manually (true).
% The definitions initialise to false:
%    \begin{macrocode}
\newif\ifchilddoc
\newif\ifchilddocmanual
%    \end{macrocode}

% \macro{\childdocname}
% \macro{\childdocjob}
% The macro |\childdocname| stores the name of the main document
% to be compiled. The macro |\childdocjob| stores the name of
% the document on which the \LaTeX{} compiler was originally invoked.
% The content of |\jobname| cannot be compared
% to filenames specified in the source due to different catcodes.
% The following code rescans |\jobname|, stores the result
% in |\childdocname| and saves a copy in |\childdocjob|:
%    \begin{macrocode}
\edef\childdocname{\scantokens\expandafter{\jobname\noexpand}}
\let\childdocjob\childdocname
%    \end{macrocode}

% \macro{\childdocdisable}
% The macro |\childdocdisable| prevents the main file
% from being processed more than once.
% At this stage, the main document command |\childdocmain|
% is assumed to be called once again where it should do nothing.
% Any subsequent call to it should prevent
% a secondary processing of the main document
% It overwrites the forwarding commands
% |\childdocof| and |\childdocforward|
% with empty macros to prevent further inclusions of the main document:
%    \begin{macrocode}
\newcommand{\childdocdisable}
{
  \renewcommand{\childdocmain}[1]{\renewcommand{\childdocmain}[1]{\endinput}}
  \renewcommand{\childdocof}[1]{}
  \renewcommand{\childdocby}[2][]{}
  \renewcommand{\childdocforward}[2][]{}
  \renewcommand{\childdocdisable}{}
}
%    \end{macrocode}

% \macro{\childdocmain}
% The macro |\childdocmain| is to be called at the top of the main file
% with nothing or the main filename (without extension) as argument.
% First, it breaks loops.
% If the argument is not empty and does not match |\childdocname|
% (which is set by the first inclusion of |childdoc.def|),
% |\ifchilddoc| is set to true, |\includeonly| is applied to the child file
% and |\jobname| is set to the main file
% (for proper handling of |.aux| files):
%    \begin{macrocode}
\newcommand{\childdocmain}[1]
{
  \childdocdisable\childdocmain{}
  \if?#1?\else
    \begingroup
      \def\childdoctmp{#1}
      \ifx\childdoctmp\childdocname
        \def\childdoctmp{}
      \else
        \def\childdoctmp
        {
          \childdoctrue
          \includeonly{\childdocname}
          \def\childdocjob{#1}
          \def\jobname{#1}
        }
      \fi
      \expandafter
    \endgroup
    \childdoctmp
  \fi
}
%    \end{macrocode}

% \macro{\childdocof}
% The command |\childdocof| redirects
% compilation to the main file |#1|.
%    \begin{macrocode}
\newcommand{\childdocof}[1]
{
  \childdocdisable
  \childdoctrue
  \includeonly{\childdocname}
  \def\jobname{#1}
  \def\childdocjob{#1}
  \input{#1}
}
%    \end{macrocode}

% \macro{\childdocby}
% The command |\childdocby| ....
%    \begin{macrocode}
\newcommand{\childdocby}[2][]
{
  \childdocdisable
  \childdoctrue
  \childdocmanualtrue
  \if?#1?\else
    \def\jobname{#2}
  \fi
  \def\childdocjob{#2}
  \input{#2}
  \endinput
}
%    \end{macrocode}

% \macro{\childdocforward}
% The command |\childdocforward| redirects
% compilation to the main file or
% (if the optional argument is given) a child file.
% Parameters are set as if the main file
% or a child file starting with |\childdocof| was compiled.
% Then compilation is handed over to the main file:
%    \begin{macrocode}
\newcommand{\childdocforward}[2][]
{
  \begingroup
    \if?#1?
      \def\childdoctmp
      {
        \def\childdocname{#2}
        \def\childdocjob{#2}
        \def\jobname{#2}
        \input{#2}
        \endinput
      }
    \else
      \def\childdoctmp
      {
        \childdocdisable
        \def\childdocname{#2}
        \childdoctrue
        \includeonly{#2}
        \def\childdocjob{#1}
        \def\jobname{#1}
        \input{#1}
        \endinput
      }
    \fi
    \expandafter
  \endgroup
  \childdoctmp
}
%    \end{macrocode}

% \macro{\childdocforwardprefix}
% The command |\childdocforwardprefix| redirects
% compilation to the main or a child file by means of a pattern.
% The prefix |#1| in the current filename is replaced by |#2|
% and the suffix of the current filename is kept
% (it is assumed that the filename does not contain the substring `|~~~|'
% which is used as a delimiter).
% Compilation is handed over to the new file by |\childdocforward|:
%    \begin{macrocode}
\newcommand{\childdocforwardprefix}[3][]
{
  \begingroup
    \def\childdocextract #2##1~~~{\def\childdoctmp{\childdocforward[#1]{#3##1}}}
    \expandafter\childdocextract\childdocname~~~
    \expandafter
  \endgroup
  \childdoctmp
}
%    \end{macrocode}

% \macro{\childdoc}
% The deprecated macro |\childdoc| is a legacy version of |\childdocmain|:
%    \begin{macrocode}
\newcommand{\childdoc}{\childdocmain}
%    \end{macrocode}

% \macro{\childdocredirect}
% The deprecated macro |\childdocredirect| is a legacy version
% of |\childdocforward| and |\childdocforwardprefix|:
%    \begin{macrocode}
\newcommand{\childdocredirect}[2][]
{
  \begingroup
    \if?#1?
      \def\childdoctmp{\childdocforward{#2}}
    \else
      \def\childdoctmp{\childdocforwardprefix{#1}{#2}}
    \fi
    \expandafter
  \endgroup
  \childdoctmp
}
%    \end{macrocode}

%\iffalse
%</package>
%\fi
%
\endinput
|
and perform the replacements as outlined below.
Instead of |\childdocmain{|\textit{main}|}| add the following code
to the top of the main file:
%
\begin{center}
\begin{tabular}{l}
|\||ifdefined\childdocname\endinput\||fi\newif\ifchilddoc|\\
|\edef\childdocname{\scantokens\expandafter{\jobname\noexpand}}|\\
|\def\childdocmain{|\textit{main}|}\||ifx\childdocmain\childdocname\||else|\\
|\childdoctrue\includeonly{\childdocname}\let\jobname\childdocmain\||fi|\\
\end{tabular}
\end{center}
%
Instead of |\childdocof{|\textit{main}|}| just include the main file
at the top of each child file:
%
\begin{center}
|\input{|\textit{main}|}|
\end{center}
%
A simple redirection |\childdocforward{|\textit{dest}|}| is achieved by:
%
\begin{center}
|\def\jobname{|\textit{dest}|}\input{\jobname}|
\end{center}
%
The redirection with prefix
|\childdocforwardprefix[|\textit{prefix}|]{|\textit{dest}|}|
is accomplished by:
%
\begin{center}
\begin{tabular}{l}
|{\edef\jobname{\scantokens\expandafter{\jobname\noexpand}}|\\
|\def\redirectjob |\textit{prefix}|#1~~~{\gdef\jobname{|\textit{dest}|#1}}|\\
|\expandafter\redirectjob\jobname~~~}\input{\jobname}|
\end{tabular}
\end{center}

In an alternative approach,
child documents can be compiled by a specific command line
without additional code or specific definitions:
%
\begin{center}
|... -jobname "|\textit{target}|" "|[\textit{flags}]%
|\includeonly{|\textit{dest}|}\input{|\textit{main}|}"|
\end{center}
%

%%%%%%%%%%%%%%%%%%%%%%%%%%%%%%%%%%%%%%%%%%%%%%%%%%%%%%%%%%%%%%%%%%%%%%%%%%%%%%%%
%%%%%%%%%%%%%%%%%%%%%%%%%%%%%%%%%%%%%%%%%%%%%%%%%%%%%%%%%%%%%%%%%%%%%%%%%%%%%%%%
\section{Information}

%%%%%%%%%%%%%%%%%%%%%%%%%%%%%%%%%%%%%%%%%%%%%%%%%%%%%%%%%%%%%%%%%%%%%%%%%%%%%%%%
\subsection{Copyright}

Copyright \copyright{} 2017--2018 Niklas Beisert

This work may be distributed and/or modified under the
conditions of the \LaTeX{} Project Public License, either version 1.3
of this license or (at your option) any later version.
The latest version of this license is in
  \url{http://www.latex-project.org/lppl.txt}
and version 1.3 or later is part of all distributions of \LaTeX{}
version 2005/12/01 or later.

This work has the LPPL maintenance status `maintained'.

The Current Maintainer of this work is Niklas Beisert.

This work consists of the files |README.txt|, |childdoc.ins| and |childdoc.dtx|
as well as the derived files |childdoc.def|, |cdocsamp.tex|
with |cdocsch1.tex|, |cdocsch2.tex|, |cdocspt3.tex|, |cdocspt4.tex|,
|cdocsdrf.tex|, |cdocsfn1.tex|, |cdocsfn2.tex|
as well as |childdoc.pdf|.

%%%%%%%%%%%%%%%%%%%%%%%%%%%%%%%%%%%%%%%%%%%%%%%%%%%%%%%%%%%%%%%%%%%%%%%%%%%%%%%%
\subsection{Files and Installation}

The package consists of the files:
%
\begin{center}
\begin{tabular}{ll}
    |README.txt|   & readme file \\
    |childdoc.ins| & installation file \\
    |childdoc.dtx| & source file \\
    |childdoc.def| & definition file \\
    |cdocsamp.tex| & sample main file \\
    |cdocsch1.tex| & sample include file \\
    |cdocsch2.tex| & sample include file \\
    |cdocspt3.tex| & sample part file \\
    |cdocspt4.tex| & sample part file \\
    |cdocsdrf.tex| & sample redirection file \\
    |cdocsfn1.tex| & sample redirection file \\
    |cdocsfn2.tex| & sample redirection file \\
    |childdoc.pdf| & manual
\end{tabular}
\end{center}
%
The distribution consists of the files
|README.txt|, |childdoc.ins| and |childdoc.dtx|.
%
\begin{itemize}
\item
Run (pdf)\LaTeX{} on |childdoc.dtx|
to compile the manual |childdoc.pdf| (this file).
\item
Run \LaTeX{} on |childdoc.ins| to create the definitions file |childdoc.def|
and the sample |cdocsamp.tex| with include files
|cdocsch1.tex|, |cdocsch2.tex|, |cdocspt3.tex|, |cdocspt4.tex|,
|cdocsdrf.tex|, |cdocsfn1.tex|, |cdocsfn2.tex|.
Then copy the file |childdoc.def| to an appropriate directory of your \LaTeX{}
distribution, e.g.\ \textit{texmf-root}|/tex/latex/childdoc|.
\end{itemize}

%%%%%%%%%%%%%%%%%%%%%%%%%%%%%%%%%%%%%%%%%%%%%%%%%%%%%%%%%%%%%%%%%%%%%%%%%%%%%%%%
\subsection{Related CTAN Packages}

There are several other packages which offer a similar functionality:
%
\begin{itemize}
\item
The packages
\href{http://ctan.org/pkg/docmute}{\textsf{docmute}},
\href{http://ctan.org/pkg/includex}{\textsf{includex}} and
\href{http://ctan.org/pkg/standalone}{\textsf{standalone}}
provide commands to include only the document body of
a child file thus allowing both files to be compiled individually.
\item
The packages \href{http://ctan.org/pkg/subdocs}{\textsf{subdocs}}
and \href{http://ctan.org/pkg/subfiles}{\textsf{subfiles}}
provide structures in which the main and child documents can be
encapsulated and allowing them to be compiled individually.
The inclusion mechanism is different from the conventional |\include|.
\item
The package \href{http://ctan.org/pkg/combine}{\textsf{combine}}
is an elaborate solution to combine several documents into one.
\end{itemize}
%
See also the CTAN topic \href{http://ctan.org/topic/subdocs}{\textsf{subdocs}}
for further related packages.
The present package differs from the above solutions in that
a document structure constructed with the conventional |\include| mechanism
just needs two extra commands at the top of every file
such that all constituent files can be compiled individually.

%%%%%%%%%%%%%%%%%%%%%%%%%%%%%%%%%%%%%%%%%%%%%%%%%%%%%%%%%%%%%%%%%%%%%%%%%%%%%%%%
%\subsection{Feature Suggestions}
%
%The following is a list of features which may be useful for future
%versions of this package:
%%
%\begin{itemize}
%\item
%\ldots
%\end{itemize}

%%%%%%%%%%%%%%%%%%%%%%%%%%%%%%%%%%%%%%%%%%%%%%%%%%%%%%%%%%%%%%%%%%%%%%%%%%%%%%%%
\subsection{Revision History}

%%%%%%%%%%%%%%%%%%%%%%%%%%%%%%%%%%%%%%%%
\paragraph{v2.0:} 2018/12/30

\begin{itemize}
\item
immediate forward processing
\item
added |\childdocby| mechanism
\item
manual restructured
\end{itemize}

%%%%%%%%%%%%%%%%%%%%%%%%%%%%%%%%%%%%%%%%
\paragraph{v1.6:} 2018/01/17

\begin{itemize}
\item
application for development of include files
\item
corrections to manual
\end{itemize}

%%%%%%%%%%%%%%%%%%%%%%%%%%%%%%%%%%%%%%%%
\paragraph{v1.5:} 2017/05/21

\begin{itemize}
\item
more complete structuring introduced
\item
|\childdocof| introduced
\item
|\childdoc| renamed to |\childdocmain|
\item
|\childredirect| renamed to |\childdocforward| and |\childdocforwardprefix|
and functionality expanded
\end{itemize}

%%%%%%%%%%%%%%%%%%%%%%%%%%%%%%%%%%%%%%%%
\paragraph{v1.0:} 2017/04/27

\begin{itemize}
\item
manual and install package
\item
first version published on CTAN
\end{itemize}

%%%%%%%%%%%%%%%%%%%%%%%%%%%%%%%%%%%%%%%%
\paragraph{v0.6:} 2017/04/26

\begin{itemize}
\item
redirection mechanism added
\end{itemize}

%%%%%%%%%%%%%%%%%%%%%%%%%%%%%%%%%%%%%%%%
\paragraph{v0.5:} 2017/04/26

\begin{itemize}
\item
functionality in definition file
\end{itemize}


%%%%%%%%%%%%%%%%%%%%%%%%%%%%%%%%%%%%%%%%%%%%%%%%%%%%%%%%%%%%%%%%%%%%%%%%%%%%%%%%
%%%%%%%%%%%%%%%%%%%%%%%%%%%%%%%%%%%%%%%%%%%%%%%%%%%%%%%%%%%%%%%%%%%%%%%%%%%%%%%%
%%%%%%%%%%%%%%%%%%%%%%%%%%%%%%%%%%%%%%%%%%%%%%%%%%%%%%%%%%%%%%%%%%%%%%%%%%%%%%%%
\appendix

\settowidth\MacroIndent{\rmfamily\scriptsize 000\ }

 \DocInput{childdoc.dtx}

\end{document}
%</driver>
% \fi
%
% %%%%%%%%%%%%%%%%%%%%%%%%%%%%%%%%%%%%%%%%%%%%%%%%%%%%%%%%%%%%%%%%%%%%%%%%%%%%%%
% %%%%%%%%%%%%%%%%%%%%%%%%%%%%%%%%%%%%%%%%%%%%%%%%%%%%%%%%%%%%%%%%%%%%%%%%%%%%%%
% \section{Sample}
%\iffalse
%<*samplemain>
%\fi
%
% The following presents a sample document
% with two chapters, two parts, a title page,
% a compile flag as well as three forwarding files to set the flag.
% It consists of eight |.tex| files:
% \begin{center}
% \begin{tabular}{ll}
% |cdocsamp.tex|&main file\\
% |cdocsch1.tex|&include file for chapter 1\\
% |cdocsch2.tex|&include file for chapter 2\\
% |cdocspt3.tex|&include file for part 3\\
% |cdocspt4.tex|&include file for part 4\\
% |cdocsdrf.tex|&forwarding file for main file in draft mode\\
% |cdocsfi1.tex|&forwarding file for final version of chapter 1\\
% |cdocsfi2.tex|&forwarding file for final version of chapter 2\\
% \end{tabular}
% \end{center}
% Each of the eight files can be compiled directly by the \LaTeX{} compiler.
%
% %%%%%%%%%%%%%%%%%%%%%%%%%%%%%%%%%%%%%%
% \paragraph{Main File.}
%
% The main file is called |cdocsamp.tex|.
%
% Load the \textsf{childdoc} definitions and
% declare the filename for the main document:
%    \begin{macrocode}
% \iffalse
%
% childdoc.dtx Copyright (C) 2017-2018 Niklas Beisert
%
% This work may be distributed and/or modified under the
% conditions of the LaTeX Project Public License, either version 1.3
% of this license or (at your option) any later version.
% The latest version of this license is in
%   http://www.latex-project.org/lppl.txt
% and version 1.3 or later is part of all distributions of LaTeX
% version 2005/12/01 or later.
%
% This work has the LPPL maintenance status `maintained'.
%
% The Current Maintainer of this work is Niklas Beisert.
%
% This work consists of the files childdoc.dtx and childdoc.ins
% and the derived files childdoc.def and cdocsamp.tex with
% cdocsch1.tex, cdocsch2.tex, cdocsdrf.tex, cdocsfn1.tex, cdocsfn2.tex.
%
%<package>\ifdefined\childdocmain\endinput\fi
%<package>\ProvidesFile{childdoc.def}[2018/12/30 v2.0 child document driver]
%<samplemain>\ProvidesFile{cdocsamp.tex}[2018/12/30 v2.0 sample for childdoc]
%<*driver>
%\ProvidesFile{childdoc.drv}[2018/12/30 v2.0 childdoc reference manual file]
\PassOptionsToClass{10pt,a4paper}{article}
\documentclass{ltxdoc}

\usepackage[margin=35mm]{geometry}
\usepackage{hyperref}
\usepackage{hyperxmp}
\usepackage[usenames]{color}

\hypersetup{colorlinks=true}
\hypersetup{pdfstartview=FitH}
\hypersetup{pdfpagemode=UseNone}
\hypersetup{pdfsource={}}
\hypersetup{pdflang={en-UK}}
\hypersetup{pdfcopyright={Copyright 2017-2018 Niklas Beisert.
  This work may be distributed and/or modified under the
  conditions of the LaTeX Project Public License, either version 1.3
  of this license or (at your option) any later version.}}
\hypersetup{pdflicenseurl={http://www.latex-project.org/lppl.txt}}
\hypersetup{pdfcontactaddress={ETH Zurich, ITP, HIT K,
  Wolfgang-Pauli-Strasse 27}}
\hypersetup{pdfcontactpostcode={8093}}
\hypersetup{pdfcontactcity={Zurich}}
\hypersetup{pdfcontactcountry={Switzerland}}
\hypersetup{pdfcontactemail={nbeisert@itp.phys.ethz.ch}}
\hypersetup{pdfcontacturl={http://people.phys.ethz.ch/\xmptilde nbeisert/}}

\newcommand{\secref}[1]{\hyperref[#1]{section \ref*{#1}}}

\parskip1ex
\parindent0pt
\let\olditemize\itemize
\def\itemize{\olditemize\parskip0pt}

\begin{document}

\title{The \textsf{childdoc} Package}
\hypersetup{pdftitle={The childdoc Package}}
\author{Niklas Beisert\\[2ex]
  Institut f\"ur Theoretische Physik\\
  Eidgen\"ossische Technische Hochschule Z\"urich\\
  Wolfgang-Pauli-Strasse 27, 8093 Z\"urich, Switzerland\\[1ex]
  \href{mailto:nbeisert@itp.phys.ethz.ch}
  {\texttt{nbeisert@itp.phys.ethz.ch}}}
\hypersetup{pdfauthor={Niklas Beisert}}
\hypersetup{pdfsubject={Manual for the LaTeX2e Package childdoc}}
\date{30 December 2018, \textsf{v2.0}}
\maketitle

\begin{abstract}\noindent
\textsf{childdoc} is a \LaTeXe{} package
that enables the direct compilation
of document sections included by |\include|
to individual files.
\end{abstract}

\begingroup
\parskip0ex
\tableofcontents
\endgroup

%%%%%%%%%%%%%%%%%%%%%%%%%%%%%%%%%%%%%%%%%%%%%%%%%%%%%%%%%%%%%%%%%%%%%%%%%%%%%%%%
%%%%%%%%%%%%%%%%%%%%%%%%%%%%%%%%%%%%%%%%%%%%%%%%%%%%%%%%%%%%%%%%%%%%%%%%%%%%%%%%
\section{Introduction}

\LaTeX{} provides a mechanism to structure a large document (such as a book)
into a main file and several child files (containing the chapters)
using the |\include| command.
This mechanism is beneficial for documents
which span hundreds of pages in order to
make the source file(s) more manageable.
Moreover, compilation can be restricted to
selected child files by means of the |\includeonly| command.
The latter feature can be used to reduce the compilation time while editing
(this was significantly more useful in the earlier days of \LaTeX{})
or to generate a smaller document which is easier to navigate.
Another application of |\includeonly| is to generate
documents consisting of selected parts of the complete document.

However, there are a few drawbacks of the plain |\include| mechanism:
\begin{itemize}
\item
The child files cannot be compiled on their own,
they can only be compiled via the main file.
A naive editing environment
(such as a text editor with an option
to have the current file processed by \LaTeX)
may require one to switch to the main file before compiling;
attempting to compile the child file produces errors.
\item
The main file must be modified (each time)
to adjust the |\includeonly| command
to the present needs. This easily leaves the main file in a messy state.
\item
The generated document will always carry the filename
of the main document. This is inconvenient if
several child files are to be compiled and
to be kept for distribution.
\end{itemize}

The present package provides a simple interface
to make child files individually compilable by \LaTeX{}.
Compiling a child file then has the same effect as compiling
the main file with an |\includeonly| command
to select the appropriate child.
Moreover the generated document will carry the name of the child
rather than the main file.
This resolves all three above issues.

This feature is meant to make the editing of books,
thesis documents and lecture notes somewhat more convenient.
However, the package can also be used efficiently for
composing a series of documents (such as exercise sheets)
which are typically distributed individually.
It then assists the author in generating the individual documents
(potentially in different versions)
as well as a document containing the collected series.
Another application is in developing style files
or other kinds of included material
where compilation of the style file could redirect
to a sample or test file.

%%%%%%%%%%%%%%%%%%%%%%%%%%%%%%%%%%%%%%%%%%%%%%%%%%%%%%%%%%%%%%%%%%%%%%%%%%%%%%%%
%%%%%%%%%%%%%%%%%%%%%%%%%%%%%%%%%%%%%%%%%%%%%%%%%%%%%%%%%%%%%%%%%%%%%%%%%%%%%%%%
\section{Usage}

First of all, the package \textsf{childdoc} is \emph{not} a standard
\LaTeXe{} |.sty| style file! Therefore it needs to be invoked in
a non-standard way.

%%%%%%%%%%%%%%%%%%%%%%%%%%%%%%%%%%%%%%%%%%%%%%%%%%%%%%%%%%%%%%%%%%%%%%%%%%%%%%%%
\subsection{Included Files}
\label{sec:include}

%%%%%%%%%%%%%%%%%%%%%%%%%%%%%%%%%%%%%%%%
\DescribeMacro{\childdocmain}
To use the package, add the commands
\begin{center}
\begin{tabular}{l}
|\input{childdoc.def}|\\
|\childdocmain{}|\\
\end{tabular}
\end{center}
at the very top of the main \LaTeX{} file,
in particular \emph{before} the |\documentclass| statement!
The argument of |\childdocmain| should be left empty
(but it must be present).

%%%%%%%%%%%%%%%%%%%%%%%%%%%%%%%%%%%%%%%%
\DescribeMacro{\childdocof}
Furthermore, add the commands
\begin{center}
\begin{tabular}{l}
|\input{childdoc.def}|\\
|\childdocof{|\textit{main}|}|\\
\end{tabular}
\end{center}
at the top of every child file \textit{child}
which is included by |\include{|\textit{child}|}|
from within the main file
(or at least for those files to be compiled individually).
The argument \textit{main} must be the filename of the main file.

There are a couple of
considerations in setting up the main and child documents:

%%%%%%%%%%%%%%%%%%%%%%%%%%%%%%%%%%%%%%%%
\paragraph{Restrictions.}

Please note the following restrictions:
\begin{itemize}
\item
|\childdocmain| must be called with one argument \textit{main}
to ensure compatibility with earlier version of the package.
It must either be empty (|\childdocmain{}|)
or precisely match the filename of the main file in which it is specified.
See \secref{sec:detection} for further information.
\item
The filename \textit{main} must be specified without the |.tex| extension.
\item
The filename \textit{main} is case sensitive
(even in case-insensitive file systems)
due to internal string comparison.
\item
The argument \textit{main} should be fully expanded, it cannot be a macro.
\item
Subdirectories and special characters should be avoided in filenames.
\item
The command |\childdocmain{|\textit{main}|}| must be followed by a whitespace.
It should not be followed immediately by another command
or by a comment mark `|%|'.
This is because the \TeX{} parser reads the token immediately following
the argument of |\childdocmain| and puts it
at the beginning of every child section;
however, a white\-space is ignored.
\end{itemize}

%%%%%%%%%%%%%%%%%%%%%%%%%%%%%%%%%%%%%%%%
\paragraph{Content of Main File.}

It is advisable to place all content in the child files included by |\include|.
Any output contained in the main file will appear in all child documents
unless suppressed manually;
it cannot be suppressed automatically by the |\includeonly| directive
and thus should normally be avoided.
A method to include some content in the main file
by means of conditional processing is described in \secref{sec:conditional}.

%%%%%%%%%%%%%%%%%%%%%%%%%%%%%%%%%%%%%%%%
\paragraph{Page Numbering.}

When only a part of the document is compiled,
the appropriate numbering of pages
(as well as other status parameters)
is determined from the |.aux| files.
The latter contain information from previous passes.
However this information needs to propagate through
all intermediate child documents.
Therefore the page numbering in child documents may well
be inconsistent until the complete document is compiled at least once.

A useful (if unconventional) way to always ensure a consistent
page numbering is to restart the numbering in each child document
and denote the pages by `\textit{child}|.|\textit{page}'
where \textit{child} represents the chapter/section number of the child file.
This can be achieved by the command
|\numberwithin{page}{|\textit{child}|}|
of the \textsf{amsmath} package
where \textit{child} can be |chapter| or |section|
depending on the chosen structuring.
Alternatively, one can modify the macro |\thepage| appropriately
and reset the counter |page| at the start of each child file.

%%%%%%%%%%%%%%%%%%%%%%%%%%%%%%%%%%%%%%%%%%%%%%%%%%%%%%%%%%%%%%%%%%%%%%%%%%%%%%%%
\subsection{Conditional Processing}
\label{sec:conditional}

The package provides a mechanism to compile different versions
of a document. To customise the versions further some conditional processing
can come in handy to distinguish which version is being compiled.
The package provides two macros to describe the compilation context:

%%%%%%%%%%%%%%%%%%%%%%%%%%%%%%%%%%%%%%%%
\DescribeMacro{\ifchilddoc}
The conditional |\ifchilddoc| distinguishes between the compilation of
child documents and the main document:
%
\begin{center}
|\ifchilddoc |\textit{child-code}| |[|\||else |\textit{main-code}]| \||fi|
\end{center}

%%%%%%%%%%%%%%%%%%%%%%%%%%%%%%%%%%%%%%%%
\DescribeMacro{\childdocname}
\DescribeMacro{\childdocjob}
The macro |\childdocname| contains the filename (without extension)
of the main or child file being processed.
Note that |\childdocjob| will always contain the name of the main file.

%%%%%%%%%%%%%%%%%%%%%%%%%%%%%%%%%%%%%%%%
\paragraph{Title Page.}

Conditional processing can be used to include a title or banner page
in the main document when proper precautions are taken.
Importantly, the code in the main file should ensure that the page counter
(as well as other status parameters which are stored in the |.aux| files)
takes the same value after the conditional processing.
Otherwise the page numbers may take divergent values
depending on which part is compiled.

For example, a title page could be declared by:
%
\begin{center}
\begin{tabular}{l}
|\ifchilddoc\||else|\\
|\addtocounter{page}{-1}|\\
\textit{code for title page}\\
|\newpage|\\
|\||fi|
\end{tabular}
\end{center}
%
A banner page for the child documents can be generated by:
%
\begin{center}
\begin{tabular}{l}
|\ifchilddoc|\\
|\addtocounter{page}{-1}|\\
\textit{code for banner page}\\
|\newpage|\\
|\||fi|
\end{tabular}
\end{center}
%
Here one could write a message such as:
\begin{center}
|This is the part \childdocname{} of \childdocjob{}.|
\end{center}

%%%%%%%%%%%%%%%%%%%%%%%%%%%%%%%%%%%%%%%%%%%%%%%%%%%%%%%%%%%%%%%%%%%%%%%%%%%%%%%%
\subsection{Flags}
\label{sec:flags}

The package makes it easy to generate different versions
of the main or child documents.
To this end compilation flags can be defined
and assigned different default values.
They will be particularly useful in conjunction
with the forwarding mechanism described in \secref{sec:forward}.

For example, it may be useful to have a flag |\version|
which can be set to |draft| or |final|.
The document source will contain some conditional code
depending on the value of |\version|.
Suppose further, the flag should default to |final| for the main file
and to |draft| for child files
which is a natural assignment for editing the document.
This is achieved by placing the following code
in the preamble of the main document
(below the |\childdocmain| directive):
%
\begin{center}
\begin{tabular}{l}
|\ifchilddoc|\\
|\providecommand{\version}{draft}|\\
|\||else|\\
|\providecommand{\version}{final}|\\
|\||fi|
\end{tabular}
\end{center}
%
The definition by |\providecommand| makes sure
that previous definitions are not overwritten.
Further statements |\providecommand{\version}{...}|
can thus be added before the above code to override it.

For the main file, one might add a line
(between |\childdocmain| and the above block)
%
\begin{center}
|%\ifchilddoc\||else\providecommand{\version}{draft}\||fi|
\end{center}
%
which can be uncommented to produce a draft version.
Likewise one can add a line to the very top of a child file
(above the |\childdocof{|\textit{main}|}| directive)
%
\begin{center}
|%\providecommand{\version}{final}|
\end{center}
%
which can be uncommented to produce the final version of this child document.

%%%%%%%%%%%%%%%%%%%%%%%%%%%%%%%%%%%%%%%%%%%%%%%%%%%%%%%%%%%%%%%%%%%%%%%%%%%%%%%%
\subsection{Forwarding}
\label{sec:forward}

Different versions of the main or child documents
using compilation flags as described in \secref{sec:flags}
can be (permanently) stored in different files
for convenient compilation, viewing and distribution.
To this end, the package defines a command
to pass on compilation to a different file:

%%%%%%%%%%%%%%%%%%%%%%%%%%%%%%%%%%%%%%%%
\DescribeMacro{\childdocforward}
The command |\childdocforward| redirects processing to
another source file:
%
\begin{center}
\begin{tabular}{l}
|\input{childdoc.def}|\\
|\childdocforward[|\textit{main}|]{|\textit{dest}|}|\\
\end{tabular}
\end{center}
%
The argument \textit{dest} is the destination file
(without extension).
It should be the main file or one of the child files.
Note that further \textsf{childdoc} directives
such as |\childdocof| and |\childdocforward|
in the indicated file will be processed in this form.
The optional argument \textit{main}
passes on directly to the main file \textit{main}
while pretending to compile the child \textit{dest}.
This form behaves as if \textit{dest}
issues |\childdocof{|\textit{main}|}| right away,
and no further \textsf{childdoc} directives will be processed.

%%%%%%%%%%%%%%%%%%%%%%%%%%%%%%%%%%%%%%%%
\DescribeMacro{\...prefix}
In the alternative form |\childdocforwardprefix|,
%
\begin{center}
\begin{tabular}{l}
|\input{childdoc.def}|\\
|\childdocforwardprefix[|\textit{main}|]{|\textit{prefix}|}{|\textit{dest}|}|
\end{tabular}
\end{center}
%
the destination file is determined by a pattern
depending on the current file:
To make this work, the current file must be called
`{\textit{prefix}\hspace{0.2em}\textit{suffix}}'
with \textit{prefix} matching precisely the argument.
Processing is then passed on to the file
`{\textit{dest}\hspace{0.2em}\textit{suffix}}'.
Surely, the same effect is achieved by
directly specifying the
argument `{\textit{dest}\hspace{0.2em}\textit{suffix}}'
in the first form.
However, that requires to set up a different file
for each child. With the alternative form of the command
all these files can have exactly the same content
which simplifies setting them up and maintaining them.

For example, the following file |draft.tex|
with a compilation flag |\version| as described in \secref{sec:flags}
compiles the main document as a draft:
%
\begin{center}
\begin{tabular}{l}
|\def\version{draft}|\\
|\input{childdoc.def}|\\
|\childdocforward{|\textit{main}|}|
\end{tabular}
\end{center}
%
Likewise, the following files |final|\textit{nn}|.tex|
compile the final version of the child document
|child|\textit{nn}|.tex|:
%
\begin{center}
\begin{tabular}{l}
|\def\version{final}|\\
|\input{childdoc.def}|\\
|\childdocforwardprefix{final}{child}|
\end{tabular}
\end{center}
%

Note that when several versions of a main file and/or of each child file
are to be generated, it may be convenient to set up a |Makefile| or
shell script to automatise the process.

%%%%%%%%%%%%%%%%%%%%%%%%%%%%%%%%%%%%%%%%%%%%%%%%%%%%%%%%%%%%%%%%%%%%%%%%%%%%%%%%
\subsection{Command Line Processing}
\label{sec:commandline}

The effect of redirection files can also be achieved by invoking
the \LaTeX{} compiler with a more elaborate command line.
Most conveniently this should be done as part
of a shell script or a |Makefile|.

When using \textsf{childdoc} in the main file, the following
command lines effectively perform a redirection
(note that depending on the shell being used,
backslashes may have to be doubled: `|\|' $\to$ `|\\|'):
%
\begin{center}
|... -jobname "|\textit{target}|" |\\|"|[\textit{flags}]%
|\input{childdoc.def}\childdocforward[|\textit{main}|]{|\textit{dest}|}"|
\end{center}
%
Here \textit{target} is the name of the output file,
\textit{main} is the name of the main file
and \textit{dest} is the name of the main or child file to be processed
(all filenames without extensions).
The optional argument \textit{main} can be omitted
if \textit{main} matches \textit{dest}.
Optionally, compilation \textit{flags} can be defined via |\def| commands.
This command line makes the \TeX{} engine believe
it is compiling the file \textit{target}
whose content is specified as the latter parameter.
The provided code then forwards the processing to
\textit{main} or \textit{dest} as described in \secref{sec:forward}.

%%%%%%%%%%%%%%%%%%%%%%%%%%%%%%%%%%%%%%%%%%%%%%%%%%%%%%%%%%%%%%%%%%%%%%%%%%%%%%%%
\subsection{Include by Input}
\label{sec:input}

Including child documents by |\include| has some restrictions by design.
Most notably, the content of a child document always occupies
its own set of pages; pages cannot be shared between child documents.
Usually, this behaviour makes perfect sense
because each child document contain an essential part of the document.
However, in some situations it may be desirable to compose
a document from a collection of parts
without having mandatory page breaks between then.
For this case, the package
provides a mechanism to include parts
by |\input| which can also be processed individually.
However, by construction this mechanism
requires manual handling of the content to be output.

%%%%%%%%%%%%%%%%%%%%%%%%%%%%%%%%%%%%%%%%
\DescribeMacro{\ifchilddocmanual}
The main file should be prepared as usual, see \secref{sec:include}.
However, the document body must make a distinction
between processing of an individual part and of the main document, e.g.:
%
\begin{center}
\begin{tabular}{l}
|\ifchilddocmanual|\\
|\input{\childdocname}|\\
|\||else|\\
\textit{document body with }|\input{|\textit{part}|}|\\
|\||fi|
\end{tabular}
\end{center}
%
The conditional |\ifchilddocmanual| is true whenever
a part to be included by |\input| is being compiled,
and the name of the part is stored in |\childdocname|.

%%%%%%%%%%%%%%%%%%%%%%%%%%%%%%%%%%%%%%%%
\DescribeMacro{\childdocby}
Each part to be included by |\input| should start with:
%
\begin{center}
\begin{tabular}{l}
|\input{childdoc.def}|\\
|\childdocby{|\textit{main}|}|\\
\end{tabular}
\end{center}
%
The directive |\childdocby| is similar to |\childdocof|
described in \secref{sec:include},
but the subsequent selection of content must be done manually.
To that end, both |\ifchilddoc| and |\ifchilddocmanual|
will be true upon processing of a part,
and the name of the part is stored in |\childdocname|.
Note that |\jobname| will be set to the filename of the current part
so that each part receives an individual |.aux| file
that does not interfere with the |.aux| file(s) of the main document.
This behaviour can be altered by the alternative form
|\childdocby[*]{|\textit{main}|}| (with a non-empty optional argument)
which uses the |.aux| file of the main document
by setting |\jobname| to \textit{main}.

%%%%%%%%%%%%%%%%%%%%%%%%%%%%%%%%%%%%%%%%%%%%%%%%%%%%%%%%%%%%%%%%%%%%%%%%%%%%%%%%
\subsection{Driver Development}
\label{sec:driver}

The \textsf{childdoc} mechanism can also be use for the development
of definition files such as \LaTeX{} styles or classes.
This case differs from the above setup with multiple parts
included by |\include| in that no |\includeonly| should be invoked.
This can be achieved by starting the include file
(before |\ProvidesPackage|) with:
%
\begin{center}
\begin{tabular}{l}
|\input{childdoc.def}|\\
|\childdocforward{|\textit{main}|}|\\
\end{tabular}
\end{center}
%
or alternatively with:
%
\begin{center}
\begin{tabular}{l}
|\input{childdoc.def}|\\
|\childdocby{|\textit{main}|}|\\
\end{tabular}
\end{center}
%
Both forms have slightly different effects as described above.
The main file is prepared as usual, see \secref{sec:include}.

%%%%%%%%%%%%%%%%%%%%%%%%%%%%%%%%%%%%%%%%%%%%%%%%%%%%%%%%%%%%%%%%%%%%%%%%%%%%%%%%
\subsection{Legacy Detection}
\label{sec:detection}

The directive |\childdocmain| in the main file can detect
whether the complete document or merely a child is to be compiled
even without using the directive |\childdocof|.
This method is deprecated because it is less robust
and there is no compelling reason to use it;
it is merely provided for backward compatibility
and it may be removed in future versions.

If the detection mechanism is to be used,
it is mandatory to correctly specify
the filename of the main file as the argument of |\childdocmain|:
%
\begin{center}
\begin{tabular}{l}
|\input{childdoc.def}|\\
|\childdocmain{|\textit{main}|}|\\
\end{tabular}
\end{center}
%
If |\jobname| does not match the argument \textit{main} of |\childdocmain|,
it is assumed that |\jobname| points to the child file to be compiled.
When using |\childdocmain| with the main file specified as argument,
it suffices to start a child file
with just |\input{|\textit{main}|}|
without loading of the package and using |\childdocof|.
If instead all processing is done
with the appropriate \textsf{childdoc} directives,
the argument of \textit{main} of |\childdocmain| can be empty.

An alternative version of the command line processing described
in \secref{sec:commandline} using the detection mechanism reads:
%
\begin{center}
|... -jobname "|\textit{target}|" "|[\textit{flags}]%
[|\def\jobname{|\textit{dest}|}|]|\input{|\textit{main}|}"|
\end{center}

%%%%%%%%%%%%%%%%%%%%%%%%%%%%%%%%%%%%%%%%%%%%%%%%%%%%%%%%%%%%%%%%%%%%%%%%%%%%%%%%
\subsection{Manual Code}
\label{sec:manual}

In case one cannot be certain whether the definitions file |childdoc.def|
is installed on the target \TeX{} distribution
and one prefers not to ship it,
it is conceivable to paste a few relevant commands into the sources.

To that end, drop all statements |\input{childdoc.def}|
and perform the replacements as outlined below.
Instead of |\childdocmain{|\textit{main}|}| add the following code
to the top of the main file:
%
\begin{center}
\begin{tabular}{l}
|\||ifdefined\childdocname\endinput\||fi\newif\ifchilddoc|\\
|\edef\childdocname{\scantokens\expandafter{\jobname\noexpand}}|\\
|\def\childdocmain{|\textit{main}|}\||ifx\childdocmain\childdocname\||else|\\
|\childdoctrue\includeonly{\childdocname}\let\jobname\childdocmain\||fi|\\
\end{tabular}
\end{center}
%
Instead of |\childdocof{|\textit{main}|}| just include the main file
at the top of each child file:
%
\begin{center}
|\input{|\textit{main}|}|
\end{center}
%
A simple redirection |\childdocforward{|\textit{dest}|}| is achieved by:
%
\begin{center}
|\def\jobname{|\textit{dest}|}\input{\jobname}|
\end{center}
%
The redirection with prefix
|\childdocforwardprefix[|\textit{prefix}|]{|\textit{dest}|}|
is accomplished by:
%
\begin{center}
\begin{tabular}{l}
|{\edef\jobname{\scantokens\expandafter{\jobname\noexpand}}|\\
|\def\redirectjob |\textit{prefix}|#1~~~{\gdef\jobname{|\textit{dest}|#1}}|\\
|\expandafter\redirectjob\jobname~~~}\input{\jobname}|
\end{tabular}
\end{center}

In an alternative approach,
child documents can be compiled by a specific command line
without additional code or specific definitions:
%
\begin{center}
|... -jobname "|\textit{target}|" "|[\textit{flags}]%
|\includeonly{|\textit{dest}|}\input{|\textit{main}|}"|
\end{center}
%

%%%%%%%%%%%%%%%%%%%%%%%%%%%%%%%%%%%%%%%%%%%%%%%%%%%%%%%%%%%%%%%%%%%%%%%%%%%%%%%%
%%%%%%%%%%%%%%%%%%%%%%%%%%%%%%%%%%%%%%%%%%%%%%%%%%%%%%%%%%%%%%%%%%%%%%%%%%%%%%%%
\section{Information}

%%%%%%%%%%%%%%%%%%%%%%%%%%%%%%%%%%%%%%%%%%%%%%%%%%%%%%%%%%%%%%%%%%%%%%%%%%%%%%%%
\subsection{Copyright}

Copyright \copyright{} 2017--2018 Niklas Beisert

This work may be distributed and/or modified under the
conditions of the \LaTeX{} Project Public License, either version 1.3
of this license or (at your option) any later version.
The latest version of this license is in
  \url{http://www.latex-project.org/lppl.txt}
and version 1.3 or later is part of all distributions of \LaTeX{}
version 2005/12/01 or later.

This work has the LPPL maintenance status `maintained'.

The Current Maintainer of this work is Niklas Beisert.

This work consists of the files |README.txt|, |childdoc.ins| and |childdoc.dtx|
as well as the derived files |childdoc.def|, |cdocsamp.tex|
with |cdocsch1.tex|, |cdocsch2.tex|, |cdocspt3.tex|, |cdocspt4.tex|,
|cdocsdrf.tex|, |cdocsfn1.tex|, |cdocsfn2.tex|
as well as |childdoc.pdf|.

%%%%%%%%%%%%%%%%%%%%%%%%%%%%%%%%%%%%%%%%%%%%%%%%%%%%%%%%%%%%%%%%%%%%%%%%%%%%%%%%
\subsection{Files and Installation}

The package consists of the files:
%
\begin{center}
\begin{tabular}{ll}
    |README.txt|   & readme file \\
    |childdoc.ins| & installation file \\
    |childdoc.dtx| & source file \\
    |childdoc.def| & definition file \\
    |cdocsamp.tex| & sample main file \\
    |cdocsch1.tex| & sample include file \\
    |cdocsch2.tex| & sample include file \\
    |cdocspt3.tex| & sample part file \\
    |cdocspt4.tex| & sample part file \\
    |cdocsdrf.tex| & sample redirection file \\
    |cdocsfn1.tex| & sample redirection file \\
    |cdocsfn2.tex| & sample redirection file \\
    |childdoc.pdf| & manual
\end{tabular}
\end{center}
%
The distribution consists of the files
|README.txt|, |childdoc.ins| and |childdoc.dtx|.
%
\begin{itemize}
\item
Run (pdf)\LaTeX{} on |childdoc.dtx|
to compile the manual |childdoc.pdf| (this file).
\item
Run \LaTeX{} on |childdoc.ins| to create the definitions file |childdoc.def|
and the sample |cdocsamp.tex| with include files
|cdocsch1.tex|, |cdocsch2.tex|, |cdocspt3.tex|, |cdocspt4.tex|,
|cdocsdrf.tex|, |cdocsfn1.tex|, |cdocsfn2.tex|.
Then copy the file |childdoc.def| to an appropriate directory of your \LaTeX{}
distribution, e.g.\ \textit{texmf-root}|/tex/latex/childdoc|.
\end{itemize}

%%%%%%%%%%%%%%%%%%%%%%%%%%%%%%%%%%%%%%%%%%%%%%%%%%%%%%%%%%%%%%%%%%%%%%%%%%%%%%%%
\subsection{Related CTAN Packages}

There are several other packages which offer a similar functionality:
%
\begin{itemize}
\item
The packages
\href{http://ctan.org/pkg/docmute}{\textsf{docmute}},
\href{http://ctan.org/pkg/includex}{\textsf{includex}} and
\href{http://ctan.org/pkg/standalone}{\textsf{standalone}}
provide commands to include only the document body of
a child file thus allowing both files to be compiled individually.
\item
The packages \href{http://ctan.org/pkg/subdocs}{\textsf{subdocs}}
and \href{http://ctan.org/pkg/subfiles}{\textsf{subfiles}}
provide structures in which the main and child documents can be
encapsulated and allowing them to be compiled individually.
The inclusion mechanism is different from the conventional |\include|.
\item
The package \href{http://ctan.org/pkg/combine}{\textsf{combine}}
is an elaborate solution to combine several documents into one.
\end{itemize}
%
See also the CTAN topic \href{http://ctan.org/topic/subdocs}{\textsf{subdocs}}
for further related packages.
The present package differs from the above solutions in that
a document structure constructed with the conventional |\include| mechanism
just needs two extra commands at the top of every file
such that all constituent files can be compiled individually.

%%%%%%%%%%%%%%%%%%%%%%%%%%%%%%%%%%%%%%%%%%%%%%%%%%%%%%%%%%%%%%%%%%%%%%%%%%%%%%%%
%\subsection{Feature Suggestions}
%
%The following is a list of features which may be useful for future
%versions of this package:
%%
%\begin{itemize}
%\item
%\ldots
%\end{itemize}

%%%%%%%%%%%%%%%%%%%%%%%%%%%%%%%%%%%%%%%%%%%%%%%%%%%%%%%%%%%%%%%%%%%%%%%%%%%%%%%%
\subsection{Revision History}

%%%%%%%%%%%%%%%%%%%%%%%%%%%%%%%%%%%%%%%%
\paragraph{v2.0:} 2018/12/30

\begin{itemize}
\item
immediate forward processing
\item
added |\childdocby| mechanism
\item
manual restructured
\end{itemize}

%%%%%%%%%%%%%%%%%%%%%%%%%%%%%%%%%%%%%%%%
\paragraph{v1.6:} 2018/01/17

\begin{itemize}
\item
application for development of include files
\item
corrections to manual
\end{itemize}

%%%%%%%%%%%%%%%%%%%%%%%%%%%%%%%%%%%%%%%%
\paragraph{v1.5:} 2017/05/21

\begin{itemize}
\item
more complete structuring introduced
\item
|\childdocof| introduced
\item
|\childdoc| renamed to |\childdocmain|
\item
|\childredirect| renamed to |\childdocforward| and |\childdocforwardprefix|
and functionality expanded
\end{itemize}

%%%%%%%%%%%%%%%%%%%%%%%%%%%%%%%%%%%%%%%%
\paragraph{v1.0:} 2017/04/27

\begin{itemize}
\item
manual and install package
\item
first version published on CTAN
\end{itemize}

%%%%%%%%%%%%%%%%%%%%%%%%%%%%%%%%%%%%%%%%
\paragraph{v0.6:} 2017/04/26

\begin{itemize}
\item
redirection mechanism added
\end{itemize}

%%%%%%%%%%%%%%%%%%%%%%%%%%%%%%%%%%%%%%%%
\paragraph{v0.5:} 2017/04/26

\begin{itemize}
\item
functionality in definition file
\end{itemize}


%%%%%%%%%%%%%%%%%%%%%%%%%%%%%%%%%%%%%%%%%%%%%%%%%%%%%%%%%%%%%%%%%%%%%%%%%%%%%%%%
%%%%%%%%%%%%%%%%%%%%%%%%%%%%%%%%%%%%%%%%%%%%%%%%%%%%%%%%%%%%%%%%%%%%%%%%%%%%%%%%
%%%%%%%%%%%%%%%%%%%%%%%%%%%%%%%%%%%%%%%%%%%%%%%%%%%%%%%%%%%%%%%%%%%%%%%%%%%%%%%%
\appendix

\settowidth\MacroIndent{\rmfamily\scriptsize 000\ }

 \DocInput{childdoc.dtx}

\end{document}
%</driver>
% \fi
%
% %%%%%%%%%%%%%%%%%%%%%%%%%%%%%%%%%%%%%%%%%%%%%%%%%%%%%%%%%%%%%%%%%%%%%%%%%%%%%%
% %%%%%%%%%%%%%%%%%%%%%%%%%%%%%%%%%%%%%%%%%%%%%%%%%%%%%%%%%%%%%%%%%%%%%%%%%%%%%%
% \section{Sample}
%\iffalse
%<*samplemain>
%\fi
%
% The following presents a sample document
% with two chapters, two parts, a title page,
% a compile flag as well as three forwarding files to set the flag.
% It consists of eight |.tex| files:
% \begin{center}
% \begin{tabular}{ll}
% |cdocsamp.tex|&main file\\
% |cdocsch1.tex|&include file for chapter 1\\
% |cdocsch2.tex|&include file for chapter 2\\
% |cdocspt3.tex|&include file for part 3\\
% |cdocspt4.tex|&include file for part 4\\
% |cdocsdrf.tex|&forwarding file for main file in draft mode\\
% |cdocsfi1.tex|&forwarding file for final version of chapter 1\\
% |cdocsfi2.tex|&forwarding file for final version of chapter 2\\
% \end{tabular}
% \end{center}
% Each of the eight files can be compiled directly by the \LaTeX{} compiler.
%
% %%%%%%%%%%%%%%%%%%%%%%%%%%%%%%%%%%%%%%
% \paragraph{Main File.}
%
% The main file is called |cdocsamp.tex|.
%
% Load the \textsf{childdoc} definitions and
% declare the filename for the main document:
%    \begin{macrocode}
\input{childdoc.def}
\childdocmain{}
%    \end{macrocode}

% Optional override for |\version| flag:
%    \begin{macrocode}
%%\ifchilddoc\else\providecommand{\version}{draft}\fi
%    \end{macrocode}

% Define the default values for the |\version| flag
% (|final| for the main file and |draft| for childs):
%    \begin{macrocode}
\ifchilddoc
\providecommand{\version}{draft}
\else
\providecommand{\version}{final}
\fi
%    \end{macrocode}

% Load the standard document class:
%    \begin{macrocode}
\documentclass[12pt]{article}
%    \end{macrocode}

% Start the document body:
%    \begin{macrocode}
\begin{document}
%    \end{macrocode}

% Declare a title page.
% Print title, part of document being processed and version flag:
%    \begin{macrocode}
\addtocounter{page}{-1}
\begin{center}
{\LARGE\bfseries{}childdoc example\par}
\vspace{1cm}
\ifchilddoc
\ifchilddocmanual part\else chapter\fi:
`\childdocname' of `\childdocjob'\par
\else
main document: `\childdocjob'\par
\fi
version: \version\par
\end{center}
\newpage
%    \end{macrocode}

% Manually include selected file,
% otherwise process as usual:
%    \begin{macrocode}
\ifchilddocmanual
\section*{part `\childdocname'}
\input{\childdocname}
\else
%    \end{macrocode}

% Include the two chapters:
%    \begin{macrocode}
\include{cdocsch1}
\include{cdocsch2}
%    \end{macrocode}

% Include the two parts unless only chapters should be displayed:
%    \begin{macrocode}
\ifchilddoc\else
\section{part three}
\input{cdocspt3}
\section{part four}
\input{cdocspt4}
\fi
%    \end{macrocode}

% Process as usual until here:
%    \begin{macrocode}
\fi
%    \end{macrocode}

% End of document body:
%    \begin{macrocode}
\end{document}
%    \end{macrocode}
%\iffalse
%</samplemain>
%\fi
%
% %%%%%%%%%%%%%%%%%%%%%%%%%%%%%%%%%%%%%%
% \paragraph{Chapter Include Files.}
%
% The include files are called |cdocsch1.tex| and |cdocsch2.tex|.
%
%\iffalse
%<*samplechap1|samplechap2>
%\fi

% Optional override for |\version| flag:
%    \begin{macrocode}
%%\providecommand{\version}{final}
%    \end{macrocode}

% Include the main document:
%    \begin{macrocode}
\input{childdoc.def}
\childdocof{cdocsamp}
%    \end{macrocode}

%\iffalse
%</samplechap1|samplechap2>
%\fi
%
%\iffalse
%<*samplechap1>
%\fi
% Some text for chapter 1:
%    \begin{macrocode}
\section{one}
some text in chapter one
%    \end{macrocode}

%\iffalse
%</samplechap1>
%\fi
% Some text for chapter 2:
%\iffalse
%<*samplechap2>
%\fi
%    \begin{macrocode}
\section{two}
more text in chapter two
%    \end{macrocode}

%\iffalse
%</samplechap2>
%\fi
%
% %%%%%%%%%%%%%%%%%%%%%%%%%%%%%%%%%%%%%%
% \paragraph{Part Include Files.}
%
% The include files are called |cdocspt3.tex| and |cdocspt4.tex|.
%
%\iffalse
%<*samplepart3|samplepart4>
%\fi

% Optional override for |\version| flag:
%    \begin{macrocode}
%%\providecommand{\version}{final}
%    \end{macrocode}

% Include the main document:
%    \begin{macrocode}
\input{childdoc.def}
\childdocby{cdocsamp}
%    \end{macrocode}

%\iffalse
%</samplepart3|samplepart4>
%\fi
%
%\iffalse
%<*samplepart3>
%\fi
% Some text for part 3:
%    \begin{macrocode}
some text in part three
%    \end{macrocode}

%\iffalse
%</samplepart3>
%\fi
% Some text for part 4:
%\iffalse
%<*samplepart4>
%\fi
%    \begin{macrocode}
more text in part four
%    \end{macrocode}

%\iffalse
%</samplepart4>
%\fi
%
% %%%%%%%%%%%%%%%%%%%%%%%%%%%%%%%%%%%%%%
% \paragraph{Forwarding for a Complete Draft.}
%
% The following forwarding file |cdocsdrf.tex|
% compiles the main document in draft mode:
%\iffalse
%<*sampledraft>
%\fi
%    \begin{macrocode}
\def\version{draft}
\input{childdoc.def}
\childdocforward{cdocsamp}
%    \end{macrocode}

%\iffalse
%</sampledraft>
%\fi
%
% %%%%%%%%%%%%%%%%%%%%%%%%%%%%%%%%%%%%%%
% \paragraph{Forwarding for Final Version of the Chapters.}
%
% The following forwarding files |cdocsfn1.tex| and |cdocsfn2.tex|
% (with identical content)
% compile the final versions of the child documents
% |cdocsch1.tex| and |cdocsch2.tex|, respectively:
%\iffalse
%<*samplefinal>
%\fi
%    \begin{macrocode}
\def\version{final}
\input{childdoc.def}
\childdocforwardprefix[cdocsamp]{cdocsfn}{cdocsch}
%    \end{macrocode}

%\iffalse
%</samplefinal>
%\fi
%
% %%%%%%%%%%%%%%%%%%%%%%%%%%%%%%%%%%%%%%
% \paragraph{Command Line Processing.}
%
% The following three command lines generate the output files
% |cdocscld|, |cdocscl1| and |cdocscl2|
% which should be identical to
% |cdocsdrf|, |cdocsch1| and |cdocsfn2|, respectively:
% \begin{center}
% \begin{tabular}{l}
% |latex -jobname cdocscld \|\\
% |  "\def\version{draft}\input{childdoc.def}\childdocforward{cdocsamp}"|\\
% |latex -jobname cdocscl1 \|\\
% |  "\input{childdoc.def}\childdocforward[cdocsamp]{cdocsch1}"|\\
% |latex -jobname cdocscl2 \|\\
% |  "\def\version{final}\input{childdoc.def}\childdocforward{cdocsch2}"|
% \end{tabular}
% \end{center}
% Note that the trailing backslash on each first line
% merely continues the input to the second line
% (for convenient cut ant paste).
% Furthermore, the command |latex| can be replaced by any
% of its alternative versions such as |pdflatex|.
%
% %%%%%%%%%%%%%%%%%%%%%%%%%%%%%%%%%%%%%%%%%%%%%%%%%%%%%%%%%%%%%%%%%%%%%%%%%%%%%%
% %%%%%%%%%%%%%%%%%%%%%%%%%%%%%%%%%%%%%%%%%%%%%%%%%%%%%%%%%%%%%%%%%%%%%%%%%%%%%%
% \section{Implementation}
%\iffalse
%<*package>
%\fi
%
% This section describes the definitions file |childdoc.def|.

% The definitions cannot be loaded using |\usepackage| or |\RequirePackage|
% which has a mechanism to prevent loading a style file more than once.
% When loading the definitions by means of |\input|
% multiple instances have to be prevented manually:
%\iffalse
%This code needs to be before the `\ProvidesFile' directive
%which is defined at the beginning of this file.
%Therefore it is also placed there and commented out here.
%</package>
%<*discard>
%\fi
%    \begin{macrocode}
\ifdefined\childdocmain\endinput\fi
%    \end{macrocode}
%\iffalse
%</discard>
%<*package>
%\fi
%
% \macro{\ifchilddoc}
% \macro{\ifchilddocmanual}
% The conditional |\ifchilddoc| tells whether a
% child (true) or main (false) document is being compiled.
% The conditional |\ifchilddocmanual| tells whether
% the |\includeonly| mechanism is used (false) or
% the selection of child files must be performed manually (true).
% The definitions initialise to false:
%    \begin{macrocode}
\newif\ifchilddoc
\newif\ifchilddocmanual
%    \end{macrocode}

% \macro{\childdocname}
% \macro{\childdocjob}
% The macro |\childdocname| stores the name of the main document
% to be compiled. The macro |\childdocjob| stores the name of
% the document on which the \LaTeX{} compiler was originally invoked.
% The content of |\jobname| cannot be compared
% to filenames specified in the source due to different catcodes.
% The following code rescans |\jobname|, stores the result
% in |\childdocname| and saves a copy in |\childdocjob|:
%    \begin{macrocode}
\edef\childdocname{\scantokens\expandafter{\jobname\noexpand}}
\let\childdocjob\childdocname
%    \end{macrocode}

% \macro{\childdocdisable}
% The macro |\childdocdisable| prevents the main file
% from being processed more than once.
% At this stage, the main document command |\childdocmain|
% is assumed to be called once again where it should do nothing.
% Any subsequent call to it should prevent
% a secondary processing of the main document
% It overwrites the forwarding commands
% |\childdocof| and |\childdocforward|
% with empty macros to prevent further inclusions of the main document:
%    \begin{macrocode}
\newcommand{\childdocdisable}
{
  \renewcommand{\childdocmain}[1]{\renewcommand{\childdocmain}[1]{\endinput}}
  \renewcommand{\childdocof}[1]{}
  \renewcommand{\childdocby}[2][]{}
  \renewcommand{\childdocforward}[2][]{}
  \renewcommand{\childdocdisable}{}
}
%    \end{macrocode}

% \macro{\childdocmain}
% The macro |\childdocmain| is to be called at the top of the main file
% with nothing or the main filename (without extension) as argument.
% First, it breaks loops.
% If the argument is not empty and does not match |\childdocname|
% (which is set by the first inclusion of |childdoc.def|),
% |\ifchilddoc| is set to true, |\includeonly| is applied to the child file
% and |\jobname| is set to the main file
% (for proper handling of |.aux| files):
%    \begin{macrocode}
\newcommand{\childdocmain}[1]
{
  \childdocdisable\childdocmain{}
  \if?#1?\else
    \begingroup
      \def\childdoctmp{#1}
      \ifx\childdoctmp\childdocname
        \def\childdoctmp{}
      \else
        \def\childdoctmp
        {
          \childdoctrue
          \includeonly{\childdocname}
          \def\childdocjob{#1}
          \def\jobname{#1}
        }
      \fi
      \expandafter
    \endgroup
    \childdoctmp
  \fi
}
%    \end{macrocode}

% \macro{\childdocof}
% The command |\childdocof| redirects
% compilation to the main file |#1|.
%    \begin{macrocode}
\newcommand{\childdocof}[1]
{
  \childdocdisable
  \childdoctrue
  \includeonly{\childdocname}
  \def\jobname{#1}
  \def\childdocjob{#1}
  \input{#1}
}
%    \end{macrocode}

% \macro{\childdocby}
% The command |\childdocby| ....
%    \begin{macrocode}
\newcommand{\childdocby}[2][]
{
  \childdocdisable
  \childdoctrue
  \childdocmanualtrue
  \if?#1?\else
    \def\jobname{#2}
  \fi
  \def\childdocjob{#2}
  \input{#2}
  \endinput
}
%    \end{macrocode}

% \macro{\childdocforward}
% The command |\childdocforward| redirects
% compilation to the main file or
% (if the optional argument is given) a child file.
% Parameters are set as if the main file
% or a child file starting with |\childdocof| was compiled.
% Then compilation is handed over to the main file:
%    \begin{macrocode}
\newcommand{\childdocforward}[2][]
{
  \begingroup
    \if?#1?
      \def\childdoctmp
      {
        \def\childdocname{#2}
        \def\childdocjob{#2}
        \def\jobname{#2}
        \input{#2}
        \endinput
      }
    \else
      \def\childdoctmp
      {
        \childdocdisable
        \def\childdocname{#2}
        \childdoctrue
        \includeonly{#2}
        \def\childdocjob{#1}
        \def\jobname{#1}
        \input{#1}
        \endinput
      }
    \fi
    \expandafter
  \endgroup
  \childdoctmp
}
%    \end{macrocode}

% \macro{\childdocforwardprefix}
% The command |\childdocforwardprefix| redirects
% compilation to the main or a child file by means of a pattern.
% The prefix |#1| in the current filename is replaced by |#2|
% and the suffix of the current filename is kept
% (it is assumed that the filename does not contain the substring `|~~~|'
% which is used as a delimiter).
% Compilation is handed over to the new file by |\childdocforward|:
%    \begin{macrocode}
\newcommand{\childdocforwardprefix}[3][]
{
  \begingroup
    \def\childdocextract #2##1~~~{\def\childdoctmp{\childdocforward[#1]{#3##1}}}
    \expandafter\childdocextract\childdocname~~~
    \expandafter
  \endgroup
  \childdoctmp
}
%    \end{macrocode}

% \macro{\childdoc}
% The deprecated macro |\childdoc| is a legacy version of |\childdocmain|:
%    \begin{macrocode}
\newcommand{\childdoc}{\childdocmain}
%    \end{macrocode}

% \macro{\childdocredirect}
% The deprecated macro |\childdocredirect| is a legacy version
% of |\childdocforward| and |\childdocforwardprefix|:
%    \begin{macrocode}
\newcommand{\childdocredirect}[2][]
{
  \begingroup
    \if?#1?
      \def\childdoctmp{\childdocforward{#2}}
    \else
      \def\childdoctmp{\childdocforwardprefix{#1}{#2}}
    \fi
    \expandafter
  \endgroup
  \childdoctmp
}
%    \end{macrocode}

%\iffalse
%</package>
%\fi
%
\endinput

\childdocmain{}
%    \end{macrocode}

% Optional override for |\version| flag:
%    \begin{macrocode}
%%\ifchilddoc\else\providecommand{\version}{draft}\fi
%    \end{macrocode}

% Define the default values for the |\version| flag
% (|final| for the main file and |draft| for childs):
%    \begin{macrocode}
\ifchilddoc
\providecommand{\version}{draft}
\else
\providecommand{\version}{final}
\fi
%    \end{macrocode}

% Load the standard document class:
%    \begin{macrocode}
\documentclass[12pt]{article}
%    \end{macrocode}

% Start the document body:
%    \begin{macrocode}
\begin{document}
%    \end{macrocode}

% Declare a title page.
% Print title, part of document being processed and version flag:
%    \begin{macrocode}
\addtocounter{page}{-1}
\begin{center}
{\LARGE\bfseries{}childdoc example\par}
\vspace{1cm}
\ifchilddoc
\ifchilddocmanual part\else chapter\fi:
`\childdocname' of `\childdocjob'\par
\else
main document: `\childdocjob'\par
\fi
version: \version\par
\end{center}
\newpage
%    \end{macrocode}

% Manually include selected file,
% otherwise process as usual:
%    \begin{macrocode}
\ifchilddocmanual
\section*{part `\childdocname'}
\input{\childdocname}
\else
%    \end{macrocode}

% Include the two chapters:
%    \begin{macrocode}
\include{cdocsch1}
\include{cdocsch2}
%    \end{macrocode}

% Include the two parts unless only chapters should be displayed:
%    \begin{macrocode}
\ifchilddoc\else
\section{part three}
\input{cdocspt3}
\section{part four}
\input{cdocspt4}
\fi
%    \end{macrocode}

% Process as usual until here:
%    \begin{macrocode}
\fi
%    \end{macrocode}

% End of document body:
%    \begin{macrocode}
\end{document}
%    \end{macrocode}
%\iffalse
%</samplemain>
%\fi
%
% %%%%%%%%%%%%%%%%%%%%%%%%%%%%%%%%%%%%%%
% \paragraph{Chapter Include Files.}
%
% The include files are called |cdocsch1.tex| and |cdocsch2.tex|.
%
%\iffalse
%<*samplechap1|samplechap2>
%\fi

% Optional override for |\version| flag:
%    \begin{macrocode}
%%\providecommand{\version}{final}
%    \end{macrocode}

% Include the main document:
%    \begin{macrocode}
% \iffalse
%
% childdoc.dtx Copyright (C) 2017-2018 Niklas Beisert
%
% This work may be distributed and/or modified under the
% conditions of the LaTeX Project Public License, either version 1.3
% of this license or (at your option) any later version.
% The latest version of this license is in
%   http://www.latex-project.org/lppl.txt
% and version 1.3 or later is part of all distributions of LaTeX
% version 2005/12/01 or later.
%
% This work has the LPPL maintenance status `maintained'.
%
% The Current Maintainer of this work is Niklas Beisert.
%
% This work consists of the files childdoc.dtx and childdoc.ins
% and the derived files childdoc.def and cdocsamp.tex with
% cdocsch1.tex, cdocsch2.tex, cdocsdrf.tex, cdocsfn1.tex, cdocsfn2.tex.
%
%<package>\ifdefined\childdocmain\endinput\fi
%<package>\ProvidesFile{childdoc.def}[2018/12/30 v2.0 child document driver]
%<samplemain>\ProvidesFile{cdocsamp.tex}[2018/12/30 v2.0 sample for childdoc]
%<*driver>
%\ProvidesFile{childdoc.drv}[2018/12/30 v2.0 childdoc reference manual file]
\PassOptionsToClass{10pt,a4paper}{article}
\documentclass{ltxdoc}

\usepackage[margin=35mm]{geometry}
\usepackage{hyperref}
\usepackage{hyperxmp}
\usepackage[usenames]{color}

\hypersetup{colorlinks=true}
\hypersetup{pdfstartview=FitH}
\hypersetup{pdfpagemode=UseNone}
\hypersetup{pdfsource={}}
\hypersetup{pdflang={en-UK}}
\hypersetup{pdfcopyright={Copyright 2017-2018 Niklas Beisert.
  This work may be distributed and/or modified under the
  conditions of the LaTeX Project Public License, either version 1.3
  of this license or (at your option) any later version.}}
\hypersetup{pdflicenseurl={http://www.latex-project.org/lppl.txt}}
\hypersetup{pdfcontactaddress={ETH Zurich, ITP, HIT K,
  Wolfgang-Pauli-Strasse 27}}
\hypersetup{pdfcontactpostcode={8093}}
\hypersetup{pdfcontactcity={Zurich}}
\hypersetup{pdfcontactcountry={Switzerland}}
\hypersetup{pdfcontactemail={nbeisert@itp.phys.ethz.ch}}
\hypersetup{pdfcontacturl={http://people.phys.ethz.ch/\xmptilde nbeisert/}}

\newcommand{\secref}[1]{\hyperref[#1]{section \ref*{#1}}}

\parskip1ex
\parindent0pt
\let\olditemize\itemize
\def\itemize{\olditemize\parskip0pt}

\begin{document}

\title{The \textsf{childdoc} Package}
\hypersetup{pdftitle={The childdoc Package}}
\author{Niklas Beisert\\[2ex]
  Institut f\"ur Theoretische Physik\\
  Eidgen\"ossische Technische Hochschule Z\"urich\\
  Wolfgang-Pauli-Strasse 27, 8093 Z\"urich, Switzerland\\[1ex]
  \href{mailto:nbeisert@itp.phys.ethz.ch}
  {\texttt{nbeisert@itp.phys.ethz.ch}}}
\hypersetup{pdfauthor={Niklas Beisert}}
\hypersetup{pdfsubject={Manual for the LaTeX2e Package childdoc}}
\date{30 December 2018, \textsf{v2.0}}
\maketitle

\begin{abstract}\noindent
\textsf{childdoc} is a \LaTeXe{} package
that enables the direct compilation
of document sections included by |\include|
to individual files.
\end{abstract}

\begingroup
\parskip0ex
\tableofcontents
\endgroup

%%%%%%%%%%%%%%%%%%%%%%%%%%%%%%%%%%%%%%%%%%%%%%%%%%%%%%%%%%%%%%%%%%%%%%%%%%%%%%%%
%%%%%%%%%%%%%%%%%%%%%%%%%%%%%%%%%%%%%%%%%%%%%%%%%%%%%%%%%%%%%%%%%%%%%%%%%%%%%%%%
\section{Introduction}

\LaTeX{} provides a mechanism to structure a large document (such as a book)
into a main file and several child files (containing the chapters)
using the |\include| command.
This mechanism is beneficial for documents
which span hundreds of pages in order to
make the source file(s) more manageable.
Moreover, compilation can be restricted to
selected child files by means of the |\includeonly| command.
The latter feature can be used to reduce the compilation time while editing
(this was significantly more useful in the earlier days of \LaTeX{})
or to generate a smaller document which is easier to navigate.
Another application of |\includeonly| is to generate
documents consisting of selected parts of the complete document.

However, there are a few drawbacks of the plain |\include| mechanism:
\begin{itemize}
\item
The child files cannot be compiled on their own,
they can only be compiled via the main file.
A naive editing environment
(such as a text editor with an option
to have the current file processed by \LaTeX)
may require one to switch to the main file before compiling;
attempting to compile the child file produces errors.
\item
The main file must be modified (each time)
to adjust the |\includeonly| command
to the present needs. This easily leaves the main file in a messy state.
\item
The generated document will always carry the filename
of the main document. This is inconvenient if
several child files are to be compiled and
to be kept for distribution.
\end{itemize}

The present package provides a simple interface
to make child files individually compilable by \LaTeX{}.
Compiling a child file then has the same effect as compiling
the main file with an |\includeonly| command
to select the appropriate child.
Moreover the generated document will carry the name of the child
rather than the main file.
This resolves all three above issues.

This feature is meant to make the editing of books,
thesis documents and lecture notes somewhat more convenient.
However, the package can also be used efficiently for
composing a series of documents (such as exercise sheets)
which are typically distributed individually.
It then assists the author in generating the individual documents
(potentially in different versions)
as well as a document containing the collected series.
Another application is in developing style files
or other kinds of included material
where compilation of the style file could redirect
to a sample or test file.

%%%%%%%%%%%%%%%%%%%%%%%%%%%%%%%%%%%%%%%%%%%%%%%%%%%%%%%%%%%%%%%%%%%%%%%%%%%%%%%%
%%%%%%%%%%%%%%%%%%%%%%%%%%%%%%%%%%%%%%%%%%%%%%%%%%%%%%%%%%%%%%%%%%%%%%%%%%%%%%%%
\section{Usage}

First of all, the package \textsf{childdoc} is \emph{not} a standard
\LaTeXe{} |.sty| style file! Therefore it needs to be invoked in
a non-standard way.

%%%%%%%%%%%%%%%%%%%%%%%%%%%%%%%%%%%%%%%%%%%%%%%%%%%%%%%%%%%%%%%%%%%%%%%%%%%%%%%%
\subsection{Included Files}
\label{sec:include}

%%%%%%%%%%%%%%%%%%%%%%%%%%%%%%%%%%%%%%%%
\DescribeMacro{\childdocmain}
To use the package, add the commands
\begin{center}
\begin{tabular}{l}
|\input{childdoc.def}|\\
|\childdocmain{}|\\
\end{tabular}
\end{center}
at the very top of the main \LaTeX{} file,
in particular \emph{before} the |\documentclass| statement!
The argument of |\childdocmain| should be left empty
(but it must be present).

%%%%%%%%%%%%%%%%%%%%%%%%%%%%%%%%%%%%%%%%
\DescribeMacro{\childdocof}
Furthermore, add the commands
\begin{center}
\begin{tabular}{l}
|\input{childdoc.def}|\\
|\childdocof{|\textit{main}|}|\\
\end{tabular}
\end{center}
at the top of every child file \textit{child}
which is included by |\include{|\textit{child}|}|
from within the main file
(or at least for those files to be compiled individually).
The argument \textit{main} must be the filename of the main file.

There are a couple of
considerations in setting up the main and child documents:

%%%%%%%%%%%%%%%%%%%%%%%%%%%%%%%%%%%%%%%%
\paragraph{Restrictions.}

Please note the following restrictions:
\begin{itemize}
\item
|\childdocmain| must be called with one argument \textit{main}
to ensure compatibility with earlier version of the package.
It must either be empty (|\childdocmain{}|)
or precisely match the filename of the main file in which it is specified.
See \secref{sec:detection} for further information.
\item
The filename \textit{main} must be specified without the |.tex| extension.
\item
The filename \textit{main} is case sensitive
(even in case-insensitive file systems)
due to internal string comparison.
\item
The argument \textit{main} should be fully expanded, it cannot be a macro.
\item
Subdirectories and special characters should be avoided in filenames.
\item
The command |\childdocmain{|\textit{main}|}| must be followed by a whitespace.
It should not be followed immediately by another command
or by a comment mark `|%|'.
This is because the \TeX{} parser reads the token immediately following
the argument of |\childdocmain| and puts it
at the beginning of every child section;
however, a white\-space is ignored.
\end{itemize}

%%%%%%%%%%%%%%%%%%%%%%%%%%%%%%%%%%%%%%%%
\paragraph{Content of Main File.}

It is advisable to place all content in the child files included by |\include|.
Any output contained in the main file will appear in all child documents
unless suppressed manually;
it cannot be suppressed automatically by the |\includeonly| directive
and thus should normally be avoided.
A method to include some content in the main file
by means of conditional processing is described in \secref{sec:conditional}.

%%%%%%%%%%%%%%%%%%%%%%%%%%%%%%%%%%%%%%%%
\paragraph{Page Numbering.}

When only a part of the document is compiled,
the appropriate numbering of pages
(as well as other status parameters)
is determined from the |.aux| files.
The latter contain information from previous passes.
However this information needs to propagate through
all intermediate child documents.
Therefore the page numbering in child documents may well
be inconsistent until the complete document is compiled at least once.

A useful (if unconventional) way to always ensure a consistent
page numbering is to restart the numbering in each child document
and denote the pages by `\textit{child}|.|\textit{page}'
where \textit{child} represents the chapter/section number of the child file.
This can be achieved by the command
|\numberwithin{page}{|\textit{child}|}|
of the \textsf{amsmath} package
where \textit{child} can be |chapter| or |section|
depending on the chosen structuring.
Alternatively, one can modify the macro |\thepage| appropriately
and reset the counter |page| at the start of each child file.

%%%%%%%%%%%%%%%%%%%%%%%%%%%%%%%%%%%%%%%%%%%%%%%%%%%%%%%%%%%%%%%%%%%%%%%%%%%%%%%%
\subsection{Conditional Processing}
\label{sec:conditional}

The package provides a mechanism to compile different versions
of a document. To customise the versions further some conditional processing
can come in handy to distinguish which version is being compiled.
The package provides two macros to describe the compilation context:

%%%%%%%%%%%%%%%%%%%%%%%%%%%%%%%%%%%%%%%%
\DescribeMacro{\ifchilddoc}
The conditional |\ifchilddoc| distinguishes between the compilation of
child documents and the main document:
%
\begin{center}
|\ifchilddoc |\textit{child-code}| |[|\||else |\textit{main-code}]| \||fi|
\end{center}

%%%%%%%%%%%%%%%%%%%%%%%%%%%%%%%%%%%%%%%%
\DescribeMacro{\childdocname}
\DescribeMacro{\childdocjob}
The macro |\childdocname| contains the filename (without extension)
of the main or child file being processed.
Note that |\childdocjob| will always contain the name of the main file.

%%%%%%%%%%%%%%%%%%%%%%%%%%%%%%%%%%%%%%%%
\paragraph{Title Page.}

Conditional processing can be used to include a title or banner page
in the main document when proper precautions are taken.
Importantly, the code in the main file should ensure that the page counter
(as well as other status parameters which are stored in the |.aux| files)
takes the same value after the conditional processing.
Otherwise the page numbers may take divergent values
depending on which part is compiled.

For example, a title page could be declared by:
%
\begin{center}
\begin{tabular}{l}
|\ifchilddoc\||else|\\
|\addtocounter{page}{-1}|\\
\textit{code for title page}\\
|\newpage|\\
|\||fi|
\end{tabular}
\end{center}
%
A banner page for the child documents can be generated by:
%
\begin{center}
\begin{tabular}{l}
|\ifchilddoc|\\
|\addtocounter{page}{-1}|\\
\textit{code for banner page}\\
|\newpage|\\
|\||fi|
\end{tabular}
\end{center}
%
Here one could write a message such as:
\begin{center}
|This is the part \childdocname{} of \childdocjob{}.|
\end{center}

%%%%%%%%%%%%%%%%%%%%%%%%%%%%%%%%%%%%%%%%%%%%%%%%%%%%%%%%%%%%%%%%%%%%%%%%%%%%%%%%
\subsection{Flags}
\label{sec:flags}

The package makes it easy to generate different versions
of the main or child documents.
To this end compilation flags can be defined
and assigned different default values.
They will be particularly useful in conjunction
with the forwarding mechanism described in \secref{sec:forward}.

For example, it may be useful to have a flag |\version|
which can be set to |draft| or |final|.
The document source will contain some conditional code
depending on the value of |\version|.
Suppose further, the flag should default to |final| for the main file
and to |draft| for child files
which is a natural assignment for editing the document.
This is achieved by placing the following code
in the preamble of the main document
(below the |\childdocmain| directive):
%
\begin{center}
\begin{tabular}{l}
|\ifchilddoc|\\
|\providecommand{\version}{draft}|\\
|\||else|\\
|\providecommand{\version}{final}|\\
|\||fi|
\end{tabular}
\end{center}
%
The definition by |\providecommand| makes sure
that previous definitions are not overwritten.
Further statements |\providecommand{\version}{...}|
can thus be added before the above code to override it.

For the main file, one might add a line
(between |\childdocmain| and the above block)
%
\begin{center}
|%\ifchilddoc\||else\providecommand{\version}{draft}\||fi|
\end{center}
%
which can be uncommented to produce a draft version.
Likewise one can add a line to the very top of a child file
(above the |\childdocof{|\textit{main}|}| directive)
%
\begin{center}
|%\providecommand{\version}{final}|
\end{center}
%
which can be uncommented to produce the final version of this child document.

%%%%%%%%%%%%%%%%%%%%%%%%%%%%%%%%%%%%%%%%%%%%%%%%%%%%%%%%%%%%%%%%%%%%%%%%%%%%%%%%
\subsection{Forwarding}
\label{sec:forward}

Different versions of the main or child documents
using compilation flags as described in \secref{sec:flags}
can be (permanently) stored in different files
for convenient compilation, viewing and distribution.
To this end, the package defines a command
to pass on compilation to a different file:

%%%%%%%%%%%%%%%%%%%%%%%%%%%%%%%%%%%%%%%%
\DescribeMacro{\childdocforward}
The command |\childdocforward| redirects processing to
another source file:
%
\begin{center}
\begin{tabular}{l}
|\input{childdoc.def}|\\
|\childdocforward[|\textit{main}|]{|\textit{dest}|}|\\
\end{tabular}
\end{center}
%
The argument \textit{dest} is the destination file
(without extension).
It should be the main file or one of the child files.
Note that further \textsf{childdoc} directives
such as |\childdocof| and |\childdocforward|
in the indicated file will be processed in this form.
The optional argument \textit{main}
passes on directly to the main file \textit{main}
while pretending to compile the child \textit{dest}.
This form behaves as if \textit{dest}
issues |\childdocof{|\textit{main}|}| right away,
and no further \textsf{childdoc} directives will be processed.

%%%%%%%%%%%%%%%%%%%%%%%%%%%%%%%%%%%%%%%%
\DescribeMacro{\...prefix}
In the alternative form |\childdocforwardprefix|,
%
\begin{center}
\begin{tabular}{l}
|\input{childdoc.def}|\\
|\childdocforwardprefix[|\textit{main}|]{|\textit{prefix}|}{|\textit{dest}|}|
\end{tabular}
\end{center}
%
the destination file is determined by a pattern
depending on the current file:
To make this work, the current file must be called
`{\textit{prefix}\hspace{0.2em}\textit{suffix}}'
with \textit{prefix} matching precisely the argument.
Processing is then passed on to the file
`{\textit{dest}\hspace{0.2em}\textit{suffix}}'.
Surely, the same effect is achieved by
directly specifying the
argument `{\textit{dest}\hspace{0.2em}\textit{suffix}}'
in the first form.
However, that requires to set up a different file
for each child. With the alternative form of the command
all these files can have exactly the same content
which simplifies setting them up and maintaining them.

For example, the following file |draft.tex|
with a compilation flag |\version| as described in \secref{sec:flags}
compiles the main document as a draft:
%
\begin{center}
\begin{tabular}{l}
|\def\version{draft}|\\
|\input{childdoc.def}|\\
|\childdocforward{|\textit{main}|}|
\end{tabular}
\end{center}
%
Likewise, the following files |final|\textit{nn}|.tex|
compile the final version of the child document
|child|\textit{nn}|.tex|:
%
\begin{center}
\begin{tabular}{l}
|\def\version{final}|\\
|\input{childdoc.def}|\\
|\childdocforwardprefix{final}{child}|
\end{tabular}
\end{center}
%

Note that when several versions of a main file and/or of each child file
are to be generated, it may be convenient to set up a |Makefile| or
shell script to automatise the process.

%%%%%%%%%%%%%%%%%%%%%%%%%%%%%%%%%%%%%%%%%%%%%%%%%%%%%%%%%%%%%%%%%%%%%%%%%%%%%%%%
\subsection{Command Line Processing}
\label{sec:commandline}

The effect of redirection files can also be achieved by invoking
the \LaTeX{} compiler with a more elaborate command line.
Most conveniently this should be done as part
of a shell script or a |Makefile|.

When using \textsf{childdoc} in the main file, the following
command lines effectively perform a redirection
(note that depending on the shell being used,
backslashes may have to be doubled: `|\|' $\to$ `|\\|'):
%
\begin{center}
|... -jobname "|\textit{target}|" |\\|"|[\textit{flags}]%
|\input{childdoc.def}\childdocforward[|\textit{main}|]{|\textit{dest}|}"|
\end{center}
%
Here \textit{target} is the name of the output file,
\textit{main} is the name of the main file
and \textit{dest} is the name of the main or child file to be processed
(all filenames without extensions).
The optional argument \textit{main} can be omitted
if \textit{main} matches \textit{dest}.
Optionally, compilation \textit{flags} can be defined via |\def| commands.
This command line makes the \TeX{} engine believe
it is compiling the file \textit{target}
whose content is specified as the latter parameter.
The provided code then forwards the processing to
\textit{main} or \textit{dest} as described in \secref{sec:forward}.

%%%%%%%%%%%%%%%%%%%%%%%%%%%%%%%%%%%%%%%%%%%%%%%%%%%%%%%%%%%%%%%%%%%%%%%%%%%%%%%%
\subsection{Include by Input}
\label{sec:input}

Including child documents by |\include| has some restrictions by design.
Most notably, the content of a child document always occupies
its own set of pages; pages cannot be shared between child documents.
Usually, this behaviour makes perfect sense
because each child document contain an essential part of the document.
However, in some situations it may be desirable to compose
a document from a collection of parts
without having mandatory page breaks between then.
For this case, the package
provides a mechanism to include parts
by |\input| which can also be processed individually.
However, by construction this mechanism
requires manual handling of the content to be output.

%%%%%%%%%%%%%%%%%%%%%%%%%%%%%%%%%%%%%%%%
\DescribeMacro{\ifchilddocmanual}
The main file should be prepared as usual, see \secref{sec:include}.
However, the document body must make a distinction
between processing of an individual part and of the main document, e.g.:
%
\begin{center}
\begin{tabular}{l}
|\ifchilddocmanual|\\
|\input{\childdocname}|\\
|\||else|\\
\textit{document body with }|\input{|\textit{part}|}|\\
|\||fi|
\end{tabular}
\end{center}
%
The conditional |\ifchilddocmanual| is true whenever
a part to be included by |\input| is being compiled,
and the name of the part is stored in |\childdocname|.

%%%%%%%%%%%%%%%%%%%%%%%%%%%%%%%%%%%%%%%%
\DescribeMacro{\childdocby}
Each part to be included by |\input| should start with:
%
\begin{center}
\begin{tabular}{l}
|\input{childdoc.def}|\\
|\childdocby{|\textit{main}|}|\\
\end{tabular}
\end{center}
%
The directive |\childdocby| is similar to |\childdocof|
described in \secref{sec:include},
but the subsequent selection of content must be done manually.
To that end, both |\ifchilddoc| and |\ifchilddocmanual|
will be true upon processing of a part,
and the name of the part is stored in |\childdocname|.
Note that |\jobname| will be set to the filename of the current part
so that each part receives an individual |.aux| file
that does not interfere with the |.aux| file(s) of the main document.
This behaviour can be altered by the alternative form
|\childdocby[*]{|\textit{main}|}| (with a non-empty optional argument)
which uses the |.aux| file of the main document
by setting |\jobname| to \textit{main}.

%%%%%%%%%%%%%%%%%%%%%%%%%%%%%%%%%%%%%%%%%%%%%%%%%%%%%%%%%%%%%%%%%%%%%%%%%%%%%%%%
\subsection{Driver Development}
\label{sec:driver}

The \textsf{childdoc} mechanism can also be use for the development
of definition files such as \LaTeX{} styles or classes.
This case differs from the above setup with multiple parts
included by |\include| in that no |\includeonly| should be invoked.
This can be achieved by starting the include file
(before |\ProvidesPackage|) with:
%
\begin{center}
\begin{tabular}{l}
|\input{childdoc.def}|\\
|\childdocforward{|\textit{main}|}|\\
\end{tabular}
\end{center}
%
or alternatively with:
%
\begin{center}
\begin{tabular}{l}
|\input{childdoc.def}|\\
|\childdocby{|\textit{main}|}|\\
\end{tabular}
\end{center}
%
Both forms have slightly different effects as described above.
The main file is prepared as usual, see \secref{sec:include}.

%%%%%%%%%%%%%%%%%%%%%%%%%%%%%%%%%%%%%%%%%%%%%%%%%%%%%%%%%%%%%%%%%%%%%%%%%%%%%%%%
\subsection{Legacy Detection}
\label{sec:detection}

The directive |\childdocmain| in the main file can detect
whether the complete document or merely a child is to be compiled
even without using the directive |\childdocof|.
This method is deprecated because it is less robust
and there is no compelling reason to use it;
it is merely provided for backward compatibility
and it may be removed in future versions.

If the detection mechanism is to be used,
it is mandatory to correctly specify
the filename of the main file as the argument of |\childdocmain|:
%
\begin{center}
\begin{tabular}{l}
|\input{childdoc.def}|\\
|\childdocmain{|\textit{main}|}|\\
\end{tabular}
\end{center}
%
If |\jobname| does not match the argument \textit{main} of |\childdocmain|,
it is assumed that |\jobname| points to the child file to be compiled.
When using |\childdocmain| with the main file specified as argument,
it suffices to start a child file
with just |\input{|\textit{main}|}|
without loading of the package and using |\childdocof|.
If instead all processing is done
with the appropriate \textsf{childdoc} directives,
the argument of \textit{main} of |\childdocmain| can be empty.

An alternative version of the command line processing described
in \secref{sec:commandline} using the detection mechanism reads:
%
\begin{center}
|... -jobname "|\textit{target}|" "|[\textit{flags}]%
[|\def\jobname{|\textit{dest}|}|]|\input{|\textit{main}|}"|
\end{center}

%%%%%%%%%%%%%%%%%%%%%%%%%%%%%%%%%%%%%%%%%%%%%%%%%%%%%%%%%%%%%%%%%%%%%%%%%%%%%%%%
\subsection{Manual Code}
\label{sec:manual}

In case one cannot be certain whether the definitions file |childdoc.def|
is installed on the target \TeX{} distribution
and one prefers not to ship it,
it is conceivable to paste a few relevant commands into the sources.

To that end, drop all statements |\input{childdoc.def}|
and perform the replacements as outlined below.
Instead of |\childdocmain{|\textit{main}|}| add the following code
to the top of the main file:
%
\begin{center}
\begin{tabular}{l}
|\||ifdefined\childdocname\endinput\||fi\newif\ifchilddoc|\\
|\edef\childdocname{\scantokens\expandafter{\jobname\noexpand}}|\\
|\def\childdocmain{|\textit{main}|}\||ifx\childdocmain\childdocname\||else|\\
|\childdoctrue\includeonly{\childdocname}\let\jobname\childdocmain\||fi|\\
\end{tabular}
\end{center}
%
Instead of |\childdocof{|\textit{main}|}| just include the main file
at the top of each child file:
%
\begin{center}
|\input{|\textit{main}|}|
\end{center}
%
A simple redirection |\childdocforward{|\textit{dest}|}| is achieved by:
%
\begin{center}
|\def\jobname{|\textit{dest}|}\input{\jobname}|
\end{center}
%
The redirection with prefix
|\childdocforwardprefix[|\textit{prefix}|]{|\textit{dest}|}|
is accomplished by:
%
\begin{center}
\begin{tabular}{l}
|{\edef\jobname{\scantokens\expandafter{\jobname\noexpand}}|\\
|\def\redirectjob |\textit{prefix}|#1~~~{\gdef\jobname{|\textit{dest}|#1}}|\\
|\expandafter\redirectjob\jobname~~~}\input{\jobname}|
\end{tabular}
\end{center}

In an alternative approach,
child documents can be compiled by a specific command line
without additional code or specific definitions:
%
\begin{center}
|... -jobname "|\textit{target}|" "|[\textit{flags}]%
|\includeonly{|\textit{dest}|}\input{|\textit{main}|}"|
\end{center}
%

%%%%%%%%%%%%%%%%%%%%%%%%%%%%%%%%%%%%%%%%%%%%%%%%%%%%%%%%%%%%%%%%%%%%%%%%%%%%%%%%
%%%%%%%%%%%%%%%%%%%%%%%%%%%%%%%%%%%%%%%%%%%%%%%%%%%%%%%%%%%%%%%%%%%%%%%%%%%%%%%%
\section{Information}

%%%%%%%%%%%%%%%%%%%%%%%%%%%%%%%%%%%%%%%%%%%%%%%%%%%%%%%%%%%%%%%%%%%%%%%%%%%%%%%%
\subsection{Copyright}

Copyright \copyright{} 2017--2018 Niklas Beisert

This work may be distributed and/or modified under the
conditions of the \LaTeX{} Project Public License, either version 1.3
of this license or (at your option) any later version.
The latest version of this license is in
  \url{http://www.latex-project.org/lppl.txt}
and version 1.3 or later is part of all distributions of \LaTeX{}
version 2005/12/01 or later.

This work has the LPPL maintenance status `maintained'.

The Current Maintainer of this work is Niklas Beisert.

This work consists of the files |README.txt|, |childdoc.ins| and |childdoc.dtx|
as well as the derived files |childdoc.def|, |cdocsamp.tex|
with |cdocsch1.tex|, |cdocsch2.tex|, |cdocspt3.tex|, |cdocspt4.tex|,
|cdocsdrf.tex|, |cdocsfn1.tex|, |cdocsfn2.tex|
as well as |childdoc.pdf|.

%%%%%%%%%%%%%%%%%%%%%%%%%%%%%%%%%%%%%%%%%%%%%%%%%%%%%%%%%%%%%%%%%%%%%%%%%%%%%%%%
\subsection{Files and Installation}

The package consists of the files:
%
\begin{center}
\begin{tabular}{ll}
    |README.txt|   & readme file \\
    |childdoc.ins| & installation file \\
    |childdoc.dtx| & source file \\
    |childdoc.def| & definition file \\
    |cdocsamp.tex| & sample main file \\
    |cdocsch1.tex| & sample include file \\
    |cdocsch2.tex| & sample include file \\
    |cdocspt3.tex| & sample part file \\
    |cdocspt4.tex| & sample part file \\
    |cdocsdrf.tex| & sample redirection file \\
    |cdocsfn1.tex| & sample redirection file \\
    |cdocsfn2.tex| & sample redirection file \\
    |childdoc.pdf| & manual
\end{tabular}
\end{center}
%
The distribution consists of the files
|README.txt|, |childdoc.ins| and |childdoc.dtx|.
%
\begin{itemize}
\item
Run (pdf)\LaTeX{} on |childdoc.dtx|
to compile the manual |childdoc.pdf| (this file).
\item
Run \LaTeX{} on |childdoc.ins| to create the definitions file |childdoc.def|
and the sample |cdocsamp.tex| with include files
|cdocsch1.tex|, |cdocsch2.tex|, |cdocspt3.tex|, |cdocspt4.tex|,
|cdocsdrf.tex|, |cdocsfn1.tex|, |cdocsfn2.tex|.
Then copy the file |childdoc.def| to an appropriate directory of your \LaTeX{}
distribution, e.g.\ \textit{texmf-root}|/tex/latex/childdoc|.
\end{itemize}

%%%%%%%%%%%%%%%%%%%%%%%%%%%%%%%%%%%%%%%%%%%%%%%%%%%%%%%%%%%%%%%%%%%%%%%%%%%%%%%%
\subsection{Related CTAN Packages}

There are several other packages which offer a similar functionality:
%
\begin{itemize}
\item
The packages
\href{http://ctan.org/pkg/docmute}{\textsf{docmute}},
\href{http://ctan.org/pkg/includex}{\textsf{includex}} and
\href{http://ctan.org/pkg/standalone}{\textsf{standalone}}
provide commands to include only the document body of
a child file thus allowing both files to be compiled individually.
\item
The packages \href{http://ctan.org/pkg/subdocs}{\textsf{subdocs}}
and \href{http://ctan.org/pkg/subfiles}{\textsf{subfiles}}
provide structures in which the main and child documents can be
encapsulated and allowing them to be compiled individually.
The inclusion mechanism is different from the conventional |\include|.
\item
The package \href{http://ctan.org/pkg/combine}{\textsf{combine}}
is an elaborate solution to combine several documents into one.
\end{itemize}
%
See also the CTAN topic \href{http://ctan.org/topic/subdocs}{\textsf{subdocs}}
for further related packages.
The present package differs from the above solutions in that
a document structure constructed with the conventional |\include| mechanism
just needs two extra commands at the top of every file
such that all constituent files can be compiled individually.

%%%%%%%%%%%%%%%%%%%%%%%%%%%%%%%%%%%%%%%%%%%%%%%%%%%%%%%%%%%%%%%%%%%%%%%%%%%%%%%%
%\subsection{Feature Suggestions}
%
%The following is a list of features which may be useful for future
%versions of this package:
%%
%\begin{itemize}
%\item
%\ldots
%\end{itemize}

%%%%%%%%%%%%%%%%%%%%%%%%%%%%%%%%%%%%%%%%%%%%%%%%%%%%%%%%%%%%%%%%%%%%%%%%%%%%%%%%
\subsection{Revision History}

%%%%%%%%%%%%%%%%%%%%%%%%%%%%%%%%%%%%%%%%
\paragraph{v2.0:} 2018/12/30

\begin{itemize}
\item
immediate forward processing
\item
added |\childdocby| mechanism
\item
manual restructured
\end{itemize}

%%%%%%%%%%%%%%%%%%%%%%%%%%%%%%%%%%%%%%%%
\paragraph{v1.6:} 2018/01/17

\begin{itemize}
\item
application for development of include files
\item
corrections to manual
\end{itemize}

%%%%%%%%%%%%%%%%%%%%%%%%%%%%%%%%%%%%%%%%
\paragraph{v1.5:} 2017/05/21

\begin{itemize}
\item
more complete structuring introduced
\item
|\childdocof| introduced
\item
|\childdoc| renamed to |\childdocmain|
\item
|\childredirect| renamed to |\childdocforward| and |\childdocforwardprefix|
and functionality expanded
\end{itemize}

%%%%%%%%%%%%%%%%%%%%%%%%%%%%%%%%%%%%%%%%
\paragraph{v1.0:} 2017/04/27

\begin{itemize}
\item
manual and install package
\item
first version published on CTAN
\end{itemize}

%%%%%%%%%%%%%%%%%%%%%%%%%%%%%%%%%%%%%%%%
\paragraph{v0.6:} 2017/04/26

\begin{itemize}
\item
redirection mechanism added
\end{itemize}

%%%%%%%%%%%%%%%%%%%%%%%%%%%%%%%%%%%%%%%%
\paragraph{v0.5:} 2017/04/26

\begin{itemize}
\item
functionality in definition file
\end{itemize}


%%%%%%%%%%%%%%%%%%%%%%%%%%%%%%%%%%%%%%%%%%%%%%%%%%%%%%%%%%%%%%%%%%%%%%%%%%%%%%%%
%%%%%%%%%%%%%%%%%%%%%%%%%%%%%%%%%%%%%%%%%%%%%%%%%%%%%%%%%%%%%%%%%%%%%%%%%%%%%%%%
%%%%%%%%%%%%%%%%%%%%%%%%%%%%%%%%%%%%%%%%%%%%%%%%%%%%%%%%%%%%%%%%%%%%%%%%%%%%%%%%
\appendix

\settowidth\MacroIndent{\rmfamily\scriptsize 000\ }

 \DocInput{childdoc.dtx}

\end{document}
%</driver>
% \fi
%
% %%%%%%%%%%%%%%%%%%%%%%%%%%%%%%%%%%%%%%%%%%%%%%%%%%%%%%%%%%%%%%%%%%%%%%%%%%%%%%
% %%%%%%%%%%%%%%%%%%%%%%%%%%%%%%%%%%%%%%%%%%%%%%%%%%%%%%%%%%%%%%%%%%%%%%%%%%%%%%
% \section{Sample}
%\iffalse
%<*samplemain>
%\fi
%
% The following presents a sample document
% with two chapters, two parts, a title page,
% a compile flag as well as three forwarding files to set the flag.
% It consists of eight |.tex| files:
% \begin{center}
% \begin{tabular}{ll}
% |cdocsamp.tex|&main file\\
% |cdocsch1.tex|&include file for chapter 1\\
% |cdocsch2.tex|&include file for chapter 2\\
% |cdocspt3.tex|&include file for part 3\\
% |cdocspt4.tex|&include file for part 4\\
% |cdocsdrf.tex|&forwarding file for main file in draft mode\\
% |cdocsfi1.tex|&forwarding file for final version of chapter 1\\
% |cdocsfi2.tex|&forwarding file for final version of chapter 2\\
% \end{tabular}
% \end{center}
% Each of the eight files can be compiled directly by the \LaTeX{} compiler.
%
% %%%%%%%%%%%%%%%%%%%%%%%%%%%%%%%%%%%%%%
% \paragraph{Main File.}
%
% The main file is called |cdocsamp.tex|.
%
% Load the \textsf{childdoc} definitions and
% declare the filename for the main document:
%    \begin{macrocode}
\input{childdoc.def}
\childdocmain{}
%    \end{macrocode}

% Optional override for |\version| flag:
%    \begin{macrocode}
%%\ifchilddoc\else\providecommand{\version}{draft}\fi
%    \end{macrocode}

% Define the default values for the |\version| flag
% (|final| for the main file and |draft| for childs):
%    \begin{macrocode}
\ifchilddoc
\providecommand{\version}{draft}
\else
\providecommand{\version}{final}
\fi
%    \end{macrocode}

% Load the standard document class:
%    \begin{macrocode}
\documentclass[12pt]{article}
%    \end{macrocode}

% Start the document body:
%    \begin{macrocode}
\begin{document}
%    \end{macrocode}

% Declare a title page.
% Print title, part of document being processed and version flag:
%    \begin{macrocode}
\addtocounter{page}{-1}
\begin{center}
{\LARGE\bfseries{}childdoc example\par}
\vspace{1cm}
\ifchilddoc
\ifchilddocmanual part\else chapter\fi:
`\childdocname' of `\childdocjob'\par
\else
main document: `\childdocjob'\par
\fi
version: \version\par
\end{center}
\newpage
%    \end{macrocode}

% Manually include selected file,
% otherwise process as usual:
%    \begin{macrocode}
\ifchilddocmanual
\section*{part `\childdocname'}
\input{\childdocname}
\else
%    \end{macrocode}

% Include the two chapters:
%    \begin{macrocode}
\include{cdocsch1}
\include{cdocsch2}
%    \end{macrocode}

% Include the two parts unless only chapters should be displayed:
%    \begin{macrocode}
\ifchilddoc\else
\section{part three}
\input{cdocspt3}
\section{part four}
\input{cdocspt4}
\fi
%    \end{macrocode}

% Process as usual until here:
%    \begin{macrocode}
\fi
%    \end{macrocode}

% End of document body:
%    \begin{macrocode}
\end{document}
%    \end{macrocode}
%\iffalse
%</samplemain>
%\fi
%
% %%%%%%%%%%%%%%%%%%%%%%%%%%%%%%%%%%%%%%
% \paragraph{Chapter Include Files.}
%
% The include files are called |cdocsch1.tex| and |cdocsch2.tex|.
%
%\iffalse
%<*samplechap1|samplechap2>
%\fi

% Optional override for |\version| flag:
%    \begin{macrocode}
%%\providecommand{\version}{final}
%    \end{macrocode}

% Include the main document:
%    \begin{macrocode}
\input{childdoc.def}
\childdocof{cdocsamp}
%    \end{macrocode}

%\iffalse
%</samplechap1|samplechap2>
%\fi
%
%\iffalse
%<*samplechap1>
%\fi
% Some text for chapter 1:
%    \begin{macrocode}
\section{one}
some text in chapter one
%    \end{macrocode}

%\iffalse
%</samplechap1>
%\fi
% Some text for chapter 2:
%\iffalse
%<*samplechap2>
%\fi
%    \begin{macrocode}
\section{two}
more text in chapter two
%    \end{macrocode}

%\iffalse
%</samplechap2>
%\fi
%
% %%%%%%%%%%%%%%%%%%%%%%%%%%%%%%%%%%%%%%
% \paragraph{Part Include Files.}
%
% The include files are called |cdocspt3.tex| and |cdocspt4.tex|.
%
%\iffalse
%<*samplepart3|samplepart4>
%\fi

% Optional override for |\version| flag:
%    \begin{macrocode}
%%\providecommand{\version}{final}
%    \end{macrocode}

% Include the main document:
%    \begin{macrocode}
\input{childdoc.def}
\childdocby{cdocsamp}
%    \end{macrocode}

%\iffalse
%</samplepart3|samplepart4>
%\fi
%
%\iffalse
%<*samplepart3>
%\fi
% Some text for part 3:
%    \begin{macrocode}
some text in part three
%    \end{macrocode}

%\iffalse
%</samplepart3>
%\fi
% Some text for part 4:
%\iffalse
%<*samplepart4>
%\fi
%    \begin{macrocode}
more text in part four
%    \end{macrocode}

%\iffalse
%</samplepart4>
%\fi
%
% %%%%%%%%%%%%%%%%%%%%%%%%%%%%%%%%%%%%%%
% \paragraph{Forwarding for a Complete Draft.}
%
% The following forwarding file |cdocsdrf.tex|
% compiles the main document in draft mode:
%\iffalse
%<*sampledraft>
%\fi
%    \begin{macrocode}
\def\version{draft}
\input{childdoc.def}
\childdocforward{cdocsamp}
%    \end{macrocode}

%\iffalse
%</sampledraft>
%\fi
%
% %%%%%%%%%%%%%%%%%%%%%%%%%%%%%%%%%%%%%%
% \paragraph{Forwarding for Final Version of the Chapters.}
%
% The following forwarding files |cdocsfn1.tex| and |cdocsfn2.tex|
% (with identical content)
% compile the final versions of the child documents
% |cdocsch1.tex| and |cdocsch2.tex|, respectively:
%\iffalse
%<*samplefinal>
%\fi
%    \begin{macrocode}
\def\version{final}
\input{childdoc.def}
\childdocforwardprefix[cdocsamp]{cdocsfn}{cdocsch}
%    \end{macrocode}

%\iffalse
%</samplefinal>
%\fi
%
% %%%%%%%%%%%%%%%%%%%%%%%%%%%%%%%%%%%%%%
% \paragraph{Command Line Processing.}
%
% The following three command lines generate the output files
% |cdocscld|, |cdocscl1| and |cdocscl2|
% which should be identical to
% |cdocsdrf|, |cdocsch1| and |cdocsfn2|, respectively:
% \begin{center}
% \begin{tabular}{l}
% |latex -jobname cdocscld \|\\
% |  "\def\version{draft}\input{childdoc.def}\childdocforward{cdocsamp}"|\\
% |latex -jobname cdocscl1 \|\\
% |  "\input{childdoc.def}\childdocforward[cdocsamp]{cdocsch1}"|\\
% |latex -jobname cdocscl2 \|\\
% |  "\def\version{final}\input{childdoc.def}\childdocforward{cdocsch2}"|
% \end{tabular}
% \end{center}
% Note that the trailing backslash on each first line
% merely continues the input to the second line
% (for convenient cut ant paste).
% Furthermore, the command |latex| can be replaced by any
% of its alternative versions such as |pdflatex|.
%
% %%%%%%%%%%%%%%%%%%%%%%%%%%%%%%%%%%%%%%%%%%%%%%%%%%%%%%%%%%%%%%%%%%%%%%%%%%%%%%
% %%%%%%%%%%%%%%%%%%%%%%%%%%%%%%%%%%%%%%%%%%%%%%%%%%%%%%%%%%%%%%%%%%%%%%%%%%%%%%
% \section{Implementation}
%\iffalse
%<*package>
%\fi
%
% This section describes the definitions file |childdoc.def|.

% The definitions cannot be loaded using |\usepackage| or |\RequirePackage|
% which has a mechanism to prevent loading a style file more than once.
% When loading the definitions by means of |\input|
% multiple instances have to be prevented manually:
%\iffalse
%This code needs to be before the `\ProvidesFile' directive
%which is defined at the beginning of this file.
%Therefore it is also placed there and commented out here.
%</package>
%<*discard>
%\fi
%    \begin{macrocode}
\ifdefined\childdocmain\endinput\fi
%    \end{macrocode}
%\iffalse
%</discard>
%<*package>
%\fi
%
% \macro{\ifchilddoc}
% \macro{\ifchilddocmanual}
% The conditional |\ifchilddoc| tells whether a
% child (true) or main (false) document is being compiled.
% The conditional |\ifchilddocmanual| tells whether
% the |\includeonly| mechanism is used (false) or
% the selection of child files must be performed manually (true).
% The definitions initialise to false:
%    \begin{macrocode}
\newif\ifchilddoc
\newif\ifchilddocmanual
%    \end{macrocode}

% \macro{\childdocname}
% \macro{\childdocjob}
% The macro |\childdocname| stores the name of the main document
% to be compiled. The macro |\childdocjob| stores the name of
% the document on which the \LaTeX{} compiler was originally invoked.
% The content of |\jobname| cannot be compared
% to filenames specified in the source due to different catcodes.
% The following code rescans |\jobname|, stores the result
% in |\childdocname| and saves a copy in |\childdocjob|:
%    \begin{macrocode}
\edef\childdocname{\scantokens\expandafter{\jobname\noexpand}}
\let\childdocjob\childdocname
%    \end{macrocode}

% \macro{\childdocdisable}
% The macro |\childdocdisable| prevents the main file
% from being processed more than once.
% At this stage, the main document command |\childdocmain|
% is assumed to be called once again where it should do nothing.
% Any subsequent call to it should prevent
% a secondary processing of the main document
% It overwrites the forwarding commands
% |\childdocof| and |\childdocforward|
% with empty macros to prevent further inclusions of the main document:
%    \begin{macrocode}
\newcommand{\childdocdisable}
{
  \renewcommand{\childdocmain}[1]{\renewcommand{\childdocmain}[1]{\endinput}}
  \renewcommand{\childdocof}[1]{}
  \renewcommand{\childdocby}[2][]{}
  \renewcommand{\childdocforward}[2][]{}
  \renewcommand{\childdocdisable}{}
}
%    \end{macrocode}

% \macro{\childdocmain}
% The macro |\childdocmain| is to be called at the top of the main file
% with nothing or the main filename (without extension) as argument.
% First, it breaks loops.
% If the argument is not empty and does not match |\childdocname|
% (which is set by the first inclusion of |childdoc.def|),
% |\ifchilddoc| is set to true, |\includeonly| is applied to the child file
% and |\jobname| is set to the main file
% (for proper handling of |.aux| files):
%    \begin{macrocode}
\newcommand{\childdocmain}[1]
{
  \childdocdisable\childdocmain{}
  \if?#1?\else
    \begingroup
      \def\childdoctmp{#1}
      \ifx\childdoctmp\childdocname
        \def\childdoctmp{}
      \else
        \def\childdoctmp
        {
          \childdoctrue
          \includeonly{\childdocname}
          \def\childdocjob{#1}
          \def\jobname{#1}
        }
      \fi
      \expandafter
    \endgroup
    \childdoctmp
  \fi
}
%    \end{macrocode}

% \macro{\childdocof}
% The command |\childdocof| redirects
% compilation to the main file |#1|.
%    \begin{macrocode}
\newcommand{\childdocof}[1]
{
  \childdocdisable
  \childdoctrue
  \includeonly{\childdocname}
  \def\jobname{#1}
  \def\childdocjob{#1}
  \input{#1}
}
%    \end{macrocode}

% \macro{\childdocby}
% The command |\childdocby| ....
%    \begin{macrocode}
\newcommand{\childdocby}[2][]
{
  \childdocdisable
  \childdoctrue
  \childdocmanualtrue
  \if?#1?\else
    \def\jobname{#2}
  \fi
  \def\childdocjob{#2}
  \input{#2}
  \endinput
}
%    \end{macrocode}

% \macro{\childdocforward}
% The command |\childdocforward| redirects
% compilation to the main file or
% (if the optional argument is given) a child file.
% Parameters are set as if the main file
% or a child file starting with |\childdocof| was compiled.
% Then compilation is handed over to the main file:
%    \begin{macrocode}
\newcommand{\childdocforward}[2][]
{
  \begingroup
    \if?#1?
      \def\childdoctmp
      {
        \def\childdocname{#2}
        \def\childdocjob{#2}
        \def\jobname{#2}
        \input{#2}
        \endinput
      }
    \else
      \def\childdoctmp
      {
        \childdocdisable
        \def\childdocname{#2}
        \childdoctrue
        \includeonly{#2}
        \def\childdocjob{#1}
        \def\jobname{#1}
        \input{#1}
        \endinput
      }
    \fi
    \expandafter
  \endgroup
  \childdoctmp
}
%    \end{macrocode}

% \macro{\childdocforwardprefix}
% The command |\childdocforwardprefix| redirects
% compilation to the main or a child file by means of a pattern.
% The prefix |#1| in the current filename is replaced by |#2|
% and the suffix of the current filename is kept
% (it is assumed that the filename does not contain the substring `|~~~|'
% which is used as a delimiter).
% Compilation is handed over to the new file by |\childdocforward|:
%    \begin{macrocode}
\newcommand{\childdocforwardprefix}[3][]
{
  \begingroup
    \def\childdocextract #2##1~~~{\def\childdoctmp{\childdocforward[#1]{#3##1}}}
    \expandafter\childdocextract\childdocname~~~
    \expandafter
  \endgroup
  \childdoctmp
}
%    \end{macrocode}

% \macro{\childdoc}
% The deprecated macro |\childdoc| is a legacy version of |\childdocmain|:
%    \begin{macrocode}
\newcommand{\childdoc}{\childdocmain}
%    \end{macrocode}

% \macro{\childdocredirect}
% The deprecated macro |\childdocredirect| is a legacy version
% of |\childdocforward| and |\childdocforwardprefix|:
%    \begin{macrocode}
\newcommand{\childdocredirect}[2][]
{
  \begingroup
    \if?#1?
      \def\childdoctmp{\childdocforward{#2}}
    \else
      \def\childdoctmp{\childdocforwardprefix{#1}{#2}}
    \fi
    \expandafter
  \endgroup
  \childdoctmp
}
%    \end{macrocode}

%\iffalse
%</package>
%\fi
%
\endinput

\childdocof{cdocsamp}
%    \end{macrocode}

%\iffalse
%</samplechap1|samplechap2>
%\fi
%
%\iffalse
%<*samplechap1>
%\fi
% Some text for chapter 1:
%    \begin{macrocode}
\section{one}
some text in chapter one
%    \end{macrocode}

%\iffalse
%</samplechap1>
%\fi
% Some text for chapter 2:
%\iffalse
%<*samplechap2>
%\fi
%    \begin{macrocode}
\section{two}
more text in chapter two
%    \end{macrocode}

%\iffalse
%</samplechap2>
%\fi
%
% %%%%%%%%%%%%%%%%%%%%%%%%%%%%%%%%%%%%%%
% \paragraph{Part Include Files.}
%
% The include files are called |cdocspt3.tex| and |cdocspt4.tex|.
%
%\iffalse
%<*samplepart3|samplepart4>
%\fi

% Optional override for |\version| flag:
%    \begin{macrocode}
%%\providecommand{\version}{final}
%    \end{macrocode}

% Include the main document:
%    \begin{macrocode}
% \iffalse
%
% childdoc.dtx Copyright (C) 2017-2018 Niklas Beisert
%
% This work may be distributed and/or modified under the
% conditions of the LaTeX Project Public License, either version 1.3
% of this license or (at your option) any later version.
% The latest version of this license is in
%   http://www.latex-project.org/lppl.txt
% and version 1.3 or later is part of all distributions of LaTeX
% version 2005/12/01 or later.
%
% This work has the LPPL maintenance status `maintained'.
%
% The Current Maintainer of this work is Niklas Beisert.
%
% This work consists of the files childdoc.dtx and childdoc.ins
% and the derived files childdoc.def and cdocsamp.tex with
% cdocsch1.tex, cdocsch2.tex, cdocsdrf.tex, cdocsfn1.tex, cdocsfn2.tex.
%
%<package>\ifdefined\childdocmain\endinput\fi
%<package>\ProvidesFile{childdoc.def}[2018/12/30 v2.0 child document driver]
%<samplemain>\ProvidesFile{cdocsamp.tex}[2018/12/30 v2.0 sample for childdoc]
%<*driver>
%\ProvidesFile{childdoc.drv}[2018/12/30 v2.0 childdoc reference manual file]
\PassOptionsToClass{10pt,a4paper}{article}
\documentclass{ltxdoc}

\usepackage[margin=35mm]{geometry}
\usepackage{hyperref}
\usepackage{hyperxmp}
\usepackage[usenames]{color}

\hypersetup{colorlinks=true}
\hypersetup{pdfstartview=FitH}
\hypersetup{pdfpagemode=UseNone}
\hypersetup{pdfsource={}}
\hypersetup{pdflang={en-UK}}
\hypersetup{pdfcopyright={Copyright 2017-2018 Niklas Beisert.
  This work may be distributed and/or modified under the
  conditions of the LaTeX Project Public License, either version 1.3
  of this license or (at your option) any later version.}}
\hypersetup{pdflicenseurl={http://www.latex-project.org/lppl.txt}}
\hypersetup{pdfcontactaddress={ETH Zurich, ITP, HIT K,
  Wolfgang-Pauli-Strasse 27}}
\hypersetup{pdfcontactpostcode={8093}}
\hypersetup{pdfcontactcity={Zurich}}
\hypersetup{pdfcontactcountry={Switzerland}}
\hypersetup{pdfcontactemail={nbeisert@itp.phys.ethz.ch}}
\hypersetup{pdfcontacturl={http://people.phys.ethz.ch/\xmptilde nbeisert/}}

\newcommand{\secref}[1]{\hyperref[#1]{section \ref*{#1}}}

\parskip1ex
\parindent0pt
\let\olditemize\itemize
\def\itemize{\olditemize\parskip0pt}

\begin{document}

\title{The \textsf{childdoc} Package}
\hypersetup{pdftitle={The childdoc Package}}
\author{Niklas Beisert\\[2ex]
  Institut f\"ur Theoretische Physik\\
  Eidgen\"ossische Technische Hochschule Z\"urich\\
  Wolfgang-Pauli-Strasse 27, 8093 Z\"urich, Switzerland\\[1ex]
  \href{mailto:nbeisert@itp.phys.ethz.ch}
  {\texttt{nbeisert@itp.phys.ethz.ch}}}
\hypersetup{pdfauthor={Niklas Beisert}}
\hypersetup{pdfsubject={Manual for the LaTeX2e Package childdoc}}
\date{30 December 2018, \textsf{v2.0}}
\maketitle

\begin{abstract}\noindent
\textsf{childdoc} is a \LaTeXe{} package
that enables the direct compilation
of document sections included by |\include|
to individual files.
\end{abstract}

\begingroup
\parskip0ex
\tableofcontents
\endgroup

%%%%%%%%%%%%%%%%%%%%%%%%%%%%%%%%%%%%%%%%%%%%%%%%%%%%%%%%%%%%%%%%%%%%%%%%%%%%%%%%
%%%%%%%%%%%%%%%%%%%%%%%%%%%%%%%%%%%%%%%%%%%%%%%%%%%%%%%%%%%%%%%%%%%%%%%%%%%%%%%%
\section{Introduction}

\LaTeX{} provides a mechanism to structure a large document (such as a book)
into a main file and several child files (containing the chapters)
using the |\include| command.
This mechanism is beneficial for documents
which span hundreds of pages in order to
make the source file(s) more manageable.
Moreover, compilation can be restricted to
selected child files by means of the |\includeonly| command.
The latter feature can be used to reduce the compilation time while editing
(this was significantly more useful in the earlier days of \LaTeX{})
or to generate a smaller document which is easier to navigate.
Another application of |\includeonly| is to generate
documents consisting of selected parts of the complete document.

However, there are a few drawbacks of the plain |\include| mechanism:
\begin{itemize}
\item
The child files cannot be compiled on their own,
they can only be compiled via the main file.
A naive editing environment
(such as a text editor with an option
to have the current file processed by \LaTeX)
may require one to switch to the main file before compiling;
attempting to compile the child file produces errors.
\item
The main file must be modified (each time)
to adjust the |\includeonly| command
to the present needs. This easily leaves the main file in a messy state.
\item
The generated document will always carry the filename
of the main document. This is inconvenient if
several child files are to be compiled and
to be kept for distribution.
\end{itemize}

The present package provides a simple interface
to make child files individually compilable by \LaTeX{}.
Compiling a child file then has the same effect as compiling
the main file with an |\includeonly| command
to select the appropriate child.
Moreover the generated document will carry the name of the child
rather than the main file.
This resolves all three above issues.

This feature is meant to make the editing of books,
thesis documents and lecture notes somewhat more convenient.
However, the package can also be used efficiently for
composing a series of documents (such as exercise sheets)
which are typically distributed individually.
It then assists the author in generating the individual documents
(potentially in different versions)
as well as a document containing the collected series.
Another application is in developing style files
or other kinds of included material
where compilation of the style file could redirect
to a sample or test file.

%%%%%%%%%%%%%%%%%%%%%%%%%%%%%%%%%%%%%%%%%%%%%%%%%%%%%%%%%%%%%%%%%%%%%%%%%%%%%%%%
%%%%%%%%%%%%%%%%%%%%%%%%%%%%%%%%%%%%%%%%%%%%%%%%%%%%%%%%%%%%%%%%%%%%%%%%%%%%%%%%
\section{Usage}

First of all, the package \textsf{childdoc} is \emph{not} a standard
\LaTeXe{} |.sty| style file! Therefore it needs to be invoked in
a non-standard way.

%%%%%%%%%%%%%%%%%%%%%%%%%%%%%%%%%%%%%%%%%%%%%%%%%%%%%%%%%%%%%%%%%%%%%%%%%%%%%%%%
\subsection{Included Files}
\label{sec:include}

%%%%%%%%%%%%%%%%%%%%%%%%%%%%%%%%%%%%%%%%
\DescribeMacro{\childdocmain}
To use the package, add the commands
\begin{center}
\begin{tabular}{l}
|\input{childdoc.def}|\\
|\childdocmain{}|\\
\end{tabular}
\end{center}
at the very top of the main \LaTeX{} file,
in particular \emph{before} the |\documentclass| statement!
The argument of |\childdocmain| should be left empty
(but it must be present).

%%%%%%%%%%%%%%%%%%%%%%%%%%%%%%%%%%%%%%%%
\DescribeMacro{\childdocof}
Furthermore, add the commands
\begin{center}
\begin{tabular}{l}
|\input{childdoc.def}|\\
|\childdocof{|\textit{main}|}|\\
\end{tabular}
\end{center}
at the top of every child file \textit{child}
which is included by |\include{|\textit{child}|}|
from within the main file
(or at least for those files to be compiled individually).
The argument \textit{main} must be the filename of the main file.

There are a couple of
considerations in setting up the main and child documents:

%%%%%%%%%%%%%%%%%%%%%%%%%%%%%%%%%%%%%%%%
\paragraph{Restrictions.}

Please note the following restrictions:
\begin{itemize}
\item
|\childdocmain| must be called with one argument \textit{main}
to ensure compatibility with earlier version of the package.
It must either be empty (|\childdocmain{}|)
or precisely match the filename of the main file in which it is specified.
See \secref{sec:detection} for further information.
\item
The filename \textit{main} must be specified without the |.tex| extension.
\item
The filename \textit{main} is case sensitive
(even in case-insensitive file systems)
due to internal string comparison.
\item
The argument \textit{main} should be fully expanded, it cannot be a macro.
\item
Subdirectories and special characters should be avoided in filenames.
\item
The command |\childdocmain{|\textit{main}|}| must be followed by a whitespace.
It should not be followed immediately by another command
or by a comment mark `|%|'.
This is because the \TeX{} parser reads the token immediately following
the argument of |\childdocmain| and puts it
at the beginning of every child section;
however, a white\-space is ignored.
\end{itemize}

%%%%%%%%%%%%%%%%%%%%%%%%%%%%%%%%%%%%%%%%
\paragraph{Content of Main File.}

It is advisable to place all content in the child files included by |\include|.
Any output contained in the main file will appear in all child documents
unless suppressed manually;
it cannot be suppressed automatically by the |\includeonly| directive
and thus should normally be avoided.
A method to include some content in the main file
by means of conditional processing is described in \secref{sec:conditional}.

%%%%%%%%%%%%%%%%%%%%%%%%%%%%%%%%%%%%%%%%
\paragraph{Page Numbering.}

When only a part of the document is compiled,
the appropriate numbering of pages
(as well as other status parameters)
is determined from the |.aux| files.
The latter contain information from previous passes.
However this information needs to propagate through
all intermediate child documents.
Therefore the page numbering in child documents may well
be inconsistent until the complete document is compiled at least once.

A useful (if unconventional) way to always ensure a consistent
page numbering is to restart the numbering in each child document
and denote the pages by `\textit{child}|.|\textit{page}'
where \textit{child} represents the chapter/section number of the child file.
This can be achieved by the command
|\numberwithin{page}{|\textit{child}|}|
of the \textsf{amsmath} package
where \textit{child} can be |chapter| or |section|
depending on the chosen structuring.
Alternatively, one can modify the macro |\thepage| appropriately
and reset the counter |page| at the start of each child file.

%%%%%%%%%%%%%%%%%%%%%%%%%%%%%%%%%%%%%%%%%%%%%%%%%%%%%%%%%%%%%%%%%%%%%%%%%%%%%%%%
\subsection{Conditional Processing}
\label{sec:conditional}

The package provides a mechanism to compile different versions
of a document. To customise the versions further some conditional processing
can come in handy to distinguish which version is being compiled.
The package provides two macros to describe the compilation context:

%%%%%%%%%%%%%%%%%%%%%%%%%%%%%%%%%%%%%%%%
\DescribeMacro{\ifchilddoc}
The conditional |\ifchilddoc| distinguishes between the compilation of
child documents and the main document:
%
\begin{center}
|\ifchilddoc |\textit{child-code}| |[|\||else |\textit{main-code}]| \||fi|
\end{center}

%%%%%%%%%%%%%%%%%%%%%%%%%%%%%%%%%%%%%%%%
\DescribeMacro{\childdocname}
\DescribeMacro{\childdocjob}
The macro |\childdocname| contains the filename (without extension)
of the main or child file being processed.
Note that |\childdocjob| will always contain the name of the main file.

%%%%%%%%%%%%%%%%%%%%%%%%%%%%%%%%%%%%%%%%
\paragraph{Title Page.}

Conditional processing can be used to include a title or banner page
in the main document when proper precautions are taken.
Importantly, the code in the main file should ensure that the page counter
(as well as other status parameters which are stored in the |.aux| files)
takes the same value after the conditional processing.
Otherwise the page numbers may take divergent values
depending on which part is compiled.

For example, a title page could be declared by:
%
\begin{center}
\begin{tabular}{l}
|\ifchilddoc\||else|\\
|\addtocounter{page}{-1}|\\
\textit{code for title page}\\
|\newpage|\\
|\||fi|
\end{tabular}
\end{center}
%
A banner page for the child documents can be generated by:
%
\begin{center}
\begin{tabular}{l}
|\ifchilddoc|\\
|\addtocounter{page}{-1}|\\
\textit{code for banner page}\\
|\newpage|\\
|\||fi|
\end{tabular}
\end{center}
%
Here one could write a message such as:
\begin{center}
|This is the part \childdocname{} of \childdocjob{}.|
\end{center}

%%%%%%%%%%%%%%%%%%%%%%%%%%%%%%%%%%%%%%%%%%%%%%%%%%%%%%%%%%%%%%%%%%%%%%%%%%%%%%%%
\subsection{Flags}
\label{sec:flags}

The package makes it easy to generate different versions
of the main or child documents.
To this end compilation flags can be defined
and assigned different default values.
They will be particularly useful in conjunction
with the forwarding mechanism described in \secref{sec:forward}.

For example, it may be useful to have a flag |\version|
which can be set to |draft| or |final|.
The document source will contain some conditional code
depending on the value of |\version|.
Suppose further, the flag should default to |final| for the main file
and to |draft| for child files
which is a natural assignment for editing the document.
This is achieved by placing the following code
in the preamble of the main document
(below the |\childdocmain| directive):
%
\begin{center}
\begin{tabular}{l}
|\ifchilddoc|\\
|\providecommand{\version}{draft}|\\
|\||else|\\
|\providecommand{\version}{final}|\\
|\||fi|
\end{tabular}
\end{center}
%
The definition by |\providecommand| makes sure
that previous definitions are not overwritten.
Further statements |\providecommand{\version}{...}|
can thus be added before the above code to override it.

For the main file, one might add a line
(between |\childdocmain| and the above block)
%
\begin{center}
|%\ifchilddoc\||else\providecommand{\version}{draft}\||fi|
\end{center}
%
which can be uncommented to produce a draft version.
Likewise one can add a line to the very top of a child file
(above the |\childdocof{|\textit{main}|}| directive)
%
\begin{center}
|%\providecommand{\version}{final}|
\end{center}
%
which can be uncommented to produce the final version of this child document.

%%%%%%%%%%%%%%%%%%%%%%%%%%%%%%%%%%%%%%%%%%%%%%%%%%%%%%%%%%%%%%%%%%%%%%%%%%%%%%%%
\subsection{Forwarding}
\label{sec:forward}

Different versions of the main or child documents
using compilation flags as described in \secref{sec:flags}
can be (permanently) stored in different files
for convenient compilation, viewing and distribution.
To this end, the package defines a command
to pass on compilation to a different file:

%%%%%%%%%%%%%%%%%%%%%%%%%%%%%%%%%%%%%%%%
\DescribeMacro{\childdocforward}
The command |\childdocforward| redirects processing to
another source file:
%
\begin{center}
\begin{tabular}{l}
|\input{childdoc.def}|\\
|\childdocforward[|\textit{main}|]{|\textit{dest}|}|\\
\end{tabular}
\end{center}
%
The argument \textit{dest} is the destination file
(without extension).
It should be the main file or one of the child files.
Note that further \textsf{childdoc} directives
such as |\childdocof| and |\childdocforward|
in the indicated file will be processed in this form.
The optional argument \textit{main}
passes on directly to the main file \textit{main}
while pretending to compile the child \textit{dest}.
This form behaves as if \textit{dest}
issues |\childdocof{|\textit{main}|}| right away,
and no further \textsf{childdoc} directives will be processed.

%%%%%%%%%%%%%%%%%%%%%%%%%%%%%%%%%%%%%%%%
\DescribeMacro{\...prefix}
In the alternative form |\childdocforwardprefix|,
%
\begin{center}
\begin{tabular}{l}
|\input{childdoc.def}|\\
|\childdocforwardprefix[|\textit{main}|]{|\textit{prefix}|}{|\textit{dest}|}|
\end{tabular}
\end{center}
%
the destination file is determined by a pattern
depending on the current file:
To make this work, the current file must be called
`{\textit{prefix}\hspace{0.2em}\textit{suffix}}'
with \textit{prefix} matching precisely the argument.
Processing is then passed on to the file
`{\textit{dest}\hspace{0.2em}\textit{suffix}}'.
Surely, the same effect is achieved by
directly specifying the
argument `{\textit{dest}\hspace{0.2em}\textit{suffix}}'
in the first form.
However, that requires to set up a different file
for each child. With the alternative form of the command
all these files can have exactly the same content
which simplifies setting them up and maintaining them.

For example, the following file |draft.tex|
with a compilation flag |\version| as described in \secref{sec:flags}
compiles the main document as a draft:
%
\begin{center}
\begin{tabular}{l}
|\def\version{draft}|\\
|\input{childdoc.def}|\\
|\childdocforward{|\textit{main}|}|
\end{tabular}
\end{center}
%
Likewise, the following files |final|\textit{nn}|.tex|
compile the final version of the child document
|child|\textit{nn}|.tex|:
%
\begin{center}
\begin{tabular}{l}
|\def\version{final}|\\
|\input{childdoc.def}|\\
|\childdocforwardprefix{final}{child}|
\end{tabular}
\end{center}
%

Note that when several versions of a main file and/or of each child file
are to be generated, it may be convenient to set up a |Makefile| or
shell script to automatise the process.

%%%%%%%%%%%%%%%%%%%%%%%%%%%%%%%%%%%%%%%%%%%%%%%%%%%%%%%%%%%%%%%%%%%%%%%%%%%%%%%%
\subsection{Command Line Processing}
\label{sec:commandline}

The effect of redirection files can also be achieved by invoking
the \LaTeX{} compiler with a more elaborate command line.
Most conveniently this should be done as part
of a shell script or a |Makefile|.

When using \textsf{childdoc} in the main file, the following
command lines effectively perform a redirection
(note that depending on the shell being used,
backslashes may have to be doubled: `|\|' $\to$ `|\\|'):
%
\begin{center}
|... -jobname "|\textit{target}|" |\\|"|[\textit{flags}]%
|\input{childdoc.def}\childdocforward[|\textit{main}|]{|\textit{dest}|}"|
\end{center}
%
Here \textit{target} is the name of the output file,
\textit{main} is the name of the main file
and \textit{dest} is the name of the main or child file to be processed
(all filenames without extensions).
The optional argument \textit{main} can be omitted
if \textit{main} matches \textit{dest}.
Optionally, compilation \textit{flags} can be defined via |\def| commands.
This command line makes the \TeX{} engine believe
it is compiling the file \textit{target}
whose content is specified as the latter parameter.
The provided code then forwards the processing to
\textit{main} or \textit{dest} as described in \secref{sec:forward}.

%%%%%%%%%%%%%%%%%%%%%%%%%%%%%%%%%%%%%%%%%%%%%%%%%%%%%%%%%%%%%%%%%%%%%%%%%%%%%%%%
\subsection{Include by Input}
\label{sec:input}

Including child documents by |\include| has some restrictions by design.
Most notably, the content of a child document always occupies
its own set of pages; pages cannot be shared between child documents.
Usually, this behaviour makes perfect sense
because each child document contain an essential part of the document.
However, in some situations it may be desirable to compose
a document from a collection of parts
without having mandatory page breaks between then.
For this case, the package
provides a mechanism to include parts
by |\input| which can also be processed individually.
However, by construction this mechanism
requires manual handling of the content to be output.

%%%%%%%%%%%%%%%%%%%%%%%%%%%%%%%%%%%%%%%%
\DescribeMacro{\ifchilddocmanual}
The main file should be prepared as usual, see \secref{sec:include}.
However, the document body must make a distinction
between processing of an individual part and of the main document, e.g.:
%
\begin{center}
\begin{tabular}{l}
|\ifchilddocmanual|\\
|\input{\childdocname}|\\
|\||else|\\
\textit{document body with }|\input{|\textit{part}|}|\\
|\||fi|
\end{tabular}
\end{center}
%
The conditional |\ifchilddocmanual| is true whenever
a part to be included by |\input| is being compiled,
and the name of the part is stored in |\childdocname|.

%%%%%%%%%%%%%%%%%%%%%%%%%%%%%%%%%%%%%%%%
\DescribeMacro{\childdocby}
Each part to be included by |\input| should start with:
%
\begin{center}
\begin{tabular}{l}
|\input{childdoc.def}|\\
|\childdocby{|\textit{main}|}|\\
\end{tabular}
\end{center}
%
The directive |\childdocby| is similar to |\childdocof|
described in \secref{sec:include},
but the subsequent selection of content must be done manually.
To that end, both |\ifchilddoc| and |\ifchilddocmanual|
will be true upon processing of a part,
and the name of the part is stored in |\childdocname|.
Note that |\jobname| will be set to the filename of the current part
so that each part receives an individual |.aux| file
that does not interfere with the |.aux| file(s) of the main document.
This behaviour can be altered by the alternative form
|\childdocby[*]{|\textit{main}|}| (with a non-empty optional argument)
which uses the |.aux| file of the main document
by setting |\jobname| to \textit{main}.

%%%%%%%%%%%%%%%%%%%%%%%%%%%%%%%%%%%%%%%%%%%%%%%%%%%%%%%%%%%%%%%%%%%%%%%%%%%%%%%%
\subsection{Driver Development}
\label{sec:driver}

The \textsf{childdoc} mechanism can also be use for the development
of definition files such as \LaTeX{} styles or classes.
This case differs from the above setup with multiple parts
included by |\include| in that no |\includeonly| should be invoked.
This can be achieved by starting the include file
(before |\ProvidesPackage|) with:
%
\begin{center}
\begin{tabular}{l}
|\input{childdoc.def}|\\
|\childdocforward{|\textit{main}|}|\\
\end{tabular}
\end{center}
%
or alternatively with:
%
\begin{center}
\begin{tabular}{l}
|\input{childdoc.def}|\\
|\childdocby{|\textit{main}|}|\\
\end{tabular}
\end{center}
%
Both forms have slightly different effects as described above.
The main file is prepared as usual, see \secref{sec:include}.

%%%%%%%%%%%%%%%%%%%%%%%%%%%%%%%%%%%%%%%%%%%%%%%%%%%%%%%%%%%%%%%%%%%%%%%%%%%%%%%%
\subsection{Legacy Detection}
\label{sec:detection}

The directive |\childdocmain| in the main file can detect
whether the complete document or merely a child is to be compiled
even without using the directive |\childdocof|.
This method is deprecated because it is less robust
and there is no compelling reason to use it;
it is merely provided for backward compatibility
and it may be removed in future versions.

If the detection mechanism is to be used,
it is mandatory to correctly specify
the filename of the main file as the argument of |\childdocmain|:
%
\begin{center}
\begin{tabular}{l}
|\input{childdoc.def}|\\
|\childdocmain{|\textit{main}|}|\\
\end{tabular}
\end{center}
%
If |\jobname| does not match the argument \textit{main} of |\childdocmain|,
it is assumed that |\jobname| points to the child file to be compiled.
When using |\childdocmain| with the main file specified as argument,
it suffices to start a child file
with just |\input{|\textit{main}|}|
without loading of the package and using |\childdocof|.
If instead all processing is done
with the appropriate \textsf{childdoc} directives,
the argument of \textit{main} of |\childdocmain| can be empty.

An alternative version of the command line processing described
in \secref{sec:commandline} using the detection mechanism reads:
%
\begin{center}
|... -jobname "|\textit{target}|" "|[\textit{flags}]%
[|\def\jobname{|\textit{dest}|}|]|\input{|\textit{main}|}"|
\end{center}

%%%%%%%%%%%%%%%%%%%%%%%%%%%%%%%%%%%%%%%%%%%%%%%%%%%%%%%%%%%%%%%%%%%%%%%%%%%%%%%%
\subsection{Manual Code}
\label{sec:manual}

In case one cannot be certain whether the definitions file |childdoc.def|
is installed on the target \TeX{} distribution
and one prefers not to ship it,
it is conceivable to paste a few relevant commands into the sources.

To that end, drop all statements |\input{childdoc.def}|
and perform the replacements as outlined below.
Instead of |\childdocmain{|\textit{main}|}| add the following code
to the top of the main file:
%
\begin{center}
\begin{tabular}{l}
|\||ifdefined\childdocname\endinput\||fi\newif\ifchilddoc|\\
|\edef\childdocname{\scantokens\expandafter{\jobname\noexpand}}|\\
|\def\childdocmain{|\textit{main}|}\||ifx\childdocmain\childdocname\||else|\\
|\childdoctrue\includeonly{\childdocname}\let\jobname\childdocmain\||fi|\\
\end{tabular}
\end{center}
%
Instead of |\childdocof{|\textit{main}|}| just include the main file
at the top of each child file:
%
\begin{center}
|\input{|\textit{main}|}|
\end{center}
%
A simple redirection |\childdocforward{|\textit{dest}|}| is achieved by:
%
\begin{center}
|\def\jobname{|\textit{dest}|}\input{\jobname}|
\end{center}
%
The redirection with prefix
|\childdocforwardprefix[|\textit{prefix}|]{|\textit{dest}|}|
is accomplished by:
%
\begin{center}
\begin{tabular}{l}
|{\edef\jobname{\scantokens\expandafter{\jobname\noexpand}}|\\
|\def\redirectjob |\textit{prefix}|#1~~~{\gdef\jobname{|\textit{dest}|#1}}|\\
|\expandafter\redirectjob\jobname~~~}\input{\jobname}|
\end{tabular}
\end{center}

In an alternative approach,
child documents can be compiled by a specific command line
without additional code or specific definitions:
%
\begin{center}
|... -jobname "|\textit{target}|" "|[\textit{flags}]%
|\includeonly{|\textit{dest}|}\input{|\textit{main}|}"|
\end{center}
%

%%%%%%%%%%%%%%%%%%%%%%%%%%%%%%%%%%%%%%%%%%%%%%%%%%%%%%%%%%%%%%%%%%%%%%%%%%%%%%%%
%%%%%%%%%%%%%%%%%%%%%%%%%%%%%%%%%%%%%%%%%%%%%%%%%%%%%%%%%%%%%%%%%%%%%%%%%%%%%%%%
\section{Information}

%%%%%%%%%%%%%%%%%%%%%%%%%%%%%%%%%%%%%%%%%%%%%%%%%%%%%%%%%%%%%%%%%%%%%%%%%%%%%%%%
\subsection{Copyright}

Copyright \copyright{} 2017--2018 Niklas Beisert

This work may be distributed and/or modified under the
conditions of the \LaTeX{} Project Public License, either version 1.3
of this license or (at your option) any later version.
The latest version of this license is in
  \url{http://www.latex-project.org/lppl.txt}
and version 1.3 or later is part of all distributions of \LaTeX{}
version 2005/12/01 or later.

This work has the LPPL maintenance status `maintained'.

The Current Maintainer of this work is Niklas Beisert.

This work consists of the files |README.txt|, |childdoc.ins| and |childdoc.dtx|
as well as the derived files |childdoc.def|, |cdocsamp.tex|
with |cdocsch1.tex|, |cdocsch2.tex|, |cdocspt3.tex|, |cdocspt4.tex|,
|cdocsdrf.tex|, |cdocsfn1.tex|, |cdocsfn2.tex|
as well as |childdoc.pdf|.

%%%%%%%%%%%%%%%%%%%%%%%%%%%%%%%%%%%%%%%%%%%%%%%%%%%%%%%%%%%%%%%%%%%%%%%%%%%%%%%%
\subsection{Files and Installation}

The package consists of the files:
%
\begin{center}
\begin{tabular}{ll}
    |README.txt|   & readme file \\
    |childdoc.ins| & installation file \\
    |childdoc.dtx| & source file \\
    |childdoc.def| & definition file \\
    |cdocsamp.tex| & sample main file \\
    |cdocsch1.tex| & sample include file \\
    |cdocsch2.tex| & sample include file \\
    |cdocspt3.tex| & sample part file \\
    |cdocspt4.tex| & sample part file \\
    |cdocsdrf.tex| & sample redirection file \\
    |cdocsfn1.tex| & sample redirection file \\
    |cdocsfn2.tex| & sample redirection file \\
    |childdoc.pdf| & manual
\end{tabular}
\end{center}
%
The distribution consists of the files
|README.txt|, |childdoc.ins| and |childdoc.dtx|.
%
\begin{itemize}
\item
Run (pdf)\LaTeX{} on |childdoc.dtx|
to compile the manual |childdoc.pdf| (this file).
\item
Run \LaTeX{} on |childdoc.ins| to create the definitions file |childdoc.def|
and the sample |cdocsamp.tex| with include files
|cdocsch1.tex|, |cdocsch2.tex|, |cdocspt3.tex|, |cdocspt4.tex|,
|cdocsdrf.tex|, |cdocsfn1.tex|, |cdocsfn2.tex|.
Then copy the file |childdoc.def| to an appropriate directory of your \LaTeX{}
distribution, e.g.\ \textit{texmf-root}|/tex/latex/childdoc|.
\end{itemize}

%%%%%%%%%%%%%%%%%%%%%%%%%%%%%%%%%%%%%%%%%%%%%%%%%%%%%%%%%%%%%%%%%%%%%%%%%%%%%%%%
\subsection{Related CTAN Packages}

There are several other packages which offer a similar functionality:
%
\begin{itemize}
\item
The packages
\href{http://ctan.org/pkg/docmute}{\textsf{docmute}},
\href{http://ctan.org/pkg/includex}{\textsf{includex}} and
\href{http://ctan.org/pkg/standalone}{\textsf{standalone}}
provide commands to include only the document body of
a child file thus allowing both files to be compiled individually.
\item
The packages \href{http://ctan.org/pkg/subdocs}{\textsf{subdocs}}
and \href{http://ctan.org/pkg/subfiles}{\textsf{subfiles}}
provide structures in which the main and child documents can be
encapsulated and allowing them to be compiled individually.
The inclusion mechanism is different from the conventional |\include|.
\item
The package \href{http://ctan.org/pkg/combine}{\textsf{combine}}
is an elaborate solution to combine several documents into one.
\end{itemize}
%
See also the CTAN topic \href{http://ctan.org/topic/subdocs}{\textsf{subdocs}}
for further related packages.
The present package differs from the above solutions in that
a document structure constructed with the conventional |\include| mechanism
just needs two extra commands at the top of every file
such that all constituent files can be compiled individually.

%%%%%%%%%%%%%%%%%%%%%%%%%%%%%%%%%%%%%%%%%%%%%%%%%%%%%%%%%%%%%%%%%%%%%%%%%%%%%%%%
%\subsection{Feature Suggestions}
%
%The following is a list of features which may be useful for future
%versions of this package:
%%
%\begin{itemize}
%\item
%\ldots
%\end{itemize}

%%%%%%%%%%%%%%%%%%%%%%%%%%%%%%%%%%%%%%%%%%%%%%%%%%%%%%%%%%%%%%%%%%%%%%%%%%%%%%%%
\subsection{Revision History}

%%%%%%%%%%%%%%%%%%%%%%%%%%%%%%%%%%%%%%%%
\paragraph{v2.0:} 2018/12/30

\begin{itemize}
\item
immediate forward processing
\item
added |\childdocby| mechanism
\item
manual restructured
\end{itemize}

%%%%%%%%%%%%%%%%%%%%%%%%%%%%%%%%%%%%%%%%
\paragraph{v1.6:} 2018/01/17

\begin{itemize}
\item
application for development of include files
\item
corrections to manual
\end{itemize}

%%%%%%%%%%%%%%%%%%%%%%%%%%%%%%%%%%%%%%%%
\paragraph{v1.5:} 2017/05/21

\begin{itemize}
\item
more complete structuring introduced
\item
|\childdocof| introduced
\item
|\childdoc| renamed to |\childdocmain|
\item
|\childredirect| renamed to |\childdocforward| and |\childdocforwardprefix|
and functionality expanded
\end{itemize}

%%%%%%%%%%%%%%%%%%%%%%%%%%%%%%%%%%%%%%%%
\paragraph{v1.0:} 2017/04/27

\begin{itemize}
\item
manual and install package
\item
first version published on CTAN
\end{itemize}

%%%%%%%%%%%%%%%%%%%%%%%%%%%%%%%%%%%%%%%%
\paragraph{v0.6:} 2017/04/26

\begin{itemize}
\item
redirection mechanism added
\end{itemize}

%%%%%%%%%%%%%%%%%%%%%%%%%%%%%%%%%%%%%%%%
\paragraph{v0.5:} 2017/04/26

\begin{itemize}
\item
functionality in definition file
\end{itemize}


%%%%%%%%%%%%%%%%%%%%%%%%%%%%%%%%%%%%%%%%%%%%%%%%%%%%%%%%%%%%%%%%%%%%%%%%%%%%%%%%
%%%%%%%%%%%%%%%%%%%%%%%%%%%%%%%%%%%%%%%%%%%%%%%%%%%%%%%%%%%%%%%%%%%%%%%%%%%%%%%%
%%%%%%%%%%%%%%%%%%%%%%%%%%%%%%%%%%%%%%%%%%%%%%%%%%%%%%%%%%%%%%%%%%%%%%%%%%%%%%%%
\appendix

\settowidth\MacroIndent{\rmfamily\scriptsize 000\ }

 \DocInput{childdoc.dtx}

\end{document}
%</driver>
% \fi
%
% %%%%%%%%%%%%%%%%%%%%%%%%%%%%%%%%%%%%%%%%%%%%%%%%%%%%%%%%%%%%%%%%%%%%%%%%%%%%%%
% %%%%%%%%%%%%%%%%%%%%%%%%%%%%%%%%%%%%%%%%%%%%%%%%%%%%%%%%%%%%%%%%%%%%%%%%%%%%%%
% \section{Sample}
%\iffalse
%<*samplemain>
%\fi
%
% The following presents a sample document
% with two chapters, two parts, a title page,
% a compile flag as well as three forwarding files to set the flag.
% It consists of eight |.tex| files:
% \begin{center}
% \begin{tabular}{ll}
% |cdocsamp.tex|&main file\\
% |cdocsch1.tex|&include file for chapter 1\\
% |cdocsch2.tex|&include file for chapter 2\\
% |cdocspt3.tex|&include file for part 3\\
% |cdocspt4.tex|&include file for part 4\\
% |cdocsdrf.tex|&forwarding file for main file in draft mode\\
% |cdocsfi1.tex|&forwarding file for final version of chapter 1\\
% |cdocsfi2.tex|&forwarding file for final version of chapter 2\\
% \end{tabular}
% \end{center}
% Each of the eight files can be compiled directly by the \LaTeX{} compiler.
%
% %%%%%%%%%%%%%%%%%%%%%%%%%%%%%%%%%%%%%%
% \paragraph{Main File.}
%
% The main file is called |cdocsamp.tex|.
%
% Load the \textsf{childdoc} definitions and
% declare the filename for the main document:
%    \begin{macrocode}
\input{childdoc.def}
\childdocmain{}
%    \end{macrocode}

% Optional override for |\version| flag:
%    \begin{macrocode}
%%\ifchilddoc\else\providecommand{\version}{draft}\fi
%    \end{macrocode}

% Define the default values for the |\version| flag
% (|final| for the main file and |draft| for childs):
%    \begin{macrocode}
\ifchilddoc
\providecommand{\version}{draft}
\else
\providecommand{\version}{final}
\fi
%    \end{macrocode}

% Load the standard document class:
%    \begin{macrocode}
\documentclass[12pt]{article}
%    \end{macrocode}

% Start the document body:
%    \begin{macrocode}
\begin{document}
%    \end{macrocode}

% Declare a title page.
% Print title, part of document being processed and version flag:
%    \begin{macrocode}
\addtocounter{page}{-1}
\begin{center}
{\LARGE\bfseries{}childdoc example\par}
\vspace{1cm}
\ifchilddoc
\ifchilddocmanual part\else chapter\fi:
`\childdocname' of `\childdocjob'\par
\else
main document: `\childdocjob'\par
\fi
version: \version\par
\end{center}
\newpage
%    \end{macrocode}

% Manually include selected file,
% otherwise process as usual:
%    \begin{macrocode}
\ifchilddocmanual
\section*{part `\childdocname'}
\input{\childdocname}
\else
%    \end{macrocode}

% Include the two chapters:
%    \begin{macrocode}
\include{cdocsch1}
\include{cdocsch2}
%    \end{macrocode}

% Include the two parts unless only chapters should be displayed:
%    \begin{macrocode}
\ifchilddoc\else
\section{part three}
\input{cdocspt3}
\section{part four}
\input{cdocspt4}
\fi
%    \end{macrocode}

% Process as usual until here:
%    \begin{macrocode}
\fi
%    \end{macrocode}

% End of document body:
%    \begin{macrocode}
\end{document}
%    \end{macrocode}
%\iffalse
%</samplemain>
%\fi
%
% %%%%%%%%%%%%%%%%%%%%%%%%%%%%%%%%%%%%%%
% \paragraph{Chapter Include Files.}
%
% The include files are called |cdocsch1.tex| and |cdocsch2.tex|.
%
%\iffalse
%<*samplechap1|samplechap2>
%\fi

% Optional override for |\version| flag:
%    \begin{macrocode}
%%\providecommand{\version}{final}
%    \end{macrocode}

% Include the main document:
%    \begin{macrocode}
\input{childdoc.def}
\childdocof{cdocsamp}
%    \end{macrocode}

%\iffalse
%</samplechap1|samplechap2>
%\fi
%
%\iffalse
%<*samplechap1>
%\fi
% Some text for chapter 1:
%    \begin{macrocode}
\section{one}
some text in chapter one
%    \end{macrocode}

%\iffalse
%</samplechap1>
%\fi
% Some text for chapter 2:
%\iffalse
%<*samplechap2>
%\fi
%    \begin{macrocode}
\section{two}
more text in chapter two
%    \end{macrocode}

%\iffalse
%</samplechap2>
%\fi
%
% %%%%%%%%%%%%%%%%%%%%%%%%%%%%%%%%%%%%%%
% \paragraph{Part Include Files.}
%
% The include files are called |cdocspt3.tex| and |cdocspt4.tex|.
%
%\iffalse
%<*samplepart3|samplepart4>
%\fi

% Optional override for |\version| flag:
%    \begin{macrocode}
%%\providecommand{\version}{final}
%    \end{macrocode}

% Include the main document:
%    \begin{macrocode}
\input{childdoc.def}
\childdocby{cdocsamp}
%    \end{macrocode}

%\iffalse
%</samplepart3|samplepart4>
%\fi
%
%\iffalse
%<*samplepart3>
%\fi
% Some text for part 3:
%    \begin{macrocode}
some text in part three
%    \end{macrocode}

%\iffalse
%</samplepart3>
%\fi
% Some text for part 4:
%\iffalse
%<*samplepart4>
%\fi
%    \begin{macrocode}
more text in part four
%    \end{macrocode}

%\iffalse
%</samplepart4>
%\fi
%
% %%%%%%%%%%%%%%%%%%%%%%%%%%%%%%%%%%%%%%
% \paragraph{Forwarding for a Complete Draft.}
%
% The following forwarding file |cdocsdrf.tex|
% compiles the main document in draft mode:
%\iffalse
%<*sampledraft>
%\fi
%    \begin{macrocode}
\def\version{draft}
\input{childdoc.def}
\childdocforward{cdocsamp}
%    \end{macrocode}

%\iffalse
%</sampledraft>
%\fi
%
% %%%%%%%%%%%%%%%%%%%%%%%%%%%%%%%%%%%%%%
% \paragraph{Forwarding for Final Version of the Chapters.}
%
% The following forwarding files |cdocsfn1.tex| and |cdocsfn2.tex|
% (with identical content)
% compile the final versions of the child documents
% |cdocsch1.tex| and |cdocsch2.tex|, respectively:
%\iffalse
%<*samplefinal>
%\fi
%    \begin{macrocode}
\def\version{final}
\input{childdoc.def}
\childdocforwardprefix[cdocsamp]{cdocsfn}{cdocsch}
%    \end{macrocode}

%\iffalse
%</samplefinal>
%\fi
%
% %%%%%%%%%%%%%%%%%%%%%%%%%%%%%%%%%%%%%%
% \paragraph{Command Line Processing.}
%
% The following three command lines generate the output files
% |cdocscld|, |cdocscl1| and |cdocscl2|
% which should be identical to
% |cdocsdrf|, |cdocsch1| and |cdocsfn2|, respectively:
% \begin{center}
% \begin{tabular}{l}
% |latex -jobname cdocscld \|\\
% |  "\def\version{draft}\input{childdoc.def}\childdocforward{cdocsamp}"|\\
% |latex -jobname cdocscl1 \|\\
% |  "\input{childdoc.def}\childdocforward[cdocsamp]{cdocsch1}"|\\
% |latex -jobname cdocscl2 \|\\
% |  "\def\version{final}\input{childdoc.def}\childdocforward{cdocsch2}"|
% \end{tabular}
% \end{center}
% Note that the trailing backslash on each first line
% merely continues the input to the second line
% (for convenient cut ant paste).
% Furthermore, the command |latex| can be replaced by any
% of its alternative versions such as |pdflatex|.
%
% %%%%%%%%%%%%%%%%%%%%%%%%%%%%%%%%%%%%%%%%%%%%%%%%%%%%%%%%%%%%%%%%%%%%%%%%%%%%%%
% %%%%%%%%%%%%%%%%%%%%%%%%%%%%%%%%%%%%%%%%%%%%%%%%%%%%%%%%%%%%%%%%%%%%%%%%%%%%%%
% \section{Implementation}
%\iffalse
%<*package>
%\fi
%
% This section describes the definitions file |childdoc.def|.

% The definitions cannot be loaded using |\usepackage| or |\RequirePackage|
% which has a mechanism to prevent loading a style file more than once.
% When loading the definitions by means of |\input|
% multiple instances have to be prevented manually:
%\iffalse
%This code needs to be before the `\ProvidesFile' directive
%which is defined at the beginning of this file.
%Therefore it is also placed there and commented out here.
%</package>
%<*discard>
%\fi
%    \begin{macrocode}
\ifdefined\childdocmain\endinput\fi
%    \end{macrocode}
%\iffalse
%</discard>
%<*package>
%\fi
%
% \macro{\ifchilddoc}
% \macro{\ifchilddocmanual}
% The conditional |\ifchilddoc| tells whether a
% child (true) or main (false) document is being compiled.
% The conditional |\ifchilddocmanual| tells whether
% the |\includeonly| mechanism is used (false) or
% the selection of child files must be performed manually (true).
% The definitions initialise to false:
%    \begin{macrocode}
\newif\ifchilddoc
\newif\ifchilddocmanual
%    \end{macrocode}

% \macro{\childdocname}
% \macro{\childdocjob}
% The macro |\childdocname| stores the name of the main document
% to be compiled. The macro |\childdocjob| stores the name of
% the document on which the \LaTeX{} compiler was originally invoked.
% The content of |\jobname| cannot be compared
% to filenames specified in the source due to different catcodes.
% The following code rescans |\jobname|, stores the result
% in |\childdocname| and saves a copy in |\childdocjob|:
%    \begin{macrocode}
\edef\childdocname{\scantokens\expandafter{\jobname\noexpand}}
\let\childdocjob\childdocname
%    \end{macrocode}

% \macro{\childdocdisable}
% The macro |\childdocdisable| prevents the main file
% from being processed more than once.
% At this stage, the main document command |\childdocmain|
% is assumed to be called once again where it should do nothing.
% Any subsequent call to it should prevent
% a secondary processing of the main document
% It overwrites the forwarding commands
% |\childdocof| and |\childdocforward|
% with empty macros to prevent further inclusions of the main document:
%    \begin{macrocode}
\newcommand{\childdocdisable}
{
  \renewcommand{\childdocmain}[1]{\renewcommand{\childdocmain}[1]{\endinput}}
  \renewcommand{\childdocof}[1]{}
  \renewcommand{\childdocby}[2][]{}
  \renewcommand{\childdocforward}[2][]{}
  \renewcommand{\childdocdisable}{}
}
%    \end{macrocode}

% \macro{\childdocmain}
% The macro |\childdocmain| is to be called at the top of the main file
% with nothing or the main filename (without extension) as argument.
% First, it breaks loops.
% If the argument is not empty and does not match |\childdocname|
% (which is set by the first inclusion of |childdoc.def|),
% |\ifchilddoc| is set to true, |\includeonly| is applied to the child file
% and |\jobname| is set to the main file
% (for proper handling of |.aux| files):
%    \begin{macrocode}
\newcommand{\childdocmain}[1]
{
  \childdocdisable\childdocmain{}
  \if?#1?\else
    \begingroup
      \def\childdoctmp{#1}
      \ifx\childdoctmp\childdocname
        \def\childdoctmp{}
      \else
        \def\childdoctmp
        {
          \childdoctrue
          \includeonly{\childdocname}
          \def\childdocjob{#1}
          \def\jobname{#1}
        }
      \fi
      \expandafter
    \endgroup
    \childdoctmp
  \fi
}
%    \end{macrocode}

% \macro{\childdocof}
% The command |\childdocof| redirects
% compilation to the main file |#1|.
%    \begin{macrocode}
\newcommand{\childdocof}[1]
{
  \childdocdisable
  \childdoctrue
  \includeonly{\childdocname}
  \def\jobname{#1}
  \def\childdocjob{#1}
  \input{#1}
}
%    \end{macrocode}

% \macro{\childdocby}
% The command |\childdocby| ....
%    \begin{macrocode}
\newcommand{\childdocby}[2][]
{
  \childdocdisable
  \childdoctrue
  \childdocmanualtrue
  \if?#1?\else
    \def\jobname{#2}
  \fi
  \def\childdocjob{#2}
  \input{#2}
  \endinput
}
%    \end{macrocode}

% \macro{\childdocforward}
% The command |\childdocforward| redirects
% compilation to the main file or
% (if the optional argument is given) a child file.
% Parameters are set as if the main file
% or a child file starting with |\childdocof| was compiled.
% Then compilation is handed over to the main file:
%    \begin{macrocode}
\newcommand{\childdocforward}[2][]
{
  \begingroup
    \if?#1?
      \def\childdoctmp
      {
        \def\childdocname{#2}
        \def\childdocjob{#2}
        \def\jobname{#2}
        \input{#2}
        \endinput
      }
    \else
      \def\childdoctmp
      {
        \childdocdisable
        \def\childdocname{#2}
        \childdoctrue
        \includeonly{#2}
        \def\childdocjob{#1}
        \def\jobname{#1}
        \input{#1}
        \endinput
      }
    \fi
    \expandafter
  \endgroup
  \childdoctmp
}
%    \end{macrocode}

% \macro{\childdocforwardprefix}
% The command |\childdocforwardprefix| redirects
% compilation to the main or a child file by means of a pattern.
% The prefix |#1| in the current filename is replaced by |#2|
% and the suffix of the current filename is kept
% (it is assumed that the filename does not contain the substring `|~~~|'
% which is used as a delimiter).
% Compilation is handed over to the new file by |\childdocforward|:
%    \begin{macrocode}
\newcommand{\childdocforwardprefix}[3][]
{
  \begingroup
    \def\childdocextract #2##1~~~{\def\childdoctmp{\childdocforward[#1]{#3##1}}}
    \expandafter\childdocextract\childdocname~~~
    \expandafter
  \endgroup
  \childdoctmp
}
%    \end{macrocode}

% \macro{\childdoc}
% The deprecated macro |\childdoc| is a legacy version of |\childdocmain|:
%    \begin{macrocode}
\newcommand{\childdoc}{\childdocmain}
%    \end{macrocode}

% \macro{\childdocredirect}
% The deprecated macro |\childdocredirect| is a legacy version
% of |\childdocforward| and |\childdocforwardprefix|:
%    \begin{macrocode}
\newcommand{\childdocredirect}[2][]
{
  \begingroup
    \if?#1?
      \def\childdoctmp{\childdocforward{#2}}
    \else
      \def\childdoctmp{\childdocforwardprefix{#1}{#2}}
    \fi
    \expandafter
  \endgroup
  \childdoctmp
}
%    \end{macrocode}

%\iffalse
%</package>
%\fi
%
\endinput

\childdocby{cdocsamp}
%    \end{macrocode}

%\iffalse
%</samplepart3|samplepart4>
%\fi
%
%\iffalse
%<*samplepart3>
%\fi
% Some text for part 3:
%    \begin{macrocode}
some text in part three
%    \end{macrocode}

%\iffalse
%</samplepart3>
%\fi
% Some text for part 4:
%\iffalse
%<*samplepart4>
%\fi
%    \begin{macrocode}
more text in part four
%    \end{macrocode}

%\iffalse
%</samplepart4>
%\fi
%
% %%%%%%%%%%%%%%%%%%%%%%%%%%%%%%%%%%%%%%
% \paragraph{Forwarding for a Complete Draft.}
%
% The following forwarding file |cdocsdrf.tex|
% compiles the main document in draft mode:
%\iffalse
%<*sampledraft>
%\fi
%    \begin{macrocode}
\def\version{draft}
% \iffalse
%
% childdoc.dtx Copyright (C) 2017-2018 Niklas Beisert
%
% This work may be distributed and/or modified under the
% conditions of the LaTeX Project Public License, either version 1.3
% of this license or (at your option) any later version.
% The latest version of this license is in
%   http://www.latex-project.org/lppl.txt
% and version 1.3 or later is part of all distributions of LaTeX
% version 2005/12/01 or later.
%
% This work has the LPPL maintenance status `maintained'.
%
% The Current Maintainer of this work is Niklas Beisert.
%
% This work consists of the files childdoc.dtx and childdoc.ins
% and the derived files childdoc.def and cdocsamp.tex with
% cdocsch1.tex, cdocsch2.tex, cdocsdrf.tex, cdocsfn1.tex, cdocsfn2.tex.
%
%<package>\ifdefined\childdocmain\endinput\fi
%<package>\ProvidesFile{childdoc.def}[2018/12/30 v2.0 child document driver]
%<samplemain>\ProvidesFile{cdocsamp.tex}[2018/12/30 v2.0 sample for childdoc]
%<*driver>
%\ProvidesFile{childdoc.drv}[2018/12/30 v2.0 childdoc reference manual file]
\PassOptionsToClass{10pt,a4paper}{article}
\documentclass{ltxdoc}

\usepackage[margin=35mm]{geometry}
\usepackage{hyperref}
\usepackage{hyperxmp}
\usepackage[usenames]{color}

\hypersetup{colorlinks=true}
\hypersetup{pdfstartview=FitH}
\hypersetup{pdfpagemode=UseNone}
\hypersetup{pdfsource={}}
\hypersetup{pdflang={en-UK}}
\hypersetup{pdfcopyright={Copyright 2017-2018 Niklas Beisert.
  This work may be distributed and/or modified under the
  conditions of the LaTeX Project Public License, either version 1.3
  of this license or (at your option) any later version.}}
\hypersetup{pdflicenseurl={http://www.latex-project.org/lppl.txt}}
\hypersetup{pdfcontactaddress={ETH Zurich, ITP, HIT K,
  Wolfgang-Pauli-Strasse 27}}
\hypersetup{pdfcontactpostcode={8093}}
\hypersetup{pdfcontactcity={Zurich}}
\hypersetup{pdfcontactcountry={Switzerland}}
\hypersetup{pdfcontactemail={nbeisert@itp.phys.ethz.ch}}
\hypersetup{pdfcontacturl={http://people.phys.ethz.ch/\xmptilde nbeisert/}}

\newcommand{\secref}[1]{\hyperref[#1]{section \ref*{#1}}}

\parskip1ex
\parindent0pt
\let\olditemize\itemize
\def\itemize{\olditemize\parskip0pt}

\begin{document}

\title{The \textsf{childdoc} Package}
\hypersetup{pdftitle={The childdoc Package}}
\author{Niklas Beisert\\[2ex]
  Institut f\"ur Theoretische Physik\\
  Eidgen\"ossische Technische Hochschule Z\"urich\\
  Wolfgang-Pauli-Strasse 27, 8093 Z\"urich, Switzerland\\[1ex]
  \href{mailto:nbeisert@itp.phys.ethz.ch}
  {\texttt{nbeisert@itp.phys.ethz.ch}}}
\hypersetup{pdfauthor={Niklas Beisert}}
\hypersetup{pdfsubject={Manual for the LaTeX2e Package childdoc}}
\date{30 December 2018, \textsf{v2.0}}
\maketitle

\begin{abstract}\noindent
\textsf{childdoc} is a \LaTeXe{} package
that enables the direct compilation
of document sections included by |\include|
to individual files.
\end{abstract}

\begingroup
\parskip0ex
\tableofcontents
\endgroup

%%%%%%%%%%%%%%%%%%%%%%%%%%%%%%%%%%%%%%%%%%%%%%%%%%%%%%%%%%%%%%%%%%%%%%%%%%%%%%%%
%%%%%%%%%%%%%%%%%%%%%%%%%%%%%%%%%%%%%%%%%%%%%%%%%%%%%%%%%%%%%%%%%%%%%%%%%%%%%%%%
\section{Introduction}

\LaTeX{} provides a mechanism to structure a large document (such as a book)
into a main file and several child files (containing the chapters)
using the |\include| command.
This mechanism is beneficial for documents
which span hundreds of pages in order to
make the source file(s) more manageable.
Moreover, compilation can be restricted to
selected child files by means of the |\includeonly| command.
The latter feature can be used to reduce the compilation time while editing
(this was significantly more useful in the earlier days of \LaTeX{})
or to generate a smaller document which is easier to navigate.
Another application of |\includeonly| is to generate
documents consisting of selected parts of the complete document.

However, there are a few drawbacks of the plain |\include| mechanism:
\begin{itemize}
\item
The child files cannot be compiled on their own,
they can only be compiled via the main file.
A naive editing environment
(such as a text editor with an option
to have the current file processed by \LaTeX)
may require one to switch to the main file before compiling;
attempting to compile the child file produces errors.
\item
The main file must be modified (each time)
to adjust the |\includeonly| command
to the present needs. This easily leaves the main file in a messy state.
\item
The generated document will always carry the filename
of the main document. This is inconvenient if
several child files are to be compiled and
to be kept for distribution.
\end{itemize}

The present package provides a simple interface
to make child files individually compilable by \LaTeX{}.
Compiling a child file then has the same effect as compiling
the main file with an |\includeonly| command
to select the appropriate child.
Moreover the generated document will carry the name of the child
rather than the main file.
This resolves all three above issues.

This feature is meant to make the editing of books,
thesis documents and lecture notes somewhat more convenient.
However, the package can also be used efficiently for
composing a series of documents (such as exercise sheets)
which are typically distributed individually.
It then assists the author in generating the individual documents
(potentially in different versions)
as well as a document containing the collected series.
Another application is in developing style files
or other kinds of included material
where compilation of the style file could redirect
to a sample or test file.

%%%%%%%%%%%%%%%%%%%%%%%%%%%%%%%%%%%%%%%%%%%%%%%%%%%%%%%%%%%%%%%%%%%%%%%%%%%%%%%%
%%%%%%%%%%%%%%%%%%%%%%%%%%%%%%%%%%%%%%%%%%%%%%%%%%%%%%%%%%%%%%%%%%%%%%%%%%%%%%%%
\section{Usage}

First of all, the package \textsf{childdoc} is \emph{not} a standard
\LaTeXe{} |.sty| style file! Therefore it needs to be invoked in
a non-standard way.

%%%%%%%%%%%%%%%%%%%%%%%%%%%%%%%%%%%%%%%%%%%%%%%%%%%%%%%%%%%%%%%%%%%%%%%%%%%%%%%%
\subsection{Included Files}
\label{sec:include}

%%%%%%%%%%%%%%%%%%%%%%%%%%%%%%%%%%%%%%%%
\DescribeMacro{\childdocmain}
To use the package, add the commands
\begin{center}
\begin{tabular}{l}
|\input{childdoc.def}|\\
|\childdocmain{}|\\
\end{tabular}
\end{center}
at the very top of the main \LaTeX{} file,
in particular \emph{before} the |\documentclass| statement!
The argument of |\childdocmain| should be left empty
(but it must be present).

%%%%%%%%%%%%%%%%%%%%%%%%%%%%%%%%%%%%%%%%
\DescribeMacro{\childdocof}
Furthermore, add the commands
\begin{center}
\begin{tabular}{l}
|\input{childdoc.def}|\\
|\childdocof{|\textit{main}|}|\\
\end{tabular}
\end{center}
at the top of every child file \textit{child}
which is included by |\include{|\textit{child}|}|
from within the main file
(or at least for those files to be compiled individually).
The argument \textit{main} must be the filename of the main file.

There are a couple of
considerations in setting up the main and child documents:

%%%%%%%%%%%%%%%%%%%%%%%%%%%%%%%%%%%%%%%%
\paragraph{Restrictions.}

Please note the following restrictions:
\begin{itemize}
\item
|\childdocmain| must be called with one argument \textit{main}
to ensure compatibility with earlier version of the package.
It must either be empty (|\childdocmain{}|)
or precisely match the filename of the main file in which it is specified.
See \secref{sec:detection} for further information.
\item
The filename \textit{main} must be specified without the |.tex| extension.
\item
The filename \textit{main} is case sensitive
(even in case-insensitive file systems)
due to internal string comparison.
\item
The argument \textit{main} should be fully expanded, it cannot be a macro.
\item
Subdirectories and special characters should be avoided in filenames.
\item
The command |\childdocmain{|\textit{main}|}| must be followed by a whitespace.
It should not be followed immediately by another command
or by a comment mark `|%|'.
This is because the \TeX{} parser reads the token immediately following
the argument of |\childdocmain| and puts it
at the beginning of every child section;
however, a white\-space is ignored.
\end{itemize}

%%%%%%%%%%%%%%%%%%%%%%%%%%%%%%%%%%%%%%%%
\paragraph{Content of Main File.}

It is advisable to place all content in the child files included by |\include|.
Any output contained in the main file will appear in all child documents
unless suppressed manually;
it cannot be suppressed automatically by the |\includeonly| directive
and thus should normally be avoided.
A method to include some content in the main file
by means of conditional processing is described in \secref{sec:conditional}.

%%%%%%%%%%%%%%%%%%%%%%%%%%%%%%%%%%%%%%%%
\paragraph{Page Numbering.}

When only a part of the document is compiled,
the appropriate numbering of pages
(as well as other status parameters)
is determined from the |.aux| files.
The latter contain information from previous passes.
However this information needs to propagate through
all intermediate child documents.
Therefore the page numbering in child documents may well
be inconsistent until the complete document is compiled at least once.

A useful (if unconventional) way to always ensure a consistent
page numbering is to restart the numbering in each child document
and denote the pages by `\textit{child}|.|\textit{page}'
where \textit{child} represents the chapter/section number of the child file.
This can be achieved by the command
|\numberwithin{page}{|\textit{child}|}|
of the \textsf{amsmath} package
where \textit{child} can be |chapter| or |section|
depending on the chosen structuring.
Alternatively, one can modify the macro |\thepage| appropriately
and reset the counter |page| at the start of each child file.

%%%%%%%%%%%%%%%%%%%%%%%%%%%%%%%%%%%%%%%%%%%%%%%%%%%%%%%%%%%%%%%%%%%%%%%%%%%%%%%%
\subsection{Conditional Processing}
\label{sec:conditional}

The package provides a mechanism to compile different versions
of a document. To customise the versions further some conditional processing
can come in handy to distinguish which version is being compiled.
The package provides two macros to describe the compilation context:

%%%%%%%%%%%%%%%%%%%%%%%%%%%%%%%%%%%%%%%%
\DescribeMacro{\ifchilddoc}
The conditional |\ifchilddoc| distinguishes between the compilation of
child documents and the main document:
%
\begin{center}
|\ifchilddoc |\textit{child-code}| |[|\||else |\textit{main-code}]| \||fi|
\end{center}

%%%%%%%%%%%%%%%%%%%%%%%%%%%%%%%%%%%%%%%%
\DescribeMacro{\childdocname}
\DescribeMacro{\childdocjob}
The macro |\childdocname| contains the filename (without extension)
of the main or child file being processed.
Note that |\childdocjob| will always contain the name of the main file.

%%%%%%%%%%%%%%%%%%%%%%%%%%%%%%%%%%%%%%%%
\paragraph{Title Page.}

Conditional processing can be used to include a title or banner page
in the main document when proper precautions are taken.
Importantly, the code in the main file should ensure that the page counter
(as well as other status parameters which are stored in the |.aux| files)
takes the same value after the conditional processing.
Otherwise the page numbers may take divergent values
depending on which part is compiled.

For example, a title page could be declared by:
%
\begin{center}
\begin{tabular}{l}
|\ifchilddoc\||else|\\
|\addtocounter{page}{-1}|\\
\textit{code for title page}\\
|\newpage|\\
|\||fi|
\end{tabular}
\end{center}
%
A banner page for the child documents can be generated by:
%
\begin{center}
\begin{tabular}{l}
|\ifchilddoc|\\
|\addtocounter{page}{-1}|\\
\textit{code for banner page}\\
|\newpage|\\
|\||fi|
\end{tabular}
\end{center}
%
Here one could write a message such as:
\begin{center}
|This is the part \childdocname{} of \childdocjob{}.|
\end{center}

%%%%%%%%%%%%%%%%%%%%%%%%%%%%%%%%%%%%%%%%%%%%%%%%%%%%%%%%%%%%%%%%%%%%%%%%%%%%%%%%
\subsection{Flags}
\label{sec:flags}

The package makes it easy to generate different versions
of the main or child documents.
To this end compilation flags can be defined
and assigned different default values.
They will be particularly useful in conjunction
with the forwarding mechanism described in \secref{sec:forward}.

For example, it may be useful to have a flag |\version|
which can be set to |draft| or |final|.
The document source will contain some conditional code
depending on the value of |\version|.
Suppose further, the flag should default to |final| for the main file
and to |draft| for child files
which is a natural assignment for editing the document.
This is achieved by placing the following code
in the preamble of the main document
(below the |\childdocmain| directive):
%
\begin{center}
\begin{tabular}{l}
|\ifchilddoc|\\
|\providecommand{\version}{draft}|\\
|\||else|\\
|\providecommand{\version}{final}|\\
|\||fi|
\end{tabular}
\end{center}
%
The definition by |\providecommand| makes sure
that previous definitions are not overwritten.
Further statements |\providecommand{\version}{...}|
can thus be added before the above code to override it.

For the main file, one might add a line
(between |\childdocmain| and the above block)
%
\begin{center}
|%\ifchilddoc\||else\providecommand{\version}{draft}\||fi|
\end{center}
%
which can be uncommented to produce a draft version.
Likewise one can add a line to the very top of a child file
(above the |\childdocof{|\textit{main}|}| directive)
%
\begin{center}
|%\providecommand{\version}{final}|
\end{center}
%
which can be uncommented to produce the final version of this child document.

%%%%%%%%%%%%%%%%%%%%%%%%%%%%%%%%%%%%%%%%%%%%%%%%%%%%%%%%%%%%%%%%%%%%%%%%%%%%%%%%
\subsection{Forwarding}
\label{sec:forward}

Different versions of the main or child documents
using compilation flags as described in \secref{sec:flags}
can be (permanently) stored in different files
for convenient compilation, viewing and distribution.
To this end, the package defines a command
to pass on compilation to a different file:

%%%%%%%%%%%%%%%%%%%%%%%%%%%%%%%%%%%%%%%%
\DescribeMacro{\childdocforward}
The command |\childdocforward| redirects processing to
another source file:
%
\begin{center}
\begin{tabular}{l}
|\input{childdoc.def}|\\
|\childdocforward[|\textit{main}|]{|\textit{dest}|}|\\
\end{tabular}
\end{center}
%
The argument \textit{dest} is the destination file
(without extension).
It should be the main file or one of the child files.
Note that further \textsf{childdoc} directives
such as |\childdocof| and |\childdocforward|
in the indicated file will be processed in this form.
The optional argument \textit{main}
passes on directly to the main file \textit{main}
while pretending to compile the child \textit{dest}.
This form behaves as if \textit{dest}
issues |\childdocof{|\textit{main}|}| right away,
and no further \textsf{childdoc} directives will be processed.

%%%%%%%%%%%%%%%%%%%%%%%%%%%%%%%%%%%%%%%%
\DescribeMacro{\...prefix}
In the alternative form |\childdocforwardprefix|,
%
\begin{center}
\begin{tabular}{l}
|\input{childdoc.def}|\\
|\childdocforwardprefix[|\textit{main}|]{|\textit{prefix}|}{|\textit{dest}|}|
\end{tabular}
\end{center}
%
the destination file is determined by a pattern
depending on the current file:
To make this work, the current file must be called
`{\textit{prefix}\hspace{0.2em}\textit{suffix}}'
with \textit{prefix} matching precisely the argument.
Processing is then passed on to the file
`{\textit{dest}\hspace{0.2em}\textit{suffix}}'.
Surely, the same effect is achieved by
directly specifying the
argument `{\textit{dest}\hspace{0.2em}\textit{suffix}}'
in the first form.
However, that requires to set up a different file
for each child. With the alternative form of the command
all these files can have exactly the same content
which simplifies setting them up and maintaining them.

For example, the following file |draft.tex|
with a compilation flag |\version| as described in \secref{sec:flags}
compiles the main document as a draft:
%
\begin{center}
\begin{tabular}{l}
|\def\version{draft}|\\
|\input{childdoc.def}|\\
|\childdocforward{|\textit{main}|}|
\end{tabular}
\end{center}
%
Likewise, the following files |final|\textit{nn}|.tex|
compile the final version of the child document
|child|\textit{nn}|.tex|:
%
\begin{center}
\begin{tabular}{l}
|\def\version{final}|\\
|\input{childdoc.def}|\\
|\childdocforwardprefix{final}{child}|
\end{tabular}
\end{center}
%

Note that when several versions of a main file and/or of each child file
are to be generated, it may be convenient to set up a |Makefile| or
shell script to automatise the process.

%%%%%%%%%%%%%%%%%%%%%%%%%%%%%%%%%%%%%%%%%%%%%%%%%%%%%%%%%%%%%%%%%%%%%%%%%%%%%%%%
\subsection{Command Line Processing}
\label{sec:commandline}

The effect of redirection files can also be achieved by invoking
the \LaTeX{} compiler with a more elaborate command line.
Most conveniently this should be done as part
of a shell script or a |Makefile|.

When using \textsf{childdoc} in the main file, the following
command lines effectively perform a redirection
(note that depending on the shell being used,
backslashes may have to be doubled: `|\|' $\to$ `|\\|'):
%
\begin{center}
|... -jobname "|\textit{target}|" |\\|"|[\textit{flags}]%
|\input{childdoc.def}\childdocforward[|\textit{main}|]{|\textit{dest}|}"|
\end{center}
%
Here \textit{target} is the name of the output file,
\textit{main} is the name of the main file
and \textit{dest} is the name of the main or child file to be processed
(all filenames without extensions).
The optional argument \textit{main} can be omitted
if \textit{main} matches \textit{dest}.
Optionally, compilation \textit{flags} can be defined via |\def| commands.
This command line makes the \TeX{} engine believe
it is compiling the file \textit{target}
whose content is specified as the latter parameter.
The provided code then forwards the processing to
\textit{main} or \textit{dest} as described in \secref{sec:forward}.

%%%%%%%%%%%%%%%%%%%%%%%%%%%%%%%%%%%%%%%%%%%%%%%%%%%%%%%%%%%%%%%%%%%%%%%%%%%%%%%%
\subsection{Include by Input}
\label{sec:input}

Including child documents by |\include| has some restrictions by design.
Most notably, the content of a child document always occupies
its own set of pages; pages cannot be shared between child documents.
Usually, this behaviour makes perfect sense
because each child document contain an essential part of the document.
However, in some situations it may be desirable to compose
a document from a collection of parts
without having mandatory page breaks between then.
For this case, the package
provides a mechanism to include parts
by |\input| which can also be processed individually.
However, by construction this mechanism
requires manual handling of the content to be output.

%%%%%%%%%%%%%%%%%%%%%%%%%%%%%%%%%%%%%%%%
\DescribeMacro{\ifchilddocmanual}
The main file should be prepared as usual, see \secref{sec:include}.
However, the document body must make a distinction
between processing of an individual part and of the main document, e.g.:
%
\begin{center}
\begin{tabular}{l}
|\ifchilddocmanual|\\
|\input{\childdocname}|\\
|\||else|\\
\textit{document body with }|\input{|\textit{part}|}|\\
|\||fi|
\end{tabular}
\end{center}
%
The conditional |\ifchilddocmanual| is true whenever
a part to be included by |\input| is being compiled,
and the name of the part is stored in |\childdocname|.

%%%%%%%%%%%%%%%%%%%%%%%%%%%%%%%%%%%%%%%%
\DescribeMacro{\childdocby}
Each part to be included by |\input| should start with:
%
\begin{center}
\begin{tabular}{l}
|\input{childdoc.def}|\\
|\childdocby{|\textit{main}|}|\\
\end{tabular}
\end{center}
%
The directive |\childdocby| is similar to |\childdocof|
described in \secref{sec:include},
but the subsequent selection of content must be done manually.
To that end, both |\ifchilddoc| and |\ifchilddocmanual|
will be true upon processing of a part,
and the name of the part is stored in |\childdocname|.
Note that |\jobname| will be set to the filename of the current part
so that each part receives an individual |.aux| file
that does not interfere with the |.aux| file(s) of the main document.
This behaviour can be altered by the alternative form
|\childdocby[*]{|\textit{main}|}| (with a non-empty optional argument)
which uses the |.aux| file of the main document
by setting |\jobname| to \textit{main}.

%%%%%%%%%%%%%%%%%%%%%%%%%%%%%%%%%%%%%%%%%%%%%%%%%%%%%%%%%%%%%%%%%%%%%%%%%%%%%%%%
\subsection{Driver Development}
\label{sec:driver}

The \textsf{childdoc} mechanism can also be use for the development
of definition files such as \LaTeX{} styles or classes.
This case differs from the above setup with multiple parts
included by |\include| in that no |\includeonly| should be invoked.
This can be achieved by starting the include file
(before |\ProvidesPackage|) with:
%
\begin{center}
\begin{tabular}{l}
|\input{childdoc.def}|\\
|\childdocforward{|\textit{main}|}|\\
\end{tabular}
\end{center}
%
or alternatively with:
%
\begin{center}
\begin{tabular}{l}
|\input{childdoc.def}|\\
|\childdocby{|\textit{main}|}|\\
\end{tabular}
\end{center}
%
Both forms have slightly different effects as described above.
The main file is prepared as usual, see \secref{sec:include}.

%%%%%%%%%%%%%%%%%%%%%%%%%%%%%%%%%%%%%%%%%%%%%%%%%%%%%%%%%%%%%%%%%%%%%%%%%%%%%%%%
\subsection{Legacy Detection}
\label{sec:detection}

The directive |\childdocmain| in the main file can detect
whether the complete document or merely a child is to be compiled
even without using the directive |\childdocof|.
This method is deprecated because it is less robust
and there is no compelling reason to use it;
it is merely provided for backward compatibility
and it may be removed in future versions.

If the detection mechanism is to be used,
it is mandatory to correctly specify
the filename of the main file as the argument of |\childdocmain|:
%
\begin{center}
\begin{tabular}{l}
|\input{childdoc.def}|\\
|\childdocmain{|\textit{main}|}|\\
\end{tabular}
\end{center}
%
If |\jobname| does not match the argument \textit{main} of |\childdocmain|,
it is assumed that |\jobname| points to the child file to be compiled.
When using |\childdocmain| with the main file specified as argument,
it suffices to start a child file
with just |\input{|\textit{main}|}|
without loading of the package and using |\childdocof|.
If instead all processing is done
with the appropriate \textsf{childdoc} directives,
the argument of \textit{main} of |\childdocmain| can be empty.

An alternative version of the command line processing described
in \secref{sec:commandline} using the detection mechanism reads:
%
\begin{center}
|... -jobname "|\textit{target}|" "|[\textit{flags}]%
[|\def\jobname{|\textit{dest}|}|]|\input{|\textit{main}|}"|
\end{center}

%%%%%%%%%%%%%%%%%%%%%%%%%%%%%%%%%%%%%%%%%%%%%%%%%%%%%%%%%%%%%%%%%%%%%%%%%%%%%%%%
\subsection{Manual Code}
\label{sec:manual}

In case one cannot be certain whether the definitions file |childdoc.def|
is installed on the target \TeX{} distribution
and one prefers not to ship it,
it is conceivable to paste a few relevant commands into the sources.

To that end, drop all statements |\input{childdoc.def}|
and perform the replacements as outlined below.
Instead of |\childdocmain{|\textit{main}|}| add the following code
to the top of the main file:
%
\begin{center}
\begin{tabular}{l}
|\||ifdefined\childdocname\endinput\||fi\newif\ifchilddoc|\\
|\edef\childdocname{\scantokens\expandafter{\jobname\noexpand}}|\\
|\def\childdocmain{|\textit{main}|}\||ifx\childdocmain\childdocname\||else|\\
|\childdoctrue\includeonly{\childdocname}\let\jobname\childdocmain\||fi|\\
\end{tabular}
\end{center}
%
Instead of |\childdocof{|\textit{main}|}| just include the main file
at the top of each child file:
%
\begin{center}
|\input{|\textit{main}|}|
\end{center}
%
A simple redirection |\childdocforward{|\textit{dest}|}| is achieved by:
%
\begin{center}
|\def\jobname{|\textit{dest}|}\input{\jobname}|
\end{center}
%
The redirection with prefix
|\childdocforwardprefix[|\textit{prefix}|]{|\textit{dest}|}|
is accomplished by:
%
\begin{center}
\begin{tabular}{l}
|{\edef\jobname{\scantokens\expandafter{\jobname\noexpand}}|\\
|\def\redirectjob |\textit{prefix}|#1~~~{\gdef\jobname{|\textit{dest}|#1}}|\\
|\expandafter\redirectjob\jobname~~~}\input{\jobname}|
\end{tabular}
\end{center}

In an alternative approach,
child documents can be compiled by a specific command line
without additional code or specific definitions:
%
\begin{center}
|... -jobname "|\textit{target}|" "|[\textit{flags}]%
|\includeonly{|\textit{dest}|}\input{|\textit{main}|}"|
\end{center}
%

%%%%%%%%%%%%%%%%%%%%%%%%%%%%%%%%%%%%%%%%%%%%%%%%%%%%%%%%%%%%%%%%%%%%%%%%%%%%%%%%
%%%%%%%%%%%%%%%%%%%%%%%%%%%%%%%%%%%%%%%%%%%%%%%%%%%%%%%%%%%%%%%%%%%%%%%%%%%%%%%%
\section{Information}

%%%%%%%%%%%%%%%%%%%%%%%%%%%%%%%%%%%%%%%%%%%%%%%%%%%%%%%%%%%%%%%%%%%%%%%%%%%%%%%%
\subsection{Copyright}

Copyright \copyright{} 2017--2018 Niklas Beisert

This work may be distributed and/or modified under the
conditions of the \LaTeX{} Project Public License, either version 1.3
of this license or (at your option) any later version.
The latest version of this license is in
  \url{http://www.latex-project.org/lppl.txt}
and version 1.3 or later is part of all distributions of \LaTeX{}
version 2005/12/01 or later.

This work has the LPPL maintenance status `maintained'.

The Current Maintainer of this work is Niklas Beisert.

This work consists of the files |README.txt|, |childdoc.ins| and |childdoc.dtx|
as well as the derived files |childdoc.def|, |cdocsamp.tex|
with |cdocsch1.tex|, |cdocsch2.tex|, |cdocspt3.tex|, |cdocspt4.tex|,
|cdocsdrf.tex|, |cdocsfn1.tex|, |cdocsfn2.tex|
as well as |childdoc.pdf|.

%%%%%%%%%%%%%%%%%%%%%%%%%%%%%%%%%%%%%%%%%%%%%%%%%%%%%%%%%%%%%%%%%%%%%%%%%%%%%%%%
\subsection{Files and Installation}

The package consists of the files:
%
\begin{center}
\begin{tabular}{ll}
    |README.txt|   & readme file \\
    |childdoc.ins| & installation file \\
    |childdoc.dtx| & source file \\
    |childdoc.def| & definition file \\
    |cdocsamp.tex| & sample main file \\
    |cdocsch1.tex| & sample include file \\
    |cdocsch2.tex| & sample include file \\
    |cdocspt3.tex| & sample part file \\
    |cdocspt4.tex| & sample part file \\
    |cdocsdrf.tex| & sample redirection file \\
    |cdocsfn1.tex| & sample redirection file \\
    |cdocsfn2.tex| & sample redirection file \\
    |childdoc.pdf| & manual
\end{tabular}
\end{center}
%
The distribution consists of the files
|README.txt|, |childdoc.ins| and |childdoc.dtx|.
%
\begin{itemize}
\item
Run (pdf)\LaTeX{} on |childdoc.dtx|
to compile the manual |childdoc.pdf| (this file).
\item
Run \LaTeX{} on |childdoc.ins| to create the definitions file |childdoc.def|
and the sample |cdocsamp.tex| with include files
|cdocsch1.tex|, |cdocsch2.tex|, |cdocspt3.tex|, |cdocspt4.tex|,
|cdocsdrf.tex|, |cdocsfn1.tex|, |cdocsfn2.tex|.
Then copy the file |childdoc.def| to an appropriate directory of your \LaTeX{}
distribution, e.g.\ \textit{texmf-root}|/tex/latex/childdoc|.
\end{itemize}

%%%%%%%%%%%%%%%%%%%%%%%%%%%%%%%%%%%%%%%%%%%%%%%%%%%%%%%%%%%%%%%%%%%%%%%%%%%%%%%%
\subsection{Related CTAN Packages}

There are several other packages which offer a similar functionality:
%
\begin{itemize}
\item
The packages
\href{http://ctan.org/pkg/docmute}{\textsf{docmute}},
\href{http://ctan.org/pkg/includex}{\textsf{includex}} and
\href{http://ctan.org/pkg/standalone}{\textsf{standalone}}
provide commands to include only the document body of
a child file thus allowing both files to be compiled individually.
\item
The packages \href{http://ctan.org/pkg/subdocs}{\textsf{subdocs}}
and \href{http://ctan.org/pkg/subfiles}{\textsf{subfiles}}
provide structures in which the main and child documents can be
encapsulated and allowing them to be compiled individually.
The inclusion mechanism is different from the conventional |\include|.
\item
The package \href{http://ctan.org/pkg/combine}{\textsf{combine}}
is an elaborate solution to combine several documents into one.
\end{itemize}
%
See also the CTAN topic \href{http://ctan.org/topic/subdocs}{\textsf{subdocs}}
for further related packages.
The present package differs from the above solutions in that
a document structure constructed with the conventional |\include| mechanism
just needs two extra commands at the top of every file
such that all constituent files can be compiled individually.

%%%%%%%%%%%%%%%%%%%%%%%%%%%%%%%%%%%%%%%%%%%%%%%%%%%%%%%%%%%%%%%%%%%%%%%%%%%%%%%%
%\subsection{Feature Suggestions}
%
%The following is a list of features which may be useful for future
%versions of this package:
%%
%\begin{itemize}
%\item
%\ldots
%\end{itemize}

%%%%%%%%%%%%%%%%%%%%%%%%%%%%%%%%%%%%%%%%%%%%%%%%%%%%%%%%%%%%%%%%%%%%%%%%%%%%%%%%
\subsection{Revision History}

%%%%%%%%%%%%%%%%%%%%%%%%%%%%%%%%%%%%%%%%
\paragraph{v2.0:} 2018/12/30

\begin{itemize}
\item
immediate forward processing
\item
added |\childdocby| mechanism
\item
manual restructured
\end{itemize}

%%%%%%%%%%%%%%%%%%%%%%%%%%%%%%%%%%%%%%%%
\paragraph{v1.6:} 2018/01/17

\begin{itemize}
\item
application for development of include files
\item
corrections to manual
\end{itemize}

%%%%%%%%%%%%%%%%%%%%%%%%%%%%%%%%%%%%%%%%
\paragraph{v1.5:} 2017/05/21

\begin{itemize}
\item
more complete structuring introduced
\item
|\childdocof| introduced
\item
|\childdoc| renamed to |\childdocmain|
\item
|\childredirect| renamed to |\childdocforward| and |\childdocforwardprefix|
and functionality expanded
\end{itemize}

%%%%%%%%%%%%%%%%%%%%%%%%%%%%%%%%%%%%%%%%
\paragraph{v1.0:} 2017/04/27

\begin{itemize}
\item
manual and install package
\item
first version published on CTAN
\end{itemize}

%%%%%%%%%%%%%%%%%%%%%%%%%%%%%%%%%%%%%%%%
\paragraph{v0.6:} 2017/04/26

\begin{itemize}
\item
redirection mechanism added
\end{itemize}

%%%%%%%%%%%%%%%%%%%%%%%%%%%%%%%%%%%%%%%%
\paragraph{v0.5:} 2017/04/26

\begin{itemize}
\item
functionality in definition file
\end{itemize}


%%%%%%%%%%%%%%%%%%%%%%%%%%%%%%%%%%%%%%%%%%%%%%%%%%%%%%%%%%%%%%%%%%%%%%%%%%%%%%%%
%%%%%%%%%%%%%%%%%%%%%%%%%%%%%%%%%%%%%%%%%%%%%%%%%%%%%%%%%%%%%%%%%%%%%%%%%%%%%%%%
%%%%%%%%%%%%%%%%%%%%%%%%%%%%%%%%%%%%%%%%%%%%%%%%%%%%%%%%%%%%%%%%%%%%%%%%%%%%%%%%
\appendix

\settowidth\MacroIndent{\rmfamily\scriptsize 000\ }

 \DocInput{childdoc.dtx}

\end{document}
%</driver>
% \fi
%
% %%%%%%%%%%%%%%%%%%%%%%%%%%%%%%%%%%%%%%%%%%%%%%%%%%%%%%%%%%%%%%%%%%%%%%%%%%%%%%
% %%%%%%%%%%%%%%%%%%%%%%%%%%%%%%%%%%%%%%%%%%%%%%%%%%%%%%%%%%%%%%%%%%%%%%%%%%%%%%
% \section{Sample}
%\iffalse
%<*samplemain>
%\fi
%
% The following presents a sample document
% with two chapters, two parts, a title page,
% a compile flag as well as three forwarding files to set the flag.
% It consists of eight |.tex| files:
% \begin{center}
% \begin{tabular}{ll}
% |cdocsamp.tex|&main file\\
% |cdocsch1.tex|&include file for chapter 1\\
% |cdocsch2.tex|&include file for chapter 2\\
% |cdocspt3.tex|&include file for part 3\\
% |cdocspt4.tex|&include file for part 4\\
% |cdocsdrf.tex|&forwarding file for main file in draft mode\\
% |cdocsfi1.tex|&forwarding file for final version of chapter 1\\
% |cdocsfi2.tex|&forwarding file for final version of chapter 2\\
% \end{tabular}
% \end{center}
% Each of the eight files can be compiled directly by the \LaTeX{} compiler.
%
% %%%%%%%%%%%%%%%%%%%%%%%%%%%%%%%%%%%%%%
% \paragraph{Main File.}
%
% The main file is called |cdocsamp.tex|.
%
% Load the \textsf{childdoc} definitions and
% declare the filename for the main document:
%    \begin{macrocode}
\input{childdoc.def}
\childdocmain{}
%    \end{macrocode}

% Optional override for |\version| flag:
%    \begin{macrocode}
%%\ifchilddoc\else\providecommand{\version}{draft}\fi
%    \end{macrocode}

% Define the default values for the |\version| flag
% (|final| for the main file and |draft| for childs):
%    \begin{macrocode}
\ifchilddoc
\providecommand{\version}{draft}
\else
\providecommand{\version}{final}
\fi
%    \end{macrocode}

% Load the standard document class:
%    \begin{macrocode}
\documentclass[12pt]{article}
%    \end{macrocode}

% Start the document body:
%    \begin{macrocode}
\begin{document}
%    \end{macrocode}

% Declare a title page.
% Print title, part of document being processed and version flag:
%    \begin{macrocode}
\addtocounter{page}{-1}
\begin{center}
{\LARGE\bfseries{}childdoc example\par}
\vspace{1cm}
\ifchilddoc
\ifchilddocmanual part\else chapter\fi:
`\childdocname' of `\childdocjob'\par
\else
main document: `\childdocjob'\par
\fi
version: \version\par
\end{center}
\newpage
%    \end{macrocode}

% Manually include selected file,
% otherwise process as usual:
%    \begin{macrocode}
\ifchilddocmanual
\section*{part `\childdocname'}
\input{\childdocname}
\else
%    \end{macrocode}

% Include the two chapters:
%    \begin{macrocode}
\include{cdocsch1}
\include{cdocsch2}
%    \end{macrocode}

% Include the two parts unless only chapters should be displayed:
%    \begin{macrocode}
\ifchilddoc\else
\section{part three}
\input{cdocspt3}
\section{part four}
\input{cdocspt4}
\fi
%    \end{macrocode}

% Process as usual until here:
%    \begin{macrocode}
\fi
%    \end{macrocode}

% End of document body:
%    \begin{macrocode}
\end{document}
%    \end{macrocode}
%\iffalse
%</samplemain>
%\fi
%
% %%%%%%%%%%%%%%%%%%%%%%%%%%%%%%%%%%%%%%
% \paragraph{Chapter Include Files.}
%
% The include files are called |cdocsch1.tex| and |cdocsch2.tex|.
%
%\iffalse
%<*samplechap1|samplechap2>
%\fi

% Optional override for |\version| flag:
%    \begin{macrocode}
%%\providecommand{\version}{final}
%    \end{macrocode}

% Include the main document:
%    \begin{macrocode}
\input{childdoc.def}
\childdocof{cdocsamp}
%    \end{macrocode}

%\iffalse
%</samplechap1|samplechap2>
%\fi
%
%\iffalse
%<*samplechap1>
%\fi
% Some text for chapter 1:
%    \begin{macrocode}
\section{one}
some text in chapter one
%    \end{macrocode}

%\iffalse
%</samplechap1>
%\fi
% Some text for chapter 2:
%\iffalse
%<*samplechap2>
%\fi
%    \begin{macrocode}
\section{two}
more text in chapter two
%    \end{macrocode}

%\iffalse
%</samplechap2>
%\fi
%
% %%%%%%%%%%%%%%%%%%%%%%%%%%%%%%%%%%%%%%
% \paragraph{Part Include Files.}
%
% The include files are called |cdocspt3.tex| and |cdocspt4.tex|.
%
%\iffalse
%<*samplepart3|samplepart4>
%\fi

% Optional override for |\version| flag:
%    \begin{macrocode}
%%\providecommand{\version}{final}
%    \end{macrocode}

% Include the main document:
%    \begin{macrocode}
\input{childdoc.def}
\childdocby{cdocsamp}
%    \end{macrocode}

%\iffalse
%</samplepart3|samplepart4>
%\fi
%
%\iffalse
%<*samplepart3>
%\fi
% Some text for part 3:
%    \begin{macrocode}
some text in part three
%    \end{macrocode}

%\iffalse
%</samplepart3>
%\fi
% Some text for part 4:
%\iffalse
%<*samplepart4>
%\fi
%    \begin{macrocode}
more text in part four
%    \end{macrocode}

%\iffalse
%</samplepart4>
%\fi
%
% %%%%%%%%%%%%%%%%%%%%%%%%%%%%%%%%%%%%%%
% \paragraph{Forwarding for a Complete Draft.}
%
% The following forwarding file |cdocsdrf.tex|
% compiles the main document in draft mode:
%\iffalse
%<*sampledraft>
%\fi
%    \begin{macrocode}
\def\version{draft}
\input{childdoc.def}
\childdocforward{cdocsamp}
%    \end{macrocode}

%\iffalse
%</sampledraft>
%\fi
%
% %%%%%%%%%%%%%%%%%%%%%%%%%%%%%%%%%%%%%%
% \paragraph{Forwarding for Final Version of the Chapters.}
%
% The following forwarding files |cdocsfn1.tex| and |cdocsfn2.tex|
% (with identical content)
% compile the final versions of the child documents
% |cdocsch1.tex| and |cdocsch2.tex|, respectively:
%\iffalse
%<*samplefinal>
%\fi
%    \begin{macrocode}
\def\version{final}
\input{childdoc.def}
\childdocforwardprefix[cdocsamp]{cdocsfn}{cdocsch}
%    \end{macrocode}

%\iffalse
%</samplefinal>
%\fi
%
% %%%%%%%%%%%%%%%%%%%%%%%%%%%%%%%%%%%%%%
% \paragraph{Command Line Processing.}
%
% The following three command lines generate the output files
% |cdocscld|, |cdocscl1| and |cdocscl2|
% which should be identical to
% |cdocsdrf|, |cdocsch1| and |cdocsfn2|, respectively:
% \begin{center}
% \begin{tabular}{l}
% |latex -jobname cdocscld \|\\
% |  "\def\version{draft}\input{childdoc.def}\childdocforward{cdocsamp}"|\\
% |latex -jobname cdocscl1 \|\\
% |  "\input{childdoc.def}\childdocforward[cdocsamp]{cdocsch1}"|\\
% |latex -jobname cdocscl2 \|\\
% |  "\def\version{final}\input{childdoc.def}\childdocforward{cdocsch2}"|
% \end{tabular}
% \end{center}
% Note that the trailing backslash on each first line
% merely continues the input to the second line
% (for convenient cut ant paste).
% Furthermore, the command |latex| can be replaced by any
% of its alternative versions such as |pdflatex|.
%
% %%%%%%%%%%%%%%%%%%%%%%%%%%%%%%%%%%%%%%%%%%%%%%%%%%%%%%%%%%%%%%%%%%%%%%%%%%%%%%
% %%%%%%%%%%%%%%%%%%%%%%%%%%%%%%%%%%%%%%%%%%%%%%%%%%%%%%%%%%%%%%%%%%%%%%%%%%%%%%
% \section{Implementation}
%\iffalse
%<*package>
%\fi
%
% This section describes the definitions file |childdoc.def|.

% The definitions cannot be loaded using |\usepackage| or |\RequirePackage|
% which has a mechanism to prevent loading a style file more than once.
% When loading the definitions by means of |\input|
% multiple instances have to be prevented manually:
%\iffalse
%This code needs to be before the `\ProvidesFile' directive
%which is defined at the beginning of this file.
%Therefore it is also placed there and commented out here.
%</package>
%<*discard>
%\fi
%    \begin{macrocode}
\ifdefined\childdocmain\endinput\fi
%    \end{macrocode}
%\iffalse
%</discard>
%<*package>
%\fi
%
% \macro{\ifchilddoc}
% \macro{\ifchilddocmanual}
% The conditional |\ifchilddoc| tells whether a
% child (true) or main (false) document is being compiled.
% The conditional |\ifchilddocmanual| tells whether
% the |\includeonly| mechanism is used (false) or
% the selection of child files must be performed manually (true).
% The definitions initialise to false:
%    \begin{macrocode}
\newif\ifchilddoc
\newif\ifchilddocmanual
%    \end{macrocode}

% \macro{\childdocname}
% \macro{\childdocjob}
% The macro |\childdocname| stores the name of the main document
% to be compiled. The macro |\childdocjob| stores the name of
% the document on which the \LaTeX{} compiler was originally invoked.
% The content of |\jobname| cannot be compared
% to filenames specified in the source due to different catcodes.
% The following code rescans |\jobname|, stores the result
% in |\childdocname| and saves a copy in |\childdocjob|:
%    \begin{macrocode}
\edef\childdocname{\scantokens\expandafter{\jobname\noexpand}}
\let\childdocjob\childdocname
%    \end{macrocode}

% \macro{\childdocdisable}
% The macro |\childdocdisable| prevents the main file
% from being processed more than once.
% At this stage, the main document command |\childdocmain|
% is assumed to be called once again where it should do nothing.
% Any subsequent call to it should prevent
% a secondary processing of the main document
% It overwrites the forwarding commands
% |\childdocof| and |\childdocforward|
% with empty macros to prevent further inclusions of the main document:
%    \begin{macrocode}
\newcommand{\childdocdisable}
{
  \renewcommand{\childdocmain}[1]{\renewcommand{\childdocmain}[1]{\endinput}}
  \renewcommand{\childdocof}[1]{}
  \renewcommand{\childdocby}[2][]{}
  \renewcommand{\childdocforward}[2][]{}
  \renewcommand{\childdocdisable}{}
}
%    \end{macrocode}

% \macro{\childdocmain}
% The macro |\childdocmain| is to be called at the top of the main file
% with nothing or the main filename (without extension) as argument.
% First, it breaks loops.
% If the argument is not empty and does not match |\childdocname|
% (which is set by the first inclusion of |childdoc.def|),
% |\ifchilddoc| is set to true, |\includeonly| is applied to the child file
% and |\jobname| is set to the main file
% (for proper handling of |.aux| files):
%    \begin{macrocode}
\newcommand{\childdocmain}[1]
{
  \childdocdisable\childdocmain{}
  \if?#1?\else
    \begingroup
      \def\childdoctmp{#1}
      \ifx\childdoctmp\childdocname
        \def\childdoctmp{}
      \else
        \def\childdoctmp
        {
          \childdoctrue
          \includeonly{\childdocname}
          \def\childdocjob{#1}
          \def\jobname{#1}
        }
      \fi
      \expandafter
    \endgroup
    \childdoctmp
  \fi
}
%    \end{macrocode}

% \macro{\childdocof}
% The command |\childdocof| redirects
% compilation to the main file |#1|.
%    \begin{macrocode}
\newcommand{\childdocof}[1]
{
  \childdocdisable
  \childdoctrue
  \includeonly{\childdocname}
  \def\jobname{#1}
  \def\childdocjob{#1}
  \input{#1}
}
%    \end{macrocode}

% \macro{\childdocby}
% The command |\childdocby| ....
%    \begin{macrocode}
\newcommand{\childdocby}[2][]
{
  \childdocdisable
  \childdoctrue
  \childdocmanualtrue
  \if?#1?\else
    \def\jobname{#2}
  \fi
  \def\childdocjob{#2}
  \input{#2}
  \endinput
}
%    \end{macrocode}

% \macro{\childdocforward}
% The command |\childdocforward| redirects
% compilation to the main file or
% (if the optional argument is given) a child file.
% Parameters are set as if the main file
% or a child file starting with |\childdocof| was compiled.
% Then compilation is handed over to the main file:
%    \begin{macrocode}
\newcommand{\childdocforward}[2][]
{
  \begingroup
    \if?#1?
      \def\childdoctmp
      {
        \def\childdocname{#2}
        \def\childdocjob{#2}
        \def\jobname{#2}
        \input{#2}
        \endinput
      }
    \else
      \def\childdoctmp
      {
        \childdocdisable
        \def\childdocname{#2}
        \childdoctrue
        \includeonly{#2}
        \def\childdocjob{#1}
        \def\jobname{#1}
        \input{#1}
        \endinput
      }
    \fi
    \expandafter
  \endgroup
  \childdoctmp
}
%    \end{macrocode}

% \macro{\childdocforwardprefix}
% The command |\childdocforwardprefix| redirects
% compilation to the main or a child file by means of a pattern.
% The prefix |#1| in the current filename is replaced by |#2|
% and the suffix of the current filename is kept
% (it is assumed that the filename does not contain the substring `|~~~|'
% which is used as a delimiter).
% Compilation is handed over to the new file by |\childdocforward|:
%    \begin{macrocode}
\newcommand{\childdocforwardprefix}[3][]
{
  \begingroup
    \def\childdocextract #2##1~~~{\def\childdoctmp{\childdocforward[#1]{#3##1}}}
    \expandafter\childdocextract\childdocname~~~
    \expandafter
  \endgroup
  \childdoctmp
}
%    \end{macrocode}

% \macro{\childdoc}
% The deprecated macro |\childdoc| is a legacy version of |\childdocmain|:
%    \begin{macrocode}
\newcommand{\childdoc}{\childdocmain}
%    \end{macrocode}

% \macro{\childdocredirect}
% The deprecated macro |\childdocredirect| is a legacy version
% of |\childdocforward| and |\childdocforwardprefix|:
%    \begin{macrocode}
\newcommand{\childdocredirect}[2][]
{
  \begingroup
    \if?#1?
      \def\childdoctmp{\childdocforward{#2}}
    \else
      \def\childdoctmp{\childdocforwardprefix{#1}{#2}}
    \fi
    \expandafter
  \endgroup
  \childdoctmp
}
%    \end{macrocode}

%\iffalse
%</package>
%\fi
%
\endinput

\childdocforward{cdocsamp}
%    \end{macrocode}

%\iffalse
%</sampledraft>
%\fi
%
% %%%%%%%%%%%%%%%%%%%%%%%%%%%%%%%%%%%%%%
% \paragraph{Forwarding for Final Version of the Chapters.}
%
% The following forwarding files |cdocsfn1.tex| and |cdocsfn2.tex|
% (with identical content)
% compile the final versions of the child documents
% |cdocsch1.tex| and |cdocsch2.tex|, respectively:
%\iffalse
%<*samplefinal>
%\fi
%    \begin{macrocode}
\def\version{final}
% \iffalse
%
% childdoc.dtx Copyright (C) 2017-2018 Niklas Beisert
%
% This work may be distributed and/or modified under the
% conditions of the LaTeX Project Public License, either version 1.3
% of this license or (at your option) any later version.
% The latest version of this license is in
%   http://www.latex-project.org/lppl.txt
% and version 1.3 or later is part of all distributions of LaTeX
% version 2005/12/01 or later.
%
% This work has the LPPL maintenance status `maintained'.
%
% The Current Maintainer of this work is Niklas Beisert.
%
% This work consists of the files childdoc.dtx and childdoc.ins
% and the derived files childdoc.def and cdocsamp.tex with
% cdocsch1.tex, cdocsch2.tex, cdocsdrf.tex, cdocsfn1.tex, cdocsfn2.tex.
%
%<package>\ifdefined\childdocmain\endinput\fi
%<package>\ProvidesFile{childdoc.def}[2018/12/30 v2.0 child document driver]
%<samplemain>\ProvidesFile{cdocsamp.tex}[2018/12/30 v2.0 sample for childdoc]
%<*driver>
%\ProvidesFile{childdoc.drv}[2018/12/30 v2.0 childdoc reference manual file]
\PassOptionsToClass{10pt,a4paper}{article}
\documentclass{ltxdoc}

\usepackage[margin=35mm]{geometry}
\usepackage{hyperref}
\usepackage{hyperxmp}
\usepackage[usenames]{color}

\hypersetup{colorlinks=true}
\hypersetup{pdfstartview=FitH}
\hypersetup{pdfpagemode=UseNone}
\hypersetup{pdfsource={}}
\hypersetup{pdflang={en-UK}}
\hypersetup{pdfcopyright={Copyright 2017-2018 Niklas Beisert.
  This work may be distributed and/or modified under the
  conditions of the LaTeX Project Public License, either version 1.3
  of this license or (at your option) any later version.}}
\hypersetup{pdflicenseurl={http://www.latex-project.org/lppl.txt}}
\hypersetup{pdfcontactaddress={ETH Zurich, ITP, HIT K,
  Wolfgang-Pauli-Strasse 27}}
\hypersetup{pdfcontactpostcode={8093}}
\hypersetup{pdfcontactcity={Zurich}}
\hypersetup{pdfcontactcountry={Switzerland}}
\hypersetup{pdfcontactemail={nbeisert@itp.phys.ethz.ch}}
\hypersetup{pdfcontacturl={http://people.phys.ethz.ch/\xmptilde nbeisert/}}

\newcommand{\secref}[1]{\hyperref[#1]{section \ref*{#1}}}

\parskip1ex
\parindent0pt
\let\olditemize\itemize
\def\itemize{\olditemize\parskip0pt}

\begin{document}

\title{The \textsf{childdoc} Package}
\hypersetup{pdftitle={The childdoc Package}}
\author{Niklas Beisert\\[2ex]
  Institut f\"ur Theoretische Physik\\
  Eidgen\"ossische Technische Hochschule Z\"urich\\
  Wolfgang-Pauli-Strasse 27, 8093 Z\"urich, Switzerland\\[1ex]
  \href{mailto:nbeisert@itp.phys.ethz.ch}
  {\texttt{nbeisert@itp.phys.ethz.ch}}}
\hypersetup{pdfauthor={Niklas Beisert}}
\hypersetup{pdfsubject={Manual for the LaTeX2e Package childdoc}}
\date{30 December 2018, \textsf{v2.0}}
\maketitle

\begin{abstract}\noindent
\textsf{childdoc} is a \LaTeXe{} package
that enables the direct compilation
of document sections included by |\include|
to individual files.
\end{abstract}

\begingroup
\parskip0ex
\tableofcontents
\endgroup

%%%%%%%%%%%%%%%%%%%%%%%%%%%%%%%%%%%%%%%%%%%%%%%%%%%%%%%%%%%%%%%%%%%%%%%%%%%%%%%%
%%%%%%%%%%%%%%%%%%%%%%%%%%%%%%%%%%%%%%%%%%%%%%%%%%%%%%%%%%%%%%%%%%%%%%%%%%%%%%%%
\section{Introduction}

\LaTeX{} provides a mechanism to structure a large document (such as a book)
into a main file and several child files (containing the chapters)
using the |\include| command.
This mechanism is beneficial for documents
which span hundreds of pages in order to
make the source file(s) more manageable.
Moreover, compilation can be restricted to
selected child files by means of the |\includeonly| command.
The latter feature can be used to reduce the compilation time while editing
(this was significantly more useful in the earlier days of \LaTeX{})
or to generate a smaller document which is easier to navigate.
Another application of |\includeonly| is to generate
documents consisting of selected parts of the complete document.

However, there are a few drawbacks of the plain |\include| mechanism:
\begin{itemize}
\item
The child files cannot be compiled on their own,
they can only be compiled via the main file.
A naive editing environment
(such as a text editor with an option
to have the current file processed by \LaTeX)
may require one to switch to the main file before compiling;
attempting to compile the child file produces errors.
\item
The main file must be modified (each time)
to adjust the |\includeonly| command
to the present needs. This easily leaves the main file in a messy state.
\item
The generated document will always carry the filename
of the main document. This is inconvenient if
several child files are to be compiled and
to be kept for distribution.
\end{itemize}

The present package provides a simple interface
to make child files individually compilable by \LaTeX{}.
Compiling a child file then has the same effect as compiling
the main file with an |\includeonly| command
to select the appropriate child.
Moreover the generated document will carry the name of the child
rather than the main file.
This resolves all three above issues.

This feature is meant to make the editing of books,
thesis documents and lecture notes somewhat more convenient.
However, the package can also be used efficiently for
composing a series of documents (such as exercise sheets)
which are typically distributed individually.
It then assists the author in generating the individual documents
(potentially in different versions)
as well as a document containing the collected series.
Another application is in developing style files
or other kinds of included material
where compilation of the style file could redirect
to a sample or test file.

%%%%%%%%%%%%%%%%%%%%%%%%%%%%%%%%%%%%%%%%%%%%%%%%%%%%%%%%%%%%%%%%%%%%%%%%%%%%%%%%
%%%%%%%%%%%%%%%%%%%%%%%%%%%%%%%%%%%%%%%%%%%%%%%%%%%%%%%%%%%%%%%%%%%%%%%%%%%%%%%%
\section{Usage}

First of all, the package \textsf{childdoc} is \emph{not} a standard
\LaTeXe{} |.sty| style file! Therefore it needs to be invoked in
a non-standard way.

%%%%%%%%%%%%%%%%%%%%%%%%%%%%%%%%%%%%%%%%%%%%%%%%%%%%%%%%%%%%%%%%%%%%%%%%%%%%%%%%
\subsection{Included Files}
\label{sec:include}

%%%%%%%%%%%%%%%%%%%%%%%%%%%%%%%%%%%%%%%%
\DescribeMacro{\childdocmain}
To use the package, add the commands
\begin{center}
\begin{tabular}{l}
|\input{childdoc.def}|\\
|\childdocmain{}|\\
\end{tabular}
\end{center}
at the very top of the main \LaTeX{} file,
in particular \emph{before} the |\documentclass| statement!
The argument of |\childdocmain| should be left empty
(but it must be present).

%%%%%%%%%%%%%%%%%%%%%%%%%%%%%%%%%%%%%%%%
\DescribeMacro{\childdocof}
Furthermore, add the commands
\begin{center}
\begin{tabular}{l}
|\input{childdoc.def}|\\
|\childdocof{|\textit{main}|}|\\
\end{tabular}
\end{center}
at the top of every child file \textit{child}
which is included by |\include{|\textit{child}|}|
from within the main file
(or at least for those files to be compiled individually).
The argument \textit{main} must be the filename of the main file.

There are a couple of
considerations in setting up the main and child documents:

%%%%%%%%%%%%%%%%%%%%%%%%%%%%%%%%%%%%%%%%
\paragraph{Restrictions.}

Please note the following restrictions:
\begin{itemize}
\item
|\childdocmain| must be called with one argument \textit{main}
to ensure compatibility with earlier version of the package.
It must either be empty (|\childdocmain{}|)
or precisely match the filename of the main file in which it is specified.
See \secref{sec:detection} for further information.
\item
The filename \textit{main} must be specified without the |.tex| extension.
\item
The filename \textit{main} is case sensitive
(even in case-insensitive file systems)
due to internal string comparison.
\item
The argument \textit{main} should be fully expanded, it cannot be a macro.
\item
Subdirectories and special characters should be avoided in filenames.
\item
The command |\childdocmain{|\textit{main}|}| must be followed by a whitespace.
It should not be followed immediately by another command
or by a comment mark `|%|'.
This is because the \TeX{} parser reads the token immediately following
the argument of |\childdocmain| and puts it
at the beginning of every child section;
however, a white\-space is ignored.
\end{itemize}

%%%%%%%%%%%%%%%%%%%%%%%%%%%%%%%%%%%%%%%%
\paragraph{Content of Main File.}

It is advisable to place all content in the child files included by |\include|.
Any output contained in the main file will appear in all child documents
unless suppressed manually;
it cannot be suppressed automatically by the |\includeonly| directive
and thus should normally be avoided.
A method to include some content in the main file
by means of conditional processing is described in \secref{sec:conditional}.

%%%%%%%%%%%%%%%%%%%%%%%%%%%%%%%%%%%%%%%%
\paragraph{Page Numbering.}

When only a part of the document is compiled,
the appropriate numbering of pages
(as well as other status parameters)
is determined from the |.aux| files.
The latter contain information from previous passes.
However this information needs to propagate through
all intermediate child documents.
Therefore the page numbering in child documents may well
be inconsistent until the complete document is compiled at least once.

A useful (if unconventional) way to always ensure a consistent
page numbering is to restart the numbering in each child document
and denote the pages by `\textit{child}|.|\textit{page}'
where \textit{child} represents the chapter/section number of the child file.
This can be achieved by the command
|\numberwithin{page}{|\textit{child}|}|
of the \textsf{amsmath} package
where \textit{child} can be |chapter| or |section|
depending on the chosen structuring.
Alternatively, one can modify the macro |\thepage| appropriately
and reset the counter |page| at the start of each child file.

%%%%%%%%%%%%%%%%%%%%%%%%%%%%%%%%%%%%%%%%%%%%%%%%%%%%%%%%%%%%%%%%%%%%%%%%%%%%%%%%
\subsection{Conditional Processing}
\label{sec:conditional}

The package provides a mechanism to compile different versions
of a document. To customise the versions further some conditional processing
can come in handy to distinguish which version is being compiled.
The package provides two macros to describe the compilation context:

%%%%%%%%%%%%%%%%%%%%%%%%%%%%%%%%%%%%%%%%
\DescribeMacro{\ifchilddoc}
The conditional |\ifchilddoc| distinguishes between the compilation of
child documents and the main document:
%
\begin{center}
|\ifchilddoc |\textit{child-code}| |[|\||else |\textit{main-code}]| \||fi|
\end{center}

%%%%%%%%%%%%%%%%%%%%%%%%%%%%%%%%%%%%%%%%
\DescribeMacro{\childdocname}
\DescribeMacro{\childdocjob}
The macro |\childdocname| contains the filename (without extension)
of the main or child file being processed.
Note that |\childdocjob| will always contain the name of the main file.

%%%%%%%%%%%%%%%%%%%%%%%%%%%%%%%%%%%%%%%%
\paragraph{Title Page.}

Conditional processing can be used to include a title or banner page
in the main document when proper precautions are taken.
Importantly, the code in the main file should ensure that the page counter
(as well as other status parameters which are stored in the |.aux| files)
takes the same value after the conditional processing.
Otherwise the page numbers may take divergent values
depending on which part is compiled.

For example, a title page could be declared by:
%
\begin{center}
\begin{tabular}{l}
|\ifchilddoc\||else|\\
|\addtocounter{page}{-1}|\\
\textit{code for title page}\\
|\newpage|\\
|\||fi|
\end{tabular}
\end{center}
%
A banner page for the child documents can be generated by:
%
\begin{center}
\begin{tabular}{l}
|\ifchilddoc|\\
|\addtocounter{page}{-1}|\\
\textit{code for banner page}\\
|\newpage|\\
|\||fi|
\end{tabular}
\end{center}
%
Here one could write a message such as:
\begin{center}
|This is the part \childdocname{} of \childdocjob{}.|
\end{center}

%%%%%%%%%%%%%%%%%%%%%%%%%%%%%%%%%%%%%%%%%%%%%%%%%%%%%%%%%%%%%%%%%%%%%%%%%%%%%%%%
\subsection{Flags}
\label{sec:flags}

The package makes it easy to generate different versions
of the main or child documents.
To this end compilation flags can be defined
and assigned different default values.
They will be particularly useful in conjunction
with the forwarding mechanism described in \secref{sec:forward}.

For example, it may be useful to have a flag |\version|
which can be set to |draft| or |final|.
The document source will contain some conditional code
depending on the value of |\version|.
Suppose further, the flag should default to |final| for the main file
and to |draft| for child files
which is a natural assignment for editing the document.
This is achieved by placing the following code
in the preamble of the main document
(below the |\childdocmain| directive):
%
\begin{center}
\begin{tabular}{l}
|\ifchilddoc|\\
|\providecommand{\version}{draft}|\\
|\||else|\\
|\providecommand{\version}{final}|\\
|\||fi|
\end{tabular}
\end{center}
%
The definition by |\providecommand| makes sure
that previous definitions are not overwritten.
Further statements |\providecommand{\version}{...}|
can thus be added before the above code to override it.

For the main file, one might add a line
(between |\childdocmain| and the above block)
%
\begin{center}
|%\ifchilddoc\||else\providecommand{\version}{draft}\||fi|
\end{center}
%
which can be uncommented to produce a draft version.
Likewise one can add a line to the very top of a child file
(above the |\childdocof{|\textit{main}|}| directive)
%
\begin{center}
|%\providecommand{\version}{final}|
\end{center}
%
which can be uncommented to produce the final version of this child document.

%%%%%%%%%%%%%%%%%%%%%%%%%%%%%%%%%%%%%%%%%%%%%%%%%%%%%%%%%%%%%%%%%%%%%%%%%%%%%%%%
\subsection{Forwarding}
\label{sec:forward}

Different versions of the main or child documents
using compilation flags as described in \secref{sec:flags}
can be (permanently) stored in different files
for convenient compilation, viewing and distribution.
To this end, the package defines a command
to pass on compilation to a different file:

%%%%%%%%%%%%%%%%%%%%%%%%%%%%%%%%%%%%%%%%
\DescribeMacro{\childdocforward}
The command |\childdocforward| redirects processing to
another source file:
%
\begin{center}
\begin{tabular}{l}
|\input{childdoc.def}|\\
|\childdocforward[|\textit{main}|]{|\textit{dest}|}|\\
\end{tabular}
\end{center}
%
The argument \textit{dest} is the destination file
(without extension).
It should be the main file or one of the child files.
Note that further \textsf{childdoc} directives
such as |\childdocof| and |\childdocforward|
in the indicated file will be processed in this form.
The optional argument \textit{main}
passes on directly to the main file \textit{main}
while pretending to compile the child \textit{dest}.
This form behaves as if \textit{dest}
issues |\childdocof{|\textit{main}|}| right away,
and no further \textsf{childdoc} directives will be processed.

%%%%%%%%%%%%%%%%%%%%%%%%%%%%%%%%%%%%%%%%
\DescribeMacro{\...prefix}
In the alternative form |\childdocforwardprefix|,
%
\begin{center}
\begin{tabular}{l}
|\input{childdoc.def}|\\
|\childdocforwardprefix[|\textit{main}|]{|\textit{prefix}|}{|\textit{dest}|}|
\end{tabular}
\end{center}
%
the destination file is determined by a pattern
depending on the current file:
To make this work, the current file must be called
`{\textit{prefix}\hspace{0.2em}\textit{suffix}}'
with \textit{prefix} matching precisely the argument.
Processing is then passed on to the file
`{\textit{dest}\hspace{0.2em}\textit{suffix}}'.
Surely, the same effect is achieved by
directly specifying the
argument `{\textit{dest}\hspace{0.2em}\textit{suffix}}'
in the first form.
However, that requires to set up a different file
for each child. With the alternative form of the command
all these files can have exactly the same content
which simplifies setting them up and maintaining them.

For example, the following file |draft.tex|
with a compilation flag |\version| as described in \secref{sec:flags}
compiles the main document as a draft:
%
\begin{center}
\begin{tabular}{l}
|\def\version{draft}|\\
|\input{childdoc.def}|\\
|\childdocforward{|\textit{main}|}|
\end{tabular}
\end{center}
%
Likewise, the following files |final|\textit{nn}|.tex|
compile the final version of the child document
|child|\textit{nn}|.tex|:
%
\begin{center}
\begin{tabular}{l}
|\def\version{final}|\\
|\input{childdoc.def}|\\
|\childdocforwardprefix{final}{child}|
\end{tabular}
\end{center}
%

Note that when several versions of a main file and/or of each child file
are to be generated, it may be convenient to set up a |Makefile| or
shell script to automatise the process.

%%%%%%%%%%%%%%%%%%%%%%%%%%%%%%%%%%%%%%%%%%%%%%%%%%%%%%%%%%%%%%%%%%%%%%%%%%%%%%%%
\subsection{Command Line Processing}
\label{sec:commandline}

The effect of redirection files can also be achieved by invoking
the \LaTeX{} compiler with a more elaborate command line.
Most conveniently this should be done as part
of a shell script or a |Makefile|.

When using \textsf{childdoc} in the main file, the following
command lines effectively perform a redirection
(note that depending on the shell being used,
backslashes may have to be doubled: `|\|' $\to$ `|\\|'):
%
\begin{center}
|... -jobname "|\textit{target}|" |\\|"|[\textit{flags}]%
|\input{childdoc.def}\childdocforward[|\textit{main}|]{|\textit{dest}|}"|
\end{center}
%
Here \textit{target} is the name of the output file,
\textit{main} is the name of the main file
and \textit{dest} is the name of the main or child file to be processed
(all filenames without extensions).
The optional argument \textit{main} can be omitted
if \textit{main} matches \textit{dest}.
Optionally, compilation \textit{flags} can be defined via |\def| commands.
This command line makes the \TeX{} engine believe
it is compiling the file \textit{target}
whose content is specified as the latter parameter.
The provided code then forwards the processing to
\textit{main} or \textit{dest} as described in \secref{sec:forward}.

%%%%%%%%%%%%%%%%%%%%%%%%%%%%%%%%%%%%%%%%%%%%%%%%%%%%%%%%%%%%%%%%%%%%%%%%%%%%%%%%
\subsection{Include by Input}
\label{sec:input}

Including child documents by |\include| has some restrictions by design.
Most notably, the content of a child document always occupies
its own set of pages; pages cannot be shared between child documents.
Usually, this behaviour makes perfect sense
because each child document contain an essential part of the document.
However, in some situations it may be desirable to compose
a document from a collection of parts
without having mandatory page breaks between then.
For this case, the package
provides a mechanism to include parts
by |\input| which can also be processed individually.
However, by construction this mechanism
requires manual handling of the content to be output.

%%%%%%%%%%%%%%%%%%%%%%%%%%%%%%%%%%%%%%%%
\DescribeMacro{\ifchilddocmanual}
The main file should be prepared as usual, see \secref{sec:include}.
However, the document body must make a distinction
between processing of an individual part and of the main document, e.g.:
%
\begin{center}
\begin{tabular}{l}
|\ifchilddocmanual|\\
|\input{\childdocname}|\\
|\||else|\\
\textit{document body with }|\input{|\textit{part}|}|\\
|\||fi|
\end{tabular}
\end{center}
%
The conditional |\ifchilddocmanual| is true whenever
a part to be included by |\input| is being compiled,
and the name of the part is stored in |\childdocname|.

%%%%%%%%%%%%%%%%%%%%%%%%%%%%%%%%%%%%%%%%
\DescribeMacro{\childdocby}
Each part to be included by |\input| should start with:
%
\begin{center}
\begin{tabular}{l}
|\input{childdoc.def}|\\
|\childdocby{|\textit{main}|}|\\
\end{tabular}
\end{center}
%
The directive |\childdocby| is similar to |\childdocof|
described in \secref{sec:include},
but the subsequent selection of content must be done manually.
To that end, both |\ifchilddoc| and |\ifchilddocmanual|
will be true upon processing of a part,
and the name of the part is stored in |\childdocname|.
Note that |\jobname| will be set to the filename of the current part
so that each part receives an individual |.aux| file
that does not interfere with the |.aux| file(s) of the main document.
This behaviour can be altered by the alternative form
|\childdocby[*]{|\textit{main}|}| (with a non-empty optional argument)
which uses the |.aux| file of the main document
by setting |\jobname| to \textit{main}.

%%%%%%%%%%%%%%%%%%%%%%%%%%%%%%%%%%%%%%%%%%%%%%%%%%%%%%%%%%%%%%%%%%%%%%%%%%%%%%%%
\subsection{Driver Development}
\label{sec:driver}

The \textsf{childdoc} mechanism can also be use for the development
of definition files such as \LaTeX{} styles or classes.
This case differs from the above setup with multiple parts
included by |\include| in that no |\includeonly| should be invoked.
This can be achieved by starting the include file
(before |\ProvidesPackage|) with:
%
\begin{center}
\begin{tabular}{l}
|\input{childdoc.def}|\\
|\childdocforward{|\textit{main}|}|\\
\end{tabular}
\end{center}
%
or alternatively with:
%
\begin{center}
\begin{tabular}{l}
|\input{childdoc.def}|\\
|\childdocby{|\textit{main}|}|\\
\end{tabular}
\end{center}
%
Both forms have slightly different effects as described above.
The main file is prepared as usual, see \secref{sec:include}.

%%%%%%%%%%%%%%%%%%%%%%%%%%%%%%%%%%%%%%%%%%%%%%%%%%%%%%%%%%%%%%%%%%%%%%%%%%%%%%%%
\subsection{Legacy Detection}
\label{sec:detection}

The directive |\childdocmain| in the main file can detect
whether the complete document or merely a child is to be compiled
even without using the directive |\childdocof|.
This method is deprecated because it is less robust
and there is no compelling reason to use it;
it is merely provided for backward compatibility
and it may be removed in future versions.

If the detection mechanism is to be used,
it is mandatory to correctly specify
the filename of the main file as the argument of |\childdocmain|:
%
\begin{center}
\begin{tabular}{l}
|\input{childdoc.def}|\\
|\childdocmain{|\textit{main}|}|\\
\end{tabular}
\end{center}
%
If |\jobname| does not match the argument \textit{main} of |\childdocmain|,
it is assumed that |\jobname| points to the child file to be compiled.
When using |\childdocmain| with the main file specified as argument,
it suffices to start a child file
with just |\input{|\textit{main}|}|
without loading of the package and using |\childdocof|.
If instead all processing is done
with the appropriate \textsf{childdoc} directives,
the argument of \textit{main} of |\childdocmain| can be empty.

An alternative version of the command line processing described
in \secref{sec:commandline} using the detection mechanism reads:
%
\begin{center}
|... -jobname "|\textit{target}|" "|[\textit{flags}]%
[|\def\jobname{|\textit{dest}|}|]|\input{|\textit{main}|}"|
\end{center}

%%%%%%%%%%%%%%%%%%%%%%%%%%%%%%%%%%%%%%%%%%%%%%%%%%%%%%%%%%%%%%%%%%%%%%%%%%%%%%%%
\subsection{Manual Code}
\label{sec:manual}

In case one cannot be certain whether the definitions file |childdoc.def|
is installed on the target \TeX{} distribution
and one prefers not to ship it,
it is conceivable to paste a few relevant commands into the sources.

To that end, drop all statements |\input{childdoc.def}|
and perform the replacements as outlined below.
Instead of |\childdocmain{|\textit{main}|}| add the following code
to the top of the main file:
%
\begin{center}
\begin{tabular}{l}
|\||ifdefined\childdocname\endinput\||fi\newif\ifchilddoc|\\
|\edef\childdocname{\scantokens\expandafter{\jobname\noexpand}}|\\
|\def\childdocmain{|\textit{main}|}\||ifx\childdocmain\childdocname\||else|\\
|\childdoctrue\includeonly{\childdocname}\let\jobname\childdocmain\||fi|\\
\end{tabular}
\end{center}
%
Instead of |\childdocof{|\textit{main}|}| just include the main file
at the top of each child file:
%
\begin{center}
|\input{|\textit{main}|}|
\end{center}
%
A simple redirection |\childdocforward{|\textit{dest}|}| is achieved by:
%
\begin{center}
|\def\jobname{|\textit{dest}|}\input{\jobname}|
\end{center}
%
The redirection with prefix
|\childdocforwardprefix[|\textit{prefix}|]{|\textit{dest}|}|
is accomplished by:
%
\begin{center}
\begin{tabular}{l}
|{\edef\jobname{\scantokens\expandafter{\jobname\noexpand}}|\\
|\def\redirectjob |\textit{prefix}|#1~~~{\gdef\jobname{|\textit{dest}|#1}}|\\
|\expandafter\redirectjob\jobname~~~}\input{\jobname}|
\end{tabular}
\end{center}

In an alternative approach,
child documents can be compiled by a specific command line
without additional code or specific definitions:
%
\begin{center}
|... -jobname "|\textit{target}|" "|[\textit{flags}]%
|\includeonly{|\textit{dest}|}\input{|\textit{main}|}"|
\end{center}
%

%%%%%%%%%%%%%%%%%%%%%%%%%%%%%%%%%%%%%%%%%%%%%%%%%%%%%%%%%%%%%%%%%%%%%%%%%%%%%%%%
%%%%%%%%%%%%%%%%%%%%%%%%%%%%%%%%%%%%%%%%%%%%%%%%%%%%%%%%%%%%%%%%%%%%%%%%%%%%%%%%
\section{Information}

%%%%%%%%%%%%%%%%%%%%%%%%%%%%%%%%%%%%%%%%%%%%%%%%%%%%%%%%%%%%%%%%%%%%%%%%%%%%%%%%
\subsection{Copyright}

Copyright \copyright{} 2017--2018 Niklas Beisert

This work may be distributed and/or modified under the
conditions of the \LaTeX{} Project Public License, either version 1.3
of this license or (at your option) any later version.
The latest version of this license is in
  \url{http://www.latex-project.org/lppl.txt}
and version 1.3 or later is part of all distributions of \LaTeX{}
version 2005/12/01 or later.

This work has the LPPL maintenance status `maintained'.

The Current Maintainer of this work is Niklas Beisert.

This work consists of the files |README.txt|, |childdoc.ins| and |childdoc.dtx|
as well as the derived files |childdoc.def|, |cdocsamp.tex|
with |cdocsch1.tex|, |cdocsch2.tex|, |cdocspt3.tex|, |cdocspt4.tex|,
|cdocsdrf.tex|, |cdocsfn1.tex|, |cdocsfn2.tex|
as well as |childdoc.pdf|.

%%%%%%%%%%%%%%%%%%%%%%%%%%%%%%%%%%%%%%%%%%%%%%%%%%%%%%%%%%%%%%%%%%%%%%%%%%%%%%%%
\subsection{Files and Installation}

The package consists of the files:
%
\begin{center}
\begin{tabular}{ll}
    |README.txt|   & readme file \\
    |childdoc.ins| & installation file \\
    |childdoc.dtx| & source file \\
    |childdoc.def| & definition file \\
    |cdocsamp.tex| & sample main file \\
    |cdocsch1.tex| & sample include file \\
    |cdocsch2.tex| & sample include file \\
    |cdocspt3.tex| & sample part file \\
    |cdocspt4.tex| & sample part file \\
    |cdocsdrf.tex| & sample redirection file \\
    |cdocsfn1.tex| & sample redirection file \\
    |cdocsfn2.tex| & sample redirection file \\
    |childdoc.pdf| & manual
\end{tabular}
\end{center}
%
The distribution consists of the files
|README.txt|, |childdoc.ins| and |childdoc.dtx|.
%
\begin{itemize}
\item
Run (pdf)\LaTeX{} on |childdoc.dtx|
to compile the manual |childdoc.pdf| (this file).
\item
Run \LaTeX{} on |childdoc.ins| to create the definitions file |childdoc.def|
and the sample |cdocsamp.tex| with include files
|cdocsch1.tex|, |cdocsch2.tex|, |cdocspt3.tex|, |cdocspt4.tex|,
|cdocsdrf.tex|, |cdocsfn1.tex|, |cdocsfn2.tex|.
Then copy the file |childdoc.def| to an appropriate directory of your \LaTeX{}
distribution, e.g.\ \textit{texmf-root}|/tex/latex/childdoc|.
\end{itemize}

%%%%%%%%%%%%%%%%%%%%%%%%%%%%%%%%%%%%%%%%%%%%%%%%%%%%%%%%%%%%%%%%%%%%%%%%%%%%%%%%
\subsection{Related CTAN Packages}

There are several other packages which offer a similar functionality:
%
\begin{itemize}
\item
The packages
\href{http://ctan.org/pkg/docmute}{\textsf{docmute}},
\href{http://ctan.org/pkg/includex}{\textsf{includex}} and
\href{http://ctan.org/pkg/standalone}{\textsf{standalone}}
provide commands to include only the document body of
a child file thus allowing both files to be compiled individually.
\item
The packages \href{http://ctan.org/pkg/subdocs}{\textsf{subdocs}}
and \href{http://ctan.org/pkg/subfiles}{\textsf{subfiles}}
provide structures in which the main and child documents can be
encapsulated and allowing them to be compiled individually.
The inclusion mechanism is different from the conventional |\include|.
\item
The package \href{http://ctan.org/pkg/combine}{\textsf{combine}}
is an elaborate solution to combine several documents into one.
\end{itemize}
%
See also the CTAN topic \href{http://ctan.org/topic/subdocs}{\textsf{subdocs}}
for further related packages.
The present package differs from the above solutions in that
a document structure constructed with the conventional |\include| mechanism
just needs two extra commands at the top of every file
such that all constituent files can be compiled individually.

%%%%%%%%%%%%%%%%%%%%%%%%%%%%%%%%%%%%%%%%%%%%%%%%%%%%%%%%%%%%%%%%%%%%%%%%%%%%%%%%
%\subsection{Feature Suggestions}
%
%The following is a list of features which may be useful for future
%versions of this package:
%%
%\begin{itemize}
%\item
%\ldots
%\end{itemize}

%%%%%%%%%%%%%%%%%%%%%%%%%%%%%%%%%%%%%%%%%%%%%%%%%%%%%%%%%%%%%%%%%%%%%%%%%%%%%%%%
\subsection{Revision History}

%%%%%%%%%%%%%%%%%%%%%%%%%%%%%%%%%%%%%%%%
\paragraph{v2.0:} 2018/12/30

\begin{itemize}
\item
immediate forward processing
\item
added |\childdocby| mechanism
\item
manual restructured
\end{itemize}

%%%%%%%%%%%%%%%%%%%%%%%%%%%%%%%%%%%%%%%%
\paragraph{v1.6:} 2018/01/17

\begin{itemize}
\item
application for development of include files
\item
corrections to manual
\end{itemize}

%%%%%%%%%%%%%%%%%%%%%%%%%%%%%%%%%%%%%%%%
\paragraph{v1.5:} 2017/05/21

\begin{itemize}
\item
more complete structuring introduced
\item
|\childdocof| introduced
\item
|\childdoc| renamed to |\childdocmain|
\item
|\childredirect| renamed to |\childdocforward| and |\childdocforwardprefix|
and functionality expanded
\end{itemize}

%%%%%%%%%%%%%%%%%%%%%%%%%%%%%%%%%%%%%%%%
\paragraph{v1.0:} 2017/04/27

\begin{itemize}
\item
manual and install package
\item
first version published on CTAN
\end{itemize}

%%%%%%%%%%%%%%%%%%%%%%%%%%%%%%%%%%%%%%%%
\paragraph{v0.6:} 2017/04/26

\begin{itemize}
\item
redirection mechanism added
\end{itemize}

%%%%%%%%%%%%%%%%%%%%%%%%%%%%%%%%%%%%%%%%
\paragraph{v0.5:} 2017/04/26

\begin{itemize}
\item
functionality in definition file
\end{itemize}


%%%%%%%%%%%%%%%%%%%%%%%%%%%%%%%%%%%%%%%%%%%%%%%%%%%%%%%%%%%%%%%%%%%%%%%%%%%%%%%%
%%%%%%%%%%%%%%%%%%%%%%%%%%%%%%%%%%%%%%%%%%%%%%%%%%%%%%%%%%%%%%%%%%%%%%%%%%%%%%%%
%%%%%%%%%%%%%%%%%%%%%%%%%%%%%%%%%%%%%%%%%%%%%%%%%%%%%%%%%%%%%%%%%%%%%%%%%%%%%%%%
\appendix

\settowidth\MacroIndent{\rmfamily\scriptsize 000\ }

 \DocInput{childdoc.dtx}

\end{document}
%</driver>
% \fi
%
% %%%%%%%%%%%%%%%%%%%%%%%%%%%%%%%%%%%%%%%%%%%%%%%%%%%%%%%%%%%%%%%%%%%%%%%%%%%%%%
% %%%%%%%%%%%%%%%%%%%%%%%%%%%%%%%%%%%%%%%%%%%%%%%%%%%%%%%%%%%%%%%%%%%%%%%%%%%%%%
% \section{Sample}
%\iffalse
%<*samplemain>
%\fi
%
% The following presents a sample document
% with two chapters, two parts, a title page,
% a compile flag as well as three forwarding files to set the flag.
% It consists of eight |.tex| files:
% \begin{center}
% \begin{tabular}{ll}
% |cdocsamp.tex|&main file\\
% |cdocsch1.tex|&include file for chapter 1\\
% |cdocsch2.tex|&include file for chapter 2\\
% |cdocspt3.tex|&include file for part 3\\
% |cdocspt4.tex|&include file for part 4\\
% |cdocsdrf.tex|&forwarding file for main file in draft mode\\
% |cdocsfi1.tex|&forwarding file for final version of chapter 1\\
% |cdocsfi2.tex|&forwarding file for final version of chapter 2\\
% \end{tabular}
% \end{center}
% Each of the eight files can be compiled directly by the \LaTeX{} compiler.
%
% %%%%%%%%%%%%%%%%%%%%%%%%%%%%%%%%%%%%%%
% \paragraph{Main File.}
%
% The main file is called |cdocsamp.tex|.
%
% Load the \textsf{childdoc} definitions and
% declare the filename for the main document:
%    \begin{macrocode}
\input{childdoc.def}
\childdocmain{}
%    \end{macrocode}

% Optional override for |\version| flag:
%    \begin{macrocode}
%%\ifchilddoc\else\providecommand{\version}{draft}\fi
%    \end{macrocode}

% Define the default values for the |\version| flag
% (|final| for the main file and |draft| for childs):
%    \begin{macrocode}
\ifchilddoc
\providecommand{\version}{draft}
\else
\providecommand{\version}{final}
\fi
%    \end{macrocode}

% Load the standard document class:
%    \begin{macrocode}
\documentclass[12pt]{article}
%    \end{macrocode}

% Start the document body:
%    \begin{macrocode}
\begin{document}
%    \end{macrocode}

% Declare a title page.
% Print title, part of document being processed and version flag:
%    \begin{macrocode}
\addtocounter{page}{-1}
\begin{center}
{\LARGE\bfseries{}childdoc example\par}
\vspace{1cm}
\ifchilddoc
\ifchilddocmanual part\else chapter\fi:
`\childdocname' of `\childdocjob'\par
\else
main document: `\childdocjob'\par
\fi
version: \version\par
\end{center}
\newpage
%    \end{macrocode}

% Manually include selected file,
% otherwise process as usual:
%    \begin{macrocode}
\ifchilddocmanual
\section*{part `\childdocname'}
\input{\childdocname}
\else
%    \end{macrocode}

% Include the two chapters:
%    \begin{macrocode}
\include{cdocsch1}
\include{cdocsch2}
%    \end{macrocode}

% Include the two parts unless only chapters should be displayed:
%    \begin{macrocode}
\ifchilddoc\else
\section{part three}
\input{cdocspt3}
\section{part four}
\input{cdocspt4}
\fi
%    \end{macrocode}

% Process as usual until here:
%    \begin{macrocode}
\fi
%    \end{macrocode}

% End of document body:
%    \begin{macrocode}
\end{document}
%    \end{macrocode}
%\iffalse
%</samplemain>
%\fi
%
% %%%%%%%%%%%%%%%%%%%%%%%%%%%%%%%%%%%%%%
% \paragraph{Chapter Include Files.}
%
% The include files are called |cdocsch1.tex| and |cdocsch2.tex|.
%
%\iffalse
%<*samplechap1|samplechap2>
%\fi

% Optional override for |\version| flag:
%    \begin{macrocode}
%%\providecommand{\version}{final}
%    \end{macrocode}

% Include the main document:
%    \begin{macrocode}
\input{childdoc.def}
\childdocof{cdocsamp}
%    \end{macrocode}

%\iffalse
%</samplechap1|samplechap2>
%\fi
%
%\iffalse
%<*samplechap1>
%\fi
% Some text for chapter 1:
%    \begin{macrocode}
\section{one}
some text in chapter one
%    \end{macrocode}

%\iffalse
%</samplechap1>
%\fi
% Some text for chapter 2:
%\iffalse
%<*samplechap2>
%\fi
%    \begin{macrocode}
\section{two}
more text in chapter two
%    \end{macrocode}

%\iffalse
%</samplechap2>
%\fi
%
% %%%%%%%%%%%%%%%%%%%%%%%%%%%%%%%%%%%%%%
% \paragraph{Part Include Files.}
%
% The include files are called |cdocspt3.tex| and |cdocspt4.tex|.
%
%\iffalse
%<*samplepart3|samplepart4>
%\fi

% Optional override for |\version| flag:
%    \begin{macrocode}
%%\providecommand{\version}{final}
%    \end{macrocode}

% Include the main document:
%    \begin{macrocode}
\input{childdoc.def}
\childdocby{cdocsamp}
%    \end{macrocode}

%\iffalse
%</samplepart3|samplepart4>
%\fi
%
%\iffalse
%<*samplepart3>
%\fi
% Some text for part 3:
%    \begin{macrocode}
some text in part three
%    \end{macrocode}

%\iffalse
%</samplepart3>
%\fi
% Some text for part 4:
%\iffalse
%<*samplepart4>
%\fi
%    \begin{macrocode}
more text in part four
%    \end{macrocode}

%\iffalse
%</samplepart4>
%\fi
%
% %%%%%%%%%%%%%%%%%%%%%%%%%%%%%%%%%%%%%%
% \paragraph{Forwarding for a Complete Draft.}
%
% The following forwarding file |cdocsdrf.tex|
% compiles the main document in draft mode:
%\iffalse
%<*sampledraft>
%\fi
%    \begin{macrocode}
\def\version{draft}
\input{childdoc.def}
\childdocforward{cdocsamp}
%    \end{macrocode}

%\iffalse
%</sampledraft>
%\fi
%
% %%%%%%%%%%%%%%%%%%%%%%%%%%%%%%%%%%%%%%
% \paragraph{Forwarding for Final Version of the Chapters.}
%
% The following forwarding files |cdocsfn1.tex| and |cdocsfn2.tex|
% (with identical content)
% compile the final versions of the child documents
% |cdocsch1.tex| and |cdocsch2.tex|, respectively:
%\iffalse
%<*samplefinal>
%\fi
%    \begin{macrocode}
\def\version{final}
\input{childdoc.def}
\childdocforwardprefix[cdocsamp]{cdocsfn}{cdocsch}
%    \end{macrocode}

%\iffalse
%</samplefinal>
%\fi
%
% %%%%%%%%%%%%%%%%%%%%%%%%%%%%%%%%%%%%%%
% \paragraph{Command Line Processing.}
%
% The following three command lines generate the output files
% |cdocscld|, |cdocscl1| and |cdocscl2|
% which should be identical to
% |cdocsdrf|, |cdocsch1| and |cdocsfn2|, respectively:
% \begin{center}
% \begin{tabular}{l}
% |latex -jobname cdocscld \|\\
% |  "\def\version{draft}\input{childdoc.def}\childdocforward{cdocsamp}"|\\
% |latex -jobname cdocscl1 \|\\
% |  "\input{childdoc.def}\childdocforward[cdocsamp]{cdocsch1}"|\\
% |latex -jobname cdocscl2 \|\\
% |  "\def\version{final}\input{childdoc.def}\childdocforward{cdocsch2}"|
% \end{tabular}
% \end{center}
% Note that the trailing backslash on each first line
% merely continues the input to the second line
% (for convenient cut ant paste).
% Furthermore, the command |latex| can be replaced by any
% of its alternative versions such as |pdflatex|.
%
% %%%%%%%%%%%%%%%%%%%%%%%%%%%%%%%%%%%%%%%%%%%%%%%%%%%%%%%%%%%%%%%%%%%%%%%%%%%%%%
% %%%%%%%%%%%%%%%%%%%%%%%%%%%%%%%%%%%%%%%%%%%%%%%%%%%%%%%%%%%%%%%%%%%%%%%%%%%%%%
% \section{Implementation}
%\iffalse
%<*package>
%\fi
%
% This section describes the definitions file |childdoc.def|.

% The definitions cannot be loaded using |\usepackage| or |\RequirePackage|
% which has a mechanism to prevent loading a style file more than once.
% When loading the definitions by means of |\input|
% multiple instances have to be prevented manually:
%\iffalse
%This code needs to be before the `\ProvidesFile' directive
%which is defined at the beginning of this file.
%Therefore it is also placed there and commented out here.
%</package>
%<*discard>
%\fi
%    \begin{macrocode}
\ifdefined\childdocmain\endinput\fi
%    \end{macrocode}
%\iffalse
%</discard>
%<*package>
%\fi
%
% \macro{\ifchilddoc}
% \macro{\ifchilddocmanual}
% The conditional |\ifchilddoc| tells whether a
% child (true) or main (false) document is being compiled.
% The conditional |\ifchilddocmanual| tells whether
% the |\includeonly| mechanism is used (false) or
% the selection of child files must be performed manually (true).
% The definitions initialise to false:
%    \begin{macrocode}
\newif\ifchilddoc
\newif\ifchilddocmanual
%    \end{macrocode}

% \macro{\childdocname}
% \macro{\childdocjob}
% The macro |\childdocname| stores the name of the main document
% to be compiled. The macro |\childdocjob| stores the name of
% the document on which the \LaTeX{} compiler was originally invoked.
% The content of |\jobname| cannot be compared
% to filenames specified in the source due to different catcodes.
% The following code rescans |\jobname|, stores the result
% in |\childdocname| and saves a copy in |\childdocjob|:
%    \begin{macrocode}
\edef\childdocname{\scantokens\expandafter{\jobname\noexpand}}
\let\childdocjob\childdocname
%    \end{macrocode}

% \macro{\childdocdisable}
% The macro |\childdocdisable| prevents the main file
% from being processed more than once.
% At this stage, the main document command |\childdocmain|
% is assumed to be called once again where it should do nothing.
% Any subsequent call to it should prevent
% a secondary processing of the main document
% It overwrites the forwarding commands
% |\childdocof| and |\childdocforward|
% with empty macros to prevent further inclusions of the main document:
%    \begin{macrocode}
\newcommand{\childdocdisable}
{
  \renewcommand{\childdocmain}[1]{\renewcommand{\childdocmain}[1]{\endinput}}
  \renewcommand{\childdocof}[1]{}
  \renewcommand{\childdocby}[2][]{}
  \renewcommand{\childdocforward}[2][]{}
  \renewcommand{\childdocdisable}{}
}
%    \end{macrocode}

% \macro{\childdocmain}
% The macro |\childdocmain| is to be called at the top of the main file
% with nothing or the main filename (without extension) as argument.
% First, it breaks loops.
% If the argument is not empty and does not match |\childdocname|
% (which is set by the first inclusion of |childdoc.def|),
% |\ifchilddoc| is set to true, |\includeonly| is applied to the child file
% and |\jobname| is set to the main file
% (for proper handling of |.aux| files):
%    \begin{macrocode}
\newcommand{\childdocmain}[1]
{
  \childdocdisable\childdocmain{}
  \if?#1?\else
    \begingroup
      \def\childdoctmp{#1}
      \ifx\childdoctmp\childdocname
        \def\childdoctmp{}
      \else
        \def\childdoctmp
        {
          \childdoctrue
          \includeonly{\childdocname}
          \def\childdocjob{#1}
          \def\jobname{#1}
        }
      \fi
      \expandafter
    \endgroup
    \childdoctmp
  \fi
}
%    \end{macrocode}

% \macro{\childdocof}
% The command |\childdocof| redirects
% compilation to the main file |#1|.
%    \begin{macrocode}
\newcommand{\childdocof}[1]
{
  \childdocdisable
  \childdoctrue
  \includeonly{\childdocname}
  \def\jobname{#1}
  \def\childdocjob{#1}
  \input{#1}
}
%    \end{macrocode}

% \macro{\childdocby}
% The command |\childdocby| ....
%    \begin{macrocode}
\newcommand{\childdocby}[2][]
{
  \childdocdisable
  \childdoctrue
  \childdocmanualtrue
  \if?#1?\else
    \def\jobname{#2}
  \fi
  \def\childdocjob{#2}
  \input{#2}
  \endinput
}
%    \end{macrocode}

% \macro{\childdocforward}
% The command |\childdocforward| redirects
% compilation to the main file or
% (if the optional argument is given) a child file.
% Parameters are set as if the main file
% or a child file starting with |\childdocof| was compiled.
% Then compilation is handed over to the main file:
%    \begin{macrocode}
\newcommand{\childdocforward}[2][]
{
  \begingroup
    \if?#1?
      \def\childdoctmp
      {
        \def\childdocname{#2}
        \def\childdocjob{#2}
        \def\jobname{#2}
        \input{#2}
        \endinput
      }
    \else
      \def\childdoctmp
      {
        \childdocdisable
        \def\childdocname{#2}
        \childdoctrue
        \includeonly{#2}
        \def\childdocjob{#1}
        \def\jobname{#1}
        \input{#1}
        \endinput
      }
    \fi
    \expandafter
  \endgroup
  \childdoctmp
}
%    \end{macrocode}

% \macro{\childdocforwardprefix}
% The command |\childdocforwardprefix| redirects
% compilation to the main or a child file by means of a pattern.
% The prefix |#1| in the current filename is replaced by |#2|
% and the suffix of the current filename is kept
% (it is assumed that the filename does not contain the substring `|~~~|'
% which is used as a delimiter).
% Compilation is handed over to the new file by |\childdocforward|:
%    \begin{macrocode}
\newcommand{\childdocforwardprefix}[3][]
{
  \begingroup
    \def\childdocextract #2##1~~~{\def\childdoctmp{\childdocforward[#1]{#3##1}}}
    \expandafter\childdocextract\childdocname~~~
    \expandafter
  \endgroup
  \childdoctmp
}
%    \end{macrocode}

% \macro{\childdoc}
% The deprecated macro |\childdoc| is a legacy version of |\childdocmain|:
%    \begin{macrocode}
\newcommand{\childdoc}{\childdocmain}
%    \end{macrocode}

% \macro{\childdocredirect}
% The deprecated macro |\childdocredirect| is a legacy version
% of |\childdocforward| and |\childdocforwardprefix|:
%    \begin{macrocode}
\newcommand{\childdocredirect}[2][]
{
  \begingroup
    \if?#1?
      \def\childdoctmp{\childdocforward{#2}}
    \else
      \def\childdoctmp{\childdocforwardprefix{#1}{#2}}
    \fi
    \expandafter
  \endgroup
  \childdoctmp
}
%    \end{macrocode}

%\iffalse
%</package>
%\fi
%
\endinput

\childdocforwardprefix[cdocsamp]{cdocsfn}{cdocsch}
%    \end{macrocode}

%\iffalse
%</samplefinal>
%\fi
%
% %%%%%%%%%%%%%%%%%%%%%%%%%%%%%%%%%%%%%%
% \paragraph{Command Line Processing.}
%
% The following three command lines generate the output files
% |cdocscld|, |cdocscl1| and |cdocscl2|
% which should be identical to
% |cdocsdrf|, |cdocsch1| and |cdocsfn2|, respectively:
% \begin{center}
% \begin{tabular}{l}
% |latex -jobname cdocscld \|\\
% |  "\def\version{draft}% \iffalse
%
% childdoc.dtx Copyright (C) 2017-2018 Niklas Beisert
%
% This work may be distributed and/or modified under the
% conditions of the LaTeX Project Public License, either version 1.3
% of this license or (at your option) any later version.
% The latest version of this license is in
%   http://www.latex-project.org/lppl.txt
% and version 1.3 or later is part of all distributions of LaTeX
% version 2005/12/01 or later.
%
% This work has the LPPL maintenance status `maintained'.
%
% The Current Maintainer of this work is Niklas Beisert.
%
% This work consists of the files childdoc.dtx and childdoc.ins
% and the derived files childdoc.def and cdocsamp.tex with
% cdocsch1.tex, cdocsch2.tex, cdocsdrf.tex, cdocsfn1.tex, cdocsfn2.tex.
%
%<package>\ifdefined\childdocmain\endinput\fi
%<package>\ProvidesFile{childdoc.def}[2018/12/30 v2.0 child document driver]
%<samplemain>\ProvidesFile{cdocsamp.tex}[2018/12/30 v2.0 sample for childdoc]
%<*driver>
%\ProvidesFile{childdoc.drv}[2018/12/30 v2.0 childdoc reference manual file]
\PassOptionsToClass{10pt,a4paper}{article}
\documentclass{ltxdoc}

\usepackage[margin=35mm]{geometry}
\usepackage{hyperref}
\usepackage{hyperxmp}
\usepackage[usenames]{color}

\hypersetup{colorlinks=true}
\hypersetup{pdfstartview=FitH}
\hypersetup{pdfpagemode=UseNone}
\hypersetup{pdfsource={}}
\hypersetup{pdflang={en-UK}}
\hypersetup{pdfcopyright={Copyright 2017-2018 Niklas Beisert.
  This work may be distributed and/or modified under the
  conditions of the LaTeX Project Public License, either version 1.3
  of this license or (at your option) any later version.}}
\hypersetup{pdflicenseurl={http://www.latex-project.org/lppl.txt}}
\hypersetup{pdfcontactaddress={ETH Zurich, ITP, HIT K,
  Wolfgang-Pauli-Strasse 27}}
\hypersetup{pdfcontactpostcode={8093}}
\hypersetup{pdfcontactcity={Zurich}}
\hypersetup{pdfcontactcountry={Switzerland}}
\hypersetup{pdfcontactemail={nbeisert@itp.phys.ethz.ch}}
\hypersetup{pdfcontacturl={http://people.phys.ethz.ch/\xmptilde nbeisert/}}

\newcommand{\secref}[1]{\hyperref[#1]{section \ref*{#1}}}

\parskip1ex
\parindent0pt
\let\olditemize\itemize
\def\itemize{\olditemize\parskip0pt}

\begin{document}

\title{The \textsf{childdoc} Package}
\hypersetup{pdftitle={The childdoc Package}}
\author{Niklas Beisert\\[2ex]
  Institut f\"ur Theoretische Physik\\
  Eidgen\"ossische Technische Hochschule Z\"urich\\
  Wolfgang-Pauli-Strasse 27, 8093 Z\"urich, Switzerland\\[1ex]
  \href{mailto:nbeisert@itp.phys.ethz.ch}
  {\texttt{nbeisert@itp.phys.ethz.ch}}}
\hypersetup{pdfauthor={Niklas Beisert}}
\hypersetup{pdfsubject={Manual for the LaTeX2e Package childdoc}}
\date{30 December 2018, \textsf{v2.0}}
\maketitle

\begin{abstract}\noindent
\textsf{childdoc} is a \LaTeXe{} package
that enables the direct compilation
of document sections included by |\include|
to individual files.
\end{abstract}

\begingroup
\parskip0ex
\tableofcontents
\endgroup

%%%%%%%%%%%%%%%%%%%%%%%%%%%%%%%%%%%%%%%%%%%%%%%%%%%%%%%%%%%%%%%%%%%%%%%%%%%%%%%%
%%%%%%%%%%%%%%%%%%%%%%%%%%%%%%%%%%%%%%%%%%%%%%%%%%%%%%%%%%%%%%%%%%%%%%%%%%%%%%%%
\section{Introduction}

\LaTeX{} provides a mechanism to structure a large document (such as a book)
into a main file and several child files (containing the chapters)
using the |\include| command.
This mechanism is beneficial for documents
which span hundreds of pages in order to
make the source file(s) more manageable.
Moreover, compilation can be restricted to
selected child files by means of the |\includeonly| command.
The latter feature can be used to reduce the compilation time while editing
(this was significantly more useful in the earlier days of \LaTeX{})
or to generate a smaller document which is easier to navigate.
Another application of |\includeonly| is to generate
documents consisting of selected parts of the complete document.

However, there are a few drawbacks of the plain |\include| mechanism:
\begin{itemize}
\item
The child files cannot be compiled on their own,
they can only be compiled via the main file.
A naive editing environment
(such as a text editor with an option
to have the current file processed by \LaTeX)
may require one to switch to the main file before compiling;
attempting to compile the child file produces errors.
\item
The main file must be modified (each time)
to adjust the |\includeonly| command
to the present needs. This easily leaves the main file in a messy state.
\item
The generated document will always carry the filename
of the main document. This is inconvenient if
several child files are to be compiled and
to be kept for distribution.
\end{itemize}

The present package provides a simple interface
to make child files individually compilable by \LaTeX{}.
Compiling a child file then has the same effect as compiling
the main file with an |\includeonly| command
to select the appropriate child.
Moreover the generated document will carry the name of the child
rather than the main file.
This resolves all three above issues.

This feature is meant to make the editing of books,
thesis documents and lecture notes somewhat more convenient.
However, the package can also be used efficiently for
composing a series of documents (such as exercise sheets)
which are typically distributed individually.
It then assists the author in generating the individual documents
(potentially in different versions)
as well as a document containing the collected series.
Another application is in developing style files
or other kinds of included material
where compilation of the style file could redirect
to a sample or test file.

%%%%%%%%%%%%%%%%%%%%%%%%%%%%%%%%%%%%%%%%%%%%%%%%%%%%%%%%%%%%%%%%%%%%%%%%%%%%%%%%
%%%%%%%%%%%%%%%%%%%%%%%%%%%%%%%%%%%%%%%%%%%%%%%%%%%%%%%%%%%%%%%%%%%%%%%%%%%%%%%%
\section{Usage}

First of all, the package \textsf{childdoc} is \emph{not} a standard
\LaTeXe{} |.sty| style file! Therefore it needs to be invoked in
a non-standard way.

%%%%%%%%%%%%%%%%%%%%%%%%%%%%%%%%%%%%%%%%%%%%%%%%%%%%%%%%%%%%%%%%%%%%%%%%%%%%%%%%
\subsection{Included Files}
\label{sec:include}

%%%%%%%%%%%%%%%%%%%%%%%%%%%%%%%%%%%%%%%%
\DescribeMacro{\childdocmain}
To use the package, add the commands
\begin{center}
\begin{tabular}{l}
|\input{childdoc.def}|\\
|\childdocmain{}|\\
\end{tabular}
\end{center}
at the very top of the main \LaTeX{} file,
in particular \emph{before} the |\documentclass| statement!
The argument of |\childdocmain| should be left empty
(but it must be present).

%%%%%%%%%%%%%%%%%%%%%%%%%%%%%%%%%%%%%%%%
\DescribeMacro{\childdocof}
Furthermore, add the commands
\begin{center}
\begin{tabular}{l}
|\input{childdoc.def}|\\
|\childdocof{|\textit{main}|}|\\
\end{tabular}
\end{center}
at the top of every child file \textit{child}
which is included by |\include{|\textit{child}|}|
from within the main file
(or at least for those files to be compiled individually).
The argument \textit{main} must be the filename of the main file.

There are a couple of
considerations in setting up the main and child documents:

%%%%%%%%%%%%%%%%%%%%%%%%%%%%%%%%%%%%%%%%
\paragraph{Restrictions.}

Please note the following restrictions:
\begin{itemize}
\item
|\childdocmain| must be called with one argument \textit{main}
to ensure compatibility with earlier version of the package.
It must either be empty (|\childdocmain{}|)
or precisely match the filename of the main file in which it is specified.
See \secref{sec:detection} for further information.
\item
The filename \textit{main} must be specified without the |.tex| extension.
\item
The filename \textit{main} is case sensitive
(even in case-insensitive file systems)
due to internal string comparison.
\item
The argument \textit{main} should be fully expanded, it cannot be a macro.
\item
Subdirectories and special characters should be avoided in filenames.
\item
The command |\childdocmain{|\textit{main}|}| must be followed by a whitespace.
It should not be followed immediately by another command
or by a comment mark `|%|'.
This is because the \TeX{} parser reads the token immediately following
the argument of |\childdocmain| and puts it
at the beginning of every child section;
however, a white\-space is ignored.
\end{itemize}

%%%%%%%%%%%%%%%%%%%%%%%%%%%%%%%%%%%%%%%%
\paragraph{Content of Main File.}

It is advisable to place all content in the child files included by |\include|.
Any output contained in the main file will appear in all child documents
unless suppressed manually;
it cannot be suppressed automatically by the |\includeonly| directive
and thus should normally be avoided.
A method to include some content in the main file
by means of conditional processing is described in \secref{sec:conditional}.

%%%%%%%%%%%%%%%%%%%%%%%%%%%%%%%%%%%%%%%%
\paragraph{Page Numbering.}

When only a part of the document is compiled,
the appropriate numbering of pages
(as well as other status parameters)
is determined from the |.aux| files.
The latter contain information from previous passes.
However this information needs to propagate through
all intermediate child documents.
Therefore the page numbering in child documents may well
be inconsistent until the complete document is compiled at least once.

A useful (if unconventional) way to always ensure a consistent
page numbering is to restart the numbering in each child document
and denote the pages by `\textit{child}|.|\textit{page}'
where \textit{child} represents the chapter/section number of the child file.
This can be achieved by the command
|\numberwithin{page}{|\textit{child}|}|
of the \textsf{amsmath} package
where \textit{child} can be |chapter| or |section|
depending on the chosen structuring.
Alternatively, one can modify the macro |\thepage| appropriately
and reset the counter |page| at the start of each child file.

%%%%%%%%%%%%%%%%%%%%%%%%%%%%%%%%%%%%%%%%%%%%%%%%%%%%%%%%%%%%%%%%%%%%%%%%%%%%%%%%
\subsection{Conditional Processing}
\label{sec:conditional}

The package provides a mechanism to compile different versions
of a document. To customise the versions further some conditional processing
can come in handy to distinguish which version is being compiled.
The package provides two macros to describe the compilation context:

%%%%%%%%%%%%%%%%%%%%%%%%%%%%%%%%%%%%%%%%
\DescribeMacro{\ifchilddoc}
The conditional |\ifchilddoc| distinguishes between the compilation of
child documents and the main document:
%
\begin{center}
|\ifchilddoc |\textit{child-code}| |[|\||else |\textit{main-code}]| \||fi|
\end{center}

%%%%%%%%%%%%%%%%%%%%%%%%%%%%%%%%%%%%%%%%
\DescribeMacro{\childdocname}
\DescribeMacro{\childdocjob}
The macro |\childdocname| contains the filename (without extension)
of the main or child file being processed.
Note that |\childdocjob| will always contain the name of the main file.

%%%%%%%%%%%%%%%%%%%%%%%%%%%%%%%%%%%%%%%%
\paragraph{Title Page.}

Conditional processing can be used to include a title or banner page
in the main document when proper precautions are taken.
Importantly, the code in the main file should ensure that the page counter
(as well as other status parameters which are stored in the |.aux| files)
takes the same value after the conditional processing.
Otherwise the page numbers may take divergent values
depending on which part is compiled.

For example, a title page could be declared by:
%
\begin{center}
\begin{tabular}{l}
|\ifchilddoc\||else|\\
|\addtocounter{page}{-1}|\\
\textit{code for title page}\\
|\newpage|\\
|\||fi|
\end{tabular}
\end{center}
%
A banner page for the child documents can be generated by:
%
\begin{center}
\begin{tabular}{l}
|\ifchilddoc|\\
|\addtocounter{page}{-1}|\\
\textit{code for banner page}\\
|\newpage|\\
|\||fi|
\end{tabular}
\end{center}
%
Here one could write a message such as:
\begin{center}
|This is the part \childdocname{} of \childdocjob{}.|
\end{center}

%%%%%%%%%%%%%%%%%%%%%%%%%%%%%%%%%%%%%%%%%%%%%%%%%%%%%%%%%%%%%%%%%%%%%%%%%%%%%%%%
\subsection{Flags}
\label{sec:flags}

The package makes it easy to generate different versions
of the main or child documents.
To this end compilation flags can be defined
and assigned different default values.
They will be particularly useful in conjunction
with the forwarding mechanism described in \secref{sec:forward}.

For example, it may be useful to have a flag |\version|
which can be set to |draft| or |final|.
The document source will contain some conditional code
depending on the value of |\version|.
Suppose further, the flag should default to |final| for the main file
and to |draft| for child files
which is a natural assignment for editing the document.
This is achieved by placing the following code
in the preamble of the main document
(below the |\childdocmain| directive):
%
\begin{center}
\begin{tabular}{l}
|\ifchilddoc|\\
|\providecommand{\version}{draft}|\\
|\||else|\\
|\providecommand{\version}{final}|\\
|\||fi|
\end{tabular}
\end{center}
%
The definition by |\providecommand| makes sure
that previous definitions are not overwritten.
Further statements |\providecommand{\version}{...}|
can thus be added before the above code to override it.

For the main file, one might add a line
(between |\childdocmain| and the above block)
%
\begin{center}
|%\ifchilddoc\||else\providecommand{\version}{draft}\||fi|
\end{center}
%
which can be uncommented to produce a draft version.
Likewise one can add a line to the very top of a child file
(above the |\childdocof{|\textit{main}|}| directive)
%
\begin{center}
|%\providecommand{\version}{final}|
\end{center}
%
which can be uncommented to produce the final version of this child document.

%%%%%%%%%%%%%%%%%%%%%%%%%%%%%%%%%%%%%%%%%%%%%%%%%%%%%%%%%%%%%%%%%%%%%%%%%%%%%%%%
\subsection{Forwarding}
\label{sec:forward}

Different versions of the main or child documents
using compilation flags as described in \secref{sec:flags}
can be (permanently) stored in different files
for convenient compilation, viewing and distribution.
To this end, the package defines a command
to pass on compilation to a different file:

%%%%%%%%%%%%%%%%%%%%%%%%%%%%%%%%%%%%%%%%
\DescribeMacro{\childdocforward}
The command |\childdocforward| redirects processing to
another source file:
%
\begin{center}
\begin{tabular}{l}
|\input{childdoc.def}|\\
|\childdocforward[|\textit{main}|]{|\textit{dest}|}|\\
\end{tabular}
\end{center}
%
The argument \textit{dest} is the destination file
(without extension).
It should be the main file or one of the child files.
Note that further \textsf{childdoc} directives
such as |\childdocof| and |\childdocforward|
in the indicated file will be processed in this form.
The optional argument \textit{main}
passes on directly to the main file \textit{main}
while pretending to compile the child \textit{dest}.
This form behaves as if \textit{dest}
issues |\childdocof{|\textit{main}|}| right away,
and no further \textsf{childdoc} directives will be processed.

%%%%%%%%%%%%%%%%%%%%%%%%%%%%%%%%%%%%%%%%
\DescribeMacro{\...prefix}
In the alternative form |\childdocforwardprefix|,
%
\begin{center}
\begin{tabular}{l}
|\input{childdoc.def}|\\
|\childdocforwardprefix[|\textit{main}|]{|\textit{prefix}|}{|\textit{dest}|}|
\end{tabular}
\end{center}
%
the destination file is determined by a pattern
depending on the current file:
To make this work, the current file must be called
`{\textit{prefix}\hspace{0.2em}\textit{suffix}}'
with \textit{prefix} matching precisely the argument.
Processing is then passed on to the file
`{\textit{dest}\hspace{0.2em}\textit{suffix}}'.
Surely, the same effect is achieved by
directly specifying the
argument `{\textit{dest}\hspace{0.2em}\textit{suffix}}'
in the first form.
However, that requires to set up a different file
for each child. With the alternative form of the command
all these files can have exactly the same content
which simplifies setting them up and maintaining them.

For example, the following file |draft.tex|
with a compilation flag |\version| as described in \secref{sec:flags}
compiles the main document as a draft:
%
\begin{center}
\begin{tabular}{l}
|\def\version{draft}|\\
|\input{childdoc.def}|\\
|\childdocforward{|\textit{main}|}|
\end{tabular}
\end{center}
%
Likewise, the following files |final|\textit{nn}|.tex|
compile the final version of the child document
|child|\textit{nn}|.tex|:
%
\begin{center}
\begin{tabular}{l}
|\def\version{final}|\\
|\input{childdoc.def}|\\
|\childdocforwardprefix{final}{child}|
\end{tabular}
\end{center}
%

Note that when several versions of a main file and/or of each child file
are to be generated, it may be convenient to set up a |Makefile| or
shell script to automatise the process.

%%%%%%%%%%%%%%%%%%%%%%%%%%%%%%%%%%%%%%%%%%%%%%%%%%%%%%%%%%%%%%%%%%%%%%%%%%%%%%%%
\subsection{Command Line Processing}
\label{sec:commandline}

The effect of redirection files can also be achieved by invoking
the \LaTeX{} compiler with a more elaborate command line.
Most conveniently this should be done as part
of a shell script or a |Makefile|.

When using \textsf{childdoc} in the main file, the following
command lines effectively perform a redirection
(note that depending on the shell being used,
backslashes may have to be doubled: `|\|' $\to$ `|\\|'):
%
\begin{center}
|... -jobname "|\textit{target}|" |\\|"|[\textit{flags}]%
|\input{childdoc.def}\childdocforward[|\textit{main}|]{|\textit{dest}|}"|
\end{center}
%
Here \textit{target} is the name of the output file,
\textit{main} is the name of the main file
and \textit{dest} is the name of the main or child file to be processed
(all filenames without extensions).
The optional argument \textit{main} can be omitted
if \textit{main} matches \textit{dest}.
Optionally, compilation \textit{flags} can be defined via |\def| commands.
This command line makes the \TeX{} engine believe
it is compiling the file \textit{target}
whose content is specified as the latter parameter.
The provided code then forwards the processing to
\textit{main} or \textit{dest} as described in \secref{sec:forward}.

%%%%%%%%%%%%%%%%%%%%%%%%%%%%%%%%%%%%%%%%%%%%%%%%%%%%%%%%%%%%%%%%%%%%%%%%%%%%%%%%
\subsection{Include by Input}
\label{sec:input}

Including child documents by |\include| has some restrictions by design.
Most notably, the content of a child document always occupies
its own set of pages; pages cannot be shared between child documents.
Usually, this behaviour makes perfect sense
because each child document contain an essential part of the document.
However, in some situations it may be desirable to compose
a document from a collection of parts
without having mandatory page breaks between then.
For this case, the package
provides a mechanism to include parts
by |\input| which can also be processed individually.
However, by construction this mechanism
requires manual handling of the content to be output.

%%%%%%%%%%%%%%%%%%%%%%%%%%%%%%%%%%%%%%%%
\DescribeMacro{\ifchilddocmanual}
The main file should be prepared as usual, see \secref{sec:include}.
However, the document body must make a distinction
between processing of an individual part and of the main document, e.g.:
%
\begin{center}
\begin{tabular}{l}
|\ifchilddocmanual|\\
|\input{\childdocname}|\\
|\||else|\\
\textit{document body with }|\input{|\textit{part}|}|\\
|\||fi|
\end{tabular}
\end{center}
%
The conditional |\ifchilddocmanual| is true whenever
a part to be included by |\input| is being compiled,
and the name of the part is stored in |\childdocname|.

%%%%%%%%%%%%%%%%%%%%%%%%%%%%%%%%%%%%%%%%
\DescribeMacro{\childdocby}
Each part to be included by |\input| should start with:
%
\begin{center}
\begin{tabular}{l}
|\input{childdoc.def}|\\
|\childdocby{|\textit{main}|}|\\
\end{tabular}
\end{center}
%
The directive |\childdocby| is similar to |\childdocof|
described in \secref{sec:include},
but the subsequent selection of content must be done manually.
To that end, both |\ifchilddoc| and |\ifchilddocmanual|
will be true upon processing of a part,
and the name of the part is stored in |\childdocname|.
Note that |\jobname| will be set to the filename of the current part
so that each part receives an individual |.aux| file
that does not interfere with the |.aux| file(s) of the main document.
This behaviour can be altered by the alternative form
|\childdocby[*]{|\textit{main}|}| (with a non-empty optional argument)
which uses the |.aux| file of the main document
by setting |\jobname| to \textit{main}.

%%%%%%%%%%%%%%%%%%%%%%%%%%%%%%%%%%%%%%%%%%%%%%%%%%%%%%%%%%%%%%%%%%%%%%%%%%%%%%%%
\subsection{Driver Development}
\label{sec:driver}

The \textsf{childdoc} mechanism can also be use for the development
of definition files such as \LaTeX{} styles or classes.
This case differs from the above setup with multiple parts
included by |\include| in that no |\includeonly| should be invoked.
This can be achieved by starting the include file
(before |\ProvidesPackage|) with:
%
\begin{center}
\begin{tabular}{l}
|\input{childdoc.def}|\\
|\childdocforward{|\textit{main}|}|\\
\end{tabular}
\end{center}
%
or alternatively with:
%
\begin{center}
\begin{tabular}{l}
|\input{childdoc.def}|\\
|\childdocby{|\textit{main}|}|\\
\end{tabular}
\end{center}
%
Both forms have slightly different effects as described above.
The main file is prepared as usual, see \secref{sec:include}.

%%%%%%%%%%%%%%%%%%%%%%%%%%%%%%%%%%%%%%%%%%%%%%%%%%%%%%%%%%%%%%%%%%%%%%%%%%%%%%%%
\subsection{Legacy Detection}
\label{sec:detection}

The directive |\childdocmain| in the main file can detect
whether the complete document or merely a child is to be compiled
even without using the directive |\childdocof|.
This method is deprecated because it is less robust
and there is no compelling reason to use it;
it is merely provided for backward compatibility
and it may be removed in future versions.

If the detection mechanism is to be used,
it is mandatory to correctly specify
the filename of the main file as the argument of |\childdocmain|:
%
\begin{center}
\begin{tabular}{l}
|\input{childdoc.def}|\\
|\childdocmain{|\textit{main}|}|\\
\end{tabular}
\end{center}
%
If |\jobname| does not match the argument \textit{main} of |\childdocmain|,
it is assumed that |\jobname| points to the child file to be compiled.
When using |\childdocmain| with the main file specified as argument,
it suffices to start a child file
with just |\input{|\textit{main}|}|
without loading of the package and using |\childdocof|.
If instead all processing is done
with the appropriate \textsf{childdoc} directives,
the argument of \textit{main} of |\childdocmain| can be empty.

An alternative version of the command line processing described
in \secref{sec:commandline} using the detection mechanism reads:
%
\begin{center}
|... -jobname "|\textit{target}|" "|[\textit{flags}]%
[|\def\jobname{|\textit{dest}|}|]|\input{|\textit{main}|}"|
\end{center}

%%%%%%%%%%%%%%%%%%%%%%%%%%%%%%%%%%%%%%%%%%%%%%%%%%%%%%%%%%%%%%%%%%%%%%%%%%%%%%%%
\subsection{Manual Code}
\label{sec:manual}

In case one cannot be certain whether the definitions file |childdoc.def|
is installed on the target \TeX{} distribution
and one prefers not to ship it,
it is conceivable to paste a few relevant commands into the sources.

To that end, drop all statements |\input{childdoc.def}|
and perform the replacements as outlined below.
Instead of |\childdocmain{|\textit{main}|}| add the following code
to the top of the main file:
%
\begin{center}
\begin{tabular}{l}
|\||ifdefined\childdocname\endinput\||fi\newif\ifchilddoc|\\
|\edef\childdocname{\scantokens\expandafter{\jobname\noexpand}}|\\
|\def\childdocmain{|\textit{main}|}\||ifx\childdocmain\childdocname\||else|\\
|\childdoctrue\includeonly{\childdocname}\let\jobname\childdocmain\||fi|\\
\end{tabular}
\end{center}
%
Instead of |\childdocof{|\textit{main}|}| just include the main file
at the top of each child file:
%
\begin{center}
|\input{|\textit{main}|}|
\end{center}
%
A simple redirection |\childdocforward{|\textit{dest}|}| is achieved by:
%
\begin{center}
|\def\jobname{|\textit{dest}|}\input{\jobname}|
\end{center}
%
The redirection with prefix
|\childdocforwardprefix[|\textit{prefix}|]{|\textit{dest}|}|
is accomplished by:
%
\begin{center}
\begin{tabular}{l}
|{\edef\jobname{\scantokens\expandafter{\jobname\noexpand}}|\\
|\def\redirectjob |\textit{prefix}|#1~~~{\gdef\jobname{|\textit{dest}|#1}}|\\
|\expandafter\redirectjob\jobname~~~}\input{\jobname}|
\end{tabular}
\end{center}

In an alternative approach,
child documents can be compiled by a specific command line
without additional code or specific definitions:
%
\begin{center}
|... -jobname "|\textit{target}|" "|[\textit{flags}]%
|\includeonly{|\textit{dest}|}\input{|\textit{main}|}"|
\end{center}
%

%%%%%%%%%%%%%%%%%%%%%%%%%%%%%%%%%%%%%%%%%%%%%%%%%%%%%%%%%%%%%%%%%%%%%%%%%%%%%%%%
%%%%%%%%%%%%%%%%%%%%%%%%%%%%%%%%%%%%%%%%%%%%%%%%%%%%%%%%%%%%%%%%%%%%%%%%%%%%%%%%
\section{Information}

%%%%%%%%%%%%%%%%%%%%%%%%%%%%%%%%%%%%%%%%%%%%%%%%%%%%%%%%%%%%%%%%%%%%%%%%%%%%%%%%
\subsection{Copyright}

Copyright \copyright{} 2017--2018 Niklas Beisert

This work may be distributed and/or modified under the
conditions of the \LaTeX{} Project Public License, either version 1.3
of this license or (at your option) any later version.
The latest version of this license is in
  \url{http://www.latex-project.org/lppl.txt}
and version 1.3 or later is part of all distributions of \LaTeX{}
version 2005/12/01 or later.

This work has the LPPL maintenance status `maintained'.

The Current Maintainer of this work is Niklas Beisert.

This work consists of the files |README.txt|, |childdoc.ins| and |childdoc.dtx|
as well as the derived files |childdoc.def|, |cdocsamp.tex|
with |cdocsch1.tex|, |cdocsch2.tex|, |cdocspt3.tex|, |cdocspt4.tex|,
|cdocsdrf.tex|, |cdocsfn1.tex|, |cdocsfn2.tex|
as well as |childdoc.pdf|.

%%%%%%%%%%%%%%%%%%%%%%%%%%%%%%%%%%%%%%%%%%%%%%%%%%%%%%%%%%%%%%%%%%%%%%%%%%%%%%%%
\subsection{Files and Installation}

The package consists of the files:
%
\begin{center}
\begin{tabular}{ll}
    |README.txt|   & readme file \\
    |childdoc.ins| & installation file \\
    |childdoc.dtx| & source file \\
    |childdoc.def| & definition file \\
    |cdocsamp.tex| & sample main file \\
    |cdocsch1.tex| & sample include file \\
    |cdocsch2.tex| & sample include file \\
    |cdocspt3.tex| & sample part file \\
    |cdocspt4.tex| & sample part file \\
    |cdocsdrf.tex| & sample redirection file \\
    |cdocsfn1.tex| & sample redirection file \\
    |cdocsfn2.tex| & sample redirection file \\
    |childdoc.pdf| & manual
\end{tabular}
\end{center}
%
The distribution consists of the files
|README.txt|, |childdoc.ins| and |childdoc.dtx|.
%
\begin{itemize}
\item
Run (pdf)\LaTeX{} on |childdoc.dtx|
to compile the manual |childdoc.pdf| (this file).
\item
Run \LaTeX{} on |childdoc.ins| to create the definitions file |childdoc.def|
and the sample |cdocsamp.tex| with include files
|cdocsch1.tex|, |cdocsch2.tex|, |cdocspt3.tex|, |cdocspt4.tex|,
|cdocsdrf.tex|, |cdocsfn1.tex|, |cdocsfn2.tex|.
Then copy the file |childdoc.def| to an appropriate directory of your \LaTeX{}
distribution, e.g.\ \textit{texmf-root}|/tex/latex/childdoc|.
\end{itemize}

%%%%%%%%%%%%%%%%%%%%%%%%%%%%%%%%%%%%%%%%%%%%%%%%%%%%%%%%%%%%%%%%%%%%%%%%%%%%%%%%
\subsection{Related CTAN Packages}

There are several other packages which offer a similar functionality:
%
\begin{itemize}
\item
The packages
\href{http://ctan.org/pkg/docmute}{\textsf{docmute}},
\href{http://ctan.org/pkg/includex}{\textsf{includex}} and
\href{http://ctan.org/pkg/standalone}{\textsf{standalone}}
provide commands to include only the document body of
a child file thus allowing both files to be compiled individually.
\item
The packages \href{http://ctan.org/pkg/subdocs}{\textsf{subdocs}}
and \href{http://ctan.org/pkg/subfiles}{\textsf{subfiles}}
provide structures in which the main and child documents can be
encapsulated and allowing them to be compiled individually.
The inclusion mechanism is different from the conventional |\include|.
\item
The package \href{http://ctan.org/pkg/combine}{\textsf{combine}}
is an elaborate solution to combine several documents into one.
\end{itemize}
%
See also the CTAN topic \href{http://ctan.org/topic/subdocs}{\textsf{subdocs}}
for further related packages.
The present package differs from the above solutions in that
a document structure constructed with the conventional |\include| mechanism
just needs two extra commands at the top of every file
such that all constituent files can be compiled individually.

%%%%%%%%%%%%%%%%%%%%%%%%%%%%%%%%%%%%%%%%%%%%%%%%%%%%%%%%%%%%%%%%%%%%%%%%%%%%%%%%
%\subsection{Feature Suggestions}
%
%The following is a list of features which may be useful for future
%versions of this package:
%%
%\begin{itemize}
%\item
%\ldots
%\end{itemize}

%%%%%%%%%%%%%%%%%%%%%%%%%%%%%%%%%%%%%%%%%%%%%%%%%%%%%%%%%%%%%%%%%%%%%%%%%%%%%%%%
\subsection{Revision History}

%%%%%%%%%%%%%%%%%%%%%%%%%%%%%%%%%%%%%%%%
\paragraph{v2.0:} 2018/12/30

\begin{itemize}
\item
immediate forward processing
\item
added |\childdocby| mechanism
\item
manual restructured
\end{itemize}

%%%%%%%%%%%%%%%%%%%%%%%%%%%%%%%%%%%%%%%%
\paragraph{v1.6:} 2018/01/17

\begin{itemize}
\item
application for development of include files
\item
corrections to manual
\end{itemize}

%%%%%%%%%%%%%%%%%%%%%%%%%%%%%%%%%%%%%%%%
\paragraph{v1.5:} 2017/05/21

\begin{itemize}
\item
more complete structuring introduced
\item
|\childdocof| introduced
\item
|\childdoc| renamed to |\childdocmain|
\item
|\childredirect| renamed to |\childdocforward| and |\childdocforwardprefix|
and functionality expanded
\end{itemize}

%%%%%%%%%%%%%%%%%%%%%%%%%%%%%%%%%%%%%%%%
\paragraph{v1.0:} 2017/04/27

\begin{itemize}
\item
manual and install package
\item
first version published on CTAN
\end{itemize}

%%%%%%%%%%%%%%%%%%%%%%%%%%%%%%%%%%%%%%%%
\paragraph{v0.6:} 2017/04/26

\begin{itemize}
\item
redirection mechanism added
\end{itemize}

%%%%%%%%%%%%%%%%%%%%%%%%%%%%%%%%%%%%%%%%
\paragraph{v0.5:} 2017/04/26

\begin{itemize}
\item
functionality in definition file
\end{itemize}


%%%%%%%%%%%%%%%%%%%%%%%%%%%%%%%%%%%%%%%%%%%%%%%%%%%%%%%%%%%%%%%%%%%%%%%%%%%%%%%%
%%%%%%%%%%%%%%%%%%%%%%%%%%%%%%%%%%%%%%%%%%%%%%%%%%%%%%%%%%%%%%%%%%%%%%%%%%%%%%%%
%%%%%%%%%%%%%%%%%%%%%%%%%%%%%%%%%%%%%%%%%%%%%%%%%%%%%%%%%%%%%%%%%%%%%%%%%%%%%%%%
\appendix

\settowidth\MacroIndent{\rmfamily\scriptsize 000\ }

 \DocInput{childdoc.dtx}

\end{document}
%</driver>
% \fi
%
% %%%%%%%%%%%%%%%%%%%%%%%%%%%%%%%%%%%%%%%%%%%%%%%%%%%%%%%%%%%%%%%%%%%%%%%%%%%%%%
% %%%%%%%%%%%%%%%%%%%%%%%%%%%%%%%%%%%%%%%%%%%%%%%%%%%%%%%%%%%%%%%%%%%%%%%%%%%%%%
% \section{Sample}
%\iffalse
%<*samplemain>
%\fi
%
% The following presents a sample document
% with two chapters, two parts, a title page,
% a compile flag as well as three forwarding files to set the flag.
% It consists of eight |.tex| files:
% \begin{center}
% \begin{tabular}{ll}
% |cdocsamp.tex|&main file\\
% |cdocsch1.tex|&include file for chapter 1\\
% |cdocsch2.tex|&include file for chapter 2\\
% |cdocspt3.tex|&include file for part 3\\
% |cdocspt4.tex|&include file for part 4\\
% |cdocsdrf.tex|&forwarding file for main file in draft mode\\
% |cdocsfi1.tex|&forwarding file for final version of chapter 1\\
% |cdocsfi2.tex|&forwarding file for final version of chapter 2\\
% \end{tabular}
% \end{center}
% Each of the eight files can be compiled directly by the \LaTeX{} compiler.
%
% %%%%%%%%%%%%%%%%%%%%%%%%%%%%%%%%%%%%%%
% \paragraph{Main File.}
%
% The main file is called |cdocsamp.tex|.
%
% Load the \textsf{childdoc} definitions and
% declare the filename for the main document:
%    \begin{macrocode}
\input{childdoc.def}
\childdocmain{}
%    \end{macrocode}

% Optional override for |\version| flag:
%    \begin{macrocode}
%%\ifchilddoc\else\providecommand{\version}{draft}\fi
%    \end{macrocode}

% Define the default values for the |\version| flag
% (|final| for the main file and |draft| for childs):
%    \begin{macrocode}
\ifchilddoc
\providecommand{\version}{draft}
\else
\providecommand{\version}{final}
\fi
%    \end{macrocode}

% Load the standard document class:
%    \begin{macrocode}
\documentclass[12pt]{article}
%    \end{macrocode}

% Start the document body:
%    \begin{macrocode}
\begin{document}
%    \end{macrocode}

% Declare a title page.
% Print title, part of document being processed and version flag:
%    \begin{macrocode}
\addtocounter{page}{-1}
\begin{center}
{\LARGE\bfseries{}childdoc example\par}
\vspace{1cm}
\ifchilddoc
\ifchilddocmanual part\else chapter\fi:
`\childdocname' of `\childdocjob'\par
\else
main document: `\childdocjob'\par
\fi
version: \version\par
\end{center}
\newpage
%    \end{macrocode}

% Manually include selected file,
% otherwise process as usual:
%    \begin{macrocode}
\ifchilddocmanual
\section*{part `\childdocname'}
\input{\childdocname}
\else
%    \end{macrocode}

% Include the two chapters:
%    \begin{macrocode}
\include{cdocsch1}
\include{cdocsch2}
%    \end{macrocode}

% Include the two parts unless only chapters should be displayed:
%    \begin{macrocode}
\ifchilddoc\else
\section{part three}
\input{cdocspt3}
\section{part four}
\input{cdocspt4}
\fi
%    \end{macrocode}

% Process as usual until here:
%    \begin{macrocode}
\fi
%    \end{macrocode}

% End of document body:
%    \begin{macrocode}
\end{document}
%    \end{macrocode}
%\iffalse
%</samplemain>
%\fi
%
% %%%%%%%%%%%%%%%%%%%%%%%%%%%%%%%%%%%%%%
% \paragraph{Chapter Include Files.}
%
% The include files are called |cdocsch1.tex| and |cdocsch2.tex|.
%
%\iffalse
%<*samplechap1|samplechap2>
%\fi

% Optional override for |\version| flag:
%    \begin{macrocode}
%%\providecommand{\version}{final}
%    \end{macrocode}

% Include the main document:
%    \begin{macrocode}
\input{childdoc.def}
\childdocof{cdocsamp}
%    \end{macrocode}

%\iffalse
%</samplechap1|samplechap2>
%\fi
%
%\iffalse
%<*samplechap1>
%\fi
% Some text for chapter 1:
%    \begin{macrocode}
\section{one}
some text in chapter one
%    \end{macrocode}

%\iffalse
%</samplechap1>
%\fi
% Some text for chapter 2:
%\iffalse
%<*samplechap2>
%\fi
%    \begin{macrocode}
\section{two}
more text in chapter two
%    \end{macrocode}

%\iffalse
%</samplechap2>
%\fi
%
% %%%%%%%%%%%%%%%%%%%%%%%%%%%%%%%%%%%%%%
% \paragraph{Part Include Files.}
%
% The include files are called |cdocspt3.tex| and |cdocspt4.tex|.
%
%\iffalse
%<*samplepart3|samplepart4>
%\fi

% Optional override for |\version| flag:
%    \begin{macrocode}
%%\providecommand{\version}{final}
%    \end{macrocode}

% Include the main document:
%    \begin{macrocode}
\input{childdoc.def}
\childdocby{cdocsamp}
%    \end{macrocode}

%\iffalse
%</samplepart3|samplepart4>
%\fi
%
%\iffalse
%<*samplepart3>
%\fi
% Some text for part 3:
%    \begin{macrocode}
some text in part three
%    \end{macrocode}

%\iffalse
%</samplepart3>
%\fi
% Some text for part 4:
%\iffalse
%<*samplepart4>
%\fi
%    \begin{macrocode}
more text in part four
%    \end{macrocode}

%\iffalse
%</samplepart4>
%\fi
%
% %%%%%%%%%%%%%%%%%%%%%%%%%%%%%%%%%%%%%%
% \paragraph{Forwarding for a Complete Draft.}
%
% The following forwarding file |cdocsdrf.tex|
% compiles the main document in draft mode:
%\iffalse
%<*sampledraft>
%\fi
%    \begin{macrocode}
\def\version{draft}
\input{childdoc.def}
\childdocforward{cdocsamp}
%    \end{macrocode}

%\iffalse
%</sampledraft>
%\fi
%
% %%%%%%%%%%%%%%%%%%%%%%%%%%%%%%%%%%%%%%
% \paragraph{Forwarding for Final Version of the Chapters.}
%
% The following forwarding files |cdocsfn1.tex| and |cdocsfn2.tex|
% (with identical content)
% compile the final versions of the child documents
% |cdocsch1.tex| and |cdocsch2.tex|, respectively:
%\iffalse
%<*samplefinal>
%\fi
%    \begin{macrocode}
\def\version{final}
\input{childdoc.def}
\childdocforwardprefix[cdocsamp]{cdocsfn}{cdocsch}
%    \end{macrocode}

%\iffalse
%</samplefinal>
%\fi
%
% %%%%%%%%%%%%%%%%%%%%%%%%%%%%%%%%%%%%%%
% \paragraph{Command Line Processing.}
%
% The following three command lines generate the output files
% |cdocscld|, |cdocscl1| and |cdocscl2|
% which should be identical to
% |cdocsdrf|, |cdocsch1| and |cdocsfn2|, respectively:
% \begin{center}
% \begin{tabular}{l}
% |latex -jobname cdocscld \|\\
% |  "\def\version{draft}\input{childdoc.def}\childdocforward{cdocsamp}"|\\
% |latex -jobname cdocscl1 \|\\
% |  "\input{childdoc.def}\childdocforward[cdocsamp]{cdocsch1}"|\\
% |latex -jobname cdocscl2 \|\\
% |  "\def\version{final}\input{childdoc.def}\childdocforward{cdocsch2}"|
% \end{tabular}
% \end{center}
% Note that the trailing backslash on each first line
% merely continues the input to the second line
% (for convenient cut ant paste).
% Furthermore, the command |latex| can be replaced by any
% of its alternative versions such as |pdflatex|.
%
% %%%%%%%%%%%%%%%%%%%%%%%%%%%%%%%%%%%%%%%%%%%%%%%%%%%%%%%%%%%%%%%%%%%%%%%%%%%%%%
% %%%%%%%%%%%%%%%%%%%%%%%%%%%%%%%%%%%%%%%%%%%%%%%%%%%%%%%%%%%%%%%%%%%%%%%%%%%%%%
% \section{Implementation}
%\iffalse
%<*package>
%\fi
%
% This section describes the definitions file |childdoc.def|.

% The definitions cannot be loaded using |\usepackage| or |\RequirePackage|
% which has a mechanism to prevent loading a style file more than once.
% When loading the definitions by means of |\input|
% multiple instances have to be prevented manually:
%\iffalse
%This code needs to be before the `\ProvidesFile' directive
%which is defined at the beginning of this file.
%Therefore it is also placed there and commented out here.
%</package>
%<*discard>
%\fi
%    \begin{macrocode}
\ifdefined\childdocmain\endinput\fi
%    \end{macrocode}
%\iffalse
%</discard>
%<*package>
%\fi
%
% \macro{\ifchilddoc}
% \macro{\ifchilddocmanual}
% The conditional |\ifchilddoc| tells whether a
% child (true) or main (false) document is being compiled.
% The conditional |\ifchilddocmanual| tells whether
% the |\includeonly| mechanism is used (false) or
% the selection of child files must be performed manually (true).
% The definitions initialise to false:
%    \begin{macrocode}
\newif\ifchilddoc
\newif\ifchilddocmanual
%    \end{macrocode}

% \macro{\childdocname}
% \macro{\childdocjob}
% The macro |\childdocname| stores the name of the main document
% to be compiled. The macro |\childdocjob| stores the name of
% the document on which the \LaTeX{} compiler was originally invoked.
% The content of |\jobname| cannot be compared
% to filenames specified in the source due to different catcodes.
% The following code rescans |\jobname|, stores the result
% in |\childdocname| and saves a copy in |\childdocjob|:
%    \begin{macrocode}
\edef\childdocname{\scantokens\expandafter{\jobname\noexpand}}
\let\childdocjob\childdocname
%    \end{macrocode}

% \macro{\childdocdisable}
% The macro |\childdocdisable| prevents the main file
% from being processed more than once.
% At this stage, the main document command |\childdocmain|
% is assumed to be called once again where it should do nothing.
% Any subsequent call to it should prevent
% a secondary processing of the main document
% It overwrites the forwarding commands
% |\childdocof| and |\childdocforward|
% with empty macros to prevent further inclusions of the main document:
%    \begin{macrocode}
\newcommand{\childdocdisable}
{
  \renewcommand{\childdocmain}[1]{\renewcommand{\childdocmain}[1]{\endinput}}
  \renewcommand{\childdocof}[1]{}
  \renewcommand{\childdocby}[2][]{}
  \renewcommand{\childdocforward}[2][]{}
  \renewcommand{\childdocdisable}{}
}
%    \end{macrocode}

% \macro{\childdocmain}
% The macro |\childdocmain| is to be called at the top of the main file
% with nothing or the main filename (without extension) as argument.
% First, it breaks loops.
% If the argument is not empty and does not match |\childdocname|
% (which is set by the first inclusion of |childdoc.def|),
% |\ifchilddoc| is set to true, |\includeonly| is applied to the child file
% and |\jobname| is set to the main file
% (for proper handling of |.aux| files):
%    \begin{macrocode}
\newcommand{\childdocmain}[1]
{
  \childdocdisable\childdocmain{}
  \if?#1?\else
    \begingroup
      \def\childdoctmp{#1}
      \ifx\childdoctmp\childdocname
        \def\childdoctmp{}
      \else
        \def\childdoctmp
        {
          \childdoctrue
          \includeonly{\childdocname}
          \def\childdocjob{#1}
          \def\jobname{#1}
        }
      \fi
      \expandafter
    \endgroup
    \childdoctmp
  \fi
}
%    \end{macrocode}

% \macro{\childdocof}
% The command |\childdocof| redirects
% compilation to the main file |#1|.
%    \begin{macrocode}
\newcommand{\childdocof}[1]
{
  \childdocdisable
  \childdoctrue
  \includeonly{\childdocname}
  \def\jobname{#1}
  \def\childdocjob{#1}
  \input{#1}
}
%    \end{macrocode}

% \macro{\childdocby}
% The command |\childdocby| ....
%    \begin{macrocode}
\newcommand{\childdocby}[2][]
{
  \childdocdisable
  \childdoctrue
  \childdocmanualtrue
  \if?#1?\else
    \def\jobname{#2}
  \fi
  \def\childdocjob{#2}
  \input{#2}
  \endinput
}
%    \end{macrocode}

% \macro{\childdocforward}
% The command |\childdocforward| redirects
% compilation to the main file or
% (if the optional argument is given) a child file.
% Parameters are set as if the main file
% or a child file starting with |\childdocof| was compiled.
% Then compilation is handed over to the main file:
%    \begin{macrocode}
\newcommand{\childdocforward}[2][]
{
  \begingroup
    \if?#1?
      \def\childdoctmp
      {
        \def\childdocname{#2}
        \def\childdocjob{#2}
        \def\jobname{#2}
        \input{#2}
        \endinput
      }
    \else
      \def\childdoctmp
      {
        \childdocdisable
        \def\childdocname{#2}
        \childdoctrue
        \includeonly{#2}
        \def\childdocjob{#1}
        \def\jobname{#1}
        \input{#1}
        \endinput
      }
    \fi
    \expandafter
  \endgroup
  \childdoctmp
}
%    \end{macrocode}

% \macro{\childdocforwardprefix}
% The command |\childdocforwardprefix| redirects
% compilation to the main or a child file by means of a pattern.
% The prefix |#1| in the current filename is replaced by |#2|
% and the suffix of the current filename is kept
% (it is assumed that the filename does not contain the substring `|~~~|'
% which is used as a delimiter).
% Compilation is handed over to the new file by |\childdocforward|:
%    \begin{macrocode}
\newcommand{\childdocforwardprefix}[3][]
{
  \begingroup
    \def\childdocextract #2##1~~~{\def\childdoctmp{\childdocforward[#1]{#3##1}}}
    \expandafter\childdocextract\childdocname~~~
    \expandafter
  \endgroup
  \childdoctmp
}
%    \end{macrocode}

% \macro{\childdoc}
% The deprecated macro |\childdoc| is a legacy version of |\childdocmain|:
%    \begin{macrocode}
\newcommand{\childdoc}{\childdocmain}
%    \end{macrocode}

% \macro{\childdocredirect}
% The deprecated macro |\childdocredirect| is a legacy version
% of |\childdocforward| and |\childdocforwardprefix|:
%    \begin{macrocode}
\newcommand{\childdocredirect}[2][]
{
  \begingroup
    \if?#1?
      \def\childdoctmp{\childdocforward{#2}}
    \else
      \def\childdoctmp{\childdocforwardprefix{#1}{#2}}
    \fi
    \expandafter
  \endgroup
  \childdoctmp
}
%    \end{macrocode}

%\iffalse
%</package>
%\fi
%
\endinput
\childdocforward{cdocsamp}"|\\
% |latex -jobname cdocscl1 \|\\
% |  "% \iffalse
%
% childdoc.dtx Copyright (C) 2017-2018 Niklas Beisert
%
% This work may be distributed and/or modified under the
% conditions of the LaTeX Project Public License, either version 1.3
% of this license or (at your option) any later version.
% The latest version of this license is in
%   http://www.latex-project.org/lppl.txt
% and version 1.3 or later is part of all distributions of LaTeX
% version 2005/12/01 or later.
%
% This work has the LPPL maintenance status `maintained'.
%
% The Current Maintainer of this work is Niklas Beisert.
%
% This work consists of the files childdoc.dtx and childdoc.ins
% and the derived files childdoc.def and cdocsamp.tex with
% cdocsch1.tex, cdocsch2.tex, cdocsdrf.tex, cdocsfn1.tex, cdocsfn2.tex.
%
%<package>\ifdefined\childdocmain\endinput\fi
%<package>\ProvidesFile{childdoc.def}[2018/12/30 v2.0 child document driver]
%<samplemain>\ProvidesFile{cdocsamp.tex}[2018/12/30 v2.0 sample for childdoc]
%<*driver>
%\ProvidesFile{childdoc.drv}[2018/12/30 v2.0 childdoc reference manual file]
\PassOptionsToClass{10pt,a4paper}{article}
\documentclass{ltxdoc}

\usepackage[margin=35mm]{geometry}
\usepackage{hyperref}
\usepackage{hyperxmp}
\usepackage[usenames]{color}

\hypersetup{colorlinks=true}
\hypersetup{pdfstartview=FitH}
\hypersetup{pdfpagemode=UseNone}
\hypersetup{pdfsource={}}
\hypersetup{pdflang={en-UK}}
\hypersetup{pdfcopyright={Copyright 2017-2018 Niklas Beisert.
  This work may be distributed and/or modified under the
  conditions of the LaTeX Project Public License, either version 1.3
  of this license or (at your option) any later version.}}
\hypersetup{pdflicenseurl={http://www.latex-project.org/lppl.txt}}
\hypersetup{pdfcontactaddress={ETH Zurich, ITP, HIT K,
  Wolfgang-Pauli-Strasse 27}}
\hypersetup{pdfcontactpostcode={8093}}
\hypersetup{pdfcontactcity={Zurich}}
\hypersetup{pdfcontactcountry={Switzerland}}
\hypersetup{pdfcontactemail={nbeisert@itp.phys.ethz.ch}}
\hypersetup{pdfcontacturl={http://people.phys.ethz.ch/\xmptilde nbeisert/}}

\newcommand{\secref}[1]{\hyperref[#1]{section \ref*{#1}}}

\parskip1ex
\parindent0pt
\let\olditemize\itemize
\def\itemize{\olditemize\parskip0pt}

\begin{document}

\title{The \textsf{childdoc} Package}
\hypersetup{pdftitle={The childdoc Package}}
\author{Niklas Beisert\\[2ex]
  Institut f\"ur Theoretische Physik\\
  Eidgen\"ossische Technische Hochschule Z\"urich\\
  Wolfgang-Pauli-Strasse 27, 8093 Z\"urich, Switzerland\\[1ex]
  \href{mailto:nbeisert@itp.phys.ethz.ch}
  {\texttt{nbeisert@itp.phys.ethz.ch}}}
\hypersetup{pdfauthor={Niklas Beisert}}
\hypersetup{pdfsubject={Manual for the LaTeX2e Package childdoc}}
\date{30 December 2018, \textsf{v2.0}}
\maketitle

\begin{abstract}\noindent
\textsf{childdoc} is a \LaTeXe{} package
that enables the direct compilation
of document sections included by |\include|
to individual files.
\end{abstract}

\begingroup
\parskip0ex
\tableofcontents
\endgroup

%%%%%%%%%%%%%%%%%%%%%%%%%%%%%%%%%%%%%%%%%%%%%%%%%%%%%%%%%%%%%%%%%%%%%%%%%%%%%%%%
%%%%%%%%%%%%%%%%%%%%%%%%%%%%%%%%%%%%%%%%%%%%%%%%%%%%%%%%%%%%%%%%%%%%%%%%%%%%%%%%
\section{Introduction}

\LaTeX{} provides a mechanism to structure a large document (such as a book)
into a main file and several child files (containing the chapters)
using the |\include| command.
This mechanism is beneficial for documents
which span hundreds of pages in order to
make the source file(s) more manageable.
Moreover, compilation can be restricted to
selected child files by means of the |\includeonly| command.
The latter feature can be used to reduce the compilation time while editing
(this was significantly more useful in the earlier days of \LaTeX{})
or to generate a smaller document which is easier to navigate.
Another application of |\includeonly| is to generate
documents consisting of selected parts of the complete document.

However, there are a few drawbacks of the plain |\include| mechanism:
\begin{itemize}
\item
The child files cannot be compiled on their own,
they can only be compiled via the main file.
A naive editing environment
(such as a text editor with an option
to have the current file processed by \LaTeX)
may require one to switch to the main file before compiling;
attempting to compile the child file produces errors.
\item
The main file must be modified (each time)
to adjust the |\includeonly| command
to the present needs. This easily leaves the main file in a messy state.
\item
The generated document will always carry the filename
of the main document. This is inconvenient if
several child files are to be compiled and
to be kept for distribution.
\end{itemize}

The present package provides a simple interface
to make child files individually compilable by \LaTeX{}.
Compiling a child file then has the same effect as compiling
the main file with an |\includeonly| command
to select the appropriate child.
Moreover the generated document will carry the name of the child
rather than the main file.
This resolves all three above issues.

This feature is meant to make the editing of books,
thesis documents and lecture notes somewhat more convenient.
However, the package can also be used efficiently for
composing a series of documents (such as exercise sheets)
which are typically distributed individually.
It then assists the author in generating the individual documents
(potentially in different versions)
as well as a document containing the collected series.
Another application is in developing style files
or other kinds of included material
where compilation of the style file could redirect
to a sample or test file.

%%%%%%%%%%%%%%%%%%%%%%%%%%%%%%%%%%%%%%%%%%%%%%%%%%%%%%%%%%%%%%%%%%%%%%%%%%%%%%%%
%%%%%%%%%%%%%%%%%%%%%%%%%%%%%%%%%%%%%%%%%%%%%%%%%%%%%%%%%%%%%%%%%%%%%%%%%%%%%%%%
\section{Usage}

First of all, the package \textsf{childdoc} is \emph{not} a standard
\LaTeXe{} |.sty| style file! Therefore it needs to be invoked in
a non-standard way.

%%%%%%%%%%%%%%%%%%%%%%%%%%%%%%%%%%%%%%%%%%%%%%%%%%%%%%%%%%%%%%%%%%%%%%%%%%%%%%%%
\subsection{Included Files}
\label{sec:include}

%%%%%%%%%%%%%%%%%%%%%%%%%%%%%%%%%%%%%%%%
\DescribeMacro{\childdocmain}
To use the package, add the commands
\begin{center}
\begin{tabular}{l}
|\input{childdoc.def}|\\
|\childdocmain{}|\\
\end{tabular}
\end{center}
at the very top of the main \LaTeX{} file,
in particular \emph{before} the |\documentclass| statement!
The argument of |\childdocmain| should be left empty
(but it must be present).

%%%%%%%%%%%%%%%%%%%%%%%%%%%%%%%%%%%%%%%%
\DescribeMacro{\childdocof}
Furthermore, add the commands
\begin{center}
\begin{tabular}{l}
|\input{childdoc.def}|\\
|\childdocof{|\textit{main}|}|\\
\end{tabular}
\end{center}
at the top of every child file \textit{child}
which is included by |\include{|\textit{child}|}|
from within the main file
(or at least for those files to be compiled individually).
The argument \textit{main} must be the filename of the main file.

There are a couple of
considerations in setting up the main and child documents:

%%%%%%%%%%%%%%%%%%%%%%%%%%%%%%%%%%%%%%%%
\paragraph{Restrictions.}

Please note the following restrictions:
\begin{itemize}
\item
|\childdocmain| must be called with one argument \textit{main}
to ensure compatibility with earlier version of the package.
It must either be empty (|\childdocmain{}|)
or precisely match the filename of the main file in which it is specified.
See \secref{sec:detection} for further information.
\item
The filename \textit{main} must be specified without the |.tex| extension.
\item
The filename \textit{main} is case sensitive
(even in case-insensitive file systems)
due to internal string comparison.
\item
The argument \textit{main} should be fully expanded, it cannot be a macro.
\item
Subdirectories and special characters should be avoided in filenames.
\item
The command |\childdocmain{|\textit{main}|}| must be followed by a whitespace.
It should not be followed immediately by another command
or by a comment mark `|%|'.
This is because the \TeX{} parser reads the token immediately following
the argument of |\childdocmain| and puts it
at the beginning of every child section;
however, a white\-space is ignored.
\end{itemize}

%%%%%%%%%%%%%%%%%%%%%%%%%%%%%%%%%%%%%%%%
\paragraph{Content of Main File.}

It is advisable to place all content in the child files included by |\include|.
Any output contained in the main file will appear in all child documents
unless suppressed manually;
it cannot be suppressed automatically by the |\includeonly| directive
and thus should normally be avoided.
A method to include some content in the main file
by means of conditional processing is described in \secref{sec:conditional}.

%%%%%%%%%%%%%%%%%%%%%%%%%%%%%%%%%%%%%%%%
\paragraph{Page Numbering.}

When only a part of the document is compiled,
the appropriate numbering of pages
(as well as other status parameters)
is determined from the |.aux| files.
The latter contain information from previous passes.
However this information needs to propagate through
all intermediate child documents.
Therefore the page numbering in child documents may well
be inconsistent until the complete document is compiled at least once.

A useful (if unconventional) way to always ensure a consistent
page numbering is to restart the numbering in each child document
and denote the pages by `\textit{child}|.|\textit{page}'
where \textit{child} represents the chapter/section number of the child file.
This can be achieved by the command
|\numberwithin{page}{|\textit{child}|}|
of the \textsf{amsmath} package
where \textit{child} can be |chapter| or |section|
depending on the chosen structuring.
Alternatively, one can modify the macro |\thepage| appropriately
and reset the counter |page| at the start of each child file.

%%%%%%%%%%%%%%%%%%%%%%%%%%%%%%%%%%%%%%%%%%%%%%%%%%%%%%%%%%%%%%%%%%%%%%%%%%%%%%%%
\subsection{Conditional Processing}
\label{sec:conditional}

The package provides a mechanism to compile different versions
of a document. To customise the versions further some conditional processing
can come in handy to distinguish which version is being compiled.
The package provides two macros to describe the compilation context:

%%%%%%%%%%%%%%%%%%%%%%%%%%%%%%%%%%%%%%%%
\DescribeMacro{\ifchilddoc}
The conditional |\ifchilddoc| distinguishes between the compilation of
child documents and the main document:
%
\begin{center}
|\ifchilddoc |\textit{child-code}| |[|\||else |\textit{main-code}]| \||fi|
\end{center}

%%%%%%%%%%%%%%%%%%%%%%%%%%%%%%%%%%%%%%%%
\DescribeMacro{\childdocname}
\DescribeMacro{\childdocjob}
The macro |\childdocname| contains the filename (without extension)
of the main or child file being processed.
Note that |\childdocjob| will always contain the name of the main file.

%%%%%%%%%%%%%%%%%%%%%%%%%%%%%%%%%%%%%%%%
\paragraph{Title Page.}

Conditional processing can be used to include a title or banner page
in the main document when proper precautions are taken.
Importantly, the code in the main file should ensure that the page counter
(as well as other status parameters which are stored in the |.aux| files)
takes the same value after the conditional processing.
Otherwise the page numbers may take divergent values
depending on which part is compiled.

For example, a title page could be declared by:
%
\begin{center}
\begin{tabular}{l}
|\ifchilddoc\||else|\\
|\addtocounter{page}{-1}|\\
\textit{code for title page}\\
|\newpage|\\
|\||fi|
\end{tabular}
\end{center}
%
A banner page for the child documents can be generated by:
%
\begin{center}
\begin{tabular}{l}
|\ifchilddoc|\\
|\addtocounter{page}{-1}|\\
\textit{code for banner page}\\
|\newpage|\\
|\||fi|
\end{tabular}
\end{center}
%
Here one could write a message such as:
\begin{center}
|This is the part \childdocname{} of \childdocjob{}.|
\end{center}

%%%%%%%%%%%%%%%%%%%%%%%%%%%%%%%%%%%%%%%%%%%%%%%%%%%%%%%%%%%%%%%%%%%%%%%%%%%%%%%%
\subsection{Flags}
\label{sec:flags}

The package makes it easy to generate different versions
of the main or child documents.
To this end compilation flags can be defined
and assigned different default values.
They will be particularly useful in conjunction
with the forwarding mechanism described in \secref{sec:forward}.

For example, it may be useful to have a flag |\version|
which can be set to |draft| or |final|.
The document source will contain some conditional code
depending on the value of |\version|.
Suppose further, the flag should default to |final| for the main file
and to |draft| for child files
which is a natural assignment for editing the document.
This is achieved by placing the following code
in the preamble of the main document
(below the |\childdocmain| directive):
%
\begin{center}
\begin{tabular}{l}
|\ifchilddoc|\\
|\providecommand{\version}{draft}|\\
|\||else|\\
|\providecommand{\version}{final}|\\
|\||fi|
\end{tabular}
\end{center}
%
The definition by |\providecommand| makes sure
that previous definitions are not overwritten.
Further statements |\providecommand{\version}{...}|
can thus be added before the above code to override it.

For the main file, one might add a line
(between |\childdocmain| and the above block)
%
\begin{center}
|%\ifchilddoc\||else\providecommand{\version}{draft}\||fi|
\end{center}
%
which can be uncommented to produce a draft version.
Likewise one can add a line to the very top of a child file
(above the |\childdocof{|\textit{main}|}| directive)
%
\begin{center}
|%\providecommand{\version}{final}|
\end{center}
%
which can be uncommented to produce the final version of this child document.

%%%%%%%%%%%%%%%%%%%%%%%%%%%%%%%%%%%%%%%%%%%%%%%%%%%%%%%%%%%%%%%%%%%%%%%%%%%%%%%%
\subsection{Forwarding}
\label{sec:forward}

Different versions of the main or child documents
using compilation flags as described in \secref{sec:flags}
can be (permanently) stored in different files
for convenient compilation, viewing and distribution.
To this end, the package defines a command
to pass on compilation to a different file:

%%%%%%%%%%%%%%%%%%%%%%%%%%%%%%%%%%%%%%%%
\DescribeMacro{\childdocforward}
The command |\childdocforward| redirects processing to
another source file:
%
\begin{center}
\begin{tabular}{l}
|\input{childdoc.def}|\\
|\childdocforward[|\textit{main}|]{|\textit{dest}|}|\\
\end{tabular}
\end{center}
%
The argument \textit{dest} is the destination file
(without extension).
It should be the main file or one of the child files.
Note that further \textsf{childdoc} directives
such as |\childdocof| and |\childdocforward|
in the indicated file will be processed in this form.
The optional argument \textit{main}
passes on directly to the main file \textit{main}
while pretending to compile the child \textit{dest}.
This form behaves as if \textit{dest}
issues |\childdocof{|\textit{main}|}| right away,
and no further \textsf{childdoc} directives will be processed.

%%%%%%%%%%%%%%%%%%%%%%%%%%%%%%%%%%%%%%%%
\DescribeMacro{\...prefix}
In the alternative form |\childdocforwardprefix|,
%
\begin{center}
\begin{tabular}{l}
|\input{childdoc.def}|\\
|\childdocforwardprefix[|\textit{main}|]{|\textit{prefix}|}{|\textit{dest}|}|
\end{tabular}
\end{center}
%
the destination file is determined by a pattern
depending on the current file:
To make this work, the current file must be called
`{\textit{prefix}\hspace{0.2em}\textit{suffix}}'
with \textit{prefix} matching precisely the argument.
Processing is then passed on to the file
`{\textit{dest}\hspace{0.2em}\textit{suffix}}'.
Surely, the same effect is achieved by
directly specifying the
argument `{\textit{dest}\hspace{0.2em}\textit{suffix}}'
in the first form.
However, that requires to set up a different file
for each child. With the alternative form of the command
all these files can have exactly the same content
which simplifies setting them up and maintaining them.

For example, the following file |draft.tex|
with a compilation flag |\version| as described in \secref{sec:flags}
compiles the main document as a draft:
%
\begin{center}
\begin{tabular}{l}
|\def\version{draft}|\\
|\input{childdoc.def}|\\
|\childdocforward{|\textit{main}|}|
\end{tabular}
\end{center}
%
Likewise, the following files |final|\textit{nn}|.tex|
compile the final version of the child document
|child|\textit{nn}|.tex|:
%
\begin{center}
\begin{tabular}{l}
|\def\version{final}|\\
|\input{childdoc.def}|\\
|\childdocforwardprefix{final}{child}|
\end{tabular}
\end{center}
%

Note that when several versions of a main file and/or of each child file
are to be generated, it may be convenient to set up a |Makefile| or
shell script to automatise the process.

%%%%%%%%%%%%%%%%%%%%%%%%%%%%%%%%%%%%%%%%%%%%%%%%%%%%%%%%%%%%%%%%%%%%%%%%%%%%%%%%
\subsection{Command Line Processing}
\label{sec:commandline}

The effect of redirection files can also be achieved by invoking
the \LaTeX{} compiler with a more elaborate command line.
Most conveniently this should be done as part
of a shell script or a |Makefile|.

When using \textsf{childdoc} in the main file, the following
command lines effectively perform a redirection
(note that depending on the shell being used,
backslashes may have to be doubled: `|\|' $\to$ `|\\|'):
%
\begin{center}
|... -jobname "|\textit{target}|" |\\|"|[\textit{flags}]%
|\input{childdoc.def}\childdocforward[|\textit{main}|]{|\textit{dest}|}"|
\end{center}
%
Here \textit{target} is the name of the output file,
\textit{main} is the name of the main file
and \textit{dest} is the name of the main or child file to be processed
(all filenames without extensions).
The optional argument \textit{main} can be omitted
if \textit{main} matches \textit{dest}.
Optionally, compilation \textit{flags} can be defined via |\def| commands.
This command line makes the \TeX{} engine believe
it is compiling the file \textit{target}
whose content is specified as the latter parameter.
The provided code then forwards the processing to
\textit{main} or \textit{dest} as described in \secref{sec:forward}.

%%%%%%%%%%%%%%%%%%%%%%%%%%%%%%%%%%%%%%%%%%%%%%%%%%%%%%%%%%%%%%%%%%%%%%%%%%%%%%%%
\subsection{Include by Input}
\label{sec:input}

Including child documents by |\include| has some restrictions by design.
Most notably, the content of a child document always occupies
its own set of pages; pages cannot be shared between child documents.
Usually, this behaviour makes perfect sense
because each child document contain an essential part of the document.
However, in some situations it may be desirable to compose
a document from a collection of parts
without having mandatory page breaks between then.
For this case, the package
provides a mechanism to include parts
by |\input| which can also be processed individually.
However, by construction this mechanism
requires manual handling of the content to be output.

%%%%%%%%%%%%%%%%%%%%%%%%%%%%%%%%%%%%%%%%
\DescribeMacro{\ifchilddocmanual}
The main file should be prepared as usual, see \secref{sec:include}.
However, the document body must make a distinction
between processing of an individual part and of the main document, e.g.:
%
\begin{center}
\begin{tabular}{l}
|\ifchilddocmanual|\\
|\input{\childdocname}|\\
|\||else|\\
\textit{document body with }|\input{|\textit{part}|}|\\
|\||fi|
\end{tabular}
\end{center}
%
The conditional |\ifchilddocmanual| is true whenever
a part to be included by |\input| is being compiled,
and the name of the part is stored in |\childdocname|.

%%%%%%%%%%%%%%%%%%%%%%%%%%%%%%%%%%%%%%%%
\DescribeMacro{\childdocby}
Each part to be included by |\input| should start with:
%
\begin{center}
\begin{tabular}{l}
|\input{childdoc.def}|\\
|\childdocby{|\textit{main}|}|\\
\end{tabular}
\end{center}
%
The directive |\childdocby| is similar to |\childdocof|
described in \secref{sec:include},
but the subsequent selection of content must be done manually.
To that end, both |\ifchilddoc| and |\ifchilddocmanual|
will be true upon processing of a part,
and the name of the part is stored in |\childdocname|.
Note that |\jobname| will be set to the filename of the current part
so that each part receives an individual |.aux| file
that does not interfere with the |.aux| file(s) of the main document.
This behaviour can be altered by the alternative form
|\childdocby[*]{|\textit{main}|}| (with a non-empty optional argument)
which uses the |.aux| file of the main document
by setting |\jobname| to \textit{main}.

%%%%%%%%%%%%%%%%%%%%%%%%%%%%%%%%%%%%%%%%%%%%%%%%%%%%%%%%%%%%%%%%%%%%%%%%%%%%%%%%
\subsection{Driver Development}
\label{sec:driver}

The \textsf{childdoc} mechanism can also be use for the development
of definition files such as \LaTeX{} styles or classes.
This case differs from the above setup with multiple parts
included by |\include| in that no |\includeonly| should be invoked.
This can be achieved by starting the include file
(before |\ProvidesPackage|) with:
%
\begin{center}
\begin{tabular}{l}
|\input{childdoc.def}|\\
|\childdocforward{|\textit{main}|}|\\
\end{tabular}
\end{center}
%
or alternatively with:
%
\begin{center}
\begin{tabular}{l}
|\input{childdoc.def}|\\
|\childdocby{|\textit{main}|}|\\
\end{tabular}
\end{center}
%
Both forms have slightly different effects as described above.
The main file is prepared as usual, see \secref{sec:include}.

%%%%%%%%%%%%%%%%%%%%%%%%%%%%%%%%%%%%%%%%%%%%%%%%%%%%%%%%%%%%%%%%%%%%%%%%%%%%%%%%
\subsection{Legacy Detection}
\label{sec:detection}

The directive |\childdocmain| in the main file can detect
whether the complete document or merely a child is to be compiled
even without using the directive |\childdocof|.
This method is deprecated because it is less robust
and there is no compelling reason to use it;
it is merely provided for backward compatibility
and it may be removed in future versions.

If the detection mechanism is to be used,
it is mandatory to correctly specify
the filename of the main file as the argument of |\childdocmain|:
%
\begin{center}
\begin{tabular}{l}
|\input{childdoc.def}|\\
|\childdocmain{|\textit{main}|}|\\
\end{tabular}
\end{center}
%
If |\jobname| does not match the argument \textit{main} of |\childdocmain|,
it is assumed that |\jobname| points to the child file to be compiled.
When using |\childdocmain| with the main file specified as argument,
it suffices to start a child file
with just |\input{|\textit{main}|}|
without loading of the package and using |\childdocof|.
If instead all processing is done
with the appropriate \textsf{childdoc} directives,
the argument of \textit{main} of |\childdocmain| can be empty.

An alternative version of the command line processing described
in \secref{sec:commandline} using the detection mechanism reads:
%
\begin{center}
|... -jobname "|\textit{target}|" "|[\textit{flags}]%
[|\def\jobname{|\textit{dest}|}|]|\input{|\textit{main}|}"|
\end{center}

%%%%%%%%%%%%%%%%%%%%%%%%%%%%%%%%%%%%%%%%%%%%%%%%%%%%%%%%%%%%%%%%%%%%%%%%%%%%%%%%
\subsection{Manual Code}
\label{sec:manual}

In case one cannot be certain whether the definitions file |childdoc.def|
is installed on the target \TeX{} distribution
and one prefers not to ship it,
it is conceivable to paste a few relevant commands into the sources.

To that end, drop all statements |\input{childdoc.def}|
and perform the replacements as outlined below.
Instead of |\childdocmain{|\textit{main}|}| add the following code
to the top of the main file:
%
\begin{center}
\begin{tabular}{l}
|\||ifdefined\childdocname\endinput\||fi\newif\ifchilddoc|\\
|\edef\childdocname{\scantokens\expandafter{\jobname\noexpand}}|\\
|\def\childdocmain{|\textit{main}|}\||ifx\childdocmain\childdocname\||else|\\
|\childdoctrue\includeonly{\childdocname}\let\jobname\childdocmain\||fi|\\
\end{tabular}
\end{center}
%
Instead of |\childdocof{|\textit{main}|}| just include the main file
at the top of each child file:
%
\begin{center}
|\input{|\textit{main}|}|
\end{center}
%
A simple redirection |\childdocforward{|\textit{dest}|}| is achieved by:
%
\begin{center}
|\def\jobname{|\textit{dest}|}\input{\jobname}|
\end{center}
%
The redirection with prefix
|\childdocforwardprefix[|\textit{prefix}|]{|\textit{dest}|}|
is accomplished by:
%
\begin{center}
\begin{tabular}{l}
|{\edef\jobname{\scantokens\expandafter{\jobname\noexpand}}|\\
|\def\redirectjob |\textit{prefix}|#1~~~{\gdef\jobname{|\textit{dest}|#1}}|\\
|\expandafter\redirectjob\jobname~~~}\input{\jobname}|
\end{tabular}
\end{center}

In an alternative approach,
child documents can be compiled by a specific command line
without additional code or specific definitions:
%
\begin{center}
|... -jobname "|\textit{target}|" "|[\textit{flags}]%
|\includeonly{|\textit{dest}|}\input{|\textit{main}|}"|
\end{center}
%

%%%%%%%%%%%%%%%%%%%%%%%%%%%%%%%%%%%%%%%%%%%%%%%%%%%%%%%%%%%%%%%%%%%%%%%%%%%%%%%%
%%%%%%%%%%%%%%%%%%%%%%%%%%%%%%%%%%%%%%%%%%%%%%%%%%%%%%%%%%%%%%%%%%%%%%%%%%%%%%%%
\section{Information}

%%%%%%%%%%%%%%%%%%%%%%%%%%%%%%%%%%%%%%%%%%%%%%%%%%%%%%%%%%%%%%%%%%%%%%%%%%%%%%%%
\subsection{Copyright}

Copyright \copyright{} 2017--2018 Niklas Beisert

This work may be distributed and/or modified under the
conditions of the \LaTeX{} Project Public License, either version 1.3
of this license or (at your option) any later version.
The latest version of this license is in
  \url{http://www.latex-project.org/lppl.txt}
and version 1.3 or later is part of all distributions of \LaTeX{}
version 2005/12/01 or later.

This work has the LPPL maintenance status `maintained'.

The Current Maintainer of this work is Niklas Beisert.

This work consists of the files |README.txt|, |childdoc.ins| and |childdoc.dtx|
as well as the derived files |childdoc.def|, |cdocsamp.tex|
with |cdocsch1.tex|, |cdocsch2.tex|, |cdocspt3.tex|, |cdocspt4.tex|,
|cdocsdrf.tex|, |cdocsfn1.tex|, |cdocsfn2.tex|
as well as |childdoc.pdf|.

%%%%%%%%%%%%%%%%%%%%%%%%%%%%%%%%%%%%%%%%%%%%%%%%%%%%%%%%%%%%%%%%%%%%%%%%%%%%%%%%
\subsection{Files and Installation}

The package consists of the files:
%
\begin{center}
\begin{tabular}{ll}
    |README.txt|   & readme file \\
    |childdoc.ins| & installation file \\
    |childdoc.dtx| & source file \\
    |childdoc.def| & definition file \\
    |cdocsamp.tex| & sample main file \\
    |cdocsch1.tex| & sample include file \\
    |cdocsch2.tex| & sample include file \\
    |cdocspt3.tex| & sample part file \\
    |cdocspt4.tex| & sample part file \\
    |cdocsdrf.tex| & sample redirection file \\
    |cdocsfn1.tex| & sample redirection file \\
    |cdocsfn2.tex| & sample redirection file \\
    |childdoc.pdf| & manual
\end{tabular}
\end{center}
%
The distribution consists of the files
|README.txt|, |childdoc.ins| and |childdoc.dtx|.
%
\begin{itemize}
\item
Run (pdf)\LaTeX{} on |childdoc.dtx|
to compile the manual |childdoc.pdf| (this file).
\item
Run \LaTeX{} on |childdoc.ins| to create the definitions file |childdoc.def|
and the sample |cdocsamp.tex| with include files
|cdocsch1.tex|, |cdocsch2.tex|, |cdocspt3.tex|, |cdocspt4.tex|,
|cdocsdrf.tex|, |cdocsfn1.tex|, |cdocsfn2.tex|.
Then copy the file |childdoc.def| to an appropriate directory of your \LaTeX{}
distribution, e.g.\ \textit{texmf-root}|/tex/latex/childdoc|.
\end{itemize}

%%%%%%%%%%%%%%%%%%%%%%%%%%%%%%%%%%%%%%%%%%%%%%%%%%%%%%%%%%%%%%%%%%%%%%%%%%%%%%%%
\subsection{Related CTAN Packages}

There are several other packages which offer a similar functionality:
%
\begin{itemize}
\item
The packages
\href{http://ctan.org/pkg/docmute}{\textsf{docmute}},
\href{http://ctan.org/pkg/includex}{\textsf{includex}} and
\href{http://ctan.org/pkg/standalone}{\textsf{standalone}}
provide commands to include only the document body of
a child file thus allowing both files to be compiled individually.
\item
The packages \href{http://ctan.org/pkg/subdocs}{\textsf{subdocs}}
and \href{http://ctan.org/pkg/subfiles}{\textsf{subfiles}}
provide structures in which the main and child documents can be
encapsulated and allowing them to be compiled individually.
The inclusion mechanism is different from the conventional |\include|.
\item
The package \href{http://ctan.org/pkg/combine}{\textsf{combine}}
is an elaborate solution to combine several documents into one.
\end{itemize}
%
See also the CTAN topic \href{http://ctan.org/topic/subdocs}{\textsf{subdocs}}
for further related packages.
The present package differs from the above solutions in that
a document structure constructed with the conventional |\include| mechanism
just needs two extra commands at the top of every file
such that all constituent files can be compiled individually.

%%%%%%%%%%%%%%%%%%%%%%%%%%%%%%%%%%%%%%%%%%%%%%%%%%%%%%%%%%%%%%%%%%%%%%%%%%%%%%%%
%\subsection{Feature Suggestions}
%
%The following is a list of features which may be useful for future
%versions of this package:
%%
%\begin{itemize}
%\item
%\ldots
%\end{itemize}

%%%%%%%%%%%%%%%%%%%%%%%%%%%%%%%%%%%%%%%%%%%%%%%%%%%%%%%%%%%%%%%%%%%%%%%%%%%%%%%%
\subsection{Revision History}

%%%%%%%%%%%%%%%%%%%%%%%%%%%%%%%%%%%%%%%%
\paragraph{v2.0:} 2018/12/30

\begin{itemize}
\item
immediate forward processing
\item
added |\childdocby| mechanism
\item
manual restructured
\end{itemize}

%%%%%%%%%%%%%%%%%%%%%%%%%%%%%%%%%%%%%%%%
\paragraph{v1.6:} 2018/01/17

\begin{itemize}
\item
application for development of include files
\item
corrections to manual
\end{itemize}

%%%%%%%%%%%%%%%%%%%%%%%%%%%%%%%%%%%%%%%%
\paragraph{v1.5:} 2017/05/21

\begin{itemize}
\item
more complete structuring introduced
\item
|\childdocof| introduced
\item
|\childdoc| renamed to |\childdocmain|
\item
|\childredirect| renamed to |\childdocforward| and |\childdocforwardprefix|
and functionality expanded
\end{itemize}

%%%%%%%%%%%%%%%%%%%%%%%%%%%%%%%%%%%%%%%%
\paragraph{v1.0:} 2017/04/27

\begin{itemize}
\item
manual and install package
\item
first version published on CTAN
\end{itemize}

%%%%%%%%%%%%%%%%%%%%%%%%%%%%%%%%%%%%%%%%
\paragraph{v0.6:} 2017/04/26

\begin{itemize}
\item
redirection mechanism added
\end{itemize}

%%%%%%%%%%%%%%%%%%%%%%%%%%%%%%%%%%%%%%%%
\paragraph{v0.5:} 2017/04/26

\begin{itemize}
\item
functionality in definition file
\end{itemize}


%%%%%%%%%%%%%%%%%%%%%%%%%%%%%%%%%%%%%%%%%%%%%%%%%%%%%%%%%%%%%%%%%%%%%%%%%%%%%%%%
%%%%%%%%%%%%%%%%%%%%%%%%%%%%%%%%%%%%%%%%%%%%%%%%%%%%%%%%%%%%%%%%%%%%%%%%%%%%%%%%
%%%%%%%%%%%%%%%%%%%%%%%%%%%%%%%%%%%%%%%%%%%%%%%%%%%%%%%%%%%%%%%%%%%%%%%%%%%%%%%%
\appendix

\settowidth\MacroIndent{\rmfamily\scriptsize 000\ }

 \DocInput{childdoc.dtx}

\end{document}
%</driver>
% \fi
%
% %%%%%%%%%%%%%%%%%%%%%%%%%%%%%%%%%%%%%%%%%%%%%%%%%%%%%%%%%%%%%%%%%%%%%%%%%%%%%%
% %%%%%%%%%%%%%%%%%%%%%%%%%%%%%%%%%%%%%%%%%%%%%%%%%%%%%%%%%%%%%%%%%%%%%%%%%%%%%%
% \section{Sample}
%\iffalse
%<*samplemain>
%\fi
%
% The following presents a sample document
% with two chapters, two parts, a title page,
% a compile flag as well as three forwarding files to set the flag.
% It consists of eight |.tex| files:
% \begin{center}
% \begin{tabular}{ll}
% |cdocsamp.tex|&main file\\
% |cdocsch1.tex|&include file for chapter 1\\
% |cdocsch2.tex|&include file for chapter 2\\
% |cdocspt3.tex|&include file for part 3\\
% |cdocspt4.tex|&include file for part 4\\
% |cdocsdrf.tex|&forwarding file for main file in draft mode\\
% |cdocsfi1.tex|&forwarding file for final version of chapter 1\\
% |cdocsfi2.tex|&forwarding file for final version of chapter 2\\
% \end{tabular}
% \end{center}
% Each of the eight files can be compiled directly by the \LaTeX{} compiler.
%
% %%%%%%%%%%%%%%%%%%%%%%%%%%%%%%%%%%%%%%
% \paragraph{Main File.}
%
% The main file is called |cdocsamp.tex|.
%
% Load the \textsf{childdoc} definitions and
% declare the filename for the main document:
%    \begin{macrocode}
\input{childdoc.def}
\childdocmain{}
%    \end{macrocode}

% Optional override for |\version| flag:
%    \begin{macrocode}
%%\ifchilddoc\else\providecommand{\version}{draft}\fi
%    \end{macrocode}

% Define the default values for the |\version| flag
% (|final| for the main file and |draft| for childs):
%    \begin{macrocode}
\ifchilddoc
\providecommand{\version}{draft}
\else
\providecommand{\version}{final}
\fi
%    \end{macrocode}

% Load the standard document class:
%    \begin{macrocode}
\documentclass[12pt]{article}
%    \end{macrocode}

% Start the document body:
%    \begin{macrocode}
\begin{document}
%    \end{macrocode}

% Declare a title page.
% Print title, part of document being processed and version flag:
%    \begin{macrocode}
\addtocounter{page}{-1}
\begin{center}
{\LARGE\bfseries{}childdoc example\par}
\vspace{1cm}
\ifchilddoc
\ifchilddocmanual part\else chapter\fi:
`\childdocname' of `\childdocjob'\par
\else
main document: `\childdocjob'\par
\fi
version: \version\par
\end{center}
\newpage
%    \end{macrocode}

% Manually include selected file,
% otherwise process as usual:
%    \begin{macrocode}
\ifchilddocmanual
\section*{part `\childdocname'}
\input{\childdocname}
\else
%    \end{macrocode}

% Include the two chapters:
%    \begin{macrocode}
\include{cdocsch1}
\include{cdocsch2}
%    \end{macrocode}

% Include the two parts unless only chapters should be displayed:
%    \begin{macrocode}
\ifchilddoc\else
\section{part three}
\input{cdocspt3}
\section{part four}
\input{cdocspt4}
\fi
%    \end{macrocode}

% Process as usual until here:
%    \begin{macrocode}
\fi
%    \end{macrocode}

% End of document body:
%    \begin{macrocode}
\end{document}
%    \end{macrocode}
%\iffalse
%</samplemain>
%\fi
%
% %%%%%%%%%%%%%%%%%%%%%%%%%%%%%%%%%%%%%%
% \paragraph{Chapter Include Files.}
%
% The include files are called |cdocsch1.tex| and |cdocsch2.tex|.
%
%\iffalse
%<*samplechap1|samplechap2>
%\fi

% Optional override for |\version| flag:
%    \begin{macrocode}
%%\providecommand{\version}{final}
%    \end{macrocode}

% Include the main document:
%    \begin{macrocode}
\input{childdoc.def}
\childdocof{cdocsamp}
%    \end{macrocode}

%\iffalse
%</samplechap1|samplechap2>
%\fi
%
%\iffalse
%<*samplechap1>
%\fi
% Some text for chapter 1:
%    \begin{macrocode}
\section{one}
some text in chapter one
%    \end{macrocode}

%\iffalse
%</samplechap1>
%\fi
% Some text for chapter 2:
%\iffalse
%<*samplechap2>
%\fi
%    \begin{macrocode}
\section{two}
more text in chapter two
%    \end{macrocode}

%\iffalse
%</samplechap2>
%\fi
%
% %%%%%%%%%%%%%%%%%%%%%%%%%%%%%%%%%%%%%%
% \paragraph{Part Include Files.}
%
% The include files are called |cdocspt3.tex| and |cdocspt4.tex|.
%
%\iffalse
%<*samplepart3|samplepart4>
%\fi

% Optional override for |\version| flag:
%    \begin{macrocode}
%%\providecommand{\version}{final}
%    \end{macrocode}

% Include the main document:
%    \begin{macrocode}
\input{childdoc.def}
\childdocby{cdocsamp}
%    \end{macrocode}

%\iffalse
%</samplepart3|samplepart4>
%\fi
%
%\iffalse
%<*samplepart3>
%\fi
% Some text for part 3:
%    \begin{macrocode}
some text in part three
%    \end{macrocode}

%\iffalse
%</samplepart3>
%\fi
% Some text for part 4:
%\iffalse
%<*samplepart4>
%\fi
%    \begin{macrocode}
more text in part four
%    \end{macrocode}

%\iffalse
%</samplepart4>
%\fi
%
% %%%%%%%%%%%%%%%%%%%%%%%%%%%%%%%%%%%%%%
% \paragraph{Forwarding for a Complete Draft.}
%
% The following forwarding file |cdocsdrf.tex|
% compiles the main document in draft mode:
%\iffalse
%<*sampledraft>
%\fi
%    \begin{macrocode}
\def\version{draft}
\input{childdoc.def}
\childdocforward{cdocsamp}
%    \end{macrocode}

%\iffalse
%</sampledraft>
%\fi
%
% %%%%%%%%%%%%%%%%%%%%%%%%%%%%%%%%%%%%%%
% \paragraph{Forwarding for Final Version of the Chapters.}
%
% The following forwarding files |cdocsfn1.tex| and |cdocsfn2.tex|
% (with identical content)
% compile the final versions of the child documents
% |cdocsch1.tex| and |cdocsch2.tex|, respectively:
%\iffalse
%<*samplefinal>
%\fi
%    \begin{macrocode}
\def\version{final}
\input{childdoc.def}
\childdocforwardprefix[cdocsamp]{cdocsfn}{cdocsch}
%    \end{macrocode}

%\iffalse
%</samplefinal>
%\fi
%
% %%%%%%%%%%%%%%%%%%%%%%%%%%%%%%%%%%%%%%
% \paragraph{Command Line Processing.}
%
% The following three command lines generate the output files
% |cdocscld|, |cdocscl1| and |cdocscl2|
% which should be identical to
% |cdocsdrf|, |cdocsch1| and |cdocsfn2|, respectively:
% \begin{center}
% \begin{tabular}{l}
% |latex -jobname cdocscld \|\\
% |  "\def\version{draft}\input{childdoc.def}\childdocforward{cdocsamp}"|\\
% |latex -jobname cdocscl1 \|\\
% |  "\input{childdoc.def}\childdocforward[cdocsamp]{cdocsch1}"|\\
% |latex -jobname cdocscl2 \|\\
% |  "\def\version{final}\input{childdoc.def}\childdocforward{cdocsch2}"|
% \end{tabular}
% \end{center}
% Note that the trailing backslash on each first line
% merely continues the input to the second line
% (for convenient cut ant paste).
% Furthermore, the command |latex| can be replaced by any
% of its alternative versions such as |pdflatex|.
%
% %%%%%%%%%%%%%%%%%%%%%%%%%%%%%%%%%%%%%%%%%%%%%%%%%%%%%%%%%%%%%%%%%%%%%%%%%%%%%%
% %%%%%%%%%%%%%%%%%%%%%%%%%%%%%%%%%%%%%%%%%%%%%%%%%%%%%%%%%%%%%%%%%%%%%%%%%%%%%%
% \section{Implementation}
%\iffalse
%<*package>
%\fi
%
% This section describes the definitions file |childdoc.def|.

% The definitions cannot be loaded using |\usepackage| or |\RequirePackage|
% which has a mechanism to prevent loading a style file more than once.
% When loading the definitions by means of |\input|
% multiple instances have to be prevented manually:
%\iffalse
%This code needs to be before the `\ProvidesFile' directive
%which is defined at the beginning of this file.
%Therefore it is also placed there and commented out here.
%</package>
%<*discard>
%\fi
%    \begin{macrocode}
\ifdefined\childdocmain\endinput\fi
%    \end{macrocode}
%\iffalse
%</discard>
%<*package>
%\fi
%
% \macro{\ifchilddoc}
% \macro{\ifchilddocmanual}
% The conditional |\ifchilddoc| tells whether a
% child (true) or main (false) document is being compiled.
% The conditional |\ifchilddocmanual| tells whether
% the |\includeonly| mechanism is used (false) or
% the selection of child files must be performed manually (true).
% The definitions initialise to false:
%    \begin{macrocode}
\newif\ifchilddoc
\newif\ifchilddocmanual
%    \end{macrocode}

% \macro{\childdocname}
% \macro{\childdocjob}
% The macro |\childdocname| stores the name of the main document
% to be compiled. The macro |\childdocjob| stores the name of
% the document on which the \LaTeX{} compiler was originally invoked.
% The content of |\jobname| cannot be compared
% to filenames specified in the source due to different catcodes.
% The following code rescans |\jobname|, stores the result
% in |\childdocname| and saves a copy in |\childdocjob|:
%    \begin{macrocode}
\edef\childdocname{\scantokens\expandafter{\jobname\noexpand}}
\let\childdocjob\childdocname
%    \end{macrocode}

% \macro{\childdocdisable}
% The macro |\childdocdisable| prevents the main file
% from being processed more than once.
% At this stage, the main document command |\childdocmain|
% is assumed to be called once again where it should do nothing.
% Any subsequent call to it should prevent
% a secondary processing of the main document
% It overwrites the forwarding commands
% |\childdocof| and |\childdocforward|
% with empty macros to prevent further inclusions of the main document:
%    \begin{macrocode}
\newcommand{\childdocdisable}
{
  \renewcommand{\childdocmain}[1]{\renewcommand{\childdocmain}[1]{\endinput}}
  \renewcommand{\childdocof}[1]{}
  \renewcommand{\childdocby}[2][]{}
  \renewcommand{\childdocforward}[2][]{}
  \renewcommand{\childdocdisable}{}
}
%    \end{macrocode}

% \macro{\childdocmain}
% The macro |\childdocmain| is to be called at the top of the main file
% with nothing or the main filename (without extension) as argument.
% First, it breaks loops.
% If the argument is not empty and does not match |\childdocname|
% (which is set by the first inclusion of |childdoc.def|),
% |\ifchilddoc| is set to true, |\includeonly| is applied to the child file
% and |\jobname| is set to the main file
% (for proper handling of |.aux| files):
%    \begin{macrocode}
\newcommand{\childdocmain}[1]
{
  \childdocdisable\childdocmain{}
  \if?#1?\else
    \begingroup
      \def\childdoctmp{#1}
      \ifx\childdoctmp\childdocname
        \def\childdoctmp{}
      \else
        \def\childdoctmp
        {
          \childdoctrue
          \includeonly{\childdocname}
          \def\childdocjob{#1}
          \def\jobname{#1}
        }
      \fi
      \expandafter
    \endgroup
    \childdoctmp
  \fi
}
%    \end{macrocode}

% \macro{\childdocof}
% The command |\childdocof| redirects
% compilation to the main file |#1|.
%    \begin{macrocode}
\newcommand{\childdocof}[1]
{
  \childdocdisable
  \childdoctrue
  \includeonly{\childdocname}
  \def\jobname{#1}
  \def\childdocjob{#1}
  \input{#1}
}
%    \end{macrocode}

% \macro{\childdocby}
% The command |\childdocby| ....
%    \begin{macrocode}
\newcommand{\childdocby}[2][]
{
  \childdocdisable
  \childdoctrue
  \childdocmanualtrue
  \if?#1?\else
    \def\jobname{#2}
  \fi
  \def\childdocjob{#2}
  \input{#2}
  \endinput
}
%    \end{macrocode}

% \macro{\childdocforward}
% The command |\childdocforward| redirects
% compilation to the main file or
% (if the optional argument is given) a child file.
% Parameters are set as if the main file
% or a child file starting with |\childdocof| was compiled.
% Then compilation is handed over to the main file:
%    \begin{macrocode}
\newcommand{\childdocforward}[2][]
{
  \begingroup
    \if?#1?
      \def\childdoctmp
      {
        \def\childdocname{#2}
        \def\childdocjob{#2}
        \def\jobname{#2}
        \input{#2}
        \endinput
      }
    \else
      \def\childdoctmp
      {
        \childdocdisable
        \def\childdocname{#2}
        \childdoctrue
        \includeonly{#2}
        \def\childdocjob{#1}
        \def\jobname{#1}
        \input{#1}
        \endinput
      }
    \fi
    \expandafter
  \endgroup
  \childdoctmp
}
%    \end{macrocode}

% \macro{\childdocforwardprefix}
% The command |\childdocforwardprefix| redirects
% compilation to the main or a child file by means of a pattern.
% The prefix |#1| in the current filename is replaced by |#2|
% and the suffix of the current filename is kept
% (it is assumed that the filename does not contain the substring `|~~~|'
% which is used as a delimiter).
% Compilation is handed over to the new file by |\childdocforward|:
%    \begin{macrocode}
\newcommand{\childdocforwardprefix}[3][]
{
  \begingroup
    \def\childdocextract #2##1~~~{\def\childdoctmp{\childdocforward[#1]{#3##1}}}
    \expandafter\childdocextract\childdocname~~~
    \expandafter
  \endgroup
  \childdoctmp
}
%    \end{macrocode}

% \macro{\childdoc}
% The deprecated macro |\childdoc| is a legacy version of |\childdocmain|:
%    \begin{macrocode}
\newcommand{\childdoc}{\childdocmain}
%    \end{macrocode}

% \macro{\childdocredirect}
% The deprecated macro |\childdocredirect| is a legacy version
% of |\childdocforward| and |\childdocforwardprefix|:
%    \begin{macrocode}
\newcommand{\childdocredirect}[2][]
{
  \begingroup
    \if?#1?
      \def\childdoctmp{\childdocforward{#2}}
    \else
      \def\childdoctmp{\childdocforwardprefix{#1}{#2}}
    \fi
    \expandafter
  \endgroup
  \childdoctmp
}
%    \end{macrocode}

%\iffalse
%</package>
%\fi
%
\endinput
\childdocforward[cdocsamp]{cdocsch1}"|\\
% |latex -jobname cdocscl2 \|\\
% |  "\def\version{final}% \iffalse
%
% childdoc.dtx Copyright (C) 2017-2018 Niklas Beisert
%
% This work may be distributed and/or modified under the
% conditions of the LaTeX Project Public License, either version 1.3
% of this license or (at your option) any later version.
% The latest version of this license is in
%   http://www.latex-project.org/lppl.txt
% and version 1.3 or later is part of all distributions of LaTeX
% version 2005/12/01 or later.
%
% This work has the LPPL maintenance status `maintained'.
%
% The Current Maintainer of this work is Niklas Beisert.
%
% This work consists of the files childdoc.dtx and childdoc.ins
% and the derived files childdoc.def and cdocsamp.tex with
% cdocsch1.tex, cdocsch2.tex, cdocsdrf.tex, cdocsfn1.tex, cdocsfn2.tex.
%
%<package>\ifdefined\childdocmain\endinput\fi
%<package>\ProvidesFile{childdoc.def}[2018/12/30 v2.0 child document driver]
%<samplemain>\ProvidesFile{cdocsamp.tex}[2018/12/30 v2.0 sample for childdoc]
%<*driver>
%\ProvidesFile{childdoc.drv}[2018/12/30 v2.0 childdoc reference manual file]
\PassOptionsToClass{10pt,a4paper}{article}
\documentclass{ltxdoc}

\usepackage[margin=35mm]{geometry}
\usepackage{hyperref}
\usepackage{hyperxmp}
\usepackage[usenames]{color}

\hypersetup{colorlinks=true}
\hypersetup{pdfstartview=FitH}
\hypersetup{pdfpagemode=UseNone}
\hypersetup{pdfsource={}}
\hypersetup{pdflang={en-UK}}
\hypersetup{pdfcopyright={Copyright 2017-2018 Niklas Beisert.
  This work may be distributed and/or modified under the
  conditions of the LaTeX Project Public License, either version 1.3
  of this license or (at your option) any later version.}}
\hypersetup{pdflicenseurl={http://www.latex-project.org/lppl.txt}}
\hypersetup{pdfcontactaddress={ETH Zurich, ITP, HIT K,
  Wolfgang-Pauli-Strasse 27}}
\hypersetup{pdfcontactpostcode={8093}}
\hypersetup{pdfcontactcity={Zurich}}
\hypersetup{pdfcontactcountry={Switzerland}}
\hypersetup{pdfcontactemail={nbeisert@itp.phys.ethz.ch}}
\hypersetup{pdfcontacturl={http://people.phys.ethz.ch/\xmptilde nbeisert/}}

\newcommand{\secref}[1]{\hyperref[#1]{section \ref*{#1}}}

\parskip1ex
\parindent0pt
\let\olditemize\itemize
\def\itemize{\olditemize\parskip0pt}

\begin{document}

\title{The \textsf{childdoc} Package}
\hypersetup{pdftitle={The childdoc Package}}
\author{Niklas Beisert\\[2ex]
  Institut f\"ur Theoretische Physik\\
  Eidgen\"ossische Technische Hochschule Z\"urich\\
  Wolfgang-Pauli-Strasse 27, 8093 Z\"urich, Switzerland\\[1ex]
  \href{mailto:nbeisert@itp.phys.ethz.ch}
  {\texttt{nbeisert@itp.phys.ethz.ch}}}
\hypersetup{pdfauthor={Niklas Beisert}}
\hypersetup{pdfsubject={Manual for the LaTeX2e Package childdoc}}
\date{30 December 2018, \textsf{v2.0}}
\maketitle

\begin{abstract}\noindent
\textsf{childdoc} is a \LaTeXe{} package
that enables the direct compilation
of document sections included by |\include|
to individual files.
\end{abstract}

\begingroup
\parskip0ex
\tableofcontents
\endgroup

%%%%%%%%%%%%%%%%%%%%%%%%%%%%%%%%%%%%%%%%%%%%%%%%%%%%%%%%%%%%%%%%%%%%%%%%%%%%%%%%
%%%%%%%%%%%%%%%%%%%%%%%%%%%%%%%%%%%%%%%%%%%%%%%%%%%%%%%%%%%%%%%%%%%%%%%%%%%%%%%%
\section{Introduction}

\LaTeX{} provides a mechanism to structure a large document (such as a book)
into a main file and several child files (containing the chapters)
using the |\include| command.
This mechanism is beneficial for documents
which span hundreds of pages in order to
make the source file(s) more manageable.
Moreover, compilation can be restricted to
selected child files by means of the |\includeonly| command.
The latter feature can be used to reduce the compilation time while editing
(this was significantly more useful in the earlier days of \LaTeX{})
or to generate a smaller document which is easier to navigate.
Another application of |\includeonly| is to generate
documents consisting of selected parts of the complete document.

However, there are a few drawbacks of the plain |\include| mechanism:
\begin{itemize}
\item
The child files cannot be compiled on their own,
they can only be compiled via the main file.
A naive editing environment
(such as a text editor with an option
to have the current file processed by \LaTeX)
may require one to switch to the main file before compiling;
attempting to compile the child file produces errors.
\item
The main file must be modified (each time)
to adjust the |\includeonly| command
to the present needs. This easily leaves the main file in a messy state.
\item
The generated document will always carry the filename
of the main document. This is inconvenient if
several child files are to be compiled and
to be kept for distribution.
\end{itemize}

The present package provides a simple interface
to make child files individually compilable by \LaTeX{}.
Compiling a child file then has the same effect as compiling
the main file with an |\includeonly| command
to select the appropriate child.
Moreover the generated document will carry the name of the child
rather than the main file.
This resolves all three above issues.

This feature is meant to make the editing of books,
thesis documents and lecture notes somewhat more convenient.
However, the package can also be used efficiently for
composing a series of documents (such as exercise sheets)
which are typically distributed individually.
It then assists the author in generating the individual documents
(potentially in different versions)
as well as a document containing the collected series.
Another application is in developing style files
or other kinds of included material
where compilation of the style file could redirect
to a sample or test file.

%%%%%%%%%%%%%%%%%%%%%%%%%%%%%%%%%%%%%%%%%%%%%%%%%%%%%%%%%%%%%%%%%%%%%%%%%%%%%%%%
%%%%%%%%%%%%%%%%%%%%%%%%%%%%%%%%%%%%%%%%%%%%%%%%%%%%%%%%%%%%%%%%%%%%%%%%%%%%%%%%
\section{Usage}

First of all, the package \textsf{childdoc} is \emph{not} a standard
\LaTeXe{} |.sty| style file! Therefore it needs to be invoked in
a non-standard way.

%%%%%%%%%%%%%%%%%%%%%%%%%%%%%%%%%%%%%%%%%%%%%%%%%%%%%%%%%%%%%%%%%%%%%%%%%%%%%%%%
\subsection{Included Files}
\label{sec:include}

%%%%%%%%%%%%%%%%%%%%%%%%%%%%%%%%%%%%%%%%
\DescribeMacro{\childdocmain}
To use the package, add the commands
\begin{center}
\begin{tabular}{l}
|\input{childdoc.def}|\\
|\childdocmain{}|\\
\end{tabular}
\end{center}
at the very top of the main \LaTeX{} file,
in particular \emph{before} the |\documentclass| statement!
The argument of |\childdocmain| should be left empty
(but it must be present).

%%%%%%%%%%%%%%%%%%%%%%%%%%%%%%%%%%%%%%%%
\DescribeMacro{\childdocof}
Furthermore, add the commands
\begin{center}
\begin{tabular}{l}
|\input{childdoc.def}|\\
|\childdocof{|\textit{main}|}|\\
\end{tabular}
\end{center}
at the top of every child file \textit{child}
which is included by |\include{|\textit{child}|}|
from within the main file
(or at least for those files to be compiled individually).
The argument \textit{main} must be the filename of the main file.

There are a couple of
considerations in setting up the main and child documents:

%%%%%%%%%%%%%%%%%%%%%%%%%%%%%%%%%%%%%%%%
\paragraph{Restrictions.}

Please note the following restrictions:
\begin{itemize}
\item
|\childdocmain| must be called with one argument \textit{main}
to ensure compatibility with earlier version of the package.
It must either be empty (|\childdocmain{}|)
or precisely match the filename of the main file in which it is specified.
See \secref{sec:detection} for further information.
\item
The filename \textit{main} must be specified without the |.tex| extension.
\item
The filename \textit{main} is case sensitive
(even in case-insensitive file systems)
due to internal string comparison.
\item
The argument \textit{main} should be fully expanded, it cannot be a macro.
\item
Subdirectories and special characters should be avoided in filenames.
\item
The command |\childdocmain{|\textit{main}|}| must be followed by a whitespace.
It should not be followed immediately by another command
or by a comment mark `|%|'.
This is because the \TeX{} parser reads the token immediately following
the argument of |\childdocmain| and puts it
at the beginning of every child section;
however, a white\-space is ignored.
\end{itemize}

%%%%%%%%%%%%%%%%%%%%%%%%%%%%%%%%%%%%%%%%
\paragraph{Content of Main File.}

It is advisable to place all content in the child files included by |\include|.
Any output contained in the main file will appear in all child documents
unless suppressed manually;
it cannot be suppressed automatically by the |\includeonly| directive
and thus should normally be avoided.
A method to include some content in the main file
by means of conditional processing is described in \secref{sec:conditional}.

%%%%%%%%%%%%%%%%%%%%%%%%%%%%%%%%%%%%%%%%
\paragraph{Page Numbering.}

When only a part of the document is compiled,
the appropriate numbering of pages
(as well as other status parameters)
is determined from the |.aux| files.
The latter contain information from previous passes.
However this information needs to propagate through
all intermediate child documents.
Therefore the page numbering in child documents may well
be inconsistent until the complete document is compiled at least once.

A useful (if unconventional) way to always ensure a consistent
page numbering is to restart the numbering in each child document
and denote the pages by `\textit{child}|.|\textit{page}'
where \textit{child} represents the chapter/section number of the child file.
This can be achieved by the command
|\numberwithin{page}{|\textit{child}|}|
of the \textsf{amsmath} package
where \textit{child} can be |chapter| or |section|
depending on the chosen structuring.
Alternatively, one can modify the macro |\thepage| appropriately
and reset the counter |page| at the start of each child file.

%%%%%%%%%%%%%%%%%%%%%%%%%%%%%%%%%%%%%%%%%%%%%%%%%%%%%%%%%%%%%%%%%%%%%%%%%%%%%%%%
\subsection{Conditional Processing}
\label{sec:conditional}

The package provides a mechanism to compile different versions
of a document. To customise the versions further some conditional processing
can come in handy to distinguish which version is being compiled.
The package provides two macros to describe the compilation context:

%%%%%%%%%%%%%%%%%%%%%%%%%%%%%%%%%%%%%%%%
\DescribeMacro{\ifchilddoc}
The conditional |\ifchilddoc| distinguishes between the compilation of
child documents and the main document:
%
\begin{center}
|\ifchilddoc |\textit{child-code}| |[|\||else |\textit{main-code}]| \||fi|
\end{center}

%%%%%%%%%%%%%%%%%%%%%%%%%%%%%%%%%%%%%%%%
\DescribeMacro{\childdocname}
\DescribeMacro{\childdocjob}
The macro |\childdocname| contains the filename (without extension)
of the main or child file being processed.
Note that |\childdocjob| will always contain the name of the main file.

%%%%%%%%%%%%%%%%%%%%%%%%%%%%%%%%%%%%%%%%
\paragraph{Title Page.}

Conditional processing can be used to include a title or banner page
in the main document when proper precautions are taken.
Importantly, the code in the main file should ensure that the page counter
(as well as other status parameters which are stored in the |.aux| files)
takes the same value after the conditional processing.
Otherwise the page numbers may take divergent values
depending on which part is compiled.

For example, a title page could be declared by:
%
\begin{center}
\begin{tabular}{l}
|\ifchilddoc\||else|\\
|\addtocounter{page}{-1}|\\
\textit{code for title page}\\
|\newpage|\\
|\||fi|
\end{tabular}
\end{center}
%
A banner page for the child documents can be generated by:
%
\begin{center}
\begin{tabular}{l}
|\ifchilddoc|\\
|\addtocounter{page}{-1}|\\
\textit{code for banner page}\\
|\newpage|\\
|\||fi|
\end{tabular}
\end{center}
%
Here one could write a message such as:
\begin{center}
|This is the part \childdocname{} of \childdocjob{}.|
\end{center}

%%%%%%%%%%%%%%%%%%%%%%%%%%%%%%%%%%%%%%%%%%%%%%%%%%%%%%%%%%%%%%%%%%%%%%%%%%%%%%%%
\subsection{Flags}
\label{sec:flags}

The package makes it easy to generate different versions
of the main or child documents.
To this end compilation flags can be defined
and assigned different default values.
They will be particularly useful in conjunction
with the forwarding mechanism described in \secref{sec:forward}.

For example, it may be useful to have a flag |\version|
which can be set to |draft| or |final|.
The document source will contain some conditional code
depending on the value of |\version|.
Suppose further, the flag should default to |final| for the main file
and to |draft| for child files
which is a natural assignment for editing the document.
This is achieved by placing the following code
in the preamble of the main document
(below the |\childdocmain| directive):
%
\begin{center}
\begin{tabular}{l}
|\ifchilddoc|\\
|\providecommand{\version}{draft}|\\
|\||else|\\
|\providecommand{\version}{final}|\\
|\||fi|
\end{tabular}
\end{center}
%
The definition by |\providecommand| makes sure
that previous definitions are not overwritten.
Further statements |\providecommand{\version}{...}|
can thus be added before the above code to override it.

For the main file, one might add a line
(between |\childdocmain| and the above block)
%
\begin{center}
|%\ifchilddoc\||else\providecommand{\version}{draft}\||fi|
\end{center}
%
which can be uncommented to produce a draft version.
Likewise one can add a line to the very top of a child file
(above the |\childdocof{|\textit{main}|}| directive)
%
\begin{center}
|%\providecommand{\version}{final}|
\end{center}
%
which can be uncommented to produce the final version of this child document.

%%%%%%%%%%%%%%%%%%%%%%%%%%%%%%%%%%%%%%%%%%%%%%%%%%%%%%%%%%%%%%%%%%%%%%%%%%%%%%%%
\subsection{Forwarding}
\label{sec:forward}

Different versions of the main or child documents
using compilation flags as described in \secref{sec:flags}
can be (permanently) stored in different files
for convenient compilation, viewing and distribution.
To this end, the package defines a command
to pass on compilation to a different file:

%%%%%%%%%%%%%%%%%%%%%%%%%%%%%%%%%%%%%%%%
\DescribeMacro{\childdocforward}
The command |\childdocforward| redirects processing to
another source file:
%
\begin{center}
\begin{tabular}{l}
|\input{childdoc.def}|\\
|\childdocforward[|\textit{main}|]{|\textit{dest}|}|\\
\end{tabular}
\end{center}
%
The argument \textit{dest} is the destination file
(without extension).
It should be the main file or one of the child files.
Note that further \textsf{childdoc} directives
such as |\childdocof| and |\childdocforward|
in the indicated file will be processed in this form.
The optional argument \textit{main}
passes on directly to the main file \textit{main}
while pretending to compile the child \textit{dest}.
This form behaves as if \textit{dest}
issues |\childdocof{|\textit{main}|}| right away,
and no further \textsf{childdoc} directives will be processed.

%%%%%%%%%%%%%%%%%%%%%%%%%%%%%%%%%%%%%%%%
\DescribeMacro{\...prefix}
In the alternative form |\childdocforwardprefix|,
%
\begin{center}
\begin{tabular}{l}
|\input{childdoc.def}|\\
|\childdocforwardprefix[|\textit{main}|]{|\textit{prefix}|}{|\textit{dest}|}|
\end{tabular}
\end{center}
%
the destination file is determined by a pattern
depending on the current file:
To make this work, the current file must be called
`{\textit{prefix}\hspace{0.2em}\textit{suffix}}'
with \textit{prefix} matching precisely the argument.
Processing is then passed on to the file
`{\textit{dest}\hspace{0.2em}\textit{suffix}}'.
Surely, the same effect is achieved by
directly specifying the
argument `{\textit{dest}\hspace{0.2em}\textit{suffix}}'
in the first form.
However, that requires to set up a different file
for each child. With the alternative form of the command
all these files can have exactly the same content
which simplifies setting them up and maintaining them.

For example, the following file |draft.tex|
with a compilation flag |\version| as described in \secref{sec:flags}
compiles the main document as a draft:
%
\begin{center}
\begin{tabular}{l}
|\def\version{draft}|\\
|\input{childdoc.def}|\\
|\childdocforward{|\textit{main}|}|
\end{tabular}
\end{center}
%
Likewise, the following files |final|\textit{nn}|.tex|
compile the final version of the child document
|child|\textit{nn}|.tex|:
%
\begin{center}
\begin{tabular}{l}
|\def\version{final}|\\
|\input{childdoc.def}|\\
|\childdocforwardprefix{final}{child}|
\end{tabular}
\end{center}
%

Note that when several versions of a main file and/or of each child file
are to be generated, it may be convenient to set up a |Makefile| or
shell script to automatise the process.

%%%%%%%%%%%%%%%%%%%%%%%%%%%%%%%%%%%%%%%%%%%%%%%%%%%%%%%%%%%%%%%%%%%%%%%%%%%%%%%%
\subsection{Command Line Processing}
\label{sec:commandline}

The effect of redirection files can also be achieved by invoking
the \LaTeX{} compiler with a more elaborate command line.
Most conveniently this should be done as part
of a shell script or a |Makefile|.

When using \textsf{childdoc} in the main file, the following
command lines effectively perform a redirection
(note that depending on the shell being used,
backslashes may have to be doubled: `|\|' $\to$ `|\\|'):
%
\begin{center}
|... -jobname "|\textit{target}|" |\\|"|[\textit{flags}]%
|\input{childdoc.def}\childdocforward[|\textit{main}|]{|\textit{dest}|}"|
\end{center}
%
Here \textit{target} is the name of the output file,
\textit{main} is the name of the main file
and \textit{dest} is the name of the main or child file to be processed
(all filenames without extensions).
The optional argument \textit{main} can be omitted
if \textit{main} matches \textit{dest}.
Optionally, compilation \textit{flags} can be defined via |\def| commands.
This command line makes the \TeX{} engine believe
it is compiling the file \textit{target}
whose content is specified as the latter parameter.
The provided code then forwards the processing to
\textit{main} or \textit{dest} as described in \secref{sec:forward}.

%%%%%%%%%%%%%%%%%%%%%%%%%%%%%%%%%%%%%%%%%%%%%%%%%%%%%%%%%%%%%%%%%%%%%%%%%%%%%%%%
\subsection{Include by Input}
\label{sec:input}

Including child documents by |\include| has some restrictions by design.
Most notably, the content of a child document always occupies
its own set of pages; pages cannot be shared between child documents.
Usually, this behaviour makes perfect sense
because each child document contain an essential part of the document.
However, in some situations it may be desirable to compose
a document from a collection of parts
without having mandatory page breaks between then.
For this case, the package
provides a mechanism to include parts
by |\input| which can also be processed individually.
However, by construction this mechanism
requires manual handling of the content to be output.

%%%%%%%%%%%%%%%%%%%%%%%%%%%%%%%%%%%%%%%%
\DescribeMacro{\ifchilddocmanual}
The main file should be prepared as usual, see \secref{sec:include}.
However, the document body must make a distinction
between processing of an individual part and of the main document, e.g.:
%
\begin{center}
\begin{tabular}{l}
|\ifchilddocmanual|\\
|\input{\childdocname}|\\
|\||else|\\
\textit{document body with }|\input{|\textit{part}|}|\\
|\||fi|
\end{tabular}
\end{center}
%
The conditional |\ifchilddocmanual| is true whenever
a part to be included by |\input| is being compiled,
and the name of the part is stored in |\childdocname|.

%%%%%%%%%%%%%%%%%%%%%%%%%%%%%%%%%%%%%%%%
\DescribeMacro{\childdocby}
Each part to be included by |\input| should start with:
%
\begin{center}
\begin{tabular}{l}
|\input{childdoc.def}|\\
|\childdocby{|\textit{main}|}|\\
\end{tabular}
\end{center}
%
The directive |\childdocby| is similar to |\childdocof|
described in \secref{sec:include},
but the subsequent selection of content must be done manually.
To that end, both |\ifchilddoc| and |\ifchilddocmanual|
will be true upon processing of a part,
and the name of the part is stored in |\childdocname|.
Note that |\jobname| will be set to the filename of the current part
so that each part receives an individual |.aux| file
that does not interfere with the |.aux| file(s) of the main document.
This behaviour can be altered by the alternative form
|\childdocby[*]{|\textit{main}|}| (with a non-empty optional argument)
which uses the |.aux| file of the main document
by setting |\jobname| to \textit{main}.

%%%%%%%%%%%%%%%%%%%%%%%%%%%%%%%%%%%%%%%%%%%%%%%%%%%%%%%%%%%%%%%%%%%%%%%%%%%%%%%%
\subsection{Driver Development}
\label{sec:driver}

The \textsf{childdoc} mechanism can also be use for the development
of definition files such as \LaTeX{} styles or classes.
This case differs from the above setup with multiple parts
included by |\include| in that no |\includeonly| should be invoked.
This can be achieved by starting the include file
(before |\ProvidesPackage|) with:
%
\begin{center}
\begin{tabular}{l}
|\input{childdoc.def}|\\
|\childdocforward{|\textit{main}|}|\\
\end{tabular}
\end{center}
%
or alternatively with:
%
\begin{center}
\begin{tabular}{l}
|\input{childdoc.def}|\\
|\childdocby{|\textit{main}|}|\\
\end{tabular}
\end{center}
%
Both forms have slightly different effects as described above.
The main file is prepared as usual, see \secref{sec:include}.

%%%%%%%%%%%%%%%%%%%%%%%%%%%%%%%%%%%%%%%%%%%%%%%%%%%%%%%%%%%%%%%%%%%%%%%%%%%%%%%%
\subsection{Legacy Detection}
\label{sec:detection}

The directive |\childdocmain| in the main file can detect
whether the complete document or merely a child is to be compiled
even without using the directive |\childdocof|.
This method is deprecated because it is less robust
and there is no compelling reason to use it;
it is merely provided for backward compatibility
and it may be removed in future versions.

If the detection mechanism is to be used,
it is mandatory to correctly specify
the filename of the main file as the argument of |\childdocmain|:
%
\begin{center}
\begin{tabular}{l}
|\input{childdoc.def}|\\
|\childdocmain{|\textit{main}|}|\\
\end{tabular}
\end{center}
%
If |\jobname| does not match the argument \textit{main} of |\childdocmain|,
it is assumed that |\jobname| points to the child file to be compiled.
When using |\childdocmain| with the main file specified as argument,
it suffices to start a child file
with just |\input{|\textit{main}|}|
without loading of the package and using |\childdocof|.
If instead all processing is done
with the appropriate \textsf{childdoc} directives,
the argument of \textit{main} of |\childdocmain| can be empty.

An alternative version of the command line processing described
in \secref{sec:commandline} using the detection mechanism reads:
%
\begin{center}
|... -jobname "|\textit{target}|" "|[\textit{flags}]%
[|\def\jobname{|\textit{dest}|}|]|\input{|\textit{main}|}"|
\end{center}

%%%%%%%%%%%%%%%%%%%%%%%%%%%%%%%%%%%%%%%%%%%%%%%%%%%%%%%%%%%%%%%%%%%%%%%%%%%%%%%%
\subsection{Manual Code}
\label{sec:manual}

In case one cannot be certain whether the definitions file |childdoc.def|
is installed on the target \TeX{} distribution
and one prefers not to ship it,
it is conceivable to paste a few relevant commands into the sources.

To that end, drop all statements |\input{childdoc.def}|
and perform the replacements as outlined below.
Instead of |\childdocmain{|\textit{main}|}| add the following code
to the top of the main file:
%
\begin{center}
\begin{tabular}{l}
|\||ifdefined\childdocname\endinput\||fi\newif\ifchilddoc|\\
|\edef\childdocname{\scantokens\expandafter{\jobname\noexpand}}|\\
|\def\childdocmain{|\textit{main}|}\||ifx\childdocmain\childdocname\||else|\\
|\childdoctrue\includeonly{\childdocname}\let\jobname\childdocmain\||fi|\\
\end{tabular}
\end{center}
%
Instead of |\childdocof{|\textit{main}|}| just include the main file
at the top of each child file:
%
\begin{center}
|\input{|\textit{main}|}|
\end{center}
%
A simple redirection |\childdocforward{|\textit{dest}|}| is achieved by:
%
\begin{center}
|\def\jobname{|\textit{dest}|}\input{\jobname}|
\end{center}
%
The redirection with prefix
|\childdocforwardprefix[|\textit{prefix}|]{|\textit{dest}|}|
is accomplished by:
%
\begin{center}
\begin{tabular}{l}
|{\edef\jobname{\scantokens\expandafter{\jobname\noexpand}}|\\
|\def\redirectjob |\textit{prefix}|#1~~~{\gdef\jobname{|\textit{dest}|#1}}|\\
|\expandafter\redirectjob\jobname~~~}\input{\jobname}|
\end{tabular}
\end{center}

In an alternative approach,
child documents can be compiled by a specific command line
without additional code or specific definitions:
%
\begin{center}
|... -jobname "|\textit{target}|" "|[\textit{flags}]%
|\includeonly{|\textit{dest}|}\input{|\textit{main}|}"|
\end{center}
%

%%%%%%%%%%%%%%%%%%%%%%%%%%%%%%%%%%%%%%%%%%%%%%%%%%%%%%%%%%%%%%%%%%%%%%%%%%%%%%%%
%%%%%%%%%%%%%%%%%%%%%%%%%%%%%%%%%%%%%%%%%%%%%%%%%%%%%%%%%%%%%%%%%%%%%%%%%%%%%%%%
\section{Information}

%%%%%%%%%%%%%%%%%%%%%%%%%%%%%%%%%%%%%%%%%%%%%%%%%%%%%%%%%%%%%%%%%%%%%%%%%%%%%%%%
\subsection{Copyright}

Copyright \copyright{} 2017--2018 Niklas Beisert

This work may be distributed and/or modified under the
conditions of the \LaTeX{} Project Public License, either version 1.3
of this license or (at your option) any later version.
The latest version of this license is in
  \url{http://www.latex-project.org/lppl.txt}
and version 1.3 or later is part of all distributions of \LaTeX{}
version 2005/12/01 or later.

This work has the LPPL maintenance status `maintained'.

The Current Maintainer of this work is Niklas Beisert.

This work consists of the files |README.txt|, |childdoc.ins| and |childdoc.dtx|
as well as the derived files |childdoc.def|, |cdocsamp.tex|
with |cdocsch1.tex|, |cdocsch2.tex|, |cdocspt3.tex|, |cdocspt4.tex|,
|cdocsdrf.tex|, |cdocsfn1.tex|, |cdocsfn2.tex|
as well as |childdoc.pdf|.

%%%%%%%%%%%%%%%%%%%%%%%%%%%%%%%%%%%%%%%%%%%%%%%%%%%%%%%%%%%%%%%%%%%%%%%%%%%%%%%%
\subsection{Files and Installation}

The package consists of the files:
%
\begin{center}
\begin{tabular}{ll}
    |README.txt|   & readme file \\
    |childdoc.ins| & installation file \\
    |childdoc.dtx| & source file \\
    |childdoc.def| & definition file \\
    |cdocsamp.tex| & sample main file \\
    |cdocsch1.tex| & sample include file \\
    |cdocsch2.tex| & sample include file \\
    |cdocspt3.tex| & sample part file \\
    |cdocspt4.tex| & sample part file \\
    |cdocsdrf.tex| & sample redirection file \\
    |cdocsfn1.tex| & sample redirection file \\
    |cdocsfn2.tex| & sample redirection file \\
    |childdoc.pdf| & manual
\end{tabular}
\end{center}
%
The distribution consists of the files
|README.txt|, |childdoc.ins| and |childdoc.dtx|.
%
\begin{itemize}
\item
Run (pdf)\LaTeX{} on |childdoc.dtx|
to compile the manual |childdoc.pdf| (this file).
\item
Run \LaTeX{} on |childdoc.ins| to create the definitions file |childdoc.def|
and the sample |cdocsamp.tex| with include files
|cdocsch1.tex|, |cdocsch2.tex|, |cdocspt3.tex|, |cdocspt4.tex|,
|cdocsdrf.tex|, |cdocsfn1.tex|, |cdocsfn2.tex|.
Then copy the file |childdoc.def| to an appropriate directory of your \LaTeX{}
distribution, e.g.\ \textit{texmf-root}|/tex/latex/childdoc|.
\end{itemize}

%%%%%%%%%%%%%%%%%%%%%%%%%%%%%%%%%%%%%%%%%%%%%%%%%%%%%%%%%%%%%%%%%%%%%%%%%%%%%%%%
\subsection{Related CTAN Packages}

There are several other packages which offer a similar functionality:
%
\begin{itemize}
\item
The packages
\href{http://ctan.org/pkg/docmute}{\textsf{docmute}},
\href{http://ctan.org/pkg/includex}{\textsf{includex}} and
\href{http://ctan.org/pkg/standalone}{\textsf{standalone}}
provide commands to include only the document body of
a child file thus allowing both files to be compiled individually.
\item
The packages \href{http://ctan.org/pkg/subdocs}{\textsf{subdocs}}
and \href{http://ctan.org/pkg/subfiles}{\textsf{subfiles}}
provide structures in which the main and child documents can be
encapsulated and allowing them to be compiled individually.
The inclusion mechanism is different from the conventional |\include|.
\item
The package \href{http://ctan.org/pkg/combine}{\textsf{combine}}
is an elaborate solution to combine several documents into one.
\end{itemize}
%
See also the CTAN topic \href{http://ctan.org/topic/subdocs}{\textsf{subdocs}}
for further related packages.
The present package differs from the above solutions in that
a document structure constructed with the conventional |\include| mechanism
just needs two extra commands at the top of every file
such that all constituent files can be compiled individually.

%%%%%%%%%%%%%%%%%%%%%%%%%%%%%%%%%%%%%%%%%%%%%%%%%%%%%%%%%%%%%%%%%%%%%%%%%%%%%%%%
%\subsection{Feature Suggestions}
%
%The following is a list of features which may be useful for future
%versions of this package:
%%
%\begin{itemize}
%\item
%\ldots
%\end{itemize}

%%%%%%%%%%%%%%%%%%%%%%%%%%%%%%%%%%%%%%%%%%%%%%%%%%%%%%%%%%%%%%%%%%%%%%%%%%%%%%%%
\subsection{Revision History}

%%%%%%%%%%%%%%%%%%%%%%%%%%%%%%%%%%%%%%%%
\paragraph{v2.0:} 2018/12/30

\begin{itemize}
\item
immediate forward processing
\item
added |\childdocby| mechanism
\item
manual restructured
\end{itemize}

%%%%%%%%%%%%%%%%%%%%%%%%%%%%%%%%%%%%%%%%
\paragraph{v1.6:} 2018/01/17

\begin{itemize}
\item
application for development of include files
\item
corrections to manual
\end{itemize}

%%%%%%%%%%%%%%%%%%%%%%%%%%%%%%%%%%%%%%%%
\paragraph{v1.5:} 2017/05/21

\begin{itemize}
\item
more complete structuring introduced
\item
|\childdocof| introduced
\item
|\childdoc| renamed to |\childdocmain|
\item
|\childredirect| renamed to |\childdocforward| and |\childdocforwardprefix|
and functionality expanded
\end{itemize}

%%%%%%%%%%%%%%%%%%%%%%%%%%%%%%%%%%%%%%%%
\paragraph{v1.0:} 2017/04/27

\begin{itemize}
\item
manual and install package
\item
first version published on CTAN
\end{itemize}

%%%%%%%%%%%%%%%%%%%%%%%%%%%%%%%%%%%%%%%%
\paragraph{v0.6:} 2017/04/26

\begin{itemize}
\item
redirection mechanism added
\end{itemize}

%%%%%%%%%%%%%%%%%%%%%%%%%%%%%%%%%%%%%%%%
\paragraph{v0.5:} 2017/04/26

\begin{itemize}
\item
functionality in definition file
\end{itemize}


%%%%%%%%%%%%%%%%%%%%%%%%%%%%%%%%%%%%%%%%%%%%%%%%%%%%%%%%%%%%%%%%%%%%%%%%%%%%%%%%
%%%%%%%%%%%%%%%%%%%%%%%%%%%%%%%%%%%%%%%%%%%%%%%%%%%%%%%%%%%%%%%%%%%%%%%%%%%%%%%%
%%%%%%%%%%%%%%%%%%%%%%%%%%%%%%%%%%%%%%%%%%%%%%%%%%%%%%%%%%%%%%%%%%%%%%%%%%%%%%%%
\appendix

\settowidth\MacroIndent{\rmfamily\scriptsize 000\ }

 \DocInput{childdoc.dtx}

\end{document}
%</driver>
% \fi
%
% %%%%%%%%%%%%%%%%%%%%%%%%%%%%%%%%%%%%%%%%%%%%%%%%%%%%%%%%%%%%%%%%%%%%%%%%%%%%%%
% %%%%%%%%%%%%%%%%%%%%%%%%%%%%%%%%%%%%%%%%%%%%%%%%%%%%%%%%%%%%%%%%%%%%%%%%%%%%%%
% \section{Sample}
%\iffalse
%<*samplemain>
%\fi
%
% The following presents a sample document
% with two chapters, two parts, a title page,
% a compile flag as well as three forwarding files to set the flag.
% It consists of eight |.tex| files:
% \begin{center}
% \begin{tabular}{ll}
% |cdocsamp.tex|&main file\\
% |cdocsch1.tex|&include file for chapter 1\\
% |cdocsch2.tex|&include file for chapter 2\\
% |cdocspt3.tex|&include file for part 3\\
% |cdocspt4.tex|&include file for part 4\\
% |cdocsdrf.tex|&forwarding file for main file in draft mode\\
% |cdocsfi1.tex|&forwarding file for final version of chapter 1\\
% |cdocsfi2.tex|&forwarding file for final version of chapter 2\\
% \end{tabular}
% \end{center}
% Each of the eight files can be compiled directly by the \LaTeX{} compiler.
%
% %%%%%%%%%%%%%%%%%%%%%%%%%%%%%%%%%%%%%%
% \paragraph{Main File.}
%
% The main file is called |cdocsamp.tex|.
%
% Load the \textsf{childdoc} definitions and
% declare the filename for the main document:
%    \begin{macrocode}
\input{childdoc.def}
\childdocmain{}
%    \end{macrocode}

% Optional override for |\version| flag:
%    \begin{macrocode}
%%\ifchilddoc\else\providecommand{\version}{draft}\fi
%    \end{macrocode}

% Define the default values for the |\version| flag
% (|final| for the main file and |draft| for childs):
%    \begin{macrocode}
\ifchilddoc
\providecommand{\version}{draft}
\else
\providecommand{\version}{final}
\fi
%    \end{macrocode}

% Load the standard document class:
%    \begin{macrocode}
\documentclass[12pt]{article}
%    \end{macrocode}

% Start the document body:
%    \begin{macrocode}
\begin{document}
%    \end{macrocode}

% Declare a title page.
% Print title, part of document being processed and version flag:
%    \begin{macrocode}
\addtocounter{page}{-1}
\begin{center}
{\LARGE\bfseries{}childdoc example\par}
\vspace{1cm}
\ifchilddoc
\ifchilddocmanual part\else chapter\fi:
`\childdocname' of `\childdocjob'\par
\else
main document: `\childdocjob'\par
\fi
version: \version\par
\end{center}
\newpage
%    \end{macrocode}

% Manually include selected file,
% otherwise process as usual:
%    \begin{macrocode}
\ifchilddocmanual
\section*{part `\childdocname'}
\input{\childdocname}
\else
%    \end{macrocode}

% Include the two chapters:
%    \begin{macrocode}
\include{cdocsch1}
\include{cdocsch2}
%    \end{macrocode}

% Include the two parts unless only chapters should be displayed:
%    \begin{macrocode}
\ifchilddoc\else
\section{part three}
\input{cdocspt3}
\section{part four}
\input{cdocspt4}
\fi
%    \end{macrocode}

% Process as usual until here:
%    \begin{macrocode}
\fi
%    \end{macrocode}

% End of document body:
%    \begin{macrocode}
\end{document}
%    \end{macrocode}
%\iffalse
%</samplemain>
%\fi
%
% %%%%%%%%%%%%%%%%%%%%%%%%%%%%%%%%%%%%%%
% \paragraph{Chapter Include Files.}
%
% The include files are called |cdocsch1.tex| and |cdocsch2.tex|.
%
%\iffalse
%<*samplechap1|samplechap2>
%\fi

% Optional override for |\version| flag:
%    \begin{macrocode}
%%\providecommand{\version}{final}
%    \end{macrocode}

% Include the main document:
%    \begin{macrocode}
\input{childdoc.def}
\childdocof{cdocsamp}
%    \end{macrocode}

%\iffalse
%</samplechap1|samplechap2>
%\fi
%
%\iffalse
%<*samplechap1>
%\fi
% Some text for chapter 1:
%    \begin{macrocode}
\section{one}
some text in chapter one
%    \end{macrocode}

%\iffalse
%</samplechap1>
%\fi
% Some text for chapter 2:
%\iffalse
%<*samplechap2>
%\fi
%    \begin{macrocode}
\section{two}
more text in chapter two
%    \end{macrocode}

%\iffalse
%</samplechap2>
%\fi
%
% %%%%%%%%%%%%%%%%%%%%%%%%%%%%%%%%%%%%%%
% \paragraph{Part Include Files.}
%
% The include files are called |cdocspt3.tex| and |cdocspt4.tex|.
%
%\iffalse
%<*samplepart3|samplepart4>
%\fi

% Optional override for |\version| flag:
%    \begin{macrocode}
%%\providecommand{\version}{final}
%    \end{macrocode}

% Include the main document:
%    \begin{macrocode}
\input{childdoc.def}
\childdocby{cdocsamp}
%    \end{macrocode}

%\iffalse
%</samplepart3|samplepart4>
%\fi
%
%\iffalse
%<*samplepart3>
%\fi
% Some text for part 3:
%    \begin{macrocode}
some text in part three
%    \end{macrocode}

%\iffalse
%</samplepart3>
%\fi
% Some text for part 4:
%\iffalse
%<*samplepart4>
%\fi
%    \begin{macrocode}
more text in part four
%    \end{macrocode}

%\iffalse
%</samplepart4>
%\fi
%
% %%%%%%%%%%%%%%%%%%%%%%%%%%%%%%%%%%%%%%
% \paragraph{Forwarding for a Complete Draft.}
%
% The following forwarding file |cdocsdrf.tex|
% compiles the main document in draft mode:
%\iffalse
%<*sampledraft>
%\fi
%    \begin{macrocode}
\def\version{draft}
\input{childdoc.def}
\childdocforward{cdocsamp}
%    \end{macrocode}

%\iffalse
%</sampledraft>
%\fi
%
% %%%%%%%%%%%%%%%%%%%%%%%%%%%%%%%%%%%%%%
% \paragraph{Forwarding for Final Version of the Chapters.}
%
% The following forwarding files |cdocsfn1.tex| and |cdocsfn2.tex|
% (with identical content)
% compile the final versions of the child documents
% |cdocsch1.tex| and |cdocsch2.tex|, respectively:
%\iffalse
%<*samplefinal>
%\fi
%    \begin{macrocode}
\def\version{final}
\input{childdoc.def}
\childdocforwardprefix[cdocsamp]{cdocsfn}{cdocsch}
%    \end{macrocode}

%\iffalse
%</samplefinal>
%\fi
%
% %%%%%%%%%%%%%%%%%%%%%%%%%%%%%%%%%%%%%%
% \paragraph{Command Line Processing.}
%
% The following three command lines generate the output files
% |cdocscld|, |cdocscl1| and |cdocscl2|
% which should be identical to
% |cdocsdrf|, |cdocsch1| and |cdocsfn2|, respectively:
% \begin{center}
% \begin{tabular}{l}
% |latex -jobname cdocscld \|\\
% |  "\def\version{draft}\input{childdoc.def}\childdocforward{cdocsamp}"|\\
% |latex -jobname cdocscl1 \|\\
% |  "\input{childdoc.def}\childdocforward[cdocsamp]{cdocsch1}"|\\
% |latex -jobname cdocscl2 \|\\
% |  "\def\version{final}\input{childdoc.def}\childdocforward{cdocsch2}"|
% \end{tabular}
% \end{center}
% Note that the trailing backslash on each first line
% merely continues the input to the second line
% (for convenient cut ant paste).
% Furthermore, the command |latex| can be replaced by any
% of its alternative versions such as |pdflatex|.
%
% %%%%%%%%%%%%%%%%%%%%%%%%%%%%%%%%%%%%%%%%%%%%%%%%%%%%%%%%%%%%%%%%%%%%%%%%%%%%%%
% %%%%%%%%%%%%%%%%%%%%%%%%%%%%%%%%%%%%%%%%%%%%%%%%%%%%%%%%%%%%%%%%%%%%%%%%%%%%%%
% \section{Implementation}
%\iffalse
%<*package>
%\fi
%
% This section describes the definitions file |childdoc.def|.

% The definitions cannot be loaded using |\usepackage| or |\RequirePackage|
% which has a mechanism to prevent loading a style file more than once.
% When loading the definitions by means of |\input|
% multiple instances have to be prevented manually:
%\iffalse
%This code needs to be before the `\ProvidesFile' directive
%which is defined at the beginning of this file.
%Therefore it is also placed there and commented out here.
%</package>
%<*discard>
%\fi
%    \begin{macrocode}
\ifdefined\childdocmain\endinput\fi
%    \end{macrocode}
%\iffalse
%</discard>
%<*package>
%\fi
%
% \macro{\ifchilddoc}
% \macro{\ifchilddocmanual}
% The conditional |\ifchilddoc| tells whether a
% child (true) or main (false) document is being compiled.
% The conditional |\ifchilddocmanual| tells whether
% the |\includeonly| mechanism is used (false) or
% the selection of child files must be performed manually (true).
% The definitions initialise to false:
%    \begin{macrocode}
\newif\ifchilddoc
\newif\ifchilddocmanual
%    \end{macrocode}

% \macro{\childdocname}
% \macro{\childdocjob}
% The macro |\childdocname| stores the name of the main document
% to be compiled. The macro |\childdocjob| stores the name of
% the document on which the \LaTeX{} compiler was originally invoked.
% The content of |\jobname| cannot be compared
% to filenames specified in the source due to different catcodes.
% The following code rescans |\jobname|, stores the result
% in |\childdocname| and saves a copy in |\childdocjob|:
%    \begin{macrocode}
\edef\childdocname{\scantokens\expandafter{\jobname\noexpand}}
\let\childdocjob\childdocname
%    \end{macrocode}

% \macro{\childdocdisable}
% The macro |\childdocdisable| prevents the main file
% from being processed more than once.
% At this stage, the main document command |\childdocmain|
% is assumed to be called once again where it should do nothing.
% Any subsequent call to it should prevent
% a secondary processing of the main document
% It overwrites the forwarding commands
% |\childdocof| and |\childdocforward|
% with empty macros to prevent further inclusions of the main document:
%    \begin{macrocode}
\newcommand{\childdocdisable}
{
  \renewcommand{\childdocmain}[1]{\renewcommand{\childdocmain}[1]{\endinput}}
  \renewcommand{\childdocof}[1]{}
  \renewcommand{\childdocby}[2][]{}
  \renewcommand{\childdocforward}[2][]{}
  \renewcommand{\childdocdisable}{}
}
%    \end{macrocode}

% \macro{\childdocmain}
% The macro |\childdocmain| is to be called at the top of the main file
% with nothing or the main filename (without extension) as argument.
% First, it breaks loops.
% If the argument is not empty and does not match |\childdocname|
% (which is set by the first inclusion of |childdoc.def|),
% |\ifchilddoc| is set to true, |\includeonly| is applied to the child file
% and |\jobname| is set to the main file
% (for proper handling of |.aux| files):
%    \begin{macrocode}
\newcommand{\childdocmain}[1]
{
  \childdocdisable\childdocmain{}
  \if?#1?\else
    \begingroup
      \def\childdoctmp{#1}
      \ifx\childdoctmp\childdocname
        \def\childdoctmp{}
      \else
        \def\childdoctmp
        {
          \childdoctrue
          \includeonly{\childdocname}
          \def\childdocjob{#1}
          \def\jobname{#1}
        }
      \fi
      \expandafter
    \endgroup
    \childdoctmp
  \fi
}
%    \end{macrocode}

% \macro{\childdocof}
% The command |\childdocof| redirects
% compilation to the main file |#1|.
%    \begin{macrocode}
\newcommand{\childdocof}[1]
{
  \childdocdisable
  \childdoctrue
  \includeonly{\childdocname}
  \def\jobname{#1}
  \def\childdocjob{#1}
  \input{#1}
}
%    \end{macrocode}

% \macro{\childdocby}
% The command |\childdocby| ....
%    \begin{macrocode}
\newcommand{\childdocby}[2][]
{
  \childdocdisable
  \childdoctrue
  \childdocmanualtrue
  \if?#1?\else
    \def\jobname{#2}
  \fi
  \def\childdocjob{#2}
  \input{#2}
  \endinput
}
%    \end{macrocode}

% \macro{\childdocforward}
% The command |\childdocforward| redirects
% compilation to the main file or
% (if the optional argument is given) a child file.
% Parameters are set as if the main file
% or a child file starting with |\childdocof| was compiled.
% Then compilation is handed over to the main file:
%    \begin{macrocode}
\newcommand{\childdocforward}[2][]
{
  \begingroup
    \if?#1?
      \def\childdoctmp
      {
        \def\childdocname{#2}
        \def\childdocjob{#2}
        \def\jobname{#2}
        \input{#2}
        \endinput
      }
    \else
      \def\childdoctmp
      {
        \childdocdisable
        \def\childdocname{#2}
        \childdoctrue
        \includeonly{#2}
        \def\childdocjob{#1}
        \def\jobname{#1}
        \input{#1}
        \endinput
      }
    \fi
    \expandafter
  \endgroup
  \childdoctmp
}
%    \end{macrocode}

% \macro{\childdocforwardprefix}
% The command |\childdocforwardprefix| redirects
% compilation to the main or a child file by means of a pattern.
% The prefix |#1| in the current filename is replaced by |#2|
% and the suffix of the current filename is kept
% (it is assumed that the filename does not contain the substring `|~~~|'
% which is used as a delimiter).
% Compilation is handed over to the new file by |\childdocforward|:
%    \begin{macrocode}
\newcommand{\childdocforwardprefix}[3][]
{
  \begingroup
    \def\childdocextract #2##1~~~{\def\childdoctmp{\childdocforward[#1]{#3##1}}}
    \expandafter\childdocextract\childdocname~~~
    \expandafter
  \endgroup
  \childdoctmp
}
%    \end{macrocode}

% \macro{\childdoc}
% The deprecated macro |\childdoc| is a legacy version of |\childdocmain|:
%    \begin{macrocode}
\newcommand{\childdoc}{\childdocmain}
%    \end{macrocode}

% \macro{\childdocredirect}
% The deprecated macro |\childdocredirect| is a legacy version
% of |\childdocforward| and |\childdocforwardprefix|:
%    \begin{macrocode}
\newcommand{\childdocredirect}[2][]
{
  \begingroup
    \if?#1?
      \def\childdoctmp{\childdocforward{#2}}
    \else
      \def\childdoctmp{\childdocforwardprefix{#1}{#2}}
    \fi
    \expandafter
  \endgroup
  \childdoctmp
}
%    \end{macrocode}

%\iffalse
%</package>
%\fi
%
\endinput
\childdocforward{cdocsch2}"|
% \end{tabular}
% \end{center}
% Note that the trailing backslash on each first line
% merely continues the input to the second line
% (for convenient cut ant paste).
% Furthermore, the command |latex| can be replaced by any
% of its alternative versions such as |pdflatex|.
%
% %%%%%%%%%%%%%%%%%%%%%%%%%%%%%%%%%%%%%%%%%%%%%%%%%%%%%%%%%%%%%%%%%%%%%%%%%%%%%%
% %%%%%%%%%%%%%%%%%%%%%%%%%%%%%%%%%%%%%%%%%%%%%%%%%%%%%%%%%%%%%%%%%%%%%%%%%%%%%%
% \section{Implementation}
%\iffalse
%<*package>
%\fi
%
% This section describes the definitions file |childdoc.def|.

% The definitions cannot be loaded using |\usepackage| or |\RequirePackage|
% which has a mechanism to prevent loading a style file more than once.
% When loading the definitions by means of |\input|
% multiple instances have to be prevented manually:
%\iffalse
%This code needs to be before the `\ProvidesFile' directive
%which is defined at the beginning of this file.
%Therefore it is also placed there and commented out here.
%</package>
%<*discard>
%\fi
%    \begin{macrocode}
\ifdefined\childdocmain\endinput\fi
%    \end{macrocode}
%\iffalse
%</discard>
%<*package>
%\fi
%
% \macro{\ifchilddoc}
% \macro{\ifchilddocmanual}
% The conditional |\ifchilddoc| tells whether a
% child (true) or main (false) document is being compiled.
% The conditional |\ifchilddocmanual| tells whether
% the |\includeonly| mechanism is used (false) or
% the selection of child files must be performed manually (true).
% The definitions initialise to false:
%    \begin{macrocode}
\newif\ifchilddoc
\newif\ifchilddocmanual
%    \end{macrocode}

% \macro{\childdocname}
% \macro{\childdocjob}
% The macro |\childdocname| stores the name of the main document
% to be compiled. The macro |\childdocjob| stores the name of
% the document on which the \LaTeX{} compiler was originally invoked.
% The content of |\jobname| cannot be compared
% to filenames specified in the source due to different catcodes.
% The following code rescans |\jobname|, stores the result
% in |\childdocname| and saves a copy in |\childdocjob|:
%    \begin{macrocode}
\edef\childdocname{\scantokens\expandafter{\jobname\noexpand}}
\let\childdocjob\childdocname
%    \end{macrocode}

% \macro{\childdocdisable}
% The macro |\childdocdisable| prevents the main file
% from being processed more than once.
% At this stage, the main document command |\childdocmain|
% is assumed to be called once again where it should do nothing.
% Any subsequent call to it should prevent
% a secondary processing of the main document
% It overwrites the forwarding commands
% |\childdocof| and |\childdocforward|
% with empty macros to prevent further inclusions of the main document:
%    \begin{macrocode}
\newcommand{\childdocdisable}
{
  \renewcommand{\childdocmain}[1]{\renewcommand{\childdocmain}[1]{\endinput}}
  \renewcommand{\childdocof}[1]{}
  \renewcommand{\childdocby}[2][]{}
  \renewcommand{\childdocforward}[2][]{}
  \renewcommand{\childdocdisable}{}
}
%    \end{macrocode}

% \macro{\childdocmain}
% The macro |\childdocmain| is to be called at the top of the main file
% with nothing or the main filename (without extension) as argument.
% First, it breaks loops.
% If the argument is not empty and does not match |\childdocname|
% (which is set by the first inclusion of |childdoc.def|),
% |\ifchilddoc| is set to true, |\includeonly| is applied to the child file
% and |\jobname| is set to the main file
% (for proper handling of |.aux| files):
%    \begin{macrocode}
\newcommand{\childdocmain}[1]
{
  \childdocdisable\childdocmain{}
  \if?#1?\else
    \begingroup
      \def\childdoctmp{#1}
      \ifx\childdoctmp\childdocname
        \def\childdoctmp{}
      \else
        \def\childdoctmp
        {
          \childdoctrue
          \includeonly{\childdocname}
          \def\childdocjob{#1}
          \def\jobname{#1}
        }
      \fi
      \expandafter
    \endgroup
    \childdoctmp
  \fi
}
%    \end{macrocode}

% \macro{\childdocof}
% The command |\childdocof| redirects
% compilation to the main file |#1|.
%    \begin{macrocode}
\newcommand{\childdocof}[1]
{
  \childdocdisable
  \childdoctrue
  \includeonly{\childdocname}
  \def\jobname{#1}
  \def\childdocjob{#1}
  \input{#1}
}
%    \end{macrocode}

% \macro{\childdocby}
% The command |\childdocby| ....
%    \begin{macrocode}
\newcommand{\childdocby}[2][]
{
  \childdocdisable
  \childdoctrue
  \childdocmanualtrue
  \if?#1?\else
    \def\jobname{#2}
  \fi
  \def\childdocjob{#2}
  \input{#2}
  \endinput
}
%    \end{macrocode}

% \macro{\childdocforward}
% The command |\childdocforward| redirects
% compilation to the main file or
% (if the optional argument is given) a child file.
% Parameters are set as if the main file
% or a child file starting with |\childdocof| was compiled.
% Then compilation is handed over to the main file:
%    \begin{macrocode}
\newcommand{\childdocforward}[2][]
{
  \begingroup
    \if?#1?
      \def\childdoctmp
      {
        \def\childdocname{#2}
        \def\childdocjob{#2}
        \def\jobname{#2}
        \input{#2}
        \endinput
      }
    \else
      \def\childdoctmp
      {
        \childdocdisable
        \def\childdocname{#2}
        \childdoctrue
        \includeonly{#2}
        \def\childdocjob{#1}
        \def\jobname{#1}
        \input{#1}
        \endinput
      }
    \fi
    \expandafter
  \endgroup
  \childdoctmp
}
%    \end{macrocode}

% \macro{\childdocforwardprefix}
% The command |\childdocforwardprefix| redirects
% compilation to the main or a child file by means of a pattern.
% The prefix |#1| in the current filename is replaced by |#2|
% and the suffix of the current filename is kept
% (it is assumed that the filename does not contain the substring `|~~~|'
% which is used as a delimiter).
% Compilation is handed over to the new file by |\childdocforward|:
%    \begin{macrocode}
\newcommand{\childdocforwardprefix}[3][]
{
  \begingroup
    \def\childdocextract #2##1~~~{\def\childdoctmp{\childdocforward[#1]{#3##1}}}
    \expandafter\childdocextract\childdocname~~~
    \expandafter
  \endgroup
  \childdoctmp
}
%    \end{macrocode}

% \macro{\childdoc}
% The deprecated macro |\childdoc| is a legacy version of |\childdocmain|:
%    \begin{macrocode}
\newcommand{\childdoc}{\childdocmain}
%    \end{macrocode}

% \macro{\childdocredirect}
% The deprecated macro |\childdocredirect| is a legacy version
% of |\childdocforward| and |\childdocforwardprefix|:
%    \begin{macrocode}
\newcommand{\childdocredirect}[2][]
{
  \begingroup
    \if?#1?
      \def\childdoctmp{\childdocforward{#2}}
    \else
      \def\childdoctmp{\childdocforwardprefix{#1}{#2}}
    \fi
    \expandafter
  \endgroup
  \childdoctmp
}
%    \end{macrocode}

%\iffalse
%</package>
%\fi
%
\endinput
|\\
|\childdocforwardprefix{final}{child}|
\end{tabular}
\end{center}
%

Note that when several versions of a main file and/or of each child file
are to be generated, it may be convenient to set up a |Makefile| or
shell script to automatise the process.

%%%%%%%%%%%%%%%%%%%%%%%%%%%%%%%%%%%%%%%%%%%%%%%%%%%%%%%%%%%%%%%%%%%%%%%%%%%%%%%%
\subsection{Command Line Processing}
\label{sec:commandline}

The effect of redirection files can also be achieved by invoking
the \LaTeX{} compiler with a more elaborate command line.
Most conveniently this should be done as part
of a shell script or a |Makefile|.

When using \textsf{childdoc} in the main file, the following
command lines effectively perform a redirection
(note that depending on the shell being used,
backslashes may have to be doubled: `|\|' $\to$ `|\\|'):
%
\begin{center}
|... -jobname "|\textit{target}|" |\\|"|[\textit{flags}]%
|% \iffalse
%
% childdoc.dtx Copyright (C) 2017-2018 Niklas Beisert
%
% This work may be distributed and/or modified under the
% conditions of the LaTeX Project Public License, either version 1.3
% of this license or (at your option) any later version.
% The latest version of this license is in
%   http://www.latex-project.org/lppl.txt
% and version 1.3 or later is part of all distributions of LaTeX
% version 2005/12/01 or later.
%
% This work has the LPPL maintenance status `maintained'.
%
% The Current Maintainer of this work is Niklas Beisert.
%
% This work consists of the files childdoc.dtx and childdoc.ins
% and the derived files childdoc.def and cdocsamp.tex with
% cdocsch1.tex, cdocsch2.tex, cdocsdrf.tex, cdocsfn1.tex, cdocsfn2.tex.
%
%<package>\ifdefined\childdocmain\endinput\fi
%<package>\ProvidesFile{childdoc.def}[2018/12/30 v2.0 child document driver]
%<samplemain>\ProvidesFile{cdocsamp.tex}[2018/12/30 v2.0 sample for childdoc]
%<*driver>
%\ProvidesFile{childdoc.drv}[2018/12/30 v2.0 childdoc reference manual file]
\PassOptionsToClass{10pt,a4paper}{article}
\documentclass{ltxdoc}

\usepackage[margin=35mm]{geometry}
\usepackage{hyperref}
\usepackage{hyperxmp}
\usepackage[usenames]{color}

\hypersetup{colorlinks=true}
\hypersetup{pdfstartview=FitH}
\hypersetup{pdfpagemode=UseNone}
\hypersetup{pdfsource={}}
\hypersetup{pdflang={en-UK}}
\hypersetup{pdfcopyright={Copyright 2017-2018 Niklas Beisert.
  This work may be distributed and/or modified under the
  conditions of the LaTeX Project Public License, either version 1.3
  of this license or (at your option) any later version.}}
\hypersetup{pdflicenseurl={http://www.latex-project.org/lppl.txt}}
\hypersetup{pdfcontactaddress={ETH Zurich, ITP, HIT K,
  Wolfgang-Pauli-Strasse 27}}
\hypersetup{pdfcontactpostcode={8093}}
\hypersetup{pdfcontactcity={Zurich}}
\hypersetup{pdfcontactcountry={Switzerland}}
\hypersetup{pdfcontactemail={nbeisert@itp.phys.ethz.ch}}
\hypersetup{pdfcontacturl={http://people.phys.ethz.ch/\xmptilde nbeisert/}}

\newcommand{\secref}[1]{\hyperref[#1]{section \ref*{#1}}}

\parskip1ex
\parindent0pt
\let\olditemize\itemize
\def\itemize{\olditemize\parskip0pt}

\begin{document}

\title{The \textsf{childdoc} Package}
\hypersetup{pdftitle={The childdoc Package}}
\author{Niklas Beisert\\[2ex]
  Institut f\"ur Theoretische Physik\\
  Eidgen\"ossische Technische Hochschule Z\"urich\\
  Wolfgang-Pauli-Strasse 27, 8093 Z\"urich, Switzerland\\[1ex]
  \href{mailto:nbeisert@itp.phys.ethz.ch}
  {\texttt{nbeisert@itp.phys.ethz.ch}}}
\hypersetup{pdfauthor={Niklas Beisert}}
\hypersetup{pdfsubject={Manual for the LaTeX2e Package childdoc}}
\date{30 December 2018, \textsf{v2.0}}
\maketitle

\begin{abstract}\noindent
\textsf{childdoc} is a \LaTeXe{} package
that enables the direct compilation
of document sections included by |\include|
to individual files.
\end{abstract}

\begingroup
\parskip0ex
\tableofcontents
\endgroup

%%%%%%%%%%%%%%%%%%%%%%%%%%%%%%%%%%%%%%%%%%%%%%%%%%%%%%%%%%%%%%%%%%%%%%%%%%%%%%%%
%%%%%%%%%%%%%%%%%%%%%%%%%%%%%%%%%%%%%%%%%%%%%%%%%%%%%%%%%%%%%%%%%%%%%%%%%%%%%%%%
\section{Introduction}

\LaTeX{} provides a mechanism to structure a large document (such as a book)
into a main file and several child files (containing the chapters)
using the |\include| command.
This mechanism is beneficial for documents
which span hundreds of pages in order to
make the source file(s) more manageable.
Moreover, compilation can be restricted to
selected child files by means of the |\includeonly| command.
The latter feature can be used to reduce the compilation time while editing
(this was significantly more useful in the earlier days of \LaTeX{})
or to generate a smaller document which is easier to navigate.
Another application of |\includeonly| is to generate
documents consisting of selected parts of the complete document.

However, there are a few drawbacks of the plain |\include| mechanism:
\begin{itemize}
\item
The child files cannot be compiled on their own,
they can only be compiled via the main file.
A naive editing environment
(such as a text editor with an option
to have the current file processed by \LaTeX)
may require one to switch to the main file before compiling;
attempting to compile the child file produces errors.
\item
The main file must be modified (each time)
to adjust the |\includeonly| command
to the present needs. This easily leaves the main file in a messy state.
\item
The generated document will always carry the filename
of the main document. This is inconvenient if
several child files are to be compiled and
to be kept for distribution.
\end{itemize}

The present package provides a simple interface
to make child files individually compilable by \LaTeX{}.
Compiling a child file then has the same effect as compiling
the main file with an |\includeonly| command
to select the appropriate child.
Moreover the generated document will carry the name of the child
rather than the main file.
This resolves all three above issues.

This feature is meant to make the editing of books,
thesis documents and lecture notes somewhat more convenient.
However, the package can also be used efficiently for
composing a series of documents (such as exercise sheets)
which are typically distributed individually.
It then assists the author in generating the individual documents
(potentially in different versions)
as well as a document containing the collected series.
Another application is in developing style files
or other kinds of included material
where compilation of the style file could redirect
to a sample or test file.

%%%%%%%%%%%%%%%%%%%%%%%%%%%%%%%%%%%%%%%%%%%%%%%%%%%%%%%%%%%%%%%%%%%%%%%%%%%%%%%%
%%%%%%%%%%%%%%%%%%%%%%%%%%%%%%%%%%%%%%%%%%%%%%%%%%%%%%%%%%%%%%%%%%%%%%%%%%%%%%%%
\section{Usage}

First of all, the package \textsf{childdoc} is \emph{not} a standard
\LaTeXe{} |.sty| style file! Therefore it needs to be invoked in
a non-standard way.

%%%%%%%%%%%%%%%%%%%%%%%%%%%%%%%%%%%%%%%%%%%%%%%%%%%%%%%%%%%%%%%%%%%%%%%%%%%%%%%%
\subsection{Included Files}
\label{sec:include}

%%%%%%%%%%%%%%%%%%%%%%%%%%%%%%%%%%%%%%%%
\DescribeMacro{\childdocmain}
To use the package, add the commands
\begin{center}
\begin{tabular}{l}
|% \iffalse
%
% childdoc.dtx Copyright (C) 2017-2018 Niklas Beisert
%
% This work may be distributed and/or modified under the
% conditions of the LaTeX Project Public License, either version 1.3
% of this license or (at your option) any later version.
% The latest version of this license is in
%   http://www.latex-project.org/lppl.txt
% and version 1.3 or later is part of all distributions of LaTeX
% version 2005/12/01 or later.
%
% This work has the LPPL maintenance status `maintained'.
%
% The Current Maintainer of this work is Niklas Beisert.
%
% This work consists of the files childdoc.dtx and childdoc.ins
% and the derived files childdoc.def and cdocsamp.tex with
% cdocsch1.tex, cdocsch2.tex, cdocsdrf.tex, cdocsfn1.tex, cdocsfn2.tex.
%
%<package>\ifdefined\childdocmain\endinput\fi
%<package>\ProvidesFile{childdoc.def}[2018/12/30 v2.0 child document driver]
%<samplemain>\ProvidesFile{cdocsamp.tex}[2018/12/30 v2.0 sample for childdoc]
%<*driver>
%\ProvidesFile{childdoc.drv}[2018/12/30 v2.0 childdoc reference manual file]
\PassOptionsToClass{10pt,a4paper}{article}
\documentclass{ltxdoc}

\usepackage[margin=35mm]{geometry}
\usepackage{hyperref}
\usepackage{hyperxmp}
\usepackage[usenames]{color}

\hypersetup{colorlinks=true}
\hypersetup{pdfstartview=FitH}
\hypersetup{pdfpagemode=UseNone}
\hypersetup{pdfsource={}}
\hypersetup{pdflang={en-UK}}
\hypersetup{pdfcopyright={Copyright 2017-2018 Niklas Beisert.
  This work may be distributed and/or modified under the
  conditions of the LaTeX Project Public License, either version 1.3
  of this license or (at your option) any later version.}}
\hypersetup{pdflicenseurl={http://www.latex-project.org/lppl.txt}}
\hypersetup{pdfcontactaddress={ETH Zurich, ITP, HIT K,
  Wolfgang-Pauli-Strasse 27}}
\hypersetup{pdfcontactpostcode={8093}}
\hypersetup{pdfcontactcity={Zurich}}
\hypersetup{pdfcontactcountry={Switzerland}}
\hypersetup{pdfcontactemail={nbeisert@itp.phys.ethz.ch}}
\hypersetup{pdfcontacturl={http://people.phys.ethz.ch/\xmptilde nbeisert/}}

\newcommand{\secref}[1]{\hyperref[#1]{section \ref*{#1}}}

\parskip1ex
\parindent0pt
\let\olditemize\itemize
\def\itemize{\olditemize\parskip0pt}

\begin{document}

\title{The \textsf{childdoc} Package}
\hypersetup{pdftitle={The childdoc Package}}
\author{Niklas Beisert\\[2ex]
  Institut f\"ur Theoretische Physik\\
  Eidgen\"ossische Technische Hochschule Z\"urich\\
  Wolfgang-Pauli-Strasse 27, 8093 Z\"urich, Switzerland\\[1ex]
  \href{mailto:nbeisert@itp.phys.ethz.ch}
  {\texttt{nbeisert@itp.phys.ethz.ch}}}
\hypersetup{pdfauthor={Niklas Beisert}}
\hypersetup{pdfsubject={Manual for the LaTeX2e Package childdoc}}
\date{30 December 2018, \textsf{v2.0}}
\maketitle

\begin{abstract}\noindent
\textsf{childdoc} is a \LaTeXe{} package
that enables the direct compilation
of document sections included by |\include|
to individual files.
\end{abstract}

\begingroup
\parskip0ex
\tableofcontents
\endgroup

%%%%%%%%%%%%%%%%%%%%%%%%%%%%%%%%%%%%%%%%%%%%%%%%%%%%%%%%%%%%%%%%%%%%%%%%%%%%%%%%
%%%%%%%%%%%%%%%%%%%%%%%%%%%%%%%%%%%%%%%%%%%%%%%%%%%%%%%%%%%%%%%%%%%%%%%%%%%%%%%%
\section{Introduction}

\LaTeX{} provides a mechanism to structure a large document (such as a book)
into a main file and several child files (containing the chapters)
using the |\include| command.
This mechanism is beneficial for documents
which span hundreds of pages in order to
make the source file(s) more manageable.
Moreover, compilation can be restricted to
selected child files by means of the |\includeonly| command.
The latter feature can be used to reduce the compilation time while editing
(this was significantly more useful in the earlier days of \LaTeX{})
or to generate a smaller document which is easier to navigate.
Another application of |\includeonly| is to generate
documents consisting of selected parts of the complete document.

However, there are a few drawbacks of the plain |\include| mechanism:
\begin{itemize}
\item
The child files cannot be compiled on their own,
they can only be compiled via the main file.
A naive editing environment
(such as a text editor with an option
to have the current file processed by \LaTeX)
may require one to switch to the main file before compiling;
attempting to compile the child file produces errors.
\item
The main file must be modified (each time)
to adjust the |\includeonly| command
to the present needs. This easily leaves the main file in a messy state.
\item
The generated document will always carry the filename
of the main document. This is inconvenient if
several child files are to be compiled and
to be kept for distribution.
\end{itemize}

The present package provides a simple interface
to make child files individually compilable by \LaTeX{}.
Compiling a child file then has the same effect as compiling
the main file with an |\includeonly| command
to select the appropriate child.
Moreover the generated document will carry the name of the child
rather than the main file.
This resolves all three above issues.

This feature is meant to make the editing of books,
thesis documents and lecture notes somewhat more convenient.
However, the package can also be used efficiently for
composing a series of documents (such as exercise sheets)
which are typically distributed individually.
It then assists the author in generating the individual documents
(potentially in different versions)
as well as a document containing the collected series.
Another application is in developing style files
or other kinds of included material
where compilation of the style file could redirect
to a sample or test file.

%%%%%%%%%%%%%%%%%%%%%%%%%%%%%%%%%%%%%%%%%%%%%%%%%%%%%%%%%%%%%%%%%%%%%%%%%%%%%%%%
%%%%%%%%%%%%%%%%%%%%%%%%%%%%%%%%%%%%%%%%%%%%%%%%%%%%%%%%%%%%%%%%%%%%%%%%%%%%%%%%
\section{Usage}

First of all, the package \textsf{childdoc} is \emph{not} a standard
\LaTeXe{} |.sty| style file! Therefore it needs to be invoked in
a non-standard way.

%%%%%%%%%%%%%%%%%%%%%%%%%%%%%%%%%%%%%%%%%%%%%%%%%%%%%%%%%%%%%%%%%%%%%%%%%%%%%%%%
\subsection{Included Files}
\label{sec:include}

%%%%%%%%%%%%%%%%%%%%%%%%%%%%%%%%%%%%%%%%
\DescribeMacro{\childdocmain}
To use the package, add the commands
\begin{center}
\begin{tabular}{l}
|\input{childdoc.def}|\\
|\childdocmain{}|\\
\end{tabular}
\end{center}
at the very top of the main \LaTeX{} file,
in particular \emph{before} the |\documentclass| statement!
The argument of |\childdocmain| should be left empty
(but it must be present).

%%%%%%%%%%%%%%%%%%%%%%%%%%%%%%%%%%%%%%%%
\DescribeMacro{\childdocof}
Furthermore, add the commands
\begin{center}
\begin{tabular}{l}
|\input{childdoc.def}|\\
|\childdocof{|\textit{main}|}|\\
\end{tabular}
\end{center}
at the top of every child file \textit{child}
which is included by |\include{|\textit{child}|}|
from within the main file
(or at least for those files to be compiled individually).
The argument \textit{main} must be the filename of the main file.

There are a couple of
considerations in setting up the main and child documents:

%%%%%%%%%%%%%%%%%%%%%%%%%%%%%%%%%%%%%%%%
\paragraph{Restrictions.}

Please note the following restrictions:
\begin{itemize}
\item
|\childdocmain| must be called with one argument \textit{main}
to ensure compatibility with earlier version of the package.
It must either be empty (|\childdocmain{}|)
or precisely match the filename of the main file in which it is specified.
See \secref{sec:detection} for further information.
\item
The filename \textit{main} must be specified without the |.tex| extension.
\item
The filename \textit{main} is case sensitive
(even in case-insensitive file systems)
due to internal string comparison.
\item
The argument \textit{main} should be fully expanded, it cannot be a macro.
\item
Subdirectories and special characters should be avoided in filenames.
\item
The command |\childdocmain{|\textit{main}|}| must be followed by a whitespace.
It should not be followed immediately by another command
or by a comment mark `|%|'.
This is because the \TeX{} parser reads the token immediately following
the argument of |\childdocmain| and puts it
at the beginning of every child section;
however, a white\-space is ignored.
\end{itemize}

%%%%%%%%%%%%%%%%%%%%%%%%%%%%%%%%%%%%%%%%
\paragraph{Content of Main File.}

It is advisable to place all content in the child files included by |\include|.
Any output contained in the main file will appear in all child documents
unless suppressed manually;
it cannot be suppressed automatically by the |\includeonly| directive
and thus should normally be avoided.
A method to include some content in the main file
by means of conditional processing is described in \secref{sec:conditional}.

%%%%%%%%%%%%%%%%%%%%%%%%%%%%%%%%%%%%%%%%
\paragraph{Page Numbering.}

When only a part of the document is compiled,
the appropriate numbering of pages
(as well as other status parameters)
is determined from the |.aux| files.
The latter contain information from previous passes.
However this information needs to propagate through
all intermediate child documents.
Therefore the page numbering in child documents may well
be inconsistent until the complete document is compiled at least once.

A useful (if unconventional) way to always ensure a consistent
page numbering is to restart the numbering in each child document
and denote the pages by `\textit{child}|.|\textit{page}'
where \textit{child} represents the chapter/section number of the child file.
This can be achieved by the command
|\numberwithin{page}{|\textit{child}|}|
of the \textsf{amsmath} package
where \textit{child} can be |chapter| or |section|
depending on the chosen structuring.
Alternatively, one can modify the macro |\thepage| appropriately
and reset the counter |page| at the start of each child file.

%%%%%%%%%%%%%%%%%%%%%%%%%%%%%%%%%%%%%%%%%%%%%%%%%%%%%%%%%%%%%%%%%%%%%%%%%%%%%%%%
\subsection{Conditional Processing}
\label{sec:conditional}

The package provides a mechanism to compile different versions
of a document. To customise the versions further some conditional processing
can come in handy to distinguish which version is being compiled.
The package provides two macros to describe the compilation context:

%%%%%%%%%%%%%%%%%%%%%%%%%%%%%%%%%%%%%%%%
\DescribeMacro{\ifchilddoc}
The conditional |\ifchilddoc| distinguishes between the compilation of
child documents and the main document:
%
\begin{center}
|\ifchilddoc |\textit{child-code}| |[|\||else |\textit{main-code}]| \||fi|
\end{center}

%%%%%%%%%%%%%%%%%%%%%%%%%%%%%%%%%%%%%%%%
\DescribeMacro{\childdocname}
\DescribeMacro{\childdocjob}
The macro |\childdocname| contains the filename (without extension)
of the main or child file being processed.
Note that |\childdocjob| will always contain the name of the main file.

%%%%%%%%%%%%%%%%%%%%%%%%%%%%%%%%%%%%%%%%
\paragraph{Title Page.}

Conditional processing can be used to include a title or banner page
in the main document when proper precautions are taken.
Importantly, the code in the main file should ensure that the page counter
(as well as other status parameters which are stored in the |.aux| files)
takes the same value after the conditional processing.
Otherwise the page numbers may take divergent values
depending on which part is compiled.

For example, a title page could be declared by:
%
\begin{center}
\begin{tabular}{l}
|\ifchilddoc\||else|\\
|\addtocounter{page}{-1}|\\
\textit{code for title page}\\
|\newpage|\\
|\||fi|
\end{tabular}
\end{center}
%
A banner page for the child documents can be generated by:
%
\begin{center}
\begin{tabular}{l}
|\ifchilddoc|\\
|\addtocounter{page}{-1}|\\
\textit{code for banner page}\\
|\newpage|\\
|\||fi|
\end{tabular}
\end{center}
%
Here one could write a message such as:
\begin{center}
|This is the part \childdocname{} of \childdocjob{}.|
\end{center}

%%%%%%%%%%%%%%%%%%%%%%%%%%%%%%%%%%%%%%%%%%%%%%%%%%%%%%%%%%%%%%%%%%%%%%%%%%%%%%%%
\subsection{Flags}
\label{sec:flags}

The package makes it easy to generate different versions
of the main or child documents.
To this end compilation flags can be defined
and assigned different default values.
They will be particularly useful in conjunction
with the forwarding mechanism described in \secref{sec:forward}.

For example, it may be useful to have a flag |\version|
which can be set to |draft| or |final|.
The document source will contain some conditional code
depending on the value of |\version|.
Suppose further, the flag should default to |final| for the main file
and to |draft| for child files
which is a natural assignment for editing the document.
This is achieved by placing the following code
in the preamble of the main document
(below the |\childdocmain| directive):
%
\begin{center}
\begin{tabular}{l}
|\ifchilddoc|\\
|\providecommand{\version}{draft}|\\
|\||else|\\
|\providecommand{\version}{final}|\\
|\||fi|
\end{tabular}
\end{center}
%
The definition by |\providecommand| makes sure
that previous definitions are not overwritten.
Further statements |\providecommand{\version}{...}|
can thus be added before the above code to override it.

For the main file, one might add a line
(between |\childdocmain| and the above block)
%
\begin{center}
|%\ifchilddoc\||else\providecommand{\version}{draft}\||fi|
\end{center}
%
which can be uncommented to produce a draft version.
Likewise one can add a line to the very top of a child file
(above the |\childdocof{|\textit{main}|}| directive)
%
\begin{center}
|%\providecommand{\version}{final}|
\end{center}
%
which can be uncommented to produce the final version of this child document.

%%%%%%%%%%%%%%%%%%%%%%%%%%%%%%%%%%%%%%%%%%%%%%%%%%%%%%%%%%%%%%%%%%%%%%%%%%%%%%%%
\subsection{Forwarding}
\label{sec:forward}

Different versions of the main or child documents
using compilation flags as described in \secref{sec:flags}
can be (permanently) stored in different files
for convenient compilation, viewing and distribution.
To this end, the package defines a command
to pass on compilation to a different file:

%%%%%%%%%%%%%%%%%%%%%%%%%%%%%%%%%%%%%%%%
\DescribeMacro{\childdocforward}
The command |\childdocforward| redirects processing to
another source file:
%
\begin{center}
\begin{tabular}{l}
|\input{childdoc.def}|\\
|\childdocforward[|\textit{main}|]{|\textit{dest}|}|\\
\end{tabular}
\end{center}
%
The argument \textit{dest} is the destination file
(without extension).
It should be the main file or one of the child files.
Note that further \textsf{childdoc} directives
such as |\childdocof| and |\childdocforward|
in the indicated file will be processed in this form.
The optional argument \textit{main}
passes on directly to the main file \textit{main}
while pretending to compile the child \textit{dest}.
This form behaves as if \textit{dest}
issues |\childdocof{|\textit{main}|}| right away,
and no further \textsf{childdoc} directives will be processed.

%%%%%%%%%%%%%%%%%%%%%%%%%%%%%%%%%%%%%%%%
\DescribeMacro{\...prefix}
In the alternative form |\childdocforwardprefix|,
%
\begin{center}
\begin{tabular}{l}
|\input{childdoc.def}|\\
|\childdocforwardprefix[|\textit{main}|]{|\textit{prefix}|}{|\textit{dest}|}|
\end{tabular}
\end{center}
%
the destination file is determined by a pattern
depending on the current file:
To make this work, the current file must be called
`{\textit{prefix}\hspace{0.2em}\textit{suffix}}'
with \textit{prefix} matching precisely the argument.
Processing is then passed on to the file
`{\textit{dest}\hspace{0.2em}\textit{suffix}}'.
Surely, the same effect is achieved by
directly specifying the
argument `{\textit{dest}\hspace{0.2em}\textit{suffix}}'
in the first form.
However, that requires to set up a different file
for each child. With the alternative form of the command
all these files can have exactly the same content
which simplifies setting them up and maintaining them.

For example, the following file |draft.tex|
with a compilation flag |\version| as described in \secref{sec:flags}
compiles the main document as a draft:
%
\begin{center}
\begin{tabular}{l}
|\def\version{draft}|\\
|\input{childdoc.def}|\\
|\childdocforward{|\textit{main}|}|
\end{tabular}
\end{center}
%
Likewise, the following files |final|\textit{nn}|.tex|
compile the final version of the child document
|child|\textit{nn}|.tex|:
%
\begin{center}
\begin{tabular}{l}
|\def\version{final}|\\
|\input{childdoc.def}|\\
|\childdocforwardprefix{final}{child}|
\end{tabular}
\end{center}
%

Note that when several versions of a main file and/or of each child file
are to be generated, it may be convenient to set up a |Makefile| or
shell script to automatise the process.

%%%%%%%%%%%%%%%%%%%%%%%%%%%%%%%%%%%%%%%%%%%%%%%%%%%%%%%%%%%%%%%%%%%%%%%%%%%%%%%%
\subsection{Command Line Processing}
\label{sec:commandline}

The effect of redirection files can also be achieved by invoking
the \LaTeX{} compiler with a more elaborate command line.
Most conveniently this should be done as part
of a shell script or a |Makefile|.

When using \textsf{childdoc} in the main file, the following
command lines effectively perform a redirection
(note that depending on the shell being used,
backslashes may have to be doubled: `|\|' $\to$ `|\\|'):
%
\begin{center}
|... -jobname "|\textit{target}|" |\\|"|[\textit{flags}]%
|\input{childdoc.def}\childdocforward[|\textit{main}|]{|\textit{dest}|}"|
\end{center}
%
Here \textit{target} is the name of the output file,
\textit{main} is the name of the main file
and \textit{dest} is the name of the main or child file to be processed
(all filenames without extensions).
The optional argument \textit{main} can be omitted
if \textit{main} matches \textit{dest}.
Optionally, compilation \textit{flags} can be defined via |\def| commands.
This command line makes the \TeX{} engine believe
it is compiling the file \textit{target}
whose content is specified as the latter parameter.
The provided code then forwards the processing to
\textit{main} or \textit{dest} as described in \secref{sec:forward}.

%%%%%%%%%%%%%%%%%%%%%%%%%%%%%%%%%%%%%%%%%%%%%%%%%%%%%%%%%%%%%%%%%%%%%%%%%%%%%%%%
\subsection{Include by Input}
\label{sec:input}

Including child documents by |\include| has some restrictions by design.
Most notably, the content of a child document always occupies
its own set of pages; pages cannot be shared between child documents.
Usually, this behaviour makes perfect sense
because each child document contain an essential part of the document.
However, in some situations it may be desirable to compose
a document from a collection of parts
without having mandatory page breaks between then.
For this case, the package
provides a mechanism to include parts
by |\input| which can also be processed individually.
However, by construction this mechanism
requires manual handling of the content to be output.

%%%%%%%%%%%%%%%%%%%%%%%%%%%%%%%%%%%%%%%%
\DescribeMacro{\ifchilddocmanual}
The main file should be prepared as usual, see \secref{sec:include}.
However, the document body must make a distinction
between processing of an individual part and of the main document, e.g.:
%
\begin{center}
\begin{tabular}{l}
|\ifchilddocmanual|\\
|\input{\childdocname}|\\
|\||else|\\
\textit{document body with }|\input{|\textit{part}|}|\\
|\||fi|
\end{tabular}
\end{center}
%
The conditional |\ifchilddocmanual| is true whenever
a part to be included by |\input| is being compiled,
and the name of the part is stored in |\childdocname|.

%%%%%%%%%%%%%%%%%%%%%%%%%%%%%%%%%%%%%%%%
\DescribeMacro{\childdocby}
Each part to be included by |\input| should start with:
%
\begin{center}
\begin{tabular}{l}
|\input{childdoc.def}|\\
|\childdocby{|\textit{main}|}|\\
\end{tabular}
\end{center}
%
The directive |\childdocby| is similar to |\childdocof|
described in \secref{sec:include},
but the subsequent selection of content must be done manually.
To that end, both |\ifchilddoc| and |\ifchilddocmanual|
will be true upon processing of a part,
and the name of the part is stored in |\childdocname|.
Note that |\jobname| will be set to the filename of the current part
so that each part receives an individual |.aux| file
that does not interfere with the |.aux| file(s) of the main document.
This behaviour can be altered by the alternative form
|\childdocby[*]{|\textit{main}|}| (with a non-empty optional argument)
which uses the |.aux| file of the main document
by setting |\jobname| to \textit{main}.

%%%%%%%%%%%%%%%%%%%%%%%%%%%%%%%%%%%%%%%%%%%%%%%%%%%%%%%%%%%%%%%%%%%%%%%%%%%%%%%%
\subsection{Driver Development}
\label{sec:driver}

The \textsf{childdoc} mechanism can also be use for the development
of definition files such as \LaTeX{} styles or classes.
This case differs from the above setup with multiple parts
included by |\include| in that no |\includeonly| should be invoked.
This can be achieved by starting the include file
(before |\ProvidesPackage|) with:
%
\begin{center}
\begin{tabular}{l}
|\input{childdoc.def}|\\
|\childdocforward{|\textit{main}|}|\\
\end{tabular}
\end{center}
%
or alternatively with:
%
\begin{center}
\begin{tabular}{l}
|\input{childdoc.def}|\\
|\childdocby{|\textit{main}|}|\\
\end{tabular}
\end{center}
%
Both forms have slightly different effects as described above.
The main file is prepared as usual, see \secref{sec:include}.

%%%%%%%%%%%%%%%%%%%%%%%%%%%%%%%%%%%%%%%%%%%%%%%%%%%%%%%%%%%%%%%%%%%%%%%%%%%%%%%%
\subsection{Legacy Detection}
\label{sec:detection}

The directive |\childdocmain| in the main file can detect
whether the complete document or merely a child is to be compiled
even without using the directive |\childdocof|.
This method is deprecated because it is less robust
and there is no compelling reason to use it;
it is merely provided for backward compatibility
and it may be removed in future versions.

If the detection mechanism is to be used,
it is mandatory to correctly specify
the filename of the main file as the argument of |\childdocmain|:
%
\begin{center}
\begin{tabular}{l}
|\input{childdoc.def}|\\
|\childdocmain{|\textit{main}|}|\\
\end{tabular}
\end{center}
%
If |\jobname| does not match the argument \textit{main} of |\childdocmain|,
it is assumed that |\jobname| points to the child file to be compiled.
When using |\childdocmain| with the main file specified as argument,
it suffices to start a child file
with just |\input{|\textit{main}|}|
without loading of the package and using |\childdocof|.
If instead all processing is done
with the appropriate \textsf{childdoc} directives,
the argument of \textit{main} of |\childdocmain| can be empty.

An alternative version of the command line processing described
in \secref{sec:commandline} using the detection mechanism reads:
%
\begin{center}
|... -jobname "|\textit{target}|" "|[\textit{flags}]%
[|\def\jobname{|\textit{dest}|}|]|\input{|\textit{main}|}"|
\end{center}

%%%%%%%%%%%%%%%%%%%%%%%%%%%%%%%%%%%%%%%%%%%%%%%%%%%%%%%%%%%%%%%%%%%%%%%%%%%%%%%%
\subsection{Manual Code}
\label{sec:manual}

In case one cannot be certain whether the definitions file |childdoc.def|
is installed on the target \TeX{} distribution
and one prefers not to ship it,
it is conceivable to paste a few relevant commands into the sources.

To that end, drop all statements |\input{childdoc.def}|
and perform the replacements as outlined below.
Instead of |\childdocmain{|\textit{main}|}| add the following code
to the top of the main file:
%
\begin{center}
\begin{tabular}{l}
|\||ifdefined\childdocname\endinput\||fi\newif\ifchilddoc|\\
|\edef\childdocname{\scantokens\expandafter{\jobname\noexpand}}|\\
|\def\childdocmain{|\textit{main}|}\||ifx\childdocmain\childdocname\||else|\\
|\childdoctrue\includeonly{\childdocname}\let\jobname\childdocmain\||fi|\\
\end{tabular}
\end{center}
%
Instead of |\childdocof{|\textit{main}|}| just include the main file
at the top of each child file:
%
\begin{center}
|\input{|\textit{main}|}|
\end{center}
%
A simple redirection |\childdocforward{|\textit{dest}|}| is achieved by:
%
\begin{center}
|\def\jobname{|\textit{dest}|}\input{\jobname}|
\end{center}
%
The redirection with prefix
|\childdocforwardprefix[|\textit{prefix}|]{|\textit{dest}|}|
is accomplished by:
%
\begin{center}
\begin{tabular}{l}
|{\edef\jobname{\scantokens\expandafter{\jobname\noexpand}}|\\
|\def\redirectjob |\textit{prefix}|#1~~~{\gdef\jobname{|\textit{dest}|#1}}|\\
|\expandafter\redirectjob\jobname~~~}\input{\jobname}|
\end{tabular}
\end{center}

In an alternative approach,
child documents can be compiled by a specific command line
without additional code or specific definitions:
%
\begin{center}
|... -jobname "|\textit{target}|" "|[\textit{flags}]%
|\includeonly{|\textit{dest}|}\input{|\textit{main}|}"|
\end{center}
%

%%%%%%%%%%%%%%%%%%%%%%%%%%%%%%%%%%%%%%%%%%%%%%%%%%%%%%%%%%%%%%%%%%%%%%%%%%%%%%%%
%%%%%%%%%%%%%%%%%%%%%%%%%%%%%%%%%%%%%%%%%%%%%%%%%%%%%%%%%%%%%%%%%%%%%%%%%%%%%%%%
\section{Information}

%%%%%%%%%%%%%%%%%%%%%%%%%%%%%%%%%%%%%%%%%%%%%%%%%%%%%%%%%%%%%%%%%%%%%%%%%%%%%%%%
\subsection{Copyright}

Copyright \copyright{} 2017--2018 Niklas Beisert

This work may be distributed and/or modified under the
conditions of the \LaTeX{} Project Public License, either version 1.3
of this license or (at your option) any later version.
The latest version of this license is in
  \url{http://www.latex-project.org/lppl.txt}
and version 1.3 or later is part of all distributions of \LaTeX{}
version 2005/12/01 or later.

This work has the LPPL maintenance status `maintained'.

The Current Maintainer of this work is Niklas Beisert.

This work consists of the files |README.txt|, |childdoc.ins| and |childdoc.dtx|
as well as the derived files |childdoc.def|, |cdocsamp.tex|
with |cdocsch1.tex|, |cdocsch2.tex|, |cdocspt3.tex|, |cdocspt4.tex|,
|cdocsdrf.tex|, |cdocsfn1.tex|, |cdocsfn2.tex|
as well as |childdoc.pdf|.

%%%%%%%%%%%%%%%%%%%%%%%%%%%%%%%%%%%%%%%%%%%%%%%%%%%%%%%%%%%%%%%%%%%%%%%%%%%%%%%%
\subsection{Files and Installation}

The package consists of the files:
%
\begin{center}
\begin{tabular}{ll}
    |README.txt|   & readme file \\
    |childdoc.ins| & installation file \\
    |childdoc.dtx| & source file \\
    |childdoc.def| & definition file \\
    |cdocsamp.tex| & sample main file \\
    |cdocsch1.tex| & sample include file \\
    |cdocsch2.tex| & sample include file \\
    |cdocspt3.tex| & sample part file \\
    |cdocspt4.tex| & sample part file \\
    |cdocsdrf.tex| & sample redirection file \\
    |cdocsfn1.tex| & sample redirection file \\
    |cdocsfn2.tex| & sample redirection file \\
    |childdoc.pdf| & manual
\end{tabular}
\end{center}
%
The distribution consists of the files
|README.txt|, |childdoc.ins| and |childdoc.dtx|.
%
\begin{itemize}
\item
Run (pdf)\LaTeX{} on |childdoc.dtx|
to compile the manual |childdoc.pdf| (this file).
\item
Run \LaTeX{} on |childdoc.ins| to create the definitions file |childdoc.def|
and the sample |cdocsamp.tex| with include files
|cdocsch1.tex|, |cdocsch2.tex|, |cdocspt3.tex|, |cdocspt4.tex|,
|cdocsdrf.tex|, |cdocsfn1.tex|, |cdocsfn2.tex|.
Then copy the file |childdoc.def| to an appropriate directory of your \LaTeX{}
distribution, e.g.\ \textit{texmf-root}|/tex/latex/childdoc|.
\end{itemize}

%%%%%%%%%%%%%%%%%%%%%%%%%%%%%%%%%%%%%%%%%%%%%%%%%%%%%%%%%%%%%%%%%%%%%%%%%%%%%%%%
\subsection{Related CTAN Packages}

There are several other packages which offer a similar functionality:
%
\begin{itemize}
\item
The packages
\href{http://ctan.org/pkg/docmute}{\textsf{docmute}},
\href{http://ctan.org/pkg/includex}{\textsf{includex}} and
\href{http://ctan.org/pkg/standalone}{\textsf{standalone}}
provide commands to include only the document body of
a child file thus allowing both files to be compiled individually.
\item
The packages \href{http://ctan.org/pkg/subdocs}{\textsf{subdocs}}
and \href{http://ctan.org/pkg/subfiles}{\textsf{subfiles}}
provide structures in which the main and child documents can be
encapsulated and allowing them to be compiled individually.
The inclusion mechanism is different from the conventional |\include|.
\item
The package \href{http://ctan.org/pkg/combine}{\textsf{combine}}
is an elaborate solution to combine several documents into one.
\end{itemize}
%
See also the CTAN topic \href{http://ctan.org/topic/subdocs}{\textsf{subdocs}}
for further related packages.
The present package differs from the above solutions in that
a document structure constructed with the conventional |\include| mechanism
just needs two extra commands at the top of every file
such that all constituent files can be compiled individually.

%%%%%%%%%%%%%%%%%%%%%%%%%%%%%%%%%%%%%%%%%%%%%%%%%%%%%%%%%%%%%%%%%%%%%%%%%%%%%%%%
%\subsection{Feature Suggestions}
%
%The following is a list of features which may be useful for future
%versions of this package:
%%
%\begin{itemize}
%\item
%\ldots
%\end{itemize}

%%%%%%%%%%%%%%%%%%%%%%%%%%%%%%%%%%%%%%%%%%%%%%%%%%%%%%%%%%%%%%%%%%%%%%%%%%%%%%%%
\subsection{Revision History}

%%%%%%%%%%%%%%%%%%%%%%%%%%%%%%%%%%%%%%%%
\paragraph{v2.0:} 2018/12/30

\begin{itemize}
\item
immediate forward processing
\item
added |\childdocby| mechanism
\item
manual restructured
\end{itemize}

%%%%%%%%%%%%%%%%%%%%%%%%%%%%%%%%%%%%%%%%
\paragraph{v1.6:} 2018/01/17

\begin{itemize}
\item
application for development of include files
\item
corrections to manual
\end{itemize}

%%%%%%%%%%%%%%%%%%%%%%%%%%%%%%%%%%%%%%%%
\paragraph{v1.5:} 2017/05/21

\begin{itemize}
\item
more complete structuring introduced
\item
|\childdocof| introduced
\item
|\childdoc| renamed to |\childdocmain|
\item
|\childredirect| renamed to |\childdocforward| and |\childdocforwardprefix|
and functionality expanded
\end{itemize}

%%%%%%%%%%%%%%%%%%%%%%%%%%%%%%%%%%%%%%%%
\paragraph{v1.0:} 2017/04/27

\begin{itemize}
\item
manual and install package
\item
first version published on CTAN
\end{itemize}

%%%%%%%%%%%%%%%%%%%%%%%%%%%%%%%%%%%%%%%%
\paragraph{v0.6:} 2017/04/26

\begin{itemize}
\item
redirection mechanism added
\end{itemize}

%%%%%%%%%%%%%%%%%%%%%%%%%%%%%%%%%%%%%%%%
\paragraph{v0.5:} 2017/04/26

\begin{itemize}
\item
functionality in definition file
\end{itemize}


%%%%%%%%%%%%%%%%%%%%%%%%%%%%%%%%%%%%%%%%%%%%%%%%%%%%%%%%%%%%%%%%%%%%%%%%%%%%%%%%
%%%%%%%%%%%%%%%%%%%%%%%%%%%%%%%%%%%%%%%%%%%%%%%%%%%%%%%%%%%%%%%%%%%%%%%%%%%%%%%%
%%%%%%%%%%%%%%%%%%%%%%%%%%%%%%%%%%%%%%%%%%%%%%%%%%%%%%%%%%%%%%%%%%%%%%%%%%%%%%%%
\appendix

\settowidth\MacroIndent{\rmfamily\scriptsize 000\ }

 \DocInput{childdoc.dtx}

\end{document}
%</driver>
% \fi
%
% %%%%%%%%%%%%%%%%%%%%%%%%%%%%%%%%%%%%%%%%%%%%%%%%%%%%%%%%%%%%%%%%%%%%%%%%%%%%%%
% %%%%%%%%%%%%%%%%%%%%%%%%%%%%%%%%%%%%%%%%%%%%%%%%%%%%%%%%%%%%%%%%%%%%%%%%%%%%%%
% \section{Sample}
%\iffalse
%<*samplemain>
%\fi
%
% The following presents a sample document
% with two chapters, two parts, a title page,
% a compile flag as well as three forwarding files to set the flag.
% It consists of eight |.tex| files:
% \begin{center}
% \begin{tabular}{ll}
% |cdocsamp.tex|&main file\\
% |cdocsch1.tex|&include file for chapter 1\\
% |cdocsch2.tex|&include file for chapter 2\\
% |cdocspt3.tex|&include file for part 3\\
% |cdocspt4.tex|&include file for part 4\\
% |cdocsdrf.tex|&forwarding file for main file in draft mode\\
% |cdocsfi1.tex|&forwarding file for final version of chapter 1\\
% |cdocsfi2.tex|&forwarding file for final version of chapter 2\\
% \end{tabular}
% \end{center}
% Each of the eight files can be compiled directly by the \LaTeX{} compiler.
%
% %%%%%%%%%%%%%%%%%%%%%%%%%%%%%%%%%%%%%%
% \paragraph{Main File.}
%
% The main file is called |cdocsamp.tex|.
%
% Load the \textsf{childdoc} definitions and
% declare the filename for the main document:
%    \begin{macrocode}
\input{childdoc.def}
\childdocmain{}
%    \end{macrocode}

% Optional override for |\version| flag:
%    \begin{macrocode}
%%\ifchilddoc\else\providecommand{\version}{draft}\fi
%    \end{macrocode}

% Define the default values for the |\version| flag
% (|final| for the main file and |draft| for childs):
%    \begin{macrocode}
\ifchilddoc
\providecommand{\version}{draft}
\else
\providecommand{\version}{final}
\fi
%    \end{macrocode}

% Load the standard document class:
%    \begin{macrocode}
\documentclass[12pt]{article}
%    \end{macrocode}

% Start the document body:
%    \begin{macrocode}
\begin{document}
%    \end{macrocode}

% Declare a title page.
% Print title, part of document being processed and version flag:
%    \begin{macrocode}
\addtocounter{page}{-1}
\begin{center}
{\LARGE\bfseries{}childdoc example\par}
\vspace{1cm}
\ifchilddoc
\ifchilddocmanual part\else chapter\fi:
`\childdocname' of `\childdocjob'\par
\else
main document: `\childdocjob'\par
\fi
version: \version\par
\end{center}
\newpage
%    \end{macrocode}

% Manually include selected file,
% otherwise process as usual:
%    \begin{macrocode}
\ifchilddocmanual
\section*{part `\childdocname'}
\input{\childdocname}
\else
%    \end{macrocode}

% Include the two chapters:
%    \begin{macrocode}
\include{cdocsch1}
\include{cdocsch2}
%    \end{macrocode}

% Include the two parts unless only chapters should be displayed:
%    \begin{macrocode}
\ifchilddoc\else
\section{part three}
\input{cdocspt3}
\section{part four}
\input{cdocspt4}
\fi
%    \end{macrocode}

% Process as usual until here:
%    \begin{macrocode}
\fi
%    \end{macrocode}

% End of document body:
%    \begin{macrocode}
\end{document}
%    \end{macrocode}
%\iffalse
%</samplemain>
%\fi
%
% %%%%%%%%%%%%%%%%%%%%%%%%%%%%%%%%%%%%%%
% \paragraph{Chapter Include Files.}
%
% The include files are called |cdocsch1.tex| and |cdocsch2.tex|.
%
%\iffalse
%<*samplechap1|samplechap2>
%\fi

% Optional override for |\version| flag:
%    \begin{macrocode}
%%\providecommand{\version}{final}
%    \end{macrocode}

% Include the main document:
%    \begin{macrocode}
\input{childdoc.def}
\childdocof{cdocsamp}
%    \end{macrocode}

%\iffalse
%</samplechap1|samplechap2>
%\fi
%
%\iffalse
%<*samplechap1>
%\fi
% Some text for chapter 1:
%    \begin{macrocode}
\section{one}
some text in chapter one
%    \end{macrocode}

%\iffalse
%</samplechap1>
%\fi
% Some text for chapter 2:
%\iffalse
%<*samplechap2>
%\fi
%    \begin{macrocode}
\section{two}
more text in chapter two
%    \end{macrocode}

%\iffalse
%</samplechap2>
%\fi
%
% %%%%%%%%%%%%%%%%%%%%%%%%%%%%%%%%%%%%%%
% \paragraph{Part Include Files.}
%
% The include files are called |cdocspt3.tex| and |cdocspt4.tex|.
%
%\iffalse
%<*samplepart3|samplepart4>
%\fi

% Optional override for |\version| flag:
%    \begin{macrocode}
%%\providecommand{\version}{final}
%    \end{macrocode}

% Include the main document:
%    \begin{macrocode}
\input{childdoc.def}
\childdocby{cdocsamp}
%    \end{macrocode}

%\iffalse
%</samplepart3|samplepart4>
%\fi
%
%\iffalse
%<*samplepart3>
%\fi
% Some text for part 3:
%    \begin{macrocode}
some text in part three
%    \end{macrocode}

%\iffalse
%</samplepart3>
%\fi
% Some text for part 4:
%\iffalse
%<*samplepart4>
%\fi
%    \begin{macrocode}
more text in part four
%    \end{macrocode}

%\iffalse
%</samplepart4>
%\fi
%
% %%%%%%%%%%%%%%%%%%%%%%%%%%%%%%%%%%%%%%
% \paragraph{Forwarding for a Complete Draft.}
%
% The following forwarding file |cdocsdrf.tex|
% compiles the main document in draft mode:
%\iffalse
%<*sampledraft>
%\fi
%    \begin{macrocode}
\def\version{draft}
\input{childdoc.def}
\childdocforward{cdocsamp}
%    \end{macrocode}

%\iffalse
%</sampledraft>
%\fi
%
% %%%%%%%%%%%%%%%%%%%%%%%%%%%%%%%%%%%%%%
% \paragraph{Forwarding for Final Version of the Chapters.}
%
% The following forwarding files |cdocsfn1.tex| and |cdocsfn2.tex|
% (with identical content)
% compile the final versions of the child documents
% |cdocsch1.tex| and |cdocsch2.tex|, respectively:
%\iffalse
%<*samplefinal>
%\fi
%    \begin{macrocode}
\def\version{final}
\input{childdoc.def}
\childdocforwardprefix[cdocsamp]{cdocsfn}{cdocsch}
%    \end{macrocode}

%\iffalse
%</samplefinal>
%\fi
%
% %%%%%%%%%%%%%%%%%%%%%%%%%%%%%%%%%%%%%%
% \paragraph{Command Line Processing.}
%
% The following three command lines generate the output files
% |cdocscld|, |cdocscl1| and |cdocscl2|
% which should be identical to
% |cdocsdrf|, |cdocsch1| and |cdocsfn2|, respectively:
% \begin{center}
% \begin{tabular}{l}
% |latex -jobname cdocscld \|\\
% |  "\def\version{draft}\input{childdoc.def}\childdocforward{cdocsamp}"|\\
% |latex -jobname cdocscl1 \|\\
% |  "\input{childdoc.def}\childdocforward[cdocsamp]{cdocsch1}"|\\
% |latex -jobname cdocscl2 \|\\
% |  "\def\version{final}\input{childdoc.def}\childdocforward{cdocsch2}"|
% \end{tabular}
% \end{center}
% Note that the trailing backslash on each first line
% merely continues the input to the second line
% (for convenient cut ant paste).
% Furthermore, the command |latex| can be replaced by any
% of its alternative versions such as |pdflatex|.
%
% %%%%%%%%%%%%%%%%%%%%%%%%%%%%%%%%%%%%%%%%%%%%%%%%%%%%%%%%%%%%%%%%%%%%%%%%%%%%%%
% %%%%%%%%%%%%%%%%%%%%%%%%%%%%%%%%%%%%%%%%%%%%%%%%%%%%%%%%%%%%%%%%%%%%%%%%%%%%%%
% \section{Implementation}
%\iffalse
%<*package>
%\fi
%
% This section describes the definitions file |childdoc.def|.

% The definitions cannot be loaded using |\usepackage| or |\RequirePackage|
% which has a mechanism to prevent loading a style file more than once.
% When loading the definitions by means of |\input|
% multiple instances have to be prevented manually:
%\iffalse
%This code needs to be before the `\ProvidesFile' directive
%which is defined at the beginning of this file.
%Therefore it is also placed there and commented out here.
%</package>
%<*discard>
%\fi
%    \begin{macrocode}
\ifdefined\childdocmain\endinput\fi
%    \end{macrocode}
%\iffalse
%</discard>
%<*package>
%\fi
%
% \macro{\ifchilddoc}
% \macro{\ifchilddocmanual}
% The conditional |\ifchilddoc| tells whether a
% child (true) or main (false) document is being compiled.
% The conditional |\ifchilddocmanual| tells whether
% the |\includeonly| mechanism is used (false) or
% the selection of child files must be performed manually (true).
% The definitions initialise to false:
%    \begin{macrocode}
\newif\ifchilddoc
\newif\ifchilddocmanual
%    \end{macrocode}

% \macro{\childdocname}
% \macro{\childdocjob}
% The macro |\childdocname| stores the name of the main document
% to be compiled. The macro |\childdocjob| stores the name of
% the document on which the \LaTeX{} compiler was originally invoked.
% The content of |\jobname| cannot be compared
% to filenames specified in the source due to different catcodes.
% The following code rescans |\jobname|, stores the result
% in |\childdocname| and saves a copy in |\childdocjob|:
%    \begin{macrocode}
\edef\childdocname{\scantokens\expandafter{\jobname\noexpand}}
\let\childdocjob\childdocname
%    \end{macrocode}

% \macro{\childdocdisable}
% The macro |\childdocdisable| prevents the main file
% from being processed more than once.
% At this stage, the main document command |\childdocmain|
% is assumed to be called once again where it should do nothing.
% Any subsequent call to it should prevent
% a secondary processing of the main document
% It overwrites the forwarding commands
% |\childdocof| and |\childdocforward|
% with empty macros to prevent further inclusions of the main document:
%    \begin{macrocode}
\newcommand{\childdocdisable}
{
  \renewcommand{\childdocmain}[1]{\renewcommand{\childdocmain}[1]{\endinput}}
  \renewcommand{\childdocof}[1]{}
  \renewcommand{\childdocby}[2][]{}
  \renewcommand{\childdocforward}[2][]{}
  \renewcommand{\childdocdisable}{}
}
%    \end{macrocode}

% \macro{\childdocmain}
% The macro |\childdocmain| is to be called at the top of the main file
% with nothing or the main filename (without extension) as argument.
% First, it breaks loops.
% If the argument is not empty and does not match |\childdocname|
% (which is set by the first inclusion of |childdoc.def|),
% |\ifchilddoc| is set to true, |\includeonly| is applied to the child file
% and |\jobname| is set to the main file
% (for proper handling of |.aux| files):
%    \begin{macrocode}
\newcommand{\childdocmain}[1]
{
  \childdocdisable\childdocmain{}
  \if?#1?\else
    \begingroup
      \def\childdoctmp{#1}
      \ifx\childdoctmp\childdocname
        \def\childdoctmp{}
      \else
        \def\childdoctmp
        {
          \childdoctrue
          \includeonly{\childdocname}
          \def\childdocjob{#1}
          \def\jobname{#1}
        }
      \fi
      \expandafter
    \endgroup
    \childdoctmp
  \fi
}
%    \end{macrocode}

% \macro{\childdocof}
% The command |\childdocof| redirects
% compilation to the main file |#1|.
%    \begin{macrocode}
\newcommand{\childdocof}[1]
{
  \childdocdisable
  \childdoctrue
  \includeonly{\childdocname}
  \def\jobname{#1}
  \def\childdocjob{#1}
  \input{#1}
}
%    \end{macrocode}

% \macro{\childdocby}
% The command |\childdocby| ....
%    \begin{macrocode}
\newcommand{\childdocby}[2][]
{
  \childdocdisable
  \childdoctrue
  \childdocmanualtrue
  \if?#1?\else
    \def\jobname{#2}
  \fi
  \def\childdocjob{#2}
  \input{#2}
  \endinput
}
%    \end{macrocode}

% \macro{\childdocforward}
% The command |\childdocforward| redirects
% compilation to the main file or
% (if the optional argument is given) a child file.
% Parameters are set as if the main file
% or a child file starting with |\childdocof| was compiled.
% Then compilation is handed over to the main file:
%    \begin{macrocode}
\newcommand{\childdocforward}[2][]
{
  \begingroup
    \if?#1?
      \def\childdoctmp
      {
        \def\childdocname{#2}
        \def\childdocjob{#2}
        \def\jobname{#2}
        \input{#2}
        \endinput
      }
    \else
      \def\childdoctmp
      {
        \childdocdisable
        \def\childdocname{#2}
        \childdoctrue
        \includeonly{#2}
        \def\childdocjob{#1}
        \def\jobname{#1}
        \input{#1}
        \endinput
      }
    \fi
    \expandafter
  \endgroup
  \childdoctmp
}
%    \end{macrocode}

% \macro{\childdocforwardprefix}
% The command |\childdocforwardprefix| redirects
% compilation to the main or a child file by means of a pattern.
% The prefix |#1| in the current filename is replaced by |#2|
% and the suffix of the current filename is kept
% (it is assumed that the filename does not contain the substring `|~~~|'
% which is used as a delimiter).
% Compilation is handed over to the new file by |\childdocforward|:
%    \begin{macrocode}
\newcommand{\childdocforwardprefix}[3][]
{
  \begingroup
    \def\childdocextract #2##1~~~{\def\childdoctmp{\childdocforward[#1]{#3##1}}}
    \expandafter\childdocextract\childdocname~~~
    \expandafter
  \endgroup
  \childdoctmp
}
%    \end{macrocode}

% \macro{\childdoc}
% The deprecated macro |\childdoc| is a legacy version of |\childdocmain|:
%    \begin{macrocode}
\newcommand{\childdoc}{\childdocmain}
%    \end{macrocode}

% \macro{\childdocredirect}
% The deprecated macro |\childdocredirect| is a legacy version
% of |\childdocforward| and |\childdocforwardprefix|:
%    \begin{macrocode}
\newcommand{\childdocredirect}[2][]
{
  \begingroup
    \if?#1?
      \def\childdoctmp{\childdocforward{#2}}
    \else
      \def\childdoctmp{\childdocforwardprefix{#1}{#2}}
    \fi
    \expandafter
  \endgroup
  \childdoctmp
}
%    \end{macrocode}

%\iffalse
%</package>
%\fi
%
\endinput
|\\
|\childdocmain{}|\\
\end{tabular}
\end{center}
at the very top of the main \LaTeX{} file,
in particular \emph{before} the |\documentclass| statement!
The argument of |\childdocmain| should be left empty
(but it must be present).

%%%%%%%%%%%%%%%%%%%%%%%%%%%%%%%%%%%%%%%%
\DescribeMacro{\childdocof}
Furthermore, add the commands
\begin{center}
\begin{tabular}{l}
|% \iffalse
%
% childdoc.dtx Copyright (C) 2017-2018 Niklas Beisert
%
% This work may be distributed and/or modified under the
% conditions of the LaTeX Project Public License, either version 1.3
% of this license or (at your option) any later version.
% The latest version of this license is in
%   http://www.latex-project.org/lppl.txt
% and version 1.3 or later is part of all distributions of LaTeX
% version 2005/12/01 or later.
%
% This work has the LPPL maintenance status `maintained'.
%
% The Current Maintainer of this work is Niklas Beisert.
%
% This work consists of the files childdoc.dtx and childdoc.ins
% and the derived files childdoc.def and cdocsamp.tex with
% cdocsch1.tex, cdocsch2.tex, cdocsdrf.tex, cdocsfn1.tex, cdocsfn2.tex.
%
%<package>\ifdefined\childdocmain\endinput\fi
%<package>\ProvidesFile{childdoc.def}[2018/12/30 v2.0 child document driver]
%<samplemain>\ProvidesFile{cdocsamp.tex}[2018/12/30 v2.0 sample for childdoc]
%<*driver>
%\ProvidesFile{childdoc.drv}[2018/12/30 v2.0 childdoc reference manual file]
\PassOptionsToClass{10pt,a4paper}{article}
\documentclass{ltxdoc}

\usepackage[margin=35mm]{geometry}
\usepackage{hyperref}
\usepackage{hyperxmp}
\usepackage[usenames]{color}

\hypersetup{colorlinks=true}
\hypersetup{pdfstartview=FitH}
\hypersetup{pdfpagemode=UseNone}
\hypersetup{pdfsource={}}
\hypersetup{pdflang={en-UK}}
\hypersetup{pdfcopyright={Copyright 2017-2018 Niklas Beisert.
  This work may be distributed and/or modified under the
  conditions of the LaTeX Project Public License, either version 1.3
  of this license or (at your option) any later version.}}
\hypersetup{pdflicenseurl={http://www.latex-project.org/lppl.txt}}
\hypersetup{pdfcontactaddress={ETH Zurich, ITP, HIT K,
  Wolfgang-Pauli-Strasse 27}}
\hypersetup{pdfcontactpostcode={8093}}
\hypersetup{pdfcontactcity={Zurich}}
\hypersetup{pdfcontactcountry={Switzerland}}
\hypersetup{pdfcontactemail={nbeisert@itp.phys.ethz.ch}}
\hypersetup{pdfcontacturl={http://people.phys.ethz.ch/\xmptilde nbeisert/}}

\newcommand{\secref}[1]{\hyperref[#1]{section \ref*{#1}}}

\parskip1ex
\parindent0pt
\let\olditemize\itemize
\def\itemize{\olditemize\parskip0pt}

\begin{document}

\title{The \textsf{childdoc} Package}
\hypersetup{pdftitle={The childdoc Package}}
\author{Niklas Beisert\\[2ex]
  Institut f\"ur Theoretische Physik\\
  Eidgen\"ossische Technische Hochschule Z\"urich\\
  Wolfgang-Pauli-Strasse 27, 8093 Z\"urich, Switzerland\\[1ex]
  \href{mailto:nbeisert@itp.phys.ethz.ch}
  {\texttt{nbeisert@itp.phys.ethz.ch}}}
\hypersetup{pdfauthor={Niklas Beisert}}
\hypersetup{pdfsubject={Manual for the LaTeX2e Package childdoc}}
\date{30 December 2018, \textsf{v2.0}}
\maketitle

\begin{abstract}\noindent
\textsf{childdoc} is a \LaTeXe{} package
that enables the direct compilation
of document sections included by |\include|
to individual files.
\end{abstract}

\begingroup
\parskip0ex
\tableofcontents
\endgroup

%%%%%%%%%%%%%%%%%%%%%%%%%%%%%%%%%%%%%%%%%%%%%%%%%%%%%%%%%%%%%%%%%%%%%%%%%%%%%%%%
%%%%%%%%%%%%%%%%%%%%%%%%%%%%%%%%%%%%%%%%%%%%%%%%%%%%%%%%%%%%%%%%%%%%%%%%%%%%%%%%
\section{Introduction}

\LaTeX{} provides a mechanism to structure a large document (such as a book)
into a main file and several child files (containing the chapters)
using the |\include| command.
This mechanism is beneficial for documents
which span hundreds of pages in order to
make the source file(s) more manageable.
Moreover, compilation can be restricted to
selected child files by means of the |\includeonly| command.
The latter feature can be used to reduce the compilation time while editing
(this was significantly more useful in the earlier days of \LaTeX{})
or to generate a smaller document which is easier to navigate.
Another application of |\includeonly| is to generate
documents consisting of selected parts of the complete document.

However, there are a few drawbacks of the plain |\include| mechanism:
\begin{itemize}
\item
The child files cannot be compiled on their own,
they can only be compiled via the main file.
A naive editing environment
(such as a text editor with an option
to have the current file processed by \LaTeX)
may require one to switch to the main file before compiling;
attempting to compile the child file produces errors.
\item
The main file must be modified (each time)
to adjust the |\includeonly| command
to the present needs. This easily leaves the main file in a messy state.
\item
The generated document will always carry the filename
of the main document. This is inconvenient if
several child files are to be compiled and
to be kept for distribution.
\end{itemize}

The present package provides a simple interface
to make child files individually compilable by \LaTeX{}.
Compiling a child file then has the same effect as compiling
the main file with an |\includeonly| command
to select the appropriate child.
Moreover the generated document will carry the name of the child
rather than the main file.
This resolves all three above issues.

This feature is meant to make the editing of books,
thesis documents and lecture notes somewhat more convenient.
However, the package can also be used efficiently for
composing a series of documents (such as exercise sheets)
which are typically distributed individually.
It then assists the author in generating the individual documents
(potentially in different versions)
as well as a document containing the collected series.
Another application is in developing style files
or other kinds of included material
where compilation of the style file could redirect
to a sample or test file.

%%%%%%%%%%%%%%%%%%%%%%%%%%%%%%%%%%%%%%%%%%%%%%%%%%%%%%%%%%%%%%%%%%%%%%%%%%%%%%%%
%%%%%%%%%%%%%%%%%%%%%%%%%%%%%%%%%%%%%%%%%%%%%%%%%%%%%%%%%%%%%%%%%%%%%%%%%%%%%%%%
\section{Usage}

First of all, the package \textsf{childdoc} is \emph{not} a standard
\LaTeXe{} |.sty| style file! Therefore it needs to be invoked in
a non-standard way.

%%%%%%%%%%%%%%%%%%%%%%%%%%%%%%%%%%%%%%%%%%%%%%%%%%%%%%%%%%%%%%%%%%%%%%%%%%%%%%%%
\subsection{Included Files}
\label{sec:include}

%%%%%%%%%%%%%%%%%%%%%%%%%%%%%%%%%%%%%%%%
\DescribeMacro{\childdocmain}
To use the package, add the commands
\begin{center}
\begin{tabular}{l}
|\input{childdoc.def}|\\
|\childdocmain{}|\\
\end{tabular}
\end{center}
at the very top of the main \LaTeX{} file,
in particular \emph{before} the |\documentclass| statement!
The argument of |\childdocmain| should be left empty
(but it must be present).

%%%%%%%%%%%%%%%%%%%%%%%%%%%%%%%%%%%%%%%%
\DescribeMacro{\childdocof}
Furthermore, add the commands
\begin{center}
\begin{tabular}{l}
|\input{childdoc.def}|\\
|\childdocof{|\textit{main}|}|\\
\end{tabular}
\end{center}
at the top of every child file \textit{child}
which is included by |\include{|\textit{child}|}|
from within the main file
(or at least for those files to be compiled individually).
The argument \textit{main} must be the filename of the main file.

There are a couple of
considerations in setting up the main and child documents:

%%%%%%%%%%%%%%%%%%%%%%%%%%%%%%%%%%%%%%%%
\paragraph{Restrictions.}

Please note the following restrictions:
\begin{itemize}
\item
|\childdocmain| must be called with one argument \textit{main}
to ensure compatibility with earlier version of the package.
It must either be empty (|\childdocmain{}|)
or precisely match the filename of the main file in which it is specified.
See \secref{sec:detection} for further information.
\item
The filename \textit{main} must be specified without the |.tex| extension.
\item
The filename \textit{main} is case sensitive
(even in case-insensitive file systems)
due to internal string comparison.
\item
The argument \textit{main} should be fully expanded, it cannot be a macro.
\item
Subdirectories and special characters should be avoided in filenames.
\item
The command |\childdocmain{|\textit{main}|}| must be followed by a whitespace.
It should not be followed immediately by another command
or by a comment mark `|%|'.
This is because the \TeX{} parser reads the token immediately following
the argument of |\childdocmain| and puts it
at the beginning of every child section;
however, a white\-space is ignored.
\end{itemize}

%%%%%%%%%%%%%%%%%%%%%%%%%%%%%%%%%%%%%%%%
\paragraph{Content of Main File.}

It is advisable to place all content in the child files included by |\include|.
Any output contained in the main file will appear in all child documents
unless suppressed manually;
it cannot be suppressed automatically by the |\includeonly| directive
and thus should normally be avoided.
A method to include some content in the main file
by means of conditional processing is described in \secref{sec:conditional}.

%%%%%%%%%%%%%%%%%%%%%%%%%%%%%%%%%%%%%%%%
\paragraph{Page Numbering.}

When only a part of the document is compiled,
the appropriate numbering of pages
(as well as other status parameters)
is determined from the |.aux| files.
The latter contain information from previous passes.
However this information needs to propagate through
all intermediate child documents.
Therefore the page numbering in child documents may well
be inconsistent until the complete document is compiled at least once.

A useful (if unconventional) way to always ensure a consistent
page numbering is to restart the numbering in each child document
and denote the pages by `\textit{child}|.|\textit{page}'
where \textit{child} represents the chapter/section number of the child file.
This can be achieved by the command
|\numberwithin{page}{|\textit{child}|}|
of the \textsf{amsmath} package
where \textit{child} can be |chapter| or |section|
depending on the chosen structuring.
Alternatively, one can modify the macro |\thepage| appropriately
and reset the counter |page| at the start of each child file.

%%%%%%%%%%%%%%%%%%%%%%%%%%%%%%%%%%%%%%%%%%%%%%%%%%%%%%%%%%%%%%%%%%%%%%%%%%%%%%%%
\subsection{Conditional Processing}
\label{sec:conditional}

The package provides a mechanism to compile different versions
of a document. To customise the versions further some conditional processing
can come in handy to distinguish which version is being compiled.
The package provides two macros to describe the compilation context:

%%%%%%%%%%%%%%%%%%%%%%%%%%%%%%%%%%%%%%%%
\DescribeMacro{\ifchilddoc}
The conditional |\ifchilddoc| distinguishes between the compilation of
child documents and the main document:
%
\begin{center}
|\ifchilddoc |\textit{child-code}| |[|\||else |\textit{main-code}]| \||fi|
\end{center}

%%%%%%%%%%%%%%%%%%%%%%%%%%%%%%%%%%%%%%%%
\DescribeMacro{\childdocname}
\DescribeMacro{\childdocjob}
The macro |\childdocname| contains the filename (without extension)
of the main or child file being processed.
Note that |\childdocjob| will always contain the name of the main file.

%%%%%%%%%%%%%%%%%%%%%%%%%%%%%%%%%%%%%%%%
\paragraph{Title Page.}

Conditional processing can be used to include a title or banner page
in the main document when proper precautions are taken.
Importantly, the code in the main file should ensure that the page counter
(as well as other status parameters which are stored in the |.aux| files)
takes the same value after the conditional processing.
Otherwise the page numbers may take divergent values
depending on which part is compiled.

For example, a title page could be declared by:
%
\begin{center}
\begin{tabular}{l}
|\ifchilddoc\||else|\\
|\addtocounter{page}{-1}|\\
\textit{code for title page}\\
|\newpage|\\
|\||fi|
\end{tabular}
\end{center}
%
A banner page for the child documents can be generated by:
%
\begin{center}
\begin{tabular}{l}
|\ifchilddoc|\\
|\addtocounter{page}{-1}|\\
\textit{code for banner page}\\
|\newpage|\\
|\||fi|
\end{tabular}
\end{center}
%
Here one could write a message such as:
\begin{center}
|This is the part \childdocname{} of \childdocjob{}.|
\end{center}

%%%%%%%%%%%%%%%%%%%%%%%%%%%%%%%%%%%%%%%%%%%%%%%%%%%%%%%%%%%%%%%%%%%%%%%%%%%%%%%%
\subsection{Flags}
\label{sec:flags}

The package makes it easy to generate different versions
of the main or child documents.
To this end compilation flags can be defined
and assigned different default values.
They will be particularly useful in conjunction
with the forwarding mechanism described in \secref{sec:forward}.

For example, it may be useful to have a flag |\version|
which can be set to |draft| or |final|.
The document source will contain some conditional code
depending on the value of |\version|.
Suppose further, the flag should default to |final| for the main file
and to |draft| for child files
which is a natural assignment for editing the document.
This is achieved by placing the following code
in the preamble of the main document
(below the |\childdocmain| directive):
%
\begin{center}
\begin{tabular}{l}
|\ifchilddoc|\\
|\providecommand{\version}{draft}|\\
|\||else|\\
|\providecommand{\version}{final}|\\
|\||fi|
\end{tabular}
\end{center}
%
The definition by |\providecommand| makes sure
that previous definitions are not overwritten.
Further statements |\providecommand{\version}{...}|
can thus be added before the above code to override it.

For the main file, one might add a line
(between |\childdocmain| and the above block)
%
\begin{center}
|%\ifchilddoc\||else\providecommand{\version}{draft}\||fi|
\end{center}
%
which can be uncommented to produce a draft version.
Likewise one can add a line to the very top of a child file
(above the |\childdocof{|\textit{main}|}| directive)
%
\begin{center}
|%\providecommand{\version}{final}|
\end{center}
%
which can be uncommented to produce the final version of this child document.

%%%%%%%%%%%%%%%%%%%%%%%%%%%%%%%%%%%%%%%%%%%%%%%%%%%%%%%%%%%%%%%%%%%%%%%%%%%%%%%%
\subsection{Forwarding}
\label{sec:forward}

Different versions of the main or child documents
using compilation flags as described in \secref{sec:flags}
can be (permanently) stored in different files
for convenient compilation, viewing and distribution.
To this end, the package defines a command
to pass on compilation to a different file:

%%%%%%%%%%%%%%%%%%%%%%%%%%%%%%%%%%%%%%%%
\DescribeMacro{\childdocforward}
The command |\childdocforward| redirects processing to
another source file:
%
\begin{center}
\begin{tabular}{l}
|\input{childdoc.def}|\\
|\childdocforward[|\textit{main}|]{|\textit{dest}|}|\\
\end{tabular}
\end{center}
%
The argument \textit{dest} is the destination file
(without extension).
It should be the main file or one of the child files.
Note that further \textsf{childdoc} directives
such as |\childdocof| and |\childdocforward|
in the indicated file will be processed in this form.
The optional argument \textit{main}
passes on directly to the main file \textit{main}
while pretending to compile the child \textit{dest}.
This form behaves as if \textit{dest}
issues |\childdocof{|\textit{main}|}| right away,
and no further \textsf{childdoc} directives will be processed.

%%%%%%%%%%%%%%%%%%%%%%%%%%%%%%%%%%%%%%%%
\DescribeMacro{\...prefix}
In the alternative form |\childdocforwardprefix|,
%
\begin{center}
\begin{tabular}{l}
|\input{childdoc.def}|\\
|\childdocforwardprefix[|\textit{main}|]{|\textit{prefix}|}{|\textit{dest}|}|
\end{tabular}
\end{center}
%
the destination file is determined by a pattern
depending on the current file:
To make this work, the current file must be called
`{\textit{prefix}\hspace{0.2em}\textit{suffix}}'
with \textit{prefix} matching precisely the argument.
Processing is then passed on to the file
`{\textit{dest}\hspace{0.2em}\textit{suffix}}'.
Surely, the same effect is achieved by
directly specifying the
argument `{\textit{dest}\hspace{0.2em}\textit{suffix}}'
in the first form.
However, that requires to set up a different file
for each child. With the alternative form of the command
all these files can have exactly the same content
which simplifies setting them up and maintaining them.

For example, the following file |draft.tex|
with a compilation flag |\version| as described in \secref{sec:flags}
compiles the main document as a draft:
%
\begin{center}
\begin{tabular}{l}
|\def\version{draft}|\\
|\input{childdoc.def}|\\
|\childdocforward{|\textit{main}|}|
\end{tabular}
\end{center}
%
Likewise, the following files |final|\textit{nn}|.tex|
compile the final version of the child document
|child|\textit{nn}|.tex|:
%
\begin{center}
\begin{tabular}{l}
|\def\version{final}|\\
|\input{childdoc.def}|\\
|\childdocforwardprefix{final}{child}|
\end{tabular}
\end{center}
%

Note that when several versions of a main file and/or of each child file
are to be generated, it may be convenient to set up a |Makefile| or
shell script to automatise the process.

%%%%%%%%%%%%%%%%%%%%%%%%%%%%%%%%%%%%%%%%%%%%%%%%%%%%%%%%%%%%%%%%%%%%%%%%%%%%%%%%
\subsection{Command Line Processing}
\label{sec:commandline}

The effect of redirection files can also be achieved by invoking
the \LaTeX{} compiler with a more elaborate command line.
Most conveniently this should be done as part
of a shell script or a |Makefile|.

When using \textsf{childdoc} in the main file, the following
command lines effectively perform a redirection
(note that depending on the shell being used,
backslashes may have to be doubled: `|\|' $\to$ `|\\|'):
%
\begin{center}
|... -jobname "|\textit{target}|" |\\|"|[\textit{flags}]%
|\input{childdoc.def}\childdocforward[|\textit{main}|]{|\textit{dest}|}"|
\end{center}
%
Here \textit{target} is the name of the output file,
\textit{main} is the name of the main file
and \textit{dest} is the name of the main or child file to be processed
(all filenames without extensions).
The optional argument \textit{main} can be omitted
if \textit{main} matches \textit{dest}.
Optionally, compilation \textit{flags} can be defined via |\def| commands.
This command line makes the \TeX{} engine believe
it is compiling the file \textit{target}
whose content is specified as the latter parameter.
The provided code then forwards the processing to
\textit{main} or \textit{dest} as described in \secref{sec:forward}.

%%%%%%%%%%%%%%%%%%%%%%%%%%%%%%%%%%%%%%%%%%%%%%%%%%%%%%%%%%%%%%%%%%%%%%%%%%%%%%%%
\subsection{Include by Input}
\label{sec:input}

Including child documents by |\include| has some restrictions by design.
Most notably, the content of a child document always occupies
its own set of pages; pages cannot be shared between child documents.
Usually, this behaviour makes perfect sense
because each child document contain an essential part of the document.
However, in some situations it may be desirable to compose
a document from a collection of parts
without having mandatory page breaks between then.
For this case, the package
provides a mechanism to include parts
by |\input| which can also be processed individually.
However, by construction this mechanism
requires manual handling of the content to be output.

%%%%%%%%%%%%%%%%%%%%%%%%%%%%%%%%%%%%%%%%
\DescribeMacro{\ifchilddocmanual}
The main file should be prepared as usual, see \secref{sec:include}.
However, the document body must make a distinction
between processing of an individual part and of the main document, e.g.:
%
\begin{center}
\begin{tabular}{l}
|\ifchilddocmanual|\\
|\input{\childdocname}|\\
|\||else|\\
\textit{document body with }|\input{|\textit{part}|}|\\
|\||fi|
\end{tabular}
\end{center}
%
The conditional |\ifchilddocmanual| is true whenever
a part to be included by |\input| is being compiled,
and the name of the part is stored in |\childdocname|.

%%%%%%%%%%%%%%%%%%%%%%%%%%%%%%%%%%%%%%%%
\DescribeMacro{\childdocby}
Each part to be included by |\input| should start with:
%
\begin{center}
\begin{tabular}{l}
|\input{childdoc.def}|\\
|\childdocby{|\textit{main}|}|\\
\end{tabular}
\end{center}
%
The directive |\childdocby| is similar to |\childdocof|
described in \secref{sec:include},
but the subsequent selection of content must be done manually.
To that end, both |\ifchilddoc| and |\ifchilddocmanual|
will be true upon processing of a part,
and the name of the part is stored in |\childdocname|.
Note that |\jobname| will be set to the filename of the current part
so that each part receives an individual |.aux| file
that does not interfere with the |.aux| file(s) of the main document.
This behaviour can be altered by the alternative form
|\childdocby[*]{|\textit{main}|}| (with a non-empty optional argument)
which uses the |.aux| file of the main document
by setting |\jobname| to \textit{main}.

%%%%%%%%%%%%%%%%%%%%%%%%%%%%%%%%%%%%%%%%%%%%%%%%%%%%%%%%%%%%%%%%%%%%%%%%%%%%%%%%
\subsection{Driver Development}
\label{sec:driver}

The \textsf{childdoc} mechanism can also be use for the development
of definition files such as \LaTeX{} styles or classes.
This case differs from the above setup with multiple parts
included by |\include| in that no |\includeonly| should be invoked.
This can be achieved by starting the include file
(before |\ProvidesPackage|) with:
%
\begin{center}
\begin{tabular}{l}
|\input{childdoc.def}|\\
|\childdocforward{|\textit{main}|}|\\
\end{tabular}
\end{center}
%
or alternatively with:
%
\begin{center}
\begin{tabular}{l}
|\input{childdoc.def}|\\
|\childdocby{|\textit{main}|}|\\
\end{tabular}
\end{center}
%
Both forms have slightly different effects as described above.
The main file is prepared as usual, see \secref{sec:include}.

%%%%%%%%%%%%%%%%%%%%%%%%%%%%%%%%%%%%%%%%%%%%%%%%%%%%%%%%%%%%%%%%%%%%%%%%%%%%%%%%
\subsection{Legacy Detection}
\label{sec:detection}

The directive |\childdocmain| in the main file can detect
whether the complete document or merely a child is to be compiled
even without using the directive |\childdocof|.
This method is deprecated because it is less robust
and there is no compelling reason to use it;
it is merely provided for backward compatibility
and it may be removed in future versions.

If the detection mechanism is to be used,
it is mandatory to correctly specify
the filename of the main file as the argument of |\childdocmain|:
%
\begin{center}
\begin{tabular}{l}
|\input{childdoc.def}|\\
|\childdocmain{|\textit{main}|}|\\
\end{tabular}
\end{center}
%
If |\jobname| does not match the argument \textit{main} of |\childdocmain|,
it is assumed that |\jobname| points to the child file to be compiled.
When using |\childdocmain| with the main file specified as argument,
it suffices to start a child file
with just |\input{|\textit{main}|}|
without loading of the package and using |\childdocof|.
If instead all processing is done
with the appropriate \textsf{childdoc} directives,
the argument of \textit{main} of |\childdocmain| can be empty.

An alternative version of the command line processing described
in \secref{sec:commandline} using the detection mechanism reads:
%
\begin{center}
|... -jobname "|\textit{target}|" "|[\textit{flags}]%
[|\def\jobname{|\textit{dest}|}|]|\input{|\textit{main}|}"|
\end{center}

%%%%%%%%%%%%%%%%%%%%%%%%%%%%%%%%%%%%%%%%%%%%%%%%%%%%%%%%%%%%%%%%%%%%%%%%%%%%%%%%
\subsection{Manual Code}
\label{sec:manual}

In case one cannot be certain whether the definitions file |childdoc.def|
is installed on the target \TeX{} distribution
and one prefers not to ship it,
it is conceivable to paste a few relevant commands into the sources.

To that end, drop all statements |\input{childdoc.def}|
and perform the replacements as outlined below.
Instead of |\childdocmain{|\textit{main}|}| add the following code
to the top of the main file:
%
\begin{center}
\begin{tabular}{l}
|\||ifdefined\childdocname\endinput\||fi\newif\ifchilddoc|\\
|\edef\childdocname{\scantokens\expandafter{\jobname\noexpand}}|\\
|\def\childdocmain{|\textit{main}|}\||ifx\childdocmain\childdocname\||else|\\
|\childdoctrue\includeonly{\childdocname}\let\jobname\childdocmain\||fi|\\
\end{tabular}
\end{center}
%
Instead of |\childdocof{|\textit{main}|}| just include the main file
at the top of each child file:
%
\begin{center}
|\input{|\textit{main}|}|
\end{center}
%
A simple redirection |\childdocforward{|\textit{dest}|}| is achieved by:
%
\begin{center}
|\def\jobname{|\textit{dest}|}\input{\jobname}|
\end{center}
%
The redirection with prefix
|\childdocforwardprefix[|\textit{prefix}|]{|\textit{dest}|}|
is accomplished by:
%
\begin{center}
\begin{tabular}{l}
|{\edef\jobname{\scantokens\expandafter{\jobname\noexpand}}|\\
|\def\redirectjob |\textit{prefix}|#1~~~{\gdef\jobname{|\textit{dest}|#1}}|\\
|\expandafter\redirectjob\jobname~~~}\input{\jobname}|
\end{tabular}
\end{center}

In an alternative approach,
child documents can be compiled by a specific command line
without additional code or specific definitions:
%
\begin{center}
|... -jobname "|\textit{target}|" "|[\textit{flags}]%
|\includeonly{|\textit{dest}|}\input{|\textit{main}|}"|
\end{center}
%

%%%%%%%%%%%%%%%%%%%%%%%%%%%%%%%%%%%%%%%%%%%%%%%%%%%%%%%%%%%%%%%%%%%%%%%%%%%%%%%%
%%%%%%%%%%%%%%%%%%%%%%%%%%%%%%%%%%%%%%%%%%%%%%%%%%%%%%%%%%%%%%%%%%%%%%%%%%%%%%%%
\section{Information}

%%%%%%%%%%%%%%%%%%%%%%%%%%%%%%%%%%%%%%%%%%%%%%%%%%%%%%%%%%%%%%%%%%%%%%%%%%%%%%%%
\subsection{Copyright}

Copyright \copyright{} 2017--2018 Niklas Beisert

This work may be distributed and/or modified under the
conditions of the \LaTeX{} Project Public License, either version 1.3
of this license or (at your option) any later version.
The latest version of this license is in
  \url{http://www.latex-project.org/lppl.txt}
and version 1.3 or later is part of all distributions of \LaTeX{}
version 2005/12/01 or later.

This work has the LPPL maintenance status `maintained'.

The Current Maintainer of this work is Niklas Beisert.

This work consists of the files |README.txt|, |childdoc.ins| and |childdoc.dtx|
as well as the derived files |childdoc.def|, |cdocsamp.tex|
with |cdocsch1.tex|, |cdocsch2.tex|, |cdocspt3.tex|, |cdocspt4.tex|,
|cdocsdrf.tex|, |cdocsfn1.tex|, |cdocsfn2.tex|
as well as |childdoc.pdf|.

%%%%%%%%%%%%%%%%%%%%%%%%%%%%%%%%%%%%%%%%%%%%%%%%%%%%%%%%%%%%%%%%%%%%%%%%%%%%%%%%
\subsection{Files and Installation}

The package consists of the files:
%
\begin{center}
\begin{tabular}{ll}
    |README.txt|   & readme file \\
    |childdoc.ins| & installation file \\
    |childdoc.dtx| & source file \\
    |childdoc.def| & definition file \\
    |cdocsamp.tex| & sample main file \\
    |cdocsch1.tex| & sample include file \\
    |cdocsch2.tex| & sample include file \\
    |cdocspt3.tex| & sample part file \\
    |cdocspt4.tex| & sample part file \\
    |cdocsdrf.tex| & sample redirection file \\
    |cdocsfn1.tex| & sample redirection file \\
    |cdocsfn2.tex| & sample redirection file \\
    |childdoc.pdf| & manual
\end{tabular}
\end{center}
%
The distribution consists of the files
|README.txt|, |childdoc.ins| and |childdoc.dtx|.
%
\begin{itemize}
\item
Run (pdf)\LaTeX{} on |childdoc.dtx|
to compile the manual |childdoc.pdf| (this file).
\item
Run \LaTeX{} on |childdoc.ins| to create the definitions file |childdoc.def|
and the sample |cdocsamp.tex| with include files
|cdocsch1.tex|, |cdocsch2.tex|, |cdocspt3.tex|, |cdocspt4.tex|,
|cdocsdrf.tex|, |cdocsfn1.tex|, |cdocsfn2.tex|.
Then copy the file |childdoc.def| to an appropriate directory of your \LaTeX{}
distribution, e.g.\ \textit{texmf-root}|/tex/latex/childdoc|.
\end{itemize}

%%%%%%%%%%%%%%%%%%%%%%%%%%%%%%%%%%%%%%%%%%%%%%%%%%%%%%%%%%%%%%%%%%%%%%%%%%%%%%%%
\subsection{Related CTAN Packages}

There are several other packages which offer a similar functionality:
%
\begin{itemize}
\item
The packages
\href{http://ctan.org/pkg/docmute}{\textsf{docmute}},
\href{http://ctan.org/pkg/includex}{\textsf{includex}} and
\href{http://ctan.org/pkg/standalone}{\textsf{standalone}}
provide commands to include only the document body of
a child file thus allowing both files to be compiled individually.
\item
The packages \href{http://ctan.org/pkg/subdocs}{\textsf{subdocs}}
and \href{http://ctan.org/pkg/subfiles}{\textsf{subfiles}}
provide structures in which the main and child documents can be
encapsulated and allowing them to be compiled individually.
The inclusion mechanism is different from the conventional |\include|.
\item
The package \href{http://ctan.org/pkg/combine}{\textsf{combine}}
is an elaborate solution to combine several documents into one.
\end{itemize}
%
See also the CTAN topic \href{http://ctan.org/topic/subdocs}{\textsf{subdocs}}
for further related packages.
The present package differs from the above solutions in that
a document structure constructed with the conventional |\include| mechanism
just needs two extra commands at the top of every file
such that all constituent files can be compiled individually.

%%%%%%%%%%%%%%%%%%%%%%%%%%%%%%%%%%%%%%%%%%%%%%%%%%%%%%%%%%%%%%%%%%%%%%%%%%%%%%%%
%\subsection{Feature Suggestions}
%
%The following is a list of features which may be useful for future
%versions of this package:
%%
%\begin{itemize}
%\item
%\ldots
%\end{itemize}

%%%%%%%%%%%%%%%%%%%%%%%%%%%%%%%%%%%%%%%%%%%%%%%%%%%%%%%%%%%%%%%%%%%%%%%%%%%%%%%%
\subsection{Revision History}

%%%%%%%%%%%%%%%%%%%%%%%%%%%%%%%%%%%%%%%%
\paragraph{v2.0:} 2018/12/30

\begin{itemize}
\item
immediate forward processing
\item
added |\childdocby| mechanism
\item
manual restructured
\end{itemize}

%%%%%%%%%%%%%%%%%%%%%%%%%%%%%%%%%%%%%%%%
\paragraph{v1.6:} 2018/01/17

\begin{itemize}
\item
application for development of include files
\item
corrections to manual
\end{itemize}

%%%%%%%%%%%%%%%%%%%%%%%%%%%%%%%%%%%%%%%%
\paragraph{v1.5:} 2017/05/21

\begin{itemize}
\item
more complete structuring introduced
\item
|\childdocof| introduced
\item
|\childdoc| renamed to |\childdocmain|
\item
|\childredirect| renamed to |\childdocforward| and |\childdocforwardprefix|
and functionality expanded
\end{itemize}

%%%%%%%%%%%%%%%%%%%%%%%%%%%%%%%%%%%%%%%%
\paragraph{v1.0:} 2017/04/27

\begin{itemize}
\item
manual and install package
\item
first version published on CTAN
\end{itemize}

%%%%%%%%%%%%%%%%%%%%%%%%%%%%%%%%%%%%%%%%
\paragraph{v0.6:} 2017/04/26

\begin{itemize}
\item
redirection mechanism added
\end{itemize}

%%%%%%%%%%%%%%%%%%%%%%%%%%%%%%%%%%%%%%%%
\paragraph{v0.5:} 2017/04/26

\begin{itemize}
\item
functionality in definition file
\end{itemize}


%%%%%%%%%%%%%%%%%%%%%%%%%%%%%%%%%%%%%%%%%%%%%%%%%%%%%%%%%%%%%%%%%%%%%%%%%%%%%%%%
%%%%%%%%%%%%%%%%%%%%%%%%%%%%%%%%%%%%%%%%%%%%%%%%%%%%%%%%%%%%%%%%%%%%%%%%%%%%%%%%
%%%%%%%%%%%%%%%%%%%%%%%%%%%%%%%%%%%%%%%%%%%%%%%%%%%%%%%%%%%%%%%%%%%%%%%%%%%%%%%%
\appendix

\settowidth\MacroIndent{\rmfamily\scriptsize 000\ }

 \DocInput{childdoc.dtx}

\end{document}
%</driver>
% \fi
%
% %%%%%%%%%%%%%%%%%%%%%%%%%%%%%%%%%%%%%%%%%%%%%%%%%%%%%%%%%%%%%%%%%%%%%%%%%%%%%%
% %%%%%%%%%%%%%%%%%%%%%%%%%%%%%%%%%%%%%%%%%%%%%%%%%%%%%%%%%%%%%%%%%%%%%%%%%%%%%%
% \section{Sample}
%\iffalse
%<*samplemain>
%\fi
%
% The following presents a sample document
% with two chapters, two parts, a title page,
% a compile flag as well as three forwarding files to set the flag.
% It consists of eight |.tex| files:
% \begin{center}
% \begin{tabular}{ll}
% |cdocsamp.tex|&main file\\
% |cdocsch1.tex|&include file for chapter 1\\
% |cdocsch2.tex|&include file for chapter 2\\
% |cdocspt3.tex|&include file for part 3\\
% |cdocspt4.tex|&include file for part 4\\
% |cdocsdrf.tex|&forwarding file for main file in draft mode\\
% |cdocsfi1.tex|&forwarding file for final version of chapter 1\\
% |cdocsfi2.tex|&forwarding file for final version of chapter 2\\
% \end{tabular}
% \end{center}
% Each of the eight files can be compiled directly by the \LaTeX{} compiler.
%
% %%%%%%%%%%%%%%%%%%%%%%%%%%%%%%%%%%%%%%
% \paragraph{Main File.}
%
% The main file is called |cdocsamp.tex|.
%
% Load the \textsf{childdoc} definitions and
% declare the filename for the main document:
%    \begin{macrocode}
\input{childdoc.def}
\childdocmain{}
%    \end{macrocode}

% Optional override for |\version| flag:
%    \begin{macrocode}
%%\ifchilddoc\else\providecommand{\version}{draft}\fi
%    \end{macrocode}

% Define the default values for the |\version| flag
% (|final| for the main file and |draft| for childs):
%    \begin{macrocode}
\ifchilddoc
\providecommand{\version}{draft}
\else
\providecommand{\version}{final}
\fi
%    \end{macrocode}

% Load the standard document class:
%    \begin{macrocode}
\documentclass[12pt]{article}
%    \end{macrocode}

% Start the document body:
%    \begin{macrocode}
\begin{document}
%    \end{macrocode}

% Declare a title page.
% Print title, part of document being processed and version flag:
%    \begin{macrocode}
\addtocounter{page}{-1}
\begin{center}
{\LARGE\bfseries{}childdoc example\par}
\vspace{1cm}
\ifchilddoc
\ifchilddocmanual part\else chapter\fi:
`\childdocname' of `\childdocjob'\par
\else
main document: `\childdocjob'\par
\fi
version: \version\par
\end{center}
\newpage
%    \end{macrocode}

% Manually include selected file,
% otherwise process as usual:
%    \begin{macrocode}
\ifchilddocmanual
\section*{part `\childdocname'}
\input{\childdocname}
\else
%    \end{macrocode}

% Include the two chapters:
%    \begin{macrocode}
\include{cdocsch1}
\include{cdocsch2}
%    \end{macrocode}

% Include the two parts unless only chapters should be displayed:
%    \begin{macrocode}
\ifchilddoc\else
\section{part three}
\input{cdocspt3}
\section{part four}
\input{cdocspt4}
\fi
%    \end{macrocode}

% Process as usual until here:
%    \begin{macrocode}
\fi
%    \end{macrocode}

% End of document body:
%    \begin{macrocode}
\end{document}
%    \end{macrocode}
%\iffalse
%</samplemain>
%\fi
%
% %%%%%%%%%%%%%%%%%%%%%%%%%%%%%%%%%%%%%%
% \paragraph{Chapter Include Files.}
%
% The include files are called |cdocsch1.tex| and |cdocsch2.tex|.
%
%\iffalse
%<*samplechap1|samplechap2>
%\fi

% Optional override for |\version| flag:
%    \begin{macrocode}
%%\providecommand{\version}{final}
%    \end{macrocode}

% Include the main document:
%    \begin{macrocode}
\input{childdoc.def}
\childdocof{cdocsamp}
%    \end{macrocode}

%\iffalse
%</samplechap1|samplechap2>
%\fi
%
%\iffalse
%<*samplechap1>
%\fi
% Some text for chapter 1:
%    \begin{macrocode}
\section{one}
some text in chapter one
%    \end{macrocode}

%\iffalse
%</samplechap1>
%\fi
% Some text for chapter 2:
%\iffalse
%<*samplechap2>
%\fi
%    \begin{macrocode}
\section{two}
more text in chapter two
%    \end{macrocode}

%\iffalse
%</samplechap2>
%\fi
%
% %%%%%%%%%%%%%%%%%%%%%%%%%%%%%%%%%%%%%%
% \paragraph{Part Include Files.}
%
% The include files are called |cdocspt3.tex| and |cdocspt4.tex|.
%
%\iffalse
%<*samplepart3|samplepart4>
%\fi

% Optional override for |\version| flag:
%    \begin{macrocode}
%%\providecommand{\version}{final}
%    \end{macrocode}

% Include the main document:
%    \begin{macrocode}
\input{childdoc.def}
\childdocby{cdocsamp}
%    \end{macrocode}

%\iffalse
%</samplepart3|samplepart4>
%\fi
%
%\iffalse
%<*samplepart3>
%\fi
% Some text for part 3:
%    \begin{macrocode}
some text in part three
%    \end{macrocode}

%\iffalse
%</samplepart3>
%\fi
% Some text for part 4:
%\iffalse
%<*samplepart4>
%\fi
%    \begin{macrocode}
more text in part four
%    \end{macrocode}

%\iffalse
%</samplepart4>
%\fi
%
% %%%%%%%%%%%%%%%%%%%%%%%%%%%%%%%%%%%%%%
% \paragraph{Forwarding for a Complete Draft.}
%
% The following forwarding file |cdocsdrf.tex|
% compiles the main document in draft mode:
%\iffalse
%<*sampledraft>
%\fi
%    \begin{macrocode}
\def\version{draft}
\input{childdoc.def}
\childdocforward{cdocsamp}
%    \end{macrocode}

%\iffalse
%</sampledraft>
%\fi
%
% %%%%%%%%%%%%%%%%%%%%%%%%%%%%%%%%%%%%%%
% \paragraph{Forwarding for Final Version of the Chapters.}
%
% The following forwarding files |cdocsfn1.tex| and |cdocsfn2.tex|
% (with identical content)
% compile the final versions of the child documents
% |cdocsch1.tex| and |cdocsch2.tex|, respectively:
%\iffalse
%<*samplefinal>
%\fi
%    \begin{macrocode}
\def\version{final}
\input{childdoc.def}
\childdocforwardprefix[cdocsamp]{cdocsfn}{cdocsch}
%    \end{macrocode}

%\iffalse
%</samplefinal>
%\fi
%
% %%%%%%%%%%%%%%%%%%%%%%%%%%%%%%%%%%%%%%
% \paragraph{Command Line Processing.}
%
% The following three command lines generate the output files
% |cdocscld|, |cdocscl1| and |cdocscl2|
% which should be identical to
% |cdocsdrf|, |cdocsch1| and |cdocsfn2|, respectively:
% \begin{center}
% \begin{tabular}{l}
% |latex -jobname cdocscld \|\\
% |  "\def\version{draft}\input{childdoc.def}\childdocforward{cdocsamp}"|\\
% |latex -jobname cdocscl1 \|\\
% |  "\input{childdoc.def}\childdocforward[cdocsamp]{cdocsch1}"|\\
% |latex -jobname cdocscl2 \|\\
% |  "\def\version{final}\input{childdoc.def}\childdocforward{cdocsch2}"|
% \end{tabular}
% \end{center}
% Note that the trailing backslash on each first line
% merely continues the input to the second line
% (for convenient cut ant paste).
% Furthermore, the command |latex| can be replaced by any
% of its alternative versions such as |pdflatex|.
%
% %%%%%%%%%%%%%%%%%%%%%%%%%%%%%%%%%%%%%%%%%%%%%%%%%%%%%%%%%%%%%%%%%%%%%%%%%%%%%%
% %%%%%%%%%%%%%%%%%%%%%%%%%%%%%%%%%%%%%%%%%%%%%%%%%%%%%%%%%%%%%%%%%%%%%%%%%%%%%%
% \section{Implementation}
%\iffalse
%<*package>
%\fi
%
% This section describes the definitions file |childdoc.def|.

% The definitions cannot be loaded using |\usepackage| or |\RequirePackage|
% which has a mechanism to prevent loading a style file more than once.
% When loading the definitions by means of |\input|
% multiple instances have to be prevented manually:
%\iffalse
%This code needs to be before the `\ProvidesFile' directive
%which is defined at the beginning of this file.
%Therefore it is also placed there and commented out here.
%</package>
%<*discard>
%\fi
%    \begin{macrocode}
\ifdefined\childdocmain\endinput\fi
%    \end{macrocode}
%\iffalse
%</discard>
%<*package>
%\fi
%
% \macro{\ifchilddoc}
% \macro{\ifchilddocmanual}
% The conditional |\ifchilddoc| tells whether a
% child (true) or main (false) document is being compiled.
% The conditional |\ifchilddocmanual| tells whether
% the |\includeonly| mechanism is used (false) or
% the selection of child files must be performed manually (true).
% The definitions initialise to false:
%    \begin{macrocode}
\newif\ifchilddoc
\newif\ifchilddocmanual
%    \end{macrocode}

% \macro{\childdocname}
% \macro{\childdocjob}
% The macro |\childdocname| stores the name of the main document
% to be compiled. The macro |\childdocjob| stores the name of
% the document on which the \LaTeX{} compiler was originally invoked.
% The content of |\jobname| cannot be compared
% to filenames specified in the source due to different catcodes.
% The following code rescans |\jobname|, stores the result
% in |\childdocname| and saves a copy in |\childdocjob|:
%    \begin{macrocode}
\edef\childdocname{\scantokens\expandafter{\jobname\noexpand}}
\let\childdocjob\childdocname
%    \end{macrocode}

% \macro{\childdocdisable}
% The macro |\childdocdisable| prevents the main file
% from being processed more than once.
% At this stage, the main document command |\childdocmain|
% is assumed to be called once again where it should do nothing.
% Any subsequent call to it should prevent
% a secondary processing of the main document
% It overwrites the forwarding commands
% |\childdocof| and |\childdocforward|
% with empty macros to prevent further inclusions of the main document:
%    \begin{macrocode}
\newcommand{\childdocdisable}
{
  \renewcommand{\childdocmain}[1]{\renewcommand{\childdocmain}[1]{\endinput}}
  \renewcommand{\childdocof}[1]{}
  \renewcommand{\childdocby}[2][]{}
  \renewcommand{\childdocforward}[2][]{}
  \renewcommand{\childdocdisable}{}
}
%    \end{macrocode}

% \macro{\childdocmain}
% The macro |\childdocmain| is to be called at the top of the main file
% with nothing or the main filename (without extension) as argument.
% First, it breaks loops.
% If the argument is not empty and does not match |\childdocname|
% (which is set by the first inclusion of |childdoc.def|),
% |\ifchilddoc| is set to true, |\includeonly| is applied to the child file
% and |\jobname| is set to the main file
% (for proper handling of |.aux| files):
%    \begin{macrocode}
\newcommand{\childdocmain}[1]
{
  \childdocdisable\childdocmain{}
  \if?#1?\else
    \begingroup
      \def\childdoctmp{#1}
      \ifx\childdoctmp\childdocname
        \def\childdoctmp{}
      \else
        \def\childdoctmp
        {
          \childdoctrue
          \includeonly{\childdocname}
          \def\childdocjob{#1}
          \def\jobname{#1}
        }
      \fi
      \expandafter
    \endgroup
    \childdoctmp
  \fi
}
%    \end{macrocode}

% \macro{\childdocof}
% The command |\childdocof| redirects
% compilation to the main file |#1|.
%    \begin{macrocode}
\newcommand{\childdocof}[1]
{
  \childdocdisable
  \childdoctrue
  \includeonly{\childdocname}
  \def\jobname{#1}
  \def\childdocjob{#1}
  \input{#1}
}
%    \end{macrocode}

% \macro{\childdocby}
% The command |\childdocby| ....
%    \begin{macrocode}
\newcommand{\childdocby}[2][]
{
  \childdocdisable
  \childdoctrue
  \childdocmanualtrue
  \if?#1?\else
    \def\jobname{#2}
  \fi
  \def\childdocjob{#2}
  \input{#2}
  \endinput
}
%    \end{macrocode}

% \macro{\childdocforward}
% The command |\childdocforward| redirects
% compilation to the main file or
% (if the optional argument is given) a child file.
% Parameters are set as if the main file
% or a child file starting with |\childdocof| was compiled.
% Then compilation is handed over to the main file:
%    \begin{macrocode}
\newcommand{\childdocforward}[2][]
{
  \begingroup
    \if?#1?
      \def\childdoctmp
      {
        \def\childdocname{#2}
        \def\childdocjob{#2}
        \def\jobname{#2}
        \input{#2}
        \endinput
      }
    \else
      \def\childdoctmp
      {
        \childdocdisable
        \def\childdocname{#2}
        \childdoctrue
        \includeonly{#2}
        \def\childdocjob{#1}
        \def\jobname{#1}
        \input{#1}
        \endinput
      }
    \fi
    \expandafter
  \endgroup
  \childdoctmp
}
%    \end{macrocode}

% \macro{\childdocforwardprefix}
% The command |\childdocforwardprefix| redirects
% compilation to the main or a child file by means of a pattern.
% The prefix |#1| in the current filename is replaced by |#2|
% and the suffix of the current filename is kept
% (it is assumed that the filename does not contain the substring `|~~~|'
% which is used as a delimiter).
% Compilation is handed over to the new file by |\childdocforward|:
%    \begin{macrocode}
\newcommand{\childdocforwardprefix}[3][]
{
  \begingroup
    \def\childdocextract #2##1~~~{\def\childdoctmp{\childdocforward[#1]{#3##1}}}
    \expandafter\childdocextract\childdocname~~~
    \expandafter
  \endgroup
  \childdoctmp
}
%    \end{macrocode}

% \macro{\childdoc}
% The deprecated macro |\childdoc| is a legacy version of |\childdocmain|:
%    \begin{macrocode}
\newcommand{\childdoc}{\childdocmain}
%    \end{macrocode}

% \macro{\childdocredirect}
% The deprecated macro |\childdocredirect| is a legacy version
% of |\childdocforward| and |\childdocforwardprefix|:
%    \begin{macrocode}
\newcommand{\childdocredirect}[2][]
{
  \begingroup
    \if?#1?
      \def\childdoctmp{\childdocforward{#2}}
    \else
      \def\childdoctmp{\childdocforwardprefix{#1}{#2}}
    \fi
    \expandafter
  \endgroup
  \childdoctmp
}
%    \end{macrocode}

%\iffalse
%</package>
%\fi
%
\endinput
|\\
|\childdocof{|\textit{main}|}|\\
\end{tabular}
\end{center}
at the top of every child file \textit{child}
which is included by |\include{|\textit{child}|}|
from within the main file
(or at least for those files to be compiled individually).
The argument \textit{main} must be the filename of the main file.

There are a couple of
considerations in setting up the main and child documents:

%%%%%%%%%%%%%%%%%%%%%%%%%%%%%%%%%%%%%%%%
\paragraph{Restrictions.}

Please note the following restrictions:
\begin{itemize}
\item
|\childdocmain| must be called with one argument \textit{main}
to ensure compatibility with earlier version of the package.
It must either be empty (|\childdocmain{}|)
or precisely match the filename of the main file in which it is specified.
See \secref{sec:detection} for further information.
\item
The filename \textit{main} must be specified without the |.tex| extension.
\item
The filename \textit{main} is case sensitive
(even in case-insensitive file systems)
due to internal string comparison.
\item
The argument \textit{main} should be fully expanded, it cannot be a macro.
\item
Subdirectories and special characters should be avoided in filenames.
\item
The command |\childdocmain{|\textit{main}|}| must be followed by a whitespace.
It should not be followed immediately by another command
or by a comment mark `|%|'.
This is because the \TeX{} parser reads the token immediately following
the argument of |\childdocmain| and puts it
at the beginning of every child section;
however, a white\-space is ignored.
\end{itemize}

%%%%%%%%%%%%%%%%%%%%%%%%%%%%%%%%%%%%%%%%
\paragraph{Content of Main File.}

It is advisable to place all content in the child files included by |\include|.
Any output contained in the main file will appear in all child documents
unless suppressed manually;
it cannot be suppressed automatically by the |\includeonly| directive
and thus should normally be avoided.
A method to include some content in the main file
by means of conditional processing is described in \secref{sec:conditional}.

%%%%%%%%%%%%%%%%%%%%%%%%%%%%%%%%%%%%%%%%
\paragraph{Page Numbering.}

When only a part of the document is compiled,
the appropriate numbering of pages
(as well as other status parameters)
is determined from the |.aux| files.
The latter contain information from previous passes.
However this information needs to propagate through
all intermediate child documents.
Therefore the page numbering in child documents may well
be inconsistent until the complete document is compiled at least once.

A useful (if unconventional) way to always ensure a consistent
page numbering is to restart the numbering in each child document
and denote the pages by `\textit{child}|.|\textit{page}'
where \textit{child} represents the chapter/section number of the child file.
This can be achieved by the command
|\numberwithin{page}{|\textit{child}|}|
of the \textsf{amsmath} package
where \textit{child} can be |chapter| or |section|
depending on the chosen structuring.
Alternatively, one can modify the macro |\thepage| appropriately
and reset the counter |page| at the start of each child file.

%%%%%%%%%%%%%%%%%%%%%%%%%%%%%%%%%%%%%%%%%%%%%%%%%%%%%%%%%%%%%%%%%%%%%%%%%%%%%%%%
\subsection{Conditional Processing}
\label{sec:conditional}

The package provides a mechanism to compile different versions
of a document. To customise the versions further some conditional processing
can come in handy to distinguish which version is being compiled.
The package provides two macros to describe the compilation context:

%%%%%%%%%%%%%%%%%%%%%%%%%%%%%%%%%%%%%%%%
\DescribeMacro{\ifchilddoc}
The conditional |\ifchilddoc| distinguishes between the compilation of
child documents and the main document:
%
\begin{center}
|\ifchilddoc |\textit{child-code}| |[|\||else |\textit{main-code}]| \||fi|
\end{center}

%%%%%%%%%%%%%%%%%%%%%%%%%%%%%%%%%%%%%%%%
\DescribeMacro{\childdocname}
\DescribeMacro{\childdocjob}
The macro |\childdocname| contains the filename (without extension)
of the main or child file being processed.
Note that |\childdocjob| will always contain the name of the main file.

%%%%%%%%%%%%%%%%%%%%%%%%%%%%%%%%%%%%%%%%
\paragraph{Title Page.}

Conditional processing can be used to include a title or banner page
in the main document when proper precautions are taken.
Importantly, the code in the main file should ensure that the page counter
(as well as other status parameters which are stored in the |.aux| files)
takes the same value after the conditional processing.
Otherwise the page numbers may take divergent values
depending on which part is compiled.

For example, a title page could be declared by:
%
\begin{center}
\begin{tabular}{l}
|\ifchilddoc\||else|\\
|\addtocounter{page}{-1}|\\
\textit{code for title page}\\
|\newpage|\\
|\||fi|
\end{tabular}
\end{center}
%
A banner page for the child documents can be generated by:
%
\begin{center}
\begin{tabular}{l}
|\ifchilddoc|\\
|\addtocounter{page}{-1}|\\
\textit{code for banner page}\\
|\newpage|\\
|\||fi|
\end{tabular}
\end{center}
%
Here one could write a message such as:
\begin{center}
|This is the part \childdocname{} of \childdocjob{}.|
\end{center}

%%%%%%%%%%%%%%%%%%%%%%%%%%%%%%%%%%%%%%%%%%%%%%%%%%%%%%%%%%%%%%%%%%%%%%%%%%%%%%%%
\subsection{Flags}
\label{sec:flags}

The package makes it easy to generate different versions
of the main or child documents.
To this end compilation flags can be defined
and assigned different default values.
They will be particularly useful in conjunction
with the forwarding mechanism described in \secref{sec:forward}.

For example, it may be useful to have a flag |\version|
which can be set to |draft| or |final|.
The document source will contain some conditional code
depending on the value of |\version|.
Suppose further, the flag should default to |final| for the main file
and to |draft| for child files
which is a natural assignment for editing the document.
This is achieved by placing the following code
in the preamble of the main document
(below the |\childdocmain| directive):
%
\begin{center}
\begin{tabular}{l}
|\ifchilddoc|\\
|\providecommand{\version}{draft}|\\
|\||else|\\
|\providecommand{\version}{final}|\\
|\||fi|
\end{tabular}
\end{center}
%
The definition by |\providecommand| makes sure
that previous definitions are not overwritten.
Further statements |\providecommand{\version}{...}|
can thus be added before the above code to override it.

For the main file, one might add a line
(between |\childdocmain| and the above block)
%
\begin{center}
|%\ifchilddoc\||else\providecommand{\version}{draft}\||fi|
\end{center}
%
which can be uncommented to produce a draft version.
Likewise one can add a line to the very top of a child file
(above the |\childdocof{|\textit{main}|}| directive)
%
\begin{center}
|%\providecommand{\version}{final}|
\end{center}
%
which can be uncommented to produce the final version of this child document.

%%%%%%%%%%%%%%%%%%%%%%%%%%%%%%%%%%%%%%%%%%%%%%%%%%%%%%%%%%%%%%%%%%%%%%%%%%%%%%%%
\subsection{Forwarding}
\label{sec:forward}

Different versions of the main or child documents
using compilation flags as described in \secref{sec:flags}
can be (permanently) stored in different files
for convenient compilation, viewing and distribution.
To this end, the package defines a command
to pass on compilation to a different file:

%%%%%%%%%%%%%%%%%%%%%%%%%%%%%%%%%%%%%%%%
\DescribeMacro{\childdocforward}
The command |\childdocforward| redirects processing to
another source file:
%
\begin{center}
\begin{tabular}{l}
|% \iffalse
%
% childdoc.dtx Copyright (C) 2017-2018 Niklas Beisert
%
% This work may be distributed and/or modified under the
% conditions of the LaTeX Project Public License, either version 1.3
% of this license or (at your option) any later version.
% The latest version of this license is in
%   http://www.latex-project.org/lppl.txt
% and version 1.3 or later is part of all distributions of LaTeX
% version 2005/12/01 or later.
%
% This work has the LPPL maintenance status `maintained'.
%
% The Current Maintainer of this work is Niklas Beisert.
%
% This work consists of the files childdoc.dtx and childdoc.ins
% and the derived files childdoc.def and cdocsamp.tex with
% cdocsch1.tex, cdocsch2.tex, cdocsdrf.tex, cdocsfn1.tex, cdocsfn2.tex.
%
%<package>\ifdefined\childdocmain\endinput\fi
%<package>\ProvidesFile{childdoc.def}[2018/12/30 v2.0 child document driver]
%<samplemain>\ProvidesFile{cdocsamp.tex}[2018/12/30 v2.0 sample for childdoc]
%<*driver>
%\ProvidesFile{childdoc.drv}[2018/12/30 v2.0 childdoc reference manual file]
\PassOptionsToClass{10pt,a4paper}{article}
\documentclass{ltxdoc}

\usepackage[margin=35mm]{geometry}
\usepackage{hyperref}
\usepackage{hyperxmp}
\usepackage[usenames]{color}

\hypersetup{colorlinks=true}
\hypersetup{pdfstartview=FitH}
\hypersetup{pdfpagemode=UseNone}
\hypersetup{pdfsource={}}
\hypersetup{pdflang={en-UK}}
\hypersetup{pdfcopyright={Copyright 2017-2018 Niklas Beisert.
  This work may be distributed and/or modified under the
  conditions of the LaTeX Project Public License, either version 1.3
  of this license or (at your option) any later version.}}
\hypersetup{pdflicenseurl={http://www.latex-project.org/lppl.txt}}
\hypersetup{pdfcontactaddress={ETH Zurich, ITP, HIT K,
  Wolfgang-Pauli-Strasse 27}}
\hypersetup{pdfcontactpostcode={8093}}
\hypersetup{pdfcontactcity={Zurich}}
\hypersetup{pdfcontactcountry={Switzerland}}
\hypersetup{pdfcontactemail={nbeisert@itp.phys.ethz.ch}}
\hypersetup{pdfcontacturl={http://people.phys.ethz.ch/\xmptilde nbeisert/}}

\newcommand{\secref}[1]{\hyperref[#1]{section \ref*{#1}}}

\parskip1ex
\parindent0pt
\let\olditemize\itemize
\def\itemize{\olditemize\parskip0pt}

\begin{document}

\title{The \textsf{childdoc} Package}
\hypersetup{pdftitle={The childdoc Package}}
\author{Niklas Beisert\\[2ex]
  Institut f\"ur Theoretische Physik\\
  Eidgen\"ossische Technische Hochschule Z\"urich\\
  Wolfgang-Pauli-Strasse 27, 8093 Z\"urich, Switzerland\\[1ex]
  \href{mailto:nbeisert@itp.phys.ethz.ch}
  {\texttt{nbeisert@itp.phys.ethz.ch}}}
\hypersetup{pdfauthor={Niklas Beisert}}
\hypersetup{pdfsubject={Manual for the LaTeX2e Package childdoc}}
\date{30 December 2018, \textsf{v2.0}}
\maketitle

\begin{abstract}\noindent
\textsf{childdoc} is a \LaTeXe{} package
that enables the direct compilation
of document sections included by |\include|
to individual files.
\end{abstract}

\begingroup
\parskip0ex
\tableofcontents
\endgroup

%%%%%%%%%%%%%%%%%%%%%%%%%%%%%%%%%%%%%%%%%%%%%%%%%%%%%%%%%%%%%%%%%%%%%%%%%%%%%%%%
%%%%%%%%%%%%%%%%%%%%%%%%%%%%%%%%%%%%%%%%%%%%%%%%%%%%%%%%%%%%%%%%%%%%%%%%%%%%%%%%
\section{Introduction}

\LaTeX{} provides a mechanism to structure a large document (such as a book)
into a main file and several child files (containing the chapters)
using the |\include| command.
This mechanism is beneficial for documents
which span hundreds of pages in order to
make the source file(s) more manageable.
Moreover, compilation can be restricted to
selected child files by means of the |\includeonly| command.
The latter feature can be used to reduce the compilation time while editing
(this was significantly more useful in the earlier days of \LaTeX{})
or to generate a smaller document which is easier to navigate.
Another application of |\includeonly| is to generate
documents consisting of selected parts of the complete document.

However, there are a few drawbacks of the plain |\include| mechanism:
\begin{itemize}
\item
The child files cannot be compiled on their own,
they can only be compiled via the main file.
A naive editing environment
(such as a text editor with an option
to have the current file processed by \LaTeX)
may require one to switch to the main file before compiling;
attempting to compile the child file produces errors.
\item
The main file must be modified (each time)
to adjust the |\includeonly| command
to the present needs. This easily leaves the main file in a messy state.
\item
The generated document will always carry the filename
of the main document. This is inconvenient if
several child files are to be compiled and
to be kept for distribution.
\end{itemize}

The present package provides a simple interface
to make child files individually compilable by \LaTeX{}.
Compiling a child file then has the same effect as compiling
the main file with an |\includeonly| command
to select the appropriate child.
Moreover the generated document will carry the name of the child
rather than the main file.
This resolves all three above issues.

This feature is meant to make the editing of books,
thesis documents and lecture notes somewhat more convenient.
However, the package can also be used efficiently for
composing a series of documents (such as exercise sheets)
which are typically distributed individually.
It then assists the author in generating the individual documents
(potentially in different versions)
as well as a document containing the collected series.
Another application is in developing style files
or other kinds of included material
where compilation of the style file could redirect
to a sample or test file.

%%%%%%%%%%%%%%%%%%%%%%%%%%%%%%%%%%%%%%%%%%%%%%%%%%%%%%%%%%%%%%%%%%%%%%%%%%%%%%%%
%%%%%%%%%%%%%%%%%%%%%%%%%%%%%%%%%%%%%%%%%%%%%%%%%%%%%%%%%%%%%%%%%%%%%%%%%%%%%%%%
\section{Usage}

First of all, the package \textsf{childdoc} is \emph{not} a standard
\LaTeXe{} |.sty| style file! Therefore it needs to be invoked in
a non-standard way.

%%%%%%%%%%%%%%%%%%%%%%%%%%%%%%%%%%%%%%%%%%%%%%%%%%%%%%%%%%%%%%%%%%%%%%%%%%%%%%%%
\subsection{Included Files}
\label{sec:include}

%%%%%%%%%%%%%%%%%%%%%%%%%%%%%%%%%%%%%%%%
\DescribeMacro{\childdocmain}
To use the package, add the commands
\begin{center}
\begin{tabular}{l}
|\input{childdoc.def}|\\
|\childdocmain{}|\\
\end{tabular}
\end{center}
at the very top of the main \LaTeX{} file,
in particular \emph{before} the |\documentclass| statement!
The argument of |\childdocmain| should be left empty
(but it must be present).

%%%%%%%%%%%%%%%%%%%%%%%%%%%%%%%%%%%%%%%%
\DescribeMacro{\childdocof}
Furthermore, add the commands
\begin{center}
\begin{tabular}{l}
|\input{childdoc.def}|\\
|\childdocof{|\textit{main}|}|\\
\end{tabular}
\end{center}
at the top of every child file \textit{child}
which is included by |\include{|\textit{child}|}|
from within the main file
(or at least for those files to be compiled individually).
The argument \textit{main} must be the filename of the main file.

There are a couple of
considerations in setting up the main and child documents:

%%%%%%%%%%%%%%%%%%%%%%%%%%%%%%%%%%%%%%%%
\paragraph{Restrictions.}

Please note the following restrictions:
\begin{itemize}
\item
|\childdocmain| must be called with one argument \textit{main}
to ensure compatibility with earlier version of the package.
It must either be empty (|\childdocmain{}|)
or precisely match the filename of the main file in which it is specified.
See \secref{sec:detection} for further information.
\item
The filename \textit{main} must be specified without the |.tex| extension.
\item
The filename \textit{main} is case sensitive
(even in case-insensitive file systems)
due to internal string comparison.
\item
The argument \textit{main} should be fully expanded, it cannot be a macro.
\item
Subdirectories and special characters should be avoided in filenames.
\item
The command |\childdocmain{|\textit{main}|}| must be followed by a whitespace.
It should not be followed immediately by another command
or by a comment mark `|%|'.
This is because the \TeX{} parser reads the token immediately following
the argument of |\childdocmain| and puts it
at the beginning of every child section;
however, a white\-space is ignored.
\end{itemize}

%%%%%%%%%%%%%%%%%%%%%%%%%%%%%%%%%%%%%%%%
\paragraph{Content of Main File.}

It is advisable to place all content in the child files included by |\include|.
Any output contained in the main file will appear in all child documents
unless suppressed manually;
it cannot be suppressed automatically by the |\includeonly| directive
and thus should normally be avoided.
A method to include some content in the main file
by means of conditional processing is described in \secref{sec:conditional}.

%%%%%%%%%%%%%%%%%%%%%%%%%%%%%%%%%%%%%%%%
\paragraph{Page Numbering.}

When only a part of the document is compiled,
the appropriate numbering of pages
(as well as other status parameters)
is determined from the |.aux| files.
The latter contain information from previous passes.
However this information needs to propagate through
all intermediate child documents.
Therefore the page numbering in child documents may well
be inconsistent until the complete document is compiled at least once.

A useful (if unconventional) way to always ensure a consistent
page numbering is to restart the numbering in each child document
and denote the pages by `\textit{child}|.|\textit{page}'
where \textit{child} represents the chapter/section number of the child file.
This can be achieved by the command
|\numberwithin{page}{|\textit{child}|}|
of the \textsf{amsmath} package
where \textit{child} can be |chapter| or |section|
depending on the chosen structuring.
Alternatively, one can modify the macro |\thepage| appropriately
and reset the counter |page| at the start of each child file.

%%%%%%%%%%%%%%%%%%%%%%%%%%%%%%%%%%%%%%%%%%%%%%%%%%%%%%%%%%%%%%%%%%%%%%%%%%%%%%%%
\subsection{Conditional Processing}
\label{sec:conditional}

The package provides a mechanism to compile different versions
of a document. To customise the versions further some conditional processing
can come in handy to distinguish which version is being compiled.
The package provides two macros to describe the compilation context:

%%%%%%%%%%%%%%%%%%%%%%%%%%%%%%%%%%%%%%%%
\DescribeMacro{\ifchilddoc}
The conditional |\ifchilddoc| distinguishes between the compilation of
child documents and the main document:
%
\begin{center}
|\ifchilddoc |\textit{child-code}| |[|\||else |\textit{main-code}]| \||fi|
\end{center}

%%%%%%%%%%%%%%%%%%%%%%%%%%%%%%%%%%%%%%%%
\DescribeMacro{\childdocname}
\DescribeMacro{\childdocjob}
The macro |\childdocname| contains the filename (without extension)
of the main or child file being processed.
Note that |\childdocjob| will always contain the name of the main file.

%%%%%%%%%%%%%%%%%%%%%%%%%%%%%%%%%%%%%%%%
\paragraph{Title Page.}

Conditional processing can be used to include a title or banner page
in the main document when proper precautions are taken.
Importantly, the code in the main file should ensure that the page counter
(as well as other status parameters which are stored in the |.aux| files)
takes the same value after the conditional processing.
Otherwise the page numbers may take divergent values
depending on which part is compiled.

For example, a title page could be declared by:
%
\begin{center}
\begin{tabular}{l}
|\ifchilddoc\||else|\\
|\addtocounter{page}{-1}|\\
\textit{code for title page}\\
|\newpage|\\
|\||fi|
\end{tabular}
\end{center}
%
A banner page for the child documents can be generated by:
%
\begin{center}
\begin{tabular}{l}
|\ifchilddoc|\\
|\addtocounter{page}{-1}|\\
\textit{code for banner page}\\
|\newpage|\\
|\||fi|
\end{tabular}
\end{center}
%
Here one could write a message such as:
\begin{center}
|This is the part \childdocname{} of \childdocjob{}.|
\end{center}

%%%%%%%%%%%%%%%%%%%%%%%%%%%%%%%%%%%%%%%%%%%%%%%%%%%%%%%%%%%%%%%%%%%%%%%%%%%%%%%%
\subsection{Flags}
\label{sec:flags}

The package makes it easy to generate different versions
of the main or child documents.
To this end compilation flags can be defined
and assigned different default values.
They will be particularly useful in conjunction
with the forwarding mechanism described in \secref{sec:forward}.

For example, it may be useful to have a flag |\version|
which can be set to |draft| or |final|.
The document source will contain some conditional code
depending on the value of |\version|.
Suppose further, the flag should default to |final| for the main file
and to |draft| for child files
which is a natural assignment for editing the document.
This is achieved by placing the following code
in the preamble of the main document
(below the |\childdocmain| directive):
%
\begin{center}
\begin{tabular}{l}
|\ifchilddoc|\\
|\providecommand{\version}{draft}|\\
|\||else|\\
|\providecommand{\version}{final}|\\
|\||fi|
\end{tabular}
\end{center}
%
The definition by |\providecommand| makes sure
that previous definitions are not overwritten.
Further statements |\providecommand{\version}{...}|
can thus be added before the above code to override it.

For the main file, one might add a line
(between |\childdocmain| and the above block)
%
\begin{center}
|%\ifchilddoc\||else\providecommand{\version}{draft}\||fi|
\end{center}
%
which can be uncommented to produce a draft version.
Likewise one can add a line to the very top of a child file
(above the |\childdocof{|\textit{main}|}| directive)
%
\begin{center}
|%\providecommand{\version}{final}|
\end{center}
%
which can be uncommented to produce the final version of this child document.

%%%%%%%%%%%%%%%%%%%%%%%%%%%%%%%%%%%%%%%%%%%%%%%%%%%%%%%%%%%%%%%%%%%%%%%%%%%%%%%%
\subsection{Forwarding}
\label{sec:forward}

Different versions of the main or child documents
using compilation flags as described in \secref{sec:flags}
can be (permanently) stored in different files
for convenient compilation, viewing and distribution.
To this end, the package defines a command
to pass on compilation to a different file:

%%%%%%%%%%%%%%%%%%%%%%%%%%%%%%%%%%%%%%%%
\DescribeMacro{\childdocforward}
The command |\childdocforward| redirects processing to
another source file:
%
\begin{center}
\begin{tabular}{l}
|\input{childdoc.def}|\\
|\childdocforward[|\textit{main}|]{|\textit{dest}|}|\\
\end{tabular}
\end{center}
%
The argument \textit{dest} is the destination file
(without extension).
It should be the main file or one of the child files.
Note that further \textsf{childdoc} directives
such as |\childdocof| and |\childdocforward|
in the indicated file will be processed in this form.
The optional argument \textit{main}
passes on directly to the main file \textit{main}
while pretending to compile the child \textit{dest}.
This form behaves as if \textit{dest}
issues |\childdocof{|\textit{main}|}| right away,
and no further \textsf{childdoc} directives will be processed.

%%%%%%%%%%%%%%%%%%%%%%%%%%%%%%%%%%%%%%%%
\DescribeMacro{\...prefix}
In the alternative form |\childdocforwardprefix|,
%
\begin{center}
\begin{tabular}{l}
|\input{childdoc.def}|\\
|\childdocforwardprefix[|\textit{main}|]{|\textit{prefix}|}{|\textit{dest}|}|
\end{tabular}
\end{center}
%
the destination file is determined by a pattern
depending on the current file:
To make this work, the current file must be called
`{\textit{prefix}\hspace{0.2em}\textit{suffix}}'
with \textit{prefix} matching precisely the argument.
Processing is then passed on to the file
`{\textit{dest}\hspace{0.2em}\textit{suffix}}'.
Surely, the same effect is achieved by
directly specifying the
argument `{\textit{dest}\hspace{0.2em}\textit{suffix}}'
in the first form.
However, that requires to set up a different file
for each child. With the alternative form of the command
all these files can have exactly the same content
which simplifies setting them up and maintaining them.

For example, the following file |draft.tex|
with a compilation flag |\version| as described in \secref{sec:flags}
compiles the main document as a draft:
%
\begin{center}
\begin{tabular}{l}
|\def\version{draft}|\\
|\input{childdoc.def}|\\
|\childdocforward{|\textit{main}|}|
\end{tabular}
\end{center}
%
Likewise, the following files |final|\textit{nn}|.tex|
compile the final version of the child document
|child|\textit{nn}|.tex|:
%
\begin{center}
\begin{tabular}{l}
|\def\version{final}|\\
|\input{childdoc.def}|\\
|\childdocforwardprefix{final}{child}|
\end{tabular}
\end{center}
%

Note that when several versions of a main file and/or of each child file
are to be generated, it may be convenient to set up a |Makefile| or
shell script to automatise the process.

%%%%%%%%%%%%%%%%%%%%%%%%%%%%%%%%%%%%%%%%%%%%%%%%%%%%%%%%%%%%%%%%%%%%%%%%%%%%%%%%
\subsection{Command Line Processing}
\label{sec:commandline}

The effect of redirection files can also be achieved by invoking
the \LaTeX{} compiler with a more elaborate command line.
Most conveniently this should be done as part
of a shell script or a |Makefile|.

When using \textsf{childdoc} in the main file, the following
command lines effectively perform a redirection
(note that depending on the shell being used,
backslashes may have to be doubled: `|\|' $\to$ `|\\|'):
%
\begin{center}
|... -jobname "|\textit{target}|" |\\|"|[\textit{flags}]%
|\input{childdoc.def}\childdocforward[|\textit{main}|]{|\textit{dest}|}"|
\end{center}
%
Here \textit{target} is the name of the output file,
\textit{main} is the name of the main file
and \textit{dest} is the name of the main or child file to be processed
(all filenames without extensions).
The optional argument \textit{main} can be omitted
if \textit{main} matches \textit{dest}.
Optionally, compilation \textit{flags} can be defined via |\def| commands.
This command line makes the \TeX{} engine believe
it is compiling the file \textit{target}
whose content is specified as the latter parameter.
The provided code then forwards the processing to
\textit{main} or \textit{dest} as described in \secref{sec:forward}.

%%%%%%%%%%%%%%%%%%%%%%%%%%%%%%%%%%%%%%%%%%%%%%%%%%%%%%%%%%%%%%%%%%%%%%%%%%%%%%%%
\subsection{Include by Input}
\label{sec:input}

Including child documents by |\include| has some restrictions by design.
Most notably, the content of a child document always occupies
its own set of pages; pages cannot be shared between child documents.
Usually, this behaviour makes perfect sense
because each child document contain an essential part of the document.
However, in some situations it may be desirable to compose
a document from a collection of parts
without having mandatory page breaks between then.
For this case, the package
provides a mechanism to include parts
by |\input| which can also be processed individually.
However, by construction this mechanism
requires manual handling of the content to be output.

%%%%%%%%%%%%%%%%%%%%%%%%%%%%%%%%%%%%%%%%
\DescribeMacro{\ifchilddocmanual}
The main file should be prepared as usual, see \secref{sec:include}.
However, the document body must make a distinction
between processing of an individual part and of the main document, e.g.:
%
\begin{center}
\begin{tabular}{l}
|\ifchilddocmanual|\\
|\input{\childdocname}|\\
|\||else|\\
\textit{document body with }|\input{|\textit{part}|}|\\
|\||fi|
\end{tabular}
\end{center}
%
The conditional |\ifchilddocmanual| is true whenever
a part to be included by |\input| is being compiled,
and the name of the part is stored in |\childdocname|.

%%%%%%%%%%%%%%%%%%%%%%%%%%%%%%%%%%%%%%%%
\DescribeMacro{\childdocby}
Each part to be included by |\input| should start with:
%
\begin{center}
\begin{tabular}{l}
|\input{childdoc.def}|\\
|\childdocby{|\textit{main}|}|\\
\end{tabular}
\end{center}
%
The directive |\childdocby| is similar to |\childdocof|
described in \secref{sec:include},
but the subsequent selection of content must be done manually.
To that end, both |\ifchilddoc| and |\ifchilddocmanual|
will be true upon processing of a part,
and the name of the part is stored in |\childdocname|.
Note that |\jobname| will be set to the filename of the current part
so that each part receives an individual |.aux| file
that does not interfere with the |.aux| file(s) of the main document.
This behaviour can be altered by the alternative form
|\childdocby[*]{|\textit{main}|}| (with a non-empty optional argument)
which uses the |.aux| file of the main document
by setting |\jobname| to \textit{main}.

%%%%%%%%%%%%%%%%%%%%%%%%%%%%%%%%%%%%%%%%%%%%%%%%%%%%%%%%%%%%%%%%%%%%%%%%%%%%%%%%
\subsection{Driver Development}
\label{sec:driver}

The \textsf{childdoc} mechanism can also be use for the development
of definition files such as \LaTeX{} styles or classes.
This case differs from the above setup with multiple parts
included by |\include| in that no |\includeonly| should be invoked.
This can be achieved by starting the include file
(before |\ProvidesPackage|) with:
%
\begin{center}
\begin{tabular}{l}
|\input{childdoc.def}|\\
|\childdocforward{|\textit{main}|}|\\
\end{tabular}
\end{center}
%
or alternatively with:
%
\begin{center}
\begin{tabular}{l}
|\input{childdoc.def}|\\
|\childdocby{|\textit{main}|}|\\
\end{tabular}
\end{center}
%
Both forms have slightly different effects as described above.
The main file is prepared as usual, see \secref{sec:include}.

%%%%%%%%%%%%%%%%%%%%%%%%%%%%%%%%%%%%%%%%%%%%%%%%%%%%%%%%%%%%%%%%%%%%%%%%%%%%%%%%
\subsection{Legacy Detection}
\label{sec:detection}

The directive |\childdocmain| in the main file can detect
whether the complete document or merely a child is to be compiled
even without using the directive |\childdocof|.
This method is deprecated because it is less robust
and there is no compelling reason to use it;
it is merely provided for backward compatibility
and it may be removed in future versions.

If the detection mechanism is to be used,
it is mandatory to correctly specify
the filename of the main file as the argument of |\childdocmain|:
%
\begin{center}
\begin{tabular}{l}
|\input{childdoc.def}|\\
|\childdocmain{|\textit{main}|}|\\
\end{tabular}
\end{center}
%
If |\jobname| does not match the argument \textit{main} of |\childdocmain|,
it is assumed that |\jobname| points to the child file to be compiled.
When using |\childdocmain| with the main file specified as argument,
it suffices to start a child file
with just |\input{|\textit{main}|}|
without loading of the package and using |\childdocof|.
If instead all processing is done
with the appropriate \textsf{childdoc} directives,
the argument of \textit{main} of |\childdocmain| can be empty.

An alternative version of the command line processing described
in \secref{sec:commandline} using the detection mechanism reads:
%
\begin{center}
|... -jobname "|\textit{target}|" "|[\textit{flags}]%
[|\def\jobname{|\textit{dest}|}|]|\input{|\textit{main}|}"|
\end{center}

%%%%%%%%%%%%%%%%%%%%%%%%%%%%%%%%%%%%%%%%%%%%%%%%%%%%%%%%%%%%%%%%%%%%%%%%%%%%%%%%
\subsection{Manual Code}
\label{sec:manual}

In case one cannot be certain whether the definitions file |childdoc.def|
is installed on the target \TeX{} distribution
and one prefers not to ship it,
it is conceivable to paste a few relevant commands into the sources.

To that end, drop all statements |\input{childdoc.def}|
and perform the replacements as outlined below.
Instead of |\childdocmain{|\textit{main}|}| add the following code
to the top of the main file:
%
\begin{center}
\begin{tabular}{l}
|\||ifdefined\childdocname\endinput\||fi\newif\ifchilddoc|\\
|\edef\childdocname{\scantokens\expandafter{\jobname\noexpand}}|\\
|\def\childdocmain{|\textit{main}|}\||ifx\childdocmain\childdocname\||else|\\
|\childdoctrue\includeonly{\childdocname}\let\jobname\childdocmain\||fi|\\
\end{tabular}
\end{center}
%
Instead of |\childdocof{|\textit{main}|}| just include the main file
at the top of each child file:
%
\begin{center}
|\input{|\textit{main}|}|
\end{center}
%
A simple redirection |\childdocforward{|\textit{dest}|}| is achieved by:
%
\begin{center}
|\def\jobname{|\textit{dest}|}\input{\jobname}|
\end{center}
%
The redirection with prefix
|\childdocforwardprefix[|\textit{prefix}|]{|\textit{dest}|}|
is accomplished by:
%
\begin{center}
\begin{tabular}{l}
|{\edef\jobname{\scantokens\expandafter{\jobname\noexpand}}|\\
|\def\redirectjob |\textit{prefix}|#1~~~{\gdef\jobname{|\textit{dest}|#1}}|\\
|\expandafter\redirectjob\jobname~~~}\input{\jobname}|
\end{tabular}
\end{center}

In an alternative approach,
child documents can be compiled by a specific command line
without additional code or specific definitions:
%
\begin{center}
|... -jobname "|\textit{target}|" "|[\textit{flags}]%
|\includeonly{|\textit{dest}|}\input{|\textit{main}|}"|
\end{center}
%

%%%%%%%%%%%%%%%%%%%%%%%%%%%%%%%%%%%%%%%%%%%%%%%%%%%%%%%%%%%%%%%%%%%%%%%%%%%%%%%%
%%%%%%%%%%%%%%%%%%%%%%%%%%%%%%%%%%%%%%%%%%%%%%%%%%%%%%%%%%%%%%%%%%%%%%%%%%%%%%%%
\section{Information}

%%%%%%%%%%%%%%%%%%%%%%%%%%%%%%%%%%%%%%%%%%%%%%%%%%%%%%%%%%%%%%%%%%%%%%%%%%%%%%%%
\subsection{Copyright}

Copyright \copyright{} 2017--2018 Niklas Beisert

This work may be distributed and/or modified under the
conditions of the \LaTeX{} Project Public License, either version 1.3
of this license or (at your option) any later version.
The latest version of this license is in
  \url{http://www.latex-project.org/lppl.txt}
and version 1.3 or later is part of all distributions of \LaTeX{}
version 2005/12/01 or later.

This work has the LPPL maintenance status `maintained'.

The Current Maintainer of this work is Niklas Beisert.

This work consists of the files |README.txt|, |childdoc.ins| and |childdoc.dtx|
as well as the derived files |childdoc.def|, |cdocsamp.tex|
with |cdocsch1.tex|, |cdocsch2.tex|, |cdocspt3.tex|, |cdocspt4.tex|,
|cdocsdrf.tex|, |cdocsfn1.tex|, |cdocsfn2.tex|
as well as |childdoc.pdf|.

%%%%%%%%%%%%%%%%%%%%%%%%%%%%%%%%%%%%%%%%%%%%%%%%%%%%%%%%%%%%%%%%%%%%%%%%%%%%%%%%
\subsection{Files and Installation}

The package consists of the files:
%
\begin{center}
\begin{tabular}{ll}
    |README.txt|   & readme file \\
    |childdoc.ins| & installation file \\
    |childdoc.dtx| & source file \\
    |childdoc.def| & definition file \\
    |cdocsamp.tex| & sample main file \\
    |cdocsch1.tex| & sample include file \\
    |cdocsch2.tex| & sample include file \\
    |cdocspt3.tex| & sample part file \\
    |cdocspt4.tex| & sample part file \\
    |cdocsdrf.tex| & sample redirection file \\
    |cdocsfn1.tex| & sample redirection file \\
    |cdocsfn2.tex| & sample redirection file \\
    |childdoc.pdf| & manual
\end{tabular}
\end{center}
%
The distribution consists of the files
|README.txt|, |childdoc.ins| and |childdoc.dtx|.
%
\begin{itemize}
\item
Run (pdf)\LaTeX{} on |childdoc.dtx|
to compile the manual |childdoc.pdf| (this file).
\item
Run \LaTeX{} on |childdoc.ins| to create the definitions file |childdoc.def|
and the sample |cdocsamp.tex| with include files
|cdocsch1.tex|, |cdocsch2.tex|, |cdocspt3.tex|, |cdocspt4.tex|,
|cdocsdrf.tex|, |cdocsfn1.tex|, |cdocsfn2.tex|.
Then copy the file |childdoc.def| to an appropriate directory of your \LaTeX{}
distribution, e.g.\ \textit{texmf-root}|/tex/latex/childdoc|.
\end{itemize}

%%%%%%%%%%%%%%%%%%%%%%%%%%%%%%%%%%%%%%%%%%%%%%%%%%%%%%%%%%%%%%%%%%%%%%%%%%%%%%%%
\subsection{Related CTAN Packages}

There are several other packages which offer a similar functionality:
%
\begin{itemize}
\item
The packages
\href{http://ctan.org/pkg/docmute}{\textsf{docmute}},
\href{http://ctan.org/pkg/includex}{\textsf{includex}} and
\href{http://ctan.org/pkg/standalone}{\textsf{standalone}}
provide commands to include only the document body of
a child file thus allowing both files to be compiled individually.
\item
The packages \href{http://ctan.org/pkg/subdocs}{\textsf{subdocs}}
and \href{http://ctan.org/pkg/subfiles}{\textsf{subfiles}}
provide structures in which the main and child documents can be
encapsulated and allowing them to be compiled individually.
The inclusion mechanism is different from the conventional |\include|.
\item
The package \href{http://ctan.org/pkg/combine}{\textsf{combine}}
is an elaborate solution to combine several documents into one.
\end{itemize}
%
See also the CTAN topic \href{http://ctan.org/topic/subdocs}{\textsf{subdocs}}
for further related packages.
The present package differs from the above solutions in that
a document structure constructed with the conventional |\include| mechanism
just needs two extra commands at the top of every file
such that all constituent files can be compiled individually.

%%%%%%%%%%%%%%%%%%%%%%%%%%%%%%%%%%%%%%%%%%%%%%%%%%%%%%%%%%%%%%%%%%%%%%%%%%%%%%%%
%\subsection{Feature Suggestions}
%
%The following is a list of features which may be useful for future
%versions of this package:
%%
%\begin{itemize}
%\item
%\ldots
%\end{itemize}

%%%%%%%%%%%%%%%%%%%%%%%%%%%%%%%%%%%%%%%%%%%%%%%%%%%%%%%%%%%%%%%%%%%%%%%%%%%%%%%%
\subsection{Revision History}

%%%%%%%%%%%%%%%%%%%%%%%%%%%%%%%%%%%%%%%%
\paragraph{v2.0:} 2018/12/30

\begin{itemize}
\item
immediate forward processing
\item
added |\childdocby| mechanism
\item
manual restructured
\end{itemize}

%%%%%%%%%%%%%%%%%%%%%%%%%%%%%%%%%%%%%%%%
\paragraph{v1.6:} 2018/01/17

\begin{itemize}
\item
application for development of include files
\item
corrections to manual
\end{itemize}

%%%%%%%%%%%%%%%%%%%%%%%%%%%%%%%%%%%%%%%%
\paragraph{v1.5:} 2017/05/21

\begin{itemize}
\item
more complete structuring introduced
\item
|\childdocof| introduced
\item
|\childdoc| renamed to |\childdocmain|
\item
|\childredirect| renamed to |\childdocforward| and |\childdocforwardprefix|
and functionality expanded
\end{itemize}

%%%%%%%%%%%%%%%%%%%%%%%%%%%%%%%%%%%%%%%%
\paragraph{v1.0:} 2017/04/27

\begin{itemize}
\item
manual and install package
\item
first version published on CTAN
\end{itemize}

%%%%%%%%%%%%%%%%%%%%%%%%%%%%%%%%%%%%%%%%
\paragraph{v0.6:} 2017/04/26

\begin{itemize}
\item
redirection mechanism added
\end{itemize}

%%%%%%%%%%%%%%%%%%%%%%%%%%%%%%%%%%%%%%%%
\paragraph{v0.5:} 2017/04/26

\begin{itemize}
\item
functionality in definition file
\end{itemize}


%%%%%%%%%%%%%%%%%%%%%%%%%%%%%%%%%%%%%%%%%%%%%%%%%%%%%%%%%%%%%%%%%%%%%%%%%%%%%%%%
%%%%%%%%%%%%%%%%%%%%%%%%%%%%%%%%%%%%%%%%%%%%%%%%%%%%%%%%%%%%%%%%%%%%%%%%%%%%%%%%
%%%%%%%%%%%%%%%%%%%%%%%%%%%%%%%%%%%%%%%%%%%%%%%%%%%%%%%%%%%%%%%%%%%%%%%%%%%%%%%%
\appendix

\settowidth\MacroIndent{\rmfamily\scriptsize 000\ }

 \DocInput{childdoc.dtx}

\end{document}
%</driver>
% \fi
%
% %%%%%%%%%%%%%%%%%%%%%%%%%%%%%%%%%%%%%%%%%%%%%%%%%%%%%%%%%%%%%%%%%%%%%%%%%%%%%%
% %%%%%%%%%%%%%%%%%%%%%%%%%%%%%%%%%%%%%%%%%%%%%%%%%%%%%%%%%%%%%%%%%%%%%%%%%%%%%%
% \section{Sample}
%\iffalse
%<*samplemain>
%\fi
%
% The following presents a sample document
% with two chapters, two parts, a title page,
% a compile flag as well as three forwarding files to set the flag.
% It consists of eight |.tex| files:
% \begin{center}
% \begin{tabular}{ll}
% |cdocsamp.tex|&main file\\
% |cdocsch1.tex|&include file for chapter 1\\
% |cdocsch2.tex|&include file for chapter 2\\
% |cdocspt3.tex|&include file for part 3\\
% |cdocspt4.tex|&include file for part 4\\
% |cdocsdrf.tex|&forwarding file for main file in draft mode\\
% |cdocsfi1.tex|&forwarding file for final version of chapter 1\\
% |cdocsfi2.tex|&forwarding file for final version of chapter 2\\
% \end{tabular}
% \end{center}
% Each of the eight files can be compiled directly by the \LaTeX{} compiler.
%
% %%%%%%%%%%%%%%%%%%%%%%%%%%%%%%%%%%%%%%
% \paragraph{Main File.}
%
% The main file is called |cdocsamp.tex|.
%
% Load the \textsf{childdoc} definitions and
% declare the filename for the main document:
%    \begin{macrocode}
\input{childdoc.def}
\childdocmain{}
%    \end{macrocode}

% Optional override for |\version| flag:
%    \begin{macrocode}
%%\ifchilddoc\else\providecommand{\version}{draft}\fi
%    \end{macrocode}

% Define the default values for the |\version| flag
% (|final| for the main file and |draft| for childs):
%    \begin{macrocode}
\ifchilddoc
\providecommand{\version}{draft}
\else
\providecommand{\version}{final}
\fi
%    \end{macrocode}

% Load the standard document class:
%    \begin{macrocode}
\documentclass[12pt]{article}
%    \end{macrocode}

% Start the document body:
%    \begin{macrocode}
\begin{document}
%    \end{macrocode}

% Declare a title page.
% Print title, part of document being processed and version flag:
%    \begin{macrocode}
\addtocounter{page}{-1}
\begin{center}
{\LARGE\bfseries{}childdoc example\par}
\vspace{1cm}
\ifchilddoc
\ifchilddocmanual part\else chapter\fi:
`\childdocname' of `\childdocjob'\par
\else
main document: `\childdocjob'\par
\fi
version: \version\par
\end{center}
\newpage
%    \end{macrocode}

% Manually include selected file,
% otherwise process as usual:
%    \begin{macrocode}
\ifchilddocmanual
\section*{part `\childdocname'}
\input{\childdocname}
\else
%    \end{macrocode}

% Include the two chapters:
%    \begin{macrocode}
\include{cdocsch1}
\include{cdocsch2}
%    \end{macrocode}

% Include the two parts unless only chapters should be displayed:
%    \begin{macrocode}
\ifchilddoc\else
\section{part three}
\input{cdocspt3}
\section{part four}
\input{cdocspt4}
\fi
%    \end{macrocode}

% Process as usual until here:
%    \begin{macrocode}
\fi
%    \end{macrocode}

% End of document body:
%    \begin{macrocode}
\end{document}
%    \end{macrocode}
%\iffalse
%</samplemain>
%\fi
%
% %%%%%%%%%%%%%%%%%%%%%%%%%%%%%%%%%%%%%%
% \paragraph{Chapter Include Files.}
%
% The include files are called |cdocsch1.tex| and |cdocsch2.tex|.
%
%\iffalse
%<*samplechap1|samplechap2>
%\fi

% Optional override for |\version| flag:
%    \begin{macrocode}
%%\providecommand{\version}{final}
%    \end{macrocode}

% Include the main document:
%    \begin{macrocode}
\input{childdoc.def}
\childdocof{cdocsamp}
%    \end{macrocode}

%\iffalse
%</samplechap1|samplechap2>
%\fi
%
%\iffalse
%<*samplechap1>
%\fi
% Some text for chapter 1:
%    \begin{macrocode}
\section{one}
some text in chapter one
%    \end{macrocode}

%\iffalse
%</samplechap1>
%\fi
% Some text for chapter 2:
%\iffalse
%<*samplechap2>
%\fi
%    \begin{macrocode}
\section{two}
more text in chapter two
%    \end{macrocode}

%\iffalse
%</samplechap2>
%\fi
%
% %%%%%%%%%%%%%%%%%%%%%%%%%%%%%%%%%%%%%%
% \paragraph{Part Include Files.}
%
% The include files are called |cdocspt3.tex| and |cdocspt4.tex|.
%
%\iffalse
%<*samplepart3|samplepart4>
%\fi

% Optional override for |\version| flag:
%    \begin{macrocode}
%%\providecommand{\version}{final}
%    \end{macrocode}

% Include the main document:
%    \begin{macrocode}
\input{childdoc.def}
\childdocby{cdocsamp}
%    \end{macrocode}

%\iffalse
%</samplepart3|samplepart4>
%\fi
%
%\iffalse
%<*samplepart3>
%\fi
% Some text for part 3:
%    \begin{macrocode}
some text in part three
%    \end{macrocode}

%\iffalse
%</samplepart3>
%\fi
% Some text for part 4:
%\iffalse
%<*samplepart4>
%\fi
%    \begin{macrocode}
more text in part four
%    \end{macrocode}

%\iffalse
%</samplepart4>
%\fi
%
% %%%%%%%%%%%%%%%%%%%%%%%%%%%%%%%%%%%%%%
% \paragraph{Forwarding for a Complete Draft.}
%
% The following forwarding file |cdocsdrf.tex|
% compiles the main document in draft mode:
%\iffalse
%<*sampledraft>
%\fi
%    \begin{macrocode}
\def\version{draft}
\input{childdoc.def}
\childdocforward{cdocsamp}
%    \end{macrocode}

%\iffalse
%</sampledraft>
%\fi
%
% %%%%%%%%%%%%%%%%%%%%%%%%%%%%%%%%%%%%%%
% \paragraph{Forwarding for Final Version of the Chapters.}
%
% The following forwarding files |cdocsfn1.tex| and |cdocsfn2.tex|
% (with identical content)
% compile the final versions of the child documents
% |cdocsch1.tex| and |cdocsch2.tex|, respectively:
%\iffalse
%<*samplefinal>
%\fi
%    \begin{macrocode}
\def\version{final}
\input{childdoc.def}
\childdocforwardprefix[cdocsamp]{cdocsfn}{cdocsch}
%    \end{macrocode}

%\iffalse
%</samplefinal>
%\fi
%
% %%%%%%%%%%%%%%%%%%%%%%%%%%%%%%%%%%%%%%
% \paragraph{Command Line Processing.}
%
% The following three command lines generate the output files
% |cdocscld|, |cdocscl1| and |cdocscl2|
% which should be identical to
% |cdocsdrf|, |cdocsch1| and |cdocsfn2|, respectively:
% \begin{center}
% \begin{tabular}{l}
% |latex -jobname cdocscld \|\\
% |  "\def\version{draft}\input{childdoc.def}\childdocforward{cdocsamp}"|\\
% |latex -jobname cdocscl1 \|\\
% |  "\input{childdoc.def}\childdocforward[cdocsamp]{cdocsch1}"|\\
% |latex -jobname cdocscl2 \|\\
% |  "\def\version{final}\input{childdoc.def}\childdocforward{cdocsch2}"|
% \end{tabular}
% \end{center}
% Note that the trailing backslash on each first line
% merely continues the input to the second line
% (for convenient cut ant paste).
% Furthermore, the command |latex| can be replaced by any
% of its alternative versions such as |pdflatex|.
%
% %%%%%%%%%%%%%%%%%%%%%%%%%%%%%%%%%%%%%%%%%%%%%%%%%%%%%%%%%%%%%%%%%%%%%%%%%%%%%%
% %%%%%%%%%%%%%%%%%%%%%%%%%%%%%%%%%%%%%%%%%%%%%%%%%%%%%%%%%%%%%%%%%%%%%%%%%%%%%%
% \section{Implementation}
%\iffalse
%<*package>
%\fi
%
% This section describes the definitions file |childdoc.def|.

% The definitions cannot be loaded using |\usepackage| or |\RequirePackage|
% which has a mechanism to prevent loading a style file more than once.
% When loading the definitions by means of |\input|
% multiple instances have to be prevented manually:
%\iffalse
%This code needs to be before the `\ProvidesFile' directive
%which is defined at the beginning of this file.
%Therefore it is also placed there and commented out here.
%</package>
%<*discard>
%\fi
%    \begin{macrocode}
\ifdefined\childdocmain\endinput\fi
%    \end{macrocode}
%\iffalse
%</discard>
%<*package>
%\fi
%
% \macro{\ifchilddoc}
% \macro{\ifchilddocmanual}
% The conditional |\ifchilddoc| tells whether a
% child (true) or main (false) document is being compiled.
% The conditional |\ifchilddocmanual| tells whether
% the |\includeonly| mechanism is used (false) or
% the selection of child files must be performed manually (true).
% The definitions initialise to false:
%    \begin{macrocode}
\newif\ifchilddoc
\newif\ifchilddocmanual
%    \end{macrocode}

% \macro{\childdocname}
% \macro{\childdocjob}
% The macro |\childdocname| stores the name of the main document
% to be compiled. The macro |\childdocjob| stores the name of
% the document on which the \LaTeX{} compiler was originally invoked.
% The content of |\jobname| cannot be compared
% to filenames specified in the source due to different catcodes.
% The following code rescans |\jobname|, stores the result
% in |\childdocname| and saves a copy in |\childdocjob|:
%    \begin{macrocode}
\edef\childdocname{\scantokens\expandafter{\jobname\noexpand}}
\let\childdocjob\childdocname
%    \end{macrocode}

% \macro{\childdocdisable}
% The macro |\childdocdisable| prevents the main file
% from being processed more than once.
% At this stage, the main document command |\childdocmain|
% is assumed to be called once again where it should do nothing.
% Any subsequent call to it should prevent
% a secondary processing of the main document
% It overwrites the forwarding commands
% |\childdocof| and |\childdocforward|
% with empty macros to prevent further inclusions of the main document:
%    \begin{macrocode}
\newcommand{\childdocdisable}
{
  \renewcommand{\childdocmain}[1]{\renewcommand{\childdocmain}[1]{\endinput}}
  \renewcommand{\childdocof}[1]{}
  \renewcommand{\childdocby}[2][]{}
  \renewcommand{\childdocforward}[2][]{}
  \renewcommand{\childdocdisable}{}
}
%    \end{macrocode}

% \macro{\childdocmain}
% The macro |\childdocmain| is to be called at the top of the main file
% with nothing or the main filename (without extension) as argument.
% First, it breaks loops.
% If the argument is not empty and does not match |\childdocname|
% (which is set by the first inclusion of |childdoc.def|),
% |\ifchilddoc| is set to true, |\includeonly| is applied to the child file
% and |\jobname| is set to the main file
% (for proper handling of |.aux| files):
%    \begin{macrocode}
\newcommand{\childdocmain}[1]
{
  \childdocdisable\childdocmain{}
  \if?#1?\else
    \begingroup
      \def\childdoctmp{#1}
      \ifx\childdoctmp\childdocname
        \def\childdoctmp{}
      \else
        \def\childdoctmp
        {
          \childdoctrue
          \includeonly{\childdocname}
          \def\childdocjob{#1}
          \def\jobname{#1}
        }
      \fi
      \expandafter
    \endgroup
    \childdoctmp
  \fi
}
%    \end{macrocode}

% \macro{\childdocof}
% The command |\childdocof| redirects
% compilation to the main file |#1|.
%    \begin{macrocode}
\newcommand{\childdocof}[1]
{
  \childdocdisable
  \childdoctrue
  \includeonly{\childdocname}
  \def\jobname{#1}
  \def\childdocjob{#1}
  \input{#1}
}
%    \end{macrocode}

% \macro{\childdocby}
% The command |\childdocby| ....
%    \begin{macrocode}
\newcommand{\childdocby}[2][]
{
  \childdocdisable
  \childdoctrue
  \childdocmanualtrue
  \if?#1?\else
    \def\jobname{#2}
  \fi
  \def\childdocjob{#2}
  \input{#2}
  \endinput
}
%    \end{macrocode}

% \macro{\childdocforward}
% The command |\childdocforward| redirects
% compilation to the main file or
% (if the optional argument is given) a child file.
% Parameters are set as if the main file
% or a child file starting with |\childdocof| was compiled.
% Then compilation is handed over to the main file:
%    \begin{macrocode}
\newcommand{\childdocforward}[2][]
{
  \begingroup
    \if?#1?
      \def\childdoctmp
      {
        \def\childdocname{#2}
        \def\childdocjob{#2}
        \def\jobname{#2}
        \input{#2}
        \endinput
      }
    \else
      \def\childdoctmp
      {
        \childdocdisable
        \def\childdocname{#2}
        \childdoctrue
        \includeonly{#2}
        \def\childdocjob{#1}
        \def\jobname{#1}
        \input{#1}
        \endinput
      }
    \fi
    \expandafter
  \endgroup
  \childdoctmp
}
%    \end{macrocode}

% \macro{\childdocforwardprefix}
% The command |\childdocforwardprefix| redirects
% compilation to the main or a child file by means of a pattern.
% The prefix |#1| in the current filename is replaced by |#2|
% and the suffix of the current filename is kept
% (it is assumed that the filename does not contain the substring `|~~~|'
% which is used as a delimiter).
% Compilation is handed over to the new file by |\childdocforward|:
%    \begin{macrocode}
\newcommand{\childdocforwardprefix}[3][]
{
  \begingroup
    \def\childdocextract #2##1~~~{\def\childdoctmp{\childdocforward[#1]{#3##1}}}
    \expandafter\childdocextract\childdocname~~~
    \expandafter
  \endgroup
  \childdoctmp
}
%    \end{macrocode}

% \macro{\childdoc}
% The deprecated macro |\childdoc| is a legacy version of |\childdocmain|:
%    \begin{macrocode}
\newcommand{\childdoc}{\childdocmain}
%    \end{macrocode}

% \macro{\childdocredirect}
% The deprecated macro |\childdocredirect| is a legacy version
% of |\childdocforward| and |\childdocforwardprefix|:
%    \begin{macrocode}
\newcommand{\childdocredirect}[2][]
{
  \begingroup
    \if?#1?
      \def\childdoctmp{\childdocforward{#2}}
    \else
      \def\childdoctmp{\childdocforwardprefix{#1}{#2}}
    \fi
    \expandafter
  \endgroup
  \childdoctmp
}
%    \end{macrocode}

%\iffalse
%</package>
%\fi
%
\endinput
|\\
|\childdocforward[|\textit{main}|]{|\textit{dest}|}|\\
\end{tabular}
\end{center}
%
The argument \textit{dest} is the destination file
(without extension).
It should be the main file or one of the child files.
Note that further \textsf{childdoc} directives
such as |\childdocof| and |\childdocforward|
in the indicated file will be processed in this form.
The optional argument \textit{main}
passes on directly to the main file \textit{main}
while pretending to compile the child \textit{dest}.
This form behaves as if \textit{dest}
issues |\childdocof{|\textit{main}|}| right away,
and no further \textsf{childdoc} directives will be processed.

%%%%%%%%%%%%%%%%%%%%%%%%%%%%%%%%%%%%%%%%
\DescribeMacro{\...prefix}
In the alternative form |\childdocforwardprefix|,
%
\begin{center}
\begin{tabular}{l}
|% \iffalse
%
% childdoc.dtx Copyright (C) 2017-2018 Niklas Beisert
%
% This work may be distributed and/or modified under the
% conditions of the LaTeX Project Public License, either version 1.3
% of this license or (at your option) any later version.
% The latest version of this license is in
%   http://www.latex-project.org/lppl.txt
% and version 1.3 or later is part of all distributions of LaTeX
% version 2005/12/01 or later.
%
% This work has the LPPL maintenance status `maintained'.
%
% The Current Maintainer of this work is Niklas Beisert.
%
% This work consists of the files childdoc.dtx and childdoc.ins
% and the derived files childdoc.def and cdocsamp.tex with
% cdocsch1.tex, cdocsch2.tex, cdocsdrf.tex, cdocsfn1.tex, cdocsfn2.tex.
%
%<package>\ifdefined\childdocmain\endinput\fi
%<package>\ProvidesFile{childdoc.def}[2018/12/30 v2.0 child document driver]
%<samplemain>\ProvidesFile{cdocsamp.tex}[2018/12/30 v2.0 sample for childdoc]
%<*driver>
%\ProvidesFile{childdoc.drv}[2018/12/30 v2.0 childdoc reference manual file]
\PassOptionsToClass{10pt,a4paper}{article}
\documentclass{ltxdoc}

\usepackage[margin=35mm]{geometry}
\usepackage{hyperref}
\usepackage{hyperxmp}
\usepackage[usenames]{color}

\hypersetup{colorlinks=true}
\hypersetup{pdfstartview=FitH}
\hypersetup{pdfpagemode=UseNone}
\hypersetup{pdfsource={}}
\hypersetup{pdflang={en-UK}}
\hypersetup{pdfcopyright={Copyright 2017-2018 Niklas Beisert.
  This work may be distributed and/or modified under the
  conditions of the LaTeX Project Public License, either version 1.3
  of this license or (at your option) any later version.}}
\hypersetup{pdflicenseurl={http://www.latex-project.org/lppl.txt}}
\hypersetup{pdfcontactaddress={ETH Zurich, ITP, HIT K,
  Wolfgang-Pauli-Strasse 27}}
\hypersetup{pdfcontactpostcode={8093}}
\hypersetup{pdfcontactcity={Zurich}}
\hypersetup{pdfcontactcountry={Switzerland}}
\hypersetup{pdfcontactemail={nbeisert@itp.phys.ethz.ch}}
\hypersetup{pdfcontacturl={http://people.phys.ethz.ch/\xmptilde nbeisert/}}

\newcommand{\secref}[1]{\hyperref[#1]{section \ref*{#1}}}

\parskip1ex
\parindent0pt
\let\olditemize\itemize
\def\itemize{\olditemize\parskip0pt}

\begin{document}

\title{The \textsf{childdoc} Package}
\hypersetup{pdftitle={The childdoc Package}}
\author{Niklas Beisert\\[2ex]
  Institut f\"ur Theoretische Physik\\
  Eidgen\"ossische Technische Hochschule Z\"urich\\
  Wolfgang-Pauli-Strasse 27, 8093 Z\"urich, Switzerland\\[1ex]
  \href{mailto:nbeisert@itp.phys.ethz.ch}
  {\texttt{nbeisert@itp.phys.ethz.ch}}}
\hypersetup{pdfauthor={Niklas Beisert}}
\hypersetup{pdfsubject={Manual for the LaTeX2e Package childdoc}}
\date{30 December 2018, \textsf{v2.0}}
\maketitle

\begin{abstract}\noindent
\textsf{childdoc} is a \LaTeXe{} package
that enables the direct compilation
of document sections included by |\include|
to individual files.
\end{abstract}

\begingroup
\parskip0ex
\tableofcontents
\endgroup

%%%%%%%%%%%%%%%%%%%%%%%%%%%%%%%%%%%%%%%%%%%%%%%%%%%%%%%%%%%%%%%%%%%%%%%%%%%%%%%%
%%%%%%%%%%%%%%%%%%%%%%%%%%%%%%%%%%%%%%%%%%%%%%%%%%%%%%%%%%%%%%%%%%%%%%%%%%%%%%%%
\section{Introduction}

\LaTeX{} provides a mechanism to structure a large document (such as a book)
into a main file and several child files (containing the chapters)
using the |\include| command.
This mechanism is beneficial for documents
which span hundreds of pages in order to
make the source file(s) more manageable.
Moreover, compilation can be restricted to
selected child files by means of the |\includeonly| command.
The latter feature can be used to reduce the compilation time while editing
(this was significantly more useful in the earlier days of \LaTeX{})
or to generate a smaller document which is easier to navigate.
Another application of |\includeonly| is to generate
documents consisting of selected parts of the complete document.

However, there are a few drawbacks of the plain |\include| mechanism:
\begin{itemize}
\item
The child files cannot be compiled on their own,
they can only be compiled via the main file.
A naive editing environment
(such as a text editor with an option
to have the current file processed by \LaTeX)
may require one to switch to the main file before compiling;
attempting to compile the child file produces errors.
\item
The main file must be modified (each time)
to adjust the |\includeonly| command
to the present needs. This easily leaves the main file in a messy state.
\item
The generated document will always carry the filename
of the main document. This is inconvenient if
several child files are to be compiled and
to be kept for distribution.
\end{itemize}

The present package provides a simple interface
to make child files individually compilable by \LaTeX{}.
Compiling a child file then has the same effect as compiling
the main file with an |\includeonly| command
to select the appropriate child.
Moreover the generated document will carry the name of the child
rather than the main file.
This resolves all three above issues.

This feature is meant to make the editing of books,
thesis documents and lecture notes somewhat more convenient.
However, the package can also be used efficiently for
composing a series of documents (such as exercise sheets)
which are typically distributed individually.
It then assists the author in generating the individual documents
(potentially in different versions)
as well as a document containing the collected series.
Another application is in developing style files
or other kinds of included material
where compilation of the style file could redirect
to a sample or test file.

%%%%%%%%%%%%%%%%%%%%%%%%%%%%%%%%%%%%%%%%%%%%%%%%%%%%%%%%%%%%%%%%%%%%%%%%%%%%%%%%
%%%%%%%%%%%%%%%%%%%%%%%%%%%%%%%%%%%%%%%%%%%%%%%%%%%%%%%%%%%%%%%%%%%%%%%%%%%%%%%%
\section{Usage}

First of all, the package \textsf{childdoc} is \emph{not} a standard
\LaTeXe{} |.sty| style file! Therefore it needs to be invoked in
a non-standard way.

%%%%%%%%%%%%%%%%%%%%%%%%%%%%%%%%%%%%%%%%%%%%%%%%%%%%%%%%%%%%%%%%%%%%%%%%%%%%%%%%
\subsection{Included Files}
\label{sec:include}

%%%%%%%%%%%%%%%%%%%%%%%%%%%%%%%%%%%%%%%%
\DescribeMacro{\childdocmain}
To use the package, add the commands
\begin{center}
\begin{tabular}{l}
|\input{childdoc.def}|\\
|\childdocmain{}|\\
\end{tabular}
\end{center}
at the very top of the main \LaTeX{} file,
in particular \emph{before} the |\documentclass| statement!
The argument of |\childdocmain| should be left empty
(but it must be present).

%%%%%%%%%%%%%%%%%%%%%%%%%%%%%%%%%%%%%%%%
\DescribeMacro{\childdocof}
Furthermore, add the commands
\begin{center}
\begin{tabular}{l}
|\input{childdoc.def}|\\
|\childdocof{|\textit{main}|}|\\
\end{tabular}
\end{center}
at the top of every child file \textit{child}
which is included by |\include{|\textit{child}|}|
from within the main file
(or at least for those files to be compiled individually).
The argument \textit{main} must be the filename of the main file.

There are a couple of
considerations in setting up the main and child documents:

%%%%%%%%%%%%%%%%%%%%%%%%%%%%%%%%%%%%%%%%
\paragraph{Restrictions.}

Please note the following restrictions:
\begin{itemize}
\item
|\childdocmain| must be called with one argument \textit{main}
to ensure compatibility with earlier version of the package.
It must either be empty (|\childdocmain{}|)
or precisely match the filename of the main file in which it is specified.
See \secref{sec:detection} for further information.
\item
The filename \textit{main} must be specified without the |.tex| extension.
\item
The filename \textit{main} is case sensitive
(even in case-insensitive file systems)
due to internal string comparison.
\item
The argument \textit{main} should be fully expanded, it cannot be a macro.
\item
Subdirectories and special characters should be avoided in filenames.
\item
The command |\childdocmain{|\textit{main}|}| must be followed by a whitespace.
It should not be followed immediately by another command
or by a comment mark `|%|'.
This is because the \TeX{} parser reads the token immediately following
the argument of |\childdocmain| and puts it
at the beginning of every child section;
however, a white\-space is ignored.
\end{itemize}

%%%%%%%%%%%%%%%%%%%%%%%%%%%%%%%%%%%%%%%%
\paragraph{Content of Main File.}

It is advisable to place all content in the child files included by |\include|.
Any output contained in the main file will appear in all child documents
unless suppressed manually;
it cannot be suppressed automatically by the |\includeonly| directive
and thus should normally be avoided.
A method to include some content in the main file
by means of conditional processing is described in \secref{sec:conditional}.

%%%%%%%%%%%%%%%%%%%%%%%%%%%%%%%%%%%%%%%%
\paragraph{Page Numbering.}

When only a part of the document is compiled,
the appropriate numbering of pages
(as well as other status parameters)
is determined from the |.aux| files.
The latter contain information from previous passes.
However this information needs to propagate through
all intermediate child documents.
Therefore the page numbering in child documents may well
be inconsistent until the complete document is compiled at least once.

A useful (if unconventional) way to always ensure a consistent
page numbering is to restart the numbering in each child document
and denote the pages by `\textit{child}|.|\textit{page}'
where \textit{child} represents the chapter/section number of the child file.
This can be achieved by the command
|\numberwithin{page}{|\textit{child}|}|
of the \textsf{amsmath} package
where \textit{child} can be |chapter| or |section|
depending on the chosen structuring.
Alternatively, one can modify the macro |\thepage| appropriately
and reset the counter |page| at the start of each child file.

%%%%%%%%%%%%%%%%%%%%%%%%%%%%%%%%%%%%%%%%%%%%%%%%%%%%%%%%%%%%%%%%%%%%%%%%%%%%%%%%
\subsection{Conditional Processing}
\label{sec:conditional}

The package provides a mechanism to compile different versions
of a document. To customise the versions further some conditional processing
can come in handy to distinguish which version is being compiled.
The package provides two macros to describe the compilation context:

%%%%%%%%%%%%%%%%%%%%%%%%%%%%%%%%%%%%%%%%
\DescribeMacro{\ifchilddoc}
The conditional |\ifchilddoc| distinguishes between the compilation of
child documents and the main document:
%
\begin{center}
|\ifchilddoc |\textit{child-code}| |[|\||else |\textit{main-code}]| \||fi|
\end{center}

%%%%%%%%%%%%%%%%%%%%%%%%%%%%%%%%%%%%%%%%
\DescribeMacro{\childdocname}
\DescribeMacro{\childdocjob}
The macro |\childdocname| contains the filename (without extension)
of the main or child file being processed.
Note that |\childdocjob| will always contain the name of the main file.

%%%%%%%%%%%%%%%%%%%%%%%%%%%%%%%%%%%%%%%%
\paragraph{Title Page.}

Conditional processing can be used to include a title or banner page
in the main document when proper precautions are taken.
Importantly, the code in the main file should ensure that the page counter
(as well as other status parameters which are stored in the |.aux| files)
takes the same value after the conditional processing.
Otherwise the page numbers may take divergent values
depending on which part is compiled.

For example, a title page could be declared by:
%
\begin{center}
\begin{tabular}{l}
|\ifchilddoc\||else|\\
|\addtocounter{page}{-1}|\\
\textit{code for title page}\\
|\newpage|\\
|\||fi|
\end{tabular}
\end{center}
%
A banner page for the child documents can be generated by:
%
\begin{center}
\begin{tabular}{l}
|\ifchilddoc|\\
|\addtocounter{page}{-1}|\\
\textit{code for banner page}\\
|\newpage|\\
|\||fi|
\end{tabular}
\end{center}
%
Here one could write a message such as:
\begin{center}
|This is the part \childdocname{} of \childdocjob{}.|
\end{center}

%%%%%%%%%%%%%%%%%%%%%%%%%%%%%%%%%%%%%%%%%%%%%%%%%%%%%%%%%%%%%%%%%%%%%%%%%%%%%%%%
\subsection{Flags}
\label{sec:flags}

The package makes it easy to generate different versions
of the main or child documents.
To this end compilation flags can be defined
and assigned different default values.
They will be particularly useful in conjunction
with the forwarding mechanism described in \secref{sec:forward}.

For example, it may be useful to have a flag |\version|
which can be set to |draft| or |final|.
The document source will contain some conditional code
depending on the value of |\version|.
Suppose further, the flag should default to |final| for the main file
and to |draft| for child files
which is a natural assignment for editing the document.
This is achieved by placing the following code
in the preamble of the main document
(below the |\childdocmain| directive):
%
\begin{center}
\begin{tabular}{l}
|\ifchilddoc|\\
|\providecommand{\version}{draft}|\\
|\||else|\\
|\providecommand{\version}{final}|\\
|\||fi|
\end{tabular}
\end{center}
%
The definition by |\providecommand| makes sure
that previous definitions are not overwritten.
Further statements |\providecommand{\version}{...}|
can thus be added before the above code to override it.

For the main file, one might add a line
(between |\childdocmain| and the above block)
%
\begin{center}
|%\ifchilddoc\||else\providecommand{\version}{draft}\||fi|
\end{center}
%
which can be uncommented to produce a draft version.
Likewise one can add a line to the very top of a child file
(above the |\childdocof{|\textit{main}|}| directive)
%
\begin{center}
|%\providecommand{\version}{final}|
\end{center}
%
which can be uncommented to produce the final version of this child document.

%%%%%%%%%%%%%%%%%%%%%%%%%%%%%%%%%%%%%%%%%%%%%%%%%%%%%%%%%%%%%%%%%%%%%%%%%%%%%%%%
\subsection{Forwarding}
\label{sec:forward}

Different versions of the main or child documents
using compilation flags as described in \secref{sec:flags}
can be (permanently) stored in different files
for convenient compilation, viewing and distribution.
To this end, the package defines a command
to pass on compilation to a different file:

%%%%%%%%%%%%%%%%%%%%%%%%%%%%%%%%%%%%%%%%
\DescribeMacro{\childdocforward}
The command |\childdocforward| redirects processing to
another source file:
%
\begin{center}
\begin{tabular}{l}
|\input{childdoc.def}|\\
|\childdocforward[|\textit{main}|]{|\textit{dest}|}|\\
\end{tabular}
\end{center}
%
The argument \textit{dest} is the destination file
(without extension).
It should be the main file or one of the child files.
Note that further \textsf{childdoc} directives
such as |\childdocof| and |\childdocforward|
in the indicated file will be processed in this form.
The optional argument \textit{main}
passes on directly to the main file \textit{main}
while pretending to compile the child \textit{dest}.
This form behaves as if \textit{dest}
issues |\childdocof{|\textit{main}|}| right away,
and no further \textsf{childdoc} directives will be processed.

%%%%%%%%%%%%%%%%%%%%%%%%%%%%%%%%%%%%%%%%
\DescribeMacro{\...prefix}
In the alternative form |\childdocforwardprefix|,
%
\begin{center}
\begin{tabular}{l}
|\input{childdoc.def}|\\
|\childdocforwardprefix[|\textit{main}|]{|\textit{prefix}|}{|\textit{dest}|}|
\end{tabular}
\end{center}
%
the destination file is determined by a pattern
depending on the current file:
To make this work, the current file must be called
`{\textit{prefix}\hspace{0.2em}\textit{suffix}}'
with \textit{prefix} matching precisely the argument.
Processing is then passed on to the file
`{\textit{dest}\hspace{0.2em}\textit{suffix}}'.
Surely, the same effect is achieved by
directly specifying the
argument `{\textit{dest}\hspace{0.2em}\textit{suffix}}'
in the first form.
However, that requires to set up a different file
for each child. With the alternative form of the command
all these files can have exactly the same content
which simplifies setting them up and maintaining them.

For example, the following file |draft.tex|
with a compilation flag |\version| as described in \secref{sec:flags}
compiles the main document as a draft:
%
\begin{center}
\begin{tabular}{l}
|\def\version{draft}|\\
|\input{childdoc.def}|\\
|\childdocforward{|\textit{main}|}|
\end{tabular}
\end{center}
%
Likewise, the following files |final|\textit{nn}|.tex|
compile the final version of the child document
|child|\textit{nn}|.tex|:
%
\begin{center}
\begin{tabular}{l}
|\def\version{final}|\\
|\input{childdoc.def}|\\
|\childdocforwardprefix{final}{child}|
\end{tabular}
\end{center}
%

Note that when several versions of a main file and/or of each child file
are to be generated, it may be convenient to set up a |Makefile| or
shell script to automatise the process.

%%%%%%%%%%%%%%%%%%%%%%%%%%%%%%%%%%%%%%%%%%%%%%%%%%%%%%%%%%%%%%%%%%%%%%%%%%%%%%%%
\subsection{Command Line Processing}
\label{sec:commandline}

The effect of redirection files can also be achieved by invoking
the \LaTeX{} compiler with a more elaborate command line.
Most conveniently this should be done as part
of a shell script or a |Makefile|.

When using \textsf{childdoc} in the main file, the following
command lines effectively perform a redirection
(note that depending on the shell being used,
backslashes may have to be doubled: `|\|' $\to$ `|\\|'):
%
\begin{center}
|... -jobname "|\textit{target}|" |\\|"|[\textit{flags}]%
|\input{childdoc.def}\childdocforward[|\textit{main}|]{|\textit{dest}|}"|
\end{center}
%
Here \textit{target} is the name of the output file,
\textit{main} is the name of the main file
and \textit{dest} is the name of the main or child file to be processed
(all filenames without extensions).
The optional argument \textit{main} can be omitted
if \textit{main} matches \textit{dest}.
Optionally, compilation \textit{flags} can be defined via |\def| commands.
This command line makes the \TeX{} engine believe
it is compiling the file \textit{target}
whose content is specified as the latter parameter.
The provided code then forwards the processing to
\textit{main} or \textit{dest} as described in \secref{sec:forward}.

%%%%%%%%%%%%%%%%%%%%%%%%%%%%%%%%%%%%%%%%%%%%%%%%%%%%%%%%%%%%%%%%%%%%%%%%%%%%%%%%
\subsection{Include by Input}
\label{sec:input}

Including child documents by |\include| has some restrictions by design.
Most notably, the content of a child document always occupies
its own set of pages; pages cannot be shared between child documents.
Usually, this behaviour makes perfect sense
because each child document contain an essential part of the document.
However, in some situations it may be desirable to compose
a document from a collection of parts
without having mandatory page breaks between then.
For this case, the package
provides a mechanism to include parts
by |\input| which can also be processed individually.
However, by construction this mechanism
requires manual handling of the content to be output.

%%%%%%%%%%%%%%%%%%%%%%%%%%%%%%%%%%%%%%%%
\DescribeMacro{\ifchilddocmanual}
The main file should be prepared as usual, see \secref{sec:include}.
However, the document body must make a distinction
between processing of an individual part and of the main document, e.g.:
%
\begin{center}
\begin{tabular}{l}
|\ifchilddocmanual|\\
|\input{\childdocname}|\\
|\||else|\\
\textit{document body with }|\input{|\textit{part}|}|\\
|\||fi|
\end{tabular}
\end{center}
%
The conditional |\ifchilddocmanual| is true whenever
a part to be included by |\input| is being compiled,
and the name of the part is stored in |\childdocname|.

%%%%%%%%%%%%%%%%%%%%%%%%%%%%%%%%%%%%%%%%
\DescribeMacro{\childdocby}
Each part to be included by |\input| should start with:
%
\begin{center}
\begin{tabular}{l}
|\input{childdoc.def}|\\
|\childdocby{|\textit{main}|}|\\
\end{tabular}
\end{center}
%
The directive |\childdocby| is similar to |\childdocof|
described in \secref{sec:include},
but the subsequent selection of content must be done manually.
To that end, both |\ifchilddoc| and |\ifchilddocmanual|
will be true upon processing of a part,
and the name of the part is stored in |\childdocname|.
Note that |\jobname| will be set to the filename of the current part
so that each part receives an individual |.aux| file
that does not interfere with the |.aux| file(s) of the main document.
This behaviour can be altered by the alternative form
|\childdocby[*]{|\textit{main}|}| (with a non-empty optional argument)
which uses the |.aux| file of the main document
by setting |\jobname| to \textit{main}.

%%%%%%%%%%%%%%%%%%%%%%%%%%%%%%%%%%%%%%%%%%%%%%%%%%%%%%%%%%%%%%%%%%%%%%%%%%%%%%%%
\subsection{Driver Development}
\label{sec:driver}

The \textsf{childdoc} mechanism can also be use for the development
of definition files such as \LaTeX{} styles or classes.
This case differs from the above setup with multiple parts
included by |\include| in that no |\includeonly| should be invoked.
This can be achieved by starting the include file
(before |\ProvidesPackage|) with:
%
\begin{center}
\begin{tabular}{l}
|\input{childdoc.def}|\\
|\childdocforward{|\textit{main}|}|\\
\end{tabular}
\end{center}
%
or alternatively with:
%
\begin{center}
\begin{tabular}{l}
|\input{childdoc.def}|\\
|\childdocby{|\textit{main}|}|\\
\end{tabular}
\end{center}
%
Both forms have slightly different effects as described above.
The main file is prepared as usual, see \secref{sec:include}.

%%%%%%%%%%%%%%%%%%%%%%%%%%%%%%%%%%%%%%%%%%%%%%%%%%%%%%%%%%%%%%%%%%%%%%%%%%%%%%%%
\subsection{Legacy Detection}
\label{sec:detection}

The directive |\childdocmain| in the main file can detect
whether the complete document or merely a child is to be compiled
even without using the directive |\childdocof|.
This method is deprecated because it is less robust
and there is no compelling reason to use it;
it is merely provided for backward compatibility
and it may be removed in future versions.

If the detection mechanism is to be used,
it is mandatory to correctly specify
the filename of the main file as the argument of |\childdocmain|:
%
\begin{center}
\begin{tabular}{l}
|\input{childdoc.def}|\\
|\childdocmain{|\textit{main}|}|\\
\end{tabular}
\end{center}
%
If |\jobname| does not match the argument \textit{main} of |\childdocmain|,
it is assumed that |\jobname| points to the child file to be compiled.
When using |\childdocmain| with the main file specified as argument,
it suffices to start a child file
with just |\input{|\textit{main}|}|
without loading of the package and using |\childdocof|.
If instead all processing is done
with the appropriate \textsf{childdoc} directives,
the argument of \textit{main} of |\childdocmain| can be empty.

An alternative version of the command line processing described
in \secref{sec:commandline} using the detection mechanism reads:
%
\begin{center}
|... -jobname "|\textit{target}|" "|[\textit{flags}]%
[|\def\jobname{|\textit{dest}|}|]|\input{|\textit{main}|}"|
\end{center}

%%%%%%%%%%%%%%%%%%%%%%%%%%%%%%%%%%%%%%%%%%%%%%%%%%%%%%%%%%%%%%%%%%%%%%%%%%%%%%%%
\subsection{Manual Code}
\label{sec:manual}

In case one cannot be certain whether the definitions file |childdoc.def|
is installed on the target \TeX{} distribution
and one prefers not to ship it,
it is conceivable to paste a few relevant commands into the sources.

To that end, drop all statements |\input{childdoc.def}|
and perform the replacements as outlined below.
Instead of |\childdocmain{|\textit{main}|}| add the following code
to the top of the main file:
%
\begin{center}
\begin{tabular}{l}
|\||ifdefined\childdocname\endinput\||fi\newif\ifchilddoc|\\
|\edef\childdocname{\scantokens\expandafter{\jobname\noexpand}}|\\
|\def\childdocmain{|\textit{main}|}\||ifx\childdocmain\childdocname\||else|\\
|\childdoctrue\includeonly{\childdocname}\let\jobname\childdocmain\||fi|\\
\end{tabular}
\end{center}
%
Instead of |\childdocof{|\textit{main}|}| just include the main file
at the top of each child file:
%
\begin{center}
|\input{|\textit{main}|}|
\end{center}
%
A simple redirection |\childdocforward{|\textit{dest}|}| is achieved by:
%
\begin{center}
|\def\jobname{|\textit{dest}|}\input{\jobname}|
\end{center}
%
The redirection with prefix
|\childdocforwardprefix[|\textit{prefix}|]{|\textit{dest}|}|
is accomplished by:
%
\begin{center}
\begin{tabular}{l}
|{\edef\jobname{\scantokens\expandafter{\jobname\noexpand}}|\\
|\def\redirectjob |\textit{prefix}|#1~~~{\gdef\jobname{|\textit{dest}|#1}}|\\
|\expandafter\redirectjob\jobname~~~}\input{\jobname}|
\end{tabular}
\end{center}

In an alternative approach,
child documents can be compiled by a specific command line
without additional code or specific definitions:
%
\begin{center}
|... -jobname "|\textit{target}|" "|[\textit{flags}]%
|\includeonly{|\textit{dest}|}\input{|\textit{main}|}"|
\end{center}
%

%%%%%%%%%%%%%%%%%%%%%%%%%%%%%%%%%%%%%%%%%%%%%%%%%%%%%%%%%%%%%%%%%%%%%%%%%%%%%%%%
%%%%%%%%%%%%%%%%%%%%%%%%%%%%%%%%%%%%%%%%%%%%%%%%%%%%%%%%%%%%%%%%%%%%%%%%%%%%%%%%
\section{Information}

%%%%%%%%%%%%%%%%%%%%%%%%%%%%%%%%%%%%%%%%%%%%%%%%%%%%%%%%%%%%%%%%%%%%%%%%%%%%%%%%
\subsection{Copyright}

Copyright \copyright{} 2017--2018 Niklas Beisert

This work may be distributed and/or modified under the
conditions of the \LaTeX{} Project Public License, either version 1.3
of this license or (at your option) any later version.
The latest version of this license is in
  \url{http://www.latex-project.org/lppl.txt}
and version 1.3 or later is part of all distributions of \LaTeX{}
version 2005/12/01 or later.

This work has the LPPL maintenance status `maintained'.

The Current Maintainer of this work is Niklas Beisert.

This work consists of the files |README.txt|, |childdoc.ins| and |childdoc.dtx|
as well as the derived files |childdoc.def|, |cdocsamp.tex|
with |cdocsch1.tex|, |cdocsch2.tex|, |cdocspt3.tex|, |cdocspt4.tex|,
|cdocsdrf.tex|, |cdocsfn1.tex|, |cdocsfn2.tex|
as well as |childdoc.pdf|.

%%%%%%%%%%%%%%%%%%%%%%%%%%%%%%%%%%%%%%%%%%%%%%%%%%%%%%%%%%%%%%%%%%%%%%%%%%%%%%%%
\subsection{Files and Installation}

The package consists of the files:
%
\begin{center}
\begin{tabular}{ll}
    |README.txt|   & readme file \\
    |childdoc.ins| & installation file \\
    |childdoc.dtx| & source file \\
    |childdoc.def| & definition file \\
    |cdocsamp.tex| & sample main file \\
    |cdocsch1.tex| & sample include file \\
    |cdocsch2.tex| & sample include file \\
    |cdocspt3.tex| & sample part file \\
    |cdocspt4.tex| & sample part file \\
    |cdocsdrf.tex| & sample redirection file \\
    |cdocsfn1.tex| & sample redirection file \\
    |cdocsfn2.tex| & sample redirection file \\
    |childdoc.pdf| & manual
\end{tabular}
\end{center}
%
The distribution consists of the files
|README.txt|, |childdoc.ins| and |childdoc.dtx|.
%
\begin{itemize}
\item
Run (pdf)\LaTeX{} on |childdoc.dtx|
to compile the manual |childdoc.pdf| (this file).
\item
Run \LaTeX{} on |childdoc.ins| to create the definitions file |childdoc.def|
and the sample |cdocsamp.tex| with include files
|cdocsch1.tex|, |cdocsch2.tex|, |cdocspt3.tex|, |cdocspt4.tex|,
|cdocsdrf.tex|, |cdocsfn1.tex|, |cdocsfn2.tex|.
Then copy the file |childdoc.def| to an appropriate directory of your \LaTeX{}
distribution, e.g.\ \textit{texmf-root}|/tex/latex/childdoc|.
\end{itemize}

%%%%%%%%%%%%%%%%%%%%%%%%%%%%%%%%%%%%%%%%%%%%%%%%%%%%%%%%%%%%%%%%%%%%%%%%%%%%%%%%
\subsection{Related CTAN Packages}

There are several other packages which offer a similar functionality:
%
\begin{itemize}
\item
The packages
\href{http://ctan.org/pkg/docmute}{\textsf{docmute}},
\href{http://ctan.org/pkg/includex}{\textsf{includex}} and
\href{http://ctan.org/pkg/standalone}{\textsf{standalone}}
provide commands to include only the document body of
a child file thus allowing both files to be compiled individually.
\item
The packages \href{http://ctan.org/pkg/subdocs}{\textsf{subdocs}}
and \href{http://ctan.org/pkg/subfiles}{\textsf{subfiles}}
provide structures in which the main and child documents can be
encapsulated and allowing them to be compiled individually.
The inclusion mechanism is different from the conventional |\include|.
\item
The package \href{http://ctan.org/pkg/combine}{\textsf{combine}}
is an elaborate solution to combine several documents into one.
\end{itemize}
%
See also the CTAN topic \href{http://ctan.org/topic/subdocs}{\textsf{subdocs}}
for further related packages.
The present package differs from the above solutions in that
a document structure constructed with the conventional |\include| mechanism
just needs two extra commands at the top of every file
such that all constituent files can be compiled individually.

%%%%%%%%%%%%%%%%%%%%%%%%%%%%%%%%%%%%%%%%%%%%%%%%%%%%%%%%%%%%%%%%%%%%%%%%%%%%%%%%
%\subsection{Feature Suggestions}
%
%The following is a list of features which may be useful for future
%versions of this package:
%%
%\begin{itemize}
%\item
%\ldots
%\end{itemize}

%%%%%%%%%%%%%%%%%%%%%%%%%%%%%%%%%%%%%%%%%%%%%%%%%%%%%%%%%%%%%%%%%%%%%%%%%%%%%%%%
\subsection{Revision History}

%%%%%%%%%%%%%%%%%%%%%%%%%%%%%%%%%%%%%%%%
\paragraph{v2.0:} 2018/12/30

\begin{itemize}
\item
immediate forward processing
\item
added |\childdocby| mechanism
\item
manual restructured
\end{itemize}

%%%%%%%%%%%%%%%%%%%%%%%%%%%%%%%%%%%%%%%%
\paragraph{v1.6:} 2018/01/17

\begin{itemize}
\item
application for development of include files
\item
corrections to manual
\end{itemize}

%%%%%%%%%%%%%%%%%%%%%%%%%%%%%%%%%%%%%%%%
\paragraph{v1.5:} 2017/05/21

\begin{itemize}
\item
more complete structuring introduced
\item
|\childdocof| introduced
\item
|\childdoc| renamed to |\childdocmain|
\item
|\childredirect| renamed to |\childdocforward| and |\childdocforwardprefix|
and functionality expanded
\end{itemize}

%%%%%%%%%%%%%%%%%%%%%%%%%%%%%%%%%%%%%%%%
\paragraph{v1.0:} 2017/04/27

\begin{itemize}
\item
manual and install package
\item
first version published on CTAN
\end{itemize}

%%%%%%%%%%%%%%%%%%%%%%%%%%%%%%%%%%%%%%%%
\paragraph{v0.6:} 2017/04/26

\begin{itemize}
\item
redirection mechanism added
\end{itemize}

%%%%%%%%%%%%%%%%%%%%%%%%%%%%%%%%%%%%%%%%
\paragraph{v0.5:} 2017/04/26

\begin{itemize}
\item
functionality in definition file
\end{itemize}


%%%%%%%%%%%%%%%%%%%%%%%%%%%%%%%%%%%%%%%%%%%%%%%%%%%%%%%%%%%%%%%%%%%%%%%%%%%%%%%%
%%%%%%%%%%%%%%%%%%%%%%%%%%%%%%%%%%%%%%%%%%%%%%%%%%%%%%%%%%%%%%%%%%%%%%%%%%%%%%%%
%%%%%%%%%%%%%%%%%%%%%%%%%%%%%%%%%%%%%%%%%%%%%%%%%%%%%%%%%%%%%%%%%%%%%%%%%%%%%%%%
\appendix

\settowidth\MacroIndent{\rmfamily\scriptsize 000\ }

 \DocInput{childdoc.dtx}

\end{document}
%</driver>
% \fi
%
% %%%%%%%%%%%%%%%%%%%%%%%%%%%%%%%%%%%%%%%%%%%%%%%%%%%%%%%%%%%%%%%%%%%%%%%%%%%%%%
% %%%%%%%%%%%%%%%%%%%%%%%%%%%%%%%%%%%%%%%%%%%%%%%%%%%%%%%%%%%%%%%%%%%%%%%%%%%%%%
% \section{Sample}
%\iffalse
%<*samplemain>
%\fi
%
% The following presents a sample document
% with two chapters, two parts, a title page,
% a compile flag as well as three forwarding files to set the flag.
% It consists of eight |.tex| files:
% \begin{center}
% \begin{tabular}{ll}
% |cdocsamp.tex|&main file\\
% |cdocsch1.tex|&include file for chapter 1\\
% |cdocsch2.tex|&include file for chapter 2\\
% |cdocspt3.tex|&include file for part 3\\
% |cdocspt4.tex|&include file for part 4\\
% |cdocsdrf.tex|&forwarding file for main file in draft mode\\
% |cdocsfi1.tex|&forwarding file for final version of chapter 1\\
% |cdocsfi2.tex|&forwarding file for final version of chapter 2\\
% \end{tabular}
% \end{center}
% Each of the eight files can be compiled directly by the \LaTeX{} compiler.
%
% %%%%%%%%%%%%%%%%%%%%%%%%%%%%%%%%%%%%%%
% \paragraph{Main File.}
%
% The main file is called |cdocsamp.tex|.
%
% Load the \textsf{childdoc} definitions and
% declare the filename for the main document:
%    \begin{macrocode}
\input{childdoc.def}
\childdocmain{}
%    \end{macrocode}

% Optional override for |\version| flag:
%    \begin{macrocode}
%%\ifchilddoc\else\providecommand{\version}{draft}\fi
%    \end{macrocode}

% Define the default values for the |\version| flag
% (|final| for the main file and |draft| for childs):
%    \begin{macrocode}
\ifchilddoc
\providecommand{\version}{draft}
\else
\providecommand{\version}{final}
\fi
%    \end{macrocode}

% Load the standard document class:
%    \begin{macrocode}
\documentclass[12pt]{article}
%    \end{macrocode}

% Start the document body:
%    \begin{macrocode}
\begin{document}
%    \end{macrocode}

% Declare a title page.
% Print title, part of document being processed and version flag:
%    \begin{macrocode}
\addtocounter{page}{-1}
\begin{center}
{\LARGE\bfseries{}childdoc example\par}
\vspace{1cm}
\ifchilddoc
\ifchilddocmanual part\else chapter\fi:
`\childdocname' of `\childdocjob'\par
\else
main document: `\childdocjob'\par
\fi
version: \version\par
\end{center}
\newpage
%    \end{macrocode}

% Manually include selected file,
% otherwise process as usual:
%    \begin{macrocode}
\ifchilddocmanual
\section*{part `\childdocname'}
\input{\childdocname}
\else
%    \end{macrocode}

% Include the two chapters:
%    \begin{macrocode}
\include{cdocsch1}
\include{cdocsch2}
%    \end{macrocode}

% Include the two parts unless only chapters should be displayed:
%    \begin{macrocode}
\ifchilddoc\else
\section{part three}
\input{cdocspt3}
\section{part four}
\input{cdocspt4}
\fi
%    \end{macrocode}

% Process as usual until here:
%    \begin{macrocode}
\fi
%    \end{macrocode}

% End of document body:
%    \begin{macrocode}
\end{document}
%    \end{macrocode}
%\iffalse
%</samplemain>
%\fi
%
% %%%%%%%%%%%%%%%%%%%%%%%%%%%%%%%%%%%%%%
% \paragraph{Chapter Include Files.}
%
% The include files are called |cdocsch1.tex| and |cdocsch2.tex|.
%
%\iffalse
%<*samplechap1|samplechap2>
%\fi

% Optional override for |\version| flag:
%    \begin{macrocode}
%%\providecommand{\version}{final}
%    \end{macrocode}

% Include the main document:
%    \begin{macrocode}
\input{childdoc.def}
\childdocof{cdocsamp}
%    \end{macrocode}

%\iffalse
%</samplechap1|samplechap2>
%\fi
%
%\iffalse
%<*samplechap1>
%\fi
% Some text for chapter 1:
%    \begin{macrocode}
\section{one}
some text in chapter one
%    \end{macrocode}

%\iffalse
%</samplechap1>
%\fi
% Some text for chapter 2:
%\iffalse
%<*samplechap2>
%\fi
%    \begin{macrocode}
\section{two}
more text in chapter two
%    \end{macrocode}

%\iffalse
%</samplechap2>
%\fi
%
% %%%%%%%%%%%%%%%%%%%%%%%%%%%%%%%%%%%%%%
% \paragraph{Part Include Files.}
%
% The include files are called |cdocspt3.tex| and |cdocspt4.tex|.
%
%\iffalse
%<*samplepart3|samplepart4>
%\fi

% Optional override for |\version| flag:
%    \begin{macrocode}
%%\providecommand{\version}{final}
%    \end{macrocode}

% Include the main document:
%    \begin{macrocode}
\input{childdoc.def}
\childdocby{cdocsamp}
%    \end{macrocode}

%\iffalse
%</samplepart3|samplepart4>
%\fi
%
%\iffalse
%<*samplepart3>
%\fi
% Some text for part 3:
%    \begin{macrocode}
some text in part three
%    \end{macrocode}

%\iffalse
%</samplepart3>
%\fi
% Some text for part 4:
%\iffalse
%<*samplepart4>
%\fi
%    \begin{macrocode}
more text in part four
%    \end{macrocode}

%\iffalse
%</samplepart4>
%\fi
%
% %%%%%%%%%%%%%%%%%%%%%%%%%%%%%%%%%%%%%%
% \paragraph{Forwarding for a Complete Draft.}
%
% The following forwarding file |cdocsdrf.tex|
% compiles the main document in draft mode:
%\iffalse
%<*sampledraft>
%\fi
%    \begin{macrocode}
\def\version{draft}
\input{childdoc.def}
\childdocforward{cdocsamp}
%    \end{macrocode}

%\iffalse
%</sampledraft>
%\fi
%
% %%%%%%%%%%%%%%%%%%%%%%%%%%%%%%%%%%%%%%
% \paragraph{Forwarding for Final Version of the Chapters.}
%
% The following forwarding files |cdocsfn1.tex| and |cdocsfn2.tex|
% (with identical content)
% compile the final versions of the child documents
% |cdocsch1.tex| and |cdocsch2.tex|, respectively:
%\iffalse
%<*samplefinal>
%\fi
%    \begin{macrocode}
\def\version{final}
\input{childdoc.def}
\childdocforwardprefix[cdocsamp]{cdocsfn}{cdocsch}
%    \end{macrocode}

%\iffalse
%</samplefinal>
%\fi
%
% %%%%%%%%%%%%%%%%%%%%%%%%%%%%%%%%%%%%%%
% \paragraph{Command Line Processing.}
%
% The following three command lines generate the output files
% |cdocscld|, |cdocscl1| and |cdocscl2|
% which should be identical to
% |cdocsdrf|, |cdocsch1| and |cdocsfn2|, respectively:
% \begin{center}
% \begin{tabular}{l}
% |latex -jobname cdocscld \|\\
% |  "\def\version{draft}\input{childdoc.def}\childdocforward{cdocsamp}"|\\
% |latex -jobname cdocscl1 \|\\
% |  "\input{childdoc.def}\childdocforward[cdocsamp]{cdocsch1}"|\\
% |latex -jobname cdocscl2 \|\\
% |  "\def\version{final}\input{childdoc.def}\childdocforward{cdocsch2}"|
% \end{tabular}
% \end{center}
% Note that the trailing backslash on each first line
% merely continues the input to the second line
% (for convenient cut ant paste).
% Furthermore, the command |latex| can be replaced by any
% of its alternative versions such as |pdflatex|.
%
% %%%%%%%%%%%%%%%%%%%%%%%%%%%%%%%%%%%%%%%%%%%%%%%%%%%%%%%%%%%%%%%%%%%%%%%%%%%%%%
% %%%%%%%%%%%%%%%%%%%%%%%%%%%%%%%%%%%%%%%%%%%%%%%%%%%%%%%%%%%%%%%%%%%%%%%%%%%%%%
% \section{Implementation}
%\iffalse
%<*package>
%\fi
%
% This section describes the definitions file |childdoc.def|.

% The definitions cannot be loaded using |\usepackage| or |\RequirePackage|
% which has a mechanism to prevent loading a style file more than once.
% When loading the definitions by means of |\input|
% multiple instances have to be prevented manually:
%\iffalse
%This code needs to be before the `\ProvidesFile' directive
%which is defined at the beginning of this file.
%Therefore it is also placed there and commented out here.
%</package>
%<*discard>
%\fi
%    \begin{macrocode}
\ifdefined\childdocmain\endinput\fi
%    \end{macrocode}
%\iffalse
%</discard>
%<*package>
%\fi
%
% \macro{\ifchilddoc}
% \macro{\ifchilddocmanual}
% The conditional |\ifchilddoc| tells whether a
% child (true) or main (false) document is being compiled.
% The conditional |\ifchilddocmanual| tells whether
% the |\includeonly| mechanism is used (false) or
% the selection of child files must be performed manually (true).
% The definitions initialise to false:
%    \begin{macrocode}
\newif\ifchilddoc
\newif\ifchilddocmanual
%    \end{macrocode}

% \macro{\childdocname}
% \macro{\childdocjob}
% The macro |\childdocname| stores the name of the main document
% to be compiled. The macro |\childdocjob| stores the name of
% the document on which the \LaTeX{} compiler was originally invoked.
% The content of |\jobname| cannot be compared
% to filenames specified in the source due to different catcodes.
% The following code rescans |\jobname|, stores the result
% in |\childdocname| and saves a copy in |\childdocjob|:
%    \begin{macrocode}
\edef\childdocname{\scantokens\expandafter{\jobname\noexpand}}
\let\childdocjob\childdocname
%    \end{macrocode}

% \macro{\childdocdisable}
% The macro |\childdocdisable| prevents the main file
% from being processed more than once.
% At this stage, the main document command |\childdocmain|
% is assumed to be called once again where it should do nothing.
% Any subsequent call to it should prevent
% a secondary processing of the main document
% It overwrites the forwarding commands
% |\childdocof| and |\childdocforward|
% with empty macros to prevent further inclusions of the main document:
%    \begin{macrocode}
\newcommand{\childdocdisable}
{
  \renewcommand{\childdocmain}[1]{\renewcommand{\childdocmain}[1]{\endinput}}
  \renewcommand{\childdocof}[1]{}
  \renewcommand{\childdocby}[2][]{}
  \renewcommand{\childdocforward}[2][]{}
  \renewcommand{\childdocdisable}{}
}
%    \end{macrocode}

% \macro{\childdocmain}
% The macro |\childdocmain| is to be called at the top of the main file
% with nothing or the main filename (without extension) as argument.
% First, it breaks loops.
% If the argument is not empty and does not match |\childdocname|
% (which is set by the first inclusion of |childdoc.def|),
% |\ifchilddoc| is set to true, |\includeonly| is applied to the child file
% and |\jobname| is set to the main file
% (for proper handling of |.aux| files):
%    \begin{macrocode}
\newcommand{\childdocmain}[1]
{
  \childdocdisable\childdocmain{}
  \if?#1?\else
    \begingroup
      \def\childdoctmp{#1}
      \ifx\childdoctmp\childdocname
        \def\childdoctmp{}
      \else
        \def\childdoctmp
        {
          \childdoctrue
          \includeonly{\childdocname}
          \def\childdocjob{#1}
          \def\jobname{#1}
        }
      \fi
      \expandafter
    \endgroup
    \childdoctmp
  \fi
}
%    \end{macrocode}

% \macro{\childdocof}
% The command |\childdocof| redirects
% compilation to the main file |#1|.
%    \begin{macrocode}
\newcommand{\childdocof}[1]
{
  \childdocdisable
  \childdoctrue
  \includeonly{\childdocname}
  \def\jobname{#1}
  \def\childdocjob{#1}
  \input{#1}
}
%    \end{macrocode}

% \macro{\childdocby}
% The command |\childdocby| ....
%    \begin{macrocode}
\newcommand{\childdocby}[2][]
{
  \childdocdisable
  \childdoctrue
  \childdocmanualtrue
  \if?#1?\else
    \def\jobname{#2}
  \fi
  \def\childdocjob{#2}
  \input{#2}
  \endinput
}
%    \end{macrocode}

% \macro{\childdocforward}
% The command |\childdocforward| redirects
% compilation to the main file or
% (if the optional argument is given) a child file.
% Parameters are set as if the main file
% or a child file starting with |\childdocof| was compiled.
% Then compilation is handed over to the main file:
%    \begin{macrocode}
\newcommand{\childdocforward}[2][]
{
  \begingroup
    \if?#1?
      \def\childdoctmp
      {
        \def\childdocname{#2}
        \def\childdocjob{#2}
        \def\jobname{#2}
        \input{#2}
        \endinput
      }
    \else
      \def\childdoctmp
      {
        \childdocdisable
        \def\childdocname{#2}
        \childdoctrue
        \includeonly{#2}
        \def\childdocjob{#1}
        \def\jobname{#1}
        \input{#1}
        \endinput
      }
    \fi
    \expandafter
  \endgroup
  \childdoctmp
}
%    \end{macrocode}

% \macro{\childdocforwardprefix}
% The command |\childdocforwardprefix| redirects
% compilation to the main or a child file by means of a pattern.
% The prefix |#1| in the current filename is replaced by |#2|
% and the suffix of the current filename is kept
% (it is assumed that the filename does not contain the substring `|~~~|'
% which is used as a delimiter).
% Compilation is handed over to the new file by |\childdocforward|:
%    \begin{macrocode}
\newcommand{\childdocforwardprefix}[3][]
{
  \begingroup
    \def\childdocextract #2##1~~~{\def\childdoctmp{\childdocforward[#1]{#3##1}}}
    \expandafter\childdocextract\childdocname~~~
    \expandafter
  \endgroup
  \childdoctmp
}
%    \end{macrocode}

% \macro{\childdoc}
% The deprecated macro |\childdoc| is a legacy version of |\childdocmain|:
%    \begin{macrocode}
\newcommand{\childdoc}{\childdocmain}
%    \end{macrocode}

% \macro{\childdocredirect}
% The deprecated macro |\childdocredirect| is a legacy version
% of |\childdocforward| and |\childdocforwardprefix|:
%    \begin{macrocode}
\newcommand{\childdocredirect}[2][]
{
  \begingroup
    \if?#1?
      \def\childdoctmp{\childdocforward{#2}}
    \else
      \def\childdoctmp{\childdocforwardprefix{#1}{#2}}
    \fi
    \expandafter
  \endgroup
  \childdoctmp
}
%    \end{macrocode}

%\iffalse
%</package>
%\fi
%
\endinput
|\\
|\childdocforwardprefix[|\textit{main}|]{|\textit{prefix}|}{|\textit{dest}|}|
\end{tabular}
\end{center}
%
the destination file is determined by a pattern
depending on the current file:
To make this work, the current file must be called
`{\textit{prefix}\hspace{0.2em}\textit{suffix}}'
with \textit{prefix} matching precisely the argument.
Processing is then passed on to the file
`{\textit{dest}\hspace{0.2em}\textit{suffix}}'.
Surely, the same effect is achieved by
directly specifying the
argument `{\textit{dest}\hspace{0.2em}\textit{suffix}}'
in the first form.
However, that requires to set up a different file
for each child. With the alternative form of the command
all these files can have exactly the same content
which simplifies setting them up and maintaining them.

For example, the following file |draft.tex|
with a compilation flag |\version| as described in \secref{sec:flags}
compiles the main document as a draft:
%
\begin{center}
\begin{tabular}{l}
|\def\version{draft}|\\
|% \iffalse
%
% childdoc.dtx Copyright (C) 2017-2018 Niklas Beisert
%
% This work may be distributed and/or modified under the
% conditions of the LaTeX Project Public License, either version 1.3
% of this license or (at your option) any later version.
% The latest version of this license is in
%   http://www.latex-project.org/lppl.txt
% and version 1.3 or later is part of all distributions of LaTeX
% version 2005/12/01 or later.
%
% This work has the LPPL maintenance status `maintained'.
%
% The Current Maintainer of this work is Niklas Beisert.
%
% This work consists of the files childdoc.dtx and childdoc.ins
% and the derived files childdoc.def and cdocsamp.tex with
% cdocsch1.tex, cdocsch2.tex, cdocsdrf.tex, cdocsfn1.tex, cdocsfn2.tex.
%
%<package>\ifdefined\childdocmain\endinput\fi
%<package>\ProvidesFile{childdoc.def}[2018/12/30 v2.0 child document driver]
%<samplemain>\ProvidesFile{cdocsamp.tex}[2018/12/30 v2.0 sample for childdoc]
%<*driver>
%\ProvidesFile{childdoc.drv}[2018/12/30 v2.0 childdoc reference manual file]
\PassOptionsToClass{10pt,a4paper}{article}
\documentclass{ltxdoc}

\usepackage[margin=35mm]{geometry}
\usepackage{hyperref}
\usepackage{hyperxmp}
\usepackage[usenames]{color}

\hypersetup{colorlinks=true}
\hypersetup{pdfstartview=FitH}
\hypersetup{pdfpagemode=UseNone}
\hypersetup{pdfsource={}}
\hypersetup{pdflang={en-UK}}
\hypersetup{pdfcopyright={Copyright 2017-2018 Niklas Beisert.
  This work may be distributed and/or modified under the
  conditions of the LaTeX Project Public License, either version 1.3
  of this license or (at your option) any later version.}}
\hypersetup{pdflicenseurl={http://www.latex-project.org/lppl.txt}}
\hypersetup{pdfcontactaddress={ETH Zurich, ITP, HIT K,
  Wolfgang-Pauli-Strasse 27}}
\hypersetup{pdfcontactpostcode={8093}}
\hypersetup{pdfcontactcity={Zurich}}
\hypersetup{pdfcontactcountry={Switzerland}}
\hypersetup{pdfcontactemail={nbeisert@itp.phys.ethz.ch}}
\hypersetup{pdfcontacturl={http://people.phys.ethz.ch/\xmptilde nbeisert/}}

\newcommand{\secref}[1]{\hyperref[#1]{section \ref*{#1}}}

\parskip1ex
\parindent0pt
\let\olditemize\itemize
\def\itemize{\olditemize\parskip0pt}

\begin{document}

\title{The \textsf{childdoc} Package}
\hypersetup{pdftitle={The childdoc Package}}
\author{Niklas Beisert\\[2ex]
  Institut f\"ur Theoretische Physik\\
  Eidgen\"ossische Technische Hochschule Z\"urich\\
  Wolfgang-Pauli-Strasse 27, 8093 Z\"urich, Switzerland\\[1ex]
  \href{mailto:nbeisert@itp.phys.ethz.ch}
  {\texttt{nbeisert@itp.phys.ethz.ch}}}
\hypersetup{pdfauthor={Niklas Beisert}}
\hypersetup{pdfsubject={Manual for the LaTeX2e Package childdoc}}
\date{30 December 2018, \textsf{v2.0}}
\maketitle

\begin{abstract}\noindent
\textsf{childdoc} is a \LaTeXe{} package
that enables the direct compilation
of document sections included by |\include|
to individual files.
\end{abstract}

\begingroup
\parskip0ex
\tableofcontents
\endgroup

%%%%%%%%%%%%%%%%%%%%%%%%%%%%%%%%%%%%%%%%%%%%%%%%%%%%%%%%%%%%%%%%%%%%%%%%%%%%%%%%
%%%%%%%%%%%%%%%%%%%%%%%%%%%%%%%%%%%%%%%%%%%%%%%%%%%%%%%%%%%%%%%%%%%%%%%%%%%%%%%%
\section{Introduction}

\LaTeX{} provides a mechanism to structure a large document (such as a book)
into a main file and several child files (containing the chapters)
using the |\include| command.
This mechanism is beneficial for documents
which span hundreds of pages in order to
make the source file(s) more manageable.
Moreover, compilation can be restricted to
selected child files by means of the |\includeonly| command.
The latter feature can be used to reduce the compilation time while editing
(this was significantly more useful in the earlier days of \LaTeX{})
or to generate a smaller document which is easier to navigate.
Another application of |\includeonly| is to generate
documents consisting of selected parts of the complete document.

However, there are a few drawbacks of the plain |\include| mechanism:
\begin{itemize}
\item
The child files cannot be compiled on their own,
they can only be compiled via the main file.
A naive editing environment
(such as a text editor with an option
to have the current file processed by \LaTeX)
may require one to switch to the main file before compiling;
attempting to compile the child file produces errors.
\item
The main file must be modified (each time)
to adjust the |\includeonly| command
to the present needs. This easily leaves the main file in a messy state.
\item
The generated document will always carry the filename
of the main document. This is inconvenient if
several child files are to be compiled and
to be kept for distribution.
\end{itemize}

The present package provides a simple interface
to make child files individually compilable by \LaTeX{}.
Compiling a child file then has the same effect as compiling
the main file with an |\includeonly| command
to select the appropriate child.
Moreover the generated document will carry the name of the child
rather than the main file.
This resolves all three above issues.

This feature is meant to make the editing of books,
thesis documents and lecture notes somewhat more convenient.
However, the package can also be used efficiently for
composing a series of documents (such as exercise sheets)
which are typically distributed individually.
It then assists the author in generating the individual documents
(potentially in different versions)
as well as a document containing the collected series.
Another application is in developing style files
or other kinds of included material
where compilation of the style file could redirect
to a sample or test file.

%%%%%%%%%%%%%%%%%%%%%%%%%%%%%%%%%%%%%%%%%%%%%%%%%%%%%%%%%%%%%%%%%%%%%%%%%%%%%%%%
%%%%%%%%%%%%%%%%%%%%%%%%%%%%%%%%%%%%%%%%%%%%%%%%%%%%%%%%%%%%%%%%%%%%%%%%%%%%%%%%
\section{Usage}

First of all, the package \textsf{childdoc} is \emph{not} a standard
\LaTeXe{} |.sty| style file! Therefore it needs to be invoked in
a non-standard way.

%%%%%%%%%%%%%%%%%%%%%%%%%%%%%%%%%%%%%%%%%%%%%%%%%%%%%%%%%%%%%%%%%%%%%%%%%%%%%%%%
\subsection{Included Files}
\label{sec:include}

%%%%%%%%%%%%%%%%%%%%%%%%%%%%%%%%%%%%%%%%
\DescribeMacro{\childdocmain}
To use the package, add the commands
\begin{center}
\begin{tabular}{l}
|\input{childdoc.def}|\\
|\childdocmain{}|\\
\end{tabular}
\end{center}
at the very top of the main \LaTeX{} file,
in particular \emph{before} the |\documentclass| statement!
The argument of |\childdocmain| should be left empty
(but it must be present).

%%%%%%%%%%%%%%%%%%%%%%%%%%%%%%%%%%%%%%%%
\DescribeMacro{\childdocof}
Furthermore, add the commands
\begin{center}
\begin{tabular}{l}
|\input{childdoc.def}|\\
|\childdocof{|\textit{main}|}|\\
\end{tabular}
\end{center}
at the top of every child file \textit{child}
which is included by |\include{|\textit{child}|}|
from within the main file
(or at least for those files to be compiled individually).
The argument \textit{main} must be the filename of the main file.

There are a couple of
considerations in setting up the main and child documents:

%%%%%%%%%%%%%%%%%%%%%%%%%%%%%%%%%%%%%%%%
\paragraph{Restrictions.}

Please note the following restrictions:
\begin{itemize}
\item
|\childdocmain| must be called with one argument \textit{main}
to ensure compatibility with earlier version of the package.
It must either be empty (|\childdocmain{}|)
or precisely match the filename of the main file in which it is specified.
See \secref{sec:detection} for further information.
\item
The filename \textit{main} must be specified without the |.tex| extension.
\item
The filename \textit{main} is case sensitive
(even in case-insensitive file systems)
due to internal string comparison.
\item
The argument \textit{main} should be fully expanded, it cannot be a macro.
\item
Subdirectories and special characters should be avoided in filenames.
\item
The command |\childdocmain{|\textit{main}|}| must be followed by a whitespace.
It should not be followed immediately by another command
or by a comment mark `|%|'.
This is because the \TeX{} parser reads the token immediately following
the argument of |\childdocmain| and puts it
at the beginning of every child section;
however, a white\-space is ignored.
\end{itemize}

%%%%%%%%%%%%%%%%%%%%%%%%%%%%%%%%%%%%%%%%
\paragraph{Content of Main File.}

It is advisable to place all content in the child files included by |\include|.
Any output contained in the main file will appear in all child documents
unless suppressed manually;
it cannot be suppressed automatically by the |\includeonly| directive
and thus should normally be avoided.
A method to include some content in the main file
by means of conditional processing is described in \secref{sec:conditional}.

%%%%%%%%%%%%%%%%%%%%%%%%%%%%%%%%%%%%%%%%
\paragraph{Page Numbering.}

When only a part of the document is compiled,
the appropriate numbering of pages
(as well as other status parameters)
is determined from the |.aux| files.
The latter contain information from previous passes.
However this information needs to propagate through
all intermediate child documents.
Therefore the page numbering in child documents may well
be inconsistent until the complete document is compiled at least once.

A useful (if unconventional) way to always ensure a consistent
page numbering is to restart the numbering in each child document
and denote the pages by `\textit{child}|.|\textit{page}'
where \textit{child} represents the chapter/section number of the child file.
This can be achieved by the command
|\numberwithin{page}{|\textit{child}|}|
of the \textsf{amsmath} package
where \textit{child} can be |chapter| or |section|
depending on the chosen structuring.
Alternatively, one can modify the macro |\thepage| appropriately
and reset the counter |page| at the start of each child file.

%%%%%%%%%%%%%%%%%%%%%%%%%%%%%%%%%%%%%%%%%%%%%%%%%%%%%%%%%%%%%%%%%%%%%%%%%%%%%%%%
\subsection{Conditional Processing}
\label{sec:conditional}

The package provides a mechanism to compile different versions
of a document. To customise the versions further some conditional processing
can come in handy to distinguish which version is being compiled.
The package provides two macros to describe the compilation context:

%%%%%%%%%%%%%%%%%%%%%%%%%%%%%%%%%%%%%%%%
\DescribeMacro{\ifchilddoc}
The conditional |\ifchilddoc| distinguishes between the compilation of
child documents and the main document:
%
\begin{center}
|\ifchilddoc |\textit{child-code}| |[|\||else |\textit{main-code}]| \||fi|
\end{center}

%%%%%%%%%%%%%%%%%%%%%%%%%%%%%%%%%%%%%%%%
\DescribeMacro{\childdocname}
\DescribeMacro{\childdocjob}
The macro |\childdocname| contains the filename (without extension)
of the main or child file being processed.
Note that |\childdocjob| will always contain the name of the main file.

%%%%%%%%%%%%%%%%%%%%%%%%%%%%%%%%%%%%%%%%
\paragraph{Title Page.}

Conditional processing can be used to include a title or banner page
in the main document when proper precautions are taken.
Importantly, the code in the main file should ensure that the page counter
(as well as other status parameters which are stored in the |.aux| files)
takes the same value after the conditional processing.
Otherwise the page numbers may take divergent values
depending on which part is compiled.

For example, a title page could be declared by:
%
\begin{center}
\begin{tabular}{l}
|\ifchilddoc\||else|\\
|\addtocounter{page}{-1}|\\
\textit{code for title page}\\
|\newpage|\\
|\||fi|
\end{tabular}
\end{center}
%
A banner page for the child documents can be generated by:
%
\begin{center}
\begin{tabular}{l}
|\ifchilddoc|\\
|\addtocounter{page}{-1}|\\
\textit{code for banner page}\\
|\newpage|\\
|\||fi|
\end{tabular}
\end{center}
%
Here one could write a message such as:
\begin{center}
|This is the part \childdocname{} of \childdocjob{}.|
\end{center}

%%%%%%%%%%%%%%%%%%%%%%%%%%%%%%%%%%%%%%%%%%%%%%%%%%%%%%%%%%%%%%%%%%%%%%%%%%%%%%%%
\subsection{Flags}
\label{sec:flags}

The package makes it easy to generate different versions
of the main or child documents.
To this end compilation flags can be defined
and assigned different default values.
They will be particularly useful in conjunction
with the forwarding mechanism described in \secref{sec:forward}.

For example, it may be useful to have a flag |\version|
which can be set to |draft| or |final|.
The document source will contain some conditional code
depending on the value of |\version|.
Suppose further, the flag should default to |final| for the main file
and to |draft| for child files
which is a natural assignment for editing the document.
This is achieved by placing the following code
in the preamble of the main document
(below the |\childdocmain| directive):
%
\begin{center}
\begin{tabular}{l}
|\ifchilddoc|\\
|\providecommand{\version}{draft}|\\
|\||else|\\
|\providecommand{\version}{final}|\\
|\||fi|
\end{tabular}
\end{center}
%
The definition by |\providecommand| makes sure
that previous definitions are not overwritten.
Further statements |\providecommand{\version}{...}|
can thus be added before the above code to override it.

For the main file, one might add a line
(between |\childdocmain| and the above block)
%
\begin{center}
|%\ifchilddoc\||else\providecommand{\version}{draft}\||fi|
\end{center}
%
which can be uncommented to produce a draft version.
Likewise one can add a line to the very top of a child file
(above the |\childdocof{|\textit{main}|}| directive)
%
\begin{center}
|%\providecommand{\version}{final}|
\end{center}
%
which can be uncommented to produce the final version of this child document.

%%%%%%%%%%%%%%%%%%%%%%%%%%%%%%%%%%%%%%%%%%%%%%%%%%%%%%%%%%%%%%%%%%%%%%%%%%%%%%%%
\subsection{Forwarding}
\label{sec:forward}

Different versions of the main or child documents
using compilation flags as described in \secref{sec:flags}
can be (permanently) stored in different files
for convenient compilation, viewing and distribution.
To this end, the package defines a command
to pass on compilation to a different file:

%%%%%%%%%%%%%%%%%%%%%%%%%%%%%%%%%%%%%%%%
\DescribeMacro{\childdocforward}
The command |\childdocforward| redirects processing to
another source file:
%
\begin{center}
\begin{tabular}{l}
|\input{childdoc.def}|\\
|\childdocforward[|\textit{main}|]{|\textit{dest}|}|\\
\end{tabular}
\end{center}
%
The argument \textit{dest} is the destination file
(without extension).
It should be the main file or one of the child files.
Note that further \textsf{childdoc} directives
such as |\childdocof| and |\childdocforward|
in the indicated file will be processed in this form.
The optional argument \textit{main}
passes on directly to the main file \textit{main}
while pretending to compile the child \textit{dest}.
This form behaves as if \textit{dest}
issues |\childdocof{|\textit{main}|}| right away,
and no further \textsf{childdoc} directives will be processed.

%%%%%%%%%%%%%%%%%%%%%%%%%%%%%%%%%%%%%%%%
\DescribeMacro{\...prefix}
In the alternative form |\childdocforwardprefix|,
%
\begin{center}
\begin{tabular}{l}
|\input{childdoc.def}|\\
|\childdocforwardprefix[|\textit{main}|]{|\textit{prefix}|}{|\textit{dest}|}|
\end{tabular}
\end{center}
%
the destination file is determined by a pattern
depending on the current file:
To make this work, the current file must be called
`{\textit{prefix}\hspace{0.2em}\textit{suffix}}'
with \textit{prefix} matching precisely the argument.
Processing is then passed on to the file
`{\textit{dest}\hspace{0.2em}\textit{suffix}}'.
Surely, the same effect is achieved by
directly specifying the
argument `{\textit{dest}\hspace{0.2em}\textit{suffix}}'
in the first form.
However, that requires to set up a different file
for each child. With the alternative form of the command
all these files can have exactly the same content
which simplifies setting them up and maintaining them.

For example, the following file |draft.tex|
with a compilation flag |\version| as described in \secref{sec:flags}
compiles the main document as a draft:
%
\begin{center}
\begin{tabular}{l}
|\def\version{draft}|\\
|\input{childdoc.def}|\\
|\childdocforward{|\textit{main}|}|
\end{tabular}
\end{center}
%
Likewise, the following files |final|\textit{nn}|.tex|
compile the final version of the child document
|child|\textit{nn}|.tex|:
%
\begin{center}
\begin{tabular}{l}
|\def\version{final}|\\
|\input{childdoc.def}|\\
|\childdocforwardprefix{final}{child}|
\end{tabular}
\end{center}
%

Note that when several versions of a main file and/or of each child file
are to be generated, it may be convenient to set up a |Makefile| or
shell script to automatise the process.

%%%%%%%%%%%%%%%%%%%%%%%%%%%%%%%%%%%%%%%%%%%%%%%%%%%%%%%%%%%%%%%%%%%%%%%%%%%%%%%%
\subsection{Command Line Processing}
\label{sec:commandline}

The effect of redirection files can also be achieved by invoking
the \LaTeX{} compiler with a more elaborate command line.
Most conveniently this should be done as part
of a shell script or a |Makefile|.

When using \textsf{childdoc} in the main file, the following
command lines effectively perform a redirection
(note that depending on the shell being used,
backslashes may have to be doubled: `|\|' $\to$ `|\\|'):
%
\begin{center}
|... -jobname "|\textit{target}|" |\\|"|[\textit{flags}]%
|\input{childdoc.def}\childdocforward[|\textit{main}|]{|\textit{dest}|}"|
\end{center}
%
Here \textit{target} is the name of the output file,
\textit{main} is the name of the main file
and \textit{dest} is the name of the main or child file to be processed
(all filenames without extensions).
The optional argument \textit{main} can be omitted
if \textit{main} matches \textit{dest}.
Optionally, compilation \textit{flags} can be defined via |\def| commands.
This command line makes the \TeX{} engine believe
it is compiling the file \textit{target}
whose content is specified as the latter parameter.
The provided code then forwards the processing to
\textit{main} or \textit{dest} as described in \secref{sec:forward}.

%%%%%%%%%%%%%%%%%%%%%%%%%%%%%%%%%%%%%%%%%%%%%%%%%%%%%%%%%%%%%%%%%%%%%%%%%%%%%%%%
\subsection{Include by Input}
\label{sec:input}

Including child documents by |\include| has some restrictions by design.
Most notably, the content of a child document always occupies
its own set of pages; pages cannot be shared between child documents.
Usually, this behaviour makes perfect sense
because each child document contain an essential part of the document.
However, in some situations it may be desirable to compose
a document from a collection of parts
without having mandatory page breaks between then.
For this case, the package
provides a mechanism to include parts
by |\input| which can also be processed individually.
However, by construction this mechanism
requires manual handling of the content to be output.

%%%%%%%%%%%%%%%%%%%%%%%%%%%%%%%%%%%%%%%%
\DescribeMacro{\ifchilddocmanual}
The main file should be prepared as usual, see \secref{sec:include}.
However, the document body must make a distinction
between processing of an individual part and of the main document, e.g.:
%
\begin{center}
\begin{tabular}{l}
|\ifchilddocmanual|\\
|\input{\childdocname}|\\
|\||else|\\
\textit{document body with }|\input{|\textit{part}|}|\\
|\||fi|
\end{tabular}
\end{center}
%
The conditional |\ifchilddocmanual| is true whenever
a part to be included by |\input| is being compiled,
and the name of the part is stored in |\childdocname|.

%%%%%%%%%%%%%%%%%%%%%%%%%%%%%%%%%%%%%%%%
\DescribeMacro{\childdocby}
Each part to be included by |\input| should start with:
%
\begin{center}
\begin{tabular}{l}
|\input{childdoc.def}|\\
|\childdocby{|\textit{main}|}|\\
\end{tabular}
\end{center}
%
The directive |\childdocby| is similar to |\childdocof|
described in \secref{sec:include},
but the subsequent selection of content must be done manually.
To that end, both |\ifchilddoc| and |\ifchilddocmanual|
will be true upon processing of a part,
and the name of the part is stored in |\childdocname|.
Note that |\jobname| will be set to the filename of the current part
so that each part receives an individual |.aux| file
that does not interfere with the |.aux| file(s) of the main document.
This behaviour can be altered by the alternative form
|\childdocby[*]{|\textit{main}|}| (with a non-empty optional argument)
which uses the |.aux| file of the main document
by setting |\jobname| to \textit{main}.

%%%%%%%%%%%%%%%%%%%%%%%%%%%%%%%%%%%%%%%%%%%%%%%%%%%%%%%%%%%%%%%%%%%%%%%%%%%%%%%%
\subsection{Driver Development}
\label{sec:driver}

The \textsf{childdoc} mechanism can also be use for the development
of definition files such as \LaTeX{} styles or classes.
This case differs from the above setup with multiple parts
included by |\include| in that no |\includeonly| should be invoked.
This can be achieved by starting the include file
(before |\ProvidesPackage|) with:
%
\begin{center}
\begin{tabular}{l}
|\input{childdoc.def}|\\
|\childdocforward{|\textit{main}|}|\\
\end{tabular}
\end{center}
%
or alternatively with:
%
\begin{center}
\begin{tabular}{l}
|\input{childdoc.def}|\\
|\childdocby{|\textit{main}|}|\\
\end{tabular}
\end{center}
%
Both forms have slightly different effects as described above.
The main file is prepared as usual, see \secref{sec:include}.

%%%%%%%%%%%%%%%%%%%%%%%%%%%%%%%%%%%%%%%%%%%%%%%%%%%%%%%%%%%%%%%%%%%%%%%%%%%%%%%%
\subsection{Legacy Detection}
\label{sec:detection}

The directive |\childdocmain| in the main file can detect
whether the complete document or merely a child is to be compiled
even without using the directive |\childdocof|.
This method is deprecated because it is less robust
and there is no compelling reason to use it;
it is merely provided for backward compatibility
and it may be removed in future versions.

If the detection mechanism is to be used,
it is mandatory to correctly specify
the filename of the main file as the argument of |\childdocmain|:
%
\begin{center}
\begin{tabular}{l}
|\input{childdoc.def}|\\
|\childdocmain{|\textit{main}|}|\\
\end{tabular}
\end{center}
%
If |\jobname| does not match the argument \textit{main} of |\childdocmain|,
it is assumed that |\jobname| points to the child file to be compiled.
When using |\childdocmain| with the main file specified as argument,
it suffices to start a child file
with just |\input{|\textit{main}|}|
without loading of the package and using |\childdocof|.
If instead all processing is done
with the appropriate \textsf{childdoc} directives,
the argument of \textit{main} of |\childdocmain| can be empty.

An alternative version of the command line processing described
in \secref{sec:commandline} using the detection mechanism reads:
%
\begin{center}
|... -jobname "|\textit{target}|" "|[\textit{flags}]%
[|\def\jobname{|\textit{dest}|}|]|\input{|\textit{main}|}"|
\end{center}

%%%%%%%%%%%%%%%%%%%%%%%%%%%%%%%%%%%%%%%%%%%%%%%%%%%%%%%%%%%%%%%%%%%%%%%%%%%%%%%%
\subsection{Manual Code}
\label{sec:manual}

In case one cannot be certain whether the definitions file |childdoc.def|
is installed on the target \TeX{} distribution
and one prefers not to ship it,
it is conceivable to paste a few relevant commands into the sources.

To that end, drop all statements |\input{childdoc.def}|
and perform the replacements as outlined below.
Instead of |\childdocmain{|\textit{main}|}| add the following code
to the top of the main file:
%
\begin{center}
\begin{tabular}{l}
|\||ifdefined\childdocname\endinput\||fi\newif\ifchilddoc|\\
|\edef\childdocname{\scantokens\expandafter{\jobname\noexpand}}|\\
|\def\childdocmain{|\textit{main}|}\||ifx\childdocmain\childdocname\||else|\\
|\childdoctrue\includeonly{\childdocname}\let\jobname\childdocmain\||fi|\\
\end{tabular}
\end{center}
%
Instead of |\childdocof{|\textit{main}|}| just include the main file
at the top of each child file:
%
\begin{center}
|\input{|\textit{main}|}|
\end{center}
%
A simple redirection |\childdocforward{|\textit{dest}|}| is achieved by:
%
\begin{center}
|\def\jobname{|\textit{dest}|}\input{\jobname}|
\end{center}
%
The redirection with prefix
|\childdocforwardprefix[|\textit{prefix}|]{|\textit{dest}|}|
is accomplished by:
%
\begin{center}
\begin{tabular}{l}
|{\edef\jobname{\scantokens\expandafter{\jobname\noexpand}}|\\
|\def\redirectjob |\textit{prefix}|#1~~~{\gdef\jobname{|\textit{dest}|#1}}|\\
|\expandafter\redirectjob\jobname~~~}\input{\jobname}|
\end{tabular}
\end{center}

In an alternative approach,
child documents can be compiled by a specific command line
without additional code or specific definitions:
%
\begin{center}
|... -jobname "|\textit{target}|" "|[\textit{flags}]%
|\includeonly{|\textit{dest}|}\input{|\textit{main}|}"|
\end{center}
%

%%%%%%%%%%%%%%%%%%%%%%%%%%%%%%%%%%%%%%%%%%%%%%%%%%%%%%%%%%%%%%%%%%%%%%%%%%%%%%%%
%%%%%%%%%%%%%%%%%%%%%%%%%%%%%%%%%%%%%%%%%%%%%%%%%%%%%%%%%%%%%%%%%%%%%%%%%%%%%%%%
\section{Information}

%%%%%%%%%%%%%%%%%%%%%%%%%%%%%%%%%%%%%%%%%%%%%%%%%%%%%%%%%%%%%%%%%%%%%%%%%%%%%%%%
\subsection{Copyright}

Copyright \copyright{} 2017--2018 Niklas Beisert

This work may be distributed and/or modified under the
conditions of the \LaTeX{} Project Public License, either version 1.3
of this license or (at your option) any later version.
The latest version of this license is in
  \url{http://www.latex-project.org/lppl.txt}
and version 1.3 or later is part of all distributions of \LaTeX{}
version 2005/12/01 or later.

This work has the LPPL maintenance status `maintained'.

The Current Maintainer of this work is Niklas Beisert.

This work consists of the files |README.txt|, |childdoc.ins| and |childdoc.dtx|
as well as the derived files |childdoc.def|, |cdocsamp.tex|
with |cdocsch1.tex|, |cdocsch2.tex|, |cdocspt3.tex|, |cdocspt4.tex|,
|cdocsdrf.tex|, |cdocsfn1.tex|, |cdocsfn2.tex|
as well as |childdoc.pdf|.

%%%%%%%%%%%%%%%%%%%%%%%%%%%%%%%%%%%%%%%%%%%%%%%%%%%%%%%%%%%%%%%%%%%%%%%%%%%%%%%%
\subsection{Files and Installation}

The package consists of the files:
%
\begin{center}
\begin{tabular}{ll}
    |README.txt|   & readme file \\
    |childdoc.ins| & installation file \\
    |childdoc.dtx| & source file \\
    |childdoc.def| & definition file \\
    |cdocsamp.tex| & sample main file \\
    |cdocsch1.tex| & sample include file \\
    |cdocsch2.tex| & sample include file \\
    |cdocspt3.tex| & sample part file \\
    |cdocspt4.tex| & sample part file \\
    |cdocsdrf.tex| & sample redirection file \\
    |cdocsfn1.tex| & sample redirection file \\
    |cdocsfn2.tex| & sample redirection file \\
    |childdoc.pdf| & manual
\end{tabular}
\end{center}
%
The distribution consists of the files
|README.txt|, |childdoc.ins| and |childdoc.dtx|.
%
\begin{itemize}
\item
Run (pdf)\LaTeX{} on |childdoc.dtx|
to compile the manual |childdoc.pdf| (this file).
\item
Run \LaTeX{} on |childdoc.ins| to create the definitions file |childdoc.def|
and the sample |cdocsamp.tex| with include files
|cdocsch1.tex|, |cdocsch2.tex|, |cdocspt3.tex|, |cdocspt4.tex|,
|cdocsdrf.tex|, |cdocsfn1.tex|, |cdocsfn2.tex|.
Then copy the file |childdoc.def| to an appropriate directory of your \LaTeX{}
distribution, e.g.\ \textit{texmf-root}|/tex/latex/childdoc|.
\end{itemize}

%%%%%%%%%%%%%%%%%%%%%%%%%%%%%%%%%%%%%%%%%%%%%%%%%%%%%%%%%%%%%%%%%%%%%%%%%%%%%%%%
\subsection{Related CTAN Packages}

There are several other packages which offer a similar functionality:
%
\begin{itemize}
\item
The packages
\href{http://ctan.org/pkg/docmute}{\textsf{docmute}},
\href{http://ctan.org/pkg/includex}{\textsf{includex}} and
\href{http://ctan.org/pkg/standalone}{\textsf{standalone}}
provide commands to include only the document body of
a child file thus allowing both files to be compiled individually.
\item
The packages \href{http://ctan.org/pkg/subdocs}{\textsf{subdocs}}
and \href{http://ctan.org/pkg/subfiles}{\textsf{subfiles}}
provide structures in which the main and child documents can be
encapsulated and allowing them to be compiled individually.
The inclusion mechanism is different from the conventional |\include|.
\item
The package \href{http://ctan.org/pkg/combine}{\textsf{combine}}
is an elaborate solution to combine several documents into one.
\end{itemize}
%
See also the CTAN topic \href{http://ctan.org/topic/subdocs}{\textsf{subdocs}}
for further related packages.
The present package differs from the above solutions in that
a document structure constructed with the conventional |\include| mechanism
just needs two extra commands at the top of every file
such that all constituent files can be compiled individually.

%%%%%%%%%%%%%%%%%%%%%%%%%%%%%%%%%%%%%%%%%%%%%%%%%%%%%%%%%%%%%%%%%%%%%%%%%%%%%%%%
%\subsection{Feature Suggestions}
%
%The following is a list of features which may be useful for future
%versions of this package:
%%
%\begin{itemize}
%\item
%\ldots
%\end{itemize}

%%%%%%%%%%%%%%%%%%%%%%%%%%%%%%%%%%%%%%%%%%%%%%%%%%%%%%%%%%%%%%%%%%%%%%%%%%%%%%%%
\subsection{Revision History}

%%%%%%%%%%%%%%%%%%%%%%%%%%%%%%%%%%%%%%%%
\paragraph{v2.0:} 2018/12/30

\begin{itemize}
\item
immediate forward processing
\item
added |\childdocby| mechanism
\item
manual restructured
\end{itemize}

%%%%%%%%%%%%%%%%%%%%%%%%%%%%%%%%%%%%%%%%
\paragraph{v1.6:} 2018/01/17

\begin{itemize}
\item
application for development of include files
\item
corrections to manual
\end{itemize}

%%%%%%%%%%%%%%%%%%%%%%%%%%%%%%%%%%%%%%%%
\paragraph{v1.5:} 2017/05/21

\begin{itemize}
\item
more complete structuring introduced
\item
|\childdocof| introduced
\item
|\childdoc| renamed to |\childdocmain|
\item
|\childredirect| renamed to |\childdocforward| and |\childdocforwardprefix|
and functionality expanded
\end{itemize}

%%%%%%%%%%%%%%%%%%%%%%%%%%%%%%%%%%%%%%%%
\paragraph{v1.0:} 2017/04/27

\begin{itemize}
\item
manual and install package
\item
first version published on CTAN
\end{itemize}

%%%%%%%%%%%%%%%%%%%%%%%%%%%%%%%%%%%%%%%%
\paragraph{v0.6:} 2017/04/26

\begin{itemize}
\item
redirection mechanism added
\end{itemize}

%%%%%%%%%%%%%%%%%%%%%%%%%%%%%%%%%%%%%%%%
\paragraph{v0.5:} 2017/04/26

\begin{itemize}
\item
functionality in definition file
\end{itemize}


%%%%%%%%%%%%%%%%%%%%%%%%%%%%%%%%%%%%%%%%%%%%%%%%%%%%%%%%%%%%%%%%%%%%%%%%%%%%%%%%
%%%%%%%%%%%%%%%%%%%%%%%%%%%%%%%%%%%%%%%%%%%%%%%%%%%%%%%%%%%%%%%%%%%%%%%%%%%%%%%%
%%%%%%%%%%%%%%%%%%%%%%%%%%%%%%%%%%%%%%%%%%%%%%%%%%%%%%%%%%%%%%%%%%%%%%%%%%%%%%%%
\appendix

\settowidth\MacroIndent{\rmfamily\scriptsize 000\ }

 \DocInput{childdoc.dtx}

\end{document}
%</driver>
% \fi
%
% %%%%%%%%%%%%%%%%%%%%%%%%%%%%%%%%%%%%%%%%%%%%%%%%%%%%%%%%%%%%%%%%%%%%%%%%%%%%%%
% %%%%%%%%%%%%%%%%%%%%%%%%%%%%%%%%%%%%%%%%%%%%%%%%%%%%%%%%%%%%%%%%%%%%%%%%%%%%%%
% \section{Sample}
%\iffalse
%<*samplemain>
%\fi
%
% The following presents a sample document
% with two chapters, two parts, a title page,
% a compile flag as well as three forwarding files to set the flag.
% It consists of eight |.tex| files:
% \begin{center}
% \begin{tabular}{ll}
% |cdocsamp.tex|&main file\\
% |cdocsch1.tex|&include file for chapter 1\\
% |cdocsch2.tex|&include file for chapter 2\\
% |cdocspt3.tex|&include file for part 3\\
% |cdocspt4.tex|&include file for part 4\\
% |cdocsdrf.tex|&forwarding file for main file in draft mode\\
% |cdocsfi1.tex|&forwarding file for final version of chapter 1\\
% |cdocsfi2.tex|&forwarding file for final version of chapter 2\\
% \end{tabular}
% \end{center}
% Each of the eight files can be compiled directly by the \LaTeX{} compiler.
%
% %%%%%%%%%%%%%%%%%%%%%%%%%%%%%%%%%%%%%%
% \paragraph{Main File.}
%
% The main file is called |cdocsamp.tex|.
%
% Load the \textsf{childdoc} definitions and
% declare the filename for the main document:
%    \begin{macrocode}
\input{childdoc.def}
\childdocmain{}
%    \end{macrocode}

% Optional override for |\version| flag:
%    \begin{macrocode}
%%\ifchilddoc\else\providecommand{\version}{draft}\fi
%    \end{macrocode}

% Define the default values for the |\version| flag
% (|final| for the main file and |draft| for childs):
%    \begin{macrocode}
\ifchilddoc
\providecommand{\version}{draft}
\else
\providecommand{\version}{final}
\fi
%    \end{macrocode}

% Load the standard document class:
%    \begin{macrocode}
\documentclass[12pt]{article}
%    \end{macrocode}

% Start the document body:
%    \begin{macrocode}
\begin{document}
%    \end{macrocode}

% Declare a title page.
% Print title, part of document being processed and version flag:
%    \begin{macrocode}
\addtocounter{page}{-1}
\begin{center}
{\LARGE\bfseries{}childdoc example\par}
\vspace{1cm}
\ifchilddoc
\ifchilddocmanual part\else chapter\fi:
`\childdocname' of `\childdocjob'\par
\else
main document: `\childdocjob'\par
\fi
version: \version\par
\end{center}
\newpage
%    \end{macrocode}

% Manually include selected file,
% otherwise process as usual:
%    \begin{macrocode}
\ifchilddocmanual
\section*{part `\childdocname'}
\input{\childdocname}
\else
%    \end{macrocode}

% Include the two chapters:
%    \begin{macrocode}
\include{cdocsch1}
\include{cdocsch2}
%    \end{macrocode}

% Include the two parts unless only chapters should be displayed:
%    \begin{macrocode}
\ifchilddoc\else
\section{part three}
\input{cdocspt3}
\section{part four}
\input{cdocspt4}
\fi
%    \end{macrocode}

% Process as usual until here:
%    \begin{macrocode}
\fi
%    \end{macrocode}

% End of document body:
%    \begin{macrocode}
\end{document}
%    \end{macrocode}
%\iffalse
%</samplemain>
%\fi
%
% %%%%%%%%%%%%%%%%%%%%%%%%%%%%%%%%%%%%%%
% \paragraph{Chapter Include Files.}
%
% The include files are called |cdocsch1.tex| and |cdocsch2.tex|.
%
%\iffalse
%<*samplechap1|samplechap2>
%\fi

% Optional override for |\version| flag:
%    \begin{macrocode}
%%\providecommand{\version}{final}
%    \end{macrocode}

% Include the main document:
%    \begin{macrocode}
\input{childdoc.def}
\childdocof{cdocsamp}
%    \end{macrocode}

%\iffalse
%</samplechap1|samplechap2>
%\fi
%
%\iffalse
%<*samplechap1>
%\fi
% Some text for chapter 1:
%    \begin{macrocode}
\section{one}
some text in chapter one
%    \end{macrocode}

%\iffalse
%</samplechap1>
%\fi
% Some text for chapter 2:
%\iffalse
%<*samplechap2>
%\fi
%    \begin{macrocode}
\section{two}
more text in chapter two
%    \end{macrocode}

%\iffalse
%</samplechap2>
%\fi
%
% %%%%%%%%%%%%%%%%%%%%%%%%%%%%%%%%%%%%%%
% \paragraph{Part Include Files.}
%
% The include files are called |cdocspt3.tex| and |cdocspt4.tex|.
%
%\iffalse
%<*samplepart3|samplepart4>
%\fi

% Optional override for |\version| flag:
%    \begin{macrocode}
%%\providecommand{\version}{final}
%    \end{macrocode}

% Include the main document:
%    \begin{macrocode}
\input{childdoc.def}
\childdocby{cdocsamp}
%    \end{macrocode}

%\iffalse
%</samplepart3|samplepart4>
%\fi
%
%\iffalse
%<*samplepart3>
%\fi
% Some text for part 3:
%    \begin{macrocode}
some text in part three
%    \end{macrocode}

%\iffalse
%</samplepart3>
%\fi
% Some text for part 4:
%\iffalse
%<*samplepart4>
%\fi
%    \begin{macrocode}
more text in part four
%    \end{macrocode}

%\iffalse
%</samplepart4>
%\fi
%
% %%%%%%%%%%%%%%%%%%%%%%%%%%%%%%%%%%%%%%
% \paragraph{Forwarding for a Complete Draft.}
%
% The following forwarding file |cdocsdrf.tex|
% compiles the main document in draft mode:
%\iffalse
%<*sampledraft>
%\fi
%    \begin{macrocode}
\def\version{draft}
\input{childdoc.def}
\childdocforward{cdocsamp}
%    \end{macrocode}

%\iffalse
%</sampledraft>
%\fi
%
% %%%%%%%%%%%%%%%%%%%%%%%%%%%%%%%%%%%%%%
% \paragraph{Forwarding for Final Version of the Chapters.}
%
% The following forwarding files |cdocsfn1.tex| and |cdocsfn2.tex|
% (with identical content)
% compile the final versions of the child documents
% |cdocsch1.tex| and |cdocsch2.tex|, respectively:
%\iffalse
%<*samplefinal>
%\fi
%    \begin{macrocode}
\def\version{final}
\input{childdoc.def}
\childdocforwardprefix[cdocsamp]{cdocsfn}{cdocsch}
%    \end{macrocode}

%\iffalse
%</samplefinal>
%\fi
%
% %%%%%%%%%%%%%%%%%%%%%%%%%%%%%%%%%%%%%%
% \paragraph{Command Line Processing.}
%
% The following three command lines generate the output files
% |cdocscld|, |cdocscl1| and |cdocscl2|
% which should be identical to
% |cdocsdrf|, |cdocsch1| and |cdocsfn2|, respectively:
% \begin{center}
% \begin{tabular}{l}
% |latex -jobname cdocscld \|\\
% |  "\def\version{draft}\input{childdoc.def}\childdocforward{cdocsamp}"|\\
% |latex -jobname cdocscl1 \|\\
% |  "\input{childdoc.def}\childdocforward[cdocsamp]{cdocsch1}"|\\
% |latex -jobname cdocscl2 \|\\
% |  "\def\version{final}\input{childdoc.def}\childdocforward{cdocsch2}"|
% \end{tabular}
% \end{center}
% Note that the trailing backslash on each first line
% merely continues the input to the second line
% (for convenient cut ant paste).
% Furthermore, the command |latex| can be replaced by any
% of its alternative versions such as |pdflatex|.
%
% %%%%%%%%%%%%%%%%%%%%%%%%%%%%%%%%%%%%%%%%%%%%%%%%%%%%%%%%%%%%%%%%%%%%%%%%%%%%%%
% %%%%%%%%%%%%%%%%%%%%%%%%%%%%%%%%%%%%%%%%%%%%%%%%%%%%%%%%%%%%%%%%%%%%%%%%%%%%%%
% \section{Implementation}
%\iffalse
%<*package>
%\fi
%
% This section describes the definitions file |childdoc.def|.

% The definitions cannot be loaded using |\usepackage| or |\RequirePackage|
% which has a mechanism to prevent loading a style file more than once.
% When loading the definitions by means of |\input|
% multiple instances have to be prevented manually:
%\iffalse
%This code needs to be before the `\ProvidesFile' directive
%which is defined at the beginning of this file.
%Therefore it is also placed there and commented out here.
%</package>
%<*discard>
%\fi
%    \begin{macrocode}
\ifdefined\childdocmain\endinput\fi
%    \end{macrocode}
%\iffalse
%</discard>
%<*package>
%\fi
%
% \macro{\ifchilddoc}
% \macro{\ifchilddocmanual}
% The conditional |\ifchilddoc| tells whether a
% child (true) or main (false) document is being compiled.
% The conditional |\ifchilddocmanual| tells whether
% the |\includeonly| mechanism is used (false) or
% the selection of child files must be performed manually (true).
% The definitions initialise to false:
%    \begin{macrocode}
\newif\ifchilddoc
\newif\ifchilddocmanual
%    \end{macrocode}

% \macro{\childdocname}
% \macro{\childdocjob}
% The macro |\childdocname| stores the name of the main document
% to be compiled. The macro |\childdocjob| stores the name of
% the document on which the \LaTeX{} compiler was originally invoked.
% The content of |\jobname| cannot be compared
% to filenames specified in the source due to different catcodes.
% The following code rescans |\jobname|, stores the result
% in |\childdocname| and saves a copy in |\childdocjob|:
%    \begin{macrocode}
\edef\childdocname{\scantokens\expandafter{\jobname\noexpand}}
\let\childdocjob\childdocname
%    \end{macrocode}

% \macro{\childdocdisable}
% The macro |\childdocdisable| prevents the main file
% from being processed more than once.
% At this stage, the main document command |\childdocmain|
% is assumed to be called once again where it should do nothing.
% Any subsequent call to it should prevent
% a secondary processing of the main document
% It overwrites the forwarding commands
% |\childdocof| and |\childdocforward|
% with empty macros to prevent further inclusions of the main document:
%    \begin{macrocode}
\newcommand{\childdocdisable}
{
  \renewcommand{\childdocmain}[1]{\renewcommand{\childdocmain}[1]{\endinput}}
  \renewcommand{\childdocof}[1]{}
  \renewcommand{\childdocby}[2][]{}
  \renewcommand{\childdocforward}[2][]{}
  \renewcommand{\childdocdisable}{}
}
%    \end{macrocode}

% \macro{\childdocmain}
% The macro |\childdocmain| is to be called at the top of the main file
% with nothing or the main filename (without extension) as argument.
% First, it breaks loops.
% If the argument is not empty and does not match |\childdocname|
% (which is set by the first inclusion of |childdoc.def|),
% |\ifchilddoc| is set to true, |\includeonly| is applied to the child file
% and |\jobname| is set to the main file
% (for proper handling of |.aux| files):
%    \begin{macrocode}
\newcommand{\childdocmain}[1]
{
  \childdocdisable\childdocmain{}
  \if?#1?\else
    \begingroup
      \def\childdoctmp{#1}
      \ifx\childdoctmp\childdocname
        \def\childdoctmp{}
      \else
        \def\childdoctmp
        {
          \childdoctrue
          \includeonly{\childdocname}
          \def\childdocjob{#1}
          \def\jobname{#1}
        }
      \fi
      \expandafter
    \endgroup
    \childdoctmp
  \fi
}
%    \end{macrocode}

% \macro{\childdocof}
% The command |\childdocof| redirects
% compilation to the main file |#1|.
%    \begin{macrocode}
\newcommand{\childdocof}[1]
{
  \childdocdisable
  \childdoctrue
  \includeonly{\childdocname}
  \def\jobname{#1}
  \def\childdocjob{#1}
  \input{#1}
}
%    \end{macrocode}

% \macro{\childdocby}
% The command |\childdocby| ....
%    \begin{macrocode}
\newcommand{\childdocby}[2][]
{
  \childdocdisable
  \childdoctrue
  \childdocmanualtrue
  \if?#1?\else
    \def\jobname{#2}
  \fi
  \def\childdocjob{#2}
  \input{#2}
  \endinput
}
%    \end{macrocode}

% \macro{\childdocforward}
% The command |\childdocforward| redirects
% compilation to the main file or
% (if the optional argument is given) a child file.
% Parameters are set as if the main file
% or a child file starting with |\childdocof| was compiled.
% Then compilation is handed over to the main file:
%    \begin{macrocode}
\newcommand{\childdocforward}[2][]
{
  \begingroup
    \if?#1?
      \def\childdoctmp
      {
        \def\childdocname{#2}
        \def\childdocjob{#2}
        \def\jobname{#2}
        \input{#2}
        \endinput
      }
    \else
      \def\childdoctmp
      {
        \childdocdisable
        \def\childdocname{#2}
        \childdoctrue
        \includeonly{#2}
        \def\childdocjob{#1}
        \def\jobname{#1}
        \input{#1}
        \endinput
      }
    \fi
    \expandafter
  \endgroup
  \childdoctmp
}
%    \end{macrocode}

% \macro{\childdocforwardprefix}
% The command |\childdocforwardprefix| redirects
% compilation to the main or a child file by means of a pattern.
% The prefix |#1| in the current filename is replaced by |#2|
% and the suffix of the current filename is kept
% (it is assumed that the filename does not contain the substring `|~~~|'
% which is used as a delimiter).
% Compilation is handed over to the new file by |\childdocforward|:
%    \begin{macrocode}
\newcommand{\childdocforwardprefix}[3][]
{
  \begingroup
    \def\childdocextract #2##1~~~{\def\childdoctmp{\childdocforward[#1]{#3##1}}}
    \expandafter\childdocextract\childdocname~~~
    \expandafter
  \endgroup
  \childdoctmp
}
%    \end{macrocode}

% \macro{\childdoc}
% The deprecated macro |\childdoc| is a legacy version of |\childdocmain|:
%    \begin{macrocode}
\newcommand{\childdoc}{\childdocmain}
%    \end{macrocode}

% \macro{\childdocredirect}
% The deprecated macro |\childdocredirect| is a legacy version
% of |\childdocforward| and |\childdocforwardprefix|:
%    \begin{macrocode}
\newcommand{\childdocredirect}[2][]
{
  \begingroup
    \if?#1?
      \def\childdoctmp{\childdocforward{#2}}
    \else
      \def\childdoctmp{\childdocforwardprefix{#1}{#2}}
    \fi
    \expandafter
  \endgroup
  \childdoctmp
}
%    \end{macrocode}

%\iffalse
%</package>
%\fi
%
\endinput
|\\
|\childdocforward{|\textit{main}|}|
\end{tabular}
\end{center}
%
Likewise, the following files |final|\textit{nn}|.tex|
compile the final version of the child document
|child|\textit{nn}|.tex|:
%
\begin{center}
\begin{tabular}{l}
|\def\version{final}|\\
|% \iffalse
%
% childdoc.dtx Copyright (C) 2017-2018 Niklas Beisert
%
% This work may be distributed and/or modified under the
% conditions of the LaTeX Project Public License, either version 1.3
% of this license or (at your option) any later version.
% The latest version of this license is in
%   http://www.latex-project.org/lppl.txt
% and version 1.3 or later is part of all distributions of LaTeX
% version 2005/12/01 or later.
%
% This work has the LPPL maintenance status `maintained'.
%
% The Current Maintainer of this work is Niklas Beisert.
%
% This work consists of the files childdoc.dtx and childdoc.ins
% and the derived files childdoc.def and cdocsamp.tex with
% cdocsch1.tex, cdocsch2.tex, cdocsdrf.tex, cdocsfn1.tex, cdocsfn2.tex.
%
%<package>\ifdefined\childdocmain\endinput\fi
%<package>\ProvidesFile{childdoc.def}[2018/12/30 v2.0 child document driver]
%<samplemain>\ProvidesFile{cdocsamp.tex}[2018/12/30 v2.0 sample for childdoc]
%<*driver>
%\ProvidesFile{childdoc.drv}[2018/12/30 v2.0 childdoc reference manual file]
\PassOptionsToClass{10pt,a4paper}{article}
\documentclass{ltxdoc}

\usepackage[margin=35mm]{geometry}
\usepackage{hyperref}
\usepackage{hyperxmp}
\usepackage[usenames]{color}

\hypersetup{colorlinks=true}
\hypersetup{pdfstartview=FitH}
\hypersetup{pdfpagemode=UseNone}
\hypersetup{pdfsource={}}
\hypersetup{pdflang={en-UK}}
\hypersetup{pdfcopyright={Copyright 2017-2018 Niklas Beisert.
  This work may be distributed and/or modified under the
  conditions of the LaTeX Project Public License, either version 1.3
  of this license or (at your option) any later version.}}
\hypersetup{pdflicenseurl={http://www.latex-project.org/lppl.txt}}
\hypersetup{pdfcontactaddress={ETH Zurich, ITP, HIT K,
  Wolfgang-Pauli-Strasse 27}}
\hypersetup{pdfcontactpostcode={8093}}
\hypersetup{pdfcontactcity={Zurich}}
\hypersetup{pdfcontactcountry={Switzerland}}
\hypersetup{pdfcontactemail={nbeisert@itp.phys.ethz.ch}}
\hypersetup{pdfcontacturl={http://people.phys.ethz.ch/\xmptilde nbeisert/}}

\newcommand{\secref}[1]{\hyperref[#1]{section \ref*{#1}}}

\parskip1ex
\parindent0pt
\let\olditemize\itemize
\def\itemize{\olditemize\parskip0pt}

\begin{document}

\title{The \textsf{childdoc} Package}
\hypersetup{pdftitle={The childdoc Package}}
\author{Niklas Beisert\\[2ex]
  Institut f\"ur Theoretische Physik\\
  Eidgen\"ossische Technische Hochschule Z\"urich\\
  Wolfgang-Pauli-Strasse 27, 8093 Z\"urich, Switzerland\\[1ex]
  \href{mailto:nbeisert@itp.phys.ethz.ch}
  {\texttt{nbeisert@itp.phys.ethz.ch}}}
\hypersetup{pdfauthor={Niklas Beisert}}
\hypersetup{pdfsubject={Manual for the LaTeX2e Package childdoc}}
\date{30 December 2018, \textsf{v2.0}}
\maketitle

\begin{abstract}\noindent
\textsf{childdoc} is a \LaTeXe{} package
that enables the direct compilation
of document sections included by |\include|
to individual files.
\end{abstract}

\begingroup
\parskip0ex
\tableofcontents
\endgroup

%%%%%%%%%%%%%%%%%%%%%%%%%%%%%%%%%%%%%%%%%%%%%%%%%%%%%%%%%%%%%%%%%%%%%%%%%%%%%%%%
%%%%%%%%%%%%%%%%%%%%%%%%%%%%%%%%%%%%%%%%%%%%%%%%%%%%%%%%%%%%%%%%%%%%%%%%%%%%%%%%
\section{Introduction}

\LaTeX{} provides a mechanism to structure a large document (such as a book)
into a main file and several child files (containing the chapters)
using the |\include| command.
This mechanism is beneficial for documents
which span hundreds of pages in order to
make the source file(s) more manageable.
Moreover, compilation can be restricted to
selected child files by means of the |\includeonly| command.
The latter feature can be used to reduce the compilation time while editing
(this was significantly more useful in the earlier days of \LaTeX{})
or to generate a smaller document which is easier to navigate.
Another application of |\includeonly| is to generate
documents consisting of selected parts of the complete document.

However, there are a few drawbacks of the plain |\include| mechanism:
\begin{itemize}
\item
The child files cannot be compiled on their own,
they can only be compiled via the main file.
A naive editing environment
(such as a text editor with an option
to have the current file processed by \LaTeX)
may require one to switch to the main file before compiling;
attempting to compile the child file produces errors.
\item
The main file must be modified (each time)
to adjust the |\includeonly| command
to the present needs. This easily leaves the main file in a messy state.
\item
The generated document will always carry the filename
of the main document. This is inconvenient if
several child files are to be compiled and
to be kept for distribution.
\end{itemize}

The present package provides a simple interface
to make child files individually compilable by \LaTeX{}.
Compiling a child file then has the same effect as compiling
the main file with an |\includeonly| command
to select the appropriate child.
Moreover the generated document will carry the name of the child
rather than the main file.
This resolves all three above issues.

This feature is meant to make the editing of books,
thesis documents and lecture notes somewhat more convenient.
However, the package can also be used efficiently for
composing a series of documents (such as exercise sheets)
which are typically distributed individually.
It then assists the author in generating the individual documents
(potentially in different versions)
as well as a document containing the collected series.
Another application is in developing style files
or other kinds of included material
where compilation of the style file could redirect
to a sample or test file.

%%%%%%%%%%%%%%%%%%%%%%%%%%%%%%%%%%%%%%%%%%%%%%%%%%%%%%%%%%%%%%%%%%%%%%%%%%%%%%%%
%%%%%%%%%%%%%%%%%%%%%%%%%%%%%%%%%%%%%%%%%%%%%%%%%%%%%%%%%%%%%%%%%%%%%%%%%%%%%%%%
\section{Usage}

First of all, the package \textsf{childdoc} is \emph{not} a standard
\LaTeXe{} |.sty| style file! Therefore it needs to be invoked in
a non-standard way.

%%%%%%%%%%%%%%%%%%%%%%%%%%%%%%%%%%%%%%%%%%%%%%%%%%%%%%%%%%%%%%%%%%%%%%%%%%%%%%%%
\subsection{Included Files}
\label{sec:include}

%%%%%%%%%%%%%%%%%%%%%%%%%%%%%%%%%%%%%%%%
\DescribeMacro{\childdocmain}
To use the package, add the commands
\begin{center}
\begin{tabular}{l}
|\input{childdoc.def}|\\
|\childdocmain{}|\\
\end{tabular}
\end{center}
at the very top of the main \LaTeX{} file,
in particular \emph{before} the |\documentclass| statement!
The argument of |\childdocmain| should be left empty
(but it must be present).

%%%%%%%%%%%%%%%%%%%%%%%%%%%%%%%%%%%%%%%%
\DescribeMacro{\childdocof}
Furthermore, add the commands
\begin{center}
\begin{tabular}{l}
|\input{childdoc.def}|\\
|\childdocof{|\textit{main}|}|\\
\end{tabular}
\end{center}
at the top of every child file \textit{child}
which is included by |\include{|\textit{child}|}|
from within the main file
(or at least for those files to be compiled individually).
The argument \textit{main} must be the filename of the main file.

There are a couple of
considerations in setting up the main and child documents:

%%%%%%%%%%%%%%%%%%%%%%%%%%%%%%%%%%%%%%%%
\paragraph{Restrictions.}

Please note the following restrictions:
\begin{itemize}
\item
|\childdocmain| must be called with one argument \textit{main}
to ensure compatibility with earlier version of the package.
It must either be empty (|\childdocmain{}|)
or precisely match the filename of the main file in which it is specified.
See \secref{sec:detection} for further information.
\item
The filename \textit{main} must be specified without the |.tex| extension.
\item
The filename \textit{main} is case sensitive
(even in case-insensitive file systems)
due to internal string comparison.
\item
The argument \textit{main} should be fully expanded, it cannot be a macro.
\item
Subdirectories and special characters should be avoided in filenames.
\item
The command |\childdocmain{|\textit{main}|}| must be followed by a whitespace.
It should not be followed immediately by another command
or by a comment mark `|%|'.
This is because the \TeX{} parser reads the token immediately following
the argument of |\childdocmain| and puts it
at the beginning of every child section;
however, a white\-space is ignored.
\end{itemize}

%%%%%%%%%%%%%%%%%%%%%%%%%%%%%%%%%%%%%%%%
\paragraph{Content of Main File.}

It is advisable to place all content in the child files included by |\include|.
Any output contained in the main file will appear in all child documents
unless suppressed manually;
it cannot be suppressed automatically by the |\includeonly| directive
and thus should normally be avoided.
A method to include some content in the main file
by means of conditional processing is described in \secref{sec:conditional}.

%%%%%%%%%%%%%%%%%%%%%%%%%%%%%%%%%%%%%%%%
\paragraph{Page Numbering.}

When only a part of the document is compiled,
the appropriate numbering of pages
(as well as other status parameters)
is determined from the |.aux| files.
The latter contain information from previous passes.
However this information needs to propagate through
all intermediate child documents.
Therefore the page numbering in child documents may well
be inconsistent until the complete document is compiled at least once.

A useful (if unconventional) way to always ensure a consistent
page numbering is to restart the numbering in each child document
and denote the pages by `\textit{child}|.|\textit{page}'
where \textit{child} represents the chapter/section number of the child file.
This can be achieved by the command
|\numberwithin{page}{|\textit{child}|}|
of the \textsf{amsmath} package
where \textit{child} can be |chapter| or |section|
depending on the chosen structuring.
Alternatively, one can modify the macro |\thepage| appropriately
and reset the counter |page| at the start of each child file.

%%%%%%%%%%%%%%%%%%%%%%%%%%%%%%%%%%%%%%%%%%%%%%%%%%%%%%%%%%%%%%%%%%%%%%%%%%%%%%%%
\subsection{Conditional Processing}
\label{sec:conditional}

The package provides a mechanism to compile different versions
of a document. To customise the versions further some conditional processing
can come in handy to distinguish which version is being compiled.
The package provides two macros to describe the compilation context:

%%%%%%%%%%%%%%%%%%%%%%%%%%%%%%%%%%%%%%%%
\DescribeMacro{\ifchilddoc}
The conditional |\ifchilddoc| distinguishes between the compilation of
child documents and the main document:
%
\begin{center}
|\ifchilddoc |\textit{child-code}| |[|\||else |\textit{main-code}]| \||fi|
\end{center}

%%%%%%%%%%%%%%%%%%%%%%%%%%%%%%%%%%%%%%%%
\DescribeMacro{\childdocname}
\DescribeMacro{\childdocjob}
The macro |\childdocname| contains the filename (without extension)
of the main or child file being processed.
Note that |\childdocjob| will always contain the name of the main file.

%%%%%%%%%%%%%%%%%%%%%%%%%%%%%%%%%%%%%%%%
\paragraph{Title Page.}

Conditional processing can be used to include a title or banner page
in the main document when proper precautions are taken.
Importantly, the code in the main file should ensure that the page counter
(as well as other status parameters which are stored in the |.aux| files)
takes the same value after the conditional processing.
Otherwise the page numbers may take divergent values
depending on which part is compiled.

For example, a title page could be declared by:
%
\begin{center}
\begin{tabular}{l}
|\ifchilddoc\||else|\\
|\addtocounter{page}{-1}|\\
\textit{code for title page}\\
|\newpage|\\
|\||fi|
\end{tabular}
\end{center}
%
A banner page for the child documents can be generated by:
%
\begin{center}
\begin{tabular}{l}
|\ifchilddoc|\\
|\addtocounter{page}{-1}|\\
\textit{code for banner page}\\
|\newpage|\\
|\||fi|
\end{tabular}
\end{center}
%
Here one could write a message such as:
\begin{center}
|This is the part \childdocname{} of \childdocjob{}.|
\end{center}

%%%%%%%%%%%%%%%%%%%%%%%%%%%%%%%%%%%%%%%%%%%%%%%%%%%%%%%%%%%%%%%%%%%%%%%%%%%%%%%%
\subsection{Flags}
\label{sec:flags}

The package makes it easy to generate different versions
of the main or child documents.
To this end compilation flags can be defined
and assigned different default values.
They will be particularly useful in conjunction
with the forwarding mechanism described in \secref{sec:forward}.

For example, it may be useful to have a flag |\version|
which can be set to |draft| or |final|.
The document source will contain some conditional code
depending on the value of |\version|.
Suppose further, the flag should default to |final| for the main file
and to |draft| for child files
which is a natural assignment for editing the document.
This is achieved by placing the following code
in the preamble of the main document
(below the |\childdocmain| directive):
%
\begin{center}
\begin{tabular}{l}
|\ifchilddoc|\\
|\providecommand{\version}{draft}|\\
|\||else|\\
|\providecommand{\version}{final}|\\
|\||fi|
\end{tabular}
\end{center}
%
The definition by |\providecommand| makes sure
that previous definitions are not overwritten.
Further statements |\providecommand{\version}{...}|
can thus be added before the above code to override it.

For the main file, one might add a line
(between |\childdocmain| and the above block)
%
\begin{center}
|%\ifchilddoc\||else\providecommand{\version}{draft}\||fi|
\end{center}
%
which can be uncommented to produce a draft version.
Likewise one can add a line to the very top of a child file
(above the |\childdocof{|\textit{main}|}| directive)
%
\begin{center}
|%\providecommand{\version}{final}|
\end{center}
%
which can be uncommented to produce the final version of this child document.

%%%%%%%%%%%%%%%%%%%%%%%%%%%%%%%%%%%%%%%%%%%%%%%%%%%%%%%%%%%%%%%%%%%%%%%%%%%%%%%%
\subsection{Forwarding}
\label{sec:forward}

Different versions of the main or child documents
using compilation flags as described in \secref{sec:flags}
can be (permanently) stored in different files
for convenient compilation, viewing and distribution.
To this end, the package defines a command
to pass on compilation to a different file:

%%%%%%%%%%%%%%%%%%%%%%%%%%%%%%%%%%%%%%%%
\DescribeMacro{\childdocforward}
The command |\childdocforward| redirects processing to
another source file:
%
\begin{center}
\begin{tabular}{l}
|\input{childdoc.def}|\\
|\childdocforward[|\textit{main}|]{|\textit{dest}|}|\\
\end{tabular}
\end{center}
%
The argument \textit{dest} is the destination file
(without extension).
It should be the main file or one of the child files.
Note that further \textsf{childdoc} directives
such as |\childdocof| and |\childdocforward|
in the indicated file will be processed in this form.
The optional argument \textit{main}
passes on directly to the main file \textit{main}
while pretending to compile the child \textit{dest}.
This form behaves as if \textit{dest}
issues |\childdocof{|\textit{main}|}| right away,
and no further \textsf{childdoc} directives will be processed.

%%%%%%%%%%%%%%%%%%%%%%%%%%%%%%%%%%%%%%%%
\DescribeMacro{\...prefix}
In the alternative form |\childdocforwardprefix|,
%
\begin{center}
\begin{tabular}{l}
|\input{childdoc.def}|\\
|\childdocforwardprefix[|\textit{main}|]{|\textit{prefix}|}{|\textit{dest}|}|
\end{tabular}
\end{center}
%
the destination file is determined by a pattern
depending on the current file:
To make this work, the current file must be called
`{\textit{prefix}\hspace{0.2em}\textit{suffix}}'
with \textit{prefix} matching precisely the argument.
Processing is then passed on to the file
`{\textit{dest}\hspace{0.2em}\textit{suffix}}'.
Surely, the same effect is achieved by
directly specifying the
argument `{\textit{dest}\hspace{0.2em}\textit{suffix}}'
in the first form.
However, that requires to set up a different file
for each child. With the alternative form of the command
all these files can have exactly the same content
which simplifies setting them up and maintaining them.

For example, the following file |draft.tex|
with a compilation flag |\version| as described in \secref{sec:flags}
compiles the main document as a draft:
%
\begin{center}
\begin{tabular}{l}
|\def\version{draft}|\\
|\input{childdoc.def}|\\
|\childdocforward{|\textit{main}|}|
\end{tabular}
\end{center}
%
Likewise, the following files |final|\textit{nn}|.tex|
compile the final version of the child document
|child|\textit{nn}|.tex|:
%
\begin{center}
\begin{tabular}{l}
|\def\version{final}|\\
|\input{childdoc.def}|\\
|\childdocforwardprefix{final}{child}|
\end{tabular}
\end{center}
%

Note that when several versions of a main file and/or of each child file
are to be generated, it may be convenient to set up a |Makefile| or
shell script to automatise the process.

%%%%%%%%%%%%%%%%%%%%%%%%%%%%%%%%%%%%%%%%%%%%%%%%%%%%%%%%%%%%%%%%%%%%%%%%%%%%%%%%
\subsection{Command Line Processing}
\label{sec:commandline}

The effect of redirection files can also be achieved by invoking
the \LaTeX{} compiler with a more elaborate command line.
Most conveniently this should be done as part
of a shell script or a |Makefile|.

When using \textsf{childdoc} in the main file, the following
command lines effectively perform a redirection
(note that depending on the shell being used,
backslashes may have to be doubled: `|\|' $\to$ `|\\|'):
%
\begin{center}
|... -jobname "|\textit{target}|" |\\|"|[\textit{flags}]%
|\input{childdoc.def}\childdocforward[|\textit{main}|]{|\textit{dest}|}"|
\end{center}
%
Here \textit{target} is the name of the output file,
\textit{main} is the name of the main file
and \textit{dest} is the name of the main or child file to be processed
(all filenames without extensions).
The optional argument \textit{main} can be omitted
if \textit{main} matches \textit{dest}.
Optionally, compilation \textit{flags} can be defined via |\def| commands.
This command line makes the \TeX{} engine believe
it is compiling the file \textit{target}
whose content is specified as the latter parameter.
The provided code then forwards the processing to
\textit{main} or \textit{dest} as described in \secref{sec:forward}.

%%%%%%%%%%%%%%%%%%%%%%%%%%%%%%%%%%%%%%%%%%%%%%%%%%%%%%%%%%%%%%%%%%%%%%%%%%%%%%%%
\subsection{Include by Input}
\label{sec:input}

Including child documents by |\include| has some restrictions by design.
Most notably, the content of a child document always occupies
its own set of pages; pages cannot be shared between child documents.
Usually, this behaviour makes perfect sense
because each child document contain an essential part of the document.
However, in some situations it may be desirable to compose
a document from a collection of parts
without having mandatory page breaks between then.
For this case, the package
provides a mechanism to include parts
by |\input| which can also be processed individually.
However, by construction this mechanism
requires manual handling of the content to be output.

%%%%%%%%%%%%%%%%%%%%%%%%%%%%%%%%%%%%%%%%
\DescribeMacro{\ifchilddocmanual}
The main file should be prepared as usual, see \secref{sec:include}.
However, the document body must make a distinction
between processing of an individual part and of the main document, e.g.:
%
\begin{center}
\begin{tabular}{l}
|\ifchilddocmanual|\\
|\input{\childdocname}|\\
|\||else|\\
\textit{document body with }|\input{|\textit{part}|}|\\
|\||fi|
\end{tabular}
\end{center}
%
The conditional |\ifchilddocmanual| is true whenever
a part to be included by |\input| is being compiled,
and the name of the part is stored in |\childdocname|.

%%%%%%%%%%%%%%%%%%%%%%%%%%%%%%%%%%%%%%%%
\DescribeMacro{\childdocby}
Each part to be included by |\input| should start with:
%
\begin{center}
\begin{tabular}{l}
|\input{childdoc.def}|\\
|\childdocby{|\textit{main}|}|\\
\end{tabular}
\end{center}
%
The directive |\childdocby| is similar to |\childdocof|
described in \secref{sec:include},
but the subsequent selection of content must be done manually.
To that end, both |\ifchilddoc| and |\ifchilddocmanual|
will be true upon processing of a part,
and the name of the part is stored in |\childdocname|.
Note that |\jobname| will be set to the filename of the current part
so that each part receives an individual |.aux| file
that does not interfere with the |.aux| file(s) of the main document.
This behaviour can be altered by the alternative form
|\childdocby[*]{|\textit{main}|}| (with a non-empty optional argument)
which uses the |.aux| file of the main document
by setting |\jobname| to \textit{main}.

%%%%%%%%%%%%%%%%%%%%%%%%%%%%%%%%%%%%%%%%%%%%%%%%%%%%%%%%%%%%%%%%%%%%%%%%%%%%%%%%
\subsection{Driver Development}
\label{sec:driver}

The \textsf{childdoc} mechanism can also be use for the development
of definition files such as \LaTeX{} styles or classes.
This case differs from the above setup with multiple parts
included by |\include| in that no |\includeonly| should be invoked.
This can be achieved by starting the include file
(before |\ProvidesPackage|) with:
%
\begin{center}
\begin{tabular}{l}
|\input{childdoc.def}|\\
|\childdocforward{|\textit{main}|}|\\
\end{tabular}
\end{center}
%
or alternatively with:
%
\begin{center}
\begin{tabular}{l}
|\input{childdoc.def}|\\
|\childdocby{|\textit{main}|}|\\
\end{tabular}
\end{center}
%
Both forms have slightly different effects as described above.
The main file is prepared as usual, see \secref{sec:include}.

%%%%%%%%%%%%%%%%%%%%%%%%%%%%%%%%%%%%%%%%%%%%%%%%%%%%%%%%%%%%%%%%%%%%%%%%%%%%%%%%
\subsection{Legacy Detection}
\label{sec:detection}

The directive |\childdocmain| in the main file can detect
whether the complete document or merely a child is to be compiled
even without using the directive |\childdocof|.
This method is deprecated because it is less robust
and there is no compelling reason to use it;
it is merely provided for backward compatibility
and it may be removed in future versions.

If the detection mechanism is to be used,
it is mandatory to correctly specify
the filename of the main file as the argument of |\childdocmain|:
%
\begin{center}
\begin{tabular}{l}
|\input{childdoc.def}|\\
|\childdocmain{|\textit{main}|}|\\
\end{tabular}
\end{center}
%
If |\jobname| does not match the argument \textit{main} of |\childdocmain|,
it is assumed that |\jobname| points to the child file to be compiled.
When using |\childdocmain| with the main file specified as argument,
it suffices to start a child file
with just |\input{|\textit{main}|}|
without loading of the package and using |\childdocof|.
If instead all processing is done
with the appropriate \textsf{childdoc} directives,
the argument of \textit{main} of |\childdocmain| can be empty.

An alternative version of the command line processing described
in \secref{sec:commandline} using the detection mechanism reads:
%
\begin{center}
|... -jobname "|\textit{target}|" "|[\textit{flags}]%
[|\def\jobname{|\textit{dest}|}|]|\input{|\textit{main}|}"|
\end{center}

%%%%%%%%%%%%%%%%%%%%%%%%%%%%%%%%%%%%%%%%%%%%%%%%%%%%%%%%%%%%%%%%%%%%%%%%%%%%%%%%
\subsection{Manual Code}
\label{sec:manual}

In case one cannot be certain whether the definitions file |childdoc.def|
is installed on the target \TeX{} distribution
and one prefers not to ship it,
it is conceivable to paste a few relevant commands into the sources.

To that end, drop all statements |\input{childdoc.def}|
and perform the replacements as outlined below.
Instead of |\childdocmain{|\textit{main}|}| add the following code
to the top of the main file:
%
\begin{center}
\begin{tabular}{l}
|\||ifdefined\childdocname\endinput\||fi\newif\ifchilddoc|\\
|\edef\childdocname{\scantokens\expandafter{\jobname\noexpand}}|\\
|\def\childdocmain{|\textit{main}|}\||ifx\childdocmain\childdocname\||else|\\
|\childdoctrue\includeonly{\childdocname}\let\jobname\childdocmain\||fi|\\
\end{tabular}
\end{center}
%
Instead of |\childdocof{|\textit{main}|}| just include the main file
at the top of each child file:
%
\begin{center}
|\input{|\textit{main}|}|
\end{center}
%
A simple redirection |\childdocforward{|\textit{dest}|}| is achieved by:
%
\begin{center}
|\def\jobname{|\textit{dest}|}\input{\jobname}|
\end{center}
%
The redirection with prefix
|\childdocforwardprefix[|\textit{prefix}|]{|\textit{dest}|}|
is accomplished by:
%
\begin{center}
\begin{tabular}{l}
|{\edef\jobname{\scantokens\expandafter{\jobname\noexpand}}|\\
|\def\redirectjob |\textit{prefix}|#1~~~{\gdef\jobname{|\textit{dest}|#1}}|\\
|\expandafter\redirectjob\jobname~~~}\input{\jobname}|
\end{tabular}
\end{center}

In an alternative approach,
child documents can be compiled by a specific command line
without additional code or specific definitions:
%
\begin{center}
|... -jobname "|\textit{target}|" "|[\textit{flags}]%
|\includeonly{|\textit{dest}|}\input{|\textit{main}|}"|
\end{center}
%

%%%%%%%%%%%%%%%%%%%%%%%%%%%%%%%%%%%%%%%%%%%%%%%%%%%%%%%%%%%%%%%%%%%%%%%%%%%%%%%%
%%%%%%%%%%%%%%%%%%%%%%%%%%%%%%%%%%%%%%%%%%%%%%%%%%%%%%%%%%%%%%%%%%%%%%%%%%%%%%%%
\section{Information}

%%%%%%%%%%%%%%%%%%%%%%%%%%%%%%%%%%%%%%%%%%%%%%%%%%%%%%%%%%%%%%%%%%%%%%%%%%%%%%%%
\subsection{Copyright}

Copyright \copyright{} 2017--2018 Niklas Beisert

This work may be distributed and/or modified under the
conditions of the \LaTeX{} Project Public License, either version 1.3
of this license or (at your option) any later version.
The latest version of this license is in
  \url{http://www.latex-project.org/lppl.txt}
and version 1.3 or later is part of all distributions of \LaTeX{}
version 2005/12/01 or later.

This work has the LPPL maintenance status `maintained'.

The Current Maintainer of this work is Niklas Beisert.

This work consists of the files |README.txt|, |childdoc.ins| and |childdoc.dtx|
as well as the derived files |childdoc.def|, |cdocsamp.tex|
with |cdocsch1.tex|, |cdocsch2.tex|, |cdocspt3.tex|, |cdocspt4.tex|,
|cdocsdrf.tex|, |cdocsfn1.tex|, |cdocsfn2.tex|
as well as |childdoc.pdf|.

%%%%%%%%%%%%%%%%%%%%%%%%%%%%%%%%%%%%%%%%%%%%%%%%%%%%%%%%%%%%%%%%%%%%%%%%%%%%%%%%
\subsection{Files and Installation}

The package consists of the files:
%
\begin{center}
\begin{tabular}{ll}
    |README.txt|   & readme file \\
    |childdoc.ins| & installation file \\
    |childdoc.dtx| & source file \\
    |childdoc.def| & definition file \\
    |cdocsamp.tex| & sample main file \\
    |cdocsch1.tex| & sample include file \\
    |cdocsch2.tex| & sample include file \\
    |cdocspt3.tex| & sample part file \\
    |cdocspt4.tex| & sample part file \\
    |cdocsdrf.tex| & sample redirection file \\
    |cdocsfn1.tex| & sample redirection file \\
    |cdocsfn2.tex| & sample redirection file \\
    |childdoc.pdf| & manual
\end{tabular}
\end{center}
%
The distribution consists of the files
|README.txt|, |childdoc.ins| and |childdoc.dtx|.
%
\begin{itemize}
\item
Run (pdf)\LaTeX{} on |childdoc.dtx|
to compile the manual |childdoc.pdf| (this file).
\item
Run \LaTeX{} on |childdoc.ins| to create the definitions file |childdoc.def|
and the sample |cdocsamp.tex| with include files
|cdocsch1.tex|, |cdocsch2.tex|, |cdocspt3.tex|, |cdocspt4.tex|,
|cdocsdrf.tex|, |cdocsfn1.tex|, |cdocsfn2.tex|.
Then copy the file |childdoc.def| to an appropriate directory of your \LaTeX{}
distribution, e.g.\ \textit{texmf-root}|/tex/latex/childdoc|.
\end{itemize}

%%%%%%%%%%%%%%%%%%%%%%%%%%%%%%%%%%%%%%%%%%%%%%%%%%%%%%%%%%%%%%%%%%%%%%%%%%%%%%%%
\subsection{Related CTAN Packages}

There are several other packages which offer a similar functionality:
%
\begin{itemize}
\item
The packages
\href{http://ctan.org/pkg/docmute}{\textsf{docmute}},
\href{http://ctan.org/pkg/includex}{\textsf{includex}} and
\href{http://ctan.org/pkg/standalone}{\textsf{standalone}}
provide commands to include only the document body of
a child file thus allowing both files to be compiled individually.
\item
The packages \href{http://ctan.org/pkg/subdocs}{\textsf{subdocs}}
and \href{http://ctan.org/pkg/subfiles}{\textsf{subfiles}}
provide structures in which the main and child documents can be
encapsulated and allowing them to be compiled individually.
The inclusion mechanism is different from the conventional |\include|.
\item
The package \href{http://ctan.org/pkg/combine}{\textsf{combine}}
is an elaborate solution to combine several documents into one.
\end{itemize}
%
See also the CTAN topic \href{http://ctan.org/topic/subdocs}{\textsf{subdocs}}
for further related packages.
The present package differs from the above solutions in that
a document structure constructed with the conventional |\include| mechanism
just needs two extra commands at the top of every file
such that all constituent files can be compiled individually.

%%%%%%%%%%%%%%%%%%%%%%%%%%%%%%%%%%%%%%%%%%%%%%%%%%%%%%%%%%%%%%%%%%%%%%%%%%%%%%%%
%\subsection{Feature Suggestions}
%
%The following is a list of features which may be useful for future
%versions of this package:
%%
%\begin{itemize}
%\item
%\ldots
%\end{itemize}

%%%%%%%%%%%%%%%%%%%%%%%%%%%%%%%%%%%%%%%%%%%%%%%%%%%%%%%%%%%%%%%%%%%%%%%%%%%%%%%%
\subsection{Revision History}

%%%%%%%%%%%%%%%%%%%%%%%%%%%%%%%%%%%%%%%%
\paragraph{v2.0:} 2018/12/30

\begin{itemize}
\item
immediate forward processing
\item
added |\childdocby| mechanism
\item
manual restructured
\end{itemize}

%%%%%%%%%%%%%%%%%%%%%%%%%%%%%%%%%%%%%%%%
\paragraph{v1.6:} 2018/01/17

\begin{itemize}
\item
application for development of include files
\item
corrections to manual
\end{itemize}

%%%%%%%%%%%%%%%%%%%%%%%%%%%%%%%%%%%%%%%%
\paragraph{v1.5:} 2017/05/21

\begin{itemize}
\item
more complete structuring introduced
\item
|\childdocof| introduced
\item
|\childdoc| renamed to |\childdocmain|
\item
|\childredirect| renamed to |\childdocforward| and |\childdocforwardprefix|
and functionality expanded
\end{itemize}

%%%%%%%%%%%%%%%%%%%%%%%%%%%%%%%%%%%%%%%%
\paragraph{v1.0:} 2017/04/27

\begin{itemize}
\item
manual and install package
\item
first version published on CTAN
\end{itemize}

%%%%%%%%%%%%%%%%%%%%%%%%%%%%%%%%%%%%%%%%
\paragraph{v0.6:} 2017/04/26

\begin{itemize}
\item
redirection mechanism added
\end{itemize}

%%%%%%%%%%%%%%%%%%%%%%%%%%%%%%%%%%%%%%%%
\paragraph{v0.5:} 2017/04/26

\begin{itemize}
\item
functionality in definition file
\end{itemize}


%%%%%%%%%%%%%%%%%%%%%%%%%%%%%%%%%%%%%%%%%%%%%%%%%%%%%%%%%%%%%%%%%%%%%%%%%%%%%%%%
%%%%%%%%%%%%%%%%%%%%%%%%%%%%%%%%%%%%%%%%%%%%%%%%%%%%%%%%%%%%%%%%%%%%%%%%%%%%%%%%
%%%%%%%%%%%%%%%%%%%%%%%%%%%%%%%%%%%%%%%%%%%%%%%%%%%%%%%%%%%%%%%%%%%%%%%%%%%%%%%%
\appendix

\settowidth\MacroIndent{\rmfamily\scriptsize 000\ }

 \DocInput{childdoc.dtx}

\end{document}
%</driver>
% \fi
%
% %%%%%%%%%%%%%%%%%%%%%%%%%%%%%%%%%%%%%%%%%%%%%%%%%%%%%%%%%%%%%%%%%%%%%%%%%%%%%%
% %%%%%%%%%%%%%%%%%%%%%%%%%%%%%%%%%%%%%%%%%%%%%%%%%%%%%%%%%%%%%%%%%%%%%%%%%%%%%%
% \section{Sample}
%\iffalse
%<*samplemain>
%\fi
%
% The following presents a sample document
% with two chapters, two parts, a title page,
% a compile flag as well as three forwarding files to set the flag.
% It consists of eight |.tex| files:
% \begin{center}
% \begin{tabular}{ll}
% |cdocsamp.tex|&main file\\
% |cdocsch1.tex|&include file for chapter 1\\
% |cdocsch2.tex|&include file for chapter 2\\
% |cdocspt3.tex|&include file for part 3\\
% |cdocspt4.tex|&include file for part 4\\
% |cdocsdrf.tex|&forwarding file for main file in draft mode\\
% |cdocsfi1.tex|&forwarding file for final version of chapter 1\\
% |cdocsfi2.tex|&forwarding file for final version of chapter 2\\
% \end{tabular}
% \end{center}
% Each of the eight files can be compiled directly by the \LaTeX{} compiler.
%
% %%%%%%%%%%%%%%%%%%%%%%%%%%%%%%%%%%%%%%
% \paragraph{Main File.}
%
% The main file is called |cdocsamp.tex|.
%
% Load the \textsf{childdoc} definitions and
% declare the filename for the main document:
%    \begin{macrocode}
\input{childdoc.def}
\childdocmain{}
%    \end{macrocode}

% Optional override for |\version| flag:
%    \begin{macrocode}
%%\ifchilddoc\else\providecommand{\version}{draft}\fi
%    \end{macrocode}

% Define the default values for the |\version| flag
% (|final| for the main file and |draft| for childs):
%    \begin{macrocode}
\ifchilddoc
\providecommand{\version}{draft}
\else
\providecommand{\version}{final}
\fi
%    \end{macrocode}

% Load the standard document class:
%    \begin{macrocode}
\documentclass[12pt]{article}
%    \end{macrocode}

% Start the document body:
%    \begin{macrocode}
\begin{document}
%    \end{macrocode}

% Declare a title page.
% Print title, part of document being processed and version flag:
%    \begin{macrocode}
\addtocounter{page}{-1}
\begin{center}
{\LARGE\bfseries{}childdoc example\par}
\vspace{1cm}
\ifchilddoc
\ifchilddocmanual part\else chapter\fi:
`\childdocname' of `\childdocjob'\par
\else
main document: `\childdocjob'\par
\fi
version: \version\par
\end{center}
\newpage
%    \end{macrocode}

% Manually include selected file,
% otherwise process as usual:
%    \begin{macrocode}
\ifchilddocmanual
\section*{part `\childdocname'}
\input{\childdocname}
\else
%    \end{macrocode}

% Include the two chapters:
%    \begin{macrocode}
\include{cdocsch1}
\include{cdocsch2}
%    \end{macrocode}

% Include the two parts unless only chapters should be displayed:
%    \begin{macrocode}
\ifchilddoc\else
\section{part three}
\input{cdocspt3}
\section{part four}
\input{cdocspt4}
\fi
%    \end{macrocode}

% Process as usual until here:
%    \begin{macrocode}
\fi
%    \end{macrocode}

% End of document body:
%    \begin{macrocode}
\end{document}
%    \end{macrocode}
%\iffalse
%</samplemain>
%\fi
%
% %%%%%%%%%%%%%%%%%%%%%%%%%%%%%%%%%%%%%%
% \paragraph{Chapter Include Files.}
%
% The include files are called |cdocsch1.tex| and |cdocsch2.tex|.
%
%\iffalse
%<*samplechap1|samplechap2>
%\fi

% Optional override for |\version| flag:
%    \begin{macrocode}
%%\providecommand{\version}{final}
%    \end{macrocode}

% Include the main document:
%    \begin{macrocode}
\input{childdoc.def}
\childdocof{cdocsamp}
%    \end{macrocode}

%\iffalse
%</samplechap1|samplechap2>
%\fi
%
%\iffalse
%<*samplechap1>
%\fi
% Some text for chapter 1:
%    \begin{macrocode}
\section{one}
some text in chapter one
%    \end{macrocode}

%\iffalse
%</samplechap1>
%\fi
% Some text for chapter 2:
%\iffalse
%<*samplechap2>
%\fi
%    \begin{macrocode}
\section{two}
more text in chapter two
%    \end{macrocode}

%\iffalse
%</samplechap2>
%\fi
%
% %%%%%%%%%%%%%%%%%%%%%%%%%%%%%%%%%%%%%%
% \paragraph{Part Include Files.}
%
% The include files are called |cdocspt3.tex| and |cdocspt4.tex|.
%
%\iffalse
%<*samplepart3|samplepart4>
%\fi

% Optional override for |\version| flag:
%    \begin{macrocode}
%%\providecommand{\version}{final}
%    \end{macrocode}

% Include the main document:
%    \begin{macrocode}
\input{childdoc.def}
\childdocby{cdocsamp}
%    \end{macrocode}

%\iffalse
%</samplepart3|samplepart4>
%\fi
%
%\iffalse
%<*samplepart3>
%\fi
% Some text for part 3:
%    \begin{macrocode}
some text in part three
%    \end{macrocode}

%\iffalse
%</samplepart3>
%\fi
% Some text for part 4:
%\iffalse
%<*samplepart4>
%\fi
%    \begin{macrocode}
more text in part four
%    \end{macrocode}

%\iffalse
%</samplepart4>
%\fi
%
% %%%%%%%%%%%%%%%%%%%%%%%%%%%%%%%%%%%%%%
% \paragraph{Forwarding for a Complete Draft.}
%
% The following forwarding file |cdocsdrf.tex|
% compiles the main document in draft mode:
%\iffalse
%<*sampledraft>
%\fi
%    \begin{macrocode}
\def\version{draft}
\input{childdoc.def}
\childdocforward{cdocsamp}
%    \end{macrocode}

%\iffalse
%</sampledraft>
%\fi
%
% %%%%%%%%%%%%%%%%%%%%%%%%%%%%%%%%%%%%%%
% \paragraph{Forwarding for Final Version of the Chapters.}
%
% The following forwarding files |cdocsfn1.tex| and |cdocsfn2.tex|
% (with identical content)
% compile the final versions of the child documents
% |cdocsch1.tex| and |cdocsch2.tex|, respectively:
%\iffalse
%<*samplefinal>
%\fi
%    \begin{macrocode}
\def\version{final}
\input{childdoc.def}
\childdocforwardprefix[cdocsamp]{cdocsfn}{cdocsch}
%    \end{macrocode}

%\iffalse
%</samplefinal>
%\fi
%
% %%%%%%%%%%%%%%%%%%%%%%%%%%%%%%%%%%%%%%
% \paragraph{Command Line Processing.}
%
% The following three command lines generate the output files
% |cdocscld|, |cdocscl1| and |cdocscl2|
% which should be identical to
% |cdocsdrf|, |cdocsch1| and |cdocsfn2|, respectively:
% \begin{center}
% \begin{tabular}{l}
% |latex -jobname cdocscld \|\\
% |  "\def\version{draft}\input{childdoc.def}\childdocforward{cdocsamp}"|\\
% |latex -jobname cdocscl1 \|\\
% |  "\input{childdoc.def}\childdocforward[cdocsamp]{cdocsch1}"|\\
% |latex -jobname cdocscl2 \|\\
% |  "\def\version{final}\input{childdoc.def}\childdocforward{cdocsch2}"|
% \end{tabular}
% \end{center}
% Note that the trailing backslash on each first line
% merely continues the input to the second line
% (for convenient cut ant paste).
% Furthermore, the command |latex| can be replaced by any
% of its alternative versions such as |pdflatex|.
%
% %%%%%%%%%%%%%%%%%%%%%%%%%%%%%%%%%%%%%%%%%%%%%%%%%%%%%%%%%%%%%%%%%%%%%%%%%%%%%%
% %%%%%%%%%%%%%%%%%%%%%%%%%%%%%%%%%%%%%%%%%%%%%%%%%%%%%%%%%%%%%%%%%%%%%%%%%%%%%%
% \section{Implementation}
%\iffalse
%<*package>
%\fi
%
% This section describes the definitions file |childdoc.def|.

% The definitions cannot be loaded using |\usepackage| or |\RequirePackage|
% which has a mechanism to prevent loading a style file more than once.
% When loading the definitions by means of |\input|
% multiple instances have to be prevented manually:
%\iffalse
%This code needs to be before the `\ProvidesFile' directive
%which is defined at the beginning of this file.
%Therefore it is also placed there and commented out here.
%</package>
%<*discard>
%\fi
%    \begin{macrocode}
\ifdefined\childdocmain\endinput\fi
%    \end{macrocode}
%\iffalse
%</discard>
%<*package>
%\fi
%
% \macro{\ifchilddoc}
% \macro{\ifchilddocmanual}
% The conditional |\ifchilddoc| tells whether a
% child (true) or main (false) document is being compiled.
% The conditional |\ifchilddocmanual| tells whether
% the |\includeonly| mechanism is used (false) or
% the selection of child files must be performed manually (true).
% The definitions initialise to false:
%    \begin{macrocode}
\newif\ifchilddoc
\newif\ifchilddocmanual
%    \end{macrocode}

% \macro{\childdocname}
% \macro{\childdocjob}
% The macro |\childdocname| stores the name of the main document
% to be compiled. The macro |\childdocjob| stores the name of
% the document on which the \LaTeX{} compiler was originally invoked.
% The content of |\jobname| cannot be compared
% to filenames specified in the source due to different catcodes.
% The following code rescans |\jobname|, stores the result
% in |\childdocname| and saves a copy in |\childdocjob|:
%    \begin{macrocode}
\edef\childdocname{\scantokens\expandafter{\jobname\noexpand}}
\let\childdocjob\childdocname
%    \end{macrocode}

% \macro{\childdocdisable}
% The macro |\childdocdisable| prevents the main file
% from being processed more than once.
% At this stage, the main document command |\childdocmain|
% is assumed to be called once again where it should do nothing.
% Any subsequent call to it should prevent
% a secondary processing of the main document
% It overwrites the forwarding commands
% |\childdocof| and |\childdocforward|
% with empty macros to prevent further inclusions of the main document:
%    \begin{macrocode}
\newcommand{\childdocdisable}
{
  \renewcommand{\childdocmain}[1]{\renewcommand{\childdocmain}[1]{\endinput}}
  \renewcommand{\childdocof}[1]{}
  \renewcommand{\childdocby}[2][]{}
  \renewcommand{\childdocforward}[2][]{}
  \renewcommand{\childdocdisable}{}
}
%    \end{macrocode}

% \macro{\childdocmain}
% The macro |\childdocmain| is to be called at the top of the main file
% with nothing or the main filename (without extension) as argument.
% First, it breaks loops.
% If the argument is not empty and does not match |\childdocname|
% (which is set by the first inclusion of |childdoc.def|),
% |\ifchilddoc| is set to true, |\includeonly| is applied to the child file
% and |\jobname| is set to the main file
% (for proper handling of |.aux| files):
%    \begin{macrocode}
\newcommand{\childdocmain}[1]
{
  \childdocdisable\childdocmain{}
  \if?#1?\else
    \begingroup
      \def\childdoctmp{#1}
      \ifx\childdoctmp\childdocname
        \def\childdoctmp{}
      \else
        \def\childdoctmp
        {
          \childdoctrue
          \includeonly{\childdocname}
          \def\childdocjob{#1}
          \def\jobname{#1}
        }
      \fi
      \expandafter
    \endgroup
    \childdoctmp
  \fi
}
%    \end{macrocode}

% \macro{\childdocof}
% The command |\childdocof| redirects
% compilation to the main file |#1|.
%    \begin{macrocode}
\newcommand{\childdocof}[1]
{
  \childdocdisable
  \childdoctrue
  \includeonly{\childdocname}
  \def\jobname{#1}
  \def\childdocjob{#1}
  \input{#1}
}
%    \end{macrocode}

% \macro{\childdocby}
% The command |\childdocby| ....
%    \begin{macrocode}
\newcommand{\childdocby}[2][]
{
  \childdocdisable
  \childdoctrue
  \childdocmanualtrue
  \if?#1?\else
    \def\jobname{#2}
  \fi
  \def\childdocjob{#2}
  \input{#2}
  \endinput
}
%    \end{macrocode}

% \macro{\childdocforward}
% The command |\childdocforward| redirects
% compilation to the main file or
% (if the optional argument is given) a child file.
% Parameters are set as if the main file
% or a child file starting with |\childdocof| was compiled.
% Then compilation is handed over to the main file:
%    \begin{macrocode}
\newcommand{\childdocforward}[2][]
{
  \begingroup
    \if?#1?
      \def\childdoctmp
      {
        \def\childdocname{#2}
        \def\childdocjob{#2}
        \def\jobname{#2}
        \input{#2}
        \endinput
      }
    \else
      \def\childdoctmp
      {
        \childdocdisable
        \def\childdocname{#2}
        \childdoctrue
        \includeonly{#2}
        \def\childdocjob{#1}
        \def\jobname{#1}
        \input{#1}
        \endinput
      }
    \fi
    \expandafter
  \endgroup
  \childdoctmp
}
%    \end{macrocode}

% \macro{\childdocforwardprefix}
% The command |\childdocforwardprefix| redirects
% compilation to the main or a child file by means of a pattern.
% The prefix |#1| in the current filename is replaced by |#2|
% and the suffix of the current filename is kept
% (it is assumed that the filename does not contain the substring `|~~~|'
% which is used as a delimiter).
% Compilation is handed over to the new file by |\childdocforward|:
%    \begin{macrocode}
\newcommand{\childdocforwardprefix}[3][]
{
  \begingroup
    \def\childdocextract #2##1~~~{\def\childdoctmp{\childdocforward[#1]{#3##1}}}
    \expandafter\childdocextract\childdocname~~~
    \expandafter
  \endgroup
  \childdoctmp
}
%    \end{macrocode}

% \macro{\childdoc}
% The deprecated macro |\childdoc| is a legacy version of |\childdocmain|:
%    \begin{macrocode}
\newcommand{\childdoc}{\childdocmain}
%    \end{macrocode}

% \macro{\childdocredirect}
% The deprecated macro |\childdocredirect| is a legacy version
% of |\childdocforward| and |\childdocforwardprefix|:
%    \begin{macrocode}
\newcommand{\childdocredirect}[2][]
{
  \begingroup
    \if?#1?
      \def\childdoctmp{\childdocforward{#2}}
    \else
      \def\childdoctmp{\childdocforwardprefix{#1}{#2}}
    \fi
    \expandafter
  \endgroup
  \childdoctmp
}
%    \end{macrocode}

%\iffalse
%</package>
%\fi
%
\endinput
|\\
|\childdocforwardprefix{final}{child}|
\end{tabular}
\end{center}
%

Note that when several versions of a main file and/or of each child file
are to be generated, it may be convenient to set up a |Makefile| or
shell script to automatise the process.

%%%%%%%%%%%%%%%%%%%%%%%%%%%%%%%%%%%%%%%%%%%%%%%%%%%%%%%%%%%%%%%%%%%%%%%%%%%%%%%%
\subsection{Command Line Processing}
\label{sec:commandline}

The effect of redirection files can also be achieved by invoking
the \LaTeX{} compiler with a more elaborate command line.
Most conveniently this should be done as part
of a shell script or a |Makefile|.

When using \textsf{childdoc} in the main file, the following
command lines effectively perform a redirection
(note that depending on the shell being used,
backslashes may have to be doubled: `|\|' $\to$ `|\\|'):
%
\begin{center}
|... -jobname "|\textit{target}|" |\\|"|[\textit{flags}]%
|% \iffalse
%
% childdoc.dtx Copyright (C) 2017-2018 Niklas Beisert
%
% This work may be distributed and/or modified under the
% conditions of the LaTeX Project Public License, either version 1.3
% of this license or (at your option) any later version.
% The latest version of this license is in
%   http://www.latex-project.org/lppl.txt
% and version 1.3 or later is part of all distributions of LaTeX
% version 2005/12/01 or later.
%
% This work has the LPPL maintenance status `maintained'.
%
% The Current Maintainer of this work is Niklas Beisert.
%
% This work consists of the files childdoc.dtx and childdoc.ins
% and the derived files childdoc.def and cdocsamp.tex with
% cdocsch1.tex, cdocsch2.tex, cdocsdrf.tex, cdocsfn1.tex, cdocsfn2.tex.
%
%<package>\ifdefined\childdocmain\endinput\fi
%<package>\ProvidesFile{childdoc.def}[2018/12/30 v2.0 child document driver]
%<samplemain>\ProvidesFile{cdocsamp.tex}[2018/12/30 v2.0 sample for childdoc]
%<*driver>
%\ProvidesFile{childdoc.drv}[2018/12/30 v2.0 childdoc reference manual file]
\PassOptionsToClass{10pt,a4paper}{article}
\documentclass{ltxdoc}

\usepackage[margin=35mm]{geometry}
\usepackage{hyperref}
\usepackage{hyperxmp}
\usepackage[usenames]{color}

\hypersetup{colorlinks=true}
\hypersetup{pdfstartview=FitH}
\hypersetup{pdfpagemode=UseNone}
\hypersetup{pdfsource={}}
\hypersetup{pdflang={en-UK}}
\hypersetup{pdfcopyright={Copyright 2017-2018 Niklas Beisert.
  This work may be distributed and/or modified under the
  conditions of the LaTeX Project Public License, either version 1.3
  of this license or (at your option) any later version.}}
\hypersetup{pdflicenseurl={http://www.latex-project.org/lppl.txt}}
\hypersetup{pdfcontactaddress={ETH Zurich, ITP, HIT K,
  Wolfgang-Pauli-Strasse 27}}
\hypersetup{pdfcontactpostcode={8093}}
\hypersetup{pdfcontactcity={Zurich}}
\hypersetup{pdfcontactcountry={Switzerland}}
\hypersetup{pdfcontactemail={nbeisert@itp.phys.ethz.ch}}
\hypersetup{pdfcontacturl={http://people.phys.ethz.ch/\xmptilde nbeisert/}}

\newcommand{\secref}[1]{\hyperref[#1]{section \ref*{#1}}}

\parskip1ex
\parindent0pt
\let\olditemize\itemize
\def\itemize{\olditemize\parskip0pt}

\begin{document}

\title{The \textsf{childdoc} Package}
\hypersetup{pdftitle={The childdoc Package}}
\author{Niklas Beisert\\[2ex]
  Institut f\"ur Theoretische Physik\\
  Eidgen\"ossische Technische Hochschule Z\"urich\\
  Wolfgang-Pauli-Strasse 27, 8093 Z\"urich, Switzerland\\[1ex]
  \href{mailto:nbeisert@itp.phys.ethz.ch}
  {\texttt{nbeisert@itp.phys.ethz.ch}}}
\hypersetup{pdfauthor={Niklas Beisert}}
\hypersetup{pdfsubject={Manual for the LaTeX2e Package childdoc}}
\date{30 December 2018, \textsf{v2.0}}
\maketitle

\begin{abstract}\noindent
\textsf{childdoc} is a \LaTeXe{} package
that enables the direct compilation
of document sections included by |\include|
to individual files.
\end{abstract}

\begingroup
\parskip0ex
\tableofcontents
\endgroup

%%%%%%%%%%%%%%%%%%%%%%%%%%%%%%%%%%%%%%%%%%%%%%%%%%%%%%%%%%%%%%%%%%%%%%%%%%%%%%%%
%%%%%%%%%%%%%%%%%%%%%%%%%%%%%%%%%%%%%%%%%%%%%%%%%%%%%%%%%%%%%%%%%%%%%%%%%%%%%%%%
\section{Introduction}

\LaTeX{} provides a mechanism to structure a large document (such as a book)
into a main file and several child files (containing the chapters)
using the |\include| command.
This mechanism is beneficial for documents
which span hundreds of pages in order to
make the source file(s) more manageable.
Moreover, compilation can be restricted to
selected child files by means of the |\includeonly| command.
The latter feature can be used to reduce the compilation time while editing
(this was significantly more useful in the earlier days of \LaTeX{})
or to generate a smaller document which is easier to navigate.
Another application of |\includeonly| is to generate
documents consisting of selected parts of the complete document.

However, there are a few drawbacks of the plain |\include| mechanism:
\begin{itemize}
\item
The child files cannot be compiled on their own,
they can only be compiled via the main file.
A naive editing environment
(such as a text editor with an option
to have the current file processed by \LaTeX)
may require one to switch to the main file before compiling;
attempting to compile the child file produces errors.
\item
The main file must be modified (each time)
to adjust the |\includeonly| command
to the present needs. This easily leaves the main file in a messy state.
\item
The generated document will always carry the filename
of the main document. This is inconvenient if
several child files are to be compiled and
to be kept for distribution.
\end{itemize}

The present package provides a simple interface
to make child files individually compilable by \LaTeX{}.
Compiling a child file then has the same effect as compiling
the main file with an |\includeonly| command
to select the appropriate child.
Moreover the generated document will carry the name of the child
rather than the main file.
This resolves all three above issues.

This feature is meant to make the editing of books,
thesis documents and lecture notes somewhat more convenient.
However, the package can also be used efficiently for
composing a series of documents (such as exercise sheets)
which are typically distributed individually.
It then assists the author in generating the individual documents
(potentially in different versions)
as well as a document containing the collected series.
Another application is in developing style files
or other kinds of included material
where compilation of the style file could redirect
to a sample or test file.

%%%%%%%%%%%%%%%%%%%%%%%%%%%%%%%%%%%%%%%%%%%%%%%%%%%%%%%%%%%%%%%%%%%%%%%%%%%%%%%%
%%%%%%%%%%%%%%%%%%%%%%%%%%%%%%%%%%%%%%%%%%%%%%%%%%%%%%%%%%%%%%%%%%%%%%%%%%%%%%%%
\section{Usage}

First of all, the package \textsf{childdoc} is \emph{not} a standard
\LaTeXe{} |.sty| style file! Therefore it needs to be invoked in
a non-standard way.

%%%%%%%%%%%%%%%%%%%%%%%%%%%%%%%%%%%%%%%%%%%%%%%%%%%%%%%%%%%%%%%%%%%%%%%%%%%%%%%%
\subsection{Included Files}
\label{sec:include}

%%%%%%%%%%%%%%%%%%%%%%%%%%%%%%%%%%%%%%%%
\DescribeMacro{\childdocmain}
To use the package, add the commands
\begin{center}
\begin{tabular}{l}
|\input{childdoc.def}|\\
|\childdocmain{}|\\
\end{tabular}
\end{center}
at the very top of the main \LaTeX{} file,
in particular \emph{before} the |\documentclass| statement!
The argument of |\childdocmain| should be left empty
(but it must be present).

%%%%%%%%%%%%%%%%%%%%%%%%%%%%%%%%%%%%%%%%
\DescribeMacro{\childdocof}
Furthermore, add the commands
\begin{center}
\begin{tabular}{l}
|\input{childdoc.def}|\\
|\childdocof{|\textit{main}|}|\\
\end{tabular}
\end{center}
at the top of every child file \textit{child}
which is included by |\include{|\textit{child}|}|
from within the main file
(or at least for those files to be compiled individually).
The argument \textit{main} must be the filename of the main file.

There are a couple of
considerations in setting up the main and child documents:

%%%%%%%%%%%%%%%%%%%%%%%%%%%%%%%%%%%%%%%%
\paragraph{Restrictions.}

Please note the following restrictions:
\begin{itemize}
\item
|\childdocmain| must be called with one argument \textit{main}
to ensure compatibility with earlier version of the package.
It must either be empty (|\childdocmain{}|)
or precisely match the filename of the main file in which it is specified.
See \secref{sec:detection} for further information.
\item
The filename \textit{main} must be specified without the |.tex| extension.
\item
The filename \textit{main} is case sensitive
(even in case-insensitive file systems)
due to internal string comparison.
\item
The argument \textit{main} should be fully expanded, it cannot be a macro.
\item
Subdirectories and special characters should be avoided in filenames.
\item
The command |\childdocmain{|\textit{main}|}| must be followed by a whitespace.
It should not be followed immediately by another command
or by a comment mark `|%|'.
This is because the \TeX{} parser reads the token immediately following
the argument of |\childdocmain| and puts it
at the beginning of every child section;
however, a white\-space is ignored.
\end{itemize}

%%%%%%%%%%%%%%%%%%%%%%%%%%%%%%%%%%%%%%%%
\paragraph{Content of Main File.}

It is advisable to place all content in the child files included by |\include|.
Any output contained in the main file will appear in all child documents
unless suppressed manually;
it cannot be suppressed automatically by the |\includeonly| directive
and thus should normally be avoided.
A method to include some content in the main file
by means of conditional processing is described in \secref{sec:conditional}.

%%%%%%%%%%%%%%%%%%%%%%%%%%%%%%%%%%%%%%%%
\paragraph{Page Numbering.}

When only a part of the document is compiled,
the appropriate numbering of pages
(as well as other status parameters)
is determined from the |.aux| files.
The latter contain information from previous passes.
However this information needs to propagate through
all intermediate child documents.
Therefore the page numbering in child documents may well
be inconsistent until the complete document is compiled at least once.

A useful (if unconventional) way to always ensure a consistent
page numbering is to restart the numbering in each child document
and denote the pages by `\textit{child}|.|\textit{page}'
where \textit{child} represents the chapter/section number of the child file.
This can be achieved by the command
|\numberwithin{page}{|\textit{child}|}|
of the \textsf{amsmath} package
where \textit{child} can be |chapter| or |section|
depending on the chosen structuring.
Alternatively, one can modify the macro |\thepage| appropriately
and reset the counter |page| at the start of each child file.

%%%%%%%%%%%%%%%%%%%%%%%%%%%%%%%%%%%%%%%%%%%%%%%%%%%%%%%%%%%%%%%%%%%%%%%%%%%%%%%%
\subsection{Conditional Processing}
\label{sec:conditional}

The package provides a mechanism to compile different versions
of a document. To customise the versions further some conditional processing
can come in handy to distinguish which version is being compiled.
The package provides two macros to describe the compilation context:

%%%%%%%%%%%%%%%%%%%%%%%%%%%%%%%%%%%%%%%%
\DescribeMacro{\ifchilddoc}
The conditional |\ifchilddoc| distinguishes between the compilation of
child documents and the main document:
%
\begin{center}
|\ifchilddoc |\textit{child-code}| |[|\||else |\textit{main-code}]| \||fi|
\end{center}

%%%%%%%%%%%%%%%%%%%%%%%%%%%%%%%%%%%%%%%%
\DescribeMacro{\childdocname}
\DescribeMacro{\childdocjob}
The macro |\childdocname| contains the filename (without extension)
of the main or child file being processed.
Note that |\childdocjob| will always contain the name of the main file.

%%%%%%%%%%%%%%%%%%%%%%%%%%%%%%%%%%%%%%%%
\paragraph{Title Page.}

Conditional processing can be used to include a title or banner page
in the main document when proper precautions are taken.
Importantly, the code in the main file should ensure that the page counter
(as well as other status parameters which are stored in the |.aux| files)
takes the same value after the conditional processing.
Otherwise the page numbers may take divergent values
depending on which part is compiled.

For example, a title page could be declared by:
%
\begin{center}
\begin{tabular}{l}
|\ifchilddoc\||else|\\
|\addtocounter{page}{-1}|\\
\textit{code for title page}\\
|\newpage|\\
|\||fi|
\end{tabular}
\end{center}
%
A banner page for the child documents can be generated by:
%
\begin{center}
\begin{tabular}{l}
|\ifchilddoc|\\
|\addtocounter{page}{-1}|\\
\textit{code for banner page}\\
|\newpage|\\
|\||fi|
\end{tabular}
\end{center}
%
Here one could write a message such as:
\begin{center}
|This is the part \childdocname{} of \childdocjob{}.|
\end{center}

%%%%%%%%%%%%%%%%%%%%%%%%%%%%%%%%%%%%%%%%%%%%%%%%%%%%%%%%%%%%%%%%%%%%%%%%%%%%%%%%
\subsection{Flags}
\label{sec:flags}

The package makes it easy to generate different versions
of the main or child documents.
To this end compilation flags can be defined
and assigned different default values.
They will be particularly useful in conjunction
with the forwarding mechanism described in \secref{sec:forward}.

For example, it may be useful to have a flag |\version|
which can be set to |draft| or |final|.
The document source will contain some conditional code
depending on the value of |\version|.
Suppose further, the flag should default to |final| for the main file
and to |draft| for child files
which is a natural assignment for editing the document.
This is achieved by placing the following code
in the preamble of the main document
(below the |\childdocmain| directive):
%
\begin{center}
\begin{tabular}{l}
|\ifchilddoc|\\
|\providecommand{\version}{draft}|\\
|\||else|\\
|\providecommand{\version}{final}|\\
|\||fi|
\end{tabular}
\end{center}
%
The definition by |\providecommand| makes sure
that previous definitions are not overwritten.
Further statements |\providecommand{\version}{...}|
can thus be added before the above code to override it.

For the main file, one might add a line
(between |\childdocmain| and the above block)
%
\begin{center}
|%\ifchilddoc\||else\providecommand{\version}{draft}\||fi|
\end{center}
%
which can be uncommented to produce a draft version.
Likewise one can add a line to the very top of a child file
(above the |\childdocof{|\textit{main}|}| directive)
%
\begin{center}
|%\providecommand{\version}{final}|
\end{center}
%
which can be uncommented to produce the final version of this child document.

%%%%%%%%%%%%%%%%%%%%%%%%%%%%%%%%%%%%%%%%%%%%%%%%%%%%%%%%%%%%%%%%%%%%%%%%%%%%%%%%
\subsection{Forwarding}
\label{sec:forward}

Different versions of the main or child documents
using compilation flags as described in \secref{sec:flags}
can be (permanently) stored in different files
for convenient compilation, viewing and distribution.
To this end, the package defines a command
to pass on compilation to a different file:

%%%%%%%%%%%%%%%%%%%%%%%%%%%%%%%%%%%%%%%%
\DescribeMacro{\childdocforward}
The command |\childdocforward| redirects processing to
another source file:
%
\begin{center}
\begin{tabular}{l}
|\input{childdoc.def}|\\
|\childdocforward[|\textit{main}|]{|\textit{dest}|}|\\
\end{tabular}
\end{center}
%
The argument \textit{dest} is the destination file
(without extension).
It should be the main file or one of the child files.
Note that further \textsf{childdoc} directives
such as |\childdocof| and |\childdocforward|
in the indicated file will be processed in this form.
The optional argument \textit{main}
passes on directly to the main file \textit{main}
while pretending to compile the child \textit{dest}.
This form behaves as if \textit{dest}
issues |\childdocof{|\textit{main}|}| right away,
and no further \textsf{childdoc} directives will be processed.

%%%%%%%%%%%%%%%%%%%%%%%%%%%%%%%%%%%%%%%%
\DescribeMacro{\...prefix}
In the alternative form |\childdocforwardprefix|,
%
\begin{center}
\begin{tabular}{l}
|\input{childdoc.def}|\\
|\childdocforwardprefix[|\textit{main}|]{|\textit{prefix}|}{|\textit{dest}|}|
\end{tabular}
\end{center}
%
the destination file is determined by a pattern
depending on the current file:
To make this work, the current file must be called
`{\textit{prefix}\hspace{0.2em}\textit{suffix}}'
with \textit{prefix} matching precisely the argument.
Processing is then passed on to the file
`{\textit{dest}\hspace{0.2em}\textit{suffix}}'.
Surely, the same effect is achieved by
directly specifying the
argument `{\textit{dest}\hspace{0.2em}\textit{suffix}}'
in the first form.
However, that requires to set up a different file
for each child. With the alternative form of the command
all these files can have exactly the same content
which simplifies setting them up and maintaining them.

For example, the following file |draft.tex|
with a compilation flag |\version| as described in \secref{sec:flags}
compiles the main document as a draft:
%
\begin{center}
\begin{tabular}{l}
|\def\version{draft}|\\
|\input{childdoc.def}|\\
|\childdocforward{|\textit{main}|}|
\end{tabular}
\end{center}
%
Likewise, the following files |final|\textit{nn}|.tex|
compile the final version of the child document
|child|\textit{nn}|.tex|:
%
\begin{center}
\begin{tabular}{l}
|\def\version{final}|\\
|\input{childdoc.def}|\\
|\childdocforwardprefix{final}{child}|
\end{tabular}
\end{center}
%

Note that when several versions of a main file and/or of each child file
are to be generated, it may be convenient to set up a |Makefile| or
shell script to automatise the process.

%%%%%%%%%%%%%%%%%%%%%%%%%%%%%%%%%%%%%%%%%%%%%%%%%%%%%%%%%%%%%%%%%%%%%%%%%%%%%%%%
\subsection{Command Line Processing}
\label{sec:commandline}

The effect of redirection files can also be achieved by invoking
the \LaTeX{} compiler with a more elaborate command line.
Most conveniently this should be done as part
of a shell script or a |Makefile|.

When using \textsf{childdoc} in the main file, the following
command lines effectively perform a redirection
(note that depending on the shell being used,
backslashes may have to be doubled: `|\|' $\to$ `|\\|'):
%
\begin{center}
|... -jobname "|\textit{target}|" |\\|"|[\textit{flags}]%
|\input{childdoc.def}\childdocforward[|\textit{main}|]{|\textit{dest}|}"|
\end{center}
%
Here \textit{target} is the name of the output file,
\textit{main} is the name of the main file
and \textit{dest} is the name of the main or child file to be processed
(all filenames without extensions).
The optional argument \textit{main} can be omitted
if \textit{main} matches \textit{dest}.
Optionally, compilation \textit{flags} can be defined via |\def| commands.
This command line makes the \TeX{} engine believe
it is compiling the file \textit{target}
whose content is specified as the latter parameter.
The provided code then forwards the processing to
\textit{main} or \textit{dest} as described in \secref{sec:forward}.

%%%%%%%%%%%%%%%%%%%%%%%%%%%%%%%%%%%%%%%%%%%%%%%%%%%%%%%%%%%%%%%%%%%%%%%%%%%%%%%%
\subsection{Include by Input}
\label{sec:input}

Including child documents by |\include| has some restrictions by design.
Most notably, the content of a child document always occupies
its own set of pages; pages cannot be shared between child documents.
Usually, this behaviour makes perfect sense
because each child document contain an essential part of the document.
However, in some situations it may be desirable to compose
a document from a collection of parts
without having mandatory page breaks between then.
For this case, the package
provides a mechanism to include parts
by |\input| which can also be processed individually.
However, by construction this mechanism
requires manual handling of the content to be output.

%%%%%%%%%%%%%%%%%%%%%%%%%%%%%%%%%%%%%%%%
\DescribeMacro{\ifchilddocmanual}
The main file should be prepared as usual, see \secref{sec:include}.
However, the document body must make a distinction
between processing of an individual part and of the main document, e.g.:
%
\begin{center}
\begin{tabular}{l}
|\ifchilddocmanual|\\
|\input{\childdocname}|\\
|\||else|\\
\textit{document body with }|\input{|\textit{part}|}|\\
|\||fi|
\end{tabular}
\end{center}
%
The conditional |\ifchilddocmanual| is true whenever
a part to be included by |\input| is being compiled,
and the name of the part is stored in |\childdocname|.

%%%%%%%%%%%%%%%%%%%%%%%%%%%%%%%%%%%%%%%%
\DescribeMacro{\childdocby}
Each part to be included by |\input| should start with:
%
\begin{center}
\begin{tabular}{l}
|\input{childdoc.def}|\\
|\childdocby{|\textit{main}|}|\\
\end{tabular}
\end{center}
%
The directive |\childdocby| is similar to |\childdocof|
described in \secref{sec:include},
but the subsequent selection of content must be done manually.
To that end, both |\ifchilddoc| and |\ifchilddocmanual|
will be true upon processing of a part,
and the name of the part is stored in |\childdocname|.
Note that |\jobname| will be set to the filename of the current part
so that each part receives an individual |.aux| file
that does not interfere with the |.aux| file(s) of the main document.
This behaviour can be altered by the alternative form
|\childdocby[*]{|\textit{main}|}| (with a non-empty optional argument)
which uses the |.aux| file of the main document
by setting |\jobname| to \textit{main}.

%%%%%%%%%%%%%%%%%%%%%%%%%%%%%%%%%%%%%%%%%%%%%%%%%%%%%%%%%%%%%%%%%%%%%%%%%%%%%%%%
\subsection{Driver Development}
\label{sec:driver}

The \textsf{childdoc} mechanism can also be use for the development
of definition files such as \LaTeX{} styles or classes.
This case differs from the above setup with multiple parts
included by |\include| in that no |\includeonly| should be invoked.
This can be achieved by starting the include file
(before |\ProvidesPackage|) with:
%
\begin{center}
\begin{tabular}{l}
|\input{childdoc.def}|\\
|\childdocforward{|\textit{main}|}|\\
\end{tabular}
\end{center}
%
or alternatively with:
%
\begin{center}
\begin{tabular}{l}
|\input{childdoc.def}|\\
|\childdocby{|\textit{main}|}|\\
\end{tabular}
\end{center}
%
Both forms have slightly different effects as described above.
The main file is prepared as usual, see \secref{sec:include}.

%%%%%%%%%%%%%%%%%%%%%%%%%%%%%%%%%%%%%%%%%%%%%%%%%%%%%%%%%%%%%%%%%%%%%%%%%%%%%%%%
\subsection{Legacy Detection}
\label{sec:detection}

The directive |\childdocmain| in the main file can detect
whether the complete document or merely a child is to be compiled
even without using the directive |\childdocof|.
This method is deprecated because it is less robust
and there is no compelling reason to use it;
it is merely provided for backward compatibility
and it may be removed in future versions.

If the detection mechanism is to be used,
it is mandatory to correctly specify
the filename of the main file as the argument of |\childdocmain|:
%
\begin{center}
\begin{tabular}{l}
|\input{childdoc.def}|\\
|\childdocmain{|\textit{main}|}|\\
\end{tabular}
\end{center}
%
If |\jobname| does not match the argument \textit{main} of |\childdocmain|,
it is assumed that |\jobname| points to the child file to be compiled.
When using |\childdocmain| with the main file specified as argument,
it suffices to start a child file
with just |\input{|\textit{main}|}|
without loading of the package and using |\childdocof|.
If instead all processing is done
with the appropriate \textsf{childdoc} directives,
the argument of \textit{main} of |\childdocmain| can be empty.

An alternative version of the command line processing described
in \secref{sec:commandline} using the detection mechanism reads:
%
\begin{center}
|... -jobname "|\textit{target}|" "|[\textit{flags}]%
[|\def\jobname{|\textit{dest}|}|]|\input{|\textit{main}|}"|
\end{center}

%%%%%%%%%%%%%%%%%%%%%%%%%%%%%%%%%%%%%%%%%%%%%%%%%%%%%%%%%%%%%%%%%%%%%%%%%%%%%%%%
\subsection{Manual Code}
\label{sec:manual}

In case one cannot be certain whether the definitions file |childdoc.def|
is installed on the target \TeX{} distribution
and one prefers not to ship it,
it is conceivable to paste a few relevant commands into the sources.

To that end, drop all statements |\input{childdoc.def}|
and perform the replacements as outlined below.
Instead of |\childdocmain{|\textit{main}|}| add the following code
to the top of the main file:
%
\begin{center}
\begin{tabular}{l}
|\||ifdefined\childdocname\endinput\||fi\newif\ifchilddoc|\\
|\edef\childdocname{\scantokens\expandafter{\jobname\noexpand}}|\\
|\def\childdocmain{|\textit{main}|}\||ifx\childdocmain\childdocname\||else|\\
|\childdoctrue\includeonly{\childdocname}\let\jobname\childdocmain\||fi|\\
\end{tabular}
\end{center}
%
Instead of |\childdocof{|\textit{main}|}| just include the main file
at the top of each child file:
%
\begin{center}
|\input{|\textit{main}|}|
\end{center}
%
A simple redirection |\childdocforward{|\textit{dest}|}| is achieved by:
%
\begin{center}
|\def\jobname{|\textit{dest}|}\input{\jobname}|
\end{center}
%
The redirection with prefix
|\childdocforwardprefix[|\textit{prefix}|]{|\textit{dest}|}|
is accomplished by:
%
\begin{center}
\begin{tabular}{l}
|{\edef\jobname{\scantokens\expandafter{\jobname\noexpand}}|\\
|\def\redirectjob |\textit{prefix}|#1~~~{\gdef\jobname{|\textit{dest}|#1}}|\\
|\expandafter\redirectjob\jobname~~~}\input{\jobname}|
\end{tabular}
\end{center}

In an alternative approach,
child documents can be compiled by a specific command line
without additional code or specific definitions:
%
\begin{center}
|... -jobname "|\textit{target}|" "|[\textit{flags}]%
|\includeonly{|\textit{dest}|}\input{|\textit{main}|}"|
\end{center}
%

%%%%%%%%%%%%%%%%%%%%%%%%%%%%%%%%%%%%%%%%%%%%%%%%%%%%%%%%%%%%%%%%%%%%%%%%%%%%%%%%
%%%%%%%%%%%%%%%%%%%%%%%%%%%%%%%%%%%%%%%%%%%%%%%%%%%%%%%%%%%%%%%%%%%%%%%%%%%%%%%%
\section{Information}

%%%%%%%%%%%%%%%%%%%%%%%%%%%%%%%%%%%%%%%%%%%%%%%%%%%%%%%%%%%%%%%%%%%%%%%%%%%%%%%%
\subsection{Copyright}

Copyright \copyright{} 2017--2018 Niklas Beisert

This work may be distributed and/or modified under the
conditions of the \LaTeX{} Project Public License, either version 1.3
of this license or (at your option) any later version.
The latest version of this license is in
  \url{http://www.latex-project.org/lppl.txt}
and version 1.3 or later is part of all distributions of \LaTeX{}
version 2005/12/01 or later.

This work has the LPPL maintenance status `maintained'.

The Current Maintainer of this work is Niklas Beisert.

This work consists of the files |README.txt|, |childdoc.ins| and |childdoc.dtx|
as well as the derived files |childdoc.def|, |cdocsamp.tex|
with |cdocsch1.tex|, |cdocsch2.tex|, |cdocspt3.tex|, |cdocspt4.tex|,
|cdocsdrf.tex|, |cdocsfn1.tex|, |cdocsfn2.tex|
as well as |childdoc.pdf|.

%%%%%%%%%%%%%%%%%%%%%%%%%%%%%%%%%%%%%%%%%%%%%%%%%%%%%%%%%%%%%%%%%%%%%%%%%%%%%%%%
\subsection{Files and Installation}

The package consists of the files:
%
\begin{center}
\begin{tabular}{ll}
    |README.txt|   & readme file \\
    |childdoc.ins| & installation file \\
    |childdoc.dtx| & source file \\
    |childdoc.def| & definition file \\
    |cdocsamp.tex| & sample main file \\
    |cdocsch1.tex| & sample include file \\
    |cdocsch2.tex| & sample include file \\
    |cdocspt3.tex| & sample part file \\
    |cdocspt4.tex| & sample part file \\
    |cdocsdrf.tex| & sample redirection file \\
    |cdocsfn1.tex| & sample redirection file \\
    |cdocsfn2.tex| & sample redirection file \\
    |childdoc.pdf| & manual
\end{tabular}
\end{center}
%
The distribution consists of the files
|README.txt|, |childdoc.ins| and |childdoc.dtx|.
%
\begin{itemize}
\item
Run (pdf)\LaTeX{} on |childdoc.dtx|
to compile the manual |childdoc.pdf| (this file).
\item
Run \LaTeX{} on |childdoc.ins| to create the definitions file |childdoc.def|
and the sample |cdocsamp.tex| with include files
|cdocsch1.tex|, |cdocsch2.tex|, |cdocspt3.tex|, |cdocspt4.tex|,
|cdocsdrf.tex|, |cdocsfn1.tex|, |cdocsfn2.tex|.
Then copy the file |childdoc.def| to an appropriate directory of your \LaTeX{}
distribution, e.g.\ \textit{texmf-root}|/tex/latex/childdoc|.
\end{itemize}

%%%%%%%%%%%%%%%%%%%%%%%%%%%%%%%%%%%%%%%%%%%%%%%%%%%%%%%%%%%%%%%%%%%%%%%%%%%%%%%%
\subsection{Related CTAN Packages}

There are several other packages which offer a similar functionality:
%
\begin{itemize}
\item
The packages
\href{http://ctan.org/pkg/docmute}{\textsf{docmute}},
\href{http://ctan.org/pkg/includex}{\textsf{includex}} and
\href{http://ctan.org/pkg/standalone}{\textsf{standalone}}
provide commands to include only the document body of
a child file thus allowing both files to be compiled individually.
\item
The packages \href{http://ctan.org/pkg/subdocs}{\textsf{subdocs}}
and \href{http://ctan.org/pkg/subfiles}{\textsf{subfiles}}
provide structures in which the main and child documents can be
encapsulated and allowing them to be compiled individually.
The inclusion mechanism is different from the conventional |\include|.
\item
The package \href{http://ctan.org/pkg/combine}{\textsf{combine}}
is an elaborate solution to combine several documents into one.
\end{itemize}
%
See also the CTAN topic \href{http://ctan.org/topic/subdocs}{\textsf{subdocs}}
for further related packages.
The present package differs from the above solutions in that
a document structure constructed with the conventional |\include| mechanism
just needs two extra commands at the top of every file
such that all constituent files can be compiled individually.

%%%%%%%%%%%%%%%%%%%%%%%%%%%%%%%%%%%%%%%%%%%%%%%%%%%%%%%%%%%%%%%%%%%%%%%%%%%%%%%%
%\subsection{Feature Suggestions}
%
%The following is a list of features which may be useful for future
%versions of this package:
%%
%\begin{itemize}
%\item
%\ldots
%\end{itemize}

%%%%%%%%%%%%%%%%%%%%%%%%%%%%%%%%%%%%%%%%%%%%%%%%%%%%%%%%%%%%%%%%%%%%%%%%%%%%%%%%
\subsection{Revision History}

%%%%%%%%%%%%%%%%%%%%%%%%%%%%%%%%%%%%%%%%
\paragraph{v2.0:} 2018/12/30

\begin{itemize}
\item
immediate forward processing
\item
added |\childdocby| mechanism
\item
manual restructured
\end{itemize}

%%%%%%%%%%%%%%%%%%%%%%%%%%%%%%%%%%%%%%%%
\paragraph{v1.6:} 2018/01/17

\begin{itemize}
\item
application for development of include files
\item
corrections to manual
\end{itemize}

%%%%%%%%%%%%%%%%%%%%%%%%%%%%%%%%%%%%%%%%
\paragraph{v1.5:} 2017/05/21

\begin{itemize}
\item
more complete structuring introduced
\item
|\childdocof| introduced
\item
|\childdoc| renamed to |\childdocmain|
\item
|\childredirect| renamed to |\childdocforward| and |\childdocforwardprefix|
and functionality expanded
\end{itemize}

%%%%%%%%%%%%%%%%%%%%%%%%%%%%%%%%%%%%%%%%
\paragraph{v1.0:} 2017/04/27

\begin{itemize}
\item
manual and install package
\item
first version published on CTAN
\end{itemize}

%%%%%%%%%%%%%%%%%%%%%%%%%%%%%%%%%%%%%%%%
\paragraph{v0.6:} 2017/04/26

\begin{itemize}
\item
redirection mechanism added
\end{itemize}

%%%%%%%%%%%%%%%%%%%%%%%%%%%%%%%%%%%%%%%%
\paragraph{v0.5:} 2017/04/26

\begin{itemize}
\item
functionality in definition file
\end{itemize}


%%%%%%%%%%%%%%%%%%%%%%%%%%%%%%%%%%%%%%%%%%%%%%%%%%%%%%%%%%%%%%%%%%%%%%%%%%%%%%%%
%%%%%%%%%%%%%%%%%%%%%%%%%%%%%%%%%%%%%%%%%%%%%%%%%%%%%%%%%%%%%%%%%%%%%%%%%%%%%%%%
%%%%%%%%%%%%%%%%%%%%%%%%%%%%%%%%%%%%%%%%%%%%%%%%%%%%%%%%%%%%%%%%%%%%%%%%%%%%%%%%
\appendix

\settowidth\MacroIndent{\rmfamily\scriptsize 000\ }

 \DocInput{childdoc.dtx}

\end{document}
%</driver>
% \fi
%
% %%%%%%%%%%%%%%%%%%%%%%%%%%%%%%%%%%%%%%%%%%%%%%%%%%%%%%%%%%%%%%%%%%%%%%%%%%%%%%
% %%%%%%%%%%%%%%%%%%%%%%%%%%%%%%%%%%%%%%%%%%%%%%%%%%%%%%%%%%%%%%%%%%%%%%%%%%%%%%
% \section{Sample}
%\iffalse
%<*samplemain>
%\fi
%
% The following presents a sample document
% with two chapters, two parts, a title page,
% a compile flag as well as three forwarding files to set the flag.
% It consists of eight |.tex| files:
% \begin{center}
% \begin{tabular}{ll}
% |cdocsamp.tex|&main file\\
% |cdocsch1.tex|&include file for chapter 1\\
% |cdocsch2.tex|&include file for chapter 2\\
% |cdocspt3.tex|&include file for part 3\\
% |cdocspt4.tex|&include file for part 4\\
% |cdocsdrf.tex|&forwarding file for main file in draft mode\\
% |cdocsfi1.tex|&forwarding file for final version of chapter 1\\
% |cdocsfi2.tex|&forwarding file for final version of chapter 2\\
% \end{tabular}
% \end{center}
% Each of the eight files can be compiled directly by the \LaTeX{} compiler.
%
% %%%%%%%%%%%%%%%%%%%%%%%%%%%%%%%%%%%%%%
% \paragraph{Main File.}
%
% The main file is called |cdocsamp.tex|.
%
% Load the \textsf{childdoc} definitions and
% declare the filename for the main document:
%    \begin{macrocode}
\input{childdoc.def}
\childdocmain{}
%    \end{macrocode}

% Optional override for |\version| flag:
%    \begin{macrocode}
%%\ifchilddoc\else\providecommand{\version}{draft}\fi
%    \end{macrocode}

% Define the default values for the |\version| flag
% (|final| for the main file and |draft| for childs):
%    \begin{macrocode}
\ifchilddoc
\providecommand{\version}{draft}
\else
\providecommand{\version}{final}
\fi
%    \end{macrocode}

% Load the standard document class:
%    \begin{macrocode}
\documentclass[12pt]{article}
%    \end{macrocode}

% Start the document body:
%    \begin{macrocode}
\begin{document}
%    \end{macrocode}

% Declare a title page.
% Print title, part of document being processed and version flag:
%    \begin{macrocode}
\addtocounter{page}{-1}
\begin{center}
{\LARGE\bfseries{}childdoc example\par}
\vspace{1cm}
\ifchilddoc
\ifchilddocmanual part\else chapter\fi:
`\childdocname' of `\childdocjob'\par
\else
main document: `\childdocjob'\par
\fi
version: \version\par
\end{center}
\newpage
%    \end{macrocode}

% Manually include selected file,
% otherwise process as usual:
%    \begin{macrocode}
\ifchilddocmanual
\section*{part `\childdocname'}
\input{\childdocname}
\else
%    \end{macrocode}

% Include the two chapters:
%    \begin{macrocode}
\include{cdocsch1}
\include{cdocsch2}
%    \end{macrocode}

% Include the two parts unless only chapters should be displayed:
%    \begin{macrocode}
\ifchilddoc\else
\section{part three}
\input{cdocspt3}
\section{part four}
\input{cdocspt4}
\fi
%    \end{macrocode}

% Process as usual until here:
%    \begin{macrocode}
\fi
%    \end{macrocode}

% End of document body:
%    \begin{macrocode}
\end{document}
%    \end{macrocode}
%\iffalse
%</samplemain>
%\fi
%
% %%%%%%%%%%%%%%%%%%%%%%%%%%%%%%%%%%%%%%
% \paragraph{Chapter Include Files.}
%
% The include files are called |cdocsch1.tex| and |cdocsch2.tex|.
%
%\iffalse
%<*samplechap1|samplechap2>
%\fi

% Optional override for |\version| flag:
%    \begin{macrocode}
%%\providecommand{\version}{final}
%    \end{macrocode}

% Include the main document:
%    \begin{macrocode}
\input{childdoc.def}
\childdocof{cdocsamp}
%    \end{macrocode}

%\iffalse
%</samplechap1|samplechap2>
%\fi
%
%\iffalse
%<*samplechap1>
%\fi
% Some text for chapter 1:
%    \begin{macrocode}
\section{one}
some text in chapter one
%    \end{macrocode}

%\iffalse
%</samplechap1>
%\fi
% Some text for chapter 2:
%\iffalse
%<*samplechap2>
%\fi
%    \begin{macrocode}
\section{two}
more text in chapter two
%    \end{macrocode}

%\iffalse
%</samplechap2>
%\fi
%
% %%%%%%%%%%%%%%%%%%%%%%%%%%%%%%%%%%%%%%
% \paragraph{Part Include Files.}
%
% The include files are called |cdocspt3.tex| and |cdocspt4.tex|.
%
%\iffalse
%<*samplepart3|samplepart4>
%\fi

% Optional override for |\version| flag:
%    \begin{macrocode}
%%\providecommand{\version}{final}
%    \end{macrocode}

% Include the main document:
%    \begin{macrocode}
\input{childdoc.def}
\childdocby{cdocsamp}
%    \end{macrocode}

%\iffalse
%</samplepart3|samplepart4>
%\fi
%
%\iffalse
%<*samplepart3>
%\fi
% Some text for part 3:
%    \begin{macrocode}
some text in part three
%    \end{macrocode}

%\iffalse
%</samplepart3>
%\fi
% Some text for part 4:
%\iffalse
%<*samplepart4>
%\fi
%    \begin{macrocode}
more text in part four
%    \end{macrocode}

%\iffalse
%</samplepart4>
%\fi
%
% %%%%%%%%%%%%%%%%%%%%%%%%%%%%%%%%%%%%%%
% \paragraph{Forwarding for a Complete Draft.}
%
% The following forwarding file |cdocsdrf.tex|
% compiles the main document in draft mode:
%\iffalse
%<*sampledraft>
%\fi
%    \begin{macrocode}
\def\version{draft}
\input{childdoc.def}
\childdocforward{cdocsamp}
%    \end{macrocode}

%\iffalse
%</sampledraft>
%\fi
%
% %%%%%%%%%%%%%%%%%%%%%%%%%%%%%%%%%%%%%%
% \paragraph{Forwarding for Final Version of the Chapters.}
%
% The following forwarding files |cdocsfn1.tex| and |cdocsfn2.tex|
% (with identical content)
% compile the final versions of the child documents
% |cdocsch1.tex| and |cdocsch2.tex|, respectively:
%\iffalse
%<*samplefinal>
%\fi
%    \begin{macrocode}
\def\version{final}
\input{childdoc.def}
\childdocforwardprefix[cdocsamp]{cdocsfn}{cdocsch}
%    \end{macrocode}

%\iffalse
%</samplefinal>
%\fi
%
% %%%%%%%%%%%%%%%%%%%%%%%%%%%%%%%%%%%%%%
% \paragraph{Command Line Processing.}
%
% The following three command lines generate the output files
% |cdocscld|, |cdocscl1| and |cdocscl2|
% which should be identical to
% |cdocsdrf|, |cdocsch1| and |cdocsfn2|, respectively:
% \begin{center}
% \begin{tabular}{l}
% |latex -jobname cdocscld \|\\
% |  "\def\version{draft}\input{childdoc.def}\childdocforward{cdocsamp}"|\\
% |latex -jobname cdocscl1 \|\\
% |  "\input{childdoc.def}\childdocforward[cdocsamp]{cdocsch1}"|\\
% |latex -jobname cdocscl2 \|\\
% |  "\def\version{final}\input{childdoc.def}\childdocforward{cdocsch2}"|
% \end{tabular}
% \end{center}
% Note that the trailing backslash on each first line
% merely continues the input to the second line
% (for convenient cut ant paste).
% Furthermore, the command |latex| can be replaced by any
% of its alternative versions such as |pdflatex|.
%
% %%%%%%%%%%%%%%%%%%%%%%%%%%%%%%%%%%%%%%%%%%%%%%%%%%%%%%%%%%%%%%%%%%%%%%%%%%%%%%
% %%%%%%%%%%%%%%%%%%%%%%%%%%%%%%%%%%%%%%%%%%%%%%%%%%%%%%%%%%%%%%%%%%%%%%%%%%%%%%
% \section{Implementation}
%\iffalse
%<*package>
%\fi
%
% This section describes the definitions file |childdoc.def|.

% The definitions cannot be loaded using |\usepackage| or |\RequirePackage|
% which has a mechanism to prevent loading a style file more than once.
% When loading the definitions by means of |\input|
% multiple instances have to be prevented manually:
%\iffalse
%This code needs to be before the `\ProvidesFile' directive
%which is defined at the beginning of this file.
%Therefore it is also placed there and commented out here.
%</package>
%<*discard>
%\fi
%    \begin{macrocode}
\ifdefined\childdocmain\endinput\fi
%    \end{macrocode}
%\iffalse
%</discard>
%<*package>
%\fi
%
% \macro{\ifchilddoc}
% \macro{\ifchilddocmanual}
% The conditional |\ifchilddoc| tells whether a
% child (true) or main (false) document is being compiled.
% The conditional |\ifchilddocmanual| tells whether
% the |\includeonly| mechanism is used (false) or
% the selection of child files must be performed manually (true).
% The definitions initialise to false:
%    \begin{macrocode}
\newif\ifchilddoc
\newif\ifchilddocmanual
%    \end{macrocode}

% \macro{\childdocname}
% \macro{\childdocjob}
% The macro |\childdocname| stores the name of the main document
% to be compiled. The macro |\childdocjob| stores the name of
% the document on which the \LaTeX{} compiler was originally invoked.
% The content of |\jobname| cannot be compared
% to filenames specified in the source due to different catcodes.
% The following code rescans |\jobname|, stores the result
% in |\childdocname| and saves a copy in |\childdocjob|:
%    \begin{macrocode}
\edef\childdocname{\scantokens\expandafter{\jobname\noexpand}}
\let\childdocjob\childdocname
%    \end{macrocode}

% \macro{\childdocdisable}
% The macro |\childdocdisable| prevents the main file
% from being processed more than once.
% At this stage, the main document command |\childdocmain|
% is assumed to be called once again where it should do nothing.
% Any subsequent call to it should prevent
% a secondary processing of the main document
% It overwrites the forwarding commands
% |\childdocof| and |\childdocforward|
% with empty macros to prevent further inclusions of the main document:
%    \begin{macrocode}
\newcommand{\childdocdisable}
{
  \renewcommand{\childdocmain}[1]{\renewcommand{\childdocmain}[1]{\endinput}}
  \renewcommand{\childdocof}[1]{}
  \renewcommand{\childdocby}[2][]{}
  \renewcommand{\childdocforward}[2][]{}
  \renewcommand{\childdocdisable}{}
}
%    \end{macrocode}

% \macro{\childdocmain}
% The macro |\childdocmain| is to be called at the top of the main file
% with nothing or the main filename (without extension) as argument.
% First, it breaks loops.
% If the argument is not empty and does not match |\childdocname|
% (which is set by the first inclusion of |childdoc.def|),
% |\ifchilddoc| is set to true, |\includeonly| is applied to the child file
% and |\jobname| is set to the main file
% (for proper handling of |.aux| files):
%    \begin{macrocode}
\newcommand{\childdocmain}[1]
{
  \childdocdisable\childdocmain{}
  \if?#1?\else
    \begingroup
      \def\childdoctmp{#1}
      \ifx\childdoctmp\childdocname
        \def\childdoctmp{}
      \else
        \def\childdoctmp
        {
          \childdoctrue
          \includeonly{\childdocname}
          \def\childdocjob{#1}
          \def\jobname{#1}
        }
      \fi
      \expandafter
    \endgroup
    \childdoctmp
  \fi
}
%    \end{macrocode}

% \macro{\childdocof}
% The command |\childdocof| redirects
% compilation to the main file |#1|.
%    \begin{macrocode}
\newcommand{\childdocof}[1]
{
  \childdocdisable
  \childdoctrue
  \includeonly{\childdocname}
  \def\jobname{#1}
  \def\childdocjob{#1}
  \input{#1}
}
%    \end{macrocode}

% \macro{\childdocby}
% The command |\childdocby| ....
%    \begin{macrocode}
\newcommand{\childdocby}[2][]
{
  \childdocdisable
  \childdoctrue
  \childdocmanualtrue
  \if?#1?\else
    \def\jobname{#2}
  \fi
  \def\childdocjob{#2}
  \input{#2}
  \endinput
}
%    \end{macrocode}

% \macro{\childdocforward}
% The command |\childdocforward| redirects
% compilation to the main file or
% (if the optional argument is given) a child file.
% Parameters are set as if the main file
% or a child file starting with |\childdocof| was compiled.
% Then compilation is handed over to the main file:
%    \begin{macrocode}
\newcommand{\childdocforward}[2][]
{
  \begingroup
    \if?#1?
      \def\childdoctmp
      {
        \def\childdocname{#2}
        \def\childdocjob{#2}
        \def\jobname{#2}
        \input{#2}
        \endinput
      }
    \else
      \def\childdoctmp
      {
        \childdocdisable
        \def\childdocname{#2}
        \childdoctrue
        \includeonly{#2}
        \def\childdocjob{#1}
        \def\jobname{#1}
        \input{#1}
        \endinput
      }
    \fi
    \expandafter
  \endgroup
  \childdoctmp
}
%    \end{macrocode}

% \macro{\childdocforwardprefix}
% The command |\childdocforwardprefix| redirects
% compilation to the main or a child file by means of a pattern.
% The prefix |#1| in the current filename is replaced by |#2|
% and the suffix of the current filename is kept
% (it is assumed that the filename does not contain the substring `|~~~|'
% which is used as a delimiter).
% Compilation is handed over to the new file by |\childdocforward|:
%    \begin{macrocode}
\newcommand{\childdocforwardprefix}[3][]
{
  \begingroup
    \def\childdocextract #2##1~~~{\def\childdoctmp{\childdocforward[#1]{#3##1}}}
    \expandafter\childdocextract\childdocname~~~
    \expandafter
  \endgroup
  \childdoctmp
}
%    \end{macrocode}

% \macro{\childdoc}
% The deprecated macro |\childdoc| is a legacy version of |\childdocmain|:
%    \begin{macrocode}
\newcommand{\childdoc}{\childdocmain}
%    \end{macrocode}

% \macro{\childdocredirect}
% The deprecated macro |\childdocredirect| is a legacy version
% of |\childdocforward| and |\childdocforwardprefix|:
%    \begin{macrocode}
\newcommand{\childdocredirect}[2][]
{
  \begingroup
    \if?#1?
      \def\childdoctmp{\childdocforward{#2}}
    \else
      \def\childdoctmp{\childdocforwardprefix{#1}{#2}}
    \fi
    \expandafter
  \endgroup
  \childdoctmp
}
%    \end{macrocode}

%\iffalse
%</package>
%\fi
%
\endinput
\childdocforward[|\textit{main}|]{|\textit{dest}|}"|
\end{center}
%
Here \textit{target} is the name of the output file,
\textit{main} is the name of the main file
and \textit{dest} is the name of the main or child file to be processed
(all filenames without extensions).
The optional argument \textit{main} can be omitted
if \textit{main} matches \textit{dest}.
Optionally, compilation \textit{flags} can be defined via |\def| commands.
This command line makes the \TeX{} engine believe
it is compiling the file \textit{target}
whose content is specified as the latter parameter.
The provided code then forwards the processing to
\textit{main} or \textit{dest} as described in \secref{sec:forward}.

%%%%%%%%%%%%%%%%%%%%%%%%%%%%%%%%%%%%%%%%%%%%%%%%%%%%%%%%%%%%%%%%%%%%%%%%%%%%%%%%
\subsection{Include by Input}
\label{sec:input}

Including child documents by |\include| has some restrictions by design.
Most notably, the content of a child document always occupies
its own set of pages; pages cannot be shared between child documents.
Usually, this behaviour makes perfect sense
because each child document contain an essential part of the document.
However, in some situations it may be desirable to compose
a document from a collection of parts
without having mandatory page breaks between then.
For this case, the package
provides a mechanism to include parts
by |\input| which can also be processed individually.
However, by construction this mechanism
requires manual handling of the content to be output.

%%%%%%%%%%%%%%%%%%%%%%%%%%%%%%%%%%%%%%%%
\DescribeMacro{\ifchilddocmanual}
The main file should be prepared as usual, see \secref{sec:include}.
However, the document body must make a distinction
between processing of an individual part and of the main document, e.g.:
%
\begin{center}
\begin{tabular}{l}
|\ifchilddocmanual|\\
|\input{\childdocname}|\\
|\||else|\\
\textit{document body with }|\input{|\textit{part}|}|\\
|\||fi|
\end{tabular}
\end{center}
%
The conditional |\ifchilddocmanual| is true whenever
a part to be included by |\input| is being compiled,
and the name of the part is stored in |\childdocname|.

%%%%%%%%%%%%%%%%%%%%%%%%%%%%%%%%%%%%%%%%
\DescribeMacro{\childdocby}
Each part to be included by |\input| should start with:
%
\begin{center}
\begin{tabular}{l}
|% \iffalse
%
% childdoc.dtx Copyright (C) 2017-2018 Niklas Beisert
%
% This work may be distributed and/or modified under the
% conditions of the LaTeX Project Public License, either version 1.3
% of this license or (at your option) any later version.
% The latest version of this license is in
%   http://www.latex-project.org/lppl.txt
% and version 1.3 or later is part of all distributions of LaTeX
% version 2005/12/01 or later.
%
% This work has the LPPL maintenance status `maintained'.
%
% The Current Maintainer of this work is Niklas Beisert.
%
% This work consists of the files childdoc.dtx and childdoc.ins
% and the derived files childdoc.def and cdocsamp.tex with
% cdocsch1.tex, cdocsch2.tex, cdocsdrf.tex, cdocsfn1.tex, cdocsfn2.tex.
%
%<package>\ifdefined\childdocmain\endinput\fi
%<package>\ProvidesFile{childdoc.def}[2018/12/30 v2.0 child document driver]
%<samplemain>\ProvidesFile{cdocsamp.tex}[2018/12/30 v2.0 sample for childdoc]
%<*driver>
%\ProvidesFile{childdoc.drv}[2018/12/30 v2.0 childdoc reference manual file]
\PassOptionsToClass{10pt,a4paper}{article}
\documentclass{ltxdoc}

\usepackage[margin=35mm]{geometry}
\usepackage{hyperref}
\usepackage{hyperxmp}
\usepackage[usenames]{color}

\hypersetup{colorlinks=true}
\hypersetup{pdfstartview=FitH}
\hypersetup{pdfpagemode=UseNone}
\hypersetup{pdfsource={}}
\hypersetup{pdflang={en-UK}}
\hypersetup{pdfcopyright={Copyright 2017-2018 Niklas Beisert.
  This work may be distributed and/or modified under the
  conditions of the LaTeX Project Public License, either version 1.3
  of this license or (at your option) any later version.}}
\hypersetup{pdflicenseurl={http://www.latex-project.org/lppl.txt}}
\hypersetup{pdfcontactaddress={ETH Zurich, ITP, HIT K,
  Wolfgang-Pauli-Strasse 27}}
\hypersetup{pdfcontactpostcode={8093}}
\hypersetup{pdfcontactcity={Zurich}}
\hypersetup{pdfcontactcountry={Switzerland}}
\hypersetup{pdfcontactemail={nbeisert@itp.phys.ethz.ch}}
\hypersetup{pdfcontacturl={http://people.phys.ethz.ch/\xmptilde nbeisert/}}

\newcommand{\secref}[1]{\hyperref[#1]{section \ref*{#1}}}

\parskip1ex
\parindent0pt
\let\olditemize\itemize
\def\itemize{\olditemize\parskip0pt}

\begin{document}

\title{The \textsf{childdoc} Package}
\hypersetup{pdftitle={The childdoc Package}}
\author{Niklas Beisert\\[2ex]
  Institut f\"ur Theoretische Physik\\
  Eidgen\"ossische Technische Hochschule Z\"urich\\
  Wolfgang-Pauli-Strasse 27, 8093 Z\"urich, Switzerland\\[1ex]
  \href{mailto:nbeisert@itp.phys.ethz.ch}
  {\texttt{nbeisert@itp.phys.ethz.ch}}}
\hypersetup{pdfauthor={Niklas Beisert}}
\hypersetup{pdfsubject={Manual for the LaTeX2e Package childdoc}}
\date{30 December 2018, \textsf{v2.0}}
\maketitle

\begin{abstract}\noindent
\textsf{childdoc} is a \LaTeXe{} package
that enables the direct compilation
of document sections included by |\include|
to individual files.
\end{abstract}

\begingroup
\parskip0ex
\tableofcontents
\endgroup

%%%%%%%%%%%%%%%%%%%%%%%%%%%%%%%%%%%%%%%%%%%%%%%%%%%%%%%%%%%%%%%%%%%%%%%%%%%%%%%%
%%%%%%%%%%%%%%%%%%%%%%%%%%%%%%%%%%%%%%%%%%%%%%%%%%%%%%%%%%%%%%%%%%%%%%%%%%%%%%%%
\section{Introduction}

\LaTeX{} provides a mechanism to structure a large document (such as a book)
into a main file and several child files (containing the chapters)
using the |\include| command.
This mechanism is beneficial for documents
which span hundreds of pages in order to
make the source file(s) more manageable.
Moreover, compilation can be restricted to
selected child files by means of the |\includeonly| command.
The latter feature can be used to reduce the compilation time while editing
(this was significantly more useful in the earlier days of \LaTeX{})
or to generate a smaller document which is easier to navigate.
Another application of |\includeonly| is to generate
documents consisting of selected parts of the complete document.

However, there are a few drawbacks of the plain |\include| mechanism:
\begin{itemize}
\item
The child files cannot be compiled on their own,
they can only be compiled via the main file.
A naive editing environment
(such as a text editor with an option
to have the current file processed by \LaTeX)
may require one to switch to the main file before compiling;
attempting to compile the child file produces errors.
\item
The main file must be modified (each time)
to adjust the |\includeonly| command
to the present needs. This easily leaves the main file in a messy state.
\item
The generated document will always carry the filename
of the main document. This is inconvenient if
several child files are to be compiled and
to be kept for distribution.
\end{itemize}

The present package provides a simple interface
to make child files individually compilable by \LaTeX{}.
Compiling a child file then has the same effect as compiling
the main file with an |\includeonly| command
to select the appropriate child.
Moreover the generated document will carry the name of the child
rather than the main file.
This resolves all three above issues.

This feature is meant to make the editing of books,
thesis documents and lecture notes somewhat more convenient.
However, the package can also be used efficiently for
composing a series of documents (such as exercise sheets)
which are typically distributed individually.
It then assists the author in generating the individual documents
(potentially in different versions)
as well as a document containing the collected series.
Another application is in developing style files
or other kinds of included material
where compilation of the style file could redirect
to a sample or test file.

%%%%%%%%%%%%%%%%%%%%%%%%%%%%%%%%%%%%%%%%%%%%%%%%%%%%%%%%%%%%%%%%%%%%%%%%%%%%%%%%
%%%%%%%%%%%%%%%%%%%%%%%%%%%%%%%%%%%%%%%%%%%%%%%%%%%%%%%%%%%%%%%%%%%%%%%%%%%%%%%%
\section{Usage}

First of all, the package \textsf{childdoc} is \emph{not} a standard
\LaTeXe{} |.sty| style file! Therefore it needs to be invoked in
a non-standard way.

%%%%%%%%%%%%%%%%%%%%%%%%%%%%%%%%%%%%%%%%%%%%%%%%%%%%%%%%%%%%%%%%%%%%%%%%%%%%%%%%
\subsection{Included Files}
\label{sec:include}

%%%%%%%%%%%%%%%%%%%%%%%%%%%%%%%%%%%%%%%%
\DescribeMacro{\childdocmain}
To use the package, add the commands
\begin{center}
\begin{tabular}{l}
|\input{childdoc.def}|\\
|\childdocmain{}|\\
\end{tabular}
\end{center}
at the very top of the main \LaTeX{} file,
in particular \emph{before} the |\documentclass| statement!
The argument of |\childdocmain| should be left empty
(but it must be present).

%%%%%%%%%%%%%%%%%%%%%%%%%%%%%%%%%%%%%%%%
\DescribeMacro{\childdocof}
Furthermore, add the commands
\begin{center}
\begin{tabular}{l}
|\input{childdoc.def}|\\
|\childdocof{|\textit{main}|}|\\
\end{tabular}
\end{center}
at the top of every child file \textit{child}
which is included by |\include{|\textit{child}|}|
from within the main file
(or at least for those files to be compiled individually).
The argument \textit{main} must be the filename of the main file.

There are a couple of
considerations in setting up the main and child documents:

%%%%%%%%%%%%%%%%%%%%%%%%%%%%%%%%%%%%%%%%
\paragraph{Restrictions.}

Please note the following restrictions:
\begin{itemize}
\item
|\childdocmain| must be called with one argument \textit{main}
to ensure compatibility with earlier version of the package.
It must either be empty (|\childdocmain{}|)
or precisely match the filename of the main file in which it is specified.
See \secref{sec:detection} for further information.
\item
The filename \textit{main} must be specified without the |.tex| extension.
\item
The filename \textit{main} is case sensitive
(even in case-insensitive file systems)
due to internal string comparison.
\item
The argument \textit{main} should be fully expanded, it cannot be a macro.
\item
Subdirectories and special characters should be avoided in filenames.
\item
The command |\childdocmain{|\textit{main}|}| must be followed by a whitespace.
It should not be followed immediately by another command
or by a comment mark `|%|'.
This is because the \TeX{} parser reads the token immediately following
the argument of |\childdocmain| and puts it
at the beginning of every child section;
however, a white\-space is ignored.
\end{itemize}

%%%%%%%%%%%%%%%%%%%%%%%%%%%%%%%%%%%%%%%%
\paragraph{Content of Main File.}

It is advisable to place all content in the child files included by |\include|.
Any output contained in the main file will appear in all child documents
unless suppressed manually;
it cannot be suppressed automatically by the |\includeonly| directive
and thus should normally be avoided.
A method to include some content in the main file
by means of conditional processing is described in \secref{sec:conditional}.

%%%%%%%%%%%%%%%%%%%%%%%%%%%%%%%%%%%%%%%%
\paragraph{Page Numbering.}

When only a part of the document is compiled,
the appropriate numbering of pages
(as well as other status parameters)
is determined from the |.aux| files.
The latter contain information from previous passes.
However this information needs to propagate through
all intermediate child documents.
Therefore the page numbering in child documents may well
be inconsistent until the complete document is compiled at least once.

A useful (if unconventional) way to always ensure a consistent
page numbering is to restart the numbering in each child document
and denote the pages by `\textit{child}|.|\textit{page}'
where \textit{child} represents the chapter/section number of the child file.
This can be achieved by the command
|\numberwithin{page}{|\textit{child}|}|
of the \textsf{amsmath} package
where \textit{child} can be |chapter| or |section|
depending on the chosen structuring.
Alternatively, one can modify the macro |\thepage| appropriately
and reset the counter |page| at the start of each child file.

%%%%%%%%%%%%%%%%%%%%%%%%%%%%%%%%%%%%%%%%%%%%%%%%%%%%%%%%%%%%%%%%%%%%%%%%%%%%%%%%
\subsection{Conditional Processing}
\label{sec:conditional}

The package provides a mechanism to compile different versions
of a document. To customise the versions further some conditional processing
can come in handy to distinguish which version is being compiled.
The package provides two macros to describe the compilation context:

%%%%%%%%%%%%%%%%%%%%%%%%%%%%%%%%%%%%%%%%
\DescribeMacro{\ifchilddoc}
The conditional |\ifchilddoc| distinguishes between the compilation of
child documents and the main document:
%
\begin{center}
|\ifchilddoc |\textit{child-code}| |[|\||else |\textit{main-code}]| \||fi|
\end{center}

%%%%%%%%%%%%%%%%%%%%%%%%%%%%%%%%%%%%%%%%
\DescribeMacro{\childdocname}
\DescribeMacro{\childdocjob}
The macro |\childdocname| contains the filename (without extension)
of the main or child file being processed.
Note that |\childdocjob| will always contain the name of the main file.

%%%%%%%%%%%%%%%%%%%%%%%%%%%%%%%%%%%%%%%%
\paragraph{Title Page.}

Conditional processing can be used to include a title or banner page
in the main document when proper precautions are taken.
Importantly, the code in the main file should ensure that the page counter
(as well as other status parameters which are stored in the |.aux| files)
takes the same value after the conditional processing.
Otherwise the page numbers may take divergent values
depending on which part is compiled.

For example, a title page could be declared by:
%
\begin{center}
\begin{tabular}{l}
|\ifchilddoc\||else|\\
|\addtocounter{page}{-1}|\\
\textit{code for title page}\\
|\newpage|\\
|\||fi|
\end{tabular}
\end{center}
%
A banner page for the child documents can be generated by:
%
\begin{center}
\begin{tabular}{l}
|\ifchilddoc|\\
|\addtocounter{page}{-1}|\\
\textit{code for banner page}\\
|\newpage|\\
|\||fi|
\end{tabular}
\end{center}
%
Here one could write a message such as:
\begin{center}
|This is the part \childdocname{} of \childdocjob{}.|
\end{center}

%%%%%%%%%%%%%%%%%%%%%%%%%%%%%%%%%%%%%%%%%%%%%%%%%%%%%%%%%%%%%%%%%%%%%%%%%%%%%%%%
\subsection{Flags}
\label{sec:flags}

The package makes it easy to generate different versions
of the main or child documents.
To this end compilation flags can be defined
and assigned different default values.
They will be particularly useful in conjunction
with the forwarding mechanism described in \secref{sec:forward}.

For example, it may be useful to have a flag |\version|
which can be set to |draft| or |final|.
The document source will contain some conditional code
depending on the value of |\version|.
Suppose further, the flag should default to |final| for the main file
and to |draft| for child files
which is a natural assignment for editing the document.
This is achieved by placing the following code
in the preamble of the main document
(below the |\childdocmain| directive):
%
\begin{center}
\begin{tabular}{l}
|\ifchilddoc|\\
|\providecommand{\version}{draft}|\\
|\||else|\\
|\providecommand{\version}{final}|\\
|\||fi|
\end{tabular}
\end{center}
%
The definition by |\providecommand| makes sure
that previous definitions are not overwritten.
Further statements |\providecommand{\version}{...}|
can thus be added before the above code to override it.

For the main file, one might add a line
(between |\childdocmain| and the above block)
%
\begin{center}
|%\ifchilddoc\||else\providecommand{\version}{draft}\||fi|
\end{center}
%
which can be uncommented to produce a draft version.
Likewise one can add a line to the very top of a child file
(above the |\childdocof{|\textit{main}|}| directive)
%
\begin{center}
|%\providecommand{\version}{final}|
\end{center}
%
which can be uncommented to produce the final version of this child document.

%%%%%%%%%%%%%%%%%%%%%%%%%%%%%%%%%%%%%%%%%%%%%%%%%%%%%%%%%%%%%%%%%%%%%%%%%%%%%%%%
\subsection{Forwarding}
\label{sec:forward}

Different versions of the main or child documents
using compilation flags as described in \secref{sec:flags}
can be (permanently) stored in different files
for convenient compilation, viewing and distribution.
To this end, the package defines a command
to pass on compilation to a different file:

%%%%%%%%%%%%%%%%%%%%%%%%%%%%%%%%%%%%%%%%
\DescribeMacro{\childdocforward}
The command |\childdocforward| redirects processing to
another source file:
%
\begin{center}
\begin{tabular}{l}
|\input{childdoc.def}|\\
|\childdocforward[|\textit{main}|]{|\textit{dest}|}|\\
\end{tabular}
\end{center}
%
The argument \textit{dest} is the destination file
(without extension).
It should be the main file or one of the child files.
Note that further \textsf{childdoc} directives
such as |\childdocof| and |\childdocforward|
in the indicated file will be processed in this form.
The optional argument \textit{main}
passes on directly to the main file \textit{main}
while pretending to compile the child \textit{dest}.
This form behaves as if \textit{dest}
issues |\childdocof{|\textit{main}|}| right away,
and no further \textsf{childdoc} directives will be processed.

%%%%%%%%%%%%%%%%%%%%%%%%%%%%%%%%%%%%%%%%
\DescribeMacro{\...prefix}
In the alternative form |\childdocforwardprefix|,
%
\begin{center}
\begin{tabular}{l}
|\input{childdoc.def}|\\
|\childdocforwardprefix[|\textit{main}|]{|\textit{prefix}|}{|\textit{dest}|}|
\end{tabular}
\end{center}
%
the destination file is determined by a pattern
depending on the current file:
To make this work, the current file must be called
`{\textit{prefix}\hspace{0.2em}\textit{suffix}}'
with \textit{prefix} matching precisely the argument.
Processing is then passed on to the file
`{\textit{dest}\hspace{0.2em}\textit{suffix}}'.
Surely, the same effect is achieved by
directly specifying the
argument `{\textit{dest}\hspace{0.2em}\textit{suffix}}'
in the first form.
However, that requires to set up a different file
for each child. With the alternative form of the command
all these files can have exactly the same content
which simplifies setting them up and maintaining them.

For example, the following file |draft.tex|
with a compilation flag |\version| as described in \secref{sec:flags}
compiles the main document as a draft:
%
\begin{center}
\begin{tabular}{l}
|\def\version{draft}|\\
|\input{childdoc.def}|\\
|\childdocforward{|\textit{main}|}|
\end{tabular}
\end{center}
%
Likewise, the following files |final|\textit{nn}|.tex|
compile the final version of the child document
|child|\textit{nn}|.tex|:
%
\begin{center}
\begin{tabular}{l}
|\def\version{final}|\\
|\input{childdoc.def}|\\
|\childdocforwardprefix{final}{child}|
\end{tabular}
\end{center}
%

Note that when several versions of a main file and/or of each child file
are to be generated, it may be convenient to set up a |Makefile| or
shell script to automatise the process.

%%%%%%%%%%%%%%%%%%%%%%%%%%%%%%%%%%%%%%%%%%%%%%%%%%%%%%%%%%%%%%%%%%%%%%%%%%%%%%%%
\subsection{Command Line Processing}
\label{sec:commandline}

The effect of redirection files can also be achieved by invoking
the \LaTeX{} compiler with a more elaborate command line.
Most conveniently this should be done as part
of a shell script or a |Makefile|.

When using \textsf{childdoc} in the main file, the following
command lines effectively perform a redirection
(note that depending on the shell being used,
backslashes may have to be doubled: `|\|' $\to$ `|\\|'):
%
\begin{center}
|... -jobname "|\textit{target}|" |\\|"|[\textit{flags}]%
|\input{childdoc.def}\childdocforward[|\textit{main}|]{|\textit{dest}|}"|
\end{center}
%
Here \textit{target} is the name of the output file,
\textit{main} is the name of the main file
and \textit{dest} is the name of the main or child file to be processed
(all filenames without extensions).
The optional argument \textit{main} can be omitted
if \textit{main} matches \textit{dest}.
Optionally, compilation \textit{flags} can be defined via |\def| commands.
This command line makes the \TeX{} engine believe
it is compiling the file \textit{target}
whose content is specified as the latter parameter.
The provided code then forwards the processing to
\textit{main} or \textit{dest} as described in \secref{sec:forward}.

%%%%%%%%%%%%%%%%%%%%%%%%%%%%%%%%%%%%%%%%%%%%%%%%%%%%%%%%%%%%%%%%%%%%%%%%%%%%%%%%
\subsection{Include by Input}
\label{sec:input}

Including child documents by |\include| has some restrictions by design.
Most notably, the content of a child document always occupies
its own set of pages; pages cannot be shared between child documents.
Usually, this behaviour makes perfect sense
because each child document contain an essential part of the document.
However, in some situations it may be desirable to compose
a document from a collection of parts
without having mandatory page breaks between then.
For this case, the package
provides a mechanism to include parts
by |\input| which can also be processed individually.
However, by construction this mechanism
requires manual handling of the content to be output.

%%%%%%%%%%%%%%%%%%%%%%%%%%%%%%%%%%%%%%%%
\DescribeMacro{\ifchilddocmanual}
The main file should be prepared as usual, see \secref{sec:include}.
However, the document body must make a distinction
between processing of an individual part and of the main document, e.g.:
%
\begin{center}
\begin{tabular}{l}
|\ifchilddocmanual|\\
|\input{\childdocname}|\\
|\||else|\\
\textit{document body with }|\input{|\textit{part}|}|\\
|\||fi|
\end{tabular}
\end{center}
%
The conditional |\ifchilddocmanual| is true whenever
a part to be included by |\input| is being compiled,
and the name of the part is stored in |\childdocname|.

%%%%%%%%%%%%%%%%%%%%%%%%%%%%%%%%%%%%%%%%
\DescribeMacro{\childdocby}
Each part to be included by |\input| should start with:
%
\begin{center}
\begin{tabular}{l}
|\input{childdoc.def}|\\
|\childdocby{|\textit{main}|}|\\
\end{tabular}
\end{center}
%
The directive |\childdocby| is similar to |\childdocof|
described in \secref{sec:include},
but the subsequent selection of content must be done manually.
To that end, both |\ifchilddoc| and |\ifchilddocmanual|
will be true upon processing of a part,
and the name of the part is stored in |\childdocname|.
Note that |\jobname| will be set to the filename of the current part
so that each part receives an individual |.aux| file
that does not interfere with the |.aux| file(s) of the main document.
This behaviour can be altered by the alternative form
|\childdocby[*]{|\textit{main}|}| (with a non-empty optional argument)
which uses the |.aux| file of the main document
by setting |\jobname| to \textit{main}.

%%%%%%%%%%%%%%%%%%%%%%%%%%%%%%%%%%%%%%%%%%%%%%%%%%%%%%%%%%%%%%%%%%%%%%%%%%%%%%%%
\subsection{Driver Development}
\label{sec:driver}

The \textsf{childdoc} mechanism can also be use for the development
of definition files such as \LaTeX{} styles or classes.
This case differs from the above setup with multiple parts
included by |\include| in that no |\includeonly| should be invoked.
This can be achieved by starting the include file
(before |\ProvidesPackage|) with:
%
\begin{center}
\begin{tabular}{l}
|\input{childdoc.def}|\\
|\childdocforward{|\textit{main}|}|\\
\end{tabular}
\end{center}
%
or alternatively with:
%
\begin{center}
\begin{tabular}{l}
|\input{childdoc.def}|\\
|\childdocby{|\textit{main}|}|\\
\end{tabular}
\end{center}
%
Both forms have slightly different effects as described above.
The main file is prepared as usual, see \secref{sec:include}.

%%%%%%%%%%%%%%%%%%%%%%%%%%%%%%%%%%%%%%%%%%%%%%%%%%%%%%%%%%%%%%%%%%%%%%%%%%%%%%%%
\subsection{Legacy Detection}
\label{sec:detection}

The directive |\childdocmain| in the main file can detect
whether the complete document or merely a child is to be compiled
even without using the directive |\childdocof|.
This method is deprecated because it is less robust
and there is no compelling reason to use it;
it is merely provided for backward compatibility
and it may be removed in future versions.

If the detection mechanism is to be used,
it is mandatory to correctly specify
the filename of the main file as the argument of |\childdocmain|:
%
\begin{center}
\begin{tabular}{l}
|\input{childdoc.def}|\\
|\childdocmain{|\textit{main}|}|\\
\end{tabular}
\end{center}
%
If |\jobname| does not match the argument \textit{main} of |\childdocmain|,
it is assumed that |\jobname| points to the child file to be compiled.
When using |\childdocmain| with the main file specified as argument,
it suffices to start a child file
with just |\input{|\textit{main}|}|
without loading of the package and using |\childdocof|.
If instead all processing is done
with the appropriate \textsf{childdoc} directives,
the argument of \textit{main} of |\childdocmain| can be empty.

An alternative version of the command line processing described
in \secref{sec:commandline} using the detection mechanism reads:
%
\begin{center}
|... -jobname "|\textit{target}|" "|[\textit{flags}]%
[|\def\jobname{|\textit{dest}|}|]|\input{|\textit{main}|}"|
\end{center}

%%%%%%%%%%%%%%%%%%%%%%%%%%%%%%%%%%%%%%%%%%%%%%%%%%%%%%%%%%%%%%%%%%%%%%%%%%%%%%%%
\subsection{Manual Code}
\label{sec:manual}

In case one cannot be certain whether the definitions file |childdoc.def|
is installed on the target \TeX{} distribution
and one prefers not to ship it,
it is conceivable to paste a few relevant commands into the sources.

To that end, drop all statements |\input{childdoc.def}|
and perform the replacements as outlined below.
Instead of |\childdocmain{|\textit{main}|}| add the following code
to the top of the main file:
%
\begin{center}
\begin{tabular}{l}
|\||ifdefined\childdocname\endinput\||fi\newif\ifchilddoc|\\
|\edef\childdocname{\scantokens\expandafter{\jobname\noexpand}}|\\
|\def\childdocmain{|\textit{main}|}\||ifx\childdocmain\childdocname\||else|\\
|\childdoctrue\includeonly{\childdocname}\let\jobname\childdocmain\||fi|\\
\end{tabular}
\end{center}
%
Instead of |\childdocof{|\textit{main}|}| just include the main file
at the top of each child file:
%
\begin{center}
|\input{|\textit{main}|}|
\end{center}
%
A simple redirection |\childdocforward{|\textit{dest}|}| is achieved by:
%
\begin{center}
|\def\jobname{|\textit{dest}|}\input{\jobname}|
\end{center}
%
The redirection with prefix
|\childdocforwardprefix[|\textit{prefix}|]{|\textit{dest}|}|
is accomplished by:
%
\begin{center}
\begin{tabular}{l}
|{\edef\jobname{\scantokens\expandafter{\jobname\noexpand}}|\\
|\def\redirectjob |\textit{prefix}|#1~~~{\gdef\jobname{|\textit{dest}|#1}}|\\
|\expandafter\redirectjob\jobname~~~}\input{\jobname}|
\end{tabular}
\end{center}

In an alternative approach,
child documents can be compiled by a specific command line
without additional code or specific definitions:
%
\begin{center}
|... -jobname "|\textit{target}|" "|[\textit{flags}]%
|\includeonly{|\textit{dest}|}\input{|\textit{main}|}"|
\end{center}
%

%%%%%%%%%%%%%%%%%%%%%%%%%%%%%%%%%%%%%%%%%%%%%%%%%%%%%%%%%%%%%%%%%%%%%%%%%%%%%%%%
%%%%%%%%%%%%%%%%%%%%%%%%%%%%%%%%%%%%%%%%%%%%%%%%%%%%%%%%%%%%%%%%%%%%%%%%%%%%%%%%
\section{Information}

%%%%%%%%%%%%%%%%%%%%%%%%%%%%%%%%%%%%%%%%%%%%%%%%%%%%%%%%%%%%%%%%%%%%%%%%%%%%%%%%
\subsection{Copyright}

Copyright \copyright{} 2017--2018 Niklas Beisert

This work may be distributed and/or modified under the
conditions of the \LaTeX{} Project Public License, either version 1.3
of this license or (at your option) any later version.
The latest version of this license is in
  \url{http://www.latex-project.org/lppl.txt}
and version 1.3 or later is part of all distributions of \LaTeX{}
version 2005/12/01 or later.

This work has the LPPL maintenance status `maintained'.

The Current Maintainer of this work is Niklas Beisert.

This work consists of the files |README.txt|, |childdoc.ins| and |childdoc.dtx|
as well as the derived files |childdoc.def|, |cdocsamp.tex|
with |cdocsch1.tex|, |cdocsch2.tex|, |cdocspt3.tex|, |cdocspt4.tex|,
|cdocsdrf.tex|, |cdocsfn1.tex|, |cdocsfn2.tex|
as well as |childdoc.pdf|.

%%%%%%%%%%%%%%%%%%%%%%%%%%%%%%%%%%%%%%%%%%%%%%%%%%%%%%%%%%%%%%%%%%%%%%%%%%%%%%%%
\subsection{Files and Installation}

The package consists of the files:
%
\begin{center}
\begin{tabular}{ll}
    |README.txt|   & readme file \\
    |childdoc.ins| & installation file \\
    |childdoc.dtx| & source file \\
    |childdoc.def| & definition file \\
    |cdocsamp.tex| & sample main file \\
    |cdocsch1.tex| & sample include file \\
    |cdocsch2.tex| & sample include file \\
    |cdocspt3.tex| & sample part file \\
    |cdocspt4.tex| & sample part file \\
    |cdocsdrf.tex| & sample redirection file \\
    |cdocsfn1.tex| & sample redirection file \\
    |cdocsfn2.tex| & sample redirection file \\
    |childdoc.pdf| & manual
\end{tabular}
\end{center}
%
The distribution consists of the files
|README.txt|, |childdoc.ins| and |childdoc.dtx|.
%
\begin{itemize}
\item
Run (pdf)\LaTeX{} on |childdoc.dtx|
to compile the manual |childdoc.pdf| (this file).
\item
Run \LaTeX{} on |childdoc.ins| to create the definitions file |childdoc.def|
and the sample |cdocsamp.tex| with include files
|cdocsch1.tex|, |cdocsch2.tex|, |cdocspt3.tex|, |cdocspt4.tex|,
|cdocsdrf.tex|, |cdocsfn1.tex|, |cdocsfn2.tex|.
Then copy the file |childdoc.def| to an appropriate directory of your \LaTeX{}
distribution, e.g.\ \textit{texmf-root}|/tex/latex/childdoc|.
\end{itemize}

%%%%%%%%%%%%%%%%%%%%%%%%%%%%%%%%%%%%%%%%%%%%%%%%%%%%%%%%%%%%%%%%%%%%%%%%%%%%%%%%
\subsection{Related CTAN Packages}

There are several other packages which offer a similar functionality:
%
\begin{itemize}
\item
The packages
\href{http://ctan.org/pkg/docmute}{\textsf{docmute}},
\href{http://ctan.org/pkg/includex}{\textsf{includex}} and
\href{http://ctan.org/pkg/standalone}{\textsf{standalone}}
provide commands to include only the document body of
a child file thus allowing both files to be compiled individually.
\item
The packages \href{http://ctan.org/pkg/subdocs}{\textsf{subdocs}}
and \href{http://ctan.org/pkg/subfiles}{\textsf{subfiles}}
provide structures in which the main and child documents can be
encapsulated and allowing them to be compiled individually.
The inclusion mechanism is different from the conventional |\include|.
\item
The package \href{http://ctan.org/pkg/combine}{\textsf{combine}}
is an elaborate solution to combine several documents into one.
\end{itemize}
%
See also the CTAN topic \href{http://ctan.org/topic/subdocs}{\textsf{subdocs}}
for further related packages.
The present package differs from the above solutions in that
a document structure constructed with the conventional |\include| mechanism
just needs two extra commands at the top of every file
such that all constituent files can be compiled individually.

%%%%%%%%%%%%%%%%%%%%%%%%%%%%%%%%%%%%%%%%%%%%%%%%%%%%%%%%%%%%%%%%%%%%%%%%%%%%%%%%
%\subsection{Feature Suggestions}
%
%The following is a list of features which may be useful for future
%versions of this package:
%%
%\begin{itemize}
%\item
%\ldots
%\end{itemize}

%%%%%%%%%%%%%%%%%%%%%%%%%%%%%%%%%%%%%%%%%%%%%%%%%%%%%%%%%%%%%%%%%%%%%%%%%%%%%%%%
\subsection{Revision History}

%%%%%%%%%%%%%%%%%%%%%%%%%%%%%%%%%%%%%%%%
\paragraph{v2.0:} 2018/12/30

\begin{itemize}
\item
immediate forward processing
\item
added |\childdocby| mechanism
\item
manual restructured
\end{itemize}

%%%%%%%%%%%%%%%%%%%%%%%%%%%%%%%%%%%%%%%%
\paragraph{v1.6:} 2018/01/17

\begin{itemize}
\item
application for development of include files
\item
corrections to manual
\end{itemize}

%%%%%%%%%%%%%%%%%%%%%%%%%%%%%%%%%%%%%%%%
\paragraph{v1.5:} 2017/05/21

\begin{itemize}
\item
more complete structuring introduced
\item
|\childdocof| introduced
\item
|\childdoc| renamed to |\childdocmain|
\item
|\childredirect| renamed to |\childdocforward| and |\childdocforwardprefix|
and functionality expanded
\end{itemize}

%%%%%%%%%%%%%%%%%%%%%%%%%%%%%%%%%%%%%%%%
\paragraph{v1.0:} 2017/04/27

\begin{itemize}
\item
manual and install package
\item
first version published on CTAN
\end{itemize}

%%%%%%%%%%%%%%%%%%%%%%%%%%%%%%%%%%%%%%%%
\paragraph{v0.6:} 2017/04/26

\begin{itemize}
\item
redirection mechanism added
\end{itemize}

%%%%%%%%%%%%%%%%%%%%%%%%%%%%%%%%%%%%%%%%
\paragraph{v0.5:} 2017/04/26

\begin{itemize}
\item
functionality in definition file
\end{itemize}


%%%%%%%%%%%%%%%%%%%%%%%%%%%%%%%%%%%%%%%%%%%%%%%%%%%%%%%%%%%%%%%%%%%%%%%%%%%%%%%%
%%%%%%%%%%%%%%%%%%%%%%%%%%%%%%%%%%%%%%%%%%%%%%%%%%%%%%%%%%%%%%%%%%%%%%%%%%%%%%%%
%%%%%%%%%%%%%%%%%%%%%%%%%%%%%%%%%%%%%%%%%%%%%%%%%%%%%%%%%%%%%%%%%%%%%%%%%%%%%%%%
\appendix

\settowidth\MacroIndent{\rmfamily\scriptsize 000\ }

 \DocInput{childdoc.dtx}

\end{document}
%</driver>
% \fi
%
% %%%%%%%%%%%%%%%%%%%%%%%%%%%%%%%%%%%%%%%%%%%%%%%%%%%%%%%%%%%%%%%%%%%%%%%%%%%%%%
% %%%%%%%%%%%%%%%%%%%%%%%%%%%%%%%%%%%%%%%%%%%%%%%%%%%%%%%%%%%%%%%%%%%%%%%%%%%%%%
% \section{Sample}
%\iffalse
%<*samplemain>
%\fi
%
% The following presents a sample document
% with two chapters, two parts, a title page,
% a compile flag as well as three forwarding files to set the flag.
% It consists of eight |.tex| files:
% \begin{center}
% \begin{tabular}{ll}
% |cdocsamp.tex|&main file\\
% |cdocsch1.tex|&include file for chapter 1\\
% |cdocsch2.tex|&include file for chapter 2\\
% |cdocspt3.tex|&include file for part 3\\
% |cdocspt4.tex|&include file for part 4\\
% |cdocsdrf.tex|&forwarding file for main file in draft mode\\
% |cdocsfi1.tex|&forwarding file for final version of chapter 1\\
% |cdocsfi2.tex|&forwarding file for final version of chapter 2\\
% \end{tabular}
% \end{center}
% Each of the eight files can be compiled directly by the \LaTeX{} compiler.
%
% %%%%%%%%%%%%%%%%%%%%%%%%%%%%%%%%%%%%%%
% \paragraph{Main File.}
%
% The main file is called |cdocsamp.tex|.
%
% Load the \textsf{childdoc} definitions and
% declare the filename for the main document:
%    \begin{macrocode}
\input{childdoc.def}
\childdocmain{}
%    \end{macrocode}

% Optional override for |\version| flag:
%    \begin{macrocode}
%%\ifchilddoc\else\providecommand{\version}{draft}\fi
%    \end{macrocode}

% Define the default values for the |\version| flag
% (|final| for the main file and |draft| for childs):
%    \begin{macrocode}
\ifchilddoc
\providecommand{\version}{draft}
\else
\providecommand{\version}{final}
\fi
%    \end{macrocode}

% Load the standard document class:
%    \begin{macrocode}
\documentclass[12pt]{article}
%    \end{macrocode}

% Start the document body:
%    \begin{macrocode}
\begin{document}
%    \end{macrocode}

% Declare a title page.
% Print title, part of document being processed and version flag:
%    \begin{macrocode}
\addtocounter{page}{-1}
\begin{center}
{\LARGE\bfseries{}childdoc example\par}
\vspace{1cm}
\ifchilddoc
\ifchilddocmanual part\else chapter\fi:
`\childdocname' of `\childdocjob'\par
\else
main document: `\childdocjob'\par
\fi
version: \version\par
\end{center}
\newpage
%    \end{macrocode}

% Manually include selected file,
% otherwise process as usual:
%    \begin{macrocode}
\ifchilddocmanual
\section*{part `\childdocname'}
\input{\childdocname}
\else
%    \end{macrocode}

% Include the two chapters:
%    \begin{macrocode}
\include{cdocsch1}
\include{cdocsch2}
%    \end{macrocode}

% Include the two parts unless only chapters should be displayed:
%    \begin{macrocode}
\ifchilddoc\else
\section{part three}
\input{cdocspt3}
\section{part four}
\input{cdocspt4}
\fi
%    \end{macrocode}

% Process as usual until here:
%    \begin{macrocode}
\fi
%    \end{macrocode}

% End of document body:
%    \begin{macrocode}
\end{document}
%    \end{macrocode}
%\iffalse
%</samplemain>
%\fi
%
% %%%%%%%%%%%%%%%%%%%%%%%%%%%%%%%%%%%%%%
% \paragraph{Chapter Include Files.}
%
% The include files are called |cdocsch1.tex| and |cdocsch2.tex|.
%
%\iffalse
%<*samplechap1|samplechap2>
%\fi

% Optional override for |\version| flag:
%    \begin{macrocode}
%%\providecommand{\version}{final}
%    \end{macrocode}

% Include the main document:
%    \begin{macrocode}
\input{childdoc.def}
\childdocof{cdocsamp}
%    \end{macrocode}

%\iffalse
%</samplechap1|samplechap2>
%\fi
%
%\iffalse
%<*samplechap1>
%\fi
% Some text for chapter 1:
%    \begin{macrocode}
\section{one}
some text in chapter one
%    \end{macrocode}

%\iffalse
%</samplechap1>
%\fi
% Some text for chapter 2:
%\iffalse
%<*samplechap2>
%\fi
%    \begin{macrocode}
\section{two}
more text in chapter two
%    \end{macrocode}

%\iffalse
%</samplechap2>
%\fi
%
% %%%%%%%%%%%%%%%%%%%%%%%%%%%%%%%%%%%%%%
% \paragraph{Part Include Files.}
%
% The include files are called |cdocspt3.tex| and |cdocspt4.tex|.
%
%\iffalse
%<*samplepart3|samplepart4>
%\fi

% Optional override for |\version| flag:
%    \begin{macrocode}
%%\providecommand{\version}{final}
%    \end{macrocode}

% Include the main document:
%    \begin{macrocode}
\input{childdoc.def}
\childdocby{cdocsamp}
%    \end{macrocode}

%\iffalse
%</samplepart3|samplepart4>
%\fi
%
%\iffalse
%<*samplepart3>
%\fi
% Some text for part 3:
%    \begin{macrocode}
some text in part three
%    \end{macrocode}

%\iffalse
%</samplepart3>
%\fi
% Some text for part 4:
%\iffalse
%<*samplepart4>
%\fi
%    \begin{macrocode}
more text in part four
%    \end{macrocode}

%\iffalse
%</samplepart4>
%\fi
%
% %%%%%%%%%%%%%%%%%%%%%%%%%%%%%%%%%%%%%%
% \paragraph{Forwarding for a Complete Draft.}
%
% The following forwarding file |cdocsdrf.tex|
% compiles the main document in draft mode:
%\iffalse
%<*sampledraft>
%\fi
%    \begin{macrocode}
\def\version{draft}
\input{childdoc.def}
\childdocforward{cdocsamp}
%    \end{macrocode}

%\iffalse
%</sampledraft>
%\fi
%
% %%%%%%%%%%%%%%%%%%%%%%%%%%%%%%%%%%%%%%
% \paragraph{Forwarding for Final Version of the Chapters.}
%
% The following forwarding files |cdocsfn1.tex| and |cdocsfn2.tex|
% (with identical content)
% compile the final versions of the child documents
% |cdocsch1.tex| and |cdocsch2.tex|, respectively:
%\iffalse
%<*samplefinal>
%\fi
%    \begin{macrocode}
\def\version{final}
\input{childdoc.def}
\childdocforwardprefix[cdocsamp]{cdocsfn}{cdocsch}
%    \end{macrocode}

%\iffalse
%</samplefinal>
%\fi
%
% %%%%%%%%%%%%%%%%%%%%%%%%%%%%%%%%%%%%%%
% \paragraph{Command Line Processing.}
%
% The following three command lines generate the output files
% |cdocscld|, |cdocscl1| and |cdocscl2|
% which should be identical to
% |cdocsdrf|, |cdocsch1| and |cdocsfn2|, respectively:
% \begin{center}
% \begin{tabular}{l}
% |latex -jobname cdocscld \|\\
% |  "\def\version{draft}\input{childdoc.def}\childdocforward{cdocsamp}"|\\
% |latex -jobname cdocscl1 \|\\
% |  "\input{childdoc.def}\childdocforward[cdocsamp]{cdocsch1}"|\\
% |latex -jobname cdocscl2 \|\\
% |  "\def\version{final}\input{childdoc.def}\childdocforward{cdocsch2}"|
% \end{tabular}
% \end{center}
% Note that the trailing backslash on each first line
% merely continues the input to the second line
% (for convenient cut ant paste).
% Furthermore, the command |latex| can be replaced by any
% of its alternative versions such as |pdflatex|.
%
% %%%%%%%%%%%%%%%%%%%%%%%%%%%%%%%%%%%%%%%%%%%%%%%%%%%%%%%%%%%%%%%%%%%%%%%%%%%%%%
% %%%%%%%%%%%%%%%%%%%%%%%%%%%%%%%%%%%%%%%%%%%%%%%%%%%%%%%%%%%%%%%%%%%%%%%%%%%%%%
% \section{Implementation}
%\iffalse
%<*package>
%\fi
%
% This section describes the definitions file |childdoc.def|.

% The definitions cannot be loaded using |\usepackage| or |\RequirePackage|
% which has a mechanism to prevent loading a style file more than once.
% When loading the definitions by means of |\input|
% multiple instances have to be prevented manually:
%\iffalse
%This code needs to be before the `\ProvidesFile' directive
%which is defined at the beginning of this file.
%Therefore it is also placed there and commented out here.
%</package>
%<*discard>
%\fi
%    \begin{macrocode}
\ifdefined\childdocmain\endinput\fi
%    \end{macrocode}
%\iffalse
%</discard>
%<*package>
%\fi
%
% \macro{\ifchilddoc}
% \macro{\ifchilddocmanual}
% The conditional |\ifchilddoc| tells whether a
% child (true) or main (false) document is being compiled.
% The conditional |\ifchilddocmanual| tells whether
% the |\includeonly| mechanism is used (false) or
% the selection of child files must be performed manually (true).
% The definitions initialise to false:
%    \begin{macrocode}
\newif\ifchilddoc
\newif\ifchilddocmanual
%    \end{macrocode}

% \macro{\childdocname}
% \macro{\childdocjob}
% The macro |\childdocname| stores the name of the main document
% to be compiled. The macro |\childdocjob| stores the name of
% the document on which the \LaTeX{} compiler was originally invoked.
% The content of |\jobname| cannot be compared
% to filenames specified in the source due to different catcodes.
% The following code rescans |\jobname|, stores the result
% in |\childdocname| and saves a copy in |\childdocjob|:
%    \begin{macrocode}
\edef\childdocname{\scantokens\expandafter{\jobname\noexpand}}
\let\childdocjob\childdocname
%    \end{macrocode}

% \macro{\childdocdisable}
% The macro |\childdocdisable| prevents the main file
% from being processed more than once.
% At this stage, the main document command |\childdocmain|
% is assumed to be called once again where it should do nothing.
% Any subsequent call to it should prevent
% a secondary processing of the main document
% It overwrites the forwarding commands
% |\childdocof| and |\childdocforward|
% with empty macros to prevent further inclusions of the main document:
%    \begin{macrocode}
\newcommand{\childdocdisable}
{
  \renewcommand{\childdocmain}[1]{\renewcommand{\childdocmain}[1]{\endinput}}
  \renewcommand{\childdocof}[1]{}
  \renewcommand{\childdocby}[2][]{}
  \renewcommand{\childdocforward}[2][]{}
  \renewcommand{\childdocdisable}{}
}
%    \end{macrocode}

% \macro{\childdocmain}
% The macro |\childdocmain| is to be called at the top of the main file
% with nothing or the main filename (without extension) as argument.
% First, it breaks loops.
% If the argument is not empty and does not match |\childdocname|
% (which is set by the first inclusion of |childdoc.def|),
% |\ifchilddoc| is set to true, |\includeonly| is applied to the child file
% and |\jobname| is set to the main file
% (for proper handling of |.aux| files):
%    \begin{macrocode}
\newcommand{\childdocmain}[1]
{
  \childdocdisable\childdocmain{}
  \if?#1?\else
    \begingroup
      \def\childdoctmp{#1}
      \ifx\childdoctmp\childdocname
        \def\childdoctmp{}
      \else
        \def\childdoctmp
        {
          \childdoctrue
          \includeonly{\childdocname}
          \def\childdocjob{#1}
          \def\jobname{#1}
        }
      \fi
      \expandafter
    \endgroup
    \childdoctmp
  \fi
}
%    \end{macrocode}

% \macro{\childdocof}
% The command |\childdocof| redirects
% compilation to the main file |#1|.
%    \begin{macrocode}
\newcommand{\childdocof}[1]
{
  \childdocdisable
  \childdoctrue
  \includeonly{\childdocname}
  \def\jobname{#1}
  \def\childdocjob{#1}
  \input{#1}
}
%    \end{macrocode}

% \macro{\childdocby}
% The command |\childdocby| ....
%    \begin{macrocode}
\newcommand{\childdocby}[2][]
{
  \childdocdisable
  \childdoctrue
  \childdocmanualtrue
  \if?#1?\else
    \def\jobname{#2}
  \fi
  \def\childdocjob{#2}
  \input{#2}
  \endinput
}
%    \end{macrocode}

% \macro{\childdocforward}
% The command |\childdocforward| redirects
% compilation to the main file or
% (if the optional argument is given) a child file.
% Parameters are set as if the main file
% or a child file starting with |\childdocof| was compiled.
% Then compilation is handed over to the main file:
%    \begin{macrocode}
\newcommand{\childdocforward}[2][]
{
  \begingroup
    \if?#1?
      \def\childdoctmp
      {
        \def\childdocname{#2}
        \def\childdocjob{#2}
        \def\jobname{#2}
        \input{#2}
        \endinput
      }
    \else
      \def\childdoctmp
      {
        \childdocdisable
        \def\childdocname{#2}
        \childdoctrue
        \includeonly{#2}
        \def\childdocjob{#1}
        \def\jobname{#1}
        \input{#1}
        \endinput
      }
    \fi
    \expandafter
  \endgroup
  \childdoctmp
}
%    \end{macrocode}

% \macro{\childdocforwardprefix}
% The command |\childdocforwardprefix| redirects
% compilation to the main or a child file by means of a pattern.
% The prefix |#1| in the current filename is replaced by |#2|
% and the suffix of the current filename is kept
% (it is assumed that the filename does not contain the substring `|~~~|'
% which is used as a delimiter).
% Compilation is handed over to the new file by |\childdocforward|:
%    \begin{macrocode}
\newcommand{\childdocforwardprefix}[3][]
{
  \begingroup
    \def\childdocextract #2##1~~~{\def\childdoctmp{\childdocforward[#1]{#3##1}}}
    \expandafter\childdocextract\childdocname~~~
    \expandafter
  \endgroup
  \childdoctmp
}
%    \end{macrocode}

% \macro{\childdoc}
% The deprecated macro |\childdoc| is a legacy version of |\childdocmain|:
%    \begin{macrocode}
\newcommand{\childdoc}{\childdocmain}
%    \end{macrocode}

% \macro{\childdocredirect}
% The deprecated macro |\childdocredirect| is a legacy version
% of |\childdocforward| and |\childdocforwardprefix|:
%    \begin{macrocode}
\newcommand{\childdocredirect}[2][]
{
  \begingroup
    \if?#1?
      \def\childdoctmp{\childdocforward{#2}}
    \else
      \def\childdoctmp{\childdocforwardprefix{#1}{#2}}
    \fi
    \expandafter
  \endgroup
  \childdoctmp
}
%    \end{macrocode}

%\iffalse
%</package>
%\fi
%
\endinput
|\\
|\childdocby{|\textit{main}|}|\\
\end{tabular}
\end{center}
%
The directive |\childdocby| is similar to |\childdocof|
described in \secref{sec:include},
but the subsequent selection of content must be done manually.
To that end, both |\ifchilddoc| and |\ifchilddocmanual|
will be true upon processing of a part,
and the name of the part is stored in |\childdocname|.
Note that |\jobname| will be set to the filename of the current part
so that each part receives an individual |.aux| file
that does not interfere with the |.aux| file(s) of the main document.
This behaviour can be altered by the alternative form
|\childdocby[*]{|\textit{main}|}| (with a non-empty optional argument)
which uses the |.aux| file of the main document
by setting |\jobname| to \textit{main}.

%%%%%%%%%%%%%%%%%%%%%%%%%%%%%%%%%%%%%%%%%%%%%%%%%%%%%%%%%%%%%%%%%%%%%%%%%%%%%%%%
\subsection{Driver Development}
\label{sec:driver}

The \textsf{childdoc} mechanism can also be use for the development
of definition files such as \LaTeX{} styles or classes.
This case differs from the above setup with multiple parts
included by |\include| in that no |\includeonly| should be invoked.
This can be achieved by starting the include file
(before |\ProvidesPackage|) with:
%
\begin{center}
\begin{tabular}{l}
|% \iffalse
%
% childdoc.dtx Copyright (C) 2017-2018 Niklas Beisert
%
% This work may be distributed and/or modified under the
% conditions of the LaTeX Project Public License, either version 1.3
% of this license or (at your option) any later version.
% The latest version of this license is in
%   http://www.latex-project.org/lppl.txt
% and version 1.3 or later is part of all distributions of LaTeX
% version 2005/12/01 or later.
%
% This work has the LPPL maintenance status `maintained'.
%
% The Current Maintainer of this work is Niklas Beisert.
%
% This work consists of the files childdoc.dtx and childdoc.ins
% and the derived files childdoc.def and cdocsamp.tex with
% cdocsch1.tex, cdocsch2.tex, cdocsdrf.tex, cdocsfn1.tex, cdocsfn2.tex.
%
%<package>\ifdefined\childdocmain\endinput\fi
%<package>\ProvidesFile{childdoc.def}[2018/12/30 v2.0 child document driver]
%<samplemain>\ProvidesFile{cdocsamp.tex}[2018/12/30 v2.0 sample for childdoc]
%<*driver>
%\ProvidesFile{childdoc.drv}[2018/12/30 v2.0 childdoc reference manual file]
\PassOptionsToClass{10pt,a4paper}{article}
\documentclass{ltxdoc}

\usepackage[margin=35mm]{geometry}
\usepackage{hyperref}
\usepackage{hyperxmp}
\usepackage[usenames]{color}

\hypersetup{colorlinks=true}
\hypersetup{pdfstartview=FitH}
\hypersetup{pdfpagemode=UseNone}
\hypersetup{pdfsource={}}
\hypersetup{pdflang={en-UK}}
\hypersetup{pdfcopyright={Copyright 2017-2018 Niklas Beisert.
  This work may be distributed and/or modified under the
  conditions of the LaTeX Project Public License, either version 1.3
  of this license or (at your option) any later version.}}
\hypersetup{pdflicenseurl={http://www.latex-project.org/lppl.txt}}
\hypersetup{pdfcontactaddress={ETH Zurich, ITP, HIT K,
  Wolfgang-Pauli-Strasse 27}}
\hypersetup{pdfcontactpostcode={8093}}
\hypersetup{pdfcontactcity={Zurich}}
\hypersetup{pdfcontactcountry={Switzerland}}
\hypersetup{pdfcontactemail={nbeisert@itp.phys.ethz.ch}}
\hypersetup{pdfcontacturl={http://people.phys.ethz.ch/\xmptilde nbeisert/}}

\newcommand{\secref}[1]{\hyperref[#1]{section \ref*{#1}}}

\parskip1ex
\parindent0pt
\let\olditemize\itemize
\def\itemize{\olditemize\parskip0pt}

\begin{document}

\title{The \textsf{childdoc} Package}
\hypersetup{pdftitle={The childdoc Package}}
\author{Niklas Beisert\\[2ex]
  Institut f\"ur Theoretische Physik\\
  Eidgen\"ossische Technische Hochschule Z\"urich\\
  Wolfgang-Pauli-Strasse 27, 8093 Z\"urich, Switzerland\\[1ex]
  \href{mailto:nbeisert@itp.phys.ethz.ch}
  {\texttt{nbeisert@itp.phys.ethz.ch}}}
\hypersetup{pdfauthor={Niklas Beisert}}
\hypersetup{pdfsubject={Manual for the LaTeX2e Package childdoc}}
\date{30 December 2018, \textsf{v2.0}}
\maketitle

\begin{abstract}\noindent
\textsf{childdoc} is a \LaTeXe{} package
that enables the direct compilation
of document sections included by |\include|
to individual files.
\end{abstract}

\begingroup
\parskip0ex
\tableofcontents
\endgroup

%%%%%%%%%%%%%%%%%%%%%%%%%%%%%%%%%%%%%%%%%%%%%%%%%%%%%%%%%%%%%%%%%%%%%%%%%%%%%%%%
%%%%%%%%%%%%%%%%%%%%%%%%%%%%%%%%%%%%%%%%%%%%%%%%%%%%%%%%%%%%%%%%%%%%%%%%%%%%%%%%
\section{Introduction}

\LaTeX{} provides a mechanism to structure a large document (such as a book)
into a main file and several child files (containing the chapters)
using the |\include| command.
This mechanism is beneficial for documents
which span hundreds of pages in order to
make the source file(s) more manageable.
Moreover, compilation can be restricted to
selected child files by means of the |\includeonly| command.
The latter feature can be used to reduce the compilation time while editing
(this was significantly more useful in the earlier days of \LaTeX{})
or to generate a smaller document which is easier to navigate.
Another application of |\includeonly| is to generate
documents consisting of selected parts of the complete document.

However, there are a few drawbacks of the plain |\include| mechanism:
\begin{itemize}
\item
The child files cannot be compiled on their own,
they can only be compiled via the main file.
A naive editing environment
(such as a text editor with an option
to have the current file processed by \LaTeX)
may require one to switch to the main file before compiling;
attempting to compile the child file produces errors.
\item
The main file must be modified (each time)
to adjust the |\includeonly| command
to the present needs. This easily leaves the main file in a messy state.
\item
The generated document will always carry the filename
of the main document. This is inconvenient if
several child files are to be compiled and
to be kept for distribution.
\end{itemize}

The present package provides a simple interface
to make child files individually compilable by \LaTeX{}.
Compiling a child file then has the same effect as compiling
the main file with an |\includeonly| command
to select the appropriate child.
Moreover the generated document will carry the name of the child
rather than the main file.
This resolves all three above issues.

This feature is meant to make the editing of books,
thesis documents and lecture notes somewhat more convenient.
However, the package can also be used efficiently for
composing a series of documents (such as exercise sheets)
which are typically distributed individually.
It then assists the author in generating the individual documents
(potentially in different versions)
as well as a document containing the collected series.
Another application is in developing style files
or other kinds of included material
where compilation of the style file could redirect
to a sample or test file.

%%%%%%%%%%%%%%%%%%%%%%%%%%%%%%%%%%%%%%%%%%%%%%%%%%%%%%%%%%%%%%%%%%%%%%%%%%%%%%%%
%%%%%%%%%%%%%%%%%%%%%%%%%%%%%%%%%%%%%%%%%%%%%%%%%%%%%%%%%%%%%%%%%%%%%%%%%%%%%%%%
\section{Usage}

First of all, the package \textsf{childdoc} is \emph{not} a standard
\LaTeXe{} |.sty| style file! Therefore it needs to be invoked in
a non-standard way.

%%%%%%%%%%%%%%%%%%%%%%%%%%%%%%%%%%%%%%%%%%%%%%%%%%%%%%%%%%%%%%%%%%%%%%%%%%%%%%%%
\subsection{Included Files}
\label{sec:include}

%%%%%%%%%%%%%%%%%%%%%%%%%%%%%%%%%%%%%%%%
\DescribeMacro{\childdocmain}
To use the package, add the commands
\begin{center}
\begin{tabular}{l}
|\input{childdoc.def}|\\
|\childdocmain{}|\\
\end{tabular}
\end{center}
at the very top of the main \LaTeX{} file,
in particular \emph{before} the |\documentclass| statement!
The argument of |\childdocmain| should be left empty
(but it must be present).

%%%%%%%%%%%%%%%%%%%%%%%%%%%%%%%%%%%%%%%%
\DescribeMacro{\childdocof}
Furthermore, add the commands
\begin{center}
\begin{tabular}{l}
|\input{childdoc.def}|\\
|\childdocof{|\textit{main}|}|\\
\end{tabular}
\end{center}
at the top of every child file \textit{child}
which is included by |\include{|\textit{child}|}|
from within the main file
(or at least for those files to be compiled individually).
The argument \textit{main} must be the filename of the main file.

There are a couple of
considerations in setting up the main and child documents:

%%%%%%%%%%%%%%%%%%%%%%%%%%%%%%%%%%%%%%%%
\paragraph{Restrictions.}

Please note the following restrictions:
\begin{itemize}
\item
|\childdocmain| must be called with one argument \textit{main}
to ensure compatibility with earlier version of the package.
It must either be empty (|\childdocmain{}|)
or precisely match the filename of the main file in which it is specified.
See \secref{sec:detection} for further information.
\item
The filename \textit{main} must be specified without the |.tex| extension.
\item
The filename \textit{main} is case sensitive
(even in case-insensitive file systems)
due to internal string comparison.
\item
The argument \textit{main} should be fully expanded, it cannot be a macro.
\item
Subdirectories and special characters should be avoided in filenames.
\item
The command |\childdocmain{|\textit{main}|}| must be followed by a whitespace.
It should not be followed immediately by another command
or by a comment mark `|%|'.
This is because the \TeX{} parser reads the token immediately following
the argument of |\childdocmain| and puts it
at the beginning of every child section;
however, a white\-space is ignored.
\end{itemize}

%%%%%%%%%%%%%%%%%%%%%%%%%%%%%%%%%%%%%%%%
\paragraph{Content of Main File.}

It is advisable to place all content in the child files included by |\include|.
Any output contained in the main file will appear in all child documents
unless suppressed manually;
it cannot be suppressed automatically by the |\includeonly| directive
and thus should normally be avoided.
A method to include some content in the main file
by means of conditional processing is described in \secref{sec:conditional}.

%%%%%%%%%%%%%%%%%%%%%%%%%%%%%%%%%%%%%%%%
\paragraph{Page Numbering.}

When only a part of the document is compiled,
the appropriate numbering of pages
(as well as other status parameters)
is determined from the |.aux| files.
The latter contain information from previous passes.
However this information needs to propagate through
all intermediate child documents.
Therefore the page numbering in child documents may well
be inconsistent until the complete document is compiled at least once.

A useful (if unconventional) way to always ensure a consistent
page numbering is to restart the numbering in each child document
and denote the pages by `\textit{child}|.|\textit{page}'
where \textit{child} represents the chapter/section number of the child file.
This can be achieved by the command
|\numberwithin{page}{|\textit{child}|}|
of the \textsf{amsmath} package
where \textit{child} can be |chapter| or |section|
depending on the chosen structuring.
Alternatively, one can modify the macro |\thepage| appropriately
and reset the counter |page| at the start of each child file.

%%%%%%%%%%%%%%%%%%%%%%%%%%%%%%%%%%%%%%%%%%%%%%%%%%%%%%%%%%%%%%%%%%%%%%%%%%%%%%%%
\subsection{Conditional Processing}
\label{sec:conditional}

The package provides a mechanism to compile different versions
of a document. To customise the versions further some conditional processing
can come in handy to distinguish which version is being compiled.
The package provides two macros to describe the compilation context:

%%%%%%%%%%%%%%%%%%%%%%%%%%%%%%%%%%%%%%%%
\DescribeMacro{\ifchilddoc}
The conditional |\ifchilddoc| distinguishes between the compilation of
child documents and the main document:
%
\begin{center}
|\ifchilddoc |\textit{child-code}| |[|\||else |\textit{main-code}]| \||fi|
\end{center}

%%%%%%%%%%%%%%%%%%%%%%%%%%%%%%%%%%%%%%%%
\DescribeMacro{\childdocname}
\DescribeMacro{\childdocjob}
The macro |\childdocname| contains the filename (without extension)
of the main or child file being processed.
Note that |\childdocjob| will always contain the name of the main file.

%%%%%%%%%%%%%%%%%%%%%%%%%%%%%%%%%%%%%%%%
\paragraph{Title Page.}

Conditional processing can be used to include a title or banner page
in the main document when proper precautions are taken.
Importantly, the code in the main file should ensure that the page counter
(as well as other status parameters which are stored in the |.aux| files)
takes the same value after the conditional processing.
Otherwise the page numbers may take divergent values
depending on which part is compiled.

For example, a title page could be declared by:
%
\begin{center}
\begin{tabular}{l}
|\ifchilddoc\||else|\\
|\addtocounter{page}{-1}|\\
\textit{code for title page}\\
|\newpage|\\
|\||fi|
\end{tabular}
\end{center}
%
A banner page for the child documents can be generated by:
%
\begin{center}
\begin{tabular}{l}
|\ifchilddoc|\\
|\addtocounter{page}{-1}|\\
\textit{code for banner page}\\
|\newpage|\\
|\||fi|
\end{tabular}
\end{center}
%
Here one could write a message such as:
\begin{center}
|This is the part \childdocname{} of \childdocjob{}.|
\end{center}

%%%%%%%%%%%%%%%%%%%%%%%%%%%%%%%%%%%%%%%%%%%%%%%%%%%%%%%%%%%%%%%%%%%%%%%%%%%%%%%%
\subsection{Flags}
\label{sec:flags}

The package makes it easy to generate different versions
of the main or child documents.
To this end compilation flags can be defined
and assigned different default values.
They will be particularly useful in conjunction
with the forwarding mechanism described in \secref{sec:forward}.

For example, it may be useful to have a flag |\version|
which can be set to |draft| or |final|.
The document source will contain some conditional code
depending on the value of |\version|.
Suppose further, the flag should default to |final| for the main file
and to |draft| for child files
which is a natural assignment for editing the document.
This is achieved by placing the following code
in the preamble of the main document
(below the |\childdocmain| directive):
%
\begin{center}
\begin{tabular}{l}
|\ifchilddoc|\\
|\providecommand{\version}{draft}|\\
|\||else|\\
|\providecommand{\version}{final}|\\
|\||fi|
\end{tabular}
\end{center}
%
The definition by |\providecommand| makes sure
that previous definitions are not overwritten.
Further statements |\providecommand{\version}{...}|
can thus be added before the above code to override it.

For the main file, one might add a line
(between |\childdocmain| and the above block)
%
\begin{center}
|%\ifchilddoc\||else\providecommand{\version}{draft}\||fi|
\end{center}
%
which can be uncommented to produce a draft version.
Likewise one can add a line to the very top of a child file
(above the |\childdocof{|\textit{main}|}| directive)
%
\begin{center}
|%\providecommand{\version}{final}|
\end{center}
%
which can be uncommented to produce the final version of this child document.

%%%%%%%%%%%%%%%%%%%%%%%%%%%%%%%%%%%%%%%%%%%%%%%%%%%%%%%%%%%%%%%%%%%%%%%%%%%%%%%%
\subsection{Forwarding}
\label{sec:forward}

Different versions of the main or child documents
using compilation flags as described in \secref{sec:flags}
can be (permanently) stored in different files
for convenient compilation, viewing and distribution.
To this end, the package defines a command
to pass on compilation to a different file:

%%%%%%%%%%%%%%%%%%%%%%%%%%%%%%%%%%%%%%%%
\DescribeMacro{\childdocforward}
The command |\childdocforward| redirects processing to
another source file:
%
\begin{center}
\begin{tabular}{l}
|\input{childdoc.def}|\\
|\childdocforward[|\textit{main}|]{|\textit{dest}|}|\\
\end{tabular}
\end{center}
%
The argument \textit{dest} is the destination file
(without extension).
It should be the main file or one of the child files.
Note that further \textsf{childdoc} directives
such as |\childdocof| and |\childdocforward|
in the indicated file will be processed in this form.
The optional argument \textit{main}
passes on directly to the main file \textit{main}
while pretending to compile the child \textit{dest}.
This form behaves as if \textit{dest}
issues |\childdocof{|\textit{main}|}| right away,
and no further \textsf{childdoc} directives will be processed.

%%%%%%%%%%%%%%%%%%%%%%%%%%%%%%%%%%%%%%%%
\DescribeMacro{\...prefix}
In the alternative form |\childdocforwardprefix|,
%
\begin{center}
\begin{tabular}{l}
|\input{childdoc.def}|\\
|\childdocforwardprefix[|\textit{main}|]{|\textit{prefix}|}{|\textit{dest}|}|
\end{tabular}
\end{center}
%
the destination file is determined by a pattern
depending on the current file:
To make this work, the current file must be called
`{\textit{prefix}\hspace{0.2em}\textit{suffix}}'
with \textit{prefix} matching precisely the argument.
Processing is then passed on to the file
`{\textit{dest}\hspace{0.2em}\textit{suffix}}'.
Surely, the same effect is achieved by
directly specifying the
argument `{\textit{dest}\hspace{0.2em}\textit{suffix}}'
in the first form.
However, that requires to set up a different file
for each child. With the alternative form of the command
all these files can have exactly the same content
which simplifies setting them up and maintaining them.

For example, the following file |draft.tex|
with a compilation flag |\version| as described in \secref{sec:flags}
compiles the main document as a draft:
%
\begin{center}
\begin{tabular}{l}
|\def\version{draft}|\\
|\input{childdoc.def}|\\
|\childdocforward{|\textit{main}|}|
\end{tabular}
\end{center}
%
Likewise, the following files |final|\textit{nn}|.tex|
compile the final version of the child document
|child|\textit{nn}|.tex|:
%
\begin{center}
\begin{tabular}{l}
|\def\version{final}|\\
|\input{childdoc.def}|\\
|\childdocforwardprefix{final}{child}|
\end{tabular}
\end{center}
%

Note that when several versions of a main file and/or of each child file
are to be generated, it may be convenient to set up a |Makefile| or
shell script to automatise the process.

%%%%%%%%%%%%%%%%%%%%%%%%%%%%%%%%%%%%%%%%%%%%%%%%%%%%%%%%%%%%%%%%%%%%%%%%%%%%%%%%
\subsection{Command Line Processing}
\label{sec:commandline}

The effect of redirection files can also be achieved by invoking
the \LaTeX{} compiler with a more elaborate command line.
Most conveniently this should be done as part
of a shell script or a |Makefile|.

When using \textsf{childdoc} in the main file, the following
command lines effectively perform a redirection
(note that depending on the shell being used,
backslashes may have to be doubled: `|\|' $\to$ `|\\|'):
%
\begin{center}
|... -jobname "|\textit{target}|" |\\|"|[\textit{flags}]%
|\input{childdoc.def}\childdocforward[|\textit{main}|]{|\textit{dest}|}"|
\end{center}
%
Here \textit{target} is the name of the output file,
\textit{main} is the name of the main file
and \textit{dest} is the name of the main or child file to be processed
(all filenames without extensions).
The optional argument \textit{main} can be omitted
if \textit{main} matches \textit{dest}.
Optionally, compilation \textit{flags} can be defined via |\def| commands.
This command line makes the \TeX{} engine believe
it is compiling the file \textit{target}
whose content is specified as the latter parameter.
The provided code then forwards the processing to
\textit{main} or \textit{dest} as described in \secref{sec:forward}.

%%%%%%%%%%%%%%%%%%%%%%%%%%%%%%%%%%%%%%%%%%%%%%%%%%%%%%%%%%%%%%%%%%%%%%%%%%%%%%%%
\subsection{Include by Input}
\label{sec:input}

Including child documents by |\include| has some restrictions by design.
Most notably, the content of a child document always occupies
its own set of pages; pages cannot be shared between child documents.
Usually, this behaviour makes perfect sense
because each child document contain an essential part of the document.
However, in some situations it may be desirable to compose
a document from a collection of parts
without having mandatory page breaks between then.
For this case, the package
provides a mechanism to include parts
by |\input| which can also be processed individually.
However, by construction this mechanism
requires manual handling of the content to be output.

%%%%%%%%%%%%%%%%%%%%%%%%%%%%%%%%%%%%%%%%
\DescribeMacro{\ifchilddocmanual}
The main file should be prepared as usual, see \secref{sec:include}.
However, the document body must make a distinction
between processing of an individual part and of the main document, e.g.:
%
\begin{center}
\begin{tabular}{l}
|\ifchilddocmanual|\\
|\input{\childdocname}|\\
|\||else|\\
\textit{document body with }|\input{|\textit{part}|}|\\
|\||fi|
\end{tabular}
\end{center}
%
The conditional |\ifchilddocmanual| is true whenever
a part to be included by |\input| is being compiled,
and the name of the part is stored in |\childdocname|.

%%%%%%%%%%%%%%%%%%%%%%%%%%%%%%%%%%%%%%%%
\DescribeMacro{\childdocby}
Each part to be included by |\input| should start with:
%
\begin{center}
\begin{tabular}{l}
|\input{childdoc.def}|\\
|\childdocby{|\textit{main}|}|\\
\end{tabular}
\end{center}
%
The directive |\childdocby| is similar to |\childdocof|
described in \secref{sec:include},
but the subsequent selection of content must be done manually.
To that end, both |\ifchilddoc| and |\ifchilddocmanual|
will be true upon processing of a part,
and the name of the part is stored in |\childdocname|.
Note that |\jobname| will be set to the filename of the current part
so that each part receives an individual |.aux| file
that does not interfere with the |.aux| file(s) of the main document.
This behaviour can be altered by the alternative form
|\childdocby[*]{|\textit{main}|}| (with a non-empty optional argument)
which uses the |.aux| file of the main document
by setting |\jobname| to \textit{main}.

%%%%%%%%%%%%%%%%%%%%%%%%%%%%%%%%%%%%%%%%%%%%%%%%%%%%%%%%%%%%%%%%%%%%%%%%%%%%%%%%
\subsection{Driver Development}
\label{sec:driver}

The \textsf{childdoc} mechanism can also be use for the development
of definition files such as \LaTeX{} styles or classes.
This case differs from the above setup with multiple parts
included by |\include| in that no |\includeonly| should be invoked.
This can be achieved by starting the include file
(before |\ProvidesPackage|) with:
%
\begin{center}
\begin{tabular}{l}
|\input{childdoc.def}|\\
|\childdocforward{|\textit{main}|}|\\
\end{tabular}
\end{center}
%
or alternatively with:
%
\begin{center}
\begin{tabular}{l}
|\input{childdoc.def}|\\
|\childdocby{|\textit{main}|}|\\
\end{tabular}
\end{center}
%
Both forms have slightly different effects as described above.
The main file is prepared as usual, see \secref{sec:include}.

%%%%%%%%%%%%%%%%%%%%%%%%%%%%%%%%%%%%%%%%%%%%%%%%%%%%%%%%%%%%%%%%%%%%%%%%%%%%%%%%
\subsection{Legacy Detection}
\label{sec:detection}

The directive |\childdocmain| in the main file can detect
whether the complete document or merely a child is to be compiled
even without using the directive |\childdocof|.
This method is deprecated because it is less robust
and there is no compelling reason to use it;
it is merely provided for backward compatibility
and it may be removed in future versions.

If the detection mechanism is to be used,
it is mandatory to correctly specify
the filename of the main file as the argument of |\childdocmain|:
%
\begin{center}
\begin{tabular}{l}
|\input{childdoc.def}|\\
|\childdocmain{|\textit{main}|}|\\
\end{tabular}
\end{center}
%
If |\jobname| does not match the argument \textit{main} of |\childdocmain|,
it is assumed that |\jobname| points to the child file to be compiled.
When using |\childdocmain| with the main file specified as argument,
it suffices to start a child file
with just |\input{|\textit{main}|}|
without loading of the package and using |\childdocof|.
If instead all processing is done
with the appropriate \textsf{childdoc} directives,
the argument of \textit{main} of |\childdocmain| can be empty.

An alternative version of the command line processing described
in \secref{sec:commandline} using the detection mechanism reads:
%
\begin{center}
|... -jobname "|\textit{target}|" "|[\textit{flags}]%
[|\def\jobname{|\textit{dest}|}|]|\input{|\textit{main}|}"|
\end{center}

%%%%%%%%%%%%%%%%%%%%%%%%%%%%%%%%%%%%%%%%%%%%%%%%%%%%%%%%%%%%%%%%%%%%%%%%%%%%%%%%
\subsection{Manual Code}
\label{sec:manual}

In case one cannot be certain whether the definitions file |childdoc.def|
is installed on the target \TeX{} distribution
and one prefers not to ship it,
it is conceivable to paste a few relevant commands into the sources.

To that end, drop all statements |\input{childdoc.def}|
and perform the replacements as outlined below.
Instead of |\childdocmain{|\textit{main}|}| add the following code
to the top of the main file:
%
\begin{center}
\begin{tabular}{l}
|\||ifdefined\childdocname\endinput\||fi\newif\ifchilddoc|\\
|\edef\childdocname{\scantokens\expandafter{\jobname\noexpand}}|\\
|\def\childdocmain{|\textit{main}|}\||ifx\childdocmain\childdocname\||else|\\
|\childdoctrue\includeonly{\childdocname}\let\jobname\childdocmain\||fi|\\
\end{tabular}
\end{center}
%
Instead of |\childdocof{|\textit{main}|}| just include the main file
at the top of each child file:
%
\begin{center}
|\input{|\textit{main}|}|
\end{center}
%
A simple redirection |\childdocforward{|\textit{dest}|}| is achieved by:
%
\begin{center}
|\def\jobname{|\textit{dest}|}\input{\jobname}|
\end{center}
%
The redirection with prefix
|\childdocforwardprefix[|\textit{prefix}|]{|\textit{dest}|}|
is accomplished by:
%
\begin{center}
\begin{tabular}{l}
|{\edef\jobname{\scantokens\expandafter{\jobname\noexpand}}|\\
|\def\redirectjob |\textit{prefix}|#1~~~{\gdef\jobname{|\textit{dest}|#1}}|\\
|\expandafter\redirectjob\jobname~~~}\input{\jobname}|
\end{tabular}
\end{center}

In an alternative approach,
child documents can be compiled by a specific command line
without additional code or specific definitions:
%
\begin{center}
|... -jobname "|\textit{target}|" "|[\textit{flags}]%
|\includeonly{|\textit{dest}|}\input{|\textit{main}|}"|
\end{center}
%

%%%%%%%%%%%%%%%%%%%%%%%%%%%%%%%%%%%%%%%%%%%%%%%%%%%%%%%%%%%%%%%%%%%%%%%%%%%%%%%%
%%%%%%%%%%%%%%%%%%%%%%%%%%%%%%%%%%%%%%%%%%%%%%%%%%%%%%%%%%%%%%%%%%%%%%%%%%%%%%%%
\section{Information}

%%%%%%%%%%%%%%%%%%%%%%%%%%%%%%%%%%%%%%%%%%%%%%%%%%%%%%%%%%%%%%%%%%%%%%%%%%%%%%%%
\subsection{Copyright}

Copyright \copyright{} 2017--2018 Niklas Beisert

This work may be distributed and/or modified under the
conditions of the \LaTeX{} Project Public License, either version 1.3
of this license or (at your option) any later version.
The latest version of this license is in
  \url{http://www.latex-project.org/lppl.txt}
and version 1.3 or later is part of all distributions of \LaTeX{}
version 2005/12/01 or later.

This work has the LPPL maintenance status `maintained'.

The Current Maintainer of this work is Niklas Beisert.

This work consists of the files |README.txt|, |childdoc.ins| and |childdoc.dtx|
as well as the derived files |childdoc.def|, |cdocsamp.tex|
with |cdocsch1.tex|, |cdocsch2.tex|, |cdocspt3.tex|, |cdocspt4.tex|,
|cdocsdrf.tex|, |cdocsfn1.tex|, |cdocsfn2.tex|
as well as |childdoc.pdf|.

%%%%%%%%%%%%%%%%%%%%%%%%%%%%%%%%%%%%%%%%%%%%%%%%%%%%%%%%%%%%%%%%%%%%%%%%%%%%%%%%
\subsection{Files and Installation}

The package consists of the files:
%
\begin{center}
\begin{tabular}{ll}
    |README.txt|   & readme file \\
    |childdoc.ins| & installation file \\
    |childdoc.dtx| & source file \\
    |childdoc.def| & definition file \\
    |cdocsamp.tex| & sample main file \\
    |cdocsch1.tex| & sample include file \\
    |cdocsch2.tex| & sample include file \\
    |cdocspt3.tex| & sample part file \\
    |cdocspt4.tex| & sample part file \\
    |cdocsdrf.tex| & sample redirection file \\
    |cdocsfn1.tex| & sample redirection file \\
    |cdocsfn2.tex| & sample redirection file \\
    |childdoc.pdf| & manual
\end{tabular}
\end{center}
%
The distribution consists of the files
|README.txt|, |childdoc.ins| and |childdoc.dtx|.
%
\begin{itemize}
\item
Run (pdf)\LaTeX{} on |childdoc.dtx|
to compile the manual |childdoc.pdf| (this file).
\item
Run \LaTeX{} on |childdoc.ins| to create the definitions file |childdoc.def|
and the sample |cdocsamp.tex| with include files
|cdocsch1.tex|, |cdocsch2.tex|, |cdocspt3.tex|, |cdocspt4.tex|,
|cdocsdrf.tex|, |cdocsfn1.tex|, |cdocsfn2.tex|.
Then copy the file |childdoc.def| to an appropriate directory of your \LaTeX{}
distribution, e.g.\ \textit{texmf-root}|/tex/latex/childdoc|.
\end{itemize}

%%%%%%%%%%%%%%%%%%%%%%%%%%%%%%%%%%%%%%%%%%%%%%%%%%%%%%%%%%%%%%%%%%%%%%%%%%%%%%%%
\subsection{Related CTAN Packages}

There are several other packages which offer a similar functionality:
%
\begin{itemize}
\item
The packages
\href{http://ctan.org/pkg/docmute}{\textsf{docmute}},
\href{http://ctan.org/pkg/includex}{\textsf{includex}} and
\href{http://ctan.org/pkg/standalone}{\textsf{standalone}}
provide commands to include only the document body of
a child file thus allowing both files to be compiled individually.
\item
The packages \href{http://ctan.org/pkg/subdocs}{\textsf{subdocs}}
and \href{http://ctan.org/pkg/subfiles}{\textsf{subfiles}}
provide structures in which the main and child documents can be
encapsulated and allowing them to be compiled individually.
The inclusion mechanism is different from the conventional |\include|.
\item
The package \href{http://ctan.org/pkg/combine}{\textsf{combine}}
is an elaborate solution to combine several documents into one.
\end{itemize}
%
See also the CTAN topic \href{http://ctan.org/topic/subdocs}{\textsf{subdocs}}
for further related packages.
The present package differs from the above solutions in that
a document structure constructed with the conventional |\include| mechanism
just needs two extra commands at the top of every file
such that all constituent files can be compiled individually.

%%%%%%%%%%%%%%%%%%%%%%%%%%%%%%%%%%%%%%%%%%%%%%%%%%%%%%%%%%%%%%%%%%%%%%%%%%%%%%%%
%\subsection{Feature Suggestions}
%
%The following is a list of features which may be useful for future
%versions of this package:
%%
%\begin{itemize}
%\item
%\ldots
%\end{itemize}

%%%%%%%%%%%%%%%%%%%%%%%%%%%%%%%%%%%%%%%%%%%%%%%%%%%%%%%%%%%%%%%%%%%%%%%%%%%%%%%%
\subsection{Revision History}

%%%%%%%%%%%%%%%%%%%%%%%%%%%%%%%%%%%%%%%%
\paragraph{v2.0:} 2018/12/30

\begin{itemize}
\item
immediate forward processing
\item
added |\childdocby| mechanism
\item
manual restructured
\end{itemize}

%%%%%%%%%%%%%%%%%%%%%%%%%%%%%%%%%%%%%%%%
\paragraph{v1.6:} 2018/01/17

\begin{itemize}
\item
application for development of include files
\item
corrections to manual
\end{itemize}

%%%%%%%%%%%%%%%%%%%%%%%%%%%%%%%%%%%%%%%%
\paragraph{v1.5:} 2017/05/21

\begin{itemize}
\item
more complete structuring introduced
\item
|\childdocof| introduced
\item
|\childdoc| renamed to |\childdocmain|
\item
|\childredirect| renamed to |\childdocforward| and |\childdocforwardprefix|
and functionality expanded
\end{itemize}

%%%%%%%%%%%%%%%%%%%%%%%%%%%%%%%%%%%%%%%%
\paragraph{v1.0:} 2017/04/27

\begin{itemize}
\item
manual and install package
\item
first version published on CTAN
\end{itemize}

%%%%%%%%%%%%%%%%%%%%%%%%%%%%%%%%%%%%%%%%
\paragraph{v0.6:} 2017/04/26

\begin{itemize}
\item
redirection mechanism added
\end{itemize}

%%%%%%%%%%%%%%%%%%%%%%%%%%%%%%%%%%%%%%%%
\paragraph{v0.5:} 2017/04/26

\begin{itemize}
\item
functionality in definition file
\end{itemize}


%%%%%%%%%%%%%%%%%%%%%%%%%%%%%%%%%%%%%%%%%%%%%%%%%%%%%%%%%%%%%%%%%%%%%%%%%%%%%%%%
%%%%%%%%%%%%%%%%%%%%%%%%%%%%%%%%%%%%%%%%%%%%%%%%%%%%%%%%%%%%%%%%%%%%%%%%%%%%%%%%
%%%%%%%%%%%%%%%%%%%%%%%%%%%%%%%%%%%%%%%%%%%%%%%%%%%%%%%%%%%%%%%%%%%%%%%%%%%%%%%%
\appendix

\settowidth\MacroIndent{\rmfamily\scriptsize 000\ }

 \DocInput{childdoc.dtx}

\end{document}
%</driver>
% \fi
%
% %%%%%%%%%%%%%%%%%%%%%%%%%%%%%%%%%%%%%%%%%%%%%%%%%%%%%%%%%%%%%%%%%%%%%%%%%%%%%%
% %%%%%%%%%%%%%%%%%%%%%%%%%%%%%%%%%%%%%%%%%%%%%%%%%%%%%%%%%%%%%%%%%%%%%%%%%%%%%%
% \section{Sample}
%\iffalse
%<*samplemain>
%\fi
%
% The following presents a sample document
% with two chapters, two parts, a title page,
% a compile flag as well as three forwarding files to set the flag.
% It consists of eight |.tex| files:
% \begin{center}
% \begin{tabular}{ll}
% |cdocsamp.tex|&main file\\
% |cdocsch1.tex|&include file for chapter 1\\
% |cdocsch2.tex|&include file for chapter 2\\
% |cdocspt3.tex|&include file for part 3\\
% |cdocspt4.tex|&include file for part 4\\
% |cdocsdrf.tex|&forwarding file for main file in draft mode\\
% |cdocsfi1.tex|&forwarding file for final version of chapter 1\\
% |cdocsfi2.tex|&forwarding file for final version of chapter 2\\
% \end{tabular}
% \end{center}
% Each of the eight files can be compiled directly by the \LaTeX{} compiler.
%
% %%%%%%%%%%%%%%%%%%%%%%%%%%%%%%%%%%%%%%
% \paragraph{Main File.}
%
% The main file is called |cdocsamp.tex|.
%
% Load the \textsf{childdoc} definitions and
% declare the filename for the main document:
%    \begin{macrocode}
\input{childdoc.def}
\childdocmain{}
%    \end{macrocode}

% Optional override for |\version| flag:
%    \begin{macrocode}
%%\ifchilddoc\else\providecommand{\version}{draft}\fi
%    \end{macrocode}

% Define the default values for the |\version| flag
% (|final| for the main file and |draft| for childs):
%    \begin{macrocode}
\ifchilddoc
\providecommand{\version}{draft}
\else
\providecommand{\version}{final}
\fi
%    \end{macrocode}

% Load the standard document class:
%    \begin{macrocode}
\documentclass[12pt]{article}
%    \end{macrocode}

% Start the document body:
%    \begin{macrocode}
\begin{document}
%    \end{macrocode}

% Declare a title page.
% Print title, part of document being processed and version flag:
%    \begin{macrocode}
\addtocounter{page}{-1}
\begin{center}
{\LARGE\bfseries{}childdoc example\par}
\vspace{1cm}
\ifchilddoc
\ifchilddocmanual part\else chapter\fi:
`\childdocname' of `\childdocjob'\par
\else
main document: `\childdocjob'\par
\fi
version: \version\par
\end{center}
\newpage
%    \end{macrocode}

% Manually include selected file,
% otherwise process as usual:
%    \begin{macrocode}
\ifchilddocmanual
\section*{part `\childdocname'}
\input{\childdocname}
\else
%    \end{macrocode}

% Include the two chapters:
%    \begin{macrocode}
\include{cdocsch1}
\include{cdocsch2}
%    \end{macrocode}

% Include the two parts unless only chapters should be displayed:
%    \begin{macrocode}
\ifchilddoc\else
\section{part three}
\input{cdocspt3}
\section{part four}
\input{cdocspt4}
\fi
%    \end{macrocode}

% Process as usual until here:
%    \begin{macrocode}
\fi
%    \end{macrocode}

% End of document body:
%    \begin{macrocode}
\end{document}
%    \end{macrocode}
%\iffalse
%</samplemain>
%\fi
%
% %%%%%%%%%%%%%%%%%%%%%%%%%%%%%%%%%%%%%%
% \paragraph{Chapter Include Files.}
%
% The include files are called |cdocsch1.tex| and |cdocsch2.tex|.
%
%\iffalse
%<*samplechap1|samplechap2>
%\fi

% Optional override for |\version| flag:
%    \begin{macrocode}
%%\providecommand{\version}{final}
%    \end{macrocode}

% Include the main document:
%    \begin{macrocode}
\input{childdoc.def}
\childdocof{cdocsamp}
%    \end{macrocode}

%\iffalse
%</samplechap1|samplechap2>
%\fi
%
%\iffalse
%<*samplechap1>
%\fi
% Some text for chapter 1:
%    \begin{macrocode}
\section{one}
some text in chapter one
%    \end{macrocode}

%\iffalse
%</samplechap1>
%\fi
% Some text for chapter 2:
%\iffalse
%<*samplechap2>
%\fi
%    \begin{macrocode}
\section{two}
more text in chapter two
%    \end{macrocode}

%\iffalse
%</samplechap2>
%\fi
%
% %%%%%%%%%%%%%%%%%%%%%%%%%%%%%%%%%%%%%%
% \paragraph{Part Include Files.}
%
% The include files are called |cdocspt3.tex| and |cdocspt4.tex|.
%
%\iffalse
%<*samplepart3|samplepart4>
%\fi

% Optional override for |\version| flag:
%    \begin{macrocode}
%%\providecommand{\version}{final}
%    \end{macrocode}

% Include the main document:
%    \begin{macrocode}
\input{childdoc.def}
\childdocby{cdocsamp}
%    \end{macrocode}

%\iffalse
%</samplepart3|samplepart4>
%\fi
%
%\iffalse
%<*samplepart3>
%\fi
% Some text for part 3:
%    \begin{macrocode}
some text in part three
%    \end{macrocode}

%\iffalse
%</samplepart3>
%\fi
% Some text for part 4:
%\iffalse
%<*samplepart4>
%\fi
%    \begin{macrocode}
more text in part four
%    \end{macrocode}

%\iffalse
%</samplepart4>
%\fi
%
% %%%%%%%%%%%%%%%%%%%%%%%%%%%%%%%%%%%%%%
% \paragraph{Forwarding for a Complete Draft.}
%
% The following forwarding file |cdocsdrf.tex|
% compiles the main document in draft mode:
%\iffalse
%<*sampledraft>
%\fi
%    \begin{macrocode}
\def\version{draft}
\input{childdoc.def}
\childdocforward{cdocsamp}
%    \end{macrocode}

%\iffalse
%</sampledraft>
%\fi
%
% %%%%%%%%%%%%%%%%%%%%%%%%%%%%%%%%%%%%%%
% \paragraph{Forwarding for Final Version of the Chapters.}
%
% The following forwarding files |cdocsfn1.tex| and |cdocsfn2.tex|
% (with identical content)
% compile the final versions of the child documents
% |cdocsch1.tex| and |cdocsch2.tex|, respectively:
%\iffalse
%<*samplefinal>
%\fi
%    \begin{macrocode}
\def\version{final}
\input{childdoc.def}
\childdocforwardprefix[cdocsamp]{cdocsfn}{cdocsch}
%    \end{macrocode}

%\iffalse
%</samplefinal>
%\fi
%
% %%%%%%%%%%%%%%%%%%%%%%%%%%%%%%%%%%%%%%
% \paragraph{Command Line Processing.}
%
% The following three command lines generate the output files
% |cdocscld|, |cdocscl1| and |cdocscl2|
% which should be identical to
% |cdocsdrf|, |cdocsch1| and |cdocsfn2|, respectively:
% \begin{center}
% \begin{tabular}{l}
% |latex -jobname cdocscld \|\\
% |  "\def\version{draft}\input{childdoc.def}\childdocforward{cdocsamp}"|\\
% |latex -jobname cdocscl1 \|\\
% |  "\input{childdoc.def}\childdocforward[cdocsamp]{cdocsch1}"|\\
% |latex -jobname cdocscl2 \|\\
% |  "\def\version{final}\input{childdoc.def}\childdocforward{cdocsch2}"|
% \end{tabular}
% \end{center}
% Note that the trailing backslash on each first line
% merely continues the input to the second line
% (for convenient cut ant paste).
% Furthermore, the command |latex| can be replaced by any
% of its alternative versions such as |pdflatex|.
%
% %%%%%%%%%%%%%%%%%%%%%%%%%%%%%%%%%%%%%%%%%%%%%%%%%%%%%%%%%%%%%%%%%%%%%%%%%%%%%%
% %%%%%%%%%%%%%%%%%%%%%%%%%%%%%%%%%%%%%%%%%%%%%%%%%%%%%%%%%%%%%%%%%%%%%%%%%%%%%%
% \section{Implementation}
%\iffalse
%<*package>
%\fi
%
% This section describes the definitions file |childdoc.def|.

% The definitions cannot be loaded using |\usepackage| or |\RequirePackage|
% which has a mechanism to prevent loading a style file more than once.
% When loading the definitions by means of |\input|
% multiple instances have to be prevented manually:
%\iffalse
%This code needs to be before the `\ProvidesFile' directive
%which is defined at the beginning of this file.
%Therefore it is also placed there and commented out here.
%</package>
%<*discard>
%\fi
%    \begin{macrocode}
\ifdefined\childdocmain\endinput\fi
%    \end{macrocode}
%\iffalse
%</discard>
%<*package>
%\fi
%
% \macro{\ifchilddoc}
% \macro{\ifchilddocmanual}
% The conditional |\ifchilddoc| tells whether a
% child (true) or main (false) document is being compiled.
% The conditional |\ifchilddocmanual| tells whether
% the |\includeonly| mechanism is used (false) or
% the selection of child files must be performed manually (true).
% The definitions initialise to false:
%    \begin{macrocode}
\newif\ifchilddoc
\newif\ifchilddocmanual
%    \end{macrocode}

% \macro{\childdocname}
% \macro{\childdocjob}
% The macro |\childdocname| stores the name of the main document
% to be compiled. The macro |\childdocjob| stores the name of
% the document on which the \LaTeX{} compiler was originally invoked.
% The content of |\jobname| cannot be compared
% to filenames specified in the source due to different catcodes.
% The following code rescans |\jobname|, stores the result
% in |\childdocname| and saves a copy in |\childdocjob|:
%    \begin{macrocode}
\edef\childdocname{\scantokens\expandafter{\jobname\noexpand}}
\let\childdocjob\childdocname
%    \end{macrocode}

% \macro{\childdocdisable}
% The macro |\childdocdisable| prevents the main file
% from being processed more than once.
% At this stage, the main document command |\childdocmain|
% is assumed to be called once again where it should do nothing.
% Any subsequent call to it should prevent
% a secondary processing of the main document
% It overwrites the forwarding commands
% |\childdocof| and |\childdocforward|
% with empty macros to prevent further inclusions of the main document:
%    \begin{macrocode}
\newcommand{\childdocdisable}
{
  \renewcommand{\childdocmain}[1]{\renewcommand{\childdocmain}[1]{\endinput}}
  \renewcommand{\childdocof}[1]{}
  \renewcommand{\childdocby}[2][]{}
  \renewcommand{\childdocforward}[2][]{}
  \renewcommand{\childdocdisable}{}
}
%    \end{macrocode}

% \macro{\childdocmain}
% The macro |\childdocmain| is to be called at the top of the main file
% with nothing or the main filename (without extension) as argument.
% First, it breaks loops.
% If the argument is not empty and does not match |\childdocname|
% (which is set by the first inclusion of |childdoc.def|),
% |\ifchilddoc| is set to true, |\includeonly| is applied to the child file
% and |\jobname| is set to the main file
% (for proper handling of |.aux| files):
%    \begin{macrocode}
\newcommand{\childdocmain}[1]
{
  \childdocdisable\childdocmain{}
  \if?#1?\else
    \begingroup
      \def\childdoctmp{#1}
      \ifx\childdoctmp\childdocname
        \def\childdoctmp{}
      \else
        \def\childdoctmp
        {
          \childdoctrue
          \includeonly{\childdocname}
          \def\childdocjob{#1}
          \def\jobname{#1}
        }
      \fi
      \expandafter
    \endgroup
    \childdoctmp
  \fi
}
%    \end{macrocode}

% \macro{\childdocof}
% The command |\childdocof| redirects
% compilation to the main file |#1|.
%    \begin{macrocode}
\newcommand{\childdocof}[1]
{
  \childdocdisable
  \childdoctrue
  \includeonly{\childdocname}
  \def\jobname{#1}
  \def\childdocjob{#1}
  \input{#1}
}
%    \end{macrocode}

% \macro{\childdocby}
% The command |\childdocby| ....
%    \begin{macrocode}
\newcommand{\childdocby}[2][]
{
  \childdocdisable
  \childdoctrue
  \childdocmanualtrue
  \if?#1?\else
    \def\jobname{#2}
  \fi
  \def\childdocjob{#2}
  \input{#2}
  \endinput
}
%    \end{macrocode}

% \macro{\childdocforward}
% The command |\childdocforward| redirects
% compilation to the main file or
% (if the optional argument is given) a child file.
% Parameters are set as if the main file
% or a child file starting with |\childdocof| was compiled.
% Then compilation is handed over to the main file:
%    \begin{macrocode}
\newcommand{\childdocforward}[2][]
{
  \begingroup
    \if?#1?
      \def\childdoctmp
      {
        \def\childdocname{#2}
        \def\childdocjob{#2}
        \def\jobname{#2}
        \input{#2}
        \endinput
      }
    \else
      \def\childdoctmp
      {
        \childdocdisable
        \def\childdocname{#2}
        \childdoctrue
        \includeonly{#2}
        \def\childdocjob{#1}
        \def\jobname{#1}
        \input{#1}
        \endinput
      }
    \fi
    \expandafter
  \endgroup
  \childdoctmp
}
%    \end{macrocode}

% \macro{\childdocforwardprefix}
% The command |\childdocforwardprefix| redirects
% compilation to the main or a child file by means of a pattern.
% The prefix |#1| in the current filename is replaced by |#2|
% and the suffix of the current filename is kept
% (it is assumed that the filename does not contain the substring `|~~~|'
% which is used as a delimiter).
% Compilation is handed over to the new file by |\childdocforward|:
%    \begin{macrocode}
\newcommand{\childdocforwardprefix}[3][]
{
  \begingroup
    \def\childdocextract #2##1~~~{\def\childdoctmp{\childdocforward[#1]{#3##1}}}
    \expandafter\childdocextract\childdocname~~~
    \expandafter
  \endgroup
  \childdoctmp
}
%    \end{macrocode}

% \macro{\childdoc}
% The deprecated macro |\childdoc| is a legacy version of |\childdocmain|:
%    \begin{macrocode}
\newcommand{\childdoc}{\childdocmain}
%    \end{macrocode}

% \macro{\childdocredirect}
% The deprecated macro |\childdocredirect| is a legacy version
% of |\childdocforward| and |\childdocforwardprefix|:
%    \begin{macrocode}
\newcommand{\childdocredirect}[2][]
{
  \begingroup
    \if?#1?
      \def\childdoctmp{\childdocforward{#2}}
    \else
      \def\childdoctmp{\childdocforwardprefix{#1}{#2}}
    \fi
    \expandafter
  \endgroup
  \childdoctmp
}
%    \end{macrocode}

%\iffalse
%</package>
%\fi
%
\endinput
|\\
|\childdocforward{|\textit{main}|}|\\
\end{tabular}
\end{center}
%
or alternatively with:
%
\begin{center}
\begin{tabular}{l}
|% \iffalse
%
% childdoc.dtx Copyright (C) 2017-2018 Niklas Beisert
%
% This work may be distributed and/or modified under the
% conditions of the LaTeX Project Public License, either version 1.3
% of this license or (at your option) any later version.
% The latest version of this license is in
%   http://www.latex-project.org/lppl.txt
% and version 1.3 or later is part of all distributions of LaTeX
% version 2005/12/01 or later.
%
% This work has the LPPL maintenance status `maintained'.
%
% The Current Maintainer of this work is Niklas Beisert.
%
% This work consists of the files childdoc.dtx and childdoc.ins
% and the derived files childdoc.def and cdocsamp.tex with
% cdocsch1.tex, cdocsch2.tex, cdocsdrf.tex, cdocsfn1.tex, cdocsfn2.tex.
%
%<package>\ifdefined\childdocmain\endinput\fi
%<package>\ProvidesFile{childdoc.def}[2018/12/30 v2.0 child document driver]
%<samplemain>\ProvidesFile{cdocsamp.tex}[2018/12/30 v2.0 sample for childdoc]
%<*driver>
%\ProvidesFile{childdoc.drv}[2018/12/30 v2.0 childdoc reference manual file]
\PassOptionsToClass{10pt,a4paper}{article}
\documentclass{ltxdoc}

\usepackage[margin=35mm]{geometry}
\usepackage{hyperref}
\usepackage{hyperxmp}
\usepackage[usenames]{color}

\hypersetup{colorlinks=true}
\hypersetup{pdfstartview=FitH}
\hypersetup{pdfpagemode=UseNone}
\hypersetup{pdfsource={}}
\hypersetup{pdflang={en-UK}}
\hypersetup{pdfcopyright={Copyright 2017-2018 Niklas Beisert.
  This work may be distributed and/or modified under the
  conditions of the LaTeX Project Public License, either version 1.3
  of this license or (at your option) any later version.}}
\hypersetup{pdflicenseurl={http://www.latex-project.org/lppl.txt}}
\hypersetup{pdfcontactaddress={ETH Zurich, ITP, HIT K,
  Wolfgang-Pauli-Strasse 27}}
\hypersetup{pdfcontactpostcode={8093}}
\hypersetup{pdfcontactcity={Zurich}}
\hypersetup{pdfcontactcountry={Switzerland}}
\hypersetup{pdfcontactemail={nbeisert@itp.phys.ethz.ch}}
\hypersetup{pdfcontacturl={http://people.phys.ethz.ch/\xmptilde nbeisert/}}

\newcommand{\secref}[1]{\hyperref[#1]{section \ref*{#1}}}

\parskip1ex
\parindent0pt
\let\olditemize\itemize
\def\itemize{\olditemize\parskip0pt}

\begin{document}

\title{The \textsf{childdoc} Package}
\hypersetup{pdftitle={The childdoc Package}}
\author{Niklas Beisert\\[2ex]
  Institut f\"ur Theoretische Physik\\
  Eidgen\"ossische Technische Hochschule Z\"urich\\
  Wolfgang-Pauli-Strasse 27, 8093 Z\"urich, Switzerland\\[1ex]
  \href{mailto:nbeisert@itp.phys.ethz.ch}
  {\texttt{nbeisert@itp.phys.ethz.ch}}}
\hypersetup{pdfauthor={Niklas Beisert}}
\hypersetup{pdfsubject={Manual for the LaTeX2e Package childdoc}}
\date{30 December 2018, \textsf{v2.0}}
\maketitle

\begin{abstract}\noindent
\textsf{childdoc} is a \LaTeXe{} package
that enables the direct compilation
of document sections included by |\include|
to individual files.
\end{abstract}

\begingroup
\parskip0ex
\tableofcontents
\endgroup

%%%%%%%%%%%%%%%%%%%%%%%%%%%%%%%%%%%%%%%%%%%%%%%%%%%%%%%%%%%%%%%%%%%%%%%%%%%%%%%%
%%%%%%%%%%%%%%%%%%%%%%%%%%%%%%%%%%%%%%%%%%%%%%%%%%%%%%%%%%%%%%%%%%%%%%%%%%%%%%%%
\section{Introduction}

\LaTeX{} provides a mechanism to structure a large document (such as a book)
into a main file and several child files (containing the chapters)
using the |\include| command.
This mechanism is beneficial for documents
which span hundreds of pages in order to
make the source file(s) more manageable.
Moreover, compilation can be restricted to
selected child files by means of the |\includeonly| command.
The latter feature can be used to reduce the compilation time while editing
(this was significantly more useful in the earlier days of \LaTeX{})
or to generate a smaller document which is easier to navigate.
Another application of |\includeonly| is to generate
documents consisting of selected parts of the complete document.

However, there are a few drawbacks of the plain |\include| mechanism:
\begin{itemize}
\item
The child files cannot be compiled on their own,
they can only be compiled via the main file.
A naive editing environment
(such as a text editor with an option
to have the current file processed by \LaTeX)
may require one to switch to the main file before compiling;
attempting to compile the child file produces errors.
\item
The main file must be modified (each time)
to adjust the |\includeonly| command
to the present needs. This easily leaves the main file in a messy state.
\item
The generated document will always carry the filename
of the main document. This is inconvenient if
several child files are to be compiled and
to be kept for distribution.
\end{itemize}

The present package provides a simple interface
to make child files individually compilable by \LaTeX{}.
Compiling a child file then has the same effect as compiling
the main file with an |\includeonly| command
to select the appropriate child.
Moreover the generated document will carry the name of the child
rather than the main file.
This resolves all three above issues.

This feature is meant to make the editing of books,
thesis documents and lecture notes somewhat more convenient.
However, the package can also be used efficiently for
composing a series of documents (such as exercise sheets)
which are typically distributed individually.
It then assists the author in generating the individual documents
(potentially in different versions)
as well as a document containing the collected series.
Another application is in developing style files
or other kinds of included material
where compilation of the style file could redirect
to a sample or test file.

%%%%%%%%%%%%%%%%%%%%%%%%%%%%%%%%%%%%%%%%%%%%%%%%%%%%%%%%%%%%%%%%%%%%%%%%%%%%%%%%
%%%%%%%%%%%%%%%%%%%%%%%%%%%%%%%%%%%%%%%%%%%%%%%%%%%%%%%%%%%%%%%%%%%%%%%%%%%%%%%%
\section{Usage}

First of all, the package \textsf{childdoc} is \emph{not} a standard
\LaTeXe{} |.sty| style file! Therefore it needs to be invoked in
a non-standard way.

%%%%%%%%%%%%%%%%%%%%%%%%%%%%%%%%%%%%%%%%%%%%%%%%%%%%%%%%%%%%%%%%%%%%%%%%%%%%%%%%
\subsection{Included Files}
\label{sec:include}

%%%%%%%%%%%%%%%%%%%%%%%%%%%%%%%%%%%%%%%%
\DescribeMacro{\childdocmain}
To use the package, add the commands
\begin{center}
\begin{tabular}{l}
|\input{childdoc.def}|\\
|\childdocmain{}|\\
\end{tabular}
\end{center}
at the very top of the main \LaTeX{} file,
in particular \emph{before} the |\documentclass| statement!
The argument of |\childdocmain| should be left empty
(but it must be present).

%%%%%%%%%%%%%%%%%%%%%%%%%%%%%%%%%%%%%%%%
\DescribeMacro{\childdocof}
Furthermore, add the commands
\begin{center}
\begin{tabular}{l}
|\input{childdoc.def}|\\
|\childdocof{|\textit{main}|}|\\
\end{tabular}
\end{center}
at the top of every child file \textit{child}
which is included by |\include{|\textit{child}|}|
from within the main file
(or at least for those files to be compiled individually).
The argument \textit{main} must be the filename of the main file.

There are a couple of
considerations in setting up the main and child documents:

%%%%%%%%%%%%%%%%%%%%%%%%%%%%%%%%%%%%%%%%
\paragraph{Restrictions.}

Please note the following restrictions:
\begin{itemize}
\item
|\childdocmain| must be called with one argument \textit{main}
to ensure compatibility with earlier version of the package.
It must either be empty (|\childdocmain{}|)
or precisely match the filename of the main file in which it is specified.
See \secref{sec:detection} for further information.
\item
The filename \textit{main} must be specified without the |.tex| extension.
\item
The filename \textit{main} is case sensitive
(even in case-insensitive file systems)
due to internal string comparison.
\item
The argument \textit{main} should be fully expanded, it cannot be a macro.
\item
Subdirectories and special characters should be avoided in filenames.
\item
The command |\childdocmain{|\textit{main}|}| must be followed by a whitespace.
It should not be followed immediately by another command
or by a comment mark `|%|'.
This is because the \TeX{} parser reads the token immediately following
the argument of |\childdocmain| and puts it
at the beginning of every child section;
however, a white\-space is ignored.
\end{itemize}

%%%%%%%%%%%%%%%%%%%%%%%%%%%%%%%%%%%%%%%%
\paragraph{Content of Main File.}

It is advisable to place all content in the child files included by |\include|.
Any output contained in the main file will appear in all child documents
unless suppressed manually;
it cannot be suppressed automatically by the |\includeonly| directive
and thus should normally be avoided.
A method to include some content in the main file
by means of conditional processing is described in \secref{sec:conditional}.

%%%%%%%%%%%%%%%%%%%%%%%%%%%%%%%%%%%%%%%%
\paragraph{Page Numbering.}

When only a part of the document is compiled,
the appropriate numbering of pages
(as well as other status parameters)
is determined from the |.aux| files.
The latter contain information from previous passes.
However this information needs to propagate through
all intermediate child documents.
Therefore the page numbering in child documents may well
be inconsistent until the complete document is compiled at least once.

A useful (if unconventional) way to always ensure a consistent
page numbering is to restart the numbering in each child document
and denote the pages by `\textit{child}|.|\textit{page}'
where \textit{child} represents the chapter/section number of the child file.
This can be achieved by the command
|\numberwithin{page}{|\textit{child}|}|
of the \textsf{amsmath} package
where \textit{child} can be |chapter| or |section|
depending on the chosen structuring.
Alternatively, one can modify the macro |\thepage| appropriately
and reset the counter |page| at the start of each child file.

%%%%%%%%%%%%%%%%%%%%%%%%%%%%%%%%%%%%%%%%%%%%%%%%%%%%%%%%%%%%%%%%%%%%%%%%%%%%%%%%
\subsection{Conditional Processing}
\label{sec:conditional}

The package provides a mechanism to compile different versions
of a document. To customise the versions further some conditional processing
can come in handy to distinguish which version is being compiled.
The package provides two macros to describe the compilation context:

%%%%%%%%%%%%%%%%%%%%%%%%%%%%%%%%%%%%%%%%
\DescribeMacro{\ifchilddoc}
The conditional |\ifchilddoc| distinguishes between the compilation of
child documents and the main document:
%
\begin{center}
|\ifchilddoc |\textit{child-code}| |[|\||else |\textit{main-code}]| \||fi|
\end{center}

%%%%%%%%%%%%%%%%%%%%%%%%%%%%%%%%%%%%%%%%
\DescribeMacro{\childdocname}
\DescribeMacro{\childdocjob}
The macro |\childdocname| contains the filename (without extension)
of the main or child file being processed.
Note that |\childdocjob| will always contain the name of the main file.

%%%%%%%%%%%%%%%%%%%%%%%%%%%%%%%%%%%%%%%%
\paragraph{Title Page.}

Conditional processing can be used to include a title or banner page
in the main document when proper precautions are taken.
Importantly, the code in the main file should ensure that the page counter
(as well as other status parameters which are stored in the |.aux| files)
takes the same value after the conditional processing.
Otherwise the page numbers may take divergent values
depending on which part is compiled.

For example, a title page could be declared by:
%
\begin{center}
\begin{tabular}{l}
|\ifchilddoc\||else|\\
|\addtocounter{page}{-1}|\\
\textit{code for title page}\\
|\newpage|\\
|\||fi|
\end{tabular}
\end{center}
%
A banner page for the child documents can be generated by:
%
\begin{center}
\begin{tabular}{l}
|\ifchilddoc|\\
|\addtocounter{page}{-1}|\\
\textit{code for banner page}\\
|\newpage|\\
|\||fi|
\end{tabular}
\end{center}
%
Here one could write a message such as:
\begin{center}
|This is the part \childdocname{} of \childdocjob{}.|
\end{center}

%%%%%%%%%%%%%%%%%%%%%%%%%%%%%%%%%%%%%%%%%%%%%%%%%%%%%%%%%%%%%%%%%%%%%%%%%%%%%%%%
\subsection{Flags}
\label{sec:flags}

The package makes it easy to generate different versions
of the main or child documents.
To this end compilation flags can be defined
and assigned different default values.
They will be particularly useful in conjunction
with the forwarding mechanism described in \secref{sec:forward}.

For example, it may be useful to have a flag |\version|
which can be set to |draft| or |final|.
The document source will contain some conditional code
depending on the value of |\version|.
Suppose further, the flag should default to |final| for the main file
and to |draft| for child files
which is a natural assignment for editing the document.
This is achieved by placing the following code
in the preamble of the main document
(below the |\childdocmain| directive):
%
\begin{center}
\begin{tabular}{l}
|\ifchilddoc|\\
|\providecommand{\version}{draft}|\\
|\||else|\\
|\providecommand{\version}{final}|\\
|\||fi|
\end{tabular}
\end{center}
%
The definition by |\providecommand| makes sure
that previous definitions are not overwritten.
Further statements |\providecommand{\version}{...}|
can thus be added before the above code to override it.

For the main file, one might add a line
(between |\childdocmain| and the above block)
%
\begin{center}
|%\ifchilddoc\||else\providecommand{\version}{draft}\||fi|
\end{center}
%
which can be uncommented to produce a draft version.
Likewise one can add a line to the very top of a child file
(above the |\childdocof{|\textit{main}|}| directive)
%
\begin{center}
|%\providecommand{\version}{final}|
\end{center}
%
which can be uncommented to produce the final version of this child document.

%%%%%%%%%%%%%%%%%%%%%%%%%%%%%%%%%%%%%%%%%%%%%%%%%%%%%%%%%%%%%%%%%%%%%%%%%%%%%%%%
\subsection{Forwarding}
\label{sec:forward}

Different versions of the main or child documents
using compilation flags as described in \secref{sec:flags}
can be (permanently) stored in different files
for convenient compilation, viewing and distribution.
To this end, the package defines a command
to pass on compilation to a different file:

%%%%%%%%%%%%%%%%%%%%%%%%%%%%%%%%%%%%%%%%
\DescribeMacro{\childdocforward}
The command |\childdocforward| redirects processing to
another source file:
%
\begin{center}
\begin{tabular}{l}
|\input{childdoc.def}|\\
|\childdocforward[|\textit{main}|]{|\textit{dest}|}|\\
\end{tabular}
\end{center}
%
The argument \textit{dest} is the destination file
(without extension).
It should be the main file or one of the child files.
Note that further \textsf{childdoc} directives
such as |\childdocof| and |\childdocforward|
in the indicated file will be processed in this form.
The optional argument \textit{main}
passes on directly to the main file \textit{main}
while pretending to compile the child \textit{dest}.
This form behaves as if \textit{dest}
issues |\childdocof{|\textit{main}|}| right away,
and no further \textsf{childdoc} directives will be processed.

%%%%%%%%%%%%%%%%%%%%%%%%%%%%%%%%%%%%%%%%
\DescribeMacro{\...prefix}
In the alternative form |\childdocforwardprefix|,
%
\begin{center}
\begin{tabular}{l}
|\input{childdoc.def}|\\
|\childdocforwardprefix[|\textit{main}|]{|\textit{prefix}|}{|\textit{dest}|}|
\end{tabular}
\end{center}
%
the destination file is determined by a pattern
depending on the current file:
To make this work, the current file must be called
`{\textit{prefix}\hspace{0.2em}\textit{suffix}}'
with \textit{prefix} matching precisely the argument.
Processing is then passed on to the file
`{\textit{dest}\hspace{0.2em}\textit{suffix}}'.
Surely, the same effect is achieved by
directly specifying the
argument `{\textit{dest}\hspace{0.2em}\textit{suffix}}'
in the first form.
However, that requires to set up a different file
for each child. With the alternative form of the command
all these files can have exactly the same content
which simplifies setting them up and maintaining them.

For example, the following file |draft.tex|
with a compilation flag |\version| as described in \secref{sec:flags}
compiles the main document as a draft:
%
\begin{center}
\begin{tabular}{l}
|\def\version{draft}|\\
|\input{childdoc.def}|\\
|\childdocforward{|\textit{main}|}|
\end{tabular}
\end{center}
%
Likewise, the following files |final|\textit{nn}|.tex|
compile the final version of the child document
|child|\textit{nn}|.tex|:
%
\begin{center}
\begin{tabular}{l}
|\def\version{final}|\\
|\input{childdoc.def}|\\
|\childdocforwardprefix{final}{child}|
\end{tabular}
\end{center}
%

Note that when several versions of a main file and/or of each child file
are to be generated, it may be convenient to set up a |Makefile| or
shell script to automatise the process.

%%%%%%%%%%%%%%%%%%%%%%%%%%%%%%%%%%%%%%%%%%%%%%%%%%%%%%%%%%%%%%%%%%%%%%%%%%%%%%%%
\subsection{Command Line Processing}
\label{sec:commandline}

The effect of redirection files can also be achieved by invoking
the \LaTeX{} compiler with a more elaborate command line.
Most conveniently this should be done as part
of a shell script or a |Makefile|.

When using \textsf{childdoc} in the main file, the following
command lines effectively perform a redirection
(note that depending on the shell being used,
backslashes may have to be doubled: `|\|' $\to$ `|\\|'):
%
\begin{center}
|... -jobname "|\textit{target}|" |\\|"|[\textit{flags}]%
|\input{childdoc.def}\childdocforward[|\textit{main}|]{|\textit{dest}|}"|
\end{center}
%
Here \textit{target} is the name of the output file,
\textit{main} is the name of the main file
and \textit{dest} is the name of the main or child file to be processed
(all filenames without extensions).
The optional argument \textit{main} can be omitted
if \textit{main} matches \textit{dest}.
Optionally, compilation \textit{flags} can be defined via |\def| commands.
This command line makes the \TeX{} engine believe
it is compiling the file \textit{target}
whose content is specified as the latter parameter.
The provided code then forwards the processing to
\textit{main} or \textit{dest} as described in \secref{sec:forward}.

%%%%%%%%%%%%%%%%%%%%%%%%%%%%%%%%%%%%%%%%%%%%%%%%%%%%%%%%%%%%%%%%%%%%%%%%%%%%%%%%
\subsection{Include by Input}
\label{sec:input}

Including child documents by |\include| has some restrictions by design.
Most notably, the content of a child document always occupies
its own set of pages; pages cannot be shared between child documents.
Usually, this behaviour makes perfect sense
because each child document contain an essential part of the document.
However, in some situations it may be desirable to compose
a document from a collection of parts
without having mandatory page breaks between then.
For this case, the package
provides a mechanism to include parts
by |\input| which can also be processed individually.
However, by construction this mechanism
requires manual handling of the content to be output.

%%%%%%%%%%%%%%%%%%%%%%%%%%%%%%%%%%%%%%%%
\DescribeMacro{\ifchilddocmanual}
The main file should be prepared as usual, see \secref{sec:include}.
However, the document body must make a distinction
between processing of an individual part and of the main document, e.g.:
%
\begin{center}
\begin{tabular}{l}
|\ifchilddocmanual|\\
|\input{\childdocname}|\\
|\||else|\\
\textit{document body with }|\input{|\textit{part}|}|\\
|\||fi|
\end{tabular}
\end{center}
%
The conditional |\ifchilddocmanual| is true whenever
a part to be included by |\input| is being compiled,
and the name of the part is stored in |\childdocname|.

%%%%%%%%%%%%%%%%%%%%%%%%%%%%%%%%%%%%%%%%
\DescribeMacro{\childdocby}
Each part to be included by |\input| should start with:
%
\begin{center}
\begin{tabular}{l}
|\input{childdoc.def}|\\
|\childdocby{|\textit{main}|}|\\
\end{tabular}
\end{center}
%
The directive |\childdocby| is similar to |\childdocof|
described in \secref{sec:include},
but the subsequent selection of content must be done manually.
To that end, both |\ifchilddoc| and |\ifchilddocmanual|
will be true upon processing of a part,
and the name of the part is stored in |\childdocname|.
Note that |\jobname| will be set to the filename of the current part
so that each part receives an individual |.aux| file
that does not interfere with the |.aux| file(s) of the main document.
This behaviour can be altered by the alternative form
|\childdocby[*]{|\textit{main}|}| (with a non-empty optional argument)
which uses the |.aux| file of the main document
by setting |\jobname| to \textit{main}.

%%%%%%%%%%%%%%%%%%%%%%%%%%%%%%%%%%%%%%%%%%%%%%%%%%%%%%%%%%%%%%%%%%%%%%%%%%%%%%%%
\subsection{Driver Development}
\label{sec:driver}

The \textsf{childdoc} mechanism can also be use for the development
of definition files such as \LaTeX{} styles or classes.
This case differs from the above setup with multiple parts
included by |\include| in that no |\includeonly| should be invoked.
This can be achieved by starting the include file
(before |\ProvidesPackage|) with:
%
\begin{center}
\begin{tabular}{l}
|\input{childdoc.def}|\\
|\childdocforward{|\textit{main}|}|\\
\end{tabular}
\end{center}
%
or alternatively with:
%
\begin{center}
\begin{tabular}{l}
|\input{childdoc.def}|\\
|\childdocby{|\textit{main}|}|\\
\end{tabular}
\end{center}
%
Both forms have slightly different effects as described above.
The main file is prepared as usual, see \secref{sec:include}.

%%%%%%%%%%%%%%%%%%%%%%%%%%%%%%%%%%%%%%%%%%%%%%%%%%%%%%%%%%%%%%%%%%%%%%%%%%%%%%%%
\subsection{Legacy Detection}
\label{sec:detection}

The directive |\childdocmain| in the main file can detect
whether the complete document or merely a child is to be compiled
even without using the directive |\childdocof|.
This method is deprecated because it is less robust
and there is no compelling reason to use it;
it is merely provided for backward compatibility
and it may be removed in future versions.

If the detection mechanism is to be used,
it is mandatory to correctly specify
the filename of the main file as the argument of |\childdocmain|:
%
\begin{center}
\begin{tabular}{l}
|\input{childdoc.def}|\\
|\childdocmain{|\textit{main}|}|\\
\end{tabular}
\end{center}
%
If |\jobname| does not match the argument \textit{main} of |\childdocmain|,
it is assumed that |\jobname| points to the child file to be compiled.
When using |\childdocmain| with the main file specified as argument,
it suffices to start a child file
with just |\input{|\textit{main}|}|
without loading of the package and using |\childdocof|.
If instead all processing is done
with the appropriate \textsf{childdoc} directives,
the argument of \textit{main} of |\childdocmain| can be empty.

An alternative version of the command line processing described
in \secref{sec:commandline} using the detection mechanism reads:
%
\begin{center}
|... -jobname "|\textit{target}|" "|[\textit{flags}]%
[|\def\jobname{|\textit{dest}|}|]|\input{|\textit{main}|}"|
\end{center}

%%%%%%%%%%%%%%%%%%%%%%%%%%%%%%%%%%%%%%%%%%%%%%%%%%%%%%%%%%%%%%%%%%%%%%%%%%%%%%%%
\subsection{Manual Code}
\label{sec:manual}

In case one cannot be certain whether the definitions file |childdoc.def|
is installed on the target \TeX{} distribution
and one prefers not to ship it,
it is conceivable to paste a few relevant commands into the sources.

To that end, drop all statements |\input{childdoc.def}|
and perform the replacements as outlined below.
Instead of |\childdocmain{|\textit{main}|}| add the following code
to the top of the main file:
%
\begin{center}
\begin{tabular}{l}
|\||ifdefined\childdocname\endinput\||fi\newif\ifchilddoc|\\
|\edef\childdocname{\scantokens\expandafter{\jobname\noexpand}}|\\
|\def\childdocmain{|\textit{main}|}\||ifx\childdocmain\childdocname\||else|\\
|\childdoctrue\includeonly{\childdocname}\let\jobname\childdocmain\||fi|\\
\end{tabular}
\end{center}
%
Instead of |\childdocof{|\textit{main}|}| just include the main file
at the top of each child file:
%
\begin{center}
|\input{|\textit{main}|}|
\end{center}
%
A simple redirection |\childdocforward{|\textit{dest}|}| is achieved by:
%
\begin{center}
|\def\jobname{|\textit{dest}|}\input{\jobname}|
\end{center}
%
The redirection with prefix
|\childdocforwardprefix[|\textit{prefix}|]{|\textit{dest}|}|
is accomplished by:
%
\begin{center}
\begin{tabular}{l}
|{\edef\jobname{\scantokens\expandafter{\jobname\noexpand}}|\\
|\def\redirectjob |\textit{prefix}|#1~~~{\gdef\jobname{|\textit{dest}|#1}}|\\
|\expandafter\redirectjob\jobname~~~}\input{\jobname}|
\end{tabular}
\end{center}

In an alternative approach,
child documents can be compiled by a specific command line
without additional code or specific definitions:
%
\begin{center}
|... -jobname "|\textit{target}|" "|[\textit{flags}]%
|\includeonly{|\textit{dest}|}\input{|\textit{main}|}"|
\end{center}
%

%%%%%%%%%%%%%%%%%%%%%%%%%%%%%%%%%%%%%%%%%%%%%%%%%%%%%%%%%%%%%%%%%%%%%%%%%%%%%%%%
%%%%%%%%%%%%%%%%%%%%%%%%%%%%%%%%%%%%%%%%%%%%%%%%%%%%%%%%%%%%%%%%%%%%%%%%%%%%%%%%
\section{Information}

%%%%%%%%%%%%%%%%%%%%%%%%%%%%%%%%%%%%%%%%%%%%%%%%%%%%%%%%%%%%%%%%%%%%%%%%%%%%%%%%
\subsection{Copyright}

Copyright \copyright{} 2017--2018 Niklas Beisert

This work may be distributed and/or modified under the
conditions of the \LaTeX{} Project Public License, either version 1.3
of this license or (at your option) any later version.
The latest version of this license is in
  \url{http://www.latex-project.org/lppl.txt}
and version 1.3 or later is part of all distributions of \LaTeX{}
version 2005/12/01 or later.

This work has the LPPL maintenance status `maintained'.

The Current Maintainer of this work is Niklas Beisert.

This work consists of the files |README.txt|, |childdoc.ins| and |childdoc.dtx|
as well as the derived files |childdoc.def|, |cdocsamp.tex|
with |cdocsch1.tex|, |cdocsch2.tex|, |cdocspt3.tex|, |cdocspt4.tex|,
|cdocsdrf.tex|, |cdocsfn1.tex|, |cdocsfn2.tex|
as well as |childdoc.pdf|.

%%%%%%%%%%%%%%%%%%%%%%%%%%%%%%%%%%%%%%%%%%%%%%%%%%%%%%%%%%%%%%%%%%%%%%%%%%%%%%%%
\subsection{Files and Installation}

The package consists of the files:
%
\begin{center}
\begin{tabular}{ll}
    |README.txt|   & readme file \\
    |childdoc.ins| & installation file \\
    |childdoc.dtx| & source file \\
    |childdoc.def| & definition file \\
    |cdocsamp.tex| & sample main file \\
    |cdocsch1.tex| & sample include file \\
    |cdocsch2.tex| & sample include file \\
    |cdocspt3.tex| & sample part file \\
    |cdocspt4.tex| & sample part file \\
    |cdocsdrf.tex| & sample redirection file \\
    |cdocsfn1.tex| & sample redirection file \\
    |cdocsfn2.tex| & sample redirection file \\
    |childdoc.pdf| & manual
\end{tabular}
\end{center}
%
The distribution consists of the files
|README.txt|, |childdoc.ins| and |childdoc.dtx|.
%
\begin{itemize}
\item
Run (pdf)\LaTeX{} on |childdoc.dtx|
to compile the manual |childdoc.pdf| (this file).
\item
Run \LaTeX{} on |childdoc.ins| to create the definitions file |childdoc.def|
and the sample |cdocsamp.tex| with include files
|cdocsch1.tex|, |cdocsch2.tex|, |cdocspt3.tex|, |cdocspt4.tex|,
|cdocsdrf.tex|, |cdocsfn1.tex|, |cdocsfn2.tex|.
Then copy the file |childdoc.def| to an appropriate directory of your \LaTeX{}
distribution, e.g.\ \textit{texmf-root}|/tex/latex/childdoc|.
\end{itemize}

%%%%%%%%%%%%%%%%%%%%%%%%%%%%%%%%%%%%%%%%%%%%%%%%%%%%%%%%%%%%%%%%%%%%%%%%%%%%%%%%
\subsection{Related CTAN Packages}

There are several other packages which offer a similar functionality:
%
\begin{itemize}
\item
The packages
\href{http://ctan.org/pkg/docmute}{\textsf{docmute}},
\href{http://ctan.org/pkg/includex}{\textsf{includex}} and
\href{http://ctan.org/pkg/standalone}{\textsf{standalone}}
provide commands to include only the document body of
a child file thus allowing both files to be compiled individually.
\item
The packages \href{http://ctan.org/pkg/subdocs}{\textsf{subdocs}}
and \href{http://ctan.org/pkg/subfiles}{\textsf{subfiles}}
provide structures in which the main and child documents can be
encapsulated and allowing them to be compiled individually.
The inclusion mechanism is different from the conventional |\include|.
\item
The package \href{http://ctan.org/pkg/combine}{\textsf{combine}}
is an elaborate solution to combine several documents into one.
\end{itemize}
%
See also the CTAN topic \href{http://ctan.org/topic/subdocs}{\textsf{subdocs}}
for further related packages.
The present package differs from the above solutions in that
a document structure constructed with the conventional |\include| mechanism
just needs two extra commands at the top of every file
such that all constituent files can be compiled individually.

%%%%%%%%%%%%%%%%%%%%%%%%%%%%%%%%%%%%%%%%%%%%%%%%%%%%%%%%%%%%%%%%%%%%%%%%%%%%%%%%
%\subsection{Feature Suggestions}
%
%The following is a list of features which may be useful for future
%versions of this package:
%%
%\begin{itemize}
%\item
%\ldots
%\end{itemize}

%%%%%%%%%%%%%%%%%%%%%%%%%%%%%%%%%%%%%%%%%%%%%%%%%%%%%%%%%%%%%%%%%%%%%%%%%%%%%%%%
\subsection{Revision History}

%%%%%%%%%%%%%%%%%%%%%%%%%%%%%%%%%%%%%%%%
\paragraph{v2.0:} 2018/12/30

\begin{itemize}
\item
immediate forward processing
\item
added |\childdocby| mechanism
\item
manual restructured
\end{itemize}

%%%%%%%%%%%%%%%%%%%%%%%%%%%%%%%%%%%%%%%%
\paragraph{v1.6:} 2018/01/17

\begin{itemize}
\item
application for development of include files
\item
corrections to manual
\end{itemize}

%%%%%%%%%%%%%%%%%%%%%%%%%%%%%%%%%%%%%%%%
\paragraph{v1.5:} 2017/05/21

\begin{itemize}
\item
more complete structuring introduced
\item
|\childdocof| introduced
\item
|\childdoc| renamed to |\childdocmain|
\item
|\childredirect| renamed to |\childdocforward| and |\childdocforwardprefix|
and functionality expanded
\end{itemize}

%%%%%%%%%%%%%%%%%%%%%%%%%%%%%%%%%%%%%%%%
\paragraph{v1.0:} 2017/04/27

\begin{itemize}
\item
manual and install package
\item
first version published on CTAN
\end{itemize}

%%%%%%%%%%%%%%%%%%%%%%%%%%%%%%%%%%%%%%%%
\paragraph{v0.6:} 2017/04/26

\begin{itemize}
\item
redirection mechanism added
\end{itemize}

%%%%%%%%%%%%%%%%%%%%%%%%%%%%%%%%%%%%%%%%
\paragraph{v0.5:} 2017/04/26

\begin{itemize}
\item
functionality in definition file
\end{itemize}


%%%%%%%%%%%%%%%%%%%%%%%%%%%%%%%%%%%%%%%%%%%%%%%%%%%%%%%%%%%%%%%%%%%%%%%%%%%%%%%%
%%%%%%%%%%%%%%%%%%%%%%%%%%%%%%%%%%%%%%%%%%%%%%%%%%%%%%%%%%%%%%%%%%%%%%%%%%%%%%%%
%%%%%%%%%%%%%%%%%%%%%%%%%%%%%%%%%%%%%%%%%%%%%%%%%%%%%%%%%%%%%%%%%%%%%%%%%%%%%%%%
\appendix

\settowidth\MacroIndent{\rmfamily\scriptsize 000\ }

 \DocInput{childdoc.dtx}

\end{document}
%</driver>
% \fi
%
% %%%%%%%%%%%%%%%%%%%%%%%%%%%%%%%%%%%%%%%%%%%%%%%%%%%%%%%%%%%%%%%%%%%%%%%%%%%%%%
% %%%%%%%%%%%%%%%%%%%%%%%%%%%%%%%%%%%%%%%%%%%%%%%%%%%%%%%%%%%%%%%%%%%%%%%%%%%%%%
% \section{Sample}
%\iffalse
%<*samplemain>
%\fi
%
% The following presents a sample document
% with two chapters, two parts, a title page,
% a compile flag as well as three forwarding files to set the flag.
% It consists of eight |.tex| files:
% \begin{center}
% \begin{tabular}{ll}
% |cdocsamp.tex|&main file\\
% |cdocsch1.tex|&include file for chapter 1\\
% |cdocsch2.tex|&include file for chapter 2\\
% |cdocspt3.tex|&include file for part 3\\
% |cdocspt4.tex|&include file for part 4\\
% |cdocsdrf.tex|&forwarding file for main file in draft mode\\
% |cdocsfi1.tex|&forwarding file for final version of chapter 1\\
% |cdocsfi2.tex|&forwarding file for final version of chapter 2\\
% \end{tabular}
% \end{center}
% Each of the eight files can be compiled directly by the \LaTeX{} compiler.
%
% %%%%%%%%%%%%%%%%%%%%%%%%%%%%%%%%%%%%%%
% \paragraph{Main File.}
%
% The main file is called |cdocsamp.tex|.
%
% Load the \textsf{childdoc} definitions and
% declare the filename for the main document:
%    \begin{macrocode}
\input{childdoc.def}
\childdocmain{}
%    \end{macrocode}

% Optional override for |\version| flag:
%    \begin{macrocode}
%%\ifchilddoc\else\providecommand{\version}{draft}\fi
%    \end{macrocode}

% Define the default values for the |\version| flag
% (|final| for the main file and |draft| for childs):
%    \begin{macrocode}
\ifchilddoc
\providecommand{\version}{draft}
\else
\providecommand{\version}{final}
\fi
%    \end{macrocode}

% Load the standard document class:
%    \begin{macrocode}
\documentclass[12pt]{article}
%    \end{macrocode}

% Start the document body:
%    \begin{macrocode}
\begin{document}
%    \end{macrocode}

% Declare a title page.
% Print title, part of document being processed and version flag:
%    \begin{macrocode}
\addtocounter{page}{-1}
\begin{center}
{\LARGE\bfseries{}childdoc example\par}
\vspace{1cm}
\ifchilddoc
\ifchilddocmanual part\else chapter\fi:
`\childdocname' of `\childdocjob'\par
\else
main document: `\childdocjob'\par
\fi
version: \version\par
\end{center}
\newpage
%    \end{macrocode}

% Manually include selected file,
% otherwise process as usual:
%    \begin{macrocode}
\ifchilddocmanual
\section*{part `\childdocname'}
\input{\childdocname}
\else
%    \end{macrocode}

% Include the two chapters:
%    \begin{macrocode}
\include{cdocsch1}
\include{cdocsch2}
%    \end{macrocode}

% Include the two parts unless only chapters should be displayed:
%    \begin{macrocode}
\ifchilddoc\else
\section{part three}
\input{cdocspt3}
\section{part four}
\input{cdocspt4}
\fi
%    \end{macrocode}

% Process as usual until here:
%    \begin{macrocode}
\fi
%    \end{macrocode}

% End of document body:
%    \begin{macrocode}
\end{document}
%    \end{macrocode}
%\iffalse
%</samplemain>
%\fi
%
% %%%%%%%%%%%%%%%%%%%%%%%%%%%%%%%%%%%%%%
% \paragraph{Chapter Include Files.}
%
% The include files are called |cdocsch1.tex| and |cdocsch2.tex|.
%
%\iffalse
%<*samplechap1|samplechap2>
%\fi

% Optional override for |\version| flag:
%    \begin{macrocode}
%%\providecommand{\version}{final}
%    \end{macrocode}

% Include the main document:
%    \begin{macrocode}
\input{childdoc.def}
\childdocof{cdocsamp}
%    \end{macrocode}

%\iffalse
%</samplechap1|samplechap2>
%\fi
%
%\iffalse
%<*samplechap1>
%\fi
% Some text for chapter 1:
%    \begin{macrocode}
\section{one}
some text in chapter one
%    \end{macrocode}

%\iffalse
%</samplechap1>
%\fi
% Some text for chapter 2:
%\iffalse
%<*samplechap2>
%\fi
%    \begin{macrocode}
\section{two}
more text in chapter two
%    \end{macrocode}

%\iffalse
%</samplechap2>
%\fi
%
% %%%%%%%%%%%%%%%%%%%%%%%%%%%%%%%%%%%%%%
% \paragraph{Part Include Files.}
%
% The include files are called |cdocspt3.tex| and |cdocspt4.tex|.
%
%\iffalse
%<*samplepart3|samplepart4>
%\fi

% Optional override for |\version| flag:
%    \begin{macrocode}
%%\providecommand{\version}{final}
%    \end{macrocode}

% Include the main document:
%    \begin{macrocode}
\input{childdoc.def}
\childdocby{cdocsamp}
%    \end{macrocode}

%\iffalse
%</samplepart3|samplepart4>
%\fi
%
%\iffalse
%<*samplepart3>
%\fi
% Some text for part 3:
%    \begin{macrocode}
some text in part three
%    \end{macrocode}

%\iffalse
%</samplepart3>
%\fi
% Some text for part 4:
%\iffalse
%<*samplepart4>
%\fi
%    \begin{macrocode}
more text in part four
%    \end{macrocode}

%\iffalse
%</samplepart4>
%\fi
%
% %%%%%%%%%%%%%%%%%%%%%%%%%%%%%%%%%%%%%%
% \paragraph{Forwarding for a Complete Draft.}
%
% The following forwarding file |cdocsdrf.tex|
% compiles the main document in draft mode:
%\iffalse
%<*sampledraft>
%\fi
%    \begin{macrocode}
\def\version{draft}
\input{childdoc.def}
\childdocforward{cdocsamp}
%    \end{macrocode}

%\iffalse
%</sampledraft>
%\fi
%
% %%%%%%%%%%%%%%%%%%%%%%%%%%%%%%%%%%%%%%
% \paragraph{Forwarding for Final Version of the Chapters.}
%
% The following forwarding files |cdocsfn1.tex| and |cdocsfn2.tex|
% (with identical content)
% compile the final versions of the child documents
% |cdocsch1.tex| and |cdocsch2.tex|, respectively:
%\iffalse
%<*samplefinal>
%\fi
%    \begin{macrocode}
\def\version{final}
\input{childdoc.def}
\childdocforwardprefix[cdocsamp]{cdocsfn}{cdocsch}
%    \end{macrocode}

%\iffalse
%</samplefinal>
%\fi
%
% %%%%%%%%%%%%%%%%%%%%%%%%%%%%%%%%%%%%%%
% \paragraph{Command Line Processing.}
%
% The following three command lines generate the output files
% |cdocscld|, |cdocscl1| and |cdocscl2|
% which should be identical to
% |cdocsdrf|, |cdocsch1| and |cdocsfn2|, respectively:
% \begin{center}
% \begin{tabular}{l}
% |latex -jobname cdocscld \|\\
% |  "\def\version{draft}\input{childdoc.def}\childdocforward{cdocsamp}"|\\
% |latex -jobname cdocscl1 \|\\
% |  "\input{childdoc.def}\childdocforward[cdocsamp]{cdocsch1}"|\\
% |latex -jobname cdocscl2 \|\\
% |  "\def\version{final}\input{childdoc.def}\childdocforward{cdocsch2}"|
% \end{tabular}
% \end{center}
% Note that the trailing backslash on each first line
% merely continues the input to the second line
% (for convenient cut ant paste).
% Furthermore, the command |latex| can be replaced by any
% of its alternative versions such as |pdflatex|.
%
% %%%%%%%%%%%%%%%%%%%%%%%%%%%%%%%%%%%%%%%%%%%%%%%%%%%%%%%%%%%%%%%%%%%%%%%%%%%%%%
% %%%%%%%%%%%%%%%%%%%%%%%%%%%%%%%%%%%%%%%%%%%%%%%%%%%%%%%%%%%%%%%%%%%%%%%%%%%%%%
% \section{Implementation}
%\iffalse
%<*package>
%\fi
%
% This section describes the definitions file |childdoc.def|.

% The definitions cannot be loaded using |\usepackage| or |\RequirePackage|
% which has a mechanism to prevent loading a style file more than once.
% When loading the definitions by means of |\input|
% multiple instances have to be prevented manually:
%\iffalse
%This code needs to be before the `\ProvidesFile' directive
%which is defined at the beginning of this file.
%Therefore it is also placed there and commented out here.
%</package>
%<*discard>
%\fi
%    \begin{macrocode}
\ifdefined\childdocmain\endinput\fi
%    \end{macrocode}
%\iffalse
%</discard>
%<*package>
%\fi
%
% \macro{\ifchilddoc}
% \macro{\ifchilddocmanual}
% The conditional |\ifchilddoc| tells whether a
% child (true) or main (false) document is being compiled.
% The conditional |\ifchilddocmanual| tells whether
% the |\includeonly| mechanism is used (false) or
% the selection of child files must be performed manually (true).
% The definitions initialise to false:
%    \begin{macrocode}
\newif\ifchilddoc
\newif\ifchilddocmanual
%    \end{macrocode}

% \macro{\childdocname}
% \macro{\childdocjob}
% The macro |\childdocname| stores the name of the main document
% to be compiled. The macro |\childdocjob| stores the name of
% the document on which the \LaTeX{} compiler was originally invoked.
% The content of |\jobname| cannot be compared
% to filenames specified in the source due to different catcodes.
% The following code rescans |\jobname|, stores the result
% in |\childdocname| and saves a copy in |\childdocjob|:
%    \begin{macrocode}
\edef\childdocname{\scantokens\expandafter{\jobname\noexpand}}
\let\childdocjob\childdocname
%    \end{macrocode}

% \macro{\childdocdisable}
% The macro |\childdocdisable| prevents the main file
% from being processed more than once.
% At this stage, the main document command |\childdocmain|
% is assumed to be called once again where it should do nothing.
% Any subsequent call to it should prevent
% a secondary processing of the main document
% It overwrites the forwarding commands
% |\childdocof| and |\childdocforward|
% with empty macros to prevent further inclusions of the main document:
%    \begin{macrocode}
\newcommand{\childdocdisable}
{
  \renewcommand{\childdocmain}[1]{\renewcommand{\childdocmain}[1]{\endinput}}
  \renewcommand{\childdocof}[1]{}
  \renewcommand{\childdocby}[2][]{}
  \renewcommand{\childdocforward}[2][]{}
  \renewcommand{\childdocdisable}{}
}
%    \end{macrocode}

% \macro{\childdocmain}
% The macro |\childdocmain| is to be called at the top of the main file
% with nothing or the main filename (without extension) as argument.
% First, it breaks loops.
% If the argument is not empty and does not match |\childdocname|
% (which is set by the first inclusion of |childdoc.def|),
% |\ifchilddoc| is set to true, |\includeonly| is applied to the child file
% and |\jobname| is set to the main file
% (for proper handling of |.aux| files):
%    \begin{macrocode}
\newcommand{\childdocmain}[1]
{
  \childdocdisable\childdocmain{}
  \if?#1?\else
    \begingroup
      \def\childdoctmp{#1}
      \ifx\childdoctmp\childdocname
        \def\childdoctmp{}
      \else
        \def\childdoctmp
        {
          \childdoctrue
          \includeonly{\childdocname}
          \def\childdocjob{#1}
          \def\jobname{#1}
        }
      \fi
      \expandafter
    \endgroup
    \childdoctmp
  \fi
}
%    \end{macrocode}

% \macro{\childdocof}
% The command |\childdocof| redirects
% compilation to the main file |#1|.
%    \begin{macrocode}
\newcommand{\childdocof}[1]
{
  \childdocdisable
  \childdoctrue
  \includeonly{\childdocname}
  \def\jobname{#1}
  \def\childdocjob{#1}
  \input{#1}
}
%    \end{macrocode}

% \macro{\childdocby}
% The command |\childdocby| ....
%    \begin{macrocode}
\newcommand{\childdocby}[2][]
{
  \childdocdisable
  \childdoctrue
  \childdocmanualtrue
  \if?#1?\else
    \def\jobname{#2}
  \fi
  \def\childdocjob{#2}
  \input{#2}
  \endinput
}
%    \end{macrocode}

% \macro{\childdocforward}
% The command |\childdocforward| redirects
% compilation to the main file or
% (if the optional argument is given) a child file.
% Parameters are set as if the main file
% or a child file starting with |\childdocof| was compiled.
% Then compilation is handed over to the main file:
%    \begin{macrocode}
\newcommand{\childdocforward}[2][]
{
  \begingroup
    \if?#1?
      \def\childdoctmp
      {
        \def\childdocname{#2}
        \def\childdocjob{#2}
        \def\jobname{#2}
        \input{#2}
        \endinput
      }
    \else
      \def\childdoctmp
      {
        \childdocdisable
        \def\childdocname{#2}
        \childdoctrue
        \includeonly{#2}
        \def\childdocjob{#1}
        \def\jobname{#1}
        \input{#1}
        \endinput
      }
    \fi
    \expandafter
  \endgroup
  \childdoctmp
}
%    \end{macrocode}

% \macro{\childdocforwardprefix}
% The command |\childdocforwardprefix| redirects
% compilation to the main or a child file by means of a pattern.
% The prefix |#1| in the current filename is replaced by |#2|
% and the suffix of the current filename is kept
% (it is assumed that the filename does not contain the substring `|~~~|'
% which is used as a delimiter).
% Compilation is handed over to the new file by |\childdocforward|:
%    \begin{macrocode}
\newcommand{\childdocforwardprefix}[3][]
{
  \begingroup
    \def\childdocextract #2##1~~~{\def\childdoctmp{\childdocforward[#1]{#3##1}}}
    \expandafter\childdocextract\childdocname~~~
    \expandafter
  \endgroup
  \childdoctmp
}
%    \end{macrocode}

% \macro{\childdoc}
% The deprecated macro |\childdoc| is a legacy version of |\childdocmain|:
%    \begin{macrocode}
\newcommand{\childdoc}{\childdocmain}
%    \end{macrocode}

% \macro{\childdocredirect}
% The deprecated macro |\childdocredirect| is a legacy version
% of |\childdocforward| and |\childdocforwardprefix|:
%    \begin{macrocode}
\newcommand{\childdocredirect}[2][]
{
  \begingroup
    \if?#1?
      \def\childdoctmp{\childdocforward{#2}}
    \else
      \def\childdoctmp{\childdocforwardprefix{#1}{#2}}
    \fi
    \expandafter
  \endgroup
  \childdoctmp
}
%    \end{macrocode}

%\iffalse
%</package>
%\fi
%
\endinput
|\\
|\childdocby{|\textit{main}|}|\\
\end{tabular}
\end{center}
%
Both forms have slightly different effects as described above.
The main file is prepared as usual, see \secref{sec:include}.

%%%%%%%%%%%%%%%%%%%%%%%%%%%%%%%%%%%%%%%%%%%%%%%%%%%%%%%%%%%%%%%%%%%%%%%%%%%%%%%%
\subsection{Legacy Detection}
\label{sec:detection}

The directive |\childdocmain| in the main file can detect
whether the complete document or merely a child is to be compiled
even without using the directive |\childdocof|.
This method is deprecated because it is less robust
and there is no compelling reason to use it;
it is merely provided for backward compatibility
and it may be removed in future versions.

If the detection mechanism is to be used,
it is mandatory to correctly specify
the filename of the main file as the argument of |\childdocmain|:
%
\begin{center}
\begin{tabular}{l}
|% \iffalse
%
% childdoc.dtx Copyright (C) 2017-2018 Niklas Beisert
%
% This work may be distributed and/or modified under the
% conditions of the LaTeX Project Public License, either version 1.3
% of this license or (at your option) any later version.
% The latest version of this license is in
%   http://www.latex-project.org/lppl.txt
% and version 1.3 or later is part of all distributions of LaTeX
% version 2005/12/01 or later.
%
% This work has the LPPL maintenance status `maintained'.
%
% The Current Maintainer of this work is Niklas Beisert.
%
% This work consists of the files childdoc.dtx and childdoc.ins
% and the derived files childdoc.def and cdocsamp.tex with
% cdocsch1.tex, cdocsch2.tex, cdocsdrf.tex, cdocsfn1.tex, cdocsfn2.tex.
%
%<package>\ifdefined\childdocmain\endinput\fi
%<package>\ProvidesFile{childdoc.def}[2018/12/30 v2.0 child document driver]
%<samplemain>\ProvidesFile{cdocsamp.tex}[2018/12/30 v2.0 sample for childdoc]
%<*driver>
%\ProvidesFile{childdoc.drv}[2018/12/30 v2.0 childdoc reference manual file]
\PassOptionsToClass{10pt,a4paper}{article}
\documentclass{ltxdoc}

\usepackage[margin=35mm]{geometry}
\usepackage{hyperref}
\usepackage{hyperxmp}
\usepackage[usenames]{color}

\hypersetup{colorlinks=true}
\hypersetup{pdfstartview=FitH}
\hypersetup{pdfpagemode=UseNone}
\hypersetup{pdfsource={}}
\hypersetup{pdflang={en-UK}}
\hypersetup{pdfcopyright={Copyright 2017-2018 Niklas Beisert.
  This work may be distributed and/or modified under the
  conditions of the LaTeX Project Public License, either version 1.3
  of this license or (at your option) any later version.}}
\hypersetup{pdflicenseurl={http://www.latex-project.org/lppl.txt}}
\hypersetup{pdfcontactaddress={ETH Zurich, ITP, HIT K,
  Wolfgang-Pauli-Strasse 27}}
\hypersetup{pdfcontactpostcode={8093}}
\hypersetup{pdfcontactcity={Zurich}}
\hypersetup{pdfcontactcountry={Switzerland}}
\hypersetup{pdfcontactemail={nbeisert@itp.phys.ethz.ch}}
\hypersetup{pdfcontacturl={http://people.phys.ethz.ch/\xmptilde nbeisert/}}

\newcommand{\secref}[1]{\hyperref[#1]{section \ref*{#1}}}

\parskip1ex
\parindent0pt
\let\olditemize\itemize
\def\itemize{\olditemize\parskip0pt}

\begin{document}

\title{The \textsf{childdoc} Package}
\hypersetup{pdftitle={The childdoc Package}}
\author{Niklas Beisert\\[2ex]
  Institut f\"ur Theoretische Physik\\
  Eidgen\"ossische Technische Hochschule Z\"urich\\
  Wolfgang-Pauli-Strasse 27, 8093 Z\"urich, Switzerland\\[1ex]
  \href{mailto:nbeisert@itp.phys.ethz.ch}
  {\texttt{nbeisert@itp.phys.ethz.ch}}}
\hypersetup{pdfauthor={Niklas Beisert}}
\hypersetup{pdfsubject={Manual for the LaTeX2e Package childdoc}}
\date{30 December 2018, \textsf{v2.0}}
\maketitle

\begin{abstract}\noindent
\textsf{childdoc} is a \LaTeXe{} package
that enables the direct compilation
of document sections included by |\include|
to individual files.
\end{abstract}

\begingroup
\parskip0ex
\tableofcontents
\endgroup

%%%%%%%%%%%%%%%%%%%%%%%%%%%%%%%%%%%%%%%%%%%%%%%%%%%%%%%%%%%%%%%%%%%%%%%%%%%%%%%%
%%%%%%%%%%%%%%%%%%%%%%%%%%%%%%%%%%%%%%%%%%%%%%%%%%%%%%%%%%%%%%%%%%%%%%%%%%%%%%%%
\section{Introduction}

\LaTeX{} provides a mechanism to structure a large document (such as a book)
into a main file and several child files (containing the chapters)
using the |\include| command.
This mechanism is beneficial for documents
which span hundreds of pages in order to
make the source file(s) more manageable.
Moreover, compilation can be restricted to
selected child files by means of the |\includeonly| command.
The latter feature can be used to reduce the compilation time while editing
(this was significantly more useful in the earlier days of \LaTeX{})
or to generate a smaller document which is easier to navigate.
Another application of |\includeonly| is to generate
documents consisting of selected parts of the complete document.

However, there are a few drawbacks of the plain |\include| mechanism:
\begin{itemize}
\item
The child files cannot be compiled on their own,
they can only be compiled via the main file.
A naive editing environment
(such as a text editor with an option
to have the current file processed by \LaTeX)
may require one to switch to the main file before compiling;
attempting to compile the child file produces errors.
\item
The main file must be modified (each time)
to adjust the |\includeonly| command
to the present needs. This easily leaves the main file in a messy state.
\item
The generated document will always carry the filename
of the main document. This is inconvenient if
several child files are to be compiled and
to be kept for distribution.
\end{itemize}

The present package provides a simple interface
to make child files individually compilable by \LaTeX{}.
Compiling a child file then has the same effect as compiling
the main file with an |\includeonly| command
to select the appropriate child.
Moreover the generated document will carry the name of the child
rather than the main file.
This resolves all three above issues.

This feature is meant to make the editing of books,
thesis documents and lecture notes somewhat more convenient.
However, the package can also be used efficiently for
composing a series of documents (such as exercise sheets)
which are typically distributed individually.
It then assists the author in generating the individual documents
(potentially in different versions)
as well as a document containing the collected series.
Another application is in developing style files
or other kinds of included material
where compilation of the style file could redirect
to a sample or test file.

%%%%%%%%%%%%%%%%%%%%%%%%%%%%%%%%%%%%%%%%%%%%%%%%%%%%%%%%%%%%%%%%%%%%%%%%%%%%%%%%
%%%%%%%%%%%%%%%%%%%%%%%%%%%%%%%%%%%%%%%%%%%%%%%%%%%%%%%%%%%%%%%%%%%%%%%%%%%%%%%%
\section{Usage}

First of all, the package \textsf{childdoc} is \emph{not} a standard
\LaTeXe{} |.sty| style file! Therefore it needs to be invoked in
a non-standard way.

%%%%%%%%%%%%%%%%%%%%%%%%%%%%%%%%%%%%%%%%%%%%%%%%%%%%%%%%%%%%%%%%%%%%%%%%%%%%%%%%
\subsection{Included Files}
\label{sec:include}

%%%%%%%%%%%%%%%%%%%%%%%%%%%%%%%%%%%%%%%%
\DescribeMacro{\childdocmain}
To use the package, add the commands
\begin{center}
\begin{tabular}{l}
|\input{childdoc.def}|\\
|\childdocmain{}|\\
\end{tabular}
\end{center}
at the very top of the main \LaTeX{} file,
in particular \emph{before} the |\documentclass| statement!
The argument of |\childdocmain| should be left empty
(but it must be present).

%%%%%%%%%%%%%%%%%%%%%%%%%%%%%%%%%%%%%%%%
\DescribeMacro{\childdocof}
Furthermore, add the commands
\begin{center}
\begin{tabular}{l}
|\input{childdoc.def}|\\
|\childdocof{|\textit{main}|}|\\
\end{tabular}
\end{center}
at the top of every child file \textit{child}
which is included by |\include{|\textit{child}|}|
from within the main file
(or at least for those files to be compiled individually).
The argument \textit{main} must be the filename of the main file.

There are a couple of
considerations in setting up the main and child documents:

%%%%%%%%%%%%%%%%%%%%%%%%%%%%%%%%%%%%%%%%
\paragraph{Restrictions.}

Please note the following restrictions:
\begin{itemize}
\item
|\childdocmain| must be called with one argument \textit{main}
to ensure compatibility with earlier version of the package.
It must either be empty (|\childdocmain{}|)
or precisely match the filename of the main file in which it is specified.
See \secref{sec:detection} for further information.
\item
The filename \textit{main} must be specified without the |.tex| extension.
\item
The filename \textit{main} is case sensitive
(even in case-insensitive file systems)
due to internal string comparison.
\item
The argument \textit{main} should be fully expanded, it cannot be a macro.
\item
Subdirectories and special characters should be avoided in filenames.
\item
The command |\childdocmain{|\textit{main}|}| must be followed by a whitespace.
It should not be followed immediately by another command
or by a comment mark `|%|'.
This is because the \TeX{} parser reads the token immediately following
the argument of |\childdocmain| and puts it
at the beginning of every child section;
however, a white\-space is ignored.
\end{itemize}

%%%%%%%%%%%%%%%%%%%%%%%%%%%%%%%%%%%%%%%%
\paragraph{Content of Main File.}

It is advisable to place all content in the child files included by |\include|.
Any output contained in the main file will appear in all child documents
unless suppressed manually;
it cannot be suppressed automatically by the |\includeonly| directive
and thus should normally be avoided.
A method to include some content in the main file
by means of conditional processing is described in \secref{sec:conditional}.

%%%%%%%%%%%%%%%%%%%%%%%%%%%%%%%%%%%%%%%%
\paragraph{Page Numbering.}

When only a part of the document is compiled,
the appropriate numbering of pages
(as well as other status parameters)
is determined from the |.aux| files.
The latter contain information from previous passes.
However this information needs to propagate through
all intermediate child documents.
Therefore the page numbering in child documents may well
be inconsistent until the complete document is compiled at least once.

A useful (if unconventional) way to always ensure a consistent
page numbering is to restart the numbering in each child document
and denote the pages by `\textit{child}|.|\textit{page}'
where \textit{child} represents the chapter/section number of the child file.
This can be achieved by the command
|\numberwithin{page}{|\textit{child}|}|
of the \textsf{amsmath} package
where \textit{child} can be |chapter| or |section|
depending on the chosen structuring.
Alternatively, one can modify the macro |\thepage| appropriately
and reset the counter |page| at the start of each child file.

%%%%%%%%%%%%%%%%%%%%%%%%%%%%%%%%%%%%%%%%%%%%%%%%%%%%%%%%%%%%%%%%%%%%%%%%%%%%%%%%
\subsection{Conditional Processing}
\label{sec:conditional}

The package provides a mechanism to compile different versions
of a document. To customise the versions further some conditional processing
can come in handy to distinguish which version is being compiled.
The package provides two macros to describe the compilation context:

%%%%%%%%%%%%%%%%%%%%%%%%%%%%%%%%%%%%%%%%
\DescribeMacro{\ifchilddoc}
The conditional |\ifchilddoc| distinguishes between the compilation of
child documents and the main document:
%
\begin{center}
|\ifchilddoc |\textit{child-code}| |[|\||else |\textit{main-code}]| \||fi|
\end{center}

%%%%%%%%%%%%%%%%%%%%%%%%%%%%%%%%%%%%%%%%
\DescribeMacro{\childdocname}
\DescribeMacro{\childdocjob}
The macro |\childdocname| contains the filename (without extension)
of the main or child file being processed.
Note that |\childdocjob| will always contain the name of the main file.

%%%%%%%%%%%%%%%%%%%%%%%%%%%%%%%%%%%%%%%%
\paragraph{Title Page.}

Conditional processing can be used to include a title or banner page
in the main document when proper precautions are taken.
Importantly, the code in the main file should ensure that the page counter
(as well as other status parameters which are stored in the |.aux| files)
takes the same value after the conditional processing.
Otherwise the page numbers may take divergent values
depending on which part is compiled.

For example, a title page could be declared by:
%
\begin{center}
\begin{tabular}{l}
|\ifchilddoc\||else|\\
|\addtocounter{page}{-1}|\\
\textit{code for title page}\\
|\newpage|\\
|\||fi|
\end{tabular}
\end{center}
%
A banner page for the child documents can be generated by:
%
\begin{center}
\begin{tabular}{l}
|\ifchilddoc|\\
|\addtocounter{page}{-1}|\\
\textit{code for banner page}\\
|\newpage|\\
|\||fi|
\end{tabular}
\end{center}
%
Here one could write a message such as:
\begin{center}
|This is the part \childdocname{} of \childdocjob{}.|
\end{center}

%%%%%%%%%%%%%%%%%%%%%%%%%%%%%%%%%%%%%%%%%%%%%%%%%%%%%%%%%%%%%%%%%%%%%%%%%%%%%%%%
\subsection{Flags}
\label{sec:flags}

The package makes it easy to generate different versions
of the main or child documents.
To this end compilation flags can be defined
and assigned different default values.
They will be particularly useful in conjunction
with the forwarding mechanism described in \secref{sec:forward}.

For example, it may be useful to have a flag |\version|
which can be set to |draft| or |final|.
The document source will contain some conditional code
depending on the value of |\version|.
Suppose further, the flag should default to |final| for the main file
and to |draft| for child files
which is a natural assignment for editing the document.
This is achieved by placing the following code
in the preamble of the main document
(below the |\childdocmain| directive):
%
\begin{center}
\begin{tabular}{l}
|\ifchilddoc|\\
|\providecommand{\version}{draft}|\\
|\||else|\\
|\providecommand{\version}{final}|\\
|\||fi|
\end{tabular}
\end{center}
%
The definition by |\providecommand| makes sure
that previous definitions are not overwritten.
Further statements |\providecommand{\version}{...}|
can thus be added before the above code to override it.

For the main file, one might add a line
(between |\childdocmain| and the above block)
%
\begin{center}
|%\ifchilddoc\||else\providecommand{\version}{draft}\||fi|
\end{center}
%
which can be uncommented to produce a draft version.
Likewise one can add a line to the very top of a child file
(above the |\childdocof{|\textit{main}|}| directive)
%
\begin{center}
|%\providecommand{\version}{final}|
\end{center}
%
which can be uncommented to produce the final version of this child document.

%%%%%%%%%%%%%%%%%%%%%%%%%%%%%%%%%%%%%%%%%%%%%%%%%%%%%%%%%%%%%%%%%%%%%%%%%%%%%%%%
\subsection{Forwarding}
\label{sec:forward}

Different versions of the main or child documents
using compilation flags as described in \secref{sec:flags}
can be (permanently) stored in different files
for convenient compilation, viewing and distribution.
To this end, the package defines a command
to pass on compilation to a different file:

%%%%%%%%%%%%%%%%%%%%%%%%%%%%%%%%%%%%%%%%
\DescribeMacro{\childdocforward}
The command |\childdocforward| redirects processing to
another source file:
%
\begin{center}
\begin{tabular}{l}
|\input{childdoc.def}|\\
|\childdocforward[|\textit{main}|]{|\textit{dest}|}|\\
\end{tabular}
\end{center}
%
The argument \textit{dest} is the destination file
(without extension).
It should be the main file or one of the child files.
Note that further \textsf{childdoc} directives
such as |\childdocof| and |\childdocforward|
in the indicated file will be processed in this form.
The optional argument \textit{main}
passes on directly to the main file \textit{main}
while pretending to compile the child \textit{dest}.
This form behaves as if \textit{dest}
issues |\childdocof{|\textit{main}|}| right away,
and no further \textsf{childdoc} directives will be processed.

%%%%%%%%%%%%%%%%%%%%%%%%%%%%%%%%%%%%%%%%
\DescribeMacro{\...prefix}
In the alternative form |\childdocforwardprefix|,
%
\begin{center}
\begin{tabular}{l}
|\input{childdoc.def}|\\
|\childdocforwardprefix[|\textit{main}|]{|\textit{prefix}|}{|\textit{dest}|}|
\end{tabular}
\end{center}
%
the destination file is determined by a pattern
depending on the current file:
To make this work, the current file must be called
`{\textit{prefix}\hspace{0.2em}\textit{suffix}}'
with \textit{prefix} matching precisely the argument.
Processing is then passed on to the file
`{\textit{dest}\hspace{0.2em}\textit{suffix}}'.
Surely, the same effect is achieved by
directly specifying the
argument `{\textit{dest}\hspace{0.2em}\textit{suffix}}'
in the first form.
However, that requires to set up a different file
for each child. With the alternative form of the command
all these files can have exactly the same content
which simplifies setting them up and maintaining them.

For example, the following file |draft.tex|
with a compilation flag |\version| as described in \secref{sec:flags}
compiles the main document as a draft:
%
\begin{center}
\begin{tabular}{l}
|\def\version{draft}|\\
|\input{childdoc.def}|\\
|\childdocforward{|\textit{main}|}|
\end{tabular}
\end{center}
%
Likewise, the following files |final|\textit{nn}|.tex|
compile the final version of the child document
|child|\textit{nn}|.tex|:
%
\begin{center}
\begin{tabular}{l}
|\def\version{final}|\\
|\input{childdoc.def}|\\
|\childdocforwardprefix{final}{child}|
\end{tabular}
\end{center}
%

Note that when several versions of a main file and/or of each child file
are to be generated, it may be convenient to set up a |Makefile| or
shell script to automatise the process.

%%%%%%%%%%%%%%%%%%%%%%%%%%%%%%%%%%%%%%%%%%%%%%%%%%%%%%%%%%%%%%%%%%%%%%%%%%%%%%%%
\subsection{Command Line Processing}
\label{sec:commandline}

The effect of redirection files can also be achieved by invoking
the \LaTeX{} compiler with a more elaborate command line.
Most conveniently this should be done as part
of a shell script or a |Makefile|.

When using \textsf{childdoc} in the main file, the following
command lines effectively perform a redirection
(note that depending on the shell being used,
backslashes may have to be doubled: `|\|' $\to$ `|\\|'):
%
\begin{center}
|... -jobname "|\textit{target}|" |\\|"|[\textit{flags}]%
|\input{childdoc.def}\childdocforward[|\textit{main}|]{|\textit{dest}|}"|
\end{center}
%
Here \textit{target} is the name of the output file,
\textit{main} is the name of the main file
and \textit{dest} is the name of the main or child file to be processed
(all filenames without extensions).
The optional argument \textit{main} can be omitted
if \textit{main} matches \textit{dest}.
Optionally, compilation \textit{flags} can be defined via |\def| commands.
This command line makes the \TeX{} engine believe
it is compiling the file \textit{target}
whose content is specified as the latter parameter.
The provided code then forwards the processing to
\textit{main} or \textit{dest} as described in \secref{sec:forward}.

%%%%%%%%%%%%%%%%%%%%%%%%%%%%%%%%%%%%%%%%%%%%%%%%%%%%%%%%%%%%%%%%%%%%%%%%%%%%%%%%
\subsection{Include by Input}
\label{sec:input}

Including child documents by |\include| has some restrictions by design.
Most notably, the content of a child document always occupies
its own set of pages; pages cannot be shared between child documents.
Usually, this behaviour makes perfect sense
because each child document contain an essential part of the document.
However, in some situations it may be desirable to compose
a document from a collection of parts
without having mandatory page breaks between then.
For this case, the package
provides a mechanism to include parts
by |\input| which can also be processed individually.
However, by construction this mechanism
requires manual handling of the content to be output.

%%%%%%%%%%%%%%%%%%%%%%%%%%%%%%%%%%%%%%%%
\DescribeMacro{\ifchilddocmanual}
The main file should be prepared as usual, see \secref{sec:include}.
However, the document body must make a distinction
between processing of an individual part and of the main document, e.g.:
%
\begin{center}
\begin{tabular}{l}
|\ifchilddocmanual|\\
|\input{\childdocname}|\\
|\||else|\\
\textit{document body with }|\input{|\textit{part}|}|\\
|\||fi|
\end{tabular}
\end{center}
%
The conditional |\ifchilddocmanual| is true whenever
a part to be included by |\input| is being compiled,
and the name of the part is stored in |\childdocname|.

%%%%%%%%%%%%%%%%%%%%%%%%%%%%%%%%%%%%%%%%
\DescribeMacro{\childdocby}
Each part to be included by |\input| should start with:
%
\begin{center}
\begin{tabular}{l}
|\input{childdoc.def}|\\
|\childdocby{|\textit{main}|}|\\
\end{tabular}
\end{center}
%
The directive |\childdocby| is similar to |\childdocof|
described in \secref{sec:include},
but the subsequent selection of content must be done manually.
To that end, both |\ifchilddoc| and |\ifchilddocmanual|
will be true upon processing of a part,
and the name of the part is stored in |\childdocname|.
Note that |\jobname| will be set to the filename of the current part
so that each part receives an individual |.aux| file
that does not interfere with the |.aux| file(s) of the main document.
This behaviour can be altered by the alternative form
|\childdocby[*]{|\textit{main}|}| (with a non-empty optional argument)
which uses the |.aux| file of the main document
by setting |\jobname| to \textit{main}.

%%%%%%%%%%%%%%%%%%%%%%%%%%%%%%%%%%%%%%%%%%%%%%%%%%%%%%%%%%%%%%%%%%%%%%%%%%%%%%%%
\subsection{Driver Development}
\label{sec:driver}

The \textsf{childdoc} mechanism can also be use for the development
of definition files such as \LaTeX{} styles or classes.
This case differs from the above setup with multiple parts
included by |\include| in that no |\includeonly| should be invoked.
This can be achieved by starting the include file
(before |\ProvidesPackage|) with:
%
\begin{center}
\begin{tabular}{l}
|\input{childdoc.def}|\\
|\childdocforward{|\textit{main}|}|\\
\end{tabular}
\end{center}
%
or alternatively with:
%
\begin{center}
\begin{tabular}{l}
|\input{childdoc.def}|\\
|\childdocby{|\textit{main}|}|\\
\end{tabular}
\end{center}
%
Both forms have slightly different effects as described above.
The main file is prepared as usual, see \secref{sec:include}.

%%%%%%%%%%%%%%%%%%%%%%%%%%%%%%%%%%%%%%%%%%%%%%%%%%%%%%%%%%%%%%%%%%%%%%%%%%%%%%%%
\subsection{Legacy Detection}
\label{sec:detection}

The directive |\childdocmain| in the main file can detect
whether the complete document or merely a child is to be compiled
even without using the directive |\childdocof|.
This method is deprecated because it is less robust
and there is no compelling reason to use it;
it is merely provided for backward compatibility
and it may be removed in future versions.

If the detection mechanism is to be used,
it is mandatory to correctly specify
the filename of the main file as the argument of |\childdocmain|:
%
\begin{center}
\begin{tabular}{l}
|\input{childdoc.def}|\\
|\childdocmain{|\textit{main}|}|\\
\end{tabular}
\end{center}
%
If |\jobname| does not match the argument \textit{main} of |\childdocmain|,
it is assumed that |\jobname| points to the child file to be compiled.
When using |\childdocmain| with the main file specified as argument,
it suffices to start a child file
with just |\input{|\textit{main}|}|
without loading of the package and using |\childdocof|.
If instead all processing is done
with the appropriate \textsf{childdoc} directives,
the argument of \textit{main} of |\childdocmain| can be empty.

An alternative version of the command line processing described
in \secref{sec:commandline} using the detection mechanism reads:
%
\begin{center}
|... -jobname "|\textit{target}|" "|[\textit{flags}]%
[|\def\jobname{|\textit{dest}|}|]|\input{|\textit{main}|}"|
\end{center}

%%%%%%%%%%%%%%%%%%%%%%%%%%%%%%%%%%%%%%%%%%%%%%%%%%%%%%%%%%%%%%%%%%%%%%%%%%%%%%%%
\subsection{Manual Code}
\label{sec:manual}

In case one cannot be certain whether the definitions file |childdoc.def|
is installed on the target \TeX{} distribution
and one prefers not to ship it,
it is conceivable to paste a few relevant commands into the sources.

To that end, drop all statements |\input{childdoc.def}|
and perform the replacements as outlined below.
Instead of |\childdocmain{|\textit{main}|}| add the following code
to the top of the main file:
%
\begin{center}
\begin{tabular}{l}
|\||ifdefined\childdocname\endinput\||fi\newif\ifchilddoc|\\
|\edef\childdocname{\scantokens\expandafter{\jobname\noexpand}}|\\
|\def\childdocmain{|\textit{main}|}\||ifx\childdocmain\childdocname\||else|\\
|\childdoctrue\includeonly{\childdocname}\let\jobname\childdocmain\||fi|\\
\end{tabular}
\end{center}
%
Instead of |\childdocof{|\textit{main}|}| just include the main file
at the top of each child file:
%
\begin{center}
|\input{|\textit{main}|}|
\end{center}
%
A simple redirection |\childdocforward{|\textit{dest}|}| is achieved by:
%
\begin{center}
|\def\jobname{|\textit{dest}|}\input{\jobname}|
\end{center}
%
The redirection with prefix
|\childdocforwardprefix[|\textit{prefix}|]{|\textit{dest}|}|
is accomplished by:
%
\begin{center}
\begin{tabular}{l}
|{\edef\jobname{\scantokens\expandafter{\jobname\noexpand}}|\\
|\def\redirectjob |\textit{prefix}|#1~~~{\gdef\jobname{|\textit{dest}|#1}}|\\
|\expandafter\redirectjob\jobname~~~}\input{\jobname}|
\end{tabular}
\end{center}

In an alternative approach,
child documents can be compiled by a specific command line
without additional code or specific definitions:
%
\begin{center}
|... -jobname "|\textit{target}|" "|[\textit{flags}]%
|\includeonly{|\textit{dest}|}\input{|\textit{main}|}"|
\end{center}
%

%%%%%%%%%%%%%%%%%%%%%%%%%%%%%%%%%%%%%%%%%%%%%%%%%%%%%%%%%%%%%%%%%%%%%%%%%%%%%%%%
%%%%%%%%%%%%%%%%%%%%%%%%%%%%%%%%%%%%%%%%%%%%%%%%%%%%%%%%%%%%%%%%%%%%%%%%%%%%%%%%
\section{Information}

%%%%%%%%%%%%%%%%%%%%%%%%%%%%%%%%%%%%%%%%%%%%%%%%%%%%%%%%%%%%%%%%%%%%%%%%%%%%%%%%
\subsection{Copyright}

Copyright \copyright{} 2017--2018 Niklas Beisert

This work may be distributed and/or modified under the
conditions of the \LaTeX{} Project Public License, either version 1.3
of this license or (at your option) any later version.
The latest version of this license is in
  \url{http://www.latex-project.org/lppl.txt}
and version 1.3 or later is part of all distributions of \LaTeX{}
version 2005/12/01 or later.

This work has the LPPL maintenance status `maintained'.

The Current Maintainer of this work is Niklas Beisert.

This work consists of the files |README.txt|, |childdoc.ins| and |childdoc.dtx|
as well as the derived files |childdoc.def|, |cdocsamp.tex|
with |cdocsch1.tex|, |cdocsch2.tex|, |cdocspt3.tex|, |cdocspt4.tex|,
|cdocsdrf.tex|, |cdocsfn1.tex|, |cdocsfn2.tex|
as well as |childdoc.pdf|.

%%%%%%%%%%%%%%%%%%%%%%%%%%%%%%%%%%%%%%%%%%%%%%%%%%%%%%%%%%%%%%%%%%%%%%%%%%%%%%%%
\subsection{Files and Installation}

The package consists of the files:
%
\begin{center}
\begin{tabular}{ll}
    |README.txt|   & readme file \\
    |childdoc.ins| & installation file \\
    |childdoc.dtx| & source file \\
    |childdoc.def| & definition file \\
    |cdocsamp.tex| & sample main file \\
    |cdocsch1.tex| & sample include file \\
    |cdocsch2.tex| & sample include file \\
    |cdocspt3.tex| & sample part file \\
    |cdocspt4.tex| & sample part file \\
    |cdocsdrf.tex| & sample redirection file \\
    |cdocsfn1.tex| & sample redirection file \\
    |cdocsfn2.tex| & sample redirection file \\
    |childdoc.pdf| & manual
\end{tabular}
\end{center}
%
The distribution consists of the files
|README.txt|, |childdoc.ins| and |childdoc.dtx|.
%
\begin{itemize}
\item
Run (pdf)\LaTeX{} on |childdoc.dtx|
to compile the manual |childdoc.pdf| (this file).
\item
Run \LaTeX{} on |childdoc.ins| to create the definitions file |childdoc.def|
and the sample |cdocsamp.tex| with include files
|cdocsch1.tex|, |cdocsch2.tex|, |cdocspt3.tex|, |cdocspt4.tex|,
|cdocsdrf.tex|, |cdocsfn1.tex|, |cdocsfn2.tex|.
Then copy the file |childdoc.def| to an appropriate directory of your \LaTeX{}
distribution, e.g.\ \textit{texmf-root}|/tex/latex/childdoc|.
\end{itemize}

%%%%%%%%%%%%%%%%%%%%%%%%%%%%%%%%%%%%%%%%%%%%%%%%%%%%%%%%%%%%%%%%%%%%%%%%%%%%%%%%
\subsection{Related CTAN Packages}

There are several other packages which offer a similar functionality:
%
\begin{itemize}
\item
The packages
\href{http://ctan.org/pkg/docmute}{\textsf{docmute}},
\href{http://ctan.org/pkg/includex}{\textsf{includex}} and
\href{http://ctan.org/pkg/standalone}{\textsf{standalone}}
provide commands to include only the document body of
a child file thus allowing both files to be compiled individually.
\item
The packages \href{http://ctan.org/pkg/subdocs}{\textsf{subdocs}}
and \href{http://ctan.org/pkg/subfiles}{\textsf{subfiles}}
provide structures in which the main and child documents can be
encapsulated and allowing them to be compiled individually.
The inclusion mechanism is different from the conventional |\include|.
\item
The package \href{http://ctan.org/pkg/combine}{\textsf{combine}}
is an elaborate solution to combine several documents into one.
\end{itemize}
%
See also the CTAN topic \href{http://ctan.org/topic/subdocs}{\textsf{subdocs}}
for further related packages.
The present package differs from the above solutions in that
a document structure constructed with the conventional |\include| mechanism
just needs two extra commands at the top of every file
such that all constituent files can be compiled individually.

%%%%%%%%%%%%%%%%%%%%%%%%%%%%%%%%%%%%%%%%%%%%%%%%%%%%%%%%%%%%%%%%%%%%%%%%%%%%%%%%
%\subsection{Feature Suggestions}
%
%The following is a list of features which may be useful for future
%versions of this package:
%%
%\begin{itemize}
%\item
%\ldots
%\end{itemize}

%%%%%%%%%%%%%%%%%%%%%%%%%%%%%%%%%%%%%%%%%%%%%%%%%%%%%%%%%%%%%%%%%%%%%%%%%%%%%%%%
\subsection{Revision History}

%%%%%%%%%%%%%%%%%%%%%%%%%%%%%%%%%%%%%%%%
\paragraph{v2.0:} 2018/12/30

\begin{itemize}
\item
immediate forward processing
\item
added |\childdocby| mechanism
\item
manual restructured
\end{itemize}

%%%%%%%%%%%%%%%%%%%%%%%%%%%%%%%%%%%%%%%%
\paragraph{v1.6:} 2018/01/17

\begin{itemize}
\item
application for development of include files
\item
corrections to manual
\end{itemize}

%%%%%%%%%%%%%%%%%%%%%%%%%%%%%%%%%%%%%%%%
\paragraph{v1.5:} 2017/05/21

\begin{itemize}
\item
more complete structuring introduced
\item
|\childdocof| introduced
\item
|\childdoc| renamed to |\childdocmain|
\item
|\childredirect| renamed to |\childdocforward| and |\childdocforwardprefix|
and functionality expanded
\end{itemize}

%%%%%%%%%%%%%%%%%%%%%%%%%%%%%%%%%%%%%%%%
\paragraph{v1.0:} 2017/04/27

\begin{itemize}
\item
manual and install package
\item
first version published on CTAN
\end{itemize}

%%%%%%%%%%%%%%%%%%%%%%%%%%%%%%%%%%%%%%%%
\paragraph{v0.6:} 2017/04/26

\begin{itemize}
\item
redirection mechanism added
\end{itemize}

%%%%%%%%%%%%%%%%%%%%%%%%%%%%%%%%%%%%%%%%
\paragraph{v0.5:} 2017/04/26

\begin{itemize}
\item
functionality in definition file
\end{itemize}


%%%%%%%%%%%%%%%%%%%%%%%%%%%%%%%%%%%%%%%%%%%%%%%%%%%%%%%%%%%%%%%%%%%%%%%%%%%%%%%%
%%%%%%%%%%%%%%%%%%%%%%%%%%%%%%%%%%%%%%%%%%%%%%%%%%%%%%%%%%%%%%%%%%%%%%%%%%%%%%%%
%%%%%%%%%%%%%%%%%%%%%%%%%%%%%%%%%%%%%%%%%%%%%%%%%%%%%%%%%%%%%%%%%%%%%%%%%%%%%%%%
\appendix

\settowidth\MacroIndent{\rmfamily\scriptsize 000\ }

 \DocInput{childdoc.dtx}

\end{document}
%</driver>
% \fi
%
% %%%%%%%%%%%%%%%%%%%%%%%%%%%%%%%%%%%%%%%%%%%%%%%%%%%%%%%%%%%%%%%%%%%%%%%%%%%%%%
% %%%%%%%%%%%%%%%%%%%%%%%%%%%%%%%%%%%%%%%%%%%%%%%%%%%%%%%%%%%%%%%%%%%%%%%%%%%%%%
% \section{Sample}
%\iffalse
%<*samplemain>
%\fi
%
% The following presents a sample document
% with two chapters, two parts, a title page,
% a compile flag as well as three forwarding files to set the flag.
% It consists of eight |.tex| files:
% \begin{center}
% \begin{tabular}{ll}
% |cdocsamp.tex|&main file\\
% |cdocsch1.tex|&include file for chapter 1\\
% |cdocsch2.tex|&include file for chapter 2\\
% |cdocspt3.tex|&include file for part 3\\
% |cdocspt4.tex|&include file for part 4\\
% |cdocsdrf.tex|&forwarding file for main file in draft mode\\
% |cdocsfi1.tex|&forwarding file for final version of chapter 1\\
% |cdocsfi2.tex|&forwarding file for final version of chapter 2\\
% \end{tabular}
% \end{center}
% Each of the eight files can be compiled directly by the \LaTeX{} compiler.
%
% %%%%%%%%%%%%%%%%%%%%%%%%%%%%%%%%%%%%%%
% \paragraph{Main File.}
%
% The main file is called |cdocsamp.tex|.
%
% Load the \textsf{childdoc} definitions and
% declare the filename for the main document:
%    \begin{macrocode}
\input{childdoc.def}
\childdocmain{}
%    \end{macrocode}

% Optional override for |\version| flag:
%    \begin{macrocode}
%%\ifchilddoc\else\providecommand{\version}{draft}\fi
%    \end{macrocode}

% Define the default values for the |\version| flag
% (|final| for the main file and |draft| for childs):
%    \begin{macrocode}
\ifchilddoc
\providecommand{\version}{draft}
\else
\providecommand{\version}{final}
\fi
%    \end{macrocode}

% Load the standard document class:
%    \begin{macrocode}
\documentclass[12pt]{article}
%    \end{macrocode}

% Start the document body:
%    \begin{macrocode}
\begin{document}
%    \end{macrocode}

% Declare a title page.
% Print title, part of document being processed and version flag:
%    \begin{macrocode}
\addtocounter{page}{-1}
\begin{center}
{\LARGE\bfseries{}childdoc example\par}
\vspace{1cm}
\ifchilddoc
\ifchilddocmanual part\else chapter\fi:
`\childdocname' of `\childdocjob'\par
\else
main document: `\childdocjob'\par
\fi
version: \version\par
\end{center}
\newpage
%    \end{macrocode}

% Manually include selected file,
% otherwise process as usual:
%    \begin{macrocode}
\ifchilddocmanual
\section*{part `\childdocname'}
\input{\childdocname}
\else
%    \end{macrocode}

% Include the two chapters:
%    \begin{macrocode}
\include{cdocsch1}
\include{cdocsch2}
%    \end{macrocode}

% Include the two parts unless only chapters should be displayed:
%    \begin{macrocode}
\ifchilddoc\else
\section{part three}
\input{cdocspt3}
\section{part four}
\input{cdocspt4}
\fi
%    \end{macrocode}

% Process as usual until here:
%    \begin{macrocode}
\fi
%    \end{macrocode}

% End of document body:
%    \begin{macrocode}
\end{document}
%    \end{macrocode}
%\iffalse
%</samplemain>
%\fi
%
% %%%%%%%%%%%%%%%%%%%%%%%%%%%%%%%%%%%%%%
% \paragraph{Chapter Include Files.}
%
% The include files are called |cdocsch1.tex| and |cdocsch2.tex|.
%
%\iffalse
%<*samplechap1|samplechap2>
%\fi

% Optional override for |\version| flag:
%    \begin{macrocode}
%%\providecommand{\version}{final}
%    \end{macrocode}

% Include the main document:
%    \begin{macrocode}
\input{childdoc.def}
\childdocof{cdocsamp}
%    \end{macrocode}

%\iffalse
%</samplechap1|samplechap2>
%\fi
%
%\iffalse
%<*samplechap1>
%\fi
% Some text for chapter 1:
%    \begin{macrocode}
\section{one}
some text in chapter one
%    \end{macrocode}

%\iffalse
%</samplechap1>
%\fi
% Some text for chapter 2:
%\iffalse
%<*samplechap2>
%\fi
%    \begin{macrocode}
\section{two}
more text in chapter two
%    \end{macrocode}

%\iffalse
%</samplechap2>
%\fi
%
% %%%%%%%%%%%%%%%%%%%%%%%%%%%%%%%%%%%%%%
% \paragraph{Part Include Files.}
%
% The include files are called |cdocspt3.tex| and |cdocspt4.tex|.
%
%\iffalse
%<*samplepart3|samplepart4>
%\fi

% Optional override for |\version| flag:
%    \begin{macrocode}
%%\providecommand{\version}{final}
%    \end{macrocode}

% Include the main document:
%    \begin{macrocode}
\input{childdoc.def}
\childdocby{cdocsamp}
%    \end{macrocode}

%\iffalse
%</samplepart3|samplepart4>
%\fi
%
%\iffalse
%<*samplepart3>
%\fi
% Some text for part 3:
%    \begin{macrocode}
some text in part three
%    \end{macrocode}

%\iffalse
%</samplepart3>
%\fi
% Some text for part 4:
%\iffalse
%<*samplepart4>
%\fi
%    \begin{macrocode}
more text in part four
%    \end{macrocode}

%\iffalse
%</samplepart4>
%\fi
%
% %%%%%%%%%%%%%%%%%%%%%%%%%%%%%%%%%%%%%%
% \paragraph{Forwarding for a Complete Draft.}
%
% The following forwarding file |cdocsdrf.tex|
% compiles the main document in draft mode:
%\iffalse
%<*sampledraft>
%\fi
%    \begin{macrocode}
\def\version{draft}
\input{childdoc.def}
\childdocforward{cdocsamp}
%    \end{macrocode}

%\iffalse
%</sampledraft>
%\fi
%
% %%%%%%%%%%%%%%%%%%%%%%%%%%%%%%%%%%%%%%
% \paragraph{Forwarding for Final Version of the Chapters.}
%
% The following forwarding files |cdocsfn1.tex| and |cdocsfn2.tex|
% (with identical content)
% compile the final versions of the child documents
% |cdocsch1.tex| and |cdocsch2.tex|, respectively:
%\iffalse
%<*samplefinal>
%\fi
%    \begin{macrocode}
\def\version{final}
\input{childdoc.def}
\childdocforwardprefix[cdocsamp]{cdocsfn}{cdocsch}
%    \end{macrocode}

%\iffalse
%</samplefinal>
%\fi
%
% %%%%%%%%%%%%%%%%%%%%%%%%%%%%%%%%%%%%%%
% \paragraph{Command Line Processing.}
%
% The following three command lines generate the output files
% |cdocscld|, |cdocscl1| and |cdocscl2|
% which should be identical to
% |cdocsdrf|, |cdocsch1| and |cdocsfn2|, respectively:
% \begin{center}
% \begin{tabular}{l}
% |latex -jobname cdocscld \|\\
% |  "\def\version{draft}\input{childdoc.def}\childdocforward{cdocsamp}"|\\
% |latex -jobname cdocscl1 \|\\
% |  "\input{childdoc.def}\childdocforward[cdocsamp]{cdocsch1}"|\\
% |latex -jobname cdocscl2 \|\\
% |  "\def\version{final}\input{childdoc.def}\childdocforward{cdocsch2}"|
% \end{tabular}
% \end{center}
% Note that the trailing backslash on each first line
% merely continues the input to the second line
% (for convenient cut ant paste).
% Furthermore, the command |latex| can be replaced by any
% of its alternative versions such as |pdflatex|.
%
% %%%%%%%%%%%%%%%%%%%%%%%%%%%%%%%%%%%%%%%%%%%%%%%%%%%%%%%%%%%%%%%%%%%%%%%%%%%%%%
% %%%%%%%%%%%%%%%%%%%%%%%%%%%%%%%%%%%%%%%%%%%%%%%%%%%%%%%%%%%%%%%%%%%%%%%%%%%%%%
% \section{Implementation}
%\iffalse
%<*package>
%\fi
%
% This section describes the definitions file |childdoc.def|.

% The definitions cannot be loaded using |\usepackage| or |\RequirePackage|
% which has a mechanism to prevent loading a style file more than once.
% When loading the definitions by means of |\input|
% multiple instances have to be prevented manually:
%\iffalse
%This code needs to be before the `\ProvidesFile' directive
%which is defined at the beginning of this file.
%Therefore it is also placed there and commented out here.
%</package>
%<*discard>
%\fi
%    \begin{macrocode}
\ifdefined\childdocmain\endinput\fi
%    \end{macrocode}
%\iffalse
%</discard>
%<*package>
%\fi
%
% \macro{\ifchilddoc}
% \macro{\ifchilddocmanual}
% The conditional |\ifchilddoc| tells whether a
% child (true) or main (false) document is being compiled.
% The conditional |\ifchilddocmanual| tells whether
% the |\includeonly| mechanism is used (false) or
% the selection of child files must be performed manually (true).
% The definitions initialise to false:
%    \begin{macrocode}
\newif\ifchilddoc
\newif\ifchilddocmanual
%    \end{macrocode}

% \macro{\childdocname}
% \macro{\childdocjob}
% The macro |\childdocname| stores the name of the main document
% to be compiled. The macro |\childdocjob| stores the name of
% the document on which the \LaTeX{} compiler was originally invoked.
% The content of |\jobname| cannot be compared
% to filenames specified in the source due to different catcodes.
% The following code rescans |\jobname|, stores the result
% in |\childdocname| and saves a copy in |\childdocjob|:
%    \begin{macrocode}
\edef\childdocname{\scantokens\expandafter{\jobname\noexpand}}
\let\childdocjob\childdocname
%    \end{macrocode}

% \macro{\childdocdisable}
% The macro |\childdocdisable| prevents the main file
% from being processed more than once.
% At this stage, the main document command |\childdocmain|
% is assumed to be called once again where it should do nothing.
% Any subsequent call to it should prevent
% a secondary processing of the main document
% It overwrites the forwarding commands
% |\childdocof| and |\childdocforward|
% with empty macros to prevent further inclusions of the main document:
%    \begin{macrocode}
\newcommand{\childdocdisable}
{
  \renewcommand{\childdocmain}[1]{\renewcommand{\childdocmain}[1]{\endinput}}
  \renewcommand{\childdocof}[1]{}
  \renewcommand{\childdocby}[2][]{}
  \renewcommand{\childdocforward}[2][]{}
  \renewcommand{\childdocdisable}{}
}
%    \end{macrocode}

% \macro{\childdocmain}
% The macro |\childdocmain| is to be called at the top of the main file
% with nothing or the main filename (without extension) as argument.
% First, it breaks loops.
% If the argument is not empty and does not match |\childdocname|
% (which is set by the first inclusion of |childdoc.def|),
% |\ifchilddoc| is set to true, |\includeonly| is applied to the child file
% and |\jobname| is set to the main file
% (for proper handling of |.aux| files):
%    \begin{macrocode}
\newcommand{\childdocmain}[1]
{
  \childdocdisable\childdocmain{}
  \if?#1?\else
    \begingroup
      \def\childdoctmp{#1}
      \ifx\childdoctmp\childdocname
        \def\childdoctmp{}
      \else
        \def\childdoctmp
        {
          \childdoctrue
          \includeonly{\childdocname}
          \def\childdocjob{#1}
          \def\jobname{#1}
        }
      \fi
      \expandafter
    \endgroup
    \childdoctmp
  \fi
}
%    \end{macrocode}

% \macro{\childdocof}
% The command |\childdocof| redirects
% compilation to the main file |#1|.
%    \begin{macrocode}
\newcommand{\childdocof}[1]
{
  \childdocdisable
  \childdoctrue
  \includeonly{\childdocname}
  \def\jobname{#1}
  \def\childdocjob{#1}
  \input{#1}
}
%    \end{macrocode}

% \macro{\childdocby}
% The command |\childdocby| ....
%    \begin{macrocode}
\newcommand{\childdocby}[2][]
{
  \childdocdisable
  \childdoctrue
  \childdocmanualtrue
  \if?#1?\else
    \def\jobname{#2}
  \fi
  \def\childdocjob{#2}
  \input{#2}
  \endinput
}
%    \end{macrocode}

% \macro{\childdocforward}
% The command |\childdocforward| redirects
% compilation to the main file or
% (if the optional argument is given) a child file.
% Parameters are set as if the main file
% or a child file starting with |\childdocof| was compiled.
% Then compilation is handed over to the main file:
%    \begin{macrocode}
\newcommand{\childdocforward}[2][]
{
  \begingroup
    \if?#1?
      \def\childdoctmp
      {
        \def\childdocname{#2}
        \def\childdocjob{#2}
        \def\jobname{#2}
        \input{#2}
        \endinput
      }
    \else
      \def\childdoctmp
      {
        \childdocdisable
        \def\childdocname{#2}
        \childdoctrue
        \includeonly{#2}
        \def\childdocjob{#1}
        \def\jobname{#1}
        \input{#1}
        \endinput
      }
    \fi
    \expandafter
  \endgroup
  \childdoctmp
}
%    \end{macrocode}

% \macro{\childdocforwardprefix}
% The command |\childdocforwardprefix| redirects
% compilation to the main or a child file by means of a pattern.
% The prefix |#1| in the current filename is replaced by |#2|
% and the suffix of the current filename is kept
% (it is assumed that the filename does not contain the substring `|~~~|'
% which is used as a delimiter).
% Compilation is handed over to the new file by |\childdocforward|:
%    \begin{macrocode}
\newcommand{\childdocforwardprefix}[3][]
{
  \begingroup
    \def\childdocextract #2##1~~~{\def\childdoctmp{\childdocforward[#1]{#3##1}}}
    \expandafter\childdocextract\childdocname~~~
    \expandafter
  \endgroup
  \childdoctmp
}
%    \end{macrocode}

% \macro{\childdoc}
% The deprecated macro |\childdoc| is a legacy version of |\childdocmain|:
%    \begin{macrocode}
\newcommand{\childdoc}{\childdocmain}
%    \end{macrocode}

% \macro{\childdocredirect}
% The deprecated macro |\childdocredirect| is a legacy version
% of |\childdocforward| and |\childdocforwardprefix|:
%    \begin{macrocode}
\newcommand{\childdocredirect}[2][]
{
  \begingroup
    \if?#1?
      \def\childdoctmp{\childdocforward{#2}}
    \else
      \def\childdoctmp{\childdocforwardprefix{#1}{#2}}
    \fi
    \expandafter
  \endgroup
  \childdoctmp
}
%    \end{macrocode}

%\iffalse
%</package>
%\fi
%
\endinput
|\\
|\childdocmain{|\textit{main}|}|\\
\end{tabular}
\end{center}
%
If |\jobname| does not match the argument \textit{main} of |\childdocmain|,
it is assumed that |\jobname| points to the child file to be compiled.
When using |\childdocmain| with the main file specified as argument,
it suffices to start a child file
with just |\input{|\textit{main}|}|
without loading of the package and using |\childdocof|.
If instead all processing is done
with the appropriate \textsf{childdoc} directives,
the argument of \textit{main} of |\childdocmain| can be empty.

An alternative version of the command line processing described
in \secref{sec:commandline} using the detection mechanism reads:
%
\begin{center}
|... -jobname "|\textit{target}|" "|[\textit{flags}]%
[|\def\jobname{|\textit{dest}|}|]|\input{|\textit{main}|}"|
\end{center}

%%%%%%%%%%%%%%%%%%%%%%%%%%%%%%%%%%%%%%%%%%%%%%%%%%%%%%%%%%%%%%%%%%%%%%%%%%%%%%%%
\subsection{Manual Code}
\label{sec:manual}

In case one cannot be certain whether the definitions file |childdoc.def|
is installed on the target \TeX{} distribution
and one prefers not to ship it,
it is conceivable to paste a few relevant commands into the sources.

To that end, drop all statements |% \iffalse
%
% childdoc.dtx Copyright (C) 2017-2018 Niklas Beisert
%
% This work may be distributed and/or modified under the
% conditions of the LaTeX Project Public License, either version 1.3
% of this license or (at your option) any later version.
% The latest version of this license is in
%   http://www.latex-project.org/lppl.txt
% and version 1.3 or later is part of all distributions of LaTeX
% version 2005/12/01 or later.
%
% This work has the LPPL maintenance status `maintained'.
%
% The Current Maintainer of this work is Niklas Beisert.
%
% This work consists of the files childdoc.dtx and childdoc.ins
% and the derived files childdoc.def and cdocsamp.tex with
% cdocsch1.tex, cdocsch2.tex, cdocsdrf.tex, cdocsfn1.tex, cdocsfn2.tex.
%
%<package>\ifdefined\childdocmain\endinput\fi
%<package>\ProvidesFile{childdoc.def}[2018/12/30 v2.0 child document driver]
%<samplemain>\ProvidesFile{cdocsamp.tex}[2018/12/30 v2.0 sample for childdoc]
%<*driver>
%\ProvidesFile{childdoc.drv}[2018/12/30 v2.0 childdoc reference manual file]
\PassOptionsToClass{10pt,a4paper}{article}
\documentclass{ltxdoc}

\usepackage[margin=35mm]{geometry}
\usepackage{hyperref}
\usepackage{hyperxmp}
\usepackage[usenames]{color}

\hypersetup{colorlinks=true}
\hypersetup{pdfstartview=FitH}
\hypersetup{pdfpagemode=UseNone}
\hypersetup{pdfsource={}}
\hypersetup{pdflang={en-UK}}
\hypersetup{pdfcopyright={Copyright 2017-2018 Niklas Beisert.
  This work may be distributed and/or modified under the
  conditions of the LaTeX Project Public License, either version 1.3
  of this license or (at your option) any later version.}}
\hypersetup{pdflicenseurl={http://www.latex-project.org/lppl.txt}}
\hypersetup{pdfcontactaddress={ETH Zurich, ITP, HIT K,
  Wolfgang-Pauli-Strasse 27}}
\hypersetup{pdfcontactpostcode={8093}}
\hypersetup{pdfcontactcity={Zurich}}
\hypersetup{pdfcontactcountry={Switzerland}}
\hypersetup{pdfcontactemail={nbeisert@itp.phys.ethz.ch}}
\hypersetup{pdfcontacturl={http://people.phys.ethz.ch/\xmptilde nbeisert/}}

\newcommand{\secref}[1]{\hyperref[#1]{section \ref*{#1}}}

\parskip1ex
\parindent0pt
\let\olditemize\itemize
\def\itemize{\olditemize\parskip0pt}

\begin{document}

\title{The \textsf{childdoc} Package}
\hypersetup{pdftitle={The childdoc Package}}
\author{Niklas Beisert\\[2ex]
  Institut f\"ur Theoretische Physik\\
  Eidgen\"ossische Technische Hochschule Z\"urich\\
  Wolfgang-Pauli-Strasse 27, 8093 Z\"urich, Switzerland\\[1ex]
  \href{mailto:nbeisert@itp.phys.ethz.ch}
  {\texttt{nbeisert@itp.phys.ethz.ch}}}
\hypersetup{pdfauthor={Niklas Beisert}}
\hypersetup{pdfsubject={Manual for the LaTeX2e Package childdoc}}
\date{30 December 2018, \textsf{v2.0}}
\maketitle

\begin{abstract}\noindent
\textsf{childdoc} is a \LaTeXe{} package
that enables the direct compilation
of document sections included by |\include|
to individual files.
\end{abstract}

\begingroup
\parskip0ex
\tableofcontents
\endgroup

%%%%%%%%%%%%%%%%%%%%%%%%%%%%%%%%%%%%%%%%%%%%%%%%%%%%%%%%%%%%%%%%%%%%%%%%%%%%%%%%
%%%%%%%%%%%%%%%%%%%%%%%%%%%%%%%%%%%%%%%%%%%%%%%%%%%%%%%%%%%%%%%%%%%%%%%%%%%%%%%%
\section{Introduction}

\LaTeX{} provides a mechanism to structure a large document (such as a book)
into a main file and several child files (containing the chapters)
using the |\include| command.
This mechanism is beneficial for documents
which span hundreds of pages in order to
make the source file(s) more manageable.
Moreover, compilation can be restricted to
selected child files by means of the |\includeonly| command.
The latter feature can be used to reduce the compilation time while editing
(this was significantly more useful in the earlier days of \LaTeX{})
or to generate a smaller document which is easier to navigate.
Another application of |\includeonly| is to generate
documents consisting of selected parts of the complete document.

However, there are a few drawbacks of the plain |\include| mechanism:
\begin{itemize}
\item
The child files cannot be compiled on their own,
they can only be compiled via the main file.
A naive editing environment
(such as a text editor with an option
to have the current file processed by \LaTeX)
may require one to switch to the main file before compiling;
attempting to compile the child file produces errors.
\item
The main file must be modified (each time)
to adjust the |\includeonly| command
to the present needs. This easily leaves the main file in a messy state.
\item
The generated document will always carry the filename
of the main document. This is inconvenient if
several child files are to be compiled and
to be kept for distribution.
\end{itemize}

The present package provides a simple interface
to make child files individually compilable by \LaTeX{}.
Compiling a child file then has the same effect as compiling
the main file with an |\includeonly| command
to select the appropriate child.
Moreover the generated document will carry the name of the child
rather than the main file.
This resolves all three above issues.

This feature is meant to make the editing of books,
thesis documents and lecture notes somewhat more convenient.
However, the package can also be used efficiently for
composing a series of documents (such as exercise sheets)
which are typically distributed individually.
It then assists the author in generating the individual documents
(potentially in different versions)
as well as a document containing the collected series.
Another application is in developing style files
or other kinds of included material
where compilation of the style file could redirect
to a sample or test file.

%%%%%%%%%%%%%%%%%%%%%%%%%%%%%%%%%%%%%%%%%%%%%%%%%%%%%%%%%%%%%%%%%%%%%%%%%%%%%%%%
%%%%%%%%%%%%%%%%%%%%%%%%%%%%%%%%%%%%%%%%%%%%%%%%%%%%%%%%%%%%%%%%%%%%%%%%%%%%%%%%
\section{Usage}

First of all, the package \textsf{childdoc} is \emph{not} a standard
\LaTeXe{} |.sty| style file! Therefore it needs to be invoked in
a non-standard way.

%%%%%%%%%%%%%%%%%%%%%%%%%%%%%%%%%%%%%%%%%%%%%%%%%%%%%%%%%%%%%%%%%%%%%%%%%%%%%%%%
\subsection{Included Files}
\label{sec:include}

%%%%%%%%%%%%%%%%%%%%%%%%%%%%%%%%%%%%%%%%
\DescribeMacro{\childdocmain}
To use the package, add the commands
\begin{center}
\begin{tabular}{l}
|\input{childdoc.def}|\\
|\childdocmain{}|\\
\end{tabular}
\end{center}
at the very top of the main \LaTeX{} file,
in particular \emph{before} the |\documentclass| statement!
The argument of |\childdocmain| should be left empty
(but it must be present).

%%%%%%%%%%%%%%%%%%%%%%%%%%%%%%%%%%%%%%%%
\DescribeMacro{\childdocof}
Furthermore, add the commands
\begin{center}
\begin{tabular}{l}
|\input{childdoc.def}|\\
|\childdocof{|\textit{main}|}|\\
\end{tabular}
\end{center}
at the top of every child file \textit{child}
which is included by |\include{|\textit{child}|}|
from within the main file
(or at least for those files to be compiled individually).
The argument \textit{main} must be the filename of the main file.

There are a couple of
considerations in setting up the main and child documents:

%%%%%%%%%%%%%%%%%%%%%%%%%%%%%%%%%%%%%%%%
\paragraph{Restrictions.}

Please note the following restrictions:
\begin{itemize}
\item
|\childdocmain| must be called with one argument \textit{main}
to ensure compatibility with earlier version of the package.
It must either be empty (|\childdocmain{}|)
or precisely match the filename of the main file in which it is specified.
See \secref{sec:detection} for further information.
\item
The filename \textit{main} must be specified without the |.tex| extension.
\item
The filename \textit{main} is case sensitive
(even in case-insensitive file systems)
due to internal string comparison.
\item
The argument \textit{main} should be fully expanded, it cannot be a macro.
\item
Subdirectories and special characters should be avoided in filenames.
\item
The command |\childdocmain{|\textit{main}|}| must be followed by a whitespace.
It should not be followed immediately by another command
or by a comment mark `|%|'.
This is because the \TeX{} parser reads the token immediately following
the argument of |\childdocmain| and puts it
at the beginning of every child section;
however, a white\-space is ignored.
\end{itemize}

%%%%%%%%%%%%%%%%%%%%%%%%%%%%%%%%%%%%%%%%
\paragraph{Content of Main File.}

It is advisable to place all content in the child files included by |\include|.
Any output contained in the main file will appear in all child documents
unless suppressed manually;
it cannot be suppressed automatically by the |\includeonly| directive
and thus should normally be avoided.
A method to include some content in the main file
by means of conditional processing is described in \secref{sec:conditional}.

%%%%%%%%%%%%%%%%%%%%%%%%%%%%%%%%%%%%%%%%
\paragraph{Page Numbering.}

When only a part of the document is compiled,
the appropriate numbering of pages
(as well as other status parameters)
is determined from the |.aux| files.
The latter contain information from previous passes.
However this information needs to propagate through
all intermediate child documents.
Therefore the page numbering in child documents may well
be inconsistent until the complete document is compiled at least once.

A useful (if unconventional) way to always ensure a consistent
page numbering is to restart the numbering in each child document
and denote the pages by `\textit{child}|.|\textit{page}'
where \textit{child} represents the chapter/section number of the child file.
This can be achieved by the command
|\numberwithin{page}{|\textit{child}|}|
of the \textsf{amsmath} package
where \textit{child} can be |chapter| or |section|
depending on the chosen structuring.
Alternatively, one can modify the macro |\thepage| appropriately
and reset the counter |page| at the start of each child file.

%%%%%%%%%%%%%%%%%%%%%%%%%%%%%%%%%%%%%%%%%%%%%%%%%%%%%%%%%%%%%%%%%%%%%%%%%%%%%%%%
\subsection{Conditional Processing}
\label{sec:conditional}

The package provides a mechanism to compile different versions
of a document. To customise the versions further some conditional processing
can come in handy to distinguish which version is being compiled.
The package provides two macros to describe the compilation context:

%%%%%%%%%%%%%%%%%%%%%%%%%%%%%%%%%%%%%%%%
\DescribeMacro{\ifchilddoc}
The conditional |\ifchilddoc| distinguishes between the compilation of
child documents and the main document:
%
\begin{center}
|\ifchilddoc |\textit{child-code}| |[|\||else |\textit{main-code}]| \||fi|
\end{center}

%%%%%%%%%%%%%%%%%%%%%%%%%%%%%%%%%%%%%%%%
\DescribeMacro{\childdocname}
\DescribeMacro{\childdocjob}
The macro |\childdocname| contains the filename (without extension)
of the main or child file being processed.
Note that |\childdocjob| will always contain the name of the main file.

%%%%%%%%%%%%%%%%%%%%%%%%%%%%%%%%%%%%%%%%
\paragraph{Title Page.}

Conditional processing can be used to include a title or banner page
in the main document when proper precautions are taken.
Importantly, the code in the main file should ensure that the page counter
(as well as other status parameters which are stored in the |.aux| files)
takes the same value after the conditional processing.
Otherwise the page numbers may take divergent values
depending on which part is compiled.

For example, a title page could be declared by:
%
\begin{center}
\begin{tabular}{l}
|\ifchilddoc\||else|\\
|\addtocounter{page}{-1}|\\
\textit{code for title page}\\
|\newpage|\\
|\||fi|
\end{tabular}
\end{center}
%
A banner page for the child documents can be generated by:
%
\begin{center}
\begin{tabular}{l}
|\ifchilddoc|\\
|\addtocounter{page}{-1}|\\
\textit{code for banner page}\\
|\newpage|\\
|\||fi|
\end{tabular}
\end{center}
%
Here one could write a message such as:
\begin{center}
|This is the part \childdocname{} of \childdocjob{}.|
\end{center}

%%%%%%%%%%%%%%%%%%%%%%%%%%%%%%%%%%%%%%%%%%%%%%%%%%%%%%%%%%%%%%%%%%%%%%%%%%%%%%%%
\subsection{Flags}
\label{sec:flags}

The package makes it easy to generate different versions
of the main or child documents.
To this end compilation flags can be defined
and assigned different default values.
They will be particularly useful in conjunction
with the forwarding mechanism described in \secref{sec:forward}.

For example, it may be useful to have a flag |\version|
which can be set to |draft| or |final|.
The document source will contain some conditional code
depending on the value of |\version|.
Suppose further, the flag should default to |final| for the main file
and to |draft| for child files
which is a natural assignment for editing the document.
This is achieved by placing the following code
in the preamble of the main document
(below the |\childdocmain| directive):
%
\begin{center}
\begin{tabular}{l}
|\ifchilddoc|\\
|\providecommand{\version}{draft}|\\
|\||else|\\
|\providecommand{\version}{final}|\\
|\||fi|
\end{tabular}
\end{center}
%
The definition by |\providecommand| makes sure
that previous definitions are not overwritten.
Further statements |\providecommand{\version}{...}|
can thus be added before the above code to override it.

For the main file, one might add a line
(between |\childdocmain| and the above block)
%
\begin{center}
|%\ifchilddoc\||else\providecommand{\version}{draft}\||fi|
\end{center}
%
which can be uncommented to produce a draft version.
Likewise one can add a line to the very top of a child file
(above the |\childdocof{|\textit{main}|}| directive)
%
\begin{center}
|%\providecommand{\version}{final}|
\end{center}
%
which can be uncommented to produce the final version of this child document.

%%%%%%%%%%%%%%%%%%%%%%%%%%%%%%%%%%%%%%%%%%%%%%%%%%%%%%%%%%%%%%%%%%%%%%%%%%%%%%%%
\subsection{Forwarding}
\label{sec:forward}

Different versions of the main or child documents
using compilation flags as described in \secref{sec:flags}
can be (permanently) stored in different files
for convenient compilation, viewing and distribution.
To this end, the package defines a command
to pass on compilation to a different file:

%%%%%%%%%%%%%%%%%%%%%%%%%%%%%%%%%%%%%%%%
\DescribeMacro{\childdocforward}
The command |\childdocforward| redirects processing to
another source file:
%
\begin{center}
\begin{tabular}{l}
|\input{childdoc.def}|\\
|\childdocforward[|\textit{main}|]{|\textit{dest}|}|\\
\end{tabular}
\end{center}
%
The argument \textit{dest} is the destination file
(without extension).
It should be the main file or one of the child files.
Note that further \textsf{childdoc} directives
such as |\childdocof| and |\childdocforward|
in the indicated file will be processed in this form.
The optional argument \textit{main}
passes on directly to the main file \textit{main}
while pretending to compile the child \textit{dest}.
This form behaves as if \textit{dest}
issues |\childdocof{|\textit{main}|}| right away,
and no further \textsf{childdoc} directives will be processed.

%%%%%%%%%%%%%%%%%%%%%%%%%%%%%%%%%%%%%%%%
\DescribeMacro{\...prefix}
In the alternative form |\childdocforwardprefix|,
%
\begin{center}
\begin{tabular}{l}
|\input{childdoc.def}|\\
|\childdocforwardprefix[|\textit{main}|]{|\textit{prefix}|}{|\textit{dest}|}|
\end{tabular}
\end{center}
%
the destination file is determined by a pattern
depending on the current file:
To make this work, the current file must be called
`{\textit{prefix}\hspace{0.2em}\textit{suffix}}'
with \textit{prefix} matching precisely the argument.
Processing is then passed on to the file
`{\textit{dest}\hspace{0.2em}\textit{suffix}}'.
Surely, the same effect is achieved by
directly specifying the
argument `{\textit{dest}\hspace{0.2em}\textit{suffix}}'
in the first form.
However, that requires to set up a different file
for each child. With the alternative form of the command
all these files can have exactly the same content
which simplifies setting them up and maintaining them.

For example, the following file |draft.tex|
with a compilation flag |\version| as described in \secref{sec:flags}
compiles the main document as a draft:
%
\begin{center}
\begin{tabular}{l}
|\def\version{draft}|\\
|\input{childdoc.def}|\\
|\childdocforward{|\textit{main}|}|
\end{tabular}
\end{center}
%
Likewise, the following files |final|\textit{nn}|.tex|
compile the final version of the child document
|child|\textit{nn}|.tex|:
%
\begin{center}
\begin{tabular}{l}
|\def\version{final}|\\
|\input{childdoc.def}|\\
|\childdocforwardprefix{final}{child}|
\end{tabular}
\end{center}
%

Note that when several versions of a main file and/or of each child file
are to be generated, it may be convenient to set up a |Makefile| or
shell script to automatise the process.

%%%%%%%%%%%%%%%%%%%%%%%%%%%%%%%%%%%%%%%%%%%%%%%%%%%%%%%%%%%%%%%%%%%%%%%%%%%%%%%%
\subsection{Command Line Processing}
\label{sec:commandline}

The effect of redirection files can also be achieved by invoking
the \LaTeX{} compiler with a more elaborate command line.
Most conveniently this should be done as part
of a shell script or a |Makefile|.

When using \textsf{childdoc} in the main file, the following
command lines effectively perform a redirection
(note that depending on the shell being used,
backslashes may have to be doubled: `|\|' $\to$ `|\\|'):
%
\begin{center}
|... -jobname "|\textit{target}|" |\\|"|[\textit{flags}]%
|\input{childdoc.def}\childdocforward[|\textit{main}|]{|\textit{dest}|}"|
\end{center}
%
Here \textit{target} is the name of the output file,
\textit{main} is the name of the main file
and \textit{dest} is the name of the main or child file to be processed
(all filenames without extensions).
The optional argument \textit{main} can be omitted
if \textit{main} matches \textit{dest}.
Optionally, compilation \textit{flags} can be defined via |\def| commands.
This command line makes the \TeX{} engine believe
it is compiling the file \textit{target}
whose content is specified as the latter parameter.
The provided code then forwards the processing to
\textit{main} or \textit{dest} as described in \secref{sec:forward}.

%%%%%%%%%%%%%%%%%%%%%%%%%%%%%%%%%%%%%%%%%%%%%%%%%%%%%%%%%%%%%%%%%%%%%%%%%%%%%%%%
\subsection{Include by Input}
\label{sec:input}

Including child documents by |\include| has some restrictions by design.
Most notably, the content of a child document always occupies
its own set of pages; pages cannot be shared between child documents.
Usually, this behaviour makes perfect sense
because each child document contain an essential part of the document.
However, in some situations it may be desirable to compose
a document from a collection of parts
without having mandatory page breaks between then.
For this case, the package
provides a mechanism to include parts
by |\input| which can also be processed individually.
However, by construction this mechanism
requires manual handling of the content to be output.

%%%%%%%%%%%%%%%%%%%%%%%%%%%%%%%%%%%%%%%%
\DescribeMacro{\ifchilddocmanual}
The main file should be prepared as usual, see \secref{sec:include}.
However, the document body must make a distinction
between processing of an individual part and of the main document, e.g.:
%
\begin{center}
\begin{tabular}{l}
|\ifchilddocmanual|\\
|\input{\childdocname}|\\
|\||else|\\
\textit{document body with }|\input{|\textit{part}|}|\\
|\||fi|
\end{tabular}
\end{center}
%
The conditional |\ifchilddocmanual| is true whenever
a part to be included by |\input| is being compiled,
and the name of the part is stored in |\childdocname|.

%%%%%%%%%%%%%%%%%%%%%%%%%%%%%%%%%%%%%%%%
\DescribeMacro{\childdocby}
Each part to be included by |\input| should start with:
%
\begin{center}
\begin{tabular}{l}
|\input{childdoc.def}|\\
|\childdocby{|\textit{main}|}|\\
\end{tabular}
\end{center}
%
The directive |\childdocby| is similar to |\childdocof|
described in \secref{sec:include},
but the subsequent selection of content must be done manually.
To that end, both |\ifchilddoc| and |\ifchilddocmanual|
will be true upon processing of a part,
and the name of the part is stored in |\childdocname|.
Note that |\jobname| will be set to the filename of the current part
so that each part receives an individual |.aux| file
that does not interfere with the |.aux| file(s) of the main document.
This behaviour can be altered by the alternative form
|\childdocby[*]{|\textit{main}|}| (with a non-empty optional argument)
which uses the |.aux| file of the main document
by setting |\jobname| to \textit{main}.

%%%%%%%%%%%%%%%%%%%%%%%%%%%%%%%%%%%%%%%%%%%%%%%%%%%%%%%%%%%%%%%%%%%%%%%%%%%%%%%%
\subsection{Driver Development}
\label{sec:driver}

The \textsf{childdoc} mechanism can also be use for the development
of definition files such as \LaTeX{} styles or classes.
This case differs from the above setup with multiple parts
included by |\include| in that no |\includeonly| should be invoked.
This can be achieved by starting the include file
(before |\ProvidesPackage|) with:
%
\begin{center}
\begin{tabular}{l}
|\input{childdoc.def}|\\
|\childdocforward{|\textit{main}|}|\\
\end{tabular}
\end{center}
%
or alternatively with:
%
\begin{center}
\begin{tabular}{l}
|\input{childdoc.def}|\\
|\childdocby{|\textit{main}|}|\\
\end{tabular}
\end{center}
%
Both forms have slightly different effects as described above.
The main file is prepared as usual, see \secref{sec:include}.

%%%%%%%%%%%%%%%%%%%%%%%%%%%%%%%%%%%%%%%%%%%%%%%%%%%%%%%%%%%%%%%%%%%%%%%%%%%%%%%%
\subsection{Legacy Detection}
\label{sec:detection}

The directive |\childdocmain| in the main file can detect
whether the complete document or merely a child is to be compiled
even without using the directive |\childdocof|.
This method is deprecated because it is less robust
and there is no compelling reason to use it;
it is merely provided for backward compatibility
and it may be removed in future versions.

If the detection mechanism is to be used,
it is mandatory to correctly specify
the filename of the main file as the argument of |\childdocmain|:
%
\begin{center}
\begin{tabular}{l}
|\input{childdoc.def}|\\
|\childdocmain{|\textit{main}|}|\\
\end{tabular}
\end{center}
%
If |\jobname| does not match the argument \textit{main} of |\childdocmain|,
it is assumed that |\jobname| points to the child file to be compiled.
When using |\childdocmain| with the main file specified as argument,
it suffices to start a child file
with just |\input{|\textit{main}|}|
without loading of the package and using |\childdocof|.
If instead all processing is done
with the appropriate \textsf{childdoc} directives,
the argument of \textit{main} of |\childdocmain| can be empty.

An alternative version of the command line processing described
in \secref{sec:commandline} using the detection mechanism reads:
%
\begin{center}
|... -jobname "|\textit{target}|" "|[\textit{flags}]%
[|\def\jobname{|\textit{dest}|}|]|\input{|\textit{main}|}"|
\end{center}

%%%%%%%%%%%%%%%%%%%%%%%%%%%%%%%%%%%%%%%%%%%%%%%%%%%%%%%%%%%%%%%%%%%%%%%%%%%%%%%%
\subsection{Manual Code}
\label{sec:manual}

In case one cannot be certain whether the definitions file |childdoc.def|
is installed on the target \TeX{} distribution
and one prefers not to ship it,
it is conceivable to paste a few relevant commands into the sources.

To that end, drop all statements |\input{childdoc.def}|
and perform the replacements as outlined below.
Instead of |\childdocmain{|\textit{main}|}| add the following code
to the top of the main file:
%
\begin{center}
\begin{tabular}{l}
|\||ifdefined\childdocname\endinput\||fi\newif\ifchilddoc|\\
|\edef\childdocname{\scantokens\expandafter{\jobname\noexpand}}|\\
|\def\childdocmain{|\textit{main}|}\||ifx\childdocmain\childdocname\||else|\\
|\childdoctrue\includeonly{\childdocname}\let\jobname\childdocmain\||fi|\\
\end{tabular}
\end{center}
%
Instead of |\childdocof{|\textit{main}|}| just include the main file
at the top of each child file:
%
\begin{center}
|\input{|\textit{main}|}|
\end{center}
%
A simple redirection |\childdocforward{|\textit{dest}|}| is achieved by:
%
\begin{center}
|\def\jobname{|\textit{dest}|}\input{\jobname}|
\end{center}
%
The redirection with prefix
|\childdocforwardprefix[|\textit{prefix}|]{|\textit{dest}|}|
is accomplished by:
%
\begin{center}
\begin{tabular}{l}
|{\edef\jobname{\scantokens\expandafter{\jobname\noexpand}}|\\
|\def\redirectjob |\textit{prefix}|#1~~~{\gdef\jobname{|\textit{dest}|#1}}|\\
|\expandafter\redirectjob\jobname~~~}\input{\jobname}|
\end{tabular}
\end{center}

In an alternative approach,
child documents can be compiled by a specific command line
without additional code or specific definitions:
%
\begin{center}
|... -jobname "|\textit{target}|" "|[\textit{flags}]%
|\includeonly{|\textit{dest}|}\input{|\textit{main}|}"|
\end{center}
%

%%%%%%%%%%%%%%%%%%%%%%%%%%%%%%%%%%%%%%%%%%%%%%%%%%%%%%%%%%%%%%%%%%%%%%%%%%%%%%%%
%%%%%%%%%%%%%%%%%%%%%%%%%%%%%%%%%%%%%%%%%%%%%%%%%%%%%%%%%%%%%%%%%%%%%%%%%%%%%%%%
\section{Information}

%%%%%%%%%%%%%%%%%%%%%%%%%%%%%%%%%%%%%%%%%%%%%%%%%%%%%%%%%%%%%%%%%%%%%%%%%%%%%%%%
\subsection{Copyright}

Copyright \copyright{} 2017--2018 Niklas Beisert

This work may be distributed and/or modified under the
conditions of the \LaTeX{} Project Public License, either version 1.3
of this license or (at your option) any later version.
The latest version of this license is in
  \url{http://www.latex-project.org/lppl.txt}
and version 1.3 or later is part of all distributions of \LaTeX{}
version 2005/12/01 or later.

This work has the LPPL maintenance status `maintained'.

The Current Maintainer of this work is Niklas Beisert.

This work consists of the files |README.txt|, |childdoc.ins| and |childdoc.dtx|
as well as the derived files |childdoc.def|, |cdocsamp.tex|
with |cdocsch1.tex|, |cdocsch2.tex|, |cdocspt3.tex|, |cdocspt4.tex|,
|cdocsdrf.tex|, |cdocsfn1.tex|, |cdocsfn2.tex|
as well as |childdoc.pdf|.

%%%%%%%%%%%%%%%%%%%%%%%%%%%%%%%%%%%%%%%%%%%%%%%%%%%%%%%%%%%%%%%%%%%%%%%%%%%%%%%%
\subsection{Files and Installation}

The package consists of the files:
%
\begin{center}
\begin{tabular}{ll}
    |README.txt|   & readme file \\
    |childdoc.ins| & installation file \\
    |childdoc.dtx| & source file \\
    |childdoc.def| & definition file \\
    |cdocsamp.tex| & sample main file \\
    |cdocsch1.tex| & sample include file \\
    |cdocsch2.tex| & sample include file \\
    |cdocspt3.tex| & sample part file \\
    |cdocspt4.tex| & sample part file \\
    |cdocsdrf.tex| & sample redirection file \\
    |cdocsfn1.tex| & sample redirection file \\
    |cdocsfn2.tex| & sample redirection file \\
    |childdoc.pdf| & manual
\end{tabular}
\end{center}
%
The distribution consists of the files
|README.txt|, |childdoc.ins| and |childdoc.dtx|.
%
\begin{itemize}
\item
Run (pdf)\LaTeX{} on |childdoc.dtx|
to compile the manual |childdoc.pdf| (this file).
\item
Run \LaTeX{} on |childdoc.ins| to create the definitions file |childdoc.def|
and the sample |cdocsamp.tex| with include files
|cdocsch1.tex|, |cdocsch2.tex|, |cdocspt3.tex|, |cdocspt4.tex|,
|cdocsdrf.tex|, |cdocsfn1.tex|, |cdocsfn2.tex|.
Then copy the file |childdoc.def| to an appropriate directory of your \LaTeX{}
distribution, e.g.\ \textit{texmf-root}|/tex/latex/childdoc|.
\end{itemize}

%%%%%%%%%%%%%%%%%%%%%%%%%%%%%%%%%%%%%%%%%%%%%%%%%%%%%%%%%%%%%%%%%%%%%%%%%%%%%%%%
\subsection{Related CTAN Packages}

There are several other packages which offer a similar functionality:
%
\begin{itemize}
\item
The packages
\href{http://ctan.org/pkg/docmute}{\textsf{docmute}},
\href{http://ctan.org/pkg/includex}{\textsf{includex}} and
\href{http://ctan.org/pkg/standalone}{\textsf{standalone}}
provide commands to include only the document body of
a child file thus allowing both files to be compiled individually.
\item
The packages \href{http://ctan.org/pkg/subdocs}{\textsf{subdocs}}
and \href{http://ctan.org/pkg/subfiles}{\textsf{subfiles}}
provide structures in which the main and child documents can be
encapsulated and allowing them to be compiled individually.
The inclusion mechanism is different from the conventional |\include|.
\item
The package \href{http://ctan.org/pkg/combine}{\textsf{combine}}
is an elaborate solution to combine several documents into one.
\end{itemize}
%
See also the CTAN topic \href{http://ctan.org/topic/subdocs}{\textsf{subdocs}}
for further related packages.
The present package differs from the above solutions in that
a document structure constructed with the conventional |\include| mechanism
just needs two extra commands at the top of every file
such that all constituent files can be compiled individually.

%%%%%%%%%%%%%%%%%%%%%%%%%%%%%%%%%%%%%%%%%%%%%%%%%%%%%%%%%%%%%%%%%%%%%%%%%%%%%%%%
%\subsection{Feature Suggestions}
%
%The following is a list of features which may be useful for future
%versions of this package:
%%
%\begin{itemize}
%\item
%\ldots
%\end{itemize}

%%%%%%%%%%%%%%%%%%%%%%%%%%%%%%%%%%%%%%%%%%%%%%%%%%%%%%%%%%%%%%%%%%%%%%%%%%%%%%%%
\subsection{Revision History}

%%%%%%%%%%%%%%%%%%%%%%%%%%%%%%%%%%%%%%%%
\paragraph{v2.0:} 2018/12/30

\begin{itemize}
\item
immediate forward processing
\item
added |\childdocby| mechanism
\item
manual restructured
\end{itemize}

%%%%%%%%%%%%%%%%%%%%%%%%%%%%%%%%%%%%%%%%
\paragraph{v1.6:} 2018/01/17

\begin{itemize}
\item
application for development of include files
\item
corrections to manual
\end{itemize}

%%%%%%%%%%%%%%%%%%%%%%%%%%%%%%%%%%%%%%%%
\paragraph{v1.5:} 2017/05/21

\begin{itemize}
\item
more complete structuring introduced
\item
|\childdocof| introduced
\item
|\childdoc| renamed to |\childdocmain|
\item
|\childredirect| renamed to |\childdocforward| and |\childdocforwardprefix|
and functionality expanded
\end{itemize}

%%%%%%%%%%%%%%%%%%%%%%%%%%%%%%%%%%%%%%%%
\paragraph{v1.0:} 2017/04/27

\begin{itemize}
\item
manual and install package
\item
first version published on CTAN
\end{itemize}

%%%%%%%%%%%%%%%%%%%%%%%%%%%%%%%%%%%%%%%%
\paragraph{v0.6:} 2017/04/26

\begin{itemize}
\item
redirection mechanism added
\end{itemize}

%%%%%%%%%%%%%%%%%%%%%%%%%%%%%%%%%%%%%%%%
\paragraph{v0.5:} 2017/04/26

\begin{itemize}
\item
functionality in definition file
\end{itemize}


%%%%%%%%%%%%%%%%%%%%%%%%%%%%%%%%%%%%%%%%%%%%%%%%%%%%%%%%%%%%%%%%%%%%%%%%%%%%%%%%
%%%%%%%%%%%%%%%%%%%%%%%%%%%%%%%%%%%%%%%%%%%%%%%%%%%%%%%%%%%%%%%%%%%%%%%%%%%%%%%%
%%%%%%%%%%%%%%%%%%%%%%%%%%%%%%%%%%%%%%%%%%%%%%%%%%%%%%%%%%%%%%%%%%%%%%%%%%%%%%%%
\appendix

\settowidth\MacroIndent{\rmfamily\scriptsize 000\ }

 \DocInput{childdoc.dtx}

\end{document}
%</driver>
% \fi
%
% %%%%%%%%%%%%%%%%%%%%%%%%%%%%%%%%%%%%%%%%%%%%%%%%%%%%%%%%%%%%%%%%%%%%%%%%%%%%%%
% %%%%%%%%%%%%%%%%%%%%%%%%%%%%%%%%%%%%%%%%%%%%%%%%%%%%%%%%%%%%%%%%%%%%%%%%%%%%%%
% \section{Sample}
%\iffalse
%<*samplemain>
%\fi
%
% The following presents a sample document
% with two chapters, two parts, a title page,
% a compile flag as well as three forwarding files to set the flag.
% It consists of eight |.tex| files:
% \begin{center}
% \begin{tabular}{ll}
% |cdocsamp.tex|&main file\\
% |cdocsch1.tex|&include file for chapter 1\\
% |cdocsch2.tex|&include file for chapter 2\\
% |cdocspt3.tex|&include file for part 3\\
% |cdocspt4.tex|&include file for part 4\\
% |cdocsdrf.tex|&forwarding file for main file in draft mode\\
% |cdocsfi1.tex|&forwarding file for final version of chapter 1\\
% |cdocsfi2.tex|&forwarding file for final version of chapter 2\\
% \end{tabular}
% \end{center}
% Each of the eight files can be compiled directly by the \LaTeX{} compiler.
%
% %%%%%%%%%%%%%%%%%%%%%%%%%%%%%%%%%%%%%%
% \paragraph{Main File.}
%
% The main file is called |cdocsamp.tex|.
%
% Load the \textsf{childdoc} definitions and
% declare the filename for the main document:
%    \begin{macrocode}
\input{childdoc.def}
\childdocmain{}
%    \end{macrocode}

% Optional override for |\version| flag:
%    \begin{macrocode}
%%\ifchilddoc\else\providecommand{\version}{draft}\fi
%    \end{macrocode}

% Define the default values for the |\version| flag
% (|final| for the main file and |draft| for childs):
%    \begin{macrocode}
\ifchilddoc
\providecommand{\version}{draft}
\else
\providecommand{\version}{final}
\fi
%    \end{macrocode}

% Load the standard document class:
%    \begin{macrocode}
\documentclass[12pt]{article}
%    \end{macrocode}

% Start the document body:
%    \begin{macrocode}
\begin{document}
%    \end{macrocode}

% Declare a title page.
% Print title, part of document being processed and version flag:
%    \begin{macrocode}
\addtocounter{page}{-1}
\begin{center}
{\LARGE\bfseries{}childdoc example\par}
\vspace{1cm}
\ifchilddoc
\ifchilddocmanual part\else chapter\fi:
`\childdocname' of `\childdocjob'\par
\else
main document: `\childdocjob'\par
\fi
version: \version\par
\end{center}
\newpage
%    \end{macrocode}

% Manually include selected file,
% otherwise process as usual:
%    \begin{macrocode}
\ifchilddocmanual
\section*{part `\childdocname'}
\input{\childdocname}
\else
%    \end{macrocode}

% Include the two chapters:
%    \begin{macrocode}
\include{cdocsch1}
\include{cdocsch2}
%    \end{macrocode}

% Include the two parts unless only chapters should be displayed:
%    \begin{macrocode}
\ifchilddoc\else
\section{part three}
\input{cdocspt3}
\section{part four}
\input{cdocspt4}
\fi
%    \end{macrocode}

% Process as usual until here:
%    \begin{macrocode}
\fi
%    \end{macrocode}

% End of document body:
%    \begin{macrocode}
\end{document}
%    \end{macrocode}
%\iffalse
%</samplemain>
%\fi
%
% %%%%%%%%%%%%%%%%%%%%%%%%%%%%%%%%%%%%%%
% \paragraph{Chapter Include Files.}
%
% The include files are called |cdocsch1.tex| and |cdocsch2.tex|.
%
%\iffalse
%<*samplechap1|samplechap2>
%\fi

% Optional override for |\version| flag:
%    \begin{macrocode}
%%\providecommand{\version}{final}
%    \end{macrocode}

% Include the main document:
%    \begin{macrocode}
\input{childdoc.def}
\childdocof{cdocsamp}
%    \end{macrocode}

%\iffalse
%</samplechap1|samplechap2>
%\fi
%
%\iffalse
%<*samplechap1>
%\fi
% Some text for chapter 1:
%    \begin{macrocode}
\section{one}
some text in chapter one
%    \end{macrocode}

%\iffalse
%</samplechap1>
%\fi
% Some text for chapter 2:
%\iffalse
%<*samplechap2>
%\fi
%    \begin{macrocode}
\section{two}
more text in chapter two
%    \end{macrocode}

%\iffalse
%</samplechap2>
%\fi
%
% %%%%%%%%%%%%%%%%%%%%%%%%%%%%%%%%%%%%%%
% \paragraph{Part Include Files.}
%
% The include files are called |cdocspt3.tex| and |cdocspt4.tex|.
%
%\iffalse
%<*samplepart3|samplepart4>
%\fi

% Optional override for |\version| flag:
%    \begin{macrocode}
%%\providecommand{\version}{final}
%    \end{macrocode}

% Include the main document:
%    \begin{macrocode}
\input{childdoc.def}
\childdocby{cdocsamp}
%    \end{macrocode}

%\iffalse
%</samplepart3|samplepart4>
%\fi
%
%\iffalse
%<*samplepart3>
%\fi
% Some text for part 3:
%    \begin{macrocode}
some text in part three
%    \end{macrocode}

%\iffalse
%</samplepart3>
%\fi
% Some text for part 4:
%\iffalse
%<*samplepart4>
%\fi
%    \begin{macrocode}
more text in part four
%    \end{macrocode}

%\iffalse
%</samplepart4>
%\fi
%
% %%%%%%%%%%%%%%%%%%%%%%%%%%%%%%%%%%%%%%
% \paragraph{Forwarding for a Complete Draft.}
%
% The following forwarding file |cdocsdrf.tex|
% compiles the main document in draft mode:
%\iffalse
%<*sampledraft>
%\fi
%    \begin{macrocode}
\def\version{draft}
\input{childdoc.def}
\childdocforward{cdocsamp}
%    \end{macrocode}

%\iffalse
%</sampledraft>
%\fi
%
% %%%%%%%%%%%%%%%%%%%%%%%%%%%%%%%%%%%%%%
% \paragraph{Forwarding for Final Version of the Chapters.}
%
% The following forwarding files |cdocsfn1.tex| and |cdocsfn2.tex|
% (with identical content)
% compile the final versions of the child documents
% |cdocsch1.tex| and |cdocsch2.tex|, respectively:
%\iffalse
%<*samplefinal>
%\fi
%    \begin{macrocode}
\def\version{final}
\input{childdoc.def}
\childdocforwardprefix[cdocsamp]{cdocsfn}{cdocsch}
%    \end{macrocode}

%\iffalse
%</samplefinal>
%\fi
%
% %%%%%%%%%%%%%%%%%%%%%%%%%%%%%%%%%%%%%%
% \paragraph{Command Line Processing.}
%
% The following three command lines generate the output files
% |cdocscld|, |cdocscl1| and |cdocscl2|
% which should be identical to
% |cdocsdrf|, |cdocsch1| and |cdocsfn2|, respectively:
% \begin{center}
% \begin{tabular}{l}
% |latex -jobname cdocscld \|\\
% |  "\def\version{draft}\input{childdoc.def}\childdocforward{cdocsamp}"|\\
% |latex -jobname cdocscl1 \|\\
% |  "\input{childdoc.def}\childdocforward[cdocsamp]{cdocsch1}"|\\
% |latex -jobname cdocscl2 \|\\
% |  "\def\version{final}\input{childdoc.def}\childdocforward{cdocsch2}"|
% \end{tabular}
% \end{center}
% Note that the trailing backslash on each first line
% merely continues the input to the second line
% (for convenient cut ant paste).
% Furthermore, the command |latex| can be replaced by any
% of its alternative versions such as |pdflatex|.
%
% %%%%%%%%%%%%%%%%%%%%%%%%%%%%%%%%%%%%%%%%%%%%%%%%%%%%%%%%%%%%%%%%%%%%%%%%%%%%%%
% %%%%%%%%%%%%%%%%%%%%%%%%%%%%%%%%%%%%%%%%%%%%%%%%%%%%%%%%%%%%%%%%%%%%%%%%%%%%%%
% \section{Implementation}
%\iffalse
%<*package>
%\fi
%
% This section describes the definitions file |childdoc.def|.

% The definitions cannot be loaded using |\usepackage| or |\RequirePackage|
% which has a mechanism to prevent loading a style file more than once.
% When loading the definitions by means of |\input|
% multiple instances have to be prevented manually:
%\iffalse
%This code needs to be before the `\ProvidesFile' directive
%which is defined at the beginning of this file.
%Therefore it is also placed there and commented out here.
%</package>
%<*discard>
%\fi
%    \begin{macrocode}
\ifdefined\childdocmain\endinput\fi
%    \end{macrocode}
%\iffalse
%</discard>
%<*package>
%\fi
%
% \macro{\ifchilddoc}
% \macro{\ifchilddocmanual}
% The conditional |\ifchilddoc| tells whether a
% child (true) or main (false) document is being compiled.
% The conditional |\ifchilddocmanual| tells whether
% the |\includeonly| mechanism is used (false) or
% the selection of child files must be performed manually (true).
% The definitions initialise to false:
%    \begin{macrocode}
\newif\ifchilddoc
\newif\ifchilddocmanual
%    \end{macrocode}

% \macro{\childdocname}
% \macro{\childdocjob}
% The macro |\childdocname| stores the name of the main document
% to be compiled. The macro |\childdocjob| stores the name of
% the document on which the \LaTeX{} compiler was originally invoked.
% The content of |\jobname| cannot be compared
% to filenames specified in the source due to different catcodes.
% The following code rescans |\jobname|, stores the result
% in |\childdocname| and saves a copy in |\childdocjob|:
%    \begin{macrocode}
\edef\childdocname{\scantokens\expandafter{\jobname\noexpand}}
\let\childdocjob\childdocname
%    \end{macrocode}

% \macro{\childdocdisable}
% The macro |\childdocdisable| prevents the main file
% from being processed more than once.
% At this stage, the main document command |\childdocmain|
% is assumed to be called once again where it should do nothing.
% Any subsequent call to it should prevent
% a secondary processing of the main document
% It overwrites the forwarding commands
% |\childdocof| and |\childdocforward|
% with empty macros to prevent further inclusions of the main document:
%    \begin{macrocode}
\newcommand{\childdocdisable}
{
  \renewcommand{\childdocmain}[1]{\renewcommand{\childdocmain}[1]{\endinput}}
  \renewcommand{\childdocof}[1]{}
  \renewcommand{\childdocby}[2][]{}
  \renewcommand{\childdocforward}[2][]{}
  \renewcommand{\childdocdisable}{}
}
%    \end{macrocode}

% \macro{\childdocmain}
% The macro |\childdocmain| is to be called at the top of the main file
% with nothing or the main filename (without extension) as argument.
% First, it breaks loops.
% If the argument is not empty and does not match |\childdocname|
% (which is set by the first inclusion of |childdoc.def|),
% |\ifchilddoc| is set to true, |\includeonly| is applied to the child file
% and |\jobname| is set to the main file
% (for proper handling of |.aux| files):
%    \begin{macrocode}
\newcommand{\childdocmain}[1]
{
  \childdocdisable\childdocmain{}
  \if?#1?\else
    \begingroup
      \def\childdoctmp{#1}
      \ifx\childdoctmp\childdocname
        \def\childdoctmp{}
      \else
        \def\childdoctmp
        {
          \childdoctrue
          \includeonly{\childdocname}
          \def\childdocjob{#1}
          \def\jobname{#1}
        }
      \fi
      \expandafter
    \endgroup
    \childdoctmp
  \fi
}
%    \end{macrocode}

% \macro{\childdocof}
% The command |\childdocof| redirects
% compilation to the main file |#1|.
%    \begin{macrocode}
\newcommand{\childdocof}[1]
{
  \childdocdisable
  \childdoctrue
  \includeonly{\childdocname}
  \def\jobname{#1}
  \def\childdocjob{#1}
  \input{#1}
}
%    \end{macrocode}

% \macro{\childdocby}
% The command |\childdocby| ....
%    \begin{macrocode}
\newcommand{\childdocby}[2][]
{
  \childdocdisable
  \childdoctrue
  \childdocmanualtrue
  \if?#1?\else
    \def\jobname{#2}
  \fi
  \def\childdocjob{#2}
  \input{#2}
  \endinput
}
%    \end{macrocode}

% \macro{\childdocforward}
% The command |\childdocforward| redirects
% compilation to the main file or
% (if the optional argument is given) a child file.
% Parameters are set as if the main file
% or a child file starting with |\childdocof| was compiled.
% Then compilation is handed over to the main file:
%    \begin{macrocode}
\newcommand{\childdocforward}[2][]
{
  \begingroup
    \if?#1?
      \def\childdoctmp
      {
        \def\childdocname{#2}
        \def\childdocjob{#2}
        \def\jobname{#2}
        \input{#2}
        \endinput
      }
    \else
      \def\childdoctmp
      {
        \childdocdisable
        \def\childdocname{#2}
        \childdoctrue
        \includeonly{#2}
        \def\childdocjob{#1}
        \def\jobname{#1}
        \input{#1}
        \endinput
      }
    \fi
    \expandafter
  \endgroup
  \childdoctmp
}
%    \end{macrocode}

% \macro{\childdocforwardprefix}
% The command |\childdocforwardprefix| redirects
% compilation to the main or a child file by means of a pattern.
% The prefix |#1| in the current filename is replaced by |#2|
% and the suffix of the current filename is kept
% (it is assumed that the filename does not contain the substring `|~~~|'
% which is used as a delimiter).
% Compilation is handed over to the new file by |\childdocforward|:
%    \begin{macrocode}
\newcommand{\childdocforwardprefix}[3][]
{
  \begingroup
    \def\childdocextract #2##1~~~{\def\childdoctmp{\childdocforward[#1]{#3##1}}}
    \expandafter\childdocextract\childdocname~~~
    \expandafter
  \endgroup
  \childdoctmp
}
%    \end{macrocode}

% \macro{\childdoc}
% The deprecated macro |\childdoc| is a legacy version of |\childdocmain|:
%    \begin{macrocode}
\newcommand{\childdoc}{\childdocmain}
%    \end{macrocode}

% \macro{\childdocredirect}
% The deprecated macro |\childdocredirect| is a legacy version
% of |\childdocforward| and |\childdocforwardprefix|:
%    \begin{macrocode}
\newcommand{\childdocredirect}[2][]
{
  \begingroup
    \if?#1?
      \def\childdoctmp{\childdocforward{#2}}
    \else
      \def\childdoctmp{\childdocforwardprefix{#1}{#2}}
    \fi
    \expandafter
  \endgroup
  \childdoctmp
}
%    \end{macrocode}

%\iffalse
%</package>
%\fi
%
\endinput
|
and perform the replacements as outlined below.
Instead of |\childdocmain{|\textit{main}|}| add the following code
to the top of the main file:
%
\begin{center}
\begin{tabular}{l}
|\||ifdefined\childdocname\endinput\||fi\newif\ifchilddoc|\\
|\edef\childdocname{\scantokens\expandafter{\jobname\noexpand}}|\\
|\def\childdocmain{|\textit{main}|}\||ifx\childdocmain\childdocname\||else|\\
|\childdoctrue\includeonly{\childdocname}\let\jobname\childdocmain\||fi|\\
\end{tabular}
\end{center}
%
Instead of |\childdocof{|\textit{main}|}| just include the main file
at the top of each child file:
%
\begin{center}
|\input{|\textit{main}|}|
\end{center}
%
A simple redirection |\childdocforward{|\textit{dest}|}| is achieved by:
%
\begin{center}
|\def\jobname{|\textit{dest}|}\input{\jobname}|
\end{center}
%
The redirection with prefix
|\childdocforwardprefix[|\textit{prefix}|]{|\textit{dest}|}|
is accomplished by:
%
\begin{center}
\begin{tabular}{l}
|{\edef\jobname{\scantokens\expandafter{\jobname\noexpand}}|\\
|\def\redirectjob |\textit{prefix}|#1~~~{\gdef\jobname{|\textit{dest}|#1}}|\\
|\expandafter\redirectjob\jobname~~~}\input{\jobname}|
\end{tabular}
\end{center}

In an alternative approach,
child documents can be compiled by a specific command line
without additional code or specific definitions:
%
\begin{center}
|... -jobname "|\textit{target}|" "|[\textit{flags}]%
|\includeonly{|\textit{dest}|}\input{|\textit{main}|}"|
\end{center}
%

%%%%%%%%%%%%%%%%%%%%%%%%%%%%%%%%%%%%%%%%%%%%%%%%%%%%%%%%%%%%%%%%%%%%%%%%%%%%%%%%
%%%%%%%%%%%%%%%%%%%%%%%%%%%%%%%%%%%%%%%%%%%%%%%%%%%%%%%%%%%%%%%%%%%%%%%%%%%%%%%%
\section{Information}

%%%%%%%%%%%%%%%%%%%%%%%%%%%%%%%%%%%%%%%%%%%%%%%%%%%%%%%%%%%%%%%%%%%%%%%%%%%%%%%%
\subsection{Copyright}

Copyright \copyright{} 2017--2018 Niklas Beisert

This work may be distributed and/or modified under the
conditions of the \LaTeX{} Project Public License, either version 1.3
of this license or (at your option) any later version.
The latest version of this license is in
  \url{http://www.latex-project.org/lppl.txt}
and version 1.3 or later is part of all distributions of \LaTeX{}
version 2005/12/01 or later.

This work has the LPPL maintenance status `maintained'.

The Current Maintainer of this work is Niklas Beisert.

This work consists of the files |README.txt|, |childdoc.ins| and |childdoc.dtx|
as well as the derived files |childdoc.def|, |cdocsamp.tex|
with |cdocsch1.tex|, |cdocsch2.tex|, |cdocspt3.tex|, |cdocspt4.tex|,
|cdocsdrf.tex|, |cdocsfn1.tex|, |cdocsfn2.tex|
as well as |childdoc.pdf|.

%%%%%%%%%%%%%%%%%%%%%%%%%%%%%%%%%%%%%%%%%%%%%%%%%%%%%%%%%%%%%%%%%%%%%%%%%%%%%%%%
\subsection{Files and Installation}

The package consists of the files:
%
\begin{center}
\begin{tabular}{ll}
    |README.txt|   & readme file \\
    |childdoc.ins| & installation file \\
    |childdoc.dtx| & source file \\
    |childdoc.def| & definition file \\
    |cdocsamp.tex| & sample main file \\
    |cdocsch1.tex| & sample include file \\
    |cdocsch2.tex| & sample include file \\
    |cdocspt3.tex| & sample part file \\
    |cdocspt4.tex| & sample part file \\
    |cdocsdrf.tex| & sample redirection file \\
    |cdocsfn1.tex| & sample redirection file \\
    |cdocsfn2.tex| & sample redirection file \\
    |childdoc.pdf| & manual
\end{tabular}
\end{center}
%
The distribution consists of the files
|README.txt|, |childdoc.ins| and |childdoc.dtx|.
%
\begin{itemize}
\item
Run (pdf)\LaTeX{} on |childdoc.dtx|
to compile the manual |childdoc.pdf| (this file).
\item
Run \LaTeX{} on |childdoc.ins| to create the definitions file |childdoc.def|
and the sample |cdocsamp.tex| with include files
|cdocsch1.tex|, |cdocsch2.tex|, |cdocspt3.tex|, |cdocspt4.tex|,
|cdocsdrf.tex|, |cdocsfn1.tex|, |cdocsfn2.tex|.
Then copy the file |childdoc.def| to an appropriate directory of your \LaTeX{}
distribution, e.g.\ \textit{texmf-root}|/tex/latex/childdoc|.
\end{itemize}

%%%%%%%%%%%%%%%%%%%%%%%%%%%%%%%%%%%%%%%%%%%%%%%%%%%%%%%%%%%%%%%%%%%%%%%%%%%%%%%%
\subsection{Related CTAN Packages}

There are several other packages which offer a similar functionality:
%
\begin{itemize}
\item
The packages
\href{http://ctan.org/pkg/docmute}{\textsf{docmute}},
\href{http://ctan.org/pkg/includex}{\textsf{includex}} and
\href{http://ctan.org/pkg/standalone}{\textsf{standalone}}
provide commands to include only the document body of
a child file thus allowing both files to be compiled individually.
\item
The packages \href{http://ctan.org/pkg/subdocs}{\textsf{subdocs}}
and \href{http://ctan.org/pkg/subfiles}{\textsf{subfiles}}
provide structures in which the main and child documents can be
encapsulated and allowing them to be compiled individually.
The inclusion mechanism is different from the conventional |\include|.
\item
The package \href{http://ctan.org/pkg/combine}{\textsf{combine}}
is an elaborate solution to combine several documents into one.
\end{itemize}
%
See also the CTAN topic \href{http://ctan.org/topic/subdocs}{\textsf{subdocs}}
for further related packages.
The present package differs from the above solutions in that
a document structure constructed with the conventional |\include| mechanism
just needs two extra commands at the top of every file
such that all constituent files can be compiled individually.

%%%%%%%%%%%%%%%%%%%%%%%%%%%%%%%%%%%%%%%%%%%%%%%%%%%%%%%%%%%%%%%%%%%%%%%%%%%%%%%%
%\subsection{Feature Suggestions}
%
%The following is a list of features which may be useful for future
%versions of this package:
%%
%\begin{itemize}
%\item
%\ldots
%\end{itemize}

%%%%%%%%%%%%%%%%%%%%%%%%%%%%%%%%%%%%%%%%%%%%%%%%%%%%%%%%%%%%%%%%%%%%%%%%%%%%%%%%
\subsection{Revision History}

%%%%%%%%%%%%%%%%%%%%%%%%%%%%%%%%%%%%%%%%
\paragraph{v2.0:} 2018/12/30

\begin{itemize}
\item
immediate forward processing
\item
added |\childdocby| mechanism
\item
manual restructured
\end{itemize}

%%%%%%%%%%%%%%%%%%%%%%%%%%%%%%%%%%%%%%%%
\paragraph{v1.6:} 2018/01/17

\begin{itemize}
\item
application for development of include files
\item
corrections to manual
\end{itemize}

%%%%%%%%%%%%%%%%%%%%%%%%%%%%%%%%%%%%%%%%
\paragraph{v1.5:} 2017/05/21

\begin{itemize}
\item
more complete structuring introduced
\item
|\childdocof| introduced
\item
|\childdoc| renamed to |\childdocmain|
\item
|\childredirect| renamed to |\childdocforward| and |\childdocforwardprefix|
and functionality expanded
\end{itemize}

%%%%%%%%%%%%%%%%%%%%%%%%%%%%%%%%%%%%%%%%
\paragraph{v1.0:} 2017/04/27

\begin{itemize}
\item
manual and install package
\item
first version published on CTAN
\end{itemize}

%%%%%%%%%%%%%%%%%%%%%%%%%%%%%%%%%%%%%%%%
\paragraph{v0.6:} 2017/04/26

\begin{itemize}
\item
redirection mechanism added
\end{itemize}

%%%%%%%%%%%%%%%%%%%%%%%%%%%%%%%%%%%%%%%%
\paragraph{v0.5:} 2017/04/26

\begin{itemize}
\item
functionality in definition file
\end{itemize}


%%%%%%%%%%%%%%%%%%%%%%%%%%%%%%%%%%%%%%%%%%%%%%%%%%%%%%%%%%%%%%%%%%%%%%%%%%%%%%%%
%%%%%%%%%%%%%%%%%%%%%%%%%%%%%%%%%%%%%%%%%%%%%%%%%%%%%%%%%%%%%%%%%%%%%%%%%%%%%%%%
%%%%%%%%%%%%%%%%%%%%%%%%%%%%%%%%%%%%%%%%%%%%%%%%%%%%%%%%%%%%%%%%%%%%%%%%%%%%%%%%
\appendix

\settowidth\MacroIndent{\rmfamily\scriptsize 000\ }

 \DocInput{childdoc.dtx}

\end{document}
%</driver>
% \fi
%
% %%%%%%%%%%%%%%%%%%%%%%%%%%%%%%%%%%%%%%%%%%%%%%%%%%%%%%%%%%%%%%%%%%%%%%%%%%%%%%
% %%%%%%%%%%%%%%%%%%%%%%%%%%%%%%%%%%%%%%%%%%%%%%%%%%%%%%%%%%%%%%%%%%%%%%%%%%%%%%
% \section{Sample}
%\iffalse
%<*samplemain>
%\fi
%
% The following presents a sample document
% with two chapters, two parts, a title page,
% a compile flag as well as three forwarding files to set the flag.
% It consists of eight |.tex| files:
% \begin{center}
% \begin{tabular}{ll}
% |cdocsamp.tex|&main file\\
% |cdocsch1.tex|&include file for chapter 1\\
% |cdocsch2.tex|&include file for chapter 2\\
% |cdocspt3.tex|&include file for part 3\\
% |cdocspt4.tex|&include file for part 4\\
% |cdocsdrf.tex|&forwarding file for main file in draft mode\\
% |cdocsfi1.tex|&forwarding file for final version of chapter 1\\
% |cdocsfi2.tex|&forwarding file for final version of chapter 2\\
% \end{tabular}
% \end{center}
% Each of the eight files can be compiled directly by the \LaTeX{} compiler.
%
% %%%%%%%%%%%%%%%%%%%%%%%%%%%%%%%%%%%%%%
% \paragraph{Main File.}
%
% The main file is called |cdocsamp.tex|.
%
% Load the \textsf{childdoc} definitions and
% declare the filename for the main document:
%    \begin{macrocode}
% \iffalse
%
% childdoc.dtx Copyright (C) 2017-2018 Niklas Beisert
%
% This work may be distributed and/or modified under the
% conditions of the LaTeX Project Public License, either version 1.3
% of this license or (at your option) any later version.
% The latest version of this license is in
%   http://www.latex-project.org/lppl.txt
% and version 1.3 or later is part of all distributions of LaTeX
% version 2005/12/01 or later.
%
% This work has the LPPL maintenance status `maintained'.
%
% The Current Maintainer of this work is Niklas Beisert.
%
% This work consists of the files childdoc.dtx and childdoc.ins
% and the derived files childdoc.def and cdocsamp.tex with
% cdocsch1.tex, cdocsch2.tex, cdocsdrf.tex, cdocsfn1.tex, cdocsfn2.tex.
%
%<package>\ifdefined\childdocmain\endinput\fi
%<package>\ProvidesFile{childdoc.def}[2018/12/30 v2.0 child document driver]
%<samplemain>\ProvidesFile{cdocsamp.tex}[2018/12/30 v2.0 sample for childdoc]
%<*driver>
%\ProvidesFile{childdoc.drv}[2018/12/30 v2.0 childdoc reference manual file]
\PassOptionsToClass{10pt,a4paper}{article}
\documentclass{ltxdoc}

\usepackage[margin=35mm]{geometry}
\usepackage{hyperref}
\usepackage{hyperxmp}
\usepackage[usenames]{color}

\hypersetup{colorlinks=true}
\hypersetup{pdfstartview=FitH}
\hypersetup{pdfpagemode=UseNone}
\hypersetup{pdfsource={}}
\hypersetup{pdflang={en-UK}}
\hypersetup{pdfcopyright={Copyright 2017-2018 Niklas Beisert.
  This work may be distributed and/or modified under the
  conditions of the LaTeX Project Public License, either version 1.3
  of this license or (at your option) any later version.}}
\hypersetup{pdflicenseurl={http://www.latex-project.org/lppl.txt}}
\hypersetup{pdfcontactaddress={ETH Zurich, ITP, HIT K,
  Wolfgang-Pauli-Strasse 27}}
\hypersetup{pdfcontactpostcode={8093}}
\hypersetup{pdfcontactcity={Zurich}}
\hypersetup{pdfcontactcountry={Switzerland}}
\hypersetup{pdfcontactemail={nbeisert@itp.phys.ethz.ch}}
\hypersetup{pdfcontacturl={http://people.phys.ethz.ch/\xmptilde nbeisert/}}

\newcommand{\secref}[1]{\hyperref[#1]{section \ref*{#1}}}

\parskip1ex
\parindent0pt
\let\olditemize\itemize
\def\itemize{\olditemize\parskip0pt}

\begin{document}

\title{The \textsf{childdoc} Package}
\hypersetup{pdftitle={The childdoc Package}}
\author{Niklas Beisert\\[2ex]
  Institut f\"ur Theoretische Physik\\
  Eidgen\"ossische Technische Hochschule Z\"urich\\
  Wolfgang-Pauli-Strasse 27, 8093 Z\"urich, Switzerland\\[1ex]
  \href{mailto:nbeisert@itp.phys.ethz.ch}
  {\texttt{nbeisert@itp.phys.ethz.ch}}}
\hypersetup{pdfauthor={Niklas Beisert}}
\hypersetup{pdfsubject={Manual for the LaTeX2e Package childdoc}}
\date{30 December 2018, \textsf{v2.0}}
\maketitle

\begin{abstract}\noindent
\textsf{childdoc} is a \LaTeXe{} package
that enables the direct compilation
of document sections included by |\include|
to individual files.
\end{abstract}

\begingroup
\parskip0ex
\tableofcontents
\endgroup

%%%%%%%%%%%%%%%%%%%%%%%%%%%%%%%%%%%%%%%%%%%%%%%%%%%%%%%%%%%%%%%%%%%%%%%%%%%%%%%%
%%%%%%%%%%%%%%%%%%%%%%%%%%%%%%%%%%%%%%%%%%%%%%%%%%%%%%%%%%%%%%%%%%%%%%%%%%%%%%%%
\section{Introduction}

\LaTeX{} provides a mechanism to structure a large document (such as a book)
into a main file and several child files (containing the chapters)
using the |\include| command.
This mechanism is beneficial for documents
which span hundreds of pages in order to
make the source file(s) more manageable.
Moreover, compilation can be restricted to
selected child files by means of the |\includeonly| command.
The latter feature can be used to reduce the compilation time while editing
(this was significantly more useful in the earlier days of \LaTeX{})
or to generate a smaller document which is easier to navigate.
Another application of |\includeonly| is to generate
documents consisting of selected parts of the complete document.

However, there are a few drawbacks of the plain |\include| mechanism:
\begin{itemize}
\item
The child files cannot be compiled on their own,
they can only be compiled via the main file.
A naive editing environment
(such as a text editor with an option
to have the current file processed by \LaTeX)
may require one to switch to the main file before compiling;
attempting to compile the child file produces errors.
\item
The main file must be modified (each time)
to adjust the |\includeonly| command
to the present needs. This easily leaves the main file in a messy state.
\item
The generated document will always carry the filename
of the main document. This is inconvenient if
several child files are to be compiled and
to be kept for distribution.
\end{itemize}

The present package provides a simple interface
to make child files individually compilable by \LaTeX{}.
Compiling a child file then has the same effect as compiling
the main file with an |\includeonly| command
to select the appropriate child.
Moreover the generated document will carry the name of the child
rather than the main file.
This resolves all three above issues.

This feature is meant to make the editing of books,
thesis documents and lecture notes somewhat more convenient.
However, the package can also be used efficiently for
composing a series of documents (such as exercise sheets)
which are typically distributed individually.
It then assists the author in generating the individual documents
(potentially in different versions)
as well as a document containing the collected series.
Another application is in developing style files
or other kinds of included material
where compilation of the style file could redirect
to a sample or test file.

%%%%%%%%%%%%%%%%%%%%%%%%%%%%%%%%%%%%%%%%%%%%%%%%%%%%%%%%%%%%%%%%%%%%%%%%%%%%%%%%
%%%%%%%%%%%%%%%%%%%%%%%%%%%%%%%%%%%%%%%%%%%%%%%%%%%%%%%%%%%%%%%%%%%%%%%%%%%%%%%%
\section{Usage}

First of all, the package \textsf{childdoc} is \emph{not} a standard
\LaTeXe{} |.sty| style file! Therefore it needs to be invoked in
a non-standard way.

%%%%%%%%%%%%%%%%%%%%%%%%%%%%%%%%%%%%%%%%%%%%%%%%%%%%%%%%%%%%%%%%%%%%%%%%%%%%%%%%
\subsection{Included Files}
\label{sec:include}

%%%%%%%%%%%%%%%%%%%%%%%%%%%%%%%%%%%%%%%%
\DescribeMacro{\childdocmain}
To use the package, add the commands
\begin{center}
\begin{tabular}{l}
|\input{childdoc.def}|\\
|\childdocmain{}|\\
\end{tabular}
\end{center}
at the very top of the main \LaTeX{} file,
in particular \emph{before} the |\documentclass| statement!
The argument of |\childdocmain| should be left empty
(but it must be present).

%%%%%%%%%%%%%%%%%%%%%%%%%%%%%%%%%%%%%%%%
\DescribeMacro{\childdocof}
Furthermore, add the commands
\begin{center}
\begin{tabular}{l}
|\input{childdoc.def}|\\
|\childdocof{|\textit{main}|}|\\
\end{tabular}
\end{center}
at the top of every child file \textit{child}
which is included by |\include{|\textit{child}|}|
from within the main file
(or at least for those files to be compiled individually).
The argument \textit{main} must be the filename of the main file.

There are a couple of
considerations in setting up the main and child documents:

%%%%%%%%%%%%%%%%%%%%%%%%%%%%%%%%%%%%%%%%
\paragraph{Restrictions.}

Please note the following restrictions:
\begin{itemize}
\item
|\childdocmain| must be called with one argument \textit{main}
to ensure compatibility with earlier version of the package.
It must either be empty (|\childdocmain{}|)
or precisely match the filename of the main file in which it is specified.
See \secref{sec:detection} for further information.
\item
The filename \textit{main} must be specified without the |.tex| extension.
\item
The filename \textit{main} is case sensitive
(even in case-insensitive file systems)
due to internal string comparison.
\item
The argument \textit{main} should be fully expanded, it cannot be a macro.
\item
Subdirectories and special characters should be avoided in filenames.
\item
The command |\childdocmain{|\textit{main}|}| must be followed by a whitespace.
It should not be followed immediately by another command
or by a comment mark `|%|'.
This is because the \TeX{} parser reads the token immediately following
the argument of |\childdocmain| and puts it
at the beginning of every child section;
however, a white\-space is ignored.
\end{itemize}

%%%%%%%%%%%%%%%%%%%%%%%%%%%%%%%%%%%%%%%%
\paragraph{Content of Main File.}

It is advisable to place all content in the child files included by |\include|.
Any output contained in the main file will appear in all child documents
unless suppressed manually;
it cannot be suppressed automatically by the |\includeonly| directive
and thus should normally be avoided.
A method to include some content in the main file
by means of conditional processing is described in \secref{sec:conditional}.

%%%%%%%%%%%%%%%%%%%%%%%%%%%%%%%%%%%%%%%%
\paragraph{Page Numbering.}

When only a part of the document is compiled,
the appropriate numbering of pages
(as well as other status parameters)
is determined from the |.aux| files.
The latter contain information from previous passes.
However this information needs to propagate through
all intermediate child documents.
Therefore the page numbering in child documents may well
be inconsistent until the complete document is compiled at least once.

A useful (if unconventional) way to always ensure a consistent
page numbering is to restart the numbering in each child document
and denote the pages by `\textit{child}|.|\textit{page}'
where \textit{child} represents the chapter/section number of the child file.
This can be achieved by the command
|\numberwithin{page}{|\textit{child}|}|
of the \textsf{amsmath} package
where \textit{child} can be |chapter| or |section|
depending on the chosen structuring.
Alternatively, one can modify the macro |\thepage| appropriately
and reset the counter |page| at the start of each child file.

%%%%%%%%%%%%%%%%%%%%%%%%%%%%%%%%%%%%%%%%%%%%%%%%%%%%%%%%%%%%%%%%%%%%%%%%%%%%%%%%
\subsection{Conditional Processing}
\label{sec:conditional}

The package provides a mechanism to compile different versions
of a document. To customise the versions further some conditional processing
can come in handy to distinguish which version is being compiled.
The package provides two macros to describe the compilation context:

%%%%%%%%%%%%%%%%%%%%%%%%%%%%%%%%%%%%%%%%
\DescribeMacro{\ifchilddoc}
The conditional |\ifchilddoc| distinguishes between the compilation of
child documents and the main document:
%
\begin{center}
|\ifchilddoc |\textit{child-code}| |[|\||else |\textit{main-code}]| \||fi|
\end{center}

%%%%%%%%%%%%%%%%%%%%%%%%%%%%%%%%%%%%%%%%
\DescribeMacro{\childdocname}
\DescribeMacro{\childdocjob}
The macro |\childdocname| contains the filename (without extension)
of the main or child file being processed.
Note that |\childdocjob| will always contain the name of the main file.

%%%%%%%%%%%%%%%%%%%%%%%%%%%%%%%%%%%%%%%%
\paragraph{Title Page.}

Conditional processing can be used to include a title or banner page
in the main document when proper precautions are taken.
Importantly, the code in the main file should ensure that the page counter
(as well as other status parameters which are stored in the |.aux| files)
takes the same value after the conditional processing.
Otherwise the page numbers may take divergent values
depending on which part is compiled.

For example, a title page could be declared by:
%
\begin{center}
\begin{tabular}{l}
|\ifchilddoc\||else|\\
|\addtocounter{page}{-1}|\\
\textit{code for title page}\\
|\newpage|\\
|\||fi|
\end{tabular}
\end{center}
%
A banner page for the child documents can be generated by:
%
\begin{center}
\begin{tabular}{l}
|\ifchilddoc|\\
|\addtocounter{page}{-1}|\\
\textit{code for banner page}\\
|\newpage|\\
|\||fi|
\end{tabular}
\end{center}
%
Here one could write a message such as:
\begin{center}
|This is the part \childdocname{} of \childdocjob{}.|
\end{center}

%%%%%%%%%%%%%%%%%%%%%%%%%%%%%%%%%%%%%%%%%%%%%%%%%%%%%%%%%%%%%%%%%%%%%%%%%%%%%%%%
\subsection{Flags}
\label{sec:flags}

The package makes it easy to generate different versions
of the main or child documents.
To this end compilation flags can be defined
and assigned different default values.
They will be particularly useful in conjunction
with the forwarding mechanism described in \secref{sec:forward}.

For example, it may be useful to have a flag |\version|
which can be set to |draft| or |final|.
The document source will contain some conditional code
depending on the value of |\version|.
Suppose further, the flag should default to |final| for the main file
and to |draft| for child files
which is a natural assignment for editing the document.
This is achieved by placing the following code
in the preamble of the main document
(below the |\childdocmain| directive):
%
\begin{center}
\begin{tabular}{l}
|\ifchilddoc|\\
|\providecommand{\version}{draft}|\\
|\||else|\\
|\providecommand{\version}{final}|\\
|\||fi|
\end{tabular}
\end{center}
%
The definition by |\providecommand| makes sure
that previous definitions are not overwritten.
Further statements |\providecommand{\version}{...}|
can thus be added before the above code to override it.

For the main file, one might add a line
(between |\childdocmain| and the above block)
%
\begin{center}
|%\ifchilddoc\||else\providecommand{\version}{draft}\||fi|
\end{center}
%
which can be uncommented to produce a draft version.
Likewise one can add a line to the very top of a child file
(above the |\childdocof{|\textit{main}|}| directive)
%
\begin{center}
|%\providecommand{\version}{final}|
\end{center}
%
which can be uncommented to produce the final version of this child document.

%%%%%%%%%%%%%%%%%%%%%%%%%%%%%%%%%%%%%%%%%%%%%%%%%%%%%%%%%%%%%%%%%%%%%%%%%%%%%%%%
\subsection{Forwarding}
\label{sec:forward}

Different versions of the main or child documents
using compilation flags as described in \secref{sec:flags}
can be (permanently) stored in different files
for convenient compilation, viewing and distribution.
To this end, the package defines a command
to pass on compilation to a different file:

%%%%%%%%%%%%%%%%%%%%%%%%%%%%%%%%%%%%%%%%
\DescribeMacro{\childdocforward}
The command |\childdocforward| redirects processing to
another source file:
%
\begin{center}
\begin{tabular}{l}
|\input{childdoc.def}|\\
|\childdocforward[|\textit{main}|]{|\textit{dest}|}|\\
\end{tabular}
\end{center}
%
The argument \textit{dest} is the destination file
(without extension).
It should be the main file or one of the child files.
Note that further \textsf{childdoc} directives
such as |\childdocof| and |\childdocforward|
in the indicated file will be processed in this form.
The optional argument \textit{main}
passes on directly to the main file \textit{main}
while pretending to compile the child \textit{dest}.
This form behaves as if \textit{dest}
issues |\childdocof{|\textit{main}|}| right away,
and no further \textsf{childdoc} directives will be processed.

%%%%%%%%%%%%%%%%%%%%%%%%%%%%%%%%%%%%%%%%
\DescribeMacro{\...prefix}
In the alternative form |\childdocforwardprefix|,
%
\begin{center}
\begin{tabular}{l}
|\input{childdoc.def}|\\
|\childdocforwardprefix[|\textit{main}|]{|\textit{prefix}|}{|\textit{dest}|}|
\end{tabular}
\end{center}
%
the destination file is determined by a pattern
depending on the current file:
To make this work, the current file must be called
`{\textit{prefix}\hspace{0.2em}\textit{suffix}}'
with \textit{prefix} matching precisely the argument.
Processing is then passed on to the file
`{\textit{dest}\hspace{0.2em}\textit{suffix}}'.
Surely, the same effect is achieved by
directly specifying the
argument `{\textit{dest}\hspace{0.2em}\textit{suffix}}'
in the first form.
However, that requires to set up a different file
for each child. With the alternative form of the command
all these files can have exactly the same content
which simplifies setting them up and maintaining them.

For example, the following file |draft.tex|
with a compilation flag |\version| as described in \secref{sec:flags}
compiles the main document as a draft:
%
\begin{center}
\begin{tabular}{l}
|\def\version{draft}|\\
|\input{childdoc.def}|\\
|\childdocforward{|\textit{main}|}|
\end{tabular}
\end{center}
%
Likewise, the following files |final|\textit{nn}|.tex|
compile the final version of the child document
|child|\textit{nn}|.tex|:
%
\begin{center}
\begin{tabular}{l}
|\def\version{final}|\\
|\input{childdoc.def}|\\
|\childdocforwardprefix{final}{child}|
\end{tabular}
\end{center}
%

Note that when several versions of a main file and/or of each child file
are to be generated, it may be convenient to set up a |Makefile| or
shell script to automatise the process.

%%%%%%%%%%%%%%%%%%%%%%%%%%%%%%%%%%%%%%%%%%%%%%%%%%%%%%%%%%%%%%%%%%%%%%%%%%%%%%%%
\subsection{Command Line Processing}
\label{sec:commandline}

The effect of redirection files can also be achieved by invoking
the \LaTeX{} compiler with a more elaborate command line.
Most conveniently this should be done as part
of a shell script or a |Makefile|.

When using \textsf{childdoc} in the main file, the following
command lines effectively perform a redirection
(note that depending on the shell being used,
backslashes may have to be doubled: `|\|' $\to$ `|\\|'):
%
\begin{center}
|... -jobname "|\textit{target}|" |\\|"|[\textit{flags}]%
|\input{childdoc.def}\childdocforward[|\textit{main}|]{|\textit{dest}|}"|
\end{center}
%
Here \textit{target} is the name of the output file,
\textit{main} is the name of the main file
and \textit{dest} is the name of the main or child file to be processed
(all filenames without extensions).
The optional argument \textit{main} can be omitted
if \textit{main} matches \textit{dest}.
Optionally, compilation \textit{flags} can be defined via |\def| commands.
This command line makes the \TeX{} engine believe
it is compiling the file \textit{target}
whose content is specified as the latter parameter.
The provided code then forwards the processing to
\textit{main} or \textit{dest} as described in \secref{sec:forward}.

%%%%%%%%%%%%%%%%%%%%%%%%%%%%%%%%%%%%%%%%%%%%%%%%%%%%%%%%%%%%%%%%%%%%%%%%%%%%%%%%
\subsection{Include by Input}
\label{sec:input}

Including child documents by |\include| has some restrictions by design.
Most notably, the content of a child document always occupies
its own set of pages; pages cannot be shared between child documents.
Usually, this behaviour makes perfect sense
because each child document contain an essential part of the document.
However, in some situations it may be desirable to compose
a document from a collection of parts
without having mandatory page breaks between then.
For this case, the package
provides a mechanism to include parts
by |\input| which can also be processed individually.
However, by construction this mechanism
requires manual handling of the content to be output.

%%%%%%%%%%%%%%%%%%%%%%%%%%%%%%%%%%%%%%%%
\DescribeMacro{\ifchilddocmanual}
The main file should be prepared as usual, see \secref{sec:include}.
However, the document body must make a distinction
between processing of an individual part and of the main document, e.g.:
%
\begin{center}
\begin{tabular}{l}
|\ifchilddocmanual|\\
|\input{\childdocname}|\\
|\||else|\\
\textit{document body with }|\input{|\textit{part}|}|\\
|\||fi|
\end{tabular}
\end{center}
%
The conditional |\ifchilddocmanual| is true whenever
a part to be included by |\input| is being compiled,
and the name of the part is stored in |\childdocname|.

%%%%%%%%%%%%%%%%%%%%%%%%%%%%%%%%%%%%%%%%
\DescribeMacro{\childdocby}
Each part to be included by |\input| should start with:
%
\begin{center}
\begin{tabular}{l}
|\input{childdoc.def}|\\
|\childdocby{|\textit{main}|}|\\
\end{tabular}
\end{center}
%
The directive |\childdocby| is similar to |\childdocof|
described in \secref{sec:include},
but the subsequent selection of content must be done manually.
To that end, both |\ifchilddoc| and |\ifchilddocmanual|
will be true upon processing of a part,
and the name of the part is stored in |\childdocname|.
Note that |\jobname| will be set to the filename of the current part
so that each part receives an individual |.aux| file
that does not interfere with the |.aux| file(s) of the main document.
This behaviour can be altered by the alternative form
|\childdocby[*]{|\textit{main}|}| (with a non-empty optional argument)
which uses the |.aux| file of the main document
by setting |\jobname| to \textit{main}.

%%%%%%%%%%%%%%%%%%%%%%%%%%%%%%%%%%%%%%%%%%%%%%%%%%%%%%%%%%%%%%%%%%%%%%%%%%%%%%%%
\subsection{Driver Development}
\label{sec:driver}

The \textsf{childdoc} mechanism can also be use for the development
of definition files such as \LaTeX{} styles or classes.
This case differs from the above setup with multiple parts
included by |\include| in that no |\includeonly| should be invoked.
This can be achieved by starting the include file
(before |\ProvidesPackage|) with:
%
\begin{center}
\begin{tabular}{l}
|\input{childdoc.def}|\\
|\childdocforward{|\textit{main}|}|\\
\end{tabular}
\end{center}
%
or alternatively with:
%
\begin{center}
\begin{tabular}{l}
|\input{childdoc.def}|\\
|\childdocby{|\textit{main}|}|\\
\end{tabular}
\end{center}
%
Both forms have slightly different effects as described above.
The main file is prepared as usual, see \secref{sec:include}.

%%%%%%%%%%%%%%%%%%%%%%%%%%%%%%%%%%%%%%%%%%%%%%%%%%%%%%%%%%%%%%%%%%%%%%%%%%%%%%%%
\subsection{Legacy Detection}
\label{sec:detection}

The directive |\childdocmain| in the main file can detect
whether the complete document or merely a child is to be compiled
even without using the directive |\childdocof|.
This method is deprecated because it is less robust
and there is no compelling reason to use it;
it is merely provided for backward compatibility
and it may be removed in future versions.

If the detection mechanism is to be used,
it is mandatory to correctly specify
the filename of the main file as the argument of |\childdocmain|:
%
\begin{center}
\begin{tabular}{l}
|\input{childdoc.def}|\\
|\childdocmain{|\textit{main}|}|\\
\end{tabular}
\end{center}
%
If |\jobname| does not match the argument \textit{main} of |\childdocmain|,
it is assumed that |\jobname| points to the child file to be compiled.
When using |\childdocmain| with the main file specified as argument,
it suffices to start a child file
with just |\input{|\textit{main}|}|
without loading of the package and using |\childdocof|.
If instead all processing is done
with the appropriate \textsf{childdoc} directives,
the argument of \textit{main} of |\childdocmain| can be empty.

An alternative version of the command line processing described
in \secref{sec:commandline} using the detection mechanism reads:
%
\begin{center}
|... -jobname "|\textit{target}|" "|[\textit{flags}]%
[|\def\jobname{|\textit{dest}|}|]|\input{|\textit{main}|}"|
\end{center}

%%%%%%%%%%%%%%%%%%%%%%%%%%%%%%%%%%%%%%%%%%%%%%%%%%%%%%%%%%%%%%%%%%%%%%%%%%%%%%%%
\subsection{Manual Code}
\label{sec:manual}

In case one cannot be certain whether the definitions file |childdoc.def|
is installed on the target \TeX{} distribution
and one prefers not to ship it,
it is conceivable to paste a few relevant commands into the sources.

To that end, drop all statements |\input{childdoc.def}|
and perform the replacements as outlined below.
Instead of |\childdocmain{|\textit{main}|}| add the following code
to the top of the main file:
%
\begin{center}
\begin{tabular}{l}
|\||ifdefined\childdocname\endinput\||fi\newif\ifchilddoc|\\
|\edef\childdocname{\scantokens\expandafter{\jobname\noexpand}}|\\
|\def\childdocmain{|\textit{main}|}\||ifx\childdocmain\childdocname\||else|\\
|\childdoctrue\includeonly{\childdocname}\let\jobname\childdocmain\||fi|\\
\end{tabular}
\end{center}
%
Instead of |\childdocof{|\textit{main}|}| just include the main file
at the top of each child file:
%
\begin{center}
|\input{|\textit{main}|}|
\end{center}
%
A simple redirection |\childdocforward{|\textit{dest}|}| is achieved by:
%
\begin{center}
|\def\jobname{|\textit{dest}|}\input{\jobname}|
\end{center}
%
The redirection with prefix
|\childdocforwardprefix[|\textit{prefix}|]{|\textit{dest}|}|
is accomplished by:
%
\begin{center}
\begin{tabular}{l}
|{\edef\jobname{\scantokens\expandafter{\jobname\noexpand}}|\\
|\def\redirectjob |\textit{prefix}|#1~~~{\gdef\jobname{|\textit{dest}|#1}}|\\
|\expandafter\redirectjob\jobname~~~}\input{\jobname}|
\end{tabular}
\end{center}

In an alternative approach,
child documents can be compiled by a specific command line
without additional code or specific definitions:
%
\begin{center}
|... -jobname "|\textit{target}|" "|[\textit{flags}]%
|\includeonly{|\textit{dest}|}\input{|\textit{main}|}"|
\end{center}
%

%%%%%%%%%%%%%%%%%%%%%%%%%%%%%%%%%%%%%%%%%%%%%%%%%%%%%%%%%%%%%%%%%%%%%%%%%%%%%%%%
%%%%%%%%%%%%%%%%%%%%%%%%%%%%%%%%%%%%%%%%%%%%%%%%%%%%%%%%%%%%%%%%%%%%%%%%%%%%%%%%
\section{Information}

%%%%%%%%%%%%%%%%%%%%%%%%%%%%%%%%%%%%%%%%%%%%%%%%%%%%%%%%%%%%%%%%%%%%%%%%%%%%%%%%
\subsection{Copyright}

Copyright \copyright{} 2017--2018 Niklas Beisert

This work may be distributed and/or modified under the
conditions of the \LaTeX{} Project Public License, either version 1.3
of this license or (at your option) any later version.
The latest version of this license is in
  \url{http://www.latex-project.org/lppl.txt}
and version 1.3 or later is part of all distributions of \LaTeX{}
version 2005/12/01 or later.

This work has the LPPL maintenance status `maintained'.

The Current Maintainer of this work is Niklas Beisert.

This work consists of the files |README.txt|, |childdoc.ins| and |childdoc.dtx|
as well as the derived files |childdoc.def|, |cdocsamp.tex|
with |cdocsch1.tex|, |cdocsch2.tex|, |cdocspt3.tex|, |cdocspt4.tex|,
|cdocsdrf.tex|, |cdocsfn1.tex|, |cdocsfn2.tex|
as well as |childdoc.pdf|.

%%%%%%%%%%%%%%%%%%%%%%%%%%%%%%%%%%%%%%%%%%%%%%%%%%%%%%%%%%%%%%%%%%%%%%%%%%%%%%%%
\subsection{Files and Installation}

The package consists of the files:
%
\begin{center}
\begin{tabular}{ll}
    |README.txt|   & readme file \\
    |childdoc.ins| & installation file \\
    |childdoc.dtx| & source file \\
    |childdoc.def| & definition file \\
    |cdocsamp.tex| & sample main file \\
    |cdocsch1.tex| & sample include file \\
    |cdocsch2.tex| & sample include file \\
    |cdocspt3.tex| & sample part file \\
    |cdocspt4.tex| & sample part file \\
    |cdocsdrf.tex| & sample redirection file \\
    |cdocsfn1.tex| & sample redirection file \\
    |cdocsfn2.tex| & sample redirection file \\
    |childdoc.pdf| & manual
\end{tabular}
\end{center}
%
The distribution consists of the files
|README.txt|, |childdoc.ins| and |childdoc.dtx|.
%
\begin{itemize}
\item
Run (pdf)\LaTeX{} on |childdoc.dtx|
to compile the manual |childdoc.pdf| (this file).
\item
Run \LaTeX{} on |childdoc.ins| to create the definitions file |childdoc.def|
and the sample |cdocsamp.tex| with include files
|cdocsch1.tex|, |cdocsch2.tex|, |cdocspt3.tex|, |cdocspt4.tex|,
|cdocsdrf.tex|, |cdocsfn1.tex|, |cdocsfn2.tex|.
Then copy the file |childdoc.def| to an appropriate directory of your \LaTeX{}
distribution, e.g.\ \textit{texmf-root}|/tex/latex/childdoc|.
\end{itemize}

%%%%%%%%%%%%%%%%%%%%%%%%%%%%%%%%%%%%%%%%%%%%%%%%%%%%%%%%%%%%%%%%%%%%%%%%%%%%%%%%
\subsection{Related CTAN Packages}

There are several other packages which offer a similar functionality:
%
\begin{itemize}
\item
The packages
\href{http://ctan.org/pkg/docmute}{\textsf{docmute}},
\href{http://ctan.org/pkg/includex}{\textsf{includex}} and
\href{http://ctan.org/pkg/standalone}{\textsf{standalone}}
provide commands to include only the document body of
a child file thus allowing both files to be compiled individually.
\item
The packages \href{http://ctan.org/pkg/subdocs}{\textsf{subdocs}}
and \href{http://ctan.org/pkg/subfiles}{\textsf{subfiles}}
provide structures in which the main and child documents can be
encapsulated and allowing them to be compiled individually.
The inclusion mechanism is different from the conventional |\include|.
\item
The package \href{http://ctan.org/pkg/combine}{\textsf{combine}}
is an elaborate solution to combine several documents into one.
\end{itemize}
%
See also the CTAN topic \href{http://ctan.org/topic/subdocs}{\textsf{subdocs}}
for further related packages.
The present package differs from the above solutions in that
a document structure constructed with the conventional |\include| mechanism
just needs two extra commands at the top of every file
such that all constituent files can be compiled individually.

%%%%%%%%%%%%%%%%%%%%%%%%%%%%%%%%%%%%%%%%%%%%%%%%%%%%%%%%%%%%%%%%%%%%%%%%%%%%%%%%
%\subsection{Feature Suggestions}
%
%The following is a list of features which may be useful for future
%versions of this package:
%%
%\begin{itemize}
%\item
%\ldots
%\end{itemize}

%%%%%%%%%%%%%%%%%%%%%%%%%%%%%%%%%%%%%%%%%%%%%%%%%%%%%%%%%%%%%%%%%%%%%%%%%%%%%%%%
\subsection{Revision History}

%%%%%%%%%%%%%%%%%%%%%%%%%%%%%%%%%%%%%%%%
\paragraph{v2.0:} 2018/12/30

\begin{itemize}
\item
immediate forward processing
\item
added |\childdocby| mechanism
\item
manual restructured
\end{itemize}

%%%%%%%%%%%%%%%%%%%%%%%%%%%%%%%%%%%%%%%%
\paragraph{v1.6:} 2018/01/17

\begin{itemize}
\item
application for development of include files
\item
corrections to manual
\end{itemize}

%%%%%%%%%%%%%%%%%%%%%%%%%%%%%%%%%%%%%%%%
\paragraph{v1.5:} 2017/05/21

\begin{itemize}
\item
more complete structuring introduced
\item
|\childdocof| introduced
\item
|\childdoc| renamed to |\childdocmain|
\item
|\childredirect| renamed to |\childdocforward| and |\childdocforwardprefix|
and functionality expanded
\end{itemize}

%%%%%%%%%%%%%%%%%%%%%%%%%%%%%%%%%%%%%%%%
\paragraph{v1.0:} 2017/04/27

\begin{itemize}
\item
manual and install package
\item
first version published on CTAN
\end{itemize}

%%%%%%%%%%%%%%%%%%%%%%%%%%%%%%%%%%%%%%%%
\paragraph{v0.6:} 2017/04/26

\begin{itemize}
\item
redirection mechanism added
\end{itemize}

%%%%%%%%%%%%%%%%%%%%%%%%%%%%%%%%%%%%%%%%
\paragraph{v0.5:} 2017/04/26

\begin{itemize}
\item
functionality in definition file
\end{itemize}


%%%%%%%%%%%%%%%%%%%%%%%%%%%%%%%%%%%%%%%%%%%%%%%%%%%%%%%%%%%%%%%%%%%%%%%%%%%%%%%%
%%%%%%%%%%%%%%%%%%%%%%%%%%%%%%%%%%%%%%%%%%%%%%%%%%%%%%%%%%%%%%%%%%%%%%%%%%%%%%%%
%%%%%%%%%%%%%%%%%%%%%%%%%%%%%%%%%%%%%%%%%%%%%%%%%%%%%%%%%%%%%%%%%%%%%%%%%%%%%%%%
\appendix

\settowidth\MacroIndent{\rmfamily\scriptsize 000\ }

 \DocInput{childdoc.dtx}

\end{document}
%</driver>
% \fi
%
% %%%%%%%%%%%%%%%%%%%%%%%%%%%%%%%%%%%%%%%%%%%%%%%%%%%%%%%%%%%%%%%%%%%%%%%%%%%%%%
% %%%%%%%%%%%%%%%%%%%%%%%%%%%%%%%%%%%%%%%%%%%%%%%%%%%%%%%%%%%%%%%%%%%%%%%%%%%%%%
% \section{Sample}
%\iffalse
%<*samplemain>
%\fi
%
% The following presents a sample document
% with two chapters, two parts, a title page,
% a compile flag as well as three forwarding files to set the flag.
% It consists of eight |.tex| files:
% \begin{center}
% \begin{tabular}{ll}
% |cdocsamp.tex|&main file\\
% |cdocsch1.tex|&include file for chapter 1\\
% |cdocsch2.tex|&include file for chapter 2\\
% |cdocspt3.tex|&include file for part 3\\
% |cdocspt4.tex|&include file for part 4\\
% |cdocsdrf.tex|&forwarding file for main file in draft mode\\
% |cdocsfi1.tex|&forwarding file for final version of chapter 1\\
% |cdocsfi2.tex|&forwarding file for final version of chapter 2\\
% \end{tabular}
% \end{center}
% Each of the eight files can be compiled directly by the \LaTeX{} compiler.
%
% %%%%%%%%%%%%%%%%%%%%%%%%%%%%%%%%%%%%%%
% \paragraph{Main File.}
%
% The main file is called |cdocsamp.tex|.
%
% Load the \textsf{childdoc} definitions and
% declare the filename for the main document:
%    \begin{macrocode}
\input{childdoc.def}
\childdocmain{}
%    \end{macrocode}

% Optional override for |\version| flag:
%    \begin{macrocode}
%%\ifchilddoc\else\providecommand{\version}{draft}\fi
%    \end{macrocode}

% Define the default values for the |\version| flag
% (|final| for the main file and |draft| for childs):
%    \begin{macrocode}
\ifchilddoc
\providecommand{\version}{draft}
\else
\providecommand{\version}{final}
\fi
%    \end{macrocode}

% Load the standard document class:
%    \begin{macrocode}
\documentclass[12pt]{article}
%    \end{macrocode}

% Start the document body:
%    \begin{macrocode}
\begin{document}
%    \end{macrocode}

% Declare a title page.
% Print title, part of document being processed and version flag:
%    \begin{macrocode}
\addtocounter{page}{-1}
\begin{center}
{\LARGE\bfseries{}childdoc example\par}
\vspace{1cm}
\ifchilddoc
\ifchilddocmanual part\else chapter\fi:
`\childdocname' of `\childdocjob'\par
\else
main document: `\childdocjob'\par
\fi
version: \version\par
\end{center}
\newpage
%    \end{macrocode}

% Manually include selected file,
% otherwise process as usual:
%    \begin{macrocode}
\ifchilddocmanual
\section*{part `\childdocname'}
\input{\childdocname}
\else
%    \end{macrocode}

% Include the two chapters:
%    \begin{macrocode}
\include{cdocsch1}
\include{cdocsch2}
%    \end{macrocode}

% Include the two parts unless only chapters should be displayed:
%    \begin{macrocode}
\ifchilddoc\else
\section{part three}
\input{cdocspt3}
\section{part four}
\input{cdocspt4}
\fi
%    \end{macrocode}

% Process as usual until here:
%    \begin{macrocode}
\fi
%    \end{macrocode}

% End of document body:
%    \begin{macrocode}
\end{document}
%    \end{macrocode}
%\iffalse
%</samplemain>
%\fi
%
% %%%%%%%%%%%%%%%%%%%%%%%%%%%%%%%%%%%%%%
% \paragraph{Chapter Include Files.}
%
% The include files are called |cdocsch1.tex| and |cdocsch2.tex|.
%
%\iffalse
%<*samplechap1|samplechap2>
%\fi

% Optional override for |\version| flag:
%    \begin{macrocode}
%%\providecommand{\version}{final}
%    \end{macrocode}

% Include the main document:
%    \begin{macrocode}
\input{childdoc.def}
\childdocof{cdocsamp}
%    \end{macrocode}

%\iffalse
%</samplechap1|samplechap2>
%\fi
%
%\iffalse
%<*samplechap1>
%\fi
% Some text for chapter 1:
%    \begin{macrocode}
\section{one}
some text in chapter one
%    \end{macrocode}

%\iffalse
%</samplechap1>
%\fi
% Some text for chapter 2:
%\iffalse
%<*samplechap2>
%\fi
%    \begin{macrocode}
\section{two}
more text in chapter two
%    \end{macrocode}

%\iffalse
%</samplechap2>
%\fi
%
% %%%%%%%%%%%%%%%%%%%%%%%%%%%%%%%%%%%%%%
% \paragraph{Part Include Files.}
%
% The include files are called |cdocspt3.tex| and |cdocspt4.tex|.
%
%\iffalse
%<*samplepart3|samplepart4>
%\fi

% Optional override for |\version| flag:
%    \begin{macrocode}
%%\providecommand{\version}{final}
%    \end{macrocode}

% Include the main document:
%    \begin{macrocode}
\input{childdoc.def}
\childdocby{cdocsamp}
%    \end{macrocode}

%\iffalse
%</samplepart3|samplepart4>
%\fi
%
%\iffalse
%<*samplepart3>
%\fi
% Some text for part 3:
%    \begin{macrocode}
some text in part three
%    \end{macrocode}

%\iffalse
%</samplepart3>
%\fi
% Some text for part 4:
%\iffalse
%<*samplepart4>
%\fi
%    \begin{macrocode}
more text in part four
%    \end{macrocode}

%\iffalse
%</samplepart4>
%\fi
%
% %%%%%%%%%%%%%%%%%%%%%%%%%%%%%%%%%%%%%%
% \paragraph{Forwarding for a Complete Draft.}
%
% The following forwarding file |cdocsdrf.tex|
% compiles the main document in draft mode:
%\iffalse
%<*sampledraft>
%\fi
%    \begin{macrocode}
\def\version{draft}
\input{childdoc.def}
\childdocforward{cdocsamp}
%    \end{macrocode}

%\iffalse
%</sampledraft>
%\fi
%
% %%%%%%%%%%%%%%%%%%%%%%%%%%%%%%%%%%%%%%
% \paragraph{Forwarding for Final Version of the Chapters.}
%
% The following forwarding files |cdocsfn1.tex| and |cdocsfn2.tex|
% (with identical content)
% compile the final versions of the child documents
% |cdocsch1.tex| and |cdocsch2.tex|, respectively:
%\iffalse
%<*samplefinal>
%\fi
%    \begin{macrocode}
\def\version{final}
\input{childdoc.def}
\childdocforwardprefix[cdocsamp]{cdocsfn}{cdocsch}
%    \end{macrocode}

%\iffalse
%</samplefinal>
%\fi
%
% %%%%%%%%%%%%%%%%%%%%%%%%%%%%%%%%%%%%%%
% \paragraph{Command Line Processing.}
%
% The following three command lines generate the output files
% |cdocscld|, |cdocscl1| and |cdocscl2|
% which should be identical to
% |cdocsdrf|, |cdocsch1| and |cdocsfn2|, respectively:
% \begin{center}
% \begin{tabular}{l}
% |latex -jobname cdocscld \|\\
% |  "\def\version{draft}\input{childdoc.def}\childdocforward{cdocsamp}"|\\
% |latex -jobname cdocscl1 \|\\
% |  "\input{childdoc.def}\childdocforward[cdocsamp]{cdocsch1}"|\\
% |latex -jobname cdocscl2 \|\\
% |  "\def\version{final}\input{childdoc.def}\childdocforward{cdocsch2}"|
% \end{tabular}
% \end{center}
% Note that the trailing backslash on each first line
% merely continues the input to the second line
% (for convenient cut ant paste).
% Furthermore, the command |latex| can be replaced by any
% of its alternative versions such as |pdflatex|.
%
% %%%%%%%%%%%%%%%%%%%%%%%%%%%%%%%%%%%%%%%%%%%%%%%%%%%%%%%%%%%%%%%%%%%%%%%%%%%%%%
% %%%%%%%%%%%%%%%%%%%%%%%%%%%%%%%%%%%%%%%%%%%%%%%%%%%%%%%%%%%%%%%%%%%%%%%%%%%%%%
% \section{Implementation}
%\iffalse
%<*package>
%\fi
%
% This section describes the definitions file |childdoc.def|.

% The definitions cannot be loaded using |\usepackage| or |\RequirePackage|
% which has a mechanism to prevent loading a style file more than once.
% When loading the definitions by means of |\input|
% multiple instances have to be prevented manually:
%\iffalse
%This code needs to be before the `\ProvidesFile' directive
%which is defined at the beginning of this file.
%Therefore it is also placed there and commented out here.
%</package>
%<*discard>
%\fi
%    \begin{macrocode}
\ifdefined\childdocmain\endinput\fi
%    \end{macrocode}
%\iffalse
%</discard>
%<*package>
%\fi
%
% \macro{\ifchilddoc}
% \macro{\ifchilddocmanual}
% The conditional |\ifchilddoc| tells whether a
% child (true) or main (false) document is being compiled.
% The conditional |\ifchilddocmanual| tells whether
% the |\includeonly| mechanism is used (false) or
% the selection of child files must be performed manually (true).
% The definitions initialise to false:
%    \begin{macrocode}
\newif\ifchilddoc
\newif\ifchilddocmanual
%    \end{macrocode}

% \macro{\childdocname}
% \macro{\childdocjob}
% The macro |\childdocname| stores the name of the main document
% to be compiled. The macro |\childdocjob| stores the name of
% the document on which the \LaTeX{} compiler was originally invoked.
% The content of |\jobname| cannot be compared
% to filenames specified in the source due to different catcodes.
% The following code rescans |\jobname|, stores the result
% in |\childdocname| and saves a copy in |\childdocjob|:
%    \begin{macrocode}
\edef\childdocname{\scantokens\expandafter{\jobname\noexpand}}
\let\childdocjob\childdocname
%    \end{macrocode}

% \macro{\childdocdisable}
% The macro |\childdocdisable| prevents the main file
% from being processed more than once.
% At this stage, the main document command |\childdocmain|
% is assumed to be called once again where it should do nothing.
% Any subsequent call to it should prevent
% a secondary processing of the main document
% It overwrites the forwarding commands
% |\childdocof| and |\childdocforward|
% with empty macros to prevent further inclusions of the main document:
%    \begin{macrocode}
\newcommand{\childdocdisable}
{
  \renewcommand{\childdocmain}[1]{\renewcommand{\childdocmain}[1]{\endinput}}
  \renewcommand{\childdocof}[1]{}
  \renewcommand{\childdocby}[2][]{}
  \renewcommand{\childdocforward}[2][]{}
  \renewcommand{\childdocdisable}{}
}
%    \end{macrocode}

% \macro{\childdocmain}
% The macro |\childdocmain| is to be called at the top of the main file
% with nothing or the main filename (without extension) as argument.
% First, it breaks loops.
% If the argument is not empty and does not match |\childdocname|
% (which is set by the first inclusion of |childdoc.def|),
% |\ifchilddoc| is set to true, |\includeonly| is applied to the child file
% and |\jobname| is set to the main file
% (for proper handling of |.aux| files):
%    \begin{macrocode}
\newcommand{\childdocmain}[1]
{
  \childdocdisable\childdocmain{}
  \if?#1?\else
    \begingroup
      \def\childdoctmp{#1}
      \ifx\childdoctmp\childdocname
        \def\childdoctmp{}
      \else
        \def\childdoctmp
        {
          \childdoctrue
          \includeonly{\childdocname}
          \def\childdocjob{#1}
          \def\jobname{#1}
        }
      \fi
      \expandafter
    \endgroup
    \childdoctmp
  \fi
}
%    \end{macrocode}

% \macro{\childdocof}
% The command |\childdocof| redirects
% compilation to the main file |#1|.
%    \begin{macrocode}
\newcommand{\childdocof}[1]
{
  \childdocdisable
  \childdoctrue
  \includeonly{\childdocname}
  \def\jobname{#1}
  \def\childdocjob{#1}
  \input{#1}
}
%    \end{macrocode}

% \macro{\childdocby}
% The command |\childdocby| ....
%    \begin{macrocode}
\newcommand{\childdocby}[2][]
{
  \childdocdisable
  \childdoctrue
  \childdocmanualtrue
  \if?#1?\else
    \def\jobname{#2}
  \fi
  \def\childdocjob{#2}
  \input{#2}
  \endinput
}
%    \end{macrocode}

% \macro{\childdocforward}
% The command |\childdocforward| redirects
% compilation to the main file or
% (if the optional argument is given) a child file.
% Parameters are set as if the main file
% or a child file starting with |\childdocof| was compiled.
% Then compilation is handed over to the main file:
%    \begin{macrocode}
\newcommand{\childdocforward}[2][]
{
  \begingroup
    \if?#1?
      \def\childdoctmp
      {
        \def\childdocname{#2}
        \def\childdocjob{#2}
        \def\jobname{#2}
        \input{#2}
        \endinput
      }
    \else
      \def\childdoctmp
      {
        \childdocdisable
        \def\childdocname{#2}
        \childdoctrue
        \includeonly{#2}
        \def\childdocjob{#1}
        \def\jobname{#1}
        \input{#1}
        \endinput
      }
    \fi
    \expandafter
  \endgroup
  \childdoctmp
}
%    \end{macrocode}

% \macro{\childdocforwardprefix}
% The command |\childdocforwardprefix| redirects
% compilation to the main or a child file by means of a pattern.
% The prefix |#1| in the current filename is replaced by |#2|
% and the suffix of the current filename is kept
% (it is assumed that the filename does not contain the substring `|~~~|'
% which is used as a delimiter).
% Compilation is handed over to the new file by |\childdocforward|:
%    \begin{macrocode}
\newcommand{\childdocforwardprefix}[3][]
{
  \begingroup
    \def\childdocextract #2##1~~~{\def\childdoctmp{\childdocforward[#1]{#3##1}}}
    \expandafter\childdocextract\childdocname~~~
    \expandafter
  \endgroup
  \childdoctmp
}
%    \end{macrocode}

% \macro{\childdoc}
% The deprecated macro |\childdoc| is a legacy version of |\childdocmain|:
%    \begin{macrocode}
\newcommand{\childdoc}{\childdocmain}
%    \end{macrocode}

% \macro{\childdocredirect}
% The deprecated macro |\childdocredirect| is a legacy version
% of |\childdocforward| and |\childdocforwardprefix|:
%    \begin{macrocode}
\newcommand{\childdocredirect}[2][]
{
  \begingroup
    \if?#1?
      \def\childdoctmp{\childdocforward{#2}}
    \else
      \def\childdoctmp{\childdocforwardprefix{#1}{#2}}
    \fi
    \expandafter
  \endgroup
  \childdoctmp
}
%    \end{macrocode}

%\iffalse
%</package>
%\fi
%
\endinput

\childdocmain{}
%    \end{macrocode}

% Optional override for |\version| flag:
%    \begin{macrocode}
%%\ifchilddoc\else\providecommand{\version}{draft}\fi
%    \end{macrocode}

% Define the default values for the |\version| flag
% (|final| for the main file and |draft| for childs):
%    \begin{macrocode}
\ifchilddoc
\providecommand{\version}{draft}
\else
\providecommand{\version}{final}
\fi
%    \end{macrocode}

% Load the standard document class:
%    \begin{macrocode}
\documentclass[12pt]{article}
%    \end{macrocode}

% Start the document body:
%    \begin{macrocode}
\begin{document}
%    \end{macrocode}

% Declare a title page.
% Print title, part of document being processed and version flag:
%    \begin{macrocode}
\addtocounter{page}{-1}
\begin{center}
{\LARGE\bfseries{}childdoc example\par}
\vspace{1cm}
\ifchilddoc
\ifchilddocmanual part\else chapter\fi:
`\childdocname' of `\childdocjob'\par
\else
main document: `\childdocjob'\par
\fi
version: \version\par
\end{center}
\newpage
%    \end{macrocode}

% Manually include selected file,
% otherwise process as usual:
%    \begin{macrocode}
\ifchilddocmanual
\section*{part `\childdocname'}
\input{\childdocname}
\else
%    \end{macrocode}

% Include the two chapters:
%    \begin{macrocode}
\include{cdocsch1}
\include{cdocsch2}
%    \end{macrocode}

% Include the two parts unless only chapters should be displayed:
%    \begin{macrocode}
\ifchilddoc\else
\section{part three}
\input{cdocspt3}
\section{part four}
\input{cdocspt4}
\fi
%    \end{macrocode}

% Process as usual until here:
%    \begin{macrocode}
\fi
%    \end{macrocode}

% End of document body:
%    \begin{macrocode}
\end{document}
%    \end{macrocode}
%\iffalse
%</samplemain>
%\fi
%
% %%%%%%%%%%%%%%%%%%%%%%%%%%%%%%%%%%%%%%
% \paragraph{Chapter Include Files.}
%
% The include files are called |cdocsch1.tex| and |cdocsch2.tex|.
%
%\iffalse
%<*samplechap1|samplechap2>
%\fi

% Optional override for |\version| flag:
%    \begin{macrocode}
%%\providecommand{\version}{final}
%    \end{macrocode}

% Include the main document:
%    \begin{macrocode}
% \iffalse
%
% childdoc.dtx Copyright (C) 2017-2018 Niklas Beisert
%
% This work may be distributed and/or modified under the
% conditions of the LaTeX Project Public License, either version 1.3
% of this license or (at your option) any later version.
% The latest version of this license is in
%   http://www.latex-project.org/lppl.txt
% and version 1.3 or later is part of all distributions of LaTeX
% version 2005/12/01 or later.
%
% This work has the LPPL maintenance status `maintained'.
%
% The Current Maintainer of this work is Niklas Beisert.
%
% This work consists of the files childdoc.dtx and childdoc.ins
% and the derived files childdoc.def and cdocsamp.tex with
% cdocsch1.tex, cdocsch2.tex, cdocsdrf.tex, cdocsfn1.tex, cdocsfn2.tex.
%
%<package>\ifdefined\childdocmain\endinput\fi
%<package>\ProvidesFile{childdoc.def}[2018/12/30 v2.0 child document driver]
%<samplemain>\ProvidesFile{cdocsamp.tex}[2018/12/30 v2.0 sample for childdoc]
%<*driver>
%\ProvidesFile{childdoc.drv}[2018/12/30 v2.0 childdoc reference manual file]
\PassOptionsToClass{10pt,a4paper}{article}
\documentclass{ltxdoc}

\usepackage[margin=35mm]{geometry}
\usepackage{hyperref}
\usepackage{hyperxmp}
\usepackage[usenames]{color}

\hypersetup{colorlinks=true}
\hypersetup{pdfstartview=FitH}
\hypersetup{pdfpagemode=UseNone}
\hypersetup{pdfsource={}}
\hypersetup{pdflang={en-UK}}
\hypersetup{pdfcopyright={Copyright 2017-2018 Niklas Beisert.
  This work may be distributed and/or modified under the
  conditions of the LaTeX Project Public License, either version 1.3
  of this license or (at your option) any later version.}}
\hypersetup{pdflicenseurl={http://www.latex-project.org/lppl.txt}}
\hypersetup{pdfcontactaddress={ETH Zurich, ITP, HIT K,
  Wolfgang-Pauli-Strasse 27}}
\hypersetup{pdfcontactpostcode={8093}}
\hypersetup{pdfcontactcity={Zurich}}
\hypersetup{pdfcontactcountry={Switzerland}}
\hypersetup{pdfcontactemail={nbeisert@itp.phys.ethz.ch}}
\hypersetup{pdfcontacturl={http://people.phys.ethz.ch/\xmptilde nbeisert/}}

\newcommand{\secref}[1]{\hyperref[#1]{section \ref*{#1}}}

\parskip1ex
\parindent0pt
\let\olditemize\itemize
\def\itemize{\olditemize\parskip0pt}

\begin{document}

\title{The \textsf{childdoc} Package}
\hypersetup{pdftitle={The childdoc Package}}
\author{Niklas Beisert\\[2ex]
  Institut f\"ur Theoretische Physik\\
  Eidgen\"ossische Technische Hochschule Z\"urich\\
  Wolfgang-Pauli-Strasse 27, 8093 Z\"urich, Switzerland\\[1ex]
  \href{mailto:nbeisert@itp.phys.ethz.ch}
  {\texttt{nbeisert@itp.phys.ethz.ch}}}
\hypersetup{pdfauthor={Niklas Beisert}}
\hypersetup{pdfsubject={Manual for the LaTeX2e Package childdoc}}
\date{30 December 2018, \textsf{v2.0}}
\maketitle

\begin{abstract}\noindent
\textsf{childdoc} is a \LaTeXe{} package
that enables the direct compilation
of document sections included by |\include|
to individual files.
\end{abstract}

\begingroup
\parskip0ex
\tableofcontents
\endgroup

%%%%%%%%%%%%%%%%%%%%%%%%%%%%%%%%%%%%%%%%%%%%%%%%%%%%%%%%%%%%%%%%%%%%%%%%%%%%%%%%
%%%%%%%%%%%%%%%%%%%%%%%%%%%%%%%%%%%%%%%%%%%%%%%%%%%%%%%%%%%%%%%%%%%%%%%%%%%%%%%%
\section{Introduction}

\LaTeX{} provides a mechanism to structure a large document (such as a book)
into a main file and several child files (containing the chapters)
using the |\include| command.
This mechanism is beneficial for documents
which span hundreds of pages in order to
make the source file(s) more manageable.
Moreover, compilation can be restricted to
selected child files by means of the |\includeonly| command.
The latter feature can be used to reduce the compilation time while editing
(this was significantly more useful in the earlier days of \LaTeX{})
or to generate a smaller document which is easier to navigate.
Another application of |\includeonly| is to generate
documents consisting of selected parts of the complete document.

However, there are a few drawbacks of the plain |\include| mechanism:
\begin{itemize}
\item
The child files cannot be compiled on their own,
they can only be compiled via the main file.
A naive editing environment
(such as a text editor with an option
to have the current file processed by \LaTeX)
may require one to switch to the main file before compiling;
attempting to compile the child file produces errors.
\item
The main file must be modified (each time)
to adjust the |\includeonly| command
to the present needs. This easily leaves the main file in a messy state.
\item
The generated document will always carry the filename
of the main document. This is inconvenient if
several child files are to be compiled and
to be kept for distribution.
\end{itemize}

The present package provides a simple interface
to make child files individually compilable by \LaTeX{}.
Compiling a child file then has the same effect as compiling
the main file with an |\includeonly| command
to select the appropriate child.
Moreover the generated document will carry the name of the child
rather than the main file.
This resolves all three above issues.

This feature is meant to make the editing of books,
thesis documents and lecture notes somewhat more convenient.
However, the package can also be used efficiently for
composing a series of documents (such as exercise sheets)
which are typically distributed individually.
It then assists the author in generating the individual documents
(potentially in different versions)
as well as a document containing the collected series.
Another application is in developing style files
or other kinds of included material
where compilation of the style file could redirect
to a sample or test file.

%%%%%%%%%%%%%%%%%%%%%%%%%%%%%%%%%%%%%%%%%%%%%%%%%%%%%%%%%%%%%%%%%%%%%%%%%%%%%%%%
%%%%%%%%%%%%%%%%%%%%%%%%%%%%%%%%%%%%%%%%%%%%%%%%%%%%%%%%%%%%%%%%%%%%%%%%%%%%%%%%
\section{Usage}

First of all, the package \textsf{childdoc} is \emph{not} a standard
\LaTeXe{} |.sty| style file! Therefore it needs to be invoked in
a non-standard way.

%%%%%%%%%%%%%%%%%%%%%%%%%%%%%%%%%%%%%%%%%%%%%%%%%%%%%%%%%%%%%%%%%%%%%%%%%%%%%%%%
\subsection{Included Files}
\label{sec:include}

%%%%%%%%%%%%%%%%%%%%%%%%%%%%%%%%%%%%%%%%
\DescribeMacro{\childdocmain}
To use the package, add the commands
\begin{center}
\begin{tabular}{l}
|\input{childdoc.def}|\\
|\childdocmain{}|\\
\end{tabular}
\end{center}
at the very top of the main \LaTeX{} file,
in particular \emph{before} the |\documentclass| statement!
The argument of |\childdocmain| should be left empty
(but it must be present).

%%%%%%%%%%%%%%%%%%%%%%%%%%%%%%%%%%%%%%%%
\DescribeMacro{\childdocof}
Furthermore, add the commands
\begin{center}
\begin{tabular}{l}
|\input{childdoc.def}|\\
|\childdocof{|\textit{main}|}|\\
\end{tabular}
\end{center}
at the top of every child file \textit{child}
which is included by |\include{|\textit{child}|}|
from within the main file
(or at least for those files to be compiled individually).
The argument \textit{main} must be the filename of the main file.

There are a couple of
considerations in setting up the main and child documents:

%%%%%%%%%%%%%%%%%%%%%%%%%%%%%%%%%%%%%%%%
\paragraph{Restrictions.}

Please note the following restrictions:
\begin{itemize}
\item
|\childdocmain| must be called with one argument \textit{main}
to ensure compatibility with earlier version of the package.
It must either be empty (|\childdocmain{}|)
or precisely match the filename of the main file in which it is specified.
See \secref{sec:detection} for further information.
\item
The filename \textit{main} must be specified without the |.tex| extension.
\item
The filename \textit{main} is case sensitive
(even in case-insensitive file systems)
due to internal string comparison.
\item
The argument \textit{main} should be fully expanded, it cannot be a macro.
\item
Subdirectories and special characters should be avoided in filenames.
\item
The command |\childdocmain{|\textit{main}|}| must be followed by a whitespace.
It should not be followed immediately by another command
or by a comment mark `|%|'.
This is because the \TeX{} parser reads the token immediately following
the argument of |\childdocmain| and puts it
at the beginning of every child section;
however, a white\-space is ignored.
\end{itemize}

%%%%%%%%%%%%%%%%%%%%%%%%%%%%%%%%%%%%%%%%
\paragraph{Content of Main File.}

It is advisable to place all content in the child files included by |\include|.
Any output contained in the main file will appear in all child documents
unless suppressed manually;
it cannot be suppressed automatically by the |\includeonly| directive
and thus should normally be avoided.
A method to include some content in the main file
by means of conditional processing is described in \secref{sec:conditional}.

%%%%%%%%%%%%%%%%%%%%%%%%%%%%%%%%%%%%%%%%
\paragraph{Page Numbering.}

When only a part of the document is compiled,
the appropriate numbering of pages
(as well as other status parameters)
is determined from the |.aux| files.
The latter contain information from previous passes.
However this information needs to propagate through
all intermediate child documents.
Therefore the page numbering in child documents may well
be inconsistent until the complete document is compiled at least once.

A useful (if unconventional) way to always ensure a consistent
page numbering is to restart the numbering in each child document
and denote the pages by `\textit{child}|.|\textit{page}'
where \textit{child} represents the chapter/section number of the child file.
This can be achieved by the command
|\numberwithin{page}{|\textit{child}|}|
of the \textsf{amsmath} package
where \textit{child} can be |chapter| or |section|
depending on the chosen structuring.
Alternatively, one can modify the macro |\thepage| appropriately
and reset the counter |page| at the start of each child file.

%%%%%%%%%%%%%%%%%%%%%%%%%%%%%%%%%%%%%%%%%%%%%%%%%%%%%%%%%%%%%%%%%%%%%%%%%%%%%%%%
\subsection{Conditional Processing}
\label{sec:conditional}

The package provides a mechanism to compile different versions
of a document. To customise the versions further some conditional processing
can come in handy to distinguish which version is being compiled.
The package provides two macros to describe the compilation context:

%%%%%%%%%%%%%%%%%%%%%%%%%%%%%%%%%%%%%%%%
\DescribeMacro{\ifchilddoc}
The conditional |\ifchilddoc| distinguishes between the compilation of
child documents and the main document:
%
\begin{center}
|\ifchilddoc |\textit{child-code}| |[|\||else |\textit{main-code}]| \||fi|
\end{center}

%%%%%%%%%%%%%%%%%%%%%%%%%%%%%%%%%%%%%%%%
\DescribeMacro{\childdocname}
\DescribeMacro{\childdocjob}
The macro |\childdocname| contains the filename (without extension)
of the main or child file being processed.
Note that |\childdocjob| will always contain the name of the main file.

%%%%%%%%%%%%%%%%%%%%%%%%%%%%%%%%%%%%%%%%
\paragraph{Title Page.}

Conditional processing can be used to include a title or banner page
in the main document when proper precautions are taken.
Importantly, the code in the main file should ensure that the page counter
(as well as other status parameters which are stored in the |.aux| files)
takes the same value after the conditional processing.
Otherwise the page numbers may take divergent values
depending on which part is compiled.

For example, a title page could be declared by:
%
\begin{center}
\begin{tabular}{l}
|\ifchilddoc\||else|\\
|\addtocounter{page}{-1}|\\
\textit{code for title page}\\
|\newpage|\\
|\||fi|
\end{tabular}
\end{center}
%
A banner page for the child documents can be generated by:
%
\begin{center}
\begin{tabular}{l}
|\ifchilddoc|\\
|\addtocounter{page}{-1}|\\
\textit{code for banner page}\\
|\newpage|\\
|\||fi|
\end{tabular}
\end{center}
%
Here one could write a message such as:
\begin{center}
|This is the part \childdocname{} of \childdocjob{}.|
\end{center}

%%%%%%%%%%%%%%%%%%%%%%%%%%%%%%%%%%%%%%%%%%%%%%%%%%%%%%%%%%%%%%%%%%%%%%%%%%%%%%%%
\subsection{Flags}
\label{sec:flags}

The package makes it easy to generate different versions
of the main or child documents.
To this end compilation flags can be defined
and assigned different default values.
They will be particularly useful in conjunction
with the forwarding mechanism described in \secref{sec:forward}.

For example, it may be useful to have a flag |\version|
which can be set to |draft| or |final|.
The document source will contain some conditional code
depending on the value of |\version|.
Suppose further, the flag should default to |final| for the main file
and to |draft| for child files
which is a natural assignment for editing the document.
This is achieved by placing the following code
in the preamble of the main document
(below the |\childdocmain| directive):
%
\begin{center}
\begin{tabular}{l}
|\ifchilddoc|\\
|\providecommand{\version}{draft}|\\
|\||else|\\
|\providecommand{\version}{final}|\\
|\||fi|
\end{tabular}
\end{center}
%
The definition by |\providecommand| makes sure
that previous definitions are not overwritten.
Further statements |\providecommand{\version}{...}|
can thus be added before the above code to override it.

For the main file, one might add a line
(between |\childdocmain| and the above block)
%
\begin{center}
|%\ifchilddoc\||else\providecommand{\version}{draft}\||fi|
\end{center}
%
which can be uncommented to produce a draft version.
Likewise one can add a line to the very top of a child file
(above the |\childdocof{|\textit{main}|}| directive)
%
\begin{center}
|%\providecommand{\version}{final}|
\end{center}
%
which can be uncommented to produce the final version of this child document.

%%%%%%%%%%%%%%%%%%%%%%%%%%%%%%%%%%%%%%%%%%%%%%%%%%%%%%%%%%%%%%%%%%%%%%%%%%%%%%%%
\subsection{Forwarding}
\label{sec:forward}

Different versions of the main or child documents
using compilation flags as described in \secref{sec:flags}
can be (permanently) stored in different files
for convenient compilation, viewing and distribution.
To this end, the package defines a command
to pass on compilation to a different file:

%%%%%%%%%%%%%%%%%%%%%%%%%%%%%%%%%%%%%%%%
\DescribeMacro{\childdocforward}
The command |\childdocforward| redirects processing to
another source file:
%
\begin{center}
\begin{tabular}{l}
|\input{childdoc.def}|\\
|\childdocforward[|\textit{main}|]{|\textit{dest}|}|\\
\end{tabular}
\end{center}
%
The argument \textit{dest} is the destination file
(without extension).
It should be the main file or one of the child files.
Note that further \textsf{childdoc} directives
such as |\childdocof| and |\childdocforward|
in the indicated file will be processed in this form.
The optional argument \textit{main}
passes on directly to the main file \textit{main}
while pretending to compile the child \textit{dest}.
This form behaves as if \textit{dest}
issues |\childdocof{|\textit{main}|}| right away,
and no further \textsf{childdoc} directives will be processed.

%%%%%%%%%%%%%%%%%%%%%%%%%%%%%%%%%%%%%%%%
\DescribeMacro{\...prefix}
In the alternative form |\childdocforwardprefix|,
%
\begin{center}
\begin{tabular}{l}
|\input{childdoc.def}|\\
|\childdocforwardprefix[|\textit{main}|]{|\textit{prefix}|}{|\textit{dest}|}|
\end{tabular}
\end{center}
%
the destination file is determined by a pattern
depending on the current file:
To make this work, the current file must be called
`{\textit{prefix}\hspace{0.2em}\textit{suffix}}'
with \textit{prefix} matching precisely the argument.
Processing is then passed on to the file
`{\textit{dest}\hspace{0.2em}\textit{suffix}}'.
Surely, the same effect is achieved by
directly specifying the
argument `{\textit{dest}\hspace{0.2em}\textit{suffix}}'
in the first form.
However, that requires to set up a different file
for each child. With the alternative form of the command
all these files can have exactly the same content
which simplifies setting them up and maintaining them.

For example, the following file |draft.tex|
with a compilation flag |\version| as described in \secref{sec:flags}
compiles the main document as a draft:
%
\begin{center}
\begin{tabular}{l}
|\def\version{draft}|\\
|\input{childdoc.def}|\\
|\childdocforward{|\textit{main}|}|
\end{tabular}
\end{center}
%
Likewise, the following files |final|\textit{nn}|.tex|
compile the final version of the child document
|child|\textit{nn}|.tex|:
%
\begin{center}
\begin{tabular}{l}
|\def\version{final}|\\
|\input{childdoc.def}|\\
|\childdocforwardprefix{final}{child}|
\end{tabular}
\end{center}
%

Note that when several versions of a main file and/or of each child file
are to be generated, it may be convenient to set up a |Makefile| or
shell script to automatise the process.

%%%%%%%%%%%%%%%%%%%%%%%%%%%%%%%%%%%%%%%%%%%%%%%%%%%%%%%%%%%%%%%%%%%%%%%%%%%%%%%%
\subsection{Command Line Processing}
\label{sec:commandline}

The effect of redirection files can also be achieved by invoking
the \LaTeX{} compiler with a more elaborate command line.
Most conveniently this should be done as part
of a shell script or a |Makefile|.

When using \textsf{childdoc} in the main file, the following
command lines effectively perform a redirection
(note that depending on the shell being used,
backslashes may have to be doubled: `|\|' $\to$ `|\\|'):
%
\begin{center}
|... -jobname "|\textit{target}|" |\\|"|[\textit{flags}]%
|\input{childdoc.def}\childdocforward[|\textit{main}|]{|\textit{dest}|}"|
\end{center}
%
Here \textit{target} is the name of the output file,
\textit{main} is the name of the main file
and \textit{dest} is the name of the main or child file to be processed
(all filenames without extensions).
The optional argument \textit{main} can be omitted
if \textit{main} matches \textit{dest}.
Optionally, compilation \textit{flags} can be defined via |\def| commands.
This command line makes the \TeX{} engine believe
it is compiling the file \textit{target}
whose content is specified as the latter parameter.
The provided code then forwards the processing to
\textit{main} or \textit{dest} as described in \secref{sec:forward}.

%%%%%%%%%%%%%%%%%%%%%%%%%%%%%%%%%%%%%%%%%%%%%%%%%%%%%%%%%%%%%%%%%%%%%%%%%%%%%%%%
\subsection{Include by Input}
\label{sec:input}

Including child documents by |\include| has some restrictions by design.
Most notably, the content of a child document always occupies
its own set of pages; pages cannot be shared between child documents.
Usually, this behaviour makes perfect sense
because each child document contain an essential part of the document.
However, in some situations it may be desirable to compose
a document from a collection of parts
without having mandatory page breaks between then.
For this case, the package
provides a mechanism to include parts
by |\input| which can also be processed individually.
However, by construction this mechanism
requires manual handling of the content to be output.

%%%%%%%%%%%%%%%%%%%%%%%%%%%%%%%%%%%%%%%%
\DescribeMacro{\ifchilddocmanual}
The main file should be prepared as usual, see \secref{sec:include}.
However, the document body must make a distinction
between processing of an individual part and of the main document, e.g.:
%
\begin{center}
\begin{tabular}{l}
|\ifchilddocmanual|\\
|\input{\childdocname}|\\
|\||else|\\
\textit{document body with }|\input{|\textit{part}|}|\\
|\||fi|
\end{tabular}
\end{center}
%
The conditional |\ifchilddocmanual| is true whenever
a part to be included by |\input| is being compiled,
and the name of the part is stored in |\childdocname|.

%%%%%%%%%%%%%%%%%%%%%%%%%%%%%%%%%%%%%%%%
\DescribeMacro{\childdocby}
Each part to be included by |\input| should start with:
%
\begin{center}
\begin{tabular}{l}
|\input{childdoc.def}|\\
|\childdocby{|\textit{main}|}|\\
\end{tabular}
\end{center}
%
The directive |\childdocby| is similar to |\childdocof|
described in \secref{sec:include},
but the subsequent selection of content must be done manually.
To that end, both |\ifchilddoc| and |\ifchilddocmanual|
will be true upon processing of a part,
and the name of the part is stored in |\childdocname|.
Note that |\jobname| will be set to the filename of the current part
so that each part receives an individual |.aux| file
that does not interfere with the |.aux| file(s) of the main document.
This behaviour can be altered by the alternative form
|\childdocby[*]{|\textit{main}|}| (with a non-empty optional argument)
which uses the |.aux| file of the main document
by setting |\jobname| to \textit{main}.

%%%%%%%%%%%%%%%%%%%%%%%%%%%%%%%%%%%%%%%%%%%%%%%%%%%%%%%%%%%%%%%%%%%%%%%%%%%%%%%%
\subsection{Driver Development}
\label{sec:driver}

The \textsf{childdoc} mechanism can also be use for the development
of definition files such as \LaTeX{} styles or classes.
This case differs from the above setup with multiple parts
included by |\include| in that no |\includeonly| should be invoked.
This can be achieved by starting the include file
(before |\ProvidesPackage|) with:
%
\begin{center}
\begin{tabular}{l}
|\input{childdoc.def}|\\
|\childdocforward{|\textit{main}|}|\\
\end{tabular}
\end{center}
%
or alternatively with:
%
\begin{center}
\begin{tabular}{l}
|\input{childdoc.def}|\\
|\childdocby{|\textit{main}|}|\\
\end{tabular}
\end{center}
%
Both forms have slightly different effects as described above.
The main file is prepared as usual, see \secref{sec:include}.

%%%%%%%%%%%%%%%%%%%%%%%%%%%%%%%%%%%%%%%%%%%%%%%%%%%%%%%%%%%%%%%%%%%%%%%%%%%%%%%%
\subsection{Legacy Detection}
\label{sec:detection}

The directive |\childdocmain| in the main file can detect
whether the complete document or merely a child is to be compiled
even without using the directive |\childdocof|.
This method is deprecated because it is less robust
and there is no compelling reason to use it;
it is merely provided for backward compatibility
and it may be removed in future versions.

If the detection mechanism is to be used,
it is mandatory to correctly specify
the filename of the main file as the argument of |\childdocmain|:
%
\begin{center}
\begin{tabular}{l}
|\input{childdoc.def}|\\
|\childdocmain{|\textit{main}|}|\\
\end{tabular}
\end{center}
%
If |\jobname| does not match the argument \textit{main} of |\childdocmain|,
it is assumed that |\jobname| points to the child file to be compiled.
When using |\childdocmain| with the main file specified as argument,
it suffices to start a child file
with just |\input{|\textit{main}|}|
without loading of the package and using |\childdocof|.
If instead all processing is done
with the appropriate \textsf{childdoc} directives,
the argument of \textit{main} of |\childdocmain| can be empty.

An alternative version of the command line processing described
in \secref{sec:commandline} using the detection mechanism reads:
%
\begin{center}
|... -jobname "|\textit{target}|" "|[\textit{flags}]%
[|\def\jobname{|\textit{dest}|}|]|\input{|\textit{main}|}"|
\end{center}

%%%%%%%%%%%%%%%%%%%%%%%%%%%%%%%%%%%%%%%%%%%%%%%%%%%%%%%%%%%%%%%%%%%%%%%%%%%%%%%%
\subsection{Manual Code}
\label{sec:manual}

In case one cannot be certain whether the definitions file |childdoc.def|
is installed on the target \TeX{} distribution
and one prefers not to ship it,
it is conceivable to paste a few relevant commands into the sources.

To that end, drop all statements |\input{childdoc.def}|
and perform the replacements as outlined below.
Instead of |\childdocmain{|\textit{main}|}| add the following code
to the top of the main file:
%
\begin{center}
\begin{tabular}{l}
|\||ifdefined\childdocname\endinput\||fi\newif\ifchilddoc|\\
|\edef\childdocname{\scantokens\expandafter{\jobname\noexpand}}|\\
|\def\childdocmain{|\textit{main}|}\||ifx\childdocmain\childdocname\||else|\\
|\childdoctrue\includeonly{\childdocname}\let\jobname\childdocmain\||fi|\\
\end{tabular}
\end{center}
%
Instead of |\childdocof{|\textit{main}|}| just include the main file
at the top of each child file:
%
\begin{center}
|\input{|\textit{main}|}|
\end{center}
%
A simple redirection |\childdocforward{|\textit{dest}|}| is achieved by:
%
\begin{center}
|\def\jobname{|\textit{dest}|}\input{\jobname}|
\end{center}
%
The redirection with prefix
|\childdocforwardprefix[|\textit{prefix}|]{|\textit{dest}|}|
is accomplished by:
%
\begin{center}
\begin{tabular}{l}
|{\edef\jobname{\scantokens\expandafter{\jobname\noexpand}}|\\
|\def\redirectjob |\textit{prefix}|#1~~~{\gdef\jobname{|\textit{dest}|#1}}|\\
|\expandafter\redirectjob\jobname~~~}\input{\jobname}|
\end{tabular}
\end{center}

In an alternative approach,
child documents can be compiled by a specific command line
without additional code or specific definitions:
%
\begin{center}
|... -jobname "|\textit{target}|" "|[\textit{flags}]%
|\includeonly{|\textit{dest}|}\input{|\textit{main}|}"|
\end{center}
%

%%%%%%%%%%%%%%%%%%%%%%%%%%%%%%%%%%%%%%%%%%%%%%%%%%%%%%%%%%%%%%%%%%%%%%%%%%%%%%%%
%%%%%%%%%%%%%%%%%%%%%%%%%%%%%%%%%%%%%%%%%%%%%%%%%%%%%%%%%%%%%%%%%%%%%%%%%%%%%%%%
\section{Information}

%%%%%%%%%%%%%%%%%%%%%%%%%%%%%%%%%%%%%%%%%%%%%%%%%%%%%%%%%%%%%%%%%%%%%%%%%%%%%%%%
\subsection{Copyright}

Copyright \copyright{} 2017--2018 Niklas Beisert

This work may be distributed and/or modified under the
conditions of the \LaTeX{} Project Public License, either version 1.3
of this license or (at your option) any later version.
The latest version of this license is in
  \url{http://www.latex-project.org/lppl.txt}
and version 1.3 or later is part of all distributions of \LaTeX{}
version 2005/12/01 or later.

This work has the LPPL maintenance status `maintained'.

The Current Maintainer of this work is Niklas Beisert.

This work consists of the files |README.txt|, |childdoc.ins| and |childdoc.dtx|
as well as the derived files |childdoc.def|, |cdocsamp.tex|
with |cdocsch1.tex|, |cdocsch2.tex|, |cdocspt3.tex|, |cdocspt4.tex|,
|cdocsdrf.tex|, |cdocsfn1.tex|, |cdocsfn2.tex|
as well as |childdoc.pdf|.

%%%%%%%%%%%%%%%%%%%%%%%%%%%%%%%%%%%%%%%%%%%%%%%%%%%%%%%%%%%%%%%%%%%%%%%%%%%%%%%%
\subsection{Files and Installation}

The package consists of the files:
%
\begin{center}
\begin{tabular}{ll}
    |README.txt|   & readme file \\
    |childdoc.ins| & installation file \\
    |childdoc.dtx| & source file \\
    |childdoc.def| & definition file \\
    |cdocsamp.tex| & sample main file \\
    |cdocsch1.tex| & sample include file \\
    |cdocsch2.tex| & sample include file \\
    |cdocspt3.tex| & sample part file \\
    |cdocspt4.tex| & sample part file \\
    |cdocsdrf.tex| & sample redirection file \\
    |cdocsfn1.tex| & sample redirection file \\
    |cdocsfn2.tex| & sample redirection file \\
    |childdoc.pdf| & manual
\end{tabular}
\end{center}
%
The distribution consists of the files
|README.txt|, |childdoc.ins| and |childdoc.dtx|.
%
\begin{itemize}
\item
Run (pdf)\LaTeX{} on |childdoc.dtx|
to compile the manual |childdoc.pdf| (this file).
\item
Run \LaTeX{} on |childdoc.ins| to create the definitions file |childdoc.def|
and the sample |cdocsamp.tex| with include files
|cdocsch1.tex|, |cdocsch2.tex|, |cdocspt3.tex|, |cdocspt4.tex|,
|cdocsdrf.tex|, |cdocsfn1.tex|, |cdocsfn2.tex|.
Then copy the file |childdoc.def| to an appropriate directory of your \LaTeX{}
distribution, e.g.\ \textit{texmf-root}|/tex/latex/childdoc|.
\end{itemize}

%%%%%%%%%%%%%%%%%%%%%%%%%%%%%%%%%%%%%%%%%%%%%%%%%%%%%%%%%%%%%%%%%%%%%%%%%%%%%%%%
\subsection{Related CTAN Packages}

There are several other packages which offer a similar functionality:
%
\begin{itemize}
\item
The packages
\href{http://ctan.org/pkg/docmute}{\textsf{docmute}},
\href{http://ctan.org/pkg/includex}{\textsf{includex}} and
\href{http://ctan.org/pkg/standalone}{\textsf{standalone}}
provide commands to include only the document body of
a child file thus allowing both files to be compiled individually.
\item
The packages \href{http://ctan.org/pkg/subdocs}{\textsf{subdocs}}
and \href{http://ctan.org/pkg/subfiles}{\textsf{subfiles}}
provide structures in which the main and child documents can be
encapsulated and allowing them to be compiled individually.
The inclusion mechanism is different from the conventional |\include|.
\item
The package \href{http://ctan.org/pkg/combine}{\textsf{combine}}
is an elaborate solution to combine several documents into one.
\end{itemize}
%
See also the CTAN topic \href{http://ctan.org/topic/subdocs}{\textsf{subdocs}}
for further related packages.
The present package differs from the above solutions in that
a document structure constructed with the conventional |\include| mechanism
just needs two extra commands at the top of every file
such that all constituent files can be compiled individually.

%%%%%%%%%%%%%%%%%%%%%%%%%%%%%%%%%%%%%%%%%%%%%%%%%%%%%%%%%%%%%%%%%%%%%%%%%%%%%%%%
%\subsection{Feature Suggestions}
%
%The following is a list of features which may be useful for future
%versions of this package:
%%
%\begin{itemize}
%\item
%\ldots
%\end{itemize}

%%%%%%%%%%%%%%%%%%%%%%%%%%%%%%%%%%%%%%%%%%%%%%%%%%%%%%%%%%%%%%%%%%%%%%%%%%%%%%%%
\subsection{Revision History}

%%%%%%%%%%%%%%%%%%%%%%%%%%%%%%%%%%%%%%%%
\paragraph{v2.0:} 2018/12/30

\begin{itemize}
\item
immediate forward processing
\item
added |\childdocby| mechanism
\item
manual restructured
\end{itemize}

%%%%%%%%%%%%%%%%%%%%%%%%%%%%%%%%%%%%%%%%
\paragraph{v1.6:} 2018/01/17

\begin{itemize}
\item
application for development of include files
\item
corrections to manual
\end{itemize}

%%%%%%%%%%%%%%%%%%%%%%%%%%%%%%%%%%%%%%%%
\paragraph{v1.5:} 2017/05/21

\begin{itemize}
\item
more complete structuring introduced
\item
|\childdocof| introduced
\item
|\childdoc| renamed to |\childdocmain|
\item
|\childredirect| renamed to |\childdocforward| and |\childdocforwardprefix|
and functionality expanded
\end{itemize}

%%%%%%%%%%%%%%%%%%%%%%%%%%%%%%%%%%%%%%%%
\paragraph{v1.0:} 2017/04/27

\begin{itemize}
\item
manual and install package
\item
first version published on CTAN
\end{itemize}

%%%%%%%%%%%%%%%%%%%%%%%%%%%%%%%%%%%%%%%%
\paragraph{v0.6:} 2017/04/26

\begin{itemize}
\item
redirection mechanism added
\end{itemize}

%%%%%%%%%%%%%%%%%%%%%%%%%%%%%%%%%%%%%%%%
\paragraph{v0.5:} 2017/04/26

\begin{itemize}
\item
functionality in definition file
\end{itemize}


%%%%%%%%%%%%%%%%%%%%%%%%%%%%%%%%%%%%%%%%%%%%%%%%%%%%%%%%%%%%%%%%%%%%%%%%%%%%%%%%
%%%%%%%%%%%%%%%%%%%%%%%%%%%%%%%%%%%%%%%%%%%%%%%%%%%%%%%%%%%%%%%%%%%%%%%%%%%%%%%%
%%%%%%%%%%%%%%%%%%%%%%%%%%%%%%%%%%%%%%%%%%%%%%%%%%%%%%%%%%%%%%%%%%%%%%%%%%%%%%%%
\appendix

\settowidth\MacroIndent{\rmfamily\scriptsize 000\ }

 \DocInput{childdoc.dtx}

\end{document}
%</driver>
% \fi
%
% %%%%%%%%%%%%%%%%%%%%%%%%%%%%%%%%%%%%%%%%%%%%%%%%%%%%%%%%%%%%%%%%%%%%%%%%%%%%%%
% %%%%%%%%%%%%%%%%%%%%%%%%%%%%%%%%%%%%%%%%%%%%%%%%%%%%%%%%%%%%%%%%%%%%%%%%%%%%%%
% \section{Sample}
%\iffalse
%<*samplemain>
%\fi
%
% The following presents a sample document
% with two chapters, two parts, a title page,
% a compile flag as well as three forwarding files to set the flag.
% It consists of eight |.tex| files:
% \begin{center}
% \begin{tabular}{ll}
% |cdocsamp.tex|&main file\\
% |cdocsch1.tex|&include file for chapter 1\\
% |cdocsch2.tex|&include file for chapter 2\\
% |cdocspt3.tex|&include file for part 3\\
% |cdocspt4.tex|&include file for part 4\\
% |cdocsdrf.tex|&forwarding file for main file in draft mode\\
% |cdocsfi1.tex|&forwarding file for final version of chapter 1\\
% |cdocsfi2.tex|&forwarding file for final version of chapter 2\\
% \end{tabular}
% \end{center}
% Each of the eight files can be compiled directly by the \LaTeX{} compiler.
%
% %%%%%%%%%%%%%%%%%%%%%%%%%%%%%%%%%%%%%%
% \paragraph{Main File.}
%
% The main file is called |cdocsamp.tex|.
%
% Load the \textsf{childdoc} definitions and
% declare the filename for the main document:
%    \begin{macrocode}
\input{childdoc.def}
\childdocmain{}
%    \end{macrocode}

% Optional override for |\version| flag:
%    \begin{macrocode}
%%\ifchilddoc\else\providecommand{\version}{draft}\fi
%    \end{macrocode}

% Define the default values for the |\version| flag
% (|final| for the main file and |draft| for childs):
%    \begin{macrocode}
\ifchilddoc
\providecommand{\version}{draft}
\else
\providecommand{\version}{final}
\fi
%    \end{macrocode}

% Load the standard document class:
%    \begin{macrocode}
\documentclass[12pt]{article}
%    \end{macrocode}

% Start the document body:
%    \begin{macrocode}
\begin{document}
%    \end{macrocode}

% Declare a title page.
% Print title, part of document being processed and version flag:
%    \begin{macrocode}
\addtocounter{page}{-1}
\begin{center}
{\LARGE\bfseries{}childdoc example\par}
\vspace{1cm}
\ifchilddoc
\ifchilddocmanual part\else chapter\fi:
`\childdocname' of `\childdocjob'\par
\else
main document: `\childdocjob'\par
\fi
version: \version\par
\end{center}
\newpage
%    \end{macrocode}

% Manually include selected file,
% otherwise process as usual:
%    \begin{macrocode}
\ifchilddocmanual
\section*{part `\childdocname'}
\input{\childdocname}
\else
%    \end{macrocode}

% Include the two chapters:
%    \begin{macrocode}
\include{cdocsch1}
\include{cdocsch2}
%    \end{macrocode}

% Include the two parts unless only chapters should be displayed:
%    \begin{macrocode}
\ifchilddoc\else
\section{part three}
\input{cdocspt3}
\section{part four}
\input{cdocspt4}
\fi
%    \end{macrocode}

% Process as usual until here:
%    \begin{macrocode}
\fi
%    \end{macrocode}

% End of document body:
%    \begin{macrocode}
\end{document}
%    \end{macrocode}
%\iffalse
%</samplemain>
%\fi
%
% %%%%%%%%%%%%%%%%%%%%%%%%%%%%%%%%%%%%%%
% \paragraph{Chapter Include Files.}
%
% The include files are called |cdocsch1.tex| and |cdocsch2.tex|.
%
%\iffalse
%<*samplechap1|samplechap2>
%\fi

% Optional override for |\version| flag:
%    \begin{macrocode}
%%\providecommand{\version}{final}
%    \end{macrocode}

% Include the main document:
%    \begin{macrocode}
\input{childdoc.def}
\childdocof{cdocsamp}
%    \end{macrocode}

%\iffalse
%</samplechap1|samplechap2>
%\fi
%
%\iffalse
%<*samplechap1>
%\fi
% Some text for chapter 1:
%    \begin{macrocode}
\section{one}
some text in chapter one
%    \end{macrocode}

%\iffalse
%</samplechap1>
%\fi
% Some text for chapter 2:
%\iffalse
%<*samplechap2>
%\fi
%    \begin{macrocode}
\section{two}
more text in chapter two
%    \end{macrocode}

%\iffalse
%</samplechap2>
%\fi
%
% %%%%%%%%%%%%%%%%%%%%%%%%%%%%%%%%%%%%%%
% \paragraph{Part Include Files.}
%
% The include files are called |cdocspt3.tex| and |cdocspt4.tex|.
%
%\iffalse
%<*samplepart3|samplepart4>
%\fi

% Optional override for |\version| flag:
%    \begin{macrocode}
%%\providecommand{\version}{final}
%    \end{macrocode}

% Include the main document:
%    \begin{macrocode}
\input{childdoc.def}
\childdocby{cdocsamp}
%    \end{macrocode}

%\iffalse
%</samplepart3|samplepart4>
%\fi
%
%\iffalse
%<*samplepart3>
%\fi
% Some text for part 3:
%    \begin{macrocode}
some text in part three
%    \end{macrocode}

%\iffalse
%</samplepart3>
%\fi
% Some text for part 4:
%\iffalse
%<*samplepart4>
%\fi
%    \begin{macrocode}
more text in part four
%    \end{macrocode}

%\iffalse
%</samplepart4>
%\fi
%
% %%%%%%%%%%%%%%%%%%%%%%%%%%%%%%%%%%%%%%
% \paragraph{Forwarding for a Complete Draft.}
%
% The following forwarding file |cdocsdrf.tex|
% compiles the main document in draft mode:
%\iffalse
%<*sampledraft>
%\fi
%    \begin{macrocode}
\def\version{draft}
\input{childdoc.def}
\childdocforward{cdocsamp}
%    \end{macrocode}

%\iffalse
%</sampledraft>
%\fi
%
% %%%%%%%%%%%%%%%%%%%%%%%%%%%%%%%%%%%%%%
% \paragraph{Forwarding for Final Version of the Chapters.}
%
% The following forwarding files |cdocsfn1.tex| and |cdocsfn2.tex|
% (with identical content)
% compile the final versions of the child documents
% |cdocsch1.tex| and |cdocsch2.tex|, respectively:
%\iffalse
%<*samplefinal>
%\fi
%    \begin{macrocode}
\def\version{final}
\input{childdoc.def}
\childdocforwardprefix[cdocsamp]{cdocsfn}{cdocsch}
%    \end{macrocode}

%\iffalse
%</samplefinal>
%\fi
%
% %%%%%%%%%%%%%%%%%%%%%%%%%%%%%%%%%%%%%%
% \paragraph{Command Line Processing.}
%
% The following three command lines generate the output files
% |cdocscld|, |cdocscl1| and |cdocscl2|
% which should be identical to
% |cdocsdrf|, |cdocsch1| and |cdocsfn2|, respectively:
% \begin{center}
% \begin{tabular}{l}
% |latex -jobname cdocscld \|\\
% |  "\def\version{draft}\input{childdoc.def}\childdocforward{cdocsamp}"|\\
% |latex -jobname cdocscl1 \|\\
% |  "\input{childdoc.def}\childdocforward[cdocsamp]{cdocsch1}"|\\
% |latex -jobname cdocscl2 \|\\
% |  "\def\version{final}\input{childdoc.def}\childdocforward{cdocsch2}"|
% \end{tabular}
% \end{center}
% Note that the trailing backslash on each first line
% merely continues the input to the second line
% (for convenient cut ant paste).
% Furthermore, the command |latex| can be replaced by any
% of its alternative versions such as |pdflatex|.
%
% %%%%%%%%%%%%%%%%%%%%%%%%%%%%%%%%%%%%%%%%%%%%%%%%%%%%%%%%%%%%%%%%%%%%%%%%%%%%%%
% %%%%%%%%%%%%%%%%%%%%%%%%%%%%%%%%%%%%%%%%%%%%%%%%%%%%%%%%%%%%%%%%%%%%%%%%%%%%%%
% \section{Implementation}
%\iffalse
%<*package>
%\fi
%
% This section describes the definitions file |childdoc.def|.

% The definitions cannot be loaded using |\usepackage| or |\RequirePackage|
% which has a mechanism to prevent loading a style file more than once.
% When loading the definitions by means of |\input|
% multiple instances have to be prevented manually:
%\iffalse
%This code needs to be before the `\ProvidesFile' directive
%which is defined at the beginning of this file.
%Therefore it is also placed there and commented out here.
%</package>
%<*discard>
%\fi
%    \begin{macrocode}
\ifdefined\childdocmain\endinput\fi
%    \end{macrocode}
%\iffalse
%</discard>
%<*package>
%\fi
%
% \macro{\ifchilddoc}
% \macro{\ifchilddocmanual}
% The conditional |\ifchilddoc| tells whether a
% child (true) or main (false) document is being compiled.
% The conditional |\ifchilddocmanual| tells whether
% the |\includeonly| mechanism is used (false) or
% the selection of child files must be performed manually (true).
% The definitions initialise to false:
%    \begin{macrocode}
\newif\ifchilddoc
\newif\ifchilddocmanual
%    \end{macrocode}

% \macro{\childdocname}
% \macro{\childdocjob}
% The macro |\childdocname| stores the name of the main document
% to be compiled. The macro |\childdocjob| stores the name of
% the document on which the \LaTeX{} compiler was originally invoked.
% The content of |\jobname| cannot be compared
% to filenames specified in the source due to different catcodes.
% The following code rescans |\jobname|, stores the result
% in |\childdocname| and saves a copy in |\childdocjob|:
%    \begin{macrocode}
\edef\childdocname{\scantokens\expandafter{\jobname\noexpand}}
\let\childdocjob\childdocname
%    \end{macrocode}

% \macro{\childdocdisable}
% The macro |\childdocdisable| prevents the main file
% from being processed more than once.
% At this stage, the main document command |\childdocmain|
% is assumed to be called once again where it should do nothing.
% Any subsequent call to it should prevent
% a secondary processing of the main document
% It overwrites the forwarding commands
% |\childdocof| and |\childdocforward|
% with empty macros to prevent further inclusions of the main document:
%    \begin{macrocode}
\newcommand{\childdocdisable}
{
  \renewcommand{\childdocmain}[1]{\renewcommand{\childdocmain}[1]{\endinput}}
  \renewcommand{\childdocof}[1]{}
  \renewcommand{\childdocby}[2][]{}
  \renewcommand{\childdocforward}[2][]{}
  \renewcommand{\childdocdisable}{}
}
%    \end{macrocode}

% \macro{\childdocmain}
% The macro |\childdocmain| is to be called at the top of the main file
% with nothing or the main filename (without extension) as argument.
% First, it breaks loops.
% If the argument is not empty and does not match |\childdocname|
% (which is set by the first inclusion of |childdoc.def|),
% |\ifchilddoc| is set to true, |\includeonly| is applied to the child file
% and |\jobname| is set to the main file
% (for proper handling of |.aux| files):
%    \begin{macrocode}
\newcommand{\childdocmain}[1]
{
  \childdocdisable\childdocmain{}
  \if?#1?\else
    \begingroup
      \def\childdoctmp{#1}
      \ifx\childdoctmp\childdocname
        \def\childdoctmp{}
      \else
        \def\childdoctmp
        {
          \childdoctrue
          \includeonly{\childdocname}
          \def\childdocjob{#1}
          \def\jobname{#1}
        }
      \fi
      \expandafter
    \endgroup
    \childdoctmp
  \fi
}
%    \end{macrocode}

% \macro{\childdocof}
% The command |\childdocof| redirects
% compilation to the main file |#1|.
%    \begin{macrocode}
\newcommand{\childdocof}[1]
{
  \childdocdisable
  \childdoctrue
  \includeonly{\childdocname}
  \def\jobname{#1}
  \def\childdocjob{#1}
  \input{#1}
}
%    \end{macrocode}

% \macro{\childdocby}
% The command |\childdocby| ....
%    \begin{macrocode}
\newcommand{\childdocby}[2][]
{
  \childdocdisable
  \childdoctrue
  \childdocmanualtrue
  \if?#1?\else
    \def\jobname{#2}
  \fi
  \def\childdocjob{#2}
  \input{#2}
  \endinput
}
%    \end{macrocode}

% \macro{\childdocforward}
% The command |\childdocforward| redirects
% compilation to the main file or
% (if the optional argument is given) a child file.
% Parameters are set as if the main file
% or a child file starting with |\childdocof| was compiled.
% Then compilation is handed over to the main file:
%    \begin{macrocode}
\newcommand{\childdocforward}[2][]
{
  \begingroup
    \if?#1?
      \def\childdoctmp
      {
        \def\childdocname{#2}
        \def\childdocjob{#2}
        \def\jobname{#2}
        \input{#2}
        \endinput
      }
    \else
      \def\childdoctmp
      {
        \childdocdisable
        \def\childdocname{#2}
        \childdoctrue
        \includeonly{#2}
        \def\childdocjob{#1}
        \def\jobname{#1}
        \input{#1}
        \endinput
      }
    \fi
    \expandafter
  \endgroup
  \childdoctmp
}
%    \end{macrocode}

% \macro{\childdocforwardprefix}
% The command |\childdocforwardprefix| redirects
% compilation to the main or a child file by means of a pattern.
% The prefix |#1| in the current filename is replaced by |#2|
% and the suffix of the current filename is kept
% (it is assumed that the filename does not contain the substring `|~~~|'
% which is used as a delimiter).
% Compilation is handed over to the new file by |\childdocforward|:
%    \begin{macrocode}
\newcommand{\childdocforwardprefix}[3][]
{
  \begingroup
    \def\childdocextract #2##1~~~{\def\childdoctmp{\childdocforward[#1]{#3##1}}}
    \expandafter\childdocextract\childdocname~~~
    \expandafter
  \endgroup
  \childdoctmp
}
%    \end{macrocode}

% \macro{\childdoc}
% The deprecated macro |\childdoc| is a legacy version of |\childdocmain|:
%    \begin{macrocode}
\newcommand{\childdoc}{\childdocmain}
%    \end{macrocode}

% \macro{\childdocredirect}
% The deprecated macro |\childdocredirect| is a legacy version
% of |\childdocforward| and |\childdocforwardprefix|:
%    \begin{macrocode}
\newcommand{\childdocredirect}[2][]
{
  \begingroup
    \if?#1?
      \def\childdoctmp{\childdocforward{#2}}
    \else
      \def\childdoctmp{\childdocforwardprefix{#1}{#2}}
    \fi
    \expandafter
  \endgroup
  \childdoctmp
}
%    \end{macrocode}

%\iffalse
%</package>
%\fi
%
\endinput

\childdocof{cdocsamp}
%    \end{macrocode}

%\iffalse
%</samplechap1|samplechap2>
%\fi
%
%\iffalse
%<*samplechap1>
%\fi
% Some text for chapter 1:
%    \begin{macrocode}
\section{one}
some text in chapter one
%    \end{macrocode}

%\iffalse
%</samplechap1>
%\fi
% Some text for chapter 2:
%\iffalse
%<*samplechap2>
%\fi
%    \begin{macrocode}
\section{two}
more text in chapter two
%    \end{macrocode}

%\iffalse
%</samplechap2>
%\fi
%
% %%%%%%%%%%%%%%%%%%%%%%%%%%%%%%%%%%%%%%
% \paragraph{Part Include Files.}
%
% The include files are called |cdocspt3.tex| and |cdocspt4.tex|.
%
%\iffalse
%<*samplepart3|samplepart4>
%\fi

% Optional override for |\version| flag:
%    \begin{macrocode}
%%\providecommand{\version}{final}
%    \end{macrocode}

% Include the main document:
%    \begin{macrocode}
% \iffalse
%
% childdoc.dtx Copyright (C) 2017-2018 Niklas Beisert
%
% This work may be distributed and/or modified under the
% conditions of the LaTeX Project Public License, either version 1.3
% of this license or (at your option) any later version.
% The latest version of this license is in
%   http://www.latex-project.org/lppl.txt
% and version 1.3 or later is part of all distributions of LaTeX
% version 2005/12/01 or later.
%
% This work has the LPPL maintenance status `maintained'.
%
% The Current Maintainer of this work is Niklas Beisert.
%
% This work consists of the files childdoc.dtx and childdoc.ins
% and the derived files childdoc.def and cdocsamp.tex with
% cdocsch1.tex, cdocsch2.tex, cdocsdrf.tex, cdocsfn1.tex, cdocsfn2.tex.
%
%<package>\ifdefined\childdocmain\endinput\fi
%<package>\ProvidesFile{childdoc.def}[2018/12/30 v2.0 child document driver]
%<samplemain>\ProvidesFile{cdocsamp.tex}[2018/12/30 v2.0 sample for childdoc]
%<*driver>
%\ProvidesFile{childdoc.drv}[2018/12/30 v2.0 childdoc reference manual file]
\PassOptionsToClass{10pt,a4paper}{article}
\documentclass{ltxdoc}

\usepackage[margin=35mm]{geometry}
\usepackage{hyperref}
\usepackage{hyperxmp}
\usepackage[usenames]{color}

\hypersetup{colorlinks=true}
\hypersetup{pdfstartview=FitH}
\hypersetup{pdfpagemode=UseNone}
\hypersetup{pdfsource={}}
\hypersetup{pdflang={en-UK}}
\hypersetup{pdfcopyright={Copyright 2017-2018 Niklas Beisert.
  This work may be distributed and/or modified under the
  conditions of the LaTeX Project Public License, either version 1.3
  of this license or (at your option) any later version.}}
\hypersetup{pdflicenseurl={http://www.latex-project.org/lppl.txt}}
\hypersetup{pdfcontactaddress={ETH Zurich, ITP, HIT K,
  Wolfgang-Pauli-Strasse 27}}
\hypersetup{pdfcontactpostcode={8093}}
\hypersetup{pdfcontactcity={Zurich}}
\hypersetup{pdfcontactcountry={Switzerland}}
\hypersetup{pdfcontactemail={nbeisert@itp.phys.ethz.ch}}
\hypersetup{pdfcontacturl={http://people.phys.ethz.ch/\xmptilde nbeisert/}}

\newcommand{\secref}[1]{\hyperref[#1]{section \ref*{#1}}}

\parskip1ex
\parindent0pt
\let\olditemize\itemize
\def\itemize{\olditemize\parskip0pt}

\begin{document}

\title{The \textsf{childdoc} Package}
\hypersetup{pdftitle={The childdoc Package}}
\author{Niklas Beisert\\[2ex]
  Institut f\"ur Theoretische Physik\\
  Eidgen\"ossische Technische Hochschule Z\"urich\\
  Wolfgang-Pauli-Strasse 27, 8093 Z\"urich, Switzerland\\[1ex]
  \href{mailto:nbeisert@itp.phys.ethz.ch}
  {\texttt{nbeisert@itp.phys.ethz.ch}}}
\hypersetup{pdfauthor={Niklas Beisert}}
\hypersetup{pdfsubject={Manual for the LaTeX2e Package childdoc}}
\date{30 December 2018, \textsf{v2.0}}
\maketitle

\begin{abstract}\noindent
\textsf{childdoc} is a \LaTeXe{} package
that enables the direct compilation
of document sections included by |\include|
to individual files.
\end{abstract}

\begingroup
\parskip0ex
\tableofcontents
\endgroup

%%%%%%%%%%%%%%%%%%%%%%%%%%%%%%%%%%%%%%%%%%%%%%%%%%%%%%%%%%%%%%%%%%%%%%%%%%%%%%%%
%%%%%%%%%%%%%%%%%%%%%%%%%%%%%%%%%%%%%%%%%%%%%%%%%%%%%%%%%%%%%%%%%%%%%%%%%%%%%%%%
\section{Introduction}

\LaTeX{} provides a mechanism to structure a large document (such as a book)
into a main file and several child files (containing the chapters)
using the |\include| command.
This mechanism is beneficial for documents
which span hundreds of pages in order to
make the source file(s) more manageable.
Moreover, compilation can be restricted to
selected child files by means of the |\includeonly| command.
The latter feature can be used to reduce the compilation time while editing
(this was significantly more useful in the earlier days of \LaTeX{})
or to generate a smaller document which is easier to navigate.
Another application of |\includeonly| is to generate
documents consisting of selected parts of the complete document.

However, there are a few drawbacks of the plain |\include| mechanism:
\begin{itemize}
\item
The child files cannot be compiled on their own,
they can only be compiled via the main file.
A naive editing environment
(such as a text editor with an option
to have the current file processed by \LaTeX)
may require one to switch to the main file before compiling;
attempting to compile the child file produces errors.
\item
The main file must be modified (each time)
to adjust the |\includeonly| command
to the present needs. This easily leaves the main file in a messy state.
\item
The generated document will always carry the filename
of the main document. This is inconvenient if
several child files are to be compiled and
to be kept for distribution.
\end{itemize}

The present package provides a simple interface
to make child files individually compilable by \LaTeX{}.
Compiling a child file then has the same effect as compiling
the main file with an |\includeonly| command
to select the appropriate child.
Moreover the generated document will carry the name of the child
rather than the main file.
This resolves all three above issues.

This feature is meant to make the editing of books,
thesis documents and lecture notes somewhat more convenient.
However, the package can also be used efficiently for
composing a series of documents (such as exercise sheets)
which are typically distributed individually.
It then assists the author in generating the individual documents
(potentially in different versions)
as well as a document containing the collected series.
Another application is in developing style files
or other kinds of included material
where compilation of the style file could redirect
to a sample or test file.

%%%%%%%%%%%%%%%%%%%%%%%%%%%%%%%%%%%%%%%%%%%%%%%%%%%%%%%%%%%%%%%%%%%%%%%%%%%%%%%%
%%%%%%%%%%%%%%%%%%%%%%%%%%%%%%%%%%%%%%%%%%%%%%%%%%%%%%%%%%%%%%%%%%%%%%%%%%%%%%%%
\section{Usage}

First of all, the package \textsf{childdoc} is \emph{not} a standard
\LaTeXe{} |.sty| style file! Therefore it needs to be invoked in
a non-standard way.

%%%%%%%%%%%%%%%%%%%%%%%%%%%%%%%%%%%%%%%%%%%%%%%%%%%%%%%%%%%%%%%%%%%%%%%%%%%%%%%%
\subsection{Included Files}
\label{sec:include}

%%%%%%%%%%%%%%%%%%%%%%%%%%%%%%%%%%%%%%%%
\DescribeMacro{\childdocmain}
To use the package, add the commands
\begin{center}
\begin{tabular}{l}
|\input{childdoc.def}|\\
|\childdocmain{}|\\
\end{tabular}
\end{center}
at the very top of the main \LaTeX{} file,
in particular \emph{before} the |\documentclass| statement!
The argument of |\childdocmain| should be left empty
(but it must be present).

%%%%%%%%%%%%%%%%%%%%%%%%%%%%%%%%%%%%%%%%
\DescribeMacro{\childdocof}
Furthermore, add the commands
\begin{center}
\begin{tabular}{l}
|\input{childdoc.def}|\\
|\childdocof{|\textit{main}|}|\\
\end{tabular}
\end{center}
at the top of every child file \textit{child}
which is included by |\include{|\textit{child}|}|
from within the main file
(or at least for those files to be compiled individually).
The argument \textit{main} must be the filename of the main file.

There are a couple of
considerations in setting up the main and child documents:

%%%%%%%%%%%%%%%%%%%%%%%%%%%%%%%%%%%%%%%%
\paragraph{Restrictions.}

Please note the following restrictions:
\begin{itemize}
\item
|\childdocmain| must be called with one argument \textit{main}
to ensure compatibility with earlier version of the package.
It must either be empty (|\childdocmain{}|)
or precisely match the filename of the main file in which it is specified.
See \secref{sec:detection} for further information.
\item
The filename \textit{main} must be specified without the |.tex| extension.
\item
The filename \textit{main} is case sensitive
(even in case-insensitive file systems)
due to internal string comparison.
\item
The argument \textit{main} should be fully expanded, it cannot be a macro.
\item
Subdirectories and special characters should be avoided in filenames.
\item
The command |\childdocmain{|\textit{main}|}| must be followed by a whitespace.
It should not be followed immediately by another command
or by a comment mark `|%|'.
This is because the \TeX{} parser reads the token immediately following
the argument of |\childdocmain| and puts it
at the beginning of every child section;
however, a white\-space is ignored.
\end{itemize}

%%%%%%%%%%%%%%%%%%%%%%%%%%%%%%%%%%%%%%%%
\paragraph{Content of Main File.}

It is advisable to place all content in the child files included by |\include|.
Any output contained in the main file will appear in all child documents
unless suppressed manually;
it cannot be suppressed automatically by the |\includeonly| directive
and thus should normally be avoided.
A method to include some content in the main file
by means of conditional processing is described in \secref{sec:conditional}.

%%%%%%%%%%%%%%%%%%%%%%%%%%%%%%%%%%%%%%%%
\paragraph{Page Numbering.}

When only a part of the document is compiled,
the appropriate numbering of pages
(as well as other status parameters)
is determined from the |.aux| files.
The latter contain information from previous passes.
However this information needs to propagate through
all intermediate child documents.
Therefore the page numbering in child documents may well
be inconsistent until the complete document is compiled at least once.

A useful (if unconventional) way to always ensure a consistent
page numbering is to restart the numbering in each child document
and denote the pages by `\textit{child}|.|\textit{page}'
where \textit{child} represents the chapter/section number of the child file.
This can be achieved by the command
|\numberwithin{page}{|\textit{child}|}|
of the \textsf{amsmath} package
where \textit{child} can be |chapter| or |section|
depending on the chosen structuring.
Alternatively, one can modify the macro |\thepage| appropriately
and reset the counter |page| at the start of each child file.

%%%%%%%%%%%%%%%%%%%%%%%%%%%%%%%%%%%%%%%%%%%%%%%%%%%%%%%%%%%%%%%%%%%%%%%%%%%%%%%%
\subsection{Conditional Processing}
\label{sec:conditional}

The package provides a mechanism to compile different versions
of a document. To customise the versions further some conditional processing
can come in handy to distinguish which version is being compiled.
The package provides two macros to describe the compilation context:

%%%%%%%%%%%%%%%%%%%%%%%%%%%%%%%%%%%%%%%%
\DescribeMacro{\ifchilddoc}
The conditional |\ifchilddoc| distinguishes between the compilation of
child documents and the main document:
%
\begin{center}
|\ifchilddoc |\textit{child-code}| |[|\||else |\textit{main-code}]| \||fi|
\end{center}

%%%%%%%%%%%%%%%%%%%%%%%%%%%%%%%%%%%%%%%%
\DescribeMacro{\childdocname}
\DescribeMacro{\childdocjob}
The macro |\childdocname| contains the filename (without extension)
of the main or child file being processed.
Note that |\childdocjob| will always contain the name of the main file.

%%%%%%%%%%%%%%%%%%%%%%%%%%%%%%%%%%%%%%%%
\paragraph{Title Page.}

Conditional processing can be used to include a title or banner page
in the main document when proper precautions are taken.
Importantly, the code in the main file should ensure that the page counter
(as well as other status parameters which are stored in the |.aux| files)
takes the same value after the conditional processing.
Otherwise the page numbers may take divergent values
depending on which part is compiled.

For example, a title page could be declared by:
%
\begin{center}
\begin{tabular}{l}
|\ifchilddoc\||else|\\
|\addtocounter{page}{-1}|\\
\textit{code for title page}\\
|\newpage|\\
|\||fi|
\end{tabular}
\end{center}
%
A banner page for the child documents can be generated by:
%
\begin{center}
\begin{tabular}{l}
|\ifchilddoc|\\
|\addtocounter{page}{-1}|\\
\textit{code for banner page}\\
|\newpage|\\
|\||fi|
\end{tabular}
\end{center}
%
Here one could write a message such as:
\begin{center}
|This is the part \childdocname{} of \childdocjob{}.|
\end{center}

%%%%%%%%%%%%%%%%%%%%%%%%%%%%%%%%%%%%%%%%%%%%%%%%%%%%%%%%%%%%%%%%%%%%%%%%%%%%%%%%
\subsection{Flags}
\label{sec:flags}

The package makes it easy to generate different versions
of the main or child documents.
To this end compilation flags can be defined
and assigned different default values.
They will be particularly useful in conjunction
with the forwarding mechanism described in \secref{sec:forward}.

For example, it may be useful to have a flag |\version|
which can be set to |draft| or |final|.
The document source will contain some conditional code
depending on the value of |\version|.
Suppose further, the flag should default to |final| for the main file
and to |draft| for child files
which is a natural assignment for editing the document.
This is achieved by placing the following code
in the preamble of the main document
(below the |\childdocmain| directive):
%
\begin{center}
\begin{tabular}{l}
|\ifchilddoc|\\
|\providecommand{\version}{draft}|\\
|\||else|\\
|\providecommand{\version}{final}|\\
|\||fi|
\end{tabular}
\end{center}
%
The definition by |\providecommand| makes sure
that previous definitions are not overwritten.
Further statements |\providecommand{\version}{...}|
can thus be added before the above code to override it.

For the main file, one might add a line
(between |\childdocmain| and the above block)
%
\begin{center}
|%\ifchilddoc\||else\providecommand{\version}{draft}\||fi|
\end{center}
%
which can be uncommented to produce a draft version.
Likewise one can add a line to the very top of a child file
(above the |\childdocof{|\textit{main}|}| directive)
%
\begin{center}
|%\providecommand{\version}{final}|
\end{center}
%
which can be uncommented to produce the final version of this child document.

%%%%%%%%%%%%%%%%%%%%%%%%%%%%%%%%%%%%%%%%%%%%%%%%%%%%%%%%%%%%%%%%%%%%%%%%%%%%%%%%
\subsection{Forwarding}
\label{sec:forward}

Different versions of the main or child documents
using compilation flags as described in \secref{sec:flags}
can be (permanently) stored in different files
for convenient compilation, viewing and distribution.
To this end, the package defines a command
to pass on compilation to a different file:

%%%%%%%%%%%%%%%%%%%%%%%%%%%%%%%%%%%%%%%%
\DescribeMacro{\childdocforward}
The command |\childdocforward| redirects processing to
another source file:
%
\begin{center}
\begin{tabular}{l}
|\input{childdoc.def}|\\
|\childdocforward[|\textit{main}|]{|\textit{dest}|}|\\
\end{tabular}
\end{center}
%
The argument \textit{dest} is the destination file
(without extension).
It should be the main file or one of the child files.
Note that further \textsf{childdoc} directives
such as |\childdocof| and |\childdocforward|
in the indicated file will be processed in this form.
The optional argument \textit{main}
passes on directly to the main file \textit{main}
while pretending to compile the child \textit{dest}.
This form behaves as if \textit{dest}
issues |\childdocof{|\textit{main}|}| right away,
and no further \textsf{childdoc} directives will be processed.

%%%%%%%%%%%%%%%%%%%%%%%%%%%%%%%%%%%%%%%%
\DescribeMacro{\...prefix}
In the alternative form |\childdocforwardprefix|,
%
\begin{center}
\begin{tabular}{l}
|\input{childdoc.def}|\\
|\childdocforwardprefix[|\textit{main}|]{|\textit{prefix}|}{|\textit{dest}|}|
\end{tabular}
\end{center}
%
the destination file is determined by a pattern
depending on the current file:
To make this work, the current file must be called
`{\textit{prefix}\hspace{0.2em}\textit{suffix}}'
with \textit{prefix} matching precisely the argument.
Processing is then passed on to the file
`{\textit{dest}\hspace{0.2em}\textit{suffix}}'.
Surely, the same effect is achieved by
directly specifying the
argument `{\textit{dest}\hspace{0.2em}\textit{suffix}}'
in the first form.
However, that requires to set up a different file
for each child. With the alternative form of the command
all these files can have exactly the same content
which simplifies setting them up and maintaining them.

For example, the following file |draft.tex|
with a compilation flag |\version| as described in \secref{sec:flags}
compiles the main document as a draft:
%
\begin{center}
\begin{tabular}{l}
|\def\version{draft}|\\
|\input{childdoc.def}|\\
|\childdocforward{|\textit{main}|}|
\end{tabular}
\end{center}
%
Likewise, the following files |final|\textit{nn}|.tex|
compile the final version of the child document
|child|\textit{nn}|.tex|:
%
\begin{center}
\begin{tabular}{l}
|\def\version{final}|\\
|\input{childdoc.def}|\\
|\childdocforwardprefix{final}{child}|
\end{tabular}
\end{center}
%

Note that when several versions of a main file and/or of each child file
are to be generated, it may be convenient to set up a |Makefile| or
shell script to automatise the process.

%%%%%%%%%%%%%%%%%%%%%%%%%%%%%%%%%%%%%%%%%%%%%%%%%%%%%%%%%%%%%%%%%%%%%%%%%%%%%%%%
\subsection{Command Line Processing}
\label{sec:commandline}

The effect of redirection files can also be achieved by invoking
the \LaTeX{} compiler with a more elaborate command line.
Most conveniently this should be done as part
of a shell script or a |Makefile|.

When using \textsf{childdoc} in the main file, the following
command lines effectively perform a redirection
(note that depending on the shell being used,
backslashes may have to be doubled: `|\|' $\to$ `|\\|'):
%
\begin{center}
|... -jobname "|\textit{target}|" |\\|"|[\textit{flags}]%
|\input{childdoc.def}\childdocforward[|\textit{main}|]{|\textit{dest}|}"|
\end{center}
%
Here \textit{target} is the name of the output file,
\textit{main} is the name of the main file
and \textit{dest} is the name of the main or child file to be processed
(all filenames without extensions).
The optional argument \textit{main} can be omitted
if \textit{main} matches \textit{dest}.
Optionally, compilation \textit{flags} can be defined via |\def| commands.
This command line makes the \TeX{} engine believe
it is compiling the file \textit{target}
whose content is specified as the latter parameter.
The provided code then forwards the processing to
\textit{main} or \textit{dest} as described in \secref{sec:forward}.

%%%%%%%%%%%%%%%%%%%%%%%%%%%%%%%%%%%%%%%%%%%%%%%%%%%%%%%%%%%%%%%%%%%%%%%%%%%%%%%%
\subsection{Include by Input}
\label{sec:input}

Including child documents by |\include| has some restrictions by design.
Most notably, the content of a child document always occupies
its own set of pages; pages cannot be shared between child documents.
Usually, this behaviour makes perfect sense
because each child document contain an essential part of the document.
However, in some situations it may be desirable to compose
a document from a collection of parts
without having mandatory page breaks between then.
For this case, the package
provides a mechanism to include parts
by |\input| which can also be processed individually.
However, by construction this mechanism
requires manual handling of the content to be output.

%%%%%%%%%%%%%%%%%%%%%%%%%%%%%%%%%%%%%%%%
\DescribeMacro{\ifchilddocmanual}
The main file should be prepared as usual, see \secref{sec:include}.
However, the document body must make a distinction
between processing of an individual part and of the main document, e.g.:
%
\begin{center}
\begin{tabular}{l}
|\ifchilddocmanual|\\
|\input{\childdocname}|\\
|\||else|\\
\textit{document body with }|\input{|\textit{part}|}|\\
|\||fi|
\end{tabular}
\end{center}
%
The conditional |\ifchilddocmanual| is true whenever
a part to be included by |\input| is being compiled,
and the name of the part is stored in |\childdocname|.

%%%%%%%%%%%%%%%%%%%%%%%%%%%%%%%%%%%%%%%%
\DescribeMacro{\childdocby}
Each part to be included by |\input| should start with:
%
\begin{center}
\begin{tabular}{l}
|\input{childdoc.def}|\\
|\childdocby{|\textit{main}|}|\\
\end{tabular}
\end{center}
%
The directive |\childdocby| is similar to |\childdocof|
described in \secref{sec:include},
but the subsequent selection of content must be done manually.
To that end, both |\ifchilddoc| and |\ifchilddocmanual|
will be true upon processing of a part,
and the name of the part is stored in |\childdocname|.
Note that |\jobname| will be set to the filename of the current part
so that each part receives an individual |.aux| file
that does not interfere with the |.aux| file(s) of the main document.
This behaviour can be altered by the alternative form
|\childdocby[*]{|\textit{main}|}| (with a non-empty optional argument)
which uses the |.aux| file of the main document
by setting |\jobname| to \textit{main}.

%%%%%%%%%%%%%%%%%%%%%%%%%%%%%%%%%%%%%%%%%%%%%%%%%%%%%%%%%%%%%%%%%%%%%%%%%%%%%%%%
\subsection{Driver Development}
\label{sec:driver}

The \textsf{childdoc} mechanism can also be use for the development
of definition files such as \LaTeX{} styles or classes.
This case differs from the above setup with multiple parts
included by |\include| in that no |\includeonly| should be invoked.
This can be achieved by starting the include file
(before |\ProvidesPackage|) with:
%
\begin{center}
\begin{tabular}{l}
|\input{childdoc.def}|\\
|\childdocforward{|\textit{main}|}|\\
\end{tabular}
\end{center}
%
or alternatively with:
%
\begin{center}
\begin{tabular}{l}
|\input{childdoc.def}|\\
|\childdocby{|\textit{main}|}|\\
\end{tabular}
\end{center}
%
Both forms have slightly different effects as described above.
The main file is prepared as usual, see \secref{sec:include}.

%%%%%%%%%%%%%%%%%%%%%%%%%%%%%%%%%%%%%%%%%%%%%%%%%%%%%%%%%%%%%%%%%%%%%%%%%%%%%%%%
\subsection{Legacy Detection}
\label{sec:detection}

The directive |\childdocmain| in the main file can detect
whether the complete document or merely a child is to be compiled
even without using the directive |\childdocof|.
This method is deprecated because it is less robust
and there is no compelling reason to use it;
it is merely provided for backward compatibility
and it may be removed in future versions.

If the detection mechanism is to be used,
it is mandatory to correctly specify
the filename of the main file as the argument of |\childdocmain|:
%
\begin{center}
\begin{tabular}{l}
|\input{childdoc.def}|\\
|\childdocmain{|\textit{main}|}|\\
\end{tabular}
\end{center}
%
If |\jobname| does not match the argument \textit{main} of |\childdocmain|,
it is assumed that |\jobname| points to the child file to be compiled.
When using |\childdocmain| with the main file specified as argument,
it suffices to start a child file
with just |\input{|\textit{main}|}|
without loading of the package and using |\childdocof|.
If instead all processing is done
with the appropriate \textsf{childdoc} directives,
the argument of \textit{main} of |\childdocmain| can be empty.

An alternative version of the command line processing described
in \secref{sec:commandline} using the detection mechanism reads:
%
\begin{center}
|... -jobname "|\textit{target}|" "|[\textit{flags}]%
[|\def\jobname{|\textit{dest}|}|]|\input{|\textit{main}|}"|
\end{center}

%%%%%%%%%%%%%%%%%%%%%%%%%%%%%%%%%%%%%%%%%%%%%%%%%%%%%%%%%%%%%%%%%%%%%%%%%%%%%%%%
\subsection{Manual Code}
\label{sec:manual}

In case one cannot be certain whether the definitions file |childdoc.def|
is installed on the target \TeX{} distribution
and one prefers not to ship it,
it is conceivable to paste a few relevant commands into the sources.

To that end, drop all statements |\input{childdoc.def}|
and perform the replacements as outlined below.
Instead of |\childdocmain{|\textit{main}|}| add the following code
to the top of the main file:
%
\begin{center}
\begin{tabular}{l}
|\||ifdefined\childdocname\endinput\||fi\newif\ifchilddoc|\\
|\edef\childdocname{\scantokens\expandafter{\jobname\noexpand}}|\\
|\def\childdocmain{|\textit{main}|}\||ifx\childdocmain\childdocname\||else|\\
|\childdoctrue\includeonly{\childdocname}\let\jobname\childdocmain\||fi|\\
\end{tabular}
\end{center}
%
Instead of |\childdocof{|\textit{main}|}| just include the main file
at the top of each child file:
%
\begin{center}
|\input{|\textit{main}|}|
\end{center}
%
A simple redirection |\childdocforward{|\textit{dest}|}| is achieved by:
%
\begin{center}
|\def\jobname{|\textit{dest}|}\input{\jobname}|
\end{center}
%
The redirection with prefix
|\childdocforwardprefix[|\textit{prefix}|]{|\textit{dest}|}|
is accomplished by:
%
\begin{center}
\begin{tabular}{l}
|{\edef\jobname{\scantokens\expandafter{\jobname\noexpand}}|\\
|\def\redirectjob |\textit{prefix}|#1~~~{\gdef\jobname{|\textit{dest}|#1}}|\\
|\expandafter\redirectjob\jobname~~~}\input{\jobname}|
\end{tabular}
\end{center}

In an alternative approach,
child documents can be compiled by a specific command line
without additional code or specific definitions:
%
\begin{center}
|... -jobname "|\textit{target}|" "|[\textit{flags}]%
|\includeonly{|\textit{dest}|}\input{|\textit{main}|}"|
\end{center}
%

%%%%%%%%%%%%%%%%%%%%%%%%%%%%%%%%%%%%%%%%%%%%%%%%%%%%%%%%%%%%%%%%%%%%%%%%%%%%%%%%
%%%%%%%%%%%%%%%%%%%%%%%%%%%%%%%%%%%%%%%%%%%%%%%%%%%%%%%%%%%%%%%%%%%%%%%%%%%%%%%%
\section{Information}

%%%%%%%%%%%%%%%%%%%%%%%%%%%%%%%%%%%%%%%%%%%%%%%%%%%%%%%%%%%%%%%%%%%%%%%%%%%%%%%%
\subsection{Copyright}

Copyright \copyright{} 2017--2018 Niklas Beisert

This work may be distributed and/or modified under the
conditions of the \LaTeX{} Project Public License, either version 1.3
of this license or (at your option) any later version.
The latest version of this license is in
  \url{http://www.latex-project.org/lppl.txt}
and version 1.3 or later is part of all distributions of \LaTeX{}
version 2005/12/01 or later.

This work has the LPPL maintenance status `maintained'.

The Current Maintainer of this work is Niklas Beisert.

This work consists of the files |README.txt|, |childdoc.ins| and |childdoc.dtx|
as well as the derived files |childdoc.def|, |cdocsamp.tex|
with |cdocsch1.tex|, |cdocsch2.tex|, |cdocspt3.tex|, |cdocspt4.tex|,
|cdocsdrf.tex|, |cdocsfn1.tex|, |cdocsfn2.tex|
as well as |childdoc.pdf|.

%%%%%%%%%%%%%%%%%%%%%%%%%%%%%%%%%%%%%%%%%%%%%%%%%%%%%%%%%%%%%%%%%%%%%%%%%%%%%%%%
\subsection{Files and Installation}

The package consists of the files:
%
\begin{center}
\begin{tabular}{ll}
    |README.txt|   & readme file \\
    |childdoc.ins| & installation file \\
    |childdoc.dtx| & source file \\
    |childdoc.def| & definition file \\
    |cdocsamp.tex| & sample main file \\
    |cdocsch1.tex| & sample include file \\
    |cdocsch2.tex| & sample include file \\
    |cdocspt3.tex| & sample part file \\
    |cdocspt4.tex| & sample part file \\
    |cdocsdrf.tex| & sample redirection file \\
    |cdocsfn1.tex| & sample redirection file \\
    |cdocsfn2.tex| & sample redirection file \\
    |childdoc.pdf| & manual
\end{tabular}
\end{center}
%
The distribution consists of the files
|README.txt|, |childdoc.ins| and |childdoc.dtx|.
%
\begin{itemize}
\item
Run (pdf)\LaTeX{} on |childdoc.dtx|
to compile the manual |childdoc.pdf| (this file).
\item
Run \LaTeX{} on |childdoc.ins| to create the definitions file |childdoc.def|
and the sample |cdocsamp.tex| with include files
|cdocsch1.tex|, |cdocsch2.tex|, |cdocspt3.tex|, |cdocspt4.tex|,
|cdocsdrf.tex|, |cdocsfn1.tex|, |cdocsfn2.tex|.
Then copy the file |childdoc.def| to an appropriate directory of your \LaTeX{}
distribution, e.g.\ \textit{texmf-root}|/tex/latex/childdoc|.
\end{itemize}

%%%%%%%%%%%%%%%%%%%%%%%%%%%%%%%%%%%%%%%%%%%%%%%%%%%%%%%%%%%%%%%%%%%%%%%%%%%%%%%%
\subsection{Related CTAN Packages}

There are several other packages which offer a similar functionality:
%
\begin{itemize}
\item
The packages
\href{http://ctan.org/pkg/docmute}{\textsf{docmute}},
\href{http://ctan.org/pkg/includex}{\textsf{includex}} and
\href{http://ctan.org/pkg/standalone}{\textsf{standalone}}
provide commands to include only the document body of
a child file thus allowing both files to be compiled individually.
\item
The packages \href{http://ctan.org/pkg/subdocs}{\textsf{subdocs}}
and \href{http://ctan.org/pkg/subfiles}{\textsf{subfiles}}
provide structures in which the main and child documents can be
encapsulated and allowing them to be compiled individually.
The inclusion mechanism is different from the conventional |\include|.
\item
The package \href{http://ctan.org/pkg/combine}{\textsf{combine}}
is an elaborate solution to combine several documents into one.
\end{itemize}
%
See also the CTAN topic \href{http://ctan.org/topic/subdocs}{\textsf{subdocs}}
for further related packages.
The present package differs from the above solutions in that
a document structure constructed with the conventional |\include| mechanism
just needs two extra commands at the top of every file
such that all constituent files can be compiled individually.

%%%%%%%%%%%%%%%%%%%%%%%%%%%%%%%%%%%%%%%%%%%%%%%%%%%%%%%%%%%%%%%%%%%%%%%%%%%%%%%%
%\subsection{Feature Suggestions}
%
%The following is a list of features which may be useful for future
%versions of this package:
%%
%\begin{itemize}
%\item
%\ldots
%\end{itemize}

%%%%%%%%%%%%%%%%%%%%%%%%%%%%%%%%%%%%%%%%%%%%%%%%%%%%%%%%%%%%%%%%%%%%%%%%%%%%%%%%
\subsection{Revision History}

%%%%%%%%%%%%%%%%%%%%%%%%%%%%%%%%%%%%%%%%
\paragraph{v2.0:} 2018/12/30

\begin{itemize}
\item
immediate forward processing
\item
added |\childdocby| mechanism
\item
manual restructured
\end{itemize}

%%%%%%%%%%%%%%%%%%%%%%%%%%%%%%%%%%%%%%%%
\paragraph{v1.6:} 2018/01/17

\begin{itemize}
\item
application for development of include files
\item
corrections to manual
\end{itemize}

%%%%%%%%%%%%%%%%%%%%%%%%%%%%%%%%%%%%%%%%
\paragraph{v1.5:} 2017/05/21

\begin{itemize}
\item
more complete structuring introduced
\item
|\childdocof| introduced
\item
|\childdoc| renamed to |\childdocmain|
\item
|\childredirect| renamed to |\childdocforward| and |\childdocforwardprefix|
and functionality expanded
\end{itemize}

%%%%%%%%%%%%%%%%%%%%%%%%%%%%%%%%%%%%%%%%
\paragraph{v1.0:} 2017/04/27

\begin{itemize}
\item
manual and install package
\item
first version published on CTAN
\end{itemize}

%%%%%%%%%%%%%%%%%%%%%%%%%%%%%%%%%%%%%%%%
\paragraph{v0.6:} 2017/04/26

\begin{itemize}
\item
redirection mechanism added
\end{itemize}

%%%%%%%%%%%%%%%%%%%%%%%%%%%%%%%%%%%%%%%%
\paragraph{v0.5:} 2017/04/26

\begin{itemize}
\item
functionality in definition file
\end{itemize}


%%%%%%%%%%%%%%%%%%%%%%%%%%%%%%%%%%%%%%%%%%%%%%%%%%%%%%%%%%%%%%%%%%%%%%%%%%%%%%%%
%%%%%%%%%%%%%%%%%%%%%%%%%%%%%%%%%%%%%%%%%%%%%%%%%%%%%%%%%%%%%%%%%%%%%%%%%%%%%%%%
%%%%%%%%%%%%%%%%%%%%%%%%%%%%%%%%%%%%%%%%%%%%%%%%%%%%%%%%%%%%%%%%%%%%%%%%%%%%%%%%
\appendix

\settowidth\MacroIndent{\rmfamily\scriptsize 000\ }

 \DocInput{childdoc.dtx}

\end{document}
%</driver>
% \fi
%
% %%%%%%%%%%%%%%%%%%%%%%%%%%%%%%%%%%%%%%%%%%%%%%%%%%%%%%%%%%%%%%%%%%%%%%%%%%%%%%
% %%%%%%%%%%%%%%%%%%%%%%%%%%%%%%%%%%%%%%%%%%%%%%%%%%%%%%%%%%%%%%%%%%%%%%%%%%%%%%
% \section{Sample}
%\iffalse
%<*samplemain>
%\fi
%
% The following presents a sample document
% with two chapters, two parts, a title page,
% a compile flag as well as three forwarding files to set the flag.
% It consists of eight |.tex| files:
% \begin{center}
% \begin{tabular}{ll}
% |cdocsamp.tex|&main file\\
% |cdocsch1.tex|&include file for chapter 1\\
% |cdocsch2.tex|&include file for chapter 2\\
% |cdocspt3.tex|&include file for part 3\\
% |cdocspt4.tex|&include file for part 4\\
% |cdocsdrf.tex|&forwarding file for main file in draft mode\\
% |cdocsfi1.tex|&forwarding file for final version of chapter 1\\
% |cdocsfi2.tex|&forwarding file for final version of chapter 2\\
% \end{tabular}
% \end{center}
% Each of the eight files can be compiled directly by the \LaTeX{} compiler.
%
% %%%%%%%%%%%%%%%%%%%%%%%%%%%%%%%%%%%%%%
% \paragraph{Main File.}
%
% The main file is called |cdocsamp.tex|.
%
% Load the \textsf{childdoc} definitions and
% declare the filename for the main document:
%    \begin{macrocode}
\input{childdoc.def}
\childdocmain{}
%    \end{macrocode}

% Optional override for |\version| flag:
%    \begin{macrocode}
%%\ifchilddoc\else\providecommand{\version}{draft}\fi
%    \end{macrocode}

% Define the default values for the |\version| flag
% (|final| for the main file and |draft| for childs):
%    \begin{macrocode}
\ifchilddoc
\providecommand{\version}{draft}
\else
\providecommand{\version}{final}
\fi
%    \end{macrocode}

% Load the standard document class:
%    \begin{macrocode}
\documentclass[12pt]{article}
%    \end{macrocode}

% Start the document body:
%    \begin{macrocode}
\begin{document}
%    \end{macrocode}

% Declare a title page.
% Print title, part of document being processed and version flag:
%    \begin{macrocode}
\addtocounter{page}{-1}
\begin{center}
{\LARGE\bfseries{}childdoc example\par}
\vspace{1cm}
\ifchilddoc
\ifchilddocmanual part\else chapter\fi:
`\childdocname' of `\childdocjob'\par
\else
main document: `\childdocjob'\par
\fi
version: \version\par
\end{center}
\newpage
%    \end{macrocode}

% Manually include selected file,
% otherwise process as usual:
%    \begin{macrocode}
\ifchilddocmanual
\section*{part `\childdocname'}
\input{\childdocname}
\else
%    \end{macrocode}

% Include the two chapters:
%    \begin{macrocode}
\include{cdocsch1}
\include{cdocsch2}
%    \end{macrocode}

% Include the two parts unless only chapters should be displayed:
%    \begin{macrocode}
\ifchilddoc\else
\section{part three}
\input{cdocspt3}
\section{part four}
\input{cdocspt4}
\fi
%    \end{macrocode}

% Process as usual until here:
%    \begin{macrocode}
\fi
%    \end{macrocode}

% End of document body:
%    \begin{macrocode}
\end{document}
%    \end{macrocode}
%\iffalse
%</samplemain>
%\fi
%
% %%%%%%%%%%%%%%%%%%%%%%%%%%%%%%%%%%%%%%
% \paragraph{Chapter Include Files.}
%
% The include files are called |cdocsch1.tex| and |cdocsch2.tex|.
%
%\iffalse
%<*samplechap1|samplechap2>
%\fi

% Optional override for |\version| flag:
%    \begin{macrocode}
%%\providecommand{\version}{final}
%    \end{macrocode}

% Include the main document:
%    \begin{macrocode}
\input{childdoc.def}
\childdocof{cdocsamp}
%    \end{macrocode}

%\iffalse
%</samplechap1|samplechap2>
%\fi
%
%\iffalse
%<*samplechap1>
%\fi
% Some text for chapter 1:
%    \begin{macrocode}
\section{one}
some text in chapter one
%    \end{macrocode}

%\iffalse
%</samplechap1>
%\fi
% Some text for chapter 2:
%\iffalse
%<*samplechap2>
%\fi
%    \begin{macrocode}
\section{two}
more text in chapter two
%    \end{macrocode}

%\iffalse
%</samplechap2>
%\fi
%
% %%%%%%%%%%%%%%%%%%%%%%%%%%%%%%%%%%%%%%
% \paragraph{Part Include Files.}
%
% The include files are called |cdocspt3.tex| and |cdocspt4.tex|.
%
%\iffalse
%<*samplepart3|samplepart4>
%\fi

% Optional override for |\version| flag:
%    \begin{macrocode}
%%\providecommand{\version}{final}
%    \end{macrocode}

% Include the main document:
%    \begin{macrocode}
\input{childdoc.def}
\childdocby{cdocsamp}
%    \end{macrocode}

%\iffalse
%</samplepart3|samplepart4>
%\fi
%
%\iffalse
%<*samplepart3>
%\fi
% Some text for part 3:
%    \begin{macrocode}
some text in part three
%    \end{macrocode}

%\iffalse
%</samplepart3>
%\fi
% Some text for part 4:
%\iffalse
%<*samplepart4>
%\fi
%    \begin{macrocode}
more text in part four
%    \end{macrocode}

%\iffalse
%</samplepart4>
%\fi
%
% %%%%%%%%%%%%%%%%%%%%%%%%%%%%%%%%%%%%%%
% \paragraph{Forwarding for a Complete Draft.}
%
% The following forwarding file |cdocsdrf.tex|
% compiles the main document in draft mode:
%\iffalse
%<*sampledraft>
%\fi
%    \begin{macrocode}
\def\version{draft}
\input{childdoc.def}
\childdocforward{cdocsamp}
%    \end{macrocode}

%\iffalse
%</sampledraft>
%\fi
%
% %%%%%%%%%%%%%%%%%%%%%%%%%%%%%%%%%%%%%%
% \paragraph{Forwarding for Final Version of the Chapters.}
%
% The following forwarding files |cdocsfn1.tex| and |cdocsfn2.tex|
% (with identical content)
% compile the final versions of the child documents
% |cdocsch1.tex| and |cdocsch2.tex|, respectively:
%\iffalse
%<*samplefinal>
%\fi
%    \begin{macrocode}
\def\version{final}
\input{childdoc.def}
\childdocforwardprefix[cdocsamp]{cdocsfn}{cdocsch}
%    \end{macrocode}

%\iffalse
%</samplefinal>
%\fi
%
% %%%%%%%%%%%%%%%%%%%%%%%%%%%%%%%%%%%%%%
% \paragraph{Command Line Processing.}
%
% The following three command lines generate the output files
% |cdocscld|, |cdocscl1| and |cdocscl2|
% which should be identical to
% |cdocsdrf|, |cdocsch1| and |cdocsfn2|, respectively:
% \begin{center}
% \begin{tabular}{l}
% |latex -jobname cdocscld \|\\
% |  "\def\version{draft}\input{childdoc.def}\childdocforward{cdocsamp}"|\\
% |latex -jobname cdocscl1 \|\\
% |  "\input{childdoc.def}\childdocforward[cdocsamp]{cdocsch1}"|\\
% |latex -jobname cdocscl2 \|\\
% |  "\def\version{final}\input{childdoc.def}\childdocforward{cdocsch2}"|
% \end{tabular}
% \end{center}
% Note that the trailing backslash on each first line
% merely continues the input to the second line
% (for convenient cut ant paste).
% Furthermore, the command |latex| can be replaced by any
% of its alternative versions such as |pdflatex|.
%
% %%%%%%%%%%%%%%%%%%%%%%%%%%%%%%%%%%%%%%%%%%%%%%%%%%%%%%%%%%%%%%%%%%%%%%%%%%%%%%
% %%%%%%%%%%%%%%%%%%%%%%%%%%%%%%%%%%%%%%%%%%%%%%%%%%%%%%%%%%%%%%%%%%%%%%%%%%%%%%
% \section{Implementation}
%\iffalse
%<*package>
%\fi
%
% This section describes the definitions file |childdoc.def|.

% The definitions cannot be loaded using |\usepackage| or |\RequirePackage|
% which has a mechanism to prevent loading a style file more than once.
% When loading the definitions by means of |\input|
% multiple instances have to be prevented manually:
%\iffalse
%This code needs to be before the `\ProvidesFile' directive
%which is defined at the beginning of this file.
%Therefore it is also placed there and commented out here.
%</package>
%<*discard>
%\fi
%    \begin{macrocode}
\ifdefined\childdocmain\endinput\fi
%    \end{macrocode}
%\iffalse
%</discard>
%<*package>
%\fi
%
% \macro{\ifchilddoc}
% \macro{\ifchilddocmanual}
% The conditional |\ifchilddoc| tells whether a
% child (true) or main (false) document is being compiled.
% The conditional |\ifchilddocmanual| tells whether
% the |\includeonly| mechanism is used (false) or
% the selection of child files must be performed manually (true).
% The definitions initialise to false:
%    \begin{macrocode}
\newif\ifchilddoc
\newif\ifchilddocmanual
%    \end{macrocode}

% \macro{\childdocname}
% \macro{\childdocjob}
% The macro |\childdocname| stores the name of the main document
% to be compiled. The macro |\childdocjob| stores the name of
% the document on which the \LaTeX{} compiler was originally invoked.
% The content of |\jobname| cannot be compared
% to filenames specified in the source due to different catcodes.
% The following code rescans |\jobname|, stores the result
% in |\childdocname| and saves a copy in |\childdocjob|:
%    \begin{macrocode}
\edef\childdocname{\scantokens\expandafter{\jobname\noexpand}}
\let\childdocjob\childdocname
%    \end{macrocode}

% \macro{\childdocdisable}
% The macro |\childdocdisable| prevents the main file
% from being processed more than once.
% At this stage, the main document command |\childdocmain|
% is assumed to be called once again where it should do nothing.
% Any subsequent call to it should prevent
% a secondary processing of the main document
% It overwrites the forwarding commands
% |\childdocof| and |\childdocforward|
% with empty macros to prevent further inclusions of the main document:
%    \begin{macrocode}
\newcommand{\childdocdisable}
{
  \renewcommand{\childdocmain}[1]{\renewcommand{\childdocmain}[1]{\endinput}}
  \renewcommand{\childdocof}[1]{}
  \renewcommand{\childdocby}[2][]{}
  \renewcommand{\childdocforward}[2][]{}
  \renewcommand{\childdocdisable}{}
}
%    \end{macrocode}

% \macro{\childdocmain}
% The macro |\childdocmain| is to be called at the top of the main file
% with nothing or the main filename (without extension) as argument.
% First, it breaks loops.
% If the argument is not empty and does not match |\childdocname|
% (which is set by the first inclusion of |childdoc.def|),
% |\ifchilddoc| is set to true, |\includeonly| is applied to the child file
% and |\jobname| is set to the main file
% (for proper handling of |.aux| files):
%    \begin{macrocode}
\newcommand{\childdocmain}[1]
{
  \childdocdisable\childdocmain{}
  \if?#1?\else
    \begingroup
      \def\childdoctmp{#1}
      \ifx\childdoctmp\childdocname
        \def\childdoctmp{}
      \else
        \def\childdoctmp
        {
          \childdoctrue
          \includeonly{\childdocname}
          \def\childdocjob{#1}
          \def\jobname{#1}
        }
      \fi
      \expandafter
    \endgroup
    \childdoctmp
  \fi
}
%    \end{macrocode}

% \macro{\childdocof}
% The command |\childdocof| redirects
% compilation to the main file |#1|.
%    \begin{macrocode}
\newcommand{\childdocof}[1]
{
  \childdocdisable
  \childdoctrue
  \includeonly{\childdocname}
  \def\jobname{#1}
  \def\childdocjob{#1}
  \input{#1}
}
%    \end{macrocode}

% \macro{\childdocby}
% The command |\childdocby| ....
%    \begin{macrocode}
\newcommand{\childdocby}[2][]
{
  \childdocdisable
  \childdoctrue
  \childdocmanualtrue
  \if?#1?\else
    \def\jobname{#2}
  \fi
  \def\childdocjob{#2}
  \input{#2}
  \endinput
}
%    \end{macrocode}

% \macro{\childdocforward}
% The command |\childdocforward| redirects
% compilation to the main file or
% (if the optional argument is given) a child file.
% Parameters are set as if the main file
% or a child file starting with |\childdocof| was compiled.
% Then compilation is handed over to the main file:
%    \begin{macrocode}
\newcommand{\childdocforward}[2][]
{
  \begingroup
    \if?#1?
      \def\childdoctmp
      {
        \def\childdocname{#2}
        \def\childdocjob{#2}
        \def\jobname{#2}
        \input{#2}
        \endinput
      }
    \else
      \def\childdoctmp
      {
        \childdocdisable
        \def\childdocname{#2}
        \childdoctrue
        \includeonly{#2}
        \def\childdocjob{#1}
        \def\jobname{#1}
        \input{#1}
        \endinput
      }
    \fi
    \expandafter
  \endgroup
  \childdoctmp
}
%    \end{macrocode}

% \macro{\childdocforwardprefix}
% The command |\childdocforwardprefix| redirects
% compilation to the main or a child file by means of a pattern.
% The prefix |#1| in the current filename is replaced by |#2|
% and the suffix of the current filename is kept
% (it is assumed that the filename does not contain the substring `|~~~|'
% which is used as a delimiter).
% Compilation is handed over to the new file by |\childdocforward|:
%    \begin{macrocode}
\newcommand{\childdocforwardprefix}[3][]
{
  \begingroup
    \def\childdocextract #2##1~~~{\def\childdoctmp{\childdocforward[#1]{#3##1}}}
    \expandafter\childdocextract\childdocname~~~
    \expandafter
  \endgroup
  \childdoctmp
}
%    \end{macrocode}

% \macro{\childdoc}
% The deprecated macro |\childdoc| is a legacy version of |\childdocmain|:
%    \begin{macrocode}
\newcommand{\childdoc}{\childdocmain}
%    \end{macrocode}

% \macro{\childdocredirect}
% The deprecated macro |\childdocredirect| is a legacy version
% of |\childdocforward| and |\childdocforwardprefix|:
%    \begin{macrocode}
\newcommand{\childdocredirect}[2][]
{
  \begingroup
    \if?#1?
      \def\childdoctmp{\childdocforward{#2}}
    \else
      \def\childdoctmp{\childdocforwardprefix{#1}{#2}}
    \fi
    \expandafter
  \endgroup
  \childdoctmp
}
%    \end{macrocode}

%\iffalse
%</package>
%\fi
%
\endinput

\childdocby{cdocsamp}
%    \end{macrocode}

%\iffalse
%</samplepart3|samplepart4>
%\fi
%
%\iffalse
%<*samplepart3>
%\fi
% Some text for part 3:
%    \begin{macrocode}
some text in part three
%    \end{macrocode}

%\iffalse
%</samplepart3>
%\fi
% Some text for part 4:
%\iffalse
%<*samplepart4>
%\fi
%    \begin{macrocode}
more text in part four
%    \end{macrocode}

%\iffalse
%</samplepart4>
%\fi
%
% %%%%%%%%%%%%%%%%%%%%%%%%%%%%%%%%%%%%%%
% \paragraph{Forwarding for a Complete Draft.}
%
% The following forwarding file |cdocsdrf.tex|
% compiles the main document in draft mode:
%\iffalse
%<*sampledraft>
%\fi
%    \begin{macrocode}
\def\version{draft}
% \iffalse
%
% childdoc.dtx Copyright (C) 2017-2018 Niklas Beisert
%
% This work may be distributed and/or modified under the
% conditions of the LaTeX Project Public License, either version 1.3
% of this license or (at your option) any later version.
% The latest version of this license is in
%   http://www.latex-project.org/lppl.txt
% and version 1.3 or later is part of all distributions of LaTeX
% version 2005/12/01 or later.
%
% This work has the LPPL maintenance status `maintained'.
%
% The Current Maintainer of this work is Niklas Beisert.
%
% This work consists of the files childdoc.dtx and childdoc.ins
% and the derived files childdoc.def and cdocsamp.tex with
% cdocsch1.tex, cdocsch2.tex, cdocsdrf.tex, cdocsfn1.tex, cdocsfn2.tex.
%
%<package>\ifdefined\childdocmain\endinput\fi
%<package>\ProvidesFile{childdoc.def}[2018/12/30 v2.0 child document driver]
%<samplemain>\ProvidesFile{cdocsamp.tex}[2018/12/30 v2.0 sample for childdoc]
%<*driver>
%\ProvidesFile{childdoc.drv}[2018/12/30 v2.0 childdoc reference manual file]
\PassOptionsToClass{10pt,a4paper}{article}
\documentclass{ltxdoc}

\usepackage[margin=35mm]{geometry}
\usepackage{hyperref}
\usepackage{hyperxmp}
\usepackage[usenames]{color}

\hypersetup{colorlinks=true}
\hypersetup{pdfstartview=FitH}
\hypersetup{pdfpagemode=UseNone}
\hypersetup{pdfsource={}}
\hypersetup{pdflang={en-UK}}
\hypersetup{pdfcopyright={Copyright 2017-2018 Niklas Beisert.
  This work may be distributed and/or modified under the
  conditions of the LaTeX Project Public License, either version 1.3
  of this license or (at your option) any later version.}}
\hypersetup{pdflicenseurl={http://www.latex-project.org/lppl.txt}}
\hypersetup{pdfcontactaddress={ETH Zurich, ITP, HIT K,
  Wolfgang-Pauli-Strasse 27}}
\hypersetup{pdfcontactpostcode={8093}}
\hypersetup{pdfcontactcity={Zurich}}
\hypersetup{pdfcontactcountry={Switzerland}}
\hypersetup{pdfcontactemail={nbeisert@itp.phys.ethz.ch}}
\hypersetup{pdfcontacturl={http://people.phys.ethz.ch/\xmptilde nbeisert/}}

\newcommand{\secref}[1]{\hyperref[#1]{section \ref*{#1}}}

\parskip1ex
\parindent0pt
\let\olditemize\itemize
\def\itemize{\olditemize\parskip0pt}

\begin{document}

\title{The \textsf{childdoc} Package}
\hypersetup{pdftitle={The childdoc Package}}
\author{Niklas Beisert\\[2ex]
  Institut f\"ur Theoretische Physik\\
  Eidgen\"ossische Technische Hochschule Z\"urich\\
  Wolfgang-Pauli-Strasse 27, 8093 Z\"urich, Switzerland\\[1ex]
  \href{mailto:nbeisert@itp.phys.ethz.ch}
  {\texttt{nbeisert@itp.phys.ethz.ch}}}
\hypersetup{pdfauthor={Niklas Beisert}}
\hypersetup{pdfsubject={Manual for the LaTeX2e Package childdoc}}
\date{30 December 2018, \textsf{v2.0}}
\maketitle

\begin{abstract}\noindent
\textsf{childdoc} is a \LaTeXe{} package
that enables the direct compilation
of document sections included by |\include|
to individual files.
\end{abstract}

\begingroup
\parskip0ex
\tableofcontents
\endgroup

%%%%%%%%%%%%%%%%%%%%%%%%%%%%%%%%%%%%%%%%%%%%%%%%%%%%%%%%%%%%%%%%%%%%%%%%%%%%%%%%
%%%%%%%%%%%%%%%%%%%%%%%%%%%%%%%%%%%%%%%%%%%%%%%%%%%%%%%%%%%%%%%%%%%%%%%%%%%%%%%%
\section{Introduction}

\LaTeX{} provides a mechanism to structure a large document (such as a book)
into a main file and several child files (containing the chapters)
using the |\include| command.
This mechanism is beneficial for documents
which span hundreds of pages in order to
make the source file(s) more manageable.
Moreover, compilation can be restricted to
selected child files by means of the |\includeonly| command.
The latter feature can be used to reduce the compilation time while editing
(this was significantly more useful in the earlier days of \LaTeX{})
or to generate a smaller document which is easier to navigate.
Another application of |\includeonly| is to generate
documents consisting of selected parts of the complete document.

However, there are a few drawbacks of the plain |\include| mechanism:
\begin{itemize}
\item
The child files cannot be compiled on their own,
they can only be compiled via the main file.
A naive editing environment
(such as a text editor with an option
to have the current file processed by \LaTeX)
may require one to switch to the main file before compiling;
attempting to compile the child file produces errors.
\item
The main file must be modified (each time)
to adjust the |\includeonly| command
to the present needs. This easily leaves the main file in a messy state.
\item
The generated document will always carry the filename
of the main document. This is inconvenient if
several child files are to be compiled and
to be kept for distribution.
\end{itemize}

The present package provides a simple interface
to make child files individually compilable by \LaTeX{}.
Compiling a child file then has the same effect as compiling
the main file with an |\includeonly| command
to select the appropriate child.
Moreover the generated document will carry the name of the child
rather than the main file.
This resolves all three above issues.

This feature is meant to make the editing of books,
thesis documents and lecture notes somewhat more convenient.
However, the package can also be used efficiently for
composing a series of documents (such as exercise sheets)
which are typically distributed individually.
It then assists the author in generating the individual documents
(potentially in different versions)
as well as a document containing the collected series.
Another application is in developing style files
or other kinds of included material
where compilation of the style file could redirect
to a sample or test file.

%%%%%%%%%%%%%%%%%%%%%%%%%%%%%%%%%%%%%%%%%%%%%%%%%%%%%%%%%%%%%%%%%%%%%%%%%%%%%%%%
%%%%%%%%%%%%%%%%%%%%%%%%%%%%%%%%%%%%%%%%%%%%%%%%%%%%%%%%%%%%%%%%%%%%%%%%%%%%%%%%
\section{Usage}

First of all, the package \textsf{childdoc} is \emph{not} a standard
\LaTeXe{} |.sty| style file! Therefore it needs to be invoked in
a non-standard way.

%%%%%%%%%%%%%%%%%%%%%%%%%%%%%%%%%%%%%%%%%%%%%%%%%%%%%%%%%%%%%%%%%%%%%%%%%%%%%%%%
\subsection{Included Files}
\label{sec:include}

%%%%%%%%%%%%%%%%%%%%%%%%%%%%%%%%%%%%%%%%
\DescribeMacro{\childdocmain}
To use the package, add the commands
\begin{center}
\begin{tabular}{l}
|\input{childdoc.def}|\\
|\childdocmain{}|\\
\end{tabular}
\end{center}
at the very top of the main \LaTeX{} file,
in particular \emph{before} the |\documentclass| statement!
The argument of |\childdocmain| should be left empty
(but it must be present).

%%%%%%%%%%%%%%%%%%%%%%%%%%%%%%%%%%%%%%%%
\DescribeMacro{\childdocof}
Furthermore, add the commands
\begin{center}
\begin{tabular}{l}
|\input{childdoc.def}|\\
|\childdocof{|\textit{main}|}|\\
\end{tabular}
\end{center}
at the top of every child file \textit{child}
which is included by |\include{|\textit{child}|}|
from within the main file
(or at least for those files to be compiled individually).
The argument \textit{main} must be the filename of the main file.

There are a couple of
considerations in setting up the main and child documents:

%%%%%%%%%%%%%%%%%%%%%%%%%%%%%%%%%%%%%%%%
\paragraph{Restrictions.}

Please note the following restrictions:
\begin{itemize}
\item
|\childdocmain| must be called with one argument \textit{main}
to ensure compatibility with earlier version of the package.
It must either be empty (|\childdocmain{}|)
or precisely match the filename of the main file in which it is specified.
See \secref{sec:detection} for further information.
\item
The filename \textit{main} must be specified without the |.tex| extension.
\item
The filename \textit{main} is case sensitive
(even in case-insensitive file systems)
due to internal string comparison.
\item
The argument \textit{main} should be fully expanded, it cannot be a macro.
\item
Subdirectories and special characters should be avoided in filenames.
\item
The command |\childdocmain{|\textit{main}|}| must be followed by a whitespace.
It should not be followed immediately by another command
or by a comment mark `|%|'.
This is because the \TeX{} parser reads the token immediately following
the argument of |\childdocmain| and puts it
at the beginning of every child section;
however, a white\-space is ignored.
\end{itemize}

%%%%%%%%%%%%%%%%%%%%%%%%%%%%%%%%%%%%%%%%
\paragraph{Content of Main File.}

It is advisable to place all content in the child files included by |\include|.
Any output contained in the main file will appear in all child documents
unless suppressed manually;
it cannot be suppressed automatically by the |\includeonly| directive
and thus should normally be avoided.
A method to include some content in the main file
by means of conditional processing is described in \secref{sec:conditional}.

%%%%%%%%%%%%%%%%%%%%%%%%%%%%%%%%%%%%%%%%
\paragraph{Page Numbering.}

When only a part of the document is compiled,
the appropriate numbering of pages
(as well as other status parameters)
is determined from the |.aux| files.
The latter contain information from previous passes.
However this information needs to propagate through
all intermediate child documents.
Therefore the page numbering in child documents may well
be inconsistent until the complete document is compiled at least once.

A useful (if unconventional) way to always ensure a consistent
page numbering is to restart the numbering in each child document
and denote the pages by `\textit{child}|.|\textit{page}'
where \textit{child} represents the chapter/section number of the child file.
This can be achieved by the command
|\numberwithin{page}{|\textit{child}|}|
of the \textsf{amsmath} package
where \textit{child} can be |chapter| or |section|
depending on the chosen structuring.
Alternatively, one can modify the macro |\thepage| appropriately
and reset the counter |page| at the start of each child file.

%%%%%%%%%%%%%%%%%%%%%%%%%%%%%%%%%%%%%%%%%%%%%%%%%%%%%%%%%%%%%%%%%%%%%%%%%%%%%%%%
\subsection{Conditional Processing}
\label{sec:conditional}

The package provides a mechanism to compile different versions
of a document. To customise the versions further some conditional processing
can come in handy to distinguish which version is being compiled.
The package provides two macros to describe the compilation context:

%%%%%%%%%%%%%%%%%%%%%%%%%%%%%%%%%%%%%%%%
\DescribeMacro{\ifchilddoc}
The conditional |\ifchilddoc| distinguishes between the compilation of
child documents and the main document:
%
\begin{center}
|\ifchilddoc |\textit{child-code}| |[|\||else |\textit{main-code}]| \||fi|
\end{center}

%%%%%%%%%%%%%%%%%%%%%%%%%%%%%%%%%%%%%%%%
\DescribeMacro{\childdocname}
\DescribeMacro{\childdocjob}
The macro |\childdocname| contains the filename (without extension)
of the main or child file being processed.
Note that |\childdocjob| will always contain the name of the main file.

%%%%%%%%%%%%%%%%%%%%%%%%%%%%%%%%%%%%%%%%
\paragraph{Title Page.}

Conditional processing can be used to include a title or banner page
in the main document when proper precautions are taken.
Importantly, the code in the main file should ensure that the page counter
(as well as other status parameters which are stored in the |.aux| files)
takes the same value after the conditional processing.
Otherwise the page numbers may take divergent values
depending on which part is compiled.

For example, a title page could be declared by:
%
\begin{center}
\begin{tabular}{l}
|\ifchilddoc\||else|\\
|\addtocounter{page}{-1}|\\
\textit{code for title page}\\
|\newpage|\\
|\||fi|
\end{tabular}
\end{center}
%
A banner page for the child documents can be generated by:
%
\begin{center}
\begin{tabular}{l}
|\ifchilddoc|\\
|\addtocounter{page}{-1}|\\
\textit{code for banner page}\\
|\newpage|\\
|\||fi|
\end{tabular}
\end{center}
%
Here one could write a message such as:
\begin{center}
|This is the part \childdocname{} of \childdocjob{}.|
\end{center}

%%%%%%%%%%%%%%%%%%%%%%%%%%%%%%%%%%%%%%%%%%%%%%%%%%%%%%%%%%%%%%%%%%%%%%%%%%%%%%%%
\subsection{Flags}
\label{sec:flags}

The package makes it easy to generate different versions
of the main or child documents.
To this end compilation flags can be defined
and assigned different default values.
They will be particularly useful in conjunction
with the forwarding mechanism described in \secref{sec:forward}.

For example, it may be useful to have a flag |\version|
which can be set to |draft| or |final|.
The document source will contain some conditional code
depending on the value of |\version|.
Suppose further, the flag should default to |final| for the main file
and to |draft| for child files
which is a natural assignment for editing the document.
This is achieved by placing the following code
in the preamble of the main document
(below the |\childdocmain| directive):
%
\begin{center}
\begin{tabular}{l}
|\ifchilddoc|\\
|\providecommand{\version}{draft}|\\
|\||else|\\
|\providecommand{\version}{final}|\\
|\||fi|
\end{tabular}
\end{center}
%
The definition by |\providecommand| makes sure
that previous definitions are not overwritten.
Further statements |\providecommand{\version}{...}|
can thus be added before the above code to override it.

For the main file, one might add a line
(between |\childdocmain| and the above block)
%
\begin{center}
|%\ifchilddoc\||else\providecommand{\version}{draft}\||fi|
\end{center}
%
which can be uncommented to produce a draft version.
Likewise one can add a line to the very top of a child file
(above the |\childdocof{|\textit{main}|}| directive)
%
\begin{center}
|%\providecommand{\version}{final}|
\end{center}
%
which can be uncommented to produce the final version of this child document.

%%%%%%%%%%%%%%%%%%%%%%%%%%%%%%%%%%%%%%%%%%%%%%%%%%%%%%%%%%%%%%%%%%%%%%%%%%%%%%%%
\subsection{Forwarding}
\label{sec:forward}

Different versions of the main or child documents
using compilation flags as described in \secref{sec:flags}
can be (permanently) stored in different files
for convenient compilation, viewing and distribution.
To this end, the package defines a command
to pass on compilation to a different file:

%%%%%%%%%%%%%%%%%%%%%%%%%%%%%%%%%%%%%%%%
\DescribeMacro{\childdocforward}
The command |\childdocforward| redirects processing to
another source file:
%
\begin{center}
\begin{tabular}{l}
|\input{childdoc.def}|\\
|\childdocforward[|\textit{main}|]{|\textit{dest}|}|\\
\end{tabular}
\end{center}
%
The argument \textit{dest} is the destination file
(without extension).
It should be the main file or one of the child files.
Note that further \textsf{childdoc} directives
such as |\childdocof| and |\childdocforward|
in the indicated file will be processed in this form.
The optional argument \textit{main}
passes on directly to the main file \textit{main}
while pretending to compile the child \textit{dest}.
This form behaves as if \textit{dest}
issues |\childdocof{|\textit{main}|}| right away,
and no further \textsf{childdoc} directives will be processed.

%%%%%%%%%%%%%%%%%%%%%%%%%%%%%%%%%%%%%%%%
\DescribeMacro{\...prefix}
In the alternative form |\childdocforwardprefix|,
%
\begin{center}
\begin{tabular}{l}
|\input{childdoc.def}|\\
|\childdocforwardprefix[|\textit{main}|]{|\textit{prefix}|}{|\textit{dest}|}|
\end{tabular}
\end{center}
%
the destination file is determined by a pattern
depending on the current file:
To make this work, the current file must be called
`{\textit{prefix}\hspace{0.2em}\textit{suffix}}'
with \textit{prefix} matching precisely the argument.
Processing is then passed on to the file
`{\textit{dest}\hspace{0.2em}\textit{suffix}}'.
Surely, the same effect is achieved by
directly specifying the
argument `{\textit{dest}\hspace{0.2em}\textit{suffix}}'
in the first form.
However, that requires to set up a different file
for each child. With the alternative form of the command
all these files can have exactly the same content
which simplifies setting them up and maintaining them.

For example, the following file |draft.tex|
with a compilation flag |\version| as described in \secref{sec:flags}
compiles the main document as a draft:
%
\begin{center}
\begin{tabular}{l}
|\def\version{draft}|\\
|\input{childdoc.def}|\\
|\childdocforward{|\textit{main}|}|
\end{tabular}
\end{center}
%
Likewise, the following files |final|\textit{nn}|.tex|
compile the final version of the child document
|child|\textit{nn}|.tex|:
%
\begin{center}
\begin{tabular}{l}
|\def\version{final}|\\
|\input{childdoc.def}|\\
|\childdocforwardprefix{final}{child}|
\end{tabular}
\end{center}
%

Note that when several versions of a main file and/or of each child file
are to be generated, it may be convenient to set up a |Makefile| or
shell script to automatise the process.

%%%%%%%%%%%%%%%%%%%%%%%%%%%%%%%%%%%%%%%%%%%%%%%%%%%%%%%%%%%%%%%%%%%%%%%%%%%%%%%%
\subsection{Command Line Processing}
\label{sec:commandline}

The effect of redirection files can also be achieved by invoking
the \LaTeX{} compiler with a more elaborate command line.
Most conveniently this should be done as part
of a shell script or a |Makefile|.

When using \textsf{childdoc} in the main file, the following
command lines effectively perform a redirection
(note that depending on the shell being used,
backslashes may have to be doubled: `|\|' $\to$ `|\\|'):
%
\begin{center}
|... -jobname "|\textit{target}|" |\\|"|[\textit{flags}]%
|\input{childdoc.def}\childdocforward[|\textit{main}|]{|\textit{dest}|}"|
\end{center}
%
Here \textit{target} is the name of the output file,
\textit{main} is the name of the main file
and \textit{dest} is the name of the main or child file to be processed
(all filenames without extensions).
The optional argument \textit{main} can be omitted
if \textit{main} matches \textit{dest}.
Optionally, compilation \textit{flags} can be defined via |\def| commands.
This command line makes the \TeX{} engine believe
it is compiling the file \textit{target}
whose content is specified as the latter parameter.
The provided code then forwards the processing to
\textit{main} or \textit{dest} as described in \secref{sec:forward}.

%%%%%%%%%%%%%%%%%%%%%%%%%%%%%%%%%%%%%%%%%%%%%%%%%%%%%%%%%%%%%%%%%%%%%%%%%%%%%%%%
\subsection{Include by Input}
\label{sec:input}

Including child documents by |\include| has some restrictions by design.
Most notably, the content of a child document always occupies
its own set of pages; pages cannot be shared between child documents.
Usually, this behaviour makes perfect sense
because each child document contain an essential part of the document.
However, in some situations it may be desirable to compose
a document from a collection of parts
without having mandatory page breaks between then.
For this case, the package
provides a mechanism to include parts
by |\input| which can also be processed individually.
However, by construction this mechanism
requires manual handling of the content to be output.

%%%%%%%%%%%%%%%%%%%%%%%%%%%%%%%%%%%%%%%%
\DescribeMacro{\ifchilddocmanual}
The main file should be prepared as usual, see \secref{sec:include}.
However, the document body must make a distinction
between processing of an individual part and of the main document, e.g.:
%
\begin{center}
\begin{tabular}{l}
|\ifchilddocmanual|\\
|\input{\childdocname}|\\
|\||else|\\
\textit{document body with }|\input{|\textit{part}|}|\\
|\||fi|
\end{tabular}
\end{center}
%
The conditional |\ifchilddocmanual| is true whenever
a part to be included by |\input| is being compiled,
and the name of the part is stored in |\childdocname|.

%%%%%%%%%%%%%%%%%%%%%%%%%%%%%%%%%%%%%%%%
\DescribeMacro{\childdocby}
Each part to be included by |\input| should start with:
%
\begin{center}
\begin{tabular}{l}
|\input{childdoc.def}|\\
|\childdocby{|\textit{main}|}|\\
\end{tabular}
\end{center}
%
The directive |\childdocby| is similar to |\childdocof|
described in \secref{sec:include},
but the subsequent selection of content must be done manually.
To that end, both |\ifchilddoc| and |\ifchilddocmanual|
will be true upon processing of a part,
and the name of the part is stored in |\childdocname|.
Note that |\jobname| will be set to the filename of the current part
so that each part receives an individual |.aux| file
that does not interfere with the |.aux| file(s) of the main document.
This behaviour can be altered by the alternative form
|\childdocby[*]{|\textit{main}|}| (with a non-empty optional argument)
which uses the |.aux| file of the main document
by setting |\jobname| to \textit{main}.

%%%%%%%%%%%%%%%%%%%%%%%%%%%%%%%%%%%%%%%%%%%%%%%%%%%%%%%%%%%%%%%%%%%%%%%%%%%%%%%%
\subsection{Driver Development}
\label{sec:driver}

The \textsf{childdoc} mechanism can also be use for the development
of definition files such as \LaTeX{} styles or classes.
This case differs from the above setup with multiple parts
included by |\include| in that no |\includeonly| should be invoked.
This can be achieved by starting the include file
(before |\ProvidesPackage|) with:
%
\begin{center}
\begin{tabular}{l}
|\input{childdoc.def}|\\
|\childdocforward{|\textit{main}|}|\\
\end{tabular}
\end{center}
%
or alternatively with:
%
\begin{center}
\begin{tabular}{l}
|\input{childdoc.def}|\\
|\childdocby{|\textit{main}|}|\\
\end{tabular}
\end{center}
%
Both forms have slightly different effects as described above.
The main file is prepared as usual, see \secref{sec:include}.

%%%%%%%%%%%%%%%%%%%%%%%%%%%%%%%%%%%%%%%%%%%%%%%%%%%%%%%%%%%%%%%%%%%%%%%%%%%%%%%%
\subsection{Legacy Detection}
\label{sec:detection}

The directive |\childdocmain| in the main file can detect
whether the complete document or merely a child is to be compiled
even without using the directive |\childdocof|.
This method is deprecated because it is less robust
and there is no compelling reason to use it;
it is merely provided for backward compatibility
and it may be removed in future versions.

If the detection mechanism is to be used,
it is mandatory to correctly specify
the filename of the main file as the argument of |\childdocmain|:
%
\begin{center}
\begin{tabular}{l}
|\input{childdoc.def}|\\
|\childdocmain{|\textit{main}|}|\\
\end{tabular}
\end{center}
%
If |\jobname| does not match the argument \textit{main} of |\childdocmain|,
it is assumed that |\jobname| points to the child file to be compiled.
When using |\childdocmain| with the main file specified as argument,
it suffices to start a child file
with just |\input{|\textit{main}|}|
without loading of the package and using |\childdocof|.
If instead all processing is done
with the appropriate \textsf{childdoc} directives,
the argument of \textit{main} of |\childdocmain| can be empty.

An alternative version of the command line processing described
in \secref{sec:commandline} using the detection mechanism reads:
%
\begin{center}
|... -jobname "|\textit{target}|" "|[\textit{flags}]%
[|\def\jobname{|\textit{dest}|}|]|\input{|\textit{main}|}"|
\end{center}

%%%%%%%%%%%%%%%%%%%%%%%%%%%%%%%%%%%%%%%%%%%%%%%%%%%%%%%%%%%%%%%%%%%%%%%%%%%%%%%%
\subsection{Manual Code}
\label{sec:manual}

In case one cannot be certain whether the definitions file |childdoc.def|
is installed on the target \TeX{} distribution
and one prefers not to ship it,
it is conceivable to paste a few relevant commands into the sources.

To that end, drop all statements |\input{childdoc.def}|
and perform the replacements as outlined below.
Instead of |\childdocmain{|\textit{main}|}| add the following code
to the top of the main file:
%
\begin{center}
\begin{tabular}{l}
|\||ifdefined\childdocname\endinput\||fi\newif\ifchilddoc|\\
|\edef\childdocname{\scantokens\expandafter{\jobname\noexpand}}|\\
|\def\childdocmain{|\textit{main}|}\||ifx\childdocmain\childdocname\||else|\\
|\childdoctrue\includeonly{\childdocname}\let\jobname\childdocmain\||fi|\\
\end{tabular}
\end{center}
%
Instead of |\childdocof{|\textit{main}|}| just include the main file
at the top of each child file:
%
\begin{center}
|\input{|\textit{main}|}|
\end{center}
%
A simple redirection |\childdocforward{|\textit{dest}|}| is achieved by:
%
\begin{center}
|\def\jobname{|\textit{dest}|}\input{\jobname}|
\end{center}
%
The redirection with prefix
|\childdocforwardprefix[|\textit{prefix}|]{|\textit{dest}|}|
is accomplished by:
%
\begin{center}
\begin{tabular}{l}
|{\edef\jobname{\scantokens\expandafter{\jobname\noexpand}}|\\
|\def\redirectjob |\textit{prefix}|#1~~~{\gdef\jobname{|\textit{dest}|#1}}|\\
|\expandafter\redirectjob\jobname~~~}\input{\jobname}|
\end{tabular}
\end{center}

In an alternative approach,
child documents can be compiled by a specific command line
without additional code or specific definitions:
%
\begin{center}
|... -jobname "|\textit{target}|" "|[\textit{flags}]%
|\includeonly{|\textit{dest}|}\input{|\textit{main}|}"|
\end{center}
%

%%%%%%%%%%%%%%%%%%%%%%%%%%%%%%%%%%%%%%%%%%%%%%%%%%%%%%%%%%%%%%%%%%%%%%%%%%%%%%%%
%%%%%%%%%%%%%%%%%%%%%%%%%%%%%%%%%%%%%%%%%%%%%%%%%%%%%%%%%%%%%%%%%%%%%%%%%%%%%%%%
\section{Information}

%%%%%%%%%%%%%%%%%%%%%%%%%%%%%%%%%%%%%%%%%%%%%%%%%%%%%%%%%%%%%%%%%%%%%%%%%%%%%%%%
\subsection{Copyright}

Copyright \copyright{} 2017--2018 Niklas Beisert

This work may be distributed and/or modified under the
conditions of the \LaTeX{} Project Public License, either version 1.3
of this license or (at your option) any later version.
The latest version of this license is in
  \url{http://www.latex-project.org/lppl.txt}
and version 1.3 or later is part of all distributions of \LaTeX{}
version 2005/12/01 or later.

This work has the LPPL maintenance status `maintained'.

The Current Maintainer of this work is Niklas Beisert.

This work consists of the files |README.txt|, |childdoc.ins| and |childdoc.dtx|
as well as the derived files |childdoc.def|, |cdocsamp.tex|
with |cdocsch1.tex|, |cdocsch2.tex|, |cdocspt3.tex|, |cdocspt4.tex|,
|cdocsdrf.tex|, |cdocsfn1.tex|, |cdocsfn2.tex|
as well as |childdoc.pdf|.

%%%%%%%%%%%%%%%%%%%%%%%%%%%%%%%%%%%%%%%%%%%%%%%%%%%%%%%%%%%%%%%%%%%%%%%%%%%%%%%%
\subsection{Files and Installation}

The package consists of the files:
%
\begin{center}
\begin{tabular}{ll}
    |README.txt|   & readme file \\
    |childdoc.ins| & installation file \\
    |childdoc.dtx| & source file \\
    |childdoc.def| & definition file \\
    |cdocsamp.tex| & sample main file \\
    |cdocsch1.tex| & sample include file \\
    |cdocsch2.tex| & sample include file \\
    |cdocspt3.tex| & sample part file \\
    |cdocspt4.tex| & sample part file \\
    |cdocsdrf.tex| & sample redirection file \\
    |cdocsfn1.tex| & sample redirection file \\
    |cdocsfn2.tex| & sample redirection file \\
    |childdoc.pdf| & manual
\end{tabular}
\end{center}
%
The distribution consists of the files
|README.txt|, |childdoc.ins| and |childdoc.dtx|.
%
\begin{itemize}
\item
Run (pdf)\LaTeX{} on |childdoc.dtx|
to compile the manual |childdoc.pdf| (this file).
\item
Run \LaTeX{} on |childdoc.ins| to create the definitions file |childdoc.def|
and the sample |cdocsamp.tex| with include files
|cdocsch1.tex|, |cdocsch2.tex|, |cdocspt3.tex|, |cdocspt4.tex|,
|cdocsdrf.tex|, |cdocsfn1.tex|, |cdocsfn2.tex|.
Then copy the file |childdoc.def| to an appropriate directory of your \LaTeX{}
distribution, e.g.\ \textit{texmf-root}|/tex/latex/childdoc|.
\end{itemize}

%%%%%%%%%%%%%%%%%%%%%%%%%%%%%%%%%%%%%%%%%%%%%%%%%%%%%%%%%%%%%%%%%%%%%%%%%%%%%%%%
\subsection{Related CTAN Packages}

There are several other packages which offer a similar functionality:
%
\begin{itemize}
\item
The packages
\href{http://ctan.org/pkg/docmute}{\textsf{docmute}},
\href{http://ctan.org/pkg/includex}{\textsf{includex}} and
\href{http://ctan.org/pkg/standalone}{\textsf{standalone}}
provide commands to include only the document body of
a child file thus allowing both files to be compiled individually.
\item
The packages \href{http://ctan.org/pkg/subdocs}{\textsf{subdocs}}
and \href{http://ctan.org/pkg/subfiles}{\textsf{subfiles}}
provide structures in which the main and child documents can be
encapsulated and allowing them to be compiled individually.
The inclusion mechanism is different from the conventional |\include|.
\item
The package \href{http://ctan.org/pkg/combine}{\textsf{combine}}
is an elaborate solution to combine several documents into one.
\end{itemize}
%
See also the CTAN topic \href{http://ctan.org/topic/subdocs}{\textsf{subdocs}}
for further related packages.
The present package differs from the above solutions in that
a document structure constructed with the conventional |\include| mechanism
just needs two extra commands at the top of every file
such that all constituent files can be compiled individually.

%%%%%%%%%%%%%%%%%%%%%%%%%%%%%%%%%%%%%%%%%%%%%%%%%%%%%%%%%%%%%%%%%%%%%%%%%%%%%%%%
%\subsection{Feature Suggestions}
%
%The following is a list of features which may be useful for future
%versions of this package:
%%
%\begin{itemize}
%\item
%\ldots
%\end{itemize}

%%%%%%%%%%%%%%%%%%%%%%%%%%%%%%%%%%%%%%%%%%%%%%%%%%%%%%%%%%%%%%%%%%%%%%%%%%%%%%%%
\subsection{Revision History}

%%%%%%%%%%%%%%%%%%%%%%%%%%%%%%%%%%%%%%%%
\paragraph{v2.0:} 2018/12/30

\begin{itemize}
\item
immediate forward processing
\item
added |\childdocby| mechanism
\item
manual restructured
\end{itemize}

%%%%%%%%%%%%%%%%%%%%%%%%%%%%%%%%%%%%%%%%
\paragraph{v1.6:} 2018/01/17

\begin{itemize}
\item
application for development of include files
\item
corrections to manual
\end{itemize}

%%%%%%%%%%%%%%%%%%%%%%%%%%%%%%%%%%%%%%%%
\paragraph{v1.5:} 2017/05/21

\begin{itemize}
\item
more complete structuring introduced
\item
|\childdocof| introduced
\item
|\childdoc| renamed to |\childdocmain|
\item
|\childredirect| renamed to |\childdocforward| and |\childdocforwardprefix|
and functionality expanded
\end{itemize}

%%%%%%%%%%%%%%%%%%%%%%%%%%%%%%%%%%%%%%%%
\paragraph{v1.0:} 2017/04/27

\begin{itemize}
\item
manual and install package
\item
first version published on CTAN
\end{itemize}

%%%%%%%%%%%%%%%%%%%%%%%%%%%%%%%%%%%%%%%%
\paragraph{v0.6:} 2017/04/26

\begin{itemize}
\item
redirection mechanism added
\end{itemize}

%%%%%%%%%%%%%%%%%%%%%%%%%%%%%%%%%%%%%%%%
\paragraph{v0.5:} 2017/04/26

\begin{itemize}
\item
functionality in definition file
\end{itemize}


%%%%%%%%%%%%%%%%%%%%%%%%%%%%%%%%%%%%%%%%%%%%%%%%%%%%%%%%%%%%%%%%%%%%%%%%%%%%%%%%
%%%%%%%%%%%%%%%%%%%%%%%%%%%%%%%%%%%%%%%%%%%%%%%%%%%%%%%%%%%%%%%%%%%%%%%%%%%%%%%%
%%%%%%%%%%%%%%%%%%%%%%%%%%%%%%%%%%%%%%%%%%%%%%%%%%%%%%%%%%%%%%%%%%%%%%%%%%%%%%%%
\appendix

\settowidth\MacroIndent{\rmfamily\scriptsize 000\ }

 \DocInput{childdoc.dtx}

\end{document}
%</driver>
% \fi
%
% %%%%%%%%%%%%%%%%%%%%%%%%%%%%%%%%%%%%%%%%%%%%%%%%%%%%%%%%%%%%%%%%%%%%%%%%%%%%%%
% %%%%%%%%%%%%%%%%%%%%%%%%%%%%%%%%%%%%%%%%%%%%%%%%%%%%%%%%%%%%%%%%%%%%%%%%%%%%%%
% \section{Sample}
%\iffalse
%<*samplemain>
%\fi
%
% The following presents a sample document
% with two chapters, two parts, a title page,
% a compile flag as well as three forwarding files to set the flag.
% It consists of eight |.tex| files:
% \begin{center}
% \begin{tabular}{ll}
% |cdocsamp.tex|&main file\\
% |cdocsch1.tex|&include file for chapter 1\\
% |cdocsch2.tex|&include file for chapter 2\\
% |cdocspt3.tex|&include file for part 3\\
% |cdocspt4.tex|&include file for part 4\\
% |cdocsdrf.tex|&forwarding file for main file in draft mode\\
% |cdocsfi1.tex|&forwarding file for final version of chapter 1\\
% |cdocsfi2.tex|&forwarding file for final version of chapter 2\\
% \end{tabular}
% \end{center}
% Each of the eight files can be compiled directly by the \LaTeX{} compiler.
%
% %%%%%%%%%%%%%%%%%%%%%%%%%%%%%%%%%%%%%%
% \paragraph{Main File.}
%
% The main file is called |cdocsamp.tex|.
%
% Load the \textsf{childdoc} definitions and
% declare the filename for the main document:
%    \begin{macrocode}
\input{childdoc.def}
\childdocmain{}
%    \end{macrocode}

% Optional override for |\version| flag:
%    \begin{macrocode}
%%\ifchilddoc\else\providecommand{\version}{draft}\fi
%    \end{macrocode}

% Define the default values for the |\version| flag
% (|final| for the main file and |draft| for childs):
%    \begin{macrocode}
\ifchilddoc
\providecommand{\version}{draft}
\else
\providecommand{\version}{final}
\fi
%    \end{macrocode}

% Load the standard document class:
%    \begin{macrocode}
\documentclass[12pt]{article}
%    \end{macrocode}

% Start the document body:
%    \begin{macrocode}
\begin{document}
%    \end{macrocode}

% Declare a title page.
% Print title, part of document being processed and version flag:
%    \begin{macrocode}
\addtocounter{page}{-1}
\begin{center}
{\LARGE\bfseries{}childdoc example\par}
\vspace{1cm}
\ifchilddoc
\ifchilddocmanual part\else chapter\fi:
`\childdocname' of `\childdocjob'\par
\else
main document: `\childdocjob'\par
\fi
version: \version\par
\end{center}
\newpage
%    \end{macrocode}

% Manually include selected file,
% otherwise process as usual:
%    \begin{macrocode}
\ifchilddocmanual
\section*{part `\childdocname'}
\input{\childdocname}
\else
%    \end{macrocode}

% Include the two chapters:
%    \begin{macrocode}
\include{cdocsch1}
\include{cdocsch2}
%    \end{macrocode}

% Include the two parts unless only chapters should be displayed:
%    \begin{macrocode}
\ifchilddoc\else
\section{part three}
\input{cdocspt3}
\section{part four}
\input{cdocspt4}
\fi
%    \end{macrocode}

% Process as usual until here:
%    \begin{macrocode}
\fi
%    \end{macrocode}

% End of document body:
%    \begin{macrocode}
\end{document}
%    \end{macrocode}
%\iffalse
%</samplemain>
%\fi
%
% %%%%%%%%%%%%%%%%%%%%%%%%%%%%%%%%%%%%%%
% \paragraph{Chapter Include Files.}
%
% The include files are called |cdocsch1.tex| and |cdocsch2.tex|.
%
%\iffalse
%<*samplechap1|samplechap2>
%\fi

% Optional override for |\version| flag:
%    \begin{macrocode}
%%\providecommand{\version}{final}
%    \end{macrocode}

% Include the main document:
%    \begin{macrocode}
\input{childdoc.def}
\childdocof{cdocsamp}
%    \end{macrocode}

%\iffalse
%</samplechap1|samplechap2>
%\fi
%
%\iffalse
%<*samplechap1>
%\fi
% Some text for chapter 1:
%    \begin{macrocode}
\section{one}
some text in chapter one
%    \end{macrocode}

%\iffalse
%</samplechap1>
%\fi
% Some text for chapter 2:
%\iffalse
%<*samplechap2>
%\fi
%    \begin{macrocode}
\section{two}
more text in chapter two
%    \end{macrocode}

%\iffalse
%</samplechap2>
%\fi
%
% %%%%%%%%%%%%%%%%%%%%%%%%%%%%%%%%%%%%%%
% \paragraph{Part Include Files.}
%
% The include files are called |cdocspt3.tex| and |cdocspt4.tex|.
%
%\iffalse
%<*samplepart3|samplepart4>
%\fi

% Optional override for |\version| flag:
%    \begin{macrocode}
%%\providecommand{\version}{final}
%    \end{macrocode}

% Include the main document:
%    \begin{macrocode}
\input{childdoc.def}
\childdocby{cdocsamp}
%    \end{macrocode}

%\iffalse
%</samplepart3|samplepart4>
%\fi
%
%\iffalse
%<*samplepart3>
%\fi
% Some text for part 3:
%    \begin{macrocode}
some text in part three
%    \end{macrocode}

%\iffalse
%</samplepart3>
%\fi
% Some text for part 4:
%\iffalse
%<*samplepart4>
%\fi
%    \begin{macrocode}
more text in part four
%    \end{macrocode}

%\iffalse
%</samplepart4>
%\fi
%
% %%%%%%%%%%%%%%%%%%%%%%%%%%%%%%%%%%%%%%
% \paragraph{Forwarding for a Complete Draft.}
%
% The following forwarding file |cdocsdrf.tex|
% compiles the main document in draft mode:
%\iffalse
%<*sampledraft>
%\fi
%    \begin{macrocode}
\def\version{draft}
\input{childdoc.def}
\childdocforward{cdocsamp}
%    \end{macrocode}

%\iffalse
%</sampledraft>
%\fi
%
% %%%%%%%%%%%%%%%%%%%%%%%%%%%%%%%%%%%%%%
% \paragraph{Forwarding for Final Version of the Chapters.}
%
% The following forwarding files |cdocsfn1.tex| and |cdocsfn2.tex|
% (with identical content)
% compile the final versions of the child documents
% |cdocsch1.tex| and |cdocsch2.tex|, respectively:
%\iffalse
%<*samplefinal>
%\fi
%    \begin{macrocode}
\def\version{final}
\input{childdoc.def}
\childdocforwardprefix[cdocsamp]{cdocsfn}{cdocsch}
%    \end{macrocode}

%\iffalse
%</samplefinal>
%\fi
%
% %%%%%%%%%%%%%%%%%%%%%%%%%%%%%%%%%%%%%%
% \paragraph{Command Line Processing.}
%
% The following three command lines generate the output files
% |cdocscld|, |cdocscl1| and |cdocscl2|
% which should be identical to
% |cdocsdrf|, |cdocsch1| and |cdocsfn2|, respectively:
% \begin{center}
% \begin{tabular}{l}
% |latex -jobname cdocscld \|\\
% |  "\def\version{draft}\input{childdoc.def}\childdocforward{cdocsamp}"|\\
% |latex -jobname cdocscl1 \|\\
% |  "\input{childdoc.def}\childdocforward[cdocsamp]{cdocsch1}"|\\
% |latex -jobname cdocscl2 \|\\
% |  "\def\version{final}\input{childdoc.def}\childdocforward{cdocsch2}"|
% \end{tabular}
% \end{center}
% Note that the trailing backslash on each first line
% merely continues the input to the second line
% (for convenient cut ant paste).
% Furthermore, the command |latex| can be replaced by any
% of its alternative versions such as |pdflatex|.
%
% %%%%%%%%%%%%%%%%%%%%%%%%%%%%%%%%%%%%%%%%%%%%%%%%%%%%%%%%%%%%%%%%%%%%%%%%%%%%%%
% %%%%%%%%%%%%%%%%%%%%%%%%%%%%%%%%%%%%%%%%%%%%%%%%%%%%%%%%%%%%%%%%%%%%%%%%%%%%%%
% \section{Implementation}
%\iffalse
%<*package>
%\fi
%
% This section describes the definitions file |childdoc.def|.

% The definitions cannot be loaded using |\usepackage| or |\RequirePackage|
% which has a mechanism to prevent loading a style file more than once.
% When loading the definitions by means of |\input|
% multiple instances have to be prevented manually:
%\iffalse
%This code needs to be before the `\ProvidesFile' directive
%which is defined at the beginning of this file.
%Therefore it is also placed there and commented out here.
%</package>
%<*discard>
%\fi
%    \begin{macrocode}
\ifdefined\childdocmain\endinput\fi
%    \end{macrocode}
%\iffalse
%</discard>
%<*package>
%\fi
%
% \macro{\ifchilddoc}
% \macro{\ifchilddocmanual}
% The conditional |\ifchilddoc| tells whether a
% child (true) or main (false) document is being compiled.
% The conditional |\ifchilddocmanual| tells whether
% the |\includeonly| mechanism is used (false) or
% the selection of child files must be performed manually (true).
% The definitions initialise to false:
%    \begin{macrocode}
\newif\ifchilddoc
\newif\ifchilddocmanual
%    \end{macrocode}

% \macro{\childdocname}
% \macro{\childdocjob}
% The macro |\childdocname| stores the name of the main document
% to be compiled. The macro |\childdocjob| stores the name of
% the document on which the \LaTeX{} compiler was originally invoked.
% The content of |\jobname| cannot be compared
% to filenames specified in the source due to different catcodes.
% The following code rescans |\jobname|, stores the result
% in |\childdocname| and saves a copy in |\childdocjob|:
%    \begin{macrocode}
\edef\childdocname{\scantokens\expandafter{\jobname\noexpand}}
\let\childdocjob\childdocname
%    \end{macrocode}

% \macro{\childdocdisable}
% The macro |\childdocdisable| prevents the main file
% from being processed more than once.
% At this stage, the main document command |\childdocmain|
% is assumed to be called once again where it should do nothing.
% Any subsequent call to it should prevent
% a secondary processing of the main document
% It overwrites the forwarding commands
% |\childdocof| and |\childdocforward|
% with empty macros to prevent further inclusions of the main document:
%    \begin{macrocode}
\newcommand{\childdocdisable}
{
  \renewcommand{\childdocmain}[1]{\renewcommand{\childdocmain}[1]{\endinput}}
  \renewcommand{\childdocof}[1]{}
  \renewcommand{\childdocby}[2][]{}
  \renewcommand{\childdocforward}[2][]{}
  \renewcommand{\childdocdisable}{}
}
%    \end{macrocode}

% \macro{\childdocmain}
% The macro |\childdocmain| is to be called at the top of the main file
% with nothing or the main filename (without extension) as argument.
% First, it breaks loops.
% If the argument is not empty and does not match |\childdocname|
% (which is set by the first inclusion of |childdoc.def|),
% |\ifchilddoc| is set to true, |\includeonly| is applied to the child file
% and |\jobname| is set to the main file
% (for proper handling of |.aux| files):
%    \begin{macrocode}
\newcommand{\childdocmain}[1]
{
  \childdocdisable\childdocmain{}
  \if?#1?\else
    \begingroup
      \def\childdoctmp{#1}
      \ifx\childdoctmp\childdocname
        \def\childdoctmp{}
      \else
        \def\childdoctmp
        {
          \childdoctrue
          \includeonly{\childdocname}
          \def\childdocjob{#1}
          \def\jobname{#1}
        }
      \fi
      \expandafter
    \endgroup
    \childdoctmp
  \fi
}
%    \end{macrocode}

% \macro{\childdocof}
% The command |\childdocof| redirects
% compilation to the main file |#1|.
%    \begin{macrocode}
\newcommand{\childdocof}[1]
{
  \childdocdisable
  \childdoctrue
  \includeonly{\childdocname}
  \def\jobname{#1}
  \def\childdocjob{#1}
  \input{#1}
}
%    \end{macrocode}

% \macro{\childdocby}
% The command |\childdocby| ....
%    \begin{macrocode}
\newcommand{\childdocby}[2][]
{
  \childdocdisable
  \childdoctrue
  \childdocmanualtrue
  \if?#1?\else
    \def\jobname{#2}
  \fi
  \def\childdocjob{#2}
  \input{#2}
  \endinput
}
%    \end{macrocode}

% \macro{\childdocforward}
% The command |\childdocforward| redirects
% compilation to the main file or
% (if the optional argument is given) a child file.
% Parameters are set as if the main file
% or a child file starting with |\childdocof| was compiled.
% Then compilation is handed over to the main file:
%    \begin{macrocode}
\newcommand{\childdocforward}[2][]
{
  \begingroup
    \if?#1?
      \def\childdoctmp
      {
        \def\childdocname{#2}
        \def\childdocjob{#2}
        \def\jobname{#2}
        \input{#2}
        \endinput
      }
    \else
      \def\childdoctmp
      {
        \childdocdisable
        \def\childdocname{#2}
        \childdoctrue
        \includeonly{#2}
        \def\childdocjob{#1}
        \def\jobname{#1}
        \input{#1}
        \endinput
      }
    \fi
    \expandafter
  \endgroup
  \childdoctmp
}
%    \end{macrocode}

% \macro{\childdocforwardprefix}
% The command |\childdocforwardprefix| redirects
% compilation to the main or a child file by means of a pattern.
% The prefix |#1| in the current filename is replaced by |#2|
% and the suffix of the current filename is kept
% (it is assumed that the filename does not contain the substring `|~~~|'
% which is used as a delimiter).
% Compilation is handed over to the new file by |\childdocforward|:
%    \begin{macrocode}
\newcommand{\childdocforwardprefix}[3][]
{
  \begingroup
    \def\childdocextract #2##1~~~{\def\childdoctmp{\childdocforward[#1]{#3##1}}}
    \expandafter\childdocextract\childdocname~~~
    \expandafter
  \endgroup
  \childdoctmp
}
%    \end{macrocode}

% \macro{\childdoc}
% The deprecated macro |\childdoc| is a legacy version of |\childdocmain|:
%    \begin{macrocode}
\newcommand{\childdoc}{\childdocmain}
%    \end{macrocode}

% \macro{\childdocredirect}
% The deprecated macro |\childdocredirect| is a legacy version
% of |\childdocforward| and |\childdocforwardprefix|:
%    \begin{macrocode}
\newcommand{\childdocredirect}[2][]
{
  \begingroup
    \if?#1?
      \def\childdoctmp{\childdocforward{#2}}
    \else
      \def\childdoctmp{\childdocforwardprefix{#1}{#2}}
    \fi
    \expandafter
  \endgroup
  \childdoctmp
}
%    \end{macrocode}

%\iffalse
%</package>
%\fi
%
\endinput

\childdocforward{cdocsamp}
%    \end{macrocode}

%\iffalse
%</sampledraft>
%\fi
%
% %%%%%%%%%%%%%%%%%%%%%%%%%%%%%%%%%%%%%%
% \paragraph{Forwarding for Final Version of the Chapters.}
%
% The following forwarding files |cdocsfn1.tex| and |cdocsfn2.tex|
% (with identical content)
% compile the final versions of the child documents
% |cdocsch1.tex| and |cdocsch2.tex|, respectively:
%\iffalse
%<*samplefinal>
%\fi
%    \begin{macrocode}
\def\version{final}
% \iffalse
%
% childdoc.dtx Copyright (C) 2017-2018 Niklas Beisert
%
% This work may be distributed and/or modified under the
% conditions of the LaTeX Project Public License, either version 1.3
% of this license or (at your option) any later version.
% The latest version of this license is in
%   http://www.latex-project.org/lppl.txt
% and version 1.3 or later is part of all distributions of LaTeX
% version 2005/12/01 or later.
%
% This work has the LPPL maintenance status `maintained'.
%
% The Current Maintainer of this work is Niklas Beisert.
%
% This work consists of the files childdoc.dtx and childdoc.ins
% and the derived files childdoc.def and cdocsamp.tex with
% cdocsch1.tex, cdocsch2.tex, cdocsdrf.tex, cdocsfn1.tex, cdocsfn2.tex.
%
%<package>\ifdefined\childdocmain\endinput\fi
%<package>\ProvidesFile{childdoc.def}[2018/12/30 v2.0 child document driver]
%<samplemain>\ProvidesFile{cdocsamp.tex}[2018/12/30 v2.0 sample for childdoc]
%<*driver>
%\ProvidesFile{childdoc.drv}[2018/12/30 v2.0 childdoc reference manual file]
\PassOptionsToClass{10pt,a4paper}{article}
\documentclass{ltxdoc}

\usepackage[margin=35mm]{geometry}
\usepackage{hyperref}
\usepackage{hyperxmp}
\usepackage[usenames]{color}

\hypersetup{colorlinks=true}
\hypersetup{pdfstartview=FitH}
\hypersetup{pdfpagemode=UseNone}
\hypersetup{pdfsource={}}
\hypersetup{pdflang={en-UK}}
\hypersetup{pdfcopyright={Copyright 2017-2018 Niklas Beisert.
  This work may be distributed and/or modified under the
  conditions of the LaTeX Project Public License, either version 1.3
  of this license or (at your option) any later version.}}
\hypersetup{pdflicenseurl={http://www.latex-project.org/lppl.txt}}
\hypersetup{pdfcontactaddress={ETH Zurich, ITP, HIT K,
  Wolfgang-Pauli-Strasse 27}}
\hypersetup{pdfcontactpostcode={8093}}
\hypersetup{pdfcontactcity={Zurich}}
\hypersetup{pdfcontactcountry={Switzerland}}
\hypersetup{pdfcontactemail={nbeisert@itp.phys.ethz.ch}}
\hypersetup{pdfcontacturl={http://people.phys.ethz.ch/\xmptilde nbeisert/}}

\newcommand{\secref}[1]{\hyperref[#1]{section \ref*{#1}}}

\parskip1ex
\parindent0pt
\let\olditemize\itemize
\def\itemize{\olditemize\parskip0pt}

\begin{document}

\title{The \textsf{childdoc} Package}
\hypersetup{pdftitle={The childdoc Package}}
\author{Niklas Beisert\\[2ex]
  Institut f\"ur Theoretische Physik\\
  Eidgen\"ossische Technische Hochschule Z\"urich\\
  Wolfgang-Pauli-Strasse 27, 8093 Z\"urich, Switzerland\\[1ex]
  \href{mailto:nbeisert@itp.phys.ethz.ch}
  {\texttt{nbeisert@itp.phys.ethz.ch}}}
\hypersetup{pdfauthor={Niklas Beisert}}
\hypersetup{pdfsubject={Manual for the LaTeX2e Package childdoc}}
\date{30 December 2018, \textsf{v2.0}}
\maketitle

\begin{abstract}\noindent
\textsf{childdoc} is a \LaTeXe{} package
that enables the direct compilation
of document sections included by |\include|
to individual files.
\end{abstract}

\begingroup
\parskip0ex
\tableofcontents
\endgroup

%%%%%%%%%%%%%%%%%%%%%%%%%%%%%%%%%%%%%%%%%%%%%%%%%%%%%%%%%%%%%%%%%%%%%%%%%%%%%%%%
%%%%%%%%%%%%%%%%%%%%%%%%%%%%%%%%%%%%%%%%%%%%%%%%%%%%%%%%%%%%%%%%%%%%%%%%%%%%%%%%
\section{Introduction}

\LaTeX{} provides a mechanism to structure a large document (such as a book)
into a main file and several child files (containing the chapters)
using the |\include| command.
This mechanism is beneficial for documents
which span hundreds of pages in order to
make the source file(s) more manageable.
Moreover, compilation can be restricted to
selected child files by means of the |\includeonly| command.
The latter feature can be used to reduce the compilation time while editing
(this was significantly more useful in the earlier days of \LaTeX{})
or to generate a smaller document which is easier to navigate.
Another application of |\includeonly| is to generate
documents consisting of selected parts of the complete document.

However, there are a few drawbacks of the plain |\include| mechanism:
\begin{itemize}
\item
The child files cannot be compiled on their own,
they can only be compiled via the main file.
A naive editing environment
(such as a text editor with an option
to have the current file processed by \LaTeX)
may require one to switch to the main file before compiling;
attempting to compile the child file produces errors.
\item
The main file must be modified (each time)
to adjust the |\includeonly| command
to the present needs. This easily leaves the main file in a messy state.
\item
The generated document will always carry the filename
of the main document. This is inconvenient if
several child files are to be compiled and
to be kept for distribution.
\end{itemize}

The present package provides a simple interface
to make child files individually compilable by \LaTeX{}.
Compiling a child file then has the same effect as compiling
the main file with an |\includeonly| command
to select the appropriate child.
Moreover the generated document will carry the name of the child
rather than the main file.
This resolves all three above issues.

This feature is meant to make the editing of books,
thesis documents and lecture notes somewhat more convenient.
However, the package can also be used efficiently for
composing a series of documents (such as exercise sheets)
which are typically distributed individually.
It then assists the author in generating the individual documents
(potentially in different versions)
as well as a document containing the collected series.
Another application is in developing style files
or other kinds of included material
where compilation of the style file could redirect
to a sample or test file.

%%%%%%%%%%%%%%%%%%%%%%%%%%%%%%%%%%%%%%%%%%%%%%%%%%%%%%%%%%%%%%%%%%%%%%%%%%%%%%%%
%%%%%%%%%%%%%%%%%%%%%%%%%%%%%%%%%%%%%%%%%%%%%%%%%%%%%%%%%%%%%%%%%%%%%%%%%%%%%%%%
\section{Usage}

First of all, the package \textsf{childdoc} is \emph{not} a standard
\LaTeXe{} |.sty| style file! Therefore it needs to be invoked in
a non-standard way.

%%%%%%%%%%%%%%%%%%%%%%%%%%%%%%%%%%%%%%%%%%%%%%%%%%%%%%%%%%%%%%%%%%%%%%%%%%%%%%%%
\subsection{Included Files}
\label{sec:include}

%%%%%%%%%%%%%%%%%%%%%%%%%%%%%%%%%%%%%%%%
\DescribeMacro{\childdocmain}
To use the package, add the commands
\begin{center}
\begin{tabular}{l}
|\input{childdoc.def}|\\
|\childdocmain{}|\\
\end{tabular}
\end{center}
at the very top of the main \LaTeX{} file,
in particular \emph{before} the |\documentclass| statement!
The argument of |\childdocmain| should be left empty
(but it must be present).

%%%%%%%%%%%%%%%%%%%%%%%%%%%%%%%%%%%%%%%%
\DescribeMacro{\childdocof}
Furthermore, add the commands
\begin{center}
\begin{tabular}{l}
|\input{childdoc.def}|\\
|\childdocof{|\textit{main}|}|\\
\end{tabular}
\end{center}
at the top of every child file \textit{child}
which is included by |\include{|\textit{child}|}|
from within the main file
(or at least for those files to be compiled individually).
The argument \textit{main} must be the filename of the main file.

There are a couple of
considerations in setting up the main and child documents:

%%%%%%%%%%%%%%%%%%%%%%%%%%%%%%%%%%%%%%%%
\paragraph{Restrictions.}

Please note the following restrictions:
\begin{itemize}
\item
|\childdocmain| must be called with one argument \textit{main}
to ensure compatibility with earlier version of the package.
It must either be empty (|\childdocmain{}|)
or precisely match the filename of the main file in which it is specified.
See \secref{sec:detection} for further information.
\item
The filename \textit{main} must be specified without the |.tex| extension.
\item
The filename \textit{main} is case sensitive
(even in case-insensitive file systems)
due to internal string comparison.
\item
The argument \textit{main} should be fully expanded, it cannot be a macro.
\item
Subdirectories and special characters should be avoided in filenames.
\item
The command |\childdocmain{|\textit{main}|}| must be followed by a whitespace.
It should not be followed immediately by another command
or by a comment mark `|%|'.
This is because the \TeX{} parser reads the token immediately following
the argument of |\childdocmain| and puts it
at the beginning of every child section;
however, a white\-space is ignored.
\end{itemize}

%%%%%%%%%%%%%%%%%%%%%%%%%%%%%%%%%%%%%%%%
\paragraph{Content of Main File.}

It is advisable to place all content in the child files included by |\include|.
Any output contained in the main file will appear in all child documents
unless suppressed manually;
it cannot be suppressed automatically by the |\includeonly| directive
and thus should normally be avoided.
A method to include some content in the main file
by means of conditional processing is described in \secref{sec:conditional}.

%%%%%%%%%%%%%%%%%%%%%%%%%%%%%%%%%%%%%%%%
\paragraph{Page Numbering.}

When only a part of the document is compiled,
the appropriate numbering of pages
(as well as other status parameters)
is determined from the |.aux| files.
The latter contain information from previous passes.
However this information needs to propagate through
all intermediate child documents.
Therefore the page numbering in child documents may well
be inconsistent until the complete document is compiled at least once.

A useful (if unconventional) way to always ensure a consistent
page numbering is to restart the numbering in each child document
and denote the pages by `\textit{child}|.|\textit{page}'
where \textit{child} represents the chapter/section number of the child file.
This can be achieved by the command
|\numberwithin{page}{|\textit{child}|}|
of the \textsf{amsmath} package
where \textit{child} can be |chapter| or |section|
depending on the chosen structuring.
Alternatively, one can modify the macro |\thepage| appropriately
and reset the counter |page| at the start of each child file.

%%%%%%%%%%%%%%%%%%%%%%%%%%%%%%%%%%%%%%%%%%%%%%%%%%%%%%%%%%%%%%%%%%%%%%%%%%%%%%%%
\subsection{Conditional Processing}
\label{sec:conditional}

The package provides a mechanism to compile different versions
of a document. To customise the versions further some conditional processing
can come in handy to distinguish which version is being compiled.
The package provides two macros to describe the compilation context:

%%%%%%%%%%%%%%%%%%%%%%%%%%%%%%%%%%%%%%%%
\DescribeMacro{\ifchilddoc}
The conditional |\ifchilddoc| distinguishes between the compilation of
child documents and the main document:
%
\begin{center}
|\ifchilddoc |\textit{child-code}| |[|\||else |\textit{main-code}]| \||fi|
\end{center}

%%%%%%%%%%%%%%%%%%%%%%%%%%%%%%%%%%%%%%%%
\DescribeMacro{\childdocname}
\DescribeMacro{\childdocjob}
The macro |\childdocname| contains the filename (without extension)
of the main or child file being processed.
Note that |\childdocjob| will always contain the name of the main file.

%%%%%%%%%%%%%%%%%%%%%%%%%%%%%%%%%%%%%%%%
\paragraph{Title Page.}

Conditional processing can be used to include a title or banner page
in the main document when proper precautions are taken.
Importantly, the code in the main file should ensure that the page counter
(as well as other status parameters which are stored in the |.aux| files)
takes the same value after the conditional processing.
Otherwise the page numbers may take divergent values
depending on which part is compiled.

For example, a title page could be declared by:
%
\begin{center}
\begin{tabular}{l}
|\ifchilddoc\||else|\\
|\addtocounter{page}{-1}|\\
\textit{code for title page}\\
|\newpage|\\
|\||fi|
\end{tabular}
\end{center}
%
A banner page for the child documents can be generated by:
%
\begin{center}
\begin{tabular}{l}
|\ifchilddoc|\\
|\addtocounter{page}{-1}|\\
\textit{code for banner page}\\
|\newpage|\\
|\||fi|
\end{tabular}
\end{center}
%
Here one could write a message such as:
\begin{center}
|This is the part \childdocname{} of \childdocjob{}.|
\end{center}

%%%%%%%%%%%%%%%%%%%%%%%%%%%%%%%%%%%%%%%%%%%%%%%%%%%%%%%%%%%%%%%%%%%%%%%%%%%%%%%%
\subsection{Flags}
\label{sec:flags}

The package makes it easy to generate different versions
of the main or child documents.
To this end compilation flags can be defined
and assigned different default values.
They will be particularly useful in conjunction
with the forwarding mechanism described in \secref{sec:forward}.

For example, it may be useful to have a flag |\version|
which can be set to |draft| or |final|.
The document source will contain some conditional code
depending on the value of |\version|.
Suppose further, the flag should default to |final| for the main file
and to |draft| for child files
which is a natural assignment for editing the document.
This is achieved by placing the following code
in the preamble of the main document
(below the |\childdocmain| directive):
%
\begin{center}
\begin{tabular}{l}
|\ifchilddoc|\\
|\providecommand{\version}{draft}|\\
|\||else|\\
|\providecommand{\version}{final}|\\
|\||fi|
\end{tabular}
\end{center}
%
The definition by |\providecommand| makes sure
that previous definitions are not overwritten.
Further statements |\providecommand{\version}{...}|
can thus be added before the above code to override it.

For the main file, one might add a line
(between |\childdocmain| and the above block)
%
\begin{center}
|%\ifchilddoc\||else\providecommand{\version}{draft}\||fi|
\end{center}
%
which can be uncommented to produce a draft version.
Likewise one can add a line to the very top of a child file
(above the |\childdocof{|\textit{main}|}| directive)
%
\begin{center}
|%\providecommand{\version}{final}|
\end{center}
%
which can be uncommented to produce the final version of this child document.

%%%%%%%%%%%%%%%%%%%%%%%%%%%%%%%%%%%%%%%%%%%%%%%%%%%%%%%%%%%%%%%%%%%%%%%%%%%%%%%%
\subsection{Forwarding}
\label{sec:forward}

Different versions of the main or child documents
using compilation flags as described in \secref{sec:flags}
can be (permanently) stored in different files
for convenient compilation, viewing and distribution.
To this end, the package defines a command
to pass on compilation to a different file:

%%%%%%%%%%%%%%%%%%%%%%%%%%%%%%%%%%%%%%%%
\DescribeMacro{\childdocforward}
The command |\childdocforward| redirects processing to
another source file:
%
\begin{center}
\begin{tabular}{l}
|\input{childdoc.def}|\\
|\childdocforward[|\textit{main}|]{|\textit{dest}|}|\\
\end{tabular}
\end{center}
%
The argument \textit{dest} is the destination file
(without extension).
It should be the main file or one of the child files.
Note that further \textsf{childdoc} directives
such as |\childdocof| and |\childdocforward|
in the indicated file will be processed in this form.
The optional argument \textit{main}
passes on directly to the main file \textit{main}
while pretending to compile the child \textit{dest}.
This form behaves as if \textit{dest}
issues |\childdocof{|\textit{main}|}| right away,
and no further \textsf{childdoc} directives will be processed.

%%%%%%%%%%%%%%%%%%%%%%%%%%%%%%%%%%%%%%%%
\DescribeMacro{\...prefix}
In the alternative form |\childdocforwardprefix|,
%
\begin{center}
\begin{tabular}{l}
|\input{childdoc.def}|\\
|\childdocforwardprefix[|\textit{main}|]{|\textit{prefix}|}{|\textit{dest}|}|
\end{tabular}
\end{center}
%
the destination file is determined by a pattern
depending on the current file:
To make this work, the current file must be called
`{\textit{prefix}\hspace{0.2em}\textit{suffix}}'
with \textit{prefix} matching precisely the argument.
Processing is then passed on to the file
`{\textit{dest}\hspace{0.2em}\textit{suffix}}'.
Surely, the same effect is achieved by
directly specifying the
argument `{\textit{dest}\hspace{0.2em}\textit{suffix}}'
in the first form.
However, that requires to set up a different file
for each child. With the alternative form of the command
all these files can have exactly the same content
which simplifies setting them up and maintaining them.

For example, the following file |draft.tex|
with a compilation flag |\version| as described in \secref{sec:flags}
compiles the main document as a draft:
%
\begin{center}
\begin{tabular}{l}
|\def\version{draft}|\\
|\input{childdoc.def}|\\
|\childdocforward{|\textit{main}|}|
\end{tabular}
\end{center}
%
Likewise, the following files |final|\textit{nn}|.tex|
compile the final version of the child document
|child|\textit{nn}|.tex|:
%
\begin{center}
\begin{tabular}{l}
|\def\version{final}|\\
|\input{childdoc.def}|\\
|\childdocforwardprefix{final}{child}|
\end{tabular}
\end{center}
%

Note that when several versions of a main file and/or of each child file
are to be generated, it may be convenient to set up a |Makefile| or
shell script to automatise the process.

%%%%%%%%%%%%%%%%%%%%%%%%%%%%%%%%%%%%%%%%%%%%%%%%%%%%%%%%%%%%%%%%%%%%%%%%%%%%%%%%
\subsection{Command Line Processing}
\label{sec:commandline}

The effect of redirection files can also be achieved by invoking
the \LaTeX{} compiler with a more elaborate command line.
Most conveniently this should be done as part
of a shell script or a |Makefile|.

When using \textsf{childdoc} in the main file, the following
command lines effectively perform a redirection
(note that depending on the shell being used,
backslashes may have to be doubled: `|\|' $\to$ `|\\|'):
%
\begin{center}
|... -jobname "|\textit{target}|" |\\|"|[\textit{flags}]%
|\input{childdoc.def}\childdocforward[|\textit{main}|]{|\textit{dest}|}"|
\end{center}
%
Here \textit{target} is the name of the output file,
\textit{main} is the name of the main file
and \textit{dest} is the name of the main or child file to be processed
(all filenames without extensions).
The optional argument \textit{main} can be omitted
if \textit{main} matches \textit{dest}.
Optionally, compilation \textit{flags} can be defined via |\def| commands.
This command line makes the \TeX{} engine believe
it is compiling the file \textit{target}
whose content is specified as the latter parameter.
The provided code then forwards the processing to
\textit{main} or \textit{dest} as described in \secref{sec:forward}.

%%%%%%%%%%%%%%%%%%%%%%%%%%%%%%%%%%%%%%%%%%%%%%%%%%%%%%%%%%%%%%%%%%%%%%%%%%%%%%%%
\subsection{Include by Input}
\label{sec:input}

Including child documents by |\include| has some restrictions by design.
Most notably, the content of a child document always occupies
its own set of pages; pages cannot be shared between child documents.
Usually, this behaviour makes perfect sense
because each child document contain an essential part of the document.
However, in some situations it may be desirable to compose
a document from a collection of parts
without having mandatory page breaks between then.
For this case, the package
provides a mechanism to include parts
by |\input| which can also be processed individually.
However, by construction this mechanism
requires manual handling of the content to be output.

%%%%%%%%%%%%%%%%%%%%%%%%%%%%%%%%%%%%%%%%
\DescribeMacro{\ifchilddocmanual}
The main file should be prepared as usual, see \secref{sec:include}.
However, the document body must make a distinction
between processing of an individual part and of the main document, e.g.:
%
\begin{center}
\begin{tabular}{l}
|\ifchilddocmanual|\\
|\input{\childdocname}|\\
|\||else|\\
\textit{document body with }|\input{|\textit{part}|}|\\
|\||fi|
\end{tabular}
\end{center}
%
The conditional |\ifchilddocmanual| is true whenever
a part to be included by |\input| is being compiled,
and the name of the part is stored in |\childdocname|.

%%%%%%%%%%%%%%%%%%%%%%%%%%%%%%%%%%%%%%%%
\DescribeMacro{\childdocby}
Each part to be included by |\input| should start with:
%
\begin{center}
\begin{tabular}{l}
|\input{childdoc.def}|\\
|\childdocby{|\textit{main}|}|\\
\end{tabular}
\end{center}
%
The directive |\childdocby| is similar to |\childdocof|
described in \secref{sec:include},
but the subsequent selection of content must be done manually.
To that end, both |\ifchilddoc| and |\ifchilddocmanual|
will be true upon processing of a part,
and the name of the part is stored in |\childdocname|.
Note that |\jobname| will be set to the filename of the current part
so that each part receives an individual |.aux| file
that does not interfere with the |.aux| file(s) of the main document.
This behaviour can be altered by the alternative form
|\childdocby[*]{|\textit{main}|}| (with a non-empty optional argument)
which uses the |.aux| file of the main document
by setting |\jobname| to \textit{main}.

%%%%%%%%%%%%%%%%%%%%%%%%%%%%%%%%%%%%%%%%%%%%%%%%%%%%%%%%%%%%%%%%%%%%%%%%%%%%%%%%
\subsection{Driver Development}
\label{sec:driver}

The \textsf{childdoc} mechanism can also be use for the development
of definition files such as \LaTeX{} styles or classes.
This case differs from the above setup with multiple parts
included by |\include| in that no |\includeonly| should be invoked.
This can be achieved by starting the include file
(before |\ProvidesPackage|) with:
%
\begin{center}
\begin{tabular}{l}
|\input{childdoc.def}|\\
|\childdocforward{|\textit{main}|}|\\
\end{tabular}
\end{center}
%
or alternatively with:
%
\begin{center}
\begin{tabular}{l}
|\input{childdoc.def}|\\
|\childdocby{|\textit{main}|}|\\
\end{tabular}
\end{center}
%
Both forms have slightly different effects as described above.
The main file is prepared as usual, see \secref{sec:include}.

%%%%%%%%%%%%%%%%%%%%%%%%%%%%%%%%%%%%%%%%%%%%%%%%%%%%%%%%%%%%%%%%%%%%%%%%%%%%%%%%
\subsection{Legacy Detection}
\label{sec:detection}

The directive |\childdocmain| in the main file can detect
whether the complete document or merely a child is to be compiled
even without using the directive |\childdocof|.
This method is deprecated because it is less robust
and there is no compelling reason to use it;
it is merely provided for backward compatibility
and it may be removed in future versions.

If the detection mechanism is to be used,
it is mandatory to correctly specify
the filename of the main file as the argument of |\childdocmain|:
%
\begin{center}
\begin{tabular}{l}
|\input{childdoc.def}|\\
|\childdocmain{|\textit{main}|}|\\
\end{tabular}
\end{center}
%
If |\jobname| does not match the argument \textit{main} of |\childdocmain|,
it is assumed that |\jobname| points to the child file to be compiled.
When using |\childdocmain| with the main file specified as argument,
it suffices to start a child file
with just |\input{|\textit{main}|}|
without loading of the package and using |\childdocof|.
If instead all processing is done
with the appropriate \textsf{childdoc} directives,
the argument of \textit{main} of |\childdocmain| can be empty.

An alternative version of the command line processing described
in \secref{sec:commandline} using the detection mechanism reads:
%
\begin{center}
|... -jobname "|\textit{target}|" "|[\textit{flags}]%
[|\def\jobname{|\textit{dest}|}|]|\input{|\textit{main}|}"|
\end{center}

%%%%%%%%%%%%%%%%%%%%%%%%%%%%%%%%%%%%%%%%%%%%%%%%%%%%%%%%%%%%%%%%%%%%%%%%%%%%%%%%
\subsection{Manual Code}
\label{sec:manual}

In case one cannot be certain whether the definitions file |childdoc.def|
is installed on the target \TeX{} distribution
and one prefers not to ship it,
it is conceivable to paste a few relevant commands into the sources.

To that end, drop all statements |\input{childdoc.def}|
and perform the replacements as outlined below.
Instead of |\childdocmain{|\textit{main}|}| add the following code
to the top of the main file:
%
\begin{center}
\begin{tabular}{l}
|\||ifdefined\childdocname\endinput\||fi\newif\ifchilddoc|\\
|\edef\childdocname{\scantokens\expandafter{\jobname\noexpand}}|\\
|\def\childdocmain{|\textit{main}|}\||ifx\childdocmain\childdocname\||else|\\
|\childdoctrue\includeonly{\childdocname}\let\jobname\childdocmain\||fi|\\
\end{tabular}
\end{center}
%
Instead of |\childdocof{|\textit{main}|}| just include the main file
at the top of each child file:
%
\begin{center}
|\input{|\textit{main}|}|
\end{center}
%
A simple redirection |\childdocforward{|\textit{dest}|}| is achieved by:
%
\begin{center}
|\def\jobname{|\textit{dest}|}\input{\jobname}|
\end{center}
%
The redirection with prefix
|\childdocforwardprefix[|\textit{prefix}|]{|\textit{dest}|}|
is accomplished by:
%
\begin{center}
\begin{tabular}{l}
|{\edef\jobname{\scantokens\expandafter{\jobname\noexpand}}|\\
|\def\redirectjob |\textit{prefix}|#1~~~{\gdef\jobname{|\textit{dest}|#1}}|\\
|\expandafter\redirectjob\jobname~~~}\input{\jobname}|
\end{tabular}
\end{center}

In an alternative approach,
child documents can be compiled by a specific command line
without additional code or specific definitions:
%
\begin{center}
|... -jobname "|\textit{target}|" "|[\textit{flags}]%
|\includeonly{|\textit{dest}|}\input{|\textit{main}|}"|
\end{center}
%

%%%%%%%%%%%%%%%%%%%%%%%%%%%%%%%%%%%%%%%%%%%%%%%%%%%%%%%%%%%%%%%%%%%%%%%%%%%%%%%%
%%%%%%%%%%%%%%%%%%%%%%%%%%%%%%%%%%%%%%%%%%%%%%%%%%%%%%%%%%%%%%%%%%%%%%%%%%%%%%%%
\section{Information}

%%%%%%%%%%%%%%%%%%%%%%%%%%%%%%%%%%%%%%%%%%%%%%%%%%%%%%%%%%%%%%%%%%%%%%%%%%%%%%%%
\subsection{Copyright}

Copyright \copyright{} 2017--2018 Niklas Beisert

This work may be distributed and/or modified under the
conditions of the \LaTeX{} Project Public License, either version 1.3
of this license or (at your option) any later version.
The latest version of this license is in
  \url{http://www.latex-project.org/lppl.txt}
and version 1.3 or later is part of all distributions of \LaTeX{}
version 2005/12/01 or later.

This work has the LPPL maintenance status `maintained'.

The Current Maintainer of this work is Niklas Beisert.

This work consists of the files |README.txt|, |childdoc.ins| and |childdoc.dtx|
as well as the derived files |childdoc.def|, |cdocsamp.tex|
with |cdocsch1.tex|, |cdocsch2.tex|, |cdocspt3.tex|, |cdocspt4.tex|,
|cdocsdrf.tex|, |cdocsfn1.tex|, |cdocsfn2.tex|
as well as |childdoc.pdf|.

%%%%%%%%%%%%%%%%%%%%%%%%%%%%%%%%%%%%%%%%%%%%%%%%%%%%%%%%%%%%%%%%%%%%%%%%%%%%%%%%
\subsection{Files and Installation}

The package consists of the files:
%
\begin{center}
\begin{tabular}{ll}
    |README.txt|   & readme file \\
    |childdoc.ins| & installation file \\
    |childdoc.dtx| & source file \\
    |childdoc.def| & definition file \\
    |cdocsamp.tex| & sample main file \\
    |cdocsch1.tex| & sample include file \\
    |cdocsch2.tex| & sample include file \\
    |cdocspt3.tex| & sample part file \\
    |cdocspt4.tex| & sample part file \\
    |cdocsdrf.tex| & sample redirection file \\
    |cdocsfn1.tex| & sample redirection file \\
    |cdocsfn2.tex| & sample redirection file \\
    |childdoc.pdf| & manual
\end{tabular}
\end{center}
%
The distribution consists of the files
|README.txt|, |childdoc.ins| and |childdoc.dtx|.
%
\begin{itemize}
\item
Run (pdf)\LaTeX{} on |childdoc.dtx|
to compile the manual |childdoc.pdf| (this file).
\item
Run \LaTeX{} on |childdoc.ins| to create the definitions file |childdoc.def|
and the sample |cdocsamp.tex| with include files
|cdocsch1.tex|, |cdocsch2.tex|, |cdocspt3.tex|, |cdocspt4.tex|,
|cdocsdrf.tex|, |cdocsfn1.tex|, |cdocsfn2.tex|.
Then copy the file |childdoc.def| to an appropriate directory of your \LaTeX{}
distribution, e.g.\ \textit{texmf-root}|/tex/latex/childdoc|.
\end{itemize}

%%%%%%%%%%%%%%%%%%%%%%%%%%%%%%%%%%%%%%%%%%%%%%%%%%%%%%%%%%%%%%%%%%%%%%%%%%%%%%%%
\subsection{Related CTAN Packages}

There are several other packages which offer a similar functionality:
%
\begin{itemize}
\item
The packages
\href{http://ctan.org/pkg/docmute}{\textsf{docmute}},
\href{http://ctan.org/pkg/includex}{\textsf{includex}} and
\href{http://ctan.org/pkg/standalone}{\textsf{standalone}}
provide commands to include only the document body of
a child file thus allowing both files to be compiled individually.
\item
The packages \href{http://ctan.org/pkg/subdocs}{\textsf{subdocs}}
and \href{http://ctan.org/pkg/subfiles}{\textsf{subfiles}}
provide structures in which the main and child documents can be
encapsulated and allowing them to be compiled individually.
The inclusion mechanism is different from the conventional |\include|.
\item
The package \href{http://ctan.org/pkg/combine}{\textsf{combine}}
is an elaborate solution to combine several documents into one.
\end{itemize}
%
See also the CTAN topic \href{http://ctan.org/topic/subdocs}{\textsf{subdocs}}
for further related packages.
The present package differs from the above solutions in that
a document structure constructed with the conventional |\include| mechanism
just needs two extra commands at the top of every file
such that all constituent files can be compiled individually.

%%%%%%%%%%%%%%%%%%%%%%%%%%%%%%%%%%%%%%%%%%%%%%%%%%%%%%%%%%%%%%%%%%%%%%%%%%%%%%%%
%\subsection{Feature Suggestions}
%
%The following is a list of features which may be useful for future
%versions of this package:
%%
%\begin{itemize}
%\item
%\ldots
%\end{itemize}

%%%%%%%%%%%%%%%%%%%%%%%%%%%%%%%%%%%%%%%%%%%%%%%%%%%%%%%%%%%%%%%%%%%%%%%%%%%%%%%%
\subsection{Revision History}

%%%%%%%%%%%%%%%%%%%%%%%%%%%%%%%%%%%%%%%%
\paragraph{v2.0:} 2018/12/30

\begin{itemize}
\item
immediate forward processing
\item
added |\childdocby| mechanism
\item
manual restructured
\end{itemize}

%%%%%%%%%%%%%%%%%%%%%%%%%%%%%%%%%%%%%%%%
\paragraph{v1.6:} 2018/01/17

\begin{itemize}
\item
application for development of include files
\item
corrections to manual
\end{itemize}

%%%%%%%%%%%%%%%%%%%%%%%%%%%%%%%%%%%%%%%%
\paragraph{v1.5:} 2017/05/21

\begin{itemize}
\item
more complete structuring introduced
\item
|\childdocof| introduced
\item
|\childdoc| renamed to |\childdocmain|
\item
|\childredirect| renamed to |\childdocforward| and |\childdocforwardprefix|
and functionality expanded
\end{itemize}

%%%%%%%%%%%%%%%%%%%%%%%%%%%%%%%%%%%%%%%%
\paragraph{v1.0:} 2017/04/27

\begin{itemize}
\item
manual and install package
\item
first version published on CTAN
\end{itemize}

%%%%%%%%%%%%%%%%%%%%%%%%%%%%%%%%%%%%%%%%
\paragraph{v0.6:} 2017/04/26

\begin{itemize}
\item
redirection mechanism added
\end{itemize}

%%%%%%%%%%%%%%%%%%%%%%%%%%%%%%%%%%%%%%%%
\paragraph{v0.5:} 2017/04/26

\begin{itemize}
\item
functionality in definition file
\end{itemize}


%%%%%%%%%%%%%%%%%%%%%%%%%%%%%%%%%%%%%%%%%%%%%%%%%%%%%%%%%%%%%%%%%%%%%%%%%%%%%%%%
%%%%%%%%%%%%%%%%%%%%%%%%%%%%%%%%%%%%%%%%%%%%%%%%%%%%%%%%%%%%%%%%%%%%%%%%%%%%%%%%
%%%%%%%%%%%%%%%%%%%%%%%%%%%%%%%%%%%%%%%%%%%%%%%%%%%%%%%%%%%%%%%%%%%%%%%%%%%%%%%%
\appendix

\settowidth\MacroIndent{\rmfamily\scriptsize 000\ }

 \DocInput{childdoc.dtx}

\end{document}
%</driver>
% \fi
%
% %%%%%%%%%%%%%%%%%%%%%%%%%%%%%%%%%%%%%%%%%%%%%%%%%%%%%%%%%%%%%%%%%%%%%%%%%%%%%%
% %%%%%%%%%%%%%%%%%%%%%%%%%%%%%%%%%%%%%%%%%%%%%%%%%%%%%%%%%%%%%%%%%%%%%%%%%%%%%%
% \section{Sample}
%\iffalse
%<*samplemain>
%\fi
%
% The following presents a sample document
% with two chapters, two parts, a title page,
% a compile flag as well as three forwarding files to set the flag.
% It consists of eight |.tex| files:
% \begin{center}
% \begin{tabular}{ll}
% |cdocsamp.tex|&main file\\
% |cdocsch1.tex|&include file for chapter 1\\
% |cdocsch2.tex|&include file for chapter 2\\
% |cdocspt3.tex|&include file for part 3\\
% |cdocspt4.tex|&include file for part 4\\
% |cdocsdrf.tex|&forwarding file for main file in draft mode\\
% |cdocsfi1.tex|&forwarding file for final version of chapter 1\\
% |cdocsfi2.tex|&forwarding file for final version of chapter 2\\
% \end{tabular}
% \end{center}
% Each of the eight files can be compiled directly by the \LaTeX{} compiler.
%
% %%%%%%%%%%%%%%%%%%%%%%%%%%%%%%%%%%%%%%
% \paragraph{Main File.}
%
% The main file is called |cdocsamp.tex|.
%
% Load the \textsf{childdoc} definitions and
% declare the filename for the main document:
%    \begin{macrocode}
\input{childdoc.def}
\childdocmain{}
%    \end{macrocode}

% Optional override for |\version| flag:
%    \begin{macrocode}
%%\ifchilddoc\else\providecommand{\version}{draft}\fi
%    \end{macrocode}

% Define the default values for the |\version| flag
% (|final| for the main file and |draft| for childs):
%    \begin{macrocode}
\ifchilddoc
\providecommand{\version}{draft}
\else
\providecommand{\version}{final}
\fi
%    \end{macrocode}

% Load the standard document class:
%    \begin{macrocode}
\documentclass[12pt]{article}
%    \end{macrocode}

% Start the document body:
%    \begin{macrocode}
\begin{document}
%    \end{macrocode}

% Declare a title page.
% Print title, part of document being processed and version flag:
%    \begin{macrocode}
\addtocounter{page}{-1}
\begin{center}
{\LARGE\bfseries{}childdoc example\par}
\vspace{1cm}
\ifchilddoc
\ifchilddocmanual part\else chapter\fi:
`\childdocname' of `\childdocjob'\par
\else
main document: `\childdocjob'\par
\fi
version: \version\par
\end{center}
\newpage
%    \end{macrocode}

% Manually include selected file,
% otherwise process as usual:
%    \begin{macrocode}
\ifchilddocmanual
\section*{part `\childdocname'}
\input{\childdocname}
\else
%    \end{macrocode}

% Include the two chapters:
%    \begin{macrocode}
\include{cdocsch1}
\include{cdocsch2}
%    \end{macrocode}

% Include the two parts unless only chapters should be displayed:
%    \begin{macrocode}
\ifchilddoc\else
\section{part three}
\input{cdocspt3}
\section{part four}
\input{cdocspt4}
\fi
%    \end{macrocode}

% Process as usual until here:
%    \begin{macrocode}
\fi
%    \end{macrocode}

% End of document body:
%    \begin{macrocode}
\end{document}
%    \end{macrocode}
%\iffalse
%</samplemain>
%\fi
%
% %%%%%%%%%%%%%%%%%%%%%%%%%%%%%%%%%%%%%%
% \paragraph{Chapter Include Files.}
%
% The include files are called |cdocsch1.tex| and |cdocsch2.tex|.
%
%\iffalse
%<*samplechap1|samplechap2>
%\fi

% Optional override for |\version| flag:
%    \begin{macrocode}
%%\providecommand{\version}{final}
%    \end{macrocode}

% Include the main document:
%    \begin{macrocode}
\input{childdoc.def}
\childdocof{cdocsamp}
%    \end{macrocode}

%\iffalse
%</samplechap1|samplechap2>
%\fi
%
%\iffalse
%<*samplechap1>
%\fi
% Some text for chapter 1:
%    \begin{macrocode}
\section{one}
some text in chapter one
%    \end{macrocode}

%\iffalse
%</samplechap1>
%\fi
% Some text for chapter 2:
%\iffalse
%<*samplechap2>
%\fi
%    \begin{macrocode}
\section{two}
more text in chapter two
%    \end{macrocode}

%\iffalse
%</samplechap2>
%\fi
%
% %%%%%%%%%%%%%%%%%%%%%%%%%%%%%%%%%%%%%%
% \paragraph{Part Include Files.}
%
% The include files are called |cdocspt3.tex| and |cdocspt4.tex|.
%
%\iffalse
%<*samplepart3|samplepart4>
%\fi

% Optional override for |\version| flag:
%    \begin{macrocode}
%%\providecommand{\version}{final}
%    \end{macrocode}

% Include the main document:
%    \begin{macrocode}
\input{childdoc.def}
\childdocby{cdocsamp}
%    \end{macrocode}

%\iffalse
%</samplepart3|samplepart4>
%\fi
%
%\iffalse
%<*samplepart3>
%\fi
% Some text for part 3:
%    \begin{macrocode}
some text in part three
%    \end{macrocode}

%\iffalse
%</samplepart3>
%\fi
% Some text for part 4:
%\iffalse
%<*samplepart4>
%\fi
%    \begin{macrocode}
more text in part four
%    \end{macrocode}

%\iffalse
%</samplepart4>
%\fi
%
% %%%%%%%%%%%%%%%%%%%%%%%%%%%%%%%%%%%%%%
% \paragraph{Forwarding for a Complete Draft.}
%
% The following forwarding file |cdocsdrf.tex|
% compiles the main document in draft mode:
%\iffalse
%<*sampledraft>
%\fi
%    \begin{macrocode}
\def\version{draft}
\input{childdoc.def}
\childdocforward{cdocsamp}
%    \end{macrocode}

%\iffalse
%</sampledraft>
%\fi
%
% %%%%%%%%%%%%%%%%%%%%%%%%%%%%%%%%%%%%%%
% \paragraph{Forwarding for Final Version of the Chapters.}
%
% The following forwarding files |cdocsfn1.tex| and |cdocsfn2.tex|
% (with identical content)
% compile the final versions of the child documents
% |cdocsch1.tex| and |cdocsch2.tex|, respectively:
%\iffalse
%<*samplefinal>
%\fi
%    \begin{macrocode}
\def\version{final}
\input{childdoc.def}
\childdocforwardprefix[cdocsamp]{cdocsfn}{cdocsch}
%    \end{macrocode}

%\iffalse
%</samplefinal>
%\fi
%
% %%%%%%%%%%%%%%%%%%%%%%%%%%%%%%%%%%%%%%
% \paragraph{Command Line Processing.}
%
% The following three command lines generate the output files
% |cdocscld|, |cdocscl1| and |cdocscl2|
% which should be identical to
% |cdocsdrf|, |cdocsch1| and |cdocsfn2|, respectively:
% \begin{center}
% \begin{tabular}{l}
% |latex -jobname cdocscld \|\\
% |  "\def\version{draft}\input{childdoc.def}\childdocforward{cdocsamp}"|\\
% |latex -jobname cdocscl1 \|\\
% |  "\input{childdoc.def}\childdocforward[cdocsamp]{cdocsch1}"|\\
% |latex -jobname cdocscl2 \|\\
% |  "\def\version{final}\input{childdoc.def}\childdocforward{cdocsch2}"|
% \end{tabular}
% \end{center}
% Note that the trailing backslash on each first line
% merely continues the input to the second line
% (for convenient cut ant paste).
% Furthermore, the command |latex| can be replaced by any
% of its alternative versions such as |pdflatex|.
%
% %%%%%%%%%%%%%%%%%%%%%%%%%%%%%%%%%%%%%%%%%%%%%%%%%%%%%%%%%%%%%%%%%%%%%%%%%%%%%%
% %%%%%%%%%%%%%%%%%%%%%%%%%%%%%%%%%%%%%%%%%%%%%%%%%%%%%%%%%%%%%%%%%%%%%%%%%%%%%%
% \section{Implementation}
%\iffalse
%<*package>
%\fi
%
% This section describes the definitions file |childdoc.def|.

% The definitions cannot be loaded using |\usepackage| or |\RequirePackage|
% which has a mechanism to prevent loading a style file more than once.
% When loading the definitions by means of |\input|
% multiple instances have to be prevented manually:
%\iffalse
%This code needs to be before the `\ProvidesFile' directive
%which is defined at the beginning of this file.
%Therefore it is also placed there and commented out here.
%</package>
%<*discard>
%\fi
%    \begin{macrocode}
\ifdefined\childdocmain\endinput\fi
%    \end{macrocode}
%\iffalse
%</discard>
%<*package>
%\fi
%
% \macro{\ifchilddoc}
% \macro{\ifchilddocmanual}
% The conditional |\ifchilddoc| tells whether a
% child (true) or main (false) document is being compiled.
% The conditional |\ifchilddocmanual| tells whether
% the |\includeonly| mechanism is used (false) or
% the selection of child files must be performed manually (true).
% The definitions initialise to false:
%    \begin{macrocode}
\newif\ifchilddoc
\newif\ifchilddocmanual
%    \end{macrocode}

% \macro{\childdocname}
% \macro{\childdocjob}
% The macro |\childdocname| stores the name of the main document
% to be compiled. The macro |\childdocjob| stores the name of
% the document on which the \LaTeX{} compiler was originally invoked.
% The content of |\jobname| cannot be compared
% to filenames specified in the source due to different catcodes.
% The following code rescans |\jobname|, stores the result
% in |\childdocname| and saves a copy in |\childdocjob|:
%    \begin{macrocode}
\edef\childdocname{\scantokens\expandafter{\jobname\noexpand}}
\let\childdocjob\childdocname
%    \end{macrocode}

% \macro{\childdocdisable}
% The macro |\childdocdisable| prevents the main file
% from being processed more than once.
% At this stage, the main document command |\childdocmain|
% is assumed to be called once again where it should do nothing.
% Any subsequent call to it should prevent
% a secondary processing of the main document
% It overwrites the forwarding commands
% |\childdocof| and |\childdocforward|
% with empty macros to prevent further inclusions of the main document:
%    \begin{macrocode}
\newcommand{\childdocdisable}
{
  \renewcommand{\childdocmain}[1]{\renewcommand{\childdocmain}[1]{\endinput}}
  \renewcommand{\childdocof}[1]{}
  \renewcommand{\childdocby}[2][]{}
  \renewcommand{\childdocforward}[2][]{}
  \renewcommand{\childdocdisable}{}
}
%    \end{macrocode}

% \macro{\childdocmain}
% The macro |\childdocmain| is to be called at the top of the main file
% with nothing or the main filename (without extension) as argument.
% First, it breaks loops.
% If the argument is not empty and does not match |\childdocname|
% (which is set by the first inclusion of |childdoc.def|),
% |\ifchilddoc| is set to true, |\includeonly| is applied to the child file
% and |\jobname| is set to the main file
% (for proper handling of |.aux| files):
%    \begin{macrocode}
\newcommand{\childdocmain}[1]
{
  \childdocdisable\childdocmain{}
  \if?#1?\else
    \begingroup
      \def\childdoctmp{#1}
      \ifx\childdoctmp\childdocname
        \def\childdoctmp{}
      \else
        \def\childdoctmp
        {
          \childdoctrue
          \includeonly{\childdocname}
          \def\childdocjob{#1}
          \def\jobname{#1}
        }
      \fi
      \expandafter
    \endgroup
    \childdoctmp
  \fi
}
%    \end{macrocode}

% \macro{\childdocof}
% The command |\childdocof| redirects
% compilation to the main file |#1|.
%    \begin{macrocode}
\newcommand{\childdocof}[1]
{
  \childdocdisable
  \childdoctrue
  \includeonly{\childdocname}
  \def\jobname{#1}
  \def\childdocjob{#1}
  \input{#1}
}
%    \end{macrocode}

% \macro{\childdocby}
% The command |\childdocby| ....
%    \begin{macrocode}
\newcommand{\childdocby}[2][]
{
  \childdocdisable
  \childdoctrue
  \childdocmanualtrue
  \if?#1?\else
    \def\jobname{#2}
  \fi
  \def\childdocjob{#2}
  \input{#2}
  \endinput
}
%    \end{macrocode}

% \macro{\childdocforward}
% The command |\childdocforward| redirects
% compilation to the main file or
% (if the optional argument is given) a child file.
% Parameters are set as if the main file
% or a child file starting with |\childdocof| was compiled.
% Then compilation is handed over to the main file:
%    \begin{macrocode}
\newcommand{\childdocforward}[2][]
{
  \begingroup
    \if?#1?
      \def\childdoctmp
      {
        \def\childdocname{#2}
        \def\childdocjob{#2}
        \def\jobname{#2}
        \input{#2}
        \endinput
      }
    \else
      \def\childdoctmp
      {
        \childdocdisable
        \def\childdocname{#2}
        \childdoctrue
        \includeonly{#2}
        \def\childdocjob{#1}
        \def\jobname{#1}
        \input{#1}
        \endinput
      }
    \fi
    \expandafter
  \endgroup
  \childdoctmp
}
%    \end{macrocode}

% \macro{\childdocforwardprefix}
% The command |\childdocforwardprefix| redirects
% compilation to the main or a child file by means of a pattern.
% The prefix |#1| in the current filename is replaced by |#2|
% and the suffix of the current filename is kept
% (it is assumed that the filename does not contain the substring `|~~~|'
% which is used as a delimiter).
% Compilation is handed over to the new file by |\childdocforward|:
%    \begin{macrocode}
\newcommand{\childdocforwardprefix}[3][]
{
  \begingroup
    \def\childdocextract #2##1~~~{\def\childdoctmp{\childdocforward[#1]{#3##1}}}
    \expandafter\childdocextract\childdocname~~~
    \expandafter
  \endgroup
  \childdoctmp
}
%    \end{macrocode}

% \macro{\childdoc}
% The deprecated macro |\childdoc| is a legacy version of |\childdocmain|:
%    \begin{macrocode}
\newcommand{\childdoc}{\childdocmain}
%    \end{macrocode}

% \macro{\childdocredirect}
% The deprecated macro |\childdocredirect| is a legacy version
% of |\childdocforward| and |\childdocforwardprefix|:
%    \begin{macrocode}
\newcommand{\childdocredirect}[2][]
{
  \begingroup
    \if?#1?
      \def\childdoctmp{\childdocforward{#2}}
    \else
      \def\childdoctmp{\childdocforwardprefix{#1}{#2}}
    \fi
    \expandafter
  \endgroup
  \childdoctmp
}
%    \end{macrocode}

%\iffalse
%</package>
%\fi
%
\endinput

\childdocforwardprefix[cdocsamp]{cdocsfn}{cdocsch}
%    \end{macrocode}

%\iffalse
%</samplefinal>
%\fi
%
% %%%%%%%%%%%%%%%%%%%%%%%%%%%%%%%%%%%%%%
% \paragraph{Command Line Processing.}
%
% The following three command lines generate the output files
% |cdocscld|, |cdocscl1| and |cdocscl2|
% which should be identical to
% |cdocsdrf|, |cdocsch1| and |cdocsfn2|, respectively:
% \begin{center}
% \begin{tabular}{l}
% |latex -jobname cdocscld \|\\
% |  "\def\version{draft}% \iffalse
%
% childdoc.dtx Copyright (C) 2017-2018 Niklas Beisert
%
% This work may be distributed and/or modified under the
% conditions of the LaTeX Project Public License, either version 1.3
% of this license or (at your option) any later version.
% The latest version of this license is in
%   http://www.latex-project.org/lppl.txt
% and version 1.3 or later is part of all distributions of LaTeX
% version 2005/12/01 or later.
%
% This work has the LPPL maintenance status `maintained'.
%
% The Current Maintainer of this work is Niklas Beisert.
%
% This work consists of the files childdoc.dtx and childdoc.ins
% and the derived files childdoc.def and cdocsamp.tex with
% cdocsch1.tex, cdocsch2.tex, cdocsdrf.tex, cdocsfn1.tex, cdocsfn2.tex.
%
%<package>\ifdefined\childdocmain\endinput\fi
%<package>\ProvidesFile{childdoc.def}[2018/12/30 v2.0 child document driver]
%<samplemain>\ProvidesFile{cdocsamp.tex}[2018/12/30 v2.0 sample for childdoc]
%<*driver>
%\ProvidesFile{childdoc.drv}[2018/12/30 v2.0 childdoc reference manual file]
\PassOptionsToClass{10pt,a4paper}{article}
\documentclass{ltxdoc}

\usepackage[margin=35mm]{geometry}
\usepackage{hyperref}
\usepackage{hyperxmp}
\usepackage[usenames]{color}

\hypersetup{colorlinks=true}
\hypersetup{pdfstartview=FitH}
\hypersetup{pdfpagemode=UseNone}
\hypersetup{pdfsource={}}
\hypersetup{pdflang={en-UK}}
\hypersetup{pdfcopyright={Copyright 2017-2018 Niklas Beisert.
  This work may be distributed and/or modified under the
  conditions of the LaTeX Project Public License, either version 1.3
  of this license or (at your option) any later version.}}
\hypersetup{pdflicenseurl={http://www.latex-project.org/lppl.txt}}
\hypersetup{pdfcontactaddress={ETH Zurich, ITP, HIT K,
  Wolfgang-Pauli-Strasse 27}}
\hypersetup{pdfcontactpostcode={8093}}
\hypersetup{pdfcontactcity={Zurich}}
\hypersetup{pdfcontactcountry={Switzerland}}
\hypersetup{pdfcontactemail={nbeisert@itp.phys.ethz.ch}}
\hypersetup{pdfcontacturl={http://people.phys.ethz.ch/\xmptilde nbeisert/}}

\newcommand{\secref}[1]{\hyperref[#1]{section \ref*{#1}}}

\parskip1ex
\parindent0pt
\let\olditemize\itemize
\def\itemize{\olditemize\parskip0pt}

\begin{document}

\title{The \textsf{childdoc} Package}
\hypersetup{pdftitle={The childdoc Package}}
\author{Niklas Beisert\\[2ex]
  Institut f\"ur Theoretische Physik\\
  Eidgen\"ossische Technische Hochschule Z\"urich\\
  Wolfgang-Pauli-Strasse 27, 8093 Z\"urich, Switzerland\\[1ex]
  \href{mailto:nbeisert@itp.phys.ethz.ch}
  {\texttt{nbeisert@itp.phys.ethz.ch}}}
\hypersetup{pdfauthor={Niklas Beisert}}
\hypersetup{pdfsubject={Manual for the LaTeX2e Package childdoc}}
\date{30 December 2018, \textsf{v2.0}}
\maketitle

\begin{abstract}\noindent
\textsf{childdoc} is a \LaTeXe{} package
that enables the direct compilation
of document sections included by |\include|
to individual files.
\end{abstract}

\begingroup
\parskip0ex
\tableofcontents
\endgroup

%%%%%%%%%%%%%%%%%%%%%%%%%%%%%%%%%%%%%%%%%%%%%%%%%%%%%%%%%%%%%%%%%%%%%%%%%%%%%%%%
%%%%%%%%%%%%%%%%%%%%%%%%%%%%%%%%%%%%%%%%%%%%%%%%%%%%%%%%%%%%%%%%%%%%%%%%%%%%%%%%
\section{Introduction}

\LaTeX{} provides a mechanism to structure a large document (such as a book)
into a main file and several child files (containing the chapters)
using the |\include| command.
This mechanism is beneficial for documents
which span hundreds of pages in order to
make the source file(s) more manageable.
Moreover, compilation can be restricted to
selected child files by means of the |\includeonly| command.
The latter feature can be used to reduce the compilation time while editing
(this was significantly more useful in the earlier days of \LaTeX{})
or to generate a smaller document which is easier to navigate.
Another application of |\includeonly| is to generate
documents consisting of selected parts of the complete document.

However, there are a few drawbacks of the plain |\include| mechanism:
\begin{itemize}
\item
The child files cannot be compiled on their own,
they can only be compiled via the main file.
A naive editing environment
(such as a text editor with an option
to have the current file processed by \LaTeX)
may require one to switch to the main file before compiling;
attempting to compile the child file produces errors.
\item
The main file must be modified (each time)
to adjust the |\includeonly| command
to the present needs. This easily leaves the main file in a messy state.
\item
The generated document will always carry the filename
of the main document. This is inconvenient if
several child files are to be compiled and
to be kept for distribution.
\end{itemize}

The present package provides a simple interface
to make child files individually compilable by \LaTeX{}.
Compiling a child file then has the same effect as compiling
the main file with an |\includeonly| command
to select the appropriate child.
Moreover the generated document will carry the name of the child
rather than the main file.
This resolves all three above issues.

This feature is meant to make the editing of books,
thesis documents and lecture notes somewhat more convenient.
However, the package can also be used efficiently for
composing a series of documents (such as exercise sheets)
which are typically distributed individually.
It then assists the author in generating the individual documents
(potentially in different versions)
as well as a document containing the collected series.
Another application is in developing style files
or other kinds of included material
where compilation of the style file could redirect
to a sample or test file.

%%%%%%%%%%%%%%%%%%%%%%%%%%%%%%%%%%%%%%%%%%%%%%%%%%%%%%%%%%%%%%%%%%%%%%%%%%%%%%%%
%%%%%%%%%%%%%%%%%%%%%%%%%%%%%%%%%%%%%%%%%%%%%%%%%%%%%%%%%%%%%%%%%%%%%%%%%%%%%%%%
\section{Usage}

First of all, the package \textsf{childdoc} is \emph{not} a standard
\LaTeXe{} |.sty| style file! Therefore it needs to be invoked in
a non-standard way.

%%%%%%%%%%%%%%%%%%%%%%%%%%%%%%%%%%%%%%%%%%%%%%%%%%%%%%%%%%%%%%%%%%%%%%%%%%%%%%%%
\subsection{Included Files}
\label{sec:include}

%%%%%%%%%%%%%%%%%%%%%%%%%%%%%%%%%%%%%%%%
\DescribeMacro{\childdocmain}
To use the package, add the commands
\begin{center}
\begin{tabular}{l}
|\input{childdoc.def}|\\
|\childdocmain{}|\\
\end{tabular}
\end{center}
at the very top of the main \LaTeX{} file,
in particular \emph{before} the |\documentclass| statement!
The argument of |\childdocmain| should be left empty
(but it must be present).

%%%%%%%%%%%%%%%%%%%%%%%%%%%%%%%%%%%%%%%%
\DescribeMacro{\childdocof}
Furthermore, add the commands
\begin{center}
\begin{tabular}{l}
|\input{childdoc.def}|\\
|\childdocof{|\textit{main}|}|\\
\end{tabular}
\end{center}
at the top of every child file \textit{child}
which is included by |\include{|\textit{child}|}|
from within the main file
(or at least for those files to be compiled individually).
The argument \textit{main} must be the filename of the main file.

There are a couple of
considerations in setting up the main and child documents:

%%%%%%%%%%%%%%%%%%%%%%%%%%%%%%%%%%%%%%%%
\paragraph{Restrictions.}

Please note the following restrictions:
\begin{itemize}
\item
|\childdocmain| must be called with one argument \textit{main}
to ensure compatibility with earlier version of the package.
It must either be empty (|\childdocmain{}|)
or precisely match the filename of the main file in which it is specified.
See \secref{sec:detection} for further information.
\item
The filename \textit{main} must be specified without the |.tex| extension.
\item
The filename \textit{main} is case sensitive
(even in case-insensitive file systems)
due to internal string comparison.
\item
The argument \textit{main} should be fully expanded, it cannot be a macro.
\item
Subdirectories and special characters should be avoided in filenames.
\item
The command |\childdocmain{|\textit{main}|}| must be followed by a whitespace.
It should not be followed immediately by another command
or by a comment mark `|%|'.
This is because the \TeX{} parser reads the token immediately following
the argument of |\childdocmain| and puts it
at the beginning of every child section;
however, a white\-space is ignored.
\end{itemize}

%%%%%%%%%%%%%%%%%%%%%%%%%%%%%%%%%%%%%%%%
\paragraph{Content of Main File.}

It is advisable to place all content in the child files included by |\include|.
Any output contained in the main file will appear in all child documents
unless suppressed manually;
it cannot be suppressed automatically by the |\includeonly| directive
and thus should normally be avoided.
A method to include some content in the main file
by means of conditional processing is described in \secref{sec:conditional}.

%%%%%%%%%%%%%%%%%%%%%%%%%%%%%%%%%%%%%%%%
\paragraph{Page Numbering.}

When only a part of the document is compiled,
the appropriate numbering of pages
(as well as other status parameters)
is determined from the |.aux| files.
The latter contain information from previous passes.
However this information needs to propagate through
all intermediate child documents.
Therefore the page numbering in child documents may well
be inconsistent until the complete document is compiled at least once.

A useful (if unconventional) way to always ensure a consistent
page numbering is to restart the numbering in each child document
and denote the pages by `\textit{child}|.|\textit{page}'
where \textit{child} represents the chapter/section number of the child file.
This can be achieved by the command
|\numberwithin{page}{|\textit{child}|}|
of the \textsf{amsmath} package
where \textit{child} can be |chapter| or |section|
depending on the chosen structuring.
Alternatively, one can modify the macro |\thepage| appropriately
and reset the counter |page| at the start of each child file.

%%%%%%%%%%%%%%%%%%%%%%%%%%%%%%%%%%%%%%%%%%%%%%%%%%%%%%%%%%%%%%%%%%%%%%%%%%%%%%%%
\subsection{Conditional Processing}
\label{sec:conditional}

The package provides a mechanism to compile different versions
of a document. To customise the versions further some conditional processing
can come in handy to distinguish which version is being compiled.
The package provides two macros to describe the compilation context:

%%%%%%%%%%%%%%%%%%%%%%%%%%%%%%%%%%%%%%%%
\DescribeMacro{\ifchilddoc}
The conditional |\ifchilddoc| distinguishes between the compilation of
child documents and the main document:
%
\begin{center}
|\ifchilddoc |\textit{child-code}| |[|\||else |\textit{main-code}]| \||fi|
\end{center}

%%%%%%%%%%%%%%%%%%%%%%%%%%%%%%%%%%%%%%%%
\DescribeMacro{\childdocname}
\DescribeMacro{\childdocjob}
The macro |\childdocname| contains the filename (without extension)
of the main or child file being processed.
Note that |\childdocjob| will always contain the name of the main file.

%%%%%%%%%%%%%%%%%%%%%%%%%%%%%%%%%%%%%%%%
\paragraph{Title Page.}

Conditional processing can be used to include a title or banner page
in the main document when proper precautions are taken.
Importantly, the code in the main file should ensure that the page counter
(as well as other status parameters which are stored in the |.aux| files)
takes the same value after the conditional processing.
Otherwise the page numbers may take divergent values
depending on which part is compiled.

For example, a title page could be declared by:
%
\begin{center}
\begin{tabular}{l}
|\ifchilddoc\||else|\\
|\addtocounter{page}{-1}|\\
\textit{code for title page}\\
|\newpage|\\
|\||fi|
\end{tabular}
\end{center}
%
A banner page for the child documents can be generated by:
%
\begin{center}
\begin{tabular}{l}
|\ifchilddoc|\\
|\addtocounter{page}{-1}|\\
\textit{code for banner page}\\
|\newpage|\\
|\||fi|
\end{tabular}
\end{center}
%
Here one could write a message such as:
\begin{center}
|This is the part \childdocname{} of \childdocjob{}.|
\end{center}

%%%%%%%%%%%%%%%%%%%%%%%%%%%%%%%%%%%%%%%%%%%%%%%%%%%%%%%%%%%%%%%%%%%%%%%%%%%%%%%%
\subsection{Flags}
\label{sec:flags}

The package makes it easy to generate different versions
of the main or child documents.
To this end compilation flags can be defined
and assigned different default values.
They will be particularly useful in conjunction
with the forwarding mechanism described in \secref{sec:forward}.

For example, it may be useful to have a flag |\version|
which can be set to |draft| or |final|.
The document source will contain some conditional code
depending on the value of |\version|.
Suppose further, the flag should default to |final| for the main file
and to |draft| for child files
which is a natural assignment for editing the document.
This is achieved by placing the following code
in the preamble of the main document
(below the |\childdocmain| directive):
%
\begin{center}
\begin{tabular}{l}
|\ifchilddoc|\\
|\providecommand{\version}{draft}|\\
|\||else|\\
|\providecommand{\version}{final}|\\
|\||fi|
\end{tabular}
\end{center}
%
The definition by |\providecommand| makes sure
that previous definitions are not overwritten.
Further statements |\providecommand{\version}{...}|
can thus be added before the above code to override it.

For the main file, one might add a line
(between |\childdocmain| and the above block)
%
\begin{center}
|%\ifchilddoc\||else\providecommand{\version}{draft}\||fi|
\end{center}
%
which can be uncommented to produce a draft version.
Likewise one can add a line to the very top of a child file
(above the |\childdocof{|\textit{main}|}| directive)
%
\begin{center}
|%\providecommand{\version}{final}|
\end{center}
%
which can be uncommented to produce the final version of this child document.

%%%%%%%%%%%%%%%%%%%%%%%%%%%%%%%%%%%%%%%%%%%%%%%%%%%%%%%%%%%%%%%%%%%%%%%%%%%%%%%%
\subsection{Forwarding}
\label{sec:forward}

Different versions of the main or child documents
using compilation flags as described in \secref{sec:flags}
can be (permanently) stored in different files
for convenient compilation, viewing and distribution.
To this end, the package defines a command
to pass on compilation to a different file:

%%%%%%%%%%%%%%%%%%%%%%%%%%%%%%%%%%%%%%%%
\DescribeMacro{\childdocforward}
The command |\childdocforward| redirects processing to
another source file:
%
\begin{center}
\begin{tabular}{l}
|\input{childdoc.def}|\\
|\childdocforward[|\textit{main}|]{|\textit{dest}|}|\\
\end{tabular}
\end{center}
%
The argument \textit{dest} is the destination file
(without extension).
It should be the main file or one of the child files.
Note that further \textsf{childdoc} directives
such as |\childdocof| and |\childdocforward|
in the indicated file will be processed in this form.
The optional argument \textit{main}
passes on directly to the main file \textit{main}
while pretending to compile the child \textit{dest}.
This form behaves as if \textit{dest}
issues |\childdocof{|\textit{main}|}| right away,
and no further \textsf{childdoc} directives will be processed.

%%%%%%%%%%%%%%%%%%%%%%%%%%%%%%%%%%%%%%%%
\DescribeMacro{\...prefix}
In the alternative form |\childdocforwardprefix|,
%
\begin{center}
\begin{tabular}{l}
|\input{childdoc.def}|\\
|\childdocforwardprefix[|\textit{main}|]{|\textit{prefix}|}{|\textit{dest}|}|
\end{tabular}
\end{center}
%
the destination file is determined by a pattern
depending on the current file:
To make this work, the current file must be called
`{\textit{prefix}\hspace{0.2em}\textit{suffix}}'
with \textit{prefix} matching precisely the argument.
Processing is then passed on to the file
`{\textit{dest}\hspace{0.2em}\textit{suffix}}'.
Surely, the same effect is achieved by
directly specifying the
argument `{\textit{dest}\hspace{0.2em}\textit{suffix}}'
in the first form.
However, that requires to set up a different file
for each child. With the alternative form of the command
all these files can have exactly the same content
which simplifies setting them up and maintaining them.

For example, the following file |draft.tex|
with a compilation flag |\version| as described in \secref{sec:flags}
compiles the main document as a draft:
%
\begin{center}
\begin{tabular}{l}
|\def\version{draft}|\\
|\input{childdoc.def}|\\
|\childdocforward{|\textit{main}|}|
\end{tabular}
\end{center}
%
Likewise, the following files |final|\textit{nn}|.tex|
compile the final version of the child document
|child|\textit{nn}|.tex|:
%
\begin{center}
\begin{tabular}{l}
|\def\version{final}|\\
|\input{childdoc.def}|\\
|\childdocforwardprefix{final}{child}|
\end{tabular}
\end{center}
%

Note that when several versions of a main file and/or of each child file
are to be generated, it may be convenient to set up a |Makefile| or
shell script to automatise the process.

%%%%%%%%%%%%%%%%%%%%%%%%%%%%%%%%%%%%%%%%%%%%%%%%%%%%%%%%%%%%%%%%%%%%%%%%%%%%%%%%
\subsection{Command Line Processing}
\label{sec:commandline}

The effect of redirection files can also be achieved by invoking
the \LaTeX{} compiler with a more elaborate command line.
Most conveniently this should be done as part
of a shell script or a |Makefile|.

When using \textsf{childdoc} in the main file, the following
command lines effectively perform a redirection
(note that depending on the shell being used,
backslashes may have to be doubled: `|\|' $\to$ `|\\|'):
%
\begin{center}
|... -jobname "|\textit{target}|" |\\|"|[\textit{flags}]%
|\input{childdoc.def}\childdocforward[|\textit{main}|]{|\textit{dest}|}"|
\end{center}
%
Here \textit{target} is the name of the output file,
\textit{main} is the name of the main file
and \textit{dest} is the name of the main or child file to be processed
(all filenames without extensions).
The optional argument \textit{main} can be omitted
if \textit{main} matches \textit{dest}.
Optionally, compilation \textit{flags} can be defined via |\def| commands.
This command line makes the \TeX{} engine believe
it is compiling the file \textit{target}
whose content is specified as the latter parameter.
The provided code then forwards the processing to
\textit{main} or \textit{dest} as described in \secref{sec:forward}.

%%%%%%%%%%%%%%%%%%%%%%%%%%%%%%%%%%%%%%%%%%%%%%%%%%%%%%%%%%%%%%%%%%%%%%%%%%%%%%%%
\subsection{Include by Input}
\label{sec:input}

Including child documents by |\include| has some restrictions by design.
Most notably, the content of a child document always occupies
its own set of pages; pages cannot be shared between child documents.
Usually, this behaviour makes perfect sense
because each child document contain an essential part of the document.
However, in some situations it may be desirable to compose
a document from a collection of parts
without having mandatory page breaks between then.
For this case, the package
provides a mechanism to include parts
by |\input| which can also be processed individually.
However, by construction this mechanism
requires manual handling of the content to be output.

%%%%%%%%%%%%%%%%%%%%%%%%%%%%%%%%%%%%%%%%
\DescribeMacro{\ifchilddocmanual}
The main file should be prepared as usual, see \secref{sec:include}.
However, the document body must make a distinction
between processing of an individual part and of the main document, e.g.:
%
\begin{center}
\begin{tabular}{l}
|\ifchilddocmanual|\\
|\input{\childdocname}|\\
|\||else|\\
\textit{document body with }|\input{|\textit{part}|}|\\
|\||fi|
\end{tabular}
\end{center}
%
The conditional |\ifchilddocmanual| is true whenever
a part to be included by |\input| is being compiled,
and the name of the part is stored in |\childdocname|.

%%%%%%%%%%%%%%%%%%%%%%%%%%%%%%%%%%%%%%%%
\DescribeMacro{\childdocby}
Each part to be included by |\input| should start with:
%
\begin{center}
\begin{tabular}{l}
|\input{childdoc.def}|\\
|\childdocby{|\textit{main}|}|\\
\end{tabular}
\end{center}
%
The directive |\childdocby| is similar to |\childdocof|
described in \secref{sec:include},
but the subsequent selection of content must be done manually.
To that end, both |\ifchilddoc| and |\ifchilddocmanual|
will be true upon processing of a part,
and the name of the part is stored in |\childdocname|.
Note that |\jobname| will be set to the filename of the current part
so that each part receives an individual |.aux| file
that does not interfere with the |.aux| file(s) of the main document.
This behaviour can be altered by the alternative form
|\childdocby[*]{|\textit{main}|}| (with a non-empty optional argument)
which uses the |.aux| file of the main document
by setting |\jobname| to \textit{main}.

%%%%%%%%%%%%%%%%%%%%%%%%%%%%%%%%%%%%%%%%%%%%%%%%%%%%%%%%%%%%%%%%%%%%%%%%%%%%%%%%
\subsection{Driver Development}
\label{sec:driver}

The \textsf{childdoc} mechanism can also be use for the development
of definition files such as \LaTeX{} styles or classes.
This case differs from the above setup with multiple parts
included by |\include| in that no |\includeonly| should be invoked.
This can be achieved by starting the include file
(before |\ProvidesPackage|) with:
%
\begin{center}
\begin{tabular}{l}
|\input{childdoc.def}|\\
|\childdocforward{|\textit{main}|}|\\
\end{tabular}
\end{center}
%
or alternatively with:
%
\begin{center}
\begin{tabular}{l}
|\input{childdoc.def}|\\
|\childdocby{|\textit{main}|}|\\
\end{tabular}
\end{center}
%
Both forms have slightly different effects as described above.
The main file is prepared as usual, see \secref{sec:include}.

%%%%%%%%%%%%%%%%%%%%%%%%%%%%%%%%%%%%%%%%%%%%%%%%%%%%%%%%%%%%%%%%%%%%%%%%%%%%%%%%
\subsection{Legacy Detection}
\label{sec:detection}

The directive |\childdocmain| in the main file can detect
whether the complete document or merely a child is to be compiled
even without using the directive |\childdocof|.
This method is deprecated because it is less robust
and there is no compelling reason to use it;
it is merely provided for backward compatibility
and it may be removed in future versions.

If the detection mechanism is to be used,
it is mandatory to correctly specify
the filename of the main file as the argument of |\childdocmain|:
%
\begin{center}
\begin{tabular}{l}
|\input{childdoc.def}|\\
|\childdocmain{|\textit{main}|}|\\
\end{tabular}
\end{center}
%
If |\jobname| does not match the argument \textit{main} of |\childdocmain|,
it is assumed that |\jobname| points to the child file to be compiled.
When using |\childdocmain| with the main file specified as argument,
it suffices to start a child file
with just |\input{|\textit{main}|}|
without loading of the package and using |\childdocof|.
If instead all processing is done
with the appropriate \textsf{childdoc} directives,
the argument of \textit{main} of |\childdocmain| can be empty.

An alternative version of the command line processing described
in \secref{sec:commandline} using the detection mechanism reads:
%
\begin{center}
|... -jobname "|\textit{target}|" "|[\textit{flags}]%
[|\def\jobname{|\textit{dest}|}|]|\input{|\textit{main}|}"|
\end{center}

%%%%%%%%%%%%%%%%%%%%%%%%%%%%%%%%%%%%%%%%%%%%%%%%%%%%%%%%%%%%%%%%%%%%%%%%%%%%%%%%
\subsection{Manual Code}
\label{sec:manual}

In case one cannot be certain whether the definitions file |childdoc.def|
is installed on the target \TeX{} distribution
and one prefers not to ship it,
it is conceivable to paste a few relevant commands into the sources.

To that end, drop all statements |\input{childdoc.def}|
and perform the replacements as outlined below.
Instead of |\childdocmain{|\textit{main}|}| add the following code
to the top of the main file:
%
\begin{center}
\begin{tabular}{l}
|\||ifdefined\childdocname\endinput\||fi\newif\ifchilddoc|\\
|\edef\childdocname{\scantokens\expandafter{\jobname\noexpand}}|\\
|\def\childdocmain{|\textit{main}|}\||ifx\childdocmain\childdocname\||else|\\
|\childdoctrue\includeonly{\childdocname}\let\jobname\childdocmain\||fi|\\
\end{tabular}
\end{center}
%
Instead of |\childdocof{|\textit{main}|}| just include the main file
at the top of each child file:
%
\begin{center}
|\input{|\textit{main}|}|
\end{center}
%
A simple redirection |\childdocforward{|\textit{dest}|}| is achieved by:
%
\begin{center}
|\def\jobname{|\textit{dest}|}\input{\jobname}|
\end{center}
%
The redirection with prefix
|\childdocforwardprefix[|\textit{prefix}|]{|\textit{dest}|}|
is accomplished by:
%
\begin{center}
\begin{tabular}{l}
|{\edef\jobname{\scantokens\expandafter{\jobname\noexpand}}|\\
|\def\redirectjob |\textit{prefix}|#1~~~{\gdef\jobname{|\textit{dest}|#1}}|\\
|\expandafter\redirectjob\jobname~~~}\input{\jobname}|
\end{tabular}
\end{center}

In an alternative approach,
child documents can be compiled by a specific command line
without additional code or specific definitions:
%
\begin{center}
|... -jobname "|\textit{target}|" "|[\textit{flags}]%
|\includeonly{|\textit{dest}|}\input{|\textit{main}|}"|
\end{center}
%

%%%%%%%%%%%%%%%%%%%%%%%%%%%%%%%%%%%%%%%%%%%%%%%%%%%%%%%%%%%%%%%%%%%%%%%%%%%%%%%%
%%%%%%%%%%%%%%%%%%%%%%%%%%%%%%%%%%%%%%%%%%%%%%%%%%%%%%%%%%%%%%%%%%%%%%%%%%%%%%%%
\section{Information}

%%%%%%%%%%%%%%%%%%%%%%%%%%%%%%%%%%%%%%%%%%%%%%%%%%%%%%%%%%%%%%%%%%%%%%%%%%%%%%%%
\subsection{Copyright}

Copyright \copyright{} 2017--2018 Niklas Beisert

This work may be distributed and/or modified under the
conditions of the \LaTeX{} Project Public License, either version 1.3
of this license or (at your option) any later version.
The latest version of this license is in
  \url{http://www.latex-project.org/lppl.txt}
and version 1.3 or later is part of all distributions of \LaTeX{}
version 2005/12/01 or later.

This work has the LPPL maintenance status `maintained'.

The Current Maintainer of this work is Niklas Beisert.

This work consists of the files |README.txt|, |childdoc.ins| and |childdoc.dtx|
as well as the derived files |childdoc.def|, |cdocsamp.tex|
with |cdocsch1.tex|, |cdocsch2.tex|, |cdocspt3.tex|, |cdocspt4.tex|,
|cdocsdrf.tex|, |cdocsfn1.tex|, |cdocsfn2.tex|
as well as |childdoc.pdf|.

%%%%%%%%%%%%%%%%%%%%%%%%%%%%%%%%%%%%%%%%%%%%%%%%%%%%%%%%%%%%%%%%%%%%%%%%%%%%%%%%
\subsection{Files and Installation}

The package consists of the files:
%
\begin{center}
\begin{tabular}{ll}
    |README.txt|   & readme file \\
    |childdoc.ins| & installation file \\
    |childdoc.dtx| & source file \\
    |childdoc.def| & definition file \\
    |cdocsamp.tex| & sample main file \\
    |cdocsch1.tex| & sample include file \\
    |cdocsch2.tex| & sample include file \\
    |cdocspt3.tex| & sample part file \\
    |cdocspt4.tex| & sample part file \\
    |cdocsdrf.tex| & sample redirection file \\
    |cdocsfn1.tex| & sample redirection file \\
    |cdocsfn2.tex| & sample redirection file \\
    |childdoc.pdf| & manual
\end{tabular}
\end{center}
%
The distribution consists of the files
|README.txt|, |childdoc.ins| and |childdoc.dtx|.
%
\begin{itemize}
\item
Run (pdf)\LaTeX{} on |childdoc.dtx|
to compile the manual |childdoc.pdf| (this file).
\item
Run \LaTeX{} on |childdoc.ins| to create the definitions file |childdoc.def|
and the sample |cdocsamp.tex| with include files
|cdocsch1.tex|, |cdocsch2.tex|, |cdocspt3.tex|, |cdocspt4.tex|,
|cdocsdrf.tex|, |cdocsfn1.tex|, |cdocsfn2.tex|.
Then copy the file |childdoc.def| to an appropriate directory of your \LaTeX{}
distribution, e.g.\ \textit{texmf-root}|/tex/latex/childdoc|.
\end{itemize}

%%%%%%%%%%%%%%%%%%%%%%%%%%%%%%%%%%%%%%%%%%%%%%%%%%%%%%%%%%%%%%%%%%%%%%%%%%%%%%%%
\subsection{Related CTAN Packages}

There are several other packages which offer a similar functionality:
%
\begin{itemize}
\item
The packages
\href{http://ctan.org/pkg/docmute}{\textsf{docmute}},
\href{http://ctan.org/pkg/includex}{\textsf{includex}} and
\href{http://ctan.org/pkg/standalone}{\textsf{standalone}}
provide commands to include only the document body of
a child file thus allowing both files to be compiled individually.
\item
The packages \href{http://ctan.org/pkg/subdocs}{\textsf{subdocs}}
and \href{http://ctan.org/pkg/subfiles}{\textsf{subfiles}}
provide structures in which the main and child documents can be
encapsulated and allowing them to be compiled individually.
The inclusion mechanism is different from the conventional |\include|.
\item
The package \href{http://ctan.org/pkg/combine}{\textsf{combine}}
is an elaborate solution to combine several documents into one.
\end{itemize}
%
See also the CTAN topic \href{http://ctan.org/topic/subdocs}{\textsf{subdocs}}
for further related packages.
The present package differs from the above solutions in that
a document structure constructed with the conventional |\include| mechanism
just needs two extra commands at the top of every file
such that all constituent files can be compiled individually.

%%%%%%%%%%%%%%%%%%%%%%%%%%%%%%%%%%%%%%%%%%%%%%%%%%%%%%%%%%%%%%%%%%%%%%%%%%%%%%%%
%\subsection{Feature Suggestions}
%
%The following is a list of features which may be useful for future
%versions of this package:
%%
%\begin{itemize}
%\item
%\ldots
%\end{itemize}

%%%%%%%%%%%%%%%%%%%%%%%%%%%%%%%%%%%%%%%%%%%%%%%%%%%%%%%%%%%%%%%%%%%%%%%%%%%%%%%%
\subsection{Revision History}

%%%%%%%%%%%%%%%%%%%%%%%%%%%%%%%%%%%%%%%%
\paragraph{v2.0:} 2018/12/30

\begin{itemize}
\item
immediate forward processing
\item
added |\childdocby| mechanism
\item
manual restructured
\end{itemize}

%%%%%%%%%%%%%%%%%%%%%%%%%%%%%%%%%%%%%%%%
\paragraph{v1.6:} 2018/01/17

\begin{itemize}
\item
application for development of include files
\item
corrections to manual
\end{itemize}

%%%%%%%%%%%%%%%%%%%%%%%%%%%%%%%%%%%%%%%%
\paragraph{v1.5:} 2017/05/21

\begin{itemize}
\item
more complete structuring introduced
\item
|\childdocof| introduced
\item
|\childdoc| renamed to |\childdocmain|
\item
|\childredirect| renamed to |\childdocforward| and |\childdocforwardprefix|
and functionality expanded
\end{itemize}

%%%%%%%%%%%%%%%%%%%%%%%%%%%%%%%%%%%%%%%%
\paragraph{v1.0:} 2017/04/27

\begin{itemize}
\item
manual and install package
\item
first version published on CTAN
\end{itemize}

%%%%%%%%%%%%%%%%%%%%%%%%%%%%%%%%%%%%%%%%
\paragraph{v0.6:} 2017/04/26

\begin{itemize}
\item
redirection mechanism added
\end{itemize}

%%%%%%%%%%%%%%%%%%%%%%%%%%%%%%%%%%%%%%%%
\paragraph{v0.5:} 2017/04/26

\begin{itemize}
\item
functionality in definition file
\end{itemize}


%%%%%%%%%%%%%%%%%%%%%%%%%%%%%%%%%%%%%%%%%%%%%%%%%%%%%%%%%%%%%%%%%%%%%%%%%%%%%%%%
%%%%%%%%%%%%%%%%%%%%%%%%%%%%%%%%%%%%%%%%%%%%%%%%%%%%%%%%%%%%%%%%%%%%%%%%%%%%%%%%
%%%%%%%%%%%%%%%%%%%%%%%%%%%%%%%%%%%%%%%%%%%%%%%%%%%%%%%%%%%%%%%%%%%%%%%%%%%%%%%%
\appendix

\settowidth\MacroIndent{\rmfamily\scriptsize 000\ }

 \DocInput{childdoc.dtx}

\end{document}
%</driver>
% \fi
%
% %%%%%%%%%%%%%%%%%%%%%%%%%%%%%%%%%%%%%%%%%%%%%%%%%%%%%%%%%%%%%%%%%%%%%%%%%%%%%%
% %%%%%%%%%%%%%%%%%%%%%%%%%%%%%%%%%%%%%%%%%%%%%%%%%%%%%%%%%%%%%%%%%%%%%%%%%%%%%%
% \section{Sample}
%\iffalse
%<*samplemain>
%\fi
%
% The following presents a sample document
% with two chapters, two parts, a title page,
% a compile flag as well as three forwarding files to set the flag.
% It consists of eight |.tex| files:
% \begin{center}
% \begin{tabular}{ll}
% |cdocsamp.tex|&main file\\
% |cdocsch1.tex|&include file for chapter 1\\
% |cdocsch2.tex|&include file for chapter 2\\
% |cdocspt3.tex|&include file for part 3\\
% |cdocspt4.tex|&include file for part 4\\
% |cdocsdrf.tex|&forwarding file for main file in draft mode\\
% |cdocsfi1.tex|&forwarding file for final version of chapter 1\\
% |cdocsfi2.tex|&forwarding file for final version of chapter 2\\
% \end{tabular}
% \end{center}
% Each of the eight files can be compiled directly by the \LaTeX{} compiler.
%
% %%%%%%%%%%%%%%%%%%%%%%%%%%%%%%%%%%%%%%
% \paragraph{Main File.}
%
% The main file is called |cdocsamp.tex|.
%
% Load the \textsf{childdoc} definitions and
% declare the filename for the main document:
%    \begin{macrocode}
\input{childdoc.def}
\childdocmain{}
%    \end{macrocode}

% Optional override for |\version| flag:
%    \begin{macrocode}
%%\ifchilddoc\else\providecommand{\version}{draft}\fi
%    \end{macrocode}

% Define the default values for the |\version| flag
% (|final| for the main file and |draft| for childs):
%    \begin{macrocode}
\ifchilddoc
\providecommand{\version}{draft}
\else
\providecommand{\version}{final}
\fi
%    \end{macrocode}

% Load the standard document class:
%    \begin{macrocode}
\documentclass[12pt]{article}
%    \end{macrocode}

% Start the document body:
%    \begin{macrocode}
\begin{document}
%    \end{macrocode}

% Declare a title page.
% Print title, part of document being processed and version flag:
%    \begin{macrocode}
\addtocounter{page}{-1}
\begin{center}
{\LARGE\bfseries{}childdoc example\par}
\vspace{1cm}
\ifchilddoc
\ifchilddocmanual part\else chapter\fi:
`\childdocname' of `\childdocjob'\par
\else
main document: `\childdocjob'\par
\fi
version: \version\par
\end{center}
\newpage
%    \end{macrocode}

% Manually include selected file,
% otherwise process as usual:
%    \begin{macrocode}
\ifchilddocmanual
\section*{part `\childdocname'}
\input{\childdocname}
\else
%    \end{macrocode}

% Include the two chapters:
%    \begin{macrocode}
\include{cdocsch1}
\include{cdocsch2}
%    \end{macrocode}

% Include the two parts unless only chapters should be displayed:
%    \begin{macrocode}
\ifchilddoc\else
\section{part three}
\input{cdocspt3}
\section{part four}
\input{cdocspt4}
\fi
%    \end{macrocode}

% Process as usual until here:
%    \begin{macrocode}
\fi
%    \end{macrocode}

% End of document body:
%    \begin{macrocode}
\end{document}
%    \end{macrocode}
%\iffalse
%</samplemain>
%\fi
%
% %%%%%%%%%%%%%%%%%%%%%%%%%%%%%%%%%%%%%%
% \paragraph{Chapter Include Files.}
%
% The include files are called |cdocsch1.tex| and |cdocsch2.tex|.
%
%\iffalse
%<*samplechap1|samplechap2>
%\fi

% Optional override for |\version| flag:
%    \begin{macrocode}
%%\providecommand{\version}{final}
%    \end{macrocode}

% Include the main document:
%    \begin{macrocode}
\input{childdoc.def}
\childdocof{cdocsamp}
%    \end{macrocode}

%\iffalse
%</samplechap1|samplechap2>
%\fi
%
%\iffalse
%<*samplechap1>
%\fi
% Some text for chapter 1:
%    \begin{macrocode}
\section{one}
some text in chapter one
%    \end{macrocode}

%\iffalse
%</samplechap1>
%\fi
% Some text for chapter 2:
%\iffalse
%<*samplechap2>
%\fi
%    \begin{macrocode}
\section{two}
more text in chapter two
%    \end{macrocode}

%\iffalse
%</samplechap2>
%\fi
%
% %%%%%%%%%%%%%%%%%%%%%%%%%%%%%%%%%%%%%%
% \paragraph{Part Include Files.}
%
% The include files are called |cdocspt3.tex| and |cdocspt4.tex|.
%
%\iffalse
%<*samplepart3|samplepart4>
%\fi

% Optional override for |\version| flag:
%    \begin{macrocode}
%%\providecommand{\version}{final}
%    \end{macrocode}

% Include the main document:
%    \begin{macrocode}
\input{childdoc.def}
\childdocby{cdocsamp}
%    \end{macrocode}

%\iffalse
%</samplepart3|samplepart4>
%\fi
%
%\iffalse
%<*samplepart3>
%\fi
% Some text for part 3:
%    \begin{macrocode}
some text in part three
%    \end{macrocode}

%\iffalse
%</samplepart3>
%\fi
% Some text for part 4:
%\iffalse
%<*samplepart4>
%\fi
%    \begin{macrocode}
more text in part four
%    \end{macrocode}

%\iffalse
%</samplepart4>
%\fi
%
% %%%%%%%%%%%%%%%%%%%%%%%%%%%%%%%%%%%%%%
% \paragraph{Forwarding for a Complete Draft.}
%
% The following forwarding file |cdocsdrf.tex|
% compiles the main document in draft mode:
%\iffalse
%<*sampledraft>
%\fi
%    \begin{macrocode}
\def\version{draft}
\input{childdoc.def}
\childdocforward{cdocsamp}
%    \end{macrocode}

%\iffalse
%</sampledraft>
%\fi
%
% %%%%%%%%%%%%%%%%%%%%%%%%%%%%%%%%%%%%%%
% \paragraph{Forwarding for Final Version of the Chapters.}
%
% The following forwarding files |cdocsfn1.tex| and |cdocsfn2.tex|
% (with identical content)
% compile the final versions of the child documents
% |cdocsch1.tex| and |cdocsch2.tex|, respectively:
%\iffalse
%<*samplefinal>
%\fi
%    \begin{macrocode}
\def\version{final}
\input{childdoc.def}
\childdocforwardprefix[cdocsamp]{cdocsfn}{cdocsch}
%    \end{macrocode}

%\iffalse
%</samplefinal>
%\fi
%
% %%%%%%%%%%%%%%%%%%%%%%%%%%%%%%%%%%%%%%
% \paragraph{Command Line Processing.}
%
% The following three command lines generate the output files
% |cdocscld|, |cdocscl1| and |cdocscl2|
% which should be identical to
% |cdocsdrf|, |cdocsch1| and |cdocsfn2|, respectively:
% \begin{center}
% \begin{tabular}{l}
% |latex -jobname cdocscld \|\\
% |  "\def\version{draft}\input{childdoc.def}\childdocforward{cdocsamp}"|\\
% |latex -jobname cdocscl1 \|\\
% |  "\input{childdoc.def}\childdocforward[cdocsamp]{cdocsch1}"|\\
% |latex -jobname cdocscl2 \|\\
% |  "\def\version{final}\input{childdoc.def}\childdocforward{cdocsch2}"|
% \end{tabular}
% \end{center}
% Note that the trailing backslash on each first line
% merely continues the input to the second line
% (for convenient cut ant paste).
% Furthermore, the command |latex| can be replaced by any
% of its alternative versions such as |pdflatex|.
%
% %%%%%%%%%%%%%%%%%%%%%%%%%%%%%%%%%%%%%%%%%%%%%%%%%%%%%%%%%%%%%%%%%%%%%%%%%%%%%%
% %%%%%%%%%%%%%%%%%%%%%%%%%%%%%%%%%%%%%%%%%%%%%%%%%%%%%%%%%%%%%%%%%%%%%%%%%%%%%%
% \section{Implementation}
%\iffalse
%<*package>
%\fi
%
% This section describes the definitions file |childdoc.def|.

% The definitions cannot be loaded using |\usepackage| or |\RequirePackage|
% which has a mechanism to prevent loading a style file more than once.
% When loading the definitions by means of |\input|
% multiple instances have to be prevented manually:
%\iffalse
%This code needs to be before the `\ProvidesFile' directive
%which is defined at the beginning of this file.
%Therefore it is also placed there and commented out here.
%</package>
%<*discard>
%\fi
%    \begin{macrocode}
\ifdefined\childdocmain\endinput\fi
%    \end{macrocode}
%\iffalse
%</discard>
%<*package>
%\fi
%
% \macro{\ifchilddoc}
% \macro{\ifchilddocmanual}
% The conditional |\ifchilddoc| tells whether a
% child (true) or main (false) document is being compiled.
% The conditional |\ifchilddocmanual| tells whether
% the |\includeonly| mechanism is used (false) or
% the selection of child files must be performed manually (true).
% The definitions initialise to false:
%    \begin{macrocode}
\newif\ifchilddoc
\newif\ifchilddocmanual
%    \end{macrocode}

% \macro{\childdocname}
% \macro{\childdocjob}
% The macro |\childdocname| stores the name of the main document
% to be compiled. The macro |\childdocjob| stores the name of
% the document on which the \LaTeX{} compiler was originally invoked.
% The content of |\jobname| cannot be compared
% to filenames specified in the source due to different catcodes.
% The following code rescans |\jobname|, stores the result
% in |\childdocname| and saves a copy in |\childdocjob|:
%    \begin{macrocode}
\edef\childdocname{\scantokens\expandafter{\jobname\noexpand}}
\let\childdocjob\childdocname
%    \end{macrocode}

% \macro{\childdocdisable}
% The macro |\childdocdisable| prevents the main file
% from being processed more than once.
% At this stage, the main document command |\childdocmain|
% is assumed to be called once again where it should do nothing.
% Any subsequent call to it should prevent
% a secondary processing of the main document
% It overwrites the forwarding commands
% |\childdocof| and |\childdocforward|
% with empty macros to prevent further inclusions of the main document:
%    \begin{macrocode}
\newcommand{\childdocdisable}
{
  \renewcommand{\childdocmain}[1]{\renewcommand{\childdocmain}[1]{\endinput}}
  \renewcommand{\childdocof}[1]{}
  \renewcommand{\childdocby}[2][]{}
  \renewcommand{\childdocforward}[2][]{}
  \renewcommand{\childdocdisable}{}
}
%    \end{macrocode}

% \macro{\childdocmain}
% The macro |\childdocmain| is to be called at the top of the main file
% with nothing or the main filename (without extension) as argument.
% First, it breaks loops.
% If the argument is not empty and does not match |\childdocname|
% (which is set by the first inclusion of |childdoc.def|),
% |\ifchilddoc| is set to true, |\includeonly| is applied to the child file
% and |\jobname| is set to the main file
% (for proper handling of |.aux| files):
%    \begin{macrocode}
\newcommand{\childdocmain}[1]
{
  \childdocdisable\childdocmain{}
  \if?#1?\else
    \begingroup
      \def\childdoctmp{#1}
      \ifx\childdoctmp\childdocname
        \def\childdoctmp{}
      \else
        \def\childdoctmp
        {
          \childdoctrue
          \includeonly{\childdocname}
          \def\childdocjob{#1}
          \def\jobname{#1}
        }
      \fi
      \expandafter
    \endgroup
    \childdoctmp
  \fi
}
%    \end{macrocode}

% \macro{\childdocof}
% The command |\childdocof| redirects
% compilation to the main file |#1|.
%    \begin{macrocode}
\newcommand{\childdocof}[1]
{
  \childdocdisable
  \childdoctrue
  \includeonly{\childdocname}
  \def\jobname{#1}
  \def\childdocjob{#1}
  \input{#1}
}
%    \end{macrocode}

% \macro{\childdocby}
% The command |\childdocby| ....
%    \begin{macrocode}
\newcommand{\childdocby}[2][]
{
  \childdocdisable
  \childdoctrue
  \childdocmanualtrue
  \if?#1?\else
    \def\jobname{#2}
  \fi
  \def\childdocjob{#2}
  \input{#2}
  \endinput
}
%    \end{macrocode}

% \macro{\childdocforward}
% The command |\childdocforward| redirects
% compilation to the main file or
% (if the optional argument is given) a child file.
% Parameters are set as if the main file
% or a child file starting with |\childdocof| was compiled.
% Then compilation is handed over to the main file:
%    \begin{macrocode}
\newcommand{\childdocforward}[2][]
{
  \begingroup
    \if?#1?
      \def\childdoctmp
      {
        \def\childdocname{#2}
        \def\childdocjob{#2}
        \def\jobname{#2}
        \input{#2}
        \endinput
      }
    \else
      \def\childdoctmp
      {
        \childdocdisable
        \def\childdocname{#2}
        \childdoctrue
        \includeonly{#2}
        \def\childdocjob{#1}
        \def\jobname{#1}
        \input{#1}
        \endinput
      }
    \fi
    \expandafter
  \endgroup
  \childdoctmp
}
%    \end{macrocode}

% \macro{\childdocforwardprefix}
% The command |\childdocforwardprefix| redirects
% compilation to the main or a child file by means of a pattern.
% The prefix |#1| in the current filename is replaced by |#2|
% and the suffix of the current filename is kept
% (it is assumed that the filename does not contain the substring `|~~~|'
% which is used as a delimiter).
% Compilation is handed over to the new file by |\childdocforward|:
%    \begin{macrocode}
\newcommand{\childdocforwardprefix}[3][]
{
  \begingroup
    \def\childdocextract #2##1~~~{\def\childdoctmp{\childdocforward[#1]{#3##1}}}
    \expandafter\childdocextract\childdocname~~~
    \expandafter
  \endgroup
  \childdoctmp
}
%    \end{macrocode}

% \macro{\childdoc}
% The deprecated macro |\childdoc| is a legacy version of |\childdocmain|:
%    \begin{macrocode}
\newcommand{\childdoc}{\childdocmain}
%    \end{macrocode}

% \macro{\childdocredirect}
% The deprecated macro |\childdocredirect| is a legacy version
% of |\childdocforward| and |\childdocforwardprefix|:
%    \begin{macrocode}
\newcommand{\childdocredirect}[2][]
{
  \begingroup
    \if?#1?
      \def\childdoctmp{\childdocforward{#2}}
    \else
      \def\childdoctmp{\childdocforwardprefix{#1}{#2}}
    \fi
    \expandafter
  \endgroup
  \childdoctmp
}
%    \end{macrocode}

%\iffalse
%</package>
%\fi
%
\endinput
\childdocforward{cdocsamp}"|\\
% |latex -jobname cdocscl1 \|\\
% |  "% \iffalse
%
% childdoc.dtx Copyright (C) 2017-2018 Niklas Beisert
%
% This work may be distributed and/or modified under the
% conditions of the LaTeX Project Public License, either version 1.3
% of this license or (at your option) any later version.
% The latest version of this license is in
%   http://www.latex-project.org/lppl.txt
% and version 1.3 or later is part of all distributions of LaTeX
% version 2005/12/01 or later.
%
% This work has the LPPL maintenance status `maintained'.
%
% The Current Maintainer of this work is Niklas Beisert.
%
% This work consists of the files childdoc.dtx and childdoc.ins
% and the derived files childdoc.def and cdocsamp.tex with
% cdocsch1.tex, cdocsch2.tex, cdocsdrf.tex, cdocsfn1.tex, cdocsfn2.tex.
%
%<package>\ifdefined\childdocmain\endinput\fi
%<package>\ProvidesFile{childdoc.def}[2018/12/30 v2.0 child document driver]
%<samplemain>\ProvidesFile{cdocsamp.tex}[2018/12/30 v2.0 sample for childdoc]
%<*driver>
%\ProvidesFile{childdoc.drv}[2018/12/30 v2.0 childdoc reference manual file]
\PassOptionsToClass{10pt,a4paper}{article}
\documentclass{ltxdoc}

\usepackage[margin=35mm]{geometry}
\usepackage{hyperref}
\usepackage{hyperxmp}
\usepackage[usenames]{color}

\hypersetup{colorlinks=true}
\hypersetup{pdfstartview=FitH}
\hypersetup{pdfpagemode=UseNone}
\hypersetup{pdfsource={}}
\hypersetup{pdflang={en-UK}}
\hypersetup{pdfcopyright={Copyright 2017-2018 Niklas Beisert.
  This work may be distributed and/or modified under the
  conditions of the LaTeX Project Public License, either version 1.3
  of this license or (at your option) any later version.}}
\hypersetup{pdflicenseurl={http://www.latex-project.org/lppl.txt}}
\hypersetup{pdfcontactaddress={ETH Zurich, ITP, HIT K,
  Wolfgang-Pauli-Strasse 27}}
\hypersetup{pdfcontactpostcode={8093}}
\hypersetup{pdfcontactcity={Zurich}}
\hypersetup{pdfcontactcountry={Switzerland}}
\hypersetup{pdfcontactemail={nbeisert@itp.phys.ethz.ch}}
\hypersetup{pdfcontacturl={http://people.phys.ethz.ch/\xmptilde nbeisert/}}

\newcommand{\secref}[1]{\hyperref[#1]{section \ref*{#1}}}

\parskip1ex
\parindent0pt
\let\olditemize\itemize
\def\itemize{\olditemize\parskip0pt}

\begin{document}

\title{The \textsf{childdoc} Package}
\hypersetup{pdftitle={The childdoc Package}}
\author{Niklas Beisert\\[2ex]
  Institut f\"ur Theoretische Physik\\
  Eidgen\"ossische Technische Hochschule Z\"urich\\
  Wolfgang-Pauli-Strasse 27, 8093 Z\"urich, Switzerland\\[1ex]
  \href{mailto:nbeisert@itp.phys.ethz.ch}
  {\texttt{nbeisert@itp.phys.ethz.ch}}}
\hypersetup{pdfauthor={Niklas Beisert}}
\hypersetup{pdfsubject={Manual for the LaTeX2e Package childdoc}}
\date{30 December 2018, \textsf{v2.0}}
\maketitle

\begin{abstract}\noindent
\textsf{childdoc} is a \LaTeXe{} package
that enables the direct compilation
of document sections included by |\include|
to individual files.
\end{abstract}

\begingroup
\parskip0ex
\tableofcontents
\endgroup

%%%%%%%%%%%%%%%%%%%%%%%%%%%%%%%%%%%%%%%%%%%%%%%%%%%%%%%%%%%%%%%%%%%%%%%%%%%%%%%%
%%%%%%%%%%%%%%%%%%%%%%%%%%%%%%%%%%%%%%%%%%%%%%%%%%%%%%%%%%%%%%%%%%%%%%%%%%%%%%%%
\section{Introduction}

\LaTeX{} provides a mechanism to structure a large document (such as a book)
into a main file and several child files (containing the chapters)
using the |\include| command.
This mechanism is beneficial for documents
which span hundreds of pages in order to
make the source file(s) more manageable.
Moreover, compilation can be restricted to
selected child files by means of the |\includeonly| command.
The latter feature can be used to reduce the compilation time while editing
(this was significantly more useful in the earlier days of \LaTeX{})
or to generate a smaller document which is easier to navigate.
Another application of |\includeonly| is to generate
documents consisting of selected parts of the complete document.

However, there are a few drawbacks of the plain |\include| mechanism:
\begin{itemize}
\item
The child files cannot be compiled on their own,
they can only be compiled via the main file.
A naive editing environment
(such as a text editor with an option
to have the current file processed by \LaTeX)
may require one to switch to the main file before compiling;
attempting to compile the child file produces errors.
\item
The main file must be modified (each time)
to adjust the |\includeonly| command
to the present needs. This easily leaves the main file in a messy state.
\item
The generated document will always carry the filename
of the main document. This is inconvenient if
several child files are to be compiled and
to be kept for distribution.
\end{itemize}

The present package provides a simple interface
to make child files individually compilable by \LaTeX{}.
Compiling a child file then has the same effect as compiling
the main file with an |\includeonly| command
to select the appropriate child.
Moreover the generated document will carry the name of the child
rather than the main file.
This resolves all three above issues.

This feature is meant to make the editing of books,
thesis documents and lecture notes somewhat more convenient.
However, the package can also be used efficiently for
composing a series of documents (such as exercise sheets)
which are typically distributed individually.
It then assists the author in generating the individual documents
(potentially in different versions)
as well as a document containing the collected series.
Another application is in developing style files
or other kinds of included material
where compilation of the style file could redirect
to a sample or test file.

%%%%%%%%%%%%%%%%%%%%%%%%%%%%%%%%%%%%%%%%%%%%%%%%%%%%%%%%%%%%%%%%%%%%%%%%%%%%%%%%
%%%%%%%%%%%%%%%%%%%%%%%%%%%%%%%%%%%%%%%%%%%%%%%%%%%%%%%%%%%%%%%%%%%%%%%%%%%%%%%%
\section{Usage}

First of all, the package \textsf{childdoc} is \emph{not} a standard
\LaTeXe{} |.sty| style file! Therefore it needs to be invoked in
a non-standard way.

%%%%%%%%%%%%%%%%%%%%%%%%%%%%%%%%%%%%%%%%%%%%%%%%%%%%%%%%%%%%%%%%%%%%%%%%%%%%%%%%
\subsection{Included Files}
\label{sec:include}

%%%%%%%%%%%%%%%%%%%%%%%%%%%%%%%%%%%%%%%%
\DescribeMacro{\childdocmain}
To use the package, add the commands
\begin{center}
\begin{tabular}{l}
|\input{childdoc.def}|\\
|\childdocmain{}|\\
\end{tabular}
\end{center}
at the very top of the main \LaTeX{} file,
in particular \emph{before} the |\documentclass| statement!
The argument of |\childdocmain| should be left empty
(but it must be present).

%%%%%%%%%%%%%%%%%%%%%%%%%%%%%%%%%%%%%%%%
\DescribeMacro{\childdocof}
Furthermore, add the commands
\begin{center}
\begin{tabular}{l}
|\input{childdoc.def}|\\
|\childdocof{|\textit{main}|}|\\
\end{tabular}
\end{center}
at the top of every child file \textit{child}
which is included by |\include{|\textit{child}|}|
from within the main file
(or at least for those files to be compiled individually).
The argument \textit{main} must be the filename of the main file.

There are a couple of
considerations in setting up the main and child documents:

%%%%%%%%%%%%%%%%%%%%%%%%%%%%%%%%%%%%%%%%
\paragraph{Restrictions.}

Please note the following restrictions:
\begin{itemize}
\item
|\childdocmain| must be called with one argument \textit{main}
to ensure compatibility with earlier version of the package.
It must either be empty (|\childdocmain{}|)
or precisely match the filename of the main file in which it is specified.
See \secref{sec:detection} for further information.
\item
The filename \textit{main} must be specified without the |.tex| extension.
\item
The filename \textit{main} is case sensitive
(even in case-insensitive file systems)
due to internal string comparison.
\item
The argument \textit{main} should be fully expanded, it cannot be a macro.
\item
Subdirectories and special characters should be avoided in filenames.
\item
The command |\childdocmain{|\textit{main}|}| must be followed by a whitespace.
It should not be followed immediately by another command
or by a comment mark `|%|'.
This is because the \TeX{} parser reads the token immediately following
the argument of |\childdocmain| and puts it
at the beginning of every child section;
however, a white\-space is ignored.
\end{itemize}

%%%%%%%%%%%%%%%%%%%%%%%%%%%%%%%%%%%%%%%%
\paragraph{Content of Main File.}

It is advisable to place all content in the child files included by |\include|.
Any output contained in the main file will appear in all child documents
unless suppressed manually;
it cannot be suppressed automatically by the |\includeonly| directive
and thus should normally be avoided.
A method to include some content in the main file
by means of conditional processing is described in \secref{sec:conditional}.

%%%%%%%%%%%%%%%%%%%%%%%%%%%%%%%%%%%%%%%%
\paragraph{Page Numbering.}

When only a part of the document is compiled,
the appropriate numbering of pages
(as well as other status parameters)
is determined from the |.aux| files.
The latter contain information from previous passes.
However this information needs to propagate through
all intermediate child documents.
Therefore the page numbering in child documents may well
be inconsistent until the complete document is compiled at least once.

A useful (if unconventional) way to always ensure a consistent
page numbering is to restart the numbering in each child document
and denote the pages by `\textit{child}|.|\textit{page}'
where \textit{child} represents the chapter/section number of the child file.
This can be achieved by the command
|\numberwithin{page}{|\textit{child}|}|
of the \textsf{amsmath} package
where \textit{child} can be |chapter| or |section|
depending on the chosen structuring.
Alternatively, one can modify the macro |\thepage| appropriately
and reset the counter |page| at the start of each child file.

%%%%%%%%%%%%%%%%%%%%%%%%%%%%%%%%%%%%%%%%%%%%%%%%%%%%%%%%%%%%%%%%%%%%%%%%%%%%%%%%
\subsection{Conditional Processing}
\label{sec:conditional}

The package provides a mechanism to compile different versions
of a document. To customise the versions further some conditional processing
can come in handy to distinguish which version is being compiled.
The package provides two macros to describe the compilation context:

%%%%%%%%%%%%%%%%%%%%%%%%%%%%%%%%%%%%%%%%
\DescribeMacro{\ifchilddoc}
The conditional |\ifchilddoc| distinguishes between the compilation of
child documents and the main document:
%
\begin{center}
|\ifchilddoc |\textit{child-code}| |[|\||else |\textit{main-code}]| \||fi|
\end{center}

%%%%%%%%%%%%%%%%%%%%%%%%%%%%%%%%%%%%%%%%
\DescribeMacro{\childdocname}
\DescribeMacro{\childdocjob}
The macro |\childdocname| contains the filename (without extension)
of the main or child file being processed.
Note that |\childdocjob| will always contain the name of the main file.

%%%%%%%%%%%%%%%%%%%%%%%%%%%%%%%%%%%%%%%%
\paragraph{Title Page.}

Conditional processing can be used to include a title or banner page
in the main document when proper precautions are taken.
Importantly, the code in the main file should ensure that the page counter
(as well as other status parameters which are stored in the |.aux| files)
takes the same value after the conditional processing.
Otherwise the page numbers may take divergent values
depending on which part is compiled.

For example, a title page could be declared by:
%
\begin{center}
\begin{tabular}{l}
|\ifchilddoc\||else|\\
|\addtocounter{page}{-1}|\\
\textit{code for title page}\\
|\newpage|\\
|\||fi|
\end{tabular}
\end{center}
%
A banner page for the child documents can be generated by:
%
\begin{center}
\begin{tabular}{l}
|\ifchilddoc|\\
|\addtocounter{page}{-1}|\\
\textit{code for banner page}\\
|\newpage|\\
|\||fi|
\end{tabular}
\end{center}
%
Here one could write a message such as:
\begin{center}
|This is the part \childdocname{} of \childdocjob{}.|
\end{center}

%%%%%%%%%%%%%%%%%%%%%%%%%%%%%%%%%%%%%%%%%%%%%%%%%%%%%%%%%%%%%%%%%%%%%%%%%%%%%%%%
\subsection{Flags}
\label{sec:flags}

The package makes it easy to generate different versions
of the main or child documents.
To this end compilation flags can be defined
and assigned different default values.
They will be particularly useful in conjunction
with the forwarding mechanism described in \secref{sec:forward}.

For example, it may be useful to have a flag |\version|
which can be set to |draft| or |final|.
The document source will contain some conditional code
depending on the value of |\version|.
Suppose further, the flag should default to |final| for the main file
and to |draft| for child files
which is a natural assignment for editing the document.
This is achieved by placing the following code
in the preamble of the main document
(below the |\childdocmain| directive):
%
\begin{center}
\begin{tabular}{l}
|\ifchilddoc|\\
|\providecommand{\version}{draft}|\\
|\||else|\\
|\providecommand{\version}{final}|\\
|\||fi|
\end{tabular}
\end{center}
%
The definition by |\providecommand| makes sure
that previous definitions are not overwritten.
Further statements |\providecommand{\version}{...}|
can thus be added before the above code to override it.

For the main file, one might add a line
(between |\childdocmain| and the above block)
%
\begin{center}
|%\ifchilddoc\||else\providecommand{\version}{draft}\||fi|
\end{center}
%
which can be uncommented to produce a draft version.
Likewise one can add a line to the very top of a child file
(above the |\childdocof{|\textit{main}|}| directive)
%
\begin{center}
|%\providecommand{\version}{final}|
\end{center}
%
which can be uncommented to produce the final version of this child document.

%%%%%%%%%%%%%%%%%%%%%%%%%%%%%%%%%%%%%%%%%%%%%%%%%%%%%%%%%%%%%%%%%%%%%%%%%%%%%%%%
\subsection{Forwarding}
\label{sec:forward}

Different versions of the main or child documents
using compilation flags as described in \secref{sec:flags}
can be (permanently) stored in different files
for convenient compilation, viewing and distribution.
To this end, the package defines a command
to pass on compilation to a different file:

%%%%%%%%%%%%%%%%%%%%%%%%%%%%%%%%%%%%%%%%
\DescribeMacro{\childdocforward}
The command |\childdocforward| redirects processing to
another source file:
%
\begin{center}
\begin{tabular}{l}
|\input{childdoc.def}|\\
|\childdocforward[|\textit{main}|]{|\textit{dest}|}|\\
\end{tabular}
\end{center}
%
The argument \textit{dest} is the destination file
(without extension).
It should be the main file or one of the child files.
Note that further \textsf{childdoc} directives
such as |\childdocof| and |\childdocforward|
in the indicated file will be processed in this form.
The optional argument \textit{main}
passes on directly to the main file \textit{main}
while pretending to compile the child \textit{dest}.
This form behaves as if \textit{dest}
issues |\childdocof{|\textit{main}|}| right away,
and no further \textsf{childdoc} directives will be processed.

%%%%%%%%%%%%%%%%%%%%%%%%%%%%%%%%%%%%%%%%
\DescribeMacro{\...prefix}
In the alternative form |\childdocforwardprefix|,
%
\begin{center}
\begin{tabular}{l}
|\input{childdoc.def}|\\
|\childdocforwardprefix[|\textit{main}|]{|\textit{prefix}|}{|\textit{dest}|}|
\end{tabular}
\end{center}
%
the destination file is determined by a pattern
depending on the current file:
To make this work, the current file must be called
`{\textit{prefix}\hspace{0.2em}\textit{suffix}}'
with \textit{prefix} matching precisely the argument.
Processing is then passed on to the file
`{\textit{dest}\hspace{0.2em}\textit{suffix}}'.
Surely, the same effect is achieved by
directly specifying the
argument `{\textit{dest}\hspace{0.2em}\textit{suffix}}'
in the first form.
However, that requires to set up a different file
for each child. With the alternative form of the command
all these files can have exactly the same content
which simplifies setting them up and maintaining them.

For example, the following file |draft.tex|
with a compilation flag |\version| as described in \secref{sec:flags}
compiles the main document as a draft:
%
\begin{center}
\begin{tabular}{l}
|\def\version{draft}|\\
|\input{childdoc.def}|\\
|\childdocforward{|\textit{main}|}|
\end{tabular}
\end{center}
%
Likewise, the following files |final|\textit{nn}|.tex|
compile the final version of the child document
|child|\textit{nn}|.tex|:
%
\begin{center}
\begin{tabular}{l}
|\def\version{final}|\\
|\input{childdoc.def}|\\
|\childdocforwardprefix{final}{child}|
\end{tabular}
\end{center}
%

Note that when several versions of a main file and/or of each child file
are to be generated, it may be convenient to set up a |Makefile| or
shell script to automatise the process.

%%%%%%%%%%%%%%%%%%%%%%%%%%%%%%%%%%%%%%%%%%%%%%%%%%%%%%%%%%%%%%%%%%%%%%%%%%%%%%%%
\subsection{Command Line Processing}
\label{sec:commandline}

The effect of redirection files can also be achieved by invoking
the \LaTeX{} compiler with a more elaborate command line.
Most conveniently this should be done as part
of a shell script or a |Makefile|.

When using \textsf{childdoc} in the main file, the following
command lines effectively perform a redirection
(note that depending on the shell being used,
backslashes may have to be doubled: `|\|' $\to$ `|\\|'):
%
\begin{center}
|... -jobname "|\textit{target}|" |\\|"|[\textit{flags}]%
|\input{childdoc.def}\childdocforward[|\textit{main}|]{|\textit{dest}|}"|
\end{center}
%
Here \textit{target} is the name of the output file,
\textit{main} is the name of the main file
and \textit{dest} is the name of the main or child file to be processed
(all filenames without extensions).
The optional argument \textit{main} can be omitted
if \textit{main} matches \textit{dest}.
Optionally, compilation \textit{flags} can be defined via |\def| commands.
This command line makes the \TeX{} engine believe
it is compiling the file \textit{target}
whose content is specified as the latter parameter.
The provided code then forwards the processing to
\textit{main} or \textit{dest} as described in \secref{sec:forward}.

%%%%%%%%%%%%%%%%%%%%%%%%%%%%%%%%%%%%%%%%%%%%%%%%%%%%%%%%%%%%%%%%%%%%%%%%%%%%%%%%
\subsection{Include by Input}
\label{sec:input}

Including child documents by |\include| has some restrictions by design.
Most notably, the content of a child document always occupies
its own set of pages; pages cannot be shared between child documents.
Usually, this behaviour makes perfect sense
because each child document contain an essential part of the document.
However, in some situations it may be desirable to compose
a document from a collection of parts
without having mandatory page breaks between then.
For this case, the package
provides a mechanism to include parts
by |\input| which can also be processed individually.
However, by construction this mechanism
requires manual handling of the content to be output.

%%%%%%%%%%%%%%%%%%%%%%%%%%%%%%%%%%%%%%%%
\DescribeMacro{\ifchilddocmanual}
The main file should be prepared as usual, see \secref{sec:include}.
However, the document body must make a distinction
between processing of an individual part and of the main document, e.g.:
%
\begin{center}
\begin{tabular}{l}
|\ifchilddocmanual|\\
|\input{\childdocname}|\\
|\||else|\\
\textit{document body with }|\input{|\textit{part}|}|\\
|\||fi|
\end{tabular}
\end{center}
%
The conditional |\ifchilddocmanual| is true whenever
a part to be included by |\input| is being compiled,
and the name of the part is stored in |\childdocname|.

%%%%%%%%%%%%%%%%%%%%%%%%%%%%%%%%%%%%%%%%
\DescribeMacro{\childdocby}
Each part to be included by |\input| should start with:
%
\begin{center}
\begin{tabular}{l}
|\input{childdoc.def}|\\
|\childdocby{|\textit{main}|}|\\
\end{tabular}
\end{center}
%
The directive |\childdocby| is similar to |\childdocof|
described in \secref{sec:include},
but the subsequent selection of content must be done manually.
To that end, both |\ifchilddoc| and |\ifchilddocmanual|
will be true upon processing of a part,
and the name of the part is stored in |\childdocname|.
Note that |\jobname| will be set to the filename of the current part
so that each part receives an individual |.aux| file
that does not interfere with the |.aux| file(s) of the main document.
This behaviour can be altered by the alternative form
|\childdocby[*]{|\textit{main}|}| (with a non-empty optional argument)
which uses the |.aux| file of the main document
by setting |\jobname| to \textit{main}.

%%%%%%%%%%%%%%%%%%%%%%%%%%%%%%%%%%%%%%%%%%%%%%%%%%%%%%%%%%%%%%%%%%%%%%%%%%%%%%%%
\subsection{Driver Development}
\label{sec:driver}

The \textsf{childdoc} mechanism can also be use for the development
of definition files such as \LaTeX{} styles or classes.
This case differs from the above setup with multiple parts
included by |\include| in that no |\includeonly| should be invoked.
This can be achieved by starting the include file
(before |\ProvidesPackage|) with:
%
\begin{center}
\begin{tabular}{l}
|\input{childdoc.def}|\\
|\childdocforward{|\textit{main}|}|\\
\end{tabular}
\end{center}
%
or alternatively with:
%
\begin{center}
\begin{tabular}{l}
|\input{childdoc.def}|\\
|\childdocby{|\textit{main}|}|\\
\end{tabular}
\end{center}
%
Both forms have slightly different effects as described above.
The main file is prepared as usual, see \secref{sec:include}.

%%%%%%%%%%%%%%%%%%%%%%%%%%%%%%%%%%%%%%%%%%%%%%%%%%%%%%%%%%%%%%%%%%%%%%%%%%%%%%%%
\subsection{Legacy Detection}
\label{sec:detection}

The directive |\childdocmain| in the main file can detect
whether the complete document or merely a child is to be compiled
even without using the directive |\childdocof|.
This method is deprecated because it is less robust
and there is no compelling reason to use it;
it is merely provided for backward compatibility
and it may be removed in future versions.

If the detection mechanism is to be used,
it is mandatory to correctly specify
the filename of the main file as the argument of |\childdocmain|:
%
\begin{center}
\begin{tabular}{l}
|\input{childdoc.def}|\\
|\childdocmain{|\textit{main}|}|\\
\end{tabular}
\end{center}
%
If |\jobname| does not match the argument \textit{main} of |\childdocmain|,
it is assumed that |\jobname| points to the child file to be compiled.
When using |\childdocmain| with the main file specified as argument,
it suffices to start a child file
with just |\input{|\textit{main}|}|
without loading of the package and using |\childdocof|.
If instead all processing is done
with the appropriate \textsf{childdoc} directives,
the argument of \textit{main} of |\childdocmain| can be empty.

An alternative version of the command line processing described
in \secref{sec:commandline} using the detection mechanism reads:
%
\begin{center}
|... -jobname "|\textit{target}|" "|[\textit{flags}]%
[|\def\jobname{|\textit{dest}|}|]|\input{|\textit{main}|}"|
\end{center}

%%%%%%%%%%%%%%%%%%%%%%%%%%%%%%%%%%%%%%%%%%%%%%%%%%%%%%%%%%%%%%%%%%%%%%%%%%%%%%%%
\subsection{Manual Code}
\label{sec:manual}

In case one cannot be certain whether the definitions file |childdoc.def|
is installed on the target \TeX{} distribution
and one prefers not to ship it,
it is conceivable to paste a few relevant commands into the sources.

To that end, drop all statements |\input{childdoc.def}|
and perform the replacements as outlined below.
Instead of |\childdocmain{|\textit{main}|}| add the following code
to the top of the main file:
%
\begin{center}
\begin{tabular}{l}
|\||ifdefined\childdocname\endinput\||fi\newif\ifchilddoc|\\
|\edef\childdocname{\scantokens\expandafter{\jobname\noexpand}}|\\
|\def\childdocmain{|\textit{main}|}\||ifx\childdocmain\childdocname\||else|\\
|\childdoctrue\includeonly{\childdocname}\let\jobname\childdocmain\||fi|\\
\end{tabular}
\end{center}
%
Instead of |\childdocof{|\textit{main}|}| just include the main file
at the top of each child file:
%
\begin{center}
|\input{|\textit{main}|}|
\end{center}
%
A simple redirection |\childdocforward{|\textit{dest}|}| is achieved by:
%
\begin{center}
|\def\jobname{|\textit{dest}|}\input{\jobname}|
\end{center}
%
The redirection with prefix
|\childdocforwardprefix[|\textit{prefix}|]{|\textit{dest}|}|
is accomplished by:
%
\begin{center}
\begin{tabular}{l}
|{\edef\jobname{\scantokens\expandafter{\jobname\noexpand}}|\\
|\def\redirectjob |\textit{prefix}|#1~~~{\gdef\jobname{|\textit{dest}|#1}}|\\
|\expandafter\redirectjob\jobname~~~}\input{\jobname}|
\end{tabular}
\end{center}

In an alternative approach,
child documents can be compiled by a specific command line
without additional code or specific definitions:
%
\begin{center}
|... -jobname "|\textit{target}|" "|[\textit{flags}]%
|\includeonly{|\textit{dest}|}\input{|\textit{main}|}"|
\end{center}
%

%%%%%%%%%%%%%%%%%%%%%%%%%%%%%%%%%%%%%%%%%%%%%%%%%%%%%%%%%%%%%%%%%%%%%%%%%%%%%%%%
%%%%%%%%%%%%%%%%%%%%%%%%%%%%%%%%%%%%%%%%%%%%%%%%%%%%%%%%%%%%%%%%%%%%%%%%%%%%%%%%
\section{Information}

%%%%%%%%%%%%%%%%%%%%%%%%%%%%%%%%%%%%%%%%%%%%%%%%%%%%%%%%%%%%%%%%%%%%%%%%%%%%%%%%
\subsection{Copyright}

Copyright \copyright{} 2017--2018 Niklas Beisert

This work may be distributed and/or modified under the
conditions of the \LaTeX{} Project Public License, either version 1.3
of this license or (at your option) any later version.
The latest version of this license is in
  \url{http://www.latex-project.org/lppl.txt}
and version 1.3 or later is part of all distributions of \LaTeX{}
version 2005/12/01 or later.

This work has the LPPL maintenance status `maintained'.

The Current Maintainer of this work is Niklas Beisert.

This work consists of the files |README.txt|, |childdoc.ins| and |childdoc.dtx|
as well as the derived files |childdoc.def|, |cdocsamp.tex|
with |cdocsch1.tex|, |cdocsch2.tex|, |cdocspt3.tex|, |cdocspt4.tex|,
|cdocsdrf.tex|, |cdocsfn1.tex|, |cdocsfn2.tex|
as well as |childdoc.pdf|.

%%%%%%%%%%%%%%%%%%%%%%%%%%%%%%%%%%%%%%%%%%%%%%%%%%%%%%%%%%%%%%%%%%%%%%%%%%%%%%%%
\subsection{Files and Installation}

The package consists of the files:
%
\begin{center}
\begin{tabular}{ll}
    |README.txt|   & readme file \\
    |childdoc.ins| & installation file \\
    |childdoc.dtx| & source file \\
    |childdoc.def| & definition file \\
    |cdocsamp.tex| & sample main file \\
    |cdocsch1.tex| & sample include file \\
    |cdocsch2.tex| & sample include file \\
    |cdocspt3.tex| & sample part file \\
    |cdocspt4.tex| & sample part file \\
    |cdocsdrf.tex| & sample redirection file \\
    |cdocsfn1.tex| & sample redirection file \\
    |cdocsfn2.tex| & sample redirection file \\
    |childdoc.pdf| & manual
\end{tabular}
\end{center}
%
The distribution consists of the files
|README.txt|, |childdoc.ins| and |childdoc.dtx|.
%
\begin{itemize}
\item
Run (pdf)\LaTeX{} on |childdoc.dtx|
to compile the manual |childdoc.pdf| (this file).
\item
Run \LaTeX{} on |childdoc.ins| to create the definitions file |childdoc.def|
and the sample |cdocsamp.tex| with include files
|cdocsch1.tex|, |cdocsch2.tex|, |cdocspt3.tex|, |cdocspt4.tex|,
|cdocsdrf.tex|, |cdocsfn1.tex|, |cdocsfn2.tex|.
Then copy the file |childdoc.def| to an appropriate directory of your \LaTeX{}
distribution, e.g.\ \textit{texmf-root}|/tex/latex/childdoc|.
\end{itemize}

%%%%%%%%%%%%%%%%%%%%%%%%%%%%%%%%%%%%%%%%%%%%%%%%%%%%%%%%%%%%%%%%%%%%%%%%%%%%%%%%
\subsection{Related CTAN Packages}

There are several other packages which offer a similar functionality:
%
\begin{itemize}
\item
The packages
\href{http://ctan.org/pkg/docmute}{\textsf{docmute}},
\href{http://ctan.org/pkg/includex}{\textsf{includex}} and
\href{http://ctan.org/pkg/standalone}{\textsf{standalone}}
provide commands to include only the document body of
a child file thus allowing both files to be compiled individually.
\item
The packages \href{http://ctan.org/pkg/subdocs}{\textsf{subdocs}}
and \href{http://ctan.org/pkg/subfiles}{\textsf{subfiles}}
provide structures in which the main and child documents can be
encapsulated and allowing them to be compiled individually.
The inclusion mechanism is different from the conventional |\include|.
\item
The package \href{http://ctan.org/pkg/combine}{\textsf{combine}}
is an elaborate solution to combine several documents into one.
\end{itemize}
%
See also the CTAN topic \href{http://ctan.org/topic/subdocs}{\textsf{subdocs}}
for further related packages.
The present package differs from the above solutions in that
a document structure constructed with the conventional |\include| mechanism
just needs two extra commands at the top of every file
such that all constituent files can be compiled individually.

%%%%%%%%%%%%%%%%%%%%%%%%%%%%%%%%%%%%%%%%%%%%%%%%%%%%%%%%%%%%%%%%%%%%%%%%%%%%%%%%
%\subsection{Feature Suggestions}
%
%The following is a list of features which may be useful for future
%versions of this package:
%%
%\begin{itemize}
%\item
%\ldots
%\end{itemize}

%%%%%%%%%%%%%%%%%%%%%%%%%%%%%%%%%%%%%%%%%%%%%%%%%%%%%%%%%%%%%%%%%%%%%%%%%%%%%%%%
\subsection{Revision History}

%%%%%%%%%%%%%%%%%%%%%%%%%%%%%%%%%%%%%%%%
\paragraph{v2.0:} 2018/12/30

\begin{itemize}
\item
immediate forward processing
\item
added |\childdocby| mechanism
\item
manual restructured
\end{itemize}

%%%%%%%%%%%%%%%%%%%%%%%%%%%%%%%%%%%%%%%%
\paragraph{v1.6:} 2018/01/17

\begin{itemize}
\item
application for development of include files
\item
corrections to manual
\end{itemize}

%%%%%%%%%%%%%%%%%%%%%%%%%%%%%%%%%%%%%%%%
\paragraph{v1.5:} 2017/05/21

\begin{itemize}
\item
more complete structuring introduced
\item
|\childdocof| introduced
\item
|\childdoc| renamed to |\childdocmain|
\item
|\childredirect| renamed to |\childdocforward| and |\childdocforwardprefix|
and functionality expanded
\end{itemize}

%%%%%%%%%%%%%%%%%%%%%%%%%%%%%%%%%%%%%%%%
\paragraph{v1.0:} 2017/04/27

\begin{itemize}
\item
manual and install package
\item
first version published on CTAN
\end{itemize}

%%%%%%%%%%%%%%%%%%%%%%%%%%%%%%%%%%%%%%%%
\paragraph{v0.6:} 2017/04/26

\begin{itemize}
\item
redirection mechanism added
\end{itemize}

%%%%%%%%%%%%%%%%%%%%%%%%%%%%%%%%%%%%%%%%
\paragraph{v0.5:} 2017/04/26

\begin{itemize}
\item
functionality in definition file
\end{itemize}


%%%%%%%%%%%%%%%%%%%%%%%%%%%%%%%%%%%%%%%%%%%%%%%%%%%%%%%%%%%%%%%%%%%%%%%%%%%%%%%%
%%%%%%%%%%%%%%%%%%%%%%%%%%%%%%%%%%%%%%%%%%%%%%%%%%%%%%%%%%%%%%%%%%%%%%%%%%%%%%%%
%%%%%%%%%%%%%%%%%%%%%%%%%%%%%%%%%%%%%%%%%%%%%%%%%%%%%%%%%%%%%%%%%%%%%%%%%%%%%%%%
\appendix

\settowidth\MacroIndent{\rmfamily\scriptsize 000\ }

 \DocInput{childdoc.dtx}

\end{document}
%</driver>
% \fi
%
% %%%%%%%%%%%%%%%%%%%%%%%%%%%%%%%%%%%%%%%%%%%%%%%%%%%%%%%%%%%%%%%%%%%%%%%%%%%%%%
% %%%%%%%%%%%%%%%%%%%%%%%%%%%%%%%%%%%%%%%%%%%%%%%%%%%%%%%%%%%%%%%%%%%%%%%%%%%%%%
% \section{Sample}
%\iffalse
%<*samplemain>
%\fi
%
% The following presents a sample document
% with two chapters, two parts, a title page,
% a compile flag as well as three forwarding files to set the flag.
% It consists of eight |.tex| files:
% \begin{center}
% \begin{tabular}{ll}
% |cdocsamp.tex|&main file\\
% |cdocsch1.tex|&include file for chapter 1\\
% |cdocsch2.tex|&include file for chapter 2\\
% |cdocspt3.tex|&include file for part 3\\
% |cdocspt4.tex|&include file for part 4\\
% |cdocsdrf.tex|&forwarding file for main file in draft mode\\
% |cdocsfi1.tex|&forwarding file for final version of chapter 1\\
% |cdocsfi2.tex|&forwarding file for final version of chapter 2\\
% \end{tabular}
% \end{center}
% Each of the eight files can be compiled directly by the \LaTeX{} compiler.
%
% %%%%%%%%%%%%%%%%%%%%%%%%%%%%%%%%%%%%%%
% \paragraph{Main File.}
%
% The main file is called |cdocsamp.tex|.
%
% Load the \textsf{childdoc} definitions and
% declare the filename for the main document:
%    \begin{macrocode}
\input{childdoc.def}
\childdocmain{}
%    \end{macrocode}

% Optional override for |\version| flag:
%    \begin{macrocode}
%%\ifchilddoc\else\providecommand{\version}{draft}\fi
%    \end{macrocode}

% Define the default values for the |\version| flag
% (|final| for the main file and |draft| for childs):
%    \begin{macrocode}
\ifchilddoc
\providecommand{\version}{draft}
\else
\providecommand{\version}{final}
\fi
%    \end{macrocode}

% Load the standard document class:
%    \begin{macrocode}
\documentclass[12pt]{article}
%    \end{macrocode}

% Start the document body:
%    \begin{macrocode}
\begin{document}
%    \end{macrocode}

% Declare a title page.
% Print title, part of document being processed and version flag:
%    \begin{macrocode}
\addtocounter{page}{-1}
\begin{center}
{\LARGE\bfseries{}childdoc example\par}
\vspace{1cm}
\ifchilddoc
\ifchilddocmanual part\else chapter\fi:
`\childdocname' of `\childdocjob'\par
\else
main document: `\childdocjob'\par
\fi
version: \version\par
\end{center}
\newpage
%    \end{macrocode}

% Manually include selected file,
% otherwise process as usual:
%    \begin{macrocode}
\ifchilddocmanual
\section*{part `\childdocname'}
\input{\childdocname}
\else
%    \end{macrocode}

% Include the two chapters:
%    \begin{macrocode}
\include{cdocsch1}
\include{cdocsch2}
%    \end{macrocode}

% Include the two parts unless only chapters should be displayed:
%    \begin{macrocode}
\ifchilddoc\else
\section{part three}
\input{cdocspt3}
\section{part four}
\input{cdocspt4}
\fi
%    \end{macrocode}

% Process as usual until here:
%    \begin{macrocode}
\fi
%    \end{macrocode}

% End of document body:
%    \begin{macrocode}
\end{document}
%    \end{macrocode}
%\iffalse
%</samplemain>
%\fi
%
% %%%%%%%%%%%%%%%%%%%%%%%%%%%%%%%%%%%%%%
% \paragraph{Chapter Include Files.}
%
% The include files are called |cdocsch1.tex| and |cdocsch2.tex|.
%
%\iffalse
%<*samplechap1|samplechap2>
%\fi

% Optional override for |\version| flag:
%    \begin{macrocode}
%%\providecommand{\version}{final}
%    \end{macrocode}

% Include the main document:
%    \begin{macrocode}
\input{childdoc.def}
\childdocof{cdocsamp}
%    \end{macrocode}

%\iffalse
%</samplechap1|samplechap2>
%\fi
%
%\iffalse
%<*samplechap1>
%\fi
% Some text for chapter 1:
%    \begin{macrocode}
\section{one}
some text in chapter one
%    \end{macrocode}

%\iffalse
%</samplechap1>
%\fi
% Some text for chapter 2:
%\iffalse
%<*samplechap2>
%\fi
%    \begin{macrocode}
\section{two}
more text in chapter two
%    \end{macrocode}

%\iffalse
%</samplechap2>
%\fi
%
% %%%%%%%%%%%%%%%%%%%%%%%%%%%%%%%%%%%%%%
% \paragraph{Part Include Files.}
%
% The include files are called |cdocspt3.tex| and |cdocspt4.tex|.
%
%\iffalse
%<*samplepart3|samplepart4>
%\fi

% Optional override for |\version| flag:
%    \begin{macrocode}
%%\providecommand{\version}{final}
%    \end{macrocode}

% Include the main document:
%    \begin{macrocode}
\input{childdoc.def}
\childdocby{cdocsamp}
%    \end{macrocode}

%\iffalse
%</samplepart3|samplepart4>
%\fi
%
%\iffalse
%<*samplepart3>
%\fi
% Some text for part 3:
%    \begin{macrocode}
some text in part three
%    \end{macrocode}

%\iffalse
%</samplepart3>
%\fi
% Some text for part 4:
%\iffalse
%<*samplepart4>
%\fi
%    \begin{macrocode}
more text in part four
%    \end{macrocode}

%\iffalse
%</samplepart4>
%\fi
%
% %%%%%%%%%%%%%%%%%%%%%%%%%%%%%%%%%%%%%%
% \paragraph{Forwarding for a Complete Draft.}
%
% The following forwarding file |cdocsdrf.tex|
% compiles the main document in draft mode:
%\iffalse
%<*sampledraft>
%\fi
%    \begin{macrocode}
\def\version{draft}
\input{childdoc.def}
\childdocforward{cdocsamp}
%    \end{macrocode}

%\iffalse
%</sampledraft>
%\fi
%
% %%%%%%%%%%%%%%%%%%%%%%%%%%%%%%%%%%%%%%
% \paragraph{Forwarding for Final Version of the Chapters.}
%
% The following forwarding files |cdocsfn1.tex| and |cdocsfn2.tex|
% (with identical content)
% compile the final versions of the child documents
% |cdocsch1.tex| and |cdocsch2.tex|, respectively:
%\iffalse
%<*samplefinal>
%\fi
%    \begin{macrocode}
\def\version{final}
\input{childdoc.def}
\childdocforwardprefix[cdocsamp]{cdocsfn}{cdocsch}
%    \end{macrocode}

%\iffalse
%</samplefinal>
%\fi
%
% %%%%%%%%%%%%%%%%%%%%%%%%%%%%%%%%%%%%%%
% \paragraph{Command Line Processing.}
%
% The following three command lines generate the output files
% |cdocscld|, |cdocscl1| and |cdocscl2|
% which should be identical to
% |cdocsdrf|, |cdocsch1| and |cdocsfn2|, respectively:
% \begin{center}
% \begin{tabular}{l}
% |latex -jobname cdocscld \|\\
% |  "\def\version{draft}\input{childdoc.def}\childdocforward{cdocsamp}"|\\
% |latex -jobname cdocscl1 \|\\
% |  "\input{childdoc.def}\childdocforward[cdocsamp]{cdocsch1}"|\\
% |latex -jobname cdocscl2 \|\\
% |  "\def\version{final}\input{childdoc.def}\childdocforward{cdocsch2}"|
% \end{tabular}
% \end{center}
% Note that the trailing backslash on each first line
% merely continues the input to the second line
% (for convenient cut ant paste).
% Furthermore, the command |latex| can be replaced by any
% of its alternative versions such as |pdflatex|.
%
% %%%%%%%%%%%%%%%%%%%%%%%%%%%%%%%%%%%%%%%%%%%%%%%%%%%%%%%%%%%%%%%%%%%%%%%%%%%%%%
% %%%%%%%%%%%%%%%%%%%%%%%%%%%%%%%%%%%%%%%%%%%%%%%%%%%%%%%%%%%%%%%%%%%%%%%%%%%%%%
% \section{Implementation}
%\iffalse
%<*package>
%\fi
%
% This section describes the definitions file |childdoc.def|.

% The definitions cannot be loaded using |\usepackage| or |\RequirePackage|
% which has a mechanism to prevent loading a style file more than once.
% When loading the definitions by means of |\input|
% multiple instances have to be prevented manually:
%\iffalse
%This code needs to be before the `\ProvidesFile' directive
%which is defined at the beginning of this file.
%Therefore it is also placed there and commented out here.
%</package>
%<*discard>
%\fi
%    \begin{macrocode}
\ifdefined\childdocmain\endinput\fi
%    \end{macrocode}
%\iffalse
%</discard>
%<*package>
%\fi
%
% \macro{\ifchilddoc}
% \macro{\ifchilddocmanual}
% The conditional |\ifchilddoc| tells whether a
% child (true) or main (false) document is being compiled.
% The conditional |\ifchilddocmanual| tells whether
% the |\includeonly| mechanism is used (false) or
% the selection of child files must be performed manually (true).
% The definitions initialise to false:
%    \begin{macrocode}
\newif\ifchilddoc
\newif\ifchilddocmanual
%    \end{macrocode}

% \macro{\childdocname}
% \macro{\childdocjob}
% The macro |\childdocname| stores the name of the main document
% to be compiled. The macro |\childdocjob| stores the name of
% the document on which the \LaTeX{} compiler was originally invoked.
% The content of |\jobname| cannot be compared
% to filenames specified in the source due to different catcodes.
% The following code rescans |\jobname|, stores the result
% in |\childdocname| and saves a copy in |\childdocjob|:
%    \begin{macrocode}
\edef\childdocname{\scantokens\expandafter{\jobname\noexpand}}
\let\childdocjob\childdocname
%    \end{macrocode}

% \macro{\childdocdisable}
% The macro |\childdocdisable| prevents the main file
% from being processed more than once.
% At this stage, the main document command |\childdocmain|
% is assumed to be called once again where it should do nothing.
% Any subsequent call to it should prevent
% a secondary processing of the main document
% It overwrites the forwarding commands
% |\childdocof| and |\childdocforward|
% with empty macros to prevent further inclusions of the main document:
%    \begin{macrocode}
\newcommand{\childdocdisable}
{
  \renewcommand{\childdocmain}[1]{\renewcommand{\childdocmain}[1]{\endinput}}
  \renewcommand{\childdocof}[1]{}
  \renewcommand{\childdocby}[2][]{}
  \renewcommand{\childdocforward}[2][]{}
  \renewcommand{\childdocdisable}{}
}
%    \end{macrocode}

% \macro{\childdocmain}
% The macro |\childdocmain| is to be called at the top of the main file
% with nothing or the main filename (without extension) as argument.
% First, it breaks loops.
% If the argument is not empty and does not match |\childdocname|
% (which is set by the first inclusion of |childdoc.def|),
% |\ifchilddoc| is set to true, |\includeonly| is applied to the child file
% and |\jobname| is set to the main file
% (for proper handling of |.aux| files):
%    \begin{macrocode}
\newcommand{\childdocmain}[1]
{
  \childdocdisable\childdocmain{}
  \if?#1?\else
    \begingroup
      \def\childdoctmp{#1}
      \ifx\childdoctmp\childdocname
        \def\childdoctmp{}
      \else
        \def\childdoctmp
        {
          \childdoctrue
          \includeonly{\childdocname}
          \def\childdocjob{#1}
          \def\jobname{#1}
        }
      \fi
      \expandafter
    \endgroup
    \childdoctmp
  \fi
}
%    \end{macrocode}

% \macro{\childdocof}
% The command |\childdocof| redirects
% compilation to the main file |#1|.
%    \begin{macrocode}
\newcommand{\childdocof}[1]
{
  \childdocdisable
  \childdoctrue
  \includeonly{\childdocname}
  \def\jobname{#1}
  \def\childdocjob{#1}
  \input{#1}
}
%    \end{macrocode}

% \macro{\childdocby}
% The command |\childdocby| ....
%    \begin{macrocode}
\newcommand{\childdocby}[2][]
{
  \childdocdisable
  \childdoctrue
  \childdocmanualtrue
  \if?#1?\else
    \def\jobname{#2}
  \fi
  \def\childdocjob{#2}
  \input{#2}
  \endinput
}
%    \end{macrocode}

% \macro{\childdocforward}
% The command |\childdocforward| redirects
% compilation to the main file or
% (if the optional argument is given) a child file.
% Parameters are set as if the main file
% or a child file starting with |\childdocof| was compiled.
% Then compilation is handed over to the main file:
%    \begin{macrocode}
\newcommand{\childdocforward}[2][]
{
  \begingroup
    \if?#1?
      \def\childdoctmp
      {
        \def\childdocname{#2}
        \def\childdocjob{#2}
        \def\jobname{#2}
        \input{#2}
        \endinput
      }
    \else
      \def\childdoctmp
      {
        \childdocdisable
        \def\childdocname{#2}
        \childdoctrue
        \includeonly{#2}
        \def\childdocjob{#1}
        \def\jobname{#1}
        \input{#1}
        \endinput
      }
    \fi
    \expandafter
  \endgroup
  \childdoctmp
}
%    \end{macrocode}

% \macro{\childdocforwardprefix}
% The command |\childdocforwardprefix| redirects
% compilation to the main or a child file by means of a pattern.
% The prefix |#1| in the current filename is replaced by |#2|
% and the suffix of the current filename is kept
% (it is assumed that the filename does not contain the substring `|~~~|'
% which is used as a delimiter).
% Compilation is handed over to the new file by |\childdocforward|:
%    \begin{macrocode}
\newcommand{\childdocforwardprefix}[3][]
{
  \begingroup
    \def\childdocextract #2##1~~~{\def\childdoctmp{\childdocforward[#1]{#3##1}}}
    \expandafter\childdocextract\childdocname~~~
    \expandafter
  \endgroup
  \childdoctmp
}
%    \end{macrocode}

% \macro{\childdoc}
% The deprecated macro |\childdoc| is a legacy version of |\childdocmain|:
%    \begin{macrocode}
\newcommand{\childdoc}{\childdocmain}
%    \end{macrocode}

% \macro{\childdocredirect}
% The deprecated macro |\childdocredirect| is a legacy version
% of |\childdocforward| and |\childdocforwardprefix|:
%    \begin{macrocode}
\newcommand{\childdocredirect}[2][]
{
  \begingroup
    \if?#1?
      \def\childdoctmp{\childdocforward{#2}}
    \else
      \def\childdoctmp{\childdocforwardprefix{#1}{#2}}
    \fi
    \expandafter
  \endgroup
  \childdoctmp
}
%    \end{macrocode}

%\iffalse
%</package>
%\fi
%
\endinput
\childdocforward[cdocsamp]{cdocsch1}"|\\
% |latex -jobname cdocscl2 \|\\
% |  "\def\version{final}% \iffalse
%
% childdoc.dtx Copyright (C) 2017-2018 Niklas Beisert
%
% This work may be distributed and/or modified under the
% conditions of the LaTeX Project Public License, either version 1.3
% of this license or (at your option) any later version.
% The latest version of this license is in
%   http://www.latex-project.org/lppl.txt
% and version 1.3 or later is part of all distributions of LaTeX
% version 2005/12/01 or later.
%
% This work has the LPPL maintenance status `maintained'.
%
% The Current Maintainer of this work is Niklas Beisert.
%
% This work consists of the files childdoc.dtx and childdoc.ins
% and the derived files childdoc.def and cdocsamp.tex with
% cdocsch1.tex, cdocsch2.tex, cdocsdrf.tex, cdocsfn1.tex, cdocsfn2.tex.
%
%<package>\ifdefined\childdocmain\endinput\fi
%<package>\ProvidesFile{childdoc.def}[2018/12/30 v2.0 child document driver]
%<samplemain>\ProvidesFile{cdocsamp.tex}[2018/12/30 v2.0 sample for childdoc]
%<*driver>
%\ProvidesFile{childdoc.drv}[2018/12/30 v2.0 childdoc reference manual file]
\PassOptionsToClass{10pt,a4paper}{article}
\documentclass{ltxdoc}

\usepackage[margin=35mm]{geometry}
\usepackage{hyperref}
\usepackage{hyperxmp}
\usepackage[usenames]{color}

\hypersetup{colorlinks=true}
\hypersetup{pdfstartview=FitH}
\hypersetup{pdfpagemode=UseNone}
\hypersetup{pdfsource={}}
\hypersetup{pdflang={en-UK}}
\hypersetup{pdfcopyright={Copyright 2017-2018 Niklas Beisert.
  This work may be distributed and/or modified under the
  conditions of the LaTeX Project Public License, either version 1.3
  of this license or (at your option) any later version.}}
\hypersetup{pdflicenseurl={http://www.latex-project.org/lppl.txt}}
\hypersetup{pdfcontactaddress={ETH Zurich, ITP, HIT K,
  Wolfgang-Pauli-Strasse 27}}
\hypersetup{pdfcontactpostcode={8093}}
\hypersetup{pdfcontactcity={Zurich}}
\hypersetup{pdfcontactcountry={Switzerland}}
\hypersetup{pdfcontactemail={nbeisert@itp.phys.ethz.ch}}
\hypersetup{pdfcontacturl={http://people.phys.ethz.ch/\xmptilde nbeisert/}}

\newcommand{\secref}[1]{\hyperref[#1]{section \ref*{#1}}}

\parskip1ex
\parindent0pt
\let\olditemize\itemize
\def\itemize{\olditemize\parskip0pt}

\begin{document}

\title{The \textsf{childdoc} Package}
\hypersetup{pdftitle={The childdoc Package}}
\author{Niklas Beisert\\[2ex]
  Institut f\"ur Theoretische Physik\\
  Eidgen\"ossische Technische Hochschule Z\"urich\\
  Wolfgang-Pauli-Strasse 27, 8093 Z\"urich, Switzerland\\[1ex]
  \href{mailto:nbeisert@itp.phys.ethz.ch}
  {\texttt{nbeisert@itp.phys.ethz.ch}}}
\hypersetup{pdfauthor={Niklas Beisert}}
\hypersetup{pdfsubject={Manual for the LaTeX2e Package childdoc}}
\date{30 December 2018, \textsf{v2.0}}
\maketitle

\begin{abstract}\noindent
\textsf{childdoc} is a \LaTeXe{} package
that enables the direct compilation
of document sections included by |\include|
to individual files.
\end{abstract}

\begingroup
\parskip0ex
\tableofcontents
\endgroup

%%%%%%%%%%%%%%%%%%%%%%%%%%%%%%%%%%%%%%%%%%%%%%%%%%%%%%%%%%%%%%%%%%%%%%%%%%%%%%%%
%%%%%%%%%%%%%%%%%%%%%%%%%%%%%%%%%%%%%%%%%%%%%%%%%%%%%%%%%%%%%%%%%%%%%%%%%%%%%%%%
\section{Introduction}

\LaTeX{} provides a mechanism to structure a large document (such as a book)
into a main file and several child files (containing the chapters)
using the |\include| command.
This mechanism is beneficial for documents
which span hundreds of pages in order to
make the source file(s) more manageable.
Moreover, compilation can be restricted to
selected child files by means of the |\includeonly| command.
The latter feature can be used to reduce the compilation time while editing
(this was significantly more useful in the earlier days of \LaTeX{})
or to generate a smaller document which is easier to navigate.
Another application of |\includeonly| is to generate
documents consisting of selected parts of the complete document.

However, there are a few drawbacks of the plain |\include| mechanism:
\begin{itemize}
\item
The child files cannot be compiled on their own,
they can only be compiled via the main file.
A naive editing environment
(such as a text editor with an option
to have the current file processed by \LaTeX)
may require one to switch to the main file before compiling;
attempting to compile the child file produces errors.
\item
The main file must be modified (each time)
to adjust the |\includeonly| command
to the present needs. This easily leaves the main file in a messy state.
\item
The generated document will always carry the filename
of the main document. This is inconvenient if
several child files are to be compiled and
to be kept for distribution.
\end{itemize}

The present package provides a simple interface
to make child files individually compilable by \LaTeX{}.
Compiling a child file then has the same effect as compiling
the main file with an |\includeonly| command
to select the appropriate child.
Moreover the generated document will carry the name of the child
rather than the main file.
This resolves all three above issues.

This feature is meant to make the editing of books,
thesis documents and lecture notes somewhat more convenient.
However, the package can also be used efficiently for
composing a series of documents (such as exercise sheets)
which are typically distributed individually.
It then assists the author in generating the individual documents
(potentially in different versions)
as well as a document containing the collected series.
Another application is in developing style files
or other kinds of included material
where compilation of the style file could redirect
to a sample or test file.

%%%%%%%%%%%%%%%%%%%%%%%%%%%%%%%%%%%%%%%%%%%%%%%%%%%%%%%%%%%%%%%%%%%%%%%%%%%%%%%%
%%%%%%%%%%%%%%%%%%%%%%%%%%%%%%%%%%%%%%%%%%%%%%%%%%%%%%%%%%%%%%%%%%%%%%%%%%%%%%%%
\section{Usage}

First of all, the package \textsf{childdoc} is \emph{not} a standard
\LaTeXe{} |.sty| style file! Therefore it needs to be invoked in
a non-standard way.

%%%%%%%%%%%%%%%%%%%%%%%%%%%%%%%%%%%%%%%%%%%%%%%%%%%%%%%%%%%%%%%%%%%%%%%%%%%%%%%%
\subsection{Included Files}
\label{sec:include}

%%%%%%%%%%%%%%%%%%%%%%%%%%%%%%%%%%%%%%%%
\DescribeMacro{\childdocmain}
To use the package, add the commands
\begin{center}
\begin{tabular}{l}
|\input{childdoc.def}|\\
|\childdocmain{}|\\
\end{tabular}
\end{center}
at the very top of the main \LaTeX{} file,
in particular \emph{before} the |\documentclass| statement!
The argument of |\childdocmain| should be left empty
(but it must be present).

%%%%%%%%%%%%%%%%%%%%%%%%%%%%%%%%%%%%%%%%
\DescribeMacro{\childdocof}
Furthermore, add the commands
\begin{center}
\begin{tabular}{l}
|\input{childdoc.def}|\\
|\childdocof{|\textit{main}|}|\\
\end{tabular}
\end{center}
at the top of every child file \textit{child}
which is included by |\include{|\textit{child}|}|
from within the main file
(or at least for those files to be compiled individually).
The argument \textit{main} must be the filename of the main file.

There are a couple of
considerations in setting up the main and child documents:

%%%%%%%%%%%%%%%%%%%%%%%%%%%%%%%%%%%%%%%%
\paragraph{Restrictions.}

Please note the following restrictions:
\begin{itemize}
\item
|\childdocmain| must be called with one argument \textit{main}
to ensure compatibility with earlier version of the package.
It must either be empty (|\childdocmain{}|)
or precisely match the filename of the main file in which it is specified.
See \secref{sec:detection} for further information.
\item
The filename \textit{main} must be specified without the |.tex| extension.
\item
The filename \textit{main} is case sensitive
(even in case-insensitive file systems)
due to internal string comparison.
\item
The argument \textit{main} should be fully expanded, it cannot be a macro.
\item
Subdirectories and special characters should be avoided in filenames.
\item
The command |\childdocmain{|\textit{main}|}| must be followed by a whitespace.
It should not be followed immediately by another command
or by a comment mark `|%|'.
This is because the \TeX{} parser reads the token immediately following
the argument of |\childdocmain| and puts it
at the beginning of every child section;
however, a white\-space is ignored.
\end{itemize}

%%%%%%%%%%%%%%%%%%%%%%%%%%%%%%%%%%%%%%%%
\paragraph{Content of Main File.}

It is advisable to place all content in the child files included by |\include|.
Any output contained in the main file will appear in all child documents
unless suppressed manually;
it cannot be suppressed automatically by the |\includeonly| directive
and thus should normally be avoided.
A method to include some content in the main file
by means of conditional processing is described in \secref{sec:conditional}.

%%%%%%%%%%%%%%%%%%%%%%%%%%%%%%%%%%%%%%%%
\paragraph{Page Numbering.}

When only a part of the document is compiled,
the appropriate numbering of pages
(as well as other status parameters)
is determined from the |.aux| files.
The latter contain information from previous passes.
However this information needs to propagate through
all intermediate child documents.
Therefore the page numbering in child documents may well
be inconsistent until the complete document is compiled at least once.

A useful (if unconventional) way to always ensure a consistent
page numbering is to restart the numbering in each child document
and denote the pages by `\textit{child}|.|\textit{page}'
where \textit{child} represents the chapter/section number of the child file.
This can be achieved by the command
|\numberwithin{page}{|\textit{child}|}|
of the \textsf{amsmath} package
where \textit{child} can be |chapter| or |section|
depending on the chosen structuring.
Alternatively, one can modify the macro |\thepage| appropriately
and reset the counter |page| at the start of each child file.

%%%%%%%%%%%%%%%%%%%%%%%%%%%%%%%%%%%%%%%%%%%%%%%%%%%%%%%%%%%%%%%%%%%%%%%%%%%%%%%%
\subsection{Conditional Processing}
\label{sec:conditional}

The package provides a mechanism to compile different versions
of a document. To customise the versions further some conditional processing
can come in handy to distinguish which version is being compiled.
The package provides two macros to describe the compilation context:

%%%%%%%%%%%%%%%%%%%%%%%%%%%%%%%%%%%%%%%%
\DescribeMacro{\ifchilddoc}
The conditional |\ifchilddoc| distinguishes between the compilation of
child documents and the main document:
%
\begin{center}
|\ifchilddoc |\textit{child-code}| |[|\||else |\textit{main-code}]| \||fi|
\end{center}

%%%%%%%%%%%%%%%%%%%%%%%%%%%%%%%%%%%%%%%%
\DescribeMacro{\childdocname}
\DescribeMacro{\childdocjob}
The macro |\childdocname| contains the filename (without extension)
of the main or child file being processed.
Note that |\childdocjob| will always contain the name of the main file.

%%%%%%%%%%%%%%%%%%%%%%%%%%%%%%%%%%%%%%%%
\paragraph{Title Page.}

Conditional processing can be used to include a title or banner page
in the main document when proper precautions are taken.
Importantly, the code in the main file should ensure that the page counter
(as well as other status parameters which are stored in the |.aux| files)
takes the same value after the conditional processing.
Otherwise the page numbers may take divergent values
depending on which part is compiled.

For example, a title page could be declared by:
%
\begin{center}
\begin{tabular}{l}
|\ifchilddoc\||else|\\
|\addtocounter{page}{-1}|\\
\textit{code for title page}\\
|\newpage|\\
|\||fi|
\end{tabular}
\end{center}
%
A banner page for the child documents can be generated by:
%
\begin{center}
\begin{tabular}{l}
|\ifchilddoc|\\
|\addtocounter{page}{-1}|\\
\textit{code for banner page}\\
|\newpage|\\
|\||fi|
\end{tabular}
\end{center}
%
Here one could write a message such as:
\begin{center}
|This is the part \childdocname{} of \childdocjob{}.|
\end{center}

%%%%%%%%%%%%%%%%%%%%%%%%%%%%%%%%%%%%%%%%%%%%%%%%%%%%%%%%%%%%%%%%%%%%%%%%%%%%%%%%
\subsection{Flags}
\label{sec:flags}

The package makes it easy to generate different versions
of the main or child documents.
To this end compilation flags can be defined
and assigned different default values.
They will be particularly useful in conjunction
with the forwarding mechanism described in \secref{sec:forward}.

For example, it may be useful to have a flag |\version|
which can be set to |draft| or |final|.
The document source will contain some conditional code
depending on the value of |\version|.
Suppose further, the flag should default to |final| for the main file
and to |draft| for child files
which is a natural assignment for editing the document.
This is achieved by placing the following code
in the preamble of the main document
(below the |\childdocmain| directive):
%
\begin{center}
\begin{tabular}{l}
|\ifchilddoc|\\
|\providecommand{\version}{draft}|\\
|\||else|\\
|\providecommand{\version}{final}|\\
|\||fi|
\end{tabular}
\end{center}
%
The definition by |\providecommand| makes sure
that previous definitions are not overwritten.
Further statements |\providecommand{\version}{...}|
can thus be added before the above code to override it.

For the main file, one might add a line
(between |\childdocmain| and the above block)
%
\begin{center}
|%\ifchilddoc\||else\providecommand{\version}{draft}\||fi|
\end{center}
%
which can be uncommented to produce a draft version.
Likewise one can add a line to the very top of a child file
(above the |\childdocof{|\textit{main}|}| directive)
%
\begin{center}
|%\providecommand{\version}{final}|
\end{center}
%
which can be uncommented to produce the final version of this child document.

%%%%%%%%%%%%%%%%%%%%%%%%%%%%%%%%%%%%%%%%%%%%%%%%%%%%%%%%%%%%%%%%%%%%%%%%%%%%%%%%
\subsection{Forwarding}
\label{sec:forward}

Different versions of the main or child documents
using compilation flags as described in \secref{sec:flags}
can be (permanently) stored in different files
for convenient compilation, viewing and distribution.
To this end, the package defines a command
to pass on compilation to a different file:

%%%%%%%%%%%%%%%%%%%%%%%%%%%%%%%%%%%%%%%%
\DescribeMacro{\childdocforward}
The command |\childdocforward| redirects processing to
another source file:
%
\begin{center}
\begin{tabular}{l}
|\input{childdoc.def}|\\
|\childdocforward[|\textit{main}|]{|\textit{dest}|}|\\
\end{tabular}
\end{center}
%
The argument \textit{dest} is the destination file
(without extension).
It should be the main file or one of the child files.
Note that further \textsf{childdoc} directives
such as |\childdocof| and |\childdocforward|
in the indicated file will be processed in this form.
The optional argument \textit{main}
passes on directly to the main file \textit{main}
while pretending to compile the child \textit{dest}.
This form behaves as if \textit{dest}
issues |\childdocof{|\textit{main}|}| right away,
and no further \textsf{childdoc} directives will be processed.

%%%%%%%%%%%%%%%%%%%%%%%%%%%%%%%%%%%%%%%%
\DescribeMacro{\...prefix}
In the alternative form |\childdocforwardprefix|,
%
\begin{center}
\begin{tabular}{l}
|\input{childdoc.def}|\\
|\childdocforwardprefix[|\textit{main}|]{|\textit{prefix}|}{|\textit{dest}|}|
\end{tabular}
\end{center}
%
the destination file is determined by a pattern
depending on the current file:
To make this work, the current file must be called
`{\textit{prefix}\hspace{0.2em}\textit{suffix}}'
with \textit{prefix} matching precisely the argument.
Processing is then passed on to the file
`{\textit{dest}\hspace{0.2em}\textit{suffix}}'.
Surely, the same effect is achieved by
directly specifying the
argument `{\textit{dest}\hspace{0.2em}\textit{suffix}}'
in the first form.
However, that requires to set up a different file
for each child. With the alternative form of the command
all these files can have exactly the same content
which simplifies setting them up and maintaining them.

For example, the following file |draft.tex|
with a compilation flag |\version| as described in \secref{sec:flags}
compiles the main document as a draft:
%
\begin{center}
\begin{tabular}{l}
|\def\version{draft}|\\
|\input{childdoc.def}|\\
|\childdocforward{|\textit{main}|}|
\end{tabular}
\end{center}
%
Likewise, the following files |final|\textit{nn}|.tex|
compile the final version of the child document
|child|\textit{nn}|.tex|:
%
\begin{center}
\begin{tabular}{l}
|\def\version{final}|\\
|\input{childdoc.def}|\\
|\childdocforwardprefix{final}{child}|
\end{tabular}
\end{center}
%

Note that when several versions of a main file and/or of each child file
are to be generated, it may be convenient to set up a |Makefile| or
shell script to automatise the process.

%%%%%%%%%%%%%%%%%%%%%%%%%%%%%%%%%%%%%%%%%%%%%%%%%%%%%%%%%%%%%%%%%%%%%%%%%%%%%%%%
\subsection{Command Line Processing}
\label{sec:commandline}

The effect of redirection files can also be achieved by invoking
the \LaTeX{} compiler with a more elaborate command line.
Most conveniently this should be done as part
of a shell script or a |Makefile|.

When using \textsf{childdoc} in the main file, the following
command lines effectively perform a redirection
(note that depending on the shell being used,
backslashes may have to be doubled: `|\|' $\to$ `|\\|'):
%
\begin{center}
|... -jobname "|\textit{target}|" |\\|"|[\textit{flags}]%
|\input{childdoc.def}\childdocforward[|\textit{main}|]{|\textit{dest}|}"|
\end{center}
%
Here \textit{target} is the name of the output file,
\textit{main} is the name of the main file
and \textit{dest} is the name of the main or child file to be processed
(all filenames without extensions).
The optional argument \textit{main} can be omitted
if \textit{main} matches \textit{dest}.
Optionally, compilation \textit{flags} can be defined via |\def| commands.
This command line makes the \TeX{} engine believe
it is compiling the file \textit{target}
whose content is specified as the latter parameter.
The provided code then forwards the processing to
\textit{main} or \textit{dest} as described in \secref{sec:forward}.

%%%%%%%%%%%%%%%%%%%%%%%%%%%%%%%%%%%%%%%%%%%%%%%%%%%%%%%%%%%%%%%%%%%%%%%%%%%%%%%%
\subsection{Include by Input}
\label{sec:input}

Including child documents by |\include| has some restrictions by design.
Most notably, the content of a child document always occupies
its own set of pages; pages cannot be shared between child documents.
Usually, this behaviour makes perfect sense
because each child document contain an essential part of the document.
However, in some situations it may be desirable to compose
a document from a collection of parts
without having mandatory page breaks between then.
For this case, the package
provides a mechanism to include parts
by |\input| which can also be processed individually.
However, by construction this mechanism
requires manual handling of the content to be output.

%%%%%%%%%%%%%%%%%%%%%%%%%%%%%%%%%%%%%%%%
\DescribeMacro{\ifchilddocmanual}
The main file should be prepared as usual, see \secref{sec:include}.
However, the document body must make a distinction
between processing of an individual part and of the main document, e.g.:
%
\begin{center}
\begin{tabular}{l}
|\ifchilddocmanual|\\
|\input{\childdocname}|\\
|\||else|\\
\textit{document body with }|\input{|\textit{part}|}|\\
|\||fi|
\end{tabular}
\end{center}
%
The conditional |\ifchilddocmanual| is true whenever
a part to be included by |\input| is being compiled,
and the name of the part is stored in |\childdocname|.

%%%%%%%%%%%%%%%%%%%%%%%%%%%%%%%%%%%%%%%%
\DescribeMacro{\childdocby}
Each part to be included by |\input| should start with:
%
\begin{center}
\begin{tabular}{l}
|\input{childdoc.def}|\\
|\childdocby{|\textit{main}|}|\\
\end{tabular}
\end{center}
%
The directive |\childdocby| is similar to |\childdocof|
described in \secref{sec:include},
but the subsequent selection of content must be done manually.
To that end, both |\ifchilddoc| and |\ifchilddocmanual|
will be true upon processing of a part,
and the name of the part is stored in |\childdocname|.
Note that |\jobname| will be set to the filename of the current part
so that each part receives an individual |.aux| file
that does not interfere with the |.aux| file(s) of the main document.
This behaviour can be altered by the alternative form
|\childdocby[*]{|\textit{main}|}| (with a non-empty optional argument)
which uses the |.aux| file of the main document
by setting |\jobname| to \textit{main}.

%%%%%%%%%%%%%%%%%%%%%%%%%%%%%%%%%%%%%%%%%%%%%%%%%%%%%%%%%%%%%%%%%%%%%%%%%%%%%%%%
\subsection{Driver Development}
\label{sec:driver}

The \textsf{childdoc} mechanism can also be use for the development
of definition files such as \LaTeX{} styles or classes.
This case differs from the above setup with multiple parts
included by |\include| in that no |\includeonly| should be invoked.
This can be achieved by starting the include file
(before |\ProvidesPackage|) with:
%
\begin{center}
\begin{tabular}{l}
|\input{childdoc.def}|\\
|\childdocforward{|\textit{main}|}|\\
\end{tabular}
\end{center}
%
or alternatively with:
%
\begin{center}
\begin{tabular}{l}
|\input{childdoc.def}|\\
|\childdocby{|\textit{main}|}|\\
\end{tabular}
\end{center}
%
Both forms have slightly different effects as described above.
The main file is prepared as usual, see \secref{sec:include}.

%%%%%%%%%%%%%%%%%%%%%%%%%%%%%%%%%%%%%%%%%%%%%%%%%%%%%%%%%%%%%%%%%%%%%%%%%%%%%%%%
\subsection{Legacy Detection}
\label{sec:detection}

The directive |\childdocmain| in the main file can detect
whether the complete document or merely a child is to be compiled
even without using the directive |\childdocof|.
This method is deprecated because it is less robust
and there is no compelling reason to use it;
it is merely provided for backward compatibility
and it may be removed in future versions.

If the detection mechanism is to be used,
it is mandatory to correctly specify
the filename of the main file as the argument of |\childdocmain|:
%
\begin{center}
\begin{tabular}{l}
|\input{childdoc.def}|\\
|\childdocmain{|\textit{main}|}|\\
\end{tabular}
\end{center}
%
If |\jobname| does not match the argument \textit{main} of |\childdocmain|,
it is assumed that |\jobname| points to the child file to be compiled.
When using |\childdocmain| with the main file specified as argument,
it suffices to start a child file
with just |\input{|\textit{main}|}|
without loading of the package and using |\childdocof|.
If instead all processing is done
with the appropriate \textsf{childdoc} directives,
the argument of \textit{main} of |\childdocmain| can be empty.

An alternative version of the command line processing described
in \secref{sec:commandline} using the detection mechanism reads:
%
\begin{center}
|... -jobname "|\textit{target}|" "|[\textit{flags}]%
[|\def\jobname{|\textit{dest}|}|]|\input{|\textit{main}|}"|
\end{center}

%%%%%%%%%%%%%%%%%%%%%%%%%%%%%%%%%%%%%%%%%%%%%%%%%%%%%%%%%%%%%%%%%%%%%%%%%%%%%%%%
\subsection{Manual Code}
\label{sec:manual}

In case one cannot be certain whether the definitions file |childdoc.def|
is installed on the target \TeX{} distribution
and one prefers not to ship it,
it is conceivable to paste a few relevant commands into the sources.

To that end, drop all statements |\input{childdoc.def}|
and perform the replacements as outlined below.
Instead of |\childdocmain{|\textit{main}|}| add the following code
to the top of the main file:
%
\begin{center}
\begin{tabular}{l}
|\||ifdefined\childdocname\endinput\||fi\newif\ifchilddoc|\\
|\edef\childdocname{\scantokens\expandafter{\jobname\noexpand}}|\\
|\def\childdocmain{|\textit{main}|}\||ifx\childdocmain\childdocname\||else|\\
|\childdoctrue\includeonly{\childdocname}\let\jobname\childdocmain\||fi|\\
\end{tabular}
\end{center}
%
Instead of |\childdocof{|\textit{main}|}| just include the main file
at the top of each child file:
%
\begin{center}
|\input{|\textit{main}|}|
\end{center}
%
A simple redirection |\childdocforward{|\textit{dest}|}| is achieved by:
%
\begin{center}
|\def\jobname{|\textit{dest}|}\input{\jobname}|
\end{center}
%
The redirection with prefix
|\childdocforwardprefix[|\textit{prefix}|]{|\textit{dest}|}|
is accomplished by:
%
\begin{center}
\begin{tabular}{l}
|{\edef\jobname{\scantokens\expandafter{\jobname\noexpand}}|\\
|\def\redirectjob |\textit{prefix}|#1~~~{\gdef\jobname{|\textit{dest}|#1}}|\\
|\expandafter\redirectjob\jobname~~~}\input{\jobname}|
\end{tabular}
\end{center}

In an alternative approach,
child documents can be compiled by a specific command line
without additional code or specific definitions:
%
\begin{center}
|... -jobname "|\textit{target}|" "|[\textit{flags}]%
|\includeonly{|\textit{dest}|}\input{|\textit{main}|}"|
\end{center}
%

%%%%%%%%%%%%%%%%%%%%%%%%%%%%%%%%%%%%%%%%%%%%%%%%%%%%%%%%%%%%%%%%%%%%%%%%%%%%%%%%
%%%%%%%%%%%%%%%%%%%%%%%%%%%%%%%%%%%%%%%%%%%%%%%%%%%%%%%%%%%%%%%%%%%%%%%%%%%%%%%%
\section{Information}

%%%%%%%%%%%%%%%%%%%%%%%%%%%%%%%%%%%%%%%%%%%%%%%%%%%%%%%%%%%%%%%%%%%%%%%%%%%%%%%%
\subsection{Copyright}

Copyright \copyright{} 2017--2018 Niklas Beisert

This work may be distributed and/or modified under the
conditions of the \LaTeX{} Project Public License, either version 1.3
of this license or (at your option) any later version.
The latest version of this license is in
  \url{http://www.latex-project.org/lppl.txt}
and version 1.3 or later is part of all distributions of \LaTeX{}
version 2005/12/01 or later.

This work has the LPPL maintenance status `maintained'.

The Current Maintainer of this work is Niklas Beisert.

This work consists of the files |README.txt|, |childdoc.ins| and |childdoc.dtx|
as well as the derived files |childdoc.def|, |cdocsamp.tex|
with |cdocsch1.tex|, |cdocsch2.tex|, |cdocspt3.tex|, |cdocspt4.tex|,
|cdocsdrf.tex|, |cdocsfn1.tex|, |cdocsfn2.tex|
as well as |childdoc.pdf|.

%%%%%%%%%%%%%%%%%%%%%%%%%%%%%%%%%%%%%%%%%%%%%%%%%%%%%%%%%%%%%%%%%%%%%%%%%%%%%%%%
\subsection{Files and Installation}

The package consists of the files:
%
\begin{center}
\begin{tabular}{ll}
    |README.txt|   & readme file \\
    |childdoc.ins| & installation file \\
    |childdoc.dtx| & source file \\
    |childdoc.def| & definition file \\
    |cdocsamp.tex| & sample main file \\
    |cdocsch1.tex| & sample include file \\
    |cdocsch2.tex| & sample include file \\
    |cdocspt3.tex| & sample part file \\
    |cdocspt4.tex| & sample part file \\
    |cdocsdrf.tex| & sample redirection file \\
    |cdocsfn1.tex| & sample redirection file \\
    |cdocsfn2.tex| & sample redirection file \\
    |childdoc.pdf| & manual
\end{tabular}
\end{center}
%
The distribution consists of the files
|README.txt|, |childdoc.ins| and |childdoc.dtx|.
%
\begin{itemize}
\item
Run (pdf)\LaTeX{} on |childdoc.dtx|
to compile the manual |childdoc.pdf| (this file).
\item
Run \LaTeX{} on |childdoc.ins| to create the definitions file |childdoc.def|
and the sample |cdocsamp.tex| with include files
|cdocsch1.tex|, |cdocsch2.tex|, |cdocspt3.tex|, |cdocspt4.tex|,
|cdocsdrf.tex|, |cdocsfn1.tex|, |cdocsfn2.tex|.
Then copy the file |childdoc.def| to an appropriate directory of your \LaTeX{}
distribution, e.g.\ \textit{texmf-root}|/tex/latex/childdoc|.
\end{itemize}

%%%%%%%%%%%%%%%%%%%%%%%%%%%%%%%%%%%%%%%%%%%%%%%%%%%%%%%%%%%%%%%%%%%%%%%%%%%%%%%%
\subsection{Related CTAN Packages}

There are several other packages which offer a similar functionality:
%
\begin{itemize}
\item
The packages
\href{http://ctan.org/pkg/docmute}{\textsf{docmute}},
\href{http://ctan.org/pkg/includex}{\textsf{includex}} and
\href{http://ctan.org/pkg/standalone}{\textsf{standalone}}
provide commands to include only the document body of
a child file thus allowing both files to be compiled individually.
\item
The packages \href{http://ctan.org/pkg/subdocs}{\textsf{subdocs}}
and \href{http://ctan.org/pkg/subfiles}{\textsf{subfiles}}
provide structures in which the main and child documents can be
encapsulated and allowing them to be compiled individually.
The inclusion mechanism is different from the conventional |\include|.
\item
The package \href{http://ctan.org/pkg/combine}{\textsf{combine}}
is an elaborate solution to combine several documents into one.
\end{itemize}
%
See also the CTAN topic \href{http://ctan.org/topic/subdocs}{\textsf{subdocs}}
for further related packages.
The present package differs from the above solutions in that
a document structure constructed with the conventional |\include| mechanism
just needs two extra commands at the top of every file
such that all constituent files can be compiled individually.

%%%%%%%%%%%%%%%%%%%%%%%%%%%%%%%%%%%%%%%%%%%%%%%%%%%%%%%%%%%%%%%%%%%%%%%%%%%%%%%%
%\subsection{Feature Suggestions}
%
%The following is a list of features which may be useful for future
%versions of this package:
%%
%\begin{itemize}
%\item
%\ldots
%\end{itemize}

%%%%%%%%%%%%%%%%%%%%%%%%%%%%%%%%%%%%%%%%%%%%%%%%%%%%%%%%%%%%%%%%%%%%%%%%%%%%%%%%
\subsection{Revision History}

%%%%%%%%%%%%%%%%%%%%%%%%%%%%%%%%%%%%%%%%
\paragraph{v2.0:} 2018/12/30

\begin{itemize}
\item
immediate forward processing
\item
added |\childdocby| mechanism
\item
manual restructured
\end{itemize}

%%%%%%%%%%%%%%%%%%%%%%%%%%%%%%%%%%%%%%%%
\paragraph{v1.6:} 2018/01/17

\begin{itemize}
\item
application for development of include files
\item
corrections to manual
\end{itemize}

%%%%%%%%%%%%%%%%%%%%%%%%%%%%%%%%%%%%%%%%
\paragraph{v1.5:} 2017/05/21

\begin{itemize}
\item
more complete structuring introduced
\item
|\childdocof| introduced
\item
|\childdoc| renamed to |\childdocmain|
\item
|\childredirect| renamed to |\childdocforward| and |\childdocforwardprefix|
and functionality expanded
\end{itemize}

%%%%%%%%%%%%%%%%%%%%%%%%%%%%%%%%%%%%%%%%
\paragraph{v1.0:} 2017/04/27

\begin{itemize}
\item
manual and install package
\item
first version published on CTAN
\end{itemize}

%%%%%%%%%%%%%%%%%%%%%%%%%%%%%%%%%%%%%%%%
\paragraph{v0.6:} 2017/04/26

\begin{itemize}
\item
redirection mechanism added
\end{itemize}

%%%%%%%%%%%%%%%%%%%%%%%%%%%%%%%%%%%%%%%%
\paragraph{v0.5:} 2017/04/26

\begin{itemize}
\item
functionality in definition file
\end{itemize}


%%%%%%%%%%%%%%%%%%%%%%%%%%%%%%%%%%%%%%%%%%%%%%%%%%%%%%%%%%%%%%%%%%%%%%%%%%%%%%%%
%%%%%%%%%%%%%%%%%%%%%%%%%%%%%%%%%%%%%%%%%%%%%%%%%%%%%%%%%%%%%%%%%%%%%%%%%%%%%%%%
%%%%%%%%%%%%%%%%%%%%%%%%%%%%%%%%%%%%%%%%%%%%%%%%%%%%%%%%%%%%%%%%%%%%%%%%%%%%%%%%
\appendix

\settowidth\MacroIndent{\rmfamily\scriptsize 000\ }

 \DocInput{childdoc.dtx}

\end{document}
%</driver>
% \fi
%
% %%%%%%%%%%%%%%%%%%%%%%%%%%%%%%%%%%%%%%%%%%%%%%%%%%%%%%%%%%%%%%%%%%%%%%%%%%%%%%
% %%%%%%%%%%%%%%%%%%%%%%%%%%%%%%%%%%%%%%%%%%%%%%%%%%%%%%%%%%%%%%%%%%%%%%%%%%%%%%
% \section{Sample}
%\iffalse
%<*samplemain>
%\fi
%
% The following presents a sample document
% with two chapters, two parts, a title page,
% a compile flag as well as three forwarding files to set the flag.
% It consists of eight |.tex| files:
% \begin{center}
% \begin{tabular}{ll}
% |cdocsamp.tex|&main file\\
% |cdocsch1.tex|&include file for chapter 1\\
% |cdocsch2.tex|&include file for chapter 2\\
% |cdocspt3.tex|&include file for part 3\\
% |cdocspt4.tex|&include file for part 4\\
% |cdocsdrf.tex|&forwarding file for main file in draft mode\\
% |cdocsfi1.tex|&forwarding file for final version of chapter 1\\
% |cdocsfi2.tex|&forwarding file for final version of chapter 2\\
% \end{tabular}
% \end{center}
% Each of the eight files can be compiled directly by the \LaTeX{} compiler.
%
% %%%%%%%%%%%%%%%%%%%%%%%%%%%%%%%%%%%%%%
% \paragraph{Main File.}
%
% The main file is called |cdocsamp.tex|.
%
% Load the \textsf{childdoc} definitions and
% declare the filename for the main document:
%    \begin{macrocode}
\input{childdoc.def}
\childdocmain{}
%    \end{macrocode}

% Optional override for |\version| flag:
%    \begin{macrocode}
%%\ifchilddoc\else\providecommand{\version}{draft}\fi
%    \end{macrocode}

% Define the default values for the |\version| flag
% (|final| for the main file and |draft| for childs):
%    \begin{macrocode}
\ifchilddoc
\providecommand{\version}{draft}
\else
\providecommand{\version}{final}
\fi
%    \end{macrocode}

% Load the standard document class:
%    \begin{macrocode}
\documentclass[12pt]{article}
%    \end{macrocode}

% Start the document body:
%    \begin{macrocode}
\begin{document}
%    \end{macrocode}

% Declare a title page.
% Print title, part of document being processed and version flag:
%    \begin{macrocode}
\addtocounter{page}{-1}
\begin{center}
{\LARGE\bfseries{}childdoc example\par}
\vspace{1cm}
\ifchilddoc
\ifchilddocmanual part\else chapter\fi:
`\childdocname' of `\childdocjob'\par
\else
main document: `\childdocjob'\par
\fi
version: \version\par
\end{center}
\newpage
%    \end{macrocode}

% Manually include selected file,
% otherwise process as usual:
%    \begin{macrocode}
\ifchilddocmanual
\section*{part `\childdocname'}
\input{\childdocname}
\else
%    \end{macrocode}

% Include the two chapters:
%    \begin{macrocode}
\include{cdocsch1}
\include{cdocsch2}
%    \end{macrocode}

% Include the two parts unless only chapters should be displayed:
%    \begin{macrocode}
\ifchilddoc\else
\section{part three}
\input{cdocspt3}
\section{part four}
\input{cdocspt4}
\fi
%    \end{macrocode}

% Process as usual until here:
%    \begin{macrocode}
\fi
%    \end{macrocode}

% End of document body:
%    \begin{macrocode}
\end{document}
%    \end{macrocode}
%\iffalse
%</samplemain>
%\fi
%
% %%%%%%%%%%%%%%%%%%%%%%%%%%%%%%%%%%%%%%
% \paragraph{Chapter Include Files.}
%
% The include files are called |cdocsch1.tex| and |cdocsch2.tex|.
%
%\iffalse
%<*samplechap1|samplechap2>
%\fi

% Optional override for |\version| flag:
%    \begin{macrocode}
%%\providecommand{\version}{final}
%    \end{macrocode}

% Include the main document:
%    \begin{macrocode}
\input{childdoc.def}
\childdocof{cdocsamp}
%    \end{macrocode}

%\iffalse
%</samplechap1|samplechap2>
%\fi
%
%\iffalse
%<*samplechap1>
%\fi
% Some text for chapter 1:
%    \begin{macrocode}
\section{one}
some text in chapter one
%    \end{macrocode}

%\iffalse
%</samplechap1>
%\fi
% Some text for chapter 2:
%\iffalse
%<*samplechap2>
%\fi
%    \begin{macrocode}
\section{two}
more text in chapter two
%    \end{macrocode}

%\iffalse
%</samplechap2>
%\fi
%
% %%%%%%%%%%%%%%%%%%%%%%%%%%%%%%%%%%%%%%
% \paragraph{Part Include Files.}
%
% The include files are called |cdocspt3.tex| and |cdocspt4.tex|.
%
%\iffalse
%<*samplepart3|samplepart4>
%\fi

% Optional override for |\version| flag:
%    \begin{macrocode}
%%\providecommand{\version}{final}
%    \end{macrocode}

% Include the main document:
%    \begin{macrocode}
\input{childdoc.def}
\childdocby{cdocsamp}
%    \end{macrocode}

%\iffalse
%</samplepart3|samplepart4>
%\fi
%
%\iffalse
%<*samplepart3>
%\fi
% Some text for part 3:
%    \begin{macrocode}
some text in part three
%    \end{macrocode}

%\iffalse
%</samplepart3>
%\fi
% Some text for part 4:
%\iffalse
%<*samplepart4>
%\fi
%    \begin{macrocode}
more text in part four
%    \end{macrocode}

%\iffalse
%</samplepart4>
%\fi
%
% %%%%%%%%%%%%%%%%%%%%%%%%%%%%%%%%%%%%%%
% \paragraph{Forwarding for a Complete Draft.}
%
% The following forwarding file |cdocsdrf.tex|
% compiles the main document in draft mode:
%\iffalse
%<*sampledraft>
%\fi
%    \begin{macrocode}
\def\version{draft}
\input{childdoc.def}
\childdocforward{cdocsamp}
%    \end{macrocode}

%\iffalse
%</sampledraft>
%\fi
%
% %%%%%%%%%%%%%%%%%%%%%%%%%%%%%%%%%%%%%%
% \paragraph{Forwarding for Final Version of the Chapters.}
%
% The following forwarding files |cdocsfn1.tex| and |cdocsfn2.tex|
% (with identical content)
% compile the final versions of the child documents
% |cdocsch1.tex| and |cdocsch2.tex|, respectively:
%\iffalse
%<*samplefinal>
%\fi
%    \begin{macrocode}
\def\version{final}
\input{childdoc.def}
\childdocforwardprefix[cdocsamp]{cdocsfn}{cdocsch}
%    \end{macrocode}

%\iffalse
%</samplefinal>
%\fi
%
% %%%%%%%%%%%%%%%%%%%%%%%%%%%%%%%%%%%%%%
% \paragraph{Command Line Processing.}
%
% The following three command lines generate the output files
% |cdocscld|, |cdocscl1| and |cdocscl2|
% which should be identical to
% |cdocsdrf|, |cdocsch1| and |cdocsfn2|, respectively:
% \begin{center}
% \begin{tabular}{l}
% |latex -jobname cdocscld \|\\
% |  "\def\version{draft}\input{childdoc.def}\childdocforward{cdocsamp}"|\\
% |latex -jobname cdocscl1 \|\\
% |  "\input{childdoc.def}\childdocforward[cdocsamp]{cdocsch1}"|\\
% |latex -jobname cdocscl2 \|\\
% |  "\def\version{final}\input{childdoc.def}\childdocforward{cdocsch2}"|
% \end{tabular}
% \end{center}
% Note that the trailing backslash on each first line
% merely continues the input to the second line
% (for convenient cut ant paste).
% Furthermore, the command |latex| can be replaced by any
% of its alternative versions such as |pdflatex|.
%
% %%%%%%%%%%%%%%%%%%%%%%%%%%%%%%%%%%%%%%%%%%%%%%%%%%%%%%%%%%%%%%%%%%%%%%%%%%%%%%
% %%%%%%%%%%%%%%%%%%%%%%%%%%%%%%%%%%%%%%%%%%%%%%%%%%%%%%%%%%%%%%%%%%%%%%%%%%%%%%
% \section{Implementation}
%\iffalse
%<*package>
%\fi
%
% This section describes the definitions file |childdoc.def|.

% The definitions cannot be loaded using |\usepackage| or |\RequirePackage|
% which has a mechanism to prevent loading a style file more than once.
% When loading the definitions by means of |\input|
% multiple instances have to be prevented manually:
%\iffalse
%This code needs to be before the `\ProvidesFile' directive
%which is defined at the beginning of this file.
%Therefore it is also placed there and commented out here.
%</package>
%<*discard>
%\fi
%    \begin{macrocode}
\ifdefined\childdocmain\endinput\fi
%    \end{macrocode}
%\iffalse
%</discard>
%<*package>
%\fi
%
% \macro{\ifchilddoc}
% \macro{\ifchilddocmanual}
% The conditional |\ifchilddoc| tells whether a
% child (true) or main (false) document is being compiled.
% The conditional |\ifchilddocmanual| tells whether
% the |\includeonly| mechanism is used (false) or
% the selection of child files must be performed manually (true).
% The definitions initialise to false:
%    \begin{macrocode}
\newif\ifchilddoc
\newif\ifchilddocmanual
%    \end{macrocode}

% \macro{\childdocname}
% \macro{\childdocjob}
% The macro |\childdocname| stores the name of the main document
% to be compiled. The macro |\childdocjob| stores the name of
% the document on which the \LaTeX{} compiler was originally invoked.
% The content of |\jobname| cannot be compared
% to filenames specified in the source due to different catcodes.
% The following code rescans |\jobname|, stores the result
% in |\childdocname| and saves a copy in |\childdocjob|:
%    \begin{macrocode}
\edef\childdocname{\scantokens\expandafter{\jobname\noexpand}}
\let\childdocjob\childdocname
%    \end{macrocode}

% \macro{\childdocdisable}
% The macro |\childdocdisable| prevents the main file
% from being processed more than once.
% At this stage, the main document command |\childdocmain|
% is assumed to be called once again where it should do nothing.
% Any subsequent call to it should prevent
% a secondary processing of the main document
% It overwrites the forwarding commands
% |\childdocof| and |\childdocforward|
% with empty macros to prevent further inclusions of the main document:
%    \begin{macrocode}
\newcommand{\childdocdisable}
{
  \renewcommand{\childdocmain}[1]{\renewcommand{\childdocmain}[1]{\endinput}}
  \renewcommand{\childdocof}[1]{}
  \renewcommand{\childdocby}[2][]{}
  \renewcommand{\childdocforward}[2][]{}
  \renewcommand{\childdocdisable}{}
}
%    \end{macrocode}

% \macro{\childdocmain}
% The macro |\childdocmain| is to be called at the top of the main file
% with nothing or the main filename (without extension) as argument.
% First, it breaks loops.
% If the argument is not empty and does not match |\childdocname|
% (which is set by the first inclusion of |childdoc.def|),
% |\ifchilddoc| is set to true, |\includeonly| is applied to the child file
% and |\jobname| is set to the main file
% (for proper handling of |.aux| files):
%    \begin{macrocode}
\newcommand{\childdocmain}[1]
{
  \childdocdisable\childdocmain{}
  \if?#1?\else
    \begingroup
      \def\childdoctmp{#1}
      \ifx\childdoctmp\childdocname
        \def\childdoctmp{}
      \else
        \def\childdoctmp
        {
          \childdoctrue
          \includeonly{\childdocname}
          \def\childdocjob{#1}
          \def\jobname{#1}
        }
      \fi
      \expandafter
    \endgroup
    \childdoctmp
  \fi
}
%    \end{macrocode}

% \macro{\childdocof}
% The command |\childdocof| redirects
% compilation to the main file |#1|.
%    \begin{macrocode}
\newcommand{\childdocof}[1]
{
  \childdocdisable
  \childdoctrue
  \includeonly{\childdocname}
  \def\jobname{#1}
  \def\childdocjob{#1}
  \input{#1}
}
%    \end{macrocode}

% \macro{\childdocby}
% The command |\childdocby| ....
%    \begin{macrocode}
\newcommand{\childdocby}[2][]
{
  \childdocdisable
  \childdoctrue
  \childdocmanualtrue
  \if?#1?\else
    \def\jobname{#2}
  \fi
  \def\childdocjob{#2}
  \input{#2}
  \endinput
}
%    \end{macrocode}

% \macro{\childdocforward}
% The command |\childdocforward| redirects
% compilation to the main file or
% (if the optional argument is given) a child file.
% Parameters are set as if the main file
% or a child file starting with |\childdocof| was compiled.
% Then compilation is handed over to the main file:
%    \begin{macrocode}
\newcommand{\childdocforward}[2][]
{
  \begingroup
    \if?#1?
      \def\childdoctmp
      {
        \def\childdocname{#2}
        \def\childdocjob{#2}
        \def\jobname{#2}
        \input{#2}
        \endinput
      }
    \else
      \def\childdoctmp
      {
        \childdocdisable
        \def\childdocname{#2}
        \childdoctrue
        \includeonly{#2}
        \def\childdocjob{#1}
        \def\jobname{#1}
        \input{#1}
        \endinput
      }
    \fi
    \expandafter
  \endgroup
  \childdoctmp
}
%    \end{macrocode}

% \macro{\childdocforwardprefix}
% The command |\childdocforwardprefix| redirects
% compilation to the main or a child file by means of a pattern.
% The prefix |#1| in the current filename is replaced by |#2|
% and the suffix of the current filename is kept
% (it is assumed that the filename does not contain the substring `|~~~|'
% which is used as a delimiter).
% Compilation is handed over to the new file by |\childdocforward|:
%    \begin{macrocode}
\newcommand{\childdocforwardprefix}[3][]
{
  \begingroup
    \def\childdocextract #2##1~~~{\def\childdoctmp{\childdocforward[#1]{#3##1}}}
    \expandafter\childdocextract\childdocname~~~
    \expandafter
  \endgroup
  \childdoctmp
}
%    \end{macrocode}

% \macro{\childdoc}
% The deprecated macro |\childdoc| is a legacy version of |\childdocmain|:
%    \begin{macrocode}
\newcommand{\childdoc}{\childdocmain}
%    \end{macrocode}

% \macro{\childdocredirect}
% The deprecated macro |\childdocredirect| is a legacy version
% of |\childdocforward| and |\childdocforwardprefix|:
%    \begin{macrocode}
\newcommand{\childdocredirect}[2][]
{
  \begingroup
    \if?#1?
      \def\childdoctmp{\childdocforward{#2}}
    \else
      \def\childdoctmp{\childdocforwardprefix{#1}{#2}}
    \fi
    \expandafter
  \endgroup
  \childdoctmp
}
%    \end{macrocode}

%\iffalse
%</package>
%\fi
%
\endinput
\childdocforward{cdocsch2}"|
% \end{tabular}
% \end{center}
% Note that the trailing backslash on each first line
% merely continues the input to the second line
% (for convenient cut ant paste).
% Furthermore, the command |latex| can be replaced by any
% of its alternative versions such as |pdflatex|.
%
% %%%%%%%%%%%%%%%%%%%%%%%%%%%%%%%%%%%%%%%%%%%%%%%%%%%%%%%%%%%%%%%%%%%%%%%%%%%%%%
% %%%%%%%%%%%%%%%%%%%%%%%%%%%%%%%%%%%%%%%%%%%%%%%%%%%%%%%%%%%%%%%%%%%%%%%%%%%%%%
% \section{Implementation}
%\iffalse
%<*package>
%\fi
%
% This section describes the definitions file |childdoc.def|.

% The definitions cannot be loaded using |\usepackage| or |\RequirePackage|
% which has a mechanism to prevent loading a style file more than once.
% When loading the definitions by means of |\input|
% multiple instances have to be prevented manually:
%\iffalse
%This code needs to be before the `\ProvidesFile' directive
%which is defined at the beginning of this file.
%Therefore it is also placed there and commented out here.
%</package>
%<*discard>
%\fi
%    \begin{macrocode}
\ifdefined\childdocmain\endinput\fi
%    \end{macrocode}
%\iffalse
%</discard>
%<*package>
%\fi
%
% \macro{\ifchilddoc}
% \macro{\ifchilddocmanual}
% The conditional |\ifchilddoc| tells whether a
% child (true) or main (false) document is being compiled.
% The conditional |\ifchilddocmanual| tells whether
% the |\includeonly| mechanism is used (false) or
% the selection of child files must be performed manually (true).
% The definitions initialise to false:
%    \begin{macrocode}
\newif\ifchilddoc
\newif\ifchilddocmanual
%    \end{macrocode}

% \macro{\childdocname}
% \macro{\childdocjob}
% The macro |\childdocname| stores the name of the main document
% to be compiled. The macro |\childdocjob| stores the name of
% the document on which the \LaTeX{} compiler was originally invoked.
% The content of |\jobname| cannot be compared
% to filenames specified in the source due to different catcodes.
% The following code rescans |\jobname|, stores the result
% in |\childdocname| and saves a copy in |\childdocjob|:
%    \begin{macrocode}
\edef\childdocname{\scantokens\expandafter{\jobname\noexpand}}
\let\childdocjob\childdocname
%    \end{macrocode}

% \macro{\childdocdisable}
% The macro |\childdocdisable| prevents the main file
% from being processed more than once.
% At this stage, the main document command |\childdocmain|
% is assumed to be called once again where it should do nothing.
% Any subsequent call to it should prevent
% a secondary processing of the main document
% It overwrites the forwarding commands
% |\childdocof| and |\childdocforward|
% with empty macros to prevent further inclusions of the main document:
%    \begin{macrocode}
\newcommand{\childdocdisable}
{
  \renewcommand{\childdocmain}[1]{\renewcommand{\childdocmain}[1]{\endinput}}
  \renewcommand{\childdocof}[1]{}
  \renewcommand{\childdocby}[2][]{}
  \renewcommand{\childdocforward}[2][]{}
  \renewcommand{\childdocdisable}{}
}
%    \end{macrocode}

% \macro{\childdocmain}
% The macro |\childdocmain| is to be called at the top of the main file
% with nothing or the main filename (without extension) as argument.
% First, it breaks loops.
% If the argument is not empty and does not match |\childdocname|
% (which is set by the first inclusion of |childdoc.def|),
% |\ifchilddoc| is set to true, |\includeonly| is applied to the child file
% and |\jobname| is set to the main file
% (for proper handling of |.aux| files):
%    \begin{macrocode}
\newcommand{\childdocmain}[1]
{
  \childdocdisable\childdocmain{}
  \if?#1?\else
    \begingroup
      \def\childdoctmp{#1}
      \ifx\childdoctmp\childdocname
        \def\childdoctmp{}
      \else
        \def\childdoctmp
        {
          \childdoctrue
          \includeonly{\childdocname}
          \def\childdocjob{#1}
          \def\jobname{#1}
        }
      \fi
      \expandafter
    \endgroup
    \childdoctmp
  \fi
}
%    \end{macrocode}

% \macro{\childdocof}
% The command |\childdocof| redirects
% compilation to the main file |#1|.
%    \begin{macrocode}
\newcommand{\childdocof}[1]
{
  \childdocdisable
  \childdoctrue
  \includeonly{\childdocname}
  \def\jobname{#1}
  \def\childdocjob{#1}
  \input{#1}
}
%    \end{macrocode}

% \macro{\childdocby}
% The command |\childdocby| ....
%    \begin{macrocode}
\newcommand{\childdocby}[2][]
{
  \childdocdisable
  \childdoctrue
  \childdocmanualtrue
  \if?#1?\else
    \def\jobname{#2}
  \fi
  \def\childdocjob{#2}
  \input{#2}
  \endinput
}
%    \end{macrocode}

% \macro{\childdocforward}
% The command |\childdocforward| redirects
% compilation to the main file or
% (if the optional argument is given) a child file.
% Parameters are set as if the main file
% or a child file starting with |\childdocof| was compiled.
% Then compilation is handed over to the main file:
%    \begin{macrocode}
\newcommand{\childdocforward}[2][]
{
  \begingroup
    \if?#1?
      \def\childdoctmp
      {
        \def\childdocname{#2}
        \def\childdocjob{#2}
        \def\jobname{#2}
        \input{#2}
        \endinput
      }
    \else
      \def\childdoctmp
      {
        \childdocdisable
        \def\childdocname{#2}
        \childdoctrue
        \includeonly{#2}
        \def\childdocjob{#1}
        \def\jobname{#1}
        \input{#1}
        \endinput
      }
    \fi
    \expandafter
  \endgroup
  \childdoctmp
}
%    \end{macrocode}

% \macro{\childdocforwardprefix}
% The command |\childdocforwardprefix| redirects
% compilation to the main or a child file by means of a pattern.
% The prefix |#1| in the current filename is replaced by |#2|
% and the suffix of the current filename is kept
% (it is assumed that the filename does not contain the substring `|~~~|'
% which is used as a delimiter).
% Compilation is handed over to the new file by |\childdocforward|:
%    \begin{macrocode}
\newcommand{\childdocforwardprefix}[3][]
{
  \begingroup
    \def\childdocextract #2##1~~~{\def\childdoctmp{\childdocforward[#1]{#3##1}}}
    \expandafter\childdocextract\childdocname~~~
    \expandafter
  \endgroup
  \childdoctmp
}
%    \end{macrocode}

% \macro{\childdoc}
% The deprecated macro |\childdoc| is a legacy version of |\childdocmain|:
%    \begin{macrocode}
\newcommand{\childdoc}{\childdocmain}
%    \end{macrocode}

% \macro{\childdocredirect}
% The deprecated macro |\childdocredirect| is a legacy version
% of |\childdocforward| and |\childdocforwardprefix|:
%    \begin{macrocode}
\newcommand{\childdocredirect}[2][]
{
  \begingroup
    \if?#1?
      \def\childdoctmp{\childdocforward{#2}}
    \else
      \def\childdoctmp{\childdocforwardprefix{#1}{#2}}
    \fi
    \expandafter
  \endgroup
  \childdoctmp
}
%    \end{macrocode}

%\iffalse
%</package>
%\fi
%
\endinput
\childdocforward[|\textit{main}|]{|\textit{dest}|}"|
\end{center}
%
Here \textit{target} is the name of the output file,
\textit{main} is the name of the main file
and \textit{dest} is the name of the main or child file to be processed
(all filenames without extensions).
The optional argument \textit{main} can be omitted
if \textit{main} matches \textit{dest}.
Optionally, compilation \textit{flags} can be defined via |\def| commands.
This command line makes the \TeX{} engine believe
it is compiling the file \textit{target}
whose content is specified as the latter parameter.
The provided code then forwards the processing to
\textit{main} or \textit{dest} as described in \secref{sec:forward}.

%%%%%%%%%%%%%%%%%%%%%%%%%%%%%%%%%%%%%%%%%%%%%%%%%%%%%%%%%%%%%%%%%%%%%%%%%%%%%%%%
\subsection{Include by Input}
\label{sec:input}

Including child documents by |\include| has some restrictions by design.
Most notably, the content of a child document always occupies
its own set of pages; pages cannot be shared between child documents.
Usually, this behaviour makes perfect sense
because each child document contain an essential part of the document.
However, in some situations it may be desirable to compose
a document from a collection of parts
without having mandatory page breaks between then.
For this case, the package
provides a mechanism to include parts
by |\input| which can also be processed individually.
However, by construction this mechanism
requires manual handling of the content to be output.

%%%%%%%%%%%%%%%%%%%%%%%%%%%%%%%%%%%%%%%%
\DescribeMacro{\ifchilddocmanual}
The main file should be prepared as usual, see \secref{sec:include}.
However, the document body must make a distinction
between processing of an individual part and of the main document, e.g.:
%
\begin{center}
\begin{tabular}{l}
|\ifchilddocmanual|\\
|\input{\childdocname}|\\
|\||else|\\
\textit{document body with }|\input{|\textit{part}|}|\\
|\||fi|
\end{tabular}
\end{center}
%
The conditional |\ifchilddocmanual| is true whenever
a part to be included by |\input| is being compiled,
and the name of the part is stored in |\childdocname|.

%%%%%%%%%%%%%%%%%%%%%%%%%%%%%%%%%%%%%%%%
\DescribeMacro{\childdocby}
Each part to be included by |\input| should start with:
%
\begin{center}
\begin{tabular}{l}
|% \iffalse
%
% childdoc.dtx Copyright (C) 2017-2018 Niklas Beisert
%
% This work may be distributed and/or modified under the
% conditions of the LaTeX Project Public License, either version 1.3
% of this license or (at your option) any later version.
% The latest version of this license is in
%   http://www.latex-project.org/lppl.txt
% and version 1.3 or later is part of all distributions of LaTeX
% version 2005/12/01 or later.
%
% This work has the LPPL maintenance status `maintained'.
%
% The Current Maintainer of this work is Niklas Beisert.
%
% This work consists of the files childdoc.dtx and childdoc.ins
% and the derived files childdoc.def and cdocsamp.tex with
% cdocsch1.tex, cdocsch2.tex, cdocsdrf.tex, cdocsfn1.tex, cdocsfn2.tex.
%
%<package>\ifdefined\childdocmain\endinput\fi
%<package>\ProvidesFile{childdoc.def}[2018/12/30 v2.0 child document driver]
%<samplemain>\ProvidesFile{cdocsamp.tex}[2018/12/30 v2.0 sample for childdoc]
%<*driver>
%\ProvidesFile{childdoc.drv}[2018/12/30 v2.0 childdoc reference manual file]
\PassOptionsToClass{10pt,a4paper}{article}
\documentclass{ltxdoc}

\usepackage[margin=35mm]{geometry}
\usepackage{hyperref}
\usepackage{hyperxmp}
\usepackage[usenames]{color}

\hypersetup{colorlinks=true}
\hypersetup{pdfstartview=FitH}
\hypersetup{pdfpagemode=UseNone}
\hypersetup{pdfsource={}}
\hypersetup{pdflang={en-UK}}
\hypersetup{pdfcopyright={Copyright 2017-2018 Niklas Beisert.
  This work may be distributed and/or modified under the
  conditions of the LaTeX Project Public License, either version 1.3
  of this license or (at your option) any later version.}}
\hypersetup{pdflicenseurl={http://www.latex-project.org/lppl.txt}}
\hypersetup{pdfcontactaddress={ETH Zurich, ITP, HIT K,
  Wolfgang-Pauli-Strasse 27}}
\hypersetup{pdfcontactpostcode={8093}}
\hypersetup{pdfcontactcity={Zurich}}
\hypersetup{pdfcontactcountry={Switzerland}}
\hypersetup{pdfcontactemail={nbeisert@itp.phys.ethz.ch}}
\hypersetup{pdfcontacturl={http://people.phys.ethz.ch/\xmptilde nbeisert/}}

\newcommand{\secref}[1]{\hyperref[#1]{section \ref*{#1}}}

\parskip1ex
\parindent0pt
\let\olditemize\itemize
\def\itemize{\olditemize\parskip0pt}

\begin{document}

\title{The \textsf{childdoc} Package}
\hypersetup{pdftitle={The childdoc Package}}
\author{Niklas Beisert\\[2ex]
  Institut f\"ur Theoretische Physik\\
  Eidgen\"ossische Technische Hochschule Z\"urich\\
  Wolfgang-Pauli-Strasse 27, 8093 Z\"urich, Switzerland\\[1ex]
  \href{mailto:nbeisert@itp.phys.ethz.ch}
  {\texttt{nbeisert@itp.phys.ethz.ch}}}
\hypersetup{pdfauthor={Niklas Beisert}}
\hypersetup{pdfsubject={Manual for the LaTeX2e Package childdoc}}
\date{30 December 2018, \textsf{v2.0}}
\maketitle

\begin{abstract}\noindent
\textsf{childdoc} is a \LaTeXe{} package
that enables the direct compilation
of document sections included by |\include|
to individual files.
\end{abstract}

\begingroup
\parskip0ex
\tableofcontents
\endgroup

%%%%%%%%%%%%%%%%%%%%%%%%%%%%%%%%%%%%%%%%%%%%%%%%%%%%%%%%%%%%%%%%%%%%%%%%%%%%%%%%
%%%%%%%%%%%%%%%%%%%%%%%%%%%%%%%%%%%%%%%%%%%%%%%%%%%%%%%%%%%%%%%%%%%%%%%%%%%%%%%%
\section{Introduction}

\LaTeX{} provides a mechanism to structure a large document (such as a book)
into a main file and several child files (containing the chapters)
using the |\include| command.
This mechanism is beneficial for documents
which span hundreds of pages in order to
make the source file(s) more manageable.
Moreover, compilation can be restricted to
selected child files by means of the |\includeonly| command.
The latter feature can be used to reduce the compilation time while editing
(this was significantly more useful in the earlier days of \LaTeX{})
or to generate a smaller document which is easier to navigate.
Another application of |\includeonly| is to generate
documents consisting of selected parts of the complete document.

However, there are a few drawbacks of the plain |\include| mechanism:
\begin{itemize}
\item
The child files cannot be compiled on their own,
they can only be compiled via the main file.
A naive editing environment
(such as a text editor with an option
to have the current file processed by \LaTeX)
may require one to switch to the main file before compiling;
attempting to compile the child file produces errors.
\item
The main file must be modified (each time)
to adjust the |\includeonly| command
to the present needs. This easily leaves the main file in a messy state.
\item
The generated document will always carry the filename
of the main document. This is inconvenient if
several child files are to be compiled and
to be kept for distribution.
\end{itemize}

The present package provides a simple interface
to make child files individually compilable by \LaTeX{}.
Compiling a child file then has the same effect as compiling
the main file with an |\includeonly| command
to select the appropriate child.
Moreover the generated document will carry the name of the child
rather than the main file.
This resolves all three above issues.

This feature is meant to make the editing of books,
thesis documents and lecture notes somewhat more convenient.
However, the package can also be used efficiently for
composing a series of documents (such as exercise sheets)
which are typically distributed individually.
It then assists the author in generating the individual documents
(potentially in different versions)
as well as a document containing the collected series.
Another application is in developing style files
or other kinds of included material
where compilation of the style file could redirect
to a sample or test file.

%%%%%%%%%%%%%%%%%%%%%%%%%%%%%%%%%%%%%%%%%%%%%%%%%%%%%%%%%%%%%%%%%%%%%%%%%%%%%%%%
%%%%%%%%%%%%%%%%%%%%%%%%%%%%%%%%%%%%%%%%%%%%%%%%%%%%%%%%%%%%%%%%%%%%%%%%%%%%%%%%
\section{Usage}

First of all, the package \textsf{childdoc} is \emph{not} a standard
\LaTeXe{} |.sty| style file! Therefore it needs to be invoked in
a non-standard way.

%%%%%%%%%%%%%%%%%%%%%%%%%%%%%%%%%%%%%%%%%%%%%%%%%%%%%%%%%%%%%%%%%%%%%%%%%%%%%%%%
\subsection{Included Files}
\label{sec:include}

%%%%%%%%%%%%%%%%%%%%%%%%%%%%%%%%%%%%%%%%
\DescribeMacro{\childdocmain}
To use the package, add the commands
\begin{center}
\begin{tabular}{l}
|% \iffalse
%
% childdoc.dtx Copyright (C) 2017-2018 Niklas Beisert
%
% This work may be distributed and/or modified under the
% conditions of the LaTeX Project Public License, either version 1.3
% of this license or (at your option) any later version.
% The latest version of this license is in
%   http://www.latex-project.org/lppl.txt
% and version 1.3 or later is part of all distributions of LaTeX
% version 2005/12/01 or later.
%
% This work has the LPPL maintenance status `maintained'.
%
% The Current Maintainer of this work is Niklas Beisert.
%
% This work consists of the files childdoc.dtx and childdoc.ins
% and the derived files childdoc.def and cdocsamp.tex with
% cdocsch1.tex, cdocsch2.tex, cdocsdrf.tex, cdocsfn1.tex, cdocsfn2.tex.
%
%<package>\ifdefined\childdocmain\endinput\fi
%<package>\ProvidesFile{childdoc.def}[2018/12/30 v2.0 child document driver]
%<samplemain>\ProvidesFile{cdocsamp.tex}[2018/12/30 v2.0 sample for childdoc]
%<*driver>
%\ProvidesFile{childdoc.drv}[2018/12/30 v2.0 childdoc reference manual file]
\PassOptionsToClass{10pt,a4paper}{article}
\documentclass{ltxdoc}

\usepackage[margin=35mm]{geometry}
\usepackage{hyperref}
\usepackage{hyperxmp}
\usepackage[usenames]{color}

\hypersetup{colorlinks=true}
\hypersetup{pdfstartview=FitH}
\hypersetup{pdfpagemode=UseNone}
\hypersetup{pdfsource={}}
\hypersetup{pdflang={en-UK}}
\hypersetup{pdfcopyright={Copyright 2017-2018 Niklas Beisert.
  This work may be distributed and/or modified under the
  conditions of the LaTeX Project Public License, either version 1.3
  of this license or (at your option) any later version.}}
\hypersetup{pdflicenseurl={http://www.latex-project.org/lppl.txt}}
\hypersetup{pdfcontactaddress={ETH Zurich, ITP, HIT K,
  Wolfgang-Pauli-Strasse 27}}
\hypersetup{pdfcontactpostcode={8093}}
\hypersetup{pdfcontactcity={Zurich}}
\hypersetup{pdfcontactcountry={Switzerland}}
\hypersetup{pdfcontactemail={nbeisert@itp.phys.ethz.ch}}
\hypersetup{pdfcontacturl={http://people.phys.ethz.ch/\xmptilde nbeisert/}}

\newcommand{\secref}[1]{\hyperref[#1]{section \ref*{#1}}}

\parskip1ex
\parindent0pt
\let\olditemize\itemize
\def\itemize{\olditemize\parskip0pt}

\begin{document}

\title{The \textsf{childdoc} Package}
\hypersetup{pdftitle={The childdoc Package}}
\author{Niklas Beisert\\[2ex]
  Institut f\"ur Theoretische Physik\\
  Eidgen\"ossische Technische Hochschule Z\"urich\\
  Wolfgang-Pauli-Strasse 27, 8093 Z\"urich, Switzerland\\[1ex]
  \href{mailto:nbeisert@itp.phys.ethz.ch}
  {\texttt{nbeisert@itp.phys.ethz.ch}}}
\hypersetup{pdfauthor={Niklas Beisert}}
\hypersetup{pdfsubject={Manual for the LaTeX2e Package childdoc}}
\date{30 December 2018, \textsf{v2.0}}
\maketitle

\begin{abstract}\noindent
\textsf{childdoc} is a \LaTeXe{} package
that enables the direct compilation
of document sections included by |\include|
to individual files.
\end{abstract}

\begingroup
\parskip0ex
\tableofcontents
\endgroup

%%%%%%%%%%%%%%%%%%%%%%%%%%%%%%%%%%%%%%%%%%%%%%%%%%%%%%%%%%%%%%%%%%%%%%%%%%%%%%%%
%%%%%%%%%%%%%%%%%%%%%%%%%%%%%%%%%%%%%%%%%%%%%%%%%%%%%%%%%%%%%%%%%%%%%%%%%%%%%%%%
\section{Introduction}

\LaTeX{} provides a mechanism to structure a large document (such as a book)
into a main file and several child files (containing the chapters)
using the |\include| command.
This mechanism is beneficial for documents
which span hundreds of pages in order to
make the source file(s) more manageable.
Moreover, compilation can be restricted to
selected child files by means of the |\includeonly| command.
The latter feature can be used to reduce the compilation time while editing
(this was significantly more useful in the earlier days of \LaTeX{})
or to generate a smaller document which is easier to navigate.
Another application of |\includeonly| is to generate
documents consisting of selected parts of the complete document.

However, there are a few drawbacks of the plain |\include| mechanism:
\begin{itemize}
\item
The child files cannot be compiled on their own,
they can only be compiled via the main file.
A naive editing environment
(such as a text editor with an option
to have the current file processed by \LaTeX)
may require one to switch to the main file before compiling;
attempting to compile the child file produces errors.
\item
The main file must be modified (each time)
to adjust the |\includeonly| command
to the present needs. This easily leaves the main file in a messy state.
\item
The generated document will always carry the filename
of the main document. This is inconvenient if
several child files are to be compiled and
to be kept for distribution.
\end{itemize}

The present package provides a simple interface
to make child files individually compilable by \LaTeX{}.
Compiling a child file then has the same effect as compiling
the main file with an |\includeonly| command
to select the appropriate child.
Moreover the generated document will carry the name of the child
rather than the main file.
This resolves all three above issues.

This feature is meant to make the editing of books,
thesis documents and lecture notes somewhat more convenient.
However, the package can also be used efficiently for
composing a series of documents (such as exercise sheets)
which are typically distributed individually.
It then assists the author in generating the individual documents
(potentially in different versions)
as well as a document containing the collected series.
Another application is in developing style files
or other kinds of included material
where compilation of the style file could redirect
to a sample or test file.

%%%%%%%%%%%%%%%%%%%%%%%%%%%%%%%%%%%%%%%%%%%%%%%%%%%%%%%%%%%%%%%%%%%%%%%%%%%%%%%%
%%%%%%%%%%%%%%%%%%%%%%%%%%%%%%%%%%%%%%%%%%%%%%%%%%%%%%%%%%%%%%%%%%%%%%%%%%%%%%%%
\section{Usage}

First of all, the package \textsf{childdoc} is \emph{not} a standard
\LaTeXe{} |.sty| style file! Therefore it needs to be invoked in
a non-standard way.

%%%%%%%%%%%%%%%%%%%%%%%%%%%%%%%%%%%%%%%%%%%%%%%%%%%%%%%%%%%%%%%%%%%%%%%%%%%%%%%%
\subsection{Included Files}
\label{sec:include}

%%%%%%%%%%%%%%%%%%%%%%%%%%%%%%%%%%%%%%%%
\DescribeMacro{\childdocmain}
To use the package, add the commands
\begin{center}
\begin{tabular}{l}
|\input{childdoc.def}|\\
|\childdocmain{}|\\
\end{tabular}
\end{center}
at the very top of the main \LaTeX{} file,
in particular \emph{before} the |\documentclass| statement!
The argument of |\childdocmain| should be left empty
(but it must be present).

%%%%%%%%%%%%%%%%%%%%%%%%%%%%%%%%%%%%%%%%
\DescribeMacro{\childdocof}
Furthermore, add the commands
\begin{center}
\begin{tabular}{l}
|\input{childdoc.def}|\\
|\childdocof{|\textit{main}|}|\\
\end{tabular}
\end{center}
at the top of every child file \textit{child}
which is included by |\include{|\textit{child}|}|
from within the main file
(or at least for those files to be compiled individually).
The argument \textit{main} must be the filename of the main file.

There are a couple of
considerations in setting up the main and child documents:

%%%%%%%%%%%%%%%%%%%%%%%%%%%%%%%%%%%%%%%%
\paragraph{Restrictions.}

Please note the following restrictions:
\begin{itemize}
\item
|\childdocmain| must be called with one argument \textit{main}
to ensure compatibility with earlier version of the package.
It must either be empty (|\childdocmain{}|)
or precisely match the filename of the main file in which it is specified.
See \secref{sec:detection} for further information.
\item
The filename \textit{main} must be specified without the |.tex| extension.
\item
The filename \textit{main} is case sensitive
(even in case-insensitive file systems)
due to internal string comparison.
\item
The argument \textit{main} should be fully expanded, it cannot be a macro.
\item
Subdirectories and special characters should be avoided in filenames.
\item
The command |\childdocmain{|\textit{main}|}| must be followed by a whitespace.
It should not be followed immediately by another command
or by a comment mark `|%|'.
This is because the \TeX{} parser reads the token immediately following
the argument of |\childdocmain| and puts it
at the beginning of every child section;
however, a white\-space is ignored.
\end{itemize}

%%%%%%%%%%%%%%%%%%%%%%%%%%%%%%%%%%%%%%%%
\paragraph{Content of Main File.}

It is advisable to place all content in the child files included by |\include|.
Any output contained in the main file will appear in all child documents
unless suppressed manually;
it cannot be suppressed automatically by the |\includeonly| directive
and thus should normally be avoided.
A method to include some content in the main file
by means of conditional processing is described in \secref{sec:conditional}.

%%%%%%%%%%%%%%%%%%%%%%%%%%%%%%%%%%%%%%%%
\paragraph{Page Numbering.}

When only a part of the document is compiled,
the appropriate numbering of pages
(as well as other status parameters)
is determined from the |.aux| files.
The latter contain information from previous passes.
However this information needs to propagate through
all intermediate child documents.
Therefore the page numbering in child documents may well
be inconsistent until the complete document is compiled at least once.

A useful (if unconventional) way to always ensure a consistent
page numbering is to restart the numbering in each child document
and denote the pages by `\textit{child}|.|\textit{page}'
where \textit{child} represents the chapter/section number of the child file.
This can be achieved by the command
|\numberwithin{page}{|\textit{child}|}|
of the \textsf{amsmath} package
where \textit{child} can be |chapter| or |section|
depending on the chosen structuring.
Alternatively, one can modify the macro |\thepage| appropriately
and reset the counter |page| at the start of each child file.

%%%%%%%%%%%%%%%%%%%%%%%%%%%%%%%%%%%%%%%%%%%%%%%%%%%%%%%%%%%%%%%%%%%%%%%%%%%%%%%%
\subsection{Conditional Processing}
\label{sec:conditional}

The package provides a mechanism to compile different versions
of a document. To customise the versions further some conditional processing
can come in handy to distinguish which version is being compiled.
The package provides two macros to describe the compilation context:

%%%%%%%%%%%%%%%%%%%%%%%%%%%%%%%%%%%%%%%%
\DescribeMacro{\ifchilddoc}
The conditional |\ifchilddoc| distinguishes between the compilation of
child documents and the main document:
%
\begin{center}
|\ifchilddoc |\textit{child-code}| |[|\||else |\textit{main-code}]| \||fi|
\end{center}

%%%%%%%%%%%%%%%%%%%%%%%%%%%%%%%%%%%%%%%%
\DescribeMacro{\childdocname}
\DescribeMacro{\childdocjob}
The macro |\childdocname| contains the filename (without extension)
of the main or child file being processed.
Note that |\childdocjob| will always contain the name of the main file.

%%%%%%%%%%%%%%%%%%%%%%%%%%%%%%%%%%%%%%%%
\paragraph{Title Page.}

Conditional processing can be used to include a title or banner page
in the main document when proper precautions are taken.
Importantly, the code in the main file should ensure that the page counter
(as well as other status parameters which are stored in the |.aux| files)
takes the same value after the conditional processing.
Otherwise the page numbers may take divergent values
depending on which part is compiled.

For example, a title page could be declared by:
%
\begin{center}
\begin{tabular}{l}
|\ifchilddoc\||else|\\
|\addtocounter{page}{-1}|\\
\textit{code for title page}\\
|\newpage|\\
|\||fi|
\end{tabular}
\end{center}
%
A banner page for the child documents can be generated by:
%
\begin{center}
\begin{tabular}{l}
|\ifchilddoc|\\
|\addtocounter{page}{-1}|\\
\textit{code for banner page}\\
|\newpage|\\
|\||fi|
\end{tabular}
\end{center}
%
Here one could write a message such as:
\begin{center}
|This is the part \childdocname{} of \childdocjob{}.|
\end{center}

%%%%%%%%%%%%%%%%%%%%%%%%%%%%%%%%%%%%%%%%%%%%%%%%%%%%%%%%%%%%%%%%%%%%%%%%%%%%%%%%
\subsection{Flags}
\label{sec:flags}

The package makes it easy to generate different versions
of the main or child documents.
To this end compilation flags can be defined
and assigned different default values.
They will be particularly useful in conjunction
with the forwarding mechanism described in \secref{sec:forward}.

For example, it may be useful to have a flag |\version|
which can be set to |draft| or |final|.
The document source will contain some conditional code
depending on the value of |\version|.
Suppose further, the flag should default to |final| for the main file
and to |draft| for child files
which is a natural assignment for editing the document.
This is achieved by placing the following code
in the preamble of the main document
(below the |\childdocmain| directive):
%
\begin{center}
\begin{tabular}{l}
|\ifchilddoc|\\
|\providecommand{\version}{draft}|\\
|\||else|\\
|\providecommand{\version}{final}|\\
|\||fi|
\end{tabular}
\end{center}
%
The definition by |\providecommand| makes sure
that previous definitions are not overwritten.
Further statements |\providecommand{\version}{...}|
can thus be added before the above code to override it.

For the main file, one might add a line
(between |\childdocmain| and the above block)
%
\begin{center}
|%\ifchilddoc\||else\providecommand{\version}{draft}\||fi|
\end{center}
%
which can be uncommented to produce a draft version.
Likewise one can add a line to the very top of a child file
(above the |\childdocof{|\textit{main}|}| directive)
%
\begin{center}
|%\providecommand{\version}{final}|
\end{center}
%
which can be uncommented to produce the final version of this child document.

%%%%%%%%%%%%%%%%%%%%%%%%%%%%%%%%%%%%%%%%%%%%%%%%%%%%%%%%%%%%%%%%%%%%%%%%%%%%%%%%
\subsection{Forwarding}
\label{sec:forward}

Different versions of the main or child documents
using compilation flags as described in \secref{sec:flags}
can be (permanently) stored in different files
for convenient compilation, viewing and distribution.
To this end, the package defines a command
to pass on compilation to a different file:

%%%%%%%%%%%%%%%%%%%%%%%%%%%%%%%%%%%%%%%%
\DescribeMacro{\childdocforward}
The command |\childdocforward| redirects processing to
another source file:
%
\begin{center}
\begin{tabular}{l}
|\input{childdoc.def}|\\
|\childdocforward[|\textit{main}|]{|\textit{dest}|}|\\
\end{tabular}
\end{center}
%
The argument \textit{dest} is the destination file
(without extension).
It should be the main file or one of the child files.
Note that further \textsf{childdoc} directives
such as |\childdocof| and |\childdocforward|
in the indicated file will be processed in this form.
The optional argument \textit{main}
passes on directly to the main file \textit{main}
while pretending to compile the child \textit{dest}.
This form behaves as if \textit{dest}
issues |\childdocof{|\textit{main}|}| right away,
and no further \textsf{childdoc} directives will be processed.

%%%%%%%%%%%%%%%%%%%%%%%%%%%%%%%%%%%%%%%%
\DescribeMacro{\...prefix}
In the alternative form |\childdocforwardprefix|,
%
\begin{center}
\begin{tabular}{l}
|\input{childdoc.def}|\\
|\childdocforwardprefix[|\textit{main}|]{|\textit{prefix}|}{|\textit{dest}|}|
\end{tabular}
\end{center}
%
the destination file is determined by a pattern
depending on the current file:
To make this work, the current file must be called
`{\textit{prefix}\hspace{0.2em}\textit{suffix}}'
with \textit{prefix} matching precisely the argument.
Processing is then passed on to the file
`{\textit{dest}\hspace{0.2em}\textit{suffix}}'.
Surely, the same effect is achieved by
directly specifying the
argument `{\textit{dest}\hspace{0.2em}\textit{suffix}}'
in the first form.
However, that requires to set up a different file
for each child. With the alternative form of the command
all these files can have exactly the same content
which simplifies setting them up and maintaining them.

For example, the following file |draft.tex|
with a compilation flag |\version| as described in \secref{sec:flags}
compiles the main document as a draft:
%
\begin{center}
\begin{tabular}{l}
|\def\version{draft}|\\
|\input{childdoc.def}|\\
|\childdocforward{|\textit{main}|}|
\end{tabular}
\end{center}
%
Likewise, the following files |final|\textit{nn}|.tex|
compile the final version of the child document
|child|\textit{nn}|.tex|:
%
\begin{center}
\begin{tabular}{l}
|\def\version{final}|\\
|\input{childdoc.def}|\\
|\childdocforwardprefix{final}{child}|
\end{tabular}
\end{center}
%

Note that when several versions of a main file and/or of each child file
are to be generated, it may be convenient to set up a |Makefile| or
shell script to automatise the process.

%%%%%%%%%%%%%%%%%%%%%%%%%%%%%%%%%%%%%%%%%%%%%%%%%%%%%%%%%%%%%%%%%%%%%%%%%%%%%%%%
\subsection{Command Line Processing}
\label{sec:commandline}

The effect of redirection files can also be achieved by invoking
the \LaTeX{} compiler with a more elaborate command line.
Most conveniently this should be done as part
of a shell script or a |Makefile|.

When using \textsf{childdoc} in the main file, the following
command lines effectively perform a redirection
(note that depending on the shell being used,
backslashes may have to be doubled: `|\|' $\to$ `|\\|'):
%
\begin{center}
|... -jobname "|\textit{target}|" |\\|"|[\textit{flags}]%
|\input{childdoc.def}\childdocforward[|\textit{main}|]{|\textit{dest}|}"|
\end{center}
%
Here \textit{target} is the name of the output file,
\textit{main} is the name of the main file
and \textit{dest} is the name of the main or child file to be processed
(all filenames without extensions).
The optional argument \textit{main} can be omitted
if \textit{main} matches \textit{dest}.
Optionally, compilation \textit{flags} can be defined via |\def| commands.
This command line makes the \TeX{} engine believe
it is compiling the file \textit{target}
whose content is specified as the latter parameter.
The provided code then forwards the processing to
\textit{main} or \textit{dest} as described in \secref{sec:forward}.

%%%%%%%%%%%%%%%%%%%%%%%%%%%%%%%%%%%%%%%%%%%%%%%%%%%%%%%%%%%%%%%%%%%%%%%%%%%%%%%%
\subsection{Include by Input}
\label{sec:input}

Including child documents by |\include| has some restrictions by design.
Most notably, the content of a child document always occupies
its own set of pages; pages cannot be shared between child documents.
Usually, this behaviour makes perfect sense
because each child document contain an essential part of the document.
However, in some situations it may be desirable to compose
a document from a collection of parts
without having mandatory page breaks between then.
For this case, the package
provides a mechanism to include parts
by |\input| which can also be processed individually.
However, by construction this mechanism
requires manual handling of the content to be output.

%%%%%%%%%%%%%%%%%%%%%%%%%%%%%%%%%%%%%%%%
\DescribeMacro{\ifchilddocmanual}
The main file should be prepared as usual, see \secref{sec:include}.
However, the document body must make a distinction
between processing of an individual part and of the main document, e.g.:
%
\begin{center}
\begin{tabular}{l}
|\ifchilddocmanual|\\
|\input{\childdocname}|\\
|\||else|\\
\textit{document body with }|\input{|\textit{part}|}|\\
|\||fi|
\end{tabular}
\end{center}
%
The conditional |\ifchilddocmanual| is true whenever
a part to be included by |\input| is being compiled,
and the name of the part is stored in |\childdocname|.

%%%%%%%%%%%%%%%%%%%%%%%%%%%%%%%%%%%%%%%%
\DescribeMacro{\childdocby}
Each part to be included by |\input| should start with:
%
\begin{center}
\begin{tabular}{l}
|\input{childdoc.def}|\\
|\childdocby{|\textit{main}|}|\\
\end{tabular}
\end{center}
%
The directive |\childdocby| is similar to |\childdocof|
described in \secref{sec:include},
but the subsequent selection of content must be done manually.
To that end, both |\ifchilddoc| and |\ifchilddocmanual|
will be true upon processing of a part,
and the name of the part is stored in |\childdocname|.
Note that |\jobname| will be set to the filename of the current part
so that each part receives an individual |.aux| file
that does not interfere with the |.aux| file(s) of the main document.
This behaviour can be altered by the alternative form
|\childdocby[*]{|\textit{main}|}| (with a non-empty optional argument)
which uses the |.aux| file of the main document
by setting |\jobname| to \textit{main}.

%%%%%%%%%%%%%%%%%%%%%%%%%%%%%%%%%%%%%%%%%%%%%%%%%%%%%%%%%%%%%%%%%%%%%%%%%%%%%%%%
\subsection{Driver Development}
\label{sec:driver}

The \textsf{childdoc} mechanism can also be use for the development
of definition files such as \LaTeX{} styles or classes.
This case differs from the above setup with multiple parts
included by |\include| in that no |\includeonly| should be invoked.
This can be achieved by starting the include file
(before |\ProvidesPackage|) with:
%
\begin{center}
\begin{tabular}{l}
|\input{childdoc.def}|\\
|\childdocforward{|\textit{main}|}|\\
\end{tabular}
\end{center}
%
or alternatively with:
%
\begin{center}
\begin{tabular}{l}
|\input{childdoc.def}|\\
|\childdocby{|\textit{main}|}|\\
\end{tabular}
\end{center}
%
Both forms have slightly different effects as described above.
The main file is prepared as usual, see \secref{sec:include}.

%%%%%%%%%%%%%%%%%%%%%%%%%%%%%%%%%%%%%%%%%%%%%%%%%%%%%%%%%%%%%%%%%%%%%%%%%%%%%%%%
\subsection{Legacy Detection}
\label{sec:detection}

The directive |\childdocmain| in the main file can detect
whether the complete document or merely a child is to be compiled
even without using the directive |\childdocof|.
This method is deprecated because it is less robust
and there is no compelling reason to use it;
it is merely provided for backward compatibility
and it may be removed in future versions.

If the detection mechanism is to be used,
it is mandatory to correctly specify
the filename of the main file as the argument of |\childdocmain|:
%
\begin{center}
\begin{tabular}{l}
|\input{childdoc.def}|\\
|\childdocmain{|\textit{main}|}|\\
\end{tabular}
\end{center}
%
If |\jobname| does not match the argument \textit{main} of |\childdocmain|,
it is assumed that |\jobname| points to the child file to be compiled.
When using |\childdocmain| with the main file specified as argument,
it suffices to start a child file
with just |\input{|\textit{main}|}|
without loading of the package and using |\childdocof|.
If instead all processing is done
with the appropriate \textsf{childdoc} directives,
the argument of \textit{main} of |\childdocmain| can be empty.

An alternative version of the command line processing described
in \secref{sec:commandline} using the detection mechanism reads:
%
\begin{center}
|... -jobname "|\textit{target}|" "|[\textit{flags}]%
[|\def\jobname{|\textit{dest}|}|]|\input{|\textit{main}|}"|
\end{center}

%%%%%%%%%%%%%%%%%%%%%%%%%%%%%%%%%%%%%%%%%%%%%%%%%%%%%%%%%%%%%%%%%%%%%%%%%%%%%%%%
\subsection{Manual Code}
\label{sec:manual}

In case one cannot be certain whether the definitions file |childdoc.def|
is installed on the target \TeX{} distribution
and one prefers not to ship it,
it is conceivable to paste a few relevant commands into the sources.

To that end, drop all statements |\input{childdoc.def}|
and perform the replacements as outlined below.
Instead of |\childdocmain{|\textit{main}|}| add the following code
to the top of the main file:
%
\begin{center}
\begin{tabular}{l}
|\||ifdefined\childdocname\endinput\||fi\newif\ifchilddoc|\\
|\edef\childdocname{\scantokens\expandafter{\jobname\noexpand}}|\\
|\def\childdocmain{|\textit{main}|}\||ifx\childdocmain\childdocname\||else|\\
|\childdoctrue\includeonly{\childdocname}\let\jobname\childdocmain\||fi|\\
\end{tabular}
\end{center}
%
Instead of |\childdocof{|\textit{main}|}| just include the main file
at the top of each child file:
%
\begin{center}
|\input{|\textit{main}|}|
\end{center}
%
A simple redirection |\childdocforward{|\textit{dest}|}| is achieved by:
%
\begin{center}
|\def\jobname{|\textit{dest}|}\input{\jobname}|
\end{center}
%
The redirection with prefix
|\childdocforwardprefix[|\textit{prefix}|]{|\textit{dest}|}|
is accomplished by:
%
\begin{center}
\begin{tabular}{l}
|{\edef\jobname{\scantokens\expandafter{\jobname\noexpand}}|\\
|\def\redirectjob |\textit{prefix}|#1~~~{\gdef\jobname{|\textit{dest}|#1}}|\\
|\expandafter\redirectjob\jobname~~~}\input{\jobname}|
\end{tabular}
\end{center}

In an alternative approach,
child documents can be compiled by a specific command line
without additional code or specific definitions:
%
\begin{center}
|... -jobname "|\textit{target}|" "|[\textit{flags}]%
|\includeonly{|\textit{dest}|}\input{|\textit{main}|}"|
\end{center}
%

%%%%%%%%%%%%%%%%%%%%%%%%%%%%%%%%%%%%%%%%%%%%%%%%%%%%%%%%%%%%%%%%%%%%%%%%%%%%%%%%
%%%%%%%%%%%%%%%%%%%%%%%%%%%%%%%%%%%%%%%%%%%%%%%%%%%%%%%%%%%%%%%%%%%%%%%%%%%%%%%%
\section{Information}

%%%%%%%%%%%%%%%%%%%%%%%%%%%%%%%%%%%%%%%%%%%%%%%%%%%%%%%%%%%%%%%%%%%%%%%%%%%%%%%%
\subsection{Copyright}

Copyright \copyright{} 2017--2018 Niklas Beisert

This work may be distributed and/or modified under the
conditions of the \LaTeX{} Project Public License, either version 1.3
of this license or (at your option) any later version.
The latest version of this license is in
  \url{http://www.latex-project.org/lppl.txt}
and version 1.3 or later is part of all distributions of \LaTeX{}
version 2005/12/01 or later.

This work has the LPPL maintenance status `maintained'.

The Current Maintainer of this work is Niklas Beisert.

This work consists of the files |README.txt|, |childdoc.ins| and |childdoc.dtx|
as well as the derived files |childdoc.def|, |cdocsamp.tex|
with |cdocsch1.tex|, |cdocsch2.tex|, |cdocspt3.tex|, |cdocspt4.tex|,
|cdocsdrf.tex|, |cdocsfn1.tex|, |cdocsfn2.tex|
as well as |childdoc.pdf|.

%%%%%%%%%%%%%%%%%%%%%%%%%%%%%%%%%%%%%%%%%%%%%%%%%%%%%%%%%%%%%%%%%%%%%%%%%%%%%%%%
\subsection{Files and Installation}

The package consists of the files:
%
\begin{center}
\begin{tabular}{ll}
    |README.txt|   & readme file \\
    |childdoc.ins| & installation file \\
    |childdoc.dtx| & source file \\
    |childdoc.def| & definition file \\
    |cdocsamp.tex| & sample main file \\
    |cdocsch1.tex| & sample include file \\
    |cdocsch2.tex| & sample include file \\
    |cdocspt3.tex| & sample part file \\
    |cdocspt4.tex| & sample part file \\
    |cdocsdrf.tex| & sample redirection file \\
    |cdocsfn1.tex| & sample redirection file \\
    |cdocsfn2.tex| & sample redirection file \\
    |childdoc.pdf| & manual
\end{tabular}
\end{center}
%
The distribution consists of the files
|README.txt|, |childdoc.ins| and |childdoc.dtx|.
%
\begin{itemize}
\item
Run (pdf)\LaTeX{} on |childdoc.dtx|
to compile the manual |childdoc.pdf| (this file).
\item
Run \LaTeX{} on |childdoc.ins| to create the definitions file |childdoc.def|
and the sample |cdocsamp.tex| with include files
|cdocsch1.tex|, |cdocsch2.tex|, |cdocspt3.tex|, |cdocspt4.tex|,
|cdocsdrf.tex|, |cdocsfn1.tex|, |cdocsfn2.tex|.
Then copy the file |childdoc.def| to an appropriate directory of your \LaTeX{}
distribution, e.g.\ \textit{texmf-root}|/tex/latex/childdoc|.
\end{itemize}

%%%%%%%%%%%%%%%%%%%%%%%%%%%%%%%%%%%%%%%%%%%%%%%%%%%%%%%%%%%%%%%%%%%%%%%%%%%%%%%%
\subsection{Related CTAN Packages}

There are several other packages which offer a similar functionality:
%
\begin{itemize}
\item
The packages
\href{http://ctan.org/pkg/docmute}{\textsf{docmute}},
\href{http://ctan.org/pkg/includex}{\textsf{includex}} and
\href{http://ctan.org/pkg/standalone}{\textsf{standalone}}
provide commands to include only the document body of
a child file thus allowing both files to be compiled individually.
\item
The packages \href{http://ctan.org/pkg/subdocs}{\textsf{subdocs}}
and \href{http://ctan.org/pkg/subfiles}{\textsf{subfiles}}
provide structures in which the main and child documents can be
encapsulated and allowing them to be compiled individually.
The inclusion mechanism is different from the conventional |\include|.
\item
The package \href{http://ctan.org/pkg/combine}{\textsf{combine}}
is an elaborate solution to combine several documents into one.
\end{itemize}
%
See also the CTAN topic \href{http://ctan.org/topic/subdocs}{\textsf{subdocs}}
for further related packages.
The present package differs from the above solutions in that
a document structure constructed with the conventional |\include| mechanism
just needs two extra commands at the top of every file
such that all constituent files can be compiled individually.

%%%%%%%%%%%%%%%%%%%%%%%%%%%%%%%%%%%%%%%%%%%%%%%%%%%%%%%%%%%%%%%%%%%%%%%%%%%%%%%%
%\subsection{Feature Suggestions}
%
%The following is a list of features which may be useful for future
%versions of this package:
%%
%\begin{itemize}
%\item
%\ldots
%\end{itemize}

%%%%%%%%%%%%%%%%%%%%%%%%%%%%%%%%%%%%%%%%%%%%%%%%%%%%%%%%%%%%%%%%%%%%%%%%%%%%%%%%
\subsection{Revision History}

%%%%%%%%%%%%%%%%%%%%%%%%%%%%%%%%%%%%%%%%
\paragraph{v2.0:} 2018/12/30

\begin{itemize}
\item
immediate forward processing
\item
added |\childdocby| mechanism
\item
manual restructured
\end{itemize}

%%%%%%%%%%%%%%%%%%%%%%%%%%%%%%%%%%%%%%%%
\paragraph{v1.6:} 2018/01/17

\begin{itemize}
\item
application for development of include files
\item
corrections to manual
\end{itemize}

%%%%%%%%%%%%%%%%%%%%%%%%%%%%%%%%%%%%%%%%
\paragraph{v1.5:} 2017/05/21

\begin{itemize}
\item
more complete structuring introduced
\item
|\childdocof| introduced
\item
|\childdoc| renamed to |\childdocmain|
\item
|\childredirect| renamed to |\childdocforward| and |\childdocforwardprefix|
and functionality expanded
\end{itemize}

%%%%%%%%%%%%%%%%%%%%%%%%%%%%%%%%%%%%%%%%
\paragraph{v1.0:} 2017/04/27

\begin{itemize}
\item
manual and install package
\item
first version published on CTAN
\end{itemize}

%%%%%%%%%%%%%%%%%%%%%%%%%%%%%%%%%%%%%%%%
\paragraph{v0.6:} 2017/04/26

\begin{itemize}
\item
redirection mechanism added
\end{itemize}

%%%%%%%%%%%%%%%%%%%%%%%%%%%%%%%%%%%%%%%%
\paragraph{v0.5:} 2017/04/26

\begin{itemize}
\item
functionality in definition file
\end{itemize}


%%%%%%%%%%%%%%%%%%%%%%%%%%%%%%%%%%%%%%%%%%%%%%%%%%%%%%%%%%%%%%%%%%%%%%%%%%%%%%%%
%%%%%%%%%%%%%%%%%%%%%%%%%%%%%%%%%%%%%%%%%%%%%%%%%%%%%%%%%%%%%%%%%%%%%%%%%%%%%%%%
%%%%%%%%%%%%%%%%%%%%%%%%%%%%%%%%%%%%%%%%%%%%%%%%%%%%%%%%%%%%%%%%%%%%%%%%%%%%%%%%
\appendix

\settowidth\MacroIndent{\rmfamily\scriptsize 000\ }

 \DocInput{childdoc.dtx}

\end{document}
%</driver>
% \fi
%
% %%%%%%%%%%%%%%%%%%%%%%%%%%%%%%%%%%%%%%%%%%%%%%%%%%%%%%%%%%%%%%%%%%%%%%%%%%%%%%
% %%%%%%%%%%%%%%%%%%%%%%%%%%%%%%%%%%%%%%%%%%%%%%%%%%%%%%%%%%%%%%%%%%%%%%%%%%%%%%
% \section{Sample}
%\iffalse
%<*samplemain>
%\fi
%
% The following presents a sample document
% with two chapters, two parts, a title page,
% a compile flag as well as three forwarding files to set the flag.
% It consists of eight |.tex| files:
% \begin{center}
% \begin{tabular}{ll}
% |cdocsamp.tex|&main file\\
% |cdocsch1.tex|&include file for chapter 1\\
% |cdocsch2.tex|&include file for chapter 2\\
% |cdocspt3.tex|&include file for part 3\\
% |cdocspt4.tex|&include file for part 4\\
% |cdocsdrf.tex|&forwarding file for main file in draft mode\\
% |cdocsfi1.tex|&forwarding file for final version of chapter 1\\
% |cdocsfi2.tex|&forwarding file for final version of chapter 2\\
% \end{tabular}
% \end{center}
% Each of the eight files can be compiled directly by the \LaTeX{} compiler.
%
% %%%%%%%%%%%%%%%%%%%%%%%%%%%%%%%%%%%%%%
% \paragraph{Main File.}
%
% The main file is called |cdocsamp.tex|.
%
% Load the \textsf{childdoc} definitions and
% declare the filename for the main document:
%    \begin{macrocode}
\input{childdoc.def}
\childdocmain{}
%    \end{macrocode}

% Optional override for |\version| flag:
%    \begin{macrocode}
%%\ifchilddoc\else\providecommand{\version}{draft}\fi
%    \end{macrocode}

% Define the default values for the |\version| flag
% (|final| for the main file and |draft| for childs):
%    \begin{macrocode}
\ifchilddoc
\providecommand{\version}{draft}
\else
\providecommand{\version}{final}
\fi
%    \end{macrocode}

% Load the standard document class:
%    \begin{macrocode}
\documentclass[12pt]{article}
%    \end{macrocode}

% Start the document body:
%    \begin{macrocode}
\begin{document}
%    \end{macrocode}

% Declare a title page.
% Print title, part of document being processed and version flag:
%    \begin{macrocode}
\addtocounter{page}{-1}
\begin{center}
{\LARGE\bfseries{}childdoc example\par}
\vspace{1cm}
\ifchilddoc
\ifchilddocmanual part\else chapter\fi:
`\childdocname' of `\childdocjob'\par
\else
main document: `\childdocjob'\par
\fi
version: \version\par
\end{center}
\newpage
%    \end{macrocode}

% Manually include selected file,
% otherwise process as usual:
%    \begin{macrocode}
\ifchilddocmanual
\section*{part `\childdocname'}
\input{\childdocname}
\else
%    \end{macrocode}

% Include the two chapters:
%    \begin{macrocode}
\include{cdocsch1}
\include{cdocsch2}
%    \end{macrocode}

% Include the two parts unless only chapters should be displayed:
%    \begin{macrocode}
\ifchilddoc\else
\section{part three}
\input{cdocspt3}
\section{part four}
\input{cdocspt4}
\fi
%    \end{macrocode}

% Process as usual until here:
%    \begin{macrocode}
\fi
%    \end{macrocode}

% End of document body:
%    \begin{macrocode}
\end{document}
%    \end{macrocode}
%\iffalse
%</samplemain>
%\fi
%
% %%%%%%%%%%%%%%%%%%%%%%%%%%%%%%%%%%%%%%
% \paragraph{Chapter Include Files.}
%
% The include files are called |cdocsch1.tex| and |cdocsch2.tex|.
%
%\iffalse
%<*samplechap1|samplechap2>
%\fi

% Optional override for |\version| flag:
%    \begin{macrocode}
%%\providecommand{\version}{final}
%    \end{macrocode}

% Include the main document:
%    \begin{macrocode}
\input{childdoc.def}
\childdocof{cdocsamp}
%    \end{macrocode}

%\iffalse
%</samplechap1|samplechap2>
%\fi
%
%\iffalse
%<*samplechap1>
%\fi
% Some text for chapter 1:
%    \begin{macrocode}
\section{one}
some text in chapter one
%    \end{macrocode}

%\iffalse
%</samplechap1>
%\fi
% Some text for chapter 2:
%\iffalse
%<*samplechap2>
%\fi
%    \begin{macrocode}
\section{two}
more text in chapter two
%    \end{macrocode}

%\iffalse
%</samplechap2>
%\fi
%
% %%%%%%%%%%%%%%%%%%%%%%%%%%%%%%%%%%%%%%
% \paragraph{Part Include Files.}
%
% The include files are called |cdocspt3.tex| and |cdocspt4.tex|.
%
%\iffalse
%<*samplepart3|samplepart4>
%\fi

% Optional override for |\version| flag:
%    \begin{macrocode}
%%\providecommand{\version}{final}
%    \end{macrocode}

% Include the main document:
%    \begin{macrocode}
\input{childdoc.def}
\childdocby{cdocsamp}
%    \end{macrocode}

%\iffalse
%</samplepart3|samplepart4>
%\fi
%
%\iffalse
%<*samplepart3>
%\fi
% Some text for part 3:
%    \begin{macrocode}
some text in part three
%    \end{macrocode}

%\iffalse
%</samplepart3>
%\fi
% Some text for part 4:
%\iffalse
%<*samplepart4>
%\fi
%    \begin{macrocode}
more text in part four
%    \end{macrocode}

%\iffalse
%</samplepart4>
%\fi
%
% %%%%%%%%%%%%%%%%%%%%%%%%%%%%%%%%%%%%%%
% \paragraph{Forwarding for a Complete Draft.}
%
% The following forwarding file |cdocsdrf.tex|
% compiles the main document in draft mode:
%\iffalse
%<*sampledraft>
%\fi
%    \begin{macrocode}
\def\version{draft}
\input{childdoc.def}
\childdocforward{cdocsamp}
%    \end{macrocode}

%\iffalse
%</sampledraft>
%\fi
%
% %%%%%%%%%%%%%%%%%%%%%%%%%%%%%%%%%%%%%%
% \paragraph{Forwarding for Final Version of the Chapters.}
%
% The following forwarding files |cdocsfn1.tex| and |cdocsfn2.tex|
% (with identical content)
% compile the final versions of the child documents
% |cdocsch1.tex| and |cdocsch2.tex|, respectively:
%\iffalse
%<*samplefinal>
%\fi
%    \begin{macrocode}
\def\version{final}
\input{childdoc.def}
\childdocforwardprefix[cdocsamp]{cdocsfn}{cdocsch}
%    \end{macrocode}

%\iffalse
%</samplefinal>
%\fi
%
% %%%%%%%%%%%%%%%%%%%%%%%%%%%%%%%%%%%%%%
% \paragraph{Command Line Processing.}
%
% The following three command lines generate the output files
% |cdocscld|, |cdocscl1| and |cdocscl2|
% which should be identical to
% |cdocsdrf|, |cdocsch1| and |cdocsfn2|, respectively:
% \begin{center}
% \begin{tabular}{l}
% |latex -jobname cdocscld \|\\
% |  "\def\version{draft}\input{childdoc.def}\childdocforward{cdocsamp}"|\\
% |latex -jobname cdocscl1 \|\\
% |  "\input{childdoc.def}\childdocforward[cdocsamp]{cdocsch1}"|\\
% |latex -jobname cdocscl2 \|\\
% |  "\def\version{final}\input{childdoc.def}\childdocforward{cdocsch2}"|
% \end{tabular}
% \end{center}
% Note that the trailing backslash on each first line
% merely continues the input to the second line
% (for convenient cut ant paste).
% Furthermore, the command |latex| can be replaced by any
% of its alternative versions such as |pdflatex|.
%
% %%%%%%%%%%%%%%%%%%%%%%%%%%%%%%%%%%%%%%%%%%%%%%%%%%%%%%%%%%%%%%%%%%%%%%%%%%%%%%
% %%%%%%%%%%%%%%%%%%%%%%%%%%%%%%%%%%%%%%%%%%%%%%%%%%%%%%%%%%%%%%%%%%%%%%%%%%%%%%
% \section{Implementation}
%\iffalse
%<*package>
%\fi
%
% This section describes the definitions file |childdoc.def|.

% The definitions cannot be loaded using |\usepackage| or |\RequirePackage|
% which has a mechanism to prevent loading a style file more than once.
% When loading the definitions by means of |\input|
% multiple instances have to be prevented manually:
%\iffalse
%This code needs to be before the `\ProvidesFile' directive
%which is defined at the beginning of this file.
%Therefore it is also placed there and commented out here.
%</package>
%<*discard>
%\fi
%    \begin{macrocode}
\ifdefined\childdocmain\endinput\fi
%    \end{macrocode}
%\iffalse
%</discard>
%<*package>
%\fi
%
% \macro{\ifchilddoc}
% \macro{\ifchilddocmanual}
% The conditional |\ifchilddoc| tells whether a
% child (true) or main (false) document is being compiled.
% The conditional |\ifchilddocmanual| tells whether
% the |\includeonly| mechanism is used (false) or
% the selection of child files must be performed manually (true).
% The definitions initialise to false:
%    \begin{macrocode}
\newif\ifchilddoc
\newif\ifchilddocmanual
%    \end{macrocode}

% \macro{\childdocname}
% \macro{\childdocjob}
% The macro |\childdocname| stores the name of the main document
% to be compiled. The macro |\childdocjob| stores the name of
% the document on which the \LaTeX{} compiler was originally invoked.
% The content of |\jobname| cannot be compared
% to filenames specified in the source due to different catcodes.
% The following code rescans |\jobname|, stores the result
% in |\childdocname| and saves a copy in |\childdocjob|:
%    \begin{macrocode}
\edef\childdocname{\scantokens\expandafter{\jobname\noexpand}}
\let\childdocjob\childdocname
%    \end{macrocode}

% \macro{\childdocdisable}
% The macro |\childdocdisable| prevents the main file
% from being processed more than once.
% At this stage, the main document command |\childdocmain|
% is assumed to be called once again where it should do nothing.
% Any subsequent call to it should prevent
% a secondary processing of the main document
% It overwrites the forwarding commands
% |\childdocof| and |\childdocforward|
% with empty macros to prevent further inclusions of the main document:
%    \begin{macrocode}
\newcommand{\childdocdisable}
{
  \renewcommand{\childdocmain}[1]{\renewcommand{\childdocmain}[1]{\endinput}}
  \renewcommand{\childdocof}[1]{}
  \renewcommand{\childdocby}[2][]{}
  \renewcommand{\childdocforward}[2][]{}
  \renewcommand{\childdocdisable}{}
}
%    \end{macrocode}

% \macro{\childdocmain}
% The macro |\childdocmain| is to be called at the top of the main file
% with nothing or the main filename (without extension) as argument.
% First, it breaks loops.
% If the argument is not empty and does not match |\childdocname|
% (which is set by the first inclusion of |childdoc.def|),
% |\ifchilddoc| is set to true, |\includeonly| is applied to the child file
% and |\jobname| is set to the main file
% (for proper handling of |.aux| files):
%    \begin{macrocode}
\newcommand{\childdocmain}[1]
{
  \childdocdisable\childdocmain{}
  \if?#1?\else
    \begingroup
      \def\childdoctmp{#1}
      \ifx\childdoctmp\childdocname
        \def\childdoctmp{}
      \else
        \def\childdoctmp
        {
          \childdoctrue
          \includeonly{\childdocname}
          \def\childdocjob{#1}
          \def\jobname{#1}
        }
      \fi
      \expandafter
    \endgroup
    \childdoctmp
  \fi
}
%    \end{macrocode}

% \macro{\childdocof}
% The command |\childdocof| redirects
% compilation to the main file |#1|.
%    \begin{macrocode}
\newcommand{\childdocof}[1]
{
  \childdocdisable
  \childdoctrue
  \includeonly{\childdocname}
  \def\jobname{#1}
  \def\childdocjob{#1}
  \input{#1}
}
%    \end{macrocode}

% \macro{\childdocby}
% The command |\childdocby| ....
%    \begin{macrocode}
\newcommand{\childdocby}[2][]
{
  \childdocdisable
  \childdoctrue
  \childdocmanualtrue
  \if?#1?\else
    \def\jobname{#2}
  \fi
  \def\childdocjob{#2}
  \input{#2}
  \endinput
}
%    \end{macrocode}

% \macro{\childdocforward}
% The command |\childdocforward| redirects
% compilation to the main file or
% (if the optional argument is given) a child file.
% Parameters are set as if the main file
% or a child file starting with |\childdocof| was compiled.
% Then compilation is handed over to the main file:
%    \begin{macrocode}
\newcommand{\childdocforward}[2][]
{
  \begingroup
    \if?#1?
      \def\childdoctmp
      {
        \def\childdocname{#2}
        \def\childdocjob{#2}
        \def\jobname{#2}
        \input{#2}
        \endinput
      }
    \else
      \def\childdoctmp
      {
        \childdocdisable
        \def\childdocname{#2}
        \childdoctrue
        \includeonly{#2}
        \def\childdocjob{#1}
        \def\jobname{#1}
        \input{#1}
        \endinput
      }
    \fi
    \expandafter
  \endgroup
  \childdoctmp
}
%    \end{macrocode}

% \macro{\childdocforwardprefix}
% The command |\childdocforwardprefix| redirects
% compilation to the main or a child file by means of a pattern.
% The prefix |#1| in the current filename is replaced by |#2|
% and the suffix of the current filename is kept
% (it is assumed that the filename does not contain the substring `|~~~|'
% which is used as a delimiter).
% Compilation is handed over to the new file by |\childdocforward|:
%    \begin{macrocode}
\newcommand{\childdocforwardprefix}[3][]
{
  \begingroup
    \def\childdocextract #2##1~~~{\def\childdoctmp{\childdocforward[#1]{#3##1}}}
    \expandafter\childdocextract\childdocname~~~
    \expandafter
  \endgroup
  \childdoctmp
}
%    \end{macrocode}

% \macro{\childdoc}
% The deprecated macro |\childdoc| is a legacy version of |\childdocmain|:
%    \begin{macrocode}
\newcommand{\childdoc}{\childdocmain}
%    \end{macrocode}

% \macro{\childdocredirect}
% The deprecated macro |\childdocredirect| is a legacy version
% of |\childdocforward| and |\childdocforwardprefix|:
%    \begin{macrocode}
\newcommand{\childdocredirect}[2][]
{
  \begingroup
    \if?#1?
      \def\childdoctmp{\childdocforward{#2}}
    \else
      \def\childdoctmp{\childdocforwardprefix{#1}{#2}}
    \fi
    \expandafter
  \endgroup
  \childdoctmp
}
%    \end{macrocode}

%\iffalse
%</package>
%\fi
%
\endinput
|\\
|\childdocmain{}|\\
\end{tabular}
\end{center}
at the very top of the main \LaTeX{} file,
in particular \emph{before} the |\documentclass| statement!
The argument of |\childdocmain| should be left empty
(but it must be present).

%%%%%%%%%%%%%%%%%%%%%%%%%%%%%%%%%%%%%%%%
\DescribeMacro{\childdocof}
Furthermore, add the commands
\begin{center}
\begin{tabular}{l}
|% \iffalse
%
% childdoc.dtx Copyright (C) 2017-2018 Niklas Beisert
%
% This work may be distributed and/or modified under the
% conditions of the LaTeX Project Public License, either version 1.3
% of this license or (at your option) any later version.
% The latest version of this license is in
%   http://www.latex-project.org/lppl.txt
% and version 1.3 or later is part of all distributions of LaTeX
% version 2005/12/01 or later.
%
% This work has the LPPL maintenance status `maintained'.
%
% The Current Maintainer of this work is Niklas Beisert.
%
% This work consists of the files childdoc.dtx and childdoc.ins
% and the derived files childdoc.def and cdocsamp.tex with
% cdocsch1.tex, cdocsch2.tex, cdocsdrf.tex, cdocsfn1.tex, cdocsfn2.tex.
%
%<package>\ifdefined\childdocmain\endinput\fi
%<package>\ProvidesFile{childdoc.def}[2018/12/30 v2.0 child document driver]
%<samplemain>\ProvidesFile{cdocsamp.tex}[2018/12/30 v2.0 sample for childdoc]
%<*driver>
%\ProvidesFile{childdoc.drv}[2018/12/30 v2.0 childdoc reference manual file]
\PassOptionsToClass{10pt,a4paper}{article}
\documentclass{ltxdoc}

\usepackage[margin=35mm]{geometry}
\usepackage{hyperref}
\usepackage{hyperxmp}
\usepackage[usenames]{color}

\hypersetup{colorlinks=true}
\hypersetup{pdfstartview=FitH}
\hypersetup{pdfpagemode=UseNone}
\hypersetup{pdfsource={}}
\hypersetup{pdflang={en-UK}}
\hypersetup{pdfcopyright={Copyright 2017-2018 Niklas Beisert.
  This work may be distributed and/or modified under the
  conditions of the LaTeX Project Public License, either version 1.3
  of this license or (at your option) any later version.}}
\hypersetup{pdflicenseurl={http://www.latex-project.org/lppl.txt}}
\hypersetup{pdfcontactaddress={ETH Zurich, ITP, HIT K,
  Wolfgang-Pauli-Strasse 27}}
\hypersetup{pdfcontactpostcode={8093}}
\hypersetup{pdfcontactcity={Zurich}}
\hypersetup{pdfcontactcountry={Switzerland}}
\hypersetup{pdfcontactemail={nbeisert@itp.phys.ethz.ch}}
\hypersetup{pdfcontacturl={http://people.phys.ethz.ch/\xmptilde nbeisert/}}

\newcommand{\secref}[1]{\hyperref[#1]{section \ref*{#1}}}

\parskip1ex
\parindent0pt
\let\olditemize\itemize
\def\itemize{\olditemize\parskip0pt}

\begin{document}

\title{The \textsf{childdoc} Package}
\hypersetup{pdftitle={The childdoc Package}}
\author{Niklas Beisert\\[2ex]
  Institut f\"ur Theoretische Physik\\
  Eidgen\"ossische Technische Hochschule Z\"urich\\
  Wolfgang-Pauli-Strasse 27, 8093 Z\"urich, Switzerland\\[1ex]
  \href{mailto:nbeisert@itp.phys.ethz.ch}
  {\texttt{nbeisert@itp.phys.ethz.ch}}}
\hypersetup{pdfauthor={Niklas Beisert}}
\hypersetup{pdfsubject={Manual for the LaTeX2e Package childdoc}}
\date{30 December 2018, \textsf{v2.0}}
\maketitle

\begin{abstract}\noindent
\textsf{childdoc} is a \LaTeXe{} package
that enables the direct compilation
of document sections included by |\include|
to individual files.
\end{abstract}

\begingroup
\parskip0ex
\tableofcontents
\endgroup

%%%%%%%%%%%%%%%%%%%%%%%%%%%%%%%%%%%%%%%%%%%%%%%%%%%%%%%%%%%%%%%%%%%%%%%%%%%%%%%%
%%%%%%%%%%%%%%%%%%%%%%%%%%%%%%%%%%%%%%%%%%%%%%%%%%%%%%%%%%%%%%%%%%%%%%%%%%%%%%%%
\section{Introduction}

\LaTeX{} provides a mechanism to structure a large document (such as a book)
into a main file and several child files (containing the chapters)
using the |\include| command.
This mechanism is beneficial for documents
which span hundreds of pages in order to
make the source file(s) more manageable.
Moreover, compilation can be restricted to
selected child files by means of the |\includeonly| command.
The latter feature can be used to reduce the compilation time while editing
(this was significantly more useful in the earlier days of \LaTeX{})
or to generate a smaller document which is easier to navigate.
Another application of |\includeonly| is to generate
documents consisting of selected parts of the complete document.

However, there are a few drawbacks of the plain |\include| mechanism:
\begin{itemize}
\item
The child files cannot be compiled on their own,
they can only be compiled via the main file.
A naive editing environment
(such as a text editor with an option
to have the current file processed by \LaTeX)
may require one to switch to the main file before compiling;
attempting to compile the child file produces errors.
\item
The main file must be modified (each time)
to adjust the |\includeonly| command
to the present needs. This easily leaves the main file in a messy state.
\item
The generated document will always carry the filename
of the main document. This is inconvenient if
several child files are to be compiled and
to be kept for distribution.
\end{itemize}

The present package provides a simple interface
to make child files individually compilable by \LaTeX{}.
Compiling a child file then has the same effect as compiling
the main file with an |\includeonly| command
to select the appropriate child.
Moreover the generated document will carry the name of the child
rather than the main file.
This resolves all three above issues.

This feature is meant to make the editing of books,
thesis documents and lecture notes somewhat more convenient.
However, the package can also be used efficiently for
composing a series of documents (such as exercise sheets)
which are typically distributed individually.
It then assists the author in generating the individual documents
(potentially in different versions)
as well as a document containing the collected series.
Another application is in developing style files
or other kinds of included material
where compilation of the style file could redirect
to a sample or test file.

%%%%%%%%%%%%%%%%%%%%%%%%%%%%%%%%%%%%%%%%%%%%%%%%%%%%%%%%%%%%%%%%%%%%%%%%%%%%%%%%
%%%%%%%%%%%%%%%%%%%%%%%%%%%%%%%%%%%%%%%%%%%%%%%%%%%%%%%%%%%%%%%%%%%%%%%%%%%%%%%%
\section{Usage}

First of all, the package \textsf{childdoc} is \emph{not} a standard
\LaTeXe{} |.sty| style file! Therefore it needs to be invoked in
a non-standard way.

%%%%%%%%%%%%%%%%%%%%%%%%%%%%%%%%%%%%%%%%%%%%%%%%%%%%%%%%%%%%%%%%%%%%%%%%%%%%%%%%
\subsection{Included Files}
\label{sec:include}

%%%%%%%%%%%%%%%%%%%%%%%%%%%%%%%%%%%%%%%%
\DescribeMacro{\childdocmain}
To use the package, add the commands
\begin{center}
\begin{tabular}{l}
|\input{childdoc.def}|\\
|\childdocmain{}|\\
\end{tabular}
\end{center}
at the very top of the main \LaTeX{} file,
in particular \emph{before} the |\documentclass| statement!
The argument of |\childdocmain| should be left empty
(but it must be present).

%%%%%%%%%%%%%%%%%%%%%%%%%%%%%%%%%%%%%%%%
\DescribeMacro{\childdocof}
Furthermore, add the commands
\begin{center}
\begin{tabular}{l}
|\input{childdoc.def}|\\
|\childdocof{|\textit{main}|}|\\
\end{tabular}
\end{center}
at the top of every child file \textit{child}
which is included by |\include{|\textit{child}|}|
from within the main file
(or at least for those files to be compiled individually).
The argument \textit{main} must be the filename of the main file.

There are a couple of
considerations in setting up the main and child documents:

%%%%%%%%%%%%%%%%%%%%%%%%%%%%%%%%%%%%%%%%
\paragraph{Restrictions.}

Please note the following restrictions:
\begin{itemize}
\item
|\childdocmain| must be called with one argument \textit{main}
to ensure compatibility with earlier version of the package.
It must either be empty (|\childdocmain{}|)
or precisely match the filename of the main file in which it is specified.
See \secref{sec:detection} for further information.
\item
The filename \textit{main} must be specified without the |.tex| extension.
\item
The filename \textit{main} is case sensitive
(even in case-insensitive file systems)
due to internal string comparison.
\item
The argument \textit{main} should be fully expanded, it cannot be a macro.
\item
Subdirectories and special characters should be avoided in filenames.
\item
The command |\childdocmain{|\textit{main}|}| must be followed by a whitespace.
It should not be followed immediately by another command
or by a comment mark `|%|'.
This is because the \TeX{} parser reads the token immediately following
the argument of |\childdocmain| and puts it
at the beginning of every child section;
however, a white\-space is ignored.
\end{itemize}

%%%%%%%%%%%%%%%%%%%%%%%%%%%%%%%%%%%%%%%%
\paragraph{Content of Main File.}

It is advisable to place all content in the child files included by |\include|.
Any output contained in the main file will appear in all child documents
unless suppressed manually;
it cannot be suppressed automatically by the |\includeonly| directive
and thus should normally be avoided.
A method to include some content in the main file
by means of conditional processing is described in \secref{sec:conditional}.

%%%%%%%%%%%%%%%%%%%%%%%%%%%%%%%%%%%%%%%%
\paragraph{Page Numbering.}

When only a part of the document is compiled,
the appropriate numbering of pages
(as well as other status parameters)
is determined from the |.aux| files.
The latter contain information from previous passes.
However this information needs to propagate through
all intermediate child documents.
Therefore the page numbering in child documents may well
be inconsistent until the complete document is compiled at least once.

A useful (if unconventional) way to always ensure a consistent
page numbering is to restart the numbering in each child document
and denote the pages by `\textit{child}|.|\textit{page}'
where \textit{child} represents the chapter/section number of the child file.
This can be achieved by the command
|\numberwithin{page}{|\textit{child}|}|
of the \textsf{amsmath} package
where \textit{child} can be |chapter| or |section|
depending on the chosen structuring.
Alternatively, one can modify the macro |\thepage| appropriately
and reset the counter |page| at the start of each child file.

%%%%%%%%%%%%%%%%%%%%%%%%%%%%%%%%%%%%%%%%%%%%%%%%%%%%%%%%%%%%%%%%%%%%%%%%%%%%%%%%
\subsection{Conditional Processing}
\label{sec:conditional}

The package provides a mechanism to compile different versions
of a document. To customise the versions further some conditional processing
can come in handy to distinguish which version is being compiled.
The package provides two macros to describe the compilation context:

%%%%%%%%%%%%%%%%%%%%%%%%%%%%%%%%%%%%%%%%
\DescribeMacro{\ifchilddoc}
The conditional |\ifchilddoc| distinguishes between the compilation of
child documents and the main document:
%
\begin{center}
|\ifchilddoc |\textit{child-code}| |[|\||else |\textit{main-code}]| \||fi|
\end{center}

%%%%%%%%%%%%%%%%%%%%%%%%%%%%%%%%%%%%%%%%
\DescribeMacro{\childdocname}
\DescribeMacro{\childdocjob}
The macro |\childdocname| contains the filename (without extension)
of the main or child file being processed.
Note that |\childdocjob| will always contain the name of the main file.

%%%%%%%%%%%%%%%%%%%%%%%%%%%%%%%%%%%%%%%%
\paragraph{Title Page.}

Conditional processing can be used to include a title or banner page
in the main document when proper precautions are taken.
Importantly, the code in the main file should ensure that the page counter
(as well as other status parameters which are stored in the |.aux| files)
takes the same value after the conditional processing.
Otherwise the page numbers may take divergent values
depending on which part is compiled.

For example, a title page could be declared by:
%
\begin{center}
\begin{tabular}{l}
|\ifchilddoc\||else|\\
|\addtocounter{page}{-1}|\\
\textit{code for title page}\\
|\newpage|\\
|\||fi|
\end{tabular}
\end{center}
%
A banner page for the child documents can be generated by:
%
\begin{center}
\begin{tabular}{l}
|\ifchilddoc|\\
|\addtocounter{page}{-1}|\\
\textit{code for banner page}\\
|\newpage|\\
|\||fi|
\end{tabular}
\end{center}
%
Here one could write a message such as:
\begin{center}
|This is the part \childdocname{} of \childdocjob{}.|
\end{center}

%%%%%%%%%%%%%%%%%%%%%%%%%%%%%%%%%%%%%%%%%%%%%%%%%%%%%%%%%%%%%%%%%%%%%%%%%%%%%%%%
\subsection{Flags}
\label{sec:flags}

The package makes it easy to generate different versions
of the main or child documents.
To this end compilation flags can be defined
and assigned different default values.
They will be particularly useful in conjunction
with the forwarding mechanism described in \secref{sec:forward}.

For example, it may be useful to have a flag |\version|
which can be set to |draft| or |final|.
The document source will contain some conditional code
depending on the value of |\version|.
Suppose further, the flag should default to |final| for the main file
and to |draft| for child files
which is a natural assignment for editing the document.
This is achieved by placing the following code
in the preamble of the main document
(below the |\childdocmain| directive):
%
\begin{center}
\begin{tabular}{l}
|\ifchilddoc|\\
|\providecommand{\version}{draft}|\\
|\||else|\\
|\providecommand{\version}{final}|\\
|\||fi|
\end{tabular}
\end{center}
%
The definition by |\providecommand| makes sure
that previous definitions are not overwritten.
Further statements |\providecommand{\version}{...}|
can thus be added before the above code to override it.

For the main file, one might add a line
(between |\childdocmain| and the above block)
%
\begin{center}
|%\ifchilddoc\||else\providecommand{\version}{draft}\||fi|
\end{center}
%
which can be uncommented to produce a draft version.
Likewise one can add a line to the very top of a child file
(above the |\childdocof{|\textit{main}|}| directive)
%
\begin{center}
|%\providecommand{\version}{final}|
\end{center}
%
which can be uncommented to produce the final version of this child document.

%%%%%%%%%%%%%%%%%%%%%%%%%%%%%%%%%%%%%%%%%%%%%%%%%%%%%%%%%%%%%%%%%%%%%%%%%%%%%%%%
\subsection{Forwarding}
\label{sec:forward}

Different versions of the main or child documents
using compilation flags as described in \secref{sec:flags}
can be (permanently) stored in different files
for convenient compilation, viewing and distribution.
To this end, the package defines a command
to pass on compilation to a different file:

%%%%%%%%%%%%%%%%%%%%%%%%%%%%%%%%%%%%%%%%
\DescribeMacro{\childdocforward}
The command |\childdocforward| redirects processing to
another source file:
%
\begin{center}
\begin{tabular}{l}
|\input{childdoc.def}|\\
|\childdocforward[|\textit{main}|]{|\textit{dest}|}|\\
\end{tabular}
\end{center}
%
The argument \textit{dest} is the destination file
(without extension).
It should be the main file or one of the child files.
Note that further \textsf{childdoc} directives
such as |\childdocof| and |\childdocforward|
in the indicated file will be processed in this form.
The optional argument \textit{main}
passes on directly to the main file \textit{main}
while pretending to compile the child \textit{dest}.
This form behaves as if \textit{dest}
issues |\childdocof{|\textit{main}|}| right away,
and no further \textsf{childdoc} directives will be processed.

%%%%%%%%%%%%%%%%%%%%%%%%%%%%%%%%%%%%%%%%
\DescribeMacro{\...prefix}
In the alternative form |\childdocforwardprefix|,
%
\begin{center}
\begin{tabular}{l}
|\input{childdoc.def}|\\
|\childdocforwardprefix[|\textit{main}|]{|\textit{prefix}|}{|\textit{dest}|}|
\end{tabular}
\end{center}
%
the destination file is determined by a pattern
depending on the current file:
To make this work, the current file must be called
`{\textit{prefix}\hspace{0.2em}\textit{suffix}}'
with \textit{prefix} matching precisely the argument.
Processing is then passed on to the file
`{\textit{dest}\hspace{0.2em}\textit{suffix}}'.
Surely, the same effect is achieved by
directly specifying the
argument `{\textit{dest}\hspace{0.2em}\textit{suffix}}'
in the first form.
However, that requires to set up a different file
for each child. With the alternative form of the command
all these files can have exactly the same content
which simplifies setting them up and maintaining them.

For example, the following file |draft.tex|
with a compilation flag |\version| as described in \secref{sec:flags}
compiles the main document as a draft:
%
\begin{center}
\begin{tabular}{l}
|\def\version{draft}|\\
|\input{childdoc.def}|\\
|\childdocforward{|\textit{main}|}|
\end{tabular}
\end{center}
%
Likewise, the following files |final|\textit{nn}|.tex|
compile the final version of the child document
|child|\textit{nn}|.tex|:
%
\begin{center}
\begin{tabular}{l}
|\def\version{final}|\\
|\input{childdoc.def}|\\
|\childdocforwardprefix{final}{child}|
\end{tabular}
\end{center}
%

Note that when several versions of a main file and/or of each child file
are to be generated, it may be convenient to set up a |Makefile| or
shell script to automatise the process.

%%%%%%%%%%%%%%%%%%%%%%%%%%%%%%%%%%%%%%%%%%%%%%%%%%%%%%%%%%%%%%%%%%%%%%%%%%%%%%%%
\subsection{Command Line Processing}
\label{sec:commandline}

The effect of redirection files can also be achieved by invoking
the \LaTeX{} compiler with a more elaborate command line.
Most conveniently this should be done as part
of a shell script or a |Makefile|.

When using \textsf{childdoc} in the main file, the following
command lines effectively perform a redirection
(note that depending on the shell being used,
backslashes may have to be doubled: `|\|' $\to$ `|\\|'):
%
\begin{center}
|... -jobname "|\textit{target}|" |\\|"|[\textit{flags}]%
|\input{childdoc.def}\childdocforward[|\textit{main}|]{|\textit{dest}|}"|
\end{center}
%
Here \textit{target} is the name of the output file,
\textit{main} is the name of the main file
and \textit{dest} is the name of the main or child file to be processed
(all filenames without extensions).
The optional argument \textit{main} can be omitted
if \textit{main} matches \textit{dest}.
Optionally, compilation \textit{flags} can be defined via |\def| commands.
This command line makes the \TeX{} engine believe
it is compiling the file \textit{target}
whose content is specified as the latter parameter.
The provided code then forwards the processing to
\textit{main} or \textit{dest} as described in \secref{sec:forward}.

%%%%%%%%%%%%%%%%%%%%%%%%%%%%%%%%%%%%%%%%%%%%%%%%%%%%%%%%%%%%%%%%%%%%%%%%%%%%%%%%
\subsection{Include by Input}
\label{sec:input}

Including child documents by |\include| has some restrictions by design.
Most notably, the content of a child document always occupies
its own set of pages; pages cannot be shared between child documents.
Usually, this behaviour makes perfect sense
because each child document contain an essential part of the document.
However, in some situations it may be desirable to compose
a document from a collection of parts
without having mandatory page breaks between then.
For this case, the package
provides a mechanism to include parts
by |\input| which can also be processed individually.
However, by construction this mechanism
requires manual handling of the content to be output.

%%%%%%%%%%%%%%%%%%%%%%%%%%%%%%%%%%%%%%%%
\DescribeMacro{\ifchilddocmanual}
The main file should be prepared as usual, see \secref{sec:include}.
However, the document body must make a distinction
between processing of an individual part and of the main document, e.g.:
%
\begin{center}
\begin{tabular}{l}
|\ifchilddocmanual|\\
|\input{\childdocname}|\\
|\||else|\\
\textit{document body with }|\input{|\textit{part}|}|\\
|\||fi|
\end{tabular}
\end{center}
%
The conditional |\ifchilddocmanual| is true whenever
a part to be included by |\input| is being compiled,
and the name of the part is stored in |\childdocname|.

%%%%%%%%%%%%%%%%%%%%%%%%%%%%%%%%%%%%%%%%
\DescribeMacro{\childdocby}
Each part to be included by |\input| should start with:
%
\begin{center}
\begin{tabular}{l}
|\input{childdoc.def}|\\
|\childdocby{|\textit{main}|}|\\
\end{tabular}
\end{center}
%
The directive |\childdocby| is similar to |\childdocof|
described in \secref{sec:include},
but the subsequent selection of content must be done manually.
To that end, both |\ifchilddoc| and |\ifchilddocmanual|
will be true upon processing of a part,
and the name of the part is stored in |\childdocname|.
Note that |\jobname| will be set to the filename of the current part
so that each part receives an individual |.aux| file
that does not interfere with the |.aux| file(s) of the main document.
This behaviour can be altered by the alternative form
|\childdocby[*]{|\textit{main}|}| (with a non-empty optional argument)
which uses the |.aux| file of the main document
by setting |\jobname| to \textit{main}.

%%%%%%%%%%%%%%%%%%%%%%%%%%%%%%%%%%%%%%%%%%%%%%%%%%%%%%%%%%%%%%%%%%%%%%%%%%%%%%%%
\subsection{Driver Development}
\label{sec:driver}

The \textsf{childdoc} mechanism can also be use for the development
of definition files such as \LaTeX{} styles or classes.
This case differs from the above setup with multiple parts
included by |\include| in that no |\includeonly| should be invoked.
This can be achieved by starting the include file
(before |\ProvidesPackage|) with:
%
\begin{center}
\begin{tabular}{l}
|\input{childdoc.def}|\\
|\childdocforward{|\textit{main}|}|\\
\end{tabular}
\end{center}
%
or alternatively with:
%
\begin{center}
\begin{tabular}{l}
|\input{childdoc.def}|\\
|\childdocby{|\textit{main}|}|\\
\end{tabular}
\end{center}
%
Both forms have slightly different effects as described above.
The main file is prepared as usual, see \secref{sec:include}.

%%%%%%%%%%%%%%%%%%%%%%%%%%%%%%%%%%%%%%%%%%%%%%%%%%%%%%%%%%%%%%%%%%%%%%%%%%%%%%%%
\subsection{Legacy Detection}
\label{sec:detection}

The directive |\childdocmain| in the main file can detect
whether the complete document or merely a child is to be compiled
even without using the directive |\childdocof|.
This method is deprecated because it is less robust
and there is no compelling reason to use it;
it is merely provided for backward compatibility
and it may be removed in future versions.

If the detection mechanism is to be used,
it is mandatory to correctly specify
the filename of the main file as the argument of |\childdocmain|:
%
\begin{center}
\begin{tabular}{l}
|\input{childdoc.def}|\\
|\childdocmain{|\textit{main}|}|\\
\end{tabular}
\end{center}
%
If |\jobname| does not match the argument \textit{main} of |\childdocmain|,
it is assumed that |\jobname| points to the child file to be compiled.
When using |\childdocmain| with the main file specified as argument,
it suffices to start a child file
with just |\input{|\textit{main}|}|
without loading of the package and using |\childdocof|.
If instead all processing is done
with the appropriate \textsf{childdoc} directives,
the argument of \textit{main} of |\childdocmain| can be empty.

An alternative version of the command line processing described
in \secref{sec:commandline} using the detection mechanism reads:
%
\begin{center}
|... -jobname "|\textit{target}|" "|[\textit{flags}]%
[|\def\jobname{|\textit{dest}|}|]|\input{|\textit{main}|}"|
\end{center}

%%%%%%%%%%%%%%%%%%%%%%%%%%%%%%%%%%%%%%%%%%%%%%%%%%%%%%%%%%%%%%%%%%%%%%%%%%%%%%%%
\subsection{Manual Code}
\label{sec:manual}

In case one cannot be certain whether the definitions file |childdoc.def|
is installed on the target \TeX{} distribution
and one prefers not to ship it,
it is conceivable to paste a few relevant commands into the sources.

To that end, drop all statements |\input{childdoc.def}|
and perform the replacements as outlined below.
Instead of |\childdocmain{|\textit{main}|}| add the following code
to the top of the main file:
%
\begin{center}
\begin{tabular}{l}
|\||ifdefined\childdocname\endinput\||fi\newif\ifchilddoc|\\
|\edef\childdocname{\scantokens\expandafter{\jobname\noexpand}}|\\
|\def\childdocmain{|\textit{main}|}\||ifx\childdocmain\childdocname\||else|\\
|\childdoctrue\includeonly{\childdocname}\let\jobname\childdocmain\||fi|\\
\end{tabular}
\end{center}
%
Instead of |\childdocof{|\textit{main}|}| just include the main file
at the top of each child file:
%
\begin{center}
|\input{|\textit{main}|}|
\end{center}
%
A simple redirection |\childdocforward{|\textit{dest}|}| is achieved by:
%
\begin{center}
|\def\jobname{|\textit{dest}|}\input{\jobname}|
\end{center}
%
The redirection with prefix
|\childdocforwardprefix[|\textit{prefix}|]{|\textit{dest}|}|
is accomplished by:
%
\begin{center}
\begin{tabular}{l}
|{\edef\jobname{\scantokens\expandafter{\jobname\noexpand}}|\\
|\def\redirectjob |\textit{prefix}|#1~~~{\gdef\jobname{|\textit{dest}|#1}}|\\
|\expandafter\redirectjob\jobname~~~}\input{\jobname}|
\end{tabular}
\end{center}

In an alternative approach,
child documents can be compiled by a specific command line
without additional code or specific definitions:
%
\begin{center}
|... -jobname "|\textit{target}|" "|[\textit{flags}]%
|\includeonly{|\textit{dest}|}\input{|\textit{main}|}"|
\end{center}
%

%%%%%%%%%%%%%%%%%%%%%%%%%%%%%%%%%%%%%%%%%%%%%%%%%%%%%%%%%%%%%%%%%%%%%%%%%%%%%%%%
%%%%%%%%%%%%%%%%%%%%%%%%%%%%%%%%%%%%%%%%%%%%%%%%%%%%%%%%%%%%%%%%%%%%%%%%%%%%%%%%
\section{Information}

%%%%%%%%%%%%%%%%%%%%%%%%%%%%%%%%%%%%%%%%%%%%%%%%%%%%%%%%%%%%%%%%%%%%%%%%%%%%%%%%
\subsection{Copyright}

Copyright \copyright{} 2017--2018 Niklas Beisert

This work may be distributed and/or modified under the
conditions of the \LaTeX{} Project Public License, either version 1.3
of this license or (at your option) any later version.
The latest version of this license is in
  \url{http://www.latex-project.org/lppl.txt}
and version 1.3 or later is part of all distributions of \LaTeX{}
version 2005/12/01 or later.

This work has the LPPL maintenance status `maintained'.

The Current Maintainer of this work is Niklas Beisert.

This work consists of the files |README.txt|, |childdoc.ins| and |childdoc.dtx|
as well as the derived files |childdoc.def|, |cdocsamp.tex|
with |cdocsch1.tex|, |cdocsch2.tex|, |cdocspt3.tex|, |cdocspt4.tex|,
|cdocsdrf.tex|, |cdocsfn1.tex|, |cdocsfn2.tex|
as well as |childdoc.pdf|.

%%%%%%%%%%%%%%%%%%%%%%%%%%%%%%%%%%%%%%%%%%%%%%%%%%%%%%%%%%%%%%%%%%%%%%%%%%%%%%%%
\subsection{Files and Installation}

The package consists of the files:
%
\begin{center}
\begin{tabular}{ll}
    |README.txt|   & readme file \\
    |childdoc.ins| & installation file \\
    |childdoc.dtx| & source file \\
    |childdoc.def| & definition file \\
    |cdocsamp.tex| & sample main file \\
    |cdocsch1.tex| & sample include file \\
    |cdocsch2.tex| & sample include file \\
    |cdocspt3.tex| & sample part file \\
    |cdocspt4.tex| & sample part file \\
    |cdocsdrf.tex| & sample redirection file \\
    |cdocsfn1.tex| & sample redirection file \\
    |cdocsfn2.tex| & sample redirection file \\
    |childdoc.pdf| & manual
\end{tabular}
\end{center}
%
The distribution consists of the files
|README.txt|, |childdoc.ins| and |childdoc.dtx|.
%
\begin{itemize}
\item
Run (pdf)\LaTeX{} on |childdoc.dtx|
to compile the manual |childdoc.pdf| (this file).
\item
Run \LaTeX{} on |childdoc.ins| to create the definitions file |childdoc.def|
and the sample |cdocsamp.tex| with include files
|cdocsch1.tex|, |cdocsch2.tex|, |cdocspt3.tex|, |cdocspt4.tex|,
|cdocsdrf.tex|, |cdocsfn1.tex|, |cdocsfn2.tex|.
Then copy the file |childdoc.def| to an appropriate directory of your \LaTeX{}
distribution, e.g.\ \textit{texmf-root}|/tex/latex/childdoc|.
\end{itemize}

%%%%%%%%%%%%%%%%%%%%%%%%%%%%%%%%%%%%%%%%%%%%%%%%%%%%%%%%%%%%%%%%%%%%%%%%%%%%%%%%
\subsection{Related CTAN Packages}

There are several other packages which offer a similar functionality:
%
\begin{itemize}
\item
The packages
\href{http://ctan.org/pkg/docmute}{\textsf{docmute}},
\href{http://ctan.org/pkg/includex}{\textsf{includex}} and
\href{http://ctan.org/pkg/standalone}{\textsf{standalone}}
provide commands to include only the document body of
a child file thus allowing both files to be compiled individually.
\item
The packages \href{http://ctan.org/pkg/subdocs}{\textsf{subdocs}}
and \href{http://ctan.org/pkg/subfiles}{\textsf{subfiles}}
provide structures in which the main and child documents can be
encapsulated and allowing them to be compiled individually.
The inclusion mechanism is different from the conventional |\include|.
\item
The package \href{http://ctan.org/pkg/combine}{\textsf{combine}}
is an elaborate solution to combine several documents into one.
\end{itemize}
%
See also the CTAN topic \href{http://ctan.org/topic/subdocs}{\textsf{subdocs}}
for further related packages.
The present package differs from the above solutions in that
a document structure constructed with the conventional |\include| mechanism
just needs two extra commands at the top of every file
such that all constituent files can be compiled individually.

%%%%%%%%%%%%%%%%%%%%%%%%%%%%%%%%%%%%%%%%%%%%%%%%%%%%%%%%%%%%%%%%%%%%%%%%%%%%%%%%
%\subsection{Feature Suggestions}
%
%The following is a list of features which may be useful for future
%versions of this package:
%%
%\begin{itemize}
%\item
%\ldots
%\end{itemize}

%%%%%%%%%%%%%%%%%%%%%%%%%%%%%%%%%%%%%%%%%%%%%%%%%%%%%%%%%%%%%%%%%%%%%%%%%%%%%%%%
\subsection{Revision History}

%%%%%%%%%%%%%%%%%%%%%%%%%%%%%%%%%%%%%%%%
\paragraph{v2.0:} 2018/12/30

\begin{itemize}
\item
immediate forward processing
\item
added |\childdocby| mechanism
\item
manual restructured
\end{itemize}

%%%%%%%%%%%%%%%%%%%%%%%%%%%%%%%%%%%%%%%%
\paragraph{v1.6:} 2018/01/17

\begin{itemize}
\item
application for development of include files
\item
corrections to manual
\end{itemize}

%%%%%%%%%%%%%%%%%%%%%%%%%%%%%%%%%%%%%%%%
\paragraph{v1.5:} 2017/05/21

\begin{itemize}
\item
more complete structuring introduced
\item
|\childdocof| introduced
\item
|\childdoc| renamed to |\childdocmain|
\item
|\childredirect| renamed to |\childdocforward| and |\childdocforwardprefix|
and functionality expanded
\end{itemize}

%%%%%%%%%%%%%%%%%%%%%%%%%%%%%%%%%%%%%%%%
\paragraph{v1.0:} 2017/04/27

\begin{itemize}
\item
manual and install package
\item
first version published on CTAN
\end{itemize}

%%%%%%%%%%%%%%%%%%%%%%%%%%%%%%%%%%%%%%%%
\paragraph{v0.6:} 2017/04/26

\begin{itemize}
\item
redirection mechanism added
\end{itemize}

%%%%%%%%%%%%%%%%%%%%%%%%%%%%%%%%%%%%%%%%
\paragraph{v0.5:} 2017/04/26

\begin{itemize}
\item
functionality in definition file
\end{itemize}


%%%%%%%%%%%%%%%%%%%%%%%%%%%%%%%%%%%%%%%%%%%%%%%%%%%%%%%%%%%%%%%%%%%%%%%%%%%%%%%%
%%%%%%%%%%%%%%%%%%%%%%%%%%%%%%%%%%%%%%%%%%%%%%%%%%%%%%%%%%%%%%%%%%%%%%%%%%%%%%%%
%%%%%%%%%%%%%%%%%%%%%%%%%%%%%%%%%%%%%%%%%%%%%%%%%%%%%%%%%%%%%%%%%%%%%%%%%%%%%%%%
\appendix

\settowidth\MacroIndent{\rmfamily\scriptsize 000\ }

 \DocInput{childdoc.dtx}

\end{document}
%</driver>
% \fi
%
% %%%%%%%%%%%%%%%%%%%%%%%%%%%%%%%%%%%%%%%%%%%%%%%%%%%%%%%%%%%%%%%%%%%%%%%%%%%%%%
% %%%%%%%%%%%%%%%%%%%%%%%%%%%%%%%%%%%%%%%%%%%%%%%%%%%%%%%%%%%%%%%%%%%%%%%%%%%%%%
% \section{Sample}
%\iffalse
%<*samplemain>
%\fi
%
% The following presents a sample document
% with two chapters, two parts, a title page,
% a compile flag as well as three forwarding files to set the flag.
% It consists of eight |.tex| files:
% \begin{center}
% \begin{tabular}{ll}
% |cdocsamp.tex|&main file\\
% |cdocsch1.tex|&include file for chapter 1\\
% |cdocsch2.tex|&include file for chapter 2\\
% |cdocspt3.tex|&include file for part 3\\
% |cdocspt4.tex|&include file for part 4\\
% |cdocsdrf.tex|&forwarding file for main file in draft mode\\
% |cdocsfi1.tex|&forwarding file for final version of chapter 1\\
% |cdocsfi2.tex|&forwarding file for final version of chapter 2\\
% \end{tabular}
% \end{center}
% Each of the eight files can be compiled directly by the \LaTeX{} compiler.
%
% %%%%%%%%%%%%%%%%%%%%%%%%%%%%%%%%%%%%%%
% \paragraph{Main File.}
%
% The main file is called |cdocsamp.tex|.
%
% Load the \textsf{childdoc} definitions and
% declare the filename for the main document:
%    \begin{macrocode}
\input{childdoc.def}
\childdocmain{}
%    \end{macrocode}

% Optional override for |\version| flag:
%    \begin{macrocode}
%%\ifchilddoc\else\providecommand{\version}{draft}\fi
%    \end{macrocode}

% Define the default values for the |\version| flag
% (|final| for the main file and |draft| for childs):
%    \begin{macrocode}
\ifchilddoc
\providecommand{\version}{draft}
\else
\providecommand{\version}{final}
\fi
%    \end{macrocode}

% Load the standard document class:
%    \begin{macrocode}
\documentclass[12pt]{article}
%    \end{macrocode}

% Start the document body:
%    \begin{macrocode}
\begin{document}
%    \end{macrocode}

% Declare a title page.
% Print title, part of document being processed and version flag:
%    \begin{macrocode}
\addtocounter{page}{-1}
\begin{center}
{\LARGE\bfseries{}childdoc example\par}
\vspace{1cm}
\ifchilddoc
\ifchilddocmanual part\else chapter\fi:
`\childdocname' of `\childdocjob'\par
\else
main document: `\childdocjob'\par
\fi
version: \version\par
\end{center}
\newpage
%    \end{macrocode}

% Manually include selected file,
% otherwise process as usual:
%    \begin{macrocode}
\ifchilddocmanual
\section*{part `\childdocname'}
\input{\childdocname}
\else
%    \end{macrocode}

% Include the two chapters:
%    \begin{macrocode}
\include{cdocsch1}
\include{cdocsch2}
%    \end{macrocode}

% Include the two parts unless only chapters should be displayed:
%    \begin{macrocode}
\ifchilddoc\else
\section{part three}
\input{cdocspt3}
\section{part four}
\input{cdocspt4}
\fi
%    \end{macrocode}

% Process as usual until here:
%    \begin{macrocode}
\fi
%    \end{macrocode}

% End of document body:
%    \begin{macrocode}
\end{document}
%    \end{macrocode}
%\iffalse
%</samplemain>
%\fi
%
% %%%%%%%%%%%%%%%%%%%%%%%%%%%%%%%%%%%%%%
% \paragraph{Chapter Include Files.}
%
% The include files are called |cdocsch1.tex| and |cdocsch2.tex|.
%
%\iffalse
%<*samplechap1|samplechap2>
%\fi

% Optional override for |\version| flag:
%    \begin{macrocode}
%%\providecommand{\version}{final}
%    \end{macrocode}

% Include the main document:
%    \begin{macrocode}
\input{childdoc.def}
\childdocof{cdocsamp}
%    \end{macrocode}

%\iffalse
%</samplechap1|samplechap2>
%\fi
%
%\iffalse
%<*samplechap1>
%\fi
% Some text for chapter 1:
%    \begin{macrocode}
\section{one}
some text in chapter one
%    \end{macrocode}

%\iffalse
%</samplechap1>
%\fi
% Some text for chapter 2:
%\iffalse
%<*samplechap2>
%\fi
%    \begin{macrocode}
\section{two}
more text in chapter two
%    \end{macrocode}

%\iffalse
%</samplechap2>
%\fi
%
% %%%%%%%%%%%%%%%%%%%%%%%%%%%%%%%%%%%%%%
% \paragraph{Part Include Files.}
%
% The include files are called |cdocspt3.tex| and |cdocspt4.tex|.
%
%\iffalse
%<*samplepart3|samplepart4>
%\fi

% Optional override for |\version| flag:
%    \begin{macrocode}
%%\providecommand{\version}{final}
%    \end{macrocode}

% Include the main document:
%    \begin{macrocode}
\input{childdoc.def}
\childdocby{cdocsamp}
%    \end{macrocode}

%\iffalse
%</samplepart3|samplepart4>
%\fi
%
%\iffalse
%<*samplepart3>
%\fi
% Some text for part 3:
%    \begin{macrocode}
some text in part three
%    \end{macrocode}

%\iffalse
%</samplepart3>
%\fi
% Some text for part 4:
%\iffalse
%<*samplepart4>
%\fi
%    \begin{macrocode}
more text in part four
%    \end{macrocode}

%\iffalse
%</samplepart4>
%\fi
%
% %%%%%%%%%%%%%%%%%%%%%%%%%%%%%%%%%%%%%%
% \paragraph{Forwarding for a Complete Draft.}
%
% The following forwarding file |cdocsdrf.tex|
% compiles the main document in draft mode:
%\iffalse
%<*sampledraft>
%\fi
%    \begin{macrocode}
\def\version{draft}
\input{childdoc.def}
\childdocforward{cdocsamp}
%    \end{macrocode}

%\iffalse
%</sampledraft>
%\fi
%
% %%%%%%%%%%%%%%%%%%%%%%%%%%%%%%%%%%%%%%
% \paragraph{Forwarding for Final Version of the Chapters.}
%
% The following forwarding files |cdocsfn1.tex| and |cdocsfn2.tex|
% (with identical content)
% compile the final versions of the child documents
% |cdocsch1.tex| and |cdocsch2.tex|, respectively:
%\iffalse
%<*samplefinal>
%\fi
%    \begin{macrocode}
\def\version{final}
\input{childdoc.def}
\childdocforwardprefix[cdocsamp]{cdocsfn}{cdocsch}
%    \end{macrocode}

%\iffalse
%</samplefinal>
%\fi
%
% %%%%%%%%%%%%%%%%%%%%%%%%%%%%%%%%%%%%%%
% \paragraph{Command Line Processing.}
%
% The following three command lines generate the output files
% |cdocscld|, |cdocscl1| and |cdocscl2|
% which should be identical to
% |cdocsdrf|, |cdocsch1| and |cdocsfn2|, respectively:
% \begin{center}
% \begin{tabular}{l}
% |latex -jobname cdocscld \|\\
% |  "\def\version{draft}\input{childdoc.def}\childdocforward{cdocsamp}"|\\
% |latex -jobname cdocscl1 \|\\
% |  "\input{childdoc.def}\childdocforward[cdocsamp]{cdocsch1}"|\\
% |latex -jobname cdocscl2 \|\\
% |  "\def\version{final}\input{childdoc.def}\childdocforward{cdocsch2}"|
% \end{tabular}
% \end{center}
% Note that the trailing backslash on each first line
% merely continues the input to the second line
% (for convenient cut ant paste).
% Furthermore, the command |latex| can be replaced by any
% of its alternative versions such as |pdflatex|.
%
% %%%%%%%%%%%%%%%%%%%%%%%%%%%%%%%%%%%%%%%%%%%%%%%%%%%%%%%%%%%%%%%%%%%%%%%%%%%%%%
% %%%%%%%%%%%%%%%%%%%%%%%%%%%%%%%%%%%%%%%%%%%%%%%%%%%%%%%%%%%%%%%%%%%%%%%%%%%%%%
% \section{Implementation}
%\iffalse
%<*package>
%\fi
%
% This section describes the definitions file |childdoc.def|.

% The definitions cannot be loaded using |\usepackage| or |\RequirePackage|
% which has a mechanism to prevent loading a style file more than once.
% When loading the definitions by means of |\input|
% multiple instances have to be prevented manually:
%\iffalse
%This code needs to be before the `\ProvidesFile' directive
%which is defined at the beginning of this file.
%Therefore it is also placed there and commented out here.
%</package>
%<*discard>
%\fi
%    \begin{macrocode}
\ifdefined\childdocmain\endinput\fi
%    \end{macrocode}
%\iffalse
%</discard>
%<*package>
%\fi
%
% \macro{\ifchilddoc}
% \macro{\ifchilddocmanual}
% The conditional |\ifchilddoc| tells whether a
% child (true) or main (false) document is being compiled.
% The conditional |\ifchilddocmanual| tells whether
% the |\includeonly| mechanism is used (false) or
% the selection of child files must be performed manually (true).
% The definitions initialise to false:
%    \begin{macrocode}
\newif\ifchilddoc
\newif\ifchilddocmanual
%    \end{macrocode}

% \macro{\childdocname}
% \macro{\childdocjob}
% The macro |\childdocname| stores the name of the main document
% to be compiled. The macro |\childdocjob| stores the name of
% the document on which the \LaTeX{} compiler was originally invoked.
% The content of |\jobname| cannot be compared
% to filenames specified in the source due to different catcodes.
% The following code rescans |\jobname|, stores the result
% in |\childdocname| and saves a copy in |\childdocjob|:
%    \begin{macrocode}
\edef\childdocname{\scantokens\expandafter{\jobname\noexpand}}
\let\childdocjob\childdocname
%    \end{macrocode}

% \macro{\childdocdisable}
% The macro |\childdocdisable| prevents the main file
% from being processed more than once.
% At this stage, the main document command |\childdocmain|
% is assumed to be called once again where it should do nothing.
% Any subsequent call to it should prevent
% a secondary processing of the main document
% It overwrites the forwarding commands
% |\childdocof| and |\childdocforward|
% with empty macros to prevent further inclusions of the main document:
%    \begin{macrocode}
\newcommand{\childdocdisable}
{
  \renewcommand{\childdocmain}[1]{\renewcommand{\childdocmain}[1]{\endinput}}
  \renewcommand{\childdocof}[1]{}
  \renewcommand{\childdocby}[2][]{}
  \renewcommand{\childdocforward}[2][]{}
  \renewcommand{\childdocdisable}{}
}
%    \end{macrocode}

% \macro{\childdocmain}
% The macro |\childdocmain| is to be called at the top of the main file
% with nothing or the main filename (without extension) as argument.
% First, it breaks loops.
% If the argument is not empty and does not match |\childdocname|
% (which is set by the first inclusion of |childdoc.def|),
% |\ifchilddoc| is set to true, |\includeonly| is applied to the child file
% and |\jobname| is set to the main file
% (for proper handling of |.aux| files):
%    \begin{macrocode}
\newcommand{\childdocmain}[1]
{
  \childdocdisable\childdocmain{}
  \if?#1?\else
    \begingroup
      \def\childdoctmp{#1}
      \ifx\childdoctmp\childdocname
        \def\childdoctmp{}
      \else
        \def\childdoctmp
        {
          \childdoctrue
          \includeonly{\childdocname}
          \def\childdocjob{#1}
          \def\jobname{#1}
        }
      \fi
      \expandafter
    \endgroup
    \childdoctmp
  \fi
}
%    \end{macrocode}

% \macro{\childdocof}
% The command |\childdocof| redirects
% compilation to the main file |#1|.
%    \begin{macrocode}
\newcommand{\childdocof}[1]
{
  \childdocdisable
  \childdoctrue
  \includeonly{\childdocname}
  \def\jobname{#1}
  \def\childdocjob{#1}
  \input{#1}
}
%    \end{macrocode}

% \macro{\childdocby}
% The command |\childdocby| ....
%    \begin{macrocode}
\newcommand{\childdocby}[2][]
{
  \childdocdisable
  \childdoctrue
  \childdocmanualtrue
  \if?#1?\else
    \def\jobname{#2}
  \fi
  \def\childdocjob{#2}
  \input{#2}
  \endinput
}
%    \end{macrocode}

% \macro{\childdocforward}
% The command |\childdocforward| redirects
% compilation to the main file or
% (if the optional argument is given) a child file.
% Parameters are set as if the main file
% or a child file starting with |\childdocof| was compiled.
% Then compilation is handed over to the main file:
%    \begin{macrocode}
\newcommand{\childdocforward}[2][]
{
  \begingroup
    \if?#1?
      \def\childdoctmp
      {
        \def\childdocname{#2}
        \def\childdocjob{#2}
        \def\jobname{#2}
        \input{#2}
        \endinput
      }
    \else
      \def\childdoctmp
      {
        \childdocdisable
        \def\childdocname{#2}
        \childdoctrue
        \includeonly{#2}
        \def\childdocjob{#1}
        \def\jobname{#1}
        \input{#1}
        \endinput
      }
    \fi
    \expandafter
  \endgroup
  \childdoctmp
}
%    \end{macrocode}

% \macro{\childdocforwardprefix}
% The command |\childdocforwardprefix| redirects
% compilation to the main or a child file by means of a pattern.
% The prefix |#1| in the current filename is replaced by |#2|
% and the suffix of the current filename is kept
% (it is assumed that the filename does not contain the substring `|~~~|'
% which is used as a delimiter).
% Compilation is handed over to the new file by |\childdocforward|:
%    \begin{macrocode}
\newcommand{\childdocforwardprefix}[3][]
{
  \begingroup
    \def\childdocextract #2##1~~~{\def\childdoctmp{\childdocforward[#1]{#3##1}}}
    \expandafter\childdocextract\childdocname~~~
    \expandafter
  \endgroup
  \childdoctmp
}
%    \end{macrocode}

% \macro{\childdoc}
% The deprecated macro |\childdoc| is a legacy version of |\childdocmain|:
%    \begin{macrocode}
\newcommand{\childdoc}{\childdocmain}
%    \end{macrocode}

% \macro{\childdocredirect}
% The deprecated macro |\childdocredirect| is a legacy version
% of |\childdocforward| and |\childdocforwardprefix|:
%    \begin{macrocode}
\newcommand{\childdocredirect}[2][]
{
  \begingroup
    \if?#1?
      \def\childdoctmp{\childdocforward{#2}}
    \else
      \def\childdoctmp{\childdocforwardprefix{#1}{#2}}
    \fi
    \expandafter
  \endgroup
  \childdoctmp
}
%    \end{macrocode}

%\iffalse
%</package>
%\fi
%
\endinput
|\\
|\childdocof{|\textit{main}|}|\\
\end{tabular}
\end{center}
at the top of every child file \textit{child}
which is included by |\include{|\textit{child}|}|
from within the main file
(or at least for those files to be compiled individually).
The argument \textit{main} must be the filename of the main file.

There are a couple of
considerations in setting up the main and child documents:

%%%%%%%%%%%%%%%%%%%%%%%%%%%%%%%%%%%%%%%%
\paragraph{Restrictions.}

Please note the following restrictions:
\begin{itemize}
\item
|\childdocmain| must be called with one argument \textit{main}
to ensure compatibility with earlier version of the package.
It must either be empty (|\childdocmain{}|)
or precisely match the filename of the main file in which it is specified.
See \secref{sec:detection} for further information.
\item
The filename \textit{main} must be specified without the |.tex| extension.
\item
The filename \textit{main} is case sensitive
(even in case-insensitive file systems)
due to internal string comparison.
\item
The argument \textit{main} should be fully expanded, it cannot be a macro.
\item
Subdirectories and special characters should be avoided in filenames.
\item
The command |\childdocmain{|\textit{main}|}| must be followed by a whitespace.
It should not be followed immediately by another command
or by a comment mark `|%|'.
This is because the \TeX{} parser reads the token immediately following
the argument of |\childdocmain| and puts it
at the beginning of every child section;
however, a white\-space is ignored.
\end{itemize}

%%%%%%%%%%%%%%%%%%%%%%%%%%%%%%%%%%%%%%%%
\paragraph{Content of Main File.}

It is advisable to place all content in the child files included by |\include|.
Any output contained in the main file will appear in all child documents
unless suppressed manually;
it cannot be suppressed automatically by the |\includeonly| directive
and thus should normally be avoided.
A method to include some content in the main file
by means of conditional processing is described in \secref{sec:conditional}.

%%%%%%%%%%%%%%%%%%%%%%%%%%%%%%%%%%%%%%%%
\paragraph{Page Numbering.}

When only a part of the document is compiled,
the appropriate numbering of pages
(as well as other status parameters)
is determined from the |.aux| files.
The latter contain information from previous passes.
However this information needs to propagate through
all intermediate child documents.
Therefore the page numbering in child documents may well
be inconsistent until the complete document is compiled at least once.

A useful (if unconventional) way to always ensure a consistent
page numbering is to restart the numbering in each child document
and denote the pages by `\textit{child}|.|\textit{page}'
where \textit{child} represents the chapter/section number of the child file.
This can be achieved by the command
|\numberwithin{page}{|\textit{child}|}|
of the \textsf{amsmath} package
where \textit{child} can be |chapter| or |section|
depending on the chosen structuring.
Alternatively, one can modify the macro |\thepage| appropriately
and reset the counter |page| at the start of each child file.

%%%%%%%%%%%%%%%%%%%%%%%%%%%%%%%%%%%%%%%%%%%%%%%%%%%%%%%%%%%%%%%%%%%%%%%%%%%%%%%%
\subsection{Conditional Processing}
\label{sec:conditional}

The package provides a mechanism to compile different versions
of a document. To customise the versions further some conditional processing
can come in handy to distinguish which version is being compiled.
The package provides two macros to describe the compilation context:

%%%%%%%%%%%%%%%%%%%%%%%%%%%%%%%%%%%%%%%%
\DescribeMacro{\ifchilddoc}
The conditional |\ifchilddoc| distinguishes between the compilation of
child documents and the main document:
%
\begin{center}
|\ifchilddoc |\textit{child-code}| |[|\||else |\textit{main-code}]| \||fi|
\end{center}

%%%%%%%%%%%%%%%%%%%%%%%%%%%%%%%%%%%%%%%%
\DescribeMacro{\childdocname}
\DescribeMacro{\childdocjob}
The macro |\childdocname| contains the filename (without extension)
of the main or child file being processed.
Note that |\childdocjob| will always contain the name of the main file.

%%%%%%%%%%%%%%%%%%%%%%%%%%%%%%%%%%%%%%%%
\paragraph{Title Page.}

Conditional processing can be used to include a title or banner page
in the main document when proper precautions are taken.
Importantly, the code in the main file should ensure that the page counter
(as well as other status parameters which are stored in the |.aux| files)
takes the same value after the conditional processing.
Otherwise the page numbers may take divergent values
depending on which part is compiled.

For example, a title page could be declared by:
%
\begin{center}
\begin{tabular}{l}
|\ifchilddoc\||else|\\
|\addtocounter{page}{-1}|\\
\textit{code for title page}\\
|\newpage|\\
|\||fi|
\end{tabular}
\end{center}
%
A banner page for the child documents can be generated by:
%
\begin{center}
\begin{tabular}{l}
|\ifchilddoc|\\
|\addtocounter{page}{-1}|\\
\textit{code for banner page}\\
|\newpage|\\
|\||fi|
\end{tabular}
\end{center}
%
Here one could write a message such as:
\begin{center}
|This is the part \childdocname{} of \childdocjob{}.|
\end{center}

%%%%%%%%%%%%%%%%%%%%%%%%%%%%%%%%%%%%%%%%%%%%%%%%%%%%%%%%%%%%%%%%%%%%%%%%%%%%%%%%
\subsection{Flags}
\label{sec:flags}

The package makes it easy to generate different versions
of the main or child documents.
To this end compilation flags can be defined
and assigned different default values.
They will be particularly useful in conjunction
with the forwarding mechanism described in \secref{sec:forward}.

For example, it may be useful to have a flag |\version|
which can be set to |draft| or |final|.
The document source will contain some conditional code
depending on the value of |\version|.
Suppose further, the flag should default to |final| for the main file
and to |draft| for child files
which is a natural assignment for editing the document.
This is achieved by placing the following code
in the preamble of the main document
(below the |\childdocmain| directive):
%
\begin{center}
\begin{tabular}{l}
|\ifchilddoc|\\
|\providecommand{\version}{draft}|\\
|\||else|\\
|\providecommand{\version}{final}|\\
|\||fi|
\end{tabular}
\end{center}
%
The definition by |\providecommand| makes sure
that previous definitions are not overwritten.
Further statements |\providecommand{\version}{...}|
can thus be added before the above code to override it.

For the main file, one might add a line
(between |\childdocmain| and the above block)
%
\begin{center}
|%\ifchilddoc\||else\providecommand{\version}{draft}\||fi|
\end{center}
%
which can be uncommented to produce a draft version.
Likewise one can add a line to the very top of a child file
(above the |\childdocof{|\textit{main}|}| directive)
%
\begin{center}
|%\providecommand{\version}{final}|
\end{center}
%
which can be uncommented to produce the final version of this child document.

%%%%%%%%%%%%%%%%%%%%%%%%%%%%%%%%%%%%%%%%%%%%%%%%%%%%%%%%%%%%%%%%%%%%%%%%%%%%%%%%
\subsection{Forwarding}
\label{sec:forward}

Different versions of the main or child documents
using compilation flags as described in \secref{sec:flags}
can be (permanently) stored in different files
for convenient compilation, viewing and distribution.
To this end, the package defines a command
to pass on compilation to a different file:

%%%%%%%%%%%%%%%%%%%%%%%%%%%%%%%%%%%%%%%%
\DescribeMacro{\childdocforward}
The command |\childdocforward| redirects processing to
another source file:
%
\begin{center}
\begin{tabular}{l}
|% \iffalse
%
% childdoc.dtx Copyright (C) 2017-2018 Niklas Beisert
%
% This work may be distributed and/or modified under the
% conditions of the LaTeX Project Public License, either version 1.3
% of this license or (at your option) any later version.
% The latest version of this license is in
%   http://www.latex-project.org/lppl.txt
% and version 1.3 or later is part of all distributions of LaTeX
% version 2005/12/01 or later.
%
% This work has the LPPL maintenance status `maintained'.
%
% The Current Maintainer of this work is Niklas Beisert.
%
% This work consists of the files childdoc.dtx and childdoc.ins
% and the derived files childdoc.def and cdocsamp.tex with
% cdocsch1.tex, cdocsch2.tex, cdocsdrf.tex, cdocsfn1.tex, cdocsfn2.tex.
%
%<package>\ifdefined\childdocmain\endinput\fi
%<package>\ProvidesFile{childdoc.def}[2018/12/30 v2.0 child document driver]
%<samplemain>\ProvidesFile{cdocsamp.tex}[2018/12/30 v2.0 sample for childdoc]
%<*driver>
%\ProvidesFile{childdoc.drv}[2018/12/30 v2.0 childdoc reference manual file]
\PassOptionsToClass{10pt,a4paper}{article}
\documentclass{ltxdoc}

\usepackage[margin=35mm]{geometry}
\usepackage{hyperref}
\usepackage{hyperxmp}
\usepackage[usenames]{color}

\hypersetup{colorlinks=true}
\hypersetup{pdfstartview=FitH}
\hypersetup{pdfpagemode=UseNone}
\hypersetup{pdfsource={}}
\hypersetup{pdflang={en-UK}}
\hypersetup{pdfcopyright={Copyright 2017-2018 Niklas Beisert.
  This work may be distributed and/or modified under the
  conditions of the LaTeX Project Public License, either version 1.3
  of this license or (at your option) any later version.}}
\hypersetup{pdflicenseurl={http://www.latex-project.org/lppl.txt}}
\hypersetup{pdfcontactaddress={ETH Zurich, ITP, HIT K,
  Wolfgang-Pauli-Strasse 27}}
\hypersetup{pdfcontactpostcode={8093}}
\hypersetup{pdfcontactcity={Zurich}}
\hypersetup{pdfcontactcountry={Switzerland}}
\hypersetup{pdfcontactemail={nbeisert@itp.phys.ethz.ch}}
\hypersetup{pdfcontacturl={http://people.phys.ethz.ch/\xmptilde nbeisert/}}

\newcommand{\secref}[1]{\hyperref[#1]{section \ref*{#1}}}

\parskip1ex
\parindent0pt
\let\olditemize\itemize
\def\itemize{\olditemize\parskip0pt}

\begin{document}

\title{The \textsf{childdoc} Package}
\hypersetup{pdftitle={The childdoc Package}}
\author{Niklas Beisert\\[2ex]
  Institut f\"ur Theoretische Physik\\
  Eidgen\"ossische Technische Hochschule Z\"urich\\
  Wolfgang-Pauli-Strasse 27, 8093 Z\"urich, Switzerland\\[1ex]
  \href{mailto:nbeisert@itp.phys.ethz.ch}
  {\texttt{nbeisert@itp.phys.ethz.ch}}}
\hypersetup{pdfauthor={Niklas Beisert}}
\hypersetup{pdfsubject={Manual for the LaTeX2e Package childdoc}}
\date{30 December 2018, \textsf{v2.0}}
\maketitle

\begin{abstract}\noindent
\textsf{childdoc} is a \LaTeXe{} package
that enables the direct compilation
of document sections included by |\include|
to individual files.
\end{abstract}

\begingroup
\parskip0ex
\tableofcontents
\endgroup

%%%%%%%%%%%%%%%%%%%%%%%%%%%%%%%%%%%%%%%%%%%%%%%%%%%%%%%%%%%%%%%%%%%%%%%%%%%%%%%%
%%%%%%%%%%%%%%%%%%%%%%%%%%%%%%%%%%%%%%%%%%%%%%%%%%%%%%%%%%%%%%%%%%%%%%%%%%%%%%%%
\section{Introduction}

\LaTeX{} provides a mechanism to structure a large document (such as a book)
into a main file and several child files (containing the chapters)
using the |\include| command.
This mechanism is beneficial for documents
which span hundreds of pages in order to
make the source file(s) more manageable.
Moreover, compilation can be restricted to
selected child files by means of the |\includeonly| command.
The latter feature can be used to reduce the compilation time while editing
(this was significantly more useful in the earlier days of \LaTeX{})
or to generate a smaller document which is easier to navigate.
Another application of |\includeonly| is to generate
documents consisting of selected parts of the complete document.

However, there are a few drawbacks of the plain |\include| mechanism:
\begin{itemize}
\item
The child files cannot be compiled on their own,
they can only be compiled via the main file.
A naive editing environment
(such as a text editor with an option
to have the current file processed by \LaTeX)
may require one to switch to the main file before compiling;
attempting to compile the child file produces errors.
\item
The main file must be modified (each time)
to adjust the |\includeonly| command
to the present needs. This easily leaves the main file in a messy state.
\item
The generated document will always carry the filename
of the main document. This is inconvenient if
several child files are to be compiled and
to be kept for distribution.
\end{itemize}

The present package provides a simple interface
to make child files individually compilable by \LaTeX{}.
Compiling a child file then has the same effect as compiling
the main file with an |\includeonly| command
to select the appropriate child.
Moreover the generated document will carry the name of the child
rather than the main file.
This resolves all three above issues.

This feature is meant to make the editing of books,
thesis documents and lecture notes somewhat more convenient.
However, the package can also be used efficiently for
composing a series of documents (such as exercise sheets)
which are typically distributed individually.
It then assists the author in generating the individual documents
(potentially in different versions)
as well as a document containing the collected series.
Another application is in developing style files
or other kinds of included material
where compilation of the style file could redirect
to a sample or test file.

%%%%%%%%%%%%%%%%%%%%%%%%%%%%%%%%%%%%%%%%%%%%%%%%%%%%%%%%%%%%%%%%%%%%%%%%%%%%%%%%
%%%%%%%%%%%%%%%%%%%%%%%%%%%%%%%%%%%%%%%%%%%%%%%%%%%%%%%%%%%%%%%%%%%%%%%%%%%%%%%%
\section{Usage}

First of all, the package \textsf{childdoc} is \emph{not} a standard
\LaTeXe{} |.sty| style file! Therefore it needs to be invoked in
a non-standard way.

%%%%%%%%%%%%%%%%%%%%%%%%%%%%%%%%%%%%%%%%%%%%%%%%%%%%%%%%%%%%%%%%%%%%%%%%%%%%%%%%
\subsection{Included Files}
\label{sec:include}

%%%%%%%%%%%%%%%%%%%%%%%%%%%%%%%%%%%%%%%%
\DescribeMacro{\childdocmain}
To use the package, add the commands
\begin{center}
\begin{tabular}{l}
|\input{childdoc.def}|\\
|\childdocmain{}|\\
\end{tabular}
\end{center}
at the very top of the main \LaTeX{} file,
in particular \emph{before} the |\documentclass| statement!
The argument of |\childdocmain| should be left empty
(but it must be present).

%%%%%%%%%%%%%%%%%%%%%%%%%%%%%%%%%%%%%%%%
\DescribeMacro{\childdocof}
Furthermore, add the commands
\begin{center}
\begin{tabular}{l}
|\input{childdoc.def}|\\
|\childdocof{|\textit{main}|}|\\
\end{tabular}
\end{center}
at the top of every child file \textit{child}
which is included by |\include{|\textit{child}|}|
from within the main file
(or at least for those files to be compiled individually).
The argument \textit{main} must be the filename of the main file.

There are a couple of
considerations in setting up the main and child documents:

%%%%%%%%%%%%%%%%%%%%%%%%%%%%%%%%%%%%%%%%
\paragraph{Restrictions.}

Please note the following restrictions:
\begin{itemize}
\item
|\childdocmain| must be called with one argument \textit{main}
to ensure compatibility with earlier version of the package.
It must either be empty (|\childdocmain{}|)
or precisely match the filename of the main file in which it is specified.
See \secref{sec:detection} for further information.
\item
The filename \textit{main} must be specified without the |.tex| extension.
\item
The filename \textit{main} is case sensitive
(even in case-insensitive file systems)
due to internal string comparison.
\item
The argument \textit{main} should be fully expanded, it cannot be a macro.
\item
Subdirectories and special characters should be avoided in filenames.
\item
The command |\childdocmain{|\textit{main}|}| must be followed by a whitespace.
It should not be followed immediately by another command
or by a comment mark `|%|'.
This is because the \TeX{} parser reads the token immediately following
the argument of |\childdocmain| and puts it
at the beginning of every child section;
however, a white\-space is ignored.
\end{itemize}

%%%%%%%%%%%%%%%%%%%%%%%%%%%%%%%%%%%%%%%%
\paragraph{Content of Main File.}

It is advisable to place all content in the child files included by |\include|.
Any output contained in the main file will appear in all child documents
unless suppressed manually;
it cannot be suppressed automatically by the |\includeonly| directive
and thus should normally be avoided.
A method to include some content in the main file
by means of conditional processing is described in \secref{sec:conditional}.

%%%%%%%%%%%%%%%%%%%%%%%%%%%%%%%%%%%%%%%%
\paragraph{Page Numbering.}

When only a part of the document is compiled,
the appropriate numbering of pages
(as well as other status parameters)
is determined from the |.aux| files.
The latter contain information from previous passes.
However this information needs to propagate through
all intermediate child documents.
Therefore the page numbering in child documents may well
be inconsistent until the complete document is compiled at least once.

A useful (if unconventional) way to always ensure a consistent
page numbering is to restart the numbering in each child document
and denote the pages by `\textit{child}|.|\textit{page}'
where \textit{child} represents the chapter/section number of the child file.
This can be achieved by the command
|\numberwithin{page}{|\textit{child}|}|
of the \textsf{amsmath} package
where \textit{child} can be |chapter| or |section|
depending on the chosen structuring.
Alternatively, one can modify the macro |\thepage| appropriately
and reset the counter |page| at the start of each child file.

%%%%%%%%%%%%%%%%%%%%%%%%%%%%%%%%%%%%%%%%%%%%%%%%%%%%%%%%%%%%%%%%%%%%%%%%%%%%%%%%
\subsection{Conditional Processing}
\label{sec:conditional}

The package provides a mechanism to compile different versions
of a document. To customise the versions further some conditional processing
can come in handy to distinguish which version is being compiled.
The package provides two macros to describe the compilation context:

%%%%%%%%%%%%%%%%%%%%%%%%%%%%%%%%%%%%%%%%
\DescribeMacro{\ifchilddoc}
The conditional |\ifchilddoc| distinguishes between the compilation of
child documents and the main document:
%
\begin{center}
|\ifchilddoc |\textit{child-code}| |[|\||else |\textit{main-code}]| \||fi|
\end{center}

%%%%%%%%%%%%%%%%%%%%%%%%%%%%%%%%%%%%%%%%
\DescribeMacro{\childdocname}
\DescribeMacro{\childdocjob}
The macro |\childdocname| contains the filename (without extension)
of the main or child file being processed.
Note that |\childdocjob| will always contain the name of the main file.

%%%%%%%%%%%%%%%%%%%%%%%%%%%%%%%%%%%%%%%%
\paragraph{Title Page.}

Conditional processing can be used to include a title or banner page
in the main document when proper precautions are taken.
Importantly, the code in the main file should ensure that the page counter
(as well as other status parameters which are stored in the |.aux| files)
takes the same value after the conditional processing.
Otherwise the page numbers may take divergent values
depending on which part is compiled.

For example, a title page could be declared by:
%
\begin{center}
\begin{tabular}{l}
|\ifchilddoc\||else|\\
|\addtocounter{page}{-1}|\\
\textit{code for title page}\\
|\newpage|\\
|\||fi|
\end{tabular}
\end{center}
%
A banner page for the child documents can be generated by:
%
\begin{center}
\begin{tabular}{l}
|\ifchilddoc|\\
|\addtocounter{page}{-1}|\\
\textit{code for banner page}\\
|\newpage|\\
|\||fi|
\end{tabular}
\end{center}
%
Here one could write a message such as:
\begin{center}
|This is the part \childdocname{} of \childdocjob{}.|
\end{center}

%%%%%%%%%%%%%%%%%%%%%%%%%%%%%%%%%%%%%%%%%%%%%%%%%%%%%%%%%%%%%%%%%%%%%%%%%%%%%%%%
\subsection{Flags}
\label{sec:flags}

The package makes it easy to generate different versions
of the main or child documents.
To this end compilation flags can be defined
and assigned different default values.
They will be particularly useful in conjunction
with the forwarding mechanism described in \secref{sec:forward}.

For example, it may be useful to have a flag |\version|
which can be set to |draft| or |final|.
The document source will contain some conditional code
depending on the value of |\version|.
Suppose further, the flag should default to |final| for the main file
and to |draft| for child files
which is a natural assignment for editing the document.
This is achieved by placing the following code
in the preamble of the main document
(below the |\childdocmain| directive):
%
\begin{center}
\begin{tabular}{l}
|\ifchilddoc|\\
|\providecommand{\version}{draft}|\\
|\||else|\\
|\providecommand{\version}{final}|\\
|\||fi|
\end{tabular}
\end{center}
%
The definition by |\providecommand| makes sure
that previous definitions are not overwritten.
Further statements |\providecommand{\version}{...}|
can thus be added before the above code to override it.

For the main file, one might add a line
(between |\childdocmain| and the above block)
%
\begin{center}
|%\ifchilddoc\||else\providecommand{\version}{draft}\||fi|
\end{center}
%
which can be uncommented to produce a draft version.
Likewise one can add a line to the very top of a child file
(above the |\childdocof{|\textit{main}|}| directive)
%
\begin{center}
|%\providecommand{\version}{final}|
\end{center}
%
which can be uncommented to produce the final version of this child document.

%%%%%%%%%%%%%%%%%%%%%%%%%%%%%%%%%%%%%%%%%%%%%%%%%%%%%%%%%%%%%%%%%%%%%%%%%%%%%%%%
\subsection{Forwarding}
\label{sec:forward}

Different versions of the main or child documents
using compilation flags as described in \secref{sec:flags}
can be (permanently) stored in different files
for convenient compilation, viewing and distribution.
To this end, the package defines a command
to pass on compilation to a different file:

%%%%%%%%%%%%%%%%%%%%%%%%%%%%%%%%%%%%%%%%
\DescribeMacro{\childdocforward}
The command |\childdocforward| redirects processing to
another source file:
%
\begin{center}
\begin{tabular}{l}
|\input{childdoc.def}|\\
|\childdocforward[|\textit{main}|]{|\textit{dest}|}|\\
\end{tabular}
\end{center}
%
The argument \textit{dest} is the destination file
(without extension).
It should be the main file or one of the child files.
Note that further \textsf{childdoc} directives
such as |\childdocof| and |\childdocforward|
in the indicated file will be processed in this form.
The optional argument \textit{main}
passes on directly to the main file \textit{main}
while pretending to compile the child \textit{dest}.
This form behaves as if \textit{dest}
issues |\childdocof{|\textit{main}|}| right away,
and no further \textsf{childdoc} directives will be processed.

%%%%%%%%%%%%%%%%%%%%%%%%%%%%%%%%%%%%%%%%
\DescribeMacro{\...prefix}
In the alternative form |\childdocforwardprefix|,
%
\begin{center}
\begin{tabular}{l}
|\input{childdoc.def}|\\
|\childdocforwardprefix[|\textit{main}|]{|\textit{prefix}|}{|\textit{dest}|}|
\end{tabular}
\end{center}
%
the destination file is determined by a pattern
depending on the current file:
To make this work, the current file must be called
`{\textit{prefix}\hspace{0.2em}\textit{suffix}}'
with \textit{prefix} matching precisely the argument.
Processing is then passed on to the file
`{\textit{dest}\hspace{0.2em}\textit{suffix}}'.
Surely, the same effect is achieved by
directly specifying the
argument `{\textit{dest}\hspace{0.2em}\textit{suffix}}'
in the first form.
However, that requires to set up a different file
for each child. With the alternative form of the command
all these files can have exactly the same content
which simplifies setting them up and maintaining them.

For example, the following file |draft.tex|
with a compilation flag |\version| as described in \secref{sec:flags}
compiles the main document as a draft:
%
\begin{center}
\begin{tabular}{l}
|\def\version{draft}|\\
|\input{childdoc.def}|\\
|\childdocforward{|\textit{main}|}|
\end{tabular}
\end{center}
%
Likewise, the following files |final|\textit{nn}|.tex|
compile the final version of the child document
|child|\textit{nn}|.tex|:
%
\begin{center}
\begin{tabular}{l}
|\def\version{final}|\\
|\input{childdoc.def}|\\
|\childdocforwardprefix{final}{child}|
\end{tabular}
\end{center}
%

Note that when several versions of a main file and/or of each child file
are to be generated, it may be convenient to set up a |Makefile| or
shell script to automatise the process.

%%%%%%%%%%%%%%%%%%%%%%%%%%%%%%%%%%%%%%%%%%%%%%%%%%%%%%%%%%%%%%%%%%%%%%%%%%%%%%%%
\subsection{Command Line Processing}
\label{sec:commandline}

The effect of redirection files can also be achieved by invoking
the \LaTeX{} compiler with a more elaborate command line.
Most conveniently this should be done as part
of a shell script or a |Makefile|.

When using \textsf{childdoc} in the main file, the following
command lines effectively perform a redirection
(note that depending on the shell being used,
backslashes may have to be doubled: `|\|' $\to$ `|\\|'):
%
\begin{center}
|... -jobname "|\textit{target}|" |\\|"|[\textit{flags}]%
|\input{childdoc.def}\childdocforward[|\textit{main}|]{|\textit{dest}|}"|
\end{center}
%
Here \textit{target} is the name of the output file,
\textit{main} is the name of the main file
and \textit{dest} is the name of the main or child file to be processed
(all filenames without extensions).
The optional argument \textit{main} can be omitted
if \textit{main} matches \textit{dest}.
Optionally, compilation \textit{flags} can be defined via |\def| commands.
This command line makes the \TeX{} engine believe
it is compiling the file \textit{target}
whose content is specified as the latter parameter.
The provided code then forwards the processing to
\textit{main} or \textit{dest} as described in \secref{sec:forward}.

%%%%%%%%%%%%%%%%%%%%%%%%%%%%%%%%%%%%%%%%%%%%%%%%%%%%%%%%%%%%%%%%%%%%%%%%%%%%%%%%
\subsection{Include by Input}
\label{sec:input}

Including child documents by |\include| has some restrictions by design.
Most notably, the content of a child document always occupies
its own set of pages; pages cannot be shared between child documents.
Usually, this behaviour makes perfect sense
because each child document contain an essential part of the document.
However, in some situations it may be desirable to compose
a document from a collection of parts
without having mandatory page breaks between then.
For this case, the package
provides a mechanism to include parts
by |\input| which can also be processed individually.
However, by construction this mechanism
requires manual handling of the content to be output.

%%%%%%%%%%%%%%%%%%%%%%%%%%%%%%%%%%%%%%%%
\DescribeMacro{\ifchilddocmanual}
The main file should be prepared as usual, see \secref{sec:include}.
However, the document body must make a distinction
between processing of an individual part and of the main document, e.g.:
%
\begin{center}
\begin{tabular}{l}
|\ifchilddocmanual|\\
|\input{\childdocname}|\\
|\||else|\\
\textit{document body with }|\input{|\textit{part}|}|\\
|\||fi|
\end{tabular}
\end{center}
%
The conditional |\ifchilddocmanual| is true whenever
a part to be included by |\input| is being compiled,
and the name of the part is stored in |\childdocname|.

%%%%%%%%%%%%%%%%%%%%%%%%%%%%%%%%%%%%%%%%
\DescribeMacro{\childdocby}
Each part to be included by |\input| should start with:
%
\begin{center}
\begin{tabular}{l}
|\input{childdoc.def}|\\
|\childdocby{|\textit{main}|}|\\
\end{tabular}
\end{center}
%
The directive |\childdocby| is similar to |\childdocof|
described in \secref{sec:include},
but the subsequent selection of content must be done manually.
To that end, both |\ifchilddoc| and |\ifchilddocmanual|
will be true upon processing of a part,
and the name of the part is stored in |\childdocname|.
Note that |\jobname| will be set to the filename of the current part
so that each part receives an individual |.aux| file
that does not interfere with the |.aux| file(s) of the main document.
This behaviour can be altered by the alternative form
|\childdocby[*]{|\textit{main}|}| (with a non-empty optional argument)
which uses the |.aux| file of the main document
by setting |\jobname| to \textit{main}.

%%%%%%%%%%%%%%%%%%%%%%%%%%%%%%%%%%%%%%%%%%%%%%%%%%%%%%%%%%%%%%%%%%%%%%%%%%%%%%%%
\subsection{Driver Development}
\label{sec:driver}

The \textsf{childdoc} mechanism can also be use for the development
of definition files such as \LaTeX{} styles or classes.
This case differs from the above setup with multiple parts
included by |\include| in that no |\includeonly| should be invoked.
This can be achieved by starting the include file
(before |\ProvidesPackage|) with:
%
\begin{center}
\begin{tabular}{l}
|\input{childdoc.def}|\\
|\childdocforward{|\textit{main}|}|\\
\end{tabular}
\end{center}
%
or alternatively with:
%
\begin{center}
\begin{tabular}{l}
|\input{childdoc.def}|\\
|\childdocby{|\textit{main}|}|\\
\end{tabular}
\end{center}
%
Both forms have slightly different effects as described above.
The main file is prepared as usual, see \secref{sec:include}.

%%%%%%%%%%%%%%%%%%%%%%%%%%%%%%%%%%%%%%%%%%%%%%%%%%%%%%%%%%%%%%%%%%%%%%%%%%%%%%%%
\subsection{Legacy Detection}
\label{sec:detection}

The directive |\childdocmain| in the main file can detect
whether the complete document or merely a child is to be compiled
even without using the directive |\childdocof|.
This method is deprecated because it is less robust
and there is no compelling reason to use it;
it is merely provided for backward compatibility
and it may be removed in future versions.

If the detection mechanism is to be used,
it is mandatory to correctly specify
the filename of the main file as the argument of |\childdocmain|:
%
\begin{center}
\begin{tabular}{l}
|\input{childdoc.def}|\\
|\childdocmain{|\textit{main}|}|\\
\end{tabular}
\end{center}
%
If |\jobname| does not match the argument \textit{main} of |\childdocmain|,
it is assumed that |\jobname| points to the child file to be compiled.
When using |\childdocmain| with the main file specified as argument,
it suffices to start a child file
with just |\input{|\textit{main}|}|
without loading of the package and using |\childdocof|.
If instead all processing is done
with the appropriate \textsf{childdoc} directives,
the argument of \textit{main} of |\childdocmain| can be empty.

An alternative version of the command line processing described
in \secref{sec:commandline} using the detection mechanism reads:
%
\begin{center}
|... -jobname "|\textit{target}|" "|[\textit{flags}]%
[|\def\jobname{|\textit{dest}|}|]|\input{|\textit{main}|}"|
\end{center}

%%%%%%%%%%%%%%%%%%%%%%%%%%%%%%%%%%%%%%%%%%%%%%%%%%%%%%%%%%%%%%%%%%%%%%%%%%%%%%%%
\subsection{Manual Code}
\label{sec:manual}

In case one cannot be certain whether the definitions file |childdoc.def|
is installed on the target \TeX{} distribution
and one prefers not to ship it,
it is conceivable to paste a few relevant commands into the sources.

To that end, drop all statements |\input{childdoc.def}|
and perform the replacements as outlined below.
Instead of |\childdocmain{|\textit{main}|}| add the following code
to the top of the main file:
%
\begin{center}
\begin{tabular}{l}
|\||ifdefined\childdocname\endinput\||fi\newif\ifchilddoc|\\
|\edef\childdocname{\scantokens\expandafter{\jobname\noexpand}}|\\
|\def\childdocmain{|\textit{main}|}\||ifx\childdocmain\childdocname\||else|\\
|\childdoctrue\includeonly{\childdocname}\let\jobname\childdocmain\||fi|\\
\end{tabular}
\end{center}
%
Instead of |\childdocof{|\textit{main}|}| just include the main file
at the top of each child file:
%
\begin{center}
|\input{|\textit{main}|}|
\end{center}
%
A simple redirection |\childdocforward{|\textit{dest}|}| is achieved by:
%
\begin{center}
|\def\jobname{|\textit{dest}|}\input{\jobname}|
\end{center}
%
The redirection with prefix
|\childdocforwardprefix[|\textit{prefix}|]{|\textit{dest}|}|
is accomplished by:
%
\begin{center}
\begin{tabular}{l}
|{\edef\jobname{\scantokens\expandafter{\jobname\noexpand}}|\\
|\def\redirectjob |\textit{prefix}|#1~~~{\gdef\jobname{|\textit{dest}|#1}}|\\
|\expandafter\redirectjob\jobname~~~}\input{\jobname}|
\end{tabular}
\end{center}

In an alternative approach,
child documents can be compiled by a specific command line
without additional code or specific definitions:
%
\begin{center}
|... -jobname "|\textit{target}|" "|[\textit{flags}]%
|\includeonly{|\textit{dest}|}\input{|\textit{main}|}"|
\end{center}
%

%%%%%%%%%%%%%%%%%%%%%%%%%%%%%%%%%%%%%%%%%%%%%%%%%%%%%%%%%%%%%%%%%%%%%%%%%%%%%%%%
%%%%%%%%%%%%%%%%%%%%%%%%%%%%%%%%%%%%%%%%%%%%%%%%%%%%%%%%%%%%%%%%%%%%%%%%%%%%%%%%
\section{Information}

%%%%%%%%%%%%%%%%%%%%%%%%%%%%%%%%%%%%%%%%%%%%%%%%%%%%%%%%%%%%%%%%%%%%%%%%%%%%%%%%
\subsection{Copyright}

Copyright \copyright{} 2017--2018 Niklas Beisert

This work may be distributed and/or modified under the
conditions of the \LaTeX{} Project Public License, either version 1.3
of this license or (at your option) any later version.
The latest version of this license is in
  \url{http://www.latex-project.org/lppl.txt}
and version 1.3 or later is part of all distributions of \LaTeX{}
version 2005/12/01 or later.

This work has the LPPL maintenance status `maintained'.

The Current Maintainer of this work is Niklas Beisert.

This work consists of the files |README.txt|, |childdoc.ins| and |childdoc.dtx|
as well as the derived files |childdoc.def|, |cdocsamp.tex|
with |cdocsch1.tex|, |cdocsch2.tex|, |cdocspt3.tex|, |cdocspt4.tex|,
|cdocsdrf.tex|, |cdocsfn1.tex|, |cdocsfn2.tex|
as well as |childdoc.pdf|.

%%%%%%%%%%%%%%%%%%%%%%%%%%%%%%%%%%%%%%%%%%%%%%%%%%%%%%%%%%%%%%%%%%%%%%%%%%%%%%%%
\subsection{Files and Installation}

The package consists of the files:
%
\begin{center}
\begin{tabular}{ll}
    |README.txt|   & readme file \\
    |childdoc.ins| & installation file \\
    |childdoc.dtx| & source file \\
    |childdoc.def| & definition file \\
    |cdocsamp.tex| & sample main file \\
    |cdocsch1.tex| & sample include file \\
    |cdocsch2.tex| & sample include file \\
    |cdocspt3.tex| & sample part file \\
    |cdocspt4.tex| & sample part file \\
    |cdocsdrf.tex| & sample redirection file \\
    |cdocsfn1.tex| & sample redirection file \\
    |cdocsfn2.tex| & sample redirection file \\
    |childdoc.pdf| & manual
\end{tabular}
\end{center}
%
The distribution consists of the files
|README.txt|, |childdoc.ins| and |childdoc.dtx|.
%
\begin{itemize}
\item
Run (pdf)\LaTeX{} on |childdoc.dtx|
to compile the manual |childdoc.pdf| (this file).
\item
Run \LaTeX{} on |childdoc.ins| to create the definitions file |childdoc.def|
and the sample |cdocsamp.tex| with include files
|cdocsch1.tex|, |cdocsch2.tex|, |cdocspt3.tex|, |cdocspt4.tex|,
|cdocsdrf.tex|, |cdocsfn1.tex|, |cdocsfn2.tex|.
Then copy the file |childdoc.def| to an appropriate directory of your \LaTeX{}
distribution, e.g.\ \textit{texmf-root}|/tex/latex/childdoc|.
\end{itemize}

%%%%%%%%%%%%%%%%%%%%%%%%%%%%%%%%%%%%%%%%%%%%%%%%%%%%%%%%%%%%%%%%%%%%%%%%%%%%%%%%
\subsection{Related CTAN Packages}

There are several other packages which offer a similar functionality:
%
\begin{itemize}
\item
The packages
\href{http://ctan.org/pkg/docmute}{\textsf{docmute}},
\href{http://ctan.org/pkg/includex}{\textsf{includex}} and
\href{http://ctan.org/pkg/standalone}{\textsf{standalone}}
provide commands to include only the document body of
a child file thus allowing both files to be compiled individually.
\item
The packages \href{http://ctan.org/pkg/subdocs}{\textsf{subdocs}}
and \href{http://ctan.org/pkg/subfiles}{\textsf{subfiles}}
provide structures in which the main and child documents can be
encapsulated and allowing them to be compiled individually.
The inclusion mechanism is different from the conventional |\include|.
\item
The package \href{http://ctan.org/pkg/combine}{\textsf{combine}}
is an elaborate solution to combine several documents into one.
\end{itemize}
%
See also the CTAN topic \href{http://ctan.org/topic/subdocs}{\textsf{subdocs}}
for further related packages.
The present package differs from the above solutions in that
a document structure constructed with the conventional |\include| mechanism
just needs two extra commands at the top of every file
such that all constituent files can be compiled individually.

%%%%%%%%%%%%%%%%%%%%%%%%%%%%%%%%%%%%%%%%%%%%%%%%%%%%%%%%%%%%%%%%%%%%%%%%%%%%%%%%
%\subsection{Feature Suggestions}
%
%The following is a list of features which may be useful for future
%versions of this package:
%%
%\begin{itemize}
%\item
%\ldots
%\end{itemize}

%%%%%%%%%%%%%%%%%%%%%%%%%%%%%%%%%%%%%%%%%%%%%%%%%%%%%%%%%%%%%%%%%%%%%%%%%%%%%%%%
\subsection{Revision History}

%%%%%%%%%%%%%%%%%%%%%%%%%%%%%%%%%%%%%%%%
\paragraph{v2.0:} 2018/12/30

\begin{itemize}
\item
immediate forward processing
\item
added |\childdocby| mechanism
\item
manual restructured
\end{itemize}

%%%%%%%%%%%%%%%%%%%%%%%%%%%%%%%%%%%%%%%%
\paragraph{v1.6:} 2018/01/17

\begin{itemize}
\item
application for development of include files
\item
corrections to manual
\end{itemize}

%%%%%%%%%%%%%%%%%%%%%%%%%%%%%%%%%%%%%%%%
\paragraph{v1.5:} 2017/05/21

\begin{itemize}
\item
more complete structuring introduced
\item
|\childdocof| introduced
\item
|\childdoc| renamed to |\childdocmain|
\item
|\childredirect| renamed to |\childdocforward| and |\childdocforwardprefix|
and functionality expanded
\end{itemize}

%%%%%%%%%%%%%%%%%%%%%%%%%%%%%%%%%%%%%%%%
\paragraph{v1.0:} 2017/04/27

\begin{itemize}
\item
manual and install package
\item
first version published on CTAN
\end{itemize}

%%%%%%%%%%%%%%%%%%%%%%%%%%%%%%%%%%%%%%%%
\paragraph{v0.6:} 2017/04/26

\begin{itemize}
\item
redirection mechanism added
\end{itemize}

%%%%%%%%%%%%%%%%%%%%%%%%%%%%%%%%%%%%%%%%
\paragraph{v0.5:} 2017/04/26

\begin{itemize}
\item
functionality in definition file
\end{itemize}


%%%%%%%%%%%%%%%%%%%%%%%%%%%%%%%%%%%%%%%%%%%%%%%%%%%%%%%%%%%%%%%%%%%%%%%%%%%%%%%%
%%%%%%%%%%%%%%%%%%%%%%%%%%%%%%%%%%%%%%%%%%%%%%%%%%%%%%%%%%%%%%%%%%%%%%%%%%%%%%%%
%%%%%%%%%%%%%%%%%%%%%%%%%%%%%%%%%%%%%%%%%%%%%%%%%%%%%%%%%%%%%%%%%%%%%%%%%%%%%%%%
\appendix

\settowidth\MacroIndent{\rmfamily\scriptsize 000\ }

 \DocInput{childdoc.dtx}

\end{document}
%</driver>
% \fi
%
% %%%%%%%%%%%%%%%%%%%%%%%%%%%%%%%%%%%%%%%%%%%%%%%%%%%%%%%%%%%%%%%%%%%%%%%%%%%%%%
% %%%%%%%%%%%%%%%%%%%%%%%%%%%%%%%%%%%%%%%%%%%%%%%%%%%%%%%%%%%%%%%%%%%%%%%%%%%%%%
% \section{Sample}
%\iffalse
%<*samplemain>
%\fi
%
% The following presents a sample document
% with two chapters, two parts, a title page,
% a compile flag as well as three forwarding files to set the flag.
% It consists of eight |.tex| files:
% \begin{center}
% \begin{tabular}{ll}
% |cdocsamp.tex|&main file\\
% |cdocsch1.tex|&include file for chapter 1\\
% |cdocsch2.tex|&include file for chapter 2\\
% |cdocspt3.tex|&include file for part 3\\
% |cdocspt4.tex|&include file for part 4\\
% |cdocsdrf.tex|&forwarding file for main file in draft mode\\
% |cdocsfi1.tex|&forwarding file for final version of chapter 1\\
% |cdocsfi2.tex|&forwarding file for final version of chapter 2\\
% \end{tabular}
% \end{center}
% Each of the eight files can be compiled directly by the \LaTeX{} compiler.
%
% %%%%%%%%%%%%%%%%%%%%%%%%%%%%%%%%%%%%%%
% \paragraph{Main File.}
%
% The main file is called |cdocsamp.tex|.
%
% Load the \textsf{childdoc} definitions and
% declare the filename for the main document:
%    \begin{macrocode}
\input{childdoc.def}
\childdocmain{}
%    \end{macrocode}

% Optional override for |\version| flag:
%    \begin{macrocode}
%%\ifchilddoc\else\providecommand{\version}{draft}\fi
%    \end{macrocode}

% Define the default values for the |\version| flag
% (|final| for the main file and |draft| for childs):
%    \begin{macrocode}
\ifchilddoc
\providecommand{\version}{draft}
\else
\providecommand{\version}{final}
\fi
%    \end{macrocode}

% Load the standard document class:
%    \begin{macrocode}
\documentclass[12pt]{article}
%    \end{macrocode}

% Start the document body:
%    \begin{macrocode}
\begin{document}
%    \end{macrocode}

% Declare a title page.
% Print title, part of document being processed and version flag:
%    \begin{macrocode}
\addtocounter{page}{-1}
\begin{center}
{\LARGE\bfseries{}childdoc example\par}
\vspace{1cm}
\ifchilddoc
\ifchilddocmanual part\else chapter\fi:
`\childdocname' of `\childdocjob'\par
\else
main document: `\childdocjob'\par
\fi
version: \version\par
\end{center}
\newpage
%    \end{macrocode}

% Manually include selected file,
% otherwise process as usual:
%    \begin{macrocode}
\ifchilddocmanual
\section*{part `\childdocname'}
\input{\childdocname}
\else
%    \end{macrocode}

% Include the two chapters:
%    \begin{macrocode}
\include{cdocsch1}
\include{cdocsch2}
%    \end{macrocode}

% Include the two parts unless only chapters should be displayed:
%    \begin{macrocode}
\ifchilddoc\else
\section{part three}
\input{cdocspt3}
\section{part four}
\input{cdocspt4}
\fi
%    \end{macrocode}

% Process as usual until here:
%    \begin{macrocode}
\fi
%    \end{macrocode}

% End of document body:
%    \begin{macrocode}
\end{document}
%    \end{macrocode}
%\iffalse
%</samplemain>
%\fi
%
% %%%%%%%%%%%%%%%%%%%%%%%%%%%%%%%%%%%%%%
% \paragraph{Chapter Include Files.}
%
% The include files are called |cdocsch1.tex| and |cdocsch2.tex|.
%
%\iffalse
%<*samplechap1|samplechap2>
%\fi

% Optional override for |\version| flag:
%    \begin{macrocode}
%%\providecommand{\version}{final}
%    \end{macrocode}

% Include the main document:
%    \begin{macrocode}
\input{childdoc.def}
\childdocof{cdocsamp}
%    \end{macrocode}

%\iffalse
%</samplechap1|samplechap2>
%\fi
%
%\iffalse
%<*samplechap1>
%\fi
% Some text for chapter 1:
%    \begin{macrocode}
\section{one}
some text in chapter one
%    \end{macrocode}

%\iffalse
%</samplechap1>
%\fi
% Some text for chapter 2:
%\iffalse
%<*samplechap2>
%\fi
%    \begin{macrocode}
\section{two}
more text in chapter two
%    \end{macrocode}

%\iffalse
%</samplechap2>
%\fi
%
% %%%%%%%%%%%%%%%%%%%%%%%%%%%%%%%%%%%%%%
% \paragraph{Part Include Files.}
%
% The include files are called |cdocspt3.tex| and |cdocspt4.tex|.
%
%\iffalse
%<*samplepart3|samplepart4>
%\fi

% Optional override for |\version| flag:
%    \begin{macrocode}
%%\providecommand{\version}{final}
%    \end{macrocode}

% Include the main document:
%    \begin{macrocode}
\input{childdoc.def}
\childdocby{cdocsamp}
%    \end{macrocode}

%\iffalse
%</samplepart3|samplepart4>
%\fi
%
%\iffalse
%<*samplepart3>
%\fi
% Some text for part 3:
%    \begin{macrocode}
some text in part three
%    \end{macrocode}

%\iffalse
%</samplepart3>
%\fi
% Some text for part 4:
%\iffalse
%<*samplepart4>
%\fi
%    \begin{macrocode}
more text in part four
%    \end{macrocode}

%\iffalse
%</samplepart4>
%\fi
%
% %%%%%%%%%%%%%%%%%%%%%%%%%%%%%%%%%%%%%%
% \paragraph{Forwarding for a Complete Draft.}
%
% The following forwarding file |cdocsdrf.tex|
% compiles the main document in draft mode:
%\iffalse
%<*sampledraft>
%\fi
%    \begin{macrocode}
\def\version{draft}
\input{childdoc.def}
\childdocforward{cdocsamp}
%    \end{macrocode}

%\iffalse
%</sampledraft>
%\fi
%
% %%%%%%%%%%%%%%%%%%%%%%%%%%%%%%%%%%%%%%
% \paragraph{Forwarding for Final Version of the Chapters.}
%
% The following forwarding files |cdocsfn1.tex| and |cdocsfn2.tex|
% (with identical content)
% compile the final versions of the child documents
% |cdocsch1.tex| and |cdocsch2.tex|, respectively:
%\iffalse
%<*samplefinal>
%\fi
%    \begin{macrocode}
\def\version{final}
\input{childdoc.def}
\childdocforwardprefix[cdocsamp]{cdocsfn}{cdocsch}
%    \end{macrocode}

%\iffalse
%</samplefinal>
%\fi
%
% %%%%%%%%%%%%%%%%%%%%%%%%%%%%%%%%%%%%%%
% \paragraph{Command Line Processing.}
%
% The following three command lines generate the output files
% |cdocscld|, |cdocscl1| and |cdocscl2|
% which should be identical to
% |cdocsdrf|, |cdocsch1| and |cdocsfn2|, respectively:
% \begin{center}
% \begin{tabular}{l}
% |latex -jobname cdocscld \|\\
% |  "\def\version{draft}\input{childdoc.def}\childdocforward{cdocsamp}"|\\
% |latex -jobname cdocscl1 \|\\
% |  "\input{childdoc.def}\childdocforward[cdocsamp]{cdocsch1}"|\\
% |latex -jobname cdocscl2 \|\\
% |  "\def\version{final}\input{childdoc.def}\childdocforward{cdocsch2}"|
% \end{tabular}
% \end{center}
% Note that the trailing backslash on each first line
% merely continues the input to the second line
% (for convenient cut ant paste).
% Furthermore, the command |latex| can be replaced by any
% of its alternative versions such as |pdflatex|.
%
% %%%%%%%%%%%%%%%%%%%%%%%%%%%%%%%%%%%%%%%%%%%%%%%%%%%%%%%%%%%%%%%%%%%%%%%%%%%%%%
% %%%%%%%%%%%%%%%%%%%%%%%%%%%%%%%%%%%%%%%%%%%%%%%%%%%%%%%%%%%%%%%%%%%%%%%%%%%%%%
% \section{Implementation}
%\iffalse
%<*package>
%\fi
%
% This section describes the definitions file |childdoc.def|.

% The definitions cannot be loaded using |\usepackage| or |\RequirePackage|
% which has a mechanism to prevent loading a style file more than once.
% When loading the definitions by means of |\input|
% multiple instances have to be prevented manually:
%\iffalse
%This code needs to be before the `\ProvidesFile' directive
%which is defined at the beginning of this file.
%Therefore it is also placed there and commented out here.
%</package>
%<*discard>
%\fi
%    \begin{macrocode}
\ifdefined\childdocmain\endinput\fi
%    \end{macrocode}
%\iffalse
%</discard>
%<*package>
%\fi
%
% \macro{\ifchilddoc}
% \macro{\ifchilddocmanual}
% The conditional |\ifchilddoc| tells whether a
% child (true) or main (false) document is being compiled.
% The conditional |\ifchilddocmanual| tells whether
% the |\includeonly| mechanism is used (false) or
% the selection of child files must be performed manually (true).
% The definitions initialise to false:
%    \begin{macrocode}
\newif\ifchilddoc
\newif\ifchilddocmanual
%    \end{macrocode}

% \macro{\childdocname}
% \macro{\childdocjob}
% The macro |\childdocname| stores the name of the main document
% to be compiled. The macro |\childdocjob| stores the name of
% the document on which the \LaTeX{} compiler was originally invoked.
% The content of |\jobname| cannot be compared
% to filenames specified in the source due to different catcodes.
% The following code rescans |\jobname|, stores the result
% in |\childdocname| and saves a copy in |\childdocjob|:
%    \begin{macrocode}
\edef\childdocname{\scantokens\expandafter{\jobname\noexpand}}
\let\childdocjob\childdocname
%    \end{macrocode}

% \macro{\childdocdisable}
% The macro |\childdocdisable| prevents the main file
% from being processed more than once.
% At this stage, the main document command |\childdocmain|
% is assumed to be called once again where it should do nothing.
% Any subsequent call to it should prevent
% a secondary processing of the main document
% It overwrites the forwarding commands
% |\childdocof| and |\childdocforward|
% with empty macros to prevent further inclusions of the main document:
%    \begin{macrocode}
\newcommand{\childdocdisable}
{
  \renewcommand{\childdocmain}[1]{\renewcommand{\childdocmain}[1]{\endinput}}
  \renewcommand{\childdocof}[1]{}
  \renewcommand{\childdocby}[2][]{}
  \renewcommand{\childdocforward}[2][]{}
  \renewcommand{\childdocdisable}{}
}
%    \end{macrocode}

% \macro{\childdocmain}
% The macro |\childdocmain| is to be called at the top of the main file
% with nothing or the main filename (without extension) as argument.
% First, it breaks loops.
% If the argument is not empty and does not match |\childdocname|
% (which is set by the first inclusion of |childdoc.def|),
% |\ifchilddoc| is set to true, |\includeonly| is applied to the child file
% and |\jobname| is set to the main file
% (for proper handling of |.aux| files):
%    \begin{macrocode}
\newcommand{\childdocmain}[1]
{
  \childdocdisable\childdocmain{}
  \if?#1?\else
    \begingroup
      \def\childdoctmp{#1}
      \ifx\childdoctmp\childdocname
        \def\childdoctmp{}
      \else
        \def\childdoctmp
        {
          \childdoctrue
          \includeonly{\childdocname}
          \def\childdocjob{#1}
          \def\jobname{#1}
        }
      \fi
      \expandafter
    \endgroup
    \childdoctmp
  \fi
}
%    \end{macrocode}

% \macro{\childdocof}
% The command |\childdocof| redirects
% compilation to the main file |#1|.
%    \begin{macrocode}
\newcommand{\childdocof}[1]
{
  \childdocdisable
  \childdoctrue
  \includeonly{\childdocname}
  \def\jobname{#1}
  \def\childdocjob{#1}
  \input{#1}
}
%    \end{macrocode}

% \macro{\childdocby}
% The command |\childdocby| ....
%    \begin{macrocode}
\newcommand{\childdocby}[2][]
{
  \childdocdisable
  \childdoctrue
  \childdocmanualtrue
  \if?#1?\else
    \def\jobname{#2}
  \fi
  \def\childdocjob{#2}
  \input{#2}
  \endinput
}
%    \end{macrocode}

% \macro{\childdocforward}
% The command |\childdocforward| redirects
% compilation to the main file or
% (if the optional argument is given) a child file.
% Parameters are set as if the main file
% or a child file starting with |\childdocof| was compiled.
% Then compilation is handed over to the main file:
%    \begin{macrocode}
\newcommand{\childdocforward}[2][]
{
  \begingroup
    \if?#1?
      \def\childdoctmp
      {
        \def\childdocname{#2}
        \def\childdocjob{#2}
        \def\jobname{#2}
        \input{#2}
        \endinput
      }
    \else
      \def\childdoctmp
      {
        \childdocdisable
        \def\childdocname{#2}
        \childdoctrue
        \includeonly{#2}
        \def\childdocjob{#1}
        \def\jobname{#1}
        \input{#1}
        \endinput
      }
    \fi
    \expandafter
  \endgroup
  \childdoctmp
}
%    \end{macrocode}

% \macro{\childdocforwardprefix}
% The command |\childdocforwardprefix| redirects
% compilation to the main or a child file by means of a pattern.
% The prefix |#1| in the current filename is replaced by |#2|
% and the suffix of the current filename is kept
% (it is assumed that the filename does not contain the substring `|~~~|'
% which is used as a delimiter).
% Compilation is handed over to the new file by |\childdocforward|:
%    \begin{macrocode}
\newcommand{\childdocforwardprefix}[3][]
{
  \begingroup
    \def\childdocextract #2##1~~~{\def\childdoctmp{\childdocforward[#1]{#3##1}}}
    \expandafter\childdocextract\childdocname~~~
    \expandafter
  \endgroup
  \childdoctmp
}
%    \end{macrocode}

% \macro{\childdoc}
% The deprecated macro |\childdoc| is a legacy version of |\childdocmain|:
%    \begin{macrocode}
\newcommand{\childdoc}{\childdocmain}
%    \end{macrocode}

% \macro{\childdocredirect}
% The deprecated macro |\childdocredirect| is a legacy version
% of |\childdocforward| and |\childdocforwardprefix|:
%    \begin{macrocode}
\newcommand{\childdocredirect}[2][]
{
  \begingroup
    \if?#1?
      \def\childdoctmp{\childdocforward{#2}}
    \else
      \def\childdoctmp{\childdocforwardprefix{#1}{#2}}
    \fi
    \expandafter
  \endgroup
  \childdoctmp
}
%    \end{macrocode}

%\iffalse
%</package>
%\fi
%
\endinput
|\\
|\childdocforward[|\textit{main}|]{|\textit{dest}|}|\\
\end{tabular}
\end{center}
%
The argument \textit{dest} is the destination file
(without extension).
It should be the main file or one of the child files.
Note that further \textsf{childdoc} directives
such as |\childdocof| and |\childdocforward|
in the indicated file will be processed in this form.
The optional argument \textit{main}
passes on directly to the main file \textit{main}
while pretending to compile the child \textit{dest}.
This form behaves as if \textit{dest}
issues |\childdocof{|\textit{main}|}| right away,
and no further \textsf{childdoc} directives will be processed.

%%%%%%%%%%%%%%%%%%%%%%%%%%%%%%%%%%%%%%%%
\DescribeMacro{\...prefix}
In the alternative form |\childdocforwardprefix|,
%
\begin{center}
\begin{tabular}{l}
|% \iffalse
%
% childdoc.dtx Copyright (C) 2017-2018 Niklas Beisert
%
% This work may be distributed and/or modified under the
% conditions of the LaTeX Project Public License, either version 1.3
% of this license or (at your option) any later version.
% The latest version of this license is in
%   http://www.latex-project.org/lppl.txt
% and version 1.3 or later is part of all distributions of LaTeX
% version 2005/12/01 or later.
%
% This work has the LPPL maintenance status `maintained'.
%
% The Current Maintainer of this work is Niklas Beisert.
%
% This work consists of the files childdoc.dtx and childdoc.ins
% and the derived files childdoc.def and cdocsamp.tex with
% cdocsch1.tex, cdocsch2.tex, cdocsdrf.tex, cdocsfn1.tex, cdocsfn2.tex.
%
%<package>\ifdefined\childdocmain\endinput\fi
%<package>\ProvidesFile{childdoc.def}[2018/12/30 v2.0 child document driver]
%<samplemain>\ProvidesFile{cdocsamp.tex}[2018/12/30 v2.0 sample for childdoc]
%<*driver>
%\ProvidesFile{childdoc.drv}[2018/12/30 v2.0 childdoc reference manual file]
\PassOptionsToClass{10pt,a4paper}{article}
\documentclass{ltxdoc}

\usepackage[margin=35mm]{geometry}
\usepackage{hyperref}
\usepackage{hyperxmp}
\usepackage[usenames]{color}

\hypersetup{colorlinks=true}
\hypersetup{pdfstartview=FitH}
\hypersetup{pdfpagemode=UseNone}
\hypersetup{pdfsource={}}
\hypersetup{pdflang={en-UK}}
\hypersetup{pdfcopyright={Copyright 2017-2018 Niklas Beisert.
  This work may be distributed and/or modified under the
  conditions of the LaTeX Project Public License, either version 1.3
  of this license or (at your option) any later version.}}
\hypersetup{pdflicenseurl={http://www.latex-project.org/lppl.txt}}
\hypersetup{pdfcontactaddress={ETH Zurich, ITP, HIT K,
  Wolfgang-Pauli-Strasse 27}}
\hypersetup{pdfcontactpostcode={8093}}
\hypersetup{pdfcontactcity={Zurich}}
\hypersetup{pdfcontactcountry={Switzerland}}
\hypersetup{pdfcontactemail={nbeisert@itp.phys.ethz.ch}}
\hypersetup{pdfcontacturl={http://people.phys.ethz.ch/\xmptilde nbeisert/}}

\newcommand{\secref}[1]{\hyperref[#1]{section \ref*{#1}}}

\parskip1ex
\parindent0pt
\let\olditemize\itemize
\def\itemize{\olditemize\parskip0pt}

\begin{document}

\title{The \textsf{childdoc} Package}
\hypersetup{pdftitle={The childdoc Package}}
\author{Niklas Beisert\\[2ex]
  Institut f\"ur Theoretische Physik\\
  Eidgen\"ossische Technische Hochschule Z\"urich\\
  Wolfgang-Pauli-Strasse 27, 8093 Z\"urich, Switzerland\\[1ex]
  \href{mailto:nbeisert@itp.phys.ethz.ch}
  {\texttt{nbeisert@itp.phys.ethz.ch}}}
\hypersetup{pdfauthor={Niklas Beisert}}
\hypersetup{pdfsubject={Manual for the LaTeX2e Package childdoc}}
\date{30 December 2018, \textsf{v2.0}}
\maketitle

\begin{abstract}\noindent
\textsf{childdoc} is a \LaTeXe{} package
that enables the direct compilation
of document sections included by |\include|
to individual files.
\end{abstract}

\begingroup
\parskip0ex
\tableofcontents
\endgroup

%%%%%%%%%%%%%%%%%%%%%%%%%%%%%%%%%%%%%%%%%%%%%%%%%%%%%%%%%%%%%%%%%%%%%%%%%%%%%%%%
%%%%%%%%%%%%%%%%%%%%%%%%%%%%%%%%%%%%%%%%%%%%%%%%%%%%%%%%%%%%%%%%%%%%%%%%%%%%%%%%
\section{Introduction}

\LaTeX{} provides a mechanism to structure a large document (such as a book)
into a main file and several child files (containing the chapters)
using the |\include| command.
This mechanism is beneficial for documents
which span hundreds of pages in order to
make the source file(s) more manageable.
Moreover, compilation can be restricted to
selected child files by means of the |\includeonly| command.
The latter feature can be used to reduce the compilation time while editing
(this was significantly more useful in the earlier days of \LaTeX{})
or to generate a smaller document which is easier to navigate.
Another application of |\includeonly| is to generate
documents consisting of selected parts of the complete document.

However, there are a few drawbacks of the plain |\include| mechanism:
\begin{itemize}
\item
The child files cannot be compiled on their own,
they can only be compiled via the main file.
A naive editing environment
(such as a text editor with an option
to have the current file processed by \LaTeX)
may require one to switch to the main file before compiling;
attempting to compile the child file produces errors.
\item
The main file must be modified (each time)
to adjust the |\includeonly| command
to the present needs. This easily leaves the main file in a messy state.
\item
The generated document will always carry the filename
of the main document. This is inconvenient if
several child files are to be compiled and
to be kept for distribution.
\end{itemize}

The present package provides a simple interface
to make child files individually compilable by \LaTeX{}.
Compiling a child file then has the same effect as compiling
the main file with an |\includeonly| command
to select the appropriate child.
Moreover the generated document will carry the name of the child
rather than the main file.
This resolves all three above issues.

This feature is meant to make the editing of books,
thesis documents and lecture notes somewhat more convenient.
However, the package can also be used efficiently for
composing a series of documents (such as exercise sheets)
which are typically distributed individually.
It then assists the author in generating the individual documents
(potentially in different versions)
as well as a document containing the collected series.
Another application is in developing style files
or other kinds of included material
where compilation of the style file could redirect
to a sample or test file.

%%%%%%%%%%%%%%%%%%%%%%%%%%%%%%%%%%%%%%%%%%%%%%%%%%%%%%%%%%%%%%%%%%%%%%%%%%%%%%%%
%%%%%%%%%%%%%%%%%%%%%%%%%%%%%%%%%%%%%%%%%%%%%%%%%%%%%%%%%%%%%%%%%%%%%%%%%%%%%%%%
\section{Usage}

First of all, the package \textsf{childdoc} is \emph{not} a standard
\LaTeXe{} |.sty| style file! Therefore it needs to be invoked in
a non-standard way.

%%%%%%%%%%%%%%%%%%%%%%%%%%%%%%%%%%%%%%%%%%%%%%%%%%%%%%%%%%%%%%%%%%%%%%%%%%%%%%%%
\subsection{Included Files}
\label{sec:include}

%%%%%%%%%%%%%%%%%%%%%%%%%%%%%%%%%%%%%%%%
\DescribeMacro{\childdocmain}
To use the package, add the commands
\begin{center}
\begin{tabular}{l}
|\input{childdoc.def}|\\
|\childdocmain{}|\\
\end{tabular}
\end{center}
at the very top of the main \LaTeX{} file,
in particular \emph{before} the |\documentclass| statement!
The argument of |\childdocmain| should be left empty
(but it must be present).

%%%%%%%%%%%%%%%%%%%%%%%%%%%%%%%%%%%%%%%%
\DescribeMacro{\childdocof}
Furthermore, add the commands
\begin{center}
\begin{tabular}{l}
|\input{childdoc.def}|\\
|\childdocof{|\textit{main}|}|\\
\end{tabular}
\end{center}
at the top of every child file \textit{child}
which is included by |\include{|\textit{child}|}|
from within the main file
(or at least for those files to be compiled individually).
The argument \textit{main} must be the filename of the main file.

There are a couple of
considerations in setting up the main and child documents:

%%%%%%%%%%%%%%%%%%%%%%%%%%%%%%%%%%%%%%%%
\paragraph{Restrictions.}

Please note the following restrictions:
\begin{itemize}
\item
|\childdocmain| must be called with one argument \textit{main}
to ensure compatibility with earlier version of the package.
It must either be empty (|\childdocmain{}|)
or precisely match the filename of the main file in which it is specified.
See \secref{sec:detection} for further information.
\item
The filename \textit{main} must be specified without the |.tex| extension.
\item
The filename \textit{main} is case sensitive
(even in case-insensitive file systems)
due to internal string comparison.
\item
The argument \textit{main} should be fully expanded, it cannot be a macro.
\item
Subdirectories and special characters should be avoided in filenames.
\item
The command |\childdocmain{|\textit{main}|}| must be followed by a whitespace.
It should not be followed immediately by another command
or by a comment mark `|%|'.
This is because the \TeX{} parser reads the token immediately following
the argument of |\childdocmain| and puts it
at the beginning of every child section;
however, a white\-space is ignored.
\end{itemize}

%%%%%%%%%%%%%%%%%%%%%%%%%%%%%%%%%%%%%%%%
\paragraph{Content of Main File.}

It is advisable to place all content in the child files included by |\include|.
Any output contained in the main file will appear in all child documents
unless suppressed manually;
it cannot be suppressed automatically by the |\includeonly| directive
and thus should normally be avoided.
A method to include some content in the main file
by means of conditional processing is described in \secref{sec:conditional}.

%%%%%%%%%%%%%%%%%%%%%%%%%%%%%%%%%%%%%%%%
\paragraph{Page Numbering.}

When only a part of the document is compiled,
the appropriate numbering of pages
(as well as other status parameters)
is determined from the |.aux| files.
The latter contain information from previous passes.
However this information needs to propagate through
all intermediate child documents.
Therefore the page numbering in child documents may well
be inconsistent until the complete document is compiled at least once.

A useful (if unconventional) way to always ensure a consistent
page numbering is to restart the numbering in each child document
and denote the pages by `\textit{child}|.|\textit{page}'
where \textit{child} represents the chapter/section number of the child file.
This can be achieved by the command
|\numberwithin{page}{|\textit{child}|}|
of the \textsf{amsmath} package
where \textit{child} can be |chapter| or |section|
depending on the chosen structuring.
Alternatively, one can modify the macro |\thepage| appropriately
and reset the counter |page| at the start of each child file.

%%%%%%%%%%%%%%%%%%%%%%%%%%%%%%%%%%%%%%%%%%%%%%%%%%%%%%%%%%%%%%%%%%%%%%%%%%%%%%%%
\subsection{Conditional Processing}
\label{sec:conditional}

The package provides a mechanism to compile different versions
of a document. To customise the versions further some conditional processing
can come in handy to distinguish which version is being compiled.
The package provides two macros to describe the compilation context:

%%%%%%%%%%%%%%%%%%%%%%%%%%%%%%%%%%%%%%%%
\DescribeMacro{\ifchilddoc}
The conditional |\ifchilddoc| distinguishes between the compilation of
child documents and the main document:
%
\begin{center}
|\ifchilddoc |\textit{child-code}| |[|\||else |\textit{main-code}]| \||fi|
\end{center}

%%%%%%%%%%%%%%%%%%%%%%%%%%%%%%%%%%%%%%%%
\DescribeMacro{\childdocname}
\DescribeMacro{\childdocjob}
The macro |\childdocname| contains the filename (without extension)
of the main or child file being processed.
Note that |\childdocjob| will always contain the name of the main file.

%%%%%%%%%%%%%%%%%%%%%%%%%%%%%%%%%%%%%%%%
\paragraph{Title Page.}

Conditional processing can be used to include a title or banner page
in the main document when proper precautions are taken.
Importantly, the code in the main file should ensure that the page counter
(as well as other status parameters which are stored in the |.aux| files)
takes the same value after the conditional processing.
Otherwise the page numbers may take divergent values
depending on which part is compiled.

For example, a title page could be declared by:
%
\begin{center}
\begin{tabular}{l}
|\ifchilddoc\||else|\\
|\addtocounter{page}{-1}|\\
\textit{code for title page}\\
|\newpage|\\
|\||fi|
\end{tabular}
\end{center}
%
A banner page for the child documents can be generated by:
%
\begin{center}
\begin{tabular}{l}
|\ifchilddoc|\\
|\addtocounter{page}{-1}|\\
\textit{code for banner page}\\
|\newpage|\\
|\||fi|
\end{tabular}
\end{center}
%
Here one could write a message such as:
\begin{center}
|This is the part \childdocname{} of \childdocjob{}.|
\end{center}

%%%%%%%%%%%%%%%%%%%%%%%%%%%%%%%%%%%%%%%%%%%%%%%%%%%%%%%%%%%%%%%%%%%%%%%%%%%%%%%%
\subsection{Flags}
\label{sec:flags}

The package makes it easy to generate different versions
of the main or child documents.
To this end compilation flags can be defined
and assigned different default values.
They will be particularly useful in conjunction
with the forwarding mechanism described in \secref{sec:forward}.

For example, it may be useful to have a flag |\version|
which can be set to |draft| or |final|.
The document source will contain some conditional code
depending on the value of |\version|.
Suppose further, the flag should default to |final| for the main file
and to |draft| for child files
which is a natural assignment for editing the document.
This is achieved by placing the following code
in the preamble of the main document
(below the |\childdocmain| directive):
%
\begin{center}
\begin{tabular}{l}
|\ifchilddoc|\\
|\providecommand{\version}{draft}|\\
|\||else|\\
|\providecommand{\version}{final}|\\
|\||fi|
\end{tabular}
\end{center}
%
The definition by |\providecommand| makes sure
that previous definitions are not overwritten.
Further statements |\providecommand{\version}{...}|
can thus be added before the above code to override it.

For the main file, one might add a line
(between |\childdocmain| and the above block)
%
\begin{center}
|%\ifchilddoc\||else\providecommand{\version}{draft}\||fi|
\end{center}
%
which can be uncommented to produce a draft version.
Likewise one can add a line to the very top of a child file
(above the |\childdocof{|\textit{main}|}| directive)
%
\begin{center}
|%\providecommand{\version}{final}|
\end{center}
%
which can be uncommented to produce the final version of this child document.

%%%%%%%%%%%%%%%%%%%%%%%%%%%%%%%%%%%%%%%%%%%%%%%%%%%%%%%%%%%%%%%%%%%%%%%%%%%%%%%%
\subsection{Forwarding}
\label{sec:forward}

Different versions of the main or child documents
using compilation flags as described in \secref{sec:flags}
can be (permanently) stored in different files
for convenient compilation, viewing and distribution.
To this end, the package defines a command
to pass on compilation to a different file:

%%%%%%%%%%%%%%%%%%%%%%%%%%%%%%%%%%%%%%%%
\DescribeMacro{\childdocforward}
The command |\childdocforward| redirects processing to
another source file:
%
\begin{center}
\begin{tabular}{l}
|\input{childdoc.def}|\\
|\childdocforward[|\textit{main}|]{|\textit{dest}|}|\\
\end{tabular}
\end{center}
%
The argument \textit{dest} is the destination file
(without extension).
It should be the main file or one of the child files.
Note that further \textsf{childdoc} directives
such as |\childdocof| and |\childdocforward|
in the indicated file will be processed in this form.
The optional argument \textit{main}
passes on directly to the main file \textit{main}
while pretending to compile the child \textit{dest}.
This form behaves as if \textit{dest}
issues |\childdocof{|\textit{main}|}| right away,
and no further \textsf{childdoc} directives will be processed.

%%%%%%%%%%%%%%%%%%%%%%%%%%%%%%%%%%%%%%%%
\DescribeMacro{\...prefix}
In the alternative form |\childdocforwardprefix|,
%
\begin{center}
\begin{tabular}{l}
|\input{childdoc.def}|\\
|\childdocforwardprefix[|\textit{main}|]{|\textit{prefix}|}{|\textit{dest}|}|
\end{tabular}
\end{center}
%
the destination file is determined by a pattern
depending on the current file:
To make this work, the current file must be called
`{\textit{prefix}\hspace{0.2em}\textit{suffix}}'
with \textit{prefix} matching precisely the argument.
Processing is then passed on to the file
`{\textit{dest}\hspace{0.2em}\textit{suffix}}'.
Surely, the same effect is achieved by
directly specifying the
argument `{\textit{dest}\hspace{0.2em}\textit{suffix}}'
in the first form.
However, that requires to set up a different file
for each child. With the alternative form of the command
all these files can have exactly the same content
which simplifies setting them up and maintaining them.

For example, the following file |draft.tex|
with a compilation flag |\version| as described in \secref{sec:flags}
compiles the main document as a draft:
%
\begin{center}
\begin{tabular}{l}
|\def\version{draft}|\\
|\input{childdoc.def}|\\
|\childdocforward{|\textit{main}|}|
\end{tabular}
\end{center}
%
Likewise, the following files |final|\textit{nn}|.tex|
compile the final version of the child document
|child|\textit{nn}|.tex|:
%
\begin{center}
\begin{tabular}{l}
|\def\version{final}|\\
|\input{childdoc.def}|\\
|\childdocforwardprefix{final}{child}|
\end{tabular}
\end{center}
%

Note that when several versions of a main file and/or of each child file
are to be generated, it may be convenient to set up a |Makefile| or
shell script to automatise the process.

%%%%%%%%%%%%%%%%%%%%%%%%%%%%%%%%%%%%%%%%%%%%%%%%%%%%%%%%%%%%%%%%%%%%%%%%%%%%%%%%
\subsection{Command Line Processing}
\label{sec:commandline}

The effect of redirection files can also be achieved by invoking
the \LaTeX{} compiler with a more elaborate command line.
Most conveniently this should be done as part
of a shell script or a |Makefile|.

When using \textsf{childdoc} in the main file, the following
command lines effectively perform a redirection
(note that depending on the shell being used,
backslashes may have to be doubled: `|\|' $\to$ `|\\|'):
%
\begin{center}
|... -jobname "|\textit{target}|" |\\|"|[\textit{flags}]%
|\input{childdoc.def}\childdocforward[|\textit{main}|]{|\textit{dest}|}"|
\end{center}
%
Here \textit{target} is the name of the output file,
\textit{main} is the name of the main file
and \textit{dest} is the name of the main or child file to be processed
(all filenames without extensions).
The optional argument \textit{main} can be omitted
if \textit{main} matches \textit{dest}.
Optionally, compilation \textit{flags} can be defined via |\def| commands.
This command line makes the \TeX{} engine believe
it is compiling the file \textit{target}
whose content is specified as the latter parameter.
The provided code then forwards the processing to
\textit{main} or \textit{dest} as described in \secref{sec:forward}.

%%%%%%%%%%%%%%%%%%%%%%%%%%%%%%%%%%%%%%%%%%%%%%%%%%%%%%%%%%%%%%%%%%%%%%%%%%%%%%%%
\subsection{Include by Input}
\label{sec:input}

Including child documents by |\include| has some restrictions by design.
Most notably, the content of a child document always occupies
its own set of pages; pages cannot be shared between child documents.
Usually, this behaviour makes perfect sense
because each child document contain an essential part of the document.
However, in some situations it may be desirable to compose
a document from a collection of parts
without having mandatory page breaks between then.
For this case, the package
provides a mechanism to include parts
by |\input| which can also be processed individually.
However, by construction this mechanism
requires manual handling of the content to be output.

%%%%%%%%%%%%%%%%%%%%%%%%%%%%%%%%%%%%%%%%
\DescribeMacro{\ifchilddocmanual}
The main file should be prepared as usual, see \secref{sec:include}.
However, the document body must make a distinction
between processing of an individual part and of the main document, e.g.:
%
\begin{center}
\begin{tabular}{l}
|\ifchilddocmanual|\\
|\input{\childdocname}|\\
|\||else|\\
\textit{document body with }|\input{|\textit{part}|}|\\
|\||fi|
\end{tabular}
\end{center}
%
The conditional |\ifchilddocmanual| is true whenever
a part to be included by |\input| is being compiled,
and the name of the part is stored in |\childdocname|.

%%%%%%%%%%%%%%%%%%%%%%%%%%%%%%%%%%%%%%%%
\DescribeMacro{\childdocby}
Each part to be included by |\input| should start with:
%
\begin{center}
\begin{tabular}{l}
|\input{childdoc.def}|\\
|\childdocby{|\textit{main}|}|\\
\end{tabular}
\end{center}
%
The directive |\childdocby| is similar to |\childdocof|
described in \secref{sec:include},
but the subsequent selection of content must be done manually.
To that end, both |\ifchilddoc| and |\ifchilddocmanual|
will be true upon processing of a part,
and the name of the part is stored in |\childdocname|.
Note that |\jobname| will be set to the filename of the current part
so that each part receives an individual |.aux| file
that does not interfere with the |.aux| file(s) of the main document.
This behaviour can be altered by the alternative form
|\childdocby[*]{|\textit{main}|}| (with a non-empty optional argument)
which uses the |.aux| file of the main document
by setting |\jobname| to \textit{main}.

%%%%%%%%%%%%%%%%%%%%%%%%%%%%%%%%%%%%%%%%%%%%%%%%%%%%%%%%%%%%%%%%%%%%%%%%%%%%%%%%
\subsection{Driver Development}
\label{sec:driver}

The \textsf{childdoc} mechanism can also be use for the development
of definition files such as \LaTeX{} styles or classes.
This case differs from the above setup with multiple parts
included by |\include| in that no |\includeonly| should be invoked.
This can be achieved by starting the include file
(before |\ProvidesPackage|) with:
%
\begin{center}
\begin{tabular}{l}
|\input{childdoc.def}|\\
|\childdocforward{|\textit{main}|}|\\
\end{tabular}
\end{center}
%
or alternatively with:
%
\begin{center}
\begin{tabular}{l}
|\input{childdoc.def}|\\
|\childdocby{|\textit{main}|}|\\
\end{tabular}
\end{center}
%
Both forms have slightly different effects as described above.
The main file is prepared as usual, see \secref{sec:include}.

%%%%%%%%%%%%%%%%%%%%%%%%%%%%%%%%%%%%%%%%%%%%%%%%%%%%%%%%%%%%%%%%%%%%%%%%%%%%%%%%
\subsection{Legacy Detection}
\label{sec:detection}

The directive |\childdocmain| in the main file can detect
whether the complete document or merely a child is to be compiled
even without using the directive |\childdocof|.
This method is deprecated because it is less robust
and there is no compelling reason to use it;
it is merely provided for backward compatibility
and it may be removed in future versions.

If the detection mechanism is to be used,
it is mandatory to correctly specify
the filename of the main file as the argument of |\childdocmain|:
%
\begin{center}
\begin{tabular}{l}
|\input{childdoc.def}|\\
|\childdocmain{|\textit{main}|}|\\
\end{tabular}
\end{center}
%
If |\jobname| does not match the argument \textit{main} of |\childdocmain|,
it is assumed that |\jobname| points to the child file to be compiled.
When using |\childdocmain| with the main file specified as argument,
it suffices to start a child file
with just |\input{|\textit{main}|}|
without loading of the package and using |\childdocof|.
If instead all processing is done
with the appropriate \textsf{childdoc} directives,
the argument of \textit{main} of |\childdocmain| can be empty.

An alternative version of the command line processing described
in \secref{sec:commandline} using the detection mechanism reads:
%
\begin{center}
|... -jobname "|\textit{target}|" "|[\textit{flags}]%
[|\def\jobname{|\textit{dest}|}|]|\input{|\textit{main}|}"|
\end{center}

%%%%%%%%%%%%%%%%%%%%%%%%%%%%%%%%%%%%%%%%%%%%%%%%%%%%%%%%%%%%%%%%%%%%%%%%%%%%%%%%
\subsection{Manual Code}
\label{sec:manual}

In case one cannot be certain whether the definitions file |childdoc.def|
is installed on the target \TeX{} distribution
and one prefers not to ship it,
it is conceivable to paste a few relevant commands into the sources.

To that end, drop all statements |\input{childdoc.def}|
and perform the replacements as outlined below.
Instead of |\childdocmain{|\textit{main}|}| add the following code
to the top of the main file:
%
\begin{center}
\begin{tabular}{l}
|\||ifdefined\childdocname\endinput\||fi\newif\ifchilddoc|\\
|\edef\childdocname{\scantokens\expandafter{\jobname\noexpand}}|\\
|\def\childdocmain{|\textit{main}|}\||ifx\childdocmain\childdocname\||else|\\
|\childdoctrue\includeonly{\childdocname}\let\jobname\childdocmain\||fi|\\
\end{tabular}
\end{center}
%
Instead of |\childdocof{|\textit{main}|}| just include the main file
at the top of each child file:
%
\begin{center}
|\input{|\textit{main}|}|
\end{center}
%
A simple redirection |\childdocforward{|\textit{dest}|}| is achieved by:
%
\begin{center}
|\def\jobname{|\textit{dest}|}\input{\jobname}|
\end{center}
%
The redirection with prefix
|\childdocforwardprefix[|\textit{prefix}|]{|\textit{dest}|}|
is accomplished by:
%
\begin{center}
\begin{tabular}{l}
|{\edef\jobname{\scantokens\expandafter{\jobname\noexpand}}|\\
|\def\redirectjob |\textit{prefix}|#1~~~{\gdef\jobname{|\textit{dest}|#1}}|\\
|\expandafter\redirectjob\jobname~~~}\input{\jobname}|
\end{tabular}
\end{center}

In an alternative approach,
child documents can be compiled by a specific command line
without additional code or specific definitions:
%
\begin{center}
|... -jobname "|\textit{target}|" "|[\textit{flags}]%
|\includeonly{|\textit{dest}|}\input{|\textit{main}|}"|
\end{center}
%

%%%%%%%%%%%%%%%%%%%%%%%%%%%%%%%%%%%%%%%%%%%%%%%%%%%%%%%%%%%%%%%%%%%%%%%%%%%%%%%%
%%%%%%%%%%%%%%%%%%%%%%%%%%%%%%%%%%%%%%%%%%%%%%%%%%%%%%%%%%%%%%%%%%%%%%%%%%%%%%%%
\section{Information}

%%%%%%%%%%%%%%%%%%%%%%%%%%%%%%%%%%%%%%%%%%%%%%%%%%%%%%%%%%%%%%%%%%%%%%%%%%%%%%%%
\subsection{Copyright}

Copyright \copyright{} 2017--2018 Niklas Beisert

This work may be distributed and/or modified under the
conditions of the \LaTeX{} Project Public License, either version 1.3
of this license or (at your option) any later version.
The latest version of this license is in
  \url{http://www.latex-project.org/lppl.txt}
and version 1.3 or later is part of all distributions of \LaTeX{}
version 2005/12/01 or later.

This work has the LPPL maintenance status `maintained'.

The Current Maintainer of this work is Niklas Beisert.

This work consists of the files |README.txt|, |childdoc.ins| and |childdoc.dtx|
as well as the derived files |childdoc.def|, |cdocsamp.tex|
with |cdocsch1.tex|, |cdocsch2.tex|, |cdocspt3.tex|, |cdocspt4.tex|,
|cdocsdrf.tex|, |cdocsfn1.tex|, |cdocsfn2.tex|
as well as |childdoc.pdf|.

%%%%%%%%%%%%%%%%%%%%%%%%%%%%%%%%%%%%%%%%%%%%%%%%%%%%%%%%%%%%%%%%%%%%%%%%%%%%%%%%
\subsection{Files and Installation}

The package consists of the files:
%
\begin{center}
\begin{tabular}{ll}
    |README.txt|   & readme file \\
    |childdoc.ins| & installation file \\
    |childdoc.dtx| & source file \\
    |childdoc.def| & definition file \\
    |cdocsamp.tex| & sample main file \\
    |cdocsch1.tex| & sample include file \\
    |cdocsch2.tex| & sample include file \\
    |cdocspt3.tex| & sample part file \\
    |cdocspt4.tex| & sample part file \\
    |cdocsdrf.tex| & sample redirection file \\
    |cdocsfn1.tex| & sample redirection file \\
    |cdocsfn2.tex| & sample redirection file \\
    |childdoc.pdf| & manual
\end{tabular}
\end{center}
%
The distribution consists of the files
|README.txt|, |childdoc.ins| and |childdoc.dtx|.
%
\begin{itemize}
\item
Run (pdf)\LaTeX{} on |childdoc.dtx|
to compile the manual |childdoc.pdf| (this file).
\item
Run \LaTeX{} on |childdoc.ins| to create the definitions file |childdoc.def|
and the sample |cdocsamp.tex| with include files
|cdocsch1.tex|, |cdocsch2.tex|, |cdocspt3.tex|, |cdocspt4.tex|,
|cdocsdrf.tex|, |cdocsfn1.tex|, |cdocsfn2.tex|.
Then copy the file |childdoc.def| to an appropriate directory of your \LaTeX{}
distribution, e.g.\ \textit{texmf-root}|/tex/latex/childdoc|.
\end{itemize}

%%%%%%%%%%%%%%%%%%%%%%%%%%%%%%%%%%%%%%%%%%%%%%%%%%%%%%%%%%%%%%%%%%%%%%%%%%%%%%%%
\subsection{Related CTAN Packages}

There are several other packages which offer a similar functionality:
%
\begin{itemize}
\item
The packages
\href{http://ctan.org/pkg/docmute}{\textsf{docmute}},
\href{http://ctan.org/pkg/includex}{\textsf{includex}} and
\href{http://ctan.org/pkg/standalone}{\textsf{standalone}}
provide commands to include only the document body of
a child file thus allowing both files to be compiled individually.
\item
The packages \href{http://ctan.org/pkg/subdocs}{\textsf{subdocs}}
and \href{http://ctan.org/pkg/subfiles}{\textsf{subfiles}}
provide structures in which the main and child documents can be
encapsulated and allowing them to be compiled individually.
The inclusion mechanism is different from the conventional |\include|.
\item
The package \href{http://ctan.org/pkg/combine}{\textsf{combine}}
is an elaborate solution to combine several documents into one.
\end{itemize}
%
See also the CTAN topic \href{http://ctan.org/topic/subdocs}{\textsf{subdocs}}
for further related packages.
The present package differs from the above solutions in that
a document structure constructed with the conventional |\include| mechanism
just needs two extra commands at the top of every file
such that all constituent files can be compiled individually.

%%%%%%%%%%%%%%%%%%%%%%%%%%%%%%%%%%%%%%%%%%%%%%%%%%%%%%%%%%%%%%%%%%%%%%%%%%%%%%%%
%\subsection{Feature Suggestions}
%
%The following is a list of features which may be useful for future
%versions of this package:
%%
%\begin{itemize}
%\item
%\ldots
%\end{itemize}

%%%%%%%%%%%%%%%%%%%%%%%%%%%%%%%%%%%%%%%%%%%%%%%%%%%%%%%%%%%%%%%%%%%%%%%%%%%%%%%%
\subsection{Revision History}

%%%%%%%%%%%%%%%%%%%%%%%%%%%%%%%%%%%%%%%%
\paragraph{v2.0:} 2018/12/30

\begin{itemize}
\item
immediate forward processing
\item
added |\childdocby| mechanism
\item
manual restructured
\end{itemize}

%%%%%%%%%%%%%%%%%%%%%%%%%%%%%%%%%%%%%%%%
\paragraph{v1.6:} 2018/01/17

\begin{itemize}
\item
application for development of include files
\item
corrections to manual
\end{itemize}

%%%%%%%%%%%%%%%%%%%%%%%%%%%%%%%%%%%%%%%%
\paragraph{v1.5:} 2017/05/21

\begin{itemize}
\item
more complete structuring introduced
\item
|\childdocof| introduced
\item
|\childdoc| renamed to |\childdocmain|
\item
|\childredirect| renamed to |\childdocforward| and |\childdocforwardprefix|
and functionality expanded
\end{itemize}

%%%%%%%%%%%%%%%%%%%%%%%%%%%%%%%%%%%%%%%%
\paragraph{v1.0:} 2017/04/27

\begin{itemize}
\item
manual and install package
\item
first version published on CTAN
\end{itemize}

%%%%%%%%%%%%%%%%%%%%%%%%%%%%%%%%%%%%%%%%
\paragraph{v0.6:} 2017/04/26

\begin{itemize}
\item
redirection mechanism added
\end{itemize}

%%%%%%%%%%%%%%%%%%%%%%%%%%%%%%%%%%%%%%%%
\paragraph{v0.5:} 2017/04/26

\begin{itemize}
\item
functionality in definition file
\end{itemize}


%%%%%%%%%%%%%%%%%%%%%%%%%%%%%%%%%%%%%%%%%%%%%%%%%%%%%%%%%%%%%%%%%%%%%%%%%%%%%%%%
%%%%%%%%%%%%%%%%%%%%%%%%%%%%%%%%%%%%%%%%%%%%%%%%%%%%%%%%%%%%%%%%%%%%%%%%%%%%%%%%
%%%%%%%%%%%%%%%%%%%%%%%%%%%%%%%%%%%%%%%%%%%%%%%%%%%%%%%%%%%%%%%%%%%%%%%%%%%%%%%%
\appendix

\settowidth\MacroIndent{\rmfamily\scriptsize 000\ }

 \DocInput{childdoc.dtx}

\end{document}
%</driver>
% \fi
%
% %%%%%%%%%%%%%%%%%%%%%%%%%%%%%%%%%%%%%%%%%%%%%%%%%%%%%%%%%%%%%%%%%%%%%%%%%%%%%%
% %%%%%%%%%%%%%%%%%%%%%%%%%%%%%%%%%%%%%%%%%%%%%%%%%%%%%%%%%%%%%%%%%%%%%%%%%%%%%%
% \section{Sample}
%\iffalse
%<*samplemain>
%\fi
%
% The following presents a sample document
% with two chapters, two parts, a title page,
% a compile flag as well as three forwarding files to set the flag.
% It consists of eight |.tex| files:
% \begin{center}
% \begin{tabular}{ll}
% |cdocsamp.tex|&main file\\
% |cdocsch1.tex|&include file for chapter 1\\
% |cdocsch2.tex|&include file for chapter 2\\
% |cdocspt3.tex|&include file for part 3\\
% |cdocspt4.tex|&include file for part 4\\
% |cdocsdrf.tex|&forwarding file for main file in draft mode\\
% |cdocsfi1.tex|&forwarding file for final version of chapter 1\\
% |cdocsfi2.tex|&forwarding file for final version of chapter 2\\
% \end{tabular}
% \end{center}
% Each of the eight files can be compiled directly by the \LaTeX{} compiler.
%
% %%%%%%%%%%%%%%%%%%%%%%%%%%%%%%%%%%%%%%
% \paragraph{Main File.}
%
% The main file is called |cdocsamp.tex|.
%
% Load the \textsf{childdoc} definitions and
% declare the filename for the main document:
%    \begin{macrocode}
\input{childdoc.def}
\childdocmain{}
%    \end{macrocode}

% Optional override for |\version| flag:
%    \begin{macrocode}
%%\ifchilddoc\else\providecommand{\version}{draft}\fi
%    \end{macrocode}

% Define the default values for the |\version| flag
% (|final| for the main file and |draft| for childs):
%    \begin{macrocode}
\ifchilddoc
\providecommand{\version}{draft}
\else
\providecommand{\version}{final}
\fi
%    \end{macrocode}

% Load the standard document class:
%    \begin{macrocode}
\documentclass[12pt]{article}
%    \end{macrocode}

% Start the document body:
%    \begin{macrocode}
\begin{document}
%    \end{macrocode}

% Declare a title page.
% Print title, part of document being processed and version flag:
%    \begin{macrocode}
\addtocounter{page}{-1}
\begin{center}
{\LARGE\bfseries{}childdoc example\par}
\vspace{1cm}
\ifchilddoc
\ifchilddocmanual part\else chapter\fi:
`\childdocname' of `\childdocjob'\par
\else
main document: `\childdocjob'\par
\fi
version: \version\par
\end{center}
\newpage
%    \end{macrocode}

% Manually include selected file,
% otherwise process as usual:
%    \begin{macrocode}
\ifchilddocmanual
\section*{part `\childdocname'}
\input{\childdocname}
\else
%    \end{macrocode}

% Include the two chapters:
%    \begin{macrocode}
\include{cdocsch1}
\include{cdocsch2}
%    \end{macrocode}

% Include the two parts unless only chapters should be displayed:
%    \begin{macrocode}
\ifchilddoc\else
\section{part three}
\input{cdocspt3}
\section{part four}
\input{cdocspt4}
\fi
%    \end{macrocode}

% Process as usual until here:
%    \begin{macrocode}
\fi
%    \end{macrocode}

% End of document body:
%    \begin{macrocode}
\end{document}
%    \end{macrocode}
%\iffalse
%</samplemain>
%\fi
%
% %%%%%%%%%%%%%%%%%%%%%%%%%%%%%%%%%%%%%%
% \paragraph{Chapter Include Files.}
%
% The include files are called |cdocsch1.tex| and |cdocsch2.tex|.
%
%\iffalse
%<*samplechap1|samplechap2>
%\fi

% Optional override for |\version| flag:
%    \begin{macrocode}
%%\providecommand{\version}{final}
%    \end{macrocode}

% Include the main document:
%    \begin{macrocode}
\input{childdoc.def}
\childdocof{cdocsamp}
%    \end{macrocode}

%\iffalse
%</samplechap1|samplechap2>
%\fi
%
%\iffalse
%<*samplechap1>
%\fi
% Some text for chapter 1:
%    \begin{macrocode}
\section{one}
some text in chapter one
%    \end{macrocode}

%\iffalse
%</samplechap1>
%\fi
% Some text for chapter 2:
%\iffalse
%<*samplechap2>
%\fi
%    \begin{macrocode}
\section{two}
more text in chapter two
%    \end{macrocode}

%\iffalse
%</samplechap2>
%\fi
%
% %%%%%%%%%%%%%%%%%%%%%%%%%%%%%%%%%%%%%%
% \paragraph{Part Include Files.}
%
% The include files are called |cdocspt3.tex| and |cdocspt4.tex|.
%
%\iffalse
%<*samplepart3|samplepart4>
%\fi

% Optional override for |\version| flag:
%    \begin{macrocode}
%%\providecommand{\version}{final}
%    \end{macrocode}

% Include the main document:
%    \begin{macrocode}
\input{childdoc.def}
\childdocby{cdocsamp}
%    \end{macrocode}

%\iffalse
%</samplepart3|samplepart4>
%\fi
%
%\iffalse
%<*samplepart3>
%\fi
% Some text for part 3:
%    \begin{macrocode}
some text in part three
%    \end{macrocode}

%\iffalse
%</samplepart3>
%\fi
% Some text for part 4:
%\iffalse
%<*samplepart4>
%\fi
%    \begin{macrocode}
more text in part four
%    \end{macrocode}

%\iffalse
%</samplepart4>
%\fi
%
% %%%%%%%%%%%%%%%%%%%%%%%%%%%%%%%%%%%%%%
% \paragraph{Forwarding for a Complete Draft.}
%
% The following forwarding file |cdocsdrf.tex|
% compiles the main document in draft mode:
%\iffalse
%<*sampledraft>
%\fi
%    \begin{macrocode}
\def\version{draft}
\input{childdoc.def}
\childdocforward{cdocsamp}
%    \end{macrocode}

%\iffalse
%</sampledraft>
%\fi
%
% %%%%%%%%%%%%%%%%%%%%%%%%%%%%%%%%%%%%%%
% \paragraph{Forwarding for Final Version of the Chapters.}
%
% The following forwarding files |cdocsfn1.tex| and |cdocsfn2.tex|
% (with identical content)
% compile the final versions of the child documents
% |cdocsch1.tex| and |cdocsch2.tex|, respectively:
%\iffalse
%<*samplefinal>
%\fi
%    \begin{macrocode}
\def\version{final}
\input{childdoc.def}
\childdocforwardprefix[cdocsamp]{cdocsfn}{cdocsch}
%    \end{macrocode}

%\iffalse
%</samplefinal>
%\fi
%
% %%%%%%%%%%%%%%%%%%%%%%%%%%%%%%%%%%%%%%
% \paragraph{Command Line Processing.}
%
% The following three command lines generate the output files
% |cdocscld|, |cdocscl1| and |cdocscl2|
% which should be identical to
% |cdocsdrf|, |cdocsch1| and |cdocsfn2|, respectively:
% \begin{center}
% \begin{tabular}{l}
% |latex -jobname cdocscld \|\\
% |  "\def\version{draft}\input{childdoc.def}\childdocforward{cdocsamp}"|\\
% |latex -jobname cdocscl1 \|\\
% |  "\input{childdoc.def}\childdocforward[cdocsamp]{cdocsch1}"|\\
% |latex -jobname cdocscl2 \|\\
% |  "\def\version{final}\input{childdoc.def}\childdocforward{cdocsch2}"|
% \end{tabular}
% \end{center}
% Note that the trailing backslash on each first line
% merely continues the input to the second line
% (for convenient cut ant paste).
% Furthermore, the command |latex| can be replaced by any
% of its alternative versions such as |pdflatex|.
%
% %%%%%%%%%%%%%%%%%%%%%%%%%%%%%%%%%%%%%%%%%%%%%%%%%%%%%%%%%%%%%%%%%%%%%%%%%%%%%%
% %%%%%%%%%%%%%%%%%%%%%%%%%%%%%%%%%%%%%%%%%%%%%%%%%%%%%%%%%%%%%%%%%%%%%%%%%%%%%%
% \section{Implementation}
%\iffalse
%<*package>
%\fi
%
% This section describes the definitions file |childdoc.def|.

% The definitions cannot be loaded using |\usepackage| or |\RequirePackage|
% which has a mechanism to prevent loading a style file more than once.
% When loading the definitions by means of |\input|
% multiple instances have to be prevented manually:
%\iffalse
%This code needs to be before the `\ProvidesFile' directive
%which is defined at the beginning of this file.
%Therefore it is also placed there and commented out here.
%</package>
%<*discard>
%\fi
%    \begin{macrocode}
\ifdefined\childdocmain\endinput\fi
%    \end{macrocode}
%\iffalse
%</discard>
%<*package>
%\fi
%
% \macro{\ifchilddoc}
% \macro{\ifchilddocmanual}
% The conditional |\ifchilddoc| tells whether a
% child (true) or main (false) document is being compiled.
% The conditional |\ifchilddocmanual| tells whether
% the |\includeonly| mechanism is used (false) or
% the selection of child files must be performed manually (true).
% The definitions initialise to false:
%    \begin{macrocode}
\newif\ifchilddoc
\newif\ifchilddocmanual
%    \end{macrocode}

% \macro{\childdocname}
% \macro{\childdocjob}
% The macro |\childdocname| stores the name of the main document
% to be compiled. The macro |\childdocjob| stores the name of
% the document on which the \LaTeX{} compiler was originally invoked.
% The content of |\jobname| cannot be compared
% to filenames specified in the source due to different catcodes.
% The following code rescans |\jobname|, stores the result
% in |\childdocname| and saves a copy in |\childdocjob|:
%    \begin{macrocode}
\edef\childdocname{\scantokens\expandafter{\jobname\noexpand}}
\let\childdocjob\childdocname
%    \end{macrocode}

% \macro{\childdocdisable}
% The macro |\childdocdisable| prevents the main file
% from being processed more than once.
% At this stage, the main document command |\childdocmain|
% is assumed to be called once again where it should do nothing.
% Any subsequent call to it should prevent
% a secondary processing of the main document
% It overwrites the forwarding commands
% |\childdocof| and |\childdocforward|
% with empty macros to prevent further inclusions of the main document:
%    \begin{macrocode}
\newcommand{\childdocdisable}
{
  \renewcommand{\childdocmain}[1]{\renewcommand{\childdocmain}[1]{\endinput}}
  \renewcommand{\childdocof}[1]{}
  \renewcommand{\childdocby}[2][]{}
  \renewcommand{\childdocforward}[2][]{}
  \renewcommand{\childdocdisable}{}
}
%    \end{macrocode}

% \macro{\childdocmain}
% The macro |\childdocmain| is to be called at the top of the main file
% with nothing or the main filename (without extension) as argument.
% First, it breaks loops.
% If the argument is not empty and does not match |\childdocname|
% (which is set by the first inclusion of |childdoc.def|),
% |\ifchilddoc| is set to true, |\includeonly| is applied to the child file
% and |\jobname| is set to the main file
% (for proper handling of |.aux| files):
%    \begin{macrocode}
\newcommand{\childdocmain}[1]
{
  \childdocdisable\childdocmain{}
  \if?#1?\else
    \begingroup
      \def\childdoctmp{#1}
      \ifx\childdoctmp\childdocname
        \def\childdoctmp{}
      \else
        \def\childdoctmp
        {
          \childdoctrue
          \includeonly{\childdocname}
          \def\childdocjob{#1}
          \def\jobname{#1}
        }
      \fi
      \expandafter
    \endgroup
    \childdoctmp
  \fi
}
%    \end{macrocode}

% \macro{\childdocof}
% The command |\childdocof| redirects
% compilation to the main file |#1|.
%    \begin{macrocode}
\newcommand{\childdocof}[1]
{
  \childdocdisable
  \childdoctrue
  \includeonly{\childdocname}
  \def\jobname{#1}
  \def\childdocjob{#1}
  \input{#1}
}
%    \end{macrocode}

% \macro{\childdocby}
% The command |\childdocby| ....
%    \begin{macrocode}
\newcommand{\childdocby}[2][]
{
  \childdocdisable
  \childdoctrue
  \childdocmanualtrue
  \if?#1?\else
    \def\jobname{#2}
  \fi
  \def\childdocjob{#2}
  \input{#2}
  \endinput
}
%    \end{macrocode}

% \macro{\childdocforward}
% The command |\childdocforward| redirects
% compilation to the main file or
% (if the optional argument is given) a child file.
% Parameters are set as if the main file
% or a child file starting with |\childdocof| was compiled.
% Then compilation is handed over to the main file:
%    \begin{macrocode}
\newcommand{\childdocforward}[2][]
{
  \begingroup
    \if?#1?
      \def\childdoctmp
      {
        \def\childdocname{#2}
        \def\childdocjob{#2}
        \def\jobname{#2}
        \input{#2}
        \endinput
      }
    \else
      \def\childdoctmp
      {
        \childdocdisable
        \def\childdocname{#2}
        \childdoctrue
        \includeonly{#2}
        \def\childdocjob{#1}
        \def\jobname{#1}
        \input{#1}
        \endinput
      }
    \fi
    \expandafter
  \endgroup
  \childdoctmp
}
%    \end{macrocode}

% \macro{\childdocforwardprefix}
% The command |\childdocforwardprefix| redirects
% compilation to the main or a child file by means of a pattern.
% The prefix |#1| in the current filename is replaced by |#2|
% and the suffix of the current filename is kept
% (it is assumed that the filename does not contain the substring `|~~~|'
% which is used as a delimiter).
% Compilation is handed over to the new file by |\childdocforward|:
%    \begin{macrocode}
\newcommand{\childdocforwardprefix}[3][]
{
  \begingroup
    \def\childdocextract #2##1~~~{\def\childdoctmp{\childdocforward[#1]{#3##1}}}
    \expandafter\childdocextract\childdocname~~~
    \expandafter
  \endgroup
  \childdoctmp
}
%    \end{macrocode}

% \macro{\childdoc}
% The deprecated macro |\childdoc| is a legacy version of |\childdocmain|:
%    \begin{macrocode}
\newcommand{\childdoc}{\childdocmain}
%    \end{macrocode}

% \macro{\childdocredirect}
% The deprecated macro |\childdocredirect| is a legacy version
% of |\childdocforward| and |\childdocforwardprefix|:
%    \begin{macrocode}
\newcommand{\childdocredirect}[2][]
{
  \begingroup
    \if?#1?
      \def\childdoctmp{\childdocforward{#2}}
    \else
      \def\childdoctmp{\childdocforwardprefix{#1}{#2}}
    \fi
    \expandafter
  \endgroup
  \childdoctmp
}
%    \end{macrocode}

%\iffalse
%</package>
%\fi
%
\endinput
|\\
|\childdocforwardprefix[|\textit{main}|]{|\textit{prefix}|}{|\textit{dest}|}|
\end{tabular}
\end{center}
%
the destination file is determined by a pattern
depending on the current file:
To make this work, the current file must be called
`{\textit{prefix}\hspace{0.2em}\textit{suffix}}'
with \textit{prefix} matching precisely the argument.
Processing is then passed on to the file
`{\textit{dest}\hspace{0.2em}\textit{suffix}}'.
Surely, the same effect is achieved by
directly specifying the
argument `{\textit{dest}\hspace{0.2em}\textit{suffix}}'
in the first form.
However, that requires to set up a different file
for each child. With the alternative form of the command
all these files can have exactly the same content
which simplifies setting them up and maintaining them.

For example, the following file |draft.tex|
with a compilation flag |\version| as described in \secref{sec:flags}
compiles the main document as a draft:
%
\begin{center}
\begin{tabular}{l}
|\def\version{draft}|\\
|% \iffalse
%
% childdoc.dtx Copyright (C) 2017-2018 Niklas Beisert
%
% This work may be distributed and/or modified under the
% conditions of the LaTeX Project Public License, either version 1.3
% of this license or (at your option) any later version.
% The latest version of this license is in
%   http://www.latex-project.org/lppl.txt
% and version 1.3 or later is part of all distributions of LaTeX
% version 2005/12/01 or later.
%
% This work has the LPPL maintenance status `maintained'.
%
% The Current Maintainer of this work is Niklas Beisert.
%
% This work consists of the files childdoc.dtx and childdoc.ins
% and the derived files childdoc.def and cdocsamp.tex with
% cdocsch1.tex, cdocsch2.tex, cdocsdrf.tex, cdocsfn1.tex, cdocsfn2.tex.
%
%<package>\ifdefined\childdocmain\endinput\fi
%<package>\ProvidesFile{childdoc.def}[2018/12/30 v2.0 child document driver]
%<samplemain>\ProvidesFile{cdocsamp.tex}[2018/12/30 v2.0 sample for childdoc]
%<*driver>
%\ProvidesFile{childdoc.drv}[2018/12/30 v2.0 childdoc reference manual file]
\PassOptionsToClass{10pt,a4paper}{article}
\documentclass{ltxdoc}

\usepackage[margin=35mm]{geometry}
\usepackage{hyperref}
\usepackage{hyperxmp}
\usepackage[usenames]{color}

\hypersetup{colorlinks=true}
\hypersetup{pdfstartview=FitH}
\hypersetup{pdfpagemode=UseNone}
\hypersetup{pdfsource={}}
\hypersetup{pdflang={en-UK}}
\hypersetup{pdfcopyright={Copyright 2017-2018 Niklas Beisert.
  This work may be distributed and/or modified under the
  conditions of the LaTeX Project Public License, either version 1.3
  of this license or (at your option) any later version.}}
\hypersetup{pdflicenseurl={http://www.latex-project.org/lppl.txt}}
\hypersetup{pdfcontactaddress={ETH Zurich, ITP, HIT K,
  Wolfgang-Pauli-Strasse 27}}
\hypersetup{pdfcontactpostcode={8093}}
\hypersetup{pdfcontactcity={Zurich}}
\hypersetup{pdfcontactcountry={Switzerland}}
\hypersetup{pdfcontactemail={nbeisert@itp.phys.ethz.ch}}
\hypersetup{pdfcontacturl={http://people.phys.ethz.ch/\xmptilde nbeisert/}}

\newcommand{\secref}[1]{\hyperref[#1]{section \ref*{#1}}}

\parskip1ex
\parindent0pt
\let\olditemize\itemize
\def\itemize{\olditemize\parskip0pt}

\begin{document}

\title{The \textsf{childdoc} Package}
\hypersetup{pdftitle={The childdoc Package}}
\author{Niklas Beisert\\[2ex]
  Institut f\"ur Theoretische Physik\\
  Eidgen\"ossische Technische Hochschule Z\"urich\\
  Wolfgang-Pauli-Strasse 27, 8093 Z\"urich, Switzerland\\[1ex]
  \href{mailto:nbeisert@itp.phys.ethz.ch}
  {\texttt{nbeisert@itp.phys.ethz.ch}}}
\hypersetup{pdfauthor={Niklas Beisert}}
\hypersetup{pdfsubject={Manual for the LaTeX2e Package childdoc}}
\date{30 December 2018, \textsf{v2.0}}
\maketitle

\begin{abstract}\noindent
\textsf{childdoc} is a \LaTeXe{} package
that enables the direct compilation
of document sections included by |\include|
to individual files.
\end{abstract}

\begingroup
\parskip0ex
\tableofcontents
\endgroup

%%%%%%%%%%%%%%%%%%%%%%%%%%%%%%%%%%%%%%%%%%%%%%%%%%%%%%%%%%%%%%%%%%%%%%%%%%%%%%%%
%%%%%%%%%%%%%%%%%%%%%%%%%%%%%%%%%%%%%%%%%%%%%%%%%%%%%%%%%%%%%%%%%%%%%%%%%%%%%%%%
\section{Introduction}

\LaTeX{} provides a mechanism to structure a large document (such as a book)
into a main file and several child files (containing the chapters)
using the |\include| command.
This mechanism is beneficial for documents
which span hundreds of pages in order to
make the source file(s) more manageable.
Moreover, compilation can be restricted to
selected child files by means of the |\includeonly| command.
The latter feature can be used to reduce the compilation time while editing
(this was significantly more useful in the earlier days of \LaTeX{})
or to generate a smaller document which is easier to navigate.
Another application of |\includeonly| is to generate
documents consisting of selected parts of the complete document.

However, there are a few drawbacks of the plain |\include| mechanism:
\begin{itemize}
\item
The child files cannot be compiled on their own,
they can only be compiled via the main file.
A naive editing environment
(such as a text editor with an option
to have the current file processed by \LaTeX)
may require one to switch to the main file before compiling;
attempting to compile the child file produces errors.
\item
The main file must be modified (each time)
to adjust the |\includeonly| command
to the present needs. This easily leaves the main file in a messy state.
\item
The generated document will always carry the filename
of the main document. This is inconvenient if
several child files are to be compiled and
to be kept for distribution.
\end{itemize}

The present package provides a simple interface
to make child files individually compilable by \LaTeX{}.
Compiling a child file then has the same effect as compiling
the main file with an |\includeonly| command
to select the appropriate child.
Moreover the generated document will carry the name of the child
rather than the main file.
This resolves all three above issues.

This feature is meant to make the editing of books,
thesis documents and lecture notes somewhat more convenient.
However, the package can also be used efficiently for
composing a series of documents (such as exercise sheets)
which are typically distributed individually.
It then assists the author in generating the individual documents
(potentially in different versions)
as well as a document containing the collected series.
Another application is in developing style files
or other kinds of included material
where compilation of the style file could redirect
to a sample or test file.

%%%%%%%%%%%%%%%%%%%%%%%%%%%%%%%%%%%%%%%%%%%%%%%%%%%%%%%%%%%%%%%%%%%%%%%%%%%%%%%%
%%%%%%%%%%%%%%%%%%%%%%%%%%%%%%%%%%%%%%%%%%%%%%%%%%%%%%%%%%%%%%%%%%%%%%%%%%%%%%%%
\section{Usage}

First of all, the package \textsf{childdoc} is \emph{not} a standard
\LaTeXe{} |.sty| style file! Therefore it needs to be invoked in
a non-standard way.

%%%%%%%%%%%%%%%%%%%%%%%%%%%%%%%%%%%%%%%%%%%%%%%%%%%%%%%%%%%%%%%%%%%%%%%%%%%%%%%%
\subsection{Included Files}
\label{sec:include}

%%%%%%%%%%%%%%%%%%%%%%%%%%%%%%%%%%%%%%%%
\DescribeMacro{\childdocmain}
To use the package, add the commands
\begin{center}
\begin{tabular}{l}
|\input{childdoc.def}|\\
|\childdocmain{}|\\
\end{tabular}
\end{center}
at the very top of the main \LaTeX{} file,
in particular \emph{before} the |\documentclass| statement!
The argument of |\childdocmain| should be left empty
(but it must be present).

%%%%%%%%%%%%%%%%%%%%%%%%%%%%%%%%%%%%%%%%
\DescribeMacro{\childdocof}
Furthermore, add the commands
\begin{center}
\begin{tabular}{l}
|\input{childdoc.def}|\\
|\childdocof{|\textit{main}|}|\\
\end{tabular}
\end{center}
at the top of every child file \textit{child}
which is included by |\include{|\textit{child}|}|
from within the main file
(or at least for those files to be compiled individually).
The argument \textit{main} must be the filename of the main file.

There are a couple of
considerations in setting up the main and child documents:

%%%%%%%%%%%%%%%%%%%%%%%%%%%%%%%%%%%%%%%%
\paragraph{Restrictions.}

Please note the following restrictions:
\begin{itemize}
\item
|\childdocmain| must be called with one argument \textit{main}
to ensure compatibility with earlier version of the package.
It must either be empty (|\childdocmain{}|)
or precisely match the filename of the main file in which it is specified.
See \secref{sec:detection} for further information.
\item
The filename \textit{main} must be specified without the |.tex| extension.
\item
The filename \textit{main} is case sensitive
(even in case-insensitive file systems)
due to internal string comparison.
\item
The argument \textit{main} should be fully expanded, it cannot be a macro.
\item
Subdirectories and special characters should be avoided in filenames.
\item
The command |\childdocmain{|\textit{main}|}| must be followed by a whitespace.
It should not be followed immediately by another command
or by a comment mark `|%|'.
This is because the \TeX{} parser reads the token immediately following
the argument of |\childdocmain| and puts it
at the beginning of every child section;
however, a white\-space is ignored.
\end{itemize}

%%%%%%%%%%%%%%%%%%%%%%%%%%%%%%%%%%%%%%%%
\paragraph{Content of Main File.}

It is advisable to place all content in the child files included by |\include|.
Any output contained in the main file will appear in all child documents
unless suppressed manually;
it cannot be suppressed automatically by the |\includeonly| directive
and thus should normally be avoided.
A method to include some content in the main file
by means of conditional processing is described in \secref{sec:conditional}.

%%%%%%%%%%%%%%%%%%%%%%%%%%%%%%%%%%%%%%%%
\paragraph{Page Numbering.}

When only a part of the document is compiled,
the appropriate numbering of pages
(as well as other status parameters)
is determined from the |.aux| files.
The latter contain information from previous passes.
However this information needs to propagate through
all intermediate child documents.
Therefore the page numbering in child documents may well
be inconsistent until the complete document is compiled at least once.

A useful (if unconventional) way to always ensure a consistent
page numbering is to restart the numbering in each child document
and denote the pages by `\textit{child}|.|\textit{page}'
where \textit{child} represents the chapter/section number of the child file.
This can be achieved by the command
|\numberwithin{page}{|\textit{child}|}|
of the \textsf{amsmath} package
where \textit{child} can be |chapter| or |section|
depending on the chosen structuring.
Alternatively, one can modify the macro |\thepage| appropriately
and reset the counter |page| at the start of each child file.

%%%%%%%%%%%%%%%%%%%%%%%%%%%%%%%%%%%%%%%%%%%%%%%%%%%%%%%%%%%%%%%%%%%%%%%%%%%%%%%%
\subsection{Conditional Processing}
\label{sec:conditional}

The package provides a mechanism to compile different versions
of a document. To customise the versions further some conditional processing
can come in handy to distinguish which version is being compiled.
The package provides two macros to describe the compilation context:

%%%%%%%%%%%%%%%%%%%%%%%%%%%%%%%%%%%%%%%%
\DescribeMacro{\ifchilddoc}
The conditional |\ifchilddoc| distinguishes between the compilation of
child documents and the main document:
%
\begin{center}
|\ifchilddoc |\textit{child-code}| |[|\||else |\textit{main-code}]| \||fi|
\end{center}

%%%%%%%%%%%%%%%%%%%%%%%%%%%%%%%%%%%%%%%%
\DescribeMacro{\childdocname}
\DescribeMacro{\childdocjob}
The macro |\childdocname| contains the filename (without extension)
of the main or child file being processed.
Note that |\childdocjob| will always contain the name of the main file.

%%%%%%%%%%%%%%%%%%%%%%%%%%%%%%%%%%%%%%%%
\paragraph{Title Page.}

Conditional processing can be used to include a title or banner page
in the main document when proper precautions are taken.
Importantly, the code in the main file should ensure that the page counter
(as well as other status parameters which are stored in the |.aux| files)
takes the same value after the conditional processing.
Otherwise the page numbers may take divergent values
depending on which part is compiled.

For example, a title page could be declared by:
%
\begin{center}
\begin{tabular}{l}
|\ifchilddoc\||else|\\
|\addtocounter{page}{-1}|\\
\textit{code for title page}\\
|\newpage|\\
|\||fi|
\end{tabular}
\end{center}
%
A banner page for the child documents can be generated by:
%
\begin{center}
\begin{tabular}{l}
|\ifchilddoc|\\
|\addtocounter{page}{-1}|\\
\textit{code for banner page}\\
|\newpage|\\
|\||fi|
\end{tabular}
\end{center}
%
Here one could write a message such as:
\begin{center}
|This is the part \childdocname{} of \childdocjob{}.|
\end{center}

%%%%%%%%%%%%%%%%%%%%%%%%%%%%%%%%%%%%%%%%%%%%%%%%%%%%%%%%%%%%%%%%%%%%%%%%%%%%%%%%
\subsection{Flags}
\label{sec:flags}

The package makes it easy to generate different versions
of the main or child documents.
To this end compilation flags can be defined
and assigned different default values.
They will be particularly useful in conjunction
with the forwarding mechanism described in \secref{sec:forward}.

For example, it may be useful to have a flag |\version|
which can be set to |draft| or |final|.
The document source will contain some conditional code
depending on the value of |\version|.
Suppose further, the flag should default to |final| for the main file
and to |draft| for child files
which is a natural assignment for editing the document.
This is achieved by placing the following code
in the preamble of the main document
(below the |\childdocmain| directive):
%
\begin{center}
\begin{tabular}{l}
|\ifchilddoc|\\
|\providecommand{\version}{draft}|\\
|\||else|\\
|\providecommand{\version}{final}|\\
|\||fi|
\end{tabular}
\end{center}
%
The definition by |\providecommand| makes sure
that previous definitions are not overwritten.
Further statements |\providecommand{\version}{...}|
can thus be added before the above code to override it.

For the main file, one might add a line
(between |\childdocmain| and the above block)
%
\begin{center}
|%\ifchilddoc\||else\providecommand{\version}{draft}\||fi|
\end{center}
%
which can be uncommented to produce a draft version.
Likewise one can add a line to the very top of a child file
(above the |\childdocof{|\textit{main}|}| directive)
%
\begin{center}
|%\providecommand{\version}{final}|
\end{center}
%
which can be uncommented to produce the final version of this child document.

%%%%%%%%%%%%%%%%%%%%%%%%%%%%%%%%%%%%%%%%%%%%%%%%%%%%%%%%%%%%%%%%%%%%%%%%%%%%%%%%
\subsection{Forwarding}
\label{sec:forward}

Different versions of the main or child documents
using compilation flags as described in \secref{sec:flags}
can be (permanently) stored in different files
for convenient compilation, viewing and distribution.
To this end, the package defines a command
to pass on compilation to a different file:

%%%%%%%%%%%%%%%%%%%%%%%%%%%%%%%%%%%%%%%%
\DescribeMacro{\childdocforward}
The command |\childdocforward| redirects processing to
another source file:
%
\begin{center}
\begin{tabular}{l}
|\input{childdoc.def}|\\
|\childdocforward[|\textit{main}|]{|\textit{dest}|}|\\
\end{tabular}
\end{center}
%
The argument \textit{dest} is the destination file
(without extension).
It should be the main file or one of the child files.
Note that further \textsf{childdoc} directives
such as |\childdocof| and |\childdocforward|
in the indicated file will be processed in this form.
The optional argument \textit{main}
passes on directly to the main file \textit{main}
while pretending to compile the child \textit{dest}.
This form behaves as if \textit{dest}
issues |\childdocof{|\textit{main}|}| right away,
and no further \textsf{childdoc} directives will be processed.

%%%%%%%%%%%%%%%%%%%%%%%%%%%%%%%%%%%%%%%%
\DescribeMacro{\...prefix}
In the alternative form |\childdocforwardprefix|,
%
\begin{center}
\begin{tabular}{l}
|\input{childdoc.def}|\\
|\childdocforwardprefix[|\textit{main}|]{|\textit{prefix}|}{|\textit{dest}|}|
\end{tabular}
\end{center}
%
the destination file is determined by a pattern
depending on the current file:
To make this work, the current file must be called
`{\textit{prefix}\hspace{0.2em}\textit{suffix}}'
with \textit{prefix} matching precisely the argument.
Processing is then passed on to the file
`{\textit{dest}\hspace{0.2em}\textit{suffix}}'.
Surely, the same effect is achieved by
directly specifying the
argument `{\textit{dest}\hspace{0.2em}\textit{suffix}}'
in the first form.
However, that requires to set up a different file
for each child. With the alternative form of the command
all these files can have exactly the same content
which simplifies setting them up and maintaining them.

For example, the following file |draft.tex|
with a compilation flag |\version| as described in \secref{sec:flags}
compiles the main document as a draft:
%
\begin{center}
\begin{tabular}{l}
|\def\version{draft}|\\
|\input{childdoc.def}|\\
|\childdocforward{|\textit{main}|}|
\end{tabular}
\end{center}
%
Likewise, the following files |final|\textit{nn}|.tex|
compile the final version of the child document
|child|\textit{nn}|.tex|:
%
\begin{center}
\begin{tabular}{l}
|\def\version{final}|\\
|\input{childdoc.def}|\\
|\childdocforwardprefix{final}{child}|
\end{tabular}
\end{center}
%

Note that when several versions of a main file and/or of each child file
are to be generated, it may be convenient to set up a |Makefile| or
shell script to automatise the process.

%%%%%%%%%%%%%%%%%%%%%%%%%%%%%%%%%%%%%%%%%%%%%%%%%%%%%%%%%%%%%%%%%%%%%%%%%%%%%%%%
\subsection{Command Line Processing}
\label{sec:commandline}

The effect of redirection files can also be achieved by invoking
the \LaTeX{} compiler with a more elaborate command line.
Most conveniently this should be done as part
of a shell script or a |Makefile|.

When using \textsf{childdoc} in the main file, the following
command lines effectively perform a redirection
(note that depending on the shell being used,
backslashes may have to be doubled: `|\|' $\to$ `|\\|'):
%
\begin{center}
|... -jobname "|\textit{target}|" |\\|"|[\textit{flags}]%
|\input{childdoc.def}\childdocforward[|\textit{main}|]{|\textit{dest}|}"|
\end{center}
%
Here \textit{target} is the name of the output file,
\textit{main} is the name of the main file
and \textit{dest} is the name of the main or child file to be processed
(all filenames without extensions).
The optional argument \textit{main} can be omitted
if \textit{main} matches \textit{dest}.
Optionally, compilation \textit{flags} can be defined via |\def| commands.
This command line makes the \TeX{} engine believe
it is compiling the file \textit{target}
whose content is specified as the latter parameter.
The provided code then forwards the processing to
\textit{main} or \textit{dest} as described in \secref{sec:forward}.

%%%%%%%%%%%%%%%%%%%%%%%%%%%%%%%%%%%%%%%%%%%%%%%%%%%%%%%%%%%%%%%%%%%%%%%%%%%%%%%%
\subsection{Include by Input}
\label{sec:input}

Including child documents by |\include| has some restrictions by design.
Most notably, the content of a child document always occupies
its own set of pages; pages cannot be shared between child documents.
Usually, this behaviour makes perfect sense
because each child document contain an essential part of the document.
However, in some situations it may be desirable to compose
a document from a collection of parts
without having mandatory page breaks between then.
For this case, the package
provides a mechanism to include parts
by |\input| which can also be processed individually.
However, by construction this mechanism
requires manual handling of the content to be output.

%%%%%%%%%%%%%%%%%%%%%%%%%%%%%%%%%%%%%%%%
\DescribeMacro{\ifchilddocmanual}
The main file should be prepared as usual, see \secref{sec:include}.
However, the document body must make a distinction
between processing of an individual part and of the main document, e.g.:
%
\begin{center}
\begin{tabular}{l}
|\ifchilddocmanual|\\
|\input{\childdocname}|\\
|\||else|\\
\textit{document body with }|\input{|\textit{part}|}|\\
|\||fi|
\end{tabular}
\end{center}
%
The conditional |\ifchilddocmanual| is true whenever
a part to be included by |\input| is being compiled,
and the name of the part is stored in |\childdocname|.

%%%%%%%%%%%%%%%%%%%%%%%%%%%%%%%%%%%%%%%%
\DescribeMacro{\childdocby}
Each part to be included by |\input| should start with:
%
\begin{center}
\begin{tabular}{l}
|\input{childdoc.def}|\\
|\childdocby{|\textit{main}|}|\\
\end{tabular}
\end{center}
%
The directive |\childdocby| is similar to |\childdocof|
described in \secref{sec:include},
but the subsequent selection of content must be done manually.
To that end, both |\ifchilddoc| and |\ifchilddocmanual|
will be true upon processing of a part,
and the name of the part is stored in |\childdocname|.
Note that |\jobname| will be set to the filename of the current part
so that each part receives an individual |.aux| file
that does not interfere with the |.aux| file(s) of the main document.
This behaviour can be altered by the alternative form
|\childdocby[*]{|\textit{main}|}| (with a non-empty optional argument)
which uses the |.aux| file of the main document
by setting |\jobname| to \textit{main}.

%%%%%%%%%%%%%%%%%%%%%%%%%%%%%%%%%%%%%%%%%%%%%%%%%%%%%%%%%%%%%%%%%%%%%%%%%%%%%%%%
\subsection{Driver Development}
\label{sec:driver}

The \textsf{childdoc} mechanism can also be use for the development
of definition files such as \LaTeX{} styles or classes.
This case differs from the above setup with multiple parts
included by |\include| in that no |\includeonly| should be invoked.
This can be achieved by starting the include file
(before |\ProvidesPackage|) with:
%
\begin{center}
\begin{tabular}{l}
|\input{childdoc.def}|\\
|\childdocforward{|\textit{main}|}|\\
\end{tabular}
\end{center}
%
or alternatively with:
%
\begin{center}
\begin{tabular}{l}
|\input{childdoc.def}|\\
|\childdocby{|\textit{main}|}|\\
\end{tabular}
\end{center}
%
Both forms have slightly different effects as described above.
The main file is prepared as usual, see \secref{sec:include}.

%%%%%%%%%%%%%%%%%%%%%%%%%%%%%%%%%%%%%%%%%%%%%%%%%%%%%%%%%%%%%%%%%%%%%%%%%%%%%%%%
\subsection{Legacy Detection}
\label{sec:detection}

The directive |\childdocmain| in the main file can detect
whether the complete document or merely a child is to be compiled
even without using the directive |\childdocof|.
This method is deprecated because it is less robust
and there is no compelling reason to use it;
it is merely provided for backward compatibility
and it may be removed in future versions.

If the detection mechanism is to be used,
it is mandatory to correctly specify
the filename of the main file as the argument of |\childdocmain|:
%
\begin{center}
\begin{tabular}{l}
|\input{childdoc.def}|\\
|\childdocmain{|\textit{main}|}|\\
\end{tabular}
\end{center}
%
If |\jobname| does not match the argument \textit{main} of |\childdocmain|,
it is assumed that |\jobname| points to the child file to be compiled.
When using |\childdocmain| with the main file specified as argument,
it suffices to start a child file
with just |\input{|\textit{main}|}|
without loading of the package and using |\childdocof|.
If instead all processing is done
with the appropriate \textsf{childdoc} directives,
the argument of \textit{main} of |\childdocmain| can be empty.

An alternative version of the command line processing described
in \secref{sec:commandline} using the detection mechanism reads:
%
\begin{center}
|... -jobname "|\textit{target}|" "|[\textit{flags}]%
[|\def\jobname{|\textit{dest}|}|]|\input{|\textit{main}|}"|
\end{center}

%%%%%%%%%%%%%%%%%%%%%%%%%%%%%%%%%%%%%%%%%%%%%%%%%%%%%%%%%%%%%%%%%%%%%%%%%%%%%%%%
\subsection{Manual Code}
\label{sec:manual}

In case one cannot be certain whether the definitions file |childdoc.def|
is installed on the target \TeX{} distribution
and one prefers not to ship it,
it is conceivable to paste a few relevant commands into the sources.

To that end, drop all statements |\input{childdoc.def}|
and perform the replacements as outlined below.
Instead of |\childdocmain{|\textit{main}|}| add the following code
to the top of the main file:
%
\begin{center}
\begin{tabular}{l}
|\||ifdefined\childdocname\endinput\||fi\newif\ifchilddoc|\\
|\edef\childdocname{\scantokens\expandafter{\jobname\noexpand}}|\\
|\def\childdocmain{|\textit{main}|}\||ifx\childdocmain\childdocname\||else|\\
|\childdoctrue\includeonly{\childdocname}\let\jobname\childdocmain\||fi|\\
\end{tabular}
\end{center}
%
Instead of |\childdocof{|\textit{main}|}| just include the main file
at the top of each child file:
%
\begin{center}
|\input{|\textit{main}|}|
\end{center}
%
A simple redirection |\childdocforward{|\textit{dest}|}| is achieved by:
%
\begin{center}
|\def\jobname{|\textit{dest}|}\input{\jobname}|
\end{center}
%
The redirection with prefix
|\childdocforwardprefix[|\textit{prefix}|]{|\textit{dest}|}|
is accomplished by:
%
\begin{center}
\begin{tabular}{l}
|{\edef\jobname{\scantokens\expandafter{\jobname\noexpand}}|\\
|\def\redirectjob |\textit{prefix}|#1~~~{\gdef\jobname{|\textit{dest}|#1}}|\\
|\expandafter\redirectjob\jobname~~~}\input{\jobname}|
\end{tabular}
\end{center}

In an alternative approach,
child documents can be compiled by a specific command line
without additional code or specific definitions:
%
\begin{center}
|... -jobname "|\textit{target}|" "|[\textit{flags}]%
|\includeonly{|\textit{dest}|}\input{|\textit{main}|}"|
\end{center}
%

%%%%%%%%%%%%%%%%%%%%%%%%%%%%%%%%%%%%%%%%%%%%%%%%%%%%%%%%%%%%%%%%%%%%%%%%%%%%%%%%
%%%%%%%%%%%%%%%%%%%%%%%%%%%%%%%%%%%%%%%%%%%%%%%%%%%%%%%%%%%%%%%%%%%%%%%%%%%%%%%%
\section{Information}

%%%%%%%%%%%%%%%%%%%%%%%%%%%%%%%%%%%%%%%%%%%%%%%%%%%%%%%%%%%%%%%%%%%%%%%%%%%%%%%%
\subsection{Copyright}

Copyright \copyright{} 2017--2018 Niklas Beisert

This work may be distributed and/or modified under the
conditions of the \LaTeX{} Project Public License, either version 1.3
of this license or (at your option) any later version.
The latest version of this license is in
  \url{http://www.latex-project.org/lppl.txt}
and version 1.3 or later is part of all distributions of \LaTeX{}
version 2005/12/01 or later.

This work has the LPPL maintenance status `maintained'.

The Current Maintainer of this work is Niklas Beisert.

This work consists of the files |README.txt|, |childdoc.ins| and |childdoc.dtx|
as well as the derived files |childdoc.def|, |cdocsamp.tex|
with |cdocsch1.tex|, |cdocsch2.tex|, |cdocspt3.tex|, |cdocspt4.tex|,
|cdocsdrf.tex|, |cdocsfn1.tex|, |cdocsfn2.tex|
as well as |childdoc.pdf|.

%%%%%%%%%%%%%%%%%%%%%%%%%%%%%%%%%%%%%%%%%%%%%%%%%%%%%%%%%%%%%%%%%%%%%%%%%%%%%%%%
\subsection{Files and Installation}

The package consists of the files:
%
\begin{center}
\begin{tabular}{ll}
    |README.txt|   & readme file \\
    |childdoc.ins| & installation file \\
    |childdoc.dtx| & source file \\
    |childdoc.def| & definition file \\
    |cdocsamp.tex| & sample main file \\
    |cdocsch1.tex| & sample include file \\
    |cdocsch2.tex| & sample include file \\
    |cdocspt3.tex| & sample part file \\
    |cdocspt4.tex| & sample part file \\
    |cdocsdrf.tex| & sample redirection file \\
    |cdocsfn1.tex| & sample redirection file \\
    |cdocsfn2.tex| & sample redirection file \\
    |childdoc.pdf| & manual
\end{tabular}
\end{center}
%
The distribution consists of the files
|README.txt|, |childdoc.ins| and |childdoc.dtx|.
%
\begin{itemize}
\item
Run (pdf)\LaTeX{} on |childdoc.dtx|
to compile the manual |childdoc.pdf| (this file).
\item
Run \LaTeX{} on |childdoc.ins| to create the definitions file |childdoc.def|
and the sample |cdocsamp.tex| with include files
|cdocsch1.tex|, |cdocsch2.tex|, |cdocspt3.tex|, |cdocspt4.tex|,
|cdocsdrf.tex|, |cdocsfn1.tex|, |cdocsfn2.tex|.
Then copy the file |childdoc.def| to an appropriate directory of your \LaTeX{}
distribution, e.g.\ \textit{texmf-root}|/tex/latex/childdoc|.
\end{itemize}

%%%%%%%%%%%%%%%%%%%%%%%%%%%%%%%%%%%%%%%%%%%%%%%%%%%%%%%%%%%%%%%%%%%%%%%%%%%%%%%%
\subsection{Related CTAN Packages}

There are several other packages which offer a similar functionality:
%
\begin{itemize}
\item
The packages
\href{http://ctan.org/pkg/docmute}{\textsf{docmute}},
\href{http://ctan.org/pkg/includex}{\textsf{includex}} and
\href{http://ctan.org/pkg/standalone}{\textsf{standalone}}
provide commands to include only the document body of
a child file thus allowing both files to be compiled individually.
\item
The packages \href{http://ctan.org/pkg/subdocs}{\textsf{subdocs}}
and \href{http://ctan.org/pkg/subfiles}{\textsf{subfiles}}
provide structures in which the main and child documents can be
encapsulated and allowing them to be compiled individually.
The inclusion mechanism is different from the conventional |\include|.
\item
The package \href{http://ctan.org/pkg/combine}{\textsf{combine}}
is an elaborate solution to combine several documents into one.
\end{itemize}
%
See also the CTAN topic \href{http://ctan.org/topic/subdocs}{\textsf{subdocs}}
for further related packages.
The present package differs from the above solutions in that
a document structure constructed with the conventional |\include| mechanism
just needs two extra commands at the top of every file
such that all constituent files can be compiled individually.

%%%%%%%%%%%%%%%%%%%%%%%%%%%%%%%%%%%%%%%%%%%%%%%%%%%%%%%%%%%%%%%%%%%%%%%%%%%%%%%%
%\subsection{Feature Suggestions}
%
%The following is a list of features which may be useful for future
%versions of this package:
%%
%\begin{itemize}
%\item
%\ldots
%\end{itemize}

%%%%%%%%%%%%%%%%%%%%%%%%%%%%%%%%%%%%%%%%%%%%%%%%%%%%%%%%%%%%%%%%%%%%%%%%%%%%%%%%
\subsection{Revision History}

%%%%%%%%%%%%%%%%%%%%%%%%%%%%%%%%%%%%%%%%
\paragraph{v2.0:} 2018/12/30

\begin{itemize}
\item
immediate forward processing
\item
added |\childdocby| mechanism
\item
manual restructured
\end{itemize}

%%%%%%%%%%%%%%%%%%%%%%%%%%%%%%%%%%%%%%%%
\paragraph{v1.6:} 2018/01/17

\begin{itemize}
\item
application for development of include files
\item
corrections to manual
\end{itemize}

%%%%%%%%%%%%%%%%%%%%%%%%%%%%%%%%%%%%%%%%
\paragraph{v1.5:} 2017/05/21

\begin{itemize}
\item
more complete structuring introduced
\item
|\childdocof| introduced
\item
|\childdoc| renamed to |\childdocmain|
\item
|\childredirect| renamed to |\childdocforward| and |\childdocforwardprefix|
and functionality expanded
\end{itemize}

%%%%%%%%%%%%%%%%%%%%%%%%%%%%%%%%%%%%%%%%
\paragraph{v1.0:} 2017/04/27

\begin{itemize}
\item
manual and install package
\item
first version published on CTAN
\end{itemize}

%%%%%%%%%%%%%%%%%%%%%%%%%%%%%%%%%%%%%%%%
\paragraph{v0.6:} 2017/04/26

\begin{itemize}
\item
redirection mechanism added
\end{itemize}

%%%%%%%%%%%%%%%%%%%%%%%%%%%%%%%%%%%%%%%%
\paragraph{v0.5:} 2017/04/26

\begin{itemize}
\item
functionality in definition file
\end{itemize}


%%%%%%%%%%%%%%%%%%%%%%%%%%%%%%%%%%%%%%%%%%%%%%%%%%%%%%%%%%%%%%%%%%%%%%%%%%%%%%%%
%%%%%%%%%%%%%%%%%%%%%%%%%%%%%%%%%%%%%%%%%%%%%%%%%%%%%%%%%%%%%%%%%%%%%%%%%%%%%%%%
%%%%%%%%%%%%%%%%%%%%%%%%%%%%%%%%%%%%%%%%%%%%%%%%%%%%%%%%%%%%%%%%%%%%%%%%%%%%%%%%
\appendix

\settowidth\MacroIndent{\rmfamily\scriptsize 000\ }

 \DocInput{childdoc.dtx}

\end{document}
%</driver>
% \fi
%
% %%%%%%%%%%%%%%%%%%%%%%%%%%%%%%%%%%%%%%%%%%%%%%%%%%%%%%%%%%%%%%%%%%%%%%%%%%%%%%
% %%%%%%%%%%%%%%%%%%%%%%%%%%%%%%%%%%%%%%%%%%%%%%%%%%%%%%%%%%%%%%%%%%%%%%%%%%%%%%
% \section{Sample}
%\iffalse
%<*samplemain>
%\fi
%
% The following presents a sample document
% with two chapters, two parts, a title page,
% a compile flag as well as three forwarding files to set the flag.
% It consists of eight |.tex| files:
% \begin{center}
% \begin{tabular}{ll}
% |cdocsamp.tex|&main file\\
% |cdocsch1.tex|&include file for chapter 1\\
% |cdocsch2.tex|&include file for chapter 2\\
% |cdocspt3.tex|&include file for part 3\\
% |cdocspt4.tex|&include file for part 4\\
% |cdocsdrf.tex|&forwarding file for main file in draft mode\\
% |cdocsfi1.tex|&forwarding file for final version of chapter 1\\
% |cdocsfi2.tex|&forwarding file for final version of chapter 2\\
% \end{tabular}
% \end{center}
% Each of the eight files can be compiled directly by the \LaTeX{} compiler.
%
% %%%%%%%%%%%%%%%%%%%%%%%%%%%%%%%%%%%%%%
% \paragraph{Main File.}
%
% The main file is called |cdocsamp.tex|.
%
% Load the \textsf{childdoc} definitions and
% declare the filename for the main document:
%    \begin{macrocode}
\input{childdoc.def}
\childdocmain{}
%    \end{macrocode}

% Optional override for |\version| flag:
%    \begin{macrocode}
%%\ifchilddoc\else\providecommand{\version}{draft}\fi
%    \end{macrocode}

% Define the default values for the |\version| flag
% (|final| for the main file and |draft| for childs):
%    \begin{macrocode}
\ifchilddoc
\providecommand{\version}{draft}
\else
\providecommand{\version}{final}
\fi
%    \end{macrocode}

% Load the standard document class:
%    \begin{macrocode}
\documentclass[12pt]{article}
%    \end{macrocode}

% Start the document body:
%    \begin{macrocode}
\begin{document}
%    \end{macrocode}

% Declare a title page.
% Print title, part of document being processed and version flag:
%    \begin{macrocode}
\addtocounter{page}{-1}
\begin{center}
{\LARGE\bfseries{}childdoc example\par}
\vspace{1cm}
\ifchilddoc
\ifchilddocmanual part\else chapter\fi:
`\childdocname' of `\childdocjob'\par
\else
main document: `\childdocjob'\par
\fi
version: \version\par
\end{center}
\newpage
%    \end{macrocode}

% Manually include selected file,
% otherwise process as usual:
%    \begin{macrocode}
\ifchilddocmanual
\section*{part `\childdocname'}
\input{\childdocname}
\else
%    \end{macrocode}

% Include the two chapters:
%    \begin{macrocode}
\include{cdocsch1}
\include{cdocsch2}
%    \end{macrocode}

% Include the two parts unless only chapters should be displayed:
%    \begin{macrocode}
\ifchilddoc\else
\section{part three}
\input{cdocspt3}
\section{part four}
\input{cdocspt4}
\fi
%    \end{macrocode}

% Process as usual until here:
%    \begin{macrocode}
\fi
%    \end{macrocode}

% End of document body:
%    \begin{macrocode}
\end{document}
%    \end{macrocode}
%\iffalse
%</samplemain>
%\fi
%
% %%%%%%%%%%%%%%%%%%%%%%%%%%%%%%%%%%%%%%
% \paragraph{Chapter Include Files.}
%
% The include files are called |cdocsch1.tex| and |cdocsch2.tex|.
%
%\iffalse
%<*samplechap1|samplechap2>
%\fi

% Optional override for |\version| flag:
%    \begin{macrocode}
%%\providecommand{\version}{final}
%    \end{macrocode}

% Include the main document:
%    \begin{macrocode}
\input{childdoc.def}
\childdocof{cdocsamp}
%    \end{macrocode}

%\iffalse
%</samplechap1|samplechap2>
%\fi
%
%\iffalse
%<*samplechap1>
%\fi
% Some text for chapter 1:
%    \begin{macrocode}
\section{one}
some text in chapter one
%    \end{macrocode}

%\iffalse
%</samplechap1>
%\fi
% Some text for chapter 2:
%\iffalse
%<*samplechap2>
%\fi
%    \begin{macrocode}
\section{two}
more text in chapter two
%    \end{macrocode}

%\iffalse
%</samplechap2>
%\fi
%
% %%%%%%%%%%%%%%%%%%%%%%%%%%%%%%%%%%%%%%
% \paragraph{Part Include Files.}
%
% The include files are called |cdocspt3.tex| and |cdocspt4.tex|.
%
%\iffalse
%<*samplepart3|samplepart4>
%\fi

% Optional override for |\version| flag:
%    \begin{macrocode}
%%\providecommand{\version}{final}
%    \end{macrocode}

% Include the main document:
%    \begin{macrocode}
\input{childdoc.def}
\childdocby{cdocsamp}
%    \end{macrocode}

%\iffalse
%</samplepart3|samplepart4>
%\fi
%
%\iffalse
%<*samplepart3>
%\fi
% Some text for part 3:
%    \begin{macrocode}
some text in part three
%    \end{macrocode}

%\iffalse
%</samplepart3>
%\fi
% Some text for part 4:
%\iffalse
%<*samplepart4>
%\fi
%    \begin{macrocode}
more text in part four
%    \end{macrocode}

%\iffalse
%</samplepart4>
%\fi
%
% %%%%%%%%%%%%%%%%%%%%%%%%%%%%%%%%%%%%%%
% \paragraph{Forwarding for a Complete Draft.}
%
% The following forwarding file |cdocsdrf.tex|
% compiles the main document in draft mode:
%\iffalse
%<*sampledraft>
%\fi
%    \begin{macrocode}
\def\version{draft}
\input{childdoc.def}
\childdocforward{cdocsamp}
%    \end{macrocode}

%\iffalse
%</sampledraft>
%\fi
%
% %%%%%%%%%%%%%%%%%%%%%%%%%%%%%%%%%%%%%%
% \paragraph{Forwarding for Final Version of the Chapters.}
%
% The following forwarding files |cdocsfn1.tex| and |cdocsfn2.tex|
% (with identical content)
% compile the final versions of the child documents
% |cdocsch1.tex| and |cdocsch2.tex|, respectively:
%\iffalse
%<*samplefinal>
%\fi
%    \begin{macrocode}
\def\version{final}
\input{childdoc.def}
\childdocforwardprefix[cdocsamp]{cdocsfn}{cdocsch}
%    \end{macrocode}

%\iffalse
%</samplefinal>
%\fi
%
% %%%%%%%%%%%%%%%%%%%%%%%%%%%%%%%%%%%%%%
% \paragraph{Command Line Processing.}
%
% The following three command lines generate the output files
% |cdocscld|, |cdocscl1| and |cdocscl2|
% which should be identical to
% |cdocsdrf|, |cdocsch1| and |cdocsfn2|, respectively:
% \begin{center}
% \begin{tabular}{l}
% |latex -jobname cdocscld \|\\
% |  "\def\version{draft}\input{childdoc.def}\childdocforward{cdocsamp}"|\\
% |latex -jobname cdocscl1 \|\\
% |  "\input{childdoc.def}\childdocforward[cdocsamp]{cdocsch1}"|\\
% |latex -jobname cdocscl2 \|\\
% |  "\def\version{final}\input{childdoc.def}\childdocforward{cdocsch2}"|
% \end{tabular}
% \end{center}
% Note that the trailing backslash on each first line
% merely continues the input to the second line
% (for convenient cut ant paste).
% Furthermore, the command |latex| can be replaced by any
% of its alternative versions such as |pdflatex|.
%
% %%%%%%%%%%%%%%%%%%%%%%%%%%%%%%%%%%%%%%%%%%%%%%%%%%%%%%%%%%%%%%%%%%%%%%%%%%%%%%
% %%%%%%%%%%%%%%%%%%%%%%%%%%%%%%%%%%%%%%%%%%%%%%%%%%%%%%%%%%%%%%%%%%%%%%%%%%%%%%
% \section{Implementation}
%\iffalse
%<*package>
%\fi
%
% This section describes the definitions file |childdoc.def|.

% The definitions cannot be loaded using |\usepackage| or |\RequirePackage|
% which has a mechanism to prevent loading a style file more than once.
% When loading the definitions by means of |\input|
% multiple instances have to be prevented manually:
%\iffalse
%This code needs to be before the `\ProvidesFile' directive
%which is defined at the beginning of this file.
%Therefore it is also placed there and commented out here.
%</package>
%<*discard>
%\fi
%    \begin{macrocode}
\ifdefined\childdocmain\endinput\fi
%    \end{macrocode}
%\iffalse
%</discard>
%<*package>
%\fi
%
% \macro{\ifchilddoc}
% \macro{\ifchilddocmanual}
% The conditional |\ifchilddoc| tells whether a
% child (true) or main (false) document is being compiled.
% The conditional |\ifchilddocmanual| tells whether
% the |\includeonly| mechanism is used (false) or
% the selection of child files must be performed manually (true).
% The definitions initialise to false:
%    \begin{macrocode}
\newif\ifchilddoc
\newif\ifchilddocmanual
%    \end{macrocode}

% \macro{\childdocname}
% \macro{\childdocjob}
% The macro |\childdocname| stores the name of the main document
% to be compiled. The macro |\childdocjob| stores the name of
% the document on which the \LaTeX{} compiler was originally invoked.
% The content of |\jobname| cannot be compared
% to filenames specified in the source due to different catcodes.
% The following code rescans |\jobname|, stores the result
% in |\childdocname| and saves a copy in |\childdocjob|:
%    \begin{macrocode}
\edef\childdocname{\scantokens\expandafter{\jobname\noexpand}}
\let\childdocjob\childdocname
%    \end{macrocode}

% \macro{\childdocdisable}
% The macro |\childdocdisable| prevents the main file
% from being processed more than once.
% At this stage, the main document command |\childdocmain|
% is assumed to be called once again where it should do nothing.
% Any subsequent call to it should prevent
% a secondary processing of the main document
% It overwrites the forwarding commands
% |\childdocof| and |\childdocforward|
% with empty macros to prevent further inclusions of the main document:
%    \begin{macrocode}
\newcommand{\childdocdisable}
{
  \renewcommand{\childdocmain}[1]{\renewcommand{\childdocmain}[1]{\endinput}}
  \renewcommand{\childdocof}[1]{}
  \renewcommand{\childdocby}[2][]{}
  \renewcommand{\childdocforward}[2][]{}
  \renewcommand{\childdocdisable}{}
}
%    \end{macrocode}

% \macro{\childdocmain}
% The macro |\childdocmain| is to be called at the top of the main file
% with nothing or the main filename (without extension) as argument.
% First, it breaks loops.
% If the argument is not empty and does not match |\childdocname|
% (which is set by the first inclusion of |childdoc.def|),
% |\ifchilddoc| is set to true, |\includeonly| is applied to the child file
% and |\jobname| is set to the main file
% (for proper handling of |.aux| files):
%    \begin{macrocode}
\newcommand{\childdocmain}[1]
{
  \childdocdisable\childdocmain{}
  \if?#1?\else
    \begingroup
      \def\childdoctmp{#1}
      \ifx\childdoctmp\childdocname
        \def\childdoctmp{}
      \else
        \def\childdoctmp
        {
          \childdoctrue
          \includeonly{\childdocname}
          \def\childdocjob{#1}
          \def\jobname{#1}
        }
      \fi
      \expandafter
    \endgroup
    \childdoctmp
  \fi
}
%    \end{macrocode}

% \macro{\childdocof}
% The command |\childdocof| redirects
% compilation to the main file |#1|.
%    \begin{macrocode}
\newcommand{\childdocof}[1]
{
  \childdocdisable
  \childdoctrue
  \includeonly{\childdocname}
  \def\jobname{#1}
  \def\childdocjob{#1}
  \input{#1}
}
%    \end{macrocode}

% \macro{\childdocby}
% The command |\childdocby| ....
%    \begin{macrocode}
\newcommand{\childdocby}[2][]
{
  \childdocdisable
  \childdoctrue
  \childdocmanualtrue
  \if?#1?\else
    \def\jobname{#2}
  \fi
  \def\childdocjob{#2}
  \input{#2}
  \endinput
}
%    \end{macrocode}

% \macro{\childdocforward}
% The command |\childdocforward| redirects
% compilation to the main file or
% (if the optional argument is given) a child file.
% Parameters are set as if the main file
% or a child file starting with |\childdocof| was compiled.
% Then compilation is handed over to the main file:
%    \begin{macrocode}
\newcommand{\childdocforward}[2][]
{
  \begingroup
    \if?#1?
      \def\childdoctmp
      {
        \def\childdocname{#2}
        \def\childdocjob{#2}
        \def\jobname{#2}
        \input{#2}
        \endinput
      }
    \else
      \def\childdoctmp
      {
        \childdocdisable
        \def\childdocname{#2}
        \childdoctrue
        \includeonly{#2}
        \def\childdocjob{#1}
        \def\jobname{#1}
        \input{#1}
        \endinput
      }
    \fi
    \expandafter
  \endgroup
  \childdoctmp
}
%    \end{macrocode}

% \macro{\childdocforwardprefix}
% The command |\childdocforwardprefix| redirects
% compilation to the main or a child file by means of a pattern.
% The prefix |#1| in the current filename is replaced by |#2|
% and the suffix of the current filename is kept
% (it is assumed that the filename does not contain the substring `|~~~|'
% which is used as a delimiter).
% Compilation is handed over to the new file by |\childdocforward|:
%    \begin{macrocode}
\newcommand{\childdocforwardprefix}[3][]
{
  \begingroup
    \def\childdocextract #2##1~~~{\def\childdoctmp{\childdocforward[#1]{#3##1}}}
    \expandafter\childdocextract\childdocname~~~
    \expandafter
  \endgroup
  \childdoctmp
}
%    \end{macrocode}

% \macro{\childdoc}
% The deprecated macro |\childdoc| is a legacy version of |\childdocmain|:
%    \begin{macrocode}
\newcommand{\childdoc}{\childdocmain}
%    \end{macrocode}

% \macro{\childdocredirect}
% The deprecated macro |\childdocredirect| is a legacy version
% of |\childdocforward| and |\childdocforwardprefix|:
%    \begin{macrocode}
\newcommand{\childdocredirect}[2][]
{
  \begingroup
    \if?#1?
      \def\childdoctmp{\childdocforward{#2}}
    \else
      \def\childdoctmp{\childdocforwardprefix{#1}{#2}}
    \fi
    \expandafter
  \endgroup
  \childdoctmp
}
%    \end{macrocode}

%\iffalse
%</package>
%\fi
%
\endinput
|\\
|\childdocforward{|\textit{main}|}|
\end{tabular}
\end{center}
%
Likewise, the following files |final|\textit{nn}|.tex|
compile the final version of the child document
|child|\textit{nn}|.tex|:
%
\begin{center}
\begin{tabular}{l}
|\def\version{final}|\\
|% \iffalse
%
% childdoc.dtx Copyright (C) 2017-2018 Niklas Beisert
%
% This work may be distributed and/or modified under the
% conditions of the LaTeX Project Public License, either version 1.3
% of this license or (at your option) any later version.
% The latest version of this license is in
%   http://www.latex-project.org/lppl.txt
% and version 1.3 or later is part of all distributions of LaTeX
% version 2005/12/01 or later.
%
% This work has the LPPL maintenance status `maintained'.
%
% The Current Maintainer of this work is Niklas Beisert.
%
% This work consists of the files childdoc.dtx and childdoc.ins
% and the derived files childdoc.def and cdocsamp.tex with
% cdocsch1.tex, cdocsch2.tex, cdocsdrf.tex, cdocsfn1.tex, cdocsfn2.tex.
%
%<package>\ifdefined\childdocmain\endinput\fi
%<package>\ProvidesFile{childdoc.def}[2018/12/30 v2.0 child document driver]
%<samplemain>\ProvidesFile{cdocsamp.tex}[2018/12/30 v2.0 sample for childdoc]
%<*driver>
%\ProvidesFile{childdoc.drv}[2018/12/30 v2.0 childdoc reference manual file]
\PassOptionsToClass{10pt,a4paper}{article}
\documentclass{ltxdoc}

\usepackage[margin=35mm]{geometry}
\usepackage{hyperref}
\usepackage{hyperxmp}
\usepackage[usenames]{color}

\hypersetup{colorlinks=true}
\hypersetup{pdfstartview=FitH}
\hypersetup{pdfpagemode=UseNone}
\hypersetup{pdfsource={}}
\hypersetup{pdflang={en-UK}}
\hypersetup{pdfcopyright={Copyright 2017-2018 Niklas Beisert.
  This work may be distributed and/or modified under the
  conditions of the LaTeX Project Public License, either version 1.3
  of this license or (at your option) any later version.}}
\hypersetup{pdflicenseurl={http://www.latex-project.org/lppl.txt}}
\hypersetup{pdfcontactaddress={ETH Zurich, ITP, HIT K,
  Wolfgang-Pauli-Strasse 27}}
\hypersetup{pdfcontactpostcode={8093}}
\hypersetup{pdfcontactcity={Zurich}}
\hypersetup{pdfcontactcountry={Switzerland}}
\hypersetup{pdfcontactemail={nbeisert@itp.phys.ethz.ch}}
\hypersetup{pdfcontacturl={http://people.phys.ethz.ch/\xmptilde nbeisert/}}

\newcommand{\secref}[1]{\hyperref[#1]{section \ref*{#1}}}

\parskip1ex
\parindent0pt
\let\olditemize\itemize
\def\itemize{\olditemize\parskip0pt}

\begin{document}

\title{The \textsf{childdoc} Package}
\hypersetup{pdftitle={The childdoc Package}}
\author{Niklas Beisert\\[2ex]
  Institut f\"ur Theoretische Physik\\
  Eidgen\"ossische Technische Hochschule Z\"urich\\
  Wolfgang-Pauli-Strasse 27, 8093 Z\"urich, Switzerland\\[1ex]
  \href{mailto:nbeisert@itp.phys.ethz.ch}
  {\texttt{nbeisert@itp.phys.ethz.ch}}}
\hypersetup{pdfauthor={Niklas Beisert}}
\hypersetup{pdfsubject={Manual for the LaTeX2e Package childdoc}}
\date{30 December 2018, \textsf{v2.0}}
\maketitle

\begin{abstract}\noindent
\textsf{childdoc} is a \LaTeXe{} package
that enables the direct compilation
of document sections included by |\include|
to individual files.
\end{abstract}

\begingroup
\parskip0ex
\tableofcontents
\endgroup

%%%%%%%%%%%%%%%%%%%%%%%%%%%%%%%%%%%%%%%%%%%%%%%%%%%%%%%%%%%%%%%%%%%%%%%%%%%%%%%%
%%%%%%%%%%%%%%%%%%%%%%%%%%%%%%%%%%%%%%%%%%%%%%%%%%%%%%%%%%%%%%%%%%%%%%%%%%%%%%%%
\section{Introduction}

\LaTeX{} provides a mechanism to structure a large document (such as a book)
into a main file and several child files (containing the chapters)
using the |\include| command.
This mechanism is beneficial for documents
which span hundreds of pages in order to
make the source file(s) more manageable.
Moreover, compilation can be restricted to
selected child files by means of the |\includeonly| command.
The latter feature can be used to reduce the compilation time while editing
(this was significantly more useful in the earlier days of \LaTeX{})
or to generate a smaller document which is easier to navigate.
Another application of |\includeonly| is to generate
documents consisting of selected parts of the complete document.

However, there are a few drawbacks of the plain |\include| mechanism:
\begin{itemize}
\item
The child files cannot be compiled on their own,
they can only be compiled via the main file.
A naive editing environment
(such as a text editor with an option
to have the current file processed by \LaTeX)
may require one to switch to the main file before compiling;
attempting to compile the child file produces errors.
\item
The main file must be modified (each time)
to adjust the |\includeonly| command
to the present needs. This easily leaves the main file in a messy state.
\item
The generated document will always carry the filename
of the main document. This is inconvenient if
several child files are to be compiled and
to be kept for distribution.
\end{itemize}

The present package provides a simple interface
to make child files individually compilable by \LaTeX{}.
Compiling a child file then has the same effect as compiling
the main file with an |\includeonly| command
to select the appropriate child.
Moreover the generated document will carry the name of the child
rather than the main file.
This resolves all three above issues.

This feature is meant to make the editing of books,
thesis documents and lecture notes somewhat more convenient.
However, the package can also be used efficiently for
composing a series of documents (such as exercise sheets)
which are typically distributed individually.
It then assists the author in generating the individual documents
(potentially in different versions)
as well as a document containing the collected series.
Another application is in developing style files
or other kinds of included material
where compilation of the style file could redirect
to a sample or test file.

%%%%%%%%%%%%%%%%%%%%%%%%%%%%%%%%%%%%%%%%%%%%%%%%%%%%%%%%%%%%%%%%%%%%%%%%%%%%%%%%
%%%%%%%%%%%%%%%%%%%%%%%%%%%%%%%%%%%%%%%%%%%%%%%%%%%%%%%%%%%%%%%%%%%%%%%%%%%%%%%%
\section{Usage}

First of all, the package \textsf{childdoc} is \emph{not} a standard
\LaTeXe{} |.sty| style file! Therefore it needs to be invoked in
a non-standard way.

%%%%%%%%%%%%%%%%%%%%%%%%%%%%%%%%%%%%%%%%%%%%%%%%%%%%%%%%%%%%%%%%%%%%%%%%%%%%%%%%
\subsection{Included Files}
\label{sec:include}

%%%%%%%%%%%%%%%%%%%%%%%%%%%%%%%%%%%%%%%%
\DescribeMacro{\childdocmain}
To use the package, add the commands
\begin{center}
\begin{tabular}{l}
|\input{childdoc.def}|\\
|\childdocmain{}|\\
\end{tabular}
\end{center}
at the very top of the main \LaTeX{} file,
in particular \emph{before} the |\documentclass| statement!
The argument of |\childdocmain| should be left empty
(but it must be present).

%%%%%%%%%%%%%%%%%%%%%%%%%%%%%%%%%%%%%%%%
\DescribeMacro{\childdocof}
Furthermore, add the commands
\begin{center}
\begin{tabular}{l}
|\input{childdoc.def}|\\
|\childdocof{|\textit{main}|}|\\
\end{tabular}
\end{center}
at the top of every child file \textit{child}
which is included by |\include{|\textit{child}|}|
from within the main file
(or at least for those files to be compiled individually).
The argument \textit{main} must be the filename of the main file.

There are a couple of
considerations in setting up the main and child documents:

%%%%%%%%%%%%%%%%%%%%%%%%%%%%%%%%%%%%%%%%
\paragraph{Restrictions.}

Please note the following restrictions:
\begin{itemize}
\item
|\childdocmain| must be called with one argument \textit{main}
to ensure compatibility with earlier version of the package.
It must either be empty (|\childdocmain{}|)
or precisely match the filename of the main file in which it is specified.
See \secref{sec:detection} for further information.
\item
The filename \textit{main} must be specified without the |.tex| extension.
\item
The filename \textit{main} is case sensitive
(even in case-insensitive file systems)
due to internal string comparison.
\item
The argument \textit{main} should be fully expanded, it cannot be a macro.
\item
Subdirectories and special characters should be avoided in filenames.
\item
The command |\childdocmain{|\textit{main}|}| must be followed by a whitespace.
It should not be followed immediately by another command
or by a comment mark `|%|'.
This is because the \TeX{} parser reads the token immediately following
the argument of |\childdocmain| and puts it
at the beginning of every child section;
however, a white\-space is ignored.
\end{itemize}

%%%%%%%%%%%%%%%%%%%%%%%%%%%%%%%%%%%%%%%%
\paragraph{Content of Main File.}

It is advisable to place all content in the child files included by |\include|.
Any output contained in the main file will appear in all child documents
unless suppressed manually;
it cannot be suppressed automatically by the |\includeonly| directive
and thus should normally be avoided.
A method to include some content in the main file
by means of conditional processing is described in \secref{sec:conditional}.

%%%%%%%%%%%%%%%%%%%%%%%%%%%%%%%%%%%%%%%%
\paragraph{Page Numbering.}

When only a part of the document is compiled,
the appropriate numbering of pages
(as well as other status parameters)
is determined from the |.aux| files.
The latter contain information from previous passes.
However this information needs to propagate through
all intermediate child documents.
Therefore the page numbering in child documents may well
be inconsistent until the complete document is compiled at least once.

A useful (if unconventional) way to always ensure a consistent
page numbering is to restart the numbering in each child document
and denote the pages by `\textit{child}|.|\textit{page}'
where \textit{child} represents the chapter/section number of the child file.
This can be achieved by the command
|\numberwithin{page}{|\textit{child}|}|
of the \textsf{amsmath} package
where \textit{child} can be |chapter| or |section|
depending on the chosen structuring.
Alternatively, one can modify the macro |\thepage| appropriately
and reset the counter |page| at the start of each child file.

%%%%%%%%%%%%%%%%%%%%%%%%%%%%%%%%%%%%%%%%%%%%%%%%%%%%%%%%%%%%%%%%%%%%%%%%%%%%%%%%
\subsection{Conditional Processing}
\label{sec:conditional}

The package provides a mechanism to compile different versions
of a document. To customise the versions further some conditional processing
can come in handy to distinguish which version is being compiled.
The package provides two macros to describe the compilation context:

%%%%%%%%%%%%%%%%%%%%%%%%%%%%%%%%%%%%%%%%
\DescribeMacro{\ifchilddoc}
The conditional |\ifchilddoc| distinguishes between the compilation of
child documents and the main document:
%
\begin{center}
|\ifchilddoc |\textit{child-code}| |[|\||else |\textit{main-code}]| \||fi|
\end{center}

%%%%%%%%%%%%%%%%%%%%%%%%%%%%%%%%%%%%%%%%
\DescribeMacro{\childdocname}
\DescribeMacro{\childdocjob}
The macro |\childdocname| contains the filename (without extension)
of the main or child file being processed.
Note that |\childdocjob| will always contain the name of the main file.

%%%%%%%%%%%%%%%%%%%%%%%%%%%%%%%%%%%%%%%%
\paragraph{Title Page.}

Conditional processing can be used to include a title or banner page
in the main document when proper precautions are taken.
Importantly, the code in the main file should ensure that the page counter
(as well as other status parameters which are stored in the |.aux| files)
takes the same value after the conditional processing.
Otherwise the page numbers may take divergent values
depending on which part is compiled.

For example, a title page could be declared by:
%
\begin{center}
\begin{tabular}{l}
|\ifchilddoc\||else|\\
|\addtocounter{page}{-1}|\\
\textit{code for title page}\\
|\newpage|\\
|\||fi|
\end{tabular}
\end{center}
%
A banner page for the child documents can be generated by:
%
\begin{center}
\begin{tabular}{l}
|\ifchilddoc|\\
|\addtocounter{page}{-1}|\\
\textit{code for banner page}\\
|\newpage|\\
|\||fi|
\end{tabular}
\end{center}
%
Here one could write a message such as:
\begin{center}
|This is the part \childdocname{} of \childdocjob{}.|
\end{center}

%%%%%%%%%%%%%%%%%%%%%%%%%%%%%%%%%%%%%%%%%%%%%%%%%%%%%%%%%%%%%%%%%%%%%%%%%%%%%%%%
\subsection{Flags}
\label{sec:flags}

The package makes it easy to generate different versions
of the main or child documents.
To this end compilation flags can be defined
and assigned different default values.
They will be particularly useful in conjunction
with the forwarding mechanism described in \secref{sec:forward}.

For example, it may be useful to have a flag |\version|
which can be set to |draft| or |final|.
The document source will contain some conditional code
depending on the value of |\version|.
Suppose further, the flag should default to |final| for the main file
and to |draft| for child files
which is a natural assignment for editing the document.
This is achieved by placing the following code
in the preamble of the main document
(below the |\childdocmain| directive):
%
\begin{center}
\begin{tabular}{l}
|\ifchilddoc|\\
|\providecommand{\version}{draft}|\\
|\||else|\\
|\providecommand{\version}{final}|\\
|\||fi|
\end{tabular}
\end{center}
%
The definition by |\providecommand| makes sure
that previous definitions are not overwritten.
Further statements |\providecommand{\version}{...}|
can thus be added before the above code to override it.

For the main file, one might add a line
(between |\childdocmain| and the above block)
%
\begin{center}
|%\ifchilddoc\||else\providecommand{\version}{draft}\||fi|
\end{center}
%
which can be uncommented to produce a draft version.
Likewise one can add a line to the very top of a child file
(above the |\childdocof{|\textit{main}|}| directive)
%
\begin{center}
|%\providecommand{\version}{final}|
\end{center}
%
which can be uncommented to produce the final version of this child document.

%%%%%%%%%%%%%%%%%%%%%%%%%%%%%%%%%%%%%%%%%%%%%%%%%%%%%%%%%%%%%%%%%%%%%%%%%%%%%%%%
\subsection{Forwarding}
\label{sec:forward}

Different versions of the main or child documents
using compilation flags as described in \secref{sec:flags}
can be (permanently) stored in different files
for convenient compilation, viewing and distribution.
To this end, the package defines a command
to pass on compilation to a different file:

%%%%%%%%%%%%%%%%%%%%%%%%%%%%%%%%%%%%%%%%
\DescribeMacro{\childdocforward}
The command |\childdocforward| redirects processing to
another source file:
%
\begin{center}
\begin{tabular}{l}
|\input{childdoc.def}|\\
|\childdocforward[|\textit{main}|]{|\textit{dest}|}|\\
\end{tabular}
\end{center}
%
The argument \textit{dest} is the destination file
(without extension).
It should be the main file or one of the child files.
Note that further \textsf{childdoc} directives
such as |\childdocof| and |\childdocforward|
in the indicated file will be processed in this form.
The optional argument \textit{main}
passes on directly to the main file \textit{main}
while pretending to compile the child \textit{dest}.
This form behaves as if \textit{dest}
issues |\childdocof{|\textit{main}|}| right away,
and no further \textsf{childdoc} directives will be processed.

%%%%%%%%%%%%%%%%%%%%%%%%%%%%%%%%%%%%%%%%
\DescribeMacro{\...prefix}
In the alternative form |\childdocforwardprefix|,
%
\begin{center}
\begin{tabular}{l}
|\input{childdoc.def}|\\
|\childdocforwardprefix[|\textit{main}|]{|\textit{prefix}|}{|\textit{dest}|}|
\end{tabular}
\end{center}
%
the destination file is determined by a pattern
depending on the current file:
To make this work, the current file must be called
`{\textit{prefix}\hspace{0.2em}\textit{suffix}}'
with \textit{prefix} matching precisely the argument.
Processing is then passed on to the file
`{\textit{dest}\hspace{0.2em}\textit{suffix}}'.
Surely, the same effect is achieved by
directly specifying the
argument `{\textit{dest}\hspace{0.2em}\textit{suffix}}'
in the first form.
However, that requires to set up a different file
for each child. With the alternative form of the command
all these files can have exactly the same content
which simplifies setting them up and maintaining them.

For example, the following file |draft.tex|
with a compilation flag |\version| as described in \secref{sec:flags}
compiles the main document as a draft:
%
\begin{center}
\begin{tabular}{l}
|\def\version{draft}|\\
|\input{childdoc.def}|\\
|\childdocforward{|\textit{main}|}|
\end{tabular}
\end{center}
%
Likewise, the following files |final|\textit{nn}|.tex|
compile the final version of the child document
|child|\textit{nn}|.tex|:
%
\begin{center}
\begin{tabular}{l}
|\def\version{final}|\\
|\input{childdoc.def}|\\
|\childdocforwardprefix{final}{child}|
\end{tabular}
\end{center}
%

Note that when several versions of a main file and/or of each child file
are to be generated, it may be convenient to set up a |Makefile| or
shell script to automatise the process.

%%%%%%%%%%%%%%%%%%%%%%%%%%%%%%%%%%%%%%%%%%%%%%%%%%%%%%%%%%%%%%%%%%%%%%%%%%%%%%%%
\subsection{Command Line Processing}
\label{sec:commandline}

The effect of redirection files can also be achieved by invoking
the \LaTeX{} compiler with a more elaborate command line.
Most conveniently this should be done as part
of a shell script or a |Makefile|.

When using \textsf{childdoc} in the main file, the following
command lines effectively perform a redirection
(note that depending on the shell being used,
backslashes may have to be doubled: `|\|' $\to$ `|\\|'):
%
\begin{center}
|... -jobname "|\textit{target}|" |\\|"|[\textit{flags}]%
|\input{childdoc.def}\childdocforward[|\textit{main}|]{|\textit{dest}|}"|
\end{center}
%
Here \textit{target} is the name of the output file,
\textit{main} is the name of the main file
and \textit{dest} is the name of the main or child file to be processed
(all filenames without extensions).
The optional argument \textit{main} can be omitted
if \textit{main} matches \textit{dest}.
Optionally, compilation \textit{flags} can be defined via |\def| commands.
This command line makes the \TeX{} engine believe
it is compiling the file \textit{target}
whose content is specified as the latter parameter.
The provided code then forwards the processing to
\textit{main} or \textit{dest} as described in \secref{sec:forward}.

%%%%%%%%%%%%%%%%%%%%%%%%%%%%%%%%%%%%%%%%%%%%%%%%%%%%%%%%%%%%%%%%%%%%%%%%%%%%%%%%
\subsection{Include by Input}
\label{sec:input}

Including child documents by |\include| has some restrictions by design.
Most notably, the content of a child document always occupies
its own set of pages; pages cannot be shared between child documents.
Usually, this behaviour makes perfect sense
because each child document contain an essential part of the document.
However, in some situations it may be desirable to compose
a document from a collection of parts
without having mandatory page breaks between then.
For this case, the package
provides a mechanism to include parts
by |\input| which can also be processed individually.
However, by construction this mechanism
requires manual handling of the content to be output.

%%%%%%%%%%%%%%%%%%%%%%%%%%%%%%%%%%%%%%%%
\DescribeMacro{\ifchilddocmanual}
The main file should be prepared as usual, see \secref{sec:include}.
However, the document body must make a distinction
between processing of an individual part and of the main document, e.g.:
%
\begin{center}
\begin{tabular}{l}
|\ifchilddocmanual|\\
|\input{\childdocname}|\\
|\||else|\\
\textit{document body with }|\input{|\textit{part}|}|\\
|\||fi|
\end{tabular}
\end{center}
%
The conditional |\ifchilddocmanual| is true whenever
a part to be included by |\input| is being compiled,
and the name of the part is stored in |\childdocname|.

%%%%%%%%%%%%%%%%%%%%%%%%%%%%%%%%%%%%%%%%
\DescribeMacro{\childdocby}
Each part to be included by |\input| should start with:
%
\begin{center}
\begin{tabular}{l}
|\input{childdoc.def}|\\
|\childdocby{|\textit{main}|}|\\
\end{tabular}
\end{center}
%
The directive |\childdocby| is similar to |\childdocof|
described in \secref{sec:include},
but the subsequent selection of content must be done manually.
To that end, both |\ifchilddoc| and |\ifchilddocmanual|
will be true upon processing of a part,
and the name of the part is stored in |\childdocname|.
Note that |\jobname| will be set to the filename of the current part
so that each part receives an individual |.aux| file
that does not interfere with the |.aux| file(s) of the main document.
This behaviour can be altered by the alternative form
|\childdocby[*]{|\textit{main}|}| (with a non-empty optional argument)
which uses the |.aux| file of the main document
by setting |\jobname| to \textit{main}.

%%%%%%%%%%%%%%%%%%%%%%%%%%%%%%%%%%%%%%%%%%%%%%%%%%%%%%%%%%%%%%%%%%%%%%%%%%%%%%%%
\subsection{Driver Development}
\label{sec:driver}

The \textsf{childdoc} mechanism can also be use for the development
of definition files such as \LaTeX{} styles or classes.
This case differs from the above setup with multiple parts
included by |\include| in that no |\includeonly| should be invoked.
This can be achieved by starting the include file
(before |\ProvidesPackage|) with:
%
\begin{center}
\begin{tabular}{l}
|\input{childdoc.def}|\\
|\childdocforward{|\textit{main}|}|\\
\end{tabular}
\end{center}
%
or alternatively with:
%
\begin{center}
\begin{tabular}{l}
|\input{childdoc.def}|\\
|\childdocby{|\textit{main}|}|\\
\end{tabular}
\end{center}
%
Both forms have slightly different effects as described above.
The main file is prepared as usual, see \secref{sec:include}.

%%%%%%%%%%%%%%%%%%%%%%%%%%%%%%%%%%%%%%%%%%%%%%%%%%%%%%%%%%%%%%%%%%%%%%%%%%%%%%%%
\subsection{Legacy Detection}
\label{sec:detection}

The directive |\childdocmain| in the main file can detect
whether the complete document or merely a child is to be compiled
even without using the directive |\childdocof|.
This method is deprecated because it is less robust
and there is no compelling reason to use it;
it is merely provided for backward compatibility
and it may be removed in future versions.

If the detection mechanism is to be used,
it is mandatory to correctly specify
the filename of the main file as the argument of |\childdocmain|:
%
\begin{center}
\begin{tabular}{l}
|\input{childdoc.def}|\\
|\childdocmain{|\textit{main}|}|\\
\end{tabular}
\end{center}
%
If |\jobname| does not match the argument \textit{main} of |\childdocmain|,
it is assumed that |\jobname| points to the child file to be compiled.
When using |\childdocmain| with the main file specified as argument,
it suffices to start a child file
with just |\input{|\textit{main}|}|
without loading of the package and using |\childdocof|.
If instead all processing is done
with the appropriate \textsf{childdoc} directives,
the argument of \textit{main} of |\childdocmain| can be empty.

An alternative version of the command line processing described
in \secref{sec:commandline} using the detection mechanism reads:
%
\begin{center}
|... -jobname "|\textit{target}|" "|[\textit{flags}]%
[|\def\jobname{|\textit{dest}|}|]|\input{|\textit{main}|}"|
\end{center}

%%%%%%%%%%%%%%%%%%%%%%%%%%%%%%%%%%%%%%%%%%%%%%%%%%%%%%%%%%%%%%%%%%%%%%%%%%%%%%%%
\subsection{Manual Code}
\label{sec:manual}

In case one cannot be certain whether the definitions file |childdoc.def|
is installed on the target \TeX{} distribution
and one prefers not to ship it,
it is conceivable to paste a few relevant commands into the sources.

To that end, drop all statements |\input{childdoc.def}|
and perform the replacements as outlined below.
Instead of |\childdocmain{|\textit{main}|}| add the following code
to the top of the main file:
%
\begin{center}
\begin{tabular}{l}
|\||ifdefined\childdocname\endinput\||fi\newif\ifchilddoc|\\
|\edef\childdocname{\scantokens\expandafter{\jobname\noexpand}}|\\
|\def\childdocmain{|\textit{main}|}\||ifx\childdocmain\childdocname\||else|\\
|\childdoctrue\includeonly{\childdocname}\let\jobname\childdocmain\||fi|\\
\end{tabular}
\end{center}
%
Instead of |\childdocof{|\textit{main}|}| just include the main file
at the top of each child file:
%
\begin{center}
|\input{|\textit{main}|}|
\end{center}
%
A simple redirection |\childdocforward{|\textit{dest}|}| is achieved by:
%
\begin{center}
|\def\jobname{|\textit{dest}|}\input{\jobname}|
\end{center}
%
The redirection with prefix
|\childdocforwardprefix[|\textit{prefix}|]{|\textit{dest}|}|
is accomplished by:
%
\begin{center}
\begin{tabular}{l}
|{\edef\jobname{\scantokens\expandafter{\jobname\noexpand}}|\\
|\def\redirectjob |\textit{prefix}|#1~~~{\gdef\jobname{|\textit{dest}|#1}}|\\
|\expandafter\redirectjob\jobname~~~}\input{\jobname}|
\end{tabular}
\end{center}

In an alternative approach,
child documents can be compiled by a specific command line
without additional code or specific definitions:
%
\begin{center}
|... -jobname "|\textit{target}|" "|[\textit{flags}]%
|\includeonly{|\textit{dest}|}\input{|\textit{main}|}"|
\end{center}
%

%%%%%%%%%%%%%%%%%%%%%%%%%%%%%%%%%%%%%%%%%%%%%%%%%%%%%%%%%%%%%%%%%%%%%%%%%%%%%%%%
%%%%%%%%%%%%%%%%%%%%%%%%%%%%%%%%%%%%%%%%%%%%%%%%%%%%%%%%%%%%%%%%%%%%%%%%%%%%%%%%
\section{Information}

%%%%%%%%%%%%%%%%%%%%%%%%%%%%%%%%%%%%%%%%%%%%%%%%%%%%%%%%%%%%%%%%%%%%%%%%%%%%%%%%
\subsection{Copyright}

Copyright \copyright{} 2017--2018 Niklas Beisert

This work may be distributed and/or modified under the
conditions of the \LaTeX{} Project Public License, either version 1.3
of this license or (at your option) any later version.
The latest version of this license is in
  \url{http://www.latex-project.org/lppl.txt}
and version 1.3 or later is part of all distributions of \LaTeX{}
version 2005/12/01 or later.

This work has the LPPL maintenance status `maintained'.

The Current Maintainer of this work is Niklas Beisert.

This work consists of the files |README.txt|, |childdoc.ins| and |childdoc.dtx|
as well as the derived files |childdoc.def|, |cdocsamp.tex|
with |cdocsch1.tex|, |cdocsch2.tex|, |cdocspt3.tex|, |cdocspt4.tex|,
|cdocsdrf.tex|, |cdocsfn1.tex|, |cdocsfn2.tex|
as well as |childdoc.pdf|.

%%%%%%%%%%%%%%%%%%%%%%%%%%%%%%%%%%%%%%%%%%%%%%%%%%%%%%%%%%%%%%%%%%%%%%%%%%%%%%%%
\subsection{Files and Installation}

The package consists of the files:
%
\begin{center}
\begin{tabular}{ll}
    |README.txt|   & readme file \\
    |childdoc.ins| & installation file \\
    |childdoc.dtx| & source file \\
    |childdoc.def| & definition file \\
    |cdocsamp.tex| & sample main file \\
    |cdocsch1.tex| & sample include file \\
    |cdocsch2.tex| & sample include file \\
    |cdocspt3.tex| & sample part file \\
    |cdocspt4.tex| & sample part file \\
    |cdocsdrf.tex| & sample redirection file \\
    |cdocsfn1.tex| & sample redirection file \\
    |cdocsfn2.tex| & sample redirection file \\
    |childdoc.pdf| & manual
\end{tabular}
\end{center}
%
The distribution consists of the files
|README.txt|, |childdoc.ins| and |childdoc.dtx|.
%
\begin{itemize}
\item
Run (pdf)\LaTeX{} on |childdoc.dtx|
to compile the manual |childdoc.pdf| (this file).
\item
Run \LaTeX{} on |childdoc.ins| to create the definitions file |childdoc.def|
and the sample |cdocsamp.tex| with include files
|cdocsch1.tex|, |cdocsch2.tex|, |cdocspt3.tex|, |cdocspt4.tex|,
|cdocsdrf.tex|, |cdocsfn1.tex|, |cdocsfn2.tex|.
Then copy the file |childdoc.def| to an appropriate directory of your \LaTeX{}
distribution, e.g.\ \textit{texmf-root}|/tex/latex/childdoc|.
\end{itemize}

%%%%%%%%%%%%%%%%%%%%%%%%%%%%%%%%%%%%%%%%%%%%%%%%%%%%%%%%%%%%%%%%%%%%%%%%%%%%%%%%
\subsection{Related CTAN Packages}

There are several other packages which offer a similar functionality:
%
\begin{itemize}
\item
The packages
\href{http://ctan.org/pkg/docmute}{\textsf{docmute}},
\href{http://ctan.org/pkg/includex}{\textsf{includex}} and
\href{http://ctan.org/pkg/standalone}{\textsf{standalone}}
provide commands to include only the document body of
a child file thus allowing both files to be compiled individually.
\item
The packages \href{http://ctan.org/pkg/subdocs}{\textsf{subdocs}}
and \href{http://ctan.org/pkg/subfiles}{\textsf{subfiles}}
provide structures in which the main and child documents can be
encapsulated and allowing them to be compiled individually.
The inclusion mechanism is different from the conventional |\include|.
\item
The package \href{http://ctan.org/pkg/combine}{\textsf{combine}}
is an elaborate solution to combine several documents into one.
\end{itemize}
%
See also the CTAN topic \href{http://ctan.org/topic/subdocs}{\textsf{subdocs}}
for further related packages.
The present package differs from the above solutions in that
a document structure constructed with the conventional |\include| mechanism
just needs two extra commands at the top of every file
such that all constituent files can be compiled individually.

%%%%%%%%%%%%%%%%%%%%%%%%%%%%%%%%%%%%%%%%%%%%%%%%%%%%%%%%%%%%%%%%%%%%%%%%%%%%%%%%
%\subsection{Feature Suggestions}
%
%The following is a list of features which may be useful for future
%versions of this package:
%%
%\begin{itemize}
%\item
%\ldots
%\end{itemize}

%%%%%%%%%%%%%%%%%%%%%%%%%%%%%%%%%%%%%%%%%%%%%%%%%%%%%%%%%%%%%%%%%%%%%%%%%%%%%%%%
\subsection{Revision History}

%%%%%%%%%%%%%%%%%%%%%%%%%%%%%%%%%%%%%%%%
\paragraph{v2.0:} 2018/12/30

\begin{itemize}
\item
immediate forward processing
\item
added |\childdocby| mechanism
\item
manual restructured
\end{itemize}

%%%%%%%%%%%%%%%%%%%%%%%%%%%%%%%%%%%%%%%%
\paragraph{v1.6:} 2018/01/17

\begin{itemize}
\item
application for development of include files
\item
corrections to manual
\end{itemize}

%%%%%%%%%%%%%%%%%%%%%%%%%%%%%%%%%%%%%%%%
\paragraph{v1.5:} 2017/05/21

\begin{itemize}
\item
more complete structuring introduced
\item
|\childdocof| introduced
\item
|\childdoc| renamed to |\childdocmain|
\item
|\childredirect| renamed to |\childdocforward| and |\childdocforwardprefix|
and functionality expanded
\end{itemize}

%%%%%%%%%%%%%%%%%%%%%%%%%%%%%%%%%%%%%%%%
\paragraph{v1.0:} 2017/04/27

\begin{itemize}
\item
manual and install package
\item
first version published on CTAN
\end{itemize}

%%%%%%%%%%%%%%%%%%%%%%%%%%%%%%%%%%%%%%%%
\paragraph{v0.6:} 2017/04/26

\begin{itemize}
\item
redirection mechanism added
\end{itemize}

%%%%%%%%%%%%%%%%%%%%%%%%%%%%%%%%%%%%%%%%
\paragraph{v0.5:} 2017/04/26

\begin{itemize}
\item
functionality in definition file
\end{itemize}


%%%%%%%%%%%%%%%%%%%%%%%%%%%%%%%%%%%%%%%%%%%%%%%%%%%%%%%%%%%%%%%%%%%%%%%%%%%%%%%%
%%%%%%%%%%%%%%%%%%%%%%%%%%%%%%%%%%%%%%%%%%%%%%%%%%%%%%%%%%%%%%%%%%%%%%%%%%%%%%%%
%%%%%%%%%%%%%%%%%%%%%%%%%%%%%%%%%%%%%%%%%%%%%%%%%%%%%%%%%%%%%%%%%%%%%%%%%%%%%%%%
\appendix

\settowidth\MacroIndent{\rmfamily\scriptsize 000\ }

 \DocInput{childdoc.dtx}

\end{document}
%</driver>
% \fi
%
% %%%%%%%%%%%%%%%%%%%%%%%%%%%%%%%%%%%%%%%%%%%%%%%%%%%%%%%%%%%%%%%%%%%%%%%%%%%%%%
% %%%%%%%%%%%%%%%%%%%%%%%%%%%%%%%%%%%%%%%%%%%%%%%%%%%%%%%%%%%%%%%%%%%%%%%%%%%%%%
% \section{Sample}
%\iffalse
%<*samplemain>
%\fi
%
% The following presents a sample document
% with two chapters, two parts, a title page,
% a compile flag as well as three forwarding files to set the flag.
% It consists of eight |.tex| files:
% \begin{center}
% \begin{tabular}{ll}
% |cdocsamp.tex|&main file\\
% |cdocsch1.tex|&include file for chapter 1\\
% |cdocsch2.tex|&include file for chapter 2\\
% |cdocspt3.tex|&include file for part 3\\
% |cdocspt4.tex|&include file for part 4\\
% |cdocsdrf.tex|&forwarding file for main file in draft mode\\
% |cdocsfi1.tex|&forwarding file for final version of chapter 1\\
% |cdocsfi2.tex|&forwarding file for final version of chapter 2\\
% \end{tabular}
% \end{center}
% Each of the eight files can be compiled directly by the \LaTeX{} compiler.
%
% %%%%%%%%%%%%%%%%%%%%%%%%%%%%%%%%%%%%%%
% \paragraph{Main File.}
%
% The main file is called |cdocsamp.tex|.
%
% Load the \textsf{childdoc} definitions and
% declare the filename for the main document:
%    \begin{macrocode}
\input{childdoc.def}
\childdocmain{}
%    \end{macrocode}

% Optional override for |\version| flag:
%    \begin{macrocode}
%%\ifchilddoc\else\providecommand{\version}{draft}\fi
%    \end{macrocode}

% Define the default values for the |\version| flag
% (|final| for the main file and |draft| for childs):
%    \begin{macrocode}
\ifchilddoc
\providecommand{\version}{draft}
\else
\providecommand{\version}{final}
\fi
%    \end{macrocode}

% Load the standard document class:
%    \begin{macrocode}
\documentclass[12pt]{article}
%    \end{macrocode}

% Start the document body:
%    \begin{macrocode}
\begin{document}
%    \end{macrocode}

% Declare a title page.
% Print title, part of document being processed and version flag:
%    \begin{macrocode}
\addtocounter{page}{-1}
\begin{center}
{\LARGE\bfseries{}childdoc example\par}
\vspace{1cm}
\ifchilddoc
\ifchilddocmanual part\else chapter\fi:
`\childdocname' of `\childdocjob'\par
\else
main document: `\childdocjob'\par
\fi
version: \version\par
\end{center}
\newpage
%    \end{macrocode}

% Manually include selected file,
% otherwise process as usual:
%    \begin{macrocode}
\ifchilddocmanual
\section*{part `\childdocname'}
\input{\childdocname}
\else
%    \end{macrocode}

% Include the two chapters:
%    \begin{macrocode}
\include{cdocsch1}
\include{cdocsch2}
%    \end{macrocode}

% Include the two parts unless only chapters should be displayed:
%    \begin{macrocode}
\ifchilddoc\else
\section{part three}
\input{cdocspt3}
\section{part four}
\input{cdocspt4}
\fi
%    \end{macrocode}

% Process as usual until here:
%    \begin{macrocode}
\fi
%    \end{macrocode}

% End of document body:
%    \begin{macrocode}
\end{document}
%    \end{macrocode}
%\iffalse
%</samplemain>
%\fi
%
% %%%%%%%%%%%%%%%%%%%%%%%%%%%%%%%%%%%%%%
% \paragraph{Chapter Include Files.}
%
% The include files are called |cdocsch1.tex| and |cdocsch2.tex|.
%
%\iffalse
%<*samplechap1|samplechap2>
%\fi

% Optional override for |\version| flag:
%    \begin{macrocode}
%%\providecommand{\version}{final}
%    \end{macrocode}

% Include the main document:
%    \begin{macrocode}
\input{childdoc.def}
\childdocof{cdocsamp}
%    \end{macrocode}

%\iffalse
%</samplechap1|samplechap2>
%\fi
%
%\iffalse
%<*samplechap1>
%\fi
% Some text for chapter 1:
%    \begin{macrocode}
\section{one}
some text in chapter one
%    \end{macrocode}

%\iffalse
%</samplechap1>
%\fi
% Some text for chapter 2:
%\iffalse
%<*samplechap2>
%\fi
%    \begin{macrocode}
\section{two}
more text in chapter two
%    \end{macrocode}

%\iffalse
%</samplechap2>
%\fi
%
% %%%%%%%%%%%%%%%%%%%%%%%%%%%%%%%%%%%%%%
% \paragraph{Part Include Files.}
%
% The include files are called |cdocspt3.tex| and |cdocspt4.tex|.
%
%\iffalse
%<*samplepart3|samplepart4>
%\fi

% Optional override for |\version| flag:
%    \begin{macrocode}
%%\providecommand{\version}{final}
%    \end{macrocode}

% Include the main document:
%    \begin{macrocode}
\input{childdoc.def}
\childdocby{cdocsamp}
%    \end{macrocode}

%\iffalse
%</samplepart3|samplepart4>
%\fi
%
%\iffalse
%<*samplepart3>
%\fi
% Some text for part 3:
%    \begin{macrocode}
some text in part three
%    \end{macrocode}

%\iffalse
%</samplepart3>
%\fi
% Some text for part 4:
%\iffalse
%<*samplepart4>
%\fi
%    \begin{macrocode}
more text in part four
%    \end{macrocode}

%\iffalse
%</samplepart4>
%\fi
%
% %%%%%%%%%%%%%%%%%%%%%%%%%%%%%%%%%%%%%%
% \paragraph{Forwarding for a Complete Draft.}
%
% The following forwarding file |cdocsdrf.tex|
% compiles the main document in draft mode:
%\iffalse
%<*sampledraft>
%\fi
%    \begin{macrocode}
\def\version{draft}
\input{childdoc.def}
\childdocforward{cdocsamp}
%    \end{macrocode}

%\iffalse
%</sampledraft>
%\fi
%
% %%%%%%%%%%%%%%%%%%%%%%%%%%%%%%%%%%%%%%
% \paragraph{Forwarding for Final Version of the Chapters.}
%
% The following forwarding files |cdocsfn1.tex| and |cdocsfn2.tex|
% (with identical content)
% compile the final versions of the child documents
% |cdocsch1.tex| and |cdocsch2.tex|, respectively:
%\iffalse
%<*samplefinal>
%\fi
%    \begin{macrocode}
\def\version{final}
\input{childdoc.def}
\childdocforwardprefix[cdocsamp]{cdocsfn}{cdocsch}
%    \end{macrocode}

%\iffalse
%</samplefinal>
%\fi
%
% %%%%%%%%%%%%%%%%%%%%%%%%%%%%%%%%%%%%%%
% \paragraph{Command Line Processing.}
%
% The following three command lines generate the output files
% |cdocscld|, |cdocscl1| and |cdocscl2|
% which should be identical to
% |cdocsdrf|, |cdocsch1| and |cdocsfn2|, respectively:
% \begin{center}
% \begin{tabular}{l}
% |latex -jobname cdocscld \|\\
% |  "\def\version{draft}\input{childdoc.def}\childdocforward{cdocsamp}"|\\
% |latex -jobname cdocscl1 \|\\
% |  "\input{childdoc.def}\childdocforward[cdocsamp]{cdocsch1}"|\\
% |latex -jobname cdocscl2 \|\\
% |  "\def\version{final}\input{childdoc.def}\childdocforward{cdocsch2}"|
% \end{tabular}
% \end{center}
% Note that the trailing backslash on each first line
% merely continues the input to the second line
% (for convenient cut ant paste).
% Furthermore, the command |latex| can be replaced by any
% of its alternative versions such as |pdflatex|.
%
% %%%%%%%%%%%%%%%%%%%%%%%%%%%%%%%%%%%%%%%%%%%%%%%%%%%%%%%%%%%%%%%%%%%%%%%%%%%%%%
% %%%%%%%%%%%%%%%%%%%%%%%%%%%%%%%%%%%%%%%%%%%%%%%%%%%%%%%%%%%%%%%%%%%%%%%%%%%%%%
% \section{Implementation}
%\iffalse
%<*package>
%\fi
%
% This section describes the definitions file |childdoc.def|.

% The definitions cannot be loaded using |\usepackage| or |\RequirePackage|
% which has a mechanism to prevent loading a style file more than once.
% When loading the definitions by means of |\input|
% multiple instances have to be prevented manually:
%\iffalse
%This code needs to be before the `\ProvidesFile' directive
%which is defined at the beginning of this file.
%Therefore it is also placed there and commented out here.
%</package>
%<*discard>
%\fi
%    \begin{macrocode}
\ifdefined\childdocmain\endinput\fi
%    \end{macrocode}
%\iffalse
%</discard>
%<*package>
%\fi
%
% \macro{\ifchilddoc}
% \macro{\ifchilddocmanual}
% The conditional |\ifchilddoc| tells whether a
% child (true) or main (false) document is being compiled.
% The conditional |\ifchilddocmanual| tells whether
% the |\includeonly| mechanism is used (false) or
% the selection of child files must be performed manually (true).
% The definitions initialise to false:
%    \begin{macrocode}
\newif\ifchilddoc
\newif\ifchilddocmanual
%    \end{macrocode}

% \macro{\childdocname}
% \macro{\childdocjob}
% The macro |\childdocname| stores the name of the main document
% to be compiled. The macro |\childdocjob| stores the name of
% the document on which the \LaTeX{} compiler was originally invoked.
% The content of |\jobname| cannot be compared
% to filenames specified in the source due to different catcodes.
% The following code rescans |\jobname|, stores the result
% in |\childdocname| and saves a copy in |\childdocjob|:
%    \begin{macrocode}
\edef\childdocname{\scantokens\expandafter{\jobname\noexpand}}
\let\childdocjob\childdocname
%    \end{macrocode}

% \macro{\childdocdisable}
% The macro |\childdocdisable| prevents the main file
% from being processed more than once.
% At this stage, the main document command |\childdocmain|
% is assumed to be called once again where it should do nothing.
% Any subsequent call to it should prevent
% a secondary processing of the main document
% It overwrites the forwarding commands
% |\childdocof| and |\childdocforward|
% with empty macros to prevent further inclusions of the main document:
%    \begin{macrocode}
\newcommand{\childdocdisable}
{
  \renewcommand{\childdocmain}[1]{\renewcommand{\childdocmain}[1]{\endinput}}
  \renewcommand{\childdocof}[1]{}
  \renewcommand{\childdocby}[2][]{}
  \renewcommand{\childdocforward}[2][]{}
  \renewcommand{\childdocdisable}{}
}
%    \end{macrocode}

% \macro{\childdocmain}
% The macro |\childdocmain| is to be called at the top of the main file
% with nothing or the main filename (without extension) as argument.
% First, it breaks loops.
% If the argument is not empty and does not match |\childdocname|
% (which is set by the first inclusion of |childdoc.def|),
% |\ifchilddoc| is set to true, |\includeonly| is applied to the child file
% and |\jobname| is set to the main file
% (for proper handling of |.aux| files):
%    \begin{macrocode}
\newcommand{\childdocmain}[1]
{
  \childdocdisable\childdocmain{}
  \if?#1?\else
    \begingroup
      \def\childdoctmp{#1}
      \ifx\childdoctmp\childdocname
        \def\childdoctmp{}
      \else
        \def\childdoctmp
        {
          \childdoctrue
          \includeonly{\childdocname}
          \def\childdocjob{#1}
          \def\jobname{#1}
        }
      \fi
      \expandafter
    \endgroup
    \childdoctmp
  \fi
}
%    \end{macrocode}

% \macro{\childdocof}
% The command |\childdocof| redirects
% compilation to the main file |#1|.
%    \begin{macrocode}
\newcommand{\childdocof}[1]
{
  \childdocdisable
  \childdoctrue
  \includeonly{\childdocname}
  \def\jobname{#1}
  \def\childdocjob{#1}
  \input{#1}
}
%    \end{macrocode}

% \macro{\childdocby}
% The command |\childdocby| ....
%    \begin{macrocode}
\newcommand{\childdocby}[2][]
{
  \childdocdisable
  \childdoctrue
  \childdocmanualtrue
  \if?#1?\else
    \def\jobname{#2}
  \fi
  \def\childdocjob{#2}
  \input{#2}
  \endinput
}
%    \end{macrocode}

% \macro{\childdocforward}
% The command |\childdocforward| redirects
% compilation to the main file or
% (if the optional argument is given) a child file.
% Parameters are set as if the main file
% or a child file starting with |\childdocof| was compiled.
% Then compilation is handed over to the main file:
%    \begin{macrocode}
\newcommand{\childdocforward}[2][]
{
  \begingroup
    \if?#1?
      \def\childdoctmp
      {
        \def\childdocname{#2}
        \def\childdocjob{#2}
        \def\jobname{#2}
        \input{#2}
        \endinput
      }
    \else
      \def\childdoctmp
      {
        \childdocdisable
        \def\childdocname{#2}
        \childdoctrue
        \includeonly{#2}
        \def\childdocjob{#1}
        \def\jobname{#1}
        \input{#1}
        \endinput
      }
    \fi
    \expandafter
  \endgroup
  \childdoctmp
}
%    \end{macrocode}

% \macro{\childdocforwardprefix}
% The command |\childdocforwardprefix| redirects
% compilation to the main or a child file by means of a pattern.
% The prefix |#1| in the current filename is replaced by |#2|
% and the suffix of the current filename is kept
% (it is assumed that the filename does not contain the substring `|~~~|'
% which is used as a delimiter).
% Compilation is handed over to the new file by |\childdocforward|:
%    \begin{macrocode}
\newcommand{\childdocforwardprefix}[3][]
{
  \begingroup
    \def\childdocextract #2##1~~~{\def\childdoctmp{\childdocforward[#1]{#3##1}}}
    \expandafter\childdocextract\childdocname~~~
    \expandafter
  \endgroup
  \childdoctmp
}
%    \end{macrocode}

% \macro{\childdoc}
% The deprecated macro |\childdoc| is a legacy version of |\childdocmain|:
%    \begin{macrocode}
\newcommand{\childdoc}{\childdocmain}
%    \end{macrocode}

% \macro{\childdocredirect}
% The deprecated macro |\childdocredirect| is a legacy version
% of |\childdocforward| and |\childdocforwardprefix|:
%    \begin{macrocode}
\newcommand{\childdocredirect}[2][]
{
  \begingroup
    \if?#1?
      \def\childdoctmp{\childdocforward{#2}}
    \else
      \def\childdoctmp{\childdocforwardprefix{#1}{#2}}
    \fi
    \expandafter
  \endgroup
  \childdoctmp
}
%    \end{macrocode}

%\iffalse
%</package>
%\fi
%
\endinput
|\\
|\childdocforwardprefix{final}{child}|
\end{tabular}
\end{center}
%

Note that when several versions of a main file and/or of each child file
are to be generated, it may be convenient to set up a |Makefile| or
shell script to automatise the process.

%%%%%%%%%%%%%%%%%%%%%%%%%%%%%%%%%%%%%%%%%%%%%%%%%%%%%%%%%%%%%%%%%%%%%%%%%%%%%%%%
\subsection{Command Line Processing}
\label{sec:commandline}

The effect of redirection files can also be achieved by invoking
the \LaTeX{} compiler with a more elaborate command line.
Most conveniently this should be done as part
of a shell script or a |Makefile|.

When using \textsf{childdoc} in the main file, the following
command lines effectively perform a redirection
(note that depending on the shell being used,
backslashes may have to be doubled: `|\|' $\to$ `|\\|'):
%
\begin{center}
|... -jobname "|\textit{target}|" |\\|"|[\textit{flags}]%
|% \iffalse
%
% childdoc.dtx Copyright (C) 2017-2018 Niklas Beisert
%
% This work may be distributed and/or modified under the
% conditions of the LaTeX Project Public License, either version 1.3
% of this license or (at your option) any later version.
% The latest version of this license is in
%   http://www.latex-project.org/lppl.txt
% and version 1.3 or later is part of all distributions of LaTeX
% version 2005/12/01 or later.
%
% This work has the LPPL maintenance status `maintained'.
%
% The Current Maintainer of this work is Niklas Beisert.
%
% This work consists of the files childdoc.dtx and childdoc.ins
% and the derived files childdoc.def and cdocsamp.tex with
% cdocsch1.tex, cdocsch2.tex, cdocsdrf.tex, cdocsfn1.tex, cdocsfn2.tex.
%
%<package>\ifdefined\childdocmain\endinput\fi
%<package>\ProvidesFile{childdoc.def}[2018/12/30 v2.0 child document driver]
%<samplemain>\ProvidesFile{cdocsamp.tex}[2018/12/30 v2.0 sample for childdoc]
%<*driver>
%\ProvidesFile{childdoc.drv}[2018/12/30 v2.0 childdoc reference manual file]
\PassOptionsToClass{10pt,a4paper}{article}
\documentclass{ltxdoc}

\usepackage[margin=35mm]{geometry}
\usepackage{hyperref}
\usepackage{hyperxmp}
\usepackage[usenames]{color}

\hypersetup{colorlinks=true}
\hypersetup{pdfstartview=FitH}
\hypersetup{pdfpagemode=UseNone}
\hypersetup{pdfsource={}}
\hypersetup{pdflang={en-UK}}
\hypersetup{pdfcopyright={Copyright 2017-2018 Niklas Beisert.
  This work may be distributed and/or modified under the
  conditions of the LaTeX Project Public License, either version 1.3
  of this license or (at your option) any later version.}}
\hypersetup{pdflicenseurl={http://www.latex-project.org/lppl.txt}}
\hypersetup{pdfcontactaddress={ETH Zurich, ITP, HIT K,
  Wolfgang-Pauli-Strasse 27}}
\hypersetup{pdfcontactpostcode={8093}}
\hypersetup{pdfcontactcity={Zurich}}
\hypersetup{pdfcontactcountry={Switzerland}}
\hypersetup{pdfcontactemail={nbeisert@itp.phys.ethz.ch}}
\hypersetup{pdfcontacturl={http://people.phys.ethz.ch/\xmptilde nbeisert/}}

\newcommand{\secref}[1]{\hyperref[#1]{section \ref*{#1}}}

\parskip1ex
\parindent0pt
\let\olditemize\itemize
\def\itemize{\olditemize\parskip0pt}

\begin{document}

\title{The \textsf{childdoc} Package}
\hypersetup{pdftitle={The childdoc Package}}
\author{Niklas Beisert\\[2ex]
  Institut f\"ur Theoretische Physik\\
  Eidgen\"ossische Technische Hochschule Z\"urich\\
  Wolfgang-Pauli-Strasse 27, 8093 Z\"urich, Switzerland\\[1ex]
  \href{mailto:nbeisert@itp.phys.ethz.ch}
  {\texttt{nbeisert@itp.phys.ethz.ch}}}
\hypersetup{pdfauthor={Niklas Beisert}}
\hypersetup{pdfsubject={Manual for the LaTeX2e Package childdoc}}
\date{30 December 2018, \textsf{v2.0}}
\maketitle

\begin{abstract}\noindent
\textsf{childdoc} is a \LaTeXe{} package
that enables the direct compilation
of document sections included by |\include|
to individual files.
\end{abstract}

\begingroup
\parskip0ex
\tableofcontents
\endgroup

%%%%%%%%%%%%%%%%%%%%%%%%%%%%%%%%%%%%%%%%%%%%%%%%%%%%%%%%%%%%%%%%%%%%%%%%%%%%%%%%
%%%%%%%%%%%%%%%%%%%%%%%%%%%%%%%%%%%%%%%%%%%%%%%%%%%%%%%%%%%%%%%%%%%%%%%%%%%%%%%%
\section{Introduction}

\LaTeX{} provides a mechanism to structure a large document (such as a book)
into a main file and several child files (containing the chapters)
using the |\include| command.
This mechanism is beneficial for documents
which span hundreds of pages in order to
make the source file(s) more manageable.
Moreover, compilation can be restricted to
selected child files by means of the |\includeonly| command.
The latter feature can be used to reduce the compilation time while editing
(this was significantly more useful in the earlier days of \LaTeX{})
or to generate a smaller document which is easier to navigate.
Another application of |\includeonly| is to generate
documents consisting of selected parts of the complete document.

However, there are a few drawbacks of the plain |\include| mechanism:
\begin{itemize}
\item
The child files cannot be compiled on their own,
they can only be compiled via the main file.
A naive editing environment
(such as a text editor with an option
to have the current file processed by \LaTeX)
may require one to switch to the main file before compiling;
attempting to compile the child file produces errors.
\item
The main file must be modified (each time)
to adjust the |\includeonly| command
to the present needs. This easily leaves the main file in a messy state.
\item
The generated document will always carry the filename
of the main document. This is inconvenient if
several child files are to be compiled and
to be kept for distribution.
\end{itemize}

The present package provides a simple interface
to make child files individually compilable by \LaTeX{}.
Compiling a child file then has the same effect as compiling
the main file with an |\includeonly| command
to select the appropriate child.
Moreover the generated document will carry the name of the child
rather than the main file.
This resolves all three above issues.

This feature is meant to make the editing of books,
thesis documents and lecture notes somewhat more convenient.
However, the package can also be used efficiently for
composing a series of documents (such as exercise sheets)
which are typically distributed individually.
It then assists the author in generating the individual documents
(potentially in different versions)
as well as a document containing the collected series.
Another application is in developing style files
or other kinds of included material
where compilation of the style file could redirect
to a sample or test file.

%%%%%%%%%%%%%%%%%%%%%%%%%%%%%%%%%%%%%%%%%%%%%%%%%%%%%%%%%%%%%%%%%%%%%%%%%%%%%%%%
%%%%%%%%%%%%%%%%%%%%%%%%%%%%%%%%%%%%%%%%%%%%%%%%%%%%%%%%%%%%%%%%%%%%%%%%%%%%%%%%
\section{Usage}

First of all, the package \textsf{childdoc} is \emph{not} a standard
\LaTeXe{} |.sty| style file! Therefore it needs to be invoked in
a non-standard way.

%%%%%%%%%%%%%%%%%%%%%%%%%%%%%%%%%%%%%%%%%%%%%%%%%%%%%%%%%%%%%%%%%%%%%%%%%%%%%%%%
\subsection{Included Files}
\label{sec:include}

%%%%%%%%%%%%%%%%%%%%%%%%%%%%%%%%%%%%%%%%
\DescribeMacro{\childdocmain}
To use the package, add the commands
\begin{center}
\begin{tabular}{l}
|\input{childdoc.def}|\\
|\childdocmain{}|\\
\end{tabular}
\end{center}
at the very top of the main \LaTeX{} file,
in particular \emph{before} the |\documentclass| statement!
The argument of |\childdocmain| should be left empty
(but it must be present).

%%%%%%%%%%%%%%%%%%%%%%%%%%%%%%%%%%%%%%%%
\DescribeMacro{\childdocof}
Furthermore, add the commands
\begin{center}
\begin{tabular}{l}
|\input{childdoc.def}|\\
|\childdocof{|\textit{main}|}|\\
\end{tabular}
\end{center}
at the top of every child file \textit{child}
which is included by |\include{|\textit{child}|}|
from within the main file
(or at least for those files to be compiled individually).
The argument \textit{main} must be the filename of the main file.

There are a couple of
considerations in setting up the main and child documents:

%%%%%%%%%%%%%%%%%%%%%%%%%%%%%%%%%%%%%%%%
\paragraph{Restrictions.}

Please note the following restrictions:
\begin{itemize}
\item
|\childdocmain| must be called with one argument \textit{main}
to ensure compatibility with earlier version of the package.
It must either be empty (|\childdocmain{}|)
or precisely match the filename of the main file in which it is specified.
See \secref{sec:detection} for further information.
\item
The filename \textit{main} must be specified without the |.tex| extension.
\item
The filename \textit{main} is case sensitive
(even in case-insensitive file systems)
due to internal string comparison.
\item
The argument \textit{main} should be fully expanded, it cannot be a macro.
\item
Subdirectories and special characters should be avoided in filenames.
\item
The command |\childdocmain{|\textit{main}|}| must be followed by a whitespace.
It should not be followed immediately by another command
or by a comment mark `|%|'.
This is because the \TeX{} parser reads the token immediately following
the argument of |\childdocmain| and puts it
at the beginning of every child section;
however, a white\-space is ignored.
\end{itemize}

%%%%%%%%%%%%%%%%%%%%%%%%%%%%%%%%%%%%%%%%
\paragraph{Content of Main File.}

It is advisable to place all content in the child files included by |\include|.
Any output contained in the main file will appear in all child documents
unless suppressed manually;
it cannot be suppressed automatically by the |\includeonly| directive
and thus should normally be avoided.
A method to include some content in the main file
by means of conditional processing is described in \secref{sec:conditional}.

%%%%%%%%%%%%%%%%%%%%%%%%%%%%%%%%%%%%%%%%
\paragraph{Page Numbering.}

When only a part of the document is compiled,
the appropriate numbering of pages
(as well as other status parameters)
is determined from the |.aux| files.
The latter contain information from previous passes.
However this information needs to propagate through
all intermediate child documents.
Therefore the page numbering in child documents may well
be inconsistent until the complete document is compiled at least once.

A useful (if unconventional) way to always ensure a consistent
page numbering is to restart the numbering in each child document
and denote the pages by `\textit{child}|.|\textit{page}'
where \textit{child} represents the chapter/section number of the child file.
This can be achieved by the command
|\numberwithin{page}{|\textit{child}|}|
of the \textsf{amsmath} package
where \textit{child} can be |chapter| or |section|
depending on the chosen structuring.
Alternatively, one can modify the macro |\thepage| appropriately
and reset the counter |page| at the start of each child file.

%%%%%%%%%%%%%%%%%%%%%%%%%%%%%%%%%%%%%%%%%%%%%%%%%%%%%%%%%%%%%%%%%%%%%%%%%%%%%%%%
\subsection{Conditional Processing}
\label{sec:conditional}

The package provides a mechanism to compile different versions
of a document. To customise the versions further some conditional processing
can come in handy to distinguish which version is being compiled.
The package provides two macros to describe the compilation context:

%%%%%%%%%%%%%%%%%%%%%%%%%%%%%%%%%%%%%%%%
\DescribeMacro{\ifchilddoc}
The conditional |\ifchilddoc| distinguishes between the compilation of
child documents and the main document:
%
\begin{center}
|\ifchilddoc |\textit{child-code}| |[|\||else |\textit{main-code}]| \||fi|
\end{center}

%%%%%%%%%%%%%%%%%%%%%%%%%%%%%%%%%%%%%%%%
\DescribeMacro{\childdocname}
\DescribeMacro{\childdocjob}
The macro |\childdocname| contains the filename (without extension)
of the main or child file being processed.
Note that |\childdocjob| will always contain the name of the main file.

%%%%%%%%%%%%%%%%%%%%%%%%%%%%%%%%%%%%%%%%
\paragraph{Title Page.}

Conditional processing can be used to include a title or banner page
in the main document when proper precautions are taken.
Importantly, the code in the main file should ensure that the page counter
(as well as other status parameters which are stored in the |.aux| files)
takes the same value after the conditional processing.
Otherwise the page numbers may take divergent values
depending on which part is compiled.

For example, a title page could be declared by:
%
\begin{center}
\begin{tabular}{l}
|\ifchilddoc\||else|\\
|\addtocounter{page}{-1}|\\
\textit{code for title page}\\
|\newpage|\\
|\||fi|
\end{tabular}
\end{center}
%
A banner page for the child documents can be generated by:
%
\begin{center}
\begin{tabular}{l}
|\ifchilddoc|\\
|\addtocounter{page}{-1}|\\
\textit{code for banner page}\\
|\newpage|\\
|\||fi|
\end{tabular}
\end{center}
%
Here one could write a message such as:
\begin{center}
|This is the part \childdocname{} of \childdocjob{}.|
\end{center}

%%%%%%%%%%%%%%%%%%%%%%%%%%%%%%%%%%%%%%%%%%%%%%%%%%%%%%%%%%%%%%%%%%%%%%%%%%%%%%%%
\subsection{Flags}
\label{sec:flags}

The package makes it easy to generate different versions
of the main or child documents.
To this end compilation flags can be defined
and assigned different default values.
They will be particularly useful in conjunction
with the forwarding mechanism described in \secref{sec:forward}.

For example, it may be useful to have a flag |\version|
which can be set to |draft| or |final|.
The document source will contain some conditional code
depending on the value of |\version|.
Suppose further, the flag should default to |final| for the main file
and to |draft| for child files
which is a natural assignment for editing the document.
This is achieved by placing the following code
in the preamble of the main document
(below the |\childdocmain| directive):
%
\begin{center}
\begin{tabular}{l}
|\ifchilddoc|\\
|\providecommand{\version}{draft}|\\
|\||else|\\
|\providecommand{\version}{final}|\\
|\||fi|
\end{tabular}
\end{center}
%
The definition by |\providecommand| makes sure
that previous definitions are not overwritten.
Further statements |\providecommand{\version}{...}|
can thus be added before the above code to override it.

For the main file, one might add a line
(between |\childdocmain| and the above block)
%
\begin{center}
|%\ifchilddoc\||else\providecommand{\version}{draft}\||fi|
\end{center}
%
which can be uncommented to produce a draft version.
Likewise one can add a line to the very top of a child file
(above the |\childdocof{|\textit{main}|}| directive)
%
\begin{center}
|%\providecommand{\version}{final}|
\end{center}
%
which can be uncommented to produce the final version of this child document.

%%%%%%%%%%%%%%%%%%%%%%%%%%%%%%%%%%%%%%%%%%%%%%%%%%%%%%%%%%%%%%%%%%%%%%%%%%%%%%%%
\subsection{Forwarding}
\label{sec:forward}

Different versions of the main or child documents
using compilation flags as described in \secref{sec:flags}
can be (permanently) stored in different files
for convenient compilation, viewing and distribution.
To this end, the package defines a command
to pass on compilation to a different file:

%%%%%%%%%%%%%%%%%%%%%%%%%%%%%%%%%%%%%%%%
\DescribeMacro{\childdocforward}
The command |\childdocforward| redirects processing to
another source file:
%
\begin{center}
\begin{tabular}{l}
|\input{childdoc.def}|\\
|\childdocforward[|\textit{main}|]{|\textit{dest}|}|\\
\end{tabular}
\end{center}
%
The argument \textit{dest} is the destination file
(without extension).
It should be the main file or one of the child files.
Note that further \textsf{childdoc} directives
such as |\childdocof| and |\childdocforward|
in the indicated file will be processed in this form.
The optional argument \textit{main}
passes on directly to the main file \textit{main}
while pretending to compile the child \textit{dest}.
This form behaves as if \textit{dest}
issues |\childdocof{|\textit{main}|}| right away,
and no further \textsf{childdoc} directives will be processed.

%%%%%%%%%%%%%%%%%%%%%%%%%%%%%%%%%%%%%%%%
\DescribeMacro{\...prefix}
In the alternative form |\childdocforwardprefix|,
%
\begin{center}
\begin{tabular}{l}
|\input{childdoc.def}|\\
|\childdocforwardprefix[|\textit{main}|]{|\textit{prefix}|}{|\textit{dest}|}|
\end{tabular}
\end{center}
%
the destination file is determined by a pattern
depending on the current file:
To make this work, the current file must be called
`{\textit{prefix}\hspace{0.2em}\textit{suffix}}'
with \textit{prefix} matching precisely the argument.
Processing is then passed on to the file
`{\textit{dest}\hspace{0.2em}\textit{suffix}}'.
Surely, the same effect is achieved by
directly specifying the
argument `{\textit{dest}\hspace{0.2em}\textit{suffix}}'
in the first form.
However, that requires to set up a different file
for each child. With the alternative form of the command
all these files can have exactly the same content
which simplifies setting them up and maintaining them.

For example, the following file |draft.tex|
with a compilation flag |\version| as described in \secref{sec:flags}
compiles the main document as a draft:
%
\begin{center}
\begin{tabular}{l}
|\def\version{draft}|\\
|\input{childdoc.def}|\\
|\childdocforward{|\textit{main}|}|
\end{tabular}
\end{center}
%
Likewise, the following files |final|\textit{nn}|.tex|
compile the final version of the child document
|child|\textit{nn}|.tex|:
%
\begin{center}
\begin{tabular}{l}
|\def\version{final}|\\
|\input{childdoc.def}|\\
|\childdocforwardprefix{final}{child}|
\end{tabular}
\end{center}
%

Note that when several versions of a main file and/or of each child file
are to be generated, it may be convenient to set up a |Makefile| or
shell script to automatise the process.

%%%%%%%%%%%%%%%%%%%%%%%%%%%%%%%%%%%%%%%%%%%%%%%%%%%%%%%%%%%%%%%%%%%%%%%%%%%%%%%%
\subsection{Command Line Processing}
\label{sec:commandline}

The effect of redirection files can also be achieved by invoking
the \LaTeX{} compiler with a more elaborate command line.
Most conveniently this should be done as part
of a shell script or a |Makefile|.

When using \textsf{childdoc} in the main file, the following
command lines effectively perform a redirection
(note that depending on the shell being used,
backslashes may have to be doubled: `|\|' $\to$ `|\\|'):
%
\begin{center}
|... -jobname "|\textit{target}|" |\\|"|[\textit{flags}]%
|\input{childdoc.def}\childdocforward[|\textit{main}|]{|\textit{dest}|}"|
\end{center}
%
Here \textit{target} is the name of the output file,
\textit{main} is the name of the main file
and \textit{dest} is the name of the main or child file to be processed
(all filenames without extensions).
The optional argument \textit{main} can be omitted
if \textit{main} matches \textit{dest}.
Optionally, compilation \textit{flags} can be defined via |\def| commands.
This command line makes the \TeX{} engine believe
it is compiling the file \textit{target}
whose content is specified as the latter parameter.
The provided code then forwards the processing to
\textit{main} or \textit{dest} as described in \secref{sec:forward}.

%%%%%%%%%%%%%%%%%%%%%%%%%%%%%%%%%%%%%%%%%%%%%%%%%%%%%%%%%%%%%%%%%%%%%%%%%%%%%%%%
\subsection{Include by Input}
\label{sec:input}

Including child documents by |\include| has some restrictions by design.
Most notably, the content of a child document always occupies
its own set of pages; pages cannot be shared between child documents.
Usually, this behaviour makes perfect sense
because each child document contain an essential part of the document.
However, in some situations it may be desirable to compose
a document from a collection of parts
without having mandatory page breaks between then.
For this case, the package
provides a mechanism to include parts
by |\input| which can also be processed individually.
However, by construction this mechanism
requires manual handling of the content to be output.

%%%%%%%%%%%%%%%%%%%%%%%%%%%%%%%%%%%%%%%%
\DescribeMacro{\ifchilddocmanual}
The main file should be prepared as usual, see \secref{sec:include}.
However, the document body must make a distinction
between processing of an individual part and of the main document, e.g.:
%
\begin{center}
\begin{tabular}{l}
|\ifchilddocmanual|\\
|\input{\childdocname}|\\
|\||else|\\
\textit{document body with }|\input{|\textit{part}|}|\\
|\||fi|
\end{tabular}
\end{center}
%
The conditional |\ifchilddocmanual| is true whenever
a part to be included by |\input| is being compiled,
and the name of the part is stored in |\childdocname|.

%%%%%%%%%%%%%%%%%%%%%%%%%%%%%%%%%%%%%%%%
\DescribeMacro{\childdocby}
Each part to be included by |\input| should start with:
%
\begin{center}
\begin{tabular}{l}
|\input{childdoc.def}|\\
|\childdocby{|\textit{main}|}|\\
\end{tabular}
\end{center}
%
The directive |\childdocby| is similar to |\childdocof|
described in \secref{sec:include},
but the subsequent selection of content must be done manually.
To that end, both |\ifchilddoc| and |\ifchilddocmanual|
will be true upon processing of a part,
and the name of the part is stored in |\childdocname|.
Note that |\jobname| will be set to the filename of the current part
so that each part receives an individual |.aux| file
that does not interfere with the |.aux| file(s) of the main document.
This behaviour can be altered by the alternative form
|\childdocby[*]{|\textit{main}|}| (with a non-empty optional argument)
which uses the |.aux| file of the main document
by setting |\jobname| to \textit{main}.

%%%%%%%%%%%%%%%%%%%%%%%%%%%%%%%%%%%%%%%%%%%%%%%%%%%%%%%%%%%%%%%%%%%%%%%%%%%%%%%%
\subsection{Driver Development}
\label{sec:driver}

The \textsf{childdoc} mechanism can also be use for the development
of definition files such as \LaTeX{} styles or classes.
This case differs from the above setup with multiple parts
included by |\include| in that no |\includeonly| should be invoked.
This can be achieved by starting the include file
(before |\ProvidesPackage|) with:
%
\begin{center}
\begin{tabular}{l}
|\input{childdoc.def}|\\
|\childdocforward{|\textit{main}|}|\\
\end{tabular}
\end{center}
%
or alternatively with:
%
\begin{center}
\begin{tabular}{l}
|\input{childdoc.def}|\\
|\childdocby{|\textit{main}|}|\\
\end{tabular}
\end{center}
%
Both forms have slightly different effects as described above.
The main file is prepared as usual, see \secref{sec:include}.

%%%%%%%%%%%%%%%%%%%%%%%%%%%%%%%%%%%%%%%%%%%%%%%%%%%%%%%%%%%%%%%%%%%%%%%%%%%%%%%%
\subsection{Legacy Detection}
\label{sec:detection}

The directive |\childdocmain| in the main file can detect
whether the complete document or merely a child is to be compiled
even without using the directive |\childdocof|.
This method is deprecated because it is less robust
and there is no compelling reason to use it;
it is merely provided for backward compatibility
and it may be removed in future versions.

If the detection mechanism is to be used,
it is mandatory to correctly specify
the filename of the main file as the argument of |\childdocmain|:
%
\begin{center}
\begin{tabular}{l}
|\input{childdoc.def}|\\
|\childdocmain{|\textit{main}|}|\\
\end{tabular}
\end{center}
%
If |\jobname| does not match the argument \textit{main} of |\childdocmain|,
it is assumed that |\jobname| points to the child file to be compiled.
When using |\childdocmain| with the main file specified as argument,
it suffices to start a child file
with just |\input{|\textit{main}|}|
without loading of the package and using |\childdocof|.
If instead all processing is done
with the appropriate \textsf{childdoc} directives,
the argument of \textit{main} of |\childdocmain| can be empty.

An alternative version of the command line processing described
in \secref{sec:commandline} using the detection mechanism reads:
%
\begin{center}
|... -jobname "|\textit{target}|" "|[\textit{flags}]%
[|\def\jobname{|\textit{dest}|}|]|\input{|\textit{main}|}"|
\end{center}

%%%%%%%%%%%%%%%%%%%%%%%%%%%%%%%%%%%%%%%%%%%%%%%%%%%%%%%%%%%%%%%%%%%%%%%%%%%%%%%%
\subsection{Manual Code}
\label{sec:manual}

In case one cannot be certain whether the definitions file |childdoc.def|
is installed on the target \TeX{} distribution
and one prefers not to ship it,
it is conceivable to paste a few relevant commands into the sources.

To that end, drop all statements |\input{childdoc.def}|
and perform the replacements as outlined below.
Instead of |\childdocmain{|\textit{main}|}| add the following code
to the top of the main file:
%
\begin{center}
\begin{tabular}{l}
|\||ifdefined\childdocname\endinput\||fi\newif\ifchilddoc|\\
|\edef\childdocname{\scantokens\expandafter{\jobname\noexpand}}|\\
|\def\childdocmain{|\textit{main}|}\||ifx\childdocmain\childdocname\||else|\\
|\childdoctrue\includeonly{\childdocname}\let\jobname\childdocmain\||fi|\\
\end{tabular}
\end{center}
%
Instead of |\childdocof{|\textit{main}|}| just include the main file
at the top of each child file:
%
\begin{center}
|\input{|\textit{main}|}|
\end{center}
%
A simple redirection |\childdocforward{|\textit{dest}|}| is achieved by:
%
\begin{center}
|\def\jobname{|\textit{dest}|}\input{\jobname}|
\end{center}
%
The redirection with prefix
|\childdocforwardprefix[|\textit{prefix}|]{|\textit{dest}|}|
is accomplished by:
%
\begin{center}
\begin{tabular}{l}
|{\edef\jobname{\scantokens\expandafter{\jobname\noexpand}}|\\
|\def\redirectjob |\textit{prefix}|#1~~~{\gdef\jobname{|\textit{dest}|#1}}|\\
|\expandafter\redirectjob\jobname~~~}\input{\jobname}|
\end{tabular}
\end{center}

In an alternative approach,
child documents can be compiled by a specific command line
without additional code or specific definitions:
%
\begin{center}
|... -jobname "|\textit{target}|" "|[\textit{flags}]%
|\includeonly{|\textit{dest}|}\input{|\textit{main}|}"|
\end{center}
%

%%%%%%%%%%%%%%%%%%%%%%%%%%%%%%%%%%%%%%%%%%%%%%%%%%%%%%%%%%%%%%%%%%%%%%%%%%%%%%%%
%%%%%%%%%%%%%%%%%%%%%%%%%%%%%%%%%%%%%%%%%%%%%%%%%%%%%%%%%%%%%%%%%%%%%%%%%%%%%%%%
\section{Information}

%%%%%%%%%%%%%%%%%%%%%%%%%%%%%%%%%%%%%%%%%%%%%%%%%%%%%%%%%%%%%%%%%%%%%%%%%%%%%%%%
\subsection{Copyright}

Copyright \copyright{} 2017--2018 Niklas Beisert

This work may be distributed and/or modified under the
conditions of the \LaTeX{} Project Public License, either version 1.3
of this license or (at your option) any later version.
The latest version of this license is in
  \url{http://www.latex-project.org/lppl.txt}
and version 1.3 or later is part of all distributions of \LaTeX{}
version 2005/12/01 or later.

This work has the LPPL maintenance status `maintained'.

The Current Maintainer of this work is Niklas Beisert.

This work consists of the files |README.txt|, |childdoc.ins| and |childdoc.dtx|
as well as the derived files |childdoc.def|, |cdocsamp.tex|
with |cdocsch1.tex|, |cdocsch2.tex|, |cdocspt3.tex|, |cdocspt4.tex|,
|cdocsdrf.tex|, |cdocsfn1.tex|, |cdocsfn2.tex|
as well as |childdoc.pdf|.

%%%%%%%%%%%%%%%%%%%%%%%%%%%%%%%%%%%%%%%%%%%%%%%%%%%%%%%%%%%%%%%%%%%%%%%%%%%%%%%%
\subsection{Files and Installation}

The package consists of the files:
%
\begin{center}
\begin{tabular}{ll}
    |README.txt|   & readme file \\
    |childdoc.ins| & installation file \\
    |childdoc.dtx| & source file \\
    |childdoc.def| & definition file \\
    |cdocsamp.tex| & sample main file \\
    |cdocsch1.tex| & sample include file \\
    |cdocsch2.tex| & sample include file \\
    |cdocspt3.tex| & sample part file \\
    |cdocspt4.tex| & sample part file \\
    |cdocsdrf.tex| & sample redirection file \\
    |cdocsfn1.tex| & sample redirection file \\
    |cdocsfn2.tex| & sample redirection file \\
    |childdoc.pdf| & manual
\end{tabular}
\end{center}
%
The distribution consists of the files
|README.txt|, |childdoc.ins| and |childdoc.dtx|.
%
\begin{itemize}
\item
Run (pdf)\LaTeX{} on |childdoc.dtx|
to compile the manual |childdoc.pdf| (this file).
\item
Run \LaTeX{} on |childdoc.ins| to create the definitions file |childdoc.def|
and the sample |cdocsamp.tex| with include files
|cdocsch1.tex|, |cdocsch2.tex|, |cdocspt3.tex|, |cdocspt4.tex|,
|cdocsdrf.tex|, |cdocsfn1.tex|, |cdocsfn2.tex|.
Then copy the file |childdoc.def| to an appropriate directory of your \LaTeX{}
distribution, e.g.\ \textit{texmf-root}|/tex/latex/childdoc|.
\end{itemize}

%%%%%%%%%%%%%%%%%%%%%%%%%%%%%%%%%%%%%%%%%%%%%%%%%%%%%%%%%%%%%%%%%%%%%%%%%%%%%%%%
\subsection{Related CTAN Packages}

There are several other packages which offer a similar functionality:
%
\begin{itemize}
\item
The packages
\href{http://ctan.org/pkg/docmute}{\textsf{docmute}},
\href{http://ctan.org/pkg/includex}{\textsf{includex}} and
\href{http://ctan.org/pkg/standalone}{\textsf{standalone}}
provide commands to include only the document body of
a child file thus allowing both files to be compiled individually.
\item
The packages \href{http://ctan.org/pkg/subdocs}{\textsf{subdocs}}
and \href{http://ctan.org/pkg/subfiles}{\textsf{subfiles}}
provide structures in which the main and child documents can be
encapsulated and allowing them to be compiled individually.
The inclusion mechanism is different from the conventional |\include|.
\item
The package \href{http://ctan.org/pkg/combine}{\textsf{combine}}
is an elaborate solution to combine several documents into one.
\end{itemize}
%
See also the CTAN topic \href{http://ctan.org/topic/subdocs}{\textsf{subdocs}}
for further related packages.
The present package differs from the above solutions in that
a document structure constructed with the conventional |\include| mechanism
just needs two extra commands at the top of every file
such that all constituent files can be compiled individually.

%%%%%%%%%%%%%%%%%%%%%%%%%%%%%%%%%%%%%%%%%%%%%%%%%%%%%%%%%%%%%%%%%%%%%%%%%%%%%%%%
%\subsection{Feature Suggestions}
%
%The following is a list of features which may be useful for future
%versions of this package:
%%
%\begin{itemize}
%\item
%\ldots
%\end{itemize}

%%%%%%%%%%%%%%%%%%%%%%%%%%%%%%%%%%%%%%%%%%%%%%%%%%%%%%%%%%%%%%%%%%%%%%%%%%%%%%%%
\subsection{Revision History}

%%%%%%%%%%%%%%%%%%%%%%%%%%%%%%%%%%%%%%%%
\paragraph{v2.0:} 2018/12/30

\begin{itemize}
\item
immediate forward processing
\item
added |\childdocby| mechanism
\item
manual restructured
\end{itemize}

%%%%%%%%%%%%%%%%%%%%%%%%%%%%%%%%%%%%%%%%
\paragraph{v1.6:} 2018/01/17

\begin{itemize}
\item
application for development of include files
\item
corrections to manual
\end{itemize}

%%%%%%%%%%%%%%%%%%%%%%%%%%%%%%%%%%%%%%%%
\paragraph{v1.5:} 2017/05/21

\begin{itemize}
\item
more complete structuring introduced
\item
|\childdocof| introduced
\item
|\childdoc| renamed to |\childdocmain|
\item
|\childredirect| renamed to |\childdocforward| and |\childdocforwardprefix|
and functionality expanded
\end{itemize}

%%%%%%%%%%%%%%%%%%%%%%%%%%%%%%%%%%%%%%%%
\paragraph{v1.0:} 2017/04/27

\begin{itemize}
\item
manual and install package
\item
first version published on CTAN
\end{itemize}

%%%%%%%%%%%%%%%%%%%%%%%%%%%%%%%%%%%%%%%%
\paragraph{v0.6:} 2017/04/26

\begin{itemize}
\item
redirection mechanism added
\end{itemize}

%%%%%%%%%%%%%%%%%%%%%%%%%%%%%%%%%%%%%%%%
\paragraph{v0.5:} 2017/04/26

\begin{itemize}
\item
functionality in definition file
\end{itemize}


%%%%%%%%%%%%%%%%%%%%%%%%%%%%%%%%%%%%%%%%%%%%%%%%%%%%%%%%%%%%%%%%%%%%%%%%%%%%%%%%
%%%%%%%%%%%%%%%%%%%%%%%%%%%%%%%%%%%%%%%%%%%%%%%%%%%%%%%%%%%%%%%%%%%%%%%%%%%%%%%%
%%%%%%%%%%%%%%%%%%%%%%%%%%%%%%%%%%%%%%%%%%%%%%%%%%%%%%%%%%%%%%%%%%%%%%%%%%%%%%%%
\appendix

\settowidth\MacroIndent{\rmfamily\scriptsize 000\ }

 \DocInput{childdoc.dtx}

\end{document}
%</driver>
% \fi
%
% %%%%%%%%%%%%%%%%%%%%%%%%%%%%%%%%%%%%%%%%%%%%%%%%%%%%%%%%%%%%%%%%%%%%%%%%%%%%%%
% %%%%%%%%%%%%%%%%%%%%%%%%%%%%%%%%%%%%%%%%%%%%%%%%%%%%%%%%%%%%%%%%%%%%%%%%%%%%%%
% \section{Sample}
%\iffalse
%<*samplemain>
%\fi
%
% The following presents a sample document
% with two chapters, two parts, a title page,
% a compile flag as well as three forwarding files to set the flag.
% It consists of eight |.tex| files:
% \begin{center}
% \begin{tabular}{ll}
% |cdocsamp.tex|&main file\\
% |cdocsch1.tex|&include file for chapter 1\\
% |cdocsch2.tex|&include file for chapter 2\\
% |cdocspt3.tex|&include file for part 3\\
% |cdocspt4.tex|&include file for part 4\\
% |cdocsdrf.tex|&forwarding file for main file in draft mode\\
% |cdocsfi1.tex|&forwarding file for final version of chapter 1\\
% |cdocsfi2.tex|&forwarding file for final version of chapter 2\\
% \end{tabular}
% \end{center}
% Each of the eight files can be compiled directly by the \LaTeX{} compiler.
%
% %%%%%%%%%%%%%%%%%%%%%%%%%%%%%%%%%%%%%%
% \paragraph{Main File.}
%
% The main file is called |cdocsamp.tex|.
%
% Load the \textsf{childdoc} definitions and
% declare the filename for the main document:
%    \begin{macrocode}
\input{childdoc.def}
\childdocmain{}
%    \end{macrocode}

% Optional override for |\version| flag:
%    \begin{macrocode}
%%\ifchilddoc\else\providecommand{\version}{draft}\fi
%    \end{macrocode}

% Define the default values for the |\version| flag
% (|final| for the main file and |draft| for childs):
%    \begin{macrocode}
\ifchilddoc
\providecommand{\version}{draft}
\else
\providecommand{\version}{final}
\fi
%    \end{macrocode}

% Load the standard document class:
%    \begin{macrocode}
\documentclass[12pt]{article}
%    \end{macrocode}

% Start the document body:
%    \begin{macrocode}
\begin{document}
%    \end{macrocode}

% Declare a title page.
% Print title, part of document being processed and version flag:
%    \begin{macrocode}
\addtocounter{page}{-1}
\begin{center}
{\LARGE\bfseries{}childdoc example\par}
\vspace{1cm}
\ifchilddoc
\ifchilddocmanual part\else chapter\fi:
`\childdocname' of `\childdocjob'\par
\else
main document: `\childdocjob'\par
\fi
version: \version\par
\end{center}
\newpage
%    \end{macrocode}

% Manually include selected file,
% otherwise process as usual:
%    \begin{macrocode}
\ifchilddocmanual
\section*{part `\childdocname'}
\input{\childdocname}
\else
%    \end{macrocode}

% Include the two chapters:
%    \begin{macrocode}
\include{cdocsch1}
\include{cdocsch2}
%    \end{macrocode}

% Include the two parts unless only chapters should be displayed:
%    \begin{macrocode}
\ifchilddoc\else
\section{part three}
\input{cdocspt3}
\section{part four}
\input{cdocspt4}
\fi
%    \end{macrocode}

% Process as usual until here:
%    \begin{macrocode}
\fi
%    \end{macrocode}

% End of document body:
%    \begin{macrocode}
\end{document}
%    \end{macrocode}
%\iffalse
%</samplemain>
%\fi
%
% %%%%%%%%%%%%%%%%%%%%%%%%%%%%%%%%%%%%%%
% \paragraph{Chapter Include Files.}
%
% The include files are called |cdocsch1.tex| and |cdocsch2.tex|.
%
%\iffalse
%<*samplechap1|samplechap2>
%\fi

% Optional override for |\version| flag:
%    \begin{macrocode}
%%\providecommand{\version}{final}
%    \end{macrocode}

% Include the main document:
%    \begin{macrocode}
\input{childdoc.def}
\childdocof{cdocsamp}
%    \end{macrocode}

%\iffalse
%</samplechap1|samplechap2>
%\fi
%
%\iffalse
%<*samplechap1>
%\fi
% Some text for chapter 1:
%    \begin{macrocode}
\section{one}
some text in chapter one
%    \end{macrocode}

%\iffalse
%</samplechap1>
%\fi
% Some text for chapter 2:
%\iffalse
%<*samplechap2>
%\fi
%    \begin{macrocode}
\section{two}
more text in chapter two
%    \end{macrocode}

%\iffalse
%</samplechap2>
%\fi
%
% %%%%%%%%%%%%%%%%%%%%%%%%%%%%%%%%%%%%%%
% \paragraph{Part Include Files.}
%
% The include files are called |cdocspt3.tex| and |cdocspt4.tex|.
%
%\iffalse
%<*samplepart3|samplepart4>
%\fi

% Optional override for |\version| flag:
%    \begin{macrocode}
%%\providecommand{\version}{final}
%    \end{macrocode}

% Include the main document:
%    \begin{macrocode}
\input{childdoc.def}
\childdocby{cdocsamp}
%    \end{macrocode}

%\iffalse
%</samplepart3|samplepart4>
%\fi
%
%\iffalse
%<*samplepart3>
%\fi
% Some text for part 3:
%    \begin{macrocode}
some text in part three
%    \end{macrocode}

%\iffalse
%</samplepart3>
%\fi
% Some text for part 4:
%\iffalse
%<*samplepart4>
%\fi
%    \begin{macrocode}
more text in part four
%    \end{macrocode}

%\iffalse
%</samplepart4>
%\fi
%
% %%%%%%%%%%%%%%%%%%%%%%%%%%%%%%%%%%%%%%
% \paragraph{Forwarding for a Complete Draft.}
%
% The following forwarding file |cdocsdrf.tex|
% compiles the main document in draft mode:
%\iffalse
%<*sampledraft>
%\fi
%    \begin{macrocode}
\def\version{draft}
\input{childdoc.def}
\childdocforward{cdocsamp}
%    \end{macrocode}

%\iffalse
%</sampledraft>
%\fi
%
% %%%%%%%%%%%%%%%%%%%%%%%%%%%%%%%%%%%%%%
% \paragraph{Forwarding for Final Version of the Chapters.}
%
% The following forwarding files |cdocsfn1.tex| and |cdocsfn2.tex|
% (with identical content)
% compile the final versions of the child documents
% |cdocsch1.tex| and |cdocsch2.tex|, respectively:
%\iffalse
%<*samplefinal>
%\fi
%    \begin{macrocode}
\def\version{final}
\input{childdoc.def}
\childdocforwardprefix[cdocsamp]{cdocsfn}{cdocsch}
%    \end{macrocode}

%\iffalse
%</samplefinal>
%\fi
%
% %%%%%%%%%%%%%%%%%%%%%%%%%%%%%%%%%%%%%%
% \paragraph{Command Line Processing.}
%
% The following three command lines generate the output files
% |cdocscld|, |cdocscl1| and |cdocscl2|
% which should be identical to
% |cdocsdrf|, |cdocsch1| and |cdocsfn2|, respectively:
% \begin{center}
% \begin{tabular}{l}
% |latex -jobname cdocscld \|\\
% |  "\def\version{draft}\input{childdoc.def}\childdocforward{cdocsamp}"|\\
% |latex -jobname cdocscl1 \|\\
% |  "\input{childdoc.def}\childdocforward[cdocsamp]{cdocsch1}"|\\
% |latex -jobname cdocscl2 \|\\
% |  "\def\version{final}\input{childdoc.def}\childdocforward{cdocsch2}"|
% \end{tabular}
% \end{center}
% Note that the trailing backslash on each first line
% merely continues the input to the second line
% (for convenient cut ant paste).
% Furthermore, the command |latex| can be replaced by any
% of its alternative versions such as |pdflatex|.
%
% %%%%%%%%%%%%%%%%%%%%%%%%%%%%%%%%%%%%%%%%%%%%%%%%%%%%%%%%%%%%%%%%%%%%%%%%%%%%%%
% %%%%%%%%%%%%%%%%%%%%%%%%%%%%%%%%%%%%%%%%%%%%%%%%%%%%%%%%%%%%%%%%%%%%%%%%%%%%%%
% \section{Implementation}
%\iffalse
%<*package>
%\fi
%
% This section describes the definitions file |childdoc.def|.

% The definitions cannot be loaded using |\usepackage| or |\RequirePackage|
% which has a mechanism to prevent loading a style file more than once.
% When loading the definitions by means of |\input|
% multiple instances have to be prevented manually:
%\iffalse
%This code needs to be before the `\ProvidesFile' directive
%which is defined at the beginning of this file.
%Therefore it is also placed there and commented out here.
%</package>
%<*discard>
%\fi
%    \begin{macrocode}
\ifdefined\childdocmain\endinput\fi
%    \end{macrocode}
%\iffalse
%</discard>
%<*package>
%\fi
%
% \macro{\ifchilddoc}
% \macro{\ifchilddocmanual}
% The conditional |\ifchilddoc| tells whether a
% child (true) or main (false) document is being compiled.
% The conditional |\ifchilddocmanual| tells whether
% the |\includeonly| mechanism is used (false) or
% the selection of child files must be performed manually (true).
% The definitions initialise to false:
%    \begin{macrocode}
\newif\ifchilddoc
\newif\ifchilddocmanual
%    \end{macrocode}

% \macro{\childdocname}
% \macro{\childdocjob}
% The macro |\childdocname| stores the name of the main document
% to be compiled. The macro |\childdocjob| stores the name of
% the document on which the \LaTeX{} compiler was originally invoked.
% The content of |\jobname| cannot be compared
% to filenames specified in the source due to different catcodes.
% The following code rescans |\jobname|, stores the result
% in |\childdocname| and saves a copy in |\childdocjob|:
%    \begin{macrocode}
\edef\childdocname{\scantokens\expandafter{\jobname\noexpand}}
\let\childdocjob\childdocname
%    \end{macrocode}

% \macro{\childdocdisable}
% The macro |\childdocdisable| prevents the main file
% from being processed more than once.
% At this stage, the main document command |\childdocmain|
% is assumed to be called once again where it should do nothing.
% Any subsequent call to it should prevent
% a secondary processing of the main document
% It overwrites the forwarding commands
% |\childdocof| and |\childdocforward|
% with empty macros to prevent further inclusions of the main document:
%    \begin{macrocode}
\newcommand{\childdocdisable}
{
  \renewcommand{\childdocmain}[1]{\renewcommand{\childdocmain}[1]{\endinput}}
  \renewcommand{\childdocof}[1]{}
  \renewcommand{\childdocby}[2][]{}
  \renewcommand{\childdocforward}[2][]{}
  \renewcommand{\childdocdisable}{}
}
%    \end{macrocode}

% \macro{\childdocmain}
% The macro |\childdocmain| is to be called at the top of the main file
% with nothing or the main filename (without extension) as argument.
% First, it breaks loops.
% If the argument is not empty and does not match |\childdocname|
% (which is set by the first inclusion of |childdoc.def|),
% |\ifchilddoc| is set to true, |\includeonly| is applied to the child file
% and |\jobname| is set to the main file
% (for proper handling of |.aux| files):
%    \begin{macrocode}
\newcommand{\childdocmain}[1]
{
  \childdocdisable\childdocmain{}
  \if?#1?\else
    \begingroup
      \def\childdoctmp{#1}
      \ifx\childdoctmp\childdocname
        \def\childdoctmp{}
      \else
        \def\childdoctmp
        {
          \childdoctrue
          \includeonly{\childdocname}
          \def\childdocjob{#1}
          \def\jobname{#1}
        }
      \fi
      \expandafter
    \endgroup
    \childdoctmp
  \fi
}
%    \end{macrocode}

% \macro{\childdocof}
% The command |\childdocof| redirects
% compilation to the main file |#1|.
%    \begin{macrocode}
\newcommand{\childdocof}[1]
{
  \childdocdisable
  \childdoctrue
  \includeonly{\childdocname}
  \def\jobname{#1}
  \def\childdocjob{#1}
  \input{#1}
}
%    \end{macrocode}

% \macro{\childdocby}
% The command |\childdocby| ....
%    \begin{macrocode}
\newcommand{\childdocby}[2][]
{
  \childdocdisable
  \childdoctrue
  \childdocmanualtrue
  \if?#1?\else
    \def\jobname{#2}
  \fi
  \def\childdocjob{#2}
  \input{#2}
  \endinput
}
%    \end{macrocode}

% \macro{\childdocforward}
% The command |\childdocforward| redirects
% compilation to the main file or
% (if the optional argument is given) a child file.
% Parameters are set as if the main file
% or a child file starting with |\childdocof| was compiled.
% Then compilation is handed over to the main file:
%    \begin{macrocode}
\newcommand{\childdocforward}[2][]
{
  \begingroup
    \if?#1?
      \def\childdoctmp
      {
        \def\childdocname{#2}
        \def\childdocjob{#2}
        \def\jobname{#2}
        \input{#2}
        \endinput
      }
    \else
      \def\childdoctmp
      {
        \childdocdisable
        \def\childdocname{#2}
        \childdoctrue
        \includeonly{#2}
        \def\childdocjob{#1}
        \def\jobname{#1}
        \input{#1}
        \endinput
      }
    \fi
    \expandafter
  \endgroup
  \childdoctmp
}
%    \end{macrocode}

% \macro{\childdocforwardprefix}
% The command |\childdocforwardprefix| redirects
% compilation to the main or a child file by means of a pattern.
% The prefix |#1| in the current filename is replaced by |#2|
% and the suffix of the current filename is kept
% (it is assumed that the filename does not contain the substring `|~~~|'
% which is used as a delimiter).
% Compilation is handed over to the new file by |\childdocforward|:
%    \begin{macrocode}
\newcommand{\childdocforwardprefix}[3][]
{
  \begingroup
    \def\childdocextract #2##1~~~{\def\childdoctmp{\childdocforward[#1]{#3##1}}}
    \expandafter\childdocextract\childdocname~~~
    \expandafter
  \endgroup
  \childdoctmp
}
%    \end{macrocode}

% \macro{\childdoc}
% The deprecated macro |\childdoc| is a legacy version of |\childdocmain|:
%    \begin{macrocode}
\newcommand{\childdoc}{\childdocmain}
%    \end{macrocode}

% \macro{\childdocredirect}
% The deprecated macro |\childdocredirect| is a legacy version
% of |\childdocforward| and |\childdocforwardprefix|:
%    \begin{macrocode}
\newcommand{\childdocredirect}[2][]
{
  \begingroup
    \if?#1?
      \def\childdoctmp{\childdocforward{#2}}
    \else
      \def\childdoctmp{\childdocforwardprefix{#1}{#2}}
    \fi
    \expandafter
  \endgroup
  \childdoctmp
}
%    \end{macrocode}

%\iffalse
%</package>
%\fi
%
\endinput
\childdocforward[|\textit{main}|]{|\textit{dest}|}"|
\end{center}
%
Here \textit{target} is the name of the output file,
\textit{main} is the name of the main file
and \textit{dest} is the name of the main or child file to be processed
(all filenames without extensions).
The optional argument \textit{main} can be omitted
if \textit{main} matches \textit{dest}.
Optionally, compilation \textit{flags} can be defined via |\def| commands.
This command line makes the \TeX{} engine believe
it is compiling the file \textit{target}
whose content is specified as the latter parameter.
The provided code then forwards the processing to
\textit{main} or \textit{dest} as described in \secref{sec:forward}.

%%%%%%%%%%%%%%%%%%%%%%%%%%%%%%%%%%%%%%%%%%%%%%%%%%%%%%%%%%%%%%%%%%%%%%%%%%%%%%%%
\subsection{Include by Input}
\label{sec:input}

Including child documents by |\include| has some restrictions by design.
Most notably, the content of a child document always occupies
its own set of pages; pages cannot be shared between child documents.
Usually, this behaviour makes perfect sense
because each child document contain an essential part of the document.
However, in some situations it may be desirable to compose
a document from a collection of parts
without having mandatory page breaks between then.
For this case, the package
provides a mechanism to include parts
by |\input| which can also be processed individually.
However, by construction this mechanism
requires manual handling of the content to be output.

%%%%%%%%%%%%%%%%%%%%%%%%%%%%%%%%%%%%%%%%
\DescribeMacro{\ifchilddocmanual}
The main file should be prepared as usual, see \secref{sec:include}.
However, the document body must make a distinction
between processing of an individual part and of the main document, e.g.:
%
\begin{center}
\begin{tabular}{l}
|\ifchilddocmanual|\\
|\input{\childdocname}|\\
|\||else|\\
\textit{document body with }|\input{|\textit{part}|}|\\
|\||fi|
\end{tabular}
\end{center}
%
The conditional |\ifchilddocmanual| is true whenever
a part to be included by |\input| is being compiled,
and the name of the part is stored in |\childdocname|.

%%%%%%%%%%%%%%%%%%%%%%%%%%%%%%%%%%%%%%%%
\DescribeMacro{\childdocby}
Each part to be included by |\input| should start with:
%
\begin{center}
\begin{tabular}{l}
|% \iffalse
%
% childdoc.dtx Copyright (C) 2017-2018 Niklas Beisert
%
% This work may be distributed and/or modified under the
% conditions of the LaTeX Project Public License, either version 1.3
% of this license or (at your option) any later version.
% The latest version of this license is in
%   http://www.latex-project.org/lppl.txt
% and version 1.3 or later is part of all distributions of LaTeX
% version 2005/12/01 or later.
%
% This work has the LPPL maintenance status `maintained'.
%
% The Current Maintainer of this work is Niklas Beisert.
%
% This work consists of the files childdoc.dtx and childdoc.ins
% and the derived files childdoc.def and cdocsamp.tex with
% cdocsch1.tex, cdocsch2.tex, cdocsdrf.tex, cdocsfn1.tex, cdocsfn2.tex.
%
%<package>\ifdefined\childdocmain\endinput\fi
%<package>\ProvidesFile{childdoc.def}[2018/12/30 v2.0 child document driver]
%<samplemain>\ProvidesFile{cdocsamp.tex}[2018/12/30 v2.0 sample for childdoc]
%<*driver>
%\ProvidesFile{childdoc.drv}[2018/12/30 v2.0 childdoc reference manual file]
\PassOptionsToClass{10pt,a4paper}{article}
\documentclass{ltxdoc}

\usepackage[margin=35mm]{geometry}
\usepackage{hyperref}
\usepackage{hyperxmp}
\usepackage[usenames]{color}

\hypersetup{colorlinks=true}
\hypersetup{pdfstartview=FitH}
\hypersetup{pdfpagemode=UseNone}
\hypersetup{pdfsource={}}
\hypersetup{pdflang={en-UK}}
\hypersetup{pdfcopyright={Copyright 2017-2018 Niklas Beisert.
  This work may be distributed and/or modified under the
  conditions of the LaTeX Project Public License, either version 1.3
  of this license or (at your option) any later version.}}
\hypersetup{pdflicenseurl={http://www.latex-project.org/lppl.txt}}
\hypersetup{pdfcontactaddress={ETH Zurich, ITP, HIT K,
  Wolfgang-Pauli-Strasse 27}}
\hypersetup{pdfcontactpostcode={8093}}
\hypersetup{pdfcontactcity={Zurich}}
\hypersetup{pdfcontactcountry={Switzerland}}
\hypersetup{pdfcontactemail={nbeisert@itp.phys.ethz.ch}}
\hypersetup{pdfcontacturl={http://people.phys.ethz.ch/\xmptilde nbeisert/}}

\newcommand{\secref}[1]{\hyperref[#1]{section \ref*{#1}}}

\parskip1ex
\parindent0pt
\let\olditemize\itemize
\def\itemize{\olditemize\parskip0pt}

\begin{document}

\title{The \textsf{childdoc} Package}
\hypersetup{pdftitle={The childdoc Package}}
\author{Niklas Beisert\\[2ex]
  Institut f\"ur Theoretische Physik\\
  Eidgen\"ossische Technische Hochschule Z\"urich\\
  Wolfgang-Pauli-Strasse 27, 8093 Z\"urich, Switzerland\\[1ex]
  \href{mailto:nbeisert@itp.phys.ethz.ch}
  {\texttt{nbeisert@itp.phys.ethz.ch}}}
\hypersetup{pdfauthor={Niklas Beisert}}
\hypersetup{pdfsubject={Manual for the LaTeX2e Package childdoc}}
\date{30 December 2018, \textsf{v2.0}}
\maketitle

\begin{abstract}\noindent
\textsf{childdoc} is a \LaTeXe{} package
that enables the direct compilation
of document sections included by |\include|
to individual files.
\end{abstract}

\begingroup
\parskip0ex
\tableofcontents
\endgroup

%%%%%%%%%%%%%%%%%%%%%%%%%%%%%%%%%%%%%%%%%%%%%%%%%%%%%%%%%%%%%%%%%%%%%%%%%%%%%%%%
%%%%%%%%%%%%%%%%%%%%%%%%%%%%%%%%%%%%%%%%%%%%%%%%%%%%%%%%%%%%%%%%%%%%%%%%%%%%%%%%
\section{Introduction}

\LaTeX{} provides a mechanism to structure a large document (such as a book)
into a main file and several child files (containing the chapters)
using the |\include| command.
This mechanism is beneficial for documents
which span hundreds of pages in order to
make the source file(s) more manageable.
Moreover, compilation can be restricted to
selected child files by means of the |\includeonly| command.
The latter feature can be used to reduce the compilation time while editing
(this was significantly more useful in the earlier days of \LaTeX{})
or to generate a smaller document which is easier to navigate.
Another application of |\includeonly| is to generate
documents consisting of selected parts of the complete document.

However, there are a few drawbacks of the plain |\include| mechanism:
\begin{itemize}
\item
The child files cannot be compiled on their own,
they can only be compiled via the main file.
A naive editing environment
(such as a text editor with an option
to have the current file processed by \LaTeX)
may require one to switch to the main file before compiling;
attempting to compile the child file produces errors.
\item
The main file must be modified (each time)
to adjust the |\includeonly| command
to the present needs. This easily leaves the main file in a messy state.
\item
The generated document will always carry the filename
of the main document. This is inconvenient if
several child files are to be compiled and
to be kept for distribution.
\end{itemize}

The present package provides a simple interface
to make child files individually compilable by \LaTeX{}.
Compiling a child file then has the same effect as compiling
the main file with an |\includeonly| command
to select the appropriate child.
Moreover the generated document will carry the name of the child
rather than the main file.
This resolves all three above issues.

This feature is meant to make the editing of books,
thesis documents and lecture notes somewhat more convenient.
However, the package can also be used efficiently for
composing a series of documents (such as exercise sheets)
which are typically distributed individually.
It then assists the author in generating the individual documents
(potentially in different versions)
as well as a document containing the collected series.
Another application is in developing style files
or other kinds of included material
where compilation of the style file could redirect
to a sample or test file.

%%%%%%%%%%%%%%%%%%%%%%%%%%%%%%%%%%%%%%%%%%%%%%%%%%%%%%%%%%%%%%%%%%%%%%%%%%%%%%%%
%%%%%%%%%%%%%%%%%%%%%%%%%%%%%%%%%%%%%%%%%%%%%%%%%%%%%%%%%%%%%%%%%%%%%%%%%%%%%%%%
\section{Usage}

First of all, the package \textsf{childdoc} is \emph{not} a standard
\LaTeXe{} |.sty| style file! Therefore it needs to be invoked in
a non-standard way.

%%%%%%%%%%%%%%%%%%%%%%%%%%%%%%%%%%%%%%%%%%%%%%%%%%%%%%%%%%%%%%%%%%%%%%%%%%%%%%%%
\subsection{Included Files}
\label{sec:include}

%%%%%%%%%%%%%%%%%%%%%%%%%%%%%%%%%%%%%%%%
\DescribeMacro{\childdocmain}
To use the package, add the commands
\begin{center}
\begin{tabular}{l}
|\input{childdoc.def}|\\
|\childdocmain{}|\\
\end{tabular}
\end{center}
at the very top of the main \LaTeX{} file,
in particular \emph{before} the |\documentclass| statement!
The argument of |\childdocmain| should be left empty
(but it must be present).

%%%%%%%%%%%%%%%%%%%%%%%%%%%%%%%%%%%%%%%%
\DescribeMacro{\childdocof}
Furthermore, add the commands
\begin{center}
\begin{tabular}{l}
|\input{childdoc.def}|\\
|\childdocof{|\textit{main}|}|\\
\end{tabular}
\end{center}
at the top of every child file \textit{child}
which is included by |\include{|\textit{child}|}|
from within the main file
(or at least for those files to be compiled individually).
The argument \textit{main} must be the filename of the main file.

There are a couple of
considerations in setting up the main and child documents:

%%%%%%%%%%%%%%%%%%%%%%%%%%%%%%%%%%%%%%%%
\paragraph{Restrictions.}

Please note the following restrictions:
\begin{itemize}
\item
|\childdocmain| must be called with one argument \textit{main}
to ensure compatibility with earlier version of the package.
It must either be empty (|\childdocmain{}|)
or precisely match the filename of the main file in which it is specified.
See \secref{sec:detection} for further information.
\item
The filename \textit{main} must be specified without the |.tex| extension.
\item
The filename \textit{main} is case sensitive
(even in case-insensitive file systems)
due to internal string comparison.
\item
The argument \textit{main} should be fully expanded, it cannot be a macro.
\item
Subdirectories and special characters should be avoided in filenames.
\item
The command |\childdocmain{|\textit{main}|}| must be followed by a whitespace.
It should not be followed immediately by another command
or by a comment mark `|%|'.
This is because the \TeX{} parser reads the token immediately following
the argument of |\childdocmain| and puts it
at the beginning of every child section;
however, a white\-space is ignored.
\end{itemize}

%%%%%%%%%%%%%%%%%%%%%%%%%%%%%%%%%%%%%%%%
\paragraph{Content of Main File.}

It is advisable to place all content in the child files included by |\include|.
Any output contained in the main file will appear in all child documents
unless suppressed manually;
it cannot be suppressed automatically by the |\includeonly| directive
and thus should normally be avoided.
A method to include some content in the main file
by means of conditional processing is described in \secref{sec:conditional}.

%%%%%%%%%%%%%%%%%%%%%%%%%%%%%%%%%%%%%%%%
\paragraph{Page Numbering.}

When only a part of the document is compiled,
the appropriate numbering of pages
(as well as other status parameters)
is determined from the |.aux| files.
The latter contain information from previous passes.
However this information needs to propagate through
all intermediate child documents.
Therefore the page numbering in child documents may well
be inconsistent until the complete document is compiled at least once.

A useful (if unconventional) way to always ensure a consistent
page numbering is to restart the numbering in each child document
and denote the pages by `\textit{child}|.|\textit{page}'
where \textit{child} represents the chapter/section number of the child file.
This can be achieved by the command
|\numberwithin{page}{|\textit{child}|}|
of the \textsf{amsmath} package
where \textit{child} can be |chapter| or |section|
depending on the chosen structuring.
Alternatively, one can modify the macro |\thepage| appropriately
and reset the counter |page| at the start of each child file.

%%%%%%%%%%%%%%%%%%%%%%%%%%%%%%%%%%%%%%%%%%%%%%%%%%%%%%%%%%%%%%%%%%%%%%%%%%%%%%%%
\subsection{Conditional Processing}
\label{sec:conditional}

The package provides a mechanism to compile different versions
of a document. To customise the versions further some conditional processing
can come in handy to distinguish which version is being compiled.
The package provides two macros to describe the compilation context:

%%%%%%%%%%%%%%%%%%%%%%%%%%%%%%%%%%%%%%%%
\DescribeMacro{\ifchilddoc}
The conditional |\ifchilddoc| distinguishes between the compilation of
child documents and the main document:
%
\begin{center}
|\ifchilddoc |\textit{child-code}| |[|\||else |\textit{main-code}]| \||fi|
\end{center}

%%%%%%%%%%%%%%%%%%%%%%%%%%%%%%%%%%%%%%%%
\DescribeMacro{\childdocname}
\DescribeMacro{\childdocjob}
The macro |\childdocname| contains the filename (without extension)
of the main or child file being processed.
Note that |\childdocjob| will always contain the name of the main file.

%%%%%%%%%%%%%%%%%%%%%%%%%%%%%%%%%%%%%%%%
\paragraph{Title Page.}

Conditional processing can be used to include a title or banner page
in the main document when proper precautions are taken.
Importantly, the code in the main file should ensure that the page counter
(as well as other status parameters which are stored in the |.aux| files)
takes the same value after the conditional processing.
Otherwise the page numbers may take divergent values
depending on which part is compiled.

For example, a title page could be declared by:
%
\begin{center}
\begin{tabular}{l}
|\ifchilddoc\||else|\\
|\addtocounter{page}{-1}|\\
\textit{code for title page}\\
|\newpage|\\
|\||fi|
\end{tabular}
\end{center}
%
A banner page for the child documents can be generated by:
%
\begin{center}
\begin{tabular}{l}
|\ifchilddoc|\\
|\addtocounter{page}{-1}|\\
\textit{code for banner page}\\
|\newpage|\\
|\||fi|
\end{tabular}
\end{center}
%
Here one could write a message such as:
\begin{center}
|This is the part \childdocname{} of \childdocjob{}.|
\end{center}

%%%%%%%%%%%%%%%%%%%%%%%%%%%%%%%%%%%%%%%%%%%%%%%%%%%%%%%%%%%%%%%%%%%%%%%%%%%%%%%%
\subsection{Flags}
\label{sec:flags}

The package makes it easy to generate different versions
of the main or child documents.
To this end compilation flags can be defined
and assigned different default values.
They will be particularly useful in conjunction
with the forwarding mechanism described in \secref{sec:forward}.

For example, it may be useful to have a flag |\version|
which can be set to |draft| or |final|.
The document source will contain some conditional code
depending on the value of |\version|.
Suppose further, the flag should default to |final| for the main file
and to |draft| for child files
which is a natural assignment for editing the document.
This is achieved by placing the following code
in the preamble of the main document
(below the |\childdocmain| directive):
%
\begin{center}
\begin{tabular}{l}
|\ifchilddoc|\\
|\providecommand{\version}{draft}|\\
|\||else|\\
|\providecommand{\version}{final}|\\
|\||fi|
\end{tabular}
\end{center}
%
The definition by |\providecommand| makes sure
that previous definitions are not overwritten.
Further statements |\providecommand{\version}{...}|
can thus be added before the above code to override it.

For the main file, one might add a line
(between |\childdocmain| and the above block)
%
\begin{center}
|%\ifchilddoc\||else\providecommand{\version}{draft}\||fi|
\end{center}
%
which can be uncommented to produce a draft version.
Likewise one can add a line to the very top of a child file
(above the |\childdocof{|\textit{main}|}| directive)
%
\begin{center}
|%\providecommand{\version}{final}|
\end{center}
%
which can be uncommented to produce the final version of this child document.

%%%%%%%%%%%%%%%%%%%%%%%%%%%%%%%%%%%%%%%%%%%%%%%%%%%%%%%%%%%%%%%%%%%%%%%%%%%%%%%%
\subsection{Forwarding}
\label{sec:forward}

Different versions of the main or child documents
using compilation flags as described in \secref{sec:flags}
can be (permanently) stored in different files
for convenient compilation, viewing and distribution.
To this end, the package defines a command
to pass on compilation to a different file:

%%%%%%%%%%%%%%%%%%%%%%%%%%%%%%%%%%%%%%%%
\DescribeMacro{\childdocforward}
The command |\childdocforward| redirects processing to
another source file:
%
\begin{center}
\begin{tabular}{l}
|\input{childdoc.def}|\\
|\childdocforward[|\textit{main}|]{|\textit{dest}|}|\\
\end{tabular}
\end{center}
%
The argument \textit{dest} is the destination file
(without extension).
It should be the main file or one of the child files.
Note that further \textsf{childdoc} directives
such as |\childdocof| and |\childdocforward|
in the indicated file will be processed in this form.
The optional argument \textit{main}
passes on directly to the main file \textit{main}
while pretending to compile the child \textit{dest}.
This form behaves as if \textit{dest}
issues |\childdocof{|\textit{main}|}| right away,
and no further \textsf{childdoc} directives will be processed.

%%%%%%%%%%%%%%%%%%%%%%%%%%%%%%%%%%%%%%%%
\DescribeMacro{\...prefix}
In the alternative form |\childdocforwardprefix|,
%
\begin{center}
\begin{tabular}{l}
|\input{childdoc.def}|\\
|\childdocforwardprefix[|\textit{main}|]{|\textit{prefix}|}{|\textit{dest}|}|
\end{tabular}
\end{center}
%
the destination file is determined by a pattern
depending on the current file:
To make this work, the current file must be called
`{\textit{prefix}\hspace{0.2em}\textit{suffix}}'
with \textit{prefix} matching precisely the argument.
Processing is then passed on to the file
`{\textit{dest}\hspace{0.2em}\textit{suffix}}'.
Surely, the same effect is achieved by
directly specifying the
argument `{\textit{dest}\hspace{0.2em}\textit{suffix}}'
in the first form.
However, that requires to set up a different file
for each child. With the alternative form of the command
all these files can have exactly the same content
which simplifies setting them up and maintaining them.

For example, the following file |draft.tex|
with a compilation flag |\version| as described in \secref{sec:flags}
compiles the main document as a draft:
%
\begin{center}
\begin{tabular}{l}
|\def\version{draft}|\\
|\input{childdoc.def}|\\
|\childdocforward{|\textit{main}|}|
\end{tabular}
\end{center}
%
Likewise, the following files |final|\textit{nn}|.tex|
compile the final version of the child document
|child|\textit{nn}|.tex|:
%
\begin{center}
\begin{tabular}{l}
|\def\version{final}|\\
|\input{childdoc.def}|\\
|\childdocforwardprefix{final}{child}|
\end{tabular}
\end{center}
%

Note that when several versions of a main file and/or of each child file
are to be generated, it may be convenient to set up a |Makefile| or
shell script to automatise the process.

%%%%%%%%%%%%%%%%%%%%%%%%%%%%%%%%%%%%%%%%%%%%%%%%%%%%%%%%%%%%%%%%%%%%%%%%%%%%%%%%
\subsection{Command Line Processing}
\label{sec:commandline}

The effect of redirection files can also be achieved by invoking
the \LaTeX{} compiler with a more elaborate command line.
Most conveniently this should be done as part
of a shell script or a |Makefile|.

When using \textsf{childdoc} in the main file, the following
command lines effectively perform a redirection
(note that depending on the shell being used,
backslashes may have to be doubled: `|\|' $\to$ `|\\|'):
%
\begin{center}
|... -jobname "|\textit{target}|" |\\|"|[\textit{flags}]%
|\input{childdoc.def}\childdocforward[|\textit{main}|]{|\textit{dest}|}"|
\end{center}
%
Here \textit{target} is the name of the output file,
\textit{main} is the name of the main file
and \textit{dest} is the name of the main or child file to be processed
(all filenames without extensions).
The optional argument \textit{main} can be omitted
if \textit{main} matches \textit{dest}.
Optionally, compilation \textit{flags} can be defined via |\def| commands.
This command line makes the \TeX{} engine believe
it is compiling the file \textit{target}
whose content is specified as the latter parameter.
The provided code then forwards the processing to
\textit{main} or \textit{dest} as described in \secref{sec:forward}.

%%%%%%%%%%%%%%%%%%%%%%%%%%%%%%%%%%%%%%%%%%%%%%%%%%%%%%%%%%%%%%%%%%%%%%%%%%%%%%%%
\subsection{Include by Input}
\label{sec:input}

Including child documents by |\include| has some restrictions by design.
Most notably, the content of a child document always occupies
its own set of pages; pages cannot be shared between child documents.
Usually, this behaviour makes perfect sense
because each child document contain an essential part of the document.
However, in some situations it may be desirable to compose
a document from a collection of parts
without having mandatory page breaks between then.
For this case, the package
provides a mechanism to include parts
by |\input| which can also be processed individually.
However, by construction this mechanism
requires manual handling of the content to be output.

%%%%%%%%%%%%%%%%%%%%%%%%%%%%%%%%%%%%%%%%
\DescribeMacro{\ifchilddocmanual}
The main file should be prepared as usual, see \secref{sec:include}.
However, the document body must make a distinction
between processing of an individual part and of the main document, e.g.:
%
\begin{center}
\begin{tabular}{l}
|\ifchilddocmanual|\\
|\input{\childdocname}|\\
|\||else|\\
\textit{document body with }|\input{|\textit{part}|}|\\
|\||fi|
\end{tabular}
\end{center}
%
The conditional |\ifchilddocmanual| is true whenever
a part to be included by |\input| is being compiled,
and the name of the part is stored in |\childdocname|.

%%%%%%%%%%%%%%%%%%%%%%%%%%%%%%%%%%%%%%%%
\DescribeMacro{\childdocby}
Each part to be included by |\input| should start with:
%
\begin{center}
\begin{tabular}{l}
|\input{childdoc.def}|\\
|\childdocby{|\textit{main}|}|\\
\end{tabular}
\end{center}
%
The directive |\childdocby| is similar to |\childdocof|
described in \secref{sec:include},
but the subsequent selection of content must be done manually.
To that end, both |\ifchilddoc| and |\ifchilddocmanual|
will be true upon processing of a part,
and the name of the part is stored in |\childdocname|.
Note that |\jobname| will be set to the filename of the current part
so that each part receives an individual |.aux| file
that does not interfere with the |.aux| file(s) of the main document.
This behaviour can be altered by the alternative form
|\childdocby[*]{|\textit{main}|}| (with a non-empty optional argument)
which uses the |.aux| file of the main document
by setting |\jobname| to \textit{main}.

%%%%%%%%%%%%%%%%%%%%%%%%%%%%%%%%%%%%%%%%%%%%%%%%%%%%%%%%%%%%%%%%%%%%%%%%%%%%%%%%
\subsection{Driver Development}
\label{sec:driver}

The \textsf{childdoc} mechanism can also be use for the development
of definition files such as \LaTeX{} styles or classes.
This case differs from the above setup with multiple parts
included by |\include| in that no |\includeonly| should be invoked.
This can be achieved by starting the include file
(before |\ProvidesPackage|) with:
%
\begin{center}
\begin{tabular}{l}
|\input{childdoc.def}|\\
|\childdocforward{|\textit{main}|}|\\
\end{tabular}
\end{center}
%
or alternatively with:
%
\begin{center}
\begin{tabular}{l}
|\input{childdoc.def}|\\
|\childdocby{|\textit{main}|}|\\
\end{tabular}
\end{center}
%
Both forms have slightly different effects as described above.
The main file is prepared as usual, see \secref{sec:include}.

%%%%%%%%%%%%%%%%%%%%%%%%%%%%%%%%%%%%%%%%%%%%%%%%%%%%%%%%%%%%%%%%%%%%%%%%%%%%%%%%
\subsection{Legacy Detection}
\label{sec:detection}

The directive |\childdocmain| in the main file can detect
whether the complete document or merely a child is to be compiled
even without using the directive |\childdocof|.
This method is deprecated because it is less robust
and there is no compelling reason to use it;
it is merely provided for backward compatibility
and it may be removed in future versions.

If the detection mechanism is to be used,
it is mandatory to correctly specify
the filename of the main file as the argument of |\childdocmain|:
%
\begin{center}
\begin{tabular}{l}
|\input{childdoc.def}|\\
|\childdocmain{|\textit{main}|}|\\
\end{tabular}
\end{center}
%
If |\jobname| does not match the argument \textit{main} of |\childdocmain|,
it is assumed that |\jobname| points to the child file to be compiled.
When using |\childdocmain| with the main file specified as argument,
it suffices to start a child file
with just |\input{|\textit{main}|}|
without loading of the package and using |\childdocof|.
If instead all processing is done
with the appropriate \textsf{childdoc} directives,
the argument of \textit{main} of |\childdocmain| can be empty.

An alternative version of the command line processing described
in \secref{sec:commandline} using the detection mechanism reads:
%
\begin{center}
|... -jobname "|\textit{target}|" "|[\textit{flags}]%
[|\def\jobname{|\textit{dest}|}|]|\input{|\textit{main}|}"|
\end{center}

%%%%%%%%%%%%%%%%%%%%%%%%%%%%%%%%%%%%%%%%%%%%%%%%%%%%%%%%%%%%%%%%%%%%%%%%%%%%%%%%
\subsection{Manual Code}
\label{sec:manual}

In case one cannot be certain whether the definitions file |childdoc.def|
is installed on the target \TeX{} distribution
and one prefers not to ship it,
it is conceivable to paste a few relevant commands into the sources.

To that end, drop all statements |\input{childdoc.def}|
and perform the replacements as outlined below.
Instead of |\childdocmain{|\textit{main}|}| add the following code
to the top of the main file:
%
\begin{center}
\begin{tabular}{l}
|\||ifdefined\childdocname\endinput\||fi\newif\ifchilddoc|\\
|\edef\childdocname{\scantokens\expandafter{\jobname\noexpand}}|\\
|\def\childdocmain{|\textit{main}|}\||ifx\childdocmain\childdocname\||else|\\
|\childdoctrue\includeonly{\childdocname}\let\jobname\childdocmain\||fi|\\
\end{tabular}
\end{center}
%
Instead of |\childdocof{|\textit{main}|}| just include the main file
at the top of each child file:
%
\begin{center}
|\input{|\textit{main}|}|
\end{center}
%
A simple redirection |\childdocforward{|\textit{dest}|}| is achieved by:
%
\begin{center}
|\def\jobname{|\textit{dest}|}\input{\jobname}|
\end{center}
%
The redirection with prefix
|\childdocforwardprefix[|\textit{prefix}|]{|\textit{dest}|}|
is accomplished by:
%
\begin{center}
\begin{tabular}{l}
|{\edef\jobname{\scantokens\expandafter{\jobname\noexpand}}|\\
|\def\redirectjob |\textit{prefix}|#1~~~{\gdef\jobname{|\textit{dest}|#1}}|\\
|\expandafter\redirectjob\jobname~~~}\input{\jobname}|
\end{tabular}
\end{center}

In an alternative approach,
child documents can be compiled by a specific command line
without additional code or specific definitions:
%
\begin{center}
|... -jobname "|\textit{target}|" "|[\textit{flags}]%
|\includeonly{|\textit{dest}|}\input{|\textit{main}|}"|
\end{center}
%

%%%%%%%%%%%%%%%%%%%%%%%%%%%%%%%%%%%%%%%%%%%%%%%%%%%%%%%%%%%%%%%%%%%%%%%%%%%%%%%%
%%%%%%%%%%%%%%%%%%%%%%%%%%%%%%%%%%%%%%%%%%%%%%%%%%%%%%%%%%%%%%%%%%%%%%%%%%%%%%%%
\section{Information}

%%%%%%%%%%%%%%%%%%%%%%%%%%%%%%%%%%%%%%%%%%%%%%%%%%%%%%%%%%%%%%%%%%%%%%%%%%%%%%%%
\subsection{Copyright}

Copyright \copyright{} 2017--2018 Niklas Beisert

This work may be distributed and/or modified under the
conditions of the \LaTeX{} Project Public License, either version 1.3
of this license or (at your option) any later version.
The latest version of this license is in
  \url{http://www.latex-project.org/lppl.txt}
and version 1.3 or later is part of all distributions of \LaTeX{}
version 2005/12/01 or later.

This work has the LPPL maintenance status `maintained'.

The Current Maintainer of this work is Niklas Beisert.

This work consists of the files |README.txt|, |childdoc.ins| and |childdoc.dtx|
as well as the derived files |childdoc.def|, |cdocsamp.tex|
with |cdocsch1.tex|, |cdocsch2.tex|, |cdocspt3.tex|, |cdocspt4.tex|,
|cdocsdrf.tex|, |cdocsfn1.tex|, |cdocsfn2.tex|
as well as |childdoc.pdf|.

%%%%%%%%%%%%%%%%%%%%%%%%%%%%%%%%%%%%%%%%%%%%%%%%%%%%%%%%%%%%%%%%%%%%%%%%%%%%%%%%
\subsection{Files and Installation}

The package consists of the files:
%
\begin{center}
\begin{tabular}{ll}
    |README.txt|   & readme file \\
    |childdoc.ins| & installation file \\
    |childdoc.dtx| & source file \\
    |childdoc.def| & definition file \\
    |cdocsamp.tex| & sample main file \\
    |cdocsch1.tex| & sample include file \\
    |cdocsch2.tex| & sample include file \\
    |cdocspt3.tex| & sample part file \\
    |cdocspt4.tex| & sample part file \\
    |cdocsdrf.tex| & sample redirection file \\
    |cdocsfn1.tex| & sample redirection file \\
    |cdocsfn2.tex| & sample redirection file \\
    |childdoc.pdf| & manual
\end{tabular}
\end{center}
%
The distribution consists of the files
|README.txt|, |childdoc.ins| and |childdoc.dtx|.
%
\begin{itemize}
\item
Run (pdf)\LaTeX{} on |childdoc.dtx|
to compile the manual |childdoc.pdf| (this file).
\item
Run \LaTeX{} on |childdoc.ins| to create the definitions file |childdoc.def|
and the sample |cdocsamp.tex| with include files
|cdocsch1.tex|, |cdocsch2.tex|, |cdocspt3.tex|, |cdocspt4.tex|,
|cdocsdrf.tex|, |cdocsfn1.tex|, |cdocsfn2.tex|.
Then copy the file |childdoc.def| to an appropriate directory of your \LaTeX{}
distribution, e.g.\ \textit{texmf-root}|/tex/latex/childdoc|.
\end{itemize}

%%%%%%%%%%%%%%%%%%%%%%%%%%%%%%%%%%%%%%%%%%%%%%%%%%%%%%%%%%%%%%%%%%%%%%%%%%%%%%%%
\subsection{Related CTAN Packages}

There are several other packages which offer a similar functionality:
%
\begin{itemize}
\item
The packages
\href{http://ctan.org/pkg/docmute}{\textsf{docmute}},
\href{http://ctan.org/pkg/includex}{\textsf{includex}} and
\href{http://ctan.org/pkg/standalone}{\textsf{standalone}}
provide commands to include only the document body of
a child file thus allowing both files to be compiled individually.
\item
The packages \href{http://ctan.org/pkg/subdocs}{\textsf{subdocs}}
and \href{http://ctan.org/pkg/subfiles}{\textsf{subfiles}}
provide structures in which the main and child documents can be
encapsulated and allowing them to be compiled individually.
The inclusion mechanism is different from the conventional |\include|.
\item
The package \href{http://ctan.org/pkg/combine}{\textsf{combine}}
is an elaborate solution to combine several documents into one.
\end{itemize}
%
See also the CTAN topic \href{http://ctan.org/topic/subdocs}{\textsf{subdocs}}
for further related packages.
The present package differs from the above solutions in that
a document structure constructed with the conventional |\include| mechanism
just needs two extra commands at the top of every file
such that all constituent files can be compiled individually.

%%%%%%%%%%%%%%%%%%%%%%%%%%%%%%%%%%%%%%%%%%%%%%%%%%%%%%%%%%%%%%%%%%%%%%%%%%%%%%%%
%\subsection{Feature Suggestions}
%
%The following is a list of features which may be useful for future
%versions of this package:
%%
%\begin{itemize}
%\item
%\ldots
%\end{itemize}

%%%%%%%%%%%%%%%%%%%%%%%%%%%%%%%%%%%%%%%%%%%%%%%%%%%%%%%%%%%%%%%%%%%%%%%%%%%%%%%%
\subsection{Revision History}

%%%%%%%%%%%%%%%%%%%%%%%%%%%%%%%%%%%%%%%%
\paragraph{v2.0:} 2018/12/30

\begin{itemize}
\item
immediate forward processing
\item
added |\childdocby| mechanism
\item
manual restructured
\end{itemize}

%%%%%%%%%%%%%%%%%%%%%%%%%%%%%%%%%%%%%%%%
\paragraph{v1.6:} 2018/01/17

\begin{itemize}
\item
application for development of include files
\item
corrections to manual
\end{itemize}

%%%%%%%%%%%%%%%%%%%%%%%%%%%%%%%%%%%%%%%%
\paragraph{v1.5:} 2017/05/21

\begin{itemize}
\item
more complete structuring introduced
\item
|\childdocof| introduced
\item
|\childdoc| renamed to |\childdocmain|
\item
|\childredirect| renamed to |\childdocforward| and |\childdocforwardprefix|
and functionality expanded
\end{itemize}

%%%%%%%%%%%%%%%%%%%%%%%%%%%%%%%%%%%%%%%%
\paragraph{v1.0:} 2017/04/27

\begin{itemize}
\item
manual and install package
\item
first version published on CTAN
\end{itemize}

%%%%%%%%%%%%%%%%%%%%%%%%%%%%%%%%%%%%%%%%
\paragraph{v0.6:} 2017/04/26

\begin{itemize}
\item
redirection mechanism added
\end{itemize}

%%%%%%%%%%%%%%%%%%%%%%%%%%%%%%%%%%%%%%%%
\paragraph{v0.5:} 2017/04/26

\begin{itemize}
\item
functionality in definition file
\end{itemize}


%%%%%%%%%%%%%%%%%%%%%%%%%%%%%%%%%%%%%%%%%%%%%%%%%%%%%%%%%%%%%%%%%%%%%%%%%%%%%%%%
%%%%%%%%%%%%%%%%%%%%%%%%%%%%%%%%%%%%%%%%%%%%%%%%%%%%%%%%%%%%%%%%%%%%%%%%%%%%%%%%
%%%%%%%%%%%%%%%%%%%%%%%%%%%%%%%%%%%%%%%%%%%%%%%%%%%%%%%%%%%%%%%%%%%%%%%%%%%%%%%%
\appendix

\settowidth\MacroIndent{\rmfamily\scriptsize 000\ }

 \DocInput{childdoc.dtx}

\end{document}
%</driver>
% \fi
%
% %%%%%%%%%%%%%%%%%%%%%%%%%%%%%%%%%%%%%%%%%%%%%%%%%%%%%%%%%%%%%%%%%%%%%%%%%%%%%%
% %%%%%%%%%%%%%%%%%%%%%%%%%%%%%%%%%%%%%%%%%%%%%%%%%%%%%%%%%%%%%%%%%%%%%%%%%%%%%%
% \section{Sample}
%\iffalse
%<*samplemain>
%\fi
%
% The following presents a sample document
% with two chapters, two parts, a title page,
% a compile flag as well as three forwarding files to set the flag.
% It consists of eight |.tex| files:
% \begin{center}
% \begin{tabular}{ll}
% |cdocsamp.tex|&main file\\
% |cdocsch1.tex|&include file for chapter 1\\
% |cdocsch2.tex|&include file for chapter 2\\
% |cdocspt3.tex|&include file for part 3\\
% |cdocspt4.tex|&include file for part 4\\
% |cdocsdrf.tex|&forwarding file for main file in draft mode\\
% |cdocsfi1.tex|&forwarding file for final version of chapter 1\\
% |cdocsfi2.tex|&forwarding file for final version of chapter 2\\
% \end{tabular}
% \end{center}
% Each of the eight files can be compiled directly by the \LaTeX{} compiler.
%
% %%%%%%%%%%%%%%%%%%%%%%%%%%%%%%%%%%%%%%
% \paragraph{Main File.}
%
% The main file is called |cdocsamp.tex|.
%
% Load the \textsf{childdoc} definitions and
% declare the filename for the main document:
%    \begin{macrocode}
\input{childdoc.def}
\childdocmain{}
%    \end{macrocode}

% Optional override for |\version| flag:
%    \begin{macrocode}
%%\ifchilddoc\else\providecommand{\version}{draft}\fi
%    \end{macrocode}

% Define the default values for the |\version| flag
% (|final| for the main file and |draft| for childs):
%    \begin{macrocode}
\ifchilddoc
\providecommand{\version}{draft}
\else
\providecommand{\version}{final}
\fi
%    \end{macrocode}

% Load the standard document class:
%    \begin{macrocode}
\documentclass[12pt]{article}
%    \end{macrocode}

% Start the document body:
%    \begin{macrocode}
\begin{document}
%    \end{macrocode}

% Declare a title page.
% Print title, part of document being processed and version flag:
%    \begin{macrocode}
\addtocounter{page}{-1}
\begin{center}
{\LARGE\bfseries{}childdoc example\par}
\vspace{1cm}
\ifchilddoc
\ifchilddocmanual part\else chapter\fi:
`\childdocname' of `\childdocjob'\par
\else
main document: `\childdocjob'\par
\fi
version: \version\par
\end{center}
\newpage
%    \end{macrocode}

% Manually include selected file,
% otherwise process as usual:
%    \begin{macrocode}
\ifchilddocmanual
\section*{part `\childdocname'}
\input{\childdocname}
\else
%    \end{macrocode}

% Include the two chapters:
%    \begin{macrocode}
\include{cdocsch1}
\include{cdocsch2}
%    \end{macrocode}

% Include the two parts unless only chapters should be displayed:
%    \begin{macrocode}
\ifchilddoc\else
\section{part three}
\input{cdocspt3}
\section{part four}
\input{cdocspt4}
\fi
%    \end{macrocode}

% Process as usual until here:
%    \begin{macrocode}
\fi
%    \end{macrocode}

% End of document body:
%    \begin{macrocode}
\end{document}
%    \end{macrocode}
%\iffalse
%</samplemain>
%\fi
%
% %%%%%%%%%%%%%%%%%%%%%%%%%%%%%%%%%%%%%%
% \paragraph{Chapter Include Files.}
%
% The include files are called |cdocsch1.tex| and |cdocsch2.tex|.
%
%\iffalse
%<*samplechap1|samplechap2>
%\fi

% Optional override for |\version| flag:
%    \begin{macrocode}
%%\providecommand{\version}{final}
%    \end{macrocode}

% Include the main document:
%    \begin{macrocode}
\input{childdoc.def}
\childdocof{cdocsamp}
%    \end{macrocode}

%\iffalse
%</samplechap1|samplechap2>
%\fi
%
%\iffalse
%<*samplechap1>
%\fi
% Some text for chapter 1:
%    \begin{macrocode}
\section{one}
some text in chapter one
%    \end{macrocode}

%\iffalse
%</samplechap1>
%\fi
% Some text for chapter 2:
%\iffalse
%<*samplechap2>
%\fi
%    \begin{macrocode}
\section{two}
more text in chapter two
%    \end{macrocode}

%\iffalse
%</samplechap2>
%\fi
%
% %%%%%%%%%%%%%%%%%%%%%%%%%%%%%%%%%%%%%%
% \paragraph{Part Include Files.}
%
% The include files are called |cdocspt3.tex| and |cdocspt4.tex|.
%
%\iffalse
%<*samplepart3|samplepart4>
%\fi

% Optional override for |\version| flag:
%    \begin{macrocode}
%%\providecommand{\version}{final}
%    \end{macrocode}

% Include the main document:
%    \begin{macrocode}
\input{childdoc.def}
\childdocby{cdocsamp}
%    \end{macrocode}

%\iffalse
%</samplepart3|samplepart4>
%\fi
%
%\iffalse
%<*samplepart3>
%\fi
% Some text for part 3:
%    \begin{macrocode}
some text in part three
%    \end{macrocode}

%\iffalse
%</samplepart3>
%\fi
% Some text for part 4:
%\iffalse
%<*samplepart4>
%\fi
%    \begin{macrocode}
more text in part four
%    \end{macrocode}

%\iffalse
%</samplepart4>
%\fi
%
% %%%%%%%%%%%%%%%%%%%%%%%%%%%%%%%%%%%%%%
% \paragraph{Forwarding for a Complete Draft.}
%
% The following forwarding file |cdocsdrf.tex|
% compiles the main document in draft mode:
%\iffalse
%<*sampledraft>
%\fi
%    \begin{macrocode}
\def\version{draft}
\input{childdoc.def}
\childdocforward{cdocsamp}
%    \end{macrocode}

%\iffalse
%</sampledraft>
%\fi
%
% %%%%%%%%%%%%%%%%%%%%%%%%%%%%%%%%%%%%%%
% \paragraph{Forwarding for Final Version of the Chapters.}
%
% The following forwarding files |cdocsfn1.tex| and |cdocsfn2.tex|
% (with identical content)
% compile the final versions of the child documents
% |cdocsch1.tex| and |cdocsch2.tex|, respectively:
%\iffalse
%<*samplefinal>
%\fi
%    \begin{macrocode}
\def\version{final}
\input{childdoc.def}
\childdocforwardprefix[cdocsamp]{cdocsfn}{cdocsch}
%    \end{macrocode}

%\iffalse
%</samplefinal>
%\fi
%
% %%%%%%%%%%%%%%%%%%%%%%%%%%%%%%%%%%%%%%
% \paragraph{Command Line Processing.}
%
% The following three command lines generate the output files
% |cdocscld|, |cdocscl1| and |cdocscl2|
% which should be identical to
% |cdocsdrf|, |cdocsch1| and |cdocsfn2|, respectively:
% \begin{center}
% \begin{tabular}{l}
% |latex -jobname cdocscld \|\\
% |  "\def\version{draft}\input{childdoc.def}\childdocforward{cdocsamp}"|\\
% |latex -jobname cdocscl1 \|\\
% |  "\input{childdoc.def}\childdocforward[cdocsamp]{cdocsch1}"|\\
% |latex -jobname cdocscl2 \|\\
% |  "\def\version{final}\input{childdoc.def}\childdocforward{cdocsch2}"|
% \end{tabular}
% \end{center}
% Note that the trailing backslash on each first line
% merely continues the input to the second line
% (for convenient cut ant paste).
% Furthermore, the command |latex| can be replaced by any
% of its alternative versions such as |pdflatex|.
%
% %%%%%%%%%%%%%%%%%%%%%%%%%%%%%%%%%%%%%%%%%%%%%%%%%%%%%%%%%%%%%%%%%%%%%%%%%%%%%%
% %%%%%%%%%%%%%%%%%%%%%%%%%%%%%%%%%%%%%%%%%%%%%%%%%%%%%%%%%%%%%%%%%%%%%%%%%%%%%%
% \section{Implementation}
%\iffalse
%<*package>
%\fi
%
% This section describes the definitions file |childdoc.def|.

% The definitions cannot be loaded using |\usepackage| or |\RequirePackage|
% which has a mechanism to prevent loading a style file more than once.
% When loading the definitions by means of |\input|
% multiple instances have to be prevented manually:
%\iffalse
%This code needs to be before the `\ProvidesFile' directive
%which is defined at the beginning of this file.
%Therefore it is also placed there and commented out here.
%</package>
%<*discard>
%\fi
%    \begin{macrocode}
\ifdefined\childdocmain\endinput\fi
%    \end{macrocode}
%\iffalse
%</discard>
%<*package>
%\fi
%
% \macro{\ifchilddoc}
% \macro{\ifchilddocmanual}
% The conditional |\ifchilddoc| tells whether a
% child (true) or main (false) document is being compiled.
% The conditional |\ifchilddocmanual| tells whether
% the |\includeonly| mechanism is used (false) or
% the selection of child files must be performed manually (true).
% The definitions initialise to false:
%    \begin{macrocode}
\newif\ifchilddoc
\newif\ifchilddocmanual
%    \end{macrocode}

% \macro{\childdocname}
% \macro{\childdocjob}
% The macro |\childdocname| stores the name of the main document
% to be compiled. The macro |\childdocjob| stores the name of
% the document on which the \LaTeX{} compiler was originally invoked.
% The content of |\jobname| cannot be compared
% to filenames specified in the source due to different catcodes.
% The following code rescans |\jobname|, stores the result
% in |\childdocname| and saves a copy in |\childdocjob|:
%    \begin{macrocode}
\edef\childdocname{\scantokens\expandafter{\jobname\noexpand}}
\let\childdocjob\childdocname
%    \end{macrocode}

% \macro{\childdocdisable}
% The macro |\childdocdisable| prevents the main file
% from being processed more than once.
% At this stage, the main document command |\childdocmain|
% is assumed to be called once again where it should do nothing.
% Any subsequent call to it should prevent
% a secondary processing of the main document
% It overwrites the forwarding commands
% |\childdocof| and |\childdocforward|
% with empty macros to prevent further inclusions of the main document:
%    \begin{macrocode}
\newcommand{\childdocdisable}
{
  \renewcommand{\childdocmain}[1]{\renewcommand{\childdocmain}[1]{\endinput}}
  \renewcommand{\childdocof}[1]{}
  \renewcommand{\childdocby}[2][]{}
  \renewcommand{\childdocforward}[2][]{}
  \renewcommand{\childdocdisable}{}
}
%    \end{macrocode}

% \macro{\childdocmain}
% The macro |\childdocmain| is to be called at the top of the main file
% with nothing or the main filename (without extension) as argument.
% First, it breaks loops.
% If the argument is not empty and does not match |\childdocname|
% (which is set by the first inclusion of |childdoc.def|),
% |\ifchilddoc| is set to true, |\includeonly| is applied to the child file
% and |\jobname| is set to the main file
% (for proper handling of |.aux| files):
%    \begin{macrocode}
\newcommand{\childdocmain}[1]
{
  \childdocdisable\childdocmain{}
  \if?#1?\else
    \begingroup
      \def\childdoctmp{#1}
      \ifx\childdoctmp\childdocname
        \def\childdoctmp{}
      \else
        \def\childdoctmp
        {
          \childdoctrue
          \includeonly{\childdocname}
          \def\childdocjob{#1}
          \def\jobname{#1}
        }
      \fi
      \expandafter
    \endgroup
    \childdoctmp
  \fi
}
%    \end{macrocode}

% \macro{\childdocof}
% The command |\childdocof| redirects
% compilation to the main file |#1|.
%    \begin{macrocode}
\newcommand{\childdocof}[1]
{
  \childdocdisable
  \childdoctrue
  \includeonly{\childdocname}
  \def\jobname{#1}
  \def\childdocjob{#1}
  \input{#1}
}
%    \end{macrocode}

% \macro{\childdocby}
% The command |\childdocby| ....
%    \begin{macrocode}
\newcommand{\childdocby}[2][]
{
  \childdocdisable
  \childdoctrue
  \childdocmanualtrue
  \if?#1?\else
    \def\jobname{#2}
  \fi
  \def\childdocjob{#2}
  \input{#2}
  \endinput
}
%    \end{macrocode}

% \macro{\childdocforward}
% The command |\childdocforward| redirects
% compilation to the main file or
% (if the optional argument is given) a child file.
% Parameters are set as if the main file
% or a child file starting with |\childdocof| was compiled.
% Then compilation is handed over to the main file:
%    \begin{macrocode}
\newcommand{\childdocforward}[2][]
{
  \begingroup
    \if?#1?
      \def\childdoctmp
      {
        \def\childdocname{#2}
        \def\childdocjob{#2}
        \def\jobname{#2}
        \input{#2}
        \endinput
      }
    \else
      \def\childdoctmp
      {
        \childdocdisable
        \def\childdocname{#2}
        \childdoctrue
        \includeonly{#2}
        \def\childdocjob{#1}
        \def\jobname{#1}
        \input{#1}
        \endinput
      }
    \fi
    \expandafter
  \endgroup
  \childdoctmp
}
%    \end{macrocode}

% \macro{\childdocforwardprefix}
% The command |\childdocforwardprefix| redirects
% compilation to the main or a child file by means of a pattern.
% The prefix |#1| in the current filename is replaced by |#2|
% and the suffix of the current filename is kept
% (it is assumed that the filename does not contain the substring `|~~~|'
% which is used as a delimiter).
% Compilation is handed over to the new file by |\childdocforward|:
%    \begin{macrocode}
\newcommand{\childdocforwardprefix}[3][]
{
  \begingroup
    \def\childdocextract #2##1~~~{\def\childdoctmp{\childdocforward[#1]{#3##1}}}
    \expandafter\childdocextract\childdocname~~~
    \expandafter
  \endgroup
  \childdoctmp
}
%    \end{macrocode}

% \macro{\childdoc}
% The deprecated macro |\childdoc| is a legacy version of |\childdocmain|:
%    \begin{macrocode}
\newcommand{\childdoc}{\childdocmain}
%    \end{macrocode}

% \macro{\childdocredirect}
% The deprecated macro |\childdocredirect| is a legacy version
% of |\childdocforward| and |\childdocforwardprefix|:
%    \begin{macrocode}
\newcommand{\childdocredirect}[2][]
{
  \begingroup
    \if?#1?
      \def\childdoctmp{\childdocforward{#2}}
    \else
      \def\childdoctmp{\childdocforwardprefix{#1}{#2}}
    \fi
    \expandafter
  \endgroup
  \childdoctmp
}
%    \end{macrocode}

%\iffalse
%</package>
%\fi
%
\endinput
|\\
|\childdocby{|\textit{main}|}|\\
\end{tabular}
\end{center}
%
The directive |\childdocby| is similar to |\childdocof|
described in \secref{sec:include},
but the subsequent selection of content must be done manually.
To that end, both |\ifchilddoc| and |\ifchilddocmanual|
will be true upon processing of a part,
and the name of the part is stored in |\childdocname|.
Note that |\jobname| will be set to the filename of the current part
so that each part receives an individual |.aux| file
that does not interfere with the |.aux| file(s) of the main document.
This behaviour can be altered by the alternative form
|\childdocby[*]{|\textit{main}|}| (with a non-empty optional argument)
which uses the |.aux| file of the main document
by setting |\jobname| to \textit{main}.

%%%%%%%%%%%%%%%%%%%%%%%%%%%%%%%%%%%%%%%%%%%%%%%%%%%%%%%%%%%%%%%%%%%%%%%%%%%%%%%%
\subsection{Driver Development}
\label{sec:driver}

The \textsf{childdoc} mechanism can also be use for the development
of definition files such as \LaTeX{} styles or classes.
This case differs from the above setup with multiple parts
included by |\include| in that no |\includeonly| should be invoked.
This can be achieved by starting the include file
(before |\ProvidesPackage|) with:
%
\begin{center}
\begin{tabular}{l}
|% \iffalse
%
% childdoc.dtx Copyright (C) 2017-2018 Niklas Beisert
%
% This work may be distributed and/or modified under the
% conditions of the LaTeX Project Public License, either version 1.3
% of this license or (at your option) any later version.
% The latest version of this license is in
%   http://www.latex-project.org/lppl.txt
% and version 1.3 or later is part of all distributions of LaTeX
% version 2005/12/01 or later.
%
% This work has the LPPL maintenance status `maintained'.
%
% The Current Maintainer of this work is Niklas Beisert.
%
% This work consists of the files childdoc.dtx and childdoc.ins
% and the derived files childdoc.def and cdocsamp.tex with
% cdocsch1.tex, cdocsch2.tex, cdocsdrf.tex, cdocsfn1.tex, cdocsfn2.tex.
%
%<package>\ifdefined\childdocmain\endinput\fi
%<package>\ProvidesFile{childdoc.def}[2018/12/30 v2.0 child document driver]
%<samplemain>\ProvidesFile{cdocsamp.tex}[2018/12/30 v2.0 sample for childdoc]
%<*driver>
%\ProvidesFile{childdoc.drv}[2018/12/30 v2.0 childdoc reference manual file]
\PassOptionsToClass{10pt,a4paper}{article}
\documentclass{ltxdoc}

\usepackage[margin=35mm]{geometry}
\usepackage{hyperref}
\usepackage{hyperxmp}
\usepackage[usenames]{color}

\hypersetup{colorlinks=true}
\hypersetup{pdfstartview=FitH}
\hypersetup{pdfpagemode=UseNone}
\hypersetup{pdfsource={}}
\hypersetup{pdflang={en-UK}}
\hypersetup{pdfcopyright={Copyright 2017-2018 Niklas Beisert.
  This work may be distributed and/or modified under the
  conditions of the LaTeX Project Public License, either version 1.3
  of this license or (at your option) any later version.}}
\hypersetup{pdflicenseurl={http://www.latex-project.org/lppl.txt}}
\hypersetup{pdfcontactaddress={ETH Zurich, ITP, HIT K,
  Wolfgang-Pauli-Strasse 27}}
\hypersetup{pdfcontactpostcode={8093}}
\hypersetup{pdfcontactcity={Zurich}}
\hypersetup{pdfcontactcountry={Switzerland}}
\hypersetup{pdfcontactemail={nbeisert@itp.phys.ethz.ch}}
\hypersetup{pdfcontacturl={http://people.phys.ethz.ch/\xmptilde nbeisert/}}

\newcommand{\secref}[1]{\hyperref[#1]{section \ref*{#1}}}

\parskip1ex
\parindent0pt
\let\olditemize\itemize
\def\itemize{\olditemize\parskip0pt}

\begin{document}

\title{The \textsf{childdoc} Package}
\hypersetup{pdftitle={The childdoc Package}}
\author{Niklas Beisert\\[2ex]
  Institut f\"ur Theoretische Physik\\
  Eidgen\"ossische Technische Hochschule Z\"urich\\
  Wolfgang-Pauli-Strasse 27, 8093 Z\"urich, Switzerland\\[1ex]
  \href{mailto:nbeisert@itp.phys.ethz.ch}
  {\texttt{nbeisert@itp.phys.ethz.ch}}}
\hypersetup{pdfauthor={Niklas Beisert}}
\hypersetup{pdfsubject={Manual for the LaTeX2e Package childdoc}}
\date{30 December 2018, \textsf{v2.0}}
\maketitle

\begin{abstract}\noindent
\textsf{childdoc} is a \LaTeXe{} package
that enables the direct compilation
of document sections included by |\include|
to individual files.
\end{abstract}

\begingroup
\parskip0ex
\tableofcontents
\endgroup

%%%%%%%%%%%%%%%%%%%%%%%%%%%%%%%%%%%%%%%%%%%%%%%%%%%%%%%%%%%%%%%%%%%%%%%%%%%%%%%%
%%%%%%%%%%%%%%%%%%%%%%%%%%%%%%%%%%%%%%%%%%%%%%%%%%%%%%%%%%%%%%%%%%%%%%%%%%%%%%%%
\section{Introduction}

\LaTeX{} provides a mechanism to structure a large document (such as a book)
into a main file and several child files (containing the chapters)
using the |\include| command.
This mechanism is beneficial for documents
which span hundreds of pages in order to
make the source file(s) more manageable.
Moreover, compilation can be restricted to
selected child files by means of the |\includeonly| command.
The latter feature can be used to reduce the compilation time while editing
(this was significantly more useful in the earlier days of \LaTeX{})
or to generate a smaller document which is easier to navigate.
Another application of |\includeonly| is to generate
documents consisting of selected parts of the complete document.

However, there are a few drawbacks of the plain |\include| mechanism:
\begin{itemize}
\item
The child files cannot be compiled on their own,
they can only be compiled via the main file.
A naive editing environment
(such as a text editor with an option
to have the current file processed by \LaTeX)
may require one to switch to the main file before compiling;
attempting to compile the child file produces errors.
\item
The main file must be modified (each time)
to adjust the |\includeonly| command
to the present needs. This easily leaves the main file in a messy state.
\item
The generated document will always carry the filename
of the main document. This is inconvenient if
several child files are to be compiled and
to be kept for distribution.
\end{itemize}

The present package provides a simple interface
to make child files individually compilable by \LaTeX{}.
Compiling a child file then has the same effect as compiling
the main file with an |\includeonly| command
to select the appropriate child.
Moreover the generated document will carry the name of the child
rather than the main file.
This resolves all three above issues.

This feature is meant to make the editing of books,
thesis documents and lecture notes somewhat more convenient.
However, the package can also be used efficiently for
composing a series of documents (such as exercise sheets)
which are typically distributed individually.
It then assists the author in generating the individual documents
(potentially in different versions)
as well as a document containing the collected series.
Another application is in developing style files
or other kinds of included material
where compilation of the style file could redirect
to a sample or test file.

%%%%%%%%%%%%%%%%%%%%%%%%%%%%%%%%%%%%%%%%%%%%%%%%%%%%%%%%%%%%%%%%%%%%%%%%%%%%%%%%
%%%%%%%%%%%%%%%%%%%%%%%%%%%%%%%%%%%%%%%%%%%%%%%%%%%%%%%%%%%%%%%%%%%%%%%%%%%%%%%%
\section{Usage}

First of all, the package \textsf{childdoc} is \emph{not} a standard
\LaTeXe{} |.sty| style file! Therefore it needs to be invoked in
a non-standard way.

%%%%%%%%%%%%%%%%%%%%%%%%%%%%%%%%%%%%%%%%%%%%%%%%%%%%%%%%%%%%%%%%%%%%%%%%%%%%%%%%
\subsection{Included Files}
\label{sec:include}

%%%%%%%%%%%%%%%%%%%%%%%%%%%%%%%%%%%%%%%%
\DescribeMacro{\childdocmain}
To use the package, add the commands
\begin{center}
\begin{tabular}{l}
|\input{childdoc.def}|\\
|\childdocmain{}|\\
\end{tabular}
\end{center}
at the very top of the main \LaTeX{} file,
in particular \emph{before} the |\documentclass| statement!
The argument of |\childdocmain| should be left empty
(but it must be present).

%%%%%%%%%%%%%%%%%%%%%%%%%%%%%%%%%%%%%%%%
\DescribeMacro{\childdocof}
Furthermore, add the commands
\begin{center}
\begin{tabular}{l}
|\input{childdoc.def}|\\
|\childdocof{|\textit{main}|}|\\
\end{tabular}
\end{center}
at the top of every child file \textit{child}
which is included by |\include{|\textit{child}|}|
from within the main file
(or at least for those files to be compiled individually).
The argument \textit{main} must be the filename of the main file.

There are a couple of
considerations in setting up the main and child documents:

%%%%%%%%%%%%%%%%%%%%%%%%%%%%%%%%%%%%%%%%
\paragraph{Restrictions.}

Please note the following restrictions:
\begin{itemize}
\item
|\childdocmain| must be called with one argument \textit{main}
to ensure compatibility with earlier version of the package.
It must either be empty (|\childdocmain{}|)
or precisely match the filename of the main file in which it is specified.
See \secref{sec:detection} for further information.
\item
The filename \textit{main} must be specified without the |.tex| extension.
\item
The filename \textit{main} is case sensitive
(even in case-insensitive file systems)
due to internal string comparison.
\item
The argument \textit{main} should be fully expanded, it cannot be a macro.
\item
Subdirectories and special characters should be avoided in filenames.
\item
The command |\childdocmain{|\textit{main}|}| must be followed by a whitespace.
It should not be followed immediately by another command
or by a comment mark `|%|'.
This is because the \TeX{} parser reads the token immediately following
the argument of |\childdocmain| and puts it
at the beginning of every child section;
however, a white\-space is ignored.
\end{itemize}

%%%%%%%%%%%%%%%%%%%%%%%%%%%%%%%%%%%%%%%%
\paragraph{Content of Main File.}

It is advisable to place all content in the child files included by |\include|.
Any output contained in the main file will appear in all child documents
unless suppressed manually;
it cannot be suppressed automatically by the |\includeonly| directive
and thus should normally be avoided.
A method to include some content in the main file
by means of conditional processing is described in \secref{sec:conditional}.

%%%%%%%%%%%%%%%%%%%%%%%%%%%%%%%%%%%%%%%%
\paragraph{Page Numbering.}

When only a part of the document is compiled,
the appropriate numbering of pages
(as well as other status parameters)
is determined from the |.aux| files.
The latter contain information from previous passes.
However this information needs to propagate through
all intermediate child documents.
Therefore the page numbering in child documents may well
be inconsistent until the complete document is compiled at least once.

A useful (if unconventional) way to always ensure a consistent
page numbering is to restart the numbering in each child document
and denote the pages by `\textit{child}|.|\textit{page}'
where \textit{child} represents the chapter/section number of the child file.
This can be achieved by the command
|\numberwithin{page}{|\textit{child}|}|
of the \textsf{amsmath} package
where \textit{child} can be |chapter| or |section|
depending on the chosen structuring.
Alternatively, one can modify the macro |\thepage| appropriately
and reset the counter |page| at the start of each child file.

%%%%%%%%%%%%%%%%%%%%%%%%%%%%%%%%%%%%%%%%%%%%%%%%%%%%%%%%%%%%%%%%%%%%%%%%%%%%%%%%
\subsection{Conditional Processing}
\label{sec:conditional}

The package provides a mechanism to compile different versions
of a document. To customise the versions further some conditional processing
can come in handy to distinguish which version is being compiled.
The package provides two macros to describe the compilation context:

%%%%%%%%%%%%%%%%%%%%%%%%%%%%%%%%%%%%%%%%
\DescribeMacro{\ifchilddoc}
The conditional |\ifchilddoc| distinguishes between the compilation of
child documents and the main document:
%
\begin{center}
|\ifchilddoc |\textit{child-code}| |[|\||else |\textit{main-code}]| \||fi|
\end{center}

%%%%%%%%%%%%%%%%%%%%%%%%%%%%%%%%%%%%%%%%
\DescribeMacro{\childdocname}
\DescribeMacro{\childdocjob}
The macro |\childdocname| contains the filename (without extension)
of the main or child file being processed.
Note that |\childdocjob| will always contain the name of the main file.

%%%%%%%%%%%%%%%%%%%%%%%%%%%%%%%%%%%%%%%%
\paragraph{Title Page.}

Conditional processing can be used to include a title or banner page
in the main document when proper precautions are taken.
Importantly, the code in the main file should ensure that the page counter
(as well as other status parameters which are stored in the |.aux| files)
takes the same value after the conditional processing.
Otherwise the page numbers may take divergent values
depending on which part is compiled.

For example, a title page could be declared by:
%
\begin{center}
\begin{tabular}{l}
|\ifchilddoc\||else|\\
|\addtocounter{page}{-1}|\\
\textit{code for title page}\\
|\newpage|\\
|\||fi|
\end{tabular}
\end{center}
%
A banner page for the child documents can be generated by:
%
\begin{center}
\begin{tabular}{l}
|\ifchilddoc|\\
|\addtocounter{page}{-1}|\\
\textit{code for banner page}\\
|\newpage|\\
|\||fi|
\end{tabular}
\end{center}
%
Here one could write a message such as:
\begin{center}
|This is the part \childdocname{} of \childdocjob{}.|
\end{center}

%%%%%%%%%%%%%%%%%%%%%%%%%%%%%%%%%%%%%%%%%%%%%%%%%%%%%%%%%%%%%%%%%%%%%%%%%%%%%%%%
\subsection{Flags}
\label{sec:flags}

The package makes it easy to generate different versions
of the main or child documents.
To this end compilation flags can be defined
and assigned different default values.
They will be particularly useful in conjunction
with the forwarding mechanism described in \secref{sec:forward}.

For example, it may be useful to have a flag |\version|
which can be set to |draft| or |final|.
The document source will contain some conditional code
depending on the value of |\version|.
Suppose further, the flag should default to |final| for the main file
and to |draft| for child files
which is a natural assignment for editing the document.
This is achieved by placing the following code
in the preamble of the main document
(below the |\childdocmain| directive):
%
\begin{center}
\begin{tabular}{l}
|\ifchilddoc|\\
|\providecommand{\version}{draft}|\\
|\||else|\\
|\providecommand{\version}{final}|\\
|\||fi|
\end{tabular}
\end{center}
%
The definition by |\providecommand| makes sure
that previous definitions are not overwritten.
Further statements |\providecommand{\version}{...}|
can thus be added before the above code to override it.

For the main file, one might add a line
(between |\childdocmain| and the above block)
%
\begin{center}
|%\ifchilddoc\||else\providecommand{\version}{draft}\||fi|
\end{center}
%
which can be uncommented to produce a draft version.
Likewise one can add a line to the very top of a child file
(above the |\childdocof{|\textit{main}|}| directive)
%
\begin{center}
|%\providecommand{\version}{final}|
\end{center}
%
which can be uncommented to produce the final version of this child document.

%%%%%%%%%%%%%%%%%%%%%%%%%%%%%%%%%%%%%%%%%%%%%%%%%%%%%%%%%%%%%%%%%%%%%%%%%%%%%%%%
\subsection{Forwarding}
\label{sec:forward}

Different versions of the main or child documents
using compilation flags as described in \secref{sec:flags}
can be (permanently) stored in different files
for convenient compilation, viewing and distribution.
To this end, the package defines a command
to pass on compilation to a different file:

%%%%%%%%%%%%%%%%%%%%%%%%%%%%%%%%%%%%%%%%
\DescribeMacro{\childdocforward}
The command |\childdocforward| redirects processing to
another source file:
%
\begin{center}
\begin{tabular}{l}
|\input{childdoc.def}|\\
|\childdocforward[|\textit{main}|]{|\textit{dest}|}|\\
\end{tabular}
\end{center}
%
The argument \textit{dest} is the destination file
(without extension).
It should be the main file or one of the child files.
Note that further \textsf{childdoc} directives
such as |\childdocof| and |\childdocforward|
in the indicated file will be processed in this form.
The optional argument \textit{main}
passes on directly to the main file \textit{main}
while pretending to compile the child \textit{dest}.
This form behaves as if \textit{dest}
issues |\childdocof{|\textit{main}|}| right away,
and no further \textsf{childdoc} directives will be processed.

%%%%%%%%%%%%%%%%%%%%%%%%%%%%%%%%%%%%%%%%
\DescribeMacro{\...prefix}
In the alternative form |\childdocforwardprefix|,
%
\begin{center}
\begin{tabular}{l}
|\input{childdoc.def}|\\
|\childdocforwardprefix[|\textit{main}|]{|\textit{prefix}|}{|\textit{dest}|}|
\end{tabular}
\end{center}
%
the destination file is determined by a pattern
depending on the current file:
To make this work, the current file must be called
`{\textit{prefix}\hspace{0.2em}\textit{suffix}}'
with \textit{prefix} matching precisely the argument.
Processing is then passed on to the file
`{\textit{dest}\hspace{0.2em}\textit{suffix}}'.
Surely, the same effect is achieved by
directly specifying the
argument `{\textit{dest}\hspace{0.2em}\textit{suffix}}'
in the first form.
However, that requires to set up a different file
for each child. With the alternative form of the command
all these files can have exactly the same content
which simplifies setting them up and maintaining them.

For example, the following file |draft.tex|
with a compilation flag |\version| as described in \secref{sec:flags}
compiles the main document as a draft:
%
\begin{center}
\begin{tabular}{l}
|\def\version{draft}|\\
|\input{childdoc.def}|\\
|\childdocforward{|\textit{main}|}|
\end{tabular}
\end{center}
%
Likewise, the following files |final|\textit{nn}|.tex|
compile the final version of the child document
|child|\textit{nn}|.tex|:
%
\begin{center}
\begin{tabular}{l}
|\def\version{final}|\\
|\input{childdoc.def}|\\
|\childdocforwardprefix{final}{child}|
\end{tabular}
\end{center}
%

Note that when several versions of a main file and/or of each child file
are to be generated, it may be convenient to set up a |Makefile| or
shell script to automatise the process.

%%%%%%%%%%%%%%%%%%%%%%%%%%%%%%%%%%%%%%%%%%%%%%%%%%%%%%%%%%%%%%%%%%%%%%%%%%%%%%%%
\subsection{Command Line Processing}
\label{sec:commandline}

The effect of redirection files can also be achieved by invoking
the \LaTeX{} compiler with a more elaborate command line.
Most conveniently this should be done as part
of a shell script or a |Makefile|.

When using \textsf{childdoc} in the main file, the following
command lines effectively perform a redirection
(note that depending on the shell being used,
backslashes may have to be doubled: `|\|' $\to$ `|\\|'):
%
\begin{center}
|... -jobname "|\textit{target}|" |\\|"|[\textit{flags}]%
|\input{childdoc.def}\childdocforward[|\textit{main}|]{|\textit{dest}|}"|
\end{center}
%
Here \textit{target} is the name of the output file,
\textit{main} is the name of the main file
and \textit{dest} is the name of the main or child file to be processed
(all filenames without extensions).
The optional argument \textit{main} can be omitted
if \textit{main} matches \textit{dest}.
Optionally, compilation \textit{flags} can be defined via |\def| commands.
This command line makes the \TeX{} engine believe
it is compiling the file \textit{target}
whose content is specified as the latter parameter.
The provided code then forwards the processing to
\textit{main} or \textit{dest} as described in \secref{sec:forward}.

%%%%%%%%%%%%%%%%%%%%%%%%%%%%%%%%%%%%%%%%%%%%%%%%%%%%%%%%%%%%%%%%%%%%%%%%%%%%%%%%
\subsection{Include by Input}
\label{sec:input}

Including child documents by |\include| has some restrictions by design.
Most notably, the content of a child document always occupies
its own set of pages; pages cannot be shared between child documents.
Usually, this behaviour makes perfect sense
because each child document contain an essential part of the document.
However, in some situations it may be desirable to compose
a document from a collection of parts
without having mandatory page breaks between then.
For this case, the package
provides a mechanism to include parts
by |\input| which can also be processed individually.
However, by construction this mechanism
requires manual handling of the content to be output.

%%%%%%%%%%%%%%%%%%%%%%%%%%%%%%%%%%%%%%%%
\DescribeMacro{\ifchilddocmanual}
The main file should be prepared as usual, see \secref{sec:include}.
However, the document body must make a distinction
between processing of an individual part and of the main document, e.g.:
%
\begin{center}
\begin{tabular}{l}
|\ifchilddocmanual|\\
|\input{\childdocname}|\\
|\||else|\\
\textit{document body with }|\input{|\textit{part}|}|\\
|\||fi|
\end{tabular}
\end{center}
%
The conditional |\ifchilddocmanual| is true whenever
a part to be included by |\input| is being compiled,
and the name of the part is stored in |\childdocname|.

%%%%%%%%%%%%%%%%%%%%%%%%%%%%%%%%%%%%%%%%
\DescribeMacro{\childdocby}
Each part to be included by |\input| should start with:
%
\begin{center}
\begin{tabular}{l}
|\input{childdoc.def}|\\
|\childdocby{|\textit{main}|}|\\
\end{tabular}
\end{center}
%
The directive |\childdocby| is similar to |\childdocof|
described in \secref{sec:include},
but the subsequent selection of content must be done manually.
To that end, both |\ifchilddoc| and |\ifchilddocmanual|
will be true upon processing of a part,
and the name of the part is stored in |\childdocname|.
Note that |\jobname| will be set to the filename of the current part
so that each part receives an individual |.aux| file
that does not interfere with the |.aux| file(s) of the main document.
This behaviour can be altered by the alternative form
|\childdocby[*]{|\textit{main}|}| (with a non-empty optional argument)
which uses the |.aux| file of the main document
by setting |\jobname| to \textit{main}.

%%%%%%%%%%%%%%%%%%%%%%%%%%%%%%%%%%%%%%%%%%%%%%%%%%%%%%%%%%%%%%%%%%%%%%%%%%%%%%%%
\subsection{Driver Development}
\label{sec:driver}

The \textsf{childdoc} mechanism can also be use for the development
of definition files such as \LaTeX{} styles or classes.
This case differs from the above setup with multiple parts
included by |\include| in that no |\includeonly| should be invoked.
This can be achieved by starting the include file
(before |\ProvidesPackage|) with:
%
\begin{center}
\begin{tabular}{l}
|\input{childdoc.def}|\\
|\childdocforward{|\textit{main}|}|\\
\end{tabular}
\end{center}
%
or alternatively with:
%
\begin{center}
\begin{tabular}{l}
|\input{childdoc.def}|\\
|\childdocby{|\textit{main}|}|\\
\end{tabular}
\end{center}
%
Both forms have slightly different effects as described above.
The main file is prepared as usual, see \secref{sec:include}.

%%%%%%%%%%%%%%%%%%%%%%%%%%%%%%%%%%%%%%%%%%%%%%%%%%%%%%%%%%%%%%%%%%%%%%%%%%%%%%%%
\subsection{Legacy Detection}
\label{sec:detection}

The directive |\childdocmain| in the main file can detect
whether the complete document or merely a child is to be compiled
even without using the directive |\childdocof|.
This method is deprecated because it is less robust
and there is no compelling reason to use it;
it is merely provided for backward compatibility
and it may be removed in future versions.

If the detection mechanism is to be used,
it is mandatory to correctly specify
the filename of the main file as the argument of |\childdocmain|:
%
\begin{center}
\begin{tabular}{l}
|\input{childdoc.def}|\\
|\childdocmain{|\textit{main}|}|\\
\end{tabular}
\end{center}
%
If |\jobname| does not match the argument \textit{main} of |\childdocmain|,
it is assumed that |\jobname| points to the child file to be compiled.
When using |\childdocmain| with the main file specified as argument,
it suffices to start a child file
with just |\input{|\textit{main}|}|
without loading of the package and using |\childdocof|.
If instead all processing is done
with the appropriate \textsf{childdoc} directives,
the argument of \textit{main} of |\childdocmain| can be empty.

An alternative version of the command line processing described
in \secref{sec:commandline} using the detection mechanism reads:
%
\begin{center}
|... -jobname "|\textit{target}|" "|[\textit{flags}]%
[|\def\jobname{|\textit{dest}|}|]|\input{|\textit{main}|}"|
\end{center}

%%%%%%%%%%%%%%%%%%%%%%%%%%%%%%%%%%%%%%%%%%%%%%%%%%%%%%%%%%%%%%%%%%%%%%%%%%%%%%%%
\subsection{Manual Code}
\label{sec:manual}

In case one cannot be certain whether the definitions file |childdoc.def|
is installed on the target \TeX{} distribution
and one prefers not to ship it,
it is conceivable to paste a few relevant commands into the sources.

To that end, drop all statements |\input{childdoc.def}|
and perform the replacements as outlined below.
Instead of |\childdocmain{|\textit{main}|}| add the following code
to the top of the main file:
%
\begin{center}
\begin{tabular}{l}
|\||ifdefined\childdocname\endinput\||fi\newif\ifchilddoc|\\
|\edef\childdocname{\scantokens\expandafter{\jobname\noexpand}}|\\
|\def\childdocmain{|\textit{main}|}\||ifx\childdocmain\childdocname\||else|\\
|\childdoctrue\includeonly{\childdocname}\let\jobname\childdocmain\||fi|\\
\end{tabular}
\end{center}
%
Instead of |\childdocof{|\textit{main}|}| just include the main file
at the top of each child file:
%
\begin{center}
|\input{|\textit{main}|}|
\end{center}
%
A simple redirection |\childdocforward{|\textit{dest}|}| is achieved by:
%
\begin{center}
|\def\jobname{|\textit{dest}|}\input{\jobname}|
\end{center}
%
The redirection with prefix
|\childdocforwardprefix[|\textit{prefix}|]{|\textit{dest}|}|
is accomplished by:
%
\begin{center}
\begin{tabular}{l}
|{\edef\jobname{\scantokens\expandafter{\jobname\noexpand}}|\\
|\def\redirectjob |\textit{prefix}|#1~~~{\gdef\jobname{|\textit{dest}|#1}}|\\
|\expandafter\redirectjob\jobname~~~}\input{\jobname}|
\end{tabular}
\end{center}

In an alternative approach,
child documents can be compiled by a specific command line
without additional code or specific definitions:
%
\begin{center}
|... -jobname "|\textit{target}|" "|[\textit{flags}]%
|\includeonly{|\textit{dest}|}\input{|\textit{main}|}"|
\end{center}
%

%%%%%%%%%%%%%%%%%%%%%%%%%%%%%%%%%%%%%%%%%%%%%%%%%%%%%%%%%%%%%%%%%%%%%%%%%%%%%%%%
%%%%%%%%%%%%%%%%%%%%%%%%%%%%%%%%%%%%%%%%%%%%%%%%%%%%%%%%%%%%%%%%%%%%%%%%%%%%%%%%
\section{Information}

%%%%%%%%%%%%%%%%%%%%%%%%%%%%%%%%%%%%%%%%%%%%%%%%%%%%%%%%%%%%%%%%%%%%%%%%%%%%%%%%
\subsection{Copyright}

Copyright \copyright{} 2017--2018 Niklas Beisert

This work may be distributed and/or modified under the
conditions of the \LaTeX{} Project Public License, either version 1.3
of this license or (at your option) any later version.
The latest version of this license is in
  \url{http://www.latex-project.org/lppl.txt}
and version 1.3 or later is part of all distributions of \LaTeX{}
version 2005/12/01 or later.

This work has the LPPL maintenance status `maintained'.

The Current Maintainer of this work is Niklas Beisert.

This work consists of the files |README.txt|, |childdoc.ins| and |childdoc.dtx|
as well as the derived files |childdoc.def|, |cdocsamp.tex|
with |cdocsch1.tex|, |cdocsch2.tex|, |cdocspt3.tex|, |cdocspt4.tex|,
|cdocsdrf.tex|, |cdocsfn1.tex|, |cdocsfn2.tex|
as well as |childdoc.pdf|.

%%%%%%%%%%%%%%%%%%%%%%%%%%%%%%%%%%%%%%%%%%%%%%%%%%%%%%%%%%%%%%%%%%%%%%%%%%%%%%%%
\subsection{Files and Installation}

The package consists of the files:
%
\begin{center}
\begin{tabular}{ll}
    |README.txt|   & readme file \\
    |childdoc.ins| & installation file \\
    |childdoc.dtx| & source file \\
    |childdoc.def| & definition file \\
    |cdocsamp.tex| & sample main file \\
    |cdocsch1.tex| & sample include file \\
    |cdocsch2.tex| & sample include file \\
    |cdocspt3.tex| & sample part file \\
    |cdocspt4.tex| & sample part file \\
    |cdocsdrf.tex| & sample redirection file \\
    |cdocsfn1.tex| & sample redirection file \\
    |cdocsfn2.tex| & sample redirection file \\
    |childdoc.pdf| & manual
\end{tabular}
\end{center}
%
The distribution consists of the files
|README.txt|, |childdoc.ins| and |childdoc.dtx|.
%
\begin{itemize}
\item
Run (pdf)\LaTeX{} on |childdoc.dtx|
to compile the manual |childdoc.pdf| (this file).
\item
Run \LaTeX{} on |childdoc.ins| to create the definitions file |childdoc.def|
and the sample |cdocsamp.tex| with include files
|cdocsch1.tex|, |cdocsch2.tex|, |cdocspt3.tex|, |cdocspt4.tex|,
|cdocsdrf.tex|, |cdocsfn1.tex|, |cdocsfn2.tex|.
Then copy the file |childdoc.def| to an appropriate directory of your \LaTeX{}
distribution, e.g.\ \textit{texmf-root}|/tex/latex/childdoc|.
\end{itemize}

%%%%%%%%%%%%%%%%%%%%%%%%%%%%%%%%%%%%%%%%%%%%%%%%%%%%%%%%%%%%%%%%%%%%%%%%%%%%%%%%
\subsection{Related CTAN Packages}

There are several other packages which offer a similar functionality:
%
\begin{itemize}
\item
The packages
\href{http://ctan.org/pkg/docmute}{\textsf{docmute}},
\href{http://ctan.org/pkg/includex}{\textsf{includex}} and
\href{http://ctan.org/pkg/standalone}{\textsf{standalone}}
provide commands to include only the document body of
a child file thus allowing both files to be compiled individually.
\item
The packages \href{http://ctan.org/pkg/subdocs}{\textsf{subdocs}}
and \href{http://ctan.org/pkg/subfiles}{\textsf{subfiles}}
provide structures in which the main and child documents can be
encapsulated and allowing them to be compiled individually.
The inclusion mechanism is different from the conventional |\include|.
\item
The package \href{http://ctan.org/pkg/combine}{\textsf{combine}}
is an elaborate solution to combine several documents into one.
\end{itemize}
%
See also the CTAN topic \href{http://ctan.org/topic/subdocs}{\textsf{subdocs}}
for further related packages.
The present package differs from the above solutions in that
a document structure constructed with the conventional |\include| mechanism
just needs two extra commands at the top of every file
such that all constituent files can be compiled individually.

%%%%%%%%%%%%%%%%%%%%%%%%%%%%%%%%%%%%%%%%%%%%%%%%%%%%%%%%%%%%%%%%%%%%%%%%%%%%%%%%
%\subsection{Feature Suggestions}
%
%The following is a list of features which may be useful for future
%versions of this package:
%%
%\begin{itemize}
%\item
%\ldots
%\end{itemize}

%%%%%%%%%%%%%%%%%%%%%%%%%%%%%%%%%%%%%%%%%%%%%%%%%%%%%%%%%%%%%%%%%%%%%%%%%%%%%%%%
\subsection{Revision History}

%%%%%%%%%%%%%%%%%%%%%%%%%%%%%%%%%%%%%%%%
\paragraph{v2.0:} 2018/12/30

\begin{itemize}
\item
immediate forward processing
\item
added |\childdocby| mechanism
\item
manual restructured
\end{itemize}

%%%%%%%%%%%%%%%%%%%%%%%%%%%%%%%%%%%%%%%%
\paragraph{v1.6:} 2018/01/17

\begin{itemize}
\item
application for development of include files
\item
corrections to manual
\end{itemize}

%%%%%%%%%%%%%%%%%%%%%%%%%%%%%%%%%%%%%%%%
\paragraph{v1.5:} 2017/05/21

\begin{itemize}
\item
more complete structuring introduced
\item
|\childdocof| introduced
\item
|\childdoc| renamed to |\childdocmain|
\item
|\childredirect| renamed to |\childdocforward| and |\childdocforwardprefix|
and functionality expanded
\end{itemize}

%%%%%%%%%%%%%%%%%%%%%%%%%%%%%%%%%%%%%%%%
\paragraph{v1.0:} 2017/04/27

\begin{itemize}
\item
manual and install package
\item
first version published on CTAN
\end{itemize}

%%%%%%%%%%%%%%%%%%%%%%%%%%%%%%%%%%%%%%%%
\paragraph{v0.6:} 2017/04/26

\begin{itemize}
\item
redirection mechanism added
\end{itemize}

%%%%%%%%%%%%%%%%%%%%%%%%%%%%%%%%%%%%%%%%
\paragraph{v0.5:} 2017/04/26

\begin{itemize}
\item
functionality in definition file
\end{itemize}


%%%%%%%%%%%%%%%%%%%%%%%%%%%%%%%%%%%%%%%%%%%%%%%%%%%%%%%%%%%%%%%%%%%%%%%%%%%%%%%%
%%%%%%%%%%%%%%%%%%%%%%%%%%%%%%%%%%%%%%%%%%%%%%%%%%%%%%%%%%%%%%%%%%%%%%%%%%%%%%%%
%%%%%%%%%%%%%%%%%%%%%%%%%%%%%%%%%%%%%%%%%%%%%%%%%%%%%%%%%%%%%%%%%%%%%%%%%%%%%%%%
\appendix

\settowidth\MacroIndent{\rmfamily\scriptsize 000\ }

 \DocInput{childdoc.dtx}

\end{document}
%</driver>
% \fi
%
% %%%%%%%%%%%%%%%%%%%%%%%%%%%%%%%%%%%%%%%%%%%%%%%%%%%%%%%%%%%%%%%%%%%%%%%%%%%%%%
% %%%%%%%%%%%%%%%%%%%%%%%%%%%%%%%%%%%%%%%%%%%%%%%%%%%%%%%%%%%%%%%%%%%%%%%%%%%%%%
% \section{Sample}
%\iffalse
%<*samplemain>
%\fi
%
% The following presents a sample document
% with two chapters, two parts, a title page,
% a compile flag as well as three forwarding files to set the flag.
% It consists of eight |.tex| files:
% \begin{center}
% \begin{tabular}{ll}
% |cdocsamp.tex|&main file\\
% |cdocsch1.tex|&include file for chapter 1\\
% |cdocsch2.tex|&include file for chapter 2\\
% |cdocspt3.tex|&include file for part 3\\
% |cdocspt4.tex|&include file for part 4\\
% |cdocsdrf.tex|&forwarding file for main file in draft mode\\
% |cdocsfi1.tex|&forwarding file for final version of chapter 1\\
% |cdocsfi2.tex|&forwarding file for final version of chapter 2\\
% \end{tabular}
% \end{center}
% Each of the eight files can be compiled directly by the \LaTeX{} compiler.
%
% %%%%%%%%%%%%%%%%%%%%%%%%%%%%%%%%%%%%%%
% \paragraph{Main File.}
%
% The main file is called |cdocsamp.tex|.
%
% Load the \textsf{childdoc} definitions and
% declare the filename for the main document:
%    \begin{macrocode}
\input{childdoc.def}
\childdocmain{}
%    \end{macrocode}

% Optional override for |\version| flag:
%    \begin{macrocode}
%%\ifchilddoc\else\providecommand{\version}{draft}\fi
%    \end{macrocode}

% Define the default values for the |\version| flag
% (|final| for the main file and |draft| for childs):
%    \begin{macrocode}
\ifchilddoc
\providecommand{\version}{draft}
\else
\providecommand{\version}{final}
\fi
%    \end{macrocode}

% Load the standard document class:
%    \begin{macrocode}
\documentclass[12pt]{article}
%    \end{macrocode}

% Start the document body:
%    \begin{macrocode}
\begin{document}
%    \end{macrocode}

% Declare a title page.
% Print title, part of document being processed and version flag:
%    \begin{macrocode}
\addtocounter{page}{-1}
\begin{center}
{\LARGE\bfseries{}childdoc example\par}
\vspace{1cm}
\ifchilddoc
\ifchilddocmanual part\else chapter\fi:
`\childdocname' of `\childdocjob'\par
\else
main document: `\childdocjob'\par
\fi
version: \version\par
\end{center}
\newpage
%    \end{macrocode}

% Manually include selected file,
% otherwise process as usual:
%    \begin{macrocode}
\ifchilddocmanual
\section*{part `\childdocname'}
\input{\childdocname}
\else
%    \end{macrocode}

% Include the two chapters:
%    \begin{macrocode}
\include{cdocsch1}
\include{cdocsch2}
%    \end{macrocode}

% Include the two parts unless only chapters should be displayed:
%    \begin{macrocode}
\ifchilddoc\else
\section{part three}
\input{cdocspt3}
\section{part four}
\input{cdocspt4}
\fi
%    \end{macrocode}

% Process as usual until here:
%    \begin{macrocode}
\fi
%    \end{macrocode}

% End of document body:
%    \begin{macrocode}
\end{document}
%    \end{macrocode}
%\iffalse
%</samplemain>
%\fi
%
% %%%%%%%%%%%%%%%%%%%%%%%%%%%%%%%%%%%%%%
% \paragraph{Chapter Include Files.}
%
% The include files are called |cdocsch1.tex| and |cdocsch2.tex|.
%
%\iffalse
%<*samplechap1|samplechap2>
%\fi

% Optional override for |\version| flag:
%    \begin{macrocode}
%%\providecommand{\version}{final}
%    \end{macrocode}

% Include the main document:
%    \begin{macrocode}
\input{childdoc.def}
\childdocof{cdocsamp}
%    \end{macrocode}

%\iffalse
%</samplechap1|samplechap2>
%\fi
%
%\iffalse
%<*samplechap1>
%\fi
% Some text for chapter 1:
%    \begin{macrocode}
\section{one}
some text in chapter one
%    \end{macrocode}

%\iffalse
%</samplechap1>
%\fi
% Some text for chapter 2:
%\iffalse
%<*samplechap2>
%\fi
%    \begin{macrocode}
\section{two}
more text in chapter two
%    \end{macrocode}

%\iffalse
%</samplechap2>
%\fi
%
% %%%%%%%%%%%%%%%%%%%%%%%%%%%%%%%%%%%%%%
% \paragraph{Part Include Files.}
%
% The include files are called |cdocspt3.tex| and |cdocspt4.tex|.
%
%\iffalse
%<*samplepart3|samplepart4>
%\fi

% Optional override for |\version| flag:
%    \begin{macrocode}
%%\providecommand{\version}{final}
%    \end{macrocode}

% Include the main document:
%    \begin{macrocode}
\input{childdoc.def}
\childdocby{cdocsamp}
%    \end{macrocode}

%\iffalse
%</samplepart3|samplepart4>
%\fi
%
%\iffalse
%<*samplepart3>
%\fi
% Some text for part 3:
%    \begin{macrocode}
some text in part three
%    \end{macrocode}

%\iffalse
%</samplepart3>
%\fi
% Some text for part 4:
%\iffalse
%<*samplepart4>
%\fi
%    \begin{macrocode}
more text in part four
%    \end{macrocode}

%\iffalse
%</samplepart4>
%\fi
%
% %%%%%%%%%%%%%%%%%%%%%%%%%%%%%%%%%%%%%%
% \paragraph{Forwarding for a Complete Draft.}
%
% The following forwarding file |cdocsdrf.tex|
% compiles the main document in draft mode:
%\iffalse
%<*sampledraft>
%\fi
%    \begin{macrocode}
\def\version{draft}
\input{childdoc.def}
\childdocforward{cdocsamp}
%    \end{macrocode}

%\iffalse
%</sampledraft>
%\fi
%
% %%%%%%%%%%%%%%%%%%%%%%%%%%%%%%%%%%%%%%
% \paragraph{Forwarding for Final Version of the Chapters.}
%
% The following forwarding files |cdocsfn1.tex| and |cdocsfn2.tex|
% (with identical content)
% compile the final versions of the child documents
% |cdocsch1.tex| and |cdocsch2.tex|, respectively:
%\iffalse
%<*samplefinal>
%\fi
%    \begin{macrocode}
\def\version{final}
\input{childdoc.def}
\childdocforwardprefix[cdocsamp]{cdocsfn}{cdocsch}
%    \end{macrocode}

%\iffalse
%</samplefinal>
%\fi
%
% %%%%%%%%%%%%%%%%%%%%%%%%%%%%%%%%%%%%%%
% \paragraph{Command Line Processing.}
%
% The following three command lines generate the output files
% |cdocscld|, |cdocscl1| and |cdocscl2|
% which should be identical to
% |cdocsdrf|, |cdocsch1| and |cdocsfn2|, respectively:
% \begin{center}
% \begin{tabular}{l}
% |latex -jobname cdocscld \|\\
% |  "\def\version{draft}\input{childdoc.def}\childdocforward{cdocsamp}"|\\
% |latex -jobname cdocscl1 \|\\
% |  "\input{childdoc.def}\childdocforward[cdocsamp]{cdocsch1}"|\\
% |latex -jobname cdocscl2 \|\\
% |  "\def\version{final}\input{childdoc.def}\childdocforward{cdocsch2}"|
% \end{tabular}
% \end{center}
% Note that the trailing backslash on each first line
% merely continues the input to the second line
% (for convenient cut ant paste).
% Furthermore, the command |latex| can be replaced by any
% of its alternative versions such as |pdflatex|.
%
% %%%%%%%%%%%%%%%%%%%%%%%%%%%%%%%%%%%%%%%%%%%%%%%%%%%%%%%%%%%%%%%%%%%%%%%%%%%%%%
% %%%%%%%%%%%%%%%%%%%%%%%%%%%%%%%%%%%%%%%%%%%%%%%%%%%%%%%%%%%%%%%%%%%%%%%%%%%%%%
% \section{Implementation}
%\iffalse
%<*package>
%\fi
%
% This section describes the definitions file |childdoc.def|.

% The definitions cannot be loaded using |\usepackage| or |\RequirePackage|
% which has a mechanism to prevent loading a style file more than once.
% When loading the definitions by means of |\input|
% multiple instances have to be prevented manually:
%\iffalse
%This code needs to be before the `\ProvidesFile' directive
%which is defined at the beginning of this file.
%Therefore it is also placed there and commented out here.
%</package>
%<*discard>
%\fi
%    \begin{macrocode}
\ifdefined\childdocmain\endinput\fi
%    \end{macrocode}
%\iffalse
%</discard>
%<*package>
%\fi
%
% \macro{\ifchilddoc}
% \macro{\ifchilddocmanual}
% The conditional |\ifchilddoc| tells whether a
% child (true) or main (false) document is being compiled.
% The conditional |\ifchilddocmanual| tells whether
% the |\includeonly| mechanism is used (false) or
% the selection of child files must be performed manually (true).
% The definitions initialise to false:
%    \begin{macrocode}
\newif\ifchilddoc
\newif\ifchilddocmanual
%    \end{macrocode}

% \macro{\childdocname}
% \macro{\childdocjob}
% The macro |\childdocname| stores the name of the main document
% to be compiled. The macro |\childdocjob| stores the name of
% the document on which the \LaTeX{} compiler was originally invoked.
% The content of |\jobname| cannot be compared
% to filenames specified in the source due to different catcodes.
% The following code rescans |\jobname|, stores the result
% in |\childdocname| and saves a copy in |\childdocjob|:
%    \begin{macrocode}
\edef\childdocname{\scantokens\expandafter{\jobname\noexpand}}
\let\childdocjob\childdocname
%    \end{macrocode}

% \macro{\childdocdisable}
% The macro |\childdocdisable| prevents the main file
% from being processed more than once.
% At this stage, the main document command |\childdocmain|
% is assumed to be called once again where it should do nothing.
% Any subsequent call to it should prevent
% a secondary processing of the main document
% It overwrites the forwarding commands
% |\childdocof| and |\childdocforward|
% with empty macros to prevent further inclusions of the main document:
%    \begin{macrocode}
\newcommand{\childdocdisable}
{
  \renewcommand{\childdocmain}[1]{\renewcommand{\childdocmain}[1]{\endinput}}
  \renewcommand{\childdocof}[1]{}
  \renewcommand{\childdocby}[2][]{}
  \renewcommand{\childdocforward}[2][]{}
  \renewcommand{\childdocdisable}{}
}
%    \end{macrocode}

% \macro{\childdocmain}
% The macro |\childdocmain| is to be called at the top of the main file
% with nothing or the main filename (without extension) as argument.
% First, it breaks loops.
% If the argument is not empty and does not match |\childdocname|
% (which is set by the first inclusion of |childdoc.def|),
% |\ifchilddoc| is set to true, |\includeonly| is applied to the child file
% and |\jobname| is set to the main file
% (for proper handling of |.aux| files):
%    \begin{macrocode}
\newcommand{\childdocmain}[1]
{
  \childdocdisable\childdocmain{}
  \if?#1?\else
    \begingroup
      \def\childdoctmp{#1}
      \ifx\childdoctmp\childdocname
        \def\childdoctmp{}
      \else
        \def\childdoctmp
        {
          \childdoctrue
          \includeonly{\childdocname}
          \def\childdocjob{#1}
          \def\jobname{#1}
        }
      \fi
      \expandafter
    \endgroup
    \childdoctmp
  \fi
}
%    \end{macrocode}

% \macro{\childdocof}
% The command |\childdocof| redirects
% compilation to the main file |#1|.
%    \begin{macrocode}
\newcommand{\childdocof}[1]
{
  \childdocdisable
  \childdoctrue
  \includeonly{\childdocname}
  \def\jobname{#1}
  \def\childdocjob{#1}
  \input{#1}
}
%    \end{macrocode}

% \macro{\childdocby}
% The command |\childdocby| ....
%    \begin{macrocode}
\newcommand{\childdocby}[2][]
{
  \childdocdisable
  \childdoctrue
  \childdocmanualtrue
  \if?#1?\else
    \def\jobname{#2}
  \fi
  \def\childdocjob{#2}
  \input{#2}
  \endinput
}
%    \end{macrocode}

% \macro{\childdocforward}
% The command |\childdocforward| redirects
% compilation to the main file or
% (if the optional argument is given) a child file.
% Parameters are set as if the main file
% or a child file starting with |\childdocof| was compiled.
% Then compilation is handed over to the main file:
%    \begin{macrocode}
\newcommand{\childdocforward}[2][]
{
  \begingroup
    \if?#1?
      \def\childdoctmp
      {
        \def\childdocname{#2}
        \def\childdocjob{#2}
        \def\jobname{#2}
        \input{#2}
        \endinput
      }
    \else
      \def\childdoctmp
      {
        \childdocdisable
        \def\childdocname{#2}
        \childdoctrue
        \includeonly{#2}
        \def\childdocjob{#1}
        \def\jobname{#1}
        \input{#1}
        \endinput
      }
    \fi
    \expandafter
  \endgroup
  \childdoctmp
}
%    \end{macrocode}

% \macro{\childdocforwardprefix}
% The command |\childdocforwardprefix| redirects
% compilation to the main or a child file by means of a pattern.
% The prefix |#1| in the current filename is replaced by |#2|
% and the suffix of the current filename is kept
% (it is assumed that the filename does not contain the substring `|~~~|'
% which is used as a delimiter).
% Compilation is handed over to the new file by |\childdocforward|:
%    \begin{macrocode}
\newcommand{\childdocforwardprefix}[3][]
{
  \begingroup
    \def\childdocextract #2##1~~~{\def\childdoctmp{\childdocforward[#1]{#3##1}}}
    \expandafter\childdocextract\childdocname~~~
    \expandafter
  \endgroup
  \childdoctmp
}
%    \end{macrocode}

% \macro{\childdoc}
% The deprecated macro |\childdoc| is a legacy version of |\childdocmain|:
%    \begin{macrocode}
\newcommand{\childdoc}{\childdocmain}
%    \end{macrocode}

% \macro{\childdocredirect}
% The deprecated macro |\childdocredirect| is a legacy version
% of |\childdocforward| and |\childdocforwardprefix|:
%    \begin{macrocode}
\newcommand{\childdocredirect}[2][]
{
  \begingroup
    \if?#1?
      \def\childdoctmp{\childdocforward{#2}}
    \else
      \def\childdoctmp{\childdocforwardprefix{#1}{#2}}
    \fi
    \expandafter
  \endgroup
  \childdoctmp
}
%    \end{macrocode}

%\iffalse
%</package>
%\fi
%
\endinput
|\\
|\childdocforward{|\textit{main}|}|\\
\end{tabular}
\end{center}
%
or alternatively with:
%
\begin{center}
\begin{tabular}{l}
|% \iffalse
%
% childdoc.dtx Copyright (C) 2017-2018 Niklas Beisert
%
% This work may be distributed and/or modified under the
% conditions of the LaTeX Project Public License, either version 1.3
% of this license or (at your option) any later version.
% The latest version of this license is in
%   http://www.latex-project.org/lppl.txt
% and version 1.3 or later is part of all distributions of LaTeX
% version 2005/12/01 or later.
%
% This work has the LPPL maintenance status `maintained'.
%
% The Current Maintainer of this work is Niklas Beisert.
%
% This work consists of the files childdoc.dtx and childdoc.ins
% and the derived files childdoc.def and cdocsamp.tex with
% cdocsch1.tex, cdocsch2.tex, cdocsdrf.tex, cdocsfn1.tex, cdocsfn2.tex.
%
%<package>\ifdefined\childdocmain\endinput\fi
%<package>\ProvidesFile{childdoc.def}[2018/12/30 v2.0 child document driver]
%<samplemain>\ProvidesFile{cdocsamp.tex}[2018/12/30 v2.0 sample for childdoc]
%<*driver>
%\ProvidesFile{childdoc.drv}[2018/12/30 v2.0 childdoc reference manual file]
\PassOptionsToClass{10pt,a4paper}{article}
\documentclass{ltxdoc}

\usepackage[margin=35mm]{geometry}
\usepackage{hyperref}
\usepackage{hyperxmp}
\usepackage[usenames]{color}

\hypersetup{colorlinks=true}
\hypersetup{pdfstartview=FitH}
\hypersetup{pdfpagemode=UseNone}
\hypersetup{pdfsource={}}
\hypersetup{pdflang={en-UK}}
\hypersetup{pdfcopyright={Copyright 2017-2018 Niklas Beisert.
  This work may be distributed and/or modified under the
  conditions of the LaTeX Project Public License, either version 1.3
  of this license or (at your option) any later version.}}
\hypersetup{pdflicenseurl={http://www.latex-project.org/lppl.txt}}
\hypersetup{pdfcontactaddress={ETH Zurich, ITP, HIT K,
  Wolfgang-Pauli-Strasse 27}}
\hypersetup{pdfcontactpostcode={8093}}
\hypersetup{pdfcontactcity={Zurich}}
\hypersetup{pdfcontactcountry={Switzerland}}
\hypersetup{pdfcontactemail={nbeisert@itp.phys.ethz.ch}}
\hypersetup{pdfcontacturl={http://people.phys.ethz.ch/\xmptilde nbeisert/}}

\newcommand{\secref}[1]{\hyperref[#1]{section \ref*{#1}}}

\parskip1ex
\parindent0pt
\let\olditemize\itemize
\def\itemize{\olditemize\parskip0pt}

\begin{document}

\title{The \textsf{childdoc} Package}
\hypersetup{pdftitle={The childdoc Package}}
\author{Niklas Beisert\\[2ex]
  Institut f\"ur Theoretische Physik\\
  Eidgen\"ossische Technische Hochschule Z\"urich\\
  Wolfgang-Pauli-Strasse 27, 8093 Z\"urich, Switzerland\\[1ex]
  \href{mailto:nbeisert@itp.phys.ethz.ch}
  {\texttt{nbeisert@itp.phys.ethz.ch}}}
\hypersetup{pdfauthor={Niklas Beisert}}
\hypersetup{pdfsubject={Manual for the LaTeX2e Package childdoc}}
\date{30 December 2018, \textsf{v2.0}}
\maketitle

\begin{abstract}\noindent
\textsf{childdoc} is a \LaTeXe{} package
that enables the direct compilation
of document sections included by |\include|
to individual files.
\end{abstract}

\begingroup
\parskip0ex
\tableofcontents
\endgroup

%%%%%%%%%%%%%%%%%%%%%%%%%%%%%%%%%%%%%%%%%%%%%%%%%%%%%%%%%%%%%%%%%%%%%%%%%%%%%%%%
%%%%%%%%%%%%%%%%%%%%%%%%%%%%%%%%%%%%%%%%%%%%%%%%%%%%%%%%%%%%%%%%%%%%%%%%%%%%%%%%
\section{Introduction}

\LaTeX{} provides a mechanism to structure a large document (such as a book)
into a main file and several child files (containing the chapters)
using the |\include| command.
This mechanism is beneficial for documents
which span hundreds of pages in order to
make the source file(s) more manageable.
Moreover, compilation can be restricted to
selected child files by means of the |\includeonly| command.
The latter feature can be used to reduce the compilation time while editing
(this was significantly more useful in the earlier days of \LaTeX{})
or to generate a smaller document which is easier to navigate.
Another application of |\includeonly| is to generate
documents consisting of selected parts of the complete document.

However, there are a few drawbacks of the plain |\include| mechanism:
\begin{itemize}
\item
The child files cannot be compiled on their own,
they can only be compiled via the main file.
A naive editing environment
(such as a text editor with an option
to have the current file processed by \LaTeX)
may require one to switch to the main file before compiling;
attempting to compile the child file produces errors.
\item
The main file must be modified (each time)
to adjust the |\includeonly| command
to the present needs. This easily leaves the main file in a messy state.
\item
The generated document will always carry the filename
of the main document. This is inconvenient if
several child files are to be compiled and
to be kept for distribution.
\end{itemize}

The present package provides a simple interface
to make child files individually compilable by \LaTeX{}.
Compiling a child file then has the same effect as compiling
the main file with an |\includeonly| command
to select the appropriate child.
Moreover the generated document will carry the name of the child
rather than the main file.
This resolves all three above issues.

This feature is meant to make the editing of books,
thesis documents and lecture notes somewhat more convenient.
However, the package can also be used efficiently for
composing a series of documents (such as exercise sheets)
which are typically distributed individually.
It then assists the author in generating the individual documents
(potentially in different versions)
as well as a document containing the collected series.
Another application is in developing style files
or other kinds of included material
where compilation of the style file could redirect
to a sample or test file.

%%%%%%%%%%%%%%%%%%%%%%%%%%%%%%%%%%%%%%%%%%%%%%%%%%%%%%%%%%%%%%%%%%%%%%%%%%%%%%%%
%%%%%%%%%%%%%%%%%%%%%%%%%%%%%%%%%%%%%%%%%%%%%%%%%%%%%%%%%%%%%%%%%%%%%%%%%%%%%%%%
\section{Usage}

First of all, the package \textsf{childdoc} is \emph{not} a standard
\LaTeXe{} |.sty| style file! Therefore it needs to be invoked in
a non-standard way.

%%%%%%%%%%%%%%%%%%%%%%%%%%%%%%%%%%%%%%%%%%%%%%%%%%%%%%%%%%%%%%%%%%%%%%%%%%%%%%%%
\subsection{Included Files}
\label{sec:include}

%%%%%%%%%%%%%%%%%%%%%%%%%%%%%%%%%%%%%%%%
\DescribeMacro{\childdocmain}
To use the package, add the commands
\begin{center}
\begin{tabular}{l}
|\input{childdoc.def}|\\
|\childdocmain{}|\\
\end{tabular}
\end{center}
at the very top of the main \LaTeX{} file,
in particular \emph{before} the |\documentclass| statement!
The argument of |\childdocmain| should be left empty
(but it must be present).

%%%%%%%%%%%%%%%%%%%%%%%%%%%%%%%%%%%%%%%%
\DescribeMacro{\childdocof}
Furthermore, add the commands
\begin{center}
\begin{tabular}{l}
|\input{childdoc.def}|\\
|\childdocof{|\textit{main}|}|\\
\end{tabular}
\end{center}
at the top of every child file \textit{child}
which is included by |\include{|\textit{child}|}|
from within the main file
(or at least for those files to be compiled individually).
The argument \textit{main} must be the filename of the main file.

There are a couple of
considerations in setting up the main and child documents:

%%%%%%%%%%%%%%%%%%%%%%%%%%%%%%%%%%%%%%%%
\paragraph{Restrictions.}

Please note the following restrictions:
\begin{itemize}
\item
|\childdocmain| must be called with one argument \textit{main}
to ensure compatibility with earlier version of the package.
It must either be empty (|\childdocmain{}|)
or precisely match the filename of the main file in which it is specified.
See \secref{sec:detection} for further information.
\item
The filename \textit{main} must be specified without the |.tex| extension.
\item
The filename \textit{main} is case sensitive
(even in case-insensitive file systems)
due to internal string comparison.
\item
The argument \textit{main} should be fully expanded, it cannot be a macro.
\item
Subdirectories and special characters should be avoided in filenames.
\item
The command |\childdocmain{|\textit{main}|}| must be followed by a whitespace.
It should not be followed immediately by another command
or by a comment mark `|%|'.
This is because the \TeX{} parser reads the token immediately following
the argument of |\childdocmain| and puts it
at the beginning of every child section;
however, a white\-space is ignored.
\end{itemize}

%%%%%%%%%%%%%%%%%%%%%%%%%%%%%%%%%%%%%%%%
\paragraph{Content of Main File.}

It is advisable to place all content in the child files included by |\include|.
Any output contained in the main file will appear in all child documents
unless suppressed manually;
it cannot be suppressed automatically by the |\includeonly| directive
and thus should normally be avoided.
A method to include some content in the main file
by means of conditional processing is described in \secref{sec:conditional}.

%%%%%%%%%%%%%%%%%%%%%%%%%%%%%%%%%%%%%%%%
\paragraph{Page Numbering.}

When only a part of the document is compiled,
the appropriate numbering of pages
(as well as other status parameters)
is determined from the |.aux| files.
The latter contain information from previous passes.
However this information needs to propagate through
all intermediate child documents.
Therefore the page numbering in child documents may well
be inconsistent until the complete document is compiled at least once.

A useful (if unconventional) way to always ensure a consistent
page numbering is to restart the numbering in each child document
and denote the pages by `\textit{child}|.|\textit{page}'
where \textit{child} represents the chapter/section number of the child file.
This can be achieved by the command
|\numberwithin{page}{|\textit{child}|}|
of the \textsf{amsmath} package
where \textit{child} can be |chapter| or |section|
depending on the chosen structuring.
Alternatively, one can modify the macro |\thepage| appropriately
and reset the counter |page| at the start of each child file.

%%%%%%%%%%%%%%%%%%%%%%%%%%%%%%%%%%%%%%%%%%%%%%%%%%%%%%%%%%%%%%%%%%%%%%%%%%%%%%%%
\subsection{Conditional Processing}
\label{sec:conditional}

The package provides a mechanism to compile different versions
of a document. To customise the versions further some conditional processing
can come in handy to distinguish which version is being compiled.
The package provides two macros to describe the compilation context:

%%%%%%%%%%%%%%%%%%%%%%%%%%%%%%%%%%%%%%%%
\DescribeMacro{\ifchilddoc}
The conditional |\ifchilddoc| distinguishes between the compilation of
child documents and the main document:
%
\begin{center}
|\ifchilddoc |\textit{child-code}| |[|\||else |\textit{main-code}]| \||fi|
\end{center}

%%%%%%%%%%%%%%%%%%%%%%%%%%%%%%%%%%%%%%%%
\DescribeMacro{\childdocname}
\DescribeMacro{\childdocjob}
The macro |\childdocname| contains the filename (without extension)
of the main or child file being processed.
Note that |\childdocjob| will always contain the name of the main file.

%%%%%%%%%%%%%%%%%%%%%%%%%%%%%%%%%%%%%%%%
\paragraph{Title Page.}

Conditional processing can be used to include a title or banner page
in the main document when proper precautions are taken.
Importantly, the code in the main file should ensure that the page counter
(as well as other status parameters which are stored in the |.aux| files)
takes the same value after the conditional processing.
Otherwise the page numbers may take divergent values
depending on which part is compiled.

For example, a title page could be declared by:
%
\begin{center}
\begin{tabular}{l}
|\ifchilddoc\||else|\\
|\addtocounter{page}{-1}|\\
\textit{code for title page}\\
|\newpage|\\
|\||fi|
\end{tabular}
\end{center}
%
A banner page for the child documents can be generated by:
%
\begin{center}
\begin{tabular}{l}
|\ifchilddoc|\\
|\addtocounter{page}{-1}|\\
\textit{code for banner page}\\
|\newpage|\\
|\||fi|
\end{tabular}
\end{center}
%
Here one could write a message such as:
\begin{center}
|This is the part \childdocname{} of \childdocjob{}.|
\end{center}

%%%%%%%%%%%%%%%%%%%%%%%%%%%%%%%%%%%%%%%%%%%%%%%%%%%%%%%%%%%%%%%%%%%%%%%%%%%%%%%%
\subsection{Flags}
\label{sec:flags}

The package makes it easy to generate different versions
of the main or child documents.
To this end compilation flags can be defined
and assigned different default values.
They will be particularly useful in conjunction
with the forwarding mechanism described in \secref{sec:forward}.

For example, it may be useful to have a flag |\version|
which can be set to |draft| or |final|.
The document source will contain some conditional code
depending on the value of |\version|.
Suppose further, the flag should default to |final| for the main file
and to |draft| for child files
which is a natural assignment for editing the document.
This is achieved by placing the following code
in the preamble of the main document
(below the |\childdocmain| directive):
%
\begin{center}
\begin{tabular}{l}
|\ifchilddoc|\\
|\providecommand{\version}{draft}|\\
|\||else|\\
|\providecommand{\version}{final}|\\
|\||fi|
\end{tabular}
\end{center}
%
The definition by |\providecommand| makes sure
that previous definitions are not overwritten.
Further statements |\providecommand{\version}{...}|
can thus be added before the above code to override it.

For the main file, one might add a line
(between |\childdocmain| and the above block)
%
\begin{center}
|%\ifchilddoc\||else\providecommand{\version}{draft}\||fi|
\end{center}
%
which can be uncommented to produce a draft version.
Likewise one can add a line to the very top of a child file
(above the |\childdocof{|\textit{main}|}| directive)
%
\begin{center}
|%\providecommand{\version}{final}|
\end{center}
%
which can be uncommented to produce the final version of this child document.

%%%%%%%%%%%%%%%%%%%%%%%%%%%%%%%%%%%%%%%%%%%%%%%%%%%%%%%%%%%%%%%%%%%%%%%%%%%%%%%%
\subsection{Forwarding}
\label{sec:forward}

Different versions of the main or child documents
using compilation flags as described in \secref{sec:flags}
can be (permanently) stored in different files
for convenient compilation, viewing and distribution.
To this end, the package defines a command
to pass on compilation to a different file:

%%%%%%%%%%%%%%%%%%%%%%%%%%%%%%%%%%%%%%%%
\DescribeMacro{\childdocforward}
The command |\childdocforward| redirects processing to
another source file:
%
\begin{center}
\begin{tabular}{l}
|\input{childdoc.def}|\\
|\childdocforward[|\textit{main}|]{|\textit{dest}|}|\\
\end{tabular}
\end{center}
%
The argument \textit{dest} is the destination file
(without extension).
It should be the main file or one of the child files.
Note that further \textsf{childdoc} directives
such as |\childdocof| and |\childdocforward|
in the indicated file will be processed in this form.
The optional argument \textit{main}
passes on directly to the main file \textit{main}
while pretending to compile the child \textit{dest}.
This form behaves as if \textit{dest}
issues |\childdocof{|\textit{main}|}| right away,
and no further \textsf{childdoc} directives will be processed.

%%%%%%%%%%%%%%%%%%%%%%%%%%%%%%%%%%%%%%%%
\DescribeMacro{\...prefix}
In the alternative form |\childdocforwardprefix|,
%
\begin{center}
\begin{tabular}{l}
|\input{childdoc.def}|\\
|\childdocforwardprefix[|\textit{main}|]{|\textit{prefix}|}{|\textit{dest}|}|
\end{tabular}
\end{center}
%
the destination file is determined by a pattern
depending on the current file:
To make this work, the current file must be called
`{\textit{prefix}\hspace{0.2em}\textit{suffix}}'
with \textit{prefix} matching precisely the argument.
Processing is then passed on to the file
`{\textit{dest}\hspace{0.2em}\textit{suffix}}'.
Surely, the same effect is achieved by
directly specifying the
argument `{\textit{dest}\hspace{0.2em}\textit{suffix}}'
in the first form.
However, that requires to set up a different file
for each child. With the alternative form of the command
all these files can have exactly the same content
which simplifies setting them up and maintaining them.

For example, the following file |draft.tex|
with a compilation flag |\version| as described in \secref{sec:flags}
compiles the main document as a draft:
%
\begin{center}
\begin{tabular}{l}
|\def\version{draft}|\\
|\input{childdoc.def}|\\
|\childdocforward{|\textit{main}|}|
\end{tabular}
\end{center}
%
Likewise, the following files |final|\textit{nn}|.tex|
compile the final version of the child document
|child|\textit{nn}|.tex|:
%
\begin{center}
\begin{tabular}{l}
|\def\version{final}|\\
|\input{childdoc.def}|\\
|\childdocforwardprefix{final}{child}|
\end{tabular}
\end{center}
%

Note that when several versions of a main file and/or of each child file
are to be generated, it may be convenient to set up a |Makefile| or
shell script to automatise the process.

%%%%%%%%%%%%%%%%%%%%%%%%%%%%%%%%%%%%%%%%%%%%%%%%%%%%%%%%%%%%%%%%%%%%%%%%%%%%%%%%
\subsection{Command Line Processing}
\label{sec:commandline}

The effect of redirection files can also be achieved by invoking
the \LaTeX{} compiler with a more elaborate command line.
Most conveniently this should be done as part
of a shell script or a |Makefile|.

When using \textsf{childdoc} in the main file, the following
command lines effectively perform a redirection
(note that depending on the shell being used,
backslashes may have to be doubled: `|\|' $\to$ `|\\|'):
%
\begin{center}
|... -jobname "|\textit{target}|" |\\|"|[\textit{flags}]%
|\input{childdoc.def}\childdocforward[|\textit{main}|]{|\textit{dest}|}"|
\end{center}
%
Here \textit{target} is the name of the output file,
\textit{main} is the name of the main file
and \textit{dest} is the name of the main or child file to be processed
(all filenames without extensions).
The optional argument \textit{main} can be omitted
if \textit{main} matches \textit{dest}.
Optionally, compilation \textit{flags} can be defined via |\def| commands.
This command line makes the \TeX{} engine believe
it is compiling the file \textit{target}
whose content is specified as the latter parameter.
The provided code then forwards the processing to
\textit{main} or \textit{dest} as described in \secref{sec:forward}.

%%%%%%%%%%%%%%%%%%%%%%%%%%%%%%%%%%%%%%%%%%%%%%%%%%%%%%%%%%%%%%%%%%%%%%%%%%%%%%%%
\subsection{Include by Input}
\label{sec:input}

Including child documents by |\include| has some restrictions by design.
Most notably, the content of a child document always occupies
its own set of pages; pages cannot be shared between child documents.
Usually, this behaviour makes perfect sense
because each child document contain an essential part of the document.
However, in some situations it may be desirable to compose
a document from a collection of parts
without having mandatory page breaks between then.
For this case, the package
provides a mechanism to include parts
by |\input| which can also be processed individually.
However, by construction this mechanism
requires manual handling of the content to be output.

%%%%%%%%%%%%%%%%%%%%%%%%%%%%%%%%%%%%%%%%
\DescribeMacro{\ifchilddocmanual}
The main file should be prepared as usual, see \secref{sec:include}.
However, the document body must make a distinction
between processing of an individual part and of the main document, e.g.:
%
\begin{center}
\begin{tabular}{l}
|\ifchilddocmanual|\\
|\input{\childdocname}|\\
|\||else|\\
\textit{document body with }|\input{|\textit{part}|}|\\
|\||fi|
\end{tabular}
\end{center}
%
The conditional |\ifchilddocmanual| is true whenever
a part to be included by |\input| is being compiled,
and the name of the part is stored in |\childdocname|.

%%%%%%%%%%%%%%%%%%%%%%%%%%%%%%%%%%%%%%%%
\DescribeMacro{\childdocby}
Each part to be included by |\input| should start with:
%
\begin{center}
\begin{tabular}{l}
|\input{childdoc.def}|\\
|\childdocby{|\textit{main}|}|\\
\end{tabular}
\end{center}
%
The directive |\childdocby| is similar to |\childdocof|
described in \secref{sec:include},
but the subsequent selection of content must be done manually.
To that end, both |\ifchilddoc| and |\ifchilddocmanual|
will be true upon processing of a part,
and the name of the part is stored in |\childdocname|.
Note that |\jobname| will be set to the filename of the current part
so that each part receives an individual |.aux| file
that does not interfere with the |.aux| file(s) of the main document.
This behaviour can be altered by the alternative form
|\childdocby[*]{|\textit{main}|}| (with a non-empty optional argument)
which uses the |.aux| file of the main document
by setting |\jobname| to \textit{main}.

%%%%%%%%%%%%%%%%%%%%%%%%%%%%%%%%%%%%%%%%%%%%%%%%%%%%%%%%%%%%%%%%%%%%%%%%%%%%%%%%
\subsection{Driver Development}
\label{sec:driver}

The \textsf{childdoc} mechanism can also be use for the development
of definition files such as \LaTeX{} styles or classes.
This case differs from the above setup with multiple parts
included by |\include| in that no |\includeonly| should be invoked.
This can be achieved by starting the include file
(before |\ProvidesPackage|) with:
%
\begin{center}
\begin{tabular}{l}
|\input{childdoc.def}|\\
|\childdocforward{|\textit{main}|}|\\
\end{tabular}
\end{center}
%
or alternatively with:
%
\begin{center}
\begin{tabular}{l}
|\input{childdoc.def}|\\
|\childdocby{|\textit{main}|}|\\
\end{tabular}
\end{center}
%
Both forms have slightly different effects as described above.
The main file is prepared as usual, see \secref{sec:include}.

%%%%%%%%%%%%%%%%%%%%%%%%%%%%%%%%%%%%%%%%%%%%%%%%%%%%%%%%%%%%%%%%%%%%%%%%%%%%%%%%
\subsection{Legacy Detection}
\label{sec:detection}

The directive |\childdocmain| in the main file can detect
whether the complete document or merely a child is to be compiled
even without using the directive |\childdocof|.
This method is deprecated because it is less robust
and there is no compelling reason to use it;
it is merely provided for backward compatibility
and it may be removed in future versions.

If the detection mechanism is to be used,
it is mandatory to correctly specify
the filename of the main file as the argument of |\childdocmain|:
%
\begin{center}
\begin{tabular}{l}
|\input{childdoc.def}|\\
|\childdocmain{|\textit{main}|}|\\
\end{tabular}
\end{center}
%
If |\jobname| does not match the argument \textit{main} of |\childdocmain|,
it is assumed that |\jobname| points to the child file to be compiled.
When using |\childdocmain| with the main file specified as argument,
it suffices to start a child file
with just |\input{|\textit{main}|}|
without loading of the package and using |\childdocof|.
If instead all processing is done
with the appropriate \textsf{childdoc} directives,
the argument of \textit{main} of |\childdocmain| can be empty.

An alternative version of the command line processing described
in \secref{sec:commandline} using the detection mechanism reads:
%
\begin{center}
|... -jobname "|\textit{target}|" "|[\textit{flags}]%
[|\def\jobname{|\textit{dest}|}|]|\input{|\textit{main}|}"|
\end{center}

%%%%%%%%%%%%%%%%%%%%%%%%%%%%%%%%%%%%%%%%%%%%%%%%%%%%%%%%%%%%%%%%%%%%%%%%%%%%%%%%
\subsection{Manual Code}
\label{sec:manual}

In case one cannot be certain whether the definitions file |childdoc.def|
is installed on the target \TeX{} distribution
and one prefers not to ship it,
it is conceivable to paste a few relevant commands into the sources.

To that end, drop all statements |\input{childdoc.def}|
and perform the replacements as outlined below.
Instead of |\childdocmain{|\textit{main}|}| add the following code
to the top of the main file:
%
\begin{center}
\begin{tabular}{l}
|\||ifdefined\childdocname\endinput\||fi\newif\ifchilddoc|\\
|\edef\childdocname{\scantokens\expandafter{\jobname\noexpand}}|\\
|\def\childdocmain{|\textit{main}|}\||ifx\childdocmain\childdocname\||else|\\
|\childdoctrue\includeonly{\childdocname}\let\jobname\childdocmain\||fi|\\
\end{tabular}
\end{center}
%
Instead of |\childdocof{|\textit{main}|}| just include the main file
at the top of each child file:
%
\begin{center}
|\input{|\textit{main}|}|
\end{center}
%
A simple redirection |\childdocforward{|\textit{dest}|}| is achieved by:
%
\begin{center}
|\def\jobname{|\textit{dest}|}\input{\jobname}|
\end{center}
%
The redirection with prefix
|\childdocforwardprefix[|\textit{prefix}|]{|\textit{dest}|}|
is accomplished by:
%
\begin{center}
\begin{tabular}{l}
|{\edef\jobname{\scantokens\expandafter{\jobname\noexpand}}|\\
|\def\redirectjob |\textit{prefix}|#1~~~{\gdef\jobname{|\textit{dest}|#1}}|\\
|\expandafter\redirectjob\jobname~~~}\input{\jobname}|
\end{tabular}
\end{center}

In an alternative approach,
child documents can be compiled by a specific command line
without additional code or specific definitions:
%
\begin{center}
|... -jobname "|\textit{target}|" "|[\textit{flags}]%
|\includeonly{|\textit{dest}|}\input{|\textit{main}|}"|
\end{center}
%

%%%%%%%%%%%%%%%%%%%%%%%%%%%%%%%%%%%%%%%%%%%%%%%%%%%%%%%%%%%%%%%%%%%%%%%%%%%%%%%%
%%%%%%%%%%%%%%%%%%%%%%%%%%%%%%%%%%%%%%%%%%%%%%%%%%%%%%%%%%%%%%%%%%%%%%%%%%%%%%%%
\section{Information}

%%%%%%%%%%%%%%%%%%%%%%%%%%%%%%%%%%%%%%%%%%%%%%%%%%%%%%%%%%%%%%%%%%%%%%%%%%%%%%%%
\subsection{Copyright}

Copyright \copyright{} 2017--2018 Niklas Beisert

This work may be distributed and/or modified under the
conditions of the \LaTeX{} Project Public License, either version 1.3
of this license or (at your option) any later version.
The latest version of this license is in
  \url{http://www.latex-project.org/lppl.txt}
and version 1.3 or later is part of all distributions of \LaTeX{}
version 2005/12/01 or later.

This work has the LPPL maintenance status `maintained'.

The Current Maintainer of this work is Niklas Beisert.

This work consists of the files |README.txt|, |childdoc.ins| and |childdoc.dtx|
as well as the derived files |childdoc.def|, |cdocsamp.tex|
with |cdocsch1.tex|, |cdocsch2.tex|, |cdocspt3.tex|, |cdocspt4.tex|,
|cdocsdrf.tex|, |cdocsfn1.tex|, |cdocsfn2.tex|
as well as |childdoc.pdf|.

%%%%%%%%%%%%%%%%%%%%%%%%%%%%%%%%%%%%%%%%%%%%%%%%%%%%%%%%%%%%%%%%%%%%%%%%%%%%%%%%
\subsection{Files and Installation}

The package consists of the files:
%
\begin{center}
\begin{tabular}{ll}
    |README.txt|   & readme file \\
    |childdoc.ins| & installation file \\
    |childdoc.dtx| & source file \\
    |childdoc.def| & definition file \\
    |cdocsamp.tex| & sample main file \\
    |cdocsch1.tex| & sample include file \\
    |cdocsch2.tex| & sample include file \\
    |cdocspt3.tex| & sample part file \\
    |cdocspt4.tex| & sample part file \\
    |cdocsdrf.tex| & sample redirection file \\
    |cdocsfn1.tex| & sample redirection file \\
    |cdocsfn2.tex| & sample redirection file \\
    |childdoc.pdf| & manual
\end{tabular}
\end{center}
%
The distribution consists of the files
|README.txt|, |childdoc.ins| and |childdoc.dtx|.
%
\begin{itemize}
\item
Run (pdf)\LaTeX{} on |childdoc.dtx|
to compile the manual |childdoc.pdf| (this file).
\item
Run \LaTeX{} on |childdoc.ins| to create the definitions file |childdoc.def|
and the sample |cdocsamp.tex| with include files
|cdocsch1.tex|, |cdocsch2.tex|, |cdocspt3.tex|, |cdocspt4.tex|,
|cdocsdrf.tex|, |cdocsfn1.tex|, |cdocsfn2.tex|.
Then copy the file |childdoc.def| to an appropriate directory of your \LaTeX{}
distribution, e.g.\ \textit{texmf-root}|/tex/latex/childdoc|.
\end{itemize}

%%%%%%%%%%%%%%%%%%%%%%%%%%%%%%%%%%%%%%%%%%%%%%%%%%%%%%%%%%%%%%%%%%%%%%%%%%%%%%%%
\subsection{Related CTAN Packages}

There are several other packages which offer a similar functionality:
%
\begin{itemize}
\item
The packages
\href{http://ctan.org/pkg/docmute}{\textsf{docmute}},
\href{http://ctan.org/pkg/includex}{\textsf{includex}} and
\href{http://ctan.org/pkg/standalone}{\textsf{standalone}}
provide commands to include only the document body of
a child file thus allowing both files to be compiled individually.
\item
The packages \href{http://ctan.org/pkg/subdocs}{\textsf{subdocs}}
and \href{http://ctan.org/pkg/subfiles}{\textsf{subfiles}}
provide structures in which the main and child documents can be
encapsulated and allowing them to be compiled individually.
The inclusion mechanism is different from the conventional |\include|.
\item
The package \href{http://ctan.org/pkg/combine}{\textsf{combine}}
is an elaborate solution to combine several documents into one.
\end{itemize}
%
See also the CTAN topic \href{http://ctan.org/topic/subdocs}{\textsf{subdocs}}
for further related packages.
The present package differs from the above solutions in that
a document structure constructed with the conventional |\include| mechanism
just needs two extra commands at the top of every file
such that all constituent files can be compiled individually.

%%%%%%%%%%%%%%%%%%%%%%%%%%%%%%%%%%%%%%%%%%%%%%%%%%%%%%%%%%%%%%%%%%%%%%%%%%%%%%%%
%\subsection{Feature Suggestions}
%
%The following is a list of features which may be useful for future
%versions of this package:
%%
%\begin{itemize}
%\item
%\ldots
%\end{itemize}

%%%%%%%%%%%%%%%%%%%%%%%%%%%%%%%%%%%%%%%%%%%%%%%%%%%%%%%%%%%%%%%%%%%%%%%%%%%%%%%%
\subsection{Revision History}

%%%%%%%%%%%%%%%%%%%%%%%%%%%%%%%%%%%%%%%%
\paragraph{v2.0:} 2018/12/30

\begin{itemize}
\item
immediate forward processing
\item
added |\childdocby| mechanism
\item
manual restructured
\end{itemize}

%%%%%%%%%%%%%%%%%%%%%%%%%%%%%%%%%%%%%%%%
\paragraph{v1.6:} 2018/01/17

\begin{itemize}
\item
application for development of include files
\item
corrections to manual
\end{itemize}

%%%%%%%%%%%%%%%%%%%%%%%%%%%%%%%%%%%%%%%%
\paragraph{v1.5:} 2017/05/21

\begin{itemize}
\item
more complete structuring introduced
\item
|\childdocof| introduced
\item
|\childdoc| renamed to |\childdocmain|
\item
|\childredirect| renamed to |\childdocforward| and |\childdocforwardprefix|
and functionality expanded
\end{itemize}

%%%%%%%%%%%%%%%%%%%%%%%%%%%%%%%%%%%%%%%%
\paragraph{v1.0:} 2017/04/27

\begin{itemize}
\item
manual and install package
\item
first version published on CTAN
\end{itemize}

%%%%%%%%%%%%%%%%%%%%%%%%%%%%%%%%%%%%%%%%
\paragraph{v0.6:} 2017/04/26

\begin{itemize}
\item
redirection mechanism added
\end{itemize}

%%%%%%%%%%%%%%%%%%%%%%%%%%%%%%%%%%%%%%%%
\paragraph{v0.5:} 2017/04/26

\begin{itemize}
\item
functionality in definition file
\end{itemize}


%%%%%%%%%%%%%%%%%%%%%%%%%%%%%%%%%%%%%%%%%%%%%%%%%%%%%%%%%%%%%%%%%%%%%%%%%%%%%%%%
%%%%%%%%%%%%%%%%%%%%%%%%%%%%%%%%%%%%%%%%%%%%%%%%%%%%%%%%%%%%%%%%%%%%%%%%%%%%%%%%
%%%%%%%%%%%%%%%%%%%%%%%%%%%%%%%%%%%%%%%%%%%%%%%%%%%%%%%%%%%%%%%%%%%%%%%%%%%%%%%%
\appendix

\settowidth\MacroIndent{\rmfamily\scriptsize 000\ }

 \DocInput{childdoc.dtx}

\end{document}
%</driver>
% \fi
%
% %%%%%%%%%%%%%%%%%%%%%%%%%%%%%%%%%%%%%%%%%%%%%%%%%%%%%%%%%%%%%%%%%%%%%%%%%%%%%%
% %%%%%%%%%%%%%%%%%%%%%%%%%%%%%%%%%%%%%%%%%%%%%%%%%%%%%%%%%%%%%%%%%%%%%%%%%%%%%%
% \section{Sample}
%\iffalse
%<*samplemain>
%\fi
%
% The following presents a sample document
% with two chapters, two parts, a title page,
% a compile flag as well as three forwarding files to set the flag.
% It consists of eight |.tex| files:
% \begin{center}
% \begin{tabular}{ll}
% |cdocsamp.tex|&main file\\
% |cdocsch1.tex|&include file for chapter 1\\
% |cdocsch2.tex|&include file for chapter 2\\
% |cdocspt3.tex|&include file for part 3\\
% |cdocspt4.tex|&include file for part 4\\
% |cdocsdrf.tex|&forwarding file for main file in draft mode\\
% |cdocsfi1.tex|&forwarding file for final version of chapter 1\\
% |cdocsfi2.tex|&forwarding file for final version of chapter 2\\
% \end{tabular}
% \end{center}
% Each of the eight files can be compiled directly by the \LaTeX{} compiler.
%
% %%%%%%%%%%%%%%%%%%%%%%%%%%%%%%%%%%%%%%
% \paragraph{Main File.}
%
% The main file is called |cdocsamp.tex|.
%
% Load the \textsf{childdoc} definitions and
% declare the filename for the main document:
%    \begin{macrocode}
\input{childdoc.def}
\childdocmain{}
%    \end{macrocode}

% Optional override for |\version| flag:
%    \begin{macrocode}
%%\ifchilddoc\else\providecommand{\version}{draft}\fi
%    \end{macrocode}

% Define the default values for the |\version| flag
% (|final| for the main file and |draft| for childs):
%    \begin{macrocode}
\ifchilddoc
\providecommand{\version}{draft}
\else
\providecommand{\version}{final}
\fi
%    \end{macrocode}

% Load the standard document class:
%    \begin{macrocode}
\documentclass[12pt]{article}
%    \end{macrocode}

% Start the document body:
%    \begin{macrocode}
\begin{document}
%    \end{macrocode}

% Declare a title page.
% Print title, part of document being processed and version flag:
%    \begin{macrocode}
\addtocounter{page}{-1}
\begin{center}
{\LARGE\bfseries{}childdoc example\par}
\vspace{1cm}
\ifchilddoc
\ifchilddocmanual part\else chapter\fi:
`\childdocname' of `\childdocjob'\par
\else
main document: `\childdocjob'\par
\fi
version: \version\par
\end{center}
\newpage
%    \end{macrocode}

% Manually include selected file,
% otherwise process as usual:
%    \begin{macrocode}
\ifchilddocmanual
\section*{part `\childdocname'}
\input{\childdocname}
\else
%    \end{macrocode}

% Include the two chapters:
%    \begin{macrocode}
\include{cdocsch1}
\include{cdocsch2}
%    \end{macrocode}

% Include the two parts unless only chapters should be displayed:
%    \begin{macrocode}
\ifchilddoc\else
\section{part three}
\input{cdocspt3}
\section{part four}
\input{cdocspt4}
\fi
%    \end{macrocode}

% Process as usual until here:
%    \begin{macrocode}
\fi
%    \end{macrocode}

% End of document body:
%    \begin{macrocode}
\end{document}
%    \end{macrocode}
%\iffalse
%</samplemain>
%\fi
%
% %%%%%%%%%%%%%%%%%%%%%%%%%%%%%%%%%%%%%%
% \paragraph{Chapter Include Files.}
%
% The include files are called |cdocsch1.tex| and |cdocsch2.tex|.
%
%\iffalse
%<*samplechap1|samplechap2>
%\fi

% Optional override for |\version| flag:
%    \begin{macrocode}
%%\providecommand{\version}{final}
%    \end{macrocode}

% Include the main document:
%    \begin{macrocode}
\input{childdoc.def}
\childdocof{cdocsamp}
%    \end{macrocode}

%\iffalse
%</samplechap1|samplechap2>
%\fi
%
%\iffalse
%<*samplechap1>
%\fi
% Some text for chapter 1:
%    \begin{macrocode}
\section{one}
some text in chapter one
%    \end{macrocode}

%\iffalse
%</samplechap1>
%\fi
% Some text for chapter 2:
%\iffalse
%<*samplechap2>
%\fi
%    \begin{macrocode}
\section{two}
more text in chapter two
%    \end{macrocode}

%\iffalse
%</samplechap2>
%\fi
%
% %%%%%%%%%%%%%%%%%%%%%%%%%%%%%%%%%%%%%%
% \paragraph{Part Include Files.}
%
% The include files are called |cdocspt3.tex| and |cdocspt4.tex|.
%
%\iffalse
%<*samplepart3|samplepart4>
%\fi

% Optional override for |\version| flag:
%    \begin{macrocode}
%%\providecommand{\version}{final}
%    \end{macrocode}

% Include the main document:
%    \begin{macrocode}
\input{childdoc.def}
\childdocby{cdocsamp}
%    \end{macrocode}

%\iffalse
%</samplepart3|samplepart4>
%\fi
%
%\iffalse
%<*samplepart3>
%\fi
% Some text for part 3:
%    \begin{macrocode}
some text in part three
%    \end{macrocode}

%\iffalse
%</samplepart3>
%\fi
% Some text for part 4:
%\iffalse
%<*samplepart4>
%\fi
%    \begin{macrocode}
more text in part four
%    \end{macrocode}

%\iffalse
%</samplepart4>
%\fi
%
% %%%%%%%%%%%%%%%%%%%%%%%%%%%%%%%%%%%%%%
% \paragraph{Forwarding for a Complete Draft.}
%
% The following forwarding file |cdocsdrf.tex|
% compiles the main document in draft mode:
%\iffalse
%<*sampledraft>
%\fi
%    \begin{macrocode}
\def\version{draft}
\input{childdoc.def}
\childdocforward{cdocsamp}
%    \end{macrocode}

%\iffalse
%</sampledraft>
%\fi
%
% %%%%%%%%%%%%%%%%%%%%%%%%%%%%%%%%%%%%%%
% \paragraph{Forwarding for Final Version of the Chapters.}
%
% The following forwarding files |cdocsfn1.tex| and |cdocsfn2.tex|
% (with identical content)
% compile the final versions of the child documents
% |cdocsch1.tex| and |cdocsch2.tex|, respectively:
%\iffalse
%<*samplefinal>
%\fi
%    \begin{macrocode}
\def\version{final}
\input{childdoc.def}
\childdocforwardprefix[cdocsamp]{cdocsfn}{cdocsch}
%    \end{macrocode}

%\iffalse
%</samplefinal>
%\fi
%
% %%%%%%%%%%%%%%%%%%%%%%%%%%%%%%%%%%%%%%
% \paragraph{Command Line Processing.}
%
% The following three command lines generate the output files
% |cdocscld|, |cdocscl1| and |cdocscl2|
% which should be identical to
% |cdocsdrf|, |cdocsch1| and |cdocsfn2|, respectively:
% \begin{center}
% \begin{tabular}{l}
% |latex -jobname cdocscld \|\\
% |  "\def\version{draft}\input{childdoc.def}\childdocforward{cdocsamp}"|\\
% |latex -jobname cdocscl1 \|\\
% |  "\input{childdoc.def}\childdocforward[cdocsamp]{cdocsch1}"|\\
% |latex -jobname cdocscl2 \|\\
% |  "\def\version{final}\input{childdoc.def}\childdocforward{cdocsch2}"|
% \end{tabular}
% \end{center}
% Note that the trailing backslash on each first line
% merely continues the input to the second line
% (for convenient cut ant paste).
% Furthermore, the command |latex| can be replaced by any
% of its alternative versions such as |pdflatex|.
%
% %%%%%%%%%%%%%%%%%%%%%%%%%%%%%%%%%%%%%%%%%%%%%%%%%%%%%%%%%%%%%%%%%%%%%%%%%%%%%%
% %%%%%%%%%%%%%%%%%%%%%%%%%%%%%%%%%%%%%%%%%%%%%%%%%%%%%%%%%%%%%%%%%%%%%%%%%%%%%%
% \section{Implementation}
%\iffalse
%<*package>
%\fi
%
% This section describes the definitions file |childdoc.def|.

% The definitions cannot be loaded using |\usepackage| or |\RequirePackage|
% which has a mechanism to prevent loading a style file more than once.
% When loading the definitions by means of |\input|
% multiple instances have to be prevented manually:
%\iffalse
%This code needs to be before the `\ProvidesFile' directive
%which is defined at the beginning of this file.
%Therefore it is also placed there and commented out here.
%</package>
%<*discard>
%\fi
%    \begin{macrocode}
\ifdefined\childdocmain\endinput\fi
%    \end{macrocode}
%\iffalse
%</discard>
%<*package>
%\fi
%
% \macro{\ifchilddoc}
% \macro{\ifchilddocmanual}
% The conditional |\ifchilddoc| tells whether a
% child (true) or main (false) document is being compiled.
% The conditional |\ifchilddocmanual| tells whether
% the |\includeonly| mechanism is used (false) or
% the selection of child files must be performed manually (true).
% The definitions initialise to false:
%    \begin{macrocode}
\newif\ifchilddoc
\newif\ifchilddocmanual
%    \end{macrocode}

% \macro{\childdocname}
% \macro{\childdocjob}
% The macro |\childdocname| stores the name of the main document
% to be compiled. The macro |\childdocjob| stores the name of
% the document on which the \LaTeX{} compiler was originally invoked.
% The content of |\jobname| cannot be compared
% to filenames specified in the source due to different catcodes.
% The following code rescans |\jobname|, stores the result
% in |\childdocname| and saves a copy in |\childdocjob|:
%    \begin{macrocode}
\edef\childdocname{\scantokens\expandafter{\jobname\noexpand}}
\let\childdocjob\childdocname
%    \end{macrocode}

% \macro{\childdocdisable}
% The macro |\childdocdisable| prevents the main file
% from being processed more than once.
% At this stage, the main document command |\childdocmain|
% is assumed to be called once again where it should do nothing.
% Any subsequent call to it should prevent
% a secondary processing of the main document
% It overwrites the forwarding commands
% |\childdocof| and |\childdocforward|
% with empty macros to prevent further inclusions of the main document:
%    \begin{macrocode}
\newcommand{\childdocdisable}
{
  \renewcommand{\childdocmain}[1]{\renewcommand{\childdocmain}[1]{\endinput}}
  \renewcommand{\childdocof}[1]{}
  \renewcommand{\childdocby}[2][]{}
  \renewcommand{\childdocforward}[2][]{}
  \renewcommand{\childdocdisable}{}
}
%    \end{macrocode}

% \macro{\childdocmain}
% The macro |\childdocmain| is to be called at the top of the main file
% with nothing or the main filename (without extension) as argument.
% First, it breaks loops.
% If the argument is not empty and does not match |\childdocname|
% (which is set by the first inclusion of |childdoc.def|),
% |\ifchilddoc| is set to true, |\includeonly| is applied to the child file
% and |\jobname| is set to the main file
% (for proper handling of |.aux| files):
%    \begin{macrocode}
\newcommand{\childdocmain}[1]
{
  \childdocdisable\childdocmain{}
  \if?#1?\else
    \begingroup
      \def\childdoctmp{#1}
      \ifx\childdoctmp\childdocname
        \def\childdoctmp{}
      \else
        \def\childdoctmp
        {
          \childdoctrue
          \includeonly{\childdocname}
          \def\childdocjob{#1}
          \def\jobname{#1}
        }
      \fi
      \expandafter
    \endgroup
    \childdoctmp
  \fi
}
%    \end{macrocode}

% \macro{\childdocof}
% The command |\childdocof| redirects
% compilation to the main file |#1|.
%    \begin{macrocode}
\newcommand{\childdocof}[1]
{
  \childdocdisable
  \childdoctrue
  \includeonly{\childdocname}
  \def\jobname{#1}
  \def\childdocjob{#1}
  \input{#1}
}
%    \end{macrocode}

% \macro{\childdocby}
% The command |\childdocby| ....
%    \begin{macrocode}
\newcommand{\childdocby}[2][]
{
  \childdocdisable
  \childdoctrue
  \childdocmanualtrue
  \if?#1?\else
    \def\jobname{#2}
  \fi
  \def\childdocjob{#2}
  \input{#2}
  \endinput
}
%    \end{macrocode}

% \macro{\childdocforward}
% The command |\childdocforward| redirects
% compilation to the main file or
% (if the optional argument is given) a child file.
% Parameters are set as if the main file
% or a child file starting with |\childdocof| was compiled.
% Then compilation is handed over to the main file:
%    \begin{macrocode}
\newcommand{\childdocforward}[2][]
{
  \begingroup
    \if?#1?
      \def\childdoctmp
      {
        \def\childdocname{#2}
        \def\childdocjob{#2}
        \def\jobname{#2}
        \input{#2}
        \endinput
      }
    \else
      \def\childdoctmp
      {
        \childdocdisable
        \def\childdocname{#2}
        \childdoctrue
        \includeonly{#2}
        \def\childdocjob{#1}
        \def\jobname{#1}
        \input{#1}
        \endinput
      }
    \fi
    \expandafter
  \endgroup
  \childdoctmp
}
%    \end{macrocode}

% \macro{\childdocforwardprefix}
% The command |\childdocforwardprefix| redirects
% compilation to the main or a child file by means of a pattern.
% The prefix |#1| in the current filename is replaced by |#2|
% and the suffix of the current filename is kept
% (it is assumed that the filename does not contain the substring `|~~~|'
% which is used as a delimiter).
% Compilation is handed over to the new file by |\childdocforward|:
%    \begin{macrocode}
\newcommand{\childdocforwardprefix}[3][]
{
  \begingroup
    \def\childdocextract #2##1~~~{\def\childdoctmp{\childdocforward[#1]{#3##1}}}
    \expandafter\childdocextract\childdocname~~~
    \expandafter
  \endgroup
  \childdoctmp
}
%    \end{macrocode}

% \macro{\childdoc}
% The deprecated macro |\childdoc| is a legacy version of |\childdocmain|:
%    \begin{macrocode}
\newcommand{\childdoc}{\childdocmain}
%    \end{macrocode}

% \macro{\childdocredirect}
% The deprecated macro |\childdocredirect| is a legacy version
% of |\childdocforward| and |\childdocforwardprefix|:
%    \begin{macrocode}
\newcommand{\childdocredirect}[2][]
{
  \begingroup
    \if?#1?
      \def\childdoctmp{\childdocforward{#2}}
    \else
      \def\childdoctmp{\childdocforwardprefix{#1}{#2}}
    \fi
    \expandafter
  \endgroup
  \childdoctmp
}
%    \end{macrocode}

%\iffalse
%</package>
%\fi
%
\endinput
|\\
|\childdocby{|\textit{main}|}|\\
\end{tabular}
\end{center}
%
Both forms have slightly different effects as described above.
The main file is prepared as usual, see \secref{sec:include}.

%%%%%%%%%%%%%%%%%%%%%%%%%%%%%%%%%%%%%%%%%%%%%%%%%%%%%%%%%%%%%%%%%%%%%%%%%%%%%%%%
\subsection{Legacy Detection}
\label{sec:detection}

The directive |\childdocmain| in the main file can detect
whether the complete document or merely a child is to be compiled
even without using the directive |\childdocof|.
This method is deprecated because it is less robust
and there is no compelling reason to use it;
it is merely provided for backward compatibility
and it may be removed in future versions.

If the detection mechanism is to be used,
it is mandatory to correctly specify
the filename of the main file as the argument of |\childdocmain|:
%
\begin{center}
\begin{tabular}{l}
|% \iffalse
%
% childdoc.dtx Copyright (C) 2017-2018 Niklas Beisert
%
% This work may be distributed and/or modified under the
% conditions of the LaTeX Project Public License, either version 1.3
% of this license or (at your option) any later version.
% The latest version of this license is in
%   http://www.latex-project.org/lppl.txt
% and version 1.3 or later is part of all distributions of LaTeX
% version 2005/12/01 or later.
%
% This work has the LPPL maintenance status `maintained'.
%
% The Current Maintainer of this work is Niklas Beisert.
%
% This work consists of the files childdoc.dtx and childdoc.ins
% and the derived files childdoc.def and cdocsamp.tex with
% cdocsch1.tex, cdocsch2.tex, cdocsdrf.tex, cdocsfn1.tex, cdocsfn2.tex.
%
%<package>\ifdefined\childdocmain\endinput\fi
%<package>\ProvidesFile{childdoc.def}[2018/12/30 v2.0 child document driver]
%<samplemain>\ProvidesFile{cdocsamp.tex}[2018/12/30 v2.0 sample for childdoc]
%<*driver>
%\ProvidesFile{childdoc.drv}[2018/12/30 v2.0 childdoc reference manual file]
\PassOptionsToClass{10pt,a4paper}{article}
\documentclass{ltxdoc}

\usepackage[margin=35mm]{geometry}
\usepackage{hyperref}
\usepackage{hyperxmp}
\usepackage[usenames]{color}

\hypersetup{colorlinks=true}
\hypersetup{pdfstartview=FitH}
\hypersetup{pdfpagemode=UseNone}
\hypersetup{pdfsource={}}
\hypersetup{pdflang={en-UK}}
\hypersetup{pdfcopyright={Copyright 2017-2018 Niklas Beisert.
  This work may be distributed and/or modified under the
  conditions of the LaTeX Project Public License, either version 1.3
  of this license or (at your option) any later version.}}
\hypersetup{pdflicenseurl={http://www.latex-project.org/lppl.txt}}
\hypersetup{pdfcontactaddress={ETH Zurich, ITP, HIT K,
  Wolfgang-Pauli-Strasse 27}}
\hypersetup{pdfcontactpostcode={8093}}
\hypersetup{pdfcontactcity={Zurich}}
\hypersetup{pdfcontactcountry={Switzerland}}
\hypersetup{pdfcontactemail={nbeisert@itp.phys.ethz.ch}}
\hypersetup{pdfcontacturl={http://people.phys.ethz.ch/\xmptilde nbeisert/}}

\newcommand{\secref}[1]{\hyperref[#1]{section \ref*{#1}}}

\parskip1ex
\parindent0pt
\let\olditemize\itemize
\def\itemize{\olditemize\parskip0pt}

\begin{document}

\title{The \textsf{childdoc} Package}
\hypersetup{pdftitle={The childdoc Package}}
\author{Niklas Beisert\\[2ex]
  Institut f\"ur Theoretische Physik\\
  Eidgen\"ossische Technische Hochschule Z\"urich\\
  Wolfgang-Pauli-Strasse 27, 8093 Z\"urich, Switzerland\\[1ex]
  \href{mailto:nbeisert@itp.phys.ethz.ch}
  {\texttt{nbeisert@itp.phys.ethz.ch}}}
\hypersetup{pdfauthor={Niklas Beisert}}
\hypersetup{pdfsubject={Manual for the LaTeX2e Package childdoc}}
\date{30 December 2018, \textsf{v2.0}}
\maketitle

\begin{abstract}\noindent
\textsf{childdoc} is a \LaTeXe{} package
that enables the direct compilation
of document sections included by |\include|
to individual files.
\end{abstract}

\begingroup
\parskip0ex
\tableofcontents
\endgroup

%%%%%%%%%%%%%%%%%%%%%%%%%%%%%%%%%%%%%%%%%%%%%%%%%%%%%%%%%%%%%%%%%%%%%%%%%%%%%%%%
%%%%%%%%%%%%%%%%%%%%%%%%%%%%%%%%%%%%%%%%%%%%%%%%%%%%%%%%%%%%%%%%%%%%%%%%%%%%%%%%
\section{Introduction}

\LaTeX{} provides a mechanism to structure a large document (such as a book)
into a main file and several child files (containing the chapters)
using the |\include| command.
This mechanism is beneficial for documents
which span hundreds of pages in order to
make the source file(s) more manageable.
Moreover, compilation can be restricted to
selected child files by means of the |\includeonly| command.
The latter feature can be used to reduce the compilation time while editing
(this was significantly more useful in the earlier days of \LaTeX{})
or to generate a smaller document which is easier to navigate.
Another application of |\includeonly| is to generate
documents consisting of selected parts of the complete document.

However, there are a few drawbacks of the plain |\include| mechanism:
\begin{itemize}
\item
The child files cannot be compiled on their own,
they can only be compiled via the main file.
A naive editing environment
(such as a text editor with an option
to have the current file processed by \LaTeX)
may require one to switch to the main file before compiling;
attempting to compile the child file produces errors.
\item
The main file must be modified (each time)
to adjust the |\includeonly| command
to the present needs. This easily leaves the main file in a messy state.
\item
The generated document will always carry the filename
of the main document. This is inconvenient if
several child files are to be compiled and
to be kept for distribution.
\end{itemize}

The present package provides a simple interface
to make child files individually compilable by \LaTeX{}.
Compiling a child file then has the same effect as compiling
the main file with an |\includeonly| command
to select the appropriate child.
Moreover the generated document will carry the name of the child
rather than the main file.
This resolves all three above issues.

This feature is meant to make the editing of books,
thesis documents and lecture notes somewhat more convenient.
However, the package can also be used efficiently for
composing a series of documents (such as exercise sheets)
which are typically distributed individually.
It then assists the author in generating the individual documents
(potentially in different versions)
as well as a document containing the collected series.
Another application is in developing style files
or other kinds of included material
where compilation of the style file could redirect
to a sample or test file.

%%%%%%%%%%%%%%%%%%%%%%%%%%%%%%%%%%%%%%%%%%%%%%%%%%%%%%%%%%%%%%%%%%%%%%%%%%%%%%%%
%%%%%%%%%%%%%%%%%%%%%%%%%%%%%%%%%%%%%%%%%%%%%%%%%%%%%%%%%%%%%%%%%%%%%%%%%%%%%%%%
\section{Usage}

First of all, the package \textsf{childdoc} is \emph{not} a standard
\LaTeXe{} |.sty| style file! Therefore it needs to be invoked in
a non-standard way.

%%%%%%%%%%%%%%%%%%%%%%%%%%%%%%%%%%%%%%%%%%%%%%%%%%%%%%%%%%%%%%%%%%%%%%%%%%%%%%%%
\subsection{Included Files}
\label{sec:include}

%%%%%%%%%%%%%%%%%%%%%%%%%%%%%%%%%%%%%%%%
\DescribeMacro{\childdocmain}
To use the package, add the commands
\begin{center}
\begin{tabular}{l}
|\input{childdoc.def}|\\
|\childdocmain{}|\\
\end{tabular}
\end{center}
at the very top of the main \LaTeX{} file,
in particular \emph{before} the |\documentclass| statement!
The argument of |\childdocmain| should be left empty
(but it must be present).

%%%%%%%%%%%%%%%%%%%%%%%%%%%%%%%%%%%%%%%%
\DescribeMacro{\childdocof}
Furthermore, add the commands
\begin{center}
\begin{tabular}{l}
|\input{childdoc.def}|\\
|\childdocof{|\textit{main}|}|\\
\end{tabular}
\end{center}
at the top of every child file \textit{child}
which is included by |\include{|\textit{child}|}|
from within the main file
(or at least for those files to be compiled individually).
The argument \textit{main} must be the filename of the main file.

There are a couple of
considerations in setting up the main and child documents:

%%%%%%%%%%%%%%%%%%%%%%%%%%%%%%%%%%%%%%%%
\paragraph{Restrictions.}

Please note the following restrictions:
\begin{itemize}
\item
|\childdocmain| must be called with one argument \textit{main}
to ensure compatibility with earlier version of the package.
It must either be empty (|\childdocmain{}|)
or precisely match the filename of the main file in which it is specified.
See \secref{sec:detection} for further information.
\item
The filename \textit{main} must be specified without the |.tex| extension.
\item
The filename \textit{main} is case sensitive
(even in case-insensitive file systems)
due to internal string comparison.
\item
The argument \textit{main} should be fully expanded, it cannot be a macro.
\item
Subdirectories and special characters should be avoided in filenames.
\item
The command |\childdocmain{|\textit{main}|}| must be followed by a whitespace.
It should not be followed immediately by another command
or by a comment mark `|%|'.
This is because the \TeX{} parser reads the token immediately following
the argument of |\childdocmain| and puts it
at the beginning of every child section;
however, a white\-space is ignored.
\end{itemize}

%%%%%%%%%%%%%%%%%%%%%%%%%%%%%%%%%%%%%%%%
\paragraph{Content of Main File.}

It is advisable to place all content in the child files included by |\include|.
Any output contained in the main file will appear in all child documents
unless suppressed manually;
it cannot be suppressed automatically by the |\includeonly| directive
and thus should normally be avoided.
A method to include some content in the main file
by means of conditional processing is described in \secref{sec:conditional}.

%%%%%%%%%%%%%%%%%%%%%%%%%%%%%%%%%%%%%%%%
\paragraph{Page Numbering.}

When only a part of the document is compiled,
the appropriate numbering of pages
(as well as other status parameters)
is determined from the |.aux| files.
The latter contain information from previous passes.
However this information needs to propagate through
all intermediate child documents.
Therefore the page numbering in child documents may well
be inconsistent until the complete document is compiled at least once.

A useful (if unconventional) way to always ensure a consistent
page numbering is to restart the numbering in each child document
and denote the pages by `\textit{child}|.|\textit{page}'
where \textit{child} represents the chapter/section number of the child file.
This can be achieved by the command
|\numberwithin{page}{|\textit{child}|}|
of the \textsf{amsmath} package
where \textit{child} can be |chapter| or |section|
depending on the chosen structuring.
Alternatively, one can modify the macro |\thepage| appropriately
and reset the counter |page| at the start of each child file.

%%%%%%%%%%%%%%%%%%%%%%%%%%%%%%%%%%%%%%%%%%%%%%%%%%%%%%%%%%%%%%%%%%%%%%%%%%%%%%%%
\subsection{Conditional Processing}
\label{sec:conditional}

The package provides a mechanism to compile different versions
of a document. To customise the versions further some conditional processing
can come in handy to distinguish which version is being compiled.
The package provides two macros to describe the compilation context:

%%%%%%%%%%%%%%%%%%%%%%%%%%%%%%%%%%%%%%%%
\DescribeMacro{\ifchilddoc}
The conditional |\ifchilddoc| distinguishes between the compilation of
child documents and the main document:
%
\begin{center}
|\ifchilddoc |\textit{child-code}| |[|\||else |\textit{main-code}]| \||fi|
\end{center}

%%%%%%%%%%%%%%%%%%%%%%%%%%%%%%%%%%%%%%%%
\DescribeMacro{\childdocname}
\DescribeMacro{\childdocjob}
The macro |\childdocname| contains the filename (without extension)
of the main or child file being processed.
Note that |\childdocjob| will always contain the name of the main file.

%%%%%%%%%%%%%%%%%%%%%%%%%%%%%%%%%%%%%%%%
\paragraph{Title Page.}

Conditional processing can be used to include a title or banner page
in the main document when proper precautions are taken.
Importantly, the code in the main file should ensure that the page counter
(as well as other status parameters which are stored in the |.aux| files)
takes the same value after the conditional processing.
Otherwise the page numbers may take divergent values
depending on which part is compiled.

For example, a title page could be declared by:
%
\begin{center}
\begin{tabular}{l}
|\ifchilddoc\||else|\\
|\addtocounter{page}{-1}|\\
\textit{code for title page}\\
|\newpage|\\
|\||fi|
\end{tabular}
\end{center}
%
A banner page for the child documents can be generated by:
%
\begin{center}
\begin{tabular}{l}
|\ifchilddoc|\\
|\addtocounter{page}{-1}|\\
\textit{code for banner page}\\
|\newpage|\\
|\||fi|
\end{tabular}
\end{center}
%
Here one could write a message such as:
\begin{center}
|This is the part \childdocname{} of \childdocjob{}.|
\end{center}

%%%%%%%%%%%%%%%%%%%%%%%%%%%%%%%%%%%%%%%%%%%%%%%%%%%%%%%%%%%%%%%%%%%%%%%%%%%%%%%%
\subsection{Flags}
\label{sec:flags}

The package makes it easy to generate different versions
of the main or child documents.
To this end compilation flags can be defined
and assigned different default values.
They will be particularly useful in conjunction
with the forwarding mechanism described in \secref{sec:forward}.

For example, it may be useful to have a flag |\version|
which can be set to |draft| or |final|.
The document source will contain some conditional code
depending on the value of |\version|.
Suppose further, the flag should default to |final| for the main file
and to |draft| for child files
which is a natural assignment for editing the document.
This is achieved by placing the following code
in the preamble of the main document
(below the |\childdocmain| directive):
%
\begin{center}
\begin{tabular}{l}
|\ifchilddoc|\\
|\providecommand{\version}{draft}|\\
|\||else|\\
|\providecommand{\version}{final}|\\
|\||fi|
\end{tabular}
\end{center}
%
The definition by |\providecommand| makes sure
that previous definitions are not overwritten.
Further statements |\providecommand{\version}{...}|
can thus be added before the above code to override it.

For the main file, one might add a line
(between |\childdocmain| and the above block)
%
\begin{center}
|%\ifchilddoc\||else\providecommand{\version}{draft}\||fi|
\end{center}
%
which can be uncommented to produce a draft version.
Likewise one can add a line to the very top of a child file
(above the |\childdocof{|\textit{main}|}| directive)
%
\begin{center}
|%\providecommand{\version}{final}|
\end{center}
%
which can be uncommented to produce the final version of this child document.

%%%%%%%%%%%%%%%%%%%%%%%%%%%%%%%%%%%%%%%%%%%%%%%%%%%%%%%%%%%%%%%%%%%%%%%%%%%%%%%%
\subsection{Forwarding}
\label{sec:forward}

Different versions of the main or child documents
using compilation flags as described in \secref{sec:flags}
can be (permanently) stored in different files
for convenient compilation, viewing and distribution.
To this end, the package defines a command
to pass on compilation to a different file:

%%%%%%%%%%%%%%%%%%%%%%%%%%%%%%%%%%%%%%%%
\DescribeMacro{\childdocforward}
The command |\childdocforward| redirects processing to
another source file:
%
\begin{center}
\begin{tabular}{l}
|\input{childdoc.def}|\\
|\childdocforward[|\textit{main}|]{|\textit{dest}|}|\\
\end{tabular}
\end{center}
%
The argument \textit{dest} is the destination file
(without extension).
It should be the main file or one of the child files.
Note that further \textsf{childdoc} directives
such as |\childdocof| and |\childdocforward|
in the indicated file will be processed in this form.
The optional argument \textit{main}
passes on directly to the main file \textit{main}
while pretending to compile the child \textit{dest}.
This form behaves as if \textit{dest}
issues |\childdocof{|\textit{main}|}| right away,
and no further \textsf{childdoc} directives will be processed.

%%%%%%%%%%%%%%%%%%%%%%%%%%%%%%%%%%%%%%%%
\DescribeMacro{\...prefix}
In the alternative form |\childdocforwardprefix|,
%
\begin{center}
\begin{tabular}{l}
|\input{childdoc.def}|\\
|\childdocforwardprefix[|\textit{main}|]{|\textit{prefix}|}{|\textit{dest}|}|
\end{tabular}
\end{center}
%
the destination file is determined by a pattern
depending on the current file:
To make this work, the current file must be called
`{\textit{prefix}\hspace{0.2em}\textit{suffix}}'
with \textit{prefix} matching precisely the argument.
Processing is then passed on to the file
`{\textit{dest}\hspace{0.2em}\textit{suffix}}'.
Surely, the same effect is achieved by
directly specifying the
argument `{\textit{dest}\hspace{0.2em}\textit{suffix}}'
in the first form.
However, that requires to set up a different file
for each child. With the alternative form of the command
all these files can have exactly the same content
which simplifies setting them up and maintaining them.

For example, the following file |draft.tex|
with a compilation flag |\version| as described in \secref{sec:flags}
compiles the main document as a draft:
%
\begin{center}
\begin{tabular}{l}
|\def\version{draft}|\\
|\input{childdoc.def}|\\
|\childdocforward{|\textit{main}|}|
\end{tabular}
\end{center}
%
Likewise, the following files |final|\textit{nn}|.tex|
compile the final version of the child document
|child|\textit{nn}|.tex|:
%
\begin{center}
\begin{tabular}{l}
|\def\version{final}|\\
|\input{childdoc.def}|\\
|\childdocforwardprefix{final}{child}|
\end{tabular}
\end{center}
%

Note that when several versions of a main file and/or of each child file
are to be generated, it may be convenient to set up a |Makefile| or
shell script to automatise the process.

%%%%%%%%%%%%%%%%%%%%%%%%%%%%%%%%%%%%%%%%%%%%%%%%%%%%%%%%%%%%%%%%%%%%%%%%%%%%%%%%
\subsection{Command Line Processing}
\label{sec:commandline}

The effect of redirection files can also be achieved by invoking
the \LaTeX{} compiler with a more elaborate command line.
Most conveniently this should be done as part
of a shell script or a |Makefile|.

When using \textsf{childdoc} in the main file, the following
command lines effectively perform a redirection
(note that depending on the shell being used,
backslashes may have to be doubled: `|\|' $\to$ `|\\|'):
%
\begin{center}
|... -jobname "|\textit{target}|" |\\|"|[\textit{flags}]%
|\input{childdoc.def}\childdocforward[|\textit{main}|]{|\textit{dest}|}"|
\end{center}
%
Here \textit{target} is the name of the output file,
\textit{main} is the name of the main file
and \textit{dest} is the name of the main or child file to be processed
(all filenames without extensions).
The optional argument \textit{main} can be omitted
if \textit{main} matches \textit{dest}.
Optionally, compilation \textit{flags} can be defined via |\def| commands.
This command line makes the \TeX{} engine believe
it is compiling the file \textit{target}
whose content is specified as the latter parameter.
The provided code then forwards the processing to
\textit{main} or \textit{dest} as described in \secref{sec:forward}.

%%%%%%%%%%%%%%%%%%%%%%%%%%%%%%%%%%%%%%%%%%%%%%%%%%%%%%%%%%%%%%%%%%%%%%%%%%%%%%%%
\subsection{Include by Input}
\label{sec:input}

Including child documents by |\include| has some restrictions by design.
Most notably, the content of a child document always occupies
its own set of pages; pages cannot be shared between child documents.
Usually, this behaviour makes perfect sense
because each child document contain an essential part of the document.
However, in some situations it may be desirable to compose
a document from a collection of parts
without having mandatory page breaks between then.
For this case, the package
provides a mechanism to include parts
by |\input| which can also be processed individually.
However, by construction this mechanism
requires manual handling of the content to be output.

%%%%%%%%%%%%%%%%%%%%%%%%%%%%%%%%%%%%%%%%
\DescribeMacro{\ifchilddocmanual}
The main file should be prepared as usual, see \secref{sec:include}.
However, the document body must make a distinction
between processing of an individual part and of the main document, e.g.:
%
\begin{center}
\begin{tabular}{l}
|\ifchilddocmanual|\\
|\input{\childdocname}|\\
|\||else|\\
\textit{document body with }|\input{|\textit{part}|}|\\
|\||fi|
\end{tabular}
\end{center}
%
The conditional |\ifchilddocmanual| is true whenever
a part to be included by |\input| is being compiled,
and the name of the part is stored in |\childdocname|.

%%%%%%%%%%%%%%%%%%%%%%%%%%%%%%%%%%%%%%%%
\DescribeMacro{\childdocby}
Each part to be included by |\input| should start with:
%
\begin{center}
\begin{tabular}{l}
|\input{childdoc.def}|\\
|\childdocby{|\textit{main}|}|\\
\end{tabular}
\end{center}
%
The directive |\childdocby| is similar to |\childdocof|
described in \secref{sec:include},
but the subsequent selection of content must be done manually.
To that end, both |\ifchilddoc| and |\ifchilddocmanual|
will be true upon processing of a part,
and the name of the part is stored in |\childdocname|.
Note that |\jobname| will be set to the filename of the current part
so that each part receives an individual |.aux| file
that does not interfere with the |.aux| file(s) of the main document.
This behaviour can be altered by the alternative form
|\childdocby[*]{|\textit{main}|}| (with a non-empty optional argument)
which uses the |.aux| file of the main document
by setting |\jobname| to \textit{main}.

%%%%%%%%%%%%%%%%%%%%%%%%%%%%%%%%%%%%%%%%%%%%%%%%%%%%%%%%%%%%%%%%%%%%%%%%%%%%%%%%
\subsection{Driver Development}
\label{sec:driver}

The \textsf{childdoc} mechanism can also be use for the development
of definition files such as \LaTeX{} styles or classes.
This case differs from the above setup with multiple parts
included by |\include| in that no |\includeonly| should be invoked.
This can be achieved by starting the include file
(before |\ProvidesPackage|) with:
%
\begin{center}
\begin{tabular}{l}
|\input{childdoc.def}|\\
|\childdocforward{|\textit{main}|}|\\
\end{tabular}
\end{center}
%
or alternatively with:
%
\begin{center}
\begin{tabular}{l}
|\input{childdoc.def}|\\
|\childdocby{|\textit{main}|}|\\
\end{tabular}
\end{center}
%
Both forms have slightly different effects as described above.
The main file is prepared as usual, see \secref{sec:include}.

%%%%%%%%%%%%%%%%%%%%%%%%%%%%%%%%%%%%%%%%%%%%%%%%%%%%%%%%%%%%%%%%%%%%%%%%%%%%%%%%
\subsection{Legacy Detection}
\label{sec:detection}

The directive |\childdocmain| in the main file can detect
whether the complete document or merely a child is to be compiled
even without using the directive |\childdocof|.
This method is deprecated because it is less robust
and there is no compelling reason to use it;
it is merely provided for backward compatibility
and it may be removed in future versions.

If the detection mechanism is to be used,
it is mandatory to correctly specify
the filename of the main file as the argument of |\childdocmain|:
%
\begin{center}
\begin{tabular}{l}
|\input{childdoc.def}|\\
|\childdocmain{|\textit{main}|}|\\
\end{tabular}
\end{center}
%
If |\jobname| does not match the argument \textit{main} of |\childdocmain|,
it is assumed that |\jobname| points to the child file to be compiled.
When using |\childdocmain| with the main file specified as argument,
it suffices to start a child file
with just |\input{|\textit{main}|}|
without loading of the package and using |\childdocof|.
If instead all processing is done
with the appropriate \textsf{childdoc} directives,
the argument of \textit{main} of |\childdocmain| can be empty.

An alternative version of the command line processing described
in \secref{sec:commandline} using the detection mechanism reads:
%
\begin{center}
|... -jobname "|\textit{target}|" "|[\textit{flags}]%
[|\def\jobname{|\textit{dest}|}|]|\input{|\textit{main}|}"|
\end{center}

%%%%%%%%%%%%%%%%%%%%%%%%%%%%%%%%%%%%%%%%%%%%%%%%%%%%%%%%%%%%%%%%%%%%%%%%%%%%%%%%
\subsection{Manual Code}
\label{sec:manual}

In case one cannot be certain whether the definitions file |childdoc.def|
is installed on the target \TeX{} distribution
and one prefers not to ship it,
it is conceivable to paste a few relevant commands into the sources.

To that end, drop all statements |\input{childdoc.def}|
and perform the replacements as outlined below.
Instead of |\childdocmain{|\textit{main}|}| add the following code
to the top of the main file:
%
\begin{center}
\begin{tabular}{l}
|\||ifdefined\childdocname\endinput\||fi\newif\ifchilddoc|\\
|\edef\childdocname{\scantokens\expandafter{\jobname\noexpand}}|\\
|\def\childdocmain{|\textit{main}|}\||ifx\childdocmain\childdocname\||else|\\
|\childdoctrue\includeonly{\childdocname}\let\jobname\childdocmain\||fi|\\
\end{tabular}
\end{center}
%
Instead of |\childdocof{|\textit{main}|}| just include the main file
at the top of each child file:
%
\begin{center}
|\input{|\textit{main}|}|
\end{center}
%
A simple redirection |\childdocforward{|\textit{dest}|}| is achieved by:
%
\begin{center}
|\def\jobname{|\textit{dest}|}\input{\jobname}|
\end{center}
%
The redirection with prefix
|\childdocforwardprefix[|\textit{prefix}|]{|\textit{dest}|}|
is accomplished by:
%
\begin{center}
\begin{tabular}{l}
|{\edef\jobname{\scantokens\expandafter{\jobname\noexpand}}|\\
|\def\redirectjob |\textit{prefix}|#1~~~{\gdef\jobname{|\textit{dest}|#1}}|\\
|\expandafter\redirectjob\jobname~~~}\input{\jobname}|
\end{tabular}
\end{center}

In an alternative approach,
child documents can be compiled by a specific command line
without additional code or specific definitions:
%
\begin{center}
|... -jobname "|\textit{target}|" "|[\textit{flags}]%
|\includeonly{|\textit{dest}|}\input{|\textit{main}|}"|
\end{center}
%

%%%%%%%%%%%%%%%%%%%%%%%%%%%%%%%%%%%%%%%%%%%%%%%%%%%%%%%%%%%%%%%%%%%%%%%%%%%%%%%%
%%%%%%%%%%%%%%%%%%%%%%%%%%%%%%%%%%%%%%%%%%%%%%%%%%%%%%%%%%%%%%%%%%%%%%%%%%%%%%%%
\section{Information}

%%%%%%%%%%%%%%%%%%%%%%%%%%%%%%%%%%%%%%%%%%%%%%%%%%%%%%%%%%%%%%%%%%%%%%%%%%%%%%%%
\subsection{Copyright}

Copyright \copyright{} 2017--2018 Niklas Beisert

This work may be distributed and/or modified under the
conditions of the \LaTeX{} Project Public License, either version 1.3
of this license or (at your option) any later version.
The latest version of this license is in
  \url{http://www.latex-project.org/lppl.txt}
and version 1.3 or later is part of all distributions of \LaTeX{}
version 2005/12/01 or later.

This work has the LPPL maintenance status `maintained'.

The Current Maintainer of this work is Niklas Beisert.

This work consists of the files |README.txt|, |childdoc.ins| and |childdoc.dtx|
as well as the derived files |childdoc.def|, |cdocsamp.tex|
with |cdocsch1.tex|, |cdocsch2.tex|, |cdocspt3.tex|, |cdocspt4.tex|,
|cdocsdrf.tex|, |cdocsfn1.tex|, |cdocsfn2.tex|
as well as |childdoc.pdf|.

%%%%%%%%%%%%%%%%%%%%%%%%%%%%%%%%%%%%%%%%%%%%%%%%%%%%%%%%%%%%%%%%%%%%%%%%%%%%%%%%
\subsection{Files and Installation}

The package consists of the files:
%
\begin{center}
\begin{tabular}{ll}
    |README.txt|   & readme file \\
    |childdoc.ins| & installation file \\
    |childdoc.dtx| & source file \\
    |childdoc.def| & definition file \\
    |cdocsamp.tex| & sample main file \\
    |cdocsch1.tex| & sample include file \\
    |cdocsch2.tex| & sample include file \\
    |cdocspt3.tex| & sample part file \\
    |cdocspt4.tex| & sample part file \\
    |cdocsdrf.tex| & sample redirection file \\
    |cdocsfn1.tex| & sample redirection file \\
    |cdocsfn2.tex| & sample redirection file \\
    |childdoc.pdf| & manual
\end{tabular}
\end{center}
%
The distribution consists of the files
|README.txt|, |childdoc.ins| and |childdoc.dtx|.
%
\begin{itemize}
\item
Run (pdf)\LaTeX{} on |childdoc.dtx|
to compile the manual |childdoc.pdf| (this file).
\item
Run \LaTeX{} on |childdoc.ins| to create the definitions file |childdoc.def|
and the sample |cdocsamp.tex| with include files
|cdocsch1.tex|, |cdocsch2.tex|, |cdocspt3.tex|, |cdocspt4.tex|,
|cdocsdrf.tex|, |cdocsfn1.tex|, |cdocsfn2.tex|.
Then copy the file |childdoc.def| to an appropriate directory of your \LaTeX{}
distribution, e.g.\ \textit{texmf-root}|/tex/latex/childdoc|.
\end{itemize}

%%%%%%%%%%%%%%%%%%%%%%%%%%%%%%%%%%%%%%%%%%%%%%%%%%%%%%%%%%%%%%%%%%%%%%%%%%%%%%%%
\subsection{Related CTAN Packages}

There are several other packages which offer a similar functionality:
%
\begin{itemize}
\item
The packages
\href{http://ctan.org/pkg/docmute}{\textsf{docmute}},
\href{http://ctan.org/pkg/includex}{\textsf{includex}} and
\href{http://ctan.org/pkg/standalone}{\textsf{standalone}}
provide commands to include only the document body of
a child file thus allowing both files to be compiled individually.
\item
The packages \href{http://ctan.org/pkg/subdocs}{\textsf{subdocs}}
and \href{http://ctan.org/pkg/subfiles}{\textsf{subfiles}}
provide structures in which the main and child documents can be
encapsulated and allowing them to be compiled individually.
The inclusion mechanism is different from the conventional |\include|.
\item
The package \href{http://ctan.org/pkg/combine}{\textsf{combine}}
is an elaborate solution to combine several documents into one.
\end{itemize}
%
See also the CTAN topic \href{http://ctan.org/topic/subdocs}{\textsf{subdocs}}
for further related packages.
The present package differs from the above solutions in that
a document structure constructed with the conventional |\include| mechanism
just needs two extra commands at the top of every file
such that all constituent files can be compiled individually.

%%%%%%%%%%%%%%%%%%%%%%%%%%%%%%%%%%%%%%%%%%%%%%%%%%%%%%%%%%%%%%%%%%%%%%%%%%%%%%%%
%\subsection{Feature Suggestions}
%
%The following is a list of features which may be useful for future
%versions of this package:
%%
%\begin{itemize}
%\item
%\ldots
%\end{itemize}

%%%%%%%%%%%%%%%%%%%%%%%%%%%%%%%%%%%%%%%%%%%%%%%%%%%%%%%%%%%%%%%%%%%%%%%%%%%%%%%%
\subsection{Revision History}

%%%%%%%%%%%%%%%%%%%%%%%%%%%%%%%%%%%%%%%%
\paragraph{v2.0:} 2018/12/30

\begin{itemize}
\item
immediate forward processing
\item
added |\childdocby| mechanism
\item
manual restructured
\end{itemize}

%%%%%%%%%%%%%%%%%%%%%%%%%%%%%%%%%%%%%%%%
\paragraph{v1.6:} 2018/01/17

\begin{itemize}
\item
application for development of include files
\item
corrections to manual
\end{itemize}

%%%%%%%%%%%%%%%%%%%%%%%%%%%%%%%%%%%%%%%%
\paragraph{v1.5:} 2017/05/21

\begin{itemize}
\item
more complete structuring introduced
\item
|\childdocof| introduced
\item
|\childdoc| renamed to |\childdocmain|
\item
|\childredirect| renamed to |\childdocforward| and |\childdocforwardprefix|
and functionality expanded
\end{itemize}

%%%%%%%%%%%%%%%%%%%%%%%%%%%%%%%%%%%%%%%%
\paragraph{v1.0:} 2017/04/27

\begin{itemize}
\item
manual and install package
\item
first version published on CTAN
\end{itemize}

%%%%%%%%%%%%%%%%%%%%%%%%%%%%%%%%%%%%%%%%
\paragraph{v0.6:} 2017/04/26

\begin{itemize}
\item
redirection mechanism added
\end{itemize}

%%%%%%%%%%%%%%%%%%%%%%%%%%%%%%%%%%%%%%%%
\paragraph{v0.5:} 2017/04/26

\begin{itemize}
\item
functionality in definition file
\end{itemize}


%%%%%%%%%%%%%%%%%%%%%%%%%%%%%%%%%%%%%%%%%%%%%%%%%%%%%%%%%%%%%%%%%%%%%%%%%%%%%%%%
%%%%%%%%%%%%%%%%%%%%%%%%%%%%%%%%%%%%%%%%%%%%%%%%%%%%%%%%%%%%%%%%%%%%%%%%%%%%%%%%
%%%%%%%%%%%%%%%%%%%%%%%%%%%%%%%%%%%%%%%%%%%%%%%%%%%%%%%%%%%%%%%%%%%%%%%%%%%%%%%%
\appendix

\settowidth\MacroIndent{\rmfamily\scriptsize 000\ }

 \DocInput{childdoc.dtx}

\end{document}
%</driver>
% \fi
%
% %%%%%%%%%%%%%%%%%%%%%%%%%%%%%%%%%%%%%%%%%%%%%%%%%%%%%%%%%%%%%%%%%%%%%%%%%%%%%%
% %%%%%%%%%%%%%%%%%%%%%%%%%%%%%%%%%%%%%%%%%%%%%%%%%%%%%%%%%%%%%%%%%%%%%%%%%%%%%%
% \section{Sample}
%\iffalse
%<*samplemain>
%\fi
%
% The following presents a sample document
% with two chapters, two parts, a title page,
% a compile flag as well as three forwarding files to set the flag.
% It consists of eight |.tex| files:
% \begin{center}
% \begin{tabular}{ll}
% |cdocsamp.tex|&main file\\
% |cdocsch1.tex|&include file for chapter 1\\
% |cdocsch2.tex|&include file for chapter 2\\
% |cdocspt3.tex|&include file for part 3\\
% |cdocspt4.tex|&include file for part 4\\
% |cdocsdrf.tex|&forwarding file for main file in draft mode\\
% |cdocsfi1.tex|&forwarding file for final version of chapter 1\\
% |cdocsfi2.tex|&forwarding file for final version of chapter 2\\
% \end{tabular}
% \end{center}
% Each of the eight files can be compiled directly by the \LaTeX{} compiler.
%
% %%%%%%%%%%%%%%%%%%%%%%%%%%%%%%%%%%%%%%
% \paragraph{Main File.}
%
% The main file is called |cdocsamp.tex|.
%
% Load the \textsf{childdoc} definitions and
% declare the filename for the main document:
%    \begin{macrocode}
\input{childdoc.def}
\childdocmain{}
%    \end{macrocode}

% Optional override for |\version| flag:
%    \begin{macrocode}
%%\ifchilddoc\else\providecommand{\version}{draft}\fi
%    \end{macrocode}

% Define the default values for the |\version| flag
% (|final| for the main file and |draft| for childs):
%    \begin{macrocode}
\ifchilddoc
\providecommand{\version}{draft}
\else
\providecommand{\version}{final}
\fi
%    \end{macrocode}

% Load the standard document class:
%    \begin{macrocode}
\documentclass[12pt]{article}
%    \end{macrocode}

% Start the document body:
%    \begin{macrocode}
\begin{document}
%    \end{macrocode}

% Declare a title page.
% Print title, part of document being processed and version flag:
%    \begin{macrocode}
\addtocounter{page}{-1}
\begin{center}
{\LARGE\bfseries{}childdoc example\par}
\vspace{1cm}
\ifchilddoc
\ifchilddocmanual part\else chapter\fi:
`\childdocname' of `\childdocjob'\par
\else
main document: `\childdocjob'\par
\fi
version: \version\par
\end{center}
\newpage
%    \end{macrocode}

% Manually include selected file,
% otherwise process as usual:
%    \begin{macrocode}
\ifchilddocmanual
\section*{part `\childdocname'}
\input{\childdocname}
\else
%    \end{macrocode}

% Include the two chapters:
%    \begin{macrocode}
\include{cdocsch1}
\include{cdocsch2}
%    \end{macrocode}

% Include the two parts unless only chapters should be displayed:
%    \begin{macrocode}
\ifchilddoc\else
\section{part three}
\input{cdocspt3}
\section{part four}
\input{cdocspt4}
\fi
%    \end{macrocode}

% Process as usual until here:
%    \begin{macrocode}
\fi
%    \end{macrocode}

% End of document body:
%    \begin{macrocode}
\end{document}
%    \end{macrocode}
%\iffalse
%</samplemain>
%\fi
%
% %%%%%%%%%%%%%%%%%%%%%%%%%%%%%%%%%%%%%%
% \paragraph{Chapter Include Files.}
%
% The include files are called |cdocsch1.tex| and |cdocsch2.tex|.
%
%\iffalse
%<*samplechap1|samplechap2>
%\fi

% Optional override for |\version| flag:
%    \begin{macrocode}
%%\providecommand{\version}{final}
%    \end{macrocode}

% Include the main document:
%    \begin{macrocode}
\input{childdoc.def}
\childdocof{cdocsamp}
%    \end{macrocode}

%\iffalse
%</samplechap1|samplechap2>
%\fi
%
%\iffalse
%<*samplechap1>
%\fi
% Some text for chapter 1:
%    \begin{macrocode}
\section{one}
some text in chapter one
%    \end{macrocode}

%\iffalse
%</samplechap1>
%\fi
% Some text for chapter 2:
%\iffalse
%<*samplechap2>
%\fi
%    \begin{macrocode}
\section{two}
more text in chapter two
%    \end{macrocode}

%\iffalse
%</samplechap2>
%\fi
%
% %%%%%%%%%%%%%%%%%%%%%%%%%%%%%%%%%%%%%%
% \paragraph{Part Include Files.}
%
% The include files are called |cdocspt3.tex| and |cdocspt4.tex|.
%
%\iffalse
%<*samplepart3|samplepart4>
%\fi

% Optional override for |\version| flag:
%    \begin{macrocode}
%%\providecommand{\version}{final}
%    \end{macrocode}

% Include the main document:
%    \begin{macrocode}
\input{childdoc.def}
\childdocby{cdocsamp}
%    \end{macrocode}

%\iffalse
%</samplepart3|samplepart4>
%\fi
%
%\iffalse
%<*samplepart3>
%\fi
% Some text for part 3:
%    \begin{macrocode}
some text in part three
%    \end{macrocode}

%\iffalse
%</samplepart3>
%\fi
% Some text for part 4:
%\iffalse
%<*samplepart4>
%\fi
%    \begin{macrocode}
more text in part four
%    \end{macrocode}

%\iffalse
%</samplepart4>
%\fi
%
% %%%%%%%%%%%%%%%%%%%%%%%%%%%%%%%%%%%%%%
% \paragraph{Forwarding for a Complete Draft.}
%
% The following forwarding file |cdocsdrf.tex|
% compiles the main document in draft mode:
%\iffalse
%<*sampledraft>
%\fi
%    \begin{macrocode}
\def\version{draft}
\input{childdoc.def}
\childdocforward{cdocsamp}
%    \end{macrocode}

%\iffalse
%</sampledraft>
%\fi
%
% %%%%%%%%%%%%%%%%%%%%%%%%%%%%%%%%%%%%%%
% \paragraph{Forwarding for Final Version of the Chapters.}
%
% The following forwarding files |cdocsfn1.tex| and |cdocsfn2.tex|
% (with identical content)
% compile the final versions of the child documents
% |cdocsch1.tex| and |cdocsch2.tex|, respectively:
%\iffalse
%<*samplefinal>
%\fi
%    \begin{macrocode}
\def\version{final}
\input{childdoc.def}
\childdocforwardprefix[cdocsamp]{cdocsfn}{cdocsch}
%    \end{macrocode}

%\iffalse
%</samplefinal>
%\fi
%
% %%%%%%%%%%%%%%%%%%%%%%%%%%%%%%%%%%%%%%
% \paragraph{Command Line Processing.}
%
% The following three command lines generate the output files
% |cdocscld|, |cdocscl1| and |cdocscl2|
% which should be identical to
% |cdocsdrf|, |cdocsch1| and |cdocsfn2|, respectively:
% \begin{center}
% \begin{tabular}{l}
% |latex -jobname cdocscld \|\\
% |  "\def\version{draft}\input{childdoc.def}\childdocforward{cdocsamp}"|\\
% |latex -jobname cdocscl1 \|\\
% |  "\input{childdoc.def}\childdocforward[cdocsamp]{cdocsch1}"|\\
% |latex -jobname cdocscl2 \|\\
% |  "\def\version{final}\input{childdoc.def}\childdocforward{cdocsch2}"|
% \end{tabular}
% \end{center}
% Note that the trailing backslash on each first line
% merely continues the input to the second line
% (for convenient cut ant paste).
% Furthermore, the command |latex| can be replaced by any
% of its alternative versions such as |pdflatex|.
%
% %%%%%%%%%%%%%%%%%%%%%%%%%%%%%%%%%%%%%%%%%%%%%%%%%%%%%%%%%%%%%%%%%%%%%%%%%%%%%%
% %%%%%%%%%%%%%%%%%%%%%%%%%%%%%%%%%%%%%%%%%%%%%%%%%%%%%%%%%%%%%%%%%%%%%%%%%%%%%%
% \section{Implementation}
%\iffalse
%<*package>
%\fi
%
% This section describes the definitions file |childdoc.def|.

% The definitions cannot be loaded using |\usepackage| or |\RequirePackage|
% which has a mechanism to prevent loading a style file more than once.
% When loading the definitions by means of |\input|
% multiple instances have to be prevented manually:
%\iffalse
%This code needs to be before the `\ProvidesFile' directive
%which is defined at the beginning of this file.
%Therefore it is also placed there and commented out here.
%</package>
%<*discard>
%\fi
%    \begin{macrocode}
\ifdefined\childdocmain\endinput\fi
%    \end{macrocode}
%\iffalse
%</discard>
%<*package>
%\fi
%
% \macro{\ifchilddoc}
% \macro{\ifchilddocmanual}
% The conditional |\ifchilddoc| tells whether a
% child (true) or main (false) document is being compiled.
% The conditional |\ifchilddocmanual| tells whether
% the |\includeonly| mechanism is used (false) or
% the selection of child files must be performed manually (true).
% The definitions initialise to false:
%    \begin{macrocode}
\newif\ifchilddoc
\newif\ifchilddocmanual
%    \end{macrocode}

% \macro{\childdocname}
% \macro{\childdocjob}
% The macro |\childdocname| stores the name of the main document
% to be compiled. The macro |\childdocjob| stores the name of
% the document on which the \LaTeX{} compiler was originally invoked.
% The content of |\jobname| cannot be compared
% to filenames specified in the source due to different catcodes.
% The following code rescans |\jobname|, stores the result
% in |\childdocname| and saves a copy in |\childdocjob|:
%    \begin{macrocode}
\edef\childdocname{\scantokens\expandafter{\jobname\noexpand}}
\let\childdocjob\childdocname
%    \end{macrocode}

% \macro{\childdocdisable}
% The macro |\childdocdisable| prevents the main file
% from being processed more than once.
% At this stage, the main document command |\childdocmain|
% is assumed to be called once again where it should do nothing.
% Any subsequent call to it should prevent
% a secondary processing of the main document
% It overwrites the forwarding commands
% |\childdocof| and |\childdocforward|
% with empty macros to prevent further inclusions of the main document:
%    \begin{macrocode}
\newcommand{\childdocdisable}
{
  \renewcommand{\childdocmain}[1]{\renewcommand{\childdocmain}[1]{\endinput}}
  \renewcommand{\childdocof}[1]{}
  \renewcommand{\childdocby}[2][]{}
  \renewcommand{\childdocforward}[2][]{}
  \renewcommand{\childdocdisable}{}
}
%    \end{macrocode}

% \macro{\childdocmain}
% The macro |\childdocmain| is to be called at the top of the main file
% with nothing or the main filename (without extension) as argument.
% First, it breaks loops.
% If the argument is not empty and does not match |\childdocname|
% (which is set by the first inclusion of |childdoc.def|),
% |\ifchilddoc| is set to true, |\includeonly| is applied to the child file
% and |\jobname| is set to the main file
% (for proper handling of |.aux| files):
%    \begin{macrocode}
\newcommand{\childdocmain}[1]
{
  \childdocdisable\childdocmain{}
  \if?#1?\else
    \begingroup
      \def\childdoctmp{#1}
      \ifx\childdoctmp\childdocname
        \def\childdoctmp{}
      \else
        \def\childdoctmp
        {
          \childdoctrue
          \includeonly{\childdocname}
          \def\childdocjob{#1}
          \def\jobname{#1}
        }
      \fi
      \expandafter
    \endgroup
    \childdoctmp
  \fi
}
%    \end{macrocode}

% \macro{\childdocof}
% The command |\childdocof| redirects
% compilation to the main file |#1|.
%    \begin{macrocode}
\newcommand{\childdocof}[1]
{
  \childdocdisable
  \childdoctrue
  \includeonly{\childdocname}
  \def\jobname{#1}
  \def\childdocjob{#1}
  \input{#1}
}
%    \end{macrocode}

% \macro{\childdocby}
% The command |\childdocby| ....
%    \begin{macrocode}
\newcommand{\childdocby}[2][]
{
  \childdocdisable
  \childdoctrue
  \childdocmanualtrue
  \if?#1?\else
    \def\jobname{#2}
  \fi
  \def\childdocjob{#2}
  \input{#2}
  \endinput
}
%    \end{macrocode}

% \macro{\childdocforward}
% The command |\childdocforward| redirects
% compilation to the main file or
% (if the optional argument is given) a child file.
% Parameters are set as if the main file
% or a child file starting with |\childdocof| was compiled.
% Then compilation is handed over to the main file:
%    \begin{macrocode}
\newcommand{\childdocforward}[2][]
{
  \begingroup
    \if?#1?
      \def\childdoctmp
      {
        \def\childdocname{#2}
        \def\childdocjob{#2}
        \def\jobname{#2}
        \input{#2}
        \endinput
      }
    \else
      \def\childdoctmp
      {
        \childdocdisable
        \def\childdocname{#2}
        \childdoctrue
        \includeonly{#2}
        \def\childdocjob{#1}
        \def\jobname{#1}
        \input{#1}
        \endinput
      }
    \fi
    \expandafter
  \endgroup
  \childdoctmp
}
%    \end{macrocode}

% \macro{\childdocforwardprefix}
% The command |\childdocforwardprefix| redirects
% compilation to the main or a child file by means of a pattern.
% The prefix |#1| in the current filename is replaced by |#2|
% and the suffix of the current filename is kept
% (it is assumed that the filename does not contain the substring `|~~~|'
% which is used as a delimiter).
% Compilation is handed over to the new file by |\childdocforward|:
%    \begin{macrocode}
\newcommand{\childdocforwardprefix}[3][]
{
  \begingroup
    \def\childdocextract #2##1~~~{\def\childdoctmp{\childdocforward[#1]{#3##1}}}
    \expandafter\childdocextract\childdocname~~~
    \expandafter
  \endgroup
  \childdoctmp
}
%    \end{macrocode}

% \macro{\childdoc}
% The deprecated macro |\childdoc| is a legacy version of |\childdocmain|:
%    \begin{macrocode}
\newcommand{\childdoc}{\childdocmain}
%    \end{macrocode}

% \macro{\childdocredirect}
% The deprecated macro |\childdocredirect| is a legacy version
% of |\childdocforward| and |\childdocforwardprefix|:
%    \begin{macrocode}
\newcommand{\childdocredirect}[2][]
{
  \begingroup
    \if?#1?
      \def\childdoctmp{\childdocforward{#2}}
    \else
      \def\childdoctmp{\childdocforwardprefix{#1}{#2}}
    \fi
    \expandafter
  \endgroup
  \childdoctmp
}
%    \end{macrocode}

%\iffalse
%</package>
%\fi
%
\endinput
|\\
|\childdocmain{|\textit{main}|}|\\
\end{tabular}
\end{center}
%
If |\jobname| does not match the argument \textit{main} of |\childdocmain|,
it is assumed that |\jobname| points to the child file to be compiled.
When using |\childdocmain| with the main file specified as argument,
it suffices to start a child file
with just |\input{|\textit{main}|}|
without loading of the package and using |\childdocof|.
If instead all processing is done
with the appropriate \textsf{childdoc} directives,
the argument of \textit{main} of |\childdocmain| can be empty.

An alternative version of the command line processing described
in \secref{sec:commandline} using the detection mechanism reads:
%
\begin{center}
|... -jobname "|\textit{target}|" "|[\textit{flags}]%
[|\def\jobname{|\textit{dest}|}|]|\input{|\textit{main}|}"|
\end{center}

%%%%%%%%%%%%%%%%%%%%%%%%%%%%%%%%%%%%%%%%%%%%%%%%%%%%%%%%%%%%%%%%%%%%%%%%%%%%%%%%
\subsection{Manual Code}
\label{sec:manual}

In case one cannot be certain whether the definitions file |childdoc.def|
is installed on the target \TeX{} distribution
and one prefers not to ship it,
it is conceivable to paste a few relevant commands into the sources.

To that end, drop all statements |% \iffalse
%
% childdoc.dtx Copyright (C) 2017-2018 Niklas Beisert
%
% This work may be distributed and/or modified under the
% conditions of the LaTeX Project Public License, either version 1.3
% of this license or (at your option) any later version.
% The latest version of this license is in
%   http://www.latex-project.org/lppl.txt
% and version 1.3 or later is part of all distributions of LaTeX
% version 2005/12/01 or later.
%
% This work has the LPPL maintenance status `maintained'.
%
% The Current Maintainer of this work is Niklas Beisert.
%
% This work consists of the files childdoc.dtx and childdoc.ins
% and the derived files childdoc.def and cdocsamp.tex with
% cdocsch1.tex, cdocsch2.tex, cdocsdrf.tex, cdocsfn1.tex, cdocsfn2.tex.
%
%<package>\ifdefined\childdocmain\endinput\fi
%<package>\ProvidesFile{childdoc.def}[2018/12/30 v2.0 child document driver]
%<samplemain>\ProvidesFile{cdocsamp.tex}[2018/12/30 v2.0 sample for childdoc]
%<*driver>
%\ProvidesFile{childdoc.drv}[2018/12/30 v2.0 childdoc reference manual file]
\PassOptionsToClass{10pt,a4paper}{article}
\documentclass{ltxdoc}

\usepackage[margin=35mm]{geometry}
\usepackage{hyperref}
\usepackage{hyperxmp}
\usepackage[usenames]{color}

\hypersetup{colorlinks=true}
\hypersetup{pdfstartview=FitH}
\hypersetup{pdfpagemode=UseNone}
\hypersetup{pdfsource={}}
\hypersetup{pdflang={en-UK}}
\hypersetup{pdfcopyright={Copyright 2017-2018 Niklas Beisert.
  This work may be distributed and/or modified under the
  conditions of the LaTeX Project Public License, either version 1.3
  of this license or (at your option) any later version.}}
\hypersetup{pdflicenseurl={http://www.latex-project.org/lppl.txt}}
\hypersetup{pdfcontactaddress={ETH Zurich, ITP, HIT K,
  Wolfgang-Pauli-Strasse 27}}
\hypersetup{pdfcontactpostcode={8093}}
\hypersetup{pdfcontactcity={Zurich}}
\hypersetup{pdfcontactcountry={Switzerland}}
\hypersetup{pdfcontactemail={nbeisert@itp.phys.ethz.ch}}
\hypersetup{pdfcontacturl={http://people.phys.ethz.ch/\xmptilde nbeisert/}}

\newcommand{\secref}[1]{\hyperref[#1]{section \ref*{#1}}}

\parskip1ex
\parindent0pt
\let\olditemize\itemize
\def\itemize{\olditemize\parskip0pt}

\begin{document}

\title{The \textsf{childdoc} Package}
\hypersetup{pdftitle={The childdoc Package}}
\author{Niklas Beisert\\[2ex]
  Institut f\"ur Theoretische Physik\\
  Eidgen\"ossische Technische Hochschule Z\"urich\\
  Wolfgang-Pauli-Strasse 27, 8093 Z\"urich, Switzerland\\[1ex]
  \href{mailto:nbeisert@itp.phys.ethz.ch}
  {\texttt{nbeisert@itp.phys.ethz.ch}}}
\hypersetup{pdfauthor={Niklas Beisert}}
\hypersetup{pdfsubject={Manual for the LaTeX2e Package childdoc}}
\date{30 December 2018, \textsf{v2.0}}
\maketitle

\begin{abstract}\noindent
\textsf{childdoc} is a \LaTeXe{} package
that enables the direct compilation
of document sections included by |\include|
to individual files.
\end{abstract}

\begingroup
\parskip0ex
\tableofcontents
\endgroup

%%%%%%%%%%%%%%%%%%%%%%%%%%%%%%%%%%%%%%%%%%%%%%%%%%%%%%%%%%%%%%%%%%%%%%%%%%%%%%%%
%%%%%%%%%%%%%%%%%%%%%%%%%%%%%%%%%%%%%%%%%%%%%%%%%%%%%%%%%%%%%%%%%%%%%%%%%%%%%%%%
\section{Introduction}

\LaTeX{} provides a mechanism to structure a large document (such as a book)
into a main file and several child files (containing the chapters)
using the |\include| command.
This mechanism is beneficial for documents
which span hundreds of pages in order to
make the source file(s) more manageable.
Moreover, compilation can be restricted to
selected child files by means of the |\includeonly| command.
The latter feature can be used to reduce the compilation time while editing
(this was significantly more useful in the earlier days of \LaTeX{})
or to generate a smaller document which is easier to navigate.
Another application of |\includeonly| is to generate
documents consisting of selected parts of the complete document.

However, there are a few drawbacks of the plain |\include| mechanism:
\begin{itemize}
\item
The child files cannot be compiled on their own,
they can only be compiled via the main file.
A naive editing environment
(such as a text editor with an option
to have the current file processed by \LaTeX)
may require one to switch to the main file before compiling;
attempting to compile the child file produces errors.
\item
The main file must be modified (each time)
to adjust the |\includeonly| command
to the present needs. This easily leaves the main file in a messy state.
\item
The generated document will always carry the filename
of the main document. This is inconvenient if
several child files are to be compiled and
to be kept for distribution.
\end{itemize}

The present package provides a simple interface
to make child files individually compilable by \LaTeX{}.
Compiling a child file then has the same effect as compiling
the main file with an |\includeonly| command
to select the appropriate child.
Moreover the generated document will carry the name of the child
rather than the main file.
This resolves all three above issues.

This feature is meant to make the editing of books,
thesis documents and lecture notes somewhat more convenient.
However, the package can also be used efficiently for
composing a series of documents (such as exercise sheets)
which are typically distributed individually.
It then assists the author in generating the individual documents
(potentially in different versions)
as well as a document containing the collected series.
Another application is in developing style files
or other kinds of included material
where compilation of the style file could redirect
to a sample or test file.

%%%%%%%%%%%%%%%%%%%%%%%%%%%%%%%%%%%%%%%%%%%%%%%%%%%%%%%%%%%%%%%%%%%%%%%%%%%%%%%%
%%%%%%%%%%%%%%%%%%%%%%%%%%%%%%%%%%%%%%%%%%%%%%%%%%%%%%%%%%%%%%%%%%%%%%%%%%%%%%%%
\section{Usage}

First of all, the package \textsf{childdoc} is \emph{not} a standard
\LaTeXe{} |.sty| style file! Therefore it needs to be invoked in
a non-standard way.

%%%%%%%%%%%%%%%%%%%%%%%%%%%%%%%%%%%%%%%%%%%%%%%%%%%%%%%%%%%%%%%%%%%%%%%%%%%%%%%%
\subsection{Included Files}
\label{sec:include}

%%%%%%%%%%%%%%%%%%%%%%%%%%%%%%%%%%%%%%%%
\DescribeMacro{\childdocmain}
To use the package, add the commands
\begin{center}
\begin{tabular}{l}
|\input{childdoc.def}|\\
|\childdocmain{}|\\
\end{tabular}
\end{center}
at the very top of the main \LaTeX{} file,
in particular \emph{before} the |\documentclass| statement!
The argument of |\childdocmain| should be left empty
(but it must be present).

%%%%%%%%%%%%%%%%%%%%%%%%%%%%%%%%%%%%%%%%
\DescribeMacro{\childdocof}
Furthermore, add the commands
\begin{center}
\begin{tabular}{l}
|\input{childdoc.def}|\\
|\childdocof{|\textit{main}|}|\\
\end{tabular}
\end{center}
at the top of every child file \textit{child}
which is included by |\include{|\textit{child}|}|
from within the main file
(or at least for those files to be compiled individually).
The argument \textit{main} must be the filename of the main file.

There are a couple of
considerations in setting up the main and child documents:

%%%%%%%%%%%%%%%%%%%%%%%%%%%%%%%%%%%%%%%%
\paragraph{Restrictions.}

Please note the following restrictions:
\begin{itemize}
\item
|\childdocmain| must be called with one argument \textit{main}
to ensure compatibility with earlier version of the package.
It must either be empty (|\childdocmain{}|)
or precisely match the filename of the main file in which it is specified.
See \secref{sec:detection} for further information.
\item
The filename \textit{main} must be specified without the |.tex| extension.
\item
The filename \textit{main} is case sensitive
(even in case-insensitive file systems)
due to internal string comparison.
\item
The argument \textit{main} should be fully expanded, it cannot be a macro.
\item
Subdirectories and special characters should be avoided in filenames.
\item
The command |\childdocmain{|\textit{main}|}| must be followed by a whitespace.
It should not be followed immediately by another command
or by a comment mark `|%|'.
This is because the \TeX{} parser reads the token immediately following
the argument of |\childdocmain| and puts it
at the beginning of every child section;
however, a white\-space is ignored.
\end{itemize}

%%%%%%%%%%%%%%%%%%%%%%%%%%%%%%%%%%%%%%%%
\paragraph{Content of Main File.}

It is advisable to place all content in the child files included by |\include|.
Any output contained in the main file will appear in all child documents
unless suppressed manually;
it cannot be suppressed automatically by the |\includeonly| directive
and thus should normally be avoided.
A method to include some content in the main file
by means of conditional processing is described in \secref{sec:conditional}.

%%%%%%%%%%%%%%%%%%%%%%%%%%%%%%%%%%%%%%%%
\paragraph{Page Numbering.}

When only a part of the document is compiled,
the appropriate numbering of pages
(as well as other status parameters)
is determined from the |.aux| files.
The latter contain information from previous passes.
However this information needs to propagate through
all intermediate child documents.
Therefore the page numbering in child documents may well
be inconsistent until the complete document is compiled at least once.

A useful (if unconventional) way to always ensure a consistent
page numbering is to restart the numbering in each child document
and denote the pages by `\textit{child}|.|\textit{page}'
where \textit{child} represents the chapter/section number of the child file.
This can be achieved by the command
|\numberwithin{page}{|\textit{child}|}|
of the \textsf{amsmath} package
where \textit{child} can be |chapter| or |section|
depending on the chosen structuring.
Alternatively, one can modify the macro |\thepage| appropriately
and reset the counter |page| at the start of each child file.

%%%%%%%%%%%%%%%%%%%%%%%%%%%%%%%%%%%%%%%%%%%%%%%%%%%%%%%%%%%%%%%%%%%%%%%%%%%%%%%%
\subsection{Conditional Processing}
\label{sec:conditional}

The package provides a mechanism to compile different versions
of a document. To customise the versions further some conditional processing
can come in handy to distinguish which version is being compiled.
The package provides two macros to describe the compilation context:

%%%%%%%%%%%%%%%%%%%%%%%%%%%%%%%%%%%%%%%%
\DescribeMacro{\ifchilddoc}
The conditional |\ifchilddoc| distinguishes between the compilation of
child documents and the main document:
%
\begin{center}
|\ifchilddoc |\textit{child-code}| |[|\||else |\textit{main-code}]| \||fi|
\end{center}

%%%%%%%%%%%%%%%%%%%%%%%%%%%%%%%%%%%%%%%%
\DescribeMacro{\childdocname}
\DescribeMacro{\childdocjob}
The macro |\childdocname| contains the filename (without extension)
of the main or child file being processed.
Note that |\childdocjob| will always contain the name of the main file.

%%%%%%%%%%%%%%%%%%%%%%%%%%%%%%%%%%%%%%%%
\paragraph{Title Page.}

Conditional processing can be used to include a title or banner page
in the main document when proper precautions are taken.
Importantly, the code in the main file should ensure that the page counter
(as well as other status parameters which are stored in the |.aux| files)
takes the same value after the conditional processing.
Otherwise the page numbers may take divergent values
depending on which part is compiled.

For example, a title page could be declared by:
%
\begin{center}
\begin{tabular}{l}
|\ifchilddoc\||else|\\
|\addtocounter{page}{-1}|\\
\textit{code for title page}\\
|\newpage|\\
|\||fi|
\end{tabular}
\end{center}
%
A banner page for the child documents can be generated by:
%
\begin{center}
\begin{tabular}{l}
|\ifchilddoc|\\
|\addtocounter{page}{-1}|\\
\textit{code for banner page}\\
|\newpage|\\
|\||fi|
\end{tabular}
\end{center}
%
Here one could write a message such as:
\begin{center}
|This is the part \childdocname{} of \childdocjob{}.|
\end{center}

%%%%%%%%%%%%%%%%%%%%%%%%%%%%%%%%%%%%%%%%%%%%%%%%%%%%%%%%%%%%%%%%%%%%%%%%%%%%%%%%
\subsection{Flags}
\label{sec:flags}

The package makes it easy to generate different versions
of the main or child documents.
To this end compilation flags can be defined
and assigned different default values.
They will be particularly useful in conjunction
with the forwarding mechanism described in \secref{sec:forward}.

For example, it may be useful to have a flag |\version|
which can be set to |draft| or |final|.
The document source will contain some conditional code
depending on the value of |\version|.
Suppose further, the flag should default to |final| for the main file
and to |draft| for child files
which is a natural assignment for editing the document.
This is achieved by placing the following code
in the preamble of the main document
(below the |\childdocmain| directive):
%
\begin{center}
\begin{tabular}{l}
|\ifchilddoc|\\
|\providecommand{\version}{draft}|\\
|\||else|\\
|\providecommand{\version}{final}|\\
|\||fi|
\end{tabular}
\end{center}
%
The definition by |\providecommand| makes sure
that previous definitions are not overwritten.
Further statements |\providecommand{\version}{...}|
can thus be added before the above code to override it.

For the main file, one might add a line
(between |\childdocmain| and the above block)
%
\begin{center}
|%\ifchilddoc\||else\providecommand{\version}{draft}\||fi|
\end{center}
%
which can be uncommented to produce a draft version.
Likewise one can add a line to the very top of a child file
(above the |\childdocof{|\textit{main}|}| directive)
%
\begin{center}
|%\providecommand{\version}{final}|
\end{center}
%
which can be uncommented to produce the final version of this child document.

%%%%%%%%%%%%%%%%%%%%%%%%%%%%%%%%%%%%%%%%%%%%%%%%%%%%%%%%%%%%%%%%%%%%%%%%%%%%%%%%
\subsection{Forwarding}
\label{sec:forward}

Different versions of the main or child documents
using compilation flags as described in \secref{sec:flags}
can be (permanently) stored in different files
for convenient compilation, viewing and distribution.
To this end, the package defines a command
to pass on compilation to a different file:

%%%%%%%%%%%%%%%%%%%%%%%%%%%%%%%%%%%%%%%%
\DescribeMacro{\childdocforward}
The command |\childdocforward| redirects processing to
another source file:
%
\begin{center}
\begin{tabular}{l}
|\input{childdoc.def}|\\
|\childdocforward[|\textit{main}|]{|\textit{dest}|}|\\
\end{tabular}
\end{center}
%
The argument \textit{dest} is the destination file
(without extension).
It should be the main file or one of the child files.
Note that further \textsf{childdoc} directives
such as |\childdocof| and |\childdocforward|
in the indicated file will be processed in this form.
The optional argument \textit{main}
passes on directly to the main file \textit{main}
while pretending to compile the child \textit{dest}.
This form behaves as if \textit{dest}
issues |\childdocof{|\textit{main}|}| right away,
and no further \textsf{childdoc} directives will be processed.

%%%%%%%%%%%%%%%%%%%%%%%%%%%%%%%%%%%%%%%%
\DescribeMacro{\...prefix}
In the alternative form |\childdocforwardprefix|,
%
\begin{center}
\begin{tabular}{l}
|\input{childdoc.def}|\\
|\childdocforwardprefix[|\textit{main}|]{|\textit{prefix}|}{|\textit{dest}|}|
\end{tabular}
\end{center}
%
the destination file is determined by a pattern
depending on the current file:
To make this work, the current file must be called
`{\textit{prefix}\hspace{0.2em}\textit{suffix}}'
with \textit{prefix} matching precisely the argument.
Processing is then passed on to the file
`{\textit{dest}\hspace{0.2em}\textit{suffix}}'.
Surely, the same effect is achieved by
directly specifying the
argument `{\textit{dest}\hspace{0.2em}\textit{suffix}}'
in the first form.
However, that requires to set up a different file
for each child. With the alternative form of the command
all these files can have exactly the same content
which simplifies setting them up and maintaining them.

For example, the following file |draft.tex|
with a compilation flag |\version| as described in \secref{sec:flags}
compiles the main document as a draft:
%
\begin{center}
\begin{tabular}{l}
|\def\version{draft}|\\
|\input{childdoc.def}|\\
|\childdocforward{|\textit{main}|}|
\end{tabular}
\end{center}
%
Likewise, the following files |final|\textit{nn}|.tex|
compile the final version of the child document
|child|\textit{nn}|.tex|:
%
\begin{center}
\begin{tabular}{l}
|\def\version{final}|\\
|\input{childdoc.def}|\\
|\childdocforwardprefix{final}{child}|
\end{tabular}
\end{center}
%

Note that when several versions of a main file and/or of each child file
are to be generated, it may be convenient to set up a |Makefile| or
shell script to automatise the process.

%%%%%%%%%%%%%%%%%%%%%%%%%%%%%%%%%%%%%%%%%%%%%%%%%%%%%%%%%%%%%%%%%%%%%%%%%%%%%%%%
\subsection{Command Line Processing}
\label{sec:commandline}

The effect of redirection files can also be achieved by invoking
the \LaTeX{} compiler with a more elaborate command line.
Most conveniently this should be done as part
of a shell script or a |Makefile|.

When using \textsf{childdoc} in the main file, the following
command lines effectively perform a redirection
(note that depending on the shell being used,
backslashes may have to be doubled: `|\|' $\to$ `|\\|'):
%
\begin{center}
|... -jobname "|\textit{target}|" |\\|"|[\textit{flags}]%
|\input{childdoc.def}\childdocforward[|\textit{main}|]{|\textit{dest}|}"|
\end{center}
%
Here \textit{target} is the name of the output file,
\textit{main} is the name of the main file
and \textit{dest} is the name of the main or child file to be processed
(all filenames without extensions).
The optional argument \textit{main} can be omitted
if \textit{main} matches \textit{dest}.
Optionally, compilation \textit{flags} can be defined via |\def| commands.
This command line makes the \TeX{} engine believe
it is compiling the file \textit{target}
whose content is specified as the latter parameter.
The provided code then forwards the processing to
\textit{main} or \textit{dest} as described in \secref{sec:forward}.

%%%%%%%%%%%%%%%%%%%%%%%%%%%%%%%%%%%%%%%%%%%%%%%%%%%%%%%%%%%%%%%%%%%%%%%%%%%%%%%%
\subsection{Include by Input}
\label{sec:input}

Including child documents by |\include| has some restrictions by design.
Most notably, the content of a child document always occupies
its own set of pages; pages cannot be shared between child documents.
Usually, this behaviour makes perfect sense
because each child document contain an essential part of the document.
However, in some situations it may be desirable to compose
a document from a collection of parts
without having mandatory page breaks between then.
For this case, the package
provides a mechanism to include parts
by |\input| which can also be processed individually.
However, by construction this mechanism
requires manual handling of the content to be output.

%%%%%%%%%%%%%%%%%%%%%%%%%%%%%%%%%%%%%%%%
\DescribeMacro{\ifchilddocmanual}
The main file should be prepared as usual, see \secref{sec:include}.
However, the document body must make a distinction
between processing of an individual part and of the main document, e.g.:
%
\begin{center}
\begin{tabular}{l}
|\ifchilddocmanual|\\
|\input{\childdocname}|\\
|\||else|\\
\textit{document body with }|\input{|\textit{part}|}|\\
|\||fi|
\end{tabular}
\end{center}
%
The conditional |\ifchilddocmanual| is true whenever
a part to be included by |\input| is being compiled,
and the name of the part is stored in |\childdocname|.

%%%%%%%%%%%%%%%%%%%%%%%%%%%%%%%%%%%%%%%%
\DescribeMacro{\childdocby}
Each part to be included by |\input| should start with:
%
\begin{center}
\begin{tabular}{l}
|\input{childdoc.def}|\\
|\childdocby{|\textit{main}|}|\\
\end{tabular}
\end{center}
%
The directive |\childdocby| is similar to |\childdocof|
described in \secref{sec:include},
but the subsequent selection of content must be done manually.
To that end, both |\ifchilddoc| and |\ifchilddocmanual|
will be true upon processing of a part,
and the name of the part is stored in |\childdocname|.
Note that |\jobname| will be set to the filename of the current part
so that each part receives an individual |.aux| file
that does not interfere with the |.aux| file(s) of the main document.
This behaviour can be altered by the alternative form
|\childdocby[*]{|\textit{main}|}| (with a non-empty optional argument)
which uses the |.aux| file of the main document
by setting |\jobname| to \textit{main}.

%%%%%%%%%%%%%%%%%%%%%%%%%%%%%%%%%%%%%%%%%%%%%%%%%%%%%%%%%%%%%%%%%%%%%%%%%%%%%%%%
\subsection{Driver Development}
\label{sec:driver}

The \textsf{childdoc} mechanism can also be use for the development
of definition files such as \LaTeX{} styles or classes.
This case differs from the above setup with multiple parts
included by |\include| in that no |\includeonly| should be invoked.
This can be achieved by starting the include file
(before |\ProvidesPackage|) with:
%
\begin{center}
\begin{tabular}{l}
|\input{childdoc.def}|\\
|\childdocforward{|\textit{main}|}|\\
\end{tabular}
\end{center}
%
or alternatively with:
%
\begin{center}
\begin{tabular}{l}
|\input{childdoc.def}|\\
|\childdocby{|\textit{main}|}|\\
\end{tabular}
\end{center}
%
Both forms have slightly different effects as described above.
The main file is prepared as usual, see \secref{sec:include}.

%%%%%%%%%%%%%%%%%%%%%%%%%%%%%%%%%%%%%%%%%%%%%%%%%%%%%%%%%%%%%%%%%%%%%%%%%%%%%%%%
\subsection{Legacy Detection}
\label{sec:detection}

The directive |\childdocmain| in the main file can detect
whether the complete document or merely a child is to be compiled
even without using the directive |\childdocof|.
This method is deprecated because it is less robust
and there is no compelling reason to use it;
it is merely provided for backward compatibility
and it may be removed in future versions.

If the detection mechanism is to be used,
it is mandatory to correctly specify
the filename of the main file as the argument of |\childdocmain|:
%
\begin{center}
\begin{tabular}{l}
|\input{childdoc.def}|\\
|\childdocmain{|\textit{main}|}|\\
\end{tabular}
\end{center}
%
If |\jobname| does not match the argument \textit{main} of |\childdocmain|,
it is assumed that |\jobname| points to the child file to be compiled.
When using |\childdocmain| with the main file specified as argument,
it suffices to start a child file
with just |\input{|\textit{main}|}|
without loading of the package and using |\childdocof|.
If instead all processing is done
with the appropriate \textsf{childdoc} directives,
the argument of \textit{main} of |\childdocmain| can be empty.

An alternative version of the command line processing described
in \secref{sec:commandline} using the detection mechanism reads:
%
\begin{center}
|... -jobname "|\textit{target}|" "|[\textit{flags}]%
[|\def\jobname{|\textit{dest}|}|]|\input{|\textit{main}|}"|
\end{center}

%%%%%%%%%%%%%%%%%%%%%%%%%%%%%%%%%%%%%%%%%%%%%%%%%%%%%%%%%%%%%%%%%%%%%%%%%%%%%%%%
\subsection{Manual Code}
\label{sec:manual}

In case one cannot be certain whether the definitions file |childdoc.def|
is installed on the target \TeX{} distribution
and one prefers not to ship it,
it is conceivable to paste a few relevant commands into the sources.

To that end, drop all statements |\input{childdoc.def}|
and perform the replacements as outlined below.
Instead of |\childdocmain{|\textit{main}|}| add the following code
to the top of the main file:
%
\begin{center}
\begin{tabular}{l}
|\||ifdefined\childdocname\endinput\||fi\newif\ifchilddoc|\\
|\edef\childdocname{\scantokens\expandafter{\jobname\noexpand}}|\\
|\def\childdocmain{|\textit{main}|}\||ifx\childdocmain\childdocname\||else|\\
|\childdoctrue\includeonly{\childdocname}\let\jobname\childdocmain\||fi|\\
\end{tabular}
\end{center}
%
Instead of |\childdocof{|\textit{main}|}| just include the main file
at the top of each child file:
%
\begin{center}
|\input{|\textit{main}|}|
\end{center}
%
A simple redirection |\childdocforward{|\textit{dest}|}| is achieved by:
%
\begin{center}
|\def\jobname{|\textit{dest}|}\input{\jobname}|
\end{center}
%
The redirection with prefix
|\childdocforwardprefix[|\textit{prefix}|]{|\textit{dest}|}|
is accomplished by:
%
\begin{center}
\begin{tabular}{l}
|{\edef\jobname{\scantokens\expandafter{\jobname\noexpand}}|\\
|\def\redirectjob |\textit{prefix}|#1~~~{\gdef\jobname{|\textit{dest}|#1}}|\\
|\expandafter\redirectjob\jobname~~~}\input{\jobname}|
\end{tabular}
\end{center}

In an alternative approach,
child documents can be compiled by a specific command line
without additional code or specific definitions:
%
\begin{center}
|... -jobname "|\textit{target}|" "|[\textit{flags}]%
|\includeonly{|\textit{dest}|}\input{|\textit{main}|}"|
\end{center}
%

%%%%%%%%%%%%%%%%%%%%%%%%%%%%%%%%%%%%%%%%%%%%%%%%%%%%%%%%%%%%%%%%%%%%%%%%%%%%%%%%
%%%%%%%%%%%%%%%%%%%%%%%%%%%%%%%%%%%%%%%%%%%%%%%%%%%%%%%%%%%%%%%%%%%%%%%%%%%%%%%%
\section{Information}

%%%%%%%%%%%%%%%%%%%%%%%%%%%%%%%%%%%%%%%%%%%%%%%%%%%%%%%%%%%%%%%%%%%%%%%%%%%%%%%%
\subsection{Copyright}

Copyright \copyright{} 2017--2018 Niklas Beisert

This work may be distributed and/or modified under the
conditions of the \LaTeX{} Project Public License, either version 1.3
of this license or (at your option) any later version.
The latest version of this license is in
  \url{http://www.latex-project.org/lppl.txt}
and version 1.3 or later is part of all distributions of \LaTeX{}
version 2005/12/01 or later.

This work has the LPPL maintenance status `maintained'.

The Current Maintainer of this work is Niklas Beisert.

This work consists of the files |README.txt|, |childdoc.ins| and |childdoc.dtx|
as well as the derived files |childdoc.def|, |cdocsamp.tex|
with |cdocsch1.tex|, |cdocsch2.tex|, |cdocspt3.tex|, |cdocspt4.tex|,
|cdocsdrf.tex|, |cdocsfn1.tex|, |cdocsfn2.tex|
as well as |childdoc.pdf|.

%%%%%%%%%%%%%%%%%%%%%%%%%%%%%%%%%%%%%%%%%%%%%%%%%%%%%%%%%%%%%%%%%%%%%%%%%%%%%%%%
\subsection{Files and Installation}

The package consists of the files:
%
\begin{center}
\begin{tabular}{ll}
    |README.txt|   & readme file \\
    |childdoc.ins| & installation file \\
    |childdoc.dtx| & source file \\
    |childdoc.def| & definition file \\
    |cdocsamp.tex| & sample main file \\
    |cdocsch1.tex| & sample include file \\
    |cdocsch2.tex| & sample include file \\
    |cdocspt3.tex| & sample part file \\
    |cdocspt4.tex| & sample part file \\
    |cdocsdrf.tex| & sample redirection file \\
    |cdocsfn1.tex| & sample redirection file \\
    |cdocsfn2.tex| & sample redirection file \\
    |childdoc.pdf| & manual
\end{tabular}
\end{center}
%
The distribution consists of the files
|README.txt|, |childdoc.ins| and |childdoc.dtx|.
%
\begin{itemize}
\item
Run (pdf)\LaTeX{} on |childdoc.dtx|
to compile the manual |childdoc.pdf| (this file).
\item
Run \LaTeX{} on |childdoc.ins| to create the definitions file |childdoc.def|
and the sample |cdocsamp.tex| with include files
|cdocsch1.tex|, |cdocsch2.tex|, |cdocspt3.tex|, |cdocspt4.tex|,
|cdocsdrf.tex|, |cdocsfn1.tex|, |cdocsfn2.tex|.
Then copy the file |childdoc.def| to an appropriate directory of your \LaTeX{}
distribution, e.g.\ \textit{texmf-root}|/tex/latex/childdoc|.
\end{itemize}

%%%%%%%%%%%%%%%%%%%%%%%%%%%%%%%%%%%%%%%%%%%%%%%%%%%%%%%%%%%%%%%%%%%%%%%%%%%%%%%%
\subsection{Related CTAN Packages}

There are several other packages which offer a similar functionality:
%
\begin{itemize}
\item
The packages
\href{http://ctan.org/pkg/docmute}{\textsf{docmute}},
\href{http://ctan.org/pkg/includex}{\textsf{includex}} and
\href{http://ctan.org/pkg/standalone}{\textsf{standalone}}
provide commands to include only the document body of
a child file thus allowing both files to be compiled individually.
\item
The packages \href{http://ctan.org/pkg/subdocs}{\textsf{subdocs}}
and \href{http://ctan.org/pkg/subfiles}{\textsf{subfiles}}
provide structures in which the main and child documents can be
encapsulated and allowing them to be compiled individually.
The inclusion mechanism is different from the conventional |\include|.
\item
The package \href{http://ctan.org/pkg/combine}{\textsf{combine}}
is an elaborate solution to combine several documents into one.
\end{itemize}
%
See also the CTAN topic \href{http://ctan.org/topic/subdocs}{\textsf{subdocs}}
for further related packages.
The present package differs from the above solutions in that
a document structure constructed with the conventional |\include| mechanism
just needs two extra commands at the top of every file
such that all constituent files can be compiled individually.

%%%%%%%%%%%%%%%%%%%%%%%%%%%%%%%%%%%%%%%%%%%%%%%%%%%%%%%%%%%%%%%%%%%%%%%%%%%%%%%%
%\subsection{Feature Suggestions}
%
%The following is a list of features which may be useful for future
%versions of this package:
%%
%\begin{itemize}
%\item
%\ldots
%\end{itemize}

%%%%%%%%%%%%%%%%%%%%%%%%%%%%%%%%%%%%%%%%%%%%%%%%%%%%%%%%%%%%%%%%%%%%%%%%%%%%%%%%
\subsection{Revision History}

%%%%%%%%%%%%%%%%%%%%%%%%%%%%%%%%%%%%%%%%
\paragraph{v2.0:} 2018/12/30

\begin{itemize}
\item
immediate forward processing
\item
added |\childdocby| mechanism
\item
manual restructured
\end{itemize}

%%%%%%%%%%%%%%%%%%%%%%%%%%%%%%%%%%%%%%%%
\paragraph{v1.6:} 2018/01/17

\begin{itemize}
\item
application for development of include files
\item
corrections to manual
\end{itemize}

%%%%%%%%%%%%%%%%%%%%%%%%%%%%%%%%%%%%%%%%
\paragraph{v1.5:} 2017/05/21

\begin{itemize}
\item
more complete structuring introduced
\item
|\childdocof| introduced
\item
|\childdoc| renamed to |\childdocmain|
\item
|\childredirect| renamed to |\childdocforward| and |\childdocforwardprefix|
and functionality expanded
\end{itemize}

%%%%%%%%%%%%%%%%%%%%%%%%%%%%%%%%%%%%%%%%
\paragraph{v1.0:} 2017/04/27

\begin{itemize}
\item
manual and install package
\item
first version published on CTAN
\end{itemize}

%%%%%%%%%%%%%%%%%%%%%%%%%%%%%%%%%%%%%%%%
\paragraph{v0.6:} 2017/04/26

\begin{itemize}
\item
redirection mechanism added
\end{itemize}

%%%%%%%%%%%%%%%%%%%%%%%%%%%%%%%%%%%%%%%%
\paragraph{v0.5:} 2017/04/26

\begin{itemize}
\item
functionality in definition file
\end{itemize}


%%%%%%%%%%%%%%%%%%%%%%%%%%%%%%%%%%%%%%%%%%%%%%%%%%%%%%%%%%%%%%%%%%%%%%%%%%%%%%%%
%%%%%%%%%%%%%%%%%%%%%%%%%%%%%%%%%%%%%%%%%%%%%%%%%%%%%%%%%%%%%%%%%%%%%%%%%%%%%%%%
%%%%%%%%%%%%%%%%%%%%%%%%%%%%%%%%%%%%%%%%%%%%%%%%%%%%%%%%%%%%%%%%%%%%%%%%%%%%%%%%
\appendix

\settowidth\MacroIndent{\rmfamily\scriptsize 000\ }

 \DocInput{childdoc.dtx}

\end{document}
%</driver>
% \fi
%
% %%%%%%%%%%%%%%%%%%%%%%%%%%%%%%%%%%%%%%%%%%%%%%%%%%%%%%%%%%%%%%%%%%%%%%%%%%%%%%
% %%%%%%%%%%%%%%%%%%%%%%%%%%%%%%%%%%%%%%%%%%%%%%%%%%%%%%%%%%%%%%%%%%%%%%%%%%%%%%
% \section{Sample}
%\iffalse
%<*samplemain>
%\fi
%
% The following presents a sample document
% with two chapters, two parts, a title page,
% a compile flag as well as three forwarding files to set the flag.
% It consists of eight |.tex| files:
% \begin{center}
% \begin{tabular}{ll}
% |cdocsamp.tex|&main file\\
% |cdocsch1.tex|&include file for chapter 1\\
% |cdocsch2.tex|&include file for chapter 2\\
% |cdocspt3.tex|&include file for part 3\\
% |cdocspt4.tex|&include file for part 4\\
% |cdocsdrf.tex|&forwarding file for main file in draft mode\\
% |cdocsfi1.tex|&forwarding file for final version of chapter 1\\
% |cdocsfi2.tex|&forwarding file for final version of chapter 2\\
% \end{tabular}
% \end{center}
% Each of the eight files can be compiled directly by the \LaTeX{} compiler.
%
% %%%%%%%%%%%%%%%%%%%%%%%%%%%%%%%%%%%%%%
% \paragraph{Main File.}
%
% The main file is called |cdocsamp.tex|.
%
% Load the \textsf{childdoc} definitions and
% declare the filename for the main document:
%    \begin{macrocode}
\input{childdoc.def}
\childdocmain{}
%    \end{macrocode}

% Optional override for |\version| flag:
%    \begin{macrocode}
%%\ifchilddoc\else\providecommand{\version}{draft}\fi
%    \end{macrocode}

% Define the default values for the |\version| flag
% (|final| for the main file and |draft| for childs):
%    \begin{macrocode}
\ifchilddoc
\providecommand{\version}{draft}
\else
\providecommand{\version}{final}
\fi
%    \end{macrocode}

% Load the standard document class:
%    \begin{macrocode}
\documentclass[12pt]{article}
%    \end{macrocode}

% Start the document body:
%    \begin{macrocode}
\begin{document}
%    \end{macrocode}

% Declare a title page.
% Print title, part of document being processed and version flag:
%    \begin{macrocode}
\addtocounter{page}{-1}
\begin{center}
{\LARGE\bfseries{}childdoc example\par}
\vspace{1cm}
\ifchilddoc
\ifchilddocmanual part\else chapter\fi:
`\childdocname' of `\childdocjob'\par
\else
main document: `\childdocjob'\par
\fi
version: \version\par
\end{center}
\newpage
%    \end{macrocode}

% Manually include selected file,
% otherwise process as usual:
%    \begin{macrocode}
\ifchilddocmanual
\section*{part `\childdocname'}
\input{\childdocname}
\else
%    \end{macrocode}

% Include the two chapters:
%    \begin{macrocode}
\include{cdocsch1}
\include{cdocsch2}
%    \end{macrocode}

% Include the two parts unless only chapters should be displayed:
%    \begin{macrocode}
\ifchilddoc\else
\section{part three}
\input{cdocspt3}
\section{part four}
\input{cdocspt4}
\fi
%    \end{macrocode}

% Process as usual until here:
%    \begin{macrocode}
\fi
%    \end{macrocode}

% End of document body:
%    \begin{macrocode}
\end{document}
%    \end{macrocode}
%\iffalse
%</samplemain>
%\fi
%
% %%%%%%%%%%%%%%%%%%%%%%%%%%%%%%%%%%%%%%
% \paragraph{Chapter Include Files.}
%
% The include files are called |cdocsch1.tex| and |cdocsch2.tex|.
%
%\iffalse
%<*samplechap1|samplechap2>
%\fi

% Optional override for |\version| flag:
%    \begin{macrocode}
%%\providecommand{\version}{final}
%    \end{macrocode}

% Include the main document:
%    \begin{macrocode}
\input{childdoc.def}
\childdocof{cdocsamp}
%    \end{macrocode}

%\iffalse
%</samplechap1|samplechap2>
%\fi
%
%\iffalse
%<*samplechap1>
%\fi
% Some text for chapter 1:
%    \begin{macrocode}
\section{one}
some text in chapter one
%    \end{macrocode}

%\iffalse
%</samplechap1>
%\fi
% Some text for chapter 2:
%\iffalse
%<*samplechap2>
%\fi
%    \begin{macrocode}
\section{two}
more text in chapter two
%    \end{macrocode}

%\iffalse
%</samplechap2>
%\fi
%
% %%%%%%%%%%%%%%%%%%%%%%%%%%%%%%%%%%%%%%
% \paragraph{Part Include Files.}
%
% The include files are called |cdocspt3.tex| and |cdocspt4.tex|.
%
%\iffalse
%<*samplepart3|samplepart4>
%\fi

% Optional override for |\version| flag:
%    \begin{macrocode}
%%\providecommand{\version}{final}
%    \end{macrocode}

% Include the main document:
%    \begin{macrocode}
\input{childdoc.def}
\childdocby{cdocsamp}
%    \end{macrocode}

%\iffalse
%</samplepart3|samplepart4>
%\fi
%
%\iffalse
%<*samplepart3>
%\fi
% Some text for part 3:
%    \begin{macrocode}
some text in part three
%    \end{macrocode}

%\iffalse
%</samplepart3>
%\fi
% Some text for part 4:
%\iffalse
%<*samplepart4>
%\fi
%    \begin{macrocode}
more text in part four
%    \end{macrocode}

%\iffalse
%</samplepart4>
%\fi
%
% %%%%%%%%%%%%%%%%%%%%%%%%%%%%%%%%%%%%%%
% \paragraph{Forwarding for a Complete Draft.}
%
% The following forwarding file |cdocsdrf.tex|
% compiles the main document in draft mode:
%\iffalse
%<*sampledraft>
%\fi
%    \begin{macrocode}
\def\version{draft}
\input{childdoc.def}
\childdocforward{cdocsamp}
%    \end{macrocode}

%\iffalse
%</sampledraft>
%\fi
%
% %%%%%%%%%%%%%%%%%%%%%%%%%%%%%%%%%%%%%%
% \paragraph{Forwarding for Final Version of the Chapters.}
%
% The following forwarding files |cdocsfn1.tex| and |cdocsfn2.tex|
% (with identical content)
% compile the final versions of the child documents
% |cdocsch1.tex| and |cdocsch2.tex|, respectively:
%\iffalse
%<*samplefinal>
%\fi
%    \begin{macrocode}
\def\version{final}
\input{childdoc.def}
\childdocforwardprefix[cdocsamp]{cdocsfn}{cdocsch}
%    \end{macrocode}

%\iffalse
%</samplefinal>
%\fi
%
% %%%%%%%%%%%%%%%%%%%%%%%%%%%%%%%%%%%%%%
% \paragraph{Command Line Processing.}
%
% The following three command lines generate the output files
% |cdocscld|, |cdocscl1| and |cdocscl2|
% which should be identical to
% |cdocsdrf|, |cdocsch1| and |cdocsfn2|, respectively:
% \begin{center}
% \begin{tabular}{l}
% |latex -jobname cdocscld \|\\
% |  "\def\version{draft}\input{childdoc.def}\childdocforward{cdocsamp}"|\\
% |latex -jobname cdocscl1 \|\\
% |  "\input{childdoc.def}\childdocforward[cdocsamp]{cdocsch1}"|\\
% |latex -jobname cdocscl2 \|\\
% |  "\def\version{final}\input{childdoc.def}\childdocforward{cdocsch2}"|
% \end{tabular}
% \end{center}
% Note that the trailing backslash on each first line
% merely continues the input to the second line
% (for convenient cut ant paste).
% Furthermore, the command |latex| can be replaced by any
% of its alternative versions such as |pdflatex|.
%
% %%%%%%%%%%%%%%%%%%%%%%%%%%%%%%%%%%%%%%%%%%%%%%%%%%%%%%%%%%%%%%%%%%%%%%%%%%%%%%
% %%%%%%%%%%%%%%%%%%%%%%%%%%%%%%%%%%%%%%%%%%%%%%%%%%%%%%%%%%%%%%%%%%%%%%%%%%%%%%
% \section{Implementation}
%\iffalse
%<*package>
%\fi
%
% This section describes the definitions file |childdoc.def|.

% The definitions cannot be loaded using |\usepackage| or |\RequirePackage|
% which has a mechanism to prevent loading a style file more than once.
% When loading the definitions by means of |\input|
% multiple instances have to be prevented manually:
%\iffalse
%This code needs to be before the `\ProvidesFile' directive
%which is defined at the beginning of this file.
%Therefore it is also placed there and commented out here.
%</package>
%<*discard>
%\fi
%    \begin{macrocode}
\ifdefined\childdocmain\endinput\fi
%    \end{macrocode}
%\iffalse
%</discard>
%<*package>
%\fi
%
% \macro{\ifchilddoc}
% \macro{\ifchilddocmanual}
% The conditional |\ifchilddoc| tells whether a
% child (true) or main (false) document is being compiled.
% The conditional |\ifchilddocmanual| tells whether
% the |\includeonly| mechanism is used (false) or
% the selection of child files must be performed manually (true).
% The definitions initialise to false:
%    \begin{macrocode}
\newif\ifchilddoc
\newif\ifchilddocmanual
%    \end{macrocode}

% \macro{\childdocname}
% \macro{\childdocjob}
% The macro |\childdocname| stores the name of the main document
% to be compiled. The macro |\childdocjob| stores the name of
% the document on which the \LaTeX{} compiler was originally invoked.
% The content of |\jobname| cannot be compared
% to filenames specified in the source due to different catcodes.
% The following code rescans |\jobname|, stores the result
% in |\childdocname| and saves a copy in |\childdocjob|:
%    \begin{macrocode}
\edef\childdocname{\scantokens\expandafter{\jobname\noexpand}}
\let\childdocjob\childdocname
%    \end{macrocode}

% \macro{\childdocdisable}
% The macro |\childdocdisable| prevents the main file
% from being processed more than once.
% At this stage, the main document command |\childdocmain|
% is assumed to be called once again where it should do nothing.
% Any subsequent call to it should prevent
% a secondary processing of the main document
% It overwrites the forwarding commands
% |\childdocof| and |\childdocforward|
% with empty macros to prevent further inclusions of the main document:
%    \begin{macrocode}
\newcommand{\childdocdisable}
{
  \renewcommand{\childdocmain}[1]{\renewcommand{\childdocmain}[1]{\endinput}}
  \renewcommand{\childdocof}[1]{}
  \renewcommand{\childdocby}[2][]{}
  \renewcommand{\childdocforward}[2][]{}
  \renewcommand{\childdocdisable}{}
}
%    \end{macrocode}

% \macro{\childdocmain}
% The macro |\childdocmain| is to be called at the top of the main file
% with nothing or the main filename (without extension) as argument.
% First, it breaks loops.
% If the argument is not empty and does not match |\childdocname|
% (which is set by the first inclusion of |childdoc.def|),
% |\ifchilddoc| is set to true, |\includeonly| is applied to the child file
% and |\jobname| is set to the main file
% (for proper handling of |.aux| files):
%    \begin{macrocode}
\newcommand{\childdocmain}[1]
{
  \childdocdisable\childdocmain{}
  \if?#1?\else
    \begingroup
      \def\childdoctmp{#1}
      \ifx\childdoctmp\childdocname
        \def\childdoctmp{}
      \else
        \def\childdoctmp
        {
          \childdoctrue
          \includeonly{\childdocname}
          \def\childdocjob{#1}
          \def\jobname{#1}
        }
      \fi
      \expandafter
    \endgroup
    \childdoctmp
  \fi
}
%    \end{macrocode}

% \macro{\childdocof}
% The command |\childdocof| redirects
% compilation to the main file |#1|.
%    \begin{macrocode}
\newcommand{\childdocof}[1]
{
  \childdocdisable
  \childdoctrue
  \includeonly{\childdocname}
  \def\jobname{#1}
  \def\childdocjob{#1}
  \input{#1}
}
%    \end{macrocode}

% \macro{\childdocby}
% The command |\childdocby| ....
%    \begin{macrocode}
\newcommand{\childdocby}[2][]
{
  \childdocdisable
  \childdoctrue
  \childdocmanualtrue
  \if?#1?\else
    \def\jobname{#2}
  \fi
  \def\childdocjob{#2}
  \input{#2}
  \endinput
}
%    \end{macrocode}

% \macro{\childdocforward}
% The command |\childdocforward| redirects
% compilation to the main file or
% (if the optional argument is given) a child file.
% Parameters are set as if the main file
% or a child file starting with |\childdocof| was compiled.
% Then compilation is handed over to the main file:
%    \begin{macrocode}
\newcommand{\childdocforward}[2][]
{
  \begingroup
    \if?#1?
      \def\childdoctmp
      {
        \def\childdocname{#2}
        \def\childdocjob{#2}
        \def\jobname{#2}
        \input{#2}
        \endinput
      }
    \else
      \def\childdoctmp
      {
        \childdocdisable
        \def\childdocname{#2}
        \childdoctrue
        \includeonly{#2}
        \def\childdocjob{#1}
        \def\jobname{#1}
        \input{#1}
        \endinput
      }
    \fi
    \expandafter
  \endgroup
  \childdoctmp
}
%    \end{macrocode}

% \macro{\childdocforwardprefix}
% The command |\childdocforwardprefix| redirects
% compilation to the main or a child file by means of a pattern.
% The prefix |#1| in the current filename is replaced by |#2|
% and the suffix of the current filename is kept
% (it is assumed that the filename does not contain the substring `|~~~|'
% which is used as a delimiter).
% Compilation is handed over to the new file by |\childdocforward|:
%    \begin{macrocode}
\newcommand{\childdocforwardprefix}[3][]
{
  \begingroup
    \def\childdocextract #2##1~~~{\def\childdoctmp{\childdocforward[#1]{#3##1}}}
    \expandafter\childdocextract\childdocname~~~
    \expandafter
  \endgroup
  \childdoctmp
}
%    \end{macrocode}

% \macro{\childdoc}
% The deprecated macro |\childdoc| is a legacy version of |\childdocmain|:
%    \begin{macrocode}
\newcommand{\childdoc}{\childdocmain}
%    \end{macrocode}

% \macro{\childdocredirect}
% The deprecated macro |\childdocredirect| is a legacy version
% of |\childdocforward| and |\childdocforwardprefix|:
%    \begin{macrocode}
\newcommand{\childdocredirect}[2][]
{
  \begingroup
    \if?#1?
      \def\childdoctmp{\childdocforward{#2}}
    \else
      \def\childdoctmp{\childdocforwardprefix{#1}{#2}}
    \fi
    \expandafter
  \endgroup
  \childdoctmp
}
%    \end{macrocode}

%\iffalse
%</package>
%\fi
%
\endinput
|
and perform the replacements as outlined below.
Instead of |\childdocmain{|\textit{main}|}| add the following code
to the top of the main file:
%
\begin{center}
\begin{tabular}{l}
|\||ifdefined\childdocname\endinput\||fi\newif\ifchilddoc|\\
|\edef\childdocname{\scantokens\expandafter{\jobname\noexpand}}|\\
|\def\childdocmain{|\textit{main}|}\||ifx\childdocmain\childdocname\||else|\\
|\childdoctrue\includeonly{\childdocname}\let\jobname\childdocmain\||fi|\\
\end{tabular}
\end{center}
%
Instead of |\childdocof{|\textit{main}|}| just include the main file
at the top of each child file:
%
\begin{center}
|\input{|\textit{main}|}|
\end{center}
%
A simple redirection |\childdocforward{|\textit{dest}|}| is achieved by:
%
\begin{center}
|\def\jobname{|\textit{dest}|}\input{\jobname}|
\end{center}
%
The redirection with prefix
|\childdocforwardprefix[|\textit{prefix}|]{|\textit{dest}|}|
is accomplished by:
%
\begin{center}
\begin{tabular}{l}
|{\edef\jobname{\scantokens\expandafter{\jobname\noexpand}}|\\
|\def\redirectjob |\textit{prefix}|#1~~~{\gdef\jobname{|\textit{dest}|#1}}|\\
|\expandafter\redirectjob\jobname~~~}\input{\jobname}|
\end{tabular}
\end{center}

In an alternative approach,
child documents can be compiled by a specific command line
without additional code or specific definitions:
%
\begin{center}
|... -jobname "|\textit{target}|" "|[\textit{flags}]%
|\includeonly{|\textit{dest}|}\input{|\textit{main}|}"|
\end{center}
%

%%%%%%%%%%%%%%%%%%%%%%%%%%%%%%%%%%%%%%%%%%%%%%%%%%%%%%%%%%%%%%%%%%%%%%%%%%%%%%%%
%%%%%%%%%%%%%%%%%%%%%%%%%%%%%%%%%%%%%%%%%%%%%%%%%%%%%%%%%%%%%%%%%%%%%%%%%%%%%%%%
\section{Information}

%%%%%%%%%%%%%%%%%%%%%%%%%%%%%%%%%%%%%%%%%%%%%%%%%%%%%%%%%%%%%%%%%%%%%%%%%%%%%%%%
\subsection{Copyright}

Copyright \copyright{} 2017--2018 Niklas Beisert

This work may be distributed and/or modified under the
conditions of the \LaTeX{} Project Public License, either version 1.3
of this license or (at your option) any later version.
The latest version of this license is in
  \url{http://www.latex-project.org/lppl.txt}
and version 1.3 or later is part of all distributions of \LaTeX{}
version 2005/12/01 or later.

This work has the LPPL maintenance status `maintained'.

The Current Maintainer of this work is Niklas Beisert.

This work consists of the files |README.txt|, |childdoc.ins| and |childdoc.dtx|
as well as the derived files |childdoc.def|, |cdocsamp.tex|
with |cdocsch1.tex|, |cdocsch2.tex|, |cdocspt3.tex|, |cdocspt4.tex|,
|cdocsdrf.tex|, |cdocsfn1.tex|, |cdocsfn2.tex|
as well as |childdoc.pdf|.

%%%%%%%%%%%%%%%%%%%%%%%%%%%%%%%%%%%%%%%%%%%%%%%%%%%%%%%%%%%%%%%%%%%%%%%%%%%%%%%%
\subsection{Files and Installation}

The package consists of the files:
%
\begin{center}
\begin{tabular}{ll}
    |README.txt|   & readme file \\
    |childdoc.ins| & installation file \\
    |childdoc.dtx| & source file \\
    |childdoc.def| & definition file \\
    |cdocsamp.tex| & sample main file \\
    |cdocsch1.tex| & sample include file \\
    |cdocsch2.tex| & sample include file \\
    |cdocspt3.tex| & sample part file \\
    |cdocspt4.tex| & sample part file \\
    |cdocsdrf.tex| & sample redirection file \\
    |cdocsfn1.tex| & sample redirection file \\
    |cdocsfn2.tex| & sample redirection file \\
    |childdoc.pdf| & manual
\end{tabular}
\end{center}
%
The distribution consists of the files
|README.txt|, |childdoc.ins| and |childdoc.dtx|.
%
\begin{itemize}
\item
Run (pdf)\LaTeX{} on |childdoc.dtx|
to compile the manual |childdoc.pdf| (this file).
\item
Run \LaTeX{} on |childdoc.ins| to create the definitions file |childdoc.def|
and the sample |cdocsamp.tex| with include files
|cdocsch1.tex|, |cdocsch2.tex|, |cdocspt3.tex|, |cdocspt4.tex|,
|cdocsdrf.tex|, |cdocsfn1.tex|, |cdocsfn2.tex|.
Then copy the file |childdoc.def| to an appropriate directory of your \LaTeX{}
distribution, e.g.\ \textit{texmf-root}|/tex/latex/childdoc|.
\end{itemize}

%%%%%%%%%%%%%%%%%%%%%%%%%%%%%%%%%%%%%%%%%%%%%%%%%%%%%%%%%%%%%%%%%%%%%%%%%%%%%%%%
\subsection{Related CTAN Packages}

There are several other packages which offer a similar functionality:
%
\begin{itemize}
\item
The packages
\href{http://ctan.org/pkg/docmute}{\textsf{docmute}},
\href{http://ctan.org/pkg/includex}{\textsf{includex}} and
\href{http://ctan.org/pkg/standalone}{\textsf{standalone}}
provide commands to include only the document body of
a child file thus allowing both files to be compiled individually.
\item
The packages \href{http://ctan.org/pkg/subdocs}{\textsf{subdocs}}
and \href{http://ctan.org/pkg/subfiles}{\textsf{subfiles}}
provide structures in which the main and child documents can be
encapsulated and allowing them to be compiled individually.
The inclusion mechanism is different from the conventional |\include|.
\item
The package \href{http://ctan.org/pkg/combine}{\textsf{combine}}
is an elaborate solution to combine several documents into one.
\end{itemize}
%
See also the CTAN topic \href{http://ctan.org/topic/subdocs}{\textsf{subdocs}}
for further related packages.
The present package differs from the above solutions in that
a document structure constructed with the conventional |\include| mechanism
just needs two extra commands at the top of every file
such that all constituent files can be compiled individually.

%%%%%%%%%%%%%%%%%%%%%%%%%%%%%%%%%%%%%%%%%%%%%%%%%%%%%%%%%%%%%%%%%%%%%%%%%%%%%%%%
%\subsection{Feature Suggestions}
%
%The following is a list of features which may be useful for future
%versions of this package:
%%
%\begin{itemize}
%\item
%\ldots
%\end{itemize}

%%%%%%%%%%%%%%%%%%%%%%%%%%%%%%%%%%%%%%%%%%%%%%%%%%%%%%%%%%%%%%%%%%%%%%%%%%%%%%%%
\subsection{Revision History}

%%%%%%%%%%%%%%%%%%%%%%%%%%%%%%%%%%%%%%%%
\paragraph{v2.0:} 2018/12/30

\begin{itemize}
\item
immediate forward processing
\item
added |\childdocby| mechanism
\item
manual restructured
\end{itemize}

%%%%%%%%%%%%%%%%%%%%%%%%%%%%%%%%%%%%%%%%
\paragraph{v1.6:} 2018/01/17

\begin{itemize}
\item
application for development of include files
\item
corrections to manual
\end{itemize}

%%%%%%%%%%%%%%%%%%%%%%%%%%%%%%%%%%%%%%%%
\paragraph{v1.5:} 2017/05/21

\begin{itemize}
\item
more complete structuring introduced
\item
|\childdocof| introduced
\item
|\childdoc| renamed to |\childdocmain|
\item
|\childredirect| renamed to |\childdocforward| and |\childdocforwardprefix|
and functionality expanded
\end{itemize}

%%%%%%%%%%%%%%%%%%%%%%%%%%%%%%%%%%%%%%%%
\paragraph{v1.0:} 2017/04/27

\begin{itemize}
\item
manual and install package
\item
first version published on CTAN
\end{itemize}

%%%%%%%%%%%%%%%%%%%%%%%%%%%%%%%%%%%%%%%%
\paragraph{v0.6:} 2017/04/26

\begin{itemize}
\item
redirection mechanism added
\end{itemize}

%%%%%%%%%%%%%%%%%%%%%%%%%%%%%%%%%%%%%%%%
\paragraph{v0.5:} 2017/04/26

\begin{itemize}
\item
functionality in definition file
\end{itemize}


%%%%%%%%%%%%%%%%%%%%%%%%%%%%%%%%%%%%%%%%%%%%%%%%%%%%%%%%%%%%%%%%%%%%%%%%%%%%%%%%
%%%%%%%%%%%%%%%%%%%%%%%%%%%%%%%%%%%%%%%%%%%%%%%%%%%%%%%%%%%%%%%%%%%%%%%%%%%%%%%%
%%%%%%%%%%%%%%%%%%%%%%%%%%%%%%%%%%%%%%%%%%%%%%%%%%%%%%%%%%%%%%%%%%%%%%%%%%%%%%%%
\appendix

\settowidth\MacroIndent{\rmfamily\scriptsize 000\ }

 \DocInput{childdoc.dtx}

\end{document}
%</driver>
% \fi
%
% %%%%%%%%%%%%%%%%%%%%%%%%%%%%%%%%%%%%%%%%%%%%%%%%%%%%%%%%%%%%%%%%%%%%%%%%%%%%%%
% %%%%%%%%%%%%%%%%%%%%%%%%%%%%%%%%%%%%%%%%%%%%%%%%%%%%%%%%%%%%%%%%%%%%%%%%%%%%%%
% \section{Sample}
%\iffalse
%<*samplemain>
%\fi
%
% The following presents a sample document
% with two chapters, two parts, a title page,
% a compile flag as well as three forwarding files to set the flag.
% It consists of eight |.tex| files:
% \begin{center}
% \begin{tabular}{ll}
% |cdocsamp.tex|&main file\\
% |cdocsch1.tex|&include file for chapter 1\\
% |cdocsch2.tex|&include file for chapter 2\\
% |cdocspt3.tex|&include file for part 3\\
% |cdocspt4.tex|&include file for part 4\\
% |cdocsdrf.tex|&forwarding file for main file in draft mode\\
% |cdocsfi1.tex|&forwarding file for final version of chapter 1\\
% |cdocsfi2.tex|&forwarding file for final version of chapter 2\\
% \end{tabular}
% \end{center}
% Each of the eight files can be compiled directly by the \LaTeX{} compiler.
%
% %%%%%%%%%%%%%%%%%%%%%%%%%%%%%%%%%%%%%%
% \paragraph{Main File.}
%
% The main file is called |cdocsamp.tex|.
%
% Load the \textsf{childdoc} definitions and
% declare the filename for the main document:
%    \begin{macrocode}
% \iffalse
%
% childdoc.dtx Copyright (C) 2017-2018 Niklas Beisert
%
% This work may be distributed and/or modified under the
% conditions of the LaTeX Project Public License, either version 1.3
% of this license or (at your option) any later version.
% The latest version of this license is in
%   http://www.latex-project.org/lppl.txt
% and version 1.3 or later is part of all distributions of LaTeX
% version 2005/12/01 or later.
%
% This work has the LPPL maintenance status `maintained'.
%
% The Current Maintainer of this work is Niklas Beisert.
%
% This work consists of the files childdoc.dtx and childdoc.ins
% and the derived files childdoc.def and cdocsamp.tex with
% cdocsch1.tex, cdocsch2.tex, cdocsdrf.tex, cdocsfn1.tex, cdocsfn2.tex.
%
%<package>\ifdefined\childdocmain\endinput\fi
%<package>\ProvidesFile{childdoc.def}[2018/12/30 v2.0 child document driver]
%<samplemain>\ProvidesFile{cdocsamp.tex}[2018/12/30 v2.0 sample for childdoc]
%<*driver>
%\ProvidesFile{childdoc.drv}[2018/12/30 v2.0 childdoc reference manual file]
\PassOptionsToClass{10pt,a4paper}{article}
\documentclass{ltxdoc}

\usepackage[margin=35mm]{geometry}
\usepackage{hyperref}
\usepackage{hyperxmp}
\usepackage[usenames]{color}

\hypersetup{colorlinks=true}
\hypersetup{pdfstartview=FitH}
\hypersetup{pdfpagemode=UseNone}
\hypersetup{pdfsource={}}
\hypersetup{pdflang={en-UK}}
\hypersetup{pdfcopyright={Copyright 2017-2018 Niklas Beisert.
  This work may be distributed and/or modified under the
  conditions of the LaTeX Project Public License, either version 1.3
  of this license or (at your option) any later version.}}
\hypersetup{pdflicenseurl={http://www.latex-project.org/lppl.txt}}
\hypersetup{pdfcontactaddress={ETH Zurich, ITP, HIT K,
  Wolfgang-Pauli-Strasse 27}}
\hypersetup{pdfcontactpostcode={8093}}
\hypersetup{pdfcontactcity={Zurich}}
\hypersetup{pdfcontactcountry={Switzerland}}
\hypersetup{pdfcontactemail={nbeisert@itp.phys.ethz.ch}}
\hypersetup{pdfcontacturl={http://people.phys.ethz.ch/\xmptilde nbeisert/}}

\newcommand{\secref}[1]{\hyperref[#1]{section \ref*{#1}}}

\parskip1ex
\parindent0pt
\let\olditemize\itemize
\def\itemize{\olditemize\parskip0pt}

\begin{document}

\title{The \textsf{childdoc} Package}
\hypersetup{pdftitle={The childdoc Package}}
\author{Niklas Beisert\\[2ex]
  Institut f\"ur Theoretische Physik\\
  Eidgen\"ossische Technische Hochschule Z\"urich\\
  Wolfgang-Pauli-Strasse 27, 8093 Z\"urich, Switzerland\\[1ex]
  \href{mailto:nbeisert@itp.phys.ethz.ch}
  {\texttt{nbeisert@itp.phys.ethz.ch}}}
\hypersetup{pdfauthor={Niklas Beisert}}
\hypersetup{pdfsubject={Manual for the LaTeX2e Package childdoc}}
\date{30 December 2018, \textsf{v2.0}}
\maketitle

\begin{abstract}\noindent
\textsf{childdoc} is a \LaTeXe{} package
that enables the direct compilation
of document sections included by |\include|
to individual files.
\end{abstract}

\begingroup
\parskip0ex
\tableofcontents
\endgroup

%%%%%%%%%%%%%%%%%%%%%%%%%%%%%%%%%%%%%%%%%%%%%%%%%%%%%%%%%%%%%%%%%%%%%%%%%%%%%%%%
%%%%%%%%%%%%%%%%%%%%%%%%%%%%%%%%%%%%%%%%%%%%%%%%%%%%%%%%%%%%%%%%%%%%%%%%%%%%%%%%
\section{Introduction}

\LaTeX{} provides a mechanism to structure a large document (such as a book)
into a main file and several child files (containing the chapters)
using the |\include| command.
This mechanism is beneficial for documents
which span hundreds of pages in order to
make the source file(s) more manageable.
Moreover, compilation can be restricted to
selected child files by means of the |\includeonly| command.
The latter feature can be used to reduce the compilation time while editing
(this was significantly more useful in the earlier days of \LaTeX{})
or to generate a smaller document which is easier to navigate.
Another application of |\includeonly| is to generate
documents consisting of selected parts of the complete document.

However, there are a few drawbacks of the plain |\include| mechanism:
\begin{itemize}
\item
The child files cannot be compiled on their own,
they can only be compiled via the main file.
A naive editing environment
(such as a text editor with an option
to have the current file processed by \LaTeX)
may require one to switch to the main file before compiling;
attempting to compile the child file produces errors.
\item
The main file must be modified (each time)
to adjust the |\includeonly| command
to the present needs. This easily leaves the main file in a messy state.
\item
The generated document will always carry the filename
of the main document. This is inconvenient if
several child files are to be compiled and
to be kept for distribution.
\end{itemize}

The present package provides a simple interface
to make child files individually compilable by \LaTeX{}.
Compiling a child file then has the same effect as compiling
the main file with an |\includeonly| command
to select the appropriate child.
Moreover the generated document will carry the name of the child
rather than the main file.
This resolves all three above issues.

This feature is meant to make the editing of books,
thesis documents and lecture notes somewhat more convenient.
However, the package can also be used efficiently for
composing a series of documents (such as exercise sheets)
which are typically distributed individually.
It then assists the author in generating the individual documents
(potentially in different versions)
as well as a document containing the collected series.
Another application is in developing style files
or other kinds of included material
where compilation of the style file could redirect
to a sample or test file.

%%%%%%%%%%%%%%%%%%%%%%%%%%%%%%%%%%%%%%%%%%%%%%%%%%%%%%%%%%%%%%%%%%%%%%%%%%%%%%%%
%%%%%%%%%%%%%%%%%%%%%%%%%%%%%%%%%%%%%%%%%%%%%%%%%%%%%%%%%%%%%%%%%%%%%%%%%%%%%%%%
\section{Usage}

First of all, the package \textsf{childdoc} is \emph{not} a standard
\LaTeXe{} |.sty| style file! Therefore it needs to be invoked in
a non-standard way.

%%%%%%%%%%%%%%%%%%%%%%%%%%%%%%%%%%%%%%%%%%%%%%%%%%%%%%%%%%%%%%%%%%%%%%%%%%%%%%%%
\subsection{Included Files}
\label{sec:include}

%%%%%%%%%%%%%%%%%%%%%%%%%%%%%%%%%%%%%%%%
\DescribeMacro{\childdocmain}
To use the package, add the commands
\begin{center}
\begin{tabular}{l}
|\input{childdoc.def}|\\
|\childdocmain{}|\\
\end{tabular}
\end{center}
at the very top of the main \LaTeX{} file,
in particular \emph{before} the |\documentclass| statement!
The argument of |\childdocmain| should be left empty
(but it must be present).

%%%%%%%%%%%%%%%%%%%%%%%%%%%%%%%%%%%%%%%%
\DescribeMacro{\childdocof}
Furthermore, add the commands
\begin{center}
\begin{tabular}{l}
|\input{childdoc.def}|\\
|\childdocof{|\textit{main}|}|\\
\end{tabular}
\end{center}
at the top of every child file \textit{child}
which is included by |\include{|\textit{child}|}|
from within the main file
(or at least for those files to be compiled individually).
The argument \textit{main} must be the filename of the main file.

There are a couple of
considerations in setting up the main and child documents:

%%%%%%%%%%%%%%%%%%%%%%%%%%%%%%%%%%%%%%%%
\paragraph{Restrictions.}

Please note the following restrictions:
\begin{itemize}
\item
|\childdocmain| must be called with one argument \textit{main}
to ensure compatibility with earlier version of the package.
It must either be empty (|\childdocmain{}|)
or precisely match the filename of the main file in which it is specified.
See \secref{sec:detection} for further information.
\item
The filename \textit{main} must be specified without the |.tex| extension.
\item
The filename \textit{main} is case sensitive
(even in case-insensitive file systems)
due to internal string comparison.
\item
The argument \textit{main} should be fully expanded, it cannot be a macro.
\item
Subdirectories and special characters should be avoided in filenames.
\item
The command |\childdocmain{|\textit{main}|}| must be followed by a whitespace.
It should not be followed immediately by another command
or by a comment mark `|%|'.
This is because the \TeX{} parser reads the token immediately following
the argument of |\childdocmain| and puts it
at the beginning of every child section;
however, a white\-space is ignored.
\end{itemize}

%%%%%%%%%%%%%%%%%%%%%%%%%%%%%%%%%%%%%%%%
\paragraph{Content of Main File.}

It is advisable to place all content in the child files included by |\include|.
Any output contained in the main file will appear in all child documents
unless suppressed manually;
it cannot be suppressed automatically by the |\includeonly| directive
and thus should normally be avoided.
A method to include some content in the main file
by means of conditional processing is described in \secref{sec:conditional}.

%%%%%%%%%%%%%%%%%%%%%%%%%%%%%%%%%%%%%%%%
\paragraph{Page Numbering.}

When only a part of the document is compiled,
the appropriate numbering of pages
(as well as other status parameters)
is determined from the |.aux| files.
The latter contain information from previous passes.
However this information needs to propagate through
all intermediate child documents.
Therefore the page numbering in child documents may well
be inconsistent until the complete document is compiled at least once.

A useful (if unconventional) way to always ensure a consistent
page numbering is to restart the numbering in each child document
and denote the pages by `\textit{child}|.|\textit{page}'
where \textit{child} represents the chapter/section number of the child file.
This can be achieved by the command
|\numberwithin{page}{|\textit{child}|}|
of the \textsf{amsmath} package
where \textit{child} can be |chapter| or |section|
depending on the chosen structuring.
Alternatively, one can modify the macro |\thepage| appropriately
and reset the counter |page| at the start of each child file.

%%%%%%%%%%%%%%%%%%%%%%%%%%%%%%%%%%%%%%%%%%%%%%%%%%%%%%%%%%%%%%%%%%%%%%%%%%%%%%%%
\subsection{Conditional Processing}
\label{sec:conditional}

The package provides a mechanism to compile different versions
of a document. To customise the versions further some conditional processing
can come in handy to distinguish which version is being compiled.
The package provides two macros to describe the compilation context:

%%%%%%%%%%%%%%%%%%%%%%%%%%%%%%%%%%%%%%%%
\DescribeMacro{\ifchilddoc}
The conditional |\ifchilddoc| distinguishes between the compilation of
child documents and the main document:
%
\begin{center}
|\ifchilddoc |\textit{child-code}| |[|\||else |\textit{main-code}]| \||fi|
\end{center}

%%%%%%%%%%%%%%%%%%%%%%%%%%%%%%%%%%%%%%%%
\DescribeMacro{\childdocname}
\DescribeMacro{\childdocjob}
The macro |\childdocname| contains the filename (without extension)
of the main or child file being processed.
Note that |\childdocjob| will always contain the name of the main file.

%%%%%%%%%%%%%%%%%%%%%%%%%%%%%%%%%%%%%%%%
\paragraph{Title Page.}

Conditional processing can be used to include a title or banner page
in the main document when proper precautions are taken.
Importantly, the code in the main file should ensure that the page counter
(as well as other status parameters which are stored in the |.aux| files)
takes the same value after the conditional processing.
Otherwise the page numbers may take divergent values
depending on which part is compiled.

For example, a title page could be declared by:
%
\begin{center}
\begin{tabular}{l}
|\ifchilddoc\||else|\\
|\addtocounter{page}{-1}|\\
\textit{code for title page}\\
|\newpage|\\
|\||fi|
\end{tabular}
\end{center}
%
A banner page for the child documents can be generated by:
%
\begin{center}
\begin{tabular}{l}
|\ifchilddoc|\\
|\addtocounter{page}{-1}|\\
\textit{code for banner page}\\
|\newpage|\\
|\||fi|
\end{tabular}
\end{center}
%
Here one could write a message such as:
\begin{center}
|This is the part \childdocname{} of \childdocjob{}.|
\end{center}

%%%%%%%%%%%%%%%%%%%%%%%%%%%%%%%%%%%%%%%%%%%%%%%%%%%%%%%%%%%%%%%%%%%%%%%%%%%%%%%%
\subsection{Flags}
\label{sec:flags}

The package makes it easy to generate different versions
of the main or child documents.
To this end compilation flags can be defined
and assigned different default values.
They will be particularly useful in conjunction
with the forwarding mechanism described in \secref{sec:forward}.

For example, it may be useful to have a flag |\version|
which can be set to |draft| or |final|.
The document source will contain some conditional code
depending on the value of |\version|.
Suppose further, the flag should default to |final| for the main file
and to |draft| for child files
which is a natural assignment for editing the document.
This is achieved by placing the following code
in the preamble of the main document
(below the |\childdocmain| directive):
%
\begin{center}
\begin{tabular}{l}
|\ifchilddoc|\\
|\providecommand{\version}{draft}|\\
|\||else|\\
|\providecommand{\version}{final}|\\
|\||fi|
\end{tabular}
\end{center}
%
The definition by |\providecommand| makes sure
that previous definitions are not overwritten.
Further statements |\providecommand{\version}{...}|
can thus be added before the above code to override it.

For the main file, one might add a line
(between |\childdocmain| and the above block)
%
\begin{center}
|%\ifchilddoc\||else\providecommand{\version}{draft}\||fi|
\end{center}
%
which can be uncommented to produce a draft version.
Likewise one can add a line to the very top of a child file
(above the |\childdocof{|\textit{main}|}| directive)
%
\begin{center}
|%\providecommand{\version}{final}|
\end{center}
%
which can be uncommented to produce the final version of this child document.

%%%%%%%%%%%%%%%%%%%%%%%%%%%%%%%%%%%%%%%%%%%%%%%%%%%%%%%%%%%%%%%%%%%%%%%%%%%%%%%%
\subsection{Forwarding}
\label{sec:forward}

Different versions of the main or child documents
using compilation flags as described in \secref{sec:flags}
can be (permanently) stored in different files
for convenient compilation, viewing and distribution.
To this end, the package defines a command
to pass on compilation to a different file:

%%%%%%%%%%%%%%%%%%%%%%%%%%%%%%%%%%%%%%%%
\DescribeMacro{\childdocforward}
The command |\childdocforward| redirects processing to
another source file:
%
\begin{center}
\begin{tabular}{l}
|\input{childdoc.def}|\\
|\childdocforward[|\textit{main}|]{|\textit{dest}|}|\\
\end{tabular}
\end{center}
%
The argument \textit{dest} is the destination file
(without extension).
It should be the main file or one of the child files.
Note that further \textsf{childdoc} directives
such as |\childdocof| and |\childdocforward|
in the indicated file will be processed in this form.
The optional argument \textit{main}
passes on directly to the main file \textit{main}
while pretending to compile the child \textit{dest}.
This form behaves as if \textit{dest}
issues |\childdocof{|\textit{main}|}| right away,
and no further \textsf{childdoc} directives will be processed.

%%%%%%%%%%%%%%%%%%%%%%%%%%%%%%%%%%%%%%%%
\DescribeMacro{\...prefix}
In the alternative form |\childdocforwardprefix|,
%
\begin{center}
\begin{tabular}{l}
|\input{childdoc.def}|\\
|\childdocforwardprefix[|\textit{main}|]{|\textit{prefix}|}{|\textit{dest}|}|
\end{tabular}
\end{center}
%
the destination file is determined by a pattern
depending on the current file:
To make this work, the current file must be called
`{\textit{prefix}\hspace{0.2em}\textit{suffix}}'
with \textit{prefix} matching precisely the argument.
Processing is then passed on to the file
`{\textit{dest}\hspace{0.2em}\textit{suffix}}'.
Surely, the same effect is achieved by
directly specifying the
argument `{\textit{dest}\hspace{0.2em}\textit{suffix}}'
in the first form.
However, that requires to set up a different file
for each child. With the alternative form of the command
all these files can have exactly the same content
which simplifies setting them up and maintaining them.

For example, the following file |draft.tex|
with a compilation flag |\version| as described in \secref{sec:flags}
compiles the main document as a draft:
%
\begin{center}
\begin{tabular}{l}
|\def\version{draft}|\\
|\input{childdoc.def}|\\
|\childdocforward{|\textit{main}|}|
\end{tabular}
\end{center}
%
Likewise, the following files |final|\textit{nn}|.tex|
compile the final version of the child document
|child|\textit{nn}|.tex|:
%
\begin{center}
\begin{tabular}{l}
|\def\version{final}|\\
|\input{childdoc.def}|\\
|\childdocforwardprefix{final}{child}|
\end{tabular}
\end{center}
%

Note that when several versions of a main file and/or of each child file
are to be generated, it may be convenient to set up a |Makefile| or
shell script to automatise the process.

%%%%%%%%%%%%%%%%%%%%%%%%%%%%%%%%%%%%%%%%%%%%%%%%%%%%%%%%%%%%%%%%%%%%%%%%%%%%%%%%
\subsection{Command Line Processing}
\label{sec:commandline}

The effect of redirection files can also be achieved by invoking
the \LaTeX{} compiler with a more elaborate command line.
Most conveniently this should be done as part
of a shell script or a |Makefile|.

When using \textsf{childdoc} in the main file, the following
command lines effectively perform a redirection
(note that depending on the shell being used,
backslashes may have to be doubled: `|\|' $\to$ `|\\|'):
%
\begin{center}
|... -jobname "|\textit{target}|" |\\|"|[\textit{flags}]%
|\input{childdoc.def}\childdocforward[|\textit{main}|]{|\textit{dest}|}"|
\end{center}
%
Here \textit{target} is the name of the output file,
\textit{main} is the name of the main file
and \textit{dest} is the name of the main or child file to be processed
(all filenames without extensions).
The optional argument \textit{main} can be omitted
if \textit{main} matches \textit{dest}.
Optionally, compilation \textit{flags} can be defined via |\def| commands.
This command line makes the \TeX{} engine believe
it is compiling the file \textit{target}
whose content is specified as the latter parameter.
The provided code then forwards the processing to
\textit{main} or \textit{dest} as described in \secref{sec:forward}.

%%%%%%%%%%%%%%%%%%%%%%%%%%%%%%%%%%%%%%%%%%%%%%%%%%%%%%%%%%%%%%%%%%%%%%%%%%%%%%%%
\subsection{Include by Input}
\label{sec:input}

Including child documents by |\include| has some restrictions by design.
Most notably, the content of a child document always occupies
its own set of pages; pages cannot be shared between child documents.
Usually, this behaviour makes perfect sense
because each child document contain an essential part of the document.
However, in some situations it may be desirable to compose
a document from a collection of parts
without having mandatory page breaks between then.
For this case, the package
provides a mechanism to include parts
by |\input| which can also be processed individually.
However, by construction this mechanism
requires manual handling of the content to be output.

%%%%%%%%%%%%%%%%%%%%%%%%%%%%%%%%%%%%%%%%
\DescribeMacro{\ifchilddocmanual}
The main file should be prepared as usual, see \secref{sec:include}.
However, the document body must make a distinction
between processing of an individual part and of the main document, e.g.:
%
\begin{center}
\begin{tabular}{l}
|\ifchilddocmanual|\\
|\input{\childdocname}|\\
|\||else|\\
\textit{document body with }|\input{|\textit{part}|}|\\
|\||fi|
\end{tabular}
\end{center}
%
The conditional |\ifchilddocmanual| is true whenever
a part to be included by |\input| is being compiled,
and the name of the part is stored in |\childdocname|.

%%%%%%%%%%%%%%%%%%%%%%%%%%%%%%%%%%%%%%%%
\DescribeMacro{\childdocby}
Each part to be included by |\input| should start with:
%
\begin{center}
\begin{tabular}{l}
|\input{childdoc.def}|\\
|\childdocby{|\textit{main}|}|\\
\end{tabular}
\end{center}
%
The directive |\childdocby| is similar to |\childdocof|
described in \secref{sec:include},
but the subsequent selection of content must be done manually.
To that end, both |\ifchilddoc| and |\ifchilddocmanual|
will be true upon processing of a part,
and the name of the part is stored in |\childdocname|.
Note that |\jobname| will be set to the filename of the current part
so that each part receives an individual |.aux| file
that does not interfere with the |.aux| file(s) of the main document.
This behaviour can be altered by the alternative form
|\childdocby[*]{|\textit{main}|}| (with a non-empty optional argument)
which uses the |.aux| file of the main document
by setting |\jobname| to \textit{main}.

%%%%%%%%%%%%%%%%%%%%%%%%%%%%%%%%%%%%%%%%%%%%%%%%%%%%%%%%%%%%%%%%%%%%%%%%%%%%%%%%
\subsection{Driver Development}
\label{sec:driver}

The \textsf{childdoc} mechanism can also be use for the development
of definition files such as \LaTeX{} styles or classes.
This case differs from the above setup with multiple parts
included by |\include| in that no |\includeonly| should be invoked.
This can be achieved by starting the include file
(before |\ProvidesPackage|) with:
%
\begin{center}
\begin{tabular}{l}
|\input{childdoc.def}|\\
|\childdocforward{|\textit{main}|}|\\
\end{tabular}
\end{center}
%
or alternatively with:
%
\begin{center}
\begin{tabular}{l}
|\input{childdoc.def}|\\
|\childdocby{|\textit{main}|}|\\
\end{tabular}
\end{center}
%
Both forms have slightly different effects as described above.
The main file is prepared as usual, see \secref{sec:include}.

%%%%%%%%%%%%%%%%%%%%%%%%%%%%%%%%%%%%%%%%%%%%%%%%%%%%%%%%%%%%%%%%%%%%%%%%%%%%%%%%
\subsection{Legacy Detection}
\label{sec:detection}

The directive |\childdocmain| in the main file can detect
whether the complete document or merely a child is to be compiled
even without using the directive |\childdocof|.
This method is deprecated because it is less robust
and there is no compelling reason to use it;
it is merely provided for backward compatibility
and it may be removed in future versions.

If the detection mechanism is to be used,
it is mandatory to correctly specify
the filename of the main file as the argument of |\childdocmain|:
%
\begin{center}
\begin{tabular}{l}
|\input{childdoc.def}|\\
|\childdocmain{|\textit{main}|}|\\
\end{tabular}
\end{center}
%
If |\jobname| does not match the argument \textit{main} of |\childdocmain|,
it is assumed that |\jobname| points to the child file to be compiled.
When using |\childdocmain| with the main file specified as argument,
it suffices to start a child file
with just |\input{|\textit{main}|}|
without loading of the package and using |\childdocof|.
If instead all processing is done
with the appropriate \textsf{childdoc} directives,
the argument of \textit{main} of |\childdocmain| can be empty.

An alternative version of the command line processing described
in \secref{sec:commandline} using the detection mechanism reads:
%
\begin{center}
|... -jobname "|\textit{target}|" "|[\textit{flags}]%
[|\def\jobname{|\textit{dest}|}|]|\input{|\textit{main}|}"|
\end{center}

%%%%%%%%%%%%%%%%%%%%%%%%%%%%%%%%%%%%%%%%%%%%%%%%%%%%%%%%%%%%%%%%%%%%%%%%%%%%%%%%
\subsection{Manual Code}
\label{sec:manual}

In case one cannot be certain whether the definitions file |childdoc.def|
is installed on the target \TeX{} distribution
and one prefers not to ship it,
it is conceivable to paste a few relevant commands into the sources.

To that end, drop all statements |\input{childdoc.def}|
and perform the replacements as outlined below.
Instead of |\childdocmain{|\textit{main}|}| add the following code
to the top of the main file:
%
\begin{center}
\begin{tabular}{l}
|\||ifdefined\childdocname\endinput\||fi\newif\ifchilddoc|\\
|\edef\childdocname{\scantokens\expandafter{\jobname\noexpand}}|\\
|\def\childdocmain{|\textit{main}|}\||ifx\childdocmain\childdocname\||else|\\
|\childdoctrue\includeonly{\childdocname}\let\jobname\childdocmain\||fi|\\
\end{tabular}
\end{center}
%
Instead of |\childdocof{|\textit{main}|}| just include the main file
at the top of each child file:
%
\begin{center}
|\input{|\textit{main}|}|
\end{center}
%
A simple redirection |\childdocforward{|\textit{dest}|}| is achieved by:
%
\begin{center}
|\def\jobname{|\textit{dest}|}\input{\jobname}|
\end{center}
%
The redirection with prefix
|\childdocforwardprefix[|\textit{prefix}|]{|\textit{dest}|}|
is accomplished by:
%
\begin{center}
\begin{tabular}{l}
|{\edef\jobname{\scantokens\expandafter{\jobname\noexpand}}|\\
|\def\redirectjob |\textit{prefix}|#1~~~{\gdef\jobname{|\textit{dest}|#1}}|\\
|\expandafter\redirectjob\jobname~~~}\input{\jobname}|
\end{tabular}
\end{center}

In an alternative approach,
child documents can be compiled by a specific command line
without additional code or specific definitions:
%
\begin{center}
|... -jobname "|\textit{target}|" "|[\textit{flags}]%
|\includeonly{|\textit{dest}|}\input{|\textit{main}|}"|
\end{center}
%

%%%%%%%%%%%%%%%%%%%%%%%%%%%%%%%%%%%%%%%%%%%%%%%%%%%%%%%%%%%%%%%%%%%%%%%%%%%%%%%%
%%%%%%%%%%%%%%%%%%%%%%%%%%%%%%%%%%%%%%%%%%%%%%%%%%%%%%%%%%%%%%%%%%%%%%%%%%%%%%%%
\section{Information}

%%%%%%%%%%%%%%%%%%%%%%%%%%%%%%%%%%%%%%%%%%%%%%%%%%%%%%%%%%%%%%%%%%%%%%%%%%%%%%%%
\subsection{Copyright}

Copyright \copyright{} 2017--2018 Niklas Beisert

This work may be distributed and/or modified under the
conditions of the \LaTeX{} Project Public License, either version 1.3
of this license or (at your option) any later version.
The latest version of this license is in
  \url{http://www.latex-project.org/lppl.txt}
and version 1.3 or later is part of all distributions of \LaTeX{}
version 2005/12/01 or later.

This work has the LPPL maintenance status `maintained'.

The Current Maintainer of this work is Niklas Beisert.

This work consists of the files |README.txt|, |childdoc.ins| and |childdoc.dtx|
as well as the derived files |childdoc.def|, |cdocsamp.tex|
with |cdocsch1.tex|, |cdocsch2.tex|, |cdocspt3.tex|, |cdocspt4.tex|,
|cdocsdrf.tex|, |cdocsfn1.tex|, |cdocsfn2.tex|
as well as |childdoc.pdf|.

%%%%%%%%%%%%%%%%%%%%%%%%%%%%%%%%%%%%%%%%%%%%%%%%%%%%%%%%%%%%%%%%%%%%%%%%%%%%%%%%
\subsection{Files and Installation}

The package consists of the files:
%
\begin{center}
\begin{tabular}{ll}
    |README.txt|   & readme file \\
    |childdoc.ins| & installation file \\
    |childdoc.dtx| & source file \\
    |childdoc.def| & definition file \\
    |cdocsamp.tex| & sample main file \\
    |cdocsch1.tex| & sample include file \\
    |cdocsch2.tex| & sample include file \\
    |cdocspt3.tex| & sample part file \\
    |cdocspt4.tex| & sample part file \\
    |cdocsdrf.tex| & sample redirection file \\
    |cdocsfn1.tex| & sample redirection file \\
    |cdocsfn2.tex| & sample redirection file \\
    |childdoc.pdf| & manual
\end{tabular}
\end{center}
%
The distribution consists of the files
|README.txt|, |childdoc.ins| and |childdoc.dtx|.
%
\begin{itemize}
\item
Run (pdf)\LaTeX{} on |childdoc.dtx|
to compile the manual |childdoc.pdf| (this file).
\item
Run \LaTeX{} on |childdoc.ins| to create the definitions file |childdoc.def|
and the sample |cdocsamp.tex| with include files
|cdocsch1.tex|, |cdocsch2.tex|, |cdocspt3.tex|, |cdocspt4.tex|,
|cdocsdrf.tex|, |cdocsfn1.tex|, |cdocsfn2.tex|.
Then copy the file |childdoc.def| to an appropriate directory of your \LaTeX{}
distribution, e.g.\ \textit{texmf-root}|/tex/latex/childdoc|.
\end{itemize}

%%%%%%%%%%%%%%%%%%%%%%%%%%%%%%%%%%%%%%%%%%%%%%%%%%%%%%%%%%%%%%%%%%%%%%%%%%%%%%%%
\subsection{Related CTAN Packages}

There are several other packages which offer a similar functionality:
%
\begin{itemize}
\item
The packages
\href{http://ctan.org/pkg/docmute}{\textsf{docmute}},
\href{http://ctan.org/pkg/includex}{\textsf{includex}} and
\href{http://ctan.org/pkg/standalone}{\textsf{standalone}}
provide commands to include only the document body of
a child file thus allowing both files to be compiled individually.
\item
The packages \href{http://ctan.org/pkg/subdocs}{\textsf{subdocs}}
and \href{http://ctan.org/pkg/subfiles}{\textsf{subfiles}}
provide structures in which the main and child documents can be
encapsulated and allowing them to be compiled individually.
The inclusion mechanism is different from the conventional |\include|.
\item
The package \href{http://ctan.org/pkg/combine}{\textsf{combine}}
is an elaborate solution to combine several documents into one.
\end{itemize}
%
See also the CTAN topic \href{http://ctan.org/topic/subdocs}{\textsf{subdocs}}
for further related packages.
The present package differs from the above solutions in that
a document structure constructed with the conventional |\include| mechanism
just needs two extra commands at the top of every file
such that all constituent files can be compiled individually.

%%%%%%%%%%%%%%%%%%%%%%%%%%%%%%%%%%%%%%%%%%%%%%%%%%%%%%%%%%%%%%%%%%%%%%%%%%%%%%%%
%\subsection{Feature Suggestions}
%
%The following is a list of features which may be useful for future
%versions of this package:
%%
%\begin{itemize}
%\item
%\ldots
%\end{itemize}

%%%%%%%%%%%%%%%%%%%%%%%%%%%%%%%%%%%%%%%%%%%%%%%%%%%%%%%%%%%%%%%%%%%%%%%%%%%%%%%%
\subsection{Revision History}

%%%%%%%%%%%%%%%%%%%%%%%%%%%%%%%%%%%%%%%%
\paragraph{v2.0:} 2018/12/30

\begin{itemize}
\item
immediate forward processing
\item
added |\childdocby| mechanism
\item
manual restructured
\end{itemize}

%%%%%%%%%%%%%%%%%%%%%%%%%%%%%%%%%%%%%%%%
\paragraph{v1.6:} 2018/01/17

\begin{itemize}
\item
application for development of include files
\item
corrections to manual
\end{itemize}

%%%%%%%%%%%%%%%%%%%%%%%%%%%%%%%%%%%%%%%%
\paragraph{v1.5:} 2017/05/21

\begin{itemize}
\item
more complete structuring introduced
\item
|\childdocof| introduced
\item
|\childdoc| renamed to |\childdocmain|
\item
|\childredirect| renamed to |\childdocforward| and |\childdocforwardprefix|
and functionality expanded
\end{itemize}

%%%%%%%%%%%%%%%%%%%%%%%%%%%%%%%%%%%%%%%%
\paragraph{v1.0:} 2017/04/27

\begin{itemize}
\item
manual and install package
\item
first version published on CTAN
\end{itemize}

%%%%%%%%%%%%%%%%%%%%%%%%%%%%%%%%%%%%%%%%
\paragraph{v0.6:} 2017/04/26

\begin{itemize}
\item
redirection mechanism added
\end{itemize}

%%%%%%%%%%%%%%%%%%%%%%%%%%%%%%%%%%%%%%%%
\paragraph{v0.5:} 2017/04/26

\begin{itemize}
\item
functionality in definition file
\end{itemize}


%%%%%%%%%%%%%%%%%%%%%%%%%%%%%%%%%%%%%%%%%%%%%%%%%%%%%%%%%%%%%%%%%%%%%%%%%%%%%%%%
%%%%%%%%%%%%%%%%%%%%%%%%%%%%%%%%%%%%%%%%%%%%%%%%%%%%%%%%%%%%%%%%%%%%%%%%%%%%%%%%
%%%%%%%%%%%%%%%%%%%%%%%%%%%%%%%%%%%%%%%%%%%%%%%%%%%%%%%%%%%%%%%%%%%%%%%%%%%%%%%%
\appendix

\settowidth\MacroIndent{\rmfamily\scriptsize 000\ }

 \DocInput{childdoc.dtx}

\end{document}
%</driver>
% \fi
%
% %%%%%%%%%%%%%%%%%%%%%%%%%%%%%%%%%%%%%%%%%%%%%%%%%%%%%%%%%%%%%%%%%%%%%%%%%%%%%%
% %%%%%%%%%%%%%%%%%%%%%%%%%%%%%%%%%%%%%%%%%%%%%%%%%%%%%%%%%%%%%%%%%%%%%%%%%%%%%%
% \section{Sample}
%\iffalse
%<*samplemain>
%\fi
%
% The following presents a sample document
% with two chapters, two parts, a title page,
% a compile flag as well as three forwarding files to set the flag.
% It consists of eight |.tex| files:
% \begin{center}
% \begin{tabular}{ll}
% |cdocsamp.tex|&main file\\
% |cdocsch1.tex|&include file for chapter 1\\
% |cdocsch2.tex|&include file for chapter 2\\
% |cdocspt3.tex|&include file for part 3\\
% |cdocspt4.tex|&include file for part 4\\
% |cdocsdrf.tex|&forwarding file for main file in draft mode\\
% |cdocsfi1.tex|&forwarding file for final version of chapter 1\\
% |cdocsfi2.tex|&forwarding file for final version of chapter 2\\
% \end{tabular}
% \end{center}
% Each of the eight files can be compiled directly by the \LaTeX{} compiler.
%
% %%%%%%%%%%%%%%%%%%%%%%%%%%%%%%%%%%%%%%
% \paragraph{Main File.}
%
% The main file is called |cdocsamp.tex|.
%
% Load the \textsf{childdoc} definitions and
% declare the filename for the main document:
%    \begin{macrocode}
\input{childdoc.def}
\childdocmain{}
%    \end{macrocode}

% Optional override for |\version| flag:
%    \begin{macrocode}
%%\ifchilddoc\else\providecommand{\version}{draft}\fi
%    \end{macrocode}

% Define the default values for the |\version| flag
% (|final| for the main file and |draft| for childs):
%    \begin{macrocode}
\ifchilddoc
\providecommand{\version}{draft}
\else
\providecommand{\version}{final}
\fi
%    \end{macrocode}

% Load the standard document class:
%    \begin{macrocode}
\documentclass[12pt]{article}
%    \end{macrocode}

% Start the document body:
%    \begin{macrocode}
\begin{document}
%    \end{macrocode}

% Declare a title page.
% Print title, part of document being processed and version flag:
%    \begin{macrocode}
\addtocounter{page}{-1}
\begin{center}
{\LARGE\bfseries{}childdoc example\par}
\vspace{1cm}
\ifchilddoc
\ifchilddocmanual part\else chapter\fi:
`\childdocname' of `\childdocjob'\par
\else
main document: `\childdocjob'\par
\fi
version: \version\par
\end{center}
\newpage
%    \end{macrocode}

% Manually include selected file,
% otherwise process as usual:
%    \begin{macrocode}
\ifchilddocmanual
\section*{part `\childdocname'}
\input{\childdocname}
\else
%    \end{macrocode}

% Include the two chapters:
%    \begin{macrocode}
\include{cdocsch1}
\include{cdocsch2}
%    \end{macrocode}

% Include the two parts unless only chapters should be displayed:
%    \begin{macrocode}
\ifchilddoc\else
\section{part three}
\input{cdocspt3}
\section{part four}
\input{cdocspt4}
\fi
%    \end{macrocode}

% Process as usual until here:
%    \begin{macrocode}
\fi
%    \end{macrocode}

% End of document body:
%    \begin{macrocode}
\end{document}
%    \end{macrocode}
%\iffalse
%</samplemain>
%\fi
%
% %%%%%%%%%%%%%%%%%%%%%%%%%%%%%%%%%%%%%%
% \paragraph{Chapter Include Files.}
%
% The include files are called |cdocsch1.tex| and |cdocsch2.tex|.
%
%\iffalse
%<*samplechap1|samplechap2>
%\fi

% Optional override for |\version| flag:
%    \begin{macrocode}
%%\providecommand{\version}{final}
%    \end{macrocode}

% Include the main document:
%    \begin{macrocode}
\input{childdoc.def}
\childdocof{cdocsamp}
%    \end{macrocode}

%\iffalse
%</samplechap1|samplechap2>
%\fi
%
%\iffalse
%<*samplechap1>
%\fi
% Some text for chapter 1:
%    \begin{macrocode}
\section{one}
some text in chapter one
%    \end{macrocode}

%\iffalse
%</samplechap1>
%\fi
% Some text for chapter 2:
%\iffalse
%<*samplechap2>
%\fi
%    \begin{macrocode}
\section{two}
more text in chapter two
%    \end{macrocode}

%\iffalse
%</samplechap2>
%\fi
%
% %%%%%%%%%%%%%%%%%%%%%%%%%%%%%%%%%%%%%%
% \paragraph{Part Include Files.}
%
% The include files are called |cdocspt3.tex| and |cdocspt4.tex|.
%
%\iffalse
%<*samplepart3|samplepart4>
%\fi

% Optional override for |\version| flag:
%    \begin{macrocode}
%%\providecommand{\version}{final}
%    \end{macrocode}

% Include the main document:
%    \begin{macrocode}
\input{childdoc.def}
\childdocby{cdocsamp}
%    \end{macrocode}

%\iffalse
%</samplepart3|samplepart4>
%\fi
%
%\iffalse
%<*samplepart3>
%\fi
% Some text for part 3:
%    \begin{macrocode}
some text in part three
%    \end{macrocode}

%\iffalse
%</samplepart3>
%\fi
% Some text for part 4:
%\iffalse
%<*samplepart4>
%\fi
%    \begin{macrocode}
more text in part four
%    \end{macrocode}

%\iffalse
%</samplepart4>
%\fi
%
% %%%%%%%%%%%%%%%%%%%%%%%%%%%%%%%%%%%%%%
% \paragraph{Forwarding for a Complete Draft.}
%
% The following forwarding file |cdocsdrf.tex|
% compiles the main document in draft mode:
%\iffalse
%<*sampledraft>
%\fi
%    \begin{macrocode}
\def\version{draft}
\input{childdoc.def}
\childdocforward{cdocsamp}
%    \end{macrocode}

%\iffalse
%</sampledraft>
%\fi
%
% %%%%%%%%%%%%%%%%%%%%%%%%%%%%%%%%%%%%%%
% \paragraph{Forwarding for Final Version of the Chapters.}
%
% The following forwarding files |cdocsfn1.tex| and |cdocsfn2.tex|
% (with identical content)
% compile the final versions of the child documents
% |cdocsch1.tex| and |cdocsch2.tex|, respectively:
%\iffalse
%<*samplefinal>
%\fi
%    \begin{macrocode}
\def\version{final}
\input{childdoc.def}
\childdocforwardprefix[cdocsamp]{cdocsfn}{cdocsch}
%    \end{macrocode}

%\iffalse
%</samplefinal>
%\fi
%
% %%%%%%%%%%%%%%%%%%%%%%%%%%%%%%%%%%%%%%
% \paragraph{Command Line Processing.}
%
% The following three command lines generate the output files
% |cdocscld|, |cdocscl1| and |cdocscl2|
% which should be identical to
% |cdocsdrf|, |cdocsch1| and |cdocsfn2|, respectively:
% \begin{center}
% \begin{tabular}{l}
% |latex -jobname cdocscld \|\\
% |  "\def\version{draft}\input{childdoc.def}\childdocforward{cdocsamp}"|\\
% |latex -jobname cdocscl1 \|\\
% |  "\input{childdoc.def}\childdocforward[cdocsamp]{cdocsch1}"|\\
% |latex -jobname cdocscl2 \|\\
% |  "\def\version{final}\input{childdoc.def}\childdocforward{cdocsch2}"|
% \end{tabular}
% \end{center}
% Note that the trailing backslash on each first line
% merely continues the input to the second line
% (for convenient cut ant paste).
% Furthermore, the command |latex| can be replaced by any
% of its alternative versions such as |pdflatex|.
%
% %%%%%%%%%%%%%%%%%%%%%%%%%%%%%%%%%%%%%%%%%%%%%%%%%%%%%%%%%%%%%%%%%%%%%%%%%%%%%%
% %%%%%%%%%%%%%%%%%%%%%%%%%%%%%%%%%%%%%%%%%%%%%%%%%%%%%%%%%%%%%%%%%%%%%%%%%%%%%%
% \section{Implementation}
%\iffalse
%<*package>
%\fi
%
% This section describes the definitions file |childdoc.def|.

% The definitions cannot be loaded using |\usepackage| or |\RequirePackage|
% which has a mechanism to prevent loading a style file more than once.
% When loading the definitions by means of |\input|
% multiple instances have to be prevented manually:
%\iffalse
%This code needs to be before the `\ProvidesFile' directive
%which is defined at the beginning of this file.
%Therefore it is also placed there and commented out here.
%</package>
%<*discard>
%\fi
%    \begin{macrocode}
\ifdefined\childdocmain\endinput\fi
%    \end{macrocode}
%\iffalse
%</discard>
%<*package>
%\fi
%
% \macro{\ifchilddoc}
% \macro{\ifchilddocmanual}
% The conditional |\ifchilddoc| tells whether a
% child (true) or main (false) document is being compiled.
% The conditional |\ifchilddocmanual| tells whether
% the |\includeonly| mechanism is used (false) or
% the selection of child files must be performed manually (true).
% The definitions initialise to false:
%    \begin{macrocode}
\newif\ifchilddoc
\newif\ifchilddocmanual
%    \end{macrocode}

% \macro{\childdocname}
% \macro{\childdocjob}
% The macro |\childdocname| stores the name of the main document
% to be compiled. The macro |\childdocjob| stores the name of
% the document on which the \LaTeX{} compiler was originally invoked.
% The content of |\jobname| cannot be compared
% to filenames specified in the source due to different catcodes.
% The following code rescans |\jobname|, stores the result
% in |\childdocname| and saves a copy in |\childdocjob|:
%    \begin{macrocode}
\edef\childdocname{\scantokens\expandafter{\jobname\noexpand}}
\let\childdocjob\childdocname
%    \end{macrocode}

% \macro{\childdocdisable}
% The macro |\childdocdisable| prevents the main file
% from being processed more than once.
% At this stage, the main document command |\childdocmain|
% is assumed to be called once again where it should do nothing.
% Any subsequent call to it should prevent
% a secondary processing of the main document
% It overwrites the forwarding commands
% |\childdocof| and |\childdocforward|
% with empty macros to prevent further inclusions of the main document:
%    \begin{macrocode}
\newcommand{\childdocdisable}
{
  \renewcommand{\childdocmain}[1]{\renewcommand{\childdocmain}[1]{\endinput}}
  \renewcommand{\childdocof}[1]{}
  \renewcommand{\childdocby}[2][]{}
  \renewcommand{\childdocforward}[2][]{}
  \renewcommand{\childdocdisable}{}
}
%    \end{macrocode}

% \macro{\childdocmain}
% The macro |\childdocmain| is to be called at the top of the main file
% with nothing or the main filename (without extension) as argument.
% First, it breaks loops.
% If the argument is not empty and does not match |\childdocname|
% (which is set by the first inclusion of |childdoc.def|),
% |\ifchilddoc| is set to true, |\includeonly| is applied to the child file
% and |\jobname| is set to the main file
% (for proper handling of |.aux| files):
%    \begin{macrocode}
\newcommand{\childdocmain}[1]
{
  \childdocdisable\childdocmain{}
  \if?#1?\else
    \begingroup
      \def\childdoctmp{#1}
      \ifx\childdoctmp\childdocname
        \def\childdoctmp{}
      \else
        \def\childdoctmp
        {
          \childdoctrue
          \includeonly{\childdocname}
          \def\childdocjob{#1}
          \def\jobname{#1}
        }
      \fi
      \expandafter
    \endgroup
    \childdoctmp
  \fi
}
%    \end{macrocode}

% \macro{\childdocof}
% The command |\childdocof| redirects
% compilation to the main file |#1|.
%    \begin{macrocode}
\newcommand{\childdocof}[1]
{
  \childdocdisable
  \childdoctrue
  \includeonly{\childdocname}
  \def\jobname{#1}
  \def\childdocjob{#1}
  \input{#1}
}
%    \end{macrocode}

% \macro{\childdocby}
% The command |\childdocby| ....
%    \begin{macrocode}
\newcommand{\childdocby}[2][]
{
  \childdocdisable
  \childdoctrue
  \childdocmanualtrue
  \if?#1?\else
    \def\jobname{#2}
  \fi
  \def\childdocjob{#2}
  \input{#2}
  \endinput
}
%    \end{macrocode}

% \macro{\childdocforward}
% The command |\childdocforward| redirects
% compilation to the main file or
% (if the optional argument is given) a child file.
% Parameters are set as if the main file
% or a child file starting with |\childdocof| was compiled.
% Then compilation is handed over to the main file:
%    \begin{macrocode}
\newcommand{\childdocforward}[2][]
{
  \begingroup
    \if?#1?
      \def\childdoctmp
      {
        \def\childdocname{#2}
        \def\childdocjob{#2}
        \def\jobname{#2}
        \input{#2}
        \endinput
      }
    \else
      \def\childdoctmp
      {
        \childdocdisable
        \def\childdocname{#2}
        \childdoctrue
        \includeonly{#2}
        \def\childdocjob{#1}
        \def\jobname{#1}
        \input{#1}
        \endinput
      }
    \fi
    \expandafter
  \endgroup
  \childdoctmp
}
%    \end{macrocode}

% \macro{\childdocforwardprefix}
% The command |\childdocforwardprefix| redirects
% compilation to the main or a child file by means of a pattern.
% The prefix |#1| in the current filename is replaced by |#2|
% and the suffix of the current filename is kept
% (it is assumed that the filename does not contain the substring `|~~~|'
% which is used as a delimiter).
% Compilation is handed over to the new file by |\childdocforward|:
%    \begin{macrocode}
\newcommand{\childdocforwardprefix}[3][]
{
  \begingroup
    \def\childdocextract #2##1~~~{\def\childdoctmp{\childdocforward[#1]{#3##1}}}
    \expandafter\childdocextract\childdocname~~~
    \expandafter
  \endgroup
  \childdoctmp
}
%    \end{macrocode}

% \macro{\childdoc}
% The deprecated macro |\childdoc| is a legacy version of |\childdocmain|:
%    \begin{macrocode}
\newcommand{\childdoc}{\childdocmain}
%    \end{macrocode}

% \macro{\childdocredirect}
% The deprecated macro |\childdocredirect| is a legacy version
% of |\childdocforward| and |\childdocforwardprefix|:
%    \begin{macrocode}
\newcommand{\childdocredirect}[2][]
{
  \begingroup
    \if?#1?
      \def\childdoctmp{\childdocforward{#2}}
    \else
      \def\childdoctmp{\childdocforwardprefix{#1}{#2}}
    \fi
    \expandafter
  \endgroup
  \childdoctmp
}
%    \end{macrocode}

%\iffalse
%</package>
%\fi
%
\endinput

\childdocmain{}
%    \end{macrocode}

% Optional override for |\version| flag:
%    \begin{macrocode}
%%\ifchilddoc\else\providecommand{\version}{draft}\fi
%    \end{macrocode}

% Define the default values for the |\version| flag
% (|final| for the main file and |draft| for childs):
%    \begin{macrocode}
\ifchilddoc
\providecommand{\version}{draft}
\else
\providecommand{\version}{final}
\fi
%    \end{macrocode}

% Load the standard document class:
%    \begin{macrocode}
\documentclass[12pt]{article}
%    \end{macrocode}

% Start the document body:
%    \begin{macrocode}
\begin{document}
%    \end{macrocode}

% Declare a title page.
% Print title, part of document being processed and version flag:
%    \begin{macrocode}
\addtocounter{page}{-1}
\begin{center}
{\LARGE\bfseries{}childdoc example\par}
\vspace{1cm}
\ifchilddoc
\ifchilddocmanual part\else chapter\fi:
`\childdocname' of `\childdocjob'\par
\else
main document: `\childdocjob'\par
\fi
version: \version\par
\end{center}
\newpage
%    \end{macrocode}

% Manually include selected file,
% otherwise process as usual:
%    \begin{macrocode}
\ifchilddocmanual
\section*{part `\childdocname'}
\input{\childdocname}
\else
%    \end{macrocode}

% Include the two chapters:
%    \begin{macrocode}
\include{cdocsch1}
\include{cdocsch2}
%    \end{macrocode}

% Include the two parts unless only chapters should be displayed:
%    \begin{macrocode}
\ifchilddoc\else
\section{part three}
\input{cdocspt3}
\section{part four}
\input{cdocspt4}
\fi
%    \end{macrocode}

% Process as usual until here:
%    \begin{macrocode}
\fi
%    \end{macrocode}

% End of document body:
%    \begin{macrocode}
\end{document}
%    \end{macrocode}
%\iffalse
%</samplemain>
%\fi
%
% %%%%%%%%%%%%%%%%%%%%%%%%%%%%%%%%%%%%%%
% \paragraph{Chapter Include Files.}
%
% The include files are called |cdocsch1.tex| and |cdocsch2.tex|.
%
%\iffalse
%<*samplechap1|samplechap2>
%\fi

% Optional override for |\version| flag:
%    \begin{macrocode}
%%\providecommand{\version}{final}
%    \end{macrocode}

% Include the main document:
%    \begin{macrocode}
% \iffalse
%
% childdoc.dtx Copyright (C) 2017-2018 Niklas Beisert
%
% This work may be distributed and/or modified under the
% conditions of the LaTeX Project Public License, either version 1.3
% of this license or (at your option) any later version.
% The latest version of this license is in
%   http://www.latex-project.org/lppl.txt
% and version 1.3 or later is part of all distributions of LaTeX
% version 2005/12/01 or later.
%
% This work has the LPPL maintenance status `maintained'.
%
% The Current Maintainer of this work is Niklas Beisert.
%
% This work consists of the files childdoc.dtx and childdoc.ins
% and the derived files childdoc.def and cdocsamp.tex with
% cdocsch1.tex, cdocsch2.tex, cdocsdrf.tex, cdocsfn1.tex, cdocsfn2.tex.
%
%<package>\ifdefined\childdocmain\endinput\fi
%<package>\ProvidesFile{childdoc.def}[2018/12/30 v2.0 child document driver]
%<samplemain>\ProvidesFile{cdocsamp.tex}[2018/12/30 v2.0 sample for childdoc]
%<*driver>
%\ProvidesFile{childdoc.drv}[2018/12/30 v2.0 childdoc reference manual file]
\PassOptionsToClass{10pt,a4paper}{article}
\documentclass{ltxdoc}

\usepackage[margin=35mm]{geometry}
\usepackage{hyperref}
\usepackage{hyperxmp}
\usepackage[usenames]{color}

\hypersetup{colorlinks=true}
\hypersetup{pdfstartview=FitH}
\hypersetup{pdfpagemode=UseNone}
\hypersetup{pdfsource={}}
\hypersetup{pdflang={en-UK}}
\hypersetup{pdfcopyright={Copyright 2017-2018 Niklas Beisert.
  This work may be distributed and/or modified under the
  conditions of the LaTeX Project Public License, either version 1.3
  of this license or (at your option) any later version.}}
\hypersetup{pdflicenseurl={http://www.latex-project.org/lppl.txt}}
\hypersetup{pdfcontactaddress={ETH Zurich, ITP, HIT K,
  Wolfgang-Pauli-Strasse 27}}
\hypersetup{pdfcontactpostcode={8093}}
\hypersetup{pdfcontactcity={Zurich}}
\hypersetup{pdfcontactcountry={Switzerland}}
\hypersetup{pdfcontactemail={nbeisert@itp.phys.ethz.ch}}
\hypersetup{pdfcontacturl={http://people.phys.ethz.ch/\xmptilde nbeisert/}}

\newcommand{\secref}[1]{\hyperref[#1]{section \ref*{#1}}}

\parskip1ex
\parindent0pt
\let\olditemize\itemize
\def\itemize{\olditemize\parskip0pt}

\begin{document}

\title{The \textsf{childdoc} Package}
\hypersetup{pdftitle={The childdoc Package}}
\author{Niklas Beisert\\[2ex]
  Institut f\"ur Theoretische Physik\\
  Eidgen\"ossische Technische Hochschule Z\"urich\\
  Wolfgang-Pauli-Strasse 27, 8093 Z\"urich, Switzerland\\[1ex]
  \href{mailto:nbeisert@itp.phys.ethz.ch}
  {\texttt{nbeisert@itp.phys.ethz.ch}}}
\hypersetup{pdfauthor={Niklas Beisert}}
\hypersetup{pdfsubject={Manual for the LaTeX2e Package childdoc}}
\date{30 December 2018, \textsf{v2.0}}
\maketitle

\begin{abstract}\noindent
\textsf{childdoc} is a \LaTeXe{} package
that enables the direct compilation
of document sections included by |\include|
to individual files.
\end{abstract}

\begingroup
\parskip0ex
\tableofcontents
\endgroup

%%%%%%%%%%%%%%%%%%%%%%%%%%%%%%%%%%%%%%%%%%%%%%%%%%%%%%%%%%%%%%%%%%%%%%%%%%%%%%%%
%%%%%%%%%%%%%%%%%%%%%%%%%%%%%%%%%%%%%%%%%%%%%%%%%%%%%%%%%%%%%%%%%%%%%%%%%%%%%%%%
\section{Introduction}

\LaTeX{} provides a mechanism to structure a large document (such as a book)
into a main file and several child files (containing the chapters)
using the |\include| command.
This mechanism is beneficial for documents
which span hundreds of pages in order to
make the source file(s) more manageable.
Moreover, compilation can be restricted to
selected child files by means of the |\includeonly| command.
The latter feature can be used to reduce the compilation time while editing
(this was significantly more useful in the earlier days of \LaTeX{})
or to generate a smaller document which is easier to navigate.
Another application of |\includeonly| is to generate
documents consisting of selected parts of the complete document.

However, there are a few drawbacks of the plain |\include| mechanism:
\begin{itemize}
\item
The child files cannot be compiled on their own,
they can only be compiled via the main file.
A naive editing environment
(such as a text editor with an option
to have the current file processed by \LaTeX)
may require one to switch to the main file before compiling;
attempting to compile the child file produces errors.
\item
The main file must be modified (each time)
to adjust the |\includeonly| command
to the present needs. This easily leaves the main file in a messy state.
\item
The generated document will always carry the filename
of the main document. This is inconvenient if
several child files are to be compiled and
to be kept for distribution.
\end{itemize}

The present package provides a simple interface
to make child files individually compilable by \LaTeX{}.
Compiling a child file then has the same effect as compiling
the main file with an |\includeonly| command
to select the appropriate child.
Moreover the generated document will carry the name of the child
rather than the main file.
This resolves all three above issues.

This feature is meant to make the editing of books,
thesis documents and lecture notes somewhat more convenient.
However, the package can also be used efficiently for
composing a series of documents (such as exercise sheets)
which are typically distributed individually.
It then assists the author in generating the individual documents
(potentially in different versions)
as well as a document containing the collected series.
Another application is in developing style files
or other kinds of included material
where compilation of the style file could redirect
to a sample or test file.

%%%%%%%%%%%%%%%%%%%%%%%%%%%%%%%%%%%%%%%%%%%%%%%%%%%%%%%%%%%%%%%%%%%%%%%%%%%%%%%%
%%%%%%%%%%%%%%%%%%%%%%%%%%%%%%%%%%%%%%%%%%%%%%%%%%%%%%%%%%%%%%%%%%%%%%%%%%%%%%%%
\section{Usage}

First of all, the package \textsf{childdoc} is \emph{not} a standard
\LaTeXe{} |.sty| style file! Therefore it needs to be invoked in
a non-standard way.

%%%%%%%%%%%%%%%%%%%%%%%%%%%%%%%%%%%%%%%%%%%%%%%%%%%%%%%%%%%%%%%%%%%%%%%%%%%%%%%%
\subsection{Included Files}
\label{sec:include}

%%%%%%%%%%%%%%%%%%%%%%%%%%%%%%%%%%%%%%%%
\DescribeMacro{\childdocmain}
To use the package, add the commands
\begin{center}
\begin{tabular}{l}
|\input{childdoc.def}|\\
|\childdocmain{}|\\
\end{tabular}
\end{center}
at the very top of the main \LaTeX{} file,
in particular \emph{before} the |\documentclass| statement!
The argument of |\childdocmain| should be left empty
(but it must be present).

%%%%%%%%%%%%%%%%%%%%%%%%%%%%%%%%%%%%%%%%
\DescribeMacro{\childdocof}
Furthermore, add the commands
\begin{center}
\begin{tabular}{l}
|\input{childdoc.def}|\\
|\childdocof{|\textit{main}|}|\\
\end{tabular}
\end{center}
at the top of every child file \textit{child}
which is included by |\include{|\textit{child}|}|
from within the main file
(or at least for those files to be compiled individually).
The argument \textit{main} must be the filename of the main file.

There are a couple of
considerations in setting up the main and child documents:

%%%%%%%%%%%%%%%%%%%%%%%%%%%%%%%%%%%%%%%%
\paragraph{Restrictions.}

Please note the following restrictions:
\begin{itemize}
\item
|\childdocmain| must be called with one argument \textit{main}
to ensure compatibility with earlier version of the package.
It must either be empty (|\childdocmain{}|)
or precisely match the filename of the main file in which it is specified.
See \secref{sec:detection} for further information.
\item
The filename \textit{main} must be specified without the |.tex| extension.
\item
The filename \textit{main} is case sensitive
(even in case-insensitive file systems)
due to internal string comparison.
\item
The argument \textit{main} should be fully expanded, it cannot be a macro.
\item
Subdirectories and special characters should be avoided in filenames.
\item
The command |\childdocmain{|\textit{main}|}| must be followed by a whitespace.
It should not be followed immediately by another command
or by a comment mark `|%|'.
This is because the \TeX{} parser reads the token immediately following
the argument of |\childdocmain| and puts it
at the beginning of every child section;
however, a white\-space is ignored.
\end{itemize}

%%%%%%%%%%%%%%%%%%%%%%%%%%%%%%%%%%%%%%%%
\paragraph{Content of Main File.}

It is advisable to place all content in the child files included by |\include|.
Any output contained in the main file will appear in all child documents
unless suppressed manually;
it cannot be suppressed automatically by the |\includeonly| directive
and thus should normally be avoided.
A method to include some content in the main file
by means of conditional processing is described in \secref{sec:conditional}.

%%%%%%%%%%%%%%%%%%%%%%%%%%%%%%%%%%%%%%%%
\paragraph{Page Numbering.}

When only a part of the document is compiled,
the appropriate numbering of pages
(as well as other status parameters)
is determined from the |.aux| files.
The latter contain information from previous passes.
However this information needs to propagate through
all intermediate child documents.
Therefore the page numbering in child documents may well
be inconsistent until the complete document is compiled at least once.

A useful (if unconventional) way to always ensure a consistent
page numbering is to restart the numbering in each child document
and denote the pages by `\textit{child}|.|\textit{page}'
where \textit{child} represents the chapter/section number of the child file.
This can be achieved by the command
|\numberwithin{page}{|\textit{child}|}|
of the \textsf{amsmath} package
where \textit{child} can be |chapter| or |section|
depending on the chosen structuring.
Alternatively, one can modify the macro |\thepage| appropriately
and reset the counter |page| at the start of each child file.

%%%%%%%%%%%%%%%%%%%%%%%%%%%%%%%%%%%%%%%%%%%%%%%%%%%%%%%%%%%%%%%%%%%%%%%%%%%%%%%%
\subsection{Conditional Processing}
\label{sec:conditional}

The package provides a mechanism to compile different versions
of a document. To customise the versions further some conditional processing
can come in handy to distinguish which version is being compiled.
The package provides two macros to describe the compilation context:

%%%%%%%%%%%%%%%%%%%%%%%%%%%%%%%%%%%%%%%%
\DescribeMacro{\ifchilddoc}
The conditional |\ifchilddoc| distinguishes between the compilation of
child documents and the main document:
%
\begin{center}
|\ifchilddoc |\textit{child-code}| |[|\||else |\textit{main-code}]| \||fi|
\end{center}

%%%%%%%%%%%%%%%%%%%%%%%%%%%%%%%%%%%%%%%%
\DescribeMacro{\childdocname}
\DescribeMacro{\childdocjob}
The macro |\childdocname| contains the filename (without extension)
of the main or child file being processed.
Note that |\childdocjob| will always contain the name of the main file.

%%%%%%%%%%%%%%%%%%%%%%%%%%%%%%%%%%%%%%%%
\paragraph{Title Page.}

Conditional processing can be used to include a title or banner page
in the main document when proper precautions are taken.
Importantly, the code in the main file should ensure that the page counter
(as well as other status parameters which are stored in the |.aux| files)
takes the same value after the conditional processing.
Otherwise the page numbers may take divergent values
depending on which part is compiled.

For example, a title page could be declared by:
%
\begin{center}
\begin{tabular}{l}
|\ifchilddoc\||else|\\
|\addtocounter{page}{-1}|\\
\textit{code for title page}\\
|\newpage|\\
|\||fi|
\end{tabular}
\end{center}
%
A banner page for the child documents can be generated by:
%
\begin{center}
\begin{tabular}{l}
|\ifchilddoc|\\
|\addtocounter{page}{-1}|\\
\textit{code for banner page}\\
|\newpage|\\
|\||fi|
\end{tabular}
\end{center}
%
Here one could write a message such as:
\begin{center}
|This is the part \childdocname{} of \childdocjob{}.|
\end{center}

%%%%%%%%%%%%%%%%%%%%%%%%%%%%%%%%%%%%%%%%%%%%%%%%%%%%%%%%%%%%%%%%%%%%%%%%%%%%%%%%
\subsection{Flags}
\label{sec:flags}

The package makes it easy to generate different versions
of the main or child documents.
To this end compilation flags can be defined
and assigned different default values.
They will be particularly useful in conjunction
with the forwarding mechanism described in \secref{sec:forward}.

For example, it may be useful to have a flag |\version|
which can be set to |draft| or |final|.
The document source will contain some conditional code
depending on the value of |\version|.
Suppose further, the flag should default to |final| for the main file
and to |draft| for child files
which is a natural assignment for editing the document.
This is achieved by placing the following code
in the preamble of the main document
(below the |\childdocmain| directive):
%
\begin{center}
\begin{tabular}{l}
|\ifchilddoc|\\
|\providecommand{\version}{draft}|\\
|\||else|\\
|\providecommand{\version}{final}|\\
|\||fi|
\end{tabular}
\end{center}
%
The definition by |\providecommand| makes sure
that previous definitions are not overwritten.
Further statements |\providecommand{\version}{...}|
can thus be added before the above code to override it.

For the main file, one might add a line
(between |\childdocmain| and the above block)
%
\begin{center}
|%\ifchilddoc\||else\providecommand{\version}{draft}\||fi|
\end{center}
%
which can be uncommented to produce a draft version.
Likewise one can add a line to the very top of a child file
(above the |\childdocof{|\textit{main}|}| directive)
%
\begin{center}
|%\providecommand{\version}{final}|
\end{center}
%
which can be uncommented to produce the final version of this child document.

%%%%%%%%%%%%%%%%%%%%%%%%%%%%%%%%%%%%%%%%%%%%%%%%%%%%%%%%%%%%%%%%%%%%%%%%%%%%%%%%
\subsection{Forwarding}
\label{sec:forward}

Different versions of the main or child documents
using compilation flags as described in \secref{sec:flags}
can be (permanently) stored in different files
for convenient compilation, viewing and distribution.
To this end, the package defines a command
to pass on compilation to a different file:

%%%%%%%%%%%%%%%%%%%%%%%%%%%%%%%%%%%%%%%%
\DescribeMacro{\childdocforward}
The command |\childdocforward| redirects processing to
another source file:
%
\begin{center}
\begin{tabular}{l}
|\input{childdoc.def}|\\
|\childdocforward[|\textit{main}|]{|\textit{dest}|}|\\
\end{tabular}
\end{center}
%
The argument \textit{dest} is the destination file
(without extension).
It should be the main file or one of the child files.
Note that further \textsf{childdoc} directives
such as |\childdocof| and |\childdocforward|
in the indicated file will be processed in this form.
The optional argument \textit{main}
passes on directly to the main file \textit{main}
while pretending to compile the child \textit{dest}.
This form behaves as if \textit{dest}
issues |\childdocof{|\textit{main}|}| right away,
and no further \textsf{childdoc} directives will be processed.

%%%%%%%%%%%%%%%%%%%%%%%%%%%%%%%%%%%%%%%%
\DescribeMacro{\...prefix}
In the alternative form |\childdocforwardprefix|,
%
\begin{center}
\begin{tabular}{l}
|\input{childdoc.def}|\\
|\childdocforwardprefix[|\textit{main}|]{|\textit{prefix}|}{|\textit{dest}|}|
\end{tabular}
\end{center}
%
the destination file is determined by a pattern
depending on the current file:
To make this work, the current file must be called
`{\textit{prefix}\hspace{0.2em}\textit{suffix}}'
with \textit{prefix} matching precisely the argument.
Processing is then passed on to the file
`{\textit{dest}\hspace{0.2em}\textit{suffix}}'.
Surely, the same effect is achieved by
directly specifying the
argument `{\textit{dest}\hspace{0.2em}\textit{suffix}}'
in the first form.
However, that requires to set up a different file
for each child. With the alternative form of the command
all these files can have exactly the same content
which simplifies setting them up and maintaining them.

For example, the following file |draft.tex|
with a compilation flag |\version| as described in \secref{sec:flags}
compiles the main document as a draft:
%
\begin{center}
\begin{tabular}{l}
|\def\version{draft}|\\
|\input{childdoc.def}|\\
|\childdocforward{|\textit{main}|}|
\end{tabular}
\end{center}
%
Likewise, the following files |final|\textit{nn}|.tex|
compile the final version of the child document
|child|\textit{nn}|.tex|:
%
\begin{center}
\begin{tabular}{l}
|\def\version{final}|\\
|\input{childdoc.def}|\\
|\childdocforwardprefix{final}{child}|
\end{tabular}
\end{center}
%

Note that when several versions of a main file and/or of each child file
are to be generated, it may be convenient to set up a |Makefile| or
shell script to automatise the process.

%%%%%%%%%%%%%%%%%%%%%%%%%%%%%%%%%%%%%%%%%%%%%%%%%%%%%%%%%%%%%%%%%%%%%%%%%%%%%%%%
\subsection{Command Line Processing}
\label{sec:commandline}

The effect of redirection files can also be achieved by invoking
the \LaTeX{} compiler with a more elaborate command line.
Most conveniently this should be done as part
of a shell script or a |Makefile|.

When using \textsf{childdoc} in the main file, the following
command lines effectively perform a redirection
(note that depending on the shell being used,
backslashes may have to be doubled: `|\|' $\to$ `|\\|'):
%
\begin{center}
|... -jobname "|\textit{target}|" |\\|"|[\textit{flags}]%
|\input{childdoc.def}\childdocforward[|\textit{main}|]{|\textit{dest}|}"|
\end{center}
%
Here \textit{target} is the name of the output file,
\textit{main} is the name of the main file
and \textit{dest} is the name of the main or child file to be processed
(all filenames without extensions).
The optional argument \textit{main} can be omitted
if \textit{main} matches \textit{dest}.
Optionally, compilation \textit{flags} can be defined via |\def| commands.
This command line makes the \TeX{} engine believe
it is compiling the file \textit{target}
whose content is specified as the latter parameter.
The provided code then forwards the processing to
\textit{main} or \textit{dest} as described in \secref{sec:forward}.

%%%%%%%%%%%%%%%%%%%%%%%%%%%%%%%%%%%%%%%%%%%%%%%%%%%%%%%%%%%%%%%%%%%%%%%%%%%%%%%%
\subsection{Include by Input}
\label{sec:input}

Including child documents by |\include| has some restrictions by design.
Most notably, the content of a child document always occupies
its own set of pages; pages cannot be shared between child documents.
Usually, this behaviour makes perfect sense
because each child document contain an essential part of the document.
However, in some situations it may be desirable to compose
a document from a collection of parts
without having mandatory page breaks between then.
For this case, the package
provides a mechanism to include parts
by |\input| which can also be processed individually.
However, by construction this mechanism
requires manual handling of the content to be output.

%%%%%%%%%%%%%%%%%%%%%%%%%%%%%%%%%%%%%%%%
\DescribeMacro{\ifchilddocmanual}
The main file should be prepared as usual, see \secref{sec:include}.
However, the document body must make a distinction
between processing of an individual part and of the main document, e.g.:
%
\begin{center}
\begin{tabular}{l}
|\ifchilddocmanual|\\
|\input{\childdocname}|\\
|\||else|\\
\textit{document body with }|\input{|\textit{part}|}|\\
|\||fi|
\end{tabular}
\end{center}
%
The conditional |\ifchilddocmanual| is true whenever
a part to be included by |\input| is being compiled,
and the name of the part is stored in |\childdocname|.

%%%%%%%%%%%%%%%%%%%%%%%%%%%%%%%%%%%%%%%%
\DescribeMacro{\childdocby}
Each part to be included by |\input| should start with:
%
\begin{center}
\begin{tabular}{l}
|\input{childdoc.def}|\\
|\childdocby{|\textit{main}|}|\\
\end{tabular}
\end{center}
%
The directive |\childdocby| is similar to |\childdocof|
described in \secref{sec:include},
but the subsequent selection of content must be done manually.
To that end, both |\ifchilddoc| and |\ifchilddocmanual|
will be true upon processing of a part,
and the name of the part is stored in |\childdocname|.
Note that |\jobname| will be set to the filename of the current part
so that each part receives an individual |.aux| file
that does not interfere with the |.aux| file(s) of the main document.
This behaviour can be altered by the alternative form
|\childdocby[*]{|\textit{main}|}| (with a non-empty optional argument)
which uses the |.aux| file of the main document
by setting |\jobname| to \textit{main}.

%%%%%%%%%%%%%%%%%%%%%%%%%%%%%%%%%%%%%%%%%%%%%%%%%%%%%%%%%%%%%%%%%%%%%%%%%%%%%%%%
\subsection{Driver Development}
\label{sec:driver}

The \textsf{childdoc} mechanism can also be use for the development
of definition files such as \LaTeX{} styles or classes.
This case differs from the above setup with multiple parts
included by |\include| in that no |\includeonly| should be invoked.
This can be achieved by starting the include file
(before |\ProvidesPackage|) with:
%
\begin{center}
\begin{tabular}{l}
|\input{childdoc.def}|\\
|\childdocforward{|\textit{main}|}|\\
\end{tabular}
\end{center}
%
or alternatively with:
%
\begin{center}
\begin{tabular}{l}
|\input{childdoc.def}|\\
|\childdocby{|\textit{main}|}|\\
\end{tabular}
\end{center}
%
Both forms have slightly different effects as described above.
The main file is prepared as usual, see \secref{sec:include}.

%%%%%%%%%%%%%%%%%%%%%%%%%%%%%%%%%%%%%%%%%%%%%%%%%%%%%%%%%%%%%%%%%%%%%%%%%%%%%%%%
\subsection{Legacy Detection}
\label{sec:detection}

The directive |\childdocmain| in the main file can detect
whether the complete document or merely a child is to be compiled
even without using the directive |\childdocof|.
This method is deprecated because it is less robust
and there is no compelling reason to use it;
it is merely provided for backward compatibility
and it may be removed in future versions.

If the detection mechanism is to be used,
it is mandatory to correctly specify
the filename of the main file as the argument of |\childdocmain|:
%
\begin{center}
\begin{tabular}{l}
|\input{childdoc.def}|\\
|\childdocmain{|\textit{main}|}|\\
\end{tabular}
\end{center}
%
If |\jobname| does not match the argument \textit{main} of |\childdocmain|,
it is assumed that |\jobname| points to the child file to be compiled.
When using |\childdocmain| with the main file specified as argument,
it suffices to start a child file
with just |\input{|\textit{main}|}|
without loading of the package and using |\childdocof|.
If instead all processing is done
with the appropriate \textsf{childdoc} directives,
the argument of \textit{main} of |\childdocmain| can be empty.

An alternative version of the command line processing described
in \secref{sec:commandline} using the detection mechanism reads:
%
\begin{center}
|... -jobname "|\textit{target}|" "|[\textit{flags}]%
[|\def\jobname{|\textit{dest}|}|]|\input{|\textit{main}|}"|
\end{center}

%%%%%%%%%%%%%%%%%%%%%%%%%%%%%%%%%%%%%%%%%%%%%%%%%%%%%%%%%%%%%%%%%%%%%%%%%%%%%%%%
\subsection{Manual Code}
\label{sec:manual}

In case one cannot be certain whether the definitions file |childdoc.def|
is installed on the target \TeX{} distribution
and one prefers not to ship it,
it is conceivable to paste a few relevant commands into the sources.

To that end, drop all statements |\input{childdoc.def}|
and perform the replacements as outlined below.
Instead of |\childdocmain{|\textit{main}|}| add the following code
to the top of the main file:
%
\begin{center}
\begin{tabular}{l}
|\||ifdefined\childdocname\endinput\||fi\newif\ifchilddoc|\\
|\edef\childdocname{\scantokens\expandafter{\jobname\noexpand}}|\\
|\def\childdocmain{|\textit{main}|}\||ifx\childdocmain\childdocname\||else|\\
|\childdoctrue\includeonly{\childdocname}\let\jobname\childdocmain\||fi|\\
\end{tabular}
\end{center}
%
Instead of |\childdocof{|\textit{main}|}| just include the main file
at the top of each child file:
%
\begin{center}
|\input{|\textit{main}|}|
\end{center}
%
A simple redirection |\childdocforward{|\textit{dest}|}| is achieved by:
%
\begin{center}
|\def\jobname{|\textit{dest}|}\input{\jobname}|
\end{center}
%
The redirection with prefix
|\childdocforwardprefix[|\textit{prefix}|]{|\textit{dest}|}|
is accomplished by:
%
\begin{center}
\begin{tabular}{l}
|{\edef\jobname{\scantokens\expandafter{\jobname\noexpand}}|\\
|\def\redirectjob |\textit{prefix}|#1~~~{\gdef\jobname{|\textit{dest}|#1}}|\\
|\expandafter\redirectjob\jobname~~~}\input{\jobname}|
\end{tabular}
\end{center}

In an alternative approach,
child documents can be compiled by a specific command line
without additional code or specific definitions:
%
\begin{center}
|... -jobname "|\textit{target}|" "|[\textit{flags}]%
|\includeonly{|\textit{dest}|}\input{|\textit{main}|}"|
\end{center}
%

%%%%%%%%%%%%%%%%%%%%%%%%%%%%%%%%%%%%%%%%%%%%%%%%%%%%%%%%%%%%%%%%%%%%%%%%%%%%%%%%
%%%%%%%%%%%%%%%%%%%%%%%%%%%%%%%%%%%%%%%%%%%%%%%%%%%%%%%%%%%%%%%%%%%%%%%%%%%%%%%%
\section{Information}

%%%%%%%%%%%%%%%%%%%%%%%%%%%%%%%%%%%%%%%%%%%%%%%%%%%%%%%%%%%%%%%%%%%%%%%%%%%%%%%%
\subsection{Copyright}

Copyright \copyright{} 2017--2018 Niklas Beisert

This work may be distributed and/or modified under the
conditions of the \LaTeX{} Project Public License, either version 1.3
of this license or (at your option) any later version.
The latest version of this license is in
  \url{http://www.latex-project.org/lppl.txt}
and version 1.3 or later is part of all distributions of \LaTeX{}
version 2005/12/01 or later.

This work has the LPPL maintenance status `maintained'.

The Current Maintainer of this work is Niklas Beisert.

This work consists of the files |README.txt|, |childdoc.ins| and |childdoc.dtx|
as well as the derived files |childdoc.def|, |cdocsamp.tex|
with |cdocsch1.tex|, |cdocsch2.tex|, |cdocspt3.tex|, |cdocspt4.tex|,
|cdocsdrf.tex|, |cdocsfn1.tex|, |cdocsfn2.tex|
as well as |childdoc.pdf|.

%%%%%%%%%%%%%%%%%%%%%%%%%%%%%%%%%%%%%%%%%%%%%%%%%%%%%%%%%%%%%%%%%%%%%%%%%%%%%%%%
\subsection{Files and Installation}

The package consists of the files:
%
\begin{center}
\begin{tabular}{ll}
    |README.txt|   & readme file \\
    |childdoc.ins| & installation file \\
    |childdoc.dtx| & source file \\
    |childdoc.def| & definition file \\
    |cdocsamp.tex| & sample main file \\
    |cdocsch1.tex| & sample include file \\
    |cdocsch2.tex| & sample include file \\
    |cdocspt3.tex| & sample part file \\
    |cdocspt4.tex| & sample part file \\
    |cdocsdrf.tex| & sample redirection file \\
    |cdocsfn1.tex| & sample redirection file \\
    |cdocsfn2.tex| & sample redirection file \\
    |childdoc.pdf| & manual
\end{tabular}
\end{center}
%
The distribution consists of the files
|README.txt|, |childdoc.ins| and |childdoc.dtx|.
%
\begin{itemize}
\item
Run (pdf)\LaTeX{} on |childdoc.dtx|
to compile the manual |childdoc.pdf| (this file).
\item
Run \LaTeX{} on |childdoc.ins| to create the definitions file |childdoc.def|
and the sample |cdocsamp.tex| with include files
|cdocsch1.tex|, |cdocsch2.tex|, |cdocspt3.tex|, |cdocspt4.tex|,
|cdocsdrf.tex|, |cdocsfn1.tex|, |cdocsfn2.tex|.
Then copy the file |childdoc.def| to an appropriate directory of your \LaTeX{}
distribution, e.g.\ \textit{texmf-root}|/tex/latex/childdoc|.
\end{itemize}

%%%%%%%%%%%%%%%%%%%%%%%%%%%%%%%%%%%%%%%%%%%%%%%%%%%%%%%%%%%%%%%%%%%%%%%%%%%%%%%%
\subsection{Related CTAN Packages}

There are several other packages which offer a similar functionality:
%
\begin{itemize}
\item
The packages
\href{http://ctan.org/pkg/docmute}{\textsf{docmute}},
\href{http://ctan.org/pkg/includex}{\textsf{includex}} and
\href{http://ctan.org/pkg/standalone}{\textsf{standalone}}
provide commands to include only the document body of
a child file thus allowing both files to be compiled individually.
\item
The packages \href{http://ctan.org/pkg/subdocs}{\textsf{subdocs}}
and \href{http://ctan.org/pkg/subfiles}{\textsf{subfiles}}
provide structures in which the main and child documents can be
encapsulated and allowing them to be compiled individually.
The inclusion mechanism is different from the conventional |\include|.
\item
The package \href{http://ctan.org/pkg/combine}{\textsf{combine}}
is an elaborate solution to combine several documents into one.
\end{itemize}
%
See also the CTAN topic \href{http://ctan.org/topic/subdocs}{\textsf{subdocs}}
for further related packages.
The present package differs from the above solutions in that
a document structure constructed with the conventional |\include| mechanism
just needs two extra commands at the top of every file
such that all constituent files can be compiled individually.

%%%%%%%%%%%%%%%%%%%%%%%%%%%%%%%%%%%%%%%%%%%%%%%%%%%%%%%%%%%%%%%%%%%%%%%%%%%%%%%%
%\subsection{Feature Suggestions}
%
%The following is a list of features which may be useful for future
%versions of this package:
%%
%\begin{itemize}
%\item
%\ldots
%\end{itemize}

%%%%%%%%%%%%%%%%%%%%%%%%%%%%%%%%%%%%%%%%%%%%%%%%%%%%%%%%%%%%%%%%%%%%%%%%%%%%%%%%
\subsection{Revision History}

%%%%%%%%%%%%%%%%%%%%%%%%%%%%%%%%%%%%%%%%
\paragraph{v2.0:} 2018/12/30

\begin{itemize}
\item
immediate forward processing
\item
added |\childdocby| mechanism
\item
manual restructured
\end{itemize}

%%%%%%%%%%%%%%%%%%%%%%%%%%%%%%%%%%%%%%%%
\paragraph{v1.6:} 2018/01/17

\begin{itemize}
\item
application for development of include files
\item
corrections to manual
\end{itemize}

%%%%%%%%%%%%%%%%%%%%%%%%%%%%%%%%%%%%%%%%
\paragraph{v1.5:} 2017/05/21

\begin{itemize}
\item
more complete structuring introduced
\item
|\childdocof| introduced
\item
|\childdoc| renamed to |\childdocmain|
\item
|\childredirect| renamed to |\childdocforward| and |\childdocforwardprefix|
and functionality expanded
\end{itemize}

%%%%%%%%%%%%%%%%%%%%%%%%%%%%%%%%%%%%%%%%
\paragraph{v1.0:} 2017/04/27

\begin{itemize}
\item
manual and install package
\item
first version published on CTAN
\end{itemize}

%%%%%%%%%%%%%%%%%%%%%%%%%%%%%%%%%%%%%%%%
\paragraph{v0.6:} 2017/04/26

\begin{itemize}
\item
redirection mechanism added
\end{itemize}

%%%%%%%%%%%%%%%%%%%%%%%%%%%%%%%%%%%%%%%%
\paragraph{v0.5:} 2017/04/26

\begin{itemize}
\item
functionality in definition file
\end{itemize}


%%%%%%%%%%%%%%%%%%%%%%%%%%%%%%%%%%%%%%%%%%%%%%%%%%%%%%%%%%%%%%%%%%%%%%%%%%%%%%%%
%%%%%%%%%%%%%%%%%%%%%%%%%%%%%%%%%%%%%%%%%%%%%%%%%%%%%%%%%%%%%%%%%%%%%%%%%%%%%%%%
%%%%%%%%%%%%%%%%%%%%%%%%%%%%%%%%%%%%%%%%%%%%%%%%%%%%%%%%%%%%%%%%%%%%%%%%%%%%%%%%
\appendix

\settowidth\MacroIndent{\rmfamily\scriptsize 000\ }

 \DocInput{childdoc.dtx}

\end{document}
%</driver>
% \fi
%
% %%%%%%%%%%%%%%%%%%%%%%%%%%%%%%%%%%%%%%%%%%%%%%%%%%%%%%%%%%%%%%%%%%%%%%%%%%%%%%
% %%%%%%%%%%%%%%%%%%%%%%%%%%%%%%%%%%%%%%%%%%%%%%%%%%%%%%%%%%%%%%%%%%%%%%%%%%%%%%
% \section{Sample}
%\iffalse
%<*samplemain>
%\fi
%
% The following presents a sample document
% with two chapters, two parts, a title page,
% a compile flag as well as three forwarding files to set the flag.
% It consists of eight |.tex| files:
% \begin{center}
% \begin{tabular}{ll}
% |cdocsamp.tex|&main file\\
% |cdocsch1.tex|&include file for chapter 1\\
% |cdocsch2.tex|&include file for chapter 2\\
% |cdocspt3.tex|&include file for part 3\\
% |cdocspt4.tex|&include file for part 4\\
% |cdocsdrf.tex|&forwarding file for main file in draft mode\\
% |cdocsfi1.tex|&forwarding file for final version of chapter 1\\
% |cdocsfi2.tex|&forwarding file for final version of chapter 2\\
% \end{tabular}
% \end{center}
% Each of the eight files can be compiled directly by the \LaTeX{} compiler.
%
% %%%%%%%%%%%%%%%%%%%%%%%%%%%%%%%%%%%%%%
% \paragraph{Main File.}
%
% The main file is called |cdocsamp.tex|.
%
% Load the \textsf{childdoc} definitions and
% declare the filename for the main document:
%    \begin{macrocode}
\input{childdoc.def}
\childdocmain{}
%    \end{macrocode}

% Optional override for |\version| flag:
%    \begin{macrocode}
%%\ifchilddoc\else\providecommand{\version}{draft}\fi
%    \end{macrocode}

% Define the default values for the |\version| flag
% (|final| for the main file and |draft| for childs):
%    \begin{macrocode}
\ifchilddoc
\providecommand{\version}{draft}
\else
\providecommand{\version}{final}
\fi
%    \end{macrocode}

% Load the standard document class:
%    \begin{macrocode}
\documentclass[12pt]{article}
%    \end{macrocode}

% Start the document body:
%    \begin{macrocode}
\begin{document}
%    \end{macrocode}

% Declare a title page.
% Print title, part of document being processed and version flag:
%    \begin{macrocode}
\addtocounter{page}{-1}
\begin{center}
{\LARGE\bfseries{}childdoc example\par}
\vspace{1cm}
\ifchilddoc
\ifchilddocmanual part\else chapter\fi:
`\childdocname' of `\childdocjob'\par
\else
main document: `\childdocjob'\par
\fi
version: \version\par
\end{center}
\newpage
%    \end{macrocode}

% Manually include selected file,
% otherwise process as usual:
%    \begin{macrocode}
\ifchilddocmanual
\section*{part `\childdocname'}
\input{\childdocname}
\else
%    \end{macrocode}

% Include the two chapters:
%    \begin{macrocode}
\include{cdocsch1}
\include{cdocsch2}
%    \end{macrocode}

% Include the two parts unless only chapters should be displayed:
%    \begin{macrocode}
\ifchilddoc\else
\section{part three}
\input{cdocspt3}
\section{part four}
\input{cdocspt4}
\fi
%    \end{macrocode}

% Process as usual until here:
%    \begin{macrocode}
\fi
%    \end{macrocode}

% End of document body:
%    \begin{macrocode}
\end{document}
%    \end{macrocode}
%\iffalse
%</samplemain>
%\fi
%
% %%%%%%%%%%%%%%%%%%%%%%%%%%%%%%%%%%%%%%
% \paragraph{Chapter Include Files.}
%
% The include files are called |cdocsch1.tex| and |cdocsch2.tex|.
%
%\iffalse
%<*samplechap1|samplechap2>
%\fi

% Optional override for |\version| flag:
%    \begin{macrocode}
%%\providecommand{\version}{final}
%    \end{macrocode}

% Include the main document:
%    \begin{macrocode}
\input{childdoc.def}
\childdocof{cdocsamp}
%    \end{macrocode}

%\iffalse
%</samplechap1|samplechap2>
%\fi
%
%\iffalse
%<*samplechap1>
%\fi
% Some text for chapter 1:
%    \begin{macrocode}
\section{one}
some text in chapter one
%    \end{macrocode}

%\iffalse
%</samplechap1>
%\fi
% Some text for chapter 2:
%\iffalse
%<*samplechap2>
%\fi
%    \begin{macrocode}
\section{two}
more text in chapter two
%    \end{macrocode}

%\iffalse
%</samplechap2>
%\fi
%
% %%%%%%%%%%%%%%%%%%%%%%%%%%%%%%%%%%%%%%
% \paragraph{Part Include Files.}
%
% The include files are called |cdocspt3.tex| and |cdocspt4.tex|.
%
%\iffalse
%<*samplepart3|samplepart4>
%\fi

% Optional override for |\version| flag:
%    \begin{macrocode}
%%\providecommand{\version}{final}
%    \end{macrocode}

% Include the main document:
%    \begin{macrocode}
\input{childdoc.def}
\childdocby{cdocsamp}
%    \end{macrocode}

%\iffalse
%</samplepart3|samplepart4>
%\fi
%
%\iffalse
%<*samplepart3>
%\fi
% Some text for part 3:
%    \begin{macrocode}
some text in part three
%    \end{macrocode}

%\iffalse
%</samplepart3>
%\fi
% Some text for part 4:
%\iffalse
%<*samplepart4>
%\fi
%    \begin{macrocode}
more text in part four
%    \end{macrocode}

%\iffalse
%</samplepart4>
%\fi
%
% %%%%%%%%%%%%%%%%%%%%%%%%%%%%%%%%%%%%%%
% \paragraph{Forwarding for a Complete Draft.}
%
% The following forwarding file |cdocsdrf.tex|
% compiles the main document in draft mode:
%\iffalse
%<*sampledraft>
%\fi
%    \begin{macrocode}
\def\version{draft}
\input{childdoc.def}
\childdocforward{cdocsamp}
%    \end{macrocode}

%\iffalse
%</sampledraft>
%\fi
%
% %%%%%%%%%%%%%%%%%%%%%%%%%%%%%%%%%%%%%%
% \paragraph{Forwarding for Final Version of the Chapters.}
%
% The following forwarding files |cdocsfn1.tex| and |cdocsfn2.tex|
% (with identical content)
% compile the final versions of the child documents
% |cdocsch1.tex| and |cdocsch2.tex|, respectively:
%\iffalse
%<*samplefinal>
%\fi
%    \begin{macrocode}
\def\version{final}
\input{childdoc.def}
\childdocforwardprefix[cdocsamp]{cdocsfn}{cdocsch}
%    \end{macrocode}

%\iffalse
%</samplefinal>
%\fi
%
% %%%%%%%%%%%%%%%%%%%%%%%%%%%%%%%%%%%%%%
% \paragraph{Command Line Processing.}
%
% The following three command lines generate the output files
% |cdocscld|, |cdocscl1| and |cdocscl2|
% which should be identical to
% |cdocsdrf|, |cdocsch1| and |cdocsfn2|, respectively:
% \begin{center}
% \begin{tabular}{l}
% |latex -jobname cdocscld \|\\
% |  "\def\version{draft}\input{childdoc.def}\childdocforward{cdocsamp}"|\\
% |latex -jobname cdocscl1 \|\\
% |  "\input{childdoc.def}\childdocforward[cdocsamp]{cdocsch1}"|\\
% |latex -jobname cdocscl2 \|\\
% |  "\def\version{final}\input{childdoc.def}\childdocforward{cdocsch2}"|
% \end{tabular}
% \end{center}
% Note that the trailing backslash on each first line
% merely continues the input to the second line
% (for convenient cut ant paste).
% Furthermore, the command |latex| can be replaced by any
% of its alternative versions such as |pdflatex|.
%
% %%%%%%%%%%%%%%%%%%%%%%%%%%%%%%%%%%%%%%%%%%%%%%%%%%%%%%%%%%%%%%%%%%%%%%%%%%%%%%
% %%%%%%%%%%%%%%%%%%%%%%%%%%%%%%%%%%%%%%%%%%%%%%%%%%%%%%%%%%%%%%%%%%%%%%%%%%%%%%
% \section{Implementation}
%\iffalse
%<*package>
%\fi
%
% This section describes the definitions file |childdoc.def|.

% The definitions cannot be loaded using |\usepackage| or |\RequirePackage|
% which has a mechanism to prevent loading a style file more than once.
% When loading the definitions by means of |\input|
% multiple instances have to be prevented manually:
%\iffalse
%This code needs to be before the `\ProvidesFile' directive
%which is defined at the beginning of this file.
%Therefore it is also placed there and commented out here.
%</package>
%<*discard>
%\fi
%    \begin{macrocode}
\ifdefined\childdocmain\endinput\fi
%    \end{macrocode}
%\iffalse
%</discard>
%<*package>
%\fi
%
% \macro{\ifchilddoc}
% \macro{\ifchilddocmanual}
% The conditional |\ifchilddoc| tells whether a
% child (true) or main (false) document is being compiled.
% The conditional |\ifchilddocmanual| tells whether
% the |\includeonly| mechanism is used (false) or
% the selection of child files must be performed manually (true).
% The definitions initialise to false:
%    \begin{macrocode}
\newif\ifchilddoc
\newif\ifchilddocmanual
%    \end{macrocode}

% \macro{\childdocname}
% \macro{\childdocjob}
% The macro |\childdocname| stores the name of the main document
% to be compiled. The macro |\childdocjob| stores the name of
% the document on which the \LaTeX{} compiler was originally invoked.
% The content of |\jobname| cannot be compared
% to filenames specified in the source due to different catcodes.
% The following code rescans |\jobname|, stores the result
% in |\childdocname| and saves a copy in |\childdocjob|:
%    \begin{macrocode}
\edef\childdocname{\scantokens\expandafter{\jobname\noexpand}}
\let\childdocjob\childdocname
%    \end{macrocode}

% \macro{\childdocdisable}
% The macro |\childdocdisable| prevents the main file
% from being processed more than once.
% At this stage, the main document command |\childdocmain|
% is assumed to be called once again where it should do nothing.
% Any subsequent call to it should prevent
% a secondary processing of the main document
% It overwrites the forwarding commands
% |\childdocof| and |\childdocforward|
% with empty macros to prevent further inclusions of the main document:
%    \begin{macrocode}
\newcommand{\childdocdisable}
{
  \renewcommand{\childdocmain}[1]{\renewcommand{\childdocmain}[1]{\endinput}}
  \renewcommand{\childdocof}[1]{}
  \renewcommand{\childdocby}[2][]{}
  \renewcommand{\childdocforward}[2][]{}
  \renewcommand{\childdocdisable}{}
}
%    \end{macrocode}

% \macro{\childdocmain}
% The macro |\childdocmain| is to be called at the top of the main file
% with nothing or the main filename (without extension) as argument.
% First, it breaks loops.
% If the argument is not empty and does not match |\childdocname|
% (which is set by the first inclusion of |childdoc.def|),
% |\ifchilddoc| is set to true, |\includeonly| is applied to the child file
% and |\jobname| is set to the main file
% (for proper handling of |.aux| files):
%    \begin{macrocode}
\newcommand{\childdocmain}[1]
{
  \childdocdisable\childdocmain{}
  \if?#1?\else
    \begingroup
      \def\childdoctmp{#1}
      \ifx\childdoctmp\childdocname
        \def\childdoctmp{}
      \else
        \def\childdoctmp
        {
          \childdoctrue
          \includeonly{\childdocname}
          \def\childdocjob{#1}
          \def\jobname{#1}
        }
      \fi
      \expandafter
    \endgroup
    \childdoctmp
  \fi
}
%    \end{macrocode}

% \macro{\childdocof}
% The command |\childdocof| redirects
% compilation to the main file |#1|.
%    \begin{macrocode}
\newcommand{\childdocof}[1]
{
  \childdocdisable
  \childdoctrue
  \includeonly{\childdocname}
  \def\jobname{#1}
  \def\childdocjob{#1}
  \input{#1}
}
%    \end{macrocode}

% \macro{\childdocby}
% The command |\childdocby| ....
%    \begin{macrocode}
\newcommand{\childdocby}[2][]
{
  \childdocdisable
  \childdoctrue
  \childdocmanualtrue
  \if?#1?\else
    \def\jobname{#2}
  \fi
  \def\childdocjob{#2}
  \input{#2}
  \endinput
}
%    \end{macrocode}

% \macro{\childdocforward}
% The command |\childdocforward| redirects
% compilation to the main file or
% (if the optional argument is given) a child file.
% Parameters are set as if the main file
% or a child file starting with |\childdocof| was compiled.
% Then compilation is handed over to the main file:
%    \begin{macrocode}
\newcommand{\childdocforward}[2][]
{
  \begingroup
    \if?#1?
      \def\childdoctmp
      {
        \def\childdocname{#2}
        \def\childdocjob{#2}
        \def\jobname{#2}
        \input{#2}
        \endinput
      }
    \else
      \def\childdoctmp
      {
        \childdocdisable
        \def\childdocname{#2}
        \childdoctrue
        \includeonly{#2}
        \def\childdocjob{#1}
        \def\jobname{#1}
        \input{#1}
        \endinput
      }
    \fi
    \expandafter
  \endgroup
  \childdoctmp
}
%    \end{macrocode}

% \macro{\childdocforwardprefix}
% The command |\childdocforwardprefix| redirects
% compilation to the main or a child file by means of a pattern.
% The prefix |#1| in the current filename is replaced by |#2|
% and the suffix of the current filename is kept
% (it is assumed that the filename does not contain the substring `|~~~|'
% which is used as a delimiter).
% Compilation is handed over to the new file by |\childdocforward|:
%    \begin{macrocode}
\newcommand{\childdocforwardprefix}[3][]
{
  \begingroup
    \def\childdocextract #2##1~~~{\def\childdoctmp{\childdocforward[#1]{#3##1}}}
    \expandafter\childdocextract\childdocname~~~
    \expandafter
  \endgroup
  \childdoctmp
}
%    \end{macrocode}

% \macro{\childdoc}
% The deprecated macro |\childdoc| is a legacy version of |\childdocmain|:
%    \begin{macrocode}
\newcommand{\childdoc}{\childdocmain}
%    \end{macrocode}

% \macro{\childdocredirect}
% The deprecated macro |\childdocredirect| is a legacy version
% of |\childdocforward| and |\childdocforwardprefix|:
%    \begin{macrocode}
\newcommand{\childdocredirect}[2][]
{
  \begingroup
    \if?#1?
      \def\childdoctmp{\childdocforward{#2}}
    \else
      \def\childdoctmp{\childdocforwardprefix{#1}{#2}}
    \fi
    \expandafter
  \endgroup
  \childdoctmp
}
%    \end{macrocode}

%\iffalse
%</package>
%\fi
%
\endinput

\childdocof{cdocsamp}
%    \end{macrocode}

%\iffalse
%</samplechap1|samplechap2>
%\fi
%
%\iffalse
%<*samplechap1>
%\fi
% Some text for chapter 1:
%    \begin{macrocode}
\section{one}
some text in chapter one
%    \end{macrocode}

%\iffalse
%</samplechap1>
%\fi
% Some text for chapter 2:
%\iffalse
%<*samplechap2>
%\fi
%    \begin{macrocode}
\section{two}
more text in chapter two
%    \end{macrocode}

%\iffalse
%</samplechap2>
%\fi
%
% %%%%%%%%%%%%%%%%%%%%%%%%%%%%%%%%%%%%%%
% \paragraph{Part Include Files.}
%
% The include files are called |cdocspt3.tex| and |cdocspt4.tex|.
%
%\iffalse
%<*samplepart3|samplepart4>
%\fi

% Optional override for |\version| flag:
%    \begin{macrocode}
%%\providecommand{\version}{final}
%    \end{macrocode}

% Include the main document:
%    \begin{macrocode}
% \iffalse
%
% childdoc.dtx Copyright (C) 2017-2018 Niklas Beisert
%
% This work may be distributed and/or modified under the
% conditions of the LaTeX Project Public License, either version 1.3
% of this license or (at your option) any later version.
% The latest version of this license is in
%   http://www.latex-project.org/lppl.txt
% and version 1.3 or later is part of all distributions of LaTeX
% version 2005/12/01 or later.
%
% This work has the LPPL maintenance status `maintained'.
%
% The Current Maintainer of this work is Niklas Beisert.
%
% This work consists of the files childdoc.dtx and childdoc.ins
% and the derived files childdoc.def and cdocsamp.tex with
% cdocsch1.tex, cdocsch2.tex, cdocsdrf.tex, cdocsfn1.tex, cdocsfn2.tex.
%
%<package>\ifdefined\childdocmain\endinput\fi
%<package>\ProvidesFile{childdoc.def}[2018/12/30 v2.0 child document driver]
%<samplemain>\ProvidesFile{cdocsamp.tex}[2018/12/30 v2.0 sample for childdoc]
%<*driver>
%\ProvidesFile{childdoc.drv}[2018/12/30 v2.0 childdoc reference manual file]
\PassOptionsToClass{10pt,a4paper}{article}
\documentclass{ltxdoc}

\usepackage[margin=35mm]{geometry}
\usepackage{hyperref}
\usepackage{hyperxmp}
\usepackage[usenames]{color}

\hypersetup{colorlinks=true}
\hypersetup{pdfstartview=FitH}
\hypersetup{pdfpagemode=UseNone}
\hypersetup{pdfsource={}}
\hypersetup{pdflang={en-UK}}
\hypersetup{pdfcopyright={Copyright 2017-2018 Niklas Beisert.
  This work may be distributed and/or modified under the
  conditions of the LaTeX Project Public License, either version 1.3
  of this license or (at your option) any later version.}}
\hypersetup{pdflicenseurl={http://www.latex-project.org/lppl.txt}}
\hypersetup{pdfcontactaddress={ETH Zurich, ITP, HIT K,
  Wolfgang-Pauli-Strasse 27}}
\hypersetup{pdfcontactpostcode={8093}}
\hypersetup{pdfcontactcity={Zurich}}
\hypersetup{pdfcontactcountry={Switzerland}}
\hypersetup{pdfcontactemail={nbeisert@itp.phys.ethz.ch}}
\hypersetup{pdfcontacturl={http://people.phys.ethz.ch/\xmptilde nbeisert/}}

\newcommand{\secref}[1]{\hyperref[#1]{section \ref*{#1}}}

\parskip1ex
\parindent0pt
\let\olditemize\itemize
\def\itemize{\olditemize\parskip0pt}

\begin{document}

\title{The \textsf{childdoc} Package}
\hypersetup{pdftitle={The childdoc Package}}
\author{Niklas Beisert\\[2ex]
  Institut f\"ur Theoretische Physik\\
  Eidgen\"ossische Technische Hochschule Z\"urich\\
  Wolfgang-Pauli-Strasse 27, 8093 Z\"urich, Switzerland\\[1ex]
  \href{mailto:nbeisert@itp.phys.ethz.ch}
  {\texttt{nbeisert@itp.phys.ethz.ch}}}
\hypersetup{pdfauthor={Niklas Beisert}}
\hypersetup{pdfsubject={Manual for the LaTeX2e Package childdoc}}
\date{30 December 2018, \textsf{v2.0}}
\maketitle

\begin{abstract}\noindent
\textsf{childdoc} is a \LaTeXe{} package
that enables the direct compilation
of document sections included by |\include|
to individual files.
\end{abstract}

\begingroup
\parskip0ex
\tableofcontents
\endgroup

%%%%%%%%%%%%%%%%%%%%%%%%%%%%%%%%%%%%%%%%%%%%%%%%%%%%%%%%%%%%%%%%%%%%%%%%%%%%%%%%
%%%%%%%%%%%%%%%%%%%%%%%%%%%%%%%%%%%%%%%%%%%%%%%%%%%%%%%%%%%%%%%%%%%%%%%%%%%%%%%%
\section{Introduction}

\LaTeX{} provides a mechanism to structure a large document (such as a book)
into a main file and several child files (containing the chapters)
using the |\include| command.
This mechanism is beneficial for documents
which span hundreds of pages in order to
make the source file(s) more manageable.
Moreover, compilation can be restricted to
selected child files by means of the |\includeonly| command.
The latter feature can be used to reduce the compilation time while editing
(this was significantly more useful in the earlier days of \LaTeX{})
or to generate a smaller document which is easier to navigate.
Another application of |\includeonly| is to generate
documents consisting of selected parts of the complete document.

However, there are a few drawbacks of the plain |\include| mechanism:
\begin{itemize}
\item
The child files cannot be compiled on their own,
they can only be compiled via the main file.
A naive editing environment
(such as a text editor with an option
to have the current file processed by \LaTeX)
may require one to switch to the main file before compiling;
attempting to compile the child file produces errors.
\item
The main file must be modified (each time)
to adjust the |\includeonly| command
to the present needs. This easily leaves the main file in a messy state.
\item
The generated document will always carry the filename
of the main document. This is inconvenient if
several child files are to be compiled and
to be kept for distribution.
\end{itemize}

The present package provides a simple interface
to make child files individually compilable by \LaTeX{}.
Compiling a child file then has the same effect as compiling
the main file with an |\includeonly| command
to select the appropriate child.
Moreover the generated document will carry the name of the child
rather than the main file.
This resolves all three above issues.

This feature is meant to make the editing of books,
thesis documents and lecture notes somewhat more convenient.
However, the package can also be used efficiently for
composing a series of documents (such as exercise sheets)
which are typically distributed individually.
It then assists the author in generating the individual documents
(potentially in different versions)
as well as a document containing the collected series.
Another application is in developing style files
or other kinds of included material
where compilation of the style file could redirect
to a sample or test file.

%%%%%%%%%%%%%%%%%%%%%%%%%%%%%%%%%%%%%%%%%%%%%%%%%%%%%%%%%%%%%%%%%%%%%%%%%%%%%%%%
%%%%%%%%%%%%%%%%%%%%%%%%%%%%%%%%%%%%%%%%%%%%%%%%%%%%%%%%%%%%%%%%%%%%%%%%%%%%%%%%
\section{Usage}

First of all, the package \textsf{childdoc} is \emph{not} a standard
\LaTeXe{} |.sty| style file! Therefore it needs to be invoked in
a non-standard way.

%%%%%%%%%%%%%%%%%%%%%%%%%%%%%%%%%%%%%%%%%%%%%%%%%%%%%%%%%%%%%%%%%%%%%%%%%%%%%%%%
\subsection{Included Files}
\label{sec:include}

%%%%%%%%%%%%%%%%%%%%%%%%%%%%%%%%%%%%%%%%
\DescribeMacro{\childdocmain}
To use the package, add the commands
\begin{center}
\begin{tabular}{l}
|\input{childdoc.def}|\\
|\childdocmain{}|\\
\end{tabular}
\end{center}
at the very top of the main \LaTeX{} file,
in particular \emph{before} the |\documentclass| statement!
The argument of |\childdocmain| should be left empty
(but it must be present).

%%%%%%%%%%%%%%%%%%%%%%%%%%%%%%%%%%%%%%%%
\DescribeMacro{\childdocof}
Furthermore, add the commands
\begin{center}
\begin{tabular}{l}
|\input{childdoc.def}|\\
|\childdocof{|\textit{main}|}|\\
\end{tabular}
\end{center}
at the top of every child file \textit{child}
which is included by |\include{|\textit{child}|}|
from within the main file
(or at least for those files to be compiled individually).
The argument \textit{main} must be the filename of the main file.

There are a couple of
considerations in setting up the main and child documents:

%%%%%%%%%%%%%%%%%%%%%%%%%%%%%%%%%%%%%%%%
\paragraph{Restrictions.}

Please note the following restrictions:
\begin{itemize}
\item
|\childdocmain| must be called with one argument \textit{main}
to ensure compatibility with earlier version of the package.
It must either be empty (|\childdocmain{}|)
or precisely match the filename of the main file in which it is specified.
See \secref{sec:detection} for further information.
\item
The filename \textit{main} must be specified without the |.tex| extension.
\item
The filename \textit{main} is case sensitive
(even in case-insensitive file systems)
due to internal string comparison.
\item
The argument \textit{main} should be fully expanded, it cannot be a macro.
\item
Subdirectories and special characters should be avoided in filenames.
\item
The command |\childdocmain{|\textit{main}|}| must be followed by a whitespace.
It should not be followed immediately by another command
or by a comment mark `|%|'.
This is because the \TeX{} parser reads the token immediately following
the argument of |\childdocmain| and puts it
at the beginning of every child section;
however, a white\-space is ignored.
\end{itemize}

%%%%%%%%%%%%%%%%%%%%%%%%%%%%%%%%%%%%%%%%
\paragraph{Content of Main File.}

It is advisable to place all content in the child files included by |\include|.
Any output contained in the main file will appear in all child documents
unless suppressed manually;
it cannot be suppressed automatically by the |\includeonly| directive
and thus should normally be avoided.
A method to include some content in the main file
by means of conditional processing is described in \secref{sec:conditional}.

%%%%%%%%%%%%%%%%%%%%%%%%%%%%%%%%%%%%%%%%
\paragraph{Page Numbering.}

When only a part of the document is compiled,
the appropriate numbering of pages
(as well as other status parameters)
is determined from the |.aux| files.
The latter contain information from previous passes.
However this information needs to propagate through
all intermediate child documents.
Therefore the page numbering in child documents may well
be inconsistent until the complete document is compiled at least once.

A useful (if unconventional) way to always ensure a consistent
page numbering is to restart the numbering in each child document
and denote the pages by `\textit{child}|.|\textit{page}'
where \textit{child} represents the chapter/section number of the child file.
This can be achieved by the command
|\numberwithin{page}{|\textit{child}|}|
of the \textsf{amsmath} package
where \textit{child} can be |chapter| or |section|
depending on the chosen structuring.
Alternatively, one can modify the macro |\thepage| appropriately
and reset the counter |page| at the start of each child file.

%%%%%%%%%%%%%%%%%%%%%%%%%%%%%%%%%%%%%%%%%%%%%%%%%%%%%%%%%%%%%%%%%%%%%%%%%%%%%%%%
\subsection{Conditional Processing}
\label{sec:conditional}

The package provides a mechanism to compile different versions
of a document. To customise the versions further some conditional processing
can come in handy to distinguish which version is being compiled.
The package provides two macros to describe the compilation context:

%%%%%%%%%%%%%%%%%%%%%%%%%%%%%%%%%%%%%%%%
\DescribeMacro{\ifchilddoc}
The conditional |\ifchilddoc| distinguishes between the compilation of
child documents and the main document:
%
\begin{center}
|\ifchilddoc |\textit{child-code}| |[|\||else |\textit{main-code}]| \||fi|
\end{center}

%%%%%%%%%%%%%%%%%%%%%%%%%%%%%%%%%%%%%%%%
\DescribeMacro{\childdocname}
\DescribeMacro{\childdocjob}
The macro |\childdocname| contains the filename (without extension)
of the main or child file being processed.
Note that |\childdocjob| will always contain the name of the main file.

%%%%%%%%%%%%%%%%%%%%%%%%%%%%%%%%%%%%%%%%
\paragraph{Title Page.}

Conditional processing can be used to include a title or banner page
in the main document when proper precautions are taken.
Importantly, the code in the main file should ensure that the page counter
(as well as other status parameters which are stored in the |.aux| files)
takes the same value after the conditional processing.
Otherwise the page numbers may take divergent values
depending on which part is compiled.

For example, a title page could be declared by:
%
\begin{center}
\begin{tabular}{l}
|\ifchilddoc\||else|\\
|\addtocounter{page}{-1}|\\
\textit{code for title page}\\
|\newpage|\\
|\||fi|
\end{tabular}
\end{center}
%
A banner page for the child documents can be generated by:
%
\begin{center}
\begin{tabular}{l}
|\ifchilddoc|\\
|\addtocounter{page}{-1}|\\
\textit{code for banner page}\\
|\newpage|\\
|\||fi|
\end{tabular}
\end{center}
%
Here one could write a message such as:
\begin{center}
|This is the part \childdocname{} of \childdocjob{}.|
\end{center}

%%%%%%%%%%%%%%%%%%%%%%%%%%%%%%%%%%%%%%%%%%%%%%%%%%%%%%%%%%%%%%%%%%%%%%%%%%%%%%%%
\subsection{Flags}
\label{sec:flags}

The package makes it easy to generate different versions
of the main or child documents.
To this end compilation flags can be defined
and assigned different default values.
They will be particularly useful in conjunction
with the forwarding mechanism described in \secref{sec:forward}.

For example, it may be useful to have a flag |\version|
which can be set to |draft| or |final|.
The document source will contain some conditional code
depending on the value of |\version|.
Suppose further, the flag should default to |final| for the main file
and to |draft| for child files
which is a natural assignment for editing the document.
This is achieved by placing the following code
in the preamble of the main document
(below the |\childdocmain| directive):
%
\begin{center}
\begin{tabular}{l}
|\ifchilddoc|\\
|\providecommand{\version}{draft}|\\
|\||else|\\
|\providecommand{\version}{final}|\\
|\||fi|
\end{tabular}
\end{center}
%
The definition by |\providecommand| makes sure
that previous definitions are not overwritten.
Further statements |\providecommand{\version}{...}|
can thus be added before the above code to override it.

For the main file, one might add a line
(between |\childdocmain| and the above block)
%
\begin{center}
|%\ifchilddoc\||else\providecommand{\version}{draft}\||fi|
\end{center}
%
which can be uncommented to produce a draft version.
Likewise one can add a line to the very top of a child file
(above the |\childdocof{|\textit{main}|}| directive)
%
\begin{center}
|%\providecommand{\version}{final}|
\end{center}
%
which can be uncommented to produce the final version of this child document.

%%%%%%%%%%%%%%%%%%%%%%%%%%%%%%%%%%%%%%%%%%%%%%%%%%%%%%%%%%%%%%%%%%%%%%%%%%%%%%%%
\subsection{Forwarding}
\label{sec:forward}

Different versions of the main or child documents
using compilation flags as described in \secref{sec:flags}
can be (permanently) stored in different files
for convenient compilation, viewing and distribution.
To this end, the package defines a command
to pass on compilation to a different file:

%%%%%%%%%%%%%%%%%%%%%%%%%%%%%%%%%%%%%%%%
\DescribeMacro{\childdocforward}
The command |\childdocforward| redirects processing to
another source file:
%
\begin{center}
\begin{tabular}{l}
|\input{childdoc.def}|\\
|\childdocforward[|\textit{main}|]{|\textit{dest}|}|\\
\end{tabular}
\end{center}
%
The argument \textit{dest} is the destination file
(without extension).
It should be the main file or one of the child files.
Note that further \textsf{childdoc} directives
such as |\childdocof| and |\childdocforward|
in the indicated file will be processed in this form.
The optional argument \textit{main}
passes on directly to the main file \textit{main}
while pretending to compile the child \textit{dest}.
This form behaves as if \textit{dest}
issues |\childdocof{|\textit{main}|}| right away,
and no further \textsf{childdoc} directives will be processed.

%%%%%%%%%%%%%%%%%%%%%%%%%%%%%%%%%%%%%%%%
\DescribeMacro{\...prefix}
In the alternative form |\childdocforwardprefix|,
%
\begin{center}
\begin{tabular}{l}
|\input{childdoc.def}|\\
|\childdocforwardprefix[|\textit{main}|]{|\textit{prefix}|}{|\textit{dest}|}|
\end{tabular}
\end{center}
%
the destination file is determined by a pattern
depending on the current file:
To make this work, the current file must be called
`{\textit{prefix}\hspace{0.2em}\textit{suffix}}'
with \textit{prefix} matching precisely the argument.
Processing is then passed on to the file
`{\textit{dest}\hspace{0.2em}\textit{suffix}}'.
Surely, the same effect is achieved by
directly specifying the
argument `{\textit{dest}\hspace{0.2em}\textit{suffix}}'
in the first form.
However, that requires to set up a different file
for each child. With the alternative form of the command
all these files can have exactly the same content
which simplifies setting them up and maintaining them.

For example, the following file |draft.tex|
with a compilation flag |\version| as described in \secref{sec:flags}
compiles the main document as a draft:
%
\begin{center}
\begin{tabular}{l}
|\def\version{draft}|\\
|\input{childdoc.def}|\\
|\childdocforward{|\textit{main}|}|
\end{tabular}
\end{center}
%
Likewise, the following files |final|\textit{nn}|.tex|
compile the final version of the child document
|child|\textit{nn}|.tex|:
%
\begin{center}
\begin{tabular}{l}
|\def\version{final}|\\
|\input{childdoc.def}|\\
|\childdocforwardprefix{final}{child}|
\end{tabular}
\end{center}
%

Note that when several versions of a main file and/or of each child file
are to be generated, it may be convenient to set up a |Makefile| or
shell script to automatise the process.

%%%%%%%%%%%%%%%%%%%%%%%%%%%%%%%%%%%%%%%%%%%%%%%%%%%%%%%%%%%%%%%%%%%%%%%%%%%%%%%%
\subsection{Command Line Processing}
\label{sec:commandline}

The effect of redirection files can also be achieved by invoking
the \LaTeX{} compiler with a more elaborate command line.
Most conveniently this should be done as part
of a shell script or a |Makefile|.

When using \textsf{childdoc} in the main file, the following
command lines effectively perform a redirection
(note that depending on the shell being used,
backslashes may have to be doubled: `|\|' $\to$ `|\\|'):
%
\begin{center}
|... -jobname "|\textit{target}|" |\\|"|[\textit{flags}]%
|\input{childdoc.def}\childdocforward[|\textit{main}|]{|\textit{dest}|}"|
\end{center}
%
Here \textit{target} is the name of the output file,
\textit{main} is the name of the main file
and \textit{dest} is the name of the main or child file to be processed
(all filenames without extensions).
The optional argument \textit{main} can be omitted
if \textit{main} matches \textit{dest}.
Optionally, compilation \textit{flags} can be defined via |\def| commands.
This command line makes the \TeX{} engine believe
it is compiling the file \textit{target}
whose content is specified as the latter parameter.
The provided code then forwards the processing to
\textit{main} or \textit{dest} as described in \secref{sec:forward}.

%%%%%%%%%%%%%%%%%%%%%%%%%%%%%%%%%%%%%%%%%%%%%%%%%%%%%%%%%%%%%%%%%%%%%%%%%%%%%%%%
\subsection{Include by Input}
\label{sec:input}

Including child documents by |\include| has some restrictions by design.
Most notably, the content of a child document always occupies
its own set of pages; pages cannot be shared between child documents.
Usually, this behaviour makes perfect sense
because each child document contain an essential part of the document.
However, in some situations it may be desirable to compose
a document from a collection of parts
without having mandatory page breaks between then.
For this case, the package
provides a mechanism to include parts
by |\input| which can also be processed individually.
However, by construction this mechanism
requires manual handling of the content to be output.

%%%%%%%%%%%%%%%%%%%%%%%%%%%%%%%%%%%%%%%%
\DescribeMacro{\ifchilddocmanual}
The main file should be prepared as usual, see \secref{sec:include}.
However, the document body must make a distinction
between processing of an individual part and of the main document, e.g.:
%
\begin{center}
\begin{tabular}{l}
|\ifchilddocmanual|\\
|\input{\childdocname}|\\
|\||else|\\
\textit{document body with }|\input{|\textit{part}|}|\\
|\||fi|
\end{tabular}
\end{center}
%
The conditional |\ifchilddocmanual| is true whenever
a part to be included by |\input| is being compiled,
and the name of the part is stored in |\childdocname|.

%%%%%%%%%%%%%%%%%%%%%%%%%%%%%%%%%%%%%%%%
\DescribeMacro{\childdocby}
Each part to be included by |\input| should start with:
%
\begin{center}
\begin{tabular}{l}
|\input{childdoc.def}|\\
|\childdocby{|\textit{main}|}|\\
\end{tabular}
\end{center}
%
The directive |\childdocby| is similar to |\childdocof|
described in \secref{sec:include},
but the subsequent selection of content must be done manually.
To that end, both |\ifchilddoc| and |\ifchilddocmanual|
will be true upon processing of a part,
and the name of the part is stored in |\childdocname|.
Note that |\jobname| will be set to the filename of the current part
so that each part receives an individual |.aux| file
that does not interfere with the |.aux| file(s) of the main document.
This behaviour can be altered by the alternative form
|\childdocby[*]{|\textit{main}|}| (with a non-empty optional argument)
which uses the |.aux| file of the main document
by setting |\jobname| to \textit{main}.

%%%%%%%%%%%%%%%%%%%%%%%%%%%%%%%%%%%%%%%%%%%%%%%%%%%%%%%%%%%%%%%%%%%%%%%%%%%%%%%%
\subsection{Driver Development}
\label{sec:driver}

The \textsf{childdoc} mechanism can also be use for the development
of definition files such as \LaTeX{} styles or classes.
This case differs from the above setup with multiple parts
included by |\include| in that no |\includeonly| should be invoked.
This can be achieved by starting the include file
(before |\ProvidesPackage|) with:
%
\begin{center}
\begin{tabular}{l}
|\input{childdoc.def}|\\
|\childdocforward{|\textit{main}|}|\\
\end{tabular}
\end{center}
%
or alternatively with:
%
\begin{center}
\begin{tabular}{l}
|\input{childdoc.def}|\\
|\childdocby{|\textit{main}|}|\\
\end{tabular}
\end{center}
%
Both forms have slightly different effects as described above.
The main file is prepared as usual, see \secref{sec:include}.

%%%%%%%%%%%%%%%%%%%%%%%%%%%%%%%%%%%%%%%%%%%%%%%%%%%%%%%%%%%%%%%%%%%%%%%%%%%%%%%%
\subsection{Legacy Detection}
\label{sec:detection}

The directive |\childdocmain| in the main file can detect
whether the complete document or merely a child is to be compiled
even without using the directive |\childdocof|.
This method is deprecated because it is less robust
and there is no compelling reason to use it;
it is merely provided for backward compatibility
and it may be removed in future versions.

If the detection mechanism is to be used,
it is mandatory to correctly specify
the filename of the main file as the argument of |\childdocmain|:
%
\begin{center}
\begin{tabular}{l}
|\input{childdoc.def}|\\
|\childdocmain{|\textit{main}|}|\\
\end{tabular}
\end{center}
%
If |\jobname| does not match the argument \textit{main} of |\childdocmain|,
it is assumed that |\jobname| points to the child file to be compiled.
When using |\childdocmain| with the main file specified as argument,
it suffices to start a child file
with just |\input{|\textit{main}|}|
without loading of the package and using |\childdocof|.
If instead all processing is done
with the appropriate \textsf{childdoc} directives,
the argument of \textit{main} of |\childdocmain| can be empty.

An alternative version of the command line processing described
in \secref{sec:commandline} using the detection mechanism reads:
%
\begin{center}
|... -jobname "|\textit{target}|" "|[\textit{flags}]%
[|\def\jobname{|\textit{dest}|}|]|\input{|\textit{main}|}"|
\end{center}

%%%%%%%%%%%%%%%%%%%%%%%%%%%%%%%%%%%%%%%%%%%%%%%%%%%%%%%%%%%%%%%%%%%%%%%%%%%%%%%%
\subsection{Manual Code}
\label{sec:manual}

In case one cannot be certain whether the definitions file |childdoc.def|
is installed on the target \TeX{} distribution
and one prefers not to ship it,
it is conceivable to paste a few relevant commands into the sources.

To that end, drop all statements |\input{childdoc.def}|
and perform the replacements as outlined below.
Instead of |\childdocmain{|\textit{main}|}| add the following code
to the top of the main file:
%
\begin{center}
\begin{tabular}{l}
|\||ifdefined\childdocname\endinput\||fi\newif\ifchilddoc|\\
|\edef\childdocname{\scantokens\expandafter{\jobname\noexpand}}|\\
|\def\childdocmain{|\textit{main}|}\||ifx\childdocmain\childdocname\||else|\\
|\childdoctrue\includeonly{\childdocname}\let\jobname\childdocmain\||fi|\\
\end{tabular}
\end{center}
%
Instead of |\childdocof{|\textit{main}|}| just include the main file
at the top of each child file:
%
\begin{center}
|\input{|\textit{main}|}|
\end{center}
%
A simple redirection |\childdocforward{|\textit{dest}|}| is achieved by:
%
\begin{center}
|\def\jobname{|\textit{dest}|}\input{\jobname}|
\end{center}
%
The redirection with prefix
|\childdocforwardprefix[|\textit{prefix}|]{|\textit{dest}|}|
is accomplished by:
%
\begin{center}
\begin{tabular}{l}
|{\edef\jobname{\scantokens\expandafter{\jobname\noexpand}}|\\
|\def\redirectjob |\textit{prefix}|#1~~~{\gdef\jobname{|\textit{dest}|#1}}|\\
|\expandafter\redirectjob\jobname~~~}\input{\jobname}|
\end{tabular}
\end{center}

In an alternative approach,
child documents can be compiled by a specific command line
without additional code or specific definitions:
%
\begin{center}
|... -jobname "|\textit{target}|" "|[\textit{flags}]%
|\includeonly{|\textit{dest}|}\input{|\textit{main}|}"|
\end{center}
%

%%%%%%%%%%%%%%%%%%%%%%%%%%%%%%%%%%%%%%%%%%%%%%%%%%%%%%%%%%%%%%%%%%%%%%%%%%%%%%%%
%%%%%%%%%%%%%%%%%%%%%%%%%%%%%%%%%%%%%%%%%%%%%%%%%%%%%%%%%%%%%%%%%%%%%%%%%%%%%%%%
\section{Information}

%%%%%%%%%%%%%%%%%%%%%%%%%%%%%%%%%%%%%%%%%%%%%%%%%%%%%%%%%%%%%%%%%%%%%%%%%%%%%%%%
\subsection{Copyright}

Copyright \copyright{} 2017--2018 Niklas Beisert

This work may be distributed and/or modified under the
conditions of the \LaTeX{} Project Public License, either version 1.3
of this license or (at your option) any later version.
The latest version of this license is in
  \url{http://www.latex-project.org/lppl.txt}
and version 1.3 or later is part of all distributions of \LaTeX{}
version 2005/12/01 or later.

This work has the LPPL maintenance status `maintained'.

The Current Maintainer of this work is Niklas Beisert.

This work consists of the files |README.txt|, |childdoc.ins| and |childdoc.dtx|
as well as the derived files |childdoc.def|, |cdocsamp.tex|
with |cdocsch1.tex|, |cdocsch2.tex|, |cdocspt3.tex|, |cdocspt4.tex|,
|cdocsdrf.tex|, |cdocsfn1.tex|, |cdocsfn2.tex|
as well as |childdoc.pdf|.

%%%%%%%%%%%%%%%%%%%%%%%%%%%%%%%%%%%%%%%%%%%%%%%%%%%%%%%%%%%%%%%%%%%%%%%%%%%%%%%%
\subsection{Files and Installation}

The package consists of the files:
%
\begin{center}
\begin{tabular}{ll}
    |README.txt|   & readme file \\
    |childdoc.ins| & installation file \\
    |childdoc.dtx| & source file \\
    |childdoc.def| & definition file \\
    |cdocsamp.tex| & sample main file \\
    |cdocsch1.tex| & sample include file \\
    |cdocsch2.tex| & sample include file \\
    |cdocspt3.tex| & sample part file \\
    |cdocspt4.tex| & sample part file \\
    |cdocsdrf.tex| & sample redirection file \\
    |cdocsfn1.tex| & sample redirection file \\
    |cdocsfn2.tex| & sample redirection file \\
    |childdoc.pdf| & manual
\end{tabular}
\end{center}
%
The distribution consists of the files
|README.txt|, |childdoc.ins| and |childdoc.dtx|.
%
\begin{itemize}
\item
Run (pdf)\LaTeX{} on |childdoc.dtx|
to compile the manual |childdoc.pdf| (this file).
\item
Run \LaTeX{} on |childdoc.ins| to create the definitions file |childdoc.def|
and the sample |cdocsamp.tex| with include files
|cdocsch1.tex|, |cdocsch2.tex|, |cdocspt3.tex|, |cdocspt4.tex|,
|cdocsdrf.tex|, |cdocsfn1.tex|, |cdocsfn2.tex|.
Then copy the file |childdoc.def| to an appropriate directory of your \LaTeX{}
distribution, e.g.\ \textit{texmf-root}|/tex/latex/childdoc|.
\end{itemize}

%%%%%%%%%%%%%%%%%%%%%%%%%%%%%%%%%%%%%%%%%%%%%%%%%%%%%%%%%%%%%%%%%%%%%%%%%%%%%%%%
\subsection{Related CTAN Packages}

There are several other packages which offer a similar functionality:
%
\begin{itemize}
\item
The packages
\href{http://ctan.org/pkg/docmute}{\textsf{docmute}},
\href{http://ctan.org/pkg/includex}{\textsf{includex}} and
\href{http://ctan.org/pkg/standalone}{\textsf{standalone}}
provide commands to include only the document body of
a child file thus allowing both files to be compiled individually.
\item
The packages \href{http://ctan.org/pkg/subdocs}{\textsf{subdocs}}
and \href{http://ctan.org/pkg/subfiles}{\textsf{subfiles}}
provide structures in which the main and child documents can be
encapsulated and allowing them to be compiled individually.
The inclusion mechanism is different from the conventional |\include|.
\item
The package \href{http://ctan.org/pkg/combine}{\textsf{combine}}
is an elaborate solution to combine several documents into one.
\end{itemize}
%
See also the CTAN topic \href{http://ctan.org/topic/subdocs}{\textsf{subdocs}}
for further related packages.
The present package differs from the above solutions in that
a document structure constructed with the conventional |\include| mechanism
just needs two extra commands at the top of every file
such that all constituent files can be compiled individually.

%%%%%%%%%%%%%%%%%%%%%%%%%%%%%%%%%%%%%%%%%%%%%%%%%%%%%%%%%%%%%%%%%%%%%%%%%%%%%%%%
%\subsection{Feature Suggestions}
%
%The following is a list of features which may be useful for future
%versions of this package:
%%
%\begin{itemize}
%\item
%\ldots
%\end{itemize}

%%%%%%%%%%%%%%%%%%%%%%%%%%%%%%%%%%%%%%%%%%%%%%%%%%%%%%%%%%%%%%%%%%%%%%%%%%%%%%%%
\subsection{Revision History}

%%%%%%%%%%%%%%%%%%%%%%%%%%%%%%%%%%%%%%%%
\paragraph{v2.0:} 2018/12/30

\begin{itemize}
\item
immediate forward processing
\item
added |\childdocby| mechanism
\item
manual restructured
\end{itemize}

%%%%%%%%%%%%%%%%%%%%%%%%%%%%%%%%%%%%%%%%
\paragraph{v1.6:} 2018/01/17

\begin{itemize}
\item
application for development of include files
\item
corrections to manual
\end{itemize}

%%%%%%%%%%%%%%%%%%%%%%%%%%%%%%%%%%%%%%%%
\paragraph{v1.5:} 2017/05/21

\begin{itemize}
\item
more complete structuring introduced
\item
|\childdocof| introduced
\item
|\childdoc| renamed to |\childdocmain|
\item
|\childredirect| renamed to |\childdocforward| and |\childdocforwardprefix|
and functionality expanded
\end{itemize}

%%%%%%%%%%%%%%%%%%%%%%%%%%%%%%%%%%%%%%%%
\paragraph{v1.0:} 2017/04/27

\begin{itemize}
\item
manual and install package
\item
first version published on CTAN
\end{itemize}

%%%%%%%%%%%%%%%%%%%%%%%%%%%%%%%%%%%%%%%%
\paragraph{v0.6:} 2017/04/26

\begin{itemize}
\item
redirection mechanism added
\end{itemize}

%%%%%%%%%%%%%%%%%%%%%%%%%%%%%%%%%%%%%%%%
\paragraph{v0.5:} 2017/04/26

\begin{itemize}
\item
functionality in definition file
\end{itemize}


%%%%%%%%%%%%%%%%%%%%%%%%%%%%%%%%%%%%%%%%%%%%%%%%%%%%%%%%%%%%%%%%%%%%%%%%%%%%%%%%
%%%%%%%%%%%%%%%%%%%%%%%%%%%%%%%%%%%%%%%%%%%%%%%%%%%%%%%%%%%%%%%%%%%%%%%%%%%%%%%%
%%%%%%%%%%%%%%%%%%%%%%%%%%%%%%%%%%%%%%%%%%%%%%%%%%%%%%%%%%%%%%%%%%%%%%%%%%%%%%%%
\appendix

\settowidth\MacroIndent{\rmfamily\scriptsize 000\ }

 \DocInput{childdoc.dtx}

\end{document}
%</driver>
% \fi
%
% %%%%%%%%%%%%%%%%%%%%%%%%%%%%%%%%%%%%%%%%%%%%%%%%%%%%%%%%%%%%%%%%%%%%%%%%%%%%%%
% %%%%%%%%%%%%%%%%%%%%%%%%%%%%%%%%%%%%%%%%%%%%%%%%%%%%%%%%%%%%%%%%%%%%%%%%%%%%%%
% \section{Sample}
%\iffalse
%<*samplemain>
%\fi
%
% The following presents a sample document
% with two chapters, two parts, a title page,
% a compile flag as well as three forwarding files to set the flag.
% It consists of eight |.tex| files:
% \begin{center}
% \begin{tabular}{ll}
% |cdocsamp.tex|&main file\\
% |cdocsch1.tex|&include file for chapter 1\\
% |cdocsch2.tex|&include file for chapter 2\\
% |cdocspt3.tex|&include file for part 3\\
% |cdocspt4.tex|&include file for part 4\\
% |cdocsdrf.tex|&forwarding file for main file in draft mode\\
% |cdocsfi1.tex|&forwarding file for final version of chapter 1\\
% |cdocsfi2.tex|&forwarding file for final version of chapter 2\\
% \end{tabular}
% \end{center}
% Each of the eight files can be compiled directly by the \LaTeX{} compiler.
%
% %%%%%%%%%%%%%%%%%%%%%%%%%%%%%%%%%%%%%%
% \paragraph{Main File.}
%
% The main file is called |cdocsamp.tex|.
%
% Load the \textsf{childdoc} definitions and
% declare the filename for the main document:
%    \begin{macrocode}
\input{childdoc.def}
\childdocmain{}
%    \end{macrocode}

% Optional override for |\version| flag:
%    \begin{macrocode}
%%\ifchilddoc\else\providecommand{\version}{draft}\fi
%    \end{macrocode}

% Define the default values for the |\version| flag
% (|final| for the main file and |draft| for childs):
%    \begin{macrocode}
\ifchilddoc
\providecommand{\version}{draft}
\else
\providecommand{\version}{final}
\fi
%    \end{macrocode}

% Load the standard document class:
%    \begin{macrocode}
\documentclass[12pt]{article}
%    \end{macrocode}

% Start the document body:
%    \begin{macrocode}
\begin{document}
%    \end{macrocode}

% Declare a title page.
% Print title, part of document being processed and version flag:
%    \begin{macrocode}
\addtocounter{page}{-1}
\begin{center}
{\LARGE\bfseries{}childdoc example\par}
\vspace{1cm}
\ifchilddoc
\ifchilddocmanual part\else chapter\fi:
`\childdocname' of `\childdocjob'\par
\else
main document: `\childdocjob'\par
\fi
version: \version\par
\end{center}
\newpage
%    \end{macrocode}

% Manually include selected file,
% otherwise process as usual:
%    \begin{macrocode}
\ifchilddocmanual
\section*{part `\childdocname'}
\input{\childdocname}
\else
%    \end{macrocode}

% Include the two chapters:
%    \begin{macrocode}
\include{cdocsch1}
\include{cdocsch2}
%    \end{macrocode}

% Include the two parts unless only chapters should be displayed:
%    \begin{macrocode}
\ifchilddoc\else
\section{part three}
\input{cdocspt3}
\section{part four}
\input{cdocspt4}
\fi
%    \end{macrocode}

% Process as usual until here:
%    \begin{macrocode}
\fi
%    \end{macrocode}

% End of document body:
%    \begin{macrocode}
\end{document}
%    \end{macrocode}
%\iffalse
%</samplemain>
%\fi
%
% %%%%%%%%%%%%%%%%%%%%%%%%%%%%%%%%%%%%%%
% \paragraph{Chapter Include Files.}
%
% The include files are called |cdocsch1.tex| and |cdocsch2.tex|.
%
%\iffalse
%<*samplechap1|samplechap2>
%\fi

% Optional override for |\version| flag:
%    \begin{macrocode}
%%\providecommand{\version}{final}
%    \end{macrocode}

% Include the main document:
%    \begin{macrocode}
\input{childdoc.def}
\childdocof{cdocsamp}
%    \end{macrocode}

%\iffalse
%</samplechap1|samplechap2>
%\fi
%
%\iffalse
%<*samplechap1>
%\fi
% Some text for chapter 1:
%    \begin{macrocode}
\section{one}
some text in chapter one
%    \end{macrocode}

%\iffalse
%</samplechap1>
%\fi
% Some text for chapter 2:
%\iffalse
%<*samplechap2>
%\fi
%    \begin{macrocode}
\section{two}
more text in chapter two
%    \end{macrocode}

%\iffalse
%</samplechap2>
%\fi
%
% %%%%%%%%%%%%%%%%%%%%%%%%%%%%%%%%%%%%%%
% \paragraph{Part Include Files.}
%
% The include files are called |cdocspt3.tex| and |cdocspt4.tex|.
%
%\iffalse
%<*samplepart3|samplepart4>
%\fi

% Optional override for |\version| flag:
%    \begin{macrocode}
%%\providecommand{\version}{final}
%    \end{macrocode}

% Include the main document:
%    \begin{macrocode}
\input{childdoc.def}
\childdocby{cdocsamp}
%    \end{macrocode}

%\iffalse
%</samplepart3|samplepart4>
%\fi
%
%\iffalse
%<*samplepart3>
%\fi
% Some text for part 3:
%    \begin{macrocode}
some text in part three
%    \end{macrocode}

%\iffalse
%</samplepart3>
%\fi
% Some text for part 4:
%\iffalse
%<*samplepart4>
%\fi
%    \begin{macrocode}
more text in part four
%    \end{macrocode}

%\iffalse
%</samplepart4>
%\fi
%
% %%%%%%%%%%%%%%%%%%%%%%%%%%%%%%%%%%%%%%
% \paragraph{Forwarding for a Complete Draft.}
%
% The following forwarding file |cdocsdrf.tex|
% compiles the main document in draft mode:
%\iffalse
%<*sampledraft>
%\fi
%    \begin{macrocode}
\def\version{draft}
\input{childdoc.def}
\childdocforward{cdocsamp}
%    \end{macrocode}

%\iffalse
%</sampledraft>
%\fi
%
% %%%%%%%%%%%%%%%%%%%%%%%%%%%%%%%%%%%%%%
% \paragraph{Forwarding for Final Version of the Chapters.}
%
% The following forwarding files |cdocsfn1.tex| and |cdocsfn2.tex|
% (with identical content)
% compile the final versions of the child documents
% |cdocsch1.tex| and |cdocsch2.tex|, respectively:
%\iffalse
%<*samplefinal>
%\fi
%    \begin{macrocode}
\def\version{final}
\input{childdoc.def}
\childdocforwardprefix[cdocsamp]{cdocsfn}{cdocsch}
%    \end{macrocode}

%\iffalse
%</samplefinal>
%\fi
%
% %%%%%%%%%%%%%%%%%%%%%%%%%%%%%%%%%%%%%%
% \paragraph{Command Line Processing.}
%
% The following three command lines generate the output files
% |cdocscld|, |cdocscl1| and |cdocscl2|
% which should be identical to
% |cdocsdrf|, |cdocsch1| and |cdocsfn2|, respectively:
% \begin{center}
% \begin{tabular}{l}
% |latex -jobname cdocscld \|\\
% |  "\def\version{draft}\input{childdoc.def}\childdocforward{cdocsamp}"|\\
% |latex -jobname cdocscl1 \|\\
% |  "\input{childdoc.def}\childdocforward[cdocsamp]{cdocsch1}"|\\
% |latex -jobname cdocscl2 \|\\
% |  "\def\version{final}\input{childdoc.def}\childdocforward{cdocsch2}"|
% \end{tabular}
% \end{center}
% Note that the trailing backslash on each first line
% merely continues the input to the second line
% (for convenient cut ant paste).
% Furthermore, the command |latex| can be replaced by any
% of its alternative versions such as |pdflatex|.
%
% %%%%%%%%%%%%%%%%%%%%%%%%%%%%%%%%%%%%%%%%%%%%%%%%%%%%%%%%%%%%%%%%%%%%%%%%%%%%%%
% %%%%%%%%%%%%%%%%%%%%%%%%%%%%%%%%%%%%%%%%%%%%%%%%%%%%%%%%%%%%%%%%%%%%%%%%%%%%%%
% \section{Implementation}
%\iffalse
%<*package>
%\fi
%
% This section describes the definitions file |childdoc.def|.

% The definitions cannot be loaded using |\usepackage| or |\RequirePackage|
% which has a mechanism to prevent loading a style file more than once.
% When loading the definitions by means of |\input|
% multiple instances have to be prevented manually:
%\iffalse
%This code needs to be before the `\ProvidesFile' directive
%which is defined at the beginning of this file.
%Therefore it is also placed there and commented out here.
%</package>
%<*discard>
%\fi
%    \begin{macrocode}
\ifdefined\childdocmain\endinput\fi
%    \end{macrocode}
%\iffalse
%</discard>
%<*package>
%\fi
%
% \macro{\ifchilddoc}
% \macro{\ifchilddocmanual}
% The conditional |\ifchilddoc| tells whether a
% child (true) or main (false) document is being compiled.
% The conditional |\ifchilddocmanual| tells whether
% the |\includeonly| mechanism is used (false) or
% the selection of child files must be performed manually (true).
% The definitions initialise to false:
%    \begin{macrocode}
\newif\ifchilddoc
\newif\ifchilddocmanual
%    \end{macrocode}

% \macro{\childdocname}
% \macro{\childdocjob}
% The macro |\childdocname| stores the name of the main document
% to be compiled. The macro |\childdocjob| stores the name of
% the document on which the \LaTeX{} compiler was originally invoked.
% The content of |\jobname| cannot be compared
% to filenames specified in the source due to different catcodes.
% The following code rescans |\jobname|, stores the result
% in |\childdocname| and saves a copy in |\childdocjob|:
%    \begin{macrocode}
\edef\childdocname{\scantokens\expandafter{\jobname\noexpand}}
\let\childdocjob\childdocname
%    \end{macrocode}

% \macro{\childdocdisable}
% The macro |\childdocdisable| prevents the main file
% from being processed more than once.
% At this stage, the main document command |\childdocmain|
% is assumed to be called once again where it should do nothing.
% Any subsequent call to it should prevent
% a secondary processing of the main document
% It overwrites the forwarding commands
% |\childdocof| and |\childdocforward|
% with empty macros to prevent further inclusions of the main document:
%    \begin{macrocode}
\newcommand{\childdocdisable}
{
  \renewcommand{\childdocmain}[1]{\renewcommand{\childdocmain}[1]{\endinput}}
  \renewcommand{\childdocof}[1]{}
  \renewcommand{\childdocby}[2][]{}
  \renewcommand{\childdocforward}[2][]{}
  \renewcommand{\childdocdisable}{}
}
%    \end{macrocode}

% \macro{\childdocmain}
% The macro |\childdocmain| is to be called at the top of the main file
% with nothing or the main filename (without extension) as argument.
% First, it breaks loops.
% If the argument is not empty and does not match |\childdocname|
% (which is set by the first inclusion of |childdoc.def|),
% |\ifchilddoc| is set to true, |\includeonly| is applied to the child file
% and |\jobname| is set to the main file
% (for proper handling of |.aux| files):
%    \begin{macrocode}
\newcommand{\childdocmain}[1]
{
  \childdocdisable\childdocmain{}
  \if?#1?\else
    \begingroup
      \def\childdoctmp{#1}
      \ifx\childdoctmp\childdocname
        \def\childdoctmp{}
      \else
        \def\childdoctmp
        {
          \childdoctrue
          \includeonly{\childdocname}
          \def\childdocjob{#1}
          \def\jobname{#1}
        }
      \fi
      \expandafter
    \endgroup
    \childdoctmp
  \fi
}
%    \end{macrocode}

% \macro{\childdocof}
% The command |\childdocof| redirects
% compilation to the main file |#1|.
%    \begin{macrocode}
\newcommand{\childdocof}[1]
{
  \childdocdisable
  \childdoctrue
  \includeonly{\childdocname}
  \def\jobname{#1}
  \def\childdocjob{#1}
  \input{#1}
}
%    \end{macrocode}

% \macro{\childdocby}
% The command |\childdocby| ....
%    \begin{macrocode}
\newcommand{\childdocby}[2][]
{
  \childdocdisable
  \childdoctrue
  \childdocmanualtrue
  \if?#1?\else
    \def\jobname{#2}
  \fi
  \def\childdocjob{#2}
  \input{#2}
  \endinput
}
%    \end{macrocode}

% \macro{\childdocforward}
% The command |\childdocforward| redirects
% compilation to the main file or
% (if the optional argument is given) a child file.
% Parameters are set as if the main file
% or a child file starting with |\childdocof| was compiled.
% Then compilation is handed over to the main file:
%    \begin{macrocode}
\newcommand{\childdocforward}[2][]
{
  \begingroup
    \if?#1?
      \def\childdoctmp
      {
        \def\childdocname{#2}
        \def\childdocjob{#2}
        \def\jobname{#2}
        \input{#2}
        \endinput
      }
    \else
      \def\childdoctmp
      {
        \childdocdisable
        \def\childdocname{#2}
        \childdoctrue
        \includeonly{#2}
        \def\childdocjob{#1}
        \def\jobname{#1}
        \input{#1}
        \endinput
      }
    \fi
    \expandafter
  \endgroup
  \childdoctmp
}
%    \end{macrocode}

% \macro{\childdocforwardprefix}
% The command |\childdocforwardprefix| redirects
% compilation to the main or a child file by means of a pattern.
% The prefix |#1| in the current filename is replaced by |#2|
% and the suffix of the current filename is kept
% (it is assumed that the filename does not contain the substring `|~~~|'
% which is used as a delimiter).
% Compilation is handed over to the new file by |\childdocforward|:
%    \begin{macrocode}
\newcommand{\childdocforwardprefix}[3][]
{
  \begingroup
    \def\childdocextract #2##1~~~{\def\childdoctmp{\childdocforward[#1]{#3##1}}}
    \expandafter\childdocextract\childdocname~~~
    \expandafter
  \endgroup
  \childdoctmp
}
%    \end{macrocode}

% \macro{\childdoc}
% The deprecated macro |\childdoc| is a legacy version of |\childdocmain|:
%    \begin{macrocode}
\newcommand{\childdoc}{\childdocmain}
%    \end{macrocode}

% \macro{\childdocredirect}
% The deprecated macro |\childdocredirect| is a legacy version
% of |\childdocforward| and |\childdocforwardprefix|:
%    \begin{macrocode}
\newcommand{\childdocredirect}[2][]
{
  \begingroup
    \if?#1?
      \def\childdoctmp{\childdocforward{#2}}
    \else
      \def\childdoctmp{\childdocforwardprefix{#1}{#2}}
    \fi
    \expandafter
  \endgroup
  \childdoctmp
}
%    \end{macrocode}

%\iffalse
%</package>
%\fi
%
\endinput

\childdocby{cdocsamp}
%    \end{macrocode}

%\iffalse
%</samplepart3|samplepart4>
%\fi
%
%\iffalse
%<*samplepart3>
%\fi
% Some text for part 3:
%    \begin{macrocode}
some text in part three
%    \end{macrocode}

%\iffalse
%</samplepart3>
%\fi
% Some text for part 4:
%\iffalse
%<*samplepart4>
%\fi
%    \begin{macrocode}
more text in part four
%    \end{macrocode}

%\iffalse
%</samplepart4>
%\fi
%
% %%%%%%%%%%%%%%%%%%%%%%%%%%%%%%%%%%%%%%
% \paragraph{Forwarding for a Complete Draft.}
%
% The following forwarding file |cdocsdrf.tex|
% compiles the main document in draft mode:
%\iffalse
%<*sampledraft>
%\fi
%    \begin{macrocode}
\def\version{draft}
% \iffalse
%
% childdoc.dtx Copyright (C) 2017-2018 Niklas Beisert
%
% This work may be distributed and/or modified under the
% conditions of the LaTeX Project Public License, either version 1.3
% of this license or (at your option) any later version.
% The latest version of this license is in
%   http://www.latex-project.org/lppl.txt
% and version 1.3 or later is part of all distributions of LaTeX
% version 2005/12/01 or later.
%
% This work has the LPPL maintenance status `maintained'.
%
% The Current Maintainer of this work is Niklas Beisert.
%
% This work consists of the files childdoc.dtx and childdoc.ins
% and the derived files childdoc.def and cdocsamp.tex with
% cdocsch1.tex, cdocsch2.tex, cdocsdrf.tex, cdocsfn1.tex, cdocsfn2.tex.
%
%<package>\ifdefined\childdocmain\endinput\fi
%<package>\ProvidesFile{childdoc.def}[2018/12/30 v2.0 child document driver]
%<samplemain>\ProvidesFile{cdocsamp.tex}[2018/12/30 v2.0 sample for childdoc]
%<*driver>
%\ProvidesFile{childdoc.drv}[2018/12/30 v2.0 childdoc reference manual file]
\PassOptionsToClass{10pt,a4paper}{article}
\documentclass{ltxdoc}

\usepackage[margin=35mm]{geometry}
\usepackage{hyperref}
\usepackage{hyperxmp}
\usepackage[usenames]{color}

\hypersetup{colorlinks=true}
\hypersetup{pdfstartview=FitH}
\hypersetup{pdfpagemode=UseNone}
\hypersetup{pdfsource={}}
\hypersetup{pdflang={en-UK}}
\hypersetup{pdfcopyright={Copyright 2017-2018 Niklas Beisert.
  This work may be distributed and/or modified under the
  conditions of the LaTeX Project Public License, either version 1.3
  of this license or (at your option) any later version.}}
\hypersetup{pdflicenseurl={http://www.latex-project.org/lppl.txt}}
\hypersetup{pdfcontactaddress={ETH Zurich, ITP, HIT K,
  Wolfgang-Pauli-Strasse 27}}
\hypersetup{pdfcontactpostcode={8093}}
\hypersetup{pdfcontactcity={Zurich}}
\hypersetup{pdfcontactcountry={Switzerland}}
\hypersetup{pdfcontactemail={nbeisert@itp.phys.ethz.ch}}
\hypersetup{pdfcontacturl={http://people.phys.ethz.ch/\xmptilde nbeisert/}}

\newcommand{\secref}[1]{\hyperref[#1]{section \ref*{#1}}}

\parskip1ex
\parindent0pt
\let\olditemize\itemize
\def\itemize{\olditemize\parskip0pt}

\begin{document}

\title{The \textsf{childdoc} Package}
\hypersetup{pdftitle={The childdoc Package}}
\author{Niklas Beisert\\[2ex]
  Institut f\"ur Theoretische Physik\\
  Eidgen\"ossische Technische Hochschule Z\"urich\\
  Wolfgang-Pauli-Strasse 27, 8093 Z\"urich, Switzerland\\[1ex]
  \href{mailto:nbeisert@itp.phys.ethz.ch}
  {\texttt{nbeisert@itp.phys.ethz.ch}}}
\hypersetup{pdfauthor={Niklas Beisert}}
\hypersetup{pdfsubject={Manual for the LaTeX2e Package childdoc}}
\date{30 December 2018, \textsf{v2.0}}
\maketitle

\begin{abstract}\noindent
\textsf{childdoc} is a \LaTeXe{} package
that enables the direct compilation
of document sections included by |\include|
to individual files.
\end{abstract}

\begingroup
\parskip0ex
\tableofcontents
\endgroup

%%%%%%%%%%%%%%%%%%%%%%%%%%%%%%%%%%%%%%%%%%%%%%%%%%%%%%%%%%%%%%%%%%%%%%%%%%%%%%%%
%%%%%%%%%%%%%%%%%%%%%%%%%%%%%%%%%%%%%%%%%%%%%%%%%%%%%%%%%%%%%%%%%%%%%%%%%%%%%%%%
\section{Introduction}

\LaTeX{} provides a mechanism to structure a large document (such as a book)
into a main file and several child files (containing the chapters)
using the |\include| command.
This mechanism is beneficial for documents
which span hundreds of pages in order to
make the source file(s) more manageable.
Moreover, compilation can be restricted to
selected child files by means of the |\includeonly| command.
The latter feature can be used to reduce the compilation time while editing
(this was significantly more useful in the earlier days of \LaTeX{})
or to generate a smaller document which is easier to navigate.
Another application of |\includeonly| is to generate
documents consisting of selected parts of the complete document.

However, there are a few drawbacks of the plain |\include| mechanism:
\begin{itemize}
\item
The child files cannot be compiled on their own,
they can only be compiled via the main file.
A naive editing environment
(such as a text editor with an option
to have the current file processed by \LaTeX)
may require one to switch to the main file before compiling;
attempting to compile the child file produces errors.
\item
The main file must be modified (each time)
to adjust the |\includeonly| command
to the present needs. This easily leaves the main file in a messy state.
\item
The generated document will always carry the filename
of the main document. This is inconvenient if
several child files are to be compiled and
to be kept for distribution.
\end{itemize}

The present package provides a simple interface
to make child files individually compilable by \LaTeX{}.
Compiling a child file then has the same effect as compiling
the main file with an |\includeonly| command
to select the appropriate child.
Moreover the generated document will carry the name of the child
rather than the main file.
This resolves all three above issues.

This feature is meant to make the editing of books,
thesis documents and lecture notes somewhat more convenient.
However, the package can also be used efficiently for
composing a series of documents (such as exercise sheets)
which are typically distributed individually.
It then assists the author in generating the individual documents
(potentially in different versions)
as well as a document containing the collected series.
Another application is in developing style files
or other kinds of included material
where compilation of the style file could redirect
to a sample or test file.

%%%%%%%%%%%%%%%%%%%%%%%%%%%%%%%%%%%%%%%%%%%%%%%%%%%%%%%%%%%%%%%%%%%%%%%%%%%%%%%%
%%%%%%%%%%%%%%%%%%%%%%%%%%%%%%%%%%%%%%%%%%%%%%%%%%%%%%%%%%%%%%%%%%%%%%%%%%%%%%%%
\section{Usage}

First of all, the package \textsf{childdoc} is \emph{not} a standard
\LaTeXe{} |.sty| style file! Therefore it needs to be invoked in
a non-standard way.

%%%%%%%%%%%%%%%%%%%%%%%%%%%%%%%%%%%%%%%%%%%%%%%%%%%%%%%%%%%%%%%%%%%%%%%%%%%%%%%%
\subsection{Included Files}
\label{sec:include}

%%%%%%%%%%%%%%%%%%%%%%%%%%%%%%%%%%%%%%%%
\DescribeMacro{\childdocmain}
To use the package, add the commands
\begin{center}
\begin{tabular}{l}
|\input{childdoc.def}|\\
|\childdocmain{}|\\
\end{tabular}
\end{center}
at the very top of the main \LaTeX{} file,
in particular \emph{before} the |\documentclass| statement!
The argument of |\childdocmain| should be left empty
(but it must be present).

%%%%%%%%%%%%%%%%%%%%%%%%%%%%%%%%%%%%%%%%
\DescribeMacro{\childdocof}
Furthermore, add the commands
\begin{center}
\begin{tabular}{l}
|\input{childdoc.def}|\\
|\childdocof{|\textit{main}|}|\\
\end{tabular}
\end{center}
at the top of every child file \textit{child}
which is included by |\include{|\textit{child}|}|
from within the main file
(or at least for those files to be compiled individually).
The argument \textit{main} must be the filename of the main file.

There are a couple of
considerations in setting up the main and child documents:

%%%%%%%%%%%%%%%%%%%%%%%%%%%%%%%%%%%%%%%%
\paragraph{Restrictions.}

Please note the following restrictions:
\begin{itemize}
\item
|\childdocmain| must be called with one argument \textit{main}
to ensure compatibility with earlier version of the package.
It must either be empty (|\childdocmain{}|)
or precisely match the filename of the main file in which it is specified.
See \secref{sec:detection} for further information.
\item
The filename \textit{main} must be specified without the |.tex| extension.
\item
The filename \textit{main} is case sensitive
(even in case-insensitive file systems)
due to internal string comparison.
\item
The argument \textit{main} should be fully expanded, it cannot be a macro.
\item
Subdirectories and special characters should be avoided in filenames.
\item
The command |\childdocmain{|\textit{main}|}| must be followed by a whitespace.
It should not be followed immediately by another command
or by a comment mark `|%|'.
This is because the \TeX{} parser reads the token immediately following
the argument of |\childdocmain| and puts it
at the beginning of every child section;
however, a white\-space is ignored.
\end{itemize}

%%%%%%%%%%%%%%%%%%%%%%%%%%%%%%%%%%%%%%%%
\paragraph{Content of Main File.}

It is advisable to place all content in the child files included by |\include|.
Any output contained in the main file will appear in all child documents
unless suppressed manually;
it cannot be suppressed automatically by the |\includeonly| directive
and thus should normally be avoided.
A method to include some content in the main file
by means of conditional processing is described in \secref{sec:conditional}.

%%%%%%%%%%%%%%%%%%%%%%%%%%%%%%%%%%%%%%%%
\paragraph{Page Numbering.}

When only a part of the document is compiled,
the appropriate numbering of pages
(as well as other status parameters)
is determined from the |.aux| files.
The latter contain information from previous passes.
However this information needs to propagate through
all intermediate child documents.
Therefore the page numbering in child documents may well
be inconsistent until the complete document is compiled at least once.

A useful (if unconventional) way to always ensure a consistent
page numbering is to restart the numbering in each child document
and denote the pages by `\textit{child}|.|\textit{page}'
where \textit{child} represents the chapter/section number of the child file.
This can be achieved by the command
|\numberwithin{page}{|\textit{child}|}|
of the \textsf{amsmath} package
where \textit{child} can be |chapter| or |section|
depending on the chosen structuring.
Alternatively, one can modify the macro |\thepage| appropriately
and reset the counter |page| at the start of each child file.

%%%%%%%%%%%%%%%%%%%%%%%%%%%%%%%%%%%%%%%%%%%%%%%%%%%%%%%%%%%%%%%%%%%%%%%%%%%%%%%%
\subsection{Conditional Processing}
\label{sec:conditional}

The package provides a mechanism to compile different versions
of a document. To customise the versions further some conditional processing
can come in handy to distinguish which version is being compiled.
The package provides two macros to describe the compilation context:

%%%%%%%%%%%%%%%%%%%%%%%%%%%%%%%%%%%%%%%%
\DescribeMacro{\ifchilddoc}
The conditional |\ifchilddoc| distinguishes between the compilation of
child documents and the main document:
%
\begin{center}
|\ifchilddoc |\textit{child-code}| |[|\||else |\textit{main-code}]| \||fi|
\end{center}

%%%%%%%%%%%%%%%%%%%%%%%%%%%%%%%%%%%%%%%%
\DescribeMacro{\childdocname}
\DescribeMacro{\childdocjob}
The macro |\childdocname| contains the filename (without extension)
of the main or child file being processed.
Note that |\childdocjob| will always contain the name of the main file.

%%%%%%%%%%%%%%%%%%%%%%%%%%%%%%%%%%%%%%%%
\paragraph{Title Page.}

Conditional processing can be used to include a title or banner page
in the main document when proper precautions are taken.
Importantly, the code in the main file should ensure that the page counter
(as well as other status parameters which are stored in the |.aux| files)
takes the same value after the conditional processing.
Otherwise the page numbers may take divergent values
depending on which part is compiled.

For example, a title page could be declared by:
%
\begin{center}
\begin{tabular}{l}
|\ifchilddoc\||else|\\
|\addtocounter{page}{-1}|\\
\textit{code for title page}\\
|\newpage|\\
|\||fi|
\end{tabular}
\end{center}
%
A banner page for the child documents can be generated by:
%
\begin{center}
\begin{tabular}{l}
|\ifchilddoc|\\
|\addtocounter{page}{-1}|\\
\textit{code for banner page}\\
|\newpage|\\
|\||fi|
\end{tabular}
\end{center}
%
Here one could write a message such as:
\begin{center}
|This is the part \childdocname{} of \childdocjob{}.|
\end{center}

%%%%%%%%%%%%%%%%%%%%%%%%%%%%%%%%%%%%%%%%%%%%%%%%%%%%%%%%%%%%%%%%%%%%%%%%%%%%%%%%
\subsection{Flags}
\label{sec:flags}

The package makes it easy to generate different versions
of the main or child documents.
To this end compilation flags can be defined
and assigned different default values.
They will be particularly useful in conjunction
with the forwarding mechanism described in \secref{sec:forward}.

For example, it may be useful to have a flag |\version|
which can be set to |draft| or |final|.
The document source will contain some conditional code
depending on the value of |\version|.
Suppose further, the flag should default to |final| for the main file
and to |draft| for child files
which is a natural assignment for editing the document.
This is achieved by placing the following code
in the preamble of the main document
(below the |\childdocmain| directive):
%
\begin{center}
\begin{tabular}{l}
|\ifchilddoc|\\
|\providecommand{\version}{draft}|\\
|\||else|\\
|\providecommand{\version}{final}|\\
|\||fi|
\end{tabular}
\end{center}
%
The definition by |\providecommand| makes sure
that previous definitions are not overwritten.
Further statements |\providecommand{\version}{...}|
can thus be added before the above code to override it.

For the main file, one might add a line
(between |\childdocmain| and the above block)
%
\begin{center}
|%\ifchilddoc\||else\providecommand{\version}{draft}\||fi|
\end{center}
%
which can be uncommented to produce a draft version.
Likewise one can add a line to the very top of a child file
(above the |\childdocof{|\textit{main}|}| directive)
%
\begin{center}
|%\providecommand{\version}{final}|
\end{center}
%
which can be uncommented to produce the final version of this child document.

%%%%%%%%%%%%%%%%%%%%%%%%%%%%%%%%%%%%%%%%%%%%%%%%%%%%%%%%%%%%%%%%%%%%%%%%%%%%%%%%
\subsection{Forwarding}
\label{sec:forward}

Different versions of the main or child documents
using compilation flags as described in \secref{sec:flags}
can be (permanently) stored in different files
for convenient compilation, viewing and distribution.
To this end, the package defines a command
to pass on compilation to a different file:

%%%%%%%%%%%%%%%%%%%%%%%%%%%%%%%%%%%%%%%%
\DescribeMacro{\childdocforward}
The command |\childdocforward| redirects processing to
another source file:
%
\begin{center}
\begin{tabular}{l}
|\input{childdoc.def}|\\
|\childdocforward[|\textit{main}|]{|\textit{dest}|}|\\
\end{tabular}
\end{center}
%
The argument \textit{dest} is the destination file
(without extension).
It should be the main file or one of the child files.
Note that further \textsf{childdoc} directives
such as |\childdocof| and |\childdocforward|
in the indicated file will be processed in this form.
The optional argument \textit{main}
passes on directly to the main file \textit{main}
while pretending to compile the child \textit{dest}.
This form behaves as if \textit{dest}
issues |\childdocof{|\textit{main}|}| right away,
and no further \textsf{childdoc} directives will be processed.

%%%%%%%%%%%%%%%%%%%%%%%%%%%%%%%%%%%%%%%%
\DescribeMacro{\...prefix}
In the alternative form |\childdocforwardprefix|,
%
\begin{center}
\begin{tabular}{l}
|\input{childdoc.def}|\\
|\childdocforwardprefix[|\textit{main}|]{|\textit{prefix}|}{|\textit{dest}|}|
\end{tabular}
\end{center}
%
the destination file is determined by a pattern
depending on the current file:
To make this work, the current file must be called
`{\textit{prefix}\hspace{0.2em}\textit{suffix}}'
with \textit{prefix} matching precisely the argument.
Processing is then passed on to the file
`{\textit{dest}\hspace{0.2em}\textit{suffix}}'.
Surely, the same effect is achieved by
directly specifying the
argument `{\textit{dest}\hspace{0.2em}\textit{suffix}}'
in the first form.
However, that requires to set up a different file
for each child. With the alternative form of the command
all these files can have exactly the same content
which simplifies setting them up and maintaining them.

For example, the following file |draft.tex|
with a compilation flag |\version| as described in \secref{sec:flags}
compiles the main document as a draft:
%
\begin{center}
\begin{tabular}{l}
|\def\version{draft}|\\
|\input{childdoc.def}|\\
|\childdocforward{|\textit{main}|}|
\end{tabular}
\end{center}
%
Likewise, the following files |final|\textit{nn}|.tex|
compile the final version of the child document
|child|\textit{nn}|.tex|:
%
\begin{center}
\begin{tabular}{l}
|\def\version{final}|\\
|\input{childdoc.def}|\\
|\childdocforwardprefix{final}{child}|
\end{tabular}
\end{center}
%

Note that when several versions of a main file and/or of each child file
are to be generated, it may be convenient to set up a |Makefile| or
shell script to automatise the process.

%%%%%%%%%%%%%%%%%%%%%%%%%%%%%%%%%%%%%%%%%%%%%%%%%%%%%%%%%%%%%%%%%%%%%%%%%%%%%%%%
\subsection{Command Line Processing}
\label{sec:commandline}

The effect of redirection files can also be achieved by invoking
the \LaTeX{} compiler with a more elaborate command line.
Most conveniently this should be done as part
of a shell script or a |Makefile|.

When using \textsf{childdoc} in the main file, the following
command lines effectively perform a redirection
(note that depending on the shell being used,
backslashes may have to be doubled: `|\|' $\to$ `|\\|'):
%
\begin{center}
|... -jobname "|\textit{target}|" |\\|"|[\textit{flags}]%
|\input{childdoc.def}\childdocforward[|\textit{main}|]{|\textit{dest}|}"|
\end{center}
%
Here \textit{target} is the name of the output file,
\textit{main} is the name of the main file
and \textit{dest} is the name of the main or child file to be processed
(all filenames without extensions).
The optional argument \textit{main} can be omitted
if \textit{main} matches \textit{dest}.
Optionally, compilation \textit{flags} can be defined via |\def| commands.
This command line makes the \TeX{} engine believe
it is compiling the file \textit{target}
whose content is specified as the latter parameter.
The provided code then forwards the processing to
\textit{main} or \textit{dest} as described in \secref{sec:forward}.

%%%%%%%%%%%%%%%%%%%%%%%%%%%%%%%%%%%%%%%%%%%%%%%%%%%%%%%%%%%%%%%%%%%%%%%%%%%%%%%%
\subsection{Include by Input}
\label{sec:input}

Including child documents by |\include| has some restrictions by design.
Most notably, the content of a child document always occupies
its own set of pages; pages cannot be shared between child documents.
Usually, this behaviour makes perfect sense
because each child document contain an essential part of the document.
However, in some situations it may be desirable to compose
a document from a collection of parts
without having mandatory page breaks between then.
For this case, the package
provides a mechanism to include parts
by |\input| which can also be processed individually.
However, by construction this mechanism
requires manual handling of the content to be output.

%%%%%%%%%%%%%%%%%%%%%%%%%%%%%%%%%%%%%%%%
\DescribeMacro{\ifchilddocmanual}
The main file should be prepared as usual, see \secref{sec:include}.
However, the document body must make a distinction
between processing of an individual part and of the main document, e.g.:
%
\begin{center}
\begin{tabular}{l}
|\ifchilddocmanual|\\
|\input{\childdocname}|\\
|\||else|\\
\textit{document body with }|\input{|\textit{part}|}|\\
|\||fi|
\end{tabular}
\end{center}
%
The conditional |\ifchilddocmanual| is true whenever
a part to be included by |\input| is being compiled,
and the name of the part is stored in |\childdocname|.

%%%%%%%%%%%%%%%%%%%%%%%%%%%%%%%%%%%%%%%%
\DescribeMacro{\childdocby}
Each part to be included by |\input| should start with:
%
\begin{center}
\begin{tabular}{l}
|\input{childdoc.def}|\\
|\childdocby{|\textit{main}|}|\\
\end{tabular}
\end{center}
%
The directive |\childdocby| is similar to |\childdocof|
described in \secref{sec:include},
but the subsequent selection of content must be done manually.
To that end, both |\ifchilddoc| and |\ifchilddocmanual|
will be true upon processing of a part,
and the name of the part is stored in |\childdocname|.
Note that |\jobname| will be set to the filename of the current part
so that each part receives an individual |.aux| file
that does not interfere with the |.aux| file(s) of the main document.
This behaviour can be altered by the alternative form
|\childdocby[*]{|\textit{main}|}| (with a non-empty optional argument)
which uses the |.aux| file of the main document
by setting |\jobname| to \textit{main}.

%%%%%%%%%%%%%%%%%%%%%%%%%%%%%%%%%%%%%%%%%%%%%%%%%%%%%%%%%%%%%%%%%%%%%%%%%%%%%%%%
\subsection{Driver Development}
\label{sec:driver}

The \textsf{childdoc} mechanism can also be use for the development
of definition files such as \LaTeX{} styles or classes.
This case differs from the above setup with multiple parts
included by |\include| in that no |\includeonly| should be invoked.
This can be achieved by starting the include file
(before |\ProvidesPackage|) with:
%
\begin{center}
\begin{tabular}{l}
|\input{childdoc.def}|\\
|\childdocforward{|\textit{main}|}|\\
\end{tabular}
\end{center}
%
or alternatively with:
%
\begin{center}
\begin{tabular}{l}
|\input{childdoc.def}|\\
|\childdocby{|\textit{main}|}|\\
\end{tabular}
\end{center}
%
Both forms have slightly different effects as described above.
The main file is prepared as usual, see \secref{sec:include}.

%%%%%%%%%%%%%%%%%%%%%%%%%%%%%%%%%%%%%%%%%%%%%%%%%%%%%%%%%%%%%%%%%%%%%%%%%%%%%%%%
\subsection{Legacy Detection}
\label{sec:detection}

The directive |\childdocmain| in the main file can detect
whether the complete document or merely a child is to be compiled
even without using the directive |\childdocof|.
This method is deprecated because it is less robust
and there is no compelling reason to use it;
it is merely provided for backward compatibility
and it may be removed in future versions.

If the detection mechanism is to be used,
it is mandatory to correctly specify
the filename of the main file as the argument of |\childdocmain|:
%
\begin{center}
\begin{tabular}{l}
|\input{childdoc.def}|\\
|\childdocmain{|\textit{main}|}|\\
\end{tabular}
\end{center}
%
If |\jobname| does not match the argument \textit{main} of |\childdocmain|,
it is assumed that |\jobname| points to the child file to be compiled.
When using |\childdocmain| with the main file specified as argument,
it suffices to start a child file
with just |\input{|\textit{main}|}|
without loading of the package and using |\childdocof|.
If instead all processing is done
with the appropriate \textsf{childdoc} directives,
the argument of \textit{main} of |\childdocmain| can be empty.

An alternative version of the command line processing described
in \secref{sec:commandline} using the detection mechanism reads:
%
\begin{center}
|... -jobname "|\textit{target}|" "|[\textit{flags}]%
[|\def\jobname{|\textit{dest}|}|]|\input{|\textit{main}|}"|
\end{center}

%%%%%%%%%%%%%%%%%%%%%%%%%%%%%%%%%%%%%%%%%%%%%%%%%%%%%%%%%%%%%%%%%%%%%%%%%%%%%%%%
\subsection{Manual Code}
\label{sec:manual}

In case one cannot be certain whether the definitions file |childdoc.def|
is installed on the target \TeX{} distribution
and one prefers not to ship it,
it is conceivable to paste a few relevant commands into the sources.

To that end, drop all statements |\input{childdoc.def}|
and perform the replacements as outlined below.
Instead of |\childdocmain{|\textit{main}|}| add the following code
to the top of the main file:
%
\begin{center}
\begin{tabular}{l}
|\||ifdefined\childdocname\endinput\||fi\newif\ifchilddoc|\\
|\edef\childdocname{\scantokens\expandafter{\jobname\noexpand}}|\\
|\def\childdocmain{|\textit{main}|}\||ifx\childdocmain\childdocname\||else|\\
|\childdoctrue\includeonly{\childdocname}\let\jobname\childdocmain\||fi|\\
\end{tabular}
\end{center}
%
Instead of |\childdocof{|\textit{main}|}| just include the main file
at the top of each child file:
%
\begin{center}
|\input{|\textit{main}|}|
\end{center}
%
A simple redirection |\childdocforward{|\textit{dest}|}| is achieved by:
%
\begin{center}
|\def\jobname{|\textit{dest}|}\input{\jobname}|
\end{center}
%
The redirection with prefix
|\childdocforwardprefix[|\textit{prefix}|]{|\textit{dest}|}|
is accomplished by:
%
\begin{center}
\begin{tabular}{l}
|{\edef\jobname{\scantokens\expandafter{\jobname\noexpand}}|\\
|\def\redirectjob |\textit{prefix}|#1~~~{\gdef\jobname{|\textit{dest}|#1}}|\\
|\expandafter\redirectjob\jobname~~~}\input{\jobname}|
\end{tabular}
\end{center}

In an alternative approach,
child documents can be compiled by a specific command line
without additional code or specific definitions:
%
\begin{center}
|... -jobname "|\textit{target}|" "|[\textit{flags}]%
|\includeonly{|\textit{dest}|}\input{|\textit{main}|}"|
\end{center}
%

%%%%%%%%%%%%%%%%%%%%%%%%%%%%%%%%%%%%%%%%%%%%%%%%%%%%%%%%%%%%%%%%%%%%%%%%%%%%%%%%
%%%%%%%%%%%%%%%%%%%%%%%%%%%%%%%%%%%%%%%%%%%%%%%%%%%%%%%%%%%%%%%%%%%%%%%%%%%%%%%%
\section{Information}

%%%%%%%%%%%%%%%%%%%%%%%%%%%%%%%%%%%%%%%%%%%%%%%%%%%%%%%%%%%%%%%%%%%%%%%%%%%%%%%%
\subsection{Copyright}

Copyright \copyright{} 2017--2018 Niklas Beisert

This work may be distributed and/or modified under the
conditions of the \LaTeX{} Project Public License, either version 1.3
of this license or (at your option) any later version.
The latest version of this license is in
  \url{http://www.latex-project.org/lppl.txt}
and version 1.3 or later is part of all distributions of \LaTeX{}
version 2005/12/01 or later.

This work has the LPPL maintenance status `maintained'.

The Current Maintainer of this work is Niklas Beisert.

This work consists of the files |README.txt|, |childdoc.ins| and |childdoc.dtx|
as well as the derived files |childdoc.def|, |cdocsamp.tex|
with |cdocsch1.tex|, |cdocsch2.tex|, |cdocspt3.tex|, |cdocspt4.tex|,
|cdocsdrf.tex|, |cdocsfn1.tex|, |cdocsfn2.tex|
as well as |childdoc.pdf|.

%%%%%%%%%%%%%%%%%%%%%%%%%%%%%%%%%%%%%%%%%%%%%%%%%%%%%%%%%%%%%%%%%%%%%%%%%%%%%%%%
\subsection{Files and Installation}

The package consists of the files:
%
\begin{center}
\begin{tabular}{ll}
    |README.txt|   & readme file \\
    |childdoc.ins| & installation file \\
    |childdoc.dtx| & source file \\
    |childdoc.def| & definition file \\
    |cdocsamp.tex| & sample main file \\
    |cdocsch1.tex| & sample include file \\
    |cdocsch2.tex| & sample include file \\
    |cdocspt3.tex| & sample part file \\
    |cdocspt4.tex| & sample part file \\
    |cdocsdrf.tex| & sample redirection file \\
    |cdocsfn1.tex| & sample redirection file \\
    |cdocsfn2.tex| & sample redirection file \\
    |childdoc.pdf| & manual
\end{tabular}
\end{center}
%
The distribution consists of the files
|README.txt|, |childdoc.ins| and |childdoc.dtx|.
%
\begin{itemize}
\item
Run (pdf)\LaTeX{} on |childdoc.dtx|
to compile the manual |childdoc.pdf| (this file).
\item
Run \LaTeX{} on |childdoc.ins| to create the definitions file |childdoc.def|
and the sample |cdocsamp.tex| with include files
|cdocsch1.tex|, |cdocsch2.tex|, |cdocspt3.tex|, |cdocspt4.tex|,
|cdocsdrf.tex|, |cdocsfn1.tex|, |cdocsfn2.tex|.
Then copy the file |childdoc.def| to an appropriate directory of your \LaTeX{}
distribution, e.g.\ \textit{texmf-root}|/tex/latex/childdoc|.
\end{itemize}

%%%%%%%%%%%%%%%%%%%%%%%%%%%%%%%%%%%%%%%%%%%%%%%%%%%%%%%%%%%%%%%%%%%%%%%%%%%%%%%%
\subsection{Related CTAN Packages}

There are several other packages which offer a similar functionality:
%
\begin{itemize}
\item
The packages
\href{http://ctan.org/pkg/docmute}{\textsf{docmute}},
\href{http://ctan.org/pkg/includex}{\textsf{includex}} and
\href{http://ctan.org/pkg/standalone}{\textsf{standalone}}
provide commands to include only the document body of
a child file thus allowing both files to be compiled individually.
\item
The packages \href{http://ctan.org/pkg/subdocs}{\textsf{subdocs}}
and \href{http://ctan.org/pkg/subfiles}{\textsf{subfiles}}
provide structures in which the main and child documents can be
encapsulated and allowing them to be compiled individually.
The inclusion mechanism is different from the conventional |\include|.
\item
The package \href{http://ctan.org/pkg/combine}{\textsf{combine}}
is an elaborate solution to combine several documents into one.
\end{itemize}
%
See also the CTAN topic \href{http://ctan.org/topic/subdocs}{\textsf{subdocs}}
for further related packages.
The present package differs from the above solutions in that
a document structure constructed with the conventional |\include| mechanism
just needs two extra commands at the top of every file
such that all constituent files can be compiled individually.

%%%%%%%%%%%%%%%%%%%%%%%%%%%%%%%%%%%%%%%%%%%%%%%%%%%%%%%%%%%%%%%%%%%%%%%%%%%%%%%%
%\subsection{Feature Suggestions}
%
%The following is a list of features which may be useful for future
%versions of this package:
%%
%\begin{itemize}
%\item
%\ldots
%\end{itemize}

%%%%%%%%%%%%%%%%%%%%%%%%%%%%%%%%%%%%%%%%%%%%%%%%%%%%%%%%%%%%%%%%%%%%%%%%%%%%%%%%
\subsection{Revision History}

%%%%%%%%%%%%%%%%%%%%%%%%%%%%%%%%%%%%%%%%
\paragraph{v2.0:} 2018/12/30

\begin{itemize}
\item
immediate forward processing
\item
added |\childdocby| mechanism
\item
manual restructured
\end{itemize}

%%%%%%%%%%%%%%%%%%%%%%%%%%%%%%%%%%%%%%%%
\paragraph{v1.6:} 2018/01/17

\begin{itemize}
\item
application for development of include files
\item
corrections to manual
\end{itemize}

%%%%%%%%%%%%%%%%%%%%%%%%%%%%%%%%%%%%%%%%
\paragraph{v1.5:} 2017/05/21

\begin{itemize}
\item
more complete structuring introduced
\item
|\childdocof| introduced
\item
|\childdoc| renamed to |\childdocmain|
\item
|\childredirect| renamed to |\childdocforward| and |\childdocforwardprefix|
and functionality expanded
\end{itemize}

%%%%%%%%%%%%%%%%%%%%%%%%%%%%%%%%%%%%%%%%
\paragraph{v1.0:} 2017/04/27

\begin{itemize}
\item
manual and install package
\item
first version published on CTAN
\end{itemize}

%%%%%%%%%%%%%%%%%%%%%%%%%%%%%%%%%%%%%%%%
\paragraph{v0.6:} 2017/04/26

\begin{itemize}
\item
redirection mechanism added
\end{itemize}

%%%%%%%%%%%%%%%%%%%%%%%%%%%%%%%%%%%%%%%%
\paragraph{v0.5:} 2017/04/26

\begin{itemize}
\item
functionality in definition file
\end{itemize}


%%%%%%%%%%%%%%%%%%%%%%%%%%%%%%%%%%%%%%%%%%%%%%%%%%%%%%%%%%%%%%%%%%%%%%%%%%%%%%%%
%%%%%%%%%%%%%%%%%%%%%%%%%%%%%%%%%%%%%%%%%%%%%%%%%%%%%%%%%%%%%%%%%%%%%%%%%%%%%%%%
%%%%%%%%%%%%%%%%%%%%%%%%%%%%%%%%%%%%%%%%%%%%%%%%%%%%%%%%%%%%%%%%%%%%%%%%%%%%%%%%
\appendix

\settowidth\MacroIndent{\rmfamily\scriptsize 000\ }

 \DocInput{childdoc.dtx}

\end{document}
%</driver>
% \fi
%
% %%%%%%%%%%%%%%%%%%%%%%%%%%%%%%%%%%%%%%%%%%%%%%%%%%%%%%%%%%%%%%%%%%%%%%%%%%%%%%
% %%%%%%%%%%%%%%%%%%%%%%%%%%%%%%%%%%%%%%%%%%%%%%%%%%%%%%%%%%%%%%%%%%%%%%%%%%%%%%
% \section{Sample}
%\iffalse
%<*samplemain>
%\fi
%
% The following presents a sample document
% with two chapters, two parts, a title page,
% a compile flag as well as three forwarding files to set the flag.
% It consists of eight |.tex| files:
% \begin{center}
% \begin{tabular}{ll}
% |cdocsamp.tex|&main file\\
% |cdocsch1.tex|&include file for chapter 1\\
% |cdocsch2.tex|&include file for chapter 2\\
% |cdocspt3.tex|&include file for part 3\\
% |cdocspt4.tex|&include file for part 4\\
% |cdocsdrf.tex|&forwarding file for main file in draft mode\\
% |cdocsfi1.tex|&forwarding file for final version of chapter 1\\
% |cdocsfi2.tex|&forwarding file for final version of chapter 2\\
% \end{tabular}
% \end{center}
% Each of the eight files can be compiled directly by the \LaTeX{} compiler.
%
% %%%%%%%%%%%%%%%%%%%%%%%%%%%%%%%%%%%%%%
% \paragraph{Main File.}
%
% The main file is called |cdocsamp.tex|.
%
% Load the \textsf{childdoc} definitions and
% declare the filename for the main document:
%    \begin{macrocode}
\input{childdoc.def}
\childdocmain{}
%    \end{macrocode}

% Optional override for |\version| flag:
%    \begin{macrocode}
%%\ifchilddoc\else\providecommand{\version}{draft}\fi
%    \end{macrocode}

% Define the default values for the |\version| flag
% (|final| for the main file and |draft| for childs):
%    \begin{macrocode}
\ifchilddoc
\providecommand{\version}{draft}
\else
\providecommand{\version}{final}
\fi
%    \end{macrocode}

% Load the standard document class:
%    \begin{macrocode}
\documentclass[12pt]{article}
%    \end{macrocode}

% Start the document body:
%    \begin{macrocode}
\begin{document}
%    \end{macrocode}

% Declare a title page.
% Print title, part of document being processed and version flag:
%    \begin{macrocode}
\addtocounter{page}{-1}
\begin{center}
{\LARGE\bfseries{}childdoc example\par}
\vspace{1cm}
\ifchilddoc
\ifchilddocmanual part\else chapter\fi:
`\childdocname' of `\childdocjob'\par
\else
main document: `\childdocjob'\par
\fi
version: \version\par
\end{center}
\newpage
%    \end{macrocode}

% Manually include selected file,
% otherwise process as usual:
%    \begin{macrocode}
\ifchilddocmanual
\section*{part `\childdocname'}
\input{\childdocname}
\else
%    \end{macrocode}

% Include the two chapters:
%    \begin{macrocode}
\include{cdocsch1}
\include{cdocsch2}
%    \end{macrocode}

% Include the two parts unless only chapters should be displayed:
%    \begin{macrocode}
\ifchilddoc\else
\section{part three}
\input{cdocspt3}
\section{part four}
\input{cdocspt4}
\fi
%    \end{macrocode}

% Process as usual until here:
%    \begin{macrocode}
\fi
%    \end{macrocode}

% End of document body:
%    \begin{macrocode}
\end{document}
%    \end{macrocode}
%\iffalse
%</samplemain>
%\fi
%
% %%%%%%%%%%%%%%%%%%%%%%%%%%%%%%%%%%%%%%
% \paragraph{Chapter Include Files.}
%
% The include files are called |cdocsch1.tex| and |cdocsch2.tex|.
%
%\iffalse
%<*samplechap1|samplechap2>
%\fi

% Optional override for |\version| flag:
%    \begin{macrocode}
%%\providecommand{\version}{final}
%    \end{macrocode}

% Include the main document:
%    \begin{macrocode}
\input{childdoc.def}
\childdocof{cdocsamp}
%    \end{macrocode}

%\iffalse
%</samplechap1|samplechap2>
%\fi
%
%\iffalse
%<*samplechap1>
%\fi
% Some text for chapter 1:
%    \begin{macrocode}
\section{one}
some text in chapter one
%    \end{macrocode}

%\iffalse
%</samplechap1>
%\fi
% Some text for chapter 2:
%\iffalse
%<*samplechap2>
%\fi
%    \begin{macrocode}
\section{two}
more text in chapter two
%    \end{macrocode}

%\iffalse
%</samplechap2>
%\fi
%
% %%%%%%%%%%%%%%%%%%%%%%%%%%%%%%%%%%%%%%
% \paragraph{Part Include Files.}
%
% The include files are called |cdocspt3.tex| and |cdocspt4.tex|.
%
%\iffalse
%<*samplepart3|samplepart4>
%\fi

% Optional override for |\version| flag:
%    \begin{macrocode}
%%\providecommand{\version}{final}
%    \end{macrocode}

% Include the main document:
%    \begin{macrocode}
\input{childdoc.def}
\childdocby{cdocsamp}
%    \end{macrocode}

%\iffalse
%</samplepart3|samplepart4>
%\fi
%
%\iffalse
%<*samplepart3>
%\fi
% Some text for part 3:
%    \begin{macrocode}
some text in part three
%    \end{macrocode}

%\iffalse
%</samplepart3>
%\fi
% Some text for part 4:
%\iffalse
%<*samplepart4>
%\fi
%    \begin{macrocode}
more text in part four
%    \end{macrocode}

%\iffalse
%</samplepart4>
%\fi
%
% %%%%%%%%%%%%%%%%%%%%%%%%%%%%%%%%%%%%%%
% \paragraph{Forwarding for a Complete Draft.}
%
% The following forwarding file |cdocsdrf.tex|
% compiles the main document in draft mode:
%\iffalse
%<*sampledraft>
%\fi
%    \begin{macrocode}
\def\version{draft}
\input{childdoc.def}
\childdocforward{cdocsamp}
%    \end{macrocode}

%\iffalse
%</sampledraft>
%\fi
%
% %%%%%%%%%%%%%%%%%%%%%%%%%%%%%%%%%%%%%%
% \paragraph{Forwarding for Final Version of the Chapters.}
%
% The following forwarding files |cdocsfn1.tex| and |cdocsfn2.tex|
% (with identical content)
% compile the final versions of the child documents
% |cdocsch1.tex| and |cdocsch2.tex|, respectively:
%\iffalse
%<*samplefinal>
%\fi
%    \begin{macrocode}
\def\version{final}
\input{childdoc.def}
\childdocforwardprefix[cdocsamp]{cdocsfn}{cdocsch}
%    \end{macrocode}

%\iffalse
%</samplefinal>
%\fi
%
% %%%%%%%%%%%%%%%%%%%%%%%%%%%%%%%%%%%%%%
% \paragraph{Command Line Processing.}
%
% The following three command lines generate the output files
% |cdocscld|, |cdocscl1| and |cdocscl2|
% which should be identical to
% |cdocsdrf|, |cdocsch1| and |cdocsfn2|, respectively:
% \begin{center}
% \begin{tabular}{l}
% |latex -jobname cdocscld \|\\
% |  "\def\version{draft}\input{childdoc.def}\childdocforward{cdocsamp}"|\\
% |latex -jobname cdocscl1 \|\\
% |  "\input{childdoc.def}\childdocforward[cdocsamp]{cdocsch1}"|\\
% |latex -jobname cdocscl2 \|\\
% |  "\def\version{final}\input{childdoc.def}\childdocforward{cdocsch2}"|
% \end{tabular}
% \end{center}
% Note that the trailing backslash on each first line
% merely continues the input to the second line
% (for convenient cut ant paste).
% Furthermore, the command |latex| can be replaced by any
% of its alternative versions such as |pdflatex|.
%
% %%%%%%%%%%%%%%%%%%%%%%%%%%%%%%%%%%%%%%%%%%%%%%%%%%%%%%%%%%%%%%%%%%%%%%%%%%%%%%
% %%%%%%%%%%%%%%%%%%%%%%%%%%%%%%%%%%%%%%%%%%%%%%%%%%%%%%%%%%%%%%%%%%%%%%%%%%%%%%
% \section{Implementation}
%\iffalse
%<*package>
%\fi
%
% This section describes the definitions file |childdoc.def|.

% The definitions cannot be loaded using |\usepackage| or |\RequirePackage|
% which has a mechanism to prevent loading a style file more than once.
% When loading the definitions by means of |\input|
% multiple instances have to be prevented manually:
%\iffalse
%This code needs to be before the `\ProvidesFile' directive
%which is defined at the beginning of this file.
%Therefore it is also placed there and commented out here.
%</package>
%<*discard>
%\fi
%    \begin{macrocode}
\ifdefined\childdocmain\endinput\fi
%    \end{macrocode}
%\iffalse
%</discard>
%<*package>
%\fi
%
% \macro{\ifchilddoc}
% \macro{\ifchilddocmanual}
% The conditional |\ifchilddoc| tells whether a
% child (true) or main (false) document is being compiled.
% The conditional |\ifchilddocmanual| tells whether
% the |\includeonly| mechanism is used (false) or
% the selection of child files must be performed manually (true).
% The definitions initialise to false:
%    \begin{macrocode}
\newif\ifchilddoc
\newif\ifchilddocmanual
%    \end{macrocode}

% \macro{\childdocname}
% \macro{\childdocjob}
% The macro |\childdocname| stores the name of the main document
% to be compiled. The macro |\childdocjob| stores the name of
% the document on which the \LaTeX{} compiler was originally invoked.
% The content of |\jobname| cannot be compared
% to filenames specified in the source due to different catcodes.
% The following code rescans |\jobname|, stores the result
% in |\childdocname| and saves a copy in |\childdocjob|:
%    \begin{macrocode}
\edef\childdocname{\scantokens\expandafter{\jobname\noexpand}}
\let\childdocjob\childdocname
%    \end{macrocode}

% \macro{\childdocdisable}
% The macro |\childdocdisable| prevents the main file
% from being processed more than once.
% At this stage, the main document command |\childdocmain|
% is assumed to be called once again where it should do nothing.
% Any subsequent call to it should prevent
% a secondary processing of the main document
% It overwrites the forwarding commands
% |\childdocof| and |\childdocforward|
% with empty macros to prevent further inclusions of the main document:
%    \begin{macrocode}
\newcommand{\childdocdisable}
{
  \renewcommand{\childdocmain}[1]{\renewcommand{\childdocmain}[1]{\endinput}}
  \renewcommand{\childdocof}[1]{}
  \renewcommand{\childdocby}[2][]{}
  \renewcommand{\childdocforward}[2][]{}
  \renewcommand{\childdocdisable}{}
}
%    \end{macrocode}

% \macro{\childdocmain}
% The macro |\childdocmain| is to be called at the top of the main file
% with nothing or the main filename (without extension) as argument.
% First, it breaks loops.
% If the argument is not empty and does not match |\childdocname|
% (which is set by the first inclusion of |childdoc.def|),
% |\ifchilddoc| is set to true, |\includeonly| is applied to the child file
% and |\jobname| is set to the main file
% (for proper handling of |.aux| files):
%    \begin{macrocode}
\newcommand{\childdocmain}[1]
{
  \childdocdisable\childdocmain{}
  \if?#1?\else
    \begingroup
      \def\childdoctmp{#1}
      \ifx\childdoctmp\childdocname
        \def\childdoctmp{}
      \else
        \def\childdoctmp
        {
          \childdoctrue
          \includeonly{\childdocname}
          \def\childdocjob{#1}
          \def\jobname{#1}
        }
      \fi
      \expandafter
    \endgroup
    \childdoctmp
  \fi
}
%    \end{macrocode}

% \macro{\childdocof}
% The command |\childdocof| redirects
% compilation to the main file |#1|.
%    \begin{macrocode}
\newcommand{\childdocof}[1]
{
  \childdocdisable
  \childdoctrue
  \includeonly{\childdocname}
  \def\jobname{#1}
  \def\childdocjob{#1}
  \input{#1}
}
%    \end{macrocode}

% \macro{\childdocby}
% The command |\childdocby| ....
%    \begin{macrocode}
\newcommand{\childdocby}[2][]
{
  \childdocdisable
  \childdoctrue
  \childdocmanualtrue
  \if?#1?\else
    \def\jobname{#2}
  \fi
  \def\childdocjob{#2}
  \input{#2}
  \endinput
}
%    \end{macrocode}

% \macro{\childdocforward}
% The command |\childdocforward| redirects
% compilation to the main file or
% (if the optional argument is given) a child file.
% Parameters are set as if the main file
% or a child file starting with |\childdocof| was compiled.
% Then compilation is handed over to the main file:
%    \begin{macrocode}
\newcommand{\childdocforward}[2][]
{
  \begingroup
    \if?#1?
      \def\childdoctmp
      {
        \def\childdocname{#2}
        \def\childdocjob{#2}
        \def\jobname{#2}
        \input{#2}
        \endinput
      }
    \else
      \def\childdoctmp
      {
        \childdocdisable
        \def\childdocname{#2}
        \childdoctrue
        \includeonly{#2}
        \def\childdocjob{#1}
        \def\jobname{#1}
        \input{#1}
        \endinput
      }
    \fi
    \expandafter
  \endgroup
  \childdoctmp
}
%    \end{macrocode}

% \macro{\childdocforwardprefix}
% The command |\childdocforwardprefix| redirects
% compilation to the main or a child file by means of a pattern.
% The prefix |#1| in the current filename is replaced by |#2|
% and the suffix of the current filename is kept
% (it is assumed that the filename does not contain the substring `|~~~|'
% which is used as a delimiter).
% Compilation is handed over to the new file by |\childdocforward|:
%    \begin{macrocode}
\newcommand{\childdocforwardprefix}[3][]
{
  \begingroup
    \def\childdocextract #2##1~~~{\def\childdoctmp{\childdocforward[#1]{#3##1}}}
    \expandafter\childdocextract\childdocname~~~
    \expandafter
  \endgroup
  \childdoctmp
}
%    \end{macrocode}

% \macro{\childdoc}
% The deprecated macro |\childdoc| is a legacy version of |\childdocmain|:
%    \begin{macrocode}
\newcommand{\childdoc}{\childdocmain}
%    \end{macrocode}

% \macro{\childdocredirect}
% The deprecated macro |\childdocredirect| is a legacy version
% of |\childdocforward| and |\childdocforwardprefix|:
%    \begin{macrocode}
\newcommand{\childdocredirect}[2][]
{
  \begingroup
    \if?#1?
      \def\childdoctmp{\childdocforward{#2}}
    \else
      \def\childdoctmp{\childdocforwardprefix{#1}{#2}}
    \fi
    \expandafter
  \endgroup
  \childdoctmp
}
%    \end{macrocode}

%\iffalse
%</package>
%\fi
%
\endinput

\childdocforward{cdocsamp}
%    \end{macrocode}

%\iffalse
%</sampledraft>
%\fi
%
% %%%%%%%%%%%%%%%%%%%%%%%%%%%%%%%%%%%%%%
% \paragraph{Forwarding for Final Version of the Chapters.}
%
% The following forwarding files |cdocsfn1.tex| and |cdocsfn2.tex|
% (with identical content)
% compile the final versions of the child documents
% |cdocsch1.tex| and |cdocsch2.tex|, respectively:
%\iffalse
%<*samplefinal>
%\fi
%    \begin{macrocode}
\def\version{final}
% \iffalse
%
% childdoc.dtx Copyright (C) 2017-2018 Niklas Beisert
%
% This work may be distributed and/or modified under the
% conditions of the LaTeX Project Public License, either version 1.3
% of this license or (at your option) any later version.
% The latest version of this license is in
%   http://www.latex-project.org/lppl.txt
% and version 1.3 or later is part of all distributions of LaTeX
% version 2005/12/01 or later.
%
% This work has the LPPL maintenance status `maintained'.
%
% The Current Maintainer of this work is Niklas Beisert.
%
% This work consists of the files childdoc.dtx and childdoc.ins
% and the derived files childdoc.def and cdocsamp.tex with
% cdocsch1.tex, cdocsch2.tex, cdocsdrf.tex, cdocsfn1.tex, cdocsfn2.tex.
%
%<package>\ifdefined\childdocmain\endinput\fi
%<package>\ProvidesFile{childdoc.def}[2018/12/30 v2.0 child document driver]
%<samplemain>\ProvidesFile{cdocsamp.tex}[2018/12/30 v2.0 sample for childdoc]
%<*driver>
%\ProvidesFile{childdoc.drv}[2018/12/30 v2.0 childdoc reference manual file]
\PassOptionsToClass{10pt,a4paper}{article}
\documentclass{ltxdoc}

\usepackage[margin=35mm]{geometry}
\usepackage{hyperref}
\usepackage{hyperxmp}
\usepackage[usenames]{color}

\hypersetup{colorlinks=true}
\hypersetup{pdfstartview=FitH}
\hypersetup{pdfpagemode=UseNone}
\hypersetup{pdfsource={}}
\hypersetup{pdflang={en-UK}}
\hypersetup{pdfcopyright={Copyright 2017-2018 Niklas Beisert.
  This work may be distributed and/or modified under the
  conditions of the LaTeX Project Public License, either version 1.3
  of this license or (at your option) any later version.}}
\hypersetup{pdflicenseurl={http://www.latex-project.org/lppl.txt}}
\hypersetup{pdfcontactaddress={ETH Zurich, ITP, HIT K,
  Wolfgang-Pauli-Strasse 27}}
\hypersetup{pdfcontactpostcode={8093}}
\hypersetup{pdfcontactcity={Zurich}}
\hypersetup{pdfcontactcountry={Switzerland}}
\hypersetup{pdfcontactemail={nbeisert@itp.phys.ethz.ch}}
\hypersetup{pdfcontacturl={http://people.phys.ethz.ch/\xmptilde nbeisert/}}

\newcommand{\secref}[1]{\hyperref[#1]{section \ref*{#1}}}

\parskip1ex
\parindent0pt
\let\olditemize\itemize
\def\itemize{\olditemize\parskip0pt}

\begin{document}

\title{The \textsf{childdoc} Package}
\hypersetup{pdftitle={The childdoc Package}}
\author{Niklas Beisert\\[2ex]
  Institut f\"ur Theoretische Physik\\
  Eidgen\"ossische Technische Hochschule Z\"urich\\
  Wolfgang-Pauli-Strasse 27, 8093 Z\"urich, Switzerland\\[1ex]
  \href{mailto:nbeisert@itp.phys.ethz.ch}
  {\texttt{nbeisert@itp.phys.ethz.ch}}}
\hypersetup{pdfauthor={Niklas Beisert}}
\hypersetup{pdfsubject={Manual for the LaTeX2e Package childdoc}}
\date{30 December 2018, \textsf{v2.0}}
\maketitle

\begin{abstract}\noindent
\textsf{childdoc} is a \LaTeXe{} package
that enables the direct compilation
of document sections included by |\include|
to individual files.
\end{abstract}

\begingroup
\parskip0ex
\tableofcontents
\endgroup

%%%%%%%%%%%%%%%%%%%%%%%%%%%%%%%%%%%%%%%%%%%%%%%%%%%%%%%%%%%%%%%%%%%%%%%%%%%%%%%%
%%%%%%%%%%%%%%%%%%%%%%%%%%%%%%%%%%%%%%%%%%%%%%%%%%%%%%%%%%%%%%%%%%%%%%%%%%%%%%%%
\section{Introduction}

\LaTeX{} provides a mechanism to structure a large document (such as a book)
into a main file and several child files (containing the chapters)
using the |\include| command.
This mechanism is beneficial for documents
which span hundreds of pages in order to
make the source file(s) more manageable.
Moreover, compilation can be restricted to
selected child files by means of the |\includeonly| command.
The latter feature can be used to reduce the compilation time while editing
(this was significantly more useful in the earlier days of \LaTeX{})
or to generate a smaller document which is easier to navigate.
Another application of |\includeonly| is to generate
documents consisting of selected parts of the complete document.

However, there are a few drawbacks of the plain |\include| mechanism:
\begin{itemize}
\item
The child files cannot be compiled on their own,
they can only be compiled via the main file.
A naive editing environment
(such as a text editor with an option
to have the current file processed by \LaTeX)
may require one to switch to the main file before compiling;
attempting to compile the child file produces errors.
\item
The main file must be modified (each time)
to adjust the |\includeonly| command
to the present needs. This easily leaves the main file in a messy state.
\item
The generated document will always carry the filename
of the main document. This is inconvenient if
several child files are to be compiled and
to be kept for distribution.
\end{itemize}

The present package provides a simple interface
to make child files individually compilable by \LaTeX{}.
Compiling a child file then has the same effect as compiling
the main file with an |\includeonly| command
to select the appropriate child.
Moreover the generated document will carry the name of the child
rather than the main file.
This resolves all three above issues.

This feature is meant to make the editing of books,
thesis documents and lecture notes somewhat more convenient.
However, the package can also be used efficiently for
composing a series of documents (such as exercise sheets)
which are typically distributed individually.
It then assists the author in generating the individual documents
(potentially in different versions)
as well as a document containing the collected series.
Another application is in developing style files
or other kinds of included material
where compilation of the style file could redirect
to a sample or test file.

%%%%%%%%%%%%%%%%%%%%%%%%%%%%%%%%%%%%%%%%%%%%%%%%%%%%%%%%%%%%%%%%%%%%%%%%%%%%%%%%
%%%%%%%%%%%%%%%%%%%%%%%%%%%%%%%%%%%%%%%%%%%%%%%%%%%%%%%%%%%%%%%%%%%%%%%%%%%%%%%%
\section{Usage}

First of all, the package \textsf{childdoc} is \emph{not} a standard
\LaTeXe{} |.sty| style file! Therefore it needs to be invoked in
a non-standard way.

%%%%%%%%%%%%%%%%%%%%%%%%%%%%%%%%%%%%%%%%%%%%%%%%%%%%%%%%%%%%%%%%%%%%%%%%%%%%%%%%
\subsection{Included Files}
\label{sec:include}

%%%%%%%%%%%%%%%%%%%%%%%%%%%%%%%%%%%%%%%%
\DescribeMacro{\childdocmain}
To use the package, add the commands
\begin{center}
\begin{tabular}{l}
|\input{childdoc.def}|\\
|\childdocmain{}|\\
\end{tabular}
\end{center}
at the very top of the main \LaTeX{} file,
in particular \emph{before} the |\documentclass| statement!
The argument of |\childdocmain| should be left empty
(but it must be present).

%%%%%%%%%%%%%%%%%%%%%%%%%%%%%%%%%%%%%%%%
\DescribeMacro{\childdocof}
Furthermore, add the commands
\begin{center}
\begin{tabular}{l}
|\input{childdoc.def}|\\
|\childdocof{|\textit{main}|}|\\
\end{tabular}
\end{center}
at the top of every child file \textit{child}
which is included by |\include{|\textit{child}|}|
from within the main file
(or at least for those files to be compiled individually).
The argument \textit{main} must be the filename of the main file.

There are a couple of
considerations in setting up the main and child documents:

%%%%%%%%%%%%%%%%%%%%%%%%%%%%%%%%%%%%%%%%
\paragraph{Restrictions.}

Please note the following restrictions:
\begin{itemize}
\item
|\childdocmain| must be called with one argument \textit{main}
to ensure compatibility with earlier version of the package.
It must either be empty (|\childdocmain{}|)
or precisely match the filename of the main file in which it is specified.
See \secref{sec:detection} for further information.
\item
The filename \textit{main} must be specified without the |.tex| extension.
\item
The filename \textit{main} is case sensitive
(even in case-insensitive file systems)
due to internal string comparison.
\item
The argument \textit{main} should be fully expanded, it cannot be a macro.
\item
Subdirectories and special characters should be avoided in filenames.
\item
The command |\childdocmain{|\textit{main}|}| must be followed by a whitespace.
It should not be followed immediately by another command
or by a comment mark `|%|'.
This is because the \TeX{} parser reads the token immediately following
the argument of |\childdocmain| and puts it
at the beginning of every child section;
however, a white\-space is ignored.
\end{itemize}

%%%%%%%%%%%%%%%%%%%%%%%%%%%%%%%%%%%%%%%%
\paragraph{Content of Main File.}

It is advisable to place all content in the child files included by |\include|.
Any output contained in the main file will appear in all child documents
unless suppressed manually;
it cannot be suppressed automatically by the |\includeonly| directive
and thus should normally be avoided.
A method to include some content in the main file
by means of conditional processing is described in \secref{sec:conditional}.

%%%%%%%%%%%%%%%%%%%%%%%%%%%%%%%%%%%%%%%%
\paragraph{Page Numbering.}

When only a part of the document is compiled,
the appropriate numbering of pages
(as well as other status parameters)
is determined from the |.aux| files.
The latter contain information from previous passes.
However this information needs to propagate through
all intermediate child documents.
Therefore the page numbering in child documents may well
be inconsistent until the complete document is compiled at least once.

A useful (if unconventional) way to always ensure a consistent
page numbering is to restart the numbering in each child document
and denote the pages by `\textit{child}|.|\textit{page}'
where \textit{child} represents the chapter/section number of the child file.
This can be achieved by the command
|\numberwithin{page}{|\textit{child}|}|
of the \textsf{amsmath} package
where \textit{child} can be |chapter| or |section|
depending on the chosen structuring.
Alternatively, one can modify the macro |\thepage| appropriately
and reset the counter |page| at the start of each child file.

%%%%%%%%%%%%%%%%%%%%%%%%%%%%%%%%%%%%%%%%%%%%%%%%%%%%%%%%%%%%%%%%%%%%%%%%%%%%%%%%
\subsection{Conditional Processing}
\label{sec:conditional}

The package provides a mechanism to compile different versions
of a document. To customise the versions further some conditional processing
can come in handy to distinguish which version is being compiled.
The package provides two macros to describe the compilation context:

%%%%%%%%%%%%%%%%%%%%%%%%%%%%%%%%%%%%%%%%
\DescribeMacro{\ifchilddoc}
The conditional |\ifchilddoc| distinguishes between the compilation of
child documents and the main document:
%
\begin{center}
|\ifchilddoc |\textit{child-code}| |[|\||else |\textit{main-code}]| \||fi|
\end{center}

%%%%%%%%%%%%%%%%%%%%%%%%%%%%%%%%%%%%%%%%
\DescribeMacro{\childdocname}
\DescribeMacro{\childdocjob}
The macro |\childdocname| contains the filename (without extension)
of the main or child file being processed.
Note that |\childdocjob| will always contain the name of the main file.

%%%%%%%%%%%%%%%%%%%%%%%%%%%%%%%%%%%%%%%%
\paragraph{Title Page.}

Conditional processing can be used to include a title or banner page
in the main document when proper precautions are taken.
Importantly, the code in the main file should ensure that the page counter
(as well as other status parameters which are stored in the |.aux| files)
takes the same value after the conditional processing.
Otherwise the page numbers may take divergent values
depending on which part is compiled.

For example, a title page could be declared by:
%
\begin{center}
\begin{tabular}{l}
|\ifchilddoc\||else|\\
|\addtocounter{page}{-1}|\\
\textit{code for title page}\\
|\newpage|\\
|\||fi|
\end{tabular}
\end{center}
%
A banner page for the child documents can be generated by:
%
\begin{center}
\begin{tabular}{l}
|\ifchilddoc|\\
|\addtocounter{page}{-1}|\\
\textit{code for banner page}\\
|\newpage|\\
|\||fi|
\end{tabular}
\end{center}
%
Here one could write a message such as:
\begin{center}
|This is the part \childdocname{} of \childdocjob{}.|
\end{center}

%%%%%%%%%%%%%%%%%%%%%%%%%%%%%%%%%%%%%%%%%%%%%%%%%%%%%%%%%%%%%%%%%%%%%%%%%%%%%%%%
\subsection{Flags}
\label{sec:flags}

The package makes it easy to generate different versions
of the main or child documents.
To this end compilation flags can be defined
and assigned different default values.
They will be particularly useful in conjunction
with the forwarding mechanism described in \secref{sec:forward}.

For example, it may be useful to have a flag |\version|
which can be set to |draft| or |final|.
The document source will contain some conditional code
depending on the value of |\version|.
Suppose further, the flag should default to |final| for the main file
and to |draft| for child files
which is a natural assignment for editing the document.
This is achieved by placing the following code
in the preamble of the main document
(below the |\childdocmain| directive):
%
\begin{center}
\begin{tabular}{l}
|\ifchilddoc|\\
|\providecommand{\version}{draft}|\\
|\||else|\\
|\providecommand{\version}{final}|\\
|\||fi|
\end{tabular}
\end{center}
%
The definition by |\providecommand| makes sure
that previous definitions are not overwritten.
Further statements |\providecommand{\version}{...}|
can thus be added before the above code to override it.

For the main file, one might add a line
(between |\childdocmain| and the above block)
%
\begin{center}
|%\ifchilddoc\||else\providecommand{\version}{draft}\||fi|
\end{center}
%
which can be uncommented to produce a draft version.
Likewise one can add a line to the very top of a child file
(above the |\childdocof{|\textit{main}|}| directive)
%
\begin{center}
|%\providecommand{\version}{final}|
\end{center}
%
which can be uncommented to produce the final version of this child document.

%%%%%%%%%%%%%%%%%%%%%%%%%%%%%%%%%%%%%%%%%%%%%%%%%%%%%%%%%%%%%%%%%%%%%%%%%%%%%%%%
\subsection{Forwarding}
\label{sec:forward}

Different versions of the main or child documents
using compilation flags as described in \secref{sec:flags}
can be (permanently) stored in different files
for convenient compilation, viewing and distribution.
To this end, the package defines a command
to pass on compilation to a different file:

%%%%%%%%%%%%%%%%%%%%%%%%%%%%%%%%%%%%%%%%
\DescribeMacro{\childdocforward}
The command |\childdocforward| redirects processing to
another source file:
%
\begin{center}
\begin{tabular}{l}
|\input{childdoc.def}|\\
|\childdocforward[|\textit{main}|]{|\textit{dest}|}|\\
\end{tabular}
\end{center}
%
The argument \textit{dest} is the destination file
(without extension).
It should be the main file or one of the child files.
Note that further \textsf{childdoc} directives
such as |\childdocof| and |\childdocforward|
in the indicated file will be processed in this form.
The optional argument \textit{main}
passes on directly to the main file \textit{main}
while pretending to compile the child \textit{dest}.
This form behaves as if \textit{dest}
issues |\childdocof{|\textit{main}|}| right away,
and no further \textsf{childdoc} directives will be processed.

%%%%%%%%%%%%%%%%%%%%%%%%%%%%%%%%%%%%%%%%
\DescribeMacro{\...prefix}
In the alternative form |\childdocforwardprefix|,
%
\begin{center}
\begin{tabular}{l}
|\input{childdoc.def}|\\
|\childdocforwardprefix[|\textit{main}|]{|\textit{prefix}|}{|\textit{dest}|}|
\end{tabular}
\end{center}
%
the destination file is determined by a pattern
depending on the current file:
To make this work, the current file must be called
`{\textit{prefix}\hspace{0.2em}\textit{suffix}}'
with \textit{prefix} matching precisely the argument.
Processing is then passed on to the file
`{\textit{dest}\hspace{0.2em}\textit{suffix}}'.
Surely, the same effect is achieved by
directly specifying the
argument `{\textit{dest}\hspace{0.2em}\textit{suffix}}'
in the first form.
However, that requires to set up a different file
for each child. With the alternative form of the command
all these files can have exactly the same content
which simplifies setting them up and maintaining them.

For example, the following file |draft.tex|
with a compilation flag |\version| as described in \secref{sec:flags}
compiles the main document as a draft:
%
\begin{center}
\begin{tabular}{l}
|\def\version{draft}|\\
|\input{childdoc.def}|\\
|\childdocforward{|\textit{main}|}|
\end{tabular}
\end{center}
%
Likewise, the following files |final|\textit{nn}|.tex|
compile the final version of the child document
|child|\textit{nn}|.tex|:
%
\begin{center}
\begin{tabular}{l}
|\def\version{final}|\\
|\input{childdoc.def}|\\
|\childdocforwardprefix{final}{child}|
\end{tabular}
\end{center}
%

Note that when several versions of a main file and/or of each child file
are to be generated, it may be convenient to set up a |Makefile| or
shell script to automatise the process.

%%%%%%%%%%%%%%%%%%%%%%%%%%%%%%%%%%%%%%%%%%%%%%%%%%%%%%%%%%%%%%%%%%%%%%%%%%%%%%%%
\subsection{Command Line Processing}
\label{sec:commandline}

The effect of redirection files can also be achieved by invoking
the \LaTeX{} compiler with a more elaborate command line.
Most conveniently this should be done as part
of a shell script or a |Makefile|.

When using \textsf{childdoc} in the main file, the following
command lines effectively perform a redirection
(note that depending on the shell being used,
backslashes may have to be doubled: `|\|' $\to$ `|\\|'):
%
\begin{center}
|... -jobname "|\textit{target}|" |\\|"|[\textit{flags}]%
|\input{childdoc.def}\childdocforward[|\textit{main}|]{|\textit{dest}|}"|
\end{center}
%
Here \textit{target} is the name of the output file,
\textit{main} is the name of the main file
and \textit{dest} is the name of the main or child file to be processed
(all filenames without extensions).
The optional argument \textit{main} can be omitted
if \textit{main} matches \textit{dest}.
Optionally, compilation \textit{flags} can be defined via |\def| commands.
This command line makes the \TeX{} engine believe
it is compiling the file \textit{target}
whose content is specified as the latter parameter.
The provided code then forwards the processing to
\textit{main} or \textit{dest} as described in \secref{sec:forward}.

%%%%%%%%%%%%%%%%%%%%%%%%%%%%%%%%%%%%%%%%%%%%%%%%%%%%%%%%%%%%%%%%%%%%%%%%%%%%%%%%
\subsection{Include by Input}
\label{sec:input}

Including child documents by |\include| has some restrictions by design.
Most notably, the content of a child document always occupies
its own set of pages; pages cannot be shared between child documents.
Usually, this behaviour makes perfect sense
because each child document contain an essential part of the document.
However, in some situations it may be desirable to compose
a document from a collection of parts
without having mandatory page breaks between then.
For this case, the package
provides a mechanism to include parts
by |\input| which can also be processed individually.
However, by construction this mechanism
requires manual handling of the content to be output.

%%%%%%%%%%%%%%%%%%%%%%%%%%%%%%%%%%%%%%%%
\DescribeMacro{\ifchilddocmanual}
The main file should be prepared as usual, see \secref{sec:include}.
However, the document body must make a distinction
between processing of an individual part and of the main document, e.g.:
%
\begin{center}
\begin{tabular}{l}
|\ifchilddocmanual|\\
|\input{\childdocname}|\\
|\||else|\\
\textit{document body with }|\input{|\textit{part}|}|\\
|\||fi|
\end{tabular}
\end{center}
%
The conditional |\ifchilddocmanual| is true whenever
a part to be included by |\input| is being compiled,
and the name of the part is stored in |\childdocname|.

%%%%%%%%%%%%%%%%%%%%%%%%%%%%%%%%%%%%%%%%
\DescribeMacro{\childdocby}
Each part to be included by |\input| should start with:
%
\begin{center}
\begin{tabular}{l}
|\input{childdoc.def}|\\
|\childdocby{|\textit{main}|}|\\
\end{tabular}
\end{center}
%
The directive |\childdocby| is similar to |\childdocof|
described in \secref{sec:include},
but the subsequent selection of content must be done manually.
To that end, both |\ifchilddoc| and |\ifchilddocmanual|
will be true upon processing of a part,
and the name of the part is stored in |\childdocname|.
Note that |\jobname| will be set to the filename of the current part
so that each part receives an individual |.aux| file
that does not interfere with the |.aux| file(s) of the main document.
This behaviour can be altered by the alternative form
|\childdocby[*]{|\textit{main}|}| (with a non-empty optional argument)
which uses the |.aux| file of the main document
by setting |\jobname| to \textit{main}.

%%%%%%%%%%%%%%%%%%%%%%%%%%%%%%%%%%%%%%%%%%%%%%%%%%%%%%%%%%%%%%%%%%%%%%%%%%%%%%%%
\subsection{Driver Development}
\label{sec:driver}

The \textsf{childdoc} mechanism can also be use for the development
of definition files such as \LaTeX{} styles or classes.
This case differs from the above setup with multiple parts
included by |\include| in that no |\includeonly| should be invoked.
This can be achieved by starting the include file
(before |\ProvidesPackage|) with:
%
\begin{center}
\begin{tabular}{l}
|\input{childdoc.def}|\\
|\childdocforward{|\textit{main}|}|\\
\end{tabular}
\end{center}
%
or alternatively with:
%
\begin{center}
\begin{tabular}{l}
|\input{childdoc.def}|\\
|\childdocby{|\textit{main}|}|\\
\end{tabular}
\end{center}
%
Both forms have slightly different effects as described above.
The main file is prepared as usual, see \secref{sec:include}.

%%%%%%%%%%%%%%%%%%%%%%%%%%%%%%%%%%%%%%%%%%%%%%%%%%%%%%%%%%%%%%%%%%%%%%%%%%%%%%%%
\subsection{Legacy Detection}
\label{sec:detection}

The directive |\childdocmain| in the main file can detect
whether the complete document or merely a child is to be compiled
even without using the directive |\childdocof|.
This method is deprecated because it is less robust
and there is no compelling reason to use it;
it is merely provided for backward compatibility
and it may be removed in future versions.

If the detection mechanism is to be used,
it is mandatory to correctly specify
the filename of the main file as the argument of |\childdocmain|:
%
\begin{center}
\begin{tabular}{l}
|\input{childdoc.def}|\\
|\childdocmain{|\textit{main}|}|\\
\end{tabular}
\end{center}
%
If |\jobname| does not match the argument \textit{main} of |\childdocmain|,
it is assumed that |\jobname| points to the child file to be compiled.
When using |\childdocmain| with the main file specified as argument,
it suffices to start a child file
with just |\input{|\textit{main}|}|
without loading of the package and using |\childdocof|.
If instead all processing is done
with the appropriate \textsf{childdoc} directives,
the argument of \textit{main} of |\childdocmain| can be empty.

An alternative version of the command line processing described
in \secref{sec:commandline} using the detection mechanism reads:
%
\begin{center}
|... -jobname "|\textit{target}|" "|[\textit{flags}]%
[|\def\jobname{|\textit{dest}|}|]|\input{|\textit{main}|}"|
\end{center}

%%%%%%%%%%%%%%%%%%%%%%%%%%%%%%%%%%%%%%%%%%%%%%%%%%%%%%%%%%%%%%%%%%%%%%%%%%%%%%%%
\subsection{Manual Code}
\label{sec:manual}

In case one cannot be certain whether the definitions file |childdoc.def|
is installed on the target \TeX{} distribution
and one prefers not to ship it,
it is conceivable to paste a few relevant commands into the sources.

To that end, drop all statements |\input{childdoc.def}|
and perform the replacements as outlined below.
Instead of |\childdocmain{|\textit{main}|}| add the following code
to the top of the main file:
%
\begin{center}
\begin{tabular}{l}
|\||ifdefined\childdocname\endinput\||fi\newif\ifchilddoc|\\
|\edef\childdocname{\scantokens\expandafter{\jobname\noexpand}}|\\
|\def\childdocmain{|\textit{main}|}\||ifx\childdocmain\childdocname\||else|\\
|\childdoctrue\includeonly{\childdocname}\let\jobname\childdocmain\||fi|\\
\end{tabular}
\end{center}
%
Instead of |\childdocof{|\textit{main}|}| just include the main file
at the top of each child file:
%
\begin{center}
|\input{|\textit{main}|}|
\end{center}
%
A simple redirection |\childdocforward{|\textit{dest}|}| is achieved by:
%
\begin{center}
|\def\jobname{|\textit{dest}|}\input{\jobname}|
\end{center}
%
The redirection with prefix
|\childdocforwardprefix[|\textit{prefix}|]{|\textit{dest}|}|
is accomplished by:
%
\begin{center}
\begin{tabular}{l}
|{\edef\jobname{\scantokens\expandafter{\jobname\noexpand}}|\\
|\def\redirectjob |\textit{prefix}|#1~~~{\gdef\jobname{|\textit{dest}|#1}}|\\
|\expandafter\redirectjob\jobname~~~}\input{\jobname}|
\end{tabular}
\end{center}

In an alternative approach,
child documents can be compiled by a specific command line
without additional code or specific definitions:
%
\begin{center}
|... -jobname "|\textit{target}|" "|[\textit{flags}]%
|\includeonly{|\textit{dest}|}\input{|\textit{main}|}"|
\end{center}
%

%%%%%%%%%%%%%%%%%%%%%%%%%%%%%%%%%%%%%%%%%%%%%%%%%%%%%%%%%%%%%%%%%%%%%%%%%%%%%%%%
%%%%%%%%%%%%%%%%%%%%%%%%%%%%%%%%%%%%%%%%%%%%%%%%%%%%%%%%%%%%%%%%%%%%%%%%%%%%%%%%
\section{Information}

%%%%%%%%%%%%%%%%%%%%%%%%%%%%%%%%%%%%%%%%%%%%%%%%%%%%%%%%%%%%%%%%%%%%%%%%%%%%%%%%
\subsection{Copyright}

Copyright \copyright{} 2017--2018 Niklas Beisert

This work may be distributed and/or modified under the
conditions of the \LaTeX{} Project Public License, either version 1.3
of this license or (at your option) any later version.
The latest version of this license is in
  \url{http://www.latex-project.org/lppl.txt}
and version 1.3 or later is part of all distributions of \LaTeX{}
version 2005/12/01 or later.

This work has the LPPL maintenance status `maintained'.

The Current Maintainer of this work is Niklas Beisert.

This work consists of the files |README.txt|, |childdoc.ins| and |childdoc.dtx|
as well as the derived files |childdoc.def|, |cdocsamp.tex|
with |cdocsch1.tex|, |cdocsch2.tex|, |cdocspt3.tex|, |cdocspt4.tex|,
|cdocsdrf.tex|, |cdocsfn1.tex|, |cdocsfn2.tex|
as well as |childdoc.pdf|.

%%%%%%%%%%%%%%%%%%%%%%%%%%%%%%%%%%%%%%%%%%%%%%%%%%%%%%%%%%%%%%%%%%%%%%%%%%%%%%%%
\subsection{Files and Installation}

The package consists of the files:
%
\begin{center}
\begin{tabular}{ll}
    |README.txt|   & readme file \\
    |childdoc.ins| & installation file \\
    |childdoc.dtx| & source file \\
    |childdoc.def| & definition file \\
    |cdocsamp.tex| & sample main file \\
    |cdocsch1.tex| & sample include file \\
    |cdocsch2.tex| & sample include file \\
    |cdocspt3.tex| & sample part file \\
    |cdocspt4.tex| & sample part file \\
    |cdocsdrf.tex| & sample redirection file \\
    |cdocsfn1.tex| & sample redirection file \\
    |cdocsfn2.tex| & sample redirection file \\
    |childdoc.pdf| & manual
\end{tabular}
\end{center}
%
The distribution consists of the files
|README.txt|, |childdoc.ins| and |childdoc.dtx|.
%
\begin{itemize}
\item
Run (pdf)\LaTeX{} on |childdoc.dtx|
to compile the manual |childdoc.pdf| (this file).
\item
Run \LaTeX{} on |childdoc.ins| to create the definitions file |childdoc.def|
and the sample |cdocsamp.tex| with include files
|cdocsch1.tex|, |cdocsch2.tex|, |cdocspt3.tex|, |cdocspt4.tex|,
|cdocsdrf.tex|, |cdocsfn1.tex|, |cdocsfn2.tex|.
Then copy the file |childdoc.def| to an appropriate directory of your \LaTeX{}
distribution, e.g.\ \textit{texmf-root}|/tex/latex/childdoc|.
\end{itemize}

%%%%%%%%%%%%%%%%%%%%%%%%%%%%%%%%%%%%%%%%%%%%%%%%%%%%%%%%%%%%%%%%%%%%%%%%%%%%%%%%
\subsection{Related CTAN Packages}

There are several other packages which offer a similar functionality:
%
\begin{itemize}
\item
The packages
\href{http://ctan.org/pkg/docmute}{\textsf{docmute}},
\href{http://ctan.org/pkg/includex}{\textsf{includex}} and
\href{http://ctan.org/pkg/standalone}{\textsf{standalone}}
provide commands to include only the document body of
a child file thus allowing both files to be compiled individually.
\item
The packages \href{http://ctan.org/pkg/subdocs}{\textsf{subdocs}}
and \href{http://ctan.org/pkg/subfiles}{\textsf{subfiles}}
provide structures in which the main and child documents can be
encapsulated and allowing them to be compiled individually.
The inclusion mechanism is different from the conventional |\include|.
\item
The package \href{http://ctan.org/pkg/combine}{\textsf{combine}}
is an elaborate solution to combine several documents into one.
\end{itemize}
%
See also the CTAN topic \href{http://ctan.org/topic/subdocs}{\textsf{subdocs}}
for further related packages.
The present package differs from the above solutions in that
a document structure constructed with the conventional |\include| mechanism
just needs two extra commands at the top of every file
such that all constituent files can be compiled individually.

%%%%%%%%%%%%%%%%%%%%%%%%%%%%%%%%%%%%%%%%%%%%%%%%%%%%%%%%%%%%%%%%%%%%%%%%%%%%%%%%
%\subsection{Feature Suggestions}
%
%The following is a list of features which may be useful for future
%versions of this package:
%%
%\begin{itemize}
%\item
%\ldots
%\end{itemize}

%%%%%%%%%%%%%%%%%%%%%%%%%%%%%%%%%%%%%%%%%%%%%%%%%%%%%%%%%%%%%%%%%%%%%%%%%%%%%%%%
\subsection{Revision History}

%%%%%%%%%%%%%%%%%%%%%%%%%%%%%%%%%%%%%%%%
\paragraph{v2.0:} 2018/12/30

\begin{itemize}
\item
immediate forward processing
\item
added |\childdocby| mechanism
\item
manual restructured
\end{itemize}

%%%%%%%%%%%%%%%%%%%%%%%%%%%%%%%%%%%%%%%%
\paragraph{v1.6:} 2018/01/17

\begin{itemize}
\item
application for development of include files
\item
corrections to manual
\end{itemize}

%%%%%%%%%%%%%%%%%%%%%%%%%%%%%%%%%%%%%%%%
\paragraph{v1.5:} 2017/05/21

\begin{itemize}
\item
more complete structuring introduced
\item
|\childdocof| introduced
\item
|\childdoc| renamed to |\childdocmain|
\item
|\childredirect| renamed to |\childdocforward| and |\childdocforwardprefix|
and functionality expanded
\end{itemize}

%%%%%%%%%%%%%%%%%%%%%%%%%%%%%%%%%%%%%%%%
\paragraph{v1.0:} 2017/04/27

\begin{itemize}
\item
manual and install package
\item
first version published on CTAN
\end{itemize}

%%%%%%%%%%%%%%%%%%%%%%%%%%%%%%%%%%%%%%%%
\paragraph{v0.6:} 2017/04/26

\begin{itemize}
\item
redirection mechanism added
\end{itemize}

%%%%%%%%%%%%%%%%%%%%%%%%%%%%%%%%%%%%%%%%
\paragraph{v0.5:} 2017/04/26

\begin{itemize}
\item
functionality in definition file
\end{itemize}


%%%%%%%%%%%%%%%%%%%%%%%%%%%%%%%%%%%%%%%%%%%%%%%%%%%%%%%%%%%%%%%%%%%%%%%%%%%%%%%%
%%%%%%%%%%%%%%%%%%%%%%%%%%%%%%%%%%%%%%%%%%%%%%%%%%%%%%%%%%%%%%%%%%%%%%%%%%%%%%%%
%%%%%%%%%%%%%%%%%%%%%%%%%%%%%%%%%%%%%%%%%%%%%%%%%%%%%%%%%%%%%%%%%%%%%%%%%%%%%%%%
\appendix

\settowidth\MacroIndent{\rmfamily\scriptsize 000\ }

 \DocInput{childdoc.dtx}

\end{document}
%</driver>
% \fi
%
% %%%%%%%%%%%%%%%%%%%%%%%%%%%%%%%%%%%%%%%%%%%%%%%%%%%%%%%%%%%%%%%%%%%%%%%%%%%%%%
% %%%%%%%%%%%%%%%%%%%%%%%%%%%%%%%%%%%%%%%%%%%%%%%%%%%%%%%%%%%%%%%%%%%%%%%%%%%%%%
% \section{Sample}
%\iffalse
%<*samplemain>
%\fi
%
% The following presents a sample document
% with two chapters, two parts, a title page,
% a compile flag as well as three forwarding files to set the flag.
% It consists of eight |.tex| files:
% \begin{center}
% \begin{tabular}{ll}
% |cdocsamp.tex|&main file\\
% |cdocsch1.tex|&include file for chapter 1\\
% |cdocsch2.tex|&include file for chapter 2\\
% |cdocspt3.tex|&include file for part 3\\
% |cdocspt4.tex|&include file for part 4\\
% |cdocsdrf.tex|&forwarding file for main file in draft mode\\
% |cdocsfi1.tex|&forwarding file for final version of chapter 1\\
% |cdocsfi2.tex|&forwarding file for final version of chapter 2\\
% \end{tabular}
% \end{center}
% Each of the eight files can be compiled directly by the \LaTeX{} compiler.
%
% %%%%%%%%%%%%%%%%%%%%%%%%%%%%%%%%%%%%%%
% \paragraph{Main File.}
%
% The main file is called |cdocsamp.tex|.
%
% Load the \textsf{childdoc} definitions and
% declare the filename for the main document:
%    \begin{macrocode}
\input{childdoc.def}
\childdocmain{}
%    \end{macrocode}

% Optional override for |\version| flag:
%    \begin{macrocode}
%%\ifchilddoc\else\providecommand{\version}{draft}\fi
%    \end{macrocode}

% Define the default values for the |\version| flag
% (|final| for the main file and |draft| for childs):
%    \begin{macrocode}
\ifchilddoc
\providecommand{\version}{draft}
\else
\providecommand{\version}{final}
\fi
%    \end{macrocode}

% Load the standard document class:
%    \begin{macrocode}
\documentclass[12pt]{article}
%    \end{macrocode}

% Start the document body:
%    \begin{macrocode}
\begin{document}
%    \end{macrocode}

% Declare a title page.
% Print title, part of document being processed and version flag:
%    \begin{macrocode}
\addtocounter{page}{-1}
\begin{center}
{\LARGE\bfseries{}childdoc example\par}
\vspace{1cm}
\ifchilddoc
\ifchilddocmanual part\else chapter\fi:
`\childdocname' of `\childdocjob'\par
\else
main document: `\childdocjob'\par
\fi
version: \version\par
\end{center}
\newpage
%    \end{macrocode}

% Manually include selected file,
% otherwise process as usual:
%    \begin{macrocode}
\ifchilddocmanual
\section*{part `\childdocname'}
\input{\childdocname}
\else
%    \end{macrocode}

% Include the two chapters:
%    \begin{macrocode}
\include{cdocsch1}
\include{cdocsch2}
%    \end{macrocode}

% Include the two parts unless only chapters should be displayed:
%    \begin{macrocode}
\ifchilddoc\else
\section{part three}
\input{cdocspt3}
\section{part four}
\input{cdocspt4}
\fi
%    \end{macrocode}

% Process as usual until here:
%    \begin{macrocode}
\fi
%    \end{macrocode}

% End of document body:
%    \begin{macrocode}
\end{document}
%    \end{macrocode}
%\iffalse
%</samplemain>
%\fi
%
% %%%%%%%%%%%%%%%%%%%%%%%%%%%%%%%%%%%%%%
% \paragraph{Chapter Include Files.}
%
% The include files are called |cdocsch1.tex| and |cdocsch2.tex|.
%
%\iffalse
%<*samplechap1|samplechap2>
%\fi

% Optional override for |\version| flag:
%    \begin{macrocode}
%%\providecommand{\version}{final}
%    \end{macrocode}

% Include the main document:
%    \begin{macrocode}
\input{childdoc.def}
\childdocof{cdocsamp}
%    \end{macrocode}

%\iffalse
%</samplechap1|samplechap2>
%\fi
%
%\iffalse
%<*samplechap1>
%\fi
% Some text for chapter 1:
%    \begin{macrocode}
\section{one}
some text in chapter one
%    \end{macrocode}

%\iffalse
%</samplechap1>
%\fi
% Some text for chapter 2:
%\iffalse
%<*samplechap2>
%\fi
%    \begin{macrocode}
\section{two}
more text in chapter two
%    \end{macrocode}

%\iffalse
%</samplechap2>
%\fi
%
% %%%%%%%%%%%%%%%%%%%%%%%%%%%%%%%%%%%%%%
% \paragraph{Part Include Files.}
%
% The include files are called |cdocspt3.tex| and |cdocspt4.tex|.
%
%\iffalse
%<*samplepart3|samplepart4>
%\fi

% Optional override for |\version| flag:
%    \begin{macrocode}
%%\providecommand{\version}{final}
%    \end{macrocode}

% Include the main document:
%    \begin{macrocode}
\input{childdoc.def}
\childdocby{cdocsamp}
%    \end{macrocode}

%\iffalse
%</samplepart3|samplepart4>
%\fi
%
%\iffalse
%<*samplepart3>
%\fi
% Some text for part 3:
%    \begin{macrocode}
some text in part three
%    \end{macrocode}

%\iffalse
%</samplepart3>
%\fi
% Some text for part 4:
%\iffalse
%<*samplepart4>
%\fi
%    \begin{macrocode}
more text in part four
%    \end{macrocode}

%\iffalse
%</samplepart4>
%\fi
%
% %%%%%%%%%%%%%%%%%%%%%%%%%%%%%%%%%%%%%%
% \paragraph{Forwarding for a Complete Draft.}
%
% The following forwarding file |cdocsdrf.tex|
% compiles the main document in draft mode:
%\iffalse
%<*sampledraft>
%\fi
%    \begin{macrocode}
\def\version{draft}
\input{childdoc.def}
\childdocforward{cdocsamp}
%    \end{macrocode}

%\iffalse
%</sampledraft>
%\fi
%
% %%%%%%%%%%%%%%%%%%%%%%%%%%%%%%%%%%%%%%
% \paragraph{Forwarding for Final Version of the Chapters.}
%
% The following forwarding files |cdocsfn1.tex| and |cdocsfn2.tex|
% (with identical content)
% compile the final versions of the child documents
% |cdocsch1.tex| and |cdocsch2.tex|, respectively:
%\iffalse
%<*samplefinal>
%\fi
%    \begin{macrocode}
\def\version{final}
\input{childdoc.def}
\childdocforwardprefix[cdocsamp]{cdocsfn}{cdocsch}
%    \end{macrocode}

%\iffalse
%</samplefinal>
%\fi
%
% %%%%%%%%%%%%%%%%%%%%%%%%%%%%%%%%%%%%%%
% \paragraph{Command Line Processing.}
%
% The following three command lines generate the output files
% |cdocscld|, |cdocscl1| and |cdocscl2|
% which should be identical to
% |cdocsdrf|, |cdocsch1| and |cdocsfn2|, respectively:
% \begin{center}
% \begin{tabular}{l}
% |latex -jobname cdocscld \|\\
% |  "\def\version{draft}\input{childdoc.def}\childdocforward{cdocsamp}"|\\
% |latex -jobname cdocscl1 \|\\
% |  "\input{childdoc.def}\childdocforward[cdocsamp]{cdocsch1}"|\\
% |latex -jobname cdocscl2 \|\\
% |  "\def\version{final}\input{childdoc.def}\childdocforward{cdocsch2}"|
% \end{tabular}
% \end{center}
% Note that the trailing backslash on each first line
% merely continues the input to the second line
% (for convenient cut ant paste).
% Furthermore, the command |latex| can be replaced by any
% of its alternative versions such as |pdflatex|.
%
% %%%%%%%%%%%%%%%%%%%%%%%%%%%%%%%%%%%%%%%%%%%%%%%%%%%%%%%%%%%%%%%%%%%%%%%%%%%%%%
% %%%%%%%%%%%%%%%%%%%%%%%%%%%%%%%%%%%%%%%%%%%%%%%%%%%%%%%%%%%%%%%%%%%%%%%%%%%%%%
% \section{Implementation}
%\iffalse
%<*package>
%\fi
%
% This section describes the definitions file |childdoc.def|.

% The definitions cannot be loaded using |\usepackage| or |\RequirePackage|
% which has a mechanism to prevent loading a style file more than once.
% When loading the definitions by means of |\input|
% multiple instances have to be prevented manually:
%\iffalse
%This code needs to be before the `\ProvidesFile' directive
%which is defined at the beginning of this file.
%Therefore it is also placed there and commented out here.
%</package>
%<*discard>
%\fi
%    \begin{macrocode}
\ifdefined\childdocmain\endinput\fi
%    \end{macrocode}
%\iffalse
%</discard>
%<*package>
%\fi
%
% \macro{\ifchilddoc}
% \macro{\ifchilddocmanual}
% The conditional |\ifchilddoc| tells whether a
% child (true) or main (false) document is being compiled.
% The conditional |\ifchilddocmanual| tells whether
% the |\includeonly| mechanism is used (false) or
% the selection of child files must be performed manually (true).
% The definitions initialise to false:
%    \begin{macrocode}
\newif\ifchilddoc
\newif\ifchilddocmanual
%    \end{macrocode}

% \macro{\childdocname}
% \macro{\childdocjob}
% The macro |\childdocname| stores the name of the main document
% to be compiled. The macro |\childdocjob| stores the name of
% the document on which the \LaTeX{} compiler was originally invoked.
% The content of |\jobname| cannot be compared
% to filenames specified in the source due to different catcodes.
% The following code rescans |\jobname|, stores the result
% in |\childdocname| and saves a copy in |\childdocjob|:
%    \begin{macrocode}
\edef\childdocname{\scantokens\expandafter{\jobname\noexpand}}
\let\childdocjob\childdocname
%    \end{macrocode}

% \macro{\childdocdisable}
% The macro |\childdocdisable| prevents the main file
% from being processed more than once.
% At this stage, the main document command |\childdocmain|
% is assumed to be called once again where it should do nothing.
% Any subsequent call to it should prevent
% a secondary processing of the main document
% It overwrites the forwarding commands
% |\childdocof| and |\childdocforward|
% with empty macros to prevent further inclusions of the main document:
%    \begin{macrocode}
\newcommand{\childdocdisable}
{
  \renewcommand{\childdocmain}[1]{\renewcommand{\childdocmain}[1]{\endinput}}
  \renewcommand{\childdocof}[1]{}
  \renewcommand{\childdocby}[2][]{}
  \renewcommand{\childdocforward}[2][]{}
  \renewcommand{\childdocdisable}{}
}
%    \end{macrocode}

% \macro{\childdocmain}
% The macro |\childdocmain| is to be called at the top of the main file
% with nothing or the main filename (without extension) as argument.
% First, it breaks loops.
% If the argument is not empty and does not match |\childdocname|
% (which is set by the first inclusion of |childdoc.def|),
% |\ifchilddoc| is set to true, |\includeonly| is applied to the child file
% and |\jobname| is set to the main file
% (for proper handling of |.aux| files):
%    \begin{macrocode}
\newcommand{\childdocmain}[1]
{
  \childdocdisable\childdocmain{}
  \if?#1?\else
    \begingroup
      \def\childdoctmp{#1}
      \ifx\childdoctmp\childdocname
        \def\childdoctmp{}
      \else
        \def\childdoctmp
        {
          \childdoctrue
          \includeonly{\childdocname}
          \def\childdocjob{#1}
          \def\jobname{#1}
        }
      \fi
      \expandafter
    \endgroup
    \childdoctmp
  \fi
}
%    \end{macrocode}

% \macro{\childdocof}
% The command |\childdocof| redirects
% compilation to the main file |#1|.
%    \begin{macrocode}
\newcommand{\childdocof}[1]
{
  \childdocdisable
  \childdoctrue
  \includeonly{\childdocname}
  \def\jobname{#1}
  \def\childdocjob{#1}
  \input{#1}
}
%    \end{macrocode}

% \macro{\childdocby}
% The command |\childdocby| ....
%    \begin{macrocode}
\newcommand{\childdocby}[2][]
{
  \childdocdisable
  \childdoctrue
  \childdocmanualtrue
  \if?#1?\else
    \def\jobname{#2}
  \fi
  \def\childdocjob{#2}
  \input{#2}
  \endinput
}
%    \end{macrocode}

% \macro{\childdocforward}
% The command |\childdocforward| redirects
% compilation to the main file or
% (if the optional argument is given) a child file.
% Parameters are set as if the main file
% or a child file starting with |\childdocof| was compiled.
% Then compilation is handed over to the main file:
%    \begin{macrocode}
\newcommand{\childdocforward}[2][]
{
  \begingroup
    \if?#1?
      \def\childdoctmp
      {
        \def\childdocname{#2}
        \def\childdocjob{#2}
        \def\jobname{#2}
        \input{#2}
        \endinput
      }
    \else
      \def\childdoctmp
      {
        \childdocdisable
        \def\childdocname{#2}
        \childdoctrue
        \includeonly{#2}
        \def\childdocjob{#1}
        \def\jobname{#1}
        \input{#1}
        \endinput
      }
    \fi
    \expandafter
  \endgroup
  \childdoctmp
}
%    \end{macrocode}

% \macro{\childdocforwardprefix}
% The command |\childdocforwardprefix| redirects
% compilation to the main or a child file by means of a pattern.
% The prefix |#1| in the current filename is replaced by |#2|
% and the suffix of the current filename is kept
% (it is assumed that the filename does not contain the substring `|~~~|'
% which is used as a delimiter).
% Compilation is handed over to the new file by |\childdocforward|:
%    \begin{macrocode}
\newcommand{\childdocforwardprefix}[3][]
{
  \begingroup
    \def\childdocextract #2##1~~~{\def\childdoctmp{\childdocforward[#1]{#3##1}}}
    \expandafter\childdocextract\childdocname~~~
    \expandafter
  \endgroup
  \childdoctmp
}
%    \end{macrocode}

% \macro{\childdoc}
% The deprecated macro |\childdoc| is a legacy version of |\childdocmain|:
%    \begin{macrocode}
\newcommand{\childdoc}{\childdocmain}
%    \end{macrocode}

% \macro{\childdocredirect}
% The deprecated macro |\childdocredirect| is a legacy version
% of |\childdocforward| and |\childdocforwardprefix|:
%    \begin{macrocode}
\newcommand{\childdocredirect}[2][]
{
  \begingroup
    \if?#1?
      \def\childdoctmp{\childdocforward{#2}}
    \else
      \def\childdoctmp{\childdocforwardprefix{#1}{#2}}
    \fi
    \expandafter
  \endgroup
  \childdoctmp
}
%    \end{macrocode}

%\iffalse
%</package>
%\fi
%
\endinput

\childdocforwardprefix[cdocsamp]{cdocsfn}{cdocsch}
%    \end{macrocode}

%\iffalse
%</samplefinal>
%\fi
%
% %%%%%%%%%%%%%%%%%%%%%%%%%%%%%%%%%%%%%%
% \paragraph{Command Line Processing.}
%
% The following three command lines generate the output files
% |cdocscld|, |cdocscl1| and |cdocscl2|
% which should be identical to
% |cdocsdrf|, |cdocsch1| and |cdocsfn2|, respectively:
% \begin{center}
% \begin{tabular}{l}
% |latex -jobname cdocscld \|\\
% |  "\def\version{draft}% \iffalse
%
% childdoc.dtx Copyright (C) 2017-2018 Niklas Beisert
%
% This work may be distributed and/or modified under the
% conditions of the LaTeX Project Public License, either version 1.3
% of this license or (at your option) any later version.
% The latest version of this license is in
%   http://www.latex-project.org/lppl.txt
% and version 1.3 or later is part of all distributions of LaTeX
% version 2005/12/01 or later.
%
% This work has the LPPL maintenance status `maintained'.
%
% The Current Maintainer of this work is Niklas Beisert.
%
% This work consists of the files childdoc.dtx and childdoc.ins
% and the derived files childdoc.def and cdocsamp.tex with
% cdocsch1.tex, cdocsch2.tex, cdocsdrf.tex, cdocsfn1.tex, cdocsfn2.tex.
%
%<package>\ifdefined\childdocmain\endinput\fi
%<package>\ProvidesFile{childdoc.def}[2018/12/30 v2.0 child document driver]
%<samplemain>\ProvidesFile{cdocsamp.tex}[2018/12/30 v2.0 sample for childdoc]
%<*driver>
%\ProvidesFile{childdoc.drv}[2018/12/30 v2.0 childdoc reference manual file]
\PassOptionsToClass{10pt,a4paper}{article}
\documentclass{ltxdoc}

\usepackage[margin=35mm]{geometry}
\usepackage{hyperref}
\usepackage{hyperxmp}
\usepackage[usenames]{color}

\hypersetup{colorlinks=true}
\hypersetup{pdfstartview=FitH}
\hypersetup{pdfpagemode=UseNone}
\hypersetup{pdfsource={}}
\hypersetup{pdflang={en-UK}}
\hypersetup{pdfcopyright={Copyright 2017-2018 Niklas Beisert.
  This work may be distributed and/or modified under the
  conditions of the LaTeX Project Public License, either version 1.3
  of this license or (at your option) any later version.}}
\hypersetup{pdflicenseurl={http://www.latex-project.org/lppl.txt}}
\hypersetup{pdfcontactaddress={ETH Zurich, ITP, HIT K,
  Wolfgang-Pauli-Strasse 27}}
\hypersetup{pdfcontactpostcode={8093}}
\hypersetup{pdfcontactcity={Zurich}}
\hypersetup{pdfcontactcountry={Switzerland}}
\hypersetup{pdfcontactemail={nbeisert@itp.phys.ethz.ch}}
\hypersetup{pdfcontacturl={http://people.phys.ethz.ch/\xmptilde nbeisert/}}

\newcommand{\secref}[1]{\hyperref[#1]{section \ref*{#1}}}

\parskip1ex
\parindent0pt
\let\olditemize\itemize
\def\itemize{\olditemize\parskip0pt}

\begin{document}

\title{The \textsf{childdoc} Package}
\hypersetup{pdftitle={The childdoc Package}}
\author{Niklas Beisert\\[2ex]
  Institut f\"ur Theoretische Physik\\
  Eidgen\"ossische Technische Hochschule Z\"urich\\
  Wolfgang-Pauli-Strasse 27, 8093 Z\"urich, Switzerland\\[1ex]
  \href{mailto:nbeisert@itp.phys.ethz.ch}
  {\texttt{nbeisert@itp.phys.ethz.ch}}}
\hypersetup{pdfauthor={Niklas Beisert}}
\hypersetup{pdfsubject={Manual for the LaTeX2e Package childdoc}}
\date{30 December 2018, \textsf{v2.0}}
\maketitle

\begin{abstract}\noindent
\textsf{childdoc} is a \LaTeXe{} package
that enables the direct compilation
of document sections included by |\include|
to individual files.
\end{abstract}

\begingroup
\parskip0ex
\tableofcontents
\endgroup

%%%%%%%%%%%%%%%%%%%%%%%%%%%%%%%%%%%%%%%%%%%%%%%%%%%%%%%%%%%%%%%%%%%%%%%%%%%%%%%%
%%%%%%%%%%%%%%%%%%%%%%%%%%%%%%%%%%%%%%%%%%%%%%%%%%%%%%%%%%%%%%%%%%%%%%%%%%%%%%%%
\section{Introduction}

\LaTeX{} provides a mechanism to structure a large document (such as a book)
into a main file and several child files (containing the chapters)
using the |\include| command.
This mechanism is beneficial for documents
which span hundreds of pages in order to
make the source file(s) more manageable.
Moreover, compilation can be restricted to
selected child files by means of the |\includeonly| command.
The latter feature can be used to reduce the compilation time while editing
(this was significantly more useful in the earlier days of \LaTeX{})
or to generate a smaller document which is easier to navigate.
Another application of |\includeonly| is to generate
documents consisting of selected parts of the complete document.

However, there are a few drawbacks of the plain |\include| mechanism:
\begin{itemize}
\item
The child files cannot be compiled on their own,
they can only be compiled via the main file.
A naive editing environment
(such as a text editor with an option
to have the current file processed by \LaTeX)
may require one to switch to the main file before compiling;
attempting to compile the child file produces errors.
\item
The main file must be modified (each time)
to adjust the |\includeonly| command
to the present needs. This easily leaves the main file in a messy state.
\item
The generated document will always carry the filename
of the main document. This is inconvenient if
several child files are to be compiled and
to be kept for distribution.
\end{itemize}

The present package provides a simple interface
to make child files individually compilable by \LaTeX{}.
Compiling a child file then has the same effect as compiling
the main file with an |\includeonly| command
to select the appropriate child.
Moreover the generated document will carry the name of the child
rather than the main file.
This resolves all three above issues.

This feature is meant to make the editing of books,
thesis documents and lecture notes somewhat more convenient.
However, the package can also be used efficiently for
composing a series of documents (such as exercise sheets)
which are typically distributed individually.
It then assists the author in generating the individual documents
(potentially in different versions)
as well as a document containing the collected series.
Another application is in developing style files
or other kinds of included material
where compilation of the style file could redirect
to a sample or test file.

%%%%%%%%%%%%%%%%%%%%%%%%%%%%%%%%%%%%%%%%%%%%%%%%%%%%%%%%%%%%%%%%%%%%%%%%%%%%%%%%
%%%%%%%%%%%%%%%%%%%%%%%%%%%%%%%%%%%%%%%%%%%%%%%%%%%%%%%%%%%%%%%%%%%%%%%%%%%%%%%%
\section{Usage}

First of all, the package \textsf{childdoc} is \emph{not} a standard
\LaTeXe{} |.sty| style file! Therefore it needs to be invoked in
a non-standard way.

%%%%%%%%%%%%%%%%%%%%%%%%%%%%%%%%%%%%%%%%%%%%%%%%%%%%%%%%%%%%%%%%%%%%%%%%%%%%%%%%
\subsection{Included Files}
\label{sec:include}

%%%%%%%%%%%%%%%%%%%%%%%%%%%%%%%%%%%%%%%%
\DescribeMacro{\childdocmain}
To use the package, add the commands
\begin{center}
\begin{tabular}{l}
|\input{childdoc.def}|\\
|\childdocmain{}|\\
\end{tabular}
\end{center}
at the very top of the main \LaTeX{} file,
in particular \emph{before} the |\documentclass| statement!
The argument of |\childdocmain| should be left empty
(but it must be present).

%%%%%%%%%%%%%%%%%%%%%%%%%%%%%%%%%%%%%%%%
\DescribeMacro{\childdocof}
Furthermore, add the commands
\begin{center}
\begin{tabular}{l}
|\input{childdoc.def}|\\
|\childdocof{|\textit{main}|}|\\
\end{tabular}
\end{center}
at the top of every child file \textit{child}
which is included by |\include{|\textit{child}|}|
from within the main file
(or at least for those files to be compiled individually).
The argument \textit{main} must be the filename of the main file.

There are a couple of
considerations in setting up the main and child documents:

%%%%%%%%%%%%%%%%%%%%%%%%%%%%%%%%%%%%%%%%
\paragraph{Restrictions.}

Please note the following restrictions:
\begin{itemize}
\item
|\childdocmain| must be called with one argument \textit{main}
to ensure compatibility with earlier version of the package.
It must either be empty (|\childdocmain{}|)
or precisely match the filename of the main file in which it is specified.
See \secref{sec:detection} for further information.
\item
The filename \textit{main} must be specified without the |.tex| extension.
\item
The filename \textit{main} is case sensitive
(even in case-insensitive file systems)
due to internal string comparison.
\item
The argument \textit{main} should be fully expanded, it cannot be a macro.
\item
Subdirectories and special characters should be avoided in filenames.
\item
The command |\childdocmain{|\textit{main}|}| must be followed by a whitespace.
It should not be followed immediately by another command
or by a comment mark `|%|'.
This is because the \TeX{} parser reads the token immediately following
the argument of |\childdocmain| and puts it
at the beginning of every child section;
however, a white\-space is ignored.
\end{itemize}

%%%%%%%%%%%%%%%%%%%%%%%%%%%%%%%%%%%%%%%%
\paragraph{Content of Main File.}

It is advisable to place all content in the child files included by |\include|.
Any output contained in the main file will appear in all child documents
unless suppressed manually;
it cannot be suppressed automatically by the |\includeonly| directive
and thus should normally be avoided.
A method to include some content in the main file
by means of conditional processing is described in \secref{sec:conditional}.

%%%%%%%%%%%%%%%%%%%%%%%%%%%%%%%%%%%%%%%%
\paragraph{Page Numbering.}

When only a part of the document is compiled,
the appropriate numbering of pages
(as well as other status parameters)
is determined from the |.aux| files.
The latter contain information from previous passes.
However this information needs to propagate through
all intermediate child documents.
Therefore the page numbering in child documents may well
be inconsistent until the complete document is compiled at least once.

A useful (if unconventional) way to always ensure a consistent
page numbering is to restart the numbering in each child document
and denote the pages by `\textit{child}|.|\textit{page}'
where \textit{child} represents the chapter/section number of the child file.
This can be achieved by the command
|\numberwithin{page}{|\textit{child}|}|
of the \textsf{amsmath} package
where \textit{child} can be |chapter| or |section|
depending on the chosen structuring.
Alternatively, one can modify the macro |\thepage| appropriately
and reset the counter |page| at the start of each child file.

%%%%%%%%%%%%%%%%%%%%%%%%%%%%%%%%%%%%%%%%%%%%%%%%%%%%%%%%%%%%%%%%%%%%%%%%%%%%%%%%
\subsection{Conditional Processing}
\label{sec:conditional}

The package provides a mechanism to compile different versions
of a document. To customise the versions further some conditional processing
can come in handy to distinguish which version is being compiled.
The package provides two macros to describe the compilation context:

%%%%%%%%%%%%%%%%%%%%%%%%%%%%%%%%%%%%%%%%
\DescribeMacro{\ifchilddoc}
The conditional |\ifchilddoc| distinguishes between the compilation of
child documents and the main document:
%
\begin{center}
|\ifchilddoc |\textit{child-code}| |[|\||else |\textit{main-code}]| \||fi|
\end{center}

%%%%%%%%%%%%%%%%%%%%%%%%%%%%%%%%%%%%%%%%
\DescribeMacro{\childdocname}
\DescribeMacro{\childdocjob}
The macro |\childdocname| contains the filename (without extension)
of the main or child file being processed.
Note that |\childdocjob| will always contain the name of the main file.

%%%%%%%%%%%%%%%%%%%%%%%%%%%%%%%%%%%%%%%%
\paragraph{Title Page.}

Conditional processing can be used to include a title or banner page
in the main document when proper precautions are taken.
Importantly, the code in the main file should ensure that the page counter
(as well as other status parameters which are stored in the |.aux| files)
takes the same value after the conditional processing.
Otherwise the page numbers may take divergent values
depending on which part is compiled.

For example, a title page could be declared by:
%
\begin{center}
\begin{tabular}{l}
|\ifchilddoc\||else|\\
|\addtocounter{page}{-1}|\\
\textit{code for title page}\\
|\newpage|\\
|\||fi|
\end{tabular}
\end{center}
%
A banner page for the child documents can be generated by:
%
\begin{center}
\begin{tabular}{l}
|\ifchilddoc|\\
|\addtocounter{page}{-1}|\\
\textit{code for banner page}\\
|\newpage|\\
|\||fi|
\end{tabular}
\end{center}
%
Here one could write a message such as:
\begin{center}
|This is the part \childdocname{} of \childdocjob{}.|
\end{center}

%%%%%%%%%%%%%%%%%%%%%%%%%%%%%%%%%%%%%%%%%%%%%%%%%%%%%%%%%%%%%%%%%%%%%%%%%%%%%%%%
\subsection{Flags}
\label{sec:flags}

The package makes it easy to generate different versions
of the main or child documents.
To this end compilation flags can be defined
and assigned different default values.
They will be particularly useful in conjunction
with the forwarding mechanism described in \secref{sec:forward}.

For example, it may be useful to have a flag |\version|
which can be set to |draft| or |final|.
The document source will contain some conditional code
depending on the value of |\version|.
Suppose further, the flag should default to |final| for the main file
and to |draft| for child files
which is a natural assignment for editing the document.
This is achieved by placing the following code
in the preamble of the main document
(below the |\childdocmain| directive):
%
\begin{center}
\begin{tabular}{l}
|\ifchilddoc|\\
|\providecommand{\version}{draft}|\\
|\||else|\\
|\providecommand{\version}{final}|\\
|\||fi|
\end{tabular}
\end{center}
%
The definition by |\providecommand| makes sure
that previous definitions are not overwritten.
Further statements |\providecommand{\version}{...}|
can thus be added before the above code to override it.

For the main file, one might add a line
(between |\childdocmain| and the above block)
%
\begin{center}
|%\ifchilddoc\||else\providecommand{\version}{draft}\||fi|
\end{center}
%
which can be uncommented to produce a draft version.
Likewise one can add a line to the very top of a child file
(above the |\childdocof{|\textit{main}|}| directive)
%
\begin{center}
|%\providecommand{\version}{final}|
\end{center}
%
which can be uncommented to produce the final version of this child document.

%%%%%%%%%%%%%%%%%%%%%%%%%%%%%%%%%%%%%%%%%%%%%%%%%%%%%%%%%%%%%%%%%%%%%%%%%%%%%%%%
\subsection{Forwarding}
\label{sec:forward}

Different versions of the main or child documents
using compilation flags as described in \secref{sec:flags}
can be (permanently) stored in different files
for convenient compilation, viewing and distribution.
To this end, the package defines a command
to pass on compilation to a different file:

%%%%%%%%%%%%%%%%%%%%%%%%%%%%%%%%%%%%%%%%
\DescribeMacro{\childdocforward}
The command |\childdocforward| redirects processing to
another source file:
%
\begin{center}
\begin{tabular}{l}
|\input{childdoc.def}|\\
|\childdocforward[|\textit{main}|]{|\textit{dest}|}|\\
\end{tabular}
\end{center}
%
The argument \textit{dest} is the destination file
(without extension).
It should be the main file or one of the child files.
Note that further \textsf{childdoc} directives
such as |\childdocof| and |\childdocforward|
in the indicated file will be processed in this form.
The optional argument \textit{main}
passes on directly to the main file \textit{main}
while pretending to compile the child \textit{dest}.
This form behaves as if \textit{dest}
issues |\childdocof{|\textit{main}|}| right away,
and no further \textsf{childdoc} directives will be processed.

%%%%%%%%%%%%%%%%%%%%%%%%%%%%%%%%%%%%%%%%
\DescribeMacro{\...prefix}
In the alternative form |\childdocforwardprefix|,
%
\begin{center}
\begin{tabular}{l}
|\input{childdoc.def}|\\
|\childdocforwardprefix[|\textit{main}|]{|\textit{prefix}|}{|\textit{dest}|}|
\end{tabular}
\end{center}
%
the destination file is determined by a pattern
depending on the current file:
To make this work, the current file must be called
`{\textit{prefix}\hspace{0.2em}\textit{suffix}}'
with \textit{prefix} matching precisely the argument.
Processing is then passed on to the file
`{\textit{dest}\hspace{0.2em}\textit{suffix}}'.
Surely, the same effect is achieved by
directly specifying the
argument `{\textit{dest}\hspace{0.2em}\textit{suffix}}'
in the first form.
However, that requires to set up a different file
for each child. With the alternative form of the command
all these files can have exactly the same content
which simplifies setting them up and maintaining them.

For example, the following file |draft.tex|
with a compilation flag |\version| as described in \secref{sec:flags}
compiles the main document as a draft:
%
\begin{center}
\begin{tabular}{l}
|\def\version{draft}|\\
|\input{childdoc.def}|\\
|\childdocforward{|\textit{main}|}|
\end{tabular}
\end{center}
%
Likewise, the following files |final|\textit{nn}|.tex|
compile the final version of the child document
|child|\textit{nn}|.tex|:
%
\begin{center}
\begin{tabular}{l}
|\def\version{final}|\\
|\input{childdoc.def}|\\
|\childdocforwardprefix{final}{child}|
\end{tabular}
\end{center}
%

Note that when several versions of a main file and/or of each child file
are to be generated, it may be convenient to set up a |Makefile| or
shell script to automatise the process.

%%%%%%%%%%%%%%%%%%%%%%%%%%%%%%%%%%%%%%%%%%%%%%%%%%%%%%%%%%%%%%%%%%%%%%%%%%%%%%%%
\subsection{Command Line Processing}
\label{sec:commandline}

The effect of redirection files can also be achieved by invoking
the \LaTeX{} compiler with a more elaborate command line.
Most conveniently this should be done as part
of a shell script or a |Makefile|.

When using \textsf{childdoc} in the main file, the following
command lines effectively perform a redirection
(note that depending on the shell being used,
backslashes may have to be doubled: `|\|' $\to$ `|\\|'):
%
\begin{center}
|... -jobname "|\textit{target}|" |\\|"|[\textit{flags}]%
|\input{childdoc.def}\childdocforward[|\textit{main}|]{|\textit{dest}|}"|
\end{center}
%
Here \textit{target} is the name of the output file,
\textit{main} is the name of the main file
and \textit{dest} is the name of the main or child file to be processed
(all filenames without extensions).
The optional argument \textit{main} can be omitted
if \textit{main} matches \textit{dest}.
Optionally, compilation \textit{flags} can be defined via |\def| commands.
This command line makes the \TeX{} engine believe
it is compiling the file \textit{target}
whose content is specified as the latter parameter.
The provided code then forwards the processing to
\textit{main} or \textit{dest} as described in \secref{sec:forward}.

%%%%%%%%%%%%%%%%%%%%%%%%%%%%%%%%%%%%%%%%%%%%%%%%%%%%%%%%%%%%%%%%%%%%%%%%%%%%%%%%
\subsection{Include by Input}
\label{sec:input}

Including child documents by |\include| has some restrictions by design.
Most notably, the content of a child document always occupies
its own set of pages; pages cannot be shared between child documents.
Usually, this behaviour makes perfect sense
because each child document contain an essential part of the document.
However, in some situations it may be desirable to compose
a document from a collection of parts
without having mandatory page breaks between then.
For this case, the package
provides a mechanism to include parts
by |\input| which can also be processed individually.
However, by construction this mechanism
requires manual handling of the content to be output.

%%%%%%%%%%%%%%%%%%%%%%%%%%%%%%%%%%%%%%%%
\DescribeMacro{\ifchilddocmanual}
The main file should be prepared as usual, see \secref{sec:include}.
However, the document body must make a distinction
between processing of an individual part and of the main document, e.g.:
%
\begin{center}
\begin{tabular}{l}
|\ifchilddocmanual|\\
|\input{\childdocname}|\\
|\||else|\\
\textit{document body with }|\input{|\textit{part}|}|\\
|\||fi|
\end{tabular}
\end{center}
%
The conditional |\ifchilddocmanual| is true whenever
a part to be included by |\input| is being compiled,
and the name of the part is stored in |\childdocname|.

%%%%%%%%%%%%%%%%%%%%%%%%%%%%%%%%%%%%%%%%
\DescribeMacro{\childdocby}
Each part to be included by |\input| should start with:
%
\begin{center}
\begin{tabular}{l}
|\input{childdoc.def}|\\
|\childdocby{|\textit{main}|}|\\
\end{tabular}
\end{center}
%
The directive |\childdocby| is similar to |\childdocof|
described in \secref{sec:include},
but the subsequent selection of content must be done manually.
To that end, both |\ifchilddoc| and |\ifchilddocmanual|
will be true upon processing of a part,
and the name of the part is stored in |\childdocname|.
Note that |\jobname| will be set to the filename of the current part
so that each part receives an individual |.aux| file
that does not interfere with the |.aux| file(s) of the main document.
This behaviour can be altered by the alternative form
|\childdocby[*]{|\textit{main}|}| (with a non-empty optional argument)
which uses the |.aux| file of the main document
by setting |\jobname| to \textit{main}.

%%%%%%%%%%%%%%%%%%%%%%%%%%%%%%%%%%%%%%%%%%%%%%%%%%%%%%%%%%%%%%%%%%%%%%%%%%%%%%%%
\subsection{Driver Development}
\label{sec:driver}

The \textsf{childdoc} mechanism can also be use for the development
of definition files such as \LaTeX{} styles or classes.
This case differs from the above setup with multiple parts
included by |\include| in that no |\includeonly| should be invoked.
This can be achieved by starting the include file
(before |\ProvidesPackage|) with:
%
\begin{center}
\begin{tabular}{l}
|\input{childdoc.def}|\\
|\childdocforward{|\textit{main}|}|\\
\end{tabular}
\end{center}
%
or alternatively with:
%
\begin{center}
\begin{tabular}{l}
|\input{childdoc.def}|\\
|\childdocby{|\textit{main}|}|\\
\end{tabular}
\end{center}
%
Both forms have slightly different effects as described above.
The main file is prepared as usual, see \secref{sec:include}.

%%%%%%%%%%%%%%%%%%%%%%%%%%%%%%%%%%%%%%%%%%%%%%%%%%%%%%%%%%%%%%%%%%%%%%%%%%%%%%%%
\subsection{Legacy Detection}
\label{sec:detection}

The directive |\childdocmain| in the main file can detect
whether the complete document or merely a child is to be compiled
even without using the directive |\childdocof|.
This method is deprecated because it is less robust
and there is no compelling reason to use it;
it is merely provided for backward compatibility
and it may be removed in future versions.

If the detection mechanism is to be used,
it is mandatory to correctly specify
the filename of the main file as the argument of |\childdocmain|:
%
\begin{center}
\begin{tabular}{l}
|\input{childdoc.def}|\\
|\childdocmain{|\textit{main}|}|\\
\end{tabular}
\end{center}
%
If |\jobname| does not match the argument \textit{main} of |\childdocmain|,
it is assumed that |\jobname| points to the child file to be compiled.
When using |\childdocmain| with the main file specified as argument,
it suffices to start a child file
with just |\input{|\textit{main}|}|
without loading of the package and using |\childdocof|.
If instead all processing is done
with the appropriate \textsf{childdoc} directives,
the argument of \textit{main} of |\childdocmain| can be empty.

An alternative version of the command line processing described
in \secref{sec:commandline} using the detection mechanism reads:
%
\begin{center}
|... -jobname "|\textit{target}|" "|[\textit{flags}]%
[|\def\jobname{|\textit{dest}|}|]|\input{|\textit{main}|}"|
\end{center}

%%%%%%%%%%%%%%%%%%%%%%%%%%%%%%%%%%%%%%%%%%%%%%%%%%%%%%%%%%%%%%%%%%%%%%%%%%%%%%%%
\subsection{Manual Code}
\label{sec:manual}

In case one cannot be certain whether the definitions file |childdoc.def|
is installed on the target \TeX{} distribution
and one prefers not to ship it,
it is conceivable to paste a few relevant commands into the sources.

To that end, drop all statements |\input{childdoc.def}|
and perform the replacements as outlined below.
Instead of |\childdocmain{|\textit{main}|}| add the following code
to the top of the main file:
%
\begin{center}
\begin{tabular}{l}
|\||ifdefined\childdocname\endinput\||fi\newif\ifchilddoc|\\
|\edef\childdocname{\scantokens\expandafter{\jobname\noexpand}}|\\
|\def\childdocmain{|\textit{main}|}\||ifx\childdocmain\childdocname\||else|\\
|\childdoctrue\includeonly{\childdocname}\let\jobname\childdocmain\||fi|\\
\end{tabular}
\end{center}
%
Instead of |\childdocof{|\textit{main}|}| just include the main file
at the top of each child file:
%
\begin{center}
|\input{|\textit{main}|}|
\end{center}
%
A simple redirection |\childdocforward{|\textit{dest}|}| is achieved by:
%
\begin{center}
|\def\jobname{|\textit{dest}|}\input{\jobname}|
\end{center}
%
The redirection with prefix
|\childdocforwardprefix[|\textit{prefix}|]{|\textit{dest}|}|
is accomplished by:
%
\begin{center}
\begin{tabular}{l}
|{\edef\jobname{\scantokens\expandafter{\jobname\noexpand}}|\\
|\def\redirectjob |\textit{prefix}|#1~~~{\gdef\jobname{|\textit{dest}|#1}}|\\
|\expandafter\redirectjob\jobname~~~}\input{\jobname}|
\end{tabular}
\end{center}

In an alternative approach,
child documents can be compiled by a specific command line
without additional code or specific definitions:
%
\begin{center}
|... -jobname "|\textit{target}|" "|[\textit{flags}]%
|\includeonly{|\textit{dest}|}\input{|\textit{main}|}"|
\end{center}
%

%%%%%%%%%%%%%%%%%%%%%%%%%%%%%%%%%%%%%%%%%%%%%%%%%%%%%%%%%%%%%%%%%%%%%%%%%%%%%%%%
%%%%%%%%%%%%%%%%%%%%%%%%%%%%%%%%%%%%%%%%%%%%%%%%%%%%%%%%%%%%%%%%%%%%%%%%%%%%%%%%
\section{Information}

%%%%%%%%%%%%%%%%%%%%%%%%%%%%%%%%%%%%%%%%%%%%%%%%%%%%%%%%%%%%%%%%%%%%%%%%%%%%%%%%
\subsection{Copyright}

Copyright \copyright{} 2017--2018 Niklas Beisert

This work may be distributed and/or modified under the
conditions of the \LaTeX{} Project Public License, either version 1.3
of this license or (at your option) any later version.
The latest version of this license is in
  \url{http://www.latex-project.org/lppl.txt}
and version 1.3 or later is part of all distributions of \LaTeX{}
version 2005/12/01 or later.

This work has the LPPL maintenance status `maintained'.

The Current Maintainer of this work is Niklas Beisert.

This work consists of the files |README.txt|, |childdoc.ins| and |childdoc.dtx|
as well as the derived files |childdoc.def|, |cdocsamp.tex|
with |cdocsch1.tex|, |cdocsch2.tex|, |cdocspt3.tex|, |cdocspt4.tex|,
|cdocsdrf.tex|, |cdocsfn1.tex|, |cdocsfn2.tex|
as well as |childdoc.pdf|.

%%%%%%%%%%%%%%%%%%%%%%%%%%%%%%%%%%%%%%%%%%%%%%%%%%%%%%%%%%%%%%%%%%%%%%%%%%%%%%%%
\subsection{Files and Installation}

The package consists of the files:
%
\begin{center}
\begin{tabular}{ll}
    |README.txt|   & readme file \\
    |childdoc.ins| & installation file \\
    |childdoc.dtx| & source file \\
    |childdoc.def| & definition file \\
    |cdocsamp.tex| & sample main file \\
    |cdocsch1.tex| & sample include file \\
    |cdocsch2.tex| & sample include file \\
    |cdocspt3.tex| & sample part file \\
    |cdocspt4.tex| & sample part file \\
    |cdocsdrf.tex| & sample redirection file \\
    |cdocsfn1.tex| & sample redirection file \\
    |cdocsfn2.tex| & sample redirection file \\
    |childdoc.pdf| & manual
\end{tabular}
\end{center}
%
The distribution consists of the files
|README.txt|, |childdoc.ins| and |childdoc.dtx|.
%
\begin{itemize}
\item
Run (pdf)\LaTeX{} on |childdoc.dtx|
to compile the manual |childdoc.pdf| (this file).
\item
Run \LaTeX{} on |childdoc.ins| to create the definitions file |childdoc.def|
and the sample |cdocsamp.tex| with include files
|cdocsch1.tex|, |cdocsch2.tex|, |cdocspt3.tex|, |cdocspt4.tex|,
|cdocsdrf.tex|, |cdocsfn1.tex|, |cdocsfn2.tex|.
Then copy the file |childdoc.def| to an appropriate directory of your \LaTeX{}
distribution, e.g.\ \textit{texmf-root}|/tex/latex/childdoc|.
\end{itemize}

%%%%%%%%%%%%%%%%%%%%%%%%%%%%%%%%%%%%%%%%%%%%%%%%%%%%%%%%%%%%%%%%%%%%%%%%%%%%%%%%
\subsection{Related CTAN Packages}

There are several other packages which offer a similar functionality:
%
\begin{itemize}
\item
The packages
\href{http://ctan.org/pkg/docmute}{\textsf{docmute}},
\href{http://ctan.org/pkg/includex}{\textsf{includex}} and
\href{http://ctan.org/pkg/standalone}{\textsf{standalone}}
provide commands to include only the document body of
a child file thus allowing both files to be compiled individually.
\item
The packages \href{http://ctan.org/pkg/subdocs}{\textsf{subdocs}}
and \href{http://ctan.org/pkg/subfiles}{\textsf{subfiles}}
provide structures in which the main and child documents can be
encapsulated and allowing them to be compiled individually.
The inclusion mechanism is different from the conventional |\include|.
\item
The package \href{http://ctan.org/pkg/combine}{\textsf{combine}}
is an elaborate solution to combine several documents into one.
\end{itemize}
%
See also the CTAN topic \href{http://ctan.org/topic/subdocs}{\textsf{subdocs}}
for further related packages.
The present package differs from the above solutions in that
a document structure constructed with the conventional |\include| mechanism
just needs two extra commands at the top of every file
such that all constituent files can be compiled individually.

%%%%%%%%%%%%%%%%%%%%%%%%%%%%%%%%%%%%%%%%%%%%%%%%%%%%%%%%%%%%%%%%%%%%%%%%%%%%%%%%
%\subsection{Feature Suggestions}
%
%The following is a list of features which may be useful for future
%versions of this package:
%%
%\begin{itemize}
%\item
%\ldots
%\end{itemize}

%%%%%%%%%%%%%%%%%%%%%%%%%%%%%%%%%%%%%%%%%%%%%%%%%%%%%%%%%%%%%%%%%%%%%%%%%%%%%%%%
\subsection{Revision History}

%%%%%%%%%%%%%%%%%%%%%%%%%%%%%%%%%%%%%%%%
\paragraph{v2.0:} 2018/12/30

\begin{itemize}
\item
immediate forward processing
\item
added |\childdocby| mechanism
\item
manual restructured
\end{itemize}

%%%%%%%%%%%%%%%%%%%%%%%%%%%%%%%%%%%%%%%%
\paragraph{v1.6:} 2018/01/17

\begin{itemize}
\item
application for development of include files
\item
corrections to manual
\end{itemize}

%%%%%%%%%%%%%%%%%%%%%%%%%%%%%%%%%%%%%%%%
\paragraph{v1.5:} 2017/05/21

\begin{itemize}
\item
more complete structuring introduced
\item
|\childdocof| introduced
\item
|\childdoc| renamed to |\childdocmain|
\item
|\childredirect| renamed to |\childdocforward| and |\childdocforwardprefix|
and functionality expanded
\end{itemize}

%%%%%%%%%%%%%%%%%%%%%%%%%%%%%%%%%%%%%%%%
\paragraph{v1.0:} 2017/04/27

\begin{itemize}
\item
manual and install package
\item
first version published on CTAN
\end{itemize}

%%%%%%%%%%%%%%%%%%%%%%%%%%%%%%%%%%%%%%%%
\paragraph{v0.6:} 2017/04/26

\begin{itemize}
\item
redirection mechanism added
\end{itemize}

%%%%%%%%%%%%%%%%%%%%%%%%%%%%%%%%%%%%%%%%
\paragraph{v0.5:} 2017/04/26

\begin{itemize}
\item
functionality in definition file
\end{itemize}


%%%%%%%%%%%%%%%%%%%%%%%%%%%%%%%%%%%%%%%%%%%%%%%%%%%%%%%%%%%%%%%%%%%%%%%%%%%%%%%%
%%%%%%%%%%%%%%%%%%%%%%%%%%%%%%%%%%%%%%%%%%%%%%%%%%%%%%%%%%%%%%%%%%%%%%%%%%%%%%%%
%%%%%%%%%%%%%%%%%%%%%%%%%%%%%%%%%%%%%%%%%%%%%%%%%%%%%%%%%%%%%%%%%%%%%%%%%%%%%%%%
\appendix

\settowidth\MacroIndent{\rmfamily\scriptsize 000\ }

 \DocInput{childdoc.dtx}

\end{document}
%</driver>
% \fi
%
% %%%%%%%%%%%%%%%%%%%%%%%%%%%%%%%%%%%%%%%%%%%%%%%%%%%%%%%%%%%%%%%%%%%%%%%%%%%%%%
% %%%%%%%%%%%%%%%%%%%%%%%%%%%%%%%%%%%%%%%%%%%%%%%%%%%%%%%%%%%%%%%%%%%%%%%%%%%%%%
% \section{Sample}
%\iffalse
%<*samplemain>
%\fi
%
% The following presents a sample document
% with two chapters, two parts, a title page,
% a compile flag as well as three forwarding files to set the flag.
% It consists of eight |.tex| files:
% \begin{center}
% \begin{tabular}{ll}
% |cdocsamp.tex|&main file\\
% |cdocsch1.tex|&include file for chapter 1\\
% |cdocsch2.tex|&include file for chapter 2\\
% |cdocspt3.tex|&include file for part 3\\
% |cdocspt4.tex|&include file for part 4\\
% |cdocsdrf.tex|&forwarding file for main file in draft mode\\
% |cdocsfi1.tex|&forwarding file for final version of chapter 1\\
% |cdocsfi2.tex|&forwarding file for final version of chapter 2\\
% \end{tabular}
% \end{center}
% Each of the eight files can be compiled directly by the \LaTeX{} compiler.
%
% %%%%%%%%%%%%%%%%%%%%%%%%%%%%%%%%%%%%%%
% \paragraph{Main File.}
%
% The main file is called |cdocsamp.tex|.
%
% Load the \textsf{childdoc} definitions and
% declare the filename for the main document:
%    \begin{macrocode}
\input{childdoc.def}
\childdocmain{}
%    \end{macrocode}

% Optional override for |\version| flag:
%    \begin{macrocode}
%%\ifchilddoc\else\providecommand{\version}{draft}\fi
%    \end{macrocode}

% Define the default values for the |\version| flag
% (|final| for the main file and |draft| for childs):
%    \begin{macrocode}
\ifchilddoc
\providecommand{\version}{draft}
\else
\providecommand{\version}{final}
\fi
%    \end{macrocode}

% Load the standard document class:
%    \begin{macrocode}
\documentclass[12pt]{article}
%    \end{macrocode}

% Start the document body:
%    \begin{macrocode}
\begin{document}
%    \end{macrocode}

% Declare a title page.
% Print title, part of document being processed and version flag:
%    \begin{macrocode}
\addtocounter{page}{-1}
\begin{center}
{\LARGE\bfseries{}childdoc example\par}
\vspace{1cm}
\ifchilddoc
\ifchilddocmanual part\else chapter\fi:
`\childdocname' of `\childdocjob'\par
\else
main document: `\childdocjob'\par
\fi
version: \version\par
\end{center}
\newpage
%    \end{macrocode}

% Manually include selected file,
% otherwise process as usual:
%    \begin{macrocode}
\ifchilddocmanual
\section*{part `\childdocname'}
\input{\childdocname}
\else
%    \end{macrocode}

% Include the two chapters:
%    \begin{macrocode}
\include{cdocsch1}
\include{cdocsch2}
%    \end{macrocode}

% Include the two parts unless only chapters should be displayed:
%    \begin{macrocode}
\ifchilddoc\else
\section{part three}
\input{cdocspt3}
\section{part four}
\input{cdocspt4}
\fi
%    \end{macrocode}

% Process as usual until here:
%    \begin{macrocode}
\fi
%    \end{macrocode}

% End of document body:
%    \begin{macrocode}
\end{document}
%    \end{macrocode}
%\iffalse
%</samplemain>
%\fi
%
% %%%%%%%%%%%%%%%%%%%%%%%%%%%%%%%%%%%%%%
% \paragraph{Chapter Include Files.}
%
% The include files are called |cdocsch1.tex| and |cdocsch2.tex|.
%
%\iffalse
%<*samplechap1|samplechap2>
%\fi

% Optional override for |\version| flag:
%    \begin{macrocode}
%%\providecommand{\version}{final}
%    \end{macrocode}

% Include the main document:
%    \begin{macrocode}
\input{childdoc.def}
\childdocof{cdocsamp}
%    \end{macrocode}

%\iffalse
%</samplechap1|samplechap2>
%\fi
%
%\iffalse
%<*samplechap1>
%\fi
% Some text for chapter 1:
%    \begin{macrocode}
\section{one}
some text in chapter one
%    \end{macrocode}

%\iffalse
%</samplechap1>
%\fi
% Some text for chapter 2:
%\iffalse
%<*samplechap2>
%\fi
%    \begin{macrocode}
\section{two}
more text in chapter two
%    \end{macrocode}

%\iffalse
%</samplechap2>
%\fi
%
% %%%%%%%%%%%%%%%%%%%%%%%%%%%%%%%%%%%%%%
% \paragraph{Part Include Files.}
%
% The include files are called |cdocspt3.tex| and |cdocspt4.tex|.
%
%\iffalse
%<*samplepart3|samplepart4>
%\fi

% Optional override for |\version| flag:
%    \begin{macrocode}
%%\providecommand{\version}{final}
%    \end{macrocode}

% Include the main document:
%    \begin{macrocode}
\input{childdoc.def}
\childdocby{cdocsamp}
%    \end{macrocode}

%\iffalse
%</samplepart3|samplepart4>
%\fi
%
%\iffalse
%<*samplepart3>
%\fi
% Some text for part 3:
%    \begin{macrocode}
some text in part three
%    \end{macrocode}

%\iffalse
%</samplepart3>
%\fi
% Some text for part 4:
%\iffalse
%<*samplepart4>
%\fi
%    \begin{macrocode}
more text in part four
%    \end{macrocode}

%\iffalse
%</samplepart4>
%\fi
%
% %%%%%%%%%%%%%%%%%%%%%%%%%%%%%%%%%%%%%%
% \paragraph{Forwarding for a Complete Draft.}
%
% The following forwarding file |cdocsdrf.tex|
% compiles the main document in draft mode:
%\iffalse
%<*sampledraft>
%\fi
%    \begin{macrocode}
\def\version{draft}
\input{childdoc.def}
\childdocforward{cdocsamp}
%    \end{macrocode}

%\iffalse
%</sampledraft>
%\fi
%
% %%%%%%%%%%%%%%%%%%%%%%%%%%%%%%%%%%%%%%
% \paragraph{Forwarding for Final Version of the Chapters.}
%
% The following forwarding files |cdocsfn1.tex| and |cdocsfn2.tex|
% (with identical content)
% compile the final versions of the child documents
% |cdocsch1.tex| and |cdocsch2.tex|, respectively:
%\iffalse
%<*samplefinal>
%\fi
%    \begin{macrocode}
\def\version{final}
\input{childdoc.def}
\childdocforwardprefix[cdocsamp]{cdocsfn}{cdocsch}
%    \end{macrocode}

%\iffalse
%</samplefinal>
%\fi
%
% %%%%%%%%%%%%%%%%%%%%%%%%%%%%%%%%%%%%%%
% \paragraph{Command Line Processing.}
%
% The following three command lines generate the output files
% |cdocscld|, |cdocscl1| and |cdocscl2|
% which should be identical to
% |cdocsdrf|, |cdocsch1| and |cdocsfn2|, respectively:
% \begin{center}
% \begin{tabular}{l}
% |latex -jobname cdocscld \|\\
% |  "\def\version{draft}\input{childdoc.def}\childdocforward{cdocsamp}"|\\
% |latex -jobname cdocscl1 \|\\
% |  "\input{childdoc.def}\childdocforward[cdocsamp]{cdocsch1}"|\\
% |latex -jobname cdocscl2 \|\\
% |  "\def\version{final}\input{childdoc.def}\childdocforward{cdocsch2}"|
% \end{tabular}
% \end{center}
% Note that the trailing backslash on each first line
% merely continues the input to the second line
% (for convenient cut ant paste).
% Furthermore, the command |latex| can be replaced by any
% of its alternative versions such as |pdflatex|.
%
% %%%%%%%%%%%%%%%%%%%%%%%%%%%%%%%%%%%%%%%%%%%%%%%%%%%%%%%%%%%%%%%%%%%%%%%%%%%%%%
% %%%%%%%%%%%%%%%%%%%%%%%%%%%%%%%%%%%%%%%%%%%%%%%%%%%%%%%%%%%%%%%%%%%%%%%%%%%%%%
% \section{Implementation}
%\iffalse
%<*package>
%\fi
%
% This section describes the definitions file |childdoc.def|.

% The definitions cannot be loaded using |\usepackage| or |\RequirePackage|
% which has a mechanism to prevent loading a style file more than once.
% When loading the definitions by means of |\input|
% multiple instances have to be prevented manually:
%\iffalse
%This code needs to be before the `\ProvidesFile' directive
%which is defined at the beginning of this file.
%Therefore it is also placed there and commented out here.
%</package>
%<*discard>
%\fi
%    \begin{macrocode}
\ifdefined\childdocmain\endinput\fi
%    \end{macrocode}
%\iffalse
%</discard>
%<*package>
%\fi
%
% \macro{\ifchilddoc}
% \macro{\ifchilddocmanual}
% The conditional |\ifchilddoc| tells whether a
% child (true) or main (false) document is being compiled.
% The conditional |\ifchilddocmanual| tells whether
% the |\includeonly| mechanism is used (false) or
% the selection of child files must be performed manually (true).
% The definitions initialise to false:
%    \begin{macrocode}
\newif\ifchilddoc
\newif\ifchilddocmanual
%    \end{macrocode}

% \macro{\childdocname}
% \macro{\childdocjob}
% The macro |\childdocname| stores the name of the main document
% to be compiled. The macro |\childdocjob| stores the name of
% the document on which the \LaTeX{} compiler was originally invoked.
% The content of |\jobname| cannot be compared
% to filenames specified in the source due to different catcodes.
% The following code rescans |\jobname|, stores the result
% in |\childdocname| and saves a copy in |\childdocjob|:
%    \begin{macrocode}
\edef\childdocname{\scantokens\expandafter{\jobname\noexpand}}
\let\childdocjob\childdocname
%    \end{macrocode}

% \macro{\childdocdisable}
% The macro |\childdocdisable| prevents the main file
% from being processed more than once.
% At this stage, the main document command |\childdocmain|
% is assumed to be called once again where it should do nothing.
% Any subsequent call to it should prevent
% a secondary processing of the main document
% It overwrites the forwarding commands
% |\childdocof| and |\childdocforward|
% with empty macros to prevent further inclusions of the main document:
%    \begin{macrocode}
\newcommand{\childdocdisable}
{
  \renewcommand{\childdocmain}[1]{\renewcommand{\childdocmain}[1]{\endinput}}
  \renewcommand{\childdocof}[1]{}
  \renewcommand{\childdocby}[2][]{}
  \renewcommand{\childdocforward}[2][]{}
  \renewcommand{\childdocdisable}{}
}
%    \end{macrocode}

% \macro{\childdocmain}
% The macro |\childdocmain| is to be called at the top of the main file
% with nothing or the main filename (without extension) as argument.
% First, it breaks loops.
% If the argument is not empty and does not match |\childdocname|
% (which is set by the first inclusion of |childdoc.def|),
% |\ifchilddoc| is set to true, |\includeonly| is applied to the child file
% and |\jobname| is set to the main file
% (for proper handling of |.aux| files):
%    \begin{macrocode}
\newcommand{\childdocmain}[1]
{
  \childdocdisable\childdocmain{}
  \if?#1?\else
    \begingroup
      \def\childdoctmp{#1}
      \ifx\childdoctmp\childdocname
        \def\childdoctmp{}
      \else
        \def\childdoctmp
        {
          \childdoctrue
          \includeonly{\childdocname}
          \def\childdocjob{#1}
          \def\jobname{#1}
        }
      \fi
      \expandafter
    \endgroup
    \childdoctmp
  \fi
}
%    \end{macrocode}

% \macro{\childdocof}
% The command |\childdocof| redirects
% compilation to the main file |#1|.
%    \begin{macrocode}
\newcommand{\childdocof}[1]
{
  \childdocdisable
  \childdoctrue
  \includeonly{\childdocname}
  \def\jobname{#1}
  \def\childdocjob{#1}
  \input{#1}
}
%    \end{macrocode}

% \macro{\childdocby}
% The command |\childdocby| ....
%    \begin{macrocode}
\newcommand{\childdocby}[2][]
{
  \childdocdisable
  \childdoctrue
  \childdocmanualtrue
  \if?#1?\else
    \def\jobname{#2}
  \fi
  \def\childdocjob{#2}
  \input{#2}
  \endinput
}
%    \end{macrocode}

% \macro{\childdocforward}
% The command |\childdocforward| redirects
% compilation to the main file or
% (if the optional argument is given) a child file.
% Parameters are set as if the main file
% or a child file starting with |\childdocof| was compiled.
% Then compilation is handed over to the main file:
%    \begin{macrocode}
\newcommand{\childdocforward}[2][]
{
  \begingroup
    \if?#1?
      \def\childdoctmp
      {
        \def\childdocname{#2}
        \def\childdocjob{#2}
        \def\jobname{#2}
        \input{#2}
        \endinput
      }
    \else
      \def\childdoctmp
      {
        \childdocdisable
        \def\childdocname{#2}
        \childdoctrue
        \includeonly{#2}
        \def\childdocjob{#1}
        \def\jobname{#1}
        \input{#1}
        \endinput
      }
    \fi
    \expandafter
  \endgroup
  \childdoctmp
}
%    \end{macrocode}

% \macro{\childdocforwardprefix}
% The command |\childdocforwardprefix| redirects
% compilation to the main or a child file by means of a pattern.
% The prefix |#1| in the current filename is replaced by |#2|
% and the suffix of the current filename is kept
% (it is assumed that the filename does not contain the substring `|~~~|'
% which is used as a delimiter).
% Compilation is handed over to the new file by |\childdocforward|:
%    \begin{macrocode}
\newcommand{\childdocforwardprefix}[3][]
{
  \begingroup
    \def\childdocextract #2##1~~~{\def\childdoctmp{\childdocforward[#1]{#3##1}}}
    \expandafter\childdocextract\childdocname~~~
    \expandafter
  \endgroup
  \childdoctmp
}
%    \end{macrocode}

% \macro{\childdoc}
% The deprecated macro |\childdoc| is a legacy version of |\childdocmain|:
%    \begin{macrocode}
\newcommand{\childdoc}{\childdocmain}
%    \end{macrocode}

% \macro{\childdocredirect}
% The deprecated macro |\childdocredirect| is a legacy version
% of |\childdocforward| and |\childdocforwardprefix|:
%    \begin{macrocode}
\newcommand{\childdocredirect}[2][]
{
  \begingroup
    \if?#1?
      \def\childdoctmp{\childdocforward{#2}}
    \else
      \def\childdoctmp{\childdocforwardprefix{#1}{#2}}
    \fi
    \expandafter
  \endgroup
  \childdoctmp
}
%    \end{macrocode}

%\iffalse
%</package>
%\fi
%
\endinput
\childdocforward{cdocsamp}"|\\
% |latex -jobname cdocscl1 \|\\
% |  "% \iffalse
%
% childdoc.dtx Copyright (C) 2017-2018 Niklas Beisert
%
% This work may be distributed and/or modified under the
% conditions of the LaTeX Project Public License, either version 1.3
% of this license or (at your option) any later version.
% The latest version of this license is in
%   http://www.latex-project.org/lppl.txt
% and version 1.3 or later is part of all distributions of LaTeX
% version 2005/12/01 or later.
%
% This work has the LPPL maintenance status `maintained'.
%
% The Current Maintainer of this work is Niklas Beisert.
%
% This work consists of the files childdoc.dtx and childdoc.ins
% and the derived files childdoc.def and cdocsamp.tex with
% cdocsch1.tex, cdocsch2.tex, cdocsdrf.tex, cdocsfn1.tex, cdocsfn2.tex.
%
%<package>\ifdefined\childdocmain\endinput\fi
%<package>\ProvidesFile{childdoc.def}[2018/12/30 v2.0 child document driver]
%<samplemain>\ProvidesFile{cdocsamp.tex}[2018/12/30 v2.0 sample for childdoc]
%<*driver>
%\ProvidesFile{childdoc.drv}[2018/12/30 v2.0 childdoc reference manual file]
\PassOptionsToClass{10pt,a4paper}{article}
\documentclass{ltxdoc}

\usepackage[margin=35mm]{geometry}
\usepackage{hyperref}
\usepackage{hyperxmp}
\usepackage[usenames]{color}

\hypersetup{colorlinks=true}
\hypersetup{pdfstartview=FitH}
\hypersetup{pdfpagemode=UseNone}
\hypersetup{pdfsource={}}
\hypersetup{pdflang={en-UK}}
\hypersetup{pdfcopyright={Copyright 2017-2018 Niklas Beisert.
  This work may be distributed and/or modified under the
  conditions of the LaTeX Project Public License, either version 1.3
  of this license or (at your option) any later version.}}
\hypersetup{pdflicenseurl={http://www.latex-project.org/lppl.txt}}
\hypersetup{pdfcontactaddress={ETH Zurich, ITP, HIT K,
  Wolfgang-Pauli-Strasse 27}}
\hypersetup{pdfcontactpostcode={8093}}
\hypersetup{pdfcontactcity={Zurich}}
\hypersetup{pdfcontactcountry={Switzerland}}
\hypersetup{pdfcontactemail={nbeisert@itp.phys.ethz.ch}}
\hypersetup{pdfcontacturl={http://people.phys.ethz.ch/\xmptilde nbeisert/}}

\newcommand{\secref}[1]{\hyperref[#1]{section \ref*{#1}}}

\parskip1ex
\parindent0pt
\let\olditemize\itemize
\def\itemize{\olditemize\parskip0pt}

\begin{document}

\title{The \textsf{childdoc} Package}
\hypersetup{pdftitle={The childdoc Package}}
\author{Niklas Beisert\\[2ex]
  Institut f\"ur Theoretische Physik\\
  Eidgen\"ossische Technische Hochschule Z\"urich\\
  Wolfgang-Pauli-Strasse 27, 8093 Z\"urich, Switzerland\\[1ex]
  \href{mailto:nbeisert@itp.phys.ethz.ch}
  {\texttt{nbeisert@itp.phys.ethz.ch}}}
\hypersetup{pdfauthor={Niklas Beisert}}
\hypersetup{pdfsubject={Manual for the LaTeX2e Package childdoc}}
\date{30 December 2018, \textsf{v2.0}}
\maketitle

\begin{abstract}\noindent
\textsf{childdoc} is a \LaTeXe{} package
that enables the direct compilation
of document sections included by |\include|
to individual files.
\end{abstract}

\begingroup
\parskip0ex
\tableofcontents
\endgroup

%%%%%%%%%%%%%%%%%%%%%%%%%%%%%%%%%%%%%%%%%%%%%%%%%%%%%%%%%%%%%%%%%%%%%%%%%%%%%%%%
%%%%%%%%%%%%%%%%%%%%%%%%%%%%%%%%%%%%%%%%%%%%%%%%%%%%%%%%%%%%%%%%%%%%%%%%%%%%%%%%
\section{Introduction}

\LaTeX{} provides a mechanism to structure a large document (such as a book)
into a main file and several child files (containing the chapters)
using the |\include| command.
This mechanism is beneficial for documents
which span hundreds of pages in order to
make the source file(s) more manageable.
Moreover, compilation can be restricted to
selected child files by means of the |\includeonly| command.
The latter feature can be used to reduce the compilation time while editing
(this was significantly more useful in the earlier days of \LaTeX{})
or to generate a smaller document which is easier to navigate.
Another application of |\includeonly| is to generate
documents consisting of selected parts of the complete document.

However, there are a few drawbacks of the plain |\include| mechanism:
\begin{itemize}
\item
The child files cannot be compiled on their own,
they can only be compiled via the main file.
A naive editing environment
(such as a text editor with an option
to have the current file processed by \LaTeX)
may require one to switch to the main file before compiling;
attempting to compile the child file produces errors.
\item
The main file must be modified (each time)
to adjust the |\includeonly| command
to the present needs. This easily leaves the main file in a messy state.
\item
The generated document will always carry the filename
of the main document. This is inconvenient if
several child files are to be compiled and
to be kept for distribution.
\end{itemize}

The present package provides a simple interface
to make child files individually compilable by \LaTeX{}.
Compiling a child file then has the same effect as compiling
the main file with an |\includeonly| command
to select the appropriate child.
Moreover the generated document will carry the name of the child
rather than the main file.
This resolves all three above issues.

This feature is meant to make the editing of books,
thesis documents and lecture notes somewhat more convenient.
However, the package can also be used efficiently for
composing a series of documents (such as exercise sheets)
which are typically distributed individually.
It then assists the author in generating the individual documents
(potentially in different versions)
as well as a document containing the collected series.
Another application is in developing style files
or other kinds of included material
where compilation of the style file could redirect
to a sample or test file.

%%%%%%%%%%%%%%%%%%%%%%%%%%%%%%%%%%%%%%%%%%%%%%%%%%%%%%%%%%%%%%%%%%%%%%%%%%%%%%%%
%%%%%%%%%%%%%%%%%%%%%%%%%%%%%%%%%%%%%%%%%%%%%%%%%%%%%%%%%%%%%%%%%%%%%%%%%%%%%%%%
\section{Usage}

First of all, the package \textsf{childdoc} is \emph{not} a standard
\LaTeXe{} |.sty| style file! Therefore it needs to be invoked in
a non-standard way.

%%%%%%%%%%%%%%%%%%%%%%%%%%%%%%%%%%%%%%%%%%%%%%%%%%%%%%%%%%%%%%%%%%%%%%%%%%%%%%%%
\subsection{Included Files}
\label{sec:include}

%%%%%%%%%%%%%%%%%%%%%%%%%%%%%%%%%%%%%%%%
\DescribeMacro{\childdocmain}
To use the package, add the commands
\begin{center}
\begin{tabular}{l}
|\input{childdoc.def}|\\
|\childdocmain{}|\\
\end{tabular}
\end{center}
at the very top of the main \LaTeX{} file,
in particular \emph{before} the |\documentclass| statement!
The argument of |\childdocmain| should be left empty
(but it must be present).

%%%%%%%%%%%%%%%%%%%%%%%%%%%%%%%%%%%%%%%%
\DescribeMacro{\childdocof}
Furthermore, add the commands
\begin{center}
\begin{tabular}{l}
|\input{childdoc.def}|\\
|\childdocof{|\textit{main}|}|\\
\end{tabular}
\end{center}
at the top of every child file \textit{child}
which is included by |\include{|\textit{child}|}|
from within the main file
(or at least for those files to be compiled individually).
The argument \textit{main} must be the filename of the main file.

There are a couple of
considerations in setting up the main and child documents:

%%%%%%%%%%%%%%%%%%%%%%%%%%%%%%%%%%%%%%%%
\paragraph{Restrictions.}

Please note the following restrictions:
\begin{itemize}
\item
|\childdocmain| must be called with one argument \textit{main}
to ensure compatibility with earlier version of the package.
It must either be empty (|\childdocmain{}|)
or precisely match the filename of the main file in which it is specified.
See \secref{sec:detection} for further information.
\item
The filename \textit{main} must be specified without the |.tex| extension.
\item
The filename \textit{main} is case sensitive
(even in case-insensitive file systems)
due to internal string comparison.
\item
The argument \textit{main} should be fully expanded, it cannot be a macro.
\item
Subdirectories and special characters should be avoided in filenames.
\item
The command |\childdocmain{|\textit{main}|}| must be followed by a whitespace.
It should not be followed immediately by another command
or by a comment mark `|%|'.
This is because the \TeX{} parser reads the token immediately following
the argument of |\childdocmain| and puts it
at the beginning of every child section;
however, a white\-space is ignored.
\end{itemize}

%%%%%%%%%%%%%%%%%%%%%%%%%%%%%%%%%%%%%%%%
\paragraph{Content of Main File.}

It is advisable to place all content in the child files included by |\include|.
Any output contained in the main file will appear in all child documents
unless suppressed manually;
it cannot be suppressed automatically by the |\includeonly| directive
and thus should normally be avoided.
A method to include some content in the main file
by means of conditional processing is described in \secref{sec:conditional}.

%%%%%%%%%%%%%%%%%%%%%%%%%%%%%%%%%%%%%%%%
\paragraph{Page Numbering.}

When only a part of the document is compiled,
the appropriate numbering of pages
(as well as other status parameters)
is determined from the |.aux| files.
The latter contain information from previous passes.
However this information needs to propagate through
all intermediate child documents.
Therefore the page numbering in child documents may well
be inconsistent until the complete document is compiled at least once.

A useful (if unconventional) way to always ensure a consistent
page numbering is to restart the numbering in each child document
and denote the pages by `\textit{child}|.|\textit{page}'
where \textit{child} represents the chapter/section number of the child file.
This can be achieved by the command
|\numberwithin{page}{|\textit{child}|}|
of the \textsf{amsmath} package
where \textit{child} can be |chapter| or |section|
depending on the chosen structuring.
Alternatively, one can modify the macro |\thepage| appropriately
and reset the counter |page| at the start of each child file.

%%%%%%%%%%%%%%%%%%%%%%%%%%%%%%%%%%%%%%%%%%%%%%%%%%%%%%%%%%%%%%%%%%%%%%%%%%%%%%%%
\subsection{Conditional Processing}
\label{sec:conditional}

The package provides a mechanism to compile different versions
of a document. To customise the versions further some conditional processing
can come in handy to distinguish which version is being compiled.
The package provides two macros to describe the compilation context:

%%%%%%%%%%%%%%%%%%%%%%%%%%%%%%%%%%%%%%%%
\DescribeMacro{\ifchilddoc}
The conditional |\ifchilddoc| distinguishes between the compilation of
child documents and the main document:
%
\begin{center}
|\ifchilddoc |\textit{child-code}| |[|\||else |\textit{main-code}]| \||fi|
\end{center}

%%%%%%%%%%%%%%%%%%%%%%%%%%%%%%%%%%%%%%%%
\DescribeMacro{\childdocname}
\DescribeMacro{\childdocjob}
The macro |\childdocname| contains the filename (without extension)
of the main or child file being processed.
Note that |\childdocjob| will always contain the name of the main file.

%%%%%%%%%%%%%%%%%%%%%%%%%%%%%%%%%%%%%%%%
\paragraph{Title Page.}

Conditional processing can be used to include a title or banner page
in the main document when proper precautions are taken.
Importantly, the code in the main file should ensure that the page counter
(as well as other status parameters which are stored in the |.aux| files)
takes the same value after the conditional processing.
Otherwise the page numbers may take divergent values
depending on which part is compiled.

For example, a title page could be declared by:
%
\begin{center}
\begin{tabular}{l}
|\ifchilddoc\||else|\\
|\addtocounter{page}{-1}|\\
\textit{code for title page}\\
|\newpage|\\
|\||fi|
\end{tabular}
\end{center}
%
A banner page for the child documents can be generated by:
%
\begin{center}
\begin{tabular}{l}
|\ifchilddoc|\\
|\addtocounter{page}{-1}|\\
\textit{code for banner page}\\
|\newpage|\\
|\||fi|
\end{tabular}
\end{center}
%
Here one could write a message such as:
\begin{center}
|This is the part \childdocname{} of \childdocjob{}.|
\end{center}

%%%%%%%%%%%%%%%%%%%%%%%%%%%%%%%%%%%%%%%%%%%%%%%%%%%%%%%%%%%%%%%%%%%%%%%%%%%%%%%%
\subsection{Flags}
\label{sec:flags}

The package makes it easy to generate different versions
of the main or child documents.
To this end compilation flags can be defined
and assigned different default values.
They will be particularly useful in conjunction
with the forwarding mechanism described in \secref{sec:forward}.

For example, it may be useful to have a flag |\version|
which can be set to |draft| or |final|.
The document source will contain some conditional code
depending on the value of |\version|.
Suppose further, the flag should default to |final| for the main file
and to |draft| for child files
which is a natural assignment for editing the document.
This is achieved by placing the following code
in the preamble of the main document
(below the |\childdocmain| directive):
%
\begin{center}
\begin{tabular}{l}
|\ifchilddoc|\\
|\providecommand{\version}{draft}|\\
|\||else|\\
|\providecommand{\version}{final}|\\
|\||fi|
\end{tabular}
\end{center}
%
The definition by |\providecommand| makes sure
that previous definitions are not overwritten.
Further statements |\providecommand{\version}{...}|
can thus be added before the above code to override it.

For the main file, one might add a line
(between |\childdocmain| and the above block)
%
\begin{center}
|%\ifchilddoc\||else\providecommand{\version}{draft}\||fi|
\end{center}
%
which can be uncommented to produce a draft version.
Likewise one can add a line to the very top of a child file
(above the |\childdocof{|\textit{main}|}| directive)
%
\begin{center}
|%\providecommand{\version}{final}|
\end{center}
%
which can be uncommented to produce the final version of this child document.

%%%%%%%%%%%%%%%%%%%%%%%%%%%%%%%%%%%%%%%%%%%%%%%%%%%%%%%%%%%%%%%%%%%%%%%%%%%%%%%%
\subsection{Forwarding}
\label{sec:forward}

Different versions of the main or child documents
using compilation flags as described in \secref{sec:flags}
can be (permanently) stored in different files
for convenient compilation, viewing and distribution.
To this end, the package defines a command
to pass on compilation to a different file:

%%%%%%%%%%%%%%%%%%%%%%%%%%%%%%%%%%%%%%%%
\DescribeMacro{\childdocforward}
The command |\childdocforward| redirects processing to
another source file:
%
\begin{center}
\begin{tabular}{l}
|\input{childdoc.def}|\\
|\childdocforward[|\textit{main}|]{|\textit{dest}|}|\\
\end{tabular}
\end{center}
%
The argument \textit{dest} is the destination file
(without extension).
It should be the main file or one of the child files.
Note that further \textsf{childdoc} directives
such as |\childdocof| and |\childdocforward|
in the indicated file will be processed in this form.
The optional argument \textit{main}
passes on directly to the main file \textit{main}
while pretending to compile the child \textit{dest}.
This form behaves as if \textit{dest}
issues |\childdocof{|\textit{main}|}| right away,
and no further \textsf{childdoc} directives will be processed.

%%%%%%%%%%%%%%%%%%%%%%%%%%%%%%%%%%%%%%%%
\DescribeMacro{\...prefix}
In the alternative form |\childdocforwardprefix|,
%
\begin{center}
\begin{tabular}{l}
|\input{childdoc.def}|\\
|\childdocforwardprefix[|\textit{main}|]{|\textit{prefix}|}{|\textit{dest}|}|
\end{tabular}
\end{center}
%
the destination file is determined by a pattern
depending on the current file:
To make this work, the current file must be called
`{\textit{prefix}\hspace{0.2em}\textit{suffix}}'
with \textit{prefix} matching precisely the argument.
Processing is then passed on to the file
`{\textit{dest}\hspace{0.2em}\textit{suffix}}'.
Surely, the same effect is achieved by
directly specifying the
argument `{\textit{dest}\hspace{0.2em}\textit{suffix}}'
in the first form.
However, that requires to set up a different file
for each child. With the alternative form of the command
all these files can have exactly the same content
which simplifies setting them up and maintaining them.

For example, the following file |draft.tex|
with a compilation flag |\version| as described in \secref{sec:flags}
compiles the main document as a draft:
%
\begin{center}
\begin{tabular}{l}
|\def\version{draft}|\\
|\input{childdoc.def}|\\
|\childdocforward{|\textit{main}|}|
\end{tabular}
\end{center}
%
Likewise, the following files |final|\textit{nn}|.tex|
compile the final version of the child document
|child|\textit{nn}|.tex|:
%
\begin{center}
\begin{tabular}{l}
|\def\version{final}|\\
|\input{childdoc.def}|\\
|\childdocforwardprefix{final}{child}|
\end{tabular}
\end{center}
%

Note that when several versions of a main file and/or of each child file
are to be generated, it may be convenient to set up a |Makefile| or
shell script to automatise the process.

%%%%%%%%%%%%%%%%%%%%%%%%%%%%%%%%%%%%%%%%%%%%%%%%%%%%%%%%%%%%%%%%%%%%%%%%%%%%%%%%
\subsection{Command Line Processing}
\label{sec:commandline}

The effect of redirection files can also be achieved by invoking
the \LaTeX{} compiler with a more elaborate command line.
Most conveniently this should be done as part
of a shell script or a |Makefile|.

When using \textsf{childdoc} in the main file, the following
command lines effectively perform a redirection
(note that depending on the shell being used,
backslashes may have to be doubled: `|\|' $\to$ `|\\|'):
%
\begin{center}
|... -jobname "|\textit{target}|" |\\|"|[\textit{flags}]%
|\input{childdoc.def}\childdocforward[|\textit{main}|]{|\textit{dest}|}"|
\end{center}
%
Here \textit{target} is the name of the output file,
\textit{main} is the name of the main file
and \textit{dest} is the name of the main or child file to be processed
(all filenames without extensions).
The optional argument \textit{main} can be omitted
if \textit{main} matches \textit{dest}.
Optionally, compilation \textit{flags} can be defined via |\def| commands.
This command line makes the \TeX{} engine believe
it is compiling the file \textit{target}
whose content is specified as the latter parameter.
The provided code then forwards the processing to
\textit{main} or \textit{dest} as described in \secref{sec:forward}.

%%%%%%%%%%%%%%%%%%%%%%%%%%%%%%%%%%%%%%%%%%%%%%%%%%%%%%%%%%%%%%%%%%%%%%%%%%%%%%%%
\subsection{Include by Input}
\label{sec:input}

Including child documents by |\include| has some restrictions by design.
Most notably, the content of a child document always occupies
its own set of pages; pages cannot be shared between child documents.
Usually, this behaviour makes perfect sense
because each child document contain an essential part of the document.
However, in some situations it may be desirable to compose
a document from a collection of parts
without having mandatory page breaks between then.
For this case, the package
provides a mechanism to include parts
by |\input| which can also be processed individually.
However, by construction this mechanism
requires manual handling of the content to be output.

%%%%%%%%%%%%%%%%%%%%%%%%%%%%%%%%%%%%%%%%
\DescribeMacro{\ifchilddocmanual}
The main file should be prepared as usual, see \secref{sec:include}.
However, the document body must make a distinction
between processing of an individual part and of the main document, e.g.:
%
\begin{center}
\begin{tabular}{l}
|\ifchilddocmanual|\\
|\input{\childdocname}|\\
|\||else|\\
\textit{document body with }|\input{|\textit{part}|}|\\
|\||fi|
\end{tabular}
\end{center}
%
The conditional |\ifchilddocmanual| is true whenever
a part to be included by |\input| is being compiled,
and the name of the part is stored in |\childdocname|.

%%%%%%%%%%%%%%%%%%%%%%%%%%%%%%%%%%%%%%%%
\DescribeMacro{\childdocby}
Each part to be included by |\input| should start with:
%
\begin{center}
\begin{tabular}{l}
|\input{childdoc.def}|\\
|\childdocby{|\textit{main}|}|\\
\end{tabular}
\end{center}
%
The directive |\childdocby| is similar to |\childdocof|
described in \secref{sec:include},
but the subsequent selection of content must be done manually.
To that end, both |\ifchilddoc| and |\ifchilddocmanual|
will be true upon processing of a part,
and the name of the part is stored in |\childdocname|.
Note that |\jobname| will be set to the filename of the current part
so that each part receives an individual |.aux| file
that does not interfere with the |.aux| file(s) of the main document.
This behaviour can be altered by the alternative form
|\childdocby[*]{|\textit{main}|}| (with a non-empty optional argument)
which uses the |.aux| file of the main document
by setting |\jobname| to \textit{main}.

%%%%%%%%%%%%%%%%%%%%%%%%%%%%%%%%%%%%%%%%%%%%%%%%%%%%%%%%%%%%%%%%%%%%%%%%%%%%%%%%
\subsection{Driver Development}
\label{sec:driver}

The \textsf{childdoc} mechanism can also be use for the development
of definition files such as \LaTeX{} styles or classes.
This case differs from the above setup with multiple parts
included by |\include| in that no |\includeonly| should be invoked.
This can be achieved by starting the include file
(before |\ProvidesPackage|) with:
%
\begin{center}
\begin{tabular}{l}
|\input{childdoc.def}|\\
|\childdocforward{|\textit{main}|}|\\
\end{tabular}
\end{center}
%
or alternatively with:
%
\begin{center}
\begin{tabular}{l}
|\input{childdoc.def}|\\
|\childdocby{|\textit{main}|}|\\
\end{tabular}
\end{center}
%
Both forms have slightly different effects as described above.
The main file is prepared as usual, see \secref{sec:include}.

%%%%%%%%%%%%%%%%%%%%%%%%%%%%%%%%%%%%%%%%%%%%%%%%%%%%%%%%%%%%%%%%%%%%%%%%%%%%%%%%
\subsection{Legacy Detection}
\label{sec:detection}

The directive |\childdocmain| in the main file can detect
whether the complete document or merely a child is to be compiled
even without using the directive |\childdocof|.
This method is deprecated because it is less robust
and there is no compelling reason to use it;
it is merely provided for backward compatibility
and it may be removed in future versions.

If the detection mechanism is to be used,
it is mandatory to correctly specify
the filename of the main file as the argument of |\childdocmain|:
%
\begin{center}
\begin{tabular}{l}
|\input{childdoc.def}|\\
|\childdocmain{|\textit{main}|}|\\
\end{tabular}
\end{center}
%
If |\jobname| does not match the argument \textit{main} of |\childdocmain|,
it is assumed that |\jobname| points to the child file to be compiled.
When using |\childdocmain| with the main file specified as argument,
it suffices to start a child file
with just |\input{|\textit{main}|}|
without loading of the package and using |\childdocof|.
If instead all processing is done
with the appropriate \textsf{childdoc} directives,
the argument of \textit{main} of |\childdocmain| can be empty.

An alternative version of the command line processing described
in \secref{sec:commandline} using the detection mechanism reads:
%
\begin{center}
|... -jobname "|\textit{target}|" "|[\textit{flags}]%
[|\def\jobname{|\textit{dest}|}|]|\input{|\textit{main}|}"|
\end{center}

%%%%%%%%%%%%%%%%%%%%%%%%%%%%%%%%%%%%%%%%%%%%%%%%%%%%%%%%%%%%%%%%%%%%%%%%%%%%%%%%
\subsection{Manual Code}
\label{sec:manual}

In case one cannot be certain whether the definitions file |childdoc.def|
is installed on the target \TeX{} distribution
and one prefers not to ship it,
it is conceivable to paste a few relevant commands into the sources.

To that end, drop all statements |\input{childdoc.def}|
and perform the replacements as outlined below.
Instead of |\childdocmain{|\textit{main}|}| add the following code
to the top of the main file:
%
\begin{center}
\begin{tabular}{l}
|\||ifdefined\childdocname\endinput\||fi\newif\ifchilddoc|\\
|\edef\childdocname{\scantokens\expandafter{\jobname\noexpand}}|\\
|\def\childdocmain{|\textit{main}|}\||ifx\childdocmain\childdocname\||else|\\
|\childdoctrue\includeonly{\childdocname}\let\jobname\childdocmain\||fi|\\
\end{tabular}
\end{center}
%
Instead of |\childdocof{|\textit{main}|}| just include the main file
at the top of each child file:
%
\begin{center}
|\input{|\textit{main}|}|
\end{center}
%
A simple redirection |\childdocforward{|\textit{dest}|}| is achieved by:
%
\begin{center}
|\def\jobname{|\textit{dest}|}\input{\jobname}|
\end{center}
%
The redirection with prefix
|\childdocforwardprefix[|\textit{prefix}|]{|\textit{dest}|}|
is accomplished by:
%
\begin{center}
\begin{tabular}{l}
|{\edef\jobname{\scantokens\expandafter{\jobname\noexpand}}|\\
|\def\redirectjob |\textit{prefix}|#1~~~{\gdef\jobname{|\textit{dest}|#1}}|\\
|\expandafter\redirectjob\jobname~~~}\input{\jobname}|
\end{tabular}
\end{center}

In an alternative approach,
child documents can be compiled by a specific command line
without additional code or specific definitions:
%
\begin{center}
|... -jobname "|\textit{target}|" "|[\textit{flags}]%
|\includeonly{|\textit{dest}|}\input{|\textit{main}|}"|
\end{center}
%

%%%%%%%%%%%%%%%%%%%%%%%%%%%%%%%%%%%%%%%%%%%%%%%%%%%%%%%%%%%%%%%%%%%%%%%%%%%%%%%%
%%%%%%%%%%%%%%%%%%%%%%%%%%%%%%%%%%%%%%%%%%%%%%%%%%%%%%%%%%%%%%%%%%%%%%%%%%%%%%%%
\section{Information}

%%%%%%%%%%%%%%%%%%%%%%%%%%%%%%%%%%%%%%%%%%%%%%%%%%%%%%%%%%%%%%%%%%%%%%%%%%%%%%%%
\subsection{Copyright}

Copyright \copyright{} 2017--2018 Niklas Beisert

This work may be distributed and/or modified under the
conditions of the \LaTeX{} Project Public License, either version 1.3
of this license or (at your option) any later version.
The latest version of this license is in
  \url{http://www.latex-project.org/lppl.txt}
and version 1.3 or later is part of all distributions of \LaTeX{}
version 2005/12/01 or later.

This work has the LPPL maintenance status `maintained'.

The Current Maintainer of this work is Niklas Beisert.

This work consists of the files |README.txt|, |childdoc.ins| and |childdoc.dtx|
as well as the derived files |childdoc.def|, |cdocsamp.tex|
with |cdocsch1.tex|, |cdocsch2.tex|, |cdocspt3.tex|, |cdocspt4.tex|,
|cdocsdrf.tex|, |cdocsfn1.tex|, |cdocsfn2.tex|
as well as |childdoc.pdf|.

%%%%%%%%%%%%%%%%%%%%%%%%%%%%%%%%%%%%%%%%%%%%%%%%%%%%%%%%%%%%%%%%%%%%%%%%%%%%%%%%
\subsection{Files and Installation}

The package consists of the files:
%
\begin{center}
\begin{tabular}{ll}
    |README.txt|   & readme file \\
    |childdoc.ins| & installation file \\
    |childdoc.dtx| & source file \\
    |childdoc.def| & definition file \\
    |cdocsamp.tex| & sample main file \\
    |cdocsch1.tex| & sample include file \\
    |cdocsch2.tex| & sample include file \\
    |cdocspt3.tex| & sample part file \\
    |cdocspt4.tex| & sample part file \\
    |cdocsdrf.tex| & sample redirection file \\
    |cdocsfn1.tex| & sample redirection file \\
    |cdocsfn2.tex| & sample redirection file \\
    |childdoc.pdf| & manual
\end{tabular}
\end{center}
%
The distribution consists of the files
|README.txt|, |childdoc.ins| and |childdoc.dtx|.
%
\begin{itemize}
\item
Run (pdf)\LaTeX{} on |childdoc.dtx|
to compile the manual |childdoc.pdf| (this file).
\item
Run \LaTeX{} on |childdoc.ins| to create the definitions file |childdoc.def|
and the sample |cdocsamp.tex| with include files
|cdocsch1.tex|, |cdocsch2.tex|, |cdocspt3.tex|, |cdocspt4.tex|,
|cdocsdrf.tex|, |cdocsfn1.tex|, |cdocsfn2.tex|.
Then copy the file |childdoc.def| to an appropriate directory of your \LaTeX{}
distribution, e.g.\ \textit{texmf-root}|/tex/latex/childdoc|.
\end{itemize}

%%%%%%%%%%%%%%%%%%%%%%%%%%%%%%%%%%%%%%%%%%%%%%%%%%%%%%%%%%%%%%%%%%%%%%%%%%%%%%%%
\subsection{Related CTAN Packages}

There are several other packages which offer a similar functionality:
%
\begin{itemize}
\item
The packages
\href{http://ctan.org/pkg/docmute}{\textsf{docmute}},
\href{http://ctan.org/pkg/includex}{\textsf{includex}} and
\href{http://ctan.org/pkg/standalone}{\textsf{standalone}}
provide commands to include only the document body of
a child file thus allowing both files to be compiled individually.
\item
The packages \href{http://ctan.org/pkg/subdocs}{\textsf{subdocs}}
and \href{http://ctan.org/pkg/subfiles}{\textsf{subfiles}}
provide structures in which the main and child documents can be
encapsulated and allowing them to be compiled individually.
The inclusion mechanism is different from the conventional |\include|.
\item
The package \href{http://ctan.org/pkg/combine}{\textsf{combine}}
is an elaborate solution to combine several documents into one.
\end{itemize}
%
See also the CTAN topic \href{http://ctan.org/topic/subdocs}{\textsf{subdocs}}
for further related packages.
The present package differs from the above solutions in that
a document structure constructed with the conventional |\include| mechanism
just needs two extra commands at the top of every file
such that all constituent files can be compiled individually.

%%%%%%%%%%%%%%%%%%%%%%%%%%%%%%%%%%%%%%%%%%%%%%%%%%%%%%%%%%%%%%%%%%%%%%%%%%%%%%%%
%\subsection{Feature Suggestions}
%
%The following is a list of features which may be useful for future
%versions of this package:
%%
%\begin{itemize}
%\item
%\ldots
%\end{itemize}

%%%%%%%%%%%%%%%%%%%%%%%%%%%%%%%%%%%%%%%%%%%%%%%%%%%%%%%%%%%%%%%%%%%%%%%%%%%%%%%%
\subsection{Revision History}

%%%%%%%%%%%%%%%%%%%%%%%%%%%%%%%%%%%%%%%%
\paragraph{v2.0:} 2018/12/30

\begin{itemize}
\item
immediate forward processing
\item
added |\childdocby| mechanism
\item
manual restructured
\end{itemize}

%%%%%%%%%%%%%%%%%%%%%%%%%%%%%%%%%%%%%%%%
\paragraph{v1.6:} 2018/01/17

\begin{itemize}
\item
application for development of include files
\item
corrections to manual
\end{itemize}

%%%%%%%%%%%%%%%%%%%%%%%%%%%%%%%%%%%%%%%%
\paragraph{v1.5:} 2017/05/21

\begin{itemize}
\item
more complete structuring introduced
\item
|\childdocof| introduced
\item
|\childdoc| renamed to |\childdocmain|
\item
|\childredirect| renamed to |\childdocforward| and |\childdocforwardprefix|
and functionality expanded
\end{itemize}

%%%%%%%%%%%%%%%%%%%%%%%%%%%%%%%%%%%%%%%%
\paragraph{v1.0:} 2017/04/27

\begin{itemize}
\item
manual and install package
\item
first version published on CTAN
\end{itemize}

%%%%%%%%%%%%%%%%%%%%%%%%%%%%%%%%%%%%%%%%
\paragraph{v0.6:} 2017/04/26

\begin{itemize}
\item
redirection mechanism added
\end{itemize}

%%%%%%%%%%%%%%%%%%%%%%%%%%%%%%%%%%%%%%%%
\paragraph{v0.5:} 2017/04/26

\begin{itemize}
\item
functionality in definition file
\end{itemize}


%%%%%%%%%%%%%%%%%%%%%%%%%%%%%%%%%%%%%%%%%%%%%%%%%%%%%%%%%%%%%%%%%%%%%%%%%%%%%%%%
%%%%%%%%%%%%%%%%%%%%%%%%%%%%%%%%%%%%%%%%%%%%%%%%%%%%%%%%%%%%%%%%%%%%%%%%%%%%%%%%
%%%%%%%%%%%%%%%%%%%%%%%%%%%%%%%%%%%%%%%%%%%%%%%%%%%%%%%%%%%%%%%%%%%%%%%%%%%%%%%%
\appendix

\settowidth\MacroIndent{\rmfamily\scriptsize 000\ }

 \DocInput{childdoc.dtx}

\end{document}
%</driver>
% \fi
%
% %%%%%%%%%%%%%%%%%%%%%%%%%%%%%%%%%%%%%%%%%%%%%%%%%%%%%%%%%%%%%%%%%%%%%%%%%%%%%%
% %%%%%%%%%%%%%%%%%%%%%%%%%%%%%%%%%%%%%%%%%%%%%%%%%%%%%%%%%%%%%%%%%%%%%%%%%%%%%%
% \section{Sample}
%\iffalse
%<*samplemain>
%\fi
%
% The following presents a sample document
% with two chapters, two parts, a title page,
% a compile flag as well as three forwarding files to set the flag.
% It consists of eight |.tex| files:
% \begin{center}
% \begin{tabular}{ll}
% |cdocsamp.tex|&main file\\
% |cdocsch1.tex|&include file for chapter 1\\
% |cdocsch2.tex|&include file for chapter 2\\
% |cdocspt3.tex|&include file for part 3\\
% |cdocspt4.tex|&include file for part 4\\
% |cdocsdrf.tex|&forwarding file for main file in draft mode\\
% |cdocsfi1.tex|&forwarding file for final version of chapter 1\\
% |cdocsfi2.tex|&forwarding file for final version of chapter 2\\
% \end{tabular}
% \end{center}
% Each of the eight files can be compiled directly by the \LaTeX{} compiler.
%
% %%%%%%%%%%%%%%%%%%%%%%%%%%%%%%%%%%%%%%
% \paragraph{Main File.}
%
% The main file is called |cdocsamp.tex|.
%
% Load the \textsf{childdoc} definitions and
% declare the filename for the main document:
%    \begin{macrocode}
\input{childdoc.def}
\childdocmain{}
%    \end{macrocode}

% Optional override for |\version| flag:
%    \begin{macrocode}
%%\ifchilddoc\else\providecommand{\version}{draft}\fi
%    \end{macrocode}

% Define the default values for the |\version| flag
% (|final| for the main file and |draft| for childs):
%    \begin{macrocode}
\ifchilddoc
\providecommand{\version}{draft}
\else
\providecommand{\version}{final}
\fi
%    \end{macrocode}

% Load the standard document class:
%    \begin{macrocode}
\documentclass[12pt]{article}
%    \end{macrocode}

% Start the document body:
%    \begin{macrocode}
\begin{document}
%    \end{macrocode}

% Declare a title page.
% Print title, part of document being processed and version flag:
%    \begin{macrocode}
\addtocounter{page}{-1}
\begin{center}
{\LARGE\bfseries{}childdoc example\par}
\vspace{1cm}
\ifchilddoc
\ifchilddocmanual part\else chapter\fi:
`\childdocname' of `\childdocjob'\par
\else
main document: `\childdocjob'\par
\fi
version: \version\par
\end{center}
\newpage
%    \end{macrocode}

% Manually include selected file,
% otherwise process as usual:
%    \begin{macrocode}
\ifchilddocmanual
\section*{part `\childdocname'}
\input{\childdocname}
\else
%    \end{macrocode}

% Include the two chapters:
%    \begin{macrocode}
\include{cdocsch1}
\include{cdocsch2}
%    \end{macrocode}

% Include the two parts unless only chapters should be displayed:
%    \begin{macrocode}
\ifchilddoc\else
\section{part three}
\input{cdocspt3}
\section{part four}
\input{cdocspt4}
\fi
%    \end{macrocode}

% Process as usual until here:
%    \begin{macrocode}
\fi
%    \end{macrocode}

% End of document body:
%    \begin{macrocode}
\end{document}
%    \end{macrocode}
%\iffalse
%</samplemain>
%\fi
%
% %%%%%%%%%%%%%%%%%%%%%%%%%%%%%%%%%%%%%%
% \paragraph{Chapter Include Files.}
%
% The include files are called |cdocsch1.tex| and |cdocsch2.tex|.
%
%\iffalse
%<*samplechap1|samplechap2>
%\fi

% Optional override for |\version| flag:
%    \begin{macrocode}
%%\providecommand{\version}{final}
%    \end{macrocode}

% Include the main document:
%    \begin{macrocode}
\input{childdoc.def}
\childdocof{cdocsamp}
%    \end{macrocode}

%\iffalse
%</samplechap1|samplechap2>
%\fi
%
%\iffalse
%<*samplechap1>
%\fi
% Some text for chapter 1:
%    \begin{macrocode}
\section{one}
some text in chapter one
%    \end{macrocode}

%\iffalse
%</samplechap1>
%\fi
% Some text for chapter 2:
%\iffalse
%<*samplechap2>
%\fi
%    \begin{macrocode}
\section{two}
more text in chapter two
%    \end{macrocode}

%\iffalse
%</samplechap2>
%\fi
%
% %%%%%%%%%%%%%%%%%%%%%%%%%%%%%%%%%%%%%%
% \paragraph{Part Include Files.}
%
% The include files are called |cdocspt3.tex| and |cdocspt4.tex|.
%
%\iffalse
%<*samplepart3|samplepart4>
%\fi

% Optional override for |\version| flag:
%    \begin{macrocode}
%%\providecommand{\version}{final}
%    \end{macrocode}

% Include the main document:
%    \begin{macrocode}
\input{childdoc.def}
\childdocby{cdocsamp}
%    \end{macrocode}

%\iffalse
%</samplepart3|samplepart4>
%\fi
%
%\iffalse
%<*samplepart3>
%\fi
% Some text for part 3:
%    \begin{macrocode}
some text in part three
%    \end{macrocode}

%\iffalse
%</samplepart3>
%\fi
% Some text for part 4:
%\iffalse
%<*samplepart4>
%\fi
%    \begin{macrocode}
more text in part four
%    \end{macrocode}

%\iffalse
%</samplepart4>
%\fi
%
% %%%%%%%%%%%%%%%%%%%%%%%%%%%%%%%%%%%%%%
% \paragraph{Forwarding for a Complete Draft.}
%
% The following forwarding file |cdocsdrf.tex|
% compiles the main document in draft mode:
%\iffalse
%<*sampledraft>
%\fi
%    \begin{macrocode}
\def\version{draft}
\input{childdoc.def}
\childdocforward{cdocsamp}
%    \end{macrocode}

%\iffalse
%</sampledraft>
%\fi
%
% %%%%%%%%%%%%%%%%%%%%%%%%%%%%%%%%%%%%%%
% \paragraph{Forwarding for Final Version of the Chapters.}
%
% The following forwarding files |cdocsfn1.tex| and |cdocsfn2.tex|
% (with identical content)
% compile the final versions of the child documents
% |cdocsch1.tex| and |cdocsch2.tex|, respectively:
%\iffalse
%<*samplefinal>
%\fi
%    \begin{macrocode}
\def\version{final}
\input{childdoc.def}
\childdocforwardprefix[cdocsamp]{cdocsfn}{cdocsch}
%    \end{macrocode}

%\iffalse
%</samplefinal>
%\fi
%
% %%%%%%%%%%%%%%%%%%%%%%%%%%%%%%%%%%%%%%
% \paragraph{Command Line Processing.}
%
% The following three command lines generate the output files
% |cdocscld|, |cdocscl1| and |cdocscl2|
% which should be identical to
% |cdocsdrf|, |cdocsch1| and |cdocsfn2|, respectively:
% \begin{center}
% \begin{tabular}{l}
% |latex -jobname cdocscld \|\\
% |  "\def\version{draft}\input{childdoc.def}\childdocforward{cdocsamp}"|\\
% |latex -jobname cdocscl1 \|\\
% |  "\input{childdoc.def}\childdocforward[cdocsamp]{cdocsch1}"|\\
% |latex -jobname cdocscl2 \|\\
% |  "\def\version{final}\input{childdoc.def}\childdocforward{cdocsch2}"|
% \end{tabular}
% \end{center}
% Note that the trailing backslash on each first line
% merely continues the input to the second line
% (for convenient cut ant paste).
% Furthermore, the command |latex| can be replaced by any
% of its alternative versions such as |pdflatex|.
%
% %%%%%%%%%%%%%%%%%%%%%%%%%%%%%%%%%%%%%%%%%%%%%%%%%%%%%%%%%%%%%%%%%%%%%%%%%%%%%%
% %%%%%%%%%%%%%%%%%%%%%%%%%%%%%%%%%%%%%%%%%%%%%%%%%%%%%%%%%%%%%%%%%%%%%%%%%%%%%%
% \section{Implementation}
%\iffalse
%<*package>
%\fi
%
% This section describes the definitions file |childdoc.def|.

% The definitions cannot be loaded using |\usepackage| or |\RequirePackage|
% which has a mechanism to prevent loading a style file more than once.
% When loading the definitions by means of |\input|
% multiple instances have to be prevented manually:
%\iffalse
%This code needs to be before the `\ProvidesFile' directive
%which is defined at the beginning of this file.
%Therefore it is also placed there and commented out here.
%</package>
%<*discard>
%\fi
%    \begin{macrocode}
\ifdefined\childdocmain\endinput\fi
%    \end{macrocode}
%\iffalse
%</discard>
%<*package>
%\fi
%
% \macro{\ifchilddoc}
% \macro{\ifchilddocmanual}
% The conditional |\ifchilddoc| tells whether a
% child (true) or main (false) document is being compiled.
% The conditional |\ifchilddocmanual| tells whether
% the |\includeonly| mechanism is used (false) or
% the selection of child files must be performed manually (true).
% The definitions initialise to false:
%    \begin{macrocode}
\newif\ifchilddoc
\newif\ifchilddocmanual
%    \end{macrocode}

% \macro{\childdocname}
% \macro{\childdocjob}
% The macro |\childdocname| stores the name of the main document
% to be compiled. The macro |\childdocjob| stores the name of
% the document on which the \LaTeX{} compiler was originally invoked.
% The content of |\jobname| cannot be compared
% to filenames specified in the source due to different catcodes.
% The following code rescans |\jobname|, stores the result
% in |\childdocname| and saves a copy in |\childdocjob|:
%    \begin{macrocode}
\edef\childdocname{\scantokens\expandafter{\jobname\noexpand}}
\let\childdocjob\childdocname
%    \end{macrocode}

% \macro{\childdocdisable}
% The macro |\childdocdisable| prevents the main file
% from being processed more than once.
% At this stage, the main document command |\childdocmain|
% is assumed to be called once again where it should do nothing.
% Any subsequent call to it should prevent
% a secondary processing of the main document
% It overwrites the forwarding commands
% |\childdocof| and |\childdocforward|
% with empty macros to prevent further inclusions of the main document:
%    \begin{macrocode}
\newcommand{\childdocdisable}
{
  \renewcommand{\childdocmain}[1]{\renewcommand{\childdocmain}[1]{\endinput}}
  \renewcommand{\childdocof}[1]{}
  \renewcommand{\childdocby}[2][]{}
  \renewcommand{\childdocforward}[2][]{}
  \renewcommand{\childdocdisable}{}
}
%    \end{macrocode}

% \macro{\childdocmain}
% The macro |\childdocmain| is to be called at the top of the main file
% with nothing or the main filename (without extension) as argument.
% First, it breaks loops.
% If the argument is not empty and does not match |\childdocname|
% (which is set by the first inclusion of |childdoc.def|),
% |\ifchilddoc| is set to true, |\includeonly| is applied to the child file
% and |\jobname| is set to the main file
% (for proper handling of |.aux| files):
%    \begin{macrocode}
\newcommand{\childdocmain}[1]
{
  \childdocdisable\childdocmain{}
  \if?#1?\else
    \begingroup
      \def\childdoctmp{#1}
      \ifx\childdoctmp\childdocname
        \def\childdoctmp{}
      \else
        \def\childdoctmp
        {
          \childdoctrue
          \includeonly{\childdocname}
          \def\childdocjob{#1}
          \def\jobname{#1}
        }
      \fi
      \expandafter
    \endgroup
    \childdoctmp
  \fi
}
%    \end{macrocode}

% \macro{\childdocof}
% The command |\childdocof| redirects
% compilation to the main file |#1|.
%    \begin{macrocode}
\newcommand{\childdocof}[1]
{
  \childdocdisable
  \childdoctrue
  \includeonly{\childdocname}
  \def\jobname{#1}
  \def\childdocjob{#1}
  \input{#1}
}
%    \end{macrocode}

% \macro{\childdocby}
% The command |\childdocby| ....
%    \begin{macrocode}
\newcommand{\childdocby}[2][]
{
  \childdocdisable
  \childdoctrue
  \childdocmanualtrue
  \if?#1?\else
    \def\jobname{#2}
  \fi
  \def\childdocjob{#2}
  \input{#2}
  \endinput
}
%    \end{macrocode}

% \macro{\childdocforward}
% The command |\childdocforward| redirects
% compilation to the main file or
% (if the optional argument is given) a child file.
% Parameters are set as if the main file
% or a child file starting with |\childdocof| was compiled.
% Then compilation is handed over to the main file:
%    \begin{macrocode}
\newcommand{\childdocforward}[2][]
{
  \begingroup
    \if?#1?
      \def\childdoctmp
      {
        \def\childdocname{#2}
        \def\childdocjob{#2}
        \def\jobname{#2}
        \input{#2}
        \endinput
      }
    \else
      \def\childdoctmp
      {
        \childdocdisable
        \def\childdocname{#2}
        \childdoctrue
        \includeonly{#2}
        \def\childdocjob{#1}
        \def\jobname{#1}
        \input{#1}
        \endinput
      }
    \fi
    \expandafter
  \endgroup
  \childdoctmp
}
%    \end{macrocode}

% \macro{\childdocforwardprefix}
% The command |\childdocforwardprefix| redirects
% compilation to the main or a child file by means of a pattern.
% The prefix |#1| in the current filename is replaced by |#2|
% and the suffix of the current filename is kept
% (it is assumed that the filename does not contain the substring `|~~~|'
% which is used as a delimiter).
% Compilation is handed over to the new file by |\childdocforward|:
%    \begin{macrocode}
\newcommand{\childdocforwardprefix}[3][]
{
  \begingroup
    \def\childdocextract #2##1~~~{\def\childdoctmp{\childdocforward[#1]{#3##1}}}
    \expandafter\childdocextract\childdocname~~~
    \expandafter
  \endgroup
  \childdoctmp
}
%    \end{macrocode}

% \macro{\childdoc}
% The deprecated macro |\childdoc| is a legacy version of |\childdocmain|:
%    \begin{macrocode}
\newcommand{\childdoc}{\childdocmain}
%    \end{macrocode}

% \macro{\childdocredirect}
% The deprecated macro |\childdocredirect| is a legacy version
% of |\childdocforward| and |\childdocforwardprefix|:
%    \begin{macrocode}
\newcommand{\childdocredirect}[2][]
{
  \begingroup
    \if?#1?
      \def\childdoctmp{\childdocforward{#2}}
    \else
      \def\childdoctmp{\childdocforwardprefix{#1}{#2}}
    \fi
    \expandafter
  \endgroup
  \childdoctmp
}
%    \end{macrocode}

%\iffalse
%</package>
%\fi
%
\endinput
\childdocforward[cdocsamp]{cdocsch1}"|\\
% |latex -jobname cdocscl2 \|\\
% |  "\def\version{final}% \iffalse
%
% childdoc.dtx Copyright (C) 2017-2018 Niklas Beisert
%
% This work may be distributed and/or modified under the
% conditions of the LaTeX Project Public License, either version 1.3
% of this license or (at your option) any later version.
% The latest version of this license is in
%   http://www.latex-project.org/lppl.txt
% and version 1.3 or later is part of all distributions of LaTeX
% version 2005/12/01 or later.
%
% This work has the LPPL maintenance status `maintained'.
%
% The Current Maintainer of this work is Niklas Beisert.
%
% This work consists of the files childdoc.dtx and childdoc.ins
% and the derived files childdoc.def and cdocsamp.tex with
% cdocsch1.tex, cdocsch2.tex, cdocsdrf.tex, cdocsfn1.tex, cdocsfn2.tex.
%
%<package>\ifdefined\childdocmain\endinput\fi
%<package>\ProvidesFile{childdoc.def}[2018/12/30 v2.0 child document driver]
%<samplemain>\ProvidesFile{cdocsamp.tex}[2018/12/30 v2.0 sample for childdoc]
%<*driver>
%\ProvidesFile{childdoc.drv}[2018/12/30 v2.0 childdoc reference manual file]
\PassOptionsToClass{10pt,a4paper}{article}
\documentclass{ltxdoc}

\usepackage[margin=35mm]{geometry}
\usepackage{hyperref}
\usepackage{hyperxmp}
\usepackage[usenames]{color}

\hypersetup{colorlinks=true}
\hypersetup{pdfstartview=FitH}
\hypersetup{pdfpagemode=UseNone}
\hypersetup{pdfsource={}}
\hypersetup{pdflang={en-UK}}
\hypersetup{pdfcopyright={Copyright 2017-2018 Niklas Beisert.
  This work may be distributed and/or modified under the
  conditions of the LaTeX Project Public License, either version 1.3
  of this license or (at your option) any later version.}}
\hypersetup{pdflicenseurl={http://www.latex-project.org/lppl.txt}}
\hypersetup{pdfcontactaddress={ETH Zurich, ITP, HIT K,
  Wolfgang-Pauli-Strasse 27}}
\hypersetup{pdfcontactpostcode={8093}}
\hypersetup{pdfcontactcity={Zurich}}
\hypersetup{pdfcontactcountry={Switzerland}}
\hypersetup{pdfcontactemail={nbeisert@itp.phys.ethz.ch}}
\hypersetup{pdfcontacturl={http://people.phys.ethz.ch/\xmptilde nbeisert/}}

\newcommand{\secref}[1]{\hyperref[#1]{section \ref*{#1}}}

\parskip1ex
\parindent0pt
\let\olditemize\itemize
\def\itemize{\olditemize\parskip0pt}

\begin{document}

\title{The \textsf{childdoc} Package}
\hypersetup{pdftitle={The childdoc Package}}
\author{Niklas Beisert\\[2ex]
  Institut f\"ur Theoretische Physik\\
  Eidgen\"ossische Technische Hochschule Z\"urich\\
  Wolfgang-Pauli-Strasse 27, 8093 Z\"urich, Switzerland\\[1ex]
  \href{mailto:nbeisert@itp.phys.ethz.ch}
  {\texttt{nbeisert@itp.phys.ethz.ch}}}
\hypersetup{pdfauthor={Niklas Beisert}}
\hypersetup{pdfsubject={Manual for the LaTeX2e Package childdoc}}
\date{30 December 2018, \textsf{v2.0}}
\maketitle

\begin{abstract}\noindent
\textsf{childdoc} is a \LaTeXe{} package
that enables the direct compilation
of document sections included by |\include|
to individual files.
\end{abstract}

\begingroup
\parskip0ex
\tableofcontents
\endgroup

%%%%%%%%%%%%%%%%%%%%%%%%%%%%%%%%%%%%%%%%%%%%%%%%%%%%%%%%%%%%%%%%%%%%%%%%%%%%%%%%
%%%%%%%%%%%%%%%%%%%%%%%%%%%%%%%%%%%%%%%%%%%%%%%%%%%%%%%%%%%%%%%%%%%%%%%%%%%%%%%%
\section{Introduction}

\LaTeX{} provides a mechanism to structure a large document (such as a book)
into a main file and several child files (containing the chapters)
using the |\include| command.
This mechanism is beneficial for documents
which span hundreds of pages in order to
make the source file(s) more manageable.
Moreover, compilation can be restricted to
selected child files by means of the |\includeonly| command.
The latter feature can be used to reduce the compilation time while editing
(this was significantly more useful in the earlier days of \LaTeX{})
or to generate a smaller document which is easier to navigate.
Another application of |\includeonly| is to generate
documents consisting of selected parts of the complete document.

However, there are a few drawbacks of the plain |\include| mechanism:
\begin{itemize}
\item
The child files cannot be compiled on their own,
they can only be compiled via the main file.
A naive editing environment
(such as a text editor with an option
to have the current file processed by \LaTeX)
may require one to switch to the main file before compiling;
attempting to compile the child file produces errors.
\item
The main file must be modified (each time)
to adjust the |\includeonly| command
to the present needs. This easily leaves the main file in a messy state.
\item
The generated document will always carry the filename
of the main document. This is inconvenient if
several child files are to be compiled and
to be kept for distribution.
\end{itemize}

The present package provides a simple interface
to make child files individually compilable by \LaTeX{}.
Compiling a child file then has the same effect as compiling
the main file with an |\includeonly| command
to select the appropriate child.
Moreover the generated document will carry the name of the child
rather than the main file.
This resolves all three above issues.

This feature is meant to make the editing of books,
thesis documents and lecture notes somewhat more convenient.
However, the package can also be used efficiently for
composing a series of documents (such as exercise sheets)
which are typically distributed individually.
It then assists the author in generating the individual documents
(potentially in different versions)
as well as a document containing the collected series.
Another application is in developing style files
or other kinds of included material
where compilation of the style file could redirect
to a sample or test file.

%%%%%%%%%%%%%%%%%%%%%%%%%%%%%%%%%%%%%%%%%%%%%%%%%%%%%%%%%%%%%%%%%%%%%%%%%%%%%%%%
%%%%%%%%%%%%%%%%%%%%%%%%%%%%%%%%%%%%%%%%%%%%%%%%%%%%%%%%%%%%%%%%%%%%%%%%%%%%%%%%
\section{Usage}

First of all, the package \textsf{childdoc} is \emph{not} a standard
\LaTeXe{} |.sty| style file! Therefore it needs to be invoked in
a non-standard way.

%%%%%%%%%%%%%%%%%%%%%%%%%%%%%%%%%%%%%%%%%%%%%%%%%%%%%%%%%%%%%%%%%%%%%%%%%%%%%%%%
\subsection{Included Files}
\label{sec:include}

%%%%%%%%%%%%%%%%%%%%%%%%%%%%%%%%%%%%%%%%
\DescribeMacro{\childdocmain}
To use the package, add the commands
\begin{center}
\begin{tabular}{l}
|\input{childdoc.def}|\\
|\childdocmain{}|\\
\end{tabular}
\end{center}
at the very top of the main \LaTeX{} file,
in particular \emph{before} the |\documentclass| statement!
The argument of |\childdocmain| should be left empty
(but it must be present).

%%%%%%%%%%%%%%%%%%%%%%%%%%%%%%%%%%%%%%%%
\DescribeMacro{\childdocof}
Furthermore, add the commands
\begin{center}
\begin{tabular}{l}
|\input{childdoc.def}|\\
|\childdocof{|\textit{main}|}|\\
\end{tabular}
\end{center}
at the top of every child file \textit{child}
which is included by |\include{|\textit{child}|}|
from within the main file
(or at least for those files to be compiled individually).
The argument \textit{main} must be the filename of the main file.

There are a couple of
considerations in setting up the main and child documents:

%%%%%%%%%%%%%%%%%%%%%%%%%%%%%%%%%%%%%%%%
\paragraph{Restrictions.}

Please note the following restrictions:
\begin{itemize}
\item
|\childdocmain| must be called with one argument \textit{main}
to ensure compatibility with earlier version of the package.
It must either be empty (|\childdocmain{}|)
or precisely match the filename of the main file in which it is specified.
See \secref{sec:detection} for further information.
\item
The filename \textit{main} must be specified without the |.tex| extension.
\item
The filename \textit{main} is case sensitive
(even in case-insensitive file systems)
due to internal string comparison.
\item
The argument \textit{main} should be fully expanded, it cannot be a macro.
\item
Subdirectories and special characters should be avoided in filenames.
\item
The command |\childdocmain{|\textit{main}|}| must be followed by a whitespace.
It should not be followed immediately by another command
or by a comment mark `|%|'.
This is because the \TeX{} parser reads the token immediately following
the argument of |\childdocmain| and puts it
at the beginning of every child section;
however, a white\-space is ignored.
\end{itemize}

%%%%%%%%%%%%%%%%%%%%%%%%%%%%%%%%%%%%%%%%
\paragraph{Content of Main File.}

It is advisable to place all content in the child files included by |\include|.
Any output contained in the main file will appear in all child documents
unless suppressed manually;
it cannot be suppressed automatically by the |\includeonly| directive
and thus should normally be avoided.
A method to include some content in the main file
by means of conditional processing is described in \secref{sec:conditional}.

%%%%%%%%%%%%%%%%%%%%%%%%%%%%%%%%%%%%%%%%
\paragraph{Page Numbering.}

When only a part of the document is compiled,
the appropriate numbering of pages
(as well as other status parameters)
is determined from the |.aux| files.
The latter contain information from previous passes.
However this information needs to propagate through
all intermediate child documents.
Therefore the page numbering in child documents may well
be inconsistent until the complete document is compiled at least once.

A useful (if unconventional) way to always ensure a consistent
page numbering is to restart the numbering in each child document
and denote the pages by `\textit{child}|.|\textit{page}'
where \textit{child} represents the chapter/section number of the child file.
This can be achieved by the command
|\numberwithin{page}{|\textit{child}|}|
of the \textsf{amsmath} package
where \textit{child} can be |chapter| or |section|
depending on the chosen structuring.
Alternatively, one can modify the macro |\thepage| appropriately
and reset the counter |page| at the start of each child file.

%%%%%%%%%%%%%%%%%%%%%%%%%%%%%%%%%%%%%%%%%%%%%%%%%%%%%%%%%%%%%%%%%%%%%%%%%%%%%%%%
\subsection{Conditional Processing}
\label{sec:conditional}

The package provides a mechanism to compile different versions
of a document. To customise the versions further some conditional processing
can come in handy to distinguish which version is being compiled.
The package provides two macros to describe the compilation context:

%%%%%%%%%%%%%%%%%%%%%%%%%%%%%%%%%%%%%%%%
\DescribeMacro{\ifchilddoc}
The conditional |\ifchilddoc| distinguishes between the compilation of
child documents and the main document:
%
\begin{center}
|\ifchilddoc |\textit{child-code}| |[|\||else |\textit{main-code}]| \||fi|
\end{center}

%%%%%%%%%%%%%%%%%%%%%%%%%%%%%%%%%%%%%%%%
\DescribeMacro{\childdocname}
\DescribeMacro{\childdocjob}
The macro |\childdocname| contains the filename (without extension)
of the main or child file being processed.
Note that |\childdocjob| will always contain the name of the main file.

%%%%%%%%%%%%%%%%%%%%%%%%%%%%%%%%%%%%%%%%
\paragraph{Title Page.}

Conditional processing can be used to include a title or banner page
in the main document when proper precautions are taken.
Importantly, the code in the main file should ensure that the page counter
(as well as other status parameters which are stored in the |.aux| files)
takes the same value after the conditional processing.
Otherwise the page numbers may take divergent values
depending on which part is compiled.

For example, a title page could be declared by:
%
\begin{center}
\begin{tabular}{l}
|\ifchilddoc\||else|\\
|\addtocounter{page}{-1}|\\
\textit{code for title page}\\
|\newpage|\\
|\||fi|
\end{tabular}
\end{center}
%
A banner page for the child documents can be generated by:
%
\begin{center}
\begin{tabular}{l}
|\ifchilddoc|\\
|\addtocounter{page}{-1}|\\
\textit{code for banner page}\\
|\newpage|\\
|\||fi|
\end{tabular}
\end{center}
%
Here one could write a message such as:
\begin{center}
|This is the part \childdocname{} of \childdocjob{}.|
\end{center}

%%%%%%%%%%%%%%%%%%%%%%%%%%%%%%%%%%%%%%%%%%%%%%%%%%%%%%%%%%%%%%%%%%%%%%%%%%%%%%%%
\subsection{Flags}
\label{sec:flags}

The package makes it easy to generate different versions
of the main or child documents.
To this end compilation flags can be defined
and assigned different default values.
They will be particularly useful in conjunction
with the forwarding mechanism described in \secref{sec:forward}.

For example, it may be useful to have a flag |\version|
which can be set to |draft| or |final|.
The document source will contain some conditional code
depending on the value of |\version|.
Suppose further, the flag should default to |final| for the main file
and to |draft| for child files
which is a natural assignment for editing the document.
This is achieved by placing the following code
in the preamble of the main document
(below the |\childdocmain| directive):
%
\begin{center}
\begin{tabular}{l}
|\ifchilddoc|\\
|\providecommand{\version}{draft}|\\
|\||else|\\
|\providecommand{\version}{final}|\\
|\||fi|
\end{tabular}
\end{center}
%
The definition by |\providecommand| makes sure
that previous definitions are not overwritten.
Further statements |\providecommand{\version}{...}|
can thus be added before the above code to override it.

For the main file, one might add a line
(between |\childdocmain| and the above block)
%
\begin{center}
|%\ifchilddoc\||else\providecommand{\version}{draft}\||fi|
\end{center}
%
which can be uncommented to produce a draft version.
Likewise one can add a line to the very top of a child file
(above the |\childdocof{|\textit{main}|}| directive)
%
\begin{center}
|%\providecommand{\version}{final}|
\end{center}
%
which can be uncommented to produce the final version of this child document.

%%%%%%%%%%%%%%%%%%%%%%%%%%%%%%%%%%%%%%%%%%%%%%%%%%%%%%%%%%%%%%%%%%%%%%%%%%%%%%%%
\subsection{Forwarding}
\label{sec:forward}

Different versions of the main or child documents
using compilation flags as described in \secref{sec:flags}
can be (permanently) stored in different files
for convenient compilation, viewing and distribution.
To this end, the package defines a command
to pass on compilation to a different file:

%%%%%%%%%%%%%%%%%%%%%%%%%%%%%%%%%%%%%%%%
\DescribeMacro{\childdocforward}
The command |\childdocforward| redirects processing to
another source file:
%
\begin{center}
\begin{tabular}{l}
|\input{childdoc.def}|\\
|\childdocforward[|\textit{main}|]{|\textit{dest}|}|\\
\end{tabular}
\end{center}
%
The argument \textit{dest} is the destination file
(without extension).
It should be the main file or one of the child files.
Note that further \textsf{childdoc} directives
such as |\childdocof| and |\childdocforward|
in the indicated file will be processed in this form.
The optional argument \textit{main}
passes on directly to the main file \textit{main}
while pretending to compile the child \textit{dest}.
This form behaves as if \textit{dest}
issues |\childdocof{|\textit{main}|}| right away,
and no further \textsf{childdoc} directives will be processed.

%%%%%%%%%%%%%%%%%%%%%%%%%%%%%%%%%%%%%%%%
\DescribeMacro{\...prefix}
In the alternative form |\childdocforwardprefix|,
%
\begin{center}
\begin{tabular}{l}
|\input{childdoc.def}|\\
|\childdocforwardprefix[|\textit{main}|]{|\textit{prefix}|}{|\textit{dest}|}|
\end{tabular}
\end{center}
%
the destination file is determined by a pattern
depending on the current file:
To make this work, the current file must be called
`{\textit{prefix}\hspace{0.2em}\textit{suffix}}'
with \textit{prefix} matching precisely the argument.
Processing is then passed on to the file
`{\textit{dest}\hspace{0.2em}\textit{suffix}}'.
Surely, the same effect is achieved by
directly specifying the
argument `{\textit{dest}\hspace{0.2em}\textit{suffix}}'
in the first form.
However, that requires to set up a different file
for each child. With the alternative form of the command
all these files can have exactly the same content
which simplifies setting them up and maintaining them.

For example, the following file |draft.tex|
with a compilation flag |\version| as described in \secref{sec:flags}
compiles the main document as a draft:
%
\begin{center}
\begin{tabular}{l}
|\def\version{draft}|\\
|\input{childdoc.def}|\\
|\childdocforward{|\textit{main}|}|
\end{tabular}
\end{center}
%
Likewise, the following files |final|\textit{nn}|.tex|
compile the final version of the child document
|child|\textit{nn}|.tex|:
%
\begin{center}
\begin{tabular}{l}
|\def\version{final}|\\
|\input{childdoc.def}|\\
|\childdocforwardprefix{final}{child}|
\end{tabular}
\end{center}
%

Note that when several versions of a main file and/or of each child file
are to be generated, it may be convenient to set up a |Makefile| or
shell script to automatise the process.

%%%%%%%%%%%%%%%%%%%%%%%%%%%%%%%%%%%%%%%%%%%%%%%%%%%%%%%%%%%%%%%%%%%%%%%%%%%%%%%%
\subsection{Command Line Processing}
\label{sec:commandline}

The effect of redirection files can also be achieved by invoking
the \LaTeX{} compiler with a more elaborate command line.
Most conveniently this should be done as part
of a shell script or a |Makefile|.

When using \textsf{childdoc} in the main file, the following
command lines effectively perform a redirection
(note that depending on the shell being used,
backslashes may have to be doubled: `|\|' $\to$ `|\\|'):
%
\begin{center}
|... -jobname "|\textit{target}|" |\\|"|[\textit{flags}]%
|\input{childdoc.def}\childdocforward[|\textit{main}|]{|\textit{dest}|}"|
\end{center}
%
Here \textit{target} is the name of the output file,
\textit{main} is the name of the main file
and \textit{dest} is the name of the main or child file to be processed
(all filenames without extensions).
The optional argument \textit{main} can be omitted
if \textit{main} matches \textit{dest}.
Optionally, compilation \textit{flags} can be defined via |\def| commands.
This command line makes the \TeX{} engine believe
it is compiling the file \textit{target}
whose content is specified as the latter parameter.
The provided code then forwards the processing to
\textit{main} or \textit{dest} as described in \secref{sec:forward}.

%%%%%%%%%%%%%%%%%%%%%%%%%%%%%%%%%%%%%%%%%%%%%%%%%%%%%%%%%%%%%%%%%%%%%%%%%%%%%%%%
\subsection{Include by Input}
\label{sec:input}

Including child documents by |\include| has some restrictions by design.
Most notably, the content of a child document always occupies
its own set of pages; pages cannot be shared between child documents.
Usually, this behaviour makes perfect sense
because each child document contain an essential part of the document.
However, in some situations it may be desirable to compose
a document from a collection of parts
without having mandatory page breaks between then.
For this case, the package
provides a mechanism to include parts
by |\input| which can also be processed individually.
However, by construction this mechanism
requires manual handling of the content to be output.

%%%%%%%%%%%%%%%%%%%%%%%%%%%%%%%%%%%%%%%%
\DescribeMacro{\ifchilddocmanual}
The main file should be prepared as usual, see \secref{sec:include}.
However, the document body must make a distinction
between processing of an individual part and of the main document, e.g.:
%
\begin{center}
\begin{tabular}{l}
|\ifchilddocmanual|\\
|\input{\childdocname}|\\
|\||else|\\
\textit{document body with }|\input{|\textit{part}|}|\\
|\||fi|
\end{tabular}
\end{center}
%
The conditional |\ifchilddocmanual| is true whenever
a part to be included by |\input| is being compiled,
and the name of the part is stored in |\childdocname|.

%%%%%%%%%%%%%%%%%%%%%%%%%%%%%%%%%%%%%%%%
\DescribeMacro{\childdocby}
Each part to be included by |\input| should start with:
%
\begin{center}
\begin{tabular}{l}
|\input{childdoc.def}|\\
|\childdocby{|\textit{main}|}|\\
\end{tabular}
\end{center}
%
The directive |\childdocby| is similar to |\childdocof|
described in \secref{sec:include},
but the subsequent selection of content must be done manually.
To that end, both |\ifchilddoc| and |\ifchilddocmanual|
will be true upon processing of a part,
and the name of the part is stored in |\childdocname|.
Note that |\jobname| will be set to the filename of the current part
so that each part receives an individual |.aux| file
that does not interfere with the |.aux| file(s) of the main document.
This behaviour can be altered by the alternative form
|\childdocby[*]{|\textit{main}|}| (with a non-empty optional argument)
which uses the |.aux| file of the main document
by setting |\jobname| to \textit{main}.

%%%%%%%%%%%%%%%%%%%%%%%%%%%%%%%%%%%%%%%%%%%%%%%%%%%%%%%%%%%%%%%%%%%%%%%%%%%%%%%%
\subsection{Driver Development}
\label{sec:driver}

The \textsf{childdoc} mechanism can also be use for the development
of definition files such as \LaTeX{} styles or classes.
This case differs from the above setup with multiple parts
included by |\include| in that no |\includeonly| should be invoked.
This can be achieved by starting the include file
(before |\ProvidesPackage|) with:
%
\begin{center}
\begin{tabular}{l}
|\input{childdoc.def}|\\
|\childdocforward{|\textit{main}|}|\\
\end{tabular}
\end{center}
%
or alternatively with:
%
\begin{center}
\begin{tabular}{l}
|\input{childdoc.def}|\\
|\childdocby{|\textit{main}|}|\\
\end{tabular}
\end{center}
%
Both forms have slightly different effects as described above.
The main file is prepared as usual, see \secref{sec:include}.

%%%%%%%%%%%%%%%%%%%%%%%%%%%%%%%%%%%%%%%%%%%%%%%%%%%%%%%%%%%%%%%%%%%%%%%%%%%%%%%%
\subsection{Legacy Detection}
\label{sec:detection}

The directive |\childdocmain| in the main file can detect
whether the complete document or merely a child is to be compiled
even without using the directive |\childdocof|.
This method is deprecated because it is less robust
and there is no compelling reason to use it;
it is merely provided for backward compatibility
and it may be removed in future versions.

If the detection mechanism is to be used,
it is mandatory to correctly specify
the filename of the main file as the argument of |\childdocmain|:
%
\begin{center}
\begin{tabular}{l}
|\input{childdoc.def}|\\
|\childdocmain{|\textit{main}|}|\\
\end{tabular}
\end{center}
%
If |\jobname| does not match the argument \textit{main} of |\childdocmain|,
it is assumed that |\jobname| points to the child file to be compiled.
When using |\childdocmain| with the main file specified as argument,
it suffices to start a child file
with just |\input{|\textit{main}|}|
without loading of the package and using |\childdocof|.
If instead all processing is done
with the appropriate \textsf{childdoc} directives,
the argument of \textit{main} of |\childdocmain| can be empty.

An alternative version of the command line processing described
in \secref{sec:commandline} using the detection mechanism reads:
%
\begin{center}
|... -jobname "|\textit{target}|" "|[\textit{flags}]%
[|\def\jobname{|\textit{dest}|}|]|\input{|\textit{main}|}"|
\end{center}

%%%%%%%%%%%%%%%%%%%%%%%%%%%%%%%%%%%%%%%%%%%%%%%%%%%%%%%%%%%%%%%%%%%%%%%%%%%%%%%%
\subsection{Manual Code}
\label{sec:manual}

In case one cannot be certain whether the definitions file |childdoc.def|
is installed on the target \TeX{} distribution
and one prefers not to ship it,
it is conceivable to paste a few relevant commands into the sources.

To that end, drop all statements |\input{childdoc.def}|
and perform the replacements as outlined below.
Instead of |\childdocmain{|\textit{main}|}| add the following code
to the top of the main file:
%
\begin{center}
\begin{tabular}{l}
|\||ifdefined\childdocname\endinput\||fi\newif\ifchilddoc|\\
|\edef\childdocname{\scantokens\expandafter{\jobname\noexpand}}|\\
|\def\childdocmain{|\textit{main}|}\||ifx\childdocmain\childdocname\||else|\\
|\childdoctrue\includeonly{\childdocname}\let\jobname\childdocmain\||fi|\\
\end{tabular}
\end{center}
%
Instead of |\childdocof{|\textit{main}|}| just include the main file
at the top of each child file:
%
\begin{center}
|\input{|\textit{main}|}|
\end{center}
%
A simple redirection |\childdocforward{|\textit{dest}|}| is achieved by:
%
\begin{center}
|\def\jobname{|\textit{dest}|}\input{\jobname}|
\end{center}
%
The redirection with prefix
|\childdocforwardprefix[|\textit{prefix}|]{|\textit{dest}|}|
is accomplished by:
%
\begin{center}
\begin{tabular}{l}
|{\edef\jobname{\scantokens\expandafter{\jobname\noexpand}}|\\
|\def\redirectjob |\textit{prefix}|#1~~~{\gdef\jobname{|\textit{dest}|#1}}|\\
|\expandafter\redirectjob\jobname~~~}\input{\jobname}|
\end{tabular}
\end{center}

In an alternative approach,
child documents can be compiled by a specific command line
without additional code or specific definitions:
%
\begin{center}
|... -jobname "|\textit{target}|" "|[\textit{flags}]%
|\includeonly{|\textit{dest}|}\input{|\textit{main}|}"|
\end{center}
%

%%%%%%%%%%%%%%%%%%%%%%%%%%%%%%%%%%%%%%%%%%%%%%%%%%%%%%%%%%%%%%%%%%%%%%%%%%%%%%%%
%%%%%%%%%%%%%%%%%%%%%%%%%%%%%%%%%%%%%%%%%%%%%%%%%%%%%%%%%%%%%%%%%%%%%%%%%%%%%%%%
\section{Information}

%%%%%%%%%%%%%%%%%%%%%%%%%%%%%%%%%%%%%%%%%%%%%%%%%%%%%%%%%%%%%%%%%%%%%%%%%%%%%%%%
\subsection{Copyright}

Copyright \copyright{} 2017--2018 Niklas Beisert

This work may be distributed and/or modified under the
conditions of the \LaTeX{} Project Public License, either version 1.3
of this license or (at your option) any later version.
The latest version of this license is in
  \url{http://www.latex-project.org/lppl.txt}
and version 1.3 or later is part of all distributions of \LaTeX{}
version 2005/12/01 or later.

This work has the LPPL maintenance status `maintained'.

The Current Maintainer of this work is Niklas Beisert.

This work consists of the files |README.txt|, |childdoc.ins| and |childdoc.dtx|
as well as the derived files |childdoc.def|, |cdocsamp.tex|
with |cdocsch1.tex|, |cdocsch2.tex|, |cdocspt3.tex|, |cdocspt4.tex|,
|cdocsdrf.tex|, |cdocsfn1.tex|, |cdocsfn2.tex|
as well as |childdoc.pdf|.

%%%%%%%%%%%%%%%%%%%%%%%%%%%%%%%%%%%%%%%%%%%%%%%%%%%%%%%%%%%%%%%%%%%%%%%%%%%%%%%%
\subsection{Files and Installation}

The package consists of the files:
%
\begin{center}
\begin{tabular}{ll}
    |README.txt|   & readme file \\
    |childdoc.ins| & installation file \\
    |childdoc.dtx| & source file \\
    |childdoc.def| & definition file \\
    |cdocsamp.tex| & sample main file \\
    |cdocsch1.tex| & sample include file \\
    |cdocsch2.tex| & sample include file \\
    |cdocspt3.tex| & sample part file \\
    |cdocspt4.tex| & sample part file \\
    |cdocsdrf.tex| & sample redirection file \\
    |cdocsfn1.tex| & sample redirection file \\
    |cdocsfn2.tex| & sample redirection file \\
    |childdoc.pdf| & manual
\end{tabular}
\end{center}
%
The distribution consists of the files
|README.txt|, |childdoc.ins| and |childdoc.dtx|.
%
\begin{itemize}
\item
Run (pdf)\LaTeX{} on |childdoc.dtx|
to compile the manual |childdoc.pdf| (this file).
\item
Run \LaTeX{} on |childdoc.ins| to create the definitions file |childdoc.def|
and the sample |cdocsamp.tex| with include files
|cdocsch1.tex|, |cdocsch2.tex|, |cdocspt3.tex|, |cdocspt4.tex|,
|cdocsdrf.tex|, |cdocsfn1.tex|, |cdocsfn2.tex|.
Then copy the file |childdoc.def| to an appropriate directory of your \LaTeX{}
distribution, e.g.\ \textit{texmf-root}|/tex/latex/childdoc|.
\end{itemize}

%%%%%%%%%%%%%%%%%%%%%%%%%%%%%%%%%%%%%%%%%%%%%%%%%%%%%%%%%%%%%%%%%%%%%%%%%%%%%%%%
\subsection{Related CTAN Packages}

There are several other packages which offer a similar functionality:
%
\begin{itemize}
\item
The packages
\href{http://ctan.org/pkg/docmute}{\textsf{docmute}},
\href{http://ctan.org/pkg/includex}{\textsf{includex}} and
\href{http://ctan.org/pkg/standalone}{\textsf{standalone}}
provide commands to include only the document body of
a child file thus allowing both files to be compiled individually.
\item
The packages \href{http://ctan.org/pkg/subdocs}{\textsf{subdocs}}
and \href{http://ctan.org/pkg/subfiles}{\textsf{subfiles}}
provide structures in which the main and child documents can be
encapsulated and allowing them to be compiled individually.
The inclusion mechanism is different from the conventional |\include|.
\item
The package \href{http://ctan.org/pkg/combine}{\textsf{combine}}
is an elaborate solution to combine several documents into one.
\end{itemize}
%
See also the CTAN topic \href{http://ctan.org/topic/subdocs}{\textsf{subdocs}}
for further related packages.
The present package differs from the above solutions in that
a document structure constructed with the conventional |\include| mechanism
just needs two extra commands at the top of every file
such that all constituent files can be compiled individually.

%%%%%%%%%%%%%%%%%%%%%%%%%%%%%%%%%%%%%%%%%%%%%%%%%%%%%%%%%%%%%%%%%%%%%%%%%%%%%%%%
%\subsection{Feature Suggestions}
%
%The following is a list of features which may be useful for future
%versions of this package:
%%
%\begin{itemize}
%\item
%\ldots
%\end{itemize}

%%%%%%%%%%%%%%%%%%%%%%%%%%%%%%%%%%%%%%%%%%%%%%%%%%%%%%%%%%%%%%%%%%%%%%%%%%%%%%%%
\subsection{Revision History}

%%%%%%%%%%%%%%%%%%%%%%%%%%%%%%%%%%%%%%%%
\paragraph{v2.0:} 2018/12/30

\begin{itemize}
\item
immediate forward processing
\item
added |\childdocby| mechanism
\item
manual restructured
\end{itemize}

%%%%%%%%%%%%%%%%%%%%%%%%%%%%%%%%%%%%%%%%
\paragraph{v1.6:} 2018/01/17

\begin{itemize}
\item
application for development of include files
\item
corrections to manual
\end{itemize}

%%%%%%%%%%%%%%%%%%%%%%%%%%%%%%%%%%%%%%%%
\paragraph{v1.5:} 2017/05/21

\begin{itemize}
\item
more complete structuring introduced
\item
|\childdocof| introduced
\item
|\childdoc| renamed to |\childdocmain|
\item
|\childredirect| renamed to |\childdocforward| and |\childdocforwardprefix|
and functionality expanded
\end{itemize}

%%%%%%%%%%%%%%%%%%%%%%%%%%%%%%%%%%%%%%%%
\paragraph{v1.0:} 2017/04/27

\begin{itemize}
\item
manual and install package
\item
first version published on CTAN
\end{itemize}

%%%%%%%%%%%%%%%%%%%%%%%%%%%%%%%%%%%%%%%%
\paragraph{v0.6:} 2017/04/26

\begin{itemize}
\item
redirection mechanism added
\end{itemize}

%%%%%%%%%%%%%%%%%%%%%%%%%%%%%%%%%%%%%%%%
\paragraph{v0.5:} 2017/04/26

\begin{itemize}
\item
functionality in definition file
\end{itemize}


%%%%%%%%%%%%%%%%%%%%%%%%%%%%%%%%%%%%%%%%%%%%%%%%%%%%%%%%%%%%%%%%%%%%%%%%%%%%%%%%
%%%%%%%%%%%%%%%%%%%%%%%%%%%%%%%%%%%%%%%%%%%%%%%%%%%%%%%%%%%%%%%%%%%%%%%%%%%%%%%%
%%%%%%%%%%%%%%%%%%%%%%%%%%%%%%%%%%%%%%%%%%%%%%%%%%%%%%%%%%%%%%%%%%%%%%%%%%%%%%%%
\appendix

\settowidth\MacroIndent{\rmfamily\scriptsize 000\ }

 \DocInput{childdoc.dtx}

\end{document}
%</driver>
% \fi
%
% %%%%%%%%%%%%%%%%%%%%%%%%%%%%%%%%%%%%%%%%%%%%%%%%%%%%%%%%%%%%%%%%%%%%%%%%%%%%%%
% %%%%%%%%%%%%%%%%%%%%%%%%%%%%%%%%%%%%%%%%%%%%%%%%%%%%%%%%%%%%%%%%%%%%%%%%%%%%%%
% \section{Sample}
%\iffalse
%<*samplemain>
%\fi
%
% The following presents a sample document
% with two chapters, two parts, a title page,
% a compile flag as well as three forwarding files to set the flag.
% It consists of eight |.tex| files:
% \begin{center}
% \begin{tabular}{ll}
% |cdocsamp.tex|&main file\\
% |cdocsch1.tex|&include file for chapter 1\\
% |cdocsch2.tex|&include file for chapter 2\\
% |cdocspt3.tex|&include file for part 3\\
% |cdocspt4.tex|&include file for part 4\\
% |cdocsdrf.tex|&forwarding file for main file in draft mode\\
% |cdocsfi1.tex|&forwarding file for final version of chapter 1\\
% |cdocsfi2.tex|&forwarding file for final version of chapter 2\\
% \end{tabular}
% \end{center}
% Each of the eight files can be compiled directly by the \LaTeX{} compiler.
%
% %%%%%%%%%%%%%%%%%%%%%%%%%%%%%%%%%%%%%%
% \paragraph{Main File.}
%
% The main file is called |cdocsamp.tex|.
%
% Load the \textsf{childdoc} definitions and
% declare the filename for the main document:
%    \begin{macrocode}
\input{childdoc.def}
\childdocmain{}
%    \end{macrocode}

% Optional override for |\version| flag:
%    \begin{macrocode}
%%\ifchilddoc\else\providecommand{\version}{draft}\fi
%    \end{macrocode}

% Define the default values for the |\version| flag
% (|final| for the main file and |draft| for childs):
%    \begin{macrocode}
\ifchilddoc
\providecommand{\version}{draft}
\else
\providecommand{\version}{final}
\fi
%    \end{macrocode}

% Load the standard document class:
%    \begin{macrocode}
\documentclass[12pt]{article}
%    \end{macrocode}

% Start the document body:
%    \begin{macrocode}
\begin{document}
%    \end{macrocode}

% Declare a title page.
% Print title, part of document being processed and version flag:
%    \begin{macrocode}
\addtocounter{page}{-1}
\begin{center}
{\LARGE\bfseries{}childdoc example\par}
\vspace{1cm}
\ifchilddoc
\ifchilddocmanual part\else chapter\fi:
`\childdocname' of `\childdocjob'\par
\else
main document: `\childdocjob'\par
\fi
version: \version\par
\end{center}
\newpage
%    \end{macrocode}

% Manually include selected file,
% otherwise process as usual:
%    \begin{macrocode}
\ifchilddocmanual
\section*{part `\childdocname'}
\input{\childdocname}
\else
%    \end{macrocode}

% Include the two chapters:
%    \begin{macrocode}
\include{cdocsch1}
\include{cdocsch2}
%    \end{macrocode}

% Include the two parts unless only chapters should be displayed:
%    \begin{macrocode}
\ifchilddoc\else
\section{part three}
\input{cdocspt3}
\section{part four}
\input{cdocspt4}
\fi
%    \end{macrocode}

% Process as usual until here:
%    \begin{macrocode}
\fi
%    \end{macrocode}

% End of document body:
%    \begin{macrocode}
\end{document}
%    \end{macrocode}
%\iffalse
%</samplemain>
%\fi
%
% %%%%%%%%%%%%%%%%%%%%%%%%%%%%%%%%%%%%%%
% \paragraph{Chapter Include Files.}
%
% The include files are called |cdocsch1.tex| and |cdocsch2.tex|.
%
%\iffalse
%<*samplechap1|samplechap2>
%\fi

% Optional override for |\version| flag:
%    \begin{macrocode}
%%\providecommand{\version}{final}
%    \end{macrocode}

% Include the main document:
%    \begin{macrocode}
\input{childdoc.def}
\childdocof{cdocsamp}
%    \end{macrocode}

%\iffalse
%</samplechap1|samplechap2>
%\fi
%
%\iffalse
%<*samplechap1>
%\fi
% Some text for chapter 1:
%    \begin{macrocode}
\section{one}
some text in chapter one
%    \end{macrocode}

%\iffalse
%</samplechap1>
%\fi
% Some text for chapter 2:
%\iffalse
%<*samplechap2>
%\fi
%    \begin{macrocode}
\section{two}
more text in chapter two
%    \end{macrocode}

%\iffalse
%</samplechap2>
%\fi
%
% %%%%%%%%%%%%%%%%%%%%%%%%%%%%%%%%%%%%%%
% \paragraph{Part Include Files.}
%
% The include files are called |cdocspt3.tex| and |cdocspt4.tex|.
%
%\iffalse
%<*samplepart3|samplepart4>
%\fi

% Optional override for |\version| flag:
%    \begin{macrocode}
%%\providecommand{\version}{final}
%    \end{macrocode}

% Include the main document:
%    \begin{macrocode}
\input{childdoc.def}
\childdocby{cdocsamp}
%    \end{macrocode}

%\iffalse
%</samplepart3|samplepart4>
%\fi
%
%\iffalse
%<*samplepart3>
%\fi
% Some text for part 3:
%    \begin{macrocode}
some text in part three
%    \end{macrocode}

%\iffalse
%</samplepart3>
%\fi
% Some text for part 4:
%\iffalse
%<*samplepart4>
%\fi
%    \begin{macrocode}
more text in part four
%    \end{macrocode}

%\iffalse
%</samplepart4>
%\fi
%
% %%%%%%%%%%%%%%%%%%%%%%%%%%%%%%%%%%%%%%
% \paragraph{Forwarding for a Complete Draft.}
%
% The following forwarding file |cdocsdrf.tex|
% compiles the main document in draft mode:
%\iffalse
%<*sampledraft>
%\fi
%    \begin{macrocode}
\def\version{draft}
\input{childdoc.def}
\childdocforward{cdocsamp}
%    \end{macrocode}

%\iffalse
%</sampledraft>
%\fi
%
% %%%%%%%%%%%%%%%%%%%%%%%%%%%%%%%%%%%%%%
% \paragraph{Forwarding for Final Version of the Chapters.}
%
% The following forwarding files |cdocsfn1.tex| and |cdocsfn2.tex|
% (with identical content)
% compile the final versions of the child documents
% |cdocsch1.tex| and |cdocsch2.tex|, respectively:
%\iffalse
%<*samplefinal>
%\fi
%    \begin{macrocode}
\def\version{final}
\input{childdoc.def}
\childdocforwardprefix[cdocsamp]{cdocsfn}{cdocsch}
%    \end{macrocode}

%\iffalse
%</samplefinal>
%\fi
%
% %%%%%%%%%%%%%%%%%%%%%%%%%%%%%%%%%%%%%%
% \paragraph{Command Line Processing.}
%
% The following three command lines generate the output files
% |cdocscld|, |cdocscl1| and |cdocscl2|
% which should be identical to
% |cdocsdrf|, |cdocsch1| and |cdocsfn2|, respectively:
% \begin{center}
% \begin{tabular}{l}
% |latex -jobname cdocscld \|\\
% |  "\def\version{draft}\input{childdoc.def}\childdocforward{cdocsamp}"|\\
% |latex -jobname cdocscl1 \|\\
% |  "\input{childdoc.def}\childdocforward[cdocsamp]{cdocsch1}"|\\
% |latex -jobname cdocscl2 \|\\
% |  "\def\version{final}\input{childdoc.def}\childdocforward{cdocsch2}"|
% \end{tabular}
% \end{center}
% Note that the trailing backslash on each first line
% merely continues the input to the second line
% (for convenient cut ant paste).
% Furthermore, the command |latex| can be replaced by any
% of its alternative versions such as |pdflatex|.
%
% %%%%%%%%%%%%%%%%%%%%%%%%%%%%%%%%%%%%%%%%%%%%%%%%%%%%%%%%%%%%%%%%%%%%%%%%%%%%%%
% %%%%%%%%%%%%%%%%%%%%%%%%%%%%%%%%%%%%%%%%%%%%%%%%%%%%%%%%%%%%%%%%%%%%%%%%%%%%%%
% \section{Implementation}
%\iffalse
%<*package>
%\fi
%
% This section describes the definitions file |childdoc.def|.

% The definitions cannot be loaded using |\usepackage| or |\RequirePackage|
% which has a mechanism to prevent loading a style file more than once.
% When loading the definitions by means of |\input|
% multiple instances have to be prevented manually:
%\iffalse
%This code needs to be before the `\ProvidesFile' directive
%which is defined at the beginning of this file.
%Therefore it is also placed there and commented out here.
%</package>
%<*discard>
%\fi
%    \begin{macrocode}
\ifdefined\childdocmain\endinput\fi
%    \end{macrocode}
%\iffalse
%</discard>
%<*package>
%\fi
%
% \macro{\ifchilddoc}
% \macro{\ifchilddocmanual}
% The conditional |\ifchilddoc| tells whether a
% child (true) or main (false) document is being compiled.
% The conditional |\ifchilddocmanual| tells whether
% the |\includeonly| mechanism is used (false) or
% the selection of child files must be performed manually (true).
% The definitions initialise to false:
%    \begin{macrocode}
\newif\ifchilddoc
\newif\ifchilddocmanual
%    \end{macrocode}

% \macro{\childdocname}
% \macro{\childdocjob}
% The macro |\childdocname| stores the name of the main document
% to be compiled. The macro |\childdocjob| stores the name of
% the document on which the \LaTeX{} compiler was originally invoked.
% The content of |\jobname| cannot be compared
% to filenames specified in the source due to different catcodes.
% The following code rescans |\jobname|, stores the result
% in |\childdocname| and saves a copy in |\childdocjob|:
%    \begin{macrocode}
\edef\childdocname{\scantokens\expandafter{\jobname\noexpand}}
\let\childdocjob\childdocname
%    \end{macrocode}

% \macro{\childdocdisable}
% The macro |\childdocdisable| prevents the main file
% from being processed more than once.
% At this stage, the main document command |\childdocmain|
% is assumed to be called once again where it should do nothing.
% Any subsequent call to it should prevent
% a secondary processing of the main document
% It overwrites the forwarding commands
% |\childdocof| and |\childdocforward|
% with empty macros to prevent further inclusions of the main document:
%    \begin{macrocode}
\newcommand{\childdocdisable}
{
  \renewcommand{\childdocmain}[1]{\renewcommand{\childdocmain}[1]{\endinput}}
  \renewcommand{\childdocof}[1]{}
  \renewcommand{\childdocby}[2][]{}
  \renewcommand{\childdocforward}[2][]{}
  \renewcommand{\childdocdisable}{}
}
%    \end{macrocode}

% \macro{\childdocmain}
% The macro |\childdocmain| is to be called at the top of the main file
% with nothing or the main filename (without extension) as argument.
% First, it breaks loops.
% If the argument is not empty and does not match |\childdocname|
% (which is set by the first inclusion of |childdoc.def|),
% |\ifchilddoc| is set to true, |\includeonly| is applied to the child file
% and |\jobname| is set to the main file
% (for proper handling of |.aux| files):
%    \begin{macrocode}
\newcommand{\childdocmain}[1]
{
  \childdocdisable\childdocmain{}
  \if?#1?\else
    \begingroup
      \def\childdoctmp{#1}
      \ifx\childdoctmp\childdocname
        \def\childdoctmp{}
      \else
        \def\childdoctmp
        {
          \childdoctrue
          \includeonly{\childdocname}
          \def\childdocjob{#1}
          \def\jobname{#1}
        }
      \fi
      \expandafter
    \endgroup
    \childdoctmp
  \fi
}
%    \end{macrocode}

% \macro{\childdocof}
% The command |\childdocof| redirects
% compilation to the main file |#1|.
%    \begin{macrocode}
\newcommand{\childdocof}[1]
{
  \childdocdisable
  \childdoctrue
  \includeonly{\childdocname}
  \def\jobname{#1}
  \def\childdocjob{#1}
  \input{#1}
}
%    \end{macrocode}

% \macro{\childdocby}
% The command |\childdocby| ....
%    \begin{macrocode}
\newcommand{\childdocby}[2][]
{
  \childdocdisable
  \childdoctrue
  \childdocmanualtrue
  \if?#1?\else
    \def\jobname{#2}
  \fi
  \def\childdocjob{#2}
  \input{#2}
  \endinput
}
%    \end{macrocode}

% \macro{\childdocforward}
% The command |\childdocforward| redirects
% compilation to the main file or
% (if the optional argument is given) a child file.
% Parameters are set as if the main file
% or a child file starting with |\childdocof| was compiled.
% Then compilation is handed over to the main file:
%    \begin{macrocode}
\newcommand{\childdocforward}[2][]
{
  \begingroup
    \if?#1?
      \def\childdoctmp
      {
        \def\childdocname{#2}
        \def\childdocjob{#2}
        \def\jobname{#2}
        \input{#2}
        \endinput
      }
    \else
      \def\childdoctmp
      {
        \childdocdisable
        \def\childdocname{#2}
        \childdoctrue
        \includeonly{#2}
        \def\childdocjob{#1}
        \def\jobname{#1}
        \input{#1}
        \endinput
      }
    \fi
    \expandafter
  \endgroup
  \childdoctmp
}
%    \end{macrocode}

% \macro{\childdocforwardprefix}
% The command |\childdocforwardprefix| redirects
% compilation to the main or a child file by means of a pattern.
% The prefix |#1| in the current filename is replaced by |#2|
% and the suffix of the current filename is kept
% (it is assumed that the filename does not contain the substring `|~~~|'
% which is used as a delimiter).
% Compilation is handed over to the new file by |\childdocforward|:
%    \begin{macrocode}
\newcommand{\childdocforwardprefix}[3][]
{
  \begingroup
    \def\childdocextract #2##1~~~{\def\childdoctmp{\childdocforward[#1]{#3##1}}}
    \expandafter\childdocextract\childdocname~~~
    \expandafter
  \endgroup
  \childdoctmp
}
%    \end{macrocode}

% \macro{\childdoc}
% The deprecated macro |\childdoc| is a legacy version of |\childdocmain|:
%    \begin{macrocode}
\newcommand{\childdoc}{\childdocmain}
%    \end{macrocode}

% \macro{\childdocredirect}
% The deprecated macro |\childdocredirect| is a legacy version
% of |\childdocforward| and |\childdocforwardprefix|:
%    \begin{macrocode}
\newcommand{\childdocredirect}[2][]
{
  \begingroup
    \if?#1?
      \def\childdoctmp{\childdocforward{#2}}
    \else
      \def\childdoctmp{\childdocforwardprefix{#1}{#2}}
    \fi
    \expandafter
  \endgroup
  \childdoctmp
}
%    \end{macrocode}

%\iffalse
%</package>
%\fi
%
\endinput
\childdocforward{cdocsch2}"|
% \end{tabular}
% \end{center}
% Note that the trailing backslash on each first line
% merely continues the input to the second line
% (for convenient cut ant paste).
% Furthermore, the command |latex| can be replaced by any
% of its alternative versions such as |pdflatex|.
%
% %%%%%%%%%%%%%%%%%%%%%%%%%%%%%%%%%%%%%%%%%%%%%%%%%%%%%%%%%%%%%%%%%%%%%%%%%%%%%%
% %%%%%%%%%%%%%%%%%%%%%%%%%%%%%%%%%%%%%%%%%%%%%%%%%%%%%%%%%%%%%%%%%%%%%%%%%%%%%%
% \section{Implementation}
%\iffalse
%<*package>
%\fi
%
% This section describes the definitions file |childdoc.def|.

% The definitions cannot be loaded using |\usepackage| or |\RequirePackage|
% which has a mechanism to prevent loading a style file more than once.
% When loading the definitions by means of |\input|
% multiple instances have to be prevented manually:
%\iffalse
%This code needs to be before the `\ProvidesFile' directive
%which is defined at the beginning of this file.
%Therefore it is also placed there and commented out here.
%</package>
%<*discard>
%\fi
%    \begin{macrocode}
\ifdefined\childdocmain\endinput\fi
%    \end{macrocode}
%\iffalse
%</discard>
%<*package>
%\fi
%
% \macro{\ifchilddoc}
% \macro{\ifchilddocmanual}
% The conditional |\ifchilddoc| tells whether a
% child (true) or main (false) document is being compiled.
% The conditional |\ifchilddocmanual| tells whether
% the |\includeonly| mechanism is used (false) or
% the selection of child files must be performed manually (true).
% The definitions initialise to false:
%    \begin{macrocode}
\newif\ifchilddoc
\newif\ifchilddocmanual
%    \end{macrocode}

% \macro{\childdocname}
% \macro{\childdocjob}
% The macro |\childdocname| stores the name of the main document
% to be compiled. The macro |\childdocjob| stores the name of
% the document on which the \LaTeX{} compiler was originally invoked.
% The content of |\jobname| cannot be compared
% to filenames specified in the source due to different catcodes.
% The following code rescans |\jobname|, stores the result
% in |\childdocname| and saves a copy in |\childdocjob|:
%    \begin{macrocode}
\edef\childdocname{\scantokens\expandafter{\jobname\noexpand}}
\let\childdocjob\childdocname
%    \end{macrocode}

% \macro{\childdocdisable}
% The macro |\childdocdisable| prevents the main file
% from being processed more than once.
% At this stage, the main document command |\childdocmain|
% is assumed to be called once again where it should do nothing.
% Any subsequent call to it should prevent
% a secondary processing of the main document
% It overwrites the forwarding commands
% |\childdocof| and |\childdocforward|
% with empty macros to prevent further inclusions of the main document:
%    \begin{macrocode}
\newcommand{\childdocdisable}
{
  \renewcommand{\childdocmain}[1]{\renewcommand{\childdocmain}[1]{\endinput}}
  \renewcommand{\childdocof}[1]{}
  \renewcommand{\childdocby}[2][]{}
  \renewcommand{\childdocforward}[2][]{}
  \renewcommand{\childdocdisable}{}
}
%    \end{macrocode}

% \macro{\childdocmain}
% The macro |\childdocmain| is to be called at the top of the main file
% with nothing or the main filename (without extension) as argument.
% First, it breaks loops.
% If the argument is not empty and does not match |\childdocname|
% (which is set by the first inclusion of |childdoc.def|),
% |\ifchilddoc| is set to true, |\includeonly| is applied to the child file
% and |\jobname| is set to the main file
% (for proper handling of |.aux| files):
%    \begin{macrocode}
\newcommand{\childdocmain}[1]
{
  \childdocdisable\childdocmain{}
  \if?#1?\else
    \begingroup
      \def\childdoctmp{#1}
      \ifx\childdoctmp\childdocname
        \def\childdoctmp{}
      \else
        \def\childdoctmp
        {
          \childdoctrue
          \includeonly{\childdocname}
          \def\childdocjob{#1}
          \def\jobname{#1}
        }
      \fi
      \expandafter
    \endgroup
    \childdoctmp
  \fi
}
%    \end{macrocode}

% \macro{\childdocof}
% The command |\childdocof| redirects
% compilation to the main file |#1|.
%    \begin{macrocode}
\newcommand{\childdocof}[1]
{
  \childdocdisable
  \childdoctrue
  \includeonly{\childdocname}
  \def\jobname{#1}
  \def\childdocjob{#1}
  \input{#1}
}
%    \end{macrocode}

% \macro{\childdocby}
% The command |\childdocby| ....
%    \begin{macrocode}
\newcommand{\childdocby}[2][]
{
  \childdocdisable
  \childdoctrue
  \childdocmanualtrue
  \if?#1?\else
    \def\jobname{#2}
  \fi
  \def\childdocjob{#2}
  \input{#2}
  \endinput
}
%    \end{macrocode}

% \macro{\childdocforward}
% The command |\childdocforward| redirects
% compilation to the main file or
% (if the optional argument is given) a child file.
% Parameters are set as if the main file
% or a child file starting with |\childdocof| was compiled.
% Then compilation is handed over to the main file:
%    \begin{macrocode}
\newcommand{\childdocforward}[2][]
{
  \begingroup
    \if?#1?
      \def\childdoctmp
      {
        \def\childdocname{#2}
        \def\childdocjob{#2}
        \def\jobname{#2}
        \input{#2}
        \endinput
      }
    \else
      \def\childdoctmp
      {
        \childdocdisable
        \def\childdocname{#2}
        \childdoctrue
        \includeonly{#2}
        \def\childdocjob{#1}
        \def\jobname{#1}
        \input{#1}
        \endinput
      }
    \fi
    \expandafter
  \endgroup
  \childdoctmp
}
%    \end{macrocode}

% \macro{\childdocforwardprefix}
% The command |\childdocforwardprefix| redirects
% compilation to the main or a child file by means of a pattern.
% The prefix |#1| in the current filename is replaced by |#2|
% and the suffix of the current filename is kept
% (it is assumed that the filename does not contain the substring `|~~~|'
% which is used as a delimiter).
% Compilation is handed over to the new file by |\childdocforward|:
%    \begin{macrocode}
\newcommand{\childdocforwardprefix}[3][]
{
  \begingroup
    \def\childdocextract #2##1~~~{\def\childdoctmp{\childdocforward[#1]{#3##1}}}
    \expandafter\childdocextract\childdocname~~~
    \expandafter
  \endgroup
  \childdoctmp
}
%    \end{macrocode}

% \macro{\childdoc}
% The deprecated macro |\childdoc| is a legacy version of |\childdocmain|:
%    \begin{macrocode}
\newcommand{\childdoc}{\childdocmain}
%    \end{macrocode}

% \macro{\childdocredirect}
% The deprecated macro |\childdocredirect| is a legacy version
% of |\childdocforward| and |\childdocforwardprefix|:
%    \begin{macrocode}
\newcommand{\childdocredirect}[2][]
{
  \begingroup
    \if?#1?
      \def\childdoctmp{\childdocforward{#2}}
    \else
      \def\childdoctmp{\childdocforwardprefix{#1}{#2}}
    \fi
    \expandafter
  \endgroup
  \childdoctmp
}
%    \end{macrocode}

%\iffalse
%</package>
%\fi
%
\endinput
|\\
|\childdocby{|\textit{main}|}|\\
\end{tabular}
\end{center}
%
The directive |\childdocby| is similar to |\childdocof|
described in \secref{sec:include},
but the subsequent selection of content must be done manually.
To that end, both |\ifchilddoc| and |\ifchilddocmanual|
will be true upon processing of a part,
and the name of the part is stored in |\childdocname|.
Note that |\jobname| will be set to the filename of the current part
so that each part receives an individual |.aux| file
that does not interfere with the |.aux| file(s) of the main document.
This behaviour can be altered by the alternative form
|\childdocby[*]{|\textit{main}|}| (with a non-empty optional argument)
which uses the |.aux| file of the main document
by setting |\jobname| to \textit{main}.

%%%%%%%%%%%%%%%%%%%%%%%%%%%%%%%%%%%%%%%%%%%%%%%%%%%%%%%%%%%%%%%%%%%%%%%%%%%%%%%%
\subsection{Driver Development}
\label{sec:driver}

The \textsf{childdoc} mechanism can also be use for the development
of definition files such as \LaTeX{} styles or classes.
This case differs from the above setup with multiple parts
included by |\include| in that no |\includeonly| should be invoked.
This can be achieved by starting the include file
(before |\ProvidesPackage|) with:
%
\begin{center}
\begin{tabular}{l}
|% \iffalse
%
% childdoc.dtx Copyright (C) 2017-2018 Niklas Beisert
%
% This work may be distributed and/or modified under the
% conditions of the LaTeX Project Public License, either version 1.3
% of this license or (at your option) any later version.
% The latest version of this license is in
%   http://www.latex-project.org/lppl.txt
% and version 1.3 or later is part of all distributions of LaTeX
% version 2005/12/01 or later.
%
% This work has the LPPL maintenance status `maintained'.
%
% The Current Maintainer of this work is Niklas Beisert.
%
% This work consists of the files childdoc.dtx and childdoc.ins
% and the derived files childdoc.def and cdocsamp.tex with
% cdocsch1.tex, cdocsch2.tex, cdocsdrf.tex, cdocsfn1.tex, cdocsfn2.tex.
%
%<package>\ifdefined\childdocmain\endinput\fi
%<package>\ProvidesFile{childdoc.def}[2018/12/30 v2.0 child document driver]
%<samplemain>\ProvidesFile{cdocsamp.tex}[2018/12/30 v2.0 sample for childdoc]
%<*driver>
%\ProvidesFile{childdoc.drv}[2018/12/30 v2.0 childdoc reference manual file]
\PassOptionsToClass{10pt,a4paper}{article}
\documentclass{ltxdoc}

\usepackage[margin=35mm]{geometry}
\usepackage{hyperref}
\usepackage{hyperxmp}
\usepackage[usenames]{color}

\hypersetup{colorlinks=true}
\hypersetup{pdfstartview=FitH}
\hypersetup{pdfpagemode=UseNone}
\hypersetup{pdfsource={}}
\hypersetup{pdflang={en-UK}}
\hypersetup{pdfcopyright={Copyright 2017-2018 Niklas Beisert.
  This work may be distributed and/or modified under the
  conditions of the LaTeX Project Public License, either version 1.3
  of this license or (at your option) any later version.}}
\hypersetup{pdflicenseurl={http://www.latex-project.org/lppl.txt}}
\hypersetup{pdfcontactaddress={ETH Zurich, ITP, HIT K,
  Wolfgang-Pauli-Strasse 27}}
\hypersetup{pdfcontactpostcode={8093}}
\hypersetup{pdfcontactcity={Zurich}}
\hypersetup{pdfcontactcountry={Switzerland}}
\hypersetup{pdfcontactemail={nbeisert@itp.phys.ethz.ch}}
\hypersetup{pdfcontacturl={http://people.phys.ethz.ch/\xmptilde nbeisert/}}

\newcommand{\secref}[1]{\hyperref[#1]{section \ref*{#1}}}

\parskip1ex
\parindent0pt
\let\olditemize\itemize
\def\itemize{\olditemize\parskip0pt}

\begin{document}

\title{The \textsf{childdoc} Package}
\hypersetup{pdftitle={The childdoc Package}}
\author{Niklas Beisert\\[2ex]
  Institut f\"ur Theoretische Physik\\
  Eidgen\"ossische Technische Hochschule Z\"urich\\
  Wolfgang-Pauli-Strasse 27, 8093 Z\"urich, Switzerland\\[1ex]
  \href{mailto:nbeisert@itp.phys.ethz.ch}
  {\texttt{nbeisert@itp.phys.ethz.ch}}}
\hypersetup{pdfauthor={Niklas Beisert}}
\hypersetup{pdfsubject={Manual for the LaTeX2e Package childdoc}}
\date{30 December 2018, \textsf{v2.0}}
\maketitle

\begin{abstract}\noindent
\textsf{childdoc} is a \LaTeXe{} package
that enables the direct compilation
of document sections included by |\include|
to individual files.
\end{abstract}

\begingroup
\parskip0ex
\tableofcontents
\endgroup

%%%%%%%%%%%%%%%%%%%%%%%%%%%%%%%%%%%%%%%%%%%%%%%%%%%%%%%%%%%%%%%%%%%%%%%%%%%%%%%%
%%%%%%%%%%%%%%%%%%%%%%%%%%%%%%%%%%%%%%%%%%%%%%%%%%%%%%%%%%%%%%%%%%%%%%%%%%%%%%%%
\section{Introduction}

\LaTeX{} provides a mechanism to structure a large document (such as a book)
into a main file and several child files (containing the chapters)
using the |\include| command.
This mechanism is beneficial for documents
which span hundreds of pages in order to
make the source file(s) more manageable.
Moreover, compilation can be restricted to
selected child files by means of the |\includeonly| command.
The latter feature can be used to reduce the compilation time while editing
(this was significantly more useful in the earlier days of \LaTeX{})
or to generate a smaller document which is easier to navigate.
Another application of |\includeonly| is to generate
documents consisting of selected parts of the complete document.

However, there are a few drawbacks of the plain |\include| mechanism:
\begin{itemize}
\item
The child files cannot be compiled on their own,
they can only be compiled via the main file.
A naive editing environment
(such as a text editor with an option
to have the current file processed by \LaTeX)
may require one to switch to the main file before compiling;
attempting to compile the child file produces errors.
\item
The main file must be modified (each time)
to adjust the |\includeonly| command
to the present needs. This easily leaves the main file in a messy state.
\item
The generated document will always carry the filename
of the main document. This is inconvenient if
several child files are to be compiled and
to be kept for distribution.
\end{itemize}

The present package provides a simple interface
to make child files individually compilable by \LaTeX{}.
Compiling a child file then has the same effect as compiling
the main file with an |\includeonly| command
to select the appropriate child.
Moreover the generated document will carry the name of the child
rather than the main file.
This resolves all three above issues.

This feature is meant to make the editing of books,
thesis documents and lecture notes somewhat more convenient.
However, the package can also be used efficiently for
composing a series of documents (such as exercise sheets)
which are typically distributed individually.
It then assists the author in generating the individual documents
(potentially in different versions)
as well as a document containing the collected series.
Another application is in developing style files
or other kinds of included material
where compilation of the style file could redirect
to a sample or test file.

%%%%%%%%%%%%%%%%%%%%%%%%%%%%%%%%%%%%%%%%%%%%%%%%%%%%%%%%%%%%%%%%%%%%%%%%%%%%%%%%
%%%%%%%%%%%%%%%%%%%%%%%%%%%%%%%%%%%%%%%%%%%%%%%%%%%%%%%%%%%%%%%%%%%%%%%%%%%%%%%%
\section{Usage}

First of all, the package \textsf{childdoc} is \emph{not} a standard
\LaTeXe{} |.sty| style file! Therefore it needs to be invoked in
a non-standard way.

%%%%%%%%%%%%%%%%%%%%%%%%%%%%%%%%%%%%%%%%%%%%%%%%%%%%%%%%%%%%%%%%%%%%%%%%%%%%%%%%
\subsection{Included Files}
\label{sec:include}

%%%%%%%%%%%%%%%%%%%%%%%%%%%%%%%%%%%%%%%%
\DescribeMacro{\childdocmain}
To use the package, add the commands
\begin{center}
\begin{tabular}{l}
|% \iffalse
%
% childdoc.dtx Copyright (C) 2017-2018 Niklas Beisert
%
% This work may be distributed and/or modified under the
% conditions of the LaTeX Project Public License, either version 1.3
% of this license or (at your option) any later version.
% The latest version of this license is in
%   http://www.latex-project.org/lppl.txt
% and version 1.3 or later is part of all distributions of LaTeX
% version 2005/12/01 or later.
%
% This work has the LPPL maintenance status `maintained'.
%
% The Current Maintainer of this work is Niklas Beisert.
%
% This work consists of the files childdoc.dtx and childdoc.ins
% and the derived files childdoc.def and cdocsamp.tex with
% cdocsch1.tex, cdocsch2.tex, cdocsdrf.tex, cdocsfn1.tex, cdocsfn2.tex.
%
%<package>\ifdefined\childdocmain\endinput\fi
%<package>\ProvidesFile{childdoc.def}[2018/12/30 v2.0 child document driver]
%<samplemain>\ProvidesFile{cdocsamp.tex}[2018/12/30 v2.0 sample for childdoc]
%<*driver>
%\ProvidesFile{childdoc.drv}[2018/12/30 v2.0 childdoc reference manual file]
\PassOptionsToClass{10pt,a4paper}{article}
\documentclass{ltxdoc}

\usepackage[margin=35mm]{geometry}
\usepackage{hyperref}
\usepackage{hyperxmp}
\usepackage[usenames]{color}

\hypersetup{colorlinks=true}
\hypersetup{pdfstartview=FitH}
\hypersetup{pdfpagemode=UseNone}
\hypersetup{pdfsource={}}
\hypersetup{pdflang={en-UK}}
\hypersetup{pdfcopyright={Copyright 2017-2018 Niklas Beisert.
  This work may be distributed and/or modified under the
  conditions of the LaTeX Project Public License, either version 1.3
  of this license or (at your option) any later version.}}
\hypersetup{pdflicenseurl={http://www.latex-project.org/lppl.txt}}
\hypersetup{pdfcontactaddress={ETH Zurich, ITP, HIT K,
  Wolfgang-Pauli-Strasse 27}}
\hypersetup{pdfcontactpostcode={8093}}
\hypersetup{pdfcontactcity={Zurich}}
\hypersetup{pdfcontactcountry={Switzerland}}
\hypersetup{pdfcontactemail={nbeisert@itp.phys.ethz.ch}}
\hypersetup{pdfcontacturl={http://people.phys.ethz.ch/\xmptilde nbeisert/}}

\newcommand{\secref}[1]{\hyperref[#1]{section \ref*{#1}}}

\parskip1ex
\parindent0pt
\let\olditemize\itemize
\def\itemize{\olditemize\parskip0pt}

\begin{document}

\title{The \textsf{childdoc} Package}
\hypersetup{pdftitle={The childdoc Package}}
\author{Niklas Beisert\\[2ex]
  Institut f\"ur Theoretische Physik\\
  Eidgen\"ossische Technische Hochschule Z\"urich\\
  Wolfgang-Pauli-Strasse 27, 8093 Z\"urich, Switzerland\\[1ex]
  \href{mailto:nbeisert@itp.phys.ethz.ch}
  {\texttt{nbeisert@itp.phys.ethz.ch}}}
\hypersetup{pdfauthor={Niklas Beisert}}
\hypersetup{pdfsubject={Manual for the LaTeX2e Package childdoc}}
\date{30 December 2018, \textsf{v2.0}}
\maketitle

\begin{abstract}\noindent
\textsf{childdoc} is a \LaTeXe{} package
that enables the direct compilation
of document sections included by |\include|
to individual files.
\end{abstract}

\begingroup
\parskip0ex
\tableofcontents
\endgroup

%%%%%%%%%%%%%%%%%%%%%%%%%%%%%%%%%%%%%%%%%%%%%%%%%%%%%%%%%%%%%%%%%%%%%%%%%%%%%%%%
%%%%%%%%%%%%%%%%%%%%%%%%%%%%%%%%%%%%%%%%%%%%%%%%%%%%%%%%%%%%%%%%%%%%%%%%%%%%%%%%
\section{Introduction}

\LaTeX{} provides a mechanism to structure a large document (such as a book)
into a main file and several child files (containing the chapters)
using the |\include| command.
This mechanism is beneficial for documents
which span hundreds of pages in order to
make the source file(s) more manageable.
Moreover, compilation can be restricted to
selected child files by means of the |\includeonly| command.
The latter feature can be used to reduce the compilation time while editing
(this was significantly more useful in the earlier days of \LaTeX{})
or to generate a smaller document which is easier to navigate.
Another application of |\includeonly| is to generate
documents consisting of selected parts of the complete document.

However, there are a few drawbacks of the plain |\include| mechanism:
\begin{itemize}
\item
The child files cannot be compiled on their own,
they can only be compiled via the main file.
A naive editing environment
(such as a text editor with an option
to have the current file processed by \LaTeX)
may require one to switch to the main file before compiling;
attempting to compile the child file produces errors.
\item
The main file must be modified (each time)
to adjust the |\includeonly| command
to the present needs. This easily leaves the main file in a messy state.
\item
The generated document will always carry the filename
of the main document. This is inconvenient if
several child files are to be compiled and
to be kept for distribution.
\end{itemize}

The present package provides a simple interface
to make child files individually compilable by \LaTeX{}.
Compiling a child file then has the same effect as compiling
the main file with an |\includeonly| command
to select the appropriate child.
Moreover the generated document will carry the name of the child
rather than the main file.
This resolves all three above issues.

This feature is meant to make the editing of books,
thesis documents and lecture notes somewhat more convenient.
However, the package can also be used efficiently for
composing a series of documents (such as exercise sheets)
which are typically distributed individually.
It then assists the author in generating the individual documents
(potentially in different versions)
as well as a document containing the collected series.
Another application is in developing style files
or other kinds of included material
where compilation of the style file could redirect
to a sample or test file.

%%%%%%%%%%%%%%%%%%%%%%%%%%%%%%%%%%%%%%%%%%%%%%%%%%%%%%%%%%%%%%%%%%%%%%%%%%%%%%%%
%%%%%%%%%%%%%%%%%%%%%%%%%%%%%%%%%%%%%%%%%%%%%%%%%%%%%%%%%%%%%%%%%%%%%%%%%%%%%%%%
\section{Usage}

First of all, the package \textsf{childdoc} is \emph{not} a standard
\LaTeXe{} |.sty| style file! Therefore it needs to be invoked in
a non-standard way.

%%%%%%%%%%%%%%%%%%%%%%%%%%%%%%%%%%%%%%%%%%%%%%%%%%%%%%%%%%%%%%%%%%%%%%%%%%%%%%%%
\subsection{Included Files}
\label{sec:include}

%%%%%%%%%%%%%%%%%%%%%%%%%%%%%%%%%%%%%%%%
\DescribeMacro{\childdocmain}
To use the package, add the commands
\begin{center}
\begin{tabular}{l}
|\input{childdoc.def}|\\
|\childdocmain{}|\\
\end{tabular}
\end{center}
at the very top of the main \LaTeX{} file,
in particular \emph{before} the |\documentclass| statement!
The argument of |\childdocmain| should be left empty
(but it must be present).

%%%%%%%%%%%%%%%%%%%%%%%%%%%%%%%%%%%%%%%%
\DescribeMacro{\childdocof}
Furthermore, add the commands
\begin{center}
\begin{tabular}{l}
|\input{childdoc.def}|\\
|\childdocof{|\textit{main}|}|\\
\end{tabular}
\end{center}
at the top of every child file \textit{child}
which is included by |\include{|\textit{child}|}|
from within the main file
(or at least for those files to be compiled individually).
The argument \textit{main} must be the filename of the main file.

There are a couple of
considerations in setting up the main and child documents:

%%%%%%%%%%%%%%%%%%%%%%%%%%%%%%%%%%%%%%%%
\paragraph{Restrictions.}

Please note the following restrictions:
\begin{itemize}
\item
|\childdocmain| must be called with one argument \textit{main}
to ensure compatibility with earlier version of the package.
It must either be empty (|\childdocmain{}|)
or precisely match the filename of the main file in which it is specified.
See \secref{sec:detection} for further information.
\item
The filename \textit{main} must be specified without the |.tex| extension.
\item
The filename \textit{main} is case sensitive
(even in case-insensitive file systems)
due to internal string comparison.
\item
The argument \textit{main} should be fully expanded, it cannot be a macro.
\item
Subdirectories and special characters should be avoided in filenames.
\item
The command |\childdocmain{|\textit{main}|}| must be followed by a whitespace.
It should not be followed immediately by another command
or by a comment mark `|%|'.
This is because the \TeX{} parser reads the token immediately following
the argument of |\childdocmain| and puts it
at the beginning of every child section;
however, a white\-space is ignored.
\end{itemize}

%%%%%%%%%%%%%%%%%%%%%%%%%%%%%%%%%%%%%%%%
\paragraph{Content of Main File.}

It is advisable to place all content in the child files included by |\include|.
Any output contained in the main file will appear in all child documents
unless suppressed manually;
it cannot be suppressed automatically by the |\includeonly| directive
and thus should normally be avoided.
A method to include some content in the main file
by means of conditional processing is described in \secref{sec:conditional}.

%%%%%%%%%%%%%%%%%%%%%%%%%%%%%%%%%%%%%%%%
\paragraph{Page Numbering.}

When only a part of the document is compiled,
the appropriate numbering of pages
(as well as other status parameters)
is determined from the |.aux| files.
The latter contain information from previous passes.
However this information needs to propagate through
all intermediate child documents.
Therefore the page numbering in child documents may well
be inconsistent until the complete document is compiled at least once.

A useful (if unconventional) way to always ensure a consistent
page numbering is to restart the numbering in each child document
and denote the pages by `\textit{child}|.|\textit{page}'
where \textit{child} represents the chapter/section number of the child file.
This can be achieved by the command
|\numberwithin{page}{|\textit{child}|}|
of the \textsf{amsmath} package
where \textit{child} can be |chapter| or |section|
depending on the chosen structuring.
Alternatively, one can modify the macro |\thepage| appropriately
and reset the counter |page| at the start of each child file.

%%%%%%%%%%%%%%%%%%%%%%%%%%%%%%%%%%%%%%%%%%%%%%%%%%%%%%%%%%%%%%%%%%%%%%%%%%%%%%%%
\subsection{Conditional Processing}
\label{sec:conditional}

The package provides a mechanism to compile different versions
of a document. To customise the versions further some conditional processing
can come in handy to distinguish which version is being compiled.
The package provides two macros to describe the compilation context:

%%%%%%%%%%%%%%%%%%%%%%%%%%%%%%%%%%%%%%%%
\DescribeMacro{\ifchilddoc}
The conditional |\ifchilddoc| distinguishes between the compilation of
child documents and the main document:
%
\begin{center}
|\ifchilddoc |\textit{child-code}| |[|\||else |\textit{main-code}]| \||fi|
\end{center}

%%%%%%%%%%%%%%%%%%%%%%%%%%%%%%%%%%%%%%%%
\DescribeMacro{\childdocname}
\DescribeMacro{\childdocjob}
The macro |\childdocname| contains the filename (without extension)
of the main or child file being processed.
Note that |\childdocjob| will always contain the name of the main file.

%%%%%%%%%%%%%%%%%%%%%%%%%%%%%%%%%%%%%%%%
\paragraph{Title Page.}

Conditional processing can be used to include a title or banner page
in the main document when proper precautions are taken.
Importantly, the code in the main file should ensure that the page counter
(as well as other status parameters which are stored in the |.aux| files)
takes the same value after the conditional processing.
Otherwise the page numbers may take divergent values
depending on which part is compiled.

For example, a title page could be declared by:
%
\begin{center}
\begin{tabular}{l}
|\ifchilddoc\||else|\\
|\addtocounter{page}{-1}|\\
\textit{code for title page}\\
|\newpage|\\
|\||fi|
\end{tabular}
\end{center}
%
A banner page for the child documents can be generated by:
%
\begin{center}
\begin{tabular}{l}
|\ifchilddoc|\\
|\addtocounter{page}{-1}|\\
\textit{code for banner page}\\
|\newpage|\\
|\||fi|
\end{tabular}
\end{center}
%
Here one could write a message such as:
\begin{center}
|This is the part \childdocname{} of \childdocjob{}.|
\end{center}

%%%%%%%%%%%%%%%%%%%%%%%%%%%%%%%%%%%%%%%%%%%%%%%%%%%%%%%%%%%%%%%%%%%%%%%%%%%%%%%%
\subsection{Flags}
\label{sec:flags}

The package makes it easy to generate different versions
of the main or child documents.
To this end compilation flags can be defined
and assigned different default values.
They will be particularly useful in conjunction
with the forwarding mechanism described in \secref{sec:forward}.

For example, it may be useful to have a flag |\version|
which can be set to |draft| or |final|.
The document source will contain some conditional code
depending on the value of |\version|.
Suppose further, the flag should default to |final| for the main file
and to |draft| for child files
which is a natural assignment for editing the document.
This is achieved by placing the following code
in the preamble of the main document
(below the |\childdocmain| directive):
%
\begin{center}
\begin{tabular}{l}
|\ifchilddoc|\\
|\providecommand{\version}{draft}|\\
|\||else|\\
|\providecommand{\version}{final}|\\
|\||fi|
\end{tabular}
\end{center}
%
The definition by |\providecommand| makes sure
that previous definitions are not overwritten.
Further statements |\providecommand{\version}{...}|
can thus be added before the above code to override it.

For the main file, one might add a line
(between |\childdocmain| and the above block)
%
\begin{center}
|%\ifchilddoc\||else\providecommand{\version}{draft}\||fi|
\end{center}
%
which can be uncommented to produce a draft version.
Likewise one can add a line to the very top of a child file
(above the |\childdocof{|\textit{main}|}| directive)
%
\begin{center}
|%\providecommand{\version}{final}|
\end{center}
%
which can be uncommented to produce the final version of this child document.

%%%%%%%%%%%%%%%%%%%%%%%%%%%%%%%%%%%%%%%%%%%%%%%%%%%%%%%%%%%%%%%%%%%%%%%%%%%%%%%%
\subsection{Forwarding}
\label{sec:forward}

Different versions of the main or child documents
using compilation flags as described in \secref{sec:flags}
can be (permanently) stored in different files
for convenient compilation, viewing and distribution.
To this end, the package defines a command
to pass on compilation to a different file:

%%%%%%%%%%%%%%%%%%%%%%%%%%%%%%%%%%%%%%%%
\DescribeMacro{\childdocforward}
The command |\childdocforward| redirects processing to
another source file:
%
\begin{center}
\begin{tabular}{l}
|\input{childdoc.def}|\\
|\childdocforward[|\textit{main}|]{|\textit{dest}|}|\\
\end{tabular}
\end{center}
%
The argument \textit{dest} is the destination file
(without extension).
It should be the main file or one of the child files.
Note that further \textsf{childdoc} directives
such as |\childdocof| and |\childdocforward|
in the indicated file will be processed in this form.
The optional argument \textit{main}
passes on directly to the main file \textit{main}
while pretending to compile the child \textit{dest}.
This form behaves as if \textit{dest}
issues |\childdocof{|\textit{main}|}| right away,
and no further \textsf{childdoc} directives will be processed.

%%%%%%%%%%%%%%%%%%%%%%%%%%%%%%%%%%%%%%%%
\DescribeMacro{\...prefix}
In the alternative form |\childdocforwardprefix|,
%
\begin{center}
\begin{tabular}{l}
|\input{childdoc.def}|\\
|\childdocforwardprefix[|\textit{main}|]{|\textit{prefix}|}{|\textit{dest}|}|
\end{tabular}
\end{center}
%
the destination file is determined by a pattern
depending on the current file:
To make this work, the current file must be called
`{\textit{prefix}\hspace{0.2em}\textit{suffix}}'
with \textit{prefix} matching precisely the argument.
Processing is then passed on to the file
`{\textit{dest}\hspace{0.2em}\textit{suffix}}'.
Surely, the same effect is achieved by
directly specifying the
argument `{\textit{dest}\hspace{0.2em}\textit{suffix}}'
in the first form.
However, that requires to set up a different file
for each child. With the alternative form of the command
all these files can have exactly the same content
which simplifies setting them up and maintaining them.

For example, the following file |draft.tex|
with a compilation flag |\version| as described in \secref{sec:flags}
compiles the main document as a draft:
%
\begin{center}
\begin{tabular}{l}
|\def\version{draft}|\\
|\input{childdoc.def}|\\
|\childdocforward{|\textit{main}|}|
\end{tabular}
\end{center}
%
Likewise, the following files |final|\textit{nn}|.tex|
compile the final version of the child document
|child|\textit{nn}|.tex|:
%
\begin{center}
\begin{tabular}{l}
|\def\version{final}|\\
|\input{childdoc.def}|\\
|\childdocforwardprefix{final}{child}|
\end{tabular}
\end{center}
%

Note that when several versions of a main file and/or of each child file
are to be generated, it may be convenient to set up a |Makefile| or
shell script to automatise the process.

%%%%%%%%%%%%%%%%%%%%%%%%%%%%%%%%%%%%%%%%%%%%%%%%%%%%%%%%%%%%%%%%%%%%%%%%%%%%%%%%
\subsection{Command Line Processing}
\label{sec:commandline}

The effect of redirection files can also be achieved by invoking
the \LaTeX{} compiler with a more elaborate command line.
Most conveniently this should be done as part
of a shell script or a |Makefile|.

When using \textsf{childdoc} in the main file, the following
command lines effectively perform a redirection
(note that depending on the shell being used,
backslashes may have to be doubled: `|\|' $\to$ `|\\|'):
%
\begin{center}
|... -jobname "|\textit{target}|" |\\|"|[\textit{flags}]%
|\input{childdoc.def}\childdocforward[|\textit{main}|]{|\textit{dest}|}"|
\end{center}
%
Here \textit{target} is the name of the output file,
\textit{main} is the name of the main file
and \textit{dest} is the name of the main or child file to be processed
(all filenames without extensions).
The optional argument \textit{main} can be omitted
if \textit{main} matches \textit{dest}.
Optionally, compilation \textit{flags} can be defined via |\def| commands.
This command line makes the \TeX{} engine believe
it is compiling the file \textit{target}
whose content is specified as the latter parameter.
The provided code then forwards the processing to
\textit{main} or \textit{dest} as described in \secref{sec:forward}.

%%%%%%%%%%%%%%%%%%%%%%%%%%%%%%%%%%%%%%%%%%%%%%%%%%%%%%%%%%%%%%%%%%%%%%%%%%%%%%%%
\subsection{Include by Input}
\label{sec:input}

Including child documents by |\include| has some restrictions by design.
Most notably, the content of a child document always occupies
its own set of pages; pages cannot be shared between child documents.
Usually, this behaviour makes perfect sense
because each child document contain an essential part of the document.
However, in some situations it may be desirable to compose
a document from a collection of parts
without having mandatory page breaks between then.
For this case, the package
provides a mechanism to include parts
by |\input| which can also be processed individually.
However, by construction this mechanism
requires manual handling of the content to be output.

%%%%%%%%%%%%%%%%%%%%%%%%%%%%%%%%%%%%%%%%
\DescribeMacro{\ifchilddocmanual}
The main file should be prepared as usual, see \secref{sec:include}.
However, the document body must make a distinction
between processing of an individual part and of the main document, e.g.:
%
\begin{center}
\begin{tabular}{l}
|\ifchilddocmanual|\\
|\input{\childdocname}|\\
|\||else|\\
\textit{document body with }|\input{|\textit{part}|}|\\
|\||fi|
\end{tabular}
\end{center}
%
The conditional |\ifchilddocmanual| is true whenever
a part to be included by |\input| is being compiled,
and the name of the part is stored in |\childdocname|.

%%%%%%%%%%%%%%%%%%%%%%%%%%%%%%%%%%%%%%%%
\DescribeMacro{\childdocby}
Each part to be included by |\input| should start with:
%
\begin{center}
\begin{tabular}{l}
|\input{childdoc.def}|\\
|\childdocby{|\textit{main}|}|\\
\end{tabular}
\end{center}
%
The directive |\childdocby| is similar to |\childdocof|
described in \secref{sec:include},
but the subsequent selection of content must be done manually.
To that end, both |\ifchilddoc| and |\ifchilddocmanual|
will be true upon processing of a part,
and the name of the part is stored in |\childdocname|.
Note that |\jobname| will be set to the filename of the current part
so that each part receives an individual |.aux| file
that does not interfere with the |.aux| file(s) of the main document.
This behaviour can be altered by the alternative form
|\childdocby[*]{|\textit{main}|}| (with a non-empty optional argument)
which uses the |.aux| file of the main document
by setting |\jobname| to \textit{main}.

%%%%%%%%%%%%%%%%%%%%%%%%%%%%%%%%%%%%%%%%%%%%%%%%%%%%%%%%%%%%%%%%%%%%%%%%%%%%%%%%
\subsection{Driver Development}
\label{sec:driver}

The \textsf{childdoc} mechanism can also be use for the development
of definition files such as \LaTeX{} styles or classes.
This case differs from the above setup with multiple parts
included by |\include| in that no |\includeonly| should be invoked.
This can be achieved by starting the include file
(before |\ProvidesPackage|) with:
%
\begin{center}
\begin{tabular}{l}
|\input{childdoc.def}|\\
|\childdocforward{|\textit{main}|}|\\
\end{tabular}
\end{center}
%
or alternatively with:
%
\begin{center}
\begin{tabular}{l}
|\input{childdoc.def}|\\
|\childdocby{|\textit{main}|}|\\
\end{tabular}
\end{center}
%
Both forms have slightly different effects as described above.
The main file is prepared as usual, see \secref{sec:include}.

%%%%%%%%%%%%%%%%%%%%%%%%%%%%%%%%%%%%%%%%%%%%%%%%%%%%%%%%%%%%%%%%%%%%%%%%%%%%%%%%
\subsection{Legacy Detection}
\label{sec:detection}

The directive |\childdocmain| in the main file can detect
whether the complete document or merely a child is to be compiled
even without using the directive |\childdocof|.
This method is deprecated because it is less robust
and there is no compelling reason to use it;
it is merely provided for backward compatibility
and it may be removed in future versions.

If the detection mechanism is to be used,
it is mandatory to correctly specify
the filename of the main file as the argument of |\childdocmain|:
%
\begin{center}
\begin{tabular}{l}
|\input{childdoc.def}|\\
|\childdocmain{|\textit{main}|}|\\
\end{tabular}
\end{center}
%
If |\jobname| does not match the argument \textit{main} of |\childdocmain|,
it is assumed that |\jobname| points to the child file to be compiled.
When using |\childdocmain| with the main file specified as argument,
it suffices to start a child file
with just |\input{|\textit{main}|}|
without loading of the package and using |\childdocof|.
If instead all processing is done
with the appropriate \textsf{childdoc} directives,
the argument of \textit{main} of |\childdocmain| can be empty.

An alternative version of the command line processing described
in \secref{sec:commandline} using the detection mechanism reads:
%
\begin{center}
|... -jobname "|\textit{target}|" "|[\textit{flags}]%
[|\def\jobname{|\textit{dest}|}|]|\input{|\textit{main}|}"|
\end{center}

%%%%%%%%%%%%%%%%%%%%%%%%%%%%%%%%%%%%%%%%%%%%%%%%%%%%%%%%%%%%%%%%%%%%%%%%%%%%%%%%
\subsection{Manual Code}
\label{sec:manual}

In case one cannot be certain whether the definitions file |childdoc.def|
is installed on the target \TeX{} distribution
and one prefers not to ship it,
it is conceivable to paste a few relevant commands into the sources.

To that end, drop all statements |\input{childdoc.def}|
and perform the replacements as outlined below.
Instead of |\childdocmain{|\textit{main}|}| add the following code
to the top of the main file:
%
\begin{center}
\begin{tabular}{l}
|\||ifdefined\childdocname\endinput\||fi\newif\ifchilddoc|\\
|\edef\childdocname{\scantokens\expandafter{\jobname\noexpand}}|\\
|\def\childdocmain{|\textit{main}|}\||ifx\childdocmain\childdocname\||else|\\
|\childdoctrue\includeonly{\childdocname}\let\jobname\childdocmain\||fi|\\
\end{tabular}
\end{center}
%
Instead of |\childdocof{|\textit{main}|}| just include the main file
at the top of each child file:
%
\begin{center}
|\input{|\textit{main}|}|
\end{center}
%
A simple redirection |\childdocforward{|\textit{dest}|}| is achieved by:
%
\begin{center}
|\def\jobname{|\textit{dest}|}\input{\jobname}|
\end{center}
%
The redirection with prefix
|\childdocforwardprefix[|\textit{prefix}|]{|\textit{dest}|}|
is accomplished by:
%
\begin{center}
\begin{tabular}{l}
|{\edef\jobname{\scantokens\expandafter{\jobname\noexpand}}|\\
|\def\redirectjob |\textit{prefix}|#1~~~{\gdef\jobname{|\textit{dest}|#1}}|\\
|\expandafter\redirectjob\jobname~~~}\input{\jobname}|
\end{tabular}
\end{center}

In an alternative approach,
child documents can be compiled by a specific command line
without additional code or specific definitions:
%
\begin{center}
|... -jobname "|\textit{target}|" "|[\textit{flags}]%
|\includeonly{|\textit{dest}|}\input{|\textit{main}|}"|
\end{center}
%

%%%%%%%%%%%%%%%%%%%%%%%%%%%%%%%%%%%%%%%%%%%%%%%%%%%%%%%%%%%%%%%%%%%%%%%%%%%%%%%%
%%%%%%%%%%%%%%%%%%%%%%%%%%%%%%%%%%%%%%%%%%%%%%%%%%%%%%%%%%%%%%%%%%%%%%%%%%%%%%%%
\section{Information}

%%%%%%%%%%%%%%%%%%%%%%%%%%%%%%%%%%%%%%%%%%%%%%%%%%%%%%%%%%%%%%%%%%%%%%%%%%%%%%%%
\subsection{Copyright}

Copyright \copyright{} 2017--2018 Niklas Beisert

This work may be distributed and/or modified under the
conditions of the \LaTeX{} Project Public License, either version 1.3
of this license or (at your option) any later version.
The latest version of this license is in
  \url{http://www.latex-project.org/lppl.txt}
and version 1.3 or later is part of all distributions of \LaTeX{}
version 2005/12/01 or later.

This work has the LPPL maintenance status `maintained'.

The Current Maintainer of this work is Niklas Beisert.

This work consists of the files |README.txt|, |childdoc.ins| and |childdoc.dtx|
as well as the derived files |childdoc.def|, |cdocsamp.tex|
with |cdocsch1.tex|, |cdocsch2.tex|, |cdocspt3.tex|, |cdocspt4.tex|,
|cdocsdrf.tex|, |cdocsfn1.tex|, |cdocsfn2.tex|
as well as |childdoc.pdf|.

%%%%%%%%%%%%%%%%%%%%%%%%%%%%%%%%%%%%%%%%%%%%%%%%%%%%%%%%%%%%%%%%%%%%%%%%%%%%%%%%
\subsection{Files and Installation}

The package consists of the files:
%
\begin{center}
\begin{tabular}{ll}
    |README.txt|   & readme file \\
    |childdoc.ins| & installation file \\
    |childdoc.dtx| & source file \\
    |childdoc.def| & definition file \\
    |cdocsamp.tex| & sample main file \\
    |cdocsch1.tex| & sample include file \\
    |cdocsch2.tex| & sample include file \\
    |cdocspt3.tex| & sample part file \\
    |cdocspt4.tex| & sample part file \\
    |cdocsdrf.tex| & sample redirection file \\
    |cdocsfn1.tex| & sample redirection file \\
    |cdocsfn2.tex| & sample redirection file \\
    |childdoc.pdf| & manual
\end{tabular}
\end{center}
%
The distribution consists of the files
|README.txt|, |childdoc.ins| and |childdoc.dtx|.
%
\begin{itemize}
\item
Run (pdf)\LaTeX{} on |childdoc.dtx|
to compile the manual |childdoc.pdf| (this file).
\item
Run \LaTeX{} on |childdoc.ins| to create the definitions file |childdoc.def|
and the sample |cdocsamp.tex| with include files
|cdocsch1.tex|, |cdocsch2.tex|, |cdocspt3.tex|, |cdocspt4.tex|,
|cdocsdrf.tex|, |cdocsfn1.tex|, |cdocsfn2.tex|.
Then copy the file |childdoc.def| to an appropriate directory of your \LaTeX{}
distribution, e.g.\ \textit{texmf-root}|/tex/latex/childdoc|.
\end{itemize}

%%%%%%%%%%%%%%%%%%%%%%%%%%%%%%%%%%%%%%%%%%%%%%%%%%%%%%%%%%%%%%%%%%%%%%%%%%%%%%%%
\subsection{Related CTAN Packages}

There are several other packages which offer a similar functionality:
%
\begin{itemize}
\item
The packages
\href{http://ctan.org/pkg/docmute}{\textsf{docmute}},
\href{http://ctan.org/pkg/includex}{\textsf{includex}} and
\href{http://ctan.org/pkg/standalone}{\textsf{standalone}}
provide commands to include only the document body of
a child file thus allowing both files to be compiled individually.
\item
The packages \href{http://ctan.org/pkg/subdocs}{\textsf{subdocs}}
and \href{http://ctan.org/pkg/subfiles}{\textsf{subfiles}}
provide structures in which the main and child documents can be
encapsulated and allowing them to be compiled individually.
The inclusion mechanism is different from the conventional |\include|.
\item
The package \href{http://ctan.org/pkg/combine}{\textsf{combine}}
is an elaborate solution to combine several documents into one.
\end{itemize}
%
See also the CTAN topic \href{http://ctan.org/topic/subdocs}{\textsf{subdocs}}
for further related packages.
The present package differs from the above solutions in that
a document structure constructed with the conventional |\include| mechanism
just needs two extra commands at the top of every file
such that all constituent files can be compiled individually.

%%%%%%%%%%%%%%%%%%%%%%%%%%%%%%%%%%%%%%%%%%%%%%%%%%%%%%%%%%%%%%%%%%%%%%%%%%%%%%%%
%\subsection{Feature Suggestions}
%
%The following is a list of features which may be useful for future
%versions of this package:
%%
%\begin{itemize}
%\item
%\ldots
%\end{itemize}

%%%%%%%%%%%%%%%%%%%%%%%%%%%%%%%%%%%%%%%%%%%%%%%%%%%%%%%%%%%%%%%%%%%%%%%%%%%%%%%%
\subsection{Revision History}

%%%%%%%%%%%%%%%%%%%%%%%%%%%%%%%%%%%%%%%%
\paragraph{v2.0:} 2018/12/30

\begin{itemize}
\item
immediate forward processing
\item
added |\childdocby| mechanism
\item
manual restructured
\end{itemize}

%%%%%%%%%%%%%%%%%%%%%%%%%%%%%%%%%%%%%%%%
\paragraph{v1.6:} 2018/01/17

\begin{itemize}
\item
application for development of include files
\item
corrections to manual
\end{itemize}

%%%%%%%%%%%%%%%%%%%%%%%%%%%%%%%%%%%%%%%%
\paragraph{v1.5:} 2017/05/21

\begin{itemize}
\item
more complete structuring introduced
\item
|\childdocof| introduced
\item
|\childdoc| renamed to |\childdocmain|
\item
|\childredirect| renamed to |\childdocforward| and |\childdocforwardprefix|
and functionality expanded
\end{itemize}

%%%%%%%%%%%%%%%%%%%%%%%%%%%%%%%%%%%%%%%%
\paragraph{v1.0:} 2017/04/27

\begin{itemize}
\item
manual and install package
\item
first version published on CTAN
\end{itemize}

%%%%%%%%%%%%%%%%%%%%%%%%%%%%%%%%%%%%%%%%
\paragraph{v0.6:} 2017/04/26

\begin{itemize}
\item
redirection mechanism added
\end{itemize}

%%%%%%%%%%%%%%%%%%%%%%%%%%%%%%%%%%%%%%%%
\paragraph{v0.5:} 2017/04/26

\begin{itemize}
\item
functionality in definition file
\end{itemize}


%%%%%%%%%%%%%%%%%%%%%%%%%%%%%%%%%%%%%%%%%%%%%%%%%%%%%%%%%%%%%%%%%%%%%%%%%%%%%%%%
%%%%%%%%%%%%%%%%%%%%%%%%%%%%%%%%%%%%%%%%%%%%%%%%%%%%%%%%%%%%%%%%%%%%%%%%%%%%%%%%
%%%%%%%%%%%%%%%%%%%%%%%%%%%%%%%%%%%%%%%%%%%%%%%%%%%%%%%%%%%%%%%%%%%%%%%%%%%%%%%%
\appendix

\settowidth\MacroIndent{\rmfamily\scriptsize 000\ }

 \DocInput{childdoc.dtx}

\end{document}
%</driver>
% \fi
%
% %%%%%%%%%%%%%%%%%%%%%%%%%%%%%%%%%%%%%%%%%%%%%%%%%%%%%%%%%%%%%%%%%%%%%%%%%%%%%%
% %%%%%%%%%%%%%%%%%%%%%%%%%%%%%%%%%%%%%%%%%%%%%%%%%%%%%%%%%%%%%%%%%%%%%%%%%%%%%%
% \section{Sample}
%\iffalse
%<*samplemain>
%\fi
%
% The following presents a sample document
% with two chapters, two parts, a title page,
% a compile flag as well as three forwarding files to set the flag.
% It consists of eight |.tex| files:
% \begin{center}
% \begin{tabular}{ll}
% |cdocsamp.tex|&main file\\
% |cdocsch1.tex|&include file for chapter 1\\
% |cdocsch2.tex|&include file for chapter 2\\
% |cdocspt3.tex|&include file for part 3\\
% |cdocspt4.tex|&include file for part 4\\
% |cdocsdrf.tex|&forwarding file for main file in draft mode\\
% |cdocsfi1.tex|&forwarding file for final version of chapter 1\\
% |cdocsfi2.tex|&forwarding file for final version of chapter 2\\
% \end{tabular}
% \end{center}
% Each of the eight files can be compiled directly by the \LaTeX{} compiler.
%
% %%%%%%%%%%%%%%%%%%%%%%%%%%%%%%%%%%%%%%
% \paragraph{Main File.}
%
% The main file is called |cdocsamp.tex|.
%
% Load the \textsf{childdoc} definitions and
% declare the filename for the main document:
%    \begin{macrocode}
\input{childdoc.def}
\childdocmain{}
%    \end{macrocode}

% Optional override for |\version| flag:
%    \begin{macrocode}
%%\ifchilddoc\else\providecommand{\version}{draft}\fi
%    \end{macrocode}

% Define the default values for the |\version| flag
% (|final| for the main file and |draft| for childs):
%    \begin{macrocode}
\ifchilddoc
\providecommand{\version}{draft}
\else
\providecommand{\version}{final}
\fi
%    \end{macrocode}

% Load the standard document class:
%    \begin{macrocode}
\documentclass[12pt]{article}
%    \end{macrocode}

% Start the document body:
%    \begin{macrocode}
\begin{document}
%    \end{macrocode}

% Declare a title page.
% Print title, part of document being processed and version flag:
%    \begin{macrocode}
\addtocounter{page}{-1}
\begin{center}
{\LARGE\bfseries{}childdoc example\par}
\vspace{1cm}
\ifchilddoc
\ifchilddocmanual part\else chapter\fi:
`\childdocname' of `\childdocjob'\par
\else
main document: `\childdocjob'\par
\fi
version: \version\par
\end{center}
\newpage
%    \end{macrocode}

% Manually include selected file,
% otherwise process as usual:
%    \begin{macrocode}
\ifchilddocmanual
\section*{part `\childdocname'}
\input{\childdocname}
\else
%    \end{macrocode}

% Include the two chapters:
%    \begin{macrocode}
\include{cdocsch1}
\include{cdocsch2}
%    \end{macrocode}

% Include the two parts unless only chapters should be displayed:
%    \begin{macrocode}
\ifchilddoc\else
\section{part three}
\input{cdocspt3}
\section{part four}
\input{cdocspt4}
\fi
%    \end{macrocode}

% Process as usual until here:
%    \begin{macrocode}
\fi
%    \end{macrocode}

% End of document body:
%    \begin{macrocode}
\end{document}
%    \end{macrocode}
%\iffalse
%</samplemain>
%\fi
%
% %%%%%%%%%%%%%%%%%%%%%%%%%%%%%%%%%%%%%%
% \paragraph{Chapter Include Files.}
%
% The include files are called |cdocsch1.tex| and |cdocsch2.tex|.
%
%\iffalse
%<*samplechap1|samplechap2>
%\fi

% Optional override for |\version| flag:
%    \begin{macrocode}
%%\providecommand{\version}{final}
%    \end{macrocode}

% Include the main document:
%    \begin{macrocode}
\input{childdoc.def}
\childdocof{cdocsamp}
%    \end{macrocode}

%\iffalse
%</samplechap1|samplechap2>
%\fi
%
%\iffalse
%<*samplechap1>
%\fi
% Some text for chapter 1:
%    \begin{macrocode}
\section{one}
some text in chapter one
%    \end{macrocode}

%\iffalse
%</samplechap1>
%\fi
% Some text for chapter 2:
%\iffalse
%<*samplechap2>
%\fi
%    \begin{macrocode}
\section{two}
more text in chapter two
%    \end{macrocode}

%\iffalse
%</samplechap2>
%\fi
%
% %%%%%%%%%%%%%%%%%%%%%%%%%%%%%%%%%%%%%%
% \paragraph{Part Include Files.}
%
% The include files are called |cdocspt3.tex| and |cdocspt4.tex|.
%
%\iffalse
%<*samplepart3|samplepart4>
%\fi

% Optional override for |\version| flag:
%    \begin{macrocode}
%%\providecommand{\version}{final}
%    \end{macrocode}

% Include the main document:
%    \begin{macrocode}
\input{childdoc.def}
\childdocby{cdocsamp}
%    \end{macrocode}

%\iffalse
%</samplepart3|samplepart4>
%\fi
%
%\iffalse
%<*samplepart3>
%\fi
% Some text for part 3:
%    \begin{macrocode}
some text in part three
%    \end{macrocode}

%\iffalse
%</samplepart3>
%\fi
% Some text for part 4:
%\iffalse
%<*samplepart4>
%\fi
%    \begin{macrocode}
more text in part four
%    \end{macrocode}

%\iffalse
%</samplepart4>
%\fi
%
% %%%%%%%%%%%%%%%%%%%%%%%%%%%%%%%%%%%%%%
% \paragraph{Forwarding for a Complete Draft.}
%
% The following forwarding file |cdocsdrf.tex|
% compiles the main document in draft mode:
%\iffalse
%<*sampledraft>
%\fi
%    \begin{macrocode}
\def\version{draft}
\input{childdoc.def}
\childdocforward{cdocsamp}
%    \end{macrocode}

%\iffalse
%</sampledraft>
%\fi
%
% %%%%%%%%%%%%%%%%%%%%%%%%%%%%%%%%%%%%%%
% \paragraph{Forwarding for Final Version of the Chapters.}
%
% The following forwarding files |cdocsfn1.tex| and |cdocsfn2.tex|
% (with identical content)
% compile the final versions of the child documents
% |cdocsch1.tex| and |cdocsch2.tex|, respectively:
%\iffalse
%<*samplefinal>
%\fi
%    \begin{macrocode}
\def\version{final}
\input{childdoc.def}
\childdocforwardprefix[cdocsamp]{cdocsfn}{cdocsch}
%    \end{macrocode}

%\iffalse
%</samplefinal>
%\fi
%
% %%%%%%%%%%%%%%%%%%%%%%%%%%%%%%%%%%%%%%
% \paragraph{Command Line Processing.}
%
% The following three command lines generate the output files
% |cdocscld|, |cdocscl1| and |cdocscl2|
% which should be identical to
% |cdocsdrf|, |cdocsch1| and |cdocsfn2|, respectively:
% \begin{center}
% \begin{tabular}{l}
% |latex -jobname cdocscld \|\\
% |  "\def\version{draft}\input{childdoc.def}\childdocforward{cdocsamp}"|\\
% |latex -jobname cdocscl1 \|\\
% |  "\input{childdoc.def}\childdocforward[cdocsamp]{cdocsch1}"|\\
% |latex -jobname cdocscl2 \|\\
% |  "\def\version{final}\input{childdoc.def}\childdocforward{cdocsch2}"|
% \end{tabular}
% \end{center}
% Note that the trailing backslash on each first line
% merely continues the input to the second line
% (for convenient cut ant paste).
% Furthermore, the command |latex| can be replaced by any
% of its alternative versions such as |pdflatex|.
%
% %%%%%%%%%%%%%%%%%%%%%%%%%%%%%%%%%%%%%%%%%%%%%%%%%%%%%%%%%%%%%%%%%%%%%%%%%%%%%%
% %%%%%%%%%%%%%%%%%%%%%%%%%%%%%%%%%%%%%%%%%%%%%%%%%%%%%%%%%%%%%%%%%%%%%%%%%%%%%%
% \section{Implementation}
%\iffalse
%<*package>
%\fi
%
% This section describes the definitions file |childdoc.def|.

% The definitions cannot be loaded using |\usepackage| or |\RequirePackage|
% which has a mechanism to prevent loading a style file more than once.
% When loading the definitions by means of |\input|
% multiple instances have to be prevented manually:
%\iffalse
%This code needs to be before the `\ProvidesFile' directive
%which is defined at the beginning of this file.
%Therefore it is also placed there and commented out here.
%</package>
%<*discard>
%\fi
%    \begin{macrocode}
\ifdefined\childdocmain\endinput\fi
%    \end{macrocode}
%\iffalse
%</discard>
%<*package>
%\fi
%
% \macro{\ifchilddoc}
% \macro{\ifchilddocmanual}
% The conditional |\ifchilddoc| tells whether a
% child (true) or main (false) document is being compiled.
% The conditional |\ifchilddocmanual| tells whether
% the |\includeonly| mechanism is used (false) or
% the selection of child files must be performed manually (true).
% The definitions initialise to false:
%    \begin{macrocode}
\newif\ifchilddoc
\newif\ifchilddocmanual
%    \end{macrocode}

% \macro{\childdocname}
% \macro{\childdocjob}
% The macro |\childdocname| stores the name of the main document
% to be compiled. The macro |\childdocjob| stores the name of
% the document on which the \LaTeX{} compiler was originally invoked.
% The content of |\jobname| cannot be compared
% to filenames specified in the source due to different catcodes.
% The following code rescans |\jobname|, stores the result
% in |\childdocname| and saves a copy in |\childdocjob|:
%    \begin{macrocode}
\edef\childdocname{\scantokens\expandafter{\jobname\noexpand}}
\let\childdocjob\childdocname
%    \end{macrocode}

% \macro{\childdocdisable}
% The macro |\childdocdisable| prevents the main file
% from being processed more than once.
% At this stage, the main document command |\childdocmain|
% is assumed to be called once again where it should do nothing.
% Any subsequent call to it should prevent
% a secondary processing of the main document
% It overwrites the forwarding commands
% |\childdocof| and |\childdocforward|
% with empty macros to prevent further inclusions of the main document:
%    \begin{macrocode}
\newcommand{\childdocdisable}
{
  \renewcommand{\childdocmain}[1]{\renewcommand{\childdocmain}[1]{\endinput}}
  \renewcommand{\childdocof}[1]{}
  \renewcommand{\childdocby}[2][]{}
  \renewcommand{\childdocforward}[2][]{}
  \renewcommand{\childdocdisable}{}
}
%    \end{macrocode}

% \macro{\childdocmain}
% The macro |\childdocmain| is to be called at the top of the main file
% with nothing or the main filename (without extension) as argument.
% First, it breaks loops.
% If the argument is not empty and does not match |\childdocname|
% (which is set by the first inclusion of |childdoc.def|),
% |\ifchilddoc| is set to true, |\includeonly| is applied to the child file
% and |\jobname| is set to the main file
% (for proper handling of |.aux| files):
%    \begin{macrocode}
\newcommand{\childdocmain}[1]
{
  \childdocdisable\childdocmain{}
  \if?#1?\else
    \begingroup
      \def\childdoctmp{#1}
      \ifx\childdoctmp\childdocname
        \def\childdoctmp{}
      \else
        \def\childdoctmp
        {
          \childdoctrue
          \includeonly{\childdocname}
          \def\childdocjob{#1}
          \def\jobname{#1}
        }
      \fi
      \expandafter
    \endgroup
    \childdoctmp
  \fi
}
%    \end{macrocode}

% \macro{\childdocof}
% The command |\childdocof| redirects
% compilation to the main file |#1|.
%    \begin{macrocode}
\newcommand{\childdocof}[1]
{
  \childdocdisable
  \childdoctrue
  \includeonly{\childdocname}
  \def\jobname{#1}
  \def\childdocjob{#1}
  \input{#1}
}
%    \end{macrocode}

% \macro{\childdocby}
% The command |\childdocby| ....
%    \begin{macrocode}
\newcommand{\childdocby}[2][]
{
  \childdocdisable
  \childdoctrue
  \childdocmanualtrue
  \if?#1?\else
    \def\jobname{#2}
  \fi
  \def\childdocjob{#2}
  \input{#2}
  \endinput
}
%    \end{macrocode}

% \macro{\childdocforward}
% The command |\childdocforward| redirects
% compilation to the main file or
% (if the optional argument is given) a child file.
% Parameters are set as if the main file
% or a child file starting with |\childdocof| was compiled.
% Then compilation is handed over to the main file:
%    \begin{macrocode}
\newcommand{\childdocforward}[2][]
{
  \begingroup
    \if?#1?
      \def\childdoctmp
      {
        \def\childdocname{#2}
        \def\childdocjob{#2}
        \def\jobname{#2}
        \input{#2}
        \endinput
      }
    \else
      \def\childdoctmp
      {
        \childdocdisable
        \def\childdocname{#2}
        \childdoctrue
        \includeonly{#2}
        \def\childdocjob{#1}
        \def\jobname{#1}
        \input{#1}
        \endinput
      }
    \fi
    \expandafter
  \endgroup
  \childdoctmp
}
%    \end{macrocode}

% \macro{\childdocforwardprefix}
% The command |\childdocforwardprefix| redirects
% compilation to the main or a child file by means of a pattern.
% The prefix |#1| in the current filename is replaced by |#2|
% and the suffix of the current filename is kept
% (it is assumed that the filename does not contain the substring `|~~~|'
% which is used as a delimiter).
% Compilation is handed over to the new file by |\childdocforward|:
%    \begin{macrocode}
\newcommand{\childdocforwardprefix}[3][]
{
  \begingroup
    \def\childdocextract #2##1~~~{\def\childdoctmp{\childdocforward[#1]{#3##1}}}
    \expandafter\childdocextract\childdocname~~~
    \expandafter
  \endgroup
  \childdoctmp
}
%    \end{macrocode}

% \macro{\childdoc}
% The deprecated macro |\childdoc| is a legacy version of |\childdocmain|:
%    \begin{macrocode}
\newcommand{\childdoc}{\childdocmain}
%    \end{macrocode}

% \macro{\childdocredirect}
% The deprecated macro |\childdocredirect| is a legacy version
% of |\childdocforward| and |\childdocforwardprefix|:
%    \begin{macrocode}
\newcommand{\childdocredirect}[2][]
{
  \begingroup
    \if?#1?
      \def\childdoctmp{\childdocforward{#2}}
    \else
      \def\childdoctmp{\childdocforwardprefix{#1}{#2}}
    \fi
    \expandafter
  \endgroup
  \childdoctmp
}
%    \end{macrocode}

%\iffalse
%</package>
%\fi
%
\endinput
|\\
|\childdocmain{}|\\
\end{tabular}
\end{center}
at the very top of the main \LaTeX{} file,
in particular \emph{before} the |\documentclass| statement!
The argument of |\childdocmain| should be left empty
(but it must be present).

%%%%%%%%%%%%%%%%%%%%%%%%%%%%%%%%%%%%%%%%
\DescribeMacro{\childdocof}
Furthermore, add the commands
\begin{center}
\begin{tabular}{l}
|% \iffalse
%
% childdoc.dtx Copyright (C) 2017-2018 Niklas Beisert
%
% This work may be distributed and/or modified under the
% conditions of the LaTeX Project Public License, either version 1.3
% of this license or (at your option) any later version.
% The latest version of this license is in
%   http://www.latex-project.org/lppl.txt
% and version 1.3 or later is part of all distributions of LaTeX
% version 2005/12/01 or later.
%
% This work has the LPPL maintenance status `maintained'.
%
% The Current Maintainer of this work is Niklas Beisert.
%
% This work consists of the files childdoc.dtx and childdoc.ins
% and the derived files childdoc.def and cdocsamp.tex with
% cdocsch1.tex, cdocsch2.tex, cdocsdrf.tex, cdocsfn1.tex, cdocsfn2.tex.
%
%<package>\ifdefined\childdocmain\endinput\fi
%<package>\ProvidesFile{childdoc.def}[2018/12/30 v2.0 child document driver]
%<samplemain>\ProvidesFile{cdocsamp.tex}[2018/12/30 v2.0 sample for childdoc]
%<*driver>
%\ProvidesFile{childdoc.drv}[2018/12/30 v2.0 childdoc reference manual file]
\PassOptionsToClass{10pt,a4paper}{article}
\documentclass{ltxdoc}

\usepackage[margin=35mm]{geometry}
\usepackage{hyperref}
\usepackage{hyperxmp}
\usepackage[usenames]{color}

\hypersetup{colorlinks=true}
\hypersetup{pdfstartview=FitH}
\hypersetup{pdfpagemode=UseNone}
\hypersetup{pdfsource={}}
\hypersetup{pdflang={en-UK}}
\hypersetup{pdfcopyright={Copyright 2017-2018 Niklas Beisert.
  This work may be distributed and/or modified under the
  conditions of the LaTeX Project Public License, either version 1.3
  of this license or (at your option) any later version.}}
\hypersetup{pdflicenseurl={http://www.latex-project.org/lppl.txt}}
\hypersetup{pdfcontactaddress={ETH Zurich, ITP, HIT K,
  Wolfgang-Pauli-Strasse 27}}
\hypersetup{pdfcontactpostcode={8093}}
\hypersetup{pdfcontactcity={Zurich}}
\hypersetup{pdfcontactcountry={Switzerland}}
\hypersetup{pdfcontactemail={nbeisert@itp.phys.ethz.ch}}
\hypersetup{pdfcontacturl={http://people.phys.ethz.ch/\xmptilde nbeisert/}}

\newcommand{\secref}[1]{\hyperref[#1]{section \ref*{#1}}}

\parskip1ex
\parindent0pt
\let\olditemize\itemize
\def\itemize{\olditemize\parskip0pt}

\begin{document}

\title{The \textsf{childdoc} Package}
\hypersetup{pdftitle={The childdoc Package}}
\author{Niklas Beisert\\[2ex]
  Institut f\"ur Theoretische Physik\\
  Eidgen\"ossische Technische Hochschule Z\"urich\\
  Wolfgang-Pauli-Strasse 27, 8093 Z\"urich, Switzerland\\[1ex]
  \href{mailto:nbeisert@itp.phys.ethz.ch}
  {\texttt{nbeisert@itp.phys.ethz.ch}}}
\hypersetup{pdfauthor={Niklas Beisert}}
\hypersetup{pdfsubject={Manual for the LaTeX2e Package childdoc}}
\date{30 December 2018, \textsf{v2.0}}
\maketitle

\begin{abstract}\noindent
\textsf{childdoc} is a \LaTeXe{} package
that enables the direct compilation
of document sections included by |\include|
to individual files.
\end{abstract}

\begingroup
\parskip0ex
\tableofcontents
\endgroup

%%%%%%%%%%%%%%%%%%%%%%%%%%%%%%%%%%%%%%%%%%%%%%%%%%%%%%%%%%%%%%%%%%%%%%%%%%%%%%%%
%%%%%%%%%%%%%%%%%%%%%%%%%%%%%%%%%%%%%%%%%%%%%%%%%%%%%%%%%%%%%%%%%%%%%%%%%%%%%%%%
\section{Introduction}

\LaTeX{} provides a mechanism to structure a large document (such as a book)
into a main file and several child files (containing the chapters)
using the |\include| command.
This mechanism is beneficial for documents
which span hundreds of pages in order to
make the source file(s) more manageable.
Moreover, compilation can be restricted to
selected child files by means of the |\includeonly| command.
The latter feature can be used to reduce the compilation time while editing
(this was significantly more useful in the earlier days of \LaTeX{})
or to generate a smaller document which is easier to navigate.
Another application of |\includeonly| is to generate
documents consisting of selected parts of the complete document.

However, there are a few drawbacks of the plain |\include| mechanism:
\begin{itemize}
\item
The child files cannot be compiled on their own,
they can only be compiled via the main file.
A naive editing environment
(such as a text editor with an option
to have the current file processed by \LaTeX)
may require one to switch to the main file before compiling;
attempting to compile the child file produces errors.
\item
The main file must be modified (each time)
to adjust the |\includeonly| command
to the present needs. This easily leaves the main file in a messy state.
\item
The generated document will always carry the filename
of the main document. This is inconvenient if
several child files are to be compiled and
to be kept for distribution.
\end{itemize}

The present package provides a simple interface
to make child files individually compilable by \LaTeX{}.
Compiling a child file then has the same effect as compiling
the main file with an |\includeonly| command
to select the appropriate child.
Moreover the generated document will carry the name of the child
rather than the main file.
This resolves all three above issues.

This feature is meant to make the editing of books,
thesis documents and lecture notes somewhat more convenient.
However, the package can also be used efficiently for
composing a series of documents (such as exercise sheets)
which are typically distributed individually.
It then assists the author in generating the individual documents
(potentially in different versions)
as well as a document containing the collected series.
Another application is in developing style files
or other kinds of included material
where compilation of the style file could redirect
to a sample or test file.

%%%%%%%%%%%%%%%%%%%%%%%%%%%%%%%%%%%%%%%%%%%%%%%%%%%%%%%%%%%%%%%%%%%%%%%%%%%%%%%%
%%%%%%%%%%%%%%%%%%%%%%%%%%%%%%%%%%%%%%%%%%%%%%%%%%%%%%%%%%%%%%%%%%%%%%%%%%%%%%%%
\section{Usage}

First of all, the package \textsf{childdoc} is \emph{not} a standard
\LaTeXe{} |.sty| style file! Therefore it needs to be invoked in
a non-standard way.

%%%%%%%%%%%%%%%%%%%%%%%%%%%%%%%%%%%%%%%%%%%%%%%%%%%%%%%%%%%%%%%%%%%%%%%%%%%%%%%%
\subsection{Included Files}
\label{sec:include}

%%%%%%%%%%%%%%%%%%%%%%%%%%%%%%%%%%%%%%%%
\DescribeMacro{\childdocmain}
To use the package, add the commands
\begin{center}
\begin{tabular}{l}
|\input{childdoc.def}|\\
|\childdocmain{}|\\
\end{tabular}
\end{center}
at the very top of the main \LaTeX{} file,
in particular \emph{before} the |\documentclass| statement!
The argument of |\childdocmain| should be left empty
(but it must be present).

%%%%%%%%%%%%%%%%%%%%%%%%%%%%%%%%%%%%%%%%
\DescribeMacro{\childdocof}
Furthermore, add the commands
\begin{center}
\begin{tabular}{l}
|\input{childdoc.def}|\\
|\childdocof{|\textit{main}|}|\\
\end{tabular}
\end{center}
at the top of every child file \textit{child}
which is included by |\include{|\textit{child}|}|
from within the main file
(or at least for those files to be compiled individually).
The argument \textit{main} must be the filename of the main file.

There are a couple of
considerations in setting up the main and child documents:

%%%%%%%%%%%%%%%%%%%%%%%%%%%%%%%%%%%%%%%%
\paragraph{Restrictions.}

Please note the following restrictions:
\begin{itemize}
\item
|\childdocmain| must be called with one argument \textit{main}
to ensure compatibility with earlier version of the package.
It must either be empty (|\childdocmain{}|)
or precisely match the filename of the main file in which it is specified.
See \secref{sec:detection} for further information.
\item
The filename \textit{main} must be specified without the |.tex| extension.
\item
The filename \textit{main} is case sensitive
(even in case-insensitive file systems)
due to internal string comparison.
\item
The argument \textit{main} should be fully expanded, it cannot be a macro.
\item
Subdirectories and special characters should be avoided in filenames.
\item
The command |\childdocmain{|\textit{main}|}| must be followed by a whitespace.
It should not be followed immediately by another command
or by a comment mark `|%|'.
This is because the \TeX{} parser reads the token immediately following
the argument of |\childdocmain| and puts it
at the beginning of every child section;
however, a white\-space is ignored.
\end{itemize}

%%%%%%%%%%%%%%%%%%%%%%%%%%%%%%%%%%%%%%%%
\paragraph{Content of Main File.}

It is advisable to place all content in the child files included by |\include|.
Any output contained in the main file will appear in all child documents
unless suppressed manually;
it cannot be suppressed automatically by the |\includeonly| directive
and thus should normally be avoided.
A method to include some content in the main file
by means of conditional processing is described in \secref{sec:conditional}.

%%%%%%%%%%%%%%%%%%%%%%%%%%%%%%%%%%%%%%%%
\paragraph{Page Numbering.}

When only a part of the document is compiled,
the appropriate numbering of pages
(as well as other status parameters)
is determined from the |.aux| files.
The latter contain information from previous passes.
However this information needs to propagate through
all intermediate child documents.
Therefore the page numbering in child documents may well
be inconsistent until the complete document is compiled at least once.

A useful (if unconventional) way to always ensure a consistent
page numbering is to restart the numbering in each child document
and denote the pages by `\textit{child}|.|\textit{page}'
where \textit{child} represents the chapter/section number of the child file.
This can be achieved by the command
|\numberwithin{page}{|\textit{child}|}|
of the \textsf{amsmath} package
where \textit{child} can be |chapter| or |section|
depending on the chosen structuring.
Alternatively, one can modify the macro |\thepage| appropriately
and reset the counter |page| at the start of each child file.

%%%%%%%%%%%%%%%%%%%%%%%%%%%%%%%%%%%%%%%%%%%%%%%%%%%%%%%%%%%%%%%%%%%%%%%%%%%%%%%%
\subsection{Conditional Processing}
\label{sec:conditional}

The package provides a mechanism to compile different versions
of a document. To customise the versions further some conditional processing
can come in handy to distinguish which version is being compiled.
The package provides two macros to describe the compilation context:

%%%%%%%%%%%%%%%%%%%%%%%%%%%%%%%%%%%%%%%%
\DescribeMacro{\ifchilddoc}
The conditional |\ifchilddoc| distinguishes between the compilation of
child documents and the main document:
%
\begin{center}
|\ifchilddoc |\textit{child-code}| |[|\||else |\textit{main-code}]| \||fi|
\end{center}

%%%%%%%%%%%%%%%%%%%%%%%%%%%%%%%%%%%%%%%%
\DescribeMacro{\childdocname}
\DescribeMacro{\childdocjob}
The macro |\childdocname| contains the filename (without extension)
of the main or child file being processed.
Note that |\childdocjob| will always contain the name of the main file.

%%%%%%%%%%%%%%%%%%%%%%%%%%%%%%%%%%%%%%%%
\paragraph{Title Page.}

Conditional processing can be used to include a title or banner page
in the main document when proper precautions are taken.
Importantly, the code in the main file should ensure that the page counter
(as well as other status parameters which are stored in the |.aux| files)
takes the same value after the conditional processing.
Otherwise the page numbers may take divergent values
depending on which part is compiled.

For example, a title page could be declared by:
%
\begin{center}
\begin{tabular}{l}
|\ifchilddoc\||else|\\
|\addtocounter{page}{-1}|\\
\textit{code for title page}\\
|\newpage|\\
|\||fi|
\end{tabular}
\end{center}
%
A banner page for the child documents can be generated by:
%
\begin{center}
\begin{tabular}{l}
|\ifchilddoc|\\
|\addtocounter{page}{-1}|\\
\textit{code for banner page}\\
|\newpage|\\
|\||fi|
\end{tabular}
\end{center}
%
Here one could write a message such as:
\begin{center}
|This is the part \childdocname{} of \childdocjob{}.|
\end{center}

%%%%%%%%%%%%%%%%%%%%%%%%%%%%%%%%%%%%%%%%%%%%%%%%%%%%%%%%%%%%%%%%%%%%%%%%%%%%%%%%
\subsection{Flags}
\label{sec:flags}

The package makes it easy to generate different versions
of the main or child documents.
To this end compilation flags can be defined
and assigned different default values.
They will be particularly useful in conjunction
with the forwarding mechanism described in \secref{sec:forward}.

For example, it may be useful to have a flag |\version|
which can be set to |draft| or |final|.
The document source will contain some conditional code
depending on the value of |\version|.
Suppose further, the flag should default to |final| for the main file
and to |draft| for child files
which is a natural assignment for editing the document.
This is achieved by placing the following code
in the preamble of the main document
(below the |\childdocmain| directive):
%
\begin{center}
\begin{tabular}{l}
|\ifchilddoc|\\
|\providecommand{\version}{draft}|\\
|\||else|\\
|\providecommand{\version}{final}|\\
|\||fi|
\end{tabular}
\end{center}
%
The definition by |\providecommand| makes sure
that previous definitions are not overwritten.
Further statements |\providecommand{\version}{...}|
can thus be added before the above code to override it.

For the main file, one might add a line
(between |\childdocmain| and the above block)
%
\begin{center}
|%\ifchilddoc\||else\providecommand{\version}{draft}\||fi|
\end{center}
%
which can be uncommented to produce a draft version.
Likewise one can add a line to the very top of a child file
(above the |\childdocof{|\textit{main}|}| directive)
%
\begin{center}
|%\providecommand{\version}{final}|
\end{center}
%
which can be uncommented to produce the final version of this child document.

%%%%%%%%%%%%%%%%%%%%%%%%%%%%%%%%%%%%%%%%%%%%%%%%%%%%%%%%%%%%%%%%%%%%%%%%%%%%%%%%
\subsection{Forwarding}
\label{sec:forward}

Different versions of the main or child documents
using compilation flags as described in \secref{sec:flags}
can be (permanently) stored in different files
for convenient compilation, viewing and distribution.
To this end, the package defines a command
to pass on compilation to a different file:

%%%%%%%%%%%%%%%%%%%%%%%%%%%%%%%%%%%%%%%%
\DescribeMacro{\childdocforward}
The command |\childdocforward| redirects processing to
another source file:
%
\begin{center}
\begin{tabular}{l}
|\input{childdoc.def}|\\
|\childdocforward[|\textit{main}|]{|\textit{dest}|}|\\
\end{tabular}
\end{center}
%
The argument \textit{dest} is the destination file
(without extension).
It should be the main file or one of the child files.
Note that further \textsf{childdoc} directives
such as |\childdocof| and |\childdocforward|
in the indicated file will be processed in this form.
The optional argument \textit{main}
passes on directly to the main file \textit{main}
while pretending to compile the child \textit{dest}.
This form behaves as if \textit{dest}
issues |\childdocof{|\textit{main}|}| right away,
and no further \textsf{childdoc} directives will be processed.

%%%%%%%%%%%%%%%%%%%%%%%%%%%%%%%%%%%%%%%%
\DescribeMacro{\...prefix}
In the alternative form |\childdocforwardprefix|,
%
\begin{center}
\begin{tabular}{l}
|\input{childdoc.def}|\\
|\childdocforwardprefix[|\textit{main}|]{|\textit{prefix}|}{|\textit{dest}|}|
\end{tabular}
\end{center}
%
the destination file is determined by a pattern
depending on the current file:
To make this work, the current file must be called
`{\textit{prefix}\hspace{0.2em}\textit{suffix}}'
with \textit{prefix} matching precisely the argument.
Processing is then passed on to the file
`{\textit{dest}\hspace{0.2em}\textit{suffix}}'.
Surely, the same effect is achieved by
directly specifying the
argument `{\textit{dest}\hspace{0.2em}\textit{suffix}}'
in the first form.
However, that requires to set up a different file
for each child. With the alternative form of the command
all these files can have exactly the same content
which simplifies setting them up and maintaining them.

For example, the following file |draft.tex|
with a compilation flag |\version| as described in \secref{sec:flags}
compiles the main document as a draft:
%
\begin{center}
\begin{tabular}{l}
|\def\version{draft}|\\
|\input{childdoc.def}|\\
|\childdocforward{|\textit{main}|}|
\end{tabular}
\end{center}
%
Likewise, the following files |final|\textit{nn}|.tex|
compile the final version of the child document
|child|\textit{nn}|.tex|:
%
\begin{center}
\begin{tabular}{l}
|\def\version{final}|\\
|\input{childdoc.def}|\\
|\childdocforwardprefix{final}{child}|
\end{tabular}
\end{center}
%

Note that when several versions of a main file and/or of each child file
are to be generated, it may be convenient to set up a |Makefile| or
shell script to automatise the process.

%%%%%%%%%%%%%%%%%%%%%%%%%%%%%%%%%%%%%%%%%%%%%%%%%%%%%%%%%%%%%%%%%%%%%%%%%%%%%%%%
\subsection{Command Line Processing}
\label{sec:commandline}

The effect of redirection files can also be achieved by invoking
the \LaTeX{} compiler with a more elaborate command line.
Most conveniently this should be done as part
of a shell script or a |Makefile|.

When using \textsf{childdoc} in the main file, the following
command lines effectively perform a redirection
(note that depending on the shell being used,
backslashes may have to be doubled: `|\|' $\to$ `|\\|'):
%
\begin{center}
|... -jobname "|\textit{target}|" |\\|"|[\textit{flags}]%
|\input{childdoc.def}\childdocforward[|\textit{main}|]{|\textit{dest}|}"|
\end{center}
%
Here \textit{target} is the name of the output file,
\textit{main} is the name of the main file
and \textit{dest} is the name of the main or child file to be processed
(all filenames without extensions).
The optional argument \textit{main} can be omitted
if \textit{main} matches \textit{dest}.
Optionally, compilation \textit{flags} can be defined via |\def| commands.
This command line makes the \TeX{} engine believe
it is compiling the file \textit{target}
whose content is specified as the latter parameter.
The provided code then forwards the processing to
\textit{main} or \textit{dest} as described in \secref{sec:forward}.

%%%%%%%%%%%%%%%%%%%%%%%%%%%%%%%%%%%%%%%%%%%%%%%%%%%%%%%%%%%%%%%%%%%%%%%%%%%%%%%%
\subsection{Include by Input}
\label{sec:input}

Including child documents by |\include| has some restrictions by design.
Most notably, the content of a child document always occupies
its own set of pages; pages cannot be shared between child documents.
Usually, this behaviour makes perfect sense
because each child document contain an essential part of the document.
However, in some situations it may be desirable to compose
a document from a collection of parts
without having mandatory page breaks between then.
For this case, the package
provides a mechanism to include parts
by |\input| which can also be processed individually.
However, by construction this mechanism
requires manual handling of the content to be output.

%%%%%%%%%%%%%%%%%%%%%%%%%%%%%%%%%%%%%%%%
\DescribeMacro{\ifchilddocmanual}
The main file should be prepared as usual, see \secref{sec:include}.
However, the document body must make a distinction
between processing of an individual part and of the main document, e.g.:
%
\begin{center}
\begin{tabular}{l}
|\ifchilddocmanual|\\
|\input{\childdocname}|\\
|\||else|\\
\textit{document body with }|\input{|\textit{part}|}|\\
|\||fi|
\end{tabular}
\end{center}
%
The conditional |\ifchilddocmanual| is true whenever
a part to be included by |\input| is being compiled,
and the name of the part is stored in |\childdocname|.

%%%%%%%%%%%%%%%%%%%%%%%%%%%%%%%%%%%%%%%%
\DescribeMacro{\childdocby}
Each part to be included by |\input| should start with:
%
\begin{center}
\begin{tabular}{l}
|\input{childdoc.def}|\\
|\childdocby{|\textit{main}|}|\\
\end{tabular}
\end{center}
%
The directive |\childdocby| is similar to |\childdocof|
described in \secref{sec:include},
but the subsequent selection of content must be done manually.
To that end, both |\ifchilddoc| and |\ifchilddocmanual|
will be true upon processing of a part,
and the name of the part is stored in |\childdocname|.
Note that |\jobname| will be set to the filename of the current part
so that each part receives an individual |.aux| file
that does not interfere with the |.aux| file(s) of the main document.
This behaviour can be altered by the alternative form
|\childdocby[*]{|\textit{main}|}| (with a non-empty optional argument)
which uses the |.aux| file of the main document
by setting |\jobname| to \textit{main}.

%%%%%%%%%%%%%%%%%%%%%%%%%%%%%%%%%%%%%%%%%%%%%%%%%%%%%%%%%%%%%%%%%%%%%%%%%%%%%%%%
\subsection{Driver Development}
\label{sec:driver}

The \textsf{childdoc} mechanism can also be use for the development
of definition files such as \LaTeX{} styles or classes.
This case differs from the above setup with multiple parts
included by |\include| in that no |\includeonly| should be invoked.
This can be achieved by starting the include file
(before |\ProvidesPackage|) with:
%
\begin{center}
\begin{tabular}{l}
|\input{childdoc.def}|\\
|\childdocforward{|\textit{main}|}|\\
\end{tabular}
\end{center}
%
or alternatively with:
%
\begin{center}
\begin{tabular}{l}
|\input{childdoc.def}|\\
|\childdocby{|\textit{main}|}|\\
\end{tabular}
\end{center}
%
Both forms have slightly different effects as described above.
The main file is prepared as usual, see \secref{sec:include}.

%%%%%%%%%%%%%%%%%%%%%%%%%%%%%%%%%%%%%%%%%%%%%%%%%%%%%%%%%%%%%%%%%%%%%%%%%%%%%%%%
\subsection{Legacy Detection}
\label{sec:detection}

The directive |\childdocmain| in the main file can detect
whether the complete document or merely a child is to be compiled
even without using the directive |\childdocof|.
This method is deprecated because it is less robust
and there is no compelling reason to use it;
it is merely provided for backward compatibility
and it may be removed in future versions.

If the detection mechanism is to be used,
it is mandatory to correctly specify
the filename of the main file as the argument of |\childdocmain|:
%
\begin{center}
\begin{tabular}{l}
|\input{childdoc.def}|\\
|\childdocmain{|\textit{main}|}|\\
\end{tabular}
\end{center}
%
If |\jobname| does not match the argument \textit{main} of |\childdocmain|,
it is assumed that |\jobname| points to the child file to be compiled.
When using |\childdocmain| with the main file specified as argument,
it suffices to start a child file
with just |\input{|\textit{main}|}|
without loading of the package and using |\childdocof|.
If instead all processing is done
with the appropriate \textsf{childdoc} directives,
the argument of \textit{main} of |\childdocmain| can be empty.

An alternative version of the command line processing described
in \secref{sec:commandline} using the detection mechanism reads:
%
\begin{center}
|... -jobname "|\textit{target}|" "|[\textit{flags}]%
[|\def\jobname{|\textit{dest}|}|]|\input{|\textit{main}|}"|
\end{center}

%%%%%%%%%%%%%%%%%%%%%%%%%%%%%%%%%%%%%%%%%%%%%%%%%%%%%%%%%%%%%%%%%%%%%%%%%%%%%%%%
\subsection{Manual Code}
\label{sec:manual}

In case one cannot be certain whether the definitions file |childdoc.def|
is installed on the target \TeX{} distribution
and one prefers not to ship it,
it is conceivable to paste a few relevant commands into the sources.

To that end, drop all statements |\input{childdoc.def}|
and perform the replacements as outlined below.
Instead of |\childdocmain{|\textit{main}|}| add the following code
to the top of the main file:
%
\begin{center}
\begin{tabular}{l}
|\||ifdefined\childdocname\endinput\||fi\newif\ifchilddoc|\\
|\edef\childdocname{\scantokens\expandafter{\jobname\noexpand}}|\\
|\def\childdocmain{|\textit{main}|}\||ifx\childdocmain\childdocname\||else|\\
|\childdoctrue\includeonly{\childdocname}\let\jobname\childdocmain\||fi|\\
\end{tabular}
\end{center}
%
Instead of |\childdocof{|\textit{main}|}| just include the main file
at the top of each child file:
%
\begin{center}
|\input{|\textit{main}|}|
\end{center}
%
A simple redirection |\childdocforward{|\textit{dest}|}| is achieved by:
%
\begin{center}
|\def\jobname{|\textit{dest}|}\input{\jobname}|
\end{center}
%
The redirection with prefix
|\childdocforwardprefix[|\textit{prefix}|]{|\textit{dest}|}|
is accomplished by:
%
\begin{center}
\begin{tabular}{l}
|{\edef\jobname{\scantokens\expandafter{\jobname\noexpand}}|\\
|\def\redirectjob |\textit{prefix}|#1~~~{\gdef\jobname{|\textit{dest}|#1}}|\\
|\expandafter\redirectjob\jobname~~~}\input{\jobname}|
\end{tabular}
\end{center}

In an alternative approach,
child documents can be compiled by a specific command line
without additional code or specific definitions:
%
\begin{center}
|... -jobname "|\textit{target}|" "|[\textit{flags}]%
|\includeonly{|\textit{dest}|}\input{|\textit{main}|}"|
\end{center}
%

%%%%%%%%%%%%%%%%%%%%%%%%%%%%%%%%%%%%%%%%%%%%%%%%%%%%%%%%%%%%%%%%%%%%%%%%%%%%%%%%
%%%%%%%%%%%%%%%%%%%%%%%%%%%%%%%%%%%%%%%%%%%%%%%%%%%%%%%%%%%%%%%%%%%%%%%%%%%%%%%%
\section{Information}

%%%%%%%%%%%%%%%%%%%%%%%%%%%%%%%%%%%%%%%%%%%%%%%%%%%%%%%%%%%%%%%%%%%%%%%%%%%%%%%%
\subsection{Copyright}

Copyright \copyright{} 2017--2018 Niklas Beisert

This work may be distributed and/or modified under the
conditions of the \LaTeX{} Project Public License, either version 1.3
of this license or (at your option) any later version.
The latest version of this license is in
  \url{http://www.latex-project.org/lppl.txt}
and version 1.3 or later is part of all distributions of \LaTeX{}
version 2005/12/01 or later.

This work has the LPPL maintenance status `maintained'.

The Current Maintainer of this work is Niklas Beisert.

This work consists of the files |README.txt|, |childdoc.ins| and |childdoc.dtx|
as well as the derived files |childdoc.def|, |cdocsamp.tex|
with |cdocsch1.tex|, |cdocsch2.tex|, |cdocspt3.tex|, |cdocspt4.tex|,
|cdocsdrf.tex|, |cdocsfn1.tex|, |cdocsfn2.tex|
as well as |childdoc.pdf|.

%%%%%%%%%%%%%%%%%%%%%%%%%%%%%%%%%%%%%%%%%%%%%%%%%%%%%%%%%%%%%%%%%%%%%%%%%%%%%%%%
\subsection{Files and Installation}

The package consists of the files:
%
\begin{center}
\begin{tabular}{ll}
    |README.txt|   & readme file \\
    |childdoc.ins| & installation file \\
    |childdoc.dtx| & source file \\
    |childdoc.def| & definition file \\
    |cdocsamp.tex| & sample main file \\
    |cdocsch1.tex| & sample include file \\
    |cdocsch2.tex| & sample include file \\
    |cdocspt3.tex| & sample part file \\
    |cdocspt4.tex| & sample part file \\
    |cdocsdrf.tex| & sample redirection file \\
    |cdocsfn1.tex| & sample redirection file \\
    |cdocsfn2.tex| & sample redirection file \\
    |childdoc.pdf| & manual
\end{tabular}
\end{center}
%
The distribution consists of the files
|README.txt|, |childdoc.ins| and |childdoc.dtx|.
%
\begin{itemize}
\item
Run (pdf)\LaTeX{} on |childdoc.dtx|
to compile the manual |childdoc.pdf| (this file).
\item
Run \LaTeX{} on |childdoc.ins| to create the definitions file |childdoc.def|
and the sample |cdocsamp.tex| with include files
|cdocsch1.tex|, |cdocsch2.tex|, |cdocspt3.tex|, |cdocspt4.tex|,
|cdocsdrf.tex|, |cdocsfn1.tex|, |cdocsfn2.tex|.
Then copy the file |childdoc.def| to an appropriate directory of your \LaTeX{}
distribution, e.g.\ \textit{texmf-root}|/tex/latex/childdoc|.
\end{itemize}

%%%%%%%%%%%%%%%%%%%%%%%%%%%%%%%%%%%%%%%%%%%%%%%%%%%%%%%%%%%%%%%%%%%%%%%%%%%%%%%%
\subsection{Related CTAN Packages}

There are several other packages which offer a similar functionality:
%
\begin{itemize}
\item
The packages
\href{http://ctan.org/pkg/docmute}{\textsf{docmute}},
\href{http://ctan.org/pkg/includex}{\textsf{includex}} and
\href{http://ctan.org/pkg/standalone}{\textsf{standalone}}
provide commands to include only the document body of
a child file thus allowing both files to be compiled individually.
\item
The packages \href{http://ctan.org/pkg/subdocs}{\textsf{subdocs}}
and \href{http://ctan.org/pkg/subfiles}{\textsf{subfiles}}
provide structures in which the main and child documents can be
encapsulated and allowing them to be compiled individually.
The inclusion mechanism is different from the conventional |\include|.
\item
The package \href{http://ctan.org/pkg/combine}{\textsf{combine}}
is an elaborate solution to combine several documents into one.
\end{itemize}
%
See also the CTAN topic \href{http://ctan.org/topic/subdocs}{\textsf{subdocs}}
for further related packages.
The present package differs from the above solutions in that
a document structure constructed with the conventional |\include| mechanism
just needs two extra commands at the top of every file
such that all constituent files can be compiled individually.

%%%%%%%%%%%%%%%%%%%%%%%%%%%%%%%%%%%%%%%%%%%%%%%%%%%%%%%%%%%%%%%%%%%%%%%%%%%%%%%%
%\subsection{Feature Suggestions}
%
%The following is a list of features which may be useful for future
%versions of this package:
%%
%\begin{itemize}
%\item
%\ldots
%\end{itemize}

%%%%%%%%%%%%%%%%%%%%%%%%%%%%%%%%%%%%%%%%%%%%%%%%%%%%%%%%%%%%%%%%%%%%%%%%%%%%%%%%
\subsection{Revision History}

%%%%%%%%%%%%%%%%%%%%%%%%%%%%%%%%%%%%%%%%
\paragraph{v2.0:} 2018/12/30

\begin{itemize}
\item
immediate forward processing
\item
added |\childdocby| mechanism
\item
manual restructured
\end{itemize}

%%%%%%%%%%%%%%%%%%%%%%%%%%%%%%%%%%%%%%%%
\paragraph{v1.6:} 2018/01/17

\begin{itemize}
\item
application for development of include files
\item
corrections to manual
\end{itemize}

%%%%%%%%%%%%%%%%%%%%%%%%%%%%%%%%%%%%%%%%
\paragraph{v1.5:} 2017/05/21

\begin{itemize}
\item
more complete structuring introduced
\item
|\childdocof| introduced
\item
|\childdoc| renamed to |\childdocmain|
\item
|\childredirect| renamed to |\childdocforward| and |\childdocforwardprefix|
and functionality expanded
\end{itemize}

%%%%%%%%%%%%%%%%%%%%%%%%%%%%%%%%%%%%%%%%
\paragraph{v1.0:} 2017/04/27

\begin{itemize}
\item
manual and install package
\item
first version published on CTAN
\end{itemize}

%%%%%%%%%%%%%%%%%%%%%%%%%%%%%%%%%%%%%%%%
\paragraph{v0.6:} 2017/04/26

\begin{itemize}
\item
redirection mechanism added
\end{itemize}

%%%%%%%%%%%%%%%%%%%%%%%%%%%%%%%%%%%%%%%%
\paragraph{v0.5:} 2017/04/26

\begin{itemize}
\item
functionality in definition file
\end{itemize}


%%%%%%%%%%%%%%%%%%%%%%%%%%%%%%%%%%%%%%%%%%%%%%%%%%%%%%%%%%%%%%%%%%%%%%%%%%%%%%%%
%%%%%%%%%%%%%%%%%%%%%%%%%%%%%%%%%%%%%%%%%%%%%%%%%%%%%%%%%%%%%%%%%%%%%%%%%%%%%%%%
%%%%%%%%%%%%%%%%%%%%%%%%%%%%%%%%%%%%%%%%%%%%%%%%%%%%%%%%%%%%%%%%%%%%%%%%%%%%%%%%
\appendix

\settowidth\MacroIndent{\rmfamily\scriptsize 000\ }

 \DocInput{childdoc.dtx}

\end{document}
%</driver>
% \fi
%
% %%%%%%%%%%%%%%%%%%%%%%%%%%%%%%%%%%%%%%%%%%%%%%%%%%%%%%%%%%%%%%%%%%%%%%%%%%%%%%
% %%%%%%%%%%%%%%%%%%%%%%%%%%%%%%%%%%%%%%%%%%%%%%%%%%%%%%%%%%%%%%%%%%%%%%%%%%%%%%
% \section{Sample}
%\iffalse
%<*samplemain>
%\fi
%
% The following presents a sample document
% with two chapters, two parts, a title page,
% a compile flag as well as three forwarding files to set the flag.
% It consists of eight |.tex| files:
% \begin{center}
% \begin{tabular}{ll}
% |cdocsamp.tex|&main file\\
% |cdocsch1.tex|&include file for chapter 1\\
% |cdocsch2.tex|&include file for chapter 2\\
% |cdocspt3.tex|&include file for part 3\\
% |cdocspt4.tex|&include file for part 4\\
% |cdocsdrf.tex|&forwarding file for main file in draft mode\\
% |cdocsfi1.tex|&forwarding file for final version of chapter 1\\
% |cdocsfi2.tex|&forwarding file for final version of chapter 2\\
% \end{tabular}
% \end{center}
% Each of the eight files can be compiled directly by the \LaTeX{} compiler.
%
% %%%%%%%%%%%%%%%%%%%%%%%%%%%%%%%%%%%%%%
% \paragraph{Main File.}
%
% The main file is called |cdocsamp.tex|.
%
% Load the \textsf{childdoc} definitions and
% declare the filename for the main document:
%    \begin{macrocode}
\input{childdoc.def}
\childdocmain{}
%    \end{macrocode}

% Optional override for |\version| flag:
%    \begin{macrocode}
%%\ifchilddoc\else\providecommand{\version}{draft}\fi
%    \end{macrocode}

% Define the default values for the |\version| flag
% (|final| for the main file and |draft| for childs):
%    \begin{macrocode}
\ifchilddoc
\providecommand{\version}{draft}
\else
\providecommand{\version}{final}
\fi
%    \end{macrocode}

% Load the standard document class:
%    \begin{macrocode}
\documentclass[12pt]{article}
%    \end{macrocode}

% Start the document body:
%    \begin{macrocode}
\begin{document}
%    \end{macrocode}

% Declare a title page.
% Print title, part of document being processed and version flag:
%    \begin{macrocode}
\addtocounter{page}{-1}
\begin{center}
{\LARGE\bfseries{}childdoc example\par}
\vspace{1cm}
\ifchilddoc
\ifchilddocmanual part\else chapter\fi:
`\childdocname' of `\childdocjob'\par
\else
main document: `\childdocjob'\par
\fi
version: \version\par
\end{center}
\newpage
%    \end{macrocode}

% Manually include selected file,
% otherwise process as usual:
%    \begin{macrocode}
\ifchilddocmanual
\section*{part `\childdocname'}
\input{\childdocname}
\else
%    \end{macrocode}

% Include the two chapters:
%    \begin{macrocode}
\include{cdocsch1}
\include{cdocsch2}
%    \end{macrocode}

% Include the two parts unless only chapters should be displayed:
%    \begin{macrocode}
\ifchilddoc\else
\section{part three}
\input{cdocspt3}
\section{part four}
\input{cdocspt4}
\fi
%    \end{macrocode}

% Process as usual until here:
%    \begin{macrocode}
\fi
%    \end{macrocode}

% End of document body:
%    \begin{macrocode}
\end{document}
%    \end{macrocode}
%\iffalse
%</samplemain>
%\fi
%
% %%%%%%%%%%%%%%%%%%%%%%%%%%%%%%%%%%%%%%
% \paragraph{Chapter Include Files.}
%
% The include files are called |cdocsch1.tex| and |cdocsch2.tex|.
%
%\iffalse
%<*samplechap1|samplechap2>
%\fi

% Optional override for |\version| flag:
%    \begin{macrocode}
%%\providecommand{\version}{final}
%    \end{macrocode}

% Include the main document:
%    \begin{macrocode}
\input{childdoc.def}
\childdocof{cdocsamp}
%    \end{macrocode}

%\iffalse
%</samplechap1|samplechap2>
%\fi
%
%\iffalse
%<*samplechap1>
%\fi
% Some text for chapter 1:
%    \begin{macrocode}
\section{one}
some text in chapter one
%    \end{macrocode}

%\iffalse
%</samplechap1>
%\fi
% Some text for chapter 2:
%\iffalse
%<*samplechap2>
%\fi
%    \begin{macrocode}
\section{two}
more text in chapter two
%    \end{macrocode}

%\iffalse
%</samplechap2>
%\fi
%
% %%%%%%%%%%%%%%%%%%%%%%%%%%%%%%%%%%%%%%
% \paragraph{Part Include Files.}
%
% The include files are called |cdocspt3.tex| and |cdocspt4.tex|.
%
%\iffalse
%<*samplepart3|samplepart4>
%\fi

% Optional override for |\version| flag:
%    \begin{macrocode}
%%\providecommand{\version}{final}
%    \end{macrocode}

% Include the main document:
%    \begin{macrocode}
\input{childdoc.def}
\childdocby{cdocsamp}
%    \end{macrocode}

%\iffalse
%</samplepart3|samplepart4>
%\fi
%
%\iffalse
%<*samplepart3>
%\fi
% Some text for part 3:
%    \begin{macrocode}
some text in part three
%    \end{macrocode}

%\iffalse
%</samplepart3>
%\fi
% Some text for part 4:
%\iffalse
%<*samplepart4>
%\fi
%    \begin{macrocode}
more text in part four
%    \end{macrocode}

%\iffalse
%</samplepart4>
%\fi
%
% %%%%%%%%%%%%%%%%%%%%%%%%%%%%%%%%%%%%%%
% \paragraph{Forwarding for a Complete Draft.}
%
% The following forwarding file |cdocsdrf.tex|
% compiles the main document in draft mode:
%\iffalse
%<*sampledraft>
%\fi
%    \begin{macrocode}
\def\version{draft}
\input{childdoc.def}
\childdocforward{cdocsamp}
%    \end{macrocode}

%\iffalse
%</sampledraft>
%\fi
%
% %%%%%%%%%%%%%%%%%%%%%%%%%%%%%%%%%%%%%%
% \paragraph{Forwarding for Final Version of the Chapters.}
%
% The following forwarding files |cdocsfn1.tex| and |cdocsfn2.tex|
% (with identical content)
% compile the final versions of the child documents
% |cdocsch1.tex| and |cdocsch2.tex|, respectively:
%\iffalse
%<*samplefinal>
%\fi
%    \begin{macrocode}
\def\version{final}
\input{childdoc.def}
\childdocforwardprefix[cdocsamp]{cdocsfn}{cdocsch}
%    \end{macrocode}

%\iffalse
%</samplefinal>
%\fi
%
% %%%%%%%%%%%%%%%%%%%%%%%%%%%%%%%%%%%%%%
% \paragraph{Command Line Processing.}
%
% The following three command lines generate the output files
% |cdocscld|, |cdocscl1| and |cdocscl2|
% which should be identical to
% |cdocsdrf|, |cdocsch1| and |cdocsfn2|, respectively:
% \begin{center}
% \begin{tabular}{l}
% |latex -jobname cdocscld \|\\
% |  "\def\version{draft}\input{childdoc.def}\childdocforward{cdocsamp}"|\\
% |latex -jobname cdocscl1 \|\\
% |  "\input{childdoc.def}\childdocforward[cdocsamp]{cdocsch1}"|\\
% |latex -jobname cdocscl2 \|\\
% |  "\def\version{final}\input{childdoc.def}\childdocforward{cdocsch2}"|
% \end{tabular}
% \end{center}
% Note that the trailing backslash on each first line
% merely continues the input to the second line
% (for convenient cut ant paste).
% Furthermore, the command |latex| can be replaced by any
% of its alternative versions such as |pdflatex|.
%
% %%%%%%%%%%%%%%%%%%%%%%%%%%%%%%%%%%%%%%%%%%%%%%%%%%%%%%%%%%%%%%%%%%%%%%%%%%%%%%
% %%%%%%%%%%%%%%%%%%%%%%%%%%%%%%%%%%%%%%%%%%%%%%%%%%%%%%%%%%%%%%%%%%%%%%%%%%%%%%
% \section{Implementation}
%\iffalse
%<*package>
%\fi
%
% This section describes the definitions file |childdoc.def|.

% The definitions cannot be loaded using |\usepackage| or |\RequirePackage|
% which has a mechanism to prevent loading a style file more than once.
% When loading the definitions by means of |\input|
% multiple instances have to be prevented manually:
%\iffalse
%This code needs to be before the `\ProvidesFile' directive
%which is defined at the beginning of this file.
%Therefore it is also placed there and commented out here.
%</package>
%<*discard>
%\fi
%    \begin{macrocode}
\ifdefined\childdocmain\endinput\fi
%    \end{macrocode}
%\iffalse
%</discard>
%<*package>
%\fi
%
% \macro{\ifchilddoc}
% \macro{\ifchilddocmanual}
% The conditional |\ifchilddoc| tells whether a
% child (true) or main (false) document is being compiled.
% The conditional |\ifchilddocmanual| tells whether
% the |\includeonly| mechanism is used (false) or
% the selection of child files must be performed manually (true).
% The definitions initialise to false:
%    \begin{macrocode}
\newif\ifchilddoc
\newif\ifchilddocmanual
%    \end{macrocode}

% \macro{\childdocname}
% \macro{\childdocjob}
% The macro |\childdocname| stores the name of the main document
% to be compiled. The macro |\childdocjob| stores the name of
% the document on which the \LaTeX{} compiler was originally invoked.
% The content of |\jobname| cannot be compared
% to filenames specified in the source due to different catcodes.
% The following code rescans |\jobname|, stores the result
% in |\childdocname| and saves a copy in |\childdocjob|:
%    \begin{macrocode}
\edef\childdocname{\scantokens\expandafter{\jobname\noexpand}}
\let\childdocjob\childdocname
%    \end{macrocode}

% \macro{\childdocdisable}
% The macro |\childdocdisable| prevents the main file
% from being processed more than once.
% At this stage, the main document command |\childdocmain|
% is assumed to be called once again where it should do nothing.
% Any subsequent call to it should prevent
% a secondary processing of the main document
% It overwrites the forwarding commands
% |\childdocof| and |\childdocforward|
% with empty macros to prevent further inclusions of the main document:
%    \begin{macrocode}
\newcommand{\childdocdisable}
{
  \renewcommand{\childdocmain}[1]{\renewcommand{\childdocmain}[1]{\endinput}}
  \renewcommand{\childdocof}[1]{}
  \renewcommand{\childdocby}[2][]{}
  \renewcommand{\childdocforward}[2][]{}
  \renewcommand{\childdocdisable}{}
}
%    \end{macrocode}

% \macro{\childdocmain}
% The macro |\childdocmain| is to be called at the top of the main file
% with nothing or the main filename (without extension) as argument.
% First, it breaks loops.
% If the argument is not empty and does not match |\childdocname|
% (which is set by the first inclusion of |childdoc.def|),
% |\ifchilddoc| is set to true, |\includeonly| is applied to the child file
% and |\jobname| is set to the main file
% (for proper handling of |.aux| files):
%    \begin{macrocode}
\newcommand{\childdocmain}[1]
{
  \childdocdisable\childdocmain{}
  \if?#1?\else
    \begingroup
      \def\childdoctmp{#1}
      \ifx\childdoctmp\childdocname
        \def\childdoctmp{}
      \else
        \def\childdoctmp
        {
          \childdoctrue
          \includeonly{\childdocname}
          \def\childdocjob{#1}
          \def\jobname{#1}
        }
      \fi
      \expandafter
    \endgroup
    \childdoctmp
  \fi
}
%    \end{macrocode}

% \macro{\childdocof}
% The command |\childdocof| redirects
% compilation to the main file |#1|.
%    \begin{macrocode}
\newcommand{\childdocof}[1]
{
  \childdocdisable
  \childdoctrue
  \includeonly{\childdocname}
  \def\jobname{#1}
  \def\childdocjob{#1}
  \input{#1}
}
%    \end{macrocode}

% \macro{\childdocby}
% The command |\childdocby| ....
%    \begin{macrocode}
\newcommand{\childdocby}[2][]
{
  \childdocdisable
  \childdoctrue
  \childdocmanualtrue
  \if?#1?\else
    \def\jobname{#2}
  \fi
  \def\childdocjob{#2}
  \input{#2}
  \endinput
}
%    \end{macrocode}

% \macro{\childdocforward}
% The command |\childdocforward| redirects
% compilation to the main file or
% (if the optional argument is given) a child file.
% Parameters are set as if the main file
% or a child file starting with |\childdocof| was compiled.
% Then compilation is handed over to the main file:
%    \begin{macrocode}
\newcommand{\childdocforward}[2][]
{
  \begingroup
    \if?#1?
      \def\childdoctmp
      {
        \def\childdocname{#2}
        \def\childdocjob{#2}
        \def\jobname{#2}
        \input{#2}
        \endinput
      }
    \else
      \def\childdoctmp
      {
        \childdocdisable
        \def\childdocname{#2}
        \childdoctrue
        \includeonly{#2}
        \def\childdocjob{#1}
        \def\jobname{#1}
        \input{#1}
        \endinput
      }
    \fi
    \expandafter
  \endgroup
  \childdoctmp
}
%    \end{macrocode}

% \macro{\childdocforwardprefix}
% The command |\childdocforwardprefix| redirects
% compilation to the main or a child file by means of a pattern.
% The prefix |#1| in the current filename is replaced by |#2|
% and the suffix of the current filename is kept
% (it is assumed that the filename does not contain the substring `|~~~|'
% which is used as a delimiter).
% Compilation is handed over to the new file by |\childdocforward|:
%    \begin{macrocode}
\newcommand{\childdocforwardprefix}[3][]
{
  \begingroup
    \def\childdocextract #2##1~~~{\def\childdoctmp{\childdocforward[#1]{#3##1}}}
    \expandafter\childdocextract\childdocname~~~
    \expandafter
  \endgroup
  \childdoctmp
}
%    \end{macrocode}

% \macro{\childdoc}
% The deprecated macro |\childdoc| is a legacy version of |\childdocmain|:
%    \begin{macrocode}
\newcommand{\childdoc}{\childdocmain}
%    \end{macrocode}

% \macro{\childdocredirect}
% The deprecated macro |\childdocredirect| is a legacy version
% of |\childdocforward| and |\childdocforwardprefix|:
%    \begin{macrocode}
\newcommand{\childdocredirect}[2][]
{
  \begingroup
    \if?#1?
      \def\childdoctmp{\childdocforward{#2}}
    \else
      \def\childdoctmp{\childdocforwardprefix{#1}{#2}}
    \fi
    \expandafter
  \endgroup
  \childdoctmp
}
%    \end{macrocode}

%\iffalse
%</package>
%\fi
%
\endinput
|\\
|\childdocof{|\textit{main}|}|\\
\end{tabular}
\end{center}
at the top of every child file \textit{child}
which is included by |\include{|\textit{child}|}|
from within the main file
(or at least for those files to be compiled individually).
The argument \textit{main} must be the filename of the main file.

There are a couple of
considerations in setting up the main and child documents:

%%%%%%%%%%%%%%%%%%%%%%%%%%%%%%%%%%%%%%%%
\paragraph{Restrictions.}

Please note the following restrictions:
\begin{itemize}
\item
|\childdocmain| must be called with one argument \textit{main}
to ensure compatibility with earlier version of the package.
It must either be empty (|\childdocmain{}|)
or precisely match the filename of the main file in which it is specified.
See \secref{sec:detection} for further information.
\item
The filename \textit{main} must be specified without the |.tex| extension.
\item
The filename \textit{main} is case sensitive
(even in case-insensitive file systems)
due to internal string comparison.
\item
The argument \textit{main} should be fully expanded, it cannot be a macro.
\item
Subdirectories and special characters should be avoided in filenames.
\item
The command |\childdocmain{|\textit{main}|}| must be followed by a whitespace.
It should not be followed immediately by another command
or by a comment mark `|%|'.
This is because the \TeX{} parser reads the token immediately following
the argument of |\childdocmain| and puts it
at the beginning of every child section;
however, a white\-space is ignored.
\end{itemize}

%%%%%%%%%%%%%%%%%%%%%%%%%%%%%%%%%%%%%%%%
\paragraph{Content of Main File.}

It is advisable to place all content in the child files included by |\include|.
Any output contained in the main file will appear in all child documents
unless suppressed manually;
it cannot be suppressed automatically by the |\includeonly| directive
and thus should normally be avoided.
A method to include some content in the main file
by means of conditional processing is described in \secref{sec:conditional}.

%%%%%%%%%%%%%%%%%%%%%%%%%%%%%%%%%%%%%%%%
\paragraph{Page Numbering.}

When only a part of the document is compiled,
the appropriate numbering of pages
(as well as other status parameters)
is determined from the |.aux| files.
The latter contain information from previous passes.
However this information needs to propagate through
all intermediate child documents.
Therefore the page numbering in child documents may well
be inconsistent until the complete document is compiled at least once.

A useful (if unconventional) way to always ensure a consistent
page numbering is to restart the numbering in each child document
and denote the pages by `\textit{child}|.|\textit{page}'
where \textit{child} represents the chapter/section number of the child file.
This can be achieved by the command
|\numberwithin{page}{|\textit{child}|}|
of the \textsf{amsmath} package
where \textit{child} can be |chapter| or |section|
depending on the chosen structuring.
Alternatively, one can modify the macro |\thepage| appropriately
and reset the counter |page| at the start of each child file.

%%%%%%%%%%%%%%%%%%%%%%%%%%%%%%%%%%%%%%%%%%%%%%%%%%%%%%%%%%%%%%%%%%%%%%%%%%%%%%%%
\subsection{Conditional Processing}
\label{sec:conditional}

The package provides a mechanism to compile different versions
of a document. To customise the versions further some conditional processing
can come in handy to distinguish which version is being compiled.
The package provides two macros to describe the compilation context:

%%%%%%%%%%%%%%%%%%%%%%%%%%%%%%%%%%%%%%%%
\DescribeMacro{\ifchilddoc}
The conditional |\ifchilddoc| distinguishes between the compilation of
child documents and the main document:
%
\begin{center}
|\ifchilddoc |\textit{child-code}| |[|\||else |\textit{main-code}]| \||fi|
\end{center}

%%%%%%%%%%%%%%%%%%%%%%%%%%%%%%%%%%%%%%%%
\DescribeMacro{\childdocname}
\DescribeMacro{\childdocjob}
The macro |\childdocname| contains the filename (without extension)
of the main or child file being processed.
Note that |\childdocjob| will always contain the name of the main file.

%%%%%%%%%%%%%%%%%%%%%%%%%%%%%%%%%%%%%%%%
\paragraph{Title Page.}

Conditional processing can be used to include a title or banner page
in the main document when proper precautions are taken.
Importantly, the code in the main file should ensure that the page counter
(as well as other status parameters which are stored in the |.aux| files)
takes the same value after the conditional processing.
Otherwise the page numbers may take divergent values
depending on which part is compiled.

For example, a title page could be declared by:
%
\begin{center}
\begin{tabular}{l}
|\ifchilddoc\||else|\\
|\addtocounter{page}{-1}|\\
\textit{code for title page}\\
|\newpage|\\
|\||fi|
\end{tabular}
\end{center}
%
A banner page for the child documents can be generated by:
%
\begin{center}
\begin{tabular}{l}
|\ifchilddoc|\\
|\addtocounter{page}{-1}|\\
\textit{code for banner page}\\
|\newpage|\\
|\||fi|
\end{tabular}
\end{center}
%
Here one could write a message such as:
\begin{center}
|This is the part \childdocname{} of \childdocjob{}.|
\end{center}

%%%%%%%%%%%%%%%%%%%%%%%%%%%%%%%%%%%%%%%%%%%%%%%%%%%%%%%%%%%%%%%%%%%%%%%%%%%%%%%%
\subsection{Flags}
\label{sec:flags}

The package makes it easy to generate different versions
of the main or child documents.
To this end compilation flags can be defined
and assigned different default values.
They will be particularly useful in conjunction
with the forwarding mechanism described in \secref{sec:forward}.

For example, it may be useful to have a flag |\version|
which can be set to |draft| or |final|.
The document source will contain some conditional code
depending on the value of |\version|.
Suppose further, the flag should default to |final| for the main file
and to |draft| for child files
which is a natural assignment for editing the document.
This is achieved by placing the following code
in the preamble of the main document
(below the |\childdocmain| directive):
%
\begin{center}
\begin{tabular}{l}
|\ifchilddoc|\\
|\providecommand{\version}{draft}|\\
|\||else|\\
|\providecommand{\version}{final}|\\
|\||fi|
\end{tabular}
\end{center}
%
The definition by |\providecommand| makes sure
that previous definitions are not overwritten.
Further statements |\providecommand{\version}{...}|
can thus be added before the above code to override it.

For the main file, one might add a line
(between |\childdocmain| and the above block)
%
\begin{center}
|%\ifchilddoc\||else\providecommand{\version}{draft}\||fi|
\end{center}
%
which can be uncommented to produce a draft version.
Likewise one can add a line to the very top of a child file
(above the |\childdocof{|\textit{main}|}| directive)
%
\begin{center}
|%\providecommand{\version}{final}|
\end{center}
%
which can be uncommented to produce the final version of this child document.

%%%%%%%%%%%%%%%%%%%%%%%%%%%%%%%%%%%%%%%%%%%%%%%%%%%%%%%%%%%%%%%%%%%%%%%%%%%%%%%%
\subsection{Forwarding}
\label{sec:forward}

Different versions of the main or child documents
using compilation flags as described in \secref{sec:flags}
can be (permanently) stored in different files
for convenient compilation, viewing and distribution.
To this end, the package defines a command
to pass on compilation to a different file:

%%%%%%%%%%%%%%%%%%%%%%%%%%%%%%%%%%%%%%%%
\DescribeMacro{\childdocforward}
The command |\childdocforward| redirects processing to
another source file:
%
\begin{center}
\begin{tabular}{l}
|% \iffalse
%
% childdoc.dtx Copyright (C) 2017-2018 Niklas Beisert
%
% This work may be distributed and/or modified under the
% conditions of the LaTeX Project Public License, either version 1.3
% of this license or (at your option) any later version.
% The latest version of this license is in
%   http://www.latex-project.org/lppl.txt
% and version 1.3 or later is part of all distributions of LaTeX
% version 2005/12/01 or later.
%
% This work has the LPPL maintenance status `maintained'.
%
% The Current Maintainer of this work is Niklas Beisert.
%
% This work consists of the files childdoc.dtx and childdoc.ins
% and the derived files childdoc.def and cdocsamp.tex with
% cdocsch1.tex, cdocsch2.tex, cdocsdrf.tex, cdocsfn1.tex, cdocsfn2.tex.
%
%<package>\ifdefined\childdocmain\endinput\fi
%<package>\ProvidesFile{childdoc.def}[2018/12/30 v2.0 child document driver]
%<samplemain>\ProvidesFile{cdocsamp.tex}[2018/12/30 v2.0 sample for childdoc]
%<*driver>
%\ProvidesFile{childdoc.drv}[2018/12/30 v2.0 childdoc reference manual file]
\PassOptionsToClass{10pt,a4paper}{article}
\documentclass{ltxdoc}

\usepackage[margin=35mm]{geometry}
\usepackage{hyperref}
\usepackage{hyperxmp}
\usepackage[usenames]{color}

\hypersetup{colorlinks=true}
\hypersetup{pdfstartview=FitH}
\hypersetup{pdfpagemode=UseNone}
\hypersetup{pdfsource={}}
\hypersetup{pdflang={en-UK}}
\hypersetup{pdfcopyright={Copyright 2017-2018 Niklas Beisert.
  This work may be distributed and/or modified under the
  conditions of the LaTeX Project Public License, either version 1.3
  of this license or (at your option) any later version.}}
\hypersetup{pdflicenseurl={http://www.latex-project.org/lppl.txt}}
\hypersetup{pdfcontactaddress={ETH Zurich, ITP, HIT K,
  Wolfgang-Pauli-Strasse 27}}
\hypersetup{pdfcontactpostcode={8093}}
\hypersetup{pdfcontactcity={Zurich}}
\hypersetup{pdfcontactcountry={Switzerland}}
\hypersetup{pdfcontactemail={nbeisert@itp.phys.ethz.ch}}
\hypersetup{pdfcontacturl={http://people.phys.ethz.ch/\xmptilde nbeisert/}}

\newcommand{\secref}[1]{\hyperref[#1]{section \ref*{#1}}}

\parskip1ex
\parindent0pt
\let\olditemize\itemize
\def\itemize{\olditemize\parskip0pt}

\begin{document}

\title{The \textsf{childdoc} Package}
\hypersetup{pdftitle={The childdoc Package}}
\author{Niklas Beisert\\[2ex]
  Institut f\"ur Theoretische Physik\\
  Eidgen\"ossische Technische Hochschule Z\"urich\\
  Wolfgang-Pauli-Strasse 27, 8093 Z\"urich, Switzerland\\[1ex]
  \href{mailto:nbeisert@itp.phys.ethz.ch}
  {\texttt{nbeisert@itp.phys.ethz.ch}}}
\hypersetup{pdfauthor={Niklas Beisert}}
\hypersetup{pdfsubject={Manual for the LaTeX2e Package childdoc}}
\date{30 December 2018, \textsf{v2.0}}
\maketitle

\begin{abstract}\noindent
\textsf{childdoc} is a \LaTeXe{} package
that enables the direct compilation
of document sections included by |\include|
to individual files.
\end{abstract}

\begingroup
\parskip0ex
\tableofcontents
\endgroup

%%%%%%%%%%%%%%%%%%%%%%%%%%%%%%%%%%%%%%%%%%%%%%%%%%%%%%%%%%%%%%%%%%%%%%%%%%%%%%%%
%%%%%%%%%%%%%%%%%%%%%%%%%%%%%%%%%%%%%%%%%%%%%%%%%%%%%%%%%%%%%%%%%%%%%%%%%%%%%%%%
\section{Introduction}

\LaTeX{} provides a mechanism to structure a large document (such as a book)
into a main file and several child files (containing the chapters)
using the |\include| command.
This mechanism is beneficial for documents
which span hundreds of pages in order to
make the source file(s) more manageable.
Moreover, compilation can be restricted to
selected child files by means of the |\includeonly| command.
The latter feature can be used to reduce the compilation time while editing
(this was significantly more useful in the earlier days of \LaTeX{})
or to generate a smaller document which is easier to navigate.
Another application of |\includeonly| is to generate
documents consisting of selected parts of the complete document.

However, there are a few drawbacks of the plain |\include| mechanism:
\begin{itemize}
\item
The child files cannot be compiled on their own,
they can only be compiled via the main file.
A naive editing environment
(such as a text editor with an option
to have the current file processed by \LaTeX)
may require one to switch to the main file before compiling;
attempting to compile the child file produces errors.
\item
The main file must be modified (each time)
to adjust the |\includeonly| command
to the present needs. This easily leaves the main file in a messy state.
\item
The generated document will always carry the filename
of the main document. This is inconvenient if
several child files are to be compiled and
to be kept for distribution.
\end{itemize}

The present package provides a simple interface
to make child files individually compilable by \LaTeX{}.
Compiling a child file then has the same effect as compiling
the main file with an |\includeonly| command
to select the appropriate child.
Moreover the generated document will carry the name of the child
rather than the main file.
This resolves all three above issues.

This feature is meant to make the editing of books,
thesis documents and lecture notes somewhat more convenient.
However, the package can also be used efficiently for
composing a series of documents (such as exercise sheets)
which are typically distributed individually.
It then assists the author in generating the individual documents
(potentially in different versions)
as well as a document containing the collected series.
Another application is in developing style files
or other kinds of included material
where compilation of the style file could redirect
to a sample or test file.

%%%%%%%%%%%%%%%%%%%%%%%%%%%%%%%%%%%%%%%%%%%%%%%%%%%%%%%%%%%%%%%%%%%%%%%%%%%%%%%%
%%%%%%%%%%%%%%%%%%%%%%%%%%%%%%%%%%%%%%%%%%%%%%%%%%%%%%%%%%%%%%%%%%%%%%%%%%%%%%%%
\section{Usage}

First of all, the package \textsf{childdoc} is \emph{not} a standard
\LaTeXe{} |.sty| style file! Therefore it needs to be invoked in
a non-standard way.

%%%%%%%%%%%%%%%%%%%%%%%%%%%%%%%%%%%%%%%%%%%%%%%%%%%%%%%%%%%%%%%%%%%%%%%%%%%%%%%%
\subsection{Included Files}
\label{sec:include}

%%%%%%%%%%%%%%%%%%%%%%%%%%%%%%%%%%%%%%%%
\DescribeMacro{\childdocmain}
To use the package, add the commands
\begin{center}
\begin{tabular}{l}
|\input{childdoc.def}|\\
|\childdocmain{}|\\
\end{tabular}
\end{center}
at the very top of the main \LaTeX{} file,
in particular \emph{before} the |\documentclass| statement!
The argument of |\childdocmain| should be left empty
(but it must be present).

%%%%%%%%%%%%%%%%%%%%%%%%%%%%%%%%%%%%%%%%
\DescribeMacro{\childdocof}
Furthermore, add the commands
\begin{center}
\begin{tabular}{l}
|\input{childdoc.def}|\\
|\childdocof{|\textit{main}|}|\\
\end{tabular}
\end{center}
at the top of every child file \textit{child}
which is included by |\include{|\textit{child}|}|
from within the main file
(or at least for those files to be compiled individually).
The argument \textit{main} must be the filename of the main file.

There are a couple of
considerations in setting up the main and child documents:

%%%%%%%%%%%%%%%%%%%%%%%%%%%%%%%%%%%%%%%%
\paragraph{Restrictions.}

Please note the following restrictions:
\begin{itemize}
\item
|\childdocmain| must be called with one argument \textit{main}
to ensure compatibility with earlier version of the package.
It must either be empty (|\childdocmain{}|)
or precisely match the filename of the main file in which it is specified.
See \secref{sec:detection} for further information.
\item
The filename \textit{main} must be specified without the |.tex| extension.
\item
The filename \textit{main} is case sensitive
(even in case-insensitive file systems)
due to internal string comparison.
\item
The argument \textit{main} should be fully expanded, it cannot be a macro.
\item
Subdirectories and special characters should be avoided in filenames.
\item
The command |\childdocmain{|\textit{main}|}| must be followed by a whitespace.
It should not be followed immediately by another command
or by a comment mark `|%|'.
This is because the \TeX{} parser reads the token immediately following
the argument of |\childdocmain| and puts it
at the beginning of every child section;
however, a white\-space is ignored.
\end{itemize}

%%%%%%%%%%%%%%%%%%%%%%%%%%%%%%%%%%%%%%%%
\paragraph{Content of Main File.}

It is advisable to place all content in the child files included by |\include|.
Any output contained in the main file will appear in all child documents
unless suppressed manually;
it cannot be suppressed automatically by the |\includeonly| directive
and thus should normally be avoided.
A method to include some content in the main file
by means of conditional processing is described in \secref{sec:conditional}.

%%%%%%%%%%%%%%%%%%%%%%%%%%%%%%%%%%%%%%%%
\paragraph{Page Numbering.}

When only a part of the document is compiled,
the appropriate numbering of pages
(as well as other status parameters)
is determined from the |.aux| files.
The latter contain information from previous passes.
However this information needs to propagate through
all intermediate child documents.
Therefore the page numbering in child documents may well
be inconsistent until the complete document is compiled at least once.

A useful (if unconventional) way to always ensure a consistent
page numbering is to restart the numbering in each child document
and denote the pages by `\textit{child}|.|\textit{page}'
where \textit{child} represents the chapter/section number of the child file.
This can be achieved by the command
|\numberwithin{page}{|\textit{child}|}|
of the \textsf{amsmath} package
where \textit{child} can be |chapter| or |section|
depending on the chosen structuring.
Alternatively, one can modify the macro |\thepage| appropriately
and reset the counter |page| at the start of each child file.

%%%%%%%%%%%%%%%%%%%%%%%%%%%%%%%%%%%%%%%%%%%%%%%%%%%%%%%%%%%%%%%%%%%%%%%%%%%%%%%%
\subsection{Conditional Processing}
\label{sec:conditional}

The package provides a mechanism to compile different versions
of a document. To customise the versions further some conditional processing
can come in handy to distinguish which version is being compiled.
The package provides two macros to describe the compilation context:

%%%%%%%%%%%%%%%%%%%%%%%%%%%%%%%%%%%%%%%%
\DescribeMacro{\ifchilddoc}
The conditional |\ifchilddoc| distinguishes between the compilation of
child documents and the main document:
%
\begin{center}
|\ifchilddoc |\textit{child-code}| |[|\||else |\textit{main-code}]| \||fi|
\end{center}

%%%%%%%%%%%%%%%%%%%%%%%%%%%%%%%%%%%%%%%%
\DescribeMacro{\childdocname}
\DescribeMacro{\childdocjob}
The macro |\childdocname| contains the filename (without extension)
of the main or child file being processed.
Note that |\childdocjob| will always contain the name of the main file.

%%%%%%%%%%%%%%%%%%%%%%%%%%%%%%%%%%%%%%%%
\paragraph{Title Page.}

Conditional processing can be used to include a title or banner page
in the main document when proper precautions are taken.
Importantly, the code in the main file should ensure that the page counter
(as well as other status parameters which are stored in the |.aux| files)
takes the same value after the conditional processing.
Otherwise the page numbers may take divergent values
depending on which part is compiled.

For example, a title page could be declared by:
%
\begin{center}
\begin{tabular}{l}
|\ifchilddoc\||else|\\
|\addtocounter{page}{-1}|\\
\textit{code for title page}\\
|\newpage|\\
|\||fi|
\end{tabular}
\end{center}
%
A banner page for the child documents can be generated by:
%
\begin{center}
\begin{tabular}{l}
|\ifchilddoc|\\
|\addtocounter{page}{-1}|\\
\textit{code for banner page}\\
|\newpage|\\
|\||fi|
\end{tabular}
\end{center}
%
Here one could write a message such as:
\begin{center}
|This is the part \childdocname{} of \childdocjob{}.|
\end{center}

%%%%%%%%%%%%%%%%%%%%%%%%%%%%%%%%%%%%%%%%%%%%%%%%%%%%%%%%%%%%%%%%%%%%%%%%%%%%%%%%
\subsection{Flags}
\label{sec:flags}

The package makes it easy to generate different versions
of the main or child documents.
To this end compilation flags can be defined
and assigned different default values.
They will be particularly useful in conjunction
with the forwarding mechanism described in \secref{sec:forward}.

For example, it may be useful to have a flag |\version|
which can be set to |draft| or |final|.
The document source will contain some conditional code
depending on the value of |\version|.
Suppose further, the flag should default to |final| for the main file
and to |draft| for child files
which is a natural assignment for editing the document.
This is achieved by placing the following code
in the preamble of the main document
(below the |\childdocmain| directive):
%
\begin{center}
\begin{tabular}{l}
|\ifchilddoc|\\
|\providecommand{\version}{draft}|\\
|\||else|\\
|\providecommand{\version}{final}|\\
|\||fi|
\end{tabular}
\end{center}
%
The definition by |\providecommand| makes sure
that previous definitions are not overwritten.
Further statements |\providecommand{\version}{...}|
can thus be added before the above code to override it.

For the main file, one might add a line
(between |\childdocmain| and the above block)
%
\begin{center}
|%\ifchilddoc\||else\providecommand{\version}{draft}\||fi|
\end{center}
%
which can be uncommented to produce a draft version.
Likewise one can add a line to the very top of a child file
(above the |\childdocof{|\textit{main}|}| directive)
%
\begin{center}
|%\providecommand{\version}{final}|
\end{center}
%
which can be uncommented to produce the final version of this child document.

%%%%%%%%%%%%%%%%%%%%%%%%%%%%%%%%%%%%%%%%%%%%%%%%%%%%%%%%%%%%%%%%%%%%%%%%%%%%%%%%
\subsection{Forwarding}
\label{sec:forward}

Different versions of the main or child documents
using compilation flags as described in \secref{sec:flags}
can be (permanently) stored in different files
for convenient compilation, viewing and distribution.
To this end, the package defines a command
to pass on compilation to a different file:

%%%%%%%%%%%%%%%%%%%%%%%%%%%%%%%%%%%%%%%%
\DescribeMacro{\childdocforward}
The command |\childdocforward| redirects processing to
another source file:
%
\begin{center}
\begin{tabular}{l}
|\input{childdoc.def}|\\
|\childdocforward[|\textit{main}|]{|\textit{dest}|}|\\
\end{tabular}
\end{center}
%
The argument \textit{dest} is the destination file
(without extension).
It should be the main file or one of the child files.
Note that further \textsf{childdoc} directives
such as |\childdocof| and |\childdocforward|
in the indicated file will be processed in this form.
The optional argument \textit{main}
passes on directly to the main file \textit{main}
while pretending to compile the child \textit{dest}.
This form behaves as if \textit{dest}
issues |\childdocof{|\textit{main}|}| right away,
and no further \textsf{childdoc} directives will be processed.

%%%%%%%%%%%%%%%%%%%%%%%%%%%%%%%%%%%%%%%%
\DescribeMacro{\...prefix}
In the alternative form |\childdocforwardprefix|,
%
\begin{center}
\begin{tabular}{l}
|\input{childdoc.def}|\\
|\childdocforwardprefix[|\textit{main}|]{|\textit{prefix}|}{|\textit{dest}|}|
\end{tabular}
\end{center}
%
the destination file is determined by a pattern
depending on the current file:
To make this work, the current file must be called
`{\textit{prefix}\hspace{0.2em}\textit{suffix}}'
with \textit{prefix} matching precisely the argument.
Processing is then passed on to the file
`{\textit{dest}\hspace{0.2em}\textit{suffix}}'.
Surely, the same effect is achieved by
directly specifying the
argument `{\textit{dest}\hspace{0.2em}\textit{suffix}}'
in the first form.
However, that requires to set up a different file
for each child. With the alternative form of the command
all these files can have exactly the same content
which simplifies setting them up and maintaining them.

For example, the following file |draft.tex|
with a compilation flag |\version| as described in \secref{sec:flags}
compiles the main document as a draft:
%
\begin{center}
\begin{tabular}{l}
|\def\version{draft}|\\
|\input{childdoc.def}|\\
|\childdocforward{|\textit{main}|}|
\end{tabular}
\end{center}
%
Likewise, the following files |final|\textit{nn}|.tex|
compile the final version of the child document
|child|\textit{nn}|.tex|:
%
\begin{center}
\begin{tabular}{l}
|\def\version{final}|\\
|\input{childdoc.def}|\\
|\childdocforwardprefix{final}{child}|
\end{tabular}
\end{center}
%

Note that when several versions of a main file and/or of each child file
are to be generated, it may be convenient to set up a |Makefile| or
shell script to automatise the process.

%%%%%%%%%%%%%%%%%%%%%%%%%%%%%%%%%%%%%%%%%%%%%%%%%%%%%%%%%%%%%%%%%%%%%%%%%%%%%%%%
\subsection{Command Line Processing}
\label{sec:commandline}

The effect of redirection files can also be achieved by invoking
the \LaTeX{} compiler with a more elaborate command line.
Most conveniently this should be done as part
of a shell script or a |Makefile|.

When using \textsf{childdoc} in the main file, the following
command lines effectively perform a redirection
(note that depending on the shell being used,
backslashes may have to be doubled: `|\|' $\to$ `|\\|'):
%
\begin{center}
|... -jobname "|\textit{target}|" |\\|"|[\textit{flags}]%
|\input{childdoc.def}\childdocforward[|\textit{main}|]{|\textit{dest}|}"|
\end{center}
%
Here \textit{target} is the name of the output file,
\textit{main} is the name of the main file
and \textit{dest} is the name of the main or child file to be processed
(all filenames without extensions).
The optional argument \textit{main} can be omitted
if \textit{main} matches \textit{dest}.
Optionally, compilation \textit{flags} can be defined via |\def| commands.
This command line makes the \TeX{} engine believe
it is compiling the file \textit{target}
whose content is specified as the latter parameter.
The provided code then forwards the processing to
\textit{main} or \textit{dest} as described in \secref{sec:forward}.

%%%%%%%%%%%%%%%%%%%%%%%%%%%%%%%%%%%%%%%%%%%%%%%%%%%%%%%%%%%%%%%%%%%%%%%%%%%%%%%%
\subsection{Include by Input}
\label{sec:input}

Including child documents by |\include| has some restrictions by design.
Most notably, the content of a child document always occupies
its own set of pages; pages cannot be shared between child documents.
Usually, this behaviour makes perfect sense
because each child document contain an essential part of the document.
However, in some situations it may be desirable to compose
a document from a collection of parts
without having mandatory page breaks between then.
For this case, the package
provides a mechanism to include parts
by |\input| which can also be processed individually.
However, by construction this mechanism
requires manual handling of the content to be output.

%%%%%%%%%%%%%%%%%%%%%%%%%%%%%%%%%%%%%%%%
\DescribeMacro{\ifchilddocmanual}
The main file should be prepared as usual, see \secref{sec:include}.
However, the document body must make a distinction
between processing of an individual part and of the main document, e.g.:
%
\begin{center}
\begin{tabular}{l}
|\ifchilddocmanual|\\
|\input{\childdocname}|\\
|\||else|\\
\textit{document body with }|\input{|\textit{part}|}|\\
|\||fi|
\end{tabular}
\end{center}
%
The conditional |\ifchilddocmanual| is true whenever
a part to be included by |\input| is being compiled,
and the name of the part is stored in |\childdocname|.

%%%%%%%%%%%%%%%%%%%%%%%%%%%%%%%%%%%%%%%%
\DescribeMacro{\childdocby}
Each part to be included by |\input| should start with:
%
\begin{center}
\begin{tabular}{l}
|\input{childdoc.def}|\\
|\childdocby{|\textit{main}|}|\\
\end{tabular}
\end{center}
%
The directive |\childdocby| is similar to |\childdocof|
described in \secref{sec:include},
but the subsequent selection of content must be done manually.
To that end, both |\ifchilddoc| and |\ifchilddocmanual|
will be true upon processing of a part,
and the name of the part is stored in |\childdocname|.
Note that |\jobname| will be set to the filename of the current part
so that each part receives an individual |.aux| file
that does not interfere with the |.aux| file(s) of the main document.
This behaviour can be altered by the alternative form
|\childdocby[*]{|\textit{main}|}| (with a non-empty optional argument)
which uses the |.aux| file of the main document
by setting |\jobname| to \textit{main}.

%%%%%%%%%%%%%%%%%%%%%%%%%%%%%%%%%%%%%%%%%%%%%%%%%%%%%%%%%%%%%%%%%%%%%%%%%%%%%%%%
\subsection{Driver Development}
\label{sec:driver}

The \textsf{childdoc} mechanism can also be use for the development
of definition files such as \LaTeX{} styles or classes.
This case differs from the above setup with multiple parts
included by |\include| in that no |\includeonly| should be invoked.
This can be achieved by starting the include file
(before |\ProvidesPackage|) with:
%
\begin{center}
\begin{tabular}{l}
|\input{childdoc.def}|\\
|\childdocforward{|\textit{main}|}|\\
\end{tabular}
\end{center}
%
or alternatively with:
%
\begin{center}
\begin{tabular}{l}
|\input{childdoc.def}|\\
|\childdocby{|\textit{main}|}|\\
\end{tabular}
\end{center}
%
Both forms have slightly different effects as described above.
The main file is prepared as usual, see \secref{sec:include}.

%%%%%%%%%%%%%%%%%%%%%%%%%%%%%%%%%%%%%%%%%%%%%%%%%%%%%%%%%%%%%%%%%%%%%%%%%%%%%%%%
\subsection{Legacy Detection}
\label{sec:detection}

The directive |\childdocmain| in the main file can detect
whether the complete document or merely a child is to be compiled
even without using the directive |\childdocof|.
This method is deprecated because it is less robust
and there is no compelling reason to use it;
it is merely provided for backward compatibility
and it may be removed in future versions.

If the detection mechanism is to be used,
it is mandatory to correctly specify
the filename of the main file as the argument of |\childdocmain|:
%
\begin{center}
\begin{tabular}{l}
|\input{childdoc.def}|\\
|\childdocmain{|\textit{main}|}|\\
\end{tabular}
\end{center}
%
If |\jobname| does not match the argument \textit{main} of |\childdocmain|,
it is assumed that |\jobname| points to the child file to be compiled.
When using |\childdocmain| with the main file specified as argument,
it suffices to start a child file
with just |\input{|\textit{main}|}|
without loading of the package and using |\childdocof|.
If instead all processing is done
with the appropriate \textsf{childdoc} directives,
the argument of \textit{main} of |\childdocmain| can be empty.

An alternative version of the command line processing described
in \secref{sec:commandline} using the detection mechanism reads:
%
\begin{center}
|... -jobname "|\textit{target}|" "|[\textit{flags}]%
[|\def\jobname{|\textit{dest}|}|]|\input{|\textit{main}|}"|
\end{center}

%%%%%%%%%%%%%%%%%%%%%%%%%%%%%%%%%%%%%%%%%%%%%%%%%%%%%%%%%%%%%%%%%%%%%%%%%%%%%%%%
\subsection{Manual Code}
\label{sec:manual}

In case one cannot be certain whether the definitions file |childdoc.def|
is installed on the target \TeX{} distribution
and one prefers not to ship it,
it is conceivable to paste a few relevant commands into the sources.

To that end, drop all statements |\input{childdoc.def}|
and perform the replacements as outlined below.
Instead of |\childdocmain{|\textit{main}|}| add the following code
to the top of the main file:
%
\begin{center}
\begin{tabular}{l}
|\||ifdefined\childdocname\endinput\||fi\newif\ifchilddoc|\\
|\edef\childdocname{\scantokens\expandafter{\jobname\noexpand}}|\\
|\def\childdocmain{|\textit{main}|}\||ifx\childdocmain\childdocname\||else|\\
|\childdoctrue\includeonly{\childdocname}\let\jobname\childdocmain\||fi|\\
\end{tabular}
\end{center}
%
Instead of |\childdocof{|\textit{main}|}| just include the main file
at the top of each child file:
%
\begin{center}
|\input{|\textit{main}|}|
\end{center}
%
A simple redirection |\childdocforward{|\textit{dest}|}| is achieved by:
%
\begin{center}
|\def\jobname{|\textit{dest}|}\input{\jobname}|
\end{center}
%
The redirection with prefix
|\childdocforwardprefix[|\textit{prefix}|]{|\textit{dest}|}|
is accomplished by:
%
\begin{center}
\begin{tabular}{l}
|{\edef\jobname{\scantokens\expandafter{\jobname\noexpand}}|\\
|\def\redirectjob |\textit{prefix}|#1~~~{\gdef\jobname{|\textit{dest}|#1}}|\\
|\expandafter\redirectjob\jobname~~~}\input{\jobname}|
\end{tabular}
\end{center}

In an alternative approach,
child documents can be compiled by a specific command line
without additional code or specific definitions:
%
\begin{center}
|... -jobname "|\textit{target}|" "|[\textit{flags}]%
|\includeonly{|\textit{dest}|}\input{|\textit{main}|}"|
\end{center}
%

%%%%%%%%%%%%%%%%%%%%%%%%%%%%%%%%%%%%%%%%%%%%%%%%%%%%%%%%%%%%%%%%%%%%%%%%%%%%%%%%
%%%%%%%%%%%%%%%%%%%%%%%%%%%%%%%%%%%%%%%%%%%%%%%%%%%%%%%%%%%%%%%%%%%%%%%%%%%%%%%%
\section{Information}

%%%%%%%%%%%%%%%%%%%%%%%%%%%%%%%%%%%%%%%%%%%%%%%%%%%%%%%%%%%%%%%%%%%%%%%%%%%%%%%%
\subsection{Copyright}

Copyright \copyright{} 2017--2018 Niklas Beisert

This work may be distributed and/or modified under the
conditions of the \LaTeX{} Project Public License, either version 1.3
of this license or (at your option) any later version.
The latest version of this license is in
  \url{http://www.latex-project.org/lppl.txt}
and version 1.3 or later is part of all distributions of \LaTeX{}
version 2005/12/01 or later.

This work has the LPPL maintenance status `maintained'.

The Current Maintainer of this work is Niklas Beisert.

This work consists of the files |README.txt|, |childdoc.ins| and |childdoc.dtx|
as well as the derived files |childdoc.def|, |cdocsamp.tex|
with |cdocsch1.tex|, |cdocsch2.tex|, |cdocspt3.tex|, |cdocspt4.tex|,
|cdocsdrf.tex|, |cdocsfn1.tex|, |cdocsfn2.tex|
as well as |childdoc.pdf|.

%%%%%%%%%%%%%%%%%%%%%%%%%%%%%%%%%%%%%%%%%%%%%%%%%%%%%%%%%%%%%%%%%%%%%%%%%%%%%%%%
\subsection{Files and Installation}

The package consists of the files:
%
\begin{center}
\begin{tabular}{ll}
    |README.txt|   & readme file \\
    |childdoc.ins| & installation file \\
    |childdoc.dtx| & source file \\
    |childdoc.def| & definition file \\
    |cdocsamp.tex| & sample main file \\
    |cdocsch1.tex| & sample include file \\
    |cdocsch2.tex| & sample include file \\
    |cdocspt3.tex| & sample part file \\
    |cdocspt4.tex| & sample part file \\
    |cdocsdrf.tex| & sample redirection file \\
    |cdocsfn1.tex| & sample redirection file \\
    |cdocsfn2.tex| & sample redirection file \\
    |childdoc.pdf| & manual
\end{tabular}
\end{center}
%
The distribution consists of the files
|README.txt|, |childdoc.ins| and |childdoc.dtx|.
%
\begin{itemize}
\item
Run (pdf)\LaTeX{} on |childdoc.dtx|
to compile the manual |childdoc.pdf| (this file).
\item
Run \LaTeX{} on |childdoc.ins| to create the definitions file |childdoc.def|
and the sample |cdocsamp.tex| with include files
|cdocsch1.tex|, |cdocsch2.tex|, |cdocspt3.tex|, |cdocspt4.tex|,
|cdocsdrf.tex|, |cdocsfn1.tex|, |cdocsfn2.tex|.
Then copy the file |childdoc.def| to an appropriate directory of your \LaTeX{}
distribution, e.g.\ \textit{texmf-root}|/tex/latex/childdoc|.
\end{itemize}

%%%%%%%%%%%%%%%%%%%%%%%%%%%%%%%%%%%%%%%%%%%%%%%%%%%%%%%%%%%%%%%%%%%%%%%%%%%%%%%%
\subsection{Related CTAN Packages}

There are several other packages which offer a similar functionality:
%
\begin{itemize}
\item
The packages
\href{http://ctan.org/pkg/docmute}{\textsf{docmute}},
\href{http://ctan.org/pkg/includex}{\textsf{includex}} and
\href{http://ctan.org/pkg/standalone}{\textsf{standalone}}
provide commands to include only the document body of
a child file thus allowing both files to be compiled individually.
\item
The packages \href{http://ctan.org/pkg/subdocs}{\textsf{subdocs}}
and \href{http://ctan.org/pkg/subfiles}{\textsf{subfiles}}
provide structures in which the main and child documents can be
encapsulated and allowing them to be compiled individually.
The inclusion mechanism is different from the conventional |\include|.
\item
The package \href{http://ctan.org/pkg/combine}{\textsf{combine}}
is an elaborate solution to combine several documents into one.
\end{itemize}
%
See also the CTAN topic \href{http://ctan.org/topic/subdocs}{\textsf{subdocs}}
for further related packages.
The present package differs from the above solutions in that
a document structure constructed with the conventional |\include| mechanism
just needs two extra commands at the top of every file
such that all constituent files can be compiled individually.

%%%%%%%%%%%%%%%%%%%%%%%%%%%%%%%%%%%%%%%%%%%%%%%%%%%%%%%%%%%%%%%%%%%%%%%%%%%%%%%%
%\subsection{Feature Suggestions}
%
%The following is a list of features which may be useful for future
%versions of this package:
%%
%\begin{itemize}
%\item
%\ldots
%\end{itemize}

%%%%%%%%%%%%%%%%%%%%%%%%%%%%%%%%%%%%%%%%%%%%%%%%%%%%%%%%%%%%%%%%%%%%%%%%%%%%%%%%
\subsection{Revision History}

%%%%%%%%%%%%%%%%%%%%%%%%%%%%%%%%%%%%%%%%
\paragraph{v2.0:} 2018/12/30

\begin{itemize}
\item
immediate forward processing
\item
added |\childdocby| mechanism
\item
manual restructured
\end{itemize}

%%%%%%%%%%%%%%%%%%%%%%%%%%%%%%%%%%%%%%%%
\paragraph{v1.6:} 2018/01/17

\begin{itemize}
\item
application for development of include files
\item
corrections to manual
\end{itemize}

%%%%%%%%%%%%%%%%%%%%%%%%%%%%%%%%%%%%%%%%
\paragraph{v1.5:} 2017/05/21

\begin{itemize}
\item
more complete structuring introduced
\item
|\childdocof| introduced
\item
|\childdoc| renamed to |\childdocmain|
\item
|\childredirect| renamed to |\childdocforward| and |\childdocforwardprefix|
and functionality expanded
\end{itemize}

%%%%%%%%%%%%%%%%%%%%%%%%%%%%%%%%%%%%%%%%
\paragraph{v1.0:} 2017/04/27

\begin{itemize}
\item
manual and install package
\item
first version published on CTAN
\end{itemize}

%%%%%%%%%%%%%%%%%%%%%%%%%%%%%%%%%%%%%%%%
\paragraph{v0.6:} 2017/04/26

\begin{itemize}
\item
redirection mechanism added
\end{itemize}

%%%%%%%%%%%%%%%%%%%%%%%%%%%%%%%%%%%%%%%%
\paragraph{v0.5:} 2017/04/26

\begin{itemize}
\item
functionality in definition file
\end{itemize}


%%%%%%%%%%%%%%%%%%%%%%%%%%%%%%%%%%%%%%%%%%%%%%%%%%%%%%%%%%%%%%%%%%%%%%%%%%%%%%%%
%%%%%%%%%%%%%%%%%%%%%%%%%%%%%%%%%%%%%%%%%%%%%%%%%%%%%%%%%%%%%%%%%%%%%%%%%%%%%%%%
%%%%%%%%%%%%%%%%%%%%%%%%%%%%%%%%%%%%%%%%%%%%%%%%%%%%%%%%%%%%%%%%%%%%%%%%%%%%%%%%
\appendix

\settowidth\MacroIndent{\rmfamily\scriptsize 000\ }

 \DocInput{childdoc.dtx}

\end{document}
%</driver>
% \fi
%
% %%%%%%%%%%%%%%%%%%%%%%%%%%%%%%%%%%%%%%%%%%%%%%%%%%%%%%%%%%%%%%%%%%%%%%%%%%%%%%
% %%%%%%%%%%%%%%%%%%%%%%%%%%%%%%%%%%%%%%%%%%%%%%%%%%%%%%%%%%%%%%%%%%%%%%%%%%%%%%
% \section{Sample}
%\iffalse
%<*samplemain>
%\fi
%
% The following presents a sample document
% with two chapters, two parts, a title page,
% a compile flag as well as three forwarding files to set the flag.
% It consists of eight |.tex| files:
% \begin{center}
% \begin{tabular}{ll}
% |cdocsamp.tex|&main file\\
% |cdocsch1.tex|&include file for chapter 1\\
% |cdocsch2.tex|&include file for chapter 2\\
% |cdocspt3.tex|&include file for part 3\\
% |cdocspt4.tex|&include file for part 4\\
% |cdocsdrf.tex|&forwarding file for main file in draft mode\\
% |cdocsfi1.tex|&forwarding file for final version of chapter 1\\
% |cdocsfi2.tex|&forwarding file for final version of chapter 2\\
% \end{tabular}
% \end{center}
% Each of the eight files can be compiled directly by the \LaTeX{} compiler.
%
% %%%%%%%%%%%%%%%%%%%%%%%%%%%%%%%%%%%%%%
% \paragraph{Main File.}
%
% The main file is called |cdocsamp.tex|.
%
% Load the \textsf{childdoc} definitions and
% declare the filename for the main document:
%    \begin{macrocode}
\input{childdoc.def}
\childdocmain{}
%    \end{macrocode}

% Optional override for |\version| flag:
%    \begin{macrocode}
%%\ifchilddoc\else\providecommand{\version}{draft}\fi
%    \end{macrocode}

% Define the default values for the |\version| flag
% (|final| for the main file and |draft| for childs):
%    \begin{macrocode}
\ifchilddoc
\providecommand{\version}{draft}
\else
\providecommand{\version}{final}
\fi
%    \end{macrocode}

% Load the standard document class:
%    \begin{macrocode}
\documentclass[12pt]{article}
%    \end{macrocode}

% Start the document body:
%    \begin{macrocode}
\begin{document}
%    \end{macrocode}

% Declare a title page.
% Print title, part of document being processed and version flag:
%    \begin{macrocode}
\addtocounter{page}{-1}
\begin{center}
{\LARGE\bfseries{}childdoc example\par}
\vspace{1cm}
\ifchilddoc
\ifchilddocmanual part\else chapter\fi:
`\childdocname' of `\childdocjob'\par
\else
main document: `\childdocjob'\par
\fi
version: \version\par
\end{center}
\newpage
%    \end{macrocode}

% Manually include selected file,
% otherwise process as usual:
%    \begin{macrocode}
\ifchilddocmanual
\section*{part `\childdocname'}
\input{\childdocname}
\else
%    \end{macrocode}

% Include the two chapters:
%    \begin{macrocode}
\include{cdocsch1}
\include{cdocsch2}
%    \end{macrocode}

% Include the two parts unless only chapters should be displayed:
%    \begin{macrocode}
\ifchilddoc\else
\section{part three}
\input{cdocspt3}
\section{part four}
\input{cdocspt4}
\fi
%    \end{macrocode}

% Process as usual until here:
%    \begin{macrocode}
\fi
%    \end{macrocode}

% End of document body:
%    \begin{macrocode}
\end{document}
%    \end{macrocode}
%\iffalse
%</samplemain>
%\fi
%
% %%%%%%%%%%%%%%%%%%%%%%%%%%%%%%%%%%%%%%
% \paragraph{Chapter Include Files.}
%
% The include files are called |cdocsch1.tex| and |cdocsch2.tex|.
%
%\iffalse
%<*samplechap1|samplechap2>
%\fi

% Optional override for |\version| flag:
%    \begin{macrocode}
%%\providecommand{\version}{final}
%    \end{macrocode}

% Include the main document:
%    \begin{macrocode}
\input{childdoc.def}
\childdocof{cdocsamp}
%    \end{macrocode}

%\iffalse
%</samplechap1|samplechap2>
%\fi
%
%\iffalse
%<*samplechap1>
%\fi
% Some text for chapter 1:
%    \begin{macrocode}
\section{one}
some text in chapter one
%    \end{macrocode}

%\iffalse
%</samplechap1>
%\fi
% Some text for chapter 2:
%\iffalse
%<*samplechap2>
%\fi
%    \begin{macrocode}
\section{two}
more text in chapter two
%    \end{macrocode}

%\iffalse
%</samplechap2>
%\fi
%
% %%%%%%%%%%%%%%%%%%%%%%%%%%%%%%%%%%%%%%
% \paragraph{Part Include Files.}
%
% The include files are called |cdocspt3.tex| and |cdocspt4.tex|.
%
%\iffalse
%<*samplepart3|samplepart4>
%\fi

% Optional override for |\version| flag:
%    \begin{macrocode}
%%\providecommand{\version}{final}
%    \end{macrocode}

% Include the main document:
%    \begin{macrocode}
\input{childdoc.def}
\childdocby{cdocsamp}
%    \end{macrocode}

%\iffalse
%</samplepart3|samplepart4>
%\fi
%
%\iffalse
%<*samplepart3>
%\fi
% Some text for part 3:
%    \begin{macrocode}
some text in part three
%    \end{macrocode}

%\iffalse
%</samplepart3>
%\fi
% Some text for part 4:
%\iffalse
%<*samplepart4>
%\fi
%    \begin{macrocode}
more text in part four
%    \end{macrocode}

%\iffalse
%</samplepart4>
%\fi
%
% %%%%%%%%%%%%%%%%%%%%%%%%%%%%%%%%%%%%%%
% \paragraph{Forwarding for a Complete Draft.}
%
% The following forwarding file |cdocsdrf.tex|
% compiles the main document in draft mode:
%\iffalse
%<*sampledraft>
%\fi
%    \begin{macrocode}
\def\version{draft}
\input{childdoc.def}
\childdocforward{cdocsamp}
%    \end{macrocode}

%\iffalse
%</sampledraft>
%\fi
%
% %%%%%%%%%%%%%%%%%%%%%%%%%%%%%%%%%%%%%%
% \paragraph{Forwarding for Final Version of the Chapters.}
%
% The following forwarding files |cdocsfn1.tex| and |cdocsfn2.tex|
% (with identical content)
% compile the final versions of the child documents
% |cdocsch1.tex| and |cdocsch2.tex|, respectively:
%\iffalse
%<*samplefinal>
%\fi
%    \begin{macrocode}
\def\version{final}
\input{childdoc.def}
\childdocforwardprefix[cdocsamp]{cdocsfn}{cdocsch}
%    \end{macrocode}

%\iffalse
%</samplefinal>
%\fi
%
% %%%%%%%%%%%%%%%%%%%%%%%%%%%%%%%%%%%%%%
% \paragraph{Command Line Processing.}
%
% The following three command lines generate the output files
% |cdocscld|, |cdocscl1| and |cdocscl2|
% which should be identical to
% |cdocsdrf|, |cdocsch1| and |cdocsfn2|, respectively:
% \begin{center}
% \begin{tabular}{l}
% |latex -jobname cdocscld \|\\
% |  "\def\version{draft}\input{childdoc.def}\childdocforward{cdocsamp}"|\\
% |latex -jobname cdocscl1 \|\\
% |  "\input{childdoc.def}\childdocforward[cdocsamp]{cdocsch1}"|\\
% |latex -jobname cdocscl2 \|\\
% |  "\def\version{final}\input{childdoc.def}\childdocforward{cdocsch2}"|
% \end{tabular}
% \end{center}
% Note that the trailing backslash on each first line
% merely continues the input to the second line
% (for convenient cut ant paste).
% Furthermore, the command |latex| can be replaced by any
% of its alternative versions such as |pdflatex|.
%
% %%%%%%%%%%%%%%%%%%%%%%%%%%%%%%%%%%%%%%%%%%%%%%%%%%%%%%%%%%%%%%%%%%%%%%%%%%%%%%
% %%%%%%%%%%%%%%%%%%%%%%%%%%%%%%%%%%%%%%%%%%%%%%%%%%%%%%%%%%%%%%%%%%%%%%%%%%%%%%
% \section{Implementation}
%\iffalse
%<*package>
%\fi
%
% This section describes the definitions file |childdoc.def|.

% The definitions cannot be loaded using |\usepackage| or |\RequirePackage|
% which has a mechanism to prevent loading a style file more than once.
% When loading the definitions by means of |\input|
% multiple instances have to be prevented manually:
%\iffalse
%This code needs to be before the `\ProvidesFile' directive
%which is defined at the beginning of this file.
%Therefore it is also placed there and commented out here.
%</package>
%<*discard>
%\fi
%    \begin{macrocode}
\ifdefined\childdocmain\endinput\fi
%    \end{macrocode}
%\iffalse
%</discard>
%<*package>
%\fi
%
% \macro{\ifchilddoc}
% \macro{\ifchilddocmanual}
% The conditional |\ifchilddoc| tells whether a
% child (true) or main (false) document is being compiled.
% The conditional |\ifchilddocmanual| tells whether
% the |\includeonly| mechanism is used (false) or
% the selection of child files must be performed manually (true).
% The definitions initialise to false:
%    \begin{macrocode}
\newif\ifchilddoc
\newif\ifchilddocmanual
%    \end{macrocode}

% \macro{\childdocname}
% \macro{\childdocjob}
% The macro |\childdocname| stores the name of the main document
% to be compiled. The macro |\childdocjob| stores the name of
% the document on which the \LaTeX{} compiler was originally invoked.
% The content of |\jobname| cannot be compared
% to filenames specified in the source due to different catcodes.
% The following code rescans |\jobname|, stores the result
% in |\childdocname| and saves a copy in |\childdocjob|:
%    \begin{macrocode}
\edef\childdocname{\scantokens\expandafter{\jobname\noexpand}}
\let\childdocjob\childdocname
%    \end{macrocode}

% \macro{\childdocdisable}
% The macro |\childdocdisable| prevents the main file
% from being processed more than once.
% At this stage, the main document command |\childdocmain|
% is assumed to be called once again where it should do nothing.
% Any subsequent call to it should prevent
% a secondary processing of the main document
% It overwrites the forwarding commands
% |\childdocof| and |\childdocforward|
% with empty macros to prevent further inclusions of the main document:
%    \begin{macrocode}
\newcommand{\childdocdisable}
{
  \renewcommand{\childdocmain}[1]{\renewcommand{\childdocmain}[1]{\endinput}}
  \renewcommand{\childdocof}[1]{}
  \renewcommand{\childdocby}[2][]{}
  \renewcommand{\childdocforward}[2][]{}
  \renewcommand{\childdocdisable}{}
}
%    \end{macrocode}

% \macro{\childdocmain}
% The macro |\childdocmain| is to be called at the top of the main file
% with nothing or the main filename (without extension) as argument.
% First, it breaks loops.
% If the argument is not empty and does not match |\childdocname|
% (which is set by the first inclusion of |childdoc.def|),
% |\ifchilddoc| is set to true, |\includeonly| is applied to the child file
% and |\jobname| is set to the main file
% (for proper handling of |.aux| files):
%    \begin{macrocode}
\newcommand{\childdocmain}[1]
{
  \childdocdisable\childdocmain{}
  \if?#1?\else
    \begingroup
      \def\childdoctmp{#1}
      \ifx\childdoctmp\childdocname
        \def\childdoctmp{}
      \else
        \def\childdoctmp
        {
          \childdoctrue
          \includeonly{\childdocname}
          \def\childdocjob{#1}
          \def\jobname{#1}
        }
      \fi
      \expandafter
    \endgroup
    \childdoctmp
  \fi
}
%    \end{macrocode}

% \macro{\childdocof}
% The command |\childdocof| redirects
% compilation to the main file |#1|.
%    \begin{macrocode}
\newcommand{\childdocof}[1]
{
  \childdocdisable
  \childdoctrue
  \includeonly{\childdocname}
  \def\jobname{#1}
  \def\childdocjob{#1}
  \input{#1}
}
%    \end{macrocode}

% \macro{\childdocby}
% The command |\childdocby| ....
%    \begin{macrocode}
\newcommand{\childdocby}[2][]
{
  \childdocdisable
  \childdoctrue
  \childdocmanualtrue
  \if?#1?\else
    \def\jobname{#2}
  \fi
  \def\childdocjob{#2}
  \input{#2}
  \endinput
}
%    \end{macrocode}

% \macro{\childdocforward}
% The command |\childdocforward| redirects
% compilation to the main file or
% (if the optional argument is given) a child file.
% Parameters are set as if the main file
% or a child file starting with |\childdocof| was compiled.
% Then compilation is handed over to the main file:
%    \begin{macrocode}
\newcommand{\childdocforward}[2][]
{
  \begingroup
    \if?#1?
      \def\childdoctmp
      {
        \def\childdocname{#2}
        \def\childdocjob{#2}
        \def\jobname{#2}
        \input{#2}
        \endinput
      }
    \else
      \def\childdoctmp
      {
        \childdocdisable
        \def\childdocname{#2}
        \childdoctrue
        \includeonly{#2}
        \def\childdocjob{#1}
        \def\jobname{#1}
        \input{#1}
        \endinput
      }
    \fi
    \expandafter
  \endgroup
  \childdoctmp
}
%    \end{macrocode}

% \macro{\childdocforwardprefix}
% The command |\childdocforwardprefix| redirects
% compilation to the main or a child file by means of a pattern.
% The prefix |#1| in the current filename is replaced by |#2|
% and the suffix of the current filename is kept
% (it is assumed that the filename does not contain the substring `|~~~|'
% which is used as a delimiter).
% Compilation is handed over to the new file by |\childdocforward|:
%    \begin{macrocode}
\newcommand{\childdocforwardprefix}[3][]
{
  \begingroup
    \def\childdocextract #2##1~~~{\def\childdoctmp{\childdocforward[#1]{#3##1}}}
    \expandafter\childdocextract\childdocname~~~
    \expandafter
  \endgroup
  \childdoctmp
}
%    \end{macrocode}

% \macro{\childdoc}
% The deprecated macro |\childdoc| is a legacy version of |\childdocmain|:
%    \begin{macrocode}
\newcommand{\childdoc}{\childdocmain}
%    \end{macrocode}

% \macro{\childdocredirect}
% The deprecated macro |\childdocredirect| is a legacy version
% of |\childdocforward| and |\childdocforwardprefix|:
%    \begin{macrocode}
\newcommand{\childdocredirect}[2][]
{
  \begingroup
    \if?#1?
      \def\childdoctmp{\childdocforward{#2}}
    \else
      \def\childdoctmp{\childdocforwardprefix{#1}{#2}}
    \fi
    \expandafter
  \endgroup
  \childdoctmp
}
%    \end{macrocode}

%\iffalse
%</package>
%\fi
%
\endinput
|\\
|\childdocforward[|\textit{main}|]{|\textit{dest}|}|\\
\end{tabular}
\end{center}
%
The argument \textit{dest} is the destination file
(without extension).
It should be the main file or one of the child files.
Note that further \textsf{childdoc} directives
such as |\childdocof| and |\childdocforward|
in the indicated file will be processed in this form.
The optional argument \textit{main}
passes on directly to the main file \textit{main}
while pretending to compile the child \textit{dest}.
This form behaves as if \textit{dest}
issues |\childdocof{|\textit{main}|}| right away,
and no further \textsf{childdoc} directives will be processed.

%%%%%%%%%%%%%%%%%%%%%%%%%%%%%%%%%%%%%%%%
\DescribeMacro{\...prefix}
In the alternative form |\childdocforwardprefix|,
%
\begin{center}
\begin{tabular}{l}
|% \iffalse
%
% childdoc.dtx Copyright (C) 2017-2018 Niklas Beisert
%
% This work may be distributed and/or modified under the
% conditions of the LaTeX Project Public License, either version 1.3
% of this license or (at your option) any later version.
% The latest version of this license is in
%   http://www.latex-project.org/lppl.txt
% and version 1.3 or later is part of all distributions of LaTeX
% version 2005/12/01 or later.
%
% This work has the LPPL maintenance status `maintained'.
%
% The Current Maintainer of this work is Niklas Beisert.
%
% This work consists of the files childdoc.dtx and childdoc.ins
% and the derived files childdoc.def and cdocsamp.tex with
% cdocsch1.tex, cdocsch2.tex, cdocsdrf.tex, cdocsfn1.tex, cdocsfn2.tex.
%
%<package>\ifdefined\childdocmain\endinput\fi
%<package>\ProvidesFile{childdoc.def}[2018/12/30 v2.0 child document driver]
%<samplemain>\ProvidesFile{cdocsamp.tex}[2018/12/30 v2.0 sample for childdoc]
%<*driver>
%\ProvidesFile{childdoc.drv}[2018/12/30 v2.0 childdoc reference manual file]
\PassOptionsToClass{10pt,a4paper}{article}
\documentclass{ltxdoc}

\usepackage[margin=35mm]{geometry}
\usepackage{hyperref}
\usepackage{hyperxmp}
\usepackage[usenames]{color}

\hypersetup{colorlinks=true}
\hypersetup{pdfstartview=FitH}
\hypersetup{pdfpagemode=UseNone}
\hypersetup{pdfsource={}}
\hypersetup{pdflang={en-UK}}
\hypersetup{pdfcopyright={Copyright 2017-2018 Niklas Beisert.
  This work may be distributed and/or modified under the
  conditions of the LaTeX Project Public License, either version 1.3
  of this license or (at your option) any later version.}}
\hypersetup{pdflicenseurl={http://www.latex-project.org/lppl.txt}}
\hypersetup{pdfcontactaddress={ETH Zurich, ITP, HIT K,
  Wolfgang-Pauli-Strasse 27}}
\hypersetup{pdfcontactpostcode={8093}}
\hypersetup{pdfcontactcity={Zurich}}
\hypersetup{pdfcontactcountry={Switzerland}}
\hypersetup{pdfcontactemail={nbeisert@itp.phys.ethz.ch}}
\hypersetup{pdfcontacturl={http://people.phys.ethz.ch/\xmptilde nbeisert/}}

\newcommand{\secref}[1]{\hyperref[#1]{section \ref*{#1}}}

\parskip1ex
\parindent0pt
\let\olditemize\itemize
\def\itemize{\olditemize\parskip0pt}

\begin{document}

\title{The \textsf{childdoc} Package}
\hypersetup{pdftitle={The childdoc Package}}
\author{Niklas Beisert\\[2ex]
  Institut f\"ur Theoretische Physik\\
  Eidgen\"ossische Technische Hochschule Z\"urich\\
  Wolfgang-Pauli-Strasse 27, 8093 Z\"urich, Switzerland\\[1ex]
  \href{mailto:nbeisert@itp.phys.ethz.ch}
  {\texttt{nbeisert@itp.phys.ethz.ch}}}
\hypersetup{pdfauthor={Niklas Beisert}}
\hypersetup{pdfsubject={Manual for the LaTeX2e Package childdoc}}
\date{30 December 2018, \textsf{v2.0}}
\maketitle

\begin{abstract}\noindent
\textsf{childdoc} is a \LaTeXe{} package
that enables the direct compilation
of document sections included by |\include|
to individual files.
\end{abstract}

\begingroup
\parskip0ex
\tableofcontents
\endgroup

%%%%%%%%%%%%%%%%%%%%%%%%%%%%%%%%%%%%%%%%%%%%%%%%%%%%%%%%%%%%%%%%%%%%%%%%%%%%%%%%
%%%%%%%%%%%%%%%%%%%%%%%%%%%%%%%%%%%%%%%%%%%%%%%%%%%%%%%%%%%%%%%%%%%%%%%%%%%%%%%%
\section{Introduction}

\LaTeX{} provides a mechanism to structure a large document (such as a book)
into a main file and several child files (containing the chapters)
using the |\include| command.
This mechanism is beneficial for documents
which span hundreds of pages in order to
make the source file(s) more manageable.
Moreover, compilation can be restricted to
selected child files by means of the |\includeonly| command.
The latter feature can be used to reduce the compilation time while editing
(this was significantly more useful in the earlier days of \LaTeX{})
or to generate a smaller document which is easier to navigate.
Another application of |\includeonly| is to generate
documents consisting of selected parts of the complete document.

However, there are a few drawbacks of the plain |\include| mechanism:
\begin{itemize}
\item
The child files cannot be compiled on their own,
they can only be compiled via the main file.
A naive editing environment
(such as a text editor with an option
to have the current file processed by \LaTeX)
may require one to switch to the main file before compiling;
attempting to compile the child file produces errors.
\item
The main file must be modified (each time)
to adjust the |\includeonly| command
to the present needs. This easily leaves the main file in a messy state.
\item
The generated document will always carry the filename
of the main document. This is inconvenient if
several child files are to be compiled and
to be kept for distribution.
\end{itemize}

The present package provides a simple interface
to make child files individually compilable by \LaTeX{}.
Compiling a child file then has the same effect as compiling
the main file with an |\includeonly| command
to select the appropriate child.
Moreover the generated document will carry the name of the child
rather than the main file.
This resolves all three above issues.

This feature is meant to make the editing of books,
thesis documents and lecture notes somewhat more convenient.
However, the package can also be used efficiently for
composing a series of documents (such as exercise sheets)
which are typically distributed individually.
It then assists the author in generating the individual documents
(potentially in different versions)
as well as a document containing the collected series.
Another application is in developing style files
or other kinds of included material
where compilation of the style file could redirect
to a sample or test file.

%%%%%%%%%%%%%%%%%%%%%%%%%%%%%%%%%%%%%%%%%%%%%%%%%%%%%%%%%%%%%%%%%%%%%%%%%%%%%%%%
%%%%%%%%%%%%%%%%%%%%%%%%%%%%%%%%%%%%%%%%%%%%%%%%%%%%%%%%%%%%%%%%%%%%%%%%%%%%%%%%
\section{Usage}

First of all, the package \textsf{childdoc} is \emph{not} a standard
\LaTeXe{} |.sty| style file! Therefore it needs to be invoked in
a non-standard way.

%%%%%%%%%%%%%%%%%%%%%%%%%%%%%%%%%%%%%%%%%%%%%%%%%%%%%%%%%%%%%%%%%%%%%%%%%%%%%%%%
\subsection{Included Files}
\label{sec:include}

%%%%%%%%%%%%%%%%%%%%%%%%%%%%%%%%%%%%%%%%
\DescribeMacro{\childdocmain}
To use the package, add the commands
\begin{center}
\begin{tabular}{l}
|\input{childdoc.def}|\\
|\childdocmain{}|\\
\end{tabular}
\end{center}
at the very top of the main \LaTeX{} file,
in particular \emph{before} the |\documentclass| statement!
The argument of |\childdocmain| should be left empty
(but it must be present).

%%%%%%%%%%%%%%%%%%%%%%%%%%%%%%%%%%%%%%%%
\DescribeMacro{\childdocof}
Furthermore, add the commands
\begin{center}
\begin{tabular}{l}
|\input{childdoc.def}|\\
|\childdocof{|\textit{main}|}|\\
\end{tabular}
\end{center}
at the top of every child file \textit{child}
which is included by |\include{|\textit{child}|}|
from within the main file
(or at least for those files to be compiled individually).
The argument \textit{main} must be the filename of the main file.

There are a couple of
considerations in setting up the main and child documents:

%%%%%%%%%%%%%%%%%%%%%%%%%%%%%%%%%%%%%%%%
\paragraph{Restrictions.}

Please note the following restrictions:
\begin{itemize}
\item
|\childdocmain| must be called with one argument \textit{main}
to ensure compatibility with earlier version of the package.
It must either be empty (|\childdocmain{}|)
or precisely match the filename of the main file in which it is specified.
See \secref{sec:detection} for further information.
\item
The filename \textit{main} must be specified without the |.tex| extension.
\item
The filename \textit{main} is case sensitive
(even in case-insensitive file systems)
due to internal string comparison.
\item
The argument \textit{main} should be fully expanded, it cannot be a macro.
\item
Subdirectories and special characters should be avoided in filenames.
\item
The command |\childdocmain{|\textit{main}|}| must be followed by a whitespace.
It should not be followed immediately by another command
or by a comment mark `|%|'.
This is because the \TeX{} parser reads the token immediately following
the argument of |\childdocmain| and puts it
at the beginning of every child section;
however, a white\-space is ignored.
\end{itemize}

%%%%%%%%%%%%%%%%%%%%%%%%%%%%%%%%%%%%%%%%
\paragraph{Content of Main File.}

It is advisable to place all content in the child files included by |\include|.
Any output contained in the main file will appear in all child documents
unless suppressed manually;
it cannot be suppressed automatically by the |\includeonly| directive
and thus should normally be avoided.
A method to include some content in the main file
by means of conditional processing is described in \secref{sec:conditional}.

%%%%%%%%%%%%%%%%%%%%%%%%%%%%%%%%%%%%%%%%
\paragraph{Page Numbering.}

When only a part of the document is compiled,
the appropriate numbering of pages
(as well as other status parameters)
is determined from the |.aux| files.
The latter contain information from previous passes.
However this information needs to propagate through
all intermediate child documents.
Therefore the page numbering in child documents may well
be inconsistent until the complete document is compiled at least once.

A useful (if unconventional) way to always ensure a consistent
page numbering is to restart the numbering in each child document
and denote the pages by `\textit{child}|.|\textit{page}'
where \textit{child} represents the chapter/section number of the child file.
This can be achieved by the command
|\numberwithin{page}{|\textit{child}|}|
of the \textsf{amsmath} package
where \textit{child} can be |chapter| or |section|
depending on the chosen structuring.
Alternatively, one can modify the macro |\thepage| appropriately
and reset the counter |page| at the start of each child file.

%%%%%%%%%%%%%%%%%%%%%%%%%%%%%%%%%%%%%%%%%%%%%%%%%%%%%%%%%%%%%%%%%%%%%%%%%%%%%%%%
\subsection{Conditional Processing}
\label{sec:conditional}

The package provides a mechanism to compile different versions
of a document. To customise the versions further some conditional processing
can come in handy to distinguish which version is being compiled.
The package provides two macros to describe the compilation context:

%%%%%%%%%%%%%%%%%%%%%%%%%%%%%%%%%%%%%%%%
\DescribeMacro{\ifchilddoc}
The conditional |\ifchilddoc| distinguishes between the compilation of
child documents and the main document:
%
\begin{center}
|\ifchilddoc |\textit{child-code}| |[|\||else |\textit{main-code}]| \||fi|
\end{center}

%%%%%%%%%%%%%%%%%%%%%%%%%%%%%%%%%%%%%%%%
\DescribeMacro{\childdocname}
\DescribeMacro{\childdocjob}
The macro |\childdocname| contains the filename (without extension)
of the main or child file being processed.
Note that |\childdocjob| will always contain the name of the main file.

%%%%%%%%%%%%%%%%%%%%%%%%%%%%%%%%%%%%%%%%
\paragraph{Title Page.}

Conditional processing can be used to include a title or banner page
in the main document when proper precautions are taken.
Importantly, the code in the main file should ensure that the page counter
(as well as other status parameters which are stored in the |.aux| files)
takes the same value after the conditional processing.
Otherwise the page numbers may take divergent values
depending on which part is compiled.

For example, a title page could be declared by:
%
\begin{center}
\begin{tabular}{l}
|\ifchilddoc\||else|\\
|\addtocounter{page}{-1}|\\
\textit{code for title page}\\
|\newpage|\\
|\||fi|
\end{tabular}
\end{center}
%
A banner page for the child documents can be generated by:
%
\begin{center}
\begin{tabular}{l}
|\ifchilddoc|\\
|\addtocounter{page}{-1}|\\
\textit{code for banner page}\\
|\newpage|\\
|\||fi|
\end{tabular}
\end{center}
%
Here one could write a message such as:
\begin{center}
|This is the part \childdocname{} of \childdocjob{}.|
\end{center}

%%%%%%%%%%%%%%%%%%%%%%%%%%%%%%%%%%%%%%%%%%%%%%%%%%%%%%%%%%%%%%%%%%%%%%%%%%%%%%%%
\subsection{Flags}
\label{sec:flags}

The package makes it easy to generate different versions
of the main or child documents.
To this end compilation flags can be defined
and assigned different default values.
They will be particularly useful in conjunction
with the forwarding mechanism described in \secref{sec:forward}.

For example, it may be useful to have a flag |\version|
which can be set to |draft| or |final|.
The document source will contain some conditional code
depending on the value of |\version|.
Suppose further, the flag should default to |final| for the main file
and to |draft| for child files
which is a natural assignment for editing the document.
This is achieved by placing the following code
in the preamble of the main document
(below the |\childdocmain| directive):
%
\begin{center}
\begin{tabular}{l}
|\ifchilddoc|\\
|\providecommand{\version}{draft}|\\
|\||else|\\
|\providecommand{\version}{final}|\\
|\||fi|
\end{tabular}
\end{center}
%
The definition by |\providecommand| makes sure
that previous definitions are not overwritten.
Further statements |\providecommand{\version}{...}|
can thus be added before the above code to override it.

For the main file, one might add a line
(between |\childdocmain| and the above block)
%
\begin{center}
|%\ifchilddoc\||else\providecommand{\version}{draft}\||fi|
\end{center}
%
which can be uncommented to produce a draft version.
Likewise one can add a line to the very top of a child file
(above the |\childdocof{|\textit{main}|}| directive)
%
\begin{center}
|%\providecommand{\version}{final}|
\end{center}
%
which can be uncommented to produce the final version of this child document.

%%%%%%%%%%%%%%%%%%%%%%%%%%%%%%%%%%%%%%%%%%%%%%%%%%%%%%%%%%%%%%%%%%%%%%%%%%%%%%%%
\subsection{Forwarding}
\label{sec:forward}

Different versions of the main or child documents
using compilation flags as described in \secref{sec:flags}
can be (permanently) stored in different files
for convenient compilation, viewing and distribution.
To this end, the package defines a command
to pass on compilation to a different file:

%%%%%%%%%%%%%%%%%%%%%%%%%%%%%%%%%%%%%%%%
\DescribeMacro{\childdocforward}
The command |\childdocforward| redirects processing to
another source file:
%
\begin{center}
\begin{tabular}{l}
|\input{childdoc.def}|\\
|\childdocforward[|\textit{main}|]{|\textit{dest}|}|\\
\end{tabular}
\end{center}
%
The argument \textit{dest} is the destination file
(without extension).
It should be the main file or one of the child files.
Note that further \textsf{childdoc} directives
such as |\childdocof| and |\childdocforward|
in the indicated file will be processed in this form.
The optional argument \textit{main}
passes on directly to the main file \textit{main}
while pretending to compile the child \textit{dest}.
This form behaves as if \textit{dest}
issues |\childdocof{|\textit{main}|}| right away,
and no further \textsf{childdoc} directives will be processed.

%%%%%%%%%%%%%%%%%%%%%%%%%%%%%%%%%%%%%%%%
\DescribeMacro{\...prefix}
In the alternative form |\childdocforwardprefix|,
%
\begin{center}
\begin{tabular}{l}
|\input{childdoc.def}|\\
|\childdocforwardprefix[|\textit{main}|]{|\textit{prefix}|}{|\textit{dest}|}|
\end{tabular}
\end{center}
%
the destination file is determined by a pattern
depending on the current file:
To make this work, the current file must be called
`{\textit{prefix}\hspace{0.2em}\textit{suffix}}'
with \textit{prefix} matching precisely the argument.
Processing is then passed on to the file
`{\textit{dest}\hspace{0.2em}\textit{suffix}}'.
Surely, the same effect is achieved by
directly specifying the
argument `{\textit{dest}\hspace{0.2em}\textit{suffix}}'
in the first form.
However, that requires to set up a different file
for each child. With the alternative form of the command
all these files can have exactly the same content
which simplifies setting them up and maintaining them.

For example, the following file |draft.tex|
with a compilation flag |\version| as described in \secref{sec:flags}
compiles the main document as a draft:
%
\begin{center}
\begin{tabular}{l}
|\def\version{draft}|\\
|\input{childdoc.def}|\\
|\childdocforward{|\textit{main}|}|
\end{tabular}
\end{center}
%
Likewise, the following files |final|\textit{nn}|.tex|
compile the final version of the child document
|child|\textit{nn}|.tex|:
%
\begin{center}
\begin{tabular}{l}
|\def\version{final}|\\
|\input{childdoc.def}|\\
|\childdocforwardprefix{final}{child}|
\end{tabular}
\end{center}
%

Note that when several versions of a main file and/or of each child file
are to be generated, it may be convenient to set up a |Makefile| or
shell script to automatise the process.

%%%%%%%%%%%%%%%%%%%%%%%%%%%%%%%%%%%%%%%%%%%%%%%%%%%%%%%%%%%%%%%%%%%%%%%%%%%%%%%%
\subsection{Command Line Processing}
\label{sec:commandline}

The effect of redirection files can also be achieved by invoking
the \LaTeX{} compiler with a more elaborate command line.
Most conveniently this should be done as part
of a shell script or a |Makefile|.

When using \textsf{childdoc} in the main file, the following
command lines effectively perform a redirection
(note that depending on the shell being used,
backslashes may have to be doubled: `|\|' $\to$ `|\\|'):
%
\begin{center}
|... -jobname "|\textit{target}|" |\\|"|[\textit{flags}]%
|\input{childdoc.def}\childdocforward[|\textit{main}|]{|\textit{dest}|}"|
\end{center}
%
Here \textit{target} is the name of the output file,
\textit{main} is the name of the main file
and \textit{dest} is the name of the main or child file to be processed
(all filenames without extensions).
The optional argument \textit{main} can be omitted
if \textit{main} matches \textit{dest}.
Optionally, compilation \textit{flags} can be defined via |\def| commands.
This command line makes the \TeX{} engine believe
it is compiling the file \textit{target}
whose content is specified as the latter parameter.
The provided code then forwards the processing to
\textit{main} or \textit{dest} as described in \secref{sec:forward}.

%%%%%%%%%%%%%%%%%%%%%%%%%%%%%%%%%%%%%%%%%%%%%%%%%%%%%%%%%%%%%%%%%%%%%%%%%%%%%%%%
\subsection{Include by Input}
\label{sec:input}

Including child documents by |\include| has some restrictions by design.
Most notably, the content of a child document always occupies
its own set of pages; pages cannot be shared between child documents.
Usually, this behaviour makes perfect sense
because each child document contain an essential part of the document.
However, in some situations it may be desirable to compose
a document from a collection of parts
without having mandatory page breaks between then.
For this case, the package
provides a mechanism to include parts
by |\input| which can also be processed individually.
However, by construction this mechanism
requires manual handling of the content to be output.

%%%%%%%%%%%%%%%%%%%%%%%%%%%%%%%%%%%%%%%%
\DescribeMacro{\ifchilddocmanual}
The main file should be prepared as usual, see \secref{sec:include}.
However, the document body must make a distinction
between processing of an individual part and of the main document, e.g.:
%
\begin{center}
\begin{tabular}{l}
|\ifchilddocmanual|\\
|\input{\childdocname}|\\
|\||else|\\
\textit{document body with }|\input{|\textit{part}|}|\\
|\||fi|
\end{tabular}
\end{center}
%
The conditional |\ifchilddocmanual| is true whenever
a part to be included by |\input| is being compiled,
and the name of the part is stored in |\childdocname|.

%%%%%%%%%%%%%%%%%%%%%%%%%%%%%%%%%%%%%%%%
\DescribeMacro{\childdocby}
Each part to be included by |\input| should start with:
%
\begin{center}
\begin{tabular}{l}
|\input{childdoc.def}|\\
|\childdocby{|\textit{main}|}|\\
\end{tabular}
\end{center}
%
The directive |\childdocby| is similar to |\childdocof|
described in \secref{sec:include},
but the subsequent selection of content must be done manually.
To that end, both |\ifchilddoc| and |\ifchilddocmanual|
will be true upon processing of a part,
and the name of the part is stored in |\childdocname|.
Note that |\jobname| will be set to the filename of the current part
so that each part receives an individual |.aux| file
that does not interfere with the |.aux| file(s) of the main document.
This behaviour can be altered by the alternative form
|\childdocby[*]{|\textit{main}|}| (with a non-empty optional argument)
which uses the |.aux| file of the main document
by setting |\jobname| to \textit{main}.

%%%%%%%%%%%%%%%%%%%%%%%%%%%%%%%%%%%%%%%%%%%%%%%%%%%%%%%%%%%%%%%%%%%%%%%%%%%%%%%%
\subsection{Driver Development}
\label{sec:driver}

The \textsf{childdoc} mechanism can also be use for the development
of definition files such as \LaTeX{} styles or classes.
This case differs from the above setup with multiple parts
included by |\include| in that no |\includeonly| should be invoked.
This can be achieved by starting the include file
(before |\ProvidesPackage|) with:
%
\begin{center}
\begin{tabular}{l}
|\input{childdoc.def}|\\
|\childdocforward{|\textit{main}|}|\\
\end{tabular}
\end{center}
%
or alternatively with:
%
\begin{center}
\begin{tabular}{l}
|\input{childdoc.def}|\\
|\childdocby{|\textit{main}|}|\\
\end{tabular}
\end{center}
%
Both forms have slightly different effects as described above.
The main file is prepared as usual, see \secref{sec:include}.

%%%%%%%%%%%%%%%%%%%%%%%%%%%%%%%%%%%%%%%%%%%%%%%%%%%%%%%%%%%%%%%%%%%%%%%%%%%%%%%%
\subsection{Legacy Detection}
\label{sec:detection}

The directive |\childdocmain| in the main file can detect
whether the complete document or merely a child is to be compiled
even without using the directive |\childdocof|.
This method is deprecated because it is less robust
and there is no compelling reason to use it;
it is merely provided for backward compatibility
and it may be removed in future versions.

If the detection mechanism is to be used,
it is mandatory to correctly specify
the filename of the main file as the argument of |\childdocmain|:
%
\begin{center}
\begin{tabular}{l}
|\input{childdoc.def}|\\
|\childdocmain{|\textit{main}|}|\\
\end{tabular}
\end{center}
%
If |\jobname| does not match the argument \textit{main} of |\childdocmain|,
it is assumed that |\jobname| points to the child file to be compiled.
When using |\childdocmain| with the main file specified as argument,
it suffices to start a child file
with just |\input{|\textit{main}|}|
without loading of the package and using |\childdocof|.
If instead all processing is done
with the appropriate \textsf{childdoc} directives,
the argument of \textit{main} of |\childdocmain| can be empty.

An alternative version of the command line processing described
in \secref{sec:commandline} using the detection mechanism reads:
%
\begin{center}
|... -jobname "|\textit{target}|" "|[\textit{flags}]%
[|\def\jobname{|\textit{dest}|}|]|\input{|\textit{main}|}"|
\end{center}

%%%%%%%%%%%%%%%%%%%%%%%%%%%%%%%%%%%%%%%%%%%%%%%%%%%%%%%%%%%%%%%%%%%%%%%%%%%%%%%%
\subsection{Manual Code}
\label{sec:manual}

In case one cannot be certain whether the definitions file |childdoc.def|
is installed on the target \TeX{} distribution
and one prefers not to ship it,
it is conceivable to paste a few relevant commands into the sources.

To that end, drop all statements |\input{childdoc.def}|
and perform the replacements as outlined below.
Instead of |\childdocmain{|\textit{main}|}| add the following code
to the top of the main file:
%
\begin{center}
\begin{tabular}{l}
|\||ifdefined\childdocname\endinput\||fi\newif\ifchilddoc|\\
|\edef\childdocname{\scantokens\expandafter{\jobname\noexpand}}|\\
|\def\childdocmain{|\textit{main}|}\||ifx\childdocmain\childdocname\||else|\\
|\childdoctrue\includeonly{\childdocname}\let\jobname\childdocmain\||fi|\\
\end{tabular}
\end{center}
%
Instead of |\childdocof{|\textit{main}|}| just include the main file
at the top of each child file:
%
\begin{center}
|\input{|\textit{main}|}|
\end{center}
%
A simple redirection |\childdocforward{|\textit{dest}|}| is achieved by:
%
\begin{center}
|\def\jobname{|\textit{dest}|}\input{\jobname}|
\end{center}
%
The redirection with prefix
|\childdocforwardprefix[|\textit{prefix}|]{|\textit{dest}|}|
is accomplished by:
%
\begin{center}
\begin{tabular}{l}
|{\edef\jobname{\scantokens\expandafter{\jobname\noexpand}}|\\
|\def\redirectjob |\textit{prefix}|#1~~~{\gdef\jobname{|\textit{dest}|#1}}|\\
|\expandafter\redirectjob\jobname~~~}\input{\jobname}|
\end{tabular}
\end{center}

In an alternative approach,
child documents can be compiled by a specific command line
without additional code or specific definitions:
%
\begin{center}
|... -jobname "|\textit{target}|" "|[\textit{flags}]%
|\includeonly{|\textit{dest}|}\input{|\textit{main}|}"|
\end{center}
%

%%%%%%%%%%%%%%%%%%%%%%%%%%%%%%%%%%%%%%%%%%%%%%%%%%%%%%%%%%%%%%%%%%%%%%%%%%%%%%%%
%%%%%%%%%%%%%%%%%%%%%%%%%%%%%%%%%%%%%%%%%%%%%%%%%%%%%%%%%%%%%%%%%%%%%%%%%%%%%%%%
\section{Information}

%%%%%%%%%%%%%%%%%%%%%%%%%%%%%%%%%%%%%%%%%%%%%%%%%%%%%%%%%%%%%%%%%%%%%%%%%%%%%%%%
\subsection{Copyright}

Copyright \copyright{} 2017--2018 Niklas Beisert

This work may be distributed and/or modified under the
conditions of the \LaTeX{} Project Public License, either version 1.3
of this license or (at your option) any later version.
The latest version of this license is in
  \url{http://www.latex-project.org/lppl.txt}
and version 1.3 or later is part of all distributions of \LaTeX{}
version 2005/12/01 or later.

This work has the LPPL maintenance status `maintained'.

The Current Maintainer of this work is Niklas Beisert.

This work consists of the files |README.txt|, |childdoc.ins| and |childdoc.dtx|
as well as the derived files |childdoc.def|, |cdocsamp.tex|
with |cdocsch1.tex|, |cdocsch2.tex|, |cdocspt3.tex|, |cdocspt4.tex|,
|cdocsdrf.tex|, |cdocsfn1.tex|, |cdocsfn2.tex|
as well as |childdoc.pdf|.

%%%%%%%%%%%%%%%%%%%%%%%%%%%%%%%%%%%%%%%%%%%%%%%%%%%%%%%%%%%%%%%%%%%%%%%%%%%%%%%%
\subsection{Files and Installation}

The package consists of the files:
%
\begin{center}
\begin{tabular}{ll}
    |README.txt|   & readme file \\
    |childdoc.ins| & installation file \\
    |childdoc.dtx| & source file \\
    |childdoc.def| & definition file \\
    |cdocsamp.tex| & sample main file \\
    |cdocsch1.tex| & sample include file \\
    |cdocsch2.tex| & sample include file \\
    |cdocspt3.tex| & sample part file \\
    |cdocspt4.tex| & sample part file \\
    |cdocsdrf.tex| & sample redirection file \\
    |cdocsfn1.tex| & sample redirection file \\
    |cdocsfn2.tex| & sample redirection file \\
    |childdoc.pdf| & manual
\end{tabular}
\end{center}
%
The distribution consists of the files
|README.txt|, |childdoc.ins| and |childdoc.dtx|.
%
\begin{itemize}
\item
Run (pdf)\LaTeX{} on |childdoc.dtx|
to compile the manual |childdoc.pdf| (this file).
\item
Run \LaTeX{} on |childdoc.ins| to create the definitions file |childdoc.def|
and the sample |cdocsamp.tex| with include files
|cdocsch1.tex|, |cdocsch2.tex|, |cdocspt3.tex|, |cdocspt4.tex|,
|cdocsdrf.tex|, |cdocsfn1.tex|, |cdocsfn2.tex|.
Then copy the file |childdoc.def| to an appropriate directory of your \LaTeX{}
distribution, e.g.\ \textit{texmf-root}|/tex/latex/childdoc|.
\end{itemize}

%%%%%%%%%%%%%%%%%%%%%%%%%%%%%%%%%%%%%%%%%%%%%%%%%%%%%%%%%%%%%%%%%%%%%%%%%%%%%%%%
\subsection{Related CTAN Packages}

There are several other packages which offer a similar functionality:
%
\begin{itemize}
\item
The packages
\href{http://ctan.org/pkg/docmute}{\textsf{docmute}},
\href{http://ctan.org/pkg/includex}{\textsf{includex}} and
\href{http://ctan.org/pkg/standalone}{\textsf{standalone}}
provide commands to include only the document body of
a child file thus allowing both files to be compiled individually.
\item
The packages \href{http://ctan.org/pkg/subdocs}{\textsf{subdocs}}
and \href{http://ctan.org/pkg/subfiles}{\textsf{subfiles}}
provide structures in which the main and child documents can be
encapsulated and allowing them to be compiled individually.
The inclusion mechanism is different from the conventional |\include|.
\item
The package \href{http://ctan.org/pkg/combine}{\textsf{combine}}
is an elaborate solution to combine several documents into one.
\end{itemize}
%
See also the CTAN topic \href{http://ctan.org/topic/subdocs}{\textsf{subdocs}}
for further related packages.
The present package differs from the above solutions in that
a document structure constructed with the conventional |\include| mechanism
just needs two extra commands at the top of every file
such that all constituent files can be compiled individually.

%%%%%%%%%%%%%%%%%%%%%%%%%%%%%%%%%%%%%%%%%%%%%%%%%%%%%%%%%%%%%%%%%%%%%%%%%%%%%%%%
%\subsection{Feature Suggestions}
%
%The following is a list of features which may be useful for future
%versions of this package:
%%
%\begin{itemize}
%\item
%\ldots
%\end{itemize}

%%%%%%%%%%%%%%%%%%%%%%%%%%%%%%%%%%%%%%%%%%%%%%%%%%%%%%%%%%%%%%%%%%%%%%%%%%%%%%%%
\subsection{Revision History}

%%%%%%%%%%%%%%%%%%%%%%%%%%%%%%%%%%%%%%%%
\paragraph{v2.0:} 2018/12/30

\begin{itemize}
\item
immediate forward processing
\item
added |\childdocby| mechanism
\item
manual restructured
\end{itemize}

%%%%%%%%%%%%%%%%%%%%%%%%%%%%%%%%%%%%%%%%
\paragraph{v1.6:} 2018/01/17

\begin{itemize}
\item
application for development of include files
\item
corrections to manual
\end{itemize}

%%%%%%%%%%%%%%%%%%%%%%%%%%%%%%%%%%%%%%%%
\paragraph{v1.5:} 2017/05/21

\begin{itemize}
\item
more complete structuring introduced
\item
|\childdocof| introduced
\item
|\childdoc| renamed to |\childdocmain|
\item
|\childredirect| renamed to |\childdocforward| and |\childdocforwardprefix|
and functionality expanded
\end{itemize}

%%%%%%%%%%%%%%%%%%%%%%%%%%%%%%%%%%%%%%%%
\paragraph{v1.0:} 2017/04/27

\begin{itemize}
\item
manual and install package
\item
first version published on CTAN
\end{itemize}

%%%%%%%%%%%%%%%%%%%%%%%%%%%%%%%%%%%%%%%%
\paragraph{v0.6:} 2017/04/26

\begin{itemize}
\item
redirection mechanism added
\end{itemize}

%%%%%%%%%%%%%%%%%%%%%%%%%%%%%%%%%%%%%%%%
\paragraph{v0.5:} 2017/04/26

\begin{itemize}
\item
functionality in definition file
\end{itemize}


%%%%%%%%%%%%%%%%%%%%%%%%%%%%%%%%%%%%%%%%%%%%%%%%%%%%%%%%%%%%%%%%%%%%%%%%%%%%%%%%
%%%%%%%%%%%%%%%%%%%%%%%%%%%%%%%%%%%%%%%%%%%%%%%%%%%%%%%%%%%%%%%%%%%%%%%%%%%%%%%%
%%%%%%%%%%%%%%%%%%%%%%%%%%%%%%%%%%%%%%%%%%%%%%%%%%%%%%%%%%%%%%%%%%%%%%%%%%%%%%%%
\appendix

\settowidth\MacroIndent{\rmfamily\scriptsize 000\ }

 \DocInput{childdoc.dtx}

\end{document}
%</driver>
% \fi
%
% %%%%%%%%%%%%%%%%%%%%%%%%%%%%%%%%%%%%%%%%%%%%%%%%%%%%%%%%%%%%%%%%%%%%%%%%%%%%%%
% %%%%%%%%%%%%%%%%%%%%%%%%%%%%%%%%%%%%%%%%%%%%%%%%%%%%%%%%%%%%%%%%%%%%%%%%%%%%%%
% \section{Sample}
%\iffalse
%<*samplemain>
%\fi
%
% The following presents a sample document
% with two chapters, two parts, a title page,
% a compile flag as well as three forwarding files to set the flag.
% It consists of eight |.tex| files:
% \begin{center}
% \begin{tabular}{ll}
% |cdocsamp.tex|&main file\\
% |cdocsch1.tex|&include file for chapter 1\\
% |cdocsch2.tex|&include file for chapter 2\\
% |cdocspt3.tex|&include file for part 3\\
% |cdocspt4.tex|&include file for part 4\\
% |cdocsdrf.tex|&forwarding file for main file in draft mode\\
% |cdocsfi1.tex|&forwarding file for final version of chapter 1\\
% |cdocsfi2.tex|&forwarding file for final version of chapter 2\\
% \end{tabular}
% \end{center}
% Each of the eight files can be compiled directly by the \LaTeX{} compiler.
%
% %%%%%%%%%%%%%%%%%%%%%%%%%%%%%%%%%%%%%%
% \paragraph{Main File.}
%
% The main file is called |cdocsamp.tex|.
%
% Load the \textsf{childdoc} definitions and
% declare the filename for the main document:
%    \begin{macrocode}
\input{childdoc.def}
\childdocmain{}
%    \end{macrocode}

% Optional override for |\version| flag:
%    \begin{macrocode}
%%\ifchilddoc\else\providecommand{\version}{draft}\fi
%    \end{macrocode}

% Define the default values for the |\version| flag
% (|final| for the main file and |draft| for childs):
%    \begin{macrocode}
\ifchilddoc
\providecommand{\version}{draft}
\else
\providecommand{\version}{final}
\fi
%    \end{macrocode}

% Load the standard document class:
%    \begin{macrocode}
\documentclass[12pt]{article}
%    \end{macrocode}

% Start the document body:
%    \begin{macrocode}
\begin{document}
%    \end{macrocode}

% Declare a title page.
% Print title, part of document being processed and version flag:
%    \begin{macrocode}
\addtocounter{page}{-1}
\begin{center}
{\LARGE\bfseries{}childdoc example\par}
\vspace{1cm}
\ifchilddoc
\ifchilddocmanual part\else chapter\fi:
`\childdocname' of `\childdocjob'\par
\else
main document: `\childdocjob'\par
\fi
version: \version\par
\end{center}
\newpage
%    \end{macrocode}

% Manually include selected file,
% otherwise process as usual:
%    \begin{macrocode}
\ifchilddocmanual
\section*{part `\childdocname'}
\input{\childdocname}
\else
%    \end{macrocode}

% Include the two chapters:
%    \begin{macrocode}
\include{cdocsch1}
\include{cdocsch2}
%    \end{macrocode}

% Include the two parts unless only chapters should be displayed:
%    \begin{macrocode}
\ifchilddoc\else
\section{part three}
\input{cdocspt3}
\section{part four}
\input{cdocspt4}
\fi
%    \end{macrocode}

% Process as usual until here:
%    \begin{macrocode}
\fi
%    \end{macrocode}

% End of document body:
%    \begin{macrocode}
\end{document}
%    \end{macrocode}
%\iffalse
%</samplemain>
%\fi
%
% %%%%%%%%%%%%%%%%%%%%%%%%%%%%%%%%%%%%%%
% \paragraph{Chapter Include Files.}
%
% The include files are called |cdocsch1.tex| and |cdocsch2.tex|.
%
%\iffalse
%<*samplechap1|samplechap2>
%\fi

% Optional override for |\version| flag:
%    \begin{macrocode}
%%\providecommand{\version}{final}
%    \end{macrocode}

% Include the main document:
%    \begin{macrocode}
\input{childdoc.def}
\childdocof{cdocsamp}
%    \end{macrocode}

%\iffalse
%</samplechap1|samplechap2>
%\fi
%
%\iffalse
%<*samplechap1>
%\fi
% Some text for chapter 1:
%    \begin{macrocode}
\section{one}
some text in chapter one
%    \end{macrocode}

%\iffalse
%</samplechap1>
%\fi
% Some text for chapter 2:
%\iffalse
%<*samplechap2>
%\fi
%    \begin{macrocode}
\section{two}
more text in chapter two
%    \end{macrocode}

%\iffalse
%</samplechap2>
%\fi
%
% %%%%%%%%%%%%%%%%%%%%%%%%%%%%%%%%%%%%%%
% \paragraph{Part Include Files.}
%
% The include files are called |cdocspt3.tex| and |cdocspt4.tex|.
%
%\iffalse
%<*samplepart3|samplepart4>
%\fi

% Optional override for |\version| flag:
%    \begin{macrocode}
%%\providecommand{\version}{final}
%    \end{macrocode}

% Include the main document:
%    \begin{macrocode}
\input{childdoc.def}
\childdocby{cdocsamp}
%    \end{macrocode}

%\iffalse
%</samplepart3|samplepart4>
%\fi
%
%\iffalse
%<*samplepart3>
%\fi
% Some text for part 3:
%    \begin{macrocode}
some text in part three
%    \end{macrocode}

%\iffalse
%</samplepart3>
%\fi
% Some text for part 4:
%\iffalse
%<*samplepart4>
%\fi
%    \begin{macrocode}
more text in part four
%    \end{macrocode}

%\iffalse
%</samplepart4>
%\fi
%
% %%%%%%%%%%%%%%%%%%%%%%%%%%%%%%%%%%%%%%
% \paragraph{Forwarding for a Complete Draft.}
%
% The following forwarding file |cdocsdrf.tex|
% compiles the main document in draft mode:
%\iffalse
%<*sampledraft>
%\fi
%    \begin{macrocode}
\def\version{draft}
\input{childdoc.def}
\childdocforward{cdocsamp}
%    \end{macrocode}

%\iffalse
%</sampledraft>
%\fi
%
% %%%%%%%%%%%%%%%%%%%%%%%%%%%%%%%%%%%%%%
% \paragraph{Forwarding for Final Version of the Chapters.}
%
% The following forwarding files |cdocsfn1.tex| and |cdocsfn2.tex|
% (with identical content)
% compile the final versions of the child documents
% |cdocsch1.tex| and |cdocsch2.tex|, respectively:
%\iffalse
%<*samplefinal>
%\fi
%    \begin{macrocode}
\def\version{final}
\input{childdoc.def}
\childdocforwardprefix[cdocsamp]{cdocsfn}{cdocsch}
%    \end{macrocode}

%\iffalse
%</samplefinal>
%\fi
%
% %%%%%%%%%%%%%%%%%%%%%%%%%%%%%%%%%%%%%%
% \paragraph{Command Line Processing.}
%
% The following three command lines generate the output files
% |cdocscld|, |cdocscl1| and |cdocscl2|
% which should be identical to
% |cdocsdrf|, |cdocsch1| and |cdocsfn2|, respectively:
% \begin{center}
% \begin{tabular}{l}
% |latex -jobname cdocscld \|\\
% |  "\def\version{draft}\input{childdoc.def}\childdocforward{cdocsamp}"|\\
% |latex -jobname cdocscl1 \|\\
% |  "\input{childdoc.def}\childdocforward[cdocsamp]{cdocsch1}"|\\
% |latex -jobname cdocscl2 \|\\
% |  "\def\version{final}\input{childdoc.def}\childdocforward{cdocsch2}"|
% \end{tabular}
% \end{center}
% Note that the trailing backslash on each first line
% merely continues the input to the second line
% (for convenient cut ant paste).
% Furthermore, the command |latex| can be replaced by any
% of its alternative versions such as |pdflatex|.
%
% %%%%%%%%%%%%%%%%%%%%%%%%%%%%%%%%%%%%%%%%%%%%%%%%%%%%%%%%%%%%%%%%%%%%%%%%%%%%%%
% %%%%%%%%%%%%%%%%%%%%%%%%%%%%%%%%%%%%%%%%%%%%%%%%%%%%%%%%%%%%%%%%%%%%%%%%%%%%%%
% \section{Implementation}
%\iffalse
%<*package>
%\fi
%
% This section describes the definitions file |childdoc.def|.

% The definitions cannot be loaded using |\usepackage| or |\RequirePackage|
% which has a mechanism to prevent loading a style file more than once.
% When loading the definitions by means of |\input|
% multiple instances have to be prevented manually:
%\iffalse
%This code needs to be before the `\ProvidesFile' directive
%which is defined at the beginning of this file.
%Therefore it is also placed there and commented out here.
%</package>
%<*discard>
%\fi
%    \begin{macrocode}
\ifdefined\childdocmain\endinput\fi
%    \end{macrocode}
%\iffalse
%</discard>
%<*package>
%\fi
%
% \macro{\ifchilddoc}
% \macro{\ifchilddocmanual}
% The conditional |\ifchilddoc| tells whether a
% child (true) or main (false) document is being compiled.
% The conditional |\ifchilddocmanual| tells whether
% the |\includeonly| mechanism is used (false) or
% the selection of child files must be performed manually (true).
% The definitions initialise to false:
%    \begin{macrocode}
\newif\ifchilddoc
\newif\ifchilddocmanual
%    \end{macrocode}

% \macro{\childdocname}
% \macro{\childdocjob}
% The macro |\childdocname| stores the name of the main document
% to be compiled. The macro |\childdocjob| stores the name of
% the document on which the \LaTeX{} compiler was originally invoked.
% The content of |\jobname| cannot be compared
% to filenames specified in the source due to different catcodes.
% The following code rescans |\jobname|, stores the result
% in |\childdocname| and saves a copy in |\childdocjob|:
%    \begin{macrocode}
\edef\childdocname{\scantokens\expandafter{\jobname\noexpand}}
\let\childdocjob\childdocname
%    \end{macrocode}

% \macro{\childdocdisable}
% The macro |\childdocdisable| prevents the main file
% from being processed more than once.
% At this stage, the main document command |\childdocmain|
% is assumed to be called once again where it should do nothing.
% Any subsequent call to it should prevent
% a secondary processing of the main document
% It overwrites the forwarding commands
% |\childdocof| and |\childdocforward|
% with empty macros to prevent further inclusions of the main document:
%    \begin{macrocode}
\newcommand{\childdocdisable}
{
  \renewcommand{\childdocmain}[1]{\renewcommand{\childdocmain}[1]{\endinput}}
  \renewcommand{\childdocof}[1]{}
  \renewcommand{\childdocby}[2][]{}
  \renewcommand{\childdocforward}[2][]{}
  \renewcommand{\childdocdisable}{}
}
%    \end{macrocode}

% \macro{\childdocmain}
% The macro |\childdocmain| is to be called at the top of the main file
% with nothing or the main filename (without extension) as argument.
% First, it breaks loops.
% If the argument is not empty and does not match |\childdocname|
% (which is set by the first inclusion of |childdoc.def|),
% |\ifchilddoc| is set to true, |\includeonly| is applied to the child file
% and |\jobname| is set to the main file
% (for proper handling of |.aux| files):
%    \begin{macrocode}
\newcommand{\childdocmain}[1]
{
  \childdocdisable\childdocmain{}
  \if?#1?\else
    \begingroup
      \def\childdoctmp{#1}
      \ifx\childdoctmp\childdocname
        \def\childdoctmp{}
      \else
        \def\childdoctmp
        {
          \childdoctrue
          \includeonly{\childdocname}
          \def\childdocjob{#1}
          \def\jobname{#1}
        }
      \fi
      \expandafter
    \endgroup
    \childdoctmp
  \fi
}
%    \end{macrocode}

% \macro{\childdocof}
% The command |\childdocof| redirects
% compilation to the main file |#1|.
%    \begin{macrocode}
\newcommand{\childdocof}[1]
{
  \childdocdisable
  \childdoctrue
  \includeonly{\childdocname}
  \def\jobname{#1}
  \def\childdocjob{#1}
  \input{#1}
}
%    \end{macrocode}

% \macro{\childdocby}
% The command |\childdocby| ....
%    \begin{macrocode}
\newcommand{\childdocby}[2][]
{
  \childdocdisable
  \childdoctrue
  \childdocmanualtrue
  \if?#1?\else
    \def\jobname{#2}
  \fi
  \def\childdocjob{#2}
  \input{#2}
  \endinput
}
%    \end{macrocode}

% \macro{\childdocforward}
% The command |\childdocforward| redirects
% compilation to the main file or
% (if the optional argument is given) a child file.
% Parameters are set as if the main file
% or a child file starting with |\childdocof| was compiled.
% Then compilation is handed over to the main file:
%    \begin{macrocode}
\newcommand{\childdocforward}[2][]
{
  \begingroup
    \if?#1?
      \def\childdoctmp
      {
        \def\childdocname{#2}
        \def\childdocjob{#2}
        \def\jobname{#2}
        \input{#2}
        \endinput
      }
    \else
      \def\childdoctmp
      {
        \childdocdisable
        \def\childdocname{#2}
        \childdoctrue
        \includeonly{#2}
        \def\childdocjob{#1}
        \def\jobname{#1}
        \input{#1}
        \endinput
      }
    \fi
    \expandafter
  \endgroup
  \childdoctmp
}
%    \end{macrocode}

% \macro{\childdocforwardprefix}
% The command |\childdocforwardprefix| redirects
% compilation to the main or a child file by means of a pattern.
% The prefix |#1| in the current filename is replaced by |#2|
% and the suffix of the current filename is kept
% (it is assumed that the filename does not contain the substring `|~~~|'
% which is used as a delimiter).
% Compilation is handed over to the new file by |\childdocforward|:
%    \begin{macrocode}
\newcommand{\childdocforwardprefix}[3][]
{
  \begingroup
    \def\childdocextract #2##1~~~{\def\childdoctmp{\childdocforward[#1]{#3##1}}}
    \expandafter\childdocextract\childdocname~~~
    \expandafter
  \endgroup
  \childdoctmp
}
%    \end{macrocode}

% \macro{\childdoc}
% The deprecated macro |\childdoc| is a legacy version of |\childdocmain|:
%    \begin{macrocode}
\newcommand{\childdoc}{\childdocmain}
%    \end{macrocode}

% \macro{\childdocredirect}
% The deprecated macro |\childdocredirect| is a legacy version
% of |\childdocforward| and |\childdocforwardprefix|:
%    \begin{macrocode}
\newcommand{\childdocredirect}[2][]
{
  \begingroup
    \if?#1?
      \def\childdoctmp{\childdocforward{#2}}
    \else
      \def\childdoctmp{\childdocforwardprefix{#1}{#2}}
    \fi
    \expandafter
  \endgroup
  \childdoctmp
}
%    \end{macrocode}

%\iffalse
%</package>
%\fi
%
\endinput
|\\
|\childdocforwardprefix[|\textit{main}|]{|\textit{prefix}|}{|\textit{dest}|}|
\end{tabular}
\end{center}
%
the destination file is determined by a pattern
depending on the current file:
To make this work, the current file must be called
`{\textit{prefix}\hspace{0.2em}\textit{suffix}}'
with \textit{prefix} matching precisely the argument.
Processing is then passed on to the file
`{\textit{dest}\hspace{0.2em}\textit{suffix}}'.
Surely, the same effect is achieved by
directly specifying the
argument `{\textit{dest}\hspace{0.2em}\textit{suffix}}'
in the first form.
However, that requires to set up a different file
for each child. With the alternative form of the command
all these files can have exactly the same content
which simplifies setting them up and maintaining them.

For example, the following file |draft.tex|
with a compilation flag |\version| as described in \secref{sec:flags}
compiles the main document as a draft:
%
\begin{center}
\begin{tabular}{l}
|\def\version{draft}|\\
|% \iffalse
%
% childdoc.dtx Copyright (C) 2017-2018 Niklas Beisert
%
% This work may be distributed and/or modified under the
% conditions of the LaTeX Project Public License, either version 1.3
% of this license or (at your option) any later version.
% The latest version of this license is in
%   http://www.latex-project.org/lppl.txt
% and version 1.3 or later is part of all distributions of LaTeX
% version 2005/12/01 or later.
%
% This work has the LPPL maintenance status `maintained'.
%
% The Current Maintainer of this work is Niklas Beisert.
%
% This work consists of the files childdoc.dtx and childdoc.ins
% and the derived files childdoc.def and cdocsamp.tex with
% cdocsch1.tex, cdocsch2.tex, cdocsdrf.tex, cdocsfn1.tex, cdocsfn2.tex.
%
%<package>\ifdefined\childdocmain\endinput\fi
%<package>\ProvidesFile{childdoc.def}[2018/12/30 v2.0 child document driver]
%<samplemain>\ProvidesFile{cdocsamp.tex}[2018/12/30 v2.0 sample for childdoc]
%<*driver>
%\ProvidesFile{childdoc.drv}[2018/12/30 v2.0 childdoc reference manual file]
\PassOptionsToClass{10pt,a4paper}{article}
\documentclass{ltxdoc}

\usepackage[margin=35mm]{geometry}
\usepackage{hyperref}
\usepackage{hyperxmp}
\usepackage[usenames]{color}

\hypersetup{colorlinks=true}
\hypersetup{pdfstartview=FitH}
\hypersetup{pdfpagemode=UseNone}
\hypersetup{pdfsource={}}
\hypersetup{pdflang={en-UK}}
\hypersetup{pdfcopyright={Copyright 2017-2018 Niklas Beisert.
  This work may be distributed and/or modified under the
  conditions of the LaTeX Project Public License, either version 1.3
  of this license or (at your option) any later version.}}
\hypersetup{pdflicenseurl={http://www.latex-project.org/lppl.txt}}
\hypersetup{pdfcontactaddress={ETH Zurich, ITP, HIT K,
  Wolfgang-Pauli-Strasse 27}}
\hypersetup{pdfcontactpostcode={8093}}
\hypersetup{pdfcontactcity={Zurich}}
\hypersetup{pdfcontactcountry={Switzerland}}
\hypersetup{pdfcontactemail={nbeisert@itp.phys.ethz.ch}}
\hypersetup{pdfcontacturl={http://people.phys.ethz.ch/\xmptilde nbeisert/}}

\newcommand{\secref}[1]{\hyperref[#1]{section \ref*{#1}}}

\parskip1ex
\parindent0pt
\let\olditemize\itemize
\def\itemize{\olditemize\parskip0pt}

\begin{document}

\title{The \textsf{childdoc} Package}
\hypersetup{pdftitle={The childdoc Package}}
\author{Niklas Beisert\\[2ex]
  Institut f\"ur Theoretische Physik\\
  Eidgen\"ossische Technische Hochschule Z\"urich\\
  Wolfgang-Pauli-Strasse 27, 8093 Z\"urich, Switzerland\\[1ex]
  \href{mailto:nbeisert@itp.phys.ethz.ch}
  {\texttt{nbeisert@itp.phys.ethz.ch}}}
\hypersetup{pdfauthor={Niklas Beisert}}
\hypersetup{pdfsubject={Manual for the LaTeX2e Package childdoc}}
\date{30 December 2018, \textsf{v2.0}}
\maketitle

\begin{abstract}\noindent
\textsf{childdoc} is a \LaTeXe{} package
that enables the direct compilation
of document sections included by |\include|
to individual files.
\end{abstract}

\begingroup
\parskip0ex
\tableofcontents
\endgroup

%%%%%%%%%%%%%%%%%%%%%%%%%%%%%%%%%%%%%%%%%%%%%%%%%%%%%%%%%%%%%%%%%%%%%%%%%%%%%%%%
%%%%%%%%%%%%%%%%%%%%%%%%%%%%%%%%%%%%%%%%%%%%%%%%%%%%%%%%%%%%%%%%%%%%%%%%%%%%%%%%
\section{Introduction}

\LaTeX{} provides a mechanism to structure a large document (such as a book)
into a main file and several child files (containing the chapters)
using the |\include| command.
This mechanism is beneficial for documents
which span hundreds of pages in order to
make the source file(s) more manageable.
Moreover, compilation can be restricted to
selected child files by means of the |\includeonly| command.
The latter feature can be used to reduce the compilation time while editing
(this was significantly more useful in the earlier days of \LaTeX{})
or to generate a smaller document which is easier to navigate.
Another application of |\includeonly| is to generate
documents consisting of selected parts of the complete document.

However, there are a few drawbacks of the plain |\include| mechanism:
\begin{itemize}
\item
The child files cannot be compiled on their own,
they can only be compiled via the main file.
A naive editing environment
(such as a text editor with an option
to have the current file processed by \LaTeX)
may require one to switch to the main file before compiling;
attempting to compile the child file produces errors.
\item
The main file must be modified (each time)
to adjust the |\includeonly| command
to the present needs. This easily leaves the main file in a messy state.
\item
The generated document will always carry the filename
of the main document. This is inconvenient if
several child files are to be compiled and
to be kept for distribution.
\end{itemize}

The present package provides a simple interface
to make child files individually compilable by \LaTeX{}.
Compiling a child file then has the same effect as compiling
the main file with an |\includeonly| command
to select the appropriate child.
Moreover the generated document will carry the name of the child
rather than the main file.
This resolves all three above issues.

This feature is meant to make the editing of books,
thesis documents and lecture notes somewhat more convenient.
However, the package can also be used efficiently for
composing a series of documents (such as exercise sheets)
which are typically distributed individually.
It then assists the author in generating the individual documents
(potentially in different versions)
as well as a document containing the collected series.
Another application is in developing style files
or other kinds of included material
where compilation of the style file could redirect
to a sample or test file.

%%%%%%%%%%%%%%%%%%%%%%%%%%%%%%%%%%%%%%%%%%%%%%%%%%%%%%%%%%%%%%%%%%%%%%%%%%%%%%%%
%%%%%%%%%%%%%%%%%%%%%%%%%%%%%%%%%%%%%%%%%%%%%%%%%%%%%%%%%%%%%%%%%%%%%%%%%%%%%%%%
\section{Usage}

First of all, the package \textsf{childdoc} is \emph{not} a standard
\LaTeXe{} |.sty| style file! Therefore it needs to be invoked in
a non-standard way.

%%%%%%%%%%%%%%%%%%%%%%%%%%%%%%%%%%%%%%%%%%%%%%%%%%%%%%%%%%%%%%%%%%%%%%%%%%%%%%%%
\subsection{Included Files}
\label{sec:include}

%%%%%%%%%%%%%%%%%%%%%%%%%%%%%%%%%%%%%%%%
\DescribeMacro{\childdocmain}
To use the package, add the commands
\begin{center}
\begin{tabular}{l}
|\input{childdoc.def}|\\
|\childdocmain{}|\\
\end{tabular}
\end{center}
at the very top of the main \LaTeX{} file,
in particular \emph{before} the |\documentclass| statement!
The argument of |\childdocmain| should be left empty
(but it must be present).

%%%%%%%%%%%%%%%%%%%%%%%%%%%%%%%%%%%%%%%%
\DescribeMacro{\childdocof}
Furthermore, add the commands
\begin{center}
\begin{tabular}{l}
|\input{childdoc.def}|\\
|\childdocof{|\textit{main}|}|\\
\end{tabular}
\end{center}
at the top of every child file \textit{child}
which is included by |\include{|\textit{child}|}|
from within the main file
(or at least for those files to be compiled individually).
The argument \textit{main} must be the filename of the main file.

There are a couple of
considerations in setting up the main and child documents:

%%%%%%%%%%%%%%%%%%%%%%%%%%%%%%%%%%%%%%%%
\paragraph{Restrictions.}

Please note the following restrictions:
\begin{itemize}
\item
|\childdocmain| must be called with one argument \textit{main}
to ensure compatibility with earlier version of the package.
It must either be empty (|\childdocmain{}|)
or precisely match the filename of the main file in which it is specified.
See \secref{sec:detection} for further information.
\item
The filename \textit{main} must be specified without the |.tex| extension.
\item
The filename \textit{main} is case sensitive
(even in case-insensitive file systems)
due to internal string comparison.
\item
The argument \textit{main} should be fully expanded, it cannot be a macro.
\item
Subdirectories and special characters should be avoided in filenames.
\item
The command |\childdocmain{|\textit{main}|}| must be followed by a whitespace.
It should not be followed immediately by another command
or by a comment mark `|%|'.
This is because the \TeX{} parser reads the token immediately following
the argument of |\childdocmain| and puts it
at the beginning of every child section;
however, a white\-space is ignored.
\end{itemize}

%%%%%%%%%%%%%%%%%%%%%%%%%%%%%%%%%%%%%%%%
\paragraph{Content of Main File.}

It is advisable to place all content in the child files included by |\include|.
Any output contained in the main file will appear in all child documents
unless suppressed manually;
it cannot be suppressed automatically by the |\includeonly| directive
and thus should normally be avoided.
A method to include some content in the main file
by means of conditional processing is described in \secref{sec:conditional}.

%%%%%%%%%%%%%%%%%%%%%%%%%%%%%%%%%%%%%%%%
\paragraph{Page Numbering.}

When only a part of the document is compiled,
the appropriate numbering of pages
(as well as other status parameters)
is determined from the |.aux| files.
The latter contain information from previous passes.
However this information needs to propagate through
all intermediate child documents.
Therefore the page numbering in child documents may well
be inconsistent until the complete document is compiled at least once.

A useful (if unconventional) way to always ensure a consistent
page numbering is to restart the numbering in each child document
and denote the pages by `\textit{child}|.|\textit{page}'
where \textit{child} represents the chapter/section number of the child file.
This can be achieved by the command
|\numberwithin{page}{|\textit{child}|}|
of the \textsf{amsmath} package
where \textit{child} can be |chapter| or |section|
depending on the chosen structuring.
Alternatively, one can modify the macro |\thepage| appropriately
and reset the counter |page| at the start of each child file.

%%%%%%%%%%%%%%%%%%%%%%%%%%%%%%%%%%%%%%%%%%%%%%%%%%%%%%%%%%%%%%%%%%%%%%%%%%%%%%%%
\subsection{Conditional Processing}
\label{sec:conditional}

The package provides a mechanism to compile different versions
of a document. To customise the versions further some conditional processing
can come in handy to distinguish which version is being compiled.
The package provides two macros to describe the compilation context:

%%%%%%%%%%%%%%%%%%%%%%%%%%%%%%%%%%%%%%%%
\DescribeMacro{\ifchilddoc}
The conditional |\ifchilddoc| distinguishes between the compilation of
child documents and the main document:
%
\begin{center}
|\ifchilddoc |\textit{child-code}| |[|\||else |\textit{main-code}]| \||fi|
\end{center}

%%%%%%%%%%%%%%%%%%%%%%%%%%%%%%%%%%%%%%%%
\DescribeMacro{\childdocname}
\DescribeMacro{\childdocjob}
The macro |\childdocname| contains the filename (without extension)
of the main or child file being processed.
Note that |\childdocjob| will always contain the name of the main file.

%%%%%%%%%%%%%%%%%%%%%%%%%%%%%%%%%%%%%%%%
\paragraph{Title Page.}

Conditional processing can be used to include a title or banner page
in the main document when proper precautions are taken.
Importantly, the code in the main file should ensure that the page counter
(as well as other status parameters which are stored in the |.aux| files)
takes the same value after the conditional processing.
Otherwise the page numbers may take divergent values
depending on which part is compiled.

For example, a title page could be declared by:
%
\begin{center}
\begin{tabular}{l}
|\ifchilddoc\||else|\\
|\addtocounter{page}{-1}|\\
\textit{code for title page}\\
|\newpage|\\
|\||fi|
\end{tabular}
\end{center}
%
A banner page for the child documents can be generated by:
%
\begin{center}
\begin{tabular}{l}
|\ifchilddoc|\\
|\addtocounter{page}{-1}|\\
\textit{code for banner page}\\
|\newpage|\\
|\||fi|
\end{tabular}
\end{center}
%
Here one could write a message such as:
\begin{center}
|This is the part \childdocname{} of \childdocjob{}.|
\end{center}

%%%%%%%%%%%%%%%%%%%%%%%%%%%%%%%%%%%%%%%%%%%%%%%%%%%%%%%%%%%%%%%%%%%%%%%%%%%%%%%%
\subsection{Flags}
\label{sec:flags}

The package makes it easy to generate different versions
of the main or child documents.
To this end compilation flags can be defined
and assigned different default values.
They will be particularly useful in conjunction
with the forwarding mechanism described in \secref{sec:forward}.

For example, it may be useful to have a flag |\version|
which can be set to |draft| or |final|.
The document source will contain some conditional code
depending on the value of |\version|.
Suppose further, the flag should default to |final| for the main file
and to |draft| for child files
which is a natural assignment for editing the document.
This is achieved by placing the following code
in the preamble of the main document
(below the |\childdocmain| directive):
%
\begin{center}
\begin{tabular}{l}
|\ifchilddoc|\\
|\providecommand{\version}{draft}|\\
|\||else|\\
|\providecommand{\version}{final}|\\
|\||fi|
\end{tabular}
\end{center}
%
The definition by |\providecommand| makes sure
that previous definitions are not overwritten.
Further statements |\providecommand{\version}{...}|
can thus be added before the above code to override it.

For the main file, one might add a line
(between |\childdocmain| and the above block)
%
\begin{center}
|%\ifchilddoc\||else\providecommand{\version}{draft}\||fi|
\end{center}
%
which can be uncommented to produce a draft version.
Likewise one can add a line to the very top of a child file
(above the |\childdocof{|\textit{main}|}| directive)
%
\begin{center}
|%\providecommand{\version}{final}|
\end{center}
%
which can be uncommented to produce the final version of this child document.

%%%%%%%%%%%%%%%%%%%%%%%%%%%%%%%%%%%%%%%%%%%%%%%%%%%%%%%%%%%%%%%%%%%%%%%%%%%%%%%%
\subsection{Forwarding}
\label{sec:forward}

Different versions of the main or child documents
using compilation flags as described in \secref{sec:flags}
can be (permanently) stored in different files
for convenient compilation, viewing and distribution.
To this end, the package defines a command
to pass on compilation to a different file:

%%%%%%%%%%%%%%%%%%%%%%%%%%%%%%%%%%%%%%%%
\DescribeMacro{\childdocforward}
The command |\childdocforward| redirects processing to
another source file:
%
\begin{center}
\begin{tabular}{l}
|\input{childdoc.def}|\\
|\childdocforward[|\textit{main}|]{|\textit{dest}|}|\\
\end{tabular}
\end{center}
%
The argument \textit{dest} is the destination file
(without extension).
It should be the main file or one of the child files.
Note that further \textsf{childdoc} directives
such as |\childdocof| and |\childdocforward|
in the indicated file will be processed in this form.
The optional argument \textit{main}
passes on directly to the main file \textit{main}
while pretending to compile the child \textit{dest}.
This form behaves as if \textit{dest}
issues |\childdocof{|\textit{main}|}| right away,
and no further \textsf{childdoc} directives will be processed.

%%%%%%%%%%%%%%%%%%%%%%%%%%%%%%%%%%%%%%%%
\DescribeMacro{\...prefix}
In the alternative form |\childdocforwardprefix|,
%
\begin{center}
\begin{tabular}{l}
|\input{childdoc.def}|\\
|\childdocforwardprefix[|\textit{main}|]{|\textit{prefix}|}{|\textit{dest}|}|
\end{tabular}
\end{center}
%
the destination file is determined by a pattern
depending on the current file:
To make this work, the current file must be called
`{\textit{prefix}\hspace{0.2em}\textit{suffix}}'
with \textit{prefix} matching precisely the argument.
Processing is then passed on to the file
`{\textit{dest}\hspace{0.2em}\textit{suffix}}'.
Surely, the same effect is achieved by
directly specifying the
argument `{\textit{dest}\hspace{0.2em}\textit{suffix}}'
in the first form.
However, that requires to set up a different file
for each child. With the alternative form of the command
all these files can have exactly the same content
which simplifies setting them up and maintaining them.

For example, the following file |draft.tex|
with a compilation flag |\version| as described in \secref{sec:flags}
compiles the main document as a draft:
%
\begin{center}
\begin{tabular}{l}
|\def\version{draft}|\\
|\input{childdoc.def}|\\
|\childdocforward{|\textit{main}|}|
\end{tabular}
\end{center}
%
Likewise, the following files |final|\textit{nn}|.tex|
compile the final version of the child document
|child|\textit{nn}|.tex|:
%
\begin{center}
\begin{tabular}{l}
|\def\version{final}|\\
|\input{childdoc.def}|\\
|\childdocforwardprefix{final}{child}|
\end{tabular}
\end{center}
%

Note that when several versions of a main file and/or of each child file
are to be generated, it may be convenient to set up a |Makefile| or
shell script to automatise the process.

%%%%%%%%%%%%%%%%%%%%%%%%%%%%%%%%%%%%%%%%%%%%%%%%%%%%%%%%%%%%%%%%%%%%%%%%%%%%%%%%
\subsection{Command Line Processing}
\label{sec:commandline}

The effect of redirection files can also be achieved by invoking
the \LaTeX{} compiler with a more elaborate command line.
Most conveniently this should be done as part
of a shell script or a |Makefile|.

When using \textsf{childdoc} in the main file, the following
command lines effectively perform a redirection
(note that depending on the shell being used,
backslashes may have to be doubled: `|\|' $\to$ `|\\|'):
%
\begin{center}
|... -jobname "|\textit{target}|" |\\|"|[\textit{flags}]%
|\input{childdoc.def}\childdocforward[|\textit{main}|]{|\textit{dest}|}"|
\end{center}
%
Here \textit{target} is the name of the output file,
\textit{main} is the name of the main file
and \textit{dest} is the name of the main or child file to be processed
(all filenames without extensions).
The optional argument \textit{main} can be omitted
if \textit{main} matches \textit{dest}.
Optionally, compilation \textit{flags} can be defined via |\def| commands.
This command line makes the \TeX{} engine believe
it is compiling the file \textit{target}
whose content is specified as the latter parameter.
The provided code then forwards the processing to
\textit{main} or \textit{dest} as described in \secref{sec:forward}.

%%%%%%%%%%%%%%%%%%%%%%%%%%%%%%%%%%%%%%%%%%%%%%%%%%%%%%%%%%%%%%%%%%%%%%%%%%%%%%%%
\subsection{Include by Input}
\label{sec:input}

Including child documents by |\include| has some restrictions by design.
Most notably, the content of a child document always occupies
its own set of pages; pages cannot be shared between child documents.
Usually, this behaviour makes perfect sense
because each child document contain an essential part of the document.
However, in some situations it may be desirable to compose
a document from a collection of parts
without having mandatory page breaks between then.
For this case, the package
provides a mechanism to include parts
by |\input| which can also be processed individually.
However, by construction this mechanism
requires manual handling of the content to be output.

%%%%%%%%%%%%%%%%%%%%%%%%%%%%%%%%%%%%%%%%
\DescribeMacro{\ifchilddocmanual}
The main file should be prepared as usual, see \secref{sec:include}.
However, the document body must make a distinction
between processing of an individual part and of the main document, e.g.:
%
\begin{center}
\begin{tabular}{l}
|\ifchilddocmanual|\\
|\input{\childdocname}|\\
|\||else|\\
\textit{document body with }|\input{|\textit{part}|}|\\
|\||fi|
\end{tabular}
\end{center}
%
The conditional |\ifchilddocmanual| is true whenever
a part to be included by |\input| is being compiled,
and the name of the part is stored in |\childdocname|.

%%%%%%%%%%%%%%%%%%%%%%%%%%%%%%%%%%%%%%%%
\DescribeMacro{\childdocby}
Each part to be included by |\input| should start with:
%
\begin{center}
\begin{tabular}{l}
|\input{childdoc.def}|\\
|\childdocby{|\textit{main}|}|\\
\end{tabular}
\end{center}
%
The directive |\childdocby| is similar to |\childdocof|
described in \secref{sec:include},
but the subsequent selection of content must be done manually.
To that end, both |\ifchilddoc| and |\ifchilddocmanual|
will be true upon processing of a part,
and the name of the part is stored in |\childdocname|.
Note that |\jobname| will be set to the filename of the current part
so that each part receives an individual |.aux| file
that does not interfere with the |.aux| file(s) of the main document.
This behaviour can be altered by the alternative form
|\childdocby[*]{|\textit{main}|}| (with a non-empty optional argument)
which uses the |.aux| file of the main document
by setting |\jobname| to \textit{main}.

%%%%%%%%%%%%%%%%%%%%%%%%%%%%%%%%%%%%%%%%%%%%%%%%%%%%%%%%%%%%%%%%%%%%%%%%%%%%%%%%
\subsection{Driver Development}
\label{sec:driver}

The \textsf{childdoc} mechanism can also be use for the development
of definition files such as \LaTeX{} styles or classes.
This case differs from the above setup with multiple parts
included by |\include| in that no |\includeonly| should be invoked.
This can be achieved by starting the include file
(before |\ProvidesPackage|) with:
%
\begin{center}
\begin{tabular}{l}
|\input{childdoc.def}|\\
|\childdocforward{|\textit{main}|}|\\
\end{tabular}
\end{center}
%
or alternatively with:
%
\begin{center}
\begin{tabular}{l}
|\input{childdoc.def}|\\
|\childdocby{|\textit{main}|}|\\
\end{tabular}
\end{center}
%
Both forms have slightly different effects as described above.
The main file is prepared as usual, see \secref{sec:include}.

%%%%%%%%%%%%%%%%%%%%%%%%%%%%%%%%%%%%%%%%%%%%%%%%%%%%%%%%%%%%%%%%%%%%%%%%%%%%%%%%
\subsection{Legacy Detection}
\label{sec:detection}

The directive |\childdocmain| in the main file can detect
whether the complete document or merely a child is to be compiled
even without using the directive |\childdocof|.
This method is deprecated because it is less robust
and there is no compelling reason to use it;
it is merely provided for backward compatibility
and it may be removed in future versions.

If the detection mechanism is to be used,
it is mandatory to correctly specify
the filename of the main file as the argument of |\childdocmain|:
%
\begin{center}
\begin{tabular}{l}
|\input{childdoc.def}|\\
|\childdocmain{|\textit{main}|}|\\
\end{tabular}
\end{center}
%
If |\jobname| does not match the argument \textit{main} of |\childdocmain|,
it is assumed that |\jobname| points to the child file to be compiled.
When using |\childdocmain| with the main file specified as argument,
it suffices to start a child file
with just |\input{|\textit{main}|}|
without loading of the package and using |\childdocof|.
If instead all processing is done
with the appropriate \textsf{childdoc} directives,
the argument of \textit{main} of |\childdocmain| can be empty.

An alternative version of the command line processing described
in \secref{sec:commandline} using the detection mechanism reads:
%
\begin{center}
|... -jobname "|\textit{target}|" "|[\textit{flags}]%
[|\def\jobname{|\textit{dest}|}|]|\input{|\textit{main}|}"|
\end{center}

%%%%%%%%%%%%%%%%%%%%%%%%%%%%%%%%%%%%%%%%%%%%%%%%%%%%%%%%%%%%%%%%%%%%%%%%%%%%%%%%
\subsection{Manual Code}
\label{sec:manual}

In case one cannot be certain whether the definitions file |childdoc.def|
is installed on the target \TeX{} distribution
and one prefers not to ship it,
it is conceivable to paste a few relevant commands into the sources.

To that end, drop all statements |\input{childdoc.def}|
and perform the replacements as outlined below.
Instead of |\childdocmain{|\textit{main}|}| add the following code
to the top of the main file:
%
\begin{center}
\begin{tabular}{l}
|\||ifdefined\childdocname\endinput\||fi\newif\ifchilddoc|\\
|\edef\childdocname{\scantokens\expandafter{\jobname\noexpand}}|\\
|\def\childdocmain{|\textit{main}|}\||ifx\childdocmain\childdocname\||else|\\
|\childdoctrue\includeonly{\childdocname}\let\jobname\childdocmain\||fi|\\
\end{tabular}
\end{center}
%
Instead of |\childdocof{|\textit{main}|}| just include the main file
at the top of each child file:
%
\begin{center}
|\input{|\textit{main}|}|
\end{center}
%
A simple redirection |\childdocforward{|\textit{dest}|}| is achieved by:
%
\begin{center}
|\def\jobname{|\textit{dest}|}\input{\jobname}|
\end{center}
%
The redirection with prefix
|\childdocforwardprefix[|\textit{prefix}|]{|\textit{dest}|}|
is accomplished by:
%
\begin{center}
\begin{tabular}{l}
|{\edef\jobname{\scantokens\expandafter{\jobname\noexpand}}|\\
|\def\redirectjob |\textit{prefix}|#1~~~{\gdef\jobname{|\textit{dest}|#1}}|\\
|\expandafter\redirectjob\jobname~~~}\input{\jobname}|
\end{tabular}
\end{center}

In an alternative approach,
child documents can be compiled by a specific command line
without additional code or specific definitions:
%
\begin{center}
|... -jobname "|\textit{target}|" "|[\textit{flags}]%
|\includeonly{|\textit{dest}|}\input{|\textit{main}|}"|
\end{center}
%

%%%%%%%%%%%%%%%%%%%%%%%%%%%%%%%%%%%%%%%%%%%%%%%%%%%%%%%%%%%%%%%%%%%%%%%%%%%%%%%%
%%%%%%%%%%%%%%%%%%%%%%%%%%%%%%%%%%%%%%%%%%%%%%%%%%%%%%%%%%%%%%%%%%%%%%%%%%%%%%%%
\section{Information}

%%%%%%%%%%%%%%%%%%%%%%%%%%%%%%%%%%%%%%%%%%%%%%%%%%%%%%%%%%%%%%%%%%%%%%%%%%%%%%%%
\subsection{Copyright}

Copyright \copyright{} 2017--2018 Niklas Beisert

This work may be distributed and/or modified under the
conditions of the \LaTeX{} Project Public License, either version 1.3
of this license or (at your option) any later version.
The latest version of this license is in
  \url{http://www.latex-project.org/lppl.txt}
and version 1.3 or later is part of all distributions of \LaTeX{}
version 2005/12/01 or later.

This work has the LPPL maintenance status `maintained'.

The Current Maintainer of this work is Niklas Beisert.

This work consists of the files |README.txt|, |childdoc.ins| and |childdoc.dtx|
as well as the derived files |childdoc.def|, |cdocsamp.tex|
with |cdocsch1.tex|, |cdocsch2.tex|, |cdocspt3.tex|, |cdocspt4.tex|,
|cdocsdrf.tex|, |cdocsfn1.tex|, |cdocsfn2.tex|
as well as |childdoc.pdf|.

%%%%%%%%%%%%%%%%%%%%%%%%%%%%%%%%%%%%%%%%%%%%%%%%%%%%%%%%%%%%%%%%%%%%%%%%%%%%%%%%
\subsection{Files and Installation}

The package consists of the files:
%
\begin{center}
\begin{tabular}{ll}
    |README.txt|   & readme file \\
    |childdoc.ins| & installation file \\
    |childdoc.dtx| & source file \\
    |childdoc.def| & definition file \\
    |cdocsamp.tex| & sample main file \\
    |cdocsch1.tex| & sample include file \\
    |cdocsch2.tex| & sample include file \\
    |cdocspt3.tex| & sample part file \\
    |cdocspt4.tex| & sample part file \\
    |cdocsdrf.tex| & sample redirection file \\
    |cdocsfn1.tex| & sample redirection file \\
    |cdocsfn2.tex| & sample redirection file \\
    |childdoc.pdf| & manual
\end{tabular}
\end{center}
%
The distribution consists of the files
|README.txt|, |childdoc.ins| and |childdoc.dtx|.
%
\begin{itemize}
\item
Run (pdf)\LaTeX{} on |childdoc.dtx|
to compile the manual |childdoc.pdf| (this file).
\item
Run \LaTeX{} on |childdoc.ins| to create the definitions file |childdoc.def|
and the sample |cdocsamp.tex| with include files
|cdocsch1.tex|, |cdocsch2.tex|, |cdocspt3.tex|, |cdocspt4.tex|,
|cdocsdrf.tex|, |cdocsfn1.tex|, |cdocsfn2.tex|.
Then copy the file |childdoc.def| to an appropriate directory of your \LaTeX{}
distribution, e.g.\ \textit{texmf-root}|/tex/latex/childdoc|.
\end{itemize}

%%%%%%%%%%%%%%%%%%%%%%%%%%%%%%%%%%%%%%%%%%%%%%%%%%%%%%%%%%%%%%%%%%%%%%%%%%%%%%%%
\subsection{Related CTAN Packages}

There are several other packages which offer a similar functionality:
%
\begin{itemize}
\item
The packages
\href{http://ctan.org/pkg/docmute}{\textsf{docmute}},
\href{http://ctan.org/pkg/includex}{\textsf{includex}} and
\href{http://ctan.org/pkg/standalone}{\textsf{standalone}}
provide commands to include only the document body of
a child file thus allowing both files to be compiled individually.
\item
The packages \href{http://ctan.org/pkg/subdocs}{\textsf{subdocs}}
and \href{http://ctan.org/pkg/subfiles}{\textsf{subfiles}}
provide structures in which the main and child documents can be
encapsulated and allowing them to be compiled individually.
The inclusion mechanism is different from the conventional |\include|.
\item
The package \href{http://ctan.org/pkg/combine}{\textsf{combine}}
is an elaborate solution to combine several documents into one.
\end{itemize}
%
See also the CTAN topic \href{http://ctan.org/topic/subdocs}{\textsf{subdocs}}
for further related packages.
The present package differs from the above solutions in that
a document structure constructed with the conventional |\include| mechanism
just needs two extra commands at the top of every file
such that all constituent files can be compiled individually.

%%%%%%%%%%%%%%%%%%%%%%%%%%%%%%%%%%%%%%%%%%%%%%%%%%%%%%%%%%%%%%%%%%%%%%%%%%%%%%%%
%\subsection{Feature Suggestions}
%
%The following is a list of features which may be useful for future
%versions of this package:
%%
%\begin{itemize}
%\item
%\ldots
%\end{itemize}

%%%%%%%%%%%%%%%%%%%%%%%%%%%%%%%%%%%%%%%%%%%%%%%%%%%%%%%%%%%%%%%%%%%%%%%%%%%%%%%%
\subsection{Revision History}

%%%%%%%%%%%%%%%%%%%%%%%%%%%%%%%%%%%%%%%%
\paragraph{v2.0:} 2018/12/30

\begin{itemize}
\item
immediate forward processing
\item
added |\childdocby| mechanism
\item
manual restructured
\end{itemize}

%%%%%%%%%%%%%%%%%%%%%%%%%%%%%%%%%%%%%%%%
\paragraph{v1.6:} 2018/01/17

\begin{itemize}
\item
application for development of include files
\item
corrections to manual
\end{itemize}

%%%%%%%%%%%%%%%%%%%%%%%%%%%%%%%%%%%%%%%%
\paragraph{v1.5:} 2017/05/21

\begin{itemize}
\item
more complete structuring introduced
\item
|\childdocof| introduced
\item
|\childdoc| renamed to |\childdocmain|
\item
|\childredirect| renamed to |\childdocforward| and |\childdocforwardprefix|
and functionality expanded
\end{itemize}

%%%%%%%%%%%%%%%%%%%%%%%%%%%%%%%%%%%%%%%%
\paragraph{v1.0:} 2017/04/27

\begin{itemize}
\item
manual and install package
\item
first version published on CTAN
\end{itemize}

%%%%%%%%%%%%%%%%%%%%%%%%%%%%%%%%%%%%%%%%
\paragraph{v0.6:} 2017/04/26

\begin{itemize}
\item
redirection mechanism added
\end{itemize}

%%%%%%%%%%%%%%%%%%%%%%%%%%%%%%%%%%%%%%%%
\paragraph{v0.5:} 2017/04/26

\begin{itemize}
\item
functionality in definition file
\end{itemize}


%%%%%%%%%%%%%%%%%%%%%%%%%%%%%%%%%%%%%%%%%%%%%%%%%%%%%%%%%%%%%%%%%%%%%%%%%%%%%%%%
%%%%%%%%%%%%%%%%%%%%%%%%%%%%%%%%%%%%%%%%%%%%%%%%%%%%%%%%%%%%%%%%%%%%%%%%%%%%%%%%
%%%%%%%%%%%%%%%%%%%%%%%%%%%%%%%%%%%%%%%%%%%%%%%%%%%%%%%%%%%%%%%%%%%%%%%%%%%%%%%%
\appendix

\settowidth\MacroIndent{\rmfamily\scriptsize 000\ }

 \DocInput{childdoc.dtx}

\end{document}
%</driver>
% \fi
%
% %%%%%%%%%%%%%%%%%%%%%%%%%%%%%%%%%%%%%%%%%%%%%%%%%%%%%%%%%%%%%%%%%%%%%%%%%%%%%%
% %%%%%%%%%%%%%%%%%%%%%%%%%%%%%%%%%%%%%%%%%%%%%%%%%%%%%%%%%%%%%%%%%%%%%%%%%%%%%%
% \section{Sample}
%\iffalse
%<*samplemain>
%\fi
%
% The following presents a sample document
% with two chapters, two parts, a title page,
% a compile flag as well as three forwarding files to set the flag.
% It consists of eight |.tex| files:
% \begin{center}
% \begin{tabular}{ll}
% |cdocsamp.tex|&main file\\
% |cdocsch1.tex|&include file for chapter 1\\
% |cdocsch2.tex|&include file for chapter 2\\
% |cdocspt3.tex|&include file for part 3\\
% |cdocspt4.tex|&include file for part 4\\
% |cdocsdrf.tex|&forwarding file for main file in draft mode\\
% |cdocsfi1.tex|&forwarding file for final version of chapter 1\\
% |cdocsfi2.tex|&forwarding file for final version of chapter 2\\
% \end{tabular}
% \end{center}
% Each of the eight files can be compiled directly by the \LaTeX{} compiler.
%
% %%%%%%%%%%%%%%%%%%%%%%%%%%%%%%%%%%%%%%
% \paragraph{Main File.}
%
% The main file is called |cdocsamp.tex|.
%
% Load the \textsf{childdoc} definitions and
% declare the filename for the main document:
%    \begin{macrocode}
\input{childdoc.def}
\childdocmain{}
%    \end{macrocode}

% Optional override for |\version| flag:
%    \begin{macrocode}
%%\ifchilddoc\else\providecommand{\version}{draft}\fi
%    \end{macrocode}

% Define the default values for the |\version| flag
% (|final| for the main file and |draft| for childs):
%    \begin{macrocode}
\ifchilddoc
\providecommand{\version}{draft}
\else
\providecommand{\version}{final}
\fi
%    \end{macrocode}

% Load the standard document class:
%    \begin{macrocode}
\documentclass[12pt]{article}
%    \end{macrocode}

% Start the document body:
%    \begin{macrocode}
\begin{document}
%    \end{macrocode}

% Declare a title page.
% Print title, part of document being processed and version flag:
%    \begin{macrocode}
\addtocounter{page}{-1}
\begin{center}
{\LARGE\bfseries{}childdoc example\par}
\vspace{1cm}
\ifchilddoc
\ifchilddocmanual part\else chapter\fi:
`\childdocname' of `\childdocjob'\par
\else
main document: `\childdocjob'\par
\fi
version: \version\par
\end{center}
\newpage
%    \end{macrocode}

% Manually include selected file,
% otherwise process as usual:
%    \begin{macrocode}
\ifchilddocmanual
\section*{part `\childdocname'}
\input{\childdocname}
\else
%    \end{macrocode}

% Include the two chapters:
%    \begin{macrocode}
\include{cdocsch1}
\include{cdocsch2}
%    \end{macrocode}

% Include the two parts unless only chapters should be displayed:
%    \begin{macrocode}
\ifchilddoc\else
\section{part three}
\input{cdocspt3}
\section{part four}
\input{cdocspt4}
\fi
%    \end{macrocode}

% Process as usual until here:
%    \begin{macrocode}
\fi
%    \end{macrocode}

% End of document body:
%    \begin{macrocode}
\end{document}
%    \end{macrocode}
%\iffalse
%</samplemain>
%\fi
%
% %%%%%%%%%%%%%%%%%%%%%%%%%%%%%%%%%%%%%%
% \paragraph{Chapter Include Files.}
%
% The include files are called |cdocsch1.tex| and |cdocsch2.tex|.
%
%\iffalse
%<*samplechap1|samplechap2>
%\fi

% Optional override for |\version| flag:
%    \begin{macrocode}
%%\providecommand{\version}{final}
%    \end{macrocode}

% Include the main document:
%    \begin{macrocode}
\input{childdoc.def}
\childdocof{cdocsamp}
%    \end{macrocode}

%\iffalse
%</samplechap1|samplechap2>
%\fi
%
%\iffalse
%<*samplechap1>
%\fi
% Some text for chapter 1:
%    \begin{macrocode}
\section{one}
some text in chapter one
%    \end{macrocode}

%\iffalse
%</samplechap1>
%\fi
% Some text for chapter 2:
%\iffalse
%<*samplechap2>
%\fi
%    \begin{macrocode}
\section{two}
more text in chapter two
%    \end{macrocode}

%\iffalse
%</samplechap2>
%\fi
%
% %%%%%%%%%%%%%%%%%%%%%%%%%%%%%%%%%%%%%%
% \paragraph{Part Include Files.}
%
% The include files are called |cdocspt3.tex| and |cdocspt4.tex|.
%
%\iffalse
%<*samplepart3|samplepart4>
%\fi

% Optional override for |\version| flag:
%    \begin{macrocode}
%%\providecommand{\version}{final}
%    \end{macrocode}

% Include the main document:
%    \begin{macrocode}
\input{childdoc.def}
\childdocby{cdocsamp}
%    \end{macrocode}

%\iffalse
%</samplepart3|samplepart4>
%\fi
%
%\iffalse
%<*samplepart3>
%\fi
% Some text for part 3:
%    \begin{macrocode}
some text in part three
%    \end{macrocode}

%\iffalse
%</samplepart3>
%\fi
% Some text for part 4:
%\iffalse
%<*samplepart4>
%\fi
%    \begin{macrocode}
more text in part four
%    \end{macrocode}

%\iffalse
%</samplepart4>
%\fi
%
% %%%%%%%%%%%%%%%%%%%%%%%%%%%%%%%%%%%%%%
% \paragraph{Forwarding for a Complete Draft.}
%
% The following forwarding file |cdocsdrf.tex|
% compiles the main document in draft mode:
%\iffalse
%<*sampledraft>
%\fi
%    \begin{macrocode}
\def\version{draft}
\input{childdoc.def}
\childdocforward{cdocsamp}
%    \end{macrocode}

%\iffalse
%</sampledraft>
%\fi
%
% %%%%%%%%%%%%%%%%%%%%%%%%%%%%%%%%%%%%%%
% \paragraph{Forwarding for Final Version of the Chapters.}
%
% The following forwarding files |cdocsfn1.tex| and |cdocsfn2.tex|
% (with identical content)
% compile the final versions of the child documents
% |cdocsch1.tex| and |cdocsch2.tex|, respectively:
%\iffalse
%<*samplefinal>
%\fi
%    \begin{macrocode}
\def\version{final}
\input{childdoc.def}
\childdocforwardprefix[cdocsamp]{cdocsfn}{cdocsch}
%    \end{macrocode}

%\iffalse
%</samplefinal>
%\fi
%
% %%%%%%%%%%%%%%%%%%%%%%%%%%%%%%%%%%%%%%
% \paragraph{Command Line Processing.}
%
% The following three command lines generate the output files
% |cdocscld|, |cdocscl1| and |cdocscl2|
% which should be identical to
% |cdocsdrf|, |cdocsch1| and |cdocsfn2|, respectively:
% \begin{center}
% \begin{tabular}{l}
% |latex -jobname cdocscld \|\\
% |  "\def\version{draft}\input{childdoc.def}\childdocforward{cdocsamp}"|\\
% |latex -jobname cdocscl1 \|\\
% |  "\input{childdoc.def}\childdocforward[cdocsamp]{cdocsch1}"|\\
% |latex -jobname cdocscl2 \|\\
% |  "\def\version{final}\input{childdoc.def}\childdocforward{cdocsch2}"|
% \end{tabular}
% \end{center}
% Note that the trailing backslash on each first line
% merely continues the input to the second line
% (for convenient cut ant paste).
% Furthermore, the command |latex| can be replaced by any
% of its alternative versions such as |pdflatex|.
%
% %%%%%%%%%%%%%%%%%%%%%%%%%%%%%%%%%%%%%%%%%%%%%%%%%%%%%%%%%%%%%%%%%%%%%%%%%%%%%%
% %%%%%%%%%%%%%%%%%%%%%%%%%%%%%%%%%%%%%%%%%%%%%%%%%%%%%%%%%%%%%%%%%%%%%%%%%%%%%%
% \section{Implementation}
%\iffalse
%<*package>
%\fi
%
% This section describes the definitions file |childdoc.def|.

% The definitions cannot be loaded using |\usepackage| or |\RequirePackage|
% which has a mechanism to prevent loading a style file more than once.
% When loading the definitions by means of |\input|
% multiple instances have to be prevented manually:
%\iffalse
%This code needs to be before the `\ProvidesFile' directive
%which is defined at the beginning of this file.
%Therefore it is also placed there and commented out here.
%</package>
%<*discard>
%\fi
%    \begin{macrocode}
\ifdefined\childdocmain\endinput\fi
%    \end{macrocode}
%\iffalse
%</discard>
%<*package>
%\fi
%
% \macro{\ifchilddoc}
% \macro{\ifchilddocmanual}
% The conditional |\ifchilddoc| tells whether a
% child (true) or main (false) document is being compiled.
% The conditional |\ifchilddocmanual| tells whether
% the |\includeonly| mechanism is used (false) or
% the selection of child files must be performed manually (true).
% The definitions initialise to false:
%    \begin{macrocode}
\newif\ifchilddoc
\newif\ifchilddocmanual
%    \end{macrocode}

% \macro{\childdocname}
% \macro{\childdocjob}
% The macro |\childdocname| stores the name of the main document
% to be compiled. The macro |\childdocjob| stores the name of
% the document on which the \LaTeX{} compiler was originally invoked.
% The content of |\jobname| cannot be compared
% to filenames specified in the source due to different catcodes.
% The following code rescans |\jobname|, stores the result
% in |\childdocname| and saves a copy in |\childdocjob|:
%    \begin{macrocode}
\edef\childdocname{\scantokens\expandafter{\jobname\noexpand}}
\let\childdocjob\childdocname
%    \end{macrocode}

% \macro{\childdocdisable}
% The macro |\childdocdisable| prevents the main file
% from being processed more than once.
% At this stage, the main document command |\childdocmain|
% is assumed to be called once again where it should do nothing.
% Any subsequent call to it should prevent
% a secondary processing of the main document
% It overwrites the forwarding commands
% |\childdocof| and |\childdocforward|
% with empty macros to prevent further inclusions of the main document:
%    \begin{macrocode}
\newcommand{\childdocdisable}
{
  \renewcommand{\childdocmain}[1]{\renewcommand{\childdocmain}[1]{\endinput}}
  \renewcommand{\childdocof}[1]{}
  \renewcommand{\childdocby}[2][]{}
  \renewcommand{\childdocforward}[2][]{}
  \renewcommand{\childdocdisable}{}
}
%    \end{macrocode}

% \macro{\childdocmain}
% The macro |\childdocmain| is to be called at the top of the main file
% with nothing or the main filename (without extension) as argument.
% First, it breaks loops.
% If the argument is not empty and does not match |\childdocname|
% (which is set by the first inclusion of |childdoc.def|),
% |\ifchilddoc| is set to true, |\includeonly| is applied to the child file
% and |\jobname| is set to the main file
% (for proper handling of |.aux| files):
%    \begin{macrocode}
\newcommand{\childdocmain}[1]
{
  \childdocdisable\childdocmain{}
  \if?#1?\else
    \begingroup
      \def\childdoctmp{#1}
      \ifx\childdoctmp\childdocname
        \def\childdoctmp{}
      \else
        \def\childdoctmp
        {
          \childdoctrue
          \includeonly{\childdocname}
          \def\childdocjob{#1}
          \def\jobname{#1}
        }
      \fi
      \expandafter
    \endgroup
    \childdoctmp
  \fi
}
%    \end{macrocode}

% \macro{\childdocof}
% The command |\childdocof| redirects
% compilation to the main file |#1|.
%    \begin{macrocode}
\newcommand{\childdocof}[1]
{
  \childdocdisable
  \childdoctrue
  \includeonly{\childdocname}
  \def\jobname{#1}
  \def\childdocjob{#1}
  \input{#1}
}
%    \end{macrocode}

% \macro{\childdocby}
% The command |\childdocby| ....
%    \begin{macrocode}
\newcommand{\childdocby}[2][]
{
  \childdocdisable
  \childdoctrue
  \childdocmanualtrue
  \if?#1?\else
    \def\jobname{#2}
  \fi
  \def\childdocjob{#2}
  \input{#2}
  \endinput
}
%    \end{macrocode}

% \macro{\childdocforward}
% The command |\childdocforward| redirects
% compilation to the main file or
% (if the optional argument is given) a child file.
% Parameters are set as if the main file
% or a child file starting with |\childdocof| was compiled.
% Then compilation is handed over to the main file:
%    \begin{macrocode}
\newcommand{\childdocforward}[2][]
{
  \begingroup
    \if?#1?
      \def\childdoctmp
      {
        \def\childdocname{#2}
        \def\childdocjob{#2}
        \def\jobname{#2}
        \input{#2}
        \endinput
      }
    \else
      \def\childdoctmp
      {
        \childdocdisable
        \def\childdocname{#2}
        \childdoctrue
        \includeonly{#2}
        \def\childdocjob{#1}
        \def\jobname{#1}
        \input{#1}
        \endinput
      }
    \fi
    \expandafter
  \endgroup
  \childdoctmp
}
%    \end{macrocode}

% \macro{\childdocforwardprefix}
% The command |\childdocforwardprefix| redirects
% compilation to the main or a child file by means of a pattern.
% The prefix |#1| in the current filename is replaced by |#2|
% and the suffix of the current filename is kept
% (it is assumed that the filename does not contain the substring `|~~~|'
% which is used as a delimiter).
% Compilation is handed over to the new file by |\childdocforward|:
%    \begin{macrocode}
\newcommand{\childdocforwardprefix}[3][]
{
  \begingroup
    \def\childdocextract #2##1~~~{\def\childdoctmp{\childdocforward[#1]{#3##1}}}
    \expandafter\childdocextract\childdocname~~~
    \expandafter
  \endgroup
  \childdoctmp
}
%    \end{macrocode}

% \macro{\childdoc}
% The deprecated macro |\childdoc| is a legacy version of |\childdocmain|:
%    \begin{macrocode}
\newcommand{\childdoc}{\childdocmain}
%    \end{macrocode}

% \macro{\childdocredirect}
% The deprecated macro |\childdocredirect| is a legacy version
% of |\childdocforward| and |\childdocforwardprefix|:
%    \begin{macrocode}
\newcommand{\childdocredirect}[2][]
{
  \begingroup
    \if?#1?
      \def\childdoctmp{\childdocforward{#2}}
    \else
      \def\childdoctmp{\childdocforwardprefix{#1}{#2}}
    \fi
    \expandafter
  \endgroup
  \childdoctmp
}
%    \end{macrocode}

%\iffalse
%</package>
%\fi
%
\endinput
|\\
|\childdocforward{|\textit{main}|}|
\end{tabular}
\end{center}
%
Likewise, the following files |final|\textit{nn}|.tex|
compile the final version of the child document
|child|\textit{nn}|.tex|:
%
\begin{center}
\begin{tabular}{l}
|\def\version{final}|\\
|% \iffalse
%
% childdoc.dtx Copyright (C) 2017-2018 Niklas Beisert
%
% This work may be distributed and/or modified under the
% conditions of the LaTeX Project Public License, either version 1.3
% of this license or (at your option) any later version.
% The latest version of this license is in
%   http://www.latex-project.org/lppl.txt
% and version 1.3 or later is part of all distributions of LaTeX
% version 2005/12/01 or later.
%
% This work has the LPPL maintenance status `maintained'.
%
% The Current Maintainer of this work is Niklas Beisert.
%
% This work consists of the files childdoc.dtx and childdoc.ins
% and the derived files childdoc.def and cdocsamp.tex with
% cdocsch1.tex, cdocsch2.tex, cdocsdrf.tex, cdocsfn1.tex, cdocsfn2.tex.
%
%<package>\ifdefined\childdocmain\endinput\fi
%<package>\ProvidesFile{childdoc.def}[2018/12/30 v2.0 child document driver]
%<samplemain>\ProvidesFile{cdocsamp.tex}[2018/12/30 v2.0 sample for childdoc]
%<*driver>
%\ProvidesFile{childdoc.drv}[2018/12/30 v2.0 childdoc reference manual file]
\PassOptionsToClass{10pt,a4paper}{article}
\documentclass{ltxdoc}

\usepackage[margin=35mm]{geometry}
\usepackage{hyperref}
\usepackage{hyperxmp}
\usepackage[usenames]{color}

\hypersetup{colorlinks=true}
\hypersetup{pdfstartview=FitH}
\hypersetup{pdfpagemode=UseNone}
\hypersetup{pdfsource={}}
\hypersetup{pdflang={en-UK}}
\hypersetup{pdfcopyright={Copyright 2017-2018 Niklas Beisert.
  This work may be distributed and/or modified under the
  conditions of the LaTeX Project Public License, either version 1.3
  of this license or (at your option) any later version.}}
\hypersetup{pdflicenseurl={http://www.latex-project.org/lppl.txt}}
\hypersetup{pdfcontactaddress={ETH Zurich, ITP, HIT K,
  Wolfgang-Pauli-Strasse 27}}
\hypersetup{pdfcontactpostcode={8093}}
\hypersetup{pdfcontactcity={Zurich}}
\hypersetup{pdfcontactcountry={Switzerland}}
\hypersetup{pdfcontactemail={nbeisert@itp.phys.ethz.ch}}
\hypersetup{pdfcontacturl={http://people.phys.ethz.ch/\xmptilde nbeisert/}}

\newcommand{\secref}[1]{\hyperref[#1]{section \ref*{#1}}}

\parskip1ex
\parindent0pt
\let\olditemize\itemize
\def\itemize{\olditemize\parskip0pt}

\begin{document}

\title{The \textsf{childdoc} Package}
\hypersetup{pdftitle={The childdoc Package}}
\author{Niklas Beisert\\[2ex]
  Institut f\"ur Theoretische Physik\\
  Eidgen\"ossische Technische Hochschule Z\"urich\\
  Wolfgang-Pauli-Strasse 27, 8093 Z\"urich, Switzerland\\[1ex]
  \href{mailto:nbeisert@itp.phys.ethz.ch}
  {\texttt{nbeisert@itp.phys.ethz.ch}}}
\hypersetup{pdfauthor={Niklas Beisert}}
\hypersetup{pdfsubject={Manual for the LaTeX2e Package childdoc}}
\date{30 December 2018, \textsf{v2.0}}
\maketitle

\begin{abstract}\noindent
\textsf{childdoc} is a \LaTeXe{} package
that enables the direct compilation
of document sections included by |\include|
to individual files.
\end{abstract}

\begingroup
\parskip0ex
\tableofcontents
\endgroup

%%%%%%%%%%%%%%%%%%%%%%%%%%%%%%%%%%%%%%%%%%%%%%%%%%%%%%%%%%%%%%%%%%%%%%%%%%%%%%%%
%%%%%%%%%%%%%%%%%%%%%%%%%%%%%%%%%%%%%%%%%%%%%%%%%%%%%%%%%%%%%%%%%%%%%%%%%%%%%%%%
\section{Introduction}

\LaTeX{} provides a mechanism to structure a large document (such as a book)
into a main file and several child files (containing the chapters)
using the |\include| command.
This mechanism is beneficial for documents
which span hundreds of pages in order to
make the source file(s) more manageable.
Moreover, compilation can be restricted to
selected child files by means of the |\includeonly| command.
The latter feature can be used to reduce the compilation time while editing
(this was significantly more useful in the earlier days of \LaTeX{})
or to generate a smaller document which is easier to navigate.
Another application of |\includeonly| is to generate
documents consisting of selected parts of the complete document.

However, there are a few drawbacks of the plain |\include| mechanism:
\begin{itemize}
\item
The child files cannot be compiled on their own,
they can only be compiled via the main file.
A naive editing environment
(such as a text editor with an option
to have the current file processed by \LaTeX)
may require one to switch to the main file before compiling;
attempting to compile the child file produces errors.
\item
The main file must be modified (each time)
to adjust the |\includeonly| command
to the present needs. This easily leaves the main file in a messy state.
\item
The generated document will always carry the filename
of the main document. This is inconvenient if
several child files are to be compiled and
to be kept for distribution.
\end{itemize}

The present package provides a simple interface
to make child files individually compilable by \LaTeX{}.
Compiling a child file then has the same effect as compiling
the main file with an |\includeonly| command
to select the appropriate child.
Moreover the generated document will carry the name of the child
rather than the main file.
This resolves all three above issues.

This feature is meant to make the editing of books,
thesis documents and lecture notes somewhat more convenient.
However, the package can also be used efficiently for
composing a series of documents (such as exercise sheets)
which are typically distributed individually.
It then assists the author in generating the individual documents
(potentially in different versions)
as well as a document containing the collected series.
Another application is in developing style files
or other kinds of included material
where compilation of the style file could redirect
to a sample or test file.

%%%%%%%%%%%%%%%%%%%%%%%%%%%%%%%%%%%%%%%%%%%%%%%%%%%%%%%%%%%%%%%%%%%%%%%%%%%%%%%%
%%%%%%%%%%%%%%%%%%%%%%%%%%%%%%%%%%%%%%%%%%%%%%%%%%%%%%%%%%%%%%%%%%%%%%%%%%%%%%%%
\section{Usage}

First of all, the package \textsf{childdoc} is \emph{not} a standard
\LaTeXe{} |.sty| style file! Therefore it needs to be invoked in
a non-standard way.

%%%%%%%%%%%%%%%%%%%%%%%%%%%%%%%%%%%%%%%%%%%%%%%%%%%%%%%%%%%%%%%%%%%%%%%%%%%%%%%%
\subsection{Included Files}
\label{sec:include}

%%%%%%%%%%%%%%%%%%%%%%%%%%%%%%%%%%%%%%%%
\DescribeMacro{\childdocmain}
To use the package, add the commands
\begin{center}
\begin{tabular}{l}
|\input{childdoc.def}|\\
|\childdocmain{}|\\
\end{tabular}
\end{center}
at the very top of the main \LaTeX{} file,
in particular \emph{before} the |\documentclass| statement!
The argument of |\childdocmain| should be left empty
(but it must be present).

%%%%%%%%%%%%%%%%%%%%%%%%%%%%%%%%%%%%%%%%
\DescribeMacro{\childdocof}
Furthermore, add the commands
\begin{center}
\begin{tabular}{l}
|\input{childdoc.def}|\\
|\childdocof{|\textit{main}|}|\\
\end{tabular}
\end{center}
at the top of every child file \textit{child}
which is included by |\include{|\textit{child}|}|
from within the main file
(or at least for those files to be compiled individually).
The argument \textit{main} must be the filename of the main file.

There are a couple of
considerations in setting up the main and child documents:

%%%%%%%%%%%%%%%%%%%%%%%%%%%%%%%%%%%%%%%%
\paragraph{Restrictions.}

Please note the following restrictions:
\begin{itemize}
\item
|\childdocmain| must be called with one argument \textit{main}
to ensure compatibility with earlier version of the package.
It must either be empty (|\childdocmain{}|)
or precisely match the filename of the main file in which it is specified.
See \secref{sec:detection} for further information.
\item
The filename \textit{main} must be specified without the |.tex| extension.
\item
The filename \textit{main} is case sensitive
(even in case-insensitive file systems)
due to internal string comparison.
\item
The argument \textit{main} should be fully expanded, it cannot be a macro.
\item
Subdirectories and special characters should be avoided in filenames.
\item
The command |\childdocmain{|\textit{main}|}| must be followed by a whitespace.
It should not be followed immediately by another command
or by a comment mark `|%|'.
This is because the \TeX{} parser reads the token immediately following
the argument of |\childdocmain| and puts it
at the beginning of every child section;
however, a white\-space is ignored.
\end{itemize}

%%%%%%%%%%%%%%%%%%%%%%%%%%%%%%%%%%%%%%%%
\paragraph{Content of Main File.}

It is advisable to place all content in the child files included by |\include|.
Any output contained in the main file will appear in all child documents
unless suppressed manually;
it cannot be suppressed automatically by the |\includeonly| directive
and thus should normally be avoided.
A method to include some content in the main file
by means of conditional processing is described in \secref{sec:conditional}.

%%%%%%%%%%%%%%%%%%%%%%%%%%%%%%%%%%%%%%%%
\paragraph{Page Numbering.}

When only a part of the document is compiled,
the appropriate numbering of pages
(as well as other status parameters)
is determined from the |.aux| files.
The latter contain information from previous passes.
However this information needs to propagate through
all intermediate child documents.
Therefore the page numbering in child documents may well
be inconsistent until the complete document is compiled at least once.

A useful (if unconventional) way to always ensure a consistent
page numbering is to restart the numbering in each child document
and denote the pages by `\textit{child}|.|\textit{page}'
where \textit{child} represents the chapter/section number of the child file.
This can be achieved by the command
|\numberwithin{page}{|\textit{child}|}|
of the \textsf{amsmath} package
where \textit{child} can be |chapter| or |section|
depending on the chosen structuring.
Alternatively, one can modify the macro |\thepage| appropriately
and reset the counter |page| at the start of each child file.

%%%%%%%%%%%%%%%%%%%%%%%%%%%%%%%%%%%%%%%%%%%%%%%%%%%%%%%%%%%%%%%%%%%%%%%%%%%%%%%%
\subsection{Conditional Processing}
\label{sec:conditional}

The package provides a mechanism to compile different versions
of a document. To customise the versions further some conditional processing
can come in handy to distinguish which version is being compiled.
The package provides two macros to describe the compilation context:

%%%%%%%%%%%%%%%%%%%%%%%%%%%%%%%%%%%%%%%%
\DescribeMacro{\ifchilddoc}
The conditional |\ifchilddoc| distinguishes between the compilation of
child documents and the main document:
%
\begin{center}
|\ifchilddoc |\textit{child-code}| |[|\||else |\textit{main-code}]| \||fi|
\end{center}

%%%%%%%%%%%%%%%%%%%%%%%%%%%%%%%%%%%%%%%%
\DescribeMacro{\childdocname}
\DescribeMacro{\childdocjob}
The macro |\childdocname| contains the filename (without extension)
of the main or child file being processed.
Note that |\childdocjob| will always contain the name of the main file.

%%%%%%%%%%%%%%%%%%%%%%%%%%%%%%%%%%%%%%%%
\paragraph{Title Page.}

Conditional processing can be used to include a title or banner page
in the main document when proper precautions are taken.
Importantly, the code in the main file should ensure that the page counter
(as well as other status parameters which are stored in the |.aux| files)
takes the same value after the conditional processing.
Otherwise the page numbers may take divergent values
depending on which part is compiled.

For example, a title page could be declared by:
%
\begin{center}
\begin{tabular}{l}
|\ifchilddoc\||else|\\
|\addtocounter{page}{-1}|\\
\textit{code for title page}\\
|\newpage|\\
|\||fi|
\end{tabular}
\end{center}
%
A banner page for the child documents can be generated by:
%
\begin{center}
\begin{tabular}{l}
|\ifchilddoc|\\
|\addtocounter{page}{-1}|\\
\textit{code for banner page}\\
|\newpage|\\
|\||fi|
\end{tabular}
\end{center}
%
Here one could write a message such as:
\begin{center}
|This is the part \childdocname{} of \childdocjob{}.|
\end{center}

%%%%%%%%%%%%%%%%%%%%%%%%%%%%%%%%%%%%%%%%%%%%%%%%%%%%%%%%%%%%%%%%%%%%%%%%%%%%%%%%
\subsection{Flags}
\label{sec:flags}

The package makes it easy to generate different versions
of the main or child documents.
To this end compilation flags can be defined
and assigned different default values.
They will be particularly useful in conjunction
with the forwarding mechanism described in \secref{sec:forward}.

For example, it may be useful to have a flag |\version|
which can be set to |draft| or |final|.
The document source will contain some conditional code
depending on the value of |\version|.
Suppose further, the flag should default to |final| for the main file
and to |draft| for child files
which is a natural assignment for editing the document.
This is achieved by placing the following code
in the preamble of the main document
(below the |\childdocmain| directive):
%
\begin{center}
\begin{tabular}{l}
|\ifchilddoc|\\
|\providecommand{\version}{draft}|\\
|\||else|\\
|\providecommand{\version}{final}|\\
|\||fi|
\end{tabular}
\end{center}
%
The definition by |\providecommand| makes sure
that previous definitions are not overwritten.
Further statements |\providecommand{\version}{...}|
can thus be added before the above code to override it.

For the main file, one might add a line
(between |\childdocmain| and the above block)
%
\begin{center}
|%\ifchilddoc\||else\providecommand{\version}{draft}\||fi|
\end{center}
%
which can be uncommented to produce a draft version.
Likewise one can add a line to the very top of a child file
(above the |\childdocof{|\textit{main}|}| directive)
%
\begin{center}
|%\providecommand{\version}{final}|
\end{center}
%
which can be uncommented to produce the final version of this child document.

%%%%%%%%%%%%%%%%%%%%%%%%%%%%%%%%%%%%%%%%%%%%%%%%%%%%%%%%%%%%%%%%%%%%%%%%%%%%%%%%
\subsection{Forwarding}
\label{sec:forward}

Different versions of the main or child documents
using compilation flags as described in \secref{sec:flags}
can be (permanently) stored in different files
for convenient compilation, viewing and distribution.
To this end, the package defines a command
to pass on compilation to a different file:

%%%%%%%%%%%%%%%%%%%%%%%%%%%%%%%%%%%%%%%%
\DescribeMacro{\childdocforward}
The command |\childdocforward| redirects processing to
another source file:
%
\begin{center}
\begin{tabular}{l}
|\input{childdoc.def}|\\
|\childdocforward[|\textit{main}|]{|\textit{dest}|}|\\
\end{tabular}
\end{center}
%
The argument \textit{dest} is the destination file
(without extension).
It should be the main file or one of the child files.
Note that further \textsf{childdoc} directives
such as |\childdocof| and |\childdocforward|
in the indicated file will be processed in this form.
The optional argument \textit{main}
passes on directly to the main file \textit{main}
while pretending to compile the child \textit{dest}.
This form behaves as if \textit{dest}
issues |\childdocof{|\textit{main}|}| right away,
and no further \textsf{childdoc} directives will be processed.

%%%%%%%%%%%%%%%%%%%%%%%%%%%%%%%%%%%%%%%%
\DescribeMacro{\...prefix}
In the alternative form |\childdocforwardprefix|,
%
\begin{center}
\begin{tabular}{l}
|\input{childdoc.def}|\\
|\childdocforwardprefix[|\textit{main}|]{|\textit{prefix}|}{|\textit{dest}|}|
\end{tabular}
\end{center}
%
the destination file is determined by a pattern
depending on the current file:
To make this work, the current file must be called
`{\textit{prefix}\hspace{0.2em}\textit{suffix}}'
with \textit{prefix} matching precisely the argument.
Processing is then passed on to the file
`{\textit{dest}\hspace{0.2em}\textit{suffix}}'.
Surely, the same effect is achieved by
directly specifying the
argument `{\textit{dest}\hspace{0.2em}\textit{suffix}}'
in the first form.
However, that requires to set up a different file
for each child. With the alternative form of the command
all these files can have exactly the same content
which simplifies setting them up and maintaining them.

For example, the following file |draft.tex|
with a compilation flag |\version| as described in \secref{sec:flags}
compiles the main document as a draft:
%
\begin{center}
\begin{tabular}{l}
|\def\version{draft}|\\
|\input{childdoc.def}|\\
|\childdocforward{|\textit{main}|}|
\end{tabular}
\end{center}
%
Likewise, the following files |final|\textit{nn}|.tex|
compile the final version of the child document
|child|\textit{nn}|.tex|:
%
\begin{center}
\begin{tabular}{l}
|\def\version{final}|\\
|\input{childdoc.def}|\\
|\childdocforwardprefix{final}{child}|
\end{tabular}
\end{center}
%

Note that when several versions of a main file and/or of each child file
are to be generated, it may be convenient to set up a |Makefile| or
shell script to automatise the process.

%%%%%%%%%%%%%%%%%%%%%%%%%%%%%%%%%%%%%%%%%%%%%%%%%%%%%%%%%%%%%%%%%%%%%%%%%%%%%%%%
\subsection{Command Line Processing}
\label{sec:commandline}

The effect of redirection files can also be achieved by invoking
the \LaTeX{} compiler with a more elaborate command line.
Most conveniently this should be done as part
of a shell script or a |Makefile|.

When using \textsf{childdoc} in the main file, the following
command lines effectively perform a redirection
(note that depending on the shell being used,
backslashes may have to be doubled: `|\|' $\to$ `|\\|'):
%
\begin{center}
|... -jobname "|\textit{target}|" |\\|"|[\textit{flags}]%
|\input{childdoc.def}\childdocforward[|\textit{main}|]{|\textit{dest}|}"|
\end{center}
%
Here \textit{target} is the name of the output file,
\textit{main} is the name of the main file
and \textit{dest} is the name of the main or child file to be processed
(all filenames without extensions).
The optional argument \textit{main} can be omitted
if \textit{main} matches \textit{dest}.
Optionally, compilation \textit{flags} can be defined via |\def| commands.
This command line makes the \TeX{} engine believe
it is compiling the file \textit{target}
whose content is specified as the latter parameter.
The provided code then forwards the processing to
\textit{main} or \textit{dest} as described in \secref{sec:forward}.

%%%%%%%%%%%%%%%%%%%%%%%%%%%%%%%%%%%%%%%%%%%%%%%%%%%%%%%%%%%%%%%%%%%%%%%%%%%%%%%%
\subsection{Include by Input}
\label{sec:input}

Including child documents by |\include| has some restrictions by design.
Most notably, the content of a child document always occupies
its own set of pages; pages cannot be shared between child documents.
Usually, this behaviour makes perfect sense
because each child document contain an essential part of the document.
However, in some situations it may be desirable to compose
a document from a collection of parts
without having mandatory page breaks between then.
For this case, the package
provides a mechanism to include parts
by |\input| which can also be processed individually.
However, by construction this mechanism
requires manual handling of the content to be output.

%%%%%%%%%%%%%%%%%%%%%%%%%%%%%%%%%%%%%%%%
\DescribeMacro{\ifchilddocmanual}
The main file should be prepared as usual, see \secref{sec:include}.
However, the document body must make a distinction
between processing of an individual part and of the main document, e.g.:
%
\begin{center}
\begin{tabular}{l}
|\ifchilddocmanual|\\
|\input{\childdocname}|\\
|\||else|\\
\textit{document body with }|\input{|\textit{part}|}|\\
|\||fi|
\end{tabular}
\end{center}
%
The conditional |\ifchilddocmanual| is true whenever
a part to be included by |\input| is being compiled,
and the name of the part is stored in |\childdocname|.

%%%%%%%%%%%%%%%%%%%%%%%%%%%%%%%%%%%%%%%%
\DescribeMacro{\childdocby}
Each part to be included by |\input| should start with:
%
\begin{center}
\begin{tabular}{l}
|\input{childdoc.def}|\\
|\childdocby{|\textit{main}|}|\\
\end{tabular}
\end{center}
%
The directive |\childdocby| is similar to |\childdocof|
described in \secref{sec:include},
but the subsequent selection of content must be done manually.
To that end, both |\ifchilddoc| and |\ifchilddocmanual|
will be true upon processing of a part,
and the name of the part is stored in |\childdocname|.
Note that |\jobname| will be set to the filename of the current part
so that each part receives an individual |.aux| file
that does not interfere with the |.aux| file(s) of the main document.
This behaviour can be altered by the alternative form
|\childdocby[*]{|\textit{main}|}| (with a non-empty optional argument)
which uses the |.aux| file of the main document
by setting |\jobname| to \textit{main}.

%%%%%%%%%%%%%%%%%%%%%%%%%%%%%%%%%%%%%%%%%%%%%%%%%%%%%%%%%%%%%%%%%%%%%%%%%%%%%%%%
\subsection{Driver Development}
\label{sec:driver}

The \textsf{childdoc} mechanism can also be use for the development
of definition files such as \LaTeX{} styles or classes.
This case differs from the above setup with multiple parts
included by |\include| in that no |\includeonly| should be invoked.
This can be achieved by starting the include file
(before |\ProvidesPackage|) with:
%
\begin{center}
\begin{tabular}{l}
|\input{childdoc.def}|\\
|\childdocforward{|\textit{main}|}|\\
\end{tabular}
\end{center}
%
or alternatively with:
%
\begin{center}
\begin{tabular}{l}
|\input{childdoc.def}|\\
|\childdocby{|\textit{main}|}|\\
\end{tabular}
\end{center}
%
Both forms have slightly different effects as described above.
The main file is prepared as usual, see \secref{sec:include}.

%%%%%%%%%%%%%%%%%%%%%%%%%%%%%%%%%%%%%%%%%%%%%%%%%%%%%%%%%%%%%%%%%%%%%%%%%%%%%%%%
\subsection{Legacy Detection}
\label{sec:detection}

The directive |\childdocmain| in the main file can detect
whether the complete document or merely a child is to be compiled
even without using the directive |\childdocof|.
This method is deprecated because it is less robust
and there is no compelling reason to use it;
it is merely provided for backward compatibility
and it may be removed in future versions.

If the detection mechanism is to be used,
it is mandatory to correctly specify
the filename of the main file as the argument of |\childdocmain|:
%
\begin{center}
\begin{tabular}{l}
|\input{childdoc.def}|\\
|\childdocmain{|\textit{main}|}|\\
\end{tabular}
\end{center}
%
If |\jobname| does not match the argument \textit{main} of |\childdocmain|,
it is assumed that |\jobname| points to the child file to be compiled.
When using |\childdocmain| with the main file specified as argument,
it suffices to start a child file
with just |\input{|\textit{main}|}|
without loading of the package and using |\childdocof|.
If instead all processing is done
with the appropriate \textsf{childdoc} directives,
the argument of \textit{main} of |\childdocmain| can be empty.

An alternative version of the command line processing described
in \secref{sec:commandline} using the detection mechanism reads:
%
\begin{center}
|... -jobname "|\textit{target}|" "|[\textit{flags}]%
[|\def\jobname{|\textit{dest}|}|]|\input{|\textit{main}|}"|
\end{center}

%%%%%%%%%%%%%%%%%%%%%%%%%%%%%%%%%%%%%%%%%%%%%%%%%%%%%%%%%%%%%%%%%%%%%%%%%%%%%%%%
\subsection{Manual Code}
\label{sec:manual}

In case one cannot be certain whether the definitions file |childdoc.def|
is installed on the target \TeX{} distribution
and one prefers not to ship it,
it is conceivable to paste a few relevant commands into the sources.

To that end, drop all statements |\input{childdoc.def}|
and perform the replacements as outlined below.
Instead of |\childdocmain{|\textit{main}|}| add the following code
to the top of the main file:
%
\begin{center}
\begin{tabular}{l}
|\||ifdefined\childdocname\endinput\||fi\newif\ifchilddoc|\\
|\edef\childdocname{\scantokens\expandafter{\jobname\noexpand}}|\\
|\def\childdocmain{|\textit{main}|}\||ifx\childdocmain\childdocname\||else|\\
|\childdoctrue\includeonly{\childdocname}\let\jobname\childdocmain\||fi|\\
\end{tabular}
\end{center}
%
Instead of |\childdocof{|\textit{main}|}| just include the main file
at the top of each child file:
%
\begin{center}
|\input{|\textit{main}|}|
\end{center}
%
A simple redirection |\childdocforward{|\textit{dest}|}| is achieved by:
%
\begin{center}
|\def\jobname{|\textit{dest}|}\input{\jobname}|
\end{center}
%
The redirection with prefix
|\childdocforwardprefix[|\textit{prefix}|]{|\textit{dest}|}|
is accomplished by:
%
\begin{center}
\begin{tabular}{l}
|{\edef\jobname{\scantokens\expandafter{\jobname\noexpand}}|\\
|\def\redirectjob |\textit{prefix}|#1~~~{\gdef\jobname{|\textit{dest}|#1}}|\\
|\expandafter\redirectjob\jobname~~~}\input{\jobname}|
\end{tabular}
\end{center}

In an alternative approach,
child documents can be compiled by a specific command line
without additional code or specific definitions:
%
\begin{center}
|... -jobname "|\textit{target}|" "|[\textit{flags}]%
|\includeonly{|\textit{dest}|}\input{|\textit{main}|}"|
\end{center}
%

%%%%%%%%%%%%%%%%%%%%%%%%%%%%%%%%%%%%%%%%%%%%%%%%%%%%%%%%%%%%%%%%%%%%%%%%%%%%%%%%
%%%%%%%%%%%%%%%%%%%%%%%%%%%%%%%%%%%%%%%%%%%%%%%%%%%%%%%%%%%%%%%%%%%%%%%%%%%%%%%%
\section{Information}

%%%%%%%%%%%%%%%%%%%%%%%%%%%%%%%%%%%%%%%%%%%%%%%%%%%%%%%%%%%%%%%%%%%%%%%%%%%%%%%%
\subsection{Copyright}

Copyright \copyright{} 2017--2018 Niklas Beisert

This work may be distributed and/or modified under the
conditions of the \LaTeX{} Project Public License, either version 1.3
of this license or (at your option) any later version.
The latest version of this license is in
  \url{http://www.latex-project.org/lppl.txt}
and version 1.3 or later is part of all distributions of \LaTeX{}
version 2005/12/01 or later.

This work has the LPPL maintenance status `maintained'.

The Current Maintainer of this work is Niklas Beisert.

This work consists of the files |README.txt|, |childdoc.ins| and |childdoc.dtx|
as well as the derived files |childdoc.def|, |cdocsamp.tex|
with |cdocsch1.tex|, |cdocsch2.tex|, |cdocspt3.tex|, |cdocspt4.tex|,
|cdocsdrf.tex|, |cdocsfn1.tex|, |cdocsfn2.tex|
as well as |childdoc.pdf|.

%%%%%%%%%%%%%%%%%%%%%%%%%%%%%%%%%%%%%%%%%%%%%%%%%%%%%%%%%%%%%%%%%%%%%%%%%%%%%%%%
\subsection{Files and Installation}

The package consists of the files:
%
\begin{center}
\begin{tabular}{ll}
    |README.txt|   & readme file \\
    |childdoc.ins| & installation file \\
    |childdoc.dtx| & source file \\
    |childdoc.def| & definition file \\
    |cdocsamp.tex| & sample main file \\
    |cdocsch1.tex| & sample include file \\
    |cdocsch2.tex| & sample include file \\
    |cdocspt3.tex| & sample part file \\
    |cdocspt4.tex| & sample part file \\
    |cdocsdrf.tex| & sample redirection file \\
    |cdocsfn1.tex| & sample redirection file \\
    |cdocsfn2.tex| & sample redirection file \\
    |childdoc.pdf| & manual
\end{tabular}
\end{center}
%
The distribution consists of the files
|README.txt|, |childdoc.ins| and |childdoc.dtx|.
%
\begin{itemize}
\item
Run (pdf)\LaTeX{} on |childdoc.dtx|
to compile the manual |childdoc.pdf| (this file).
\item
Run \LaTeX{} on |childdoc.ins| to create the definitions file |childdoc.def|
and the sample |cdocsamp.tex| with include files
|cdocsch1.tex|, |cdocsch2.tex|, |cdocspt3.tex|, |cdocspt4.tex|,
|cdocsdrf.tex|, |cdocsfn1.tex|, |cdocsfn2.tex|.
Then copy the file |childdoc.def| to an appropriate directory of your \LaTeX{}
distribution, e.g.\ \textit{texmf-root}|/tex/latex/childdoc|.
\end{itemize}

%%%%%%%%%%%%%%%%%%%%%%%%%%%%%%%%%%%%%%%%%%%%%%%%%%%%%%%%%%%%%%%%%%%%%%%%%%%%%%%%
\subsection{Related CTAN Packages}

There are several other packages which offer a similar functionality:
%
\begin{itemize}
\item
The packages
\href{http://ctan.org/pkg/docmute}{\textsf{docmute}},
\href{http://ctan.org/pkg/includex}{\textsf{includex}} and
\href{http://ctan.org/pkg/standalone}{\textsf{standalone}}
provide commands to include only the document body of
a child file thus allowing both files to be compiled individually.
\item
The packages \href{http://ctan.org/pkg/subdocs}{\textsf{subdocs}}
and \href{http://ctan.org/pkg/subfiles}{\textsf{subfiles}}
provide structures in which the main and child documents can be
encapsulated and allowing them to be compiled individually.
The inclusion mechanism is different from the conventional |\include|.
\item
The package \href{http://ctan.org/pkg/combine}{\textsf{combine}}
is an elaborate solution to combine several documents into one.
\end{itemize}
%
See also the CTAN topic \href{http://ctan.org/topic/subdocs}{\textsf{subdocs}}
for further related packages.
The present package differs from the above solutions in that
a document structure constructed with the conventional |\include| mechanism
just needs two extra commands at the top of every file
such that all constituent files can be compiled individually.

%%%%%%%%%%%%%%%%%%%%%%%%%%%%%%%%%%%%%%%%%%%%%%%%%%%%%%%%%%%%%%%%%%%%%%%%%%%%%%%%
%\subsection{Feature Suggestions}
%
%The following is a list of features which may be useful for future
%versions of this package:
%%
%\begin{itemize}
%\item
%\ldots
%\end{itemize}

%%%%%%%%%%%%%%%%%%%%%%%%%%%%%%%%%%%%%%%%%%%%%%%%%%%%%%%%%%%%%%%%%%%%%%%%%%%%%%%%
\subsection{Revision History}

%%%%%%%%%%%%%%%%%%%%%%%%%%%%%%%%%%%%%%%%
\paragraph{v2.0:} 2018/12/30

\begin{itemize}
\item
immediate forward processing
\item
added |\childdocby| mechanism
\item
manual restructured
\end{itemize}

%%%%%%%%%%%%%%%%%%%%%%%%%%%%%%%%%%%%%%%%
\paragraph{v1.6:} 2018/01/17

\begin{itemize}
\item
application for development of include files
\item
corrections to manual
\end{itemize}

%%%%%%%%%%%%%%%%%%%%%%%%%%%%%%%%%%%%%%%%
\paragraph{v1.5:} 2017/05/21

\begin{itemize}
\item
more complete structuring introduced
\item
|\childdocof| introduced
\item
|\childdoc| renamed to |\childdocmain|
\item
|\childredirect| renamed to |\childdocforward| and |\childdocforwardprefix|
and functionality expanded
\end{itemize}

%%%%%%%%%%%%%%%%%%%%%%%%%%%%%%%%%%%%%%%%
\paragraph{v1.0:} 2017/04/27

\begin{itemize}
\item
manual and install package
\item
first version published on CTAN
\end{itemize}

%%%%%%%%%%%%%%%%%%%%%%%%%%%%%%%%%%%%%%%%
\paragraph{v0.6:} 2017/04/26

\begin{itemize}
\item
redirection mechanism added
\end{itemize}

%%%%%%%%%%%%%%%%%%%%%%%%%%%%%%%%%%%%%%%%
\paragraph{v0.5:} 2017/04/26

\begin{itemize}
\item
functionality in definition file
\end{itemize}


%%%%%%%%%%%%%%%%%%%%%%%%%%%%%%%%%%%%%%%%%%%%%%%%%%%%%%%%%%%%%%%%%%%%%%%%%%%%%%%%
%%%%%%%%%%%%%%%%%%%%%%%%%%%%%%%%%%%%%%%%%%%%%%%%%%%%%%%%%%%%%%%%%%%%%%%%%%%%%%%%
%%%%%%%%%%%%%%%%%%%%%%%%%%%%%%%%%%%%%%%%%%%%%%%%%%%%%%%%%%%%%%%%%%%%%%%%%%%%%%%%
\appendix

\settowidth\MacroIndent{\rmfamily\scriptsize 000\ }

 \DocInput{childdoc.dtx}

\end{document}
%</driver>
% \fi
%
% %%%%%%%%%%%%%%%%%%%%%%%%%%%%%%%%%%%%%%%%%%%%%%%%%%%%%%%%%%%%%%%%%%%%%%%%%%%%%%
% %%%%%%%%%%%%%%%%%%%%%%%%%%%%%%%%%%%%%%%%%%%%%%%%%%%%%%%%%%%%%%%%%%%%%%%%%%%%%%
% \section{Sample}
%\iffalse
%<*samplemain>
%\fi
%
% The following presents a sample document
% with two chapters, two parts, a title page,
% a compile flag as well as three forwarding files to set the flag.
% It consists of eight |.tex| files:
% \begin{center}
% \begin{tabular}{ll}
% |cdocsamp.tex|&main file\\
% |cdocsch1.tex|&include file for chapter 1\\
% |cdocsch2.tex|&include file for chapter 2\\
% |cdocspt3.tex|&include file for part 3\\
% |cdocspt4.tex|&include file for part 4\\
% |cdocsdrf.tex|&forwarding file for main file in draft mode\\
% |cdocsfi1.tex|&forwarding file for final version of chapter 1\\
% |cdocsfi2.tex|&forwarding file for final version of chapter 2\\
% \end{tabular}
% \end{center}
% Each of the eight files can be compiled directly by the \LaTeX{} compiler.
%
% %%%%%%%%%%%%%%%%%%%%%%%%%%%%%%%%%%%%%%
% \paragraph{Main File.}
%
% The main file is called |cdocsamp.tex|.
%
% Load the \textsf{childdoc} definitions and
% declare the filename for the main document:
%    \begin{macrocode}
\input{childdoc.def}
\childdocmain{}
%    \end{macrocode}

% Optional override for |\version| flag:
%    \begin{macrocode}
%%\ifchilddoc\else\providecommand{\version}{draft}\fi
%    \end{macrocode}

% Define the default values for the |\version| flag
% (|final| for the main file and |draft| for childs):
%    \begin{macrocode}
\ifchilddoc
\providecommand{\version}{draft}
\else
\providecommand{\version}{final}
\fi
%    \end{macrocode}

% Load the standard document class:
%    \begin{macrocode}
\documentclass[12pt]{article}
%    \end{macrocode}

% Start the document body:
%    \begin{macrocode}
\begin{document}
%    \end{macrocode}

% Declare a title page.
% Print title, part of document being processed and version flag:
%    \begin{macrocode}
\addtocounter{page}{-1}
\begin{center}
{\LARGE\bfseries{}childdoc example\par}
\vspace{1cm}
\ifchilddoc
\ifchilddocmanual part\else chapter\fi:
`\childdocname' of `\childdocjob'\par
\else
main document: `\childdocjob'\par
\fi
version: \version\par
\end{center}
\newpage
%    \end{macrocode}

% Manually include selected file,
% otherwise process as usual:
%    \begin{macrocode}
\ifchilddocmanual
\section*{part `\childdocname'}
\input{\childdocname}
\else
%    \end{macrocode}

% Include the two chapters:
%    \begin{macrocode}
\include{cdocsch1}
\include{cdocsch2}
%    \end{macrocode}

% Include the two parts unless only chapters should be displayed:
%    \begin{macrocode}
\ifchilddoc\else
\section{part three}
\input{cdocspt3}
\section{part four}
\input{cdocspt4}
\fi
%    \end{macrocode}

% Process as usual until here:
%    \begin{macrocode}
\fi
%    \end{macrocode}

% End of document body:
%    \begin{macrocode}
\end{document}
%    \end{macrocode}
%\iffalse
%</samplemain>
%\fi
%
% %%%%%%%%%%%%%%%%%%%%%%%%%%%%%%%%%%%%%%
% \paragraph{Chapter Include Files.}
%
% The include files are called |cdocsch1.tex| and |cdocsch2.tex|.
%
%\iffalse
%<*samplechap1|samplechap2>
%\fi

% Optional override for |\version| flag:
%    \begin{macrocode}
%%\providecommand{\version}{final}
%    \end{macrocode}

% Include the main document:
%    \begin{macrocode}
\input{childdoc.def}
\childdocof{cdocsamp}
%    \end{macrocode}

%\iffalse
%</samplechap1|samplechap2>
%\fi
%
%\iffalse
%<*samplechap1>
%\fi
% Some text for chapter 1:
%    \begin{macrocode}
\section{one}
some text in chapter one
%    \end{macrocode}

%\iffalse
%</samplechap1>
%\fi
% Some text for chapter 2:
%\iffalse
%<*samplechap2>
%\fi
%    \begin{macrocode}
\section{two}
more text in chapter two
%    \end{macrocode}

%\iffalse
%</samplechap2>
%\fi
%
% %%%%%%%%%%%%%%%%%%%%%%%%%%%%%%%%%%%%%%
% \paragraph{Part Include Files.}
%
% The include files are called |cdocspt3.tex| and |cdocspt4.tex|.
%
%\iffalse
%<*samplepart3|samplepart4>
%\fi

% Optional override for |\version| flag:
%    \begin{macrocode}
%%\providecommand{\version}{final}
%    \end{macrocode}

% Include the main document:
%    \begin{macrocode}
\input{childdoc.def}
\childdocby{cdocsamp}
%    \end{macrocode}

%\iffalse
%</samplepart3|samplepart4>
%\fi
%
%\iffalse
%<*samplepart3>
%\fi
% Some text for part 3:
%    \begin{macrocode}
some text in part three
%    \end{macrocode}

%\iffalse
%</samplepart3>
%\fi
% Some text for part 4:
%\iffalse
%<*samplepart4>
%\fi
%    \begin{macrocode}
more text in part four
%    \end{macrocode}

%\iffalse
%</samplepart4>
%\fi
%
% %%%%%%%%%%%%%%%%%%%%%%%%%%%%%%%%%%%%%%
% \paragraph{Forwarding for a Complete Draft.}
%
% The following forwarding file |cdocsdrf.tex|
% compiles the main document in draft mode:
%\iffalse
%<*sampledraft>
%\fi
%    \begin{macrocode}
\def\version{draft}
\input{childdoc.def}
\childdocforward{cdocsamp}
%    \end{macrocode}

%\iffalse
%</sampledraft>
%\fi
%
% %%%%%%%%%%%%%%%%%%%%%%%%%%%%%%%%%%%%%%
% \paragraph{Forwarding for Final Version of the Chapters.}
%
% The following forwarding files |cdocsfn1.tex| and |cdocsfn2.tex|
% (with identical content)
% compile the final versions of the child documents
% |cdocsch1.tex| and |cdocsch2.tex|, respectively:
%\iffalse
%<*samplefinal>
%\fi
%    \begin{macrocode}
\def\version{final}
\input{childdoc.def}
\childdocforwardprefix[cdocsamp]{cdocsfn}{cdocsch}
%    \end{macrocode}

%\iffalse
%</samplefinal>
%\fi
%
% %%%%%%%%%%%%%%%%%%%%%%%%%%%%%%%%%%%%%%
% \paragraph{Command Line Processing.}
%
% The following three command lines generate the output files
% |cdocscld|, |cdocscl1| and |cdocscl2|
% which should be identical to
% |cdocsdrf|, |cdocsch1| and |cdocsfn2|, respectively:
% \begin{center}
% \begin{tabular}{l}
% |latex -jobname cdocscld \|\\
% |  "\def\version{draft}\input{childdoc.def}\childdocforward{cdocsamp}"|\\
% |latex -jobname cdocscl1 \|\\
% |  "\input{childdoc.def}\childdocforward[cdocsamp]{cdocsch1}"|\\
% |latex -jobname cdocscl2 \|\\
% |  "\def\version{final}\input{childdoc.def}\childdocforward{cdocsch2}"|
% \end{tabular}
% \end{center}
% Note that the trailing backslash on each first line
% merely continues the input to the second line
% (for convenient cut ant paste).
% Furthermore, the command |latex| can be replaced by any
% of its alternative versions such as |pdflatex|.
%
% %%%%%%%%%%%%%%%%%%%%%%%%%%%%%%%%%%%%%%%%%%%%%%%%%%%%%%%%%%%%%%%%%%%%%%%%%%%%%%
% %%%%%%%%%%%%%%%%%%%%%%%%%%%%%%%%%%%%%%%%%%%%%%%%%%%%%%%%%%%%%%%%%%%%%%%%%%%%%%
% \section{Implementation}
%\iffalse
%<*package>
%\fi
%
% This section describes the definitions file |childdoc.def|.

% The definitions cannot be loaded using |\usepackage| or |\RequirePackage|
% which has a mechanism to prevent loading a style file more than once.
% When loading the definitions by means of |\input|
% multiple instances have to be prevented manually:
%\iffalse
%This code needs to be before the `\ProvidesFile' directive
%which is defined at the beginning of this file.
%Therefore it is also placed there and commented out here.
%</package>
%<*discard>
%\fi
%    \begin{macrocode}
\ifdefined\childdocmain\endinput\fi
%    \end{macrocode}
%\iffalse
%</discard>
%<*package>
%\fi
%
% \macro{\ifchilddoc}
% \macro{\ifchilddocmanual}
% The conditional |\ifchilddoc| tells whether a
% child (true) or main (false) document is being compiled.
% The conditional |\ifchilddocmanual| tells whether
% the |\includeonly| mechanism is used (false) or
% the selection of child files must be performed manually (true).
% The definitions initialise to false:
%    \begin{macrocode}
\newif\ifchilddoc
\newif\ifchilddocmanual
%    \end{macrocode}

% \macro{\childdocname}
% \macro{\childdocjob}
% The macro |\childdocname| stores the name of the main document
% to be compiled. The macro |\childdocjob| stores the name of
% the document on which the \LaTeX{} compiler was originally invoked.
% The content of |\jobname| cannot be compared
% to filenames specified in the source due to different catcodes.
% The following code rescans |\jobname|, stores the result
% in |\childdocname| and saves a copy in |\childdocjob|:
%    \begin{macrocode}
\edef\childdocname{\scantokens\expandafter{\jobname\noexpand}}
\let\childdocjob\childdocname
%    \end{macrocode}

% \macro{\childdocdisable}
% The macro |\childdocdisable| prevents the main file
% from being processed more than once.
% At this stage, the main document command |\childdocmain|
% is assumed to be called once again where it should do nothing.
% Any subsequent call to it should prevent
% a secondary processing of the main document
% It overwrites the forwarding commands
% |\childdocof| and |\childdocforward|
% with empty macros to prevent further inclusions of the main document:
%    \begin{macrocode}
\newcommand{\childdocdisable}
{
  \renewcommand{\childdocmain}[1]{\renewcommand{\childdocmain}[1]{\endinput}}
  \renewcommand{\childdocof}[1]{}
  \renewcommand{\childdocby}[2][]{}
  \renewcommand{\childdocforward}[2][]{}
  \renewcommand{\childdocdisable}{}
}
%    \end{macrocode}

% \macro{\childdocmain}
% The macro |\childdocmain| is to be called at the top of the main file
% with nothing or the main filename (without extension) as argument.
% First, it breaks loops.
% If the argument is not empty and does not match |\childdocname|
% (which is set by the first inclusion of |childdoc.def|),
% |\ifchilddoc| is set to true, |\includeonly| is applied to the child file
% and |\jobname| is set to the main file
% (for proper handling of |.aux| files):
%    \begin{macrocode}
\newcommand{\childdocmain}[1]
{
  \childdocdisable\childdocmain{}
  \if?#1?\else
    \begingroup
      \def\childdoctmp{#1}
      \ifx\childdoctmp\childdocname
        \def\childdoctmp{}
      \else
        \def\childdoctmp
        {
          \childdoctrue
          \includeonly{\childdocname}
          \def\childdocjob{#1}
          \def\jobname{#1}
        }
      \fi
      \expandafter
    \endgroup
    \childdoctmp
  \fi
}
%    \end{macrocode}

% \macro{\childdocof}
% The command |\childdocof| redirects
% compilation to the main file |#1|.
%    \begin{macrocode}
\newcommand{\childdocof}[1]
{
  \childdocdisable
  \childdoctrue
  \includeonly{\childdocname}
  \def\jobname{#1}
  \def\childdocjob{#1}
  \input{#1}
}
%    \end{macrocode}

% \macro{\childdocby}
% The command |\childdocby| ....
%    \begin{macrocode}
\newcommand{\childdocby}[2][]
{
  \childdocdisable
  \childdoctrue
  \childdocmanualtrue
  \if?#1?\else
    \def\jobname{#2}
  \fi
  \def\childdocjob{#2}
  \input{#2}
  \endinput
}
%    \end{macrocode}

% \macro{\childdocforward}
% The command |\childdocforward| redirects
% compilation to the main file or
% (if the optional argument is given) a child file.
% Parameters are set as if the main file
% or a child file starting with |\childdocof| was compiled.
% Then compilation is handed over to the main file:
%    \begin{macrocode}
\newcommand{\childdocforward}[2][]
{
  \begingroup
    \if?#1?
      \def\childdoctmp
      {
        \def\childdocname{#2}
        \def\childdocjob{#2}
        \def\jobname{#2}
        \input{#2}
        \endinput
      }
    \else
      \def\childdoctmp
      {
        \childdocdisable
        \def\childdocname{#2}
        \childdoctrue
        \includeonly{#2}
        \def\childdocjob{#1}
        \def\jobname{#1}
        \input{#1}
        \endinput
      }
    \fi
    \expandafter
  \endgroup
  \childdoctmp
}
%    \end{macrocode}

% \macro{\childdocforwardprefix}
% The command |\childdocforwardprefix| redirects
% compilation to the main or a child file by means of a pattern.
% The prefix |#1| in the current filename is replaced by |#2|
% and the suffix of the current filename is kept
% (it is assumed that the filename does not contain the substring `|~~~|'
% which is used as a delimiter).
% Compilation is handed over to the new file by |\childdocforward|:
%    \begin{macrocode}
\newcommand{\childdocforwardprefix}[3][]
{
  \begingroup
    \def\childdocextract #2##1~~~{\def\childdoctmp{\childdocforward[#1]{#3##1}}}
    \expandafter\childdocextract\childdocname~~~
    \expandafter
  \endgroup
  \childdoctmp
}
%    \end{macrocode}

% \macro{\childdoc}
% The deprecated macro |\childdoc| is a legacy version of |\childdocmain|:
%    \begin{macrocode}
\newcommand{\childdoc}{\childdocmain}
%    \end{macrocode}

% \macro{\childdocredirect}
% The deprecated macro |\childdocredirect| is a legacy version
% of |\childdocforward| and |\childdocforwardprefix|:
%    \begin{macrocode}
\newcommand{\childdocredirect}[2][]
{
  \begingroup
    \if?#1?
      \def\childdoctmp{\childdocforward{#2}}
    \else
      \def\childdoctmp{\childdocforwardprefix{#1}{#2}}
    \fi
    \expandafter
  \endgroup
  \childdoctmp
}
%    \end{macrocode}

%\iffalse
%</package>
%\fi
%
\endinput
|\\
|\childdocforwardprefix{final}{child}|
\end{tabular}
\end{center}
%

Note that when several versions of a main file and/or of each child file
are to be generated, it may be convenient to set up a |Makefile| or
shell script to automatise the process.

%%%%%%%%%%%%%%%%%%%%%%%%%%%%%%%%%%%%%%%%%%%%%%%%%%%%%%%%%%%%%%%%%%%%%%%%%%%%%%%%
\subsection{Command Line Processing}
\label{sec:commandline}

The effect of redirection files can also be achieved by invoking
the \LaTeX{} compiler with a more elaborate command line.
Most conveniently this should be done as part
of a shell script or a |Makefile|.

When using \textsf{childdoc} in the main file, the following
command lines effectively perform a redirection
(note that depending on the shell being used,
backslashes may have to be doubled: `|\|' $\to$ `|\\|'):
%
\begin{center}
|... -jobname "|\textit{target}|" |\\|"|[\textit{flags}]%
|% \iffalse
%
% childdoc.dtx Copyright (C) 2017-2018 Niklas Beisert
%
% This work may be distributed and/or modified under the
% conditions of the LaTeX Project Public License, either version 1.3
% of this license or (at your option) any later version.
% The latest version of this license is in
%   http://www.latex-project.org/lppl.txt
% and version 1.3 or later is part of all distributions of LaTeX
% version 2005/12/01 or later.
%
% This work has the LPPL maintenance status `maintained'.
%
% The Current Maintainer of this work is Niklas Beisert.
%
% This work consists of the files childdoc.dtx and childdoc.ins
% and the derived files childdoc.def and cdocsamp.tex with
% cdocsch1.tex, cdocsch2.tex, cdocsdrf.tex, cdocsfn1.tex, cdocsfn2.tex.
%
%<package>\ifdefined\childdocmain\endinput\fi
%<package>\ProvidesFile{childdoc.def}[2018/12/30 v2.0 child document driver]
%<samplemain>\ProvidesFile{cdocsamp.tex}[2018/12/30 v2.0 sample for childdoc]
%<*driver>
%\ProvidesFile{childdoc.drv}[2018/12/30 v2.0 childdoc reference manual file]
\PassOptionsToClass{10pt,a4paper}{article}
\documentclass{ltxdoc}

\usepackage[margin=35mm]{geometry}
\usepackage{hyperref}
\usepackage{hyperxmp}
\usepackage[usenames]{color}

\hypersetup{colorlinks=true}
\hypersetup{pdfstartview=FitH}
\hypersetup{pdfpagemode=UseNone}
\hypersetup{pdfsource={}}
\hypersetup{pdflang={en-UK}}
\hypersetup{pdfcopyright={Copyright 2017-2018 Niklas Beisert.
  This work may be distributed and/or modified under the
  conditions of the LaTeX Project Public License, either version 1.3
  of this license or (at your option) any later version.}}
\hypersetup{pdflicenseurl={http://www.latex-project.org/lppl.txt}}
\hypersetup{pdfcontactaddress={ETH Zurich, ITP, HIT K,
  Wolfgang-Pauli-Strasse 27}}
\hypersetup{pdfcontactpostcode={8093}}
\hypersetup{pdfcontactcity={Zurich}}
\hypersetup{pdfcontactcountry={Switzerland}}
\hypersetup{pdfcontactemail={nbeisert@itp.phys.ethz.ch}}
\hypersetup{pdfcontacturl={http://people.phys.ethz.ch/\xmptilde nbeisert/}}

\newcommand{\secref}[1]{\hyperref[#1]{section \ref*{#1}}}

\parskip1ex
\parindent0pt
\let\olditemize\itemize
\def\itemize{\olditemize\parskip0pt}

\begin{document}

\title{The \textsf{childdoc} Package}
\hypersetup{pdftitle={The childdoc Package}}
\author{Niklas Beisert\\[2ex]
  Institut f\"ur Theoretische Physik\\
  Eidgen\"ossische Technische Hochschule Z\"urich\\
  Wolfgang-Pauli-Strasse 27, 8093 Z\"urich, Switzerland\\[1ex]
  \href{mailto:nbeisert@itp.phys.ethz.ch}
  {\texttt{nbeisert@itp.phys.ethz.ch}}}
\hypersetup{pdfauthor={Niklas Beisert}}
\hypersetup{pdfsubject={Manual for the LaTeX2e Package childdoc}}
\date{30 December 2018, \textsf{v2.0}}
\maketitle

\begin{abstract}\noindent
\textsf{childdoc} is a \LaTeXe{} package
that enables the direct compilation
of document sections included by |\include|
to individual files.
\end{abstract}

\begingroup
\parskip0ex
\tableofcontents
\endgroup

%%%%%%%%%%%%%%%%%%%%%%%%%%%%%%%%%%%%%%%%%%%%%%%%%%%%%%%%%%%%%%%%%%%%%%%%%%%%%%%%
%%%%%%%%%%%%%%%%%%%%%%%%%%%%%%%%%%%%%%%%%%%%%%%%%%%%%%%%%%%%%%%%%%%%%%%%%%%%%%%%
\section{Introduction}

\LaTeX{} provides a mechanism to structure a large document (such as a book)
into a main file and several child files (containing the chapters)
using the |\include| command.
This mechanism is beneficial for documents
which span hundreds of pages in order to
make the source file(s) more manageable.
Moreover, compilation can be restricted to
selected child files by means of the |\includeonly| command.
The latter feature can be used to reduce the compilation time while editing
(this was significantly more useful in the earlier days of \LaTeX{})
or to generate a smaller document which is easier to navigate.
Another application of |\includeonly| is to generate
documents consisting of selected parts of the complete document.

However, there are a few drawbacks of the plain |\include| mechanism:
\begin{itemize}
\item
The child files cannot be compiled on their own,
they can only be compiled via the main file.
A naive editing environment
(such as a text editor with an option
to have the current file processed by \LaTeX)
may require one to switch to the main file before compiling;
attempting to compile the child file produces errors.
\item
The main file must be modified (each time)
to adjust the |\includeonly| command
to the present needs. This easily leaves the main file in a messy state.
\item
The generated document will always carry the filename
of the main document. This is inconvenient if
several child files are to be compiled and
to be kept for distribution.
\end{itemize}

The present package provides a simple interface
to make child files individually compilable by \LaTeX{}.
Compiling a child file then has the same effect as compiling
the main file with an |\includeonly| command
to select the appropriate child.
Moreover the generated document will carry the name of the child
rather than the main file.
This resolves all three above issues.

This feature is meant to make the editing of books,
thesis documents and lecture notes somewhat more convenient.
However, the package can also be used efficiently for
composing a series of documents (such as exercise sheets)
which are typically distributed individually.
It then assists the author in generating the individual documents
(potentially in different versions)
as well as a document containing the collected series.
Another application is in developing style files
or other kinds of included material
where compilation of the style file could redirect
to a sample or test file.

%%%%%%%%%%%%%%%%%%%%%%%%%%%%%%%%%%%%%%%%%%%%%%%%%%%%%%%%%%%%%%%%%%%%%%%%%%%%%%%%
%%%%%%%%%%%%%%%%%%%%%%%%%%%%%%%%%%%%%%%%%%%%%%%%%%%%%%%%%%%%%%%%%%%%%%%%%%%%%%%%
\section{Usage}

First of all, the package \textsf{childdoc} is \emph{not} a standard
\LaTeXe{} |.sty| style file! Therefore it needs to be invoked in
a non-standard way.

%%%%%%%%%%%%%%%%%%%%%%%%%%%%%%%%%%%%%%%%%%%%%%%%%%%%%%%%%%%%%%%%%%%%%%%%%%%%%%%%
\subsection{Included Files}
\label{sec:include}

%%%%%%%%%%%%%%%%%%%%%%%%%%%%%%%%%%%%%%%%
\DescribeMacro{\childdocmain}
To use the package, add the commands
\begin{center}
\begin{tabular}{l}
|\input{childdoc.def}|\\
|\childdocmain{}|\\
\end{tabular}
\end{center}
at the very top of the main \LaTeX{} file,
in particular \emph{before} the |\documentclass| statement!
The argument of |\childdocmain| should be left empty
(but it must be present).

%%%%%%%%%%%%%%%%%%%%%%%%%%%%%%%%%%%%%%%%
\DescribeMacro{\childdocof}
Furthermore, add the commands
\begin{center}
\begin{tabular}{l}
|\input{childdoc.def}|\\
|\childdocof{|\textit{main}|}|\\
\end{tabular}
\end{center}
at the top of every child file \textit{child}
which is included by |\include{|\textit{child}|}|
from within the main file
(or at least for those files to be compiled individually).
The argument \textit{main} must be the filename of the main file.

There are a couple of
considerations in setting up the main and child documents:

%%%%%%%%%%%%%%%%%%%%%%%%%%%%%%%%%%%%%%%%
\paragraph{Restrictions.}

Please note the following restrictions:
\begin{itemize}
\item
|\childdocmain| must be called with one argument \textit{main}
to ensure compatibility with earlier version of the package.
It must either be empty (|\childdocmain{}|)
or precisely match the filename of the main file in which it is specified.
See \secref{sec:detection} for further information.
\item
The filename \textit{main} must be specified without the |.tex| extension.
\item
The filename \textit{main} is case sensitive
(even in case-insensitive file systems)
due to internal string comparison.
\item
The argument \textit{main} should be fully expanded, it cannot be a macro.
\item
Subdirectories and special characters should be avoided in filenames.
\item
The command |\childdocmain{|\textit{main}|}| must be followed by a whitespace.
It should not be followed immediately by another command
or by a comment mark `|%|'.
This is because the \TeX{} parser reads the token immediately following
the argument of |\childdocmain| and puts it
at the beginning of every child section;
however, a white\-space is ignored.
\end{itemize}

%%%%%%%%%%%%%%%%%%%%%%%%%%%%%%%%%%%%%%%%
\paragraph{Content of Main File.}

It is advisable to place all content in the child files included by |\include|.
Any output contained in the main file will appear in all child documents
unless suppressed manually;
it cannot be suppressed automatically by the |\includeonly| directive
and thus should normally be avoided.
A method to include some content in the main file
by means of conditional processing is described in \secref{sec:conditional}.

%%%%%%%%%%%%%%%%%%%%%%%%%%%%%%%%%%%%%%%%
\paragraph{Page Numbering.}

When only a part of the document is compiled,
the appropriate numbering of pages
(as well as other status parameters)
is determined from the |.aux| files.
The latter contain information from previous passes.
However this information needs to propagate through
all intermediate child documents.
Therefore the page numbering in child documents may well
be inconsistent until the complete document is compiled at least once.

A useful (if unconventional) way to always ensure a consistent
page numbering is to restart the numbering in each child document
and denote the pages by `\textit{child}|.|\textit{page}'
where \textit{child} represents the chapter/section number of the child file.
This can be achieved by the command
|\numberwithin{page}{|\textit{child}|}|
of the \textsf{amsmath} package
where \textit{child} can be |chapter| or |section|
depending on the chosen structuring.
Alternatively, one can modify the macro |\thepage| appropriately
and reset the counter |page| at the start of each child file.

%%%%%%%%%%%%%%%%%%%%%%%%%%%%%%%%%%%%%%%%%%%%%%%%%%%%%%%%%%%%%%%%%%%%%%%%%%%%%%%%
\subsection{Conditional Processing}
\label{sec:conditional}

The package provides a mechanism to compile different versions
of a document. To customise the versions further some conditional processing
can come in handy to distinguish which version is being compiled.
The package provides two macros to describe the compilation context:

%%%%%%%%%%%%%%%%%%%%%%%%%%%%%%%%%%%%%%%%
\DescribeMacro{\ifchilddoc}
The conditional |\ifchilddoc| distinguishes between the compilation of
child documents and the main document:
%
\begin{center}
|\ifchilddoc |\textit{child-code}| |[|\||else |\textit{main-code}]| \||fi|
\end{center}

%%%%%%%%%%%%%%%%%%%%%%%%%%%%%%%%%%%%%%%%
\DescribeMacro{\childdocname}
\DescribeMacro{\childdocjob}
The macro |\childdocname| contains the filename (without extension)
of the main or child file being processed.
Note that |\childdocjob| will always contain the name of the main file.

%%%%%%%%%%%%%%%%%%%%%%%%%%%%%%%%%%%%%%%%
\paragraph{Title Page.}

Conditional processing can be used to include a title or banner page
in the main document when proper precautions are taken.
Importantly, the code in the main file should ensure that the page counter
(as well as other status parameters which are stored in the |.aux| files)
takes the same value after the conditional processing.
Otherwise the page numbers may take divergent values
depending on which part is compiled.

For example, a title page could be declared by:
%
\begin{center}
\begin{tabular}{l}
|\ifchilddoc\||else|\\
|\addtocounter{page}{-1}|\\
\textit{code for title page}\\
|\newpage|\\
|\||fi|
\end{tabular}
\end{center}
%
A banner page for the child documents can be generated by:
%
\begin{center}
\begin{tabular}{l}
|\ifchilddoc|\\
|\addtocounter{page}{-1}|\\
\textit{code for banner page}\\
|\newpage|\\
|\||fi|
\end{tabular}
\end{center}
%
Here one could write a message such as:
\begin{center}
|This is the part \childdocname{} of \childdocjob{}.|
\end{center}

%%%%%%%%%%%%%%%%%%%%%%%%%%%%%%%%%%%%%%%%%%%%%%%%%%%%%%%%%%%%%%%%%%%%%%%%%%%%%%%%
\subsection{Flags}
\label{sec:flags}

The package makes it easy to generate different versions
of the main or child documents.
To this end compilation flags can be defined
and assigned different default values.
They will be particularly useful in conjunction
with the forwarding mechanism described in \secref{sec:forward}.

For example, it may be useful to have a flag |\version|
which can be set to |draft| or |final|.
The document source will contain some conditional code
depending on the value of |\version|.
Suppose further, the flag should default to |final| for the main file
and to |draft| for child files
which is a natural assignment for editing the document.
This is achieved by placing the following code
in the preamble of the main document
(below the |\childdocmain| directive):
%
\begin{center}
\begin{tabular}{l}
|\ifchilddoc|\\
|\providecommand{\version}{draft}|\\
|\||else|\\
|\providecommand{\version}{final}|\\
|\||fi|
\end{tabular}
\end{center}
%
The definition by |\providecommand| makes sure
that previous definitions are not overwritten.
Further statements |\providecommand{\version}{...}|
can thus be added before the above code to override it.

For the main file, one might add a line
(between |\childdocmain| and the above block)
%
\begin{center}
|%\ifchilddoc\||else\providecommand{\version}{draft}\||fi|
\end{center}
%
which can be uncommented to produce a draft version.
Likewise one can add a line to the very top of a child file
(above the |\childdocof{|\textit{main}|}| directive)
%
\begin{center}
|%\providecommand{\version}{final}|
\end{center}
%
which can be uncommented to produce the final version of this child document.

%%%%%%%%%%%%%%%%%%%%%%%%%%%%%%%%%%%%%%%%%%%%%%%%%%%%%%%%%%%%%%%%%%%%%%%%%%%%%%%%
\subsection{Forwarding}
\label{sec:forward}

Different versions of the main or child documents
using compilation flags as described in \secref{sec:flags}
can be (permanently) stored in different files
for convenient compilation, viewing and distribution.
To this end, the package defines a command
to pass on compilation to a different file:

%%%%%%%%%%%%%%%%%%%%%%%%%%%%%%%%%%%%%%%%
\DescribeMacro{\childdocforward}
The command |\childdocforward| redirects processing to
another source file:
%
\begin{center}
\begin{tabular}{l}
|\input{childdoc.def}|\\
|\childdocforward[|\textit{main}|]{|\textit{dest}|}|\\
\end{tabular}
\end{center}
%
The argument \textit{dest} is the destination file
(without extension).
It should be the main file or one of the child files.
Note that further \textsf{childdoc} directives
such as |\childdocof| and |\childdocforward|
in the indicated file will be processed in this form.
The optional argument \textit{main}
passes on directly to the main file \textit{main}
while pretending to compile the child \textit{dest}.
This form behaves as if \textit{dest}
issues |\childdocof{|\textit{main}|}| right away,
and no further \textsf{childdoc} directives will be processed.

%%%%%%%%%%%%%%%%%%%%%%%%%%%%%%%%%%%%%%%%
\DescribeMacro{\...prefix}
In the alternative form |\childdocforwardprefix|,
%
\begin{center}
\begin{tabular}{l}
|\input{childdoc.def}|\\
|\childdocforwardprefix[|\textit{main}|]{|\textit{prefix}|}{|\textit{dest}|}|
\end{tabular}
\end{center}
%
the destination file is determined by a pattern
depending on the current file:
To make this work, the current file must be called
`{\textit{prefix}\hspace{0.2em}\textit{suffix}}'
with \textit{prefix} matching precisely the argument.
Processing is then passed on to the file
`{\textit{dest}\hspace{0.2em}\textit{suffix}}'.
Surely, the same effect is achieved by
directly specifying the
argument `{\textit{dest}\hspace{0.2em}\textit{suffix}}'
in the first form.
However, that requires to set up a different file
for each child. With the alternative form of the command
all these files can have exactly the same content
which simplifies setting them up and maintaining them.

For example, the following file |draft.tex|
with a compilation flag |\version| as described in \secref{sec:flags}
compiles the main document as a draft:
%
\begin{center}
\begin{tabular}{l}
|\def\version{draft}|\\
|\input{childdoc.def}|\\
|\childdocforward{|\textit{main}|}|
\end{tabular}
\end{center}
%
Likewise, the following files |final|\textit{nn}|.tex|
compile the final version of the child document
|child|\textit{nn}|.tex|:
%
\begin{center}
\begin{tabular}{l}
|\def\version{final}|\\
|\input{childdoc.def}|\\
|\childdocforwardprefix{final}{child}|
\end{tabular}
\end{center}
%

Note that when several versions of a main file and/or of each child file
are to be generated, it may be convenient to set up a |Makefile| or
shell script to automatise the process.

%%%%%%%%%%%%%%%%%%%%%%%%%%%%%%%%%%%%%%%%%%%%%%%%%%%%%%%%%%%%%%%%%%%%%%%%%%%%%%%%
\subsection{Command Line Processing}
\label{sec:commandline}

The effect of redirection files can also be achieved by invoking
the \LaTeX{} compiler with a more elaborate command line.
Most conveniently this should be done as part
of a shell script or a |Makefile|.

When using \textsf{childdoc} in the main file, the following
command lines effectively perform a redirection
(note that depending on the shell being used,
backslashes may have to be doubled: `|\|' $\to$ `|\\|'):
%
\begin{center}
|... -jobname "|\textit{target}|" |\\|"|[\textit{flags}]%
|\input{childdoc.def}\childdocforward[|\textit{main}|]{|\textit{dest}|}"|
\end{center}
%
Here \textit{target} is the name of the output file,
\textit{main} is the name of the main file
and \textit{dest} is the name of the main or child file to be processed
(all filenames without extensions).
The optional argument \textit{main} can be omitted
if \textit{main} matches \textit{dest}.
Optionally, compilation \textit{flags} can be defined via |\def| commands.
This command line makes the \TeX{} engine believe
it is compiling the file \textit{target}
whose content is specified as the latter parameter.
The provided code then forwards the processing to
\textit{main} or \textit{dest} as described in \secref{sec:forward}.

%%%%%%%%%%%%%%%%%%%%%%%%%%%%%%%%%%%%%%%%%%%%%%%%%%%%%%%%%%%%%%%%%%%%%%%%%%%%%%%%
\subsection{Include by Input}
\label{sec:input}

Including child documents by |\include| has some restrictions by design.
Most notably, the content of a child document always occupies
its own set of pages; pages cannot be shared between child documents.
Usually, this behaviour makes perfect sense
because each child document contain an essential part of the document.
However, in some situations it may be desirable to compose
a document from a collection of parts
without having mandatory page breaks between then.
For this case, the package
provides a mechanism to include parts
by |\input| which can also be processed individually.
However, by construction this mechanism
requires manual handling of the content to be output.

%%%%%%%%%%%%%%%%%%%%%%%%%%%%%%%%%%%%%%%%
\DescribeMacro{\ifchilddocmanual}
The main file should be prepared as usual, see \secref{sec:include}.
However, the document body must make a distinction
between processing of an individual part and of the main document, e.g.:
%
\begin{center}
\begin{tabular}{l}
|\ifchilddocmanual|\\
|\input{\childdocname}|\\
|\||else|\\
\textit{document body with }|\input{|\textit{part}|}|\\
|\||fi|
\end{tabular}
\end{center}
%
The conditional |\ifchilddocmanual| is true whenever
a part to be included by |\input| is being compiled,
and the name of the part is stored in |\childdocname|.

%%%%%%%%%%%%%%%%%%%%%%%%%%%%%%%%%%%%%%%%
\DescribeMacro{\childdocby}
Each part to be included by |\input| should start with:
%
\begin{center}
\begin{tabular}{l}
|\input{childdoc.def}|\\
|\childdocby{|\textit{main}|}|\\
\end{tabular}
\end{center}
%
The directive |\childdocby| is similar to |\childdocof|
described in \secref{sec:include},
but the subsequent selection of content must be done manually.
To that end, both |\ifchilddoc| and |\ifchilddocmanual|
will be true upon processing of a part,
and the name of the part is stored in |\childdocname|.
Note that |\jobname| will be set to the filename of the current part
so that each part receives an individual |.aux| file
that does not interfere with the |.aux| file(s) of the main document.
This behaviour can be altered by the alternative form
|\childdocby[*]{|\textit{main}|}| (with a non-empty optional argument)
which uses the |.aux| file of the main document
by setting |\jobname| to \textit{main}.

%%%%%%%%%%%%%%%%%%%%%%%%%%%%%%%%%%%%%%%%%%%%%%%%%%%%%%%%%%%%%%%%%%%%%%%%%%%%%%%%
\subsection{Driver Development}
\label{sec:driver}

The \textsf{childdoc} mechanism can also be use for the development
of definition files such as \LaTeX{} styles or classes.
This case differs from the above setup with multiple parts
included by |\include| in that no |\includeonly| should be invoked.
This can be achieved by starting the include file
(before |\ProvidesPackage|) with:
%
\begin{center}
\begin{tabular}{l}
|\input{childdoc.def}|\\
|\childdocforward{|\textit{main}|}|\\
\end{tabular}
\end{center}
%
or alternatively with:
%
\begin{center}
\begin{tabular}{l}
|\input{childdoc.def}|\\
|\childdocby{|\textit{main}|}|\\
\end{tabular}
\end{center}
%
Both forms have slightly different effects as described above.
The main file is prepared as usual, see \secref{sec:include}.

%%%%%%%%%%%%%%%%%%%%%%%%%%%%%%%%%%%%%%%%%%%%%%%%%%%%%%%%%%%%%%%%%%%%%%%%%%%%%%%%
\subsection{Legacy Detection}
\label{sec:detection}

The directive |\childdocmain| in the main file can detect
whether the complete document or merely a child is to be compiled
even without using the directive |\childdocof|.
This method is deprecated because it is less robust
and there is no compelling reason to use it;
it is merely provided for backward compatibility
and it may be removed in future versions.

If the detection mechanism is to be used,
it is mandatory to correctly specify
the filename of the main file as the argument of |\childdocmain|:
%
\begin{center}
\begin{tabular}{l}
|\input{childdoc.def}|\\
|\childdocmain{|\textit{main}|}|\\
\end{tabular}
\end{center}
%
If |\jobname| does not match the argument \textit{main} of |\childdocmain|,
it is assumed that |\jobname| points to the child file to be compiled.
When using |\childdocmain| with the main file specified as argument,
it suffices to start a child file
with just |\input{|\textit{main}|}|
without loading of the package and using |\childdocof|.
If instead all processing is done
with the appropriate \textsf{childdoc} directives,
the argument of \textit{main} of |\childdocmain| can be empty.

An alternative version of the command line processing described
in \secref{sec:commandline} using the detection mechanism reads:
%
\begin{center}
|... -jobname "|\textit{target}|" "|[\textit{flags}]%
[|\def\jobname{|\textit{dest}|}|]|\input{|\textit{main}|}"|
\end{center}

%%%%%%%%%%%%%%%%%%%%%%%%%%%%%%%%%%%%%%%%%%%%%%%%%%%%%%%%%%%%%%%%%%%%%%%%%%%%%%%%
\subsection{Manual Code}
\label{sec:manual}

In case one cannot be certain whether the definitions file |childdoc.def|
is installed on the target \TeX{} distribution
and one prefers not to ship it,
it is conceivable to paste a few relevant commands into the sources.

To that end, drop all statements |\input{childdoc.def}|
and perform the replacements as outlined below.
Instead of |\childdocmain{|\textit{main}|}| add the following code
to the top of the main file:
%
\begin{center}
\begin{tabular}{l}
|\||ifdefined\childdocname\endinput\||fi\newif\ifchilddoc|\\
|\edef\childdocname{\scantokens\expandafter{\jobname\noexpand}}|\\
|\def\childdocmain{|\textit{main}|}\||ifx\childdocmain\childdocname\||else|\\
|\childdoctrue\includeonly{\childdocname}\let\jobname\childdocmain\||fi|\\
\end{tabular}
\end{center}
%
Instead of |\childdocof{|\textit{main}|}| just include the main file
at the top of each child file:
%
\begin{center}
|\input{|\textit{main}|}|
\end{center}
%
A simple redirection |\childdocforward{|\textit{dest}|}| is achieved by:
%
\begin{center}
|\def\jobname{|\textit{dest}|}\input{\jobname}|
\end{center}
%
The redirection with prefix
|\childdocforwardprefix[|\textit{prefix}|]{|\textit{dest}|}|
is accomplished by:
%
\begin{center}
\begin{tabular}{l}
|{\edef\jobname{\scantokens\expandafter{\jobname\noexpand}}|\\
|\def\redirectjob |\textit{prefix}|#1~~~{\gdef\jobname{|\textit{dest}|#1}}|\\
|\expandafter\redirectjob\jobname~~~}\input{\jobname}|
\end{tabular}
\end{center}

In an alternative approach,
child documents can be compiled by a specific command line
without additional code or specific definitions:
%
\begin{center}
|... -jobname "|\textit{target}|" "|[\textit{flags}]%
|\includeonly{|\textit{dest}|}\input{|\textit{main}|}"|
\end{center}
%

%%%%%%%%%%%%%%%%%%%%%%%%%%%%%%%%%%%%%%%%%%%%%%%%%%%%%%%%%%%%%%%%%%%%%%%%%%%%%%%%
%%%%%%%%%%%%%%%%%%%%%%%%%%%%%%%%%%%%%%%%%%%%%%%%%%%%%%%%%%%%%%%%%%%%%%%%%%%%%%%%
\section{Information}

%%%%%%%%%%%%%%%%%%%%%%%%%%%%%%%%%%%%%%%%%%%%%%%%%%%%%%%%%%%%%%%%%%%%%%%%%%%%%%%%
\subsection{Copyright}

Copyright \copyright{} 2017--2018 Niklas Beisert

This work may be distributed and/or modified under the
conditions of the \LaTeX{} Project Public License, either version 1.3
of this license or (at your option) any later version.
The latest version of this license is in
  \url{http://www.latex-project.org/lppl.txt}
and version 1.3 or later is part of all distributions of \LaTeX{}
version 2005/12/01 or later.

This work has the LPPL maintenance status `maintained'.

The Current Maintainer of this work is Niklas Beisert.

This work consists of the files |README.txt|, |childdoc.ins| and |childdoc.dtx|
as well as the derived files |childdoc.def|, |cdocsamp.tex|
with |cdocsch1.tex|, |cdocsch2.tex|, |cdocspt3.tex|, |cdocspt4.tex|,
|cdocsdrf.tex|, |cdocsfn1.tex|, |cdocsfn2.tex|
as well as |childdoc.pdf|.

%%%%%%%%%%%%%%%%%%%%%%%%%%%%%%%%%%%%%%%%%%%%%%%%%%%%%%%%%%%%%%%%%%%%%%%%%%%%%%%%
\subsection{Files and Installation}

The package consists of the files:
%
\begin{center}
\begin{tabular}{ll}
    |README.txt|   & readme file \\
    |childdoc.ins| & installation file \\
    |childdoc.dtx| & source file \\
    |childdoc.def| & definition file \\
    |cdocsamp.tex| & sample main file \\
    |cdocsch1.tex| & sample include file \\
    |cdocsch2.tex| & sample include file \\
    |cdocspt3.tex| & sample part file \\
    |cdocspt4.tex| & sample part file \\
    |cdocsdrf.tex| & sample redirection file \\
    |cdocsfn1.tex| & sample redirection file \\
    |cdocsfn2.tex| & sample redirection file \\
    |childdoc.pdf| & manual
\end{tabular}
\end{center}
%
The distribution consists of the files
|README.txt|, |childdoc.ins| and |childdoc.dtx|.
%
\begin{itemize}
\item
Run (pdf)\LaTeX{} on |childdoc.dtx|
to compile the manual |childdoc.pdf| (this file).
\item
Run \LaTeX{} on |childdoc.ins| to create the definitions file |childdoc.def|
and the sample |cdocsamp.tex| with include files
|cdocsch1.tex|, |cdocsch2.tex|, |cdocspt3.tex|, |cdocspt4.tex|,
|cdocsdrf.tex|, |cdocsfn1.tex|, |cdocsfn2.tex|.
Then copy the file |childdoc.def| to an appropriate directory of your \LaTeX{}
distribution, e.g.\ \textit{texmf-root}|/tex/latex/childdoc|.
\end{itemize}

%%%%%%%%%%%%%%%%%%%%%%%%%%%%%%%%%%%%%%%%%%%%%%%%%%%%%%%%%%%%%%%%%%%%%%%%%%%%%%%%
\subsection{Related CTAN Packages}

There are several other packages which offer a similar functionality:
%
\begin{itemize}
\item
The packages
\href{http://ctan.org/pkg/docmute}{\textsf{docmute}},
\href{http://ctan.org/pkg/includex}{\textsf{includex}} and
\href{http://ctan.org/pkg/standalone}{\textsf{standalone}}
provide commands to include only the document body of
a child file thus allowing both files to be compiled individually.
\item
The packages \href{http://ctan.org/pkg/subdocs}{\textsf{subdocs}}
and \href{http://ctan.org/pkg/subfiles}{\textsf{subfiles}}
provide structures in which the main and child documents can be
encapsulated and allowing them to be compiled individually.
The inclusion mechanism is different from the conventional |\include|.
\item
The package \href{http://ctan.org/pkg/combine}{\textsf{combine}}
is an elaborate solution to combine several documents into one.
\end{itemize}
%
See also the CTAN topic \href{http://ctan.org/topic/subdocs}{\textsf{subdocs}}
for further related packages.
The present package differs from the above solutions in that
a document structure constructed with the conventional |\include| mechanism
just needs two extra commands at the top of every file
such that all constituent files can be compiled individually.

%%%%%%%%%%%%%%%%%%%%%%%%%%%%%%%%%%%%%%%%%%%%%%%%%%%%%%%%%%%%%%%%%%%%%%%%%%%%%%%%
%\subsection{Feature Suggestions}
%
%The following is a list of features which may be useful for future
%versions of this package:
%%
%\begin{itemize}
%\item
%\ldots
%\end{itemize}

%%%%%%%%%%%%%%%%%%%%%%%%%%%%%%%%%%%%%%%%%%%%%%%%%%%%%%%%%%%%%%%%%%%%%%%%%%%%%%%%
\subsection{Revision History}

%%%%%%%%%%%%%%%%%%%%%%%%%%%%%%%%%%%%%%%%
\paragraph{v2.0:} 2018/12/30

\begin{itemize}
\item
immediate forward processing
\item
added |\childdocby| mechanism
\item
manual restructured
\end{itemize}

%%%%%%%%%%%%%%%%%%%%%%%%%%%%%%%%%%%%%%%%
\paragraph{v1.6:} 2018/01/17

\begin{itemize}
\item
application for development of include files
\item
corrections to manual
\end{itemize}

%%%%%%%%%%%%%%%%%%%%%%%%%%%%%%%%%%%%%%%%
\paragraph{v1.5:} 2017/05/21

\begin{itemize}
\item
more complete structuring introduced
\item
|\childdocof| introduced
\item
|\childdoc| renamed to |\childdocmain|
\item
|\childredirect| renamed to |\childdocforward| and |\childdocforwardprefix|
and functionality expanded
\end{itemize}

%%%%%%%%%%%%%%%%%%%%%%%%%%%%%%%%%%%%%%%%
\paragraph{v1.0:} 2017/04/27

\begin{itemize}
\item
manual and install package
\item
first version published on CTAN
\end{itemize}

%%%%%%%%%%%%%%%%%%%%%%%%%%%%%%%%%%%%%%%%
\paragraph{v0.6:} 2017/04/26

\begin{itemize}
\item
redirection mechanism added
\end{itemize}

%%%%%%%%%%%%%%%%%%%%%%%%%%%%%%%%%%%%%%%%
\paragraph{v0.5:} 2017/04/26

\begin{itemize}
\item
functionality in definition file
\end{itemize}


%%%%%%%%%%%%%%%%%%%%%%%%%%%%%%%%%%%%%%%%%%%%%%%%%%%%%%%%%%%%%%%%%%%%%%%%%%%%%%%%
%%%%%%%%%%%%%%%%%%%%%%%%%%%%%%%%%%%%%%%%%%%%%%%%%%%%%%%%%%%%%%%%%%%%%%%%%%%%%%%%
%%%%%%%%%%%%%%%%%%%%%%%%%%%%%%%%%%%%%%%%%%%%%%%%%%%%%%%%%%%%%%%%%%%%%%%%%%%%%%%%
\appendix

\settowidth\MacroIndent{\rmfamily\scriptsize 000\ }

 \DocInput{childdoc.dtx}

\end{document}
%</driver>
% \fi
%
% %%%%%%%%%%%%%%%%%%%%%%%%%%%%%%%%%%%%%%%%%%%%%%%%%%%%%%%%%%%%%%%%%%%%%%%%%%%%%%
% %%%%%%%%%%%%%%%%%%%%%%%%%%%%%%%%%%%%%%%%%%%%%%%%%%%%%%%%%%%%%%%%%%%%%%%%%%%%%%
% \section{Sample}
%\iffalse
%<*samplemain>
%\fi
%
% The following presents a sample document
% with two chapters, two parts, a title page,
% a compile flag as well as three forwarding files to set the flag.
% It consists of eight |.tex| files:
% \begin{center}
% \begin{tabular}{ll}
% |cdocsamp.tex|&main file\\
% |cdocsch1.tex|&include file for chapter 1\\
% |cdocsch2.tex|&include file for chapter 2\\
% |cdocspt3.tex|&include file for part 3\\
% |cdocspt4.tex|&include file for part 4\\
% |cdocsdrf.tex|&forwarding file for main file in draft mode\\
% |cdocsfi1.tex|&forwarding file for final version of chapter 1\\
% |cdocsfi2.tex|&forwarding file for final version of chapter 2\\
% \end{tabular}
% \end{center}
% Each of the eight files can be compiled directly by the \LaTeX{} compiler.
%
% %%%%%%%%%%%%%%%%%%%%%%%%%%%%%%%%%%%%%%
% \paragraph{Main File.}
%
% The main file is called |cdocsamp.tex|.
%
% Load the \textsf{childdoc} definitions and
% declare the filename for the main document:
%    \begin{macrocode}
\input{childdoc.def}
\childdocmain{}
%    \end{macrocode}

% Optional override for |\version| flag:
%    \begin{macrocode}
%%\ifchilddoc\else\providecommand{\version}{draft}\fi
%    \end{macrocode}

% Define the default values for the |\version| flag
% (|final| for the main file and |draft| for childs):
%    \begin{macrocode}
\ifchilddoc
\providecommand{\version}{draft}
\else
\providecommand{\version}{final}
\fi
%    \end{macrocode}

% Load the standard document class:
%    \begin{macrocode}
\documentclass[12pt]{article}
%    \end{macrocode}

% Start the document body:
%    \begin{macrocode}
\begin{document}
%    \end{macrocode}

% Declare a title page.
% Print title, part of document being processed and version flag:
%    \begin{macrocode}
\addtocounter{page}{-1}
\begin{center}
{\LARGE\bfseries{}childdoc example\par}
\vspace{1cm}
\ifchilddoc
\ifchilddocmanual part\else chapter\fi:
`\childdocname' of `\childdocjob'\par
\else
main document: `\childdocjob'\par
\fi
version: \version\par
\end{center}
\newpage
%    \end{macrocode}

% Manually include selected file,
% otherwise process as usual:
%    \begin{macrocode}
\ifchilddocmanual
\section*{part `\childdocname'}
\input{\childdocname}
\else
%    \end{macrocode}

% Include the two chapters:
%    \begin{macrocode}
\include{cdocsch1}
\include{cdocsch2}
%    \end{macrocode}

% Include the two parts unless only chapters should be displayed:
%    \begin{macrocode}
\ifchilddoc\else
\section{part three}
\input{cdocspt3}
\section{part four}
\input{cdocspt4}
\fi
%    \end{macrocode}

% Process as usual until here:
%    \begin{macrocode}
\fi
%    \end{macrocode}

% End of document body:
%    \begin{macrocode}
\end{document}
%    \end{macrocode}
%\iffalse
%</samplemain>
%\fi
%
% %%%%%%%%%%%%%%%%%%%%%%%%%%%%%%%%%%%%%%
% \paragraph{Chapter Include Files.}
%
% The include files are called |cdocsch1.tex| and |cdocsch2.tex|.
%
%\iffalse
%<*samplechap1|samplechap2>
%\fi

% Optional override for |\version| flag:
%    \begin{macrocode}
%%\providecommand{\version}{final}
%    \end{macrocode}

% Include the main document:
%    \begin{macrocode}
\input{childdoc.def}
\childdocof{cdocsamp}
%    \end{macrocode}

%\iffalse
%</samplechap1|samplechap2>
%\fi
%
%\iffalse
%<*samplechap1>
%\fi
% Some text for chapter 1:
%    \begin{macrocode}
\section{one}
some text in chapter one
%    \end{macrocode}

%\iffalse
%</samplechap1>
%\fi
% Some text for chapter 2:
%\iffalse
%<*samplechap2>
%\fi
%    \begin{macrocode}
\section{two}
more text in chapter two
%    \end{macrocode}

%\iffalse
%</samplechap2>
%\fi
%
% %%%%%%%%%%%%%%%%%%%%%%%%%%%%%%%%%%%%%%
% \paragraph{Part Include Files.}
%
% The include files are called |cdocspt3.tex| and |cdocspt4.tex|.
%
%\iffalse
%<*samplepart3|samplepart4>
%\fi

% Optional override for |\version| flag:
%    \begin{macrocode}
%%\providecommand{\version}{final}
%    \end{macrocode}

% Include the main document:
%    \begin{macrocode}
\input{childdoc.def}
\childdocby{cdocsamp}
%    \end{macrocode}

%\iffalse
%</samplepart3|samplepart4>
%\fi
%
%\iffalse
%<*samplepart3>
%\fi
% Some text for part 3:
%    \begin{macrocode}
some text in part three
%    \end{macrocode}

%\iffalse
%</samplepart3>
%\fi
% Some text for part 4:
%\iffalse
%<*samplepart4>
%\fi
%    \begin{macrocode}
more text in part four
%    \end{macrocode}

%\iffalse
%</samplepart4>
%\fi
%
% %%%%%%%%%%%%%%%%%%%%%%%%%%%%%%%%%%%%%%
% \paragraph{Forwarding for a Complete Draft.}
%
% The following forwarding file |cdocsdrf.tex|
% compiles the main document in draft mode:
%\iffalse
%<*sampledraft>
%\fi
%    \begin{macrocode}
\def\version{draft}
\input{childdoc.def}
\childdocforward{cdocsamp}
%    \end{macrocode}

%\iffalse
%</sampledraft>
%\fi
%
% %%%%%%%%%%%%%%%%%%%%%%%%%%%%%%%%%%%%%%
% \paragraph{Forwarding for Final Version of the Chapters.}
%
% The following forwarding files |cdocsfn1.tex| and |cdocsfn2.tex|
% (with identical content)
% compile the final versions of the child documents
% |cdocsch1.tex| and |cdocsch2.tex|, respectively:
%\iffalse
%<*samplefinal>
%\fi
%    \begin{macrocode}
\def\version{final}
\input{childdoc.def}
\childdocforwardprefix[cdocsamp]{cdocsfn}{cdocsch}
%    \end{macrocode}

%\iffalse
%</samplefinal>
%\fi
%
% %%%%%%%%%%%%%%%%%%%%%%%%%%%%%%%%%%%%%%
% \paragraph{Command Line Processing.}
%
% The following three command lines generate the output files
% |cdocscld|, |cdocscl1| and |cdocscl2|
% which should be identical to
% |cdocsdrf|, |cdocsch1| and |cdocsfn2|, respectively:
% \begin{center}
% \begin{tabular}{l}
% |latex -jobname cdocscld \|\\
% |  "\def\version{draft}\input{childdoc.def}\childdocforward{cdocsamp}"|\\
% |latex -jobname cdocscl1 \|\\
% |  "\input{childdoc.def}\childdocforward[cdocsamp]{cdocsch1}"|\\
% |latex -jobname cdocscl2 \|\\
% |  "\def\version{final}\input{childdoc.def}\childdocforward{cdocsch2}"|
% \end{tabular}
% \end{center}
% Note that the trailing backslash on each first line
% merely continues the input to the second line
% (for convenient cut ant paste).
% Furthermore, the command |latex| can be replaced by any
% of its alternative versions such as |pdflatex|.
%
% %%%%%%%%%%%%%%%%%%%%%%%%%%%%%%%%%%%%%%%%%%%%%%%%%%%%%%%%%%%%%%%%%%%%%%%%%%%%%%
% %%%%%%%%%%%%%%%%%%%%%%%%%%%%%%%%%%%%%%%%%%%%%%%%%%%%%%%%%%%%%%%%%%%%%%%%%%%%%%
% \section{Implementation}
%\iffalse
%<*package>
%\fi
%
% This section describes the definitions file |childdoc.def|.

% The definitions cannot be loaded using |\usepackage| or |\RequirePackage|
% which has a mechanism to prevent loading a style file more than once.
% When loading the definitions by means of |\input|
% multiple instances have to be prevented manually:
%\iffalse
%This code needs to be before the `\ProvidesFile' directive
%which is defined at the beginning of this file.
%Therefore it is also placed there and commented out here.
%</package>
%<*discard>
%\fi
%    \begin{macrocode}
\ifdefined\childdocmain\endinput\fi
%    \end{macrocode}
%\iffalse
%</discard>
%<*package>
%\fi
%
% \macro{\ifchilddoc}
% \macro{\ifchilddocmanual}
% The conditional |\ifchilddoc| tells whether a
% child (true) or main (false) document is being compiled.
% The conditional |\ifchilddocmanual| tells whether
% the |\includeonly| mechanism is used (false) or
% the selection of child files must be performed manually (true).
% The definitions initialise to false:
%    \begin{macrocode}
\newif\ifchilddoc
\newif\ifchilddocmanual
%    \end{macrocode}

% \macro{\childdocname}
% \macro{\childdocjob}
% The macro |\childdocname| stores the name of the main document
% to be compiled. The macro |\childdocjob| stores the name of
% the document on which the \LaTeX{} compiler was originally invoked.
% The content of |\jobname| cannot be compared
% to filenames specified in the source due to different catcodes.
% The following code rescans |\jobname|, stores the result
% in |\childdocname| and saves a copy in |\childdocjob|:
%    \begin{macrocode}
\edef\childdocname{\scantokens\expandafter{\jobname\noexpand}}
\let\childdocjob\childdocname
%    \end{macrocode}

% \macro{\childdocdisable}
% The macro |\childdocdisable| prevents the main file
% from being processed more than once.
% At this stage, the main document command |\childdocmain|
% is assumed to be called once again where it should do nothing.
% Any subsequent call to it should prevent
% a secondary processing of the main document
% It overwrites the forwarding commands
% |\childdocof| and |\childdocforward|
% with empty macros to prevent further inclusions of the main document:
%    \begin{macrocode}
\newcommand{\childdocdisable}
{
  \renewcommand{\childdocmain}[1]{\renewcommand{\childdocmain}[1]{\endinput}}
  \renewcommand{\childdocof}[1]{}
  \renewcommand{\childdocby}[2][]{}
  \renewcommand{\childdocforward}[2][]{}
  \renewcommand{\childdocdisable}{}
}
%    \end{macrocode}

% \macro{\childdocmain}
% The macro |\childdocmain| is to be called at the top of the main file
% with nothing or the main filename (without extension) as argument.
% First, it breaks loops.
% If the argument is not empty and does not match |\childdocname|
% (which is set by the first inclusion of |childdoc.def|),
% |\ifchilddoc| is set to true, |\includeonly| is applied to the child file
% and |\jobname| is set to the main file
% (for proper handling of |.aux| files):
%    \begin{macrocode}
\newcommand{\childdocmain}[1]
{
  \childdocdisable\childdocmain{}
  \if?#1?\else
    \begingroup
      \def\childdoctmp{#1}
      \ifx\childdoctmp\childdocname
        \def\childdoctmp{}
      \else
        \def\childdoctmp
        {
          \childdoctrue
          \includeonly{\childdocname}
          \def\childdocjob{#1}
          \def\jobname{#1}
        }
      \fi
      \expandafter
    \endgroup
    \childdoctmp
  \fi
}
%    \end{macrocode}

% \macro{\childdocof}
% The command |\childdocof| redirects
% compilation to the main file |#1|.
%    \begin{macrocode}
\newcommand{\childdocof}[1]
{
  \childdocdisable
  \childdoctrue
  \includeonly{\childdocname}
  \def\jobname{#1}
  \def\childdocjob{#1}
  \input{#1}
}
%    \end{macrocode}

% \macro{\childdocby}
% The command |\childdocby| ....
%    \begin{macrocode}
\newcommand{\childdocby}[2][]
{
  \childdocdisable
  \childdoctrue
  \childdocmanualtrue
  \if?#1?\else
    \def\jobname{#2}
  \fi
  \def\childdocjob{#2}
  \input{#2}
  \endinput
}
%    \end{macrocode}

% \macro{\childdocforward}
% The command |\childdocforward| redirects
% compilation to the main file or
% (if the optional argument is given) a child file.
% Parameters are set as if the main file
% or a child file starting with |\childdocof| was compiled.
% Then compilation is handed over to the main file:
%    \begin{macrocode}
\newcommand{\childdocforward}[2][]
{
  \begingroup
    \if?#1?
      \def\childdoctmp
      {
        \def\childdocname{#2}
        \def\childdocjob{#2}
        \def\jobname{#2}
        \input{#2}
        \endinput
      }
    \else
      \def\childdoctmp
      {
        \childdocdisable
        \def\childdocname{#2}
        \childdoctrue
        \includeonly{#2}
        \def\childdocjob{#1}
        \def\jobname{#1}
        \input{#1}
        \endinput
      }
    \fi
    \expandafter
  \endgroup
  \childdoctmp
}
%    \end{macrocode}

% \macro{\childdocforwardprefix}
% The command |\childdocforwardprefix| redirects
% compilation to the main or a child file by means of a pattern.
% The prefix |#1| in the current filename is replaced by |#2|
% and the suffix of the current filename is kept
% (it is assumed that the filename does not contain the substring `|~~~|'
% which is used as a delimiter).
% Compilation is handed over to the new file by |\childdocforward|:
%    \begin{macrocode}
\newcommand{\childdocforwardprefix}[3][]
{
  \begingroup
    \def\childdocextract #2##1~~~{\def\childdoctmp{\childdocforward[#1]{#3##1}}}
    \expandafter\childdocextract\childdocname~~~
    \expandafter
  \endgroup
  \childdoctmp
}
%    \end{macrocode}

% \macro{\childdoc}
% The deprecated macro |\childdoc| is a legacy version of |\childdocmain|:
%    \begin{macrocode}
\newcommand{\childdoc}{\childdocmain}
%    \end{macrocode}

% \macro{\childdocredirect}
% The deprecated macro |\childdocredirect| is a legacy version
% of |\childdocforward| and |\childdocforwardprefix|:
%    \begin{macrocode}
\newcommand{\childdocredirect}[2][]
{
  \begingroup
    \if?#1?
      \def\childdoctmp{\childdocforward{#2}}
    \else
      \def\childdoctmp{\childdocforwardprefix{#1}{#2}}
    \fi
    \expandafter
  \endgroup
  \childdoctmp
}
%    \end{macrocode}

%\iffalse
%</package>
%\fi
%
\endinput
\childdocforward[|\textit{main}|]{|\textit{dest}|}"|
\end{center}
%
Here \textit{target} is the name of the output file,
\textit{main} is the name of the main file
and \textit{dest} is the name of the main or child file to be processed
(all filenames without extensions).
The optional argument \textit{main} can be omitted
if \textit{main} matches \textit{dest}.
Optionally, compilation \textit{flags} can be defined via |\def| commands.
This command line makes the \TeX{} engine believe
it is compiling the file \textit{target}
whose content is specified as the latter parameter.
The provided code then forwards the processing to
\textit{main} or \textit{dest} as described in \secref{sec:forward}.

%%%%%%%%%%%%%%%%%%%%%%%%%%%%%%%%%%%%%%%%%%%%%%%%%%%%%%%%%%%%%%%%%%%%%%%%%%%%%%%%
\subsection{Include by Input}
\label{sec:input}

Including child documents by |\include| has some restrictions by design.
Most notably, the content of a child document always occupies
its own set of pages; pages cannot be shared between child documents.
Usually, this behaviour makes perfect sense
because each child document contain an essential part of the document.
However, in some situations it may be desirable to compose
a document from a collection of parts
without having mandatory page breaks between then.
For this case, the package
provides a mechanism to include parts
by |\input| which can also be processed individually.
However, by construction this mechanism
requires manual handling of the content to be output.

%%%%%%%%%%%%%%%%%%%%%%%%%%%%%%%%%%%%%%%%
\DescribeMacro{\ifchilddocmanual}
The main file should be prepared as usual, see \secref{sec:include}.
However, the document body must make a distinction
between processing of an individual part and of the main document, e.g.:
%
\begin{center}
\begin{tabular}{l}
|\ifchilddocmanual|\\
|\input{\childdocname}|\\
|\||else|\\
\textit{document body with }|\input{|\textit{part}|}|\\
|\||fi|
\end{tabular}
\end{center}
%
The conditional |\ifchilddocmanual| is true whenever
a part to be included by |\input| is being compiled,
and the name of the part is stored in |\childdocname|.

%%%%%%%%%%%%%%%%%%%%%%%%%%%%%%%%%%%%%%%%
\DescribeMacro{\childdocby}
Each part to be included by |\input| should start with:
%
\begin{center}
\begin{tabular}{l}
|% \iffalse
%
% childdoc.dtx Copyright (C) 2017-2018 Niklas Beisert
%
% This work may be distributed and/or modified under the
% conditions of the LaTeX Project Public License, either version 1.3
% of this license or (at your option) any later version.
% The latest version of this license is in
%   http://www.latex-project.org/lppl.txt
% and version 1.3 or later is part of all distributions of LaTeX
% version 2005/12/01 or later.
%
% This work has the LPPL maintenance status `maintained'.
%
% The Current Maintainer of this work is Niklas Beisert.
%
% This work consists of the files childdoc.dtx and childdoc.ins
% and the derived files childdoc.def and cdocsamp.tex with
% cdocsch1.tex, cdocsch2.tex, cdocsdrf.tex, cdocsfn1.tex, cdocsfn2.tex.
%
%<package>\ifdefined\childdocmain\endinput\fi
%<package>\ProvidesFile{childdoc.def}[2018/12/30 v2.0 child document driver]
%<samplemain>\ProvidesFile{cdocsamp.tex}[2018/12/30 v2.0 sample for childdoc]
%<*driver>
%\ProvidesFile{childdoc.drv}[2018/12/30 v2.0 childdoc reference manual file]
\PassOptionsToClass{10pt,a4paper}{article}
\documentclass{ltxdoc}

\usepackage[margin=35mm]{geometry}
\usepackage{hyperref}
\usepackage{hyperxmp}
\usepackage[usenames]{color}

\hypersetup{colorlinks=true}
\hypersetup{pdfstartview=FitH}
\hypersetup{pdfpagemode=UseNone}
\hypersetup{pdfsource={}}
\hypersetup{pdflang={en-UK}}
\hypersetup{pdfcopyright={Copyright 2017-2018 Niklas Beisert.
  This work may be distributed and/or modified under the
  conditions of the LaTeX Project Public License, either version 1.3
  of this license or (at your option) any later version.}}
\hypersetup{pdflicenseurl={http://www.latex-project.org/lppl.txt}}
\hypersetup{pdfcontactaddress={ETH Zurich, ITP, HIT K,
  Wolfgang-Pauli-Strasse 27}}
\hypersetup{pdfcontactpostcode={8093}}
\hypersetup{pdfcontactcity={Zurich}}
\hypersetup{pdfcontactcountry={Switzerland}}
\hypersetup{pdfcontactemail={nbeisert@itp.phys.ethz.ch}}
\hypersetup{pdfcontacturl={http://people.phys.ethz.ch/\xmptilde nbeisert/}}

\newcommand{\secref}[1]{\hyperref[#1]{section \ref*{#1}}}

\parskip1ex
\parindent0pt
\let\olditemize\itemize
\def\itemize{\olditemize\parskip0pt}

\begin{document}

\title{The \textsf{childdoc} Package}
\hypersetup{pdftitle={The childdoc Package}}
\author{Niklas Beisert\\[2ex]
  Institut f\"ur Theoretische Physik\\
  Eidgen\"ossische Technische Hochschule Z\"urich\\
  Wolfgang-Pauli-Strasse 27, 8093 Z\"urich, Switzerland\\[1ex]
  \href{mailto:nbeisert@itp.phys.ethz.ch}
  {\texttt{nbeisert@itp.phys.ethz.ch}}}
\hypersetup{pdfauthor={Niklas Beisert}}
\hypersetup{pdfsubject={Manual for the LaTeX2e Package childdoc}}
\date{30 December 2018, \textsf{v2.0}}
\maketitle

\begin{abstract}\noindent
\textsf{childdoc} is a \LaTeXe{} package
that enables the direct compilation
of document sections included by |\include|
to individual files.
\end{abstract}

\begingroup
\parskip0ex
\tableofcontents
\endgroup

%%%%%%%%%%%%%%%%%%%%%%%%%%%%%%%%%%%%%%%%%%%%%%%%%%%%%%%%%%%%%%%%%%%%%%%%%%%%%%%%
%%%%%%%%%%%%%%%%%%%%%%%%%%%%%%%%%%%%%%%%%%%%%%%%%%%%%%%%%%%%%%%%%%%%%%%%%%%%%%%%
\section{Introduction}

\LaTeX{} provides a mechanism to structure a large document (such as a book)
into a main file and several child files (containing the chapters)
using the |\include| command.
This mechanism is beneficial for documents
which span hundreds of pages in order to
make the source file(s) more manageable.
Moreover, compilation can be restricted to
selected child files by means of the |\includeonly| command.
The latter feature can be used to reduce the compilation time while editing
(this was significantly more useful in the earlier days of \LaTeX{})
or to generate a smaller document which is easier to navigate.
Another application of |\includeonly| is to generate
documents consisting of selected parts of the complete document.

However, there are a few drawbacks of the plain |\include| mechanism:
\begin{itemize}
\item
The child files cannot be compiled on their own,
they can only be compiled via the main file.
A naive editing environment
(such as a text editor with an option
to have the current file processed by \LaTeX)
may require one to switch to the main file before compiling;
attempting to compile the child file produces errors.
\item
The main file must be modified (each time)
to adjust the |\includeonly| command
to the present needs. This easily leaves the main file in a messy state.
\item
The generated document will always carry the filename
of the main document. This is inconvenient if
several child files are to be compiled and
to be kept for distribution.
\end{itemize}

The present package provides a simple interface
to make child files individually compilable by \LaTeX{}.
Compiling a child file then has the same effect as compiling
the main file with an |\includeonly| command
to select the appropriate child.
Moreover the generated document will carry the name of the child
rather than the main file.
This resolves all three above issues.

This feature is meant to make the editing of books,
thesis documents and lecture notes somewhat more convenient.
However, the package can also be used efficiently for
composing a series of documents (such as exercise sheets)
which are typically distributed individually.
It then assists the author in generating the individual documents
(potentially in different versions)
as well as a document containing the collected series.
Another application is in developing style files
or other kinds of included material
where compilation of the style file could redirect
to a sample or test file.

%%%%%%%%%%%%%%%%%%%%%%%%%%%%%%%%%%%%%%%%%%%%%%%%%%%%%%%%%%%%%%%%%%%%%%%%%%%%%%%%
%%%%%%%%%%%%%%%%%%%%%%%%%%%%%%%%%%%%%%%%%%%%%%%%%%%%%%%%%%%%%%%%%%%%%%%%%%%%%%%%
\section{Usage}

First of all, the package \textsf{childdoc} is \emph{not} a standard
\LaTeXe{} |.sty| style file! Therefore it needs to be invoked in
a non-standard way.

%%%%%%%%%%%%%%%%%%%%%%%%%%%%%%%%%%%%%%%%%%%%%%%%%%%%%%%%%%%%%%%%%%%%%%%%%%%%%%%%
\subsection{Included Files}
\label{sec:include}

%%%%%%%%%%%%%%%%%%%%%%%%%%%%%%%%%%%%%%%%
\DescribeMacro{\childdocmain}
To use the package, add the commands
\begin{center}
\begin{tabular}{l}
|\input{childdoc.def}|\\
|\childdocmain{}|\\
\end{tabular}
\end{center}
at the very top of the main \LaTeX{} file,
in particular \emph{before} the |\documentclass| statement!
The argument of |\childdocmain| should be left empty
(but it must be present).

%%%%%%%%%%%%%%%%%%%%%%%%%%%%%%%%%%%%%%%%
\DescribeMacro{\childdocof}
Furthermore, add the commands
\begin{center}
\begin{tabular}{l}
|\input{childdoc.def}|\\
|\childdocof{|\textit{main}|}|\\
\end{tabular}
\end{center}
at the top of every child file \textit{child}
which is included by |\include{|\textit{child}|}|
from within the main file
(or at least for those files to be compiled individually).
The argument \textit{main} must be the filename of the main file.

There are a couple of
considerations in setting up the main and child documents:

%%%%%%%%%%%%%%%%%%%%%%%%%%%%%%%%%%%%%%%%
\paragraph{Restrictions.}

Please note the following restrictions:
\begin{itemize}
\item
|\childdocmain| must be called with one argument \textit{main}
to ensure compatibility with earlier version of the package.
It must either be empty (|\childdocmain{}|)
or precisely match the filename of the main file in which it is specified.
See \secref{sec:detection} for further information.
\item
The filename \textit{main} must be specified without the |.tex| extension.
\item
The filename \textit{main} is case sensitive
(even in case-insensitive file systems)
due to internal string comparison.
\item
The argument \textit{main} should be fully expanded, it cannot be a macro.
\item
Subdirectories and special characters should be avoided in filenames.
\item
The command |\childdocmain{|\textit{main}|}| must be followed by a whitespace.
It should not be followed immediately by another command
or by a comment mark `|%|'.
This is because the \TeX{} parser reads the token immediately following
the argument of |\childdocmain| and puts it
at the beginning of every child section;
however, a white\-space is ignored.
\end{itemize}

%%%%%%%%%%%%%%%%%%%%%%%%%%%%%%%%%%%%%%%%
\paragraph{Content of Main File.}

It is advisable to place all content in the child files included by |\include|.
Any output contained in the main file will appear in all child documents
unless suppressed manually;
it cannot be suppressed automatically by the |\includeonly| directive
and thus should normally be avoided.
A method to include some content in the main file
by means of conditional processing is described in \secref{sec:conditional}.

%%%%%%%%%%%%%%%%%%%%%%%%%%%%%%%%%%%%%%%%
\paragraph{Page Numbering.}

When only a part of the document is compiled,
the appropriate numbering of pages
(as well as other status parameters)
is determined from the |.aux| files.
The latter contain information from previous passes.
However this information needs to propagate through
all intermediate child documents.
Therefore the page numbering in child documents may well
be inconsistent until the complete document is compiled at least once.

A useful (if unconventional) way to always ensure a consistent
page numbering is to restart the numbering in each child document
and denote the pages by `\textit{child}|.|\textit{page}'
where \textit{child} represents the chapter/section number of the child file.
This can be achieved by the command
|\numberwithin{page}{|\textit{child}|}|
of the \textsf{amsmath} package
where \textit{child} can be |chapter| or |section|
depending on the chosen structuring.
Alternatively, one can modify the macro |\thepage| appropriately
and reset the counter |page| at the start of each child file.

%%%%%%%%%%%%%%%%%%%%%%%%%%%%%%%%%%%%%%%%%%%%%%%%%%%%%%%%%%%%%%%%%%%%%%%%%%%%%%%%
\subsection{Conditional Processing}
\label{sec:conditional}

The package provides a mechanism to compile different versions
of a document. To customise the versions further some conditional processing
can come in handy to distinguish which version is being compiled.
The package provides two macros to describe the compilation context:

%%%%%%%%%%%%%%%%%%%%%%%%%%%%%%%%%%%%%%%%
\DescribeMacro{\ifchilddoc}
The conditional |\ifchilddoc| distinguishes between the compilation of
child documents and the main document:
%
\begin{center}
|\ifchilddoc |\textit{child-code}| |[|\||else |\textit{main-code}]| \||fi|
\end{center}

%%%%%%%%%%%%%%%%%%%%%%%%%%%%%%%%%%%%%%%%
\DescribeMacro{\childdocname}
\DescribeMacro{\childdocjob}
The macro |\childdocname| contains the filename (without extension)
of the main or child file being processed.
Note that |\childdocjob| will always contain the name of the main file.

%%%%%%%%%%%%%%%%%%%%%%%%%%%%%%%%%%%%%%%%
\paragraph{Title Page.}

Conditional processing can be used to include a title or banner page
in the main document when proper precautions are taken.
Importantly, the code in the main file should ensure that the page counter
(as well as other status parameters which are stored in the |.aux| files)
takes the same value after the conditional processing.
Otherwise the page numbers may take divergent values
depending on which part is compiled.

For example, a title page could be declared by:
%
\begin{center}
\begin{tabular}{l}
|\ifchilddoc\||else|\\
|\addtocounter{page}{-1}|\\
\textit{code for title page}\\
|\newpage|\\
|\||fi|
\end{tabular}
\end{center}
%
A banner page for the child documents can be generated by:
%
\begin{center}
\begin{tabular}{l}
|\ifchilddoc|\\
|\addtocounter{page}{-1}|\\
\textit{code for banner page}\\
|\newpage|\\
|\||fi|
\end{tabular}
\end{center}
%
Here one could write a message such as:
\begin{center}
|This is the part \childdocname{} of \childdocjob{}.|
\end{center}

%%%%%%%%%%%%%%%%%%%%%%%%%%%%%%%%%%%%%%%%%%%%%%%%%%%%%%%%%%%%%%%%%%%%%%%%%%%%%%%%
\subsection{Flags}
\label{sec:flags}

The package makes it easy to generate different versions
of the main or child documents.
To this end compilation flags can be defined
and assigned different default values.
They will be particularly useful in conjunction
with the forwarding mechanism described in \secref{sec:forward}.

For example, it may be useful to have a flag |\version|
which can be set to |draft| or |final|.
The document source will contain some conditional code
depending on the value of |\version|.
Suppose further, the flag should default to |final| for the main file
and to |draft| for child files
which is a natural assignment for editing the document.
This is achieved by placing the following code
in the preamble of the main document
(below the |\childdocmain| directive):
%
\begin{center}
\begin{tabular}{l}
|\ifchilddoc|\\
|\providecommand{\version}{draft}|\\
|\||else|\\
|\providecommand{\version}{final}|\\
|\||fi|
\end{tabular}
\end{center}
%
The definition by |\providecommand| makes sure
that previous definitions are not overwritten.
Further statements |\providecommand{\version}{...}|
can thus be added before the above code to override it.

For the main file, one might add a line
(between |\childdocmain| and the above block)
%
\begin{center}
|%\ifchilddoc\||else\providecommand{\version}{draft}\||fi|
\end{center}
%
which can be uncommented to produce a draft version.
Likewise one can add a line to the very top of a child file
(above the |\childdocof{|\textit{main}|}| directive)
%
\begin{center}
|%\providecommand{\version}{final}|
\end{center}
%
which can be uncommented to produce the final version of this child document.

%%%%%%%%%%%%%%%%%%%%%%%%%%%%%%%%%%%%%%%%%%%%%%%%%%%%%%%%%%%%%%%%%%%%%%%%%%%%%%%%
\subsection{Forwarding}
\label{sec:forward}

Different versions of the main or child documents
using compilation flags as described in \secref{sec:flags}
can be (permanently) stored in different files
for convenient compilation, viewing and distribution.
To this end, the package defines a command
to pass on compilation to a different file:

%%%%%%%%%%%%%%%%%%%%%%%%%%%%%%%%%%%%%%%%
\DescribeMacro{\childdocforward}
The command |\childdocforward| redirects processing to
another source file:
%
\begin{center}
\begin{tabular}{l}
|\input{childdoc.def}|\\
|\childdocforward[|\textit{main}|]{|\textit{dest}|}|\\
\end{tabular}
\end{center}
%
The argument \textit{dest} is the destination file
(without extension).
It should be the main file or one of the child files.
Note that further \textsf{childdoc} directives
such as |\childdocof| and |\childdocforward|
in the indicated file will be processed in this form.
The optional argument \textit{main}
passes on directly to the main file \textit{main}
while pretending to compile the child \textit{dest}.
This form behaves as if \textit{dest}
issues |\childdocof{|\textit{main}|}| right away,
and no further \textsf{childdoc} directives will be processed.

%%%%%%%%%%%%%%%%%%%%%%%%%%%%%%%%%%%%%%%%
\DescribeMacro{\...prefix}
In the alternative form |\childdocforwardprefix|,
%
\begin{center}
\begin{tabular}{l}
|\input{childdoc.def}|\\
|\childdocforwardprefix[|\textit{main}|]{|\textit{prefix}|}{|\textit{dest}|}|
\end{tabular}
\end{center}
%
the destination file is determined by a pattern
depending on the current file:
To make this work, the current file must be called
`{\textit{prefix}\hspace{0.2em}\textit{suffix}}'
with \textit{prefix} matching precisely the argument.
Processing is then passed on to the file
`{\textit{dest}\hspace{0.2em}\textit{suffix}}'.
Surely, the same effect is achieved by
directly specifying the
argument `{\textit{dest}\hspace{0.2em}\textit{suffix}}'
in the first form.
However, that requires to set up a different file
for each child. With the alternative form of the command
all these files can have exactly the same content
which simplifies setting them up and maintaining them.

For example, the following file |draft.tex|
with a compilation flag |\version| as described in \secref{sec:flags}
compiles the main document as a draft:
%
\begin{center}
\begin{tabular}{l}
|\def\version{draft}|\\
|\input{childdoc.def}|\\
|\childdocforward{|\textit{main}|}|
\end{tabular}
\end{center}
%
Likewise, the following files |final|\textit{nn}|.tex|
compile the final version of the child document
|child|\textit{nn}|.tex|:
%
\begin{center}
\begin{tabular}{l}
|\def\version{final}|\\
|\input{childdoc.def}|\\
|\childdocforwardprefix{final}{child}|
\end{tabular}
\end{center}
%

Note that when several versions of a main file and/or of each child file
are to be generated, it may be convenient to set up a |Makefile| or
shell script to automatise the process.

%%%%%%%%%%%%%%%%%%%%%%%%%%%%%%%%%%%%%%%%%%%%%%%%%%%%%%%%%%%%%%%%%%%%%%%%%%%%%%%%
\subsection{Command Line Processing}
\label{sec:commandline}

The effect of redirection files can also be achieved by invoking
the \LaTeX{} compiler with a more elaborate command line.
Most conveniently this should be done as part
of a shell script or a |Makefile|.

When using \textsf{childdoc} in the main file, the following
command lines effectively perform a redirection
(note that depending on the shell being used,
backslashes may have to be doubled: `|\|' $\to$ `|\\|'):
%
\begin{center}
|... -jobname "|\textit{target}|" |\\|"|[\textit{flags}]%
|\input{childdoc.def}\childdocforward[|\textit{main}|]{|\textit{dest}|}"|
\end{center}
%
Here \textit{target} is the name of the output file,
\textit{main} is the name of the main file
and \textit{dest} is the name of the main or child file to be processed
(all filenames without extensions).
The optional argument \textit{main} can be omitted
if \textit{main} matches \textit{dest}.
Optionally, compilation \textit{flags} can be defined via |\def| commands.
This command line makes the \TeX{} engine believe
it is compiling the file \textit{target}
whose content is specified as the latter parameter.
The provided code then forwards the processing to
\textit{main} or \textit{dest} as described in \secref{sec:forward}.

%%%%%%%%%%%%%%%%%%%%%%%%%%%%%%%%%%%%%%%%%%%%%%%%%%%%%%%%%%%%%%%%%%%%%%%%%%%%%%%%
\subsection{Include by Input}
\label{sec:input}

Including child documents by |\include| has some restrictions by design.
Most notably, the content of a child document always occupies
its own set of pages; pages cannot be shared between child documents.
Usually, this behaviour makes perfect sense
because each child document contain an essential part of the document.
However, in some situations it may be desirable to compose
a document from a collection of parts
without having mandatory page breaks between then.
For this case, the package
provides a mechanism to include parts
by |\input| which can also be processed individually.
However, by construction this mechanism
requires manual handling of the content to be output.

%%%%%%%%%%%%%%%%%%%%%%%%%%%%%%%%%%%%%%%%
\DescribeMacro{\ifchilddocmanual}
The main file should be prepared as usual, see \secref{sec:include}.
However, the document body must make a distinction
between processing of an individual part and of the main document, e.g.:
%
\begin{center}
\begin{tabular}{l}
|\ifchilddocmanual|\\
|\input{\childdocname}|\\
|\||else|\\
\textit{document body with }|\input{|\textit{part}|}|\\
|\||fi|
\end{tabular}
\end{center}
%
The conditional |\ifchilddocmanual| is true whenever
a part to be included by |\input| is being compiled,
and the name of the part is stored in |\childdocname|.

%%%%%%%%%%%%%%%%%%%%%%%%%%%%%%%%%%%%%%%%
\DescribeMacro{\childdocby}
Each part to be included by |\input| should start with:
%
\begin{center}
\begin{tabular}{l}
|\input{childdoc.def}|\\
|\childdocby{|\textit{main}|}|\\
\end{tabular}
\end{center}
%
The directive |\childdocby| is similar to |\childdocof|
described in \secref{sec:include},
but the subsequent selection of content must be done manually.
To that end, both |\ifchilddoc| and |\ifchilddocmanual|
will be true upon processing of a part,
and the name of the part is stored in |\childdocname|.
Note that |\jobname| will be set to the filename of the current part
so that each part receives an individual |.aux| file
that does not interfere with the |.aux| file(s) of the main document.
This behaviour can be altered by the alternative form
|\childdocby[*]{|\textit{main}|}| (with a non-empty optional argument)
which uses the |.aux| file of the main document
by setting |\jobname| to \textit{main}.

%%%%%%%%%%%%%%%%%%%%%%%%%%%%%%%%%%%%%%%%%%%%%%%%%%%%%%%%%%%%%%%%%%%%%%%%%%%%%%%%
\subsection{Driver Development}
\label{sec:driver}

The \textsf{childdoc} mechanism can also be use for the development
of definition files such as \LaTeX{} styles or classes.
This case differs from the above setup with multiple parts
included by |\include| in that no |\includeonly| should be invoked.
This can be achieved by starting the include file
(before |\ProvidesPackage|) with:
%
\begin{center}
\begin{tabular}{l}
|\input{childdoc.def}|\\
|\childdocforward{|\textit{main}|}|\\
\end{tabular}
\end{center}
%
or alternatively with:
%
\begin{center}
\begin{tabular}{l}
|\input{childdoc.def}|\\
|\childdocby{|\textit{main}|}|\\
\end{tabular}
\end{center}
%
Both forms have slightly different effects as described above.
The main file is prepared as usual, see \secref{sec:include}.

%%%%%%%%%%%%%%%%%%%%%%%%%%%%%%%%%%%%%%%%%%%%%%%%%%%%%%%%%%%%%%%%%%%%%%%%%%%%%%%%
\subsection{Legacy Detection}
\label{sec:detection}

The directive |\childdocmain| in the main file can detect
whether the complete document or merely a child is to be compiled
even without using the directive |\childdocof|.
This method is deprecated because it is less robust
and there is no compelling reason to use it;
it is merely provided for backward compatibility
and it may be removed in future versions.

If the detection mechanism is to be used,
it is mandatory to correctly specify
the filename of the main file as the argument of |\childdocmain|:
%
\begin{center}
\begin{tabular}{l}
|\input{childdoc.def}|\\
|\childdocmain{|\textit{main}|}|\\
\end{tabular}
\end{center}
%
If |\jobname| does not match the argument \textit{main} of |\childdocmain|,
it is assumed that |\jobname| points to the child file to be compiled.
When using |\childdocmain| with the main file specified as argument,
it suffices to start a child file
with just |\input{|\textit{main}|}|
without loading of the package and using |\childdocof|.
If instead all processing is done
with the appropriate \textsf{childdoc} directives,
the argument of \textit{main} of |\childdocmain| can be empty.

An alternative version of the command line processing described
in \secref{sec:commandline} using the detection mechanism reads:
%
\begin{center}
|... -jobname "|\textit{target}|" "|[\textit{flags}]%
[|\def\jobname{|\textit{dest}|}|]|\input{|\textit{main}|}"|
\end{center}

%%%%%%%%%%%%%%%%%%%%%%%%%%%%%%%%%%%%%%%%%%%%%%%%%%%%%%%%%%%%%%%%%%%%%%%%%%%%%%%%
\subsection{Manual Code}
\label{sec:manual}

In case one cannot be certain whether the definitions file |childdoc.def|
is installed on the target \TeX{} distribution
and one prefers not to ship it,
it is conceivable to paste a few relevant commands into the sources.

To that end, drop all statements |\input{childdoc.def}|
and perform the replacements as outlined below.
Instead of |\childdocmain{|\textit{main}|}| add the following code
to the top of the main file:
%
\begin{center}
\begin{tabular}{l}
|\||ifdefined\childdocname\endinput\||fi\newif\ifchilddoc|\\
|\edef\childdocname{\scantokens\expandafter{\jobname\noexpand}}|\\
|\def\childdocmain{|\textit{main}|}\||ifx\childdocmain\childdocname\||else|\\
|\childdoctrue\includeonly{\childdocname}\let\jobname\childdocmain\||fi|\\
\end{tabular}
\end{center}
%
Instead of |\childdocof{|\textit{main}|}| just include the main file
at the top of each child file:
%
\begin{center}
|\input{|\textit{main}|}|
\end{center}
%
A simple redirection |\childdocforward{|\textit{dest}|}| is achieved by:
%
\begin{center}
|\def\jobname{|\textit{dest}|}\input{\jobname}|
\end{center}
%
The redirection with prefix
|\childdocforwardprefix[|\textit{prefix}|]{|\textit{dest}|}|
is accomplished by:
%
\begin{center}
\begin{tabular}{l}
|{\edef\jobname{\scantokens\expandafter{\jobname\noexpand}}|\\
|\def\redirectjob |\textit{prefix}|#1~~~{\gdef\jobname{|\textit{dest}|#1}}|\\
|\expandafter\redirectjob\jobname~~~}\input{\jobname}|
\end{tabular}
\end{center}

In an alternative approach,
child documents can be compiled by a specific command line
without additional code or specific definitions:
%
\begin{center}
|... -jobname "|\textit{target}|" "|[\textit{flags}]%
|\includeonly{|\textit{dest}|}\input{|\textit{main}|}"|
\end{center}
%

%%%%%%%%%%%%%%%%%%%%%%%%%%%%%%%%%%%%%%%%%%%%%%%%%%%%%%%%%%%%%%%%%%%%%%%%%%%%%%%%
%%%%%%%%%%%%%%%%%%%%%%%%%%%%%%%%%%%%%%%%%%%%%%%%%%%%%%%%%%%%%%%%%%%%%%%%%%%%%%%%
\section{Information}

%%%%%%%%%%%%%%%%%%%%%%%%%%%%%%%%%%%%%%%%%%%%%%%%%%%%%%%%%%%%%%%%%%%%%%%%%%%%%%%%
\subsection{Copyright}

Copyright \copyright{} 2017--2018 Niklas Beisert

This work may be distributed and/or modified under the
conditions of the \LaTeX{} Project Public License, either version 1.3
of this license or (at your option) any later version.
The latest version of this license is in
  \url{http://www.latex-project.org/lppl.txt}
and version 1.3 or later is part of all distributions of \LaTeX{}
version 2005/12/01 or later.

This work has the LPPL maintenance status `maintained'.

The Current Maintainer of this work is Niklas Beisert.

This work consists of the files |README.txt|, |childdoc.ins| and |childdoc.dtx|
as well as the derived files |childdoc.def|, |cdocsamp.tex|
with |cdocsch1.tex|, |cdocsch2.tex|, |cdocspt3.tex|, |cdocspt4.tex|,
|cdocsdrf.tex|, |cdocsfn1.tex|, |cdocsfn2.tex|
as well as |childdoc.pdf|.

%%%%%%%%%%%%%%%%%%%%%%%%%%%%%%%%%%%%%%%%%%%%%%%%%%%%%%%%%%%%%%%%%%%%%%%%%%%%%%%%
\subsection{Files and Installation}

The package consists of the files:
%
\begin{center}
\begin{tabular}{ll}
    |README.txt|   & readme file \\
    |childdoc.ins| & installation file \\
    |childdoc.dtx| & source file \\
    |childdoc.def| & definition file \\
    |cdocsamp.tex| & sample main file \\
    |cdocsch1.tex| & sample include file \\
    |cdocsch2.tex| & sample include file \\
    |cdocspt3.tex| & sample part file \\
    |cdocspt4.tex| & sample part file \\
    |cdocsdrf.tex| & sample redirection file \\
    |cdocsfn1.tex| & sample redirection file \\
    |cdocsfn2.tex| & sample redirection file \\
    |childdoc.pdf| & manual
\end{tabular}
\end{center}
%
The distribution consists of the files
|README.txt|, |childdoc.ins| and |childdoc.dtx|.
%
\begin{itemize}
\item
Run (pdf)\LaTeX{} on |childdoc.dtx|
to compile the manual |childdoc.pdf| (this file).
\item
Run \LaTeX{} on |childdoc.ins| to create the definitions file |childdoc.def|
and the sample |cdocsamp.tex| with include files
|cdocsch1.tex|, |cdocsch2.tex|, |cdocspt3.tex|, |cdocspt4.tex|,
|cdocsdrf.tex|, |cdocsfn1.tex|, |cdocsfn2.tex|.
Then copy the file |childdoc.def| to an appropriate directory of your \LaTeX{}
distribution, e.g.\ \textit{texmf-root}|/tex/latex/childdoc|.
\end{itemize}

%%%%%%%%%%%%%%%%%%%%%%%%%%%%%%%%%%%%%%%%%%%%%%%%%%%%%%%%%%%%%%%%%%%%%%%%%%%%%%%%
\subsection{Related CTAN Packages}

There are several other packages which offer a similar functionality:
%
\begin{itemize}
\item
The packages
\href{http://ctan.org/pkg/docmute}{\textsf{docmute}},
\href{http://ctan.org/pkg/includex}{\textsf{includex}} and
\href{http://ctan.org/pkg/standalone}{\textsf{standalone}}
provide commands to include only the document body of
a child file thus allowing both files to be compiled individually.
\item
The packages \href{http://ctan.org/pkg/subdocs}{\textsf{subdocs}}
and \href{http://ctan.org/pkg/subfiles}{\textsf{subfiles}}
provide structures in which the main and child documents can be
encapsulated and allowing them to be compiled individually.
The inclusion mechanism is different from the conventional |\include|.
\item
The package \href{http://ctan.org/pkg/combine}{\textsf{combine}}
is an elaborate solution to combine several documents into one.
\end{itemize}
%
See also the CTAN topic \href{http://ctan.org/topic/subdocs}{\textsf{subdocs}}
for further related packages.
The present package differs from the above solutions in that
a document structure constructed with the conventional |\include| mechanism
just needs two extra commands at the top of every file
such that all constituent files can be compiled individually.

%%%%%%%%%%%%%%%%%%%%%%%%%%%%%%%%%%%%%%%%%%%%%%%%%%%%%%%%%%%%%%%%%%%%%%%%%%%%%%%%
%\subsection{Feature Suggestions}
%
%The following is a list of features which may be useful for future
%versions of this package:
%%
%\begin{itemize}
%\item
%\ldots
%\end{itemize}

%%%%%%%%%%%%%%%%%%%%%%%%%%%%%%%%%%%%%%%%%%%%%%%%%%%%%%%%%%%%%%%%%%%%%%%%%%%%%%%%
\subsection{Revision History}

%%%%%%%%%%%%%%%%%%%%%%%%%%%%%%%%%%%%%%%%
\paragraph{v2.0:} 2018/12/30

\begin{itemize}
\item
immediate forward processing
\item
added |\childdocby| mechanism
\item
manual restructured
\end{itemize}

%%%%%%%%%%%%%%%%%%%%%%%%%%%%%%%%%%%%%%%%
\paragraph{v1.6:} 2018/01/17

\begin{itemize}
\item
application for development of include files
\item
corrections to manual
\end{itemize}

%%%%%%%%%%%%%%%%%%%%%%%%%%%%%%%%%%%%%%%%
\paragraph{v1.5:} 2017/05/21

\begin{itemize}
\item
more complete structuring introduced
\item
|\childdocof| introduced
\item
|\childdoc| renamed to |\childdocmain|
\item
|\childredirect| renamed to |\childdocforward| and |\childdocforwardprefix|
and functionality expanded
\end{itemize}

%%%%%%%%%%%%%%%%%%%%%%%%%%%%%%%%%%%%%%%%
\paragraph{v1.0:} 2017/04/27

\begin{itemize}
\item
manual and install package
\item
first version published on CTAN
\end{itemize}

%%%%%%%%%%%%%%%%%%%%%%%%%%%%%%%%%%%%%%%%
\paragraph{v0.6:} 2017/04/26

\begin{itemize}
\item
redirection mechanism added
\end{itemize}

%%%%%%%%%%%%%%%%%%%%%%%%%%%%%%%%%%%%%%%%
\paragraph{v0.5:} 2017/04/26

\begin{itemize}
\item
functionality in definition file
\end{itemize}


%%%%%%%%%%%%%%%%%%%%%%%%%%%%%%%%%%%%%%%%%%%%%%%%%%%%%%%%%%%%%%%%%%%%%%%%%%%%%%%%
%%%%%%%%%%%%%%%%%%%%%%%%%%%%%%%%%%%%%%%%%%%%%%%%%%%%%%%%%%%%%%%%%%%%%%%%%%%%%%%%
%%%%%%%%%%%%%%%%%%%%%%%%%%%%%%%%%%%%%%%%%%%%%%%%%%%%%%%%%%%%%%%%%%%%%%%%%%%%%%%%
\appendix

\settowidth\MacroIndent{\rmfamily\scriptsize 000\ }

 \DocInput{childdoc.dtx}

\end{document}
%</driver>
% \fi
%
% %%%%%%%%%%%%%%%%%%%%%%%%%%%%%%%%%%%%%%%%%%%%%%%%%%%%%%%%%%%%%%%%%%%%%%%%%%%%%%
% %%%%%%%%%%%%%%%%%%%%%%%%%%%%%%%%%%%%%%%%%%%%%%%%%%%%%%%%%%%%%%%%%%%%%%%%%%%%%%
% \section{Sample}
%\iffalse
%<*samplemain>
%\fi
%
% The following presents a sample document
% with two chapters, two parts, a title page,
% a compile flag as well as three forwarding files to set the flag.
% It consists of eight |.tex| files:
% \begin{center}
% \begin{tabular}{ll}
% |cdocsamp.tex|&main file\\
% |cdocsch1.tex|&include file for chapter 1\\
% |cdocsch2.tex|&include file for chapter 2\\
% |cdocspt3.tex|&include file for part 3\\
% |cdocspt4.tex|&include file for part 4\\
% |cdocsdrf.tex|&forwarding file for main file in draft mode\\
% |cdocsfi1.tex|&forwarding file for final version of chapter 1\\
% |cdocsfi2.tex|&forwarding file for final version of chapter 2\\
% \end{tabular}
% \end{center}
% Each of the eight files can be compiled directly by the \LaTeX{} compiler.
%
% %%%%%%%%%%%%%%%%%%%%%%%%%%%%%%%%%%%%%%
% \paragraph{Main File.}
%
% The main file is called |cdocsamp.tex|.
%
% Load the \textsf{childdoc} definitions and
% declare the filename for the main document:
%    \begin{macrocode}
\input{childdoc.def}
\childdocmain{}
%    \end{macrocode}

% Optional override for |\version| flag:
%    \begin{macrocode}
%%\ifchilddoc\else\providecommand{\version}{draft}\fi
%    \end{macrocode}

% Define the default values for the |\version| flag
% (|final| for the main file and |draft| for childs):
%    \begin{macrocode}
\ifchilddoc
\providecommand{\version}{draft}
\else
\providecommand{\version}{final}
\fi
%    \end{macrocode}

% Load the standard document class:
%    \begin{macrocode}
\documentclass[12pt]{article}
%    \end{macrocode}

% Start the document body:
%    \begin{macrocode}
\begin{document}
%    \end{macrocode}

% Declare a title page.
% Print title, part of document being processed and version flag:
%    \begin{macrocode}
\addtocounter{page}{-1}
\begin{center}
{\LARGE\bfseries{}childdoc example\par}
\vspace{1cm}
\ifchilddoc
\ifchilddocmanual part\else chapter\fi:
`\childdocname' of `\childdocjob'\par
\else
main document: `\childdocjob'\par
\fi
version: \version\par
\end{center}
\newpage
%    \end{macrocode}

% Manually include selected file,
% otherwise process as usual:
%    \begin{macrocode}
\ifchilddocmanual
\section*{part `\childdocname'}
\input{\childdocname}
\else
%    \end{macrocode}

% Include the two chapters:
%    \begin{macrocode}
\include{cdocsch1}
\include{cdocsch2}
%    \end{macrocode}

% Include the two parts unless only chapters should be displayed:
%    \begin{macrocode}
\ifchilddoc\else
\section{part three}
\input{cdocspt3}
\section{part four}
\input{cdocspt4}
\fi
%    \end{macrocode}

% Process as usual until here:
%    \begin{macrocode}
\fi
%    \end{macrocode}

% End of document body:
%    \begin{macrocode}
\end{document}
%    \end{macrocode}
%\iffalse
%</samplemain>
%\fi
%
% %%%%%%%%%%%%%%%%%%%%%%%%%%%%%%%%%%%%%%
% \paragraph{Chapter Include Files.}
%
% The include files are called |cdocsch1.tex| and |cdocsch2.tex|.
%
%\iffalse
%<*samplechap1|samplechap2>
%\fi

% Optional override for |\version| flag:
%    \begin{macrocode}
%%\providecommand{\version}{final}
%    \end{macrocode}

% Include the main document:
%    \begin{macrocode}
\input{childdoc.def}
\childdocof{cdocsamp}
%    \end{macrocode}

%\iffalse
%</samplechap1|samplechap2>
%\fi
%
%\iffalse
%<*samplechap1>
%\fi
% Some text for chapter 1:
%    \begin{macrocode}
\section{one}
some text in chapter one
%    \end{macrocode}

%\iffalse
%</samplechap1>
%\fi
% Some text for chapter 2:
%\iffalse
%<*samplechap2>
%\fi
%    \begin{macrocode}
\section{two}
more text in chapter two
%    \end{macrocode}

%\iffalse
%</samplechap2>
%\fi
%
% %%%%%%%%%%%%%%%%%%%%%%%%%%%%%%%%%%%%%%
% \paragraph{Part Include Files.}
%
% The include files are called |cdocspt3.tex| and |cdocspt4.tex|.
%
%\iffalse
%<*samplepart3|samplepart4>
%\fi

% Optional override for |\version| flag:
%    \begin{macrocode}
%%\providecommand{\version}{final}
%    \end{macrocode}

% Include the main document:
%    \begin{macrocode}
\input{childdoc.def}
\childdocby{cdocsamp}
%    \end{macrocode}

%\iffalse
%</samplepart3|samplepart4>
%\fi
%
%\iffalse
%<*samplepart3>
%\fi
% Some text for part 3:
%    \begin{macrocode}
some text in part three
%    \end{macrocode}

%\iffalse
%</samplepart3>
%\fi
% Some text for part 4:
%\iffalse
%<*samplepart4>
%\fi
%    \begin{macrocode}
more text in part four
%    \end{macrocode}

%\iffalse
%</samplepart4>
%\fi
%
% %%%%%%%%%%%%%%%%%%%%%%%%%%%%%%%%%%%%%%
% \paragraph{Forwarding for a Complete Draft.}
%
% The following forwarding file |cdocsdrf.tex|
% compiles the main document in draft mode:
%\iffalse
%<*sampledraft>
%\fi
%    \begin{macrocode}
\def\version{draft}
\input{childdoc.def}
\childdocforward{cdocsamp}
%    \end{macrocode}

%\iffalse
%</sampledraft>
%\fi
%
% %%%%%%%%%%%%%%%%%%%%%%%%%%%%%%%%%%%%%%
% \paragraph{Forwarding for Final Version of the Chapters.}
%
% The following forwarding files |cdocsfn1.tex| and |cdocsfn2.tex|
% (with identical content)
% compile the final versions of the child documents
% |cdocsch1.tex| and |cdocsch2.tex|, respectively:
%\iffalse
%<*samplefinal>
%\fi
%    \begin{macrocode}
\def\version{final}
\input{childdoc.def}
\childdocforwardprefix[cdocsamp]{cdocsfn}{cdocsch}
%    \end{macrocode}

%\iffalse
%</samplefinal>
%\fi
%
% %%%%%%%%%%%%%%%%%%%%%%%%%%%%%%%%%%%%%%
% \paragraph{Command Line Processing.}
%
% The following three command lines generate the output files
% |cdocscld|, |cdocscl1| and |cdocscl2|
% which should be identical to
% |cdocsdrf|, |cdocsch1| and |cdocsfn2|, respectively:
% \begin{center}
% \begin{tabular}{l}
% |latex -jobname cdocscld \|\\
% |  "\def\version{draft}\input{childdoc.def}\childdocforward{cdocsamp}"|\\
% |latex -jobname cdocscl1 \|\\
% |  "\input{childdoc.def}\childdocforward[cdocsamp]{cdocsch1}"|\\
% |latex -jobname cdocscl2 \|\\
% |  "\def\version{final}\input{childdoc.def}\childdocforward{cdocsch2}"|
% \end{tabular}
% \end{center}
% Note that the trailing backslash on each first line
% merely continues the input to the second line
% (for convenient cut ant paste).
% Furthermore, the command |latex| can be replaced by any
% of its alternative versions such as |pdflatex|.
%
% %%%%%%%%%%%%%%%%%%%%%%%%%%%%%%%%%%%%%%%%%%%%%%%%%%%%%%%%%%%%%%%%%%%%%%%%%%%%%%
% %%%%%%%%%%%%%%%%%%%%%%%%%%%%%%%%%%%%%%%%%%%%%%%%%%%%%%%%%%%%%%%%%%%%%%%%%%%%%%
% \section{Implementation}
%\iffalse
%<*package>
%\fi
%
% This section describes the definitions file |childdoc.def|.

% The definitions cannot be loaded using |\usepackage| or |\RequirePackage|
% which has a mechanism to prevent loading a style file more than once.
% When loading the definitions by means of |\input|
% multiple instances have to be prevented manually:
%\iffalse
%This code needs to be before the `\ProvidesFile' directive
%which is defined at the beginning of this file.
%Therefore it is also placed there and commented out here.
%</package>
%<*discard>
%\fi
%    \begin{macrocode}
\ifdefined\childdocmain\endinput\fi
%    \end{macrocode}
%\iffalse
%</discard>
%<*package>
%\fi
%
% \macro{\ifchilddoc}
% \macro{\ifchilddocmanual}
% The conditional |\ifchilddoc| tells whether a
% child (true) or main (false) document is being compiled.
% The conditional |\ifchilddocmanual| tells whether
% the |\includeonly| mechanism is used (false) or
% the selection of child files must be performed manually (true).
% The definitions initialise to false:
%    \begin{macrocode}
\newif\ifchilddoc
\newif\ifchilddocmanual
%    \end{macrocode}

% \macro{\childdocname}
% \macro{\childdocjob}
% The macro |\childdocname| stores the name of the main document
% to be compiled. The macro |\childdocjob| stores the name of
% the document on which the \LaTeX{} compiler was originally invoked.
% The content of |\jobname| cannot be compared
% to filenames specified in the source due to different catcodes.
% The following code rescans |\jobname|, stores the result
% in |\childdocname| and saves a copy in |\childdocjob|:
%    \begin{macrocode}
\edef\childdocname{\scantokens\expandafter{\jobname\noexpand}}
\let\childdocjob\childdocname
%    \end{macrocode}

% \macro{\childdocdisable}
% The macro |\childdocdisable| prevents the main file
% from being processed more than once.
% At this stage, the main document command |\childdocmain|
% is assumed to be called once again where it should do nothing.
% Any subsequent call to it should prevent
% a secondary processing of the main document
% It overwrites the forwarding commands
% |\childdocof| and |\childdocforward|
% with empty macros to prevent further inclusions of the main document:
%    \begin{macrocode}
\newcommand{\childdocdisable}
{
  \renewcommand{\childdocmain}[1]{\renewcommand{\childdocmain}[1]{\endinput}}
  \renewcommand{\childdocof}[1]{}
  \renewcommand{\childdocby}[2][]{}
  \renewcommand{\childdocforward}[2][]{}
  \renewcommand{\childdocdisable}{}
}
%    \end{macrocode}

% \macro{\childdocmain}
% The macro |\childdocmain| is to be called at the top of the main file
% with nothing or the main filename (without extension) as argument.
% First, it breaks loops.
% If the argument is not empty and does not match |\childdocname|
% (which is set by the first inclusion of |childdoc.def|),
% |\ifchilddoc| is set to true, |\includeonly| is applied to the child file
% and |\jobname| is set to the main file
% (for proper handling of |.aux| files):
%    \begin{macrocode}
\newcommand{\childdocmain}[1]
{
  \childdocdisable\childdocmain{}
  \if?#1?\else
    \begingroup
      \def\childdoctmp{#1}
      \ifx\childdoctmp\childdocname
        \def\childdoctmp{}
      \else
        \def\childdoctmp
        {
          \childdoctrue
          \includeonly{\childdocname}
          \def\childdocjob{#1}
          \def\jobname{#1}
        }
      \fi
      \expandafter
    \endgroup
    \childdoctmp
  \fi
}
%    \end{macrocode}

% \macro{\childdocof}
% The command |\childdocof| redirects
% compilation to the main file |#1|.
%    \begin{macrocode}
\newcommand{\childdocof}[1]
{
  \childdocdisable
  \childdoctrue
  \includeonly{\childdocname}
  \def\jobname{#1}
  \def\childdocjob{#1}
  \input{#1}
}
%    \end{macrocode}

% \macro{\childdocby}
% The command |\childdocby| ....
%    \begin{macrocode}
\newcommand{\childdocby}[2][]
{
  \childdocdisable
  \childdoctrue
  \childdocmanualtrue
  \if?#1?\else
    \def\jobname{#2}
  \fi
  \def\childdocjob{#2}
  \input{#2}
  \endinput
}
%    \end{macrocode}

% \macro{\childdocforward}
% The command |\childdocforward| redirects
% compilation to the main file or
% (if the optional argument is given) a child file.
% Parameters are set as if the main file
% or a child file starting with |\childdocof| was compiled.
% Then compilation is handed over to the main file:
%    \begin{macrocode}
\newcommand{\childdocforward}[2][]
{
  \begingroup
    \if?#1?
      \def\childdoctmp
      {
        \def\childdocname{#2}
        \def\childdocjob{#2}
        \def\jobname{#2}
        \input{#2}
        \endinput
      }
    \else
      \def\childdoctmp
      {
        \childdocdisable
        \def\childdocname{#2}
        \childdoctrue
        \includeonly{#2}
        \def\childdocjob{#1}
        \def\jobname{#1}
        \input{#1}
        \endinput
      }
    \fi
    \expandafter
  \endgroup
  \childdoctmp
}
%    \end{macrocode}

% \macro{\childdocforwardprefix}
% The command |\childdocforwardprefix| redirects
% compilation to the main or a child file by means of a pattern.
% The prefix |#1| in the current filename is replaced by |#2|
% and the suffix of the current filename is kept
% (it is assumed that the filename does not contain the substring `|~~~|'
% which is used as a delimiter).
% Compilation is handed over to the new file by |\childdocforward|:
%    \begin{macrocode}
\newcommand{\childdocforwardprefix}[3][]
{
  \begingroup
    \def\childdocextract #2##1~~~{\def\childdoctmp{\childdocforward[#1]{#3##1}}}
    \expandafter\childdocextract\childdocname~~~
    \expandafter
  \endgroup
  \childdoctmp
}
%    \end{macrocode}

% \macro{\childdoc}
% The deprecated macro |\childdoc| is a legacy version of |\childdocmain|:
%    \begin{macrocode}
\newcommand{\childdoc}{\childdocmain}
%    \end{macrocode}

% \macro{\childdocredirect}
% The deprecated macro |\childdocredirect| is a legacy version
% of |\childdocforward| and |\childdocforwardprefix|:
%    \begin{macrocode}
\newcommand{\childdocredirect}[2][]
{
  \begingroup
    \if?#1?
      \def\childdoctmp{\childdocforward{#2}}
    \else
      \def\childdoctmp{\childdocforwardprefix{#1}{#2}}
    \fi
    \expandafter
  \endgroup
  \childdoctmp
}
%    \end{macrocode}

%\iffalse
%</package>
%\fi
%
\endinput
|\\
|\childdocby{|\textit{main}|}|\\
\end{tabular}
\end{center}
%
The directive |\childdocby| is similar to |\childdocof|
described in \secref{sec:include},
but the subsequent selection of content must be done manually.
To that end, both |\ifchilddoc| and |\ifchilddocmanual|
will be true upon processing of a part,
and the name of the part is stored in |\childdocname|.
Note that |\jobname| will be set to the filename of the current part
so that each part receives an individual |.aux| file
that does not interfere with the |.aux| file(s) of the main document.
This behaviour can be altered by the alternative form
|\childdocby[*]{|\textit{main}|}| (with a non-empty optional argument)
which uses the |.aux| file of the main document
by setting |\jobname| to \textit{main}.

%%%%%%%%%%%%%%%%%%%%%%%%%%%%%%%%%%%%%%%%%%%%%%%%%%%%%%%%%%%%%%%%%%%%%%%%%%%%%%%%
\subsection{Driver Development}
\label{sec:driver}

The \textsf{childdoc} mechanism can also be use for the development
of definition files such as \LaTeX{} styles or classes.
This case differs from the above setup with multiple parts
included by |\include| in that no |\includeonly| should be invoked.
This can be achieved by starting the include file
(before |\ProvidesPackage|) with:
%
\begin{center}
\begin{tabular}{l}
|% \iffalse
%
% childdoc.dtx Copyright (C) 2017-2018 Niklas Beisert
%
% This work may be distributed and/or modified under the
% conditions of the LaTeX Project Public License, either version 1.3
% of this license or (at your option) any later version.
% The latest version of this license is in
%   http://www.latex-project.org/lppl.txt
% and version 1.3 or later is part of all distributions of LaTeX
% version 2005/12/01 or later.
%
% This work has the LPPL maintenance status `maintained'.
%
% The Current Maintainer of this work is Niklas Beisert.
%
% This work consists of the files childdoc.dtx and childdoc.ins
% and the derived files childdoc.def and cdocsamp.tex with
% cdocsch1.tex, cdocsch2.tex, cdocsdrf.tex, cdocsfn1.tex, cdocsfn2.tex.
%
%<package>\ifdefined\childdocmain\endinput\fi
%<package>\ProvidesFile{childdoc.def}[2018/12/30 v2.0 child document driver]
%<samplemain>\ProvidesFile{cdocsamp.tex}[2018/12/30 v2.0 sample for childdoc]
%<*driver>
%\ProvidesFile{childdoc.drv}[2018/12/30 v2.0 childdoc reference manual file]
\PassOptionsToClass{10pt,a4paper}{article}
\documentclass{ltxdoc}

\usepackage[margin=35mm]{geometry}
\usepackage{hyperref}
\usepackage{hyperxmp}
\usepackage[usenames]{color}

\hypersetup{colorlinks=true}
\hypersetup{pdfstartview=FitH}
\hypersetup{pdfpagemode=UseNone}
\hypersetup{pdfsource={}}
\hypersetup{pdflang={en-UK}}
\hypersetup{pdfcopyright={Copyright 2017-2018 Niklas Beisert.
  This work may be distributed and/or modified under the
  conditions of the LaTeX Project Public License, either version 1.3
  of this license or (at your option) any later version.}}
\hypersetup{pdflicenseurl={http://www.latex-project.org/lppl.txt}}
\hypersetup{pdfcontactaddress={ETH Zurich, ITP, HIT K,
  Wolfgang-Pauli-Strasse 27}}
\hypersetup{pdfcontactpostcode={8093}}
\hypersetup{pdfcontactcity={Zurich}}
\hypersetup{pdfcontactcountry={Switzerland}}
\hypersetup{pdfcontactemail={nbeisert@itp.phys.ethz.ch}}
\hypersetup{pdfcontacturl={http://people.phys.ethz.ch/\xmptilde nbeisert/}}

\newcommand{\secref}[1]{\hyperref[#1]{section \ref*{#1}}}

\parskip1ex
\parindent0pt
\let\olditemize\itemize
\def\itemize{\olditemize\parskip0pt}

\begin{document}

\title{The \textsf{childdoc} Package}
\hypersetup{pdftitle={The childdoc Package}}
\author{Niklas Beisert\\[2ex]
  Institut f\"ur Theoretische Physik\\
  Eidgen\"ossische Technische Hochschule Z\"urich\\
  Wolfgang-Pauli-Strasse 27, 8093 Z\"urich, Switzerland\\[1ex]
  \href{mailto:nbeisert@itp.phys.ethz.ch}
  {\texttt{nbeisert@itp.phys.ethz.ch}}}
\hypersetup{pdfauthor={Niklas Beisert}}
\hypersetup{pdfsubject={Manual for the LaTeX2e Package childdoc}}
\date{30 December 2018, \textsf{v2.0}}
\maketitle

\begin{abstract}\noindent
\textsf{childdoc} is a \LaTeXe{} package
that enables the direct compilation
of document sections included by |\include|
to individual files.
\end{abstract}

\begingroup
\parskip0ex
\tableofcontents
\endgroup

%%%%%%%%%%%%%%%%%%%%%%%%%%%%%%%%%%%%%%%%%%%%%%%%%%%%%%%%%%%%%%%%%%%%%%%%%%%%%%%%
%%%%%%%%%%%%%%%%%%%%%%%%%%%%%%%%%%%%%%%%%%%%%%%%%%%%%%%%%%%%%%%%%%%%%%%%%%%%%%%%
\section{Introduction}

\LaTeX{} provides a mechanism to structure a large document (such as a book)
into a main file and several child files (containing the chapters)
using the |\include| command.
This mechanism is beneficial for documents
which span hundreds of pages in order to
make the source file(s) more manageable.
Moreover, compilation can be restricted to
selected child files by means of the |\includeonly| command.
The latter feature can be used to reduce the compilation time while editing
(this was significantly more useful in the earlier days of \LaTeX{})
or to generate a smaller document which is easier to navigate.
Another application of |\includeonly| is to generate
documents consisting of selected parts of the complete document.

However, there are a few drawbacks of the plain |\include| mechanism:
\begin{itemize}
\item
The child files cannot be compiled on their own,
they can only be compiled via the main file.
A naive editing environment
(such as a text editor with an option
to have the current file processed by \LaTeX)
may require one to switch to the main file before compiling;
attempting to compile the child file produces errors.
\item
The main file must be modified (each time)
to adjust the |\includeonly| command
to the present needs. This easily leaves the main file in a messy state.
\item
The generated document will always carry the filename
of the main document. This is inconvenient if
several child files are to be compiled and
to be kept for distribution.
\end{itemize}

The present package provides a simple interface
to make child files individually compilable by \LaTeX{}.
Compiling a child file then has the same effect as compiling
the main file with an |\includeonly| command
to select the appropriate child.
Moreover the generated document will carry the name of the child
rather than the main file.
This resolves all three above issues.

This feature is meant to make the editing of books,
thesis documents and lecture notes somewhat more convenient.
However, the package can also be used efficiently for
composing a series of documents (such as exercise sheets)
which are typically distributed individually.
It then assists the author in generating the individual documents
(potentially in different versions)
as well as a document containing the collected series.
Another application is in developing style files
or other kinds of included material
where compilation of the style file could redirect
to a sample or test file.

%%%%%%%%%%%%%%%%%%%%%%%%%%%%%%%%%%%%%%%%%%%%%%%%%%%%%%%%%%%%%%%%%%%%%%%%%%%%%%%%
%%%%%%%%%%%%%%%%%%%%%%%%%%%%%%%%%%%%%%%%%%%%%%%%%%%%%%%%%%%%%%%%%%%%%%%%%%%%%%%%
\section{Usage}

First of all, the package \textsf{childdoc} is \emph{not} a standard
\LaTeXe{} |.sty| style file! Therefore it needs to be invoked in
a non-standard way.

%%%%%%%%%%%%%%%%%%%%%%%%%%%%%%%%%%%%%%%%%%%%%%%%%%%%%%%%%%%%%%%%%%%%%%%%%%%%%%%%
\subsection{Included Files}
\label{sec:include}

%%%%%%%%%%%%%%%%%%%%%%%%%%%%%%%%%%%%%%%%
\DescribeMacro{\childdocmain}
To use the package, add the commands
\begin{center}
\begin{tabular}{l}
|\input{childdoc.def}|\\
|\childdocmain{}|\\
\end{tabular}
\end{center}
at the very top of the main \LaTeX{} file,
in particular \emph{before} the |\documentclass| statement!
The argument of |\childdocmain| should be left empty
(but it must be present).

%%%%%%%%%%%%%%%%%%%%%%%%%%%%%%%%%%%%%%%%
\DescribeMacro{\childdocof}
Furthermore, add the commands
\begin{center}
\begin{tabular}{l}
|\input{childdoc.def}|\\
|\childdocof{|\textit{main}|}|\\
\end{tabular}
\end{center}
at the top of every child file \textit{child}
which is included by |\include{|\textit{child}|}|
from within the main file
(or at least for those files to be compiled individually).
The argument \textit{main} must be the filename of the main file.

There are a couple of
considerations in setting up the main and child documents:

%%%%%%%%%%%%%%%%%%%%%%%%%%%%%%%%%%%%%%%%
\paragraph{Restrictions.}

Please note the following restrictions:
\begin{itemize}
\item
|\childdocmain| must be called with one argument \textit{main}
to ensure compatibility with earlier version of the package.
It must either be empty (|\childdocmain{}|)
or precisely match the filename of the main file in which it is specified.
See \secref{sec:detection} for further information.
\item
The filename \textit{main} must be specified without the |.tex| extension.
\item
The filename \textit{main} is case sensitive
(even in case-insensitive file systems)
due to internal string comparison.
\item
The argument \textit{main} should be fully expanded, it cannot be a macro.
\item
Subdirectories and special characters should be avoided in filenames.
\item
The command |\childdocmain{|\textit{main}|}| must be followed by a whitespace.
It should not be followed immediately by another command
or by a comment mark `|%|'.
This is because the \TeX{} parser reads the token immediately following
the argument of |\childdocmain| and puts it
at the beginning of every child section;
however, a white\-space is ignored.
\end{itemize}

%%%%%%%%%%%%%%%%%%%%%%%%%%%%%%%%%%%%%%%%
\paragraph{Content of Main File.}

It is advisable to place all content in the child files included by |\include|.
Any output contained in the main file will appear in all child documents
unless suppressed manually;
it cannot be suppressed automatically by the |\includeonly| directive
and thus should normally be avoided.
A method to include some content in the main file
by means of conditional processing is described in \secref{sec:conditional}.

%%%%%%%%%%%%%%%%%%%%%%%%%%%%%%%%%%%%%%%%
\paragraph{Page Numbering.}

When only a part of the document is compiled,
the appropriate numbering of pages
(as well as other status parameters)
is determined from the |.aux| files.
The latter contain information from previous passes.
However this information needs to propagate through
all intermediate child documents.
Therefore the page numbering in child documents may well
be inconsistent until the complete document is compiled at least once.

A useful (if unconventional) way to always ensure a consistent
page numbering is to restart the numbering in each child document
and denote the pages by `\textit{child}|.|\textit{page}'
where \textit{child} represents the chapter/section number of the child file.
This can be achieved by the command
|\numberwithin{page}{|\textit{child}|}|
of the \textsf{amsmath} package
where \textit{child} can be |chapter| or |section|
depending on the chosen structuring.
Alternatively, one can modify the macro |\thepage| appropriately
and reset the counter |page| at the start of each child file.

%%%%%%%%%%%%%%%%%%%%%%%%%%%%%%%%%%%%%%%%%%%%%%%%%%%%%%%%%%%%%%%%%%%%%%%%%%%%%%%%
\subsection{Conditional Processing}
\label{sec:conditional}

The package provides a mechanism to compile different versions
of a document. To customise the versions further some conditional processing
can come in handy to distinguish which version is being compiled.
The package provides two macros to describe the compilation context:

%%%%%%%%%%%%%%%%%%%%%%%%%%%%%%%%%%%%%%%%
\DescribeMacro{\ifchilddoc}
The conditional |\ifchilddoc| distinguishes between the compilation of
child documents and the main document:
%
\begin{center}
|\ifchilddoc |\textit{child-code}| |[|\||else |\textit{main-code}]| \||fi|
\end{center}

%%%%%%%%%%%%%%%%%%%%%%%%%%%%%%%%%%%%%%%%
\DescribeMacro{\childdocname}
\DescribeMacro{\childdocjob}
The macro |\childdocname| contains the filename (without extension)
of the main or child file being processed.
Note that |\childdocjob| will always contain the name of the main file.

%%%%%%%%%%%%%%%%%%%%%%%%%%%%%%%%%%%%%%%%
\paragraph{Title Page.}

Conditional processing can be used to include a title or banner page
in the main document when proper precautions are taken.
Importantly, the code in the main file should ensure that the page counter
(as well as other status parameters which are stored in the |.aux| files)
takes the same value after the conditional processing.
Otherwise the page numbers may take divergent values
depending on which part is compiled.

For example, a title page could be declared by:
%
\begin{center}
\begin{tabular}{l}
|\ifchilddoc\||else|\\
|\addtocounter{page}{-1}|\\
\textit{code for title page}\\
|\newpage|\\
|\||fi|
\end{tabular}
\end{center}
%
A banner page for the child documents can be generated by:
%
\begin{center}
\begin{tabular}{l}
|\ifchilddoc|\\
|\addtocounter{page}{-1}|\\
\textit{code for banner page}\\
|\newpage|\\
|\||fi|
\end{tabular}
\end{center}
%
Here one could write a message such as:
\begin{center}
|This is the part \childdocname{} of \childdocjob{}.|
\end{center}

%%%%%%%%%%%%%%%%%%%%%%%%%%%%%%%%%%%%%%%%%%%%%%%%%%%%%%%%%%%%%%%%%%%%%%%%%%%%%%%%
\subsection{Flags}
\label{sec:flags}

The package makes it easy to generate different versions
of the main or child documents.
To this end compilation flags can be defined
and assigned different default values.
They will be particularly useful in conjunction
with the forwarding mechanism described in \secref{sec:forward}.

For example, it may be useful to have a flag |\version|
which can be set to |draft| or |final|.
The document source will contain some conditional code
depending on the value of |\version|.
Suppose further, the flag should default to |final| for the main file
and to |draft| for child files
which is a natural assignment for editing the document.
This is achieved by placing the following code
in the preamble of the main document
(below the |\childdocmain| directive):
%
\begin{center}
\begin{tabular}{l}
|\ifchilddoc|\\
|\providecommand{\version}{draft}|\\
|\||else|\\
|\providecommand{\version}{final}|\\
|\||fi|
\end{tabular}
\end{center}
%
The definition by |\providecommand| makes sure
that previous definitions are not overwritten.
Further statements |\providecommand{\version}{...}|
can thus be added before the above code to override it.

For the main file, one might add a line
(between |\childdocmain| and the above block)
%
\begin{center}
|%\ifchilddoc\||else\providecommand{\version}{draft}\||fi|
\end{center}
%
which can be uncommented to produce a draft version.
Likewise one can add a line to the very top of a child file
(above the |\childdocof{|\textit{main}|}| directive)
%
\begin{center}
|%\providecommand{\version}{final}|
\end{center}
%
which can be uncommented to produce the final version of this child document.

%%%%%%%%%%%%%%%%%%%%%%%%%%%%%%%%%%%%%%%%%%%%%%%%%%%%%%%%%%%%%%%%%%%%%%%%%%%%%%%%
\subsection{Forwarding}
\label{sec:forward}

Different versions of the main or child documents
using compilation flags as described in \secref{sec:flags}
can be (permanently) stored in different files
for convenient compilation, viewing and distribution.
To this end, the package defines a command
to pass on compilation to a different file:

%%%%%%%%%%%%%%%%%%%%%%%%%%%%%%%%%%%%%%%%
\DescribeMacro{\childdocforward}
The command |\childdocforward| redirects processing to
another source file:
%
\begin{center}
\begin{tabular}{l}
|\input{childdoc.def}|\\
|\childdocforward[|\textit{main}|]{|\textit{dest}|}|\\
\end{tabular}
\end{center}
%
The argument \textit{dest} is the destination file
(without extension).
It should be the main file or one of the child files.
Note that further \textsf{childdoc} directives
such as |\childdocof| and |\childdocforward|
in the indicated file will be processed in this form.
The optional argument \textit{main}
passes on directly to the main file \textit{main}
while pretending to compile the child \textit{dest}.
This form behaves as if \textit{dest}
issues |\childdocof{|\textit{main}|}| right away,
and no further \textsf{childdoc} directives will be processed.

%%%%%%%%%%%%%%%%%%%%%%%%%%%%%%%%%%%%%%%%
\DescribeMacro{\...prefix}
In the alternative form |\childdocforwardprefix|,
%
\begin{center}
\begin{tabular}{l}
|\input{childdoc.def}|\\
|\childdocforwardprefix[|\textit{main}|]{|\textit{prefix}|}{|\textit{dest}|}|
\end{tabular}
\end{center}
%
the destination file is determined by a pattern
depending on the current file:
To make this work, the current file must be called
`{\textit{prefix}\hspace{0.2em}\textit{suffix}}'
with \textit{prefix} matching precisely the argument.
Processing is then passed on to the file
`{\textit{dest}\hspace{0.2em}\textit{suffix}}'.
Surely, the same effect is achieved by
directly specifying the
argument `{\textit{dest}\hspace{0.2em}\textit{suffix}}'
in the first form.
However, that requires to set up a different file
for each child. With the alternative form of the command
all these files can have exactly the same content
which simplifies setting them up and maintaining them.

For example, the following file |draft.tex|
with a compilation flag |\version| as described in \secref{sec:flags}
compiles the main document as a draft:
%
\begin{center}
\begin{tabular}{l}
|\def\version{draft}|\\
|\input{childdoc.def}|\\
|\childdocforward{|\textit{main}|}|
\end{tabular}
\end{center}
%
Likewise, the following files |final|\textit{nn}|.tex|
compile the final version of the child document
|child|\textit{nn}|.tex|:
%
\begin{center}
\begin{tabular}{l}
|\def\version{final}|\\
|\input{childdoc.def}|\\
|\childdocforwardprefix{final}{child}|
\end{tabular}
\end{center}
%

Note that when several versions of a main file and/or of each child file
are to be generated, it may be convenient to set up a |Makefile| or
shell script to automatise the process.

%%%%%%%%%%%%%%%%%%%%%%%%%%%%%%%%%%%%%%%%%%%%%%%%%%%%%%%%%%%%%%%%%%%%%%%%%%%%%%%%
\subsection{Command Line Processing}
\label{sec:commandline}

The effect of redirection files can also be achieved by invoking
the \LaTeX{} compiler with a more elaborate command line.
Most conveniently this should be done as part
of a shell script or a |Makefile|.

When using \textsf{childdoc} in the main file, the following
command lines effectively perform a redirection
(note that depending on the shell being used,
backslashes may have to be doubled: `|\|' $\to$ `|\\|'):
%
\begin{center}
|... -jobname "|\textit{target}|" |\\|"|[\textit{flags}]%
|\input{childdoc.def}\childdocforward[|\textit{main}|]{|\textit{dest}|}"|
\end{center}
%
Here \textit{target} is the name of the output file,
\textit{main} is the name of the main file
and \textit{dest} is the name of the main or child file to be processed
(all filenames without extensions).
The optional argument \textit{main} can be omitted
if \textit{main} matches \textit{dest}.
Optionally, compilation \textit{flags} can be defined via |\def| commands.
This command line makes the \TeX{} engine believe
it is compiling the file \textit{target}
whose content is specified as the latter parameter.
The provided code then forwards the processing to
\textit{main} or \textit{dest} as described in \secref{sec:forward}.

%%%%%%%%%%%%%%%%%%%%%%%%%%%%%%%%%%%%%%%%%%%%%%%%%%%%%%%%%%%%%%%%%%%%%%%%%%%%%%%%
\subsection{Include by Input}
\label{sec:input}

Including child documents by |\include| has some restrictions by design.
Most notably, the content of a child document always occupies
its own set of pages; pages cannot be shared between child documents.
Usually, this behaviour makes perfect sense
because each child document contain an essential part of the document.
However, in some situations it may be desirable to compose
a document from a collection of parts
without having mandatory page breaks between then.
For this case, the package
provides a mechanism to include parts
by |\input| which can also be processed individually.
However, by construction this mechanism
requires manual handling of the content to be output.

%%%%%%%%%%%%%%%%%%%%%%%%%%%%%%%%%%%%%%%%
\DescribeMacro{\ifchilddocmanual}
The main file should be prepared as usual, see \secref{sec:include}.
However, the document body must make a distinction
between processing of an individual part and of the main document, e.g.:
%
\begin{center}
\begin{tabular}{l}
|\ifchilddocmanual|\\
|\input{\childdocname}|\\
|\||else|\\
\textit{document body with }|\input{|\textit{part}|}|\\
|\||fi|
\end{tabular}
\end{center}
%
The conditional |\ifchilddocmanual| is true whenever
a part to be included by |\input| is being compiled,
and the name of the part is stored in |\childdocname|.

%%%%%%%%%%%%%%%%%%%%%%%%%%%%%%%%%%%%%%%%
\DescribeMacro{\childdocby}
Each part to be included by |\input| should start with:
%
\begin{center}
\begin{tabular}{l}
|\input{childdoc.def}|\\
|\childdocby{|\textit{main}|}|\\
\end{tabular}
\end{center}
%
The directive |\childdocby| is similar to |\childdocof|
described in \secref{sec:include},
but the subsequent selection of content must be done manually.
To that end, both |\ifchilddoc| and |\ifchilddocmanual|
will be true upon processing of a part,
and the name of the part is stored in |\childdocname|.
Note that |\jobname| will be set to the filename of the current part
so that each part receives an individual |.aux| file
that does not interfere with the |.aux| file(s) of the main document.
This behaviour can be altered by the alternative form
|\childdocby[*]{|\textit{main}|}| (with a non-empty optional argument)
which uses the |.aux| file of the main document
by setting |\jobname| to \textit{main}.

%%%%%%%%%%%%%%%%%%%%%%%%%%%%%%%%%%%%%%%%%%%%%%%%%%%%%%%%%%%%%%%%%%%%%%%%%%%%%%%%
\subsection{Driver Development}
\label{sec:driver}

The \textsf{childdoc} mechanism can also be use for the development
of definition files such as \LaTeX{} styles or classes.
This case differs from the above setup with multiple parts
included by |\include| in that no |\includeonly| should be invoked.
This can be achieved by starting the include file
(before |\ProvidesPackage|) with:
%
\begin{center}
\begin{tabular}{l}
|\input{childdoc.def}|\\
|\childdocforward{|\textit{main}|}|\\
\end{tabular}
\end{center}
%
or alternatively with:
%
\begin{center}
\begin{tabular}{l}
|\input{childdoc.def}|\\
|\childdocby{|\textit{main}|}|\\
\end{tabular}
\end{center}
%
Both forms have slightly different effects as described above.
The main file is prepared as usual, see \secref{sec:include}.

%%%%%%%%%%%%%%%%%%%%%%%%%%%%%%%%%%%%%%%%%%%%%%%%%%%%%%%%%%%%%%%%%%%%%%%%%%%%%%%%
\subsection{Legacy Detection}
\label{sec:detection}

The directive |\childdocmain| in the main file can detect
whether the complete document or merely a child is to be compiled
even without using the directive |\childdocof|.
This method is deprecated because it is less robust
and there is no compelling reason to use it;
it is merely provided for backward compatibility
and it may be removed in future versions.

If the detection mechanism is to be used,
it is mandatory to correctly specify
the filename of the main file as the argument of |\childdocmain|:
%
\begin{center}
\begin{tabular}{l}
|\input{childdoc.def}|\\
|\childdocmain{|\textit{main}|}|\\
\end{tabular}
\end{center}
%
If |\jobname| does not match the argument \textit{main} of |\childdocmain|,
it is assumed that |\jobname| points to the child file to be compiled.
When using |\childdocmain| with the main file specified as argument,
it suffices to start a child file
with just |\input{|\textit{main}|}|
without loading of the package and using |\childdocof|.
If instead all processing is done
with the appropriate \textsf{childdoc} directives,
the argument of \textit{main} of |\childdocmain| can be empty.

An alternative version of the command line processing described
in \secref{sec:commandline} using the detection mechanism reads:
%
\begin{center}
|... -jobname "|\textit{target}|" "|[\textit{flags}]%
[|\def\jobname{|\textit{dest}|}|]|\input{|\textit{main}|}"|
\end{center}

%%%%%%%%%%%%%%%%%%%%%%%%%%%%%%%%%%%%%%%%%%%%%%%%%%%%%%%%%%%%%%%%%%%%%%%%%%%%%%%%
\subsection{Manual Code}
\label{sec:manual}

In case one cannot be certain whether the definitions file |childdoc.def|
is installed on the target \TeX{} distribution
and one prefers not to ship it,
it is conceivable to paste a few relevant commands into the sources.

To that end, drop all statements |\input{childdoc.def}|
and perform the replacements as outlined below.
Instead of |\childdocmain{|\textit{main}|}| add the following code
to the top of the main file:
%
\begin{center}
\begin{tabular}{l}
|\||ifdefined\childdocname\endinput\||fi\newif\ifchilddoc|\\
|\edef\childdocname{\scantokens\expandafter{\jobname\noexpand}}|\\
|\def\childdocmain{|\textit{main}|}\||ifx\childdocmain\childdocname\||else|\\
|\childdoctrue\includeonly{\childdocname}\let\jobname\childdocmain\||fi|\\
\end{tabular}
\end{center}
%
Instead of |\childdocof{|\textit{main}|}| just include the main file
at the top of each child file:
%
\begin{center}
|\input{|\textit{main}|}|
\end{center}
%
A simple redirection |\childdocforward{|\textit{dest}|}| is achieved by:
%
\begin{center}
|\def\jobname{|\textit{dest}|}\input{\jobname}|
\end{center}
%
The redirection with prefix
|\childdocforwardprefix[|\textit{prefix}|]{|\textit{dest}|}|
is accomplished by:
%
\begin{center}
\begin{tabular}{l}
|{\edef\jobname{\scantokens\expandafter{\jobname\noexpand}}|\\
|\def\redirectjob |\textit{prefix}|#1~~~{\gdef\jobname{|\textit{dest}|#1}}|\\
|\expandafter\redirectjob\jobname~~~}\input{\jobname}|
\end{tabular}
\end{center}

In an alternative approach,
child documents can be compiled by a specific command line
without additional code or specific definitions:
%
\begin{center}
|... -jobname "|\textit{target}|" "|[\textit{flags}]%
|\includeonly{|\textit{dest}|}\input{|\textit{main}|}"|
\end{center}
%

%%%%%%%%%%%%%%%%%%%%%%%%%%%%%%%%%%%%%%%%%%%%%%%%%%%%%%%%%%%%%%%%%%%%%%%%%%%%%%%%
%%%%%%%%%%%%%%%%%%%%%%%%%%%%%%%%%%%%%%%%%%%%%%%%%%%%%%%%%%%%%%%%%%%%%%%%%%%%%%%%
\section{Information}

%%%%%%%%%%%%%%%%%%%%%%%%%%%%%%%%%%%%%%%%%%%%%%%%%%%%%%%%%%%%%%%%%%%%%%%%%%%%%%%%
\subsection{Copyright}

Copyright \copyright{} 2017--2018 Niklas Beisert

This work may be distributed and/or modified under the
conditions of the \LaTeX{} Project Public License, either version 1.3
of this license or (at your option) any later version.
The latest version of this license is in
  \url{http://www.latex-project.org/lppl.txt}
and version 1.3 or later is part of all distributions of \LaTeX{}
version 2005/12/01 or later.

This work has the LPPL maintenance status `maintained'.

The Current Maintainer of this work is Niklas Beisert.

This work consists of the files |README.txt|, |childdoc.ins| and |childdoc.dtx|
as well as the derived files |childdoc.def|, |cdocsamp.tex|
with |cdocsch1.tex|, |cdocsch2.tex|, |cdocspt3.tex|, |cdocspt4.tex|,
|cdocsdrf.tex|, |cdocsfn1.tex|, |cdocsfn2.tex|
as well as |childdoc.pdf|.

%%%%%%%%%%%%%%%%%%%%%%%%%%%%%%%%%%%%%%%%%%%%%%%%%%%%%%%%%%%%%%%%%%%%%%%%%%%%%%%%
\subsection{Files and Installation}

The package consists of the files:
%
\begin{center}
\begin{tabular}{ll}
    |README.txt|   & readme file \\
    |childdoc.ins| & installation file \\
    |childdoc.dtx| & source file \\
    |childdoc.def| & definition file \\
    |cdocsamp.tex| & sample main file \\
    |cdocsch1.tex| & sample include file \\
    |cdocsch2.tex| & sample include file \\
    |cdocspt3.tex| & sample part file \\
    |cdocspt4.tex| & sample part file \\
    |cdocsdrf.tex| & sample redirection file \\
    |cdocsfn1.tex| & sample redirection file \\
    |cdocsfn2.tex| & sample redirection file \\
    |childdoc.pdf| & manual
\end{tabular}
\end{center}
%
The distribution consists of the files
|README.txt|, |childdoc.ins| and |childdoc.dtx|.
%
\begin{itemize}
\item
Run (pdf)\LaTeX{} on |childdoc.dtx|
to compile the manual |childdoc.pdf| (this file).
\item
Run \LaTeX{} on |childdoc.ins| to create the definitions file |childdoc.def|
and the sample |cdocsamp.tex| with include files
|cdocsch1.tex|, |cdocsch2.tex|, |cdocspt3.tex|, |cdocspt4.tex|,
|cdocsdrf.tex|, |cdocsfn1.tex|, |cdocsfn2.tex|.
Then copy the file |childdoc.def| to an appropriate directory of your \LaTeX{}
distribution, e.g.\ \textit{texmf-root}|/tex/latex/childdoc|.
\end{itemize}

%%%%%%%%%%%%%%%%%%%%%%%%%%%%%%%%%%%%%%%%%%%%%%%%%%%%%%%%%%%%%%%%%%%%%%%%%%%%%%%%
\subsection{Related CTAN Packages}

There are several other packages which offer a similar functionality:
%
\begin{itemize}
\item
The packages
\href{http://ctan.org/pkg/docmute}{\textsf{docmute}},
\href{http://ctan.org/pkg/includex}{\textsf{includex}} and
\href{http://ctan.org/pkg/standalone}{\textsf{standalone}}
provide commands to include only the document body of
a child file thus allowing both files to be compiled individually.
\item
The packages \href{http://ctan.org/pkg/subdocs}{\textsf{subdocs}}
and \href{http://ctan.org/pkg/subfiles}{\textsf{subfiles}}
provide structures in which the main and child documents can be
encapsulated and allowing them to be compiled individually.
The inclusion mechanism is different from the conventional |\include|.
\item
The package \href{http://ctan.org/pkg/combine}{\textsf{combine}}
is an elaborate solution to combine several documents into one.
\end{itemize}
%
See also the CTAN topic \href{http://ctan.org/topic/subdocs}{\textsf{subdocs}}
for further related packages.
The present package differs from the above solutions in that
a document structure constructed with the conventional |\include| mechanism
just needs two extra commands at the top of every file
such that all constituent files can be compiled individually.

%%%%%%%%%%%%%%%%%%%%%%%%%%%%%%%%%%%%%%%%%%%%%%%%%%%%%%%%%%%%%%%%%%%%%%%%%%%%%%%%
%\subsection{Feature Suggestions}
%
%The following is a list of features which may be useful for future
%versions of this package:
%%
%\begin{itemize}
%\item
%\ldots
%\end{itemize}

%%%%%%%%%%%%%%%%%%%%%%%%%%%%%%%%%%%%%%%%%%%%%%%%%%%%%%%%%%%%%%%%%%%%%%%%%%%%%%%%
\subsection{Revision History}

%%%%%%%%%%%%%%%%%%%%%%%%%%%%%%%%%%%%%%%%
\paragraph{v2.0:} 2018/12/30

\begin{itemize}
\item
immediate forward processing
\item
added |\childdocby| mechanism
\item
manual restructured
\end{itemize}

%%%%%%%%%%%%%%%%%%%%%%%%%%%%%%%%%%%%%%%%
\paragraph{v1.6:} 2018/01/17

\begin{itemize}
\item
application for development of include files
\item
corrections to manual
\end{itemize}

%%%%%%%%%%%%%%%%%%%%%%%%%%%%%%%%%%%%%%%%
\paragraph{v1.5:} 2017/05/21

\begin{itemize}
\item
more complete structuring introduced
\item
|\childdocof| introduced
\item
|\childdoc| renamed to |\childdocmain|
\item
|\childredirect| renamed to |\childdocforward| and |\childdocforwardprefix|
and functionality expanded
\end{itemize}

%%%%%%%%%%%%%%%%%%%%%%%%%%%%%%%%%%%%%%%%
\paragraph{v1.0:} 2017/04/27

\begin{itemize}
\item
manual and install package
\item
first version published on CTAN
\end{itemize}

%%%%%%%%%%%%%%%%%%%%%%%%%%%%%%%%%%%%%%%%
\paragraph{v0.6:} 2017/04/26

\begin{itemize}
\item
redirection mechanism added
\end{itemize}

%%%%%%%%%%%%%%%%%%%%%%%%%%%%%%%%%%%%%%%%
\paragraph{v0.5:} 2017/04/26

\begin{itemize}
\item
functionality in definition file
\end{itemize}


%%%%%%%%%%%%%%%%%%%%%%%%%%%%%%%%%%%%%%%%%%%%%%%%%%%%%%%%%%%%%%%%%%%%%%%%%%%%%%%%
%%%%%%%%%%%%%%%%%%%%%%%%%%%%%%%%%%%%%%%%%%%%%%%%%%%%%%%%%%%%%%%%%%%%%%%%%%%%%%%%
%%%%%%%%%%%%%%%%%%%%%%%%%%%%%%%%%%%%%%%%%%%%%%%%%%%%%%%%%%%%%%%%%%%%%%%%%%%%%%%%
\appendix

\settowidth\MacroIndent{\rmfamily\scriptsize 000\ }

 \DocInput{childdoc.dtx}

\end{document}
%</driver>
% \fi
%
% %%%%%%%%%%%%%%%%%%%%%%%%%%%%%%%%%%%%%%%%%%%%%%%%%%%%%%%%%%%%%%%%%%%%%%%%%%%%%%
% %%%%%%%%%%%%%%%%%%%%%%%%%%%%%%%%%%%%%%%%%%%%%%%%%%%%%%%%%%%%%%%%%%%%%%%%%%%%%%
% \section{Sample}
%\iffalse
%<*samplemain>
%\fi
%
% The following presents a sample document
% with two chapters, two parts, a title page,
% a compile flag as well as three forwarding files to set the flag.
% It consists of eight |.tex| files:
% \begin{center}
% \begin{tabular}{ll}
% |cdocsamp.tex|&main file\\
% |cdocsch1.tex|&include file for chapter 1\\
% |cdocsch2.tex|&include file for chapter 2\\
% |cdocspt3.tex|&include file for part 3\\
% |cdocspt4.tex|&include file for part 4\\
% |cdocsdrf.tex|&forwarding file for main file in draft mode\\
% |cdocsfi1.tex|&forwarding file for final version of chapter 1\\
% |cdocsfi2.tex|&forwarding file for final version of chapter 2\\
% \end{tabular}
% \end{center}
% Each of the eight files can be compiled directly by the \LaTeX{} compiler.
%
% %%%%%%%%%%%%%%%%%%%%%%%%%%%%%%%%%%%%%%
% \paragraph{Main File.}
%
% The main file is called |cdocsamp.tex|.
%
% Load the \textsf{childdoc} definitions and
% declare the filename for the main document:
%    \begin{macrocode}
\input{childdoc.def}
\childdocmain{}
%    \end{macrocode}

% Optional override for |\version| flag:
%    \begin{macrocode}
%%\ifchilddoc\else\providecommand{\version}{draft}\fi
%    \end{macrocode}

% Define the default values for the |\version| flag
% (|final| for the main file and |draft| for childs):
%    \begin{macrocode}
\ifchilddoc
\providecommand{\version}{draft}
\else
\providecommand{\version}{final}
\fi
%    \end{macrocode}

% Load the standard document class:
%    \begin{macrocode}
\documentclass[12pt]{article}
%    \end{macrocode}

% Start the document body:
%    \begin{macrocode}
\begin{document}
%    \end{macrocode}

% Declare a title page.
% Print title, part of document being processed and version flag:
%    \begin{macrocode}
\addtocounter{page}{-1}
\begin{center}
{\LARGE\bfseries{}childdoc example\par}
\vspace{1cm}
\ifchilddoc
\ifchilddocmanual part\else chapter\fi:
`\childdocname' of `\childdocjob'\par
\else
main document: `\childdocjob'\par
\fi
version: \version\par
\end{center}
\newpage
%    \end{macrocode}

% Manually include selected file,
% otherwise process as usual:
%    \begin{macrocode}
\ifchilddocmanual
\section*{part `\childdocname'}
\input{\childdocname}
\else
%    \end{macrocode}

% Include the two chapters:
%    \begin{macrocode}
\include{cdocsch1}
\include{cdocsch2}
%    \end{macrocode}

% Include the two parts unless only chapters should be displayed:
%    \begin{macrocode}
\ifchilddoc\else
\section{part three}
\input{cdocspt3}
\section{part four}
\input{cdocspt4}
\fi
%    \end{macrocode}

% Process as usual until here:
%    \begin{macrocode}
\fi
%    \end{macrocode}

% End of document body:
%    \begin{macrocode}
\end{document}
%    \end{macrocode}
%\iffalse
%</samplemain>
%\fi
%
% %%%%%%%%%%%%%%%%%%%%%%%%%%%%%%%%%%%%%%
% \paragraph{Chapter Include Files.}
%
% The include files are called |cdocsch1.tex| and |cdocsch2.tex|.
%
%\iffalse
%<*samplechap1|samplechap2>
%\fi

% Optional override for |\version| flag:
%    \begin{macrocode}
%%\providecommand{\version}{final}
%    \end{macrocode}

% Include the main document:
%    \begin{macrocode}
\input{childdoc.def}
\childdocof{cdocsamp}
%    \end{macrocode}

%\iffalse
%</samplechap1|samplechap2>
%\fi
%
%\iffalse
%<*samplechap1>
%\fi
% Some text for chapter 1:
%    \begin{macrocode}
\section{one}
some text in chapter one
%    \end{macrocode}

%\iffalse
%</samplechap1>
%\fi
% Some text for chapter 2:
%\iffalse
%<*samplechap2>
%\fi
%    \begin{macrocode}
\section{two}
more text in chapter two
%    \end{macrocode}

%\iffalse
%</samplechap2>
%\fi
%
% %%%%%%%%%%%%%%%%%%%%%%%%%%%%%%%%%%%%%%
% \paragraph{Part Include Files.}
%
% The include files are called |cdocspt3.tex| and |cdocspt4.tex|.
%
%\iffalse
%<*samplepart3|samplepart4>
%\fi

% Optional override for |\version| flag:
%    \begin{macrocode}
%%\providecommand{\version}{final}
%    \end{macrocode}

% Include the main document:
%    \begin{macrocode}
\input{childdoc.def}
\childdocby{cdocsamp}
%    \end{macrocode}

%\iffalse
%</samplepart3|samplepart4>
%\fi
%
%\iffalse
%<*samplepart3>
%\fi
% Some text for part 3:
%    \begin{macrocode}
some text in part three
%    \end{macrocode}

%\iffalse
%</samplepart3>
%\fi
% Some text for part 4:
%\iffalse
%<*samplepart4>
%\fi
%    \begin{macrocode}
more text in part four
%    \end{macrocode}

%\iffalse
%</samplepart4>
%\fi
%
% %%%%%%%%%%%%%%%%%%%%%%%%%%%%%%%%%%%%%%
% \paragraph{Forwarding for a Complete Draft.}
%
% The following forwarding file |cdocsdrf.tex|
% compiles the main document in draft mode:
%\iffalse
%<*sampledraft>
%\fi
%    \begin{macrocode}
\def\version{draft}
\input{childdoc.def}
\childdocforward{cdocsamp}
%    \end{macrocode}

%\iffalse
%</sampledraft>
%\fi
%
% %%%%%%%%%%%%%%%%%%%%%%%%%%%%%%%%%%%%%%
% \paragraph{Forwarding for Final Version of the Chapters.}
%
% The following forwarding files |cdocsfn1.tex| and |cdocsfn2.tex|
% (with identical content)
% compile the final versions of the child documents
% |cdocsch1.tex| and |cdocsch2.tex|, respectively:
%\iffalse
%<*samplefinal>
%\fi
%    \begin{macrocode}
\def\version{final}
\input{childdoc.def}
\childdocforwardprefix[cdocsamp]{cdocsfn}{cdocsch}
%    \end{macrocode}

%\iffalse
%</samplefinal>
%\fi
%
% %%%%%%%%%%%%%%%%%%%%%%%%%%%%%%%%%%%%%%
% \paragraph{Command Line Processing.}
%
% The following three command lines generate the output files
% |cdocscld|, |cdocscl1| and |cdocscl2|
% which should be identical to
% |cdocsdrf|, |cdocsch1| and |cdocsfn2|, respectively:
% \begin{center}
% \begin{tabular}{l}
% |latex -jobname cdocscld \|\\
% |  "\def\version{draft}\input{childdoc.def}\childdocforward{cdocsamp}"|\\
% |latex -jobname cdocscl1 \|\\
% |  "\input{childdoc.def}\childdocforward[cdocsamp]{cdocsch1}"|\\
% |latex -jobname cdocscl2 \|\\
% |  "\def\version{final}\input{childdoc.def}\childdocforward{cdocsch2}"|
% \end{tabular}
% \end{center}
% Note that the trailing backslash on each first line
% merely continues the input to the second line
% (for convenient cut ant paste).
% Furthermore, the command |latex| can be replaced by any
% of its alternative versions such as |pdflatex|.
%
% %%%%%%%%%%%%%%%%%%%%%%%%%%%%%%%%%%%%%%%%%%%%%%%%%%%%%%%%%%%%%%%%%%%%%%%%%%%%%%
% %%%%%%%%%%%%%%%%%%%%%%%%%%%%%%%%%%%%%%%%%%%%%%%%%%%%%%%%%%%%%%%%%%%%%%%%%%%%%%
% \section{Implementation}
%\iffalse
%<*package>
%\fi
%
% This section describes the definitions file |childdoc.def|.

% The definitions cannot be loaded using |\usepackage| or |\RequirePackage|
% which has a mechanism to prevent loading a style file more than once.
% When loading the definitions by means of |\input|
% multiple instances have to be prevented manually:
%\iffalse
%This code needs to be before the `\ProvidesFile' directive
%which is defined at the beginning of this file.
%Therefore it is also placed there and commented out here.
%</package>
%<*discard>
%\fi
%    \begin{macrocode}
\ifdefined\childdocmain\endinput\fi
%    \end{macrocode}
%\iffalse
%</discard>
%<*package>
%\fi
%
% \macro{\ifchilddoc}
% \macro{\ifchilddocmanual}
% The conditional |\ifchilddoc| tells whether a
% child (true) or main (false) document is being compiled.
% The conditional |\ifchilddocmanual| tells whether
% the |\includeonly| mechanism is used (false) or
% the selection of child files must be performed manually (true).
% The definitions initialise to false:
%    \begin{macrocode}
\newif\ifchilddoc
\newif\ifchilddocmanual
%    \end{macrocode}

% \macro{\childdocname}
% \macro{\childdocjob}
% The macro |\childdocname| stores the name of the main document
% to be compiled. The macro |\childdocjob| stores the name of
% the document on which the \LaTeX{} compiler was originally invoked.
% The content of |\jobname| cannot be compared
% to filenames specified in the source due to different catcodes.
% The following code rescans |\jobname|, stores the result
% in |\childdocname| and saves a copy in |\childdocjob|:
%    \begin{macrocode}
\edef\childdocname{\scantokens\expandafter{\jobname\noexpand}}
\let\childdocjob\childdocname
%    \end{macrocode}

% \macro{\childdocdisable}
% The macro |\childdocdisable| prevents the main file
% from being processed more than once.
% At this stage, the main document command |\childdocmain|
% is assumed to be called once again where it should do nothing.
% Any subsequent call to it should prevent
% a secondary processing of the main document
% It overwrites the forwarding commands
% |\childdocof| and |\childdocforward|
% with empty macros to prevent further inclusions of the main document:
%    \begin{macrocode}
\newcommand{\childdocdisable}
{
  \renewcommand{\childdocmain}[1]{\renewcommand{\childdocmain}[1]{\endinput}}
  \renewcommand{\childdocof}[1]{}
  \renewcommand{\childdocby}[2][]{}
  \renewcommand{\childdocforward}[2][]{}
  \renewcommand{\childdocdisable}{}
}
%    \end{macrocode}

% \macro{\childdocmain}
% The macro |\childdocmain| is to be called at the top of the main file
% with nothing or the main filename (without extension) as argument.
% First, it breaks loops.
% If the argument is not empty and does not match |\childdocname|
% (which is set by the first inclusion of |childdoc.def|),
% |\ifchilddoc| is set to true, |\includeonly| is applied to the child file
% and |\jobname| is set to the main file
% (for proper handling of |.aux| files):
%    \begin{macrocode}
\newcommand{\childdocmain}[1]
{
  \childdocdisable\childdocmain{}
  \if?#1?\else
    \begingroup
      \def\childdoctmp{#1}
      \ifx\childdoctmp\childdocname
        \def\childdoctmp{}
      \else
        \def\childdoctmp
        {
          \childdoctrue
          \includeonly{\childdocname}
          \def\childdocjob{#1}
          \def\jobname{#1}
        }
      \fi
      \expandafter
    \endgroup
    \childdoctmp
  \fi
}
%    \end{macrocode}

% \macro{\childdocof}
% The command |\childdocof| redirects
% compilation to the main file |#1|.
%    \begin{macrocode}
\newcommand{\childdocof}[1]
{
  \childdocdisable
  \childdoctrue
  \includeonly{\childdocname}
  \def\jobname{#1}
  \def\childdocjob{#1}
  \input{#1}
}
%    \end{macrocode}

% \macro{\childdocby}
% The command |\childdocby| ....
%    \begin{macrocode}
\newcommand{\childdocby}[2][]
{
  \childdocdisable
  \childdoctrue
  \childdocmanualtrue
  \if?#1?\else
    \def\jobname{#2}
  \fi
  \def\childdocjob{#2}
  \input{#2}
  \endinput
}
%    \end{macrocode}

% \macro{\childdocforward}
% The command |\childdocforward| redirects
% compilation to the main file or
% (if the optional argument is given) a child file.
% Parameters are set as if the main file
% or a child file starting with |\childdocof| was compiled.
% Then compilation is handed over to the main file:
%    \begin{macrocode}
\newcommand{\childdocforward}[2][]
{
  \begingroup
    \if?#1?
      \def\childdoctmp
      {
        \def\childdocname{#2}
        \def\childdocjob{#2}
        \def\jobname{#2}
        \input{#2}
        \endinput
      }
    \else
      \def\childdoctmp
      {
        \childdocdisable
        \def\childdocname{#2}
        \childdoctrue
        \includeonly{#2}
        \def\childdocjob{#1}
        \def\jobname{#1}
        \input{#1}
        \endinput
      }
    \fi
    \expandafter
  \endgroup
  \childdoctmp
}
%    \end{macrocode}

% \macro{\childdocforwardprefix}
% The command |\childdocforwardprefix| redirects
% compilation to the main or a child file by means of a pattern.
% The prefix |#1| in the current filename is replaced by |#2|
% and the suffix of the current filename is kept
% (it is assumed that the filename does not contain the substring `|~~~|'
% which is used as a delimiter).
% Compilation is handed over to the new file by |\childdocforward|:
%    \begin{macrocode}
\newcommand{\childdocforwardprefix}[3][]
{
  \begingroup
    \def\childdocextract #2##1~~~{\def\childdoctmp{\childdocforward[#1]{#3##1}}}
    \expandafter\childdocextract\childdocname~~~
    \expandafter
  \endgroup
  \childdoctmp
}
%    \end{macrocode}

% \macro{\childdoc}
% The deprecated macro |\childdoc| is a legacy version of |\childdocmain|:
%    \begin{macrocode}
\newcommand{\childdoc}{\childdocmain}
%    \end{macrocode}

% \macro{\childdocredirect}
% The deprecated macro |\childdocredirect| is a legacy version
% of |\childdocforward| and |\childdocforwardprefix|:
%    \begin{macrocode}
\newcommand{\childdocredirect}[2][]
{
  \begingroup
    \if?#1?
      \def\childdoctmp{\childdocforward{#2}}
    \else
      \def\childdoctmp{\childdocforwardprefix{#1}{#2}}
    \fi
    \expandafter
  \endgroup
  \childdoctmp
}
%    \end{macrocode}

%\iffalse
%</package>
%\fi
%
\endinput
|\\
|\childdocforward{|\textit{main}|}|\\
\end{tabular}
\end{center}
%
or alternatively with:
%
\begin{center}
\begin{tabular}{l}
|% \iffalse
%
% childdoc.dtx Copyright (C) 2017-2018 Niklas Beisert
%
% This work may be distributed and/or modified under the
% conditions of the LaTeX Project Public License, either version 1.3
% of this license or (at your option) any later version.
% The latest version of this license is in
%   http://www.latex-project.org/lppl.txt
% and version 1.3 or later is part of all distributions of LaTeX
% version 2005/12/01 or later.
%
% This work has the LPPL maintenance status `maintained'.
%
% The Current Maintainer of this work is Niklas Beisert.
%
% This work consists of the files childdoc.dtx and childdoc.ins
% and the derived files childdoc.def and cdocsamp.tex with
% cdocsch1.tex, cdocsch2.tex, cdocsdrf.tex, cdocsfn1.tex, cdocsfn2.tex.
%
%<package>\ifdefined\childdocmain\endinput\fi
%<package>\ProvidesFile{childdoc.def}[2018/12/30 v2.0 child document driver]
%<samplemain>\ProvidesFile{cdocsamp.tex}[2018/12/30 v2.0 sample for childdoc]
%<*driver>
%\ProvidesFile{childdoc.drv}[2018/12/30 v2.0 childdoc reference manual file]
\PassOptionsToClass{10pt,a4paper}{article}
\documentclass{ltxdoc}

\usepackage[margin=35mm]{geometry}
\usepackage{hyperref}
\usepackage{hyperxmp}
\usepackage[usenames]{color}

\hypersetup{colorlinks=true}
\hypersetup{pdfstartview=FitH}
\hypersetup{pdfpagemode=UseNone}
\hypersetup{pdfsource={}}
\hypersetup{pdflang={en-UK}}
\hypersetup{pdfcopyright={Copyright 2017-2018 Niklas Beisert.
  This work may be distributed and/or modified under the
  conditions of the LaTeX Project Public License, either version 1.3
  of this license or (at your option) any later version.}}
\hypersetup{pdflicenseurl={http://www.latex-project.org/lppl.txt}}
\hypersetup{pdfcontactaddress={ETH Zurich, ITP, HIT K,
  Wolfgang-Pauli-Strasse 27}}
\hypersetup{pdfcontactpostcode={8093}}
\hypersetup{pdfcontactcity={Zurich}}
\hypersetup{pdfcontactcountry={Switzerland}}
\hypersetup{pdfcontactemail={nbeisert@itp.phys.ethz.ch}}
\hypersetup{pdfcontacturl={http://people.phys.ethz.ch/\xmptilde nbeisert/}}

\newcommand{\secref}[1]{\hyperref[#1]{section \ref*{#1}}}

\parskip1ex
\parindent0pt
\let\olditemize\itemize
\def\itemize{\olditemize\parskip0pt}

\begin{document}

\title{The \textsf{childdoc} Package}
\hypersetup{pdftitle={The childdoc Package}}
\author{Niklas Beisert\\[2ex]
  Institut f\"ur Theoretische Physik\\
  Eidgen\"ossische Technische Hochschule Z\"urich\\
  Wolfgang-Pauli-Strasse 27, 8093 Z\"urich, Switzerland\\[1ex]
  \href{mailto:nbeisert@itp.phys.ethz.ch}
  {\texttt{nbeisert@itp.phys.ethz.ch}}}
\hypersetup{pdfauthor={Niklas Beisert}}
\hypersetup{pdfsubject={Manual for the LaTeX2e Package childdoc}}
\date{30 December 2018, \textsf{v2.0}}
\maketitle

\begin{abstract}\noindent
\textsf{childdoc} is a \LaTeXe{} package
that enables the direct compilation
of document sections included by |\include|
to individual files.
\end{abstract}

\begingroup
\parskip0ex
\tableofcontents
\endgroup

%%%%%%%%%%%%%%%%%%%%%%%%%%%%%%%%%%%%%%%%%%%%%%%%%%%%%%%%%%%%%%%%%%%%%%%%%%%%%%%%
%%%%%%%%%%%%%%%%%%%%%%%%%%%%%%%%%%%%%%%%%%%%%%%%%%%%%%%%%%%%%%%%%%%%%%%%%%%%%%%%
\section{Introduction}

\LaTeX{} provides a mechanism to structure a large document (such as a book)
into a main file and several child files (containing the chapters)
using the |\include| command.
This mechanism is beneficial for documents
which span hundreds of pages in order to
make the source file(s) more manageable.
Moreover, compilation can be restricted to
selected child files by means of the |\includeonly| command.
The latter feature can be used to reduce the compilation time while editing
(this was significantly more useful in the earlier days of \LaTeX{})
or to generate a smaller document which is easier to navigate.
Another application of |\includeonly| is to generate
documents consisting of selected parts of the complete document.

However, there are a few drawbacks of the plain |\include| mechanism:
\begin{itemize}
\item
The child files cannot be compiled on their own,
they can only be compiled via the main file.
A naive editing environment
(such as a text editor with an option
to have the current file processed by \LaTeX)
may require one to switch to the main file before compiling;
attempting to compile the child file produces errors.
\item
The main file must be modified (each time)
to adjust the |\includeonly| command
to the present needs. This easily leaves the main file in a messy state.
\item
The generated document will always carry the filename
of the main document. This is inconvenient if
several child files are to be compiled and
to be kept for distribution.
\end{itemize}

The present package provides a simple interface
to make child files individually compilable by \LaTeX{}.
Compiling a child file then has the same effect as compiling
the main file with an |\includeonly| command
to select the appropriate child.
Moreover the generated document will carry the name of the child
rather than the main file.
This resolves all three above issues.

This feature is meant to make the editing of books,
thesis documents and lecture notes somewhat more convenient.
However, the package can also be used efficiently for
composing a series of documents (such as exercise sheets)
which are typically distributed individually.
It then assists the author in generating the individual documents
(potentially in different versions)
as well as a document containing the collected series.
Another application is in developing style files
or other kinds of included material
where compilation of the style file could redirect
to a sample or test file.

%%%%%%%%%%%%%%%%%%%%%%%%%%%%%%%%%%%%%%%%%%%%%%%%%%%%%%%%%%%%%%%%%%%%%%%%%%%%%%%%
%%%%%%%%%%%%%%%%%%%%%%%%%%%%%%%%%%%%%%%%%%%%%%%%%%%%%%%%%%%%%%%%%%%%%%%%%%%%%%%%
\section{Usage}

First of all, the package \textsf{childdoc} is \emph{not} a standard
\LaTeXe{} |.sty| style file! Therefore it needs to be invoked in
a non-standard way.

%%%%%%%%%%%%%%%%%%%%%%%%%%%%%%%%%%%%%%%%%%%%%%%%%%%%%%%%%%%%%%%%%%%%%%%%%%%%%%%%
\subsection{Included Files}
\label{sec:include}

%%%%%%%%%%%%%%%%%%%%%%%%%%%%%%%%%%%%%%%%
\DescribeMacro{\childdocmain}
To use the package, add the commands
\begin{center}
\begin{tabular}{l}
|\input{childdoc.def}|\\
|\childdocmain{}|\\
\end{tabular}
\end{center}
at the very top of the main \LaTeX{} file,
in particular \emph{before} the |\documentclass| statement!
The argument of |\childdocmain| should be left empty
(but it must be present).

%%%%%%%%%%%%%%%%%%%%%%%%%%%%%%%%%%%%%%%%
\DescribeMacro{\childdocof}
Furthermore, add the commands
\begin{center}
\begin{tabular}{l}
|\input{childdoc.def}|\\
|\childdocof{|\textit{main}|}|\\
\end{tabular}
\end{center}
at the top of every child file \textit{child}
which is included by |\include{|\textit{child}|}|
from within the main file
(or at least for those files to be compiled individually).
The argument \textit{main} must be the filename of the main file.

There are a couple of
considerations in setting up the main and child documents:

%%%%%%%%%%%%%%%%%%%%%%%%%%%%%%%%%%%%%%%%
\paragraph{Restrictions.}

Please note the following restrictions:
\begin{itemize}
\item
|\childdocmain| must be called with one argument \textit{main}
to ensure compatibility with earlier version of the package.
It must either be empty (|\childdocmain{}|)
or precisely match the filename of the main file in which it is specified.
See \secref{sec:detection} for further information.
\item
The filename \textit{main} must be specified without the |.tex| extension.
\item
The filename \textit{main} is case sensitive
(even in case-insensitive file systems)
due to internal string comparison.
\item
The argument \textit{main} should be fully expanded, it cannot be a macro.
\item
Subdirectories and special characters should be avoided in filenames.
\item
The command |\childdocmain{|\textit{main}|}| must be followed by a whitespace.
It should not be followed immediately by another command
or by a comment mark `|%|'.
This is because the \TeX{} parser reads the token immediately following
the argument of |\childdocmain| and puts it
at the beginning of every child section;
however, a white\-space is ignored.
\end{itemize}

%%%%%%%%%%%%%%%%%%%%%%%%%%%%%%%%%%%%%%%%
\paragraph{Content of Main File.}

It is advisable to place all content in the child files included by |\include|.
Any output contained in the main file will appear in all child documents
unless suppressed manually;
it cannot be suppressed automatically by the |\includeonly| directive
and thus should normally be avoided.
A method to include some content in the main file
by means of conditional processing is described in \secref{sec:conditional}.

%%%%%%%%%%%%%%%%%%%%%%%%%%%%%%%%%%%%%%%%
\paragraph{Page Numbering.}

When only a part of the document is compiled,
the appropriate numbering of pages
(as well as other status parameters)
is determined from the |.aux| files.
The latter contain information from previous passes.
However this information needs to propagate through
all intermediate child documents.
Therefore the page numbering in child documents may well
be inconsistent until the complete document is compiled at least once.

A useful (if unconventional) way to always ensure a consistent
page numbering is to restart the numbering in each child document
and denote the pages by `\textit{child}|.|\textit{page}'
where \textit{child} represents the chapter/section number of the child file.
This can be achieved by the command
|\numberwithin{page}{|\textit{child}|}|
of the \textsf{amsmath} package
where \textit{child} can be |chapter| or |section|
depending on the chosen structuring.
Alternatively, one can modify the macro |\thepage| appropriately
and reset the counter |page| at the start of each child file.

%%%%%%%%%%%%%%%%%%%%%%%%%%%%%%%%%%%%%%%%%%%%%%%%%%%%%%%%%%%%%%%%%%%%%%%%%%%%%%%%
\subsection{Conditional Processing}
\label{sec:conditional}

The package provides a mechanism to compile different versions
of a document. To customise the versions further some conditional processing
can come in handy to distinguish which version is being compiled.
The package provides two macros to describe the compilation context:

%%%%%%%%%%%%%%%%%%%%%%%%%%%%%%%%%%%%%%%%
\DescribeMacro{\ifchilddoc}
The conditional |\ifchilddoc| distinguishes between the compilation of
child documents and the main document:
%
\begin{center}
|\ifchilddoc |\textit{child-code}| |[|\||else |\textit{main-code}]| \||fi|
\end{center}

%%%%%%%%%%%%%%%%%%%%%%%%%%%%%%%%%%%%%%%%
\DescribeMacro{\childdocname}
\DescribeMacro{\childdocjob}
The macro |\childdocname| contains the filename (without extension)
of the main or child file being processed.
Note that |\childdocjob| will always contain the name of the main file.

%%%%%%%%%%%%%%%%%%%%%%%%%%%%%%%%%%%%%%%%
\paragraph{Title Page.}

Conditional processing can be used to include a title or banner page
in the main document when proper precautions are taken.
Importantly, the code in the main file should ensure that the page counter
(as well as other status parameters which are stored in the |.aux| files)
takes the same value after the conditional processing.
Otherwise the page numbers may take divergent values
depending on which part is compiled.

For example, a title page could be declared by:
%
\begin{center}
\begin{tabular}{l}
|\ifchilddoc\||else|\\
|\addtocounter{page}{-1}|\\
\textit{code for title page}\\
|\newpage|\\
|\||fi|
\end{tabular}
\end{center}
%
A banner page for the child documents can be generated by:
%
\begin{center}
\begin{tabular}{l}
|\ifchilddoc|\\
|\addtocounter{page}{-1}|\\
\textit{code for banner page}\\
|\newpage|\\
|\||fi|
\end{tabular}
\end{center}
%
Here one could write a message such as:
\begin{center}
|This is the part \childdocname{} of \childdocjob{}.|
\end{center}

%%%%%%%%%%%%%%%%%%%%%%%%%%%%%%%%%%%%%%%%%%%%%%%%%%%%%%%%%%%%%%%%%%%%%%%%%%%%%%%%
\subsection{Flags}
\label{sec:flags}

The package makes it easy to generate different versions
of the main or child documents.
To this end compilation flags can be defined
and assigned different default values.
They will be particularly useful in conjunction
with the forwarding mechanism described in \secref{sec:forward}.

For example, it may be useful to have a flag |\version|
which can be set to |draft| or |final|.
The document source will contain some conditional code
depending on the value of |\version|.
Suppose further, the flag should default to |final| for the main file
and to |draft| for child files
which is a natural assignment for editing the document.
This is achieved by placing the following code
in the preamble of the main document
(below the |\childdocmain| directive):
%
\begin{center}
\begin{tabular}{l}
|\ifchilddoc|\\
|\providecommand{\version}{draft}|\\
|\||else|\\
|\providecommand{\version}{final}|\\
|\||fi|
\end{tabular}
\end{center}
%
The definition by |\providecommand| makes sure
that previous definitions are not overwritten.
Further statements |\providecommand{\version}{...}|
can thus be added before the above code to override it.

For the main file, one might add a line
(between |\childdocmain| and the above block)
%
\begin{center}
|%\ifchilddoc\||else\providecommand{\version}{draft}\||fi|
\end{center}
%
which can be uncommented to produce a draft version.
Likewise one can add a line to the very top of a child file
(above the |\childdocof{|\textit{main}|}| directive)
%
\begin{center}
|%\providecommand{\version}{final}|
\end{center}
%
which can be uncommented to produce the final version of this child document.

%%%%%%%%%%%%%%%%%%%%%%%%%%%%%%%%%%%%%%%%%%%%%%%%%%%%%%%%%%%%%%%%%%%%%%%%%%%%%%%%
\subsection{Forwarding}
\label{sec:forward}

Different versions of the main or child documents
using compilation flags as described in \secref{sec:flags}
can be (permanently) stored in different files
for convenient compilation, viewing and distribution.
To this end, the package defines a command
to pass on compilation to a different file:

%%%%%%%%%%%%%%%%%%%%%%%%%%%%%%%%%%%%%%%%
\DescribeMacro{\childdocforward}
The command |\childdocforward| redirects processing to
another source file:
%
\begin{center}
\begin{tabular}{l}
|\input{childdoc.def}|\\
|\childdocforward[|\textit{main}|]{|\textit{dest}|}|\\
\end{tabular}
\end{center}
%
The argument \textit{dest} is the destination file
(without extension).
It should be the main file or one of the child files.
Note that further \textsf{childdoc} directives
such as |\childdocof| and |\childdocforward|
in the indicated file will be processed in this form.
The optional argument \textit{main}
passes on directly to the main file \textit{main}
while pretending to compile the child \textit{dest}.
This form behaves as if \textit{dest}
issues |\childdocof{|\textit{main}|}| right away,
and no further \textsf{childdoc} directives will be processed.

%%%%%%%%%%%%%%%%%%%%%%%%%%%%%%%%%%%%%%%%
\DescribeMacro{\...prefix}
In the alternative form |\childdocforwardprefix|,
%
\begin{center}
\begin{tabular}{l}
|\input{childdoc.def}|\\
|\childdocforwardprefix[|\textit{main}|]{|\textit{prefix}|}{|\textit{dest}|}|
\end{tabular}
\end{center}
%
the destination file is determined by a pattern
depending on the current file:
To make this work, the current file must be called
`{\textit{prefix}\hspace{0.2em}\textit{suffix}}'
with \textit{prefix} matching precisely the argument.
Processing is then passed on to the file
`{\textit{dest}\hspace{0.2em}\textit{suffix}}'.
Surely, the same effect is achieved by
directly specifying the
argument `{\textit{dest}\hspace{0.2em}\textit{suffix}}'
in the first form.
However, that requires to set up a different file
for each child. With the alternative form of the command
all these files can have exactly the same content
which simplifies setting them up and maintaining them.

For example, the following file |draft.tex|
with a compilation flag |\version| as described in \secref{sec:flags}
compiles the main document as a draft:
%
\begin{center}
\begin{tabular}{l}
|\def\version{draft}|\\
|\input{childdoc.def}|\\
|\childdocforward{|\textit{main}|}|
\end{tabular}
\end{center}
%
Likewise, the following files |final|\textit{nn}|.tex|
compile the final version of the child document
|child|\textit{nn}|.tex|:
%
\begin{center}
\begin{tabular}{l}
|\def\version{final}|\\
|\input{childdoc.def}|\\
|\childdocforwardprefix{final}{child}|
\end{tabular}
\end{center}
%

Note that when several versions of a main file and/or of each child file
are to be generated, it may be convenient to set up a |Makefile| or
shell script to automatise the process.

%%%%%%%%%%%%%%%%%%%%%%%%%%%%%%%%%%%%%%%%%%%%%%%%%%%%%%%%%%%%%%%%%%%%%%%%%%%%%%%%
\subsection{Command Line Processing}
\label{sec:commandline}

The effect of redirection files can also be achieved by invoking
the \LaTeX{} compiler with a more elaborate command line.
Most conveniently this should be done as part
of a shell script or a |Makefile|.

When using \textsf{childdoc} in the main file, the following
command lines effectively perform a redirection
(note that depending on the shell being used,
backslashes may have to be doubled: `|\|' $\to$ `|\\|'):
%
\begin{center}
|... -jobname "|\textit{target}|" |\\|"|[\textit{flags}]%
|\input{childdoc.def}\childdocforward[|\textit{main}|]{|\textit{dest}|}"|
\end{center}
%
Here \textit{target} is the name of the output file,
\textit{main} is the name of the main file
and \textit{dest} is the name of the main or child file to be processed
(all filenames without extensions).
The optional argument \textit{main} can be omitted
if \textit{main} matches \textit{dest}.
Optionally, compilation \textit{flags} can be defined via |\def| commands.
This command line makes the \TeX{} engine believe
it is compiling the file \textit{target}
whose content is specified as the latter parameter.
The provided code then forwards the processing to
\textit{main} or \textit{dest} as described in \secref{sec:forward}.

%%%%%%%%%%%%%%%%%%%%%%%%%%%%%%%%%%%%%%%%%%%%%%%%%%%%%%%%%%%%%%%%%%%%%%%%%%%%%%%%
\subsection{Include by Input}
\label{sec:input}

Including child documents by |\include| has some restrictions by design.
Most notably, the content of a child document always occupies
its own set of pages; pages cannot be shared between child documents.
Usually, this behaviour makes perfect sense
because each child document contain an essential part of the document.
However, in some situations it may be desirable to compose
a document from a collection of parts
without having mandatory page breaks between then.
For this case, the package
provides a mechanism to include parts
by |\input| which can also be processed individually.
However, by construction this mechanism
requires manual handling of the content to be output.

%%%%%%%%%%%%%%%%%%%%%%%%%%%%%%%%%%%%%%%%
\DescribeMacro{\ifchilddocmanual}
The main file should be prepared as usual, see \secref{sec:include}.
However, the document body must make a distinction
between processing of an individual part and of the main document, e.g.:
%
\begin{center}
\begin{tabular}{l}
|\ifchilddocmanual|\\
|\input{\childdocname}|\\
|\||else|\\
\textit{document body with }|\input{|\textit{part}|}|\\
|\||fi|
\end{tabular}
\end{center}
%
The conditional |\ifchilddocmanual| is true whenever
a part to be included by |\input| is being compiled,
and the name of the part is stored in |\childdocname|.

%%%%%%%%%%%%%%%%%%%%%%%%%%%%%%%%%%%%%%%%
\DescribeMacro{\childdocby}
Each part to be included by |\input| should start with:
%
\begin{center}
\begin{tabular}{l}
|\input{childdoc.def}|\\
|\childdocby{|\textit{main}|}|\\
\end{tabular}
\end{center}
%
The directive |\childdocby| is similar to |\childdocof|
described in \secref{sec:include},
but the subsequent selection of content must be done manually.
To that end, both |\ifchilddoc| and |\ifchilddocmanual|
will be true upon processing of a part,
and the name of the part is stored in |\childdocname|.
Note that |\jobname| will be set to the filename of the current part
so that each part receives an individual |.aux| file
that does not interfere with the |.aux| file(s) of the main document.
This behaviour can be altered by the alternative form
|\childdocby[*]{|\textit{main}|}| (with a non-empty optional argument)
which uses the |.aux| file of the main document
by setting |\jobname| to \textit{main}.

%%%%%%%%%%%%%%%%%%%%%%%%%%%%%%%%%%%%%%%%%%%%%%%%%%%%%%%%%%%%%%%%%%%%%%%%%%%%%%%%
\subsection{Driver Development}
\label{sec:driver}

The \textsf{childdoc} mechanism can also be use for the development
of definition files such as \LaTeX{} styles or classes.
This case differs from the above setup with multiple parts
included by |\include| in that no |\includeonly| should be invoked.
This can be achieved by starting the include file
(before |\ProvidesPackage|) with:
%
\begin{center}
\begin{tabular}{l}
|\input{childdoc.def}|\\
|\childdocforward{|\textit{main}|}|\\
\end{tabular}
\end{center}
%
or alternatively with:
%
\begin{center}
\begin{tabular}{l}
|\input{childdoc.def}|\\
|\childdocby{|\textit{main}|}|\\
\end{tabular}
\end{center}
%
Both forms have slightly different effects as described above.
The main file is prepared as usual, see \secref{sec:include}.

%%%%%%%%%%%%%%%%%%%%%%%%%%%%%%%%%%%%%%%%%%%%%%%%%%%%%%%%%%%%%%%%%%%%%%%%%%%%%%%%
\subsection{Legacy Detection}
\label{sec:detection}

The directive |\childdocmain| in the main file can detect
whether the complete document or merely a child is to be compiled
even without using the directive |\childdocof|.
This method is deprecated because it is less robust
and there is no compelling reason to use it;
it is merely provided for backward compatibility
and it may be removed in future versions.

If the detection mechanism is to be used,
it is mandatory to correctly specify
the filename of the main file as the argument of |\childdocmain|:
%
\begin{center}
\begin{tabular}{l}
|\input{childdoc.def}|\\
|\childdocmain{|\textit{main}|}|\\
\end{tabular}
\end{center}
%
If |\jobname| does not match the argument \textit{main} of |\childdocmain|,
it is assumed that |\jobname| points to the child file to be compiled.
When using |\childdocmain| with the main file specified as argument,
it suffices to start a child file
with just |\input{|\textit{main}|}|
without loading of the package and using |\childdocof|.
If instead all processing is done
with the appropriate \textsf{childdoc} directives,
the argument of \textit{main} of |\childdocmain| can be empty.

An alternative version of the command line processing described
in \secref{sec:commandline} using the detection mechanism reads:
%
\begin{center}
|... -jobname "|\textit{target}|" "|[\textit{flags}]%
[|\def\jobname{|\textit{dest}|}|]|\input{|\textit{main}|}"|
\end{center}

%%%%%%%%%%%%%%%%%%%%%%%%%%%%%%%%%%%%%%%%%%%%%%%%%%%%%%%%%%%%%%%%%%%%%%%%%%%%%%%%
\subsection{Manual Code}
\label{sec:manual}

In case one cannot be certain whether the definitions file |childdoc.def|
is installed on the target \TeX{} distribution
and one prefers not to ship it,
it is conceivable to paste a few relevant commands into the sources.

To that end, drop all statements |\input{childdoc.def}|
and perform the replacements as outlined below.
Instead of |\childdocmain{|\textit{main}|}| add the following code
to the top of the main file:
%
\begin{center}
\begin{tabular}{l}
|\||ifdefined\childdocname\endinput\||fi\newif\ifchilddoc|\\
|\edef\childdocname{\scantokens\expandafter{\jobname\noexpand}}|\\
|\def\childdocmain{|\textit{main}|}\||ifx\childdocmain\childdocname\||else|\\
|\childdoctrue\includeonly{\childdocname}\let\jobname\childdocmain\||fi|\\
\end{tabular}
\end{center}
%
Instead of |\childdocof{|\textit{main}|}| just include the main file
at the top of each child file:
%
\begin{center}
|\input{|\textit{main}|}|
\end{center}
%
A simple redirection |\childdocforward{|\textit{dest}|}| is achieved by:
%
\begin{center}
|\def\jobname{|\textit{dest}|}\input{\jobname}|
\end{center}
%
The redirection with prefix
|\childdocforwardprefix[|\textit{prefix}|]{|\textit{dest}|}|
is accomplished by:
%
\begin{center}
\begin{tabular}{l}
|{\edef\jobname{\scantokens\expandafter{\jobname\noexpand}}|\\
|\def\redirectjob |\textit{prefix}|#1~~~{\gdef\jobname{|\textit{dest}|#1}}|\\
|\expandafter\redirectjob\jobname~~~}\input{\jobname}|
\end{tabular}
\end{center}

In an alternative approach,
child documents can be compiled by a specific command line
without additional code or specific definitions:
%
\begin{center}
|... -jobname "|\textit{target}|" "|[\textit{flags}]%
|\includeonly{|\textit{dest}|}\input{|\textit{main}|}"|
\end{center}
%

%%%%%%%%%%%%%%%%%%%%%%%%%%%%%%%%%%%%%%%%%%%%%%%%%%%%%%%%%%%%%%%%%%%%%%%%%%%%%%%%
%%%%%%%%%%%%%%%%%%%%%%%%%%%%%%%%%%%%%%%%%%%%%%%%%%%%%%%%%%%%%%%%%%%%%%%%%%%%%%%%
\section{Information}

%%%%%%%%%%%%%%%%%%%%%%%%%%%%%%%%%%%%%%%%%%%%%%%%%%%%%%%%%%%%%%%%%%%%%%%%%%%%%%%%
\subsection{Copyright}

Copyright \copyright{} 2017--2018 Niklas Beisert

This work may be distributed and/or modified under the
conditions of the \LaTeX{} Project Public License, either version 1.3
of this license or (at your option) any later version.
The latest version of this license is in
  \url{http://www.latex-project.org/lppl.txt}
and version 1.3 or later is part of all distributions of \LaTeX{}
version 2005/12/01 or later.

This work has the LPPL maintenance status `maintained'.

The Current Maintainer of this work is Niklas Beisert.

This work consists of the files |README.txt|, |childdoc.ins| and |childdoc.dtx|
as well as the derived files |childdoc.def|, |cdocsamp.tex|
with |cdocsch1.tex|, |cdocsch2.tex|, |cdocspt3.tex|, |cdocspt4.tex|,
|cdocsdrf.tex|, |cdocsfn1.tex|, |cdocsfn2.tex|
as well as |childdoc.pdf|.

%%%%%%%%%%%%%%%%%%%%%%%%%%%%%%%%%%%%%%%%%%%%%%%%%%%%%%%%%%%%%%%%%%%%%%%%%%%%%%%%
\subsection{Files and Installation}

The package consists of the files:
%
\begin{center}
\begin{tabular}{ll}
    |README.txt|   & readme file \\
    |childdoc.ins| & installation file \\
    |childdoc.dtx| & source file \\
    |childdoc.def| & definition file \\
    |cdocsamp.tex| & sample main file \\
    |cdocsch1.tex| & sample include file \\
    |cdocsch2.tex| & sample include file \\
    |cdocspt3.tex| & sample part file \\
    |cdocspt4.tex| & sample part file \\
    |cdocsdrf.tex| & sample redirection file \\
    |cdocsfn1.tex| & sample redirection file \\
    |cdocsfn2.tex| & sample redirection file \\
    |childdoc.pdf| & manual
\end{tabular}
\end{center}
%
The distribution consists of the files
|README.txt|, |childdoc.ins| and |childdoc.dtx|.
%
\begin{itemize}
\item
Run (pdf)\LaTeX{} on |childdoc.dtx|
to compile the manual |childdoc.pdf| (this file).
\item
Run \LaTeX{} on |childdoc.ins| to create the definitions file |childdoc.def|
and the sample |cdocsamp.tex| with include files
|cdocsch1.tex|, |cdocsch2.tex|, |cdocspt3.tex|, |cdocspt4.tex|,
|cdocsdrf.tex|, |cdocsfn1.tex|, |cdocsfn2.tex|.
Then copy the file |childdoc.def| to an appropriate directory of your \LaTeX{}
distribution, e.g.\ \textit{texmf-root}|/tex/latex/childdoc|.
\end{itemize}

%%%%%%%%%%%%%%%%%%%%%%%%%%%%%%%%%%%%%%%%%%%%%%%%%%%%%%%%%%%%%%%%%%%%%%%%%%%%%%%%
\subsection{Related CTAN Packages}

There are several other packages which offer a similar functionality:
%
\begin{itemize}
\item
The packages
\href{http://ctan.org/pkg/docmute}{\textsf{docmute}},
\href{http://ctan.org/pkg/includex}{\textsf{includex}} and
\href{http://ctan.org/pkg/standalone}{\textsf{standalone}}
provide commands to include only the document body of
a child file thus allowing both files to be compiled individually.
\item
The packages \href{http://ctan.org/pkg/subdocs}{\textsf{subdocs}}
and \href{http://ctan.org/pkg/subfiles}{\textsf{subfiles}}
provide structures in which the main and child documents can be
encapsulated and allowing them to be compiled individually.
The inclusion mechanism is different from the conventional |\include|.
\item
The package \href{http://ctan.org/pkg/combine}{\textsf{combine}}
is an elaborate solution to combine several documents into one.
\end{itemize}
%
See also the CTAN topic \href{http://ctan.org/topic/subdocs}{\textsf{subdocs}}
for further related packages.
The present package differs from the above solutions in that
a document structure constructed with the conventional |\include| mechanism
just needs two extra commands at the top of every file
such that all constituent files can be compiled individually.

%%%%%%%%%%%%%%%%%%%%%%%%%%%%%%%%%%%%%%%%%%%%%%%%%%%%%%%%%%%%%%%%%%%%%%%%%%%%%%%%
%\subsection{Feature Suggestions}
%
%The following is a list of features which may be useful for future
%versions of this package:
%%
%\begin{itemize}
%\item
%\ldots
%\end{itemize}

%%%%%%%%%%%%%%%%%%%%%%%%%%%%%%%%%%%%%%%%%%%%%%%%%%%%%%%%%%%%%%%%%%%%%%%%%%%%%%%%
\subsection{Revision History}

%%%%%%%%%%%%%%%%%%%%%%%%%%%%%%%%%%%%%%%%
\paragraph{v2.0:} 2018/12/30

\begin{itemize}
\item
immediate forward processing
\item
added |\childdocby| mechanism
\item
manual restructured
\end{itemize}

%%%%%%%%%%%%%%%%%%%%%%%%%%%%%%%%%%%%%%%%
\paragraph{v1.6:} 2018/01/17

\begin{itemize}
\item
application for development of include files
\item
corrections to manual
\end{itemize}

%%%%%%%%%%%%%%%%%%%%%%%%%%%%%%%%%%%%%%%%
\paragraph{v1.5:} 2017/05/21

\begin{itemize}
\item
more complete structuring introduced
\item
|\childdocof| introduced
\item
|\childdoc| renamed to |\childdocmain|
\item
|\childredirect| renamed to |\childdocforward| and |\childdocforwardprefix|
and functionality expanded
\end{itemize}

%%%%%%%%%%%%%%%%%%%%%%%%%%%%%%%%%%%%%%%%
\paragraph{v1.0:} 2017/04/27

\begin{itemize}
\item
manual and install package
\item
first version published on CTAN
\end{itemize}

%%%%%%%%%%%%%%%%%%%%%%%%%%%%%%%%%%%%%%%%
\paragraph{v0.6:} 2017/04/26

\begin{itemize}
\item
redirection mechanism added
\end{itemize}

%%%%%%%%%%%%%%%%%%%%%%%%%%%%%%%%%%%%%%%%
\paragraph{v0.5:} 2017/04/26

\begin{itemize}
\item
functionality in definition file
\end{itemize}


%%%%%%%%%%%%%%%%%%%%%%%%%%%%%%%%%%%%%%%%%%%%%%%%%%%%%%%%%%%%%%%%%%%%%%%%%%%%%%%%
%%%%%%%%%%%%%%%%%%%%%%%%%%%%%%%%%%%%%%%%%%%%%%%%%%%%%%%%%%%%%%%%%%%%%%%%%%%%%%%%
%%%%%%%%%%%%%%%%%%%%%%%%%%%%%%%%%%%%%%%%%%%%%%%%%%%%%%%%%%%%%%%%%%%%%%%%%%%%%%%%
\appendix

\settowidth\MacroIndent{\rmfamily\scriptsize 000\ }

 \DocInput{childdoc.dtx}

\end{document}
%</driver>
% \fi
%
% %%%%%%%%%%%%%%%%%%%%%%%%%%%%%%%%%%%%%%%%%%%%%%%%%%%%%%%%%%%%%%%%%%%%%%%%%%%%%%
% %%%%%%%%%%%%%%%%%%%%%%%%%%%%%%%%%%%%%%%%%%%%%%%%%%%%%%%%%%%%%%%%%%%%%%%%%%%%%%
% \section{Sample}
%\iffalse
%<*samplemain>
%\fi
%
% The following presents a sample document
% with two chapters, two parts, a title page,
% a compile flag as well as three forwarding files to set the flag.
% It consists of eight |.tex| files:
% \begin{center}
% \begin{tabular}{ll}
% |cdocsamp.tex|&main file\\
% |cdocsch1.tex|&include file for chapter 1\\
% |cdocsch2.tex|&include file for chapter 2\\
% |cdocspt3.tex|&include file for part 3\\
% |cdocspt4.tex|&include file for part 4\\
% |cdocsdrf.tex|&forwarding file for main file in draft mode\\
% |cdocsfi1.tex|&forwarding file for final version of chapter 1\\
% |cdocsfi2.tex|&forwarding file for final version of chapter 2\\
% \end{tabular}
% \end{center}
% Each of the eight files can be compiled directly by the \LaTeX{} compiler.
%
% %%%%%%%%%%%%%%%%%%%%%%%%%%%%%%%%%%%%%%
% \paragraph{Main File.}
%
% The main file is called |cdocsamp.tex|.
%
% Load the \textsf{childdoc} definitions and
% declare the filename for the main document:
%    \begin{macrocode}
\input{childdoc.def}
\childdocmain{}
%    \end{macrocode}

% Optional override for |\version| flag:
%    \begin{macrocode}
%%\ifchilddoc\else\providecommand{\version}{draft}\fi
%    \end{macrocode}

% Define the default values for the |\version| flag
% (|final| for the main file and |draft| for childs):
%    \begin{macrocode}
\ifchilddoc
\providecommand{\version}{draft}
\else
\providecommand{\version}{final}
\fi
%    \end{macrocode}

% Load the standard document class:
%    \begin{macrocode}
\documentclass[12pt]{article}
%    \end{macrocode}

% Start the document body:
%    \begin{macrocode}
\begin{document}
%    \end{macrocode}

% Declare a title page.
% Print title, part of document being processed and version flag:
%    \begin{macrocode}
\addtocounter{page}{-1}
\begin{center}
{\LARGE\bfseries{}childdoc example\par}
\vspace{1cm}
\ifchilddoc
\ifchilddocmanual part\else chapter\fi:
`\childdocname' of `\childdocjob'\par
\else
main document: `\childdocjob'\par
\fi
version: \version\par
\end{center}
\newpage
%    \end{macrocode}

% Manually include selected file,
% otherwise process as usual:
%    \begin{macrocode}
\ifchilddocmanual
\section*{part `\childdocname'}
\input{\childdocname}
\else
%    \end{macrocode}

% Include the two chapters:
%    \begin{macrocode}
\include{cdocsch1}
\include{cdocsch2}
%    \end{macrocode}

% Include the two parts unless only chapters should be displayed:
%    \begin{macrocode}
\ifchilddoc\else
\section{part three}
\input{cdocspt3}
\section{part four}
\input{cdocspt4}
\fi
%    \end{macrocode}

% Process as usual until here:
%    \begin{macrocode}
\fi
%    \end{macrocode}

% End of document body:
%    \begin{macrocode}
\end{document}
%    \end{macrocode}
%\iffalse
%</samplemain>
%\fi
%
% %%%%%%%%%%%%%%%%%%%%%%%%%%%%%%%%%%%%%%
% \paragraph{Chapter Include Files.}
%
% The include files are called |cdocsch1.tex| and |cdocsch2.tex|.
%
%\iffalse
%<*samplechap1|samplechap2>
%\fi

% Optional override for |\version| flag:
%    \begin{macrocode}
%%\providecommand{\version}{final}
%    \end{macrocode}

% Include the main document:
%    \begin{macrocode}
\input{childdoc.def}
\childdocof{cdocsamp}
%    \end{macrocode}

%\iffalse
%</samplechap1|samplechap2>
%\fi
%
%\iffalse
%<*samplechap1>
%\fi
% Some text for chapter 1:
%    \begin{macrocode}
\section{one}
some text in chapter one
%    \end{macrocode}

%\iffalse
%</samplechap1>
%\fi
% Some text for chapter 2:
%\iffalse
%<*samplechap2>
%\fi
%    \begin{macrocode}
\section{two}
more text in chapter two
%    \end{macrocode}

%\iffalse
%</samplechap2>
%\fi
%
% %%%%%%%%%%%%%%%%%%%%%%%%%%%%%%%%%%%%%%
% \paragraph{Part Include Files.}
%
% The include files are called |cdocspt3.tex| and |cdocspt4.tex|.
%
%\iffalse
%<*samplepart3|samplepart4>
%\fi

% Optional override for |\version| flag:
%    \begin{macrocode}
%%\providecommand{\version}{final}
%    \end{macrocode}

% Include the main document:
%    \begin{macrocode}
\input{childdoc.def}
\childdocby{cdocsamp}
%    \end{macrocode}

%\iffalse
%</samplepart3|samplepart4>
%\fi
%
%\iffalse
%<*samplepart3>
%\fi
% Some text for part 3:
%    \begin{macrocode}
some text in part three
%    \end{macrocode}

%\iffalse
%</samplepart3>
%\fi
% Some text for part 4:
%\iffalse
%<*samplepart4>
%\fi
%    \begin{macrocode}
more text in part four
%    \end{macrocode}

%\iffalse
%</samplepart4>
%\fi
%
% %%%%%%%%%%%%%%%%%%%%%%%%%%%%%%%%%%%%%%
% \paragraph{Forwarding for a Complete Draft.}
%
% The following forwarding file |cdocsdrf.tex|
% compiles the main document in draft mode:
%\iffalse
%<*sampledraft>
%\fi
%    \begin{macrocode}
\def\version{draft}
\input{childdoc.def}
\childdocforward{cdocsamp}
%    \end{macrocode}

%\iffalse
%</sampledraft>
%\fi
%
% %%%%%%%%%%%%%%%%%%%%%%%%%%%%%%%%%%%%%%
% \paragraph{Forwarding for Final Version of the Chapters.}
%
% The following forwarding files |cdocsfn1.tex| and |cdocsfn2.tex|
% (with identical content)
% compile the final versions of the child documents
% |cdocsch1.tex| and |cdocsch2.tex|, respectively:
%\iffalse
%<*samplefinal>
%\fi
%    \begin{macrocode}
\def\version{final}
\input{childdoc.def}
\childdocforwardprefix[cdocsamp]{cdocsfn}{cdocsch}
%    \end{macrocode}

%\iffalse
%</samplefinal>
%\fi
%
% %%%%%%%%%%%%%%%%%%%%%%%%%%%%%%%%%%%%%%
% \paragraph{Command Line Processing.}
%
% The following three command lines generate the output files
% |cdocscld|, |cdocscl1| and |cdocscl2|
% which should be identical to
% |cdocsdrf|, |cdocsch1| and |cdocsfn2|, respectively:
% \begin{center}
% \begin{tabular}{l}
% |latex -jobname cdocscld \|\\
% |  "\def\version{draft}\input{childdoc.def}\childdocforward{cdocsamp}"|\\
% |latex -jobname cdocscl1 \|\\
% |  "\input{childdoc.def}\childdocforward[cdocsamp]{cdocsch1}"|\\
% |latex -jobname cdocscl2 \|\\
% |  "\def\version{final}\input{childdoc.def}\childdocforward{cdocsch2}"|
% \end{tabular}
% \end{center}
% Note that the trailing backslash on each first line
% merely continues the input to the second line
% (for convenient cut ant paste).
% Furthermore, the command |latex| can be replaced by any
% of its alternative versions such as |pdflatex|.
%
% %%%%%%%%%%%%%%%%%%%%%%%%%%%%%%%%%%%%%%%%%%%%%%%%%%%%%%%%%%%%%%%%%%%%%%%%%%%%%%
% %%%%%%%%%%%%%%%%%%%%%%%%%%%%%%%%%%%%%%%%%%%%%%%%%%%%%%%%%%%%%%%%%%%%%%%%%%%%%%
% \section{Implementation}
%\iffalse
%<*package>
%\fi
%
% This section describes the definitions file |childdoc.def|.

% The definitions cannot be loaded using |\usepackage| or |\RequirePackage|
% which has a mechanism to prevent loading a style file more than once.
% When loading the definitions by means of |\input|
% multiple instances have to be prevented manually:
%\iffalse
%This code needs to be before the `\ProvidesFile' directive
%which is defined at the beginning of this file.
%Therefore it is also placed there and commented out here.
%</package>
%<*discard>
%\fi
%    \begin{macrocode}
\ifdefined\childdocmain\endinput\fi
%    \end{macrocode}
%\iffalse
%</discard>
%<*package>
%\fi
%
% \macro{\ifchilddoc}
% \macro{\ifchilddocmanual}
% The conditional |\ifchilddoc| tells whether a
% child (true) or main (false) document is being compiled.
% The conditional |\ifchilddocmanual| tells whether
% the |\includeonly| mechanism is used (false) or
% the selection of child files must be performed manually (true).
% The definitions initialise to false:
%    \begin{macrocode}
\newif\ifchilddoc
\newif\ifchilddocmanual
%    \end{macrocode}

% \macro{\childdocname}
% \macro{\childdocjob}
% The macro |\childdocname| stores the name of the main document
% to be compiled. The macro |\childdocjob| stores the name of
% the document on which the \LaTeX{} compiler was originally invoked.
% The content of |\jobname| cannot be compared
% to filenames specified in the source due to different catcodes.
% The following code rescans |\jobname|, stores the result
% in |\childdocname| and saves a copy in |\childdocjob|:
%    \begin{macrocode}
\edef\childdocname{\scantokens\expandafter{\jobname\noexpand}}
\let\childdocjob\childdocname
%    \end{macrocode}

% \macro{\childdocdisable}
% The macro |\childdocdisable| prevents the main file
% from being processed more than once.
% At this stage, the main document command |\childdocmain|
% is assumed to be called once again where it should do nothing.
% Any subsequent call to it should prevent
% a secondary processing of the main document
% It overwrites the forwarding commands
% |\childdocof| and |\childdocforward|
% with empty macros to prevent further inclusions of the main document:
%    \begin{macrocode}
\newcommand{\childdocdisable}
{
  \renewcommand{\childdocmain}[1]{\renewcommand{\childdocmain}[1]{\endinput}}
  \renewcommand{\childdocof}[1]{}
  \renewcommand{\childdocby}[2][]{}
  \renewcommand{\childdocforward}[2][]{}
  \renewcommand{\childdocdisable}{}
}
%    \end{macrocode}

% \macro{\childdocmain}
% The macro |\childdocmain| is to be called at the top of the main file
% with nothing or the main filename (without extension) as argument.
% First, it breaks loops.
% If the argument is not empty and does not match |\childdocname|
% (which is set by the first inclusion of |childdoc.def|),
% |\ifchilddoc| is set to true, |\includeonly| is applied to the child file
% and |\jobname| is set to the main file
% (for proper handling of |.aux| files):
%    \begin{macrocode}
\newcommand{\childdocmain}[1]
{
  \childdocdisable\childdocmain{}
  \if?#1?\else
    \begingroup
      \def\childdoctmp{#1}
      \ifx\childdoctmp\childdocname
        \def\childdoctmp{}
      \else
        \def\childdoctmp
        {
          \childdoctrue
          \includeonly{\childdocname}
          \def\childdocjob{#1}
          \def\jobname{#1}
        }
      \fi
      \expandafter
    \endgroup
    \childdoctmp
  \fi
}
%    \end{macrocode}

% \macro{\childdocof}
% The command |\childdocof| redirects
% compilation to the main file |#1|.
%    \begin{macrocode}
\newcommand{\childdocof}[1]
{
  \childdocdisable
  \childdoctrue
  \includeonly{\childdocname}
  \def\jobname{#1}
  \def\childdocjob{#1}
  \input{#1}
}
%    \end{macrocode}

% \macro{\childdocby}
% The command |\childdocby| ....
%    \begin{macrocode}
\newcommand{\childdocby}[2][]
{
  \childdocdisable
  \childdoctrue
  \childdocmanualtrue
  \if?#1?\else
    \def\jobname{#2}
  \fi
  \def\childdocjob{#2}
  \input{#2}
  \endinput
}
%    \end{macrocode}

% \macro{\childdocforward}
% The command |\childdocforward| redirects
% compilation to the main file or
% (if the optional argument is given) a child file.
% Parameters are set as if the main file
% or a child file starting with |\childdocof| was compiled.
% Then compilation is handed over to the main file:
%    \begin{macrocode}
\newcommand{\childdocforward}[2][]
{
  \begingroup
    \if?#1?
      \def\childdoctmp
      {
        \def\childdocname{#2}
        \def\childdocjob{#2}
        \def\jobname{#2}
        \input{#2}
        \endinput
      }
    \else
      \def\childdoctmp
      {
        \childdocdisable
        \def\childdocname{#2}
        \childdoctrue
        \includeonly{#2}
        \def\childdocjob{#1}
        \def\jobname{#1}
        \input{#1}
        \endinput
      }
    \fi
    \expandafter
  \endgroup
  \childdoctmp
}
%    \end{macrocode}

% \macro{\childdocforwardprefix}
% The command |\childdocforwardprefix| redirects
% compilation to the main or a child file by means of a pattern.
% The prefix |#1| in the current filename is replaced by |#2|
% and the suffix of the current filename is kept
% (it is assumed that the filename does not contain the substring `|~~~|'
% which is used as a delimiter).
% Compilation is handed over to the new file by |\childdocforward|:
%    \begin{macrocode}
\newcommand{\childdocforwardprefix}[3][]
{
  \begingroup
    \def\childdocextract #2##1~~~{\def\childdoctmp{\childdocforward[#1]{#3##1}}}
    \expandafter\childdocextract\childdocname~~~
    \expandafter
  \endgroup
  \childdoctmp
}
%    \end{macrocode}

% \macro{\childdoc}
% The deprecated macro |\childdoc| is a legacy version of |\childdocmain|:
%    \begin{macrocode}
\newcommand{\childdoc}{\childdocmain}
%    \end{macrocode}

% \macro{\childdocredirect}
% The deprecated macro |\childdocredirect| is a legacy version
% of |\childdocforward| and |\childdocforwardprefix|:
%    \begin{macrocode}
\newcommand{\childdocredirect}[2][]
{
  \begingroup
    \if?#1?
      \def\childdoctmp{\childdocforward{#2}}
    \else
      \def\childdoctmp{\childdocforwardprefix{#1}{#2}}
    \fi
    \expandafter
  \endgroup
  \childdoctmp
}
%    \end{macrocode}

%\iffalse
%</package>
%\fi
%
\endinput
|\\
|\childdocby{|\textit{main}|}|\\
\end{tabular}
\end{center}
%
Both forms have slightly different effects as described above.
The main file is prepared as usual, see \secref{sec:include}.

%%%%%%%%%%%%%%%%%%%%%%%%%%%%%%%%%%%%%%%%%%%%%%%%%%%%%%%%%%%%%%%%%%%%%%%%%%%%%%%%
\subsection{Legacy Detection}
\label{sec:detection}

The directive |\childdocmain| in the main file can detect
whether the complete document or merely a child is to be compiled
even without using the directive |\childdocof|.
This method is deprecated because it is less robust
and there is no compelling reason to use it;
it is merely provided for backward compatibility
and it may be removed in future versions.

If the detection mechanism is to be used,
it is mandatory to correctly specify
the filename of the main file as the argument of |\childdocmain|:
%
\begin{center}
\begin{tabular}{l}
|% \iffalse
%
% childdoc.dtx Copyright (C) 2017-2018 Niklas Beisert
%
% This work may be distributed and/or modified under the
% conditions of the LaTeX Project Public License, either version 1.3
% of this license or (at your option) any later version.
% The latest version of this license is in
%   http://www.latex-project.org/lppl.txt
% and version 1.3 or later is part of all distributions of LaTeX
% version 2005/12/01 or later.
%
% This work has the LPPL maintenance status `maintained'.
%
% The Current Maintainer of this work is Niklas Beisert.
%
% This work consists of the files childdoc.dtx and childdoc.ins
% and the derived files childdoc.def and cdocsamp.tex with
% cdocsch1.tex, cdocsch2.tex, cdocsdrf.tex, cdocsfn1.tex, cdocsfn2.tex.
%
%<package>\ifdefined\childdocmain\endinput\fi
%<package>\ProvidesFile{childdoc.def}[2018/12/30 v2.0 child document driver]
%<samplemain>\ProvidesFile{cdocsamp.tex}[2018/12/30 v2.0 sample for childdoc]
%<*driver>
%\ProvidesFile{childdoc.drv}[2018/12/30 v2.0 childdoc reference manual file]
\PassOptionsToClass{10pt,a4paper}{article}
\documentclass{ltxdoc}

\usepackage[margin=35mm]{geometry}
\usepackage{hyperref}
\usepackage{hyperxmp}
\usepackage[usenames]{color}

\hypersetup{colorlinks=true}
\hypersetup{pdfstartview=FitH}
\hypersetup{pdfpagemode=UseNone}
\hypersetup{pdfsource={}}
\hypersetup{pdflang={en-UK}}
\hypersetup{pdfcopyright={Copyright 2017-2018 Niklas Beisert.
  This work may be distributed and/or modified under the
  conditions of the LaTeX Project Public License, either version 1.3
  of this license or (at your option) any later version.}}
\hypersetup{pdflicenseurl={http://www.latex-project.org/lppl.txt}}
\hypersetup{pdfcontactaddress={ETH Zurich, ITP, HIT K,
  Wolfgang-Pauli-Strasse 27}}
\hypersetup{pdfcontactpostcode={8093}}
\hypersetup{pdfcontactcity={Zurich}}
\hypersetup{pdfcontactcountry={Switzerland}}
\hypersetup{pdfcontactemail={nbeisert@itp.phys.ethz.ch}}
\hypersetup{pdfcontacturl={http://people.phys.ethz.ch/\xmptilde nbeisert/}}

\newcommand{\secref}[1]{\hyperref[#1]{section \ref*{#1}}}

\parskip1ex
\parindent0pt
\let\olditemize\itemize
\def\itemize{\olditemize\parskip0pt}

\begin{document}

\title{The \textsf{childdoc} Package}
\hypersetup{pdftitle={The childdoc Package}}
\author{Niklas Beisert\\[2ex]
  Institut f\"ur Theoretische Physik\\
  Eidgen\"ossische Technische Hochschule Z\"urich\\
  Wolfgang-Pauli-Strasse 27, 8093 Z\"urich, Switzerland\\[1ex]
  \href{mailto:nbeisert@itp.phys.ethz.ch}
  {\texttt{nbeisert@itp.phys.ethz.ch}}}
\hypersetup{pdfauthor={Niklas Beisert}}
\hypersetup{pdfsubject={Manual for the LaTeX2e Package childdoc}}
\date{30 December 2018, \textsf{v2.0}}
\maketitle

\begin{abstract}\noindent
\textsf{childdoc} is a \LaTeXe{} package
that enables the direct compilation
of document sections included by |\include|
to individual files.
\end{abstract}

\begingroup
\parskip0ex
\tableofcontents
\endgroup

%%%%%%%%%%%%%%%%%%%%%%%%%%%%%%%%%%%%%%%%%%%%%%%%%%%%%%%%%%%%%%%%%%%%%%%%%%%%%%%%
%%%%%%%%%%%%%%%%%%%%%%%%%%%%%%%%%%%%%%%%%%%%%%%%%%%%%%%%%%%%%%%%%%%%%%%%%%%%%%%%
\section{Introduction}

\LaTeX{} provides a mechanism to structure a large document (such as a book)
into a main file and several child files (containing the chapters)
using the |\include| command.
This mechanism is beneficial for documents
which span hundreds of pages in order to
make the source file(s) more manageable.
Moreover, compilation can be restricted to
selected child files by means of the |\includeonly| command.
The latter feature can be used to reduce the compilation time while editing
(this was significantly more useful in the earlier days of \LaTeX{})
or to generate a smaller document which is easier to navigate.
Another application of |\includeonly| is to generate
documents consisting of selected parts of the complete document.

However, there are a few drawbacks of the plain |\include| mechanism:
\begin{itemize}
\item
The child files cannot be compiled on their own,
they can only be compiled via the main file.
A naive editing environment
(such as a text editor with an option
to have the current file processed by \LaTeX)
may require one to switch to the main file before compiling;
attempting to compile the child file produces errors.
\item
The main file must be modified (each time)
to adjust the |\includeonly| command
to the present needs. This easily leaves the main file in a messy state.
\item
The generated document will always carry the filename
of the main document. This is inconvenient if
several child files are to be compiled and
to be kept for distribution.
\end{itemize}

The present package provides a simple interface
to make child files individually compilable by \LaTeX{}.
Compiling a child file then has the same effect as compiling
the main file with an |\includeonly| command
to select the appropriate child.
Moreover the generated document will carry the name of the child
rather than the main file.
This resolves all three above issues.

This feature is meant to make the editing of books,
thesis documents and lecture notes somewhat more convenient.
However, the package can also be used efficiently for
composing a series of documents (such as exercise sheets)
which are typically distributed individually.
It then assists the author in generating the individual documents
(potentially in different versions)
as well as a document containing the collected series.
Another application is in developing style files
or other kinds of included material
where compilation of the style file could redirect
to a sample or test file.

%%%%%%%%%%%%%%%%%%%%%%%%%%%%%%%%%%%%%%%%%%%%%%%%%%%%%%%%%%%%%%%%%%%%%%%%%%%%%%%%
%%%%%%%%%%%%%%%%%%%%%%%%%%%%%%%%%%%%%%%%%%%%%%%%%%%%%%%%%%%%%%%%%%%%%%%%%%%%%%%%
\section{Usage}

First of all, the package \textsf{childdoc} is \emph{not} a standard
\LaTeXe{} |.sty| style file! Therefore it needs to be invoked in
a non-standard way.

%%%%%%%%%%%%%%%%%%%%%%%%%%%%%%%%%%%%%%%%%%%%%%%%%%%%%%%%%%%%%%%%%%%%%%%%%%%%%%%%
\subsection{Included Files}
\label{sec:include}

%%%%%%%%%%%%%%%%%%%%%%%%%%%%%%%%%%%%%%%%
\DescribeMacro{\childdocmain}
To use the package, add the commands
\begin{center}
\begin{tabular}{l}
|\input{childdoc.def}|\\
|\childdocmain{}|\\
\end{tabular}
\end{center}
at the very top of the main \LaTeX{} file,
in particular \emph{before} the |\documentclass| statement!
The argument of |\childdocmain| should be left empty
(but it must be present).

%%%%%%%%%%%%%%%%%%%%%%%%%%%%%%%%%%%%%%%%
\DescribeMacro{\childdocof}
Furthermore, add the commands
\begin{center}
\begin{tabular}{l}
|\input{childdoc.def}|\\
|\childdocof{|\textit{main}|}|\\
\end{tabular}
\end{center}
at the top of every child file \textit{child}
which is included by |\include{|\textit{child}|}|
from within the main file
(or at least for those files to be compiled individually).
The argument \textit{main} must be the filename of the main file.

There are a couple of
considerations in setting up the main and child documents:

%%%%%%%%%%%%%%%%%%%%%%%%%%%%%%%%%%%%%%%%
\paragraph{Restrictions.}

Please note the following restrictions:
\begin{itemize}
\item
|\childdocmain| must be called with one argument \textit{main}
to ensure compatibility with earlier version of the package.
It must either be empty (|\childdocmain{}|)
or precisely match the filename of the main file in which it is specified.
See \secref{sec:detection} for further information.
\item
The filename \textit{main} must be specified without the |.tex| extension.
\item
The filename \textit{main} is case sensitive
(even in case-insensitive file systems)
due to internal string comparison.
\item
The argument \textit{main} should be fully expanded, it cannot be a macro.
\item
Subdirectories and special characters should be avoided in filenames.
\item
The command |\childdocmain{|\textit{main}|}| must be followed by a whitespace.
It should not be followed immediately by another command
or by a comment mark `|%|'.
This is because the \TeX{} parser reads the token immediately following
the argument of |\childdocmain| and puts it
at the beginning of every child section;
however, a white\-space is ignored.
\end{itemize}

%%%%%%%%%%%%%%%%%%%%%%%%%%%%%%%%%%%%%%%%
\paragraph{Content of Main File.}

It is advisable to place all content in the child files included by |\include|.
Any output contained in the main file will appear in all child documents
unless suppressed manually;
it cannot be suppressed automatically by the |\includeonly| directive
and thus should normally be avoided.
A method to include some content in the main file
by means of conditional processing is described in \secref{sec:conditional}.

%%%%%%%%%%%%%%%%%%%%%%%%%%%%%%%%%%%%%%%%
\paragraph{Page Numbering.}

When only a part of the document is compiled,
the appropriate numbering of pages
(as well as other status parameters)
is determined from the |.aux| files.
The latter contain information from previous passes.
However this information needs to propagate through
all intermediate child documents.
Therefore the page numbering in child documents may well
be inconsistent until the complete document is compiled at least once.

A useful (if unconventional) way to always ensure a consistent
page numbering is to restart the numbering in each child document
and denote the pages by `\textit{child}|.|\textit{page}'
where \textit{child} represents the chapter/section number of the child file.
This can be achieved by the command
|\numberwithin{page}{|\textit{child}|}|
of the \textsf{amsmath} package
where \textit{child} can be |chapter| or |section|
depending on the chosen structuring.
Alternatively, one can modify the macro |\thepage| appropriately
and reset the counter |page| at the start of each child file.

%%%%%%%%%%%%%%%%%%%%%%%%%%%%%%%%%%%%%%%%%%%%%%%%%%%%%%%%%%%%%%%%%%%%%%%%%%%%%%%%
\subsection{Conditional Processing}
\label{sec:conditional}

The package provides a mechanism to compile different versions
of a document. To customise the versions further some conditional processing
can come in handy to distinguish which version is being compiled.
The package provides two macros to describe the compilation context:

%%%%%%%%%%%%%%%%%%%%%%%%%%%%%%%%%%%%%%%%
\DescribeMacro{\ifchilddoc}
The conditional |\ifchilddoc| distinguishes between the compilation of
child documents and the main document:
%
\begin{center}
|\ifchilddoc |\textit{child-code}| |[|\||else |\textit{main-code}]| \||fi|
\end{center}

%%%%%%%%%%%%%%%%%%%%%%%%%%%%%%%%%%%%%%%%
\DescribeMacro{\childdocname}
\DescribeMacro{\childdocjob}
The macro |\childdocname| contains the filename (without extension)
of the main or child file being processed.
Note that |\childdocjob| will always contain the name of the main file.

%%%%%%%%%%%%%%%%%%%%%%%%%%%%%%%%%%%%%%%%
\paragraph{Title Page.}

Conditional processing can be used to include a title or banner page
in the main document when proper precautions are taken.
Importantly, the code in the main file should ensure that the page counter
(as well as other status parameters which are stored in the |.aux| files)
takes the same value after the conditional processing.
Otherwise the page numbers may take divergent values
depending on which part is compiled.

For example, a title page could be declared by:
%
\begin{center}
\begin{tabular}{l}
|\ifchilddoc\||else|\\
|\addtocounter{page}{-1}|\\
\textit{code for title page}\\
|\newpage|\\
|\||fi|
\end{tabular}
\end{center}
%
A banner page for the child documents can be generated by:
%
\begin{center}
\begin{tabular}{l}
|\ifchilddoc|\\
|\addtocounter{page}{-1}|\\
\textit{code for banner page}\\
|\newpage|\\
|\||fi|
\end{tabular}
\end{center}
%
Here one could write a message such as:
\begin{center}
|This is the part \childdocname{} of \childdocjob{}.|
\end{center}

%%%%%%%%%%%%%%%%%%%%%%%%%%%%%%%%%%%%%%%%%%%%%%%%%%%%%%%%%%%%%%%%%%%%%%%%%%%%%%%%
\subsection{Flags}
\label{sec:flags}

The package makes it easy to generate different versions
of the main or child documents.
To this end compilation flags can be defined
and assigned different default values.
They will be particularly useful in conjunction
with the forwarding mechanism described in \secref{sec:forward}.

For example, it may be useful to have a flag |\version|
which can be set to |draft| or |final|.
The document source will contain some conditional code
depending on the value of |\version|.
Suppose further, the flag should default to |final| for the main file
and to |draft| for child files
which is a natural assignment for editing the document.
This is achieved by placing the following code
in the preamble of the main document
(below the |\childdocmain| directive):
%
\begin{center}
\begin{tabular}{l}
|\ifchilddoc|\\
|\providecommand{\version}{draft}|\\
|\||else|\\
|\providecommand{\version}{final}|\\
|\||fi|
\end{tabular}
\end{center}
%
The definition by |\providecommand| makes sure
that previous definitions are not overwritten.
Further statements |\providecommand{\version}{...}|
can thus be added before the above code to override it.

For the main file, one might add a line
(between |\childdocmain| and the above block)
%
\begin{center}
|%\ifchilddoc\||else\providecommand{\version}{draft}\||fi|
\end{center}
%
which can be uncommented to produce a draft version.
Likewise one can add a line to the very top of a child file
(above the |\childdocof{|\textit{main}|}| directive)
%
\begin{center}
|%\providecommand{\version}{final}|
\end{center}
%
which can be uncommented to produce the final version of this child document.

%%%%%%%%%%%%%%%%%%%%%%%%%%%%%%%%%%%%%%%%%%%%%%%%%%%%%%%%%%%%%%%%%%%%%%%%%%%%%%%%
\subsection{Forwarding}
\label{sec:forward}

Different versions of the main or child documents
using compilation flags as described in \secref{sec:flags}
can be (permanently) stored in different files
for convenient compilation, viewing and distribution.
To this end, the package defines a command
to pass on compilation to a different file:

%%%%%%%%%%%%%%%%%%%%%%%%%%%%%%%%%%%%%%%%
\DescribeMacro{\childdocforward}
The command |\childdocforward| redirects processing to
another source file:
%
\begin{center}
\begin{tabular}{l}
|\input{childdoc.def}|\\
|\childdocforward[|\textit{main}|]{|\textit{dest}|}|\\
\end{tabular}
\end{center}
%
The argument \textit{dest} is the destination file
(without extension).
It should be the main file or one of the child files.
Note that further \textsf{childdoc} directives
such as |\childdocof| and |\childdocforward|
in the indicated file will be processed in this form.
The optional argument \textit{main}
passes on directly to the main file \textit{main}
while pretending to compile the child \textit{dest}.
This form behaves as if \textit{dest}
issues |\childdocof{|\textit{main}|}| right away,
and no further \textsf{childdoc} directives will be processed.

%%%%%%%%%%%%%%%%%%%%%%%%%%%%%%%%%%%%%%%%
\DescribeMacro{\...prefix}
In the alternative form |\childdocforwardprefix|,
%
\begin{center}
\begin{tabular}{l}
|\input{childdoc.def}|\\
|\childdocforwardprefix[|\textit{main}|]{|\textit{prefix}|}{|\textit{dest}|}|
\end{tabular}
\end{center}
%
the destination file is determined by a pattern
depending on the current file:
To make this work, the current file must be called
`{\textit{prefix}\hspace{0.2em}\textit{suffix}}'
with \textit{prefix} matching precisely the argument.
Processing is then passed on to the file
`{\textit{dest}\hspace{0.2em}\textit{suffix}}'.
Surely, the same effect is achieved by
directly specifying the
argument `{\textit{dest}\hspace{0.2em}\textit{suffix}}'
in the first form.
However, that requires to set up a different file
for each child. With the alternative form of the command
all these files can have exactly the same content
which simplifies setting them up and maintaining them.

For example, the following file |draft.tex|
with a compilation flag |\version| as described in \secref{sec:flags}
compiles the main document as a draft:
%
\begin{center}
\begin{tabular}{l}
|\def\version{draft}|\\
|\input{childdoc.def}|\\
|\childdocforward{|\textit{main}|}|
\end{tabular}
\end{center}
%
Likewise, the following files |final|\textit{nn}|.tex|
compile the final version of the child document
|child|\textit{nn}|.tex|:
%
\begin{center}
\begin{tabular}{l}
|\def\version{final}|\\
|\input{childdoc.def}|\\
|\childdocforwardprefix{final}{child}|
\end{tabular}
\end{center}
%

Note that when several versions of a main file and/or of each child file
are to be generated, it may be convenient to set up a |Makefile| or
shell script to automatise the process.

%%%%%%%%%%%%%%%%%%%%%%%%%%%%%%%%%%%%%%%%%%%%%%%%%%%%%%%%%%%%%%%%%%%%%%%%%%%%%%%%
\subsection{Command Line Processing}
\label{sec:commandline}

The effect of redirection files can also be achieved by invoking
the \LaTeX{} compiler with a more elaborate command line.
Most conveniently this should be done as part
of a shell script or a |Makefile|.

When using \textsf{childdoc} in the main file, the following
command lines effectively perform a redirection
(note that depending on the shell being used,
backslashes may have to be doubled: `|\|' $\to$ `|\\|'):
%
\begin{center}
|... -jobname "|\textit{target}|" |\\|"|[\textit{flags}]%
|\input{childdoc.def}\childdocforward[|\textit{main}|]{|\textit{dest}|}"|
\end{center}
%
Here \textit{target} is the name of the output file,
\textit{main} is the name of the main file
and \textit{dest} is the name of the main or child file to be processed
(all filenames without extensions).
The optional argument \textit{main} can be omitted
if \textit{main} matches \textit{dest}.
Optionally, compilation \textit{flags} can be defined via |\def| commands.
This command line makes the \TeX{} engine believe
it is compiling the file \textit{target}
whose content is specified as the latter parameter.
The provided code then forwards the processing to
\textit{main} or \textit{dest} as described in \secref{sec:forward}.

%%%%%%%%%%%%%%%%%%%%%%%%%%%%%%%%%%%%%%%%%%%%%%%%%%%%%%%%%%%%%%%%%%%%%%%%%%%%%%%%
\subsection{Include by Input}
\label{sec:input}

Including child documents by |\include| has some restrictions by design.
Most notably, the content of a child document always occupies
its own set of pages; pages cannot be shared between child documents.
Usually, this behaviour makes perfect sense
because each child document contain an essential part of the document.
However, in some situations it may be desirable to compose
a document from a collection of parts
without having mandatory page breaks between then.
For this case, the package
provides a mechanism to include parts
by |\input| which can also be processed individually.
However, by construction this mechanism
requires manual handling of the content to be output.

%%%%%%%%%%%%%%%%%%%%%%%%%%%%%%%%%%%%%%%%
\DescribeMacro{\ifchilddocmanual}
The main file should be prepared as usual, see \secref{sec:include}.
However, the document body must make a distinction
between processing of an individual part and of the main document, e.g.:
%
\begin{center}
\begin{tabular}{l}
|\ifchilddocmanual|\\
|\input{\childdocname}|\\
|\||else|\\
\textit{document body with }|\input{|\textit{part}|}|\\
|\||fi|
\end{tabular}
\end{center}
%
The conditional |\ifchilddocmanual| is true whenever
a part to be included by |\input| is being compiled,
and the name of the part is stored in |\childdocname|.

%%%%%%%%%%%%%%%%%%%%%%%%%%%%%%%%%%%%%%%%
\DescribeMacro{\childdocby}
Each part to be included by |\input| should start with:
%
\begin{center}
\begin{tabular}{l}
|\input{childdoc.def}|\\
|\childdocby{|\textit{main}|}|\\
\end{tabular}
\end{center}
%
The directive |\childdocby| is similar to |\childdocof|
described in \secref{sec:include},
but the subsequent selection of content must be done manually.
To that end, both |\ifchilddoc| and |\ifchilddocmanual|
will be true upon processing of a part,
and the name of the part is stored in |\childdocname|.
Note that |\jobname| will be set to the filename of the current part
so that each part receives an individual |.aux| file
that does not interfere with the |.aux| file(s) of the main document.
This behaviour can be altered by the alternative form
|\childdocby[*]{|\textit{main}|}| (with a non-empty optional argument)
which uses the |.aux| file of the main document
by setting |\jobname| to \textit{main}.

%%%%%%%%%%%%%%%%%%%%%%%%%%%%%%%%%%%%%%%%%%%%%%%%%%%%%%%%%%%%%%%%%%%%%%%%%%%%%%%%
\subsection{Driver Development}
\label{sec:driver}

The \textsf{childdoc} mechanism can also be use for the development
of definition files such as \LaTeX{} styles or classes.
This case differs from the above setup with multiple parts
included by |\include| in that no |\includeonly| should be invoked.
This can be achieved by starting the include file
(before |\ProvidesPackage|) with:
%
\begin{center}
\begin{tabular}{l}
|\input{childdoc.def}|\\
|\childdocforward{|\textit{main}|}|\\
\end{tabular}
\end{center}
%
or alternatively with:
%
\begin{center}
\begin{tabular}{l}
|\input{childdoc.def}|\\
|\childdocby{|\textit{main}|}|\\
\end{tabular}
\end{center}
%
Both forms have slightly different effects as described above.
The main file is prepared as usual, see \secref{sec:include}.

%%%%%%%%%%%%%%%%%%%%%%%%%%%%%%%%%%%%%%%%%%%%%%%%%%%%%%%%%%%%%%%%%%%%%%%%%%%%%%%%
\subsection{Legacy Detection}
\label{sec:detection}

The directive |\childdocmain| in the main file can detect
whether the complete document or merely a child is to be compiled
even without using the directive |\childdocof|.
This method is deprecated because it is less robust
and there is no compelling reason to use it;
it is merely provided for backward compatibility
and it may be removed in future versions.

If the detection mechanism is to be used,
it is mandatory to correctly specify
the filename of the main file as the argument of |\childdocmain|:
%
\begin{center}
\begin{tabular}{l}
|\input{childdoc.def}|\\
|\childdocmain{|\textit{main}|}|\\
\end{tabular}
\end{center}
%
If |\jobname| does not match the argument \textit{main} of |\childdocmain|,
it is assumed that |\jobname| points to the child file to be compiled.
When using |\childdocmain| with the main file specified as argument,
it suffices to start a child file
with just |\input{|\textit{main}|}|
without loading of the package and using |\childdocof|.
If instead all processing is done
with the appropriate \textsf{childdoc} directives,
the argument of \textit{main} of |\childdocmain| can be empty.

An alternative version of the command line processing described
in \secref{sec:commandline} using the detection mechanism reads:
%
\begin{center}
|... -jobname "|\textit{target}|" "|[\textit{flags}]%
[|\def\jobname{|\textit{dest}|}|]|\input{|\textit{main}|}"|
\end{center}

%%%%%%%%%%%%%%%%%%%%%%%%%%%%%%%%%%%%%%%%%%%%%%%%%%%%%%%%%%%%%%%%%%%%%%%%%%%%%%%%
\subsection{Manual Code}
\label{sec:manual}

In case one cannot be certain whether the definitions file |childdoc.def|
is installed on the target \TeX{} distribution
and one prefers not to ship it,
it is conceivable to paste a few relevant commands into the sources.

To that end, drop all statements |\input{childdoc.def}|
and perform the replacements as outlined below.
Instead of |\childdocmain{|\textit{main}|}| add the following code
to the top of the main file:
%
\begin{center}
\begin{tabular}{l}
|\||ifdefined\childdocname\endinput\||fi\newif\ifchilddoc|\\
|\edef\childdocname{\scantokens\expandafter{\jobname\noexpand}}|\\
|\def\childdocmain{|\textit{main}|}\||ifx\childdocmain\childdocname\||else|\\
|\childdoctrue\includeonly{\childdocname}\let\jobname\childdocmain\||fi|\\
\end{tabular}
\end{center}
%
Instead of |\childdocof{|\textit{main}|}| just include the main file
at the top of each child file:
%
\begin{center}
|\input{|\textit{main}|}|
\end{center}
%
A simple redirection |\childdocforward{|\textit{dest}|}| is achieved by:
%
\begin{center}
|\def\jobname{|\textit{dest}|}\input{\jobname}|
\end{center}
%
The redirection with prefix
|\childdocforwardprefix[|\textit{prefix}|]{|\textit{dest}|}|
is accomplished by:
%
\begin{center}
\begin{tabular}{l}
|{\edef\jobname{\scantokens\expandafter{\jobname\noexpand}}|\\
|\def\redirectjob |\textit{prefix}|#1~~~{\gdef\jobname{|\textit{dest}|#1}}|\\
|\expandafter\redirectjob\jobname~~~}\input{\jobname}|
\end{tabular}
\end{center}

In an alternative approach,
child documents can be compiled by a specific command line
without additional code or specific definitions:
%
\begin{center}
|... -jobname "|\textit{target}|" "|[\textit{flags}]%
|\includeonly{|\textit{dest}|}\input{|\textit{main}|}"|
\end{center}
%

%%%%%%%%%%%%%%%%%%%%%%%%%%%%%%%%%%%%%%%%%%%%%%%%%%%%%%%%%%%%%%%%%%%%%%%%%%%%%%%%
%%%%%%%%%%%%%%%%%%%%%%%%%%%%%%%%%%%%%%%%%%%%%%%%%%%%%%%%%%%%%%%%%%%%%%%%%%%%%%%%
\section{Information}

%%%%%%%%%%%%%%%%%%%%%%%%%%%%%%%%%%%%%%%%%%%%%%%%%%%%%%%%%%%%%%%%%%%%%%%%%%%%%%%%
\subsection{Copyright}

Copyright \copyright{} 2017--2018 Niklas Beisert

This work may be distributed and/or modified under the
conditions of the \LaTeX{} Project Public License, either version 1.3
of this license or (at your option) any later version.
The latest version of this license is in
  \url{http://www.latex-project.org/lppl.txt}
and version 1.3 or later is part of all distributions of \LaTeX{}
version 2005/12/01 or later.

This work has the LPPL maintenance status `maintained'.

The Current Maintainer of this work is Niklas Beisert.

This work consists of the files |README.txt|, |childdoc.ins| and |childdoc.dtx|
as well as the derived files |childdoc.def|, |cdocsamp.tex|
with |cdocsch1.tex|, |cdocsch2.tex|, |cdocspt3.tex|, |cdocspt4.tex|,
|cdocsdrf.tex|, |cdocsfn1.tex|, |cdocsfn2.tex|
as well as |childdoc.pdf|.

%%%%%%%%%%%%%%%%%%%%%%%%%%%%%%%%%%%%%%%%%%%%%%%%%%%%%%%%%%%%%%%%%%%%%%%%%%%%%%%%
\subsection{Files and Installation}

The package consists of the files:
%
\begin{center}
\begin{tabular}{ll}
    |README.txt|   & readme file \\
    |childdoc.ins| & installation file \\
    |childdoc.dtx| & source file \\
    |childdoc.def| & definition file \\
    |cdocsamp.tex| & sample main file \\
    |cdocsch1.tex| & sample include file \\
    |cdocsch2.tex| & sample include file \\
    |cdocspt3.tex| & sample part file \\
    |cdocspt4.tex| & sample part file \\
    |cdocsdrf.tex| & sample redirection file \\
    |cdocsfn1.tex| & sample redirection file \\
    |cdocsfn2.tex| & sample redirection file \\
    |childdoc.pdf| & manual
\end{tabular}
\end{center}
%
The distribution consists of the files
|README.txt|, |childdoc.ins| and |childdoc.dtx|.
%
\begin{itemize}
\item
Run (pdf)\LaTeX{} on |childdoc.dtx|
to compile the manual |childdoc.pdf| (this file).
\item
Run \LaTeX{} on |childdoc.ins| to create the definitions file |childdoc.def|
and the sample |cdocsamp.tex| with include files
|cdocsch1.tex|, |cdocsch2.tex|, |cdocspt3.tex|, |cdocspt4.tex|,
|cdocsdrf.tex|, |cdocsfn1.tex|, |cdocsfn2.tex|.
Then copy the file |childdoc.def| to an appropriate directory of your \LaTeX{}
distribution, e.g.\ \textit{texmf-root}|/tex/latex/childdoc|.
\end{itemize}

%%%%%%%%%%%%%%%%%%%%%%%%%%%%%%%%%%%%%%%%%%%%%%%%%%%%%%%%%%%%%%%%%%%%%%%%%%%%%%%%
\subsection{Related CTAN Packages}

There are several other packages which offer a similar functionality:
%
\begin{itemize}
\item
The packages
\href{http://ctan.org/pkg/docmute}{\textsf{docmute}},
\href{http://ctan.org/pkg/includex}{\textsf{includex}} and
\href{http://ctan.org/pkg/standalone}{\textsf{standalone}}
provide commands to include only the document body of
a child file thus allowing both files to be compiled individually.
\item
The packages \href{http://ctan.org/pkg/subdocs}{\textsf{subdocs}}
and \href{http://ctan.org/pkg/subfiles}{\textsf{subfiles}}
provide structures in which the main and child documents can be
encapsulated and allowing them to be compiled individually.
The inclusion mechanism is different from the conventional |\include|.
\item
The package \href{http://ctan.org/pkg/combine}{\textsf{combine}}
is an elaborate solution to combine several documents into one.
\end{itemize}
%
See also the CTAN topic \href{http://ctan.org/topic/subdocs}{\textsf{subdocs}}
for further related packages.
The present package differs from the above solutions in that
a document structure constructed with the conventional |\include| mechanism
just needs two extra commands at the top of every file
such that all constituent files can be compiled individually.

%%%%%%%%%%%%%%%%%%%%%%%%%%%%%%%%%%%%%%%%%%%%%%%%%%%%%%%%%%%%%%%%%%%%%%%%%%%%%%%%
%\subsection{Feature Suggestions}
%
%The following is a list of features which may be useful for future
%versions of this package:
%%
%\begin{itemize}
%\item
%\ldots
%\end{itemize}

%%%%%%%%%%%%%%%%%%%%%%%%%%%%%%%%%%%%%%%%%%%%%%%%%%%%%%%%%%%%%%%%%%%%%%%%%%%%%%%%
\subsection{Revision History}

%%%%%%%%%%%%%%%%%%%%%%%%%%%%%%%%%%%%%%%%
\paragraph{v2.0:} 2018/12/30

\begin{itemize}
\item
immediate forward processing
\item
added |\childdocby| mechanism
\item
manual restructured
\end{itemize}

%%%%%%%%%%%%%%%%%%%%%%%%%%%%%%%%%%%%%%%%
\paragraph{v1.6:} 2018/01/17

\begin{itemize}
\item
application for development of include files
\item
corrections to manual
\end{itemize}

%%%%%%%%%%%%%%%%%%%%%%%%%%%%%%%%%%%%%%%%
\paragraph{v1.5:} 2017/05/21

\begin{itemize}
\item
more complete structuring introduced
\item
|\childdocof| introduced
\item
|\childdoc| renamed to |\childdocmain|
\item
|\childredirect| renamed to |\childdocforward| and |\childdocforwardprefix|
and functionality expanded
\end{itemize}

%%%%%%%%%%%%%%%%%%%%%%%%%%%%%%%%%%%%%%%%
\paragraph{v1.0:} 2017/04/27

\begin{itemize}
\item
manual and install package
\item
first version published on CTAN
\end{itemize}

%%%%%%%%%%%%%%%%%%%%%%%%%%%%%%%%%%%%%%%%
\paragraph{v0.6:} 2017/04/26

\begin{itemize}
\item
redirection mechanism added
\end{itemize}

%%%%%%%%%%%%%%%%%%%%%%%%%%%%%%%%%%%%%%%%
\paragraph{v0.5:} 2017/04/26

\begin{itemize}
\item
functionality in definition file
\end{itemize}


%%%%%%%%%%%%%%%%%%%%%%%%%%%%%%%%%%%%%%%%%%%%%%%%%%%%%%%%%%%%%%%%%%%%%%%%%%%%%%%%
%%%%%%%%%%%%%%%%%%%%%%%%%%%%%%%%%%%%%%%%%%%%%%%%%%%%%%%%%%%%%%%%%%%%%%%%%%%%%%%%
%%%%%%%%%%%%%%%%%%%%%%%%%%%%%%%%%%%%%%%%%%%%%%%%%%%%%%%%%%%%%%%%%%%%%%%%%%%%%%%%
\appendix

\settowidth\MacroIndent{\rmfamily\scriptsize 000\ }

 \DocInput{childdoc.dtx}

\end{document}
%</driver>
% \fi
%
% %%%%%%%%%%%%%%%%%%%%%%%%%%%%%%%%%%%%%%%%%%%%%%%%%%%%%%%%%%%%%%%%%%%%%%%%%%%%%%
% %%%%%%%%%%%%%%%%%%%%%%%%%%%%%%%%%%%%%%%%%%%%%%%%%%%%%%%%%%%%%%%%%%%%%%%%%%%%%%
% \section{Sample}
%\iffalse
%<*samplemain>
%\fi
%
% The following presents a sample document
% with two chapters, two parts, a title page,
% a compile flag as well as three forwarding files to set the flag.
% It consists of eight |.tex| files:
% \begin{center}
% \begin{tabular}{ll}
% |cdocsamp.tex|&main file\\
% |cdocsch1.tex|&include file for chapter 1\\
% |cdocsch2.tex|&include file for chapter 2\\
% |cdocspt3.tex|&include file for part 3\\
% |cdocspt4.tex|&include file for part 4\\
% |cdocsdrf.tex|&forwarding file for main file in draft mode\\
% |cdocsfi1.tex|&forwarding file for final version of chapter 1\\
% |cdocsfi2.tex|&forwarding file for final version of chapter 2\\
% \end{tabular}
% \end{center}
% Each of the eight files can be compiled directly by the \LaTeX{} compiler.
%
% %%%%%%%%%%%%%%%%%%%%%%%%%%%%%%%%%%%%%%
% \paragraph{Main File.}
%
% The main file is called |cdocsamp.tex|.
%
% Load the \textsf{childdoc} definitions and
% declare the filename for the main document:
%    \begin{macrocode}
\input{childdoc.def}
\childdocmain{}
%    \end{macrocode}

% Optional override for |\version| flag:
%    \begin{macrocode}
%%\ifchilddoc\else\providecommand{\version}{draft}\fi
%    \end{macrocode}

% Define the default values for the |\version| flag
% (|final| for the main file and |draft| for childs):
%    \begin{macrocode}
\ifchilddoc
\providecommand{\version}{draft}
\else
\providecommand{\version}{final}
\fi
%    \end{macrocode}

% Load the standard document class:
%    \begin{macrocode}
\documentclass[12pt]{article}
%    \end{macrocode}

% Start the document body:
%    \begin{macrocode}
\begin{document}
%    \end{macrocode}

% Declare a title page.
% Print title, part of document being processed and version flag:
%    \begin{macrocode}
\addtocounter{page}{-1}
\begin{center}
{\LARGE\bfseries{}childdoc example\par}
\vspace{1cm}
\ifchilddoc
\ifchilddocmanual part\else chapter\fi:
`\childdocname' of `\childdocjob'\par
\else
main document: `\childdocjob'\par
\fi
version: \version\par
\end{center}
\newpage
%    \end{macrocode}

% Manually include selected file,
% otherwise process as usual:
%    \begin{macrocode}
\ifchilddocmanual
\section*{part `\childdocname'}
\input{\childdocname}
\else
%    \end{macrocode}

% Include the two chapters:
%    \begin{macrocode}
\include{cdocsch1}
\include{cdocsch2}
%    \end{macrocode}

% Include the two parts unless only chapters should be displayed:
%    \begin{macrocode}
\ifchilddoc\else
\section{part three}
\input{cdocspt3}
\section{part four}
\input{cdocspt4}
\fi
%    \end{macrocode}

% Process as usual until here:
%    \begin{macrocode}
\fi
%    \end{macrocode}

% End of document body:
%    \begin{macrocode}
\end{document}
%    \end{macrocode}
%\iffalse
%</samplemain>
%\fi
%
% %%%%%%%%%%%%%%%%%%%%%%%%%%%%%%%%%%%%%%
% \paragraph{Chapter Include Files.}
%
% The include files are called |cdocsch1.tex| and |cdocsch2.tex|.
%
%\iffalse
%<*samplechap1|samplechap2>
%\fi

% Optional override for |\version| flag:
%    \begin{macrocode}
%%\providecommand{\version}{final}
%    \end{macrocode}

% Include the main document:
%    \begin{macrocode}
\input{childdoc.def}
\childdocof{cdocsamp}
%    \end{macrocode}

%\iffalse
%</samplechap1|samplechap2>
%\fi
%
%\iffalse
%<*samplechap1>
%\fi
% Some text for chapter 1:
%    \begin{macrocode}
\section{one}
some text in chapter one
%    \end{macrocode}

%\iffalse
%</samplechap1>
%\fi
% Some text for chapter 2:
%\iffalse
%<*samplechap2>
%\fi
%    \begin{macrocode}
\section{two}
more text in chapter two
%    \end{macrocode}

%\iffalse
%</samplechap2>
%\fi
%
% %%%%%%%%%%%%%%%%%%%%%%%%%%%%%%%%%%%%%%
% \paragraph{Part Include Files.}
%
% The include files are called |cdocspt3.tex| and |cdocspt4.tex|.
%
%\iffalse
%<*samplepart3|samplepart4>
%\fi

% Optional override for |\version| flag:
%    \begin{macrocode}
%%\providecommand{\version}{final}
%    \end{macrocode}

% Include the main document:
%    \begin{macrocode}
\input{childdoc.def}
\childdocby{cdocsamp}
%    \end{macrocode}

%\iffalse
%</samplepart3|samplepart4>
%\fi
%
%\iffalse
%<*samplepart3>
%\fi
% Some text for part 3:
%    \begin{macrocode}
some text in part three
%    \end{macrocode}

%\iffalse
%</samplepart3>
%\fi
% Some text for part 4:
%\iffalse
%<*samplepart4>
%\fi
%    \begin{macrocode}
more text in part four
%    \end{macrocode}

%\iffalse
%</samplepart4>
%\fi
%
% %%%%%%%%%%%%%%%%%%%%%%%%%%%%%%%%%%%%%%
% \paragraph{Forwarding for a Complete Draft.}
%
% The following forwarding file |cdocsdrf.tex|
% compiles the main document in draft mode:
%\iffalse
%<*sampledraft>
%\fi
%    \begin{macrocode}
\def\version{draft}
\input{childdoc.def}
\childdocforward{cdocsamp}
%    \end{macrocode}

%\iffalse
%</sampledraft>
%\fi
%
% %%%%%%%%%%%%%%%%%%%%%%%%%%%%%%%%%%%%%%
% \paragraph{Forwarding for Final Version of the Chapters.}
%
% The following forwarding files |cdocsfn1.tex| and |cdocsfn2.tex|
% (with identical content)
% compile the final versions of the child documents
% |cdocsch1.tex| and |cdocsch2.tex|, respectively:
%\iffalse
%<*samplefinal>
%\fi
%    \begin{macrocode}
\def\version{final}
\input{childdoc.def}
\childdocforwardprefix[cdocsamp]{cdocsfn}{cdocsch}
%    \end{macrocode}

%\iffalse
%</samplefinal>
%\fi
%
% %%%%%%%%%%%%%%%%%%%%%%%%%%%%%%%%%%%%%%
% \paragraph{Command Line Processing.}
%
% The following three command lines generate the output files
% |cdocscld|, |cdocscl1| and |cdocscl2|
% which should be identical to
% |cdocsdrf|, |cdocsch1| and |cdocsfn2|, respectively:
% \begin{center}
% \begin{tabular}{l}
% |latex -jobname cdocscld \|\\
% |  "\def\version{draft}\input{childdoc.def}\childdocforward{cdocsamp}"|\\
% |latex -jobname cdocscl1 \|\\
% |  "\input{childdoc.def}\childdocforward[cdocsamp]{cdocsch1}"|\\
% |latex -jobname cdocscl2 \|\\
% |  "\def\version{final}\input{childdoc.def}\childdocforward{cdocsch2}"|
% \end{tabular}
% \end{center}
% Note that the trailing backslash on each first line
% merely continues the input to the second line
% (for convenient cut ant paste).
% Furthermore, the command |latex| can be replaced by any
% of its alternative versions such as |pdflatex|.
%
% %%%%%%%%%%%%%%%%%%%%%%%%%%%%%%%%%%%%%%%%%%%%%%%%%%%%%%%%%%%%%%%%%%%%%%%%%%%%%%
% %%%%%%%%%%%%%%%%%%%%%%%%%%%%%%%%%%%%%%%%%%%%%%%%%%%%%%%%%%%%%%%%%%%%%%%%%%%%%%
% \section{Implementation}
%\iffalse
%<*package>
%\fi
%
% This section describes the definitions file |childdoc.def|.

% The definitions cannot be loaded using |\usepackage| or |\RequirePackage|
% which has a mechanism to prevent loading a style file more than once.
% When loading the definitions by means of |\input|
% multiple instances have to be prevented manually:
%\iffalse
%This code needs to be before the `\ProvidesFile' directive
%which is defined at the beginning of this file.
%Therefore it is also placed there and commented out here.
%</package>
%<*discard>
%\fi
%    \begin{macrocode}
\ifdefined\childdocmain\endinput\fi
%    \end{macrocode}
%\iffalse
%</discard>
%<*package>
%\fi
%
% \macro{\ifchilddoc}
% \macro{\ifchilddocmanual}
% The conditional |\ifchilddoc| tells whether a
% child (true) or main (false) document is being compiled.
% The conditional |\ifchilddocmanual| tells whether
% the |\includeonly| mechanism is used (false) or
% the selection of child files must be performed manually (true).
% The definitions initialise to false:
%    \begin{macrocode}
\newif\ifchilddoc
\newif\ifchilddocmanual
%    \end{macrocode}

% \macro{\childdocname}
% \macro{\childdocjob}
% The macro |\childdocname| stores the name of the main document
% to be compiled. The macro |\childdocjob| stores the name of
% the document on which the \LaTeX{} compiler was originally invoked.
% The content of |\jobname| cannot be compared
% to filenames specified in the source due to different catcodes.
% The following code rescans |\jobname|, stores the result
% in |\childdocname| and saves a copy in |\childdocjob|:
%    \begin{macrocode}
\edef\childdocname{\scantokens\expandafter{\jobname\noexpand}}
\let\childdocjob\childdocname
%    \end{macrocode}

% \macro{\childdocdisable}
% The macro |\childdocdisable| prevents the main file
% from being processed more than once.
% At this stage, the main document command |\childdocmain|
% is assumed to be called once again where it should do nothing.
% Any subsequent call to it should prevent
% a secondary processing of the main document
% It overwrites the forwarding commands
% |\childdocof| and |\childdocforward|
% with empty macros to prevent further inclusions of the main document:
%    \begin{macrocode}
\newcommand{\childdocdisable}
{
  \renewcommand{\childdocmain}[1]{\renewcommand{\childdocmain}[1]{\endinput}}
  \renewcommand{\childdocof}[1]{}
  \renewcommand{\childdocby}[2][]{}
  \renewcommand{\childdocforward}[2][]{}
  \renewcommand{\childdocdisable}{}
}
%    \end{macrocode}

% \macro{\childdocmain}
% The macro |\childdocmain| is to be called at the top of the main file
% with nothing or the main filename (without extension) as argument.
% First, it breaks loops.
% If the argument is not empty and does not match |\childdocname|
% (which is set by the first inclusion of |childdoc.def|),
% |\ifchilddoc| is set to true, |\includeonly| is applied to the child file
% and |\jobname| is set to the main file
% (for proper handling of |.aux| files):
%    \begin{macrocode}
\newcommand{\childdocmain}[1]
{
  \childdocdisable\childdocmain{}
  \if?#1?\else
    \begingroup
      \def\childdoctmp{#1}
      \ifx\childdoctmp\childdocname
        \def\childdoctmp{}
      \else
        \def\childdoctmp
        {
          \childdoctrue
          \includeonly{\childdocname}
          \def\childdocjob{#1}
          \def\jobname{#1}
        }
      \fi
      \expandafter
    \endgroup
    \childdoctmp
  \fi
}
%    \end{macrocode}

% \macro{\childdocof}
% The command |\childdocof| redirects
% compilation to the main file |#1|.
%    \begin{macrocode}
\newcommand{\childdocof}[1]
{
  \childdocdisable
  \childdoctrue
  \includeonly{\childdocname}
  \def\jobname{#1}
  \def\childdocjob{#1}
  \input{#1}
}
%    \end{macrocode}

% \macro{\childdocby}
% The command |\childdocby| ....
%    \begin{macrocode}
\newcommand{\childdocby}[2][]
{
  \childdocdisable
  \childdoctrue
  \childdocmanualtrue
  \if?#1?\else
    \def\jobname{#2}
  \fi
  \def\childdocjob{#2}
  \input{#2}
  \endinput
}
%    \end{macrocode}

% \macro{\childdocforward}
% The command |\childdocforward| redirects
% compilation to the main file or
% (if the optional argument is given) a child file.
% Parameters are set as if the main file
% or a child file starting with |\childdocof| was compiled.
% Then compilation is handed over to the main file:
%    \begin{macrocode}
\newcommand{\childdocforward}[2][]
{
  \begingroup
    \if?#1?
      \def\childdoctmp
      {
        \def\childdocname{#2}
        \def\childdocjob{#2}
        \def\jobname{#2}
        \input{#2}
        \endinput
      }
    \else
      \def\childdoctmp
      {
        \childdocdisable
        \def\childdocname{#2}
        \childdoctrue
        \includeonly{#2}
        \def\childdocjob{#1}
        \def\jobname{#1}
        \input{#1}
        \endinput
      }
    \fi
    \expandafter
  \endgroup
  \childdoctmp
}
%    \end{macrocode}

% \macro{\childdocforwardprefix}
% The command |\childdocforwardprefix| redirects
% compilation to the main or a child file by means of a pattern.
% The prefix |#1| in the current filename is replaced by |#2|
% and the suffix of the current filename is kept
% (it is assumed that the filename does not contain the substring `|~~~|'
% which is used as a delimiter).
% Compilation is handed over to the new file by |\childdocforward|:
%    \begin{macrocode}
\newcommand{\childdocforwardprefix}[3][]
{
  \begingroup
    \def\childdocextract #2##1~~~{\def\childdoctmp{\childdocforward[#1]{#3##1}}}
    \expandafter\childdocextract\childdocname~~~
    \expandafter
  \endgroup
  \childdoctmp
}
%    \end{macrocode}

% \macro{\childdoc}
% The deprecated macro |\childdoc| is a legacy version of |\childdocmain|:
%    \begin{macrocode}
\newcommand{\childdoc}{\childdocmain}
%    \end{macrocode}

% \macro{\childdocredirect}
% The deprecated macro |\childdocredirect| is a legacy version
% of |\childdocforward| and |\childdocforwardprefix|:
%    \begin{macrocode}
\newcommand{\childdocredirect}[2][]
{
  \begingroup
    \if?#1?
      \def\childdoctmp{\childdocforward{#2}}
    \else
      \def\childdoctmp{\childdocforwardprefix{#1}{#2}}
    \fi
    \expandafter
  \endgroup
  \childdoctmp
}
%    \end{macrocode}

%\iffalse
%</package>
%\fi
%
\endinput
|\\
|\childdocmain{|\textit{main}|}|\\
\end{tabular}
\end{center}
%
If |\jobname| does not match the argument \textit{main} of |\childdocmain|,
it is assumed that |\jobname| points to the child file to be compiled.
When using |\childdocmain| with the main file specified as argument,
it suffices to start a child file
with just |\input{|\textit{main}|}|
without loading of the package and using |\childdocof|.
If instead all processing is done
with the appropriate \textsf{childdoc} directives,
the argument of \textit{main} of |\childdocmain| can be empty.

An alternative version of the command line processing described
in \secref{sec:commandline} using the detection mechanism reads:
%
\begin{center}
|... -jobname "|\textit{target}|" "|[\textit{flags}]%
[|\def\jobname{|\textit{dest}|}|]|\input{|\textit{main}|}"|
\end{center}

%%%%%%%%%%%%%%%%%%%%%%%%%%%%%%%%%%%%%%%%%%%%%%%%%%%%%%%%%%%%%%%%%%%%%%%%%%%%%%%%
\subsection{Manual Code}
\label{sec:manual}

In case one cannot be certain whether the definitions file |childdoc.def|
is installed on the target \TeX{} distribution
and one prefers not to ship it,
it is conceivable to paste a few relevant commands into the sources.

To that end, drop all statements |% \iffalse
%
% childdoc.dtx Copyright (C) 2017-2018 Niklas Beisert
%
% This work may be distributed and/or modified under the
% conditions of the LaTeX Project Public License, either version 1.3
% of this license or (at your option) any later version.
% The latest version of this license is in
%   http://www.latex-project.org/lppl.txt
% and version 1.3 or later is part of all distributions of LaTeX
% version 2005/12/01 or later.
%
% This work has the LPPL maintenance status `maintained'.
%
% The Current Maintainer of this work is Niklas Beisert.
%
% This work consists of the files childdoc.dtx and childdoc.ins
% and the derived files childdoc.def and cdocsamp.tex with
% cdocsch1.tex, cdocsch2.tex, cdocsdrf.tex, cdocsfn1.tex, cdocsfn2.tex.
%
%<package>\ifdefined\childdocmain\endinput\fi
%<package>\ProvidesFile{childdoc.def}[2018/12/30 v2.0 child document driver]
%<samplemain>\ProvidesFile{cdocsamp.tex}[2018/12/30 v2.0 sample for childdoc]
%<*driver>
%\ProvidesFile{childdoc.drv}[2018/12/30 v2.0 childdoc reference manual file]
\PassOptionsToClass{10pt,a4paper}{article}
\documentclass{ltxdoc}

\usepackage[margin=35mm]{geometry}
\usepackage{hyperref}
\usepackage{hyperxmp}
\usepackage[usenames]{color}

\hypersetup{colorlinks=true}
\hypersetup{pdfstartview=FitH}
\hypersetup{pdfpagemode=UseNone}
\hypersetup{pdfsource={}}
\hypersetup{pdflang={en-UK}}
\hypersetup{pdfcopyright={Copyright 2017-2018 Niklas Beisert.
  This work may be distributed and/or modified under the
  conditions of the LaTeX Project Public License, either version 1.3
  of this license or (at your option) any later version.}}
\hypersetup{pdflicenseurl={http://www.latex-project.org/lppl.txt}}
\hypersetup{pdfcontactaddress={ETH Zurich, ITP, HIT K,
  Wolfgang-Pauli-Strasse 27}}
\hypersetup{pdfcontactpostcode={8093}}
\hypersetup{pdfcontactcity={Zurich}}
\hypersetup{pdfcontactcountry={Switzerland}}
\hypersetup{pdfcontactemail={nbeisert@itp.phys.ethz.ch}}
\hypersetup{pdfcontacturl={http://people.phys.ethz.ch/\xmptilde nbeisert/}}

\newcommand{\secref}[1]{\hyperref[#1]{section \ref*{#1}}}

\parskip1ex
\parindent0pt
\let\olditemize\itemize
\def\itemize{\olditemize\parskip0pt}

\begin{document}

\title{The \textsf{childdoc} Package}
\hypersetup{pdftitle={The childdoc Package}}
\author{Niklas Beisert\\[2ex]
  Institut f\"ur Theoretische Physik\\
  Eidgen\"ossische Technische Hochschule Z\"urich\\
  Wolfgang-Pauli-Strasse 27, 8093 Z\"urich, Switzerland\\[1ex]
  \href{mailto:nbeisert@itp.phys.ethz.ch}
  {\texttt{nbeisert@itp.phys.ethz.ch}}}
\hypersetup{pdfauthor={Niklas Beisert}}
\hypersetup{pdfsubject={Manual for the LaTeX2e Package childdoc}}
\date{30 December 2018, \textsf{v2.0}}
\maketitle

\begin{abstract}\noindent
\textsf{childdoc} is a \LaTeXe{} package
that enables the direct compilation
of document sections included by |\include|
to individual files.
\end{abstract}

\begingroup
\parskip0ex
\tableofcontents
\endgroup

%%%%%%%%%%%%%%%%%%%%%%%%%%%%%%%%%%%%%%%%%%%%%%%%%%%%%%%%%%%%%%%%%%%%%%%%%%%%%%%%
%%%%%%%%%%%%%%%%%%%%%%%%%%%%%%%%%%%%%%%%%%%%%%%%%%%%%%%%%%%%%%%%%%%%%%%%%%%%%%%%
\section{Introduction}

\LaTeX{} provides a mechanism to structure a large document (such as a book)
into a main file and several child files (containing the chapters)
using the |\include| command.
This mechanism is beneficial for documents
which span hundreds of pages in order to
make the source file(s) more manageable.
Moreover, compilation can be restricted to
selected child files by means of the |\includeonly| command.
The latter feature can be used to reduce the compilation time while editing
(this was significantly more useful in the earlier days of \LaTeX{})
or to generate a smaller document which is easier to navigate.
Another application of |\includeonly| is to generate
documents consisting of selected parts of the complete document.

However, there are a few drawbacks of the plain |\include| mechanism:
\begin{itemize}
\item
The child files cannot be compiled on their own,
they can only be compiled via the main file.
A naive editing environment
(such as a text editor with an option
to have the current file processed by \LaTeX)
may require one to switch to the main file before compiling;
attempting to compile the child file produces errors.
\item
The main file must be modified (each time)
to adjust the |\includeonly| command
to the present needs. This easily leaves the main file in a messy state.
\item
The generated document will always carry the filename
of the main document. This is inconvenient if
several child files are to be compiled and
to be kept for distribution.
\end{itemize}

The present package provides a simple interface
to make child files individually compilable by \LaTeX{}.
Compiling a child file then has the same effect as compiling
the main file with an |\includeonly| command
to select the appropriate child.
Moreover the generated document will carry the name of the child
rather than the main file.
This resolves all three above issues.

This feature is meant to make the editing of books,
thesis documents and lecture notes somewhat more convenient.
However, the package can also be used efficiently for
composing a series of documents (such as exercise sheets)
which are typically distributed individually.
It then assists the author in generating the individual documents
(potentially in different versions)
as well as a document containing the collected series.
Another application is in developing style files
or other kinds of included material
where compilation of the style file could redirect
to a sample or test file.

%%%%%%%%%%%%%%%%%%%%%%%%%%%%%%%%%%%%%%%%%%%%%%%%%%%%%%%%%%%%%%%%%%%%%%%%%%%%%%%%
%%%%%%%%%%%%%%%%%%%%%%%%%%%%%%%%%%%%%%%%%%%%%%%%%%%%%%%%%%%%%%%%%%%%%%%%%%%%%%%%
\section{Usage}

First of all, the package \textsf{childdoc} is \emph{not} a standard
\LaTeXe{} |.sty| style file! Therefore it needs to be invoked in
a non-standard way.

%%%%%%%%%%%%%%%%%%%%%%%%%%%%%%%%%%%%%%%%%%%%%%%%%%%%%%%%%%%%%%%%%%%%%%%%%%%%%%%%
\subsection{Included Files}
\label{sec:include}

%%%%%%%%%%%%%%%%%%%%%%%%%%%%%%%%%%%%%%%%
\DescribeMacro{\childdocmain}
To use the package, add the commands
\begin{center}
\begin{tabular}{l}
|\input{childdoc.def}|\\
|\childdocmain{}|\\
\end{tabular}
\end{center}
at the very top of the main \LaTeX{} file,
in particular \emph{before} the |\documentclass| statement!
The argument of |\childdocmain| should be left empty
(but it must be present).

%%%%%%%%%%%%%%%%%%%%%%%%%%%%%%%%%%%%%%%%
\DescribeMacro{\childdocof}
Furthermore, add the commands
\begin{center}
\begin{tabular}{l}
|\input{childdoc.def}|\\
|\childdocof{|\textit{main}|}|\\
\end{tabular}
\end{center}
at the top of every child file \textit{child}
which is included by |\include{|\textit{child}|}|
from within the main file
(or at least for those files to be compiled individually).
The argument \textit{main} must be the filename of the main file.

There are a couple of
considerations in setting up the main and child documents:

%%%%%%%%%%%%%%%%%%%%%%%%%%%%%%%%%%%%%%%%
\paragraph{Restrictions.}

Please note the following restrictions:
\begin{itemize}
\item
|\childdocmain| must be called with one argument \textit{main}
to ensure compatibility with earlier version of the package.
It must either be empty (|\childdocmain{}|)
or precisely match the filename of the main file in which it is specified.
See \secref{sec:detection} for further information.
\item
The filename \textit{main} must be specified without the |.tex| extension.
\item
The filename \textit{main} is case sensitive
(even in case-insensitive file systems)
due to internal string comparison.
\item
The argument \textit{main} should be fully expanded, it cannot be a macro.
\item
Subdirectories and special characters should be avoided in filenames.
\item
The command |\childdocmain{|\textit{main}|}| must be followed by a whitespace.
It should not be followed immediately by another command
or by a comment mark `|%|'.
This is because the \TeX{} parser reads the token immediately following
the argument of |\childdocmain| and puts it
at the beginning of every child section;
however, a white\-space is ignored.
\end{itemize}

%%%%%%%%%%%%%%%%%%%%%%%%%%%%%%%%%%%%%%%%
\paragraph{Content of Main File.}

It is advisable to place all content in the child files included by |\include|.
Any output contained in the main file will appear in all child documents
unless suppressed manually;
it cannot be suppressed automatically by the |\includeonly| directive
and thus should normally be avoided.
A method to include some content in the main file
by means of conditional processing is described in \secref{sec:conditional}.

%%%%%%%%%%%%%%%%%%%%%%%%%%%%%%%%%%%%%%%%
\paragraph{Page Numbering.}

When only a part of the document is compiled,
the appropriate numbering of pages
(as well as other status parameters)
is determined from the |.aux| files.
The latter contain information from previous passes.
However this information needs to propagate through
all intermediate child documents.
Therefore the page numbering in child documents may well
be inconsistent until the complete document is compiled at least once.

A useful (if unconventional) way to always ensure a consistent
page numbering is to restart the numbering in each child document
and denote the pages by `\textit{child}|.|\textit{page}'
where \textit{child} represents the chapter/section number of the child file.
This can be achieved by the command
|\numberwithin{page}{|\textit{child}|}|
of the \textsf{amsmath} package
where \textit{child} can be |chapter| or |section|
depending on the chosen structuring.
Alternatively, one can modify the macro |\thepage| appropriately
and reset the counter |page| at the start of each child file.

%%%%%%%%%%%%%%%%%%%%%%%%%%%%%%%%%%%%%%%%%%%%%%%%%%%%%%%%%%%%%%%%%%%%%%%%%%%%%%%%
\subsection{Conditional Processing}
\label{sec:conditional}

The package provides a mechanism to compile different versions
of a document. To customise the versions further some conditional processing
can come in handy to distinguish which version is being compiled.
The package provides two macros to describe the compilation context:

%%%%%%%%%%%%%%%%%%%%%%%%%%%%%%%%%%%%%%%%
\DescribeMacro{\ifchilddoc}
The conditional |\ifchilddoc| distinguishes between the compilation of
child documents and the main document:
%
\begin{center}
|\ifchilddoc |\textit{child-code}| |[|\||else |\textit{main-code}]| \||fi|
\end{center}

%%%%%%%%%%%%%%%%%%%%%%%%%%%%%%%%%%%%%%%%
\DescribeMacro{\childdocname}
\DescribeMacro{\childdocjob}
The macro |\childdocname| contains the filename (without extension)
of the main or child file being processed.
Note that |\childdocjob| will always contain the name of the main file.

%%%%%%%%%%%%%%%%%%%%%%%%%%%%%%%%%%%%%%%%
\paragraph{Title Page.}

Conditional processing can be used to include a title or banner page
in the main document when proper precautions are taken.
Importantly, the code in the main file should ensure that the page counter
(as well as other status parameters which are stored in the |.aux| files)
takes the same value after the conditional processing.
Otherwise the page numbers may take divergent values
depending on which part is compiled.

For example, a title page could be declared by:
%
\begin{center}
\begin{tabular}{l}
|\ifchilddoc\||else|\\
|\addtocounter{page}{-1}|\\
\textit{code for title page}\\
|\newpage|\\
|\||fi|
\end{tabular}
\end{center}
%
A banner page for the child documents can be generated by:
%
\begin{center}
\begin{tabular}{l}
|\ifchilddoc|\\
|\addtocounter{page}{-1}|\\
\textit{code for banner page}\\
|\newpage|\\
|\||fi|
\end{tabular}
\end{center}
%
Here one could write a message such as:
\begin{center}
|This is the part \childdocname{} of \childdocjob{}.|
\end{center}

%%%%%%%%%%%%%%%%%%%%%%%%%%%%%%%%%%%%%%%%%%%%%%%%%%%%%%%%%%%%%%%%%%%%%%%%%%%%%%%%
\subsection{Flags}
\label{sec:flags}

The package makes it easy to generate different versions
of the main or child documents.
To this end compilation flags can be defined
and assigned different default values.
They will be particularly useful in conjunction
with the forwarding mechanism described in \secref{sec:forward}.

For example, it may be useful to have a flag |\version|
which can be set to |draft| or |final|.
The document source will contain some conditional code
depending on the value of |\version|.
Suppose further, the flag should default to |final| for the main file
and to |draft| for child files
which is a natural assignment for editing the document.
This is achieved by placing the following code
in the preamble of the main document
(below the |\childdocmain| directive):
%
\begin{center}
\begin{tabular}{l}
|\ifchilddoc|\\
|\providecommand{\version}{draft}|\\
|\||else|\\
|\providecommand{\version}{final}|\\
|\||fi|
\end{tabular}
\end{center}
%
The definition by |\providecommand| makes sure
that previous definitions are not overwritten.
Further statements |\providecommand{\version}{...}|
can thus be added before the above code to override it.

For the main file, one might add a line
(between |\childdocmain| and the above block)
%
\begin{center}
|%\ifchilddoc\||else\providecommand{\version}{draft}\||fi|
\end{center}
%
which can be uncommented to produce a draft version.
Likewise one can add a line to the very top of a child file
(above the |\childdocof{|\textit{main}|}| directive)
%
\begin{center}
|%\providecommand{\version}{final}|
\end{center}
%
which can be uncommented to produce the final version of this child document.

%%%%%%%%%%%%%%%%%%%%%%%%%%%%%%%%%%%%%%%%%%%%%%%%%%%%%%%%%%%%%%%%%%%%%%%%%%%%%%%%
\subsection{Forwarding}
\label{sec:forward}

Different versions of the main or child documents
using compilation flags as described in \secref{sec:flags}
can be (permanently) stored in different files
for convenient compilation, viewing and distribution.
To this end, the package defines a command
to pass on compilation to a different file:

%%%%%%%%%%%%%%%%%%%%%%%%%%%%%%%%%%%%%%%%
\DescribeMacro{\childdocforward}
The command |\childdocforward| redirects processing to
another source file:
%
\begin{center}
\begin{tabular}{l}
|\input{childdoc.def}|\\
|\childdocforward[|\textit{main}|]{|\textit{dest}|}|\\
\end{tabular}
\end{center}
%
The argument \textit{dest} is the destination file
(without extension).
It should be the main file or one of the child files.
Note that further \textsf{childdoc} directives
such as |\childdocof| and |\childdocforward|
in the indicated file will be processed in this form.
The optional argument \textit{main}
passes on directly to the main file \textit{main}
while pretending to compile the child \textit{dest}.
This form behaves as if \textit{dest}
issues |\childdocof{|\textit{main}|}| right away,
and no further \textsf{childdoc} directives will be processed.

%%%%%%%%%%%%%%%%%%%%%%%%%%%%%%%%%%%%%%%%
\DescribeMacro{\...prefix}
In the alternative form |\childdocforwardprefix|,
%
\begin{center}
\begin{tabular}{l}
|\input{childdoc.def}|\\
|\childdocforwardprefix[|\textit{main}|]{|\textit{prefix}|}{|\textit{dest}|}|
\end{tabular}
\end{center}
%
the destination file is determined by a pattern
depending on the current file:
To make this work, the current file must be called
`{\textit{prefix}\hspace{0.2em}\textit{suffix}}'
with \textit{prefix} matching precisely the argument.
Processing is then passed on to the file
`{\textit{dest}\hspace{0.2em}\textit{suffix}}'.
Surely, the same effect is achieved by
directly specifying the
argument `{\textit{dest}\hspace{0.2em}\textit{suffix}}'
in the first form.
However, that requires to set up a different file
for each child. With the alternative form of the command
all these files can have exactly the same content
which simplifies setting them up and maintaining them.

For example, the following file |draft.tex|
with a compilation flag |\version| as described in \secref{sec:flags}
compiles the main document as a draft:
%
\begin{center}
\begin{tabular}{l}
|\def\version{draft}|\\
|\input{childdoc.def}|\\
|\childdocforward{|\textit{main}|}|
\end{tabular}
\end{center}
%
Likewise, the following files |final|\textit{nn}|.tex|
compile the final version of the child document
|child|\textit{nn}|.tex|:
%
\begin{center}
\begin{tabular}{l}
|\def\version{final}|\\
|\input{childdoc.def}|\\
|\childdocforwardprefix{final}{child}|
\end{tabular}
\end{center}
%

Note that when several versions of a main file and/or of each child file
are to be generated, it may be convenient to set up a |Makefile| or
shell script to automatise the process.

%%%%%%%%%%%%%%%%%%%%%%%%%%%%%%%%%%%%%%%%%%%%%%%%%%%%%%%%%%%%%%%%%%%%%%%%%%%%%%%%
\subsection{Command Line Processing}
\label{sec:commandline}

The effect of redirection files can also be achieved by invoking
the \LaTeX{} compiler with a more elaborate command line.
Most conveniently this should be done as part
of a shell script or a |Makefile|.

When using \textsf{childdoc} in the main file, the following
command lines effectively perform a redirection
(note that depending on the shell being used,
backslashes may have to be doubled: `|\|' $\to$ `|\\|'):
%
\begin{center}
|... -jobname "|\textit{target}|" |\\|"|[\textit{flags}]%
|\input{childdoc.def}\childdocforward[|\textit{main}|]{|\textit{dest}|}"|
\end{center}
%
Here \textit{target} is the name of the output file,
\textit{main} is the name of the main file
and \textit{dest} is the name of the main or child file to be processed
(all filenames without extensions).
The optional argument \textit{main} can be omitted
if \textit{main} matches \textit{dest}.
Optionally, compilation \textit{flags} can be defined via |\def| commands.
This command line makes the \TeX{} engine believe
it is compiling the file \textit{target}
whose content is specified as the latter parameter.
The provided code then forwards the processing to
\textit{main} or \textit{dest} as described in \secref{sec:forward}.

%%%%%%%%%%%%%%%%%%%%%%%%%%%%%%%%%%%%%%%%%%%%%%%%%%%%%%%%%%%%%%%%%%%%%%%%%%%%%%%%
\subsection{Include by Input}
\label{sec:input}

Including child documents by |\include| has some restrictions by design.
Most notably, the content of a child document always occupies
its own set of pages; pages cannot be shared between child documents.
Usually, this behaviour makes perfect sense
because each child document contain an essential part of the document.
However, in some situations it may be desirable to compose
a document from a collection of parts
without having mandatory page breaks between then.
For this case, the package
provides a mechanism to include parts
by |\input| which can also be processed individually.
However, by construction this mechanism
requires manual handling of the content to be output.

%%%%%%%%%%%%%%%%%%%%%%%%%%%%%%%%%%%%%%%%
\DescribeMacro{\ifchilddocmanual}
The main file should be prepared as usual, see \secref{sec:include}.
However, the document body must make a distinction
between processing of an individual part and of the main document, e.g.:
%
\begin{center}
\begin{tabular}{l}
|\ifchilddocmanual|\\
|\input{\childdocname}|\\
|\||else|\\
\textit{document body with }|\input{|\textit{part}|}|\\
|\||fi|
\end{tabular}
\end{center}
%
The conditional |\ifchilddocmanual| is true whenever
a part to be included by |\input| is being compiled,
and the name of the part is stored in |\childdocname|.

%%%%%%%%%%%%%%%%%%%%%%%%%%%%%%%%%%%%%%%%
\DescribeMacro{\childdocby}
Each part to be included by |\input| should start with:
%
\begin{center}
\begin{tabular}{l}
|\input{childdoc.def}|\\
|\childdocby{|\textit{main}|}|\\
\end{tabular}
\end{center}
%
The directive |\childdocby| is similar to |\childdocof|
described in \secref{sec:include},
but the subsequent selection of content must be done manually.
To that end, both |\ifchilddoc| and |\ifchilddocmanual|
will be true upon processing of a part,
and the name of the part is stored in |\childdocname|.
Note that |\jobname| will be set to the filename of the current part
so that each part receives an individual |.aux| file
that does not interfere with the |.aux| file(s) of the main document.
This behaviour can be altered by the alternative form
|\childdocby[*]{|\textit{main}|}| (with a non-empty optional argument)
which uses the |.aux| file of the main document
by setting |\jobname| to \textit{main}.

%%%%%%%%%%%%%%%%%%%%%%%%%%%%%%%%%%%%%%%%%%%%%%%%%%%%%%%%%%%%%%%%%%%%%%%%%%%%%%%%
\subsection{Driver Development}
\label{sec:driver}

The \textsf{childdoc} mechanism can also be use for the development
of definition files such as \LaTeX{} styles or classes.
This case differs from the above setup with multiple parts
included by |\include| in that no |\includeonly| should be invoked.
This can be achieved by starting the include file
(before |\ProvidesPackage|) with:
%
\begin{center}
\begin{tabular}{l}
|\input{childdoc.def}|\\
|\childdocforward{|\textit{main}|}|\\
\end{tabular}
\end{center}
%
or alternatively with:
%
\begin{center}
\begin{tabular}{l}
|\input{childdoc.def}|\\
|\childdocby{|\textit{main}|}|\\
\end{tabular}
\end{center}
%
Both forms have slightly different effects as described above.
The main file is prepared as usual, see \secref{sec:include}.

%%%%%%%%%%%%%%%%%%%%%%%%%%%%%%%%%%%%%%%%%%%%%%%%%%%%%%%%%%%%%%%%%%%%%%%%%%%%%%%%
\subsection{Legacy Detection}
\label{sec:detection}

The directive |\childdocmain| in the main file can detect
whether the complete document or merely a child is to be compiled
even without using the directive |\childdocof|.
This method is deprecated because it is less robust
and there is no compelling reason to use it;
it is merely provided for backward compatibility
and it may be removed in future versions.

If the detection mechanism is to be used,
it is mandatory to correctly specify
the filename of the main file as the argument of |\childdocmain|:
%
\begin{center}
\begin{tabular}{l}
|\input{childdoc.def}|\\
|\childdocmain{|\textit{main}|}|\\
\end{tabular}
\end{center}
%
If |\jobname| does not match the argument \textit{main} of |\childdocmain|,
it is assumed that |\jobname| points to the child file to be compiled.
When using |\childdocmain| with the main file specified as argument,
it suffices to start a child file
with just |\input{|\textit{main}|}|
without loading of the package and using |\childdocof|.
If instead all processing is done
with the appropriate \textsf{childdoc} directives,
the argument of \textit{main} of |\childdocmain| can be empty.

An alternative version of the command line processing described
in \secref{sec:commandline} using the detection mechanism reads:
%
\begin{center}
|... -jobname "|\textit{target}|" "|[\textit{flags}]%
[|\def\jobname{|\textit{dest}|}|]|\input{|\textit{main}|}"|
\end{center}

%%%%%%%%%%%%%%%%%%%%%%%%%%%%%%%%%%%%%%%%%%%%%%%%%%%%%%%%%%%%%%%%%%%%%%%%%%%%%%%%
\subsection{Manual Code}
\label{sec:manual}

In case one cannot be certain whether the definitions file |childdoc.def|
is installed on the target \TeX{} distribution
and one prefers not to ship it,
it is conceivable to paste a few relevant commands into the sources.

To that end, drop all statements |\input{childdoc.def}|
and perform the replacements as outlined below.
Instead of |\childdocmain{|\textit{main}|}| add the following code
to the top of the main file:
%
\begin{center}
\begin{tabular}{l}
|\||ifdefined\childdocname\endinput\||fi\newif\ifchilddoc|\\
|\edef\childdocname{\scantokens\expandafter{\jobname\noexpand}}|\\
|\def\childdocmain{|\textit{main}|}\||ifx\childdocmain\childdocname\||else|\\
|\childdoctrue\includeonly{\childdocname}\let\jobname\childdocmain\||fi|\\
\end{tabular}
\end{center}
%
Instead of |\childdocof{|\textit{main}|}| just include the main file
at the top of each child file:
%
\begin{center}
|\input{|\textit{main}|}|
\end{center}
%
A simple redirection |\childdocforward{|\textit{dest}|}| is achieved by:
%
\begin{center}
|\def\jobname{|\textit{dest}|}\input{\jobname}|
\end{center}
%
The redirection with prefix
|\childdocforwardprefix[|\textit{prefix}|]{|\textit{dest}|}|
is accomplished by:
%
\begin{center}
\begin{tabular}{l}
|{\edef\jobname{\scantokens\expandafter{\jobname\noexpand}}|\\
|\def\redirectjob |\textit{prefix}|#1~~~{\gdef\jobname{|\textit{dest}|#1}}|\\
|\expandafter\redirectjob\jobname~~~}\input{\jobname}|
\end{tabular}
\end{center}

In an alternative approach,
child documents can be compiled by a specific command line
without additional code or specific definitions:
%
\begin{center}
|... -jobname "|\textit{target}|" "|[\textit{flags}]%
|\includeonly{|\textit{dest}|}\input{|\textit{main}|}"|
\end{center}
%

%%%%%%%%%%%%%%%%%%%%%%%%%%%%%%%%%%%%%%%%%%%%%%%%%%%%%%%%%%%%%%%%%%%%%%%%%%%%%%%%
%%%%%%%%%%%%%%%%%%%%%%%%%%%%%%%%%%%%%%%%%%%%%%%%%%%%%%%%%%%%%%%%%%%%%%%%%%%%%%%%
\section{Information}

%%%%%%%%%%%%%%%%%%%%%%%%%%%%%%%%%%%%%%%%%%%%%%%%%%%%%%%%%%%%%%%%%%%%%%%%%%%%%%%%
\subsection{Copyright}

Copyright \copyright{} 2017--2018 Niklas Beisert

This work may be distributed and/or modified under the
conditions of the \LaTeX{} Project Public License, either version 1.3
of this license or (at your option) any later version.
The latest version of this license is in
  \url{http://www.latex-project.org/lppl.txt}
and version 1.3 or later is part of all distributions of \LaTeX{}
version 2005/12/01 or later.

This work has the LPPL maintenance status `maintained'.

The Current Maintainer of this work is Niklas Beisert.

This work consists of the files |README.txt|, |childdoc.ins| and |childdoc.dtx|
as well as the derived files |childdoc.def|, |cdocsamp.tex|
with |cdocsch1.tex|, |cdocsch2.tex|, |cdocspt3.tex|, |cdocspt4.tex|,
|cdocsdrf.tex|, |cdocsfn1.tex|, |cdocsfn2.tex|
as well as |childdoc.pdf|.

%%%%%%%%%%%%%%%%%%%%%%%%%%%%%%%%%%%%%%%%%%%%%%%%%%%%%%%%%%%%%%%%%%%%%%%%%%%%%%%%
\subsection{Files and Installation}

The package consists of the files:
%
\begin{center}
\begin{tabular}{ll}
    |README.txt|   & readme file \\
    |childdoc.ins| & installation file \\
    |childdoc.dtx| & source file \\
    |childdoc.def| & definition file \\
    |cdocsamp.tex| & sample main file \\
    |cdocsch1.tex| & sample include file \\
    |cdocsch2.tex| & sample include file \\
    |cdocspt3.tex| & sample part file \\
    |cdocspt4.tex| & sample part file \\
    |cdocsdrf.tex| & sample redirection file \\
    |cdocsfn1.tex| & sample redirection file \\
    |cdocsfn2.tex| & sample redirection file \\
    |childdoc.pdf| & manual
\end{tabular}
\end{center}
%
The distribution consists of the files
|README.txt|, |childdoc.ins| and |childdoc.dtx|.
%
\begin{itemize}
\item
Run (pdf)\LaTeX{} on |childdoc.dtx|
to compile the manual |childdoc.pdf| (this file).
\item
Run \LaTeX{} on |childdoc.ins| to create the definitions file |childdoc.def|
and the sample |cdocsamp.tex| with include files
|cdocsch1.tex|, |cdocsch2.tex|, |cdocspt3.tex|, |cdocspt4.tex|,
|cdocsdrf.tex|, |cdocsfn1.tex|, |cdocsfn2.tex|.
Then copy the file |childdoc.def| to an appropriate directory of your \LaTeX{}
distribution, e.g.\ \textit{texmf-root}|/tex/latex/childdoc|.
\end{itemize}

%%%%%%%%%%%%%%%%%%%%%%%%%%%%%%%%%%%%%%%%%%%%%%%%%%%%%%%%%%%%%%%%%%%%%%%%%%%%%%%%
\subsection{Related CTAN Packages}

There are several other packages which offer a similar functionality:
%
\begin{itemize}
\item
The packages
\href{http://ctan.org/pkg/docmute}{\textsf{docmute}},
\href{http://ctan.org/pkg/includex}{\textsf{includex}} and
\href{http://ctan.org/pkg/standalone}{\textsf{standalone}}
provide commands to include only the document body of
a child file thus allowing both files to be compiled individually.
\item
The packages \href{http://ctan.org/pkg/subdocs}{\textsf{subdocs}}
and \href{http://ctan.org/pkg/subfiles}{\textsf{subfiles}}
provide structures in which the main and child documents can be
encapsulated and allowing them to be compiled individually.
The inclusion mechanism is different from the conventional |\include|.
\item
The package \href{http://ctan.org/pkg/combine}{\textsf{combine}}
is an elaborate solution to combine several documents into one.
\end{itemize}
%
See also the CTAN topic \href{http://ctan.org/topic/subdocs}{\textsf{subdocs}}
for further related packages.
The present package differs from the above solutions in that
a document structure constructed with the conventional |\include| mechanism
just needs two extra commands at the top of every file
such that all constituent files can be compiled individually.

%%%%%%%%%%%%%%%%%%%%%%%%%%%%%%%%%%%%%%%%%%%%%%%%%%%%%%%%%%%%%%%%%%%%%%%%%%%%%%%%
%\subsection{Feature Suggestions}
%
%The following is a list of features which may be useful for future
%versions of this package:
%%
%\begin{itemize}
%\item
%\ldots
%\end{itemize}

%%%%%%%%%%%%%%%%%%%%%%%%%%%%%%%%%%%%%%%%%%%%%%%%%%%%%%%%%%%%%%%%%%%%%%%%%%%%%%%%
\subsection{Revision History}

%%%%%%%%%%%%%%%%%%%%%%%%%%%%%%%%%%%%%%%%
\paragraph{v2.0:} 2018/12/30

\begin{itemize}
\item
immediate forward processing
\item
added |\childdocby| mechanism
\item
manual restructured
\end{itemize}

%%%%%%%%%%%%%%%%%%%%%%%%%%%%%%%%%%%%%%%%
\paragraph{v1.6:} 2018/01/17

\begin{itemize}
\item
application for development of include files
\item
corrections to manual
\end{itemize}

%%%%%%%%%%%%%%%%%%%%%%%%%%%%%%%%%%%%%%%%
\paragraph{v1.5:} 2017/05/21

\begin{itemize}
\item
more complete structuring introduced
\item
|\childdocof| introduced
\item
|\childdoc| renamed to |\childdocmain|
\item
|\childredirect| renamed to |\childdocforward| and |\childdocforwardprefix|
and functionality expanded
\end{itemize}

%%%%%%%%%%%%%%%%%%%%%%%%%%%%%%%%%%%%%%%%
\paragraph{v1.0:} 2017/04/27

\begin{itemize}
\item
manual and install package
\item
first version published on CTAN
\end{itemize}

%%%%%%%%%%%%%%%%%%%%%%%%%%%%%%%%%%%%%%%%
\paragraph{v0.6:} 2017/04/26

\begin{itemize}
\item
redirection mechanism added
\end{itemize}

%%%%%%%%%%%%%%%%%%%%%%%%%%%%%%%%%%%%%%%%
\paragraph{v0.5:} 2017/04/26

\begin{itemize}
\item
functionality in definition file
\end{itemize}


%%%%%%%%%%%%%%%%%%%%%%%%%%%%%%%%%%%%%%%%%%%%%%%%%%%%%%%%%%%%%%%%%%%%%%%%%%%%%%%%
%%%%%%%%%%%%%%%%%%%%%%%%%%%%%%%%%%%%%%%%%%%%%%%%%%%%%%%%%%%%%%%%%%%%%%%%%%%%%%%%
%%%%%%%%%%%%%%%%%%%%%%%%%%%%%%%%%%%%%%%%%%%%%%%%%%%%%%%%%%%%%%%%%%%%%%%%%%%%%%%%
\appendix

\settowidth\MacroIndent{\rmfamily\scriptsize 000\ }

 \DocInput{childdoc.dtx}

\end{document}
%</driver>
% \fi
%
% %%%%%%%%%%%%%%%%%%%%%%%%%%%%%%%%%%%%%%%%%%%%%%%%%%%%%%%%%%%%%%%%%%%%%%%%%%%%%%
% %%%%%%%%%%%%%%%%%%%%%%%%%%%%%%%%%%%%%%%%%%%%%%%%%%%%%%%%%%%%%%%%%%%%%%%%%%%%%%
% \section{Sample}
%\iffalse
%<*samplemain>
%\fi
%
% The following presents a sample document
% with two chapters, two parts, a title page,
% a compile flag as well as three forwarding files to set the flag.
% It consists of eight |.tex| files:
% \begin{center}
% \begin{tabular}{ll}
% |cdocsamp.tex|&main file\\
% |cdocsch1.tex|&include file for chapter 1\\
% |cdocsch2.tex|&include file for chapter 2\\
% |cdocspt3.tex|&include file for part 3\\
% |cdocspt4.tex|&include file for part 4\\
% |cdocsdrf.tex|&forwarding file for main file in draft mode\\
% |cdocsfi1.tex|&forwarding file for final version of chapter 1\\
% |cdocsfi2.tex|&forwarding file for final version of chapter 2\\
% \end{tabular}
% \end{center}
% Each of the eight files can be compiled directly by the \LaTeX{} compiler.
%
% %%%%%%%%%%%%%%%%%%%%%%%%%%%%%%%%%%%%%%
% \paragraph{Main File.}
%
% The main file is called |cdocsamp.tex|.
%
% Load the \textsf{childdoc} definitions and
% declare the filename for the main document:
%    \begin{macrocode}
\input{childdoc.def}
\childdocmain{}
%    \end{macrocode}

% Optional override for |\version| flag:
%    \begin{macrocode}
%%\ifchilddoc\else\providecommand{\version}{draft}\fi
%    \end{macrocode}

% Define the default values for the |\version| flag
% (|final| for the main file and |draft| for childs):
%    \begin{macrocode}
\ifchilddoc
\providecommand{\version}{draft}
\else
\providecommand{\version}{final}
\fi
%    \end{macrocode}

% Load the standard document class:
%    \begin{macrocode}
\documentclass[12pt]{article}
%    \end{macrocode}

% Start the document body:
%    \begin{macrocode}
\begin{document}
%    \end{macrocode}

% Declare a title page.
% Print title, part of document being processed and version flag:
%    \begin{macrocode}
\addtocounter{page}{-1}
\begin{center}
{\LARGE\bfseries{}childdoc example\par}
\vspace{1cm}
\ifchilddoc
\ifchilddocmanual part\else chapter\fi:
`\childdocname' of `\childdocjob'\par
\else
main document: `\childdocjob'\par
\fi
version: \version\par
\end{center}
\newpage
%    \end{macrocode}

% Manually include selected file,
% otherwise process as usual:
%    \begin{macrocode}
\ifchilddocmanual
\section*{part `\childdocname'}
\input{\childdocname}
\else
%    \end{macrocode}

% Include the two chapters:
%    \begin{macrocode}
\include{cdocsch1}
\include{cdocsch2}
%    \end{macrocode}

% Include the two parts unless only chapters should be displayed:
%    \begin{macrocode}
\ifchilddoc\else
\section{part three}
\input{cdocspt3}
\section{part four}
\input{cdocspt4}
\fi
%    \end{macrocode}

% Process as usual until here:
%    \begin{macrocode}
\fi
%    \end{macrocode}

% End of document body:
%    \begin{macrocode}
\end{document}
%    \end{macrocode}
%\iffalse
%</samplemain>
%\fi
%
% %%%%%%%%%%%%%%%%%%%%%%%%%%%%%%%%%%%%%%
% \paragraph{Chapter Include Files.}
%
% The include files are called |cdocsch1.tex| and |cdocsch2.tex|.
%
%\iffalse
%<*samplechap1|samplechap2>
%\fi

% Optional override for |\version| flag:
%    \begin{macrocode}
%%\providecommand{\version}{final}
%    \end{macrocode}

% Include the main document:
%    \begin{macrocode}
\input{childdoc.def}
\childdocof{cdocsamp}
%    \end{macrocode}

%\iffalse
%</samplechap1|samplechap2>
%\fi
%
%\iffalse
%<*samplechap1>
%\fi
% Some text for chapter 1:
%    \begin{macrocode}
\section{one}
some text in chapter one
%    \end{macrocode}

%\iffalse
%</samplechap1>
%\fi
% Some text for chapter 2:
%\iffalse
%<*samplechap2>
%\fi
%    \begin{macrocode}
\section{two}
more text in chapter two
%    \end{macrocode}

%\iffalse
%</samplechap2>
%\fi
%
% %%%%%%%%%%%%%%%%%%%%%%%%%%%%%%%%%%%%%%
% \paragraph{Part Include Files.}
%
% The include files are called |cdocspt3.tex| and |cdocspt4.tex|.
%
%\iffalse
%<*samplepart3|samplepart4>
%\fi

% Optional override for |\version| flag:
%    \begin{macrocode}
%%\providecommand{\version}{final}
%    \end{macrocode}

% Include the main document:
%    \begin{macrocode}
\input{childdoc.def}
\childdocby{cdocsamp}
%    \end{macrocode}

%\iffalse
%</samplepart3|samplepart4>
%\fi
%
%\iffalse
%<*samplepart3>
%\fi
% Some text for part 3:
%    \begin{macrocode}
some text in part three
%    \end{macrocode}

%\iffalse
%</samplepart3>
%\fi
% Some text for part 4:
%\iffalse
%<*samplepart4>
%\fi
%    \begin{macrocode}
more text in part four
%    \end{macrocode}

%\iffalse
%</samplepart4>
%\fi
%
% %%%%%%%%%%%%%%%%%%%%%%%%%%%%%%%%%%%%%%
% \paragraph{Forwarding for a Complete Draft.}
%
% The following forwarding file |cdocsdrf.tex|
% compiles the main document in draft mode:
%\iffalse
%<*sampledraft>
%\fi
%    \begin{macrocode}
\def\version{draft}
\input{childdoc.def}
\childdocforward{cdocsamp}
%    \end{macrocode}

%\iffalse
%</sampledraft>
%\fi
%
% %%%%%%%%%%%%%%%%%%%%%%%%%%%%%%%%%%%%%%
% \paragraph{Forwarding for Final Version of the Chapters.}
%
% The following forwarding files |cdocsfn1.tex| and |cdocsfn2.tex|
% (with identical content)
% compile the final versions of the child documents
% |cdocsch1.tex| and |cdocsch2.tex|, respectively:
%\iffalse
%<*samplefinal>
%\fi
%    \begin{macrocode}
\def\version{final}
\input{childdoc.def}
\childdocforwardprefix[cdocsamp]{cdocsfn}{cdocsch}
%    \end{macrocode}

%\iffalse
%</samplefinal>
%\fi
%
% %%%%%%%%%%%%%%%%%%%%%%%%%%%%%%%%%%%%%%
% \paragraph{Command Line Processing.}
%
% The following three command lines generate the output files
% |cdocscld|, |cdocscl1| and |cdocscl2|
% which should be identical to
% |cdocsdrf|, |cdocsch1| and |cdocsfn2|, respectively:
% \begin{center}
% \begin{tabular}{l}
% |latex -jobname cdocscld \|\\
% |  "\def\version{draft}\input{childdoc.def}\childdocforward{cdocsamp}"|\\
% |latex -jobname cdocscl1 \|\\
% |  "\input{childdoc.def}\childdocforward[cdocsamp]{cdocsch1}"|\\
% |latex -jobname cdocscl2 \|\\
% |  "\def\version{final}\input{childdoc.def}\childdocforward{cdocsch2}"|
% \end{tabular}
% \end{center}
% Note that the trailing backslash on each first line
% merely continues the input to the second line
% (for convenient cut ant paste).
% Furthermore, the command |latex| can be replaced by any
% of its alternative versions such as |pdflatex|.
%
% %%%%%%%%%%%%%%%%%%%%%%%%%%%%%%%%%%%%%%%%%%%%%%%%%%%%%%%%%%%%%%%%%%%%%%%%%%%%%%
% %%%%%%%%%%%%%%%%%%%%%%%%%%%%%%%%%%%%%%%%%%%%%%%%%%%%%%%%%%%%%%%%%%%%%%%%%%%%%%
% \section{Implementation}
%\iffalse
%<*package>
%\fi
%
% This section describes the definitions file |childdoc.def|.

% The definitions cannot be loaded using |\usepackage| or |\RequirePackage|
% which has a mechanism to prevent loading a style file more than once.
% When loading the definitions by means of |\input|
% multiple instances have to be prevented manually:
%\iffalse
%This code needs to be before the `\ProvidesFile' directive
%which is defined at the beginning of this file.
%Therefore it is also placed there and commented out here.
%</package>
%<*discard>
%\fi
%    \begin{macrocode}
\ifdefined\childdocmain\endinput\fi
%    \end{macrocode}
%\iffalse
%</discard>
%<*package>
%\fi
%
% \macro{\ifchilddoc}
% \macro{\ifchilddocmanual}
% The conditional |\ifchilddoc| tells whether a
% child (true) or main (false) document is being compiled.
% The conditional |\ifchilddocmanual| tells whether
% the |\includeonly| mechanism is used (false) or
% the selection of child files must be performed manually (true).
% The definitions initialise to false:
%    \begin{macrocode}
\newif\ifchilddoc
\newif\ifchilddocmanual
%    \end{macrocode}

% \macro{\childdocname}
% \macro{\childdocjob}
% The macro |\childdocname| stores the name of the main document
% to be compiled. The macro |\childdocjob| stores the name of
% the document on which the \LaTeX{} compiler was originally invoked.
% The content of |\jobname| cannot be compared
% to filenames specified in the source due to different catcodes.
% The following code rescans |\jobname|, stores the result
% in |\childdocname| and saves a copy in |\childdocjob|:
%    \begin{macrocode}
\edef\childdocname{\scantokens\expandafter{\jobname\noexpand}}
\let\childdocjob\childdocname
%    \end{macrocode}

% \macro{\childdocdisable}
% The macro |\childdocdisable| prevents the main file
% from being processed more than once.
% At this stage, the main document command |\childdocmain|
% is assumed to be called once again where it should do nothing.
% Any subsequent call to it should prevent
% a secondary processing of the main document
% It overwrites the forwarding commands
% |\childdocof| and |\childdocforward|
% with empty macros to prevent further inclusions of the main document:
%    \begin{macrocode}
\newcommand{\childdocdisable}
{
  \renewcommand{\childdocmain}[1]{\renewcommand{\childdocmain}[1]{\endinput}}
  \renewcommand{\childdocof}[1]{}
  \renewcommand{\childdocby}[2][]{}
  \renewcommand{\childdocforward}[2][]{}
  \renewcommand{\childdocdisable}{}
}
%    \end{macrocode}

% \macro{\childdocmain}
% The macro |\childdocmain| is to be called at the top of the main file
% with nothing or the main filename (without extension) as argument.
% First, it breaks loops.
% If the argument is not empty and does not match |\childdocname|
% (which is set by the first inclusion of |childdoc.def|),
% |\ifchilddoc| is set to true, |\includeonly| is applied to the child file
% and |\jobname| is set to the main file
% (for proper handling of |.aux| files):
%    \begin{macrocode}
\newcommand{\childdocmain}[1]
{
  \childdocdisable\childdocmain{}
  \if?#1?\else
    \begingroup
      \def\childdoctmp{#1}
      \ifx\childdoctmp\childdocname
        \def\childdoctmp{}
      \else
        \def\childdoctmp
        {
          \childdoctrue
          \includeonly{\childdocname}
          \def\childdocjob{#1}
          \def\jobname{#1}
        }
      \fi
      \expandafter
    \endgroup
    \childdoctmp
  \fi
}
%    \end{macrocode}

% \macro{\childdocof}
% The command |\childdocof| redirects
% compilation to the main file |#1|.
%    \begin{macrocode}
\newcommand{\childdocof}[1]
{
  \childdocdisable
  \childdoctrue
  \includeonly{\childdocname}
  \def\jobname{#1}
  \def\childdocjob{#1}
  \input{#1}
}
%    \end{macrocode}

% \macro{\childdocby}
% The command |\childdocby| ....
%    \begin{macrocode}
\newcommand{\childdocby}[2][]
{
  \childdocdisable
  \childdoctrue
  \childdocmanualtrue
  \if?#1?\else
    \def\jobname{#2}
  \fi
  \def\childdocjob{#2}
  \input{#2}
  \endinput
}
%    \end{macrocode}

% \macro{\childdocforward}
% The command |\childdocforward| redirects
% compilation to the main file or
% (if the optional argument is given) a child file.
% Parameters are set as if the main file
% or a child file starting with |\childdocof| was compiled.
% Then compilation is handed over to the main file:
%    \begin{macrocode}
\newcommand{\childdocforward}[2][]
{
  \begingroup
    \if?#1?
      \def\childdoctmp
      {
        \def\childdocname{#2}
        \def\childdocjob{#2}
        \def\jobname{#2}
        \input{#2}
        \endinput
      }
    \else
      \def\childdoctmp
      {
        \childdocdisable
        \def\childdocname{#2}
        \childdoctrue
        \includeonly{#2}
        \def\childdocjob{#1}
        \def\jobname{#1}
        \input{#1}
        \endinput
      }
    \fi
    \expandafter
  \endgroup
  \childdoctmp
}
%    \end{macrocode}

% \macro{\childdocforwardprefix}
% The command |\childdocforwardprefix| redirects
% compilation to the main or a child file by means of a pattern.
% The prefix |#1| in the current filename is replaced by |#2|
% and the suffix of the current filename is kept
% (it is assumed that the filename does not contain the substring `|~~~|'
% which is used as a delimiter).
% Compilation is handed over to the new file by |\childdocforward|:
%    \begin{macrocode}
\newcommand{\childdocforwardprefix}[3][]
{
  \begingroup
    \def\childdocextract #2##1~~~{\def\childdoctmp{\childdocforward[#1]{#3##1}}}
    \expandafter\childdocextract\childdocname~~~
    \expandafter
  \endgroup
  \childdoctmp
}
%    \end{macrocode}

% \macro{\childdoc}
% The deprecated macro |\childdoc| is a legacy version of |\childdocmain|:
%    \begin{macrocode}
\newcommand{\childdoc}{\childdocmain}
%    \end{macrocode}

% \macro{\childdocredirect}
% The deprecated macro |\childdocredirect| is a legacy version
% of |\childdocforward| and |\childdocforwardprefix|:
%    \begin{macrocode}
\newcommand{\childdocredirect}[2][]
{
  \begingroup
    \if?#1?
      \def\childdoctmp{\childdocforward{#2}}
    \else
      \def\childdoctmp{\childdocforwardprefix{#1}{#2}}
    \fi
    \expandafter
  \endgroup
  \childdoctmp
}
%    \end{macrocode}

%\iffalse
%</package>
%\fi
%
\endinput
|
and perform the replacements as outlined below.
Instead of |\childdocmain{|\textit{main}|}| add the following code
to the top of the main file:
%
\begin{center}
\begin{tabular}{l}
|\||ifdefined\childdocname\endinput\||fi\newif\ifchilddoc|\\
|\edef\childdocname{\scantokens\expandafter{\jobname\noexpand}}|\\
|\def\childdocmain{|\textit{main}|}\||ifx\childdocmain\childdocname\||else|\\
|\childdoctrue\includeonly{\childdocname}\let\jobname\childdocmain\||fi|\\
\end{tabular}
\end{center}
%
Instead of |\childdocof{|\textit{main}|}| just include the main file
at the top of each child file:
%
\begin{center}
|\input{|\textit{main}|}|
\end{center}
%
A simple redirection |\childdocforward{|\textit{dest}|}| is achieved by:
%
\begin{center}
|\def\jobname{|\textit{dest}|}\input{\jobname}|
\end{center}
%
The redirection with prefix
|\childdocforwardprefix[|\textit{prefix}|]{|\textit{dest}|}|
is accomplished by:
%
\begin{center}
\begin{tabular}{l}
|{\edef\jobname{\scantokens\expandafter{\jobname\noexpand}}|\\
|\def\redirectjob |\textit{prefix}|#1~~~{\gdef\jobname{|\textit{dest}|#1}}|\\
|\expandafter\redirectjob\jobname~~~}\input{\jobname}|
\end{tabular}
\end{center}

In an alternative approach,
child documents can be compiled by a specific command line
without additional code or specific definitions:
%
\begin{center}
|... -jobname "|\textit{target}|" "|[\textit{flags}]%
|\includeonly{|\textit{dest}|}\input{|\textit{main}|}"|
\end{center}
%

%%%%%%%%%%%%%%%%%%%%%%%%%%%%%%%%%%%%%%%%%%%%%%%%%%%%%%%%%%%%%%%%%%%%%%%%%%%%%%%%
%%%%%%%%%%%%%%%%%%%%%%%%%%%%%%%%%%%%%%%%%%%%%%%%%%%%%%%%%%%%%%%%%%%%%%%%%%%%%%%%
\section{Information}

%%%%%%%%%%%%%%%%%%%%%%%%%%%%%%%%%%%%%%%%%%%%%%%%%%%%%%%%%%%%%%%%%%%%%%%%%%%%%%%%
\subsection{Copyright}

Copyright \copyright{} 2017--2018 Niklas Beisert

This work may be distributed and/or modified under the
conditions of the \LaTeX{} Project Public License, either version 1.3
of this license or (at your option) any later version.
The latest version of this license is in
  \url{http://www.latex-project.org/lppl.txt}
and version 1.3 or later is part of all distributions of \LaTeX{}
version 2005/12/01 or later.

This work has the LPPL maintenance status `maintained'.

The Current Maintainer of this work is Niklas Beisert.

This work consists of the files |README.txt|, |childdoc.ins| and |childdoc.dtx|
as well as the derived files |childdoc.def|, |cdocsamp.tex|
with |cdocsch1.tex|, |cdocsch2.tex|, |cdocspt3.tex|, |cdocspt4.tex|,
|cdocsdrf.tex|, |cdocsfn1.tex|, |cdocsfn2.tex|
as well as |childdoc.pdf|.

%%%%%%%%%%%%%%%%%%%%%%%%%%%%%%%%%%%%%%%%%%%%%%%%%%%%%%%%%%%%%%%%%%%%%%%%%%%%%%%%
\subsection{Files and Installation}

The package consists of the files:
%
\begin{center}
\begin{tabular}{ll}
    |README.txt|   & readme file \\
    |childdoc.ins| & installation file \\
    |childdoc.dtx| & source file \\
    |childdoc.def| & definition file \\
    |cdocsamp.tex| & sample main file \\
    |cdocsch1.tex| & sample include file \\
    |cdocsch2.tex| & sample include file \\
    |cdocspt3.tex| & sample part file \\
    |cdocspt4.tex| & sample part file \\
    |cdocsdrf.tex| & sample redirection file \\
    |cdocsfn1.tex| & sample redirection file \\
    |cdocsfn2.tex| & sample redirection file \\
    |childdoc.pdf| & manual
\end{tabular}
\end{center}
%
The distribution consists of the files
|README.txt|, |childdoc.ins| and |childdoc.dtx|.
%
\begin{itemize}
\item
Run (pdf)\LaTeX{} on |childdoc.dtx|
to compile the manual |childdoc.pdf| (this file).
\item
Run \LaTeX{} on |childdoc.ins| to create the definitions file |childdoc.def|
and the sample |cdocsamp.tex| with include files
|cdocsch1.tex|, |cdocsch2.tex|, |cdocspt3.tex|, |cdocspt4.tex|,
|cdocsdrf.tex|, |cdocsfn1.tex|, |cdocsfn2.tex|.
Then copy the file |childdoc.def| to an appropriate directory of your \LaTeX{}
distribution, e.g.\ \textit{texmf-root}|/tex/latex/childdoc|.
\end{itemize}

%%%%%%%%%%%%%%%%%%%%%%%%%%%%%%%%%%%%%%%%%%%%%%%%%%%%%%%%%%%%%%%%%%%%%%%%%%%%%%%%
\subsection{Related CTAN Packages}

There are several other packages which offer a similar functionality:
%
\begin{itemize}
\item
The packages
\href{http://ctan.org/pkg/docmute}{\textsf{docmute}},
\href{http://ctan.org/pkg/includex}{\textsf{includex}} and
\href{http://ctan.org/pkg/standalone}{\textsf{standalone}}
provide commands to include only the document body of
a child file thus allowing both files to be compiled individually.
\item
The packages \href{http://ctan.org/pkg/subdocs}{\textsf{subdocs}}
and \href{http://ctan.org/pkg/subfiles}{\textsf{subfiles}}
provide structures in which the main and child documents can be
encapsulated and allowing them to be compiled individually.
The inclusion mechanism is different from the conventional |\include|.
\item
The package \href{http://ctan.org/pkg/combine}{\textsf{combine}}
is an elaborate solution to combine several documents into one.
\end{itemize}
%
See also the CTAN topic \href{http://ctan.org/topic/subdocs}{\textsf{subdocs}}
for further related packages.
The present package differs from the above solutions in that
a document structure constructed with the conventional |\include| mechanism
just needs two extra commands at the top of every file
such that all constituent files can be compiled individually.

%%%%%%%%%%%%%%%%%%%%%%%%%%%%%%%%%%%%%%%%%%%%%%%%%%%%%%%%%%%%%%%%%%%%%%%%%%%%%%%%
%\subsection{Feature Suggestions}
%
%The following is a list of features which may be useful for future
%versions of this package:
%%
%\begin{itemize}
%\item
%\ldots
%\end{itemize}

%%%%%%%%%%%%%%%%%%%%%%%%%%%%%%%%%%%%%%%%%%%%%%%%%%%%%%%%%%%%%%%%%%%%%%%%%%%%%%%%
\subsection{Revision History}

%%%%%%%%%%%%%%%%%%%%%%%%%%%%%%%%%%%%%%%%
\paragraph{v2.0:} 2018/12/30

\begin{itemize}
\item
immediate forward processing
\item
added |\childdocby| mechanism
\item
manual restructured
\end{itemize}

%%%%%%%%%%%%%%%%%%%%%%%%%%%%%%%%%%%%%%%%
\paragraph{v1.6:} 2018/01/17

\begin{itemize}
\item
application for development of include files
\item
corrections to manual
\end{itemize}

%%%%%%%%%%%%%%%%%%%%%%%%%%%%%%%%%%%%%%%%
\paragraph{v1.5:} 2017/05/21

\begin{itemize}
\item
more complete structuring introduced
\item
|\childdocof| introduced
\item
|\childdoc| renamed to |\childdocmain|
\item
|\childredirect| renamed to |\childdocforward| and |\childdocforwardprefix|
and functionality expanded
\end{itemize}

%%%%%%%%%%%%%%%%%%%%%%%%%%%%%%%%%%%%%%%%
\paragraph{v1.0:} 2017/04/27

\begin{itemize}
\item
manual and install package
\item
first version published on CTAN
\end{itemize}

%%%%%%%%%%%%%%%%%%%%%%%%%%%%%%%%%%%%%%%%
\paragraph{v0.6:} 2017/04/26

\begin{itemize}
\item
redirection mechanism added
\end{itemize}

%%%%%%%%%%%%%%%%%%%%%%%%%%%%%%%%%%%%%%%%
\paragraph{v0.5:} 2017/04/26

\begin{itemize}
\item
functionality in definition file
\end{itemize}


%%%%%%%%%%%%%%%%%%%%%%%%%%%%%%%%%%%%%%%%%%%%%%%%%%%%%%%%%%%%%%%%%%%%%%%%%%%%%%%%
%%%%%%%%%%%%%%%%%%%%%%%%%%%%%%%%%%%%%%%%%%%%%%%%%%%%%%%%%%%%%%%%%%%%%%%%%%%%%%%%
%%%%%%%%%%%%%%%%%%%%%%%%%%%%%%%%%%%%%%%%%%%%%%%%%%%%%%%%%%%%%%%%%%%%%%%%%%%%%%%%
\appendix

\settowidth\MacroIndent{\rmfamily\scriptsize 000\ }

 \DocInput{childdoc.dtx}

\end{document}
%</driver>
% \fi
%
% %%%%%%%%%%%%%%%%%%%%%%%%%%%%%%%%%%%%%%%%%%%%%%%%%%%%%%%%%%%%%%%%%%%%%%%%%%%%%%
% %%%%%%%%%%%%%%%%%%%%%%%%%%%%%%%%%%%%%%%%%%%%%%%%%%%%%%%%%%%%%%%%%%%%%%%%%%%%%%
% \section{Sample}
%\iffalse
%<*samplemain>
%\fi
%
% The following presents a sample document
% with two chapters, two parts, a title page,
% a compile flag as well as three forwarding files to set the flag.
% It consists of eight |.tex| files:
% \begin{center}
% \begin{tabular}{ll}
% |cdocsamp.tex|&main file\\
% |cdocsch1.tex|&include file for chapter 1\\
% |cdocsch2.tex|&include file for chapter 2\\
% |cdocspt3.tex|&include file for part 3\\
% |cdocspt4.tex|&include file for part 4\\
% |cdocsdrf.tex|&forwarding file for main file in draft mode\\
% |cdocsfi1.tex|&forwarding file for final version of chapter 1\\
% |cdocsfi2.tex|&forwarding file for final version of chapter 2\\
% \end{tabular}
% \end{center}
% Each of the eight files can be compiled directly by the \LaTeX{} compiler.
%
% %%%%%%%%%%%%%%%%%%%%%%%%%%%%%%%%%%%%%%
% \paragraph{Main File.}
%
% The main file is called |cdocsamp.tex|.
%
% Load the \textsf{childdoc} definitions and
% declare the filename for the main document:
%    \begin{macrocode}
% \iffalse
%
% childdoc.dtx Copyright (C) 2017-2018 Niklas Beisert
%
% This work may be distributed and/or modified under the
% conditions of the LaTeX Project Public License, either version 1.3
% of this license or (at your option) any later version.
% The latest version of this license is in
%   http://www.latex-project.org/lppl.txt
% and version 1.3 or later is part of all distributions of LaTeX
% version 2005/12/01 or later.
%
% This work has the LPPL maintenance status `maintained'.
%
% The Current Maintainer of this work is Niklas Beisert.
%
% This work consists of the files childdoc.dtx and childdoc.ins
% and the derived files childdoc.def and cdocsamp.tex with
% cdocsch1.tex, cdocsch2.tex, cdocsdrf.tex, cdocsfn1.tex, cdocsfn2.tex.
%
%<package>\ifdefined\childdocmain\endinput\fi
%<package>\ProvidesFile{childdoc.def}[2018/12/30 v2.0 child document driver]
%<samplemain>\ProvidesFile{cdocsamp.tex}[2018/12/30 v2.0 sample for childdoc]
%<*driver>
%\ProvidesFile{childdoc.drv}[2018/12/30 v2.0 childdoc reference manual file]
\PassOptionsToClass{10pt,a4paper}{article}
\documentclass{ltxdoc}

\usepackage[margin=35mm]{geometry}
\usepackage{hyperref}
\usepackage{hyperxmp}
\usepackage[usenames]{color}

\hypersetup{colorlinks=true}
\hypersetup{pdfstartview=FitH}
\hypersetup{pdfpagemode=UseNone}
\hypersetup{pdfsource={}}
\hypersetup{pdflang={en-UK}}
\hypersetup{pdfcopyright={Copyright 2017-2018 Niklas Beisert.
  This work may be distributed and/or modified under the
  conditions of the LaTeX Project Public License, either version 1.3
  of this license or (at your option) any later version.}}
\hypersetup{pdflicenseurl={http://www.latex-project.org/lppl.txt}}
\hypersetup{pdfcontactaddress={ETH Zurich, ITP, HIT K,
  Wolfgang-Pauli-Strasse 27}}
\hypersetup{pdfcontactpostcode={8093}}
\hypersetup{pdfcontactcity={Zurich}}
\hypersetup{pdfcontactcountry={Switzerland}}
\hypersetup{pdfcontactemail={nbeisert@itp.phys.ethz.ch}}
\hypersetup{pdfcontacturl={http://people.phys.ethz.ch/\xmptilde nbeisert/}}

\newcommand{\secref}[1]{\hyperref[#1]{section \ref*{#1}}}

\parskip1ex
\parindent0pt
\let\olditemize\itemize
\def\itemize{\olditemize\parskip0pt}

\begin{document}

\title{The \textsf{childdoc} Package}
\hypersetup{pdftitle={The childdoc Package}}
\author{Niklas Beisert\\[2ex]
  Institut f\"ur Theoretische Physik\\
  Eidgen\"ossische Technische Hochschule Z\"urich\\
  Wolfgang-Pauli-Strasse 27, 8093 Z\"urich, Switzerland\\[1ex]
  \href{mailto:nbeisert@itp.phys.ethz.ch}
  {\texttt{nbeisert@itp.phys.ethz.ch}}}
\hypersetup{pdfauthor={Niklas Beisert}}
\hypersetup{pdfsubject={Manual for the LaTeX2e Package childdoc}}
\date{30 December 2018, \textsf{v2.0}}
\maketitle

\begin{abstract}\noindent
\textsf{childdoc} is a \LaTeXe{} package
that enables the direct compilation
of document sections included by |\include|
to individual files.
\end{abstract}

\begingroup
\parskip0ex
\tableofcontents
\endgroup

%%%%%%%%%%%%%%%%%%%%%%%%%%%%%%%%%%%%%%%%%%%%%%%%%%%%%%%%%%%%%%%%%%%%%%%%%%%%%%%%
%%%%%%%%%%%%%%%%%%%%%%%%%%%%%%%%%%%%%%%%%%%%%%%%%%%%%%%%%%%%%%%%%%%%%%%%%%%%%%%%
\section{Introduction}

\LaTeX{} provides a mechanism to structure a large document (such as a book)
into a main file and several child files (containing the chapters)
using the |\include| command.
This mechanism is beneficial for documents
which span hundreds of pages in order to
make the source file(s) more manageable.
Moreover, compilation can be restricted to
selected child files by means of the |\includeonly| command.
The latter feature can be used to reduce the compilation time while editing
(this was significantly more useful in the earlier days of \LaTeX{})
or to generate a smaller document which is easier to navigate.
Another application of |\includeonly| is to generate
documents consisting of selected parts of the complete document.

However, there are a few drawbacks of the plain |\include| mechanism:
\begin{itemize}
\item
The child files cannot be compiled on their own,
they can only be compiled via the main file.
A naive editing environment
(such as a text editor with an option
to have the current file processed by \LaTeX)
may require one to switch to the main file before compiling;
attempting to compile the child file produces errors.
\item
The main file must be modified (each time)
to adjust the |\includeonly| command
to the present needs. This easily leaves the main file in a messy state.
\item
The generated document will always carry the filename
of the main document. This is inconvenient if
several child files are to be compiled and
to be kept for distribution.
\end{itemize}

The present package provides a simple interface
to make child files individually compilable by \LaTeX{}.
Compiling a child file then has the same effect as compiling
the main file with an |\includeonly| command
to select the appropriate child.
Moreover the generated document will carry the name of the child
rather than the main file.
This resolves all three above issues.

This feature is meant to make the editing of books,
thesis documents and lecture notes somewhat more convenient.
However, the package can also be used efficiently for
composing a series of documents (such as exercise sheets)
which are typically distributed individually.
It then assists the author in generating the individual documents
(potentially in different versions)
as well as a document containing the collected series.
Another application is in developing style files
or other kinds of included material
where compilation of the style file could redirect
to a sample or test file.

%%%%%%%%%%%%%%%%%%%%%%%%%%%%%%%%%%%%%%%%%%%%%%%%%%%%%%%%%%%%%%%%%%%%%%%%%%%%%%%%
%%%%%%%%%%%%%%%%%%%%%%%%%%%%%%%%%%%%%%%%%%%%%%%%%%%%%%%%%%%%%%%%%%%%%%%%%%%%%%%%
\section{Usage}

First of all, the package \textsf{childdoc} is \emph{not} a standard
\LaTeXe{} |.sty| style file! Therefore it needs to be invoked in
a non-standard way.

%%%%%%%%%%%%%%%%%%%%%%%%%%%%%%%%%%%%%%%%%%%%%%%%%%%%%%%%%%%%%%%%%%%%%%%%%%%%%%%%
\subsection{Included Files}
\label{sec:include}

%%%%%%%%%%%%%%%%%%%%%%%%%%%%%%%%%%%%%%%%
\DescribeMacro{\childdocmain}
To use the package, add the commands
\begin{center}
\begin{tabular}{l}
|\input{childdoc.def}|\\
|\childdocmain{}|\\
\end{tabular}
\end{center}
at the very top of the main \LaTeX{} file,
in particular \emph{before} the |\documentclass| statement!
The argument of |\childdocmain| should be left empty
(but it must be present).

%%%%%%%%%%%%%%%%%%%%%%%%%%%%%%%%%%%%%%%%
\DescribeMacro{\childdocof}
Furthermore, add the commands
\begin{center}
\begin{tabular}{l}
|\input{childdoc.def}|\\
|\childdocof{|\textit{main}|}|\\
\end{tabular}
\end{center}
at the top of every child file \textit{child}
which is included by |\include{|\textit{child}|}|
from within the main file
(or at least for those files to be compiled individually).
The argument \textit{main} must be the filename of the main file.

There are a couple of
considerations in setting up the main and child documents:

%%%%%%%%%%%%%%%%%%%%%%%%%%%%%%%%%%%%%%%%
\paragraph{Restrictions.}

Please note the following restrictions:
\begin{itemize}
\item
|\childdocmain| must be called with one argument \textit{main}
to ensure compatibility with earlier version of the package.
It must either be empty (|\childdocmain{}|)
or precisely match the filename of the main file in which it is specified.
See \secref{sec:detection} for further information.
\item
The filename \textit{main} must be specified without the |.tex| extension.
\item
The filename \textit{main} is case sensitive
(even in case-insensitive file systems)
due to internal string comparison.
\item
The argument \textit{main} should be fully expanded, it cannot be a macro.
\item
Subdirectories and special characters should be avoided in filenames.
\item
The command |\childdocmain{|\textit{main}|}| must be followed by a whitespace.
It should not be followed immediately by another command
or by a comment mark `|%|'.
This is because the \TeX{} parser reads the token immediately following
the argument of |\childdocmain| and puts it
at the beginning of every child section;
however, a white\-space is ignored.
\end{itemize}

%%%%%%%%%%%%%%%%%%%%%%%%%%%%%%%%%%%%%%%%
\paragraph{Content of Main File.}

It is advisable to place all content in the child files included by |\include|.
Any output contained in the main file will appear in all child documents
unless suppressed manually;
it cannot be suppressed automatically by the |\includeonly| directive
and thus should normally be avoided.
A method to include some content in the main file
by means of conditional processing is described in \secref{sec:conditional}.

%%%%%%%%%%%%%%%%%%%%%%%%%%%%%%%%%%%%%%%%
\paragraph{Page Numbering.}

When only a part of the document is compiled,
the appropriate numbering of pages
(as well as other status parameters)
is determined from the |.aux| files.
The latter contain information from previous passes.
However this information needs to propagate through
all intermediate child documents.
Therefore the page numbering in child documents may well
be inconsistent until the complete document is compiled at least once.

A useful (if unconventional) way to always ensure a consistent
page numbering is to restart the numbering in each child document
and denote the pages by `\textit{child}|.|\textit{page}'
where \textit{child} represents the chapter/section number of the child file.
This can be achieved by the command
|\numberwithin{page}{|\textit{child}|}|
of the \textsf{amsmath} package
where \textit{child} can be |chapter| or |section|
depending on the chosen structuring.
Alternatively, one can modify the macro |\thepage| appropriately
and reset the counter |page| at the start of each child file.

%%%%%%%%%%%%%%%%%%%%%%%%%%%%%%%%%%%%%%%%%%%%%%%%%%%%%%%%%%%%%%%%%%%%%%%%%%%%%%%%
\subsection{Conditional Processing}
\label{sec:conditional}

The package provides a mechanism to compile different versions
of a document. To customise the versions further some conditional processing
can come in handy to distinguish which version is being compiled.
The package provides two macros to describe the compilation context:

%%%%%%%%%%%%%%%%%%%%%%%%%%%%%%%%%%%%%%%%
\DescribeMacro{\ifchilddoc}
The conditional |\ifchilddoc| distinguishes between the compilation of
child documents and the main document:
%
\begin{center}
|\ifchilddoc |\textit{child-code}| |[|\||else |\textit{main-code}]| \||fi|
\end{center}

%%%%%%%%%%%%%%%%%%%%%%%%%%%%%%%%%%%%%%%%
\DescribeMacro{\childdocname}
\DescribeMacro{\childdocjob}
The macro |\childdocname| contains the filename (without extension)
of the main or child file being processed.
Note that |\childdocjob| will always contain the name of the main file.

%%%%%%%%%%%%%%%%%%%%%%%%%%%%%%%%%%%%%%%%
\paragraph{Title Page.}

Conditional processing can be used to include a title or banner page
in the main document when proper precautions are taken.
Importantly, the code in the main file should ensure that the page counter
(as well as other status parameters which are stored in the |.aux| files)
takes the same value after the conditional processing.
Otherwise the page numbers may take divergent values
depending on which part is compiled.

For example, a title page could be declared by:
%
\begin{center}
\begin{tabular}{l}
|\ifchilddoc\||else|\\
|\addtocounter{page}{-1}|\\
\textit{code for title page}\\
|\newpage|\\
|\||fi|
\end{tabular}
\end{center}
%
A banner page for the child documents can be generated by:
%
\begin{center}
\begin{tabular}{l}
|\ifchilddoc|\\
|\addtocounter{page}{-1}|\\
\textit{code for banner page}\\
|\newpage|\\
|\||fi|
\end{tabular}
\end{center}
%
Here one could write a message such as:
\begin{center}
|This is the part \childdocname{} of \childdocjob{}.|
\end{center}

%%%%%%%%%%%%%%%%%%%%%%%%%%%%%%%%%%%%%%%%%%%%%%%%%%%%%%%%%%%%%%%%%%%%%%%%%%%%%%%%
\subsection{Flags}
\label{sec:flags}

The package makes it easy to generate different versions
of the main or child documents.
To this end compilation flags can be defined
and assigned different default values.
They will be particularly useful in conjunction
with the forwarding mechanism described in \secref{sec:forward}.

For example, it may be useful to have a flag |\version|
which can be set to |draft| or |final|.
The document source will contain some conditional code
depending on the value of |\version|.
Suppose further, the flag should default to |final| for the main file
and to |draft| for child files
which is a natural assignment for editing the document.
This is achieved by placing the following code
in the preamble of the main document
(below the |\childdocmain| directive):
%
\begin{center}
\begin{tabular}{l}
|\ifchilddoc|\\
|\providecommand{\version}{draft}|\\
|\||else|\\
|\providecommand{\version}{final}|\\
|\||fi|
\end{tabular}
\end{center}
%
The definition by |\providecommand| makes sure
that previous definitions are not overwritten.
Further statements |\providecommand{\version}{...}|
can thus be added before the above code to override it.

For the main file, one might add a line
(between |\childdocmain| and the above block)
%
\begin{center}
|%\ifchilddoc\||else\providecommand{\version}{draft}\||fi|
\end{center}
%
which can be uncommented to produce a draft version.
Likewise one can add a line to the very top of a child file
(above the |\childdocof{|\textit{main}|}| directive)
%
\begin{center}
|%\providecommand{\version}{final}|
\end{center}
%
which can be uncommented to produce the final version of this child document.

%%%%%%%%%%%%%%%%%%%%%%%%%%%%%%%%%%%%%%%%%%%%%%%%%%%%%%%%%%%%%%%%%%%%%%%%%%%%%%%%
\subsection{Forwarding}
\label{sec:forward}

Different versions of the main or child documents
using compilation flags as described in \secref{sec:flags}
can be (permanently) stored in different files
for convenient compilation, viewing and distribution.
To this end, the package defines a command
to pass on compilation to a different file:

%%%%%%%%%%%%%%%%%%%%%%%%%%%%%%%%%%%%%%%%
\DescribeMacro{\childdocforward}
The command |\childdocforward| redirects processing to
another source file:
%
\begin{center}
\begin{tabular}{l}
|\input{childdoc.def}|\\
|\childdocforward[|\textit{main}|]{|\textit{dest}|}|\\
\end{tabular}
\end{center}
%
The argument \textit{dest} is the destination file
(without extension).
It should be the main file or one of the child files.
Note that further \textsf{childdoc} directives
such as |\childdocof| and |\childdocforward|
in the indicated file will be processed in this form.
The optional argument \textit{main}
passes on directly to the main file \textit{main}
while pretending to compile the child \textit{dest}.
This form behaves as if \textit{dest}
issues |\childdocof{|\textit{main}|}| right away,
and no further \textsf{childdoc} directives will be processed.

%%%%%%%%%%%%%%%%%%%%%%%%%%%%%%%%%%%%%%%%
\DescribeMacro{\...prefix}
In the alternative form |\childdocforwardprefix|,
%
\begin{center}
\begin{tabular}{l}
|\input{childdoc.def}|\\
|\childdocforwardprefix[|\textit{main}|]{|\textit{prefix}|}{|\textit{dest}|}|
\end{tabular}
\end{center}
%
the destination file is determined by a pattern
depending on the current file:
To make this work, the current file must be called
`{\textit{prefix}\hspace{0.2em}\textit{suffix}}'
with \textit{prefix} matching precisely the argument.
Processing is then passed on to the file
`{\textit{dest}\hspace{0.2em}\textit{suffix}}'.
Surely, the same effect is achieved by
directly specifying the
argument `{\textit{dest}\hspace{0.2em}\textit{suffix}}'
in the first form.
However, that requires to set up a different file
for each child. With the alternative form of the command
all these files can have exactly the same content
which simplifies setting them up and maintaining them.

For example, the following file |draft.tex|
with a compilation flag |\version| as described in \secref{sec:flags}
compiles the main document as a draft:
%
\begin{center}
\begin{tabular}{l}
|\def\version{draft}|\\
|\input{childdoc.def}|\\
|\childdocforward{|\textit{main}|}|
\end{tabular}
\end{center}
%
Likewise, the following files |final|\textit{nn}|.tex|
compile the final version of the child document
|child|\textit{nn}|.tex|:
%
\begin{center}
\begin{tabular}{l}
|\def\version{final}|\\
|\input{childdoc.def}|\\
|\childdocforwardprefix{final}{child}|
\end{tabular}
\end{center}
%

Note that when several versions of a main file and/or of each child file
are to be generated, it may be convenient to set up a |Makefile| or
shell script to automatise the process.

%%%%%%%%%%%%%%%%%%%%%%%%%%%%%%%%%%%%%%%%%%%%%%%%%%%%%%%%%%%%%%%%%%%%%%%%%%%%%%%%
\subsection{Command Line Processing}
\label{sec:commandline}

The effect of redirection files can also be achieved by invoking
the \LaTeX{} compiler with a more elaborate command line.
Most conveniently this should be done as part
of a shell script or a |Makefile|.

When using \textsf{childdoc} in the main file, the following
command lines effectively perform a redirection
(note that depending on the shell being used,
backslashes may have to be doubled: `|\|' $\to$ `|\\|'):
%
\begin{center}
|... -jobname "|\textit{target}|" |\\|"|[\textit{flags}]%
|\input{childdoc.def}\childdocforward[|\textit{main}|]{|\textit{dest}|}"|
\end{center}
%
Here \textit{target} is the name of the output file,
\textit{main} is the name of the main file
and \textit{dest} is the name of the main or child file to be processed
(all filenames without extensions).
The optional argument \textit{main} can be omitted
if \textit{main} matches \textit{dest}.
Optionally, compilation \textit{flags} can be defined via |\def| commands.
This command line makes the \TeX{} engine believe
it is compiling the file \textit{target}
whose content is specified as the latter parameter.
The provided code then forwards the processing to
\textit{main} or \textit{dest} as described in \secref{sec:forward}.

%%%%%%%%%%%%%%%%%%%%%%%%%%%%%%%%%%%%%%%%%%%%%%%%%%%%%%%%%%%%%%%%%%%%%%%%%%%%%%%%
\subsection{Include by Input}
\label{sec:input}

Including child documents by |\include| has some restrictions by design.
Most notably, the content of a child document always occupies
its own set of pages; pages cannot be shared between child documents.
Usually, this behaviour makes perfect sense
because each child document contain an essential part of the document.
However, in some situations it may be desirable to compose
a document from a collection of parts
without having mandatory page breaks between then.
For this case, the package
provides a mechanism to include parts
by |\input| which can also be processed individually.
However, by construction this mechanism
requires manual handling of the content to be output.

%%%%%%%%%%%%%%%%%%%%%%%%%%%%%%%%%%%%%%%%
\DescribeMacro{\ifchilddocmanual}
The main file should be prepared as usual, see \secref{sec:include}.
However, the document body must make a distinction
between processing of an individual part and of the main document, e.g.:
%
\begin{center}
\begin{tabular}{l}
|\ifchilddocmanual|\\
|\input{\childdocname}|\\
|\||else|\\
\textit{document body with }|\input{|\textit{part}|}|\\
|\||fi|
\end{tabular}
\end{center}
%
The conditional |\ifchilddocmanual| is true whenever
a part to be included by |\input| is being compiled,
and the name of the part is stored in |\childdocname|.

%%%%%%%%%%%%%%%%%%%%%%%%%%%%%%%%%%%%%%%%
\DescribeMacro{\childdocby}
Each part to be included by |\input| should start with:
%
\begin{center}
\begin{tabular}{l}
|\input{childdoc.def}|\\
|\childdocby{|\textit{main}|}|\\
\end{tabular}
\end{center}
%
The directive |\childdocby| is similar to |\childdocof|
described in \secref{sec:include},
but the subsequent selection of content must be done manually.
To that end, both |\ifchilddoc| and |\ifchilddocmanual|
will be true upon processing of a part,
and the name of the part is stored in |\childdocname|.
Note that |\jobname| will be set to the filename of the current part
so that each part receives an individual |.aux| file
that does not interfere with the |.aux| file(s) of the main document.
This behaviour can be altered by the alternative form
|\childdocby[*]{|\textit{main}|}| (with a non-empty optional argument)
which uses the |.aux| file of the main document
by setting |\jobname| to \textit{main}.

%%%%%%%%%%%%%%%%%%%%%%%%%%%%%%%%%%%%%%%%%%%%%%%%%%%%%%%%%%%%%%%%%%%%%%%%%%%%%%%%
\subsection{Driver Development}
\label{sec:driver}

The \textsf{childdoc} mechanism can also be use for the development
of definition files such as \LaTeX{} styles or classes.
This case differs from the above setup with multiple parts
included by |\include| in that no |\includeonly| should be invoked.
This can be achieved by starting the include file
(before |\ProvidesPackage|) with:
%
\begin{center}
\begin{tabular}{l}
|\input{childdoc.def}|\\
|\childdocforward{|\textit{main}|}|\\
\end{tabular}
\end{center}
%
or alternatively with:
%
\begin{center}
\begin{tabular}{l}
|\input{childdoc.def}|\\
|\childdocby{|\textit{main}|}|\\
\end{tabular}
\end{center}
%
Both forms have slightly different effects as described above.
The main file is prepared as usual, see \secref{sec:include}.

%%%%%%%%%%%%%%%%%%%%%%%%%%%%%%%%%%%%%%%%%%%%%%%%%%%%%%%%%%%%%%%%%%%%%%%%%%%%%%%%
\subsection{Legacy Detection}
\label{sec:detection}

The directive |\childdocmain| in the main file can detect
whether the complete document or merely a child is to be compiled
even without using the directive |\childdocof|.
This method is deprecated because it is less robust
and there is no compelling reason to use it;
it is merely provided for backward compatibility
and it may be removed in future versions.

If the detection mechanism is to be used,
it is mandatory to correctly specify
the filename of the main file as the argument of |\childdocmain|:
%
\begin{center}
\begin{tabular}{l}
|\input{childdoc.def}|\\
|\childdocmain{|\textit{main}|}|\\
\end{tabular}
\end{center}
%
If |\jobname| does not match the argument \textit{main} of |\childdocmain|,
it is assumed that |\jobname| points to the child file to be compiled.
When using |\childdocmain| with the main file specified as argument,
it suffices to start a child file
with just |\input{|\textit{main}|}|
without loading of the package and using |\childdocof|.
If instead all processing is done
with the appropriate \textsf{childdoc} directives,
the argument of \textit{main} of |\childdocmain| can be empty.

An alternative version of the command line processing described
in \secref{sec:commandline} using the detection mechanism reads:
%
\begin{center}
|... -jobname "|\textit{target}|" "|[\textit{flags}]%
[|\def\jobname{|\textit{dest}|}|]|\input{|\textit{main}|}"|
\end{center}

%%%%%%%%%%%%%%%%%%%%%%%%%%%%%%%%%%%%%%%%%%%%%%%%%%%%%%%%%%%%%%%%%%%%%%%%%%%%%%%%
\subsection{Manual Code}
\label{sec:manual}

In case one cannot be certain whether the definitions file |childdoc.def|
is installed on the target \TeX{} distribution
and one prefers not to ship it,
it is conceivable to paste a few relevant commands into the sources.

To that end, drop all statements |\input{childdoc.def}|
and perform the replacements as outlined below.
Instead of |\childdocmain{|\textit{main}|}| add the following code
to the top of the main file:
%
\begin{center}
\begin{tabular}{l}
|\||ifdefined\childdocname\endinput\||fi\newif\ifchilddoc|\\
|\edef\childdocname{\scantokens\expandafter{\jobname\noexpand}}|\\
|\def\childdocmain{|\textit{main}|}\||ifx\childdocmain\childdocname\||else|\\
|\childdoctrue\includeonly{\childdocname}\let\jobname\childdocmain\||fi|\\
\end{tabular}
\end{center}
%
Instead of |\childdocof{|\textit{main}|}| just include the main file
at the top of each child file:
%
\begin{center}
|\input{|\textit{main}|}|
\end{center}
%
A simple redirection |\childdocforward{|\textit{dest}|}| is achieved by:
%
\begin{center}
|\def\jobname{|\textit{dest}|}\input{\jobname}|
\end{center}
%
The redirection with prefix
|\childdocforwardprefix[|\textit{prefix}|]{|\textit{dest}|}|
is accomplished by:
%
\begin{center}
\begin{tabular}{l}
|{\edef\jobname{\scantokens\expandafter{\jobname\noexpand}}|\\
|\def\redirectjob |\textit{prefix}|#1~~~{\gdef\jobname{|\textit{dest}|#1}}|\\
|\expandafter\redirectjob\jobname~~~}\input{\jobname}|
\end{tabular}
\end{center}

In an alternative approach,
child documents can be compiled by a specific command line
without additional code or specific definitions:
%
\begin{center}
|... -jobname "|\textit{target}|" "|[\textit{flags}]%
|\includeonly{|\textit{dest}|}\input{|\textit{main}|}"|
\end{center}
%

%%%%%%%%%%%%%%%%%%%%%%%%%%%%%%%%%%%%%%%%%%%%%%%%%%%%%%%%%%%%%%%%%%%%%%%%%%%%%%%%
%%%%%%%%%%%%%%%%%%%%%%%%%%%%%%%%%%%%%%%%%%%%%%%%%%%%%%%%%%%%%%%%%%%%%%%%%%%%%%%%
\section{Information}

%%%%%%%%%%%%%%%%%%%%%%%%%%%%%%%%%%%%%%%%%%%%%%%%%%%%%%%%%%%%%%%%%%%%%%%%%%%%%%%%
\subsection{Copyright}

Copyright \copyright{} 2017--2018 Niklas Beisert

This work may be distributed and/or modified under the
conditions of the \LaTeX{} Project Public License, either version 1.3
of this license or (at your option) any later version.
The latest version of this license is in
  \url{http://www.latex-project.org/lppl.txt}
and version 1.3 or later is part of all distributions of \LaTeX{}
version 2005/12/01 or later.

This work has the LPPL maintenance status `maintained'.

The Current Maintainer of this work is Niklas Beisert.

This work consists of the files |README.txt|, |childdoc.ins| and |childdoc.dtx|
as well as the derived files |childdoc.def|, |cdocsamp.tex|
with |cdocsch1.tex|, |cdocsch2.tex|, |cdocspt3.tex|, |cdocspt4.tex|,
|cdocsdrf.tex|, |cdocsfn1.tex|, |cdocsfn2.tex|
as well as |childdoc.pdf|.

%%%%%%%%%%%%%%%%%%%%%%%%%%%%%%%%%%%%%%%%%%%%%%%%%%%%%%%%%%%%%%%%%%%%%%%%%%%%%%%%
\subsection{Files and Installation}

The package consists of the files:
%
\begin{center}
\begin{tabular}{ll}
    |README.txt|   & readme file \\
    |childdoc.ins| & installation file \\
    |childdoc.dtx| & source file \\
    |childdoc.def| & definition file \\
    |cdocsamp.tex| & sample main file \\
    |cdocsch1.tex| & sample include file \\
    |cdocsch2.tex| & sample include file \\
    |cdocspt3.tex| & sample part file \\
    |cdocspt4.tex| & sample part file \\
    |cdocsdrf.tex| & sample redirection file \\
    |cdocsfn1.tex| & sample redirection file \\
    |cdocsfn2.tex| & sample redirection file \\
    |childdoc.pdf| & manual
\end{tabular}
\end{center}
%
The distribution consists of the files
|README.txt|, |childdoc.ins| and |childdoc.dtx|.
%
\begin{itemize}
\item
Run (pdf)\LaTeX{} on |childdoc.dtx|
to compile the manual |childdoc.pdf| (this file).
\item
Run \LaTeX{} on |childdoc.ins| to create the definitions file |childdoc.def|
and the sample |cdocsamp.tex| with include files
|cdocsch1.tex|, |cdocsch2.tex|, |cdocspt3.tex|, |cdocspt4.tex|,
|cdocsdrf.tex|, |cdocsfn1.tex|, |cdocsfn2.tex|.
Then copy the file |childdoc.def| to an appropriate directory of your \LaTeX{}
distribution, e.g.\ \textit{texmf-root}|/tex/latex/childdoc|.
\end{itemize}

%%%%%%%%%%%%%%%%%%%%%%%%%%%%%%%%%%%%%%%%%%%%%%%%%%%%%%%%%%%%%%%%%%%%%%%%%%%%%%%%
\subsection{Related CTAN Packages}

There are several other packages which offer a similar functionality:
%
\begin{itemize}
\item
The packages
\href{http://ctan.org/pkg/docmute}{\textsf{docmute}},
\href{http://ctan.org/pkg/includex}{\textsf{includex}} and
\href{http://ctan.org/pkg/standalone}{\textsf{standalone}}
provide commands to include only the document body of
a child file thus allowing both files to be compiled individually.
\item
The packages \href{http://ctan.org/pkg/subdocs}{\textsf{subdocs}}
and \href{http://ctan.org/pkg/subfiles}{\textsf{subfiles}}
provide structures in which the main and child documents can be
encapsulated and allowing them to be compiled individually.
The inclusion mechanism is different from the conventional |\include|.
\item
The package \href{http://ctan.org/pkg/combine}{\textsf{combine}}
is an elaborate solution to combine several documents into one.
\end{itemize}
%
See also the CTAN topic \href{http://ctan.org/topic/subdocs}{\textsf{subdocs}}
for further related packages.
The present package differs from the above solutions in that
a document structure constructed with the conventional |\include| mechanism
just needs two extra commands at the top of every file
such that all constituent files can be compiled individually.

%%%%%%%%%%%%%%%%%%%%%%%%%%%%%%%%%%%%%%%%%%%%%%%%%%%%%%%%%%%%%%%%%%%%%%%%%%%%%%%%
%\subsection{Feature Suggestions}
%
%The following is a list of features which may be useful for future
%versions of this package:
%%
%\begin{itemize}
%\item
%\ldots
%\end{itemize}

%%%%%%%%%%%%%%%%%%%%%%%%%%%%%%%%%%%%%%%%%%%%%%%%%%%%%%%%%%%%%%%%%%%%%%%%%%%%%%%%
\subsection{Revision History}

%%%%%%%%%%%%%%%%%%%%%%%%%%%%%%%%%%%%%%%%
\paragraph{v2.0:} 2018/12/30

\begin{itemize}
\item
immediate forward processing
\item
added |\childdocby| mechanism
\item
manual restructured
\end{itemize}

%%%%%%%%%%%%%%%%%%%%%%%%%%%%%%%%%%%%%%%%
\paragraph{v1.6:} 2018/01/17

\begin{itemize}
\item
application for development of include files
\item
corrections to manual
\end{itemize}

%%%%%%%%%%%%%%%%%%%%%%%%%%%%%%%%%%%%%%%%
\paragraph{v1.5:} 2017/05/21

\begin{itemize}
\item
more complete structuring introduced
\item
|\childdocof| introduced
\item
|\childdoc| renamed to |\childdocmain|
\item
|\childredirect| renamed to |\childdocforward| and |\childdocforwardprefix|
and functionality expanded
\end{itemize}

%%%%%%%%%%%%%%%%%%%%%%%%%%%%%%%%%%%%%%%%
\paragraph{v1.0:} 2017/04/27

\begin{itemize}
\item
manual and install package
\item
first version published on CTAN
\end{itemize}

%%%%%%%%%%%%%%%%%%%%%%%%%%%%%%%%%%%%%%%%
\paragraph{v0.6:} 2017/04/26

\begin{itemize}
\item
redirection mechanism added
\end{itemize}

%%%%%%%%%%%%%%%%%%%%%%%%%%%%%%%%%%%%%%%%
\paragraph{v0.5:} 2017/04/26

\begin{itemize}
\item
functionality in definition file
\end{itemize}


%%%%%%%%%%%%%%%%%%%%%%%%%%%%%%%%%%%%%%%%%%%%%%%%%%%%%%%%%%%%%%%%%%%%%%%%%%%%%%%%
%%%%%%%%%%%%%%%%%%%%%%%%%%%%%%%%%%%%%%%%%%%%%%%%%%%%%%%%%%%%%%%%%%%%%%%%%%%%%%%%
%%%%%%%%%%%%%%%%%%%%%%%%%%%%%%%%%%%%%%%%%%%%%%%%%%%%%%%%%%%%%%%%%%%%%%%%%%%%%%%%
\appendix

\settowidth\MacroIndent{\rmfamily\scriptsize 000\ }

 \DocInput{childdoc.dtx}

\end{document}
%</driver>
% \fi
%
% %%%%%%%%%%%%%%%%%%%%%%%%%%%%%%%%%%%%%%%%%%%%%%%%%%%%%%%%%%%%%%%%%%%%%%%%%%%%%%
% %%%%%%%%%%%%%%%%%%%%%%%%%%%%%%%%%%%%%%%%%%%%%%%%%%%%%%%%%%%%%%%%%%%%%%%%%%%%%%
% \section{Sample}
%\iffalse
%<*samplemain>
%\fi
%
% The following presents a sample document
% with two chapters, two parts, a title page,
% a compile flag as well as three forwarding files to set the flag.
% It consists of eight |.tex| files:
% \begin{center}
% \begin{tabular}{ll}
% |cdocsamp.tex|&main file\\
% |cdocsch1.tex|&include file for chapter 1\\
% |cdocsch2.tex|&include file for chapter 2\\
% |cdocspt3.tex|&include file for part 3\\
% |cdocspt4.tex|&include file for part 4\\
% |cdocsdrf.tex|&forwarding file for main file in draft mode\\
% |cdocsfi1.tex|&forwarding file for final version of chapter 1\\
% |cdocsfi2.tex|&forwarding file for final version of chapter 2\\
% \end{tabular}
% \end{center}
% Each of the eight files can be compiled directly by the \LaTeX{} compiler.
%
% %%%%%%%%%%%%%%%%%%%%%%%%%%%%%%%%%%%%%%
% \paragraph{Main File.}
%
% The main file is called |cdocsamp.tex|.
%
% Load the \textsf{childdoc} definitions and
% declare the filename for the main document:
%    \begin{macrocode}
\input{childdoc.def}
\childdocmain{}
%    \end{macrocode}

% Optional override for |\version| flag:
%    \begin{macrocode}
%%\ifchilddoc\else\providecommand{\version}{draft}\fi
%    \end{macrocode}

% Define the default values for the |\version| flag
% (|final| for the main file and |draft| for childs):
%    \begin{macrocode}
\ifchilddoc
\providecommand{\version}{draft}
\else
\providecommand{\version}{final}
\fi
%    \end{macrocode}

% Load the standard document class:
%    \begin{macrocode}
\documentclass[12pt]{article}
%    \end{macrocode}

% Start the document body:
%    \begin{macrocode}
\begin{document}
%    \end{macrocode}

% Declare a title page.
% Print title, part of document being processed and version flag:
%    \begin{macrocode}
\addtocounter{page}{-1}
\begin{center}
{\LARGE\bfseries{}childdoc example\par}
\vspace{1cm}
\ifchilddoc
\ifchilddocmanual part\else chapter\fi:
`\childdocname' of `\childdocjob'\par
\else
main document: `\childdocjob'\par
\fi
version: \version\par
\end{center}
\newpage
%    \end{macrocode}

% Manually include selected file,
% otherwise process as usual:
%    \begin{macrocode}
\ifchilddocmanual
\section*{part `\childdocname'}
\input{\childdocname}
\else
%    \end{macrocode}

% Include the two chapters:
%    \begin{macrocode}
\include{cdocsch1}
\include{cdocsch2}
%    \end{macrocode}

% Include the two parts unless only chapters should be displayed:
%    \begin{macrocode}
\ifchilddoc\else
\section{part three}
\input{cdocspt3}
\section{part four}
\input{cdocspt4}
\fi
%    \end{macrocode}

% Process as usual until here:
%    \begin{macrocode}
\fi
%    \end{macrocode}

% End of document body:
%    \begin{macrocode}
\end{document}
%    \end{macrocode}
%\iffalse
%</samplemain>
%\fi
%
% %%%%%%%%%%%%%%%%%%%%%%%%%%%%%%%%%%%%%%
% \paragraph{Chapter Include Files.}
%
% The include files are called |cdocsch1.tex| and |cdocsch2.tex|.
%
%\iffalse
%<*samplechap1|samplechap2>
%\fi

% Optional override for |\version| flag:
%    \begin{macrocode}
%%\providecommand{\version}{final}
%    \end{macrocode}

% Include the main document:
%    \begin{macrocode}
\input{childdoc.def}
\childdocof{cdocsamp}
%    \end{macrocode}

%\iffalse
%</samplechap1|samplechap2>
%\fi
%
%\iffalse
%<*samplechap1>
%\fi
% Some text for chapter 1:
%    \begin{macrocode}
\section{one}
some text in chapter one
%    \end{macrocode}

%\iffalse
%</samplechap1>
%\fi
% Some text for chapter 2:
%\iffalse
%<*samplechap2>
%\fi
%    \begin{macrocode}
\section{two}
more text in chapter two
%    \end{macrocode}

%\iffalse
%</samplechap2>
%\fi
%
% %%%%%%%%%%%%%%%%%%%%%%%%%%%%%%%%%%%%%%
% \paragraph{Part Include Files.}
%
% The include files are called |cdocspt3.tex| and |cdocspt4.tex|.
%
%\iffalse
%<*samplepart3|samplepart4>
%\fi

% Optional override for |\version| flag:
%    \begin{macrocode}
%%\providecommand{\version}{final}
%    \end{macrocode}

% Include the main document:
%    \begin{macrocode}
\input{childdoc.def}
\childdocby{cdocsamp}
%    \end{macrocode}

%\iffalse
%</samplepart3|samplepart4>
%\fi
%
%\iffalse
%<*samplepart3>
%\fi
% Some text for part 3:
%    \begin{macrocode}
some text in part three
%    \end{macrocode}

%\iffalse
%</samplepart3>
%\fi
% Some text for part 4:
%\iffalse
%<*samplepart4>
%\fi
%    \begin{macrocode}
more text in part four
%    \end{macrocode}

%\iffalse
%</samplepart4>
%\fi
%
% %%%%%%%%%%%%%%%%%%%%%%%%%%%%%%%%%%%%%%
% \paragraph{Forwarding for a Complete Draft.}
%
% The following forwarding file |cdocsdrf.tex|
% compiles the main document in draft mode:
%\iffalse
%<*sampledraft>
%\fi
%    \begin{macrocode}
\def\version{draft}
\input{childdoc.def}
\childdocforward{cdocsamp}
%    \end{macrocode}

%\iffalse
%</sampledraft>
%\fi
%
% %%%%%%%%%%%%%%%%%%%%%%%%%%%%%%%%%%%%%%
% \paragraph{Forwarding for Final Version of the Chapters.}
%
% The following forwarding files |cdocsfn1.tex| and |cdocsfn2.tex|
% (with identical content)
% compile the final versions of the child documents
% |cdocsch1.tex| and |cdocsch2.tex|, respectively:
%\iffalse
%<*samplefinal>
%\fi
%    \begin{macrocode}
\def\version{final}
\input{childdoc.def}
\childdocforwardprefix[cdocsamp]{cdocsfn}{cdocsch}
%    \end{macrocode}

%\iffalse
%</samplefinal>
%\fi
%
% %%%%%%%%%%%%%%%%%%%%%%%%%%%%%%%%%%%%%%
% \paragraph{Command Line Processing.}
%
% The following three command lines generate the output files
% |cdocscld|, |cdocscl1| and |cdocscl2|
% which should be identical to
% |cdocsdrf|, |cdocsch1| and |cdocsfn2|, respectively:
% \begin{center}
% \begin{tabular}{l}
% |latex -jobname cdocscld \|\\
% |  "\def\version{draft}\input{childdoc.def}\childdocforward{cdocsamp}"|\\
% |latex -jobname cdocscl1 \|\\
% |  "\input{childdoc.def}\childdocforward[cdocsamp]{cdocsch1}"|\\
% |latex -jobname cdocscl2 \|\\
% |  "\def\version{final}\input{childdoc.def}\childdocforward{cdocsch2}"|
% \end{tabular}
% \end{center}
% Note that the trailing backslash on each first line
% merely continues the input to the second line
% (for convenient cut ant paste).
% Furthermore, the command |latex| can be replaced by any
% of its alternative versions such as |pdflatex|.
%
% %%%%%%%%%%%%%%%%%%%%%%%%%%%%%%%%%%%%%%%%%%%%%%%%%%%%%%%%%%%%%%%%%%%%%%%%%%%%%%
% %%%%%%%%%%%%%%%%%%%%%%%%%%%%%%%%%%%%%%%%%%%%%%%%%%%%%%%%%%%%%%%%%%%%%%%%%%%%%%
% \section{Implementation}
%\iffalse
%<*package>
%\fi
%
% This section describes the definitions file |childdoc.def|.

% The definitions cannot be loaded using |\usepackage| or |\RequirePackage|
% which has a mechanism to prevent loading a style file more than once.
% When loading the definitions by means of |\input|
% multiple instances have to be prevented manually:
%\iffalse
%This code needs to be before the `\ProvidesFile' directive
%which is defined at the beginning of this file.
%Therefore it is also placed there and commented out here.
%</package>
%<*discard>
%\fi
%    \begin{macrocode}
\ifdefined\childdocmain\endinput\fi
%    \end{macrocode}
%\iffalse
%</discard>
%<*package>
%\fi
%
% \macro{\ifchilddoc}
% \macro{\ifchilddocmanual}
% The conditional |\ifchilddoc| tells whether a
% child (true) or main (false) document is being compiled.
% The conditional |\ifchilddocmanual| tells whether
% the |\includeonly| mechanism is used (false) or
% the selection of child files must be performed manually (true).
% The definitions initialise to false:
%    \begin{macrocode}
\newif\ifchilddoc
\newif\ifchilddocmanual
%    \end{macrocode}

% \macro{\childdocname}
% \macro{\childdocjob}
% The macro |\childdocname| stores the name of the main document
% to be compiled. The macro |\childdocjob| stores the name of
% the document on which the \LaTeX{} compiler was originally invoked.
% The content of |\jobname| cannot be compared
% to filenames specified in the source due to different catcodes.
% The following code rescans |\jobname|, stores the result
% in |\childdocname| and saves a copy in |\childdocjob|:
%    \begin{macrocode}
\edef\childdocname{\scantokens\expandafter{\jobname\noexpand}}
\let\childdocjob\childdocname
%    \end{macrocode}

% \macro{\childdocdisable}
% The macro |\childdocdisable| prevents the main file
% from being processed more than once.
% At this stage, the main document command |\childdocmain|
% is assumed to be called once again where it should do nothing.
% Any subsequent call to it should prevent
% a secondary processing of the main document
% It overwrites the forwarding commands
% |\childdocof| and |\childdocforward|
% with empty macros to prevent further inclusions of the main document:
%    \begin{macrocode}
\newcommand{\childdocdisable}
{
  \renewcommand{\childdocmain}[1]{\renewcommand{\childdocmain}[1]{\endinput}}
  \renewcommand{\childdocof}[1]{}
  \renewcommand{\childdocby}[2][]{}
  \renewcommand{\childdocforward}[2][]{}
  \renewcommand{\childdocdisable}{}
}
%    \end{macrocode}

% \macro{\childdocmain}
% The macro |\childdocmain| is to be called at the top of the main file
% with nothing or the main filename (without extension) as argument.
% First, it breaks loops.
% If the argument is not empty and does not match |\childdocname|
% (which is set by the first inclusion of |childdoc.def|),
% |\ifchilddoc| is set to true, |\includeonly| is applied to the child file
% and |\jobname| is set to the main file
% (for proper handling of |.aux| files):
%    \begin{macrocode}
\newcommand{\childdocmain}[1]
{
  \childdocdisable\childdocmain{}
  \if?#1?\else
    \begingroup
      \def\childdoctmp{#1}
      \ifx\childdoctmp\childdocname
        \def\childdoctmp{}
      \else
        \def\childdoctmp
        {
          \childdoctrue
          \includeonly{\childdocname}
          \def\childdocjob{#1}
          \def\jobname{#1}
        }
      \fi
      \expandafter
    \endgroup
    \childdoctmp
  \fi
}
%    \end{macrocode}

% \macro{\childdocof}
% The command |\childdocof| redirects
% compilation to the main file |#1|.
%    \begin{macrocode}
\newcommand{\childdocof}[1]
{
  \childdocdisable
  \childdoctrue
  \includeonly{\childdocname}
  \def\jobname{#1}
  \def\childdocjob{#1}
  \input{#1}
}
%    \end{macrocode}

% \macro{\childdocby}
% The command |\childdocby| ....
%    \begin{macrocode}
\newcommand{\childdocby}[2][]
{
  \childdocdisable
  \childdoctrue
  \childdocmanualtrue
  \if?#1?\else
    \def\jobname{#2}
  \fi
  \def\childdocjob{#2}
  \input{#2}
  \endinput
}
%    \end{macrocode}

% \macro{\childdocforward}
% The command |\childdocforward| redirects
% compilation to the main file or
% (if the optional argument is given) a child file.
% Parameters are set as if the main file
% or a child file starting with |\childdocof| was compiled.
% Then compilation is handed over to the main file:
%    \begin{macrocode}
\newcommand{\childdocforward}[2][]
{
  \begingroup
    \if?#1?
      \def\childdoctmp
      {
        \def\childdocname{#2}
        \def\childdocjob{#2}
        \def\jobname{#2}
        \input{#2}
        \endinput
      }
    \else
      \def\childdoctmp
      {
        \childdocdisable
        \def\childdocname{#2}
        \childdoctrue
        \includeonly{#2}
        \def\childdocjob{#1}
        \def\jobname{#1}
        \input{#1}
        \endinput
      }
    \fi
    \expandafter
  \endgroup
  \childdoctmp
}
%    \end{macrocode}

% \macro{\childdocforwardprefix}
% The command |\childdocforwardprefix| redirects
% compilation to the main or a child file by means of a pattern.
% The prefix |#1| in the current filename is replaced by |#2|
% and the suffix of the current filename is kept
% (it is assumed that the filename does not contain the substring `|~~~|'
% which is used as a delimiter).
% Compilation is handed over to the new file by |\childdocforward|:
%    \begin{macrocode}
\newcommand{\childdocforwardprefix}[3][]
{
  \begingroup
    \def\childdocextract #2##1~~~{\def\childdoctmp{\childdocforward[#1]{#3##1}}}
    \expandafter\childdocextract\childdocname~~~
    \expandafter
  \endgroup
  \childdoctmp
}
%    \end{macrocode}

% \macro{\childdoc}
% The deprecated macro |\childdoc| is a legacy version of |\childdocmain|:
%    \begin{macrocode}
\newcommand{\childdoc}{\childdocmain}
%    \end{macrocode}

% \macro{\childdocredirect}
% The deprecated macro |\childdocredirect| is a legacy version
% of |\childdocforward| and |\childdocforwardprefix|:
%    \begin{macrocode}
\newcommand{\childdocredirect}[2][]
{
  \begingroup
    \if?#1?
      \def\childdoctmp{\childdocforward{#2}}
    \else
      \def\childdoctmp{\childdocforwardprefix{#1}{#2}}
    \fi
    \expandafter
  \endgroup
  \childdoctmp
}
%    \end{macrocode}

%\iffalse
%</package>
%\fi
%
\endinput

\childdocmain{}
%    \end{macrocode}

% Optional override for |\version| flag:
%    \begin{macrocode}
%%\ifchilddoc\else\providecommand{\version}{draft}\fi
%    \end{macrocode}

% Define the default values for the |\version| flag
% (|final| for the main file and |draft| for childs):
%    \begin{macrocode}
\ifchilddoc
\providecommand{\version}{draft}
\else
\providecommand{\version}{final}
\fi
%    \end{macrocode}

% Load the standard document class:
%    \begin{macrocode}
\documentclass[12pt]{article}
%    \end{macrocode}

% Start the document body:
%    \begin{macrocode}
\begin{document}
%    \end{macrocode}

% Declare a title page.
% Print title, part of document being processed and version flag:
%    \begin{macrocode}
\addtocounter{page}{-1}
\begin{center}
{\LARGE\bfseries{}childdoc example\par}
\vspace{1cm}
\ifchilddoc
\ifchilddocmanual part\else chapter\fi:
`\childdocname' of `\childdocjob'\par
\else
main document: `\childdocjob'\par
\fi
version: \version\par
\end{center}
\newpage
%    \end{macrocode}

% Manually include selected file,
% otherwise process as usual:
%    \begin{macrocode}
\ifchilddocmanual
\section*{part `\childdocname'}
\input{\childdocname}
\else
%    \end{macrocode}

% Include the two chapters:
%    \begin{macrocode}
\include{cdocsch1}
\include{cdocsch2}
%    \end{macrocode}

% Include the two parts unless only chapters should be displayed:
%    \begin{macrocode}
\ifchilddoc\else
\section{part three}
\input{cdocspt3}
\section{part four}
\input{cdocspt4}
\fi
%    \end{macrocode}

% Process as usual until here:
%    \begin{macrocode}
\fi
%    \end{macrocode}

% End of document body:
%    \begin{macrocode}
\end{document}
%    \end{macrocode}
%\iffalse
%</samplemain>
%\fi
%
% %%%%%%%%%%%%%%%%%%%%%%%%%%%%%%%%%%%%%%
% \paragraph{Chapter Include Files.}
%
% The include files are called |cdocsch1.tex| and |cdocsch2.tex|.
%
%\iffalse
%<*samplechap1|samplechap2>
%\fi

% Optional override for |\version| flag:
%    \begin{macrocode}
%%\providecommand{\version}{final}
%    \end{macrocode}

% Include the main document:
%    \begin{macrocode}
% \iffalse
%
% childdoc.dtx Copyright (C) 2017-2018 Niklas Beisert
%
% This work may be distributed and/or modified under the
% conditions of the LaTeX Project Public License, either version 1.3
% of this license or (at your option) any later version.
% The latest version of this license is in
%   http://www.latex-project.org/lppl.txt
% and version 1.3 or later is part of all distributions of LaTeX
% version 2005/12/01 or later.
%
% This work has the LPPL maintenance status `maintained'.
%
% The Current Maintainer of this work is Niklas Beisert.
%
% This work consists of the files childdoc.dtx and childdoc.ins
% and the derived files childdoc.def and cdocsamp.tex with
% cdocsch1.tex, cdocsch2.tex, cdocsdrf.tex, cdocsfn1.tex, cdocsfn2.tex.
%
%<package>\ifdefined\childdocmain\endinput\fi
%<package>\ProvidesFile{childdoc.def}[2018/12/30 v2.0 child document driver]
%<samplemain>\ProvidesFile{cdocsamp.tex}[2018/12/30 v2.0 sample for childdoc]
%<*driver>
%\ProvidesFile{childdoc.drv}[2018/12/30 v2.0 childdoc reference manual file]
\PassOptionsToClass{10pt,a4paper}{article}
\documentclass{ltxdoc}

\usepackage[margin=35mm]{geometry}
\usepackage{hyperref}
\usepackage{hyperxmp}
\usepackage[usenames]{color}

\hypersetup{colorlinks=true}
\hypersetup{pdfstartview=FitH}
\hypersetup{pdfpagemode=UseNone}
\hypersetup{pdfsource={}}
\hypersetup{pdflang={en-UK}}
\hypersetup{pdfcopyright={Copyright 2017-2018 Niklas Beisert.
  This work may be distributed and/or modified under the
  conditions of the LaTeX Project Public License, either version 1.3
  of this license or (at your option) any later version.}}
\hypersetup{pdflicenseurl={http://www.latex-project.org/lppl.txt}}
\hypersetup{pdfcontactaddress={ETH Zurich, ITP, HIT K,
  Wolfgang-Pauli-Strasse 27}}
\hypersetup{pdfcontactpostcode={8093}}
\hypersetup{pdfcontactcity={Zurich}}
\hypersetup{pdfcontactcountry={Switzerland}}
\hypersetup{pdfcontactemail={nbeisert@itp.phys.ethz.ch}}
\hypersetup{pdfcontacturl={http://people.phys.ethz.ch/\xmptilde nbeisert/}}

\newcommand{\secref}[1]{\hyperref[#1]{section \ref*{#1}}}

\parskip1ex
\parindent0pt
\let\olditemize\itemize
\def\itemize{\olditemize\parskip0pt}

\begin{document}

\title{The \textsf{childdoc} Package}
\hypersetup{pdftitle={The childdoc Package}}
\author{Niklas Beisert\\[2ex]
  Institut f\"ur Theoretische Physik\\
  Eidgen\"ossische Technische Hochschule Z\"urich\\
  Wolfgang-Pauli-Strasse 27, 8093 Z\"urich, Switzerland\\[1ex]
  \href{mailto:nbeisert@itp.phys.ethz.ch}
  {\texttt{nbeisert@itp.phys.ethz.ch}}}
\hypersetup{pdfauthor={Niklas Beisert}}
\hypersetup{pdfsubject={Manual for the LaTeX2e Package childdoc}}
\date{30 December 2018, \textsf{v2.0}}
\maketitle

\begin{abstract}\noindent
\textsf{childdoc} is a \LaTeXe{} package
that enables the direct compilation
of document sections included by |\include|
to individual files.
\end{abstract}

\begingroup
\parskip0ex
\tableofcontents
\endgroup

%%%%%%%%%%%%%%%%%%%%%%%%%%%%%%%%%%%%%%%%%%%%%%%%%%%%%%%%%%%%%%%%%%%%%%%%%%%%%%%%
%%%%%%%%%%%%%%%%%%%%%%%%%%%%%%%%%%%%%%%%%%%%%%%%%%%%%%%%%%%%%%%%%%%%%%%%%%%%%%%%
\section{Introduction}

\LaTeX{} provides a mechanism to structure a large document (such as a book)
into a main file and several child files (containing the chapters)
using the |\include| command.
This mechanism is beneficial for documents
which span hundreds of pages in order to
make the source file(s) more manageable.
Moreover, compilation can be restricted to
selected child files by means of the |\includeonly| command.
The latter feature can be used to reduce the compilation time while editing
(this was significantly more useful in the earlier days of \LaTeX{})
or to generate a smaller document which is easier to navigate.
Another application of |\includeonly| is to generate
documents consisting of selected parts of the complete document.

However, there are a few drawbacks of the plain |\include| mechanism:
\begin{itemize}
\item
The child files cannot be compiled on their own,
they can only be compiled via the main file.
A naive editing environment
(such as a text editor with an option
to have the current file processed by \LaTeX)
may require one to switch to the main file before compiling;
attempting to compile the child file produces errors.
\item
The main file must be modified (each time)
to adjust the |\includeonly| command
to the present needs. This easily leaves the main file in a messy state.
\item
The generated document will always carry the filename
of the main document. This is inconvenient if
several child files are to be compiled and
to be kept for distribution.
\end{itemize}

The present package provides a simple interface
to make child files individually compilable by \LaTeX{}.
Compiling a child file then has the same effect as compiling
the main file with an |\includeonly| command
to select the appropriate child.
Moreover the generated document will carry the name of the child
rather than the main file.
This resolves all three above issues.

This feature is meant to make the editing of books,
thesis documents and lecture notes somewhat more convenient.
However, the package can also be used efficiently for
composing a series of documents (such as exercise sheets)
which are typically distributed individually.
It then assists the author in generating the individual documents
(potentially in different versions)
as well as a document containing the collected series.
Another application is in developing style files
or other kinds of included material
where compilation of the style file could redirect
to a sample or test file.

%%%%%%%%%%%%%%%%%%%%%%%%%%%%%%%%%%%%%%%%%%%%%%%%%%%%%%%%%%%%%%%%%%%%%%%%%%%%%%%%
%%%%%%%%%%%%%%%%%%%%%%%%%%%%%%%%%%%%%%%%%%%%%%%%%%%%%%%%%%%%%%%%%%%%%%%%%%%%%%%%
\section{Usage}

First of all, the package \textsf{childdoc} is \emph{not} a standard
\LaTeXe{} |.sty| style file! Therefore it needs to be invoked in
a non-standard way.

%%%%%%%%%%%%%%%%%%%%%%%%%%%%%%%%%%%%%%%%%%%%%%%%%%%%%%%%%%%%%%%%%%%%%%%%%%%%%%%%
\subsection{Included Files}
\label{sec:include}

%%%%%%%%%%%%%%%%%%%%%%%%%%%%%%%%%%%%%%%%
\DescribeMacro{\childdocmain}
To use the package, add the commands
\begin{center}
\begin{tabular}{l}
|\input{childdoc.def}|\\
|\childdocmain{}|\\
\end{tabular}
\end{center}
at the very top of the main \LaTeX{} file,
in particular \emph{before} the |\documentclass| statement!
The argument of |\childdocmain| should be left empty
(but it must be present).

%%%%%%%%%%%%%%%%%%%%%%%%%%%%%%%%%%%%%%%%
\DescribeMacro{\childdocof}
Furthermore, add the commands
\begin{center}
\begin{tabular}{l}
|\input{childdoc.def}|\\
|\childdocof{|\textit{main}|}|\\
\end{tabular}
\end{center}
at the top of every child file \textit{child}
which is included by |\include{|\textit{child}|}|
from within the main file
(or at least for those files to be compiled individually).
The argument \textit{main} must be the filename of the main file.

There are a couple of
considerations in setting up the main and child documents:

%%%%%%%%%%%%%%%%%%%%%%%%%%%%%%%%%%%%%%%%
\paragraph{Restrictions.}

Please note the following restrictions:
\begin{itemize}
\item
|\childdocmain| must be called with one argument \textit{main}
to ensure compatibility with earlier version of the package.
It must either be empty (|\childdocmain{}|)
or precisely match the filename of the main file in which it is specified.
See \secref{sec:detection} for further information.
\item
The filename \textit{main} must be specified without the |.tex| extension.
\item
The filename \textit{main} is case sensitive
(even in case-insensitive file systems)
due to internal string comparison.
\item
The argument \textit{main} should be fully expanded, it cannot be a macro.
\item
Subdirectories and special characters should be avoided in filenames.
\item
The command |\childdocmain{|\textit{main}|}| must be followed by a whitespace.
It should not be followed immediately by another command
or by a comment mark `|%|'.
This is because the \TeX{} parser reads the token immediately following
the argument of |\childdocmain| and puts it
at the beginning of every child section;
however, a white\-space is ignored.
\end{itemize}

%%%%%%%%%%%%%%%%%%%%%%%%%%%%%%%%%%%%%%%%
\paragraph{Content of Main File.}

It is advisable to place all content in the child files included by |\include|.
Any output contained in the main file will appear in all child documents
unless suppressed manually;
it cannot be suppressed automatically by the |\includeonly| directive
and thus should normally be avoided.
A method to include some content in the main file
by means of conditional processing is described in \secref{sec:conditional}.

%%%%%%%%%%%%%%%%%%%%%%%%%%%%%%%%%%%%%%%%
\paragraph{Page Numbering.}

When only a part of the document is compiled,
the appropriate numbering of pages
(as well as other status parameters)
is determined from the |.aux| files.
The latter contain information from previous passes.
However this information needs to propagate through
all intermediate child documents.
Therefore the page numbering in child documents may well
be inconsistent until the complete document is compiled at least once.

A useful (if unconventional) way to always ensure a consistent
page numbering is to restart the numbering in each child document
and denote the pages by `\textit{child}|.|\textit{page}'
where \textit{child} represents the chapter/section number of the child file.
This can be achieved by the command
|\numberwithin{page}{|\textit{child}|}|
of the \textsf{amsmath} package
where \textit{child} can be |chapter| or |section|
depending on the chosen structuring.
Alternatively, one can modify the macro |\thepage| appropriately
and reset the counter |page| at the start of each child file.

%%%%%%%%%%%%%%%%%%%%%%%%%%%%%%%%%%%%%%%%%%%%%%%%%%%%%%%%%%%%%%%%%%%%%%%%%%%%%%%%
\subsection{Conditional Processing}
\label{sec:conditional}

The package provides a mechanism to compile different versions
of a document. To customise the versions further some conditional processing
can come in handy to distinguish which version is being compiled.
The package provides two macros to describe the compilation context:

%%%%%%%%%%%%%%%%%%%%%%%%%%%%%%%%%%%%%%%%
\DescribeMacro{\ifchilddoc}
The conditional |\ifchilddoc| distinguishes between the compilation of
child documents and the main document:
%
\begin{center}
|\ifchilddoc |\textit{child-code}| |[|\||else |\textit{main-code}]| \||fi|
\end{center}

%%%%%%%%%%%%%%%%%%%%%%%%%%%%%%%%%%%%%%%%
\DescribeMacro{\childdocname}
\DescribeMacro{\childdocjob}
The macro |\childdocname| contains the filename (without extension)
of the main or child file being processed.
Note that |\childdocjob| will always contain the name of the main file.

%%%%%%%%%%%%%%%%%%%%%%%%%%%%%%%%%%%%%%%%
\paragraph{Title Page.}

Conditional processing can be used to include a title or banner page
in the main document when proper precautions are taken.
Importantly, the code in the main file should ensure that the page counter
(as well as other status parameters which are stored in the |.aux| files)
takes the same value after the conditional processing.
Otherwise the page numbers may take divergent values
depending on which part is compiled.

For example, a title page could be declared by:
%
\begin{center}
\begin{tabular}{l}
|\ifchilddoc\||else|\\
|\addtocounter{page}{-1}|\\
\textit{code for title page}\\
|\newpage|\\
|\||fi|
\end{tabular}
\end{center}
%
A banner page for the child documents can be generated by:
%
\begin{center}
\begin{tabular}{l}
|\ifchilddoc|\\
|\addtocounter{page}{-1}|\\
\textit{code for banner page}\\
|\newpage|\\
|\||fi|
\end{tabular}
\end{center}
%
Here one could write a message such as:
\begin{center}
|This is the part \childdocname{} of \childdocjob{}.|
\end{center}

%%%%%%%%%%%%%%%%%%%%%%%%%%%%%%%%%%%%%%%%%%%%%%%%%%%%%%%%%%%%%%%%%%%%%%%%%%%%%%%%
\subsection{Flags}
\label{sec:flags}

The package makes it easy to generate different versions
of the main or child documents.
To this end compilation flags can be defined
and assigned different default values.
They will be particularly useful in conjunction
with the forwarding mechanism described in \secref{sec:forward}.

For example, it may be useful to have a flag |\version|
which can be set to |draft| or |final|.
The document source will contain some conditional code
depending on the value of |\version|.
Suppose further, the flag should default to |final| for the main file
and to |draft| for child files
which is a natural assignment for editing the document.
This is achieved by placing the following code
in the preamble of the main document
(below the |\childdocmain| directive):
%
\begin{center}
\begin{tabular}{l}
|\ifchilddoc|\\
|\providecommand{\version}{draft}|\\
|\||else|\\
|\providecommand{\version}{final}|\\
|\||fi|
\end{tabular}
\end{center}
%
The definition by |\providecommand| makes sure
that previous definitions are not overwritten.
Further statements |\providecommand{\version}{...}|
can thus be added before the above code to override it.

For the main file, one might add a line
(between |\childdocmain| and the above block)
%
\begin{center}
|%\ifchilddoc\||else\providecommand{\version}{draft}\||fi|
\end{center}
%
which can be uncommented to produce a draft version.
Likewise one can add a line to the very top of a child file
(above the |\childdocof{|\textit{main}|}| directive)
%
\begin{center}
|%\providecommand{\version}{final}|
\end{center}
%
which can be uncommented to produce the final version of this child document.

%%%%%%%%%%%%%%%%%%%%%%%%%%%%%%%%%%%%%%%%%%%%%%%%%%%%%%%%%%%%%%%%%%%%%%%%%%%%%%%%
\subsection{Forwarding}
\label{sec:forward}

Different versions of the main or child documents
using compilation flags as described in \secref{sec:flags}
can be (permanently) stored in different files
for convenient compilation, viewing and distribution.
To this end, the package defines a command
to pass on compilation to a different file:

%%%%%%%%%%%%%%%%%%%%%%%%%%%%%%%%%%%%%%%%
\DescribeMacro{\childdocforward}
The command |\childdocforward| redirects processing to
another source file:
%
\begin{center}
\begin{tabular}{l}
|\input{childdoc.def}|\\
|\childdocforward[|\textit{main}|]{|\textit{dest}|}|\\
\end{tabular}
\end{center}
%
The argument \textit{dest} is the destination file
(without extension).
It should be the main file or one of the child files.
Note that further \textsf{childdoc} directives
such as |\childdocof| and |\childdocforward|
in the indicated file will be processed in this form.
The optional argument \textit{main}
passes on directly to the main file \textit{main}
while pretending to compile the child \textit{dest}.
This form behaves as if \textit{dest}
issues |\childdocof{|\textit{main}|}| right away,
and no further \textsf{childdoc} directives will be processed.

%%%%%%%%%%%%%%%%%%%%%%%%%%%%%%%%%%%%%%%%
\DescribeMacro{\...prefix}
In the alternative form |\childdocforwardprefix|,
%
\begin{center}
\begin{tabular}{l}
|\input{childdoc.def}|\\
|\childdocforwardprefix[|\textit{main}|]{|\textit{prefix}|}{|\textit{dest}|}|
\end{tabular}
\end{center}
%
the destination file is determined by a pattern
depending on the current file:
To make this work, the current file must be called
`{\textit{prefix}\hspace{0.2em}\textit{suffix}}'
with \textit{prefix} matching precisely the argument.
Processing is then passed on to the file
`{\textit{dest}\hspace{0.2em}\textit{suffix}}'.
Surely, the same effect is achieved by
directly specifying the
argument `{\textit{dest}\hspace{0.2em}\textit{suffix}}'
in the first form.
However, that requires to set up a different file
for each child. With the alternative form of the command
all these files can have exactly the same content
which simplifies setting them up and maintaining them.

For example, the following file |draft.tex|
with a compilation flag |\version| as described in \secref{sec:flags}
compiles the main document as a draft:
%
\begin{center}
\begin{tabular}{l}
|\def\version{draft}|\\
|\input{childdoc.def}|\\
|\childdocforward{|\textit{main}|}|
\end{tabular}
\end{center}
%
Likewise, the following files |final|\textit{nn}|.tex|
compile the final version of the child document
|child|\textit{nn}|.tex|:
%
\begin{center}
\begin{tabular}{l}
|\def\version{final}|\\
|\input{childdoc.def}|\\
|\childdocforwardprefix{final}{child}|
\end{tabular}
\end{center}
%

Note that when several versions of a main file and/or of each child file
are to be generated, it may be convenient to set up a |Makefile| or
shell script to automatise the process.

%%%%%%%%%%%%%%%%%%%%%%%%%%%%%%%%%%%%%%%%%%%%%%%%%%%%%%%%%%%%%%%%%%%%%%%%%%%%%%%%
\subsection{Command Line Processing}
\label{sec:commandline}

The effect of redirection files can also be achieved by invoking
the \LaTeX{} compiler with a more elaborate command line.
Most conveniently this should be done as part
of a shell script or a |Makefile|.

When using \textsf{childdoc} in the main file, the following
command lines effectively perform a redirection
(note that depending on the shell being used,
backslashes may have to be doubled: `|\|' $\to$ `|\\|'):
%
\begin{center}
|... -jobname "|\textit{target}|" |\\|"|[\textit{flags}]%
|\input{childdoc.def}\childdocforward[|\textit{main}|]{|\textit{dest}|}"|
\end{center}
%
Here \textit{target} is the name of the output file,
\textit{main} is the name of the main file
and \textit{dest} is the name of the main or child file to be processed
(all filenames without extensions).
The optional argument \textit{main} can be omitted
if \textit{main} matches \textit{dest}.
Optionally, compilation \textit{flags} can be defined via |\def| commands.
This command line makes the \TeX{} engine believe
it is compiling the file \textit{target}
whose content is specified as the latter parameter.
The provided code then forwards the processing to
\textit{main} or \textit{dest} as described in \secref{sec:forward}.

%%%%%%%%%%%%%%%%%%%%%%%%%%%%%%%%%%%%%%%%%%%%%%%%%%%%%%%%%%%%%%%%%%%%%%%%%%%%%%%%
\subsection{Include by Input}
\label{sec:input}

Including child documents by |\include| has some restrictions by design.
Most notably, the content of a child document always occupies
its own set of pages; pages cannot be shared between child documents.
Usually, this behaviour makes perfect sense
because each child document contain an essential part of the document.
However, in some situations it may be desirable to compose
a document from a collection of parts
without having mandatory page breaks between then.
For this case, the package
provides a mechanism to include parts
by |\input| which can also be processed individually.
However, by construction this mechanism
requires manual handling of the content to be output.

%%%%%%%%%%%%%%%%%%%%%%%%%%%%%%%%%%%%%%%%
\DescribeMacro{\ifchilddocmanual}
The main file should be prepared as usual, see \secref{sec:include}.
However, the document body must make a distinction
between processing of an individual part and of the main document, e.g.:
%
\begin{center}
\begin{tabular}{l}
|\ifchilddocmanual|\\
|\input{\childdocname}|\\
|\||else|\\
\textit{document body with }|\input{|\textit{part}|}|\\
|\||fi|
\end{tabular}
\end{center}
%
The conditional |\ifchilddocmanual| is true whenever
a part to be included by |\input| is being compiled,
and the name of the part is stored in |\childdocname|.

%%%%%%%%%%%%%%%%%%%%%%%%%%%%%%%%%%%%%%%%
\DescribeMacro{\childdocby}
Each part to be included by |\input| should start with:
%
\begin{center}
\begin{tabular}{l}
|\input{childdoc.def}|\\
|\childdocby{|\textit{main}|}|\\
\end{tabular}
\end{center}
%
The directive |\childdocby| is similar to |\childdocof|
described in \secref{sec:include},
but the subsequent selection of content must be done manually.
To that end, both |\ifchilddoc| and |\ifchilddocmanual|
will be true upon processing of a part,
and the name of the part is stored in |\childdocname|.
Note that |\jobname| will be set to the filename of the current part
so that each part receives an individual |.aux| file
that does not interfere with the |.aux| file(s) of the main document.
This behaviour can be altered by the alternative form
|\childdocby[*]{|\textit{main}|}| (with a non-empty optional argument)
which uses the |.aux| file of the main document
by setting |\jobname| to \textit{main}.

%%%%%%%%%%%%%%%%%%%%%%%%%%%%%%%%%%%%%%%%%%%%%%%%%%%%%%%%%%%%%%%%%%%%%%%%%%%%%%%%
\subsection{Driver Development}
\label{sec:driver}

The \textsf{childdoc} mechanism can also be use for the development
of definition files such as \LaTeX{} styles or classes.
This case differs from the above setup with multiple parts
included by |\include| in that no |\includeonly| should be invoked.
This can be achieved by starting the include file
(before |\ProvidesPackage|) with:
%
\begin{center}
\begin{tabular}{l}
|\input{childdoc.def}|\\
|\childdocforward{|\textit{main}|}|\\
\end{tabular}
\end{center}
%
or alternatively with:
%
\begin{center}
\begin{tabular}{l}
|\input{childdoc.def}|\\
|\childdocby{|\textit{main}|}|\\
\end{tabular}
\end{center}
%
Both forms have slightly different effects as described above.
The main file is prepared as usual, see \secref{sec:include}.

%%%%%%%%%%%%%%%%%%%%%%%%%%%%%%%%%%%%%%%%%%%%%%%%%%%%%%%%%%%%%%%%%%%%%%%%%%%%%%%%
\subsection{Legacy Detection}
\label{sec:detection}

The directive |\childdocmain| in the main file can detect
whether the complete document or merely a child is to be compiled
even without using the directive |\childdocof|.
This method is deprecated because it is less robust
and there is no compelling reason to use it;
it is merely provided for backward compatibility
and it may be removed in future versions.

If the detection mechanism is to be used,
it is mandatory to correctly specify
the filename of the main file as the argument of |\childdocmain|:
%
\begin{center}
\begin{tabular}{l}
|\input{childdoc.def}|\\
|\childdocmain{|\textit{main}|}|\\
\end{tabular}
\end{center}
%
If |\jobname| does not match the argument \textit{main} of |\childdocmain|,
it is assumed that |\jobname| points to the child file to be compiled.
When using |\childdocmain| with the main file specified as argument,
it suffices to start a child file
with just |\input{|\textit{main}|}|
without loading of the package and using |\childdocof|.
If instead all processing is done
with the appropriate \textsf{childdoc} directives,
the argument of \textit{main} of |\childdocmain| can be empty.

An alternative version of the command line processing described
in \secref{sec:commandline} using the detection mechanism reads:
%
\begin{center}
|... -jobname "|\textit{target}|" "|[\textit{flags}]%
[|\def\jobname{|\textit{dest}|}|]|\input{|\textit{main}|}"|
\end{center}

%%%%%%%%%%%%%%%%%%%%%%%%%%%%%%%%%%%%%%%%%%%%%%%%%%%%%%%%%%%%%%%%%%%%%%%%%%%%%%%%
\subsection{Manual Code}
\label{sec:manual}

In case one cannot be certain whether the definitions file |childdoc.def|
is installed on the target \TeX{} distribution
and one prefers not to ship it,
it is conceivable to paste a few relevant commands into the sources.

To that end, drop all statements |\input{childdoc.def}|
and perform the replacements as outlined below.
Instead of |\childdocmain{|\textit{main}|}| add the following code
to the top of the main file:
%
\begin{center}
\begin{tabular}{l}
|\||ifdefined\childdocname\endinput\||fi\newif\ifchilddoc|\\
|\edef\childdocname{\scantokens\expandafter{\jobname\noexpand}}|\\
|\def\childdocmain{|\textit{main}|}\||ifx\childdocmain\childdocname\||else|\\
|\childdoctrue\includeonly{\childdocname}\let\jobname\childdocmain\||fi|\\
\end{tabular}
\end{center}
%
Instead of |\childdocof{|\textit{main}|}| just include the main file
at the top of each child file:
%
\begin{center}
|\input{|\textit{main}|}|
\end{center}
%
A simple redirection |\childdocforward{|\textit{dest}|}| is achieved by:
%
\begin{center}
|\def\jobname{|\textit{dest}|}\input{\jobname}|
\end{center}
%
The redirection with prefix
|\childdocforwardprefix[|\textit{prefix}|]{|\textit{dest}|}|
is accomplished by:
%
\begin{center}
\begin{tabular}{l}
|{\edef\jobname{\scantokens\expandafter{\jobname\noexpand}}|\\
|\def\redirectjob |\textit{prefix}|#1~~~{\gdef\jobname{|\textit{dest}|#1}}|\\
|\expandafter\redirectjob\jobname~~~}\input{\jobname}|
\end{tabular}
\end{center}

In an alternative approach,
child documents can be compiled by a specific command line
without additional code or specific definitions:
%
\begin{center}
|... -jobname "|\textit{target}|" "|[\textit{flags}]%
|\includeonly{|\textit{dest}|}\input{|\textit{main}|}"|
\end{center}
%

%%%%%%%%%%%%%%%%%%%%%%%%%%%%%%%%%%%%%%%%%%%%%%%%%%%%%%%%%%%%%%%%%%%%%%%%%%%%%%%%
%%%%%%%%%%%%%%%%%%%%%%%%%%%%%%%%%%%%%%%%%%%%%%%%%%%%%%%%%%%%%%%%%%%%%%%%%%%%%%%%
\section{Information}

%%%%%%%%%%%%%%%%%%%%%%%%%%%%%%%%%%%%%%%%%%%%%%%%%%%%%%%%%%%%%%%%%%%%%%%%%%%%%%%%
\subsection{Copyright}

Copyright \copyright{} 2017--2018 Niklas Beisert

This work may be distributed and/or modified under the
conditions of the \LaTeX{} Project Public License, either version 1.3
of this license or (at your option) any later version.
The latest version of this license is in
  \url{http://www.latex-project.org/lppl.txt}
and version 1.3 or later is part of all distributions of \LaTeX{}
version 2005/12/01 or later.

This work has the LPPL maintenance status `maintained'.

The Current Maintainer of this work is Niklas Beisert.

This work consists of the files |README.txt|, |childdoc.ins| and |childdoc.dtx|
as well as the derived files |childdoc.def|, |cdocsamp.tex|
with |cdocsch1.tex|, |cdocsch2.tex|, |cdocspt3.tex|, |cdocspt4.tex|,
|cdocsdrf.tex|, |cdocsfn1.tex|, |cdocsfn2.tex|
as well as |childdoc.pdf|.

%%%%%%%%%%%%%%%%%%%%%%%%%%%%%%%%%%%%%%%%%%%%%%%%%%%%%%%%%%%%%%%%%%%%%%%%%%%%%%%%
\subsection{Files and Installation}

The package consists of the files:
%
\begin{center}
\begin{tabular}{ll}
    |README.txt|   & readme file \\
    |childdoc.ins| & installation file \\
    |childdoc.dtx| & source file \\
    |childdoc.def| & definition file \\
    |cdocsamp.tex| & sample main file \\
    |cdocsch1.tex| & sample include file \\
    |cdocsch2.tex| & sample include file \\
    |cdocspt3.tex| & sample part file \\
    |cdocspt4.tex| & sample part file \\
    |cdocsdrf.tex| & sample redirection file \\
    |cdocsfn1.tex| & sample redirection file \\
    |cdocsfn2.tex| & sample redirection file \\
    |childdoc.pdf| & manual
\end{tabular}
\end{center}
%
The distribution consists of the files
|README.txt|, |childdoc.ins| and |childdoc.dtx|.
%
\begin{itemize}
\item
Run (pdf)\LaTeX{} on |childdoc.dtx|
to compile the manual |childdoc.pdf| (this file).
\item
Run \LaTeX{} on |childdoc.ins| to create the definitions file |childdoc.def|
and the sample |cdocsamp.tex| with include files
|cdocsch1.tex|, |cdocsch2.tex|, |cdocspt3.tex|, |cdocspt4.tex|,
|cdocsdrf.tex|, |cdocsfn1.tex|, |cdocsfn2.tex|.
Then copy the file |childdoc.def| to an appropriate directory of your \LaTeX{}
distribution, e.g.\ \textit{texmf-root}|/tex/latex/childdoc|.
\end{itemize}

%%%%%%%%%%%%%%%%%%%%%%%%%%%%%%%%%%%%%%%%%%%%%%%%%%%%%%%%%%%%%%%%%%%%%%%%%%%%%%%%
\subsection{Related CTAN Packages}

There are several other packages which offer a similar functionality:
%
\begin{itemize}
\item
The packages
\href{http://ctan.org/pkg/docmute}{\textsf{docmute}},
\href{http://ctan.org/pkg/includex}{\textsf{includex}} and
\href{http://ctan.org/pkg/standalone}{\textsf{standalone}}
provide commands to include only the document body of
a child file thus allowing both files to be compiled individually.
\item
The packages \href{http://ctan.org/pkg/subdocs}{\textsf{subdocs}}
and \href{http://ctan.org/pkg/subfiles}{\textsf{subfiles}}
provide structures in which the main and child documents can be
encapsulated and allowing them to be compiled individually.
The inclusion mechanism is different from the conventional |\include|.
\item
The package \href{http://ctan.org/pkg/combine}{\textsf{combine}}
is an elaborate solution to combine several documents into one.
\end{itemize}
%
See also the CTAN topic \href{http://ctan.org/topic/subdocs}{\textsf{subdocs}}
for further related packages.
The present package differs from the above solutions in that
a document structure constructed with the conventional |\include| mechanism
just needs two extra commands at the top of every file
such that all constituent files can be compiled individually.

%%%%%%%%%%%%%%%%%%%%%%%%%%%%%%%%%%%%%%%%%%%%%%%%%%%%%%%%%%%%%%%%%%%%%%%%%%%%%%%%
%\subsection{Feature Suggestions}
%
%The following is a list of features which may be useful for future
%versions of this package:
%%
%\begin{itemize}
%\item
%\ldots
%\end{itemize}

%%%%%%%%%%%%%%%%%%%%%%%%%%%%%%%%%%%%%%%%%%%%%%%%%%%%%%%%%%%%%%%%%%%%%%%%%%%%%%%%
\subsection{Revision History}

%%%%%%%%%%%%%%%%%%%%%%%%%%%%%%%%%%%%%%%%
\paragraph{v2.0:} 2018/12/30

\begin{itemize}
\item
immediate forward processing
\item
added |\childdocby| mechanism
\item
manual restructured
\end{itemize}

%%%%%%%%%%%%%%%%%%%%%%%%%%%%%%%%%%%%%%%%
\paragraph{v1.6:} 2018/01/17

\begin{itemize}
\item
application for development of include files
\item
corrections to manual
\end{itemize}

%%%%%%%%%%%%%%%%%%%%%%%%%%%%%%%%%%%%%%%%
\paragraph{v1.5:} 2017/05/21

\begin{itemize}
\item
more complete structuring introduced
\item
|\childdocof| introduced
\item
|\childdoc| renamed to |\childdocmain|
\item
|\childredirect| renamed to |\childdocforward| and |\childdocforwardprefix|
and functionality expanded
\end{itemize}

%%%%%%%%%%%%%%%%%%%%%%%%%%%%%%%%%%%%%%%%
\paragraph{v1.0:} 2017/04/27

\begin{itemize}
\item
manual and install package
\item
first version published on CTAN
\end{itemize}

%%%%%%%%%%%%%%%%%%%%%%%%%%%%%%%%%%%%%%%%
\paragraph{v0.6:} 2017/04/26

\begin{itemize}
\item
redirection mechanism added
\end{itemize}

%%%%%%%%%%%%%%%%%%%%%%%%%%%%%%%%%%%%%%%%
\paragraph{v0.5:} 2017/04/26

\begin{itemize}
\item
functionality in definition file
\end{itemize}


%%%%%%%%%%%%%%%%%%%%%%%%%%%%%%%%%%%%%%%%%%%%%%%%%%%%%%%%%%%%%%%%%%%%%%%%%%%%%%%%
%%%%%%%%%%%%%%%%%%%%%%%%%%%%%%%%%%%%%%%%%%%%%%%%%%%%%%%%%%%%%%%%%%%%%%%%%%%%%%%%
%%%%%%%%%%%%%%%%%%%%%%%%%%%%%%%%%%%%%%%%%%%%%%%%%%%%%%%%%%%%%%%%%%%%%%%%%%%%%%%%
\appendix

\settowidth\MacroIndent{\rmfamily\scriptsize 000\ }

 \DocInput{childdoc.dtx}

\end{document}
%</driver>
% \fi
%
% %%%%%%%%%%%%%%%%%%%%%%%%%%%%%%%%%%%%%%%%%%%%%%%%%%%%%%%%%%%%%%%%%%%%%%%%%%%%%%
% %%%%%%%%%%%%%%%%%%%%%%%%%%%%%%%%%%%%%%%%%%%%%%%%%%%%%%%%%%%%%%%%%%%%%%%%%%%%%%
% \section{Sample}
%\iffalse
%<*samplemain>
%\fi
%
% The following presents a sample document
% with two chapters, two parts, a title page,
% a compile flag as well as three forwarding files to set the flag.
% It consists of eight |.tex| files:
% \begin{center}
% \begin{tabular}{ll}
% |cdocsamp.tex|&main file\\
% |cdocsch1.tex|&include file for chapter 1\\
% |cdocsch2.tex|&include file for chapter 2\\
% |cdocspt3.tex|&include file for part 3\\
% |cdocspt4.tex|&include file for part 4\\
% |cdocsdrf.tex|&forwarding file for main file in draft mode\\
% |cdocsfi1.tex|&forwarding file for final version of chapter 1\\
% |cdocsfi2.tex|&forwarding file for final version of chapter 2\\
% \end{tabular}
% \end{center}
% Each of the eight files can be compiled directly by the \LaTeX{} compiler.
%
% %%%%%%%%%%%%%%%%%%%%%%%%%%%%%%%%%%%%%%
% \paragraph{Main File.}
%
% The main file is called |cdocsamp.tex|.
%
% Load the \textsf{childdoc} definitions and
% declare the filename for the main document:
%    \begin{macrocode}
\input{childdoc.def}
\childdocmain{}
%    \end{macrocode}

% Optional override for |\version| flag:
%    \begin{macrocode}
%%\ifchilddoc\else\providecommand{\version}{draft}\fi
%    \end{macrocode}

% Define the default values for the |\version| flag
% (|final| for the main file and |draft| for childs):
%    \begin{macrocode}
\ifchilddoc
\providecommand{\version}{draft}
\else
\providecommand{\version}{final}
\fi
%    \end{macrocode}

% Load the standard document class:
%    \begin{macrocode}
\documentclass[12pt]{article}
%    \end{macrocode}

% Start the document body:
%    \begin{macrocode}
\begin{document}
%    \end{macrocode}

% Declare a title page.
% Print title, part of document being processed and version flag:
%    \begin{macrocode}
\addtocounter{page}{-1}
\begin{center}
{\LARGE\bfseries{}childdoc example\par}
\vspace{1cm}
\ifchilddoc
\ifchilddocmanual part\else chapter\fi:
`\childdocname' of `\childdocjob'\par
\else
main document: `\childdocjob'\par
\fi
version: \version\par
\end{center}
\newpage
%    \end{macrocode}

% Manually include selected file,
% otherwise process as usual:
%    \begin{macrocode}
\ifchilddocmanual
\section*{part `\childdocname'}
\input{\childdocname}
\else
%    \end{macrocode}

% Include the two chapters:
%    \begin{macrocode}
\include{cdocsch1}
\include{cdocsch2}
%    \end{macrocode}

% Include the two parts unless only chapters should be displayed:
%    \begin{macrocode}
\ifchilddoc\else
\section{part three}
\input{cdocspt3}
\section{part four}
\input{cdocspt4}
\fi
%    \end{macrocode}

% Process as usual until here:
%    \begin{macrocode}
\fi
%    \end{macrocode}

% End of document body:
%    \begin{macrocode}
\end{document}
%    \end{macrocode}
%\iffalse
%</samplemain>
%\fi
%
% %%%%%%%%%%%%%%%%%%%%%%%%%%%%%%%%%%%%%%
% \paragraph{Chapter Include Files.}
%
% The include files are called |cdocsch1.tex| and |cdocsch2.tex|.
%
%\iffalse
%<*samplechap1|samplechap2>
%\fi

% Optional override for |\version| flag:
%    \begin{macrocode}
%%\providecommand{\version}{final}
%    \end{macrocode}

% Include the main document:
%    \begin{macrocode}
\input{childdoc.def}
\childdocof{cdocsamp}
%    \end{macrocode}

%\iffalse
%</samplechap1|samplechap2>
%\fi
%
%\iffalse
%<*samplechap1>
%\fi
% Some text for chapter 1:
%    \begin{macrocode}
\section{one}
some text in chapter one
%    \end{macrocode}

%\iffalse
%</samplechap1>
%\fi
% Some text for chapter 2:
%\iffalse
%<*samplechap2>
%\fi
%    \begin{macrocode}
\section{two}
more text in chapter two
%    \end{macrocode}

%\iffalse
%</samplechap2>
%\fi
%
% %%%%%%%%%%%%%%%%%%%%%%%%%%%%%%%%%%%%%%
% \paragraph{Part Include Files.}
%
% The include files are called |cdocspt3.tex| and |cdocspt4.tex|.
%
%\iffalse
%<*samplepart3|samplepart4>
%\fi

% Optional override for |\version| flag:
%    \begin{macrocode}
%%\providecommand{\version}{final}
%    \end{macrocode}

% Include the main document:
%    \begin{macrocode}
\input{childdoc.def}
\childdocby{cdocsamp}
%    \end{macrocode}

%\iffalse
%</samplepart3|samplepart4>
%\fi
%
%\iffalse
%<*samplepart3>
%\fi
% Some text for part 3:
%    \begin{macrocode}
some text in part three
%    \end{macrocode}

%\iffalse
%</samplepart3>
%\fi
% Some text for part 4:
%\iffalse
%<*samplepart4>
%\fi
%    \begin{macrocode}
more text in part four
%    \end{macrocode}

%\iffalse
%</samplepart4>
%\fi
%
% %%%%%%%%%%%%%%%%%%%%%%%%%%%%%%%%%%%%%%
% \paragraph{Forwarding for a Complete Draft.}
%
% The following forwarding file |cdocsdrf.tex|
% compiles the main document in draft mode:
%\iffalse
%<*sampledraft>
%\fi
%    \begin{macrocode}
\def\version{draft}
\input{childdoc.def}
\childdocforward{cdocsamp}
%    \end{macrocode}

%\iffalse
%</sampledraft>
%\fi
%
% %%%%%%%%%%%%%%%%%%%%%%%%%%%%%%%%%%%%%%
% \paragraph{Forwarding for Final Version of the Chapters.}
%
% The following forwarding files |cdocsfn1.tex| and |cdocsfn2.tex|
% (with identical content)
% compile the final versions of the child documents
% |cdocsch1.tex| and |cdocsch2.tex|, respectively:
%\iffalse
%<*samplefinal>
%\fi
%    \begin{macrocode}
\def\version{final}
\input{childdoc.def}
\childdocforwardprefix[cdocsamp]{cdocsfn}{cdocsch}
%    \end{macrocode}

%\iffalse
%</samplefinal>
%\fi
%
% %%%%%%%%%%%%%%%%%%%%%%%%%%%%%%%%%%%%%%
% \paragraph{Command Line Processing.}
%
% The following three command lines generate the output files
% |cdocscld|, |cdocscl1| and |cdocscl2|
% which should be identical to
% |cdocsdrf|, |cdocsch1| and |cdocsfn2|, respectively:
% \begin{center}
% \begin{tabular}{l}
% |latex -jobname cdocscld \|\\
% |  "\def\version{draft}\input{childdoc.def}\childdocforward{cdocsamp}"|\\
% |latex -jobname cdocscl1 \|\\
% |  "\input{childdoc.def}\childdocforward[cdocsamp]{cdocsch1}"|\\
% |latex -jobname cdocscl2 \|\\
% |  "\def\version{final}\input{childdoc.def}\childdocforward{cdocsch2}"|
% \end{tabular}
% \end{center}
% Note that the trailing backslash on each first line
% merely continues the input to the second line
% (for convenient cut ant paste).
% Furthermore, the command |latex| can be replaced by any
% of its alternative versions such as |pdflatex|.
%
% %%%%%%%%%%%%%%%%%%%%%%%%%%%%%%%%%%%%%%%%%%%%%%%%%%%%%%%%%%%%%%%%%%%%%%%%%%%%%%
% %%%%%%%%%%%%%%%%%%%%%%%%%%%%%%%%%%%%%%%%%%%%%%%%%%%%%%%%%%%%%%%%%%%%%%%%%%%%%%
% \section{Implementation}
%\iffalse
%<*package>
%\fi
%
% This section describes the definitions file |childdoc.def|.

% The definitions cannot be loaded using |\usepackage| or |\RequirePackage|
% which has a mechanism to prevent loading a style file more than once.
% When loading the definitions by means of |\input|
% multiple instances have to be prevented manually:
%\iffalse
%This code needs to be before the `\ProvidesFile' directive
%which is defined at the beginning of this file.
%Therefore it is also placed there and commented out here.
%</package>
%<*discard>
%\fi
%    \begin{macrocode}
\ifdefined\childdocmain\endinput\fi
%    \end{macrocode}
%\iffalse
%</discard>
%<*package>
%\fi
%
% \macro{\ifchilddoc}
% \macro{\ifchilddocmanual}
% The conditional |\ifchilddoc| tells whether a
% child (true) or main (false) document is being compiled.
% The conditional |\ifchilddocmanual| tells whether
% the |\includeonly| mechanism is used (false) or
% the selection of child files must be performed manually (true).
% The definitions initialise to false:
%    \begin{macrocode}
\newif\ifchilddoc
\newif\ifchilddocmanual
%    \end{macrocode}

% \macro{\childdocname}
% \macro{\childdocjob}
% The macro |\childdocname| stores the name of the main document
% to be compiled. The macro |\childdocjob| stores the name of
% the document on which the \LaTeX{} compiler was originally invoked.
% The content of |\jobname| cannot be compared
% to filenames specified in the source due to different catcodes.
% The following code rescans |\jobname|, stores the result
% in |\childdocname| and saves a copy in |\childdocjob|:
%    \begin{macrocode}
\edef\childdocname{\scantokens\expandafter{\jobname\noexpand}}
\let\childdocjob\childdocname
%    \end{macrocode}

% \macro{\childdocdisable}
% The macro |\childdocdisable| prevents the main file
% from being processed more than once.
% At this stage, the main document command |\childdocmain|
% is assumed to be called once again where it should do nothing.
% Any subsequent call to it should prevent
% a secondary processing of the main document
% It overwrites the forwarding commands
% |\childdocof| and |\childdocforward|
% with empty macros to prevent further inclusions of the main document:
%    \begin{macrocode}
\newcommand{\childdocdisable}
{
  \renewcommand{\childdocmain}[1]{\renewcommand{\childdocmain}[1]{\endinput}}
  \renewcommand{\childdocof}[1]{}
  \renewcommand{\childdocby}[2][]{}
  \renewcommand{\childdocforward}[2][]{}
  \renewcommand{\childdocdisable}{}
}
%    \end{macrocode}

% \macro{\childdocmain}
% The macro |\childdocmain| is to be called at the top of the main file
% with nothing or the main filename (without extension) as argument.
% First, it breaks loops.
% If the argument is not empty and does not match |\childdocname|
% (which is set by the first inclusion of |childdoc.def|),
% |\ifchilddoc| is set to true, |\includeonly| is applied to the child file
% and |\jobname| is set to the main file
% (for proper handling of |.aux| files):
%    \begin{macrocode}
\newcommand{\childdocmain}[1]
{
  \childdocdisable\childdocmain{}
  \if?#1?\else
    \begingroup
      \def\childdoctmp{#1}
      \ifx\childdoctmp\childdocname
        \def\childdoctmp{}
      \else
        \def\childdoctmp
        {
          \childdoctrue
          \includeonly{\childdocname}
          \def\childdocjob{#1}
          \def\jobname{#1}
        }
      \fi
      \expandafter
    \endgroup
    \childdoctmp
  \fi
}
%    \end{macrocode}

% \macro{\childdocof}
% The command |\childdocof| redirects
% compilation to the main file |#1|.
%    \begin{macrocode}
\newcommand{\childdocof}[1]
{
  \childdocdisable
  \childdoctrue
  \includeonly{\childdocname}
  \def\jobname{#1}
  \def\childdocjob{#1}
  \input{#1}
}
%    \end{macrocode}

% \macro{\childdocby}
% The command |\childdocby| ....
%    \begin{macrocode}
\newcommand{\childdocby}[2][]
{
  \childdocdisable
  \childdoctrue
  \childdocmanualtrue
  \if?#1?\else
    \def\jobname{#2}
  \fi
  \def\childdocjob{#2}
  \input{#2}
  \endinput
}
%    \end{macrocode}

% \macro{\childdocforward}
% The command |\childdocforward| redirects
% compilation to the main file or
% (if the optional argument is given) a child file.
% Parameters are set as if the main file
% or a child file starting with |\childdocof| was compiled.
% Then compilation is handed over to the main file:
%    \begin{macrocode}
\newcommand{\childdocforward}[2][]
{
  \begingroup
    \if?#1?
      \def\childdoctmp
      {
        \def\childdocname{#2}
        \def\childdocjob{#2}
        \def\jobname{#2}
        \input{#2}
        \endinput
      }
    \else
      \def\childdoctmp
      {
        \childdocdisable
        \def\childdocname{#2}
        \childdoctrue
        \includeonly{#2}
        \def\childdocjob{#1}
        \def\jobname{#1}
        \input{#1}
        \endinput
      }
    \fi
    \expandafter
  \endgroup
  \childdoctmp
}
%    \end{macrocode}

% \macro{\childdocforwardprefix}
% The command |\childdocforwardprefix| redirects
% compilation to the main or a child file by means of a pattern.
% The prefix |#1| in the current filename is replaced by |#2|
% and the suffix of the current filename is kept
% (it is assumed that the filename does not contain the substring `|~~~|'
% which is used as a delimiter).
% Compilation is handed over to the new file by |\childdocforward|:
%    \begin{macrocode}
\newcommand{\childdocforwardprefix}[3][]
{
  \begingroup
    \def\childdocextract #2##1~~~{\def\childdoctmp{\childdocforward[#1]{#3##1}}}
    \expandafter\childdocextract\childdocname~~~
    \expandafter
  \endgroup
  \childdoctmp
}
%    \end{macrocode}

% \macro{\childdoc}
% The deprecated macro |\childdoc| is a legacy version of |\childdocmain|:
%    \begin{macrocode}
\newcommand{\childdoc}{\childdocmain}
%    \end{macrocode}

% \macro{\childdocredirect}
% The deprecated macro |\childdocredirect| is a legacy version
% of |\childdocforward| and |\childdocforwardprefix|:
%    \begin{macrocode}
\newcommand{\childdocredirect}[2][]
{
  \begingroup
    \if?#1?
      \def\childdoctmp{\childdocforward{#2}}
    \else
      \def\childdoctmp{\childdocforwardprefix{#1}{#2}}
    \fi
    \expandafter
  \endgroup
  \childdoctmp
}
%    \end{macrocode}

%\iffalse
%</package>
%\fi
%
\endinput

\childdocof{cdocsamp}
%    \end{macrocode}

%\iffalse
%</samplechap1|samplechap2>
%\fi
%
%\iffalse
%<*samplechap1>
%\fi
% Some text for chapter 1:
%    \begin{macrocode}
\section{one}
some text in chapter one
%    \end{macrocode}

%\iffalse
%</samplechap1>
%\fi
% Some text for chapter 2:
%\iffalse
%<*samplechap2>
%\fi
%    \begin{macrocode}
\section{two}
more text in chapter two
%    \end{macrocode}

%\iffalse
%</samplechap2>
%\fi
%
% %%%%%%%%%%%%%%%%%%%%%%%%%%%%%%%%%%%%%%
% \paragraph{Part Include Files.}
%
% The include files are called |cdocspt3.tex| and |cdocspt4.tex|.
%
%\iffalse
%<*samplepart3|samplepart4>
%\fi

% Optional override for |\version| flag:
%    \begin{macrocode}
%%\providecommand{\version}{final}
%    \end{macrocode}

% Include the main document:
%    \begin{macrocode}
% \iffalse
%
% childdoc.dtx Copyright (C) 2017-2018 Niklas Beisert
%
% This work may be distributed and/or modified under the
% conditions of the LaTeX Project Public License, either version 1.3
% of this license or (at your option) any later version.
% The latest version of this license is in
%   http://www.latex-project.org/lppl.txt
% and version 1.3 or later is part of all distributions of LaTeX
% version 2005/12/01 or later.
%
% This work has the LPPL maintenance status `maintained'.
%
% The Current Maintainer of this work is Niklas Beisert.
%
% This work consists of the files childdoc.dtx and childdoc.ins
% and the derived files childdoc.def and cdocsamp.tex with
% cdocsch1.tex, cdocsch2.tex, cdocsdrf.tex, cdocsfn1.tex, cdocsfn2.tex.
%
%<package>\ifdefined\childdocmain\endinput\fi
%<package>\ProvidesFile{childdoc.def}[2018/12/30 v2.0 child document driver]
%<samplemain>\ProvidesFile{cdocsamp.tex}[2018/12/30 v2.0 sample for childdoc]
%<*driver>
%\ProvidesFile{childdoc.drv}[2018/12/30 v2.0 childdoc reference manual file]
\PassOptionsToClass{10pt,a4paper}{article}
\documentclass{ltxdoc}

\usepackage[margin=35mm]{geometry}
\usepackage{hyperref}
\usepackage{hyperxmp}
\usepackage[usenames]{color}

\hypersetup{colorlinks=true}
\hypersetup{pdfstartview=FitH}
\hypersetup{pdfpagemode=UseNone}
\hypersetup{pdfsource={}}
\hypersetup{pdflang={en-UK}}
\hypersetup{pdfcopyright={Copyright 2017-2018 Niklas Beisert.
  This work may be distributed and/or modified under the
  conditions of the LaTeX Project Public License, either version 1.3
  of this license or (at your option) any later version.}}
\hypersetup{pdflicenseurl={http://www.latex-project.org/lppl.txt}}
\hypersetup{pdfcontactaddress={ETH Zurich, ITP, HIT K,
  Wolfgang-Pauli-Strasse 27}}
\hypersetup{pdfcontactpostcode={8093}}
\hypersetup{pdfcontactcity={Zurich}}
\hypersetup{pdfcontactcountry={Switzerland}}
\hypersetup{pdfcontactemail={nbeisert@itp.phys.ethz.ch}}
\hypersetup{pdfcontacturl={http://people.phys.ethz.ch/\xmptilde nbeisert/}}

\newcommand{\secref}[1]{\hyperref[#1]{section \ref*{#1}}}

\parskip1ex
\parindent0pt
\let\olditemize\itemize
\def\itemize{\olditemize\parskip0pt}

\begin{document}

\title{The \textsf{childdoc} Package}
\hypersetup{pdftitle={The childdoc Package}}
\author{Niklas Beisert\\[2ex]
  Institut f\"ur Theoretische Physik\\
  Eidgen\"ossische Technische Hochschule Z\"urich\\
  Wolfgang-Pauli-Strasse 27, 8093 Z\"urich, Switzerland\\[1ex]
  \href{mailto:nbeisert@itp.phys.ethz.ch}
  {\texttt{nbeisert@itp.phys.ethz.ch}}}
\hypersetup{pdfauthor={Niklas Beisert}}
\hypersetup{pdfsubject={Manual for the LaTeX2e Package childdoc}}
\date{30 December 2018, \textsf{v2.0}}
\maketitle

\begin{abstract}\noindent
\textsf{childdoc} is a \LaTeXe{} package
that enables the direct compilation
of document sections included by |\include|
to individual files.
\end{abstract}

\begingroup
\parskip0ex
\tableofcontents
\endgroup

%%%%%%%%%%%%%%%%%%%%%%%%%%%%%%%%%%%%%%%%%%%%%%%%%%%%%%%%%%%%%%%%%%%%%%%%%%%%%%%%
%%%%%%%%%%%%%%%%%%%%%%%%%%%%%%%%%%%%%%%%%%%%%%%%%%%%%%%%%%%%%%%%%%%%%%%%%%%%%%%%
\section{Introduction}

\LaTeX{} provides a mechanism to structure a large document (such as a book)
into a main file and several child files (containing the chapters)
using the |\include| command.
This mechanism is beneficial for documents
which span hundreds of pages in order to
make the source file(s) more manageable.
Moreover, compilation can be restricted to
selected child files by means of the |\includeonly| command.
The latter feature can be used to reduce the compilation time while editing
(this was significantly more useful in the earlier days of \LaTeX{})
or to generate a smaller document which is easier to navigate.
Another application of |\includeonly| is to generate
documents consisting of selected parts of the complete document.

However, there are a few drawbacks of the plain |\include| mechanism:
\begin{itemize}
\item
The child files cannot be compiled on their own,
they can only be compiled via the main file.
A naive editing environment
(such as a text editor with an option
to have the current file processed by \LaTeX)
may require one to switch to the main file before compiling;
attempting to compile the child file produces errors.
\item
The main file must be modified (each time)
to adjust the |\includeonly| command
to the present needs. This easily leaves the main file in a messy state.
\item
The generated document will always carry the filename
of the main document. This is inconvenient if
several child files are to be compiled and
to be kept for distribution.
\end{itemize}

The present package provides a simple interface
to make child files individually compilable by \LaTeX{}.
Compiling a child file then has the same effect as compiling
the main file with an |\includeonly| command
to select the appropriate child.
Moreover the generated document will carry the name of the child
rather than the main file.
This resolves all three above issues.

This feature is meant to make the editing of books,
thesis documents and lecture notes somewhat more convenient.
However, the package can also be used efficiently for
composing a series of documents (such as exercise sheets)
which are typically distributed individually.
It then assists the author in generating the individual documents
(potentially in different versions)
as well as a document containing the collected series.
Another application is in developing style files
or other kinds of included material
where compilation of the style file could redirect
to a sample or test file.

%%%%%%%%%%%%%%%%%%%%%%%%%%%%%%%%%%%%%%%%%%%%%%%%%%%%%%%%%%%%%%%%%%%%%%%%%%%%%%%%
%%%%%%%%%%%%%%%%%%%%%%%%%%%%%%%%%%%%%%%%%%%%%%%%%%%%%%%%%%%%%%%%%%%%%%%%%%%%%%%%
\section{Usage}

First of all, the package \textsf{childdoc} is \emph{not} a standard
\LaTeXe{} |.sty| style file! Therefore it needs to be invoked in
a non-standard way.

%%%%%%%%%%%%%%%%%%%%%%%%%%%%%%%%%%%%%%%%%%%%%%%%%%%%%%%%%%%%%%%%%%%%%%%%%%%%%%%%
\subsection{Included Files}
\label{sec:include}

%%%%%%%%%%%%%%%%%%%%%%%%%%%%%%%%%%%%%%%%
\DescribeMacro{\childdocmain}
To use the package, add the commands
\begin{center}
\begin{tabular}{l}
|\input{childdoc.def}|\\
|\childdocmain{}|\\
\end{tabular}
\end{center}
at the very top of the main \LaTeX{} file,
in particular \emph{before} the |\documentclass| statement!
The argument of |\childdocmain| should be left empty
(but it must be present).

%%%%%%%%%%%%%%%%%%%%%%%%%%%%%%%%%%%%%%%%
\DescribeMacro{\childdocof}
Furthermore, add the commands
\begin{center}
\begin{tabular}{l}
|\input{childdoc.def}|\\
|\childdocof{|\textit{main}|}|\\
\end{tabular}
\end{center}
at the top of every child file \textit{child}
which is included by |\include{|\textit{child}|}|
from within the main file
(or at least for those files to be compiled individually).
The argument \textit{main} must be the filename of the main file.

There are a couple of
considerations in setting up the main and child documents:

%%%%%%%%%%%%%%%%%%%%%%%%%%%%%%%%%%%%%%%%
\paragraph{Restrictions.}

Please note the following restrictions:
\begin{itemize}
\item
|\childdocmain| must be called with one argument \textit{main}
to ensure compatibility with earlier version of the package.
It must either be empty (|\childdocmain{}|)
or precisely match the filename of the main file in which it is specified.
See \secref{sec:detection} for further information.
\item
The filename \textit{main} must be specified without the |.tex| extension.
\item
The filename \textit{main} is case sensitive
(even in case-insensitive file systems)
due to internal string comparison.
\item
The argument \textit{main} should be fully expanded, it cannot be a macro.
\item
Subdirectories and special characters should be avoided in filenames.
\item
The command |\childdocmain{|\textit{main}|}| must be followed by a whitespace.
It should not be followed immediately by another command
or by a comment mark `|%|'.
This is because the \TeX{} parser reads the token immediately following
the argument of |\childdocmain| and puts it
at the beginning of every child section;
however, a white\-space is ignored.
\end{itemize}

%%%%%%%%%%%%%%%%%%%%%%%%%%%%%%%%%%%%%%%%
\paragraph{Content of Main File.}

It is advisable to place all content in the child files included by |\include|.
Any output contained in the main file will appear in all child documents
unless suppressed manually;
it cannot be suppressed automatically by the |\includeonly| directive
and thus should normally be avoided.
A method to include some content in the main file
by means of conditional processing is described in \secref{sec:conditional}.

%%%%%%%%%%%%%%%%%%%%%%%%%%%%%%%%%%%%%%%%
\paragraph{Page Numbering.}

When only a part of the document is compiled,
the appropriate numbering of pages
(as well as other status parameters)
is determined from the |.aux| files.
The latter contain information from previous passes.
However this information needs to propagate through
all intermediate child documents.
Therefore the page numbering in child documents may well
be inconsistent until the complete document is compiled at least once.

A useful (if unconventional) way to always ensure a consistent
page numbering is to restart the numbering in each child document
and denote the pages by `\textit{child}|.|\textit{page}'
where \textit{child} represents the chapter/section number of the child file.
This can be achieved by the command
|\numberwithin{page}{|\textit{child}|}|
of the \textsf{amsmath} package
where \textit{child} can be |chapter| or |section|
depending on the chosen structuring.
Alternatively, one can modify the macro |\thepage| appropriately
and reset the counter |page| at the start of each child file.

%%%%%%%%%%%%%%%%%%%%%%%%%%%%%%%%%%%%%%%%%%%%%%%%%%%%%%%%%%%%%%%%%%%%%%%%%%%%%%%%
\subsection{Conditional Processing}
\label{sec:conditional}

The package provides a mechanism to compile different versions
of a document. To customise the versions further some conditional processing
can come in handy to distinguish which version is being compiled.
The package provides two macros to describe the compilation context:

%%%%%%%%%%%%%%%%%%%%%%%%%%%%%%%%%%%%%%%%
\DescribeMacro{\ifchilddoc}
The conditional |\ifchilddoc| distinguishes between the compilation of
child documents and the main document:
%
\begin{center}
|\ifchilddoc |\textit{child-code}| |[|\||else |\textit{main-code}]| \||fi|
\end{center}

%%%%%%%%%%%%%%%%%%%%%%%%%%%%%%%%%%%%%%%%
\DescribeMacro{\childdocname}
\DescribeMacro{\childdocjob}
The macro |\childdocname| contains the filename (without extension)
of the main or child file being processed.
Note that |\childdocjob| will always contain the name of the main file.

%%%%%%%%%%%%%%%%%%%%%%%%%%%%%%%%%%%%%%%%
\paragraph{Title Page.}

Conditional processing can be used to include a title or banner page
in the main document when proper precautions are taken.
Importantly, the code in the main file should ensure that the page counter
(as well as other status parameters which are stored in the |.aux| files)
takes the same value after the conditional processing.
Otherwise the page numbers may take divergent values
depending on which part is compiled.

For example, a title page could be declared by:
%
\begin{center}
\begin{tabular}{l}
|\ifchilddoc\||else|\\
|\addtocounter{page}{-1}|\\
\textit{code for title page}\\
|\newpage|\\
|\||fi|
\end{tabular}
\end{center}
%
A banner page for the child documents can be generated by:
%
\begin{center}
\begin{tabular}{l}
|\ifchilddoc|\\
|\addtocounter{page}{-1}|\\
\textit{code for banner page}\\
|\newpage|\\
|\||fi|
\end{tabular}
\end{center}
%
Here one could write a message such as:
\begin{center}
|This is the part \childdocname{} of \childdocjob{}.|
\end{center}

%%%%%%%%%%%%%%%%%%%%%%%%%%%%%%%%%%%%%%%%%%%%%%%%%%%%%%%%%%%%%%%%%%%%%%%%%%%%%%%%
\subsection{Flags}
\label{sec:flags}

The package makes it easy to generate different versions
of the main or child documents.
To this end compilation flags can be defined
and assigned different default values.
They will be particularly useful in conjunction
with the forwarding mechanism described in \secref{sec:forward}.

For example, it may be useful to have a flag |\version|
which can be set to |draft| or |final|.
The document source will contain some conditional code
depending on the value of |\version|.
Suppose further, the flag should default to |final| for the main file
and to |draft| for child files
which is a natural assignment for editing the document.
This is achieved by placing the following code
in the preamble of the main document
(below the |\childdocmain| directive):
%
\begin{center}
\begin{tabular}{l}
|\ifchilddoc|\\
|\providecommand{\version}{draft}|\\
|\||else|\\
|\providecommand{\version}{final}|\\
|\||fi|
\end{tabular}
\end{center}
%
The definition by |\providecommand| makes sure
that previous definitions are not overwritten.
Further statements |\providecommand{\version}{...}|
can thus be added before the above code to override it.

For the main file, one might add a line
(between |\childdocmain| and the above block)
%
\begin{center}
|%\ifchilddoc\||else\providecommand{\version}{draft}\||fi|
\end{center}
%
which can be uncommented to produce a draft version.
Likewise one can add a line to the very top of a child file
(above the |\childdocof{|\textit{main}|}| directive)
%
\begin{center}
|%\providecommand{\version}{final}|
\end{center}
%
which can be uncommented to produce the final version of this child document.

%%%%%%%%%%%%%%%%%%%%%%%%%%%%%%%%%%%%%%%%%%%%%%%%%%%%%%%%%%%%%%%%%%%%%%%%%%%%%%%%
\subsection{Forwarding}
\label{sec:forward}

Different versions of the main or child documents
using compilation flags as described in \secref{sec:flags}
can be (permanently) stored in different files
for convenient compilation, viewing and distribution.
To this end, the package defines a command
to pass on compilation to a different file:

%%%%%%%%%%%%%%%%%%%%%%%%%%%%%%%%%%%%%%%%
\DescribeMacro{\childdocforward}
The command |\childdocforward| redirects processing to
another source file:
%
\begin{center}
\begin{tabular}{l}
|\input{childdoc.def}|\\
|\childdocforward[|\textit{main}|]{|\textit{dest}|}|\\
\end{tabular}
\end{center}
%
The argument \textit{dest} is the destination file
(without extension).
It should be the main file or one of the child files.
Note that further \textsf{childdoc} directives
such as |\childdocof| and |\childdocforward|
in the indicated file will be processed in this form.
The optional argument \textit{main}
passes on directly to the main file \textit{main}
while pretending to compile the child \textit{dest}.
This form behaves as if \textit{dest}
issues |\childdocof{|\textit{main}|}| right away,
and no further \textsf{childdoc} directives will be processed.

%%%%%%%%%%%%%%%%%%%%%%%%%%%%%%%%%%%%%%%%
\DescribeMacro{\...prefix}
In the alternative form |\childdocforwardprefix|,
%
\begin{center}
\begin{tabular}{l}
|\input{childdoc.def}|\\
|\childdocforwardprefix[|\textit{main}|]{|\textit{prefix}|}{|\textit{dest}|}|
\end{tabular}
\end{center}
%
the destination file is determined by a pattern
depending on the current file:
To make this work, the current file must be called
`{\textit{prefix}\hspace{0.2em}\textit{suffix}}'
with \textit{prefix} matching precisely the argument.
Processing is then passed on to the file
`{\textit{dest}\hspace{0.2em}\textit{suffix}}'.
Surely, the same effect is achieved by
directly specifying the
argument `{\textit{dest}\hspace{0.2em}\textit{suffix}}'
in the first form.
However, that requires to set up a different file
for each child. With the alternative form of the command
all these files can have exactly the same content
which simplifies setting them up and maintaining them.

For example, the following file |draft.tex|
with a compilation flag |\version| as described in \secref{sec:flags}
compiles the main document as a draft:
%
\begin{center}
\begin{tabular}{l}
|\def\version{draft}|\\
|\input{childdoc.def}|\\
|\childdocforward{|\textit{main}|}|
\end{tabular}
\end{center}
%
Likewise, the following files |final|\textit{nn}|.tex|
compile the final version of the child document
|child|\textit{nn}|.tex|:
%
\begin{center}
\begin{tabular}{l}
|\def\version{final}|\\
|\input{childdoc.def}|\\
|\childdocforwardprefix{final}{child}|
\end{tabular}
\end{center}
%

Note that when several versions of a main file and/or of each child file
are to be generated, it may be convenient to set up a |Makefile| or
shell script to automatise the process.

%%%%%%%%%%%%%%%%%%%%%%%%%%%%%%%%%%%%%%%%%%%%%%%%%%%%%%%%%%%%%%%%%%%%%%%%%%%%%%%%
\subsection{Command Line Processing}
\label{sec:commandline}

The effect of redirection files can also be achieved by invoking
the \LaTeX{} compiler with a more elaborate command line.
Most conveniently this should be done as part
of a shell script or a |Makefile|.

When using \textsf{childdoc} in the main file, the following
command lines effectively perform a redirection
(note that depending on the shell being used,
backslashes may have to be doubled: `|\|' $\to$ `|\\|'):
%
\begin{center}
|... -jobname "|\textit{target}|" |\\|"|[\textit{flags}]%
|\input{childdoc.def}\childdocforward[|\textit{main}|]{|\textit{dest}|}"|
\end{center}
%
Here \textit{target} is the name of the output file,
\textit{main} is the name of the main file
and \textit{dest} is the name of the main or child file to be processed
(all filenames without extensions).
The optional argument \textit{main} can be omitted
if \textit{main} matches \textit{dest}.
Optionally, compilation \textit{flags} can be defined via |\def| commands.
This command line makes the \TeX{} engine believe
it is compiling the file \textit{target}
whose content is specified as the latter parameter.
The provided code then forwards the processing to
\textit{main} or \textit{dest} as described in \secref{sec:forward}.

%%%%%%%%%%%%%%%%%%%%%%%%%%%%%%%%%%%%%%%%%%%%%%%%%%%%%%%%%%%%%%%%%%%%%%%%%%%%%%%%
\subsection{Include by Input}
\label{sec:input}

Including child documents by |\include| has some restrictions by design.
Most notably, the content of a child document always occupies
its own set of pages; pages cannot be shared between child documents.
Usually, this behaviour makes perfect sense
because each child document contain an essential part of the document.
However, in some situations it may be desirable to compose
a document from a collection of parts
without having mandatory page breaks between then.
For this case, the package
provides a mechanism to include parts
by |\input| which can also be processed individually.
However, by construction this mechanism
requires manual handling of the content to be output.

%%%%%%%%%%%%%%%%%%%%%%%%%%%%%%%%%%%%%%%%
\DescribeMacro{\ifchilddocmanual}
The main file should be prepared as usual, see \secref{sec:include}.
However, the document body must make a distinction
between processing of an individual part and of the main document, e.g.:
%
\begin{center}
\begin{tabular}{l}
|\ifchilddocmanual|\\
|\input{\childdocname}|\\
|\||else|\\
\textit{document body with }|\input{|\textit{part}|}|\\
|\||fi|
\end{tabular}
\end{center}
%
The conditional |\ifchilddocmanual| is true whenever
a part to be included by |\input| is being compiled,
and the name of the part is stored in |\childdocname|.

%%%%%%%%%%%%%%%%%%%%%%%%%%%%%%%%%%%%%%%%
\DescribeMacro{\childdocby}
Each part to be included by |\input| should start with:
%
\begin{center}
\begin{tabular}{l}
|\input{childdoc.def}|\\
|\childdocby{|\textit{main}|}|\\
\end{tabular}
\end{center}
%
The directive |\childdocby| is similar to |\childdocof|
described in \secref{sec:include},
but the subsequent selection of content must be done manually.
To that end, both |\ifchilddoc| and |\ifchilddocmanual|
will be true upon processing of a part,
and the name of the part is stored in |\childdocname|.
Note that |\jobname| will be set to the filename of the current part
so that each part receives an individual |.aux| file
that does not interfere with the |.aux| file(s) of the main document.
This behaviour can be altered by the alternative form
|\childdocby[*]{|\textit{main}|}| (with a non-empty optional argument)
which uses the |.aux| file of the main document
by setting |\jobname| to \textit{main}.

%%%%%%%%%%%%%%%%%%%%%%%%%%%%%%%%%%%%%%%%%%%%%%%%%%%%%%%%%%%%%%%%%%%%%%%%%%%%%%%%
\subsection{Driver Development}
\label{sec:driver}

The \textsf{childdoc} mechanism can also be use for the development
of definition files such as \LaTeX{} styles or classes.
This case differs from the above setup with multiple parts
included by |\include| in that no |\includeonly| should be invoked.
This can be achieved by starting the include file
(before |\ProvidesPackage|) with:
%
\begin{center}
\begin{tabular}{l}
|\input{childdoc.def}|\\
|\childdocforward{|\textit{main}|}|\\
\end{tabular}
\end{center}
%
or alternatively with:
%
\begin{center}
\begin{tabular}{l}
|\input{childdoc.def}|\\
|\childdocby{|\textit{main}|}|\\
\end{tabular}
\end{center}
%
Both forms have slightly different effects as described above.
The main file is prepared as usual, see \secref{sec:include}.

%%%%%%%%%%%%%%%%%%%%%%%%%%%%%%%%%%%%%%%%%%%%%%%%%%%%%%%%%%%%%%%%%%%%%%%%%%%%%%%%
\subsection{Legacy Detection}
\label{sec:detection}

The directive |\childdocmain| in the main file can detect
whether the complete document or merely a child is to be compiled
even without using the directive |\childdocof|.
This method is deprecated because it is less robust
and there is no compelling reason to use it;
it is merely provided for backward compatibility
and it may be removed in future versions.

If the detection mechanism is to be used,
it is mandatory to correctly specify
the filename of the main file as the argument of |\childdocmain|:
%
\begin{center}
\begin{tabular}{l}
|\input{childdoc.def}|\\
|\childdocmain{|\textit{main}|}|\\
\end{tabular}
\end{center}
%
If |\jobname| does not match the argument \textit{main} of |\childdocmain|,
it is assumed that |\jobname| points to the child file to be compiled.
When using |\childdocmain| with the main file specified as argument,
it suffices to start a child file
with just |\input{|\textit{main}|}|
without loading of the package and using |\childdocof|.
If instead all processing is done
with the appropriate \textsf{childdoc} directives,
the argument of \textit{main} of |\childdocmain| can be empty.

An alternative version of the command line processing described
in \secref{sec:commandline} using the detection mechanism reads:
%
\begin{center}
|... -jobname "|\textit{target}|" "|[\textit{flags}]%
[|\def\jobname{|\textit{dest}|}|]|\input{|\textit{main}|}"|
\end{center}

%%%%%%%%%%%%%%%%%%%%%%%%%%%%%%%%%%%%%%%%%%%%%%%%%%%%%%%%%%%%%%%%%%%%%%%%%%%%%%%%
\subsection{Manual Code}
\label{sec:manual}

In case one cannot be certain whether the definitions file |childdoc.def|
is installed on the target \TeX{} distribution
and one prefers not to ship it,
it is conceivable to paste a few relevant commands into the sources.

To that end, drop all statements |\input{childdoc.def}|
and perform the replacements as outlined below.
Instead of |\childdocmain{|\textit{main}|}| add the following code
to the top of the main file:
%
\begin{center}
\begin{tabular}{l}
|\||ifdefined\childdocname\endinput\||fi\newif\ifchilddoc|\\
|\edef\childdocname{\scantokens\expandafter{\jobname\noexpand}}|\\
|\def\childdocmain{|\textit{main}|}\||ifx\childdocmain\childdocname\||else|\\
|\childdoctrue\includeonly{\childdocname}\let\jobname\childdocmain\||fi|\\
\end{tabular}
\end{center}
%
Instead of |\childdocof{|\textit{main}|}| just include the main file
at the top of each child file:
%
\begin{center}
|\input{|\textit{main}|}|
\end{center}
%
A simple redirection |\childdocforward{|\textit{dest}|}| is achieved by:
%
\begin{center}
|\def\jobname{|\textit{dest}|}\input{\jobname}|
\end{center}
%
The redirection with prefix
|\childdocforwardprefix[|\textit{prefix}|]{|\textit{dest}|}|
is accomplished by:
%
\begin{center}
\begin{tabular}{l}
|{\edef\jobname{\scantokens\expandafter{\jobname\noexpand}}|\\
|\def\redirectjob |\textit{prefix}|#1~~~{\gdef\jobname{|\textit{dest}|#1}}|\\
|\expandafter\redirectjob\jobname~~~}\input{\jobname}|
\end{tabular}
\end{center}

In an alternative approach,
child documents can be compiled by a specific command line
without additional code or specific definitions:
%
\begin{center}
|... -jobname "|\textit{target}|" "|[\textit{flags}]%
|\includeonly{|\textit{dest}|}\input{|\textit{main}|}"|
\end{center}
%

%%%%%%%%%%%%%%%%%%%%%%%%%%%%%%%%%%%%%%%%%%%%%%%%%%%%%%%%%%%%%%%%%%%%%%%%%%%%%%%%
%%%%%%%%%%%%%%%%%%%%%%%%%%%%%%%%%%%%%%%%%%%%%%%%%%%%%%%%%%%%%%%%%%%%%%%%%%%%%%%%
\section{Information}

%%%%%%%%%%%%%%%%%%%%%%%%%%%%%%%%%%%%%%%%%%%%%%%%%%%%%%%%%%%%%%%%%%%%%%%%%%%%%%%%
\subsection{Copyright}

Copyright \copyright{} 2017--2018 Niklas Beisert

This work may be distributed and/or modified under the
conditions of the \LaTeX{} Project Public License, either version 1.3
of this license or (at your option) any later version.
The latest version of this license is in
  \url{http://www.latex-project.org/lppl.txt}
and version 1.3 or later is part of all distributions of \LaTeX{}
version 2005/12/01 or later.

This work has the LPPL maintenance status `maintained'.

The Current Maintainer of this work is Niklas Beisert.

This work consists of the files |README.txt|, |childdoc.ins| and |childdoc.dtx|
as well as the derived files |childdoc.def|, |cdocsamp.tex|
with |cdocsch1.tex|, |cdocsch2.tex|, |cdocspt3.tex|, |cdocspt4.tex|,
|cdocsdrf.tex|, |cdocsfn1.tex|, |cdocsfn2.tex|
as well as |childdoc.pdf|.

%%%%%%%%%%%%%%%%%%%%%%%%%%%%%%%%%%%%%%%%%%%%%%%%%%%%%%%%%%%%%%%%%%%%%%%%%%%%%%%%
\subsection{Files and Installation}

The package consists of the files:
%
\begin{center}
\begin{tabular}{ll}
    |README.txt|   & readme file \\
    |childdoc.ins| & installation file \\
    |childdoc.dtx| & source file \\
    |childdoc.def| & definition file \\
    |cdocsamp.tex| & sample main file \\
    |cdocsch1.tex| & sample include file \\
    |cdocsch2.tex| & sample include file \\
    |cdocspt3.tex| & sample part file \\
    |cdocspt4.tex| & sample part file \\
    |cdocsdrf.tex| & sample redirection file \\
    |cdocsfn1.tex| & sample redirection file \\
    |cdocsfn2.tex| & sample redirection file \\
    |childdoc.pdf| & manual
\end{tabular}
\end{center}
%
The distribution consists of the files
|README.txt|, |childdoc.ins| and |childdoc.dtx|.
%
\begin{itemize}
\item
Run (pdf)\LaTeX{} on |childdoc.dtx|
to compile the manual |childdoc.pdf| (this file).
\item
Run \LaTeX{} on |childdoc.ins| to create the definitions file |childdoc.def|
and the sample |cdocsamp.tex| with include files
|cdocsch1.tex|, |cdocsch2.tex|, |cdocspt3.tex|, |cdocspt4.tex|,
|cdocsdrf.tex|, |cdocsfn1.tex|, |cdocsfn2.tex|.
Then copy the file |childdoc.def| to an appropriate directory of your \LaTeX{}
distribution, e.g.\ \textit{texmf-root}|/tex/latex/childdoc|.
\end{itemize}

%%%%%%%%%%%%%%%%%%%%%%%%%%%%%%%%%%%%%%%%%%%%%%%%%%%%%%%%%%%%%%%%%%%%%%%%%%%%%%%%
\subsection{Related CTAN Packages}

There are several other packages which offer a similar functionality:
%
\begin{itemize}
\item
The packages
\href{http://ctan.org/pkg/docmute}{\textsf{docmute}},
\href{http://ctan.org/pkg/includex}{\textsf{includex}} and
\href{http://ctan.org/pkg/standalone}{\textsf{standalone}}
provide commands to include only the document body of
a child file thus allowing both files to be compiled individually.
\item
The packages \href{http://ctan.org/pkg/subdocs}{\textsf{subdocs}}
and \href{http://ctan.org/pkg/subfiles}{\textsf{subfiles}}
provide structures in which the main and child documents can be
encapsulated and allowing them to be compiled individually.
The inclusion mechanism is different from the conventional |\include|.
\item
The package \href{http://ctan.org/pkg/combine}{\textsf{combine}}
is an elaborate solution to combine several documents into one.
\end{itemize}
%
See also the CTAN topic \href{http://ctan.org/topic/subdocs}{\textsf{subdocs}}
for further related packages.
The present package differs from the above solutions in that
a document structure constructed with the conventional |\include| mechanism
just needs two extra commands at the top of every file
such that all constituent files can be compiled individually.

%%%%%%%%%%%%%%%%%%%%%%%%%%%%%%%%%%%%%%%%%%%%%%%%%%%%%%%%%%%%%%%%%%%%%%%%%%%%%%%%
%\subsection{Feature Suggestions}
%
%The following is a list of features which may be useful for future
%versions of this package:
%%
%\begin{itemize}
%\item
%\ldots
%\end{itemize}

%%%%%%%%%%%%%%%%%%%%%%%%%%%%%%%%%%%%%%%%%%%%%%%%%%%%%%%%%%%%%%%%%%%%%%%%%%%%%%%%
\subsection{Revision History}

%%%%%%%%%%%%%%%%%%%%%%%%%%%%%%%%%%%%%%%%
\paragraph{v2.0:} 2018/12/30

\begin{itemize}
\item
immediate forward processing
\item
added |\childdocby| mechanism
\item
manual restructured
\end{itemize}

%%%%%%%%%%%%%%%%%%%%%%%%%%%%%%%%%%%%%%%%
\paragraph{v1.6:} 2018/01/17

\begin{itemize}
\item
application for development of include files
\item
corrections to manual
\end{itemize}

%%%%%%%%%%%%%%%%%%%%%%%%%%%%%%%%%%%%%%%%
\paragraph{v1.5:} 2017/05/21

\begin{itemize}
\item
more complete structuring introduced
\item
|\childdocof| introduced
\item
|\childdoc| renamed to |\childdocmain|
\item
|\childredirect| renamed to |\childdocforward| and |\childdocforwardprefix|
and functionality expanded
\end{itemize}

%%%%%%%%%%%%%%%%%%%%%%%%%%%%%%%%%%%%%%%%
\paragraph{v1.0:} 2017/04/27

\begin{itemize}
\item
manual and install package
\item
first version published on CTAN
\end{itemize}

%%%%%%%%%%%%%%%%%%%%%%%%%%%%%%%%%%%%%%%%
\paragraph{v0.6:} 2017/04/26

\begin{itemize}
\item
redirection mechanism added
\end{itemize}

%%%%%%%%%%%%%%%%%%%%%%%%%%%%%%%%%%%%%%%%
\paragraph{v0.5:} 2017/04/26

\begin{itemize}
\item
functionality in definition file
\end{itemize}


%%%%%%%%%%%%%%%%%%%%%%%%%%%%%%%%%%%%%%%%%%%%%%%%%%%%%%%%%%%%%%%%%%%%%%%%%%%%%%%%
%%%%%%%%%%%%%%%%%%%%%%%%%%%%%%%%%%%%%%%%%%%%%%%%%%%%%%%%%%%%%%%%%%%%%%%%%%%%%%%%
%%%%%%%%%%%%%%%%%%%%%%%%%%%%%%%%%%%%%%%%%%%%%%%%%%%%%%%%%%%%%%%%%%%%%%%%%%%%%%%%
\appendix

\settowidth\MacroIndent{\rmfamily\scriptsize 000\ }

 \DocInput{childdoc.dtx}

\end{document}
%</driver>
% \fi
%
% %%%%%%%%%%%%%%%%%%%%%%%%%%%%%%%%%%%%%%%%%%%%%%%%%%%%%%%%%%%%%%%%%%%%%%%%%%%%%%
% %%%%%%%%%%%%%%%%%%%%%%%%%%%%%%%%%%%%%%%%%%%%%%%%%%%%%%%%%%%%%%%%%%%%%%%%%%%%%%
% \section{Sample}
%\iffalse
%<*samplemain>
%\fi
%
% The following presents a sample document
% with two chapters, two parts, a title page,
% a compile flag as well as three forwarding files to set the flag.
% It consists of eight |.tex| files:
% \begin{center}
% \begin{tabular}{ll}
% |cdocsamp.tex|&main file\\
% |cdocsch1.tex|&include file for chapter 1\\
% |cdocsch2.tex|&include file for chapter 2\\
% |cdocspt3.tex|&include file for part 3\\
% |cdocspt4.tex|&include file for part 4\\
% |cdocsdrf.tex|&forwarding file for main file in draft mode\\
% |cdocsfi1.tex|&forwarding file for final version of chapter 1\\
% |cdocsfi2.tex|&forwarding file for final version of chapter 2\\
% \end{tabular}
% \end{center}
% Each of the eight files can be compiled directly by the \LaTeX{} compiler.
%
% %%%%%%%%%%%%%%%%%%%%%%%%%%%%%%%%%%%%%%
% \paragraph{Main File.}
%
% The main file is called |cdocsamp.tex|.
%
% Load the \textsf{childdoc} definitions and
% declare the filename for the main document:
%    \begin{macrocode}
\input{childdoc.def}
\childdocmain{}
%    \end{macrocode}

% Optional override for |\version| flag:
%    \begin{macrocode}
%%\ifchilddoc\else\providecommand{\version}{draft}\fi
%    \end{macrocode}

% Define the default values for the |\version| flag
% (|final| for the main file and |draft| for childs):
%    \begin{macrocode}
\ifchilddoc
\providecommand{\version}{draft}
\else
\providecommand{\version}{final}
\fi
%    \end{macrocode}

% Load the standard document class:
%    \begin{macrocode}
\documentclass[12pt]{article}
%    \end{macrocode}

% Start the document body:
%    \begin{macrocode}
\begin{document}
%    \end{macrocode}

% Declare a title page.
% Print title, part of document being processed and version flag:
%    \begin{macrocode}
\addtocounter{page}{-1}
\begin{center}
{\LARGE\bfseries{}childdoc example\par}
\vspace{1cm}
\ifchilddoc
\ifchilddocmanual part\else chapter\fi:
`\childdocname' of `\childdocjob'\par
\else
main document: `\childdocjob'\par
\fi
version: \version\par
\end{center}
\newpage
%    \end{macrocode}

% Manually include selected file,
% otherwise process as usual:
%    \begin{macrocode}
\ifchilddocmanual
\section*{part `\childdocname'}
\input{\childdocname}
\else
%    \end{macrocode}

% Include the two chapters:
%    \begin{macrocode}
\include{cdocsch1}
\include{cdocsch2}
%    \end{macrocode}

% Include the two parts unless only chapters should be displayed:
%    \begin{macrocode}
\ifchilddoc\else
\section{part three}
\input{cdocspt3}
\section{part four}
\input{cdocspt4}
\fi
%    \end{macrocode}

% Process as usual until here:
%    \begin{macrocode}
\fi
%    \end{macrocode}

% End of document body:
%    \begin{macrocode}
\end{document}
%    \end{macrocode}
%\iffalse
%</samplemain>
%\fi
%
% %%%%%%%%%%%%%%%%%%%%%%%%%%%%%%%%%%%%%%
% \paragraph{Chapter Include Files.}
%
% The include files are called |cdocsch1.tex| and |cdocsch2.tex|.
%
%\iffalse
%<*samplechap1|samplechap2>
%\fi

% Optional override for |\version| flag:
%    \begin{macrocode}
%%\providecommand{\version}{final}
%    \end{macrocode}

% Include the main document:
%    \begin{macrocode}
\input{childdoc.def}
\childdocof{cdocsamp}
%    \end{macrocode}

%\iffalse
%</samplechap1|samplechap2>
%\fi
%
%\iffalse
%<*samplechap1>
%\fi
% Some text for chapter 1:
%    \begin{macrocode}
\section{one}
some text in chapter one
%    \end{macrocode}

%\iffalse
%</samplechap1>
%\fi
% Some text for chapter 2:
%\iffalse
%<*samplechap2>
%\fi
%    \begin{macrocode}
\section{two}
more text in chapter two
%    \end{macrocode}

%\iffalse
%</samplechap2>
%\fi
%
% %%%%%%%%%%%%%%%%%%%%%%%%%%%%%%%%%%%%%%
% \paragraph{Part Include Files.}
%
% The include files are called |cdocspt3.tex| and |cdocspt4.tex|.
%
%\iffalse
%<*samplepart3|samplepart4>
%\fi

% Optional override for |\version| flag:
%    \begin{macrocode}
%%\providecommand{\version}{final}
%    \end{macrocode}

% Include the main document:
%    \begin{macrocode}
\input{childdoc.def}
\childdocby{cdocsamp}
%    \end{macrocode}

%\iffalse
%</samplepart3|samplepart4>
%\fi
%
%\iffalse
%<*samplepart3>
%\fi
% Some text for part 3:
%    \begin{macrocode}
some text in part three
%    \end{macrocode}

%\iffalse
%</samplepart3>
%\fi
% Some text for part 4:
%\iffalse
%<*samplepart4>
%\fi
%    \begin{macrocode}
more text in part four
%    \end{macrocode}

%\iffalse
%</samplepart4>
%\fi
%
% %%%%%%%%%%%%%%%%%%%%%%%%%%%%%%%%%%%%%%
% \paragraph{Forwarding for a Complete Draft.}
%
% The following forwarding file |cdocsdrf.tex|
% compiles the main document in draft mode:
%\iffalse
%<*sampledraft>
%\fi
%    \begin{macrocode}
\def\version{draft}
\input{childdoc.def}
\childdocforward{cdocsamp}
%    \end{macrocode}

%\iffalse
%</sampledraft>
%\fi
%
% %%%%%%%%%%%%%%%%%%%%%%%%%%%%%%%%%%%%%%
% \paragraph{Forwarding for Final Version of the Chapters.}
%
% The following forwarding files |cdocsfn1.tex| and |cdocsfn2.tex|
% (with identical content)
% compile the final versions of the child documents
% |cdocsch1.tex| and |cdocsch2.tex|, respectively:
%\iffalse
%<*samplefinal>
%\fi
%    \begin{macrocode}
\def\version{final}
\input{childdoc.def}
\childdocforwardprefix[cdocsamp]{cdocsfn}{cdocsch}
%    \end{macrocode}

%\iffalse
%</samplefinal>
%\fi
%
% %%%%%%%%%%%%%%%%%%%%%%%%%%%%%%%%%%%%%%
% \paragraph{Command Line Processing.}
%
% The following three command lines generate the output files
% |cdocscld|, |cdocscl1| and |cdocscl2|
% which should be identical to
% |cdocsdrf|, |cdocsch1| and |cdocsfn2|, respectively:
% \begin{center}
% \begin{tabular}{l}
% |latex -jobname cdocscld \|\\
% |  "\def\version{draft}\input{childdoc.def}\childdocforward{cdocsamp}"|\\
% |latex -jobname cdocscl1 \|\\
% |  "\input{childdoc.def}\childdocforward[cdocsamp]{cdocsch1}"|\\
% |latex -jobname cdocscl2 \|\\
% |  "\def\version{final}\input{childdoc.def}\childdocforward{cdocsch2}"|
% \end{tabular}
% \end{center}
% Note that the trailing backslash on each first line
% merely continues the input to the second line
% (for convenient cut ant paste).
% Furthermore, the command |latex| can be replaced by any
% of its alternative versions such as |pdflatex|.
%
% %%%%%%%%%%%%%%%%%%%%%%%%%%%%%%%%%%%%%%%%%%%%%%%%%%%%%%%%%%%%%%%%%%%%%%%%%%%%%%
% %%%%%%%%%%%%%%%%%%%%%%%%%%%%%%%%%%%%%%%%%%%%%%%%%%%%%%%%%%%%%%%%%%%%%%%%%%%%%%
% \section{Implementation}
%\iffalse
%<*package>
%\fi
%
% This section describes the definitions file |childdoc.def|.

% The definitions cannot be loaded using |\usepackage| or |\RequirePackage|
% which has a mechanism to prevent loading a style file more than once.
% When loading the definitions by means of |\input|
% multiple instances have to be prevented manually:
%\iffalse
%This code needs to be before the `\ProvidesFile' directive
%which is defined at the beginning of this file.
%Therefore it is also placed there and commented out here.
%</package>
%<*discard>
%\fi
%    \begin{macrocode}
\ifdefined\childdocmain\endinput\fi
%    \end{macrocode}
%\iffalse
%</discard>
%<*package>
%\fi
%
% \macro{\ifchilddoc}
% \macro{\ifchilddocmanual}
% The conditional |\ifchilddoc| tells whether a
% child (true) or main (false) document is being compiled.
% The conditional |\ifchilddocmanual| tells whether
% the |\includeonly| mechanism is used (false) or
% the selection of child files must be performed manually (true).
% The definitions initialise to false:
%    \begin{macrocode}
\newif\ifchilddoc
\newif\ifchilddocmanual
%    \end{macrocode}

% \macro{\childdocname}
% \macro{\childdocjob}
% The macro |\childdocname| stores the name of the main document
% to be compiled. The macro |\childdocjob| stores the name of
% the document on which the \LaTeX{} compiler was originally invoked.
% The content of |\jobname| cannot be compared
% to filenames specified in the source due to different catcodes.
% The following code rescans |\jobname|, stores the result
% in |\childdocname| and saves a copy in |\childdocjob|:
%    \begin{macrocode}
\edef\childdocname{\scantokens\expandafter{\jobname\noexpand}}
\let\childdocjob\childdocname
%    \end{macrocode}

% \macro{\childdocdisable}
% The macro |\childdocdisable| prevents the main file
% from being processed more than once.
% At this stage, the main document command |\childdocmain|
% is assumed to be called once again where it should do nothing.
% Any subsequent call to it should prevent
% a secondary processing of the main document
% It overwrites the forwarding commands
% |\childdocof| and |\childdocforward|
% with empty macros to prevent further inclusions of the main document:
%    \begin{macrocode}
\newcommand{\childdocdisable}
{
  \renewcommand{\childdocmain}[1]{\renewcommand{\childdocmain}[1]{\endinput}}
  \renewcommand{\childdocof}[1]{}
  \renewcommand{\childdocby}[2][]{}
  \renewcommand{\childdocforward}[2][]{}
  \renewcommand{\childdocdisable}{}
}
%    \end{macrocode}

% \macro{\childdocmain}
% The macro |\childdocmain| is to be called at the top of the main file
% with nothing or the main filename (without extension) as argument.
% First, it breaks loops.
% If the argument is not empty and does not match |\childdocname|
% (which is set by the first inclusion of |childdoc.def|),
% |\ifchilddoc| is set to true, |\includeonly| is applied to the child file
% and |\jobname| is set to the main file
% (for proper handling of |.aux| files):
%    \begin{macrocode}
\newcommand{\childdocmain}[1]
{
  \childdocdisable\childdocmain{}
  \if?#1?\else
    \begingroup
      \def\childdoctmp{#1}
      \ifx\childdoctmp\childdocname
        \def\childdoctmp{}
      \else
        \def\childdoctmp
        {
          \childdoctrue
          \includeonly{\childdocname}
          \def\childdocjob{#1}
          \def\jobname{#1}
        }
      \fi
      \expandafter
    \endgroup
    \childdoctmp
  \fi
}
%    \end{macrocode}

% \macro{\childdocof}
% The command |\childdocof| redirects
% compilation to the main file |#1|.
%    \begin{macrocode}
\newcommand{\childdocof}[1]
{
  \childdocdisable
  \childdoctrue
  \includeonly{\childdocname}
  \def\jobname{#1}
  \def\childdocjob{#1}
  \input{#1}
}
%    \end{macrocode}

% \macro{\childdocby}
% The command |\childdocby| ....
%    \begin{macrocode}
\newcommand{\childdocby}[2][]
{
  \childdocdisable
  \childdoctrue
  \childdocmanualtrue
  \if?#1?\else
    \def\jobname{#2}
  \fi
  \def\childdocjob{#2}
  \input{#2}
  \endinput
}
%    \end{macrocode}

% \macro{\childdocforward}
% The command |\childdocforward| redirects
% compilation to the main file or
% (if the optional argument is given) a child file.
% Parameters are set as if the main file
% or a child file starting with |\childdocof| was compiled.
% Then compilation is handed over to the main file:
%    \begin{macrocode}
\newcommand{\childdocforward}[2][]
{
  \begingroup
    \if?#1?
      \def\childdoctmp
      {
        \def\childdocname{#2}
        \def\childdocjob{#2}
        \def\jobname{#2}
        \input{#2}
        \endinput
      }
    \else
      \def\childdoctmp
      {
        \childdocdisable
        \def\childdocname{#2}
        \childdoctrue
        \includeonly{#2}
        \def\childdocjob{#1}
        \def\jobname{#1}
        \input{#1}
        \endinput
      }
    \fi
    \expandafter
  \endgroup
  \childdoctmp
}
%    \end{macrocode}

% \macro{\childdocforwardprefix}
% The command |\childdocforwardprefix| redirects
% compilation to the main or a child file by means of a pattern.
% The prefix |#1| in the current filename is replaced by |#2|
% and the suffix of the current filename is kept
% (it is assumed that the filename does not contain the substring `|~~~|'
% which is used as a delimiter).
% Compilation is handed over to the new file by |\childdocforward|:
%    \begin{macrocode}
\newcommand{\childdocforwardprefix}[3][]
{
  \begingroup
    \def\childdocextract #2##1~~~{\def\childdoctmp{\childdocforward[#1]{#3##1}}}
    \expandafter\childdocextract\childdocname~~~
    \expandafter
  \endgroup
  \childdoctmp
}
%    \end{macrocode}

% \macro{\childdoc}
% The deprecated macro |\childdoc| is a legacy version of |\childdocmain|:
%    \begin{macrocode}
\newcommand{\childdoc}{\childdocmain}
%    \end{macrocode}

% \macro{\childdocredirect}
% The deprecated macro |\childdocredirect| is a legacy version
% of |\childdocforward| and |\childdocforwardprefix|:
%    \begin{macrocode}
\newcommand{\childdocredirect}[2][]
{
  \begingroup
    \if?#1?
      \def\childdoctmp{\childdocforward{#2}}
    \else
      \def\childdoctmp{\childdocforwardprefix{#1}{#2}}
    \fi
    \expandafter
  \endgroup
  \childdoctmp
}
%    \end{macrocode}

%\iffalse
%</package>
%\fi
%
\endinput

\childdocby{cdocsamp}
%    \end{macrocode}

%\iffalse
%</samplepart3|samplepart4>
%\fi
%
%\iffalse
%<*samplepart3>
%\fi
% Some text for part 3:
%    \begin{macrocode}
some text in part three
%    \end{macrocode}

%\iffalse
%</samplepart3>
%\fi
% Some text for part 4:
%\iffalse
%<*samplepart4>
%\fi
%    \begin{macrocode}
more text in part four
%    \end{macrocode}

%\iffalse
%</samplepart4>
%\fi
%
% %%%%%%%%%%%%%%%%%%%%%%%%%%%%%%%%%%%%%%
% \paragraph{Forwarding for a Complete Draft.}
%
% The following forwarding file |cdocsdrf.tex|
% compiles the main document in draft mode:
%\iffalse
%<*sampledraft>
%\fi
%    \begin{macrocode}
\def\version{draft}
% \iffalse
%
% childdoc.dtx Copyright (C) 2017-2018 Niklas Beisert
%
% This work may be distributed and/or modified under the
% conditions of the LaTeX Project Public License, either version 1.3
% of this license or (at your option) any later version.
% The latest version of this license is in
%   http://www.latex-project.org/lppl.txt
% and version 1.3 or later is part of all distributions of LaTeX
% version 2005/12/01 or later.
%
% This work has the LPPL maintenance status `maintained'.
%
% The Current Maintainer of this work is Niklas Beisert.
%
% This work consists of the files childdoc.dtx and childdoc.ins
% and the derived files childdoc.def and cdocsamp.tex with
% cdocsch1.tex, cdocsch2.tex, cdocsdrf.tex, cdocsfn1.tex, cdocsfn2.tex.
%
%<package>\ifdefined\childdocmain\endinput\fi
%<package>\ProvidesFile{childdoc.def}[2018/12/30 v2.0 child document driver]
%<samplemain>\ProvidesFile{cdocsamp.tex}[2018/12/30 v2.0 sample for childdoc]
%<*driver>
%\ProvidesFile{childdoc.drv}[2018/12/30 v2.0 childdoc reference manual file]
\PassOptionsToClass{10pt,a4paper}{article}
\documentclass{ltxdoc}

\usepackage[margin=35mm]{geometry}
\usepackage{hyperref}
\usepackage{hyperxmp}
\usepackage[usenames]{color}

\hypersetup{colorlinks=true}
\hypersetup{pdfstartview=FitH}
\hypersetup{pdfpagemode=UseNone}
\hypersetup{pdfsource={}}
\hypersetup{pdflang={en-UK}}
\hypersetup{pdfcopyright={Copyright 2017-2018 Niklas Beisert.
  This work may be distributed and/or modified under the
  conditions of the LaTeX Project Public License, either version 1.3
  of this license or (at your option) any later version.}}
\hypersetup{pdflicenseurl={http://www.latex-project.org/lppl.txt}}
\hypersetup{pdfcontactaddress={ETH Zurich, ITP, HIT K,
  Wolfgang-Pauli-Strasse 27}}
\hypersetup{pdfcontactpostcode={8093}}
\hypersetup{pdfcontactcity={Zurich}}
\hypersetup{pdfcontactcountry={Switzerland}}
\hypersetup{pdfcontactemail={nbeisert@itp.phys.ethz.ch}}
\hypersetup{pdfcontacturl={http://people.phys.ethz.ch/\xmptilde nbeisert/}}

\newcommand{\secref}[1]{\hyperref[#1]{section \ref*{#1}}}

\parskip1ex
\parindent0pt
\let\olditemize\itemize
\def\itemize{\olditemize\parskip0pt}

\begin{document}

\title{The \textsf{childdoc} Package}
\hypersetup{pdftitle={The childdoc Package}}
\author{Niklas Beisert\\[2ex]
  Institut f\"ur Theoretische Physik\\
  Eidgen\"ossische Technische Hochschule Z\"urich\\
  Wolfgang-Pauli-Strasse 27, 8093 Z\"urich, Switzerland\\[1ex]
  \href{mailto:nbeisert@itp.phys.ethz.ch}
  {\texttt{nbeisert@itp.phys.ethz.ch}}}
\hypersetup{pdfauthor={Niklas Beisert}}
\hypersetup{pdfsubject={Manual for the LaTeX2e Package childdoc}}
\date{30 December 2018, \textsf{v2.0}}
\maketitle

\begin{abstract}\noindent
\textsf{childdoc} is a \LaTeXe{} package
that enables the direct compilation
of document sections included by |\include|
to individual files.
\end{abstract}

\begingroup
\parskip0ex
\tableofcontents
\endgroup

%%%%%%%%%%%%%%%%%%%%%%%%%%%%%%%%%%%%%%%%%%%%%%%%%%%%%%%%%%%%%%%%%%%%%%%%%%%%%%%%
%%%%%%%%%%%%%%%%%%%%%%%%%%%%%%%%%%%%%%%%%%%%%%%%%%%%%%%%%%%%%%%%%%%%%%%%%%%%%%%%
\section{Introduction}

\LaTeX{} provides a mechanism to structure a large document (such as a book)
into a main file and several child files (containing the chapters)
using the |\include| command.
This mechanism is beneficial for documents
which span hundreds of pages in order to
make the source file(s) more manageable.
Moreover, compilation can be restricted to
selected child files by means of the |\includeonly| command.
The latter feature can be used to reduce the compilation time while editing
(this was significantly more useful in the earlier days of \LaTeX{})
or to generate a smaller document which is easier to navigate.
Another application of |\includeonly| is to generate
documents consisting of selected parts of the complete document.

However, there are a few drawbacks of the plain |\include| mechanism:
\begin{itemize}
\item
The child files cannot be compiled on their own,
they can only be compiled via the main file.
A naive editing environment
(such as a text editor with an option
to have the current file processed by \LaTeX)
may require one to switch to the main file before compiling;
attempting to compile the child file produces errors.
\item
The main file must be modified (each time)
to adjust the |\includeonly| command
to the present needs. This easily leaves the main file in a messy state.
\item
The generated document will always carry the filename
of the main document. This is inconvenient if
several child files are to be compiled and
to be kept for distribution.
\end{itemize}

The present package provides a simple interface
to make child files individually compilable by \LaTeX{}.
Compiling a child file then has the same effect as compiling
the main file with an |\includeonly| command
to select the appropriate child.
Moreover the generated document will carry the name of the child
rather than the main file.
This resolves all three above issues.

This feature is meant to make the editing of books,
thesis documents and lecture notes somewhat more convenient.
However, the package can also be used efficiently for
composing a series of documents (such as exercise sheets)
which are typically distributed individually.
It then assists the author in generating the individual documents
(potentially in different versions)
as well as a document containing the collected series.
Another application is in developing style files
or other kinds of included material
where compilation of the style file could redirect
to a sample or test file.

%%%%%%%%%%%%%%%%%%%%%%%%%%%%%%%%%%%%%%%%%%%%%%%%%%%%%%%%%%%%%%%%%%%%%%%%%%%%%%%%
%%%%%%%%%%%%%%%%%%%%%%%%%%%%%%%%%%%%%%%%%%%%%%%%%%%%%%%%%%%%%%%%%%%%%%%%%%%%%%%%
\section{Usage}

First of all, the package \textsf{childdoc} is \emph{not} a standard
\LaTeXe{} |.sty| style file! Therefore it needs to be invoked in
a non-standard way.

%%%%%%%%%%%%%%%%%%%%%%%%%%%%%%%%%%%%%%%%%%%%%%%%%%%%%%%%%%%%%%%%%%%%%%%%%%%%%%%%
\subsection{Included Files}
\label{sec:include}

%%%%%%%%%%%%%%%%%%%%%%%%%%%%%%%%%%%%%%%%
\DescribeMacro{\childdocmain}
To use the package, add the commands
\begin{center}
\begin{tabular}{l}
|\input{childdoc.def}|\\
|\childdocmain{}|\\
\end{tabular}
\end{center}
at the very top of the main \LaTeX{} file,
in particular \emph{before} the |\documentclass| statement!
The argument of |\childdocmain| should be left empty
(but it must be present).

%%%%%%%%%%%%%%%%%%%%%%%%%%%%%%%%%%%%%%%%
\DescribeMacro{\childdocof}
Furthermore, add the commands
\begin{center}
\begin{tabular}{l}
|\input{childdoc.def}|\\
|\childdocof{|\textit{main}|}|\\
\end{tabular}
\end{center}
at the top of every child file \textit{child}
which is included by |\include{|\textit{child}|}|
from within the main file
(or at least for those files to be compiled individually).
The argument \textit{main} must be the filename of the main file.

There are a couple of
considerations in setting up the main and child documents:

%%%%%%%%%%%%%%%%%%%%%%%%%%%%%%%%%%%%%%%%
\paragraph{Restrictions.}

Please note the following restrictions:
\begin{itemize}
\item
|\childdocmain| must be called with one argument \textit{main}
to ensure compatibility with earlier version of the package.
It must either be empty (|\childdocmain{}|)
or precisely match the filename of the main file in which it is specified.
See \secref{sec:detection} for further information.
\item
The filename \textit{main} must be specified without the |.tex| extension.
\item
The filename \textit{main} is case sensitive
(even in case-insensitive file systems)
due to internal string comparison.
\item
The argument \textit{main} should be fully expanded, it cannot be a macro.
\item
Subdirectories and special characters should be avoided in filenames.
\item
The command |\childdocmain{|\textit{main}|}| must be followed by a whitespace.
It should not be followed immediately by another command
or by a comment mark `|%|'.
This is because the \TeX{} parser reads the token immediately following
the argument of |\childdocmain| and puts it
at the beginning of every child section;
however, a white\-space is ignored.
\end{itemize}

%%%%%%%%%%%%%%%%%%%%%%%%%%%%%%%%%%%%%%%%
\paragraph{Content of Main File.}

It is advisable to place all content in the child files included by |\include|.
Any output contained in the main file will appear in all child documents
unless suppressed manually;
it cannot be suppressed automatically by the |\includeonly| directive
and thus should normally be avoided.
A method to include some content in the main file
by means of conditional processing is described in \secref{sec:conditional}.

%%%%%%%%%%%%%%%%%%%%%%%%%%%%%%%%%%%%%%%%
\paragraph{Page Numbering.}

When only a part of the document is compiled,
the appropriate numbering of pages
(as well as other status parameters)
is determined from the |.aux| files.
The latter contain information from previous passes.
However this information needs to propagate through
all intermediate child documents.
Therefore the page numbering in child documents may well
be inconsistent until the complete document is compiled at least once.

A useful (if unconventional) way to always ensure a consistent
page numbering is to restart the numbering in each child document
and denote the pages by `\textit{child}|.|\textit{page}'
where \textit{child} represents the chapter/section number of the child file.
This can be achieved by the command
|\numberwithin{page}{|\textit{child}|}|
of the \textsf{amsmath} package
where \textit{child} can be |chapter| or |section|
depending on the chosen structuring.
Alternatively, one can modify the macro |\thepage| appropriately
and reset the counter |page| at the start of each child file.

%%%%%%%%%%%%%%%%%%%%%%%%%%%%%%%%%%%%%%%%%%%%%%%%%%%%%%%%%%%%%%%%%%%%%%%%%%%%%%%%
\subsection{Conditional Processing}
\label{sec:conditional}

The package provides a mechanism to compile different versions
of a document. To customise the versions further some conditional processing
can come in handy to distinguish which version is being compiled.
The package provides two macros to describe the compilation context:

%%%%%%%%%%%%%%%%%%%%%%%%%%%%%%%%%%%%%%%%
\DescribeMacro{\ifchilddoc}
The conditional |\ifchilddoc| distinguishes between the compilation of
child documents and the main document:
%
\begin{center}
|\ifchilddoc |\textit{child-code}| |[|\||else |\textit{main-code}]| \||fi|
\end{center}

%%%%%%%%%%%%%%%%%%%%%%%%%%%%%%%%%%%%%%%%
\DescribeMacro{\childdocname}
\DescribeMacro{\childdocjob}
The macro |\childdocname| contains the filename (without extension)
of the main or child file being processed.
Note that |\childdocjob| will always contain the name of the main file.

%%%%%%%%%%%%%%%%%%%%%%%%%%%%%%%%%%%%%%%%
\paragraph{Title Page.}

Conditional processing can be used to include a title or banner page
in the main document when proper precautions are taken.
Importantly, the code in the main file should ensure that the page counter
(as well as other status parameters which are stored in the |.aux| files)
takes the same value after the conditional processing.
Otherwise the page numbers may take divergent values
depending on which part is compiled.

For example, a title page could be declared by:
%
\begin{center}
\begin{tabular}{l}
|\ifchilddoc\||else|\\
|\addtocounter{page}{-1}|\\
\textit{code for title page}\\
|\newpage|\\
|\||fi|
\end{tabular}
\end{center}
%
A banner page for the child documents can be generated by:
%
\begin{center}
\begin{tabular}{l}
|\ifchilddoc|\\
|\addtocounter{page}{-1}|\\
\textit{code for banner page}\\
|\newpage|\\
|\||fi|
\end{tabular}
\end{center}
%
Here one could write a message such as:
\begin{center}
|This is the part \childdocname{} of \childdocjob{}.|
\end{center}

%%%%%%%%%%%%%%%%%%%%%%%%%%%%%%%%%%%%%%%%%%%%%%%%%%%%%%%%%%%%%%%%%%%%%%%%%%%%%%%%
\subsection{Flags}
\label{sec:flags}

The package makes it easy to generate different versions
of the main or child documents.
To this end compilation flags can be defined
and assigned different default values.
They will be particularly useful in conjunction
with the forwarding mechanism described in \secref{sec:forward}.

For example, it may be useful to have a flag |\version|
which can be set to |draft| or |final|.
The document source will contain some conditional code
depending on the value of |\version|.
Suppose further, the flag should default to |final| for the main file
and to |draft| for child files
which is a natural assignment for editing the document.
This is achieved by placing the following code
in the preamble of the main document
(below the |\childdocmain| directive):
%
\begin{center}
\begin{tabular}{l}
|\ifchilddoc|\\
|\providecommand{\version}{draft}|\\
|\||else|\\
|\providecommand{\version}{final}|\\
|\||fi|
\end{tabular}
\end{center}
%
The definition by |\providecommand| makes sure
that previous definitions are not overwritten.
Further statements |\providecommand{\version}{...}|
can thus be added before the above code to override it.

For the main file, one might add a line
(between |\childdocmain| and the above block)
%
\begin{center}
|%\ifchilddoc\||else\providecommand{\version}{draft}\||fi|
\end{center}
%
which can be uncommented to produce a draft version.
Likewise one can add a line to the very top of a child file
(above the |\childdocof{|\textit{main}|}| directive)
%
\begin{center}
|%\providecommand{\version}{final}|
\end{center}
%
which can be uncommented to produce the final version of this child document.

%%%%%%%%%%%%%%%%%%%%%%%%%%%%%%%%%%%%%%%%%%%%%%%%%%%%%%%%%%%%%%%%%%%%%%%%%%%%%%%%
\subsection{Forwarding}
\label{sec:forward}

Different versions of the main or child documents
using compilation flags as described in \secref{sec:flags}
can be (permanently) stored in different files
for convenient compilation, viewing and distribution.
To this end, the package defines a command
to pass on compilation to a different file:

%%%%%%%%%%%%%%%%%%%%%%%%%%%%%%%%%%%%%%%%
\DescribeMacro{\childdocforward}
The command |\childdocforward| redirects processing to
another source file:
%
\begin{center}
\begin{tabular}{l}
|\input{childdoc.def}|\\
|\childdocforward[|\textit{main}|]{|\textit{dest}|}|\\
\end{tabular}
\end{center}
%
The argument \textit{dest} is the destination file
(without extension).
It should be the main file or one of the child files.
Note that further \textsf{childdoc} directives
such as |\childdocof| and |\childdocforward|
in the indicated file will be processed in this form.
The optional argument \textit{main}
passes on directly to the main file \textit{main}
while pretending to compile the child \textit{dest}.
This form behaves as if \textit{dest}
issues |\childdocof{|\textit{main}|}| right away,
and no further \textsf{childdoc} directives will be processed.

%%%%%%%%%%%%%%%%%%%%%%%%%%%%%%%%%%%%%%%%
\DescribeMacro{\...prefix}
In the alternative form |\childdocforwardprefix|,
%
\begin{center}
\begin{tabular}{l}
|\input{childdoc.def}|\\
|\childdocforwardprefix[|\textit{main}|]{|\textit{prefix}|}{|\textit{dest}|}|
\end{tabular}
\end{center}
%
the destination file is determined by a pattern
depending on the current file:
To make this work, the current file must be called
`{\textit{prefix}\hspace{0.2em}\textit{suffix}}'
with \textit{prefix} matching precisely the argument.
Processing is then passed on to the file
`{\textit{dest}\hspace{0.2em}\textit{suffix}}'.
Surely, the same effect is achieved by
directly specifying the
argument `{\textit{dest}\hspace{0.2em}\textit{suffix}}'
in the first form.
However, that requires to set up a different file
for each child. With the alternative form of the command
all these files can have exactly the same content
which simplifies setting them up and maintaining them.

For example, the following file |draft.tex|
with a compilation flag |\version| as described in \secref{sec:flags}
compiles the main document as a draft:
%
\begin{center}
\begin{tabular}{l}
|\def\version{draft}|\\
|\input{childdoc.def}|\\
|\childdocforward{|\textit{main}|}|
\end{tabular}
\end{center}
%
Likewise, the following files |final|\textit{nn}|.tex|
compile the final version of the child document
|child|\textit{nn}|.tex|:
%
\begin{center}
\begin{tabular}{l}
|\def\version{final}|\\
|\input{childdoc.def}|\\
|\childdocforwardprefix{final}{child}|
\end{tabular}
\end{center}
%

Note that when several versions of a main file and/or of each child file
are to be generated, it may be convenient to set up a |Makefile| or
shell script to automatise the process.

%%%%%%%%%%%%%%%%%%%%%%%%%%%%%%%%%%%%%%%%%%%%%%%%%%%%%%%%%%%%%%%%%%%%%%%%%%%%%%%%
\subsection{Command Line Processing}
\label{sec:commandline}

The effect of redirection files can also be achieved by invoking
the \LaTeX{} compiler with a more elaborate command line.
Most conveniently this should be done as part
of a shell script or a |Makefile|.

When using \textsf{childdoc} in the main file, the following
command lines effectively perform a redirection
(note that depending on the shell being used,
backslashes may have to be doubled: `|\|' $\to$ `|\\|'):
%
\begin{center}
|... -jobname "|\textit{target}|" |\\|"|[\textit{flags}]%
|\input{childdoc.def}\childdocforward[|\textit{main}|]{|\textit{dest}|}"|
\end{center}
%
Here \textit{target} is the name of the output file,
\textit{main} is the name of the main file
and \textit{dest} is the name of the main or child file to be processed
(all filenames without extensions).
The optional argument \textit{main} can be omitted
if \textit{main} matches \textit{dest}.
Optionally, compilation \textit{flags} can be defined via |\def| commands.
This command line makes the \TeX{} engine believe
it is compiling the file \textit{target}
whose content is specified as the latter parameter.
The provided code then forwards the processing to
\textit{main} or \textit{dest} as described in \secref{sec:forward}.

%%%%%%%%%%%%%%%%%%%%%%%%%%%%%%%%%%%%%%%%%%%%%%%%%%%%%%%%%%%%%%%%%%%%%%%%%%%%%%%%
\subsection{Include by Input}
\label{sec:input}

Including child documents by |\include| has some restrictions by design.
Most notably, the content of a child document always occupies
its own set of pages; pages cannot be shared between child documents.
Usually, this behaviour makes perfect sense
because each child document contain an essential part of the document.
However, in some situations it may be desirable to compose
a document from a collection of parts
without having mandatory page breaks between then.
For this case, the package
provides a mechanism to include parts
by |\input| which can also be processed individually.
However, by construction this mechanism
requires manual handling of the content to be output.

%%%%%%%%%%%%%%%%%%%%%%%%%%%%%%%%%%%%%%%%
\DescribeMacro{\ifchilddocmanual}
The main file should be prepared as usual, see \secref{sec:include}.
However, the document body must make a distinction
between processing of an individual part and of the main document, e.g.:
%
\begin{center}
\begin{tabular}{l}
|\ifchilddocmanual|\\
|\input{\childdocname}|\\
|\||else|\\
\textit{document body with }|\input{|\textit{part}|}|\\
|\||fi|
\end{tabular}
\end{center}
%
The conditional |\ifchilddocmanual| is true whenever
a part to be included by |\input| is being compiled,
and the name of the part is stored in |\childdocname|.

%%%%%%%%%%%%%%%%%%%%%%%%%%%%%%%%%%%%%%%%
\DescribeMacro{\childdocby}
Each part to be included by |\input| should start with:
%
\begin{center}
\begin{tabular}{l}
|\input{childdoc.def}|\\
|\childdocby{|\textit{main}|}|\\
\end{tabular}
\end{center}
%
The directive |\childdocby| is similar to |\childdocof|
described in \secref{sec:include},
but the subsequent selection of content must be done manually.
To that end, both |\ifchilddoc| and |\ifchilddocmanual|
will be true upon processing of a part,
and the name of the part is stored in |\childdocname|.
Note that |\jobname| will be set to the filename of the current part
so that each part receives an individual |.aux| file
that does not interfere with the |.aux| file(s) of the main document.
This behaviour can be altered by the alternative form
|\childdocby[*]{|\textit{main}|}| (with a non-empty optional argument)
which uses the |.aux| file of the main document
by setting |\jobname| to \textit{main}.

%%%%%%%%%%%%%%%%%%%%%%%%%%%%%%%%%%%%%%%%%%%%%%%%%%%%%%%%%%%%%%%%%%%%%%%%%%%%%%%%
\subsection{Driver Development}
\label{sec:driver}

The \textsf{childdoc} mechanism can also be use for the development
of definition files such as \LaTeX{} styles or classes.
This case differs from the above setup with multiple parts
included by |\include| in that no |\includeonly| should be invoked.
This can be achieved by starting the include file
(before |\ProvidesPackage|) with:
%
\begin{center}
\begin{tabular}{l}
|\input{childdoc.def}|\\
|\childdocforward{|\textit{main}|}|\\
\end{tabular}
\end{center}
%
or alternatively with:
%
\begin{center}
\begin{tabular}{l}
|\input{childdoc.def}|\\
|\childdocby{|\textit{main}|}|\\
\end{tabular}
\end{center}
%
Both forms have slightly different effects as described above.
The main file is prepared as usual, see \secref{sec:include}.

%%%%%%%%%%%%%%%%%%%%%%%%%%%%%%%%%%%%%%%%%%%%%%%%%%%%%%%%%%%%%%%%%%%%%%%%%%%%%%%%
\subsection{Legacy Detection}
\label{sec:detection}

The directive |\childdocmain| in the main file can detect
whether the complete document or merely a child is to be compiled
even without using the directive |\childdocof|.
This method is deprecated because it is less robust
and there is no compelling reason to use it;
it is merely provided for backward compatibility
and it may be removed in future versions.

If the detection mechanism is to be used,
it is mandatory to correctly specify
the filename of the main file as the argument of |\childdocmain|:
%
\begin{center}
\begin{tabular}{l}
|\input{childdoc.def}|\\
|\childdocmain{|\textit{main}|}|\\
\end{tabular}
\end{center}
%
If |\jobname| does not match the argument \textit{main} of |\childdocmain|,
it is assumed that |\jobname| points to the child file to be compiled.
When using |\childdocmain| with the main file specified as argument,
it suffices to start a child file
with just |\input{|\textit{main}|}|
without loading of the package and using |\childdocof|.
If instead all processing is done
with the appropriate \textsf{childdoc} directives,
the argument of \textit{main} of |\childdocmain| can be empty.

An alternative version of the command line processing described
in \secref{sec:commandline} using the detection mechanism reads:
%
\begin{center}
|... -jobname "|\textit{target}|" "|[\textit{flags}]%
[|\def\jobname{|\textit{dest}|}|]|\input{|\textit{main}|}"|
\end{center}

%%%%%%%%%%%%%%%%%%%%%%%%%%%%%%%%%%%%%%%%%%%%%%%%%%%%%%%%%%%%%%%%%%%%%%%%%%%%%%%%
\subsection{Manual Code}
\label{sec:manual}

In case one cannot be certain whether the definitions file |childdoc.def|
is installed on the target \TeX{} distribution
and one prefers not to ship it,
it is conceivable to paste a few relevant commands into the sources.

To that end, drop all statements |\input{childdoc.def}|
and perform the replacements as outlined below.
Instead of |\childdocmain{|\textit{main}|}| add the following code
to the top of the main file:
%
\begin{center}
\begin{tabular}{l}
|\||ifdefined\childdocname\endinput\||fi\newif\ifchilddoc|\\
|\edef\childdocname{\scantokens\expandafter{\jobname\noexpand}}|\\
|\def\childdocmain{|\textit{main}|}\||ifx\childdocmain\childdocname\||else|\\
|\childdoctrue\includeonly{\childdocname}\let\jobname\childdocmain\||fi|\\
\end{tabular}
\end{center}
%
Instead of |\childdocof{|\textit{main}|}| just include the main file
at the top of each child file:
%
\begin{center}
|\input{|\textit{main}|}|
\end{center}
%
A simple redirection |\childdocforward{|\textit{dest}|}| is achieved by:
%
\begin{center}
|\def\jobname{|\textit{dest}|}\input{\jobname}|
\end{center}
%
The redirection with prefix
|\childdocforwardprefix[|\textit{prefix}|]{|\textit{dest}|}|
is accomplished by:
%
\begin{center}
\begin{tabular}{l}
|{\edef\jobname{\scantokens\expandafter{\jobname\noexpand}}|\\
|\def\redirectjob |\textit{prefix}|#1~~~{\gdef\jobname{|\textit{dest}|#1}}|\\
|\expandafter\redirectjob\jobname~~~}\input{\jobname}|
\end{tabular}
\end{center}

In an alternative approach,
child documents can be compiled by a specific command line
without additional code or specific definitions:
%
\begin{center}
|... -jobname "|\textit{target}|" "|[\textit{flags}]%
|\includeonly{|\textit{dest}|}\input{|\textit{main}|}"|
\end{center}
%

%%%%%%%%%%%%%%%%%%%%%%%%%%%%%%%%%%%%%%%%%%%%%%%%%%%%%%%%%%%%%%%%%%%%%%%%%%%%%%%%
%%%%%%%%%%%%%%%%%%%%%%%%%%%%%%%%%%%%%%%%%%%%%%%%%%%%%%%%%%%%%%%%%%%%%%%%%%%%%%%%
\section{Information}

%%%%%%%%%%%%%%%%%%%%%%%%%%%%%%%%%%%%%%%%%%%%%%%%%%%%%%%%%%%%%%%%%%%%%%%%%%%%%%%%
\subsection{Copyright}

Copyright \copyright{} 2017--2018 Niklas Beisert

This work may be distributed and/or modified under the
conditions of the \LaTeX{} Project Public License, either version 1.3
of this license or (at your option) any later version.
The latest version of this license is in
  \url{http://www.latex-project.org/lppl.txt}
and version 1.3 or later is part of all distributions of \LaTeX{}
version 2005/12/01 or later.

This work has the LPPL maintenance status `maintained'.

The Current Maintainer of this work is Niklas Beisert.

This work consists of the files |README.txt|, |childdoc.ins| and |childdoc.dtx|
as well as the derived files |childdoc.def|, |cdocsamp.tex|
with |cdocsch1.tex|, |cdocsch2.tex|, |cdocspt3.tex|, |cdocspt4.tex|,
|cdocsdrf.tex|, |cdocsfn1.tex|, |cdocsfn2.tex|
as well as |childdoc.pdf|.

%%%%%%%%%%%%%%%%%%%%%%%%%%%%%%%%%%%%%%%%%%%%%%%%%%%%%%%%%%%%%%%%%%%%%%%%%%%%%%%%
\subsection{Files and Installation}

The package consists of the files:
%
\begin{center}
\begin{tabular}{ll}
    |README.txt|   & readme file \\
    |childdoc.ins| & installation file \\
    |childdoc.dtx| & source file \\
    |childdoc.def| & definition file \\
    |cdocsamp.tex| & sample main file \\
    |cdocsch1.tex| & sample include file \\
    |cdocsch2.tex| & sample include file \\
    |cdocspt3.tex| & sample part file \\
    |cdocspt4.tex| & sample part file \\
    |cdocsdrf.tex| & sample redirection file \\
    |cdocsfn1.tex| & sample redirection file \\
    |cdocsfn2.tex| & sample redirection file \\
    |childdoc.pdf| & manual
\end{tabular}
\end{center}
%
The distribution consists of the files
|README.txt|, |childdoc.ins| and |childdoc.dtx|.
%
\begin{itemize}
\item
Run (pdf)\LaTeX{} on |childdoc.dtx|
to compile the manual |childdoc.pdf| (this file).
\item
Run \LaTeX{} on |childdoc.ins| to create the definitions file |childdoc.def|
and the sample |cdocsamp.tex| with include files
|cdocsch1.tex|, |cdocsch2.tex|, |cdocspt3.tex|, |cdocspt4.tex|,
|cdocsdrf.tex|, |cdocsfn1.tex|, |cdocsfn2.tex|.
Then copy the file |childdoc.def| to an appropriate directory of your \LaTeX{}
distribution, e.g.\ \textit{texmf-root}|/tex/latex/childdoc|.
\end{itemize}

%%%%%%%%%%%%%%%%%%%%%%%%%%%%%%%%%%%%%%%%%%%%%%%%%%%%%%%%%%%%%%%%%%%%%%%%%%%%%%%%
\subsection{Related CTAN Packages}

There are several other packages which offer a similar functionality:
%
\begin{itemize}
\item
The packages
\href{http://ctan.org/pkg/docmute}{\textsf{docmute}},
\href{http://ctan.org/pkg/includex}{\textsf{includex}} and
\href{http://ctan.org/pkg/standalone}{\textsf{standalone}}
provide commands to include only the document body of
a child file thus allowing both files to be compiled individually.
\item
The packages \href{http://ctan.org/pkg/subdocs}{\textsf{subdocs}}
and \href{http://ctan.org/pkg/subfiles}{\textsf{subfiles}}
provide structures in which the main and child documents can be
encapsulated and allowing them to be compiled individually.
The inclusion mechanism is different from the conventional |\include|.
\item
The package \href{http://ctan.org/pkg/combine}{\textsf{combine}}
is an elaborate solution to combine several documents into one.
\end{itemize}
%
See also the CTAN topic \href{http://ctan.org/topic/subdocs}{\textsf{subdocs}}
for further related packages.
The present package differs from the above solutions in that
a document structure constructed with the conventional |\include| mechanism
just needs two extra commands at the top of every file
such that all constituent files can be compiled individually.

%%%%%%%%%%%%%%%%%%%%%%%%%%%%%%%%%%%%%%%%%%%%%%%%%%%%%%%%%%%%%%%%%%%%%%%%%%%%%%%%
%\subsection{Feature Suggestions}
%
%The following is a list of features which may be useful for future
%versions of this package:
%%
%\begin{itemize}
%\item
%\ldots
%\end{itemize}

%%%%%%%%%%%%%%%%%%%%%%%%%%%%%%%%%%%%%%%%%%%%%%%%%%%%%%%%%%%%%%%%%%%%%%%%%%%%%%%%
\subsection{Revision History}

%%%%%%%%%%%%%%%%%%%%%%%%%%%%%%%%%%%%%%%%
\paragraph{v2.0:} 2018/12/30

\begin{itemize}
\item
immediate forward processing
\item
added |\childdocby| mechanism
\item
manual restructured
\end{itemize}

%%%%%%%%%%%%%%%%%%%%%%%%%%%%%%%%%%%%%%%%
\paragraph{v1.6:} 2018/01/17

\begin{itemize}
\item
application for development of include files
\item
corrections to manual
\end{itemize}

%%%%%%%%%%%%%%%%%%%%%%%%%%%%%%%%%%%%%%%%
\paragraph{v1.5:} 2017/05/21

\begin{itemize}
\item
more complete structuring introduced
\item
|\childdocof| introduced
\item
|\childdoc| renamed to |\childdocmain|
\item
|\childredirect| renamed to |\childdocforward| and |\childdocforwardprefix|
and functionality expanded
\end{itemize}

%%%%%%%%%%%%%%%%%%%%%%%%%%%%%%%%%%%%%%%%
\paragraph{v1.0:} 2017/04/27

\begin{itemize}
\item
manual and install package
\item
first version published on CTAN
\end{itemize}

%%%%%%%%%%%%%%%%%%%%%%%%%%%%%%%%%%%%%%%%
\paragraph{v0.6:} 2017/04/26

\begin{itemize}
\item
redirection mechanism added
\end{itemize}

%%%%%%%%%%%%%%%%%%%%%%%%%%%%%%%%%%%%%%%%
\paragraph{v0.5:} 2017/04/26

\begin{itemize}
\item
functionality in definition file
\end{itemize}


%%%%%%%%%%%%%%%%%%%%%%%%%%%%%%%%%%%%%%%%%%%%%%%%%%%%%%%%%%%%%%%%%%%%%%%%%%%%%%%%
%%%%%%%%%%%%%%%%%%%%%%%%%%%%%%%%%%%%%%%%%%%%%%%%%%%%%%%%%%%%%%%%%%%%%%%%%%%%%%%%
%%%%%%%%%%%%%%%%%%%%%%%%%%%%%%%%%%%%%%%%%%%%%%%%%%%%%%%%%%%%%%%%%%%%%%%%%%%%%%%%
\appendix

\settowidth\MacroIndent{\rmfamily\scriptsize 000\ }

 \DocInput{childdoc.dtx}

\end{document}
%</driver>
% \fi
%
% %%%%%%%%%%%%%%%%%%%%%%%%%%%%%%%%%%%%%%%%%%%%%%%%%%%%%%%%%%%%%%%%%%%%%%%%%%%%%%
% %%%%%%%%%%%%%%%%%%%%%%%%%%%%%%%%%%%%%%%%%%%%%%%%%%%%%%%%%%%%%%%%%%%%%%%%%%%%%%
% \section{Sample}
%\iffalse
%<*samplemain>
%\fi
%
% The following presents a sample document
% with two chapters, two parts, a title page,
% a compile flag as well as three forwarding files to set the flag.
% It consists of eight |.tex| files:
% \begin{center}
% \begin{tabular}{ll}
% |cdocsamp.tex|&main file\\
% |cdocsch1.tex|&include file for chapter 1\\
% |cdocsch2.tex|&include file for chapter 2\\
% |cdocspt3.tex|&include file for part 3\\
% |cdocspt4.tex|&include file for part 4\\
% |cdocsdrf.tex|&forwarding file for main file in draft mode\\
% |cdocsfi1.tex|&forwarding file for final version of chapter 1\\
% |cdocsfi2.tex|&forwarding file for final version of chapter 2\\
% \end{tabular}
% \end{center}
% Each of the eight files can be compiled directly by the \LaTeX{} compiler.
%
% %%%%%%%%%%%%%%%%%%%%%%%%%%%%%%%%%%%%%%
% \paragraph{Main File.}
%
% The main file is called |cdocsamp.tex|.
%
% Load the \textsf{childdoc} definitions and
% declare the filename for the main document:
%    \begin{macrocode}
\input{childdoc.def}
\childdocmain{}
%    \end{macrocode}

% Optional override for |\version| flag:
%    \begin{macrocode}
%%\ifchilddoc\else\providecommand{\version}{draft}\fi
%    \end{macrocode}

% Define the default values for the |\version| flag
% (|final| for the main file and |draft| for childs):
%    \begin{macrocode}
\ifchilddoc
\providecommand{\version}{draft}
\else
\providecommand{\version}{final}
\fi
%    \end{macrocode}

% Load the standard document class:
%    \begin{macrocode}
\documentclass[12pt]{article}
%    \end{macrocode}

% Start the document body:
%    \begin{macrocode}
\begin{document}
%    \end{macrocode}

% Declare a title page.
% Print title, part of document being processed and version flag:
%    \begin{macrocode}
\addtocounter{page}{-1}
\begin{center}
{\LARGE\bfseries{}childdoc example\par}
\vspace{1cm}
\ifchilddoc
\ifchilddocmanual part\else chapter\fi:
`\childdocname' of `\childdocjob'\par
\else
main document: `\childdocjob'\par
\fi
version: \version\par
\end{center}
\newpage
%    \end{macrocode}

% Manually include selected file,
% otherwise process as usual:
%    \begin{macrocode}
\ifchilddocmanual
\section*{part `\childdocname'}
\input{\childdocname}
\else
%    \end{macrocode}

% Include the two chapters:
%    \begin{macrocode}
\include{cdocsch1}
\include{cdocsch2}
%    \end{macrocode}

% Include the two parts unless only chapters should be displayed:
%    \begin{macrocode}
\ifchilddoc\else
\section{part three}
\input{cdocspt3}
\section{part four}
\input{cdocspt4}
\fi
%    \end{macrocode}

% Process as usual until here:
%    \begin{macrocode}
\fi
%    \end{macrocode}

% End of document body:
%    \begin{macrocode}
\end{document}
%    \end{macrocode}
%\iffalse
%</samplemain>
%\fi
%
% %%%%%%%%%%%%%%%%%%%%%%%%%%%%%%%%%%%%%%
% \paragraph{Chapter Include Files.}
%
% The include files are called |cdocsch1.tex| and |cdocsch2.tex|.
%
%\iffalse
%<*samplechap1|samplechap2>
%\fi

% Optional override for |\version| flag:
%    \begin{macrocode}
%%\providecommand{\version}{final}
%    \end{macrocode}

% Include the main document:
%    \begin{macrocode}
\input{childdoc.def}
\childdocof{cdocsamp}
%    \end{macrocode}

%\iffalse
%</samplechap1|samplechap2>
%\fi
%
%\iffalse
%<*samplechap1>
%\fi
% Some text for chapter 1:
%    \begin{macrocode}
\section{one}
some text in chapter one
%    \end{macrocode}

%\iffalse
%</samplechap1>
%\fi
% Some text for chapter 2:
%\iffalse
%<*samplechap2>
%\fi
%    \begin{macrocode}
\section{two}
more text in chapter two
%    \end{macrocode}

%\iffalse
%</samplechap2>
%\fi
%
% %%%%%%%%%%%%%%%%%%%%%%%%%%%%%%%%%%%%%%
% \paragraph{Part Include Files.}
%
% The include files are called |cdocspt3.tex| and |cdocspt4.tex|.
%
%\iffalse
%<*samplepart3|samplepart4>
%\fi

% Optional override for |\version| flag:
%    \begin{macrocode}
%%\providecommand{\version}{final}
%    \end{macrocode}

% Include the main document:
%    \begin{macrocode}
\input{childdoc.def}
\childdocby{cdocsamp}
%    \end{macrocode}

%\iffalse
%</samplepart3|samplepart4>
%\fi
%
%\iffalse
%<*samplepart3>
%\fi
% Some text for part 3:
%    \begin{macrocode}
some text in part three
%    \end{macrocode}

%\iffalse
%</samplepart3>
%\fi
% Some text for part 4:
%\iffalse
%<*samplepart4>
%\fi
%    \begin{macrocode}
more text in part four
%    \end{macrocode}

%\iffalse
%</samplepart4>
%\fi
%
% %%%%%%%%%%%%%%%%%%%%%%%%%%%%%%%%%%%%%%
% \paragraph{Forwarding for a Complete Draft.}
%
% The following forwarding file |cdocsdrf.tex|
% compiles the main document in draft mode:
%\iffalse
%<*sampledraft>
%\fi
%    \begin{macrocode}
\def\version{draft}
\input{childdoc.def}
\childdocforward{cdocsamp}
%    \end{macrocode}

%\iffalse
%</sampledraft>
%\fi
%
% %%%%%%%%%%%%%%%%%%%%%%%%%%%%%%%%%%%%%%
% \paragraph{Forwarding for Final Version of the Chapters.}
%
% The following forwarding files |cdocsfn1.tex| and |cdocsfn2.tex|
% (with identical content)
% compile the final versions of the child documents
% |cdocsch1.tex| and |cdocsch2.tex|, respectively:
%\iffalse
%<*samplefinal>
%\fi
%    \begin{macrocode}
\def\version{final}
\input{childdoc.def}
\childdocforwardprefix[cdocsamp]{cdocsfn}{cdocsch}
%    \end{macrocode}

%\iffalse
%</samplefinal>
%\fi
%
% %%%%%%%%%%%%%%%%%%%%%%%%%%%%%%%%%%%%%%
% \paragraph{Command Line Processing.}
%
% The following three command lines generate the output files
% |cdocscld|, |cdocscl1| and |cdocscl2|
% which should be identical to
% |cdocsdrf|, |cdocsch1| and |cdocsfn2|, respectively:
% \begin{center}
% \begin{tabular}{l}
% |latex -jobname cdocscld \|\\
% |  "\def\version{draft}\input{childdoc.def}\childdocforward{cdocsamp}"|\\
% |latex -jobname cdocscl1 \|\\
% |  "\input{childdoc.def}\childdocforward[cdocsamp]{cdocsch1}"|\\
% |latex -jobname cdocscl2 \|\\
% |  "\def\version{final}\input{childdoc.def}\childdocforward{cdocsch2}"|
% \end{tabular}
% \end{center}
% Note that the trailing backslash on each first line
% merely continues the input to the second line
% (for convenient cut ant paste).
% Furthermore, the command |latex| can be replaced by any
% of its alternative versions such as |pdflatex|.
%
% %%%%%%%%%%%%%%%%%%%%%%%%%%%%%%%%%%%%%%%%%%%%%%%%%%%%%%%%%%%%%%%%%%%%%%%%%%%%%%
% %%%%%%%%%%%%%%%%%%%%%%%%%%%%%%%%%%%%%%%%%%%%%%%%%%%%%%%%%%%%%%%%%%%%%%%%%%%%%%
% \section{Implementation}
%\iffalse
%<*package>
%\fi
%
% This section describes the definitions file |childdoc.def|.

% The definitions cannot be loaded using |\usepackage| or |\RequirePackage|
% which has a mechanism to prevent loading a style file more than once.
% When loading the definitions by means of |\input|
% multiple instances have to be prevented manually:
%\iffalse
%This code needs to be before the `\ProvidesFile' directive
%which is defined at the beginning of this file.
%Therefore it is also placed there and commented out here.
%</package>
%<*discard>
%\fi
%    \begin{macrocode}
\ifdefined\childdocmain\endinput\fi
%    \end{macrocode}
%\iffalse
%</discard>
%<*package>
%\fi
%
% \macro{\ifchilddoc}
% \macro{\ifchilddocmanual}
% The conditional |\ifchilddoc| tells whether a
% child (true) or main (false) document is being compiled.
% The conditional |\ifchilddocmanual| tells whether
% the |\includeonly| mechanism is used (false) or
% the selection of child files must be performed manually (true).
% The definitions initialise to false:
%    \begin{macrocode}
\newif\ifchilddoc
\newif\ifchilddocmanual
%    \end{macrocode}

% \macro{\childdocname}
% \macro{\childdocjob}
% The macro |\childdocname| stores the name of the main document
% to be compiled. The macro |\childdocjob| stores the name of
% the document on which the \LaTeX{} compiler was originally invoked.
% The content of |\jobname| cannot be compared
% to filenames specified in the source due to different catcodes.
% The following code rescans |\jobname|, stores the result
% in |\childdocname| and saves a copy in |\childdocjob|:
%    \begin{macrocode}
\edef\childdocname{\scantokens\expandafter{\jobname\noexpand}}
\let\childdocjob\childdocname
%    \end{macrocode}

% \macro{\childdocdisable}
% The macro |\childdocdisable| prevents the main file
% from being processed more than once.
% At this stage, the main document command |\childdocmain|
% is assumed to be called once again where it should do nothing.
% Any subsequent call to it should prevent
% a secondary processing of the main document
% It overwrites the forwarding commands
% |\childdocof| and |\childdocforward|
% with empty macros to prevent further inclusions of the main document:
%    \begin{macrocode}
\newcommand{\childdocdisable}
{
  \renewcommand{\childdocmain}[1]{\renewcommand{\childdocmain}[1]{\endinput}}
  \renewcommand{\childdocof}[1]{}
  \renewcommand{\childdocby}[2][]{}
  \renewcommand{\childdocforward}[2][]{}
  \renewcommand{\childdocdisable}{}
}
%    \end{macrocode}

% \macro{\childdocmain}
% The macro |\childdocmain| is to be called at the top of the main file
% with nothing or the main filename (without extension) as argument.
% First, it breaks loops.
% If the argument is not empty and does not match |\childdocname|
% (which is set by the first inclusion of |childdoc.def|),
% |\ifchilddoc| is set to true, |\includeonly| is applied to the child file
% and |\jobname| is set to the main file
% (for proper handling of |.aux| files):
%    \begin{macrocode}
\newcommand{\childdocmain}[1]
{
  \childdocdisable\childdocmain{}
  \if?#1?\else
    \begingroup
      \def\childdoctmp{#1}
      \ifx\childdoctmp\childdocname
        \def\childdoctmp{}
      \else
        \def\childdoctmp
        {
          \childdoctrue
          \includeonly{\childdocname}
          \def\childdocjob{#1}
          \def\jobname{#1}
        }
      \fi
      \expandafter
    \endgroup
    \childdoctmp
  \fi
}
%    \end{macrocode}

% \macro{\childdocof}
% The command |\childdocof| redirects
% compilation to the main file |#1|.
%    \begin{macrocode}
\newcommand{\childdocof}[1]
{
  \childdocdisable
  \childdoctrue
  \includeonly{\childdocname}
  \def\jobname{#1}
  \def\childdocjob{#1}
  \input{#1}
}
%    \end{macrocode}

% \macro{\childdocby}
% The command |\childdocby| ....
%    \begin{macrocode}
\newcommand{\childdocby}[2][]
{
  \childdocdisable
  \childdoctrue
  \childdocmanualtrue
  \if?#1?\else
    \def\jobname{#2}
  \fi
  \def\childdocjob{#2}
  \input{#2}
  \endinput
}
%    \end{macrocode}

% \macro{\childdocforward}
% The command |\childdocforward| redirects
% compilation to the main file or
% (if the optional argument is given) a child file.
% Parameters are set as if the main file
% or a child file starting with |\childdocof| was compiled.
% Then compilation is handed over to the main file:
%    \begin{macrocode}
\newcommand{\childdocforward}[2][]
{
  \begingroup
    \if?#1?
      \def\childdoctmp
      {
        \def\childdocname{#2}
        \def\childdocjob{#2}
        \def\jobname{#2}
        \input{#2}
        \endinput
      }
    \else
      \def\childdoctmp
      {
        \childdocdisable
        \def\childdocname{#2}
        \childdoctrue
        \includeonly{#2}
        \def\childdocjob{#1}
        \def\jobname{#1}
        \input{#1}
        \endinput
      }
    \fi
    \expandafter
  \endgroup
  \childdoctmp
}
%    \end{macrocode}

% \macro{\childdocforwardprefix}
% The command |\childdocforwardprefix| redirects
% compilation to the main or a child file by means of a pattern.
% The prefix |#1| in the current filename is replaced by |#2|
% and the suffix of the current filename is kept
% (it is assumed that the filename does not contain the substring `|~~~|'
% which is used as a delimiter).
% Compilation is handed over to the new file by |\childdocforward|:
%    \begin{macrocode}
\newcommand{\childdocforwardprefix}[3][]
{
  \begingroup
    \def\childdocextract #2##1~~~{\def\childdoctmp{\childdocforward[#1]{#3##1}}}
    \expandafter\childdocextract\childdocname~~~
    \expandafter
  \endgroup
  \childdoctmp
}
%    \end{macrocode}

% \macro{\childdoc}
% The deprecated macro |\childdoc| is a legacy version of |\childdocmain|:
%    \begin{macrocode}
\newcommand{\childdoc}{\childdocmain}
%    \end{macrocode}

% \macro{\childdocredirect}
% The deprecated macro |\childdocredirect| is a legacy version
% of |\childdocforward| and |\childdocforwardprefix|:
%    \begin{macrocode}
\newcommand{\childdocredirect}[2][]
{
  \begingroup
    \if?#1?
      \def\childdoctmp{\childdocforward{#2}}
    \else
      \def\childdoctmp{\childdocforwardprefix{#1}{#2}}
    \fi
    \expandafter
  \endgroup
  \childdoctmp
}
%    \end{macrocode}

%\iffalse
%</package>
%\fi
%
\endinput

\childdocforward{cdocsamp}
%    \end{macrocode}

%\iffalse
%</sampledraft>
%\fi
%
% %%%%%%%%%%%%%%%%%%%%%%%%%%%%%%%%%%%%%%
% \paragraph{Forwarding for Final Version of the Chapters.}
%
% The following forwarding files |cdocsfn1.tex| and |cdocsfn2.tex|
% (with identical content)
% compile the final versions of the child documents
% |cdocsch1.tex| and |cdocsch2.tex|, respectively:
%\iffalse
%<*samplefinal>
%\fi
%    \begin{macrocode}
\def\version{final}
% \iffalse
%
% childdoc.dtx Copyright (C) 2017-2018 Niklas Beisert
%
% This work may be distributed and/or modified under the
% conditions of the LaTeX Project Public License, either version 1.3
% of this license or (at your option) any later version.
% The latest version of this license is in
%   http://www.latex-project.org/lppl.txt
% and version 1.3 or later is part of all distributions of LaTeX
% version 2005/12/01 or later.
%
% This work has the LPPL maintenance status `maintained'.
%
% The Current Maintainer of this work is Niklas Beisert.
%
% This work consists of the files childdoc.dtx and childdoc.ins
% and the derived files childdoc.def and cdocsamp.tex with
% cdocsch1.tex, cdocsch2.tex, cdocsdrf.tex, cdocsfn1.tex, cdocsfn2.tex.
%
%<package>\ifdefined\childdocmain\endinput\fi
%<package>\ProvidesFile{childdoc.def}[2018/12/30 v2.0 child document driver]
%<samplemain>\ProvidesFile{cdocsamp.tex}[2018/12/30 v2.0 sample for childdoc]
%<*driver>
%\ProvidesFile{childdoc.drv}[2018/12/30 v2.0 childdoc reference manual file]
\PassOptionsToClass{10pt,a4paper}{article}
\documentclass{ltxdoc}

\usepackage[margin=35mm]{geometry}
\usepackage{hyperref}
\usepackage{hyperxmp}
\usepackage[usenames]{color}

\hypersetup{colorlinks=true}
\hypersetup{pdfstartview=FitH}
\hypersetup{pdfpagemode=UseNone}
\hypersetup{pdfsource={}}
\hypersetup{pdflang={en-UK}}
\hypersetup{pdfcopyright={Copyright 2017-2018 Niklas Beisert.
  This work may be distributed and/or modified under the
  conditions of the LaTeX Project Public License, either version 1.3
  of this license or (at your option) any later version.}}
\hypersetup{pdflicenseurl={http://www.latex-project.org/lppl.txt}}
\hypersetup{pdfcontactaddress={ETH Zurich, ITP, HIT K,
  Wolfgang-Pauli-Strasse 27}}
\hypersetup{pdfcontactpostcode={8093}}
\hypersetup{pdfcontactcity={Zurich}}
\hypersetup{pdfcontactcountry={Switzerland}}
\hypersetup{pdfcontactemail={nbeisert@itp.phys.ethz.ch}}
\hypersetup{pdfcontacturl={http://people.phys.ethz.ch/\xmptilde nbeisert/}}

\newcommand{\secref}[1]{\hyperref[#1]{section \ref*{#1}}}

\parskip1ex
\parindent0pt
\let\olditemize\itemize
\def\itemize{\olditemize\parskip0pt}

\begin{document}

\title{The \textsf{childdoc} Package}
\hypersetup{pdftitle={The childdoc Package}}
\author{Niklas Beisert\\[2ex]
  Institut f\"ur Theoretische Physik\\
  Eidgen\"ossische Technische Hochschule Z\"urich\\
  Wolfgang-Pauli-Strasse 27, 8093 Z\"urich, Switzerland\\[1ex]
  \href{mailto:nbeisert@itp.phys.ethz.ch}
  {\texttt{nbeisert@itp.phys.ethz.ch}}}
\hypersetup{pdfauthor={Niklas Beisert}}
\hypersetup{pdfsubject={Manual for the LaTeX2e Package childdoc}}
\date{30 December 2018, \textsf{v2.0}}
\maketitle

\begin{abstract}\noindent
\textsf{childdoc} is a \LaTeXe{} package
that enables the direct compilation
of document sections included by |\include|
to individual files.
\end{abstract}

\begingroup
\parskip0ex
\tableofcontents
\endgroup

%%%%%%%%%%%%%%%%%%%%%%%%%%%%%%%%%%%%%%%%%%%%%%%%%%%%%%%%%%%%%%%%%%%%%%%%%%%%%%%%
%%%%%%%%%%%%%%%%%%%%%%%%%%%%%%%%%%%%%%%%%%%%%%%%%%%%%%%%%%%%%%%%%%%%%%%%%%%%%%%%
\section{Introduction}

\LaTeX{} provides a mechanism to structure a large document (such as a book)
into a main file and several child files (containing the chapters)
using the |\include| command.
This mechanism is beneficial for documents
which span hundreds of pages in order to
make the source file(s) more manageable.
Moreover, compilation can be restricted to
selected child files by means of the |\includeonly| command.
The latter feature can be used to reduce the compilation time while editing
(this was significantly more useful in the earlier days of \LaTeX{})
or to generate a smaller document which is easier to navigate.
Another application of |\includeonly| is to generate
documents consisting of selected parts of the complete document.

However, there are a few drawbacks of the plain |\include| mechanism:
\begin{itemize}
\item
The child files cannot be compiled on their own,
they can only be compiled via the main file.
A naive editing environment
(such as a text editor with an option
to have the current file processed by \LaTeX)
may require one to switch to the main file before compiling;
attempting to compile the child file produces errors.
\item
The main file must be modified (each time)
to adjust the |\includeonly| command
to the present needs. This easily leaves the main file in a messy state.
\item
The generated document will always carry the filename
of the main document. This is inconvenient if
several child files are to be compiled and
to be kept for distribution.
\end{itemize}

The present package provides a simple interface
to make child files individually compilable by \LaTeX{}.
Compiling a child file then has the same effect as compiling
the main file with an |\includeonly| command
to select the appropriate child.
Moreover the generated document will carry the name of the child
rather than the main file.
This resolves all three above issues.

This feature is meant to make the editing of books,
thesis documents and lecture notes somewhat more convenient.
However, the package can also be used efficiently for
composing a series of documents (such as exercise sheets)
which are typically distributed individually.
It then assists the author in generating the individual documents
(potentially in different versions)
as well as a document containing the collected series.
Another application is in developing style files
or other kinds of included material
where compilation of the style file could redirect
to a sample or test file.

%%%%%%%%%%%%%%%%%%%%%%%%%%%%%%%%%%%%%%%%%%%%%%%%%%%%%%%%%%%%%%%%%%%%%%%%%%%%%%%%
%%%%%%%%%%%%%%%%%%%%%%%%%%%%%%%%%%%%%%%%%%%%%%%%%%%%%%%%%%%%%%%%%%%%%%%%%%%%%%%%
\section{Usage}

First of all, the package \textsf{childdoc} is \emph{not} a standard
\LaTeXe{} |.sty| style file! Therefore it needs to be invoked in
a non-standard way.

%%%%%%%%%%%%%%%%%%%%%%%%%%%%%%%%%%%%%%%%%%%%%%%%%%%%%%%%%%%%%%%%%%%%%%%%%%%%%%%%
\subsection{Included Files}
\label{sec:include}

%%%%%%%%%%%%%%%%%%%%%%%%%%%%%%%%%%%%%%%%
\DescribeMacro{\childdocmain}
To use the package, add the commands
\begin{center}
\begin{tabular}{l}
|\input{childdoc.def}|\\
|\childdocmain{}|\\
\end{tabular}
\end{center}
at the very top of the main \LaTeX{} file,
in particular \emph{before} the |\documentclass| statement!
The argument of |\childdocmain| should be left empty
(but it must be present).

%%%%%%%%%%%%%%%%%%%%%%%%%%%%%%%%%%%%%%%%
\DescribeMacro{\childdocof}
Furthermore, add the commands
\begin{center}
\begin{tabular}{l}
|\input{childdoc.def}|\\
|\childdocof{|\textit{main}|}|\\
\end{tabular}
\end{center}
at the top of every child file \textit{child}
which is included by |\include{|\textit{child}|}|
from within the main file
(or at least for those files to be compiled individually).
The argument \textit{main} must be the filename of the main file.

There are a couple of
considerations in setting up the main and child documents:

%%%%%%%%%%%%%%%%%%%%%%%%%%%%%%%%%%%%%%%%
\paragraph{Restrictions.}

Please note the following restrictions:
\begin{itemize}
\item
|\childdocmain| must be called with one argument \textit{main}
to ensure compatibility with earlier version of the package.
It must either be empty (|\childdocmain{}|)
or precisely match the filename of the main file in which it is specified.
See \secref{sec:detection} for further information.
\item
The filename \textit{main} must be specified without the |.tex| extension.
\item
The filename \textit{main} is case sensitive
(even in case-insensitive file systems)
due to internal string comparison.
\item
The argument \textit{main} should be fully expanded, it cannot be a macro.
\item
Subdirectories and special characters should be avoided in filenames.
\item
The command |\childdocmain{|\textit{main}|}| must be followed by a whitespace.
It should not be followed immediately by another command
or by a comment mark `|%|'.
This is because the \TeX{} parser reads the token immediately following
the argument of |\childdocmain| and puts it
at the beginning of every child section;
however, a white\-space is ignored.
\end{itemize}

%%%%%%%%%%%%%%%%%%%%%%%%%%%%%%%%%%%%%%%%
\paragraph{Content of Main File.}

It is advisable to place all content in the child files included by |\include|.
Any output contained in the main file will appear in all child documents
unless suppressed manually;
it cannot be suppressed automatically by the |\includeonly| directive
and thus should normally be avoided.
A method to include some content in the main file
by means of conditional processing is described in \secref{sec:conditional}.

%%%%%%%%%%%%%%%%%%%%%%%%%%%%%%%%%%%%%%%%
\paragraph{Page Numbering.}

When only a part of the document is compiled,
the appropriate numbering of pages
(as well as other status parameters)
is determined from the |.aux| files.
The latter contain information from previous passes.
However this information needs to propagate through
all intermediate child documents.
Therefore the page numbering in child documents may well
be inconsistent until the complete document is compiled at least once.

A useful (if unconventional) way to always ensure a consistent
page numbering is to restart the numbering in each child document
and denote the pages by `\textit{child}|.|\textit{page}'
where \textit{child} represents the chapter/section number of the child file.
This can be achieved by the command
|\numberwithin{page}{|\textit{child}|}|
of the \textsf{amsmath} package
where \textit{child} can be |chapter| or |section|
depending on the chosen structuring.
Alternatively, one can modify the macro |\thepage| appropriately
and reset the counter |page| at the start of each child file.

%%%%%%%%%%%%%%%%%%%%%%%%%%%%%%%%%%%%%%%%%%%%%%%%%%%%%%%%%%%%%%%%%%%%%%%%%%%%%%%%
\subsection{Conditional Processing}
\label{sec:conditional}

The package provides a mechanism to compile different versions
of a document. To customise the versions further some conditional processing
can come in handy to distinguish which version is being compiled.
The package provides two macros to describe the compilation context:

%%%%%%%%%%%%%%%%%%%%%%%%%%%%%%%%%%%%%%%%
\DescribeMacro{\ifchilddoc}
The conditional |\ifchilddoc| distinguishes between the compilation of
child documents and the main document:
%
\begin{center}
|\ifchilddoc |\textit{child-code}| |[|\||else |\textit{main-code}]| \||fi|
\end{center}

%%%%%%%%%%%%%%%%%%%%%%%%%%%%%%%%%%%%%%%%
\DescribeMacro{\childdocname}
\DescribeMacro{\childdocjob}
The macro |\childdocname| contains the filename (without extension)
of the main or child file being processed.
Note that |\childdocjob| will always contain the name of the main file.

%%%%%%%%%%%%%%%%%%%%%%%%%%%%%%%%%%%%%%%%
\paragraph{Title Page.}

Conditional processing can be used to include a title or banner page
in the main document when proper precautions are taken.
Importantly, the code in the main file should ensure that the page counter
(as well as other status parameters which are stored in the |.aux| files)
takes the same value after the conditional processing.
Otherwise the page numbers may take divergent values
depending on which part is compiled.

For example, a title page could be declared by:
%
\begin{center}
\begin{tabular}{l}
|\ifchilddoc\||else|\\
|\addtocounter{page}{-1}|\\
\textit{code for title page}\\
|\newpage|\\
|\||fi|
\end{tabular}
\end{center}
%
A banner page for the child documents can be generated by:
%
\begin{center}
\begin{tabular}{l}
|\ifchilddoc|\\
|\addtocounter{page}{-1}|\\
\textit{code for banner page}\\
|\newpage|\\
|\||fi|
\end{tabular}
\end{center}
%
Here one could write a message such as:
\begin{center}
|This is the part \childdocname{} of \childdocjob{}.|
\end{center}

%%%%%%%%%%%%%%%%%%%%%%%%%%%%%%%%%%%%%%%%%%%%%%%%%%%%%%%%%%%%%%%%%%%%%%%%%%%%%%%%
\subsection{Flags}
\label{sec:flags}

The package makes it easy to generate different versions
of the main or child documents.
To this end compilation flags can be defined
and assigned different default values.
They will be particularly useful in conjunction
with the forwarding mechanism described in \secref{sec:forward}.

For example, it may be useful to have a flag |\version|
which can be set to |draft| or |final|.
The document source will contain some conditional code
depending on the value of |\version|.
Suppose further, the flag should default to |final| for the main file
and to |draft| for child files
which is a natural assignment for editing the document.
This is achieved by placing the following code
in the preamble of the main document
(below the |\childdocmain| directive):
%
\begin{center}
\begin{tabular}{l}
|\ifchilddoc|\\
|\providecommand{\version}{draft}|\\
|\||else|\\
|\providecommand{\version}{final}|\\
|\||fi|
\end{tabular}
\end{center}
%
The definition by |\providecommand| makes sure
that previous definitions are not overwritten.
Further statements |\providecommand{\version}{...}|
can thus be added before the above code to override it.

For the main file, one might add a line
(between |\childdocmain| and the above block)
%
\begin{center}
|%\ifchilddoc\||else\providecommand{\version}{draft}\||fi|
\end{center}
%
which can be uncommented to produce a draft version.
Likewise one can add a line to the very top of a child file
(above the |\childdocof{|\textit{main}|}| directive)
%
\begin{center}
|%\providecommand{\version}{final}|
\end{center}
%
which can be uncommented to produce the final version of this child document.

%%%%%%%%%%%%%%%%%%%%%%%%%%%%%%%%%%%%%%%%%%%%%%%%%%%%%%%%%%%%%%%%%%%%%%%%%%%%%%%%
\subsection{Forwarding}
\label{sec:forward}

Different versions of the main or child documents
using compilation flags as described in \secref{sec:flags}
can be (permanently) stored in different files
for convenient compilation, viewing and distribution.
To this end, the package defines a command
to pass on compilation to a different file:

%%%%%%%%%%%%%%%%%%%%%%%%%%%%%%%%%%%%%%%%
\DescribeMacro{\childdocforward}
The command |\childdocforward| redirects processing to
another source file:
%
\begin{center}
\begin{tabular}{l}
|\input{childdoc.def}|\\
|\childdocforward[|\textit{main}|]{|\textit{dest}|}|\\
\end{tabular}
\end{center}
%
The argument \textit{dest} is the destination file
(without extension).
It should be the main file or one of the child files.
Note that further \textsf{childdoc} directives
such as |\childdocof| and |\childdocforward|
in the indicated file will be processed in this form.
The optional argument \textit{main}
passes on directly to the main file \textit{main}
while pretending to compile the child \textit{dest}.
This form behaves as if \textit{dest}
issues |\childdocof{|\textit{main}|}| right away,
and no further \textsf{childdoc} directives will be processed.

%%%%%%%%%%%%%%%%%%%%%%%%%%%%%%%%%%%%%%%%
\DescribeMacro{\...prefix}
In the alternative form |\childdocforwardprefix|,
%
\begin{center}
\begin{tabular}{l}
|\input{childdoc.def}|\\
|\childdocforwardprefix[|\textit{main}|]{|\textit{prefix}|}{|\textit{dest}|}|
\end{tabular}
\end{center}
%
the destination file is determined by a pattern
depending on the current file:
To make this work, the current file must be called
`{\textit{prefix}\hspace{0.2em}\textit{suffix}}'
with \textit{prefix} matching precisely the argument.
Processing is then passed on to the file
`{\textit{dest}\hspace{0.2em}\textit{suffix}}'.
Surely, the same effect is achieved by
directly specifying the
argument `{\textit{dest}\hspace{0.2em}\textit{suffix}}'
in the first form.
However, that requires to set up a different file
for each child. With the alternative form of the command
all these files can have exactly the same content
which simplifies setting them up and maintaining them.

For example, the following file |draft.tex|
with a compilation flag |\version| as described in \secref{sec:flags}
compiles the main document as a draft:
%
\begin{center}
\begin{tabular}{l}
|\def\version{draft}|\\
|\input{childdoc.def}|\\
|\childdocforward{|\textit{main}|}|
\end{tabular}
\end{center}
%
Likewise, the following files |final|\textit{nn}|.tex|
compile the final version of the child document
|child|\textit{nn}|.tex|:
%
\begin{center}
\begin{tabular}{l}
|\def\version{final}|\\
|\input{childdoc.def}|\\
|\childdocforwardprefix{final}{child}|
\end{tabular}
\end{center}
%

Note that when several versions of a main file and/or of each child file
are to be generated, it may be convenient to set up a |Makefile| or
shell script to automatise the process.

%%%%%%%%%%%%%%%%%%%%%%%%%%%%%%%%%%%%%%%%%%%%%%%%%%%%%%%%%%%%%%%%%%%%%%%%%%%%%%%%
\subsection{Command Line Processing}
\label{sec:commandline}

The effect of redirection files can also be achieved by invoking
the \LaTeX{} compiler with a more elaborate command line.
Most conveniently this should be done as part
of a shell script or a |Makefile|.

When using \textsf{childdoc} in the main file, the following
command lines effectively perform a redirection
(note that depending on the shell being used,
backslashes may have to be doubled: `|\|' $\to$ `|\\|'):
%
\begin{center}
|... -jobname "|\textit{target}|" |\\|"|[\textit{flags}]%
|\input{childdoc.def}\childdocforward[|\textit{main}|]{|\textit{dest}|}"|
\end{center}
%
Here \textit{target} is the name of the output file,
\textit{main} is the name of the main file
and \textit{dest} is the name of the main or child file to be processed
(all filenames without extensions).
The optional argument \textit{main} can be omitted
if \textit{main} matches \textit{dest}.
Optionally, compilation \textit{flags} can be defined via |\def| commands.
This command line makes the \TeX{} engine believe
it is compiling the file \textit{target}
whose content is specified as the latter parameter.
The provided code then forwards the processing to
\textit{main} or \textit{dest} as described in \secref{sec:forward}.

%%%%%%%%%%%%%%%%%%%%%%%%%%%%%%%%%%%%%%%%%%%%%%%%%%%%%%%%%%%%%%%%%%%%%%%%%%%%%%%%
\subsection{Include by Input}
\label{sec:input}

Including child documents by |\include| has some restrictions by design.
Most notably, the content of a child document always occupies
its own set of pages; pages cannot be shared between child documents.
Usually, this behaviour makes perfect sense
because each child document contain an essential part of the document.
However, in some situations it may be desirable to compose
a document from a collection of parts
without having mandatory page breaks between then.
For this case, the package
provides a mechanism to include parts
by |\input| which can also be processed individually.
However, by construction this mechanism
requires manual handling of the content to be output.

%%%%%%%%%%%%%%%%%%%%%%%%%%%%%%%%%%%%%%%%
\DescribeMacro{\ifchilddocmanual}
The main file should be prepared as usual, see \secref{sec:include}.
However, the document body must make a distinction
between processing of an individual part and of the main document, e.g.:
%
\begin{center}
\begin{tabular}{l}
|\ifchilddocmanual|\\
|\input{\childdocname}|\\
|\||else|\\
\textit{document body with }|\input{|\textit{part}|}|\\
|\||fi|
\end{tabular}
\end{center}
%
The conditional |\ifchilddocmanual| is true whenever
a part to be included by |\input| is being compiled,
and the name of the part is stored in |\childdocname|.

%%%%%%%%%%%%%%%%%%%%%%%%%%%%%%%%%%%%%%%%
\DescribeMacro{\childdocby}
Each part to be included by |\input| should start with:
%
\begin{center}
\begin{tabular}{l}
|\input{childdoc.def}|\\
|\childdocby{|\textit{main}|}|\\
\end{tabular}
\end{center}
%
The directive |\childdocby| is similar to |\childdocof|
described in \secref{sec:include},
but the subsequent selection of content must be done manually.
To that end, both |\ifchilddoc| and |\ifchilddocmanual|
will be true upon processing of a part,
and the name of the part is stored in |\childdocname|.
Note that |\jobname| will be set to the filename of the current part
so that each part receives an individual |.aux| file
that does not interfere with the |.aux| file(s) of the main document.
This behaviour can be altered by the alternative form
|\childdocby[*]{|\textit{main}|}| (with a non-empty optional argument)
which uses the |.aux| file of the main document
by setting |\jobname| to \textit{main}.

%%%%%%%%%%%%%%%%%%%%%%%%%%%%%%%%%%%%%%%%%%%%%%%%%%%%%%%%%%%%%%%%%%%%%%%%%%%%%%%%
\subsection{Driver Development}
\label{sec:driver}

The \textsf{childdoc} mechanism can also be use for the development
of definition files such as \LaTeX{} styles or classes.
This case differs from the above setup with multiple parts
included by |\include| in that no |\includeonly| should be invoked.
This can be achieved by starting the include file
(before |\ProvidesPackage|) with:
%
\begin{center}
\begin{tabular}{l}
|\input{childdoc.def}|\\
|\childdocforward{|\textit{main}|}|\\
\end{tabular}
\end{center}
%
or alternatively with:
%
\begin{center}
\begin{tabular}{l}
|\input{childdoc.def}|\\
|\childdocby{|\textit{main}|}|\\
\end{tabular}
\end{center}
%
Both forms have slightly different effects as described above.
The main file is prepared as usual, see \secref{sec:include}.

%%%%%%%%%%%%%%%%%%%%%%%%%%%%%%%%%%%%%%%%%%%%%%%%%%%%%%%%%%%%%%%%%%%%%%%%%%%%%%%%
\subsection{Legacy Detection}
\label{sec:detection}

The directive |\childdocmain| in the main file can detect
whether the complete document or merely a child is to be compiled
even without using the directive |\childdocof|.
This method is deprecated because it is less robust
and there is no compelling reason to use it;
it is merely provided for backward compatibility
and it may be removed in future versions.

If the detection mechanism is to be used,
it is mandatory to correctly specify
the filename of the main file as the argument of |\childdocmain|:
%
\begin{center}
\begin{tabular}{l}
|\input{childdoc.def}|\\
|\childdocmain{|\textit{main}|}|\\
\end{tabular}
\end{center}
%
If |\jobname| does not match the argument \textit{main} of |\childdocmain|,
it is assumed that |\jobname| points to the child file to be compiled.
When using |\childdocmain| with the main file specified as argument,
it suffices to start a child file
with just |\input{|\textit{main}|}|
without loading of the package and using |\childdocof|.
If instead all processing is done
with the appropriate \textsf{childdoc} directives,
the argument of \textit{main} of |\childdocmain| can be empty.

An alternative version of the command line processing described
in \secref{sec:commandline} using the detection mechanism reads:
%
\begin{center}
|... -jobname "|\textit{target}|" "|[\textit{flags}]%
[|\def\jobname{|\textit{dest}|}|]|\input{|\textit{main}|}"|
\end{center}

%%%%%%%%%%%%%%%%%%%%%%%%%%%%%%%%%%%%%%%%%%%%%%%%%%%%%%%%%%%%%%%%%%%%%%%%%%%%%%%%
\subsection{Manual Code}
\label{sec:manual}

In case one cannot be certain whether the definitions file |childdoc.def|
is installed on the target \TeX{} distribution
and one prefers not to ship it,
it is conceivable to paste a few relevant commands into the sources.

To that end, drop all statements |\input{childdoc.def}|
and perform the replacements as outlined below.
Instead of |\childdocmain{|\textit{main}|}| add the following code
to the top of the main file:
%
\begin{center}
\begin{tabular}{l}
|\||ifdefined\childdocname\endinput\||fi\newif\ifchilddoc|\\
|\edef\childdocname{\scantokens\expandafter{\jobname\noexpand}}|\\
|\def\childdocmain{|\textit{main}|}\||ifx\childdocmain\childdocname\||else|\\
|\childdoctrue\includeonly{\childdocname}\let\jobname\childdocmain\||fi|\\
\end{tabular}
\end{center}
%
Instead of |\childdocof{|\textit{main}|}| just include the main file
at the top of each child file:
%
\begin{center}
|\input{|\textit{main}|}|
\end{center}
%
A simple redirection |\childdocforward{|\textit{dest}|}| is achieved by:
%
\begin{center}
|\def\jobname{|\textit{dest}|}\input{\jobname}|
\end{center}
%
The redirection with prefix
|\childdocforwardprefix[|\textit{prefix}|]{|\textit{dest}|}|
is accomplished by:
%
\begin{center}
\begin{tabular}{l}
|{\edef\jobname{\scantokens\expandafter{\jobname\noexpand}}|\\
|\def\redirectjob |\textit{prefix}|#1~~~{\gdef\jobname{|\textit{dest}|#1}}|\\
|\expandafter\redirectjob\jobname~~~}\input{\jobname}|
\end{tabular}
\end{center}

In an alternative approach,
child documents can be compiled by a specific command line
without additional code or specific definitions:
%
\begin{center}
|... -jobname "|\textit{target}|" "|[\textit{flags}]%
|\includeonly{|\textit{dest}|}\input{|\textit{main}|}"|
\end{center}
%

%%%%%%%%%%%%%%%%%%%%%%%%%%%%%%%%%%%%%%%%%%%%%%%%%%%%%%%%%%%%%%%%%%%%%%%%%%%%%%%%
%%%%%%%%%%%%%%%%%%%%%%%%%%%%%%%%%%%%%%%%%%%%%%%%%%%%%%%%%%%%%%%%%%%%%%%%%%%%%%%%
\section{Information}

%%%%%%%%%%%%%%%%%%%%%%%%%%%%%%%%%%%%%%%%%%%%%%%%%%%%%%%%%%%%%%%%%%%%%%%%%%%%%%%%
\subsection{Copyright}

Copyright \copyright{} 2017--2018 Niklas Beisert

This work may be distributed and/or modified under the
conditions of the \LaTeX{} Project Public License, either version 1.3
of this license or (at your option) any later version.
The latest version of this license is in
  \url{http://www.latex-project.org/lppl.txt}
and version 1.3 or later is part of all distributions of \LaTeX{}
version 2005/12/01 or later.

This work has the LPPL maintenance status `maintained'.

The Current Maintainer of this work is Niklas Beisert.

This work consists of the files |README.txt|, |childdoc.ins| and |childdoc.dtx|
as well as the derived files |childdoc.def|, |cdocsamp.tex|
with |cdocsch1.tex|, |cdocsch2.tex|, |cdocspt3.tex|, |cdocspt4.tex|,
|cdocsdrf.tex|, |cdocsfn1.tex|, |cdocsfn2.tex|
as well as |childdoc.pdf|.

%%%%%%%%%%%%%%%%%%%%%%%%%%%%%%%%%%%%%%%%%%%%%%%%%%%%%%%%%%%%%%%%%%%%%%%%%%%%%%%%
\subsection{Files and Installation}

The package consists of the files:
%
\begin{center}
\begin{tabular}{ll}
    |README.txt|   & readme file \\
    |childdoc.ins| & installation file \\
    |childdoc.dtx| & source file \\
    |childdoc.def| & definition file \\
    |cdocsamp.tex| & sample main file \\
    |cdocsch1.tex| & sample include file \\
    |cdocsch2.tex| & sample include file \\
    |cdocspt3.tex| & sample part file \\
    |cdocspt4.tex| & sample part file \\
    |cdocsdrf.tex| & sample redirection file \\
    |cdocsfn1.tex| & sample redirection file \\
    |cdocsfn2.tex| & sample redirection file \\
    |childdoc.pdf| & manual
\end{tabular}
\end{center}
%
The distribution consists of the files
|README.txt|, |childdoc.ins| and |childdoc.dtx|.
%
\begin{itemize}
\item
Run (pdf)\LaTeX{} on |childdoc.dtx|
to compile the manual |childdoc.pdf| (this file).
\item
Run \LaTeX{} on |childdoc.ins| to create the definitions file |childdoc.def|
and the sample |cdocsamp.tex| with include files
|cdocsch1.tex|, |cdocsch2.tex|, |cdocspt3.tex|, |cdocspt4.tex|,
|cdocsdrf.tex|, |cdocsfn1.tex|, |cdocsfn2.tex|.
Then copy the file |childdoc.def| to an appropriate directory of your \LaTeX{}
distribution, e.g.\ \textit{texmf-root}|/tex/latex/childdoc|.
\end{itemize}

%%%%%%%%%%%%%%%%%%%%%%%%%%%%%%%%%%%%%%%%%%%%%%%%%%%%%%%%%%%%%%%%%%%%%%%%%%%%%%%%
\subsection{Related CTAN Packages}

There are several other packages which offer a similar functionality:
%
\begin{itemize}
\item
The packages
\href{http://ctan.org/pkg/docmute}{\textsf{docmute}},
\href{http://ctan.org/pkg/includex}{\textsf{includex}} and
\href{http://ctan.org/pkg/standalone}{\textsf{standalone}}
provide commands to include only the document body of
a child file thus allowing both files to be compiled individually.
\item
The packages \href{http://ctan.org/pkg/subdocs}{\textsf{subdocs}}
and \href{http://ctan.org/pkg/subfiles}{\textsf{subfiles}}
provide structures in which the main and child documents can be
encapsulated and allowing them to be compiled individually.
The inclusion mechanism is different from the conventional |\include|.
\item
The package \href{http://ctan.org/pkg/combine}{\textsf{combine}}
is an elaborate solution to combine several documents into one.
\end{itemize}
%
See also the CTAN topic \href{http://ctan.org/topic/subdocs}{\textsf{subdocs}}
for further related packages.
The present package differs from the above solutions in that
a document structure constructed with the conventional |\include| mechanism
just needs two extra commands at the top of every file
such that all constituent files can be compiled individually.

%%%%%%%%%%%%%%%%%%%%%%%%%%%%%%%%%%%%%%%%%%%%%%%%%%%%%%%%%%%%%%%%%%%%%%%%%%%%%%%%
%\subsection{Feature Suggestions}
%
%The following is a list of features which may be useful for future
%versions of this package:
%%
%\begin{itemize}
%\item
%\ldots
%\end{itemize}

%%%%%%%%%%%%%%%%%%%%%%%%%%%%%%%%%%%%%%%%%%%%%%%%%%%%%%%%%%%%%%%%%%%%%%%%%%%%%%%%
\subsection{Revision History}

%%%%%%%%%%%%%%%%%%%%%%%%%%%%%%%%%%%%%%%%
\paragraph{v2.0:} 2018/12/30

\begin{itemize}
\item
immediate forward processing
\item
added |\childdocby| mechanism
\item
manual restructured
\end{itemize}

%%%%%%%%%%%%%%%%%%%%%%%%%%%%%%%%%%%%%%%%
\paragraph{v1.6:} 2018/01/17

\begin{itemize}
\item
application for development of include files
\item
corrections to manual
\end{itemize}

%%%%%%%%%%%%%%%%%%%%%%%%%%%%%%%%%%%%%%%%
\paragraph{v1.5:} 2017/05/21

\begin{itemize}
\item
more complete structuring introduced
\item
|\childdocof| introduced
\item
|\childdoc| renamed to |\childdocmain|
\item
|\childredirect| renamed to |\childdocforward| and |\childdocforwardprefix|
and functionality expanded
\end{itemize}

%%%%%%%%%%%%%%%%%%%%%%%%%%%%%%%%%%%%%%%%
\paragraph{v1.0:} 2017/04/27

\begin{itemize}
\item
manual and install package
\item
first version published on CTAN
\end{itemize}

%%%%%%%%%%%%%%%%%%%%%%%%%%%%%%%%%%%%%%%%
\paragraph{v0.6:} 2017/04/26

\begin{itemize}
\item
redirection mechanism added
\end{itemize}

%%%%%%%%%%%%%%%%%%%%%%%%%%%%%%%%%%%%%%%%
\paragraph{v0.5:} 2017/04/26

\begin{itemize}
\item
functionality in definition file
\end{itemize}


%%%%%%%%%%%%%%%%%%%%%%%%%%%%%%%%%%%%%%%%%%%%%%%%%%%%%%%%%%%%%%%%%%%%%%%%%%%%%%%%
%%%%%%%%%%%%%%%%%%%%%%%%%%%%%%%%%%%%%%%%%%%%%%%%%%%%%%%%%%%%%%%%%%%%%%%%%%%%%%%%
%%%%%%%%%%%%%%%%%%%%%%%%%%%%%%%%%%%%%%%%%%%%%%%%%%%%%%%%%%%%%%%%%%%%%%%%%%%%%%%%
\appendix

\settowidth\MacroIndent{\rmfamily\scriptsize 000\ }

 \DocInput{childdoc.dtx}

\end{document}
%</driver>
% \fi
%
% %%%%%%%%%%%%%%%%%%%%%%%%%%%%%%%%%%%%%%%%%%%%%%%%%%%%%%%%%%%%%%%%%%%%%%%%%%%%%%
% %%%%%%%%%%%%%%%%%%%%%%%%%%%%%%%%%%%%%%%%%%%%%%%%%%%%%%%%%%%%%%%%%%%%%%%%%%%%%%
% \section{Sample}
%\iffalse
%<*samplemain>
%\fi
%
% The following presents a sample document
% with two chapters, two parts, a title page,
% a compile flag as well as three forwarding files to set the flag.
% It consists of eight |.tex| files:
% \begin{center}
% \begin{tabular}{ll}
% |cdocsamp.tex|&main file\\
% |cdocsch1.tex|&include file for chapter 1\\
% |cdocsch2.tex|&include file for chapter 2\\
% |cdocspt3.tex|&include file for part 3\\
% |cdocspt4.tex|&include file for part 4\\
% |cdocsdrf.tex|&forwarding file for main file in draft mode\\
% |cdocsfi1.tex|&forwarding file for final version of chapter 1\\
% |cdocsfi2.tex|&forwarding file for final version of chapter 2\\
% \end{tabular}
% \end{center}
% Each of the eight files can be compiled directly by the \LaTeX{} compiler.
%
% %%%%%%%%%%%%%%%%%%%%%%%%%%%%%%%%%%%%%%
% \paragraph{Main File.}
%
% The main file is called |cdocsamp.tex|.
%
% Load the \textsf{childdoc} definitions and
% declare the filename for the main document:
%    \begin{macrocode}
\input{childdoc.def}
\childdocmain{}
%    \end{macrocode}

% Optional override for |\version| flag:
%    \begin{macrocode}
%%\ifchilddoc\else\providecommand{\version}{draft}\fi
%    \end{macrocode}

% Define the default values for the |\version| flag
% (|final| for the main file and |draft| for childs):
%    \begin{macrocode}
\ifchilddoc
\providecommand{\version}{draft}
\else
\providecommand{\version}{final}
\fi
%    \end{macrocode}

% Load the standard document class:
%    \begin{macrocode}
\documentclass[12pt]{article}
%    \end{macrocode}

% Start the document body:
%    \begin{macrocode}
\begin{document}
%    \end{macrocode}

% Declare a title page.
% Print title, part of document being processed and version flag:
%    \begin{macrocode}
\addtocounter{page}{-1}
\begin{center}
{\LARGE\bfseries{}childdoc example\par}
\vspace{1cm}
\ifchilddoc
\ifchilddocmanual part\else chapter\fi:
`\childdocname' of `\childdocjob'\par
\else
main document: `\childdocjob'\par
\fi
version: \version\par
\end{center}
\newpage
%    \end{macrocode}

% Manually include selected file,
% otherwise process as usual:
%    \begin{macrocode}
\ifchilddocmanual
\section*{part `\childdocname'}
\input{\childdocname}
\else
%    \end{macrocode}

% Include the two chapters:
%    \begin{macrocode}
\include{cdocsch1}
\include{cdocsch2}
%    \end{macrocode}

% Include the two parts unless only chapters should be displayed:
%    \begin{macrocode}
\ifchilddoc\else
\section{part three}
\input{cdocspt3}
\section{part four}
\input{cdocspt4}
\fi
%    \end{macrocode}

% Process as usual until here:
%    \begin{macrocode}
\fi
%    \end{macrocode}

% End of document body:
%    \begin{macrocode}
\end{document}
%    \end{macrocode}
%\iffalse
%</samplemain>
%\fi
%
% %%%%%%%%%%%%%%%%%%%%%%%%%%%%%%%%%%%%%%
% \paragraph{Chapter Include Files.}
%
% The include files are called |cdocsch1.tex| and |cdocsch2.tex|.
%
%\iffalse
%<*samplechap1|samplechap2>
%\fi

% Optional override for |\version| flag:
%    \begin{macrocode}
%%\providecommand{\version}{final}
%    \end{macrocode}

% Include the main document:
%    \begin{macrocode}
\input{childdoc.def}
\childdocof{cdocsamp}
%    \end{macrocode}

%\iffalse
%</samplechap1|samplechap2>
%\fi
%
%\iffalse
%<*samplechap1>
%\fi
% Some text for chapter 1:
%    \begin{macrocode}
\section{one}
some text in chapter one
%    \end{macrocode}

%\iffalse
%</samplechap1>
%\fi
% Some text for chapter 2:
%\iffalse
%<*samplechap2>
%\fi
%    \begin{macrocode}
\section{two}
more text in chapter two
%    \end{macrocode}

%\iffalse
%</samplechap2>
%\fi
%
% %%%%%%%%%%%%%%%%%%%%%%%%%%%%%%%%%%%%%%
% \paragraph{Part Include Files.}
%
% The include files are called |cdocspt3.tex| and |cdocspt4.tex|.
%
%\iffalse
%<*samplepart3|samplepart4>
%\fi

% Optional override for |\version| flag:
%    \begin{macrocode}
%%\providecommand{\version}{final}
%    \end{macrocode}

% Include the main document:
%    \begin{macrocode}
\input{childdoc.def}
\childdocby{cdocsamp}
%    \end{macrocode}

%\iffalse
%</samplepart3|samplepart4>
%\fi
%
%\iffalse
%<*samplepart3>
%\fi
% Some text for part 3:
%    \begin{macrocode}
some text in part three
%    \end{macrocode}

%\iffalse
%</samplepart3>
%\fi
% Some text for part 4:
%\iffalse
%<*samplepart4>
%\fi
%    \begin{macrocode}
more text in part four
%    \end{macrocode}

%\iffalse
%</samplepart4>
%\fi
%
% %%%%%%%%%%%%%%%%%%%%%%%%%%%%%%%%%%%%%%
% \paragraph{Forwarding for a Complete Draft.}
%
% The following forwarding file |cdocsdrf.tex|
% compiles the main document in draft mode:
%\iffalse
%<*sampledraft>
%\fi
%    \begin{macrocode}
\def\version{draft}
\input{childdoc.def}
\childdocforward{cdocsamp}
%    \end{macrocode}

%\iffalse
%</sampledraft>
%\fi
%
% %%%%%%%%%%%%%%%%%%%%%%%%%%%%%%%%%%%%%%
% \paragraph{Forwarding for Final Version of the Chapters.}
%
% The following forwarding files |cdocsfn1.tex| and |cdocsfn2.tex|
% (with identical content)
% compile the final versions of the child documents
% |cdocsch1.tex| and |cdocsch2.tex|, respectively:
%\iffalse
%<*samplefinal>
%\fi
%    \begin{macrocode}
\def\version{final}
\input{childdoc.def}
\childdocforwardprefix[cdocsamp]{cdocsfn}{cdocsch}
%    \end{macrocode}

%\iffalse
%</samplefinal>
%\fi
%
% %%%%%%%%%%%%%%%%%%%%%%%%%%%%%%%%%%%%%%
% \paragraph{Command Line Processing.}
%
% The following three command lines generate the output files
% |cdocscld|, |cdocscl1| and |cdocscl2|
% which should be identical to
% |cdocsdrf|, |cdocsch1| and |cdocsfn2|, respectively:
% \begin{center}
% \begin{tabular}{l}
% |latex -jobname cdocscld \|\\
% |  "\def\version{draft}\input{childdoc.def}\childdocforward{cdocsamp}"|\\
% |latex -jobname cdocscl1 \|\\
% |  "\input{childdoc.def}\childdocforward[cdocsamp]{cdocsch1}"|\\
% |latex -jobname cdocscl2 \|\\
% |  "\def\version{final}\input{childdoc.def}\childdocforward{cdocsch2}"|
% \end{tabular}
% \end{center}
% Note that the trailing backslash on each first line
% merely continues the input to the second line
% (for convenient cut ant paste).
% Furthermore, the command |latex| can be replaced by any
% of its alternative versions such as |pdflatex|.
%
% %%%%%%%%%%%%%%%%%%%%%%%%%%%%%%%%%%%%%%%%%%%%%%%%%%%%%%%%%%%%%%%%%%%%%%%%%%%%%%
% %%%%%%%%%%%%%%%%%%%%%%%%%%%%%%%%%%%%%%%%%%%%%%%%%%%%%%%%%%%%%%%%%%%%%%%%%%%%%%
% \section{Implementation}
%\iffalse
%<*package>
%\fi
%
% This section describes the definitions file |childdoc.def|.

% The definitions cannot be loaded using |\usepackage| or |\RequirePackage|
% which has a mechanism to prevent loading a style file more than once.
% When loading the definitions by means of |\input|
% multiple instances have to be prevented manually:
%\iffalse
%This code needs to be before the `\ProvidesFile' directive
%which is defined at the beginning of this file.
%Therefore it is also placed there and commented out here.
%</package>
%<*discard>
%\fi
%    \begin{macrocode}
\ifdefined\childdocmain\endinput\fi
%    \end{macrocode}
%\iffalse
%</discard>
%<*package>
%\fi
%
% \macro{\ifchilddoc}
% \macro{\ifchilddocmanual}
% The conditional |\ifchilddoc| tells whether a
% child (true) or main (false) document is being compiled.
% The conditional |\ifchilddocmanual| tells whether
% the |\includeonly| mechanism is used (false) or
% the selection of child files must be performed manually (true).
% The definitions initialise to false:
%    \begin{macrocode}
\newif\ifchilddoc
\newif\ifchilddocmanual
%    \end{macrocode}

% \macro{\childdocname}
% \macro{\childdocjob}
% The macro |\childdocname| stores the name of the main document
% to be compiled. The macro |\childdocjob| stores the name of
% the document on which the \LaTeX{} compiler was originally invoked.
% The content of |\jobname| cannot be compared
% to filenames specified in the source due to different catcodes.
% The following code rescans |\jobname|, stores the result
% in |\childdocname| and saves a copy in |\childdocjob|:
%    \begin{macrocode}
\edef\childdocname{\scantokens\expandafter{\jobname\noexpand}}
\let\childdocjob\childdocname
%    \end{macrocode}

% \macro{\childdocdisable}
% The macro |\childdocdisable| prevents the main file
% from being processed more than once.
% At this stage, the main document command |\childdocmain|
% is assumed to be called once again where it should do nothing.
% Any subsequent call to it should prevent
% a secondary processing of the main document
% It overwrites the forwarding commands
% |\childdocof| and |\childdocforward|
% with empty macros to prevent further inclusions of the main document:
%    \begin{macrocode}
\newcommand{\childdocdisable}
{
  \renewcommand{\childdocmain}[1]{\renewcommand{\childdocmain}[1]{\endinput}}
  \renewcommand{\childdocof}[1]{}
  \renewcommand{\childdocby}[2][]{}
  \renewcommand{\childdocforward}[2][]{}
  \renewcommand{\childdocdisable}{}
}
%    \end{macrocode}

% \macro{\childdocmain}
% The macro |\childdocmain| is to be called at the top of the main file
% with nothing or the main filename (without extension) as argument.
% First, it breaks loops.
% If the argument is not empty and does not match |\childdocname|
% (which is set by the first inclusion of |childdoc.def|),
% |\ifchilddoc| is set to true, |\includeonly| is applied to the child file
% and |\jobname| is set to the main file
% (for proper handling of |.aux| files):
%    \begin{macrocode}
\newcommand{\childdocmain}[1]
{
  \childdocdisable\childdocmain{}
  \if?#1?\else
    \begingroup
      \def\childdoctmp{#1}
      \ifx\childdoctmp\childdocname
        \def\childdoctmp{}
      \else
        \def\childdoctmp
        {
          \childdoctrue
          \includeonly{\childdocname}
          \def\childdocjob{#1}
          \def\jobname{#1}
        }
      \fi
      \expandafter
    \endgroup
    \childdoctmp
  \fi
}
%    \end{macrocode}

% \macro{\childdocof}
% The command |\childdocof| redirects
% compilation to the main file |#1|.
%    \begin{macrocode}
\newcommand{\childdocof}[1]
{
  \childdocdisable
  \childdoctrue
  \includeonly{\childdocname}
  \def\jobname{#1}
  \def\childdocjob{#1}
  \input{#1}
}
%    \end{macrocode}

% \macro{\childdocby}
% The command |\childdocby| ....
%    \begin{macrocode}
\newcommand{\childdocby}[2][]
{
  \childdocdisable
  \childdoctrue
  \childdocmanualtrue
  \if?#1?\else
    \def\jobname{#2}
  \fi
  \def\childdocjob{#2}
  \input{#2}
  \endinput
}
%    \end{macrocode}

% \macro{\childdocforward}
% The command |\childdocforward| redirects
% compilation to the main file or
% (if the optional argument is given) a child file.
% Parameters are set as if the main file
% or a child file starting with |\childdocof| was compiled.
% Then compilation is handed over to the main file:
%    \begin{macrocode}
\newcommand{\childdocforward}[2][]
{
  \begingroup
    \if?#1?
      \def\childdoctmp
      {
        \def\childdocname{#2}
        \def\childdocjob{#2}
        \def\jobname{#2}
        \input{#2}
        \endinput
      }
    \else
      \def\childdoctmp
      {
        \childdocdisable
        \def\childdocname{#2}
        \childdoctrue
        \includeonly{#2}
        \def\childdocjob{#1}
        \def\jobname{#1}
        \input{#1}
        \endinput
      }
    \fi
    \expandafter
  \endgroup
  \childdoctmp
}
%    \end{macrocode}

% \macro{\childdocforwardprefix}
% The command |\childdocforwardprefix| redirects
% compilation to the main or a child file by means of a pattern.
% The prefix |#1| in the current filename is replaced by |#2|
% and the suffix of the current filename is kept
% (it is assumed that the filename does not contain the substring `|~~~|'
% which is used as a delimiter).
% Compilation is handed over to the new file by |\childdocforward|:
%    \begin{macrocode}
\newcommand{\childdocforwardprefix}[3][]
{
  \begingroup
    \def\childdocextract #2##1~~~{\def\childdoctmp{\childdocforward[#1]{#3##1}}}
    \expandafter\childdocextract\childdocname~~~
    \expandafter
  \endgroup
  \childdoctmp
}
%    \end{macrocode}

% \macro{\childdoc}
% The deprecated macro |\childdoc| is a legacy version of |\childdocmain|:
%    \begin{macrocode}
\newcommand{\childdoc}{\childdocmain}
%    \end{macrocode}

% \macro{\childdocredirect}
% The deprecated macro |\childdocredirect| is a legacy version
% of |\childdocforward| and |\childdocforwardprefix|:
%    \begin{macrocode}
\newcommand{\childdocredirect}[2][]
{
  \begingroup
    \if?#1?
      \def\childdoctmp{\childdocforward{#2}}
    \else
      \def\childdoctmp{\childdocforwardprefix{#1}{#2}}
    \fi
    \expandafter
  \endgroup
  \childdoctmp
}
%    \end{macrocode}

%\iffalse
%</package>
%\fi
%
\endinput

\childdocforwardprefix[cdocsamp]{cdocsfn}{cdocsch}
%    \end{macrocode}

%\iffalse
%</samplefinal>
%\fi
%
% %%%%%%%%%%%%%%%%%%%%%%%%%%%%%%%%%%%%%%
% \paragraph{Command Line Processing.}
%
% The following three command lines generate the output files
% |cdocscld|, |cdocscl1| and |cdocscl2|
% which should be identical to
% |cdocsdrf|, |cdocsch1| and |cdocsfn2|, respectively:
% \begin{center}
% \begin{tabular}{l}
% |latex -jobname cdocscld \|\\
% |  "\def\version{draft}% \iffalse
%
% childdoc.dtx Copyright (C) 2017-2018 Niklas Beisert
%
% This work may be distributed and/or modified under the
% conditions of the LaTeX Project Public License, either version 1.3
% of this license or (at your option) any later version.
% The latest version of this license is in
%   http://www.latex-project.org/lppl.txt
% and version 1.3 or later is part of all distributions of LaTeX
% version 2005/12/01 or later.
%
% This work has the LPPL maintenance status `maintained'.
%
% The Current Maintainer of this work is Niklas Beisert.
%
% This work consists of the files childdoc.dtx and childdoc.ins
% and the derived files childdoc.def and cdocsamp.tex with
% cdocsch1.tex, cdocsch2.tex, cdocsdrf.tex, cdocsfn1.tex, cdocsfn2.tex.
%
%<package>\ifdefined\childdocmain\endinput\fi
%<package>\ProvidesFile{childdoc.def}[2018/12/30 v2.0 child document driver]
%<samplemain>\ProvidesFile{cdocsamp.tex}[2018/12/30 v2.0 sample for childdoc]
%<*driver>
%\ProvidesFile{childdoc.drv}[2018/12/30 v2.0 childdoc reference manual file]
\PassOptionsToClass{10pt,a4paper}{article}
\documentclass{ltxdoc}

\usepackage[margin=35mm]{geometry}
\usepackage{hyperref}
\usepackage{hyperxmp}
\usepackage[usenames]{color}

\hypersetup{colorlinks=true}
\hypersetup{pdfstartview=FitH}
\hypersetup{pdfpagemode=UseNone}
\hypersetup{pdfsource={}}
\hypersetup{pdflang={en-UK}}
\hypersetup{pdfcopyright={Copyright 2017-2018 Niklas Beisert.
  This work may be distributed and/or modified under the
  conditions of the LaTeX Project Public License, either version 1.3
  of this license or (at your option) any later version.}}
\hypersetup{pdflicenseurl={http://www.latex-project.org/lppl.txt}}
\hypersetup{pdfcontactaddress={ETH Zurich, ITP, HIT K,
  Wolfgang-Pauli-Strasse 27}}
\hypersetup{pdfcontactpostcode={8093}}
\hypersetup{pdfcontactcity={Zurich}}
\hypersetup{pdfcontactcountry={Switzerland}}
\hypersetup{pdfcontactemail={nbeisert@itp.phys.ethz.ch}}
\hypersetup{pdfcontacturl={http://people.phys.ethz.ch/\xmptilde nbeisert/}}

\newcommand{\secref}[1]{\hyperref[#1]{section \ref*{#1}}}

\parskip1ex
\parindent0pt
\let\olditemize\itemize
\def\itemize{\olditemize\parskip0pt}

\begin{document}

\title{The \textsf{childdoc} Package}
\hypersetup{pdftitle={The childdoc Package}}
\author{Niklas Beisert\\[2ex]
  Institut f\"ur Theoretische Physik\\
  Eidgen\"ossische Technische Hochschule Z\"urich\\
  Wolfgang-Pauli-Strasse 27, 8093 Z\"urich, Switzerland\\[1ex]
  \href{mailto:nbeisert@itp.phys.ethz.ch}
  {\texttt{nbeisert@itp.phys.ethz.ch}}}
\hypersetup{pdfauthor={Niklas Beisert}}
\hypersetup{pdfsubject={Manual for the LaTeX2e Package childdoc}}
\date{30 December 2018, \textsf{v2.0}}
\maketitle

\begin{abstract}\noindent
\textsf{childdoc} is a \LaTeXe{} package
that enables the direct compilation
of document sections included by |\include|
to individual files.
\end{abstract}

\begingroup
\parskip0ex
\tableofcontents
\endgroup

%%%%%%%%%%%%%%%%%%%%%%%%%%%%%%%%%%%%%%%%%%%%%%%%%%%%%%%%%%%%%%%%%%%%%%%%%%%%%%%%
%%%%%%%%%%%%%%%%%%%%%%%%%%%%%%%%%%%%%%%%%%%%%%%%%%%%%%%%%%%%%%%%%%%%%%%%%%%%%%%%
\section{Introduction}

\LaTeX{} provides a mechanism to structure a large document (such as a book)
into a main file and several child files (containing the chapters)
using the |\include| command.
This mechanism is beneficial for documents
which span hundreds of pages in order to
make the source file(s) more manageable.
Moreover, compilation can be restricted to
selected child files by means of the |\includeonly| command.
The latter feature can be used to reduce the compilation time while editing
(this was significantly more useful in the earlier days of \LaTeX{})
or to generate a smaller document which is easier to navigate.
Another application of |\includeonly| is to generate
documents consisting of selected parts of the complete document.

However, there are a few drawbacks of the plain |\include| mechanism:
\begin{itemize}
\item
The child files cannot be compiled on their own,
they can only be compiled via the main file.
A naive editing environment
(such as a text editor with an option
to have the current file processed by \LaTeX)
may require one to switch to the main file before compiling;
attempting to compile the child file produces errors.
\item
The main file must be modified (each time)
to adjust the |\includeonly| command
to the present needs. This easily leaves the main file in a messy state.
\item
The generated document will always carry the filename
of the main document. This is inconvenient if
several child files are to be compiled and
to be kept for distribution.
\end{itemize}

The present package provides a simple interface
to make child files individually compilable by \LaTeX{}.
Compiling a child file then has the same effect as compiling
the main file with an |\includeonly| command
to select the appropriate child.
Moreover the generated document will carry the name of the child
rather than the main file.
This resolves all three above issues.

This feature is meant to make the editing of books,
thesis documents and lecture notes somewhat more convenient.
However, the package can also be used efficiently for
composing a series of documents (such as exercise sheets)
which are typically distributed individually.
It then assists the author in generating the individual documents
(potentially in different versions)
as well as a document containing the collected series.
Another application is in developing style files
or other kinds of included material
where compilation of the style file could redirect
to a sample or test file.

%%%%%%%%%%%%%%%%%%%%%%%%%%%%%%%%%%%%%%%%%%%%%%%%%%%%%%%%%%%%%%%%%%%%%%%%%%%%%%%%
%%%%%%%%%%%%%%%%%%%%%%%%%%%%%%%%%%%%%%%%%%%%%%%%%%%%%%%%%%%%%%%%%%%%%%%%%%%%%%%%
\section{Usage}

First of all, the package \textsf{childdoc} is \emph{not} a standard
\LaTeXe{} |.sty| style file! Therefore it needs to be invoked in
a non-standard way.

%%%%%%%%%%%%%%%%%%%%%%%%%%%%%%%%%%%%%%%%%%%%%%%%%%%%%%%%%%%%%%%%%%%%%%%%%%%%%%%%
\subsection{Included Files}
\label{sec:include}

%%%%%%%%%%%%%%%%%%%%%%%%%%%%%%%%%%%%%%%%
\DescribeMacro{\childdocmain}
To use the package, add the commands
\begin{center}
\begin{tabular}{l}
|\input{childdoc.def}|\\
|\childdocmain{}|\\
\end{tabular}
\end{center}
at the very top of the main \LaTeX{} file,
in particular \emph{before} the |\documentclass| statement!
The argument of |\childdocmain| should be left empty
(but it must be present).

%%%%%%%%%%%%%%%%%%%%%%%%%%%%%%%%%%%%%%%%
\DescribeMacro{\childdocof}
Furthermore, add the commands
\begin{center}
\begin{tabular}{l}
|\input{childdoc.def}|\\
|\childdocof{|\textit{main}|}|\\
\end{tabular}
\end{center}
at the top of every child file \textit{child}
which is included by |\include{|\textit{child}|}|
from within the main file
(or at least for those files to be compiled individually).
The argument \textit{main} must be the filename of the main file.

There are a couple of
considerations in setting up the main and child documents:

%%%%%%%%%%%%%%%%%%%%%%%%%%%%%%%%%%%%%%%%
\paragraph{Restrictions.}

Please note the following restrictions:
\begin{itemize}
\item
|\childdocmain| must be called with one argument \textit{main}
to ensure compatibility with earlier version of the package.
It must either be empty (|\childdocmain{}|)
or precisely match the filename of the main file in which it is specified.
See \secref{sec:detection} for further information.
\item
The filename \textit{main} must be specified without the |.tex| extension.
\item
The filename \textit{main} is case sensitive
(even in case-insensitive file systems)
due to internal string comparison.
\item
The argument \textit{main} should be fully expanded, it cannot be a macro.
\item
Subdirectories and special characters should be avoided in filenames.
\item
The command |\childdocmain{|\textit{main}|}| must be followed by a whitespace.
It should not be followed immediately by another command
or by a comment mark `|%|'.
This is because the \TeX{} parser reads the token immediately following
the argument of |\childdocmain| and puts it
at the beginning of every child section;
however, a white\-space is ignored.
\end{itemize}

%%%%%%%%%%%%%%%%%%%%%%%%%%%%%%%%%%%%%%%%
\paragraph{Content of Main File.}

It is advisable to place all content in the child files included by |\include|.
Any output contained in the main file will appear in all child documents
unless suppressed manually;
it cannot be suppressed automatically by the |\includeonly| directive
and thus should normally be avoided.
A method to include some content in the main file
by means of conditional processing is described in \secref{sec:conditional}.

%%%%%%%%%%%%%%%%%%%%%%%%%%%%%%%%%%%%%%%%
\paragraph{Page Numbering.}

When only a part of the document is compiled,
the appropriate numbering of pages
(as well as other status parameters)
is determined from the |.aux| files.
The latter contain information from previous passes.
However this information needs to propagate through
all intermediate child documents.
Therefore the page numbering in child documents may well
be inconsistent until the complete document is compiled at least once.

A useful (if unconventional) way to always ensure a consistent
page numbering is to restart the numbering in each child document
and denote the pages by `\textit{child}|.|\textit{page}'
where \textit{child} represents the chapter/section number of the child file.
This can be achieved by the command
|\numberwithin{page}{|\textit{child}|}|
of the \textsf{amsmath} package
where \textit{child} can be |chapter| or |section|
depending on the chosen structuring.
Alternatively, one can modify the macro |\thepage| appropriately
and reset the counter |page| at the start of each child file.

%%%%%%%%%%%%%%%%%%%%%%%%%%%%%%%%%%%%%%%%%%%%%%%%%%%%%%%%%%%%%%%%%%%%%%%%%%%%%%%%
\subsection{Conditional Processing}
\label{sec:conditional}

The package provides a mechanism to compile different versions
of a document. To customise the versions further some conditional processing
can come in handy to distinguish which version is being compiled.
The package provides two macros to describe the compilation context:

%%%%%%%%%%%%%%%%%%%%%%%%%%%%%%%%%%%%%%%%
\DescribeMacro{\ifchilddoc}
The conditional |\ifchilddoc| distinguishes between the compilation of
child documents and the main document:
%
\begin{center}
|\ifchilddoc |\textit{child-code}| |[|\||else |\textit{main-code}]| \||fi|
\end{center}

%%%%%%%%%%%%%%%%%%%%%%%%%%%%%%%%%%%%%%%%
\DescribeMacro{\childdocname}
\DescribeMacro{\childdocjob}
The macro |\childdocname| contains the filename (without extension)
of the main or child file being processed.
Note that |\childdocjob| will always contain the name of the main file.

%%%%%%%%%%%%%%%%%%%%%%%%%%%%%%%%%%%%%%%%
\paragraph{Title Page.}

Conditional processing can be used to include a title or banner page
in the main document when proper precautions are taken.
Importantly, the code in the main file should ensure that the page counter
(as well as other status parameters which are stored in the |.aux| files)
takes the same value after the conditional processing.
Otherwise the page numbers may take divergent values
depending on which part is compiled.

For example, a title page could be declared by:
%
\begin{center}
\begin{tabular}{l}
|\ifchilddoc\||else|\\
|\addtocounter{page}{-1}|\\
\textit{code for title page}\\
|\newpage|\\
|\||fi|
\end{tabular}
\end{center}
%
A banner page for the child documents can be generated by:
%
\begin{center}
\begin{tabular}{l}
|\ifchilddoc|\\
|\addtocounter{page}{-1}|\\
\textit{code for banner page}\\
|\newpage|\\
|\||fi|
\end{tabular}
\end{center}
%
Here one could write a message such as:
\begin{center}
|This is the part \childdocname{} of \childdocjob{}.|
\end{center}

%%%%%%%%%%%%%%%%%%%%%%%%%%%%%%%%%%%%%%%%%%%%%%%%%%%%%%%%%%%%%%%%%%%%%%%%%%%%%%%%
\subsection{Flags}
\label{sec:flags}

The package makes it easy to generate different versions
of the main or child documents.
To this end compilation flags can be defined
and assigned different default values.
They will be particularly useful in conjunction
with the forwarding mechanism described in \secref{sec:forward}.

For example, it may be useful to have a flag |\version|
which can be set to |draft| or |final|.
The document source will contain some conditional code
depending on the value of |\version|.
Suppose further, the flag should default to |final| for the main file
and to |draft| for child files
which is a natural assignment for editing the document.
This is achieved by placing the following code
in the preamble of the main document
(below the |\childdocmain| directive):
%
\begin{center}
\begin{tabular}{l}
|\ifchilddoc|\\
|\providecommand{\version}{draft}|\\
|\||else|\\
|\providecommand{\version}{final}|\\
|\||fi|
\end{tabular}
\end{center}
%
The definition by |\providecommand| makes sure
that previous definitions are not overwritten.
Further statements |\providecommand{\version}{...}|
can thus be added before the above code to override it.

For the main file, one might add a line
(between |\childdocmain| and the above block)
%
\begin{center}
|%\ifchilddoc\||else\providecommand{\version}{draft}\||fi|
\end{center}
%
which can be uncommented to produce a draft version.
Likewise one can add a line to the very top of a child file
(above the |\childdocof{|\textit{main}|}| directive)
%
\begin{center}
|%\providecommand{\version}{final}|
\end{center}
%
which can be uncommented to produce the final version of this child document.

%%%%%%%%%%%%%%%%%%%%%%%%%%%%%%%%%%%%%%%%%%%%%%%%%%%%%%%%%%%%%%%%%%%%%%%%%%%%%%%%
\subsection{Forwarding}
\label{sec:forward}

Different versions of the main or child documents
using compilation flags as described in \secref{sec:flags}
can be (permanently) stored in different files
for convenient compilation, viewing and distribution.
To this end, the package defines a command
to pass on compilation to a different file:

%%%%%%%%%%%%%%%%%%%%%%%%%%%%%%%%%%%%%%%%
\DescribeMacro{\childdocforward}
The command |\childdocforward| redirects processing to
another source file:
%
\begin{center}
\begin{tabular}{l}
|\input{childdoc.def}|\\
|\childdocforward[|\textit{main}|]{|\textit{dest}|}|\\
\end{tabular}
\end{center}
%
The argument \textit{dest} is the destination file
(without extension).
It should be the main file or one of the child files.
Note that further \textsf{childdoc} directives
such as |\childdocof| and |\childdocforward|
in the indicated file will be processed in this form.
The optional argument \textit{main}
passes on directly to the main file \textit{main}
while pretending to compile the child \textit{dest}.
This form behaves as if \textit{dest}
issues |\childdocof{|\textit{main}|}| right away,
and no further \textsf{childdoc} directives will be processed.

%%%%%%%%%%%%%%%%%%%%%%%%%%%%%%%%%%%%%%%%
\DescribeMacro{\...prefix}
In the alternative form |\childdocforwardprefix|,
%
\begin{center}
\begin{tabular}{l}
|\input{childdoc.def}|\\
|\childdocforwardprefix[|\textit{main}|]{|\textit{prefix}|}{|\textit{dest}|}|
\end{tabular}
\end{center}
%
the destination file is determined by a pattern
depending on the current file:
To make this work, the current file must be called
`{\textit{prefix}\hspace{0.2em}\textit{suffix}}'
with \textit{prefix} matching precisely the argument.
Processing is then passed on to the file
`{\textit{dest}\hspace{0.2em}\textit{suffix}}'.
Surely, the same effect is achieved by
directly specifying the
argument `{\textit{dest}\hspace{0.2em}\textit{suffix}}'
in the first form.
However, that requires to set up a different file
for each child. With the alternative form of the command
all these files can have exactly the same content
which simplifies setting them up and maintaining them.

For example, the following file |draft.tex|
with a compilation flag |\version| as described in \secref{sec:flags}
compiles the main document as a draft:
%
\begin{center}
\begin{tabular}{l}
|\def\version{draft}|\\
|\input{childdoc.def}|\\
|\childdocforward{|\textit{main}|}|
\end{tabular}
\end{center}
%
Likewise, the following files |final|\textit{nn}|.tex|
compile the final version of the child document
|child|\textit{nn}|.tex|:
%
\begin{center}
\begin{tabular}{l}
|\def\version{final}|\\
|\input{childdoc.def}|\\
|\childdocforwardprefix{final}{child}|
\end{tabular}
\end{center}
%

Note that when several versions of a main file and/or of each child file
are to be generated, it may be convenient to set up a |Makefile| or
shell script to automatise the process.

%%%%%%%%%%%%%%%%%%%%%%%%%%%%%%%%%%%%%%%%%%%%%%%%%%%%%%%%%%%%%%%%%%%%%%%%%%%%%%%%
\subsection{Command Line Processing}
\label{sec:commandline}

The effect of redirection files can also be achieved by invoking
the \LaTeX{} compiler with a more elaborate command line.
Most conveniently this should be done as part
of a shell script or a |Makefile|.

When using \textsf{childdoc} in the main file, the following
command lines effectively perform a redirection
(note that depending on the shell being used,
backslashes may have to be doubled: `|\|' $\to$ `|\\|'):
%
\begin{center}
|... -jobname "|\textit{target}|" |\\|"|[\textit{flags}]%
|\input{childdoc.def}\childdocforward[|\textit{main}|]{|\textit{dest}|}"|
\end{center}
%
Here \textit{target} is the name of the output file,
\textit{main} is the name of the main file
and \textit{dest} is the name of the main or child file to be processed
(all filenames without extensions).
The optional argument \textit{main} can be omitted
if \textit{main} matches \textit{dest}.
Optionally, compilation \textit{flags} can be defined via |\def| commands.
This command line makes the \TeX{} engine believe
it is compiling the file \textit{target}
whose content is specified as the latter parameter.
The provided code then forwards the processing to
\textit{main} or \textit{dest} as described in \secref{sec:forward}.

%%%%%%%%%%%%%%%%%%%%%%%%%%%%%%%%%%%%%%%%%%%%%%%%%%%%%%%%%%%%%%%%%%%%%%%%%%%%%%%%
\subsection{Include by Input}
\label{sec:input}

Including child documents by |\include| has some restrictions by design.
Most notably, the content of a child document always occupies
its own set of pages; pages cannot be shared between child documents.
Usually, this behaviour makes perfect sense
because each child document contain an essential part of the document.
However, in some situations it may be desirable to compose
a document from a collection of parts
without having mandatory page breaks between then.
For this case, the package
provides a mechanism to include parts
by |\input| which can also be processed individually.
However, by construction this mechanism
requires manual handling of the content to be output.

%%%%%%%%%%%%%%%%%%%%%%%%%%%%%%%%%%%%%%%%
\DescribeMacro{\ifchilddocmanual}
The main file should be prepared as usual, see \secref{sec:include}.
However, the document body must make a distinction
between processing of an individual part and of the main document, e.g.:
%
\begin{center}
\begin{tabular}{l}
|\ifchilddocmanual|\\
|\input{\childdocname}|\\
|\||else|\\
\textit{document body with }|\input{|\textit{part}|}|\\
|\||fi|
\end{tabular}
\end{center}
%
The conditional |\ifchilddocmanual| is true whenever
a part to be included by |\input| is being compiled,
and the name of the part is stored in |\childdocname|.

%%%%%%%%%%%%%%%%%%%%%%%%%%%%%%%%%%%%%%%%
\DescribeMacro{\childdocby}
Each part to be included by |\input| should start with:
%
\begin{center}
\begin{tabular}{l}
|\input{childdoc.def}|\\
|\childdocby{|\textit{main}|}|\\
\end{tabular}
\end{center}
%
The directive |\childdocby| is similar to |\childdocof|
described in \secref{sec:include},
but the subsequent selection of content must be done manually.
To that end, both |\ifchilddoc| and |\ifchilddocmanual|
will be true upon processing of a part,
and the name of the part is stored in |\childdocname|.
Note that |\jobname| will be set to the filename of the current part
so that each part receives an individual |.aux| file
that does not interfere with the |.aux| file(s) of the main document.
This behaviour can be altered by the alternative form
|\childdocby[*]{|\textit{main}|}| (with a non-empty optional argument)
which uses the |.aux| file of the main document
by setting |\jobname| to \textit{main}.

%%%%%%%%%%%%%%%%%%%%%%%%%%%%%%%%%%%%%%%%%%%%%%%%%%%%%%%%%%%%%%%%%%%%%%%%%%%%%%%%
\subsection{Driver Development}
\label{sec:driver}

The \textsf{childdoc} mechanism can also be use for the development
of definition files such as \LaTeX{} styles or classes.
This case differs from the above setup with multiple parts
included by |\include| in that no |\includeonly| should be invoked.
This can be achieved by starting the include file
(before |\ProvidesPackage|) with:
%
\begin{center}
\begin{tabular}{l}
|\input{childdoc.def}|\\
|\childdocforward{|\textit{main}|}|\\
\end{tabular}
\end{center}
%
or alternatively with:
%
\begin{center}
\begin{tabular}{l}
|\input{childdoc.def}|\\
|\childdocby{|\textit{main}|}|\\
\end{tabular}
\end{center}
%
Both forms have slightly different effects as described above.
The main file is prepared as usual, see \secref{sec:include}.

%%%%%%%%%%%%%%%%%%%%%%%%%%%%%%%%%%%%%%%%%%%%%%%%%%%%%%%%%%%%%%%%%%%%%%%%%%%%%%%%
\subsection{Legacy Detection}
\label{sec:detection}

The directive |\childdocmain| in the main file can detect
whether the complete document or merely a child is to be compiled
even without using the directive |\childdocof|.
This method is deprecated because it is less robust
and there is no compelling reason to use it;
it is merely provided for backward compatibility
and it may be removed in future versions.

If the detection mechanism is to be used,
it is mandatory to correctly specify
the filename of the main file as the argument of |\childdocmain|:
%
\begin{center}
\begin{tabular}{l}
|\input{childdoc.def}|\\
|\childdocmain{|\textit{main}|}|\\
\end{tabular}
\end{center}
%
If |\jobname| does not match the argument \textit{main} of |\childdocmain|,
it is assumed that |\jobname| points to the child file to be compiled.
When using |\childdocmain| with the main file specified as argument,
it suffices to start a child file
with just |\input{|\textit{main}|}|
without loading of the package and using |\childdocof|.
If instead all processing is done
with the appropriate \textsf{childdoc} directives,
the argument of \textit{main} of |\childdocmain| can be empty.

An alternative version of the command line processing described
in \secref{sec:commandline} using the detection mechanism reads:
%
\begin{center}
|... -jobname "|\textit{target}|" "|[\textit{flags}]%
[|\def\jobname{|\textit{dest}|}|]|\input{|\textit{main}|}"|
\end{center}

%%%%%%%%%%%%%%%%%%%%%%%%%%%%%%%%%%%%%%%%%%%%%%%%%%%%%%%%%%%%%%%%%%%%%%%%%%%%%%%%
\subsection{Manual Code}
\label{sec:manual}

In case one cannot be certain whether the definitions file |childdoc.def|
is installed on the target \TeX{} distribution
and one prefers not to ship it,
it is conceivable to paste a few relevant commands into the sources.

To that end, drop all statements |\input{childdoc.def}|
and perform the replacements as outlined below.
Instead of |\childdocmain{|\textit{main}|}| add the following code
to the top of the main file:
%
\begin{center}
\begin{tabular}{l}
|\||ifdefined\childdocname\endinput\||fi\newif\ifchilddoc|\\
|\edef\childdocname{\scantokens\expandafter{\jobname\noexpand}}|\\
|\def\childdocmain{|\textit{main}|}\||ifx\childdocmain\childdocname\||else|\\
|\childdoctrue\includeonly{\childdocname}\let\jobname\childdocmain\||fi|\\
\end{tabular}
\end{center}
%
Instead of |\childdocof{|\textit{main}|}| just include the main file
at the top of each child file:
%
\begin{center}
|\input{|\textit{main}|}|
\end{center}
%
A simple redirection |\childdocforward{|\textit{dest}|}| is achieved by:
%
\begin{center}
|\def\jobname{|\textit{dest}|}\input{\jobname}|
\end{center}
%
The redirection with prefix
|\childdocforwardprefix[|\textit{prefix}|]{|\textit{dest}|}|
is accomplished by:
%
\begin{center}
\begin{tabular}{l}
|{\edef\jobname{\scantokens\expandafter{\jobname\noexpand}}|\\
|\def\redirectjob |\textit{prefix}|#1~~~{\gdef\jobname{|\textit{dest}|#1}}|\\
|\expandafter\redirectjob\jobname~~~}\input{\jobname}|
\end{tabular}
\end{center}

In an alternative approach,
child documents can be compiled by a specific command line
without additional code or specific definitions:
%
\begin{center}
|... -jobname "|\textit{target}|" "|[\textit{flags}]%
|\includeonly{|\textit{dest}|}\input{|\textit{main}|}"|
\end{center}
%

%%%%%%%%%%%%%%%%%%%%%%%%%%%%%%%%%%%%%%%%%%%%%%%%%%%%%%%%%%%%%%%%%%%%%%%%%%%%%%%%
%%%%%%%%%%%%%%%%%%%%%%%%%%%%%%%%%%%%%%%%%%%%%%%%%%%%%%%%%%%%%%%%%%%%%%%%%%%%%%%%
\section{Information}

%%%%%%%%%%%%%%%%%%%%%%%%%%%%%%%%%%%%%%%%%%%%%%%%%%%%%%%%%%%%%%%%%%%%%%%%%%%%%%%%
\subsection{Copyright}

Copyright \copyright{} 2017--2018 Niklas Beisert

This work may be distributed and/or modified under the
conditions of the \LaTeX{} Project Public License, either version 1.3
of this license or (at your option) any later version.
The latest version of this license is in
  \url{http://www.latex-project.org/lppl.txt}
and version 1.3 or later is part of all distributions of \LaTeX{}
version 2005/12/01 or later.

This work has the LPPL maintenance status `maintained'.

The Current Maintainer of this work is Niklas Beisert.

This work consists of the files |README.txt|, |childdoc.ins| and |childdoc.dtx|
as well as the derived files |childdoc.def|, |cdocsamp.tex|
with |cdocsch1.tex|, |cdocsch2.tex|, |cdocspt3.tex|, |cdocspt4.tex|,
|cdocsdrf.tex|, |cdocsfn1.tex|, |cdocsfn2.tex|
as well as |childdoc.pdf|.

%%%%%%%%%%%%%%%%%%%%%%%%%%%%%%%%%%%%%%%%%%%%%%%%%%%%%%%%%%%%%%%%%%%%%%%%%%%%%%%%
\subsection{Files and Installation}

The package consists of the files:
%
\begin{center}
\begin{tabular}{ll}
    |README.txt|   & readme file \\
    |childdoc.ins| & installation file \\
    |childdoc.dtx| & source file \\
    |childdoc.def| & definition file \\
    |cdocsamp.tex| & sample main file \\
    |cdocsch1.tex| & sample include file \\
    |cdocsch2.tex| & sample include file \\
    |cdocspt3.tex| & sample part file \\
    |cdocspt4.tex| & sample part file \\
    |cdocsdrf.tex| & sample redirection file \\
    |cdocsfn1.tex| & sample redirection file \\
    |cdocsfn2.tex| & sample redirection file \\
    |childdoc.pdf| & manual
\end{tabular}
\end{center}
%
The distribution consists of the files
|README.txt|, |childdoc.ins| and |childdoc.dtx|.
%
\begin{itemize}
\item
Run (pdf)\LaTeX{} on |childdoc.dtx|
to compile the manual |childdoc.pdf| (this file).
\item
Run \LaTeX{} on |childdoc.ins| to create the definitions file |childdoc.def|
and the sample |cdocsamp.tex| with include files
|cdocsch1.tex|, |cdocsch2.tex|, |cdocspt3.tex|, |cdocspt4.tex|,
|cdocsdrf.tex|, |cdocsfn1.tex|, |cdocsfn2.tex|.
Then copy the file |childdoc.def| to an appropriate directory of your \LaTeX{}
distribution, e.g.\ \textit{texmf-root}|/tex/latex/childdoc|.
\end{itemize}

%%%%%%%%%%%%%%%%%%%%%%%%%%%%%%%%%%%%%%%%%%%%%%%%%%%%%%%%%%%%%%%%%%%%%%%%%%%%%%%%
\subsection{Related CTAN Packages}

There are several other packages which offer a similar functionality:
%
\begin{itemize}
\item
The packages
\href{http://ctan.org/pkg/docmute}{\textsf{docmute}},
\href{http://ctan.org/pkg/includex}{\textsf{includex}} and
\href{http://ctan.org/pkg/standalone}{\textsf{standalone}}
provide commands to include only the document body of
a child file thus allowing both files to be compiled individually.
\item
The packages \href{http://ctan.org/pkg/subdocs}{\textsf{subdocs}}
and \href{http://ctan.org/pkg/subfiles}{\textsf{subfiles}}
provide structures in which the main and child documents can be
encapsulated and allowing them to be compiled individually.
The inclusion mechanism is different from the conventional |\include|.
\item
The package \href{http://ctan.org/pkg/combine}{\textsf{combine}}
is an elaborate solution to combine several documents into one.
\end{itemize}
%
See also the CTAN topic \href{http://ctan.org/topic/subdocs}{\textsf{subdocs}}
for further related packages.
The present package differs from the above solutions in that
a document structure constructed with the conventional |\include| mechanism
just needs two extra commands at the top of every file
such that all constituent files can be compiled individually.

%%%%%%%%%%%%%%%%%%%%%%%%%%%%%%%%%%%%%%%%%%%%%%%%%%%%%%%%%%%%%%%%%%%%%%%%%%%%%%%%
%\subsection{Feature Suggestions}
%
%The following is a list of features which may be useful for future
%versions of this package:
%%
%\begin{itemize}
%\item
%\ldots
%\end{itemize}

%%%%%%%%%%%%%%%%%%%%%%%%%%%%%%%%%%%%%%%%%%%%%%%%%%%%%%%%%%%%%%%%%%%%%%%%%%%%%%%%
\subsection{Revision History}

%%%%%%%%%%%%%%%%%%%%%%%%%%%%%%%%%%%%%%%%
\paragraph{v2.0:} 2018/12/30

\begin{itemize}
\item
immediate forward processing
\item
added |\childdocby| mechanism
\item
manual restructured
\end{itemize}

%%%%%%%%%%%%%%%%%%%%%%%%%%%%%%%%%%%%%%%%
\paragraph{v1.6:} 2018/01/17

\begin{itemize}
\item
application for development of include files
\item
corrections to manual
\end{itemize}

%%%%%%%%%%%%%%%%%%%%%%%%%%%%%%%%%%%%%%%%
\paragraph{v1.5:} 2017/05/21

\begin{itemize}
\item
more complete structuring introduced
\item
|\childdocof| introduced
\item
|\childdoc| renamed to |\childdocmain|
\item
|\childredirect| renamed to |\childdocforward| and |\childdocforwardprefix|
and functionality expanded
\end{itemize}

%%%%%%%%%%%%%%%%%%%%%%%%%%%%%%%%%%%%%%%%
\paragraph{v1.0:} 2017/04/27

\begin{itemize}
\item
manual and install package
\item
first version published on CTAN
\end{itemize}

%%%%%%%%%%%%%%%%%%%%%%%%%%%%%%%%%%%%%%%%
\paragraph{v0.6:} 2017/04/26

\begin{itemize}
\item
redirection mechanism added
\end{itemize}

%%%%%%%%%%%%%%%%%%%%%%%%%%%%%%%%%%%%%%%%
\paragraph{v0.5:} 2017/04/26

\begin{itemize}
\item
functionality in definition file
\end{itemize}


%%%%%%%%%%%%%%%%%%%%%%%%%%%%%%%%%%%%%%%%%%%%%%%%%%%%%%%%%%%%%%%%%%%%%%%%%%%%%%%%
%%%%%%%%%%%%%%%%%%%%%%%%%%%%%%%%%%%%%%%%%%%%%%%%%%%%%%%%%%%%%%%%%%%%%%%%%%%%%%%%
%%%%%%%%%%%%%%%%%%%%%%%%%%%%%%%%%%%%%%%%%%%%%%%%%%%%%%%%%%%%%%%%%%%%%%%%%%%%%%%%
\appendix

\settowidth\MacroIndent{\rmfamily\scriptsize 000\ }

 \DocInput{childdoc.dtx}

\end{document}
%</driver>
% \fi
%
% %%%%%%%%%%%%%%%%%%%%%%%%%%%%%%%%%%%%%%%%%%%%%%%%%%%%%%%%%%%%%%%%%%%%%%%%%%%%%%
% %%%%%%%%%%%%%%%%%%%%%%%%%%%%%%%%%%%%%%%%%%%%%%%%%%%%%%%%%%%%%%%%%%%%%%%%%%%%%%
% \section{Sample}
%\iffalse
%<*samplemain>
%\fi
%
% The following presents a sample document
% with two chapters, two parts, a title page,
% a compile flag as well as three forwarding files to set the flag.
% It consists of eight |.tex| files:
% \begin{center}
% \begin{tabular}{ll}
% |cdocsamp.tex|&main file\\
% |cdocsch1.tex|&include file for chapter 1\\
% |cdocsch2.tex|&include file for chapter 2\\
% |cdocspt3.tex|&include file for part 3\\
% |cdocspt4.tex|&include file for part 4\\
% |cdocsdrf.tex|&forwarding file for main file in draft mode\\
% |cdocsfi1.tex|&forwarding file for final version of chapter 1\\
% |cdocsfi2.tex|&forwarding file for final version of chapter 2\\
% \end{tabular}
% \end{center}
% Each of the eight files can be compiled directly by the \LaTeX{} compiler.
%
% %%%%%%%%%%%%%%%%%%%%%%%%%%%%%%%%%%%%%%
% \paragraph{Main File.}
%
% The main file is called |cdocsamp.tex|.
%
% Load the \textsf{childdoc} definitions and
% declare the filename for the main document:
%    \begin{macrocode}
\input{childdoc.def}
\childdocmain{}
%    \end{macrocode}

% Optional override for |\version| flag:
%    \begin{macrocode}
%%\ifchilddoc\else\providecommand{\version}{draft}\fi
%    \end{macrocode}

% Define the default values for the |\version| flag
% (|final| for the main file and |draft| for childs):
%    \begin{macrocode}
\ifchilddoc
\providecommand{\version}{draft}
\else
\providecommand{\version}{final}
\fi
%    \end{macrocode}

% Load the standard document class:
%    \begin{macrocode}
\documentclass[12pt]{article}
%    \end{macrocode}

% Start the document body:
%    \begin{macrocode}
\begin{document}
%    \end{macrocode}

% Declare a title page.
% Print title, part of document being processed and version flag:
%    \begin{macrocode}
\addtocounter{page}{-1}
\begin{center}
{\LARGE\bfseries{}childdoc example\par}
\vspace{1cm}
\ifchilddoc
\ifchilddocmanual part\else chapter\fi:
`\childdocname' of `\childdocjob'\par
\else
main document: `\childdocjob'\par
\fi
version: \version\par
\end{center}
\newpage
%    \end{macrocode}

% Manually include selected file,
% otherwise process as usual:
%    \begin{macrocode}
\ifchilddocmanual
\section*{part `\childdocname'}
\input{\childdocname}
\else
%    \end{macrocode}

% Include the two chapters:
%    \begin{macrocode}
\include{cdocsch1}
\include{cdocsch2}
%    \end{macrocode}

% Include the two parts unless only chapters should be displayed:
%    \begin{macrocode}
\ifchilddoc\else
\section{part three}
\input{cdocspt3}
\section{part four}
\input{cdocspt4}
\fi
%    \end{macrocode}

% Process as usual until here:
%    \begin{macrocode}
\fi
%    \end{macrocode}

% End of document body:
%    \begin{macrocode}
\end{document}
%    \end{macrocode}
%\iffalse
%</samplemain>
%\fi
%
% %%%%%%%%%%%%%%%%%%%%%%%%%%%%%%%%%%%%%%
% \paragraph{Chapter Include Files.}
%
% The include files are called |cdocsch1.tex| and |cdocsch2.tex|.
%
%\iffalse
%<*samplechap1|samplechap2>
%\fi

% Optional override for |\version| flag:
%    \begin{macrocode}
%%\providecommand{\version}{final}
%    \end{macrocode}

% Include the main document:
%    \begin{macrocode}
\input{childdoc.def}
\childdocof{cdocsamp}
%    \end{macrocode}

%\iffalse
%</samplechap1|samplechap2>
%\fi
%
%\iffalse
%<*samplechap1>
%\fi
% Some text for chapter 1:
%    \begin{macrocode}
\section{one}
some text in chapter one
%    \end{macrocode}

%\iffalse
%</samplechap1>
%\fi
% Some text for chapter 2:
%\iffalse
%<*samplechap2>
%\fi
%    \begin{macrocode}
\section{two}
more text in chapter two
%    \end{macrocode}

%\iffalse
%</samplechap2>
%\fi
%
% %%%%%%%%%%%%%%%%%%%%%%%%%%%%%%%%%%%%%%
% \paragraph{Part Include Files.}
%
% The include files are called |cdocspt3.tex| and |cdocspt4.tex|.
%
%\iffalse
%<*samplepart3|samplepart4>
%\fi

% Optional override for |\version| flag:
%    \begin{macrocode}
%%\providecommand{\version}{final}
%    \end{macrocode}

% Include the main document:
%    \begin{macrocode}
\input{childdoc.def}
\childdocby{cdocsamp}
%    \end{macrocode}

%\iffalse
%</samplepart3|samplepart4>
%\fi
%
%\iffalse
%<*samplepart3>
%\fi
% Some text for part 3:
%    \begin{macrocode}
some text in part three
%    \end{macrocode}

%\iffalse
%</samplepart3>
%\fi
% Some text for part 4:
%\iffalse
%<*samplepart4>
%\fi
%    \begin{macrocode}
more text in part four
%    \end{macrocode}

%\iffalse
%</samplepart4>
%\fi
%
% %%%%%%%%%%%%%%%%%%%%%%%%%%%%%%%%%%%%%%
% \paragraph{Forwarding for a Complete Draft.}
%
% The following forwarding file |cdocsdrf.tex|
% compiles the main document in draft mode:
%\iffalse
%<*sampledraft>
%\fi
%    \begin{macrocode}
\def\version{draft}
\input{childdoc.def}
\childdocforward{cdocsamp}
%    \end{macrocode}

%\iffalse
%</sampledraft>
%\fi
%
% %%%%%%%%%%%%%%%%%%%%%%%%%%%%%%%%%%%%%%
% \paragraph{Forwarding for Final Version of the Chapters.}
%
% The following forwarding files |cdocsfn1.tex| and |cdocsfn2.tex|
% (with identical content)
% compile the final versions of the child documents
% |cdocsch1.tex| and |cdocsch2.tex|, respectively:
%\iffalse
%<*samplefinal>
%\fi
%    \begin{macrocode}
\def\version{final}
\input{childdoc.def}
\childdocforwardprefix[cdocsamp]{cdocsfn}{cdocsch}
%    \end{macrocode}

%\iffalse
%</samplefinal>
%\fi
%
% %%%%%%%%%%%%%%%%%%%%%%%%%%%%%%%%%%%%%%
% \paragraph{Command Line Processing.}
%
% The following three command lines generate the output files
% |cdocscld|, |cdocscl1| and |cdocscl2|
% which should be identical to
% |cdocsdrf|, |cdocsch1| and |cdocsfn2|, respectively:
% \begin{center}
% \begin{tabular}{l}
% |latex -jobname cdocscld \|\\
% |  "\def\version{draft}\input{childdoc.def}\childdocforward{cdocsamp}"|\\
% |latex -jobname cdocscl1 \|\\
% |  "\input{childdoc.def}\childdocforward[cdocsamp]{cdocsch1}"|\\
% |latex -jobname cdocscl2 \|\\
% |  "\def\version{final}\input{childdoc.def}\childdocforward{cdocsch2}"|
% \end{tabular}
% \end{center}
% Note that the trailing backslash on each first line
% merely continues the input to the second line
% (for convenient cut ant paste).
% Furthermore, the command |latex| can be replaced by any
% of its alternative versions such as |pdflatex|.
%
% %%%%%%%%%%%%%%%%%%%%%%%%%%%%%%%%%%%%%%%%%%%%%%%%%%%%%%%%%%%%%%%%%%%%%%%%%%%%%%
% %%%%%%%%%%%%%%%%%%%%%%%%%%%%%%%%%%%%%%%%%%%%%%%%%%%%%%%%%%%%%%%%%%%%%%%%%%%%%%
% \section{Implementation}
%\iffalse
%<*package>
%\fi
%
% This section describes the definitions file |childdoc.def|.

% The definitions cannot be loaded using |\usepackage| or |\RequirePackage|
% which has a mechanism to prevent loading a style file more than once.
% When loading the definitions by means of |\input|
% multiple instances have to be prevented manually:
%\iffalse
%This code needs to be before the `\ProvidesFile' directive
%which is defined at the beginning of this file.
%Therefore it is also placed there and commented out here.
%</package>
%<*discard>
%\fi
%    \begin{macrocode}
\ifdefined\childdocmain\endinput\fi
%    \end{macrocode}
%\iffalse
%</discard>
%<*package>
%\fi
%
% \macro{\ifchilddoc}
% \macro{\ifchilddocmanual}
% The conditional |\ifchilddoc| tells whether a
% child (true) or main (false) document is being compiled.
% The conditional |\ifchilddocmanual| tells whether
% the |\includeonly| mechanism is used (false) or
% the selection of child files must be performed manually (true).
% The definitions initialise to false:
%    \begin{macrocode}
\newif\ifchilddoc
\newif\ifchilddocmanual
%    \end{macrocode}

% \macro{\childdocname}
% \macro{\childdocjob}
% The macro |\childdocname| stores the name of the main document
% to be compiled. The macro |\childdocjob| stores the name of
% the document on which the \LaTeX{} compiler was originally invoked.
% The content of |\jobname| cannot be compared
% to filenames specified in the source due to different catcodes.
% The following code rescans |\jobname|, stores the result
% in |\childdocname| and saves a copy in |\childdocjob|:
%    \begin{macrocode}
\edef\childdocname{\scantokens\expandafter{\jobname\noexpand}}
\let\childdocjob\childdocname
%    \end{macrocode}

% \macro{\childdocdisable}
% The macro |\childdocdisable| prevents the main file
% from being processed more than once.
% At this stage, the main document command |\childdocmain|
% is assumed to be called once again where it should do nothing.
% Any subsequent call to it should prevent
% a secondary processing of the main document
% It overwrites the forwarding commands
% |\childdocof| and |\childdocforward|
% with empty macros to prevent further inclusions of the main document:
%    \begin{macrocode}
\newcommand{\childdocdisable}
{
  \renewcommand{\childdocmain}[1]{\renewcommand{\childdocmain}[1]{\endinput}}
  \renewcommand{\childdocof}[1]{}
  \renewcommand{\childdocby}[2][]{}
  \renewcommand{\childdocforward}[2][]{}
  \renewcommand{\childdocdisable}{}
}
%    \end{macrocode}

% \macro{\childdocmain}
% The macro |\childdocmain| is to be called at the top of the main file
% with nothing or the main filename (without extension) as argument.
% First, it breaks loops.
% If the argument is not empty and does not match |\childdocname|
% (which is set by the first inclusion of |childdoc.def|),
% |\ifchilddoc| is set to true, |\includeonly| is applied to the child file
% and |\jobname| is set to the main file
% (for proper handling of |.aux| files):
%    \begin{macrocode}
\newcommand{\childdocmain}[1]
{
  \childdocdisable\childdocmain{}
  \if?#1?\else
    \begingroup
      \def\childdoctmp{#1}
      \ifx\childdoctmp\childdocname
        \def\childdoctmp{}
      \else
        \def\childdoctmp
        {
          \childdoctrue
          \includeonly{\childdocname}
          \def\childdocjob{#1}
          \def\jobname{#1}
        }
      \fi
      \expandafter
    \endgroup
    \childdoctmp
  \fi
}
%    \end{macrocode}

% \macro{\childdocof}
% The command |\childdocof| redirects
% compilation to the main file |#1|.
%    \begin{macrocode}
\newcommand{\childdocof}[1]
{
  \childdocdisable
  \childdoctrue
  \includeonly{\childdocname}
  \def\jobname{#1}
  \def\childdocjob{#1}
  \input{#1}
}
%    \end{macrocode}

% \macro{\childdocby}
% The command |\childdocby| ....
%    \begin{macrocode}
\newcommand{\childdocby}[2][]
{
  \childdocdisable
  \childdoctrue
  \childdocmanualtrue
  \if?#1?\else
    \def\jobname{#2}
  \fi
  \def\childdocjob{#2}
  \input{#2}
  \endinput
}
%    \end{macrocode}

% \macro{\childdocforward}
% The command |\childdocforward| redirects
% compilation to the main file or
% (if the optional argument is given) a child file.
% Parameters are set as if the main file
% or a child file starting with |\childdocof| was compiled.
% Then compilation is handed over to the main file:
%    \begin{macrocode}
\newcommand{\childdocforward}[2][]
{
  \begingroup
    \if?#1?
      \def\childdoctmp
      {
        \def\childdocname{#2}
        \def\childdocjob{#2}
        \def\jobname{#2}
        \input{#2}
        \endinput
      }
    \else
      \def\childdoctmp
      {
        \childdocdisable
        \def\childdocname{#2}
        \childdoctrue
        \includeonly{#2}
        \def\childdocjob{#1}
        \def\jobname{#1}
        \input{#1}
        \endinput
      }
    \fi
    \expandafter
  \endgroup
  \childdoctmp
}
%    \end{macrocode}

% \macro{\childdocforwardprefix}
% The command |\childdocforwardprefix| redirects
% compilation to the main or a child file by means of a pattern.
% The prefix |#1| in the current filename is replaced by |#2|
% and the suffix of the current filename is kept
% (it is assumed that the filename does not contain the substring `|~~~|'
% which is used as a delimiter).
% Compilation is handed over to the new file by |\childdocforward|:
%    \begin{macrocode}
\newcommand{\childdocforwardprefix}[3][]
{
  \begingroup
    \def\childdocextract #2##1~~~{\def\childdoctmp{\childdocforward[#1]{#3##1}}}
    \expandafter\childdocextract\childdocname~~~
    \expandafter
  \endgroup
  \childdoctmp
}
%    \end{macrocode}

% \macro{\childdoc}
% The deprecated macro |\childdoc| is a legacy version of |\childdocmain|:
%    \begin{macrocode}
\newcommand{\childdoc}{\childdocmain}
%    \end{macrocode}

% \macro{\childdocredirect}
% The deprecated macro |\childdocredirect| is a legacy version
% of |\childdocforward| and |\childdocforwardprefix|:
%    \begin{macrocode}
\newcommand{\childdocredirect}[2][]
{
  \begingroup
    \if?#1?
      \def\childdoctmp{\childdocforward{#2}}
    \else
      \def\childdoctmp{\childdocforwardprefix{#1}{#2}}
    \fi
    \expandafter
  \endgroup
  \childdoctmp
}
%    \end{macrocode}

%\iffalse
%</package>
%\fi
%
\endinput
\childdocforward{cdocsamp}"|\\
% |latex -jobname cdocscl1 \|\\
% |  "% \iffalse
%
% childdoc.dtx Copyright (C) 2017-2018 Niklas Beisert
%
% This work may be distributed and/or modified under the
% conditions of the LaTeX Project Public License, either version 1.3
% of this license or (at your option) any later version.
% The latest version of this license is in
%   http://www.latex-project.org/lppl.txt
% and version 1.3 or later is part of all distributions of LaTeX
% version 2005/12/01 or later.
%
% This work has the LPPL maintenance status `maintained'.
%
% The Current Maintainer of this work is Niklas Beisert.
%
% This work consists of the files childdoc.dtx and childdoc.ins
% and the derived files childdoc.def and cdocsamp.tex with
% cdocsch1.tex, cdocsch2.tex, cdocsdrf.tex, cdocsfn1.tex, cdocsfn2.tex.
%
%<package>\ifdefined\childdocmain\endinput\fi
%<package>\ProvidesFile{childdoc.def}[2018/12/30 v2.0 child document driver]
%<samplemain>\ProvidesFile{cdocsamp.tex}[2018/12/30 v2.0 sample for childdoc]
%<*driver>
%\ProvidesFile{childdoc.drv}[2018/12/30 v2.0 childdoc reference manual file]
\PassOptionsToClass{10pt,a4paper}{article}
\documentclass{ltxdoc}

\usepackage[margin=35mm]{geometry}
\usepackage{hyperref}
\usepackage{hyperxmp}
\usepackage[usenames]{color}

\hypersetup{colorlinks=true}
\hypersetup{pdfstartview=FitH}
\hypersetup{pdfpagemode=UseNone}
\hypersetup{pdfsource={}}
\hypersetup{pdflang={en-UK}}
\hypersetup{pdfcopyright={Copyright 2017-2018 Niklas Beisert.
  This work may be distributed and/or modified under the
  conditions of the LaTeX Project Public License, either version 1.3
  of this license or (at your option) any later version.}}
\hypersetup{pdflicenseurl={http://www.latex-project.org/lppl.txt}}
\hypersetup{pdfcontactaddress={ETH Zurich, ITP, HIT K,
  Wolfgang-Pauli-Strasse 27}}
\hypersetup{pdfcontactpostcode={8093}}
\hypersetup{pdfcontactcity={Zurich}}
\hypersetup{pdfcontactcountry={Switzerland}}
\hypersetup{pdfcontactemail={nbeisert@itp.phys.ethz.ch}}
\hypersetup{pdfcontacturl={http://people.phys.ethz.ch/\xmptilde nbeisert/}}

\newcommand{\secref}[1]{\hyperref[#1]{section \ref*{#1}}}

\parskip1ex
\parindent0pt
\let\olditemize\itemize
\def\itemize{\olditemize\parskip0pt}

\begin{document}

\title{The \textsf{childdoc} Package}
\hypersetup{pdftitle={The childdoc Package}}
\author{Niklas Beisert\\[2ex]
  Institut f\"ur Theoretische Physik\\
  Eidgen\"ossische Technische Hochschule Z\"urich\\
  Wolfgang-Pauli-Strasse 27, 8093 Z\"urich, Switzerland\\[1ex]
  \href{mailto:nbeisert@itp.phys.ethz.ch}
  {\texttt{nbeisert@itp.phys.ethz.ch}}}
\hypersetup{pdfauthor={Niklas Beisert}}
\hypersetup{pdfsubject={Manual for the LaTeX2e Package childdoc}}
\date{30 December 2018, \textsf{v2.0}}
\maketitle

\begin{abstract}\noindent
\textsf{childdoc} is a \LaTeXe{} package
that enables the direct compilation
of document sections included by |\include|
to individual files.
\end{abstract}

\begingroup
\parskip0ex
\tableofcontents
\endgroup

%%%%%%%%%%%%%%%%%%%%%%%%%%%%%%%%%%%%%%%%%%%%%%%%%%%%%%%%%%%%%%%%%%%%%%%%%%%%%%%%
%%%%%%%%%%%%%%%%%%%%%%%%%%%%%%%%%%%%%%%%%%%%%%%%%%%%%%%%%%%%%%%%%%%%%%%%%%%%%%%%
\section{Introduction}

\LaTeX{} provides a mechanism to structure a large document (such as a book)
into a main file and several child files (containing the chapters)
using the |\include| command.
This mechanism is beneficial for documents
which span hundreds of pages in order to
make the source file(s) more manageable.
Moreover, compilation can be restricted to
selected child files by means of the |\includeonly| command.
The latter feature can be used to reduce the compilation time while editing
(this was significantly more useful in the earlier days of \LaTeX{})
or to generate a smaller document which is easier to navigate.
Another application of |\includeonly| is to generate
documents consisting of selected parts of the complete document.

However, there are a few drawbacks of the plain |\include| mechanism:
\begin{itemize}
\item
The child files cannot be compiled on their own,
they can only be compiled via the main file.
A naive editing environment
(such as a text editor with an option
to have the current file processed by \LaTeX)
may require one to switch to the main file before compiling;
attempting to compile the child file produces errors.
\item
The main file must be modified (each time)
to adjust the |\includeonly| command
to the present needs. This easily leaves the main file in a messy state.
\item
The generated document will always carry the filename
of the main document. This is inconvenient if
several child files are to be compiled and
to be kept for distribution.
\end{itemize}

The present package provides a simple interface
to make child files individually compilable by \LaTeX{}.
Compiling a child file then has the same effect as compiling
the main file with an |\includeonly| command
to select the appropriate child.
Moreover the generated document will carry the name of the child
rather than the main file.
This resolves all three above issues.

This feature is meant to make the editing of books,
thesis documents and lecture notes somewhat more convenient.
However, the package can also be used efficiently for
composing a series of documents (such as exercise sheets)
which are typically distributed individually.
It then assists the author in generating the individual documents
(potentially in different versions)
as well as a document containing the collected series.
Another application is in developing style files
or other kinds of included material
where compilation of the style file could redirect
to a sample or test file.

%%%%%%%%%%%%%%%%%%%%%%%%%%%%%%%%%%%%%%%%%%%%%%%%%%%%%%%%%%%%%%%%%%%%%%%%%%%%%%%%
%%%%%%%%%%%%%%%%%%%%%%%%%%%%%%%%%%%%%%%%%%%%%%%%%%%%%%%%%%%%%%%%%%%%%%%%%%%%%%%%
\section{Usage}

First of all, the package \textsf{childdoc} is \emph{not} a standard
\LaTeXe{} |.sty| style file! Therefore it needs to be invoked in
a non-standard way.

%%%%%%%%%%%%%%%%%%%%%%%%%%%%%%%%%%%%%%%%%%%%%%%%%%%%%%%%%%%%%%%%%%%%%%%%%%%%%%%%
\subsection{Included Files}
\label{sec:include}

%%%%%%%%%%%%%%%%%%%%%%%%%%%%%%%%%%%%%%%%
\DescribeMacro{\childdocmain}
To use the package, add the commands
\begin{center}
\begin{tabular}{l}
|\input{childdoc.def}|\\
|\childdocmain{}|\\
\end{tabular}
\end{center}
at the very top of the main \LaTeX{} file,
in particular \emph{before} the |\documentclass| statement!
The argument of |\childdocmain| should be left empty
(but it must be present).

%%%%%%%%%%%%%%%%%%%%%%%%%%%%%%%%%%%%%%%%
\DescribeMacro{\childdocof}
Furthermore, add the commands
\begin{center}
\begin{tabular}{l}
|\input{childdoc.def}|\\
|\childdocof{|\textit{main}|}|\\
\end{tabular}
\end{center}
at the top of every child file \textit{child}
which is included by |\include{|\textit{child}|}|
from within the main file
(or at least for those files to be compiled individually).
The argument \textit{main} must be the filename of the main file.

There are a couple of
considerations in setting up the main and child documents:

%%%%%%%%%%%%%%%%%%%%%%%%%%%%%%%%%%%%%%%%
\paragraph{Restrictions.}

Please note the following restrictions:
\begin{itemize}
\item
|\childdocmain| must be called with one argument \textit{main}
to ensure compatibility with earlier version of the package.
It must either be empty (|\childdocmain{}|)
or precisely match the filename of the main file in which it is specified.
See \secref{sec:detection} for further information.
\item
The filename \textit{main} must be specified without the |.tex| extension.
\item
The filename \textit{main} is case sensitive
(even in case-insensitive file systems)
due to internal string comparison.
\item
The argument \textit{main} should be fully expanded, it cannot be a macro.
\item
Subdirectories and special characters should be avoided in filenames.
\item
The command |\childdocmain{|\textit{main}|}| must be followed by a whitespace.
It should not be followed immediately by another command
or by a comment mark `|%|'.
This is because the \TeX{} parser reads the token immediately following
the argument of |\childdocmain| and puts it
at the beginning of every child section;
however, a white\-space is ignored.
\end{itemize}

%%%%%%%%%%%%%%%%%%%%%%%%%%%%%%%%%%%%%%%%
\paragraph{Content of Main File.}

It is advisable to place all content in the child files included by |\include|.
Any output contained in the main file will appear in all child documents
unless suppressed manually;
it cannot be suppressed automatically by the |\includeonly| directive
and thus should normally be avoided.
A method to include some content in the main file
by means of conditional processing is described in \secref{sec:conditional}.

%%%%%%%%%%%%%%%%%%%%%%%%%%%%%%%%%%%%%%%%
\paragraph{Page Numbering.}

When only a part of the document is compiled,
the appropriate numbering of pages
(as well as other status parameters)
is determined from the |.aux| files.
The latter contain information from previous passes.
However this information needs to propagate through
all intermediate child documents.
Therefore the page numbering in child documents may well
be inconsistent until the complete document is compiled at least once.

A useful (if unconventional) way to always ensure a consistent
page numbering is to restart the numbering in each child document
and denote the pages by `\textit{child}|.|\textit{page}'
where \textit{child} represents the chapter/section number of the child file.
This can be achieved by the command
|\numberwithin{page}{|\textit{child}|}|
of the \textsf{amsmath} package
where \textit{child} can be |chapter| or |section|
depending on the chosen structuring.
Alternatively, one can modify the macro |\thepage| appropriately
and reset the counter |page| at the start of each child file.

%%%%%%%%%%%%%%%%%%%%%%%%%%%%%%%%%%%%%%%%%%%%%%%%%%%%%%%%%%%%%%%%%%%%%%%%%%%%%%%%
\subsection{Conditional Processing}
\label{sec:conditional}

The package provides a mechanism to compile different versions
of a document. To customise the versions further some conditional processing
can come in handy to distinguish which version is being compiled.
The package provides two macros to describe the compilation context:

%%%%%%%%%%%%%%%%%%%%%%%%%%%%%%%%%%%%%%%%
\DescribeMacro{\ifchilddoc}
The conditional |\ifchilddoc| distinguishes between the compilation of
child documents and the main document:
%
\begin{center}
|\ifchilddoc |\textit{child-code}| |[|\||else |\textit{main-code}]| \||fi|
\end{center}

%%%%%%%%%%%%%%%%%%%%%%%%%%%%%%%%%%%%%%%%
\DescribeMacro{\childdocname}
\DescribeMacro{\childdocjob}
The macro |\childdocname| contains the filename (without extension)
of the main or child file being processed.
Note that |\childdocjob| will always contain the name of the main file.

%%%%%%%%%%%%%%%%%%%%%%%%%%%%%%%%%%%%%%%%
\paragraph{Title Page.}

Conditional processing can be used to include a title or banner page
in the main document when proper precautions are taken.
Importantly, the code in the main file should ensure that the page counter
(as well as other status parameters which are stored in the |.aux| files)
takes the same value after the conditional processing.
Otherwise the page numbers may take divergent values
depending on which part is compiled.

For example, a title page could be declared by:
%
\begin{center}
\begin{tabular}{l}
|\ifchilddoc\||else|\\
|\addtocounter{page}{-1}|\\
\textit{code for title page}\\
|\newpage|\\
|\||fi|
\end{tabular}
\end{center}
%
A banner page for the child documents can be generated by:
%
\begin{center}
\begin{tabular}{l}
|\ifchilddoc|\\
|\addtocounter{page}{-1}|\\
\textit{code for banner page}\\
|\newpage|\\
|\||fi|
\end{tabular}
\end{center}
%
Here one could write a message such as:
\begin{center}
|This is the part \childdocname{} of \childdocjob{}.|
\end{center}

%%%%%%%%%%%%%%%%%%%%%%%%%%%%%%%%%%%%%%%%%%%%%%%%%%%%%%%%%%%%%%%%%%%%%%%%%%%%%%%%
\subsection{Flags}
\label{sec:flags}

The package makes it easy to generate different versions
of the main or child documents.
To this end compilation flags can be defined
and assigned different default values.
They will be particularly useful in conjunction
with the forwarding mechanism described in \secref{sec:forward}.

For example, it may be useful to have a flag |\version|
which can be set to |draft| or |final|.
The document source will contain some conditional code
depending on the value of |\version|.
Suppose further, the flag should default to |final| for the main file
and to |draft| for child files
which is a natural assignment for editing the document.
This is achieved by placing the following code
in the preamble of the main document
(below the |\childdocmain| directive):
%
\begin{center}
\begin{tabular}{l}
|\ifchilddoc|\\
|\providecommand{\version}{draft}|\\
|\||else|\\
|\providecommand{\version}{final}|\\
|\||fi|
\end{tabular}
\end{center}
%
The definition by |\providecommand| makes sure
that previous definitions are not overwritten.
Further statements |\providecommand{\version}{...}|
can thus be added before the above code to override it.

For the main file, one might add a line
(between |\childdocmain| and the above block)
%
\begin{center}
|%\ifchilddoc\||else\providecommand{\version}{draft}\||fi|
\end{center}
%
which can be uncommented to produce a draft version.
Likewise one can add a line to the very top of a child file
(above the |\childdocof{|\textit{main}|}| directive)
%
\begin{center}
|%\providecommand{\version}{final}|
\end{center}
%
which can be uncommented to produce the final version of this child document.

%%%%%%%%%%%%%%%%%%%%%%%%%%%%%%%%%%%%%%%%%%%%%%%%%%%%%%%%%%%%%%%%%%%%%%%%%%%%%%%%
\subsection{Forwarding}
\label{sec:forward}

Different versions of the main or child documents
using compilation flags as described in \secref{sec:flags}
can be (permanently) stored in different files
for convenient compilation, viewing and distribution.
To this end, the package defines a command
to pass on compilation to a different file:

%%%%%%%%%%%%%%%%%%%%%%%%%%%%%%%%%%%%%%%%
\DescribeMacro{\childdocforward}
The command |\childdocforward| redirects processing to
another source file:
%
\begin{center}
\begin{tabular}{l}
|\input{childdoc.def}|\\
|\childdocforward[|\textit{main}|]{|\textit{dest}|}|\\
\end{tabular}
\end{center}
%
The argument \textit{dest} is the destination file
(without extension).
It should be the main file or one of the child files.
Note that further \textsf{childdoc} directives
such as |\childdocof| and |\childdocforward|
in the indicated file will be processed in this form.
The optional argument \textit{main}
passes on directly to the main file \textit{main}
while pretending to compile the child \textit{dest}.
This form behaves as if \textit{dest}
issues |\childdocof{|\textit{main}|}| right away,
and no further \textsf{childdoc} directives will be processed.

%%%%%%%%%%%%%%%%%%%%%%%%%%%%%%%%%%%%%%%%
\DescribeMacro{\...prefix}
In the alternative form |\childdocforwardprefix|,
%
\begin{center}
\begin{tabular}{l}
|\input{childdoc.def}|\\
|\childdocforwardprefix[|\textit{main}|]{|\textit{prefix}|}{|\textit{dest}|}|
\end{tabular}
\end{center}
%
the destination file is determined by a pattern
depending on the current file:
To make this work, the current file must be called
`{\textit{prefix}\hspace{0.2em}\textit{suffix}}'
with \textit{prefix} matching precisely the argument.
Processing is then passed on to the file
`{\textit{dest}\hspace{0.2em}\textit{suffix}}'.
Surely, the same effect is achieved by
directly specifying the
argument `{\textit{dest}\hspace{0.2em}\textit{suffix}}'
in the first form.
However, that requires to set up a different file
for each child. With the alternative form of the command
all these files can have exactly the same content
which simplifies setting them up and maintaining them.

For example, the following file |draft.tex|
with a compilation flag |\version| as described in \secref{sec:flags}
compiles the main document as a draft:
%
\begin{center}
\begin{tabular}{l}
|\def\version{draft}|\\
|\input{childdoc.def}|\\
|\childdocforward{|\textit{main}|}|
\end{tabular}
\end{center}
%
Likewise, the following files |final|\textit{nn}|.tex|
compile the final version of the child document
|child|\textit{nn}|.tex|:
%
\begin{center}
\begin{tabular}{l}
|\def\version{final}|\\
|\input{childdoc.def}|\\
|\childdocforwardprefix{final}{child}|
\end{tabular}
\end{center}
%

Note that when several versions of a main file and/or of each child file
are to be generated, it may be convenient to set up a |Makefile| or
shell script to automatise the process.

%%%%%%%%%%%%%%%%%%%%%%%%%%%%%%%%%%%%%%%%%%%%%%%%%%%%%%%%%%%%%%%%%%%%%%%%%%%%%%%%
\subsection{Command Line Processing}
\label{sec:commandline}

The effect of redirection files can also be achieved by invoking
the \LaTeX{} compiler with a more elaborate command line.
Most conveniently this should be done as part
of a shell script or a |Makefile|.

When using \textsf{childdoc} in the main file, the following
command lines effectively perform a redirection
(note that depending on the shell being used,
backslashes may have to be doubled: `|\|' $\to$ `|\\|'):
%
\begin{center}
|... -jobname "|\textit{target}|" |\\|"|[\textit{flags}]%
|\input{childdoc.def}\childdocforward[|\textit{main}|]{|\textit{dest}|}"|
\end{center}
%
Here \textit{target} is the name of the output file,
\textit{main} is the name of the main file
and \textit{dest} is the name of the main or child file to be processed
(all filenames without extensions).
The optional argument \textit{main} can be omitted
if \textit{main} matches \textit{dest}.
Optionally, compilation \textit{flags} can be defined via |\def| commands.
This command line makes the \TeX{} engine believe
it is compiling the file \textit{target}
whose content is specified as the latter parameter.
The provided code then forwards the processing to
\textit{main} or \textit{dest} as described in \secref{sec:forward}.

%%%%%%%%%%%%%%%%%%%%%%%%%%%%%%%%%%%%%%%%%%%%%%%%%%%%%%%%%%%%%%%%%%%%%%%%%%%%%%%%
\subsection{Include by Input}
\label{sec:input}

Including child documents by |\include| has some restrictions by design.
Most notably, the content of a child document always occupies
its own set of pages; pages cannot be shared between child documents.
Usually, this behaviour makes perfect sense
because each child document contain an essential part of the document.
However, in some situations it may be desirable to compose
a document from a collection of parts
without having mandatory page breaks between then.
For this case, the package
provides a mechanism to include parts
by |\input| which can also be processed individually.
However, by construction this mechanism
requires manual handling of the content to be output.

%%%%%%%%%%%%%%%%%%%%%%%%%%%%%%%%%%%%%%%%
\DescribeMacro{\ifchilddocmanual}
The main file should be prepared as usual, see \secref{sec:include}.
However, the document body must make a distinction
between processing of an individual part and of the main document, e.g.:
%
\begin{center}
\begin{tabular}{l}
|\ifchilddocmanual|\\
|\input{\childdocname}|\\
|\||else|\\
\textit{document body with }|\input{|\textit{part}|}|\\
|\||fi|
\end{tabular}
\end{center}
%
The conditional |\ifchilddocmanual| is true whenever
a part to be included by |\input| is being compiled,
and the name of the part is stored in |\childdocname|.

%%%%%%%%%%%%%%%%%%%%%%%%%%%%%%%%%%%%%%%%
\DescribeMacro{\childdocby}
Each part to be included by |\input| should start with:
%
\begin{center}
\begin{tabular}{l}
|\input{childdoc.def}|\\
|\childdocby{|\textit{main}|}|\\
\end{tabular}
\end{center}
%
The directive |\childdocby| is similar to |\childdocof|
described in \secref{sec:include},
but the subsequent selection of content must be done manually.
To that end, both |\ifchilddoc| and |\ifchilddocmanual|
will be true upon processing of a part,
and the name of the part is stored in |\childdocname|.
Note that |\jobname| will be set to the filename of the current part
so that each part receives an individual |.aux| file
that does not interfere with the |.aux| file(s) of the main document.
This behaviour can be altered by the alternative form
|\childdocby[*]{|\textit{main}|}| (with a non-empty optional argument)
which uses the |.aux| file of the main document
by setting |\jobname| to \textit{main}.

%%%%%%%%%%%%%%%%%%%%%%%%%%%%%%%%%%%%%%%%%%%%%%%%%%%%%%%%%%%%%%%%%%%%%%%%%%%%%%%%
\subsection{Driver Development}
\label{sec:driver}

The \textsf{childdoc} mechanism can also be use for the development
of definition files such as \LaTeX{} styles or classes.
This case differs from the above setup with multiple parts
included by |\include| in that no |\includeonly| should be invoked.
This can be achieved by starting the include file
(before |\ProvidesPackage|) with:
%
\begin{center}
\begin{tabular}{l}
|\input{childdoc.def}|\\
|\childdocforward{|\textit{main}|}|\\
\end{tabular}
\end{center}
%
or alternatively with:
%
\begin{center}
\begin{tabular}{l}
|\input{childdoc.def}|\\
|\childdocby{|\textit{main}|}|\\
\end{tabular}
\end{center}
%
Both forms have slightly different effects as described above.
The main file is prepared as usual, see \secref{sec:include}.

%%%%%%%%%%%%%%%%%%%%%%%%%%%%%%%%%%%%%%%%%%%%%%%%%%%%%%%%%%%%%%%%%%%%%%%%%%%%%%%%
\subsection{Legacy Detection}
\label{sec:detection}

The directive |\childdocmain| in the main file can detect
whether the complete document or merely a child is to be compiled
even without using the directive |\childdocof|.
This method is deprecated because it is less robust
and there is no compelling reason to use it;
it is merely provided for backward compatibility
and it may be removed in future versions.

If the detection mechanism is to be used,
it is mandatory to correctly specify
the filename of the main file as the argument of |\childdocmain|:
%
\begin{center}
\begin{tabular}{l}
|\input{childdoc.def}|\\
|\childdocmain{|\textit{main}|}|\\
\end{tabular}
\end{center}
%
If |\jobname| does not match the argument \textit{main} of |\childdocmain|,
it is assumed that |\jobname| points to the child file to be compiled.
When using |\childdocmain| with the main file specified as argument,
it suffices to start a child file
with just |\input{|\textit{main}|}|
without loading of the package and using |\childdocof|.
If instead all processing is done
with the appropriate \textsf{childdoc} directives,
the argument of \textit{main} of |\childdocmain| can be empty.

An alternative version of the command line processing described
in \secref{sec:commandline} using the detection mechanism reads:
%
\begin{center}
|... -jobname "|\textit{target}|" "|[\textit{flags}]%
[|\def\jobname{|\textit{dest}|}|]|\input{|\textit{main}|}"|
\end{center}

%%%%%%%%%%%%%%%%%%%%%%%%%%%%%%%%%%%%%%%%%%%%%%%%%%%%%%%%%%%%%%%%%%%%%%%%%%%%%%%%
\subsection{Manual Code}
\label{sec:manual}

In case one cannot be certain whether the definitions file |childdoc.def|
is installed on the target \TeX{} distribution
and one prefers not to ship it,
it is conceivable to paste a few relevant commands into the sources.

To that end, drop all statements |\input{childdoc.def}|
and perform the replacements as outlined below.
Instead of |\childdocmain{|\textit{main}|}| add the following code
to the top of the main file:
%
\begin{center}
\begin{tabular}{l}
|\||ifdefined\childdocname\endinput\||fi\newif\ifchilddoc|\\
|\edef\childdocname{\scantokens\expandafter{\jobname\noexpand}}|\\
|\def\childdocmain{|\textit{main}|}\||ifx\childdocmain\childdocname\||else|\\
|\childdoctrue\includeonly{\childdocname}\let\jobname\childdocmain\||fi|\\
\end{tabular}
\end{center}
%
Instead of |\childdocof{|\textit{main}|}| just include the main file
at the top of each child file:
%
\begin{center}
|\input{|\textit{main}|}|
\end{center}
%
A simple redirection |\childdocforward{|\textit{dest}|}| is achieved by:
%
\begin{center}
|\def\jobname{|\textit{dest}|}\input{\jobname}|
\end{center}
%
The redirection with prefix
|\childdocforwardprefix[|\textit{prefix}|]{|\textit{dest}|}|
is accomplished by:
%
\begin{center}
\begin{tabular}{l}
|{\edef\jobname{\scantokens\expandafter{\jobname\noexpand}}|\\
|\def\redirectjob |\textit{prefix}|#1~~~{\gdef\jobname{|\textit{dest}|#1}}|\\
|\expandafter\redirectjob\jobname~~~}\input{\jobname}|
\end{tabular}
\end{center}

In an alternative approach,
child documents can be compiled by a specific command line
without additional code or specific definitions:
%
\begin{center}
|... -jobname "|\textit{target}|" "|[\textit{flags}]%
|\includeonly{|\textit{dest}|}\input{|\textit{main}|}"|
\end{center}
%

%%%%%%%%%%%%%%%%%%%%%%%%%%%%%%%%%%%%%%%%%%%%%%%%%%%%%%%%%%%%%%%%%%%%%%%%%%%%%%%%
%%%%%%%%%%%%%%%%%%%%%%%%%%%%%%%%%%%%%%%%%%%%%%%%%%%%%%%%%%%%%%%%%%%%%%%%%%%%%%%%
\section{Information}

%%%%%%%%%%%%%%%%%%%%%%%%%%%%%%%%%%%%%%%%%%%%%%%%%%%%%%%%%%%%%%%%%%%%%%%%%%%%%%%%
\subsection{Copyright}

Copyright \copyright{} 2017--2018 Niklas Beisert

This work may be distributed and/or modified under the
conditions of the \LaTeX{} Project Public License, either version 1.3
of this license or (at your option) any later version.
The latest version of this license is in
  \url{http://www.latex-project.org/lppl.txt}
and version 1.3 or later is part of all distributions of \LaTeX{}
version 2005/12/01 or later.

This work has the LPPL maintenance status `maintained'.

The Current Maintainer of this work is Niklas Beisert.

This work consists of the files |README.txt|, |childdoc.ins| and |childdoc.dtx|
as well as the derived files |childdoc.def|, |cdocsamp.tex|
with |cdocsch1.tex|, |cdocsch2.tex|, |cdocspt3.tex|, |cdocspt4.tex|,
|cdocsdrf.tex|, |cdocsfn1.tex|, |cdocsfn2.tex|
as well as |childdoc.pdf|.

%%%%%%%%%%%%%%%%%%%%%%%%%%%%%%%%%%%%%%%%%%%%%%%%%%%%%%%%%%%%%%%%%%%%%%%%%%%%%%%%
\subsection{Files and Installation}

The package consists of the files:
%
\begin{center}
\begin{tabular}{ll}
    |README.txt|   & readme file \\
    |childdoc.ins| & installation file \\
    |childdoc.dtx| & source file \\
    |childdoc.def| & definition file \\
    |cdocsamp.tex| & sample main file \\
    |cdocsch1.tex| & sample include file \\
    |cdocsch2.tex| & sample include file \\
    |cdocspt3.tex| & sample part file \\
    |cdocspt4.tex| & sample part file \\
    |cdocsdrf.tex| & sample redirection file \\
    |cdocsfn1.tex| & sample redirection file \\
    |cdocsfn2.tex| & sample redirection file \\
    |childdoc.pdf| & manual
\end{tabular}
\end{center}
%
The distribution consists of the files
|README.txt|, |childdoc.ins| and |childdoc.dtx|.
%
\begin{itemize}
\item
Run (pdf)\LaTeX{} on |childdoc.dtx|
to compile the manual |childdoc.pdf| (this file).
\item
Run \LaTeX{} on |childdoc.ins| to create the definitions file |childdoc.def|
and the sample |cdocsamp.tex| with include files
|cdocsch1.tex|, |cdocsch2.tex|, |cdocspt3.tex|, |cdocspt4.tex|,
|cdocsdrf.tex|, |cdocsfn1.tex|, |cdocsfn2.tex|.
Then copy the file |childdoc.def| to an appropriate directory of your \LaTeX{}
distribution, e.g.\ \textit{texmf-root}|/tex/latex/childdoc|.
\end{itemize}

%%%%%%%%%%%%%%%%%%%%%%%%%%%%%%%%%%%%%%%%%%%%%%%%%%%%%%%%%%%%%%%%%%%%%%%%%%%%%%%%
\subsection{Related CTAN Packages}

There are several other packages which offer a similar functionality:
%
\begin{itemize}
\item
The packages
\href{http://ctan.org/pkg/docmute}{\textsf{docmute}},
\href{http://ctan.org/pkg/includex}{\textsf{includex}} and
\href{http://ctan.org/pkg/standalone}{\textsf{standalone}}
provide commands to include only the document body of
a child file thus allowing both files to be compiled individually.
\item
The packages \href{http://ctan.org/pkg/subdocs}{\textsf{subdocs}}
and \href{http://ctan.org/pkg/subfiles}{\textsf{subfiles}}
provide structures in which the main and child documents can be
encapsulated and allowing them to be compiled individually.
The inclusion mechanism is different from the conventional |\include|.
\item
The package \href{http://ctan.org/pkg/combine}{\textsf{combine}}
is an elaborate solution to combine several documents into one.
\end{itemize}
%
See also the CTAN topic \href{http://ctan.org/topic/subdocs}{\textsf{subdocs}}
for further related packages.
The present package differs from the above solutions in that
a document structure constructed with the conventional |\include| mechanism
just needs two extra commands at the top of every file
such that all constituent files can be compiled individually.

%%%%%%%%%%%%%%%%%%%%%%%%%%%%%%%%%%%%%%%%%%%%%%%%%%%%%%%%%%%%%%%%%%%%%%%%%%%%%%%%
%\subsection{Feature Suggestions}
%
%The following is a list of features which may be useful for future
%versions of this package:
%%
%\begin{itemize}
%\item
%\ldots
%\end{itemize}

%%%%%%%%%%%%%%%%%%%%%%%%%%%%%%%%%%%%%%%%%%%%%%%%%%%%%%%%%%%%%%%%%%%%%%%%%%%%%%%%
\subsection{Revision History}

%%%%%%%%%%%%%%%%%%%%%%%%%%%%%%%%%%%%%%%%
\paragraph{v2.0:} 2018/12/30

\begin{itemize}
\item
immediate forward processing
\item
added |\childdocby| mechanism
\item
manual restructured
\end{itemize}

%%%%%%%%%%%%%%%%%%%%%%%%%%%%%%%%%%%%%%%%
\paragraph{v1.6:} 2018/01/17

\begin{itemize}
\item
application for development of include files
\item
corrections to manual
\end{itemize}

%%%%%%%%%%%%%%%%%%%%%%%%%%%%%%%%%%%%%%%%
\paragraph{v1.5:} 2017/05/21

\begin{itemize}
\item
more complete structuring introduced
\item
|\childdocof| introduced
\item
|\childdoc| renamed to |\childdocmain|
\item
|\childredirect| renamed to |\childdocforward| and |\childdocforwardprefix|
and functionality expanded
\end{itemize}

%%%%%%%%%%%%%%%%%%%%%%%%%%%%%%%%%%%%%%%%
\paragraph{v1.0:} 2017/04/27

\begin{itemize}
\item
manual and install package
\item
first version published on CTAN
\end{itemize}

%%%%%%%%%%%%%%%%%%%%%%%%%%%%%%%%%%%%%%%%
\paragraph{v0.6:} 2017/04/26

\begin{itemize}
\item
redirection mechanism added
\end{itemize}

%%%%%%%%%%%%%%%%%%%%%%%%%%%%%%%%%%%%%%%%
\paragraph{v0.5:} 2017/04/26

\begin{itemize}
\item
functionality in definition file
\end{itemize}


%%%%%%%%%%%%%%%%%%%%%%%%%%%%%%%%%%%%%%%%%%%%%%%%%%%%%%%%%%%%%%%%%%%%%%%%%%%%%%%%
%%%%%%%%%%%%%%%%%%%%%%%%%%%%%%%%%%%%%%%%%%%%%%%%%%%%%%%%%%%%%%%%%%%%%%%%%%%%%%%%
%%%%%%%%%%%%%%%%%%%%%%%%%%%%%%%%%%%%%%%%%%%%%%%%%%%%%%%%%%%%%%%%%%%%%%%%%%%%%%%%
\appendix

\settowidth\MacroIndent{\rmfamily\scriptsize 000\ }

 \DocInput{childdoc.dtx}

\end{document}
%</driver>
% \fi
%
% %%%%%%%%%%%%%%%%%%%%%%%%%%%%%%%%%%%%%%%%%%%%%%%%%%%%%%%%%%%%%%%%%%%%%%%%%%%%%%
% %%%%%%%%%%%%%%%%%%%%%%%%%%%%%%%%%%%%%%%%%%%%%%%%%%%%%%%%%%%%%%%%%%%%%%%%%%%%%%
% \section{Sample}
%\iffalse
%<*samplemain>
%\fi
%
% The following presents a sample document
% with two chapters, two parts, a title page,
% a compile flag as well as three forwarding files to set the flag.
% It consists of eight |.tex| files:
% \begin{center}
% \begin{tabular}{ll}
% |cdocsamp.tex|&main file\\
% |cdocsch1.tex|&include file for chapter 1\\
% |cdocsch2.tex|&include file for chapter 2\\
% |cdocspt3.tex|&include file for part 3\\
% |cdocspt4.tex|&include file for part 4\\
% |cdocsdrf.tex|&forwarding file for main file in draft mode\\
% |cdocsfi1.tex|&forwarding file for final version of chapter 1\\
% |cdocsfi2.tex|&forwarding file for final version of chapter 2\\
% \end{tabular}
% \end{center}
% Each of the eight files can be compiled directly by the \LaTeX{} compiler.
%
% %%%%%%%%%%%%%%%%%%%%%%%%%%%%%%%%%%%%%%
% \paragraph{Main File.}
%
% The main file is called |cdocsamp.tex|.
%
% Load the \textsf{childdoc} definitions and
% declare the filename for the main document:
%    \begin{macrocode}
\input{childdoc.def}
\childdocmain{}
%    \end{macrocode}

% Optional override for |\version| flag:
%    \begin{macrocode}
%%\ifchilddoc\else\providecommand{\version}{draft}\fi
%    \end{macrocode}

% Define the default values for the |\version| flag
% (|final| for the main file and |draft| for childs):
%    \begin{macrocode}
\ifchilddoc
\providecommand{\version}{draft}
\else
\providecommand{\version}{final}
\fi
%    \end{macrocode}

% Load the standard document class:
%    \begin{macrocode}
\documentclass[12pt]{article}
%    \end{macrocode}

% Start the document body:
%    \begin{macrocode}
\begin{document}
%    \end{macrocode}

% Declare a title page.
% Print title, part of document being processed and version flag:
%    \begin{macrocode}
\addtocounter{page}{-1}
\begin{center}
{\LARGE\bfseries{}childdoc example\par}
\vspace{1cm}
\ifchilddoc
\ifchilddocmanual part\else chapter\fi:
`\childdocname' of `\childdocjob'\par
\else
main document: `\childdocjob'\par
\fi
version: \version\par
\end{center}
\newpage
%    \end{macrocode}

% Manually include selected file,
% otherwise process as usual:
%    \begin{macrocode}
\ifchilddocmanual
\section*{part `\childdocname'}
\input{\childdocname}
\else
%    \end{macrocode}

% Include the two chapters:
%    \begin{macrocode}
\include{cdocsch1}
\include{cdocsch2}
%    \end{macrocode}

% Include the two parts unless only chapters should be displayed:
%    \begin{macrocode}
\ifchilddoc\else
\section{part three}
\input{cdocspt3}
\section{part four}
\input{cdocspt4}
\fi
%    \end{macrocode}

% Process as usual until here:
%    \begin{macrocode}
\fi
%    \end{macrocode}

% End of document body:
%    \begin{macrocode}
\end{document}
%    \end{macrocode}
%\iffalse
%</samplemain>
%\fi
%
% %%%%%%%%%%%%%%%%%%%%%%%%%%%%%%%%%%%%%%
% \paragraph{Chapter Include Files.}
%
% The include files are called |cdocsch1.tex| and |cdocsch2.tex|.
%
%\iffalse
%<*samplechap1|samplechap2>
%\fi

% Optional override for |\version| flag:
%    \begin{macrocode}
%%\providecommand{\version}{final}
%    \end{macrocode}

% Include the main document:
%    \begin{macrocode}
\input{childdoc.def}
\childdocof{cdocsamp}
%    \end{macrocode}

%\iffalse
%</samplechap1|samplechap2>
%\fi
%
%\iffalse
%<*samplechap1>
%\fi
% Some text for chapter 1:
%    \begin{macrocode}
\section{one}
some text in chapter one
%    \end{macrocode}

%\iffalse
%</samplechap1>
%\fi
% Some text for chapter 2:
%\iffalse
%<*samplechap2>
%\fi
%    \begin{macrocode}
\section{two}
more text in chapter two
%    \end{macrocode}

%\iffalse
%</samplechap2>
%\fi
%
% %%%%%%%%%%%%%%%%%%%%%%%%%%%%%%%%%%%%%%
% \paragraph{Part Include Files.}
%
% The include files are called |cdocspt3.tex| and |cdocspt4.tex|.
%
%\iffalse
%<*samplepart3|samplepart4>
%\fi

% Optional override for |\version| flag:
%    \begin{macrocode}
%%\providecommand{\version}{final}
%    \end{macrocode}

% Include the main document:
%    \begin{macrocode}
\input{childdoc.def}
\childdocby{cdocsamp}
%    \end{macrocode}

%\iffalse
%</samplepart3|samplepart4>
%\fi
%
%\iffalse
%<*samplepart3>
%\fi
% Some text for part 3:
%    \begin{macrocode}
some text in part three
%    \end{macrocode}

%\iffalse
%</samplepart3>
%\fi
% Some text for part 4:
%\iffalse
%<*samplepart4>
%\fi
%    \begin{macrocode}
more text in part four
%    \end{macrocode}

%\iffalse
%</samplepart4>
%\fi
%
% %%%%%%%%%%%%%%%%%%%%%%%%%%%%%%%%%%%%%%
% \paragraph{Forwarding for a Complete Draft.}
%
% The following forwarding file |cdocsdrf.tex|
% compiles the main document in draft mode:
%\iffalse
%<*sampledraft>
%\fi
%    \begin{macrocode}
\def\version{draft}
\input{childdoc.def}
\childdocforward{cdocsamp}
%    \end{macrocode}

%\iffalse
%</sampledraft>
%\fi
%
% %%%%%%%%%%%%%%%%%%%%%%%%%%%%%%%%%%%%%%
% \paragraph{Forwarding for Final Version of the Chapters.}
%
% The following forwarding files |cdocsfn1.tex| and |cdocsfn2.tex|
% (with identical content)
% compile the final versions of the child documents
% |cdocsch1.tex| and |cdocsch2.tex|, respectively:
%\iffalse
%<*samplefinal>
%\fi
%    \begin{macrocode}
\def\version{final}
\input{childdoc.def}
\childdocforwardprefix[cdocsamp]{cdocsfn}{cdocsch}
%    \end{macrocode}

%\iffalse
%</samplefinal>
%\fi
%
% %%%%%%%%%%%%%%%%%%%%%%%%%%%%%%%%%%%%%%
% \paragraph{Command Line Processing.}
%
% The following three command lines generate the output files
% |cdocscld|, |cdocscl1| and |cdocscl2|
% which should be identical to
% |cdocsdrf|, |cdocsch1| and |cdocsfn2|, respectively:
% \begin{center}
% \begin{tabular}{l}
% |latex -jobname cdocscld \|\\
% |  "\def\version{draft}\input{childdoc.def}\childdocforward{cdocsamp}"|\\
% |latex -jobname cdocscl1 \|\\
% |  "\input{childdoc.def}\childdocforward[cdocsamp]{cdocsch1}"|\\
% |latex -jobname cdocscl2 \|\\
% |  "\def\version{final}\input{childdoc.def}\childdocforward{cdocsch2}"|
% \end{tabular}
% \end{center}
% Note that the trailing backslash on each first line
% merely continues the input to the second line
% (for convenient cut ant paste).
% Furthermore, the command |latex| can be replaced by any
% of its alternative versions such as |pdflatex|.
%
% %%%%%%%%%%%%%%%%%%%%%%%%%%%%%%%%%%%%%%%%%%%%%%%%%%%%%%%%%%%%%%%%%%%%%%%%%%%%%%
% %%%%%%%%%%%%%%%%%%%%%%%%%%%%%%%%%%%%%%%%%%%%%%%%%%%%%%%%%%%%%%%%%%%%%%%%%%%%%%
% \section{Implementation}
%\iffalse
%<*package>
%\fi
%
% This section describes the definitions file |childdoc.def|.

% The definitions cannot be loaded using |\usepackage| or |\RequirePackage|
% which has a mechanism to prevent loading a style file more than once.
% When loading the definitions by means of |\input|
% multiple instances have to be prevented manually:
%\iffalse
%This code needs to be before the `\ProvidesFile' directive
%which is defined at the beginning of this file.
%Therefore it is also placed there and commented out here.
%</package>
%<*discard>
%\fi
%    \begin{macrocode}
\ifdefined\childdocmain\endinput\fi
%    \end{macrocode}
%\iffalse
%</discard>
%<*package>
%\fi
%
% \macro{\ifchilddoc}
% \macro{\ifchilddocmanual}
% The conditional |\ifchilddoc| tells whether a
% child (true) or main (false) document is being compiled.
% The conditional |\ifchilddocmanual| tells whether
% the |\includeonly| mechanism is used (false) or
% the selection of child files must be performed manually (true).
% The definitions initialise to false:
%    \begin{macrocode}
\newif\ifchilddoc
\newif\ifchilddocmanual
%    \end{macrocode}

% \macro{\childdocname}
% \macro{\childdocjob}
% The macro |\childdocname| stores the name of the main document
% to be compiled. The macro |\childdocjob| stores the name of
% the document on which the \LaTeX{} compiler was originally invoked.
% The content of |\jobname| cannot be compared
% to filenames specified in the source due to different catcodes.
% The following code rescans |\jobname|, stores the result
% in |\childdocname| and saves a copy in |\childdocjob|:
%    \begin{macrocode}
\edef\childdocname{\scantokens\expandafter{\jobname\noexpand}}
\let\childdocjob\childdocname
%    \end{macrocode}

% \macro{\childdocdisable}
% The macro |\childdocdisable| prevents the main file
% from being processed more than once.
% At this stage, the main document command |\childdocmain|
% is assumed to be called once again where it should do nothing.
% Any subsequent call to it should prevent
% a secondary processing of the main document
% It overwrites the forwarding commands
% |\childdocof| and |\childdocforward|
% with empty macros to prevent further inclusions of the main document:
%    \begin{macrocode}
\newcommand{\childdocdisable}
{
  \renewcommand{\childdocmain}[1]{\renewcommand{\childdocmain}[1]{\endinput}}
  \renewcommand{\childdocof}[1]{}
  \renewcommand{\childdocby}[2][]{}
  \renewcommand{\childdocforward}[2][]{}
  \renewcommand{\childdocdisable}{}
}
%    \end{macrocode}

% \macro{\childdocmain}
% The macro |\childdocmain| is to be called at the top of the main file
% with nothing or the main filename (without extension) as argument.
% First, it breaks loops.
% If the argument is not empty and does not match |\childdocname|
% (which is set by the first inclusion of |childdoc.def|),
% |\ifchilddoc| is set to true, |\includeonly| is applied to the child file
% and |\jobname| is set to the main file
% (for proper handling of |.aux| files):
%    \begin{macrocode}
\newcommand{\childdocmain}[1]
{
  \childdocdisable\childdocmain{}
  \if?#1?\else
    \begingroup
      \def\childdoctmp{#1}
      \ifx\childdoctmp\childdocname
        \def\childdoctmp{}
      \else
        \def\childdoctmp
        {
          \childdoctrue
          \includeonly{\childdocname}
          \def\childdocjob{#1}
          \def\jobname{#1}
        }
      \fi
      \expandafter
    \endgroup
    \childdoctmp
  \fi
}
%    \end{macrocode}

% \macro{\childdocof}
% The command |\childdocof| redirects
% compilation to the main file |#1|.
%    \begin{macrocode}
\newcommand{\childdocof}[1]
{
  \childdocdisable
  \childdoctrue
  \includeonly{\childdocname}
  \def\jobname{#1}
  \def\childdocjob{#1}
  \input{#1}
}
%    \end{macrocode}

% \macro{\childdocby}
% The command |\childdocby| ....
%    \begin{macrocode}
\newcommand{\childdocby}[2][]
{
  \childdocdisable
  \childdoctrue
  \childdocmanualtrue
  \if?#1?\else
    \def\jobname{#2}
  \fi
  \def\childdocjob{#2}
  \input{#2}
  \endinput
}
%    \end{macrocode}

% \macro{\childdocforward}
% The command |\childdocforward| redirects
% compilation to the main file or
% (if the optional argument is given) a child file.
% Parameters are set as if the main file
% or a child file starting with |\childdocof| was compiled.
% Then compilation is handed over to the main file:
%    \begin{macrocode}
\newcommand{\childdocforward}[2][]
{
  \begingroup
    \if?#1?
      \def\childdoctmp
      {
        \def\childdocname{#2}
        \def\childdocjob{#2}
        \def\jobname{#2}
        \input{#2}
        \endinput
      }
    \else
      \def\childdoctmp
      {
        \childdocdisable
        \def\childdocname{#2}
        \childdoctrue
        \includeonly{#2}
        \def\childdocjob{#1}
        \def\jobname{#1}
        \input{#1}
        \endinput
      }
    \fi
    \expandafter
  \endgroup
  \childdoctmp
}
%    \end{macrocode}

% \macro{\childdocforwardprefix}
% The command |\childdocforwardprefix| redirects
% compilation to the main or a child file by means of a pattern.
% The prefix |#1| in the current filename is replaced by |#2|
% and the suffix of the current filename is kept
% (it is assumed that the filename does not contain the substring `|~~~|'
% which is used as a delimiter).
% Compilation is handed over to the new file by |\childdocforward|:
%    \begin{macrocode}
\newcommand{\childdocforwardprefix}[3][]
{
  \begingroup
    \def\childdocextract #2##1~~~{\def\childdoctmp{\childdocforward[#1]{#3##1}}}
    \expandafter\childdocextract\childdocname~~~
    \expandafter
  \endgroup
  \childdoctmp
}
%    \end{macrocode}

% \macro{\childdoc}
% The deprecated macro |\childdoc| is a legacy version of |\childdocmain|:
%    \begin{macrocode}
\newcommand{\childdoc}{\childdocmain}
%    \end{macrocode}

% \macro{\childdocredirect}
% The deprecated macro |\childdocredirect| is a legacy version
% of |\childdocforward| and |\childdocforwardprefix|:
%    \begin{macrocode}
\newcommand{\childdocredirect}[2][]
{
  \begingroup
    \if?#1?
      \def\childdoctmp{\childdocforward{#2}}
    \else
      \def\childdoctmp{\childdocforwardprefix{#1}{#2}}
    \fi
    \expandafter
  \endgroup
  \childdoctmp
}
%    \end{macrocode}

%\iffalse
%</package>
%\fi
%
\endinput
\childdocforward[cdocsamp]{cdocsch1}"|\\
% |latex -jobname cdocscl2 \|\\
% |  "\def\version{final}% \iffalse
%
% childdoc.dtx Copyright (C) 2017-2018 Niklas Beisert
%
% This work may be distributed and/or modified under the
% conditions of the LaTeX Project Public License, either version 1.3
% of this license or (at your option) any later version.
% The latest version of this license is in
%   http://www.latex-project.org/lppl.txt
% and version 1.3 or later is part of all distributions of LaTeX
% version 2005/12/01 or later.
%
% This work has the LPPL maintenance status `maintained'.
%
% The Current Maintainer of this work is Niklas Beisert.
%
% This work consists of the files childdoc.dtx and childdoc.ins
% and the derived files childdoc.def and cdocsamp.tex with
% cdocsch1.tex, cdocsch2.tex, cdocsdrf.tex, cdocsfn1.tex, cdocsfn2.tex.
%
%<package>\ifdefined\childdocmain\endinput\fi
%<package>\ProvidesFile{childdoc.def}[2018/12/30 v2.0 child document driver]
%<samplemain>\ProvidesFile{cdocsamp.tex}[2018/12/30 v2.0 sample for childdoc]
%<*driver>
%\ProvidesFile{childdoc.drv}[2018/12/30 v2.0 childdoc reference manual file]
\PassOptionsToClass{10pt,a4paper}{article}
\documentclass{ltxdoc}

\usepackage[margin=35mm]{geometry}
\usepackage{hyperref}
\usepackage{hyperxmp}
\usepackage[usenames]{color}

\hypersetup{colorlinks=true}
\hypersetup{pdfstartview=FitH}
\hypersetup{pdfpagemode=UseNone}
\hypersetup{pdfsource={}}
\hypersetup{pdflang={en-UK}}
\hypersetup{pdfcopyright={Copyright 2017-2018 Niklas Beisert.
  This work may be distributed and/or modified under the
  conditions of the LaTeX Project Public License, either version 1.3
  of this license or (at your option) any later version.}}
\hypersetup{pdflicenseurl={http://www.latex-project.org/lppl.txt}}
\hypersetup{pdfcontactaddress={ETH Zurich, ITP, HIT K,
  Wolfgang-Pauli-Strasse 27}}
\hypersetup{pdfcontactpostcode={8093}}
\hypersetup{pdfcontactcity={Zurich}}
\hypersetup{pdfcontactcountry={Switzerland}}
\hypersetup{pdfcontactemail={nbeisert@itp.phys.ethz.ch}}
\hypersetup{pdfcontacturl={http://people.phys.ethz.ch/\xmptilde nbeisert/}}

\newcommand{\secref}[1]{\hyperref[#1]{section \ref*{#1}}}

\parskip1ex
\parindent0pt
\let\olditemize\itemize
\def\itemize{\olditemize\parskip0pt}

\begin{document}

\title{The \textsf{childdoc} Package}
\hypersetup{pdftitle={The childdoc Package}}
\author{Niklas Beisert\\[2ex]
  Institut f\"ur Theoretische Physik\\
  Eidgen\"ossische Technische Hochschule Z\"urich\\
  Wolfgang-Pauli-Strasse 27, 8093 Z\"urich, Switzerland\\[1ex]
  \href{mailto:nbeisert@itp.phys.ethz.ch}
  {\texttt{nbeisert@itp.phys.ethz.ch}}}
\hypersetup{pdfauthor={Niklas Beisert}}
\hypersetup{pdfsubject={Manual for the LaTeX2e Package childdoc}}
\date{30 December 2018, \textsf{v2.0}}
\maketitle

\begin{abstract}\noindent
\textsf{childdoc} is a \LaTeXe{} package
that enables the direct compilation
of document sections included by |\include|
to individual files.
\end{abstract}

\begingroup
\parskip0ex
\tableofcontents
\endgroup

%%%%%%%%%%%%%%%%%%%%%%%%%%%%%%%%%%%%%%%%%%%%%%%%%%%%%%%%%%%%%%%%%%%%%%%%%%%%%%%%
%%%%%%%%%%%%%%%%%%%%%%%%%%%%%%%%%%%%%%%%%%%%%%%%%%%%%%%%%%%%%%%%%%%%%%%%%%%%%%%%
\section{Introduction}

\LaTeX{} provides a mechanism to structure a large document (such as a book)
into a main file and several child files (containing the chapters)
using the |\include| command.
This mechanism is beneficial for documents
which span hundreds of pages in order to
make the source file(s) more manageable.
Moreover, compilation can be restricted to
selected child files by means of the |\includeonly| command.
The latter feature can be used to reduce the compilation time while editing
(this was significantly more useful in the earlier days of \LaTeX{})
or to generate a smaller document which is easier to navigate.
Another application of |\includeonly| is to generate
documents consisting of selected parts of the complete document.

However, there are a few drawbacks of the plain |\include| mechanism:
\begin{itemize}
\item
The child files cannot be compiled on their own,
they can only be compiled via the main file.
A naive editing environment
(such as a text editor with an option
to have the current file processed by \LaTeX)
may require one to switch to the main file before compiling;
attempting to compile the child file produces errors.
\item
The main file must be modified (each time)
to adjust the |\includeonly| command
to the present needs. This easily leaves the main file in a messy state.
\item
The generated document will always carry the filename
of the main document. This is inconvenient if
several child files are to be compiled and
to be kept for distribution.
\end{itemize}

The present package provides a simple interface
to make child files individually compilable by \LaTeX{}.
Compiling a child file then has the same effect as compiling
the main file with an |\includeonly| command
to select the appropriate child.
Moreover the generated document will carry the name of the child
rather than the main file.
This resolves all three above issues.

This feature is meant to make the editing of books,
thesis documents and lecture notes somewhat more convenient.
However, the package can also be used efficiently for
composing a series of documents (such as exercise sheets)
which are typically distributed individually.
It then assists the author in generating the individual documents
(potentially in different versions)
as well as a document containing the collected series.
Another application is in developing style files
or other kinds of included material
where compilation of the style file could redirect
to a sample or test file.

%%%%%%%%%%%%%%%%%%%%%%%%%%%%%%%%%%%%%%%%%%%%%%%%%%%%%%%%%%%%%%%%%%%%%%%%%%%%%%%%
%%%%%%%%%%%%%%%%%%%%%%%%%%%%%%%%%%%%%%%%%%%%%%%%%%%%%%%%%%%%%%%%%%%%%%%%%%%%%%%%
\section{Usage}

First of all, the package \textsf{childdoc} is \emph{not} a standard
\LaTeXe{} |.sty| style file! Therefore it needs to be invoked in
a non-standard way.

%%%%%%%%%%%%%%%%%%%%%%%%%%%%%%%%%%%%%%%%%%%%%%%%%%%%%%%%%%%%%%%%%%%%%%%%%%%%%%%%
\subsection{Included Files}
\label{sec:include}

%%%%%%%%%%%%%%%%%%%%%%%%%%%%%%%%%%%%%%%%
\DescribeMacro{\childdocmain}
To use the package, add the commands
\begin{center}
\begin{tabular}{l}
|\input{childdoc.def}|\\
|\childdocmain{}|\\
\end{tabular}
\end{center}
at the very top of the main \LaTeX{} file,
in particular \emph{before} the |\documentclass| statement!
The argument of |\childdocmain| should be left empty
(but it must be present).

%%%%%%%%%%%%%%%%%%%%%%%%%%%%%%%%%%%%%%%%
\DescribeMacro{\childdocof}
Furthermore, add the commands
\begin{center}
\begin{tabular}{l}
|\input{childdoc.def}|\\
|\childdocof{|\textit{main}|}|\\
\end{tabular}
\end{center}
at the top of every child file \textit{child}
which is included by |\include{|\textit{child}|}|
from within the main file
(or at least for those files to be compiled individually).
The argument \textit{main} must be the filename of the main file.

There are a couple of
considerations in setting up the main and child documents:

%%%%%%%%%%%%%%%%%%%%%%%%%%%%%%%%%%%%%%%%
\paragraph{Restrictions.}

Please note the following restrictions:
\begin{itemize}
\item
|\childdocmain| must be called with one argument \textit{main}
to ensure compatibility with earlier version of the package.
It must either be empty (|\childdocmain{}|)
or precisely match the filename of the main file in which it is specified.
See \secref{sec:detection} for further information.
\item
The filename \textit{main} must be specified without the |.tex| extension.
\item
The filename \textit{main} is case sensitive
(even in case-insensitive file systems)
due to internal string comparison.
\item
The argument \textit{main} should be fully expanded, it cannot be a macro.
\item
Subdirectories and special characters should be avoided in filenames.
\item
The command |\childdocmain{|\textit{main}|}| must be followed by a whitespace.
It should not be followed immediately by another command
or by a comment mark `|%|'.
This is because the \TeX{} parser reads the token immediately following
the argument of |\childdocmain| and puts it
at the beginning of every child section;
however, a white\-space is ignored.
\end{itemize}

%%%%%%%%%%%%%%%%%%%%%%%%%%%%%%%%%%%%%%%%
\paragraph{Content of Main File.}

It is advisable to place all content in the child files included by |\include|.
Any output contained in the main file will appear in all child documents
unless suppressed manually;
it cannot be suppressed automatically by the |\includeonly| directive
and thus should normally be avoided.
A method to include some content in the main file
by means of conditional processing is described in \secref{sec:conditional}.

%%%%%%%%%%%%%%%%%%%%%%%%%%%%%%%%%%%%%%%%
\paragraph{Page Numbering.}

When only a part of the document is compiled,
the appropriate numbering of pages
(as well as other status parameters)
is determined from the |.aux| files.
The latter contain information from previous passes.
However this information needs to propagate through
all intermediate child documents.
Therefore the page numbering in child documents may well
be inconsistent until the complete document is compiled at least once.

A useful (if unconventional) way to always ensure a consistent
page numbering is to restart the numbering in each child document
and denote the pages by `\textit{child}|.|\textit{page}'
where \textit{child} represents the chapter/section number of the child file.
This can be achieved by the command
|\numberwithin{page}{|\textit{child}|}|
of the \textsf{amsmath} package
where \textit{child} can be |chapter| or |section|
depending on the chosen structuring.
Alternatively, one can modify the macro |\thepage| appropriately
and reset the counter |page| at the start of each child file.

%%%%%%%%%%%%%%%%%%%%%%%%%%%%%%%%%%%%%%%%%%%%%%%%%%%%%%%%%%%%%%%%%%%%%%%%%%%%%%%%
\subsection{Conditional Processing}
\label{sec:conditional}

The package provides a mechanism to compile different versions
of a document. To customise the versions further some conditional processing
can come in handy to distinguish which version is being compiled.
The package provides two macros to describe the compilation context:

%%%%%%%%%%%%%%%%%%%%%%%%%%%%%%%%%%%%%%%%
\DescribeMacro{\ifchilddoc}
The conditional |\ifchilddoc| distinguishes between the compilation of
child documents and the main document:
%
\begin{center}
|\ifchilddoc |\textit{child-code}| |[|\||else |\textit{main-code}]| \||fi|
\end{center}

%%%%%%%%%%%%%%%%%%%%%%%%%%%%%%%%%%%%%%%%
\DescribeMacro{\childdocname}
\DescribeMacro{\childdocjob}
The macro |\childdocname| contains the filename (without extension)
of the main or child file being processed.
Note that |\childdocjob| will always contain the name of the main file.

%%%%%%%%%%%%%%%%%%%%%%%%%%%%%%%%%%%%%%%%
\paragraph{Title Page.}

Conditional processing can be used to include a title or banner page
in the main document when proper precautions are taken.
Importantly, the code in the main file should ensure that the page counter
(as well as other status parameters which are stored in the |.aux| files)
takes the same value after the conditional processing.
Otherwise the page numbers may take divergent values
depending on which part is compiled.

For example, a title page could be declared by:
%
\begin{center}
\begin{tabular}{l}
|\ifchilddoc\||else|\\
|\addtocounter{page}{-1}|\\
\textit{code for title page}\\
|\newpage|\\
|\||fi|
\end{tabular}
\end{center}
%
A banner page for the child documents can be generated by:
%
\begin{center}
\begin{tabular}{l}
|\ifchilddoc|\\
|\addtocounter{page}{-1}|\\
\textit{code for banner page}\\
|\newpage|\\
|\||fi|
\end{tabular}
\end{center}
%
Here one could write a message such as:
\begin{center}
|This is the part \childdocname{} of \childdocjob{}.|
\end{center}

%%%%%%%%%%%%%%%%%%%%%%%%%%%%%%%%%%%%%%%%%%%%%%%%%%%%%%%%%%%%%%%%%%%%%%%%%%%%%%%%
\subsection{Flags}
\label{sec:flags}

The package makes it easy to generate different versions
of the main or child documents.
To this end compilation flags can be defined
and assigned different default values.
They will be particularly useful in conjunction
with the forwarding mechanism described in \secref{sec:forward}.

For example, it may be useful to have a flag |\version|
which can be set to |draft| or |final|.
The document source will contain some conditional code
depending on the value of |\version|.
Suppose further, the flag should default to |final| for the main file
and to |draft| for child files
which is a natural assignment for editing the document.
This is achieved by placing the following code
in the preamble of the main document
(below the |\childdocmain| directive):
%
\begin{center}
\begin{tabular}{l}
|\ifchilddoc|\\
|\providecommand{\version}{draft}|\\
|\||else|\\
|\providecommand{\version}{final}|\\
|\||fi|
\end{tabular}
\end{center}
%
The definition by |\providecommand| makes sure
that previous definitions are not overwritten.
Further statements |\providecommand{\version}{...}|
can thus be added before the above code to override it.

For the main file, one might add a line
(between |\childdocmain| and the above block)
%
\begin{center}
|%\ifchilddoc\||else\providecommand{\version}{draft}\||fi|
\end{center}
%
which can be uncommented to produce a draft version.
Likewise one can add a line to the very top of a child file
(above the |\childdocof{|\textit{main}|}| directive)
%
\begin{center}
|%\providecommand{\version}{final}|
\end{center}
%
which can be uncommented to produce the final version of this child document.

%%%%%%%%%%%%%%%%%%%%%%%%%%%%%%%%%%%%%%%%%%%%%%%%%%%%%%%%%%%%%%%%%%%%%%%%%%%%%%%%
\subsection{Forwarding}
\label{sec:forward}

Different versions of the main or child documents
using compilation flags as described in \secref{sec:flags}
can be (permanently) stored in different files
for convenient compilation, viewing and distribution.
To this end, the package defines a command
to pass on compilation to a different file:

%%%%%%%%%%%%%%%%%%%%%%%%%%%%%%%%%%%%%%%%
\DescribeMacro{\childdocforward}
The command |\childdocforward| redirects processing to
another source file:
%
\begin{center}
\begin{tabular}{l}
|\input{childdoc.def}|\\
|\childdocforward[|\textit{main}|]{|\textit{dest}|}|\\
\end{tabular}
\end{center}
%
The argument \textit{dest} is the destination file
(without extension).
It should be the main file or one of the child files.
Note that further \textsf{childdoc} directives
such as |\childdocof| and |\childdocforward|
in the indicated file will be processed in this form.
The optional argument \textit{main}
passes on directly to the main file \textit{main}
while pretending to compile the child \textit{dest}.
This form behaves as if \textit{dest}
issues |\childdocof{|\textit{main}|}| right away,
and no further \textsf{childdoc} directives will be processed.

%%%%%%%%%%%%%%%%%%%%%%%%%%%%%%%%%%%%%%%%
\DescribeMacro{\...prefix}
In the alternative form |\childdocforwardprefix|,
%
\begin{center}
\begin{tabular}{l}
|\input{childdoc.def}|\\
|\childdocforwardprefix[|\textit{main}|]{|\textit{prefix}|}{|\textit{dest}|}|
\end{tabular}
\end{center}
%
the destination file is determined by a pattern
depending on the current file:
To make this work, the current file must be called
`{\textit{prefix}\hspace{0.2em}\textit{suffix}}'
with \textit{prefix} matching precisely the argument.
Processing is then passed on to the file
`{\textit{dest}\hspace{0.2em}\textit{suffix}}'.
Surely, the same effect is achieved by
directly specifying the
argument `{\textit{dest}\hspace{0.2em}\textit{suffix}}'
in the first form.
However, that requires to set up a different file
for each child. With the alternative form of the command
all these files can have exactly the same content
which simplifies setting them up and maintaining them.

For example, the following file |draft.tex|
with a compilation flag |\version| as described in \secref{sec:flags}
compiles the main document as a draft:
%
\begin{center}
\begin{tabular}{l}
|\def\version{draft}|\\
|\input{childdoc.def}|\\
|\childdocforward{|\textit{main}|}|
\end{tabular}
\end{center}
%
Likewise, the following files |final|\textit{nn}|.tex|
compile the final version of the child document
|child|\textit{nn}|.tex|:
%
\begin{center}
\begin{tabular}{l}
|\def\version{final}|\\
|\input{childdoc.def}|\\
|\childdocforwardprefix{final}{child}|
\end{tabular}
\end{center}
%

Note that when several versions of a main file and/or of each child file
are to be generated, it may be convenient to set up a |Makefile| or
shell script to automatise the process.

%%%%%%%%%%%%%%%%%%%%%%%%%%%%%%%%%%%%%%%%%%%%%%%%%%%%%%%%%%%%%%%%%%%%%%%%%%%%%%%%
\subsection{Command Line Processing}
\label{sec:commandline}

The effect of redirection files can also be achieved by invoking
the \LaTeX{} compiler with a more elaborate command line.
Most conveniently this should be done as part
of a shell script or a |Makefile|.

When using \textsf{childdoc} in the main file, the following
command lines effectively perform a redirection
(note that depending on the shell being used,
backslashes may have to be doubled: `|\|' $\to$ `|\\|'):
%
\begin{center}
|... -jobname "|\textit{target}|" |\\|"|[\textit{flags}]%
|\input{childdoc.def}\childdocforward[|\textit{main}|]{|\textit{dest}|}"|
\end{center}
%
Here \textit{target} is the name of the output file,
\textit{main} is the name of the main file
and \textit{dest} is the name of the main or child file to be processed
(all filenames without extensions).
The optional argument \textit{main} can be omitted
if \textit{main} matches \textit{dest}.
Optionally, compilation \textit{flags} can be defined via |\def| commands.
This command line makes the \TeX{} engine believe
it is compiling the file \textit{target}
whose content is specified as the latter parameter.
The provided code then forwards the processing to
\textit{main} or \textit{dest} as described in \secref{sec:forward}.

%%%%%%%%%%%%%%%%%%%%%%%%%%%%%%%%%%%%%%%%%%%%%%%%%%%%%%%%%%%%%%%%%%%%%%%%%%%%%%%%
\subsection{Include by Input}
\label{sec:input}

Including child documents by |\include| has some restrictions by design.
Most notably, the content of a child document always occupies
its own set of pages; pages cannot be shared between child documents.
Usually, this behaviour makes perfect sense
because each child document contain an essential part of the document.
However, in some situations it may be desirable to compose
a document from a collection of parts
without having mandatory page breaks between then.
For this case, the package
provides a mechanism to include parts
by |\input| which can also be processed individually.
However, by construction this mechanism
requires manual handling of the content to be output.

%%%%%%%%%%%%%%%%%%%%%%%%%%%%%%%%%%%%%%%%
\DescribeMacro{\ifchilddocmanual}
The main file should be prepared as usual, see \secref{sec:include}.
However, the document body must make a distinction
between processing of an individual part and of the main document, e.g.:
%
\begin{center}
\begin{tabular}{l}
|\ifchilddocmanual|\\
|\input{\childdocname}|\\
|\||else|\\
\textit{document body with }|\input{|\textit{part}|}|\\
|\||fi|
\end{tabular}
\end{center}
%
The conditional |\ifchilddocmanual| is true whenever
a part to be included by |\input| is being compiled,
and the name of the part is stored in |\childdocname|.

%%%%%%%%%%%%%%%%%%%%%%%%%%%%%%%%%%%%%%%%
\DescribeMacro{\childdocby}
Each part to be included by |\input| should start with:
%
\begin{center}
\begin{tabular}{l}
|\input{childdoc.def}|\\
|\childdocby{|\textit{main}|}|\\
\end{tabular}
\end{center}
%
The directive |\childdocby| is similar to |\childdocof|
described in \secref{sec:include},
but the subsequent selection of content must be done manually.
To that end, both |\ifchilddoc| and |\ifchilddocmanual|
will be true upon processing of a part,
and the name of the part is stored in |\childdocname|.
Note that |\jobname| will be set to the filename of the current part
so that each part receives an individual |.aux| file
that does not interfere with the |.aux| file(s) of the main document.
This behaviour can be altered by the alternative form
|\childdocby[*]{|\textit{main}|}| (with a non-empty optional argument)
which uses the |.aux| file of the main document
by setting |\jobname| to \textit{main}.

%%%%%%%%%%%%%%%%%%%%%%%%%%%%%%%%%%%%%%%%%%%%%%%%%%%%%%%%%%%%%%%%%%%%%%%%%%%%%%%%
\subsection{Driver Development}
\label{sec:driver}

The \textsf{childdoc} mechanism can also be use for the development
of definition files such as \LaTeX{} styles or classes.
This case differs from the above setup with multiple parts
included by |\include| in that no |\includeonly| should be invoked.
This can be achieved by starting the include file
(before |\ProvidesPackage|) with:
%
\begin{center}
\begin{tabular}{l}
|\input{childdoc.def}|\\
|\childdocforward{|\textit{main}|}|\\
\end{tabular}
\end{center}
%
or alternatively with:
%
\begin{center}
\begin{tabular}{l}
|\input{childdoc.def}|\\
|\childdocby{|\textit{main}|}|\\
\end{tabular}
\end{center}
%
Both forms have slightly different effects as described above.
The main file is prepared as usual, see \secref{sec:include}.

%%%%%%%%%%%%%%%%%%%%%%%%%%%%%%%%%%%%%%%%%%%%%%%%%%%%%%%%%%%%%%%%%%%%%%%%%%%%%%%%
\subsection{Legacy Detection}
\label{sec:detection}

The directive |\childdocmain| in the main file can detect
whether the complete document or merely a child is to be compiled
even without using the directive |\childdocof|.
This method is deprecated because it is less robust
and there is no compelling reason to use it;
it is merely provided for backward compatibility
and it may be removed in future versions.

If the detection mechanism is to be used,
it is mandatory to correctly specify
the filename of the main file as the argument of |\childdocmain|:
%
\begin{center}
\begin{tabular}{l}
|\input{childdoc.def}|\\
|\childdocmain{|\textit{main}|}|\\
\end{tabular}
\end{center}
%
If |\jobname| does not match the argument \textit{main} of |\childdocmain|,
it is assumed that |\jobname| points to the child file to be compiled.
When using |\childdocmain| with the main file specified as argument,
it suffices to start a child file
with just |\input{|\textit{main}|}|
without loading of the package and using |\childdocof|.
If instead all processing is done
with the appropriate \textsf{childdoc} directives,
the argument of \textit{main} of |\childdocmain| can be empty.

An alternative version of the command line processing described
in \secref{sec:commandline} using the detection mechanism reads:
%
\begin{center}
|... -jobname "|\textit{target}|" "|[\textit{flags}]%
[|\def\jobname{|\textit{dest}|}|]|\input{|\textit{main}|}"|
\end{center}

%%%%%%%%%%%%%%%%%%%%%%%%%%%%%%%%%%%%%%%%%%%%%%%%%%%%%%%%%%%%%%%%%%%%%%%%%%%%%%%%
\subsection{Manual Code}
\label{sec:manual}

In case one cannot be certain whether the definitions file |childdoc.def|
is installed on the target \TeX{} distribution
and one prefers not to ship it,
it is conceivable to paste a few relevant commands into the sources.

To that end, drop all statements |\input{childdoc.def}|
and perform the replacements as outlined below.
Instead of |\childdocmain{|\textit{main}|}| add the following code
to the top of the main file:
%
\begin{center}
\begin{tabular}{l}
|\||ifdefined\childdocname\endinput\||fi\newif\ifchilddoc|\\
|\edef\childdocname{\scantokens\expandafter{\jobname\noexpand}}|\\
|\def\childdocmain{|\textit{main}|}\||ifx\childdocmain\childdocname\||else|\\
|\childdoctrue\includeonly{\childdocname}\let\jobname\childdocmain\||fi|\\
\end{tabular}
\end{center}
%
Instead of |\childdocof{|\textit{main}|}| just include the main file
at the top of each child file:
%
\begin{center}
|\input{|\textit{main}|}|
\end{center}
%
A simple redirection |\childdocforward{|\textit{dest}|}| is achieved by:
%
\begin{center}
|\def\jobname{|\textit{dest}|}\input{\jobname}|
\end{center}
%
The redirection with prefix
|\childdocforwardprefix[|\textit{prefix}|]{|\textit{dest}|}|
is accomplished by:
%
\begin{center}
\begin{tabular}{l}
|{\edef\jobname{\scantokens\expandafter{\jobname\noexpand}}|\\
|\def\redirectjob |\textit{prefix}|#1~~~{\gdef\jobname{|\textit{dest}|#1}}|\\
|\expandafter\redirectjob\jobname~~~}\input{\jobname}|
\end{tabular}
\end{center}

In an alternative approach,
child documents can be compiled by a specific command line
without additional code or specific definitions:
%
\begin{center}
|... -jobname "|\textit{target}|" "|[\textit{flags}]%
|\includeonly{|\textit{dest}|}\input{|\textit{main}|}"|
\end{center}
%

%%%%%%%%%%%%%%%%%%%%%%%%%%%%%%%%%%%%%%%%%%%%%%%%%%%%%%%%%%%%%%%%%%%%%%%%%%%%%%%%
%%%%%%%%%%%%%%%%%%%%%%%%%%%%%%%%%%%%%%%%%%%%%%%%%%%%%%%%%%%%%%%%%%%%%%%%%%%%%%%%
\section{Information}

%%%%%%%%%%%%%%%%%%%%%%%%%%%%%%%%%%%%%%%%%%%%%%%%%%%%%%%%%%%%%%%%%%%%%%%%%%%%%%%%
\subsection{Copyright}

Copyright \copyright{} 2017--2018 Niklas Beisert

This work may be distributed and/or modified under the
conditions of the \LaTeX{} Project Public License, either version 1.3
of this license or (at your option) any later version.
The latest version of this license is in
  \url{http://www.latex-project.org/lppl.txt}
and version 1.3 or later is part of all distributions of \LaTeX{}
version 2005/12/01 or later.

This work has the LPPL maintenance status `maintained'.

The Current Maintainer of this work is Niklas Beisert.

This work consists of the files |README.txt|, |childdoc.ins| and |childdoc.dtx|
as well as the derived files |childdoc.def|, |cdocsamp.tex|
with |cdocsch1.tex|, |cdocsch2.tex|, |cdocspt3.tex|, |cdocspt4.tex|,
|cdocsdrf.tex|, |cdocsfn1.tex|, |cdocsfn2.tex|
as well as |childdoc.pdf|.

%%%%%%%%%%%%%%%%%%%%%%%%%%%%%%%%%%%%%%%%%%%%%%%%%%%%%%%%%%%%%%%%%%%%%%%%%%%%%%%%
\subsection{Files and Installation}

The package consists of the files:
%
\begin{center}
\begin{tabular}{ll}
    |README.txt|   & readme file \\
    |childdoc.ins| & installation file \\
    |childdoc.dtx| & source file \\
    |childdoc.def| & definition file \\
    |cdocsamp.tex| & sample main file \\
    |cdocsch1.tex| & sample include file \\
    |cdocsch2.tex| & sample include file \\
    |cdocspt3.tex| & sample part file \\
    |cdocspt4.tex| & sample part file \\
    |cdocsdrf.tex| & sample redirection file \\
    |cdocsfn1.tex| & sample redirection file \\
    |cdocsfn2.tex| & sample redirection file \\
    |childdoc.pdf| & manual
\end{tabular}
\end{center}
%
The distribution consists of the files
|README.txt|, |childdoc.ins| and |childdoc.dtx|.
%
\begin{itemize}
\item
Run (pdf)\LaTeX{} on |childdoc.dtx|
to compile the manual |childdoc.pdf| (this file).
\item
Run \LaTeX{} on |childdoc.ins| to create the definitions file |childdoc.def|
and the sample |cdocsamp.tex| with include files
|cdocsch1.tex|, |cdocsch2.tex|, |cdocspt3.tex|, |cdocspt4.tex|,
|cdocsdrf.tex|, |cdocsfn1.tex|, |cdocsfn2.tex|.
Then copy the file |childdoc.def| to an appropriate directory of your \LaTeX{}
distribution, e.g.\ \textit{texmf-root}|/tex/latex/childdoc|.
\end{itemize}

%%%%%%%%%%%%%%%%%%%%%%%%%%%%%%%%%%%%%%%%%%%%%%%%%%%%%%%%%%%%%%%%%%%%%%%%%%%%%%%%
\subsection{Related CTAN Packages}

There are several other packages which offer a similar functionality:
%
\begin{itemize}
\item
The packages
\href{http://ctan.org/pkg/docmute}{\textsf{docmute}},
\href{http://ctan.org/pkg/includex}{\textsf{includex}} and
\href{http://ctan.org/pkg/standalone}{\textsf{standalone}}
provide commands to include only the document body of
a child file thus allowing both files to be compiled individually.
\item
The packages \href{http://ctan.org/pkg/subdocs}{\textsf{subdocs}}
and \href{http://ctan.org/pkg/subfiles}{\textsf{subfiles}}
provide structures in which the main and child documents can be
encapsulated and allowing them to be compiled individually.
The inclusion mechanism is different from the conventional |\include|.
\item
The package \href{http://ctan.org/pkg/combine}{\textsf{combine}}
is an elaborate solution to combine several documents into one.
\end{itemize}
%
See also the CTAN topic \href{http://ctan.org/topic/subdocs}{\textsf{subdocs}}
for further related packages.
The present package differs from the above solutions in that
a document structure constructed with the conventional |\include| mechanism
just needs two extra commands at the top of every file
such that all constituent files can be compiled individually.

%%%%%%%%%%%%%%%%%%%%%%%%%%%%%%%%%%%%%%%%%%%%%%%%%%%%%%%%%%%%%%%%%%%%%%%%%%%%%%%%
%\subsection{Feature Suggestions}
%
%The following is a list of features which may be useful for future
%versions of this package:
%%
%\begin{itemize}
%\item
%\ldots
%\end{itemize}

%%%%%%%%%%%%%%%%%%%%%%%%%%%%%%%%%%%%%%%%%%%%%%%%%%%%%%%%%%%%%%%%%%%%%%%%%%%%%%%%
\subsection{Revision History}

%%%%%%%%%%%%%%%%%%%%%%%%%%%%%%%%%%%%%%%%
\paragraph{v2.0:} 2018/12/30

\begin{itemize}
\item
immediate forward processing
\item
added |\childdocby| mechanism
\item
manual restructured
\end{itemize}

%%%%%%%%%%%%%%%%%%%%%%%%%%%%%%%%%%%%%%%%
\paragraph{v1.6:} 2018/01/17

\begin{itemize}
\item
application for development of include files
\item
corrections to manual
\end{itemize}

%%%%%%%%%%%%%%%%%%%%%%%%%%%%%%%%%%%%%%%%
\paragraph{v1.5:} 2017/05/21

\begin{itemize}
\item
more complete structuring introduced
\item
|\childdocof| introduced
\item
|\childdoc| renamed to |\childdocmain|
\item
|\childredirect| renamed to |\childdocforward| and |\childdocforwardprefix|
and functionality expanded
\end{itemize}

%%%%%%%%%%%%%%%%%%%%%%%%%%%%%%%%%%%%%%%%
\paragraph{v1.0:} 2017/04/27

\begin{itemize}
\item
manual and install package
\item
first version published on CTAN
\end{itemize}

%%%%%%%%%%%%%%%%%%%%%%%%%%%%%%%%%%%%%%%%
\paragraph{v0.6:} 2017/04/26

\begin{itemize}
\item
redirection mechanism added
\end{itemize}

%%%%%%%%%%%%%%%%%%%%%%%%%%%%%%%%%%%%%%%%
\paragraph{v0.5:} 2017/04/26

\begin{itemize}
\item
functionality in definition file
\end{itemize}


%%%%%%%%%%%%%%%%%%%%%%%%%%%%%%%%%%%%%%%%%%%%%%%%%%%%%%%%%%%%%%%%%%%%%%%%%%%%%%%%
%%%%%%%%%%%%%%%%%%%%%%%%%%%%%%%%%%%%%%%%%%%%%%%%%%%%%%%%%%%%%%%%%%%%%%%%%%%%%%%%
%%%%%%%%%%%%%%%%%%%%%%%%%%%%%%%%%%%%%%%%%%%%%%%%%%%%%%%%%%%%%%%%%%%%%%%%%%%%%%%%
\appendix

\settowidth\MacroIndent{\rmfamily\scriptsize 000\ }

 \DocInput{childdoc.dtx}

\end{document}
%</driver>
% \fi
%
% %%%%%%%%%%%%%%%%%%%%%%%%%%%%%%%%%%%%%%%%%%%%%%%%%%%%%%%%%%%%%%%%%%%%%%%%%%%%%%
% %%%%%%%%%%%%%%%%%%%%%%%%%%%%%%%%%%%%%%%%%%%%%%%%%%%%%%%%%%%%%%%%%%%%%%%%%%%%%%
% \section{Sample}
%\iffalse
%<*samplemain>
%\fi
%
% The following presents a sample document
% with two chapters, two parts, a title page,
% a compile flag as well as three forwarding files to set the flag.
% It consists of eight |.tex| files:
% \begin{center}
% \begin{tabular}{ll}
% |cdocsamp.tex|&main file\\
% |cdocsch1.tex|&include file for chapter 1\\
% |cdocsch2.tex|&include file for chapter 2\\
% |cdocspt3.tex|&include file for part 3\\
% |cdocspt4.tex|&include file for part 4\\
% |cdocsdrf.tex|&forwarding file for main file in draft mode\\
% |cdocsfi1.tex|&forwarding file for final version of chapter 1\\
% |cdocsfi2.tex|&forwarding file for final version of chapter 2\\
% \end{tabular}
% \end{center}
% Each of the eight files can be compiled directly by the \LaTeX{} compiler.
%
% %%%%%%%%%%%%%%%%%%%%%%%%%%%%%%%%%%%%%%
% \paragraph{Main File.}
%
% The main file is called |cdocsamp.tex|.
%
% Load the \textsf{childdoc} definitions and
% declare the filename for the main document:
%    \begin{macrocode}
\input{childdoc.def}
\childdocmain{}
%    \end{macrocode}

% Optional override for |\version| flag:
%    \begin{macrocode}
%%\ifchilddoc\else\providecommand{\version}{draft}\fi
%    \end{macrocode}

% Define the default values for the |\version| flag
% (|final| for the main file and |draft| for childs):
%    \begin{macrocode}
\ifchilddoc
\providecommand{\version}{draft}
\else
\providecommand{\version}{final}
\fi
%    \end{macrocode}

% Load the standard document class:
%    \begin{macrocode}
\documentclass[12pt]{article}
%    \end{macrocode}

% Start the document body:
%    \begin{macrocode}
\begin{document}
%    \end{macrocode}

% Declare a title page.
% Print title, part of document being processed and version flag:
%    \begin{macrocode}
\addtocounter{page}{-1}
\begin{center}
{\LARGE\bfseries{}childdoc example\par}
\vspace{1cm}
\ifchilddoc
\ifchilddocmanual part\else chapter\fi:
`\childdocname' of `\childdocjob'\par
\else
main document: `\childdocjob'\par
\fi
version: \version\par
\end{center}
\newpage
%    \end{macrocode}

% Manually include selected file,
% otherwise process as usual:
%    \begin{macrocode}
\ifchilddocmanual
\section*{part `\childdocname'}
\input{\childdocname}
\else
%    \end{macrocode}

% Include the two chapters:
%    \begin{macrocode}
\include{cdocsch1}
\include{cdocsch2}
%    \end{macrocode}

% Include the two parts unless only chapters should be displayed:
%    \begin{macrocode}
\ifchilddoc\else
\section{part three}
\input{cdocspt3}
\section{part four}
\input{cdocspt4}
\fi
%    \end{macrocode}

% Process as usual until here:
%    \begin{macrocode}
\fi
%    \end{macrocode}

% End of document body:
%    \begin{macrocode}
\end{document}
%    \end{macrocode}
%\iffalse
%</samplemain>
%\fi
%
% %%%%%%%%%%%%%%%%%%%%%%%%%%%%%%%%%%%%%%
% \paragraph{Chapter Include Files.}
%
% The include files are called |cdocsch1.tex| and |cdocsch2.tex|.
%
%\iffalse
%<*samplechap1|samplechap2>
%\fi

% Optional override for |\version| flag:
%    \begin{macrocode}
%%\providecommand{\version}{final}
%    \end{macrocode}

% Include the main document:
%    \begin{macrocode}
\input{childdoc.def}
\childdocof{cdocsamp}
%    \end{macrocode}

%\iffalse
%</samplechap1|samplechap2>
%\fi
%
%\iffalse
%<*samplechap1>
%\fi
% Some text for chapter 1:
%    \begin{macrocode}
\section{one}
some text in chapter one
%    \end{macrocode}

%\iffalse
%</samplechap1>
%\fi
% Some text for chapter 2:
%\iffalse
%<*samplechap2>
%\fi
%    \begin{macrocode}
\section{two}
more text in chapter two
%    \end{macrocode}

%\iffalse
%</samplechap2>
%\fi
%
% %%%%%%%%%%%%%%%%%%%%%%%%%%%%%%%%%%%%%%
% \paragraph{Part Include Files.}
%
% The include files are called |cdocspt3.tex| and |cdocspt4.tex|.
%
%\iffalse
%<*samplepart3|samplepart4>
%\fi

% Optional override for |\version| flag:
%    \begin{macrocode}
%%\providecommand{\version}{final}
%    \end{macrocode}

% Include the main document:
%    \begin{macrocode}
\input{childdoc.def}
\childdocby{cdocsamp}
%    \end{macrocode}

%\iffalse
%</samplepart3|samplepart4>
%\fi
%
%\iffalse
%<*samplepart3>
%\fi
% Some text for part 3:
%    \begin{macrocode}
some text in part three
%    \end{macrocode}

%\iffalse
%</samplepart3>
%\fi
% Some text for part 4:
%\iffalse
%<*samplepart4>
%\fi
%    \begin{macrocode}
more text in part four
%    \end{macrocode}

%\iffalse
%</samplepart4>
%\fi
%
% %%%%%%%%%%%%%%%%%%%%%%%%%%%%%%%%%%%%%%
% \paragraph{Forwarding for a Complete Draft.}
%
% The following forwarding file |cdocsdrf.tex|
% compiles the main document in draft mode:
%\iffalse
%<*sampledraft>
%\fi
%    \begin{macrocode}
\def\version{draft}
\input{childdoc.def}
\childdocforward{cdocsamp}
%    \end{macrocode}

%\iffalse
%</sampledraft>
%\fi
%
% %%%%%%%%%%%%%%%%%%%%%%%%%%%%%%%%%%%%%%
% \paragraph{Forwarding for Final Version of the Chapters.}
%
% The following forwarding files |cdocsfn1.tex| and |cdocsfn2.tex|
% (with identical content)
% compile the final versions of the child documents
% |cdocsch1.tex| and |cdocsch2.tex|, respectively:
%\iffalse
%<*samplefinal>
%\fi
%    \begin{macrocode}
\def\version{final}
\input{childdoc.def}
\childdocforwardprefix[cdocsamp]{cdocsfn}{cdocsch}
%    \end{macrocode}

%\iffalse
%</samplefinal>
%\fi
%
% %%%%%%%%%%%%%%%%%%%%%%%%%%%%%%%%%%%%%%
% \paragraph{Command Line Processing.}
%
% The following three command lines generate the output files
% |cdocscld|, |cdocscl1| and |cdocscl2|
% which should be identical to
% |cdocsdrf|, |cdocsch1| and |cdocsfn2|, respectively:
% \begin{center}
% \begin{tabular}{l}
% |latex -jobname cdocscld \|\\
% |  "\def\version{draft}\input{childdoc.def}\childdocforward{cdocsamp}"|\\
% |latex -jobname cdocscl1 \|\\
% |  "\input{childdoc.def}\childdocforward[cdocsamp]{cdocsch1}"|\\
% |latex -jobname cdocscl2 \|\\
% |  "\def\version{final}\input{childdoc.def}\childdocforward{cdocsch2}"|
% \end{tabular}
% \end{center}
% Note that the trailing backslash on each first line
% merely continues the input to the second line
% (for convenient cut ant paste).
% Furthermore, the command |latex| can be replaced by any
% of its alternative versions such as |pdflatex|.
%
% %%%%%%%%%%%%%%%%%%%%%%%%%%%%%%%%%%%%%%%%%%%%%%%%%%%%%%%%%%%%%%%%%%%%%%%%%%%%%%
% %%%%%%%%%%%%%%%%%%%%%%%%%%%%%%%%%%%%%%%%%%%%%%%%%%%%%%%%%%%%%%%%%%%%%%%%%%%%%%
% \section{Implementation}
%\iffalse
%<*package>
%\fi
%
% This section describes the definitions file |childdoc.def|.

% The definitions cannot be loaded using |\usepackage| or |\RequirePackage|
% which has a mechanism to prevent loading a style file more than once.
% When loading the definitions by means of |\input|
% multiple instances have to be prevented manually:
%\iffalse
%This code needs to be before the `\ProvidesFile' directive
%which is defined at the beginning of this file.
%Therefore it is also placed there and commented out here.
%</package>
%<*discard>
%\fi
%    \begin{macrocode}
\ifdefined\childdocmain\endinput\fi
%    \end{macrocode}
%\iffalse
%</discard>
%<*package>
%\fi
%
% \macro{\ifchilddoc}
% \macro{\ifchilddocmanual}
% The conditional |\ifchilddoc| tells whether a
% child (true) or main (false) document is being compiled.
% The conditional |\ifchilddocmanual| tells whether
% the |\includeonly| mechanism is used (false) or
% the selection of child files must be performed manually (true).
% The definitions initialise to false:
%    \begin{macrocode}
\newif\ifchilddoc
\newif\ifchilddocmanual
%    \end{macrocode}

% \macro{\childdocname}
% \macro{\childdocjob}
% The macro |\childdocname| stores the name of the main document
% to be compiled. The macro |\childdocjob| stores the name of
% the document on which the \LaTeX{} compiler was originally invoked.
% The content of |\jobname| cannot be compared
% to filenames specified in the source due to different catcodes.
% The following code rescans |\jobname|, stores the result
% in |\childdocname| and saves a copy in |\childdocjob|:
%    \begin{macrocode}
\edef\childdocname{\scantokens\expandafter{\jobname\noexpand}}
\let\childdocjob\childdocname
%    \end{macrocode}

% \macro{\childdocdisable}
% The macro |\childdocdisable| prevents the main file
% from being processed more than once.
% At this stage, the main document command |\childdocmain|
% is assumed to be called once again where it should do nothing.
% Any subsequent call to it should prevent
% a secondary processing of the main document
% It overwrites the forwarding commands
% |\childdocof| and |\childdocforward|
% with empty macros to prevent further inclusions of the main document:
%    \begin{macrocode}
\newcommand{\childdocdisable}
{
  \renewcommand{\childdocmain}[1]{\renewcommand{\childdocmain}[1]{\endinput}}
  \renewcommand{\childdocof}[1]{}
  \renewcommand{\childdocby}[2][]{}
  \renewcommand{\childdocforward}[2][]{}
  \renewcommand{\childdocdisable}{}
}
%    \end{macrocode}

% \macro{\childdocmain}
% The macro |\childdocmain| is to be called at the top of the main file
% with nothing or the main filename (without extension) as argument.
% First, it breaks loops.
% If the argument is not empty and does not match |\childdocname|
% (which is set by the first inclusion of |childdoc.def|),
% |\ifchilddoc| is set to true, |\includeonly| is applied to the child file
% and |\jobname| is set to the main file
% (for proper handling of |.aux| files):
%    \begin{macrocode}
\newcommand{\childdocmain}[1]
{
  \childdocdisable\childdocmain{}
  \if?#1?\else
    \begingroup
      \def\childdoctmp{#1}
      \ifx\childdoctmp\childdocname
        \def\childdoctmp{}
      \else
        \def\childdoctmp
        {
          \childdoctrue
          \includeonly{\childdocname}
          \def\childdocjob{#1}
          \def\jobname{#1}
        }
      \fi
      \expandafter
    \endgroup
    \childdoctmp
  \fi
}
%    \end{macrocode}

% \macro{\childdocof}
% The command |\childdocof| redirects
% compilation to the main file |#1|.
%    \begin{macrocode}
\newcommand{\childdocof}[1]
{
  \childdocdisable
  \childdoctrue
  \includeonly{\childdocname}
  \def\jobname{#1}
  \def\childdocjob{#1}
  \input{#1}
}
%    \end{macrocode}

% \macro{\childdocby}
% The command |\childdocby| ....
%    \begin{macrocode}
\newcommand{\childdocby}[2][]
{
  \childdocdisable
  \childdoctrue
  \childdocmanualtrue
  \if?#1?\else
    \def\jobname{#2}
  \fi
  \def\childdocjob{#2}
  \input{#2}
  \endinput
}
%    \end{macrocode}

% \macro{\childdocforward}
% The command |\childdocforward| redirects
% compilation to the main file or
% (if the optional argument is given) a child file.
% Parameters are set as if the main file
% or a child file starting with |\childdocof| was compiled.
% Then compilation is handed over to the main file:
%    \begin{macrocode}
\newcommand{\childdocforward}[2][]
{
  \begingroup
    \if?#1?
      \def\childdoctmp
      {
        \def\childdocname{#2}
        \def\childdocjob{#2}
        \def\jobname{#2}
        \input{#2}
        \endinput
      }
    \else
      \def\childdoctmp
      {
        \childdocdisable
        \def\childdocname{#2}
        \childdoctrue
        \includeonly{#2}
        \def\childdocjob{#1}
        \def\jobname{#1}
        \input{#1}
        \endinput
      }
    \fi
    \expandafter
  \endgroup
  \childdoctmp
}
%    \end{macrocode}

% \macro{\childdocforwardprefix}
% The command |\childdocforwardprefix| redirects
% compilation to the main or a child file by means of a pattern.
% The prefix |#1| in the current filename is replaced by |#2|
% and the suffix of the current filename is kept
% (it is assumed that the filename does not contain the substring `|~~~|'
% which is used as a delimiter).
% Compilation is handed over to the new file by |\childdocforward|:
%    \begin{macrocode}
\newcommand{\childdocforwardprefix}[3][]
{
  \begingroup
    \def\childdocextract #2##1~~~{\def\childdoctmp{\childdocforward[#1]{#3##1}}}
    \expandafter\childdocextract\childdocname~~~
    \expandafter
  \endgroup
  \childdoctmp
}
%    \end{macrocode}

% \macro{\childdoc}
% The deprecated macro |\childdoc| is a legacy version of |\childdocmain|:
%    \begin{macrocode}
\newcommand{\childdoc}{\childdocmain}
%    \end{macrocode}

% \macro{\childdocredirect}
% The deprecated macro |\childdocredirect| is a legacy version
% of |\childdocforward| and |\childdocforwardprefix|:
%    \begin{macrocode}
\newcommand{\childdocredirect}[2][]
{
  \begingroup
    \if?#1?
      \def\childdoctmp{\childdocforward{#2}}
    \else
      \def\childdoctmp{\childdocforwardprefix{#1}{#2}}
    \fi
    \expandafter
  \endgroup
  \childdoctmp
}
%    \end{macrocode}

%\iffalse
%</package>
%\fi
%
\endinput
\childdocforward{cdocsch2}"|
% \end{tabular}
% \end{center}
% Note that the trailing backslash on each first line
% merely continues the input to the second line
% (for convenient cut ant paste).
% Furthermore, the command |latex| can be replaced by any
% of its alternative versions such as |pdflatex|.
%
% %%%%%%%%%%%%%%%%%%%%%%%%%%%%%%%%%%%%%%%%%%%%%%%%%%%%%%%%%%%%%%%%%%%%%%%%%%%%%%
% %%%%%%%%%%%%%%%%%%%%%%%%%%%%%%%%%%%%%%%%%%%%%%%%%%%%%%%%%%%%%%%%%%%%%%%%%%%%%%
% \section{Implementation}
%\iffalse
%<*package>
%\fi
%
% This section describes the definitions file |childdoc.def|.

% The definitions cannot be loaded using |\usepackage| or |\RequirePackage|
% which has a mechanism to prevent loading a style file more than once.
% When loading the definitions by means of |\input|
% multiple instances have to be prevented manually:
%\iffalse
%This code needs to be before the `\ProvidesFile' directive
%which is defined at the beginning of this file.
%Therefore it is also placed there and commented out here.
%</package>
%<*discard>
%\fi
%    \begin{macrocode}
\ifdefined\childdocmain\endinput\fi
%    \end{macrocode}
%\iffalse
%</discard>
%<*package>
%\fi
%
% \macro{\ifchilddoc}
% \macro{\ifchilddocmanual}
% The conditional |\ifchilddoc| tells whether a
% child (true) or main (false) document is being compiled.
% The conditional |\ifchilddocmanual| tells whether
% the |\includeonly| mechanism is used (false) or
% the selection of child files must be performed manually (true).
% The definitions initialise to false:
%    \begin{macrocode}
\newif\ifchilddoc
\newif\ifchilddocmanual
%    \end{macrocode}

% \macro{\childdocname}
% \macro{\childdocjob}
% The macro |\childdocname| stores the name of the main document
% to be compiled. The macro |\childdocjob| stores the name of
% the document on which the \LaTeX{} compiler was originally invoked.
% The content of |\jobname| cannot be compared
% to filenames specified in the source due to different catcodes.
% The following code rescans |\jobname|, stores the result
% in |\childdocname| and saves a copy in |\childdocjob|:
%    \begin{macrocode}
\edef\childdocname{\scantokens\expandafter{\jobname\noexpand}}
\let\childdocjob\childdocname
%    \end{macrocode}

% \macro{\childdocdisable}
% The macro |\childdocdisable| prevents the main file
% from being processed more than once.
% At this stage, the main document command |\childdocmain|
% is assumed to be called once again where it should do nothing.
% Any subsequent call to it should prevent
% a secondary processing of the main document
% It overwrites the forwarding commands
% |\childdocof| and |\childdocforward|
% with empty macros to prevent further inclusions of the main document:
%    \begin{macrocode}
\newcommand{\childdocdisable}
{
  \renewcommand{\childdocmain}[1]{\renewcommand{\childdocmain}[1]{\endinput}}
  \renewcommand{\childdocof}[1]{}
  \renewcommand{\childdocby}[2][]{}
  \renewcommand{\childdocforward}[2][]{}
  \renewcommand{\childdocdisable}{}
}
%    \end{macrocode}

% \macro{\childdocmain}
% The macro |\childdocmain| is to be called at the top of the main file
% with nothing or the main filename (without extension) as argument.
% First, it breaks loops.
% If the argument is not empty and does not match |\childdocname|
% (which is set by the first inclusion of |childdoc.def|),
% |\ifchilddoc| is set to true, |\includeonly| is applied to the child file
% and |\jobname| is set to the main file
% (for proper handling of |.aux| files):
%    \begin{macrocode}
\newcommand{\childdocmain}[1]
{
  \childdocdisable\childdocmain{}
  \if?#1?\else
    \begingroup
      \def\childdoctmp{#1}
      \ifx\childdoctmp\childdocname
        \def\childdoctmp{}
      \else
        \def\childdoctmp
        {
          \childdoctrue
          \includeonly{\childdocname}
          \def\childdocjob{#1}
          \def\jobname{#1}
        }
      \fi
      \expandafter
    \endgroup
    \childdoctmp
  \fi
}
%    \end{macrocode}

% \macro{\childdocof}
% The command |\childdocof| redirects
% compilation to the main file |#1|.
%    \begin{macrocode}
\newcommand{\childdocof}[1]
{
  \childdocdisable
  \childdoctrue
  \includeonly{\childdocname}
  \def\jobname{#1}
  \def\childdocjob{#1}
  \input{#1}
}
%    \end{macrocode}

% \macro{\childdocby}
% The command |\childdocby| ....
%    \begin{macrocode}
\newcommand{\childdocby}[2][]
{
  \childdocdisable
  \childdoctrue
  \childdocmanualtrue
  \if?#1?\else
    \def\jobname{#2}
  \fi
  \def\childdocjob{#2}
  \input{#2}
  \endinput
}
%    \end{macrocode}

% \macro{\childdocforward}
% The command |\childdocforward| redirects
% compilation to the main file or
% (if the optional argument is given) a child file.
% Parameters are set as if the main file
% or a child file starting with |\childdocof| was compiled.
% Then compilation is handed over to the main file:
%    \begin{macrocode}
\newcommand{\childdocforward}[2][]
{
  \begingroup
    \if?#1?
      \def\childdoctmp
      {
        \def\childdocname{#2}
        \def\childdocjob{#2}
        \def\jobname{#2}
        \input{#2}
        \endinput
      }
    \else
      \def\childdoctmp
      {
        \childdocdisable
        \def\childdocname{#2}
        \childdoctrue
        \includeonly{#2}
        \def\childdocjob{#1}
        \def\jobname{#1}
        \input{#1}
        \endinput
      }
    \fi
    \expandafter
  \endgroup
  \childdoctmp
}
%    \end{macrocode}

% \macro{\childdocforwardprefix}
% The command |\childdocforwardprefix| redirects
% compilation to the main or a child file by means of a pattern.
% The prefix |#1| in the current filename is replaced by |#2|
% and the suffix of the current filename is kept
% (it is assumed that the filename does not contain the substring `|~~~|'
% which is used as a delimiter).
% Compilation is handed over to the new file by |\childdocforward|:
%    \begin{macrocode}
\newcommand{\childdocforwardprefix}[3][]
{
  \begingroup
    \def\childdocextract #2##1~~~{\def\childdoctmp{\childdocforward[#1]{#3##1}}}
    \expandafter\childdocextract\childdocname~~~
    \expandafter
  \endgroup
  \childdoctmp
}
%    \end{macrocode}

% \macro{\childdoc}
% The deprecated macro |\childdoc| is a legacy version of |\childdocmain|:
%    \begin{macrocode}
\newcommand{\childdoc}{\childdocmain}
%    \end{macrocode}

% \macro{\childdocredirect}
% The deprecated macro |\childdocredirect| is a legacy version
% of |\childdocforward| and |\childdocforwardprefix|:
%    \begin{macrocode}
\newcommand{\childdocredirect}[2][]
{
  \begingroup
    \if?#1?
      \def\childdoctmp{\childdocforward{#2}}
    \else
      \def\childdoctmp{\childdocforwardprefix{#1}{#2}}
    \fi
    \expandafter
  \endgroup
  \childdoctmp
}
%    \end{macrocode}

%\iffalse
%</package>
%\fi
%
\endinput
|\\
|\childdocforward{|\textit{main}|}|\\
\end{tabular}
\end{center}
%
or alternatively with:
%
\begin{center}
\begin{tabular}{l}
|% \iffalse
%
% childdoc.dtx Copyright (C) 2017-2018 Niklas Beisert
%
% This work may be distributed and/or modified under the
% conditions of the LaTeX Project Public License, either version 1.3
% of this license or (at your option) any later version.
% The latest version of this license is in
%   http://www.latex-project.org/lppl.txt
% and version 1.3 or later is part of all distributions of LaTeX
% version 2005/12/01 or later.
%
% This work has the LPPL maintenance status `maintained'.
%
% The Current Maintainer of this work is Niklas Beisert.
%
% This work consists of the files childdoc.dtx and childdoc.ins
% and the derived files childdoc.def and cdocsamp.tex with
% cdocsch1.tex, cdocsch2.tex, cdocsdrf.tex, cdocsfn1.tex, cdocsfn2.tex.
%
%<package>\ifdefined\childdocmain\endinput\fi
%<package>\ProvidesFile{childdoc.def}[2018/12/30 v2.0 child document driver]
%<samplemain>\ProvidesFile{cdocsamp.tex}[2018/12/30 v2.0 sample for childdoc]
%<*driver>
%\ProvidesFile{childdoc.drv}[2018/12/30 v2.0 childdoc reference manual file]
\PassOptionsToClass{10pt,a4paper}{article}
\documentclass{ltxdoc}

\usepackage[margin=35mm]{geometry}
\usepackage{hyperref}
\usepackage{hyperxmp}
\usepackage[usenames]{color}

\hypersetup{colorlinks=true}
\hypersetup{pdfstartview=FitH}
\hypersetup{pdfpagemode=UseNone}
\hypersetup{pdfsource={}}
\hypersetup{pdflang={en-UK}}
\hypersetup{pdfcopyright={Copyright 2017-2018 Niklas Beisert.
  This work may be distributed and/or modified under the
  conditions of the LaTeX Project Public License, either version 1.3
  of this license or (at your option) any later version.}}
\hypersetup{pdflicenseurl={http://www.latex-project.org/lppl.txt}}
\hypersetup{pdfcontactaddress={ETH Zurich, ITP, HIT K,
  Wolfgang-Pauli-Strasse 27}}
\hypersetup{pdfcontactpostcode={8093}}
\hypersetup{pdfcontactcity={Zurich}}
\hypersetup{pdfcontactcountry={Switzerland}}
\hypersetup{pdfcontactemail={nbeisert@itp.phys.ethz.ch}}
\hypersetup{pdfcontacturl={http://people.phys.ethz.ch/\xmptilde nbeisert/}}

\newcommand{\secref}[1]{\hyperref[#1]{section \ref*{#1}}}

\parskip1ex
\parindent0pt
\let\olditemize\itemize
\def\itemize{\olditemize\parskip0pt}

\begin{document}

\title{The \textsf{childdoc} Package}
\hypersetup{pdftitle={The childdoc Package}}
\author{Niklas Beisert\\[2ex]
  Institut f\"ur Theoretische Physik\\
  Eidgen\"ossische Technische Hochschule Z\"urich\\
  Wolfgang-Pauli-Strasse 27, 8093 Z\"urich, Switzerland\\[1ex]
  \href{mailto:nbeisert@itp.phys.ethz.ch}
  {\texttt{nbeisert@itp.phys.ethz.ch}}}
\hypersetup{pdfauthor={Niklas Beisert}}
\hypersetup{pdfsubject={Manual for the LaTeX2e Package childdoc}}
\date{30 December 2018, \textsf{v2.0}}
\maketitle

\begin{abstract}\noindent
\textsf{childdoc} is a \LaTeXe{} package
that enables the direct compilation
of document sections included by |\include|
to individual files.
\end{abstract}

\begingroup
\parskip0ex
\tableofcontents
\endgroup

%%%%%%%%%%%%%%%%%%%%%%%%%%%%%%%%%%%%%%%%%%%%%%%%%%%%%%%%%%%%%%%%%%%%%%%%%%%%%%%%
%%%%%%%%%%%%%%%%%%%%%%%%%%%%%%%%%%%%%%%%%%%%%%%%%%%%%%%%%%%%%%%%%%%%%%%%%%%%%%%%
\section{Introduction}

\LaTeX{} provides a mechanism to structure a large document (such as a book)
into a main file and several child files (containing the chapters)
using the |\include| command.
This mechanism is beneficial for documents
which span hundreds of pages in order to
make the source file(s) more manageable.
Moreover, compilation can be restricted to
selected child files by means of the |\includeonly| command.
The latter feature can be used to reduce the compilation time while editing
(this was significantly more useful in the earlier days of \LaTeX{})
or to generate a smaller document which is easier to navigate.
Another application of |\includeonly| is to generate
documents consisting of selected parts of the complete document.

However, there are a few drawbacks of the plain |\include| mechanism:
\begin{itemize}
\item
The child files cannot be compiled on their own,
they can only be compiled via the main file.
A naive editing environment
(such as a text editor with an option
to have the current file processed by \LaTeX)
may require one to switch to the main file before compiling;
attempting to compile the child file produces errors.
\item
The main file must be modified (each time)
to adjust the |\includeonly| command
to the present needs. This easily leaves the main file in a messy state.
\item
The generated document will always carry the filename
of the main document. This is inconvenient if
several child files are to be compiled and
to be kept for distribution.
\end{itemize}

The present package provides a simple interface
to make child files individually compilable by \LaTeX{}.
Compiling a child file then has the same effect as compiling
the main file with an |\includeonly| command
to select the appropriate child.
Moreover the generated document will carry the name of the child
rather than the main file.
This resolves all three above issues.

This feature is meant to make the editing of books,
thesis documents and lecture notes somewhat more convenient.
However, the package can also be used efficiently for
composing a series of documents (such as exercise sheets)
which are typically distributed individually.
It then assists the author in generating the individual documents
(potentially in different versions)
as well as a document containing the collected series.
Another application is in developing style files
or other kinds of included material
where compilation of the style file could redirect
to a sample or test file.

%%%%%%%%%%%%%%%%%%%%%%%%%%%%%%%%%%%%%%%%%%%%%%%%%%%%%%%%%%%%%%%%%%%%%%%%%%%%%%%%
%%%%%%%%%%%%%%%%%%%%%%%%%%%%%%%%%%%%%%%%%%%%%%%%%%%%%%%%%%%%%%%%%%%%%%%%%%%%%%%%
\section{Usage}

First of all, the package \textsf{childdoc} is \emph{not} a standard
\LaTeXe{} |.sty| style file! Therefore it needs to be invoked in
a non-standard way.

%%%%%%%%%%%%%%%%%%%%%%%%%%%%%%%%%%%%%%%%%%%%%%%%%%%%%%%%%%%%%%%%%%%%%%%%%%%%%%%%
\subsection{Included Files}
\label{sec:include}

%%%%%%%%%%%%%%%%%%%%%%%%%%%%%%%%%%%%%%%%
\DescribeMacro{\childdocmain}
To use the package, add the commands
\begin{center}
\begin{tabular}{l}
|% \iffalse
%
% childdoc.dtx Copyright (C) 2017-2018 Niklas Beisert
%
% This work may be distributed and/or modified under the
% conditions of the LaTeX Project Public License, either version 1.3
% of this license or (at your option) any later version.
% The latest version of this license is in
%   http://www.latex-project.org/lppl.txt
% and version 1.3 or later is part of all distributions of LaTeX
% version 2005/12/01 or later.
%
% This work has the LPPL maintenance status `maintained'.
%
% The Current Maintainer of this work is Niklas Beisert.
%
% This work consists of the files childdoc.dtx and childdoc.ins
% and the derived files childdoc.def and cdocsamp.tex with
% cdocsch1.tex, cdocsch2.tex, cdocsdrf.tex, cdocsfn1.tex, cdocsfn2.tex.
%
%<package>\ifdefined\childdocmain\endinput\fi
%<package>\ProvidesFile{childdoc.def}[2018/12/30 v2.0 child document driver]
%<samplemain>\ProvidesFile{cdocsamp.tex}[2018/12/30 v2.0 sample for childdoc]
%<*driver>
%\ProvidesFile{childdoc.drv}[2018/12/30 v2.0 childdoc reference manual file]
\PassOptionsToClass{10pt,a4paper}{article}
\documentclass{ltxdoc}

\usepackage[margin=35mm]{geometry}
\usepackage{hyperref}
\usepackage{hyperxmp}
\usepackage[usenames]{color}

\hypersetup{colorlinks=true}
\hypersetup{pdfstartview=FitH}
\hypersetup{pdfpagemode=UseNone}
\hypersetup{pdfsource={}}
\hypersetup{pdflang={en-UK}}
\hypersetup{pdfcopyright={Copyright 2017-2018 Niklas Beisert.
  This work may be distributed and/or modified under the
  conditions of the LaTeX Project Public License, either version 1.3
  of this license or (at your option) any later version.}}
\hypersetup{pdflicenseurl={http://www.latex-project.org/lppl.txt}}
\hypersetup{pdfcontactaddress={ETH Zurich, ITP, HIT K,
  Wolfgang-Pauli-Strasse 27}}
\hypersetup{pdfcontactpostcode={8093}}
\hypersetup{pdfcontactcity={Zurich}}
\hypersetup{pdfcontactcountry={Switzerland}}
\hypersetup{pdfcontactemail={nbeisert@itp.phys.ethz.ch}}
\hypersetup{pdfcontacturl={http://people.phys.ethz.ch/\xmptilde nbeisert/}}

\newcommand{\secref}[1]{\hyperref[#1]{section \ref*{#1}}}

\parskip1ex
\parindent0pt
\let\olditemize\itemize
\def\itemize{\olditemize\parskip0pt}

\begin{document}

\title{The \textsf{childdoc} Package}
\hypersetup{pdftitle={The childdoc Package}}
\author{Niklas Beisert\\[2ex]
  Institut f\"ur Theoretische Physik\\
  Eidgen\"ossische Technische Hochschule Z\"urich\\
  Wolfgang-Pauli-Strasse 27, 8093 Z\"urich, Switzerland\\[1ex]
  \href{mailto:nbeisert@itp.phys.ethz.ch}
  {\texttt{nbeisert@itp.phys.ethz.ch}}}
\hypersetup{pdfauthor={Niklas Beisert}}
\hypersetup{pdfsubject={Manual for the LaTeX2e Package childdoc}}
\date{30 December 2018, \textsf{v2.0}}
\maketitle

\begin{abstract}\noindent
\textsf{childdoc} is a \LaTeXe{} package
that enables the direct compilation
of document sections included by |\include|
to individual files.
\end{abstract}

\begingroup
\parskip0ex
\tableofcontents
\endgroup

%%%%%%%%%%%%%%%%%%%%%%%%%%%%%%%%%%%%%%%%%%%%%%%%%%%%%%%%%%%%%%%%%%%%%%%%%%%%%%%%
%%%%%%%%%%%%%%%%%%%%%%%%%%%%%%%%%%%%%%%%%%%%%%%%%%%%%%%%%%%%%%%%%%%%%%%%%%%%%%%%
\section{Introduction}

\LaTeX{} provides a mechanism to structure a large document (such as a book)
into a main file and several child files (containing the chapters)
using the |\include| command.
This mechanism is beneficial for documents
which span hundreds of pages in order to
make the source file(s) more manageable.
Moreover, compilation can be restricted to
selected child files by means of the |\includeonly| command.
The latter feature can be used to reduce the compilation time while editing
(this was significantly more useful in the earlier days of \LaTeX{})
or to generate a smaller document which is easier to navigate.
Another application of |\includeonly| is to generate
documents consisting of selected parts of the complete document.

However, there are a few drawbacks of the plain |\include| mechanism:
\begin{itemize}
\item
The child files cannot be compiled on their own,
they can only be compiled via the main file.
A naive editing environment
(such as a text editor with an option
to have the current file processed by \LaTeX)
may require one to switch to the main file before compiling;
attempting to compile the child file produces errors.
\item
The main file must be modified (each time)
to adjust the |\includeonly| command
to the present needs. This easily leaves the main file in a messy state.
\item
The generated document will always carry the filename
of the main document. This is inconvenient if
several child files are to be compiled and
to be kept for distribution.
\end{itemize}

The present package provides a simple interface
to make child files individually compilable by \LaTeX{}.
Compiling a child file then has the same effect as compiling
the main file with an |\includeonly| command
to select the appropriate child.
Moreover the generated document will carry the name of the child
rather than the main file.
This resolves all three above issues.

This feature is meant to make the editing of books,
thesis documents and lecture notes somewhat more convenient.
However, the package can also be used efficiently for
composing a series of documents (such as exercise sheets)
which are typically distributed individually.
It then assists the author in generating the individual documents
(potentially in different versions)
as well as a document containing the collected series.
Another application is in developing style files
or other kinds of included material
where compilation of the style file could redirect
to a sample or test file.

%%%%%%%%%%%%%%%%%%%%%%%%%%%%%%%%%%%%%%%%%%%%%%%%%%%%%%%%%%%%%%%%%%%%%%%%%%%%%%%%
%%%%%%%%%%%%%%%%%%%%%%%%%%%%%%%%%%%%%%%%%%%%%%%%%%%%%%%%%%%%%%%%%%%%%%%%%%%%%%%%
\section{Usage}

First of all, the package \textsf{childdoc} is \emph{not} a standard
\LaTeXe{} |.sty| style file! Therefore it needs to be invoked in
a non-standard way.

%%%%%%%%%%%%%%%%%%%%%%%%%%%%%%%%%%%%%%%%%%%%%%%%%%%%%%%%%%%%%%%%%%%%%%%%%%%%%%%%
\subsection{Included Files}
\label{sec:include}

%%%%%%%%%%%%%%%%%%%%%%%%%%%%%%%%%%%%%%%%
\DescribeMacro{\childdocmain}
To use the package, add the commands
\begin{center}
\begin{tabular}{l}
|\input{childdoc.def}|\\
|\childdocmain{}|\\
\end{tabular}
\end{center}
at the very top of the main \LaTeX{} file,
in particular \emph{before} the |\documentclass| statement!
The argument of |\childdocmain| should be left empty
(but it must be present).

%%%%%%%%%%%%%%%%%%%%%%%%%%%%%%%%%%%%%%%%
\DescribeMacro{\childdocof}
Furthermore, add the commands
\begin{center}
\begin{tabular}{l}
|\input{childdoc.def}|\\
|\childdocof{|\textit{main}|}|\\
\end{tabular}
\end{center}
at the top of every child file \textit{child}
which is included by |\include{|\textit{child}|}|
from within the main file
(or at least for those files to be compiled individually).
The argument \textit{main} must be the filename of the main file.

There are a couple of
considerations in setting up the main and child documents:

%%%%%%%%%%%%%%%%%%%%%%%%%%%%%%%%%%%%%%%%
\paragraph{Restrictions.}

Please note the following restrictions:
\begin{itemize}
\item
|\childdocmain| must be called with one argument \textit{main}
to ensure compatibility with earlier version of the package.
It must either be empty (|\childdocmain{}|)
or precisely match the filename of the main file in which it is specified.
See \secref{sec:detection} for further information.
\item
The filename \textit{main} must be specified without the |.tex| extension.
\item
The filename \textit{main} is case sensitive
(even in case-insensitive file systems)
due to internal string comparison.
\item
The argument \textit{main} should be fully expanded, it cannot be a macro.
\item
Subdirectories and special characters should be avoided in filenames.
\item
The command |\childdocmain{|\textit{main}|}| must be followed by a whitespace.
It should not be followed immediately by another command
or by a comment mark `|%|'.
This is because the \TeX{} parser reads the token immediately following
the argument of |\childdocmain| and puts it
at the beginning of every child section;
however, a white\-space is ignored.
\end{itemize}

%%%%%%%%%%%%%%%%%%%%%%%%%%%%%%%%%%%%%%%%
\paragraph{Content of Main File.}

It is advisable to place all content in the child files included by |\include|.
Any output contained in the main file will appear in all child documents
unless suppressed manually;
it cannot be suppressed automatically by the |\includeonly| directive
and thus should normally be avoided.
A method to include some content in the main file
by means of conditional processing is described in \secref{sec:conditional}.

%%%%%%%%%%%%%%%%%%%%%%%%%%%%%%%%%%%%%%%%
\paragraph{Page Numbering.}

When only a part of the document is compiled,
the appropriate numbering of pages
(as well as other status parameters)
is determined from the |.aux| files.
The latter contain information from previous passes.
However this information needs to propagate through
all intermediate child documents.
Therefore the page numbering in child documents may well
be inconsistent until the complete document is compiled at least once.

A useful (if unconventional) way to always ensure a consistent
page numbering is to restart the numbering in each child document
and denote the pages by `\textit{child}|.|\textit{page}'
where \textit{child} represents the chapter/section number of the child file.
This can be achieved by the command
|\numberwithin{page}{|\textit{child}|}|
of the \textsf{amsmath} package
where \textit{child} can be |chapter| or |section|
depending on the chosen structuring.
Alternatively, one can modify the macro |\thepage| appropriately
and reset the counter |page| at the start of each child file.

%%%%%%%%%%%%%%%%%%%%%%%%%%%%%%%%%%%%%%%%%%%%%%%%%%%%%%%%%%%%%%%%%%%%%%%%%%%%%%%%
\subsection{Conditional Processing}
\label{sec:conditional}

The package provides a mechanism to compile different versions
of a document. To customise the versions further some conditional processing
can come in handy to distinguish which version is being compiled.
The package provides two macros to describe the compilation context:

%%%%%%%%%%%%%%%%%%%%%%%%%%%%%%%%%%%%%%%%
\DescribeMacro{\ifchilddoc}
The conditional |\ifchilddoc| distinguishes between the compilation of
child documents and the main document:
%
\begin{center}
|\ifchilddoc |\textit{child-code}| |[|\||else |\textit{main-code}]| \||fi|
\end{center}

%%%%%%%%%%%%%%%%%%%%%%%%%%%%%%%%%%%%%%%%
\DescribeMacro{\childdocname}
\DescribeMacro{\childdocjob}
The macro |\childdocname| contains the filename (without extension)
of the main or child file being processed.
Note that |\childdocjob| will always contain the name of the main file.

%%%%%%%%%%%%%%%%%%%%%%%%%%%%%%%%%%%%%%%%
\paragraph{Title Page.}

Conditional processing can be used to include a title or banner page
in the main document when proper precautions are taken.
Importantly, the code in the main file should ensure that the page counter
(as well as other status parameters which are stored in the |.aux| files)
takes the same value after the conditional processing.
Otherwise the page numbers may take divergent values
depending on which part is compiled.

For example, a title page could be declared by:
%
\begin{center}
\begin{tabular}{l}
|\ifchilddoc\||else|\\
|\addtocounter{page}{-1}|\\
\textit{code for title page}\\
|\newpage|\\
|\||fi|
\end{tabular}
\end{center}
%
A banner page for the child documents can be generated by:
%
\begin{center}
\begin{tabular}{l}
|\ifchilddoc|\\
|\addtocounter{page}{-1}|\\
\textit{code for banner page}\\
|\newpage|\\
|\||fi|
\end{tabular}
\end{center}
%
Here one could write a message such as:
\begin{center}
|This is the part \childdocname{} of \childdocjob{}.|
\end{center}

%%%%%%%%%%%%%%%%%%%%%%%%%%%%%%%%%%%%%%%%%%%%%%%%%%%%%%%%%%%%%%%%%%%%%%%%%%%%%%%%
\subsection{Flags}
\label{sec:flags}

The package makes it easy to generate different versions
of the main or child documents.
To this end compilation flags can be defined
and assigned different default values.
They will be particularly useful in conjunction
with the forwarding mechanism described in \secref{sec:forward}.

For example, it may be useful to have a flag |\version|
which can be set to |draft| or |final|.
The document source will contain some conditional code
depending on the value of |\version|.
Suppose further, the flag should default to |final| for the main file
and to |draft| for child files
which is a natural assignment for editing the document.
This is achieved by placing the following code
in the preamble of the main document
(below the |\childdocmain| directive):
%
\begin{center}
\begin{tabular}{l}
|\ifchilddoc|\\
|\providecommand{\version}{draft}|\\
|\||else|\\
|\providecommand{\version}{final}|\\
|\||fi|
\end{tabular}
\end{center}
%
The definition by |\providecommand| makes sure
that previous definitions are not overwritten.
Further statements |\providecommand{\version}{...}|
can thus be added before the above code to override it.

For the main file, one might add a line
(between |\childdocmain| and the above block)
%
\begin{center}
|%\ifchilddoc\||else\providecommand{\version}{draft}\||fi|
\end{center}
%
which can be uncommented to produce a draft version.
Likewise one can add a line to the very top of a child file
(above the |\childdocof{|\textit{main}|}| directive)
%
\begin{center}
|%\providecommand{\version}{final}|
\end{center}
%
which can be uncommented to produce the final version of this child document.

%%%%%%%%%%%%%%%%%%%%%%%%%%%%%%%%%%%%%%%%%%%%%%%%%%%%%%%%%%%%%%%%%%%%%%%%%%%%%%%%
\subsection{Forwarding}
\label{sec:forward}

Different versions of the main or child documents
using compilation flags as described in \secref{sec:flags}
can be (permanently) stored in different files
for convenient compilation, viewing and distribution.
To this end, the package defines a command
to pass on compilation to a different file:

%%%%%%%%%%%%%%%%%%%%%%%%%%%%%%%%%%%%%%%%
\DescribeMacro{\childdocforward}
The command |\childdocforward| redirects processing to
another source file:
%
\begin{center}
\begin{tabular}{l}
|\input{childdoc.def}|\\
|\childdocforward[|\textit{main}|]{|\textit{dest}|}|\\
\end{tabular}
\end{center}
%
The argument \textit{dest} is the destination file
(without extension).
It should be the main file or one of the child files.
Note that further \textsf{childdoc} directives
such as |\childdocof| and |\childdocforward|
in the indicated file will be processed in this form.
The optional argument \textit{main}
passes on directly to the main file \textit{main}
while pretending to compile the child \textit{dest}.
This form behaves as if \textit{dest}
issues |\childdocof{|\textit{main}|}| right away,
and no further \textsf{childdoc} directives will be processed.

%%%%%%%%%%%%%%%%%%%%%%%%%%%%%%%%%%%%%%%%
\DescribeMacro{\...prefix}
In the alternative form |\childdocforwardprefix|,
%
\begin{center}
\begin{tabular}{l}
|\input{childdoc.def}|\\
|\childdocforwardprefix[|\textit{main}|]{|\textit{prefix}|}{|\textit{dest}|}|
\end{tabular}
\end{center}
%
the destination file is determined by a pattern
depending on the current file:
To make this work, the current file must be called
`{\textit{prefix}\hspace{0.2em}\textit{suffix}}'
with \textit{prefix} matching precisely the argument.
Processing is then passed on to the file
`{\textit{dest}\hspace{0.2em}\textit{suffix}}'.
Surely, the same effect is achieved by
directly specifying the
argument `{\textit{dest}\hspace{0.2em}\textit{suffix}}'
in the first form.
However, that requires to set up a different file
for each child. With the alternative form of the command
all these files can have exactly the same content
which simplifies setting them up and maintaining them.

For example, the following file |draft.tex|
with a compilation flag |\version| as described in \secref{sec:flags}
compiles the main document as a draft:
%
\begin{center}
\begin{tabular}{l}
|\def\version{draft}|\\
|\input{childdoc.def}|\\
|\childdocforward{|\textit{main}|}|
\end{tabular}
\end{center}
%
Likewise, the following files |final|\textit{nn}|.tex|
compile the final version of the child document
|child|\textit{nn}|.tex|:
%
\begin{center}
\begin{tabular}{l}
|\def\version{final}|\\
|\input{childdoc.def}|\\
|\childdocforwardprefix{final}{child}|
\end{tabular}
\end{center}
%

Note that when several versions of a main file and/or of each child file
are to be generated, it may be convenient to set up a |Makefile| or
shell script to automatise the process.

%%%%%%%%%%%%%%%%%%%%%%%%%%%%%%%%%%%%%%%%%%%%%%%%%%%%%%%%%%%%%%%%%%%%%%%%%%%%%%%%
\subsection{Command Line Processing}
\label{sec:commandline}

The effect of redirection files can also be achieved by invoking
the \LaTeX{} compiler with a more elaborate command line.
Most conveniently this should be done as part
of a shell script or a |Makefile|.

When using \textsf{childdoc} in the main file, the following
command lines effectively perform a redirection
(note that depending on the shell being used,
backslashes may have to be doubled: `|\|' $\to$ `|\\|'):
%
\begin{center}
|... -jobname "|\textit{target}|" |\\|"|[\textit{flags}]%
|\input{childdoc.def}\childdocforward[|\textit{main}|]{|\textit{dest}|}"|
\end{center}
%
Here \textit{target} is the name of the output file,
\textit{main} is the name of the main file
and \textit{dest} is the name of the main or child file to be processed
(all filenames without extensions).
The optional argument \textit{main} can be omitted
if \textit{main} matches \textit{dest}.
Optionally, compilation \textit{flags} can be defined via |\def| commands.
This command line makes the \TeX{} engine believe
it is compiling the file \textit{target}
whose content is specified as the latter parameter.
The provided code then forwards the processing to
\textit{main} or \textit{dest} as described in \secref{sec:forward}.

%%%%%%%%%%%%%%%%%%%%%%%%%%%%%%%%%%%%%%%%%%%%%%%%%%%%%%%%%%%%%%%%%%%%%%%%%%%%%%%%
\subsection{Include by Input}
\label{sec:input}

Including child documents by |\include| has some restrictions by design.
Most notably, the content of a child document always occupies
its own set of pages; pages cannot be shared between child documents.
Usually, this behaviour makes perfect sense
because each child document contain an essential part of the document.
However, in some situations it may be desirable to compose
a document from a collection of parts
without having mandatory page breaks between then.
For this case, the package
provides a mechanism to include parts
by |\input| which can also be processed individually.
However, by construction this mechanism
requires manual handling of the content to be output.

%%%%%%%%%%%%%%%%%%%%%%%%%%%%%%%%%%%%%%%%
\DescribeMacro{\ifchilddocmanual}
The main file should be prepared as usual, see \secref{sec:include}.
However, the document body must make a distinction
between processing of an individual part and of the main document, e.g.:
%
\begin{center}
\begin{tabular}{l}
|\ifchilddocmanual|\\
|\input{\childdocname}|\\
|\||else|\\
\textit{document body with }|\input{|\textit{part}|}|\\
|\||fi|
\end{tabular}
\end{center}
%
The conditional |\ifchilddocmanual| is true whenever
a part to be included by |\input| is being compiled,
and the name of the part is stored in |\childdocname|.

%%%%%%%%%%%%%%%%%%%%%%%%%%%%%%%%%%%%%%%%
\DescribeMacro{\childdocby}
Each part to be included by |\input| should start with:
%
\begin{center}
\begin{tabular}{l}
|\input{childdoc.def}|\\
|\childdocby{|\textit{main}|}|\\
\end{tabular}
\end{center}
%
The directive |\childdocby| is similar to |\childdocof|
described in \secref{sec:include},
but the subsequent selection of content must be done manually.
To that end, both |\ifchilddoc| and |\ifchilddocmanual|
will be true upon processing of a part,
and the name of the part is stored in |\childdocname|.
Note that |\jobname| will be set to the filename of the current part
so that each part receives an individual |.aux| file
that does not interfere with the |.aux| file(s) of the main document.
This behaviour can be altered by the alternative form
|\childdocby[*]{|\textit{main}|}| (with a non-empty optional argument)
which uses the |.aux| file of the main document
by setting |\jobname| to \textit{main}.

%%%%%%%%%%%%%%%%%%%%%%%%%%%%%%%%%%%%%%%%%%%%%%%%%%%%%%%%%%%%%%%%%%%%%%%%%%%%%%%%
\subsection{Driver Development}
\label{sec:driver}

The \textsf{childdoc} mechanism can also be use for the development
of definition files such as \LaTeX{} styles or classes.
This case differs from the above setup with multiple parts
included by |\include| in that no |\includeonly| should be invoked.
This can be achieved by starting the include file
(before |\ProvidesPackage|) with:
%
\begin{center}
\begin{tabular}{l}
|\input{childdoc.def}|\\
|\childdocforward{|\textit{main}|}|\\
\end{tabular}
\end{center}
%
or alternatively with:
%
\begin{center}
\begin{tabular}{l}
|\input{childdoc.def}|\\
|\childdocby{|\textit{main}|}|\\
\end{tabular}
\end{center}
%
Both forms have slightly different effects as described above.
The main file is prepared as usual, see \secref{sec:include}.

%%%%%%%%%%%%%%%%%%%%%%%%%%%%%%%%%%%%%%%%%%%%%%%%%%%%%%%%%%%%%%%%%%%%%%%%%%%%%%%%
\subsection{Legacy Detection}
\label{sec:detection}

The directive |\childdocmain| in the main file can detect
whether the complete document or merely a child is to be compiled
even without using the directive |\childdocof|.
This method is deprecated because it is less robust
and there is no compelling reason to use it;
it is merely provided for backward compatibility
and it may be removed in future versions.

If the detection mechanism is to be used,
it is mandatory to correctly specify
the filename of the main file as the argument of |\childdocmain|:
%
\begin{center}
\begin{tabular}{l}
|\input{childdoc.def}|\\
|\childdocmain{|\textit{main}|}|\\
\end{tabular}
\end{center}
%
If |\jobname| does not match the argument \textit{main} of |\childdocmain|,
it is assumed that |\jobname| points to the child file to be compiled.
When using |\childdocmain| with the main file specified as argument,
it suffices to start a child file
with just |\input{|\textit{main}|}|
without loading of the package and using |\childdocof|.
If instead all processing is done
with the appropriate \textsf{childdoc} directives,
the argument of \textit{main} of |\childdocmain| can be empty.

An alternative version of the command line processing described
in \secref{sec:commandline} using the detection mechanism reads:
%
\begin{center}
|... -jobname "|\textit{target}|" "|[\textit{flags}]%
[|\def\jobname{|\textit{dest}|}|]|\input{|\textit{main}|}"|
\end{center}

%%%%%%%%%%%%%%%%%%%%%%%%%%%%%%%%%%%%%%%%%%%%%%%%%%%%%%%%%%%%%%%%%%%%%%%%%%%%%%%%
\subsection{Manual Code}
\label{sec:manual}

In case one cannot be certain whether the definitions file |childdoc.def|
is installed on the target \TeX{} distribution
and one prefers not to ship it,
it is conceivable to paste a few relevant commands into the sources.

To that end, drop all statements |\input{childdoc.def}|
and perform the replacements as outlined below.
Instead of |\childdocmain{|\textit{main}|}| add the following code
to the top of the main file:
%
\begin{center}
\begin{tabular}{l}
|\||ifdefined\childdocname\endinput\||fi\newif\ifchilddoc|\\
|\edef\childdocname{\scantokens\expandafter{\jobname\noexpand}}|\\
|\def\childdocmain{|\textit{main}|}\||ifx\childdocmain\childdocname\||else|\\
|\childdoctrue\includeonly{\childdocname}\let\jobname\childdocmain\||fi|\\
\end{tabular}
\end{center}
%
Instead of |\childdocof{|\textit{main}|}| just include the main file
at the top of each child file:
%
\begin{center}
|\input{|\textit{main}|}|
\end{center}
%
A simple redirection |\childdocforward{|\textit{dest}|}| is achieved by:
%
\begin{center}
|\def\jobname{|\textit{dest}|}\input{\jobname}|
\end{center}
%
The redirection with prefix
|\childdocforwardprefix[|\textit{prefix}|]{|\textit{dest}|}|
is accomplished by:
%
\begin{center}
\begin{tabular}{l}
|{\edef\jobname{\scantokens\expandafter{\jobname\noexpand}}|\\
|\def\redirectjob |\textit{prefix}|#1~~~{\gdef\jobname{|\textit{dest}|#1}}|\\
|\expandafter\redirectjob\jobname~~~}\input{\jobname}|
\end{tabular}
\end{center}

In an alternative approach,
child documents can be compiled by a specific command line
without additional code or specific definitions:
%
\begin{center}
|... -jobname "|\textit{target}|" "|[\textit{flags}]%
|\includeonly{|\textit{dest}|}\input{|\textit{main}|}"|
\end{center}
%

%%%%%%%%%%%%%%%%%%%%%%%%%%%%%%%%%%%%%%%%%%%%%%%%%%%%%%%%%%%%%%%%%%%%%%%%%%%%%%%%
%%%%%%%%%%%%%%%%%%%%%%%%%%%%%%%%%%%%%%%%%%%%%%%%%%%%%%%%%%%%%%%%%%%%%%%%%%%%%%%%
\section{Information}

%%%%%%%%%%%%%%%%%%%%%%%%%%%%%%%%%%%%%%%%%%%%%%%%%%%%%%%%%%%%%%%%%%%%%%%%%%%%%%%%
\subsection{Copyright}

Copyright \copyright{} 2017--2018 Niklas Beisert

This work may be distributed and/or modified under the
conditions of the \LaTeX{} Project Public License, either version 1.3
of this license or (at your option) any later version.
The latest version of this license is in
  \url{http://www.latex-project.org/lppl.txt}
and version 1.3 or later is part of all distributions of \LaTeX{}
version 2005/12/01 or later.

This work has the LPPL maintenance status `maintained'.

The Current Maintainer of this work is Niklas Beisert.

This work consists of the files |README.txt|, |childdoc.ins| and |childdoc.dtx|
as well as the derived files |childdoc.def|, |cdocsamp.tex|
with |cdocsch1.tex|, |cdocsch2.tex|, |cdocspt3.tex|, |cdocspt4.tex|,
|cdocsdrf.tex|, |cdocsfn1.tex|, |cdocsfn2.tex|
as well as |childdoc.pdf|.

%%%%%%%%%%%%%%%%%%%%%%%%%%%%%%%%%%%%%%%%%%%%%%%%%%%%%%%%%%%%%%%%%%%%%%%%%%%%%%%%
\subsection{Files and Installation}

The package consists of the files:
%
\begin{center}
\begin{tabular}{ll}
    |README.txt|   & readme file \\
    |childdoc.ins| & installation file \\
    |childdoc.dtx| & source file \\
    |childdoc.def| & definition file \\
    |cdocsamp.tex| & sample main file \\
    |cdocsch1.tex| & sample include file \\
    |cdocsch2.tex| & sample include file \\
    |cdocspt3.tex| & sample part file \\
    |cdocspt4.tex| & sample part file \\
    |cdocsdrf.tex| & sample redirection file \\
    |cdocsfn1.tex| & sample redirection file \\
    |cdocsfn2.tex| & sample redirection file \\
    |childdoc.pdf| & manual
\end{tabular}
\end{center}
%
The distribution consists of the files
|README.txt|, |childdoc.ins| and |childdoc.dtx|.
%
\begin{itemize}
\item
Run (pdf)\LaTeX{} on |childdoc.dtx|
to compile the manual |childdoc.pdf| (this file).
\item
Run \LaTeX{} on |childdoc.ins| to create the definitions file |childdoc.def|
and the sample |cdocsamp.tex| with include files
|cdocsch1.tex|, |cdocsch2.tex|, |cdocspt3.tex|, |cdocspt4.tex|,
|cdocsdrf.tex|, |cdocsfn1.tex|, |cdocsfn2.tex|.
Then copy the file |childdoc.def| to an appropriate directory of your \LaTeX{}
distribution, e.g.\ \textit{texmf-root}|/tex/latex/childdoc|.
\end{itemize}

%%%%%%%%%%%%%%%%%%%%%%%%%%%%%%%%%%%%%%%%%%%%%%%%%%%%%%%%%%%%%%%%%%%%%%%%%%%%%%%%
\subsection{Related CTAN Packages}

There are several other packages which offer a similar functionality:
%
\begin{itemize}
\item
The packages
\href{http://ctan.org/pkg/docmute}{\textsf{docmute}},
\href{http://ctan.org/pkg/includex}{\textsf{includex}} and
\href{http://ctan.org/pkg/standalone}{\textsf{standalone}}
provide commands to include only the document body of
a child file thus allowing both files to be compiled individually.
\item
The packages \href{http://ctan.org/pkg/subdocs}{\textsf{subdocs}}
and \href{http://ctan.org/pkg/subfiles}{\textsf{subfiles}}
provide structures in which the main and child documents can be
encapsulated and allowing them to be compiled individually.
The inclusion mechanism is different from the conventional |\include|.
\item
The package \href{http://ctan.org/pkg/combine}{\textsf{combine}}
is an elaborate solution to combine several documents into one.
\end{itemize}
%
See also the CTAN topic \href{http://ctan.org/topic/subdocs}{\textsf{subdocs}}
for further related packages.
The present package differs from the above solutions in that
a document structure constructed with the conventional |\include| mechanism
just needs two extra commands at the top of every file
such that all constituent files can be compiled individually.

%%%%%%%%%%%%%%%%%%%%%%%%%%%%%%%%%%%%%%%%%%%%%%%%%%%%%%%%%%%%%%%%%%%%%%%%%%%%%%%%
%\subsection{Feature Suggestions}
%
%The following is a list of features which may be useful for future
%versions of this package:
%%
%\begin{itemize}
%\item
%\ldots
%\end{itemize}

%%%%%%%%%%%%%%%%%%%%%%%%%%%%%%%%%%%%%%%%%%%%%%%%%%%%%%%%%%%%%%%%%%%%%%%%%%%%%%%%
\subsection{Revision History}

%%%%%%%%%%%%%%%%%%%%%%%%%%%%%%%%%%%%%%%%
\paragraph{v2.0:} 2018/12/30

\begin{itemize}
\item
immediate forward processing
\item
added |\childdocby| mechanism
\item
manual restructured
\end{itemize}

%%%%%%%%%%%%%%%%%%%%%%%%%%%%%%%%%%%%%%%%
\paragraph{v1.6:} 2018/01/17

\begin{itemize}
\item
application for development of include files
\item
corrections to manual
\end{itemize}

%%%%%%%%%%%%%%%%%%%%%%%%%%%%%%%%%%%%%%%%
\paragraph{v1.5:} 2017/05/21

\begin{itemize}
\item
more complete structuring introduced
\item
|\childdocof| introduced
\item
|\childdoc| renamed to |\childdocmain|
\item
|\childredirect| renamed to |\childdocforward| and |\childdocforwardprefix|
and functionality expanded
\end{itemize}

%%%%%%%%%%%%%%%%%%%%%%%%%%%%%%%%%%%%%%%%
\paragraph{v1.0:} 2017/04/27

\begin{itemize}
\item
manual and install package
\item
first version published on CTAN
\end{itemize}

%%%%%%%%%%%%%%%%%%%%%%%%%%%%%%%%%%%%%%%%
\paragraph{v0.6:} 2017/04/26

\begin{itemize}
\item
redirection mechanism added
\end{itemize}

%%%%%%%%%%%%%%%%%%%%%%%%%%%%%%%%%%%%%%%%
\paragraph{v0.5:} 2017/04/26

\begin{itemize}
\item
functionality in definition file
\end{itemize}


%%%%%%%%%%%%%%%%%%%%%%%%%%%%%%%%%%%%%%%%%%%%%%%%%%%%%%%%%%%%%%%%%%%%%%%%%%%%%%%%
%%%%%%%%%%%%%%%%%%%%%%%%%%%%%%%%%%%%%%%%%%%%%%%%%%%%%%%%%%%%%%%%%%%%%%%%%%%%%%%%
%%%%%%%%%%%%%%%%%%%%%%%%%%%%%%%%%%%%%%%%%%%%%%%%%%%%%%%%%%%%%%%%%%%%%%%%%%%%%%%%
\appendix

\settowidth\MacroIndent{\rmfamily\scriptsize 000\ }

 \DocInput{childdoc.dtx}

\end{document}
%</driver>
% \fi
%
% %%%%%%%%%%%%%%%%%%%%%%%%%%%%%%%%%%%%%%%%%%%%%%%%%%%%%%%%%%%%%%%%%%%%%%%%%%%%%%
% %%%%%%%%%%%%%%%%%%%%%%%%%%%%%%%%%%%%%%%%%%%%%%%%%%%%%%%%%%%%%%%%%%%%%%%%%%%%%%
% \section{Sample}
%\iffalse
%<*samplemain>
%\fi
%
% The following presents a sample document
% with two chapters, two parts, a title page,
% a compile flag as well as three forwarding files to set the flag.
% It consists of eight |.tex| files:
% \begin{center}
% \begin{tabular}{ll}
% |cdocsamp.tex|&main file\\
% |cdocsch1.tex|&include file for chapter 1\\
% |cdocsch2.tex|&include file for chapter 2\\
% |cdocspt3.tex|&include file for part 3\\
% |cdocspt4.tex|&include file for part 4\\
% |cdocsdrf.tex|&forwarding file for main file in draft mode\\
% |cdocsfi1.tex|&forwarding file for final version of chapter 1\\
% |cdocsfi2.tex|&forwarding file for final version of chapter 2\\
% \end{tabular}
% \end{center}
% Each of the eight files can be compiled directly by the \LaTeX{} compiler.
%
% %%%%%%%%%%%%%%%%%%%%%%%%%%%%%%%%%%%%%%
% \paragraph{Main File.}
%
% The main file is called |cdocsamp.tex|.
%
% Load the \textsf{childdoc} definitions and
% declare the filename for the main document:
%    \begin{macrocode}
\input{childdoc.def}
\childdocmain{}
%    \end{macrocode}

% Optional override for |\version| flag:
%    \begin{macrocode}
%%\ifchilddoc\else\providecommand{\version}{draft}\fi
%    \end{macrocode}

% Define the default values for the |\version| flag
% (|final| for the main file and |draft| for childs):
%    \begin{macrocode}
\ifchilddoc
\providecommand{\version}{draft}
\else
\providecommand{\version}{final}
\fi
%    \end{macrocode}

% Load the standard document class:
%    \begin{macrocode}
\documentclass[12pt]{article}
%    \end{macrocode}

% Start the document body:
%    \begin{macrocode}
\begin{document}
%    \end{macrocode}

% Declare a title page.
% Print title, part of document being processed and version flag:
%    \begin{macrocode}
\addtocounter{page}{-1}
\begin{center}
{\LARGE\bfseries{}childdoc example\par}
\vspace{1cm}
\ifchilddoc
\ifchilddocmanual part\else chapter\fi:
`\childdocname' of `\childdocjob'\par
\else
main document: `\childdocjob'\par
\fi
version: \version\par
\end{center}
\newpage
%    \end{macrocode}

% Manually include selected file,
% otherwise process as usual:
%    \begin{macrocode}
\ifchilddocmanual
\section*{part `\childdocname'}
\input{\childdocname}
\else
%    \end{macrocode}

% Include the two chapters:
%    \begin{macrocode}
\include{cdocsch1}
\include{cdocsch2}
%    \end{macrocode}

% Include the two parts unless only chapters should be displayed:
%    \begin{macrocode}
\ifchilddoc\else
\section{part three}
\input{cdocspt3}
\section{part four}
\input{cdocspt4}
\fi
%    \end{macrocode}

% Process as usual until here:
%    \begin{macrocode}
\fi
%    \end{macrocode}

% End of document body:
%    \begin{macrocode}
\end{document}
%    \end{macrocode}
%\iffalse
%</samplemain>
%\fi
%
% %%%%%%%%%%%%%%%%%%%%%%%%%%%%%%%%%%%%%%
% \paragraph{Chapter Include Files.}
%
% The include files are called |cdocsch1.tex| and |cdocsch2.tex|.
%
%\iffalse
%<*samplechap1|samplechap2>
%\fi

% Optional override for |\version| flag:
%    \begin{macrocode}
%%\providecommand{\version}{final}
%    \end{macrocode}

% Include the main document:
%    \begin{macrocode}
\input{childdoc.def}
\childdocof{cdocsamp}
%    \end{macrocode}

%\iffalse
%</samplechap1|samplechap2>
%\fi
%
%\iffalse
%<*samplechap1>
%\fi
% Some text for chapter 1:
%    \begin{macrocode}
\section{one}
some text in chapter one
%    \end{macrocode}

%\iffalse
%</samplechap1>
%\fi
% Some text for chapter 2:
%\iffalse
%<*samplechap2>
%\fi
%    \begin{macrocode}
\section{two}
more text in chapter two
%    \end{macrocode}

%\iffalse
%</samplechap2>
%\fi
%
% %%%%%%%%%%%%%%%%%%%%%%%%%%%%%%%%%%%%%%
% \paragraph{Part Include Files.}
%
% The include files are called |cdocspt3.tex| and |cdocspt4.tex|.
%
%\iffalse
%<*samplepart3|samplepart4>
%\fi

% Optional override for |\version| flag:
%    \begin{macrocode}
%%\providecommand{\version}{final}
%    \end{macrocode}

% Include the main document:
%    \begin{macrocode}
\input{childdoc.def}
\childdocby{cdocsamp}
%    \end{macrocode}

%\iffalse
%</samplepart3|samplepart4>
%\fi
%
%\iffalse
%<*samplepart3>
%\fi
% Some text for part 3:
%    \begin{macrocode}
some text in part three
%    \end{macrocode}

%\iffalse
%</samplepart3>
%\fi
% Some text for part 4:
%\iffalse
%<*samplepart4>
%\fi
%    \begin{macrocode}
more text in part four
%    \end{macrocode}

%\iffalse
%</samplepart4>
%\fi
%
% %%%%%%%%%%%%%%%%%%%%%%%%%%%%%%%%%%%%%%
% \paragraph{Forwarding for a Complete Draft.}
%
% The following forwarding file |cdocsdrf.tex|
% compiles the main document in draft mode:
%\iffalse
%<*sampledraft>
%\fi
%    \begin{macrocode}
\def\version{draft}
\input{childdoc.def}
\childdocforward{cdocsamp}
%    \end{macrocode}

%\iffalse
%</sampledraft>
%\fi
%
% %%%%%%%%%%%%%%%%%%%%%%%%%%%%%%%%%%%%%%
% \paragraph{Forwarding for Final Version of the Chapters.}
%
% The following forwarding files |cdocsfn1.tex| and |cdocsfn2.tex|
% (with identical content)
% compile the final versions of the child documents
% |cdocsch1.tex| and |cdocsch2.tex|, respectively:
%\iffalse
%<*samplefinal>
%\fi
%    \begin{macrocode}
\def\version{final}
\input{childdoc.def}
\childdocforwardprefix[cdocsamp]{cdocsfn}{cdocsch}
%    \end{macrocode}

%\iffalse
%</samplefinal>
%\fi
%
% %%%%%%%%%%%%%%%%%%%%%%%%%%%%%%%%%%%%%%
% \paragraph{Command Line Processing.}
%
% The following three command lines generate the output files
% |cdocscld|, |cdocscl1| and |cdocscl2|
% which should be identical to
% |cdocsdrf|, |cdocsch1| and |cdocsfn2|, respectively:
% \begin{center}
% \begin{tabular}{l}
% |latex -jobname cdocscld \|\\
% |  "\def\version{draft}\input{childdoc.def}\childdocforward{cdocsamp}"|\\
% |latex -jobname cdocscl1 \|\\
% |  "\input{childdoc.def}\childdocforward[cdocsamp]{cdocsch1}"|\\
% |latex -jobname cdocscl2 \|\\
% |  "\def\version{final}\input{childdoc.def}\childdocforward{cdocsch2}"|
% \end{tabular}
% \end{center}
% Note that the trailing backslash on each first line
% merely continues the input to the second line
% (for convenient cut ant paste).
% Furthermore, the command |latex| can be replaced by any
% of its alternative versions such as |pdflatex|.
%
% %%%%%%%%%%%%%%%%%%%%%%%%%%%%%%%%%%%%%%%%%%%%%%%%%%%%%%%%%%%%%%%%%%%%%%%%%%%%%%
% %%%%%%%%%%%%%%%%%%%%%%%%%%%%%%%%%%%%%%%%%%%%%%%%%%%%%%%%%%%%%%%%%%%%%%%%%%%%%%
% \section{Implementation}
%\iffalse
%<*package>
%\fi
%
% This section describes the definitions file |childdoc.def|.

% The definitions cannot be loaded using |\usepackage| or |\RequirePackage|
% which has a mechanism to prevent loading a style file more than once.
% When loading the definitions by means of |\input|
% multiple instances have to be prevented manually:
%\iffalse
%This code needs to be before the `\ProvidesFile' directive
%which is defined at the beginning of this file.
%Therefore it is also placed there and commented out here.
%</package>
%<*discard>
%\fi
%    \begin{macrocode}
\ifdefined\childdocmain\endinput\fi
%    \end{macrocode}
%\iffalse
%</discard>
%<*package>
%\fi
%
% \macro{\ifchilddoc}
% \macro{\ifchilddocmanual}
% The conditional |\ifchilddoc| tells whether a
% child (true) or main (false) document is being compiled.
% The conditional |\ifchilddocmanual| tells whether
% the |\includeonly| mechanism is used (false) or
% the selection of child files must be performed manually (true).
% The definitions initialise to false:
%    \begin{macrocode}
\newif\ifchilddoc
\newif\ifchilddocmanual
%    \end{macrocode}

% \macro{\childdocname}
% \macro{\childdocjob}
% The macro |\childdocname| stores the name of the main document
% to be compiled. The macro |\childdocjob| stores the name of
% the document on which the \LaTeX{} compiler was originally invoked.
% The content of |\jobname| cannot be compared
% to filenames specified in the source due to different catcodes.
% The following code rescans |\jobname|, stores the result
% in |\childdocname| and saves a copy in |\childdocjob|:
%    \begin{macrocode}
\edef\childdocname{\scantokens\expandafter{\jobname\noexpand}}
\let\childdocjob\childdocname
%    \end{macrocode}

% \macro{\childdocdisable}
% The macro |\childdocdisable| prevents the main file
% from being processed more than once.
% At this stage, the main document command |\childdocmain|
% is assumed to be called once again where it should do nothing.
% Any subsequent call to it should prevent
% a secondary processing of the main document
% It overwrites the forwarding commands
% |\childdocof| and |\childdocforward|
% with empty macros to prevent further inclusions of the main document:
%    \begin{macrocode}
\newcommand{\childdocdisable}
{
  \renewcommand{\childdocmain}[1]{\renewcommand{\childdocmain}[1]{\endinput}}
  \renewcommand{\childdocof}[1]{}
  \renewcommand{\childdocby}[2][]{}
  \renewcommand{\childdocforward}[2][]{}
  \renewcommand{\childdocdisable}{}
}
%    \end{macrocode}

% \macro{\childdocmain}
% The macro |\childdocmain| is to be called at the top of the main file
% with nothing or the main filename (without extension) as argument.
% First, it breaks loops.
% If the argument is not empty and does not match |\childdocname|
% (which is set by the first inclusion of |childdoc.def|),
% |\ifchilddoc| is set to true, |\includeonly| is applied to the child file
% and |\jobname| is set to the main file
% (for proper handling of |.aux| files):
%    \begin{macrocode}
\newcommand{\childdocmain}[1]
{
  \childdocdisable\childdocmain{}
  \if?#1?\else
    \begingroup
      \def\childdoctmp{#1}
      \ifx\childdoctmp\childdocname
        \def\childdoctmp{}
      \else
        \def\childdoctmp
        {
          \childdoctrue
          \includeonly{\childdocname}
          \def\childdocjob{#1}
          \def\jobname{#1}
        }
      \fi
      \expandafter
    \endgroup
    \childdoctmp
  \fi
}
%    \end{macrocode}

% \macro{\childdocof}
% The command |\childdocof| redirects
% compilation to the main file |#1|.
%    \begin{macrocode}
\newcommand{\childdocof}[1]
{
  \childdocdisable
  \childdoctrue
  \includeonly{\childdocname}
  \def\jobname{#1}
  \def\childdocjob{#1}
  \input{#1}
}
%    \end{macrocode}

% \macro{\childdocby}
% The command |\childdocby| ....
%    \begin{macrocode}
\newcommand{\childdocby}[2][]
{
  \childdocdisable
  \childdoctrue
  \childdocmanualtrue
  \if?#1?\else
    \def\jobname{#2}
  \fi
  \def\childdocjob{#2}
  \input{#2}
  \endinput
}
%    \end{macrocode}

% \macro{\childdocforward}
% The command |\childdocforward| redirects
% compilation to the main file or
% (if the optional argument is given) a child file.
% Parameters are set as if the main file
% or a child file starting with |\childdocof| was compiled.
% Then compilation is handed over to the main file:
%    \begin{macrocode}
\newcommand{\childdocforward}[2][]
{
  \begingroup
    \if?#1?
      \def\childdoctmp
      {
        \def\childdocname{#2}
        \def\childdocjob{#2}
        \def\jobname{#2}
        \input{#2}
        \endinput
      }
    \else
      \def\childdoctmp
      {
        \childdocdisable
        \def\childdocname{#2}
        \childdoctrue
        \includeonly{#2}
        \def\childdocjob{#1}
        \def\jobname{#1}
        \input{#1}
        \endinput
      }
    \fi
    \expandafter
  \endgroup
  \childdoctmp
}
%    \end{macrocode}

% \macro{\childdocforwardprefix}
% The command |\childdocforwardprefix| redirects
% compilation to the main or a child file by means of a pattern.
% The prefix |#1| in the current filename is replaced by |#2|
% and the suffix of the current filename is kept
% (it is assumed that the filename does not contain the substring `|~~~|'
% which is used as a delimiter).
% Compilation is handed over to the new file by |\childdocforward|:
%    \begin{macrocode}
\newcommand{\childdocforwardprefix}[3][]
{
  \begingroup
    \def\childdocextract #2##1~~~{\def\childdoctmp{\childdocforward[#1]{#3##1}}}
    \expandafter\childdocextract\childdocname~~~
    \expandafter
  \endgroup
  \childdoctmp
}
%    \end{macrocode}

% \macro{\childdoc}
% The deprecated macro |\childdoc| is a legacy version of |\childdocmain|:
%    \begin{macrocode}
\newcommand{\childdoc}{\childdocmain}
%    \end{macrocode}

% \macro{\childdocredirect}
% The deprecated macro |\childdocredirect| is a legacy version
% of |\childdocforward| and |\childdocforwardprefix|:
%    \begin{macrocode}
\newcommand{\childdocredirect}[2][]
{
  \begingroup
    \if?#1?
      \def\childdoctmp{\childdocforward{#2}}
    \else
      \def\childdoctmp{\childdocforwardprefix{#1}{#2}}
    \fi
    \expandafter
  \endgroup
  \childdoctmp
}
%    \end{macrocode}

%\iffalse
%</package>
%\fi
%
\endinput
|\\
|\childdocmain{}|\\
\end{tabular}
\end{center}
at the very top of the main \LaTeX{} file,
in particular \emph{before} the |\documentclass| statement!
The argument of |\childdocmain| should be left empty
(but it must be present).

%%%%%%%%%%%%%%%%%%%%%%%%%%%%%%%%%%%%%%%%
\DescribeMacro{\childdocof}
Furthermore, add the commands
\begin{center}
\begin{tabular}{l}
|% \iffalse
%
% childdoc.dtx Copyright (C) 2017-2018 Niklas Beisert
%
% This work may be distributed and/or modified under the
% conditions of the LaTeX Project Public License, either version 1.3
% of this license or (at your option) any later version.
% The latest version of this license is in
%   http://www.latex-project.org/lppl.txt
% and version 1.3 or later is part of all distributions of LaTeX
% version 2005/12/01 or later.
%
% This work has the LPPL maintenance status `maintained'.
%
% The Current Maintainer of this work is Niklas Beisert.
%
% This work consists of the files childdoc.dtx and childdoc.ins
% and the derived files childdoc.def and cdocsamp.tex with
% cdocsch1.tex, cdocsch2.tex, cdocsdrf.tex, cdocsfn1.tex, cdocsfn2.tex.
%
%<package>\ifdefined\childdocmain\endinput\fi
%<package>\ProvidesFile{childdoc.def}[2018/12/30 v2.0 child document driver]
%<samplemain>\ProvidesFile{cdocsamp.tex}[2018/12/30 v2.0 sample for childdoc]
%<*driver>
%\ProvidesFile{childdoc.drv}[2018/12/30 v2.0 childdoc reference manual file]
\PassOptionsToClass{10pt,a4paper}{article}
\documentclass{ltxdoc}

\usepackage[margin=35mm]{geometry}
\usepackage{hyperref}
\usepackage{hyperxmp}
\usepackage[usenames]{color}

\hypersetup{colorlinks=true}
\hypersetup{pdfstartview=FitH}
\hypersetup{pdfpagemode=UseNone}
\hypersetup{pdfsource={}}
\hypersetup{pdflang={en-UK}}
\hypersetup{pdfcopyright={Copyright 2017-2018 Niklas Beisert.
  This work may be distributed and/or modified under the
  conditions of the LaTeX Project Public License, either version 1.3
  of this license or (at your option) any later version.}}
\hypersetup{pdflicenseurl={http://www.latex-project.org/lppl.txt}}
\hypersetup{pdfcontactaddress={ETH Zurich, ITP, HIT K,
  Wolfgang-Pauli-Strasse 27}}
\hypersetup{pdfcontactpostcode={8093}}
\hypersetup{pdfcontactcity={Zurich}}
\hypersetup{pdfcontactcountry={Switzerland}}
\hypersetup{pdfcontactemail={nbeisert@itp.phys.ethz.ch}}
\hypersetup{pdfcontacturl={http://people.phys.ethz.ch/\xmptilde nbeisert/}}

\newcommand{\secref}[1]{\hyperref[#1]{section \ref*{#1}}}

\parskip1ex
\parindent0pt
\let\olditemize\itemize
\def\itemize{\olditemize\parskip0pt}

\begin{document}

\title{The \textsf{childdoc} Package}
\hypersetup{pdftitle={The childdoc Package}}
\author{Niklas Beisert\\[2ex]
  Institut f\"ur Theoretische Physik\\
  Eidgen\"ossische Technische Hochschule Z\"urich\\
  Wolfgang-Pauli-Strasse 27, 8093 Z\"urich, Switzerland\\[1ex]
  \href{mailto:nbeisert@itp.phys.ethz.ch}
  {\texttt{nbeisert@itp.phys.ethz.ch}}}
\hypersetup{pdfauthor={Niklas Beisert}}
\hypersetup{pdfsubject={Manual for the LaTeX2e Package childdoc}}
\date{30 December 2018, \textsf{v2.0}}
\maketitle

\begin{abstract}\noindent
\textsf{childdoc} is a \LaTeXe{} package
that enables the direct compilation
of document sections included by |\include|
to individual files.
\end{abstract}

\begingroup
\parskip0ex
\tableofcontents
\endgroup

%%%%%%%%%%%%%%%%%%%%%%%%%%%%%%%%%%%%%%%%%%%%%%%%%%%%%%%%%%%%%%%%%%%%%%%%%%%%%%%%
%%%%%%%%%%%%%%%%%%%%%%%%%%%%%%%%%%%%%%%%%%%%%%%%%%%%%%%%%%%%%%%%%%%%%%%%%%%%%%%%
\section{Introduction}

\LaTeX{} provides a mechanism to structure a large document (such as a book)
into a main file and several child files (containing the chapters)
using the |\include| command.
This mechanism is beneficial for documents
which span hundreds of pages in order to
make the source file(s) more manageable.
Moreover, compilation can be restricted to
selected child files by means of the |\includeonly| command.
The latter feature can be used to reduce the compilation time while editing
(this was significantly more useful in the earlier days of \LaTeX{})
or to generate a smaller document which is easier to navigate.
Another application of |\includeonly| is to generate
documents consisting of selected parts of the complete document.

However, there are a few drawbacks of the plain |\include| mechanism:
\begin{itemize}
\item
The child files cannot be compiled on their own,
they can only be compiled via the main file.
A naive editing environment
(such as a text editor with an option
to have the current file processed by \LaTeX)
may require one to switch to the main file before compiling;
attempting to compile the child file produces errors.
\item
The main file must be modified (each time)
to adjust the |\includeonly| command
to the present needs. This easily leaves the main file in a messy state.
\item
The generated document will always carry the filename
of the main document. This is inconvenient if
several child files are to be compiled and
to be kept for distribution.
\end{itemize}

The present package provides a simple interface
to make child files individually compilable by \LaTeX{}.
Compiling a child file then has the same effect as compiling
the main file with an |\includeonly| command
to select the appropriate child.
Moreover the generated document will carry the name of the child
rather than the main file.
This resolves all three above issues.

This feature is meant to make the editing of books,
thesis documents and lecture notes somewhat more convenient.
However, the package can also be used efficiently for
composing a series of documents (such as exercise sheets)
which are typically distributed individually.
It then assists the author in generating the individual documents
(potentially in different versions)
as well as a document containing the collected series.
Another application is in developing style files
or other kinds of included material
where compilation of the style file could redirect
to a sample or test file.

%%%%%%%%%%%%%%%%%%%%%%%%%%%%%%%%%%%%%%%%%%%%%%%%%%%%%%%%%%%%%%%%%%%%%%%%%%%%%%%%
%%%%%%%%%%%%%%%%%%%%%%%%%%%%%%%%%%%%%%%%%%%%%%%%%%%%%%%%%%%%%%%%%%%%%%%%%%%%%%%%
\section{Usage}

First of all, the package \textsf{childdoc} is \emph{not} a standard
\LaTeXe{} |.sty| style file! Therefore it needs to be invoked in
a non-standard way.

%%%%%%%%%%%%%%%%%%%%%%%%%%%%%%%%%%%%%%%%%%%%%%%%%%%%%%%%%%%%%%%%%%%%%%%%%%%%%%%%
\subsection{Included Files}
\label{sec:include}

%%%%%%%%%%%%%%%%%%%%%%%%%%%%%%%%%%%%%%%%
\DescribeMacro{\childdocmain}
To use the package, add the commands
\begin{center}
\begin{tabular}{l}
|\input{childdoc.def}|\\
|\childdocmain{}|\\
\end{tabular}
\end{center}
at the very top of the main \LaTeX{} file,
in particular \emph{before} the |\documentclass| statement!
The argument of |\childdocmain| should be left empty
(but it must be present).

%%%%%%%%%%%%%%%%%%%%%%%%%%%%%%%%%%%%%%%%
\DescribeMacro{\childdocof}
Furthermore, add the commands
\begin{center}
\begin{tabular}{l}
|\input{childdoc.def}|\\
|\childdocof{|\textit{main}|}|\\
\end{tabular}
\end{center}
at the top of every child file \textit{child}
which is included by |\include{|\textit{child}|}|
from within the main file
(or at least for those files to be compiled individually).
The argument \textit{main} must be the filename of the main file.

There are a couple of
considerations in setting up the main and child documents:

%%%%%%%%%%%%%%%%%%%%%%%%%%%%%%%%%%%%%%%%
\paragraph{Restrictions.}

Please note the following restrictions:
\begin{itemize}
\item
|\childdocmain| must be called with one argument \textit{main}
to ensure compatibility with earlier version of the package.
It must either be empty (|\childdocmain{}|)
or precisely match the filename of the main file in which it is specified.
See \secref{sec:detection} for further information.
\item
The filename \textit{main} must be specified without the |.tex| extension.
\item
The filename \textit{main} is case sensitive
(even in case-insensitive file systems)
due to internal string comparison.
\item
The argument \textit{main} should be fully expanded, it cannot be a macro.
\item
Subdirectories and special characters should be avoided in filenames.
\item
The command |\childdocmain{|\textit{main}|}| must be followed by a whitespace.
It should not be followed immediately by another command
or by a comment mark `|%|'.
This is because the \TeX{} parser reads the token immediately following
the argument of |\childdocmain| and puts it
at the beginning of every child section;
however, a white\-space is ignored.
\end{itemize}

%%%%%%%%%%%%%%%%%%%%%%%%%%%%%%%%%%%%%%%%
\paragraph{Content of Main File.}

It is advisable to place all content in the child files included by |\include|.
Any output contained in the main file will appear in all child documents
unless suppressed manually;
it cannot be suppressed automatically by the |\includeonly| directive
and thus should normally be avoided.
A method to include some content in the main file
by means of conditional processing is described in \secref{sec:conditional}.

%%%%%%%%%%%%%%%%%%%%%%%%%%%%%%%%%%%%%%%%
\paragraph{Page Numbering.}

When only a part of the document is compiled,
the appropriate numbering of pages
(as well as other status parameters)
is determined from the |.aux| files.
The latter contain information from previous passes.
However this information needs to propagate through
all intermediate child documents.
Therefore the page numbering in child documents may well
be inconsistent until the complete document is compiled at least once.

A useful (if unconventional) way to always ensure a consistent
page numbering is to restart the numbering in each child document
and denote the pages by `\textit{child}|.|\textit{page}'
where \textit{child} represents the chapter/section number of the child file.
This can be achieved by the command
|\numberwithin{page}{|\textit{child}|}|
of the \textsf{amsmath} package
where \textit{child} can be |chapter| or |section|
depending on the chosen structuring.
Alternatively, one can modify the macro |\thepage| appropriately
and reset the counter |page| at the start of each child file.

%%%%%%%%%%%%%%%%%%%%%%%%%%%%%%%%%%%%%%%%%%%%%%%%%%%%%%%%%%%%%%%%%%%%%%%%%%%%%%%%
\subsection{Conditional Processing}
\label{sec:conditional}

The package provides a mechanism to compile different versions
of a document. To customise the versions further some conditional processing
can come in handy to distinguish which version is being compiled.
The package provides two macros to describe the compilation context:

%%%%%%%%%%%%%%%%%%%%%%%%%%%%%%%%%%%%%%%%
\DescribeMacro{\ifchilddoc}
The conditional |\ifchilddoc| distinguishes between the compilation of
child documents and the main document:
%
\begin{center}
|\ifchilddoc |\textit{child-code}| |[|\||else |\textit{main-code}]| \||fi|
\end{center}

%%%%%%%%%%%%%%%%%%%%%%%%%%%%%%%%%%%%%%%%
\DescribeMacro{\childdocname}
\DescribeMacro{\childdocjob}
The macro |\childdocname| contains the filename (without extension)
of the main or child file being processed.
Note that |\childdocjob| will always contain the name of the main file.

%%%%%%%%%%%%%%%%%%%%%%%%%%%%%%%%%%%%%%%%
\paragraph{Title Page.}

Conditional processing can be used to include a title or banner page
in the main document when proper precautions are taken.
Importantly, the code in the main file should ensure that the page counter
(as well as other status parameters which are stored in the |.aux| files)
takes the same value after the conditional processing.
Otherwise the page numbers may take divergent values
depending on which part is compiled.

For example, a title page could be declared by:
%
\begin{center}
\begin{tabular}{l}
|\ifchilddoc\||else|\\
|\addtocounter{page}{-1}|\\
\textit{code for title page}\\
|\newpage|\\
|\||fi|
\end{tabular}
\end{center}
%
A banner page for the child documents can be generated by:
%
\begin{center}
\begin{tabular}{l}
|\ifchilddoc|\\
|\addtocounter{page}{-1}|\\
\textit{code for banner page}\\
|\newpage|\\
|\||fi|
\end{tabular}
\end{center}
%
Here one could write a message such as:
\begin{center}
|This is the part \childdocname{} of \childdocjob{}.|
\end{center}

%%%%%%%%%%%%%%%%%%%%%%%%%%%%%%%%%%%%%%%%%%%%%%%%%%%%%%%%%%%%%%%%%%%%%%%%%%%%%%%%
\subsection{Flags}
\label{sec:flags}

The package makes it easy to generate different versions
of the main or child documents.
To this end compilation flags can be defined
and assigned different default values.
They will be particularly useful in conjunction
with the forwarding mechanism described in \secref{sec:forward}.

For example, it may be useful to have a flag |\version|
which can be set to |draft| or |final|.
The document source will contain some conditional code
depending on the value of |\version|.
Suppose further, the flag should default to |final| for the main file
and to |draft| for child files
which is a natural assignment for editing the document.
This is achieved by placing the following code
in the preamble of the main document
(below the |\childdocmain| directive):
%
\begin{center}
\begin{tabular}{l}
|\ifchilddoc|\\
|\providecommand{\version}{draft}|\\
|\||else|\\
|\providecommand{\version}{final}|\\
|\||fi|
\end{tabular}
\end{center}
%
The definition by |\providecommand| makes sure
that previous definitions are not overwritten.
Further statements |\providecommand{\version}{...}|
can thus be added before the above code to override it.

For the main file, one might add a line
(between |\childdocmain| and the above block)
%
\begin{center}
|%\ifchilddoc\||else\providecommand{\version}{draft}\||fi|
\end{center}
%
which can be uncommented to produce a draft version.
Likewise one can add a line to the very top of a child file
(above the |\childdocof{|\textit{main}|}| directive)
%
\begin{center}
|%\providecommand{\version}{final}|
\end{center}
%
which can be uncommented to produce the final version of this child document.

%%%%%%%%%%%%%%%%%%%%%%%%%%%%%%%%%%%%%%%%%%%%%%%%%%%%%%%%%%%%%%%%%%%%%%%%%%%%%%%%
\subsection{Forwarding}
\label{sec:forward}

Different versions of the main or child documents
using compilation flags as described in \secref{sec:flags}
can be (permanently) stored in different files
for convenient compilation, viewing and distribution.
To this end, the package defines a command
to pass on compilation to a different file:

%%%%%%%%%%%%%%%%%%%%%%%%%%%%%%%%%%%%%%%%
\DescribeMacro{\childdocforward}
The command |\childdocforward| redirects processing to
another source file:
%
\begin{center}
\begin{tabular}{l}
|\input{childdoc.def}|\\
|\childdocforward[|\textit{main}|]{|\textit{dest}|}|\\
\end{tabular}
\end{center}
%
The argument \textit{dest} is the destination file
(without extension).
It should be the main file or one of the child files.
Note that further \textsf{childdoc} directives
such as |\childdocof| and |\childdocforward|
in the indicated file will be processed in this form.
The optional argument \textit{main}
passes on directly to the main file \textit{main}
while pretending to compile the child \textit{dest}.
This form behaves as if \textit{dest}
issues |\childdocof{|\textit{main}|}| right away,
and no further \textsf{childdoc} directives will be processed.

%%%%%%%%%%%%%%%%%%%%%%%%%%%%%%%%%%%%%%%%
\DescribeMacro{\...prefix}
In the alternative form |\childdocforwardprefix|,
%
\begin{center}
\begin{tabular}{l}
|\input{childdoc.def}|\\
|\childdocforwardprefix[|\textit{main}|]{|\textit{prefix}|}{|\textit{dest}|}|
\end{tabular}
\end{center}
%
the destination file is determined by a pattern
depending on the current file:
To make this work, the current file must be called
`{\textit{prefix}\hspace{0.2em}\textit{suffix}}'
with \textit{prefix} matching precisely the argument.
Processing is then passed on to the file
`{\textit{dest}\hspace{0.2em}\textit{suffix}}'.
Surely, the same effect is achieved by
directly specifying the
argument `{\textit{dest}\hspace{0.2em}\textit{suffix}}'
in the first form.
However, that requires to set up a different file
for each child. With the alternative form of the command
all these files can have exactly the same content
which simplifies setting them up and maintaining them.

For example, the following file |draft.tex|
with a compilation flag |\version| as described in \secref{sec:flags}
compiles the main document as a draft:
%
\begin{center}
\begin{tabular}{l}
|\def\version{draft}|\\
|\input{childdoc.def}|\\
|\childdocforward{|\textit{main}|}|
\end{tabular}
\end{center}
%
Likewise, the following files |final|\textit{nn}|.tex|
compile the final version of the child document
|child|\textit{nn}|.tex|:
%
\begin{center}
\begin{tabular}{l}
|\def\version{final}|\\
|\input{childdoc.def}|\\
|\childdocforwardprefix{final}{child}|
\end{tabular}
\end{center}
%

Note that when several versions of a main file and/or of each child file
are to be generated, it may be convenient to set up a |Makefile| or
shell script to automatise the process.

%%%%%%%%%%%%%%%%%%%%%%%%%%%%%%%%%%%%%%%%%%%%%%%%%%%%%%%%%%%%%%%%%%%%%%%%%%%%%%%%
\subsection{Command Line Processing}
\label{sec:commandline}

The effect of redirection files can also be achieved by invoking
the \LaTeX{} compiler with a more elaborate command line.
Most conveniently this should be done as part
of a shell script or a |Makefile|.

When using \textsf{childdoc} in the main file, the following
command lines effectively perform a redirection
(note that depending on the shell being used,
backslashes may have to be doubled: `|\|' $\to$ `|\\|'):
%
\begin{center}
|... -jobname "|\textit{target}|" |\\|"|[\textit{flags}]%
|\input{childdoc.def}\childdocforward[|\textit{main}|]{|\textit{dest}|}"|
\end{center}
%
Here \textit{target} is the name of the output file,
\textit{main} is the name of the main file
and \textit{dest} is the name of the main or child file to be processed
(all filenames without extensions).
The optional argument \textit{main} can be omitted
if \textit{main} matches \textit{dest}.
Optionally, compilation \textit{flags} can be defined via |\def| commands.
This command line makes the \TeX{} engine believe
it is compiling the file \textit{target}
whose content is specified as the latter parameter.
The provided code then forwards the processing to
\textit{main} or \textit{dest} as described in \secref{sec:forward}.

%%%%%%%%%%%%%%%%%%%%%%%%%%%%%%%%%%%%%%%%%%%%%%%%%%%%%%%%%%%%%%%%%%%%%%%%%%%%%%%%
\subsection{Include by Input}
\label{sec:input}

Including child documents by |\include| has some restrictions by design.
Most notably, the content of a child document always occupies
its own set of pages; pages cannot be shared between child documents.
Usually, this behaviour makes perfect sense
because each child document contain an essential part of the document.
However, in some situations it may be desirable to compose
a document from a collection of parts
without having mandatory page breaks between then.
For this case, the package
provides a mechanism to include parts
by |\input| which can also be processed individually.
However, by construction this mechanism
requires manual handling of the content to be output.

%%%%%%%%%%%%%%%%%%%%%%%%%%%%%%%%%%%%%%%%
\DescribeMacro{\ifchilddocmanual}
The main file should be prepared as usual, see \secref{sec:include}.
However, the document body must make a distinction
between processing of an individual part and of the main document, e.g.:
%
\begin{center}
\begin{tabular}{l}
|\ifchilddocmanual|\\
|\input{\childdocname}|\\
|\||else|\\
\textit{document body with }|\input{|\textit{part}|}|\\
|\||fi|
\end{tabular}
\end{center}
%
The conditional |\ifchilddocmanual| is true whenever
a part to be included by |\input| is being compiled,
and the name of the part is stored in |\childdocname|.

%%%%%%%%%%%%%%%%%%%%%%%%%%%%%%%%%%%%%%%%
\DescribeMacro{\childdocby}
Each part to be included by |\input| should start with:
%
\begin{center}
\begin{tabular}{l}
|\input{childdoc.def}|\\
|\childdocby{|\textit{main}|}|\\
\end{tabular}
\end{center}
%
The directive |\childdocby| is similar to |\childdocof|
described in \secref{sec:include},
but the subsequent selection of content must be done manually.
To that end, both |\ifchilddoc| and |\ifchilddocmanual|
will be true upon processing of a part,
and the name of the part is stored in |\childdocname|.
Note that |\jobname| will be set to the filename of the current part
so that each part receives an individual |.aux| file
that does not interfere with the |.aux| file(s) of the main document.
This behaviour can be altered by the alternative form
|\childdocby[*]{|\textit{main}|}| (with a non-empty optional argument)
which uses the |.aux| file of the main document
by setting |\jobname| to \textit{main}.

%%%%%%%%%%%%%%%%%%%%%%%%%%%%%%%%%%%%%%%%%%%%%%%%%%%%%%%%%%%%%%%%%%%%%%%%%%%%%%%%
\subsection{Driver Development}
\label{sec:driver}

The \textsf{childdoc} mechanism can also be use for the development
of definition files such as \LaTeX{} styles or classes.
This case differs from the above setup with multiple parts
included by |\include| in that no |\includeonly| should be invoked.
This can be achieved by starting the include file
(before |\ProvidesPackage|) with:
%
\begin{center}
\begin{tabular}{l}
|\input{childdoc.def}|\\
|\childdocforward{|\textit{main}|}|\\
\end{tabular}
\end{center}
%
or alternatively with:
%
\begin{center}
\begin{tabular}{l}
|\input{childdoc.def}|\\
|\childdocby{|\textit{main}|}|\\
\end{tabular}
\end{center}
%
Both forms have slightly different effects as described above.
The main file is prepared as usual, see \secref{sec:include}.

%%%%%%%%%%%%%%%%%%%%%%%%%%%%%%%%%%%%%%%%%%%%%%%%%%%%%%%%%%%%%%%%%%%%%%%%%%%%%%%%
\subsection{Legacy Detection}
\label{sec:detection}

The directive |\childdocmain| in the main file can detect
whether the complete document or merely a child is to be compiled
even without using the directive |\childdocof|.
This method is deprecated because it is less robust
and there is no compelling reason to use it;
it is merely provided for backward compatibility
and it may be removed in future versions.

If the detection mechanism is to be used,
it is mandatory to correctly specify
the filename of the main file as the argument of |\childdocmain|:
%
\begin{center}
\begin{tabular}{l}
|\input{childdoc.def}|\\
|\childdocmain{|\textit{main}|}|\\
\end{tabular}
\end{center}
%
If |\jobname| does not match the argument \textit{main} of |\childdocmain|,
it is assumed that |\jobname| points to the child file to be compiled.
When using |\childdocmain| with the main file specified as argument,
it suffices to start a child file
with just |\input{|\textit{main}|}|
without loading of the package and using |\childdocof|.
If instead all processing is done
with the appropriate \textsf{childdoc} directives,
the argument of \textit{main} of |\childdocmain| can be empty.

An alternative version of the command line processing described
in \secref{sec:commandline} using the detection mechanism reads:
%
\begin{center}
|... -jobname "|\textit{target}|" "|[\textit{flags}]%
[|\def\jobname{|\textit{dest}|}|]|\input{|\textit{main}|}"|
\end{center}

%%%%%%%%%%%%%%%%%%%%%%%%%%%%%%%%%%%%%%%%%%%%%%%%%%%%%%%%%%%%%%%%%%%%%%%%%%%%%%%%
\subsection{Manual Code}
\label{sec:manual}

In case one cannot be certain whether the definitions file |childdoc.def|
is installed on the target \TeX{} distribution
and one prefers not to ship it,
it is conceivable to paste a few relevant commands into the sources.

To that end, drop all statements |\input{childdoc.def}|
and perform the replacements as outlined below.
Instead of |\childdocmain{|\textit{main}|}| add the following code
to the top of the main file:
%
\begin{center}
\begin{tabular}{l}
|\||ifdefined\childdocname\endinput\||fi\newif\ifchilddoc|\\
|\edef\childdocname{\scantokens\expandafter{\jobname\noexpand}}|\\
|\def\childdocmain{|\textit{main}|}\||ifx\childdocmain\childdocname\||else|\\
|\childdoctrue\includeonly{\childdocname}\let\jobname\childdocmain\||fi|\\
\end{tabular}
\end{center}
%
Instead of |\childdocof{|\textit{main}|}| just include the main file
at the top of each child file:
%
\begin{center}
|\input{|\textit{main}|}|
\end{center}
%
A simple redirection |\childdocforward{|\textit{dest}|}| is achieved by:
%
\begin{center}
|\def\jobname{|\textit{dest}|}\input{\jobname}|
\end{center}
%
The redirection with prefix
|\childdocforwardprefix[|\textit{prefix}|]{|\textit{dest}|}|
is accomplished by:
%
\begin{center}
\begin{tabular}{l}
|{\edef\jobname{\scantokens\expandafter{\jobname\noexpand}}|\\
|\def\redirectjob |\textit{prefix}|#1~~~{\gdef\jobname{|\textit{dest}|#1}}|\\
|\expandafter\redirectjob\jobname~~~}\input{\jobname}|
\end{tabular}
\end{center}

In an alternative approach,
child documents can be compiled by a specific command line
without additional code or specific definitions:
%
\begin{center}
|... -jobname "|\textit{target}|" "|[\textit{flags}]%
|\includeonly{|\textit{dest}|}\input{|\textit{main}|}"|
\end{center}
%

%%%%%%%%%%%%%%%%%%%%%%%%%%%%%%%%%%%%%%%%%%%%%%%%%%%%%%%%%%%%%%%%%%%%%%%%%%%%%%%%
%%%%%%%%%%%%%%%%%%%%%%%%%%%%%%%%%%%%%%%%%%%%%%%%%%%%%%%%%%%%%%%%%%%%%%%%%%%%%%%%
\section{Information}

%%%%%%%%%%%%%%%%%%%%%%%%%%%%%%%%%%%%%%%%%%%%%%%%%%%%%%%%%%%%%%%%%%%%%%%%%%%%%%%%
\subsection{Copyright}

Copyright \copyright{} 2017--2018 Niklas Beisert

This work may be distributed and/or modified under the
conditions of the \LaTeX{} Project Public License, either version 1.3
of this license or (at your option) any later version.
The latest version of this license is in
  \url{http://www.latex-project.org/lppl.txt}
and version 1.3 or later is part of all distributions of \LaTeX{}
version 2005/12/01 or later.

This work has the LPPL maintenance status `maintained'.

The Current Maintainer of this work is Niklas Beisert.

This work consists of the files |README.txt|, |childdoc.ins| and |childdoc.dtx|
as well as the derived files |childdoc.def|, |cdocsamp.tex|
with |cdocsch1.tex|, |cdocsch2.tex|, |cdocspt3.tex|, |cdocspt4.tex|,
|cdocsdrf.tex|, |cdocsfn1.tex|, |cdocsfn2.tex|
as well as |childdoc.pdf|.

%%%%%%%%%%%%%%%%%%%%%%%%%%%%%%%%%%%%%%%%%%%%%%%%%%%%%%%%%%%%%%%%%%%%%%%%%%%%%%%%
\subsection{Files and Installation}

The package consists of the files:
%
\begin{center}
\begin{tabular}{ll}
    |README.txt|   & readme file \\
    |childdoc.ins| & installation file \\
    |childdoc.dtx| & source file \\
    |childdoc.def| & definition file \\
    |cdocsamp.tex| & sample main file \\
    |cdocsch1.tex| & sample include file \\
    |cdocsch2.tex| & sample include file \\
    |cdocspt3.tex| & sample part file \\
    |cdocspt4.tex| & sample part file \\
    |cdocsdrf.tex| & sample redirection file \\
    |cdocsfn1.tex| & sample redirection file \\
    |cdocsfn2.tex| & sample redirection file \\
    |childdoc.pdf| & manual
\end{tabular}
\end{center}
%
The distribution consists of the files
|README.txt|, |childdoc.ins| and |childdoc.dtx|.
%
\begin{itemize}
\item
Run (pdf)\LaTeX{} on |childdoc.dtx|
to compile the manual |childdoc.pdf| (this file).
\item
Run \LaTeX{} on |childdoc.ins| to create the definitions file |childdoc.def|
and the sample |cdocsamp.tex| with include files
|cdocsch1.tex|, |cdocsch2.tex|, |cdocspt3.tex|, |cdocspt4.tex|,
|cdocsdrf.tex|, |cdocsfn1.tex|, |cdocsfn2.tex|.
Then copy the file |childdoc.def| to an appropriate directory of your \LaTeX{}
distribution, e.g.\ \textit{texmf-root}|/tex/latex/childdoc|.
\end{itemize}

%%%%%%%%%%%%%%%%%%%%%%%%%%%%%%%%%%%%%%%%%%%%%%%%%%%%%%%%%%%%%%%%%%%%%%%%%%%%%%%%
\subsection{Related CTAN Packages}

There are several other packages which offer a similar functionality:
%
\begin{itemize}
\item
The packages
\href{http://ctan.org/pkg/docmute}{\textsf{docmute}},
\href{http://ctan.org/pkg/includex}{\textsf{includex}} and
\href{http://ctan.org/pkg/standalone}{\textsf{standalone}}
provide commands to include only the document body of
a child file thus allowing both files to be compiled individually.
\item
The packages \href{http://ctan.org/pkg/subdocs}{\textsf{subdocs}}
and \href{http://ctan.org/pkg/subfiles}{\textsf{subfiles}}
provide structures in which the main and child documents can be
encapsulated and allowing them to be compiled individually.
The inclusion mechanism is different from the conventional |\include|.
\item
The package \href{http://ctan.org/pkg/combine}{\textsf{combine}}
is an elaborate solution to combine several documents into one.
\end{itemize}
%
See also the CTAN topic \href{http://ctan.org/topic/subdocs}{\textsf{subdocs}}
for further related packages.
The present package differs from the above solutions in that
a document structure constructed with the conventional |\include| mechanism
just needs two extra commands at the top of every file
such that all constituent files can be compiled individually.

%%%%%%%%%%%%%%%%%%%%%%%%%%%%%%%%%%%%%%%%%%%%%%%%%%%%%%%%%%%%%%%%%%%%%%%%%%%%%%%%
%\subsection{Feature Suggestions}
%
%The following is a list of features which may be useful for future
%versions of this package:
%%
%\begin{itemize}
%\item
%\ldots
%\end{itemize}

%%%%%%%%%%%%%%%%%%%%%%%%%%%%%%%%%%%%%%%%%%%%%%%%%%%%%%%%%%%%%%%%%%%%%%%%%%%%%%%%
\subsection{Revision History}

%%%%%%%%%%%%%%%%%%%%%%%%%%%%%%%%%%%%%%%%
\paragraph{v2.0:} 2018/12/30

\begin{itemize}
\item
immediate forward processing
\item
added |\childdocby| mechanism
\item
manual restructured
\end{itemize}

%%%%%%%%%%%%%%%%%%%%%%%%%%%%%%%%%%%%%%%%
\paragraph{v1.6:} 2018/01/17

\begin{itemize}
\item
application for development of include files
\item
corrections to manual
\end{itemize}

%%%%%%%%%%%%%%%%%%%%%%%%%%%%%%%%%%%%%%%%
\paragraph{v1.5:} 2017/05/21

\begin{itemize}
\item
more complete structuring introduced
\item
|\childdocof| introduced
\item
|\childdoc| renamed to |\childdocmain|
\item
|\childredirect| renamed to |\childdocforward| and |\childdocforwardprefix|
and functionality expanded
\end{itemize}

%%%%%%%%%%%%%%%%%%%%%%%%%%%%%%%%%%%%%%%%
\paragraph{v1.0:} 2017/04/27

\begin{itemize}
\item
manual and install package
\item
first version published on CTAN
\end{itemize}

%%%%%%%%%%%%%%%%%%%%%%%%%%%%%%%%%%%%%%%%
\paragraph{v0.6:} 2017/04/26

\begin{itemize}
\item
redirection mechanism added
\end{itemize}

%%%%%%%%%%%%%%%%%%%%%%%%%%%%%%%%%%%%%%%%
\paragraph{v0.5:} 2017/04/26

\begin{itemize}
\item
functionality in definition file
\end{itemize}


%%%%%%%%%%%%%%%%%%%%%%%%%%%%%%%%%%%%%%%%%%%%%%%%%%%%%%%%%%%%%%%%%%%%%%%%%%%%%%%%
%%%%%%%%%%%%%%%%%%%%%%%%%%%%%%%%%%%%%%%%%%%%%%%%%%%%%%%%%%%%%%%%%%%%%%%%%%%%%%%%
%%%%%%%%%%%%%%%%%%%%%%%%%%%%%%%%%%%%%%%%%%%%%%%%%%%%%%%%%%%%%%%%%%%%%%%%%%%%%%%%
\appendix

\settowidth\MacroIndent{\rmfamily\scriptsize 000\ }

 \DocInput{childdoc.dtx}

\end{document}
%</driver>
% \fi
%
% %%%%%%%%%%%%%%%%%%%%%%%%%%%%%%%%%%%%%%%%%%%%%%%%%%%%%%%%%%%%%%%%%%%%%%%%%%%%%%
% %%%%%%%%%%%%%%%%%%%%%%%%%%%%%%%%%%%%%%%%%%%%%%%%%%%%%%%%%%%%%%%%%%%%%%%%%%%%%%
% \section{Sample}
%\iffalse
%<*samplemain>
%\fi
%
% The following presents a sample document
% with two chapters, two parts, a title page,
% a compile flag as well as three forwarding files to set the flag.
% It consists of eight |.tex| files:
% \begin{center}
% \begin{tabular}{ll}
% |cdocsamp.tex|&main file\\
% |cdocsch1.tex|&include file for chapter 1\\
% |cdocsch2.tex|&include file for chapter 2\\
% |cdocspt3.tex|&include file for part 3\\
% |cdocspt4.tex|&include file for part 4\\
% |cdocsdrf.tex|&forwarding file for main file in draft mode\\
% |cdocsfi1.tex|&forwarding file for final version of chapter 1\\
% |cdocsfi2.tex|&forwarding file for final version of chapter 2\\
% \end{tabular}
% \end{center}
% Each of the eight files can be compiled directly by the \LaTeX{} compiler.
%
% %%%%%%%%%%%%%%%%%%%%%%%%%%%%%%%%%%%%%%
% \paragraph{Main File.}
%
% The main file is called |cdocsamp.tex|.
%
% Load the \textsf{childdoc} definitions and
% declare the filename for the main document:
%    \begin{macrocode}
\input{childdoc.def}
\childdocmain{}
%    \end{macrocode}

% Optional override for |\version| flag:
%    \begin{macrocode}
%%\ifchilddoc\else\providecommand{\version}{draft}\fi
%    \end{macrocode}

% Define the default values for the |\version| flag
% (|final| for the main file and |draft| for childs):
%    \begin{macrocode}
\ifchilddoc
\providecommand{\version}{draft}
\else
\providecommand{\version}{final}
\fi
%    \end{macrocode}

% Load the standard document class:
%    \begin{macrocode}
\documentclass[12pt]{article}
%    \end{macrocode}

% Start the document body:
%    \begin{macrocode}
\begin{document}
%    \end{macrocode}

% Declare a title page.
% Print title, part of document being processed and version flag:
%    \begin{macrocode}
\addtocounter{page}{-1}
\begin{center}
{\LARGE\bfseries{}childdoc example\par}
\vspace{1cm}
\ifchilddoc
\ifchilddocmanual part\else chapter\fi:
`\childdocname' of `\childdocjob'\par
\else
main document: `\childdocjob'\par
\fi
version: \version\par
\end{center}
\newpage
%    \end{macrocode}

% Manually include selected file,
% otherwise process as usual:
%    \begin{macrocode}
\ifchilddocmanual
\section*{part `\childdocname'}
\input{\childdocname}
\else
%    \end{macrocode}

% Include the two chapters:
%    \begin{macrocode}
\include{cdocsch1}
\include{cdocsch2}
%    \end{macrocode}

% Include the two parts unless only chapters should be displayed:
%    \begin{macrocode}
\ifchilddoc\else
\section{part three}
\input{cdocspt3}
\section{part four}
\input{cdocspt4}
\fi
%    \end{macrocode}

% Process as usual until here:
%    \begin{macrocode}
\fi
%    \end{macrocode}

% End of document body:
%    \begin{macrocode}
\end{document}
%    \end{macrocode}
%\iffalse
%</samplemain>
%\fi
%
% %%%%%%%%%%%%%%%%%%%%%%%%%%%%%%%%%%%%%%
% \paragraph{Chapter Include Files.}
%
% The include files are called |cdocsch1.tex| and |cdocsch2.tex|.
%
%\iffalse
%<*samplechap1|samplechap2>
%\fi

% Optional override for |\version| flag:
%    \begin{macrocode}
%%\providecommand{\version}{final}
%    \end{macrocode}

% Include the main document:
%    \begin{macrocode}
\input{childdoc.def}
\childdocof{cdocsamp}
%    \end{macrocode}

%\iffalse
%</samplechap1|samplechap2>
%\fi
%
%\iffalse
%<*samplechap1>
%\fi
% Some text for chapter 1:
%    \begin{macrocode}
\section{one}
some text in chapter one
%    \end{macrocode}

%\iffalse
%</samplechap1>
%\fi
% Some text for chapter 2:
%\iffalse
%<*samplechap2>
%\fi
%    \begin{macrocode}
\section{two}
more text in chapter two
%    \end{macrocode}

%\iffalse
%</samplechap2>
%\fi
%
% %%%%%%%%%%%%%%%%%%%%%%%%%%%%%%%%%%%%%%
% \paragraph{Part Include Files.}
%
% The include files are called |cdocspt3.tex| and |cdocspt4.tex|.
%
%\iffalse
%<*samplepart3|samplepart4>
%\fi

% Optional override for |\version| flag:
%    \begin{macrocode}
%%\providecommand{\version}{final}
%    \end{macrocode}

% Include the main document:
%    \begin{macrocode}
\input{childdoc.def}
\childdocby{cdocsamp}
%    \end{macrocode}

%\iffalse
%</samplepart3|samplepart4>
%\fi
%
%\iffalse
%<*samplepart3>
%\fi
% Some text for part 3:
%    \begin{macrocode}
some text in part three
%    \end{macrocode}

%\iffalse
%</samplepart3>
%\fi
% Some text for part 4:
%\iffalse
%<*samplepart4>
%\fi
%    \begin{macrocode}
more text in part four
%    \end{macrocode}

%\iffalse
%</samplepart4>
%\fi
%
% %%%%%%%%%%%%%%%%%%%%%%%%%%%%%%%%%%%%%%
% \paragraph{Forwarding for a Complete Draft.}
%
% The following forwarding file |cdocsdrf.tex|
% compiles the main document in draft mode:
%\iffalse
%<*sampledraft>
%\fi
%    \begin{macrocode}
\def\version{draft}
\input{childdoc.def}
\childdocforward{cdocsamp}
%    \end{macrocode}

%\iffalse
%</sampledraft>
%\fi
%
% %%%%%%%%%%%%%%%%%%%%%%%%%%%%%%%%%%%%%%
% \paragraph{Forwarding for Final Version of the Chapters.}
%
% The following forwarding files |cdocsfn1.tex| and |cdocsfn2.tex|
% (with identical content)
% compile the final versions of the child documents
% |cdocsch1.tex| and |cdocsch2.tex|, respectively:
%\iffalse
%<*samplefinal>
%\fi
%    \begin{macrocode}
\def\version{final}
\input{childdoc.def}
\childdocforwardprefix[cdocsamp]{cdocsfn}{cdocsch}
%    \end{macrocode}

%\iffalse
%</samplefinal>
%\fi
%
% %%%%%%%%%%%%%%%%%%%%%%%%%%%%%%%%%%%%%%
% \paragraph{Command Line Processing.}
%
% The following three command lines generate the output files
% |cdocscld|, |cdocscl1| and |cdocscl2|
% which should be identical to
% |cdocsdrf|, |cdocsch1| and |cdocsfn2|, respectively:
% \begin{center}
% \begin{tabular}{l}
% |latex -jobname cdocscld \|\\
% |  "\def\version{draft}\input{childdoc.def}\childdocforward{cdocsamp}"|\\
% |latex -jobname cdocscl1 \|\\
% |  "\input{childdoc.def}\childdocforward[cdocsamp]{cdocsch1}"|\\
% |latex -jobname cdocscl2 \|\\
% |  "\def\version{final}\input{childdoc.def}\childdocforward{cdocsch2}"|
% \end{tabular}
% \end{center}
% Note that the trailing backslash on each first line
% merely continues the input to the second line
% (for convenient cut ant paste).
% Furthermore, the command |latex| can be replaced by any
% of its alternative versions such as |pdflatex|.
%
% %%%%%%%%%%%%%%%%%%%%%%%%%%%%%%%%%%%%%%%%%%%%%%%%%%%%%%%%%%%%%%%%%%%%%%%%%%%%%%
% %%%%%%%%%%%%%%%%%%%%%%%%%%%%%%%%%%%%%%%%%%%%%%%%%%%%%%%%%%%%%%%%%%%%%%%%%%%%%%
% \section{Implementation}
%\iffalse
%<*package>
%\fi
%
% This section describes the definitions file |childdoc.def|.

% The definitions cannot be loaded using |\usepackage| or |\RequirePackage|
% which has a mechanism to prevent loading a style file more than once.
% When loading the definitions by means of |\input|
% multiple instances have to be prevented manually:
%\iffalse
%This code needs to be before the `\ProvidesFile' directive
%which is defined at the beginning of this file.
%Therefore it is also placed there and commented out here.
%</package>
%<*discard>
%\fi
%    \begin{macrocode}
\ifdefined\childdocmain\endinput\fi
%    \end{macrocode}
%\iffalse
%</discard>
%<*package>
%\fi
%
% \macro{\ifchilddoc}
% \macro{\ifchilddocmanual}
% The conditional |\ifchilddoc| tells whether a
% child (true) or main (false) document is being compiled.
% The conditional |\ifchilddocmanual| tells whether
% the |\includeonly| mechanism is used (false) or
% the selection of child files must be performed manually (true).
% The definitions initialise to false:
%    \begin{macrocode}
\newif\ifchilddoc
\newif\ifchilddocmanual
%    \end{macrocode}

% \macro{\childdocname}
% \macro{\childdocjob}
% The macro |\childdocname| stores the name of the main document
% to be compiled. The macro |\childdocjob| stores the name of
% the document on which the \LaTeX{} compiler was originally invoked.
% The content of |\jobname| cannot be compared
% to filenames specified in the source due to different catcodes.
% The following code rescans |\jobname|, stores the result
% in |\childdocname| and saves a copy in |\childdocjob|:
%    \begin{macrocode}
\edef\childdocname{\scantokens\expandafter{\jobname\noexpand}}
\let\childdocjob\childdocname
%    \end{macrocode}

% \macro{\childdocdisable}
% The macro |\childdocdisable| prevents the main file
% from being processed more than once.
% At this stage, the main document command |\childdocmain|
% is assumed to be called once again where it should do nothing.
% Any subsequent call to it should prevent
% a secondary processing of the main document
% It overwrites the forwarding commands
% |\childdocof| and |\childdocforward|
% with empty macros to prevent further inclusions of the main document:
%    \begin{macrocode}
\newcommand{\childdocdisable}
{
  \renewcommand{\childdocmain}[1]{\renewcommand{\childdocmain}[1]{\endinput}}
  \renewcommand{\childdocof}[1]{}
  \renewcommand{\childdocby}[2][]{}
  \renewcommand{\childdocforward}[2][]{}
  \renewcommand{\childdocdisable}{}
}
%    \end{macrocode}

% \macro{\childdocmain}
% The macro |\childdocmain| is to be called at the top of the main file
% with nothing or the main filename (without extension) as argument.
% First, it breaks loops.
% If the argument is not empty and does not match |\childdocname|
% (which is set by the first inclusion of |childdoc.def|),
% |\ifchilddoc| is set to true, |\includeonly| is applied to the child file
% and |\jobname| is set to the main file
% (for proper handling of |.aux| files):
%    \begin{macrocode}
\newcommand{\childdocmain}[1]
{
  \childdocdisable\childdocmain{}
  \if?#1?\else
    \begingroup
      \def\childdoctmp{#1}
      \ifx\childdoctmp\childdocname
        \def\childdoctmp{}
      \else
        \def\childdoctmp
        {
          \childdoctrue
          \includeonly{\childdocname}
          \def\childdocjob{#1}
          \def\jobname{#1}
        }
      \fi
      \expandafter
    \endgroup
    \childdoctmp
  \fi
}
%    \end{macrocode}

% \macro{\childdocof}
% The command |\childdocof| redirects
% compilation to the main file |#1|.
%    \begin{macrocode}
\newcommand{\childdocof}[1]
{
  \childdocdisable
  \childdoctrue
  \includeonly{\childdocname}
  \def\jobname{#1}
  \def\childdocjob{#1}
  \input{#1}
}
%    \end{macrocode}

% \macro{\childdocby}
% The command |\childdocby| ....
%    \begin{macrocode}
\newcommand{\childdocby}[2][]
{
  \childdocdisable
  \childdoctrue
  \childdocmanualtrue
  \if?#1?\else
    \def\jobname{#2}
  \fi
  \def\childdocjob{#2}
  \input{#2}
  \endinput
}
%    \end{macrocode}

% \macro{\childdocforward}
% The command |\childdocforward| redirects
% compilation to the main file or
% (if the optional argument is given) a child file.
% Parameters are set as if the main file
% or a child file starting with |\childdocof| was compiled.
% Then compilation is handed over to the main file:
%    \begin{macrocode}
\newcommand{\childdocforward}[2][]
{
  \begingroup
    \if?#1?
      \def\childdoctmp
      {
        \def\childdocname{#2}
        \def\childdocjob{#2}
        \def\jobname{#2}
        \input{#2}
        \endinput
      }
    \else
      \def\childdoctmp
      {
        \childdocdisable
        \def\childdocname{#2}
        \childdoctrue
        \includeonly{#2}
        \def\childdocjob{#1}
        \def\jobname{#1}
        \input{#1}
        \endinput
      }
    \fi
    \expandafter
  \endgroup
  \childdoctmp
}
%    \end{macrocode}

% \macro{\childdocforwardprefix}
% The command |\childdocforwardprefix| redirects
% compilation to the main or a child file by means of a pattern.
% The prefix |#1| in the current filename is replaced by |#2|
% and the suffix of the current filename is kept
% (it is assumed that the filename does not contain the substring `|~~~|'
% which is used as a delimiter).
% Compilation is handed over to the new file by |\childdocforward|:
%    \begin{macrocode}
\newcommand{\childdocforwardprefix}[3][]
{
  \begingroup
    \def\childdocextract #2##1~~~{\def\childdoctmp{\childdocforward[#1]{#3##1}}}
    \expandafter\childdocextract\childdocname~~~
    \expandafter
  \endgroup
  \childdoctmp
}
%    \end{macrocode}

% \macro{\childdoc}
% The deprecated macro |\childdoc| is a legacy version of |\childdocmain|:
%    \begin{macrocode}
\newcommand{\childdoc}{\childdocmain}
%    \end{macrocode}

% \macro{\childdocredirect}
% The deprecated macro |\childdocredirect| is a legacy version
% of |\childdocforward| and |\childdocforwardprefix|:
%    \begin{macrocode}
\newcommand{\childdocredirect}[2][]
{
  \begingroup
    \if?#1?
      \def\childdoctmp{\childdocforward{#2}}
    \else
      \def\childdoctmp{\childdocforwardprefix{#1}{#2}}
    \fi
    \expandafter
  \endgroup
  \childdoctmp
}
%    \end{macrocode}

%\iffalse
%</package>
%\fi
%
\endinput
|\\
|\childdocof{|\textit{main}|}|\\
\end{tabular}
\end{center}
at the top of every child file \textit{child}
which is included by |\include{|\textit{child}|}|
from within the main file
(or at least for those files to be compiled individually).
The argument \textit{main} must be the filename of the main file.

There are a couple of
considerations in setting up the main and child documents:

%%%%%%%%%%%%%%%%%%%%%%%%%%%%%%%%%%%%%%%%
\paragraph{Restrictions.}

Please note the following restrictions:
\begin{itemize}
\item
|\childdocmain| must be called with one argument \textit{main}
to ensure compatibility with earlier version of the package.
It must either be empty (|\childdocmain{}|)
or precisely match the filename of the main file in which it is specified.
See \secref{sec:detection} for further information.
\item
The filename \textit{main} must be specified without the |.tex| extension.
\item
The filename \textit{main} is case sensitive
(even in case-insensitive file systems)
due to internal string comparison.
\item
The argument \textit{main} should be fully expanded, it cannot be a macro.
\item
Subdirectories and special characters should be avoided in filenames.
\item
The command |\childdocmain{|\textit{main}|}| must be followed by a whitespace.
It should not be followed immediately by another command
or by a comment mark `|%|'.
This is because the \TeX{} parser reads the token immediately following
the argument of |\childdocmain| and puts it
at the beginning of every child section;
however, a white\-space is ignored.
\end{itemize}

%%%%%%%%%%%%%%%%%%%%%%%%%%%%%%%%%%%%%%%%
\paragraph{Content of Main File.}

It is advisable to place all content in the child files included by |\include|.
Any output contained in the main file will appear in all child documents
unless suppressed manually;
it cannot be suppressed automatically by the |\includeonly| directive
and thus should normally be avoided.
A method to include some content in the main file
by means of conditional processing is described in \secref{sec:conditional}.

%%%%%%%%%%%%%%%%%%%%%%%%%%%%%%%%%%%%%%%%
\paragraph{Page Numbering.}

When only a part of the document is compiled,
the appropriate numbering of pages
(as well as other status parameters)
is determined from the |.aux| files.
The latter contain information from previous passes.
However this information needs to propagate through
all intermediate child documents.
Therefore the page numbering in child documents may well
be inconsistent until the complete document is compiled at least once.

A useful (if unconventional) way to always ensure a consistent
page numbering is to restart the numbering in each child document
and denote the pages by `\textit{child}|.|\textit{page}'
where \textit{child} represents the chapter/section number of the child file.
This can be achieved by the command
|\numberwithin{page}{|\textit{child}|}|
of the \textsf{amsmath} package
where \textit{child} can be |chapter| or |section|
depending on the chosen structuring.
Alternatively, one can modify the macro |\thepage| appropriately
and reset the counter |page| at the start of each child file.

%%%%%%%%%%%%%%%%%%%%%%%%%%%%%%%%%%%%%%%%%%%%%%%%%%%%%%%%%%%%%%%%%%%%%%%%%%%%%%%%
\subsection{Conditional Processing}
\label{sec:conditional}

The package provides a mechanism to compile different versions
of a document. To customise the versions further some conditional processing
can come in handy to distinguish which version is being compiled.
The package provides two macros to describe the compilation context:

%%%%%%%%%%%%%%%%%%%%%%%%%%%%%%%%%%%%%%%%
\DescribeMacro{\ifchilddoc}
The conditional |\ifchilddoc| distinguishes between the compilation of
child documents and the main document:
%
\begin{center}
|\ifchilddoc |\textit{child-code}| |[|\||else |\textit{main-code}]| \||fi|
\end{center}

%%%%%%%%%%%%%%%%%%%%%%%%%%%%%%%%%%%%%%%%
\DescribeMacro{\childdocname}
\DescribeMacro{\childdocjob}
The macro |\childdocname| contains the filename (without extension)
of the main or child file being processed.
Note that |\childdocjob| will always contain the name of the main file.

%%%%%%%%%%%%%%%%%%%%%%%%%%%%%%%%%%%%%%%%
\paragraph{Title Page.}

Conditional processing can be used to include a title or banner page
in the main document when proper precautions are taken.
Importantly, the code in the main file should ensure that the page counter
(as well as other status parameters which are stored in the |.aux| files)
takes the same value after the conditional processing.
Otherwise the page numbers may take divergent values
depending on which part is compiled.

For example, a title page could be declared by:
%
\begin{center}
\begin{tabular}{l}
|\ifchilddoc\||else|\\
|\addtocounter{page}{-1}|\\
\textit{code for title page}\\
|\newpage|\\
|\||fi|
\end{tabular}
\end{center}
%
A banner page for the child documents can be generated by:
%
\begin{center}
\begin{tabular}{l}
|\ifchilddoc|\\
|\addtocounter{page}{-1}|\\
\textit{code for banner page}\\
|\newpage|\\
|\||fi|
\end{tabular}
\end{center}
%
Here one could write a message such as:
\begin{center}
|This is the part \childdocname{} of \childdocjob{}.|
\end{center}

%%%%%%%%%%%%%%%%%%%%%%%%%%%%%%%%%%%%%%%%%%%%%%%%%%%%%%%%%%%%%%%%%%%%%%%%%%%%%%%%
\subsection{Flags}
\label{sec:flags}

The package makes it easy to generate different versions
of the main or child documents.
To this end compilation flags can be defined
and assigned different default values.
They will be particularly useful in conjunction
with the forwarding mechanism described in \secref{sec:forward}.

For example, it may be useful to have a flag |\version|
which can be set to |draft| or |final|.
The document source will contain some conditional code
depending on the value of |\version|.
Suppose further, the flag should default to |final| for the main file
and to |draft| for child files
which is a natural assignment for editing the document.
This is achieved by placing the following code
in the preamble of the main document
(below the |\childdocmain| directive):
%
\begin{center}
\begin{tabular}{l}
|\ifchilddoc|\\
|\providecommand{\version}{draft}|\\
|\||else|\\
|\providecommand{\version}{final}|\\
|\||fi|
\end{tabular}
\end{center}
%
The definition by |\providecommand| makes sure
that previous definitions are not overwritten.
Further statements |\providecommand{\version}{...}|
can thus be added before the above code to override it.

For the main file, one might add a line
(between |\childdocmain| and the above block)
%
\begin{center}
|%\ifchilddoc\||else\providecommand{\version}{draft}\||fi|
\end{center}
%
which can be uncommented to produce a draft version.
Likewise one can add a line to the very top of a child file
(above the |\childdocof{|\textit{main}|}| directive)
%
\begin{center}
|%\providecommand{\version}{final}|
\end{center}
%
which can be uncommented to produce the final version of this child document.

%%%%%%%%%%%%%%%%%%%%%%%%%%%%%%%%%%%%%%%%%%%%%%%%%%%%%%%%%%%%%%%%%%%%%%%%%%%%%%%%
\subsection{Forwarding}
\label{sec:forward}

Different versions of the main or child documents
using compilation flags as described in \secref{sec:flags}
can be (permanently) stored in different files
for convenient compilation, viewing and distribution.
To this end, the package defines a command
to pass on compilation to a different file:

%%%%%%%%%%%%%%%%%%%%%%%%%%%%%%%%%%%%%%%%
\DescribeMacro{\childdocforward}
The command |\childdocforward| redirects processing to
another source file:
%
\begin{center}
\begin{tabular}{l}
|% \iffalse
%
% childdoc.dtx Copyright (C) 2017-2018 Niklas Beisert
%
% This work may be distributed and/or modified under the
% conditions of the LaTeX Project Public License, either version 1.3
% of this license or (at your option) any later version.
% The latest version of this license is in
%   http://www.latex-project.org/lppl.txt
% and version 1.3 or later is part of all distributions of LaTeX
% version 2005/12/01 or later.
%
% This work has the LPPL maintenance status `maintained'.
%
% The Current Maintainer of this work is Niklas Beisert.
%
% This work consists of the files childdoc.dtx and childdoc.ins
% and the derived files childdoc.def and cdocsamp.tex with
% cdocsch1.tex, cdocsch2.tex, cdocsdrf.tex, cdocsfn1.tex, cdocsfn2.tex.
%
%<package>\ifdefined\childdocmain\endinput\fi
%<package>\ProvidesFile{childdoc.def}[2018/12/30 v2.0 child document driver]
%<samplemain>\ProvidesFile{cdocsamp.tex}[2018/12/30 v2.0 sample for childdoc]
%<*driver>
%\ProvidesFile{childdoc.drv}[2018/12/30 v2.0 childdoc reference manual file]
\PassOptionsToClass{10pt,a4paper}{article}
\documentclass{ltxdoc}

\usepackage[margin=35mm]{geometry}
\usepackage{hyperref}
\usepackage{hyperxmp}
\usepackage[usenames]{color}

\hypersetup{colorlinks=true}
\hypersetup{pdfstartview=FitH}
\hypersetup{pdfpagemode=UseNone}
\hypersetup{pdfsource={}}
\hypersetup{pdflang={en-UK}}
\hypersetup{pdfcopyright={Copyright 2017-2018 Niklas Beisert.
  This work may be distributed and/or modified under the
  conditions of the LaTeX Project Public License, either version 1.3
  of this license or (at your option) any later version.}}
\hypersetup{pdflicenseurl={http://www.latex-project.org/lppl.txt}}
\hypersetup{pdfcontactaddress={ETH Zurich, ITP, HIT K,
  Wolfgang-Pauli-Strasse 27}}
\hypersetup{pdfcontactpostcode={8093}}
\hypersetup{pdfcontactcity={Zurich}}
\hypersetup{pdfcontactcountry={Switzerland}}
\hypersetup{pdfcontactemail={nbeisert@itp.phys.ethz.ch}}
\hypersetup{pdfcontacturl={http://people.phys.ethz.ch/\xmptilde nbeisert/}}

\newcommand{\secref}[1]{\hyperref[#1]{section \ref*{#1}}}

\parskip1ex
\parindent0pt
\let\olditemize\itemize
\def\itemize{\olditemize\parskip0pt}

\begin{document}

\title{The \textsf{childdoc} Package}
\hypersetup{pdftitle={The childdoc Package}}
\author{Niklas Beisert\\[2ex]
  Institut f\"ur Theoretische Physik\\
  Eidgen\"ossische Technische Hochschule Z\"urich\\
  Wolfgang-Pauli-Strasse 27, 8093 Z\"urich, Switzerland\\[1ex]
  \href{mailto:nbeisert@itp.phys.ethz.ch}
  {\texttt{nbeisert@itp.phys.ethz.ch}}}
\hypersetup{pdfauthor={Niklas Beisert}}
\hypersetup{pdfsubject={Manual for the LaTeX2e Package childdoc}}
\date{30 December 2018, \textsf{v2.0}}
\maketitle

\begin{abstract}\noindent
\textsf{childdoc} is a \LaTeXe{} package
that enables the direct compilation
of document sections included by |\include|
to individual files.
\end{abstract}

\begingroup
\parskip0ex
\tableofcontents
\endgroup

%%%%%%%%%%%%%%%%%%%%%%%%%%%%%%%%%%%%%%%%%%%%%%%%%%%%%%%%%%%%%%%%%%%%%%%%%%%%%%%%
%%%%%%%%%%%%%%%%%%%%%%%%%%%%%%%%%%%%%%%%%%%%%%%%%%%%%%%%%%%%%%%%%%%%%%%%%%%%%%%%
\section{Introduction}

\LaTeX{} provides a mechanism to structure a large document (such as a book)
into a main file and several child files (containing the chapters)
using the |\include| command.
This mechanism is beneficial for documents
which span hundreds of pages in order to
make the source file(s) more manageable.
Moreover, compilation can be restricted to
selected child files by means of the |\includeonly| command.
The latter feature can be used to reduce the compilation time while editing
(this was significantly more useful in the earlier days of \LaTeX{})
or to generate a smaller document which is easier to navigate.
Another application of |\includeonly| is to generate
documents consisting of selected parts of the complete document.

However, there are a few drawbacks of the plain |\include| mechanism:
\begin{itemize}
\item
The child files cannot be compiled on their own,
they can only be compiled via the main file.
A naive editing environment
(such as a text editor with an option
to have the current file processed by \LaTeX)
may require one to switch to the main file before compiling;
attempting to compile the child file produces errors.
\item
The main file must be modified (each time)
to adjust the |\includeonly| command
to the present needs. This easily leaves the main file in a messy state.
\item
The generated document will always carry the filename
of the main document. This is inconvenient if
several child files are to be compiled and
to be kept for distribution.
\end{itemize}

The present package provides a simple interface
to make child files individually compilable by \LaTeX{}.
Compiling a child file then has the same effect as compiling
the main file with an |\includeonly| command
to select the appropriate child.
Moreover the generated document will carry the name of the child
rather than the main file.
This resolves all three above issues.

This feature is meant to make the editing of books,
thesis documents and lecture notes somewhat more convenient.
However, the package can also be used efficiently for
composing a series of documents (such as exercise sheets)
which are typically distributed individually.
It then assists the author in generating the individual documents
(potentially in different versions)
as well as a document containing the collected series.
Another application is in developing style files
or other kinds of included material
where compilation of the style file could redirect
to a sample or test file.

%%%%%%%%%%%%%%%%%%%%%%%%%%%%%%%%%%%%%%%%%%%%%%%%%%%%%%%%%%%%%%%%%%%%%%%%%%%%%%%%
%%%%%%%%%%%%%%%%%%%%%%%%%%%%%%%%%%%%%%%%%%%%%%%%%%%%%%%%%%%%%%%%%%%%%%%%%%%%%%%%
\section{Usage}

First of all, the package \textsf{childdoc} is \emph{not} a standard
\LaTeXe{} |.sty| style file! Therefore it needs to be invoked in
a non-standard way.

%%%%%%%%%%%%%%%%%%%%%%%%%%%%%%%%%%%%%%%%%%%%%%%%%%%%%%%%%%%%%%%%%%%%%%%%%%%%%%%%
\subsection{Included Files}
\label{sec:include}

%%%%%%%%%%%%%%%%%%%%%%%%%%%%%%%%%%%%%%%%
\DescribeMacro{\childdocmain}
To use the package, add the commands
\begin{center}
\begin{tabular}{l}
|\input{childdoc.def}|\\
|\childdocmain{}|\\
\end{tabular}
\end{center}
at the very top of the main \LaTeX{} file,
in particular \emph{before} the |\documentclass| statement!
The argument of |\childdocmain| should be left empty
(but it must be present).

%%%%%%%%%%%%%%%%%%%%%%%%%%%%%%%%%%%%%%%%
\DescribeMacro{\childdocof}
Furthermore, add the commands
\begin{center}
\begin{tabular}{l}
|\input{childdoc.def}|\\
|\childdocof{|\textit{main}|}|\\
\end{tabular}
\end{center}
at the top of every child file \textit{child}
which is included by |\include{|\textit{child}|}|
from within the main file
(or at least for those files to be compiled individually).
The argument \textit{main} must be the filename of the main file.

There are a couple of
considerations in setting up the main and child documents:

%%%%%%%%%%%%%%%%%%%%%%%%%%%%%%%%%%%%%%%%
\paragraph{Restrictions.}

Please note the following restrictions:
\begin{itemize}
\item
|\childdocmain| must be called with one argument \textit{main}
to ensure compatibility with earlier version of the package.
It must either be empty (|\childdocmain{}|)
or precisely match the filename of the main file in which it is specified.
See \secref{sec:detection} for further information.
\item
The filename \textit{main} must be specified without the |.tex| extension.
\item
The filename \textit{main} is case sensitive
(even in case-insensitive file systems)
due to internal string comparison.
\item
The argument \textit{main} should be fully expanded, it cannot be a macro.
\item
Subdirectories and special characters should be avoided in filenames.
\item
The command |\childdocmain{|\textit{main}|}| must be followed by a whitespace.
It should not be followed immediately by another command
or by a comment mark `|%|'.
This is because the \TeX{} parser reads the token immediately following
the argument of |\childdocmain| and puts it
at the beginning of every child section;
however, a white\-space is ignored.
\end{itemize}

%%%%%%%%%%%%%%%%%%%%%%%%%%%%%%%%%%%%%%%%
\paragraph{Content of Main File.}

It is advisable to place all content in the child files included by |\include|.
Any output contained in the main file will appear in all child documents
unless suppressed manually;
it cannot be suppressed automatically by the |\includeonly| directive
and thus should normally be avoided.
A method to include some content in the main file
by means of conditional processing is described in \secref{sec:conditional}.

%%%%%%%%%%%%%%%%%%%%%%%%%%%%%%%%%%%%%%%%
\paragraph{Page Numbering.}

When only a part of the document is compiled,
the appropriate numbering of pages
(as well as other status parameters)
is determined from the |.aux| files.
The latter contain information from previous passes.
However this information needs to propagate through
all intermediate child documents.
Therefore the page numbering in child documents may well
be inconsistent until the complete document is compiled at least once.

A useful (if unconventional) way to always ensure a consistent
page numbering is to restart the numbering in each child document
and denote the pages by `\textit{child}|.|\textit{page}'
where \textit{child} represents the chapter/section number of the child file.
This can be achieved by the command
|\numberwithin{page}{|\textit{child}|}|
of the \textsf{amsmath} package
where \textit{child} can be |chapter| or |section|
depending on the chosen structuring.
Alternatively, one can modify the macro |\thepage| appropriately
and reset the counter |page| at the start of each child file.

%%%%%%%%%%%%%%%%%%%%%%%%%%%%%%%%%%%%%%%%%%%%%%%%%%%%%%%%%%%%%%%%%%%%%%%%%%%%%%%%
\subsection{Conditional Processing}
\label{sec:conditional}

The package provides a mechanism to compile different versions
of a document. To customise the versions further some conditional processing
can come in handy to distinguish which version is being compiled.
The package provides two macros to describe the compilation context:

%%%%%%%%%%%%%%%%%%%%%%%%%%%%%%%%%%%%%%%%
\DescribeMacro{\ifchilddoc}
The conditional |\ifchilddoc| distinguishes between the compilation of
child documents and the main document:
%
\begin{center}
|\ifchilddoc |\textit{child-code}| |[|\||else |\textit{main-code}]| \||fi|
\end{center}

%%%%%%%%%%%%%%%%%%%%%%%%%%%%%%%%%%%%%%%%
\DescribeMacro{\childdocname}
\DescribeMacro{\childdocjob}
The macro |\childdocname| contains the filename (without extension)
of the main or child file being processed.
Note that |\childdocjob| will always contain the name of the main file.

%%%%%%%%%%%%%%%%%%%%%%%%%%%%%%%%%%%%%%%%
\paragraph{Title Page.}

Conditional processing can be used to include a title or banner page
in the main document when proper precautions are taken.
Importantly, the code in the main file should ensure that the page counter
(as well as other status parameters which are stored in the |.aux| files)
takes the same value after the conditional processing.
Otherwise the page numbers may take divergent values
depending on which part is compiled.

For example, a title page could be declared by:
%
\begin{center}
\begin{tabular}{l}
|\ifchilddoc\||else|\\
|\addtocounter{page}{-1}|\\
\textit{code for title page}\\
|\newpage|\\
|\||fi|
\end{tabular}
\end{center}
%
A banner page for the child documents can be generated by:
%
\begin{center}
\begin{tabular}{l}
|\ifchilddoc|\\
|\addtocounter{page}{-1}|\\
\textit{code for banner page}\\
|\newpage|\\
|\||fi|
\end{tabular}
\end{center}
%
Here one could write a message such as:
\begin{center}
|This is the part \childdocname{} of \childdocjob{}.|
\end{center}

%%%%%%%%%%%%%%%%%%%%%%%%%%%%%%%%%%%%%%%%%%%%%%%%%%%%%%%%%%%%%%%%%%%%%%%%%%%%%%%%
\subsection{Flags}
\label{sec:flags}

The package makes it easy to generate different versions
of the main or child documents.
To this end compilation flags can be defined
and assigned different default values.
They will be particularly useful in conjunction
with the forwarding mechanism described in \secref{sec:forward}.

For example, it may be useful to have a flag |\version|
which can be set to |draft| or |final|.
The document source will contain some conditional code
depending on the value of |\version|.
Suppose further, the flag should default to |final| for the main file
and to |draft| for child files
which is a natural assignment for editing the document.
This is achieved by placing the following code
in the preamble of the main document
(below the |\childdocmain| directive):
%
\begin{center}
\begin{tabular}{l}
|\ifchilddoc|\\
|\providecommand{\version}{draft}|\\
|\||else|\\
|\providecommand{\version}{final}|\\
|\||fi|
\end{tabular}
\end{center}
%
The definition by |\providecommand| makes sure
that previous definitions are not overwritten.
Further statements |\providecommand{\version}{...}|
can thus be added before the above code to override it.

For the main file, one might add a line
(between |\childdocmain| and the above block)
%
\begin{center}
|%\ifchilddoc\||else\providecommand{\version}{draft}\||fi|
\end{center}
%
which can be uncommented to produce a draft version.
Likewise one can add a line to the very top of a child file
(above the |\childdocof{|\textit{main}|}| directive)
%
\begin{center}
|%\providecommand{\version}{final}|
\end{center}
%
which can be uncommented to produce the final version of this child document.

%%%%%%%%%%%%%%%%%%%%%%%%%%%%%%%%%%%%%%%%%%%%%%%%%%%%%%%%%%%%%%%%%%%%%%%%%%%%%%%%
\subsection{Forwarding}
\label{sec:forward}

Different versions of the main or child documents
using compilation flags as described in \secref{sec:flags}
can be (permanently) stored in different files
for convenient compilation, viewing and distribution.
To this end, the package defines a command
to pass on compilation to a different file:

%%%%%%%%%%%%%%%%%%%%%%%%%%%%%%%%%%%%%%%%
\DescribeMacro{\childdocforward}
The command |\childdocforward| redirects processing to
another source file:
%
\begin{center}
\begin{tabular}{l}
|\input{childdoc.def}|\\
|\childdocforward[|\textit{main}|]{|\textit{dest}|}|\\
\end{tabular}
\end{center}
%
The argument \textit{dest} is the destination file
(without extension).
It should be the main file or one of the child files.
Note that further \textsf{childdoc} directives
such as |\childdocof| and |\childdocforward|
in the indicated file will be processed in this form.
The optional argument \textit{main}
passes on directly to the main file \textit{main}
while pretending to compile the child \textit{dest}.
This form behaves as if \textit{dest}
issues |\childdocof{|\textit{main}|}| right away,
and no further \textsf{childdoc} directives will be processed.

%%%%%%%%%%%%%%%%%%%%%%%%%%%%%%%%%%%%%%%%
\DescribeMacro{\...prefix}
In the alternative form |\childdocforwardprefix|,
%
\begin{center}
\begin{tabular}{l}
|\input{childdoc.def}|\\
|\childdocforwardprefix[|\textit{main}|]{|\textit{prefix}|}{|\textit{dest}|}|
\end{tabular}
\end{center}
%
the destination file is determined by a pattern
depending on the current file:
To make this work, the current file must be called
`{\textit{prefix}\hspace{0.2em}\textit{suffix}}'
with \textit{prefix} matching precisely the argument.
Processing is then passed on to the file
`{\textit{dest}\hspace{0.2em}\textit{suffix}}'.
Surely, the same effect is achieved by
directly specifying the
argument `{\textit{dest}\hspace{0.2em}\textit{suffix}}'
in the first form.
However, that requires to set up a different file
for each child. With the alternative form of the command
all these files can have exactly the same content
which simplifies setting them up and maintaining them.

For example, the following file |draft.tex|
with a compilation flag |\version| as described in \secref{sec:flags}
compiles the main document as a draft:
%
\begin{center}
\begin{tabular}{l}
|\def\version{draft}|\\
|\input{childdoc.def}|\\
|\childdocforward{|\textit{main}|}|
\end{tabular}
\end{center}
%
Likewise, the following files |final|\textit{nn}|.tex|
compile the final version of the child document
|child|\textit{nn}|.tex|:
%
\begin{center}
\begin{tabular}{l}
|\def\version{final}|\\
|\input{childdoc.def}|\\
|\childdocforwardprefix{final}{child}|
\end{tabular}
\end{center}
%

Note that when several versions of a main file and/or of each child file
are to be generated, it may be convenient to set up a |Makefile| or
shell script to automatise the process.

%%%%%%%%%%%%%%%%%%%%%%%%%%%%%%%%%%%%%%%%%%%%%%%%%%%%%%%%%%%%%%%%%%%%%%%%%%%%%%%%
\subsection{Command Line Processing}
\label{sec:commandline}

The effect of redirection files can also be achieved by invoking
the \LaTeX{} compiler with a more elaborate command line.
Most conveniently this should be done as part
of a shell script or a |Makefile|.

When using \textsf{childdoc} in the main file, the following
command lines effectively perform a redirection
(note that depending on the shell being used,
backslashes may have to be doubled: `|\|' $\to$ `|\\|'):
%
\begin{center}
|... -jobname "|\textit{target}|" |\\|"|[\textit{flags}]%
|\input{childdoc.def}\childdocforward[|\textit{main}|]{|\textit{dest}|}"|
\end{center}
%
Here \textit{target} is the name of the output file,
\textit{main} is the name of the main file
and \textit{dest} is the name of the main or child file to be processed
(all filenames without extensions).
The optional argument \textit{main} can be omitted
if \textit{main} matches \textit{dest}.
Optionally, compilation \textit{flags} can be defined via |\def| commands.
This command line makes the \TeX{} engine believe
it is compiling the file \textit{target}
whose content is specified as the latter parameter.
The provided code then forwards the processing to
\textit{main} or \textit{dest} as described in \secref{sec:forward}.

%%%%%%%%%%%%%%%%%%%%%%%%%%%%%%%%%%%%%%%%%%%%%%%%%%%%%%%%%%%%%%%%%%%%%%%%%%%%%%%%
\subsection{Include by Input}
\label{sec:input}

Including child documents by |\include| has some restrictions by design.
Most notably, the content of a child document always occupies
its own set of pages; pages cannot be shared between child documents.
Usually, this behaviour makes perfect sense
because each child document contain an essential part of the document.
However, in some situations it may be desirable to compose
a document from a collection of parts
without having mandatory page breaks between then.
For this case, the package
provides a mechanism to include parts
by |\input| which can also be processed individually.
However, by construction this mechanism
requires manual handling of the content to be output.

%%%%%%%%%%%%%%%%%%%%%%%%%%%%%%%%%%%%%%%%
\DescribeMacro{\ifchilddocmanual}
The main file should be prepared as usual, see \secref{sec:include}.
However, the document body must make a distinction
between processing of an individual part and of the main document, e.g.:
%
\begin{center}
\begin{tabular}{l}
|\ifchilddocmanual|\\
|\input{\childdocname}|\\
|\||else|\\
\textit{document body with }|\input{|\textit{part}|}|\\
|\||fi|
\end{tabular}
\end{center}
%
The conditional |\ifchilddocmanual| is true whenever
a part to be included by |\input| is being compiled,
and the name of the part is stored in |\childdocname|.

%%%%%%%%%%%%%%%%%%%%%%%%%%%%%%%%%%%%%%%%
\DescribeMacro{\childdocby}
Each part to be included by |\input| should start with:
%
\begin{center}
\begin{tabular}{l}
|\input{childdoc.def}|\\
|\childdocby{|\textit{main}|}|\\
\end{tabular}
\end{center}
%
The directive |\childdocby| is similar to |\childdocof|
described in \secref{sec:include},
but the subsequent selection of content must be done manually.
To that end, both |\ifchilddoc| and |\ifchilddocmanual|
will be true upon processing of a part,
and the name of the part is stored in |\childdocname|.
Note that |\jobname| will be set to the filename of the current part
so that each part receives an individual |.aux| file
that does not interfere with the |.aux| file(s) of the main document.
This behaviour can be altered by the alternative form
|\childdocby[*]{|\textit{main}|}| (with a non-empty optional argument)
which uses the |.aux| file of the main document
by setting |\jobname| to \textit{main}.

%%%%%%%%%%%%%%%%%%%%%%%%%%%%%%%%%%%%%%%%%%%%%%%%%%%%%%%%%%%%%%%%%%%%%%%%%%%%%%%%
\subsection{Driver Development}
\label{sec:driver}

The \textsf{childdoc} mechanism can also be use for the development
of definition files such as \LaTeX{} styles or classes.
This case differs from the above setup with multiple parts
included by |\include| in that no |\includeonly| should be invoked.
This can be achieved by starting the include file
(before |\ProvidesPackage|) with:
%
\begin{center}
\begin{tabular}{l}
|\input{childdoc.def}|\\
|\childdocforward{|\textit{main}|}|\\
\end{tabular}
\end{center}
%
or alternatively with:
%
\begin{center}
\begin{tabular}{l}
|\input{childdoc.def}|\\
|\childdocby{|\textit{main}|}|\\
\end{tabular}
\end{center}
%
Both forms have slightly different effects as described above.
The main file is prepared as usual, see \secref{sec:include}.

%%%%%%%%%%%%%%%%%%%%%%%%%%%%%%%%%%%%%%%%%%%%%%%%%%%%%%%%%%%%%%%%%%%%%%%%%%%%%%%%
\subsection{Legacy Detection}
\label{sec:detection}

The directive |\childdocmain| in the main file can detect
whether the complete document or merely a child is to be compiled
even without using the directive |\childdocof|.
This method is deprecated because it is less robust
and there is no compelling reason to use it;
it is merely provided for backward compatibility
and it may be removed in future versions.

If the detection mechanism is to be used,
it is mandatory to correctly specify
the filename of the main file as the argument of |\childdocmain|:
%
\begin{center}
\begin{tabular}{l}
|\input{childdoc.def}|\\
|\childdocmain{|\textit{main}|}|\\
\end{tabular}
\end{center}
%
If |\jobname| does not match the argument \textit{main} of |\childdocmain|,
it is assumed that |\jobname| points to the child file to be compiled.
When using |\childdocmain| with the main file specified as argument,
it suffices to start a child file
with just |\input{|\textit{main}|}|
without loading of the package and using |\childdocof|.
If instead all processing is done
with the appropriate \textsf{childdoc} directives,
the argument of \textit{main} of |\childdocmain| can be empty.

An alternative version of the command line processing described
in \secref{sec:commandline} using the detection mechanism reads:
%
\begin{center}
|... -jobname "|\textit{target}|" "|[\textit{flags}]%
[|\def\jobname{|\textit{dest}|}|]|\input{|\textit{main}|}"|
\end{center}

%%%%%%%%%%%%%%%%%%%%%%%%%%%%%%%%%%%%%%%%%%%%%%%%%%%%%%%%%%%%%%%%%%%%%%%%%%%%%%%%
\subsection{Manual Code}
\label{sec:manual}

In case one cannot be certain whether the definitions file |childdoc.def|
is installed on the target \TeX{} distribution
and one prefers not to ship it,
it is conceivable to paste a few relevant commands into the sources.

To that end, drop all statements |\input{childdoc.def}|
and perform the replacements as outlined below.
Instead of |\childdocmain{|\textit{main}|}| add the following code
to the top of the main file:
%
\begin{center}
\begin{tabular}{l}
|\||ifdefined\childdocname\endinput\||fi\newif\ifchilddoc|\\
|\edef\childdocname{\scantokens\expandafter{\jobname\noexpand}}|\\
|\def\childdocmain{|\textit{main}|}\||ifx\childdocmain\childdocname\||else|\\
|\childdoctrue\includeonly{\childdocname}\let\jobname\childdocmain\||fi|\\
\end{tabular}
\end{center}
%
Instead of |\childdocof{|\textit{main}|}| just include the main file
at the top of each child file:
%
\begin{center}
|\input{|\textit{main}|}|
\end{center}
%
A simple redirection |\childdocforward{|\textit{dest}|}| is achieved by:
%
\begin{center}
|\def\jobname{|\textit{dest}|}\input{\jobname}|
\end{center}
%
The redirection with prefix
|\childdocforwardprefix[|\textit{prefix}|]{|\textit{dest}|}|
is accomplished by:
%
\begin{center}
\begin{tabular}{l}
|{\edef\jobname{\scantokens\expandafter{\jobname\noexpand}}|\\
|\def\redirectjob |\textit{prefix}|#1~~~{\gdef\jobname{|\textit{dest}|#1}}|\\
|\expandafter\redirectjob\jobname~~~}\input{\jobname}|
\end{tabular}
\end{center}

In an alternative approach,
child documents can be compiled by a specific command line
without additional code or specific definitions:
%
\begin{center}
|... -jobname "|\textit{target}|" "|[\textit{flags}]%
|\includeonly{|\textit{dest}|}\input{|\textit{main}|}"|
\end{center}
%

%%%%%%%%%%%%%%%%%%%%%%%%%%%%%%%%%%%%%%%%%%%%%%%%%%%%%%%%%%%%%%%%%%%%%%%%%%%%%%%%
%%%%%%%%%%%%%%%%%%%%%%%%%%%%%%%%%%%%%%%%%%%%%%%%%%%%%%%%%%%%%%%%%%%%%%%%%%%%%%%%
\section{Information}

%%%%%%%%%%%%%%%%%%%%%%%%%%%%%%%%%%%%%%%%%%%%%%%%%%%%%%%%%%%%%%%%%%%%%%%%%%%%%%%%
\subsection{Copyright}

Copyright \copyright{} 2017--2018 Niklas Beisert

This work may be distributed and/or modified under the
conditions of the \LaTeX{} Project Public License, either version 1.3
of this license or (at your option) any later version.
The latest version of this license is in
  \url{http://www.latex-project.org/lppl.txt}
and version 1.3 or later is part of all distributions of \LaTeX{}
version 2005/12/01 or later.

This work has the LPPL maintenance status `maintained'.

The Current Maintainer of this work is Niklas Beisert.

This work consists of the files |README.txt|, |childdoc.ins| and |childdoc.dtx|
as well as the derived files |childdoc.def|, |cdocsamp.tex|
with |cdocsch1.tex|, |cdocsch2.tex|, |cdocspt3.tex|, |cdocspt4.tex|,
|cdocsdrf.tex|, |cdocsfn1.tex|, |cdocsfn2.tex|
as well as |childdoc.pdf|.

%%%%%%%%%%%%%%%%%%%%%%%%%%%%%%%%%%%%%%%%%%%%%%%%%%%%%%%%%%%%%%%%%%%%%%%%%%%%%%%%
\subsection{Files and Installation}

The package consists of the files:
%
\begin{center}
\begin{tabular}{ll}
    |README.txt|   & readme file \\
    |childdoc.ins| & installation file \\
    |childdoc.dtx| & source file \\
    |childdoc.def| & definition file \\
    |cdocsamp.tex| & sample main file \\
    |cdocsch1.tex| & sample include file \\
    |cdocsch2.tex| & sample include file \\
    |cdocspt3.tex| & sample part file \\
    |cdocspt4.tex| & sample part file \\
    |cdocsdrf.tex| & sample redirection file \\
    |cdocsfn1.tex| & sample redirection file \\
    |cdocsfn2.tex| & sample redirection file \\
    |childdoc.pdf| & manual
\end{tabular}
\end{center}
%
The distribution consists of the files
|README.txt|, |childdoc.ins| and |childdoc.dtx|.
%
\begin{itemize}
\item
Run (pdf)\LaTeX{} on |childdoc.dtx|
to compile the manual |childdoc.pdf| (this file).
\item
Run \LaTeX{} on |childdoc.ins| to create the definitions file |childdoc.def|
and the sample |cdocsamp.tex| with include files
|cdocsch1.tex|, |cdocsch2.tex|, |cdocspt3.tex|, |cdocspt4.tex|,
|cdocsdrf.tex|, |cdocsfn1.tex|, |cdocsfn2.tex|.
Then copy the file |childdoc.def| to an appropriate directory of your \LaTeX{}
distribution, e.g.\ \textit{texmf-root}|/tex/latex/childdoc|.
\end{itemize}

%%%%%%%%%%%%%%%%%%%%%%%%%%%%%%%%%%%%%%%%%%%%%%%%%%%%%%%%%%%%%%%%%%%%%%%%%%%%%%%%
\subsection{Related CTAN Packages}

There are several other packages which offer a similar functionality:
%
\begin{itemize}
\item
The packages
\href{http://ctan.org/pkg/docmute}{\textsf{docmute}},
\href{http://ctan.org/pkg/includex}{\textsf{includex}} and
\href{http://ctan.org/pkg/standalone}{\textsf{standalone}}
provide commands to include only the document body of
a child file thus allowing both files to be compiled individually.
\item
The packages \href{http://ctan.org/pkg/subdocs}{\textsf{subdocs}}
and \href{http://ctan.org/pkg/subfiles}{\textsf{subfiles}}
provide structures in which the main and child documents can be
encapsulated and allowing them to be compiled individually.
The inclusion mechanism is different from the conventional |\include|.
\item
The package \href{http://ctan.org/pkg/combine}{\textsf{combine}}
is an elaborate solution to combine several documents into one.
\end{itemize}
%
See also the CTAN topic \href{http://ctan.org/topic/subdocs}{\textsf{subdocs}}
for further related packages.
The present package differs from the above solutions in that
a document structure constructed with the conventional |\include| mechanism
just needs two extra commands at the top of every file
such that all constituent files can be compiled individually.

%%%%%%%%%%%%%%%%%%%%%%%%%%%%%%%%%%%%%%%%%%%%%%%%%%%%%%%%%%%%%%%%%%%%%%%%%%%%%%%%
%\subsection{Feature Suggestions}
%
%The following is a list of features which may be useful for future
%versions of this package:
%%
%\begin{itemize}
%\item
%\ldots
%\end{itemize}

%%%%%%%%%%%%%%%%%%%%%%%%%%%%%%%%%%%%%%%%%%%%%%%%%%%%%%%%%%%%%%%%%%%%%%%%%%%%%%%%
\subsection{Revision History}

%%%%%%%%%%%%%%%%%%%%%%%%%%%%%%%%%%%%%%%%
\paragraph{v2.0:} 2018/12/30

\begin{itemize}
\item
immediate forward processing
\item
added |\childdocby| mechanism
\item
manual restructured
\end{itemize}

%%%%%%%%%%%%%%%%%%%%%%%%%%%%%%%%%%%%%%%%
\paragraph{v1.6:} 2018/01/17

\begin{itemize}
\item
application for development of include files
\item
corrections to manual
\end{itemize}

%%%%%%%%%%%%%%%%%%%%%%%%%%%%%%%%%%%%%%%%
\paragraph{v1.5:} 2017/05/21

\begin{itemize}
\item
more complete structuring introduced
\item
|\childdocof| introduced
\item
|\childdoc| renamed to |\childdocmain|
\item
|\childredirect| renamed to |\childdocforward| and |\childdocforwardprefix|
and functionality expanded
\end{itemize}

%%%%%%%%%%%%%%%%%%%%%%%%%%%%%%%%%%%%%%%%
\paragraph{v1.0:} 2017/04/27

\begin{itemize}
\item
manual and install package
\item
first version published on CTAN
\end{itemize}

%%%%%%%%%%%%%%%%%%%%%%%%%%%%%%%%%%%%%%%%
\paragraph{v0.6:} 2017/04/26

\begin{itemize}
\item
redirection mechanism added
\end{itemize}

%%%%%%%%%%%%%%%%%%%%%%%%%%%%%%%%%%%%%%%%
\paragraph{v0.5:} 2017/04/26

\begin{itemize}
\item
functionality in definition file
\end{itemize}


%%%%%%%%%%%%%%%%%%%%%%%%%%%%%%%%%%%%%%%%%%%%%%%%%%%%%%%%%%%%%%%%%%%%%%%%%%%%%%%%
%%%%%%%%%%%%%%%%%%%%%%%%%%%%%%%%%%%%%%%%%%%%%%%%%%%%%%%%%%%%%%%%%%%%%%%%%%%%%%%%
%%%%%%%%%%%%%%%%%%%%%%%%%%%%%%%%%%%%%%%%%%%%%%%%%%%%%%%%%%%%%%%%%%%%%%%%%%%%%%%%
\appendix

\settowidth\MacroIndent{\rmfamily\scriptsize 000\ }

 \DocInput{childdoc.dtx}

\end{document}
%</driver>
% \fi
%
% %%%%%%%%%%%%%%%%%%%%%%%%%%%%%%%%%%%%%%%%%%%%%%%%%%%%%%%%%%%%%%%%%%%%%%%%%%%%%%
% %%%%%%%%%%%%%%%%%%%%%%%%%%%%%%%%%%%%%%%%%%%%%%%%%%%%%%%%%%%%%%%%%%%%%%%%%%%%%%
% \section{Sample}
%\iffalse
%<*samplemain>
%\fi
%
% The following presents a sample document
% with two chapters, two parts, a title page,
% a compile flag as well as three forwarding files to set the flag.
% It consists of eight |.tex| files:
% \begin{center}
% \begin{tabular}{ll}
% |cdocsamp.tex|&main file\\
% |cdocsch1.tex|&include file for chapter 1\\
% |cdocsch2.tex|&include file for chapter 2\\
% |cdocspt3.tex|&include file for part 3\\
% |cdocspt4.tex|&include file for part 4\\
% |cdocsdrf.tex|&forwarding file for main file in draft mode\\
% |cdocsfi1.tex|&forwarding file for final version of chapter 1\\
% |cdocsfi2.tex|&forwarding file for final version of chapter 2\\
% \end{tabular}
% \end{center}
% Each of the eight files can be compiled directly by the \LaTeX{} compiler.
%
% %%%%%%%%%%%%%%%%%%%%%%%%%%%%%%%%%%%%%%
% \paragraph{Main File.}
%
% The main file is called |cdocsamp.tex|.
%
% Load the \textsf{childdoc} definitions and
% declare the filename for the main document:
%    \begin{macrocode}
\input{childdoc.def}
\childdocmain{}
%    \end{macrocode}

% Optional override for |\version| flag:
%    \begin{macrocode}
%%\ifchilddoc\else\providecommand{\version}{draft}\fi
%    \end{macrocode}

% Define the default values for the |\version| flag
% (|final| for the main file and |draft| for childs):
%    \begin{macrocode}
\ifchilddoc
\providecommand{\version}{draft}
\else
\providecommand{\version}{final}
\fi
%    \end{macrocode}

% Load the standard document class:
%    \begin{macrocode}
\documentclass[12pt]{article}
%    \end{macrocode}

% Start the document body:
%    \begin{macrocode}
\begin{document}
%    \end{macrocode}

% Declare a title page.
% Print title, part of document being processed and version flag:
%    \begin{macrocode}
\addtocounter{page}{-1}
\begin{center}
{\LARGE\bfseries{}childdoc example\par}
\vspace{1cm}
\ifchilddoc
\ifchilddocmanual part\else chapter\fi:
`\childdocname' of `\childdocjob'\par
\else
main document: `\childdocjob'\par
\fi
version: \version\par
\end{center}
\newpage
%    \end{macrocode}

% Manually include selected file,
% otherwise process as usual:
%    \begin{macrocode}
\ifchilddocmanual
\section*{part `\childdocname'}
\input{\childdocname}
\else
%    \end{macrocode}

% Include the two chapters:
%    \begin{macrocode}
\include{cdocsch1}
\include{cdocsch2}
%    \end{macrocode}

% Include the two parts unless only chapters should be displayed:
%    \begin{macrocode}
\ifchilddoc\else
\section{part three}
\input{cdocspt3}
\section{part four}
\input{cdocspt4}
\fi
%    \end{macrocode}

% Process as usual until here:
%    \begin{macrocode}
\fi
%    \end{macrocode}

% End of document body:
%    \begin{macrocode}
\end{document}
%    \end{macrocode}
%\iffalse
%</samplemain>
%\fi
%
% %%%%%%%%%%%%%%%%%%%%%%%%%%%%%%%%%%%%%%
% \paragraph{Chapter Include Files.}
%
% The include files are called |cdocsch1.tex| and |cdocsch2.tex|.
%
%\iffalse
%<*samplechap1|samplechap2>
%\fi

% Optional override for |\version| flag:
%    \begin{macrocode}
%%\providecommand{\version}{final}
%    \end{macrocode}

% Include the main document:
%    \begin{macrocode}
\input{childdoc.def}
\childdocof{cdocsamp}
%    \end{macrocode}

%\iffalse
%</samplechap1|samplechap2>
%\fi
%
%\iffalse
%<*samplechap1>
%\fi
% Some text for chapter 1:
%    \begin{macrocode}
\section{one}
some text in chapter one
%    \end{macrocode}

%\iffalse
%</samplechap1>
%\fi
% Some text for chapter 2:
%\iffalse
%<*samplechap2>
%\fi
%    \begin{macrocode}
\section{two}
more text in chapter two
%    \end{macrocode}

%\iffalse
%</samplechap2>
%\fi
%
% %%%%%%%%%%%%%%%%%%%%%%%%%%%%%%%%%%%%%%
% \paragraph{Part Include Files.}
%
% The include files are called |cdocspt3.tex| and |cdocspt4.tex|.
%
%\iffalse
%<*samplepart3|samplepart4>
%\fi

% Optional override for |\version| flag:
%    \begin{macrocode}
%%\providecommand{\version}{final}
%    \end{macrocode}

% Include the main document:
%    \begin{macrocode}
\input{childdoc.def}
\childdocby{cdocsamp}
%    \end{macrocode}

%\iffalse
%</samplepart3|samplepart4>
%\fi
%
%\iffalse
%<*samplepart3>
%\fi
% Some text for part 3:
%    \begin{macrocode}
some text in part three
%    \end{macrocode}

%\iffalse
%</samplepart3>
%\fi
% Some text for part 4:
%\iffalse
%<*samplepart4>
%\fi
%    \begin{macrocode}
more text in part four
%    \end{macrocode}

%\iffalse
%</samplepart4>
%\fi
%
% %%%%%%%%%%%%%%%%%%%%%%%%%%%%%%%%%%%%%%
% \paragraph{Forwarding for a Complete Draft.}
%
% The following forwarding file |cdocsdrf.tex|
% compiles the main document in draft mode:
%\iffalse
%<*sampledraft>
%\fi
%    \begin{macrocode}
\def\version{draft}
\input{childdoc.def}
\childdocforward{cdocsamp}
%    \end{macrocode}

%\iffalse
%</sampledraft>
%\fi
%
% %%%%%%%%%%%%%%%%%%%%%%%%%%%%%%%%%%%%%%
% \paragraph{Forwarding for Final Version of the Chapters.}
%
% The following forwarding files |cdocsfn1.tex| and |cdocsfn2.tex|
% (with identical content)
% compile the final versions of the child documents
% |cdocsch1.tex| and |cdocsch2.tex|, respectively:
%\iffalse
%<*samplefinal>
%\fi
%    \begin{macrocode}
\def\version{final}
\input{childdoc.def}
\childdocforwardprefix[cdocsamp]{cdocsfn}{cdocsch}
%    \end{macrocode}

%\iffalse
%</samplefinal>
%\fi
%
% %%%%%%%%%%%%%%%%%%%%%%%%%%%%%%%%%%%%%%
% \paragraph{Command Line Processing.}
%
% The following three command lines generate the output files
% |cdocscld|, |cdocscl1| and |cdocscl2|
% which should be identical to
% |cdocsdrf|, |cdocsch1| and |cdocsfn2|, respectively:
% \begin{center}
% \begin{tabular}{l}
% |latex -jobname cdocscld \|\\
% |  "\def\version{draft}\input{childdoc.def}\childdocforward{cdocsamp}"|\\
% |latex -jobname cdocscl1 \|\\
% |  "\input{childdoc.def}\childdocforward[cdocsamp]{cdocsch1}"|\\
% |latex -jobname cdocscl2 \|\\
% |  "\def\version{final}\input{childdoc.def}\childdocforward{cdocsch2}"|
% \end{tabular}
% \end{center}
% Note that the trailing backslash on each first line
% merely continues the input to the second line
% (for convenient cut ant paste).
% Furthermore, the command |latex| can be replaced by any
% of its alternative versions such as |pdflatex|.
%
% %%%%%%%%%%%%%%%%%%%%%%%%%%%%%%%%%%%%%%%%%%%%%%%%%%%%%%%%%%%%%%%%%%%%%%%%%%%%%%
% %%%%%%%%%%%%%%%%%%%%%%%%%%%%%%%%%%%%%%%%%%%%%%%%%%%%%%%%%%%%%%%%%%%%%%%%%%%%%%
% \section{Implementation}
%\iffalse
%<*package>
%\fi
%
% This section describes the definitions file |childdoc.def|.

% The definitions cannot be loaded using |\usepackage| or |\RequirePackage|
% which has a mechanism to prevent loading a style file more than once.
% When loading the definitions by means of |\input|
% multiple instances have to be prevented manually:
%\iffalse
%This code needs to be before the `\ProvidesFile' directive
%which is defined at the beginning of this file.
%Therefore it is also placed there and commented out here.
%</package>
%<*discard>
%\fi
%    \begin{macrocode}
\ifdefined\childdocmain\endinput\fi
%    \end{macrocode}
%\iffalse
%</discard>
%<*package>
%\fi
%
% \macro{\ifchilddoc}
% \macro{\ifchilddocmanual}
% The conditional |\ifchilddoc| tells whether a
% child (true) or main (false) document is being compiled.
% The conditional |\ifchilddocmanual| tells whether
% the |\includeonly| mechanism is used (false) or
% the selection of child files must be performed manually (true).
% The definitions initialise to false:
%    \begin{macrocode}
\newif\ifchilddoc
\newif\ifchilddocmanual
%    \end{macrocode}

% \macro{\childdocname}
% \macro{\childdocjob}
% The macro |\childdocname| stores the name of the main document
% to be compiled. The macro |\childdocjob| stores the name of
% the document on which the \LaTeX{} compiler was originally invoked.
% The content of |\jobname| cannot be compared
% to filenames specified in the source due to different catcodes.
% The following code rescans |\jobname|, stores the result
% in |\childdocname| and saves a copy in |\childdocjob|:
%    \begin{macrocode}
\edef\childdocname{\scantokens\expandafter{\jobname\noexpand}}
\let\childdocjob\childdocname
%    \end{macrocode}

% \macro{\childdocdisable}
% The macro |\childdocdisable| prevents the main file
% from being processed more than once.
% At this stage, the main document command |\childdocmain|
% is assumed to be called once again where it should do nothing.
% Any subsequent call to it should prevent
% a secondary processing of the main document
% It overwrites the forwarding commands
% |\childdocof| and |\childdocforward|
% with empty macros to prevent further inclusions of the main document:
%    \begin{macrocode}
\newcommand{\childdocdisable}
{
  \renewcommand{\childdocmain}[1]{\renewcommand{\childdocmain}[1]{\endinput}}
  \renewcommand{\childdocof}[1]{}
  \renewcommand{\childdocby}[2][]{}
  \renewcommand{\childdocforward}[2][]{}
  \renewcommand{\childdocdisable}{}
}
%    \end{macrocode}

% \macro{\childdocmain}
% The macro |\childdocmain| is to be called at the top of the main file
% with nothing or the main filename (without extension) as argument.
% First, it breaks loops.
% If the argument is not empty and does not match |\childdocname|
% (which is set by the first inclusion of |childdoc.def|),
% |\ifchilddoc| is set to true, |\includeonly| is applied to the child file
% and |\jobname| is set to the main file
% (for proper handling of |.aux| files):
%    \begin{macrocode}
\newcommand{\childdocmain}[1]
{
  \childdocdisable\childdocmain{}
  \if?#1?\else
    \begingroup
      \def\childdoctmp{#1}
      \ifx\childdoctmp\childdocname
        \def\childdoctmp{}
      \else
        \def\childdoctmp
        {
          \childdoctrue
          \includeonly{\childdocname}
          \def\childdocjob{#1}
          \def\jobname{#1}
        }
      \fi
      \expandafter
    \endgroup
    \childdoctmp
  \fi
}
%    \end{macrocode}

% \macro{\childdocof}
% The command |\childdocof| redirects
% compilation to the main file |#1|.
%    \begin{macrocode}
\newcommand{\childdocof}[1]
{
  \childdocdisable
  \childdoctrue
  \includeonly{\childdocname}
  \def\jobname{#1}
  \def\childdocjob{#1}
  \input{#1}
}
%    \end{macrocode}

% \macro{\childdocby}
% The command |\childdocby| ....
%    \begin{macrocode}
\newcommand{\childdocby}[2][]
{
  \childdocdisable
  \childdoctrue
  \childdocmanualtrue
  \if?#1?\else
    \def\jobname{#2}
  \fi
  \def\childdocjob{#2}
  \input{#2}
  \endinput
}
%    \end{macrocode}

% \macro{\childdocforward}
% The command |\childdocforward| redirects
% compilation to the main file or
% (if the optional argument is given) a child file.
% Parameters are set as if the main file
% or a child file starting with |\childdocof| was compiled.
% Then compilation is handed over to the main file:
%    \begin{macrocode}
\newcommand{\childdocforward}[2][]
{
  \begingroup
    \if?#1?
      \def\childdoctmp
      {
        \def\childdocname{#2}
        \def\childdocjob{#2}
        \def\jobname{#2}
        \input{#2}
        \endinput
      }
    \else
      \def\childdoctmp
      {
        \childdocdisable
        \def\childdocname{#2}
        \childdoctrue
        \includeonly{#2}
        \def\childdocjob{#1}
        \def\jobname{#1}
        \input{#1}
        \endinput
      }
    \fi
    \expandafter
  \endgroup
  \childdoctmp
}
%    \end{macrocode}

% \macro{\childdocforwardprefix}
% The command |\childdocforwardprefix| redirects
% compilation to the main or a child file by means of a pattern.
% The prefix |#1| in the current filename is replaced by |#2|
% and the suffix of the current filename is kept
% (it is assumed that the filename does not contain the substring `|~~~|'
% which is used as a delimiter).
% Compilation is handed over to the new file by |\childdocforward|:
%    \begin{macrocode}
\newcommand{\childdocforwardprefix}[3][]
{
  \begingroup
    \def\childdocextract #2##1~~~{\def\childdoctmp{\childdocforward[#1]{#3##1}}}
    \expandafter\childdocextract\childdocname~~~
    \expandafter
  \endgroup
  \childdoctmp
}
%    \end{macrocode}

% \macro{\childdoc}
% The deprecated macro |\childdoc| is a legacy version of |\childdocmain|:
%    \begin{macrocode}
\newcommand{\childdoc}{\childdocmain}
%    \end{macrocode}

% \macro{\childdocredirect}
% The deprecated macro |\childdocredirect| is a legacy version
% of |\childdocforward| and |\childdocforwardprefix|:
%    \begin{macrocode}
\newcommand{\childdocredirect}[2][]
{
  \begingroup
    \if?#1?
      \def\childdoctmp{\childdocforward{#2}}
    \else
      \def\childdoctmp{\childdocforwardprefix{#1}{#2}}
    \fi
    \expandafter
  \endgroup
  \childdoctmp
}
%    \end{macrocode}

%\iffalse
%</package>
%\fi
%
\endinput
|\\
|\childdocforward[|\textit{main}|]{|\textit{dest}|}|\\
\end{tabular}
\end{center}
%
The argument \textit{dest} is the destination file
(without extension).
It should be the main file or one of the child files.
Note that further \textsf{childdoc} directives
such as |\childdocof| and |\childdocforward|
in the indicated file will be processed in this form.
The optional argument \textit{main}
passes on directly to the main file \textit{main}
while pretending to compile the child \textit{dest}.
This form behaves as if \textit{dest}
issues |\childdocof{|\textit{main}|}| right away,
and no further \textsf{childdoc} directives will be processed.

%%%%%%%%%%%%%%%%%%%%%%%%%%%%%%%%%%%%%%%%
\DescribeMacro{\...prefix}
In the alternative form |\childdocforwardprefix|,
%
\begin{center}
\begin{tabular}{l}
|% \iffalse
%
% childdoc.dtx Copyright (C) 2017-2018 Niklas Beisert
%
% This work may be distributed and/or modified under the
% conditions of the LaTeX Project Public License, either version 1.3
% of this license or (at your option) any later version.
% The latest version of this license is in
%   http://www.latex-project.org/lppl.txt
% and version 1.3 or later is part of all distributions of LaTeX
% version 2005/12/01 or later.
%
% This work has the LPPL maintenance status `maintained'.
%
% The Current Maintainer of this work is Niklas Beisert.
%
% This work consists of the files childdoc.dtx and childdoc.ins
% and the derived files childdoc.def and cdocsamp.tex with
% cdocsch1.tex, cdocsch2.tex, cdocsdrf.tex, cdocsfn1.tex, cdocsfn2.tex.
%
%<package>\ifdefined\childdocmain\endinput\fi
%<package>\ProvidesFile{childdoc.def}[2018/12/30 v2.0 child document driver]
%<samplemain>\ProvidesFile{cdocsamp.tex}[2018/12/30 v2.0 sample for childdoc]
%<*driver>
%\ProvidesFile{childdoc.drv}[2018/12/30 v2.0 childdoc reference manual file]
\PassOptionsToClass{10pt,a4paper}{article}
\documentclass{ltxdoc}

\usepackage[margin=35mm]{geometry}
\usepackage{hyperref}
\usepackage{hyperxmp}
\usepackage[usenames]{color}

\hypersetup{colorlinks=true}
\hypersetup{pdfstartview=FitH}
\hypersetup{pdfpagemode=UseNone}
\hypersetup{pdfsource={}}
\hypersetup{pdflang={en-UK}}
\hypersetup{pdfcopyright={Copyright 2017-2018 Niklas Beisert.
  This work may be distributed and/or modified under the
  conditions of the LaTeX Project Public License, either version 1.3
  of this license or (at your option) any later version.}}
\hypersetup{pdflicenseurl={http://www.latex-project.org/lppl.txt}}
\hypersetup{pdfcontactaddress={ETH Zurich, ITP, HIT K,
  Wolfgang-Pauli-Strasse 27}}
\hypersetup{pdfcontactpostcode={8093}}
\hypersetup{pdfcontactcity={Zurich}}
\hypersetup{pdfcontactcountry={Switzerland}}
\hypersetup{pdfcontactemail={nbeisert@itp.phys.ethz.ch}}
\hypersetup{pdfcontacturl={http://people.phys.ethz.ch/\xmptilde nbeisert/}}

\newcommand{\secref}[1]{\hyperref[#1]{section \ref*{#1}}}

\parskip1ex
\parindent0pt
\let\olditemize\itemize
\def\itemize{\olditemize\parskip0pt}

\begin{document}

\title{The \textsf{childdoc} Package}
\hypersetup{pdftitle={The childdoc Package}}
\author{Niklas Beisert\\[2ex]
  Institut f\"ur Theoretische Physik\\
  Eidgen\"ossische Technische Hochschule Z\"urich\\
  Wolfgang-Pauli-Strasse 27, 8093 Z\"urich, Switzerland\\[1ex]
  \href{mailto:nbeisert@itp.phys.ethz.ch}
  {\texttt{nbeisert@itp.phys.ethz.ch}}}
\hypersetup{pdfauthor={Niklas Beisert}}
\hypersetup{pdfsubject={Manual for the LaTeX2e Package childdoc}}
\date{30 December 2018, \textsf{v2.0}}
\maketitle

\begin{abstract}\noindent
\textsf{childdoc} is a \LaTeXe{} package
that enables the direct compilation
of document sections included by |\include|
to individual files.
\end{abstract}

\begingroup
\parskip0ex
\tableofcontents
\endgroup

%%%%%%%%%%%%%%%%%%%%%%%%%%%%%%%%%%%%%%%%%%%%%%%%%%%%%%%%%%%%%%%%%%%%%%%%%%%%%%%%
%%%%%%%%%%%%%%%%%%%%%%%%%%%%%%%%%%%%%%%%%%%%%%%%%%%%%%%%%%%%%%%%%%%%%%%%%%%%%%%%
\section{Introduction}

\LaTeX{} provides a mechanism to structure a large document (such as a book)
into a main file and several child files (containing the chapters)
using the |\include| command.
This mechanism is beneficial for documents
which span hundreds of pages in order to
make the source file(s) more manageable.
Moreover, compilation can be restricted to
selected child files by means of the |\includeonly| command.
The latter feature can be used to reduce the compilation time while editing
(this was significantly more useful in the earlier days of \LaTeX{})
or to generate a smaller document which is easier to navigate.
Another application of |\includeonly| is to generate
documents consisting of selected parts of the complete document.

However, there are a few drawbacks of the plain |\include| mechanism:
\begin{itemize}
\item
The child files cannot be compiled on their own,
they can only be compiled via the main file.
A naive editing environment
(such as a text editor with an option
to have the current file processed by \LaTeX)
may require one to switch to the main file before compiling;
attempting to compile the child file produces errors.
\item
The main file must be modified (each time)
to adjust the |\includeonly| command
to the present needs. This easily leaves the main file in a messy state.
\item
The generated document will always carry the filename
of the main document. This is inconvenient if
several child files are to be compiled and
to be kept for distribution.
\end{itemize}

The present package provides a simple interface
to make child files individually compilable by \LaTeX{}.
Compiling a child file then has the same effect as compiling
the main file with an |\includeonly| command
to select the appropriate child.
Moreover the generated document will carry the name of the child
rather than the main file.
This resolves all three above issues.

This feature is meant to make the editing of books,
thesis documents and lecture notes somewhat more convenient.
However, the package can also be used efficiently for
composing a series of documents (such as exercise sheets)
which are typically distributed individually.
It then assists the author in generating the individual documents
(potentially in different versions)
as well as a document containing the collected series.
Another application is in developing style files
or other kinds of included material
where compilation of the style file could redirect
to a sample or test file.

%%%%%%%%%%%%%%%%%%%%%%%%%%%%%%%%%%%%%%%%%%%%%%%%%%%%%%%%%%%%%%%%%%%%%%%%%%%%%%%%
%%%%%%%%%%%%%%%%%%%%%%%%%%%%%%%%%%%%%%%%%%%%%%%%%%%%%%%%%%%%%%%%%%%%%%%%%%%%%%%%
\section{Usage}

First of all, the package \textsf{childdoc} is \emph{not} a standard
\LaTeXe{} |.sty| style file! Therefore it needs to be invoked in
a non-standard way.

%%%%%%%%%%%%%%%%%%%%%%%%%%%%%%%%%%%%%%%%%%%%%%%%%%%%%%%%%%%%%%%%%%%%%%%%%%%%%%%%
\subsection{Included Files}
\label{sec:include}

%%%%%%%%%%%%%%%%%%%%%%%%%%%%%%%%%%%%%%%%
\DescribeMacro{\childdocmain}
To use the package, add the commands
\begin{center}
\begin{tabular}{l}
|\input{childdoc.def}|\\
|\childdocmain{}|\\
\end{tabular}
\end{center}
at the very top of the main \LaTeX{} file,
in particular \emph{before} the |\documentclass| statement!
The argument of |\childdocmain| should be left empty
(but it must be present).

%%%%%%%%%%%%%%%%%%%%%%%%%%%%%%%%%%%%%%%%
\DescribeMacro{\childdocof}
Furthermore, add the commands
\begin{center}
\begin{tabular}{l}
|\input{childdoc.def}|\\
|\childdocof{|\textit{main}|}|\\
\end{tabular}
\end{center}
at the top of every child file \textit{child}
which is included by |\include{|\textit{child}|}|
from within the main file
(or at least for those files to be compiled individually).
The argument \textit{main} must be the filename of the main file.

There are a couple of
considerations in setting up the main and child documents:

%%%%%%%%%%%%%%%%%%%%%%%%%%%%%%%%%%%%%%%%
\paragraph{Restrictions.}

Please note the following restrictions:
\begin{itemize}
\item
|\childdocmain| must be called with one argument \textit{main}
to ensure compatibility with earlier version of the package.
It must either be empty (|\childdocmain{}|)
or precisely match the filename of the main file in which it is specified.
See \secref{sec:detection} for further information.
\item
The filename \textit{main} must be specified without the |.tex| extension.
\item
The filename \textit{main} is case sensitive
(even in case-insensitive file systems)
due to internal string comparison.
\item
The argument \textit{main} should be fully expanded, it cannot be a macro.
\item
Subdirectories and special characters should be avoided in filenames.
\item
The command |\childdocmain{|\textit{main}|}| must be followed by a whitespace.
It should not be followed immediately by another command
or by a comment mark `|%|'.
This is because the \TeX{} parser reads the token immediately following
the argument of |\childdocmain| and puts it
at the beginning of every child section;
however, a white\-space is ignored.
\end{itemize}

%%%%%%%%%%%%%%%%%%%%%%%%%%%%%%%%%%%%%%%%
\paragraph{Content of Main File.}

It is advisable to place all content in the child files included by |\include|.
Any output contained in the main file will appear in all child documents
unless suppressed manually;
it cannot be suppressed automatically by the |\includeonly| directive
and thus should normally be avoided.
A method to include some content in the main file
by means of conditional processing is described in \secref{sec:conditional}.

%%%%%%%%%%%%%%%%%%%%%%%%%%%%%%%%%%%%%%%%
\paragraph{Page Numbering.}

When only a part of the document is compiled,
the appropriate numbering of pages
(as well as other status parameters)
is determined from the |.aux| files.
The latter contain information from previous passes.
However this information needs to propagate through
all intermediate child documents.
Therefore the page numbering in child documents may well
be inconsistent until the complete document is compiled at least once.

A useful (if unconventional) way to always ensure a consistent
page numbering is to restart the numbering in each child document
and denote the pages by `\textit{child}|.|\textit{page}'
where \textit{child} represents the chapter/section number of the child file.
This can be achieved by the command
|\numberwithin{page}{|\textit{child}|}|
of the \textsf{amsmath} package
where \textit{child} can be |chapter| or |section|
depending on the chosen structuring.
Alternatively, one can modify the macro |\thepage| appropriately
and reset the counter |page| at the start of each child file.

%%%%%%%%%%%%%%%%%%%%%%%%%%%%%%%%%%%%%%%%%%%%%%%%%%%%%%%%%%%%%%%%%%%%%%%%%%%%%%%%
\subsection{Conditional Processing}
\label{sec:conditional}

The package provides a mechanism to compile different versions
of a document. To customise the versions further some conditional processing
can come in handy to distinguish which version is being compiled.
The package provides two macros to describe the compilation context:

%%%%%%%%%%%%%%%%%%%%%%%%%%%%%%%%%%%%%%%%
\DescribeMacro{\ifchilddoc}
The conditional |\ifchilddoc| distinguishes between the compilation of
child documents and the main document:
%
\begin{center}
|\ifchilddoc |\textit{child-code}| |[|\||else |\textit{main-code}]| \||fi|
\end{center}

%%%%%%%%%%%%%%%%%%%%%%%%%%%%%%%%%%%%%%%%
\DescribeMacro{\childdocname}
\DescribeMacro{\childdocjob}
The macro |\childdocname| contains the filename (without extension)
of the main or child file being processed.
Note that |\childdocjob| will always contain the name of the main file.

%%%%%%%%%%%%%%%%%%%%%%%%%%%%%%%%%%%%%%%%
\paragraph{Title Page.}

Conditional processing can be used to include a title or banner page
in the main document when proper precautions are taken.
Importantly, the code in the main file should ensure that the page counter
(as well as other status parameters which are stored in the |.aux| files)
takes the same value after the conditional processing.
Otherwise the page numbers may take divergent values
depending on which part is compiled.

For example, a title page could be declared by:
%
\begin{center}
\begin{tabular}{l}
|\ifchilddoc\||else|\\
|\addtocounter{page}{-1}|\\
\textit{code for title page}\\
|\newpage|\\
|\||fi|
\end{tabular}
\end{center}
%
A banner page for the child documents can be generated by:
%
\begin{center}
\begin{tabular}{l}
|\ifchilddoc|\\
|\addtocounter{page}{-1}|\\
\textit{code for banner page}\\
|\newpage|\\
|\||fi|
\end{tabular}
\end{center}
%
Here one could write a message such as:
\begin{center}
|This is the part \childdocname{} of \childdocjob{}.|
\end{center}

%%%%%%%%%%%%%%%%%%%%%%%%%%%%%%%%%%%%%%%%%%%%%%%%%%%%%%%%%%%%%%%%%%%%%%%%%%%%%%%%
\subsection{Flags}
\label{sec:flags}

The package makes it easy to generate different versions
of the main or child documents.
To this end compilation flags can be defined
and assigned different default values.
They will be particularly useful in conjunction
with the forwarding mechanism described in \secref{sec:forward}.

For example, it may be useful to have a flag |\version|
which can be set to |draft| or |final|.
The document source will contain some conditional code
depending on the value of |\version|.
Suppose further, the flag should default to |final| for the main file
and to |draft| for child files
which is a natural assignment for editing the document.
This is achieved by placing the following code
in the preamble of the main document
(below the |\childdocmain| directive):
%
\begin{center}
\begin{tabular}{l}
|\ifchilddoc|\\
|\providecommand{\version}{draft}|\\
|\||else|\\
|\providecommand{\version}{final}|\\
|\||fi|
\end{tabular}
\end{center}
%
The definition by |\providecommand| makes sure
that previous definitions are not overwritten.
Further statements |\providecommand{\version}{...}|
can thus be added before the above code to override it.

For the main file, one might add a line
(between |\childdocmain| and the above block)
%
\begin{center}
|%\ifchilddoc\||else\providecommand{\version}{draft}\||fi|
\end{center}
%
which can be uncommented to produce a draft version.
Likewise one can add a line to the very top of a child file
(above the |\childdocof{|\textit{main}|}| directive)
%
\begin{center}
|%\providecommand{\version}{final}|
\end{center}
%
which can be uncommented to produce the final version of this child document.

%%%%%%%%%%%%%%%%%%%%%%%%%%%%%%%%%%%%%%%%%%%%%%%%%%%%%%%%%%%%%%%%%%%%%%%%%%%%%%%%
\subsection{Forwarding}
\label{sec:forward}

Different versions of the main or child documents
using compilation flags as described in \secref{sec:flags}
can be (permanently) stored in different files
for convenient compilation, viewing and distribution.
To this end, the package defines a command
to pass on compilation to a different file:

%%%%%%%%%%%%%%%%%%%%%%%%%%%%%%%%%%%%%%%%
\DescribeMacro{\childdocforward}
The command |\childdocforward| redirects processing to
another source file:
%
\begin{center}
\begin{tabular}{l}
|\input{childdoc.def}|\\
|\childdocforward[|\textit{main}|]{|\textit{dest}|}|\\
\end{tabular}
\end{center}
%
The argument \textit{dest} is the destination file
(without extension).
It should be the main file or one of the child files.
Note that further \textsf{childdoc} directives
such as |\childdocof| and |\childdocforward|
in the indicated file will be processed in this form.
The optional argument \textit{main}
passes on directly to the main file \textit{main}
while pretending to compile the child \textit{dest}.
This form behaves as if \textit{dest}
issues |\childdocof{|\textit{main}|}| right away,
and no further \textsf{childdoc} directives will be processed.

%%%%%%%%%%%%%%%%%%%%%%%%%%%%%%%%%%%%%%%%
\DescribeMacro{\...prefix}
In the alternative form |\childdocforwardprefix|,
%
\begin{center}
\begin{tabular}{l}
|\input{childdoc.def}|\\
|\childdocforwardprefix[|\textit{main}|]{|\textit{prefix}|}{|\textit{dest}|}|
\end{tabular}
\end{center}
%
the destination file is determined by a pattern
depending on the current file:
To make this work, the current file must be called
`{\textit{prefix}\hspace{0.2em}\textit{suffix}}'
with \textit{prefix} matching precisely the argument.
Processing is then passed on to the file
`{\textit{dest}\hspace{0.2em}\textit{suffix}}'.
Surely, the same effect is achieved by
directly specifying the
argument `{\textit{dest}\hspace{0.2em}\textit{suffix}}'
in the first form.
However, that requires to set up a different file
for each child. With the alternative form of the command
all these files can have exactly the same content
which simplifies setting them up and maintaining them.

For example, the following file |draft.tex|
with a compilation flag |\version| as described in \secref{sec:flags}
compiles the main document as a draft:
%
\begin{center}
\begin{tabular}{l}
|\def\version{draft}|\\
|\input{childdoc.def}|\\
|\childdocforward{|\textit{main}|}|
\end{tabular}
\end{center}
%
Likewise, the following files |final|\textit{nn}|.tex|
compile the final version of the child document
|child|\textit{nn}|.tex|:
%
\begin{center}
\begin{tabular}{l}
|\def\version{final}|\\
|\input{childdoc.def}|\\
|\childdocforwardprefix{final}{child}|
\end{tabular}
\end{center}
%

Note that when several versions of a main file and/or of each child file
are to be generated, it may be convenient to set up a |Makefile| or
shell script to automatise the process.

%%%%%%%%%%%%%%%%%%%%%%%%%%%%%%%%%%%%%%%%%%%%%%%%%%%%%%%%%%%%%%%%%%%%%%%%%%%%%%%%
\subsection{Command Line Processing}
\label{sec:commandline}

The effect of redirection files can also be achieved by invoking
the \LaTeX{} compiler with a more elaborate command line.
Most conveniently this should be done as part
of a shell script or a |Makefile|.

When using \textsf{childdoc} in the main file, the following
command lines effectively perform a redirection
(note that depending on the shell being used,
backslashes may have to be doubled: `|\|' $\to$ `|\\|'):
%
\begin{center}
|... -jobname "|\textit{target}|" |\\|"|[\textit{flags}]%
|\input{childdoc.def}\childdocforward[|\textit{main}|]{|\textit{dest}|}"|
\end{center}
%
Here \textit{target} is the name of the output file,
\textit{main} is the name of the main file
and \textit{dest} is the name of the main or child file to be processed
(all filenames without extensions).
The optional argument \textit{main} can be omitted
if \textit{main} matches \textit{dest}.
Optionally, compilation \textit{flags} can be defined via |\def| commands.
This command line makes the \TeX{} engine believe
it is compiling the file \textit{target}
whose content is specified as the latter parameter.
The provided code then forwards the processing to
\textit{main} or \textit{dest} as described in \secref{sec:forward}.

%%%%%%%%%%%%%%%%%%%%%%%%%%%%%%%%%%%%%%%%%%%%%%%%%%%%%%%%%%%%%%%%%%%%%%%%%%%%%%%%
\subsection{Include by Input}
\label{sec:input}

Including child documents by |\include| has some restrictions by design.
Most notably, the content of a child document always occupies
its own set of pages; pages cannot be shared between child documents.
Usually, this behaviour makes perfect sense
because each child document contain an essential part of the document.
However, in some situations it may be desirable to compose
a document from a collection of parts
without having mandatory page breaks between then.
For this case, the package
provides a mechanism to include parts
by |\input| which can also be processed individually.
However, by construction this mechanism
requires manual handling of the content to be output.

%%%%%%%%%%%%%%%%%%%%%%%%%%%%%%%%%%%%%%%%
\DescribeMacro{\ifchilddocmanual}
The main file should be prepared as usual, see \secref{sec:include}.
However, the document body must make a distinction
between processing of an individual part and of the main document, e.g.:
%
\begin{center}
\begin{tabular}{l}
|\ifchilddocmanual|\\
|\input{\childdocname}|\\
|\||else|\\
\textit{document body with }|\input{|\textit{part}|}|\\
|\||fi|
\end{tabular}
\end{center}
%
The conditional |\ifchilddocmanual| is true whenever
a part to be included by |\input| is being compiled,
and the name of the part is stored in |\childdocname|.

%%%%%%%%%%%%%%%%%%%%%%%%%%%%%%%%%%%%%%%%
\DescribeMacro{\childdocby}
Each part to be included by |\input| should start with:
%
\begin{center}
\begin{tabular}{l}
|\input{childdoc.def}|\\
|\childdocby{|\textit{main}|}|\\
\end{tabular}
\end{center}
%
The directive |\childdocby| is similar to |\childdocof|
described in \secref{sec:include},
but the subsequent selection of content must be done manually.
To that end, both |\ifchilddoc| and |\ifchilddocmanual|
will be true upon processing of a part,
and the name of the part is stored in |\childdocname|.
Note that |\jobname| will be set to the filename of the current part
so that each part receives an individual |.aux| file
that does not interfere with the |.aux| file(s) of the main document.
This behaviour can be altered by the alternative form
|\childdocby[*]{|\textit{main}|}| (with a non-empty optional argument)
which uses the |.aux| file of the main document
by setting |\jobname| to \textit{main}.

%%%%%%%%%%%%%%%%%%%%%%%%%%%%%%%%%%%%%%%%%%%%%%%%%%%%%%%%%%%%%%%%%%%%%%%%%%%%%%%%
\subsection{Driver Development}
\label{sec:driver}

The \textsf{childdoc} mechanism can also be use for the development
of definition files such as \LaTeX{} styles or classes.
This case differs from the above setup with multiple parts
included by |\include| in that no |\includeonly| should be invoked.
This can be achieved by starting the include file
(before |\ProvidesPackage|) with:
%
\begin{center}
\begin{tabular}{l}
|\input{childdoc.def}|\\
|\childdocforward{|\textit{main}|}|\\
\end{tabular}
\end{center}
%
or alternatively with:
%
\begin{center}
\begin{tabular}{l}
|\input{childdoc.def}|\\
|\childdocby{|\textit{main}|}|\\
\end{tabular}
\end{center}
%
Both forms have slightly different effects as described above.
The main file is prepared as usual, see \secref{sec:include}.

%%%%%%%%%%%%%%%%%%%%%%%%%%%%%%%%%%%%%%%%%%%%%%%%%%%%%%%%%%%%%%%%%%%%%%%%%%%%%%%%
\subsection{Legacy Detection}
\label{sec:detection}

The directive |\childdocmain| in the main file can detect
whether the complete document or merely a child is to be compiled
even without using the directive |\childdocof|.
This method is deprecated because it is less robust
and there is no compelling reason to use it;
it is merely provided for backward compatibility
and it may be removed in future versions.

If the detection mechanism is to be used,
it is mandatory to correctly specify
the filename of the main file as the argument of |\childdocmain|:
%
\begin{center}
\begin{tabular}{l}
|\input{childdoc.def}|\\
|\childdocmain{|\textit{main}|}|\\
\end{tabular}
\end{center}
%
If |\jobname| does not match the argument \textit{main} of |\childdocmain|,
it is assumed that |\jobname| points to the child file to be compiled.
When using |\childdocmain| with the main file specified as argument,
it suffices to start a child file
with just |\input{|\textit{main}|}|
without loading of the package and using |\childdocof|.
If instead all processing is done
with the appropriate \textsf{childdoc} directives,
the argument of \textit{main} of |\childdocmain| can be empty.

An alternative version of the command line processing described
in \secref{sec:commandline} using the detection mechanism reads:
%
\begin{center}
|... -jobname "|\textit{target}|" "|[\textit{flags}]%
[|\def\jobname{|\textit{dest}|}|]|\input{|\textit{main}|}"|
\end{center}

%%%%%%%%%%%%%%%%%%%%%%%%%%%%%%%%%%%%%%%%%%%%%%%%%%%%%%%%%%%%%%%%%%%%%%%%%%%%%%%%
\subsection{Manual Code}
\label{sec:manual}

In case one cannot be certain whether the definitions file |childdoc.def|
is installed on the target \TeX{} distribution
and one prefers not to ship it,
it is conceivable to paste a few relevant commands into the sources.

To that end, drop all statements |\input{childdoc.def}|
and perform the replacements as outlined below.
Instead of |\childdocmain{|\textit{main}|}| add the following code
to the top of the main file:
%
\begin{center}
\begin{tabular}{l}
|\||ifdefined\childdocname\endinput\||fi\newif\ifchilddoc|\\
|\edef\childdocname{\scantokens\expandafter{\jobname\noexpand}}|\\
|\def\childdocmain{|\textit{main}|}\||ifx\childdocmain\childdocname\||else|\\
|\childdoctrue\includeonly{\childdocname}\let\jobname\childdocmain\||fi|\\
\end{tabular}
\end{center}
%
Instead of |\childdocof{|\textit{main}|}| just include the main file
at the top of each child file:
%
\begin{center}
|\input{|\textit{main}|}|
\end{center}
%
A simple redirection |\childdocforward{|\textit{dest}|}| is achieved by:
%
\begin{center}
|\def\jobname{|\textit{dest}|}\input{\jobname}|
\end{center}
%
The redirection with prefix
|\childdocforwardprefix[|\textit{prefix}|]{|\textit{dest}|}|
is accomplished by:
%
\begin{center}
\begin{tabular}{l}
|{\edef\jobname{\scantokens\expandafter{\jobname\noexpand}}|\\
|\def\redirectjob |\textit{prefix}|#1~~~{\gdef\jobname{|\textit{dest}|#1}}|\\
|\expandafter\redirectjob\jobname~~~}\input{\jobname}|
\end{tabular}
\end{center}

In an alternative approach,
child documents can be compiled by a specific command line
without additional code or specific definitions:
%
\begin{center}
|... -jobname "|\textit{target}|" "|[\textit{flags}]%
|\includeonly{|\textit{dest}|}\input{|\textit{main}|}"|
\end{center}
%

%%%%%%%%%%%%%%%%%%%%%%%%%%%%%%%%%%%%%%%%%%%%%%%%%%%%%%%%%%%%%%%%%%%%%%%%%%%%%%%%
%%%%%%%%%%%%%%%%%%%%%%%%%%%%%%%%%%%%%%%%%%%%%%%%%%%%%%%%%%%%%%%%%%%%%%%%%%%%%%%%
\section{Information}

%%%%%%%%%%%%%%%%%%%%%%%%%%%%%%%%%%%%%%%%%%%%%%%%%%%%%%%%%%%%%%%%%%%%%%%%%%%%%%%%
\subsection{Copyright}

Copyright \copyright{} 2017--2018 Niklas Beisert

This work may be distributed and/or modified under the
conditions of the \LaTeX{} Project Public License, either version 1.3
of this license or (at your option) any later version.
The latest version of this license is in
  \url{http://www.latex-project.org/lppl.txt}
and version 1.3 or later is part of all distributions of \LaTeX{}
version 2005/12/01 or later.

This work has the LPPL maintenance status `maintained'.

The Current Maintainer of this work is Niklas Beisert.

This work consists of the files |README.txt|, |childdoc.ins| and |childdoc.dtx|
as well as the derived files |childdoc.def|, |cdocsamp.tex|
with |cdocsch1.tex|, |cdocsch2.tex|, |cdocspt3.tex|, |cdocspt4.tex|,
|cdocsdrf.tex|, |cdocsfn1.tex|, |cdocsfn2.tex|
as well as |childdoc.pdf|.

%%%%%%%%%%%%%%%%%%%%%%%%%%%%%%%%%%%%%%%%%%%%%%%%%%%%%%%%%%%%%%%%%%%%%%%%%%%%%%%%
\subsection{Files and Installation}

The package consists of the files:
%
\begin{center}
\begin{tabular}{ll}
    |README.txt|   & readme file \\
    |childdoc.ins| & installation file \\
    |childdoc.dtx| & source file \\
    |childdoc.def| & definition file \\
    |cdocsamp.tex| & sample main file \\
    |cdocsch1.tex| & sample include file \\
    |cdocsch2.tex| & sample include file \\
    |cdocspt3.tex| & sample part file \\
    |cdocspt4.tex| & sample part file \\
    |cdocsdrf.tex| & sample redirection file \\
    |cdocsfn1.tex| & sample redirection file \\
    |cdocsfn2.tex| & sample redirection file \\
    |childdoc.pdf| & manual
\end{tabular}
\end{center}
%
The distribution consists of the files
|README.txt|, |childdoc.ins| and |childdoc.dtx|.
%
\begin{itemize}
\item
Run (pdf)\LaTeX{} on |childdoc.dtx|
to compile the manual |childdoc.pdf| (this file).
\item
Run \LaTeX{} on |childdoc.ins| to create the definitions file |childdoc.def|
and the sample |cdocsamp.tex| with include files
|cdocsch1.tex|, |cdocsch2.tex|, |cdocspt3.tex|, |cdocspt4.tex|,
|cdocsdrf.tex|, |cdocsfn1.tex|, |cdocsfn2.tex|.
Then copy the file |childdoc.def| to an appropriate directory of your \LaTeX{}
distribution, e.g.\ \textit{texmf-root}|/tex/latex/childdoc|.
\end{itemize}

%%%%%%%%%%%%%%%%%%%%%%%%%%%%%%%%%%%%%%%%%%%%%%%%%%%%%%%%%%%%%%%%%%%%%%%%%%%%%%%%
\subsection{Related CTAN Packages}

There are several other packages which offer a similar functionality:
%
\begin{itemize}
\item
The packages
\href{http://ctan.org/pkg/docmute}{\textsf{docmute}},
\href{http://ctan.org/pkg/includex}{\textsf{includex}} and
\href{http://ctan.org/pkg/standalone}{\textsf{standalone}}
provide commands to include only the document body of
a child file thus allowing both files to be compiled individually.
\item
The packages \href{http://ctan.org/pkg/subdocs}{\textsf{subdocs}}
and \href{http://ctan.org/pkg/subfiles}{\textsf{subfiles}}
provide structures in which the main and child documents can be
encapsulated and allowing them to be compiled individually.
The inclusion mechanism is different from the conventional |\include|.
\item
The package \href{http://ctan.org/pkg/combine}{\textsf{combine}}
is an elaborate solution to combine several documents into one.
\end{itemize}
%
See also the CTAN topic \href{http://ctan.org/topic/subdocs}{\textsf{subdocs}}
for further related packages.
The present package differs from the above solutions in that
a document structure constructed with the conventional |\include| mechanism
just needs two extra commands at the top of every file
such that all constituent files can be compiled individually.

%%%%%%%%%%%%%%%%%%%%%%%%%%%%%%%%%%%%%%%%%%%%%%%%%%%%%%%%%%%%%%%%%%%%%%%%%%%%%%%%
%\subsection{Feature Suggestions}
%
%The following is a list of features which may be useful for future
%versions of this package:
%%
%\begin{itemize}
%\item
%\ldots
%\end{itemize}

%%%%%%%%%%%%%%%%%%%%%%%%%%%%%%%%%%%%%%%%%%%%%%%%%%%%%%%%%%%%%%%%%%%%%%%%%%%%%%%%
\subsection{Revision History}

%%%%%%%%%%%%%%%%%%%%%%%%%%%%%%%%%%%%%%%%
\paragraph{v2.0:} 2018/12/30

\begin{itemize}
\item
immediate forward processing
\item
added |\childdocby| mechanism
\item
manual restructured
\end{itemize}

%%%%%%%%%%%%%%%%%%%%%%%%%%%%%%%%%%%%%%%%
\paragraph{v1.6:} 2018/01/17

\begin{itemize}
\item
application for development of include files
\item
corrections to manual
\end{itemize}

%%%%%%%%%%%%%%%%%%%%%%%%%%%%%%%%%%%%%%%%
\paragraph{v1.5:} 2017/05/21

\begin{itemize}
\item
more complete structuring introduced
\item
|\childdocof| introduced
\item
|\childdoc| renamed to |\childdocmain|
\item
|\childredirect| renamed to |\childdocforward| and |\childdocforwardprefix|
and functionality expanded
\end{itemize}

%%%%%%%%%%%%%%%%%%%%%%%%%%%%%%%%%%%%%%%%
\paragraph{v1.0:} 2017/04/27

\begin{itemize}
\item
manual and install package
\item
first version published on CTAN
\end{itemize}

%%%%%%%%%%%%%%%%%%%%%%%%%%%%%%%%%%%%%%%%
\paragraph{v0.6:} 2017/04/26

\begin{itemize}
\item
redirection mechanism added
\end{itemize}

%%%%%%%%%%%%%%%%%%%%%%%%%%%%%%%%%%%%%%%%
\paragraph{v0.5:} 2017/04/26

\begin{itemize}
\item
functionality in definition file
\end{itemize}


%%%%%%%%%%%%%%%%%%%%%%%%%%%%%%%%%%%%%%%%%%%%%%%%%%%%%%%%%%%%%%%%%%%%%%%%%%%%%%%%
%%%%%%%%%%%%%%%%%%%%%%%%%%%%%%%%%%%%%%%%%%%%%%%%%%%%%%%%%%%%%%%%%%%%%%%%%%%%%%%%
%%%%%%%%%%%%%%%%%%%%%%%%%%%%%%%%%%%%%%%%%%%%%%%%%%%%%%%%%%%%%%%%%%%%%%%%%%%%%%%%
\appendix

\settowidth\MacroIndent{\rmfamily\scriptsize 000\ }

 \DocInput{childdoc.dtx}

\end{document}
%</driver>
% \fi
%
% %%%%%%%%%%%%%%%%%%%%%%%%%%%%%%%%%%%%%%%%%%%%%%%%%%%%%%%%%%%%%%%%%%%%%%%%%%%%%%
% %%%%%%%%%%%%%%%%%%%%%%%%%%%%%%%%%%%%%%%%%%%%%%%%%%%%%%%%%%%%%%%%%%%%%%%%%%%%%%
% \section{Sample}
%\iffalse
%<*samplemain>
%\fi
%
% The following presents a sample document
% with two chapters, two parts, a title page,
% a compile flag as well as three forwarding files to set the flag.
% It consists of eight |.tex| files:
% \begin{center}
% \begin{tabular}{ll}
% |cdocsamp.tex|&main file\\
% |cdocsch1.tex|&include file for chapter 1\\
% |cdocsch2.tex|&include file for chapter 2\\
% |cdocspt3.tex|&include file for part 3\\
% |cdocspt4.tex|&include file for part 4\\
% |cdocsdrf.tex|&forwarding file for main file in draft mode\\
% |cdocsfi1.tex|&forwarding file for final version of chapter 1\\
% |cdocsfi2.tex|&forwarding file for final version of chapter 2\\
% \end{tabular}
% \end{center}
% Each of the eight files can be compiled directly by the \LaTeX{} compiler.
%
% %%%%%%%%%%%%%%%%%%%%%%%%%%%%%%%%%%%%%%
% \paragraph{Main File.}
%
% The main file is called |cdocsamp.tex|.
%
% Load the \textsf{childdoc} definitions and
% declare the filename for the main document:
%    \begin{macrocode}
\input{childdoc.def}
\childdocmain{}
%    \end{macrocode}

% Optional override for |\version| flag:
%    \begin{macrocode}
%%\ifchilddoc\else\providecommand{\version}{draft}\fi
%    \end{macrocode}

% Define the default values for the |\version| flag
% (|final| for the main file and |draft| for childs):
%    \begin{macrocode}
\ifchilddoc
\providecommand{\version}{draft}
\else
\providecommand{\version}{final}
\fi
%    \end{macrocode}

% Load the standard document class:
%    \begin{macrocode}
\documentclass[12pt]{article}
%    \end{macrocode}

% Start the document body:
%    \begin{macrocode}
\begin{document}
%    \end{macrocode}

% Declare a title page.
% Print title, part of document being processed and version flag:
%    \begin{macrocode}
\addtocounter{page}{-1}
\begin{center}
{\LARGE\bfseries{}childdoc example\par}
\vspace{1cm}
\ifchilddoc
\ifchilddocmanual part\else chapter\fi:
`\childdocname' of `\childdocjob'\par
\else
main document: `\childdocjob'\par
\fi
version: \version\par
\end{center}
\newpage
%    \end{macrocode}

% Manually include selected file,
% otherwise process as usual:
%    \begin{macrocode}
\ifchilddocmanual
\section*{part `\childdocname'}
\input{\childdocname}
\else
%    \end{macrocode}

% Include the two chapters:
%    \begin{macrocode}
\include{cdocsch1}
\include{cdocsch2}
%    \end{macrocode}

% Include the two parts unless only chapters should be displayed:
%    \begin{macrocode}
\ifchilddoc\else
\section{part three}
\input{cdocspt3}
\section{part four}
\input{cdocspt4}
\fi
%    \end{macrocode}

% Process as usual until here:
%    \begin{macrocode}
\fi
%    \end{macrocode}

% End of document body:
%    \begin{macrocode}
\end{document}
%    \end{macrocode}
%\iffalse
%</samplemain>
%\fi
%
% %%%%%%%%%%%%%%%%%%%%%%%%%%%%%%%%%%%%%%
% \paragraph{Chapter Include Files.}
%
% The include files are called |cdocsch1.tex| and |cdocsch2.tex|.
%
%\iffalse
%<*samplechap1|samplechap2>
%\fi

% Optional override for |\version| flag:
%    \begin{macrocode}
%%\providecommand{\version}{final}
%    \end{macrocode}

% Include the main document:
%    \begin{macrocode}
\input{childdoc.def}
\childdocof{cdocsamp}
%    \end{macrocode}

%\iffalse
%</samplechap1|samplechap2>
%\fi
%
%\iffalse
%<*samplechap1>
%\fi
% Some text for chapter 1:
%    \begin{macrocode}
\section{one}
some text in chapter one
%    \end{macrocode}

%\iffalse
%</samplechap1>
%\fi
% Some text for chapter 2:
%\iffalse
%<*samplechap2>
%\fi
%    \begin{macrocode}
\section{two}
more text in chapter two
%    \end{macrocode}

%\iffalse
%</samplechap2>
%\fi
%
% %%%%%%%%%%%%%%%%%%%%%%%%%%%%%%%%%%%%%%
% \paragraph{Part Include Files.}
%
% The include files are called |cdocspt3.tex| and |cdocspt4.tex|.
%
%\iffalse
%<*samplepart3|samplepart4>
%\fi

% Optional override for |\version| flag:
%    \begin{macrocode}
%%\providecommand{\version}{final}
%    \end{macrocode}

% Include the main document:
%    \begin{macrocode}
\input{childdoc.def}
\childdocby{cdocsamp}
%    \end{macrocode}

%\iffalse
%</samplepart3|samplepart4>
%\fi
%
%\iffalse
%<*samplepart3>
%\fi
% Some text for part 3:
%    \begin{macrocode}
some text in part three
%    \end{macrocode}

%\iffalse
%</samplepart3>
%\fi
% Some text for part 4:
%\iffalse
%<*samplepart4>
%\fi
%    \begin{macrocode}
more text in part four
%    \end{macrocode}

%\iffalse
%</samplepart4>
%\fi
%
% %%%%%%%%%%%%%%%%%%%%%%%%%%%%%%%%%%%%%%
% \paragraph{Forwarding for a Complete Draft.}
%
% The following forwarding file |cdocsdrf.tex|
% compiles the main document in draft mode:
%\iffalse
%<*sampledraft>
%\fi
%    \begin{macrocode}
\def\version{draft}
\input{childdoc.def}
\childdocforward{cdocsamp}
%    \end{macrocode}

%\iffalse
%</sampledraft>
%\fi
%
% %%%%%%%%%%%%%%%%%%%%%%%%%%%%%%%%%%%%%%
% \paragraph{Forwarding for Final Version of the Chapters.}
%
% The following forwarding files |cdocsfn1.tex| and |cdocsfn2.tex|
% (with identical content)
% compile the final versions of the child documents
% |cdocsch1.tex| and |cdocsch2.tex|, respectively:
%\iffalse
%<*samplefinal>
%\fi
%    \begin{macrocode}
\def\version{final}
\input{childdoc.def}
\childdocforwardprefix[cdocsamp]{cdocsfn}{cdocsch}
%    \end{macrocode}

%\iffalse
%</samplefinal>
%\fi
%
% %%%%%%%%%%%%%%%%%%%%%%%%%%%%%%%%%%%%%%
% \paragraph{Command Line Processing.}
%
% The following three command lines generate the output files
% |cdocscld|, |cdocscl1| and |cdocscl2|
% which should be identical to
% |cdocsdrf|, |cdocsch1| and |cdocsfn2|, respectively:
% \begin{center}
% \begin{tabular}{l}
% |latex -jobname cdocscld \|\\
% |  "\def\version{draft}\input{childdoc.def}\childdocforward{cdocsamp}"|\\
% |latex -jobname cdocscl1 \|\\
% |  "\input{childdoc.def}\childdocforward[cdocsamp]{cdocsch1}"|\\
% |latex -jobname cdocscl2 \|\\
% |  "\def\version{final}\input{childdoc.def}\childdocforward{cdocsch2}"|
% \end{tabular}
% \end{center}
% Note that the trailing backslash on each first line
% merely continues the input to the second line
% (for convenient cut ant paste).
% Furthermore, the command |latex| can be replaced by any
% of its alternative versions such as |pdflatex|.
%
% %%%%%%%%%%%%%%%%%%%%%%%%%%%%%%%%%%%%%%%%%%%%%%%%%%%%%%%%%%%%%%%%%%%%%%%%%%%%%%
% %%%%%%%%%%%%%%%%%%%%%%%%%%%%%%%%%%%%%%%%%%%%%%%%%%%%%%%%%%%%%%%%%%%%%%%%%%%%%%
% \section{Implementation}
%\iffalse
%<*package>
%\fi
%
% This section describes the definitions file |childdoc.def|.

% The definitions cannot be loaded using |\usepackage| or |\RequirePackage|
% which has a mechanism to prevent loading a style file more than once.
% When loading the definitions by means of |\input|
% multiple instances have to be prevented manually:
%\iffalse
%This code needs to be before the `\ProvidesFile' directive
%which is defined at the beginning of this file.
%Therefore it is also placed there and commented out here.
%</package>
%<*discard>
%\fi
%    \begin{macrocode}
\ifdefined\childdocmain\endinput\fi
%    \end{macrocode}
%\iffalse
%</discard>
%<*package>
%\fi
%
% \macro{\ifchilddoc}
% \macro{\ifchilddocmanual}
% The conditional |\ifchilddoc| tells whether a
% child (true) or main (false) document is being compiled.
% The conditional |\ifchilddocmanual| tells whether
% the |\includeonly| mechanism is used (false) or
% the selection of child files must be performed manually (true).
% The definitions initialise to false:
%    \begin{macrocode}
\newif\ifchilddoc
\newif\ifchilddocmanual
%    \end{macrocode}

% \macro{\childdocname}
% \macro{\childdocjob}
% The macro |\childdocname| stores the name of the main document
% to be compiled. The macro |\childdocjob| stores the name of
% the document on which the \LaTeX{} compiler was originally invoked.
% The content of |\jobname| cannot be compared
% to filenames specified in the source due to different catcodes.
% The following code rescans |\jobname|, stores the result
% in |\childdocname| and saves a copy in |\childdocjob|:
%    \begin{macrocode}
\edef\childdocname{\scantokens\expandafter{\jobname\noexpand}}
\let\childdocjob\childdocname
%    \end{macrocode}

% \macro{\childdocdisable}
% The macro |\childdocdisable| prevents the main file
% from being processed more than once.
% At this stage, the main document command |\childdocmain|
% is assumed to be called once again where it should do nothing.
% Any subsequent call to it should prevent
% a secondary processing of the main document
% It overwrites the forwarding commands
% |\childdocof| and |\childdocforward|
% with empty macros to prevent further inclusions of the main document:
%    \begin{macrocode}
\newcommand{\childdocdisable}
{
  \renewcommand{\childdocmain}[1]{\renewcommand{\childdocmain}[1]{\endinput}}
  \renewcommand{\childdocof}[1]{}
  \renewcommand{\childdocby}[2][]{}
  \renewcommand{\childdocforward}[2][]{}
  \renewcommand{\childdocdisable}{}
}
%    \end{macrocode}

% \macro{\childdocmain}
% The macro |\childdocmain| is to be called at the top of the main file
% with nothing or the main filename (without extension) as argument.
% First, it breaks loops.
% If the argument is not empty and does not match |\childdocname|
% (which is set by the first inclusion of |childdoc.def|),
% |\ifchilddoc| is set to true, |\includeonly| is applied to the child file
% and |\jobname| is set to the main file
% (for proper handling of |.aux| files):
%    \begin{macrocode}
\newcommand{\childdocmain}[1]
{
  \childdocdisable\childdocmain{}
  \if?#1?\else
    \begingroup
      \def\childdoctmp{#1}
      \ifx\childdoctmp\childdocname
        \def\childdoctmp{}
      \else
        \def\childdoctmp
        {
          \childdoctrue
          \includeonly{\childdocname}
          \def\childdocjob{#1}
          \def\jobname{#1}
        }
      \fi
      \expandafter
    \endgroup
    \childdoctmp
  \fi
}
%    \end{macrocode}

% \macro{\childdocof}
% The command |\childdocof| redirects
% compilation to the main file |#1|.
%    \begin{macrocode}
\newcommand{\childdocof}[1]
{
  \childdocdisable
  \childdoctrue
  \includeonly{\childdocname}
  \def\jobname{#1}
  \def\childdocjob{#1}
  \input{#1}
}
%    \end{macrocode}

% \macro{\childdocby}
% The command |\childdocby| ....
%    \begin{macrocode}
\newcommand{\childdocby}[2][]
{
  \childdocdisable
  \childdoctrue
  \childdocmanualtrue
  \if?#1?\else
    \def\jobname{#2}
  \fi
  \def\childdocjob{#2}
  \input{#2}
  \endinput
}
%    \end{macrocode}

% \macro{\childdocforward}
% The command |\childdocforward| redirects
% compilation to the main file or
% (if the optional argument is given) a child file.
% Parameters are set as if the main file
% or a child file starting with |\childdocof| was compiled.
% Then compilation is handed over to the main file:
%    \begin{macrocode}
\newcommand{\childdocforward}[2][]
{
  \begingroup
    \if?#1?
      \def\childdoctmp
      {
        \def\childdocname{#2}
        \def\childdocjob{#2}
        \def\jobname{#2}
        \input{#2}
        \endinput
      }
    \else
      \def\childdoctmp
      {
        \childdocdisable
        \def\childdocname{#2}
        \childdoctrue
        \includeonly{#2}
        \def\childdocjob{#1}
        \def\jobname{#1}
        \input{#1}
        \endinput
      }
    \fi
    \expandafter
  \endgroup
  \childdoctmp
}
%    \end{macrocode}

% \macro{\childdocforwardprefix}
% The command |\childdocforwardprefix| redirects
% compilation to the main or a child file by means of a pattern.
% The prefix |#1| in the current filename is replaced by |#2|
% and the suffix of the current filename is kept
% (it is assumed that the filename does not contain the substring `|~~~|'
% which is used as a delimiter).
% Compilation is handed over to the new file by |\childdocforward|:
%    \begin{macrocode}
\newcommand{\childdocforwardprefix}[3][]
{
  \begingroup
    \def\childdocextract #2##1~~~{\def\childdoctmp{\childdocforward[#1]{#3##1}}}
    \expandafter\childdocextract\childdocname~~~
    \expandafter
  \endgroup
  \childdoctmp
}
%    \end{macrocode}

% \macro{\childdoc}
% The deprecated macro |\childdoc| is a legacy version of |\childdocmain|:
%    \begin{macrocode}
\newcommand{\childdoc}{\childdocmain}
%    \end{macrocode}

% \macro{\childdocredirect}
% The deprecated macro |\childdocredirect| is a legacy version
% of |\childdocforward| and |\childdocforwardprefix|:
%    \begin{macrocode}
\newcommand{\childdocredirect}[2][]
{
  \begingroup
    \if?#1?
      \def\childdoctmp{\childdocforward{#2}}
    \else
      \def\childdoctmp{\childdocforwardprefix{#1}{#2}}
    \fi
    \expandafter
  \endgroup
  \childdoctmp
}
%    \end{macrocode}

%\iffalse
%</package>
%\fi
%
\endinput
|\\
|\childdocforwardprefix[|\textit{main}|]{|\textit{prefix}|}{|\textit{dest}|}|
\end{tabular}
\end{center}
%
the destination file is determined by a pattern
depending on the current file:
To make this work, the current file must be called
`{\textit{prefix}\hspace{0.2em}\textit{suffix}}'
with \textit{prefix} matching precisely the argument.
Processing is then passed on to the file
`{\textit{dest}\hspace{0.2em}\textit{suffix}}'.
Surely, the same effect is achieved by
directly specifying the
argument `{\textit{dest}\hspace{0.2em}\textit{suffix}}'
in the first form.
However, that requires to set up a different file
for each child. With the alternative form of the command
all these files can have exactly the same content
which simplifies setting them up and maintaining them.

For example, the following file |draft.tex|
with a compilation flag |\version| as described in \secref{sec:flags}
compiles the main document as a draft:
%
\begin{center}
\begin{tabular}{l}
|\def\version{draft}|\\
|% \iffalse
%
% childdoc.dtx Copyright (C) 2017-2018 Niklas Beisert
%
% This work may be distributed and/or modified under the
% conditions of the LaTeX Project Public License, either version 1.3
% of this license or (at your option) any later version.
% The latest version of this license is in
%   http://www.latex-project.org/lppl.txt
% and version 1.3 or later is part of all distributions of LaTeX
% version 2005/12/01 or later.
%
% This work has the LPPL maintenance status `maintained'.
%
% The Current Maintainer of this work is Niklas Beisert.
%
% This work consists of the files childdoc.dtx and childdoc.ins
% and the derived files childdoc.def and cdocsamp.tex with
% cdocsch1.tex, cdocsch2.tex, cdocsdrf.tex, cdocsfn1.tex, cdocsfn2.tex.
%
%<package>\ifdefined\childdocmain\endinput\fi
%<package>\ProvidesFile{childdoc.def}[2018/12/30 v2.0 child document driver]
%<samplemain>\ProvidesFile{cdocsamp.tex}[2018/12/30 v2.0 sample for childdoc]
%<*driver>
%\ProvidesFile{childdoc.drv}[2018/12/30 v2.0 childdoc reference manual file]
\PassOptionsToClass{10pt,a4paper}{article}
\documentclass{ltxdoc}

\usepackage[margin=35mm]{geometry}
\usepackage{hyperref}
\usepackage{hyperxmp}
\usepackage[usenames]{color}

\hypersetup{colorlinks=true}
\hypersetup{pdfstartview=FitH}
\hypersetup{pdfpagemode=UseNone}
\hypersetup{pdfsource={}}
\hypersetup{pdflang={en-UK}}
\hypersetup{pdfcopyright={Copyright 2017-2018 Niklas Beisert.
  This work may be distributed and/or modified under the
  conditions of the LaTeX Project Public License, either version 1.3
  of this license or (at your option) any later version.}}
\hypersetup{pdflicenseurl={http://www.latex-project.org/lppl.txt}}
\hypersetup{pdfcontactaddress={ETH Zurich, ITP, HIT K,
  Wolfgang-Pauli-Strasse 27}}
\hypersetup{pdfcontactpostcode={8093}}
\hypersetup{pdfcontactcity={Zurich}}
\hypersetup{pdfcontactcountry={Switzerland}}
\hypersetup{pdfcontactemail={nbeisert@itp.phys.ethz.ch}}
\hypersetup{pdfcontacturl={http://people.phys.ethz.ch/\xmptilde nbeisert/}}

\newcommand{\secref}[1]{\hyperref[#1]{section \ref*{#1}}}

\parskip1ex
\parindent0pt
\let\olditemize\itemize
\def\itemize{\olditemize\parskip0pt}

\begin{document}

\title{The \textsf{childdoc} Package}
\hypersetup{pdftitle={The childdoc Package}}
\author{Niklas Beisert\\[2ex]
  Institut f\"ur Theoretische Physik\\
  Eidgen\"ossische Technische Hochschule Z\"urich\\
  Wolfgang-Pauli-Strasse 27, 8093 Z\"urich, Switzerland\\[1ex]
  \href{mailto:nbeisert@itp.phys.ethz.ch}
  {\texttt{nbeisert@itp.phys.ethz.ch}}}
\hypersetup{pdfauthor={Niklas Beisert}}
\hypersetup{pdfsubject={Manual for the LaTeX2e Package childdoc}}
\date{30 December 2018, \textsf{v2.0}}
\maketitle

\begin{abstract}\noindent
\textsf{childdoc} is a \LaTeXe{} package
that enables the direct compilation
of document sections included by |\include|
to individual files.
\end{abstract}

\begingroup
\parskip0ex
\tableofcontents
\endgroup

%%%%%%%%%%%%%%%%%%%%%%%%%%%%%%%%%%%%%%%%%%%%%%%%%%%%%%%%%%%%%%%%%%%%%%%%%%%%%%%%
%%%%%%%%%%%%%%%%%%%%%%%%%%%%%%%%%%%%%%%%%%%%%%%%%%%%%%%%%%%%%%%%%%%%%%%%%%%%%%%%
\section{Introduction}

\LaTeX{} provides a mechanism to structure a large document (such as a book)
into a main file and several child files (containing the chapters)
using the |\include| command.
This mechanism is beneficial for documents
which span hundreds of pages in order to
make the source file(s) more manageable.
Moreover, compilation can be restricted to
selected child files by means of the |\includeonly| command.
The latter feature can be used to reduce the compilation time while editing
(this was significantly more useful in the earlier days of \LaTeX{})
or to generate a smaller document which is easier to navigate.
Another application of |\includeonly| is to generate
documents consisting of selected parts of the complete document.

However, there are a few drawbacks of the plain |\include| mechanism:
\begin{itemize}
\item
The child files cannot be compiled on their own,
they can only be compiled via the main file.
A naive editing environment
(such as a text editor with an option
to have the current file processed by \LaTeX)
may require one to switch to the main file before compiling;
attempting to compile the child file produces errors.
\item
The main file must be modified (each time)
to adjust the |\includeonly| command
to the present needs. This easily leaves the main file in a messy state.
\item
The generated document will always carry the filename
of the main document. This is inconvenient if
several child files are to be compiled and
to be kept for distribution.
\end{itemize}

The present package provides a simple interface
to make child files individually compilable by \LaTeX{}.
Compiling a child file then has the same effect as compiling
the main file with an |\includeonly| command
to select the appropriate child.
Moreover the generated document will carry the name of the child
rather than the main file.
This resolves all three above issues.

This feature is meant to make the editing of books,
thesis documents and lecture notes somewhat more convenient.
However, the package can also be used efficiently for
composing a series of documents (such as exercise sheets)
which are typically distributed individually.
It then assists the author in generating the individual documents
(potentially in different versions)
as well as a document containing the collected series.
Another application is in developing style files
or other kinds of included material
where compilation of the style file could redirect
to a sample or test file.

%%%%%%%%%%%%%%%%%%%%%%%%%%%%%%%%%%%%%%%%%%%%%%%%%%%%%%%%%%%%%%%%%%%%%%%%%%%%%%%%
%%%%%%%%%%%%%%%%%%%%%%%%%%%%%%%%%%%%%%%%%%%%%%%%%%%%%%%%%%%%%%%%%%%%%%%%%%%%%%%%
\section{Usage}

First of all, the package \textsf{childdoc} is \emph{not} a standard
\LaTeXe{} |.sty| style file! Therefore it needs to be invoked in
a non-standard way.

%%%%%%%%%%%%%%%%%%%%%%%%%%%%%%%%%%%%%%%%%%%%%%%%%%%%%%%%%%%%%%%%%%%%%%%%%%%%%%%%
\subsection{Included Files}
\label{sec:include}

%%%%%%%%%%%%%%%%%%%%%%%%%%%%%%%%%%%%%%%%
\DescribeMacro{\childdocmain}
To use the package, add the commands
\begin{center}
\begin{tabular}{l}
|\input{childdoc.def}|\\
|\childdocmain{}|\\
\end{tabular}
\end{center}
at the very top of the main \LaTeX{} file,
in particular \emph{before} the |\documentclass| statement!
The argument of |\childdocmain| should be left empty
(but it must be present).

%%%%%%%%%%%%%%%%%%%%%%%%%%%%%%%%%%%%%%%%
\DescribeMacro{\childdocof}
Furthermore, add the commands
\begin{center}
\begin{tabular}{l}
|\input{childdoc.def}|\\
|\childdocof{|\textit{main}|}|\\
\end{tabular}
\end{center}
at the top of every child file \textit{child}
which is included by |\include{|\textit{child}|}|
from within the main file
(or at least for those files to be compiled individually).
The argument \textit{main} must be the filename of the main file.

There are a couple of
considerations in setting up the main and child documents:

%%%%%%%%%%%%%%%%%%%%%%%%%%%%%%%%%%%%%%%%
\paragraph{Restrictions.}

Please note the following restrictions:
\begin{itemize}
\item
|\childdocmain| must be called with one argument \textit{main}
to ensure compatibility with earlier version of the package.
It must either be empty (|\childdocmain{}|)
or precisely match the filename of the main file in which it is specified.
See \secref{sec:detection} for further information.
\item
The filename \textit{main} must be specified without the |.tex| extension.
\item
The filename \textit{main} is case sensitive
(even in case-insensitive file systems)
due to internal string comparison.
\item
The argument \textit{main} should be fully expanded, it cannot be a macro.
\item
Subdirectories and special characters should be avoided in filenames.
\item
The command |\childdocmain{|\textit{main}|}| must be followed by a whitespace.
It should not be followed immediately by another command
or by a comment mark `|%|'.
This is because the \TeX{} parser reads the token immediately following
the argument of |\childdocmain| and puts it
at the beginning of every child section;
however, a white\-space is ignored.
\end{itemize}

%%%%%%%%%%%%%%%%%%%%%%%%%%%%%%%%%%%%%%%%
\paragraph{Content of Main File.}

It is advisable to place all content in the child files included by |\include|.
Any output contained in the main file will appear in all child documents
unless suppressed manually;
it cannot be suppressed automatically by the |\includeonly| directive
and thus should normally be avoided.
A method to include some content in the main file
by means of conditional processing is described in \secref{sec:conditional}.

%%%%%%%%%%%%%%%%%%%%%%%%%%%%%%%%%%%%%%%%
\paragraph{Page Numbering.}

When only a part of the document is compiled,
the appropriate numbering of pages
(as well as other status parameters)
is determined from the |.aux| files.
The latter contain information from previous passes.
However this information needs to propagate through
all intermediate child documents.
Therefore the page numbering in child documents may well
be inconsistent until the complete document is compiled at least once.

A useful (if unconventional) way to always ensure a consistent
page numbering is to restart the numbering in each child document
and denote the pages by `\textit{child}|.|\textit{page}'
where \textit{child} represents the chapter/section number of the child file.
This can be achieved by the command
|\numberwithin{page}{|\textit{child}|}|
of the \textsf{amsmath} package
where \textit{child} can be |chapter| or |section|
depending on the chosen structuring.
Alternatively, one can modify the macro |\thepage| appropriately
and reset the counter |page| at the start of each child file.

%%%%%%%%%%%%%%%%%%%%%%%%%%%%%%%%%%%%%%%%%%%%%%%%%%%%%%%%%%%%%%%%%%%%%%%%%%%%%%%%
\subsection{Conditional Processing}
\label{sec:conditional}

The package provides a mechanism to compile different versions
of a document. To customise the versions further some conditional processing
can come in handy to distinguish which version is being compiled.
The package provides two macros to describe the compilation context:

%%%%%%%%%%%%%%%%%%%%%%%%%%%%%%%%%%%%%%%%
\DescribeMacro{\ifchilddoc}
The conditional |\ifchilddoc| distinguishes between the compilation of
child documents and the main document:
%
\begin{center}
|\ifchilddoc |\textit{child-code}| |[|\||else |\textit{main-code}]| \||fi|
\end{center}

%%%%%%%%%%%%%%%%%%%%%%%%%%%%%%%%%%%%%%%%
\DescribeMacro{\childdocname}
\DescribeMacro{\childdocjob}
The macro |\childdocname| contains the filename (without extension)
of the main or child file being processed.
Note that |\childdocjob| will always contain the name of the main file.

%%%%%%%%%%%%%%%%%%%%%%%%%%%%%%%%%%%%%%%%
\paragraph{Title Page.}

Conditional processing can be used to include a title or banner page
in the main document when proper precautions are taken.
Importantly, the code in the main file should ensure that the page counter
(as well as other status parameters which are stored in the |.aux| files)
takes the same value after the conditional processing.
Otherwise the page numbers may take divergent values
depending on which part is compiled.

For example, a title page could be declared by:
%
\begin{center}
\begin{tabular}{l}
|\ifchilddoc\||else|\\
|\addtocounter{page}{-1}|\\
\textit{code for title page}\\
|\newpage|\\
|\||fi|
\end{tabular}
\end{center}
%
A banner page for the child documents can be generated by:
%
\begin{center}
\begin{tabular}{l}
|\ifchilddoc|\\
|\addtocounter{page}{-1}|\\
\textit{code for banner page}\\
|\newpage|\\
|\||fi|
\end{tabular}
\end{center}
%
Here one could write a message such as:
\begin{center}
|This is the part \childdocname{} of \childdocjob{}.|
\end{center}

%%%%%%%%%%%%%%%%%%%%%%%%%%%%%%%%%%%%%%%%%%%%%%%%%%%%%%%%%%%%%%%%%%%%%%%%%%%%%%%%
\subsection{Flags}
\label{sec:flags}

The package makes it easy to generate different versions
of the main or child documents.
To this end compilation flags can be defined
and assigned different default values.
They will be particularly useful in conjunction
with the forwarding mechanism described in \secref{sec:forward}.

For example, it may be useful to have a flag |\version|
which can be set to |draft| or |final|.
The document source will contain some conditional code
depending on the value of |\version|.
Suppose further, the flag should default to |final| for the main file
and to |draft| for child files
which is a natural assignment for editing the document.
This is achieved by placing the following code
in the preamble of the main document
(below the |\childdocmain| directive):
%
\begin{center}
\begin{tabular}{l}
|\ifchilddoc|\\
|\providecommand{\version}{draft}|\\
|\||else|\\
|\providecommand{\version}{final}|\\
|\||fi|
\end{tabular}
\end{center}
%
The definition by |\providecommand| makes sure
that previous definitions are not overwritten.
Further statements |\providecommand{\version}{...}|
can thus be added before the above code to override it.

For the main file, one might add a line
(between |\childdocmain| and the above block)
%
\begin{center}
|%\ifchilddoc\||else\providecommand{\version}{draft}\||fi|
\end{center}
%
which can be uncommented to produce a draft version.
Likewise one can add a line to the very top of a child file
(above the |\childdocof{|\textit{main}|}| directive)
%
\begin{center}
|%\providecommand{\version}{final}|
\end{center}
%
which can be uncommented to produce the final version of this child document.

%%%%%%%%%%%%%%%%%%%%%%%%%%%%%%%%%%%%%%%%%%%%%%%%%%%%%%%%%%%%%%%%%%%%%%%%%%%%%%%%
\subsection{Forwarding}
\label{sec:forward}

Different versions of the main or child documents
using compilation flags as described in \secref{sec:flags}
can be (permanently) stored in different files
for convenient compilation, viewing and distribution.
To this end, the package defines a command
to pass on compilation to a different file:

%%%%%%%%%%%%%%%%%%%%%%%%%%%%%%%%%%%%%%%%
\DescribeMacro{\childdocforward}
The command |\childdocforward| redirects processing to
another source file:
%
\begin{center}
\begin{tabular}{l}
|\input{childdoc.def}|\\
|\childdocforward[|\textit{main}|]{|\textit{dest}|}|\\
\end{tabular}
\end{center}
%
The argument \textit{dest} is the destination file
(without extension).
It should be the main file or one of the child files.
Note that further \textsf{childdoc} directives
such as |\childdocof| and |\childdocforward|
in the indicated file will be processed in this form.
The optional argument \textit{main}
passes on directly to the main file \textit{main}
while pretending to compile the child \textit{dest}.
This form behaves as if \textit{dest}
issues |\childdocof{|\textit{main}|}| right away,
and no further \textsf{childdoc} directives will be processed.

%%%%%%%%%%%%%%%%%%%%%%%%%%%%%%%%%%%%%%%%
\DescribeMacro{\...prefix}
In the alternative form |\childdocforwardprefix|,
%
\begin{center}
\begin{tabular}{l}
|\input{childdoc.def}|\\
|\childdocforwardprefix[|\textit{main}|]{|\textit{prefix}|}{|\textit{dest}|}|
\end{tabular}
\end{center}
%
the destination file is determined by a pattern
depending on the current file:
To make this work, the current file must be called
`{\textit{prefix}\hspace{0.2em}\textit{suffix}}'
with \textit{prefix} matching precisely the argument.
Processing is then passed on to the file
`{\textit{dest}\hspace{0.2em}\textit{suffix}}'.
Surely, the same effect is achieved by
directly specifying the
argument `{\textit{dest}\hspace{0.2em}\textit{suffix}}'
in the first form.
However, that requires to set up a different file
for each child. With the alternative form of the command
all these files can have exactly the same content
which simplifies setting them up and maintaining them.

For example, the following file |draft.tex|
with a compilation flag |\version| as described in \secref{sec:flags}
compiles the main document as a draft:
%
\begin{center}
\begin{tabular}{l}
|\def\version{draft}|\\
|\input{childdoc.def}|\\
|\childdocforward{|\textit{main}|}|
\end{tabular}
\end{center}
%
Likewise, the following files |final|\textit{nn}|.tex|
compile the final version of the child document
|child|\textit{nn}|.tex|:
%
\begin{center}
\begin{tabular}{l}
|\def\version{final}|\\
|\input{childdoc.def}|\\
|\childdocforwardprefix{final}{child}|
\end{tabular}
\end{center}
%

Note that when several versions of a main file and/or of each child file
are to be generated, it may be convenient to set up a |Makefile| or
shell script to automatise the process.

%%%%%%%%%%%%%%%%%%%%%%%%%%%%%%%%%%%%%%%%%%%%%%%%%%%%%%%%%%%%%%%%%%%%%%%%%%%%%%%%
\subsection{Command Line Processing}
\label{sec:commandline}

The effect of redirection files can also be achieved by invoking
the \LaTeX{} compiler with a more elaborate command line.
Most conveniently this should be done as part
of a shell script or a |Makefile|.

When using \textsf{childdoc} in the main file, the following
command lines effectively perform a redirection
(note that depending on the shell being used,
backslashes may have to be doubled: `|\|' $\to$ `|\\|'):
%
\begin{center}
|... -jobname "|\textit{target}|" |\\|"|[\textit{flags}]%
|\input{childdoc.def}\childdocforward[|\textit{main}|]{|\textit{dest}|}"|
\end{center}
%
Here \textit{target} is the name of the output file,
\textit{main} is the name of the main file
and \textit{dest} is the name of the main or child file to be processed
(all filenames without extensions).
The optional argument \textit{main} can be omitted
if \textit{main} matches \textit{dest}.
Optionally, compilation \textit{flags} can be defined via |\def| commands.
This command line makes the \TeX{} engine believe
it is compiling the file \textit{target}
whose content is specified as the latter parameter.
The provided code then forwards the processing to
\textit{main} or \textit{dest} as described in \secref{sec:forward}.

%%%%%%%%%%%%%%%%%%%%%%%%%%%%%%%%%%%%%%%%%%%%%%%%%%%%%%%%%%%%%%%%%%%%%%%%%%%%%%%%
\subsection{Include by Input}
\label{sec:input}

Including child documents by |\include| has some restrictions by design.
Most notably, the content of a child document always occupies
its own set of pages; pages cannot be shared between child documents.
Usually, this behaviour makes perfect sense
because each child document contain an essential part of the document.
However, in some situations it may be desirable to compose
a document from a collection of parts
without having mandatory page breaks between then.
For this case, the package
provides a mechanism to include parts
by |\input| which can also be processed individually.
However, by construction this mechanism
requires manual handling of the content to be output.

%%%%%%%%%%%%%%%%%%%%%%%%%%%%%%%%%%%%%%%%
\DescribeMacro{\ifchilddocmanual}
The main file should be prepared as usual, see \secref{sec:include}.
However, the document body must make a distinction
between processing of an individual part and of the main document, e.g.:
%
\begin{center}
\begin{tabular}{l}
|\ifchilddocmanual|\\
|\input{\childdocname}|\\
|\||else|\\
\textit{document body with }|\input{|\textit{part}|}|\\
|\||fi|
\end{tabular}
\end{center}
%
The conditional |\ifchilddocmanual| is true whenever
a part to be included by |\input| is being compiled,
and the name of the part is stored in |\childdocname|.

%%%%%%%%%%%%%%%%%%%%%%%%%%%%%%%%%%%%%%%%
\DescribeMacro{\childdocby}
Each part to be included by |\input| should start with:
%
\begin{center}
\begin{tabular}{l}
|\input{childdoc.def}|\\
|\childdocby{|\textit{main}|}|\\
\end{tabular}
\end{center}
%
The directive |\childdocby| is similar to |\childdocof|
described in \secref{sec:include},
but the subsequent selection of content must be done manually.
To that end, both |\ifchilddoc| and |\ifchilddocmanual|
will be true upon processing of a part,
and the name of the part is stored in |\childdocname|.
Note that |\jobname| will be set to the filename of the current part
so that each part receives an individual |.aux| file
that does not interfere with the |.aux| file(s) of the main document.
This behaviour can be altered by the alternative form
|\childdocby[*]{|\textit{main}|}| (with a non-empty optional argument)
which uses the |.aux| file of the main document
by setting |\jobname| to \textit{main}.

%%%%%%%%%%%%%%%%%%%%%%%%%%%%%%%%%%%%%%%%%%%%%%%%%%%%%%%%%%%%%%%%%%%%%%%%%%%%%%%%
\subsection{Driver Development}
\label{sec:driver}

The \textsf{childdoc} mechanism can also be use for the development
of definition files such as \LaTeX{} styles or classes.
This case differs from the above setup with multiple parts
included by |\include| in that no |\includeonly| should be invoked.
This can be achieved by starting the include file
(before |\ProvidesPackage|) with:
%
\begin{center}
\begin{tabular}{l}
|\input{childdoc.def}|\\
|\childdocforward{|\textit{main}|}|\\
\end{tabular}
\end{center}
%
or alternatively with:
%
\begin{center}
\begin{tabular}{l}
|\input{childdoc.def}|\\
|\childdocby{|\textit{main}|}|\\
\end{tabular}
\end{center}
%
Both forms have slightly different effects as described above.
The main file is prepared as usual, see \secref{sec:include}.

%%%%%%%%%%%%%%%%%%%%%%%%%%%%%%%%%%%%%%%%%%%%%%%%%%%%%%%%%%%%%%%%%%%%%%%%%%%%%%%%
\subsection{Legacy Detection}
\label{sec:detection}

The directive |\childdocmain| in the main file can detect
whether the complete document or merely a child is to be compiled
even without using the directive |\childdocof|.
This method is deprecated because it is less robust
and there is no compelling reason to use it;
it is merely provided for backward compatibility
and it may be removed in future versions.

If the detection mechanism is to be used,
it is mandatory to correctly specify
the filename of the main file as the argument of |\childdocmain|:
%
\begin{center}
\begin{tabular}{l}
|\input{childdoc.def}|\\
|\childdocmain{|\textit{main}|}|\\
\end{tabular}
\end{center}
%
If |\jobname| does not match the argument \textit{main} of |\childdocmain|,
it is assumed that |\jobname| points to the child file to be compiled.
When using |\childdocmain| with the main file specified as argument,
it suffices to start a child file
with just |\input{|\textit{main}|}|
without loading of the package and using |\childdocof|.
If instead all processing is done
with the appropriate \textsf{childdoc} directives,
the argument of \textit{main} of |\childdocmain| can be empty.

An alternative version of the command line processing described
in \secref{sec:commandline} using the detection mechanism reads:
%
\begin{center}
|... -jobname "|\textit{target}|" "|[\textit{flags}]%
[|\def\jobname{|\textit{dest}|}|]|\input{|\textit{main}|}"|
\end{center}

%%%%%%%%%%%%%%%%%%%%%%%%%%%%%%%%%%%%%%%%%%%%%%%%%%%%%%%%%%%%%%%%%%%%%%%%%%%%%%%%
\subsection{Manual Code}
\label{sec:manual}

In case one cannot be certain whether the definitions file |childdoc.def|
is installed on the target \TeX{} distribution
and one prefers not to ship it,
it is conceivable to paste a few relevant commands into the sources.

To that end, drop all statements |\input{childdoc.def}|
and perform the replacements as outlined below.
Instead of |\childdocmain{|\textit{main}|}| add the following code
to the top of the main file:
%
\begin{center}
\begin{tabular}{l}
|\||ifdefined\childdocname\endinput\||fi\newif\ifchilddoc|\\
|\edef\childdocname{\scantokens\expandafter{\jobname\noexpand}}|\\
|\def\childdocmain{|\textit{main}|}\||ifx\childdocmain\childdocname\||else|\\
|\childdoctrue\includeonly{\childdocname}\let\jobname\childdocmain\||fi|\\
\end{tabular}
\end{center}
%
Instead of |\childdocof{|\textit{main}|}| just include the main file
at the top of each child file:
%
\begin{center}
|\input{|\textit{main}|}|
\end{center}
%
A simple redirection |\childdocforward{|\textit{dest}|}| is achieved by:
%
\begin{center}
|\def\jobname{|\textit{dest}|}\input{\jobname}|
\end{center}
%
The redirection with prefix
|\childdocforwardprefix[|\textit{prefix}|]{|\textit{dest}|}|
is accomplished by:
%
\begin{center}
\begin{tabular}{l}
|{\edef\jobname{\scantokens\expandafter{\jobname\noexpand}}|\\
|\def\redirectjob |\textit{prefix}|#1~~~{\gdef\jobname{|\textit{dest}|#1}}|\\
|\expandafter\redirectjob\jobname~~~}\input{\jobname}|
\end{tabular}
\end{center}

In an alternative approach,
child documents can be compiled by a specific command line
without additional code or specific definitions:
%
\begin{center}
|... -jobname "|\textit{target}|" "|[\textit{flags}]%
|\includeonly{|\textit{dest}|}\input{|\textit{main}|}"|
\end{center}
%

%%%%%%%%%%%%%%%%%%%%%%%%%%%%%%%%%%%%%%%%%%%%%%%%%%%%%%%%%%%%%%%%%%%%%%%%%%%%%%%%
%%%%%%%%%%%%%%%%%%%%%%%%%%%%%%%%%%%%%%%%%%%%%%%%%%%%%%%%%%%%%%%%%%%%%%%%%%%%%%%%
\section{Information}

%%%%%%%%%%%%%%%%%%%%%%%%%%%%%%%%%%%%%%%%%%%%%%%%%%%%%%%%%%%%%%%%%%%%%%%%%%%%%%%%
\subsection{Copyright}

Copyright \copyright{} 2017--2018 Niklas Beisert

This work may be distributed and/or modified under the
conditions of the \LaTeX{} Project Public License, either version 1.3
of this license or (at your option) any later version.
The latest version of this license is in
  \url{http://www.latex-project.org/lppl.txt}
and version 1.3 or later is part of all distributions of \LaTeX{}
version 2005/12/01 or later.

This work has the LPPL maintenance status `maintained'.

The Current Maintainer of this work is Niklas Beisert.

This work consists of the files |README.txt|, |childdoc.ins| and |childdoc.dtx|
as well as the derived files |childdoc.def|, |cdocsamp.tex|
with |cdocsch1.tex|, |cdocsch2.tex|, |cdocspt3.tex|, |cdocspt4.tex|,
|cdocsdrf.tex|, |cdocsfn1.tex|, |cdocsfn2.tex|
as well as |childdoc.pdf|.

%%%%%%%%%%%%%%%%%%%%%%%%%%%%%%%%%%%%%%%%%%%%%%%%%%%%%%%%%%%%%%%%%%%%%%%%%%%%%%%%
\subsection{Files and Installation}

The package consists of the files:
%
\begin{center}
\begin{tabular}{ll}
    |README.txt|   & readme file \\
    |childdoc.ins| & installation file \\
    |childdoc.dtx| & source file \\
    |childdoc.def| & definition file \\
    |cdocsamp.tex| & sample main file \\
    |cdocsch1.tex| & sample include file \\
    |cdocsch2.tex| & sample include file \\
    |cdocspt3.tex| & sample part file \\
    |cdocspt4.tex| & sample part file \\
    |cdocsdrf.tex| & sample redirection file \\
    |cdocsfn1.tex| & sample redirection file \\
    |cdocsfn2.tex| & sample redirection file \\
    |childdoc.pdf| & manual
\end{tabular}
\end{center}
%
The distribution consists of the files
|README.txt|, |childdoc.ins| and |childdoc.dtx|.
%
\begin{itemize}
\item
Run (pdf)\LaTeX{} on |childdoc.dtx|
to compile the manual |childdoc.pdf| (this file).
\item
Run \LaTeX{} on |childdoc.ins| to create the definitions file |childdoc.def|
and the sample |cdocsamp.tex| with include files
|cdocsch1.tex|, |cdocsch2.tex|, |cdocspt3.tex|, |cdocspt4.tex|,
|cdocsdrf.tex|, |cdocsfn1.tex|, |cdocsfn2.tex|.
Then copy the file |childdoc.def| to an appropriate directory of your \LaTeX{}
distribution, e.g.\ \textit{texmf-root}|/tex/latex/childdoc|.
\end{itemize}

%%%%%%%%%%%%%%%%%%%%%%%%%%%%%%%%%%%%%%%%%%%%%%%%%%%%%%%%%%%%%%%%%%%%%%%%%%%%%%%%
\subsection{Related CTAN Packages}

There are several other packages which offer a similar functionality:
%
\begin{itemize}
\item
The packages
\href{http://ctan.org/pkg/docmute}{\textsf{docmute}},
\href{http://ctan.org/pkg/includex}{\textsf{includex}} and
\href{http://ctan.org/pkg/standalone}{\textsf{standalone}}
provide commands to include only the document body of
a child file thus allowing both files to be compiled individually.
\item
The packages \href{http://ctan.org/pkg/subdocs}{\textsf{subdocs}}
and \href{http://ctan.org/pkg/subfiles}{\textsf{subfiles}}
provide structures in which the main and child documents can be
encapsulated and allowing them to be compiled individually.
The inclusion mechanism is different from the conventional |\include|.
\item
The package \href{http://ctan.org/pkg/combine}{\textsf{combine}}
is an elaborate solution to combine several documents into one.
\end{itemize}
%
See also the CTAN topic \href{http://ctan.org/topic/subdocs}{\textsf{subdocs}}
for further related packages.
The present package differs from the above solutions in that
a document structure constructed with the conventional |\include| mechanism
just needs two extra commands at the top of every file
such that all constituent files can be compiled individually.

%%%%%%%%%%%%%%%%%%%%%%%%%%%%%%%%%%%%%%%%%%%%%%%%%%%%%%%%%%%%%%%%%%%%%%%%%%%%%%%%
%\subsection{Feature Suggestions}
%
%The following is a list of features which may be useful for future
%versions of this package:
%%
%\begin{itemize}
%\item
%\ldots
%\end{itemize}

%%%%%%%%%%%%%%%%%%%%%%%%%%%%%%%%%%%%%%%%%%%%%%%%%%%%%%%%%%%%%%%%%%%%%%%%%%%%%%%%
\subsection{Revision History}

%%%%%%%%%%%%%%%%%%%%%%%%%%%%%%%%%%%%%%%%
\paragraph{v2.0:} 2018/12/30

\begin{itemize}
\item
immediate forward processing
\item
added |\childdocby| mechanism
\item
manual restructured
\end{itemize}

%%%%%%%%%%%%%%%%%%%%%%%%%%%%%%%%%%%%%%%%
\paragraph{v1.6:} 2018/01/17

\begin{itemize}
\item
application for development of include files
\item
corrections to manual
\end{itemize}

%%%%%%%%%%%%%%%%%%%%%%%%%%%%%%%%%%%%%%%%
\paragraph{v1.5:} 2017/05/21

\begin{itemize}
\item
more complete structuring introduced
\item
|\childdocof| introduced
\item
|\childdoc| renamed to |\childdocmain|
\item
|\childredirect| renamed to |\childdocforward| and |\childdocforwardprefix|
and functionality expanded
\end{itemize}

%%%%%%%%%%%%%%%%%%%%%%%%%%%%%%%%%%%%%%%%
\paragraph{v1.0:} 2017/04/27

\begin{itemize}
\item
manual and install package
\item
first version published on CTAN
\end{itemize}

%%%%%%%%%%%%%%%%%%%%%%%%%%%%%%%%%%%%%%%%
\paragraph{v0.6:} 2017/04/26

\begin{itemize}
\item
redirection mechanism added
\end{itemize}

%%%%%%%%%%%%%%%%%%%%%%%%%%%%%%%%%%%%%%%%
\paragraph{v0.5:} 2017/04/26

\begin{itemize}
\item
functionality in definition file
\end{itemize}


%%%%%%%%%%%%%%%%%%%%%%%%%%%%%%%%%%%%%%%%%%%%%%%%%%%%%%%%%%%%%%%%%%%%%%%%%%%%%%%%
%%%%%%%%%%%%%%%%%%%%%%%%%%%%%%%%%%%%%%%%%%%%%%%%%%%%%%%%%%%%%%%%%%%%%%%%%%%%%%%%
%%%%%%%%%%%%%%%%%%%%%%%%%%%%%%%%%%%%%%%%%%%%%%%%%%%%%%%%%%%%%%%%%%%%%%%%%%%%%%%%
\appendix

\settowidth\MacroIndent{\rmfamily\scriptsize 000\ }

 \DocInput{childdoc.dtx}

\end{document}
%</driver>
% \fi
%
% %%%%%%%%%%%%%%%%%%%%%%%%%%%%%%%%%%%%%%%%%%%%%%%%%%%%%%%%%%%%%%%%%%%%%%%%%%%%%%
% %%%%%%%%%%%%%%%%%%%%%%%%%%%%%%%%%%%%%%%%%%%%%%%%%%%%%%%%%%%%%%%%%%%%%%%%%%%%%%
% \section{Sample}
%\iffalse
%<*samplemain>
%\fi
%
% The following presents a sample document
% with two chapters, two parts, a title page,
% a compile flag as well as three forwarding files to set the flag.
% It consists of eight |.tex| files:
% \begin{center}
% \begin{tabular}{ll}
% |cdocsamp.tex|&main file\\
% |cdocsch1.tex|&include file for chapter 1\\
% |cdocsch2.tex|&include file for chapter 2\\
% |cdocspt3.tex|&include file for part 3\\
% |cdocspt4.tex|&include file for part 4\\
% |cdocsdrf.tex|&forwarding file for main file in draft mode\\
% |cdocsfi1.tex|&forwarding file for final version of chapter 1\\
% |cdocsfi2.tex|&forwarding file for final version of chapter 2\\
% \end{tabular}
% \end{center}
% Each of the eight files can be compiled directly by the \LaTeX{} compiler.
%
% %%%%%%%%%%%%%%%%%%%%%%%%%%%%%%%%%%%%%%
% \paragraph{Main File.}
%
% The main file is called |cdocsamp.tex|.
%
% Load the \textsf{childdoc} definitions and
% declare the filename for the main document:
%    \begin{macrocode}
\input{childdoc.def}
\childdocmain{}
%    \end{macrocode}

% Optional override for |\version| flag:
%    \begin{macrocode}
%%\ifchilddoc\else\providecommand{\version}{draft}\fi
%    \end{macrocode}

% Define the default values for the |\version| flag
% (|final| for the main file and |draft| for childs):
%    \begin{macrocode}
\ifchilddoc
\providecommand{\version}{draft}
\else
\providecommand{\version}{final}
\fi
%    \end{macrocode}

% Load the standard document class:
%    \begin{macrocode}
\documentclass[12pt]{article}
%    \end{macrocode}

% Start the document body:
%    \begin{macrocode}
\begin{document}
%    \end{macrocode}

% Declare a title page.
% Print title, part of document being processed and version flag:
%    \begin{macrocode}
\addtocounter{page}{-1}
\begin{center}
{\LARGE\bfseries{}childdoc example\par}
\vspace{1cm}
\ifchilddoc
\ifchilddocmanual part\else chapter\fi:
`\childdocname' of `\childdocjob'\par
\else
main document: `\childdocjob'\par
\fi
version: \version\par
\end{center}
\newpage
%    \end{macrocode}

% Manually include selected file,
% otherwise process as usual:
%    \begin{macrocode}
\ifchilddocmanual
\section*{part `\childdocname'}
\input{\childdocname}
\else
%    \end{macrocode}

% Include the two chapters:
%    \begin{macrocode}
\include{cdocsch1}
\include{cdocsch2}
%    \end{macrocode}

% Include the two parts unless only chapters should be displayed:
%    \begin{macrocode}
\ifchilddoc\else
\section{part three}
\input{cdocspt3}
\section{part four}
\input{cdocspt4}
\fi
%    \end{macrocode}

% Process as usual until here:
%    \begin{macrocode}
\fi
%    \end{macrocode}

% End of document body:
%    \begin{macrocode}
\end{document}
%    \end{macrocode}
%\iffalse
%</samplemain>
%\fi
%
% %%%%%%%%%%%%%%%%%%%%%%%%%%%%%%%%%%%%%%
% \paragraph{Chapter Include Files.}
%
% The include files are called |cdocsch1.tex| and |cdocsch2.tex|.
%
%\iffalse
%<*samplechap1|samplechap2>
%\fi

% Optional override for |\version| flag:
%    \begin{macrocode}
%%\providecommand{\version}{final}
%    \end{macrocode}

% Include the main document:
%    \begin{macrocode}
\input{childdoc.def}
\childdocof{cdocsamp}
%    \end{macrocode}

%\iffalse
%</samplechap1|samplechap2>
%\fi
%
%\iffalse
%<*samplechap1>
%\fi
% Some text for chapter 1:
%    \begin{macrocode}
\section{one}
some text in chapter one
%    \end{macrocode}

%\iffalse
%</samplechap1>
%\fi
% Some text for chapter 2:
%\iffalse
%<*samplechap2>
%\fi
%    \begin{macrocode}
\section{two}
more text in chapter two
%    \end{macrocode}

%\iffalse
%</samplechap2>
%\fi
%
% %%%%%%%%%%%%%%%%%%%%%%%%%%%%%%%%%%%%%%
% \paragraph{Part Include Files.}
%
% The include files are called |cdocspt3.tex| and |cdocspt4.tex|.
%
%\iffalse
%<*samplepart3|samplepart4>
%\fi

% Optional override for |\version| flag:
%    \begin{macrocode}
%%\providecommand{\version}{final}
%    \end{macrocode}

% Include the main document:
%    \begin{macrocode}
\input{childdoc.def}
\childdocby{cdocsamp}
%    \end{macrocode}

%\iffalse
%</samplepart3|samplepart4>
%\fi
%
%\iffalse
%<*samplepart3>
%\fi
% Some text for part 3:
%    \begin{macrocode}
some text in part three
%    \end{macrocode}

%\iffalse
%</samplepart3>
%\fi
% Some text for part 4:
%\iffalse
%<*samplepart4>
%\fi
%    \begin{macrocode}
more text in part four
%    \end{macrocode}

%\iffalse
%</samplepart4>
%\fi
%
% %%%%%%%%%%%%%%%%%%%%%%%%%%%%%%%%%%%%%%
% \paragraph{Forwarding for a Complete Draft.}
%
% The following forwarding file |cdocsdrf.tex|
% compiles the main document in draft mode:
%\iffalse
%<*sampledraft>
%\fi
%    \begin{macrocode}
\def\version{draft}
\input{childdoc.def}
\childdocforward{cdocsamp}
%    \end{macrocode}

%\iffalse
%</sampledraft>
%\fi
%
% %%%%%%%%%%%%%%%%%%%%%%%%%%%%%%%%%%%%%%
% \paragraph{Forwarding for Final Version of the Chapters.}
%
% The following forwarding files |cdocsfn1.tex| and |cdocsfn2.tex|
% (with identical content)
% compile the final versions of the child documents
% |cdocsch1.tex| and |cdocsch2.tex|, respectively:
%\iffalse
%<*samplefinal>
%\fi
%    \begin{macrocode}
\def\version{final}
\input{childdoc.def}
\childdocforwardprefix[cdocsamp]{cdocsfn}{cdocsch}
%    \end{macrocode}

%\iffalse
%</samplefinal>
%\fi
%
% %%%%%%%%%%%%%%%%%%%%%%%%%%%%%%%%%%%%%%
% \paragraph{Command Line Processing.}
%
% The following three command lines generate the output files
% |cdocscld|, |cdocscl1| and |cdocscl2|
% which should be identical to
% |cdocsdrf|, |cdocsch1| and |cdocsfn2|, respectively:
% \begin{center}
% \begin{tabular}{l}
% |latex -jobname cdocscld \|\\
% |  "\def\version{draft}\input{childdoc.def}\childdocforward{cdocsamp}"|\\
% |latex -jobname cdocscl1 \|\\
% |  "\input{childdoc.def}\childdocforward[cdocsamp]{cdocsch1}"|\\
% |latex -jobname cdocscl2 \|\\
% |  "\def\version{final}\input{childdoc.def}\childdocforward{cdocsch2}"|
% \end{tabular}
% \end{center}
% Note that the trailing backslash on each first line
% merely continues the input to the second line
% (for convenient cut ant paste).
% Furthermore, the command |latex| can be replaced by any
% of its alternative versions such as |pdflatex|.
%
% %%%%%%%%%%%%%%%%%%%%%%%%%%%%%%%%%%%%%%%%%%%%%%%%%%%%%%%%%%%%%%%%%%%%%%%%%%%%%%
% %%%%%%%%%%%%%%%%%%%%%%%%%%%%%%%%%%%%%%%%%%%%%%%%%%%%%%%%%%%%%%%%%%%%%%%%%%%%%%
% \section{Implementation}
%\iffalse
%<*package>
%\fi
%
% This section describes the definitions file |childdoc.def|.

% The definitions cannot be loaded using |\usepackage| or |\RequirePackage|
% which has a mechanism to prevent loading a style file more than once.
% When loading the definitions by means of |\input|
% multiple instances have to be prevented manually:
%\iffalse
%This code needs to be before the `\ProvidesFile' directive
%which is defined at the beginning of this file.
%Therefore it is also placed there and commented out here.
%</package>
%<*discard>
%\fi
%    \begin{macrocode}
\ifdefined\childdocmain\endinput\fi
%    \end{macrocode}
%\iffalse
%</discard>
%<*package>
%\fi
%
% \macro{\ifchilddoc}
% \macro{\ifchilddocmanual}
% The conditional |\ifchilddoc| tells whether a
% child (true) or main (false) document is being compiled.
% The conditional |\ifchilddocmanual| tells whether
% the |\includeonly| mechanism is used (false) or
% the selection of child files must be performed manually (true).
% The definitions initialise to false:
%    \begin{macrocode}
\newif\ifchilddoc
\newif\ifchilddocmanual
%    \end{macrocode}

% \macro{\childdocname}
% \macro{\childdocjob}
% The macro |\childdocname| stores the name of the main document
% to be compiled. The macro |\childdocjob| stores the name of
% the document on which the \LaTeX{} compiler was originally invoked.
% The content of |\jobname| cannot be compared
% to filenames specified in the source due to different catcodes.
% The following code rescans |\jobname|, stores the result
% in |\childdocname| and saves a copy in |\childdocjob|:
%    \begin{macrocode}
\edef\childdocname{\scantokens\expandafter{\jobname\noexpand}}
\let\childdocjob\childdocname
%    \end{macrocode}

% \macro{\childdocdisable}
% The macro |\childdocdisable| prevents the main file
% from being processed more than once.
% At this stage, the main document command |\childdocmain|
% is assumed to be called once again where it should do nothing.
% Any subsequent call to it should prevent
% a secondary processing of the main document
% It overwrites the forwarding commands
% |\childdocof| and |\childdocforward|
% with empty macros to prevent further inclusions of the main document:
%    \begin{macrocode}
\newcommand{\childdocdisable}
{
  \renewcommand{\childdocmain}[1]{\renewcommand{\childdocmain}[1]{\endinput}}
  \renewcommand{\childdocof}[1]{}
  \renewcommand{\childdocby}[2][]{}
  \renewcommand{\childdocforward}[2][]{}
  \renewcommand{\childdocdisable}{}
}
%    \end{macrocode}

% \macro{\childdocmain}
% The macro |\childdocmain| is to be called at the top of the main file
% with nothing or the main filename (without extension) as argument.
% First, it breaks loops.
% If the argument is not empty and does not match |\childdocname|
% (which is set by the first inclusion of |childdoc.def|),
% |\ifchilddoc| is set to true, |\includeonly| is applied to the child file
% and |\jobname| is set to the main file
% (for proper handling of |.aux| files):
%    \begin{macrocode}
\newcommand{\childdocmain}[1]
{
  \childdocdisable\childdocmain{}
  \if?#1?\else
    \begingroup
      \def\childdoctmp{#1}
      \ifx\childdoctmp\childdocname
        \def\childdoctmp{}
      \else
        \def\childdoctmp
        {
          \childdoctrue
          \includeonly{\childdocname}
          \def\childdocjob{#1}
          \def\jobname{#1}
        }
      \fi
      \expandafter
    \endgroup
    \childdoctmp
  \fi
}
%    \end{macrocode}

% \macro{\childdocof}
% The command |\childdocof| redirects
% compilation to the main file |#1|.
%    \begin{macrocode}
\newcommand{\childdocof}[1]
{
  \childdocdisable
  \childdoctrue
  \includeonly{\childdocname}
  \def\jobname{#1}
  \def\childdocjob{#1}
  \input{#1}
}
%    \end{macrocode}

% \macro{\childdocby}
% The command |\childdocby| ....
%    \begin{macrocode}
\newcommand{\childdocby}[2][]
{
  \childdocdisable
  \childdoctrue
  \childdocmanualtrue
  \if?#1?\else
    \def\jobname{#2}
  \fi
  \def\childdocjob{#2}
  \input{#2}
  \endinput
}
%    \end{macrocode}

% \macro{\childdocforward}
% The command |\childdocforward| redirects
% compilation to the main file or
% (if the optional argument is given) a child file.
% Parameters are set as if the main file
% or a child file starting with |\childdocof| was compiled.
% Then compilation is handed over to the main file:
%    \begin{macrocode}
\newcommand{\childdocforward}[2][]
{
  \begingroup
    \if?#1?
      \def\childdoctmp
      {
        \def\childdocname{#2}
        \def\childdocjob{#2}
        \def\jobname{#2}
        \input{#2}
        \endinput
      }
    \else
      \def\childdoctmp
      {
        \childdocdisable
        \def\childdocname{#2}
        \childdoctrue
        \includeonly{#2}
        \def\childdocjob{#1}
        \def\jobname{#1}
        \input{#1}
        \endinput
      }
    \fi
    \expandafter
  \endgroup
  \childdoctmp
}
%    \end{macrocode}

% \macro{\childdocforwardprefix}
% The command |\childdocforwardprefix| redirects
% compilation to the main or a child file by means of a pattern.
% The prefix |#1| in the current filename is replaced by |#2|
% and the suffix of the current filename is kept
% (it is assumed that the filename does not contain the substring `|~~~|'
% which is used as a delimiter).
% Compilation is handed over to the new file by |\childdocforward|:
%    \begin{macrocode}
\newcommand{\childdocforwardprefix}[3][]
{
  \begingroup
    \def\childdocextract #2##1~~~{\def\childdoctmp{\childdocforward[#1]{#3##1}}}
    \expandafter\childdocextract\childdocname~~~
    \expandafter
  \endgroup
  \childdoctmp
}
%    \end{macrocode}

% \macro{\childdoc}
% The deprecated macro |\childdoc| is a legacy version of |\childdocmain|:
%    \begin{macrocode}
\newcommand{\childdoc}{\childdocmain}
%    \end{macrocode}

% \macro{\childdocredirect}
% The deprecated macro |\childdocredirect| is a legacy version
% of |\childdocforward| and |\childdocforwardprefix|:
%    \begin{macrocode}
\newcommand{\childdocredirect}[2][]
{
  \begingroup
    \if?#1?
      \def\childdoctmp{\childdocforward{#2}}
    \else
      \def\childdoctmp{\childdocforwardprefix{#1}{#2}}
    \fi
    \expandafter
  \endgroup
  \childdoctmp
}
%    \end{macrocode}

%\iffalse
%</package>
%\fi
%
\endinput
|\\
|\childdocforward{|\textit{main}|}|
\end{tabular}
\end{center}
%
Likewise, the following files |final|\textit{nn}|.tex|
compile the final version of the child document
|child|\textit{nn}|.tex|:
%
\begin{center}
\begin{tabular}{l}
|\def\version{final}|\\
|% \iffalse
%
% childdoc.dtx Copyright (C) 2017-2018 Niklas Beisert
%
% This work may be distributed and/or modified under the
% conditions of the LaTeX Project Public License, either version 1.3
% of this license or (at your option) any later version.
% The latest version of this license is in
%   http://www.latex-project.org/lppl.txt
% and version 1.3 or later is part of all distributions of LaTeX
% version 2005/12/01 or later.
%
% This work has the LPPL maintenance status `maintained'.
%
% The Current Maintainer of this work is Niklas Beisert.
%
% This work consists of the files childdoc.dtx and childdoc.ins
% and the derived files childdoc.def and cdocsamp.tex with
% cdocsch1.tex, cdocsch2.tex, cdocsdrf.tex, cdocsfn1.tex, cdocsfn2.tex.
%
%<package>\ifdefined\childdocmain\endinput\fi
%<package>\ProvidesFile{childdoc.def}[2018/12/30 v2.0 child document driver]
%<samplemain>\ProvidesFile{cdocsamp.tex}[2018/12/30 v2.0 sample for childdoc]
%<*driver>
%\ProvidesFile{childdoc.drv}[2018/12/30 v2.0 childdoc reference manual file]
\PassOptionsToClass{10pt,a4paper}{article}
\documentclass{ltxdoc}

\usepackage[margin=35mm]{geometry}
\usepackage{hyperref}
\usepackage{hyperxmp}
\usepackage[usenames]{color}

\hypersetup{colorlinks=true}
\hypersetup{pdfstartview=FitH}
\hypersetup{pdfpagemode=UseNone}
\hypersetup{pdfsource={}}
\hypersetup{pdflang={en-UK}}
\hypersetup{pdfcopyright={Copyright 2017-2018 Niklas Beisert.
  This work may be distributed and/or modified under the
  conditions of the LaTeX Project Public License, either version 1.3
  of this license or (at your option) any later version.}}
\hypersetup{pdflicenseurl={http://www.latex-project.org/lppl.txt}}
\hypersetup{pdfcontactaddress={ETH Zurich, ITP, HIT K,
  Wolfgang-Pauli-Strasse 27}}
\hypersetup{pdfcontactpostcode={8093}}
\hypersetup{pdfcontactcity={Zurich}}
\hypersetup{pdfcontactcountry={Switzerland}}
\hypersetup{pdfcontactemail={nbeisert@itp.phys.ethz.ch}}
\hypersetup{pdfcontacturl={http://people.phys.ethz.ch/\xmptilde nbeisert/}}

\newcommand{\secref}[1]{\hyperref[#1]{section \ref*{#1}}}

\parskip1ex
\parindent0pt
\let\olditemize\itemize
\def\itemize{\olditemize\parskip0pt}

\begin{document}

\title{The \textsf{childdoc} Package}
\hypersetup{pdftitle={The childdoc Package}}
\author{Niklas Beisert\\[2ex]
  Institut f\"ur Theoretische Physik\\
  Eidgen\"ossische Technische Hochschule Z\"urich\\
  Wolfgang-Pauli-Strasse 27, 8093 Z\"urich, Switzerland\\[1ex]
  \href{mailto:nbeisert@itp.phys.ethz.ch}
  {\texttt{nbeisert@itp.phys.ethz.ch}}}
\hypersetup{pdfauthor={Niklas Beisert}}
\hypersetup{pdfsubject={Manual for the LaTeX2e Package childdoc}}
\date{30 December 2018, \textsf{v2.0}}
\maketitle

\begin{abstract}\noindent
\textsf{childdoc} is a \LaTeXe{} package
that enables the direct compilation
of document sections included by |\include|
to individual files.
\end{abstract}

\begingroup
\parskip0ex
\tableofcontents
\endgroup

%%%%%%%%%%%%%%%%%%%%%%%%%%%%%%%%%%%%%%%%%%%%%%%%%%%%%%%%%%%%%%%%%%%%%%%%%%%%%%%%
%%%%%%%%%%%%%%%%%%%%%%%%%%%%%%%%%%%%%%%%%%%%%%%%%%%%%%%%%%%%%%%%%%%%%%%%%%%%%%%%
\section{Introduction}

\LaTeX{} provides a mechanism to structure a large document (such as a book)
into a main file and several child files (containing the chapters)
using the |\include| command.
This mechanism is beneficial for documents
which span hundreds of pages in order to
make the source file(s) more manageable.
Moreover, compilation can be restricted to
selected child files by means of the |\includeonly| command.
The latter feature can be used to reduce the compilation time while editing
(this was significantly more useful in the earlier days of \LaTeX{})
or to generate a smaller document which is easier to navigate.
Another application of |\includeonly| is to generate
documents consisting of selected parts of the complete document.

However, there are a few drawbacks of the plain |\include| mechanism:
\begin{itemize}
\item
The child files cannot be compiled on their own,
they can only be compiled via the main file.
A naive editing environment
(such as a text editor with an option
to have the current file processed by \LaTeX)
may require one to switch to the main file before compiling;
attempting to compile the child file produces errors.
\item
The main file must be modified (each time)
to adjust the |\includeonly| command
to the present needs. This easily leaves the main file in a messy state.
\item
The generated document will always carry the filename
of the main document. This is inconvenient if
several child files are to be compiled and
to be kept for distribution.
\end{itemize}

The present package provides a simple interface
to make child files individually compilable by \LaTeX{}.
Compiling a child file then has the same effect as compiling
the main file with an |\includeonly| command
to select the appropriate child.
Moreover the generated document will carry the name of the child
rather than the main file.
This resolves all three above issues.

This feature is meant to make the editing of books,
thesis documents and lecture notes somewhat more convenient.
However, the package can also be used efficiently for
composing a series of documents (such as exercise sheets)
which are typically distributed individually.
It then assists the author in generating the individual documents
(potentially in different versions)
as well as a document containing the collected series.
Another application is in developing style files
or other kinds of included material
where compilation of the style file could redirect
to a sample or test file.

%%%%%%%%%%%%%%%%%%%%%%%%%%%%%%%%%%%%%%%%%%%%%%%%%%%%%%%%%%%%%%%%%%%%%%%%%%%%%%%%
%%%%%%%%%%%%%%%%%%%%%%%%%%%%%%%%%%%%%%%%%%%%%%%%%%%%%%%%%%%%%%%%%%%%%%%%%%%%%%%%
\section{Usage}

First of all, the package \textsf{childdoc} is \emph{not} a standard
\LaTeXe{} |.sty| style file! Therefore it needs to be invoked in
a non-standard way.

%%%%%%%%%%%%%%%%%%%%%%%%%%%%%%%%%%%%%%%%%%%%%%%%%%%%%%%%%%%%%%%%%%%%%%%%%%%%%%%%
\subsection{Included Files}
\label{sec:include}

%%%%%%%%%%%%%%%%%%%%%%%%%%%%%%%%%%%%%%%%
\DescribeMacro{\childdocmain}
To use the package, add the commands
\begin{center}
\begin{tabular}{l}
|\input{childdoc.def}|\\
|\childdocmain{}|\\
\end{tabular}
\end{center}
at the very top of the main \LaTeX{} file,
in particular \emph{before} the |\documentclass| statement!
The argument of |\childdocmain| should be left empty
(but it must be present).

%%%%%%%%%%%%%%%%%%%%%%%%%%%%%%%%%%%%%%%%
\DescribeMacro{\childdocof}
Furthermore, add the commands
\begin{center}
\begin{tabular}{l}
|\input{childdoc.def}|\\
|\childdocof{|\textit{main}|}|\\
\end{tabular}
\end{center}
at the top of every child file \textit{child}
which is included by |\include{|\textit{child}|}|
from within the main file
(or at least for those files to be compiled individually).
The argument \textit{main} must be the filename of the main file.

There are a couple of
considerations in setting up the main and child documents:

%%%%%%%%%%%%%%%%%%%%%%%%%%%%%%%%%%%%%%%%
\paragraph{Restrictions.}

Please note the following restrictions:
\begin{itemize}
\item
|\childdocmain| must be called with one argument \textit{main}
to ensure compatibility with earlier version of the package.
It must either be empty (|\childdocmain{}|)
or precisely match the filename of the main file in which it is specified.
See \secref{sec:detection} for further information.
\item
The filename \textit{main} must be specified without the |.tex| extension.
\item
The filename \textit{main} is case sensitive
(even in case-insensitive file systems)
due to internal string comparison.
\item
The argument \textit{main} should be fully expanded, it cannot be a macro.
\item
Subdirectories and special characters should be avoided in filenames.
\item
The command |\childdocmain{|\textit{main}|}| must be followed by a whitespace.
It should not be followed immediately by another command
or by a comment mark `|%|'.
This is because the \TeX{} parser reads the token immediately following
the argument of |\childdocmain| and puts it
at the beginning of every child section;
however, a white\-space is ignored.
\end{itemize}

%%%%%%%%%%%%%%%%%%%%%%%%%%%%%%%%%%%%%%%%
\paragraph{Content of Main File.}

It is advisable to place all content in the child files included by |\include|.
Any output contained in the main file will appear in all child documents
unless suppressed manually;
it cannot be suppressed automatically by the |\includeonly| directive
and thus should normally be avoided.
A method to include some content in the main file
by means of conditional processing is described in \secref{sec:conditional}.

%%%%%%%%%%%%%%%%%%%%%%%%%%%%%%%%%%%%%%%%
\paragraph{Page Numbering.}

When only a part of the document is compiled,
the appropriate numbering of pages
(as well as other status parameters)
is determined from the |.aux| files.
The latter contain information from previous passes.
However this information needs to propagate through
all intermediate child documents.
Therefore the page numbering in child documents may well
be inconsistent until the complete document is compiled at least once.

A useful (if unconventional) way to always ensure a consistent
page numbering is to restart the numbering in each child document
and denote the pages by `\textit{child}|.|\textit{page}'
where \textit{child} represents the chapter/section number of the child file.
This can be achieved by the command
|\numberwithin{page}{|\textit{child}|}|
of the \textsf{amsmath} package
where \textit{child} can be |chapter| or |section|
depending on the chosen structuring.
Alternatively, one can modify the macro |\thepage| appropriately
and reset the counter |page| at the start of each child file.

%%%%%%%%%%%%%%%%%%%%%%%%%%%%%%%%%%%%%%%%%%%%%%%%%%%%%%%%%%%%%%%%%%%%%%%%%%%%%%%%
\subsection{Conditional Processing}
\label{sec:conditional}

The package provides a mechanism to compile different versions
of a document. To customise the versions further some conditional processing
can come in handy to distinguish which version is being compiled.
The package provides two macros to describe the compilation context:

%%%%%%%%%%%%%%%%%%%%%%%%%%%%%%%%%%%%%%%%
\DescribeMacro{\ifchilddoc}
The conditional |\ifchilddoc| distinguishes between the compilation of
child documents and the main document:
%
\begin{center}
|\ifchilddoc |\textit{child-code}| |[|\||else |\textit{main-code}]| \||fi|
\end{center}

%%%%%%%%%%%%%%%%%%%%%%%%%%%%%%%%%%%%%%%%
\DescribeMacro{\childdocname}
\DescribeMacro{\childdocjob}
The macro |\childdocname| contains the filename (without extension)
of the main or child file being processed.
Note that |\childdocjob| will always contain the name of the main file.

%%%%%%%%%%%%%%%%%%%%%%%%%%%%%%%%%%%%%%%%
\paragraph{Title Page.}

Conditional processing can be used to include a title or banner page
in the main document when proper precautions are taken.
Importantly, the code in the main file should ensure that the page counter
(as well as other status parameters which are stored in the |.aux| files)
takes the same value after the conditional processing.
Otherwise the page numbers may take divergent values
depending on which part is compiled.

For example, a title page could be declared by:
%
\begin{center}
\begin{tabular}{l}
|\ifchilddoc\||else|\\
|\addtocounter{page}{-1}|\\
\textit{code for title page}\\
|\newpage|\\
|\||fi|
\end{tabular}
\end{center}
%
A banner page for the child documents can be generated by:
%
\begin{center}
\begin{tabular}{l}
|\ifchilddoc|\\
|\addtocounter{page}{-1}|\\
\textit{code for banner page}\\
|\newpage|\\
|\||fi|
\end{tabular}
\end{center}
%
Here one could write a message such as:
\begin{center}
|This is the part \childdocname{} of \childdocjob{}.|
\end{center}

%%%%%%%%%%%%%%%%%%%%%%%%%%%%%%%%%%%%%%%%%%%%%%%%%%%%%%%%%%%%%%%%%%%%%%%%%%%%%%%%
\subsection{Flags}
\label{sec:flags}

The package makes it easy to generate different versions
of the main or child documents.
To this end compilation flags can be defined
and assigned different default values.
They will be particularly useful in conjunction
with the forwarding mechanism described in \secref{sec:forward}.

For example, it may be useful to have a flag |\version|
which can be set to |draft| or |final|.
The document source will contain some conditional code
depending on the value of |\version|.
Suppose further, the flag should default to |final| for the main file
and to |draft| for child files
which is a natural assignment for editing the document.
This is achieved by placing the following code
in the preamble of the main document
(below the |\childdocmain| directive):
%
\begin{center}
\begin{tabular}{l}
|\ifchilddoc|\\
|\providecommand{\version}{draft}|\\
|\||else|\\
|\providecommand{\version}{final}|\\
|\||fi|
\end{tabular}
\end{center}
%
The definition by |\providecommand| makes sure
that previous definitions are not overwritten.
Further statements |\providecommand{\version}{...}|
can thus be added before the above code to override it.

For the main file, one might add a line
(between |\childdocmain| and the above block)
%
\begin{center}
|%\ifchilddoc\||else\providecommand{\version}{draft}\||fi|
\end{center}
%
which can be uncommented to produce a draft version.
Likewise one can add a line to the very top of a child file
(above the |\childdocof{|\textit{main}|}| directive)
%
\begin{center}
|%\providecommand{\version}{final}|
\end{center}
%
which can be uncommented to produce the final version of this child document.

%%%%%%%%%%%%%%%%%%%%%%%%%%%%%%%%%%%%%%%%%%%%%%%%%%%%%%%%%%%%%%%%%%%%%%%%%%%%%%%%
\subsection{Forwarding}
\label{sec:forward}

Different versions of the main or child documents
using compilation flags as described in \secref{sec:flags}
can be (permanently) stored in different files
for convenient compilation, viewing and distribution.
To this end, the package defines a command
to pass on compilation to a different file:

%%%%%%%%%%%%%%%%%%%%%%%%%%%%%%%%%%%%%%%%
\DescribeMacro{\childdocforward}
The command |\childdocforward| redirects processing to
another source file:
%
\begin{center}
\begin{tabular}{l}
|\input{childdoc.def}|\\
|\childdocforward[|\textit{main}|]{|\textit{dest}|}|\\
\end{tabular}
\end{center}
%
The argument \textit{dest} is the destination file
(without extension).
It should be the main file or one of the child files.
Note that further \textsf{childdoc} directives
such as |\childdocof| and |\childdocforward|
in the indicated file will be processed in this form.
The optional argument \textit{main}
passes on directly to the main file \textit{main}
while pretending to compile the child \textit{dest}.
This form behaves as if \textit{dest}
issues |\childdocof{|\textit{main}|}| right away,
and no further \textsf{childdoc} directives will be processed.

%%%%%%%%%%%%%%%%%%%%%%%%%%%%%%%%%%%%%%%%
\DescribeMacro{\...prefix}
In the alternative form |\childdocforwardprefix|,
%
\begin{center}
\begin{tabular}{l}
|\input{childdoc.def}|\\
|\childdocforwardprefix[|\textit{main}|]{|\textit{prefix}|}{|\textit{dest}|}|
\end{tabular}
\end{center}
%
the destination file is determined by a pattern
depending on the current file:
To make this work, the current file must be called
`{\textit{prefix}\hspace{0.2em}\textit{suffix}}'
with \textit{prefix} matching precisely the argument.
Processing is then passed on to the file
`{\textit{dest}\hspace{0.2em}\textit{suffix}}'.
Surely, the same effect is achieved by
directly specifying the
argument `{\textit{dest}\hspace{0.2em}\textit{suffix}}'
in the first form.
However, that requires to set up a different file
for each child. With the alternative form of the command
all these files can have exactly the same content
which simplifies setting them up and maintaining them.

For example, the following file |draft.tex|
with a compilation flag |\version| as described in \secref{sec:flags}
compiles the main document as a draft:
%
\begin{center}
\begin{tabular}{l}
|\def\version{draft}|\\
|\input{childdoc.def}|\\
|\childdocforward{|\textit{main}|}|
\end{tabular}
\end{center}
%
Likewise, the following files |final|\textit{nn}|.tex|
compile the final version of the child document
|child|\textit{nn}|.tex|:
%
\begin{center}
\begin{tabular}{l}
|\def\version{final}|\\
|\input{childdoc.def}|\\
|\childdocforwardprefix{final}{child}|
\end{tabular}
\end{center}
%

Note that when several versions of a main file and/or of each child file
are to be generated, it may be convenient to set up a |Makefile| or
shell script to automatise the process.

%%%%%%%%%%%%%%%%%%%%%%%%%%%%%%%%%%%%%%%%%%%%%%%%%%%%%%%%%%%%%%%%%%%%%%%%%%%%%%%%
\subsection{Command Line Processing}
\label{sec:commandline}

The effect of redirection files can also be achieved by invoking
the \LaTeX{} compiler with a more elaborate command line.
Most conveniently this should be done as part
of a shell script or a |Makefile|.

When using \textsf{childdoc} in the main file, the following
command lines effectively perform a redirection
(note that depending on the shell being used,
backslashes may have to be doubled: `|\|' $\to$ `|\\|'):
%
\begin{center}
|... -jobname "|\textit{target}|" |\\|"|[\textit{flags}]%
|\input{childdoc.def}\childdocforward[|\textit{main}|]{|\textit{dest}|}"|
\end{center}
%
Here \textit{target} is the name of the output file,
\textit{main} is the name of the main file
and \textit{dest} is the name of the main or child file to be processed
(all filenames without extensions).
The optional argument \textit{main} can be omitted
if \textit{main} matches \textit{dest}.
Optionally, compilation \textit{flags} can be defined via |\def| commands.
This command line makes the \TeX{} engine believe
it is compiling the file \textit{target}
whose content is specified as the latter parameter.
The provided code then forwards the processing to
\textit{main} or \textit{dest} as described in \secref{sec:forward}.

%%%%%%%%%%%%%%%%%%%%%%%%%%%%%%%%%%%%%%%%%%%%%%%%%%%%%%%%%%%%%%%%%%%%%%%%%%%%%%%%
\subsection{Include by Input}
\label{sec:input}

Including child documents by |\include| has some restrictions by design.
Most notably, the content of a child document always occupies
its own set of pages; pages cannot be shared between child documents.
Usually, this behaviour makes perfect sense
because each child document contain an essential part of the document.
However, in some situations it may be desirable to compose
a document from a collection of parts
without having mandatory page breaks between then.
For this case, the package
provides a mechanism to include parts
by |\input| which can also be processed individually.
However, by construction this mechanism
requires manual handling of the content to be output.

%%%%%%%%%%%%%%%%%%%%%%%%%%%%%%%%%%%%%%%%
\DescribeMacro{\ifchilddocmanual}
The main file should be prepared as usual, see \secref{sec:include}.
However, the document body must make a distinction
between processing of an individual part and of the main document, e.g.:
%
\begin{center}
\begin{tabular}{l}
|\ifchilddocmanual|\\
|\input{\childdocname}|\\
|\||else|\\
\textit{document body with }|\input{|\textit{part}|}|\\
|\||fi|
\end{tabular}
\end{center}
%
The conditional |\ifchilddocmanual| is true whenever
a part to be included by |\input| is being compiled,
and the name of the part is stored in |\childdocname|.

%%%%%%%%%%%%%%%%%%%%%%%%%%%%%%%%%%%%%%%%
\DescribeMacro{\childdocby}
Each part to be included by |\input| should start with:
%
\begin{center}
\begin{tabular}{l}
|\input{childdoc.def}|\\
|\childdocby{|\textit{main}|}|\\
\end{tabular}
\end{center}
%
The directive |\childdocby| is similar to |\childdocof|
described in \secref{sec:include},
but the subsequent selection of content must be done manually.
To that end, both |\ifchilddoc| and |\ifchilddocmanual|
will be true upon processing of a part,
and the name of the part is stored in |\childdocname|.
Note that |\jobname| will be set to the filename of the current part
so that each part receives an individual |.aux| file
that does not interfere with the |.aux| file(s) of the main document.
This behaviour can be altered by the alternative form
|\childdocby[*]{|\textit{main}|}| (with a non-empty optional argument)
which uses the |.aux| file of the main document
by setting |\jobname| to \textit{main}.

%%%%%%%%%%%%%%%%%%%%%%%%%%%%%%%%%%%%%%%%%%%%%%%%%%%%%%%%%%%%%%%%%%%%%%%%%%%%%%%%
\subsection{Driver Development}
\label{sec:driver}

The \textsf{childdoc} mechanism can also be use for the development
of definition files such as \LaTeX{} styles or classes.
This case differs from the above setup with multiple parts
included by |\include| in that no |\includeonly| should be invoked.
This can be achieved by starting the include file
(before |\ProvidesPackage|) with:
%
\begin{center}
\begin{tabular}{l}
|\input{childdoc.def}|\\
|\childdocforward{|\textit{main}|}|\\
\end{tabular}
\end{center}
%
or alternatively with:
%
\begin{center}
\begin{tabular}{l}
|\input{childdoc.def}|\\
|\childdocby{|\textit{main}|}|\\
\end{tabular}
\end{center}
%
Both forms have slightly different effects as described above.
The main file is prepared as usual, see \secref{sec:include}.

%%%%%%%%%%%%%%%%%%%%%%%%%%%%%%%%%%%%%%%%%%%%%%%%%%%%%%%%%%%%%%%%%%%%%%%%%%%%%%%%
\subsection{Legacy Detection}
\label{sec:detection}

The directive |\childdocmain| in the main file can detect
whether the complete document or merely a child is to be compiled
even without using the directive |\childdocof|.
This method is deprecated because it is less robust
and there is no compelling reason to use it;
it is merely provided for backward compatibility
and it may be removed in future versions.

If the detection mechanism is to be used,
it is mandatory to correctly specify
the filename of the main file as the argument of |\childdocmain|:
%
\begin{center}
\begin{tabular}{l}
|\input{childdoc.def}|\\
|\childdocmain{|\textit{main}|}|\\
\end{tabular}
\end{center}
%
If |\jobname| does not match the argument \textit{main} of |\childdocmain|,
it is assumed that |\jobname| points to the child file to be compiled.
When using |\childdocmain| with the main file specified as argument,
it suffices to start a child file
with just |\input{|\textit{main}|}|
without loading of the package and using |\childdocof|.
If instead all processing is done
with the appropriate \textsf{childdoc} directives,
the argument of \textit{main} of |\childdocmain| can be empty.

An alternative version of the command line processing described
in \secref{sec:commandline} using the detection mechanism reads:
%
\begin{center}
|... -jobname "|\textit{target}|" "|[\textit{flags}]%
[|\def\jobname{|\textit{dest}|}|]|\input{|\textit{main}|}"|
\end{center}

%%%%%%%%%%%%%%%%%%%%%%%%%%%%%%%%%%%%%%%%%%%%%%%%%%%%%%%%%%%%%%%%%%%%%%%%%%%%%%%%
\subsection{Manual Code}
\label{sec:manual}

In case one cannot be certain whether the definitions file |childdoc.def|
is installed on the target \TeX{} distribution
and one prefers not to ship it,
it is conceivable to paste a few relevant commands into the sources.

To that end, drop all statements |\input{childdoc.def}|
and perform the replacements as outlined below.
Instead of |\childdocmain{|\textit{main}|}| add the following code
to the top of the main file:
%
\begin{center}
\begin{tabular}{l}
|\||ifdefined\childdocname\endinput\||fi\newif\ifchilddoc|\\
|\edef\childdocname{\scantokens\expandafter{\jobname\noexpand}}|\\
|\def\childdocmain{|\textit{main}|}\||ifx\childdocmain\childdocname\||else|\\
|\childdoctrue\includeonly{\childdocname}\let\jobname\childdocmain\||fi|\\
\end{tabular}
\end{center}
%
Instead of |\childdocof{|\textit{main}|}| just include the main file
at the top of each child file:
%
\begin{center}
|\input{|\textit{main}|}|
\end{center}
%
A simple redirection |\childdocforward{|\textit{dest}|}| is achieved by:
%
\begin{center}
|\def\jobname{|\textit{dest}|}\input{\jobname}|
\end{center}
%
The redirection with prefix
|\childdocforwardprefix[|\textit{prefix}|]{|\textit{dest}|}|
is accomplished by:
%
\begin{center}
\begin{tabular}{l}
|{\edef\jobname{\scantokens\expandafter{\jobname\noexpand}}|\\
|\def\redirectjob |\textit{prefix}|#1~~~{\gdef\jobname{|\textit{dest}|#1}}|\\
|\expandafter\redirectjob\jobname~~~}\input{\jobname}|
\end{tabular}
\end{center}

In an alternative approach,
child documents can be compiled by a specific command line
without additional code or specific definitions:
%
\begin{center}
|... -jobname "|\textit{target}|" "|[\textit{flags}]%
|\includeonly{|\textit{dest}|}\input{|\textit{main}|}"|
\end{center}
%

%%%%%%%%%%%%%%%%%%%%%%%%%%%%%%%%%%%%%%%%%%%%%%%%%%%%%%%%%%%%%%%%%%%%%%%%%%%%%%%%
%%%%%%%%%%%%%%%%%%%%%%%%%%%%%%%%%%%%%%%%%%%%%%%%%%%%%%%%%%%%%%%%%%%%%%%%%%%%%%%%
\section{Information}

%%%%%%%%%%%%%%%%%%%%%%%%%%%%%%%%%%%%%%%%%%%%%%%%%%%%%%%%%%%%%%%%%%%%%%%%%%%%%%%%
\subsection{Copyright}

Copyright \copyright{} 2017--2018 Niklas Beisert

This work may be distributed and/or modified under the
conditions of the \LaTeX{} Project Public License, either version 1.3
of this license or (at your option) any later version.
The latest version of this license is in
  \url{http://www.latex-project.org/lppl.txt}
and version 1.3 or later is part of all distributions of \LaTeX{}
version 2005/12/01 or later.

This work has the LPPL maintenance status `maintained'.

The Current Maintainer of this work is Niklas Beisert.

This work consists of the files |README.txt|, |childdoc.ins| and |childdoc.dtx|
as well as the derived files |childdoc.def|, |cdocsamp.tex|
with |cdocsch1.tex|, |cdocsch2.tex|, |cdocspt3.tex|, |cdocspt4.tex|,
|cdocsdrf.tex|, |cdocsfn1.tex|, |cdocsfn2.tex|
as well as |childdoc.pdf|.

%%%%%%%%%%%%%%%%%%%%%%%%%%%%%%%%%%%%%%%%%%%%%%%%%%%%%%%%%%%%%%%%%%%%%%%%%%%%%%%%
\subsection{Files and Installation}

The package consists of the files:
%
\begin{center}
\begin{tabular}{ll}
    |README.txt|   & readme file \\
    |childdoc.ins| & installation file \\
    |childdoc.dtx| & source file \\
    |childdoc.def| & definition file \\
    |cdocsamp.tex| & sample main file \\
    |cdocsch1.tex| & sample include file \\
    |cdocsch2.tex| & sample include file \\
    |cdocspt3.tex| & sample part file \\
    |cdocspt4.tex| & sample part file \\
    |cdocsdrf.tex| & sample redirection file \\
    |cdocsfn1.tex| & sample redirection file \\
    |cdocsfn2.tex| & sample redirection file \\
    |childdoc.pdf| & manual
\end{tabular}
\end{center}
%
The distribution consists of the files
|README.txt|, |childdoc.ins| and |childdoc.dtx|.
%
\begin{itemize}
\item
Run (pdf)\LaTeX{} on |childdoc.dtx|
to compile the manual |childdoc.pdf| (this file).
\item
Run \LaTeX{} on |childdoc.ins| to create the definitions file |childdoc.def|
and the sample |cdocsamp.tex| with include files
|cdocsch1.tex|, |cdocsch2.tex|, |cdocspt3.tex|, |cdocspt4.tex|,
|cdocsdrf.tex|, |cdocsfn1.tex|, |cdocsfn2.tex|.
Then copy the file |childdoc.def| to an appropriate directory of your \LaTeX{}
distribution, e.g.\ \textit{texmf-root}|/tex/latex/childdoc|.
\end{itemize}

%%%%%%%%%%%%%%%%%%%%%%%%%%%%%%%%%%%%%%%%%%%%%%%%%%%%%%%%%%%%%%%%%%%%%%%%%%%%%%%%
\subsection{Related CTAN Packages}

There are several other packages which offer a similar functionality:
%
\begin{itemize}
\item
The packages
\href{http://ctan.org/pkg/docmute}{\textsf{docmute}},
\href{http://ctan.org/pkg/includex}{\textsf{includex}} and
\href{http://ctan.org/pkg/standalone}{\textsf{standalone}}
provide commands to include only the document body of
a child file thus allowing both files to be compiled individually.
\item
The packages \href{http://ctan.org/pkg/subdocs}{\textsf{subdocs}}
and \href{http://ctan.org/pkg/subfiles}{\textsf{subfiles}}
provide structures in which the main and child documents can be
encapsulated and allowing them to be compiled individually.
The inclusion mechanism is different from the conventional |\include|.
\item
The package \href{http://ctan.org/pkg/combine}{\textsf{combine}}
is an elaborate solution to combine several documents into one.
\end{itemize}
%
See also the CTAN topic \href{http://ctan.org/topic/subdocs}{\textsf{subdocs}}
for further related packages.
The present package differs from the above solutions in that
a document structure constructed with the conventional |\include| mechanism
just needs two extra commands at the top of every file
such that all constituent files can be compiled individually.

%%%%%%%%%%%%%%%%%%%%%%%%%%%%%%%%%%%%%%%%%%%%%%%%%%%%%%%%%%%%%%%%%%%%%%%%%%%%%%%%
%\subsection{Feature Suggestions}
%
%The following is a list of features which may be useful for future
%versions of this package:
%%
%\begin{itemize}
%\item
%\ldots
%\end{itemize}

%%%%%%%%%%%%%%%%%%%%%%%%%%%%%%%%%%%%%%%%%%%%%%%%%%%%%%%%%%%%%%%%%%%%%%%%%%%%%%%%
\subsection{Revision History}

%%%%%%%%%%%%%%%%%%%%%%%%%%%%%%%%%%%%%%%%
\paragraph{v2.0:} 2018/12/30

\begin{itemize}
\item
immediate forward processing
\item
added |\childdocby| mechanism
\item
manual restructured
\end{itemize}

%%%%%%%%%%%%%%%%%%%%%%%%%%%%%%%%%%%%%%%%
\paragraph{v1.6:} 2018/01/17

\begin{itemize}
\item
application for development of include files
\item
corrections to manual
\end{itemize}

%%%%%%%%%%%%%%%%%%%%%%%%%%%%%%%%%%%%%%%%
\paragraph{v1.5:} 2017/05/21

\begin{itemize}
\item
more complete structuring introduced
\item
|\childdocof| introduced
\item
|\childdoc| renamed to |\childdocmain|
\item
|\childredirect| renamed to |\childdocforward| and |\childdocforwardprefix|
and functionality expanded
\end{itemize}

%%%%%%%%%%%%%%%%%%%%%%%%%%%%%%%%%%%%%%%%
\paragraph{v1.0:} 2017/04/27

\begin{itemize}
\item
manual and install package
\item
first version published on CTAN
\end{itemize}

%%%%%%%%%%%%%%%%%%%%%%%%%%%%%%%%%%%%%%%%
\paragraph{v0.6:} 2017/04/26

\begin{itemize}
\item
redirection mechanism added
\end{itemize}

%%%%%%%%%%%%%%%%%%%%%%%%%%%%%%%%%%%%%%%%
\paragraph{v0.5:} 2017/04/26

\begin{itemize}
\item
functionality in definition file
\end{itemize}


%%%%%%%%%%%%%%%%%%%%%%%%%%%%%%%%%%%%%%%%%%%%%%%%%%%%%%%%%%%%%%%%%%%%%%%%%%%%%%%%
%%%%%%%%%%%%%%%%%%%%%%%%%%%%%%%%%%%%%%%%%%%%%%%%%%%%%%%%%%%%%%%%%%%%%%%%%%%%%%%%
%%%%%%%%%%%%%%%%%%%%%%%%%%%%%%%%%%%%%%%%%%%%%%%%%%%%%%%%%%%%%%%%%%%%%%%%%%%%%%%%
\appendix

\settowidth\MacroIndent{\rmfamily\scriptsize 000\ }

 \DocInput{childdoc.dtx}

\end{document}
%</driver>
% \fi
%
% %%%%%%%%%%%%%%%%%%%%%%%%%%%%%%%%%%%%%%%%%%%%%%%%%%%%%%%%%%%%%%%%%%%%%%%%%%%%%%
% %%%%%%%%%%%%%%%%%%%%%%%%%%%%%%%%%%%%%%%%%%%%%%%%%%%%%%%%%%%%%%%%%%%%%%%%%%%%%%
% \section{Sample}
%\iffalse
%<*samplemain>
%\fi
%
% The following presents a sample document
% with two chapters, two parts, a title page,
% a compile flag as well as three forwarding files to set the flag.
% It consists of eight |.tex| files:
% \begin{center}
% \begin{tabular}{ll}
% |cdocsamp.tex|&main file\\
% |cdocsch1.tex|&include file for chapter 1\\
% |cdocsch2.tex|&include file for chapter 2\\
% |cdocspt3.tex|&include file for part 3\\
% |cdocspt4.tex|&include file for part 4\\
% |cdocsdrf.tex|&forwarding file for main file in draft mode\\
% |cdocsfi1.tex|&forwarding file for final version of chapter 1\\
% |cdocsfi2.tex|&forwarding file for final version of chapter 2\\
% \end{tabular}
% \end{center}
% Each of the eight files can be compiled directly by the \LaTeX{} compiler.
%
% %%%%%%%%%%%%%%%%%%%%%%%%%%%%%%%%%%%%%%
% \paragraph{Main File.}
%
% The main file is called |cdocsamp.tex|.
%
% Load the \textsf{childdoc} definitions and
% declare the filename for the main document:
%    \begin{macrocode}
\input{childdoc.def}
\childdocmain{}
%    \end{macrocode}

% Optional override for |\version| flag:
%    \begin{macrocode}
%%\ifchilddoc\else\providecommand{\version}{draft}\fi
%    \end{macrocode}

% Define the default values for the |\version| flag
% (|final| for the main file and |draft| for childs):
%    \begin{macrocode}
\ifchilddoc
\providecommand{\version}{draft}
\else
\providecommand{\version}{final}
\fi
%    \end{macrocode}

% Load the standard document class:
%    \begin{macrocode}
\documentclass[12pt]{article}
%    \end{macrocode}

% Start the document body:
%    \begin{macrocode}
\begin{document}
%    \end{macrocode}

% Declare a title page.
% Print title, part of document being processed and version flag:
%    \begin{macrocode}
\addtocounter{page}{-1}
\begin{center}
{\LARGE\bfseries{}childdoc example\par}
\vspace{1cm}
\ifchilddoc
\ifchilddocmanual part\else chapter\fi:
`\childdocname' of `\childdocjob'\par
\else
main document: `\childdocjob'\par
\fi
version: \version\par
\end{center}
\newpage
%    \end{macrocode}

% Manually include selected file,
% otherwise process as usual:
%    \begin{macrocode}
\ifchilddocmanual
\section*{part `\childdocname'}
\input{\childdocname}
\else
%    \end{macrocode}

% Include the two chapters:
%    \begin{macrocode}
\include{cdocsch1}
\include{cdocsch2}
%    \end{macrocode}

% Include the two parts unless only chapters should be displayed:
%    \begin{macrocode}
\ifchilddoc\else
\section{part three}
\input{cdocspt3}
\section{part four}
\input{cdocspt4}
\fi
%    \end{macrocode}

% Process as usual until here:
%    \begin{macrocode}
\fi
%    \end{macrocode}

% End of document body:
%    \begin{macrocode}
\end{document}
%    \end{macrocode}
%\iffalse
%</samplemain>
%\fi
%
% %%%%%%%%%%%%%%%%%%%%%%%%%%%%%%%%%%%%%%
% \paragraph{Chapter Include Files.}
%
% The include files are called |cdocsch1.tex| and |cdocsch2.tex|.
%
%\iffalse
%<*samplechap1|samplechap2>
%\fi

% Optional override for |\version| flag:
%    \begin{macrocode}
%%\providecommand{\version}{final}
%    \end{macrocode}

% Include the main document:
%    \begin{macrocode}
\input{childdoc.def}
\childdocof{cdocsamp}
%    \end{macrocode}

%\iffalse
%</samplechap1|samplechap2>
%\fi
%
%\iffalse
%<*samplechap1>
%\fi
% Some text for chapter 1:
%    \begin{macrocode}
\section{one}
some text in chapter one
%    \end{macrocode}

%\iffalse
%</samplechap1>
%\fi
% Some text for chapter 2:
%\iffalse
%<*samplechap2>
%\fi
%    \begin{macrocode}
\section{two}
more text in chapter two
%    \end{macrocode}

%\iffalse
%</samplechap2>
%\fi
%
% %%%%%%%%%%%%%%%%%%%%%%%%%%%%%%%%%%%%%%
% \paragraph{Part Include Files.}
%
% The include files are called |cdocspt3.tex| and |cdocspt4.tex|.
%
%\iffalse
%<*samplepart3|samplepart4>
%\fi

% Optional override for |\version| flag:
%    \begin{macrocode}
%%\providecommand{\version}{final}
%    \end{macrocode}

% Include the main document:
%    \begin{macrocode}
\input{childdoc.def}
\childdocby{cdocsamp}
%    \end{macrocode}

%\iffalse
%</samplepart3|samplepart4>
%\fi
%
%\iffalse
%<*samplepart3>
%\fi
% Some text for part 3:
%    \begin{macrocode}
some text in part three
%    \end{macrocode}

%\iffalse
%</samplepart3>
%\fi
% Some text for part 4:
%\iffalse
%<*samplepart4>
%\fi
%    \begin{macrocode}
more text in part four
%    \end{macrocode}

%\iffalse
%</samplepart4>
%\fi
%
% %%%%%%%%%%%%%%%%%%%%%%%%%%%%%%%%%%%%%%
% \paragraph{Forwarding for a Complete Draft.}
%
% The following forwarding file |cdocsdrf.tex|
% compiles the main document in draft mode:
%\iffalse
%<*sampledraft>
%\fi
%    \begin{macrocode}
\def\version{draft}
\input{childdoc.def}
\childdocforward{cdocsamp}
%    \end{macrocode}

%\iffalse
%</sampledraft>
%\fi
%
% %%%%%%%%%%%%%%%%%%%%%%%%%%%%%%%%%%%%%%
% \paragraph{Forwarding for Final Version of the Chapters.}
%
% The following forwarding files |cdocsfn1.tex| and |cdocsfn2.tex|
% (with identical content)
% compile the final versions of the child documents
% |cdocsch1.tex| and |cdocsch2.tex|, respectively:
%\iffalse
%<*samplefinal>
%\fi
%    \begin{macrocode}
\def\version{final}
\input{childdoc.def}
\childdocforwardprefix[cdocsamp]{cdocsfn}{cdocsch}
%    \end{macrocode}

%\iffalse
%</samplefinal>
%\fi
%
% %%%%%%%%%%%%%%%%%%%%%%%%%%%%%%%%%%%%%%
% \paragraph{Command Line Processing.}
%
% The following three command lines generate the output files
% |cdocscld|, |cdocscl1| and |cdocscl2|
% which should be identical to
% |cdocsdrf|, |cdocsch1| and |cdocsfn2|, respectively:
% \begin{center}
% \begin{tabular}{l}
% |latex -jobname cdocscld \|\\
% |  "\def\version{draft}\input{childdoc.def}\childdocforward{cdocsamp}"|\\
% |latex -jobname cdocscl1 \|\\
% |  "\input{childdoc.def}\childdocforward[cdocsamp]{cdocsch1}"|\\
% |latex -jobname cdocscl2 \|\\
% |  "\def\version{final}\input{childdoc.def}\childdocforward{cdocsch2}"|
% \end{tabular}
% \end{center}
% Note that the trailing backslash on each first line
% merely continues the input to the second line
% (for convenient cut ant paste).
% Furthermore, the command |latex| can be replaced by any
% of its alternative versions such as |pdflatex|.
%
% %%%%%%%%%%%%%%%%%%%%%%%%%%%%%%%%%%%%%%%%%%%%%%%%%%%%%%%%%%%%%%%%%%%%%%%%%%%%%%
% %%%%%%%%%%%%%%%%%%%%%%%%%%%%%%%%%%%%%%%%%%%%%%%%%%%%%%%%%%%%%%%%%%%%%%%%%%%%%%
% \section{Implementation}
%\iffalse
%<*package>
%\fi
%
% This section describes the definitions file |childdoc.def|.

% The definitions cannot be loaded using |\usepackage| or |\RequirePackage|
% which has a mechanism to prevent loading a style file more than once.
% When loading the definitions by means of |\input|
% multiple instances have to be prevented manually:
%\iffalse
%This code needs to be before the `\ProvidesFile' directive
%which is defined at the beginning of this file.
%Therefore it is also placed there and commented out here.
%</package>
%<*discard>
%\fi
%    \begin{macrocode}
\ifdefined\childdocmain\endinput\fi
%    \end{macrocode}
%\iffalse
%</discard>
%<*package>
%\fi
%
% \macro{\ifchilddoc}
% \macro{\ifchilddocmanual}
% The conditional |\ifchilddoc| tells whether a
% child (true) or main (false) document is being compiled.
% The conditional |\ifchilddocmanual| tells whether
% the |\includeonly| mechanism is used (false) or
% the selection of child files must be performed manually (true).
% The definitions initialise to false:
%    \begin{macrocode}
\newif\ifchilddoc
\newif\ifchilddocmanual
%    \end{macrocode}

% \macro{\childdocname}
% \macro{\childdocjob}
% The macro |\childdocname| stores the name of the main document
% to be compiled. The macro |\childdocjob| stores the name of
% the document on which the \LaTeX{} compiler was originally invoked.
% The content of |\jobname| cannot be compared
% to filenames specified in the source due to different catcodes.
% The following code rescans |\jobname|, stores the result
% in |\childdocname| and saves a copy in |\childdocjob|:
%    \begin{macrocode}
\edef\childdocname{\scantokens\expandafter{\jobname\noexpand}}
\let\childdocjob\childdocname
%    \end{macrocode}

% \macro{\childdocdisable}
% The macro |\childdocdisable| prevents the main file
% from being processed more than once.
% At this stage, the main document command |\childdocmain|
% is assumed to be called once again where it should do nothing.
% Any subsequent call to it should prevent
% a secondary processing of the main document
% It overwrites the forwarding commands
% |\childdocof| and |\childdocforward|
% with empty macros to prevent further inclusions of the main document:
%    \begin{macrocode}
\newcommand{\childdocdisable}
{
  \renewcommand{\childdocmain}[1]{\renewcommand{\childdocmain}[1]{\endinput}}
  \renewcommand{\childdocof}[1]{}
  \renewcommand{\childdocby}[2][]{}
  \renewcommand{\childdocforward}[2][]{}
  \renewcommand{\childdocdisable}{}
}
%    \end{macrocode}

% \macro{\childdocmain}
% The macro |\childdocmain| is to be called at the top of the main file
% with nothing or the main filename (without extension) as argument.
% First, it breaks loops.
% If the argument is not empty and does not match |\childdocname|
% (which is set by the first inclusion of |childdoc.def|),
% |\ifchilddoc| is set to true, |\includeonly| is applied to the child file
% and |\jobname| is set to the main file
% (for proper handling of |.aux| files):
%    \begin{macrocode}
\newcommand{\childdocmain}[1]
{
  \childdocdisable\childdocmain{}
  \if?#1?\else
    \begingroup
      \def\childdoctmp{#1}
      \ifx\childdoctmp\childdocname
        \def\childdoctmp{}
      \else
        \def\childdoctmp
        {
          \childdoctrue
          \includeonly{\childdocname}
          \def\childdocjob{#1}
          \def\jobname{#1}
        }
      \fi
      \expandafter
    \endgroup
    \childdoctmp
  \fi
}
%    \end{macrocode}

% \macro{\childdocof}
% The command |\childdocof| redirects
% compilation to the main file |#1|.
%    \begin{macrocode}
\newcommand{\childdocof}[1]
{
  \childdocdisable
  \childdoctrue
  \includeonly{\childdocname}
  \def\jobname{#1}
  \def\childdocjob{#1}
  \input{#1}
}
%    \end{macrocode}

% \macro{\childdocby}
% The command |\childdocby| ....
%    \begin{macrocode}
\newcommand{\childdocby}[2][]
{
  \childdocdisable
  \childdoctrue
  \childdocmanualtrue
  \if?#1?\else
    \def\jobname{#2}
  \fi
  \def\childdocjob{#2}
  \input{#2}
  \endinput
}
%    \end{macrocode}

% \macro{\childdocforward}
% The command |\childdocforward| redirects
% compilation to the main file or
% (if the optional argument is given) a child file.
% Parameters are set as if the main file
% or a child file starting with |\childdocof| was compiled.
% Then compilation is handed over to the main file:
%    \begin{macrocode}
\newcommand{\childdocforward}[2][]
{
  \begingroup
    \if?#1?
      \def\childdoctmp
      {
        \def\childdocname{#2}
        \def\childdocjob{#2}
        \def\jobname{#2}
        \input{#2}
        \endinput
      }
    \else
      \def\childdoctmp
      {
        \childdocdisable
        \def\childdocname{#2}
        \childdoctrue
        \includeonly{#2}
        \def\childdocjob{#1}
        \def\jobname{#1}
        \input{#1}
        \endinput
      }
    \fi
    \expandafter
  \endgroup
  \childdoctmp
}
%    \end{macrocode}

% \macro{\childdocforwardprefix}
% The command |\childdocforwardprefix| redirects
% compilation to the main or a child file by means of a pattern.
% The prefix |#1| in the current filename is replaced by |#2|
% and the suffix of the current filename is kept
% (it is assumed that the filename does not contain the substring `|~~~|'
% which is used as a delimiter).
% Compilation is handed over to the new file by |\childdocforward|:
%    \begin{macrocode}
\newcommand{\childdocforwardprefix}[3][]
{
  \begingroup
    \def\childdocextract #2##1~~~{\def\childdoctmp{\childdocforward[#1]{#3##1}}}
    \expandafter\childdocextract\childdocname~~~
    \expandafter
  \endgroup
  \childdoctmp
}
%    \end{macrocode}

% \macro{\childdoc}
% The deprecated macro |\childdoc| is a legacy version of |\childdocmain|:
%    \begin{macrocode}
\newcommand{\childdoc}{\childdocmain}
%    \end{macrocode}

% \macro{\childdocredirect}
% The deprecated macro |\childdocredirect| is a legacy version
% of |\childdocforward| and |\childdocforwardprefix|:
%    \begin{macrocode}
\newcommand{\childdocredirect}[2][]
{
  \begingroup
    \if?#1?
      \def\childdoctmp{\childdocforward{#2}}
    \else
      \def\childdoctmp{\childdocforwardprefix{#1}{#2}}
    \fi
    \expandafter
  \endgroup
  \childdoctmp
}
%    \end{macrocode}

%\iffalse
%</package>
%\fi
%
\endinput
|\\
|\childdocforwardprefix{final}{child}|
\end{tabular}
\end{center}
%

Note that when several versions of a main file and/or of each child file
are to be generated, it may be convenient to set up a |Makefile| or
shell script to automatise the process.

%%%%%%%%%%%%%%%%%%%%%%%%%%%%%%%%%%%%%%%%%%%%%%%%%%%%%%%%%%%%%%%%%%%%%%%%%%%%%%%%
\subsection{Command Line Processing}
\label{sec:commandline}

The effect of redirection files can also be achieved by invoking
the \LaTeX{} compiler with a more elaborate command line.
Most conveniently this should be done as part
of a shell script or a |Makefile|.

When using \textsf{childdoc} in the main file, the following
command lines effectively perform a redirection
(note that depending on the shell being used,
backslashes may have to be doubled: `|\|' $\to$ `|\\|'):
%
\begin{center}
|... -jobname "|\textit{target}|" |\\|"|[\textit{flags}]%
|% \iffalse
%
% childdoc.dtx Copyright (C) 2017-2018 Niklas Beisert
%
% This work may be distributed and/or modified under the
% conditions of the LaTeX Project Public License, either version 1.3
% of this license or (at your option) any later version.
% The latest version of this license is in
%   http://www.latex-project.org/lppl.txt
% and version 1.3 or later is part of all distributions of LaTeX
% version 2005/12/01 or later.
%
% This work has the LPPL maintenance status `maintained'.
%
% The Current Maintainer of this work is Niklas Beisert.
%
% This work consists of the files childdoc.dtx and childdoc.ins
% and the derived files childdoc.def and cdocsamp.tex with
% cdocsch1.tex, cdocsch2.tex, cdocsdrf.tex, cdocsfn1.tex, cdocsfn2.tex.
%
%<package>\ifdefined\childdocmain\endinput\fi
%<package>\ProvidesFile{childdoc.def}[2018/12/30 v2.0 child document driver]
%<samplemain>\ProvidesFile{cdocsamp.tex}[2018/12/30 v2.0 sample for childdoc]
%<*driver>
%\ProvidesFile{childdoc.drv}[2018/12/30 v2.0 childdoc reference manual file]
\PassOptionsToClass{10pt,a4paper}{article}
\documentclass{ltxdoc}

\usepackage[margin=35mm]{geometry}
\usepackage{hyperref}
\usepackage{hyperxmp}
\usepackage[usenames]{color}

\hypersetup{colorlinks=true}
\hypersetup{pdfstartview=FitH}
\hypersetup{pdfpagemode=UseNone}
\hypersetup{pdfsource={}}
\hypersetup{pdflang={en-UK}}
\hypersetup{pdfcopyright={Copyright 2017-2018 Niklas Beisert.
  This work may be distributed and/or modified under the
  conditions of the LaTeX Project Public License, either version 1.3
  of this license or (at your option) any later version.}}
\hypersetup{pdflicenseurl={http://www.latex-project.org/lppl.txt}}
\hypersetup{pdfcontactaddress={ETH Zurich, ITP, HIT K,
  Wolfgang-Pauli-Strasse 27}}
\hypersetup{pdfcontactpostcode={8093}}
\hypersetup{pdfcontactcity={Zurich}}
\hypersetup{pdfcontactcountry={Switzerland}}
\hypersetup{pdfcontactemail={nbeisert@itp.phys.ethz.ch}}
\hypersetup{pdfcontacturl={http://people.phys.ethz.ch/\xmptilde nbeisert/}}

\newcommand{\secref}[1]{\hyperref[#1]{section \ref*{#1}}}

\parskip1ex
\parindent0pt
\let\olditemize\itemize
\def\itemize{\olditemize\parskip0pt}

\begin{document}

\title{The \textsf{childdoc} Package}
\hypersetup{pdftitle={The childdoc Package}}
\author{Niklas Beisert\\[2ex]
  Institut f\"ur Theoretische Physik\\
  Eidgen\"ossische Technische Hochschule Z\"urich\\
  Wolfgang-Pauli-Strasse 27, 8093 Z\"urich, Switzerland\\[1ex]
  \href{mailto:nbeisert@itp.phys.ethz.ch}
  {\texttt{nbeisert@itp.phys.ethz.ch}}}
\hypersetup{pdfauthor={Niklas Beisert}}
\hypersetup{pdfsubject={Manual for the LaTeX2e Package childdoc}}
\date{30 December 2018, \textsf{v2.0}}
\maketitle

\begin{abstract}\noindent
\textsf{childdoc} is a \LaTeXe{} package
that enables the direct compilation
of document sections included by |\include|
to individual files.
\end{abstract}

\begingroup
\parskip0ex
\tableofcontents
\endgroup

%%%%%%%%%%%%%%%%%%%%%%%%%%%%%%%%%%%%%%%%%%%%%%%%%%%%%%%%%%%%%%%%%%%%%%%%%%%%%%%%
%%%%%%%%%%%%%%%%%%%%%%%%%%%%%%%%%%%%%%%%%%%%%%%%%%%%%%%%%%%%%%%%%%%%%%%%%%%%%%%%
\section{Introduction}

\LaTeX{} provides a mechanism to structure a large document (such as a book)
into a main file and several child files (containing the chapters)
using the |\include| command.
This mechanism is beneficial for documents
which span hundreds of pages in order to
make the source file(s) more manageable.
Moreover, compilation can be restricted to
selected child files by means of the |\includeonly| command.
The latter feature can be used to reduce the compilation time while editing
(this was significantly more useful in the earlier days of \LaTeX{})
or to generate a smaller document which is easier to navigate.
Another application of |\includeonly| is to generate
documents consisting of selected parts of the complete document.

However, there are a few drawbacks of the plain |\include| mechanism:
\begin{itemize}
\item
The child files cannot be compiled on their own,
they can only be compiled via the main file.
A naive editing environment
(such as a text editor with an option
to have the current file processed by \LaTeX)
may require one to switch to the main file before compiling;
attempting to compile the child file produces errors.
\item
The main file must be modified (each time)
to adjust the |\includeonly| command
to the present needs. This easily leaves the main file in a messy state.
\item
The generated document will always carry the filename
of the main document. This is inconvenient if
several child files are to be compiled and
to be kept for distribution.
\end{itemize}

The present package provides a simple interface
to make child files individually compilable by \LaTeX{}.
Compiling a child file then has the same effect as compiling
the main file with an |\includeonly| command
to select the appropriate child.
Moreover the generated document will carry the name of the child
rather than the main file.
This resolves all three above issues.

This feature is meant to make the editing of books,
thesis documents and lecture notes somewhat more convenient.
However, the package can also be used efficiently for
composing a series of documents (such as exercise sheets)
which are typically distributed individually.
It then assists the author in generating the individual documents
(potentially in different versions)
as well as a document containing the collected series.
Another application is in developing style files
or other kinds of included material
where compilation of the style file could redirect
to a sample or test file.

%%%%%%%%%%%%%%%%%%%%%%%%%%%%%%%%%%%%%%%%%%%%%%%%%%%%%%%%%%%%%%%%%%%%%%%%%%%%%%%%
%%%%%%%%%%%%%%%%%%%%%%%%%%%%%%%%%%%%%%%%%%%%%%%%%%%%%%%%%%%%%%%%%%%%%%%%%%%%%%%%
\section{Usage}

First of all, the package \textsf{childdoc} is \emph{not} a standard
\LaTeXe{} |.sty| style file! Therefore it needs to be invoked in
a non-standard way.

%%%%%%%%%%%%%%%%%%%%%%%%%%%%%%%%%%%%%%%%%%%%%%%%%%%%%%%%%%%%%%%%%%%%%%%%%%%%%%%%
\subsection{Included Files}
\label{sec:include}

%%%%%%%%%%%%%%%%%%%%%%%%%%%%%%%%%%%%%%%%
\DescribeMacro{\childdocmain}
To use the package, add the commands
\begin{center}
\begin{tabular}{l}
|\input{childdoc.def}|\\
|\childdocmain{}|\\
\end{tabular}
\end{center}
at the very top of the main \LaTeX{} file,
in particular \emph{before} the |\documentclass| statement!
The argument of |\childdocmain| should be left empty
(but it must be present).

%%%%%%%%%%%%%%%%%%%%%%%%%%%%%%%%%%%%%%%%
\DescribeMacro{\childdocof}
Furthermore, add the commands
\begin{center}
\begin{tabular}{l}
|\input{childdoc.def}|\\
|\childdocof{|\textit{main}|}|\\
\end{tabular}
\end{center}
at the top of every child file \textit{child}
which is included by |\include{|\textit{child}|}|
from within the main file
(or at least for those files to be compiled individually).
The argument \textit{main} must be the filename of the main file.

There are a couple of
considerations in setting up the main and child documents:

%%%%%%%%%%%%%%%%%%%%%%%%%%%%%%%%%%%%%%%%
\paragraph{Restrictions.}

Please note the following restrictions:
\begin{itemize}
\item
|\childdocmain| must be called with one argument \textit{main}
to ensure compatibility with earlier version of the package.
It must either be empty (|\childdocmain{}|)
or precisely match the filename of the main file in which it is specified.
See \secref{sec:detection} for further information.
\item
The filename \textit{main} must be specified without the |.tex| extension.
\item
The filename \textit{main} is case sensitive
(even in case-insensitive file systems)
due to internal string comparison.
\item
The argument \textit{main} should be fully expanded, it cannot be a macro.
\item
Subdirectories and special characters should be avoided in filenames.
\item
The command |\childdocmain{|\textit{main}|}| must be followed by a whitespace.
It should not be followed immediately by another command
or by a comment mark `|%|'.
This is because the \TeX{} parser reads the token immediately following
the argument of |\childdocmain| and puts it
at the beginning of every child section;
however, a white\-space is ignored.
\end{itemize}

%%%%%%%%%%%%%%%%%%%%%%%%%%%%%%%%%%%%%%%%
\paragraph{Content of Main File.}

It is advisable to place all content in the child files included by |\include|.
Any output contained in the main file will appear in all child documents
unless suppressed manually;
it cannot be suppressed automatically by the |\includeonly| directive
and thus should normally be avoided.
A method to include some content in the main file
by means of conditional processing is described in \secref{sec:conditional}.

%%%%%%%%%%%%%%%%%%%%%%%%%%%%%%%%%%%%%%%%
\paragraph{Page Numbering.}

When only a part of the document is compiled,
the appropriate numbering of pages
(as well as other status parameters)
is determined from the |.aux| files.
The latter contain information from previous passes.
However this information needs to propagate through
all intermediate child documents.
Therefore the page numbering in child documents may well
be inconsistent until the complete document is compiled at least once.

A useful (if unconventional) way to always ensure a consistent
page numbering is to restart the numbering in each child document
and denote the pages by `\textit{child}|.|\textit{page}'
where \textit{child} represents the chapter/section number of the child file.
This can be achieved by the command
|\numberwithin{page}{|\textit{child}|}|
of the \textsf{amsmath} package
where \textit{child} can be |chapter| or |section|
depending on the chosen structuring.
Alternatively, one can modify the macro |\thepage| appropriately
and reset the counter |page| at the start of each child file.

%%%%%%%%%%%%%%%%%%%%%%%%%%%%%%%%%%%%%%%%%%%%%%%%%%%%%%%%%%%%%%%%%%%%%%%%%%%%%%%%
\subsection{Conditional Processing}
\label{sec:conditional}

The package provides a mechanism to compile different versions
of a document. To customise the versions further some conditional processing
can come in handy to distinguish which version is being compiled.
The package provides two macros to describe the compilation context:

%%%%%%%%%%%%%%%%%%%%%%%%%%%%%%%%%%%%%%%%
\DescribeMacro{\ifchilddoc}
The conditional |\ifchilddoc| distinguishes between the compilation of
child documents and the main document:
%
\begin{center}
|\ifchilddoc |\textit{child-code}| |[|\||else |\textit{main-code}]| \||fi|
\end{center}

%%%%%%%%%%%%%%%%%%%%%%%%%%%%%%%%%%%%%%%%
\DescribeMacro{\childdocname}
\DescribeMacro{\childdocjob}
The macro |\childdocname| contains the filename (without extension)
of the main or child file being processed.
Note that |\childdocjob| will always contain the name of the main file.

%%%%%%%%%%%%%%%%%%%%%%%%%%%%%%%%%%%%%%%%
\paragraph{Title Page.}

Conditional processing can be used to include a title or banner page
in the main document when proper precautions are taken.
Importantly, the code in the main file should ensure that the page counter
(as well as other status parameters which are stored in the |.aux| files)
takes the same value after the conditional processing.
Otherwise the page numbers may take divergent values
depending on which part is compiled.

For example, a title page could be declared by:
%
\begin{center}
\begin{tabular}{l}
|\ifchilddoc\||else|\\
|\addtocounter{page}{-1}|\\
\textit{code for title page}\\
|\newpage|\\
|\||fi|
\end{tabular}
\end{center}
%
A banner page for the child documents can be generated by:
%
\begin{center}
\begin{tabular}{l}
|\ifchilddoc|\\
|\addtocounter{page}{-1}|\\
\textit{code for banner page}\\
|\newpage|\\
|\||fi|
\end{tabular}
\end{center}
%
Here one could write a message such as:
\begin{center}
|This is the part \childdocname{} of \childdocjob{}.|
\end{center}

%%%%%%%%%%%%%%%%%%%%%%%%%%%%%%%%%%%%%%%%%%%%%%%%%%%%%%%%%%%%%%%%%%%%%%%%%%%%%%%%
\subsection{Flags}
\label{sec:flags}

The package makes it easy to generate different versions
of the main or child documents.
To this end compilation flags can be defined
and assigned different default values.
They will be particularly useful in conjunction
with the forwarding mechanism described in \secref{sec:forward}.

For example, it may be useful to have a flag |\version|
which can be set to |draft| or |final|.
The document source will contain some conditional code
depending on the value of |\version|.
Suppose further, the flag should default to |final| for the main file
and to |draft| for child files
which is a natural assignment for editing the document.
This is achieved by placing the following code
in the preamble of the main document
(below the |\childdocmain| directive):
%
\begin{center}
\begin{tabular}{l}
|\ifchilddoc|\\
|\providecommand{\version}{draft}|\\
|\||else|\\
|\providecommand{\version}{final}|\\
|\||fi|
\end{tabular}
\end{center}
%
The definition by |\providecommand| makes sure
that previous definitions are not overwritten.
Further statements |\providecommand{\version}{...}|
can thus be added before the above code to override it.

For the main file, one might add a line
(between |\childdocmain| and the above block)
%
\begin{center}
|%\ifchilddoc\||else\providecommand{\version}{draft}\||fi|
\end{center}
%
which can be uncommented to produce a draft version.
Likewise one can add a line to the very top of a child file
(above the |\childdocof{|\textit{main}|}| directive)
%
\begin{center}
|%\providecommand{\version}{final}|
\end{center}
%
which can be uncommented to produce the final version of this child document.

%%%%%%%%%%%%%%%%%%%%%%%%%%%%%%%%%%%%%%%%%%%%%%%%%%%%%%%%%%%%%%%%%%%%%%%%%%%%%%%%
\subsection{Forwarding}
\label{sec:forward}

Different versions of the main or child documents
using compilation flags as described in \secref{sec:flags}
can be (permanently) stored in different files
for convenient compilation, viewing and distribution.
To this end, the package defines a command
to pass on compilation to a different file:

%%%%%%%%%%%%%%%%%%%%%%%%%%%%%%%%%%%%%%%%
\DescribeMacro{\childdocforward}
The command |\childdocforward| redirects processing to
another source file:
%
\begin{center}
\begin{tabular}{l}
|\input{childdoc.def}|\\
|\childdocforward[|\textit{main}|]{|\textit{dest}|}|\\
\end{tabular}
\end{center}
%
The argument \textit{dest} is the destination file
(without extension).
It should be the main file or one of the child files.
Note that further \textsf{childdoc} directives
such as |\childdocof| and |\childdocforward|
in the indicated file will be processed in this form.
The optional argument \textit{main}
passes on directly to the main file \textit{main}
while pretending to compile the child \textit{dest}.
This form behaves as if \textit{dest}
issues |\childdocof{|\textit{main}|}| right away,
and no further \textsf{childdoc} directives will be processed.

%%%%%%%%%%%%%%%%%%%%%%%%%%%%%%%%%%%%%%%%
\DescribeMacro{\...prefix}
In the alternative form |\childdocforwardprefix|,
%
\begin{center}
\begin{tabular}{l}
|\input{childdoc.def}|\\
|\childdocforwardprefix[|\textit{main}|]{|\textit{prefix}|}{|\textit{dest}|}|
\end{tabular}
\end{center}
%
the destination file is determined by a pattern
depending on the current file:
To make this work, the current file must be called
`{\textit{prefix}\hspace{0.2em}\textit{suffix}}'
with \textit{prefix} matching precisely the argument.
Processing is then passed on to the file
`{\textit{dest}\hspace{0.2em}\textit{suffix}}'.
Surely, the same effect is achieved by
directly specifying the
argument `{\textit{dest}\hspace{0.2em}\textit{suffix}}'
in the first form.
However, that requires to set up a different file
for each child. With the alternative form of the command
all these files can have exactly the same content
which simplifies setting them up and maintaining them.

For example, the following file |draft.tex|
with a compilation flag |\version| as described in \secref{sec:flags}
compiles the main document as a draft:
%
\begin{center}
\begin{tabular}{l}
|\def\version{draft}|\\
|\input{childdoc.def}|\\
|\childdocforward{|\textit{main}|}|
\end{tabular}
\end{center}
%
Likewise, the following files |final|\textit{nn}|.tex|
compile the final version of the child document
|child|\textit{nn}|.tex|:
%
\begin{center}
\begin{tabular}{l}
|\def\version{final}|\\
|\input{childdoc.def}|\\
|\childdocforwardprefix{final}{child}|
\end{tabular}
\end{center}
%

Note that when several versions of a main file and/or of each child file
are to be generated, it may be convenient to set up a |Makefile| or
shell script to automatise the process.

%%%%%%%%%%%%%%%%%%%%%%%%%%%%%%%%%%%%%%%%%%%%%%%%%%%%%%%%%%%%%%%%%%%%%%%%%%%%%%%%
\subsection{Command Line Processing}
\label{sec:commandline}

The effect of redirection files can also be achieved by invoking
the \LaTeX{} compiler with a more elaborate command line.
Most conveniently this should be done as part
of a shell script or a |Makefile|.

When using \textsf{childdoc} in the main file, the following
command lines effectively perform a redirection
(note that depending on the shell being used,
backslashes may have to be doubled: `|\|' $\to$ `|\\|'):
%
\begin{center}
|... -jobname "|\textit{target}|" |\\|"|[\textit{flags}]%
|\input{childdoc.def}\childdocforward[|\textit{main}|]{|\textit{dest}|}"|
\end{center}
%
Here \textit{target} is the name of the output file,
\textit{main} is the name of the main file
and \textit{dest} is the name of the main or child file to be processed
(all filenames without extensions).
The optional argument \textit{main} can be omitted
if \textit{main} matches \textit{dest}.
Optionally, compilation \textit{flags} can be defined via |\def| commands.
This command line makes the \TeX{} engine believe
it is compiling the file \textit{target}
whose content is specified as the latter parameter.
The provided code then forwards the processing to
\textit{main} or \textit{dest} as described in \secref{sec:forward}.

%%%%%%%%%%%%%%%%%%%%%%%%%%%%%%%%%%%%%%%%%%%%%%%%%%%%%%%%%%%%%%%%%%%%%%%%%%%%%%%%
\subsection{Include by Input}
\label{sec:input}

Including child documents by |\include| has some restrictions by design.
Most notably, the content of a child document always occupies
its own set of pages; pages cannot be shared between child documents.
Usually, this behaviour makes perfect sense
because each child document contain an essential part of the document.
However, in some situations it may be desirable to compose
a document from a collection of parts
without having mandatory page breaks between then.
For this case, the package
provides a mechanism to include parts
by |\input| which can also be processed individually.
However, by construction this mechanism
requires manual handling of the content to be output.

%%%%%%%%%%%%%%%%%%%%%%%%%%%%%%%%%%%%%%%%
\DescribeMacro{\ifchilddocmanual}
The main file should be prepared as usual, see \secref{sec:include}.
However, the document body must make a distinction
between processing of an individual part and of the main document, e.g.:
%
\begin{center}
\begin{tabular}{l}
|\ifchilddocmanual|\\
|\input{\childdocname}|\\
|\||else|\\
\textit{document body with }|\input{|\textit{part}|}|\\
|\||fi|
\end{tabular}
\end{center}
%
The conditional |\ifchilddocmanual| is true whenever
a part to be included by |\input| is being compiled,
and the name of the part is stored in |\childdocname|.

%%%%%%%%%%%%%%%%%%%%%%%%%%%%%%%%%%%%%%%%
\DescribeMacro{\childdocby}
Each part to be included by |\input| should start with:
%
\begin{center}
\begin{tabular}{l}
|\input{childdoc.def}|\\
|\childdocby{|\textit{main}|}|\\
\end{tabular}
\end{center}
%
The directive |\childdocby| is similar to |\childdocof|
described in \secref{sec:include},
but the subsequent selection of content must be done manually.
To that end, both |\ifchilddoc| and |\ifchilddocmanual|
will be true upon processing of a part,
and the name of the part is stored in |\childdocname|.
Note that |\jobname| will be set to the filename of the current part
so that each part receives an individual |.aux| file
that does not interfere with the |.aux| file(s) of the main document.
This behaviour can be altered by the alternative form
|\childdocby[*]{|\textit{main}|}| (with a non-empty optional argument)
which uses the |.aux| file of the main document
by setting |\jobname| to \textit{main}.

%%%%%%%%%%%%%%%%%%%%%%%%%%%%%%%%%%%%%%%%%%%%%%%%%%%%%%%%%%%%%%%%%%%%%%%%%%%%%%%%
\subsection{Driver Development}
\label{sec:driver}

The \textsf{childdoc} mechanism can also be use for the development
of definition files such as \LaTeX{} styles or classes.
This case differs from the above setup with multiple parts
included by |\include| in that no |\includeonly| should be invoked.
This can be achieved by starting the include file
(before |\ProvidesPackage|) with:
%
\begin{center}
\begin{tabular}{l}
|\input{childdoc.def}|\\
|\childdocforward{|\textit{main}|}|\\
\end{tabular}
\end{center}
%
or alternatively with:
%
\begin{center}
\begin{tabular}{l}
|\input{childdoc.def}|\\
|\childdocby{|\textit{main}|}|\\
\end{tabular}
\end{center}
%
Both forms have slightly different effects as described above.
The main file is prepared as usual, see \secref{sec:include}.

%%%%%%%%%%%%%%%%%%%%%%%%%%%%%%%%%%%%%%%%%%%%%%%%%%%%%%%%%%%%%%%%%%%%%%%%%%%%%%%%
\subsection{Legacy Detection}
\label{sec:detection}

The directive |\childdocmain| in the main file can detect
whether the complete document or merely a child is to be compiled
even without using the directive |\childdocof|.
This method is deprecated because it is less robust
and there is no compelling reason to use it;
it is merely provided for backward compatibility
and it may be removed in future versions.

If the detection mechanism is to be used,
it is mandatory to correctly specify
the filename of the main file as the argument of |\childdocmain|:
%
\begin{center}
\begin{tabular}{l}
|\input{childdoc.def}|\\
|\childdocmain{|\textit{main}|}|\\
\end{tabular}
\end{center}
%
If |\jobname| does not match the argument \textit{main} of |\childdocmain|,
it is assumed that |\jobname| points to the child file to be compiled.
When using |\childdocmain| with the main file specified as argument,
it suffices to start a child file
with just |\input{|\textit{main}|}|
without loading of the package and using |\childdocof|.
If instead all processing is done
with the appropriate \textsf{childdoc} directives,
the argument of \textit{main} of |\childdocmain| can be empty.

An alternative version of the command line processing described
in \secref{sec:commandline} using the detection mechanism reads:
%
\begin{center}
|... -jobname "|\textit{target}|" "|[\textit{flags}]%
[|\def\jobname{|\textit{dest}|}|]|\input{|\textit{main}|}"|
\end{center}

%%%%%%%%%%%%%%%%%%%%%%%%%%%%%%%%%%%%%%%%%%%%%%%%%%%%%%%%%%%%%%%%%%%%%%%%%%%%%%%%
\subsection{Manual Code}
\label{sec:manual}

In case one cannot be certain whether the definitions file |childdoc.def|
is installed on the target \TeX{} distribution
and one prefers not to ship it,
it is conceivable to paste a few relevant commands into the sources.

To that end, drop all statements |\input{childdoc.def}|
and perform the replacements as outlined below.
Instead of |\childdocmain{|\textit{main}|}| add the following code
to the top of the main file:
%
\begin{center}
\begin{tabular}{l}
|\||ifdefined\childdocname\endinput\||fi\newif\ifchilddoc|\\
|\edef\childdocname{\scantokens\expandafter{\jobname\noexpand}}|\\
|\def\childdocmain{|\textit{main}|}\||ifx\childdocmain\childdocname\||else|\\
|\childdoctrue\includeonly{\childdocname}\let\jobname\childdocmain\||fi|\\
\end{tabular}
\end{center}
%
Instead of |\childdocof{|\textit{main}|}| just include the main file
at the top of each child file:
%
\begin{center}
|\input{|\textit{main}|}|
\end{center}
%
A simple redirection |\childdocforward{|\textit{dest}|}| is achieved by:
%
\begin{center}
|\def\jobname{|\textit{dest}|}\input{\jobname}|
\end{center}
%
The redirection with prefix
|\childdocforwardprefix[|\textit{prefix}|]{|\textit{dest}|}|
is accomplished by:
%
\begin{center}
\begin{tabular}{l}
|{\edef\jobname{\scantokens\expandafter{\jobname\noexpand}}|\\
|\def\redirectjob |\textit{prefix}|#1~~~{\gdef\jobname{|\textit{dest}|#1}}|\\
|\expandafter\redirectjob\jobname~~~}\input{\jobname}|
\end{tabular}
\end{center}

In an alternative approach,
child documents can be compiled by a specific command line
without additional code or specific definitions:
%
\begin{center}
|... -jobname "|\textit{target}|" "|[\textit{flags}]%
|\includeonly{|\textit{dest}|}\input{|\textit{main}|}"|
\end{center}
%

%%%%%%%%%%%%%%%%%%%%%%%%%%%%%%%%%%%%%%%%%%%%%%%%%%%%%%%%%%%%%%%%%%%%%%%%%%%%%%%%
%%%%%%%%%%%%%%%%%%%%%%%%%%%%%%%%%%%%%%%%%%%%%%%%%%%%%%%%%%%%%%%%%%%%%%%%%%%%%%%%
\section{Information}

%%%%%%%%%%%%%%%%%%%%%%%%%%%%%%%%%%%%%%%%%%%%%%%%%%%%%%%%%%%%%%%%%%%%%%%%%%%%%%%%
\subsection{Copyright}

Copyright \copyright{} 2017--2018 Niklas Beisert

This work may be distributed and/or modified under the
conditions of the \LaTeX{} Project Public License, either version 1.3
of this license or (at your option) any later version.
The latest version of this license is in
  \url{http://www.latex-project.org/lppl.txt}
and version 1.3 or later is part of all distributions of \LaTeX{}
version 2005/12/01 or later.

This work has the LPPL maintenance status `maintained'.

The Current Maintainer of this work is Niklas Beisert.

This work consists of the files |README.txt|, |childdoc.ins| and |childdoc.dtx|
as well as the derived files |childdoc.def|, |cdocsamp.tex|
with |cdocsch1.tex|, |cdocsch2.tex|, |cdocspt3.tex|, |cdocspt4.tex|,
|cdocsdrf.tex|, |cdocsfn1.tex|, |cdocsfn2.tex|
as well as |childdoc.pdf|.

%%%%%%%%%%%%%%%%%%%%%%%%%%%%%%%%%%%%%%%%%%%%%%%%%%%%%%%%%%%%%%%%%%%%%%%%%%%%%%%%
\subsection{Files and Installation}

The package consists of the files:
%
\begin{center}
\begin{tabular}{ll}
    |README.txt|   & readme file \\
    |childdoc.ins| & installation file \\
    |childdoc.dtx| & source file \\
    |childdoc.def| & definition file \\
    |cdocsamp.tex| & sample main file \\
    |cdocsch1.tex| & sample include file \\
    |cdocsch2.tex| & sample include file \\
    |cdocspt3.tex| & sample part file \\
    |cdocspt4.tex| & sample part file \\
    |cdocsdrf.tex| & sample redirection file \\
    |cdocsfn1.tex| & sample redirection file \\
    |cdocsfn2.tex| & sample redirection file \\
    |childdoc.pdf| & manual
\end{tabular}
\end{center}
%
The distribution consists of the files
|README.txt|, |childdoc.ins| and |childdoc.dtx|.
%
\begin{itemize}
\item
Run (pdf)\LaTeX{} on |childdoc.dtx|
to compile the manual |childdoc.pdf| (this file).
\item
Run \LaTeX{} on |childdoc.ins| to create the definitions file |childdoc.def|
and the sample |cdocsamp.tex| with include files
|cdocsch1.tex|, |cdocsch2.tex|, |cdocspt3.tex|, |cdocspt4.tex|,
|cdocsdrf.tex|, |cdocsfn1.tex|, |cdocsfn2.tex|.
Then copy the file |childdoc.def| to an appropriate directory of your \LaTeX{}
distribution, e.g.\ \textit{texmf-root}|/tex/latex/childdoc|.
\end{itemize}

%%%%%%%%%%%%%%%%%%%%%%%%%%%%%%%%%%%%%%%%%%%%%%%%%%%%%%%%%%%%%%%%%%%%%%%%%%%%%%%%
\subsection{Related CTAN Packages}

There are several other packages which offer a similar functionality:
%
\begin{itemize}
\item
The packages
\href{http://ctan.org/pkg/docmute}{\textsf{docmute}},
\href{http://ctan.org/pkg/includex}{\textsf{includex}} and
\href{http://ctan.org/pkg/standalone}{\textsf{standalone}}
provide commands to include only the document body of
a child file thus allowing both files to be compiled individually.
\item
The packages \href{http://ctan.org/pkg/subdocs}{\textsf{subdocs}}
and \href{http://ctan.org/pkg/subfiles}{\textsf{subfiles}}
provide structures in which the main and child documents can be
encapsulated and allowing them to be compiled individually.
The inclusion mechanism is different from the conventional |\include|.
\item
The package \href{http://ctan.org/pkg/combine}{\textsf{combine}}
is an elaborate solution to combine several documents into one.
\end{itemize}
%
See also the CTAN topic \href{http://ctan.org/topic/subdocs}{\textsf{subdocs}}
for further related packages.
The present package differs from the above solutions in that
a document structure constructed with the conventional |\include| mechanism
just needs two extra commands at the top of every file
such that all constituent files can be compiled individually.

%%%%%%%%%%%%%%%%%%%%%%%%%%%%%%%%%%%%%%%%%%%%%%%%%%%%%%%%%%%%%%%%%%%%%%%%%%%%%%%%
%\subsection{Feature Suggestions}
%
%The following is a list of features which may be useful for future
%versions of this package:
%%
%\begin{itemize}
%\item
%\ldots
%\end{itemize}

%%%%%%%%%%%%%%%%%%%%%%%%%%%%%%%%%%%%%%%%%%%%%%%%%%%%%%%%%%%%%%%%%%%%%%%%%%%%%%%%
\subsection{Revision History}

%%%%%%%%%%%%%%%%%%%%%%%%%%%%%%%%%%%%%%%%
\paragraph{v2.0:} 2018/12/30

\begin{itemize}
\item
immediate forward processing
\item
added |\childdocby| mechanism
\item
manual restructured
\end{itemize}

%%%%%%%%%%%%%%%%%%%%%%%%%%%%%%%%%%%%%%%%
\paragraph{v1.6:} 2018/01/17

\begin{itemize}
\item
application for development of include files
\item
corrections to manual
\end{itemize}

%%%%%%%%%%%%%%%%%%%%%%%%%%%%%%%%%%%%%%%%
\paragraph{v1.5:} 2017/05/21

\begin{itemize}
\item
more complete structuring introduced
\item
|\childdocof| introduced
\item
|\childdoc| renamed to |\childdocmain|
\item
|\childredirect| renamed to |\childdocforward| and |\childdocforwardprefix|
and functionality expanded
\end{itemize}

%%%%%%%%%%%%%%%%%%%%%%%%%%%%%%%%%%%%%%%%
\paragraph{v1.0:} 2017/04/27

\begin{itemize}
\item
manual and install package
\item
first version published on CTAN
\end{itemize}

%%%%%%%%%%%%%%%%%%%%%%%%%%%%%%%%%%%%%%%%
\paragraph{v0.6:} 2017/04/26

\begin{itemize}
\item
redirection mechanism added
\end{itemize}

%%%%%%%%%%%%%%%%%%%%%%%%%%%%%%%%%%%%%%%%
\paragraph{v0.5:} 2017/04/26

\begin{itemize}
\item
functionality in definition file
\end{itemize}


%%%%%%%%%%%%%%%%%%%%%%%%%%%%%%%%%%%%%%%%%%%%%%%%%%%%%%%%%%%%%%%%%%%%%%%%%%%%%%%%
%%%%%%%%%%%%%%%%%%%%%%%%%%%%%%%%%%%%%%%%%%%%%%%%%%%%%%%%%%%%%%%%%%%%%%%%%%%%%%%%
%%%%%%%%%%%%%%%%%%%%%%%%%%%%%%%%%%%%%%%%%%%%%%%%%%%%%%%%%%%%%%%%%%%%%%%%%%%%%%%%
\appendix

\settowidth\MacroIndent{\rmfamily\scriptsize 000\ }

 \DocInput{childdoc.dtx}

\end{document}
%</driver>
% \fi
%
% %%%%%%%%%%%%%%%%%%%%%%%%%%%%%%%%%%%%%%%%%%%%%%%%%%%%%%%%%%%%%%%%%%%%%%%%%%%%%%
% %%%%%%%%%%%%%%%%%%%%%%%%%%%%%%%%%%%%%%%%%%%%%%%%%%%%%%%%%%%%%%%%%%%%%%%%%%%%%%
% \section{Sample}
%\iffalse
%<*samplemain>
%\fi
%
% The following presents a sample document
% with two chapters, two parts, a title page,
% a compile flag as well as three forwarding files to set the flag.
% It consists of eight |.tex| files:
% \begin{center}
% \begin{tabular}{ll}
% |cdocsamp.tex|&main file\\
% |cdocsch1.tex|&include file for chapter 1\\
% |cdocsch2.tex|&include file for chapter 2\\
% |cdocspt3.tex|&include file for part 3\\
% |cdocspt4.tex|&include file for part 4\\
% |cdocsdrf.tex|&forwarding file for main file in draft mode\\
% |cdocsfi1.tex|&forwarding file for final version of chapter 1\\
% |cdocsfi2.tex|&forwarding file for final version of chapter 2\\
% \end{tabular}
% \end{center}
% Each of the eight files can be compiled directly by the \LaTeX{} compiler.
%
% %%%%%%%%%%%%%%%%%%%%%%%%%%%%%%%%%%%%%%
% \paragraph{Main File.}
%
% The main file is called |cdocsamp.tex|.
%
% Load the \textsf{childdoc} definitions and
% declare the filename for the main document:
%    \begin{macrocode}
\input{childdoc.def}
\childdocmain{}
%    \end{macrocode}

% Optional override for |\version| flag:
%    \begin{macrocode}
%%\ifchilddoc\else\providecommand{\version}{draft}\fi
%    \end{macrocode}

% Define the default values for the |\version| flag
% (|final| for the main file and |draft| for childs):
%    \begin{macrocode}
\ifchilddoc
\providecommand{\version}{draft}
\else
\providecommand{\version}{final}
\fi
%    \end{macrocode}

% Load the standard document class:
%    \begin{macrocode}
\documentclass[12pt]{article}
%    \end{macrocode}

% Start the document body:
%    \begin{macrocode}
\begin{document}
%    \end{macrocode}

% Declare a title page.
% Print title, part of document being processed and version flag:
%    \begin{macrocode}
\addtocounter{page}{-1}
\begin{center}
{\LARGE\bfseries{}childdoc example\par}
\vspace{1cm}
\ifchilddoc
\ifchilddocmanual part\else chapter\fi:
`\childdocname' of `\childdocjob'\par
\else
main document: `\childdocjob'\par
\fi
version: \version\par
\end{center}
\newpage
%    \end{macrocode}

% Manually include selected file,
% otherwise process as usual:
%    \begin{macrocode}
\ifchilddocmanual
\section*{part `\childdocname'}
\input{\childdocname}
\else
%    \end{macrocode}

% Include the two chapters:
%    \begin{macrocode}
\include{cdocsch1}
\include{cdocsch2}
%    \end{macrocode}

% Include the two parts unless only chapters should be displayed:
%    \begin{macrocode}
\ifchilddoc\else
\section{part three}
\input{cdocspt3}
\section{part four}
\input{cdocspt4}
\fi
%    \end{macrocode}

% Process as usual until here:
%    \begin{macrocode}
\fi
%    \end{macrocode}

% End of document body:
%    \begin{macrocode}
\end{document}
%    \end{macrocode}
%\iffalse
%</samplemain>
%\fi
%
% %%%%%%%%%%%%%%%%%%%%%%%%%%%%%%%%%%%%%%
% \paragraph{Chapter Include Files.}
%
% The include files are called |cdocsch1.tex| and |cdocsch2.tex|.
%
%\iffalse
%<*samplechap1|samplechap2>
%\fi

% Optional override for |\version| flag:
%    \begin{macrocode}
%%\providecommand{\version}{final}
%    \end{macrocode}

% Include the main document:
%    \begin{macrocode}
\input{childdoc.def}
\childdocof{cdocsamp}
%    \end{macrocode}

%\iffalse
%</samplechap1|samplechap2>
%\fi
%
%\iffalse
%<*samplechap1>
%\fi
% Some text for chapter 1:
%    \begin{macrocode}
\section{one}
some text in chapter one
%    \end{macrocode}

%\iffalse
%</samplechap1>
%\fi
% Some text for chapter 2:
%\iffalse
%<*samplechap2>
%\fi
%    \begin{macrocode}
\section{two}
more text in chapter two
%    \end{macrocode}

%\iffalse
%</samplechap2>
%\fi
%
% %%%%%%%%%%%%%%%%%%%%%%%%%%%%%%%%%%%%%%
% \paragraph{Part Include Files.}
%
% The include files are called |cdocspt3.tex| and |cdocspt4.tex|.
%
%\iffalse
%<*samplepart3|samplepart4>
%\fi

% Optional override for |\version| flag:
%    \begin{macrocode}
%%\providecommand{\version}{final}
%    \end{macrocode}

% Include the main document:
%    \begin{macrocode}
\input{childdoc.def}
\childdocby{cdocsamp}
%    \end{macrocode}

%\iffalse
%</samplepart3|samplepart4>
%\fi
%
%\iffalse
%<*samplepart3>
%\fi
% Some text for part 3:
%    \begin{macrocode}
some text in part three
%    \end{macrocode}

%\iffalse
%</samplepart3>
%\fi
% Some text for part 4:
%\iffalse
%<*samplepart4>
%\fi
%    \begin{macrocode}
more text in part four
%    \end{macrocode}

%\iffalse
%</samplepart4>
%\fi
%
% %%%%%%%%%%%%%%%%%%%%%%%%%%%%%%%%%%%%%%
% \paragraph{Forwarding for a Complete Draft.}
%
% The following forwarding file |cdocsdrf.tex|
% compiles the main document in draft mode:
%\iffalse
%<*sampledraft>
%\fi
%    \begin{macrocode}
\def\version{draft}
\input{childdoc.def}
\childdocforward{cdocsamp}
%    \end{macrocode}

%\iffalse
%</sampledraft>
%\fi
%
% %%%%%%%%%%%%%%%%%%%%%%%%%%%%%%%%%%%%%%
% \paragraph{Forwarding for Final Version of the Chapters.}
%
% The following forwarding files |cdocsfn1.tex| and |cdocsfn2.tex|
% (with identical content)
% compile the final versions of the child documents
% |cdocsch1.tex| and |cdocsch2.tex|, respectively:
%\iffalse
%<*samplefinal>
%\fi
%    \begin{macrocode}
\def\version{final}
\input{childdoc.def}
\childdocforwardprefix[cdocsamp]{cdocsfn}{cdocsch}
%    \end{macrocode}

%\iffalse
%</samplefinal>
%\fi
%
% %%%%%%%%%%%%%%%%%%%%%%%%%%%%%%%%%%%%%%
% \paragraph{Command Line Processing.}
%
% The following three command lines generate the output files
% |cdocscld|, |cdocscl1| and |cdocscl2|
% which should be identical to
% |cdocsdrf|, |cdocsch1| and |cdocsfn2|, respectively:
% \begin{center}
% \begin{tabular}{l}
% |latex -jobname cdocscld \|\\
% |  "\def\version{draft}\input{childdoc.def}\childdocforward{cdocsamp}"|\\
% |latex -jobname cdocscl1 \|\\
% |  "\input{childdoc.def}\childdocforward[cdocsamp]{cdocsch1}"|\\
% |latex -jobname cdocscl2 \|\\
% |  "\def\version{final}\input{childdoc.def}\childdocforward{cdocsch2}"|
% \end{tabular}
% \end{center}
% Note that the trailing backslash on each first line
% merely continues the input to the second line
% (for convenient cut ant paste).
% Furthermore, the command |latex| can be replaced by any
% of its alternative versions such as |pdflatex|.
%
% %%%%%%%%%%%%%%%%%%%%%%%%%%%%%%%%%%%%%%%%%%%%%%%%%%%%%%%%%%%%%%%%%%%%%%%%%%%%%%
% %%%%%%%%%%%%%%%%%%%%%%%%%%%%%%%%%%%%%%%%%%%%%%%%%%%%%%%%%%%%%%%%%%%%%%%%%%%%%%
% \section{Implementation}
%\iffalse
%<*package>
%\fi
%
% This section describes the definitions file |childdoc.def|.

% The definitions cannot be loaded using |\usepackage| or |\RequirePackage|
% which has a mechanism to prevent loading a style file more than once.
% When loading the definitions by means of |\input|
% multiple instances have to be prevented manually:
%\iffalse
%This code needs to be before the `\ProvidesFile' directive
%which is defined at the beginning of this file.
%Therefore it is also placed there and commented out here.
%</package>
%<*discard>
%\fi
%    \begin{macrocode}
\ifdefined\childdocmain\endinput\fi
%    \end{macrocode}
%\iffalse
%</discard>
%<*package>
%\fi
%
% \macro{\ifchilddoc}
% \macro{\ifchilddocmanual}
% The conditional |\ifchilddoc| tells whether a
% child (true) or main (false) document is being compiled.
% The conditional |\ifchilddocmanual| tells whether
% the |\includeonly| mechanism is used (false) or
% the selection of child files must be performed manually (true).
% The definitions initialise to false:
%    \begin{macrocode}
\newif\ifchilddoc
\newif\ifchilddocmanual
%    \end{macrocode}

% \macro{\childdocname}
% \macro{\childdocjob}
% The macro |\childdocname| stores the name of the main document
% to be compiled. The macro |\childdocjob| stores the name of
% the document on which the \LaTeX{} compiler was originally invoked.
% The content of |\jobname| cannot be compared
% to filenames specified in the source due to different catcodes.
% The following code rescans |\jobname|, stores the result
% in |\childdocname| and saves a copy in |\childdocjob|:
%    \begin{macrocode}
\edef\childdocname{\scantokens\expandafter{\jobname\noexpand}}
\let\childdocjob\childdocname
%    \end{macrocode}

% \macro{\childdocdisable}
% The macro |\childdocdisable| prevents the main file
% from being processed more than once.
% At this stage, the main document command |\childdocmain|
% is assumed to be called once again where it should do nothing.
% Any subsequent call to it should prevent
% a secondary processing of the main document
% It overwrites the forwarding commands
% |\childdocof| and |\childdocforward|
% with empty macros to prevent further inclusions of the main document:
%    \begin{macrocode}
\newcommand{\childdocdisable}
{
  \renewcommand{\childdocmain}[1]{\renewcommand{\childdocmain}[1]{\endinput}}
  \renewcommand{\childdocof}[1]{}
  \renewcommand{\childdocby}[2][]{}
  \renewcommand{\childdocforward}[2][]{}
  \renewcommand{\childdocdisable}{}
}
%    \end{macrocode}

% \macro{\childdocmain}
% The macro |\childdocmain| is to be called at the top of the main file
% with nothing or the main filename (without extension) as argument.
% First, it breaks loops.
% If the argument is not empty and does not match |\childdocname|
% (which is set by the first inclusion of |childdoc.def|),
% |\ifchilddoc| is set to true, |\includeonly| is applied to the child file
% and |\jobname| is set to the main file
% (for proper handling of |.aux| files):
%    \begin{macrocode}
\newcommand{\childdocmain}[1]
{
  \childdocdisable\childdocmain{}
  \if?#1?\else
    \begingroup
      \def\childdoctmp{#1}
      \ifx\childdoctmp\childdocname
        \def\childdoctmp{}
      \else
        \def\childdoctmp
        {
          \childdoctrue
          \includeonly{\childdocname}
          \def\childdocjob{#1}
          \def\jobname{#1}
        }
      \fi
      \expandafter
    \endgroup
    \childdoctmp
  \fi
}
%    \end{macrocode}

% \macro{\childdocof}
% The command |\childdocof| redirects
% compilation to the main file |#1|.
%    \begin{macrocode}
\newcommand{\childdocof}[1]
{
  \childdocdisable
  \childdoctrue
  \includeonly{\childdocname}
  \def\jobname{#1}
  \def\childdocjob{#1}
  \input{#1}
}
%    \end{macrocode}

% \macro{\childdocby}
% The command |\childdocby| ....
%    \begin{macrocode}
\newcommand{\childdocby}[2][]
{
  \childdocdisable
  \childdoctrue
  \childdocmanualtrue
  \if?#1?\else
    \def\jobname{#2}
  \fi
  \def\childdocjob{#2}
  \input{#2}
  \endinput
}
%    \end{macrocode}

% \macro{\childdocforward}
% The command |\childdocforward| redirects
% compilation to the main file or
% (if the optional argument is given) a child file.
% Parameters are set as if the main file
% or a child file starting with |\childdocof| was compiled.
% Then compilation is handed over to the main file:
%    \begin{macrocode}
\newcommand{\childdocforward}[2][]
{
  \begingroup
    \if?#1?
      \def\childdoctmp
      {
        \def\childdocname{#2}
        \def\childdocjob{#2}
        \def\jobname{#2}
        \input{#2}
        \endinput
      }
    \else
      \def\childdoctmp
      {
        \childdocdisable
        \def\childdocname{#2}
        \childdoctrue
        \includeonly{#2}
        \def\childdocjob{#1}
        \def\jobname{#1}
        \input{#1}
        \endinput
      }
    \fi
    \expandafter
  \endgroup
  \childdoctmp
}
%    \end{macrocode}

% \macro{\childdocforwardprefix}
% The command |\childdocforwardprefix| redirects
% compilation to the main or a child file by means of a pattern.
% The prefix |#1| in the current filename is replaced by |#2|
% and the suffix of the current filename is kept
% (it is assumed that the filename does not contain the substring `|~~~|'
% which is used as a delimiter).
% Compilation is handed over to the new file by |\childdocforward|:
%    \begin{macrocode}
\newcommand{\childdocforwardprefix}[3][]
{
  \begingroup
    \def\childdocextract #2##1~~~{\def\childdoctmp{\childdocforward[#1]{#3##1}}}
    \expandafter\childdocextract\childdocname~~~
    \expandafter
  \endgroup
  \childdoctmp
}
%    \end{macrocode}

% \macro{\childdoc}
% The deprecated macro |\childdoc| is a legacy version of |\childdocmain|:
%    \begin{macrocode}
\newcommand{\childdoc}{\childdocmain}
%    \end{macrocode}

% \macro{\childdocredirect}
% The deprecated macro |\childdocredirect| is a legacy version
% of |\childdocforward| and |\childdocforwardprefix|:
%    \begin{macrocode}
\newcommand{\childdocredirect}[2][]
{
  \begingroup
    \if?#1?
      \def\childdoctmp{\childdocforward{#2}}
    \else
      \def\childdoctmp{\childdocforwardprefix{#1}{#2}}
    \fi
    \expandafter
  \endgroup
  \childdoctmp
}
%    \end{macrocode}

%\iffalse
%</package>
%\fi
%
\endinput
\childdocforward[|\textit{main}|]{|\textit{dest}|}"|
\end{center}
%
Here \textit{target} is the name of the output file,
\textit{main} is the name of the main file
and \textit{dest} is the name of the main or child file to be processed
(all filenames without extensions).
The optional argument \textit{main} can be omitted
if \textit{main} matches \textit{dest}.
Optionally, compilation \textit{flags} can be defined via |\def| commands.
This command line makes the \TeX{} engine believe
it is compiling the file \textit{target}
whose content is specified as the latter parameter.
The provided code then forwards the processing to
\textit{main} or \textit{dest} as described in \secref{sec:forward}.

%%%%%%%%%%%%%%%%%%%%%%%%%%%%%%%%%%%%%%%%%%%%%%%%%%%%%%%%%%%%%%%%%%%%%%%%%%%%%%%%
\subsection{Include by Input}
\label{sec:input}

Including child documents by |\include| has some restrictions by design.
Most notably, the content of a child document always occupies
its own set of pages; pages cannot be shared between child documents.
Usually, this behaviour makes perfect sense
because each child document contain an essential part of the document.
However, in some situations it may be desirable to compose
a document from a collection of parts
without having mandatory page breaks between then.
For this case, the package
provides a mechanism to include parts
by |\input| which can also be processed individually.
However, by construction this mechanism
requires manual handling of the content to be output.

%%%%%%%%%%%%%%%%%%%%%%%%%%%%%%%%%%%%%%%%
\DescribeMacro{\ifchilddocmanual}
The main file should be prepared as usual, see \secref{sec:include}.
However, the document body must make a distinction
between processing of an individual part and of the main document, e.g.:
%
\begin{center}
\begin{tabular}{l}
|\ifchilddocmanual|\\
|\input{\childdocname}|\\
|\||else|\\
\textit{document body with }|\input{|\textit{part}|}|\\
|\||fi|
\end{tabular}
\end{center}
%
The conditional |\ifchilddocmanual| is true whenever
a part to be included by |\input| is being compiled,
and the name of the part is stored in |\childdocname|.

%%%%%%%%%%%%%%%%%%%%%%%%%%%%%%%%%%%%%%%%
\DescribeMacro{\childdocby}
Each part to be included by |\input| should start with:
%
\begin{center}
\begin{tabular}{l}
|% \iffalse
%
% childdoc.dtx Copyright (C) 2017-2018 Niklas Beisert
%
% This work may be distributed and/or modified under the
% conditions of the LaTeX Project Public License, either version 1.3
% of this license or (at your option) any later version.
% The latest version of this license is in
%   http://www.latex-project.org/lppl.txt
% and version 1.3 or later is part of all distributions of LaTeX
% version 2005/12/01 or later.
%
% This work has the LPPL maintenance status `maintained'.
%
% The Current Maintainer of this work is Niklas Beisert.
%
% This work consists of the files childdoc.dtx and childdoc.ins
% and the derived files childdoc.def and cdocsamp.tex with
% cdocsch1.tex, cdocsch2.tex, cdocsdrf.tex, cdocsfn1.tex, cdocsfn2.tex.
%
%<package>\ifdefined\childdocmain\endinput\fi
%<package>\ProvidesFile{childdoc.def}[2018/12/30 v2.0 child document driver]
%<samplemain>\ProvidesFile{cdocsamp.tex}[2018/12/30 v2.0 sample for childdoc]
%<*driver>
%\ProvidesFile{childdoc.drv}[2018/12/30 v2.0 childdoc reference manual file]
\PassOptionsToClass{10pt,a4paper}{article}
\documentclass{ltxdoc}

\usepackage[margin=35mm]{geometry}
\usepackage{hyperref}
\usepackage{hyperxmp}
\usepackage[usenames]{color}

\hypersetup{colorlinks=true}
\hypersetup{pdfstartview=FitH}
\hypersetup{pdfpagemode=UseNone}
\hypersetup{pdfsource={}}
\hypersetup{pdflang={en-UK}}
\hypersetup{pdfcopyright={Copyright 2017-2018 Niklas Beisert.
  This work may be distributed and/or modified under the
  conditions of the LaTeX Project Public License, either version 1.3
  of this license or (at your option) any later version.}}
\hypersetup{pdflicenseurl={http://www.latex-project.org/lppl.txt}}
\hypersetup{pdfcontactaddress={ETH Zurich, ITP, HIT K,
  Wolfgang-Pauli-Strasse 27}}
\hypersetup{pdfcontactpostcode={8093}}
\hypersetup{pdfcontactcity={Zurich}}
\hypersetup{pdfcontactcountry={Switzerland}}
\hypersetup{pdfcontactemail={nbeisert@itp.phys.ethz.ch}}
\hypersetup{pdfcontacturl={http://people.phys.ethz.ch/\xmptilde nbeisert/}}

\newcommand{\secref}[1]{\hyperref[#1]{section \ref*{#1}}}

\parskip1ex
\parindent0pt
\let\olditemize\itemize
\def\itemize{\olditemize\parskip0pt}

\begin{document}

\title{The \textsf{childdoc} Package}
\hypersetup{pdftitle={The childdoc Package}}
\author{Niklas Beisert\\[2ex]
  Institut f\"ur Theoretische Physik\\
  Eidgen\"ossische Technische Hochschule Z\"urich\\
  Wolfgang-Pauli-Strasse 27, 8093 Z\"urich, Switzerland\\[1ex]
  \href{mailto:nbeisert@itp.phys.ethz.ch}
  {\texttt{nbeisert@itp.phys.ethz.ch}}}
\hypersetup{pdfauthor={Niklas Beisert}}
\hypersetup{pdfsubject={Manual for the LaTeX2e Package childdoc}}
\date{30 December 2018, \textsf{v2.0}}
\maketitle

\begin{abstract}\noindent
\textsf{childdoc} is a \LaTeXe{} package
that enables the direct compilation
of document sections included by |\include|
to individual files.
\end{abstract}

\begingroup
\parskip0ex
\tableofcontents
\endgroup

%%%%%%%%%%%%%%%%%%%%%%%%%%%%%%%%%%%%%%%%%%%%%%%%%%%%%%%%%%%%%%%%%%%%%%%%%%%%%%%%
%%%%%%%%%%%%%%%%%%%%%%%%%%%%%%%%%%%%%%%%%%%%%%%%%%%%%%%%%%%%%%%%%%%%%%%%%%%%%%%%
\section{Introduction}

\LaTeX{} provides a mechanism to structure a large document (such as a book)
into a main file and several child files (containing the chapters)
using the |\include| command.
This mechanism is beneficial for documents
which span hundreds of pages in order to
make the source file(s) more manageable.
Moreover, compilation can be restricted to
selected child files by means of the |\includeonly| command.
The latter feature can be used to reduce the compilation time while editing
(this was significantly more useful in the earlier days of \LaTeX{})
or to generate a smaller document which is easier to navigate.
Another application of |\includeonly| is to generate
documents consisting of selected parts of the complete document.

However, there are a few drawbacks of the plain |\include| mechanism:
\begin{itemize}
\item
The child files cannot be compiled on their own,
they can only be compiled via the main file.
A naive editing environment
(such as a text editor with an option
to have the current file processed by \LaTeX)
may require one to switch to the main file before compiling;
attempting to compile the child file produces errors.
\item
The main file must be modified (each time)
to adjust the |\includeonly| command
to the present needs. This easily leaves the main file in a messy state.
\item
The generated document will always carry the filename
of the main document. This is inconvenient if
several child files are to be compiled and
to be kept for distribution.
\end{itemize}

The present package provides a simple interface
to make child files individually compilable by \LaTeX{}.
Compiling a child file then has the same effect as compiling
the main file with an |\includeonly| command
to select the appropriate child.
Moreover the generated document will carry the name of the child
rather than the main file.
This resolves all three above issues.

This feature is meant to make the editing of books,
thesis documents and lecture notes somewhat more convenient.
However, the package can also be used efficiently for
composing a series of documents (such as exercise sheets)
which are typically distributed individually.
It then assists the author in generating the individual documents
(potentially in different versions)
as well as a document containing the collected series.
Another application is in developing style files
or other kinds of included material
where compilation of the style file could redirect
to a sample or test file.

%%%%%%%%%%%%%%%%%%%%%%%%%%%%%%%%%%%%%%%%%%%%%%%%%%%%%%%%%%%%%%%%%%%%%%%%%%%%%%%%
%%%%%%%%%%%%%%%%%%%%%%%%%%%%%%%%%%%%%%%%%%%%%%%%%%%%%%%%%%%%%%%%%%%%%%%%%%%%%%%%
\section{Usage}

First of all, the package \textsf{childdoc} is \emph{not} a standard
\LaTeXe{} |.sty| style file! Therefore it needs to be invoked in
a non-standard way.

%%%%%%%%%%%%%%%%%%%%%%%%%%%%%%%%%%%%%%%%%%%%%%%%%%%%%%%%%%%%%%%%%%%%%%%%%%%%%%%%
\subsection{Included Files}
\label{sec:include}

%%%%%%%%%%%%%%%%%%%%%%%%%%%%%%%%%%%%%%%%
\DescribeMacro{\childdocmain}
To use the package, add the commands
\begin{center}
\begin{tabular}{l}
|\input{childdoc.def}|\\
|\childdocmain{}|\\
\end{tabular}
\end{center}
at the very top of the main \LaTeX{} file,
in particular \emph{before} the |\documentclass| statement!
The argument of |\childdocmain| should be left empty
(but it must be present).

%%%%%%%%%%%%%%%%%%%%%%%%%%%%%%%%%%%%%%%%
\DescribeMacro{\childdocof}
Furthermore, add the commands
\begin{center}
\begin{tabular}{l}
|\input{childdoc.def}|\\
|\childdocof{|\textit{main}|}|\\
\end{tabular}
\end{center}
at the top of every child file \textit{child}
which is included by |\include{|\textit{child}|}|
from within the main file
(or at least for those files to be compiled individually).
The argument \textit{main} must be the filename of the main file.

There are a couple of
considerations in setting up the main and child documents:

%%%%%%%%%%%%%%%%%%%%%%%%%%%%%%%%%%%%%%%%
\paragraph{Restrictions.}

Please note the following restrictions:
\begin{itemize}
\item
|\childdocmain| must be called with one argument \textit{main}
to ensure compatibility with earlier version of the package.
It must either be empty (|\childdocmain{}|)
or precisely match the filename of the main file in which it is specified.
See \secref{sec:detection} for further information.
\item
The filename \textit{main} must be specified without the |.tex| extension.
\item
The filename \textit{main} is case sensitive
(even in case-insensitive file systems)
due to internal string comparison.
\item
The argument \textit{main} should be fully expanded, it cannot be a macro.
\item
Subdirectories and special characters should be avoided in filenames.
\item
The command |\childdocmain{|\textit{main}|}| must be followed by a whitespace.
It should not be followed immediately by another command
or by a comment mark `|%|'.
This is because the \TeX{} parser reads the token immediately following
the argument of |\childdocmain| and puts it
at the beginning of every child section;
however, a white\-space is ignored.
\end{itemize}

%%%%%%%%%%%%%%%%%%%%%%%%%%%%%%%%%%%%%%%%
\paragraph{Content of Main File.}

It is advisable to place all content in the child files included by |\include|.
Any output contained in the main file will appear in all child documents
unless suppressed manually;
it cannot be suppressed automatically by the |\includeonly| directive
and thus should normally be avoided.
A method to include some content in the main file
by means of conditional processing is described in \secref{sec:conditional}.

%%%%%%%%%%%%%%%%%%%%%%%%%%%%%%%%%%%%%%%%
\paragraph{Page Numbering.}

When only a part of the document is compiled,
the appropriate numbering of pages
(as well as other status parameters)
is determined from the |.aux| files.
The latter contain information from previous passes.
However this information needs to propagate through
all intermediate child documents.
Therefore the page numbering in child documents may well
be inconsistent until the complete document is compiled at least once.

A useful (if unconventional) way to always ensure a consistent
page numbering is to restart the numbering in each child document
and denote the pages by `\textit{child}|.|\textit{page}'
where \textit{child} represents the chapter/section number of the child file.
This can be achieved by the command
|\numberwithin{page}{|\textit{child}|}|
of the \textsf{amsmath} package
where \textit{child} can be |chapter| or |section|
depending on the chosen structuring.
Alternatively, one can modify the macro |\thepage| appropriately
and reset the counter |page| at the start of each child file.

%%%%%%%%%%%%%%%%%%%%%%%%%%%%%%%%%%%%%%%%%%%%%%%%%%%%%%%%%%%%%%%%%%%%%%%%%%%%%%%%
\subsection{Conditional Processing}
\label{sec:conditional}

The package provides a mechanism to compile different versions
of a document. To customise the versions further some conditional processing
can come in handy to distinguish which version is being compiled.
The package provides two macros to describe the compilation context:

%%%%%%%%%%%%%%%%%%%%%%%%%%%%%%%%%%%%%%%%
\DescribeMacro{\ifchilddoc}
The conditional |\ifchilddoc| distinguishes between the compilation of
child documents and the main document:
%
\begin{center}
|\ifchilddoc |\textit{child-code}| |[|\||else |\textit{main-code}]| \||fi|
\end{center}

%%%%%%%%%%%%%%%%%%%%%%%%%%%%%%%%%%%%%%%%
\DescribeMacro{\childdocname}
\DescribeMacro{\childdocjob}
The macro |\childdocname| contains the filename (without extension)
of the main or child file being processed.
Note that |\childdocjob| will always contain the name of the main file.

%%%%%%%%%%%%%%%%%%%%%%%%%%%%%%%%%%%%%%%%
\paragraph{Title Page.}

Conditional processing can be used to include a title or banner page
in the main document when proper precautions are taken.
Importantly, the code in the main file should ensure that the page counter
(as well as other status parameters which are stored in the |.aux| files)
takes the same value after the conditional processing.
Otherwise the page numbers may take divergent values
depending on which part is compiled.

For example, a title page could be declared by:
%
\begin{center}
\begin{tabular}{l}
|\ifchilddoc\||else|\\
|\addtocounter{page}{-1}|\\
\textit{code for title page}\\
|\newpage|\\
|\||fi|
\end{tabular}
\end{center}
%
A banner page for the child documents can be generated by:
%
\begin{center}
\begin{tabular}{l}
|\ifchilddoc|\\
|\addtocounter{page}{-1}|\\
\textit{code for banner page}\\
|\newpage|\\
|\||fi|
\end{tabular}
\end{center}
%
Here one could write a message such as:
\begin{center}
|This is the part \childdocname{} of \childdocjob{}.|
\end{center}

%%%%%%%%%%%%%%%%%%%%%%%%%%%%%%%%%%%%%%%%%%%%%%%%%%%%%%%%%%%%%%%%%%%%%%%%%%%%%%%%
\subsection{Flags}
\label{sec:flags}

The package makes it easy to generate different versions
of the main or child documents.
To this end compilation flags can be defined
and assigned different default values.
They will be particularly useful in conjunction
with the forwarding mechanism described in \secref{sec:forward}.

For example, it may be useful to have a flag |\version|
which can be set to |draft| or |final|.
The document source will contain some conditional code
depending on the value of |\version|.
Suppose further, the flag should default to |final| for the main file
and to |draft| for child files
which is a natural assignment for editing the document.
This is achieved by placing the following code
in the preamble of the main document
(below the |\childdocmain| directive):
%
\begin{center}
\begin{tabular}{l}
|\ifchilddoc|\\
|\providecommand{\version}{draft}|\\
|\||else|\\
|\providecommand{\version}{final}|\\
|\||fi|
\end{tabular}
\end{center}
%
The definition by |\providecommand| makes sure
that previous definitions are not overwritten.
Further statements |\providecommand{\version}{...}|
can thus be added before the above code to override it.

For the main file, one might add a line
(between |\childdocmain| and the above block)
%
\begin{center}
|%\ifchilddoc\||else\providecommand{\version}{draft}\||fi|
\end{center}
%
which can be uncommented to produce a draft version.
Likewise one can add a line to the very top of a child file
(above the |\childdocof{|\textit{main}|}| directive)
%
\begin{center}
|%\providecommand{\version}{final}|
\end{center}
%
which can be uncommented to produce the final version of this child document.

%%%%%%%%%%%%%%%%%%%%%%%%%%%%%%%%%%%%%%%%%%%%%%%%%%%%%%%%%%%%%%%%%%%%%%%%%%%%%%%%
\subsection{Forwarding}
\label{sec:forward}

Different versions of the main or child documents
using compilation flags as described in \secref{sec:flags}
can be (permanently) stored in different files
for convenient compilation, viewing and distribution.
To this end, the package defines a command
to pass on compilation to a different file:

%%%%%%%%%%%%%%%%%%%%%%%%%%%%%%%%%%%%%%%%
\DescribeMacro{\childdocforward}
The command |\childdocforward| redirects processing to
another source file:
%
\begin{center}
\begin{tabular}{l}
|\input{childdoc.def}|\\
|\childdocforward[|\textit{main}|]{|\textit{dest}|}|\\
\end{tabular}
\end{center}
%
The argument \textit{dest} is the destination file
(without extension).
It should be the main file or one of the child files.
Note that further \textsf{childdoc} directives
such as |\childdocof| and |\childdocforward|
in the indicated file will be processed in this form.
The optional argument \textit{main}
passes on directly to the main file \textit{main}
while pretending to compile the child \textit{dest}.
This form behaves as if \textit{dest}
issues |\childdocof{|\textit{main}|}| right away,
and no further \textsf{childdoc} directives will be processed.

%%%%%%%%%%%%%%%%%%%%%%%%%%%%%%%%%%%%%%%%
\DescribeMacro{\...prefix}
In the alternative form |\childdocforwardprefix|,
%
\begin{center}
\begin{tabular}{l}
|\input{childdoc.def}|\\
|\childdocforwardprefix[|\textit{main}|]{|\textit{prefix}|}{|\textit{dest}|}|
\end{tabular}
\end{center}
%
the destination file is determined by a pattern
depending on the current file:
To make this work, the current file must be called
`{\textit{prefix}\hspace{0.2em}\textit{suffix}}'
with \textit{prefix} matching precisely the argument.
Processing is then passed on to the file
`{\textit{dest}\hspace{0.2em}\textit{suffix}}'.
Surely, the same effect is achieved by
directly specifying the
argument `{\textit{dest}\hspace{0.2em}\textit{suffix}}'
in the first form.
However, that requires to set up a different file
for each child. With the alternative form of the command
all these files can have exactly the same content
which simplifies setting them up and maintaining them.

For example, the following file |draft.tex|
with a compilation flag |\version| as described in \secref{sec:flags}
compiles the main document as a draft:
%
\begin{center}
\begin{tabular}{l}
|\def\version{draft}|\\
|\input{childdoc.def}|\\
|\childdocforward{|\textit{main}|}|
\end{tabular}
\end{center}
%
Likewise, the following files |final|\textit{nn}|.tex|
compile the final version of the child document
|child|\textit{nn}|.tex|:
%
\begin{center}
\begin{tabular}{l}
|\def\version{final}|\\
|\input{childdoc.def}|\\
|\childdocforwardprefix{final}{child}|
\end{tabular}
\end{center}
%

Note that when several versions of a main file and/or of each child file
are to be generated, it may be convenient to set up a |Makefile| or
shell script to automatise the process.

%%%%%%%%%%%%%%%%%%%%%%%%%%%%%%%%%%%%%%%%%%%%%%%%%%%%%%%%%%%%%%%%%%%%%%%%%%%%%%%%
\subsection{Command Line Processing}
\label{sec:commandline}

The effect of redirection files can also be achieved by invoking
the \LaTeX{} compiler with a more elaborate command line.
Most conveniently this should be done as part
of a shell script or a |Makefile|.

When using \textsf{childdoc} in the main file, the following
command lines effectively perform a redirection
(note that depending on the shell being used,
backslashes may have to be doubled: `|\|' $\to$ `|\\|'):
%
\begin{center}
|... -jobname "|\textit{target}|" |\\|"|[\textit{flags}]%
|\input{childdoc.def}\childdocforward[|\textit{main}|]{|\textit{dest}|}"|
\end{center}
%
Here \textit{target} is the name of the output file,
\textit{main} is the name of the main file
and \textit{dest} is the name of the main or child file to be processed
(all filenames without extensions).
The optional argument \textit{main} can be omitted
if \textit{main} matches \textit{dest}.
Optionally, compilation \textit{flags} can be defined via |\def| commands.
This command line makes the \TeX{} engine believe
it is compiling the file \textit{target}
whose content is specified as the latter parameter.
The provided code then forwards the processing to
\textit{main} or \textit{dest} as described in \secref{sec:forward}.

%%%%%%%%%%%%%%%%%%%%%%%%%%%%%%%%%%%%%%%%%%%%%%%%%%%%%%%%%%%%%%%%%%%%%%%%%%%%%%%%
\subsection{Include by Input}
\label{sec:input}

Including child documents by |\include| has some restrictions by design.
Most notably, the content of a child document always occupies
its own set of pages; pages cannot be shared between child documents.
Usually, this behaviour makes perfect sense
because each child document contain an essential part of the document.
However, in some situations it may be desirable to compose
a document from a collection of parts
without having mandatory page breaks between then.
For this case, the package
provides a mechanism to include parts
by |\input| which can also be processed individually.
However, by construction this mechanism
requires manual handling of the content to be output.

%%%%%%%%%%%%%%%%%%%%%%%%%%%%%%%%%%%%%%%%
\DescribeMacro{\ifchilddocmanual}
The main file should be prepared as usual, see \secref{sec:include}.
However, the document body must make a distinction
between processing of an individual part and of the main document, e.g.:
%
\begin{center}
\begin{tabular}{l}
|\ifchilddocmanual|\\
|\input{\childdocname}|\\
|\||else|\\
\textit{document body with }|\input{|\textit{part}|}|\\
|\||fi|
\end{tabular}
\end{center}
%
The conditional |\ifchilddocmanual| is true whenever
a part to be included by |\input| is being compiled,
and the name of the part is stored in |\childdocname|.

%%%%%%%%%%%%%%%%%%%%%%%%%%%%%%%%%%%%%%%%
\DescribeMacro{\childdocby}
Each part to be included by |\input| should start with:
%
\begin{center}
\begin{tabular}{l}
|\input{childdoc.def}|\\
|\childdocby{|\textit{main}|}|\\
\end{tabular}
\end{center}
%
The directive |\childdocby| is similar to |\childdocof|
described in \secref{sec:include},
but the subsequent selection of content must be done manually.
To that end, both |\ifchilddoc| and |\ifchilddocmanual|
will be true upon processing of a part,
and the name of the part is stored in |\childdocname|.
Note that |\jobname| will be set to the filename of the current part
so that each part receives an individual |.aux| file
that does not interfere with the |.aux| file(s) of the main document.
This behaviour can be altered by the alternative form
|\childdocby[*]{|\textit{main}|}| (with a non-empty optional argument)
which uses the |.aux| file of the main document
by setting |\jobname| to \textit{main}.

%%%%%%%%%%%%%%%%%%%%%%%%%%%%%%%%%%%%%%%%%%%%%%%%%%%%%%%%%%%%%%%%%%%%%%%%%%%%%%%%
\subsection{Driver Development}
\label{sec:driver}

The \textsf{childdoc} mechanism can also be use for the development
of definition files such as \LaTeX{} styles or classes.
This case differs from the above setup with multiple parts
included by |\include| in that no |\includeonly| should be invoked.
This can be achieved by starting the include file
(before |\ProvidesPackage|) with:
%
\begin{center}
\begin{tabular}{l}
|\input{childdoc.def}|\\
|\childdocforward{|\textit{main}|}|\\
\end{tabular}
\end{center}
%
or alternatively with:
%
\begin{center}
\begin{tabular}{l}
|\input{childdoc.def}|\\
|\childdocby{|\textit{main}|}|\\
\end{tabular}
\end{center}
%
Both forms have slightly different effects as described above.
The main file is prepared as usual, see \secref{sec:include}.

%%%%%%%%%%%%%%%%%%%%%%%%%%%%%%%%%%%%%%%%%%%%%%%%%%%%%%%%%%%%%%%%%%%%%%%%%%%%%%%%
\subsection{Legacy Detection}
\label{sec:detection}

The directive |\childdocmain| in the main file can detect
whether the complete document or merely a child is to be compiled
even without using the directive |\childdocof|.
This method is deprecated because it is less robust
and there is no compelling reason to use it;
it is merely provided for backward compatibility
and it may be removed in future versions.

If the detection mechanism is to be used,
it is mandatory to correctly specify
the filename of the main file as the argument of |\childdocmain|:
%
\begin{center}
\begin{tabular}{l}
|\input{childdoc.def}|\\
|\childdocmain{|\textit{main}|}|\\
\end{tabular}
\end{center}
%
If |\jobname| does not match the argument \textit{main} of |\childdocmain|,
it is assumed that |\jobname| points to the child file to be compiled.
When using |\childdocmain| with the main file specified as argument,
it suffices to start a child file
with just |\input{|\textit{main}|}|
without loading of the package and using |\childdocof|.
If instead all processing is done
with the appropriate \textsf{childdoc} directives,
the argument of \textit{main} of |\childdocmain| can be empty.

An alternative version of the command line processing described
in \secref{sec:commandline} using the detection mechanism reads:
%
\begin{center}
|... -jobname "|\textit{target}|" "|[\textit{flags}]%
[|\def\jobname{|\textit{dest}|}|]|\input{|\textit{main}|}"|
\end{center}

%%%%%%%%%%%%%%%%%%%%%%%%%%%%%%%%%%%%%%%%%%%%%%%%%%%%%%%%%%%%%%%%%%%%%%%%%%%%%%%%
\subsection{Manual Code}
\label{sec:manual}

In case one cannot be certain whether the definitions file |childdoc.def|
is installed on the target \TeX{} distribution
and one prefers not to ship it,
it is conceivable to paste a few relevant commands into the sources.

To that end, drop all statements |\input{childdoc.def}|
and perform the replacements as outlined below.
Instead of |\childdocmain{|\textit{main}|}| add the following code
to the top of the main file:
%
\begin{center}
\begin{tabular}{l}
|\||ifdefined\childdocname\endinput\||fi\newif\ifchilddoc|\\
|\edef\childdocname{\scantokens\expandafter{\jobname\noexpand}}|\\
|\def\childdocmain{|\textit{main}|}\||ifx\childdocmain\childdocname\||else|\\
|\childdoctrue\includeonly{\childdocname}\let\jobname\childdocmain\||fi|\\
\end{tabular}
\end{center}
%
Instead of |\childdocof{|\textit{main}|}| just include the main file
at the top of each child file:
%
\begin{center}
|\input{|\textit{main}|}|
\end{center}
%
A simple redirection |\childdocforward{|\textit{dest}|}| is achieved by:
%
\begin{center}
|\def\jobname{|\textit{dest}|}\input{\jobname}|
\end{center}
%
The redirection with prefix
|\childdocforwardprefix[|\textit{prefix}|]{|\textit{dest}|}|
is accomplished by:
%
\begin{center}
\begin{tabular}{l}
|{\edef\jobname{\scantokens\expandafter{\jobname\noexpand}}|\\
|\def\redirectjob |\textit{prefix}|#1~~~{\gdef\jobname{|\textit{dest}|#1}}|\\
|\expandafter\redirectjob\jobname~~~}\input{\jobname}|
\end{tabular}
\end{center}

In an alternative approach,
child documents can be compiled by a specific command line
without additional code or specific definitions:
%
\begin{center}
|... -jobname "|\textit{target}|" "|[\textit{flags}]%
|\includeonly{|\textit{dest}|}\input{|\textit{main}|}"|
\end{center}
%

%%%%%%%%%%%%%%%%%%%%%%%%%%%%%%%%%%%%%%%%%%%%%%%%%%%%%%%%%%%%%%%%%%%%%%%%%%%%%%%%
%%%%%%%%%%%%%%%%%%%%%%%%%%%%%%%%%%%%%%%%%%%%%%%%%%%%%%%%%%%%%%%%%%%%%%%%%%%%%%%%
\section{Information}

%%%%%%%%%%%%%%%%%%%%%%%%%%%%%%%%%%%%%%%%%%%%%%%%%%%%%%%%%%%%%%%%%%%%%%%%%%%%%%%%
\subsection{Copyright}

Copyright \copyright{} 2017--2018 Niklas Beisert

This work may be distributed and/or modified under the
conditions of the \LaTeX{} Project Public License, either version 1.3
of this license or (at your option) any later version.
The latest version of this license is in
  \url{http://www.latex-project.org/lppl.txt}
and version 1.3 or later is part of all distributions of \LaTeX{}
version 2005/12/01 or later.

This work has the LPPL maintenance status `maintained'.

The Current Maintainer of this work is Niklas Beisert.

This work consists of the files |README.txt|, |childdoc.ins| and |childdoc.dtx|
as well as the derived files |childdoc.def|, |cdocsamp.tex|
with |cdocsch1.tex|, |cdocsch2.tex|, |cdocspt3.tex|, |cdocspt4.tex|,
|cdocsdrf.tex|, |cdocsfn1.tex|, |cdocsfn2.tex|
as well as |childdoc.pdf|.

%%%%%%%%%%%%%%%%%%%%%%%%%%%%%%%%%%%%%%%%%%%%%%%%%%%%%%%%%%%%%%%%%%%%%%%%%%%%%%%%
\subsection{Files and Installation}

The package consists of the files:
%
\begin{center}
\begin{tabular}{ll}
    |README.txt|   & readme file \\
    |childdoc.ins| & installation file \\
    |childdoc.dtx| & source file \\
    |childdoc.def| & definition file \\
    |cdocsamp.tex| & sample main file \\
    |cdocsch1.tex| & sample include file \\
    |cdocsch2.tex| & sample include file \\
    |cdocspt3.tex| & sample part file \\
    |cdocspt4.tex| & sample part file \\
    |cdocsdrf.tex| & sample redirection file \\
    |cdocsfn1.tex| & sample redirection file \\
    |cdocsfn2.tex| & sample redirection file \\
    |childdoc.pdf| & manual
\end{tabular}
\end{center}
%
The distribution consists of the files
|README.txt|, |childdoc.ins| and |childdoc.dtx|.
%
\begin{itemize}
\item
Run (pdf)\LaTeX{} on |childdoc.dtx|
to compile the manual |childdoc.pdf| (this file).
\item
Run \LaTeX{} on |childdoc.ins| to create the definitions file |childdoc.def|
and the sample |cdocsamp.tex| with include files
|cdocsch1.tex|, |cdocsch2.tex|, |cdocspt3.tex|, |cdocspt4.tex|,
|cdocsdrf.tex|, |cdocsfn1.tex|, |cdocsfn2.tex|.
Then copy the file |childdoc.def| to an appropriate directory of your \LaTeX{}
distribution, e.g.\ \textit{texmf-root}|/tex/latex/childdoc|.
\end{itemize}

%%%%%%%%%%%%%%%%%%%%%%%%%%%%%%%%%%%%%%%%%%%%%%%%%%%%%%%%%%%%%%%%%%%%%%%%%%%%%%%%
\subsection{Related CTAN Packages}

There are several other packages which offer a similar functionality:
%
\begin{itemize}
\item
The packages
\href{http://ctan.org/pkg/docmute}{\textsf{docmute}},
\href{http://ctan.org/pkg/includex}{\textsf{includex}} and
\href{http://ctan.org/pkg/standalone}{\textsf{standalone}}
provide commands to include only the document body of
a child file thus allowing both files to be compiled individually.
\item
The packages \href{http://ctan.org/pkg/subdocs}{\textsf{subdocs}}
and \href{http://ctan.org/pkg/subfiles}{\textsf{subfiles}}
provide structures in which the main and child documents can be
encapsulated and allowing them to be compiled individually.
The inclusion mechanism is different from the conventional |\include|.
\item
The package \href{http://ctan.org/pkg/combine}{\textsf{combine}}
is an elaborate solution to combine several documents into one.
\end{itemize}
%
See also the CTAN topic \href{http://ctan.org/topic/subdocs}{\textsf{subdocs}}
for further related packages.
The present package differs from the above solutions in that
a document structure constructed with the conventional |\include| mechanism
just needs two extra commands at the top of every file
such that all constituent files can be compiled individually.

%%%%%%%%%%%%%%%%%%%%%%%%%%%%%%%%%%%%%%%%%%%%%%%%%%%%%%%%%%%%%%%%%%%%%%%%%%%%%%%%
%\subsection{Feature Suggestions}
%
%The following is a list of features which may be useful for future
%versions of this package:
%%
%\begin{itemize}
%\item
%\ldots
%\end{itemize}

%%%%%%%%%%%%%%%%%%%%%%%%%%%%%%%%%%%%%%%%%%%%%%%%%%%%%%%%%%%%%%%%%%%%%%%%%%%%%%%%
\subsection{Revision History}

%%%%%%%%%%%%%%%%%%%%%%%%%%%%%%%%%%%%%%%%
\paragraph{v2.0:} 2018/12/30

\begin{itemize}
\item
immediate forward processing
\item
added |\childdocby| mechanism
\item
manual restructured
\end{itemize}

%%%%%%%%%%%%%%%%%%%%%%%%%%%%%%%%%%%%%%%%
\paragraph{v1.6:} 2018/01/17

\begin{itemize}
\item
application for development of include files
\item
corrections to manual
\end{itemize}

%%%%%%%%%%%%%%%%%%%%%%%%%%%%%%%%%%%%%%%%
\paragraph{v1.5:} 2017/05/21

\begin{itemize}
\item
more complete structuring introduced
\item
|\childdocof| introduced
\item
|\childdoc| renamed to |\childdocmain|
\item
|\childredirect| renamed to |\childdocforward| and |\childdocforwardprefix|
and functionality expanded
\end{itemize}

%%%%%%%%%%%%%%%%%%%%%%%%%%%%%%%%%%%%%%%%
\paragraph{v1.0:} 2017/04/27

\begin{itemize}
\item
manual and install package
\item
first version published on CTAN
\end{itemize}

%%%%%%%%%%%%%%%%%%%%%%%%%%%%%%%%%%%%%%%%
\paragraph{v0.6:} 2017/04/26

\begin{itemize}
\item
redirection mechanism added
\end{itemize}

%%%%%%%%%%%%%%%%%%%%%%%%%%%%%%%%%%%%%%%%
\paragraph{v0.5:} 2017/04/26

\begin{itemize}
\item
functionality in definition file
\end{itemize}


%%%%%%%%%%%%%%%%%%%%%%%%%%%%%%%%%%%%%%%%%%%%%%%%%%%%%%%%%%%%%%%%%%%%%%%%%%%%%%%%
%%%%%%%%%%%%%%%%%%%%%%%%%%%%%%%%%%%%%%%%%%%%%%%%%%%%%%%%%%%%%%%%%%%%%%%%%%%%%%%%
%%%%%%%%%%%%%%%%%%%%%%%%%%%%%%%%%%%%%%%%%%%%%%%%%%%%%%%%%%%%%%%%%%%%%%%%%%%%%%%%
\appendix

\settowidth\MacroIndent{\rmfamily\scriptsize 000\ }

 \DocInput{childdoc.dtx}

\end{document}
%</driver>
% \fi
%
% %%%%%%%%%%%%%%%%%%%%%%%%%%%%%%%%%%%%%%%%%%%%%%%%%%%%%%%%%%%%%%%%%%%%%%%%%%%%%%
% %%%%%%%%%%%%%%%%%%%%%%%%%%%%%%%%%%%%%%%%%%%%%%%%%%%%%%%%%%%%%%%%%%%%%%%%%%%%%%
% \section{Sample}
%\iffalse
%<*samplemain>
%\fi
%
% The following presents a sample document
% with two chapters, two parts, a title page,
% a compile flag as well as three forwarding files to set the flag.
% It consists of eight |.tex| files:
% \begin{center}
% \begin{tabular}{ll}
% |cdocsamp.tex|&main file\\
% |cdocsch1.tex|&include file for chapter 1\\
% |cdocsch2.tex|&include file for chapter 2\\
% |cdocspt3.tex|&include file for part 3\\
% |cdocspt4.tex|&include file for part 4\\
% |cdocsdrf.tex|&forwarding file for main file in draft mode\\
% |cdocsfi1.tex|&forwarding file for final version of chapter 1\\
% |cdocsfi2.tex|&forwarding file for final version of chapter 2\\
% \end{tabular}
% \end{center}
% Each of the eight files can be compiled directly by the \LaTeX{} compiler.
%
% %%%%%%%%%%%%%%%%%%%%%%%%%%%%%%%%%%%%%%
% \paragraph{Main File.}
%
% The main file is called |cdocsamp.tex|.
%
% Load the \textsf{childdoc} definitions and
% declare the filename for the main document:
%    \begin{macrocode}
\input{childdoc.def}
\childdocmain{}
%    \end{macrocode}

% Optional override for |\version| flag:
%    \begin{macrocode}
%%\ifchilddoc\else\providecommand{\version}{draft}\fi
%    \end{macrocode}

% Define the default values for the |\version| flag
% (|final| for the main file and |draft| for childs):
%    \begin{macrocode}
\ifchilddoc
\providecommand{\version}{draft}
\else
\providecommand{\version}{final}
\fi
%    \end{macrocode}

% Load the standard document class:
%    \begin{macrocode}
\documentclass[12pt]{article}
%    \end{macrocode}

% Start the document body:
%    \begin{macrocode}
\begin{document}
%    \end{macrocode}

% Declare a title page.
% Print title, part of document being processed and version flag:
%    \begin{macrocode}
\addtocounter{page}{-1}
\begin{center}
{\LARGE\bfseries{}childdoc example\par}
\vspace{1cm}
\ifchilddoc
\ifchilddocmanual part\else chapter\fi:
`\childdocname' of `\childdocjob'\par
\else
main document: `\childdocjob'\par
\fi
version: \version\par
\end{center}
\newpage
%    \end{macrocode}

% Manually include selected file,
% otherwise process as usual:
%    \begin{macrocode}
\ifchilddocmanual
\section*{part `\childdocname'}
\input{\childdocname}
\else
%    \end{macrocode}

% Include the two chapters:
%    \begin{macrocode}
\include{cdocsch1}
\include{cdocsch2}
%    \end{macrocode}

% Include the two parts unless only chapters should be displayed:
%    \begin{macrocode}
\ifchilddoc\else
\section{part three}
\input{cdocspt3}
\section{part four}
\input{cdocspt4}
\fi
%    \end{macrocode}

% Process as usual until here:
%    \begin{macrocode}
\fi
%    \end{macrocode}

% End of document body:
%    \begin{macrocode}
\end{document}
%    \end{macrocode}
%\iffalse
%</samplemain>
%\fi
%
% %%%%%%%%%%%%%%%%%%%%%%%%%%%%%%%%%%%%%%
% \paragraph{Chapter Include Files.}
%
% The include files are called |cdocsch1.tex| and |cdocsch2.tex|.
%
%\iffalse
%<*samplechap1|samplechap2>
%\fi

% Optional override for |\version| flag:
%    \begin{macrocode}
%%\providecommand{\version}{final}
%    \end{macrocode}

% Include the main document:
%    \begin{macrocode}
\input{childdoc.def}
\childdocof{cdocsamp}
%    \end{macrocode}

%\iffalse
%</samplechap1|samplechap2>
%\fi
%
%\iffalse
%<*samplechap1>
%\fi
% Some text for chapter 1:
%    \begin{macrocode}
\section{one}
some text in chapter one
%    \end{macrocode}

%\iffalse
%</samplechap1>
%\fi
% Some text for chapter 2:
%\iffalse
%<*samplechap2>
%\fi
%    \begin{macrocode}
\section{two}
more text in chapter two
%    \end{macrocode}

%\iffalse
%</samplechap2>
%\fi
%
% %%%%%%%%%%%%%%%%%%%%%%%%%%%%%%%%%%%%%%
% \paragraph{Part Include Files.}
%
% The include files are called |cdocspt3.tex| and |cdocspt4.tex|.
%
%\iffalse
%<*samplepart3|samplepart4>
%\fi

% Optional override for |\version| flag:
%    \begin{macrocode}
%%\providecommand{\version}{final}
%    \end{macrocode}

% Include the main document:
%    \begin{macrocode}
\input{childdoc.def}
\childdocby{cdocsamp}
%    \end{macrocode}

%\iffalse
%</samplepart3|samplepart4>
%\fi
%
%\iffalse
%<*samplepart3>
%\fi
% Some text for part 3:
%    \begin{macrocode}
some text in part three
%    \end{macrocode}

%\iffalse
%</samplepart3>
%\fi
% Some text for part 4:
%\iffalse
%<*samplepart4>
%\fi
%    \begin{macrocode}
more text in part four
%    \end{macrocode}

%\iffalse
%</samplepart4>
%\fi
%
% %%%%%%%%%%%%%%%%%%%%%%%%%%%%%%%%%%%%%%
% \paragraph{Forwarding for a Complete Draft.}
%
% The following forwarding file |cdocsdrf.tex|
% compiles the main document in draft mode:
%\iffalse
%<*sampledraft>
%\fi
%    \begin{macrocode}
\def\version{draft}
\input{childdoc.def}
\childdocforward{cdocsamp}
%    \end{macrocode}

%\iffalse
%</sampledraft>
%\fi
%
% %%%%%%%%%%%%%%%%%%%%%%%%%%%%%%%%%%%%%%
% \paragraph{Forwarding for Final Version of the Chapters.}
%
% The following forwarding files |cdocsfn1.tex| and |cdocsfn2.tex|
% (with identical content)
% compile the final versions of the child documents
% |cdocsch1.tex| and |cdocsch2.tex|, respectively:
%\iffalse
%<*samplefinal>
%\fi
%    \begin{macrocode}
\def\version{final}
\input{childdoc.def}
\childdocforwardprefix[cdocsamp]{cdocsfn}{cdocsch}
%    \end{macrocode}

%\iffalse
%</samplefinal>
%\fi
%
% %%%%%%%%%%%%%%%%%%%%%%%%%%%%%%%%%%%%%%
% \paragraph{Command Line Processing.}
%
% The following three command lines generate the output files
% |cdocscld|, |cdocscl1| and |cdocscl2|
% which should be identical to
% |cdocsdrf|, |cdocsch1| and |cdocsfn2|, respectively:
% \begin{center}
% \begin{tabular}{l}
% |latex -jobname cdocscld \|\\
% |  "\def\version{draft}\input{childdoc.def}\childdocforward{cdocsamp}"|\\
% |latex -jobname cdocscl1 \|\\
% |  "\input{childdoc.def}\childdocforward[cdocsamp]{cdocsch1}"|\\
% |latex -jobname cdocscl2 \|\\
% |  "\def\version{final}\input{childdoc.def}\childdocforward{cdocsch2}"|
% \end{tabular}
% \end{center}
% Note that the trailing backslash on each first line
% merely continues the input to the second line
% (for convenient cut ant paste).
% Furthermore, the command |latex| can be replaced by any
% of its alternative versions such as |pdflatex|.
%
% %%%%%%%%%%%%%%%%%%%%%%%%%%%%%%%%%%%%%%%%%%%%%%%%%%%%%%%%%%%%%%%%%%%%%%%%%%%%%%
% %%%%%%%%%%%%%%%%%%%%%%%%%%%%%%%%%%%%%%%%%%%%%%%%%%%%%%%%%%%%%%%%%%%%%%%%%%%%%%
% \section{Implementation}
%\iffalse
%<*package>
%\fi
%
% This section describes the definitions file |childdoc.def|.

% The definitions cannot be loaded using |\usepackage| or |\RequirePackage|
% which has a mechanism to prevent loading a style file more than once.
% When loading the definitions by means of |\input|
% multiple instances have to be prevented manually:
%\iffalse
%This code needs to be before the `\ProvidesFile' directive
%which is defined at the beginning of this file.
%Therefore it is also placed there and commented out here.
%</package>
%<*discard>
%\fi
%    \begin{macrocode}
\ifdefined\childdocmain\endinput\fi
%    \end{macrocode}
%\iffalse
%</discard>
%<*package>
%\fi
%
% \macro{\ifchilddoc}
% \macro{\ifchilddocmanual}
% The conditional |\ifchilddoc| tells whether a
% child (true) or main (false) document is being compiled.
% The conditional |\ifchilddocmanual| tells whether
% the |\includeonly| mechanism is used (false) or
% the selection of child files must be performed manually (true).
% The definitions initialise to false:
%    \begin{macrocode}
\newif\ifchilddoc
\newif\ifchilddocmanual
%    \end{macrocode}

% \macro{\childdocname}
% \macro{\childdocjob}
% The macro |\childdocname| stores the name of the main document
% to be compiled. The macro |\childdocjob| stores the name of
% the document on which the \LaTeX{} compiler was originally invoked.
% The content of |\jobname| cannot be compared
% to filenames specified in the source due to different catcodes.
% The following code rescans |\jobname|, stores the result
% in |\childdocname| and saves a copy in |\childdocjob|:
%    \begin{macrocode}
\edef\childdocname{\scantokens\expandafter{\jobname\noexpand}}
\let\childdocjob\childdocname
%    \end{macrocode}

% \macro{\childdocdisable}
% The macro |\childdocdisable| prevents the main file
% from being processed more than once.
% At this stage, the main document command |\childdocmain|
% is assumed to be called once again where it should do nothing.
% Any subsequent call to it should prevent
% a secondary processing of the main document
% It overwrites the forwarding commands
% |\childdocof| and |\childdocforward|
% with empty macros to prevent further inclusions of the main document:
%    \begin{macrocode}
\newcommand{\childdocdisable}
{
  \renewcommand{\childdocmain}[1]{\renewcommand{\childdocmain}[1]{\endinput}}
  \renewcommand{\childdocof}[1]{}
  \renewcommand{\childdocby}[2][]{}
  \renewcommand{\childdocforward}[2][]{}
  \renewcommand{\childdocdisable}{}
}
%    \end{macrocode}

% \macro{\childdocmain}
% The macro |\childdocmain| is to be called at the top of the main file
% with nothing or the main filename (without extension) as argument.
% First, it breaks loops.
% If the argument is not empty and does not match |\childdocname|
% (which is set by the first inclusion of |childdoc.def|),
% |\ifchilddoc| is set to true, |\includeonly| is applied to the child file
% and |\jobname| is set to the main file
% (for proper handling of |.aux| files):
%    \begin{macrocode}
\newcommand{\childdocmain}[1]
{
  \childdocdisable\childdocmain{}
  \if?#1?\else
    \begingroup
      \def\childdoctmp{#1}
      \ifx\childdoctmp\childdocname
        \def\childdoctmp{}
      \else
        \def\childdoctmp
        {
          \childdoctrue
          \includeonly{\childdocname}
          \def\childdocjob{#1}
          \def\jobname{#1}
        }
      \fi
      \expandafter
    \endgroup
    \childdoctmp
  \fi
}
%    \end{macrocode}

% \macro{\childdocof}
% The command |\childdocof| redirects
% compilation to the main file |#1|.
%    \begin{macrocode}
\newcommand{\childdocof}[1]
{
  \childdocdisable
  \childdoctrue
  \includeonly{\childdocname}
  \def\jobname{#1}
  \def\childdocjob{#1}
  \input{#1}
}
%    \end{macrocode}

% \macro{\childdocby}
% The command |\childdocby| ....
%    \begin{macrocode}
\newcommand{\childdocby}[2][]
{
  \childdocdisable
  \childdoctrue
  \childdocmanualtrue
  \if?#1?\else
    \def\jobname{#2}
  \fi
  \def\childdocjob{#2}
  \input{#2}
  \endinput
}
%    \end{macrocode}

% \macro{\childdocforward}
% The command |\childdocforward| redirects
% compilation to the main file or
% (if the optional argument is given) a child file.
% Parameters are set as if the main file
% or a child file starting with |\childdocof| was compiled.
% Then compilation is handed over to the main file:
%    \begin{macrocode}
\newcommand{\childdocforward}[2][]
{
  \begingroup
    \if?#1?
      \def\childdoctmp
      {
        \def\childdocname{#2}
        \def\childdocjob{#2}
        \def\jobname{#2}
        \input{#2}
        \endinput
      }
    \else
      \def\childdoctmp
      {
        \childdocdisable
        \def\childdocname{#2}
        \childdoctrue
        \includeonly{#2}
        \def\childdocjob{#1}
        \def\jobname{#1}
        \input{#1}
        \endinput
      }
    \fi
    \expandafter
  \endgroup
  \childdoctmp
}
%    \end{macrocode}

% \macro{\childdocforwardprefix}
% The command |\childdocforwardprefix| redirects
% compilation to the main or a child file by means of a pattern.
% The prefix |#1| in the current filename is replaced by |#2|
% and the suffix of the current filename is kept
% (it is assumed that the filename does not contain the substring `|~~~|'
% which is used as a delimiter).
% Compilation is handed over to the new file by |\childdocforward|:
%    \begin{macrocode}
\newcommand{\childdocforwardprefix}[3][]
{
  \begingroup
    \def\childdocextract #2##1~~~{\def\childdoctmp{\childdocforward[#1]{#3##1}}}
    \expandafter\childdocextract\childdocname~~~
    \expandafter
  \endgroup
  \childdoctmp
}
%    \end{macrocode}

% \macro{\childdoc}
% The deprecated macro |\childdoc| is a legacy version of |\childdocmain|:
%    \begin{macrocode}
\newcommand{\childdoc}{\childdocmain}
%    \end{macrocode}

% \macro{\childdocredirect}
% The deprecated macro |\childdocredirect| is a legacy version
% of |\childdocforward| and |\childdocforwardprefix|:
%    \begin{macrocode}
\newcommand{\childdocredirect}[2][]
{
  \begingroup
    \if?#1?
      \def\childdoctmp{\childdocforward{#2}}
    \else
      \def\childdoctmp{\childdocforwardprefix{#1}{#2}}
    \fi
    \expandafter
  \endgroup
  \childdoctmp
}
%    \end{macrocode}

%\iffalse
%</package>
%\fi
%
\endinput
|\\
|\childdocby{|\textit{main}|}|\\
\end{tabular}
\end{center}
%
The directive |\childdocby| is similar to |\childdocof|
described in \secref{sec:include},
but the subsequent selection of content must be done manually.
To that end, both |\ifchilddoc| and |\ifchilddocmanual|
will be true upon processing of a part,
and the name of the part is stored in |\childdocname|.
Note that |\jobname| will be set to the filename of the current part
so that each part receives an individual |.aux| file
that does not interfere with the |.aux| file(s) of the main document.
This behaviour can be altered by the alternative form
|\childdocby[*]{|\textit{main}|}| (with a non-empty optional argument)
which uses the |.aux| file of the main document
by setting |\jobname| to \textit{main}.

%%%%%%%%%%%%%%%%%%%%%%%%%%%%%%%%%%%%%%%%%%%%%%%%%%%%%%%%%%%%%%%%%%%%%%%%%%%%%%%%
\subsection{Driver Development}
\label{sec:driver}

The \textsf{childdoc} mechanism can also be use for the development
of definition files such as \LaTeX{} styles or classes.
This case differs from the above setup with multiple parts
included by |\include| in that no |\includeonly| should be invoked.
This can be achieved by starting the include file
(before |\ProvidesPackage|) with:
%
\begin{center}
\begin{tabular}{l}
|% \iffalse
%
% childdoc.dtx Copyright (C) 2017-2018 Niklas Beisert
%
% This work may be distributed and/or modified under the
% conditions of the LaTeX Project Public License, either version 1.3
% of this license or (at your option) any later version.
% The latest version of this license is in
%   http://www.latex-project.org/lppl.txt
% and version 1.3 or later is part of all distributions of LaTeX
% version 2005/12/01 or later.
%
% This work has the LPPL maintenance status `maintained'.
%
% The Current Maintainer of this work is Niklas Beisert.
%
% This work consists of the files childdoc.dtx and childdoc.ins
% and the derived files childdoc.def and cdocsamp.tex with
% cdocsch1.tex, cdocsch2.tex, cdocsdrf.tex, cdocsfn1.tex, cdocsfn2.tex.
%
%<package>\ifdefined\childdocmain\endinput\fi
%<package>\ProvidesFile{childdoc.def}[2018/12/30 v2.0 child document driver]
%<samplemain>\ProvidesFile{cdocsamp.tex}[2018/12/30 v2.0 sample for childdoc]
%<*driver>
%\ProvidesFile{childdoc.drv}[2018/12/30 v2.0 childdoc reference manual file]
\PassOptionsToClass{10pt,a4paper}{article}
\documentclass{ltxdoc}

\usepackage[margin=35mm]{geometry}
\usepackage{hyperref}
\usepackage{hyperxmp}
\usepackage[usenames]{color}

\hypersetup{colorlinks=true}
\hypersetup{pdfstartview=FitH}
\hypersetup{pdfpagemode=UseNone}
\hypersetup{pdfsource={}}
\hypersetup{pdflang={en-UK}}
\hypersetup{pdfcopyright={Copyright 2017-2018 Niklas Beisert.
  This work may be distributed and/or modified under the
  conditions of the LaTeX Project Public License, either version 1.3
  of this license or (at your option) any later version.}}
\hypersetup{pdflicenseurl={http://www.latex-project.org/lppl.txt}}
\hypersetup{pdfcontactaddress={ETH Zurich, ITP, HIT K,
  Wolfgang-Pauli-Strasse 27}}
\hypersetup{pdfcontactpostcode={8093}}
\hypersetup{pdfcontactcity={Zurich}}
\hypersetup{pdfcontactcountry={Switzerland}}
\hypersetup{pdfcontactemail={nbeisert@itp.phys.ethz.ch}}
\hypersetup{pdfcontacturl={http://people.phys.ethz.ch/\xmptilde nbeisert/}}

\newcommand{\secref}[1]{\hyperref[#1]{section \ref*{#1}}}

\parskip1ex
\parindent0pt
\let\olditemize\itemize
\def\itemize{\olditemize\parskip0pt}

\begin{document}

\title{The \textsf{childdoc} Package}
\hypersetup{pdftitle={The childdoc Package}}
\author{Niklas Beisert\\[2ex]
  Institut f\"ur Theoretische Physik\\
  Eidgen\"ossische Technische Hochschule Z\"urich\\
  Wolfgang-Pauli-Strasse 27, 8093 Z\"urich, Switzerland\\[1ex]
  \href{mailto:nbeisert@itp.phys.ethz.ch}
  {\texttt{nbeisert@itp.phys.ethz.ch}}}
\hypersetup{pdfauthor={Niklas Beisert}}
\hypersetup{pdfsubject={Manual for the LaTeX2e Package childdoc}}
\date{30 December 2018, \textsf{v2.0}}
\maketitle

\begin{abstract}\noindent
\textsf{childdoc} is a \LaTeXe{} package
that enables the direct compilation
of document sections included by |\include|
to individual files.
\end{abstract}

\begingroup
\parskip0ex
\tableofcontents
\endgroup

%%%%%%%%%%%%%%%%%%%%%%%%%%%%%%%%%%%%%%%%%%%%%%%%%%%%%%%%%%%%%%%%%%%%%%%%%%%%%%%%
%%%%%%%%%%%%%%%%%%%%%%%%%%%%%%%%%%%%%%%%%%%%%%%%%%%%%%%%%%%%%%%%%%%%%%%%%%%%%%%%
\section{Introduction}

\LaTeX{} provides a mechanism to structure a large document (such as a book)
into a main file and several child files (containing the chapters)
using the |\include| command.
This mechanism is beneficial for documents
which span hundreds of pages in order to
make the source file(s) more manageable.
Moreover, compilation can be restricted to
selected child files by means of the |\includeonly| command.
The latter feature can be used to reduce the compilation time while editing
(this was significantly more useful in the earlier days of \LaTeX{})
or to generate a smaller document which is easier to navigate.
Another application of |\includeonly| is to generate
documents consisting of selected parts of the complete document.

However, there are a few drawbacks of the plain |\include| mechanism:
\begin{itemize}
\item
The child files cannot be compiled on their own,
they can only be compiled via the main file.
A naive editing environment
(such as a text editor with an option
to have the current file processed by \LaTeX)
may require one to switch to the main file before compiling;
attempting to compile the child file produces errors.
\item
The main file must be modified (each time)
to adjust the |\includeonly| command
to the present needs. This easily leaves the main file in a messy state.
\item
The generated document will always carry the filename
of the main document. This is inconvenient if
several child files are to be compiled and
to be kept for distribution.
\end{itemize}

The present package provides a simple interface
to make child files individually compilable by \LaTeX{}.
Compiling a child file then has the same effect as compiling
the main file with an |\includeonly| command
to select the appropriate child.
Moreover the generated document will carry the name of the child
rather than the main file.
This resolves all three above issues.

This feature is meant to make the editing of books,
thesis documents and lecture notes somewhat more convenient.
However, the package can also be used efficiently for
composing a series of documents (such as exercise sheets)
which are typically distributed individually.
It then assists the author in generating the individual documents
(potentially in different versions)
as well as a document containing the collected series.
Another application is in developing style files
or other kinds of included material
where compilation of the style file could redirect
to a sample or test file.

%%%%%%%%%%%%%%%%%%%%%%%%%%%%%%%%%%%%%%%%%%%%%%%%%%%%%%%%%%%%%%%%%%%%%%%%%%%%%%%%
%%%%%%%%%%%%%%%%%%%%%%%%%%%%%%%%%%%%%%%%%%%%%%%%%%%%%%%%%%%%%%%%%%%%%%%%%%%%%%%%
\section{Usage}

First of all, the package \textsf{childdoc} is \emph{not} a standard
\LaTeXe{} |.sty| style file! Therefore it needs to be invoked in
a non-standard way.

%%%%%%%%%%%%%%%%%%%%%%%%%%%%%%%%%%%%%%%%%%%%%%%%%%%%%%%%%%%%%%%%%%%%%%%%%%%%%%%%
\subsection{Included Files}
\label{sec:include}

%%%%%%%%%%%%%%%%%%%%%%%%%%%%%%%%%%%%%%%%
\DescribeMacro{\childdocmain}
To use the package, add the commands
\begin{center}
\begin{tabular}{l}
|\input{childdoc.def}|\\
|\childdocmain{}|\\
\end{tabular}
\end{center}
at the very top of the main \LaTeX{} file,
in particular \emph{before} the |\documentclass| statement!
The argument of |\childdocmain| should be left empty
(but it must be present).

%%%%%%%%%%%%%%%%%%%%%%%%%%%%%%%%%%%%%%%%
\DescribeMacro{\childdocof}
Furthermore, add the commands
\begin{center}
\begin{tabular}{l}
|\input{childdoc.def}|\\
|\childdocof{|\textit{main}|}|\\
\end{tabular}
\end{center}
at the top of every child file \textit{child}
which is included by |\include{|\textit{child}|}|
from within the main file
(or at least for those files to be compiled individually).
The argument \textit{main} must be the filename of the main file.

There are a couple of
considerations in setting up the main and child documents:

%%%%%%%%%%%%%%%%%%%%%%%%%%%%%%%%%%%%%%%%
\paragraph{Restrictions.}

Please note the following restrictions:
\begin{itemize}
\item
|\childdocmain| must be called with one argument \textit{main}
to ensure compatibility with earlier version of the package.
It must either be empty (|\childdocmain{}|)
or precisely match the filename of the main file in which it is specified.
See \secref{sec:detection} for further information.
\item
The filename \textit{main} must be specified without the |.tex| extension.
\item
The filename \textit{main} is case sensitive
(even in case-insensitive file systems)
due to internal string comparison.
\item
The argument \textit{main} should be fully expanded, it cannot be a macro.
\item
Subdirectories and special characters should be avoided in filenames.
\item
The command |\childdocmain{|\textit{main}|}| must be followed by a whitespace.
It should not be followed immediately by another command
or by a comment mark `|%|'.
This is because the \TeX{} parser reads the token immediately following
the argument of |\childdocmain| and puts it
at the beginning of every child section;
however, a white\-space is ignored.
\end{itemize}

%%%%%%%%%%%%%%%%%%%%%%%%%%%%%%%%%%%%%%%%
\paragraph{Content of Main File.}

It is advisable to place all content in the child files included by |\include|.
Any output contained in the main file will appear in all child documents
unless suppressed manually;
it cannot be suppressed automatically by the |\includeonly| directive
and thus should normally be avoided.
A method to include some content in the main file
by means of conditional processing is described in \secref{sec:conditional}.

%%%%%%%%%%%%%%%%%%%%%%%%%%%%%%%%%%%%%%%%
\paragraph{Page Numbering.}

When only a part of the document is compiled,
the appropriate numbering of pages
(as well as other status parameters)
is determined from the |.aux| files.
The latter contain information from previous passes.
However this information needs to propagate through
all intermediate child documents.
Therefore the page numbering in child documents may well
be inconsistent until the complete document is compiled at least once.

A useful (if unconventional) way to always ensure a consistent
page numbering is to restart the numbering in each child document
and denote the pages by `\textit{child}|.|\textit{page}'
where \textit{child} represents the chapter/section number of the child file.
This can be achieved by the command
|\numberwithin{page}{|\textit{child}|}|
of the \textsf{amsmath} package
where \textit{child} can be |chapter| or |section|
depending on the chosen structuring.
Alternatively, one can modify the macro |\thepage| appropriately
and reset the counter |page| at the start of each child file.

%%%%%%%%%%%%%%%%%%%%%%%%%%%%%%%%%%%%%%%%%%%%%%%%%%%%%%%%%%%%%%%%%%%%%%%%%%%%%%%%
\subsection{Conditional Processing}
\label{sec:conditional}

The package provides a mechanism to compile different versions
of a document. To customise the versions further some conditional processing
can come in handy to distinguish which version is being compiled.
The package provides two macros to describe the compilation context:

%%%%%%%%%%%%%%%%%%%%%%%%%%%%%%%%%%%%%%%%
\DescribeMacro{\ifchilddoc}
The conditional |\ifchilddoc| distinguishes between the compilation of
child documents and the main document:
%
\begin{center}
|\ifchilddoc |\textit{child-code}| |[|\||else |\textit{main-code}]| \||fi|
\end{center}

%%%%%%%%%%%%%%%%%%%%%%%%%%%%%%%%%%%%%%%%
\DescribeMacro{\childdocname}
\DescribeMacro{\childdocjob}
The macro |\childdocname| contains the filename (without extension)
of the main or child file being processed.
Note that |\childdocjob| will always contain the name of the main file.

%%%%%%%%%%%%%%%%%%%%%%%%%%%%%%%%%%%%%%%%
\paragraph{Title Page.}

Conditional processing can be used to include a title or banner page
in the main document when proper precautions are taken.
Importantly, the code in the main file should ensure that the page counter
(as well as other status parameters which are stored in the |.aux| files)
takes the same value after the conditional processing.
Otherwise the page numbers may take divergent values
depending on which part is compiled.

For example, a title page could be declared by:
%
\begin{center}
\begin{tabular}{l}
|\ifchilddoc\||else|\\
|\addtocounter{page}{-1}|\\
\textit{code for title page}\\
|\newpage|\\
|\||fi|
\end{tabular}
\end{center}
%
A banner page for the child documents can be generated by:
%
\begin{center}
\begin{tabular}{l}
|\ifchilddoc|\\
|\addtocounter{page}{-1}|\\
\textit{code for banner page}\\
|\newpage|\\
|\||fi|
\end{tabular}
\end{center}
%
Here one could write a message such as:
\begin{center}
|This is the part \childdocname{} of \childdocjob{}.|
\end{center}

%%%%%%%%%%%%%%%%%%%%%%%%%%%%%%%%%%%%%%%%%%%%%%%%%%%%%%%%%%%%%%%%%%%%%%%%%%%%%%%%
\subsection{Flags}
\label{sec:flags}

The package makes it easy to generate different versions
of the main or child documents.
To this end compilation flags can be defined
and assigned different default values.
They will be particularly useful in conjunction
with the forwarding mechanism described in \secref{sec:forward}.

For example, it may be useful to have a flag |\version|
which can be set to |draft| or |final|.
The document source will contain some conditional code
depending on the value of |\version|.
Suppose further, the flag should default to |final| for the main file
and to |draft| for child files
which is a natural assignment for editing the document.
This is achieved by placing the following code
in the preamble of the main document
(below the |\childdocmain| directive):
%
\begin{center}
\begin{tabular}{l}
|\ifchilddoc|\\
|\providecommand{\version}{draft}|\\
|\||else|\\
|\providecommand{\version}{final}|\\
|\||fi|
\end{tabular}
\end{center}
%
The definition by |\providecommand| makes sure
that previous definitions are not overwritten.
Further statements |\providecommand{\version}{...}|
can thus be added before the above code to override it.

For the main file, one might add a line
(between |\childdocmain| and the above block)
%
\begin{center}
|%\ifchilddoc\||else\providecommand{\version}{draft}\||fi|
\end{center}
%
which can be uncommented to produce a draft version.
Likewise one can add a line to the very top of a child file
(above the |\childdocof{|\textit{main}|}| directive)
%
\begin{center}
|%\providecommand{\version}{final}|
\end{center}
%
which can be uncommented to produce the final version of this child document.

%%%%%%%%%%%%%%%%%%%%%%%%%%%%%%%%%%%%%%%%%%%%%%%%%%%%%%%%%%%%%%%%%%%%%%%%%%%%%%%%
\subsection{Forwarding}
\label{sec:forward}

Different versions of the main or child documents
using compilation flags as described in \secref{sec:flags}
can be (permanently) stored in different files
for convenient compilation, viewing and distribution.
To this end, the package defines a command
to pass on compilation to a different file:

%%%%%%%%%%%%%%%%%%%%%%%%%%%%%%%%%%%%%%%%
\DescribeMacro{\childdocforward}
The command |\childdocforward| redirects processing to
another source file:
%
\begin{center}
\begin{tabular}{l}
|\input{childdoc.def}|\\
|\childdocforward[|\textit{main}|]{|\textit{dest}|}|\\
\end{tabular}
\end{center}
%
The argument \textit{dest} is the destination file
(without extension).
It should be the main file or one of the child files.
Note that further \textsf{childdoc} directives
such as |\childdocof| and |\childdocforward|
in the indicated file will be processed in this form.
The optional argument \textit{main}
passes on directly to the main file \textit{main}
while pretending to compile the child \textit{dest}.
This form behaves as if \textit{dest}
issues |\childdocof{|\textit{main}|}| right away,
and no further \textsf{childdoc} directives will be processed.

%%%%%%%%%%%%%%%%%%%%%%%%%%%%%%%%%%%%%%%%
\DescribeMacro{\...prefix}
In the alternative form |\childdocforwardprefix|,
%
\begin{center}
\begin{tabular}{l}
|\input{childdoc.def}|\\
|\childdocforwardprefix[|\textit{main}|]{|\textit{prefix}|}{|\textit{dest}|}|
\end{tabular}
\end{center}
%
the destination file is determined by a pattern
depending on the current file:
To make this work, the current file must be called
`{\textit{prefix}\hspace{0.2em}\textit{suffix}}'
with \textit{prefix} matching precisely the argument.
Processing is then passed on to the file
`{\textit{dest}\hspace{0.2em}\textit{suffix}}'.
Surely, the same effect is achieved by
directly specifying the
argument `{\textit{dest}\hspace{0.2em}\textit{suffix}}'
in the first form.
However, that requires to set up a different file
for each child. With the alternative form of the command
all these files can have exactly the same content
which simplifies setting them up and maintaining them.

For example, the following file |draft.tex|
with a compilation flag |\version| as described in \secref{sec:flags}
compiles the main document as a draft:
%
\begin{center}
\begin{tabular}{l}
|\def\version{draft}|\\
|\input{childdoc.def}|\\
|\childdocforward{|\textit{main}|}|
\end{tabular}
\end{center}
%
Likewise, the following files |final|\textit{nn}|.tex|
compile the final version of the child document
|child|\textit{nn}|.tex|:
%
\begin{center}
\begin{tabular}{l}
|\def\version{final}|\\
|\input{childdoc.def}|\\
|\childdocforwardprefix{final}{child}|
\end{tabular}
\end{center}
%

Note that when several versions of a main file and/or of each child file
are to be generated, it may be convenient to set up a |Makefile| or
shell script to automatise the process.

%%%%%%%%%%%%%%%%%%%%%%%%%%%%%%%%%%%%%%%%%%%%%%%%%%%%%%%%%%%%%%%%%%%%%%%%%%%%%%%%
\subsection{Command Line Processing}
\label{sec:commandline}

The effect of redirection files can also be achieved by invoking
the \LaTeX{} compiler with a more elaborate command line.
Most conveniently this should be done as part
of a shell script or a |Makefile|.

When using \textsf{childdoc} in the main file, the following
command lines effectively perform a redirection
(note that depending on the shell being used,
backslashes may have to be doubled: `|\|' $\to$ `|\\|'):
%
\begin{center}
|... -jobname "|\textit{target}|" |\\|"|[\textit{flags}]%
|\input{childdoc.def}\childdocforward[|\textit{main}|]{|\textit{dest}|}"|
\end{center}
%
Here \textit{target} is the name of the output file,
\textit{main} is the name of the main file
and \textit{dest} is the name of the main or child file to be processed
(all filenames without extensions).
The optional argument \textit{main} can be omitted
if \textit{main} matches \textit{dest}.
Optionally, compilation \textit{flags} can be defined via |\def| commands.
This command line makes the \TeX{} engine believe
it is compiling the file \textit{target}
whose content is specified as the latter parameter.
The provided code then forwards the processing to
\textit{main} or \textit{dest} as described in \secref{sec:forward}.

%%%%%%%%%%%%%%%%%%%%%%%%%%%%%%%%%%%%%%%%%%%%%%%%%%%%%%%%%%%%%%%%%%%%%%%%%%%%%%%%
\subsection{Include by Input}
\label{sec:input}

Including child documents by |\include| has some restrictions by design.
Most notably, the content of a child document always occupies
its own set of pages; pages cannot be shared between child documents.
Usually, this behaviour makes perfect sense
because each child document contain an essential part of the document.
However, in some situations it may be desirable to compose
a document from a collection of parts
without having mandatory page breaks between then.
For this case, the package
provides a mechanism to include parts
by |\input| which can also be processed individually.
However, by construction this mechanism
requires manual handling of the content to be output.

%%%%%%%%%%%%%%%%%%%%%%%%%%%%%%%%%%%%%%%%
\DescribeMacro{\ifchilddocmanual}
The main file should be prepared as usual, see \secref{sec:include}.
However, the document body must make a distinction
between processing of an individual part and of the main document, e.g.:
%
\begin{center}
\begin{tabular}{l}
|\ifchilddocmanual|\\
|\input{\childdocname}|\\
|\||else|\\
\textit{document body with }|\input{|\textit{part}|}|\\
|\||fi|
\end{tabular}
\end{center}
%
The conditional |\ifchilddocmanual| is true whenever
a part to be included by |\input| is being compiled,
and the name of the part is stored in |\childdocname|.

%%%%%%%%%%%%%%%%%%%%%%%%%%%%%%%%%%%%%%%%
\DescribeMacro{\childdocby}
Each part to be included by |\input| should start with:
%
\begin{center}
\begin{tabular}{l}
|\input{childdoc.def}|\\
|\childdocby{|\textit{main}|}|\\
\end{tabular}
\end{center}
%
The directive |\childdocby| is similar to |\childdocof|
described in \secref{sec:include},
but the subsequent selection of content must be done manually.
To that end, both |\ifchilddoc| and |\ifchilddocmanual|
will be true upon processing of a part,
and the name of the part is stored in |\childdocname|.
Note that |\jobname| will be set to the filename of the current part
so that each part receives an individual |.aux| file
that does not interfere with the |.aux| file(s) of the main document.
This behaviour can be altered by the alternative form
|\childdocby[*]{|\textit{main}|}| (with a non-empty optional argument)
which uses the |.aux| file of the main document
by setting |\jobname| to \textit{main}.

%%%%%%%%%%%%%%%%%%%%%%%%%%%%%%%%%%%%%%%%%%%%%%%%%%%%%%%%%%%%%%%%%%%%%%%%%%%%%%%%
\subsection{Driver Development}
\label{sec:driver}

The \textsf{childdoc} mechanism can also be use for the development
of definition files such as \LaTeX{} styles or classes.
This case differs from the above setup with multiple parts
included by |\include| in that no |\includeonly| should be invoked.
This can be achieved by starting the include file
(before |\ProvidesPackage|) with:
%
\begin{center}
\begin{tabular}{l}
|\input{childdoc.def}|\\
|\childdocforward{|\textit{main}|}|\\
\end{tabular}
\end{center}
%
or alternatively with:
%
\begin{center}
\begin{tabular}{l}
|\input{childdoc.def}|\\
|\childdocby{|\textit{main}|}|\\
\end{tabular}
\end{center}
%
Both forms have slightly different effects as described above.
The main file is prepared as usual, see \secref{sec:include}.

%%%%%%%%%%%%%%%%%%%%%%%%%%%%%%%%%%%%%%%%%%%%%%%%%%%%%%%%%%%%%%%%%%%%%%%%%%%%%%%%
\subsection{Legacy Detection}
\label{sec:detection}

The directive |\childdocmain| in the main file can detect
whether the complete document or merely a child is to be compiled
even without using the directive |\childdocof|.
This method is deprecated because it is less robust
and there is no compelling reason to use it;
it is merely provided for backward compatibility
and it may be removed in future versions.

If the detection mechanism is to be used,
it is mandatory to correctly specify
the filename of the main file as the argument of |\childdocmain|:
%
\begin{center}
\begin{tabular}{l}
|\input{childdoc.def}|\\
|\childdocmain{|\textit{main}|}|\\
\end{tabular}
\end{center}
%
If |\jobname| does not match the argument \textit{main} of |\childdocmain|,
it is assumed that |\jobname| points to the child file to be compiled.
When using |\childdocmain| with the main file specified as argument,
it suffices to start a child file
with just |\input{|\textit{main}|}|
without loading of the package and using |\childdocof|.
If instead all processing is done
with the appropriate \textsf{childdoc} directives,
the argument of \textit{main} of |\childdocmain| can be empty.

An alternative version of the command line processing described
in \secref{sec:commandline} using the detection mechanism reads:
%
\begin{center}
|... -jobname "|\textit{target}|" "|[\textit{flags}]%
[|\def\jobname{|\textit{dest}|}|]|\input{|\textit{main}|}"|
\end{center}

%%%%%%%%%%%%%%%%%%%%%%%%%%%%%%%%%%%%%%%%%%%%%%%%%%%%%%%%%%%%%%%%%%%%%%%%%%%%%%%%
\subsection{Manual Code}
\label{sec:manual}

In case one cannot be certain whether the definitions file |childdoc.def|
is installed on the target \TeX{} distribution
and one prefers not to ship it,
it is conceivable to paste a few relevant commands into the sources.

To that end, drop all statements |\input{childdoc.def}|
and perform the replacements as outlined below.
Instead of |\childdocmain{|\textit{main}|}| add the following code
to the top of the main file:
%
\begin{center}
\begin{tabular}{l}
|\||ifdefined\childdocname\endinput\||fi\newif\ifchilddoc|\\
|\edef\childdocname{\scantokens\expandafter{\jobname\noexpand}}|\\
|\def\childdocmain{|\textit{main}|}\||ifx\childdocmain\childdocname\||else|\\
|\childdoctrue\includeonly{\childdocname}\let\jobname\childdocmain\||fi|\\
\end{tabular}
\end{center}
%
Instead of |\childdocof{|\textit{main}|}| just include the main file
at the top of each child file:
%
\begin{center}
|\input{|\textit{main}|}|
\end{center}
%
A simple redirection |\childdocforward{|\textit{dest}|}| is achieved by:
%
\begin{center}
|\def\jobname{|\textit{dest}|}\input{\jobname}|
\end{center}
%
The redirection with prefix
|\childdocforwardprefix[|\textit{prefix}|]{|\textit{dest}|}|
is accomplished by:
%
\begin{center}
\begin{tabular}{l}
|{\edef\jobname{\scantokens\expandafter{\jobname\noexpand}}|\\
|\def\redirectjob |\textit{prefix}|#1~~~{\gdef\jobname{|\textit{dest}|#1}}|\\
|\expandafter\redirectjob\jobname~~~}\input{\jobname}|
\end{tabular}
\end{center}

In an alternative approach,
child documents can be compiled by a specific command line
without additional code or specific definitions:
%
\begin{center}
|... -jobname "|\textit{target}|" "|[\textit{flags}]%
|\includeonly{|\textit{dest}|}\input{|\textit{main}|}"|
\end{center}
%

%%%%%%%%%%%%%%%%%%%%%%%%%%%%%%%%%%%%%%%%%%%%%%%%%%%%%%%%%%%%%%%%%%%%%%%%%%%%%%%%
%%%%%%%%%%%%%%%%%%%%%%%%%%%%%%%%%%%%%%%%%%%%%%%%%%%%%%%%%%%%%%%%%%%%%%%%%%%%%%%%
\section{Information}

%%%%%%%%%%%%%%%%%%%%%%%%%%%%%%%%%%%%%%%%%%%%%%%%%%%%%%%%%%%%%%%%%%%%%%%%%%%%%%%%
\subsection{Copyright}

Copyright \copyright{} 2017--2018 Niklas Beisert

This work may be distributed and/or modified under the
conditions of the \LaTeX{} Project Public License, either version 1.3
of this license or (at your option) any later version.
The latest version of this license is in
  \url{http://www.latex-project.org/lppl.txt}
and version 1.3 or later is part of all distributions of \LaTeX{}
version 2005/12/01 or later.

This work has the LPPL maintenance status `maintained'.

The Current Maintainer of this work is Niklas Beisert.

This work consists of the files |README.txt|, |childdoc.ins| and |childdoc.dtx|
as well as the derived files |childdoc.def|, |cdocsamp.tex|
with |cdocsch1.tex|, |cdocsch2.tex|, |cdocspt3.tex|, |cdocspt4.tex|,
|cdocsdrf.tex|, |cdocsfn1.tex|, |cdocsfn2.tex|
as well as |childdoc.pdf|.

%%%%%%%%%%%%%%%%%%%%%%%%%%%%%%%%%%%%%%%%%%%%%%%%%%%%%%%%%%%%%%%%%%%%%%%%%%%%%%%%
\subsection{Files and Installation}

The package consists of the files:
%
\begin{center}
\begin{tabular}{ll}
    |README.txt|   & readme file \\
    |childdoc.ins| & installation file \\
    |childdoc.dtx| & source file \\
    |childdoc.def| & definition file \\
    |cdocsamp.tex| & sample main file \\
    |cdocsch1.tex| & sample include file \\
    |cdocsch2.tex| & sample include file \\
    |cdocspt3.tex| & sample part file \\
    |cdocspt4.tex| & sample part file \\
    |cdocsdrf.tex| & sample redirection file \\
    |cdocsfn1.tex| & sample redirection file \\
    |cdocsfn2.tex| & sample redirection file \\
    |childdoc.pdf| & manual
\end{tabular}
\end{center}
%
The distribution consists of the files
|README.txt|, |childdoc.ins| and |childdoc.dtx|.
%
\begin{itemize}
\item
Run (pdf)\LaTeX{} on |childdoc.dtx|
to compile the manual |childdoc.pdf| (this file).
\item
Run \LaTeX{} on |childdoc.ins| to create the definitions file |childdoc.def|
and the sample |cdocsamp.tex| with include files
|cdocsch1.tex|, |cdocsch2.tex|, |cdocspt3.tex|, |cdocspt4.tex|,
|cdocsdrf.tex|, |cdocsfn1.tex|, |cdocsfn2.tex|.
Then copy the file |childdoc.def| to an appropriate directory of your \LaTeX{}
distribution, e.g.\ \textit{texmf-root}|/tex/latex/childdoc|.
\end{itemize}

%%%%%%%%%%%%%%%%%%%%%%%%%%%%%%%%%%%%%%%%%%%%%%%%%%%%%%%%%%%%%%%%%%%%%%%%%%%%%%%%
\subsection{Related CTAN Packages}

There are several other packages which offer a similar functionality:
%
\begin{itemize}
\item
The packages
\href{http://ctan.org/pkg/docmute}{\textsf{docmute}},
\href{http://ctan.org/pkg/includex}{\textsf{includex}} and
\href{http://ctan.org/pkg/standalone}{\textsf{standalone}}
provide commands to include only the document body of
a child file thus allowing both files to be compiled individually.
\item
The packages \href{http://ctan.org/pkg/subdocs}{\textsf{subdocs}}
and \href{http://ctan.org/pkg/subfiles}{\textsf{subfiles}}
provide structures in which the main and child documents can be
encapsulated and allowing them to be compiled individually.
The inclusion mechanism is different from the conventional |\include|.
\item
The package \href{http://ctan.org/pkg/combine}{\textsf{combine}}
is an elaborate solution to combine several documents into one.
\end{itemize}
%
See also the CTAN topic \href{http://ctan.org/topic/subdocs}{\textsf{subdocs}}
for further related packages.
The present package differs from the above solutions in that
a document structure constructed with the conventional |\include| mechanism
just needs two extra commands at the top of every file
such that all constituent files can be compiled individually.

%%%%%%%%%%%%%%%%%%%%%%%%%%%%%%%%%%%%%%%%%%%%%%%%%%%%%%%%%%%%%%%%%%%%%%%%%%%%%%%%
%\subsection{Feature Suggestions}
%
%The following is a list of features which may be useful for future
%versions of this package:
%%
%\begin{itemize}
%\item
%\ldots
%\end{itemize}

%%%%%%%%%%%%%%%%%%%%%%%%%%%%%%%%%%%%%%%%%%%%%%%%%%%%%%%%%%%%%%%%%%%%%%%%%%%%%%%%
\subsection{Revision History}

%%%%%%%%%%%%%%%%%%%%%%%%%%%%%%%%%%%%%%%%
\paragraph{v2.0:} 2018/12/30

\begin{itemize}
\item
immediate forward processing
\item
added |\childdocby| mechanism
\item
manual restructured
\end{itemize}

%%%%%%%%%%%%%%%%%%%%%%%%%%%%%%%%%%%%%%%%
\paragraph{v1.6:} 2018/01/17

\begin{itemize}
\item
application for development of include files
\item
corrections to manual
\end{itemize}

%%%%%%%%%%%%%%%%%%%%%%%%%%%%%%%%%%%%%%%%
\paragraph{v1.5:} 2017/05/21

\begin{itemize}
\item
more complete structuring introduced
\item
|\childdocof| introduced
\item
|\childdoc| renamed to |\childdocmain|
\item
|\childredirect| renamed to |\childdocforward| and |\childdocforwardprefix|
and functionality expanded
\end{itemize}

%%%%%%%%%%%%%%%%%%%%%%%%%%%%%%%%%%%%%%%%
\paragraph{v1.0:} 2017/04/27

\begin{itemize}
\item
manual and install package
\item
first version published on CTAN
\end{itemize}

%%%%%%%%%%%%%%%%%%%%%%%%%%%%%%%%%%%%%%%%
\paragraph{v0.6:} 2017/04/26

\begin{itemize}
\item
redirection mechanism added
\end{itemize}

%%%%%%%%%%%%%%%%%%%%%%%%%%%%%%%%%%%%%%%%
\paragraph{v0.5:} 2017/04/26

\begin{itemize}
\item
functionality in definition file
\end{itemize}


%%%%%%%%%%%%%%%%%%%%%%%%%%%%%%%%%%%%%%%%%%%%%%%%%%%%%%%%%%%%%%%%%%%%%%%%%%%%%%%%
%%%%%%%%%%%%%%%%%%%%%%%%%%%%%%%%%%%%%%%%%%%%%%%%%%%%%%%%%%%%%%%%%%%%%%%%%%%%%%%%
%%%%%%%%%%%%%%%%%%%%%%%%%%%%%%%%%%%%%%%%%%%%%%%%%%%%%%%%%%%%%%%%%%%%%%%%%%%%%%%%
\appendix

\settowidth\MacroIndent{\rmfamily\scriptsize 000\ }

 \DocInput{childdoc.dtx}

\end{document}
%</driver>
% \fi
%
% %%%%%%%%%%%%%%%%%%%%%%%%%%%%%%%%%%%%%%%%%%%%%%%%%%%%%%%%%%%%%%%%%%%%%%%%%%%%%%
% %%%%%%%%%%%%%%%%%%%%%%%%%%%%%%%%%%%%%%%%%%%%%%%%%%%%%%%%%%%%%%%%%%%%%%%%%%%%%%
% \section{Sample}
%\iffalse
%<*samplemain>
%\fi
%
% The following presents a sample document
% with two chapters, two parts, a title page,
% a compile flag as well as three forwarding files to set the flag.
% It consists of eight |.tex| files:
% \begin{center}
% \begin{tabular}{ll}
% |cdocsamp.tex|&main file\\
% |cdocsch1.tex|&include file for chapter 1\\
% |cdocsch2.tex|&include file for chapter 2\\
% |cdocspt3.tex|&include file for part 3\\
% |cdocspt4.tex|&include file for part 4\\
% |cdocsdrf.tex|&forwarding file for main file in draft mode\\
% |cdocsfi1.tex|&forwarding file for final version of chapter 1\\
% |cdocsfi2.tex|&forwarding file for final version of chapter 2\\
% \end{tabular}
% \end{center}
% Each of the eight files can be compiled directly by the \LaTeX{} compiler.
%
% %%%%%%%%%%%%%%%%%%%%%%%%%%%%%%%%%%%%%%
% \paragraph{Main File.}
%
% The main file is called |cdocsamp.tex|.
%
% Load the \textsf{childdoc} definitions and
% declare the filename for the main document:
%    \begin{macrocode}
\input{childdoc.def}
\childdocmain{}
%    \end{macrocode}

% Optional override for |\version| flag:
%    \begin{macrocode}
%%\ifchilddoc\else\providecommand{\version}{draft}\fi
%    \end{macrocode}

% Define the default values for the |\version| flag
% (|final| for the main file and |draft| for childs):
%    \begin{macrocode}
\ifchilddoc
\providecommand{\version}{draft}
\else
\providecommand{\version}{final}
\fi
%    \end{macrocode}

% Load the standard document class:
%    \begin{macrocode}
\documentclass[12pt]{article}
%    \end{macrocode}

% Start the document body:
%    \begin{macrocode}
\begin{document}
%    \end{macrocode}

% Declare a title page.
% Print title, part of document being processed and version flag:
%    \begin{macrocode}
\addtocounter{page}{-1}
\begin{center}
{\LARGE\bfseries{}childdoc example\par}
\vspace{1cm}
\ifchilddoc
\ifchilddocmanual part\else chapter\fi:
`\childdocname' of `\childdocjob'\par
\else
main document: `\childdocjob'\par
\fi
version: \version\par
\end{center}
\newpage
%    \end{macrocode}

% Manually include selected file,
% otherwise process as usual:
%    \begin{macrocode}
\ifchilddocmanual
\section*{part `\childdocname'}
\input{\childdocname}
\else
%    \end{macrocode}

% Include the two chapters:
%    \begin{macrocode}
\include{cdocsch1}
\include{cdocsch2}
%    \end{macrocode}

% Include the two parts unless only chapters should be displayed:
%    \begin{macrocode}
\ifchilddoc\else
\section{part three}
\input{cdocspt3}
\section{part four}
\input{cdocspt4}
\fi
%    \end{macrocode}

% Process as usual until here:
%    \begin{macrocode}
\fi
%    \end{macrocode}

% End of document body:
%    \begin{macrocode}
\end{document}
%    \end{macrocode}
%\iffalse
%</samplemain>
%\fi
%
% %%%%%%%%%%%%%%%%%%%%%%%%%%%%%%%%%%%%%%
% \paragraph{Chapter Include Files.}
%
% The include files are called |cdocsch1.tex| and |cdocsch2.tex|.
%
%\iffalse
%<*samplechap1|samplechap2>
%\fi

% Optional override for |\version| flag:
%    \begin{macrocode}
%%\providecommand{\version}{final}
%    \end{macrocode}

% Include the main document:
%    \begin{macrocode}
\input{childdoc.def}
\childdocof{cdocsamp}
%    \end{macrocode}

%\iffalse
%</samplechap1|samplechap2>
%\fi
%
%\iffalse
%<*samplechap1>
%\fi
% Some text for chapter 1:
%    \begin{macrocode}
\section{one}
some text in chapter one
%    \end{macrocode}

%\iffalse
%</samplechap1>
%\fi
% Some text for chapter 2:
%\iffalse
%<*samplechap2>
%\fi
%    \begin{macrocode}
\section{two}
more text in chapter two
%    \end{macrocode}

%\iffalse
%</samplechap2>
%\fi
%
% %%%%%%%%%%%%%%%%%%%%%%%%%%%%%%%%%%%%%%
% \paragraph{Part Include Files.}
%
% The include files are called |cdocspt3.tex| and |cdocspt4.tex|.
%
%\iffalse
%<*samplepart3|samplepart4>
%\fi

% Optional override for |\version| flag:
%    \begin{macrocode}
%%\providecommand{\version}{final}
%    \end{macrocode}

% Include the main document:
%    \begin{macrocode}
\input{childdoc.def}
\childdocby{cdocsamp}
%    \end{macrocode}

%\iffalse
%</samplepart3|samplepart4>
%\fi
%
%\iffalse
%<*samplepart3>
%\fi
% Some text for part 3:
%    \begin{macrocode}
some text in part three
%    \end{macrocode}

%\iffalse
%</samplepart3>
%\fi
% Some text for part 4:
%\iffalse
%<*samplepart4>
%\fi
%    \begin{macrocode}
more text in part four
%    \end{macrocode}

%\iffalse
%</samplepart4>
%\fi
%
% %%%%%%%%%%%%%%%%%%%%%%%%%%%%%%%%%%%%%%
% \paragraph{Forwarding for a Complete Draft.}
%
% The following forwarding file |cdocsdrf.tex|
% compiles the main document in draft mode:
%\iffalse
%<*sampledraft>
%\fi
%    \begin{macrocode}
\def\version{draft}
\input{childdoc.def}
\childdocforward{cdocsamp}
%    \end{macrocode}

%\iffalse
%</sampledraft>
%\fi
%
% %%%%%%%%%%%%%%%%%%%%%%%%%%%%%%%%%%%%%%
% \paragraph{Forwarding for Final Version of the Chapters.}
%
% The following forwarding files |cdocsfn1.tex| and |cdocsfn2.tex|
% (with identical content)
% compile the final versions of the child documents
% |cdocsch1.tex| and |cdocsch2.tex|, respectively:
%\iffalse
%<*samplefinal>
%\fi
%    \begin{macrocode}
\def\version{final}
\input{childdoc.def}
\childdocforwardprefix[cdocsamp]{cdocsfn}{cdocsch}
%    \end{macrocode}

%\iffalse
%</samplefinal>
%\fi
%
% %%%%%%%%%%%%%%%%%%%%%%%%%%%%%%%%%%%%%%
% \paragraph{Command Line Processing.}
%
% The following three command lines generate the output files
% |cdocscld|, |cdocscl1| and |cdocscl2|
% which should be identical to
% |cdocsdrf|, |cdocsch1| and |cdocsfn2|, respectively:
% \begin{center}
% \begin{tabular}{l}
% |latex -jobname cdocscld \|\\
% |  "\def\version{draft}\input{childdoc.def}\childdocforward{cdocsamp}"|\\
% |latex -jobname cdocscl1 \|\\
% |  "\input{childdoc.def}\childdocforward[cdocsamp]{cdocsch1}"|\\
% |latex -jobname cdocscl2 \|\\
% |  "\def\version{final}\input{childdoc.def}\childdocforward{cdocsch2}"|
% \end{tabular}
% \end{center}
% Note that the trailing backslash on each first line
% merely continues the input to the second line
% (for convenient cut ant paste).
% Furthermore, the command |latex| can be replaced by any
% of its alternative versions such as |pdflatex|.
%
% %%%%%%%%%%%%%%%%%%%%%%%%%%%%%%%%%%%%%%%%%%%%%%%%%%%%%%%%%%%%%%%%%%%%%%%%%%%%%%
% %%%%%%%%%%%%%%%%%%%%%%%%%%%%%%%%%%%%%%%%%%%%%%%%%%%%%%%%%%%%%%%%%%%%%%%%%%%%%%
% \section{Implementation}
%\iffalse
%<*package>
%\fi
%
% This section describes the definitions file |childdoc.def|.

% The definitions cannot be loaded using |\usepackage| or |\RequirePackage|
% which has a mechanism to prevent loading a style file more than once.
% When loading the definitions by means of |\input|
% multiple instances have to be prevented manually:
%\iffalse
%This code needs to be before the `\ProvidesFile' directive
%which is defined at the beginning of this file.
%Therefore it is also placed there and commented out here.
%</package>
%<*discard>
%\fi
%    \begin{macrocode}
\ifdefined\childdocmain\endinput\fi
%    \end{macrocode}
%\iffalse
%</discard>
%<*package>
%\fi
%
% \macro{\ifchilddoc}
% \macro{\ifchilddocmanual}
% The conditional |\ifchilddoc| tells whether a
% child (true) or main (false) document is being compiled.
% The conditional |\ifchilddocmanual| tells whether
% the |\includeonly| mechanism is used (false) or
% the selection of child files must be performed manually (true).
% The definitions initialise to false:
%    \begin{macrocode}
\newif\ifchilddoc
\newif\ifchilddocmanual
%    \end{macrocode}

% \macro{\childdocname}
% \macro{\childdocjob}
% The macro |\childdocname| stores the name of the main document
% to be compiled. The macro |\childdocjob| stores the name of
% the document on which the \LaTeX{} compiler was originally invoked.
% The content of |\jobname| cannot be compared
% to filenames specified in the source due to different catcodes.
% The following code rescans |\jobname|, stores the result
% in |\childdocname| and saves a copy in |\childdocjob|:
%    \begin{macrocode}
\edef\childdocname{\scantokens\expandafter{\jobname\noexpand}}
\let\childdocjob\childdocname
%    \end{macrocode}

% \macro{\childdocdisable}
% The macro |\childdocdisable| prevents the main file
% from being processed more than once.
% At this stage, the main document command |\childdocmain|
% is assumed to be called once again where it should do nothing.
% Any subsequent call to it should prevent
% a secondary processing of the main document
% It overwrites the forwarding commands
% |\childdocof| and |\childdocforward|
% with empty macros to prevent further inclusions of the main document:
%    \begin{macrocode}
\newcommand{\childdocdisable}
{
  \renewcommand{\childdocmain}[1]{\renewcommand{\childdocmain}[1]{\endinput}}
  \renewcommand{\childdocof}[1]{}
  \renewcommand{\childdocby}[2][]{}
  \renewcommand{\childdocforward}[2][]{}
  \renewcommand{\childdocdisable}{}
}
%    \end{macrocode}

% \macro{\childdocmain}
% The macro |\childdocmain| is to be called at the top of the main file
% with nothing or the main filename (without extension) as argument.
% First, it breaks loops.
% If the argument is not empty and does not match |\childdocname|
% (which is set by the first inclusion of |childdoc.def|),
% |\ifchilddoc| is set to true, |\includeonly| is applied to the child file
% and |\jobname| is set to the main file
% (for proper handling of |.aux| files):
%    \begin{macrocode}
\newcommand{\childdocmain}[1]
{
  \childdocdisable\childdocmain{}
  \if?#1?\else
    \begingroup
      \def\childdoctmp{#1}
      \ifx\childdoctmp\childdocname
        \def\childdoctmp{}
      \else
        \def\childdoctmp
        {
          \childdoctrue
          \includeonly{\childdocname}
          \def\childdocjob{#1}
          \def\jobname{#1}
        }
      \fi
      \expandafter
    \endgroup
    \childdoctmp
  \fi
}
%    \end{macrocode}

% \macro{\childdocof}
% The command |\childdocof| redirects
% compilation to the main file |#1|.
%    \begin{macrocode}
\newcommand{\childdocof}[1]
{
  \childdocdisable
  \childdoctrue
  \includeonly{\childdocname}
  \def\jobname{#1}
  \def\childdocjob{#1}
  \input{#1}
}
%    \end{macrocode}

% \macro{\childdocby}
% The command |\childdocby| ....
%    \begin{macrocode}
\newcommand{\childdocby}[2][]
{
  \childdocdisable
  \childdoctrue
  \childdocmanualtrue
  \if?#1?\else
    \def\jobname{#2}
  \fi
  \def\childdocjob{#2}
  \input{#2}
  \endinput
}
%    \end{macrocode}

% \macro{\childdocforward}
% The command |\childdocforward| redirects
% compilation to the main file or
% (if the optional argument is given) a child file.
% Parameters are set as if the main file
% or a child file starting with |\childdocof| was compiled.
% Then compilation is handed over to the main file:
%    \begin{macrocode}
\newcommand{\childdocforward}[2][]
{
  \begingroup
    \if?#1?
      \def\childdoctmp
      {
        \def\childdocname{#2}
        \def\childdocjob{#2}
        \def\jobname{#2}
        \input{#2}
        \endinput
      }
    \else
      \def\childdoctmp
      {
        \childdocdisable
        \def\childdocname{#2}
        \childdoctrue
        \includeonly{#2}
        \def\childdocjob{#1}
        \def\jobname{#1}
        \input{#1}
        \endinput
      }
    \fi
    \expandafter
  \endgroup
  \childdoctmp
}
%    \end{macrocode}

% \macro{\childdocforwardprefix}
% The command |\childdocforwardprefix| redirects
% compilation to the main or a child file by means of a pattern.
% The prefix |#1| in the current filename is replaced by |#2|
% and the suffix of the current filename is kept
% (it is assumed that the filename does not contain the substring `|~~~|'
% which is used as a delimiter).
% Compilation is handed over to the new file by |\childdocforward|:
%    \begin{macrocode}
\newcommand{\childdocforwardprefix}[3][]
{
  \begingroup
    \def\childdocextract #2##1~~~{\def\childdoctmp{\childdocforward[#1]{#3##1}}}
    \expandafter\childdocextract\childdocname~~~
    \expandafter
  \endgroup
  \childdoctmp
}
%    \end{macrocode}

% \macro{\childdoc}
% The deprecated macro |\childdoc| is a legacy version of |\childdocmain|:
%    \begin{macrocode}
\newcommand{\childdoc}{\childdocmain}
%    \end{macrocode}

% \macro{\childdocredirect}
% The deprecated macro |\childdocredirect| is a legacy version
% of |\childdocforward| and |\childdocforwardprefix|:
%    \begin{macrocode}
\newcommand{\childdocredirect}[2][]
{
  \begingroup
    \if?#1?
      \def\childdoctmp{\childdocforward{#2}}
    \else
      \def\childdoctmp{\childdocforwardprefix{#1}{#2}}
    \fi
    \expandafter
  \endgroup
  \childdoctmp
}
%    \end{macrocode}

%\iffalse
%</package>
%\fi
%
\endinput
|\\
|\childdocforward{|\textit{main}|}|\\
\end{tabular}
\end{center}
%
or alternatively with:
%
\begin{center}
\begin{tabular}{l}
|% \iffalse
%
% childdoc.dtx Copyright (C) 2017-2018 Niklas Beisert
%
% This work may be distributed and/or modified under the
% conditions of the LaTeX Project Public License, either version 1.3
% of this license or (at your option) any later version.
% The latest version of this license is in
%   http://www.latex-project.org/lppl.txt
% and version 1.3 or later is part of all distributions of LaTeX
% version 2005/12/01 or later.
%
% This work has the LPPL maintenance status `maintained'.
%
% The Current Maintainer of this work is Niklas Beisert.
%
% This work consists of the files childdoc.dtx and childdoc.ins
% and the derived files childdoc.def and cdocsamp.tex with
% cdocsch1.tex, cdocsch2.tex, cdocsdrf.tex, cdocsfn1.tex, cdocsfn2.tex.
%
%<package>\ifdefined\childdocmain\endinput\fi
%<package>\ProvidesFile{childdoc.def}[2018/12/30 v2.0 child document driver]
%<samplemain>\ProvidesFile{cdocsamp.tex}[2018/12/30 v2.0 sample for childdoc]
%<*driver>
%\ProvidesFile{childdoc.drv}[2018/12/30 v2.0 childdoc reference manual file]
\PassOptionsToClass{10pt,a4paper}{article}
\documentclass{ltxdoc}

\usepackage[margin=35mm]{geometry}
\usepackage{hyperref}
\usepackage{hyperxmp}
\usepackage[usenames]{color}

\hypersetup{colorlinks=true}
\hypersetup{pdfstartview=FitH}
\hypersetup{pdfpagemode=UseNone}
\hypersetup{pdfsource={}}
\hypersetup{pdflang={en-UK}}
\hypersetup{pdfcopyright={Copyright 2017-2018 Niklas Beisert.
  This work may be distributed and/or modified under the
  conditions of the LaTeX Project Public License, either version 1.3
  of this license or (at your option) any later version.}}
\hypersetup{pdflicenseurl={http://www.latex-project.org/lppl.txt}}
\hypersetup{pdfcontactaddress={ETH Zurich, ITP, HIT K,
  Wolfgang-Pauli-Strasse 27}}
\hypersetup{pdfcontactpostcode={8093}}
\hypersetup{pdfcontactcity={Zurich}}
\hypersetup{pdfcontactcountry={Switzerland}}
\hypersetup{pdfcontactemail={nbeisert@itp.phys.ethz.ch}}
\hypersetup{pdfcontacturl={http://people.phys.ethz.ch/\xmptilde nbeisert/}}

\newcommand{\secref}[1]{\hyperref[#1]{section \ref*{#1}}}

\parskip1ex
\parindent0pt
\let\olditemize\itemize
\def\itemize{\olditemize\parskip0pt}

\begin{document}

\title{The \textsf{childdoc} Package}
\hypersetup{pdftitle={The childdoc Package}}
\author{Niklas Beisert\\[2ex]
  Institut f\"ur Theoretische Physik\\
  Eidgen\"ossische Technische Hochschule Z\"urich\\
  Wolfgang-Pauli-Strasse 27, 8093 Z\"urich, Switzerland\\[1ex]
  \href{mailto:nbeisert@itp.phys.ethz.ch}
  {\texttt{nbeisert@itp.phys.ethz.ch}}}
\hypersetup{pdfauthor={Niklas Beisert}}
\hypersetup{pdfsubject={Manual for the LaTeX2e Package childdoc}}
\date{30 December 2018, \textsf{v2.0}}
\maketitle

\begin{abstract}\noindent
\textsf{childdoc} is a \LaTeXe{} package
that enables the direct compilation
of document sections included by |\include|
to individual files.
\end{abstract}

\begingroup
\parskip0ex
\tableofcontents
\endgroup

%%%%%%%%%%%%%%%%%%%%%%%%%%%%%%%%%%%%%%%%%%%%%%%%%%%%%%%%%%%%%%%%%%%%%%%%%%%%%%%%
%%%%%%%%%%%%%%%%%%%%%%%%%%%%%%%%%%%%%%%%%%%%%%%%%%%%%%%%%%%%%%%%%%%%%%%%%%%%%%%%
\section{Introduction}

\LaTeX{} provides a mechanism to structure a large document (such as a book)
into a main file and several child files (containing the chapters)
using the |\include| command.
This mechanism is beneficial for documents
which span hundreds of pages in order to
make the source file(s) more manageable.
Moreover, compilation can be restricted to
selected child files by means of the |\includeonly| command.
The latter feature can be used to reduce the compilation time while editing
(this was significantly more useful in the earlier days of \LaTeX{})
or to generate a smaller document which is easier to navigate.
Another application of |\includeonly| is to generate
documents consisting of selected parts of the complete document.

However, there are a few drawbacks of the plain |\include| mechanism:
\begin{itemize}
\item
The child files cannot be compiled on their own,
they can only be compiled via the main file.
A naive editing environment
(such as a text editor with an option
to have the current file processed by \LaTeX)
may require one to switch to the main file before compiling;
attempting to compile the child file produces errors.
\item
The main file must be modified (each time)
to adjust the |\includeonly| command
to the present needs. This easily leaves the main file in a messy state.
\item
The generated document will always carry the filename
of the main document. This is inconvenient if
several child files are to be compiled and
to be kept for distribution.
\end{itemize}

The present package provides a simple interface
to make child files individually compilable by \LaTeX{}.
Compiling a child file then has the same effect as compiling
the main file with an |\includeonly| command
to select the appropriate child.
Moreover the generated document will carry the name of the child
rather than the main file.
This resolves all three above issues.

This feature is meant to make the editing of books,
thesis documents and lecture notes somewhat more convenient.
However, the package can also be used efficiently for
composing a series of documents (such as exercise sheets)
which are typically distributed individually.
It then assists the author in generating the individual documents
(potentially in different versions)
as well as a document containing the collected series.
Another application is in developing style files
or other kinds of included material
where compilation of the style file could redirect
to a sample or test file.

%%%%%%%%%%%%%%%%%%%%%%%%%%%%%%%%%%%%%%%%%%%%%%%%%%%%%%%%%%%%%%%%%%%%%%%%%%%%%%%%
%%%%%%%%%%%%%%%%%%%%%%%%%%%%%%%%%%%%%%%%%%%%%%%%%%%%%%%%%%%%%%%%%%%%%%%%%%%%%%%%
\section{Usage}

First of all, the package \textsf{childdoc} is \emph{not} a standard
\LaTeXe{} |.sty| style file! Therefore it needs to be invoked in
a non-standard way.

%%%%%%%%%%%%%%%%%%%%%%%%%%%%%%%%%%%%%%%%%%%%%%%%%%%%%%%%%%%%%%%%%%%%%%%%%%%%%%%%
\subsection{Included Files}
\label{sec:include}

%%%%%%%%%%%%%%%%%%%%%%%%%%%%%%%%%%%%%%%%
\DescribeMacro{\childdocmain}
To use the package, add the commands
\begin{center}
\begin{tabular}{l}
|\input{childdoc.def}|\\
|\childdocmain{}|\\
\end{tabular}
\end{center}
at the very top of the main \LaTeX{} file,
in particular \emph{before} the |\documentclass| statement!
The argument of |\childdocmain| should be left empty
(but it must be present).

%%%%%%%%%%%%%%%%%%%%%%%%%%%%%%%%%%%%%%%%
\DescribeMacro{\childdocof}
Furthermore, add the commands
\begin{center}
\begin{tabular}{l}
|\input{childdoc.def}|\\
|\childdocof{|\textit{main}|}|\\
\end{tabular}
\end{center}
at the top of every child file \textit{child}
which is included by |\include{|\textit{child}|}|
from within the main file
(or at least for those files to be compiled individually).
The argument \textit{main} must be the filename of the main file.

There are a couple of
considerations in setting up the main and child documents:

%%%%%%%%%%%%%%%%%%%%%%%%%%%%%%%%%%%%%%%%
\paragraph{Restrictions.}

Please note the following restrictions:
\begin{itemize}
\item
|\childdocmain| must be called with one argument \textit{main}
to ensure compatibility with earlier version of the package.
It must either be empty (|\childdocmain{}|)
or precisely match the filename of the main file in which it is specified.
See \secref{sec:detection} for further information.
\item
The filename \textit{main} must be specified without the |.tex| extension.
\item
The filename \textit{main} is case sensitive
(even in case-insensitive file systems)
due to internal string comparison.
\item
The argument \textit{main} should be fully expanded, it cannot be a macro.
\item
Subdirectories and special characters should be avoided in filenames.
\item
The command |\childdocmain{|\textit{main}|}| must be followed by a whitespace.
It should not be followed immediately by another command
or by a comment mark `|%|'.
This is because the \TeX{} parser reads the token immediately following
the argument of |\childdocmain| and puts it
at the beginning of every child section;
however, a white\-space is ignored.
\end{itemize}

%%%%%%%%%%%%%%%%%%%%%%%%%%%%%%%%%%%%%%%%
\paragraph{Content of Main File.}

It is advisable to place all content in the child files included by |\include|.
Any output contained in the main file will appear in all child documents
unless suppressed manually;
it cannot be suppressed automatically by the |\includeonly| directive
and thus should normally be avoided.
A method to include some content in the main file
by means of conditional processing is described in \secref{sec:conditional}.

%%%%%%%%%%%%%%%%%%%%%%%%%%%%%%%%%%%%%%%%
\paragraph{Page Numbering.}

When only a part of the document is compiled,
the appropriate numbering of pages
(as well as other status parameters)
is determined from the |.aux| files.
The latter contain information from previous passes.
However this information needs to propagate through
all intermediate child documents.
Therefore the page numbering in child documents may well
be inconsistent until the complete document is compiled at least once.

A useful (if unconventional) way to always ensure a consistent
page numbering is to restart the numbering in each child document
and denote the pages by `\textit{child}|.|\textit{page}'
where \textit{child} represents the chapter/section number of the child file.
This can be achieved by the command
|\numberwithin{page}{|\textit{child}|}|
of the \textsf{amsmath} package
where \textit{child} can be |chapter| or |section|
depending on the chosen structuring.
Alternatively, one can modify the macro |\thepage| appropriately
and reset the counter |page| at the start of each child file.

%%%%%%%%%%%%%%%%%%%%%%%%%%%%%%%%%%%%%%%%%%%%%%%%%%%%%%%%%%%%%%%%%%%%%%%%%%%%%%%%
\subsection{Conditional Processing}
\label{sec:conditional}

The package provides a mechanism to compile different versions
of a document. To customise the versions further some conditional processing
can come in handy to distinguish which version is being compiled.
The package provides two macros to describe the compilation context:

%%%%%%%%%%%%%%%%%%%%%%%%%%%%%%%%%%%%%%%%
\DescribeMacro{\ifchilddoc}
The conditional |\ifchilddoc| distinguishes between the compilation of
child documents and the main document:
%
\begin{center}
|\ifchilddoc |\textit{child-code}| |[|\||else |\textit{main-code}]| \||fi|
\end{center}

%%%%%%%%%%%%%%%%%%%%%%%%%%%%%%%%%%%%%%%%
\DescribeMacro{\childdocname}
\DescribeMacro{\childdocjob}
The macro |\childdocname| contains the filename (without extension)
of the main or child file being processed.
Note that |\childdocjob| will always contain the name of the main file.

%%%%%%%%%%%%%%%%%%%%%%%%%%%%%%%%%%%%%%%%
\paragraph{Title Page.}

Conditional processing can be used to include a title or banner page
in the main document when proper precautions are taken.
Importantly, the code in the main file should ensure that the page counter
(as well as other status parameters which are stored in the |.aux| files)
takes the same value after the conditional processing.
Otherwise the page numbers may take divergent values
depending on which part is compiled.

For example, a title page could be declared by:
%
\begin{center}
\begin{tabular}{l}
|\ifchilddoc\||else|\\
|\addtocounter{page}{-1}|\\
\textit{code for title page}\\
|\newpage|\\
|\||fi|
\end{tabular}
\end{center}
%
A banner page for the child documents can be generated by:
%
\begin{center}
\begin{tabular}{l}
|\ifchilddoc|\\
|\addtocounter{page}{-1}|\\
\textit{code for banner page}\\
|\newpage|\\
|\||fi|
\end{tabular}
\end{center}
%
Here one could write a message such as:
\begin{center}
|This is the part \childdocname{} of \childdocjob{}.|
\end{center}

%%%%%%%%%%%%%%%%%%%%%%%%%%%%%%%%%%%%%%%%%%%%%%%%%%%%%%%%%%%%%%%%%%%%%%%%%%%%%%%%
\subsection{Flags}
\label{sec:flags}

The package makes it easy to generate different versions
of the main or child documents.
To this end compilation flags can be defined
and assigned different default values.
They will be particularly useful in conjunction
with the forwarding mechanism described in \secref{sec:forward}.

For example, it may be useful to have a flag |\version|
which can be set to |draft| or |final|.
The document source will contain some conditional code
depending on the value of |\version|.
Suppose further, the flag should default to |final| for the main file
and to |draft| for child files
which is a natural assignment for editing the document.
This is achieved by placing the following code
in the preamble of the main document
(below the |\childdocmain| directive):
%
\begin{center}
\begin{tabular}{l}
|\ifchilddoc|\\
|\providecommand{\version}{draft}|\\
|\||else|\\
|\providecommand{\version}{final}|\\
|\||fi|
\end{tabular}
\end{center}
%
The definition by |\providecommand| makes sure
that previous definitions are not overwritten.
Further statements |\providecommand{\version}{...}|
can thus be added before the above code to override it.

For the main file, one might add a line
(between |\childdocmain| and the above block)
%
\begin{center}
|%\ifchilddoc\||else\providecommand{\version}{draft}\||fi|
\end{center}
%
which can be uncommented to produce a draft version.
Likewise one can add a line to the very top of a child file
(above the |\childdocof{|\textit{main}|}| directive)
%
\begin{center}
|%\providecommand{\version}{final}|
\end{center}
%
which can be uncommented to produce the final version of this child document.

%%%%%%%%%%%%%%%%%%%%%%%%%%%%%%%%%%%%%%%%%%%%%%%%%%%%%%%%%%%%%%%%%%%%%%%%%%%%%%%%
\subsection{Forwarding}
\label{sec:forward}

Different versions of the main or child documents
using compilation flags as described in \secref{sec:flags}
can be (permanently) stored in different files
for convenient compilation, viewing and distribution.
To this end, the package defines a command
to pass on compilation to a different file:

%%%%%%%%%%%%%%%%%%%%%%%%%%%%%%%%%%%%%%%%
\DescribeMacro{\childdocforward}
The command |\childdocforward| redirects processing to
another source file:
%
\begin{center}
\begin{tabular}{l}
|\input{childdoc.def}|\\
|\childdocforward[|\textit{main}|]{|\textit{dest}|}|\\
\end{tabular}
\end{center}
%
The argument \textit{dest} is the destination file
(without extension).
It should be the main file or one of the child files.
Note that further \textsf{childdoc} directives
such as |\childdocof| and |\childdocforward|
in the indicated file will be processed in this form.
The optional argument \textit{main}
passes on directly to the main file \textit{main}
while pretending to compile the child \textit{dest}.
This form behaves as if \textit{dest}
issues |\childdocof{|\textit{main}|}| right away,
and no further \textsf{childdoc} directives will be processed.

%%%%%%%%%%%%%%%%%%%%%%%%%%%%%%%%%%%%%%%%
\DescribeMacro{\...prefix}
In the alternative form |\childdocforwardprefix|,
%
\begin{center}
\begin{tabular}{l}
|\input{childdoc.def}|\\
|\childdocforwardprefix[|\textit{main}|]{|\textit{prefix}|}{|\textit{dest}|}|
\end{tabular}
\end{center}
%
the destination file is determined by a pattern
depending on the current file:
To make this work, the current file must be called
`{\textit{prefix}\hspace{0.2em}\textit{suffix}}'
with \textit{prefix} matching precisely the argument.
Processing is then passed on to the file
`{\textit{dest}\hspace{0.2em}\textit{suffix}}'.
Surely, the same effect is achieved by
directly specifying the
argument `{\textit{dest}\hspace{0.2em}\textit{suffix}}'
in the first form.
However, that requires to set up a different file
for each child. With the alternative form of the command
all these files can have exactly the same content
which simplifies setting them up and maintaining them.

For example, the following file |draft.tex|
with a compilation flag |\version| as described in \secref{sec:flags}
compiles the main document as a draft:
%
\begin{center}
\begin{tabular}{l}
|\def\version{draft}|\\
|\input{childdoc.def}|\\
|\childdocforward{|\textit{main}|}|
\end{tabular}
\end{center}
%
Likewise, the following files |final|\textit{nn}|.tex|
compile the final version of the child document
|child|\textit{nn}|.tex|:
%
\begin{center}
\begin{tabular}{l}
|\def\version{final}|\\
|\input{childdoc.def}|\\
|\childdocforwardprefix{final}{child}|
\end{tabular}
\end{center}
%

Note that when several versions of a main file and/or of each child file
are to be generated, it may be convenient to set up a |Makefile| or
shell script to automatise the process.

%%%%%%%%%%%%%%%%%%%%%%%%%%%%%%%%%%%%%%%%%%%%%%%%%%%%%%%%%%%%%%%%%%%%%%%%%%%%%%%%
\subsection{Command Line Processing}
\label{sec:commandline}

The effect of redirection files can also be achieved by invoking
the \LaTeX{} compiler with a more elaborate command line.
Most conveniently this should be done as part
of a shell script or a |Makefile|.

When using \textsf{childdoc} in the main file, the following
command lines effectively perform a redirection
(note that depending on the shell being used,
backslashes may have to be doubled: `|\|' $\to$ `|\\|'):
%
\begin{center}
|... -jobname "|\textit{target}|" |\\|"|[\textit{flags}]%
|\input{childdoc.def}\childdocforward[|\textit{main}|]{|\textit{dest}|}"|
\end{center}
%
Here \textit{target} is the name of the output file,
\textit{main} is the name of the main file
and \textit{dest} is the name of the main or child file to be processed
(all filenames without extensions).
The optional argument \textit{main} can be omitted
if \textit{main} matches \textit{dest}.
Optionally, compilation \textit{flags} can be defined via |\def| commands.
This command line makes the \TeX{} engine believe
it is compiling the file \textit{target}
whose content is specified as the latter parameter.
The provided code then forwards the processing to
\textit{main} or \textit{dest} as described in \secref{sec:forward}.

%%%%%%%%%%%%%%%%%%%%%%%%%%%%%%%%%%%%%%%%%%%%%%%%%%%%%%%%%%%%%%%%%%%%%%%%%%%%%%%%
\subsection{Include by Input}
\label{sec:input}

Including child documents by |\include| has some restrictions by design.
Most notably, the content of a child document always occupies
its own set of pages; pages cannot be shared between child documents.
Usually, this behaviour makes perfect sense
because each child document contain an essential part of the document.
However, in some situations it may be desirable to compose
a document from a collection of parts
without having mandatory page breaks between then.
For this case, the package
provides a mechanism to include parts
by |\input| which can also be processed individually.
However, by construction this mechanism
requires manual handling of the content to be output.

%%%%%%%%%%%%%%%%%%%%%%%%%%%%%%%%%%%%%%%%
\DescribeMacro{\ifchilddocmanual}
The main file should be prepared as usual, see \secref{sec:include}.
However, the document body must make a distinction
between processing of an individual part and of the main document, e.g.:
%
\begin{center}
\begin{tabular}{l}
|\ifchilddocmanual|\\
|\input{\childdocname}|\\
|\||else|\\
\textit{document body with }|\input{|\textit{part}|}|\\
|\||fi|
\end{tabular}
\end{center}
%
The conditional |\ifchilddocmanual| is true whenever
a part to be included by |\input| is being compiled,
and the name of the part is stored in |\childdocname|.

%%%%%%%%%%%%%%%%%%%%%%%%%%%%%%%%%%%%%%%%
\DescribeMacro{\childdocby}
Each part to be included by |\input| should start with:
%
\begin{center}
\begin{tabular}{l}
|\input{childdoc.def}|\\
|\childdocby{|\textit{main}|}|\\
\end{tabular}
\end{center}
%
The directive |\childdocby| is similar to |\childdocof|
described in \secref{sec:include},
but the subsequent selection of content must be done manually.
To that end, both |\ifchilddoc| and |\ifchilddocmanual|
will be true upon processing of a part,
and the name of the part is stored in |\childdocname|.
Note that |\jobname| will be set to the filename of the current part
so that each part receives an individual |.aux| file
that does not interfere with the |.aux| file(s) of the main document.
This behaviour can be altered by the alternative form
|\childdocby[*]{|\textit{main}|}| (with a non-empty optional argument)
which uses the |.aux| file of the main document
by setting |\jobname| to \textit{main}.

%%%%%%%%%%%%%%%%%%%%%%%%%%%%%%%%%%%%%%%%%%%%%%%%%%%%%%%%%%%%%%%%%%%%%%%%%%%%%%%%
\subsection{Driver Development}
\label{sec:driver}

The \textsf{childdoc} mechanism can also be use for the development
of definition files such as \LaTeX{} styles or classes.
This case differs from the above setup with multiple parts
included by |\include| in that no |\includeonly| should be invoked.
This can be achieved by starting the include file
(before |\ProvidesPackage|) with:
%
\begin{center}
\begin{tabular}{l}
|\input{childdoc.def}|\\
|\childdocforward{|\textit{main}|}|\\
\end{tabular}
\end{center}
%
or alternatively with:
%
\begin{center}
\begin{tabular}{l}
|\input{childdoc.def}|\\
|\childdocby{|\textit{main}|}|\\
\end{tabular}
\end{center}
%
Both forms have slightly different effects as described above.
The main file is prepared as usual, see \secref{sec:include}.

%%%%%%%%%%%%%%%%%%%%%%%%%%%%%%%%%%%%%%%%%%%%%%%%%%%%%%%%%%%%%%%%%%%%%%%%%%%%%%%%
\subsection{Legacy Detection}
\label{sec:detection}

The directive |\childdocmain| in the main file can detect
whether the complete document or merely a child is to be compiled
even without using the directive |\childdocof|.
This method is deprecated because it is less robust
and there is no compelling reason to use it;
it is merely provided for backward compatibility
and it may be removed in future versions.

If the detection mechanism is to be used,
it is mandatory to correctly specify
the filename of the main file as the argument of |\childdocmain|:
%
\begin{center}
\begin{tabular}{l}
|\input{childdoc.def}|\\
|\childdocmain{|\textit{main}|}|\\
\end{tabular}
\end{center}
%
If |\jobname| does not match the argument \textit{main} of |\childdocmain|,
it is assumed that |\jobname| points to the child file to be compiled.
When using |\childdocmain| with the main file specified as argument,
it suffices to start a child file
with just |\input{|\textit{main}|}|
without loading of the package and using |\childdocof|.
If instead all processing is done
with the appropriate \textsf{childdoc} directives,
the argument of \textit{main} of |\childdocmain| can be empty.

An alternative version of the command line processing described
in \secref{sec:commandline} using the detection mechanism reads:
%
\begin{center}
|... -jobname "|\textit{target}|" "|[\textit{flags}]%
[|\def\jobname{|\textit{dest}|}|]|\input{|\textit{main}|}"|
\end{center}

%%%%%%%%%%%%%%%%%%%%%%%%%%%%%%%%%%%%%%%%%%%%%%%%%%%%%%%%%%%%%%%%%%%%%%%%%%%%%%%%
\subsection{Manual Code}
\label{sec:manual}

In case one cannot be certain whether the definitions file |childdoc.def|
is installed on the target \TeX{} distribution
and one prefers not to ship it,
it is conceivable to paste a few relevant commands into the sources.

To that end, drop all statements |\input{childdoc.def}|
and perform the replacements as outlined below.
Instead of |\childdocmain{|\textit{main}|}| add the following code
to the top of the main file:
%
\begin{center}
\begin{tabular}{l}
|\||ifdefined\childdocname\endinput\||fi\newif\ifchilddoc|\\
|\edef\childdocname{\scantokens\expandafter{\jobname\noexpand}}|\\
|\def\childdocmain{|\textit{main}|}\||ifx\childdocmain\childdocname\||else|\\
|\childdoctrue\includeonly{\childdocname}\let\jobname\childdocmain\||fi|\\
\end{tabular}
\end{center}
%
Instead of |\childdocof{|\textit{main}|}| just include the main file
at the top of each child file:
%
\begin{center}
|\input{|\textit{main}|}|
\end{center}
%
A simple redirection |\childdocforward{|\textit{dest}|}| is achieved by:
%
\begin{center}
|\def\jobname{|\textit{dest}|}\input{\jobname}|
\end{center}
%
The redirection with prefix
|\childdocforwardprefix[|\textit{prefix}|]{|\textit{dest}|}|
is accomplished by:
%
\begin{center}
\begin{tabular}{l}
|{\edef\jobname{\scantokens\expandafter{\jobname\noexpand}}|\\
|\def\redirectjob |\textit{prefix}|#1~~~{\gdef\jobname{|\textit{dest}|#1}}|\\
|\expandafter\redirectjob\jobname~~~}\input{\jobname}|
\end{tabular}
\end{center}

In an alternative approach,
child documents can be compiled by a specific command line
without additional code or specific definitions:
%
\begin{center}
|... -jobname "|\textit{target}|" "|[\textit{flags}]%
|\includeonly{|\textit{dest}|}\input{|\textit{main}|}"|
\end{center}
%

%%%%%%%%%%%%%%%%%%%%%%%%%%%%%%%%%%%%%%%%%%%%%%%%%%%%%%%%%%%%%%%%%%%%%%%%%%%%%%%%
%%%%%%%%%%%%%%%%%%%%%%%%%%%%%%%%%%%%%%%%%%%%%%%%%%%%%%%%%%%%%%%%%%%%%%%%%%%%%%%%
\section{Information}

%%%%%%%%%%%%%%%%%%%%%%%%%%%%%%%%%%%%%%%%%%%%%%%%%%%%%%%%%%%%%%%%%%%%%%%%%%%%%%%%
\subsection{Copyright}

Copyright \copyright{} 2017--2018 Niklas Beisert

This work may be distributed and/or modified under the
conditions of the \LaTeX{} Project Public License, either version 1.3
of this license or (at your option) any later version.
The latest version of this license is in
  \url{http://www.latex-project.org/lppl.txt}
and version 1.3 or later is part of all distributions of \LaTeX{}
version 2005/12/01 or later.

This work has the LPPL maintenance status `maintained'.

The Current Maintainer of this work is Niklas Beisert.

This work consists of the files |README.txt|, |childdoc.ins| and |childdoc.dtx|
as well as the derived files |childdoc.def|, |cdocsamp.tex|
with |cdocsch1.tex|, |cdocsch2.tex|, |cdocspt3.tex|, |cdocspt4.tex|,
|cdocsdrf.tex|, |cdocsfn1.tex|, |cdocsfn2.tex|
as well as |childdoc.pdf|.

%%%%%%%%%%%%%%%%%%%%%%%%%%%%%%%%%%%%%%%%%%%%%%%%%%%%%%%%%%%%%%%%%%%%%%%%%%%%%%%%
\subsection{Files and Installation}

The package consists of the files:
%
\begin{center}
\begin{tabular}{ll}
    |README.txt|   & readme file \\
    |childdoc.ins| & installation file \\
    |childdoc.dtx| & source file \\
    |childdoc.def| & definition file \\
    |cdocsamp.tex| & sample main file \\
    |cdocsch1.tex| & sample include file \\
    |cdocsch2.tex| & sample include file \\
    |cdocspt3.tex| & sample part file \\
    |cdocspt4.tex| & sample part file \\
    |cdocsdrf.tex| & sample redirection file \\
    |cdocsfn1.tex| & sample redirection file \\
    |cdocsfn2.tex| & sample redirection file \\
    |childdoc.pdf| & manual
\end{tabular}
\end{center}
%
The distribution consists of the files
|README.txt|, |childdoc.ins| and |childdoc.dtx|.
%
\begin{itemize}
\item
Run (pdf)\LaTeX{} on |childdoc.dtx|
to compile the manual |childdoc.pdf| (this file).
\item
Run \LaTeX{} on |childdoc.ins| to create the definitions file |childdoc.def|
and the sample |cdocsamp.tex| with include files
|cdocsch1.tex|, |cdocsch2.tex|, |cdocspt3.tex|, |cdocspt4.tex|,
|cdocsdrf.tex|, |cdocsfn1.tex|, |cdocsfn2.tex|.
Then copy the file |childdoc.def| to an appropriate directory of your \LaTeX{}
distribution, e.g.\ \textit{texmf-root}|/tex/latex/childdoc|.
\end{itemize}

%%%%%%%%%%%%%%%%%%%%%%%%%%%%%%%%%%%%%%%%%%%%%%%%%%%%%%%%%%%%%%%%%%%%%%%%%%%%%%%%
\subsection{Related CTAN Packages}

There are several other packages which offer a similar functionality:
%
\begin{itemize}
\item
The packages
\href{http://ctan.org/pkg/docmute}{\textsf{docmute}},
\href{http://ctan.org/pkg/includex}{\textsf{includex}} and
\href{http://ctan.org/pkg/standalone}{\textsf{standalone}}
provide commands to include only the document body of
a child file thus allowing both files to be compiled individually.
\item
The packages \href{http://ctan.org/pkg/subdocs}{\textsf{subdocs}}
and \href{http://ctan.org/pkg/subfiles}{\textsf{subfiles}}
provide structures in which the main and child documents can be
encapsulated and allowing them to be compiled individually.
The inclusion mechanism is different from the conventional |\include|.
\item
The package \href{http://ctan.org/pkg/combine}{\textsf{combine}}
is an elaborate solution to combine several documents into one.
\end{itemize}
%
See also the CTAN topic \href{http://ctan.org/topic/subdocs}{\textsf{subdocs}}
for further related packages.
The present package differs from the above solutions in that
a document structure constructed with the conventional |\include| mechanism
just needs two extra commands at the top of every file
such that all constituent files can be compiled individually.

%%%%%%%%%%%%%%%%%%%%%%%%%%%%%%%%%%%%%%%%%%%%%%%%%%%%%%%%%%%%%%%%%%%%%%%%%%%%%%%%
%\subsection{Feature Suggestions}
%
%The following is a list of features which may be useful for future
%versions of this package:
%%
%\begin{itemize}
%\item
%\ldots
%\end{itemize}

%%%%%%%%%%%%%%%%%%%%%%%%%%%%%%%%%%%%%%%%%%%%%%%%%%%%%%%%%%%%%%%%%%%%%%%%%%%%%%%%
\subsection{Revision History}

%%%%%%%%%%%%%%%%%%%%%%%%%%%%%%%%%%%%%%%%
\paragraph{v2.0:} 2018/12/30

\begin{itemize}
\item
immediate forward processing
\item
added |\childdocby| mechanism
\item
manual restructured
\end{itemize}

%%%%%%%%%%%%%%%%%%%%%%%%%%%%%%%%%%%%%%%%
\paragraph{v1.6:} 2018/01/17

\begin{itemize}
\item
application for development of include files
\item
corrections to manual
\end{itemize}

%%%%%%%%%%%%%%%%%%%%%%%%%%%%%%%%%%%%%%%%
\paragraph{v1.5:} 2017/05/21

\begin{itemize}
\item
more complete structuring introduced
\item
|\childdocof| introduced
\item
|\childdoc| renamed to |\childdocmain|
\item
|\childredirect| renamed to |\childdocforward| and |\childdocforwardprefix|
and functionality expanded
\end{itemize}

%%%%%%%%%%%%%%%%%%%%%%%%%%%%%%%%%%%%%%%%
\paragraph{v1.0:} 2017/04/27

\begin{itemize}
\item
manual and install package
\item
first version published on CTAN
\end{itemize}

%%%%%%%%%%%%%%%%%%%%%%%%%%%%%%%%%%%%%%%%
\paragraph{v0.6:} 2017/04/26

\begin{itemize}
\item
redirection mechanism added
\end{itemize}

%%%%%%%%%%%%%%%%%%%%%%%%%%%%%%%%%%%%%%%%
\paragraph{v0.5:} 2017/04/26

\begin{itemize}
\item
functionality in definition file
\end{itemize}


%%%%%%%%%%%%%%%%%%%%%%%%%%%%%%%%%%%%%%%%%%%%%%%%%%%%%%%%%%%%%%%%%%%%%%%%%%%%%%%%
%%%%%%%%%%%%%%%%%%%%%%%%%%%%%%%%%%%%%%%%%%%%%%%%%%%%%%%%%%%%%%%%%%%%%%%%%%%%%%%%
%%%%%%%%%%%%%%%%%%%%%%%%%%%%%%%%%%%%%%%%%%%%%%%%%%%%%%%%%%%%%%%%%%%%%%%%%%%%%%%%
\appendix

\settowidth\MacroIndent{\rmfamily\scriptsize 000\ }

 \DocInput{childdoc.dtx}

\end{document}
%</driver>
% \fi
%
% %%%%%%%%%%%%%%%%%%%%%%%%%%%%%%%%%%%%%%%%%%%%%%%%%%%%%%%%%%%%%%%%%%%%%%%%%%%%%%
% %%%%%%%%%%%%%%%%%%%%%%%%%%%%%%%%%%%%%%%%%%%%%%%%%%%%%%%%%%%%%%%%%%%%%%%%%%%%%%
% \section{Sample}
%\iffalse
%<*samplemain>
%\fi
%
% The following presents a sample document
% with two chapters, two parts, a title page,
% a compile flag as well as three forwarding files to set the flag.
% It consists of eight |.tex| files:
% \begin{center}
% \begin{tabular}{ll}
% |cdocsamp.tex|&main file\\
% |cdocsch1.tex|&include file for chapter 1\\
% |cdocsch2.tex|&include file for chapter 2\\
% |cdocspt3.tex|&include file for part 3\\
% |cdocspt4.tex|&include file for part 4\\
% |cdocsdrf.tex|&forwarding file for main file in draft mode\\
% |cdocsfi1.tex|&forwarding file for final version of chapter 1\\
% |cdocsfi2.tex|&forwarding file for final version of chapter 2\\
% \end{tabular}
% \end{center}
% Each of the eight files can be compiled directly by the \LaTeX{} compiler.
%
% %%%%%%%%%%%%%%%%%%%%%%%%%%%%%%%%%%%%%%
% \paragraph{Main File.}
%
% The main file is called |cdocsamp.tex|.
%
% Load the \textsf{childdoc} definitions and
% declare the filename for the main document:
%    \begin{macrocode}
\input{childdoc.def}
\childdocmain{}
%    \end{macrocode}

% Optional override for |\version| flag:
%    \begin{macrocode}
%%\ifchilddoc\else\providecommand{\version}{draft}\fi
%    \end{macrocode}

% Define the default values for the |\version| flag
% (|final| for the main file and |draft| for childs):
%    \begin{macrocode}
\ifchilddoc
\providecommand{\version}{draft}
\else
\providecommand{\version}{final}
\fi
%    \end{macrocode}

% Load the standard document class:
%    \begin{macrocode}
\documentclass[12pt]{article}
%    \end{macrocode}

% Start the document body:
%    \begin{macrocode}
\begin{document}
%    \end{macrocode}

% Declare a title page.
% Print title, part of document being processed and version flag:
%    \begin{macrocode}
\addtocounter{page}{-1}
\begin{center}
{\LARGE\bfseries{}childdoc example\par}
\vspace{1cm}
\ifchilddoc
\ifchilddocmanual part\else chapter\fi:
`\childdocname' of `\childdocjob'\par
\else
main document: `\childdocjob'\par
\fi
version: \version\par
\end{center}
\newpage
%    \end{macrocode}

% Manually include selected file,
% otherwise process as usual:
%    \begin{macrocode}
\ifchilddocmanual
\section*{part `\childdocname'}
\input{\childdocname}
\else
%    \end{macrocode}

% Include the two chapters:
%    \begin{macrocode}
\include{cdocsch1}
\include{cdocsch2}
%    \end{macrocode}

% Include the two parts unless only chapters should be displayed:
%    \begin{macrocode}
\ifchilddoc\else
\section{part three}
\input{cdocspt3}
\section{part four}
\input{cdocspt4}
\fi
%    \end{macrocode}

% Process as usual until here:
%    \begin{macrocode}
\fi
%    \end{macrocode}

% End of document body:
%    \begin{macrocode}
\end{document}
%    \end{macrocode}
%\iffalse
%</samplemain>
%\fi
%
% %%%%%%%%%%%%%%%%%%%%%%%%%%%%%%%%%%%%%%
% \paragraph{Chapter Include Files.}
%
% The include files are called |cdocsch1.tex| and |cdocsch2.tex|.
%
%\iffalse
%<*samplechap1|samplechap2>
%\fi

% Optional override for |\version| flag:
%    \begin{macrocode}
%%\providecommand{\version}{final}
%    \end{macrocode}

% Include the main document:
%    \begin{macrocode}
\input{childdoc.def}
\childdocof{cdocsamp}
%    \end{macrocode}

%\iffalse
%</samplechap1|samplechap2>
%\fi
%
%\iffalse
%<*samplechap1>
%\fi
% Some text for chapter 1:
%    \begin{macrocode}
\section{one}
some text in chapter one
%    \end{macrocode}

%\iffalse
%</samplechap1>
%\fi
% Some text for chapter 2:
%\iffalse
%<*samplechap2>
%\fi
%    \begin{macrocode}
\section{two}
more text in chapter two
%    \end{macrocode}

%\iffalse
%</samplechap2>
%\fi
%
% %%%%%%%%%%%%%%%%%%%%%%%%%%%%%%%%%%%%%%
% \paragraph{Part Include Files.}
%
% The include files are called |cdocspt3.tex| and |cdocspt4.tex|.
%
%\iffalse
%<*samplepart3|samplepart4>
%\fi

% Optional override for |\version| flag:
%    \begin{macrocode}
%%\providecommand{\version}{final}
%    \end{macrocode}

% Include the main document:
%    \begin{macrocode}
\input{childdoc.def}
\childdocby{cdocsamp}
%    \end{macrocode}

%\iffalse
%</samplepart3|samplepart4>
%\fi
%
%\iffalse
%<*samplepart3>
%\fi
% Some text for part 3:
%    \begin{macrocode}
some text in part three
%    \end{macrocode}

%\iffalse
%</samplepart3>
%\fi
% Some text for part 4:
%\iffalse
%<*samplepart4>
%\fi
%    \begin{macrocode}
more text in part four
%    \end{macrocode}

%\iffalse
%</samplepart4>
%\fi
%
% %%%%%%%%%%%%%%%%%%%%%%%%%%%%%%%%%%%%%%
% \paragraph{Forwarding for a Complete Draft.}
%
% The following forwarding file |cdocsdrf.tex|
% compiles the main document in draft mode:
%\iffalse
%<*sampledraft>
%\fi
%    \begin{macrocode}
\def\version{draft}
\input{childdoc.def}
\childdocforward{cdocsamp}
%    \end{macrocode}

%\iffalse
%</sampledraft>
%\fi
%
% %%%%%%%%%%%%%%%%%%%%%%%%%%%%%%%%%%%%%%
% \paragraph{Forwarding for Final Version of the Chapters.}
%
% The following forwarding files |cdocsfn1.tex| and |cdocsfn2.tex|
% (with identical content)
% compile the final versions of the child documents
% |cdocsch1.tex| and |cdocsch2.tex|, respectively:
%\iffalse
%<*samplefinal>
%\fi
%    \begin{macrocode}
\def\version{final}
\input{childdoc.def}
\childdocforwardprefix[cdocsamp]{cdocsfn}{cdocsch}
%    \end{macrocode}

%\iffalse
%</samplefinal>
%\fi
%
% %%%%%%%%%%%%%%%%%%%%%%%%%%%%%%%%%%%%%%
% \paragraph{Command Line Processing.}
%
% The following three command lines generate the output files
% |cdocscld|, |cdocscl1| and |cdocscl2|
% which should be identical to
% |cdocsdrf|, |cdocsch1| and |cdocsfn2|, respectively:
% \begin{center}
% \begin{tabular}{l}
% |latex -jobname cdocscld \|\\
% |  "\def\version{draft}\input{childdoc.def}\childdocforward{cdocsamp}"|\\
% |latex -jobname cdocscl1 \|\\
% |  "\input{childdoc.def}\childdocforward[cdocsamp]{cdocsch1}"|\\
% |latex -jobname cdocscl2 \|\\
% |  "\def\version{final}\input{childdoc.def}\childdocforward{cdocsch2}"|
% \end{tabular}
% \end{center}
% Note that the trailing backslash on each first line
% merely continues the input to the second line
% (for convenient cut ant paste).
% Furthermore, the command |latex| can be replaced by any
% of its alternative versions such as |pdflatex|.
%
% %%%%%%%%%%%%%%%%%%%%%%%%%%%%%%%%%%%%%%%%%%%%%%%%%%%%%%%%%%%%%%%%%%%%%%%%%%%%%%
% %%%%%%%%%%%%%%%%%%%%%%%%%%%%%%%%%%%%%%%%%%%%%%%%%%%%%%%%%%%%%%%%%%%%%%%%%%%%%%
% \section{Implementation}
%\iffalse
%<*package>
%\fi
%
% This section describes the definitions file |childdoc.def|.

% The definitions cannot be loaded using |\usepackage| or |\RequirePackage|
% which has a mechanism to prevent loading a style file more than once.
% When loading the definitions by means of |\input|
% multiple instances have to be prevented manually:
%\iffalse
%This code needs to be before the `\ProvidesFile' directive
%which is defined at the beginning of this file.
%Therefore it is also placed there and commented out here.
%</package>
%<*discard>
%\fi
%    \begin{macrocode}
\ifdefined\childdocmain\endinput\fi
%    \end{macrocode}
%\iffalse
%</discard>
%<*package>
%\fi
%
% \macro{\ifchilddoc}
% \macro{\ifchilddocmanual}
% The conditional |\ifchilddoc| tells whether a
% child (true) or main (false) document is being compiled.
% The conditional |\ifchilddocmanual| tells whether
% the |\includeonly| mechanism is used (false) or
% the selection of child files must be performed manually (true).
% The definitions initialise to false:
%    \begin{macrocode}
\newif\ifchilddoc
\newif\ifchilddocmanual
%    \end{macrocode}

% \macro{\childdocname}
% \macro{\childdocjob}
% The macro |\childdocname| stores the name of the main document
% to be compiled. The macro |\childdocjob| stores the name of
% the document on which the \LaTeX{} compiler was originally invoked.
% The content of |\jobname| cannot be compared
% to filenames specified in the source due to different catcodes.
% The following code rescans |\jobname|, stores the result
% in |\childdocname| and saves a copy in |\childdocjob|:
%    \begin{macrocode}
\edef\childdocname{\scantokens\expandafter{\jobname\noexpand}}
\let\childdocjob\childdocname
%    \end{macrocode}

% \macro{\childdocdisable}
% The macro |\childdocdisable| prevents the main file
% from being processed more than once.
% At this stage, the main document command |\childdocmain|
% is assumed to be called once again where it should do nothing.
% Any subsequent call to it should prevent
% a secondary processing of the main document
% It overwrites the forwarding commands
% |\childdocof| and |\childdocforward|
% with empty macros to prevent further inclusions of the main document:
%    \begin{macrocode}
\newcommand{\childdocdisable}
{
  \renewcommand{\childdocmain}[1]{\renewcommand{\childdocmain}[1]{\endinput}}
  \renewcommand{\childdocof}[1]{}
  \renewcommand{\childdocby}[2][]{}
  \renewcommand{\childdocforward}[2][]{}
  \renewcommand{\childdocdisable}{}
}
%    \end{macrocode}

% \macro{\childdocmain}
% The macro |\childdocmain| is to be called at the top of the main file
% with nothing or the main filename (without extension) as argument.
% First, it breaks loops.
% If the argument is not empty and does not match |\childdocname|
% (which is set by the first inclusion of |childdoc.def|),
% |\ifchilddoc| is set to true, |\includeonly| is applied to the child file
% and |\jobname| is set to the main file
% (for proper handling of |.aux| files):
%    \begin{macrocode}
\newcommand{\childdocmain}[1]
{
  \childdocdisable\childdocmain{}
  \if?#1?\else
    \begingroup
      \def\childdoctmp{#1}
      \ifx\childdoctmp\childdocname
        \def\childdoctmp{}
      \else
        \def\childdoctmp
        {
          \childdoctrue
          \includeonly{\childdocname}
          \def\childdocjob{#1}
          \def\jobname{#1}
        }
      \fi
      \expandafter
    \endgroup
    \childdoctmp
  \fi
}
%    \end{macrocode}

% \macro{\childdocof}
% The command |\childdocof| redirects
% compilation to the main file |#1|.
%    \begin{macrocode}
\newcommand{\childdocof}[1]
{
  \childdocdisable
  \childdoctrue
  \includeonly{\childdocname}
  \def\jobname{#1}
  \def\childdocjob{#1}
  \input{#1}
}
%    \end{macrocode}

% \macro{\childdocby}
% The command |\childdocby| ....
%    \begin{macrocode}
\newcommand{\childdocby}[2][]
{
  \childdocdisable
  \childdoctrue
  \childdocmanualtrue
  \if?#1?\else
    \def\jobname{#2}
  \fi
  \def\childdocjob{#2}
  \input{#2}
  \endinput
}
%    \end{macrocode}

% \macro{\childdocforward}
% The command |\childdocforward| redirects
% compilation to the main file or
% (if the optional argument is given) a child file.
% Parameters are set as if the main file
% or a child file starting with |\childdocof| was compiled.
% Then compilation is handed over to the main file:
%    \begin{macrocode}
\newcommand{\childdocforward}[2][]
{
  \begingroup
    \if?#1?
      \def\childdoctmp
      {
        \def\childdocname{#2}
        \def\childdocjob{#2}
        \def\jobname{#2}
        \input{#2}
        \endinput
      }
    \else
      \def\childdoctmp
      {
        \childdocdisable
        \def\childdocname{#2}
        \childdoctrue
        \includeonly{#2}
        \def\childdocjob{#1}
        \def\jobname{#1}
        \input{#1}
        \endinput
      }
    \fi
    \expandafter
  \endgroup
  \childdoctmp
}
%    \end{macrocode}

% \macro{\childdocforwardprefix}
% The command |\childdocforwardprefix| redirects
% compilation to the main or a child file by means of a pattern.
% The prefix |#1| in the current filename is replaced by |#2|
% and the suffix of the current filename is kept
% (it is assumed that the filename does not contain the substring `|~~~|'
% which is used as a delimiter).
% Compilation is handed over to the new file by |\childdocforward|:
%    \begin{macrocode}
\newcommand{\childdocforwardprefix}[3][]
{
  \begingroup
    \def\childdocextract #2##1~~~{\def\childdoctmp{\childdocforward[#1]{#3##1}}}
    \expandafter\childdocextract\childdocname~~~
    \expandafter
  \endgroup
  \childdoctmp
}
%    \end{macrocode}

% \macro{\childdoc}
% The deprecated macro |\childdoc| is a legacy version of |\childdocmain|:
%    \begin{macrocode}
\newcommand{\childdoc}{\childdocmain}
%    \end{macrocode}

% \macro{\childdocredirect}
% The deprecated macro |\childdocredirect| is a legacy version
% of |\childdocforward| and |\childdocforwardprefix|:
%    \begin{macrocode}
\newcommand{\childdocredirect}[2][]
{
  \begingroup
    \if?#1?
      \def\childdoctmp{\childdocforward{#2}}
    \else
      \def\childdoctmp{\childdocforwardprefix{#1}{#2}}
    \fi
    \expandafter
  \endgroup
  \childdoctmp
}
%    \end{macrocode}

%\iffalse
%</package>
%\fi
%
\endinput
|\\
|\childdocby{|\textit{main}|}|\\
\end{tabular}
\end{center}
%
Both forms have slightly different effects as described above.
The main file is prepared as usual, see \secref{sec:include}.

%%%%%%%%%%%%%%%%%%%%%%%%%%%%%%%%%%%%%%%%%%%%%%%%%%%%%%%%%%%%%%%%%%%%%%%%%%%%%%%%
\subsection{Legacy Detection}
\label{sec:detection}

The directive |\childdocmain| in the main file can detect
whether the complete document or merely a child is to be compiled
even without using the directive |\childdocof|.
This method is deprecated because it is less robust
and there is no compelling reason to use it;
it is merely provided for backward compatibility
and it may be removed in future versions.

If the detection mechanism is to be used,
it is mandatory to correctly specify
the filename of the main file as the argument of |\childdocmain|:
%
\begin{center}
\begin{tabular}{l}
|% \iffalse
%
% childdoc.dtx Copyright (C) 2017-2018 Niklas Beisert
%
% This work may be distributed and/or modified under the
% conditions of the LaTeX Project Public License, either version 1.3
% of this license or (at your option) any later version.
% The latest version of this license is in
%   http://www.latex-project.org/lppl.txt
% and version 1.3 or later is part of all distributions of LaTeX
% version 2005/12/01 or later.
%
% This work has the LPPL maintenance status `maintained'.
%
% The Current Maintainer of this work is Niklas Beisert.
%
% This work consists of the files childdoc.dtx and childdoc.ins
% and the derived files childdoc.def and cdocsamp.tex with
% cdocsch1.tex, cdocsch2.tex, cdocsdrf.tex, cdocsfn1.tex, cdocsfn2.tex.
%
%<package>\ifdefined\childdocmain\endinput\fi
%<package>\ProvidesFile{childdoc.def}[2018/12/30 v2.0 child document driver]
%<samplemain>\ProvidesFile{cdocsamp.tex}[2018/12/30 v2.0 sample for childdoc]
%<*driver>
%\ProvidesFile{childdoc.drv}[2018/12/30 v2.0 childdoc reference manual file]
\PassOptionsToClass{10pt,a4paper}{article}
\documentclass{ltxdoc}

\usepackage[margin=35mm]{geometry}
\usepackage{hyperref}
\usepackage{hyperxmp}
\usepackage[usenames]{color}

\hypersetup{colorlinks=true}
\hypersetup{pdfstartview=FitH}
\hypersetup{pdfpagemode=UseNone}
\hypersetup{pdfsource={}}
\hypersetup{pdflang={en-UK}}
\hypersetup{pdfcopyright={Copyright 2017-2018 Niklas Beisert.
  This work may be distributed and/or modified under the
  conditions of the LaTeX Project Public License, either version 1.3
  of this license or (at your option) any later version.}}
\hypersetup{pdflicenseurl={http://www.latex-project.org/lppl.txt}}
\hypersetup{pdfcontactaddress={ETH Zurich, ITP, HIT K,
  Wolfgang-Pauli-Strasse 27}}
\hypersetup{pdfcontactpostcode={8093}}
\hypersetup{pdfcontactcity={Zurich}}
\hypersetup{pdfcontactcountry={Switzerland}}
\hypersetup{pdfcontactemail={nbeisert@itp.phys.ethz.ch}}
\hypersetup{pdfcontacturl={http://people.phys.ethz.ch/\xmptilde nbeisert/}}

\newcommand{\secref}[1]{\hyperref[#1]{section \ref*{#1}}}

\parskip1ex
\parindent0pt
\let\olditemize\itemize
\def\itemize{\olditemize\parskip0pt}

\begin{document}

\title{The \textsf{childdoc} Package}
\hypersetup{pdftitle={The childdoc Package}}
\author{Niklas Beisert\\[2ex]
  Institut f\"ur Theoretische Physik\\
  Eidgen\"ossische Technische Hochschule Z\"urich\\
  Wolfgang-Pauli-Strasse 27, 8093 Z\"urich, Switzerland\\[1ex]
  \href{mailto:nbeisert@itp.phys.ethz.ch}
  {\texttt{nbeisert@itp.phys.ethz.ch}}}
\hypersetup{pdfauthor={Niklas Beisert}}
\hypersetup{pdfsubject={Manual for the LaTeX2e Package childdoc}}
\date{30 December 2018, \textsf{v2.0}}
\maketitle

\begin{abstract}\noindent
\textsf{childdoc} is a \LaTeXe{} package
that enables the direct compilation
of document sections included by |\include|
to individual files.
\end{abstract}

\begingroup
\parskip0ex
\tableofcontents
\endgroup

%%%%%%%%%%%%%%%%%%%%%%%%%%%%%%%%%%%%%%%%%%%%%%%%%%%%%%%%%%%%%%%%%%%%%%%%%%%%%%%%
%%%%%%%%%%%%%%%%%%%%%%%%%%%%%%%%%%%%%%%%%%%%%%%%%%%%%%%%%%%%%%%%%%%%%%%%%%%%%%%%
\section{Introduction}

\LaTeX{} provides a mechanism to structure a large document (such as a book)
into a main file and several child files (containing the chapters)
using the |\include| command.
This mechanism is beneficial for documents
which span hundreds of pages in order to
make the source file(s) more manageable.
Moreover, compilation can be restricted to
selected child files by means of the |\includeonly| command.
The latter feature can be used to reduce the compilation time while editing
(this was significantly more useful in the earlier days of \LaTeX{})
or to generate a smaller document which is easier to navigate.
Another application of |\includeonly| is to generate
documents consisting of selected parts of the complete document.

However, there are a few drawbacks of the plain |\include| mechanism:
\begin{itemize}
\item
The child files cannot be compiled on their own,
they can only be compiled via the main file.
A naive editing environment
(such as a text editor with an option
to have the current file processed by \LaTeX)
may require one to switch to the main file before compiling;
attempting to compile the child file produces errors.
\item
The main file must be modified (each time)
to adjust the |\includeonly| command
to the present needs. This easily leaves the main file in a messy state.
\item
The generated document will always carry the filename
of the main document. This is inconvenient if
several child files are to be compiled and
to be kept for distribution.
\end{itemize}

The present package provides a simple interface
to make child files individually compilable by \LaTeX{}.
Compiling a child file then has the same effect as compiling
the main file with an |\includeonly| command
to select the appropriate child.
Moreover the generated document will carry the name of the child
rather than the main file.
This resolves all three above issues.

This feature is meant to make the editing of books,
thesis documents and lecture notes somewhat more convenient.
However, the package can also be used efficiently for
composing a series of documents (such as exercise sheets)
which are typically distributed individually.
It then assists the author in generating the individual documents
(potentially in different versions)
as well as a document containing the collected series.
Another application is in developing style files
or other kinds of included material
where compilation of the style file could redirect
to a sample or test file.

%%%%%%%%%%%%%%%%%%%%%%%%%%%%%%%%%%%%%%%%%%%%%%%%%%%%%%%%%%%%%%%%%%%%%%%%%%%%%%%%
%%%%%%%%%%%%%%%%%%%%%%%%%%%%%%%%%%%%%%%%%%%%%%%%%%%%%%%%%%%%%%%%%%%%%%%%%%%%%%%%
\section{Usage}

First of all, the package \textsf{childdoc} is \emph{not} a standard
\LaTeXe{} |.sty| style file! Therefore it needs to be invoked in
a non-standard way.

%%%%%%%%%%%%%%%%%%%%%%%%%%%%%%%%%%%%%%%%%%%%%%%%%%%%%%%%%%%%%%%%%%%%%%%%%%%%%%%%
\subsection{Included Files}
\label{sec:include}

%%%%%%%%%%%%%%%%%%%%%%%%%%%%%%%%%%%%%%%%
\DescribeMacro{\childdocmain}
To use the package, add the commands
\begin{center}
\begin{tabular}{l}
|\input{childdoc.def}|\\
|\childdocmain{}|\\
\end{tabular}
\end{center}
at the very top of the main \LaTeX{} file,
in particular \emph{before} the |\documentclass| statement!
The argument of |\childdocmain| should be left empty
(but it must be present).

%%%%%%%%%%%%%%%%%%%%%%%%%%%%%%%%%%%%%%%%
\DescribeMacro{\childdocof}
Furthermore, add the commands
\begin{center}
\begin{tabular}{l}
|\input{childdoc.def}|\\
|\childdocof{|\textit{main}|}|\\
\end{tabular}
\end{center}
at the top of every child file \textit{child}
which is included by |\include{|\textit{child}|}|
from within the main file
(or at least for those files to be compiled individually).
The argument \textit{main} must be the filename of the main file.

There are a couple of
considerations in setting up the main and child documents:

%%%%%%%%%%%%%%%%%%%%%%%%%%%%%%%%%%%%%%%%
\paragraph{Restrictions.}

Please note the following restrictions:
\begin{itemize}
\item
|\childdocmain| must be called with one argument \textit{main}
to ensure compatibility with earlier version of the package.
It must either be empty (|\childdocmain{}|)
or precisely match the filename of the main file in which it is specified.
See \secref{sec:detection} for further information.
\item
The filename \textit{main} must be specified without the |.tex| extension.
\item
The filename \textit{main} is case sensitive
(even in case-insensitive file systems)
due to internal string comparison.
\item
The argument \textit{main} should be fully expanded, it cannot be a macro.
\item
Subdirectories and special characters should be avoided in filenames.
\item
The command |\childdocmain{|\textit{main}|}| must be followed by a whitespace.
It should not be followed immediately by another command
or by a comment mark `|%|'.
This is because the \TeX{} parser reads the token immediately following
the argument of |\childdocmain| and puts it
at the beginning of every child section;
however, a white\-space is ignored.
\end{itemize}

%%%%%%%%%%%%%%%%%%%%%%%%%%%%%%%%%%%%%%%%
\paragraph{Content of Main File.}

It is advisable to place all content in the child files included by |\include|.
Any output contained in the main file will appear in all child documents
unless suppressed manually;
it cannot be suppressed automatically by the |\includeonly| directive
and thus should normally be avoided.
A method to include some content in the main file
by means of conditional processing is described in \secref{sec:conditional}.

%%%%%%%%%%%%%%%%%%%%%%%%%%%%%%%%%%%%%%%%
\paragraph{Page Numbering.}

When only a part of the document is compiled,
the appropriate numbering of pages
(as well as other status parameters)
is determined from the |.aux| files.
The latter contain information from previous passes.
However this information needs to propagate through
all intermediate child documents.
Therefore the page numbering in child documents may well
be inconsistent until the complete document is compiled at least once.

A useful (if unconventional) way to always ensure a consistent
page numbering is to restart the numbering in each child document
and denote the pages by `\textit{child}|.|\textit{page}'
where \textit{child} represents the chapter/section number of the child file.
This can be achieved by the command
|\numberwithin{page}{|\textit{child}|}|
of the \textsf{amsmath} package
where \textit{child} can be |chapter| or |section|
depending on the chosen structuring.
Alternatively, one can modify the macro |\thepage| appropriately
and reset the counter |page| at the start of each child file.

%%%%%%%%%%%%%%%%%%%%%%%%%%%%%%%%%%%%%%%%%%%%%%%%%%%%%%%%%%%%%%%%%%%%%%%%%%%%%%%%
\subsection{Conditional Processing}
\label{sec:conditional}

The package provides a mechanism to compile different versions
of a document. To customise the versions further some conditional processing
can come in handy to distinguish which version is being compiled.
The package provides two macros to describe the compilation context:

%%%%%%%%%%%%%%%%%%%%%%%%%%%%%%%%%%%%%%%%
\DescribeMacro{\ifchilddoc}
The conditional |\ifchilddoc| distinguishes between the compilation of
child documents and the main document:
%
\begin{center}
|\ifchilddoc |\textit{child-code}| |[|\||else |\textit{main-code}]| \||fi|
\end{center}

%%%%%%%%%%%%%%%%%%%%%%%%%%%%%%%%%%%%%%%%
\DescribeMacro{\childdocname}
\DescribeMacro{\childdocjob}
The macro |\childdocname| contains the filename (without extension)
of the main or child file being processed.
Note that |\childdocjob| will always contain the name of the main file.

%%%%%%%%%%%%%%%%%%%%%%%%%%%%%%%%%%%%%%%%
\paragraph{Title Page.}

Conditional processing can be used to include a title or banner page
in the main document when proper precautions are taken.
Importantly, the code in the main file should ensure that the page counter
(as well as other status parameters which are stored in the |.aux| files)
takes the same value after the conditional processing.
Otherwise the page numbers may take divergent values
depending on which part is compiled.

For example, a title page could be declared by:
%
\begin{center}
\begin{tabular}{l}
|\ifchilddoc\||else|\\
|\addtocounter{page}{-1}|\\
\textit{code for title page}\\
|\newpage|\\
|\||fi|
\end{tabular}
\end{center}
%
A banner page for the child documents can be generated by:
%
\begin{center}
\begin{tabular}{l}
|\ifchilddoc|\\
|\addtocounter{page}{-1}|\\
\textit{code for banner page}\\
|\newpage|\\
|\||fi|
\end{tabular}
\end{center}
%
Here one could write a message such as:
\begin{center}
|This is the part \childdocname{} of \childdocjob{}.|
\end{center}

%%%%%%%%%%%%%%%%%%%%%%%%%%%%%%%%%%%%%%%%%%%%%%%%%%%%%%%%%%%%%%%%%%%%%%%%%%%%%%%%
\subsection{Flags}
\label{sec:flags}

The package makes it easy to generate different versions
of the main or child documents.
To this end compilation flags can be defined
and assigned different default values.
They will be particularly useful in conjunction
with the forwarding mechanism described in \secref{sec:forward}.

For example, it may be useful to have a flag |\version|
which can be set to |draft| or |final|.
The document source will contain some conditional code
depending on the value of |\version|.
Suppose further, the flag should default to |final| for the main file
and to |draft| for child files
which is a natural assignment for editing the document.
This is achieved by placing the following code
in the preamble of the main document
(below the |\childdocmain| directive):
%
\begin{center}
\begin{tabular}{l}
|\ifchilddoc|\\
|\providecommand{\version}{draft}|\\
|\||else|\\
|\providecommand{\version}{final}|\\
|\||fi|
\end{tabular}
\end{center}
%
The definition by |\providecommand| makes sure
that previous definitions are not overwritten.
Further statements |\providecommand{\version}{...}|
can thus be added before the above code to override it.

For the main file, one might add a line
(between |\childdocmain| and the above block)
%
\begin{center}
|%\ifchilddoc\||else\providecommand{\version}{draft}\||fi|
\end{center}
%
which can be uncommented to produce a draft version.
Likewise one can add a line to the very top of a child file
(above the |\childdocof{|\textit{main}|}| directive)
%
\begin{center}
|%\providecommand{\version}{final}|
\end{center}
%
which can be uncommented to produce the final version of this child document.

%%%%%%%%%%%%%%%%%%%%%%%%%%%%%%%%%%%%%%%%%%%%%%%%%%%%%%%%%%%%%%%%%%%%%%%%%%%%%%%%
\subsection{Forwarding}
\label{sec:forward}

Different versions of the main or child documents
using compilation flags as described in \secref{sec:flags}
can be (permanently) stored in different files
for convenient compilation, viewing and distribution.
To this end, the package defines a command
to pass on compilation to a different file:

%%%%%%%%%%%%%%%%%%%%%%%%%%%%%%%%%%%%%%%%
\DescribeMacro{\childdocforward}
The command |\childdocforward| redirects processing to
another source file:
%
\begin{center}
\begin{tabular}{l}
|\input{childdoc.def}|\\
|\childdocforward[|\textit{main}|]{|\textit{dest}|}|\\
\end{tabular}
\end{center}
%
The argument \textit{dest} is the destination file
(without extension).
It should be the main file or one of the child files.
Note that further \textsf{childdoc} directives
such as |\childdocof| and |\childdocforward|
in the indicated file will be processed in this form.
The optional argument \textit{main}
passes on directly to the main file \textit{main}
while pretending to compile the child \textit{dest}.
This form behaves as if \textit{dest}
issues |\childdocof{|\textit{main}|}| right away,
and no further \textsf{childdoc} directives will be processed.

%%%%%%%%%%%%%%%%%%%%%%%%%%%%%%%%%%%%%%%%
\DescribeMacro{\...prefix}
In the alternative form |\childdocforwardprefix|,
%
\begin{center}
\begin{tabular}{l}
|\input{childdoc.def}|\\
|\childdocforwardprefix[|\textit{main}|]{|\textit{prefix}|}{|\textit{dest}|}|
\end{tabular}
\end{center}
%
the destination file is determined by a pattern
depending on the current file:
To make this work, the current file must be called
`{\textit{prefix}\hspace{0.2em}\textit{suffix}}'
with \textit{prefix} matching precisely the argument.
Processing is then passed on to the file
`{\textit{dest}\hspace{0.2em}\textit{suffix}}'.
Surely, the same effect is achieved by
directly specifying the
argument `{\textit{dest}\hspace{0.2em}\textit{suffix}}'
in the first form.
However, that requires to set up a different file
for each child. With the alternative form of the command
all these files can have exactly the same content
which simplifies setting them up and maintaining them.

For example, the following file |draft.tex|
with a compilation flag |\version| as described in \secref{sec:flags}
compiles the main document as a draft:
%
\begin{center}
\begin{tabular}{l}
|\def\version{draft}|\\
|\input{childdoc.def}|\\
|\childdocforward{|\textit{main}|}|
\end{tabular}
\end{center}
%
Likewise, the following files |final|\textit{nn}|.tex|
compile the final version of the child document
|child|\textit{nn}|.tex|:
%
\begin{center}
\begin{tabular}{l}
|\def\version{final}|\\
|\input{childdoc.def}|\\
|\childdocforwardprefix{final}{child}|
\end{tabular}
\end{center}
%

Note that when several versions of a main file and/or of each child file
are to be generated, it may be convenient to set up a |Makefile| or
shell script to automatise the process.

%%%%%%%%%%%%%%%%%%%%%%%%%%%%%%%%%%%%%%%%%%%%%%%%%%%%%%%%%%%%%%%%%%%%%%%%%%%%%%%%
\subsection{Command Line Processing}
\label{sec:commandline}

The effect of redirection files can also be achieved by invoking
the \LaTeX{} compiler with a more elaborate command line.
Most conveniently this should be done as part
of a shell script or a |Makefile|.

When using \textsf{childdoc} in the main file, the following
command lines effectively perform a redirection
(note that depending on the shell being used,
backslashes may have to be doubled: `|\|' $\to$ `|\\|'):
%
\begin{center}
|... -jobname "|\textit{target}|" |\\|"|[\textit{flags}]%
|\input{childdoc.def}\childdocforward[|\textit{main}|]{|\textit{dest}|}"|
\end{center}
%
Here \textit{target} is the name of the output file,
\textit{main} is the name of the main file
and \textit{dest} is the name of the main or child file to be processed
(all filenames without extensions).
The optional argument \textit{main} can be omitted
if \textit{main} matches \textit{dest}.
Optionally, compilation \textit{flags} can be defined via |\def| commands.
This command line makes the \TeX{} engine believe
it is compiling the file \textit{target}
whose content is specified as the latter parameter.
The provided code then forwards the processing to
\textit{main} or \textit{dest} as described in \secref{sec:forward}.

%%%%%%%%%%%%%%%%%%%%%%%%%%%%%%%%%%%%%%%%%%%%%%%%%%%%%%%%%%%%%%%%%%%%%%%%%%%%%%%%
\subsection{Include by Input}
\label{sec:input}

Including child documents by |\include| has some restrictions by design.
Most notably, the content of a child document always occupies
its own set of pages; pages cannot be shared between child documents.
Usually, this behaviour makes perfect sense
because each child document contain an essential part of the document.
However, in some situations it may be desirable to compose
a document from a collection of parts
without having mandatory page breaks between then.
For this case, the package
provides a mechanism to include parts
by |\input| which can also be processed individually.
However, by construction this mechanism
requires manual handling of the content to be output.

%%%%%%%%%%%%%%%%%%%%%%%%%%%%%%%%%%%%%%%%
\DescribeMacro{\ifchilddocmanual}
The main file should be prepared as usual, see \secref{sec:include}.
However, the document body must make a distinction
between processing of an individual part and of the main document, e.g.:
%
\begin{center}
\begin{tabular}{l}
|\ifchilddocmanual|\\
|\input{\childdocname}|\\
|\||else|\\
\textit{document body with }|\input{|\textit{part}|}|\\
|\||fi|
\end{tabular}
\end{center}
%
The conditional |\ifchilddocmanual| is true whenever
a part to be included by |\input| is being compiled,
and the name of the part is stored in |\childdocname|.

%%%%%%%%%%%%%%%%%%%%%%%%%%%%%%%%%%%%%%%%
\DescribeMacro{\childdocby}
Each part to be included by |\input| should start with:
%
\begin{center}
\begin{tabular}{l}
|\input{childdoc.def}|\\
|\childdocby{|\textit{main}|}|\\
\end{tabular}
\end{center}
%
The directive |\childdocby| is similar to |\childdocof|
described in \secref{sec:include},
but the subsequent selection of content must be done manually.
To that end, both |\ifchilddoc| and |\ifchilddocmanual|
will be true upon processing of a part,
and the name of the part is stored in |\childdocname|.
Note that |\jobname| will be set to the filename of the current part
so that each part receives an individual |.aux| file
that does not interfere with the |.aux| file(s) of the main document.
This behaviour can be altered by the alternative form
|\childdocby[*]{|\textit{main}|}| (with a non-empty optional argument)
which uses the |.aux| file of the main document
by setting |\jobname| to \textit{main}.

%%%%%%%%%%%%%%%%%%%%%%%%%%%%%%%%%%%%%%%%%%%%%%%%%%%%%%%%%%%%%%%%%%%%%%%%%%%%%%%%
\subsection{Driver Development}
\label{sec:driver}

The \textsf{childdoc} mechanism can also be use for the development
of definition files such as \LaTeX{} styles or classes.
This case differs from the above setup with multiple parts
included by |\include| in that no |\includeonly| should be invoked.
This can be achieved by starting the include file
(before |\ProvidesPackage|) with:
%
\begin{center}
\begin{tabular}{l}
|\input{childdoc.def}|\\
|\childdocforward{|\textit{main}|}|\\
\end{tabular}
\end{center}
%
or alternatively with:
%
\begin{center}
\begin{tabular}{l}
|\input{childdoc.def}|\\
|\childdocby{|\textit{main}|}|\\
\end{tabular}
\end{center}
%
Both forms have slightly different effects as described above.
The main file is prepared as usual, see \secref{sec:include}.

%%%%%%%%%%%%%%%%%%%%%%%%%%%%%%%%%%%%%%%%%%%%%%%%%%%%%%%%%%%%%%%%%%%%%%%%%%%%%%%%
\subsection{Legacy Detection}
\label{sec:detection}

The directive |\childdocmain| in the main file can detect
whether the complete document or merely a child is to be compiled
even without using the directive |\childdocof|.
This method is deprecated because it is less robust
and there is no compelling reason to use it;
it is merely provided for backward compatibility
and it may be removed in future versions.

If the detection mechanism is to be used,
it is mandatory to correctly specify
the filename of the main file as the argument of |\childdocmain|:
%
\begin{center}
\begin{tabular}{l}
|\input{childdoc.def}|\\
|\childdocmain{|\textit{main}|}|\\
\end{tabular}
\end{center}
%
If |\jobname| does not match the argument \textit{main} of |\childdocmain|,
it is assumed that |\jobname| points to the child file to be compiled.
When using |\childdocmain| with the main file specified as argument,
it suffices to start a child file
with just |\input{|\textit{main}|}|
without loading of the package and using |\childdocof|.
If instead all processing is done
with the appropriate \textsf{childdoc} directives,
the argument of \textit{main} of |\childdocmain| can be empty.

An alternative version of the command line processing described
in \secref{sec:commandline} using the detection mechanism reads:
%
\begin{center}
|... -jobname "|\textit{target}|" "|[\textit{flags}]%
[|\def\jobname{|\textit{dest}|}|]|\input{|\textit{main}|}"|
\end{center}

%%%%%%%%%%%%%%%%%%%%%%%%%%%%%%%%%%%%%%%%%%%%%%%%%%%%%%%%%%%%%%%%%%%%%%%%%%%%%%%%
\subsection{Manual Code}
\label{sec:manual}

In case one cannot be certain whether the definitions file |childdoc.def|
is installed on the target \TeX{} distribution
and one prefers not to ship it,
it is conceivable to paste a few relevant commands into the sources.

To that end, drop all statements |\input{childdoc.def}|
and perform the replacements as outlined below.
Instead of |\childdocmain{|\textit{main}|}| add the following code
to the top of the main file:
%
\begin{center}
\begin{tabular}{l}
|\||ifdefined\childdocname\endinput\||fi\newif\ifchilddoc|\\
|\edef\childdocname{\scantokens\expandafter{\jobname\noexpand}}|\\
|\def\childdocmain{|\textit{main}|}\||ifx\childdocmain\childdocname\||else|\\
|\childdoctrue\includeonly{\childdocname}\let\jobname\childdocmain\||fi|\\
\end{tabular}
\end{center}
%
Instead of |\childdocof{|\textit{main}|}| just include the main file
at the top of each child file:
%
\begin{center}
|\input{|\textit{main}|}|
\end{center}
%
A simple redirection |\childdocforward{|\textit{dest}|}| is achieved by:
%
\begin{center}
|\def\jobname{|\textit{dest}|}\input{\jobname}|
\end{center}
%
The redirection with prefix
|\childdocforwardprefix[|\textit{prefix}|]{|\textit{dest}|}|
is accomplished by:
%
\begin{center}
\begin{tabular}{l}
|{\edef\jobname{\scantokens\expandafter{\jobname\noexpand}}|\\
|\def\redirectjob |\textit{prefix}|#1~~~{\gdef\jobname{|\textit{dest}|#1}}|\\
|\expandafter\redirectjob\jobname~~~}\input{\jobname}|
\end{tabular}
\end{center}

In an alternative approach,
child documents can be compiled by a specific command line
without additional code or specific definitions:
%
\begin{center}
|... -jobname "|\textit{target}|" "|[\textit{flags}]%
|\includeonly{|\textit{dest}|}\input{|\textit{main}|}"|
\end{center}
%

%%%%%%%%%%%%%%%%%%%%%%%%%%%%%%%%%%%%%%%%%%%%%%%%%%%%%%%%%%%%%%%%%%%%%%%%%%%%%%%%
%%%%%%%%%%%%%%%%%%%%%%%%%%%%%%%%%%%%%%%%%%%%%%%%%%%%%%%%%%%%%%%%%%%%%%%%%%%%%%%%
\section{Information}

%%%%%%%%%%%%%%%%%%%%%%%%%%%%%%%%%%%%%%%%%%%%%%%%%%%%%%%%%%%%%%%%%%%%%%%%%%%%%%%%
\subsection{Copyright}

Copyright \copyright{} 2017--2018 Niklas Beisert

This work may be distributed and/or modified under the
conditions of the \LaTeX{} Project Public License, either version 1.3
of this license or (at your option) any later version.
The latest version of this license is in
  \url{http://www.latex-project.org/lppl.txt}
and version 1.3 or later is part of all distributions of \LaTeX{}
version 2005/12/01 or later.

This work has the LPPL maintenance status `maintained'.

The Current Maintainer of this work is Niklas Beisert.

This work consists of the files |README.txt|, |childdoc.ins| and |childdoc.dtx|
as well as the derived files |childdoc.def|, |cdocsamp.tex|
with |cdocsch1.tex|, |cdocsch2.tex|, |cdocspt3.tex|, |cdocspt4.tex|,
|cdocsdrf.tex|, |cdocsfn1.tex|, |cdocsfn2.tex|
as well as |childdoc.pdf|.

%%%%%%%%%%%%%%%%%%%%%%%%%%%%%%%%%%%%%%%%%%%%%%%%%%%%%%%%%%%%%%%%%%%%%%%%%%%%%%%%
\subsection{Files and Installation}

The package consists of the files:
%
\begin{center}
\begin{tabular}{ll}
    |README.txt|   & readme file \\
    |childdoc.ins| & installation file \\
    |childdoc.dtx| & source file \\
    |childdoc.def| & definition file \\
    |cdocsamp.tex| & sample main file \\
    |cdocsch1.tex| & sample include file \\
    |cdocsch2.tex| & sample include file \\
    |cdocspt3.tex| & sample part file \\
    |cdocspt4.tex| & sample part file \\
    |cdocsdrf.tex| & sample redirection file \\
    |cdocsfn1.tex| & sample redirection file \\
    |cdocsfn2.tex| & sample redirection file \\
    |childdoc.pdf| & manual
\end{tabular}
\end{center}
%
The distribution consists of the files
|README.txt|, |childdoc.ins| and |childdoc.dtx|.
%
\begin{itemize}
\item
Run (pdf)\LaTeX{} on |childdoc.dtx|
to compile the manual |childdoc.pdf| (this file).
\item
Run \LaTeX{} on |childdoc.ins| to create the definitions file |childdoc.def|
and the sample |cdocsamp.tex| with include files
|cdocsch1.tex|, |cdocsch2.tex|, |cdocspt3.tex|, |cdocspt4.tex|,
|cdocsdrf.tex|, |cdocsfn1.tex|, |cdocsfn2.tex|.
Then copy the file |childdoc.def| to an appropriate directory of your \LaTeX{}
distribution, e.g.\ \textit{texmf-root}|/tex/latex/childdoc|.
\end{itemize}

%%%%%%%%%%%%%%%%%%%%%%%%%%%%%%%%%%%%%%%%%%%%%%%%%%%%%%%%%%%%%%%%%%%%%%%%%%%%%%%%
\subsection{Related CTAN Packages}

There are several other packages which offer a similar functionality:
%
\begin{itemize}
\item
The packages
\href{http://ctan.org/pkg/docmute}{\textsf{docmute}},
\href{http://ctan.org/pkg/includex}{\textsf{includex}} and
\href{http://ctan.org/pkg/standalone}{\textsf{standalone}}
provide commands to include only the document body of
a child file thus allowing both files to be compiled individually.
\item
The packages \href{http://ctan.org/pkg/subdocs}{\textsf{subdocs}}
and \href{http://ctan.org/pkg/subfiles}{\textsf{subfiles}}
provide structures in which the main and child documents can be
encapsulated and allowing them to be compiled individually.
The inclusion mechanism is different from the conventional |\include|.
\item
The package \href{http://ctan.org/pkg/combine}{\textsf{combine}}
is an elaborate solution to combine several documents into one.
\end{itemize}
%
See also the CTAN topic \href{http://ctan.org/topic/subdocs}{\textsf{subdocs}}
for further related packages.
The present package differs from the above solutions in that
a document structure constructed with the conventional |\include| mechanism
just needs two extra commands at the top of every file
such that all constituent files can be compiled individually.

%%%%%%%%%%%%%%%%%%%%%%%%%%%%%%%%%%%%%%%%%%%%%%%%%%%%%%%%%%%%%%%%%%%%%%%%%%%%%%%%
%\subsection{Feature Suggestions}
%
%The following is a list of features which may be useful for future
%versions of this package:
%%
%\begin{itemize}
%\item
%\ldots
%\end{itemize}

%%%%%%%%%%%%%%%%%%%%%%%%%%%%%%%%%%%%%%%%%%%%%%%%%%%%%%%%%%%%%%%%%%%%%%%%%%%%%%%%
\subsection{Revision History}

%%%%%%%%%%%%%%%%%%%%%%%%%%%%%%%%%%%%%%%%
\paragraph{v2.0:} 2018/12/30

\begin{itemize}
\item
immediate forward processing
\item
added |\childdocby| mechanism
\item
manual restructured
\end{itemize}

%%%%%%%%%%%%%%%%%%%%%%%%%%%%%%%%%%%%%%%%
\paragraph{v1.6:} 2018/01/17

\begin{itemize}
\item
application for development of include files
\item
corrections to manual
\end{itemize}

%%%%%%%%%%%%%%%%%%%%%%%%%%%%%%%%%%%%%%%%
\paragraph{v1.5:} 2017/05/21

\begin{itemize}
\item
more complete structuring introduced
\item
|\childdocof| introduced
\item
|\childdoc| renamed to |\childdocmain|
\item
|\childredirect| renamed to |\childdocforward| and |\childdocforwardprefix|
and functionality expanded
\end{itemize}

%%%%%%%%%%%%%%%%%%%%%%%%%%%%%%%%%%%%%%%%
\paragraph{v1.0:} 2017/04/27

\begin{itemize}
\item
manual and install package
\item
first version published on CTAN
\end{itemize}

%%%%%%%%%%%%%%%%%%%%%%%%%%%%%%%%%%%%%%%%
\paragraph{v0.6:} 2017/04/26

\begin{itemize}
\item
redirection mechanism added
\end{itemize}

%%%%%%%%%%%%%%%%%%%%%%%%%%%%%%%%%%%%%%%%
\paragraph{v0.5:} 2017/04/26

\begin{itemize}
\item
functionality in definition file
\end{itemize}


%%%%%%%%%%%%%%%%%%%%%%%%%%%%%%%%%%%%%%%%%%%%%%%%%%%%%%%%%%%%%%%%%%%%%%%%%%%%%%%%
%%%%%%%%%%%%%%%%%%%%%%%%%%%%%%%%%%%%%%%%%%%%%%%%%%%%%%%%%%%%%%%%%%%%%%%%%%%%%%%%
%%%%%%%%%%%%%%%%%%%%%%%%%%%%%%%%%%%%%%%%%%%%%%%%%%%%%%%%%%%%%%%%%%%%%%%%%%%%%%%%
\appendix

\settowidth\MacroIndent{\rmfamily\scriptsize 000\ }

 \DocInput{childdoc.dtx}

\end{document}
%</driver>
% \fi
%
% %%%%%%%%%%%%%%%%%%%%%%%%%%%%%%%%%%%%%%%%%%%%%%%%%%%%%%%%%%%%%%%%%%%%%%%%%%%%%%
% %%%%%%%%%%%%%%%%%%%%%%%%%%%%%%%%%%%%%%%%%%%%%%%%%%%%%%%%%%%%%%%%%%%%%%%%%%%%%%
% \section{Sample}
%\iffalse
%<*samplemain>
%\fi
%
% The following presents a sample document
% with two chapters, two parts, a title page,
% a compile flag as well as three forwarding files to set the flag.
% It consists of eight |.tex| files:
% \begin{center}
% \begin{tabular}{ll}
% |cdocsamp.tex|&main file\\
% |cdocsch1.tex|&include file for chapter 1\\
% |cdocsch2.tex|&include file for chapter 2\\
% |cdocspt3.tex|&include file for part 3\\
% |cdocspt4.tex|&include file for part 4\\
% |cdocsdrf.tex|&forwarding file for main file in draft mode\\
% |cdocsfi1.tex|&forwarding file for final version of chapter 1\\
% |cdocsfi2.tex|&forwarding file for final version of chapter 2\\
% \end{tabular}
% \end{center}
% Each of the eight files can be compiled directly by the \LaTeX{} compiler.
%
% %%%%%%%%%%%%%%%%%%%%%%%%%%%%%%%%%%%%%%
% \paragraph{Main File.}
%
% The main file is called |cdocsamp.tex|.
%
% Load the \textsf{childdoc} definitions and
% declare the filename for the main document:
%    \begin{macrocode}
\input{childdoc.def}
\childdocmain{}
%    \end{macrocode}

% Optional override for |\version| flag:
%    \begin{macrocode}
%%\ifchilddoc\else\providecommand{\version}{draft}\fi
%    \end{macrocode}

% Define the default values for the |\version| flag
% (|final| for the main file and |draft| for childs):
%    \begin{macrocode}
\ifchilddoc
\providecommand{\version}{draft}
\else
\providecommand{\version}{final}
\fi
%    \end{macrocode}

% Load the standard document class:
%    \begin{macrocode}
\documentclass[12pt]{article}
%    \end{macrocode}

% Start the document body:
%    \begin{macrocode}
\begin{document}
%    \end{macrocode}

% Declare a title page.
% Print title, part of document being processed and version flag:
%    \begin{macrocode}
\addtocounter{page}{-1}
\begin{center}
{\LARGE\bfseries{}childdoc example\par}
\vspace{1cm}
\ifchilddoc
\ifchilddocmanual part\else chapter\fi:
`\childdocname' of `\childdocjob'\par
\else
main document: `\childdocjob'\par
\fi
version: \version\par
\end{center}
\newpage
%    \end{macrocode}

% Manually include selected file,
% otherwise process as usual:
%    \begin{macrocode}
\ifchilddocmanual
\section*{part `\childdocname'}
\input{\childdocname}
\else
%    \end{macrocode}

% Include the two chapters:
%    \begin{macrocode}
\include{cdocsch1}
\include{cdocsch2}
%    \end{macrocode}

% Include the two parts unless only chapters should be displayed:
%    \begin{macrocode}
\ifchilddoc\else
\section{part three}
\input{cdocspt3}
\section{part four}
\input{cdocspt4}
\fi
%    \end{macrocode}

% Process as usual until here:
%    \begin{macrocode}
\fi
%    \end{macrocode}

% End of document body:
%    \begin{macrocode}
\end{document}
%    \end{macrocode}
%\iffalse
%</samplemain>
%\fi
%
% %%%%%%%%%%%%%%%%%%%%%%%%%%%%%%%%%%%%%%
% \paragraph{Chapter Include Files.}
%
% The include files are called |cdocsch1.tex| and |cdocsch2.tex|.
%
%\iffalse
%<*samplechap1|samplechap2>
%\fi

% Optional override for |\version| flag:
%    \begin{macrocode}
%%\providecommand{\version}{final}
%    \end{macrocode}

% Include the main document:
%    \begin{macrocode}
\input{childdoc.def}
\childdocof{cdocsamp}
%    \end{macrocode}

%\iffalse
%</samplechap1|samplechap2>
%\fi
%
%\iffalse
%<*samplechap1>
%\fi
% Some text for chapter 1:
%    \begin{macrocode}
\section{one}
some text in chapter one
%    \end{macrocode}

%\iffalse
%</samplechap1>
%\fi
% Some text for chapter 2:
%\iffalse
%<*samplechap2>
%\fi
%    \begin{macrocode}
\section{two}
more text in chapter two
%    \end{macrocode}

%\iffalse
%</samplechap2>
%\fi
%
% %%%%%%%%%%%%%%%%%%%%%%%%%%%%%%%%%%%%%%
% \paragraph{Part Include Files.}
%
% The include files are called |cdocspt3.tex| and |cdocspt4.tex|.
%
%\iffalse
%<*samplepart3|samplepart4>
%\fi

% Optional override for |\version| flag:
%    \begin{macrocode}
%%\providecommand{\version}{final}
%    \end{macrocode}

% Include the main document:
%    \begin{macrocode}
\input{childdoc.def}
\childdocby{cdocsamp}
%    \end{macrocode}

%\iffalse
%</samplepart3|samplepart4>
%\fi
%
%\iffalse
%<*samplepart3>
%\fi
% Some text for part 3:
%    \begin{macrocode}
some text in part three
%    \end{macrocode}

%\iffalse
%</samplepart3>
%\fi
% Some text for part 4:
%\iffalse
%<*samplepart4>
%\fi
%    \begin{macrocode}
more text in part four
%    \end{macrocode}

%\iffalse
%</samplepart4>
%\fi
%
% %%%%%%%%%%%%%%%%%%%%%%%%%%%%%%%%%%%%%%
% \paragraph{Forwarding for a Complete Draft.}
%
% The following forwarding file |cdocsdrf.tex|
% compiles the main document in draft mode:
%\iffalse
%<*sampledraft>
%\fi
%    \begin{macrocode}
\def\version{draft}
\input{childdoc.def}
\childdocforward{cdocsamp}
%    \end{macrocode}

%\iffalse
%</sampledraft>
%\fi
%
% %%%%%%%%%%%%%%%%%%%%%%%%%%%%%%%%%%%%%%
% \paragraph{Forwarding for Final Version of the Chapters.}
%
% The following forwarding files |cdocsfn1.tex| and |cdocsfn2.tex|
% (with identical content)
% compile the final versions of the child documents
% |cdocsch1.tex| and |cdocsch2.tex|, respectively:
%\iffalse
%<*samplefinal>
%\fi
%    \begin{macrocode}
\def\version{final}
\input{childdoc.def}
\childdocforwardprefix[cdocsamp]{cdocsfn}{cdocsch}
%    \end{macrocode}

%\iffalse
%</samplefinal>
%\fi
%
% %%%%%%%%%%%%%%%%%%%%%%%%%%%%%%%%%%%%%%
% \paragraph{Command Line Processing.}
%
% The following three command lines generate the output files
% |cdocscld|, |cdocscl1| and |cdocscl2|
% which should be identical to
% |cdocsdrf|, |cdocsch1| and |cdocsfn2|, respectively:
% \begin{center}
% \begin{tabular}{l}
% |latex -jobname cdocscld \|\\
% |  "\def\version{draft}\input{childdoc.def}\childdocforward{cdocsamp}"|\\
% |latex -jobname cdocscl1 \|\\
% |  "\input{childdoc.def}\childdocforward[cdocsamp]{cdocsch1}"|\\
% |latex -jobname cdocscl2 \|\\
% |  "\def\version{final}\input{childdoc.def}\childdocforward{cdocsch2}"|
% \end{tabular}
% \end{center}
% Note that the trailing backslash on each first line
% merely continues the input to the second line
% (for convenient cut ant paste).
% Furthermore, the command |latex| can be replaced by any
% of its alternative versions such as |pdflatex|.
%
% %%%%%%%%%%%%%%%%%%%%%%%%%%%%%%%%%%%%%%%%%%%%%%%%%%%%%%%%%%%%%%%%%%%%%%%%%%%%%%
% %%%%%%%%%%%%%%%%%%%%%%%%%%%%%%%%%%%%%%%%%%%%%%%%%%%%%%%%%%%%%%%%%%%%%%%%%%%%%%
% \section{Implementation}
%\iffalse
%<*package>
%\fi
%
% This section describes the definitions file |childdoc.def|.

% The definitions cannot be loaded using |\usepackage| or |\RequirePackage|
% which has a mechanism to prevent loading a style file more than once.
% When loading the definitions by means of |\input|
% multiple instances have to be prevented manually:
%\iffalse
%This code needs to be before the `\ProvidesFile' directive
%which is defined at the beginning of this file.
%Therefore it is also placed there and commented out here.
%</package>
%<*discard>
%\fi
%    \begin{macrocode}
\ifdefined\childdocmain\endinput\fi
%    \end{macrocode}
%\iffalse
%</discard>
%<*package>
%\fi
%
% \macro{\ifchilddoc}
% \macro{\ifchilddocmanual}
% The conditional |\ifchilddoc| tells whether a
% child (true) or main (false) document is being compiled.
% The conditional |\ifchilddocmanual| tells whether
% the |\includeonly| mechanism is used (false) or
% the selection of child files must be performed manually (true).
% The definitions initialise to false:
%    \begin{macrocode}
\newif\ifchilddoc
\newif\ifchilddocmanual
%    \end{macrocode}

% \macro{\childdocname}
% \macro{\childdocjob}
% The macro |\childdocname| stores the name of the main document
% to be compiled. The macro |\childdocjob| stores the name of
% the document on which the \LaTeX{} compiler was originally invoked.
% The content of |\jobname| cannot be compared
% to filenames specified in the source due to different catcodes.
% The following code rescans |\jobname|, stores the result
% in |\childdocname| and saves a copy in |\childdocjob|:
%    \begin{macrocode}
\edef\childdocname{\scantokens\expandafter{\jobname\noexpand}}
\let\childdocjob\childdocname
%    \end{macrocode}

% \macro{\childdocdisable}
% The macro |\childdocdisable| prevents the main file
% from being processed more than once.
% At this stage, the main document command |\childdocmain|
% is assumed to be called once again where it should do nothing.
% Any subsequent call to it should prevent
% a secondary processing of the main document
% It overwrites the forwarding commands
% |\childdocof| and |\childdocforward|
% with empty macros to prevent further inclusions of the main document:
%    \begin{macrocode}
\newcommand{\childdocdisable}
{
  \renewcommand{\childdocmain}[1]{\renewcommand{\childdocmain}[1]{\endinput}}
  \renewcommand{\childdocof}[1]{}
  \renewcommand{\childdocby}[2][]{}
  \renewcommand{\childdocforward}[2][]{}
  \renewcommand{\childdocdisable}{}
}
%    \end{macrocode}

% \macro{\childdocmain}
% The macro |\childdocmain| is to be called at the top of the main file
% with nothing or the main filename (without extension) as argument.
% First, it breaks loops.
% If the argument is not empty and does not match |\childdocname|
% (which is set by the first inclusion of |childdoc.def|),
% |\ifchilddoc| is set to true, |\includeonly| is applied to the child file
% and |\jobname| is set to the main file
% (for proper handling of |.aux| files):
%    \begin{macrocode}
\newcommand{\childdocmain}[1]
{
  \childdocdisable\childdocmain{}
  \if?#1?\else
    \begingroup
      \def\childdoctmp{#1}
      \ifx\childdoctmp\childdocname
        \def\childdoctmp{}
      \else
        \def\childdoctmp
        {
          \childdoctrue
          \includeonly{\childdocname}
          \def\childdocjob{#1}
          \def\jobname{#1}
        }
      \fi
      \expandafter
    \endgroup
    \childdoctmp
  \fi
}
%    \end{macrocode}

% \macro{\childdocof}
% The command |\childdocof| redirects
% compilation to the main file |#1|.
%    \begin{macrocode}
\newcommand{\childdocof}[1]
{
  \childdocdisable
  \childdoctrue
  \includeonly{\childdocname}
  \def\jobname{#1}
  \def\childdocjob{#1}
  \input{#1}
}
%    \end{macrocode}

% \macro{\childdocby}
% The command |\childdocby| ....
%    \begin{macrocode}
\newcommand{\childdocby}[2][]
{
  \childdocdisable
  \childdoctrue
  \childdocmanualtrue
  \if?#1?\else
    \def\jobname{#2}
  \fi
  \def\childdocjob{#2}
  \input{#2}
  \endinput
}
%    \end{macrocode}

% \macro{\childdocforward}
% The command |\childdocforward| redirects
% compilation to the main file or
% (if the optional argument is given) a child file.
% Parameters are set as if the main file
% or a child file starting with |\childdocof| was compiled.
% Then compilation is handed over to the main file:
%    \begin{macrocode}
\newcommand{\childdocforward}[2][]
{
  \begingroup
    \if?#1?
      \def\childdoctmp
      {
        \def\childdocname{#2}
        \def\childdocjob{#2}
        \def\jobname{#2}
        \input{#2}
        \endinput
      }
    \else
      \def\childdoctmp
      {
        \childdocdisable
        \def\childdocname{#2}
        \childdoctrue
        \includeonly{#2}
        \def\childdocjob{#1}
        \def\jobname{#1}
        \input{#1}
        \endinput
      }
    \fi
    \expandafter
  \endgroup
  \childdoctmp
}
%    \end{macrocode}

% \macro{\childdocforwardprefix}
% The command |\childdocforwardprefix| redirects
% compilation to the main or a child file by means of a pattern.
% The prefix |#1| in the current filename is replaced by |#2|
% and the suffix of the current filename is kept
% (it is assumed that the filename does not contain the substring `|~~~|'
% which is used as a delimiter).
% Compilation is handed over to the new file by |\childdocforward|:
%    \begin{macrocode}
\newcommand{\childdocforwardprefix}[3][]
{
  \begingroup
    \def\childdocextract #2##1~~~{\def\childdoctmp{\childdocforward[#1]{#3##1}}}
    \expandafter\childdocextract\childdocname~~~
    \expandafter
  \endgroup
  \childdoctmp
}
%    \end{macrocode}

% \macro{\childdoc}
% The deprecated macro |\childdoc| is a legacy version of |\childdocmain|:
%    \begin{macrocode}
\newcommand{\childdoc}{\childdocmain}
%    \end{macrocode}

% \macro{\childdocredirect}
% The deprecated macro |\childdocredirect| is a legacy version
% of |\childdocforward| and |\childdocforwardprefix|:
%    \begin{macrocode}
\newcommand{\childdocredirect}[2][]
{
  \begingroup
    \if?#1?
      \def\childdoctmp{\childdocforward{#2}}
    \else
      \def\childdoctmp{\childdocforwardprefix{#1}{#2}}
    \fi
    \expandafter
  \endgroup
  \childdoctmp
}
%    \end{macrocode}

%\iffalse
%</package>
%\fi
%
\endinput
|\\
|\childdocmain{|\textit{main}|}|\\
\end{tabular}
\end{center}
%
If |\jobname| does not match the argument \textit{main} of |\childdocmain|,
it is assumed that |\jobname| points to the child file to be compiled.
When using |\childdocmain| with the main file specified as argument,
it suffices to start a child file
with just |\input{|\textit{main}|}|
without loading of the package and using |\childdocof|.
If instead all processing is done
with the appropriate \textsf{childdoc} directives,
the argument of \textit{main} of |\childdocmain| can be empty.

An alternative version of the command line processing described
in \secref{sec:commandline} using the detection mechanism reads:
%
\begin{center}
|... -jobname "|\textit{target}|" "|[\textit{flags}]%
[|\def\jobname{|\textit{dest}|}|]|\input{|\textit{main}|}"|
\end{center}

%%%%%%%%%%%%%%%%%%%%%%%%%%%%%%%%%%%%%%%%%%%%%%%%%%%%%%%%%%%%%%%%%%%%%%%%%%%%%%%%
\subsection{Manual Code}
\label{sec:manual}

In case one cannot be certain whether the definitions file |childdoc.def|
is installed on the target \TeX{} distribution
and one prefers not to ship it,
it is conceivable to paste a few relevant commands into the sources.

To that end, drop all statements |% \iffalse
%
% childdoc.dtx Copyright (C) 2017-2018 Niklas Beisert
%
% This work may be distributed and/or modified under the
% conditions of the LaTeX Project Public License, either version 1.3
% of this license or (at your option) any later version.
% The latest version of this license is in
%   http://www.latex-project.org/lppl.txt
% and version 1.3 or later is part of all distributions of LaTeX
% version 2005/12/01 or later.
%
% This work has the LPPL maintenance status `maintained'.
%
% The Current Maintainer of this work is Niklas Beisert.
%
% This work consists of the files childdoc.dtx and childdoc.ins
% and the derived files childdoc.def and cdocsamp.tex with
% cdocsch1.tex, cdocsch2.tex, cdocsdrf.tex, cdocsfn1.tex, cdocsfn2.tex.
%
%<package>\ifdefined\childdocmain\endinput\fi
%<package>\ProvidesFile{childdoc.def}[2018/12/30 v2.0 child document driver]
%<samplemain>\ProvidesFile{cdocsamp.tex}[2018/12/30 v2.0 sample for childdoc]
%<*driver>
%\ProvidesFile{childdoc.drv}[2018/12/30 v2.0 childdoc reference manual file]
\PassOptionsToClass{10pt,a4paper}{article}
\documentclass{ltxdoc}

\usepackage[margin=35mm]{geometry}
\usepackage{hyperref}
\usepackage{hyperxmp}
\usepackage[usenames]{color}

\hypersetup{colorlinks=true}
\hypersetup{pdfstartview=FitH}
\hypersetup{pdfpagemode=UseNone}
\hypersetup{pdfsource={}}
\hypersetup{pdflang={en-UK}}
\hypersetup{pdfcopyright={Copyright 2017-2018 Niklas Beisert.
  This work may be distributed and/or modified under the
  conditions of the LaTeX Project Public License, either version 1.3
  of this license or (at your option) any later version.}}
\hypersetup{pdflicenseurl={http://www.latex-project.org/lppl.txt}}
\hypersetup{pdfcontactaddress={ETH Zurich, ITP, HIT K,
  Wolfgang-Pauli-Strasse 27}}
\hypersetup{pdfcontactpostcode={8093}}
\hypersetup{pdfcontactcity={Zurich}}
\hypersetup{pdfcontactcountry={Switzerland}}
\hypersetup{pdfcontactemail={nbeisert@itp.phys.ethz.ch}}
\hypersetup{pdfcontacturl={http://people.phys.ethz.ch/\xmptilde nbeisert/}}

\newcommand{\secref}[1]{\hyperref[#1]{section \ref*{#1}}}

\parskip1ex
\parindent0pt
\let\olditemize\itemize
\def\itemize{\olditemize\parskip0pt}

\begin{document}

\title{The \textsf{childdoc} Package}
\hypersetup{pdftitle={The childdoc Package}}
\author{Niklas Beisert\\[2ex]
  Institut f\"ur Theoretische Physik\\
  Eidgen\"ossische Technische Hochschule Z\"urich\\
  Wolfgang-Pauli-Strasse 27, 8093 Z\"urich, Switzerland\\[1ex]
  \href{mailto:nbeisert@itp.phys.ethz.ch}
  {\texttt{nbeisert@itp.phys.ethz.ch}}}
\hypersetup{pdfauthor={Niklas Beisert}}
\hypersetup{pdfsubject={Manual for the LaTeX2e Package childdoc}}
\date{30 December 2018, \textsf{v2.0}}
\maketitle

\begin{abstract}\noindent
\textsf{childdoc} is a \LaTeXe{} package
that enables the direct compilation
of document sections included by |\include|
to individual files.
\end{abstract}

\begingroup
\parskip0ex
\tableofcontents
\endgroup

%%%%%%%%%%%%%%%%%%%%%%%%%%%%%%%%%%%%%%%%%%%%%%%%%%%%%%%%%%%%%%%%%%%%%%%%%%%%%%%%
%%%%%%%%%%%%%%%%%%%%%%%%%%%%%%%%%%%%%%%%%%%%%%%%%%%%%%%%%%%%%%%%%%%%%%%%%%%%%%%%
\section{Introduction}

\LaTeX{} provides a mechanism to structure a large document (such as a book)
into a main file and several child files (containing the chapters)
using the |\include| command.
This mechanism is beneficial for documents
which span hundreds of pages in order to
make the source file(s) more manageable.
Moreover, compilation can be restricted to
selected child files by means of the |\includeonly| command.
The latter feature can be used to reduce the compilation time while editing
(this was significantly more useful in the earlier days of \LaTeX{})
or to generate a smaller document which is easier to navigate.
Another application of |\includeonly| is to generate
documents consisting of selected parts of the complete document.

However, there are a few drawbacks of the plain |\include| mechanism:
\begin{itemize}
\item
The child files cannot be compiled on their own,
they can only be compiled via the main file.
A naive editing environment
(such as a text editor with an option
to have the current file processed by \LaTeX)
may require one to switch to the main file before compiling;
attempting to compile the child file produces errors.
\item
The main file must be modified (each time)
to adjust the |\includeonly| command
to the present needs. This easily leaves the main file in a messy state.
\item
The generated document will always carry the filename
of the main document. This is inconvenient if
several child files are to be compiled and
to be kept for distribution.
\end{itemize}

The present package provides a simple interface
to make child files individually compilable by \LaTeX{}.
Compiling a child file then has the same effect as compiling
the main file with an |\includeonly| command
to select the appropriate child.
Moreover the generated document will carry the name of the child
rather than the main file.
This resolves all three above issues.

This feature is meant to make the editing of books,
thesis documents and lecture notes somewhat more convenient.
However, the package can also be used efficiently for
composing a series of documents (such as exercise sheets)
which are typically distributed individually.
It then assists the author in generating the individual documents
(potentially in different versions)
as well as a document containing the collected series.
Another application is in developing style files
or other kinds of included material
where compilation of the style file could redirect
to a sample or test file.

%%%%%%%%%%%%%%%%%%%%%%%%%%%%%%%%%%%%%%%%%%%%%%%%%%%%%%%%%%%%%%%%%%%%%%%%%%%%%%%%
%%%%%%%%%%%%%%%%%%%%%%%%%%%%%%%%%%%%%%%%%%%%%%%%%%%%%%%%%%%%%%%%%%%%%%%%%%%%%%%%
\section{Usage}

First of all, the package \textsf{childdoc} is \emph{not} a standard
\LaTeXe{} |.sty| style file! Therefore it needs to be invoked in
a non-standard way.

%%%%%%%%%%%%%%%%%%%%%%%%%%%%%%%%%%%%%%%%%%%%%%%%%%%%%%%%%%%%%%%%%%%%%%%%%%%%%%%%
\subsection{Included Files}
\label{sec:include}

%%%%%%%%%%%%%%%%%%%%%%%%%%%%%%%%%%%%%%%%
\DescribeMacro{\childdocmain}
To use the package, add the commands
\begin{center}
\begin{tabular}{l}
|\input{childdoc.def}|\\
|\childdocmain{}|\\
\end{tabular}
\end{center}
at the very top of the main \LaTeX{} file,
in particular \emph{before} the |\documentclass| statement!
The argument of |\childdocmain| should be left empty
(but it must be present).

%%%%%%%%%%%%%%%%%%%%%%%%%%%%%%%%%%%%%%%%
\DescribeMacro{\childdocof}
Furthermore, add the commands
\begin{center}
\begin{tabular}{l}
|\input{childdoc.def}|\\
|\childdocof{|\textit{main}|}|\\
\end{tabular}
\end{center}
at the top of every child file \textit{child}
which is included by |\include{|\textit{child}|}|
from within the main file
(or at least for those files to be compiled individually).
The argument \textit{main} must be the filename of the main file.

There are a couple of
considerations in setting up the main and child documents:

%%%%%%%%%%%%%%%%%%%%%%%%%%%%%%%%%%%%%%%%
\paragraph{Restrictions.}

Please note the following restrictions:
\begin{itemize}
\item
|\childdocmain| must be called with one argument \textit{main}
to ensure compatibility with earlier version of the package.
It must either be empty (|\childdocmain{}|)
or precisely match the filename of the main file in which it is specified.
See \secref{sec:detection} for further information.
\item
The filename \textit{main} must be specified without the |.tex| extension.
\item
The filename \textit{main} is case sensitive
(even in case-insensitive file systems)
due to internal string comparison.
\item
The argument \textit{main} should be fully expanded, it cannot be a macro.
\item
Subdirectories and special characters should be avoided in filenames.
\item
The command |\childdocmain{|\textit{main}|}| must be followed by a whitespace.
It should not be followed immediately by another command
or by a comment mark `|%|'.
This is because the \TeX{} parser reads the token immediately following
the argument of |\childdocmain| and puts it
at the beginning of every child section;
however, a white\-space is ignored.
\end{itemize}

%%%%%%%%%%%%%%%%%%%%%%%%%%%%%%%%%%%%%%%%
\paragraph{Content of Main File.}

It is advisable to place all content in the child files included by |\include|.
Any output contained in the main file will appear in all child documents
unless suppressed manually;
it cannot be suppressed automatically by the |\includeonly| directive
and thus should normally be avoided.
A method to include some content in the main file
by means of conditional processing is described in \secref{sec:conditional}.

%%%%%%%%%%%%%%%%%%%%%%%%%%%%%%%%%%%%%%%%
\paragraph{Page Numbering.}

When only a part of the document is compiled,
the appropriate numbering of pages
(as well as other status parameters)
is determined from the |.aux| files.
The latter contain information from previous passes.
However this information needs to propagate through
all intermediate child documents.
Therefore the page numbering in child documents may well
be inconsistent until the complete document is compiled at least once.

A useful (if unconventional) way to always ensure a consistent
page numbering is to restart the numbering in each child document
and denote the pages by `\textit{child}|.|\textit{page}'
where \textit{child} represents the chapter/section number of the child file.
This can be achieved by the command
|\numberwithin{page}{|\textit{child}|}|
of the \textsf{amsmath} package
where \textit{child} can be |chapter| or |section|
depending on the chosen structuring.
Alternatively, one can modify the macro |\thepage| appropriately
and reset the counter |page| at the start of each child file.

%%%%%%%%%%%%%%%%%%%%%%%%%%%%%%%%%%%%%%%%%%%%%%%%%%%%%%%%%%%%%%%%%%%%%%%%%%%%%%%%
\subsection{Conditional Processing}
\label{sec:conditional}

The package provides a mechanism to compile different versions
of a document. To customise the versions further some conditional processing
can come in handy to distinguish which version is being compiled.
The package provides two macros to describe the compilation context:

%%%%%%%%%%%%%%%%%%%%%%%%%%%%%%%%%%%%%%%%
\DescribeMacro{\ifchilddoc}
The conditional |\ifchilddoc| distinguishes between the compilation of
child documents and the main document:
%
\begin{center}
|\ifchilddoc |\textit{child-code}| |[|\||else |\textit{main-code}]| \||fi|
\end{center}

%%%%%%%%%%%%%%%%%%%%%%%%%%%%%%%%%%%%%%%%
\DescribeMacro{\childdocname}
\DescribeMacro{\childdocjob}
The macro |\childdocname| contains the filename (without extension)
of the main or child file being processed.
Note that |\childdocjob| will always contain the name of the main file.

%%%%%%%%%%%%%%%%%%%%%%%%%%%%%%%%%%%%%%%%
\paragraph{Title Page.}

Conditional processing can be used to include a title or banner page
in the main document when proper precautions are taken.
Importantly, the code in the main file should ensure that the page counter
(as well as other status parameters which are stored in the |.aux| files)
takes the same value after the conditional processing.
Otherwise the page numbers may take divergent values
depending on which part is compiled.

For example, a title page could be declared by:
%
\begin{center}
\begin{tabular}{l}
|\ifchilddoc\||else|\\
|\addtocounter{page}{-1}|\\
\textit{code for title page}\\
|\newpage|\\
|\||fi|
\end{tabular}
\end{center}
%
A banner page for the child documents can be generated by:
%
\begin{center}
\begin{tabular}{l}
|\ifchilddoc|\\
|\addtocounter{page}{-1}|\\
\textit{code for banner page}\\
|\newpage|\\
|\||fi|
\end{tabular}
\end{center}
%
Here one could write a message such as:
\begin{center}
|This is the part \childdocname{} of \childdocjob{}.|
\end{center}

%%%%%%%%%%%%%%%%%%%%%%%%%%%%%%%%%%%%%%%%%%%%%%%%%%%%%%%%%%%%%%%%%%%%%%%%%%%%%%%%
\subsection{Flags}
\label{sec:flags}

The package makes it easy to generate different versions
of the main or child documents.
To this end compilation flags can be defined
and assigned different default values.
They will be particularly useful in conjunction
with the forwarding mechanism described in \secref{sec:forward}.

For example, it may be useful to have a flag |\version|
which can be set to |draft| or |final|.
The document source will contain some conditional code
depending on the value of |\version|.
Suppose further, the flag should default to |final| for the main file
and to |draft| for child files
which is a natural assignment for editing the document.
This is achieved by placing the following code
in the preamble of the main document
(below the |\childdocmain| directive):
%
\begin{center}
\begin{tabular}{l}
|\ifchilddoc|\\
|\providecommand{\version}{draft}|\\
|\||else|\\
|\providecommand{\version}{final}|\\
|\||fi|
\end{tabular}
\end{center}
%
The definition by |\providecommand| makes sure
that previous definitions are not overwritten.
Further statements |\providecommand{\version}{...}|
can thus be added before the above code to override it.

For the main file, one might add a line
(between |\childdocmain| and the above block)
%
\begin{center}
|%\ifchilddoc\||else\providecommand{\version}{draft}\||fi|
\end{center}
%
which can be uncommented to produce a draft version.
Likewise one can add a line to the very top of a child file
(above the |\childdocof{|\textit{main}|}| directive)
%
\begin{center}
|%\providecommand{\version}{final}|
\end{center}
%
which can be uncommented to produce the final version of this child document.

%%%%%%%%%%%%%%%%%%%%%%%%%%%%%%%%%%%%%%%%%%%%%%%%%%%%%%%%%%%%%%%%%%%%%%%%%%%%%%%%
\subsection{Forwarding}
\label{sec:forward}

Different versions of the main or child documents
using compilation flags as described in \secref{sec:flags}
can be (permanently) stored in different files
for convenient compilation, viewing and distribution.
To this end, the package defines a command
to pass on compilation to a different file:

%%%%%%%%%%%%%%%%%%%%%%%%%%%%%%%%%%%%%%%%
\DescribeMacro{\childdocforward}
The command |\childdocforward| redirects processing to
another source file:
%
\begin{center}
\begin{tabular}{l}
|\input{childdoc.def}|\\
|\childdocforward[|\textit{main}|]{|\textit{dest}|}|\\
\end{tabular}
\end{center}
%
The argument \textit{dest} is the destination file
(without extension).
It should be the main file or one of the child files.
Note that further \textsf{childdoc} directives
such as |\childdocof| and |\childdocforward|
in the indicated file will be processed in this form.
The optional argument \textit{main}
passes on directly to the main file \textit{main}
while pretending to compile the child \textit{dest}.
This form behaves as if \textit{dest}
issues |\childdocof{|\textit{main}|}| right away,
and no further \textsf{childdoc} directives will be processed.

%%%%%%%%%%%%%%%%%%%%%%%%%%%%%%%%%%%%%%%%
\DescribeMacro{\...prefix}
In the alternative form |\childdocforwardprefix|,
%
\begin{center}
\begin{tabular}{l}
|\input{childdoc.def}|\\
|\childdocforwardprefix[|\textit{main}|]{|\textit{prefix}|}{|\textit{dest}|}|
\end{tabular}
\end{center}
%
the destination file is determined by a pattern
depending on the current file:
To make this work, the current file must be called
`{\textit{prefix}\hspace{0.2em}\textit{suffix}}'
with \textit{prefix} matching precisely the argument.
Processing is then passed on to the file
`{\textit{dest}\hspace{0.2em}\textit{suffix}}'.
Surely, the same effect is achieved by
directly specifying the
argument `{\textit{dest}\hspace{0.2em}\textit{suffix}}'
in the first form.
However, that requires to set up a different file
for each child. With the alternative form of the command
all these files can have exactly the same content
which simplifies setting them up and maintaining them.

For example, the following file |draft.tex|
with a compilation flag |\version| as described in \secref{sec:flags}
compiles the main document as a draft:
%
\begin{center}
\begin{tabular}{l}
|\def\version{draft}|\\
|\input{childdoc.def}|\\
|\childdocforward{|\textit{main}|}|
\end{tabular}
\end{center}
%
Likewise, the following files |final|\textit{nn}|.tex|
compile the final version of the child document
|child|\textit{nn}|.tex|:
%
\begin{center}
\begin{tabular}{l}
|\def\version{final}|\\
|\input{childdoc.def}|\\
|\childdocforwardprefix{final}{child}|
\end{tabular}
\end{center}
%

Note that when several versions of a main file and/or of each child file
are to be generated, it may be convenient to set up a |Makefile| or
shell script to automatise the process.

%%%%%%%%%%%%%%%%%%%%%%%%%%%%%%%%%%%%%%%%%%%%%%%%%%%%%%%%%%%%%%%%%%%%%%%%%%%%%%%%
\subsection{Command Line Processing}
\label{sec:commandline}

The effect of redirection files can also be achieved by invoking
the \LaTeX{} compiler with a more elaborate command line.
Most conveniently this should be done as part
of a shell script or a |Makefile|.

When using \textsf{childdoc} in the main file, the following
command lines effectively perform a redirection
(note that depending on the shell being used,
backslashes may have to be doubled: `|\|' $\to$ `|\\|'):
%
\begin{center}
|... -jobname "|\textit{target}|" |\\|"|[\textit{flags}]%
|\input{childdoc.def}\childdocforward[|\textit{main}|]{|\textit{dest}|}"|
\end{center}
%
Here \textit{target} is the name of the output file,
\textit{main} is the name of the main file
and \textit{dest} is the name of the main or child file to be processed
(all filenames without extensions).
The optional argument \textit{main} can be omitted
if \textit{main} matches \textit{dest}.
Optionally, compilation \textit{flags} can be defined via |\def| commands.
This command line makes the \TeX{} engine believe
it is compiling the file \textit{target}
whose content is specified as the latter parameter.
The provided code then forwards the processing to
\textit{main} or \textit{dest} as described in \secref{sec:forward}.

%%%%%%%%%%%%%%%%%%%%%%%%%%%%%%%%%%%%%%%%%%%%%%%%%%%%%%%%%%%%%%%%%%%%%%%%%%%%%%%%
\subsection{Include by Input}
\label{sec:input}

Including child documents by |\include| has some restrictions by design.
Most notably, the content of a child document always occupies
its own set of pages; pages cannot be shared between child documents.
Usually, this behaviour makes perfect sense
because each child document contain an essential part of the document.
However, in some situations it may be desirable to compose
a document from a collection of parts
without having mandatory page breaks between then.
For this case, the package
provides a mechanism to include parts
by |\input| which can also be processed individually.
However, by construction this mechanism
requires manual handling of the content to be output.

%%%%%%%%%%%%%%%%%%%%%%%%%%%%%%%%%%%%%%%%
\DescribeMacro{\ifchilddocmanual}
The main file should be prepared as usual, see \secref{sec:include}.
However, the document body must make a distinction
between processing of an individual part and of the main document, e.g.:
%
\begin{center}
\begin{tabular}{l}
|\ifchilddocmanual|\\
|\input{\childdocname}|\\
|\||else|\\
\textit{document body with }|\input{|\textit{part}|}|\\
|\||fi|
\end{tabular}
\end{center}
%
The conditional |\ifchilddocmanual| is true whenever
a part to be included by |\input| is being compiled,
and the name of the part is stored in |\childdocname|.

%%%%%%%%%%%%%%%%%%%%%%%%%%%%%%%%%%%%%%%%
\DescribeMacro{\childdocby}
Each part to be included by |\input| should start with:
%
\begin{center}
\begin{tabular}{l}
|\input{childdoc.def}|\\
|\childdocby{|\textit{main}|}|\\
\end{tabular}
\end{center}
%
The directive |\childdocby| is similar to |\childdocof|
described in \secref{sec:include},
but the subsequent selection of content must be done manually.
To that end, both |\ifchilddoc| and |\ifchilddocmanual|
will be true upon processing of a part,
and the name of the part is stored in |\childdocname|.
Note that |\jobname| will be set to the filename of the current part
so that each part receives an individual |.aux| file
that does not interfere with the |.aux| file(s) of the main document.
This behaviour can be altered by the alternative form
|\childdocby[*]{|\textit{main}|}| (with a non-empty optional argument)
which uses the |.aux| file of the main document
by setting |\jobname| to \textit{main}.

%%%%%%%%%%%%%%%%%%%%%%%%%%%%%%%%%%%%%%%%%%%%%%%%%%%%%%%%%%%%%%%%%%%%%%%%%%%%%%%%
\subsection{Driver Development}
\label{sec:driver}

The \textsf{childdoc} mechanism can also be use for the development
of definition files such as \LaTeX{} styles or classes.
This case differs from the above setup with multiple parts
included by |\include| in that no |\includeonly| should be invoked.
This can be achieved by starting the include file
(before |\ProvidesPackage|) with:
%
\begin{center}
\begin{tabular}{l}
|\input{childdoc.def}|\\
|\childdocforward{|\textit{main}|}|\\
\end{tabular}
\end{center}
%
or alternatively with:
%
\begin{center}
\begin{tabular}{l}
|\input{childdoc.def}|\\
|\childdocby{|\textit{main}|}|\\
\end{tabular}
\end{center}
%
Both forms have slightly different effects as described above.
The main file is prepared as usual, see \secref{sec:include}.

%%%%%%%%%%%%%%%%%%%%%%%%%%%%%%%%%%%%%%%%%%%%%%%%%%%%%%%%%%%%%%%%%%%%%%%%%%%%%%%%
\subsection{Legacy Detection}
\label{sec:detection}

The directive |\childdocmain| in the main file can detect
whether the complete document or merely a child is to be compiled
even without using the directive |\childdocof|.
This method is deprecated because it is less robust
and there is no compelling reason to use it;
it is merely provided for backward compatibility
and it may be removed in future versions.

If the detection mechanism is to be used,
it is mandatory to correctly specify
the filename of the main file as the argument of |\childdocmain|:
%
\begin{center}
\begin{tabular}{l}
|\input{childdoc.def}|\\
|\childdocmain{|\textit{main}|}|\\
\end{tabular}
\end{center}
%
If |\jobname| does not match the argument \textit{main} of |\childdocmain|,
it is assumed that |\jobname| points to the child file to be compiled.
When using |\childdocmain| with the main file specified as argument,
it suffices to start a child file
with just |\input{|\textit{main}|}|
without loading of the package and using |\childdocof|.
If instead all processing is done
with the appropriate \textsf{childdoc} directives,
the argument of \textit{main} of |\childdocmain| can be empty.

An alternative version of the command line processing described
in \secref{sec:commandline} using the detection mechanism reads:
%
\begin{center}
|... -jobname "|\textit{target}|" "|[\textit{flags}]%
[|\def\jobname{|\textit{dest}|}|]|\input{|\textit{main}|}"|
\end{center}

%%%%%%%%%%%%%%%%%%%%%%%%%%%%%%%%%%%%%%%%%%%%%%%%%%%%%%%%%%%%%%%%%%%%%%%%%%%%%%%%
\subsection{Manual Code}
\label{sec:manual}

In case one cannot be certain whether the definitions file |childdoc.def|
is installed on the target \TeX{} distribution
and one prefers not to ship it,
it is conceivable to paste a few relevant commands into the sources.

To that end, drop all statements |\input{childdoc.def}|
and perform the replacements as outlined below.
Instead of |\childdocmain{|\textit{main}|}| add the following code
to the top of the main file:
%
\begin{center}
\begin{tabular}{l}
|\||ifdefined\childdocname\endinput\||fi\newif\ifchilddoc|\\
|\edef\childdocname{\scantokens\expandafter{\jobname\noexpand}}|\\
|\def\childdocmain{|\textit{main}|}\||ifx\childdocmain\childdocname\||else|\\
|\childdoctrue\includeonly{\childdocname}\let\jobname\childdocmain\||fi|\\
\end{tabular}
\end{center}
%
Instead of |\childdocof{|\textit{main}|}| just include the main file
at the top of each child file:
%
\begin{center}
|\input{|\textit{main}|}|
\end{center}
%
A simple redirection |\childdocforward{|\textit{dest}|}| is achieved by:
%
\begin{center}
|\def\jobname{|\textit{dest}|}\input{\jobname}|
\end{center}
%
The redirection with prefix
|\childdocforwardprefix[|\textit{prefix}|]{|\textit{dest}|}|
is accomplished by:
%
\begin{center}
\begin{tabular}{l}
|{\edef\jobname{\scantokens\expandafter{\jobname\noexpand}}|\\
|\def\redirectjob |\textit{prefix}|#1~~~{\gdef\jobname{|\textit{dest}|#1}}|\\
|\expandafter\redirectjob\jobname~~~}\input{\jobname}|
\end{tabular}
\end{center}

In an alternative approach,
child documents can be compiled by a specific command line
without additional code or specific definitions:
%
\begin{center}
|... -jobname "|\textit{target}|" "|[\textit{flags}]%
|\includeonly{|\textit{dest}|}\input{|\textit{main}|}"|
\end{center}
%

%%%%%%%%%%%%%%%%%%%%%%%%%%%%%%%%%%%%%%%%%%%%%%%%%%%%%%%%%%%%%%%%%%%%%%%%%%%%%%%%
%%%%%%%%%%%%%%%%%%%%%%%%%%%%%%%%%%%%%%%%%%%%%%%%%%%%%%%%%%%%%%%%%%%%%%%%%%%%%%%%
\section{Information}

%%%%%%%%%%%%%%%%%%%%%%%%%%%%%%%%%%%%%%%%%%%%%%%%%%%%%%%%%%%%%%%%%%%%%%%%%%%%%%%%
\subsection{Copyright}

Copyright \copyright{} 2017--2018 Niklas Beisert

This work may be distributed and/or modified under the
conditions of the \LaTeX{} Project Public License, either version 1.3
of this license or (at your option) any later version.
The latest version of this license is in
  \url{http://www.latex-project.org/lppl.txt}
and version 1.3 or later is part of all distributions of \LaTeX{}
version 2005/12/01 or later.

This work has the LPPL maintenance status `maintained'.

The Current Maintainer of this work is Niklas Beisert.

This work consists of the files |README.txt|, |childdoc.ins| and |childdoc.dtx|
as well as the derived files |childdoc.def|, |cdocsamp.tex|
with |cdocsch1.tex|, |cdocsch2.tex|, |cdocspt3.tex|, |cdocspt4.tex|,
|cdocsdrf.tex|, |cdocsfn1.tex|, |cdocsfn2.tex|
as well as |childdoc.pdf|.

%%%%%%%%%%%%%%%%%%%%%%%%%%%%%%%%%%%%%%%%%%%%%%%%%%%%%%%%%%%%%%%%%%%%%%%%%%%%%%%%
\subsection{Files and Installation}

The package consists of the files:
%
\begin{center}
\begin{tabular}{ll}
    |README.txt|   & readme file \\
    |childdoc.ins| & installation file \\
    |childdoc.dtx| & source file \\
    |childdoc.def| & definition file \\
    |cdocsamp.tex| & sample main file \\
    |cdocsch1.tex| & sample include file \\
    |cdocsch2.tex| & sample include file \\
    |cdocspt3.tex| & sample part file \\
    |cdocspt4.tex| & sample part file \\
    |cdocsdrf.tex| & sample redirection file \\
    |cdocsfn1.tex| & sample redirection file \\
    |cdocsfn2.tex| & sample redirection file \\
    |childdoc.pdf| & manual
\end{tabular}
\end{center}
%
The distribution consists of the files
|README.txt|, |childdoc.ins| and |childdoc.dtx|.
%
\begin{itemize}
\item
Run (pdf)\LaTeX{} on |childdoc.dtx|
to compile the manual |childdoc.pdf| (this file).
\item
Run \LaTeX{} on |childdoc.ins| to create the definitions file |childdoc.def|
and the sample |cdocsamp.tex| with include files
|cdocsch1.tex|, |cdocsch2.tex|, |cdocspt3.tex|, |cdocspt4.tex|,
|cdocsdrf.tex|, |cdocsfn1.tex|, |cdocsfn2.tex|.
Then copy the file |childdoc.def| to an appropriate directory of your \LaTeX{}
distribution, e.g.\ \textit{texmf-root}|/tex/latex/childdoc|.
\end{itemize}

%%%%%%%%%%%%%%%%%%%%%%%%%%%%%%%%%%%%%%%%%%%%%%%%%%%%%%%%%%%%%%%%%%%%%%%%%%%%%%%%
\subsection{Related CTAN Packages}

There are several other packages which offer a similar functionality:
%
\begin{itemize}
\item
The packages
\href{http://ctan.org/pkg/docmute}{\textsf{docmute}},
\href{http://ctan.org/pkg/includex}{\textsf{includex}} and
\href{http://ctan.org/pkg/standalone}{\textsf{standalone}}
provide commands to include only the document body of
a child file thus allowing both files to be compiled individually.
\item
The packages \href{http://ctan.org/pkg/subdocs}{\textsf{subdocs}}
and \href{http://ctan.org/pkg/subfiles}{\textsf{subfiles}}
provide structures in which the main and child documents can be
encapsulated and allowing them to be compiled individually.
The inclusion mechanism is different from the conventional |\include|.
\item
The package \href{http://ctan.org/pkg/combine}{\textsf{combine}}
is an elaborate solution to combine several documents into one.
\end{itemize}
%
See also the CTAN topic \href{http://ctan.org/topic/subdocs}{\textsf{subdocs}}
for further related packages.
The present package differs from the above solutions in that
a document structure constructed with the conventional |\include| mechanism
just needs two extra commands at the top of every file
such that all constituent files can be compiled individually.

%%%%%%%%%%%%%%%%%%%%%%%%%%%%%%%%%%%%%%%%%%%%%%%%%%%%%%%%%%%%%%%%%%%%%%%%%%%%%%%%
%\subsection{Feature Suggestions}
%
%The following is a list of features which may be useful for future
%versions of this package:
%%
%\begin{itemize}
%\item
%\ldots
%\end{itemize}

%%%%%%%%%%%%%%%%%%%%%%%%%%%%%%%%%%%%%%%%%%%%%%%%%%%%%%%%%%%%%%%%%%%%%%%%%%%%%%%%
\subsection{Revision History}

%%%%%%%%%%%%%%%%%%%%%%%%%%%%%%%%%%%%%%%%
\paragraph{v2.0:} 2018/12/30

\begin{itemize}
\item
immediate forward processing
\item
added |\childdocby| mechanism
\item
manual restructured
\end{itemize}

%%%%%%%%%%%%%%%%%%%%%%%%%%%%%%%%%%%%%%%%
\paragraph{v1.6:} 2018/01/17

\begin{itemize}
\item
application for development of include files
\item
corrections to manual
\end{itemize}

%%%%%%%%%%%%%%%%%%%%%%%%%%%%%%%%%%%%%%%%
\paragraph{v1.5:} 2017/05/21

\begin{itemize}
\item
more complete structuring introduced
\item
|\childdocof| introduced
\item
|\childdoc| renamed to |\childdocmain|
\item
|\childredirect| renamed to |\childdocforward| and |\childdocforwardprefix|
and functionality expanded
\end{itemize}

%%%%%%%%%%%%%%%%%%%%%%%%%%%%%%%%%%%%%%%%
\paragraph{v1.0:} 2017/04/27

\begin{itemize}
\item
manual and install package
\item
first version published on CTAN
\end{itemize}

%%%%%%%%%%%%%%%%%%%%%%%%%%%%%%%%%%%%%%%%
\paragraph{v0.6:} 2017/04/26

\begin{itemize}
\item
redirection mechanism added
\end{itemize}

%%%%%%%%%%%%%%%%%%%%%%%%%%%%%%%%%%%%%%%%
\paragraph{v0.5:} 2017/04/26

\begin{itemize}
\item
functionality in definition file
\end{itemize}


%%%%%%%%%%%%%%%%%%%%%%%%%%%%%%%%%%%%%%%%%%%%%%%%%%%%%%%%%%%%%%%%%%%%%%%%%%%%%%%%
%%%%%%%%%%%%%%%%%%%%%%%%%%%%%%%%%%%%%%%%%%%%%%%%%%%%%%%%%%%%%%%%%%%%%%%%%%%%%%%%
%%%%%%%%%%%%%%%%%%%%%%%%%%%%%%%%%%%%%%%%%%%%%%%%%%%%%%%%%%%%%%%%%%%%%%%%%%%%%%%%
\appendix

\settowidth\MacroIndent{\rmfamily\scriptsize 000\ }

 \DocInput{childdoc.dtx}

\end{document}
%</driver>
% \fi
%
% %%%%%%%%%%%%%%%%%%%%%%%%%%%%%%%%%%%%%%%%%%%%%%%%%%%%%%%%%%%%%%%%%%%%%%%%%%%%%%
% %%%%%%%%%%%%%%%%%%%%%%%%%%%%%%%%%%%%%%%%%%%%%%%%%%%%%%%%%%%%%%%%%%%%%%%%%%%%%%
% \section{Sample}
%\iffalse
%<*samplemain>
%\fi
%
% The following presents a sample document
% with two chapters, two parts, a title page,
% a compile flag as well as three forwarding files to set the flag.
% It consists of eight |.tex| files:
% \begin{center}
% \begin{tabular}{ll}
% |cdocsamp.tex|&main file\\
% |cdocsch1.tex|&include file for chapter 1\\
% |cdocsch2.tex|&include file for chapter 2\\
% |cdocspt3.tex|&include file for part 3\\
% |cdocspt4.tex|&include file for part 4\\
% |cdocsdrf.tex|&forwarding file for main file in draft mode\\
% |cdocsfi1.tex|&forwarding file for final version of chapter 1\\
% |cdocsfi2.tex|&forwarding file for final version of chapter 2\\
% \end{tabular}
% \end{center}
% Each of the eight files can be compiled directly by the \LaTeX{} compiler.
%
% %%%%%%%%%%%%%%%%%%%%%%%%%%%%%%%%%%%%%%
% \paragraph{Main File.}
%
% The main file is called |cdocsamp.tex|.
%
% Load the \textsf{childdoc} definitions and
% declare the filename for the main document:
%    \begin{macrocode}
\input{childdoc.def}
\childdocmain{}
%    \end{macrocode}

% Optional override for |\version| flag:
%    \begin{macrocode}
%%\ifchilddoc\else\providecommand{\version}{draft}\fi
%    \end{macrocode}

% Define the default values for the |\version| flag
% (|final| for the main file and |draft| for childs):
%    \begin{macrocode}
\ifchilddoc
\providecommand{\version}{draft}
\else
\providecommand{\version}{final}
\fi
%    \end{macrocode}

% Load the standard document class:
%    \begin{macrocode}
\documentclass[12pt]{article}
%    \end{macrocode}

% Start the document body:
%    \begin{macrocode}
\begin{document}
%    \end{macrocode}

% Declare a title page.
% Print title, part of document being processed and version flag:
%    \begin{macrocode}
\addtocounter{page}{-1}
\begin{center}
{\LARGE\bfseries{}childdoc example\par}
\vspace{1cm}
\ifchilddoc
\ifchilddocmanual part\else chapter\fi:
`\childdocname' of `\childdocjob'\par
\else
main document: `\childdocjob'\par
\fi
version: \version\par
\end{center}
\newpage
%    \end{macrocode}

% Manually include selected file,
% otherwise process as usual:
%    \begin{macrocode}
\ifchilddocmanual
\section*{part `\childdocname'}
\input{\childdocname}
\else
%    \end{macrocode}

% Include the two chapters:
%    \begin{macrocode}
\include{cdocsch1}
\include{cdocsch2}
%    \end{macrocode}

% Include the two parts unless only chapters should be displayed:
%    \begin{macrocode}
\ifchilddoc\else
\section{part three}
\input{cdocspt3}
\section{part four}
\input{cdocspt4}
\fi
%    \end{macrocode}

% Process as usual until here:
%    \begin{macrocode}
\fi
%    \end{macrocode}

% End of document body:
%    \begin{macrocode}
\end{document}
%    \end{macrocode}
%\iffalse
%</samplemain>
%\fi
%
% %%%%%%%%%%%%%%%%%%%%%%%%%%%%%%%%%%%%%%
% \paragraph{Chapter Include Files.}
%
% The include files are called |cdocsch1.tex| and |cdocsch2.tex|.
%
%\iffalse
%<*samplechap1|samplechap2>
%\fi

% Optional override for |\version| flag:
%    \begin{macrocode}
%%\providecommand{\version}{final}
%    \end{macrocode}

% Include the main document:
%    \begin{macrocode}
\input{childdoc.def}
\childdocof{cdocsamp}
%    \end{macrocode}

%\iffalse
%</samplechap1|samplechap2>
%\fi
%
%\iffalse
%<*samplechap1>
%\fi
% Some text for chapter 1:
%    \begin{macrocode}
\section{one}
some text in chapter one
%    \end{macrocode}

%\iffalse
%</samplechap1>
%\fi
% Some text for chapter 2:
%\iffalse
%<*samplechap2>
%\fi
%    \begin{macrocode}
\section{two}
more text in chapter two
%    \end{macrocode}

%\iffalse
%</samplechap2>
%\fi
%
% %%%%%%%%%%%%%%%%%%%%%%%%%%%%%%%%%%%%%%
% \paragraph{Part Include Files.}
%
% The include files are called |cdocspt3.tex| and |cdocspt4.tex|.
%
%\iffalse
%<*samplepart3|samplepart4>
%\fi

% Optional override for |\version| flag:
%    \begin{macrocode}
%%\providecommand{\version}{final}
%    \end{macrocode}

% Include the main document:
%    \begin{macrocode}
\input{childdoc.def}
\childdocby{cdocsamp}
%    \end{macrocode}

%\iffalse
%</samplepart3|samplepart4>
%\fi
%
%\iffalse
%<*samplepart3>
%\fi
% Some text for part 3:
%    \begin{macrocode}
some text in part three
%    \end{macrocode}

%\iffalse
%</samplepart3>
%\fi
% Some text for part 4:
%\iffalse
%<*samplepart4>
%\fi
%    \begin{macrocode}
more text in part four
%    \end{macrocode}

%\iffalse
%</samplepart4>
%\fi
%
% %%%%%%%%%%%%%%%%%%%%%%%%%%%%%%%%%%%%%%
% \paragraph{Forwarding for a Complete Draft.}
%
% The following forwarding file |cdocsdrf.tex|
% compiles the main document in draft mode:
%\iffalse
%<*sampledraft>
%\fi
%    \begin{macrocode}
\def\version{draft}
\input{childdoc.def}
\childdocforward{cdocsamp}
%    \end{macrocode}

%\iffalse
%</sampledraft>
%\fi
%
% %%%%%%%%%%%%%%%%%%%%%%%%%%%%%%%%%%%%%%
% \paragraph{Forwarding for Final Version of the Chapters.}
%
% The following forwarding files |cdocsfn1.tex| and |cdocsfn2.tex|
% (with identical content)
% compile the final versions of the child documents
% |cdocsch1.tex| and |cdocsch2.tex|, respectively:
%\iffalse
%<*samplefinal>
%\fi
%    \begin{macrocode}
\def\version{final}
\input{childdoc.def}
\childdocforwardprefix[cdocsamp]{cdocsfn}{cdocsch}
%    \end{macrocode}

%\iffalse
%</samplefinal>
%\fi
%
% %%%%%%%%%%%%%%%%%%%%%%%%%%%%%%%%%%%%%%
% \paragraph{Command Line Processing.}
%
% The following three command lines generate the output files
% |cdocscld|, |cdocscl1| and |cdocscl2|
% which should be identical to
% |cdocsdrf|, |cdocsch1| and |cdocsfn2|, respectively:
% \begin{center}
% \begin{tabular}{l}
% |latex -jobname cdocscld \|\\
% |  "\def\version{draft}\input{childdoc.def}\childdocforward{cdocsamp}"|\\
% |latex -jobname cdocscl1 \|\\
% |  "\input{childdoc.def}\childdocforward[cdocsamp]{cdocsch1}"|\\
% |latex -jobname cdocscl2 \|\\
% |  "\def\version{final}\input{childdoc.def}\childdocforward{cdocsch2}"|
% \end{tabular}
% \end{center}
% Note that the trailing backslash on each first line
% merely continues the input to the second line
% (for convenient cut ant paste).
% Furthermore, the command |latex| can be replaced by any
% of its alternative versions such as |pdflatex|.
%
% %%%%%%%%%%%%%%%%%%%%%%%%%%%%%%%%%%%%%%%%%%%%%%%%%%%%%%%%%%%%%%%%%%%%%%%%%%%%%%
% %%%%%%%%%%%%%%%%%%%%%%%%%%%%%%%%%%%%%%%%%%%%%%%%%%%%%%%%%%%%%%%%%%%%%%%%%%%%%%
% \section{Implementation}
%\iffalse
%<*package>
%\fi
%
% This section describes the definitions file |childdoc.def|.

% The definitions cannot be loaded using |\usepackage| or |\RequirePackage|
% which has a mechanism to prevent loading a style file more than once.
% When loading the definitions by means of |\input|
% multiple instances have to be prevented manually:
%\iffalse
%This code needs to be before the `\ProvidesFile' directive
%which is defined at the beginning of this file.
%Therefore it is also placed there and commented out here.
%</package>
%<*discard>
%\fi
%    \begin{macrocode}
\ifdefined\childdocmain\endinput\fi
%    \end{macrocode}
%\iffalse
%</discard>
%<*package>
%\fi
%
% \macro{\ifchilddoc}
% \macro{\ifchilddocmanual}
% The conditional |\ifchilddoc| tells whether a
% child (true) or main (false) document is being compiled.
% The conditional |\ifchilddocmanual| tells whether
% the |\includeonly| mechanism is used (false) or
% the selection of child files must be performed manually (true).
% The definitions initialise to false:
%    \begin{macrocode}
\newif\ifchilddoc
\newif\ifchilddocmanual
%    \end{macrocode}

% \macro{\childdocname}
% \macro{\childdocjob}
% The macro |\childdocname| stores the name of the main document
% to be compiled. The macro |\childdocjob| stores the name of
% the document on which the \LaTeX{} compiler was originally invoked.
% The content of |\jobname| cannot be compared
% to filenames specified in the source due to different catcodes.
% The following code rescans |\jobname|, stores the result
% in |\childdocname| and saves a copy in |\childdocjob|:
%    \begin{macrocode}
\edef\childdocname{\scantokens\expandafter{\jobname\noexpand}}
\let\childdocjob\childdocname
%    \end{macrocode}

% \macro{\childdocdisable}
% The macro |\childdocdisable| prevents the main file
% from being processed more than once.
% At this stage, the main document command |\childdocmain|
% is assumed to be called once again where it should do nothing.
% Any subsequent call to it should prevent
% a secondary processing of the main document
% It overwrites the forwarding commands
% |\childdocof| and |\childdocforward|
% with empty macros to prevent further inclusions of the main document:
%    \begin{macrocode}
\newcommand{\childdocdisable}
{
  \renewcommand{\childdocmain}[1]{\renewcommand{\childdocmain}[1]{\endinput}}
  \renewcommand{\childdocof}[1]{}
  \renewcommand{\childdocby}[2][]{}
  \renewcommand{\childdocforward}[2][]{}
  \renewcommand{\childdocdisable}{}
}
%    \end{macrocode}

% \macro{\childdocmain}
% The macro |\childdocmain| is to be called at the top of the main file
% with nothing or the main filename (without extension) as argument.
% First, it breaks loops.
% If the argument is not empty and does not match |\childdocname|
% (which is set by the first inclusion of |childdoc.def|),
% |\ifchilddoc| is set to true, |\includeonly| is applied to the child file
% and |\jobname| is set to the main file
% (for proper handling of |.aux| files):
%    \begin{macrocode}
\newcommand{\childdocmain}[1]
{
  \childdocdisable\childdocmain{}
  \if?#1?\else
    \begingroup
      \def\childdoctmp{#1}
      \ifx\childdoctmp\childdocname
        \def\childdoctmp{}
      \else
        \def\childdoctmp
        {
          \childdoctrue
          \includeonly{\childdocname}
          \def\childdocjob{#1}
          \def\jobname{#1}
        }
      \fi
      \expandafter
    \endgroup
    \childdoctmp
  \fi
}
%    \end{macrocode}

% \macro{\childdocof}
% The command |\childdocof| redirects
% compilation to the main file |#1|.
%    \begin{macrocode}
\newcommand{\childdocof}[1]
{
  \childdocdisable
  \childdoctrue
  \includeonly{\childdocname}
  \def\jobname{#1}
  \def\childdocjob{#1}
  \input{#1}
}
%    \end{macrocode}

% \macro{\childdocby}
% The command |\childdocby| ....
%    \begin{macrocode}
\newcommand{\childdocby}[2][]
{
  \childdocdisable
  \childdoctrue
  \childdocmanualtrue
  \if?#1?\else
    \def\jobname{#2}
  \fi
  \def\childdocjob{#2}
  \input{#2}
  \endinput
}
%    \end{macrocode}

% \macro{\childdocforward}
% The command |\childdocforward| redirects
% compilation to the main file or
% (if the optional argument is given) a child file.
% Parameters are set as if the main file
% or a child file starting with |\childdocof| was compiled.
% Then compilation is handed over to the main file:
%    \begin{macrocode}
\newcommand{\childdocforward}[2][]
{
  \begingroup
    \if?#1?
      \def\childdoctmp
      {
        \def\childdocname{#2}
        \def\childdocjob{#2}
        \def\jobname{#2}
        \input{#2}
        \endinput
      }
    \else
      \def\childdoctmp
      {
        \childdocdisable
        \def\childdocname{#2}
        \childdoctrue
        \includeonly{#2}
        \def\childdocjob{#1}
        \def\jobname{#1}
        \input{#1}
        \endinput
      }
    \fi
    \expandafter
  \endgroup
  \childdoctmp
}
%    \end{macrocode}

% \macro{\childdocforwardprefix}
% The command |\childdocforwardprefix| redirects
% compilation to the main or a child file by means of a pattern.
% The prefix |#1| in the current filename is replaced by |#2|
% and the suffix of the current filename is kept
% (it is assumed that the filename does not contain the substring `|~~~|'
% which is used as a delimiter).
% Compilation is handed over to the new file by |\childdocforward|:
%    \begin{macrocode}
\newcommand{\childdocforwardprefix}[3][]
{
  \begingroup
    \def\childdocextract #2##1~~~{\def\childdoctmp{\childdocforward[#1]{#3##1}}}
    \expandafter\childdocextract\childdocname~~~
    \expandafter
  \endgroup
  \childdoctmp
}
%    \end{macrocode}

% \macro{\childdoc}
% The deprecated macro |\childdoc| is a legacy version of |\childdocmain|:
%    \begin{macrocode}
\newcommand{\childdoc}{\childdocmain}
%    \end{macrocode}

% \macro{\childdocredirect}
% The deprecated macro |\childdocredirect| is a legacy version
% of |\childdocforward| and |\childdocforwardprefix|:
%    \begin{macrocode}
\newcommand{\childdocredirect}[2][]
{
  \begingroup
    \if?#1?
      \def\childdoctmp{\childdocforward{#2}}
    \else
      \def\childdoctmp{\childdocforwardprefix{#1}{#2}}
    \fi
    \expandafter
  \endgroup
  \childdoctmp
}
%    \end{macrocode}

%\iffalse
%</package>
%\fi
%
\endinput
|
and perform the replacements as outlined below.
Instead of |\childdocmain{|\textit{main}|}| add the following code
to the top of the main file:
%
\begin{center}
\begin{tabular}{l}
|\||ifdefined\childdocname\endinput\||fi\newif\ifchilddoc|\\
|\edef\childdocname{\scantokens\expandafter{\jobname\noexpand}}|\\
|\def\childdocmain{|\textit{main}|}\||ifx\childdocmain\childdocname\||else|\\
|\childdoctrue\includeonly{\childdocname}\let\jobname\childdocmain\||fi|\\
\end{tabular}
\end{center}
%
Instead of |\childdocof{|\textit{main}|}| just include the main file
at the top of each child file:
%
\begin{center}
|\input{|\textit{main}|}|
\end{center}
%
A simple redirection |\childdocforward{|\textit{dest}|}| is achieved by:
%
\begin{center}
|\def\jobname{|\textit{dest}|}\input{\jobname}|
\end{center}
%
The redirection with prefix
|\childdocforwardprefix[|\textit{prefix}|]{|\textit{dest}|}|
is accomplished by:
%
\begin{center}
\begin{tabular}{l}
|{\edef\jobname{\scantokens\expandafter{\jobname\noexpand}}|\\
|\def\redirectjob |\textit{prefix}|#1~~~{\gdef\jobname{|\textit{dest}|#1}}|\\
|\expandafter\redirectjob\jobname~~~}\input{\jobname}|
\end{tabular}
\end{center}

In an alternative approach,
child documents can be compiled by a specific command line
without additional code or specific definitions:
%
\begin{center}
|... -jobname "|\textit{target}|" "|[\textit{flags}]%
|\includeonly{|\textit{dest}|}\input{|\textit{main}|}"|
\end{center}
%

%%%%%%%%%%%%%%%%%%%%%%%%%%%%%%%%%%%%%%%%%%%%%%%%%%%%%%%%%%%%%%%%%%%%%%%%%%%%%%%%
%%%%%%%%%%%%%%%%%%%%%%%%%%%%%%%%%%%%%%%%%%%%%%%%%%%%%%%%%%%%%%%%%%%%%%%%%%%%%%%%
\section{Information}

%%%%%%%%%%%%%%%%%%%%%%%%%%%%%%%%%%%%%%%%%%%%%%%%%%%%%%%%%%%%%%%%%%%%%%%%%%%%%%%%
\subsection{Copyright}

Copyright \copyright{} 2017--2018 Niklas Beisert

This work may be distributed and/or modified under the
conditions of the \LaTeX{} Project Public License, either version 1.3
of this license or (at your option) any later version.
The latest version of this license is in
  \url{http://www.latex-project.org/lppl.txt}
and version 1.3 or later is part of all distributions of \LaTeX{}
version 2005/12/01 or later.

This work has the LPPL maintenance status `maintained'.

The Current Maintainer of this work is Niklas Beisert.

This work consists of the files |README.txt|, |childdoc.ins| and |childdoc.dtx|
as well as the derived files |childdoc.def|, |cdocsamp.tex|
with |cdocsch1.tex|, |cdocsch2.tex|, |cdocspt3.tex|, |cdocspt4.tex|,
|cdocsdrf.tex|, |cdocsfn1.tex|, |cdocsfn2.tex|
as well as |childdoc.pdf|.

%%%%%%%%%%%%%%%%%%%%%%%%%%%%%%%%%%%%%%%%%%%%%%%%%%%%%%%%%%%%%%%%%%%%%%%%%%%%%%%%
\subsection{Files and Installation}

The package consists of the files:
%
\begin{center}
\begin{tabular}{ll}
    |README.txt|   & readme file \\
    |childdoc.ins| & installation file \\
    |childdoc.dtx| & source file \\
    |childdoc.def| & definition file \\
    |cdocsamp.tex| & sample main file \\
    |cdocsch1.tex| & sample include file \\
    |cdocsch2.tex| & sample include file \\
    |cdocspt3.tex| & sample part file \\
    |cdocspt4.tex| & sample part file \\
    |cdocsdrf.tex| & sample redirection file \\
    |cdocsfn1.tex| & sample redirection file \\
    |cdocsfn2.tex| & sample redirection file \\
    |childdoc.pdf| & manual
\end{tabular}
\end{center}
%
The distribution consists of the files
|README.txt|, |childdoc.ins| and |childdoc.dtx|.
%
\begin{itemize}
\item
Run (pdf)\LaTeX{} on |childdoc.dtx|
to compile the manual |childdoc.pdf| (this file).
\item
Run \LaTeX{} on |childdoc.ins| to create the definitions file |childdoc.def|
and the sample |cdocsamp.tex| with include files
|cdocsch1.tex|, |cdocsch2.tex|, |cdocspt3.tex|, |cdocspt4.tex|,
|cdocsdrf.tex|, |cdocsfn1.tex|, |cdocsfn2.tex|.
Then copy the file |childdoc.def| to an appropriate directory of your \LaTeX{}
distribution, e.g.\ \textit{texmf-root}|/tex/latex/childdoc|.
\end{itemize}

%%%%%%%%%%%%%%%%%%%%%%%%%%%%%%%%%%%%%%%%%%%%%%%%%%%%%%%%%%%%%%%%%%%%%%%%%%%%%%%%
\subsection{Related CTAN Packages}

There are several other packages which offer a similar functionality:
%
\begin{itemize}
\item
The packages
\href{http://ctan.org/pkg/docmute}{\textsf{docmute}},
\href{http://ctan.org/pkg/includex}{\textsf{includex}} and
\href{http://ctan.org/pkg/standalone}{\textsf{standalone}}
provide commands to include only the document body of
a child file thus allowing both files to be compiled individually.
\item
The packages \href{http://ctan.org/pkg/subdocs}{\textsf{subdocs}}
and \href{http://ctan.org/pkg/subfiles}{\textsf{subfiles}}
provide structures in which the main and child documents can be
encapsulated and allowing them to be compiled individually.
The inclusion mechanism is different from the conventional |\include|.
\item
The package \href{http://ctan.org/pkg/combine}{\textsf{combine}}
is an elaborate solution to combine several documents into one.
\end{itemize}
%
See also the CTAN topic \href{http://ctan.org/topic/subdocs}{\textsf{subdocs}}
for further related packages.
The present package differs from the above solutions in that
a document structure constructed with the conventional |\include| mechanism
just needs two extra commands at the top of every file
such that all constituent files can be compiled individually.

%%%%%%%%%%%%%%%%%%%%%%%%%%%%%%%%%%%%%%%%%%%%%%%%%%%%%%%%%%%%%%%%%%%%%%%%%%%%%%%%
%\subsection{Feature Suggestions}
%
%The following is a list of features which may be useful for future
%versions of this package:
%%
%\begin{itemize}
%\item
%\ldots
%\end{itemize}

%%%%%%%%%%%%%%%%%%%%%%%%%%%%%%%%%%%%%%%%%%%%%%%%%%%%%%%%%%%%%%%%%%%%%%%%%%%%%%%%
\subsection{Revision History}

%%%%%%%%%%%%%%%%%%%%%%%%%%%%%%%%%%%%%%%%
\paragraph{v2.0:} 2018/12/30

\begin{itemize}
\item
immediate forward processing
\item
added |\childdocby| mechanism
\item
manual restructured
\end{itemize}

%%%%%%%%%%%%%%%%%%%%%%%%%%%%%%%%%%%%%%%%
\paragraph{v1.6:} 2018/01/17

\begin{itemize}
\item
application for development of include files
\item
corrections to manual
\end{itemize}

%%%%%%%%%%%%%%%%%%%%%%%%%%%%%%%%%%%%%%%%
\paragraph{v1.5:} 2017/05/21

\begin{itemize}
\item
more complete structuring introduced
\item
|\childdocof| introduced
\item
|\childdoc| renamed to |\childdocmain|
\item
|\childredirect| renamed to |\childdocforward| and |\childdocforwardprefix|
and functionality expanded
\end{itemize}

%%%%%%%%%%%%%%%%%%%%%%%%%%%%%%%%%%%%%%%%
\paragraph{v1.0:} 2017/04/27

\begin{itemize}
\item
manual and install package
\item
first version published on CTAN
\end{itemize}

%%%%%%%%%%%%%%%%%%%%%%%%%%%%%%%%%%%%%%%%
\paragraph{v0.6:} 2017/04/26

\begin{itemize}
\item
redirection mechanism added
\end{itemize}

%%%%%%%%%%%%%%%%%%%%%%%%%%%%%%%%%%%%%%%%
\paragraph{v0.5:} 2017/04/26

\begin{itemize}
\item
functionality in definition file
\end{itemize}


%%%%%%%%%%%%%%%%%%%%%%%%%%%%%%%%%%%%%%%%%%%%%%%%%%%%%%%%%%%%%%%%%%%%%%%%%%%%%%%%
%%%%%%%%%%%%%%%%%%%%%%%%%%%%%%%%%%%%%%%%%%%%%%%%%%%%%%%%%%%%%%%%%%%%%%%%%%%%%%%%
%%%%%%%%%%%%%%%%%%%%%%%%%%%%%%%%%%%%%%%%%%%%%%%%%%%%%%%%%%%%%%%%%%%%%%%%%%%%%%%%
\appendix

\settowidth\MacroIndent{\rmfamily\scriptsize 000\ }

 \DocInput{childdoc.dtx}

\end{document}
%</driver>
% \fi
%
% %%%%%%%%%%%%%%%%%%%%%%%%%%%%%%%%%%%%%%%%%%%%%%%%%%%%%%%%%%%%%%%%%%%%%%%%%%%%%%
% %%%%%%%%%%%%%%%%%%%%%%%%%%%%%%%%%%%%%%%%%%%%%%%%%%%%%%%%%%%%%%%%%%%%%%%%%%%%%%
% \section{Sample}
%\iffalse
%<*samplemain>
%\fi
%
% The following presents a sample document
% with two chapters, two parts, a title page,
% a compile flag as well as three forwarding files to set the flag.
% It consists of eight |.tex| files:
% \begin{center}
% \begin{tabular}{ll}
% |cdocsamp.tex|&main file\\
% |cdocsch1.tex|&include file for chapter 1\\
% |cdocsch2.tex|&include file for chapter 2\\
% |cdocspt3.tex|&include file for part 3\\
% |cdocspt4.tex|&include file for part 4\\
% |cdocsdrf.tex|&forwarding file for main file in draft mode\\
% |cdocsfi1.tex|&forwarding file for final version of chapter 1\\
% |cdocsfi2.tex|&forwarding file for final version of chapter 2\\
% \end{tabular}
% \end{center}
% Each of the eight files can be compiled directly by the \LaTeX{} compiler.
%
% %%%%%%%%%%%%%%%%%%%%%%%%%%%%%%%%%%%%%%
% \paragraph{Main File.}
%
% The main file is called |cdocsamp.tex|.
%
% Load the \textsf{childdoc} definitions and
% declare the filename for the main document:
%    \begin{macrocode}
% \iffalse
%
% childdoc.dtx Copyright (C) 2017-2018 Niklas Beisert
%
% This work may be distributed and/or modified under the
% conditions of the LaTeX Project Public License, either version 1.3
% of this license or (at your option) any later version.
% The latest version of this license is in
%   http://www.latex-project.org/lppl.txt
% and version 1.3 or later is part of all distributions of LaTeX
% version 2005/12/01 or later.
%
% This work has the LPPL maintenance status `maintained'.
%
% The Current Maintainer of this work is Niklas Beisert.
%
% This work consists of the files childdoc.dtx and childdoc.ins
% and the derived files childdoc.def and cdocsamp.tex with
% cdocsch1.tex, cdocsch2.tex, cdocsdrf.tex, cdocsfn1.tex, cdocsfn2.tex.
%
%<package>\ifdefined\childdocmain\endinput\fi
%<package>\ProvidesFile{childdoc.def}[2018/12/30 v2.0 child document driver]
%<samplemain>\ProvidesFile{cdocsamp.tex}[2018/12/30 v2.0 sample for childdoc]
%<*driver>
%\ProvidesFile{childdoc.drv}[2018/12/30 v2.0 childdoc reference manual file]
\PassOptionsToClass{10pt,a4paper}{article}
\documentclass{ltxdoc}

\usepackage[margin=35mm]{geometry}
\usepackage{hyperref}
\usepackage{hyperxmp}
\usepackage[usenames]{color}

\hypersetup{colorlinks=true}
\hypersetup{pdfstartview=FitH}
\hypersetup{pdfpagemode=UseNone}
\hypersetup{pdfsource={}}
\hypersetup{pdflang={en-UK}}
\hypersetup{pdfcopyright={Copyright 2017-2018 Niklas Beisert.
  This work may be distributed and/or modified under the
  conditions of the LaTeX Project Public License, either version 1.3
  of this license or (at your option) any later version.}}
\hypersetup{pdflicenseurl={http://www.latex-project.org/lppl.txt}}
\hypersetup{pdfcontactaddress={ETH Zurich, ITP, HIT K,
  Wolfgang-Pauli-Strasse 27}}
\hypersetup{pdfcontactpostcode={8093}}
\hypersetup{pdfcontactcity={Zurich}}
\hypersetup{pdfcontactcountry={Switzerland}}
\hypersetup{pdfcontactemail={nbeisert@itp.phys.ethz.ch}}
\hypersetup{pdfcontacturl={http://people.phys.ethz.ch/\xmptilde nbeisert/}}

\newcommand{\secref}[1]{\hyperref[#1]{section \ref*{#1}}}

\parskip1ex
\parindent0pt
\let\olditemize\itemize
\def\itemize{\olditemize\parskip0pt}

\begin{document}

\title{The \textsf{childdoc} Package}
\hypersetup{pdftitle={The childdoc Package}}
\author{Niklas Beisert\\[2ex]
  Institut f\"ur Theoretische Physik\\
  Eidgen\"ossische Technische Hochschule Z\"urich\\
  Wolfgang-Pauli-Strasse 27, 8093 Z\"urich, Switzerland\\[1ex]
  \href{mailto:nbeisert@itp.phys.ethz.ch}
  {\texttt{nbeisert@itp.phys.ethz.ch}}}
\hypersetup{pdfauthor={Niklas Beisert}}
\hypersetup{pdfsubject={Manual for the LaTeX2e Package childdoc}}
\date{30 December 2018, \textsf{v2.0}}
\maketitle

\begin{abstract}\noindent
\textsf{childdoc} is a \LaTeXe{} package
that enables the direct compilation
of document sections included by |\include|
to individual files.
\end{abstract}

\begingroup
\parskip0ex
\tableofcontents
\endgroup

%%%%%%%%%%%%%%%%%%%%%%%%%%%%%%%%%%%%%%%%%%%%%%%%%%%%%%%%%%%%%%%%%%%%%%%%%%%%%%%%
%%%%%%%%%%%%%%%%%%%%%%%%%%%%%%%%%%%%%%%%%%%%%%%%%%%%%%%%%%%%%%%%%%%%%%%%%%%%%%%%
\section{Introduction}

\LaTeX{} provides a mechanism to structure a large document (such as a book)
into a main file and several child files (containing the chapters)
using the |\include| command.
This mechanism is beneficial for documents
which span hundreds of pages in order to
make the source file(s) more manageable.
Moreover, compilation can be restricted to
selected child files by means of the |\includeonly| command.
The latter feature can be used to reduce the compilation time while editing
(this was significantly more useful in the earlier days of \LaTeX{})
or to generate a smaller document which is easier to navigate.
Another application of |\includeonly| is to generate
documents consisting of selected parts of the complete document.

However, there are a few drawbacks of the plain |\include| mechanism:
\begin{itemize}
\item
The child files cannot be compiled on their own,
they can only be compiled via the main file.
A naive editing environment
(such as a text editor with an option
to have the current file processed by \LaTeX)
may require one to switch to the main file before compiling;
attempting to compile the child file produces errors.
\item
The main file must be modified (each time)
to adjust the |\includeonly| command
to the present needs. This easily leaves the main file in a messy state.
\item
The generated document will always carry the filename
of the main document. This is inconvenient if
several child files are to be compiled and
to be kept for distribution.
\end{itemize}

The present package provides a simple interface
to make child files individually compilable by \LaTeX{}.
Compiling a child file then has the same effect as compiling
the main file with an |\includeonly| command
to select the appropriate child.
Moreover the generated document will carry the name of the child
rather than the main file.
This resolves all three above issues.

This feature is meant to make the editing of books,
thesis documents and lecture notes somewhat more convenient.
However, the package can also be used efficiently for
composing a series of documents (such as exercise sheets)
which are typically distributed individually.
It then assists the author in generating the individual documents
(potentially in different versions)
as well as a document containing the collected series.
Another application is in developing style files
or other kinds of included material
where compilation of the style file could redirect
to a sample or test file.

%%%%%%%%%%%%%%%%%%%%%%%%%%%%%%%%%%%%%%%%%%%%%%%%%%%%%%%%%%%%%%%%%%%%%%%%%%%%%%%%
%%%%%%%%%%%%%%%%%%%%%%%%%%%%%%%%%%%%%%%%%%%%%%%%%%%%%%%%%%%%%%%%%%%%%%%%%%%%%%%%
\section{Usage}

First of all, the package \textsf{childdoc} is \emph{not} a standard
\LaTeXe{} |.sty| style file! Therefore it needs to be invoked in
a non-standard way.

%%%%%%%%%%%%%%%%%%%%%%%%%%%%%%%%%%%%%%%%%%%%%%%%%%%%%%%%%%%%%%%%%%%%%%%%%%%%%%%%
\subsection{Included Files}
\label{sec:include}

%%%%%%%%%%%%%%%%%%%%%%%%%%%%%%%%%%%%%%%%
\DescribeMacro{\childdocmain}
To use the package, add the commands
\begin{center}
\begin{tabular}{l}
|\input{childdoc.def}|\\
|\childdocmain{}|\\
\end{tabular}
\end{center}
at the very top of the main \LaTeX{} file,
in particular \emph{before} the |\documentclass| statement!
The argument of |\childdocmain| should be left empty
(but it must be present).

%%%%%%%%%%%%%%%%%%%%%%%%%%%%%%%%%%%%%%%%
\DescribeMacro{\childdocof}
Furthermore, add the commands
\begin{center}
\begin{tabular}{l}
|\input{childdoc.def}|\\
|\childdocof{|\textit{main}|}|\\
\end{tabular}
\end{center}
at the top of every child file \textit{child}
which is included by |\include{|\textit{child}|}|
from within the main file
(or at least for those files to be compiled individually).
The argument \textit{main} must be the filename of the main file.

There are a couple of
considerations in setting up the main and child documents:

%%%%%%%%%%%%%%%%%%%%%%%%%%%%%%%%%%%%%%%%
\paragraph{Restrictions.}

Please note the following restrictions:
\begin{itemize}
\item
|\childdocmain| must be called with one argument \textit{main}
to ensure compatibility with earlier version of the package.
It must either be empty (|\childdocmain{}|)
or precisely match the filename of the main file in which it is specified.
See \secref{sec:detection} for further information.
\item
The filename \textit{main} must be specified without the |.tex| extension.
\item
The filename \textit{main} is case sensitive
(even in case-insensitive file systems)
due to internal string comparison.
\item
The argument \textit{main} should be fully expanded, it cannot be a macro.
\item
Subdirectories and special characters should be avoided in filenames.
\item
The command |\childdocmain{|\textit{main}|}| must be followed by a whitespace.
It should not be followed immediately by another command
or by a comment mark `|%|'.
This is because the \TeX{} parser reads the token immediately following
the argument of |\childdocmain| and puts it
at the beginning of every child section;
however, a white\-space is ignored.
\end{itemize}

%%%%%%%%%%%%%%%%%%%%%%%%%%%%%%%%%%%%%%%%
\paragraph{Content of Main File.}

It is advisable to place all content in the child files included by |\include|.
Any output contained in the main file will appear in all child documents
unless suppressed manually;
it cannot be suppressed automatically by the |\includeonly| directive
and thus should normally be avoided.
A method to include some content in the main file
by means of conditional processing is described in \secref{sec:conditional}.

%%%%%%%%%%%%%%%%%%%%%%%%%%%%%%%%%%%%%%%%
\paragraph{Page Numbering.}

When only a part of the document is compiled,
the appropriate numbering of pages
(as well as other status parameters)
is determined from the |.aux| files.
The latter contain information from previous passes.
However this information needs to propagate through
all intermediate child documents.
Therefore the page numbering in child documents may well
be inconsistent until the complete document is compiled at least once.

A useful (if unconventional) way to always ensure a consistent
page numbering is to restart the numbering in each child document
and denote the pages by `\textit{child}|.|\textit{page}'
where \textit{child} represents the chapter/section number of the child file.
This can be achieved by the command
|\numberwithin{page}{|\textit{child}|}|
of the \textsf{amsmath} package
where \textit{child} can be |chapter| or |section|
depending on the chosen structuring.
Alternatively, one can modify the macro |\thepage| appropriately
and reset the counter |page| at the start of each child file.

%%%%%%%%%%%%%%%%%%%%%%%%%%%%%%%%%%%%%%%%%%%%%%%%%%%%%%%%%%%%%%%%%%%%%%%%%%%%%%%%
\subsection{Conditional Processing}
\label{sec:conditional}

The package provides a mechanism to compile different versions
of a document. To customise the versions further some conditional processing
can come in handy to distinguish which version is being compiled.
The package provides two macros to describe the compilation context:

%%%%%%%%%%%%%%%%%%%%%%%%%%%%%%%%%%%%%%%%
\DescribeMacro{\ifchilddoc}
The conditional |\ifchilddoc| distinguishes between the compilation of
child documents and the main document:
%
\begin{center}
|\ifchilddoc |\textit{child-code}| |[|\||else |\textit{main-code}]| \||fi|
\end{center}

%%%%%%%%%%%%%%%%%%%%%%%%%%%%%%%%%%%%%%%%
\DescribeMacro{\childdocname}
\DescribeMacro{\childdocjob}
The macro |\childdocname| contains the filename (without extension)
of the main or child file being processed.
Note that |\childdocjob| will always contain the name of the main file.

%%%%%%%%%%%%%%%%%%%%%%%%%%%%%%%%%%%%%%%%
\paragraph{Title Page.}

Conditional processing can be used to include a title or banner page
in the main document when proper precautions are taken.
Importantly, the code in the main file should ensure that the page counter
(as well as other status parameters which are stored in the |.aux| files)
takes the same value after the conditional processing.
Otherwise the page numbers may take divergent values
depending on which part is compiled.

For example, a title page could be declared by:
%
\begin{center}
\begin{tabular}{l}
|\ifchilddoc\||else|\\
|\addtocounter{page}{-1}|\\
\textit{code for title page}\\
|\newpage|\\
|\||fi|
\end{tabular}
\end{center}
%
A banner page for the child documents can be generated by:
%
\begin{center}
\begin{tabular}{l}
|\ifchilddoc|\\
|\addtocounter{page}{-1}|\\
\textit{code for banner page}\\
|\newpage|\\
|\||fi|
\end{tabular}
\end{center}
%
Here one could write a message such as:
\begin{center}
|This is the part \childdocname{} of \childdocjob{}.|
\end{center}

%%%%%%%%%%%%%%%%%%%%%%%%%%%%%%%%%%%%%%%%%%%%%%%%%%%%%%%%%%%%%%%%%%%%%%%%%%%%%%%%
\subsection{Flags}
\label{sec:flags}

The package makes it easy to generate different versions
of the main or child documents.
To this end compilation flags can be defined
and assigned different default values.
They will be particularly useful in conjunction
with the forwarding mechanism described in \secref{sec:forward}.

For example, it may be useful to have a flag |\version|
which can be set to |draft| or |final|.
The document source will contain some conditional code
depending on the value of |\version|.
Suppose further, the flag should default to |final| for the main file
and to |draft| for child files
which is a natural assignment for editing the document.
This is achieved by placing the following code
in the preamble of the main document
(below the |\childdocmain| directive):
%
\begin{center}
\begin{tabular}{l}
|\ifchilddoc|\\
|\providecommand{\version}{draft}|\\
|\||else|\\
|\providecommand{\version}{final}|\\
|\||fi|
\end{tabular}
\end{center}
%
The definition by |\providecommand| makes sure
that previous definitions are not overwritten.
Further statements |\providecommand{\version}{...}|
can thus be added before the above code to override it.

For the main file, one might add a line
(between |\childdocmain| and the above block)
%
\begin{center}
|%\ifchilddoc\||else\providecommand{\version}{draft}\||fi|
\end{center}
%
which can be uncommented to produce a draft version.
Likewise one can add a line to the very top of a child file
(above the |\childdocof{|\textit{main}|}| directive)
%
\begin{center}
|%\providecommand{\version}{final}|
\end{center}
%
which can be uncommented to produce the final version of this child document.

%%%%%%%%%%%%%%%%%%%%%%%%%%%%%%%%%%%%%%%%%%%%%%%%%%%%%%%%%%%%%%%%%%%%%%%%%%%%%%%%
\subsection{Forwarding}
\label{sec:forward}

Different versions of the main or child documents
using compilation flags as described in \secref{sec:flags}
can be (permanently) stored in different files
for convenient compilation, viewing and distribution.
To this end, the package defines a command
to pass on compilation to a different file:

%%%%%%%%%%%%%%%%%%%%%%%%%%%%%%%%%%%%%%%%
\DescribeMacro{\childdocforward}
The command |\childdocforward| redirects processing to
another source file:
%
\begin{center}
\begin{tabular}{l}
|\input{childdoc.def}|\\
|\childdocforward[|\textit{main}|]{|\textit{dest}|}|\\
\end{tabular}
\end{center}
%
The argument \textit{dest} is the destination file
(without extension).
It should be the main file or one of the child files.
Note that further \textsf{childdoc} directives
such as |\childdocof| and |\childdocforward|
in the indicated file will be processed in this form.
The optional argument \textit{main}
passes on directly to the main file \textit{main}
while pretending to compile the child \textit{dest}.
This form behaves as if \textit{dest}
issues |\childdocof{|\textit{main}|}| right away,
and no further \textsf{childdoc} directives will be processed.

%%%%%%%%%%%%%%%%%%%%%%%%%%%%%%%%%%%%%%%%
\DescribeMacro{\...prefix}
In the alternative form |\childdocforwardprefix|,
%
\begin{center}
\begin{tabular}{l}
|\input{childdoc.def}|\\
|\childdocforwardprefix[|\textit{main}|]{|\textit{prefix}|}{|\textit{dest}|}|
\end{tabular}
\end{center}
%
the destination file is determined by a pattern
depending on the current file:
To make this work, the current file must be called
`{\textit{prefix}\hspace{0.2em}\textit{suffix}}'
with \textit{prefix} matching precisely the argument.
Processing is then passed on to the file
`{\textit{dest}\hspace{0.2em}\textit{suffix}}'.
Surely, the same effect is achieved by
directly specifying the
argument `{\textit{dest}\hspace{0.2em}\textit{suffix}}'
in the first form.
However, that requires to set up a different file
for each child. With the alternative form of the command
all these files can have exactly the same content
which simplifies setting them up and maintaining them.

For example, the following file |draft.tex|
with a compilation flag |\version| as described in \secref{sec:flags}
compiles the main document as a draft:
%
\begin{center}
\begin{tabular}{l}
|\def\version{draft}|\\
|\input{childdoc.def}|\\
|\childdocforward{|\textit{main}|}|
\end{tabular}
\end{center}
%
Likewise, the following files |final|\textit{nn}|.tex|
compile the final version of the child document
|child|\textit{nn}|.tex|:
%
\begin{center}
\begin{tabular}{l}
|\def\version{final}|\\
|\input{childdoc.def}|\\
|\childdocforwardprefix{final}{child}|
\end{tabular}
\end{center}
%

Note that when several versions of a main file and/or of each child file
are to be generated, it may be convenient to set up a |Makefile| or
shell script to automatise the process.

%%%%%%%%%%%%%%%%%%%%%%%%%%%%%%%%%%%%%%%%%%%%%%%%%%%%%%%%%%%%%%%%%%%%%%%%%%%%%%%%
\subsection{Command Line Processing}
\label{sec:commandline}

The effect of redirection files can also be achieved by invoking
the \LaTeX{} compiler with a more elaborate command line.
Most conveniently this should be done as part
of a shell script or a |Makefile|.

When using \textsf{childdoc} in the main file, the following
command lines effectively perform a redirection
(note that depending on the shell being used,
backslashes may have to be doubled: `|\|' $\to$ `|\\|'):
%
\begin{center}
|... -jobname "|\textit{target}|" |\\|"|[\textit{flags}]%
|\input{childdoc.def}\childdocforward[|\textit{main}|]{|\textit{dest}|}"|
\end{center}
%
Here \textit{target} is the name of the output file,
\textit{main} is the name of the main file
and \textit{dest} is the name of the main or child file to be processed
(all filenames without extensions).
The optional argument \textit{main} can be omitted
if \textit{main} matches \textit{dest}.
Optionally, compilation \textit{flags} can be defined via |\def| commands.
This command line makes the \TeX{} engine believe
it is compiling the file \textit{target}
whose content is specified as the latter parameter.
The provided code then forwards the processing to
\textit{main} or \textit{dest} as described in \secref{sec:forward}.

%%%%%%%%%%%%%%%%%%%%%%%%%%%%%%%%%%%%%%%%%%%%%%%%%%%%%%%%%%%%%%%%%%%%%%%%%%%%%%%%
\subsection{Include by Input}
\label{sec:input}

Including child documents by |\include| has some restrictions by design.
Most notably, the content of a child document always occupies
its own set of pages; pages cannot be shared between child documents.
Usually, this behaviour makes perfect sense
because each child document contain an essential part of the document.
However, in some situations it may be desirable to compose
a document from a collection of parts
without having mandatory page breaks between then.
For this case, the package
provides a mechanism to include parts
by |\input| which can also be processed individually.
However, by construction this mechanism
requires manual handling of the content to be output.

%%%%%%%%%%%%%%%%%%%%%%%%%%%%%%%%%%%%%%%%
\DescribeMacro{\ifchilddocmanual}
The main file should be prepared as usual, see \secref{sec:include}.
However, the document body must make a distinction
between processing of an individual part and of the main document, e.g.:
%
\begin{center}
\begin{tabular}{l}
|\ifchilddocmanual|\\
|\input{\childdocname}|\\
|\||else|\\
\textit{document body with }|\input{|\textit{part}|}|\\
|\||fi|
\end{tabular}
\end{center}
%
The conditional |\ifchilddocmanual| is true whenever
a part to be included by |\input| is being compiled,
and the name of the part is stored in |\childdocname|.

%%%%%%%%%%%%%%%%%%%%%%%%%%%%%%%%%%%%%%%%
\DescribeMacro{\childdocby}
Each part to be included by |\input| should start with:
%
\begin{center}
\begin{tabular}{l}
|\input{childdoc.def}|\\
|\childdocby{|\textit{main}|}|\\
\end{tabular}
\end{center}
%
The directive |\childdocby| is similar to |\childdocof|
described in \secref{sec:include},
but the subsequent selection of content must be done manually.
To that end, both |\ifchilddoc| and |\ifchilddocmanual|
will be true upon processing of a part,
and the name of the part is stored in |\childdocname|.
Note that |\jobname| will be set to the filename of the current part
so that each part receives an individual |.aux| file
that does not interfere with the |.aux| file(s) of the main document.
This behaviour can be altered by the alternative form
|\childdocby[*]{|\textit{main}|}| (with a non-empty optional argument)
which uses the |.aux| file of the main document
by setting |\jobname| to \textit{main}.

%%%%%%%%%%%%%%%%%%%%%%%%%%%%%%%%%%%%%%%%%%%%%%%%%%%%%%%%%%%%%%%%%%%%%%%%%%%%%%%%
\subsection{Driver Development}
\label{sec:driver}

The \textsf{childdoc} mechanism can also be use for the development
of definition files such as \LaTeX{} styles or classes.
This case differs from the above setup with multiple parts
included by |\include| in that no |\includeonly| should be invoked.
This can be achieved by starting the include file
(before |\ProvidesPackage|) with:
%
\begin{center}
\begin{tabular}{l}
|\input{childdoc.def}|\\
|\childdocforward{|\textit{main}|}|\\
\end{tabular}
\end{center}
%
or alternatively with:
%
\begin{center}
\begin{tabular}{l}
|\input{childdoc.def}|\\
|\childdocby{|\textit{main}|}|\\
\end{tabular}
\end{center}
%
Both forms have slightly different effects as described above.
The main file is prepared as usual, see \secref{sec:include}.

%%%%%%%%%%%%%%%%%%%%%%%%%%%%%%%%%%%%%%%%%%%%%%%%%%%%%%%%%%%%%%%%%%%%%%%%%%%%%%%%
\subsection{Legacy Detection}
\label{sec:detection}

The directive |\childdocmain| in the main file can detect
whether the complete document or merely a child is to be compiled
even without using the directive |\childdocof|.
This method is deprecated because it is less robust
and there is no compelling reason to use it;
it is merely provided for backward compatibility
and it may be removed in future versions.

If the detection mechanism is to be used,
it is mandatory to correctly specify
the filename of the main file as the argument of |\childdocmain|:
%
\begin{center}
\begin{tabular}{l}
|\input{childdoc.def}|\\
|\childdocmain{|\textit{main}|}|\\
\end{tabular}
\end{center}
%
If |\jobname| does not match the argument \textit{main} of |\childdocmain|,
it is assumed that |\jobname| points to the child file to be compiled.
When using |\childdocmain| with the main file specified as argument,
it suffices to start a child file
with just |\input{|\textit{main}|}|
without loading of the package and using |\childdocof|.
If instead all processing is done
with the appropriate \textsf{childdoc} directives,
the argument of \textit{main} of |\childdocmain| can be empty.

An alternative version of the command line processing described
in \secref{sec:commandline} using the detection mechanism reads:
%
\begin{center}
|... -jobname "|\textit{target}|" "|[\textit{flags}]%
[|\def\jobname{|\textit{dest}|}|]|\input{|\textit{main}|}"|
\end{center}

%%%%%%%%%%%%%%%%%%%%%%%%%%%%%%%%%%%%%%%%%%%%%%%%%%%%%%%%%%%%%%%%%%%%%%%%%%%%%%%%
\subsection{Manual Code}
\label{sec:manual}

In case one cannot be certain whether the definitions file |childdoc.def|
is installed on the target \TeX{} distribution
and one prefers not to ship it,
it is conceivable to paste a few relevant commands into the sources.

To that end, drop all statements |\input{childdoc.def}|
and perform the replacements as outlined below.
Instead of |\childdocmain{|\textit{main}|}| add the following code
to the top of the main file:
%
\begin{center}
\begin{tabular}{l}
|\||ifdefined\childdocname\endinput\||fi\newif\ifchilddoc|\\
|\edef\childdocname{\scantokens\expandafter{\jobname\noexpand}}|\\
|\def\childdocmain{|\textit{main}|}\||ifx\childdocmain\childdocname\||else|\\
|\childdoctrue\includeonly{\childdocname}\let\jobname\childdocmain\||fi|\\
\end{tabular}
\end{center}
%
Instead of |\childdocof{|\textit{main}|}| just include the main file
at the top of each child file:
%
\begin{center}
|\input{|\textit{main}|}|
\end{center}
%
A simple redirection |\childdocforward{|\textit{dest}|}| is achieved by:
%
\begin{center}
|\def\jobname{|\textit{dest}|}\input{\jobname}|
\end{center}
%
The redirection with prefix
|\childdocforwardprefix[|\textit{prefix}|]{|\textit{dest}|}|
is accomplished by:
%
\begin{center}
\begin{tabular}{l}
|{\edef\jobname{\scantokens\expandafter{\jobname\noexpand}}|\\
|\def\redirectjob |\textit{prefix}|#1~~~{\gdef\jobname{|\textit{dest}|#1}}|\\
|\expandafter\redirectjob\jobname~~~}\input{\jobname}|
\end{tabular}
\end{center}

In an alternative approach,
child documents can be compiled by a specific command line
without additional code or specific definitions:
%
\begin{center}
|... -jobname "|\textit{target}|" "|[\textit{flags}]%
|\includeonly{|\textit{dest}|}\input{|\textit{main}|}"|
\end{center}
%

%%%%%%%%%%%%%%%%%%%%%%%%%%%%%%%%%%%%%%%%%%%%%%%%%%%%%%%%%%%%%%%%%%%%%%%%%%%%%%%%
%%%%%%%%%%%%%%%%%%%%%%%%%%%%%%%%%%%%%%%%%%%%%%%%%%%%%%%%%%%%%%%%%%%%%%%%%%%%%%%%
\section{Information}

%%%%%%%%%%%%%%%%%%%%%%%%%%%%%%%%%%%%%%%%%%%%%%%%%%%%%%%%%%%%%%%%%%%%%%%%%%%%%%%%
\subsection{Copyright}

Copyright \copyright{} 2017--2018 Niklas Beisert

This work may be distributed and/or modified under the
conditions of the \LaTeX{} Project Public License, either version 1.3
of this license or (at your option) any later version.
The latest version of this license is in
  \url{http://www.latex-project.org/lppl.txt}
and version 1.3 or later is part of all distributions of \LaTeX{}
version 2005/12/01 or later.

This work has the LPPL maintenance status `maintained'.

The Current Maintainer of this work is Niklas Beisert.

This work consists of the files |README.txt|, |childdoc.ins| and |childdoc.dtx|
as well as the derived files |childdoc.def|, |cdocsamp.tex|
with |cdocsch1.tex|, |cdocsch2.tex|, |cdocspt3.tex|, |cdocspt4.tex|,
|cdocsdrf.tex|, |cdocsfn1.tex|, |cdocsfn2.tex|
as well as |childdoc.pdf|.

%%%%%%%%%%%%%%%%%%%%%%%%%%%%%%%%%%%%%%%%%%%%%%%%%%%%%%%%%%%%%%%%%%%%%%%%%%%%%%%%
\subsection{Files and Installation}

The package consists of the files:
%
\begin{center}
\begin{tabular}{ll}
    |README.txt|   & readme file \\
    |childdoc.ins| & installation file \\
    |childdoc.dtx| & source file \\
    |childdoc.def| & definition file \\
    |cdocsamp.tex| & sample main file \\
    |cdocsch1.tex| & sample include file \\
    |cdocsch2.tex| & sample include file \\
    |cdocspt3.tex| & sample part file \\
    |cdocspt4.tex| & sample part file \\
    |cdocsdrf.tex| & sample redirection file \\
    |cdocsfn1.tex| & sample redirection file \\
    |cdocsfn2.tex| & sample redirection file \\
    |childdoc.pdf| & manual
\end{tabular}
\end{center}
%
The distribution consists of the files
|README.txt|, |childdoc.ins| and |childdoc.dtx|.
%
\begin{itemize}
\item
Run (pdf)\LaTeX{} on |childdoc.dtx|
to compile the manual |childdoc.pdf| (this file).
\item
Run \LaTeX{} on |childdoc.ins| to create the definitions file |childdoc.def|
and the sample |cdocsamp.tex| with include files
|cdocsch1.tex|, |cdocsch2.tex|, |cdocspt3.tex|, |cdocspt4.tex|,
|cdocsdrf.tex|, |cdocsfn1.tex|, |cdocsfn2.tex|.
Then copy the file |childdoc.def| to an appropriate directory of your \LaTeX{}
distribution, e.g.\ \textit{texmf-root}|/tex/latex/childdoc|.
\end{itemize}

%%%%%%%%%%%%%%%%%%%%%%%%%%%%%%%%%%%%%%%%%%%%%%%%%%%%%%%%%%%%%%%%%%%%%%%%%%%%%%%%
\subsection{Related CTAN Packages}

There are several other packages which offer a similar functionality:
%
\begin{itemize}
\item
The packages
\href{http://ctan.org/pkg/docmute}{\textsf{docmute}},
\href{http://ctan.org/pkg/includex}{\textsf{includex}} and
\href{http://ctan.org/pkg/standalone}{\textsf{standalone}}
provide commands to include only the document body of
a child file thus allowing both files to be compiled individually.
\item
The packages \href{http://ctan.org/pkg/subdocs}{\textsf{subdocs}}
and \href{http://ctan.org/pkg/subfiles}{\textsf{subfiles}}
provide structures in which the main and child documents can be
encapsulated and allowing them to be compiled individually.
The inclusion mechanism is different from the conventional |\include|.
\item
The package \href{http://ctan.org/pkg/combine}{\textsf{combine}}
is an elaborate solution to combine several documents into one.
\end{itemize}
%
See also the CTAN topic \href{http://ctan.org/topic/subdocs}{\textsf{subdocs}}
for further related packages.
The present package differs from the above solutions in that
a document structure constructed with the conventional |\include| mechanism
just needs two extra commands at the top of every file
such that all constituent files can be compiled individually.

%%%%%%%%%%%%%%%%%%%%%%%%%%%%%%%%%%%%%%%%%%%%%%%%%%%%%%%%%%%%%%%%%%%%%%%%%%%%%%%%
%\subsection{Feature Suggestions}
%
%The following is a list of features which may be useful for future
%versions of this package:
%%
%\begin{itemize}
%\item
%\ldots
%\end{itemize}

%%%%%%%%%%%%%%%%%%%%%%%%%%%%%%%%%%%%%%%%%%%%%%%%%%%%%%%%%%%%%%%%%%%%%%%%%%%%%%%%
\subsection{Revision History}

%%%%%%%%%%%%%%%%%%%%%%%%%%%%%%%%%%%%%%%%
\paragraph{v2.0:} 2018/12/30

\begin{itemize}
\item
immediate forward processing
\item
added |\childdocby| mechanism
\item
manual restructured
\end{itemize}

%%%%%%%%%%%%%%%%%%%%%%%%%%%%%%%%%%%%%%%%
\paragraph{v1.6:} 2018/01/17

\begin{itemize}
\item
application for development of include files
\item
corrections to manual
\end{itemize}

%%%%%%%%%%%%%%%%%%%%%%%%%%%%%%%%%%%%%%%%
\paragraph{v1.5:} 2017/05/21

\begin{itemize}
\item
more complete structuring introduced
\item
|\childdocof| introduced
\item
|\childdoc| renamed to |\childdocmain|
\item
|\childredirect| renamed to |\childdocforward| and |\childdocforwardprefix|
and functionality expanded
\end{itemize}

%%%%%%%%%%%%%%%%%%%%%%%%%%%%%%%%%%%%%%%%
\paragraph{v1.0:} 2017/04/27

\begin{itemize}
\item
manual and install package
\item
first version published on CTAN
\end{itemize}

%%%%%%%%%%%%%%%%%%%%%%%%%%%%%%%%%%%%%%%%
\paragraph{v0.6:} 2017/04/26

\begin{itemize}
\item
redirection mechanism added
\end{itemize}

%%%%%%%%%%%%%%%%%%%%%%%%%%%%%%%%%%%%%%%%
\paragraph{v0.5:} 2017/04/26

\begin{itemize}
\item
functionality in definition file
\end{itemize}


%%%%%%%%%%%%%%%%%%%%%%%%%%%%%%%%%%%%%%%%%%%%%%%%%%%%%%%%%%%%%%%%%%%%%%%%%%%%%%%%
%%%%%%%%%%%%%%%%%%%%%%%%%%%%%%%%%%%%%%%%%%%%%%%%%%%%%%%%%%%%%%%%%%%%%%%%%%%%%%%%
%%%%%%%%%%%%%%%%%%%%%%%%%%%%%%%%%%%%%%%%%%%%%%%%%%%%%%%%%%%%%%%%%%%%%%%%%%%%%%%%
\appendix

\settowidth\MacroIndent{\rmfamily\scriptsize 000\ }

 \DocInput{childdoc.dtx}

\end{document}
%</driver>
% \fi
%
% %%%%%%%%%%%%%%%%%%%%%%%%%%%%%%%%%%%%%%%%%%%%%%%%%%%%%%%%%%%%%%%%%%%%%%%%%%%%%%
% %%%%%%%%%%%%%%%%%%%%%%%%%%%%%%%%%%%%%%%%%%%%%%%%%%%%%%%%%%%%%%%%%%%%%%%%%%%%%%
% \section{Sample}
%\iffalse
%<*samplemain>
%\fi
%
% The following presents a sample document
% with two chapters, two parts, a title page,
% a compile flag as well as three forwarding files to set the flag.
% It consists of eight |.tex| files:
% \begin{center}
% \begin{tabular}{ll}
% |cdocsamp.tex|&main file\\
% |cdocsch1.tex|&include file for chapter 1\\
% |cdocsch2.tex|&include file for chapter 2\\
% |cdocspt3.tex|&include file for part 3\\
% |cdocspt4.tex|&include file for part 4\\
% |cdocsdrf.tex|&forwarding file for main file in draft mode\\
% |cdocsfi1.tex|&forwarding file for final version of chapter 1\\
% |cdocsfi2.tex|&forwarding file for final version of chapter 2\\
% \end{tabular}
% \end{center}
% Each of the eight files can be compiled directly by the \LaTeX{} compiler.
%
% %%%%%%%%%%%%%%%%%%%%%%%%%%%%%%%%%%%%%%
% \paragraph{Main File.}
%
% The main file is called |cdocsamp.tex|.
%
% Load the \textsf{childdoc} definitions and
% declare the filename for the main document:
%    \begin{macrocode}
\input{childdoc.def}
\childdocmain{}
%    \end{macrocode}

% Optional override for |\version| flag:
%    \begin{macrocode}
%%\ifchilddoc\else\providecommand{\version}{draft}\fi
%    \end{macrocode}

% Define the default values for the |\version| flag
% (|final| for the main file and |draft| for childs):
%    \begin{macrocode}
\ifchilddoc
\providecommand{\version}{draft}
\else
\providecommand{\version}{final}
\fi
%    \end{macrocode}

% Load the standard document class:
%    \begin{macrocode}
\documentclass[12pt]{article}
%    \end{macrocode}

% Start the document body:
%    \begin{macrocode}
\begin{document}
%    \end{macrocode}

% Declare a title page.
% Print title, part of document being processed and version flag:
%    \begin{macrocode}
\addtocounter{page}{-1}
\begin{center}
{\LARGE\bfseries{}childdoc example\par}
\vspace{1cm}
\ifchilddoc
\ifchilddocmanual part\else chapter\fi:
`\childdocname' of `\childdocjob'\par
\else
main document: `\childdocjob'\par
\fi
version: \version\par
\end{center}
\newpage
%    \end{macrocode}

% Manually include selected file,
% otherwise process as usual:
%    \begin{macrocode}
\ifchilddocmanual
\section*{part `\childdocname'}
\input{\childdocname}
\else
%    \end{macrocode}

% Include the two chapters:
%    \begin{macrocode}
\include{cdocsch1}
\include{cdocsch2}
%    \end{macrocode}

% Include the two parts unless only chapters should be displayed:
%    \begin{macrocode}
\ifchilddoc\else
\section{part three}
\input{cdocspt3}
\section{part four}
\input{cdocspt4}
\fi
%    \end{macrocode}

% Process as usual until here:
%    \begin{macrocode}
\fi
%    \end{macrocode}

% End of document body:
%    \begin{macrocode}
\end{document}
%    \end{macrocode}
%\iffalse
%</samplemain>
%\fi
%
% %%%%%%%%%%%%%%%%%%%%%%%%%%%%%%%%%%%%%%
% \paragraph{Chapter Include Files.}
%
% The include files are called |cdocsch1.tex| and |cdocsch2.tex|.
%
%\iffalse
%<*samplechap1|samplechap2>
%\fi

% Optional override for |\version| flag:
%    \begin{macrocode}
%%\providecommand{\version}{final}
%    \end{macrocode}

% Include the main document:
%    \begin{macrocode}
\input{childdoc.def}
\childdocof{cdocsamp}
%    \end{macrocode}

%\iffalse
%</samplechap1|samplechap2>
%\fi
%
%\iffalse
%<*samplechap1>
%\fi
% Some text for chapter 1:
%    \begin{macrocode}
\section{one}
some text in chapter one
%    \end{macrocode}

%\iffalse
%</samplechap1>
%\fi
% Some text for chapter 2:
%\iffalse
%<*samplechap2>
%\fi
%    \begin{macrocode}
\section{two}
more text in chapter two
%    \end{macrocode}

%\iffalse
%</samplechap2>
%\fi
%
% %%%%%%%%%%%%%%%%%%%%%%%%%%%%%%%%%%%%%%
% \paragraph{Part Include Files.}
%
% The include files are called |cdocspt3.tex| and |cdocspt4.tex|.
%
%\iffalse
%<*samplepart3|samplepart4>
%\fi

% Optional override for |\version| flag:
%    \begin{macrocode}
%%\providecommand{\version}{final}
%    \end{macrocode}

% Include the main document:
%    \begin{macrocode}
\input{childdoc.def}
\childdocby{cdocsamp}
%    \end{macrocode}

%\iffalse
%</samplepart3|samplepart4>
%\fi
%
%\iffalse
%<*samplepart3>
%\fi
% Some text for part 3:
%    \begin{macrocode}
some text in part three
%    \end{macrocode}

%\iffalse
%</samplepart3>
%\fi
% Some text for part 4:
%\iffalse
%<*samplepart4>
%\fi
%    \begin{macrocode}
more text in part four
%    \end{macrocode}

%\iffalse
%</samplepart4>
%\fi
%
% %%%%%%%%%%%%%%%%%%%%%%%%%%%%%%%%%%%%%%
% \paragraph{Forwarding for a Complete Draft.}
%
% The following forwarding file |cdocsdrf.tex|
% compiles the main document in draft mode:
%\iffalse
%<*sampledraft>
%\fi
%    \begin{macrocode}
\def\version{draft}
\input{childdoc.def}
\childdocforward{cdocsamp}
%    \end{macrocode}

%\iffalse
%</sampledraft>
%\fi
%
% %%%%%%%%%%%%%%%%%%%%%%%%%%%%%%%%%%%%%%
% \paragraph{Forwarding for Final Version of the Chapters.}
%
% The following forwarding files |cdocsfn1.tex| and |cdocsfn2.tex|
% (with identical content)
% compile the final versions of the child documents
% |cdocsch1.tex| and |cdocsch2.tex|, respectively:
%\iffalse
%<*samplefinal>
%\fi
%    \begin{macrocode}
\def\version{final}
\input{childdoc.def}
\childdocforwardprefix[cdocsamp]{cdocsfn}{cdocsch}
%    \end{macrocode}

%\iffalse
%</samplefinal>
%\fi
%
% %%%%%%%%%%%%%%%%%%%%%%%%%%%%%%%%%%%%%%
% \paragraph{Command Line Processing.}
%
% The following three command lines generate the output files
% |cdocscld|, |cdocscl1| and |cdocscl2|
% which should be identical to
% |cdocsdrf|, |cdocsch1| and |cdocsfn2|, respectively:
% \begin{center}
% \begin{tabular}{l}
% |latex -jobname cdocscld \|\\
% |  "\def\version{draft}\input{childdoc.def}\childdocforward{cdocsamp}"|\\
% |latex -jobname cdocscl1 \|\\
% |  "\input{childdoc.def}\childdocforward[cdocsamp]{cdocsch1}"|\\
% |latex -jobname cdocscl2 \|\\
% |  "\def\version{final}\input{childdoc.def}\childdocforward{cdocsch2}"|
% \end{tabular}
% \end{center}
% Note that the trailing backslash on each first line
% merely continues the input to the second line
% (for convenient cut ant paste).
% Furthermore, the command |latex| can be replaced by any
% of its alternative versions such as |pdflatex|.
%
% %%%%%%%%%%%%%%%%%%%%%%%%%%%%%%%%%%%%%%%%%%%%%%%%%%%%%%%%%%%%%%%%%%%%%%%%%%%%%%
% %%%%%%%%%%%%%%%%%%%%%%%%%%%%%%%%%%%%%%%%%%%%%%%%%%%%%%%%%%%%%%%%%%%%%%%%%%%%%%
% \section{Implementation}
%\iffalse
%<*package>
%\fi
%
% This section describes the definitions file |childdoc.def|.

% The definitions cannot be loaded using |\usepackage| or |\RequirePackage|
% which has a mechanism to prevent loading a style file more than once.
% When loading the definitions by means of |\input|
% multiple instances have to be prevented manually:
%\iffalse
%This code needs to be before the `\ProvidesFile' directive
%which is defined at the beginning of this file.
%Therefore it is also placed there and commented out here.
%</package>
%<*discard>
%\fi
%    \begin{macrocode}
\ifdefined\childdocmain\endinput\fi
%    \end{macrocode}
%\iffalse
%</discard>
%<*package>
%\fi
%
% \macro{\ifchilddoc}
% \macro{\ifchilddocmanual}
% The conditional |\ifchilddoc| tells whether a
% child (true) or main (false) document is being compiled.
% The conditional |\ifchilddocmanual| tells whether
% the |\includeonly| mechanism is used (false) or
% the selection of child files must be performed manually (true).
% The definitions initialise to false:
%    \begin{macrocode}
\newif\ifchilddoc
\newif\ifchilddocmanual
%    \end{macrocode}

% \macro{\childdocname}
% \macro{\childdocjob}
% The macro |\childdocname| stores the name of the main document
% to be compiled. The macro |\childdocjob| stores the name of
% the document on which the \LaTeX{} compiler was originally invoked.
% The content of |\jobname| cannot be compared
% to filenames specified in the source due to different catcodes.
% The following code rescans |\jobname|, stores the result
% in |\childdocname| and saves a copy in |\childdocjob|:
%    \begin{macrocode}
\edef\childdocname{\scantokens\expandafter{\jobname\noexpand}}
\let\childdocjob\childdocname
%    \end{macrocode}

% \macro{\childdocdisable}
% The macro |\childdocdisable| prevents the main file
% from being processed more than once.
% At this stage, the main document command |\childdocmain|
% is assumed to be called once again where it should do nothing.
% Any subsequent call to it should prevent
% a secondary processing of the main document
% It overwrites the forwarding commands
% |\childdocof| and |\childdocforward|
% with empty macros to prevent further inclusions of the main document:
%    \begin{macrocode}
\newcommand{\childdocdisable}
{
  \renewcommand{\childdocmain}[1]{\renewcommand{\childdocmain}[1]{\endinput}}
  \renewcommand{\childdocof}[1]{}
  \renewcommand{\childdocby}[2][]{}
  \renewcommand{\childdocforward}[2][]{}
  \renewcommand{\childdocdisable}{}
}
%    \end{macrocode}

% \macro{\childdocmain}
% The macro |\childdocmain| is to be called at the top of the main file
% with nothing or the main filename (without extension) as argument.
% First, it breaks loops.
% If the argument is not empty and does not match |\childdocname|
% (which is set by the first inclusion of |childdoc.def|),
% |\ifchilddoc| is set to true, |\includeonly| is applied to the child file
% and |\jobname| is set to the main file
% (for proper handling of |.aux| files):
%    \begin{macrocode}
\newcommand{\childdocmain}[1]
{
  \childdocdisable\childdocmain{}
  \if?#1?\else
    \begingroup
      \def\childdoctmp{#1}
      \ifx\childdoctmp\childdocname
        \def\childdoctmp{}
      \else
        \def\childdoctmp
        {
          \childdoctrue
          \includeonly{\childdocname}
          \def\childdocjob{#1}
          \def\jobname{#1}
        }
      \fi
      \expandafter
    \endgroup
    \childdoctmp
  \fi
}
%    \end{macrocode}

% \macro{\childdocof}
% The command |\childdocof| redirects
% compilation to the main file |#1|.
%    \begin{macrocode}
\newcommand{\childdocof}[1]
{
  \childdocdisable
  \childdoctrue
  \includeonly{\childdocname}
  \def\jobname{#1}
  \def\childdocjob{#1}
  \input{#1}
}
%    \end{macrocode}

% \macro{\childdocby}
% The command |\childdocby| ....
%    \begin{macrocode}
\newcommand{\childdocby}[2][]
{
  \childdocdisable
  \childdoctrue
  \childdocmanualtrue
  \if?#1?\else
    \def\jobname{#2}
  \fi
  \def\childdocjob{#2}
  \input{#2}
  \endinput
}
%    \end{macrocode}

% \macro{\childdocforward}
% The command |\childdocforward| redirects
% compilation to the main file or
% (if the optional argument is given) a child file.
% Parameters are set as if the main file
% or a child file starting with |\childdocof| was compiled.
% Then compilation is handed over to the main file:
%    \begin{macrocode}
\newcommand{\childdocforward}[2][]
{
  \begingroup
    \if?#1?
      \def\childdoctmp
      {
        \def\childdocname{#2}
        \def\childdocjob{#2}
        \def\jobname{#2}
        \input{#2}
        \endinput
      }
    \else
      \def\childdoctmp
      {
        \childdocdisable
        \def\childdocname{#2}
        \childdoctrue
        \includeonly{#2}
        \def\childdocjob{#1}
        \def\jobname{#1}
        \input{#1}
        \endinput
      }
    \fi
    \expandafter
  \endgroup
  \childdoctmp
}
%    \end{macrocode}

% \macro{\childdocforwardprefix}
% The command |\childdocforwardprefix| redirects
% compilation to the main or a child file by means of a pattern.
% The prefix |#1| in the current filename is replaced by |#2|
% and the suffix of the current filename is kept
% (it is assumed that the filename does not contain the substring `|~~~|'
% which is used as a delimiter).
% Compilation is handed over to the new file by |\childdocforward|:
%    \begin{macrocode}
\newcommand{\childdocforwardprefix}[3][]
{
  \begingroup
    \def\childdocextract #2##1~~~{\def\childdoctmp{\childdocforward[#1]{#3##1}}}
    \expandafter\childdocextract\childdocname~~~
    \expandafter
  \endgroup
  \childdoctmp
}
%    \end{macrocode}

% \macro{\childdoc}
% The deprecated macro |\childdoc| is a legacy version of |\childdocmain|:
%    \begin{macrocode}
\newcommand{\childdoc}{\childdocmain}
%    \end{macrocode}

% \macro{\childdocredirect}
% The deprecated macro |\childdocredirect| is a legacy version
% of |\childdocforward| and |\childdocforwardprefix|:
%    \begin{macrocode}
\newcommand{\childdocredirect}[2][]
{
  \begingroup
    \if?#1?
      \def\childdoctmp{\childdocforward{#2}}
    \else
      \def\childdoctmp{\childdocforwardprefix{#1}{#2}}
    \fi
    \expandafter
  \endgroup
  \childdoctmp
}
%    \end{macrocode}

%\iffalse
%</package>
%\fi
%
\endinput

\childdocmain{}
%    \end{macrocode}

% Optional override for |\version| flag:
%    \begin{macrocode}
%%\ifchilddoc\else\providecommand{\version}{draft}\fi
%    \end{macrocode}

% Define the default values for the |\version| flag
% (|final| for the main file and |draft| for childs):
%    \begin{macrocode}
\ifchilddoc
\providecommand{\version}{draft}
\else
\providecommand{\version}{final}
\fi
%    \end{macrocode}

% Load the standard document class:
%    \begin{macrocode}
\documentclass[12pt]{article}
%    \end{macrocode}

% Start the document body:
%    \begin{macrocode}
\begin{document}
%    \end{macrocode}

% Declare a title page.
% Print title, part of document being processed and version flag:
%    \begin{macrocode}
\addtocounter{page}{-1}
\begin{center}
{\LARGE\bfseries{}childdoc example\par}
\vspace{1cm}
\ifchilddoc
\ifchilddocmanual part\else chapter\fi:
`\childdocname' of `\childdocjob'\par
\else
main document: `\childdocjob'\par
\fi
version: \version\par
\end{center}
\newpage
%    \end{macrocode}

% Manually include selected file,
% otherwise process as usual:
%    \begin{macrocode}
\ifchilddocmanual
\section*{part `\childdocname'}
\input{\childdocname}
\else
%    \end{macrocode}

% Include the two chapters:
%    \begin{macrocode}
\include{cdocsch1}
\include{cdocsch2}
%    \end{macrocode}

% Include the two parts unless only chapters should be displayed:
%    \begin{macrocode}
\ifchilddoc\else
\section{part three}
\input{cdocspt3}
\section{part four}
\input{cdocspt4}
\fi
%    \end{macrocode}

% Process as usual until here:
%    \begin{macrocode}
\fi
%    \end{macrocode}

% End of document body:
%    \begin{macrocode}
\end{document}
%    \end{macrocode}
%\iffalse
%</samplemain>
%\fi
%
% %%%%%%%%%%%%%%%%%%%%%%%%%%%%%%%%%%%%%%
% \paragraph{Chapter Include Files.}
%
% The include files are called |cdocsch1.tex| and |cdocsch2.tex|.
%
%\iffalse
%<*samplechap1|samplechap2>
%\fi

% Optional override for |\version| flag:
%    \begin{macrocode}
%%\providecommand{\version}{final}
%    \end{macrocode}

% Include the main document:
%    \begin{macrocode}
% \iffalse
%
% childdoc.dtx Copyright (C) 2017-2018 Niklas Beisert
%
% This work may be distributed and/or modified under the
% conditions of the LaTeX Project Public License, either version 1.3
% of this license or (at your option) any later version.
% The latest version of this license is in
%   http://www.latex-project.org/lppl.txt
% and version 1.3 or later is part of all distributions of LaTeX
% version 2005/12/01 or later.
%
% This work has the LPPL maintenance status `maintained'.
%
% The Current Maintainer of this work is Niklas Beisert.
%
% This work consists of the files childdoc.dtx and childdoc.ins
% and the derived files childdoc.def and cdocsamp.tex with
% cdocsch1.tex, cdocsch2.tex, cdocsdrf.tex, cdocsfn1.tex, cdocsfn2.tex.
%
%<package>\ifdefined\childdocmain\endinput\fi
%<package>\ProvidesFile{childdoc.def}[2018/12/30 v2.0 child document driver]
%<samplemain>\ProvidesFile{cdocsamp.tex}[2018/12/30 v2.0 sample for childdoc]
%<*driver>
%\ProvidesFile{childdoc.drv}[2018/12/30 v2.0 childdoc reference manual file]
\PassOptionsToClass{10pt,a4paper}{article}
\documentclass{ltxdoc}

\usepackage[margin=35mm]{geometry}
\usepackage{hyperref}
\usepackage{hyperxmp}
\usepackage[usenames]{color}

\hypersetup{colorlinks=true}
\hypersetup{pdfstartview=FitH}
\hypersetup{pdfpagemode=UseNone}
\hypersetup{pdfsource={}}
\hypersetup{pdflang={en-UK}}
\hypersetup{pdfcopyright={Copyright 2017-2018 Niklas Beisert.
  This work may be distributed and/or modified under the
  conditions of the LaTeX Project Public License, either version 1.3
  of this license or (at your option) any later version.}}
\hypersetup{pdflicenseurl={http://www.latex-project.org/lppl.txt}}
\hypersetup{pdfcontactaddress={ETH Zurich, ITP, HIT K,
  Wolfgang-Pauli-Strasse 27}}
\hypersetup{pdfcontactpostcode={8093}}
\hypersetup{pdfcontactcity={Zurich}}
\hypersetup{pdfcontactcountry={Switzerland}}
\hypersetup{pdfcontactemail={nbeisert@itp.phys.ethz.ch}}
\hypersetup{pdfcontacturl={http://people.phys.ethz.ch/\xmptilde nbeisert/}}

\newcommand{\secref}[1]{\hyperref[#1]{section \ref*{#1}}}

\parskip1ex
\parindent0pt
\let\olditemize\itemize
\def\itemize{\olditemize\parskip0pt}

\begin{document}

\title{The \textsf{childdoc} Package}
\hypersetup{pdftitle={The childdoc Package}}
\author{Niklas Beisert\\[2ex]
  Institut f\"ur Theoretische Physik\\
  Eidgen\"ossische Technische Hochschule Z\"urich\\
  Wolfgang-Pauli-Strasse 27, 8093 Z\"urich, Switzerland\\[1ex]
  \href{mailto:nbeisert@itp.phys.ethz.ch}
  {\texttt{nbeisert@itp.phys.ethz.ch}}}
\hypersetup{pdfauthor={Niklas Beisert}}
\hypersetup{pdfsubject={Manual for the LaTeX2e Package childdoc}}
\date{30 December 2018, \textsf{v2.0}}
\maketitle

\begin{abstract}\noindent
\textsf{childdoc} is a \LaTeXe{} package
that enables the direct compilation
of document sections included by |\include|
to individual files.
\end{abstract}

\begingroup
\parskip0ex
\tableofcontents
\endgroup

%%%%%%%%%%%%%%%%%%%%%%%%%%%%%%%%%%%%%%%%%%%%%%%%%%%%%%%%%%%%%%%%%%%%%%%%%%%%%%%%
%%%%%%%%%%%%%%%%%%%%%%%%%%%%%%%%%%%%%%%%%%%%%%%%%%%%%%%%%%%%%%%%%%%%%%%%%%%%%%%%
\section{Introduction}

\LaTeX{} provides a mechanism to structure a large document (such as a book)
into a main file and several child files (containing the chapters)
using the |\include| command.
This mechanism is beneficial for documents
which span hundreds of pages in order to
make the source file(s) more manageable.
Moreover, compilation can be restricted to
selected child files by means of the |\includeonly| command.
The latter feature can be used to reduce the compilation time while editing
(this was significantly more useful in the earlier days of \LaTeX{})
or to generate a smaller document which is easier to navigate.
Another application of |\includeonly| is to generate
documents consisting of selected parts of the complete document.

However, there are a few drawbacks of the plain |\include| mechanism:
\begin{itemize}
\item
The child files cannot be compiled on their own,
they can only be compiled via the main file.
A naive editing environment
(such as a text editor with an option
to have the current file processed by \LaTeX)
may require one to switch to the main file before compiling;
attempting to compile the child file produces errors.
\item
The main file must be modified (each time)
to adjust the |\includeonly| command
to the present needs. This easily leaves the main file in a messy state.
\item
The generated document will always carry the filename
of the main document. This is inconvenient if
several child files are to be compiled and
to be kept for distribution.
\end{itemize}

The present package provides a simple interface
to make child files individually compilable by \LaTeX{}.
Compiling a child file then has the same effect as compiling
the main file with an |\includeonly| command
to select the appropriate child.
Moreover the generated document will carry the name of the child
rather than the main file.
This resolves all three above issues.

This feature is meant to make the editing of books,
thesis documents and lecture notes somewhat more convenient.
However, the package can also be used efficiently for
composing a series of documents (such as exercise sheets)
which are typically distributed individually.
It then assists the author in generating the individual documents
(potentially in different versions)
as well as a document containing the collected series.
Another application is in developing style files
or other kinds of included material
where compilation of the style file could redirect
to a sample or test file.

%%%%%%%%%%%%%%%%%%%%%%%%%%%%%%%%%%%%%%%%%%%%%%%%%%%%%%%%%%%%%%%%%%%%%%%%%%%%%%%%
%%%%%%%%%%%%%%%%%%%%%%%%%%%%%%%%%%%%%%%%%%%%%%%%%%%%%%%%%%%%%%%%%%%%%%%%%%%%%%%%
\section{Usage}

First of all, the package \textsf{childdoc} is \emph{not} a standard
\LaTeXe{} |.sty| style file! Therefore it needs to be invoked in
a non-standard way.

%%%%%%%%%%%%%%%%%%%%%%%%%%%%%%%%%%%%%%%%%%%%%%%%%%%%%%%%%%%%%%%%%%%%%%%%%%%%%%%%
\subsection{Included Files}
\label{sec:include}

%%%%%%%%%%%%%%%%%%%%%%%%%%%%%%%%%%%%%%%%
\DescribeMacro{\childdocmain}
To use the package, add the commands
\begin{center}
\begin{tabular}{l}
|\input{childdoc.def}|\\
|\childdocmain{}|\\
\end{tabular}
\end{center}
at the very top of the main \LaTeX{} file,
in particular \emph{before} the |\documentclass| statement!
The argument of |\childdocmain| should be left empty
(but it must be present).

%%%%%%%%%%%%%%%%%%%%%%%%%%%%%%%%%%%%%%%%
\DescribeMacro{\childdocof}
Furthermore, add the commands
\begin{center}
\begin{tabular}{l}
|\input{childdoc.def}|\\
|\childdocof{|\textit{main}|}|\\
\end{tabular}
\end{center}
at the top of every child file \textit{child}
which is included by |\include{|\textit{child}|}|
from within the main file
(or at least for those files to be compiled individually).
The argument \textit{main} must be the filename of the main file.

There are a couple of
considerations in setting up the main and child documents:

%%%%%%%%%%%%%%%%%%%%%%%%%%%%%%%%%%%%%%%%
\paragraph{Restrictions.}

Please note the following restrictions:
\begin{itemize}
\item
|\childdocmain| must be called with one argument \textit{main}
to ensure compatibility with earlier version of the package.
It must either be empty (|\childdocmain{}|)
or precisely match the filename of the main file in which it is specified.
See \secref{sec:detection} for further information.
\item
The filename \textit{main} must be specified without the |.tex| extension.
\item
The filename \textit{main} is case sensitive
(even in case-insensitive file systems)
due to internal string comparison.
\item
The argument \textit{main} should be fully expanded, it cannot be a macro.
\item
Subdirectories and special characters should be avoided in filenames.
\item
The command |\childdocmain{|\textit{main}|}| must be followed by a whitespace.
It should not be followed immediately by another command
or by a comment mark `|%|'.
This is because the \TeX{} parser reads the token immediately following
the argument of |\childdocmain| and puts it
at the beginning of every child section;
however, a white\-space is ignored.
\end{itemize}

%%%%%%%%%%%%%%%%%%%%%%%%%%%%%%%%%%%%%%%%
\paragraph{Content of Main File.}

It is advisable to place all content in the child files included by |\include|.
Any output contained in the main file will appear in all child documents
unless suppressed manually;
it cannot be suppressed automatically by the |\includeonly| directive
and thus should normally be avoided.
A method to include some content in the main file
by means of conditional processing is described in \secref{sec:conditional}.

%%%%%%%%%%%%%%%%%%%%%%%%%%%%%%%%%%%%%%%%
\paragraph{Page Numbering.}

When only a part of the document is compiled,
the appropriate numbering of pages
(as well as other status parameters)
is determined from the |.aux| files.
The latter contain information from previous passes.
However this information needs to propagate through
all intermediate child documents.
Therefore the page numbering in child documents may well
be inconsistent until the complete document is compiled at least once.

A useful (if unconventional) way to always ensure a consistent
page numbering is to restart the numbering in each child document
and denote the pages by `\textit{child}|.|\textit{page}'
where \textit{child} represents the chapter/section number of the child file.
This can be achieved by the command
|\numberwithin{page}{|\textit{child}|}|
of the \textsf{amsmath} package
where \textit{child} can be |chapter| or |section|
depending on the chosen structuring.
Alternatively, one can modify the macro |\thepage| appropriately
and reset the counter |page| at the start of each child file.

%%%%%%%%%%%%%%%%%%%%%%%%%%%%%%%%%%%%%%%%%%%%%%%%%%%%%%%%%%%%%%%%%%%%%%%%%%%%%%%%
\subsection{Conditional Processing}
\label{sec:conditional}

The package provides a mechanism to compile different versions
of a document. To customise the versions further some conditional processing
can come in handy to distinguish which version is being compiled.
The package provides two macros to describe the compilation context:

%%%%%%%%%%%%%%%%%%%%%%%%%%%%%%%%%%%%%%%%
\DescribeMacro{\ifchilddoc}
The conditional |\ifchilddoc| distinguishes between the compilation of
child documents and the main document:
%
\begin{center}
|\ifchilddoc |\textit{child-code}| |[|\||else |\textit{main-code}]| \||fi|
\end{center}

%%%%%%%%%%%%%%%%%%%%%%%%%%%%%%%%%%%%%%%%
\DescribeMacro{\childdocname}
\DescribeMacro{\childdocjob}
The macro |\childdocname| contains the filename (without extension)
of the main or child file being processed.
Note that |\childdocjob| will always contain the name of the main file.

%%%%%%%%%%%%%%%%%%%%%%%%%%%%%%%%%%%%%%%%
\paragraph{Title Page.}

Conditional processing can be used to include a title or banner page
in the main document when proper precautions are taken.
Importantly, the code in the main file should ensure that the page counter
(as well as other status parameters which are stored in the |.aux| files)
takes the same value after the conditional processing.
Otherwise the page numbers may take divergent values
depending on which part is compiled.

For example, a title page could be declared by:
%
\begin{center}
\begin{tabular}{l}
|\ifchilddoc\||else|\\
|\addtocounter{page}{-1}|\\
\textit{code for title page}\\
|\newpage|\\
|\||fi|
\end{tabular}
\end{center}
%
A banner page for the child documents can be generated by:
%
\begin{center}
\begin{tabular}{l}
|\ifchilddoc|\\
|\addtocounter{page}{-1}|\\
\textit{code for banner page}\\
|\newpage|\\
|\||fi|
\end{tabular}
\end{center}
%
Here one could write a message such as:
\begin{center}
|This is the part \childdocname{} of \childdocjob{}.|
\end{center}

%%%%%%%%%%%%%%%%%%%%%%%%%%%%%%%%%%%%%%%%%%%%%%%%%%%%%%%%%%%%%%%%%%%%%%%%%%%%%%%%
\subsection{Flags}
\label{sec:flags}

The package makes it easy to generate different versions
of the main or child documents.
To this end compilation flags can be defined
and assigned different default values.
They will be particularly useful in conjunction
with the forwarding mechanism described in \secref{sec:forward}.

For example, it may be useful to have a flag |\version|
which can be set to |draft| or |final|.
The document source will contain some conditional code
depending on the value of |\version|.
Suppose further, the flag should default to |final| for the main file
and to |draft| for child files
which is a natural assignment for editing the document.
This is achieved by placing the following code
in the preamble of the main document
(below the |\childdocmain| directive):
%
\begin{center}
\begin{tabular}{l}
|\ifchilddoc|\\
|\providecommand{\version}{draft}|\\
|\||else|\\
|\providecommand{\version}{final}|\\
|\||fi|
\end{tabular}
\end{center}
%
The definition by |\providecommand| makes sure
that previous definitions are not overwritten.
Further statements |\providecommand{\version}{...}|
can thus be added before the above code to override it.

For the main file, one might add a line
(between |\childdocmain| and the above block)
%
\begin{center}
|%\ifchilddoc\||else\providecommand{\version}{draft}\||fi|
\end{center}
%
which can be uncommented to produce a draft version.
Likewise one can add a line to the very top of a child file
(above the |\childdocof{|\textit{main}|}| directive)
%
\begin{center}
|%\providecommand{\version}{final}|
\end{center}
%
which can be uncommented to produce the final version of this child document.

%%%%%%%%%%%%%%%%%%%%%%%%%%%%%%%%%%%%%%%%%%%%%%%%%%%%%%%%%%%%%%%%%%%%%%%%%%%%%%%%
\subsection{Forwarding}
\label{sec:forward}

Different versions of the main or child documents
using compilation flags as described in \secref{sec:flags}
can be (permanently) stored in different files
for convenient compilation, viewing and distribution.
To this end, the package defines a command
to pass on compilation to a different file:

%%%%%%%%%%%%%%%%%%%%%%%%%%%%%%%%%%%%%%%%
\DescribeMacro{\childdocforward}
The command |\childdocforward| redirects processing to
another source file:
%
\begin{center}
\begin{tabular}{l}
|\input{childdoc.def}|\\
|\childdocforward[|\textit{main}|]{|\textit{dest}|}|\\
\end{tabular}
\end{center}
%
The argument \textit{dest} is the destination file
(without extension).
It should be the main file or one of the child files.
Note that further \textsf{childdoc} directives
such as |\childdocof| and |\childdocforward|
in the indicated file will be processed in this form.
The optional argument \textit{main}
passes on directly to the main file \textit{main}
while pretending to compile the child \textit{dest}.
This form behaves as if \textit{dest}
issues |\childdocof{|\textit{main}|}| right away,
and no further \textsf{childdoc} directives will be processed.

%%%%%%%%%%%%%%%%%%%%%%%%%%%%%%%%%%%%%%%%
\DescribeMacro{\...prefix}
In the alternative form |\childdocforwardprefix|,
%
\begin{center}
\begin{tabular}{l}
|\input{childdoc.def}|\\
|\childdocforwardprefix[|\textit{main}|]{|\textit{prefix}|}{|\textit{dest}|}|
\end{tabular}
\end{center}
%
the destination file is determined by a pattern
depending on the current file:
To make this work, the current file must be called
`{\textit{prefix}\hspace{0.2em}\textit{suffix}}'
with \textit{prefix} matching precisely the argument.
Processing is then passed on to the file
`{\textit{dest}\hspace{0.2em}\textit{suffix}}'.
Surely, the same effect is achieved by
directly specifying the
argument `{\textit{dest}\hspace{0.2em}\textit{suffix}}'
in the first form.
However, that requires to set up a different file
for each child. With the alternative form of the command
all these files can have exactly the same content
which simplifies setting them up and maintaining them.

For example, the following file |draft.tex|
with a compilation flag |\version| as described in \secref{sec:flags}
compiles the main document as a draft:
%
\begin{center}
\begin{tabular}{l}
|\def\version{draft}|\\
|\input{childdoc.def}|\\
|\childdocforward{|\textit{main}|}|
\end{tabular}
\end{center}
%
Likewise, the following files |final|\textit{nn}|.tex|
compile the final version of the child document
|child|\textit{nn}|.tex|:
%
\begin{center}
\begin{tabular}{l}
|\def\version{final}|\\
|\input{childdoc.def}|\\
|\childdocforwardprefix{final}{child}|
\end{tabular}
\end{center}
%

Note that when several versions of a main file and/or of each child file
are to be generated, it may be convenient to set up a |Makefile| or
shell script to automatise the process.

%%%%%%%%%%%%%%%%%%%%%%%%%%%%%%%%%%%%%%%%%%%%%%%%%%%%%%%%%%%%%%%%%%%%%%%%%%%%%%%%
\subsection{Command Line Processing}
\label{sec:commandline}

The effect of redirection files can also be achieved by invoking
the \LaTeX{} compiler with a more elaborate command line.
Most conveniently this should be done as part
of a shell script or a |Makefile|.

When using \textsf{childdoc} in the main file, the following
command lines effectively perform a redirection
(note that depending on the shell being used,
backslashes may have to be doubled: `|\|' $\to$ `|\\|'):
%
\begin{center}
|... -jobname "|\textit{target}|" |\\|"|[\textit{flags}]%
|\input{childdoc.def}\childdocforward[|\textit{main}|]{|\textit{dest}|}"|
\end{center}
%
Here \textit{target} is the name of the output file,
\textit{main} is the name of the main file
and \textit{dest} is the name of the main or child file to be processed
(all filenames without extensions).
The optional argument \textit{main} can be omitted
if \textit{main} matches \textit{dest}.
Optionally, compilation \textit{flags} can be defined via |\def| commands.
This command line makes the \TeX{} engine believe
it is compiling the file \textit{target}
whose content is specified as the latter parameter.
The provided code then forwards the processing to
\textit{main} or \textit{dest} as described in \secref{sec:forward}.

%%%%%%%%%%%%%%%%%%%%%%%%%%%%%%%%%%%%%%%%%%%%%%%%%%%%%%%%%%%%%%%%%%%%%%%%%%%%%%%%
\subsection{Include by Input}
\label{sec:input}

Including child documents by |\include| has some restrictions by design.
Most notably, the content of a child document always occupies
its own set of pages; pages cannot be shared between child documents.
Usually, this behaviour makes perfect sense
because each child document contain an essential part of the document.
However, in some situations it may be desirable to compose
a document from a collection of parts
without having mandatory page breaks between then.
For this case, the package
provides a mechanism to include parts
by |\input| which can also be processed individually.
However, by construction this mechanism
requires manual handling of the content to be output.

%%%%%%%%%%%%%%%%%%%%%%%%%%%%%%%%%%%%%%%%
\DescribeMacro{\ifchilddocmanual}
The main file should be prepared as usual, see \secref{sec:include}.
However, the document body must make a distinction
between processing of an individual part and of the main document, e.g.:
%
\begin{center}
\begin{tabular}{l}
|\ifchilddocmanual|\\
|\input{\childdocname}|\\
|\||else|\\
\textit{document body with }|\input{|\textit{part}|}|\\
|\||fi|
\end{tabular}
\end{center}
%
The conditional |\ifchilddocmanual| is true whenever
a part to be included by |\input| is being compiled,
and the name of the part is stored in |\childdocname|.

%%%%%%%%%%%%%%%%%%%%%%%%%%%%%%%%%%%%%%%%
\DescribeMacro{\childdocby}
Each part to be included by |\input| should start with:
%
\begin{center}
\begin{tabular}{l}
|\input{childdoc.def}|\\
|\childdocby{|\textit{main}|}|\\
\end{tabular}
\end{center}
%
The directive |\childdocby| is similar to |\childdocof|
described in \secref{sec:include},
but the subsequent selection of content must be done manually.
To that end, both |\ifchilddoc| and |\ifchilddocmanual|
will be true upon processing of a part,
and the name of the part is stored in |\childdocname|.
Note that |\jobname| will be set to the filename of the current part
so that each part receives an individual |.aux| file
that does not interfere with the |.aux| file(s) of the main document.
This behaviour can be altered by the alternative form
|\childdocby[*]{|\textit{main}|}| (with a non-empty optional argument)
which uses the |.aux| file of the main document
by setting |\jobname| to \textit{main}.

%%%%%%%%%%%%%%%%%%%%%%%%%%%%%%%%%%%%%%%%%%%%%%%%%%%%%%%%%%%%%%%%%%%%%%%%%%%%%%%%
\subsection{Driver Development}
\label{sec:driver}

The \textsf{childdoc} mechanism can also be use for the development
of definition files such as \LaTeX{} styles or classes.
This case differs from the above setup with multiple parts
included by |\include| in that no |\includeonly| should be invoked.
This can be achieved by starting the include file
(before |\ProvidesPackage|) with:
%
\begin{center}
\begin{tabular}{l}
|\input{childdoc.def}|\\
|\childdocforward{|\textit{main}|}|\\
\end{tabular}
\end{center}
%
or alternatively with:
%
\begin{center}
\begin{tabular}{l}
|\input{childdoc.def}|\\
|\childdocby{|\textit{main}|}|\\
\end{tabular}
\end{center}
%
Both forms have slightly different effects as described above.
The main file is prepared as usual, see \secref{sec:include}.

%%%%%%%%%%%%%%%%%%%%%%%%%%%%%%%%%%%%%%%%%%%%%%%%%%%%%%%%%%%%%%%%%%%%%%%%%%%%%%%%
\subsection{Legacy Detection}
\label{sec:detection}

The directive |\childdocmain| in the main file can detect
whether the complete document or merely a child is to be compiled
even without using the directive |\childdocof|.
This method is deprecated because it is less robust
and there is no compelling reason to use it;
it is merely provided for backward compatibility
and it may be removed in future versions.

If the detection mechanism is to be used,
it is mandatory to correctly specify
the filename of the main file as the argument of |\childdocmain|:
%
\begin{center}
\begin{tabular}{l}
|\input{childdoc.def}|\\
|\childdocmain{|\textit{main}|}|\\
\end{tabular}
\end{center}
%
If |\jobname| does not match the argument \textit{main} of |\childdocmain|,
it is assumed that |\jobname| points to the child file to be compiled.
When using |\childdocmain| with the main file specified as argument,
it suffices to start a child file
with just |\input{|\textit{main}|}|
without loading of the package and using |\childdocof|.
If instead all processing is done
with the appropriate \textsf{childdoc} directives,
the argument of \textit{main} of |\childdocmain| can be empty.

An alternative version of the command line processing described
in \secref{sec:commandline} using the detection mechanism reads:
%
\begin{center}
|... -jobname "|\textit{target}|" "|[\textit{flags}]%
[|\def\jobname{|\textit{dest}|}|]|\input{|\textit{main}|}"|
\end{center}

%%%%%%%%%%%%%%%%%%%%%%%%%%%%%%%%%%%%%%%%%%%%%%%%%%%%%%%%%%%%%%%%%%%%%%%%%%%%%%%%
\subsection{Manual Code}
\label{sec:manual}

In case one cannot be certain whether the definitions file |childdoc.def|
is installed on the target \TeX{} distribution
and one prefers not to ship it,
it is conceivable to paste a few relevant commands into the sources.

To that end, drop all statements |\input{childdoc.def}|
and perform the replacements as outlined below.
Instead of |\childdocmain{|\textit{main}|}| add the following code
to the top of the main file:
%
\begin{center}
\begin{tabular}{l}
|\||ifdefined\childdocname\endinput\||fi\newif\ifchilddoc|\\
|\edef\childdocname{\scantokens\expandafter{\jobname\noexpand}}|\\
|\def\childdocmain{|\textit{main}|}\||ifx\childdocmain\childdocname\||else|\\
|\childdoctrue\includeonly{\childdocname}\let\jobname\childdocmain\||fi|\\
\end{tabular}
\end{center}
%
Instead of |\childdocof{|\textit{main}|}| just include the main file
at the top of each child file:
%
\begin{center}
|\input{|\textit{main}|}|
\end{center}
%
A simple redirection |\childdocforward{|\textit{dest}|}| is achieved by:
%
\begin{center}
|\def\jobname{|\textit{dest}|}\input{\jobname}|
\end{center}
%
The redirection with prefix
|\childdocforwardprefix[|\textit{prefix}|]{|\textit{dest}|}|
is accomplished by:
%
\begin{center}
\begin{tabular}{l}
|{\edef\jobname{\scantokens\expandafter{\jobname\noexpand}}|\\
|\def\redirectjob |\textit{prefix}|#1~~~{\gdef\jobname{|\textit{dest}|#1}}|\\
|\expandafter\redirectjob\jobname~~~}\input{\jobname}|
\end{tabular}
\end{center}

In an alternative approach,
child documents can be compiled by a specific command line
without additional code or specific definitions:
%
\begin{center}
|... -jobname "|\textit{target}|" "|[\textit{flags}]%
|\includeonly{|\textit{dest}|}\input{|\textit{main}|}"|
\end{center}
%

%%%%%%%%%%%%%%%%%%%%%%%%%%%%%%%%%%%%%%%%%%%%%%%%%%%%%%%%%%%%%%%%%%%%%%%%%%%%%%%%
%%%%%%%%%%%%%%%%%%%%%%%%%%%%%%%%%%%%%%%%%%%%%%%%%%%%%%%%%%%%%%%%%%%%%%%%%%%%%%%%
\section{Information}

%%%%%%%%%%%%%%%%%%%%%%%%%%%%%%%%%%%%%%%%%%%%%%%%%%%%%%%%%%%%%%%%%%%%%%%%%%%%%%%%
\subsection{Copyright}

Copyright \copyright{} 2017--2018 Niklas Beisert

This work may be distributed and/or modified under the
conditions of the \LaTeX{} Project Public License, either version 1.3
of this license or (at your option) any later version.
The latest version of this license is in
  \url{http://www.latex-project.org/lppl.txt}
and version 1.3 or later is part of all distributions of \LaTeX{}
version 2005/12/01 or later.

This work has the LPPL maintenance status `maintained'.

The Current Maintainer of this work is Niklas Beisert.

This work consists of the files |README.txt|, |childdoc.ins| and |childdoc.dtx|
as well as the derived files |childdoc.def|, |cdocsamp.tex|
with |cdocsch1.tex|, |cdocsch2.tex|, |cdocspt3.tex|, |cdocspt4.tex|,
|cdocsdrf.tex|, |cdocsfn1.tex|, |cdocsfn2.tex|
as well as |childdoc.pdf|.

%%%%%%%%%%%%%%%%%%%%%%%%%%%%%%%%%%%%%%%%%%%%%%%%%%%%%%%%%%%%%%%%%%%%%%%%%%%%%%%%
\subsection{Files and Installation}

The package consists of the files:
%
\begin{center}
\begin{tabular}{ll}
    |README.txt|   & readme file \\
    |childdoc.ins| & installation file \\
    |childdoc.dtx| & source file \\
    |childdoc.def| & definition file \\
    |cdocsamp.tex| & sample main file \\
    |cdocsch1.tex| & sample include file \\
    |cdocsch2.tex| & sample include file \\
    |cdocspt3.tex| & sample part file \\
    |cdocspt4.tex| & sample part file \\
    |cdocsdrf.tex| & sample redirection file \\
    |cdocsfn1.tex| & sample redirection file \\
    |cdocsfn2.tex| & sample redirection file \\
    |childdoc.pdf| & manual
\end{tabular}
\end{center}
%
The distribution consists of the files
|README.txt|, |childdoc.ins| and |childdoc.dtx|.
%
\begin{itemize}
\item
Run (pdf)\LaTeX{} on |childdoc.dtx|
to compile the manual |childdoc.pdf| (this file).
\item
Run \LaTeX{} on |childdoc.ins| to create the definitions file |childdoc.def|
and the sample |cdocsamp.tex| with include files
|cdocsch1.tex|, |cdocsch2.tex|, |cdocspt3.tex|, |cdocspt4.tex|,
|cdocsdrf.tex|, |cdocsfn1.tex|, |cdocsfn2.tex|.
Then copy the file |childdoc.def| to an appropriate directory of your \LaTeX{}
distribution, e.g.\ \textit{texmf-root}|/tex/latex/childdoc|.
\end{itemize}

%%%%%%%%%%%%%%%%%%%%%%%%%%%%%%%%%%%%%%%%%%%%%%%%%%%%%%%%%%%%%%%%%%%%%%%%%%%%%%%%
\subsection{Related CTAN Packages}

There are several other packages which offer a similar functionality:
%
\begin{itemize}
\item
The packages
\href{http://ctan.org/pkg/docmute}{\textsf{docmute}},
\href{http://ctan.org/pkg/includex}{\textsf{includex}} and
\href{http://ctan.org/pkg/standalone}{\textsf{standalone}}
provide commands to include only the document body of
a child file thus allowing both files to be compiled individually.
\item
The packages \href{http://ctan.org/pkg/subdocs}{\textsf{subdocs}}
and \href{http://ctan.org/pkg/subfiles}{\textsf{subfiles}}
provide structures in which the main and child documents can be
encapsulated and allowing them to be compiled individually.
The inclusion mechanism is different from the conventional |\include|.
\item
The package \href{http://ctan.org/pkg/combine}{\textsf{combine}}
is an elaborate solution to combine several documents into one.
\end{itemize}
%
See also the CTAN topic \href{http://ctan.org/topic/subdocs}{\textsf{subdocs}}
for further related packages.
The present package differs from the above solutions in that
a document structure constructed with the conventional |\include| mechanism
just needs two extra commands at the top of every file
such that all constituent files can be compiled individually.

%%%%%%%%%%%%%%%%%%%%%%%%%%%%%%%%%%%%%%%%%%%%%%%%%%%%%%%%%%%%%%%%%%%%%%%%%%%%%%%%
%\subsection{Feature Suggestions}
%
%The following is a list of features which may be useful for future
%versions of this package:
%%
%\begin{itemize}
%\item
%\ldots
%\end{itemize}

%%%%%%%%%%%%%%%%%%%%%%%%%%%%%%%%%%%%%%%%%%%%%%%%%%%%%%%%%%%%%%%%%%%%%%%%%%%%%%%%
\subsection{Revision History}

%%%%%%%%%%%%%%%%%%%%%%%%%%%%%%%%%%%%%%%%
\paragraph{v2.0:} 2018/12/30

\begin{itemize}
\item
immediate forward processing
\item
added |\childdocby| mechanism
\item
manual restructured
\end{itemize}

%%%%%%%%%%%%%%%%%%%%%%%%%%%%%%%%%%%%%%%%
\paragraph{v1.6:} 2018/01/17

\begin{itemize}
\item
application for development of include files
\item
corrections to manual
\end{itemize}

%%%%%%%%%%%%%%%%%%%%%%%%%%%%%%%%%%%%%%%%
\paragraph{v1.5:} 2017/05/21

\begin{itemize}
\item
more complete structuring introduced
\item
|\childdocof| introduced
\item
|\childdoc| renamed to |\childdocmain|
\item
|\childredirect| renamed to |\childdocforward| and |\childdocforwardprefix|
and functionality expanded
\end{itemize}

%%%%%%%%%%%%%%%%%%%%%%%%%%%%%%%%%%%%%%%%
\paragraph{v1.0:} 2017/04/27

\begin{itemize}
\item
manual and install package
\item
first version published on CTAN
\end{itemize}

%%%%%%%%%%%%%%%%%%%%%%%%%%%%%%%%%%%%%%%%
\paragraph{v0.6:} 2017/04/26

\begin{itemize}
\item
redirection mechanism added
\end{itemize}

%%%%%%%%%%%%%%%%%%%%%%%%%%%%%%%%%%%%%%%%
\paragraph{v0.5:} 2017/04/26

\begin{itemize}
\item
functionality in definition file
\end{itemize}


%%%%%%%%%%%%%%%%%%%%%%%%%%%%%%%%%%%%%%%%%%%%%%%%%%%%%%%%%%%%%%%%%%%%%%%%%%%%%%%%
%%%%%%%%%%%%%%%%%%%%%%%%%%%%%%%%%%%%%%%%%%%%%%%%%%%%%%%%%%%%%%%%%%%%%%%%%%%%%%%%
%%%%%%%%%%%%%%%%%%%%%%%%%%%%%%%%%%%%%%%%%%%%%%%%%%%%%%%%%%%%%%%%%%%%%%%%%%%%%%%%
\appendix

\settowidth\MacroIndent{\rmfamily\scriptsize 000\ }

 \DocInput{childdoc.dtx}

\end{document}
%</driver>
% \fi
%
% %%%%%%%%%%%%%%%%%%%%%%%%%%%%%%%%%%%%%%%%%%%%%%%%%%%%%%%%%%%%%%%%%%%%%%%%%%%%%%
% %%%%%%%%%%%%%%%%%%%%%%%%%%%%%%%%%%%%%%%%%%%%%%%%%%%%%%%%%%%%%%%%%%%%%%%%%%%%%%
% \section{Sample}
%\iffalse
%<*samplemain>
%\fi
%
% The following presents a sample document
% with two chapters, two parts, a title page,
% a compile flag as well as three forwarding files to set the flag.
% It consists of eight |.tex| files:
% \begin{center}
% \begin{tabular}{ll}
% |cdocsamp.tex|&main file\\
% |cdocsch1.tex|&include file for chapter 1\\
% |cdocsch2.tex|&include file for chapter 2\\
% |cdocspt3.tex|&include file for part 3\\
% |cdocspt4.tex|&include file for part 4\\
% |cdocsdrf.tex|&forwarding file for main file in draft mode\\
% |cdocsfi1.tex|&forwarding file for final version of chapter 1\\
% |cdocsfi2.tex|&forwarding file for final version of chapter 2\\
% \end{tabular}
% \end{center}
% Each of the eight files can be compiled directly by the \LaTeX{} compiler.
%
% %%%%%%%%%%%%%%%%%%%%%%%%%%%%%%%%%%%%%%
% \paragraph{Main File.}
%
% The main file is called |cdocsamp.tex|.
%
% Load the \textsf{childdoc} definitions and
% declare the filename for the main document:
%    \begin{macrocode}
\input{childdoc.def}
\childdocmain{}
%    \end{macrocode}

% Optional override for |\version| flag:
%    \begin{macrocode}
%%\ifchilddoc\else\providecommand{\version}{draft}\fi
%    \end{macrocode}

% Define the default values for the |\version| flag
% (|final| for the main file and |draft| for childs):
%    \begin{macrocode}
\ifchilddoc
\providecommand{\version}{draft}
\else
\providecommand{\version}{final}
\fi
%    \end{macrocode}

% Load the standard document class:
%    \begin{macrocode}
\documentclass[12pt]{article}
%    \end{macrocode}

% Start the document body:
%    \begin{macrocode}
\begin{document}
%    \end{macrocode}

% Declare a title page.
% Print title, part of document being processed and version flag:
%    \begin{macrocode}
\addtocounter{page}{-1}
\begin{center}
{\LARGE\bfseries{}childdoc example\par}
\vspace{1cm}
\ifchilddoc
\ifchilddocmanual part\else chapter\fi:
`\childdocname' of `\childdocjob'\par
\else
main document: `\childdocjob'\par
\fi
version: \version\par
\end{center}
\newpage
%    \end{macrocode}

% Manually include selected file,
% otherwise process as usual:
%    \begin{macrocode}
\ifchilddocmanual
\section*{part `\childdocname'}
\input{\childdocname}
\else
%    \end{macrocode}

% Include the two chapters:
%    \begin{macrocode}
\include{cdocsch1}
\include{cdocsch2}
%    \end{macrocode}

% Include the two parts unless only chapters should be displayed:
%    \begin{macrocode}
\ifchilddoc\else
\section{part three}
\input{cdocspt3}
\section{part four}
\input{cdocspt4}
\fi
%    \end{macrocode}

% Process as usual until here:
%    \begin{macrocode}
\fi
%    \end{macrocode}

% End of document body:
%    \begin{macrocode}
\end{document}
%    \end{macrocode}
%\iffalse
%</samplemain>
%\fi
%
% %%%%%%%%%%%%%%%%%%%%%%%%%%%%%%%%%%%%%%
% \paragraph{Chapter Include Files.}
%
% The include files are called |cdocsch1.tex| and |cdocsch2.tex|.
%
%\iffalse
%<*samplechap1|samplechap2>
%\fi

% Optional override for |\version| flag:
%    \begin{macrocode}
%%\providecommand{\version}{final}
%    \end{macrocode}

% Include the main document:
%    \begin{macrocode}
\input{childdoc.def}
\childdocof{cdocsamp}
%    \end{macrocode}

%\iffalse
%</samplechap1|samplechap2>
%\fi
%
%\iffalse
%<*samplechap1>
%\fi
% Some text for chapter 1:
%    \begin{macrocode}
\section{one}
some text in chapter one
%    \end{macrocode}

%\iffalse
%</samplechap1>
%\fi
% Some text for chapter 2:
%\iffalse
%<*samplechap2>
%\fi
%    \begin{macrocode}
\section{two}
more text in chapter two
%    \end{macrocode}

%\iffalse
%</samplechap2>
%\fi
%
% %%%%%%%%%%%%%%%%%%%%%%%%%%%%%%%%%%%%%%
% \paragraph{Part Include Files.}
%
% The include files are called |cdocspt3.tex| and |cdocspt4.tex|.
%
%\iffalse
%<*samplepart3|samplepart4>
%\fi

% Optional override for |\version| flag:
%    \begin{macrocode}
%%\providecommand{\version}{final}
%    \end{macrocode}

% Include the main document:
%    \begin{macrocode}
\input{childdoc.def}
\childdocby{cdocsamp}
%    \end{macrocode}

%\iffalse
%</samplepart3|samplepart4>
%\fi
%
%\iffalse
%<*samplepart3>
%\fi
% Some text for part 3:
%    \begin{macrocode}
some text in part three
%    \end{macrocode}

%\iffalse
%</samplepart3>
%\fi
% Some text for part 4:
%\iffalse
%<*samplepart4>
%\fi
%    \begin{macrocode}
more text in part four
%    \end{macrocode}

%\iffalse
%</samplepart4>
%\fi
%
% %%%%%%%%%%%%%%%%%%%%%%%%%%%%%%%%%%%%%%
% \paragraph{Forwarding for a Complete Draft.}
%
% The following forwarding file |cdocsdrf.tex|
% compiles the main document in draft mode:
%\iffalse
%<*sampledraft>
%\fi
%    \begin{macrocode}
\def\version{draft}
\input{childdoc.def}
\childdocforward{cdocsamp}
%    \end{macrocode}

%\iffalse
%</sampledraft>
%\fi
%
% %%%%%%%%%%%%%%%%%%%%%%%%%%%%%%%%%%%%%%
% \paragraph{Forwarding for Final Version of the Chapters.}
%
% The following forwarding files |cdocsfn1.tex| and |cdocsfn2.tex|
% (with identical content)
% compile the final versions of the child documents
% |cdocsch1.tex| and |cdocsch2.tex|, respectively:
%\iffalse
%<*samplefinal>
%\fi
%    \begin{macrocode}
\def\version{final}
\input{childdoc.def}
\childdocforwardprefix[cdocsamp]{cdocsfn}{cdocsch}
%    \end{macrocode}

%\iffalse
%</samplefinal>
%\fi
%
% %%%%%%%%%%%%%%%%%%%%%%%%%%%%%%%%%%%%%%
% \paragraph{Command Line Processing.}
%
% The following three command lines generate the output files
% |cdocscld|, |cdocscl1| and |cdocscl2|
% which should be identical to
% |cdocsdrf|, |cdocsch1| and |cdocsfn2|, respectively:
% \begin{center}
% \begin{tabular}{l}
% |latex -jobname cdocscld \|\\
% |  "\def\version{draft}\input{childdoc.def}\childdocforward{cdocsamp}"|\\
% |latex -jobname cdocscl1 \|\\
% |  "\input{childdoc.def}\childdocforward[cdocsamp]{cdocsch1}"|\\
% |latex -jobname cdocscl2 \|\\
% |  "\def\version{final}\input{childdoc.def}\childdocforward{cdocsch2}"|
% \end{tabular}
% \end{center}
% Note that the trailing backslash on each first line
% merely continues the input to the second line
% (for convenient cut ant paste).
% Furthermore, the command |latex| can be replaced by any
% of its alternative versions such as |pdflatex|.
%
% %%%%%%%%%%%%%%%%%%%%%%%%%%%%%%%%%%%%%%%%%%%%%%%%%%%%%%%%%%%%%%%%%%%%%%%%%%%%%%
% %%%%%%%%%%%%%%%%%%%%%%%%%%%%%%%%%%%%%%%%%%%%%%%%%%%%%%%%%%%%%%%%%%%%%%%%%%%%%%
% \section{Implementation}
%\iffalse
%<*package>
%\fi
%
% This section describes the definitions file |childdoc.def|.

% The definitions cannot be loaded using |\usepackage| or |\RequirePackage|
% which has a mechanism to prevent loading a style file more than once.
% When loading the definitions by means of |\input|
% multiple instances have to be prevented manually:
%\iffalse
%This code needs to be before the `\ProvidesFile' directive
%which is defined at the beginning of this file.
%Therefore it is also placed there and commented out here.
%</package>
%<*discard>
%\fi
%    \begin{macrocode}
\ifdefined\childdocmain\endinput\fi
%    \end{macrocode}
%\iffalse
%</discard>
%<*package>
%\fi
%
% \macro{\ifchilddoc}
% \macro{\ifchilddocmanual}
% The conditional |\ifchilddoc| tells whether a
% child (true) or main (false) document is being compiled.
% The conditional |\ifchilddocmanual| tells whether
% the |\includeonly| mechanism is used (false) or
% the selection of child files must be performed manually (true).
% The definitions initialise to false:
%    \begin{macrocode}
\newif\ifchilddoc
\newif\ifchilddocmanual
%    \end{macrocode}

% \macro{\childdocname}
% \macro{\childdocjob}
% The macro |\childdocname| stores the name of the main document
% to be compiled. The macro |\childdocjob| stores the name of
% the document on which the \LaTeX{} compiler was originally invoked.
% The content of |\jobname| cannot be compared
% to filenames specified in the source due to different catcodes.
% The following code rescans |\jobname|, stores the result
% in |\childdocname| and saves a copy in |\childdocjob|:
%    \begin{macrocode}
\edef\childdocname{\scantokens\expandafter{\jobname\noexpand}}
\let\childdocjob\childdocname
%    \end{macrocode}

% \macro{\childdocdisable}
% The macro |\childdocdisable| prevents the main file
% from being processed more than once.
% At this stage, the main document command |\childdocmain|
% is assumed to be called once again where it should do nothing.
% Any subsequent call to it should prevent
% a secondary processing of the main document
% It overwrites the forwarding commands
% |\childdocof| and |\childdocforward|
% with empty macros to prevent further inclusions of the main document:
%    \begin{macrocode}
\newcommand{\childdocdisable}
{
  \renewcommand{\childdocmain}[1]{\renewcommand{\childdocmain}[1]{\endinput}}
  \renewcommand{\childdocof}[1]{}
  \renewcommand{\childdocby}[2][]{}
  \renewcommand{\childdocforward}[2][]{}
  \renewcommand{\childdocdisable}{}
}
%    \end{macrocode}

% \macro{\childdocmain}
% The macro |\childdocmain| is to be called at the top of the main file
% with nothing or the main filename (without extension) as argument.
% First, it breaks loops.
% If the argument is not empty and does not match |\childdocname|
% (which is set by the first inclusion of |childdoc.def|),
% |\ifchilddoc| is set to true, |\includeonly| is applied to the child file
% and |\jobname| is set to the main file
% (for proper handling of |.aux| files):
%    \begin{macrocode}
\newcommand{\childdocmain}[1]
{
  \childdocdisable\childdocmain{}
  \if?#1?\else
    \begingroup
      \def\childdoctmp{#1}
      \ifx\childdoctmp\childdocname
        \def\childdoctmp{}
      \else
        \def\childdoctmp
        {
          \childdoctrue
          \includeonly{\childdocname}
          \def\childdocjob{#1}
          \def\jobname{#1}
        }
      \fi
      \expandafter
    \endgroup
    \childdoctmp
  \fi
}
%    \end{macrocode}

% \macro{\childdocof}
% The command |\childdocof| redirects
% compilation to the main file |#1|.
%    \begin{macrocode}
\newcommand{\childdocof}[1]
{
  \childdocdisable
  \childdoctrue
  \includeonly{\childdocname}
  \def\jobname{#1}
  \def\childdocjob{#1}
  \input{#1}
}
%    \end{macrocode}

% \macro{\childdocby}
% The command |\childdocby| ....
%    \begin{macrocode}
\newcommand{\childdocby}[2][]
{
  \childdocdisable
  \childdoctrue
  \childdocmanualtrue
  \if?#1?\else
    \def\jobname{#2}
  \fi
  \def\childdocjob{#2}
  \input{#2}
  \endinput
}
%    \end{macrocode}

% \macro{\childdocforward}
% The command |\childdocforward| redirects
% compilation to the main file or
% (if the optional argument is given) a child file.
% Parameters are set as if the main file
% or a child file starting with |\childdocof| was compiled.
% Then compilation is handed over to the main file:
%    \begin{macrocode}
\newcommand{\childdocforward}[2][]
{
  \begingroup
    \if?#1?
      \def\childdoctmp
      {
        \def\childdocname{#2}
        \def\childdocjob{#2}
        \def\jobname{#2}
        \input{#2}
        \endinput
      }
    \else
      \def\childdoctmp
      {
        \childdocdisable
        \def\childdocname{#2}
        \childdoctrue
        \includeonly{#2}
        \def\childdocjob{#1}
        \def\jobname{#1}
        \input{#1}
        \endinput
      }
    \fi
    \expandafter
  \endgroup
  \childdoctmp
}
%    \end{macrocode}

% \macro{\childdocforwardprefix}
% The command |\childdocforwardprefix| redirects
% compilation to the main or a child file by means of a pattern.
% The prefix |#1| in the current filename is replaced by |#2|
% and the suffix of the current filename is kept
% (it is assumed that the filename does not contain the substring `|~~~|'
% which is used as a delimiter).
% Compilation is handed over to the new file by |\childdocforward|:
%    \begin{macrocode}
\newcommand{\childdocforwardprefix}[3][]
{
  \begingroup
    \def\childdocextract #2##1~~~{\def\childdoctmp{\childdocforward[#1]{#3##1}}}
    \expandafter\childdocextract\childdocname~~~
    \expandafter
  \endgroup
  \childdoctmp
}
%    \end{macrocode}

% \macro{\childdoc}
% The deprecated macro |\childdoc| is a legacy version of |\childdocmain|:
%    \begin{macrocode}
\newcommand{\childdoc}{\childdocmain}
%    \end{macrocode}

% \macro{\childdocredirect}
% The deprecated macro |\childdocredirect| is a legacy version
% of |\childdocforward| and |\childdocforwardprefix|:
%    \begin{macrocode}
\newcommand{\childdocredirect}[2][]
{
  \begingroup
    \if?#1?
      \def\childdoctmp{\childdocforward{#2}}
    \else
      \def\childdoctmp{\childdocforwardprefix{#1}{#2}}
    \fi
    \expandafter
  \endgroup
  \childdoctmp
}
%    \end{macrocode}

%\iffalse
%</package>
%\fi
%
\endinput

\childdocof{cdocsamp}
%    \end{macrocode}

%\iffalse
%</samplechap1|samplechap2>
%\fi
%
%\iffalse
%<*samplechap1>
%\fi
% Some text for chapter 1:
%    \begin{macrocode}
\section{one}
some text in chapter one
%    \end{macrocode}

%\iffalse
%</samplechap1>
%\fi
% Some text for chapter 2:
%\iffalse
%<*samplechap2>
%\fi
%    \begin{macrocode}
\section{two}
more text in chapter two
%    \end{macrocode}

%\iffalse
%</samplechap2>
%\fi
%
% %%%%%%%%%%%%%%%%%%%%%%%%%%%%%%%%%%%%%%
% \paragraph{Part Include Files.}
%
% The include files are called |cdocspt3.tex| and |cdocspt4.tex|.
%
%\iffalse
%<*samplepart3|samplepart4>
%\fi

% Optional override for |\version| flag:
%    \begin{macrocode}
%%\providecommand{\version}{final}
%    \end{macrocode}

% Include the main document:
%    \begin{macrocode}
% \iffalse
%
% childdoc.dtx Copyright (C) 2017-2018 Niklas Beisert
%
% This work may be distributed and/or modified under the
% conditions of the LaTeX Project Public License, either version 1.3
% of this license or (at your option) any later version.
% The latest version of this license is in
%   http://www.latex-project.org/lppl.txt
% and version 1.3 or later is part of all distributions of LaTeX
% version 2005/12/01 or later.
%
% This work has the LPPL maintenance status `maintained'.
%
% The Current Maintainer of this work is Niklas Beisert.
%
% This work consists of the files childdoc.dtx and childdoc.ins
% and the derived files childdoc.def and cdocsamp.tex with
% cdocsch1.tex, cdocsch2.tex, cdocsdrf.tex, cdocsfn1.tex, cdocsfn2.tex.
%
%<package>\ifdefined\childdocmain\endinput\fi
%<package>\ProvidesFile{childdoc.def}[2018/12/30 v2.0 child document driver]
%<samplemain>\ProvidesFile{cdocsamp.tex}[2018/12/30 v2.0 sample for childdoc]
%<*driver>
%\ProvidesFile{childdoc.drv}[2018/12/30 v2.0 childdoc reference manual file]
\PassOptionsToClass{10pt,a4paper}{article}
\documentclass{ltxdoc}

\usepackage[margin=35mm]{geometry}
\usepackage{hyperref}
\usepackage{hyperxmp}
\usepackage[usenames]{color}

\hypersetup{colorlinks=true}
\hypersetup{pdfstartview=FitH}
\hypersetup{pdfpagemode=UseNone}
\hypersetup{pdfsource={}}
\hypersetup{pdflang={en-UK}}
\hypersetup{pdfcopyright={Copyright 2017-2018 Niklas Beisert.
  This work may be distributed and/or modified under the
  conditions of the LaTeX Project Public License, either version 1.3
  of this license or (at your option) any later version.}}
\hypersetup{pdflicenseurl={http://www.latex-project.org/lppl.txt}}
\hypersetup{pdfcontactaddress={ETH Zurich, ITP, HIT K,
  Wolfgang-Pauli-Strasse 27}}
\hypersetup{pdfcontactpostcode={8093}}
\hypersetup{pdfcontactcity={Zurich}}
\hypersetup{pdfcontactcountry={Switzerland}}
\hypersetup{pdfcontactemail={nbeisert@itp.phys.ethz.ch}}
\hypersetup{pdfcontacturl={http://people.phys.ethz.ch/\xmptilde nbeisert/}}

\newcommand{\secref}[1]{\hyperref[#1]{section \ref*{#1}}}

\parskip1ex
\parindent0pt
\let\olditemize\itemize
\def\itemize{\olditemize\parskip0pt}

\begin{document}

\title{The \textsf{childdoc} Package}
\hypersetup{pdftitle={The childdoc Package}}
\author{Niklas Beisert\\[2ex]
  Institut f\"ur Theoretische Physik\\
  Eidgen\"ossische Technische Hochschule Z\"urich\\
  Wolfgang-Pauli-Strasse 27, 8093 Z\"urich, Switzerland\\[1ex]
  \href{mailto:nbeisert@itp.phys.ethz.ch}
  {\texttt{nbeisert@itp.phys.ethz.ch}}}
\hypersetup{pdfauthor={Niklas Beisert}}
\hypersetup{pdfsubject={Manual for the LaTeX2e Package childdoc}}
\date{30 December 2018, \textsf{v2.0}}
\maketitle

\begin{abstract}\noindent
\textsf{childdoc} is a \LaTeXe{} package
that enables the direct compilation
of document sections included by |\include|
to individual files.
\end{abstract}

\begingroup
\parskip0ex
\tableofcontents
\endgroup

%%%%%%%%%%%%%%%%%%%%%%%%%%%%%%%%%%%%%%%%%%%%%%%%%%%%%%%%%%%%%%%%%%%%%%%%%%%%%%%%
%%%%%%%%%%%%%%%%%%%%%%%%%%%%%%%%%%%%%%%%%%%%%%%%%%%%%%%%%%%%%%%%%%%%%%%%%%%%%%%%
\section{Introduction}

\LaTeX{} provides a mechanism to structure a large document (such as a book)
into a main file and several child files (containing the chapters)
using the |\include| command.
This mechanism is beneficial for documents
which span hundreds of pages in order to
make the source file(s) more manageable.
Moreover, compilation can be restricted to
selected child files by means of the |\includeonly| command.
The latter feature can be used to reduce the compilation time while editing
(this was significantly more useful in the earlier days of \LaTeX{})
or to generate a smaller document which is easier to navigate.
Another application of |\includeonly| is to generate
documents consisting of selected parts of the complete document.

However, there are a few drawbacks of the plain |\include| mechanism:
\begin{itemize}
\item
The child files cannot be compiled on their own,
they can only be compiled via the main file.
A naive editing environment
(such as a text editor with an option
to have the current file processed by \LaTeX)
may require one to switch to the main file before compiling;
attempting to compile the child file produces errors.
\item
The main file must be modified (each time)
to adjust the |\includeonly| command
to the present needs. This easily leaves the main file in a messy state.
\item
The generated document will always carry the filename
of the main document. This is inconvenient if
several child files are to be compiled and
to be kept for distribution.
\end{itemize}

The present package provides a simple interface
to make child files individually compilable by \LaTeX{}.
Compiling a child file then has the same effect as compiling
the main file with an |\includeonly| command
to select the appropriate child.
Moreover the generated document will carry the name of the child
rather than the main file.
This resolves all three above issues.

This feature is meant to make the editing of books,
thesis documents and lecture notes somewhat more convenient.
However, the package can also be used efficiently for
composing a series of documents (such as exercise sheets)
which are typically distributed individually.
It then assists the author in generating the individual documents
(potentially in different versions)
as well as a document containing the collected series.
Another application is in developing style files
or other kinds of included material
where compilation of the style file could redirect
to a sample or test file.

%%%%%%%%%%%%%%%%%%%%%%%%%%%%%%%%%%%%%%%%%%%%%%%%%%%%%%%%%%%%%%%%%%%%%%%%%%%%%%%%
%%%%%%%%%%%%%%%%%%%%%%%%%%%%%%%%%%%%%%%%%%%%%%%%%%%%%%%%%%%%%%%%%%%%%%%%%%%%%%%%
\section{Usage}

First of all, the package \textsf{childdoc} is \emph{not} a standard
\LaTeXe{} |.sty| style file! Therefore it needs to be invoked in
a non-standard way.

%%%%%%%%%%%%%%%%%%%%%%%%%%%%%%%%%%%%%%%%%%%%%%%%%%%%%%%%%%%%%%%%%%%%%%%%%%%%%%%%
\subsection{Included Files}
\label{sec:include}

%%%%%%%%%%%%%%%%%%%%%%%%%%%%%%%%%%%%%%%%
\DescribeMacro{\childdocmain}
To use the package, add the commands
\begin{center}
\begin{tabular}{l}
|\input{childdoc.def}|\\
|\childdocmain{}|\\
\end{tabular}
\end{center}
at the very top of the main \LaTeX{} file,
in particular \emph{before} the |\documentclass| statement!
The argument of |\childdocmain| should be left empty
(but it must be present).

%%%%%%%%%%%%%%%%%%%%%%%%%%%%%%%%%%%%%%%%
\DescribeMacro{\childdocof}
Furthermore, add the commands
\begin{center}
\begin{tabular}{l}
|\input{childdoc.def}|\\
|\childdocof{|\textit{main}|}|\\
\end{tabular}
\end{center}
at the top of every child file \textit{child}
which is included by |\include{|\textit{child}|}|
from within the main file
(or at least for those files to be compiled individually).
The argument \textit{main} must be the filename of the main file.

There are a couple of
considerations in setting up the main and child documents:

%%%%%%%%%%%%%%%%%%%%%%%%%%%%%%%%%%%%%%%%
\paragraph{Restrictions.}

Please note the following restrictions:
\begin{itemize}
\item
|\childdocmain| must be called with one argument \textit{main}
to ensure compatibility with earlier version of the package.
It must either be empty (|\childdocmain{}|)
or precisely match the filename of the main file in which it is specified.
See \secref{sec:detection} for further information.
\item
The filename \textit{main} must be specified without the |.tex| extension.
\item
The filename \textit{main} is case sensitive
(even in case-insensitive file systems)
due to internal string comparison.
\item
The argument \textit{main} should be fully expanded, it cannot be a macro.
\item
Subdirectories and special characters should be avoided in filenames.
\item
The command |\childdocmain{|\textit{main}|}| must be followed by a whitespace.
It should not be followed immediately by another command
or by a comment mark `|%|'.
This is because the \TeX{} parser reads the token immediately following
the argument of |\childdocmain| and puts it
at the beginning of every child section;
however, a white\-space is ignored.
\end{itemize}

%%%%%%%%%%%%%%%%%%%%%%%%%%%%%%%%%%%%%%%%
\paragraph{Content of Main File.}

It is advisable to place all content in the child files included by |\include|.
Any output contained in the main file will appear in all child documents
unless suppressed manually;
it cannot be suppressed automatically by the |\includeonly| directive
and thus should normally be avoided.
A method to include some content in the main file
by means of conditional processing is described in \secref{sec:conditional}.

%%%%%%%%%%%%%%%%%%%%%%%%%%%%%%%%%%%%%%%%
\paragraph{Page Numbering.}

When only a part of the document is compiled,
the appropriate numbering of pages
(as well as other status parameters)
is determined from the |.aux| files.
The latter contain information from previous passes.
However this information needs to propagate through
all intermediate child documents.
Therefore the page numbering in child documents may well
be inconsistent until the complete document is compiled at least once.

A useful (if unconventional) way to always ensure a consistent
page numbering is to restart the numbering in each child document
and denote the pages by `\textit{child}|.|\textit{page}'
where \textit{child} represents the chapter/section number of the child file.
This can be achieved by the command
|\numberwithin{page}{|\textit{child}|}|
of the \textsf{amsmath} package
where \textit{child} can be |chapter| or |section|
depending on the chosen structuring.
Alternatively, one can modify the macro |\thepage| appropriately
and reset the counter |page| at the start of each child file.

%%%%%%%%%%%%%%%%%%%%%%%%%%%%%%%%%%%%%%%%%%%%%%%%%%%%%%%%%%%%%%%%%%%%%%%%%%%%%%%%
\subsection{Conditional Processing}
\label{sec:conditional}

The package provides a mechanism to compile different versions
of a document. To customise the versions further some conditional processing
can come in handy to distinguish which version is being compiled.
The package provides two macros to describe the compilation context:

%%%%%%%%%%%%%%%%%%%%%%%%%%%%%%%%%%%%%%%%
\DescribeMacro{\ifchilddoc}
The conditional |\ifchilddoc| distinguishes between the compilation of
child documents and the main document:
%
\begin{center}
|\ifchilddoc |\textit{child-code}| |[|\||else |\textit{main-code}]| \||fi|
\end{center}

%%%%%%%%%%%%%%%%%%%%%%%%%%%%%%%%%%%%%%%%
\DescribeMacro{\childdocname}
\DescribeMacro{\childdocjob}
The macro |\childdocname| contains the filename (without extension)
of the main or child file being processed.
Note that |\childdocjob| will always contain the name of the main file.

%%%%%%%%%%%%%%%%%%%%%%%%%%%%%%%%%%%%%%%%
\paragraph{Title Page.}

Conditional processing can be used to include a title or banner page
in the main document when proper precautions are taken.
Importantly, the code in the main file should ensure that the page counter
(as well as other status parameters which are stored in the |.aux| files)
takes the same value after the conditional processing.
Otherwise the page numbers may take divergent values
depending on which part is compiled.

For example, a title page could be declared by:
%
\begin{center}
\begin{tabular}{l}
|\ifchilddoc\||else|\\
|\addtocounter{page}{-1}|\\
\textit{code for title page}\\
|\newpage|\\
|\||fi|
\end{tabular}
\end{center}
%
A banner page for the child documents can be generated by:
%
\begin{center}
\begin{tabular}{l}
|\ifchilddoc|\\
|\addtocounter{page}{-1}|\\
\textit{code for banner page}\\
|\newpage|\\
|\||fi|
\end{tabular}
\end{center}
%
Here one could write a message such as:
\begin{center}
|This is the part \childdocname{} of \childdocjob{}.|
\end{center}

%%%%%%%%%%%%%%%%%%%%%%%%%%%%%%%%%%%%%%%%%%%%%%%%%%%%%%%%%%%%%%%%%%%%%%%%%%%%%%%%
\subsection{Flags}
\label{sec:flags}

The package makes it easy to generate different versions
of the main or child documents.
To this end compilation flags can be defined
and assigned different default values.
They will be particularly useful in conjunction
with the forwarding mechanism described in \secref{sec:forward}.

For example, it may be useful to have a flag |\version|
which can be set to |draft| or |final|.
The document source will contain some conditional code
depending on the value of |\version|.
Suppose further, the flag should default to |final| for the main file
and to |draft| for child files
which is a natural assignment for editing the document.
This is achieved by placing the following code
in the preamble of the main document
(below the |\childdocmain| directive):
%
\begin{center}
\begin{tabular}{l}
|\ifchilddoc|\\
|\providecommand{\version}{draft}|\\
|\||else|\\
|\providecommand{\version}{final}|\\
|\||fi|
\end{tabular}
\end{center}
%
The definition by |\providecommand| makes sure
that previous definitions are not overwritten.
Further statements |\providecommand{\version}{...}|
can thus be added before the above code to override it.

For the main file, one might add a line
(between |\childdocmain| and the above block)
%
\begin{center}
|%\ifchilddoc\||else\providecommand{\version}{draft}\||fi|
\end{center}
%
which can be uncommented to produce a draft version.
Likewise one can add a line to the very top of a child file
(above the |\childdocof{|\textit{main}|}| directive)
%
\begin{center}
|%\providecommand{\version}{final}|
\end{center}
%
which can be uncommented to produce the final version of this child document.

%%%%%%%%%%%%%%%%%%%%%%%%%%%%%%%%%%%%%%%%%%%%%%%%%%%%%%%%%%%%%%%%%%%%%%%%%%%%%%%%
\subsection{Forwarding}
\label{sec:forward}

Different versions of the main or child documents
using compilation flags as described in \secref{sec:flags}
can be (permanently) stored in different files
for convenient compilation, viewing and distribution.
To this end, the package defines a command
to pass on compilation to a different file:

%%%%%%%%%%%%%%%%%%%%%%%%%%%%%%%%%%%%%%%%
\DescribeMacro{\childdocforward}
The command |\childdocforward| redirects processing to
another source file:
%
\begin{center}
\begin{tabular}{l}
|\input{childdoc.def}|\\
|\childdocforward[|\textit{main}|]{|\textit{dest}|}|\\
\end{tabular}
\end{center}
%
The argument \textit{dest} is the destination file
(without extension).
It should be the main file or one of the child files.
Note that further \textsf{childdoc} directives
such as |\childdocof| and |\childdocforward|
in the indicated file will be processed in this form.
The optional argument \textit{main}
passes on directly to the main file \textit{main}
while pretending to compile the child \textit{dest}.
This form behaves as if \textit{dest}
issues |\childdocof{|\textit{main}|}| right away,
and no further \textsf{childdoc} directives will be processed.

%%%%%%%%%%%%%%%%%%%%%%%%%%%%%%%%%%%%%%%%
\DescribeMacro{\...prefix}
In the alternative form |\childdocforwardprefix|,
%
\begin{center}
\begin{tabular}{l}
|\input{childdoc.def}|\\
|\childdocforwardprefix[|\textit{main}|]{|\textit{prefix}|}{|\textit{dest}|}|
\end{tabular}
\end{center}
%
the destination file is determined by a pattern
depending on the current file:
To make this work, the current file must be called
`{\textit{prefix}\hspace{0.2em}\textit{suffix}}'
with \textit{prefix} matching precisely the argument.
Processing is then passed on to the file
`{\textit{dest}\hspace{0.2em}\textit{suffix}}'.
Surely, the same effect is achieved by
directly specifying the
argument `{\textit{dest}\hspace{0.2em}\textit{suffix}}'
in the first form.
However, that requires to set up a different file
for each child. With the alternative form of the command
all these files can have exactly the same content
which simplifies setting them up and maintaining them.

For example, the following file |draft.tex|
with a compilation flag |\version| as described in \secref{sec:flags}
compiles the main document as a draft:
%
\begin{center}
\begin{tabular}{l}
|\def\version{draft}|\\
|\input{childdoc.def}|\\
|\childdocforward{|\textit{main}|}|
\end{tabular}
\end{center}
%
Likewise, the following files |final|\textit{nn}|.tex|
compile the final version of the child document
|child|\textit{nn}|.tex|:
%
\begin{center}
\begin{tabular}{l}
|\def\version{final}|\\
|\input{childdoc.def}|\\
|\childdocforwardprefix{final}{child}|
\end{tabular}
\end{center}
%

Note that when several versions of a main file and/or of each child file
are to be generated, it may be convenient to set up a |Makefile| or
shell script to automatise the process.

%%%%%%%%%%%%%%%%%%%%%%%%%%%%%%%%%%%%%%%%%%%%%%%%%%%%%%%%%%%%%%%%%%%%%%%%%%%%%%%%
\subsection{Command Line Processing}
\label{sec:commandline}

The effect of redirection files can also be achieved by invoking
the \LaTeX{} compiler with a more elaborate command line.
Most conveniently this should be done as part
of a shell script or a |Makefile|.

When using \textsf{childdoc} in the main file, the following
command lines effectively perform a redirection
(note that depending on the shell being used,
backslashes may have to be doubled: `|\|' $\to$ `|\\|'):
%
\begin{center}
|... -jobname "|\textit{target}|" |\\|"|[\textit{flags}]%
|\input{childdoc.def}\childdocforward[|\textit{main}|]{|\textit{dest}|}"|
\end{center}
%
Here \textit{target} is the name of the output file,
\textit{main} is the name of the main file
and \textit{dest} is the name of the main or child file to be processed
(all filenames without extensions).
The optional argument \textit{main} can be omitted
if \textit{main} matches \textit{dest}.
Optionally, compilation \textit{flags} can be defined via |\def| commands.
This command line makes the \TeX{} engine believe
it is compiling the file \textit{target}
whose content is specified as the latter parameter.
The provided code then forwards the processing to
\textit{main} or \textit{dest} as described in \secref{sec:forward}.

%%%%%%%%%%%%%%%%%%%%%%%%%%%%%%%%%%%%%%%%%%%%%%%%%%%%%%%%%%%%%%%%%%%%%%%%%%%%%%%%
\subsection{Include by Input}
\label{sec:input}

Including child documents by |\include| has some restrictions by design.
Most notably, the content of a child document always occupies
its own set of pages; pages cannot be shared between child documents.
Usually, this behaviour makes perfect sense
because each child document contain an essential part of the document.
However, in some situations it may be desirable to compose
a document from a collection of parts
without having mandatory page breaks between then.
For this case, the package
provides a mechanism to include parts
by |\input| which can also be processed individually.
However, by construction this mechanism
requires manual handling of the content to be output.

%%%%%%%%%%%%%%%%%%%%%%%%%%%%%%%%%%%%%%%%
\DescribeMacro{\ifchilddocmanual}
The main file should be prepared as usual, see \secref{sec:include}.
However, the document body must make a distinction
between processing of an individual part and of the main document, e.g.:
%
\begin{center}
\begin{tabular}{l}
|\ifchilddocmanual|\\
|\input{\childdocname}|\\
|\||else|\\
\textit{document body with }|\input{|\textit{part}|}|\\
|\||fi|
\end{tabular}
\end{center}
%
The conditional |\ifchilddocmanual| is true whenever
a part to be included by |\input| is being compiled,
and the name of the part is stored in |\childdocname|.

%%%%%%%%%%%%%%%%%%%%%%%%%%%%%%%%%%%%%%%%
\DescribeMacro{\childdocby}
Each part to be included by |\input| should start with:
%
\begin{center}
\begin{tabular}{l}
|\input{childdoc.def}|\\
|\childdocby{|\textit{main}|}|\\
\end{tabular}
\end{center}
%
The directive |\childdocby| is similar to |\childdocof|
described in \secref{sec:include},
but the subsequent selection of content must be done manually.
To that end, both |\ifchilddoc| and |\ifchilddocmanual|
will be true upon processing of a part,
and the name of the part is stored in |\childdocname|.
Note that |\jobname| will be set to the filename of the current part
so that each part receives an individual |.aux| file
that does not interfere with the |.aux| file(s) of the main document.
This behaviour can be altered by the alternative form
|\childdocby[*]{|\textit{main}|}| (with a non-empty optional argument)
which uses the |.aux| file of the main document
by setting |\jobname| to \textit{main}.

%%%%%%%%%%%%%%%%%%%%%%%%%%%%%%%%%%%%%%%%%%%%%%%%%%%%%%%%%%%%%%%%%%%%%%%%%%%%%%%%
\subsection{Driver Development}
\label{sec:driver}

The \textsf{childdoc} mechanism can also be use for the development
of definition files such as \LaTeX{} styles or classes.
This case differs from the above setup with multiple parts
included by |\include| in that no |\includeonly| should be invoked.
This can be achieved by starting the include file
(before |\ProvidesPackage|) with:
%
\begin{center}
\begin{tabular}{l}
|\input{childdoc.def}|\\
|\childdocforward{|\textit{main}|}|\\
\end{tabular}
\end{center}
%
or alternatively with:
%
\begin{center}
\begin{tabular}{l}
|\input{childdoc.def}|\\
|\childdocby{|\textit{main}|}|\\
\end{tabular}
\end{center}
%
Both forms have slightly different effects as described above.
The main file is prepared as usual, see \secref{sec:include}.

%%%%%%%%%%%%%%%%%%%%%%%%%%%%%%%%%%%%%%%%%%%%%%%%%%%%%%%%%%%%%%%%%%%%%%%%%%%%%%%%
\subsection{Legacy Detection}
\label{sec:detection}

The directive |\childdocmain| in the main file can detect
whether the complete document or merely a child is to be compiled
even without using the directive |\childdocof|.
This method is deprecated because it is less robust
and there is no compelling reason to use it;
it is merely provided for backward compatibility
and it may be removed in future versions.

If the detection mechanism is to be used,
it is mandatory to correctly specify
the filename of the main file as the argument of |\childdocmain|:
%
\begin{center}
\begin{tabular}{l}
|\input{childdoc.def}|\\
|\childdocmain{|\textit{main}|}|\\
\end{tabular}
\end{center}
%
If |\jobname| does not match the argument \textit{main} of |\childdocmain|,
it is assumed that |\jobname| points to the child file to be compiled.
When using |\childdocmain| with the main file specified as argument,
it suffices to start a child file
with just |\input{|\textit{main}|}|
without loading of the package and using |\childdocof|.
If instead all processing is done
with the appropriate \textsf{childdoc} directives,
the argument of \textit{main} of |\childdocmain| can be empty.

An alternative version of the command line processing described
in \secref{sec:commandline} using the detection mechanism reads:
%
\begin{center}
|... -jobname "|\textit{target}|" "|[\textit{flags}]%
[|\def\jobname{|\textit{dest}|}|]|\input{|\textit{main}|}"|
\end{center}

%%%%%%%%%%%%%%%%%%%%%%%%%%%%%%%%%%%%%%%%%%%%%%%%%%%%%%%%%%%%%%%%%%%%%%%%%%%%%%%%
\subsection{Manual Code}
\label{sec:manual}

In case one cannot be certain whether the definitions file |childdoc.def|
is installed on the target \TeX{} distribution
and one prefers not to ship it,
it is conceivable to paste a few relevant commands into the sources.

To that end, drop all statements |\input{childdoc.def}|
and perform the replacements as outlined below.
Instead of |\childdocmain{|\textit{main}|}| add the following code
to the top of the main file:
%
\begin{center}
\begin{tabular}{l}
|\||ifdefined\childdocname\endinput\||fi\newif\ifchilddoc|\\
|\edef\childdocname{\scantokens\expandafter{\jobname\noexpand}}|\\
|\def\childdocmain{|\textit{main}|}\||ifx\childdocmain\childdocname\||else|\\
|\childdoctrue\includeonly{\childdocname}\let\jobname\childdocmain\||fi|\\
\end{tabular}
\end{center}
%
Instead of |\childdocof{|\textit{main}|}| just include the main file
at the top of each child file:
%
\begin{center}
|\input{|\textit{main}|}|
\end{center}
%
A simple redirection |\childdocforward{|\textit{dest}|}| is achieved by:
%
\begin{center}
|\def\jobname{|\textit{dest}|}\input{\jobname}|
\end{center}
%
The redirection with prefix
|\childdocforwardprefix[|\textit{prefix}|]{|\textit{dest}|}|
is accomplished by:
%
\begin{center}
\begin{tabular}{l}
|{\edef\jobname{\scantokens\expandafter{\jobname\noexpand}}|\\
|\def\redirectjob |\textit{prefix}|#1~~~{\gdef\jobname{|\textit{dest}|#1}}|\\
|\expandafter\redirectjob\jobname~~~}\input{\jobname}|
\end{tabular}
\end{center}

In an alternative approach,
child documents can be compiled by a specific command line
without additional code or specific definitions:
%
\begin{center}
|... -jobname "|\textit{target}|" "|[\textit{flags}]%
|\includeonly{|\textit{dest}|}\input{|\textit{main}|}"|
\end{center}
%

%%%%%%%%%%%%%%%%%%%%%%%%%%%%%%%%%%%%%%%%%%%%%%%%%%%%%%%%%%%%%%%%%%%%%%%%%%%%%%%%
%%%%%%%%%%%%%%%%%%%%%%%%%%%%%%%%%%%%%%%%%%%%%%%%%%%%%%%%%%%%%%%%%%%%%%%%%%%%%%%%
\section{Information}

%%%%%%%%%%%%%%%%%%%%%%%%%%%%%%%%%%%%%%%%%%%%%%%%%%%%%%%%%%%%%%%%%%%%%%%%%%%%%%%%
\subsection{Copyright}

Copyright \copyright{} 2017--2018 Niklas Beisert

This work may be distributed and/or modified under the
conditions of the \LaTeX{} Project Public License, either version 1.3
of this license or (at your option) any later version.
The latest version of this license is in
  \url{http://www.latex-project.org/lppl.txt}
and version 1.3 or later is part of all distributions of \LaTeX{}
version 2005/12/01 or later.

This work has the LPPL maintenance status `maintained'.

The Current Maintainer of this work is Niklas Beisert.

This work consists of the files |README.txt|, |childdoc.ins| and |childdoc.dtx|
as well as the derived files |childdoc.def|, |cdocsamp.tex|
with |cdocsch1.tex|, |cdocsch2.tex|, |cdocspt3.tex|, |cdocspt4.tex|,
|cdocsdrf.tex|, |cdocsfn1.tex|, |cdocsfn2.tex|
as well as |childdoc.pdf|.

%%%%%%%%%%%%%%%%%%%%%%%%%%%%%%%%%%%%%%%%%%%%%%%%%%%%%%%%%%%%%%%%%%%%%%%%%%%%%%%%
\subsection{Files and Installation}

The package consists of the files:
%
\begin{center}
\begin{tabular}{ll}
    |README.txt|   & readme file \\
    |childdoc.ins| & installation file \\
    |childdoc.dtx| & source file \\
    |childdoc.def| & definition file \\
    |cdocsamp.tex| & sample main file \\
    |cdocsch1.tex| & sample include file \\
    |cdocsch2.tex| & sample include file \\
    |cdocspt3.tex| & sample part file \\
    |cdocspt4.tex| & sample part file \\
    |cdocsdrf.tex| & sample redirection file \\
    |cdocsfn1.tex| & sample redirection file \\
    |cdocsfn2.tex| & sample redirection file \\
    |childdoc.pdf| & manual
\end{tabular}
\end{center}
%
The distribution consists of the files
|README.txt|, |childdoc.ins| and |childdoc.dtx|.
%
\begin{itemize}
\item
Run (pdf)\LaTeX{} on |childdoc.dtx|
to compile the manual |childdoc.pdf| (this file).
\item
Run \LaTeX{} on |childdoc.ins| to create the definitions file |childdoc.def|
and the sample |cdocsamp.tex| with include files
|cdocsch1.tex|, |cdocsch2.tex|, |cdocspt3.tex|, |cdocspt4.tex|,
|cdocsdrf.tex|, |cdocsfn1.tex|, |cdocsfn2.tex|.
Then copy the file |childdoc.def| to an appropriate directory of your \LaTeX{}
distribution, e.g.\ \textit{texmf-root}|/tex/latex/childdoc|.
\end{itemize}

%%%%%%%%%%%%%%%%%%%%%%%%%%%%%%%%%%%%%%%%%%%%%%%%%%%%%%%%%%%%%%%%%%%%%%%%%%%%%%%%
\subsection{Related CTAN Packages}

There are several other packages which offer a similar functionality:
%
\begin{itemize}
\item
The packages
\href{http://ctan.org/pkg/docmute}{\textsf{docmute}},
\href{http://ctan.org/pkg/includex}{\textsf{includex}} and
\href{http://ctan.org/pkg/standalone}{\textsf{standalone}}
provide commands to include only the document body of
a child file thus allowing both files to be compiled individually.
\item
The packages \href{http://ctan.org/pkg/subdocs}{\textsf{subdocs}}
and \href{http://ctan.org/pkg/subfiles}{\textsf{subfiles}}
provide structures in which the main and child documents can be
encapsulated and allowing them to be compiled individually.
The inclusion mechanism is different from the conventional |\include|.
\item
The package \href{http://ctan.org/pkg/combine}{\textsf{combine}}
is an elaborate solution to combine several documents into one.
\end{itemize}
%
See also the CTAN topic \href{http://ctan.org/topic/subdocs}{\textsf{subdocs}}
for further related packages.
The present package differs from the above solutions in that
a document structure constructed with the conventional |\include| mechanism
just needs two extra commands at the top of every file
such that all constituent files can be compiled individually.

%%%%%%%%%%%%%%%%%%%%%%%%%%%%%%%%%%%%%%%%%%%%%%%%%%%%%%%%%%%%%%%%%%%%%%%%%%%%%%%%
%\subsection{Feature Suggestions}
%
%The following is a list of features which may be useful for future
%versions of this package:
%%
%\begin{itemize}
%\item
%\ldots
%\end{itemize}

%%%%%%%%%%%%%%%%%%%%%%%%%%%%%%%%%%%%%%%%%%%%%%%%%%%%%%%%%%%%%%%%%%%%%%%%%%%%%%%%
\subsection{Revision History}

%%%%%%%%%%%%%%%%%%%%%%%%%%%%%%%%%%%%%%%%
\paragraph{v2.0:} 2018/12/30

\begin{itemize}
\item
immediate forward processing
\item
added |\childdocby| mechanism
\item
manual restructured
\end{itemize}

%%%%%%%%%%%%%%%%%%%%%%%%%%%%%%%%%%%%%%%%
\paragraph{v1.6:} 2018/01/17

\begin{itemize}
\item
application for development of include files
\item
corrections to manual
\end{itemize}

%%%%%%%%%%%%%%%%%%%%%%%%%%%%%%%%%%%%%%%%
\paragraph{v1.5:} 2017/05/21

\begin{itemize}
\item
more complete structuring introduced
\item
|\childdocof| introduced
\item
|\childdoc| renamed to |\childdocmain|
\item
|\childredirect| renamed to |\childdocforward| and |\childdocforwardprefix|
and functionality expanded
\end{itemize}

%%%%%%%%%%%%%%%%%%%%%%%%%%%%%%%%%%%%%%%%
\paragraph{v1.0:} 2017/04/27

\begin{itemize}
\item
manual and install package
\item
first version published on CTAN
\end{itemize}

%%%%%%%%%%%%%%%%%%%%%%%%%%%%%%%%%%%%%%%%
\paragraph{v0.6:} 2017/04/26

\begin{itemize}
\item
redirection mechanism added
\end{itemize}

%%%%%%%%%%%%%%%%%%%%%%%%%%%%%%%%%%%%%%%%
\paragraph{v0.5:} 2017/04/26

\begin{itemize}
\item
functionality in definition file
\end{itemize}


%%%%%%%%%%%%%%%%%%%%%%%%%%%%%%%%%%%%%%%%%%%%%%%%%%%%%%%%%%%%%%%%%%%%%%%%%%%%%%%%
%%%%%%%%%%%%%%%%%%%%%%%%%%%%%%%%%%%%%%%%%%%%%%%%%%%%%%%%%%%%%%%%%%%%%%%%%%%%%%%%
%%%%%%%%%%%%%%%%%%%%%%%%%%%%%%%%%%%%%%%%%%%%%%%%%%%%%%%%%%%%%%%%%%%%%%%%%%%%%%%%
\appendix

\settowidth\MacroIndent{\rmfamily\scriptsize 000\ }

 \DocInput{childdoc.dtx}

\end{document}
%</driver>
% \fi
%
% %%%%%%%%%%%%%%%%%%%%%%%%%%%%%%%%%%%%%%%%%%%%%%%%%%%%%%%%%%%%%%%%%%%%%%%%%%%%%%
% %%%%%%%%%%%%%%%%%%%%%%%%%%%%%%%%%%%%%%%%%%%%%%%%%%%%%%%%%%%%%%%%%%%%%%%%%%%%%%
% \section{Sample}
%\iffalse
%<*samplemain>
%\fi
%
% The following presents a sample document
% with two chapters, two parts, a title page,
% a compile flag as well as three forwarding files to set the flag.
% It consists of eight |.tex| files:
% \begin{center}
% \begin{tabular}{ll}
% |cdocsamp.tex|&main file\\
% |cdocsch1.tex|&include file for chapter 1\\
% |cdocsch2.tex|&include file for chapter 2\\
% |cdocspt3.tex|&include file for part 3\\
% |cdocspt4.tex|&include file for part 4\\
% |cdocsdrf.tex|&forwarding file for main file in draft mode\\
% |cdocsfi1.tex|&forwarding file for final version of chapter 1\\
% |cdocsfi2.tex|&forwarding file for final version of chapter 2\\
% \end{tabular}
% \end{center}
% Each of the eight files can be compiled directly by the \LaTeX{} compiler.
%
% %%%%%%%%%%%%%%%%%%%%%%%%%%%%%%%%%%%%%%
% \paragraph{Main File.}
%
% The main file is called |cdocsamp.tex|.
%
% Load the \textsf{childdoc} definitions and
% declare the filename for the main document:
%    \begin{macrocode}
\input{childdoc.def}
\childdocmain{}
%    \end{macrocode}

% Optional override for |\version| flag:
%    \begin{macrocode}
%%\ifchilddoc\else\providecommand{\version}{draft}\fi
%    \end{macrocode}

% Define the default values for the |\version| flag
% (|final| for the main file and |draft| for childs):
%    \begin{macrocode}
\ifchilddoc
\providecommand{\version}{draft}
\else
\providecommand{\version}{final}
\fi
%    \end{macrocode}

% Load the standard document class:
%    \begin{macrocode}
\documentclass[12pt]{article}
%    \end{macrocode}

% Start the document body:
%    \begin{macrocode}
\begin{document}
%    \end{macrocode}

% Declare a title page.
% Print title, part of document being processed and version flag:
%    \begin{macrocode}
\addtocounter{page}{-1}
\begin{center}
{\LARGE\bfseries{}childdoc example\par}
\vspace{1cm}
\ifchilddoc
\ifchilddocmanual part\else chapter\fi:
`\childdocname' of `\childdocjob'\par
\else
main document: `\childdocjob'\par
\fi
version: \version\par
\end{center}
\newpage
%    \end{macrocode}

% Manually include selected file,
% otherwise process as usual:
%    \begin{macrocode}
\ifchilddocmanual
\section*{part `\childdocname'}
\input{\childdocname}
\else
%    \end{macrocode}

% Include the two chapters:
%    \begin{macrocode}
\include{cdocsch1}
\include{cdocsch2}
%    \end{macrocode}

% Include the two parts unless only chapters should be displayed:
%    \begin{macrocode}
\ifchilddoc\else
\section{part three}
\input{cdocspt3}
\section{part four}
\input{cdocspt4}
\fi
%    \end{macrocode}

% Process as usual until here:
%    \begin{macrocode}
\fi
%    \end{macrocode}

% End of document body:
%    \begin{macrocode}
\end{document}
%    \end{macrocode}
%\iffalse
%</samplemain>
%\fi
%
% %%%%%%%%%%%%%%%%%%%%%%%%%%%%%%%%%%%%%%
% \paragraph{Chapter Include Files.}
%
% The include files are called |cdocsch1.tex| and |cdocsch2.tex|.
%
%\iffalse
%<*samplechap1|samplechap2>
%\fi

% Optional override for |\version| flag:
%    \begin{macrocode}
%%\providecommand{\version}{final}
%    \end{macrocode}

% Include the main document:
%    \begin{macrocode}
\input{childdoc.def}
\childdocof{cdocsamp}
%    \end{macrocode}

%\iffalse
%</samplechap1|samplechap2>
%\fi
%
%\iffalse
%<*samplechap1>
%\fi
% Some text for chapter 1:
%    \begin{macrocode}
\section{one}
some text in chapter one
%    \end{macrocode}

%\iffalse
%</samplechap1>
%\fi
% Some text for chapter 2:
%\iffalse
%<*samplechap2>
%\fi
%    \begin{macrocode}
\section{two}
more text in chapter two
%    \end{macrocode}

%\iffalse
%</samplechap2>
%\fi
%
% %%%%%%%%%%%%%%%%%%%%%%%%%%%%%%%%%%%%%%
% \paragraph{Part Include Files.}
%
% The include files are called |cdocspt3.tex| and |cdocspt4.tex|.
%
%\iffalse
%<*samplepart3|samplepart4>
%\fi

% Optional override for |\version| flag:
%    \begin{macrocode}
%%\providecommand{\version}{final}
%    \end{macrocode}

% Include the main document:
%    \begin{macrocode}
\input{childdoc.def}
\childdocby{cdocsamp}
%    \end{macrocode}

%\iffalse
%</samplepart3|samplepart4>
%\fi
%
%\iffalse
%<*samplepart3>
%\fi
% Some text for part 3:
%    \begin{macrocode}
some text in part three
%    \end{macrocode}

%\iffalse
%</samplepart3>
%\fi
% Some text for part 4:
%\iffalse
%<*samplepart4>
%\fi
%    \begin{macrocode}
more text in part four
%    \end{macrocode}

%\iffalse
%</samplepart4>
%\fi
%
% %%%%%%%%%%%%%%%%%%%%%%%%%%%%%%%%%%%%%%
% \paragraph{Forwarding for a Complete Draft.}
%
% The following forwarding file |cdocsdrf.tex|
% compiles the main document in draft mode:
%\iffalse
%<*sampledraft>
%\fi
%    \begin{macrocode}
\def\version{draft}
\input{childdoc.def}
\childdocforward{cdocsamp}
%    \end{macrocode}

%\iffalse
%</sampledraft>
%\fi
%
% %%%%%%%%%%%%%%%%%%%%%%%%%%%%%%%%%%%%%%
% \paragraph{Forwarding for Final Version of the Chapters.}
%
% The following forwarding files |cdocsfn1.tex| and |cdocsfn2.tex|
% (with identical content)
% compile the final versions of the child documents
% |cdocsch1.tex| and |cdocsch2.tex|, respectively:
%\iffalse
%<*samplefinal>
%\fi
%    \begin{macrocode}
\def\version{final}
\input{childdoc.def}
\childdocforwardprefix[cdocsamp]{cdocsfn}{cdocsch}
%    \end{macrocode}

%\iffalse
%</samplefinal>
%\fi
%
% %%%%%%%%%%%%%%%%%%%%%%%%%%%%%%%%%%%%%%
% \paragraph{Command Line Processing.}
%
% The following three command lines generate the output files
% |cdocscld|, |cdocscl1| and |cdocscl2|
% which should be identical to
% |cdocsdrf|, |cdocsch1| and |cdocsfn2|, respectively:
% \begin{center}
% \begin{tabular}{l}
% |latex -jobname cdocscld \|\\
% |  "\def\version{draft}\input{childdoc.def}\childdocforward{cdocsamp}"|\\
% |latex -jobname cdocscl1 \|\\
% |  "\input{childdoc.def}\childdocforward[cdocsamp]{cdocsch1}"|\\
% |latex -jobname cdocscl2 \|\\
% |  "\def\version{final}\input{childdoc.def}\childdocforward{cdocsch2}"|
% \end{tabular}
% \end{center}
% Note that the trailing backslash on each first line
% merely continues the input to the second line
% (for convenient cut ant paste).
% Furthermore, the command |latex| can be replaced by any
% of its alternative versions such as |pdflatex|.
%
% %%%%%%%%%%%%%%%%%%%%%%%%%%%%%%%%%%%%%%%%%%%%%%%%%%%%%%%%%%%%%%%%%%%%%%%%%%%%%%
% %%%%%%%%%%%%%%%%%%%%%%%%%%%%%%%%%%%%%%%%%%%%%%%%%%%%%%%%%%%%%%%%%%%%%%%%%%%%%%
% \section{Implementation}
%\iffalse
%<*package>
%\fi
%
% This section describes the definitions file |childdoc.def|.

% The definitions cannot be loaded using |\usepackage| or |\RequirePackage|
% which has a mechanism to prevent loading a style file more than once.
% When loading the definitions by means of |\input|
% multiple instances have to be prevented manually:
%\iffalse
%This code needs to be before the `\ProvidesFile' directive
%which is defined at the beginning of this file.
%Therefore it is also placed there and commented out here.
%</package>
%<*discard>
%\fi
%    \begin{macrocode}
\ifdefined\childdocmain\endinput\fi
%    \end{macrocode}
%\iffalse
%</discard>
%<*package>
%\fi
%
% \macro{\ifchilddoc}
% \macro{\ifchilddocmanual}
% The conditional |\ifchilddoc| tells whether a
% child (true) or main (false) document is being compiled.
% The conditional |\ifchilddocmanual| tells whether
% the |\includeonly| mechanism is used (false) or
% the selection of child files must be performed manually (true).
% The definitions initialise to false:
%    \begin{macrocode}
\newif\ifchilddoc
\newif\ifchilddocmanual
%    \end{macrocode}

% \macro{\childdocname}
% \macro{\childdocjob}
% The macro |\childdocname| stores the name of the main document
% to be compiled. The macro |\childdocjob| stores the name of
% the document on which the \LaTeX{} compiler was originally invoked.
% The content of |\jobname| cannot be compared
% to filenames specified in the source due to different catcodes.
% The following code rescans |\jobname|, stores the result
% in |\childdocname| and saves a copy in |\childdocjob|:
%    \begin{macrocode}
\edef\childdocname{\scantokens\expandafter{\jobname\noexpand}}
\let\childdocjob\childdocname
%    \end{macrocode}

% \macro{\childdocdisable}
% The macro |\childdocdisable| prevents the main file
% from being processed more than once.
% At this stage, the main document command |\childdocmain|
% is assumed to be called once again where it should do nothing.
% Any subsequent call to it should prevent
% a secondary processing of the main document
% It overwrites the forwarding commands
% |\childdocof| and |\childdocforward|
% with empty macros to prevent further inclusions of the main document:
%    \begin{macrocode}
\newcommand{\childdocdisable}
{
  \renewcommand{\childdocmain}[1]{\renewcommand{\childdocmain}[1]{\endinput}}
  \renewcommand{\childdocof}[1]{}
  \renewcommand{\childdocby}[2][]{}
  \renewcommand{\childdocforward}[2][]{}
  \renewcommand{\childdocdisable}{}
}
%    \end{macrocode}

% \macro{\childdocmain}
% The macro |\childdocmain| is to be called at the top of the main file
% with nothing or the main filename (without extension) as argument.
% First, it breaks loops.
% If the argument is not empty and does not match |\childdocname|
% (which is set by the first inclusion of |childdoc.def|),
% |\ifchilddoc| is set to true, |\includeonly| is applied to the child file
% and |\jobname| is set to the main file
% (for proper handling of |.aux| files):
%    \begin{macrocode}
\newcommand{\childdocmain}[1]
{
  \childdocdisable\childdocmain{}
  \if?#1?\else
    \begingroup
      \def\childdoctmp{#1}
      \ifx\childdoctmp\childdocname
        \def\childdoctmp{}
      \else
        \def\childdoctmp
        {
          \childdoctrue
          \includeonly{\childdocname}
          \def\childdocjob{#1}
          \def\jobname{#1}
        }
      \fi
      \expandafter
    \endgroup
    \childdoctmp
  \fi
}
%    \end{macrocode}

% \macro{\childdocof}
% The command |\childdocof| redirects
% compilation to the main file |#1|.
%    \begin{macrocode}
\newcommand{\childdocof}[1]
{
  \childdocdisable
  \childdoctrue
  \includeonly{\childdocname}
  \def\jobname{#1}
  \def\childdocjob{#1}
  \input{#1}
}
%    \end{macrocode}

% \macro{\childdocby}
% The command |\childdocby| ....
%    \begin{macrocode}
\newcommand{\childdocby}[2][]
{
  \childdocdisable
  \childdoctrue
  \childdocmanualtrue
  \if?#1?\else
    \def\jobname{#2}
  \fi
  \def\childdocjob{#2}
  \input{#2}
  \endinput
}
%    \end{macrocode}

% \macro{\childdocforward}
% The command |\childdocforward| redirects
% compilation to the main file or
% (if the optional argument is given) a child file.
% Parameters are set as if the main file
% or a child file starting with |\childdocof| was compiled.
% Then compilation is handed over to the main file:
%    \begin{macrocode}
\newcommand{\childdocforward}[2][]
{
  \begingroup
    \if?#1?
      \def\childdoctmp
      {
        \def\childdocname{#2}
        \def\childdocjob{#2}
        \def\jobname{#2}
        \input{#2}
        \endinput
      }
    \else
      \def\childdoctmp
      {
        \childdocdisable
        \def\childdocname{#2}
        \childdoctrue
        \includeonly{#2}
        \def\childdocjob{#1}
        \def\jobname{#1}
        \input{#1}
        \endinput
      }
    \fi
    \expandafter
  \endgroup
  \childdoctmp
}
%    \end{macrocode}

% \macro{\childdocforwardprefix}
% The command |\childdocforwardprefix| redirects
% compilation to the main or a child file by means of a pattern.
% The prefix |#1| in the current filename is replaced by |#2|
% and the suffix of the current filename is kept
% (it is assumed that the filename does not contain the substring `|~~~|'
% which is used as a delimiter).
% Compilation is handed over to the new file by |\childdocforward|:
%    \begin{macrocode}
\newcommand{\childdocforwardprefix}[3][]
{
  \begingroup
    \def\childdocextract #2##1~~~{\def\childdoctmp{\childdocforward[#1]{#3##1}}}
    \expandafter\childdocextract\childdocname~~~
    \expandafter
  \endgroup
  \childdoctmp
}
%    \end{macrocode}

% \macro{\childdoc}
% The deprecated macro |\childdoc| is a legacy version of |\childdocmain|:
%    \begin{macrocode}
\newcommand{\childdoc}{\childdocmain}
%    \end{macrocode}

% \macro{\childdocredirect}
% The deprecated macro |\childdocredirect| is a legacy version
% of |\childdocforward| and |\childdocforwardprefix|:
%    \begin{macrocode}
\newcommand{\childdocredirect}[2][]
{
  \begingroup
    \if?#1?
      \def\childdoctmp{\childdocforward{#2}}
    \else
      \def\childdoctmp{\childdocforwardprefix{#1}{#2}}
    \fi
    \expandafter
  \endgroup
  \childdoctmp
}
%    \end{macrocode}

%\iffalse
%</package>
%\fi
%
\endinput

\childdocby{cdocsamp}
%    \end{macrocode}

%\iffalse
%</samplepart3|samplepart4>
%\fi
%
%\iffalse
%<*samplepart3>
%\fi
% Some text for part 3:
%    \begin{macrocode}
some text in part three
%    \end{macrocode}

%\iffalse
%</samplepart3>
%\fi
% Some text for part 4:
%\iffalse
%<*samplepart4>
%\fi
%    \begin{macrocode}
more text in part four
%    \end{macrocode}

%\iffalse
%</samplepart4>
%\fi
%
% %%%%%%%%%%%%%%%%%%%%%%%%%%%%%%%%%%%%%%
% \paragraph{Forwarding for a Complete Draft.}
%
% The following forwarding file |cdocsdrf.tex|
% compiles the main document in draft mode:
%\iffalse
%<*sampledraft>
%\fi
%    \begin{macrocode}
\def\version{draft}
% \iffalse
%
% childdoc.dtx Copyright (C) 2017-2018 Niklas Beisert
%
% This work may be distributed and/or modified under the
% conditions of the LaTeX Project Public License, either version 1.3
% of this license or (at your option) any later version.
% The latest version of this license is in
%   http://www.latex-project.org/lppl.txt
% and version 1.3 or later is part of all distributions of LaTeX
% version 2005/12/01 or later.
%
% This work has the LPPL maintenance status `maintained'.
%
% The Current Maintainer of this work is Niklas Beisert.
%
% This work consists of the files childdoc.dtx and childdoc.ins
% and the derived files childdoc.def and cdocsamp.tex with
% cdocsch1.tex, cdocsch2.tex, cdocsdrf.tex, cdocsfn1.tex, cdocsfn2.tex.
%
%<package>\ifdefined\childdocmain\endinput\fi
%<package>\ProvidesFile{childdoc.def}[2018/12/30 v2.0 child document driver]
%<samplemain>\ProvidesFile{cdocsamp.tex}[2018/12/30 v2.0 sample for childdoc]
%<*driver>
%\ProvidesFile{childdoc.drv}[2018/12/30 v2.0 childdoc reference manual file]
\PassOptionsToClass{10pt,a4paper}{article}
\documentclass{ltxdoc}

\usepackage[margin=35mm]{geometry}
\usepackage{hyperref}
\usepackage{hyperxmp}
\usepackage[usenames]{color}

\hypersetup{colorlinks=true}
\hypersetup{pdfstartview=FitH}
\hypersetup{pdfpagemode=UseNone}
\hypersetup{pdfsource={}}
\hypersetup{pdflang={en-UK}}
\hypersetup{pdfcopyright={Copyright 2017-2018 Niklas Beisert.
  This work may be distributed and/or modified under the
  conditions of the LaTeX Project Public License, either version 1.3
  of this license or (at your option) any later version.}}
\hypersetup{pdflicenseurl={http://www.latex-project.org/lppl.txt}}
\hypersetup{pdfcontactaddress={ETH Zurich, ITP, HIT K,
  Wolfgang-Pauli-Strasse 27}}
\hypersetup{pdfcontactpostcode={8093}}
\hypersetup{pdfcontactcity={Zurich}}
\hypersetup{pdfcontactcountry={Switzerland}}
\hypersetup{pdfcontactemail={nbeisert@itp.phys.ethz.ch}}
\hypersetup{pdfcontacturl={http://people.phys.ethz.ch/\xmptilde nbeisert/}}

\newcommand{\secref}[1]{\hyperref[#1]{section \ref*{#1}}}

\parskip1ex
\parindent0pt
\let\olditemize\itemize
\def\itemize{\olditemize\parskip0pt}

\begin{document}

\title{The \textsf{childdoc} Package}
\hypersetup{pdftitle={The childdoc Package}}
\author{Niklas Beisert\\[2ex]
  Institut f\"ur Theoretische Physik\\
  Eidgen\"ossische Technische Hochschule Z\"urich\\
  Wolfgang-Pauli-Strasse 27, 8093 Z\"urich, Switzerland\\[1ex]
  \href{mailto:nbeisert@itp.phys.ethz.ch}
  {\texttt{nbeisert@itp.phys.ethz.ch}}}
\hypersetup{pdfauthor={Niklas Beisert}}
\hypersetup{pdfsubject={Manual for the LaTeX2e Package childdoc}}
\date{30 December 2018, \textsf{v2.0}}
\maketitle

\begin{abstract}\noindent
\textsf{childdoc} is a \LaTeXe{} package
that enables the direct compilation
of document sections included by |\include|
to individual files.
\end{abstract}

\begingroup
\parskip0ex
\tableofcontents
\endgroup

%%%%%%%%%%%%%%%%%%%%%%%%%%%%%%%%%%%%%%%%%%%%%%%%%%%%%%%%%%%%%%%%%%%%%%%%%%%%%%%%
%%%%%%%%%%%%%%%%%%%%%%%%%%%%%%%%%%%%%%%%%%%%%%%%%%%%%%%%%%%%%%%%%%%%%%%%%%%%%%%%
\section{Introduction}

\LaTeX{} provides a mechanism to structure a large document (such as a book)
into a main file and several child files (containing the chapters)
using the |\include| command.
This mechanism is beneficial for documents
which span hundreds of pages in order to
make the source file(s) more manageable.
Moreover, compilation can be restricted to
selected child files by means of the |\includeonly| command.
The latter feature can be used to reduce the compilation time while editing
(this was significantly more useful in the earlier days of \LaTeX{})
or to generate a smaller document which is easier to navigate.
Another application of |\includeonly| is to generate
documents consisting of selected parts of the complete document.

However, there are a few drawbacks of the plain |\include| mechanism:
\begin{itemize}
\item
The child files cannot be compiled on their own,
they can only be compiled via the main file.
A naive editing environment
(such as a text editor with an option
to have the current file processed by \LaTeX)
may require one to switch to the main file before compiling;
attempting to compile the child file produces errors.
\item
The main file must be modified (each time)
to adjust the |\includeonly| command
to the present needs. This easily leaves the main file in a messy state.
\item
The generated document will always carry the filename
of the main document. This is inconvenient if
several child files are to be compiled and
to be kept for distribution.
\end{itemize}

The present package provides a simple interface
to make child files individually compilable by \LaTeX{}.
Compiling a child file then has the same effect as compiling
the main file with an |\includeonly| command
to select the appropriate child.
Moreover the generated document will carry the name of the child
rather than the main file.
This resolves all three above issues.

This feature is meant to make the editing of books,
thesis documents and lecture notes somewhat more convenient.
However, the package can also be used efficiently for
composing a series of documents (such as exercise sheets)
which are typically distributed individually.
It then assists the author in generating the individual documents
(potentially in different versions)
as well as a document containing the collected series.
Another application is in developing style files
or other kinds of included material
where compilation of the style file could redirect
to a sample or test file.

%%%%%%%%%%%%%%%%%%%%%%%%%%%%%%%%%%%%%%%%%%%%%%%%%%%%%%%%%%%%%%%%%%%%%%%%%%%%%%%%
%%%%%%%%%%%%%%%%%%%%%%%%%%%%%%%%%%%%%%%%%%%%%%%%%%%%%%%%%%%%%%%%%%%%%%%%%%%%%%%%
\section{Usage}

First of all, the package \textsf{childdoc} is \emph{not} a standard
\LaTeXe{} |.sty| style file! Therefore it needs to be invoked in
a non-standard way.

%%%%%%%%%%%%%%%%%%%%%%%%%%%%%%%%%%%%%%%%%%%%%%%%%%%%%%%%%%%%%%%%%%%%%%%%%%%%%%%%
\subsection{Included Files}
\label{sec:include}

%%%%%%%%%%%%%%%%%%%%%%%%%%%%%%%%%%%%%%%%
\DescribeMacro{\childdocmain}
To use the package, add the commands
\begin{center}
\begin{tabular}{l}
|\input{childdoc.def}|\\
|\childdocmain{}|\\
\end{tabular}
\end{center}
at the very top of the main \LaTeX{} file,
in particular \emph{before} the |\documentclass| statement!
The argument of |\childdocmain| should be left empty
(but it must be present).

%%%%%%%%%%%%%%%%%%%%%%%%%%%%%%%%%%%%%%%%
\DescribeMacro{\childdocof}
Furthermore, add the commands
\begin{center}
\begin{tabular}{l}
|\input{childdoc.def}|\\
|\childdocof{|\textit{main}|}|\\
\end{tabular}
\end{center}
at the top of every child file \textit{child}
which is included by |\include{|\textit{child}|}|
from within the main file
(or at least for those files to be compiled individually).
The argument \textit{main} must be the filename of the main file.

There are a couple of
considerations in setting up the main and child documents:

%%%%%%%%%%%%%%%%%%%%%%%%%%%%%%%%%%%%%%%%
\paragraph{Restrictions.}

Please note the following restrictions:
\begin{itemize}
\item
|\childdocmain| must be called with one argument \textit{main}
to ensure compatibility with earlier version of the package.
It must either be empty (|\childdocmain{}|)
or precisely match the filename of the main file in which it is specified.
See \secref{sec:detection} for further information.
\item
The filename \textit{main} must be specified without the |.tex| extension.
\item
The filename \textit{main} is case sensitive
(even in case-insensitive file systems)
due to internal string comparison.
\item
The argument \textit{main} should be fully expanded, it cannot be a macro.
\item
Subdirectories and special characters should be avoided in filenames.
\item
The command |\childdocmain{|\textit{main}|}| must be followed by a whitespace.
It should not be followed immediately by another command
or by a comment mark `|%|'.
This is because the \TeX{} parser reads the token immediately following
the argument of |\childdocmain| and puts it
at the beginning of every child section;
however, a white\-space is ignored.
\end{itemize}

%%%%%%%%%%%%%%%%%%%%%%%%%%%%%%%%%%%%%%%%
\paragraph{Content of Main File.}

It is advisable to place all content in the child files included by |\include|.
Any output contained in the main file will appear in all child documents
unless suppressed manually;
it cannot be suppressed automatically by the |\includeonly| directive
and thus should normally be avoided.
A method to include some content in the main file
by means of conditional processing is described in \secref{sec:conditional}.

%%%%%%%%%%%%%%%%%%%%%%%%%%%%%%%%%%%%%%%%
\paragraph{Page Numbering.}

When only a part of the document is compiled,
the appropriate numbering of pages
(as well as other status parameters)
is determined from the |.aux| files.
The latter contain information from previous passes.
However this information needs to propagate through
all intermediate child documents.
Therefore the page numbering in child documents may well
be inconsistent until the complete document is compiled at least once.

A useful (if unconventional) way to always ensure a consistent
page numbering is to restart the numbering in each child document
and denote the pages by `\textit{child}|.|\textit{page}'
where \textit{child} represents the chapter/section number of the child file.
This can be achieved by the command
|\numberwithin{page}{|\textit{child}|}|
of the \textsf{amsmath} package
where \textit{child} can be |chapter| or |section|
depending on the chosen structuring.
Alternatively, one can modify the macro |\thepage| appropriately
and reset the counter |page| at the start of each child file.

%%%%%%%%%%%%%%%%%%%%%%%%%%%%%%%%%%%%%%%%%%%%%%%%%%%%%%%%%%%%%%%%%%%%%%%%%%%%%%%%
\subsection{Conditional Processing}
\label{sec:conditional}

The package provides a mechanism to compile different versions
of a document. To customise the versions further some conditional processing
can come in handy to distinguish which version is being compiled.
The package provides two macros to describe the compilation context:

%%%%%%%%%%%%%%%%%%%%%%%%%%%%%%%%%%%%%%%%
\DescribeMacro{\ifchilddoc}
The conditional |\ifchilddoc| distinguishes between the compilation of
child documents and the main document:
%
\begin{center}
|\ifchilddoc |\textit{child-code}| |[|\||else |\textit{main-code}]| \||fi|
\end{center}

%%%%%%%%%%%%%%%%%%%%%%%%%%%%%%%%%%%%%%%%
\DescribeMacro{\childdocname}
\DescribeMacro{\childdocjob}
The macro |\childdocname| contains the filename (without extension)
of the main or child file being processed.
Note that |\childdocjob| will always contain the name of the main file.

%%%%%%%%%%%%%%%%%%%%%%%%%%%%%%%%%%%%%%%%
\paragraph{Title Page.}

Conditional processing can be used to include a title or banner page
in the main document when proper precautions are taken.
Importantly, the code in the main file should ensure that the page counter
(as well as other status parameters which are stored in the |.aux| files)
takes the same value after the conditional processing.
Otherwise the page numbers may take divergent values
depending on which part is compiled.

For example, a title page could be declared by:
%
\begin{center}
\begin{tabular}{l}
|\ifchilddoc\||else|\\
|\addtocounter{page}{-1}|\\
\textit{code for title page}\\
|\newpage|\\
|\||fi|
\end{tabular}
\end{center}
%
A banner page for the child documents can be generated by:
%
\begin{center}
\begin{tabular}{l}
|\ifchilddoc|\\
|\addtocounter{page}{-1}|\\
\textit{code for banner page}\\
|\newpage|\\
|\||fi|
\end{tabular}
\end{center}
%
Here one could write a message such as:
\begin{center}
|This is the part \childdocname{} of \childdocjob{}.|
\end{center}

%%%%%%%%%%%%%%%%%%%%%%%%%%%%%%%%%%%%%%%%%%%%%%%%%%%%%%%%%%%%%%%%%%%%%%%%%%%%%%%%
\subsection{Flags}
\label{sec:flags}

The package makes it easy to generate different versions
of the main or child documents.
To this end compilation flags can be defined
and assigned different default values.
They will be particularly useful in conjunction
with the forwarding mechanism described in \secref{sec:forward}.

For example, it may be useful to have a flag |\version|
which can be set to |draft| or |final|.
The document source will contain some conditional code
depending on the value of |\version|.
Suppose further, the flag should default to |final| for the main file
and to |draft| for child files
which is a natural assignment for editing the document.
This is achieved by placing the following code
in the preamble of the main document
(below the |\childdocmain| directive):
%
\begin{center}
\begin{tabular}{l}
|\ifchilddoc|\\
|\providecommand{\version}{draft}|\\
|\||else|\\
|\providecommand{\version}{final}|\\
|\||fi|
\end{tabular}
\end{center}
%
The definition by |\providecommand| makes sure
that previous definitions are not overwritten.
Further statements |\providecommand{\version}{...}|
can thus be added before the above code to override it.

For the main file, one might add a line
(between |\childdocmain| and the above block)
%
\begin{center}
|%\ifchilddoc\||else\providecommand{\version}{draft}\||fi|
\end{center}
%
which can be uncommented to produce a draft version.
Likewise one can add a line to the very top of a child file
(above the |\childdocof{|\textit{main}|}| directive)
%
\begin{center}
|%\providecommand{\version}{final}|
\end{center}
%
which can be uncommented to produce the final version of this child document.

%%%%%%%%%%%%%%%%%%%%%%%%%%%%%%%%%%%%%%%%%%%%%%%%%%%%%%%%%%%%%%%%%%%%%%%%%%%%%%%%
\subsection{Forwarding}
\label{sec:forward}

Different versions of the main or child documents
using compilation flags as described in \secref{sec:flags}
can be (permanently) stored in different files
for convenient compilation, viewing and distribution.
To this end, the package defines a command
to pass on compilation to a different file:

%%%%%%%%%%%%%%%%%%%%%%%%%%%%%%%%%%%%%%%%
\DescribeMacro{\childdocforward}
The command |\childdocforward| redirects processing to
another source file:
%
\begin{center}
\begin{tabular}{l}
|\input{childdoc.def}|\\
|\childdocforward[|\textit{main}|]{|\textit{dest}|}|\\
\end{tabular}
\end{center}
%
The argument \textit{dest} is the destination file
(without extension).
It should be the main file or one of the child files.
Note that further \textsf{childdoc} directives
such as |\childdocof| and |\childdocforward|
in the indicated file will be processed in this form.
The optional argument \textit{main}
passes on directly to the main file \textit{main}
while pretending to compile the child \textit{dest}.
This form behaves as if \textit{dest}
issues |\childdocof{|\textit{main}|}| right away,
and no further \textsf{childdoc} directives will be processed.

%%%%%%%%%%%%%%%%%%%%%%%%%%%%%%%%%%%%%%%%
\DescribeMacro{\...prefix}
In the alternative form |\childdocforwardprefix|,
%
\begin{center}
\begin{tabular}{l}
|\input{childdoc.def}|\\
|\childdocforwardprefix[|\textit{main}|]{|\textit{prefix}|}{|\textit{dest}|}|
\end{tabular}
\end{center}
%
the destination file is determined by a pattern
depending on the current file:
To make this work, the current file must be called
`{\textit{prefix}\hspace{0.2em}\textit{suffix}}'
with \textit{prefix} matching precisely the argument.
Processing is then passed on to the file
`{\textit{dest}\hspace{0.2em}\textit{suffix}}'.
Surely, the same effect is achieved by
directly specifying the
argument `{\textit{dest}\hspace{0.2em}\textit{suffix}}'
in the first form.
However, that requires to set up a different file
for each child. With the alternative form of the command
all these files can have exactly the same content
which simplifies setting them up and maintaining them.

For example, the following file |draft.tex|
with a compilation flag |\version| as described in \secref{sec:flags}
compiles the main document as a draft:
%
\begin{center}
\begin{tabular}{l}
|\def\version{draft}|\\
|\input{childdoc.def}|\\
|\childdocforward{|\textit{main}|}|
\end{tabular}
\end{center}
%
Likewise, the following files |final|\textit{nn}|.tex|
compile the final version of the child document
|child|\textit{nn}|.tex|:
%
\begin{center}
\begin{tabular}{l}
|\def\version{final}|\\
|\input{childdoc.def}|\\
|\childdocforwardprefix{final}{child}|
\end{tabular}
\end{center}
%

Note that when several versions of a main file and/or of each child file
are to be generated, it may be convenient to set up a |Makefile| or
shell script to automatise the process.

%%%%%%%%%%%%%%%%%%%%%%%%%%%%%%%%%%%%%%%%%%%%%%%%%%%%%%%%%%%%%%%%%%%%%%%%%%%%%%%%
\subsection{Command Line Processing}
\label{sec:commandline}

The effect of redirection files can also be achieved by invoking
the \LaTeX{} compiler with a more elaborate command line.
Most conveniently this should be done as part
of a shell script or a |Makefile|.

When using \textsf{childdoc} in the main file, the following
command lines effectively perform a redirection
(note that depending on the shell being used,
backslashes may have to be doubled: `|\|' $\to$ `|\\|'):
%
\begin{center}
|... -jobname "|\textit{target}|" |\\|"|[\textit{flags}]%
|\input{childdoc.def}\childdocforward[|\textit{main}|]{|\textit{dest}|}"|
\end{center}
%
Here \textit{target} is the name of the output file,
\textit{main} is the name of the main file
and \textit{dest} is the name of the main or child file to be processed
(all filenames without extensions).
The optional argument \textit{main} can be omitted
if \textit{main} matches \textit{dest}.
Optionally, compilation \textit{flags} can be defined via |\def| commands.
This command line makes the \TeX{} engine believe
it is compiling the file \textit{target}
whose content is specified as the latter parameter.
The provided code then forwards the processing to
\textit{main} or \textit{dest} as described in \secref{sec:forward}.

%%%%%%%%%%%%%%%%%%%%%%%%%%%%%%%%%%%%%%%%%%%%%%%%%%%%%%%%%%%%%%%%%%%%%%%%%%%%%%%%
\subsection{Include by Input}
\label{sec:input}

Including child documents by |\include| has some restrictions by design.
Most notably, the content of a child document always occupies
its own set of pages; pages cannot be shared between child documents.
Usually, this behaviour makes perfect sense
because each child document contain an essential part of the document.
However, in some situations it may be desirable to compose
a document from a collection of parts
without having mandatory page breaks between then.
For this case, the package
provides a mechanism to include parts
by |\input| which can also be processed individually.
However, by construction this mechanism
requires manual handling of the content to be output.

%%%%%%%%%%%%%%%%%%%%%%%%%%%%%%%%%%%%%%%%
\DescribeMacro{\ifchilddocmanual}
The main file should be prepared as usual, see \secref{sec:include}.
However, the document body must make a distinction
between processing of an individual part and of the main document, e.g.:
%
\begin{center}
\begin{tabular}{l}
|\ifchilddocmanual|\\
|\input{\childdocname}|\\
|\||else|\\
\textit{document body with }|\input{|\textit{part}|}|\\
|\||fi|
\end{tabular}
\end{center}
%
The conditional |\ifchilddocmanual| is true whenever
a part to be included by |\input| is being compiled,
and the name of the part is stored in |\childdocname|.

%%%%%%%%%%%%%%%%%%%%%%%%%%%%%%%%%%%%%%%%
\DescribeMacro{\childdocby}
Each part to be included by |\input| should start with:
%
\begin{center}
\begin{tabular}{l}
|\input{childdoc.def}|\\
|\childdocby{|\textit{main}|}|\\
\end{tabular}
\end{center}
%
The directive |\childdocby| is similar to |\childdocof|
described in \secref{sec:include},
but the subsequent selection of content must be done manually.
To that end, both |\ifchilddoc| and |\ifchilddocmanual|
will be true upon processing of a part,
and the name of the part is stored in |\childdocname|.
Note that |\jobname| will be set to the filename of the current part
so that each part receives an individual |.aux| file
that does not interfere with the |.aux| file(s) of the main document.
This behaviour can be altered by the alternative form
|\childdocby[*]{|\textit{main}|}| (with a non-empty optional argument)
which uses the |.aux| file of the main document
by setting |\jobname| to \textit{main}.

%%%%%%%%%%%%%%%%%%%%%%%%%%%%%%%%%%%%%%%%%%%%%%%%%%%%%%%%%%%%%%%%%%%%%%%%%%%%%%%%
\subsection{Driver Development}
\label{sec:driver}

The \textsf{childdoc} mechanism can also be use for the development
of definition files such as \LaTeX{} styles or classes.
This case differs from the above setup with multiple parts
included by |\include| in that no |\includeonly| should be invoked.
This can be achieved by starting the include file
(before |\ProvidesPackage|) with:
%
\begin{center}
\begin{tabular}{l}
|\input{childdoc.def}|\\
|\childdocforward{|\textit{main}|}|\\
\end{tabular}
\end{center}
%
or alternatively with:
%
\begin{center}
\begin{tabular}{l}
|\input{childdoc.def}|\\
|\childdocby{|\textit{main}|}|\\
\end{tabular}
\end{center}
%
Both forms have slightly different effects as described above.
The main file is prepared as usual, see \secref{sec:include}.

%%%%%%%%%%%%%%%%%%%%%%%%%%%%%%%%%%%%%%%%%%%%%%%%%%%%%%%%%%%%%%%%%%%%%%%%%%%%%%%%
\subsection{Legacy Detection}
\label{sec:detection}

The directive |\childdocmain| in the main file can detect
whether the complete document or merely a child is to be compiled
even without using the directive |\childdocof|.
This method is deprecated because it is less robust
and there is no compelling reason to use it;
it is merely provided for backward compatibility
and it may be removed in future versions.

If the detection mechanism is to be used,
it is mandatory to correctly specify
the filename of the main file as the argument of |\childdocmain|:
%
\begin{center}
\begin{tabular}{l}
|\input{childdoc.def}|\\
|\childdocmain{|\textit{main}|}|\\
\end{tabular}
\end{center}
%
If |\jobname| does not match the argument \textit{main} of |\childdocmain|,
it is assumed that |\jobname| points to the child file to be compiled.
When using |\childdocmain| with the main file specified as argument,
it suffices to start a child file
with just |\input{|\textit{main}|}|
without loading of the package and using |\childdocof|.
If instead all processing is done
with the appropriate \textsf{childdoc} directives,
the argument of \textit{main} of |\childdocmain| can be empty.

An alternative version of the command line processing described
in \secref{sec:commandline} using the detection mechanism reads:
%
\begin{center}
|... -jobname "|\textit{target}|" "|[\textit{flags}]%
[|\def\jobname{|\textit{dest}|}|]|\input{|\textit{main}|}"|
\end{center}

%%%%%%%%%%%%%%%%%%%%%%%%%%%%%%%%%%%%%%%%%%%%%%%%%%%%%%%%%%%%%%%%%%%%%%%%%%%%%%%%
\subsection{Manual Code}
\label{sec:manual}

In case one cannot be certain whether the definitions file |childdoc.def|
is installed on the target \TeX{} distribution
and one prefers not to ship it,
it is conceivable to paste a few relevant commands into the sources.

To that end, drop all statements |\input{childdoc.def}|
and perform the replacements as outlined below.
Instead of |\childdocmain{|\textit{main}|}| add the following code
to the top of the main file:
%
\begin{center}
\begin{tabular}{l}
|\||ifdefined\childdocname\endinput\||fi\newif\ifchilddoc|\\
|\edef\childdocname{\scantokens\expandafter{\jobname\noexpand}}|\\
|\def\childdocmain{|\textit{main}|}\||ifx\childdocmain\childdocname\||else|\\
|\childdoctrue\includeonly{\childdocname}\let\jobname\childdocmain\||fi|\\
\end{tabular}
\end{center}
%
Instead of |\childdocof{|\textit{main}|}| just include the main file
at the top of each child file:
%
\begin{center}
|\input{|\textit{main}|}|
\end{center}
%
A simple redirection |\childdocforward{|\textit{dest}|}| is achieved by:
%
\begin{center}
|\def\jobname{|\textit{dest}|}\input{\jobname}|
\end{center}
%
The redirection with prefix
|\childdocforwardprefix[|\textit{prefix}|]{|\textit{dest}|}|
is accomplished by:
%
\begin{center}
\begin{tabular}{l}
|{\edef\jobname{\scantokens\expandafter{\jobname\noexpand}}|\\
|\def\redirectjob |\textit{prefix}|#1~~~{\gdef\jobname{|\textit{dest}|#1}}|\\
|\expandafter\redirectjob\jobname~~~}\input{\jobname}|
\end{tabular}
\end{center}

In an alternative approach,
child documents can be compiled by a specific command line
without additional code or specific definitions:
%
\begin{center}
|... -jobname "|\textit{target}|" "|[\textit{flags}]%
|\includeonly{|\textit{dest}|}\input{|\textit{main}|}"|
\end{center}
%

%%%%%%%%%%%%%%%%%%%%%%%%%%%%%%%%%%%%%%%%%%%%%%%%%%%%%%%%%%%%%%%%%%%%%%%%%%%%%%%%
%%%%%%%%%%%%%%%%%%%%%%%%%%%%%%%%%%%%%%%%%%%%%%%%%%%%%%%%%%%%%%%%%%%%%%%%%%%%%%%%
\section{Information}

%%%%%%%%%%%%%%%%%%%%%%%%%%%%%%%%%%%%%%%%%%%%%%%%%%%%%%%%%%%%%%%%%%%%%%%%%%%%%%%%
\subsection{Copyright}

Copyright \copyright{} 2017--2018 Niklas Beisert

This work may be distributed and/or modified under the
conditions of the \LaTeX{} Project Public License, either version 1.3
of this license or (at your option) any later version.
The latest version of this license is in
  \url{http://www.latex-project.org/lppl.txt}
and version 1.3 or later is part of all distributions of \LaTeX{}
version 2005/12/01 or later.

This work has the LPPL maintenance status `maintained'.

The Current Maintainer of this work is Niklas Beisert.

This work consists of the files |README.txt|, |childdoc.ins| and |childdoc.dtx|
as well as the derived files |childdoc.def|, |cdocsamp.tex|
with |cdocsch1.tex|, |cdocsch2.tex|, |cdocspt3.tex|, |cdocspt4.tex|,
|cdocsdrf.tex|, |cdocsfn1.tex|, |cdocsfn2.tex|
as well as |childdoc.pdf|.

%%%%%%%%%%%%%%%%%%%%%%%%%%%%%%%%%%%%%%%%%%%%%%%%%%%%%%%%%%%%%%%%%%%%%%%%%%%%%%%%
\subsection{Files and Installation}

The package consists of the files:
%
\begin{center}
\begin{tabular}{ll}
    |README.txt|   & readme file \\
    |childdoc.ins| & installation file \\
    |childdoc.dtx| & source file \\
    |childdoc.def| & definition file \\
    |cdocsamp.tex| & sample main file \\
    |cdocsch1.tex| & sample include file \\
    |cdocsch2.tex| & sample include file \\
    |cdocspt3.tex| & sample part file \\
    |cdocspt4.tex| & sample part file \\
    |cdocsdrf.tex| & sample redirection file \\
    |cdocsfn1.tex| & sample redirection file \\
    |cdocsfn2.tex| & sample redirection file \\
    |childdoc.pdf| & manual
\end{tabular}
\end{center}
%
The distribution consists of the files
|README.txt|, |childdoc.ins| and |childdoc.dtx|.
%
\begin{itemize}
\item
Run (pdf)\LaTeX{} on |childdoc.dtx|
to compile the manual |childdoc.pdf| (this file).
\item
Run \LaTeX{} on |childdoc.ins| to create the definitions file |childdoc.def|
and the sample |cdocsamp.tex| with include files
|cdocsch1.tex|, |cdocsch2.tex|, |cdocspt3.tex|, |cdocspt4.tex|,
|cdocsdrf.tex|, |cdocsfn1.tex|, |cdocsfn2.tex|.
Then copy the file |childdoc.def| to an appropriate directory of your \LaTeX{}
distribution, e.g.\ \textit{texmf-root}|/tex/latex/childdoc|.
\end{itemize}

%%%%%%%%%%%%%%%%%%%%%%%%%%%%%%%%%%%%%%%%%%%%%%%%%%%%%%%%%%%%%%%%%%%%%%%%%%%%%%%%
\subsection{Related CTAN Packages}

There are several other packages which offer a similar functionality:
%
\begin{itemize}
\item
The packages
\href{http://ctan.org/pkg/docmute}{\textsf{docmute}},
\href{http://ctan.org/pkg/includex}{\textsf{includex}} and
\href{http://ctan.org/pkg/standalone}{\textsf{standalone}}
provide commands to include only the document body of
a child file thus allowing both files to be compiled individually.
\item
The packages \href{http://ctan.org/pkg/subdocs}{\textsf{subdocs}}
and \href{http://ctan.org/pkg/subfiles}{\textsf{subfiles}}
provide structures in which the main and child documents can be
encapsulated and allowing them to be compiled individually.
The inclusion mechanism is different from the conventional |\include|.
\item
The package \href{http://ctan.org/pkg/combine}{\textsf{combine}}
is an elaborate solution to combine several documents into one.
\end{itemize}
%
See also the CTAN topic \href{http://ctan.org/topic/subdocs}{\textsf{subdocs}}
for further related packages.
The present package differs from the above solutions in that
a document structure constructed with the conventional |\include| mechanism
just needs two extra commands at the top of every file
such that all constituent files can be compiled individually.

%%%%%%%%%%%%%%%%%%%%%%%%%%%%%%%%%%%%%%%%%%%%%%%%%%%%%%%%%%%%%%%%%%%%%%%%%%%%%%%%
%\subsection{Feature Suggestions}
%
%The following is a list of features which may be useful for future
%versions of this package:
%%
%\begin{itemize}
%\item
%\ldots
%\end{itemize}

%%%%%%%%%%%%%%%%%%%%%%%%%%%%%%%%%%%%%%%%%%%%%%%%%%%%%%%%%%%%%%%%%%%%%%%%%%%%%%%%
\subsection{Revision History}

%%%%%%%%%%%%%%%%%%%%%%%%%%%%%%%%%%%%%%%%
\paragraph{v2.0:} 2018/12/30

\begin{itemize}
\item
immediate forward processing
\item
added |\childdocby| mechanism
\item
manual restructured
\end{itemize}

%%%%%%%%%%%%%%%%%%%%%%%%%%%%%%%%%%%%%%%%
\paragraph{v1.6:} 2018/01/17

\begin{itemize}
\item
application for development of include files
\item
corrections to manual
\end{itemize}

%%%%%%%%%%%%%%%%%%%%%%%%%%%%%%%%%%%%%%%%
\paragraph{v1.5:} 2017/05/21

\begin{itemize}
\item
more complete structuring introduced
\item
|\childdocof| introduced
\item
|\childdoc| renamed to |\childdocmain|
\item
|\childredirect| renamed to |\childdocforward| and |\childdocforwardprefix|
and functionality expanded
\end{itemize}

%%%%%%%%%%%%%%%%%%%%%%%%%%%%%%%%%%%%%%%%
\paragraph{v1.0:} 2017/04/27

\begin{itemize}
\item
manual and install package
\item
first version published on CTAN
\end{itemize}

%%%%%%%%%%%%%%%%%%%%%%%%%%%%%%%%%%%%%%%%
\paragraph{v0.6:} 2017/04/26

\begin{itemize}
\item
redirection mechanism added
\end{itemize}

%%%%%%%%%%%%%%%%%%%%%%%%%%%%%%%%%%%%%%%%
\paragraph{v0.5:} 2017/04/26

\begin{itemize}
\item
functionality in definition file
\end{itemize}


%%%%%%%%%%%%%%%%%%%%%%%%%%%%%%%%%%%%%%%%%%%%%%%%%%%%%%%%%%%%%%%%%%%%%%%%%%%%%%%%
%%%%%%%%%%%%%%%%%%%%%%%%%%%%%%%%%%%%%%%%%%%%%%%%%%%%%%%%%%%%%%%%%%%%%%%%%%%%%%%%
%%%%%%%%%%%%%%%%%%%%%%%%%%%%%%%%%%%%%%%%%%%%%%%%%%%%%%%%%%%%%%%%%%%%%%%%%%%%%%%%
\appendix

\settowidth\MacroIndent{\rmfamily\scriptsize 000\ }

 \DocInput{childdoc.dtx}

\end{document}
%</driver>
% \fi
%
% %%%%%%%%%%%%%%%%%%%%%%%%%%%%%%%%%%%%%%%%%%%%%%%%%%%%%%%%%%%%%%%%%%%%%%%%%%%%%%
% %%%%%%%%%%%%%%%%%%%%%%%%%%%%%%%%%%%%%%%%%%%%%%%%%%%%%%%%%%%%%%%%%%%%%%%%%%%%%%
% \section{Sample}
%\iffalse
%<*samplemain>
%\fi
%
% The following presents a sample document
% with two chapters, two parts, a title page,
% a compile flag as well as three forwarding files to set the flag.
% It consists of eight |.tex| files:
% \begin{center}
% \begin{tabular}{ll}
% |cdocsamp.tex|&main file\\
% |cdocsch1.tex|&include file for chapter 1\\
% |cdocsch2.tex|&include file for chapter 2\\
% |cdocspt3.tex|&include file for part 3\\
% |cdocspt4.tex|&include file for part 4\\
% |cdocsdrf.tex|&forwarding file for main file in draft mode\\
% |cdocsfi1.tex|&forwarding file for final version of chapter 1\\
% |cdocsfi2.tex|&forwarding file for final version of chapter 2\\
% \end{tabular}
% \end{center}
% Each of the eight files can be compiled directly by the \LaTeX{} compiler.
%
% %%%%%%%%%%%%%%%%%%%%%%%%%%%%%%%%%%%%%%
% \paragraph{Main File.}
%
% The main file is called |cdocsamp.tex|.
%
% Load the \textsf{childdoc} definitions and
% declare the filename for the main document:
%    \begin{macrocode}
\input{childdoc.def}
\childdocmain{}
%    \end{macrocode}

% Optional override for |\version| flag:
%    \begin{macrocode}
%%\ifchilddoc\else\providecommand{\version}{draft}\fi
%    \end{macrocode}

% Define the default values for the |\version| flag
% (|final| for the main file and |draft| for childs):
%    \begin{macrocode}
\ifchilddoc
\providecommand{\version}{draft}
\else
\providecommand{\version}{final}
\fi
%    \end{macrocode}

% Load the standard document class:
%    \begin{macrocode}
\documentclass[12pt]{article}
%    \end{macrocode}

% Start the document body:
%    \begin{macrocode}
\begin{document}
%    \end{macrocode}

% Declare a title page.
% Print title, part of document being processed and version flag:
%    \begin{macrocode}
\addtocounter{page}{-1}
\begin{center}
{\LARGE\bfseries{}childdoc example\par}
\vspace{1cm}
\ifchilddoc
\ifchilddocmanual part\else chapter\fi:
`\childdocname' of `\childdocjob'\par
\else
main document: `\childdocjob'\par
\fi
version: \version\par
\end{center}
\newpage
%    \end{macrocode}

% Manually include selected file,
% otherwise process as usual:
%    \begin{macrocode}
\ifchilddocmanual
\section*{part `\childdocname'}
\input{\childdocname}
\else
%    \end{macrocode}

% Include the two chapters:
%    \begin{macrocode}
\include{cdocsch1}
\include{cdocsch2}
%    \end{macrocode}

% Include the two parts unless only chapters should be displayed:
%    \begin{macrocode}
\ifchilddoc\else
\section{part three}
\input{cdocspt3}
\section{part four}
\input{cdocspt4}
\fi
%    \end{macrocode}

% Process as usual until here:
%    \begin{macrocode}
\fi
%    \end{macrocode}

% End of document body:
%    \begin{macrocode}
\end{document}
%    \end{macrocode}
%\iffalse
%</samplemain>
%\fi
%
% %%%%%%%%%%%%%%%%%%%%%%%%%%%%%%%%%%%%%%
% \paragraph{Chapter Include Files.}
%
% The include files are called |cdocsch1.tex| and |cdocsch2.tex|.
%
%\iffalse
%<*samplechap1|samplechap2>
%\fi

% Optional override for |\version| flag:
%    \begin{macrocode}
%%\providecommand{\version}{final}
%    \end{macrocode}

% Include the main document:
%    \begin{macrocode}
\input{childdoc.def}
\childdocof{cdocsamp}
%    \end{macrocode}

%\iffalse
%</samplechap1|samplechap2>
%\fi
%
%\iffalse
%<*samplechap1>
%\fi
% Some text for chapter 1:
%    \begin{macrocode}
\section{one}
some text in chapter one
%    \end{macrocode}

%\iffalse
%</samplechap1>
%\fi
% Some text for chapter 2:
%\iffalse
%<*samplechap2>
%\fi
%    \begin{macrocode}
\section{two}
more text in chapter two
%    \end{macrocode}

%\iffalse
%</samplechap2>
%\fi
%
% %%%%%%%%%%%%%%%%%%%%%%%%%%%%%%%%%%%%%%
% \paragraph{Part Include Files.}
%
% The include files are called |cdocspt3.tex| and |cdocspt4.tex|.
%
%\iffalse
%<*samplepart3|samplepart4>
%\fi

% Optional override for |\version| flag:
%    \begin{macrocode}
%%\providecommand{\version}{final}
%    \end{macrocode}

% Include the main document:
%    \begin{macrocode}
\input{childdoc.def}
\childdocby{cdocsamp}
%    \end{macrocode}

%\iffalse
%</samplepart3|samplepart4>
%\fi
%
%\iffalse
%<*samplepart3>
%\fi
% Some text for part 3:
%    \begin{macrocode}
some text in part three
%    \end{macrocode}

%\iffalse
%</samplepart3>
%\fi
% Some text for part 4:
%\iffalse
%<*samplepart4>
%\fi
%    \begin{macrocode}
more text in part four
%    \end{macrocode}

%\iffalse
%</samplepart4>
%\fi
%
% %%%%%%%%%%%%%%%%%%%%%%%%%%%%%%%%%%%%%%
% \paragraph{Forwarding for a Complete Draft.}
%
% The following forwarding file |cdocsdrf.tex|
% compiles the main document in draft mode:
%\iffalse
%<*sampledraft>
%\fi
%    \begin{macrocode}
\def\version{draft}
\input{childdoc.def}
\childdocforward{cdocsamp}
%    \end{macrocode}

%\iffalse
%</sampledraft>
%\fi
%
% %%%%%%%%%%%%%%%%%%%%%%%%%%%%%%%%%%%%%%
% \paragraph{Forwarding for Final Version of the Chapters.}
%
% The following forwarding files |cdocsfn1.tex| and |cdocsfn2.tex|
% (with identical content)
% compile the final versions of the child documents
% |cdocsch1.tex| and |cdocsch2.tex|, respectively:
%\iffalse
%<*samplefinal>
%\fi
%    \begin{macrocode}
\def\version{final}
\input{childdoc.def}
\childdocforwardprefix[cdocsamp]{cdocsfn}{cdocsch}
%    \end{macrocode}

%\iffalse
%</samplefinal>
%\fi
%
% %%%%%%%%%%%%%%%%%%%%%%%%%%%%%%%%%%%%%%
% \paragraph{Command Line Processing.}
%
% The following three command lines generate the output files
% |cdocscld|, |cdocscl1| and |cdocscl2|
% which should be identical to
% |cdocsdrf|, |cdocsch1| and |cdocsfn2|, respectively:
% \begin{center}
% \begin{tabular}{l}
% |latex -jobname cdocscld \|\\
% |  "\def\version{draft}\input{childdoc.def}\childdocforward{cdocsamp}"|\\
% |latex -jobname cdocscl1 \|\\
% |  "\input{childdoc.def}\childdocforward[cdocsamp]{cdocsch1}"|\\
% |latex -jobname cdocscl2 \|\\
% |  "\def\version{final}\input{childdoc.def}\childdocforward{cdocsch2}"|
% \end{tabular}
% \end{center}
% Note that the trailing backslash on each first line
% merely continues the input to the second line
% (for convenient cut ant paste).
% Furthermore, the command |latex| can be replaced by any
% of its alternative versions such as |pdflatex|.
%
% %%%%%%%%%%%%%%%%%%%%%%%%%%%%%%%%%%%%%%%%%%%%%%%%%%%%%%%%%%%%%%%%%%%%%%%%%%%%%%
% %%%%%%%%%%%%%%%%%%%%%%%%%%%%%%%%%%%%%%%%%%%%%%%%%%%%%%%%%%%%%%%%%%%%%%%%%%%%%%
% \section{Implementation}
%\iffalse
%<*package>
%\fi
%
% This section describes the definitions file |childdoc.def|.

% The definitions cannot be loaded using |\usepackage| or |\RequirePackage|
% which has a mechanism to prevent loading a style file more than once.
% When loading the definitions by means of |\input|
% multiple instances have to be prevented manually:
%\iffalse
%This code needs to be before the `\ProvidesFile' directive
%which is defined at the beginning of this file.
%Therefore it is also placed there and commented out here.
%</package>
%<*discard>
%\fi
%    \begin{macrocode}
\ifdefined\childdocmain\endinput\fi
%    \end{macrocode}
%\iffalse
%</discard>
%<*package>
%\fi
%
% \macro{\ifchilddoc}
% \macro{\ifchilddocmanual}
% The conditional |\ifchilddoc| tells whether a
% child (true) or main (false) document is being compiled.
% The conditional |\ifchilddocmanual| tells whether
% the |\includeonly| mechanism is used (false) or
% the selection of child files must be performed manually (true).
% The definitions initialise to false:
%    \begin{macrocode}
\newif\ifchilddoc
\newif\ifchilddocmanual
%    \end{macrocode}

% \macro{\childdocname}
% \macro{\childdocjob}
% The macro |\childdocname| stores the name of the main document
% to be compiled. The macro |\childdocjob| stores the name of
% the document on which the \LaTeX{} compiler was originally invoked.
% The content of |\jobname| cannot be compared
% to filenames specified in the source due to different catcodes.
% The following code rescans |\jobname|, stores the result
% in |\childdocname| and saves a copy in |\childdocjob|:
%    \begin{macrocode}
\edef\childdocname{\scantokens\expandafter{\jobname\noexpand}}
\let\childdocjob\childdocname
%    \end{macrocode}

% \macro{\childdocdisable}
% The macro |\childdocdisable| prevents the main file
% from being processed more than once.
% At this stage, the main document command |\childdocmain|
% is assumed to be called once again where it should do nothing.
% Any subsequent call to it should prevent
% a secondary processing of the main document
% It overwrites the forwarding commands
% |\childdocof| and |\childdocforward|
% with empty macros to prevent further inclusions of the main document:
%    \begin{macrocode}
\newcommand{\childdocdisable}
{
  \renewcommand{\childdocmain}[1]{\renewcommand{\childdocmain}[1]{\endinput}}
  \renewcommand{\childdocof}[1]{}
  \renewcommand{\childdocby}[2][]{}
  \renewcommand{\childdocforward}[2][]{}
  \renewcommand{\childdocdisable}{}
}
%    \end{macrocode}

% \macro{\childdocmain}
% The macro |\childdocmain| is to be called at the top of the main file
% with nothing or the main filename (without extension) as argument.
% First, it breaks loops.
% If the argument is not empty and does not match |\childdocname|
% (which is set by the first inclusion of |childdoc.def|),
% |\ifchilddoc| is set to true, |\includeonly| is applied to the child file
% and |\jobname| is set to the main file
% (for proper handling of |.aux| files):
%    \begin{macrocode}
\newcommand{\childdocmain}[1]
{
  \childdocdisable\childdocmain{}
  \if?#1?\else
    \begingroup
      \def\childdoctmp{#1}
      \ifx\childdoctmp\childdocname
        \def\childdoctmp{}
      \else
        \def\childdoctmp
        {
          \childdoctrue
          \includeonly{\childdocname}
          \def\childdocjob{#1}
          \def\jobname{#1}
        }
      \fi
      \expandafter
    \endgroup
    \childdoctmp
  \fi
}
%    \end{macrocode}

% \macro{\childdocof}
% The command |\childdocof| redirects
% compilation to the main file |#1|.
%    \begin{macrocode}
\newcommand{\childdocof}[1]
{
  \childdocdisable
  \childdoctrue
  \includeonly{\childdocname}
  \def\jobname{#1}
  \def\childdocjob{#1}
  \input{#1}
}
%    \end{macrocode}

% \macro{\childdocby}
% The command |\childdocby| ....
%    \begin{macrocode}
\newcommand{\childdocby}[2][]
{
  \childdocdisable
  \childdoctrue
  \childdocmanualtrue
  \if?#1?\else
    \def\jobname{#2}
  \fi
  \def\childdocjob{#2}
  \input{#2}
  \endinput
}
%    \end{macrocode}

% \macro{\childdocforward}
% The command |\childdocforward| redirects
% compilation to the main file or
% (if the optional argument is given) a child file.
% Parameters are set as if the main file
% or a child file starting with |\childdocof| was compiled.
% Then compilation is handed over to the main file:
%    \begin{macrocode}
\newcommand{\childdocforward}[2][]
{
  \begingroup
    \if?#1?
      \def\childdoctmp
      {
        \def\childdocname{#2}
        \def\childdocjob{#2}
        \def\jobname{#2}
        \input{#2}
        \endinput
      }
    \else
      \def\childdoctmp
      {
        \childdocdisable
        \def\childdocname{#2}
        \childdoctrue
        \includeonly{#2}
        \def\childdocjob{#1}
        \def\jobname{#1}
        \input{#1}
        \endinput
      }
    \fi
    \expandafter
  \endgroup
  \childdoctmp
}
%    \end{macrocode}

% \macro{\childdocforwardprefix}
% The command |\childdocforwardprefix| redirects
% compilation to the main or a child file by means of a pattern.
% The prefix |#1| in the current filename is replaced by |#2|
% and the suffix of the current filename is kept
% (it is assumed that the filename does not contain the substring `|~~~|'
% which is used as a delimiter).
% Compilation is handed over to the new file by |\childdocforward|:
%    \begin{macrocode}
\newcommand{\childdocforwardprefix}[3][]
{
  \begingroup
    \def\childdocextract #2##1~~~{\def\childdoctmp{\childdocforward[#1]{#3##1}}}
    \expandafter\childdocextract\childdocname~~~
    \expandafter
  \endgroup
  \childdoctmp
}
%    \end{macrocode}

% \macro{\childdoc}
% The deprecated macro |\childdoc| is a legacy version of |\childdocmain|:
%    \begin{macrocode}
\newcommand{\childdoc}{\childdocmain}
%    \end{macrocode}

% \macro{\childdocredirect}
% The deprecated macro |\childdocredirect| is a legacy version
% of |\childdocforward| and |\childdocforwardprefix|:
%    \begin{macrocode}
\newcommand{\childdocredirect}[2][]
{
  \begingroup
    \if?#1?
      \def\childdoctmp{\childdocforward{#2}}
    \else
      \def\childdoctmp{\childdocforwardprefix{#1}{#2}}
    \fi
    \expandafter
  \endgroup
  \childdoctmp
}
%    \end{macrocode}

%\iffalse
%</package>
%\fi
%
\endinput

\childdocforward{cdocsamp}
%    \end{macrocode}

%\iffalse
%</sampledraft>
%\fi
%
% %%%%%%%%%%%%%%%%%%%%%%%%%%%%%%%%%%%%%%
% \paragraph{Forwarding for Final Version of the Chapters.}
%
% The following forwarding files |cdocsfn1.tex| and |cdocsfn2.tex|
% (with identical content)
% compile the final versions of the child documents
% |cdocsch1.tex| and |cdocsch2.tex|, respectively:
%\iffalse
%<*samplefinal>
%\fi
%    \begin{macrocode}
\def\version{final}
% \iffalse
%
% childdoc.dtx Copyright (C) 2017-2018 Niklas Beisert
%
% This work may be distributed and/or modified under the
% conditions of the LaTeX Project Public License, either version 1.3
% of this license or (at your option) any later version.
% The latest version of this license is in
%   http://www.latex-project.org/lppl.txt
% and version 1.3 or later is part of all distributions of LaTeX
% version 2005/12/01 or later.
%
% This work has the LPPL maintenance status `maintained'.
%
% The Current Maintainer of this work is Niklas Beisert.
%
% This work consists of the files childdoc.dtx and childdoc.ins
% and the derived files childdoc.def and cdocsamp.tex with
% cdocsch1.tex, cdocsch2.tex, cdocsdrf.tex, cdocsfn1.tex, cdocsfn2.tex.
%
%<package>\ifdefined\childdocmain\endinput\fi
%<package>\ProvidesFile{childdoc.def}[2018/12/30 v2.0 child document driver]
%<samplemain>\ProvidesFile{cdocsamp.tex}[2018/12/30 v2.0 sample for childdoc]
%<*driver>
%\ProvidesFile{childdoc.drv}[2018/12/30 v2.0 childdoc reference manual file]
\PassOptionsToClass{10pt,a4paper}{article}
\documentclass{ltxdoc}

\usepackage[margin=35mm]{geometry}
\usepackage{hyperref}
\usepackage{hyperxmp}
\usepackage[usenames]{color}

\hypersetup{colorlinks=true}
\hypersetup{pdfstartview=FitH}
\hypersetup{pdfpagemode=UseNone}
\hypersetup{pdfsource={}}
\hypersetup{pdflang={en-UK}}
\hypersetup{pdfcopyright={Copyright 2017-2018 Niklas Beisert.
  This work may be distributed and/or modified under the
  conditions of the LaTeX Project Public License, either version 1.3
  of this license or (at your option) any later version.}}
\hypersetup{pdflicenseurl={http://www.latex-project.org/lppl.txt}}
\hypersetup{pdfcontactaddress={ETH Zurich, ITP, HIT K,
  Wolfgang-Pauli-Strasse 27}}
\hypersetup{pdfcontactpostcode={8093}}
\hypersetup{pdfcontactcity={Zurich}}
\hypersetup{pdfcontactcountry={Switzerland}}
\hypersetup{pdfcontactemail={nbeisert@itp.phys.ethz.ch}}
\hypersetup{pdfcontacturl={http://people.phys.ethz.ch/\xmptilde nbeisert/}}

\newcommand{\secref}[1]{\hyperref[#1]{section \ref*{#1}}}

\parskip1ex
\parindent0pt
\let\olditemize\itemize
\def\itemize{\olditemize\parskip0pt}

\begin{document}

\title{The \textsf{childdoc} Package}
\hypersetup{pdftitle={The childdoc Package}}
\author{Niklas Beisert\\[2ex]
  Institut f\"ur Theoretische Physik\\
  Eidgen\"ossische Technische Hochschule Z\"urich\\
  Wolfgang-Pauli-Strasse 27, 8093 Z\"urich, Switzerland\\[1ex]
  \href{mailto:nbeisert@itp.phys.ethz.ch}
  {\texttt{nbeisert@itp.phys.ethz.ch}}}
\hypersetup{pdfauthor={Niklas Beisert}}
\hypersetup{pdfsubject={Manual for the LaTeX2e Package childdoc}}
\date{30 December 2018, \textsf{v2.0}}
\maketitle

\begin{abstract}\noindent
\textsf{childdoc} is a \LaTeXe{} package
that enables the direct compilation
of document sections included by |\include|
to individual files.
\end{abstract}

\begingroup
\parskip0ex
\tableofcontents
\endgroup

%%%%%%%%%%%%%%%%%%%%%%%%%%%%%%%%%%%%%%%%%%%%%%%%%%%%%%%%%%%%%%%%%%%%%%%%%%%%%%%%
%%%%%%%%%%%%%%%%%%%%%%%%%%%%%%%%%%%%%%%%%%%%%%%%%%%%%%%%%%%%%%%%%%%%%%%%%%%%%%%%
\section{Introduction}

\LaTeX{} provides a mechanism to structure a large document (such as a book)
into a main file and several child files (containing the chapters)
using the |\include| command.
This mechanism is beneficial for documents
which span hundreds of pages in order to
make the source file(s) more manageable.
Moreover, compilation can be restricted to
selected child files by means of the |\includeonly| command.
The latter feature can be used to reduce the compilation time while editing
(this was significantly more useful in the earlier days of \LaTeX{})
or to generate a smaller document which is easier to navigate.
Another application of |\includeonly| is to generate
documents consisting of selected parts of the complete document.

However, there are a few drawbacks of the plain |\include| mechanism:
\begin{itemize}
\item
The child files cannot be compiled on their own,
they can only be compiled via the main file.
A naive editing environment
(such as a text editor with an option
to have the current file processed by \LaTeX)
may require one to switch to the main file before compiling;
attempting to compile the child file produces errors.
\item
The main file must be modified (each time)
to adjust the |\includeonly| command
to the present needs. This easily leaves the main file in a messy state.
\item
The generated document will always carry the filename
of the main document. This is inconvenient if
several child files are to be compiled and
to be kept for distribution.
\end{itemize}

The present package provides a simple interface
to make child files individually compilable by \LaTeX{}.
Compiling a child file then has the same effect as compiling
the main file with an |\includeonly| command
to select the appropriate child.
Moreover the generated document will carry the name of the child
rather than the main file.
This resolves all three above issues.

This feature is meant to make the editing of books,
thesis documents and lecture notes somewhat more convenient.
However, the package can also be used efficiently for
composing a series of documents (such as exercise sheets)
which are typically distributed individually.
It then assists the author in generating the individual documents
(potentially in different versions)
as well as a document containing the collected series.
Another application is in developing style files
or other kinds of included material
where compilation of the style file could redirect
to a sample or test file.

%%%%%%%%%%%%%%%%%%%%%%%%%%%%%%%%%%%%%%%%%%%%%%%%%%%%%%%%%%%%%%%%%%%%%%%%%%%%%%%%
%%%%%%%%%%%%%%%%%%%%%%%%%%%%%%%%%%%%%%%%%%%%%%%%%%%%%%%%%%%%%%%%%%%%%%%%%%%%%%%%
\section{Usage}

First of all, the package \textsf{childdoc} is \emph{not} a standard
\LaTeXe{} |.sty| style file! Therefore it needs to be invoked in
a non-standard way.

%%%%%%%%%%%%%%%%%%%%%%%%%%%%%%%%%%%%%%%%%%%%%%%%%%%%%%%%%%%%%%%%%%%%%%%%%%%%%%%%
\subsection{Included Files}
\label{sec:include}

%%%%%%%%%%%%%%%%%%%%%%%%%%%%%%%%%%%%%%%%
\DescribeMacro{\childdocmain}
To use the package, add the commands
\begin{center}
\begin{tabular}{l}
|\input{childdoc.def}|\\
|\childdocmain{}|\\
\end{tabular}
\end{center}
at the very top of the main \LaTeX{} file,
in particular \emph{before} the |\documentclass| statement!
The argument of |\childdocmain| should be left empty
(but it must be present).

%%%%%%%%%%%%%%%%%%%%%%%%%%%%%%%%%%%%%%%%
\DescribeMacro{\childdocof}
Furthermore, add the commands
\begin{center}
\begin{tabular}{l}
|\input{childdoc.def}|\\
|\childdocof{|\textit{main}|}|\\
\end{tabular}
\end{center}
at the top of every child file \textit{child}
which is included by |\include{|\textit{child}|}|
from within the main file
(or at least for those files to be compiled individually).
The argument \textit{main} must be the filename of the main file.

There are a couple of
considerations in setting up the main and child documents:

%%%%%%%%%%%%%%%%%%%%%%%%%%%%%%%%%%%%%%%%
\paragraph{Restrictions.}

Please note the following restrictions:
\begin{itemize}
\item
|\childdocmain| must be called with one argument \textit{main}
to ensure compatibility with earlier version of the package.
It must either be empty (|\childdocmain{}|)
or precisely match the filename of the main file in which it is specified.
See \secref{sec:detection} for further information.
\item
The filename \textit{main} must be specified without the |.tex| extension.
\item
The filename \textit{main} is case sensitive
(even in case-insensitive file systems)
due to internal string comparison.
\item
The argument \textit{main} should be fully expanded, it cannot be a macro.
\item
Subdirectories and special characters should be avoided in filenames.
\item
The command |\childdocmain{|\textit{main}|}| must be followed by a whitespace.
It should not be followed immediately by another command
or by a comment mark `|%|'.
This is because the \TeX{} parser reads the token immediately following
the argument of |\childdocmain| and puts it
at the beginning of every child section;
however, a white\-space is ignored.
\end{itemize}

%%%%%%%%%%%%%%%%%%%%%%%%%%%%%%%%%%%%%%%%
\paragraph{Content of Main File.}

It is advisable to place all content in the child files included by |\include|.
Any output contained in the main file will appear in all child documents
unless suppressed manually;
it cannot be suppressed automatically by the |\includeonly| directive
and thus should normally be avoided.
A method to include some content in the main file
by means of conditional processing is described in \secref{sec:conditional}.

%%%%%%%%%%%%%%%%%%%%%%%%%%%%%%%%%%%%%%%%
\paragraph{Page Numbering.}

When only a part of the document is compiled,
the appropriate numbering of pages
(as well as other status parameters)
is determined from the |.aux| files.
The latter contain information from previous passes.
However this information needs to propagate through
all intermediate child documents.
Therefore the page numbering in child documents may well
be inconsistent until the complete document is compiled at least once.

A useful (if unconventional) way to always ensure a consistent
page numbering is to restart the numbering in each child document
and denote the pages by `\textit{child}|.|\textit{page}'
where \textit{child} represents the chapter/section number of the child file.
This can be achieved by the command
|\numberwithin{page}{|\textit{child}|}|
of the \textsf{amsmath} package
where \textit{child} can be |chapter| or |section|
depending on the chosen structuring.
Alternatively, one can modify the macro |\thepage| appropriately
and reset the counter |page| at the start of each child file.

%%%%%%%%%%%%%%%%%%%%%%%%%%%%%%%%%%%%%%%%%%%%%%%%%%%%%%%%%%%%%%%%%%%%%%%%%%%%%%%%
\subsection{Conditional Processing}
\label{sec:conditional}

The package provides a mechanism to compile different versions
of a document. To customise the versions further some conditional processing
can come in handy to distinguish which version is being compiled.
The package provides two macros to describe the compilation context:

%%%%%%%%%%%%%%%%%%%%%%%%%%%%%%%%%%%%%%%%
\DescribeMacro{\ifchilddoc}
The conditional |\ifchilddoc| distinguishes between the compilation of
child documents and the main document:
%
\begin{center}
|\ifchilddoc |\textit{child-code}| |[|\||else |\textit{main-code}]| \||fi|
\end{center}

%%%%%%%%%%%%%%%%%%%%%%%%%%%%%%%%%%%%%%%%
\DescribeMacro{\childdocname}
\DescribeMacro{\childdocjob}
The macro |\childdocname| contains the filename (without extension)
of the main or child file being processed.
Note that |\childdocjob| will always contain the name of the main file.

%%%%%%%%%%%%%%%%%%%%%%%%%%%%%%%%%%%%%%%%
\paragraph{Title Page.}

Conditional processing can be used to include a title or banner page
in the main document when proper precautions are taken.
Importantly, the code in the main file should ensure that the page counter
(as well as other status parameters which are stored in the |.aux| files)
takes the same value after the conditional processing.
Otherwise the page numbers may take divergent values
depending on which part is compiled.

For example, a title page could be declared by:
%
\begin{center}
\begin{tabular}{l}
|\ifchilddoc\||else|\\
|\addtocounter{page}{-1}|\\
\textit{code for title page}\\
|\newpage|\\
|\||fi|
\end{tabular}
\end{center}
%
A banner page for the child documents can be generated by:
%
\begin{center}
\begin{tabular}{l}
|\ifchilddoc|\\
|\addtocounter{page}{-1}|\\
\textit{code for banner page}\\
|\newpage|\\
|\||fi|
\end{tabular}
\end{center}
%
Here one could write a message such as:
\begin{center}
|This is the part \childdocname{} of \childdocjob{}.|
\end{center}

%%%%%%%%%%%%%%%%%%%%%%%%%%%%%%%%%%%%%%%%%%%%%%%%%%%%%%%%%%%%%%%%%%%%%%%%%%%%%%%%
\subsection{Flags}
\label{sec:flags}

The package makes it easy to generate different versions
of the main or child documents.
To this end compilation flags can be defined
and assigned different default values.
They will be particularly useful in conjunction
with the forwarding mechanism described in \secref{sec:forward}.

For example, it may be useful to have a flag |\version|
which can be set to |draft| or |final|.
The document source will contain some conditional code
depending on the value of |\version|.
Suppose further, the flag should default to |final| for the main file
and to |draft| for child files
which is a natural assignment for editing the document.
This is achieved by placing the following code
in the preamble of the main document
(below the |\childdocmain| directive):
%
\begin{center}
\begin{tabular}{l}
|\ifchilddoc|\\
|\providecommand{\version}{draft}|\\
|\||else|\\
|\providecommand{\version}{final}|\\
|\||fi|
\end{tabular}
\end{center}
%
The definition by |\providecommand| makes sure
that previous definitions are not overwritten.
Further statements |\providecommand{\version}{...}|
can thus be added before the above code to override it.

For the main file, one might add a line
(between |\childdocmain| and the above block)
%
\begin{center}
|%\ifchilddoc\||else\providecommand{\version}{draft}\||fi|
\end{center}
%
which can be uncommented to produce a draft version.
Likewise one can add a line to the very top of a child file
(above the |\childdocof{|\textit{main}|}| directive)
%
\begin{center}
|%\providecommand{\version}{final}|
\end{center}
%
which can be uncommented to produce the final version of this child document.

%%%%%%%%%%%%%%%%%%%%%%%%%%%%%%%%%%%%%%%%%%%%%%%%%%%%%%%%%%%%%%%%%%%%%%%%%%%%%%%%
\subsection{Forwarding}
\label{sec:forward}

Different versions of the main or child documents
using compilation flags as described in \secref{sec:flags}
can be (permanently) stored in different files
for convenient compilation, viewing and distribution.
To this end, the package defines a command
to pass on compilation to a different file:

%%%%%%%%%%%%%%%%%%%%%%%%%%%%%%%%%%%%%%%%
\DescribeMacro{\childdocforward}
The command |\childdocforward| redirects processing to
another source file:
%
\begin{center}
\begin{tabular}{l}
|\input{childdoc.def}|\\
|\childdocforward[|\textit{main}|]{|\textit{dest}|}|\\
\end{tabular}
\end{center}
%
The argument \textit{dest} is the destination file
(without extension).
It should be the main file or one of the child files.
Note that further \textsf{childdoc} directives
such as |\childdocof| and |\childdocforward|
in the indicated file will be processed in this form.
The optional argument \textit{main}
passes on directly to the main file \textit{main}
while pretending to compile the child \textit{dest}.
This form behaves as if \textit{dest}
issues |\childdocof{|\textit{main}|}| right away,
and no further \textsf{childdoc} directives will be processed.

%%%%%%%%%%%%%%%%%%%%%%%%%%%%%%%%%%%%%%%%
\DescribeMacro{\...prefix}
In the alternative form |\childdocforwardprefix|,
%
\begin{center}
\begin{tabular}{l}
|\input{childdoc.def}|\\
|\childdocforwardprefix[|\textit{main}|]{|\textit{prefix}|}{|\textit{dest}|}|
\end{tabular}
\end{center}
%
the destination file is determined by a pattern
depending on the current file:
To make this work, the current file must be called
`{\textit{prefix}\hspace{0.2em}\textit{suffix}}'
with \textit{prefix} matching precisely the argument.
Processing is then passed on to the file
`{\textit{dest}\hspace{0.2em}\textit{suffix}}'.
Surely, the same effect is achieved by
directly specifying the
argument `{\textit{dest}\hspace{0.2em}\textit{suffix}}'
in the first form.
However, that requires to set up a different file
for each child. With the alternative form of the command
all these files can have exactly the same content
which simplifies setting them up and maintaining them.

For example, the following file |draft.tex|
with a compilation flag |\version| as described in \secref{sec:flags}
compiles the main document as a draft:
%
\begin{center}
\begin{tabular}{l}
|\def\version{draft}|\\
|\input{childdoc.def}|\\
|\childdocforward{|\textit{main}|}|
\end{tabular}
\end{center}
%
Likewise, the following files |final|\textit{nn}|.tex|
compile the final version of the child document
|child|\textit{nn}|.tex|:
%
\begin{center}
\begin{tabular}{l}
|\def\version{final}|\\
|\input{childdoc.def}|\\
|\childdocforwardprefix{final}{child}|
\end{tabular}
\end{center}
%

Note that when several versions of a main file and/or of each child file
are to be generated, it may be convenient to set up a |Makefile| or
shell script to automatise the process.

%%%%%%%%%%%%%%%%%%%%%%%%%%%%%%%%%%%%%%%%%%%%%%%%%%%%%%%%%%%%%%%%%%%%%%%%%%%%%%%%
\subsection{Command Line Processing}
\label{sec:commandline}

The effect of redirection files can also be achieved by invoking
the \LaTeX{} compiler with a more elaborate command line.
Most conveniently this should be done as part
of a shell script or a |Makefile|.

When using \textsf{childdoc} in the main file, the following
command lines effectively perform a redirection
(note that depending on the shell being used,
backslashes may have to be doubled: `|\|' $\to$ `|\\|'):
%
\begin{center}
|... -jobname "|\textit{target}|" |\\|"|[\textit{flags}]%
|\input{childdoc.def}\childdocforward[|\textit{main}|]{|\textit{dest}|}"|
\end{center}
%
Here \textit{target} is the name of the output file,
\textit{main} is the name of the main file
and \textit{dest} is the name of the main or child file to be processed
(all filenames without extensions).
The optional argument \textit{main} can be omitted
if \textit{main} matches \textit{dest}.
Optionally, compilation \textit{flags} can be defined via |\def| commands.
This command line makes the \TeX{} engine believe
it is compiling the file \textit{target}
whose content is specified as the latter parameter.
The provided code then forwards the processing to
\textit{main} or \textit{dest} as described in \secref{sec:forward}.

%%%%%%%%%%%%%%%%%%%%%%%%%%%%%%%%%%%%%%%%%%%%%%%%%%%%%%%%%%%%%%%%%%%%%%%%%%%%%%%%
\subsection{Include by Input}
\label{sec:input}

Including child documents by |\include| has some restrictions by design.
Most notably, the content of a child document always occupies
its own set of pages; pages cannot be shared between child documents.
Usually, this behaviour makes perfect sense
because each child document contain an essential part of the document.
However, in some situations it may be desirable to compose
a document from a collection of parts
without having mandatory page breaks between then.
For this case, the package
provides a mechanism to include parts
by |\input| which can also be processed individually.
However, by construction this mechanism
requires manual handling of the content to be output.

%%%%%%%%%%%%%%%%%%%%%%%%%%%%%%%%%%%%%%%%
\DescribeMacro{\ifchilddocmanual}
The main file should be prepared as usual, see \secref{sec:include}.
However, the document body must make a distinction
between processing of an individual part and of the main document, e.g.:
%
\begin{center}
\begin{tabular}{l}
|\ifchilddocmanual|\\
|\input{\childdocname}|\\
|\||else|\\
\textit{document body with }|\input{|\textit{part}|}|\\
|\||fi|
\end{tabular}
\end{center}
%
The conditional |\ifchilddocmanual| is true whenever
a part to be included by |\input| is being compiled,
and the name of the part is stored in |\childdocname|.

%%%%%%%%%%%%%%%%%%%%%%%%%%%%%%%%%%%%%%%%
\DescribeMacro{\childdocby}
Each part to be included by |\input| should start with:
%
\begin{center}
\begin{tabular}{l}
|\input{childdoc.def}|\\
|\childdocby{|\textit{main}|}|\\
\end{tabular}
\end{center}
%
The directive |\childdocby| is similar to |\childdocof|
described in \secref{sec:include},
but the subsequent selection of content must be done manually.
To that end, both |\ifchilddoc| and |\ifchilddocmanual|
will be true upon processing of a part,
and the name of the part is stored in |\childdocname|.
Note that |\jobname| will be set to the filename of the current part
so that each part receives an individual |.aux| file
that does not interfere with the |.aux| file(s) of the main document.
This behaviour can be altered by the alternative form
|\childdocby[*]{|\textit{main}|}| (with a non-empty optional argument)
which uses the |.aux| file of the main document
by setting |\jobname| to \textit{main}.

%%%%%%%%%%%%%%%%%%%%%%%%%%%%%%%%%%%%%%%%%%%%%%%%%%%%%%%%%%%%%%%%%%%%%%%%%%%%%%%%
\subsection{Driver Development}
\label{sec:driver}

The \textsf{childdoc} mechanism can also be use for the development
of definition files such as \LaTeX{} styles or classes.
This case differs from the above setup with multiple parts
included by |\include| in that no |\includeonly| should be invoked.
This can be achieved by starting the include file
(before |\ProvidesPackage|) with:
%
\begin{center}
\begin{tabular}{l}
|\input{childdoc.def}|\\
|\childdocforward{|\textit{main}|}|\\
\end{tabular}
\end{center}
%
or alternatively with:
%
\begin{center}
\begin{tabular}{l}
|\input{childdoc.def}|\\
|\childdocby{|\textit{main}|}|\\
\end{tabular}
\end{center}
%
Both forms have slightly different effects as described above.
The main file is prepared as usual, see \secref{sec:include}.

%%%%%%%%%%%%%%%%%%%%%%%%%%%%%%%%%%%%%%%%%%%%%%%%%%%%%%%%%%%%%%%%%%%%%%%%%%%%%%%%
\subsection{Legacy Detection}
\label{sec:detection}

The directive |\childdocmain| in the main file can detect
whether the complete document or merely a child is to be compiled
even without using the directive |\childdocof|.
This method is deprecated because it is less robust
and there is no compelling reason to use it;
it is merely provided for backward compatibility
and it may be removed in future versions.

If the detection mechanism is to be used,
it is mandatory to correctly specify
the filename of the main file as the argument of |\childdocmain|:
%
\begin{center}
\begin{tabular}{l}
|\input{childdoc.def}|\\
|\childdocmain{|\textit{main}|}|\\
\end{tabular}
\end{center}
%
If |\jobname| does not match the argument \textit{main} of |\childdocmain|,
it is assumed that |\jobname| points to the child file to be compiled.
When using |\childdocmain| with the main file specified as argument,
it suffices to start a child file
with just |\input{|\textit{main}|}|
without loading of the package and using |\childdocof|.
If instead all processing is done
with the appropriate \textsf{childdoc} directives,
the argument of \textit{main} of |\childdocmain| can be empty.

An alternative version of the command line processing described
in \secref{sec:commandline} using the detection mechanism reads:
%
\begin{center}
|... -jobname "|\textit{target}|" "|[\textit{flags}]%
[|\def\jobname{|\textit{dest}|}|]|\input{|\textit{main}|}"|
\end{center}

%%%%%%%%%%%%%%%%%%%%%%%%%%%%%%%%%%%%%%%%%%%%%%%%%%%%%%%%%%%%%%%%%%%%%%%%%%%%%%%%
\subsection{Manual Code}
\label{sec:manual}

In case one cannot be certain whether the definitions file |childdoc.def|
is installed on the target \TeX{} distribution
and one prefers not to ship it,
it is conceivable to paste a few relevant commands into the sources.

To that end, drop all statements |\input{childdoc.def}|
and perform the replacements as outlined below.
Instead of |\childdocmain{|\textit{main}|}| add the following code
to the top of the main file:
%
\begin{center}
\begin{tabular}{l}
|\||ifdefined\childdocname\endinput\||fi\newif\ifchilddoc|\\
|\edef\childdocname{\scantokens\expandafter{\jobname\noexpand}}|\\
|\def\childdocmain{|\textit{main}|}\||ifx\childdocmain\childdocname\||else|\\
|\childdoctrue\includeonly{\childdocname}\let\jobname\childdocmain\||fi|\\
\end{tabular}
\end{center}
%
Instead of |\childdocof{|\textit{main}|}| just include the main file
at the top of each child file:
%
\begin{center}
|\input{|\textit{main}|}|
\end{center}
%
A simple redirection |\childdocforward{|\textit{dest}|}| is achieved by:
%
\begin{center}
|\def\jobname{|\textit{dest}|}\input{\jobname}|
\end{center}
%
The redirection with prefix
|\childdocforwardprefix[|\textit{prefix}|]{|\textit{dest}|}|
is accomplished by:
%
\begin{center}
\begin{tabular}{l}
|{\edef\jobname{\scantokens\expandafter{\jobname\noexpand}}|\\
|\def\redirectjob |\textit{prefix}|#1~~~{\gdef\jobname{|\textit{dest}|#1}}|\\
|\expandafter\redirectjob\jobname~~~}\input{\jobname}|
\end{tabular}
\end{center}

In an alternative approach,
child documents can be compiled by a specific command line
without additional code or specific definitions:
%
\begin{center}
|... -jobname "|\textit{target}|" "|[\textit{flags}]%
|\includeonly{|\textit{dest}|}\input{|\textit{main}|}"|
\end{center}
%

%%%%%%%%%%%%%%%%%%%%%%%%%%%%%%%%%%%%%%%%%%%%%%%%%%%%%%%%%%%%%%%%%%%%%%%%%%%%%%%%
%%%%%%%%%%%%%%%%%%%%%%%%%%%%%%%%%%%%%%%%%%%%%%%%%%%%%%%%%%%%%%%%%%%%%%%%%%%%%%%%
\section{Information}

%%%%%%%%%%%%%%%%%%%%%%%%%%%%%%%%%%%%%%%%%%%%%%%%%%%%%%%%%%%%%%%%%%%%%%%%%%%%%%%%
\subsection{Copyright}

Copyright \copyright{} 2017--2018 Niklas Beisert

This work may be distributed and/or modified under the
conditions of the \LaTeX{} Project Public License, either version 1.3
of this license or (at your option) any later version.
The latest version of this license is in
  \url{http://www.latex-project.org/lppl.txt}
and version 1.3 or later is part of all distributions of \LaTeX{}
version 2005/12/01 or later.

This work has the LPPL maintenance status `maintained'.

The Current Maintainer of this work is Niklas Beisert.

This work consists of the files |README.txt|, |childdoc.ins| and |childdoc.dtx|
as well as the derived files |childdoc.def|, |cdocsamp.tex|
with |cdocsch1.tex|, |cdocsch2.tex|, |cdocspt3.tex|, |cdocspt4.tex|,
|cdocsdrf.tex|, |cdocsfn1.tex|, |cdocsfn2.tex|
as well as |childdoc.pdf|.

%%%%%%%%%%%%%%%%%%%%%%%%%%%%%%%%%%%%%%%%%%%%%%%%%%%%%%%%%%%%%%%%%%%%%%%%%%%%%%%%
\subsection{Files and Installation}

The package consists of the files:
%
\begin{center}
\begin{tabular}{ll}
    |README.txt|   & readme file \\
    |childdoc.ins| & installation file \\
    |childdoc.dtx| & source file \\
    |childdoc.def| & definition file \\
    |cdocsamp.tex| & sample main file \\
    |cdocsch1.tex| & sample include file \\
    |cdocsch2.tex| & sample include file \\
    |cdocspt3.tex| & sample part file \\
    |cdocspt4.tex| & sample part file \\
    |cdocsdrf.tex| & sample redirection file \\
    |cdocsfn1.tex| & sample redirection file \\
    |cdocsfn2.tex| & sample redirection file \\
    |childdoc.pdf| & manual
\end{tabular}
\end{center}
%
The distribution consists of the files
|README.txt|, |childdoc.ins| and |childdoc.dtx|.
%
\begin{itemize}
\item
Run (pdf)\LaTeX{} on |childdoc.dtx|
to compile the manual |childdoc.pdf| (this file).
\item
Run \LaTeX{} on |childdoc.ins| to create the definitions file |childdoc.def|
and the sample |cdocsamp.tex| with include files
|cdocsch1.tex|, |cdocsch2.tex|, |cdocspt3.tex|, |cdocspt4.tex|,
|cdocsdrf.tex|, |cdocsfn1.tex|, |cdocsfn2.tex|.
Then copy the file |childdoc.def| to an appropriate directory of your \LaTeX{}
distribution, e.g.\ \textit{texmf-root}|/tex/latex/childdoc|.
\end{itemize}

%%%%%%%%%%%%%%%%%%%%%%%%%%%%%%%%%%%%%%%%%%%%%%%%%%%%%%%%%%%%%%%%%%%%%%%%%%%%%%%%
\subsection{Related CTAN Packages}

There are several other packages which offer a similar functionality:
%
\begin{itemize}
\item
The packages
\href{http://ctan.org/pkg/docmute}{\textsf{docmute}},
\href{http://ctan.org/pkg/includex}{\textsf{includex}} and
\href{http://ctan.org/pkg/standalone}{\textsf{standalone}}
provide commands to include only the document body of
a child file thus allowing both files to be compiled individually.
\item
The packages \href{http://ctan.org/pkg/subdocs}{\textsf{subdocs}}
and \href{http://ctan.org/pkg/subfiles}{\textsf{subfiles}}
provide structures in which the main and child documents can be
encapsulated and allowing them to be compiled individually.
The inclusion mechanism is different from the conventional |\include|.
\item
The package \href{http://ctan.org/pkg/combine}{\textsf{combine}}
is an elaborate solution to combine several documents into one.
\end{itemize}
%
See also the CTAN topic \href{http://ctan.org/topic/subdocs}{\textsf{subdocs}}
for further related packages.
The present package differs from the above solutions in that
a document structure constructed with the conventional |\include| mechanism
just needs two extra commands at the top of every file
such that all constituent files can be compiled individually.

%%%%%%%%%%%%%%%%%%%%%%%%%%%%%%%%%%%%%%%%%%%%%%%%%%%%%%%%%%%%%%%%%%%%%%%%%%%%%%%%
%\subsection{Feature Suggestions}
%
%The following is a list of features which may be useful for future
%versions of this package:
%%
%\begin{itemize}
%\item
%\ldots
%\end{itemize}

%%%%%%%%%%%%%%%%%%%%%%%%%%%%%%%%%%%%%%%%%%%%%%%%%%%%%%%%%%%%%%%%%%%%%%%%%%%%%%%%
\subsection{Revision History}

%%%%%%%%%%%%%%%%%%%%%%%%%%%%%%%%%%%%%%%%
\paragraph{v2.0:} 2018/12/30

\begin{itemize}
\item
immediate forward processing
\item
added |\childdocby| mechanism
\item
manual restructured
\end{itemize}

%%%%%%%%%%%%%%%%%%%%%%%%%%%%%%%%%%%%%%%%
\paragraph{v1.6:} 2018/01/17

\begin{itemize}
\item
application for development of include files
\item
corrections to manual
\end{itemize}

%%%%%%%%%%%%%%%%%%%%%%%%%%%%%%%%%%%%%%%%
\paragraph{v1.5:} 2017/05/21

\begin{itemize}
\item
more complete structuring introduced
\item
|\childdocof| introduced
\item
|\childdoc| renamed to |\childdocmain|
\item
|\childredirect| renamed to |\childdocforward| and |\childdocforwardprefix|
and functionality expanded
\end{itemize}

%%%%%%%%%%%%%%%%%%%%%%%%%%%%%%%%%%%%%%%%
\paragraph{v1.0:} 2017/04/27

\begin{itemize}
\item
manual and install package
\item
first version published on CTAN
\end{itemize}

%%%%%%%%%%%%%%%%%%%%%%%%%%%%%%%%%%%%%%%%
\paragraph{v0.6:} 2017/04/26

\begin{itemize}
\item
redirection mechanism added
\end{itemize}

%%%%%%%%%%%%%%%%%%%%%%%%%%%%%%%%%%%%%%%%
\paragraph{v0.5:} 2017/04/26

\begin{itemize}
\item
functionality in definition file
\end{itemize}


%%%%%%%%%%%%%%%%%%%%%%%%%%%%%%%%%%%%%%%%%%%%%%%%%%%%%%%%%%%%%%%%%%%%%%%%%%%%%%%%
%%%%%%%%%%%%%%%%%%%%%%%%%%%%%%%%%%%%%%%%%%%%%%%%%%%%%%%%%%%%%%%%%%%%%%%%%%%%%%%%
%%%%%%%%%%%%%%%%%%%%%%%%%%%%%%%%%%%%%%%%%%%%%%%%%%%%%%%%%%%%%%%%%%%%%%%%%%%%%%%%
\appendix

\settowidth\MacroIndent{\rmfamily\scriptsize 000\ }

 \DocInput{childdoc.dtx}

\end{document}
%</driver>
% \fi
%
% %%%%%%%%%%%%%%%%%%%%%%%%%%%%%%%%%%%%%%%%%%%%%%%%%%%%%%%%%%%%%%%%%%%%%%%%%%%%%%
% %%%%%%%%%%%%%%%%%%%%%%%%%%%%%%%%%%%%%%%%%%%%%%%%%%%%%%%%%%%%%%%%%%%%%%%%%%%%%%
% \section{Sample}
%\iffalse
%<*samplemain>
%\fi
%
% The following presents a sample document
% with two chapters, two parts, a title page,
% a compile flag as well as three forwarding files to set the flag.
% It consists of eight |.tex| files:
% \begin{center}
% \begin{tabular}{ll}
% |cdocsamp.tex|&main file\\
% |cdocsch1.tex|&include file for chapter 1\\
% |cdocsch2.tex|&include file for chapter 2\\
% |cdocspt3.tex|&include file for part 3\\
% |cdocspt4.tex|&include file for part 4\\
% |cdocsdrf.tex|&forwarding file for main file in draft mode\\
% |cdocsfi1.tex|&forwarding file for final version of chapter 1\\
% |cdocsfi2.tex|&forwarding file for final version of chapter 2\\
% \end{tabular}
% \end{center}
% Each of the eight files can be compiled directly by the \LaTeX{} compiler.
%
% %%%%%%%%%%%%%%%%%%%%%%%%%%%%%%%%%%%%%%
% \paragraph{Main File.}
%
% The main file is called |cdocsamp.tex|.
%
% Load the \textsf{childdoc} definitions and
% declare the filename for the main document:
%    \begin{macrocode}
\input{childdoc.def}
\childdocmain{}
%    \end{macrocode}

% Optional override for |\version| flag:
%    \begin{macrocode}
%%\ifchilddoc\else\providecommand{\version}{draft}\fi
%    \end{macrocode}

% Define the default values for the |\version| flag
% (|final| for the main file and |draft| for childs):
%    \begin{macrocode}
\ifchilddoc
\providecommand{\version}{draft}
\else
\providecommand{\version}{final}
\fi
%    \end{macrocode}

% Load the standard document class:
%    \begin{macrocode}
\documentclass[12pt]{article}
%    \end{macrocode}

% Start the document body:
%    \begin{macrocode}
\begin{document}
%    \end{macrocode}

% Declare a title page.
% Print title, part of document being processed and version flag:
%    \begin{macrocode}
\addtocounter{page}{-1}
\begin{center}
{\LARGE\bfseries{}childdoc example\par}
\vspace{1cm}
\ifchilddoc
\ifchilddocmanual part\else chapter\fi:
`\childdocname' of `\childdocjob'\par
\else
main document: `\childdocjob'\par
\fi
version: \version\par
\end{center}
\newpage
%    \end{macrocode}

% Manually include selected file,
% otherwise process as usual:
%    \begin{macrocode}
\ifchilddocmanual
\section*{part `\childdocname'}
\input{\childdocname}
\else
%    \end{macrocode}

% Include the two chapters:
%    \begin{macrocode}
\include{cdocsch1}
\include{cdocsch2}
%    \end{macrocode}

% Include the two parts unless only chapters should be displayed:
%    \begin{macrocode}
\ifchilddoc\else
\section{part three}
\input{cdocspt3}
\section{part four}
\input{cdocspt4}
\fi
%    \end{macrocode}

% Process as usual until here:
%    \begin{macrocode}
\fi
%    \end{macrocode}

% End of document body:
%    \begin{macrocode}
\end{document}
%    \end{macrocode}
%\iffalse
%</samplemain>
%\fi
%
% %%%%%%%%%%%%%%%%%%%%%%%%%%%%%%%%%%%%%%
% \paragraph{Chapter Include Files.}
%
% The include files are called |cdocsch1.tex| and |cdocsch2.tex|.
%
%\iffalse
%<*samplechap1|samplechap2>
%\fi

% Optional override for |\version| flag:
%    \begin{macrocode}
%%\providecommand{\version}{final}
%    \end{macrocode}

% Include the main document:
%    \begin{macrocode}
\input{childdoc.def}
\childdocof{cdocsamp}
%    \end{macrocode}

%\iffalse
%</samplechap1|samplechap2>
%\fi
%
%\iffalse
%<*samplechap1>
%\fi
% Some text for chapter 1:
%    \begin{macrocode}
\section{one}
some text in chapter one
%    \end{macrocode}

%\iffalse
%</samplechap1>
%\fi
% Some text for chapter 2:
%\iffalse
%<*samplechap2>
%\fi
%    \begin{macrocode}
\section{two}
more text in chapter two
%    \end{macrocode}

%\iffalse
%</samplechap2>
%\fi
%
% %%%%%%%%%%%%%%%%%%%%%%%%%%%%%%%%%%%%%%
% \paragraph{Part Include Files.}
%
% The include files are called |cdocspt3.tex| and |cdocspt4.tex|.
%
%\iffalse
%<*samplepart3|samplepart4>
%\fi

% Optional override for |\version| flag:
%    \begin{macrocode}
%%\providecommand{\version}{final}
%    \end{macrocode}

% Include the main document:
%    \begin{macrocode}
\input{childdoc.def}
\childdocby{cdocsamp}
%    \end{macrocode}

%\iffalse
%</samplepart3|samplepart4>
%\fi
%
%\iffalse
%<*samplepart3>
%\fi
% Some text for part 3:
%    \begin{macrocode}
some text in part three
%    \end{macrocode}

%\iffalse
%</samplepart3>
%\fi
% Some text for part 4:
%\iffalse
%<*samplepart4>
%\fi
%    \begin{macrocode}
more text in part four
%    \end{macrocode}

%\iffalse
%</samplepart4>
%\fi
%
% %%%%%%%%%%%%%%%%%%%%%%%%%%%%%%%%%%%%%%
% \paragraph{Forwarding for a Complete Draft.}
%
% The following forwarding file |cdocsdrf.tex|
% compiles the main document in draft mode:
%\iffalse
%<*sampledraft>
%\fi
%    \begin{macrocode}
\def\version{draft}
\input{childdoc.def}
\childdocforward{cdocsamp}
%    \end{macrocode}

%\iffalse
%</sampledraft>
%\fi
%
% %%%%%%%%%%%%%%%%%%%%%%%%%%%%%%%%%%%%%%
% \paragraph{Forwarding for Final Version of the Chapters.}
%
% The following forwarding files |cdocsfn1.tex| and |cdocsfn2.tex|
% (with identical content)
% compile the final versions of the child documents
% |cdocsch1.tex| and |cdocsch2.tex|, respectively:
%\iffalse
%<*samplefinal>
%\fi
%    \begin{macrocode}
\def\version{final}
\input{childdoc.def}
\childdocforwardprefix[cdocsamp]{cdocsfn}{cdocsch}
%    \end{macrocode}

%\iffalse
%</samplefinal>
%\fi
%
% %%%%%%%%%%%%%%%%%%%%%%%%%%%%%%%%%%%%%%
% \paragraph{Command Line Processing.}
%
% The following three command lines generate the output files
% |cdocscld|, |cdocscl1| and |cdocscl2|
% which should be identical to
% |cdocsdrf|, |cdocsch1| and |cdocsfn2|, respectively:
% \begin{center}
% \begin{tabular}{l}
% |latex -jobname cdocscld \|\\
% |  "\def\version{draft}\input{childdoc.def}\childdocforward{cdocsamp}"|\\
% |latex -jobname cdocscl1 \|\\
% |  "\input{childdoc.def}\childdocforward[cdocsamp]{cdocsch1}"|\\
% |latex -jobname cdocscl2 \|\\
% |  "\def\version{final}\input{childdoc.def}\childdocforward{cdocsch2}"|
% \end{tabular}
% \end{center}
% Note that the trailing backslash on each first line
% merely continues the input to the second line
% (for convenient cut ant paste).
% Furthermore, the command |latex| can be replaced by any
% of its alternative versions such as |pdflatex|.
%
% %%%%%%%%%%%%%%%%%%%%%%%%%%%%%%%%%%%%%%%%%%%%%%%%%%%%%%%%%%%%%%%%%%%%%%%%%%%%%%
% %%%%%%%%%%%%%%%%%%%%%%%%%%%%%%%%%%%%%%%%%%%%%%%%%%%%%%%%%%%%%%%%%%%%%%%%%%%%%%
% \section{Implementation}
%\iffalse
%<*package>
%\fi
%
% This section describes the definitions file |childdoc.def|.

% The definitions cannot be loaded using |\usepackage| or |\RequirePackage|
% which has a mechanism to prevent loading a style file more than once.
% When loading the definitions by means of |\input|
% multiple instances have to be prevented manually:
%\iffalse
%This code needs to be before the `\ProvidesFile' directive
%which is defined at the beginning of this file.
%Therefore it is also placed there and commented out here.
%</package>
%<*discard>
%\fi
%    \begin{macrocode}
\ifdefined\childdocmain\endinput\fi
%    \end{macrocode}
%\iffalse
%</discard>
%<*package>
%\fi
%
% \macro{\ifchilddoc}
% \macro{\ifchilddocmanual}
% The conditional |\ifchilddoc| tells whether a
% child (true) or main (false) document is being compiled.
% The conditional |\ifchilddocmanual| tells whether
% the |\includeonly| mechanism is used (false) or
% the selection of child files must be performed manually (true).
% The definitions initialise to false:
%    \begin{macrocode}
\newif\ifchilddoc
\newif\ifchilddocmanual
%    \end{macrocode}

% \macro{\childdocname}
% \macro{\childdocjob}
% The macro |\childdocname| stores the name of the main document
% to be compiled. The macro |\childdocjob| stores the name of
% the document on which the \LaTeX{} compiler was originally invoked.
% The content of |\jobname| cannot be compared
% to filenames specified in the source due to different catcodes.
% The following code rescans |\jobname|, stores the result
% in |\childdocname| and saves a copy in |\childdocjob|:
%    \begin{macrocode}
\edef\childdocname{\scantokens\expandafter{\jobname\noexpand}}
\let\childdocjob\childdocname
%    \end{macrocode}

% \macro{\childdocdisable}
% The macro |\childdocdisable| prevents the main file
% from being processed more than once.
% At this stage, the main document command |\childdocmain|
% is assumed to be called once again where it should do nothing.
% Any subsequent call to it should prevent
% a secondary processing of the main document
% It overwrites the forwarding commands
% |\childdocof| and |\childdocforward|
% with empty macros to prevent further inclusions of the main document:
%    \begin{macrocode}
\newcommand{\childdocdisable}
{
  \renewcommand{\childdocmain}[1]{\renewcommand{\childdocmain}[1]{\endinput}}
  \renewcommand{\childdocof}[1]{}
  \renewcommand{\childdocby}[2][]{}
  \renewcommand{\childdocforward}[2][]{}
  \renewcommand{\childdocdisable}{}
}
%    \end{macrocode}

% \macro{\childdocmain}
% The macro |\childdocmain| is to be called at the top of the main file
% with nothing or the main filename (without extension) as argument.
% First, it breaks loops.
% If the argument is not empty and does not match |\childdocname|
% (which is set by the first inclusion of |childdoc.def|),
% |\ifchilddoc| is set to true, |\includeonly| is applied to the child file
% and |\jobname| is set to the main file
% (for proper handling of |.aux| files):
%    \begin{macrocode}
\newcommand{\childdocmain}[1]
{
  \childdocdisable\childdocmain{}
  \if?#1?\else
    \begingroup
      \def\childdoctmp{#1}
      \ifx\childdoctmp\childdocname
        \def\childdoctmp{}
      \else
        \def\childdoctmp
        {
          \childdoctrue
          \includeonly{\childdocname}
          \def\childdocjob{#1}
          \def\jobname{#1}
        }
      \fi
      \expandafter
    \endgroup
    \childdoctmp
  \fi
}
%    \end{macrocode}

% \macro{\childdocof}
% The command |\childdocof| redirects
% compilation to the main file |#1|.
%    \begin{macrocode}
\newcommand{\childdocof}[1]
{
  \childdocdisable
  \childdoctrue
  \includeonly{\childdocname}
  \def\jobname{#1}
  \def\childdocjob{#1}
  \input{#1}
}
%    \end{macrocode}

% \macro{\childdocby}
% The command |\childdocby| ....
%    \begin{macrocode}
\newcommand{\childdocby}[2][]
{
  \childdocdisable
  \childdoctrue
  \childdocmanualtrue
  \if?#1?\else
    \def\jobname{#2}
  \fi
  \def\childdocjob{#2}
  \input{#2}
  \endinput
}
%    \end{macrocode}

% \macro{\childdocforward}
% The command |\childdocforward| redirects
% compilation to the main file or
% (if the optional argument is given) a child file.
% Parameters are set as if the main file
% or a child file starting with |\childdocof| was compiled.
% Then compilation is handed over to the main file:
%    \begin{macrocode}
\newcommand{\childdocforward}[2][]
{
  \begingroup
    \if?#1?
      \def\childdoctmp
      {
        \def\childdocname{#2}
        \def\childdocjob{#2}
        \def\jobname{#2}
        \input{#2}
        \endinput
      }
    \else
      \def\childdoctmp
      {
        \childdocdisable
        \def\childdocname{#2}
        \childdoctrue
        \includeonly{#2}
        \def\childdocjob{#1}
        \def\jobname{#1}
        \input{#1}
        \endinput
      }
    \fi
    \expandafter
  \endgroup
  \childdoctmp
}
%    \end{macrocode}

% \macro{\childdocforwardprefix}
% The command |\childdocforwardprefix| redirects
% compilation to the main or a child file by means of a pattern.
% The prefix |#1| in the current filename is replaced by |#2|
% and the suffix of the current filename is kept
% (it is assumed that the filename does not contain the substring `|~~~|'
% which is used as a delimiter).
% Compilation is handed over to the new file by |\childdocforward|:
%    \begin{macrocode}
\newcommand{\childdocforwardprefix}[3][]
{
  \begingroup
    \def\childdocextract #2##1~~~{\def\childdoctmp{\childdocforward[#1]{#3##1}}}
    \expandafter\childdocextract\childdocname~~~
    \expandafter
  \endgroup
  \childdoctmp
}
%    \end{macrocode}

% \macro{\childdoc}
% The deprecated macro |\childdoc| is a legacy version of |\childdocmain|:
%    \begin{macrocode}
\newcommand{\childdoc}{\childdocmain}
%    \end{macrocode}

% \macro{\childdocredirect}
% The deprecated macro |\childdocredirect| is a legacy version
% of |\childdocforward| and |\childdocforwardprefix|:
%    \begin{macrocode}
\newcommand{\childdocredirect}[2][]
{
  \begingroup
    \if?#1?
      \def\childdoctmp{\childdocforward{#2}}
    \else
      \def\childdoctmp{\childdocforwardprefix{#1}{#2}}
    \fi
    \expandafter
  \endgroup
  \childdoctmp
}
%    \end{macrocode}

%\iffalse
%</package>
%\fi
%
\endinput

\childdocforwardprefix[cdocsamp]{cdocsfn}{cdocsch}
%    \end{macrocode}

%\iffalse
%</samplefinal>
%\fi
%
% %%%%%%%%%%%%%%%%%%%%%%%%%%%%%%%%%%%%%%
% \paragraph{Command Line Processing.}
%
% The following three command lines generate the output files
% |cdocscld|, |cdocscl1| and |cdocscl2|
% which should be identical to
% |cdocsdrf|, |cdocsch1| and |cdocsfn2|, respectively:
% \begin{center}
% \begin{tabular}{l}
% |latex -jobname cdocscld \|\\
% |  "\def\version{draft}% \iffalse
%
% childdoc.dtx Copyright (C) 2017-2018 Niklas Beisert
%
% This work may be distributed and/or modified under the
% conditions of the LaTeX Project Public License, either version 1.3
% of this license or (at your option) any later version.
% The latest version of this license is in
%   http://www.latex-project.org/lppl.txt
% and version 1.3 or later is part of all distributions of LaTeX
% version 2005/12/01 or later.
%
% This work has the LPPL maintenance status `maintained'.
%
% The Current Maintainer of this work is Niklas Beisert.
%
% This work consists of the files childdoc.dtx and childdoc.ins
% and the derived files childdoc.def and cdocsamp.tex with
% cdocsch1.tex, cdocsch2.tex, cdocsdrf.tex, cdocsfn1.tex, cdocsfn2.tex.
%
%<package>\ifdefined\childdocmain\endinput\fi
%<package>\ProvidesFile{childdoc.def}[2018/12/30 v2.0 child document driver]
%<samplemain>\ProvidesFile{cdocsamp.tex}[2018/12/30 v2.0 sample for childdoc]
%<*driver>
%\ProvidesFile{childdoc.drv}[2018/12/30 v2.0 childdoc reference manual file]
\PassOptionsToClass{10pt,a4paper}{article}
\documentclass{ltxdoc}

\usepackage[margin=35mm]{geometry}
\usepackage{hyperref}
\usepackage{hyperxmp}
\usepackage[usenames]{color}

\hypersetup{colorlinks=true}
\hypersetup{pdfstartview=FitH}
\hypersetup{pdfpagemode=UseNone}
\hypersetup{pdfsource={}}
\hypersetup{pdflang={en-UK}}
\hypersetup{pdfcopyright={Copyright 2017-2018 Niklas Beisert.
  This work may be distributed and/or modified under the
  conditions of the LaTeX Project Public License, either version 1.3
  of this license or (at your option) any later version.}}
\hypersetup{pdflicenseurl={http://www.latex-project.org/lppl.txt}}
\hypersetup{pdfcontactaddress={ETH Zurich, ITP, HIT K,
  Wolfgang-Pauli-Strasse 27}}
\hypersetup{pdfcontactpostcode={8093}}
\hypersetup{pdfcontactcity={Zurich}}
\hypersetup{pdfcontactcountry={Switzerland}}
\hypersetup{pdfcontactemail={nbeisert@itp.phys.ethz.ch}}
\hypersetup{pdfcontacturl={http://people.phys.ethz.ch/\xmptilde nbeisert/}}

\newcommand{\secref}[1]{\hyperref[#1]{section \ref*{#1}}}

\parskip1ex
\parindent0pt
\let\olditemize\itemize
\def\itemize{\olditemize\parskip0pt}

\begin{document}

\title{The \textsf{childdoc} Package}
\hypersetup{pdftitle={The childdoc Package}}
\author{Niklas Beisert\\[2ex]
  Institut f\"ur Theoretische Physik\\
  Eidgen\"ossische Technische Hochschule Z\"urich\\
  Wolfgang-Pauli-Strasse 27, 8093 Z\"urich, Switzerland\\[1ex]
  \href{mailto:nbeisert@itp.phys.ethz.ch}
  {\texttt{nbeisert@itp.phys.ethz.ch}}}
\hypersetup{pdfauthor={Niklas Beisert}}
\hypersetup{pdfsubject={Manual for the LaTeX2e Package childdoc}}
\date{30 December 2018, \textsf{v2.0}}
\maketitle

\begin{abstract}\noindent
\textsf{childdoc} is a \LaTeXe{} package
that enables the direct compilation
of document sections included by |\include|
to individual files.
\end{abstract}

\begingroup
\parskip0ex
\tableofcontents
\endgroup

%%%%%%%%%%%%%%%%%%%%%%%%%%%%%%%%%%%%%%%%%%%%%%%%%%%%%%%%%%%%%%%%%%%%%%%%%%%%%%%%
%%%%%%%%%%%%%%%%%%%%%%%%%%%%%%%%%%%%%%%%%%%%%%%%%%%%%%%%%%%%%%%%%%%%%%%%%%%%%%%%
\section{Introduction}

\LaTeX{} provides a mechanism to structure a large document (such as a book)
into a main file and several child files (containing the chapters)
using the |\include| command.
This mechanism is beneficial for documents
which span hundreds of pages in order to
make the source file(s) more manageable.
Moreover, compilation can be restricted to
selected child files by means of the |\includeonly| command.
The latter feature can be used to reduce the compilation time while editing
(this was significantly more useful in the earlier days of \LaTeX{})
or to generate a smaller document which is easier to navigate.
Another application of |\includeonly| is to generate
documents consisting of selected parts of the complete document.

However, there are a few drawbacks of the plain |\include| mechanism:
\begin{itemize}
\item
The child files cannot be compiled on their own,
they can only be compiled via the main file.
A naive editing environment
(such as a text editor with an option
to have the current file processed by \LaTeX)
may require one to switch to the main file before compiling;
attempting to compile the child file produces errors.
\item
The main file must be modified (each time)
to adjust the |\includeonly| command
to the present needs. This easily leaves the main file in a messy state.
\item
The generated document will always carry the filename
of the main document. This is inconvenient if
several child files are to be compiled and
to be kept for distribution.
\end{itemize}

The present package provides a simple interface
to make child files individually compilable by \LaTeX{}.
Compiling a child file then has the same effect as compiling
the main file with an |\includeonly| command
to select the appropriate child.
Moreover the generated document will carry the name of the child
rather than the main file.
This resolves all three above issues.

This feature is meant to make the editing of books,
thesis documents and lecture notes somewhat more convenient.
However, the package can also be used efficiently for
composing a series of documents (such as exercise sheets)
which are typically distributed individually.
It then assists the author in generating the individual documents
(potentially in different versions)
as well as a document containing the collected series.
Another application is in developing style files
or other kinds of included material
where compilation of the style file could redirect
to a sample or test file.

%%%%%%%%%%%%%%%%%%%%%%%%%%%%%%%%%%%%%%%%%%%%%%%%%%%%%%%%%%%%%%%%%%%%%%%%%%%%%%%%
%%%%%%%%%%%%%%%%%%%%%%%%%%%%%%%%%%%%%%%%%%%%%%%%%%%%%%%%%%%%%%%%%%%%%%%%%%%%%%%%
\section{Usage}

First of all, the package \textsf{childdoc} is \emph{not} a standard
\LaTeXe{} |.sty| style file! Therefore it needs to be invoked in
a non-standard way.

%%%%%%%%%%%%%%%%%%%%%%%%%%%%%%%%%%%%%%%%%%%%%%%%%%%%%%%%%%%%%%%%%%%%%%%%%%%%%%%%
\subsection{Included Files}
\label{sec:include}

%%%%%%%%%%%%%%%%%%%%%%%%%%%%%%%%%%%%%%%%
\DescribeMacro{\childdocmain}
To use the package, add the commands
\begin{center}
\begin{tabular}{l}
|\input{childdoc.def}|\\
|\childdocmain{}|\\
\end{tabular}
\end{center}
at the very top of the main \LaTeX{} file,
in particular \emph{before} the |\documentclass| statement!
The argument of |\childdocmain| should be left empty
(but it must be present).

%%%%%%%%%%%%%%%%%%%%%%%%%%%%%%%%%%%%%%%%
\DescribeMacro{\childdocof}
Furthermore, add the commands
\begin{center}
\begin{tabular}{l}
|\input{childdoc.def}|\\
|\childdocof{|\textit{main}|}|\\
\end{tabular}
\end{center}
at the top of every child file \textit{child}
which is included by |\include{|\textit{child}|}|
from within the main file
(or at least for those files to be compiled individually).
The argument \textit{main} must be the filename of the main file.

There are a couple of
considerations in setting up the main and child documents:

%%%%%%%%%%%%%%%%%%%%%%%%%%%%%%%%%%%%%%%%
\paragraph{Restrictions.}

Please note the following restrictions:
\begin{itemize}
\item
|\childdocmain| must be called with one argument \textit{main}
to ensure compatibility with earlier version of the package.
It must either be empty (|\childdocmain{}|)
or precisely match the filename of the main file in which it is specified.
See \secref{sec:detection} for further information.
\item
The filename \textit{main} must be specified without the |.tex| extension.
\item
The filename \textit{main} is case sensitive
(even in case-insensitive file systems)
due to internal string comparison.
\item
The argument \textit{main} should be fully expanded, it cannot be a macro.
\item
Subdirectories and special characters should be avoided in filenames.
\item
The command |\childdocmain{|\textit{main}|}| must be followed by a whitespace.
It should not be followed immediately by another command
or by a comment mark `|%|'.
This is because the \TeX{} parser reads the token immediately following
the argument of |\childdocmain| and puts it
at the beginning of every child section;
however, a white\-space is ignored.
\end{itemize}

%%%%%%%%%%%%%%%%%%%%%%%%%%%%%%%%%%%%%%%%
\paragraph{Content of Main File.}

It is advisable to place all content in the child files included by |\include|.
Any output contained in the main file will appear in all child documents
unless suppressed manually;
it cannot be suppressed automatically by the |\includeonly| directive
and thus should normally be avoided.
A method to include some content in the main file
by means of conditional processing is described in \secref{sec:conditional}.

%%%%%%%%%%%%%%%%%%%%%%%%%%%%%%%%%%%%%%%%
\paragraph{Page Numbering.}

When only a part of the document is compiled,
the appropriate numbering of pages
(as well as other status parameters)
is determined from the |.aux| files.
The latter contain information from previous passes.
However this information needs to propagate through
all intermediate child documents.
Therefore the page numbering in child documents may well
be inconsistent until the complete document is compiled at least once.

A useful (if unconventional) way to always ensure a consistent
page numbering is to restart the numbering in each child document
and denote the pages by `\textit{child}|.|\textit{page}'
where \textit{child} represents the chapter/section number of the child file.
This can be achieved by the command
|\numberwithin{page}{|\textit{child}|}|
of the \textsf{amsmath} package
where \textit{child} can be |chapter| or |section|
depending on the chosen structuring.
Alternatively, one can modify the macro |\thepage| appropriately
and reset the counter |page| at the start of each child file.

%%%%%%%%%%%%%%%%%%%%%%%%%%%%%%%%%%%%%%%%%%%%%%%%%%%%%%%%%%%%%%%%%%%%%%%%%%%%%%%%
\subsection{Conditional Processing}
\label{sec:conditional}

The package provides a mechanism to compile different versions
of a document. To customise the versions further some conditional processing
can come in handy to distinguish which version is being compiled.
The package provides two macros to describe the compilation context:

%%%%%%%%%%%%%%%%%%%%%%%%%%%%%%%%%%%%%%%%
\DescribeMacro{\ifchilddoc}
The conditional |\ifchilddoc| distinguishes between the compilation of
child documents and the main document:
%
\begin{center}
|\ifchilddoc |\textit{child-code}| |[|\||else |\textit{main-code}]| \||fi|
\end{center}

%%%%%%%%%%%%%%%%%%%%%%%%%%%%%%%%%%%%%%%%
\DescribeMacro{\childdocname}
\DescribeMacro{\childdocjob}
The macro |\childdocname| contains the filename (without extension)
of the main or child file being processed.
Note that |\childdocjob| will always contain the name of the main file.

%%%%%%%%%%%%%%%%%%%%%%%%%%%%%%%%%%%%%%%%
\paragraph{Title Page.}

Conditional processing can be used to include a title or banner page
in the main document when proper precautions are taken.
Importantly, the code in the main file should ensure that the page counter
(as well as other status parameters which are stored in the |.aux| files)
takes the same value after the conditional processing.
Otherwise the page numbers may take divergent values
depending on which part is compiled.

For example, a title page could be declared by:
%
\begin{center}
\begin{tabular}{l}
|\ifchilddoc\||else|\\
|\addtocounter{page}{-1}|\\
\textit{code for title page}\\
|\newpage|\\
|\||fi|
\end{tabular}
\end{center}
%
A banner page for the child documents can be generated by:
%
\begin{center}
\begin{tabular}{l}
|\ifchilddoc|\\
|\addtocounter{page}{-1}|\\
\textit{code for banner page}\\
|\newpage|\\
|\||fi|
\end{tabular}
\end{center}
%
Here one could write a message such as:
\begin{center}
|This is the part \childdocname{} of \childdocjob{}.|
\end{center}

%%%%%%%%%%%%%%%%%%%%%%%%%%%%%%%%%%%%%%%%%%%%%%%%%%%%%%%%%%%%%%%%%%%%%%%%%%%%%%%%
\subsection{Flags}
\label{sec:flags}

The package makes it easy to generate different versions
of the main or child documents.
To this end compilation flags can be defined
and assigned different default values.
They will be particularly useful in conjunction
with the forwarding mechanism described in \secref{sec:forward}.

For example, it may be useful to have a flag |\version|
which can be set to |draft| or |final|.
The document source will contain some conditional code
depending on the value of |\version|.
Suppose further, the flag should default to |final| for the main file
and to |draft| for child files
which is a natural assignment for editing the document.
This is achieved by placing the following code
in the preamble of the main document
(below the |\childdocmain| directive):
%
\begin{center}
\begin{tabular}{l}
|\ifchilddoc|\\
|\providecommand{\version}{draft}|\\
|\||else|\\
|\providecommand{\version}{final}|\\
|\||fi|
\end{tabular}
\end{center}
%
The definition by |\providecommand| makes sure
that previous definitions are not overwritten.
Further statements |\providecommand{\version}{...}|
can thus be added before the above code to override it.

For the main file, one might add a line
(between |\childdocmain| and the above block)
%
\begin{center}
|%\ifchilddoc\||else\providecommand{\version}{draft}\||fi|
\end{center}
%
which can be uncommented to produce a draft version.
Likewise one can add a line to the very top of a child file
(above the |\childdocof{|\textit{main}|}| directive)
%
\begin{center}
|%\providecommand{\version}{final}|
\end{center}
%
which can be uncommented to produce the final version of this child document.

%%%%%%%%%%%%%%%%%%%%%%%%%%%%%%%%%%%%%%%%%%%%%%%%%%%%%%%%%%%%%%%%%%%%%%%%%%%%%%%%
\subsection{Forwarding}
\label{sec:forward}

Different versions of the main or child documents
using compilation flags as described in \secref{sec:flags}
can be (permanently) stored in different files
for convenient compilation, viewing and distribution.
To this end, the package defines a command
to pass on compilation to a different file:

%%%%%%%%%%%%%%%%%%%%%%%%%%%%%%%%%%%%%%%%
\DescribeMacro{\childdocforward}
The command |\childdocforward| redirects processing to
another source file:
%
\begin{center}
\begin{tabular}{l}
|\input{childdoc.def}|\\
|\childdocforward[|\textit{main}|]{|\textit{dest}|}|\\
\end{tabular}
\end{center}
%
The argument \textit{dest} is the destination file
(without extension).
It should be the main file or one of the child files.
Note that further \textsf{childdoc} directives
such as |\childdocof| and |\childdocforward|
in the indicated file will be processed in this form.
The optional argument \textit{main}
passes on directly to the main file \textit{main}
while pretending to compile the child \textit{dest}.
This form behaves as if \textit{dest}
issues |\childdocof{|\textit{main}|}| right away,
and no further \textsf{childdoc} directives will be processed.

%%%%%%%%%%%%%%%%%%%%%%%%%%%%%%%%%%%%%%%%
\DescribeMacro{\...prefix}
In the alternative form |\childdocforwardprefix|,
%
\begin{center}
\begin{tabular}{l}
|\input{childdoc.def}|\\
|\childdocforwardprefix[|\textit{main}|]{|\textit{prefix}|}{|\textit{dest}|}|
\end{tabular}
\end{center}
%
the destination file is determined by a pattern
depending on the current file:
To make this work, the current file must be called
`{\textit{prefix}\hspace{0.2em}\textit{suffix}}'
with \textit{prefix} matching precisely the argument.
Processing is then passed on to the file
`{\textit{dest}\hspace{0.2em}\textit{suffix}}'.
Surely, the same effect is achieved by
directly specifying the
argument `{\textit{dest}\hspace{0.2em}\textit{suffix}}'
in the first form.
However, that requires to set up a different file
for each child. With the alternative form of the command
all these files can have exactly the same content
which simplifies setting them up and maintaining them.

For example, the following file |draft.tex|
with a compilation flag |\version| as described in \secref{sec:flags}
compiles the main document as a draft:
%
\begin{center}
\begin{tabular}{l}
|\def\version{draft}|\\
|\input{childdoc.def}|\\
|\childdocforward{|\textit{main}|}|
\end{tabular}
\end{center}
%
Likewise, the following files |final|\textit{nn}|.tex|
compile the final version of the child document
|child|\textit{nn}|.tex|:
%
\begin{center}
\begin{tabular}{l}
|\def\version{final}|\\
|\input{childdoc.def}|\\
|\childdocforwardprefix{final}{child}|
\end{tabular}
\end{center}
%

Note that when several versions of a main file and/or of each child file
are to be generated, it may be convenient to set up a |Makefile| or
shell script to automatise the process.

%%%%%%%%%%%%%%%%%%%%%%%%%%%%%%%%%%%%%%%%%%%%%%%%%%%%%%%%%%%%%%%%%%%%%%%%%%%%%%%%
\subsection{Command Line Processing}
\label{sec:commandline}

The effect of redirection files can also be achieved by invoking
the \LaTeX{} compiler with a more elaborate command line.
Most conveniently this should be done as part
of a shell script or a |Makefile|.

When using \textsf{childdoc} in the main file, the following
command lines effectively perform a redirection
(note that depending on the shell being used,
backslashes may have to be doubled: `|\|' $\to$ `|\\|'):
%
\begin{center}
|... -jobname "|\textit{target}|" |\\|"|[\textit{flags}]%
|\input{childdoc.def}\childdocforward[|\textit{main}|]{|\textit{dest}|}"|
\end{center}
%
Here \textit{target} is the name of the output file,
\textit{main} is the name of the main file
and \textit{dest} is the name of the main or child file to be processed
(all filenames without extensions).
The optional argument \textit{main} can be omitted
if \textit{main} matches \textit{dest}.
Optionally, compilation \textit{flags} can be defined via |\def| commands.
This command line makes the \TeX{} engine believe
it is compiling the file \textit{target}
whose content is specified as the latter parameter.
The provided code then forwards the processing to
\textit{main} or \textit{dest} as described in \secref{sec:forward}.

%%%%%%%%%%%%%%%%%%%%%%%%%%%%%%%%%%%%%%%%%%%%%%%%%%%%%%%%%%%%%%%%%%%%%%%%%%%%%%%%
\subsection{Include by Input}
\label{sec:input}

Including child documents by |\include| has some restrictions by design.
Most notably, the content of a child document always occupies
its own set of pages; pages cannot be shared between child documents.
Usually, this behaviour makes perfect sense
because each child document contain an essential part of the document.
However, in some situations it may be desirable to compose
a document from a collection of parts
without having mandatory page breaks between then.
For this case, the package
provides a mechanism to include parts
by |\input| which can also be processed individually.
However, by construction this mechanism
requires manual handling of the content to be output.

%%%%%%%%%%%%%%%%%%%%%%%%%%%%%%%%%%%%%%%%
\DescribeMacro{\ifchilddocmanual}
The main file should be prepared as usual, see \secref{sec:include}.
However, the document body must make a distinction
between processing of an individual part and of the main document, e.g.:
%
\begin{center}
\begin{tabular}{l}
|\ifchilddocmanual|\\
|\input{\childdocname}|\\
|\||else|\\
\textit{document body with }|\input{|\textit{part}|}|\\
|\||fi|
\end{tabular}
\end{center}
%
The conditional |\ifchilddocmanual| is true whenever
a part to be included by |\input| is being compiled,
and the name of the part is stored in |\childdocname|.

%%%%%%%%%%%%%%%%%%%%%%%%%%%%%%%%%%%%%%%%
\DescribeMacro{\childdocby}
Each part to be included by |\input| should start with:
%
\begin{center}
\begin{tabular}{l}
|\input{childdoc.def}|\\
|\childdocby{|\textit{main}|}|\\
\end{tabular}
\end{center}
%
The directive |\childdocby| is similar to |\childdocof|
described in \secref{sec:include},
but the subsequent selection of content must be done manually.
To that end, both |\ifchilddoc| and |\ifchilddocmanual|
will be true upon processing of a part,
and the name of the part is stored in |\childdocname|.
Note that |\jobname| will be set to the filename of the current part
so that each part receives an individual |.aux| file
that does not interfere with the |.aux| file(s) of the main document.
This behaviour can be altered by the alternative form
|\childdocby[*]{|\textit{main}|}| (with a non-empty optional argument)
which uses the |.aux| file of the main document
by setting |\jobname| to \textit{main}.

%%%%%%%%%%%%%%%%%%%%%%%%%%%%%%%%%%%%%%%%%%%%%%%%%%%%%%%%%%%%%%%%%%%%%%%%%%%%%%%%
\subsection{Driver Development}
\label{sec:driver}

The \textsf{childdoc} mechanism can also be use for the development
of definition files such as \LaTeX{} styles or classes.
This case differs from the above setup with multiple parts
included by |\include| in that no |\includeonly| should be invoked.
This can be achieved by starting the include file
(before |\ProvidesPackage|) with:
%
\begin{center}
\begin{tabular}{l}
|\input{childdoc.def}|\\
|\childdocforward{|\textit{main}|}|\\
\end{tabular}
\end{center}
%
or alternatively with:
%
\begin{center}
\begin{tabular}{l}
|\input{childdoc.def}|\\
|\childdocby{|\textit{main}|}|\\
\end{tabular}
\end{center}
%
Both forms have slightly different effects as described above.
The main file is prepared as usual, see \secref{sec:include}.

%%%%%%%%%%%%%%%%%%%%%%%%%%%%%%%%%%%%%%%%%%%%%%%%%%%%%%%%%%%%%%%%%%%%%%%%%%%%%%%%
\subsection{Legacy Detection}
\label{sec:detection}

The directive |\childdocmain| in the main file can detect
whether the complete document or merely a child is to be compiled
even without using the directive |\childdocof|.
This method is deprecated because it is less robust
and there is no compelling reason to use it;
it is merely provided for backward compatibility
and it may be removed in future versions.

If the detection mechanism is to be used,
it is mandatory to correctly specify
the filename of the main file as the argument of |\childdocmain|:
%
\begin{center}
\begin{tabular}{l}
|\input{childdoc.def}|\\
|\childdocmain{|\textit{main}|}|\\
\end{tabular}
\end{center}
%
If |\jobname| does not match the argument \textit{main} of |\childdocmain|,
it is assumed that |\jobname| points to the child file to be compiled.
When using |\childdocmain| with the main file specified as argument,
it suffices to start a child file
with just |\input{|\textit{main}|}|
without loading of the package and using |\childdocof|.
If instead all processing is done
with the appropriate \textsf{childdoc} directives,
the argument of \textit{main} of |\childdocmain| can be empty.

An alternative version of the command line processing described
in \secref{sec:commandline} using the detection mechanism reads:
%
\begin{center}
|... -jobname "|\textit{target}|" "|[\textit{flags}]%
[|\def\jobname{|\textit{dest}|}|]|\input{|\textit{main}|}"|
\end{center}

%%%%%%%%%%%%%%%%%%%%%%%%%%%%%%%%%%%%%%%%%%%%%%%%%%%%%%%%%%%%%%%%%%%%%%%%%%%%%%%%
\subsection{Manual Code}
\label{sec:manual}

In case one cannot be certain whether the definitions file |childdoc.def|
is installed on the target \TeX{} distribution
and one prefers not to ship it,
it is conceivable to paste a few relevant commands into the sources.

To that end, drop all statements |\input{childdoc.def}|
and perform the replacements as outlined below.
Instead of |\childdocmain{|\textit{main}|}| add the following code
to the top of the main file:
%
\begin{center}
\begin{tabular}{l}
|\||ifdefined\childdocname\endinput\||fi\newif\ifchilddoc|\\
|\edef\childdocname{\scantokens\expandafter{\jobname\noexpand}}|\\
|\def\childdocmain{|\textit{main}|}\||ifx\childdocmain\childdocname\||else|\\
|\childdoctrue\includeonly{\childdocname}\let\jobname\childdocmain\||fi|\\
\end{tabular}
\end{center}
%
Instead of |\childdocof{|\textit{main}|}| just include the main file
at the top of each child file:
%
\begin{center}
|\input{|\textit{main}|}|
\end{center}
%
A simple redirection |\childdocforward{|\textit{dest}|}| is achieved by:
%
\begin{center}
|\def\jobname{|\textit{dest}|}\input{\jobname}|
\end{center}
%
The redirection with prefix
|\childdocforwardprefix[|\textit{prefix}|]{|\textit{dest}|}|
is accomplished by:
%
\begin{center}
\begin{tabular}{l}
|{\edef\jobname{\scantokens\expandafter{\jobname\noexpand}}|\\
|\def\redirectjob |\textit{prefix}|#1~~~{\gdef\jobname{|\textit{dest}|#1}}|\\
|\expandafter\redirectjob\jobname~~~}\input{\jobname}|
\end{tabular}
\end{center}

In an alternative approach,
child documents can be compiled by a specific command line
without additional code or specific definitions:
%
\begin{center}
|... -jobname "|\textit{target}|" "|[\textit{flags}]%
|\includeonly{|\textit{dest}|}\input{|\textit{main}|}"|
\end{center}
%

%%%%%%%%%%%%%%%%%%%%%%%%%%%%%%%%%%%%%%%%%%%%%%%%%%%%%%%%%%%%%%%%%%%%%%%%%%%%%%%%
%%%%%%%%%%%%%%%%%%%%%%%%%%%%%%%%%%%%%%%%%%%%%%%%%%%%%%%%%%%%%%%%%%%%%%%%%%%%%%%%
\section{Information}

%%%%%%%%%%%%%%%%%%%%%%%%%%%%%%%%%%%%%%%%%%%%%%%%%%%%%%%%%%%%%%%%%%%%%%%%%%%%%%%%
\subsection{Copyright}

Copyright \copyright{} 2017--2018 Niklas Beisert

This work may be distributed and/or modified under the
conditions of the \LaTeX{} Project Public License, either version 1.3
of this license or (at your option) any later version.
The latest version of this license is in
  \url{http://www.latex-project.org/lppl.txt}
and version 1.3 or later is part of all distributions of \LaTeX{}
version 2005/12/01 or later.

This work has the LPPL maintenance status `maintained'.

The Current Maintainer of this work is Niklas Beisert.

This work consists of the files |README.txt|, |childdoc.ins| and |childdoc.dtx|
as well as the derived files |childdoc.def|, |cdocsamp.tex|
with |cdocsch1.tex|, |cdocsch2.tex|, |cdocspt3.tex|, |cdocspt4.tex|,
|cdocsdrf.tex|, |cdocsfn1.tex|, |cdocsfn2.tex|
as well as |childdoc.pdf|.

%%%%%%%%%%%%%%%%%%%%%%%%%%%%%%%%%%%%%%%%%%%%%%%%%%%%%%%%%%%%%%%%%%%%%%%%%%%%%%%%
\subsection{Files and Installation}

The package consists of the files:
%
\begin{center}
\begin{tabular}{ll}
    |README.txt|   & readme file \\
    |childdoc.ins| & installation file \\
    |childdoc.dtx| & source file \\
    |childdoc.def| & definition file \\
    |cdocsamp.tex| & sample main file \\
    |cdocsch1.tex| & sample include file \\
    |cdocsch2.tex| & sample include file \\
    |cdocspt3.tex| & sample part file \\
    |cdocspt4.tex| & sample part file \\
    |cdocsdrf.tex| & sample redirection file \\
    |cdocsfn1.tex| & sample redirection file \\
    |cdocsfn2.tex| & sample redirection file \\
    |childdoc.pdf| & manual
\end{tabular}
\end{center}
%
The distribution consists of the files
|README.txt|, |childdoc.ins| and |childdoc.dtx|.
%
\begin{itemize}
\item
Run (pdf)\LaTeX{} on |childdoc.dtx|
to compile the manual |childdoc.pdf| (this file).
\item
Run \LaTeX{} on |childdoc.ins| to create the definitions file |childdoc.def|
and the sample |cdocsamp.tex| with include files
|cdocsch1.tex|, |cdocsch2.tex|, |cdocspt3.tex|, |cdocspt4.tex|,
|cdocsdrf.tex|, |cdocsfn1.tex|, |cdocsfn2.tex|.
Then copy the file |childdoc.def| to an appropriate directory of your \LaTeX{}
distribution, e.g.\ \textit{texmf-root}|/tex/latex/childdoc|.
\end{itemize}

%%%%%%%%%%%%%%%%%%%%%%%%%%%%%%%%%%%%%%%%%%%%%%%%%%%%%%%%%%%%%%%%%%%%%%%%%%%%%%%%
\subsection{Related CTAN Packages}

There are several other packages which offer a similar functionality:
%
\begin{itemize}
\item
The packages
\href{http://ctan.org/pkg/docmute}{\textsf{docmute}},
\href{http://ctan.org/pkg/includex}{\textsf{includex}} and
\href{http://ctan.org/pkg/standalone}{\textsf{standalone}}
provide commands to include only the document body of
a child file thus allowing both files to be compiled individually.
\item
The packages \href{http://ctan.org/pkg/subdocs}{\textsf{subdocs}}
and \href{http://ctan.org/pkg/subfiles}{\textsf{subfiles}}
provide structures in which the main and child documents can be
encapsulated and allowing them to be compiled individually.
The inclusion mechanism is different from the conventional |\include|.
\item
The package \href{http://ctan.org/pkg/combine}{\textsf{combine}}
is an elaborate solution to combine several documents into one.
\end{itemize}
%
See also the CTAN topic \href{http://ctan.org/topic/subdocs}{\textsf{subdocs}}
for further related packages.
The present package differs from the above solutions in that
a document structure constructed with the conventional |\include| mechanism
just needs two extra commands at the top of every file
such that all constituent files can be compiled individually.

%%%%%%%%%%%%%%%%%%%%%%%%%%%%%%%%%%%%%%%%%%%%%%%%%%%%%%%%%%%%%%%%%%%%%%%%%%%%%%%%
%\subsection{Feature Suggestions}
%
%The following is a list of features which may be useful for future
%versions of this package:
%%
%\begin{itemize}
%\item
%\ldots
%\end{itemize}

%%%%%%%%%%%%%%%%%%%%%%%%%%%%%%%%%%%%%%%%%%%%%%%%%%%%%%%%%%%%%%%%%%%%%%%%%%%%%%%%
\subsection{Revision History}

%%%%%%%%%%%%%%%%%%%%%%%%%%%%%%%%%%%%%%%%
\paragraph{v2.0:} 2018/12/30

\begin{itemize}
\item
immediate forward processing
\item
added |\childdocby| mechanism
\item
manual restructured
\end{itemize}

%%%%%%%%%%%%%%%%%%%%%%%%%%%%%%%%%%%%%%%%
\paragraph{v1.6:} 2018/01/17

\begin{itemize}
\item
application for development of include files
\item
corrections to manual
\end{itemize}

%%%%%%%%%%%%%%%%%%%%%%%%%%%%%%%%%%%%%%%%
\paragraph{v1.5:} 2017/05/21

\begin{itemize}
\item
more complete structuring introduced
\item
|\childdocof| introduced
\item
|\childdoc| renamed to |\childdocmain|
\item
|\childredirect| renamed to |\childdocforward| and |\childdocforwardprefix|
and functionality expanded
\end{itemize}

%%%%%%%%%%%%%%%%%%%%%%%%%%%%%%%%%%%%%%%%
\paragraph{v1.0:} 2017/04/27

\begin{itemize}
\item
manual and install package
\item
first version published on CTAN
\end{itemize}

%%%%%%%%%%%%%%%%%%%%%%%%%%%%%%%%%%%%%%%%
\paragraph{v0.6:} 2017/04/26

\begin{itemize}
\item
redirection mechanism added
\end{itemize}

%%%%%%%%%%%%%%%%%%%%%%%%%%%%%%%%%%%%%%%%
\paragraph{v0.5:} 2017/04/26

\begin{itemize}
\item
functionality in definition file
\end{itemize}


%%%%%%%%%%%%%%%%%%%%%%%%%%%%%%%%%%%%%%%%%%%%%%%%%%%%%%%%%%%%%%%%%%%%%%%%%%%%%%%%
%%%%%%%%%%%%%%%%%%%%%%%%%%%%%%%%%%%%%%%%%%%%%%%%%%%%%%%%%%%%%%%%%%%%%%%%%%%%%%%%
%%%%%%%%%%%%%%%%%%%%%%%%%%%%%%%%%%%%%%%%%%%%%%%%%%%%%%%%%%%%%%%%%%%%%%%%%%%%%%%%
\appendix

\settowidth\MacroIndent{\rmfamily\scriptsize 000\ }

 \DocInput{childdoc.dtx}

\end{document}
%</driver>
% \fi
%
% %%%%%%%%%%%%%%%%%%%%%%%%%%%%%%%%%%%%%%%%%%%%%%%%%%%%%%%%%%%%%%%%%%%%%%%%%%%%%%
% %%%%%%%%%%%%%%%%%%%%%%%%%%%%%%%%%%%%%%%%%%%%%%%%%%%%%%%%%%%%%%%%%%%%%%%%%%%%%%
% \section{Sample}
%\iffalse
%<*samplemain>
%\fi
%
% The following presents a sample document
% with two chapters, two parts, a title page,
% a compile flag as well as three forwarding files to set the flag.
% It consists of eight |.tex| files:
% \begin{center}
% \begin{tabular}{ll}
% |cdocsamp.tex|&main file\\
% |cdocsch1.tex|&include file for chapter 1\\
% |cdocsch2.tex|&include file for chapter 2\\
% |cdocspt3.tex|&include file for part 3\\
% |cdocspt4.tex|&include file for part 4\\
% |cdocsdrf.tex|&forwarding file for main file in draft mode\\
% |cdocsfi1.tex|&forwarding file for final version of chapter 1\\
% |cdocsfi2.tex|&forwarding file for final version of chapter 2\\
% \end{tabular}
% \end{center}
% Each of the eight files can be compiled directly by the \LaTeX{} compiler.
%
% %%%%%%%%%%%%%%%%%%%%%%%%%%%%%%%%%%%%%%
% \paragraph{Main File.}
%
% The main file is called |cdocsamp.tex|.
%
% Load the \textsf{childdoc} definitions and
% declare the filename for the main document:
%    \begin{macrocode}
\input{childdoc.def}
\childdocmain{}
%    \end{macrocode}

% Optional override for |\version| flag:
%    \begin{macrocode}
%%\ifchilddoc\else\providecommand{\version}{draft}\fi
%    \end{macrocode}

% Define the default values for the |\version| flag
% (|final| for the main file and |draft| for childs):
%    \begin{macrocode}
\ifchilddoc
\providecommand{\version}{draft}
\else
\providecommand{\version}{final}
\fi
%    \end{macrocode}

% Load the standard document class:
%    \begin{macrocode}
\documentclass[12pt]{article}
%    \end{macrocode}

% Start the document body:
%    \begin{macrocode}
\begin{document}
%    \end{macrocode}

% Declare a title page.
% Print title, part of document being processed and version flag:
%    \begin{macrocode}
\addtocounter{page}{-1}
\begin{center}
{\LARGE\bfseries{}childdoc example\par}
\vspace{1cm}
\ifchilddoc
\ifchilddocmanual part\else chapter\fi:
`\childdocname' of `\childdocjob'\par
\else
main document: `\childdocjob'\par
\fi
version: \version\par
\end{center}
\newpage
%    \end{macrocode}

% Manually include selected file,
% otherwise process as usual:
%    \begin{macrocode}
\ifchilddocmanual
\section*{part `\childdocname'}
\input{\childdocname}
\else
%    \end{macrocode}

% Include the two chapters:
%    \begin{macrocode}
\include{cdocsch1}
\include{cdocsch2}
%    \end{macrocode}

% Include the two parts unless only chapters should be displayed:
%    \begin{macrocode}
\ifchilddoc\else
\section{part three}
\input{cdocspt3}
\section{part four}
\input{cdocspt4}
\fi
%    \end{macrocode}

% Process as usual until here:
%    \begin{macrocode}
\fi
%    \end{macrocode}

% End of document body:
%    \begin{macrocode}
\end{document}
%    \end{macrocode}
%\iffalse
%</samplemain>
%\fi
%
% %%%%%%%%%%%%%%%%%%%%%%%%%%%%%%%%%%%%%%
% \paragraph{Chapter Include Files.}
%
% The include files are called |cdocsch1.tex| and |cdocsch2.tex|.
%
%\iffalse
%<*samplechap1|samplechap2>
%\fi

% Optional override for |\version| flag:
%    \begin{macrocode}
%%\providecommand{\version}{final}
%    \end{macrocode}

% Include the main document:
%    \begin{macrocode}
\input{childdoc.def}
\childdocof{cdocsamp}
%    \end{macrocode}

%\iffalse
%</samplechap1|samplechap2>
%\fi
%
%\iffalse
%<*samplechap1>
%\fi
% Some text for chapter 1:
%    \begin{macrocode}
\section{one}
some text in chapter one
%    \end{macrocode}

%\iffalse
%</samplechap1>
%\fi
% Some text for chapter 2:
%\iffalse
%<*samplechap2>
%\fi
%    \begin{macrocode}
\section{two}
more text in chapter two
%    \end{macrocode}

%\iffalse
%</samplechap2>
%\fi
%
% %%%%%%%%%%%%%%%%%%%%%%%%%%%%%%%%%%%%%%
% \paragraph{Part Include Files.}
%
% The include files are called |cdocspt3.tex| and |cdocspt4.tex|.
%
%\iffalse
%<*samplepart3|samplepart4>
%\fi

% Optional override for |\version| flag:
%    \begin{macrocode}
%%\providecommand{\version}{final}
%    \end{macrocode}

% Include the main document:
%    \begin{macrocode}
\input{childdoc.def}
\childdocby{cdocsamp}
%    \end{macrocode}

%\iffalse
%</samplepart3|samplepart4>
%\fi
%
%\iffalse
%<*samplepart3>
%\fi
% Some text for part 3:
%    \begin{macrocode}
some text in part three
%    \end{macrocode}

%\iffalse
%</samplepart3>
%\fi
% Some text for part 4:
%\iffalse
%<*samplepart4>
%\fi
%    \begin{macrocode}
more text in part four
%    \end{macrocode}

%\iffalse
%</samplepart4>
%\fi
%
% %%%%%%%%%%%%%%%%%%%%%%%%%%%%%%%%%%%%%%
% \paragraph{Forwarding for a Complete Draft.}
%
% The following forwarding file |cdocsdrf.tex|
% compiles the main document in draft mode:
%\iffalse
%<*sampledraft>
%\fi
%    \begin{macrocode}
\def\version{draft}
\input{childdoc.def}
\childdocforward{cdocsamp}
%    \end{macrocode}

%\iffalse
%</sampledraft>
%\fi
%
% %%%%%%%%%%%%%%%%%%%%%%%%%%%%%%%%%%%%%%
% \paragraph{Forwarding for Final Version of the Chapters.}
%
% The following forwarding files |cdocsfn1.tex| and |cdocsfn2.tex|
% (with identical content)
% compile the final versions of the child documents
% |cdocsch1.tex| and |cdocsch2.tex|, respectively:
%\iffalse
%<*samplefinal>
%\fi
%    \begin{macrocode}
\def\version{final}
\input{childdoc.def}
\childdocforwardprefix[cdocsamp]{cdocsfn}{cdocsch}
%    \end{macrocode}

%\iffalse
%</samplefinal>
%\fi
%
% %%%%%%%%%%%%%%%%%%%%%%%%%%%%%%%%%%%%%%
% \paragraph{Command Line Processing.}
%
% The following three command lines generate the output files
% |cdocscld|, |cdocscl1| and |cdocscl2|
% which should be identical to
% |cdocsdrf|, |cdocsch1| and |cdocsfn2|, respectively:
% \begin{center}
% \begin{tabular}{l}
% |latex -jobname cdocscld \|\\
% |  "\def\version{draft}\input{childdoc.def}\childdocforward{cdocsamp}"|\\
% |latex -jobname cdocscl1 \|\\
% |  "\input{childdoc.def}\childdocforward[cdocsamp]{cdocsch1}"|\\
% |latex -jobname cdocscl2 \|\\
% |  "\def\version{final}\input{childdoc.def}\childdocforward{cdocsch2}"|
% \end{tabular}
% \end{center}
% Note that the trailing backslash on each first line
% merely continues the input to the second line
% (for convenient cut ant paste).
% Furthermore, the command |latex| can be replaced by any
% of its alternative versions such as |pdflatex|.
%
% %%%%%%%%%%%%%%%%%%%%%%%%%%%%%%%%%%%%%%%%%%%%%%%%%%%%%%%%%%%%%%%%%%%%%%%%%%%%%%
% %%%%%%%%%%%%%%%%%%%%%%%%%%%%%%%%%%%%%%%%%%%%%%%%%%%%%%%%%%%%%%%%%%%%%%%%%%%%%%
% \section{Implementation}
%\iffalse
%<*package>
%\fi
%
% This section describes the definitions file |childdoc.def|.

% The definitions cannot be loaded using |\usepackage| or |\RequirePackage|
% which has a mechanism to prevent loading a style file more than once.
% When loading the definitions by means of |\input|
% multiple instances have to be prevented manually:
%\iffalse
%This code needs to be before the `\ProvidesFile' directive
%which is defined at the beginning of this file.
%Therefore it is also placed there and commented out here.
%</package>
%<*discard>
%\fi
%    \begin{macrocode}
\ifdefined\childdocmain\endinput\fi
%    \end{macrocode}
%\iffalse
%</discard>
%<*package>
%\fi
%
% \macro{\ifchilddoc}
% \macro{\ifchilddocmanual}
% The conditional |\ifchilddoc| tells whether a
% child (true) or main (false) document is being compiled.
% The conditional |\ifchilddocmanual| tells whether
% the |\includeonly| mechanism is used (false) or
% the selection of child files must be performed manually (true).
% The definitions initialise to false:
%    \begin{macrocode}
\newif\ifchilddoc
\newif\ifchilddocmanual
%    \end{macrocode}

% \macro{\childdocname}
% \macro{\childdocjob}
% The macro |\childdocname| stores the name of the main document
% to be compiled. The macro |\childdocjob| stores the name of
% the document on which the \LaTeX{} compiler was originally invoked.
% The content of |\jobname| cannot be compared
% to filenames specified in the source due to different catcodes.
% The following code rescans |\jobname|, stores the result
% in |\childdocname| and saves a copy in |\childdocjob|:
%    \begin{macrocode}
\edef\childdocname{\scantokens\expandafter{\jobname\noexpand}}
\let\childdocjob\childdocname
%    \end{macrocode}

% \macro{\childdocdisable}
% The macro |\childdocdisable| prevents the main file
% from being processed more than once.
% At this stage, the main document command |\childdocmain|
% is assumed to be called once again where it should do nothing.
% Any subsequent call to it should prevent
% a secondary processing of the main document
% It overwrites the forwarding commands
% |\childdocof| and |\childdocforward|
% with empty macros to prevent further inclusions of the main document:
%    \begin{macrocode}
\newcommand{\childdocdisable}
{
  \renewcommand{\childdocmain}[1]{\renewcommand{\childdocmain}[1]{\endinput}}
  \renewcommand{\childdocof}[1]{}
  \renewcommand{\childdocby}[2][]{}
  \renewcommand{\childdocforward}[2][]{}
  \renewcommand{\childdocdisable}{}
}
%    \end{macrocode}

% \macro{\childdocmain}
% The macro |\childdocmain| is to be called at the top of the main file
% with nothing or the main filename (without extension) as argument.
% First, it breaks loops.
% If the argument is not empty and does not match |\childdocname|
% (which is set by the first inclusion of |childdoc.def|),
% |\ifchilddoc| is set to true, |\includeonly| is applied to the child file
% and |\jobname| is set to the main file
% (for proper handling of |.aux| files):
%    \begin{macrocode}
\newcommand{\childdocmain}[1]
{
  \childdocdisable\childdocmain{}
  \if?#1?\else
    \begingroup
      \def\childdoctmp{#1}
      \ifx\childdoctmp\childdocname
        \def\childdoctmp{}
      \else
        \def\childdoctmp
        {
          \childdoctrue
          \includeonly{\childdocname}
          \def\childdocjob{#1}
          \def\jobname{#1}
        }
      \fi
      \expandafter
    \endgroup
    \childdoctmp
  \fi
}
%    \end{macrocode}

% \macro{\childdocof}
% The command |\childdocof| redirects
% compilation to the main file |#1|.
%    \begin{macrocode}
\newcommand{\childdocof}[1]
{
  \childdocdisable
  \childdoctrue
  \includeonly{\childdocname}
  \def\jobname{#1}
  \def\childdocjob{#1}
  \input{#1}
}
%    \end{macrocode}

% \macro{\childdocby}
% The command |\childdocby| ....
%    \begin{macrocode}
\newcommand{\childdocby}[2][]
{
  \childdocdisable
  \childdoctrue
  \childdocmanualtrue
  \if?#1?\else
    \def\jobname{#2}
  \fi
  \def\childdocjob{#2}
  \input{#2}
  \endinput
}
%    \end{macrocode}

% \macro{\childdocforward}
% The command |\childdocforward| redirects
% compilation to the main file or
% (if the optional argument is given) a child file.
% Parameters are set as if the main file
% or a child file starting with |\childdocof| was compiled.
% Then compilation is handed over to the main file:
%    \begin{macrocode}
\newcommand{\childdocforward}[2][]
{
  \begingroup
    \if?#1?
      \def\childdoctmp
      {
        \def\childdocname{#2}
        \def\childdocjob{#2}
        \def\jobname{#2}
        \input{#2}
        \endinput
      }
    \else
      \def\childdoctmp
      {
        \childdocdisable
        \def\childdocname{#2}
        \childdoctrue
        \includeonly{#2}
        \def\childdocjob{#1}
        \def\jobname{#1}
        \input{#1}
        \endinput
      }
    \fi
    \expandafter
  \endgroup
  \childdoctmp
}
%    \end{macrocode}

% \macro{\childdocforwardprefix}
% The command |\childdocforwardprefix| redirects
% compilation to the main or a child file by means of a pattern.
% The prefix |#1| in the current filename is replaced by |#2|
% and the suffix of the current filename is kept
% (it is assumed that the filename does not contain the substring `|~~~|'
% which is used as a delimiter).
% Compilation is handed over to the new file by |\childdocforward|:
%    \begin{macrocode}
\newcommand{\childdocforwardprefix}[3][]
{
  \begingroup
    \def\childdocextract #2##1~~~{\def\childdoctmp{\childdocforward[#1]{#3##1}}}
    \expandafter\childdocextract\childdocname~~~
    \expandafter
  \endgroup
  \childdoctmp
}
%    \end{macrocode}

% \macro{\childdoc}
% The deprecated macro |\childdoc| is a legacy version of |\childdocmain|:
%    \begin{macrocode}
\newcommand{\childdoc}{\childdocmain}
%    \end{macrocode}

% \macro{\childdocredirect}
% The deprecated macro |\childdocredirect| is a legacy version
% of |\childdocforward| and |\childdocforwardprefix|:
%    \begin{macrocode}
\newcommand{\childdocredirect}[2][]
{
  \begingroup
    \if?#1?
      \def\childdoctmp{\childdocforward{#2}}
    \else
      \def\childdoctmp{\childdocforwardprefix{#1}{#2}}
    \fi
    \expandafter
  \endgroup
  \childdoctmp
}
%    \end{macrocode}

%\iffalse
%</package>
%\fi
%
\endinput
\childdocforward{cdocsamp}"|\\
% |latex -jobname cdocscl1 \|\\
% |  "% \iffalse
%
% childdoc.dtx Copyright (C) 2017-2018 Niklas Beisert
%
% This work may be distributed and/or modified under the
% conditions of the LaTeX Project Public License, either version 1.3
% of this license or (at your option) any later version.
% The latest version of this license is in
%   http://www.latex-project.org/lppl.txt
% and version 1.3 or later is part of all distributions of LaTeX
% version 2005/12/01 or later.
%
% This work has the LPPL maintenance status `maintained'.
%
% The Current Maintainer of this work is Niklas Beisert.
%
% This work consists of the files childdoc.dtx and childdoc.ins
% and the derived files childdoc.def and cdocsamp.tex with
% cdocsch1.tex, cdocsch2.tex, cdocsdrf.tex, cdocsfn1.tex, cdocsfn2.tex.
%
%<package>\ifdefined\childdocmain\endinput\fi
%<package>\ProvidesFile{childdoc.def}[2018/12/30 v2.0 child document driver]
%<samplemain>\ProvidesFile{cdocsamp.tex}[2018/12/30 v2.0 sample for childdoc]
%<*driver>
%\ProvidesFile{childdoc.drv}[2018/12/30 v2.0 childdoc reference manual file]
\PassOptionsToClass{10pt,a4paper}{article}
\documentclass{ltxdoc}

\usepackage[margin=35mm]{geometry}
\usepackage{hyperref}
\usepackage{hyperxmp}
\usepackage[usenames]{color}

\hypersetup{colorlinks=true}
\hypersetup{pdfstartview=FitH}
\hypersetup{pdfpagemode=UseNone}
\hypersetup{pdfsource={}}
\hypersetup{pdflang={en-UK}}
\hypersetup{pdfcopyright={Copyright 2017-2018 Niklas Beisert.
  This work may be distributed and/or modified under the
  conditions of the LaTeX Project Public License, either version 1.3
  of this license or (at your option) any later version.}}
\hypersetup{pdflicenseurl={http://www.latex-project.org/lppl.txt}}
\hypersetup{pdfcontactaddress={ETH Zurich, ITP, HIT K,
  Wolfgang-Pauli-Strasse 27}}
\hypersetup{pdfcontactpostcode={8093}}
\hypersetup{pdfcontactcity={Zurich}}
\hypersetup{pdfcontactcountry={Switzerland}}
\hypersetup{pdfcontactemail={nbeisert@itp.phys.ethz.ch}}
\hypersetup{pdfcontacturl={http://people.phys.ethz.ch/\xmptilde nbeisert/}}

\newcommand{\secref}[1]{\hyperref[#1]{section \ref*{#1}}}

\parskip1ex
\parindent0pt
\let\olditemize\itemize
\def\itemize{\olditemize\parskip0pt}

\begin{document}

\title{The \textsf{childdoc} Package}
\hypersetup{pdftitle={The childdoc Package}}
\author{Niklas Beisert\\[2ex]
  Institut f\"ur Theoretische Physik\\
  Eidgen\"ossische Technische Hochschule Z\"urich\\
  Wolfgang-Pauli-Strasse 27, 8093 Z\"urich, Switzerland\\[1ex]
  \href{mailto:nbeisert@itp.phys.ethz.ch}
  {\texttt{nbeisert@itp.phys.ethz.ch}}}
\hypersetup{pdfauthor={Niklas Beisert}}
\hypersetup{pdfsubject={Manual for the LaTeX2e Package childdoc}}
\date{30 December 2018, \textsf{v2.0}}
\maketitle

\begin{abstract}\noindent
\textsf{childdoc} is a \LaTeXe{} package
that enables the direct compilation
of document sections included by |\include|
to individual files.
\end{abstract}

\begingroup
\parskip0ex
\tableofcontents
\endgroup

%%%%%%%%%%%%%%%%%%%%%%%%%%%%%%%%%%%%%%%%%%%%%%%%%%%%%%%%%%%%%%%%%%%%%%%%%%%%%%%%
%%%%%%%%%%%%%%%%%%%%%%%%%%%%%%%%%%%%%%%%%%%%%%%%%%%%%%%%%%%%%%%%%%%%%%%%%%%%%%%%
\section{Introduction}

\LaTeX{} provides a mechanism to structure a large document (such as a book)
into a main file and several child files (containing the chapters)
using the |\include| command.
This mechanism is beneficial for documents
which span hundreds of pages in order to
make the source file(s) more manageable.
Moreover, compilation can be restricted to
selected child files by means of the |\includeonly| command.
The latter feature can be used to reduce the compilation time while editing
(this was significantly more useful in the earlier days of \LaTeX{})
or to generate a smaller document which is easier to navigate.
Another application of |\includeonly| is to generate
documents consisting of selected parts of the complete document.

However, there are a few drawbacks of the plain |\include| mechanism:
\begin{itemize}
\item
The child files cannot be compiled on their own,
they can only be compiled via the main file.
A naive editing environment
(such as a text editor with an option
to have the current file processed by \LaTeX)
may require one to switch to the main file before compiling;
attempting to compile the child file produces errors.
\item
The main file must be modified (each time)
to adjust the |\includeonly| command
to the present needs. This easily leaves the main file in a messy state.
\item
The generated document will always carry the filename
of the main document. This is inconvenient if
several child files are to be compiled and
to be kept for distribution.
\end{itemize}

The present package provides a simple interface
to make child files individually compilable by \LaTeX{}.
Compiling a child file then has the same effect as compiling
the main file with an |\includeonly| command
to select the appropriate child.
Moreover the generated document will carry the name of the child
rather than the main file.
This resolves all three above issues.

This feature is meant to make the editing of books,
thesis documents and lecture notes somewhat more convenient.
However, the package can also be used efficiently for
composing a series of documents (such as exercise sheets)
which are typically distributed individually.
It then assists the author in generating the individual documents
(potentially in different versions)
as well as a document containing the collected series.
Another application is in developing style files
or other kinds of included material
where compilation of the style file could redirect
to a sample or test file.

%%%%%%%%%%%%%%%%%%%%%%%%%%%%%%%%%%%%%%%%%%%%%%%%%%%%%%%%%%%%%%%%%%%%%%%%%%%%%%%%
%%%%%%%%%%%%%%%%%%%%%%%%%%%%%%%%%%%%%%%%%%%%%%%%%%%%%%%%%%%%%%%%%%%%%%%%%%%%%%%%
\section{Usage}

First of all, the package \textsf{childdoc} is \emph{not} a standard
\LaTeXe{} |.sty| style file! Therefore it needs to be invoked in
a non-standard way.

%%%%%%%%%%%%%%%%%%%%%%%%%%%%%%%%%%%%%%%%%%%%%%%%%%%%%%%%%%%%%%%%%%%%%%%%%%%%%%%%
\subsection{Included Files}
\label{sec:include}

%%%%%%%%%%%%%%%%%%%%%%%%%%%%%%%%%%%%%%%%
\DescribeMacro{\childdocmain}
To use the package, add the commands
\begin{center}
\begin{tabular}{l}
|\input{childdoc.def}|\\
|\childdocmain{}|\\
\end{tabular}
\end{center}
at the very top of the main \LaTeX{} file,
in particular \emph{before} the |\documentclass| statement!
The argument of |\childdocmain| should be left empty
(but it must be present).

%%%%%%%%%%%%%%%%%%%%%%%%%%%%%%%%%%%%%%%%
\DescribeMacro{\childdocof}
Furthermore, add the commands
\begin{center}
\begin{tabular}{l}
|\input{childdoc.def}|\\
|\childdocof{|\textit{main}|}|\\
\end{tabular}
\end{center}
at the top of every child file \textit{child}
which is included by |\include{|\textit{child}|}|
from within the main file
(or at least for those files to be compiled individually).
The argument \textit{main} must be the filename of the main file.

There are a couple of
considerations in setting up the main and child documents:

%%%%%%%%%%%%%%%%%%%%%%%%%%%%%%%%%%%%%%%%
\paragraph{Restrictions.}

Please note the following restrictions:
\begin{itemize}
\item
|\childdocmain| must be called with one argument \textit{main}
to ensure compatibility with earlier version of the package.
It must either be empty (|\childdocmain{}|)
or precisely match the filename of the main file in which it is specified.
See \secref{sec:detection} for further information.
\item
The filename \textit{main} must be specified without the |.tex| extension.
\item
The filename \textit{main} is case sensitive
(even in case-insensitive file systems)
due to internal string comparison.
\item
The argument \textit{main} should be fully expanded, it cannot be a macro.
\item
Subdirectories and special characters should be avoided in filenames.
\item
The command |\childdocmain{|\textit{main}|}| must be followed by a whitespace.
It should not be followed immediately by another command
or by a comment mark `|%|'.
This is because the \TeX{} parser reads the token immediately following
the argument of |\childdocmain| and puts it
at the beginning of every child section;
however, a white\-space is ignored.
\end{itemize}

%%%%%%%%%%%%%%%%%%%%%%%%%%%%%%%%%%%%%%%%
\paragraph{Content of Main File.}

It is advisable to place all content in the child files included by |\include|.
Any output contained in the main file will appear in all child documents
unless suppressed manually;
it cannot be suppressed automatically by the |\includeonly| directive
and thus should normally be avoided.
A method to include some content in the main file
by means of conditional processing is described in \secref{sec:conditional}.

%%%%%%%%%%%%%%%%%%%%%%%%%%%%%%%%%%%%%%%%
\paragraph{Page Numbering.}

When only a part of the document is compiled,
the appropriate numbering of pages
(as well as other status parameters)
is determined from the |.aux| files.
The latter contain information from previous passes.
However this information needs to propagate through
all intermediate child documents.
Therefore the page numbering in child documents may well
be inconsistent until the complete document is compiled at least once.

A useful (if unconventional) way to always ensure a consistent
page numbering is to restart the numbering in each child document
and denote the pages by `\textit{child}|.|\textit{page}'
where \textit{child} represents the chapter/section number of the child file.
This can be achieved by the command
|\numberwithin{page}{|\textit{child}|}|
of the \textsf{amsmath} package
where \textit{child} can be |chapter| or |section|
depending on the chosen structuring.
Alternatively, one can modify the macro |\thepage| appropriately
and reset the counter |page| at the start of each child file.

%%%%%%%%%%%%%%%%%%%%%%%%%%%%%%%%%%%%%%%%%%%%%%%%%%%%%%%%%%%%%%%%%%%%%%%%%%%%%%%%
\subsection{Conditional Processing}
\label{sec:conditional}

The package provides a mechanism to compile different versions
of a document. To customise the versions further some conditional processing
can come in handy to distinguish which version is being compiled.
The package provides two macros to describe the compilation context:

%%%%%%%%%%%%%%%%%%%%%%%%%%%%%%%%%%%%%%%%
\DescribeMacro{\ifchilddoc}
The conditional |\ifchilddoc| distinguishes between the compilation of
child documents and the main document:
%
\begin{center}
|\ifchilddoc |\textit{child-code}| |[|\||else |\textit{main-code}]| \||fi|
\end{center}

%%%%%%%%%%%%%%%%%%%%%%%%%%%%%%%%%%%%%%%%
\DescribeMacro{\childdocname}
\DescribeMacro{\childdocjob}
The macro |\childdocname| contains the filename (without extension)
of the main or child file being processed.
Note that |\childdocjob| will always contain the name of the main file.

%%%%%%%%%%%%%%%%%%%%%%%%%%%%%%%%%%%%%%%%
\paragraph{Title Page.}

Conditional processing can be used to include a title or banner page
in the main document when proper precautions are taken.
Importantly, the code in the main file should ensure that the page counter
(as well as other status parameters which are stored in the |.aux| files)
takes the same value after the conditional processing.
Otherwise the page numbers may take divergent values
depending on which part is compiled.

For example, a title page could be declared by:
%
\begin{center}
\begin{tabular}{l}
|\ifchilddoc\||else|\\
|\addtocounter{page}{-1}|\\
\textit{code for title page}\\
|\newpage|\\
|\||fi|
\end{tabular}
\end{center}
%
A banner page for the child documents can be generated by:
%
\begin{center}
\begin{tabular}{l}
|\ifchilddoc|\\
|\addtocounter{page}{-1}|\\
\textit{code for banner page}\\
|\newpage|\\
|\||fi|
\end{tabular}
\end{center}
%
Here one could write a message such as:
\begin{center}
|This is the part \childdocname{} of \childdocjob{}.|
\end{center}

%%%%%%%%%%%%%%%%%%%%%%%%%%%%%%%%%%%%%%%%%%%%%%%%%%%%%%%%%%%%%%%%%%%%%%%%%%%%%%%%
\subsection{Flags}
\label{sec:flags}

The package makes it easy to generate different versions
of the main or child documents.
To this end compilation flags can be defined
and assigned different default values.
They will be particularly useful in conjunction
with the forwarding mechanism described in \secref{sec:forward}.

For example, it may be useful to have a flag |\version|
which can be set to |draft| or |final|.
The document source will contain some conditional code
depending on the value of |\version|.
Suppose further, the flag should default to |final| for the main file
and to |draft| for child files
which is a natural assignment for editing the document.
This is achieved by placing the following code
in the preamble of the main document
(below the |\childdocmain| directive):
%
\begin{center}
\begin{tabular}{l}
|\ifchilddoc|\\
|\providecommand{\version}{draft}|\\
|\||else|\\
|\providecommand{\version}{final}|\\
|\||fi|
\end{tabular}
\end{center}
%
The definition by |\providecommand| makes sure
that previous definitions are not overwritten.
Further statements |\providecommand{\version}{...}|
can thus be added before the above code to override it.

For the main file, one might add a line
(between |\childdocmain| and the above block)
%
\begin{center}
|%\ifchilddoc\||else\providecommand{\version}{draft}\||fi|
\end{center}
%
which can be uncommented to produce a draft version.
Likewise one can add a line to the very top of a child file
(above the |\childdocof{|\textit{main}|}| directive)
%
\begin{center}
|%\providecommand{\version}{final}|
\end{center}
%
which can be uncommented to produce the final version of this child document.

%%%%%%%%%%%%%%%%%%%%%%%%%%%%%%%%%%%%%%%%%%%%%%%%%%%%%%%%%%%%%%%%%%%%%%%%%%%%%%%%
\subsection{Forwarding}
\label{sec:forward}

Different versions of the main or child documents
using compilation flags as described in \secref{sec:flags}
can be (permanently) stored in different files
for convenient compilation, viewing and distribution.
To this end, the package defines a command
to pass on compilation to a different file:

%%%%%%%%%%%%%%%%%%%%%%%%%%%%%%%%%%%%%%%%
\DescribeMacro{\childdocforward}
The command |\childdocforward| redirects processing to
another source file:
%
\begin{center}
\begin{tabular}{l}
|\input{childdoc.def}|\\
|\childdocforward[|\textit{main}|]{|\textit{dest}|}|\\
\end{tabular}
\end{center}
%
The argument \textit{dest} is the destination file
(without extension).
It should be the main file or one of the child files.
Note that further \textsf{childdoc} directives
such as |\childdocof| and |\childdocforward|
in the indicated file will be processed in this form.
The optional argument \textit{main}
passes on directly to the main file \textit{main}
while pretending to compile the child \textit{dest}.
This form behaves as if \textit{dest}
issues |\childdocof{|\textit{main}|}| right away,
and no further \textsf{childdoc} directives will be processed.

%%%%%%%%%%%%%%%%%%%%%%%%%%%%%%%%%%%%%%%%
\DescribeMacro{\...prefix}
In the alternative form |\childdocforwardprefix|,
%
\begin{center}
\begin{tabular}{l}
|\input{childdoc.def}|\\
|\childdocforwardprefix[|\textit{main}|]{|\textit{prefix}|}{|\textit{dest}|}|
\end{tabular}
\end{center}
%
the destination file is determined by a pattern
depending on the current file:
To make this work, the current file must be called
`{\textit{prefix}\hspace{0.2em}\textit{suffix}}'
with \textit{prefix} matching precisely the argument.
Processing is then passed on to the file
`{\textit{dest}\hspace{0.2em}\textit{suffix}}'.
Surely, the same effect is achieved by
directly specifying the
argument `{\textit{dest}\hspace{0.2em}\textit{suffix}}'
in the first form.
However, that requires to set up a different file
for each child. With the alternative form of the command
all these files can have exactly the same content
which simplifies setting them up and maintaining them.

For example, the following file |draft.tex|
with a compilation flag |\version| as described in \secref{sec:flags}
compiles the main document as a draft:
%
\begin{center}
\begin{tabular}{l}
|\def\version{draft}|\\
|\input{childdoc.def}|\\
|\childdocforward{|\textit{main}|}|
\end{tabular}
\end{center}
%
Likewise, the following files |final|\textit{nn}|.tex|
compile the final version of the child document
|child|\textit{nn}|.tex|:
%
\begin{center}
\begin{tabular}{l}
|\def\version{final}|\\
|\input{childdoc.def}|\\
|\childdocforwardprefix{final}{child}|
\end{tabular}
\end{center}
%

Note that when several versions of a main file and/or of each child file
are to be generated, it may be convenient to set up a |Makefile| or
shell script to automatise the process.

%%%%%%%%%%%%%%%%%%%%%%%%%%%%%%%%%%%%%%%%%%%%%%%%%%%%%%%%%%%%%%%%%%%%%%%%%%%%%%%%
\subsection{Command Line Processing}
\label{sec:commandline}

The effect of redirection files can also be achieved by invoking
the \LaTeX{} compiler with a more elaborate command line.
Most conveniently this should be done as part
of a shell script or a |Makefile|.

When using \textsf{childdoc} in the main file, the following
command lines effectively perform a redirection
(note that depending on the shell being used,
backslashes may have to be doubled: `|\|' $\to$ `|\\|'):
%
\begin{center}
|... -jobname "|\textit{target}|" |\\|"|[\textit{flags}]%
|\input{childdoc.def}\childdocforward[|\textit{main}|]{|\textit{dest}|}"|
\end{center}
%
Here \textit{target} is the name of the output file,
\textit{main} is the name of the main file
and \textit{dest} is the name of the main or child file to be processed
(all filenames without extensions).
The optional argument \textit{main} can be omitted
if \textit{main} matches \textit{dest}.
Optionally, compilation \textit{flags} can be defined via |\def| commands.
This command line makes the \TeX{} engine believe
it is compiling the file \textit{target}
whose content is specified as the latter parameter.
The provided code then forwards the processing to
\textit{main} or \textit{dest} as described in \secref{sec:forward}.

%%%%%%%%%%%%%%%%%%%%%%%%%%%%%%%%%%%%%%%%%%%%%%%%%%%%%%%%%%%%%%%%%%%%%%%%%%%%%%%%
\subsection{Include by Input}
\label{sec:input}

Including child documents by |\include| has some restrictions by design.
Most notably, the content of a child document always occupies
its own set of pages; pages cannot be shared between child documents.
Usually, this behaviour makes perfect sense
because each child document contain an essential part of the document.
However, in some situations it may be desirable to compose
a document from a collection of parts
without having mandatory page breaks between then.
For this case, the package
provides a mechanism to include parts
by |\input| which can also be processed individually.
However, by construction this mechanism
requires manual handling of the content to be output.

%%%%%%%%%%%%%%%%%%%%%%%%%%%%%%%%%%%%%%%%
\DescribeMacro{\ifchilddocmanual}
The main file should be prepared as usual, see \secref{sec:include}.
However, the document body must make a distinction
between processing of an individual part and of the main document, e.g.:
%
\begin{center}
\begin{tabular}{l}
|\ifchilddocmanual|\\
|\input{\childdocname}|\\
|\||else|\\
\textit{document body with }|\input{|\textit{part}|}|\\
|\||fi|
\end{tabular}
\end{center}
%
The conditional |\ifchilddocmanual| is true whenever
a part to be included by |\input| is being compiled,
and the name of the part is stored in |\childdocname|.

%%%%%%%%%%%%%%%%%%%%%%%%%%%%%%%%%%%%%%%%
\DescribeMacro{\childdocby}
Each part to be included by |\input| should start with:
%
\begin{center}
\begin{tabular}{l}
|\input{childdoc.def}|\\
|\childdocby{|\textit{main}|}|\\
\end{tabular}
\end{center}
%
The directive |\childdocby| is similar to |\childdocof|
described in \secref{sec:include},
but the subsequent selection of content must be done manually.
To that end, both |\ifchilddoc| and |\ifchilddocmanual|
will be true upon processing of a part,
and the name of the part is stored in |\childdocname|.
Note that |\jobname| will be set to the filename of the current part
so that each part receives an individual |.aux| file
that does not interfere with the |.aux| file(s) of the main document.
This behaviour can be altered by the alternative form
|\childdocby[*]{|\textit{main}|}| (with a non-empty optional argument)
which uses the |.aux| file of the main document
by setting |\jobname| to \textit{main}.

%%%%%%%%%%%%%%%%%%%%%%%%%%%%%%%%%%%%%%%%%%%%%%%%%%%%%%%%%%%%%%%%%%%%%%%%%%%%%%%%
\subsection{Driver Development}
\label{sec:driver}

The \textsf{childdoc} mechanism can also be use for the development
of definition files such as \LaTeX{} styles or classes.
This case differs from the above setup with multiple parts
included by |\include| in that no |\includeonly| should be invoked.
This can be achieved by starting the include file
(before |\ProvidesPackage|) with:
%
\begin{center}
\begin{tabular}{l}
|\input{childdoc.def}|\\
|\childdocforward{|\textit{main}|}|\\
\end{tabular}
\end{center}
%
or alternatively with:
%
\begin{center}
\begin{tabular}{l}
|\input{childdoc.def}|\\
|\childdocby{|\textit{main}|}|\\
\end{tabular}
\end{center}
%
Both forms have slightly different effects as described above.
The main file is prepared as usual, see \secref{sec:include}.

%%%%%%%%%%%%%%%%%%%%%%%%%%%%%%%%%%%%%%%%%%%%%%%%%%%%%%%%%%%%%%%%%%%%%%%%%%%%%%%%
\subsection{Legacy Detection}
\label{sec:detection}

The directive |\childdocmain| in the main file can detect
whether the complete document or merely a child is to be compiled
even without using the directive |\childdocof|.
This method is deprecated because it is less robust
and there is no compelling reason to use it;
it is merely provided for backward compatibility
and it may be removed in future versions.

If the detection mechanism is to be used,
it is mandatory to correctly specify
the filename of the main file as the argument of |\childdocmain|:
%
\begin{center}
\begin{tabular}{l}
|\input{childdoc.def}|\\
|\childdocmain{|\textit{main}|}|\\
\end{tabular}
\end{center}
%
If |\jobname| does not match the argument \textit{main} of |\childdocmain|,
it is assumed that |\jobname| points to the child file to be compiled.
When using |\childdocmain| with the main file specified as argument,
it suffices to start a child file
with just |\input{|\textit{main}|}|
without loading of the package and using |\childdocof|.
If instead all processing is done
with the appropriate \textsf{childdoc} directives,
the argument of \textit{main} of |\childdocmain| can be empty.

An alternative version of the command line processing described
in \secref{sec:commandline} using the detection mechanism reads:
%
\begin{center}
|... -jobname "|\textit{target}|" "|[\textit{flags}]%
[|\def\jobname{|\textit{dest}|}|]|\input{|\textit{main}|}"|
\end{center}

%%%%%%%%%%%%%%%%%%%%%%%%%%%%%%%%%%%%%%%%%%%%%%%%%%%%%%%%%%%%%%%%%%%%%%%%%%%%%%%%
\subsection{Manual Code}
\label{sec:manual}

In case one cannot be certain whether the definitions file |childdoc.def|
is installed on the target \TeX{} distribution
and one prefers not to ship it,
it is conceivable to paste a few relevant commands into the sources.

To that end, drop all statements |\input{childdoc.def}|
and perform the replacements as outlined below.
Instead of |\childdocmain{|\textit{main}|}| add the following code
to the top of the main file:
%
\begin{center}
\begin{tabular}{l}
|\||ifdefined\childdocname\endinput\||fi\newif\ifchilddoc|\\
|\edef\childdocname{\scantokens\expandafter{\jobname\noexpand}}|\\
|\def\childdocmain{|\textit{main}|}\||ifx\childdocmain\childdocname\||else|\\
|\childdoctrue\includeonly{\childdocname}\let\jobname\childdocmain\||fi|\\
\end{tabular}
\end{center}
%
Instead of |\childdocof{|\textit{main}|}| just include the main file
at the top of each child file:
%
\begin{center}
|\input{|\textit{main}|}|
\end{center}
%
A simple redirection |\childdocforward{|\textit{dest}|}| is achieved by:
%
\begin{center}
|\def\jobname{|\textit{dest}|}\input{\jobname}|
\end{center}
%
The redirection with prefix
|\childdocforwardprefix[|\textit{prefix}|]{|\textit{dest}|}|
is accomplished by:
%
\begin{center}
\begin{tabular}{l}
|{\edef\jobname{\scantokens\expandafter{\jobname\noexpand}}|\\
|\def\redirectjob |\textit{prefix}|#1~~~{\gdef\jobname{|\textit{dest}|#1}}|\\
|\expandafter\redirectjob\jobname~~~}\input{\jobname}|
\end{tabular}
\end{center}

In an alternative approach,
child documents can be compiled by a specific command line
without additional code or specific definitions:
%
\begin{center}
|... -jobname "|\textit{target}|" "|[\textit{flags}]%
|\includeonly{|\textit{dest}|}\input{|\textit{main}|}"|
\end{center}
%

%%%%%%%%%%%%%%%%%%%%%%%%%%%%%%%%%%%%%%%%%%%%%%%%%%%%%%%%%%%%%%%%%%%%%%%%%%%%%%%%
%%%%%%%%%%%%%%%%%%%%%%%%%%%%%%%%%%%%%%%%%%%%%%%%%%%%%%%%%%%%%%%%%%%%%%%%%%%%%%%%
\section{Information}

%%%%%%%%%%%%%%%%%%%%%%%%%%%%%%%%%%%%%%%%%%%%%%%%%%%%%%%%%%%%%%%%%%%%%%%%%%%%%%%%
\subsection{Copyright}

Copyright \copyright{} 2017--2018 Niklas Beisert

This work may be distributed and/or modified under the
conditions of the \LaTeX{} Project Public License, either version 1.3
of this license or (at your option) any later version.
The latest version of this license is in
  \url{http://www.latex-project.org/lppl.txt}
and version 1.3 or later is part of all distributions of \LaTeX{}
version 2005/12/01 or later.

This work has the LPPL maintenance status `maintained'.

The Current Maintainer of this work is Niklas Beisert.

This work consists of the files |README.txt|, |childdoc.ins| and |childdoc.dtx|
as well as the derived files |childdoc.def|, |cdocsamp.tex|
with |cdocsch1.tex|, |cdocsch2.tex|, |cdocspt3.tex|, |cdocspt4.tex|,
|cdocsdrf.tex|, |cdocsfn1.tex|, |cdocsfn2.tex|
as well as |childdoc.pdf|.

%%%%%%%%%%%%%%%%%%%%%%%%%%%%%%%%%%%%%%%%%%%%%%%%%%%%%%%%%%%%%%%%%%%%%%%%%%%%%%%%
\subsection{Files and Installation}

The package consists of the files:
%
\begin{center}
\begin{tabular}{ll}
    |README.txt|   & readme file \\
    |childdoc.ins| & installation file \\
    |childdoc.dtx| & source file \\
    |childdoc.def| & definition file \\
    |cdocsamp.tex| & sample main file \\
    |cdocsch1.tex| & sample include file \\
    |cdocsch2.tex| & sample include file \\
    |cdocspt3.tex| & sample part file \\
    |cdocspt4.tex| & sample part file \\
    |cdocsdrf.tex| & sample redirection file \\
    |cdocsfn1.tex| & sample redirection file \\
    |cdocsfn2.tex| & sample redirection file \\
    |childdoc.pdf| & manual
\end{tabular}
\end{center}
%
The distribution consists of the files
|README.txt|, |childdoc.ins| and |childdoc.dtx|.
%
\begin{itemize}
\item
Run (pdf)\LaTeX{} on |childdoc.dtx|
to compile the manual |childdoc.pdf| (this file).
\item
Run \LaTeX{} on |childdoc.ins| to create the definitions file |childdoc.def|
and the sample |cdocsamp.tex| with include files
|cdocsch1.tex|, |cdocsch2.tex|, |cdocspt3.tex|, |cdocspt4.tex|,
|cdocsdrf.tex|, |cdocsfn1.tex|, |cdocsfn2.tex|.
Then copy the file |childdoc.def| to an appropriate directory of your \LaTeX{}
distribution, e.g.\ \textit{texmf-root}|/tex/latex/childdoc|.
\end{itemize}

%%%%%%%%%%%%%%%%%%%%%%%%%%%%%%%%%%%%%%%%%%%%%%%%%%%%%%%%%%%%%%%%%%%%%%%%%%%%%%%%
\subsection{Related CTAN Packages}

There are several other packages which offer a similar functionality:
%
\begin{itemize}
\item
The packages
\href{http://ctan.org/pkg/docmute}{\textsf{docmute}},
\href{http://ctan.org/pkg/includex}{\textsf{includex}} and
\href{http://ctan.org/pkg/standalone}{\textsf{standalone}}
provide commands to include only the document body of
a child file thus allowing both files to be compiled individually.
\item
The packages \href{http://ctan.org/pkg/subdocs}{\textsf{subdocs}}
and \href{http://ctan.org/pkg/subfiles}{\textsf{subfiles}}
provide structures in which the main and child documents can be
encapsulated and allowing them to be compiled individually.
The inclusion mechanism is different from the conventional |\include|.
\item
The package \href{http://ctan.org/pkg/combine}{\textsf{combine}}
is an elaborate solution to combine several documents into one.
\end{itemize}
%
See also the CTAN topic \href{http://ctan.org/topic/subdocs}{\textsf{subdocs}}
for further related packages.
The present package differs from the above solutions in that
a document structure constructed with the conventional |\include| mechanism
just needs two extra commands at the top of every file
such that all constituent files can be compiled individually.

%%%%%%%%%%%%%%%%%%%%%%%%%%%%%%%%%%%%%%%%%%%%%%%%%%%%%%%%%%%%%%%%%%%%%%%%%%%%%%%%
%\subsection{Feature Suggestions}
%
%The following is a list of features which may be useful for future
%versions of this package:
%%
%\begin{itemize}
%\item
%\ldots
%\end{itemize}

%%%%%%%%%%%%%%%%%%%%%%%%%%%%%%%%%%%%%%%%%%%%%%%%%%%%%%%%%%%%%%%%%%%%%%%%%%%%%%%%
\subsection{Revision History}

%%%%%%%%%%%%%%%%%%%%%%%%%%%%%%%%%%%%%%%%
\paragraph{v2.0:} 2018/12/30

\begin{itemize}
\item
immediate forward processing
\item
added |\childdocby| mechanism
\item
manual restructured
\end{itemize}

%%%%%%%%%%%%%%%%%%%%%%%%%%%%%%%%%%%%%%%%
\paragraph{v1.6:} 2018/01/17

\begin{itemize}
\item
application for development of include files
\item
corrections to manual
\end{itemize}

%%%%%%%%%%%%%%%%%%%%%%%%%%%%%%%%%%%%%%%%
\paragraph{v1.5:} 2017/05/21

\begin{itemize}
\item
more complete structuring introduced
\item
|\childdocof| introduced
\item
|\childdoc| renamed to |\childdocmain|
\item
|\childredirect| renamed to |\childdocforward| and |\childdocforwardprefix|
and functionality expanded
\end{itemize}

%%%%%%%%%%%%%%%%%%%%%%%%%%%%%%%%%%%%%%%%
\paragraph{v1.0:} 2017/04/27

\begin{itemize}
\item
manual and install package
\item
first version published on CTAN
\end{itemize}

%%%%%%%%%%%%%%%%%%%%%%%%%%%%%%%%%%%%%%%%
\paragraph{v0.6:} 2017/04/26

\begin{itemize}
\item
redirection mechanism added
\end{itemize}

%%%%%%%%%%%%%%%%%%%%%%%%%%%%%%%%%%%%%%%%
\paragraph{v0.5:} 2017/04/26

\begin{itemize}
\item
functionality in definition file
\end{itemize}


%%%%%%%%%%%%%%%%%%%%%%%%%%%%%%%%%%%%%%%%%%%%%%%%%%%%%%%%%%%%%%%%%%%%%%%%%%%%%%%%
%%%%%%%%%%%%%%%%%%%%%%%%%%%%%%%%%%%%%%%%%%%%%%%%%%%%%%%%%%%%%%%%%%%%%%%%%%%%%%%%
%%%%%%%%%%%%%%%%%%%%%%%%%%%%%%%%%%%%%%%%%%%%%%%%%%%%%%%%%%%%%%%%%%%%%%%%%%%%%%%%
\appendix

\settowidth\MacroIndent{\rmfamily\scriptsize 000\ }

 \DocInput{childdoc.dtx}

\end{document}
%</driver>
% \fi
%
% %%%%%%%%%%%%%%%%%%%%%%%%%%%%%%%%%%%%%%%%%%%%%%%%%%%%%%%%%%%%%%%%%%%%%%%%%%%%%%
% %%%%%%%%%%%%%%%%%%%%%%%%%%%%%%%%%%%%%%%%%%%%%%%%%%%%%%%%%%%%%%%%%%%%%%%%%%%%%%
% \section{Sample}
%\iffalse
%<*samplemain>
%\fi
%
% The following presents a sample document
% with two chapters, two parts, a title page,
% a compile flag as well as three forwarding files to set the flag.
% It consists of eight |.tex| files:
% \begin{center}
% \begin{tabular}{ll}
% |cdocsamp.tex|&main file\\
% |cdocsch1.tex|&include file for chapter 1\\
% |cdocsch2.tex|&include file for chapter 2\\
% |cdocspt3.tex|&include file for part 3\\
% |cdocspt4.tex|&include file for part 4\\
% |cdocsdrf.tex|&forwarding file for main file in draft mode\\
% |cdocsfi1.tex|&forwarding file for final version of chapter 1\\
% |cdocsfi2.tex|&forwarding file for final version of chapter 2\\
% \end{tabular}
% \end{center}
% Each of the eight files can be compiled directly by the \LaTeX{} compiler.
%
% %%%%%%%%%%%%%%%%%%%%%%%%%%%%%%%%%%%%%%
% \paragraph{Main File.}
%
% The main file is called |cdocsamp.tex|.
%
% Load the \textsf{childdoc} definitions and
% declare the filename for the main document:
%    \begin{macrocode}
\input{childdoc.def}
\childdocmain{}
%    \end{macrocode}

% Optional override for |\version| flag:
%    \begin{macrocode}
%%\ifchilddoc\else\providecommand{\version}{draft}\fi
%    \end{macrocode}

% Define the default values for the |\version| flag
% (|final| for the main file and |draft| for childs):
%    \begin{macrocode}
\ifchilddoc
\providecommand{\version}{draft}
\else
\providecommand{\version}{final}
\fi
%    \end{macrocode}

% Load the standard document class:
%    \begin{macrocode}
\documentclass[12pt]{article}
%    \end{macrocode}

% Start the document body:
%    \begin{macrocode}
\begin{document}
%    \end{macrocode}

% Declare a title page.
% Print title, part of document being processed and version flag:
%    \begin{macrocode}
\addtocounter{page}{-1}
\begin{center}
{\LARGE\bfseries{}childdoc example\par}
\vspace{1cm}
\ifchilddoc
\ifchilddocmanual part\else chapter\fi:
`\childdocname' of `\childdocjob'\par
\else
main document: `\childdocjob'\par
\fi
version: \version\par
\end{center}
\newpage
%    \end{macrocode}

% Manually include selected file,
% otherwise process as usual:
%    \begin{macrocode}
\ifchilddocmanual
\section*{part `\childdocname'}
\input{\childdocname}
\else
%    \end{macrocode}

% Include the two chapters:
%    \begin{macrocode}
\include{cdocsch1}
\include{cdocsch2}
%    \end{macrocode}

% Include the two parts unless only chapters should be displayed:
%    \begin{macrocode}
\ifchilddoc\else
\section{part three}
\input{cdocspt3}
\section{part four}
\input{cdocspt4}
\fi
%    \end{macrocode}

% Process as usual until here:
%    \begin{macrocode}
\fi
%    \end{macrocode}

% End of document body:
%    \begin{macrocode}
\end{document}
%    \end{macrocode}
%\iffalse
%</samplemain>
%\fi
%
% %%%%%%%%%%%%%%%%%%%%%%%%%%%%%%%%%%%%%%
% \paragraph{Chapter Include Files.}
%
% The include files are called |cdocsch1.tex| and |cdocsch2.tex|.
%
%\iffalse
%<*samplechap1|samplechap2>
%\fi

% Optional override for |\version| flag:
%    \begin{macrocode}
%%\providecommand{\version}{final}
%    \end{macrocode}

% Include the main document:
%    \begin{macrocode}
\input{childdoc.def}
\childdocof{cdocsamp}
%    \end{macrocode}

%\iffalse
%</samplechap1|samplechap2>
%\fi
%
%\iffalse
%<*samplechap1>
%\fi
% Some text for chapter 1:
%    \begin{macrocode}
\section{one}
some text in chapter one
%    \end{macrocode}

%\iffalse
%</samplechap1>
%\fi
% Some text for chapter 2:
%\iffalse
%<*samplechap2>
%\fi
%    \begin{macrocode}
\section{two}
more text in chapter two
%    \end{macrocode}

%\iffalse
%</samplechap2>
%\fi
%
% %%%%%%%%%%%%%%%%%%%%%%%%%%%%%%%%%%%%%%
% \paragraph{Part Include Files.}
%
% The include files are called |cdocspt3.tex| and |cdocspt4.tex|.
%
%\iffalse
%<*samplepart3|samplepart4>
%\fi

% Optional override for |\version| flag:
%    \begin{macrocode}
%%\providecommand{\version}{final}
%    \end{macrocode}

% Include the main document:
%    \begin{macrocode}
\input{childdoc.def}
\childdocby{cdocsamp}
%    \end{macrocode}

%\iffalse
%</samplepart3|samplepart4>
%\fi
%
%\iffalse
%<*samplepart3>
%\fi
% Some text for part 3:
%    \begin{macrocode}
some text in part three
%    \end{macrocode}

%\iffalse
%</samplepart3>
%\fi
% Some text for part 4:
%\iffalse
%<*samplepart4>
%\fi
%    \begin{macrocode}
more text in part four
%    \end{macrocode}

%\iffalse
%</samplepart4>
%\fi
%
% %%%%%%%%%%%%%%%%%%%%%%%%%%%%%%%%%%%%%%
% \paragraph{Forwarding for a Complete Draft.}
%
% The following forwarding file |cdocsdrf.tex|
% compiles the main document in draft mode:
%\iffalse
%<*sampledraft>
%\fi
%    \begin{macrocode}
\def\version{draft}
\input{childdoc.def}
\childdocforward{cdocsamp}
%    \end{macrocode}

%\iffalse
%</sampledraft>
%\fi
%
% %%%%%%%%%%%%%%%%%%%%%%%%%%%%%%%%%%%%%%
% \paragraph{Forwarding for Final Version of the Chapters.}
%
% The following forwarding files |cdocsfn1.tex| and |cdocsfn2.tex|
% (with identical content)
% compile the final versions of the child documents
% |cdocsch1.tex| and |cdocsch2.tex|, respectively:
%\iffalse
%<*samplefinal>
%\fi
%    \begin{macrocode}
\def\version{final}
\input{childdoc.def}
\childdocforwardprefix[cdocsamp]{cdocsfn}{cdocsch}
%    \end{macrocode}

%\iffalse
%</samplefinal>
%\fi
%
% %%%%%%%%%%%%%%%%%%%%%%%%%%%%%%%%%%%%%%
% \paragraph{Command Line Processing.}
%
% The following three command lines generate the output files
% |cdocscld|, |cdocscl1| and |cdocscl2|
% which should be identical to
% |cdocsdrf|, |cdocsch1| and |cdocsfn2|, respectively:
% \begin{center}
% \begin{tabular}{l}
% |latex -jobname cdocscld \|\\
% |  "\def\version{draft}\input{childdoc.def}\childdocforward{cdocsamp}"|\\
% |latex -jobname cdocscl1 \|\\
% |  "\input{childdoc.def}\childdocforward[cdocsamp]{cdocsch1}"|\\
% |latex -jobname cdocscl2 \|\\
% |  "\def\version{final}\input{childdoc.def}\childdocforward{cdocsch2}"|
% \end{tabular}
% \end{center}
% Note that the trailing backslash on each first line
% merely continues the input to the second line
% (for convenient cut ant paste).
% Furthermore, the command |latex| can be replaced by any
% of its alternative versions such as |pdflatex|.
%
% %%%%%%%%%%%%%%%%%%%%%%%%%%%%%%%%%%%%%%%%%%%%%%%%%%%%%%%%%%%%%%%%%%%%%%%%%%%%%%
% %%%%%%%%%%%%%%%%%%%%%%%%%%%%%%%%%%%%%%%%%%%%%%%%%%%%%%%%%%%%%%%%%%%%%%%%%%%%%%
% \section{Implementation}
%\iffalse
%<*package>
%\fi
%
% This section describes the definitions file |childdoc.def|.

% The definitions cannot be loaded using |\usepackage| or |\RequirePackage|
% which has a mechanism to prevent loading a style file more than once.
% When loading the definitions by means of |\input|
% multiple instances have to be prevented manually:
%\iffalse
%This code needs to be before the `\ProvidesFile' directive
%which is defined at the beginning of this file.
%Therefore it is also placed there and commented out here.
%</package>
%<*discard>
%\fi
%    \begin{macrocode}
\ifdefined\childdocmain\endinput\fi
%    \end{macrocode}
%\iffalse
%</discard>
%<*package>
%\fi
%
% \macro{\ifchilddoc}
% \macro{\ifchilddocmanual}
% The conditional |\ifchilddoc| tells whether a
% child (true) or main (false) document is being compiled.
% The conditional |\ifchilddocmanual| tells whether
% the |\includeonly| mechanism is used (false) or
% the selection of child files must be performed manually (true).
% The definitions initialise to false:
%    \begin{macrocode}
\newif\ifchilddoc
\newif\ifchilddocmanual
%    \end{macrocode}

% \macro{\childdocname}
% \macro{\childdocjob}
% The macro |\childdocname| stores the name of the main document
% to be compiled. The macro |\childdocjob| stores the name of
% the document on which the \LaTeX{} compiler was originally invoked.
% The content of |\jobname| cannot be compared
% to filenames specified in the source due to different catcodes.
% The following code rescans |\jobname|, stores the result
% in |\childdocname| and saves a copy in |\childdocjob|:
%    \begin{macrocode}
\edef\childdocname{\scantokens\expandafter{\jobname\noexpand}}
\let\childdocjob\childdocname
%    \end{macrocode}

% \macro{\childdocdisable}
% The macro |\childdocdisable| prevents the main file
% from being processed more than once.
% At this stage, the main document command |\childdocmain|
% is assumed to be called once again where it should do nothing.
% Any subsequent call to it should prevent
% a secondary processing of the main document
% It overwrites the forwarding commands
% |\childdocof| and |\childdocforward|
% with empty macros to prevent further inclusions of the main document:
%    \begin{macrocode}
\newcommand{\childdocdisable}
{
  \renewcommand{\childdocmain}[1]{\renewcommand{\childdocmain}[1]{\endinput}}
  \renewcommand{\childdocof}[1]{}
  \renewcommand{\childdocby}[2][]{}
  \renewcommand{\childdocforward}[2][]{}
  \renewcommand{\childdocdisable}{}
}
%    \end{macrocode}

% \macro{\childdocmain}
% The macro |\childdocmain| is to be called at the top of the main file
% with nothing or the main filename (without extension) as argument.
% First, it breaks loops.
% If the argument is not empty and does not match |\childdocname|
% (which is set by the first inclusion of |childdoc.def|),
% |\ifchilddoc| is set to true, |\includeonly| is applied to the child file
% and |\jobname| is set to the main file
% (for proper handling of |.aux| files):
%    \begin{macrocode}
\newcommand{\childdocmain}[1]
{
  \childdocdisable\childdocmain{}
  \if?#1?\else
    \begingroup
      \def\childdoctmp{#1}
      \ifx\childdoctmp\childdocname
        \def\childdoctmp{}
      \else
        \def\childdoctmp
        {
          \childdoctrue
          \includeonly{\childdocname}
          \def\childdocjob{#1}
          \def\jobname{#1}
        }
      \fi
      \expandafter
    \endgroup
    \childdoctmp
  \fi
}
%    \end{macrocode}

% \macro{\childdocof}
% The command |\childdocof| redirects
% compilation to the main file |#1|.
%    \begin{macrocode}
\newcommand{\childdocof}[1]
{
  \childdocdisable
  \childdoctrue
  \includeonly{\childdocname}
  \def\jobname{#1}
  \def\childdocjob{#1}
  \input{#1}
}
%    \end{macrocode}

% \macro{\childdocby}
% The command |\childdocby| ....
%    \begin{macrocode}
\newcommand{\childdocby}[2][]
{
  \childdocdisable
  \childdoctrue
  \childdocmanualtrue
  \if?#1?\else
    \def\jobname{#2}
  \fi
  \def\childdocjob{#2}
  \input{#2}
  \endinput
}
%    \end{macrocode}

% \macro{\childdocforward}
% The command |\childdocforward| redirects
% compilation to the main file or
% (if the optional argument is given) a child file.
% Parameters are set as if the main file
% or a child file starting with |\childdocof| was compiled.
% Then compilation is handed over to the main file:
%    \begin{macrocode}
\newcommand{\childdocforward}[2][]
{
  \begingroup
    \if?#1?
      \def\childdoctmp
      {
        \def\childdocname{#2}
        \def\childdocjob{#2}
        \def\jobname{#2}
        \input{#2}
        \endinput
      }
    \else
      \def\childdoctmp
      {
        \childdocdisable
        \def\childdocname{#2}
        \childdoctrue
        \includeonly{#2}
        \def\childdocjob{#1}
        \def\jobname{#1}
        \input{#1}
        \endinput
      }
    \fi
    \expandafter
  \endgroup
  \childdoctmp
}
%    \end{macrocode}

% \macro{\childdocforwardprefix}
% The command |\childdocforwardprefix| redirects
% compilation to the main or a child file by means of a pattern.
% The prefix |#1| in the current filename is replaced by |#2|
% and the suffix of the current filename is kept
% (it is assumed that the filename does not contain the substring `|~~~|'
% which is used as a delimiter).
% Compilation is handed over to the new file by |\childdocforward|:
%    \begin{macrocode}
\newcommand{\childdocforwardprefix}[3][]
{
  \begingroup
    \def\childdocextract #2##1~~~{\def\childdoctmp{\childdocforward[#1]{#3##1}}}
    \expandafter\childdocextract\childdocname~~~
    \expandafter
  \endgroup
  \childdoctmp
}
%    \end{macrocode}

% \macro{\childdoc}
% The deprecated macro |\childdoc| is a legacy version of |\childdocmain|:
%    \begin{macrocode}
\newcommand{\childdoc}{\childdocmain}
%    \end{macrocode}

% \macro{\childdocredirect}
% The deprecated macro |\childdocredirect| is a legacy version
% of |\childdocforward| and |\childdocforwardprefix|:
%    \begin{macrocode}
\newcommand{\childdocredirect}[2][]
{
  \begingroup
    \if?#1?
      \def\childdoctmp{\childdocforward{#2}}
    \else
      \def\childdoctmp{\childdocforwardprefix{#1}{#2}}
    \fi
    \expandafter
  \endgroup
  \childdoctmp
}
%    \end{macrocode}

%\iffalse
%</package>
%\fi
%
\endinput
\childdocforward[cdocsamp]{cdocsch1}"|\\
% |latex -jobname cdocscl2 \|\\
% |  "\def\version{final}% \iffalse
%
% childdoc.dtx Copyright (C) 2017-2018 Niklas Beisert
%
% This work may be distributed and/or modified under the
% conditions of the LaTeX Project Public License, either version 1.3
% of this license or (at your option) any later version.
% The latest version of this license is in
%   http://www.latex-project.org/lppl.txt
% and version 1.3 or later is part of all distributions of LaTeX
% version 2005/12/01 or later.
%
% This work has the LPPL maintenance status `maintained'.
%
% The Current Maintainer of this work is Niklas Beisert.
%
% This work consists of the files childdoc.dtx and childdoc.ins
% and the derived files childdoc.def and cdocsamp.tex with
% cdocsch1.tex, cdocsch2.tex, cdocsdrf.tex, cdocsfn1.tex, cdocsfn2.tex.
%
%<package>\ifdefined\childdocmain\endinput\fi
%<package>\ProvidesFile{childdoc.def}[2018/12/30 v2.0 child document driver]
%<samplemain>\ProvidesFile{cdocsamp.tex}[2018/12/30 v2.0 sample for childdoc]
%<*driver>
%\ProvidesFile{childdoc.drv}[2018/12/30 v2.0 childdoc reference manual file]
\PassOptionsToClass{10pt,a4paper}{article}
\documentclass{ltxdoc}

\usepackage[margin=35mm]{geometry}
\usepackage{hyperref}
\usepackage{hyperxmp}
\usepackage[usenames]{color}

\hypersetup{colorlinks=true}
\hypersetup{pdfstartview=FitH}
\hypersetup{pdfpagemode=UseNone}
\hypersetup{pdfsource={}}
\hypersetup{pdflang={en-UK}}
\hypersetup{pdfcopyright={Copyright 2017-2018 Niklas Beisert.
  This work may be distributed and/or modified under the
  conditions of the LaTeX Project Public License, either version 1.3
  of this license or (at your option) any later version.}}
\hypersetup{pdflicenseurl={http://www.latex-project.org/lppl.txt}}
\hypersetup{pdfcontactaddress={ETH Zurich, ITP, HIT K,
  Wolfgang-Pauli-Strasse 27}}
\hypersetup{pdfcontactpostcode={8093}}
\hypersetup{pdfcontactcity={Zurich}}
\hypersetup{pdfcontactcountry={Switzerland}}
\hypersetup{pdfcontactemail={nbeisert@itp.phys.ethz.ch}}
\hypersetup{pdfcontacturl={http://people.phys.ethz.ch/\xmptilde nbeisert/}}

\newcommand{\secref}[1]{\hyperref[#1]{section \ref*{#1}}}

\parskip1ex
\parindent0pt
\let\olditemize\itemize
\def\itemize{\olditemize\parskip0pt}

\begin{document}

\title{The \textsf{childdoc} Package}
\hypersetup{pdftitle={The childdoc Package}}
\author{Niklas Beisert\\[2ex]
  Institut f\"ur Theoretische Physik\\
  Eidgen\"ossische Technische Hochschule Z\"urich\\
  Wolfgang-Pauli-Strasse 27, 8093 Z\"urich, Switzerland\\[1ex]
  \href{mailto:nbeisert@itp.phys.ethz.ch}
  {\texttt{nbeisert@itp.phys.ethz.ch}}}
\hypersetup{pdfauthor={Niklas Beisert}}
\hypersetup{pdfsubject={Manual for the LaTeX2e Package childdoc}}
\date{30 December 2018, \textsf{v2.0}}
\maketitle

\begin{abstract}\noindent
\textsf{childdoc} is a \LaTeXe{} package
that enables the direct compilation
of document sections included by |\include|
to individual files.
\end{abstract}

\begingroup
\parskip0ex
\tableofcontents
\endgroup

%%%%%%%%%%%%%%%%%%%%%%%%%%%%%%%%%%%%%%%%%%%%%%%%%%%%%%%%%%%%%%%%%%%%%%%%%%%%%%%%
%%%%%%%%%%%%%%%%%%%%%%%%%%%%%%%%%%%%%%%%%%%%%%%%%%%%%%%%%%%%%%%%%%%%%%%%%%%%%%%%
\section{Introduction}

\LaTeX{} provides a mechanism to structure a large document (such as a book)
into a main file and several child files (containing the chapters)
using the |\include| command.
This mechanism is beneficial for documents
which span hundreds of pages in order to
make the source file(s) more manageable.
Moreover, compilation can be restricted to
selected child files by means of the |\includeonly| command.
The latter feature can be used to reduce the compilation time while editing
(this was significantly more useful in the earlier days of \LaTeX{})
or to generate a smaller document which is easier to navigate.
Another application of |\includeonly| is to generate
documents consisting of selected parts of the complete document.

However, there are a few drawbacks of the plain |\include| mechanism:
\begin{itemize}
\item
The child files cannot be compiled on their own,
they can only be compiled via the main file.
A naive editing environment
(such as a text editor with an option
to have the current file processed by \LaTeX)
may require one to switch to the main file before compiling;
attempting to compile the child file produces errors.
\item
The main file must be modified (each time)
to adjust the |\includeonly| command
to the present needs. This easily leaves the main file in a messy state.
\item
The generated document will always carry the filename
of the main document. This is inconvenient if
several child files are to be compiled and
to be kept for distribution.
\end{itemize}

The present package provides a simple interface
to make child files individually compilable by \LaTeX{}.
Compiling a child file then has the same effect as compiling
the main file with an |\includeonly| command
to select the appropriate child.
Moreover the generated document will carry the name of the child
rather than the main file.
This resolves all three above issues.

This feature is meant to make the editing of books,
thesis documents and lecture notes somewhat more convenient.
However, the package can also be used efficiently for
composing a series of documents (such as exercise sheets)
which are typically distributed individually.
It then assists the author in generating the individual documents
(potentially in different versions)
as well as a document containing the collected series.
Another application is in developing style files
or other kinds of included material
where compilation of the style file could redirect
to a sample or test file.

%%%%%%%%%%%%%%%%%%%%%%%%%%%%%%%%%%%%%%%%%%%%%%%%%%%%%%%%%%%%%%%%%%%%%%%%%%%%%%%%
%%%%%%%%%%%%%%%%%%%%%%%%%%%%%%%%%%%%%%%%%%%%%%%%%%%%%%%%%%%%%%%%%%%%%%%%%%%%%%%%
\section{Usage}

First of all, the package \textsf{childdoc} is \emph{not} a standard
\LaTeXe{} |.sty| style file! Therefore it needs to be invoked in
a non-standard way.

%%%%%%%%%%%%%%%%%%%%%%%%%%%%%%%%%%%%%%%%%%%%%%%%%%%%%%%%%%%%%%%%%%%%%%%%%%%%%%%%
\subsection{Included Files}
\label{sec:include}

%%%%%%%%%%%%%%%%%%%%%%%%%%%%%%%%%%%%%%%%
\DescribeMacro{\childdocmain}
To use the package, add the commands
\begin{center}
\begin{tabular}{l}
|\input{childdoc.def}|\\
|\childdocmain{}|\\
\end{tabular}
\end{center}
at the very top of the main \LaTeX{} file,
in particular \emph{before} the |\documentclass| statement!
The argument of |\childdocmain| should be left empty
(but it must be present).

%%%%%%%%%%%%%%%%%%%%%%%%%%%%%%%%%%%%%%%%
\DescribeMacro{\childdocof}
Furthermore, add the commands
\begin{center}
\begin{tabular}{l}
|\input{childdoc.def}|\\
|\childdocof{|\textit{main}|}|\\
\end{tabular}
\end{center}
at the top of every child file \textit{child}
which is included by |\include{|\textit{child}|}|
from within the main file
(or at least for those files to be compiled individually).
The argument \textit{main} must be the filename of the main file.

There are a couple of
considerations in setting up the main and child documents:

%%%%%%%%%%%%%%%%%%%%%%%%%%%%%%%%%%%%%%%%
\paragraph{Restrictions.}

Please note the following restrictions:
\begin{itemize}
\item
|\childdocmain| must be called with one argument \textit{main}
to ensure compatibility with earlier version of the package.
It must either be empty (|\childdocmain{}|)
or precisely match the filename of the main file in which it is specified.
See \secref{sec:detection} for further information.
\item
The filename \textit{main} must be specified without the |.tex| extension.
\item
The filename \textit{main} is case sensitive
(even in case-insensitive file systems)
due to internal string comparison.
\item
The argument \textit{main} should be fully expanded, it cannot be a macro.
\item
Subdirectories and special characters should be avoided in filenames.
\item
The command |\childdocmain{|\textit{main}|}| must be followed by a whitespace.
It should not be followed immediately by another command
or by a comment mark `|%|'.
This is because the \TeX{} parser reads the token immediately following
the argument of |\childdocmain| and puts it
at the beginning of every child section;
however, a white\-space is ignored.
\end{itemize}

%%%%%%%%%%%%%%%%%%%%%%%%%%%%%%%%%%%%%%%%
\paragraph{Content of Main File.}

It is advisable to place all content in the child files included by |\include|.
Any output contained in the main file will appear in all child documents
unless suppressed manually;
it cannot be suppressed automatically by the |\includeonly| directive
and thus should normally be avoided.
A method to include some content in the main file
by means of conditional processing is described in \secref{sec:conditional}.

%%%%%%%%%%%%%%%%%%%%%%%%%%%%%%%%%%%%%%%%
\paragraph{Page Numbering.}

When only a part of the document is compiled,
the appropriate numbering of pages
(as well as other status parameters)
is determined from the |.aux| files.
The latter contain information from previous passes.
However this information needs to propagate through
all intermediate child documents.
Therefore the page numbering in child documents may well
be inconsistent until the complete document is compiled at least once.

A useful (if unconventional) way to always ensure a consistent
page numbering is to restart the numbering in each child document
and denote the pages by `\textit{child}|.|\textit{page}'
where \textit{child} represents the chapter/section number of the child file.
This can be achieved by the command
|\numberwithin{page}{|\textit{child}|}|
of the \textsf{amsmath} package
where \textit{child} can be |chapter| or |section|
depending on the chosen structuring.
Alternatively, one can modify the macro |\thepage| appropriately
and reset the counter |page| at the start of each child file.

%%%%%%%%%%%%%%%%%%%%%%%%%%%%%%%%%%%%%%%%%%%%%%%%%%%%%%%%%%%%%%%%%%%%%%%%%%%%%%%%
\subsection{Conditional Processing}
\label{sec:conditional}

The package provides a mechanism to compile different versions
of a document. To customise the versions further some conditional processing
can come in handy to distinguish which version is being compiled.
The package provides two macros to describe the compilation context:

%%%%%%%%%%%%%%%%%%%%%%%%%%%%%%%%%%%%%%%%
\DescribeMacro{\ifchilddoc}
The conditional |\ifchilddoc| distinguishes between the compilation of
child documents and the main document:
%
\begin{center}
|\ifchilddoc |\textit{child-code}| |[|\||else |\textit{main-code}]| \||fi|
\end{center}

%%%%%%%%%%%%%%%%%%%%%%%%%%%%%%%%%%%%%%%%
\DescribeMacro{\childdocname}
\DescribeMacro{\childdocjob}
The macro |\childdocname| contains the filename (without extension)
of the main or child file being processed.
Note that |\childdocjob| will always contain the name of the main file.

%%%%%%%%%%%%%%%%%%%%%%%%%%%%%%%%%%%%%%%%
\paragraph{Title Page.}

Conditional processing can be used to include a title or banner page
in the main document when proper precautions are taken.
Importantly, the code in the main file should ensure that the page counter
(as well as other status parameters which are stored in the |.aux| files)
takes the same value after the conditional processing.
Otherwise the page numbers may take divergent values
depending on which part is compiled.

For example, a title page could be declared by:
%
\begin{center}
\begin{tabular}{l}
|\ifchilddoc\||else|\\
|\addtocounter{page}{-1}|\\
\textit{code for title page}\\
|\newpage|\\
|\||fi|
\end{tabular}
\end{center}
%
A banner page for the child documents can be generated by:
%
\begin{center}
\begin{tabular}{l}
|\ifchilddoc|\\
|\addtocounter{page}{-1}|\\
\textit{code for banner page}\\
|\newpage|\\
|\||fi|
\end{tabular}
\end{center}
%
Here one could write a message such as:
\begin{center}
|This is the part \childdocname{} of \childdocjob{}.|
\end{center}

%%%%%%%%%%%%%%%%%%%%%%%%%%%%%%%%%%%%%%%%%%%%%%%%%%%%%%%%%%%%%%%%%%%%%%%%%%%%%%%%
\subsection{Flags}
\label{sec:flags}

The package makes it easy to generate different versions
of the main or child documents.
To this end compilation flags can be defined
and assigned different default values.
They will be particularly useful in conjunction
with the forwarding mechanism described in \secref{sec:forward}.

For example, it may be useful to have a flag |\version|
which can be set to |draft| or |final|.
The document source will contain some conditional code
depending on the value of |\version|.
Suppose further, the flag should default to |final| for the main file
and to |draft| for child files
which is a natural assignment for editing the document.
This is achieved by placing the following code
in the preamble of the main document
(below the |\childdocmain| directive):
%
\begin{center}
\begin{tabular}{l}
|\ifchilddoc|\\
|\providecommand{\version}{draft}|\\
|\||else|\\
|\providecommand{\version}{final}|\\
|\||fi|
\end{tabular}
\end{center}
%
The definition by |\providecommand| makes sure
that previous definitions are not overwritten.
Further statements |\providecommand{\version}{...}|
can thus be added before the above code to override it.

For the main file, one might add a line
(between |\childdocmain| and the above block)
%
\begin{center}
|%\ifchilddoc\||else\providecommand{\version}{draft}\||fi|
\end{center}
%
which can be uncommented to produce a draft version.
Likewise one can add a line to the very top of a child file
(above the |\childdocof{|\textit{main}|}| directive)
%
\begin{center}
|%\providecommand{\version}{final}|
\end{center}
%
which can be uncommented to produce the final version of this child document.

%%%%%%%%%%%%%%%%%%%%%%%%%%%%%%%%%%%%%%%%%%%%%%%%%%%%%%%%%%%%%%%%%%%%%%%%%%%%%%%%
\subsection{Forwarding}
\label{sec:forward}

Different versions of the main or child documents
using compilation flags as described in \secref{sec:flags}
can be (permanently) stored in different files
for convenient compilation, viewing and distribution.
To this end, the package defines a command
to pass on compilation to a different file:

%%%%%%%%%%%%%%%%%%%%%%%%%%%%%%%%%%%%%%%%
\DescribeMacro{\childdocforward}
The command |\childdocforward| redirects processing to
another source file:
%
\begin{center}
\begin{tabular}{l}
|\input{childdoc.def}|\\
|\childdocforward[|\textit{main}|]{|\textit{dest}|}|\\
\end{tabular}
\end{center}
%
The argument \textit{dest} is the destination file
(without extension).
It should be the main file or one of the child files.
Note that further \textsf{childdoc} directives
such as |\childdocof| and |\childdocforward|
in the indicated file will be processed in this form.
The optional argument \textit{main}
passes on directly to the main file \textit{main}
while pretending to compile the child \textit{dest}.
This form behaves as if \textit{dest}
issues |\childdocof{|\textit{main}|}| right away,
and no further \textsf{childdoc} directives will be processed.

%%%%%%%%%%%%%%%%%%%%%%%%%%%%%%%%%%%%%%%%
\DescribeMacro{\...prefix}
In the alternative form |\childdocforwardprefix|,
%
\begin{center}
\begin{tabular}{l}
|\input{childdoc.def}|\\
|\childdocforwardprefix[|\textit{main}|]{|\textit{prefix}|}{|\textit{dest}|}|
\end{tabular}
\end{center}
%
the destination file is determined by a pattern
depending on the current file:
To make this work, the current file must be called
`{\textit{prefix}\hspace{0.2em}\textit{suffix}}'
with \textit{prefix} matching precisely the argument.
Processing is then passed on to the file
`{\textit{dest}\hspace{0.2em}\textit{suffix}}'.
Surely, the same effect is achieved by
directly specifying the
argument `{\textit{dest}\hspace{0.2em}\textit{suffix}}'
in the first form.
However, that requires to set up a different file
for each child. With the alternative form of the command
all these files can have exactly the same content
which simplifies setting them up and maintaining them.

For example, the following file |draft.tex|
with a compilation flag |\version| as described in \secref{sec:flags}
compiles the main document as a draft:
%
\begin{center}
\begin{tabular}{l}
|\def\version{draft}|\\
|\input{childdoc.def}|\\
|\childdocforward{|\textit{main}|}|
\end{tabular}
\end{center}
%
Likewise, the following files |final|\textit{nn}|.tex|
compile the final version of the child document
|child|\textit{nn}|.tex|:
%
\begin{center}
\begin{tabular}{l}
|\def\version{final}|\\
|\input{childdoc.def}|\\
|\childdocforwardprefix{final}{child}|
\end{tabular}
\end{center}
%

Note that when several versions of a main file and/or of each child file
are to be generated, it may be convenient to set up a |Makefile| or
shell script to automatise the process.

%%%%%%%%%%%%%%%%%%%%%%%%%%%%%%%%%%%%%%%%%%%%%%%%%%%%%%%%%%%%%%%%%%%%%%%%%%%%%%%%
\subsection{Command Line Processing}
\label{sec:commandline}

The effect of redirection files can also be achieved by invoking
the \LaTeX{} compiler with a more elaborate command line.
Most conveniently this should be done as part
of a shell script or a |Makefile|.

When using \textsf{childdoc} in the main file, the following
command lines effectively perform a redirection
(note that depending on the shell being used,
backslashes may have to be doubled: `|\|' $\to$ `|\\|'):
%
\begin{center}
|... -jobname "|\textit{target}|" |\\|"|[\textit{flags}]%
|\input{childdoc.def}\childdocforward[|\textit{main}|]{|\textit{dest}|}"|
\end{center}
%
Here \textit{target} is the name of the output file,
\textit{main} is the name of the main file
and \textit{dest} is the name of the main or child file to be processed
(all filenames without extensions).
The optional argument \textit{main} can be omitted
if \textit{main} matches \textit{dest}.
Optionally, compilation \textit{flags} can be defined via |\def| commands.
This command line makes the \TeX{} engine believe
it is compiling the file \textit{target}
whose content is specified as the latter parameter.
The provided code then forwards the processing to
\textit{main} or \textit{dest} as described in \secref{sec:forward}.

%%%%%%%%%%%%%%%%%%%%%%%%%%%%%%%%%%%%%%%%%%%%%%%%%%%%%%%%%%%%%%%%%%%%%%%%%%%%%%%%
\subsection{Include by Input}
\label{sec:input}

Including child documents by |\include| has some restrictions by design.
Most notably, the content of a child document always occupies
its own set of pages; pages cannot be shared between child documents.
Usually, this behaviour makes perfect sense
because each child document contain an essential part of the document.
However, in some situations it may be desirable to compose
a document from a collection of parts
without having mandatory page breaks between then.
For this case, the package
provides a mechanism to include parts
by |\input| which can also be processed individually.
However, by construction this mechanism
requires manual handling of the content to be output.

%%%%%%%%%%%%%%%%%%%%%%%%%%%%%%%%%%%%%%%%
\DescribeMacro{\ifchilddocmanual}
The main file should be prepared as usual, see \secref{sec:include}.
However, the document body must make a distinction
between processing of an individual part and of the main document, e.g.:
%
\begin{center}
\begin{tabular}{l}
|\ifchilddocmanual|\\
|\input{\childdocname}|\\
|\||else|\\
\textit{document body with }|\input{|\textit{part}|}|\\
|\||fi|
\end{tabular}
\end{center}
%
The conditional |\ifchilddocmanual| is true whenever
a part to be included by |\input| is being compiled,
and the name of the part is stored in |\childdocname|.

%%%%%%%%%%%%%%%%%%%%%%%%%%%%%%%%%%%%%%%%
\DescribeMacro{\childdocby}
Each part to be included by |\input| should start with:
%
\begin{center}
\begin{tabular}{l}
|\input{childdoc.def}|\\
|\childdocby{|\textit{main}|}|\\
\end{tabular}
\end{center}
%
The directive |\childdocby| is similar to |\childdocof|
described in \secref{sec:include},
but the subsequent selection of content must be done manually.
To that end, both |\ifchilddoc| and |\ifchilddocmanual|
will be true upon processing of a part,
and the name of the part is stored in |\childdocname|.
Note that |\jobname| will be set to the filename of the current part
so that each part receives an individual |.aux| file
that does not interfere with the |.aux| file(s) of the main document.
This behaviour can be altered by the alternative form
|\childdocby[*]{|\textit{main}|}| (with a non-empty optional argument)
which uses the |.aux| file of the main document
by setting |\jobname| to \textit{main}.

%%%%%%%%%%%%%%%%%%%%%%%%%%%%%%%%%%%%%%%%%%%%%%%%%%%%%%%%%%%%%%%%%%%%%%%%%%%%%%%%
\subsection{Driver Development}
\label{sec:driver}

The \textsf{childdoc} mechanism can also be use for the development
of definition files such as \LaTeX{} styles or classes.
This case differs from the above setup with multiple parts
included by |\include| in that no |\includeonly| should be invoked.
This can be achieved by starting the include file
(before |\ProvidesPackage|) with:
%
\begin{center}
\begin{tabular}{l}
|\input{childdoc.def}|\\
|\childdocforward{|\textit{main}|}|\\
\end{tabular}
\end{center}
%
or alternatively with:
%
\begin{center}
\begin{tabular}{l}
|\input{childdoc.def}|\\
|\childdocby{|\textit{main}|}|\\
\end{tabular}
\end{center}
%
Both forms have slightly different effects as described above.
The main file is prepared as usual, see \secref{sec:include}.

%%%%%%%%%%%%%%%%%%%%%%%%%%%%%%%%%%%%%%%%%%%%%%%%%%%%%%%%%%%%%%%%%%%%%%%%%%%%%%%%
\subsection{Legacy Detection}
\label{sec:detection}

The directive |\childdocmain| in the main file can detect
whether the complete document or merely a child is to be compiled
even without using the directive |\childdocof|.
This method is deprecated because it is less robust
and there is no compelling reason to use it;
it is merely provided for backward compatibility
and it may be removed in future versions.

If the detection mechanism is to be used,
it is mandatory to correctly specify
the filename of the main file as the argument of |\childdocmain|:
%
\begin{center}
\begin{tabular}{l}
|\input{childdoc.def}|\\
|\childdocmain{|\textit{main}|}|\\
\end{tabular}
\end{center}
%
If |\jobname| does not match the argument \textit{main} of |\childdocmain|,
it is assumed that |\jobname| points to the child file to be compiled.
When using |\childdocmain| with the main file specified as argument,
it suffices to start a child file
with just |\input{|\textit{main}|}|
without loading of the package and using |\childdocof|.
If instead all processing is done
with the appropriate \textsf{childdoc} directives,
the argument of \textit{main} of |\childdocmain| can be empty.

An alternative version of the command line processing described
in \secref{sec:commandline} using the detection mechanism reads:
%
\begin{center}
|... -jobname "|\textit{target}|" "|[\textit{flags}]%
[|\def\jobname{|\textit{dest}|}|]|\input{|\textit{main}|}"|
\end{center}

%%%%%%%%%%%%%%%%%%%%%%%%%%%%%%%%%%%%%%%%%%%%%%%%%%%%%%%%%%%%%%%%%%%%%%%%%%%%%%%%
\subsection{Manual Code}
\label{sec:manual}

In case one cannot be certain whether the definitions file |childdoc.def|
is installed on the target \TeX{} distribution
and one prefers not to ship it,
it is conceivable to paste a few relevant commands into the sources.

To that end, drop all statements |\input{childdoc.def}|
and perform the replacements as outlined below.
Instead of |\childdocmain{|\textit{main}|}| add the following code
to the top of the main file:
%
\begin{center}
\begin{tabular}{l}
|\||ifdefined\childdocname\endinput\||fi\newif\ifchilddoc|\\
|\edef\childdocname{\scantokens\expandafter{\jobname\noexpand}}|\\
|\def\childdocmain{|\textit{main}|}\||ifx\childdocmain\childdocname\||else|\\
|\childdoctrue\includeonly{\childdocname}\let\jobname\childdocmain\||fi|\\
\end{tabular}
\end{center}
%
Instead of |\childdocof{|\textit{main}|}| just include the main file
at the top of each child file:
%
\begin{center}
|\input{|\textit{main}|}|
\end{center}
%
A simple redirection |\childdocforward{|\textit{dest}|}| is achieved by:
%
\begin{center}
|\def\jobname{|\textit{dest}|}\input{\jobname}|
\end{center}
%
The redirection with prefix
|\childdocforwardprefix[|\textit{prefix}|]{|\textit{dest}|}|
is accomplished by:
%
\begin{center}
\begin{tabular}{l}
|{\edef\jobname{\scantokens\expandafter{\jobname\noexpand}}|\\
|\def\redirectjob |\textit{prefix}|#1~~~{\gdef\jobname{|\textit{dest}|#1}}|\\
|\expandafter\redirectjob\jobname~~~}\input{\jobname}|
\end{tabular}
\end{center}

In an alternative approach,
child documents can be compiled by a specific command line
without additional code or specific definitions:
%
\begin{center}
|... -jobname "|\textit{target}|" "|[\textit{flags}]%
|\includeonly{|\textit{dest}|}\input{|\textit{main}|}"|
\end{center}
%

%%%%%%%%%%%%%%%%%%%%%%%%%%%%%%%%%%%%%%%%%%%%%%%%%%%%%%%%%%%%%%%%%%%%%%%%%%%%%%%%
%%%%%%%%%%%%%%%%%%%%%%%%%%%%%%%%%%%%%%%%%%%%%%%%%%%%%%%%%%%%%%%%%%%%%%%%%%%%%%%%
\section{Information}

%%%%%%%%%%%%%%%%%%%%%%%%%%%%%%%%%%%%%%%%%%%%%%%%%%%%%%%%%%%%%%%%%%%%%%%%%%%%%%%%
\subsection{Copyright}

Copyright \copyright{} 2017--2018 Niklas Beisert

This work may be distributed and/or modified under the
conditions of the \LaTeX{} Project Public License, either version 1.3
of this license or (at your option) any later version.
The latest version of this license is in
  \url{http://www.latex-project.org/lppl.txt}
and version 1.3 or later is part of all distributions of \LaTeX{}
version 2005/12/01 or later.

This work has the LPPL maintenance status `maintained'.

The Current Maintainer of this work is Niklas Beisert.

This work consists of the files |README.txt|, |childdoc.ins| and |childdoc.dtx|
as well as the derived files |childdoc.def|, |cdocsamp.tex|
with |cdocsch1.tex|, |cdocsch2.tex|, |cdocspt3.tex|, |cdocspt4.tex|,
|cdocsdrf.tex|, |cdocsfn1.tex|, |cdocsfn2.tex|
as well as |childdoc.pdf|.

%%%%%%%%%%%%%%%%%%%%%%%%%%%%%%%%%%%%%%%%%%%%%%%%%%%%%%%%%%%%%%%%%%%%%%%%%%%%%%%%
\subsection{Files and Installation}

The package consists of the files:
%
\begin{center}
\begin{tabular}{ll}
    |README.txt|   & readme file \\
    |childdoc.ins| & installation file \\
    |childdoc.dtx| & source file \\
    |childdoc.def| & definition file \\
    |cdocsamp.tex| & sample main file \\
    |cdocsch1.tex| & sample include file \\
    |cdocsch2.tex| & sample include file \\
    |cdocspt3.tex| & sample part file \\
    |cdocspt4.tex| & sample part file \\
    |cdocsdrf.tex| & sample redirection file \\
    |cdocsfn1.tex| & sample redirection file \\
    |cdocsfn2.tex| & sample redirection file \\
    |childdoc.pdf| & manual
\end{tabular}
\end{center}
%
The distribution consists of the files
|README.txt|, |childdoc.ins| and |childdoc.dtx|.
%
\begin{itemize}
\item
Run (pdf)\LaTeX{} on |childdoc.dtx|
to compile the manual |childdoc.pdf| (this file).
\item
Run \LaTeX{} on |childdoc.ins| to create the definitions file |childdoc.def|
and the sample |cdocsamp.tex| with include files
|cdocsch1.tex|, |cdocsch2.tex|, |cdocspt3.tex|, |cdocspt4.tex|,
|cdocsdrf.tex|, |cdocsfn1.tex|, |cdocsfn2.tex|.
Then copy the file |childdoc.def| to an appropriate directory of your \LaTeX{}
distribution, e.g.\ \textit{texmf-root}|/tex/latex/childdoc|.
\end{itemize}

%%%%%%%%%%%%%%%%%%%%%%%%%%%%%%%%%%%%%%%%%%%%%%%%%%%%%%%%%%%%%%%%%%%%%%%%%%%%%%%%
\subsection{Related CTAN Packages}

There are several other packages which offer a similar functionality:
%
\begin{itemize}
\item
The packages
\href{http://ctan.org/pkg/docmute}{\textsf{docmute}},
\href{http://ctan.org/pkg/includex}{\textsf{includex}} and
\href{http://ctan.org/pkg/standalone}{\textsf{standalone}}
provide commands to include only the document body of
a child file thus allowing both files to be compiled individually.
\item
The packages \href{http://ctan.org/pkg/subdocs}{\textsf{subdocs}}
and \href{http://ctan.org/pkg/subfiles}{\textsf{subfiles}}
provide structures in which the main and child documents can be
encapsulated and allowing them to be compiled individually.
The inclusion mechanism is different from the conventional |\include|.
\item
The package \href{http://ctan.org/pkg/combine}{\textsf{combine}}
is an elaborate solution to combine several documents into one.
\end{itemize}
%
See also the CTAN topic \href{http://ctan.org/topic/subdocs}{\textsf{subdocs}}
for further related packages.
The present package differs from the above solutions in that
a document structure constructed with the conventional |\include| mechanism
just needs two extra commands at the top of every file
such that all constituent files can be compiled individually.

%%%%%%%%%%%%%%%%%%%%%%%%%%%%%%%%%%%%%%%%%%%%%%%%%%%%%%%%%%%%%%%%%%%%%%%%%%%%%%%%
%\subsection{Feature Suggestions}
%
%The following is a list of features which may be useful for future
%versions of this package:
%%
%\begin{itemize}
%\item
%\ldots
%\end{itemize}

%%%%%%%%%%%%%%%%%%%%%%%%%%%%%%%%%%%%%%%%%%%%%%%%%%%%%%%%%%%%%%%%%%%%%%%%%%%%%%%%
\subsection{Revision History}

%%%%%%%%%%%%%%%%%%%%%%%%%%%%%%%%%%%%%%%%
\paragraph{v2.0:} 2018/12/30

\begin{itemize}
\item
immediate forward processing
\item
added |\childdocby| mechanism
\item
manual restructured
\end{itemize}

%%%%%%%%%%%%%%%%%%%%%%%%%%%%%%%%%%%%%%%%
\paragraph{v1.6:} 2018/01/17

\begin{itemize}
\item
application for development of include files
\item
corrections to manual
\end{itemize}

%%%%%%%%%%%%%%%%%%%%%%%%%%%%%%%%%%%%%%%%
\paragraph{v1.5:} 2017/05/21

\begin{itemize}
\item
more complete structuring introduced
\item
|\childdocof| introduced
\item
|\childdoc| renamed to |\childdocmain|
\item
|\childredirect| renamed to |\childdocforward| and |\childdocforwardprefix|
and functionality expanded
\end{itemize}

%%%%%%%%%%%%%%%%%%%%%%%%%%%%%%%%%%%%%%%%
\paragraph{v1.0:} 2017/04/27

\begin{itemize}
\item
manual and install package
\item
first version published on CTAN
\end{itemize}

%%%%%%%%%%%%%%%%%%%%%%%%%%%%%%%%%%%%%%%%
\paragraph{v0.6:} 2017/04/26

\begin{itemize}
\item
redirection mechanism added
\end{itemize}

%%%%%%%%%%%%%%%%%%%%%%%%%%%%%%%%%%%%%%%%
\paragraph{v0.5:} 2017/04/26

\begin{itemize}
\item
functionality in definition file
\end{itemize}


%%%%%%%%%%%%%%%%%%%%%%%%%%%%%%%%%%%%%%%%%%%%%%%%%%%%%%%%%%%%%%%%%%%%%%%%%%%%%%%%
%%%%%%%%%%%%%%%%%%%%%%%%%%%%%%%%%%%%%%%%%%%%%%%%%%%%%%%%%%%%%%%%%%%%%%%%%%%%%%%%
%%%%%%%%%%%%%%%%%%%%%%%%%%%%%%%%%%%%%%%%%%%%%%%%%%%%%%%%%%%%%%%%%%%%%%%%%%%%%%%%
\appendix

\settowidth\MacroIndent{\rmfamily\scriptsize 000\ }

 \DocInput{childdoc.dtx}

\end{document}
%</driver>
% \fi
%
% %%%%%%%%%%%%%%%%%%%%%%%%%%%%%%%%%%%%%%%%%%%%%%%%%%%%%%%%%%%%%%%%%%%%%%%%%%%%%%
% %%%%%%%%%%%%%%%%%%%%%%%%%%%%%%%%%%%%%%%%%%%%%%%%%%%%%%%%%%%%%%%%%%%%%%%%%%%%%%
% \section{Sample}
%\iffalse
%<*samplemain>
%\fi
%
% The following presents a sample document
% with two chapters, two parts, a title page,
% a compile flag as well as three forwarding files to set the flag.
% It consists of eight |.tex| files:
% \begin{center}
% \begin{tabular}{ll}
% |cdocsamp.tex|&main file\\
% |cdocsch1.tex|&include file for chapter 1\\
% |cdocsch2.tex|&include file for chapter 2\\
% |cdocspt3.tex|&include file for part 3\\
% |cdocspt4.tex|&include file for part 4\\
% |cdocsdrf.tex|&forwarding file for main file in draft mode\\
% |cdocsfi1.tex|&forwarding file for final version of chapter 1\\
% |cdocsfi2.tex|&forwarding file for final version of chapter 2\\
% \end{tabular}
% \end{center}
% Each of the eight files can be compiled directly by the \LaTeX{} compiler.
%
% %%%%%%%%%%%%%%%%%%%%%%%%%%%%%%%%%%%%%%
% \paragraph{Main File.}
%
% The main file is called |cdocsamp.tex|.
%
% Load the \textsf{childdoc} definitions and
% declare the filename for the main document:
%    \begin{macrocode}
\input{childdoc.def}
\childdocmain{}
%    \end{macrocode}

% Optional override for |\version| flag:
%    \begin{macrocode}
%%\ifchilddoc\else\providecommand{\version}{draft}\fi
%    \end{macrocode}

% Define the default values for the |\version| flag
% (|final| for the main file and |draft| for childs):
%    \begin{macrocode}
\ifchilddoc
\providecommand{\version}{draft}
\else
\providecommand{\version}{final}
\fi
%    \end{macrocode}

% Load the standard document class:
%    \begin{macrocode}
\documentclass[12pt]{article}
%    \end{macrocode}

% Start the document body:
%    \begin{macrocode}
\begin{document}
%    \end{macrocode}

% Declare a title page.
% Print title, part of document being processed and version flag:
%    \begin{macrocode}
\addtocounter{page}{-1}
\begin{center}
{\LARGE\bfseries{}childdoc example\par}
\vspace{1cm}
\ifchilddoc
\ifchilddocmanual part\else chapter\fi:
`\childdocname' of `\childdocjob'\par
\else
main document: `\childdocjob'\par
\fi
version: \version\par
\end{center}
\newpage
%    \end{macrocode}

% Manually include selected file,
% otherwise process as usual:
%    \begin{macrocode}
\ifchilddocmanual
\section*{part `\childdocname'}
\input{\childdocname}
\else
%    \end{macrocode}

% Include the two chapters:
%    \begin{macrocode}
\include{cdocsch1}
\include{cdocsch2}
%    \end{macrocode}

% Include the two parts unless only chapters should be displayed:
%    \begin{macrocode}
\ifchilddoc\else
\section{part three}
\input{cdocspt3}
\section{part four}
\input{cdocspt4}
\fi
%    \end{macrocode}

% Process as usual until here:
%    \begin{macrocode}
\fi
%    \end{macrocode}

% End of document body:
%    \begin{macrocode}
\end{document}
%    \end{macrocode}
%\iffalse
%</samplemain>
%\fi
%
% %%%%%%%%%%%%%%%%%%%%%%%%%%%%%%%%%%%%%%
% \paragraph{Chapter Include Files.}
%
% The include files are called |cdocsch1.tex| and |cdocsch2.tex|.
%
%\iffalse
%<*samplechap1|samplechap2>
%\fi

% Optional override for |\version| flag:
%    \begin{macrocode}
%%\providecommand{\version}{final}
%    \end{macrocode}

% Include the main document:
%    \begin{macrocode}
\input{childdoc.def}
\childdocof{cdocsamp}
%    \end{macrocode}

%\iffalse
%</samplechap1|samplechap2>
%\fi
%
%\iffalse
%<*samplechap1>
%\fi
% Some text for chapter 1:
%    \begin{macrocode}
\section{one}
some text in chapter one
%    \end{macrocode}

%\iffalse
%</samplechap1>
%\fi
% Some text for chapter 2:
%\iffalse
%<*samplechap2>
%\fi
%    \begin{macrocode}
\section{two}
more text in chapter two
%    \end{macrocode}

%\iffalse
%</samplechap2>
%\fi
%
% %%%%%%%%%%%%%%%%%%%%%%%%%%%%%%%%%%%%%%
% \paragraph{Part Include Files.}
%
% The include files are called |cdocspt3.tex| and |cdocspt4.tex|.
%
%\iffalse
%<*samplepart3|samplepart4>
%\fi

% Optional override for |\version| flag:
%    \begin{macrocode}
%%\providecommand{\version}{final}
%    \end{macrocode}

% Include the main document:
%    \begin{macrocode}
\input{childdoc.def}
\childdocby{cdocsamp}
%    \end{macrocode}

%\iffalse
%</samplepart3|samplepart4>
%\fi
%
%\iffalse
%<*samplepart3>
%\fi
% Some text for part 3:
%    \begin{macrocode}
some text in part three
%    \end{macrocode}

%\iffalse
%</samplepart3>
%\fi
% Some text for part 4:
%\iffalse
%<*samplepart4>
%\fi
%    \begin{macrocode}
more text in part four
%    \end{macrocode}

%\iffalse
%</samplepart4>
%\fi
%
% %%%%%%%%%%%%%%%%%%%%%%%%%%%%%%%%%%%%%%
% \paragraph{Forwarding for a Complete Draft.}
%
% The following forwarding file |cdocsdrf.tex|
% compiles the main document in draft mode:
%\iffalse
%<*sampledraft>
%\fi
%    \begin{macrocode}
\def\version{draft}
\input{childdoc.def}
\childdocforward{cdocsamp}
%    \end{macrocode}

%\iffalse
%</sampledraft>
%\fi
%
% %%%%%%%%%%%%%%%%%%%%%%%%%%%%%%%%%%%%%%
% \paragraph{Forwarding for Final Version of the Chapters.}
%
% The following forwarding files |cdocsfn1.tex| and |cdocsfn2.tex|
% (with identical content)
% compile the final versions of the child documents
% |cdocsch1.tex| and |cdocsch2.tex|, respectively:
%\iffalse
%<*samplefinal>
%\fi
%    \begin{macrocode}
\def\version{final}
\input{childdoc.def}
\childdocforwardprefix[cdocsamp]{cdocsfn}{cdocsch}
%    \end{macrocode}

%\iffalse
%</samplefinal>
%\fi
%
% %%%%%%%%%%%%%%%%%%%%%%%%%%%%%%%%%%%%%%
% \paragraph{Command Line Processing.}
%
% The following three command lines generate the output files
% |cdocscld|, |cdocscl1| and |cdocscl2|
% which should be identical to
% |cdocsdrf|, |cdocsch1| and |cdocsfn2|, respectively:
% \begin{center}
% \begin{tabular}{l}
% |latex -jobname cdocscld \|\\
% |  "\def\version{draft}\input{childdoc.def}\childdocforward{cdocsamp}"|\\
% |latex -jobname cdocscl1 \|\\
% |  "\input{childdoc.def}\childdocforward[cdocsamp]{cdocsch1}"|\\
% |latex -jobname cdocscl2 \|\\
% |  "\def\version{final}\input{childdoc.def}\childdocforward{cdocsch2}"|
% \end{tabular}
% \end{center}
% Note that the trailing backslash on each first line
% merely continues the input to the second line
% (for convenient cut ant paste).
% Furthermore, the command |latex| can be replaced by any
% of its alternative versions such as |pdflatex|.
%
% %%%%%%%%%%%%%%%%%%%%%%%%%%%%%%%%%%%%%%%%%%%%%%%%%%%%%%%%%%%%%%%%%%%%%%%%%%%%%%
% %%%%%%%%%%%%%%%%%%%%%%%%%%%%%%%%%%%%%%%%%%%%%%%%%%%%%%%%%%%%%%%%%%%%%%%%%%%%%%
% \section{Implementation}
%\iffalse
%<*package>
%\fi
%
% This section describes the definitions file |childdoc.def|.

% The definitions cannot be loaded using |\usepackage| or |\RequirePackage|
% which has a mechanism to prevent loading a style file more than once.
% When loading the definitions by means of |\input|
% multiple instances have to be prevented manually:
%\iffalse
%This code needs to be before the `\ProvidesFile' directive
%which is defined at the beginning of this file.
%Therefore it is also placed there and commented out here.
%</package>
%<*discard>
%\fi
%    \begin{macrocode}
\ifdefined\childdocmain\endinput\fi
%    \end{macrocode}
%\iffalse
%</discard>
%<*package>
%\fi
%
% \macro{\ifchilddoc}
% \macro{\ifchilddocmanual}
% The conditional |\ifchilddoc| tells whether a
% child (true) or main (false) document is being compiled.
% The conditional |\ifchilddocmanual| tells whether
% the |\includeonly| mechanism is used (false) or
% the selection of child files must be performed manually (true).
% The definitions initialise to false:
%    \begin{macrocode}
\newif\ifchilddoc
\newif\ifchilddocmanual
%    \end{macrocode}

% \macro{\childdocname}
% \macro{\childdocjob}
% The macro |\childdocname| stores the name of the main document
% to be compiled. The macro |\childdocjob| stores the name of
% the document on which the \LaTeX{} compiler was originally invoked.
% The content of |\jobname| cannot be compared
% to filenames specified in the source due to different catcodes.
% The following code rescans |\jobname|, stores the result
% in |\childdocname| and saves a copy in |\childdocjob|:
%    \begin{macrocode}
\edef\childdocname{\scantokens\expandafter{\jobname\noexpand}}
\let\childdocjob\childdocname
%    \end{macrocode}

% \macro{\childdocdisable}
% The macro |\childdocdisable| prevents the main file
% from being processed more than once.
% At this stage, the main document command |\childdocmain|
% is assumed to be called once again where it should do nothing.
% Any subsequent call to it should prevent
% a secondary processing of the main document
% It overwrites the forwarding commands
% |\childdocof| and |\childdocforward|
% with empty macros to prevent further inclusions of the main document:
%    \begin{macrocode}
\newcommand{\childdocdisable}
{
  \renewcommand{\childdocmain}[1]{\renewcommand{\childdocmain}[1]{\endinput}}
  \renewcommand{\childdocof}[1]{}
  \renewcommand{\childdocby}[2][]{}
  \renewcommand{\childdocforward}[2][]{}
  \renewcommand{\childdocdisable}{}
}
%    \end{macrocode}

% \macro{\childdocmain}
% The macro |\childdocmain| is to be called at the top of the main file
% with nothing or the main filename (without extension) as argument.
% First, it breaks loops.
% If the argument is not empty and does not match |\childdocname|
% (which is set by the first inclusion of |childdoc.def|),
% |\ifchilddoc| is set to true, |\includeonly| is applied to the child file
% and |\jobname| is set to the main file
% (for proper handling of |.aux| files):
%    \begin{macrocode}
\newcommand{\childdocmain}[1]
{
  \childdocdisable\childdocmain{}
  \if?#1?\else
    \begingroup
      \def\childdoctmp{#1}
      \ifx\childdoctmp\childdocname
        \def\childdoctmp{}
      \else
        \def\childdoctmp
        {
          \childdoctrue
          \includeonly{\childdocname}
          \def\childdocjob{#1}
          \def\jobname{#1}
        }
      \fi
      \expandafter
    \endgroup
    \childdoctmp
  \fi
}
%    \end{macrocode}

% \macro{\childdocof}
% The command |\childdocof| redirects
% compilation to the main file |#1|.
%    \begin{macrocode}
\newcommand{\childdocof}[1]
{
  \childdocdisable
  \childdoctrue
  \includeonly{\childdocname}
  \def\jobname{#1}
  \def\childdocjob{#1}
  \input{#1}
}
%    \end{macrocode}

% \macro{\childdocby}
% The command |\childdocby| ....
%    \begin{macrocode}
\newcommand{\childdocby}[2][]
{
  \childdocdisable
  \childdoctrue
  \childdocmanualtrue
  \if?#1?\else
    \def\jobname{#2}
  \fi
  \def\childdocjob{#2}
  \input{#2}
  \endinput
}
%    \end{macrocode}

% \macro{\childdocforward}
% The command |\childdocforward| redirects
% compilation to the main file or
% (if the optional argument is given) a child file.
% Parameters are set as if the main file
% or a child file starting with |\childdocof| was compiled.
% Then compilation is handed over to the main file:
%    \begin{macrocode}
\newcommand{\childdocforward}[2][]
{
  \begingroup
    \if?#1?
      \def\childdoctmp
      {
        \def\childdocname{#2}
        \def\childdocjob{#2}
        \def\jobname{#2}
        \input{#2}
        \endinput
      }
    \else
      \def\childdoctmp
      {
        \childdocdisable
        \def\childdocname{#2}
        \childdoctrue
        \includeonly{#2}
        \def\childdocjob{#1}
        \def\jobname{#1}
        \input{#1}
        \endinput
      }
    \fi
    \expandafter
  \endgroup
  \childdoctmp
}
%    \end{macrocode}

% \macro{\childdocforwardprefix}
% The command |\childdocforwardprefix| redirects
% compilation to the main or a child file by means of a pattern.
% The prefix |#1| in the current filename is replaced by |#2|
% and the suffix of the current filename is kept
% (it is assumed that the filename does not contain the substring `|~~~|'
% which is used as a delimiter).
% Compilation is handed over to the new file by |\childdocforward|:
%    \begin{macrocode}
\newcommand{\childdocforwardprefix}[3][]
{
  \begingroup
    \def\childdocextract #2##1~~~{\def\childdoctmp{\childdocforward[#1]{#3##1}}}
    \expandafter\childdocextract\childdocname~~~
    \expandafter
  \endgroup
  \childdoctmp
}
%    \end{macrocode}

% \macro{\childdoc}
% The deprecated macro |\childdoc| is a legacy version of |\childdocmain|:
%    \begin{macrocode}
\newcommand{\childdoc}{\childdocmain}
%    \end{macrocode}

% \macro{\childdocredirect}
% The deprecated macro |\childdocredirect| is a legacy version
% of |\childdocforward| and |\childdocforwardprefix|:
%    \begin{macrocode}
\newcommand{\childdocredirect}[2][]
{
  \begingroup
    \if?#1?
      \def\childdoctmp{\childdocforward{#2}}
    \else
      \def\childdoctmp{\childdocforwardprefix{#1}{#2}}
    \fi
    \expandafter
  \endgroup
  \childdoctmp
}
%    \end{macrocode}

%\iffalse
%</package>
%\fi
%
\endinput
\childdocforward{cdocsch2}"|
% \end{tabular}
% \end{center}
% Note that the trailing backslash on each first line
% merely continues the input to the second line
% (for convenient cut ant paste).
% Furthermore, the command |latex| can be replaced by any
% of its alternative versions such as |pdflatex|.
%
% %%%%%%%%%%%%%%%%%%%%%%%%%%%%%%%%%%%%%%%%%%%%%%%%%%%%%%%%%%%%%%%%%%%%%%%%%%%%%%
% %%%%%%%%%%%%%%%%%%%%%%%%%%%%%%%%%%%%%%%%%%%%%%%%%%%%%%%%%%%%%%%%%%%%%%%%%%%%%%
% \section{Implementation}
%\iffalse
%<*package>
%\fi
%
% This section describes the definitions file |childdoc.def|.

% The definitions cannot be loaded using |\usepackage| or |\RequirePackage|
% which has a mechanism to prevent loading a style file more than once.
% When loading the definitions by means of |\input|
% multiple instances have to be prevented manually:
%\iffalse
%This code needs to be before the `\ProvidesFile' directive
%which is defined at the beginning of this file.
%Therefore it is also placed there and commented out here.
%</package>
%<*discard>
%\fi
%    \begin{macrocode}
\ifdefined\childdocmain\endinput\fi
%    \end{macrocode}
%\iffalse
%</discard>
%<*package>
%\fi
%
% \macro{\ifchilddoc}
% \macro{\ifchilddocmanual}
% The conditional |\ifchilddoc| tells whether a
% child (true) or main (false) document is being compiled.
% The conditional |\ifchilddocmanual| tells whether
% the |\includeonly| mechanism is used (false) or
% the selection of child files must be performed manually (true).
% The definitions initialise to false:
%    \begin{macrocode}
\newif\ifchilddoc
\newif\ifchilddocmanual
%    \end{macrocode}

% \macro{\childdocname}
% \macro{\childdocjob}
% The macro |\childdocname| stores the name of the main document
% to be compiled. The macro |\childdocjob| stores the name of
% the document on which the \LaTeX{} compiler was originally invoked.
% The content of |\jobname| cannot be compared
% to filenames specified in the source due to different catcodes.
% The following code rescans |\jobname|, stores the result
% in |\childdocname| and saves a copy in |\childdocjob|:
%    \begin{macrocode}
\edef\childdocname{\scantokens\expandafter{\jobname\noexpand}}
\let\childdocjob\childdocname
%    \end{macrocode}

% \macro{\childdocdisable}
% The macro |\childdocdisable| prevents the main file
% from being processed more than once.
% At this stage, the main document command |\childdocmain|
% is assumed to be called once again where it should do nothing.
% Any subsequent call to it should prevent
% a secondary processing of the main document
% It overwrites the forwarding commands
% |\childdocof| and |\childdocforward|
% with empty macros to prevent further inclusions of the main document:
%    \begin{macrocode}
\newcommand{\childdocdisable}
{
  \renewcommand{\childdocmain}[1]{\renewcommand{\childdocmain}[1]{\endinput}}
  \renewcommand{\childdocof}[1]{}
  \renewcommand{\childdocby}[2][]{}
  \renewcommand{\childdocforward}[2][]{}
  \renewcommand{\childdocdisable}{}
}
%    \end{macrocode}

% \macro{\childdocmain}
% The macro |\childdocmain| is to be called at the top of the main file
% with nothing or the main filename (without extension) as argument.
% First, it breaks loops.
% If the argument is not empty and does not match |\childdocname|
% (which is set by the first inclusion of |childdoc.def|),
% |\ifchilddoc| is set to true, |\includeonly| is applied to the child file
% and |\jobname| is set to the main file
% (for proper handling of |.aux| files):
%    \begin{macrocode}
\newcommand{\childdocmain}[1]
{
  \childdocdisable\childdocmain{}
  \if?#1?\else
    \begingroup
      \def\childdoctmp{#1}
      \ifx\childdoctmp\childdocname
        \def\childdoctmp{}
      \else
        \def\childdoctmp
        {
          \childdoctrue
          \includeonly{\childdocname}
          \def\childdocjob{#1}
          \def\jobname{#1}
        }
      \fi
      \expandafter
    \endgroup
    \childdoctmp
  \fi
}
%    \end{macrocode}

% \macro{\childdocof}
% The command |\childdocof| redirects
% compilation to the main file |#1|.
%    \begin{macrocode}
\newcommand{\childdocof}[1]
{
  \childdocdisable
  \childdoctrue
  \includeonly{\childdocname}
  \def\jobname{#1}
  \def\childdocjob{#1}
  \input{#1}
}
%    \end{macrocode}

% \macro{\childdocby}
% The command |\childdocby| ....
%    \begin{macrocode}
\newcommand{\childdocby}[2][]
{
  \childdocdisable
  \childdoctrue
  \childdocmanualtrue
  \if?#1?\else
    \def\jobname{#2}
  \fi
  \def\childdocjob{#2}
  \input{#2}
  \endinput
}
%    \end{macrocode}

% \macro{\childdocforward}
% The command |\childdocforward| redirects
% compilation to the main file or
% (if the optional argument is given) a child file.
% Parameters are set as if the main file
% or a child file starting with |\childdocof| was compiled.
% Then compilation is handed over to the main file:
%    \begin{macrocode}
\newcommand{\childdocforward}[2][]
{
  \begingroup
    \if?#1?
      \def\childdoctmp
      {
        \def\childdocname{#2}
        \def\childdocjob{#2}
        \def\jobname{#2}
        \input{#2}
        \endinput
      }
    \else
      \def\childdoctmp
      {
        \childdocdisable
        \def\childdocname{#2}
        \childdoctrue
        \includeonly{#2}
        \def\childdocjob{#1}
        \def\jobname{#1}
        \input{#1}
        \endinput
      }
    \fi
    \expandafter
  \endgroup
  \childdoctmp
}
%    \end{macrocode}

% \macro{\childdocforwardprefix}
% The command |\childdocforwardprefix| redirects
% compilation to the main or a child file by means of a pattern.
% The prefix |#1| in the current filename is replaced by |#2|
% and the suffix of the current filename is kept
% (it is assumed that the filename does not contain the substring `|~~~|'
% which is used as a delimiter).
% Compilation is handed over to the new file by |\childdocforward|:
%    \begin{macrocode}
\newcommand{\childdocforwardprefix}[3][]
{
  \begingroup
    \def\childdocextract #2##1~~~{\def\childdoctmp{\childdocforward[#1]{#3##1}}}
    \expandafter\childdocextract\childdocname~~~
    \expandafter
  \endgroup
  \childdoctmp
}
%    \end{macrocode}

% \macro{\childdoc}
% The deprecated macro |\childdoc| is a legacy version of |\childdocmain|:
%    \begin{macrocode}
\newcommand{\childdoc}{\childdocmain}
%    \end{macrocode}

% \macro{\childdocredirect}
% The deprecated macro |\childdocredirect| is a legacy version
% of |\childdocforward| and |\childdocforwardprefix|:
%    \begin{macrocode}
\newcommand{\childdocredirect}[2][]
{
  \begingroup
    \if?#1?
      \def\childdoctmp{\childdocforward{#2}}
    \else
      \def\childdoctmp{\childdocforwardprefix{#1}{#2}}
    \fi
    \expandafter
  \endgroup
  \childdoctmp
}
%    \end{macrocode}

%\iffalse
%</package>
%\fi
%
\endinput
|\\
|\childdocby{|\textit{main}|}|\\
\end{tabular}
\end{center}
%
Both forms have slightly different effects as described above.
The main file is prepared as usual, see \secref{sec:include}.

%%%%%%%%%%%%%%%%%%%%%%%%%%%%%%%%%%%%%%%%%%%%%%%%%%%%%%%%%%%%%%%%%%%%%%%%%%%%%%%%
\subsection{Legacy Detection}
\label{sec:detection}

The directive |\childdocmain| in the main file can detect
whether the complete document or merely a child is to be compiled
even without using the directive |\childdocof|.
This method is deprecated because it is less robust
and there is no compelling reason to use it;
it is merely provided for backward compatibility
and it may be removed in future versions.

If the detection mechanism is to be used,
it is mandatory to correctly specify
the filename of the main file as the argument of |\childdocmain|:
%
\begin{center}
\begin{tabular}{l}
|% \iffalse
%
% childdoc.dtx Copyright (C) 2017-2018 Niklas Beisert
%
% This work may be distributed and/or modified under the
% conditions of the LaTeX Project Public License, either version 1.3
% of this license or (at your option) any later version.
% The latest version of this license is in
%   http://www.latex-project.org/lppl.txt
% and version 1.3 or later is part of all distributions of LaTeX
% version 2005/12/01 or later.
%
% This work has the LPPL maintenance status `maintained'.
%
% The Current Maintainer of this work is Niklas Beisert.
%
% This work consists of the files childdoc.dtx and childdoc.ins
% and the derived files childdoc.def and cdocsamp.tex with
% cdocsch1.tex, cdocsch2.tex, cdocsdrf.tex, cdocsfn1.tex, cdocsfn2.tex.
%
%<package>\ifdefined\childdocmain\endinput\fi
%<package>\ProvidesFile{childdoc.def}[2018/12/30 v2.0 child document driver]
%<samplemain>\ProvidesFile{cdocsamp.tex}[2018/12/30 v2.0 sample for childdoc]
%<*driver>
%\ProvidesFile{childdoc.drv}[2018/12/30 v2.0 childdoc reference manual file]
\PassOptionsToClass{10pt,a4paper}{article}
\documentclass{ltxdoc}

\usepackage[margin=35mm]{geometry}
\usepackage{hyperref}
\usepackage{hyperxmp}
\usepackage[usenames]{color}

\hypersetup{colorlinks=true}
\hypersetup{pdfstartview=FitH}
\hypersetup{pdfpagemode=UseNone}
\hypersetup{pdfsource={}}
\hypersetup{pdflang={en-UK}}
\hypersetup{pdfcopyright={Copyright 2017-2018 Niklas Beisert.
  This work may be distributed and/or modified under the
  conditions of the LaTeX Project Public License, either version 1.3
  of this license or (at your option) any later version.}}
\hypersetup{pdflicenseurl={http://www.latex-project.org/lppl.txt}}
\hypersetup{pdfcontactaddress={ETH Zurich, ITP, HIT K,
  Wolfgang-Pauli-Strasse 27}}
\hypersetup{pdfcontactpostcode={8093}}
\hypersetup{pdfcontactcity={Zurich}}
\hypersetup{pdfcontactcountry={Switzerland}}
\hypersetup{pdfcontactemail={nbeisert@itp.phys.ethz.ch}}
\hypersetup{pdfcontacturl={http://people.phys.ethz.ch/\xmptilde nbeisert/}}

\newcommand{\secref}[1]{\hyperref[#1]{section \ref*{#1}}}

\parskip1ex
\parindent0pt
\let\olditemize\itemize
\def\itemize{\olditemize\parskip0pt}

\begin{document}

\title{The \textsf{childdoc} Package}
\hypersetup{pdftitle={The childdoc Package}}
\author{Niklas Beisert\\[2ex]
  Institut f\"ur Theoretische Physik\\
  Eidgen\"ossische Technische Hochschule Z\"urich\\
  Wolfgang-Pauli-Strasse 27, 8093 Z\"urich, Switzerland\\[1ex]
  \href{mailto:nbeisert@itp.phys.ethz.ch}
  {\texttt{nbeisert@itp.phys.ethz.ch}}}
\hypersetup{pdfauthor={Niklas Beisert}}
\hypersetup{pdfsubject={Manual for the LaTeX2e Package childdoc}}
\date{30 December 2018, \textsf{v2.0}}
\maketitle

\begin{abstract}\noindent
\textsf{childdoc} is a \LaTeXe{} package
that enables the direct compilation
of document sections included by |\include|
to individual files.
\end{abstract}

\begingroup
\parskip0ex
\tableofcontents
\endgroup

%%%%%%%%%%%%%%%%%%%%%%%%%%%%%%%%%%%%%%%%%%%%%%%%%%%%%%%%%%%%%%%%%%%%%%%%%%%%%%%%
%%%%%%%%%%%%%%%%%%%%%%%%%%%%%%%%%%%%%%%%%%%%%%%%%%%%%%%%%%%%%%%%%%%%%%%%%%%%%%%%
\section{Introduction}

\LaTeX{} provides a mechanism to structure a large document (such as a book)
into a main file and several child files (containing the chapters)
using the |\include| command.
This mechanism is beneficial for documents
which span hundreds of pages in order to
make the source file(s) more manageable.
Moreover, compilation can be restricted to
selected child files by means of the |\includeonly| command.
The latter feature can be used to reduce the compilation time while editing
(this was significantly more useful in the earlier days of \LaTeX{})
or to generate a smaller document which is easier to navigate.
Another application of |\includeonly| is to generate
documents consisting of selected parts of the complete document.

However, there are a few drawbacks of the plain |\include| mechanism:
\begin{itemize}
\item
The child files cannot be compiled on their own,
they can only be compiled via the main file.
A naive editing environment
(such as a text editor with an option
to have the current file processed by \LaTeX)
may require one to switch to the main file before compiling;
attempting to compile the child file produces errors.
\item
The main file must be modified (each time)
to adjust the |\includeonly| command
to the present needs. This easily leaves the main file in a messy state.
\item
The generated document will always carry the filename
of the main document. This is inconvenient if
several child files are to be compiled and
to be kept for distribution.
\end{itemize}

The present package provides a simple interface
to make child files individually compilable by \LaTeX{}.
Compiling a child file then has the same effect as compiling
the main file with an |\includeonly| command
to select the appropriate child.
Moreover the generated document will carry the name of the child
rather than the main file.
This resolves all three above issues.

This feature is meant to make the editing of books,
thesis documents and lecture notes somewhat more convenient.
However, the package can also be used efficiently for
composing a series of documents (such as exercise sheets)
which are typically distributed individually.
It then assists the author in generating the individual documents
(potentially in different versions)
as well as a document containing the collected series.
Another application is in developing style files
or other kinds of included material
where compilation of the style file could redirect
to a sample or test file.

%%%%%%%%%%%%%%%%%%%%%%%%%%%%%%%%%%%%%%%%%%%%%%%%%%%%%%%%%%%%%%%%%%%%%%%%%%%%%%%%
%%%%%%%%%%%%%%%%%%%%%%%%%%%%%%%%%%%%%%%%%%%%%%%%%%%%%%%%%%%%%%%%%%%%%%%%%%%%%%%%
\section{Usage}

First of all, the package \textsf{childdoc} is \emph{not} a standard
\LaTeXe{} |.sty| style file! Therefore it needs to be invoked in
a non-standard way.

%%%%%%%%%%%%%%%%%%%%%%%%%%%%%%%%%%%%%%%%%%%%%%%%%%%%%%%%%%%%%%%%%%%%%%%%%%%%%%%%
\subsection{Included Files}
\label{sec:include}

%%%%%%%%%%%%%%%%%%%%%%%%%%%%%%%%%%%%%%%%
\DescribeMacro{\childdocmain}
To use the package, add the commands
\begin{center}
\begin{tabular}{l}
|% \iffalse
%
% childdoc.dtx Copyright (C) 2017-2018 Niklas Beisert
%
% This work may be distributed and/or modified under the
% conditions of the LaTeX Project Public License, either version 1.3
% of this license or (at your option) any later version.
% The latest version of this license is in
%   http://www.latex-project.org/lppl.txt
% and version 1.3 or later is part of all distributions of LaTeX
% version 2005/12/01 or later.
%
% This work has the LPPL maintenance status `maintained'.
%
% The Current Maintainer of this work is Niklas Beisert.
%
% This work consists of the files childdoc.dtx and childdoc.ins
% and the derived files childdoc.def and cdocsamp.tex with
% cdocsch1.tex, cdocsch2.tex, cdocsdrf.tex, cdocsfn1.tex, cdocsfn2.tex.
%
%<package>\ifdefined\childdocmain\endinput\fi
%<package>\ProvidesFile{childdoc.def}[2018/12/30 v2.0 child document driver]
%<samplemain>\ProvidesFile{cdocsamp.tex}[2018/12/30 v2.0 sample for childdoc]
%<*driver>
%\ProvidesFile{childdoc.drv}[2018/12/30 v2.0 childdoc reference manual file]
\PassOptionsToClass{10pt,a4paper}{article}
\documentclass{ltxdoc}

\usepackage[margin=35mm]{geometry}
\usepackage{hyperref}
\usepackage{hyperxmp}
\usepackage[usenames]{color}

\hypersetup{colorlinks=true}
\hypersetup{pdfstartview=FitH}
\hypersetup{pdfpagemode=UseNone}
\hypersetup{pdfsource={}}
\hypersetup{pdflang={en-UK}}
\hypersetup{pdfcopyright={Copyright 2017-2018 Niklas Beisert.
  This work may be distributed and/or modified under the
  conditions of the LaTeX Project Public License, either version 1.3
  of this license or (at your option) any later version.}}
\hypersetup{pdflicenseurl={http://www.latex-project.org/lppl.txt}}
\hypersetup{pdfcontactaddress={ETH Zurich, ITP, HIT K,
  Wolfgang-Pauli-Strasse 27}}
\hypersetup{pdfcontactpostcode={8093}}
\hypersetup{pdfcontactcity={Zurich}}
\hypersetup{pdfcontactcountry={Switzerland}}
\hypersetup{pdfcontactemail={nbeisert@itp.phys.ethz.ch}}
\hypersetup{pdfcontacturl={http://people.phys.ethz.ch/\xmptilde nbeisert/}}

\newcommand{\secref}[1]{\hyperref[#1]{section \ref*{#1}}}

\parskip1ex
\parindent0pt
\let\olditemize\itemize
\def\itemize{\olditemize\parskip0pt}

\begin{document}

\title{The \textsf{childdoc} Package}
\hypersetup{pdftitle={The childdoc Package}}
\author{Niklas Beisert\\[2ex]
  Institut f\"ur Theoretische Physik\\
  Eidgen\"ossische Technische Hochschule Z\"urich\\
  Wolfgang-Pauli-Strasse 27, 8093 Z\"urich, Switzerland\\[1ex]
  \href{mailto:nbeisert@itp.phys.ethz.ch}
  {\texttt{nbeisert@itp.phys.ethz.ch}}}
\hypersetup{pdfauthor={Niklas Beisert}}
\hypersetup{pdfsubject={Manual for the LaTeX2e Package childdoc}}
\date{30 December 2018, \textsf{v2.0}}
\maketitle

\begin{abstract}\noindent
\textsf{childdoc} is a \LaTeXe{} package
that enables the direct compilation
of document sections included by |\include|
to individual files.
\end{abstract}

\begingroup
\parskip0ex
\tableofcontents
\endgroup

%%%%%%%%%%%%%%%%%%%%%%%%%%%%%%%%%%%%%%%%%%%%%%%%%%%%%%%%%%%%%%%%%%%%%%%%%%%%%%%%
%%%%%%%%%%%%%%%%%%%%%%%%%%%%%%%%%%%%%%%%%%%%%%%%%%%%%%%%%%%%%%%%%%%%%%%%%%%%%%%%
\section{Introduction}

\LaTeX{} provides a mechanism to structure a large document (such as a book)
into a main file and several child files (containing the chapters)
using the |\include| command.
This mechanism is beneficial for documents
which span hundreds of pages in order to
make the source file(s) more manageable.
Moreover, compilation can be restricted to
selected child files by means of the |\includeonly| command.
The latter feature can be used to reduce the compilation time while editing
(this was significantly more useful in the earlier days of \LaTeX{})
or to generate a smaller document which is easier to navigate.
Another application of |\includeonly| is to generate
documents consisting of selected parts of the complete document.

However, there are a few drawbacks of the plain |\include| mechanism:
\begin{itemize}
\item
The child files cannot be compiled on their own,
they can only be compiled via the main file.
A naive editing environment
(such as a text editor with an option
to have the current file processed by \LaTeX)
may require one to switch to the main file before compiling;
attempting to compile the child file produces errors.
\item
The main file must be modified (each time)
to adjust the |\includeonly| command
to the present needs. This easily leaves the main file in a messy state.
\item
The generated document will always carry the filename
of the main document. This is inconvenient if
several child files are to be compiled and
to be kept for distribution.
\end{itemize}

The present package provides a simple interface
to make child files individually compilable by \LaTeX{}.
Compiling a child file then has the same effect as compiling
the main file with an |\includeonly| command
to select the appropriate child.
Moreover the generated document will carry the name of the child
rather than the main file.
This resolves all three above issues.

This feature is meant to make the editing of books,
thesis documents and lecture notes somewhat more convenient.
However, the package can also be used efficiently for
composing a series of documents (such as exercise sheets)
which are typically distributed individually.
It then assists the author in generating the individual documents
(potentially in different versions)
as well as a document containing the collected series.
Another application is in developing style files
or other kinds of included material
where compilation of the style file could redirect
to a sample or test file.

%%%%%%%%%%%%%%%%%%%%%%%%%%%%%%%%%%%%%%%%%%%%%%%%%%%%%%%%%%%%%%%%%%%%%%%%%%%%%%%%
%%%%%%%%%%%%%%%%%%%%%%%%%%%%%%%%%%%%%%%%%%%%%%%%%%%%%%%%%%%%%%%%%%%%%%%%%%%%%%%%
\section{Usage}

First of all, the package \textsf{childdoc} is \emph{not} a standard
\LaTeXe{} |.sty| style file! Therefore it needs to be invoked in
a non-standard way.

%%%%%%%%%%%%%%%%%%%%%%%%%%%%%%%%%%%%%%%%%%%%%%%%%%%%%%%%%%%%%%%%%%%%%%%%%%%%%%%%
\subsection{Included Files}
\label{sec:include}

%%%%%%%%%%%%%%%%%%%%%%%%%%%%%%%%%%%%%%%%
\DescribeMacro{\childdocmain}
To use the package, add the commands
\begin{center}
\begin{tabular}{l}
|\input{childdoc.def}|\\
|\childdocmain{}|\\
\end{tabular}
\end{center}
at the very top of the main \LaTeX{} file,
in particular \emph{before} the |\documentclass| statement!
The argument of |\childdocmain| should be left empty
(but it must be present).

%%%%%%%%%%%%%%%%%%%%%%%%%%%%%%%%%%%%%%%%
\DescribeMacro{\childdocof}
Furthermore, add the commands
\begin{center}
\begin{tabular}{l}
|\input{childdoc.def}|\\
|\childdocof{|\textit{main}|}|\\
\end{tabular}
\end{center}
at the top of every child file \textit{child}
which is included by |\include{|\textit{child}|}|
from within the main file
(or at least for those files to be compiled individually).
The argument \textit{main} must be the filename of the main file.

There are a couple of
considerations in setting up the main and child documents:

%%%%%%%%%%%%%%%%%%%%%%%%%%%%%%%%%%%%%%%%
\paragraph{Restrictions.}

Please note the following restrictions:
\begin{itemize}
\item
|\childdocmain| must be called with one argument \textit{main}
to ensure compatibility with earlier version of the package.
It must either be empty (|\childdocmain{}|)
or precisely match the filename of the main file in which it is specified.
See \secref{sec:detection} for further information.
\item
The filename \textit{main} must be specified without the |.tex| extension.
\item
The filename \textit{main} is case sensitive
(even in case-insensitive file systems)
due to internal string comparison.
\item
The argument \textit{main} should be fully expanded, it cannot be a macro.
\item
Subdirectories and special characters should be avoided in filenames.
\item
The command |\childdocmain{|\textit{main}|}| must be followed by a whitespace.
It should not be followed immediately by another command
or by a comment mark `|%|'.
This is because the \TeX{} parser reads the token immediately following
the argument of |\childdocmain| and puts it
at the beginning of every child section;
however, a white\-space is ignored.
\end{itemize}

%%%%%%%%%%%%%%%%%%%%%%%%%%%%%%%%%%%%%%%%
\paragraph{Content of Main File.}

It is advisable to place all content in the child files included by |\include|.
Any output contained in the main file will appear in all child documents
unless suppressed manually;
it cannot be suppressed automatically by the |\includeonly| directive
and thus should normally be avoided.
A method to include some content in the main file
by means of conditional processing is described in \secref{sec:conditional}.

%%%%%%%%%%%%%%%%%%%%%%%%%%%%%%%%%%%%%%%%
\paragraph{Page Numbering.}

When only a part of the document is compiled,
the appropriate numbering of pages
(as well as other status parameters)
is determined from the |.aux| files.
The latter contain information from previous passes.
However this information needs to propagate through
all intermediate child documents.
Therefore the page numbering in child documents may well
be inconsistent until the complete document is compiled at least once.

A useful (if unconventional) way to always ensure a consistent
page numbering is to restart the numbering in each child document
and denote the pages by `\textit{child}|.|\textit{page}'
where \textit{child} represents the chapter/section number of the child file.
This can be achieved by the command
|\numberwithin{page}{|\textit{child}|}|
of the \textsf{amsmath} package
where \textit{child} can be |chapter| or |section|
depending on the chosen structuring.
Alternatively, one can modify the macro |\thepage| appropriately
and reset the counter |page| at the start of each child file.

%%%%%%%%%%%%%%%%%%%%%%%%%%%%%%%%%%%%%%%%%%%%%%%%%%%%%%%%%%%%%%%%%%%%%%%%%%%%%%%%
\subsection{Conditional Processing}
\label{sec:conditional}

The package provides a mechanism to compile different versions
of a document. To customise the versions further some conditional processing
can come in handy to distinguish which version is being compiled.
The package provides two macros to describe the compilation context:

%%%%%%%%%%%%%%%%%%%%%%%%%%%%%%%%%%%%%%%%
\DescribeMacro{\ifchilddoc}
The conditional |\ifchilddoc| distinguishes between the compilation of
child documents and the main document:
%
\begin{center}
|\ifchilddoc |\textit{child-code}| |[|\||else |\textit{main-code}]| \||fi|
\end{center}

%%%%%%%%%%%%%%%%%%%%%%%%%%%%%%%%%%%%%%%%
\DescribeMacro{\childdocname}
\DescribeMacro{\childdocjob}
The macro |\childdocname| contains the filename (without extension)
of the main or child file being processed.
Note that |\childdocjob| will always contain the name of the main file.

%%%%%%%%%%%%%%%%%%%%%%%%%%%%%%%%%%%%%%%%
\paragraph{Title Page.}

Conditional processing can be used to include a title or banner page
in the main document when proper precautions are taken.
Importantly, the code in the main file should ensure that the page counter
(as well as other status parameters which are stored in the |.aux| files)
takes the same value after the conditional processing.
Otherwise the page numbers may take divergent values
depending on which part is compiled.

For example, a title page could be declared by:
%
\begin{center}
\begin{tabular}{l}
|\ifchilddoc\||else|\\
|\addtocounter{page}{-1}|\\
\textit{code for title page}\\
|\newpage|\\
|\||fi|
\end{tabular}
\end{center}
%
A banner page for the child documents can be generated by:
%
\begin{center}
\begin{tabular}{l}
|\ifchilddoc|\\
|\addtocounter{page}{-1}|\\
\textit{code for banner page}\\
|\newpage|\\
|\||fi|
\end{tabular}
\end{center}
%
Here one could write a message such as:
\begin{center}
|This is the part \childdocname{} of \childdocjob{}.|
\end{center}

%%%%%%%%%%%%%%%%%%%%%%%%%%%%%%%%%%%%%%%%%%%%%%%%%%%%%%%%%%%%%%%%%%%%%%%%%%%%%%%%
\subsection{Flags}
\label{sec:flags}

The package makes it easy to generate different versions
of the main or child documents.
To this end compilation flags can be defined
and assigned different default values.
They will be particularly useful in conjunction
with the forwarding mechanism described in \secref{sec:forward}.

For example, it may be useful to have a flag |\version|
which can be set to |draft| or |final|.
The document source will contain some conditional code
depending on the value of |\version|.
Suppose further, the flag should default to |final| for the main file
and to |draft| for child files
which is a natural assignment for editing the document.
This is achieved by placing the following code
in the preamble of the main document
(below the |\childdocmain| directive):
%
\begin{center}
\begin{tabular}{l}
|\ifchilddoc|\\
|\providecommand{\version}{draft}|\\
|\||else|\\
|\providecommand{\version}{final}|\\
|\||fi|
\end{tabular}
\end{center}
%
The definition by |\providecommand| makes sure
that previous definitions are not overwritten.
Further statements |\providecommand{\version}{...}|
can thus be added before the above code to override it.

For the main file, one might add a line
(between |\childdocmain| and the above block)
%
\begin{center}
|%\ifchilddoc\||else\providecommand{\version}{draft}\||fi|
\end{center}
%
which can be uncommented to produce a draft version.
Likewise one can add a line to the very top of a child file
(above the |\childdocof{|\textit{main}|}| directive)
%
\begin{center}
|%\providecommand{\version}{final}|
\end{center}
%
which can be uncommented to produce the final version of this child document.

%%%%%%%%%%%%%%%%%%%%%%%%%%%%%%%%%%%%%%%%%%%%%%%%%%%%%%%%%%%%%%%%%%%%%%%%%%%%%%%%
\subsection{Forwarding}
\label{sec:forward}

Different versions of the main or child documents
using compilation flags as described in \secref{sec:flags}
can be (permanently) stored in different files
for convenient compilation, viewing and distribution.
To this end, the package defines a command
to pass on compilation to a different file:

%%%%%%%%%%%%%%%%%%%%%%%%%%%%%%%%%%%%%%%%
\DescribeMacro{\childdocforward}
The command |\childdocforward| redirects processing to
another source file:
%
\begin{center}
\begin{tabular}{l}
|\input{childdoc.def}|\\
|\childdocforward[|\textit{main}|]{|\textit{dest}|}|\\
\end{tabular}
\end{center}
%
The argument \textit{dest} is the destination file
(without extension).
It should be the main file or one of the child files.
Note that further \textsf{childdoc} directives
such as |\childdocof| and |\childdocforward|
in the indicated file will be processed in this form.
The optional argument \textit{main}
passes on directly to the main file \textit{main}
while pretending to compile the child \textit{dest}.
This form behaves as if \textit{dest}
issues |\childdocof{|\textit{main}|}| right away,
and no further \textsf{childdoc} directives will be processed.

%%%%%%%%%%%%%%%%%%%%%%%%%%%%%%%%%%%%%%%%
\DescribeMacro{\...prefix}
In the alternative form |\childdocforwardprefix|,
%
\begin{center}
\begin{tabular}{l}
|\input{childdoc.def}|\\
|\childdocforwardprefix[|\textit{main}|]{|\textit{prefix}|}{|\textit{dest}|}|
\end{tabular}
\end{center}
%
the destination file is determined by a pattern
depending on the current file:
To make this work, the current file must be called
`{\textit{prefix}\hspace{0.2em}\textit{suffix}}'
with \textit{prefix} matching precisely the argument.
Processing is then passed on to the file
`{\textit{dest}\hspace{0.2em}\textit{suffix}}'.
Surely, the same effect is achieved by
directly specifying the
argument `{\textit{dest}\hspace{0.2em}\textit{suffix}}'
in the first form.
However, that requires to set up a different file
for each child. With the alternative form of the command
all these files can have exactly the same content
which simplifies setting them up and maintaining them.

For example, the following file |draft.tex|
with a compilation flag |\version| as described in \secref{sec:flags}
compiles the main document as a draft:
%
\begin{center}
\begin{tabular}{l}
|\def\version{draft}|\\
|\input{childdoc.def}|\\
|\childdocforward{|\textit{main}|}|
\end{tabular}
\end{center}
%
Likewise, the following files |final|\textit{nn}|.tex|
compile the final version of the child document
|child|\textit{nn}|.tex|:
%
\begin{center}
\begin{tabular}{l}
|\def\version{final}|\\
|\input{childdoc.def}|\\
|\childdocforwardprefix{final}{child}|
\end{tabular}
\end{center}
%

Note that when several versions of a main file and/or of each child file
are to be generated, it may be convenient to set up a |Makefile| or
shell script to automatise the process.

%%%%%%%%%%%%%%%%%%%%%%%%%%%%%%%%%%%%%%%%%%%%%%%%%%%%%%%%%%%%%%%%%%%%%%%%%%%%%%%%
\subsection{Command Line Processing}
\label{sec:commandline}

The effect of redirection files can also be achieved by invoking
the \LaTeX{} compiler with a more elaborate command line.
Most conveniently this should be done as part
of a shell script or a |Makefile|.

When using \textsf{childdoc} in the main file, the following
command lines effectively perform a redirection
(note that depending on the shell being used,
backslashes may have to be doubled: `|\|' $\to$ `|\\|'):
%
\begin{center}
|... -jobname "|\textit{target}|" |\\|"|[\textit{flags}]%
|\input{childdoc.def}\childdocforward[|\textit{main}|]{|\textit{dest}|}"|
\end{center}
%
Here \textit{target} is the name of the output file,
\textit{main} is the name of the main file
and \textit{dest} is the name of the main or child file to be processed
(all filenames without extensions).
The optional argument \textit{main} can be omitted
if \textit{main} matches \textit{dest}.
Optionally, compilation \textit{flags} can be defined via |\def| commands.
This command line makes the \TeX{} engine believe
it is compiling the file \textit{target}
whose content is specified as the latter parameter.
The provided code then forwards the processing to
\textit{main} or \textit{dest} as described in \secref{sec:forward}.

%%%%%%%%%%%%%%%%%%%%%%%%%%%%%%%%%%%%%%%%%%%%%%%%%%%%%%%%%%%%%%%%%%%%%%%%%%%%%%%%
\subsection{Include by Input}
\label{sec:input}

Including child documents by |\include| has some restrictions by design.
Most notably, the content of a child document always occupies
its own set of pages; pages cannot be shared between child documents.
Usually, this behaviour makes perfect sense
because each child document contain an essential part of the document.
However, in some situations it may be desirable to compose
a document from a collection of parts
without having mandatory page breaks between then.
For this case, the package
provides a mechanism to include parts
by |\input| which can also be processed individually.
However, by construction this mechanism
requires manual handling of the content to be output.

%%%%%%%%%%%%%%%%%%%%%%%%%%%%%%%%%%%%%%%%
\DescribeMacro{\ifchilddocmanual}
The main file should be prepared as usual, see \secref{sec:include}.
However, the document body must make a distinction
between processing of an individual part and of the main document, e.g.:
%
\begin{center}
\begin{tabular}{l}
|\ifchilddocmanual|\\
|\input{\childdocname}|\\
|\||else|\\
\textit{document body with }|\input{|\textit{part}|}|\\
|\||fi|
\end{tabular}
\end{center}
%
The conditional |\ifchilddocmanual| is true whenever
a part to be included by |\input| is being compiled,
and the name of the part is stored in |\childdocname|.

%%%%%%%%%%%%%%%%%%%%%%%%%%%%%%%%%%%%%%%%
\DescribeMacro{\childdocby}
Each part to be included by |\input| should start with:
%
\begin{center}
\begin{tabular}{l}
|\input{childdoc.def}|\\
|\childdocby{|\textit{main}|}|\\
\end{tabular}
\end{center}
%
The directive |\childdocby| is similar to |\childdocof|
described in \secref{sec:include},
but the subsequent selection of content must be done manually.
To that end, both |\ifchilddoc| and |\ifchilddocmanual|
will be true upon processing of a part,
and the name of the part is stored in |\childdocname|.
Note that |\jobname| will be set to the filename of the current part
so that each part receives an individual |.aux| file
that does not interfere with the |.aux| file(s) of the main document.
This behaviour can be altered by the alternative form
|\childdocby[*]{|\textit{main}|}| (with a non-empty optional argument)
which uses the |.aux| file of the main document
by setting |\jobname| to \textit{main}.

%%%%%%%%%%%%%%%%%%%%%%%%%%%%%%%%%%%%%%%%%%%%%%%%%%%%%%%%%%%%%%%%%%%%%%%%%%%%%%%%
\subsection{Driver Development}
\label{sec:driver}

The \textsf{childdoc} mechanism can also be use for the development
of definition files such as \LaTeX{} styles or classes.
This case differs from the above setup with multiple parts
included by |\include| in that no |\includeonly| should be invoked.
This can be achieved by starting the include file
(before |\ProvidesPackage|) with:
%
\begin{center}
\begin{tabular}{l}
|\input{childdoc.def}|\\
|\childdocforward{|\textit{main}|}|\\
\end{tabular}
\end{center}
%
or alternatively with:
%
\begin{center}
\begin{tabular}{l}
|\input{childdoc.def}|\\
|\childdocby{|\textit{main}|}|\\
\end{tabular}
\end{center}
%
Both forms have slightly different effects as described above.
The main file is prepared as usual, see \secref{sec:include}.

%%%%%%%%%%%%%%%%%%%%%%%%%%%%%%%%%%%%%%%%%%%%%%%%%%%%%%%%%%%%%%%%%%%%%%%%%%%%%%%%
\subsection{Legacy Detection}
\label{sec:detection}

The directive |\childdocmain| in the main file can detect
whether the complete document or merely a child is to be compiled
even without using the directive |\childdocof|.
This method is deprecated because it is less robust
and there is no compelling reason to use it;
it is merely provided for backward compatibility
and it may be removed in future versions.

If the detection mechanism is to be used,
it is mandatory to correctly specify
the filename of the main file as the argument of |\childdocmain|:
%
\begin{center}
\begin{tabular}{l}
|\input{childdoc.def}|\\
|\childdocmain{|\textit{main}|}|\\
\end{tabular}
\end{center}
%
If |\jobname| does not match the argument \textit{main} of |\childdocmain|,
it is assumed that |\jobname| points to the child file to be compiled.
When using |\childdocmain| with the main file specified as argument,
it suffices to start a child file
with just |\input{|\textit{main}|}|
without loading of the package and using |\childdocof|.
If instead all processing is done
with the appropriate \textsf{childdoc} directives,
the argument of \textit{main} of |\childdocmain| can be empty.

An alternative version of the command line processing described
in \secref{sec:commandline} using the detection mechanism reads:
%
\begin{center}
|... -jobname "|\textit{target}|" "|[\textit{flags}]%
[|\def\jobname{|\textit{dest}|}|]|\input{|\textit{main}|}"|
\end{center}

%%%%%%%%%%%%%%%%%%%%%%%%%%%%%%%%%%%%%%%%%%%%%%%%%%%%%%%%%%%%%%%%%%%%%%%%%%%%%%%%
\subsection{Manual Code}
\label{sec:manual}

In case one cannot be certain whether the definitions file |childdoc.def|
is installed on the target \TeX{} distribution
and one prefers not to ship it,
it is conceivable to paste a few relevant commands into the sources.

To that end, drop all statements |\input{childdoc.def}|
and perform the replacements as outlined below.
Instead of |\childdocmain{|\textit{main}|}| add the following code
to the top of the main file:
%
\begin{center}
\begin{tabular}{l}
|\||ifdefined\childdocname\endinput\||fi\newif\ifchilddoc|\\
|\edef\childdocname{\scantokens\expandafter{\jobname\noexpand}}|\\
|\def\childdocmain{|\textit{main}|}\||ifx\childdocmain\childdocname\||else|\\
|\childdoctrue\includeonly{\childdocname}\let\jobname\childdocmain\||fi|\\
\end{tabular}
\end{center}
%
Instead of |\childdocof{|\textit{main}|}| just include the main file
at the top of each child file:
%
\begin{center}
|\input{|\textit{main}|}|
\end{center}
%
A simple redirection |\childdocforward{|\textit{dest}|}| is achieved by:
%
\begin{center}
|\def\jobname{|\textit{dest}|}\input{\jobname}|
\end{center}
%
The redirection with prefix
|\childdocforwardprefix[|\textit{prefix}|]{|\textit{dest}|}|
is accomplished by:
%
\begin{center}
\begin{tabular}{l}
|{\edef\jobname{\scantokens\expandafter{\jobname\noexpand}}|\\
|\def\redirectjob |\textit{prefix}|#1~~~{\gdef\jobname{|\textit{dest}|#1}}|\\
|\expandafter\redirectjob\jobname~~~}\input{\jobname}|
\end{tabular}
\end{center}

In an alternative approach,
child documents can be compiled by a specific command line
without additional code or specific definitions:
%
\begin{center}
|... -jobname "|\textit{target}|" "|[\textit{flags}]%
|\includeonly{|\textit{dest}|}\input{|\textit{main}|}"|
\end{center}
%

%%%%%%%%%%%%%%%%%%%%%%%%%%%%%%%%%%%%%%%%%%%%%%%%%%%%%%%%%%%%%%%%%%%%%%%%%%%%%%%%
%%%%%%%%%%%%%%%%%%%%%%%%%%%%%%%%%%%%%%%%%%%%%%%%%%%%%%%%%%%%%%%%%%%%%%%%%%%%%%%%
\section{Information}

%%%%%%%%%%%%%%%%%%%%%%%%%%%%%%%%%%%%%%%%%%%%%%%%%%%%%%%%%%%%%%%%%%%%%%%%%%%%%%%%
\subsection{Copyright}

Copyright \copyright{} 2017--2018 Niklas Beisert

This work may be distributed and/or modified under the
conditions of the \LaTeX{} Project Public License, either version 1.3
of this license or (at your option) any later version.
The latest version of this license is in
  \url{http://www.latex-project.org/lppl.txt}
and version 1.3 or later is part of all distributions of \LaTeX{}
version 2005/12/01 or later.

This work has the LPPL maintenance status `maintained'.

The Current Maintainer of this work is Niklas Beisert.

This work consists of the files |README.txt|, |childdoc.ins| and |childdoc.dtx|
as well as the derived files |childdoc.def|, |cdocsamp.tex|
with |cdocsch1.tex|, |cdocsch2.tex|, |cdocspt3.tex|, |cdocspt4.tex|,
|cdocsdrf.tex|, |cdocsfn1.tex|, |cdocsfn2.tex|
as well as |childdoc.pdf|.

%%%%%%%%%%%%%%%%%%%%%%%%%%%%%%%%%%%%%%%%%%%%%%%%%%%%%%%%%%%%%%%%%%%%%%%%%%%%%%%%
\subsection{Files and Installation}

The package consists of the files:
%
\begin{center}
\begin{tabular}{ll}
    |README.txt|   & readme file \\
    |childdoc.ins| & installation file \\
    |childdoc.dtx| & source file \\
    |childdoc.def| & definition file \\
    |cdocsamp.tex| & sample main file \\
    |cdocsch1.tex| & sample include file \\
    |cdocsch2.tex| & sample include file \\
    |cdocspt3.tex| & sample part file \\
    |cdocspt4.tex| & sample part file \\
    |cdocsdrf.tex| & sample redirection file \\
    |cdocsfn1.tex| & sample redirection file \\
    |cdocsfn2.tex| & sample redirection file \\
    |childdoc.pdf| & manual
\end{tabular}
\end{center}
%
The distribution consists of the files
|README.txt|, |childdoc.ins| and |childdoc.dtx|.
%
\begin{itemize}
\item
Run (pdf)\LaTeX{} on |childdoc.dtx|
to compile the manual |childdoc.pdf| (this file).
\item
Run \LaTeX{} on |childdoc.ins| to create the definitions file |childdoc.def|
and the sample |cdocsamp.tex| with include files
|cdocsch1.tex|, |cdocsch2.tex|, |cdocspt3.tex|, |cdocspt4.tex|,
|cdocsdrf.tex|, |cdocsfn1.tex|, |cdocsfn2.tex|.
Then copy the file |childdoc.def| to an appropriate directory of your \LaTeX{}
distribution, e.g.\ \textit{texmf-root}|/tex/latex/childdoc|.
\end{itemize}

%%%%%%%%%%%%%%%%%%%%%%%%%%%%%%%%%%%%%%%%%%%%%%%%%%%%%%%%%%%%%%%%%%%%%%%%%%%%%%%%
\subsection{Related CTAN Packages}

There are several other packages which offer a similar functionality:
%
\begin{itemize}
\item
The packages
\href{http://ctan.org/pkg/docmute}{\textsf{docmute}},
\href{http://ctan.org/pkg/includex}{\textsf{includex}} and
\href{http://ctan.org/pkg/standalone}{\textsf{standalone}}
provide commands to include only the document body of
a child file thus allowing both files to be compiled individually.
\item
The packages \href{http://ctan.org/pkg/subdocs}{\textsf{subdocs}}
and \href{http://ctan.org/pkg/subfiles}{\textsf{subfiles}}
provide structures in which the main and child documents can be
encapsulated and allowing them to be compiled individually.
The inclusion mechanism is different from the conventional |\include|.
\item
The package \href{http://ctan.org/pkg/combine}{\textsf{combine}}
is an elaborate solution to combine several documents into one.
\end{itemize}
%
See also the CTAN topic \href{http://ctan.org/topic/subdocs}{\textsf{subdocs}}
for further related packages.
The present package differs from the above solutions in that
a document structure constructed with the conventional |\include| mechanism
just needs two extra commands at the top of every file
such that all constituent files can be compiled individually.

%%%%%%%%%%%%%%%%%%%%%%%%%%%%%%%%%%%%%%%%%%%%%%%%%%%%%%%%%%%%%%%%%%%%%%%%%%%%%%%%
%\subsection{Feature Suggestions}
%
%The following is a list of features which may be useful for future
%versions of this package:
%%
%\begin{itemize}
%\item
%\ldots
%\end{itemize}

%%%%%%%%%%%%%%%%%%%%%%%%%%%%%%%%%%%%%%%%%%%%%%%%%%%%%%%%%%%%%%%%%%%%%%%%%%%%%%%%
\subsection{Revision History}

%%%%%%%%%%%%%%%%%%%%%%%%%%%%%%%%%%%%%%%%
\paragraph{v2.0:} 2018/12/30

\begin{itemize}
\item
immediate forward processing
\item
added |\childdocby| mechanism
\item
manual restructured
\end{itemize}

%%%%%%%%%%%%%%%%%%%%%%%%%%%%%%%%%%%%%%%%
\paragraph{v1.6:} 2018/01/17

\begin{itemize}
\item
application for development of include files
\item
corrections to manual
\end{itemize}

%%%%%%%%%%%%%%%%%%%%%%%%%%%%%%%%%%%%%%%%
\paragraph{v1.5:} 2017/05/21

\begin{itemize}
\item
more complete structuring introduced
\item
|\childdocof| introduced
\item
|\childdoc| renamed to |\childdocmain|
\item
|\childredirect| renamed to |\childdocforward| and |\childdocforwardprefix|
and functionality expanded
\end{itemize}

%%%%%%%%%%%%%%%%%%%%%%%%%%%%%%%%%%%%%%%%
\paragraph{v1.0:} 2017/04/27

\begin{itemize}
\item
manual and install package
\item
first version published on CTAN
\end{itemize}

%%%%%%%%%%%%%%%%%%%%%%%%%%%%%%%%%%%%%%%%
\paragraph{v0.6:} 2017/04/26

\begin{itemize}
\item
redirection mechanism added
\end{itemize}

%%%%%%%%%%%%%%%%%%%%%%%%%%%%%%%%%%%%%%%%
\paragraph{v0.5:} 2017/04/26

\begin{itemize}
\item
functionality in definition file
\end{itemize}


%%%%%%%%%%%%%%%%%%%%%%%%%%%%%%%%%%%%%%%%%%%%%%%%%%%%%%%%%%%%%%%%%%%%%%%%%%%%%%%%
%%%%%%%%%%%%%%%%%%%%%%%%%%%%%%%%%%%%%%%%%%%%%%%%%%%%%%%%%%%%%%%%%%%%%%%%%%%%%%%%
%%%%%%%%%%%%%%%%%%%%%%%%%%%%%%%%%%%%%%%%%%%%%%%%%%%%%%%%%%%%%%%%%%%%%%%%%%%%%%%%
\appendix

\settowidth\MacroIndent{\rmfamily\scriptsize 000\ }

 \DocInput{childdoc.dtx}

\end{document}
%</driver>
% \fi
%
% %%%%%%%%%%%%%%%%%%%%%%%%%%%%%%%%%%%%%%%%%%%%%%%%%%%%%%%%%%%%%%%%%%%%%%%%%%%%%%
% %%%%%%%%%%%%%%%%%%%%%%%%%%%%%%%%%%%%%%%%%%%%%%%%%%%%%%%%%%%%%%%%%%%%%%%%%%%%%%
% \section{Sample}
%\iffalse
%<*samplemain>
%\fi
%
% The following presents a sample document
% with two chapters, two parts, a title page,
% a compile flag as well as three forwarding files to set the flag.
% It consists of eight |.tex| files:
% \begin{center}
% \begin{tabular}{ll}
% |cdocsamp.tex|&main file\\
% |cdocsch1.tex|&include file for chapter 1\\
% |cdocsch2.tex|&include file for chapter 2\\
% |cdocspt3.tex|&include file for part 3\\
% |cdocspt4.tex|&include file for part 4\\
% |cdocsdrf.tex|&forwarding file for main file in draft mode\\
% |cdocsfi1.tex|&forwarding file for final version of chapter 1\\
% |cdocsfi2.tex|&forwarding file for final version of chapter 2\\
% \end{tabular}
% \end{center}
% Each of the eight files can be compiled directly by the \LaTeX{} compiler.
%
% %%%%%%%%%%%%%%%%%%%%%%%%%%%%%%%%%%%%%%
% \paragraph{Main File.}
%
% The main file is called |cdocsamp.tex|.
%
% Load the \textsf{childdoc} definitions and
% declare the filename for the main document:
%    \begin{macrocode}
\input{childdoc.def}
\childdocmain{}
%    \end{macrocode}

% Optional override for |\version| flag:
%    \begin{macrocode}
%%\ifchilddoc\else\providecommand{\version}{draft}\fi
%    \end{macrocode}

% Define the default values for the |\version| flag
% (|final| for the main file and |draft| for childs):
%    \begin{macrocode}
\ifchilddoc
\providecommand{\version}{draft}
\else
\providecommand{\version}{final}
\fi
%    \end{macrocode}

% Load the standard document class:
%    \begin{macrocode}
\documentclass[12pt]{article}
%    \end{macrocode}

% Start the document body:
%    \begin{macrocode}
\begin{document}
%    \end{macrocode}

% Declare a title page.
% Print title, part of document being processed and version flag:
%    \begin{macrocode}
\addtocounter{page}{-1}
\begin{center}
{\LARGE\bfseries{}childdoc example\par}
\vspace{1cm}
\ifchilddoc
\ifchilddocmanual part\else chapter\fi:
`\childdocname' of `\childdocjob'\par
\else
main document: `\childdocjob'\par
\fi
version: \version\par
\end{center}
\newpage
%    \end{macrocode}

% Manually include selected file,
% otherwise process as usual:
%    \begin{macrocode}
\ifchilddocmanual
\section*{part `\childdocname'}
\input{\childdocname}
\else
%    \end{macrocode}

% Include the two chapters:
%    \begin{macrocode}
\include{cdocsch1}
\include{cdocsch2}
%    \end{macrocode}

% Include the two parts unless only chapters should be displayed:
%    \begin{macrocode}
\ifchilddoc\else
\section{part three}
\input{cdocspt3}
\section{part four}
\input{cdocspt4}
\fi
%    \end{macrocode}

% Process as usual until here:
%    \begin{macrocode}
\fi
%    \end{macrocode}

% End of document body:
%    \begin{macrocode}
\end{document}
%    \end{macrocode}
%\iffalse
%</samplemain>
%\fi
%
% %%%%%%%%%%%%%%%%%%%%%%%%%%%%%%%%%%%%%%
% \paragraph{Chapter Include Files.}
%
% The include files are called |cdocsch1.tex| and |cdocsch2.tex|.
%
%\iffalse
%<*samplechap1|samplechap2>
%\fi

% Optional override for |\version| flag:
%    \begin{macrocode}
%%\providecommand{\version}{final}
%    \end{macrocode}

% Include the main document:
%    \begin{macrocode}
\input{childdoc.def}
\childdocof{cdocsamp}
%    \end{macrocode}

%\iffalse
%</samplechap1|samplechap2>
%\fi
%
%\iffalse
%<*samplechap1>
%\fi
% Some text for chapter 1:
%    \begin{macrocode}
\section{one}
some text in chapter one
%    \end{macrocode}

%\iffalse
%</samplechap1>
%\fi
% Some text for chapter 2:
%\iffalse
%<*samplechap2>
%\fi
%    \begin{macrocode}
\section{two}
more text in chapter two
%    \end{macrocode}

%\iffalse
%</samplechap2>
%\fi
%
% %%%%%%%%%%%%%%%%%%%%%%%%%%%%%%%%%%%%%%
% \paragraph{Part Include Files.}
%
% The include files are called |cdocspt3.tex| and |cdocspt4.tex|.
%
%\iffalse
%<*samplepart3|samplepart4>
%\fi

% Optional override for |\version| flag:
%    \begin{macrocode}
%%\providecommand{\version}{final}
%    \end{macrocode}

% Include the main document:
%    \begin{macrocode}
\input{childdoc.def}
\childdocby{cdocsamp}
%    \end{macrocode}

%\iffalse
%</samplepart3|samplepart4>
%\fi
%
%\iffalse
%<*samplepart3>
%\fi
% Some text for part 3:
%    \begin{macrocode}
some text in part three
%    \end{macrocode}

%\iffalse
%</samplepart3>
%\fi
% Some text for part 4:
%\iffalse
%<*samplepart4>
%\fi
%    \begin{macrocode}
more text in part four
%    \end{macrocode}

%\iffalse
%</samplepart4>
%\fi
%
% %%%%%%%%%%%%%%%%%%%%%%%%%%%%%%%%%%%%%%
% \paragraph{Forwarding for a Complete Draft.}
%
% The following forwarding file |cdocsdrf.tex|
% compiles the main document in draft mode:
%\iffalse
%<*sampledraft>
%\fi
%    \begin{macrocode}
\def\version{draft}
\input{childdoc.def}
\childdocforward{cdocsamp}
%    \end{macrocode}

%\iffalse
%</sampledraft>
%\fi
%
% %%%%%%%%%%%%%%%%%%%%%%%%%%%%%%%%%%%%%%
% \paragraph{Forwarding for Final Version of the Chapters.}
%
% The following forwarding files |cdocsfn1.tex| and |cdocsfn2.tex|
% (with identical content)
% compile the final versions of the child documents
% |cdocsch1.tex| and |cdocsch2.tex|, respectively:
%\iffalse
%<*samplefinal>
%\fi
%    \begin{macrocode}
\def\version{final}
\input{childdoc.def}
\childdocforwardprefix[cdocsamp]{cdocsfn}{cdocsch}
%    \end{macrocode}

%\iffalse
%</samplefinal>
%\fi
%
% %%%%%%%%%%%%%%%%%%%%%%%%%%%%%%%%%%%%%%
% \paragraph{Command Line Processing.}
%
% The following three command lines generate the output files
% |cdocscld|, |cdocscl1| and |cdocscl2|
% which should be identical to
% |cdocsdrf|, |cdocsch1| and |cdocsfn2|, respectively:
% \begin{center}
% \begin{tabular}{l}
% |latex -jobname cdocscld \|\\
% |  "\def\version{draft}\input{childdoc.def}\childdocforward{cdocsamp}"|\\
% |latex -jobname cdocscl1 \|\\
% |  "\input{childdoc.def}\childdocforward[cdocsamp]{cdocsch1}"|\\
% |latex -jobname cdocscl2 \|\\
% |  "\def\version{final}\input{childdoc.def}\childdocforward{cdocsch2}"|
% \end{tabular}
% \end{center}
% Note that the trailing backslash on each first line
% merely continues the input to the second line
% (for convenient cut ant paste).
% Furthermore, the command |latex| can be replaced by any
% of its alternative versions such as |pdflatex|.
%
% %%%%%%%%%%%%%%%%%%%%%%%%%%%%%%%%%%%%%%%%%%%%%%%%%%%%%%%%%%%%%%%%%%%%%%%%%%%%%%
% %%%%%%%%%%%%%%%%%%%%%%%%%%%%%%%%%%%%%%%%%%%%%%%%%%%%%%%%%%%%%%%%%%%%%%%%%%%%%%
% \section{Implementation}
%\iffalse
%<*package>
%\fi
%
% This section describes the definitions file |childdoc.def|.

% The definitions cannot be loaded using |\usepackage| or |\RequirePackage|
% which has a mechanism to prevent loading a style file more than once.
% When loading the definitions by means of |\input|
% multiple instances have to be prevented manually:
%\iffalse
%This code needs to be before the `\ProvidesFile' directive
%which is defined at the beginning of this file.
%Therefore it is also placed there and commented out here.
%</package>
%<*discard>
%\fi
%    \begin{macrocode}
\ifdefined\childdocmain\endinput\fi
%    \end{macrocode}
%\iffalse
%</discard>
%<*package>
%\fi
%
% \macro{\ifchilddoc}
% \macro{\ifchilddocmanual}
% The conditional |\ifchilddoc| tells whether a
% child (true) or main (false) document is being compiled.
% The conditional |\ifchilddocmanual| tells whether
% the |\includeonly| mechanism is used (false) or
% the selection of child files must be performed manually (true).
% The definitions initialise to false:
%    \begin{macrocode}
\newif\ifchilddoc
\newif\ifchilddocmanual
%    \end{macrocode}

% \macro{\childdocname}
% \macro{\childdocjob}
% The macro |\childdocname| stores the name of the main document
% to be compiled. The macro |\childdocjob| stores the name of
% the document on which the \LaTeX{} compiler was originally invoked.
% The content of |\jobname| cannot be compared
% to filenames specified in the source due to different catcodes.
% The following code rescans |\jobname|, stores the result
% in |\childdocname| and saves a copy in |\childdocjob|:
%    \begin{macrocode}
\edef\childdocname{\scantokens\expandafter{\jobname\noexpand}}
\let\childdocjob\childdocname
%    \end{macrocode}

% \macro{\childdocdisable}
% The macro |\childdocdisable| prevents the main file
% from being processed more than once.
% At this stage, the main document command |\childdocmain|
% is assumed to be called once again where it should do nothing.
% Any subsequent call to it should prevent
% a secondary processing of the main document
% It overwrites the forwarding commands
% |\childdocof| and |\childdocforward|
% with empty macros to prevent further inclusions of the main document:
%    \begin{macrocode}
\newcommand{\childdocdisable}
{
  \renewcommand{\childdocmain}[1]{\renewcommand{\childdocmain}[1]{\endinput}}
  \renewcommand{\childdocof}[1]{}
  \renewcommand{\childdocby}[2][]{}
  \renewcommand{\childdocforward}[2][]{}
  \renewcommand{\childdocdisable}{}
}
%    \end{macrocode}

% \macro{\childdocmain}
% The macro |\childdocmain| is to be called at the top of the main file
% with nothing or the main filename (without extension) as argument.
% First, it breaks loops.
% If the argument is not empty and does not match |\childdocname|
% (which is set by the first inclusion of |childdoc.def|),
% |\ifchilddoc| is set to true, |\includeonly| is applied to the child file
% and |\jobname| is set to the main file
% (for proper handling of |.aux| files):
%    \begin{macrocode}
\newcommand{\childdocmain}[1]
{
  \childdocdisable\childdocmain{}
  \if?#1?\else
    \begingroup
      \def\childdoctmp{#1}
      \ifx\childdoctmp\childdocname
        \def\childdoctmp{}
      \else
        \def\childdoctmp
        {
          \childdoctrue
          \includeonly{\childdocname}
          \def\childdocjob{#1}
          \def\jobname{#1}
        }
      \fi
      \expandafter
    \endgroup
    \childdoctmp
  \fi
}
%    \end{macrocode}

% \macro{\childdocof}
% The command |\childdocof| redirects
% compilation to the main file |#1|.
%    \begin{macrocode}
\newcommand{\childdocof}[1]
{
  \childdocdisable
  \childdoctrue
  \includeonly{\childdocname}
  \def\jobname{#1}
  \def\childdocjob{#1}
  \input{#1}
}
%    \end{macrocode}

% \macro{\childdocby}
% The command |\childdocby| ....
%    \begin{macrocode}
\newcommand{\childdocby}[2][]
{
  \childdocdisable
  \childdoctrue
  \childdocmanualtrue
  \if?#1?\else
    \def\jobname{#2}
  \fi
  \def\childdocjob{#2}
  \input{#2}
  \endinput
}
%    \end{macrocode}

% \macro{\childdocforward}
% The command |\childdocforward| redirects
% compilation to the main file or
% (if the optional argument is given) a child file.
% Parameters are set as if the main file
% or a child file starting with |\childdocof| was compiled.
% Then compilation is handed over to the main file:
%    \begin{macrocode}
\newcommand{\childdocforward}[2][]
{
  \begingroup
    \if?#1?
      \def\childdoctmp
      {
        \def\childdocname{#2}
        \def\childdocjob{#2}
        \def\jobname{#2}
        \input{#2}
        \endinput
      }
    \else
      \def\childdoctmp
      {
        \childdocdisable
        \def\childdocname{#2}
        \childdoctrue
        \includeonly{#2}
        \def\childdocjob{#1}
        \def\jobname{#1}
        \input{#1}
        \endinput
      }
    \fi
    \expandafter
  \endgroup
  \childdoctmp
}
%    \end{macrocode}

% \macro{\childdocforwardprefix}
% The command |\childdocforwardprefix| redirects
% compilation to the main or a child file by means of a pattern.
% The prefix |#1| in the current filename is replaced by |#2|
% and the suffix of the current filename is kept
% (it is assumed that the filename does not contain the substring `|~~~|'
% which is used as a delimiter).
% Compilation is handed over to the new file by |\childdocforward|:
%    \begin{macrocode}
\newcommand{\childdocforwardprefix}[3][]
{
  \begingroup
    \def\childdocextract #2##1~~~{\def\childdoctmp{\childdocforward[#1]{#3##1}}}
    \expandafter\childdocextract\childdocname~~~
    \expandafter
  \endgroup
  \childdoctmp
}
%    \end{macrocode}

% \macro{\childdoc}
% The deprecated macro |\childdoc| is a legacy version of |\childdocmain|:
%    \begin{macrocode}
\newcommand{\childdoc}{\childdocmain}
%    \end{macrocode}

% \macro{\childdocredirect}
% The deprecated macro |\childdocredirect| is a legacy version
% of |\childdocforward| and |\childdocforwardprefix|:
%    \begin{macrocode}
\newcommand{\childdocredirect}[2][]
{
  \begingroup
    \if?#1?
      \def\childdoctmp{\childdocforward{#2}}
    \else
      \def\childdoctmp{\childdocforwardprefix{#1}{#2}}
    \fi
    \expandafter
  \endgroup
  \childdoctmp
}
%    \end{macrocode}

%\iffalse
%</package>
%\fi
%
\endinput
|\\
|\childdocmain{}|\\
\end{tabular}
\end{center}
at the very top of the main \LaTeX{} file,
in particular \emph{before} the |\documentclass| statement!
The argument of |\childdocmain| should be left empty
(but it must be present).

%%%%%%%%%%%%%%%%%%%%%%%%%%%%%%%%%%%%%%%%
\DescribeMacro{\childdocof}
Furthermore, add the commands
\begin{center}
\begin{tabular}{l}
|% \iffalse
%
% childdoc.dtx Copyright (C) 2017-2018 Niklas Beisert
%
% This work may be distributed and/or modified under the
% conditions of the LaTeX Project Public License, either version 1.3
% of this license or (at your option) any later version.
% The latest version of this license is in
%   http://www.latex-project.org/lppl.txt
% and version 1.3 or later is part of all distributions of LaTeX
% version 2005/12/01 or later.
%
% This work has the LPPL maintenance status `maintained'.
%
% The Current Maintainer of this work is Niklas Beisert.
%
% This work consists of the files childdoc.dtx and childdoc.ins
% and the derived files childdoc.def and cdocsamp.tex with
% cdocsch1.tex, cdocsch2.tex, cdocsdrf.tex, cdocsfn1.tex, cdocsfn2.tex.
%
%<package>\ifdefined\childdocmain\endinput\fi
%<package>\ProvidesFile{childdoc.def}[2018/12/30 v2.0 child document driver]
%<samplemain>\ProvidesFile{cdocsamp.tex}[2018/12/30 v2.0 sample for childdoc]
%<*driver>
%\ProvidesFile{childdoc.drv}[2018/12/30 v2.0 childdoc reference manual file]
\PassOptionsToClass{10pt,a4paper}{article}
\documentclass{ltxdoc}

\usepackage[margin=35mm]{geometry}
\usepackage{hyperref}
\usepackage{hyperxmp}
\usepackage[usenames]{color}

\hypersetup{colorlinks=true}
\hypersetup{pdfstartview=FitH}
\hypersetup{pdfpagemode=UseNone}
\hypersetup{pdfsource={}}
\hypersetup{pdflang={en-UK}}
\hypersetup{pdfcopyright={Copyright 2017-2018 Niklas Beisert.
  This work may be distributed and/or modified under the
  conditions of the LaTeX Project Public License, either version 1.3
  of this license or (at your option) any later version.}}
\hypersetup{pdflicenseurl={http://www.latex-project.org/lppl.txt}}
\hypersetup{pdfcontactaddress={ETH Zurich, ITP, HIT K,
  Wolfgang-Pauli-Strasse 27}}
\hypersetup{pdfcontactpostcode={8093}}
\hypersetup{pdfcontactcity={Zurich}}
\hypersetup{pdfcontactcountry={Switzerland}}
\hypersetup{pdfcontactemail={nbeisert@itp.phys.ethz.ch}}
\hypersetup{pdfcontacturl={http://people.phys.ethz.ch/\xmptilde nbeisert/}}

\newcommand{\secref}[1]{\hyperref[#1]{section \ref*{#1}}}

\parskip1ex
\parindent0pt
\let\olditemize\itemize
\def\itemize{\olditemize\parskip0pt}

\begin{document}

\title{The \textsf{childdoc} Package}
\hypersetup{pdftitle={The childdoc Package}}
\author{Niklas Beisert\\[2ex]
  Institut f\"ur Theoretische Physik\\
  Eidgen\"ossische Technische Hochschule Z\"urich\\
  Wolfgang-Pauli-Strasse 27, 8093 Z\"urich, Switzerland\\[1ex]
  \href{mailto:nbeisert@itp.phys.ethz.ch}
  {\texttt{nbeisert@itp.phys.ethz.ch}}}
\hypersetup{pdfauthor={Niklas Beisert}}
\hypersetup{pdfsubject={Manual for the LaTeX2e Package childdoc}}
\date{30 December 2018, \textsf{v2.0}}
\maketitle

\begin{abstract}\noindent
\textsf{childdoc} is a \LaTeXe{} package
that enables the direct compilation
of document sections included by |\include|
to individual files.
\end{abstract}

\begingroup
\parskip0ex
\tableofcontents
\endgroup

%%%%%%%%%%%%%%%%%%%%%%%%%%%%%%%%%%%%%%%%%%%%%%%%%%%%%%%%%%%%%%%%%%%%%%%%%%%%%%%%
%%%%%%%%%%%%%%%%%%%%%%%%%%%%%%%%%%%%%%%%%%%%%%%%%%%%%%%%%%%%%%%%%%%%%%%%%%%%%%%%
\section{Introduction}

\LaTeX{} provides a mechanism to structure a large document (such as a book)
into a main file and several child files (containing the chapters)
using the |\include| command.
This mechanism is beneficial for documents
which span hundreds of pages in order to
make the source file(s) more manageable.
Moreover, compilation can be restricted to
selected child files by means of the |\includeonly| command.
The latter feature can be used to reduce the compilation time while editing
(this was significantly more useful in the earlier days of \LaTeX{})
or to generate a smaller document which is easier to navigate.
Another application of |\includeonly| is to generate
documents consisting of selected parts of the complete document.

However, there are a few drawbacks of the plain |\include| mechanism:
\begin{itemize}
\item
The child files cannot be compiled on their own,
they can only be compiled via the main file.
A naive editing environment
(such as a text editor with an option
to have the current file processed by \LaTeX)
may require one to switch to the main file before compiling;
attempting to compile the child file produces errors.
\item
The main file must be modified (each time)
to adjust the |\includeonly| command
to the present needs. This easily leaves the main file in a messy state.
\item
The generated document will always carry the filename
of the main document. This is inconvenient if
several child files are to be compiled and
to be kept for distribution.
\end{itemize}

The present package provides a simple interface
to make child files individually compilable by \LaTeX{}.
Compiling a child file then has the same effect as compiling
the main file with an |\includeonly| command
to select the appropriate child.
Moreover the generated document will carry the name of the child
rather than the main file.
This resolves all three above issues.

This feature is meant to make the editing of books,
thesis documents and lecture notes somewhat more convenient.
However, the package can also be used efficiently for
composing a series of documents (such as exercise sheets)
which are typically distributed individually.
It then assists the author in generating the individual documents
(potentially in different versions)
as well as a document containing the collected series.
Another application is in developing style files
or other kinds of included material
where compilation of the style file could redirect
to a sample or test file.

%%%%%%%%%%%%%%%%%%%%%%%%%%%%%%%%%%%%%%%%%%%%%%%%%%%%%%%%%%%%%%%%%%%%%%%%%%%%%%%%
%%%%%%%%%%%%%%%%%%%%%%%%%%%%%%%%%%%%%%%%%%%%%%%%%%%%%%%%%%%%%%%%%%%%%%%%%%%%%%%%
\section{Usage}

First of all, the package \textsf{childdoc} is \emph{not} a standard
\LaTeXe{} |.sty| style file! Therefore it needs to be invoked in
a non-standard way.

%%%%%%%%%%%%%%%%%%%%%%%%%%%%%%%%%%%%%%%%%%%%%%%%%%%%%%%%%%%%%%%%%%%%%%%%%%%%%%%%
\subsection{Included Files}
\label{sec:include}

%%%%%%%%%%%%%%%%%%%%%%%%%%%%%%%%%%%%%%%%
\DescribeMacro{\childdocmain}
To use the package, add the commands
\begin{center}
\begin{tabular}{l}
|\input{childdoc.def}|\\
|\childdocmain{}|\\
\end{tabular}
\end{center}
at the very top of the main \LaTeX{} file,
in particular \emph{before} the |\documentclass| statement!
The argument of |\childdocmain| should be left empty
(but it must be present).

%%%%%%%%%%%%%%%%%%%%%%%%%%%%%%%%%%%%%%%%
\DescribeMacro{\childdocof}
Furthermore, add the commands
\begin{center}
\begin{tabular}{l}
|\input{childdoc.def}|\\
|\childdocof{|\textit{main}|}|\\
\end{tabular}
\end{center}
at the top of every child file \textit{child}
which is included by |\include{|\textit{child}|}|
from within the main file
(or at least for those files to be compiled individually).
The argument \textit{main} must be the filename of the main file.

There are a couple of
considerations in setting up the main and child documents:

%%%%%%%%%%%%%%%%%%%%%%%%%%%%%%%%%%%%%%%%
\paragraph{Restrictions.}

Please note the following restrictions:
\begin{itemize}
\item
|\childdocmain| must be called with one argument \textit{main}
to ensure compatibility with earlier version of the package.
It must either be empty (|\childdocmain{}|)
or precisely match the filename of the main file in which it is specified.
See \secref{sec:detection} for further information.
\item
The filename \textit{main} must be specified without the |.tex| extension.
\item
The filename \textit{main} is case sensitive
(even in case-insensitive file systems)
due to internal string comparison.
\item
The argument \textit{main} should be fully expanded, it cannot be a macro.
\item
Subdirectories and special characters should be avoided in filenames.
\item
The command |\childdocmain{|\textit{main}|}| must be followed by a whitespace.
It should not be followed immediately by another command
or by a comment mark `|%|'.
This is because the \TeX{} parser reads the token immediately following
the argument of |\childdocmain| and puts it
at the beginning of every child section;
however, a white\-space is ignored.
\end{itemize}

%%%%%%%%%%%%%%%%%%%%%%%%%%%%%%%%%%%%%%%%
\paragraph{Content of Main File.}

It is advisable to place all content in the child files included by |\include|.
Any output contained in the main file will appear in all child documents
unless suppressed manually;
it cannot be suppressed automatically by the |\includeonly| directive
and thus should normally be avoided.
A method to include some content in the main file
by means of conditional processing is described in \secref{sec:conditional}.

%%%%%%%%%%%%%%%%%%%%%%%%%%%%%%%%%%%%%%%%
\paragraph{Page Numbering.}

When only a part of the document is compiled,
the appropriate numbering of pages
(as well as other status parameters)
is determined from the |.aux| files.
The latter contain information from previous passes.
However this information needs to propagate through
all intermediate child documents.
Therefore the page numbering in child documents may well
be inconsistent until the complete document is compiled at least once.

A useful (if unconventional) way to always ensure a consistent
page numbering is to restart the numbering in each child document
and denote the pages by `\textit{child}|.|\textit{page}'
where \textit{child} represents the chapter/section number of the child file.
This can be achieved by the command
|\numberwithin{page}{|\textit{child}|}|
of the \textsf{amsmath} package
where \textit{child} can be |chapter| or |section|
depending on the chosen structuring.
Alternatively, one can modify the macro |\thepage| appropriately
and reset the counter |page| at the start of each child file.

%%%%%%%%%%%%%%%%%%%%%%%%%%%%%%%%%%%%%%%%%%%%%%%%%%%%%%%%%%%%%%%%%%%%%%%%%%%%%%%%
\subsection{Conditional Processing}
\label{sec:conditional}

The package provides a mechanism to compile different versions
of a document. To customise the versions further some conditional processing
can come in handy to distinguish which version is being compiled.
The package provides two macros to describe the compilation context:

%%%%%%%%%%%%%%%%%%%%%%%%%%%%%%%%%%%%%%%%
\DescribeMacro{\ifchilddoc}
The conditional |\ifchilddoc| distinguishes between the compilation of
child documents and the main document:
%
\begin{center}
|\ifchilddoc |\textit{child-code}| |[|\||else |\textit{main-code}]| \||fi|
\end{center}

%%%%%%%%%%%%%%%%%%%%%%%%%%%%%%%%%%%%%%%%
\DescribeMacro{\childdocname}
\DescribeMacro{\childdocjob}
The macro |\childdocname| contains the filename (without extension)
of the main or child file being processed.
Note that |\childdocjob| will always contain the name of the main file.

%%%%%%%%%%%%%%%%%%%%%%%%%%%%%%%%%%%%%%%%
\paragraph{Title Page.}

Conditional processing can be used to include a title or banner page
in the main document when proper precautions are taken.
Importantly, the code in the main file should ensure that the page counter
(as well as other status parameters which are stored in the |.aux| files)
takes the same value after the conditional processing.
Otherwise the page numbers may take divergent values
depending on which part is compiled.

For example, a title page could be declared by:
%
\begin{center}
\begin{tabular}{l}
|\ifchilddoc\||else|\\
|\addtocounter{page}{-1}|\\
\textit{code for title page}\\
|\newpage|\\
|\||fi|
\end{tabular}
\end{center}
%
A banner page for the child documents can be generated by:
%
\begin{center}
\begin{tabular}{l}
|\ifchilddoc|\\
|\addtocounter{page}{-1}|\\
\textit{code for banner page}\\
|\newpage|\\
|\||fi|
\end{tabular}
\end{center}
%
Here one could write a message such as:
\begin{center}
|This is the part \childdocname{} of \childdocjob{}.|
\end{center}

%%%%%%%%%%%%%%%%%%%%%%%%%%%%%%%%%%%%%%%%%%%%%%%%%%%%%%%%%%%%%%%%%%%%%%%%%%%%%%%%
\subsection{Flags}
\label{sec:flags}

The package makes it easy to generate different versions
of the main or child documents.
To this end compilation flags can be defined
and assigned different default values.
They will be particularly useful in conjunction
with the forwarding mechanism described in \secref{sec:forward}.

For example, it may be useful to have a flag |\version|
which can be set to |draft| or |final|.
The document source will contain some conditional code
depending on the value of |\version|.
Suppose further, the flag should default to |final| for the main file
and to |draft| for child files
which is a natural assignment for editing the document.
This is achieved by placing the following code
in the preamble of the main document
(below the |\childdocmain| directive):
%
\begin{center}
\begin{tabular}{l}
|\ifchilddoc|\\
|\providecommand{\version}{draft}|\\
|\||else|\\
|\providecommand{\version}{final}|\\
|\||fi|
\end{tabular}
\end{center}
%
The definition by |\providecommand| makes sure
that previous definitions are not overwritten.
Further statements |\providecommand{\version}{...}|
can thus be added before the above code to override it.

For the main file, one might add a line
(between |\childdocmain| and the above block)
%
\begin{center}
|%\ifchilddoc\||else\providecommand{\version}{draft}\||fi|
\end{center}
%
which can be uncommented to produce a draft version.
Likewise one can add a line to the very top of a child file
(above the |\childdocof{|\textit{main}|}| directive)
%
\begin{center}
|%\providecommand{\version}{final}|
\end{center}
%
which can be uncommented to produce the final version of this child document.

%%%%%%%%%%%%%%%%%%%%%%%%%%%%%%%%%%%%%%%%%%%%%%%%%%%%%%%%%%%%%%%%%%%%%%%%%%%%%%%%
\subsection{Forwarding}
\label{sec:forward}

Different versions of the main or child documents
using compilation flags as described in \secref{sec:flags}
can be (permanently) stored in different files
for convenient compilation, viewing and distribution.
To this end, the package defines a command
to pass on compilation to a different file:

%%%%%%%%%%%%%%%%%%%%%%%%%%%%%%%%%%%%%%%%
\DescribeMacro{\childdocforward}
The command |\childdocforward| redirects processing to
another source file:
%
\begin{center}
\begin{tabular}{l}
|\input{childdoc.def}|\\
|\childdocforward[|\textit{main}|]{|\textit{dest}|}|\\
\end{tabular}
\end{center}
%
The argument \textit{dest} is the destination file
(without extension).
It should be the main file or one of the child files.
Note that further \textsf{childdoc} directives
such as |\childdocof| and |\childdocforward|
in the indicated file will be processed in this form.
The optional argument \textit{main}
passes on directly to the main file \textit{main}
while pretending to compile the child \textit{dest}.
This form behaves as if \textit{dest}
issues |\childdocof{|\textit{main}|}| right away,
and no further \textsf{childdoc} directives will be processed.

%%%%%%%%%%%%%%%%%%%%%%%%%%%%%%%%%%%%%%%%
\DescribeMacro{\...prefix}
In the alternative form |\childdocforwardprefix|,
%
\begin{center}
\begin{tabular}{l}
|\input{childdoc.def}|\\
|\childdocforwardprefix[|\textit{main}|]{|\textit{prefix}|}{|\textit{dest}|}|
\end{tabular}
\end{center}
%
the destination file is determined by a pattern
depending on the current file:
To make this work, the current file must be called
`{\textit{prefix}\hspace{0.2em}\textit{suffix}}'
with \textit{prefix} matching precisely the argument.
Processing is then passed on to the file
`{\textit{dest}\hspace{0.2em}\textit{suffix}}'.
Surely, the same effect is achieved by
directly specifying the
argument `{\textit{dest}\hspace{0.2em}\textit{suffix}}'
in the first form.
However, that requires to set up a different file
for each child. With the alternative form of the command
all these files can have exactly the same content
which simplifies setting them up and maintaining them.

For example, the following file |draft.tex|
with a compilation flag |\version| as described in \secref{sec:flags}
compiles the main document as a draft:
%
\begin{center}
\begin{tabular}{l}
|\def\version{draft}|\\
|\input{childdoc.def}|\\
|\childdocforward{|\textit{main}|}|
\end{tabular}
\end{center}
%
Likewise, the following files |final|\textit{nn}|.tex|
compile the final version of the child document
|child|\textit{nn}|.tex|:
%
\begin{center}
\begin{tabular}{l}
|\def\version{final}|\\
|\input{childdoc.def}|\\
|\childdocforwardprefix{final}{child}|
\end{tabular}
\end{center}
%

Note that when several versions of a main file and/or of each child file
are to be generated, it may be convenient to set up a |Makefile| or
shell script to automatise the process.

%%%%%%%%%%%%%%%%%%%%%%%%%%%%%%%%%%%%%%%%%%%%%%%%%%%%%%%%%%%%%%%%%%%%%%%%%%%%%%%%
\subsection{Command Line Processing}
\label{sec:commandline}

The effect of redirection files can also be achieved by invoking
the \LaTeX{} compiler with a more elaborate command line.
Most conveniently this should be done as part
of a shell script or a |Makefile|.

When using \textsf{childdoc} in the main file, the following
command lines effectively perform a redirection
(note that depending on the shell being used,
backslashes may have to be doubled: `|\|' $\to$ `|\\|'):
%
\begin{center}
|... -jobname "|\textit{target}|" |\\|"|[\textit{flags}]%
|\input{childdoc.def}\childdocforward[|\textit{main}|]{|\textit{dest}|}"|
\end{center}
%
Here \textit{target} is the name of the output file,
\textit{main} is the name of the main file
and \textit{dest} is the name of the main or child file to be processed
(all filenames without extensions).
The optional argument \textit{main} can be omitted
if \textit{main} matches \textit{dest}.
Optionally, compilation \textit{flags} can be defined via |\def| commands.
This command line makes the \TeX{} engine believe
it is compiling the file \textit{target}
whose content is specified as the latter parameter.
The provided code then forwards the processing to
\textit{main} or \textit{dest} as described in \secref{sec:forward}.

%%%%%%%%%%%%%%%%%%%%%%%%%%%%%%%%%%%%%%%%%%%%%%%%%%%%%%%%%%%%%%%%%%%%%%%%%%%%%%%%
\subsection{Include by Input}
\label{sec:input}

Including child documents by |\include| has some restrictions by design.
Most notably, the content of a child document always occupies
its own set of pages; pages cannot be shared between child documents.
Usually, this behaviour makes perfect sense
because each child document contain an essential part of the document.
However, in some situations it may be desirable to compose
a document from a collection of parts
without having mandatory page breaks between then.
For this case, the package
provides a mechanism to include parts
by |\input| which can also be processed individually.
However, by construction this mechanism
requires manual handling of the content to be output.

%%%%%%%%%%%%%%%%%%%%%%%%%%%%%%%%%%%%%%%%
\DescribeMacro{\ifchilddocmanual}
The main file should be prepared as usual, see \secref{sec:include}.
However, the document body must make a distinction
between processing of an individual part and of the main document, e.g.:
%
\begin{center}
\begin{tabular}{l}
|\ifchilddocmanual|\\
|\input{\childdocname}|\\
|\||else|\\
\textit{document body with }|\input{|\textit{part}|}|\\
|\||fi|
\end{tabular}
\end{center}
%
The conditional |\ifchilddocmanual| is true whenever
a part to be included by |\input| is being compiled,
and the name of the part is stored in |\childdocname|.

%%%%%%%%%%%%%%%%%%%%%%%%%%%%%%%%%%%%%%%%
\DescribeMacro{\childdocby}
Each part to be included by |\input| should start with:
%
\begin{center}
\begin{tabular}{l}
|\input{childdoc.def}|\\
|\childdocby{|\textit{main}|}|\\
\end{tabular}
\end{center}
%
The directive |\childdocby| is similar to |\childdocof|
described in \secref{sec:include},
but the subsequent selection of content must be done manually.
To that end, both |\ifchilddoc| and |\ifchilddocmanual|
will be true upon processing of a part,
and the name of the part is stored in |\childdocname|.
Note that |\jobname| will be set to the filename of the current part
so that each part receives an individual |.aux| file
that does not interfere with the |.aux| file(s) of the main document.
This behaviour can be altered by the alternative form
|\childdocby[*]{|\textit{main}|}| (with a non-empty optional argument)
which uses the |.aux| file of the main document
by setting |\jobname| to \textit{main}.

%%%%%%%%%%%%%%%%%%%%%%%%%%%%%%%%%%%%%%%%%%%%%%%%%%%%%%%%%%%%%%%%%%%%%%%%%%%%%%%%
\subsection{Driver Development}
\label{sec:driver}

The \textsf{childdoc} mechanism can also be use for the development
of definition files such as \LaTeX{} styles or classes.
This case differs from the above setup with multiple parts
included by |\include| in that no |\includeonly| should be invoked.
This can be achieved by starting the include file
(before |\ProvidesPackage|) with:
%
\begin{center}
\begin{tabular}{l}
|\input{childdoc.def}|\\
|\childdocforward{|\textit{main}|}|\\
\end{tabular}
\end{center}
%
or alternatively with:
%
\begin{center}
\begin{tabular}{l}
|\input{childdoc.def}|\\
|\childdocby{|\textit{main}|}|\\
\end{tabular}
\end{center}
%
Both forms have slightly different effects as described above.
The main file is prepared as usual, see \secref{sec:include}.

%%%%%%%%%%%%%%%%%%%%%%%%%%%%%%%%%%%%%%%%%%%%%%%%%%%%%%%%%%%%%%%%%%%%%%%%%%%%%%%%
\subsection{Legacy Detection}
\label{sec:detection}

The directive |\childdocmain| in the main file can detect
whether the complete document or merely a child is to be compiled
even without using the directive |\childdocof|.
This method is deprecated because it is less robust
and there is no compelling reason to use it;
it is merely provided for backward compatibility
and it may be removed in future versions.

If the detection mechanism is to be used,
it is mandatory to correctly specify
the filename of the main file as the argument of |\childdocmain|:
%
\begin{center}
\begin{tabular}{l}
|\input{childdoc.def}|\\
|\childdocmain{|\textit{main}|}|\\
\end{tabular}
\end{center}
%
If |\jobname| does not match the argument \textit{main} of |\childdocmain|,
it is assumed that |\jobname| points to the child file to be compiled.
When using |\childdocmain| with the main file specified as argument,
it suffices to start a child file
with just |\input{|\textit{main}|}|
without loading of the package and using |\childdocof|.
If instead all processing is done
with the appropriate \textsf{childdoc} directives,
the argument of \textit{main} of |\childdocmain| can be empty.

An alternative version of the command line processing described
in \secref{sec:commandline} using the detection mechanism reads:
%
\begin{center}
|... -jobname "|\textit{target}|" "|[\textit{flags}]%
[|\def\jobname{|\textit{dest}|}|]|\input{|\textit{main}|}"|
\end{center}

%%%%%%%%%%%%%%%%%%%%%%%%%%%%%%%%%%%%%%%%%%%%%%%%%%%%%%%%%%%%%%%%%%%%%%%%%%%%%%%%
\subsection{Manual Code}
\label{sec:manual}

In case one cannot be certain whether the definitions file |childdoc.def|
is installed on the target \TeX{} distribution
and one prefers not to ship it,
it is conceivable to paste a few relevant commands into the sources.

To that end, drop all statements |\input{childdoc.def}|
and perform the replacements as outlined below.
Instead of |\childdocmain{|\textit{main}|}| add the following code
to the top of the main file:
%
\begin{center}
\begin{tabular}{l}
|\||ifdefined\childdocname\endinput\||fi\newif\ifchilddoc|\\
|\edef\childdocname{\scantokens\expandafter{\jobname\noexpand}}|\\
|\def\childdocmain{|\textit{main}|}\||ifx\childdocmain\childdocname\||else|\\
|\childdoctrue\includeonly{\childdocname}\let\jobname\childdocmain\||fi|\\
\end{tabular}
\end{center}
%
Instead of |\childdocof{|\textit{main}|}| just include the main file
at the top of each child file:
%
\begin{center}
|\input{|\textit{main}|}|
\end{center}
%
A simple redirection |\childdocforward{|\textit{dest}|}| is achieved by:
%
\begin{center}
|\def\jobname{|\textit{dest}|}\input{\jobname}|
\end{center}
%
The redirection with prefix
|\childdocforwardprefix[|\textit{prefix}|]{|\textit{dest}|}|
is accomplished by:
%
\begin{center}
\begin{tabular}{l}
|{\edef\jobname{\scantokens\expandafter{\jobname\noexpand}}|\\
|\def\redirectjob |\textit{prefix}|#1~~~{\gdef\jobname{|\textit{dest}|#1}}|\\
|\expandafter\redirectjob\jobname~~~}\input{\jobname}|
\end{tabular}
\end{center}

In an alternative approach,
child documents can be compiled by a specific command line
without additional code or specific definitions:
%
\begin{center}
|... -jobname "|\textit{target}|" "|[\textit{flags}]%
|\includeonly{|\textit{dest}|}\input{|\textit{main}|}"|
\end{center}
%

%%%%%%%%%%%%%%%%%%%%%%%%%%%%%%%%%%%%%%%%%%%%%%%%%%%%%%%%%%%%%%%%%%%%%%%%%%%%%%%%
%%%%%%%%%%%%%%%%%%%%%%%%%%%%%%%%%%%%%%%%%%%%%%%%%%%%%%%%%%%%%%%%%%%%%%%%%%%%%%%%
\section{Information}

%%%%%%%%%%%%%%%%%%%%%%%%%%%%%%%%%%%%%%%%%%%%%%%%%%%%%%%%%%%%%%%%%%%%%%%%%%%%%%%%
\subsection{Copyright}

Copyright \copyright{} 2017--2018 Niklas Beisert

This work may be distributed and/or modified under the
conditions of the \LaTeX{} Project Public License, either version 1.3
of this license or (at your option) any later version.
The latest version of this license is in
  \url{http://www.latex-project.org/lppl.txt}
and version 1.3 or later is part of all distributions of \LaTeX{}
version 2005/12/01 or later.

This work has the LPPL maintenance status `maintained'.

The Current Maintainer of this work is Niklas Beisert.

This work consists of the files |README.txt|, |childdoc.ins| and |childdoc.dtx|
as well as the derived files |childdoc.def|, |cdocsamp.tex|
with |cdocsch1.tex|, |cdocsch2.tex|, |cdocspt3.tex|, |cdocspt4.tex|,
|cdocsdrf.tex|, |cdocsfn1.tex|, |cdocsfn2.tex|
as well as |childdoc.pdf|.

%%%%%%%%%%%%%%%%%%%%%%%%%%%%%%%%%%%%%%%%%%%%%%%%%%%%%%%%%%%%%%%%%%%%%%%%%%%%%%%%
\subsection{Files and Installation}

The package consists of the files:
%
\begin{center}
\begin{tabular}{ll}
    |README.txt|   & readme file \\
    |childdoc.ins| & installation file \\
    |childdoc.dtx| & source file \\
    |childdoc.def| & definition file \\
    |cdocsamp.tex| & sample main file \\
    |cdocsch1.tex| & sample include file \\
    |cdocsch2.tex| & sample include file \\
    |cdocspt3.tex| & sample part file \\
    |cdocspt4.tex| & sample part file \\
    |cdocsdrf.tex| & sample redirection file \\
    |cdocsfn1.tex| & sample redirection file \\
    |cdocsfn2.tex| & sample redirection file \\
    |childdoc.pdf| & manual
\end{tabular}
\end{center}
%
The distribution consists of the files
|README.txt|, |childdoc.ins| and |childdoc.dtx|.
%
\begin{itemize}
\item
Run (pdf)\LaTeX{} on |childdoc.dtx|
to compile the manual |childdoc.pdf| (this file).
\item
Run \LaTeX{} on |childdoc.ins| to create the definitions file |childdoc.def|
and the sample |cdocsamp.tex| with include files
|cdocsch1.tex|, |cdocsch2.tex|, |cdocspt3.tex|, |cdocspt4.tex|,
|cdocsdrf.tex|, |cdocsfn1.tex|, |cdocsfn2.tex|.
Then copy the file |childdoc.def| to an appropriate directory of your \LaTeX{}
distribution, e.g.\ \textit{texmf-root}|/tex/latex/childdoc|.
\end{itemize}

%%%%%%%%%%%%%%%%%%%%%%%%%%%%%%%%%%%%%%%%%%%%%%%%%%%%%%%%%%%%%%%%%%%%%%%%%%%%%%%%
\subsection{Related CTAN Packages}

There are several other packages which offer a similar functionality:
%
\begin{itemize}
\item
The packages
\href{http://ctan.org/pkg/docmute}{\textsf{docmute}},
\href{http://ctan.org/pkg/includex}{\textsf{includex}} and
\href{http://ctan.org/pkg/standalone}{\textsf{standalone}}
provide commands to include only the document body of
a child file thus allowing both files to be compiled individually.
\item
The packages \href{http://ctan.org/pkg/subdocs}{\textsf{subdocs}}
and \href{http://ctan.org/pkg/subfiles}{\textsf{subfiles}}
provide structures in which the main and child documents can be
encapsulated and allowing them to be compiled individually.
The inclusion mechanism is different from the conventional |\include|.
\item
The package \href{http://ctan.org/pkg/combine}{\textsf{combine}}
is an elaborate solution to combine several documents into one.
\end{itemize}
%
See also the CTAN topic \href{http://ctan.org/topic/subdocs}{\textsf{subdocs}}
for further related packages.
The present package differs from the above solutions in that
a document structure constructed with the conventional |\include| mechanism
just needs two extra commands at the top of every file
such that all constituent files can be compiled individually.

%%%%%%%%%%%%%%%%%%%%%%%%%%%%%%%%%%%%%%%%%%%%%%%%%%%%%%%%%%%%%%%%%%%%%%%%%%%%%%%%
%\subsection{Feature Suggestions}
%
%The following is a list of features which may be useful for future
%versions of this package:
%%
%\begin{itemize}
%\item
%\ldots
%\end{itemize}

%%%%%%%%%%%%%%%%%%%%%%%%%%%%%%%%%%%%%%%%%%%%%%%%%%%%%%%%%%%%%%%%%%%%%%%%%%%%%%%%
\subsection{Revision History}

%%%%%%%%%%%%%%%%%%%%%%%%%%%%%%%%%%%%%%%%
\paragraph{v2.0:} 2018/12/30

\begin{itemize}
\item
immediate forward processing
\item
added |\childdocby| mechanism
\item
manual restructured
\end{itemize}

%%%%%%%%%%%%%%%%%%%%%%%%%%%%%%%%%%%%%%%%
\paragraph{v1.6:} 2018/01/17

\begin{itemize}
\item
application for development of include files
\item
corrections to manual
\end{itemize}

%%%%%%%%%%%%%%%%%%%%%%%%%%%%%%%%%%%%%%%%
\paragraph{v1.5:} 2017/05/21

\begin{itemize}
\item
more complete structuring introduced
\item
|\childdocof| introduced
\item
|\childdoc| renamed to |\childdocmain|
\item
|\childredirect| renamed to |\childdocforward| and |\childdocforwardprefix|
and functionality expanded
\end{itemize}

%%%%%%%%%%%%%%%%%%%%%%%%%%%%%%%%%%%%%%%%
\paragraph{v1.0:} 2017/04/27

\begin{itemize}
\item
manual and install package
\item
first version published on CTAN
\end{itemize}

%%%%%%%%%%%%%%%%%%%%%%%%%%%%%%%%%%%%%%%%
\paragraph{v0.6:} 2017/04/26

\begin{itemize}
\item
redirection mechanism added
\end{itemize}

%%%%%%%%%%%%%%%%%%%%%%%%%%%%%%%%%%%%%%%%
\paragraph{v0.5:} 2017/04/26

\begin{itemize}
\item
functionality in definition file
\end{itemize}


%%%%%%%%%%%%%%%%%%%%%%%%%%%%%%%%%%%%%%%%%%%%%%%%%%%%%%%%%%%%%%%%%%%%%%%%%%%%%%%%
%%%%%%%%%%%%%%%%%%%%%%%%%%%%%%%%%%%%%%%%%%%%%%%%%%%%%%%%%%%%%%%%%%%%%%%%%%%%%%%%
%%%%%%%%%%%%%%%%%%%%%%%%%%%%%%%%%%%%%%%%%%%%%%%%%%%%%%%%%%%%%%%%%%%%%%%%%%%%%%%%
\appendix

\settowidth\MacroIndent{\rmfamily\scriptsize 000\ }

 \DocInput{childdoc.dtx}

\end{document}
%</driver>
% \fi
%
% %%%%%%%%%%%%%%%%%%%%%%%%%%%%%%%%%%%%%%%%%%%%%%%%%%%%%%%%%%%%%%%%%%%%%%%%%%%%%%
% %%%%%%%%%%%%%%%%%%%%%%%%%%%%%%%%%%%%%%%%%%%%%%%%%%%%%%%%%%%%%%%%%%%%%%%%%%%%%%
% \section{Sample}
%\iffalse
%<*samplemain>
%\fi
%
% The following presents a sample document
% with two chapters, two parts, a title page,
% a compile flag as well as three forwarding files to set the flag.
% It consists of eight |.tex| files:
% \begin{center}
% \begin{tabular}{ll}
% |cdocsamp.tex|&main file\\
% |cdocsch1.tex|&include file for chapter 1\\
% |cdocsch2.tex|&include file for chapter 2\\
% |cdocspt3.tex|&include file for part 3\\
% |cdocspt4.tex|&include file for part 4\\
% |cdocsdrf.tex|&forwarding file for main file in draft mode\\
% |cdocsfi1.tex|&forwarding file for final version of chapter 1\\
% |cdocsfi2.tex|&forwarding file for final version of chapter 2\\
% \end{tabular}
% \end{center}
% Each of the eight files can be compiled directly by the \LaTeX{} compiler.
%
% %%%%%%%%%%%%%%%%%%%%%%%%%%%%%%%%%%%%%%
% \paragraph{Main File.}
%
% The main file is called |cdocsamp.tex|.
%
% Load the \textsf{childdoc} definitions and
% declare the filename for the main document:
%    \begin{macrocode}
\input{childdoc.def}
\childdocmain{}
%    \end{macrocode}

% Optional override for |\version| flag:
%    \begin{macrocode}
%%\ifchilddoc\else\providecommand{\version}{draft}\fi
%    \end{macrocode}

% Define the default values for the |\version| flag
% (|final| for the main file and |draft| for childs):
%    \begin{macrocode}
\ifchilddoc
\providecommand{\version}{draft}
\else
\providecommand{\version}{final}
\fi
%    \end{macrocode}

% Load the standard document class:
%    \begin{macrocode}
\documentclass[12pt]{article}
%    \end{macrocode}

% Start the document body:
%    \begin{macrocode}
\begin{document}
%    \end{macrocode}

% Declare a title page.
% Print title, part of document being processed and version flag:
%    \begin{macrocode}
\addtocounter{page}{-1}
\begin{center}
{\LARGE\bfseries{}childdoc example\par}
\vspace{1cm}
\ifchilddoc
\ifchilddocmanual part\else chapter\fi:
`\childdocname' of `\childdocjob'\par
\else
main document: `\childdocjob'\par
\fi
version: \version\par
\end{center}
\newpage
%    \end{macrocode}

% Manually include selected file,
% otherwise process as usual:
%    \begin{macrocode}
\ifchilddocmanual
\section*{part `\childdocname'}
\input{\childdocname}
\else
%    \end{macrocode}

% Include the two chapters:
%    \begin{macrocode}
\include{cdocsch1}
\include{cdocsch2}
%    \end{macrocode}

% Include the two parts unless only chapters should be displayed:
%    \begin{macrocode}
\ifchilddoc\else
\section{part three}
\input{cdocspt3}
\section{part four}
\input{cdocspt4}
\fi
%    \end{macrocode}

% Process as usual until here:
%    \begin{macrocode}
\fi
%    \end{macrocode}

% End of document body:
%    \begin{macrocode}
\end{document}
%    \end{macrocode}
%\iffalse
%</samplemain>
%\fi
%
% %%%%%%%%%%%%%%%%%%%%%%%%%%%%%%%%%%%%%%
% \paragraph{Chapter Include Files.}
%
% The include files are called |cdocsch1.tex| and |cdocsch2.tex|.
%
%\iffalse
%<*samplechap1|samplechap2>
%\fi

% Optional override for |\version| flag:
%    \begin{macrocode}
%%\providecommand{\version}{final}
%    \end{macrocode}

% Include the main document:
%    \begin{macrocode}
\input{childdoc.def}
\childdocof{cdocsamp}
%    \end{macrocode}

%\iffalse
%</samplechap1|samplechap2>
%\fi
%
%\iffalse
%<*samplechap1>
%\fi
% Some text for chapter 1:
%    \begin{macrocode}
\section{one}
some text in chapter one
%    \end{macrocode}

%\iffalse
%</samplechap1>
%\fi
% Some text for chapter 2:
%\iffalse
%<*samplechap2>
%\fi
%    \begin{macrocode}
\section{two}
more text in chapter two
%    \end{macrocode}

%\iffalse
%</samplechap2>
%\fi
%
% %%%%%%%%%%%%%%%%%%%%%%%%%%%%%%%%%%%%%%
% \paragraph{Part Include Files.}
%
% The include files are called |cdocspt3.tex| and |cdocspt4.tex|.
%
%\iffalse
%<*samplepart3|samplepart4>
%\fi

% Optional override for |\version| flag:
%    \begin{macrocode}
%%\providecommand{\version}{final}
%    \end{macrocode}

% Include the main document:
%    \begin{macrocode}
\input{childdoc.def}
\childdocby{cdocsamp}
%    \end{macrocode}

%\iffalse
%</samplepart3|samplepart4>
%\fi
%
%\iffalse
%<*samplepart3>
%\fi
% Some text for part 3:
%    \begin{macrocode}
some text in part three
%    \end{macrocode}

%\iffalse
%</samplepart3>
%\fi
% Some text for part 4:
%\iffalse
%<*samplepart4>
%\fi
%    \begin{macrocode}
more text in part four
%    \end{macrocode}

%\iffalse
%</samplepart4>
%\fi
%
% %%%%%%%%%%%%%%%%%%%%%%%%%%%%%%%%%%%%%%
% \paragraph{Forwarding for a Complete Draft.}
%
% The following forwarding file |cdocsdrf.tex|
% compiles the main document in draft mode:
%\iffalse
%<*sampledraft>
%\fi
%    \begin{macrocode}
\def\version{draft}
\input{childdoc.def}
\childdocforward{cdocsamp}
%    \end{macrocode}

%\iffalse
%</sampledraft>
%\fi
%
% %%%%%%%%%%%%%%%%%%%%%%%%%%%%%%%%%%%%%%
% \paragraph{Forwarding for Final Version of the Chapters.}
%
% The following forwarding files |cdocsfn1.tex| and |cdocsfn2.tex|
% (with identical content)
% compile the final versions of the child documents
% |cdocsch1.tex| and |cdocsch2.tex|, respectively:
%\iffalse
%<*samplefinal>
%\fi
%    \begin{macrocode}
\def\version{final}
\input{childdoc.def}
\childdocforwardprefix[cdocsamp]{cdocsfn}{cdocsch}
%    \end{macrocode}

%\iffalse
%</samplefinal>
%\fi
%
% %%%%%%%%%%%%%%%%%%%%%%%%%%%%%%%%%%%%%%
% \paragraph{Command Line Processing.}
%
% The following three command lines generate the output files
% |cdocscld|, |cdocscl1| and |cdocscl2|
% which should be identical to
% |cdocsdrf|, |cdocsch1| and |cdocsfn2|, respectively:
% \begin{center}
% \begin{tabular}{l}
% |latex -jobname cdocscld \|\\
% |  "\def\version{draft}\input{childdoc.def}\childdocforward{cdocsamp}"|\\
% |latex -jobname cdocscl1 \|\\
% |  "\input{childdoc.def}\childdocforward[cdocsamp]{cdocsch1}"|\\
% |latex -jobname cdocscl2 \|\\
% |  "\def\version{final}\input{childdoc.def}\childdocforward{cdocsch2}"|
% \end{tabular}
% \end{center}
% Note that the trailing backslash on each first line
% merely continues the input to the second line
% (for convenient cut ant paste).
% Furthermore, the command |latex| can be replaced by any
% of its alternative versions such as |pdflatex|.
%
% %%%%%%%%%%%%%%%%%%%%%%%%%%%%%%%%%%%%%%%%%%%%%%%%%%%%%%%%%%%%%%%%%%%%%%%%%%%%%%
% %%%%%%%%%%%%%%%%%%%%%%%%%%%%%%%%%%%%%%%%%%%%%%%%%%%%%%%%%%%%%%%%%%%%%%%%%%%%%%
% \section{Implementation}
%\iffalse
%<*package>
%\fi
%
% This section describes the definitions file |childdoc.def|.

% The definitions cannot be loaded using |\usepackage| or |\RequirePackage|
% which has a mechanism to prevent loading a style file more than once.
% When loading the definitions by means of |\input|
% multiple instances have to be prevented manually:
%\iffalse
%This code needs to be before the `\ProvidesFile' directive
%which is defined at the beginning of this file.
%Therefore it is also placed there and commented out here.
%</package>
%<*discard>
%\fi
%    \begin{macrocode}
\ifdefined\childdocmain\endinput\fi
%    \end{macrocode}
%\iffalse
%</discard>
%<*package>
%\fi
%
% \macro{\ifchilddoc}
% \macro{\ifchilddocmanual}
% The conditional |\ifchilddoc| tells whether a
% child (true) or main (false) document is being compiled.
% The conditional |\ifchilddocmanual| tells whether
% the |\includeonly| mechanism is used (false) or
% the selection of child files must be performed manually (true).
% The definitions initialise to false:
%    \begin{macrocode}
\newif\ifchilddoc
\newif\ifchilddocmanual
%    \end{macrocode}

% \macro{\childdocname}
% \macro{\childdocjob}
% The macro |\childdocname| stores the name of the main document
% to be compiled. The macro |\childdocjob| stores the name of
% the document on which the \LaTeX{} compiler was originally invoked.
% The content of |\jobname| cannot be compared
% to filenames specified in the source due to different catcodes.
% The following code rescans |\jobname|, stores the result
% in |\childdocname| and saves a copy in |\childdocjob|:
%    \begin{macrocode}
\edef\childdocname{\scantokens\expandafter{\jobname\noexpand}}
\let\childdocjob\childdocname
%    \end{macrocode}

% \macro{\childdocdisable}
% The macro |\childdocdisable| prevents the main file
% from being processed more than once.
% At this stage, the main document command |\childdocmain|
% is assumed to be called once again where it should do nothing.
% Any subsequent call to it should prevent
% a secondary processing of the main document
% It overwrites the forwarding commands
% |\childdocof| and |\childdocforward|
% with empty macros to prevent further inclusions of the main document:
%    \begin{macrocode}
\newcommand{\childdocdisable}
{
  \renewcommand{\childdocmain}[1]{\renewcommand{\childdocmain}[1]{\endinput}}
  \renewcommand{\childdocof}[1]{}
  \renewcommand{\childdocby}[2][]{}
  \renewcommand{\childdocforward}[2][]{}
  \renewcommand{\childdocdisable}{}
}
%    \end{macrocode}

% \macro{\childdocmain}
% The macro |\childdocmain| is to be called at the top of the main file
% with nothing or the main filename (without extension) as argument.
% First, it breaks loops.
% If the argument is not empty and does not match |\childdocname|
% (which is set by the first inclusion of |childdoc.def|),
% |\ifchilddoc| is set to true, |\includeonly| is applied to the child file
% and |\jobname| is set to the main file
% (for proper handling of |.aux| files):
%    \begin{macrocode}
\newcommand{\childdocmain}[1]
{
  \childdocdisable\childdocmain{}
  \if?#1?\else
    \begingroup
      \def\childdoctmp{#1}
      \ifx\childdoctmp\childdocname
        \def\childdoctmp{}
      \else
        \def\childdoctmp
        {
          \childdoctrue
          \includeonly{\childdocname}
          \def\childdocjob{#1}
          \def\jobname{#1}
        }
      \fi
      \expandafter
    \endgroup
    \childdoctmp
  \fi
}
%    \end{macrocode}

% \macro{\childdocof}
% The command |\childdocof| redirects
% compilation to the main file |#1|.
%    \begin{macrocode}
\newcommand{\childdocof}[1]
{
  \childdocdisable
  \childdoctrue
  \includeonly{\childdocname}
  \def\jobname{#1}
  \def\childdocjob{#1}
  \input{#1}
}
%    \end{macrocode}

% \macro{\childdocby}
% The command |\childdocby| ....
%    \begin{macrocode}
\newcommand{\childdocby}[2][]
{
  \childdocdisable
  \childdoctrue
  \childdocmanualtrue
  \if?#1?\else
    \def\jobname{#2}
  \fi
  \def\childdocjob{#2}
  \input{#2}
  \endinput
}
%    \end{macrocode}

% \macro{\childdocforward}
% The command |\childdocforward| redirects
% compilation to the main file or
% (if the optional argument is given) a child file.
% Parameters are set as if the main file
% or a child file starting with |\childdocof| was compiled.
% Then compilation is handed over to the main file:
%    \begin{macrocode}
\newcommand{\childdocforward}[2][]
{
  \begingroup
    \if?#1?
      \def\childdoctmp
      {
        \def\childdocname{#2}
        \def\childdocjob{#2}
        \def\jobname{#2}
        \input{#2}
        \endinput
      }
    \else
      \def\childdoctmp
      {
        \childdocdisable
        \def\childdocname{#2}
        \childdoctrue
        \includeonly{#2}
        \def\childdocjob{#1}
        \def\jobname{#1}
        \input{#1}
        \endinput
      }
    \fi
    \expandafter
  \endgroup
  \childdoctmp
}
%    \end{macrocode}

% \macro{\childdocforwardprefix}
% The command |\childdocforwardprefix| redirects
% compilation to the main or a child file by means of a pattern.
% The prefix |#1| in the current filename is replaced by |#2|
% and the suffix of the current filename is kept
% (it is assumed that the filename does not contain the substring `|~~~|'
% which is used as a delimiter).
% Compilation is handed over to the new file by |\childdocforward|:
%    \begin{macrocode}
\newcommand{\childdocforwardprefix}[3][]
{
  \begingroup
    \def\childdocextract #2##1~~~{\def\childdoctmp{\childdocforward[#1]{#3##1}}}
    \expandafter\childdocextract\childdocname~~~
    \expandafter
  \endgroup
  \childdoctmp
}
%    \end{macrocode}

% \macro{\childdoc}
% The deprecated macro |\childdoc| is a legacy version of |\childdocmain|:
%    \begin{macrocode}
\newcommand{\childdoc}{\childdocmain}
%    \end{macrocode}

% \macro{\childdocredirect}
% The deprecated macro |\childdocredirect| is a legacy version
% of |\childdocforward| and |\childdocforwardprefix|:
%    \begin{macrocode}
\newcommand{\childdocredirect}[2][]
{
  \begingroup
    \if?#1?
      \def\childdoctmp{\childdocforward{#2}}
    \else
      \def\childdoctmp{\childdocforwardprefix{#1}{#2}}
    \fi
    \expandafter
  \endgroup
  \childdoctmp
}
%    \end{macrocode}

%\iffalse
%</package>
%\fi
%
\endinput
|\\
|\childdocof{|\textit{main}|}|\\
\end{tabular}
\end{center}
at the top of every child file \textit{child}
which is included by |\include{|\textit{child}|}|
from within the main file
(or at least for those files to be compiled individually).
The argument \textit{main} must be the filename of the main file.

There are a couple of
considerations in setting up the main and child documents:

%%%%%%%%%%%%%%%%%%%%%%%%%%%%%%%%%%%%%%%%
\paragraph{Restrictions.}

Please note the following restrictions:
\begin{itemize}
\item
|\childdocmain| must be called with one argument \textit{main}
to ensure compatibility with earlier version of the package.
It must either be empty (|\childdocmain{}|)
or precisely match the filename of the main file in which it is specified.
See \secref{sec:detection} for further information.
\item
The filename \textit{main} must be specified without the |.tex| extension.
\item
The filename \textit{main} is case sensitive
(even in case-insensitive file systems)
due to internal string comparison.
\item
The argument \textit{main} should be fully expanded, it cannot be a macro.
\item
Subdirectories and special characters should be avoided in filenames.
\item
The command |\childdocmain{|\textit{main}|}| must be followed by a whitespace.
It should not be followed immediately by another command
or by a comment mark `|%|'.
This is because the \TeX{} parser reads the token immediately following
the argument of |\childdocmain| and puts it
at the beginning of every child section;
however, a white\-space is ignored.
\end{itemize}

%%%%%%%%%%%%%%%%%%%%%%%%%%%%%%%%%%%%%%%%
\paragraph{Content of Main File.}

It is advisable to place all content in the child files included by |\include|.
Any output contained in the main file will appear in all child documents
unless suppressed manually;
it cannot be suppressed automatically by the |\includeonly| directive
and thus should normally be avoided.
A method to include some content in the main file
by means of conditional processing is described in \secref{sec:conditional}.

%%%%%%%%%%%%%%%%%%%%%%%%%%%%%%%%%%%%%%%%
\paragraph{Page Numbering.}

When only a part of the document is compiled,
the appropriate numbering of pages
(as well as other status parameters)
is determined from the |.aux| files.
The latter contain information from previous passes.
However this information needs to propagate through
all intermediate child documents.
Therefore the page numbering in child documents may well
be inconsistent until the complete document is compiled at least once.

A useful (if unconventional) way to always ensure a consistent
page numbering is to restart the numbering in each child document
and denote the pages by `\textit{child}|.|\textit{page}'
where \textit{child} represents the chapter/section number of the child file.
This can be achieved by the command
|\numberwithin{page}{|\textit{child}|}|
of the \textsf{amsmath} package
where \textit{child} can be |chapter| or |section|
depending on the chosen structuring.
Alternatively, one can modify the macro |\thepage| appropriately
and reset the counter |page| at the start of each child file.

%%%%%%%%%%%%%%%%%%%%%%%%%%%%%%%%%%%%%%%%%%%%%%%%%%%%%%%%%%%%%%%%%%%%%%%%%%%%%%%%
\subsection{Conditional Processing}
\label{sec:conditional}

The package provides a mechanism to compile different versions
of a document. To customise the versions further some conditional processing
can come in handy to distinguish which version is being compiled.
The package provides two macros to describe the compilation context:

%%%%%%%%%%%%%%%%%%%%%%%%%%%%%%%%%%%%%%%%
\DescribeMacro{\ifchilddoc}
The conditional |\ifchilddoc| distinguishes between the compilation of
child documents and the main document:
%
\begin{center}
|\ifchilddoc |\textit{child-code}| |[|\||else |\textit{main-code}]| \||fi|
\end{center}

%%%%%%%%%%%%%%%%%%%%%%%%%%%%%%%%%%%%%%%%
\DescribeMacro{\childdocname}
\DescribeMacro{\childdocjob}
The macro |\childdocname| contains the filename (without extension)
of the main or child file being processed.
Note that |\childdocjob| will always contain the name of the main file.

%%%%%%%%%%%%%%%%%%%%%%%%%%%%%%%%%%%%%%%%
\paragraph{Title Page.}

Conditional processing can be used to include a title or banner page
in the main document when proper precautions are taken.
Importantly, the code in the main file should ensure that the page counter
(as well as other status parameters which are stored in the |.aux| files)
takes the same value after the conditional processing.
Otherwise the page numbers may take divergent values
depending on which part is compiled.

For example, a title page could be declared by:
%
\begin{center}
\begin{tabular}{l}
|\ifchilddoc\||else|\\
|\addtocounter{page}{-1}|\\
\textit{code for title page}\\
|\newpage|\\
|\||fi|
\end{tabular}
\end{center}
%
A banner page for the child documents can be generated by:
%
\begin{center}
\begin{tabular}{l}
|\ifchilddoc|\\
|\addtocounter{page}{-1}|\\
\textit{code for banner page}\\
|\newpage|\\
|\||fi|
\end{tabular}
\end{center}
%
Here one could write a message such as:
\begin{center}
|This is the part \childdocname{} of \childdocjob{}.|
\end{center}

%%%%%%%%%%%%%%%%%%%%%%%%%%%%%%%%%%%%%%%%%%%%%%%%%%%%%%%%%%%%%%%%%%%%%%%%%%%%%%%%
\subsection{Flags}
\label{sec:flags}

The package makes it easy to generate different versions
of the main or child documents.
To this end compilation flags can be defined
and assigned different default values.
They will be particularly useful in conjunction
with the forwarding mechanism described in \secref{sec:forward}.

For example, it may be useful to have a flag |\version|
which can be set to |draft| or |final|.
The document source will contain some conditional code
depending on the value of |\version|.
Suppose further, the flag should default to |final| for the main file
and to |draft| for child files
which is a natural assignment for editing the document.
This is achieved by placing the following code
in the preamble of the main document
(below the |\childdocmain| directive):
%
\begin{center}
\begin{tabular}{l}
|\ifchilddoc|\\
|\providecommand{\version}{draft}|\\
|\||else|\\
|\providecommand{\version}{final}|\\
|\||fi|
\end{tabular}
\end{center}
%
The definition by |\providecommand| makes sure
that previous definitions are not overwritten.
Further statements |\providecommand{\version}{...}|
can thus be added before the above code to override it.

For the main file, one might add a line
(between |\childdocmain| and the above block)
%
\begin{center}
|%\ifchilddoc\||else\providecommand{\version}{draft}\||fi|
\end{center}
%
which can be uncommented to produce a draft version.
Likewise one can add a line to the very top of a child file
(above the |\childdocof{|\textit{main}|}| directive)
%
\begin{center}
|%\providecommand{\version}{final}|
\end{center}
%
which can be uncommented to produce the final version of this child document.

%%%%%%%%%%%%%%%%%%%%%%%%%%%%%%%%%%%%%%%%%%%%%%%%%%%%%%%%%%%%%%%%%%%%%%%%%%%%%%%%
\subsection{Forwarding}
\label{sec:forward}

Different versions of the main or child documents
using compilation flags as described in \secref{sec:flags}
can be (permanently) stored in different files
for convenient compilation, viewing and distribution.
To this end, the package defines a command
to pass on compilation to a different file:

%%%%%%%%%%%%%%%%%%%%%%%%%%%%%%%%%%%%%%%%
\DescribeMacro{\childdocforward}
The command |\childdocforward| redirects processing to
another source file:
%
\begin{center}
\begin{tabular}{l}
|% \iffalse
%
% childdoc.dtx Copyright (C) 2017-2018 Niklas Beisert
%
% This work may be distributed and/or modified under the
% conditions of the LaTeX Project Public License, either version 1.3
% of this license or (at your option) any later version.
% The latest version of this license is in
%   http://www.latex-project.org/lppl.txt
% and version 1.3 or later is part of all distributions of LaTeX
% version 2005/12/01 or later.
%
% This work has the LPPL maintenance status `maintained'.
%
% The Current Maintainer of this work is Niklas Beisert.
%
% This work consists of the files childdoc.dtx and childdoc.ins
% and the derived files childdoc.def and cdocsamp.tex with
% cdocsch1.tex, cdocsch2.tex, cdocsdrf.tex, cdocsfn1.tex, cdocsfn2.tex.
%
%<package>\ifdefined\childdocmain\endinput\fi
%<package>\ProvidesFile{childdoc.def}[2018/12/30 v2.0 child document driver]
%<samplemain>\ProvidesFile{cdocsamp.tex}[2018/12/30 v2.0 sample for childdoc]
%<*driver>
%\ProvidesFile{childdoc.drv}[2018/12/30 v2.0 childdoc reference manual file]
\PassOptionsToClass{10pt,a4paper}{article}
\documentclass{ltxdoc}

\usepackage[margin=35mm]{geometry}
\usepackage{hyperref}
\usepackage{hyperxmp}
\usepackage[usenames]{color}

\hypersetup{colorlinks=true}
\hypersetup{pdfstartview=FitH}
\hypersetup{pdfpagemode=UseNone}
\hypersetup{pdfsource={}}
\hypersetup{pdflang={en-UK}}
\hypersetup{pdfcopyright={Copyright 2017-2018 Niklas Beisert.
  This work may be distributed and/or modified under the
  conditions of the LaTeX Project Public License, either version 1.3
  of this license or (at your option) any later version.}}
\hypersetup{pdflicenseurl={http://www.latex-project.org/lppl.txt}}
\hypersetup{pdfcontactaddress={ETH Zurich, ITP, HIT K,
  Wolfgang-Pauli-Strasse 27}}
\hypersetup{pdfcontactpostcode={8093}}
\hypersetup{pdfcontactcity={Zurich}}
\hypersetup{pdfcontactcountry={Switzerland}}
\hypersetup{pdfcontactemail={nbeisert@itp.phys.ethz.ch}}
\hypersetup{pdfcontacturl={http://people.phys.ethz.ch/\xmptilde nbeisert/}}

\newcommand{\secref}[1]{\hyperref[#1]{section \ref*{#1}}}

\parskip1ex
\parindent0pt
\let\olditemize\itemize
\def\itemize{\olditemize\parskip0pt}

\begin{document}

\title{The \textsf{childdoc} Package}
\hypersetup{pdftitle={The childdoc Package}}
\author{Niklas Beisert\\[2ex]
  Institut f\"ur Theoretische Physik\\
  Eidgen\"ossische Technische Hochschule Z\"urich\\
  Wolfgang-Pauli-Strasse 27, 8093 Z\"urich, Switzerland\\[1ex]
  \href{mailto:nbeisert@itp.phys.ethz.ch}
  {\texttt{nbeisert@itp.phys.ethz.ch}}}
\hypersetup{pdfauthor={Niklas Beisert}}
\hypersetup{pdfsubject={Manual for the LaTeX2e Package childdoc}}
\date{30 December 2018, \textsf{v2.0}}
\maketitle

\begin{abstract}\noindent
\textsf{childdoc} is a \LaTeXe{} package
that enables the direct compilation
of document sections included by |\include|
to individual files.
\end{abstract}

\begingroup
\parskip0ex
\tableofcontents
\endgroup

%%%%%%%%%%%%%%%%%%%%%%%%%%%%%%%%%%%%%%%%%%%%%%%%%%%%%%%%%%%%%%%%%%%%%%%%%%%%%%%%
%%%%%%%%%%%%%%%%%%%%%%%%%%%%%%%%%%%%%%%%%%%%%%%%%%%%%%%%%%%%%%%%%%%%%%%%%%%%%%%%
\section{Introduction}

\LaTeX{} provides a mechanism to structure a large document (such as a book)
into a main file and several child files (containing the chapters)
using the |\include| command.
This mechanism is beneficial for documents
which span hundreds of pages in order to
make the source file(s) more manageable.
Moreover, compilation can be restricted to
selected child files by means of the |\includeonly| command.
The latter feature can be used to reduce the compilation time while editing
(this was significantly more useful in the earlier days of \LaTeX{})
or to generate a smaller document which is easier to navigate.
Another application of |\includeonly| is to generate
documents consisting of selected parts of the complete document.

However, there are a few drawbacks of the plain |\include| mechanism:
\begin{itemize}
\item
The child files cannot be compiled on their own,
they can only be compiled via the main file.
A naive editing environment
(such as a text editor with an option
to have the current file processed by \LaTeX)
may require one to switch to the main file before compiling;
attempting to compile the child file produces errors.
\item
The main file must be modified (each time)
to adjust the |\includeonly| command
to the present needs. This easily leaves the main file in a messy state.
\item
The generated document will always carry the filename
of the main document. This is inconvenient if
several child files are to be compiled and
to be kept for distribution.
\end{itemize}

The present package provides a simple interface
to make child files individually compilable by \LaTeX{}.
Compiling a child file then has the same effect as compiling
the main file with an |\includeonly| command
to select the appropriate child.
Moreover the generated document will carry the name of the child
rather than the main file.
This resolves all three above issues.

This feature is meant to make the editing of books,
thesis documents and lecture notes somewhat more convenient.
However, the package can also be used efficiently for
composing a series of documents (such as exercise sheets)
which are typically distributed individually.
It then assists the author in generating the individual documents
(potentially in different versions)
as well as a document containing the collected series.
Another application is in developing style files
or other kinds of included material
where compilation of the style file could redirect
to a sample or test file.

%%%%%%%%%%%%%%%%%%%%%%%%%%%%%%%%%%%%%%%%%%%%%%%%%%%%%%%%%%%%%%%%%%%%%%%%%%%%%%%%
%%%%%%%%%%%%%%%%%%%%%%%%%%%%%%%%%%%%%%%%%%%%%%%%%%%%%%%%%%%%%%%%%%%%%%%%%%%%%%%%
\section{Usage}

First of all, the package \textsf{childdoc} is \emph{not} a standard
\LaTeXe{} |.sty| style file! Therefore it needs to be invoked in
a non-standard way.

%%%%%%%%%%%%%%%%%%%%%%%%%%%%%%%%%%%%%%%%%%%%%%%%%%%%%%%%%%%%%%%%%%%%%%%%%%%%%%%%
\subsection{Included Files}
\label{sec:include}

%%%%%%%%%%%%%%%%%%%%%%%%%%%%%%%%%%%%%%%%
\DescribeMacro{\childdocmain}
To use the package, add the commands
\begin{center}
\begin{tabular}{l}
|\input{childdoc.def}|\\
|\childdocmain{}|\\
\end{tabular}
\end{center}
at the very top of the main \LaTeX{} file,
in particular \emph{before} the |\documentclass| statement!
The argument of |\childdocmain| should be left empty
(but it must be present).

%%%%%%%%%%%%%%%%%%%%%%%%%%%%%%%%%%%%%%%%
\DescribeMacro{\childdocof}
Furthermore, add the commands
\begin{center}
\begin{tabular}{l}
|\input{childdoc.def}|\\
|\childdocof{|\textit{main}|}|\\
\end{tabular}
\end{center}
at the top of every child file \textit{child}
which is included by |\include{|\textit{child}|}|
from within the main file
(or at least for those files to be compiled individually).
The argument \textit{main} must be the filename of the main file.

There are a couple of
considerations in setting up the main and child documents:

%%%%%%%%%%%%%%%%%%%%%%%%%%%%%%%%%%%%%%%%
\paragraph{Restrictions.}

Please note the following restrictions:
\begin{itemize}
\item
|\childdocmain| must be called with one argument \textit{main}
to ensure compatibility with earlier version of the package.
It must either be empty (|\childdocmain{}|)
or precisely match the filename of the main file in which it is specified.
See \secref{sec:detection} for further information.
\item
The filename \textit{main} must be specified without the |.tex| extension.
\item
The filename \textit{main} is case sensitive
(even in case-insensitive file systems)
due to internal string comparison.
\item
The argument \textit{main} should be fully expanded, it cannot be a macro.
\item
Subdirectories and special characters should be avoided in filenames.
\item
The command |\childdocmain{|\textit{main}|}| must be followed by a whitespace.
It should not be followed immediately by another command
or by a comment mark `|%|'.
This is because the \TeX{} parser reads the token immediately following
the argument of |\childdocmain| and puts it
at the beginning of every child section;
however, a white\-space is ignored.
\end{itemize}

%%%%%%%%%%%%%%%%%%%%%%%%%%%%%%%%%%%%%%%%
\paragraph{Content of Main File.}

It is advisable to place all content in the child files included by |\include|.
Any output contained in the main file will appear in all child documents
unless suppressed manually;
it cannot be suppressed automatically by the |\includeonly| directive
and thus should normally be avoided.
A method to include some content in the main file
by means of conditional processing is described in \secref{sec:conditional}.

%%%%%%%%%%%%%%%%%%%%%%%%%%%%%%%%%%%%%%%%
\paragraph{Page Numbering.}

When only a part of the document is compiled,
the appropriate numbering of pages
(as well as other status parameters)
is determined from the |.aux| files.
The latter contain information from previous passes.
However this information needs to propagate through
all intermediate child documents.
Therefore the page numbering in child documents may well
be inconsistent until the complete document is compiled at least once.

A useful (if unconventional) way to always ensure a consistent
page numbering is to restart the numbering in each child document
and denote the pages by `\textit{child}|.|\textit{page}'
where \textit{child} represents the chapter/section number of the child file.
This can be achieved by the command
|\numberwithin{page}{|\textit{child}|}|
of the \textsf{amsmath} package
where \textit{child} can be |chapter| or |section|
depending on the chosen structuring.
Alternatively, one can modify the macro |\thepage| appropriately
and reset the counter |page| at the start of each child file.

%%%%%%%%%%%%%%%%%%%%%%%%%%%%%%%%%%%%%%%%%%%%%%%%%%%%%%%%%%%%%%%%%%%%%%%%%%%%%%%%
\subsection{Conditional Processing}
\label{sec:conditional}

The package provides a mechanism to compile different versions
of a document. To customise the versions further some conditional processing
can come in handy to distinguish which version is being compiled.
The package provides two macros to describe the compilation context:

%%%%%%%%%%%%%%%%%%%%%%%%%%%%%%%%%%%%%%%%
\DescribeMacro{\ifchilddoc}
The conditional |\ifchilddoc| distinguishes between the compilation of
child documents and the main document:
%
\begin{center}
|\ifchilddoc |\textit{child-code}| |[|\||else |\textit{main-code}]| \||fi|
\end{center}

%%%%%%%%%%%%%%%%%%%%%%%%%%%%%%%%%%%%%%%%
\DescribeMacro{\childdocname}
\DescribeMacro{\childdocjob}
The macro |\childdocname| contains the filename (without extension)
of the main or child file being processed.
Note that |\childdocjob| will always contain the name of the main file.

%%%%%%%%%%%%%%%%%%%%%%%%%%%%%%%%%%%%%%%%
\paragraph{Title Page.}

Conditional processing can be used to include a title or banner page
in the main document when proper precautions are taken.
Importantly, the code in the main file should ensure that the page counter
(as well as other status parameters which are stored in the |.aux| files)
takes the same value after the conditional processing.
Otherwise the page numbers may take divergent values
depending on which part is compiled.

For example, a title page could be declared by:
%
\begin{center}
\begin{tabular}{l}
|\ifchilddoc\||else|\\
|\addtocounter{page}{-1}|\\
\textit{code for title page}\\
|\newpage|\\
|\||fi|
\end{tabular}
\end{center}
%
A banner page for the child documents can be generated by:
%
\begin{center}
\begin{tabular}{l}
|\ifchilddoc|\\
|\addtocounter{page}{-1}|\\
\textit{code for banner page}\\
|\newpage|\\
|\||fi|
\end{tabular}
\end{center}
%
Here one could write a message such as:
\begin{center}
|This is the part \childdocname{} of \childdocjob{}.|
\end{center}

%%%%%%%%%%%%%%%%%%%%%%%%%%%%%%%%%%%%%%%%%%%%%%%%%%%%%%%%%%%%%%%%%%%%%%%%%%%%%%%%
\subsection{Flags}
\label{sec:flags}

The package makes it easy to generate different versions
of the main or child documents.
To this end compilation flags can be defined
and assigned different default values.
They will be particularly useful in conjunction
with the forwarding mechanism described in \secref{sec:forward}.

For example, it may be useful to have a flag |\version|
which can be set to |draft| or |final|.
The document source will contain some conditional code
depending on the value of |\version|.
Suppose further, the flag should default to |final| for the main file
and to |draft| for child files
which is a natural assignment for editing the document.
This is achieved by placing the following code
in the preamble of the main document
(below the |\childdocmain| directive):
%
\begin{center}
\begin{tabular}{l}
|\ifchilddoc|\\
|\providecommand{\version}{draft}|\\
|\||else|\\
|\providecommand{\version}{final}|\\
|\||fi|
\end{tabular}
\end{center}
%
The definition by |\providecommand| makes sure
that previous definitions are not overwritten.
Further statements |\providecommand{\version}{...}|
can thus be added before the above code to override it.

For the main file, one might add a line
(between |\childdocmain| and the above block)
%
\begin{center}
|%\ifchilddoc\||else\providecommand{\version}{draft}\||fi|
\end{center}
%
which can be uncommented to produce a draft version.
Likewise one can add a line to the very top of a child file
(above the |\childdocof{|\textit{main}|}| directive)
%
\begin{center}
|%\providecommand{\version}{final}|
\end{center}
%
which can be uncommented to produce the final version of this child document.

%%%%%%%%%%%%%%%%%%%%%%%%%%%%%%%%%%%%%%%%%%%%%%%%%%%%%%%%%%%%%%%%%%%%%%%%%%%%%%%%
\subsection{Forwarding}
\label{sec:forward}

Different versions of the main or child documents
using compilation flags as described in \secref{sec:flags}
can be (permanently) stored in different files
for convenient compilation, viewing and distribution.
To this end, the package defines a command
to pass on compilation to a different file:

%%%%%%%%%%%%%%%%%%%%%%%%%%%%%%%%%%%%%%%%
\DescribeMacro{\childdocforward}
The command |\childdocforward| redirects processing to
another source file:
%
\begin{center}
\begin{tabular}{l}
|\input{childdoc.def}|\\
|\childdocforward[|\textit{main}|]{|\textit{dest}|}|\\
\end{tabular}
\end{center}
%
The argument \textit{dest} is the destination file
(without extension).
It should be the main file or one of the child files.
Note that further \textsf{childdoc} directives
such as |\childdocof| and |\childdocforward|
in the indicated file will be processed in this form.
The optional argument \textit{main}
passes on directly to the main file \textit{main}
while pretending to compile the child \textit{dest}.
This form behaves as if \textit{dest}
issues |\childdocof{|\textit{main}|}| right away,
and no further \textsf{childdoc} directives will be processed.

%%%%%%%%%%%%%%%%%%%%%%%%%%%%%%%%%%%%%%%%
\DescribeMacro{\...prefix}
In the alternative form |\childdocforwardprefix|,
%
\begin{center}
\begin{tabular}{l}
|\input{childdoc.def}|\\
|\childdocforwardprefix[|\textit{main}|]{|\textit{prefix}|}{|\textit{dest}|}|
\end{tabular}
\end{center}
%
the destination file is determined by a pattern
depending on the current file:
To make this work, the current file must be called
`{\textit{prefix}\hspace{0.2em}\textit{suffix}}'
with \textit{prefix} matching precisely the argument.
Processing is then passed on to the file
`{\textit{dest}\hspace{0.2em}\textit{suffix}}'.
Surely, the same effect is achieved by
directly specifying the
argument `{\textit{dest}\hspace{0.2em}\textit{suffix}}'
in the first form.
However, that requires to set up a different file
for each child. With the alternative form of the command
all these files can have exactly the same content
which simplifies setting them up and maintaining them.

For example, the following file |draft.tex|
with a compilation flag |\version| as described in \secref{sec:flags}
compiles the main document as a draft:
%
\begin{center}
\begin{tabular}{l}
|\def\version{draft}|\\
|\input{childdoc.def}|\\
|\childdocforward{|\textit{main}|}|
\end{tabular}
\end{center}
%
Likewise, the following files |final|\textit{nn}|.tex|
compile the final version of the child document
|child|\textit{nn}|.tex|:
%
\begin{center}
\begin{tabular}{l}
|\def\version{final}|\\
|\input{childdoc.def}|\\
|\childdocforwardprefix{final}{child}|
\end{tabular}
\end{center}
%

Note that when several versions of a main file and/or of each child file
are to be generated, it may be convenient to set up a |Makefile| or
shell script to automatise the process.

%%%%%%%%%%%%%%%%%%%%%%%%%%%%%%%%%%%%%%%%%%%%%%%%%%%%%%%%%%%%%%%%%%%%%%%%%%%%%%%%
\subsection{Command Line Processing}
\label{sec:commandline}

The effect of redirection files can also be achieved by invoking
the \LaTeX{} compiler with a more elaborate command line.
Most conveniently this should be done as part
of a shell script or a |Makefile|.

When using \textsf{childdoc} in the main file, the following
command lines effectively perform a redirection
(note that depending on the shell being used,
backslashes may have to be doubled: `|\|' $\to$ `|\\|'):
%
\begin{center}
|... -jobname "|\textit{target}|" |\\|"|[\textit{flags}]%
|\input{childdoc.def}\childdocforward[|\textit{main}|]{|\textit{dest}|}"|
\end{center}
%
Here \textit{target} is the name of the output file,
\textit{main} is the name of the main file
and \textit{dest} is the name of the main or child file to be processed
(all filenames without extensions).
The optional argument \textit{main} can be omitted
if \textit{main} matches \textit{dest}.
Optionally, compilation \textit{flags} can be defined via |\def| commands.
This command line makes the \TeX{} engine believe
it is compiling the file \textit{target}
whose content is specified as the latter parameter.
The provided code then forwards the processing to
\textit{main} or \textit{dest} as described in \secref{sec:forward}.

%%%%%%%%%%%%%%%%%%%%%%%%%%%%%%%%%%%%%%%%%%%%%%%%%%%%%%%%%%%%%%%%%%%%%%%%%%%%%%%%
\subsection{Include by Input}
\label{sec:input}

Including child documents by |\include| has some restrictions by design.
Most notably, the content of a child document always occupies
its own set of pages; pages cannot be shared between child documents.
Usually, this behaviour makes perfect sense
because each child document contain an essential part of the document.
However, in some situations it may be desirable to compose
a document from a collection of parts
without having mandatory page breaks between then.
For this case, the package
provides a mechanism to include parts
by |\input| which can also be processed individually.
However, by construction this mechanism
requires manual handling of the content to be output.

%%%%%%%%%%%%%%%%%%%%%%%%%%%%%%%%%%%%%%%%
\DescribeMacro{\ifchilddocmanual}
The main file should be prepared as usual, see \secref{sec:include}.
However, the document body must make a distinction
between processing of an individual part and of the main document, e.g.:
%
\begin{center}
\begin{tabular}{l}
|\ifchilddocmanual|\\
|\input{\childdocname}|\\
|\||else|\\
\textit{document body with }|\input{|\textit{part}|}|\\
|\||fi|
\end{tabular}
\end{center}
%
The conditional |\ifchilddocmanual| is true whenever
a part to be included by |\input| is being compiled,
and the name of the part is stored in |\childdocname|.

%%%%%%%%%%%%%%%%%%%%%%%%%%%%%%%%%%%%%%%%
\DescribeMacro{\childdocby}
Each part to be included by |\input| should start with:
%
\begin{center}
\begin{tabular}{l}
|\input{childdoc.def}|\\
|\childdocby{|\textit{main}|}|\\
\end{tabular}
\end{center}
%
The directive |\childdocby| is similar to |\childdocof|
described in \secref{sec:include},
but the subsequent selection of content must be done manually.
To that end, both |\ifchilddoc| and |\ifchilddocmanual|
will be true upon processing of a part,
and the name of the part is stored in |\childdocname|.
Note that |\jobname| will be set to the filename of the current part
so that each part receives an individual |.aux| file
that does not interfere with the |.aux| file(s) of the main document.
This behaviour can be altered by the alternative form
|\childdocby[*]{|\textit{main}|}| (with a non-empty optional argument)
which uses the |.aux| file of the main document
by setting |\jobname| to \textit{main}.

%%%%%%%%%%%%%%%%%%%%%%%%%%%%%%%%%%%%%%%%%%%%%%%%%%%%%%%%%%%%%%%%%%%%%%%%%%%%%%%%
\subsection{Driver Development}
\label{sec:driver}

The \textsf{childdoc} mechanism can also be use for the development
of definition files such as \LaTeX{} styles or classes.
This case differs from the above setup with multiple parts
included by |\include| in that no |\includeonly| should be invoked.
This can be achieved by starting the include file
(before |\ProvidesPackage|) with:
%
\begin{center}
\begin{tabular}{l}
|\input{childdoc.def}|\\
|\childdocforward{|\textit{main}|}|\\
\end{tabular}
\end{center}
%
or alternatively with:
%
\begin{center}
\begin{tabular}{l}
|\input{childdoc.def}|\\
|\childdocby{|\textit{main}|}|\\
\end{tabular}
\end{center}
%
Both forms have slightly different effects as described above.
The main file is prepared as usual, see \secref{sec:include}.

%%%%%%%%%%%%%%%%%%%%%%%%%%%%%%%%%%%%%%%%%%%%%%%%%%%%%%%%%%%%%%%%%%%%%%%%%%%%%%%%
\subsection{Legacy Detection}
\label{sec:detection}

The directive |\childdocmain| in the main file can detect
whether the complete document or merely a child is to be compiled
even without using the directive |\childdocof|.
This method is deprecated because it is less robust
and there is no compelling reason to use it;
it is merely provided for backward compatibility
and it may be removed in future versions.

If the detection mechanism is to be used,
it is mandatory to correctly specify
the filename of the main file as the argument of |\childdocmain|:
%
\begin{center}
\begin{tabular}{l}
|\input{childdoc.def}|\\
|\childdocmain{|\textit{main}|}|\\
\end{tabular}
\end{center}
%
If |\jobname| does not match the argument \textit{main} of |\childdocmain|,
it is assumed that |\jobname| points to the child file to be compiled.
When using |\childdocmain| with the main file specified as argument,
it suffices to start a child file
with just |\input{|\textit{main}|}|
without loading of the package and using |\childdocof|.
If instead all processing is done
with the appropriate \textsf{childdoc} directives,
the argument of \textit{main} of |\childdocmain| can be empty.

An alternative version of the command line processing described
in \secref{sec:commandline} using the detection mechanism reads:
%
\begin{center}
|... -jobname "|\textit{target}|" "|[\textit{flags}]%
[|\def\jobname{|\textit{dest}|}|]|\input{|\textit{main}|}"|
\end{center}

%%%%%%%%%%%%%%%%%%%%%%%%%%%%%%%%%%%%%%%%%%%%%%%%%%%%%%%%%%%%%%%%%%%%%%%%%%%%%%%%
\subsection{Manual Code}
\label{sec:manual}

In case one cannot be certain whether the definitions file |childdoc.def|
is installed on the target \TeX{} distribution
and one prefers not to ship it,
it is conceivable to paste a few relevant commands into the sources.

To that end, drop all statements |\input{childdoc.def}|
and perform the replacements as outlined below.
Instead of |\childdocmain{|\textit{main}|}| add the following code
to the top of the main file:
%
\begin{center}
\begin{tabular}{l}
|\||ifdefined\childdocname\endinput\||fi\newif\ifchilddoc|\\
|\edef\childdocname{\scantokens\expandafter{\jobname\noexpand}}|\\
|\def\childdocmain{|\textit{main}|}\||ifx\childdocmain\childdocname\||else|\\
|\childdoctrue\includeonly{\childdocname}\let\jobname\childdocmain\||fi|\\
\end{tabular}
\end{center}
%
Instead of |\childdocof{|\textit{main}|}| just include the main file
at the top of each child file:
%
\begin{center}
|\input{|\textit{main}|}|
\end{center}
%
A simple redirection |\childdocforward{|\textit{dest}|}| is achieved by:
%
\begin{center}
|\def\jobname{|\textit{dest}|}\input{\jobname}|
\end{center}
%
The redirection with prefix
|\childdocforwardprefix[|\textit{prefix}|]{|\textit{dest}|}|
is accomplished by:
%
\begin{center}
\begin{tabular}{l}
|{\edef\jobname{\scantokens\expandafter{\jobname\noexpand}}|\\
|\def\redirectjob |\textit{prefix}|#1~~~{\gdef\jobname{|\textit{dest}|#1}}|\\
|\expandafter\redirectjob\jobname~~~}\input{\jobname}|
\end{tabular}
\end{center}

In an alternative approach,
child documents can be compiled by a specific command line
without additional code or specific definitions:
%
\begin{center}
|... -jobname "|\textit{target}|" "|[\textit{flags}]%
|\includeonly{|\textit{dest}|}\input{|\textit{main}|}"|
\end{center}
%

%%%%%%%%%%%%%%%%%%%%%%%%%%%%%%%%%%%%%%%%%%%%%%%%%%%%%%%%%%%%%%%%%%%%%%%%%%%%%%%%
%%%%%%%%%%%%%%%%%%%%%%%%%%%%%%%%%%%%%%%%%%%%%%%%%%%%%%%%%%%%%%%%%%%%%%%%%%%%%%%%
\section{Information}

%%%%%%%%%%%%%%%%%%%%%%%%%%%%%%%%%%%%%%%%%%%%%%%%%%%%%%%%%%%%%%%%%%%%%%%%%%%%%%%%
\subsection{Copyright}

Copyright \copyright{} 2017--2018 Niklas Beisert

This work may be distributed and/or modified under the
conditions of the \LaTeX{} Project Public License, either version 1.3
of this license or (at your option) any later version.
The latest version of this license is in
  \url{http://www.latex-project.org/lppl.txt}
and version 1.3 or later is part of all distributions of \LaTeX{}
version 2005/12/01 or later.

This work has the LPPL maintenance status `maintained'.

The Current Maintainer of this work is Niklas Beisert.

This work consists of the files |README.txt|, |childdoc.ins| and |childdoc.dtx|
as well as the derived files |childdoc.def|, |cdocsamp.tex|
with |cdocsch1.tex|, |cdocsch2.tex|, |cdocspt3.tex|, |cdocspt4.tex|,
|cdocsdrf.tex|, |cdocsfn1.tex|, |cdocsfn2.tex|
as well as |childdoc.pdf|.

%%%%%%%%%%%%%%%%%%%%%%%%%%%%%%%%%%%%%%%%%%%%%%%%%%%%%%%%%%%%%%%%%%%%%%%%%%%%%%%%
\subsection{Files and Installation}

The package consists of the files:
%
\begin{center}
\begin{tabular}{ll}
    |README.txt|   & readme file \\
    |childdoc.ins| & installation file \\
    |childdoc.dtx| & source file \\
    |childdoc.def| & definition file \\
    |cdocsamp.tex| & sample main file \\
    |cdocsch1.tex| & sample include file \\
    |cdocsch2.tex| & sample include file \\
    |cdocspt3.tex| & sample part file \\
    |cdocspt4.tex| & sample part file \\
    |cdocsdrf.tex| & sample redirection file \\
    |cdocsfn1.tex| & sample redirection file \\
    |cdocsfn2.tex| & sample redirection file \\
    |childdoc.pdf| & manual
\end{tabular}
\end{center}
%
The distribution consists of the files
|README.txt|, |childdoc.ins| and |childdoc.dtx|.
%
\begin{itemize}
\item
Run (pdf)\LaTeX{} on |childdoc.dtx|
to compile the manual |childdoc.pdf| (this file).
\item
Run \LaTeX{} on |childdoc.ins| to create the definitions file |childdoc.def|
and the sample |cdocsamp.tex| with include files
|cdocsch1.tex|, |cdocsch2.tex|, |cdocspt3.tex|, |cdocspt4.tex|,
|cdocsdrf.tex|, |cdocsfn1.tex|, |cdocsfn2.tex|.
Then copy the file |childdoc.def| to an appropriate directory of your \LaTeX{}
distribution, e.g.\ \textit{texmf-root}|/tex/latex/childdoc|.
\end{itemize}

%%%%%%%%%%%%%%%%%%%%%%%%%%%%%%%%%%%%%%%%%%%%%%%%%%%%%%%%%%%%%%%%%%%%%%%%%%%%%%%%
\subsection{Related CTAN Packages}

There are several other packages which offer a similar functionality:
%
\begin{itemize}
\item
The packages
\href{http://ctan.org/pkg/docmute}{\textsf{docmute}},
\href{http://ctan.org/pkg/includex}{\textsf{includex}} and
\href{http://ctan.org/pkg/standalone}{\textsf{standalone}}
provide commands to include only the document body of
a child file thus allowing both files to be compiled individually.
\item
The packages \href{http://ctan.org/pkg/subdocs}{\textsf{subdocs}}
and \href{http://ctan.org/pkg/subfiles}{\textsf{subfiles}}
provide structures in which the main and child documents can be
encapsulated and allowing them to be compiled individually.
The inclusion mechanism is different from the conventional |\include|.
\item
The package \href{http://ctan.org/pkg/combine}{\textsf{combine}}
is an elaborate solution to combine several documents into one.
\end{itemize}
%
See also the CTAN topic \href{http://ctan.org/topic/subdocs}{\textsf{subdocs}}
for further related packages.
The present package differs from the above solutions in that
a document structure constructed with the conventional |\include| mechanism
just needs two extra commands at the top of every file
such that all constituent files can be compiled individually.

%%%%%%%%%%%%%%%%%%%%%%%%%%%%%%%%%%%%%%%%%%%%%%%%%%%%%%%%%%%%%%%%%%%%%%%%%%%%%%%%
%\subsection{Feature Suggestions}
%
%The following is a list of features which may be useful for future
%versions of this package:
%%
%\begin{itemize}
%\item
%\ldots
%\end{itemize}

%%%%%%%%%%%%%%%%%%%%%%%%%%%%%%%%%%%%%%%%%%%%%%%%%%%%%%%%%%%%%%%%%%%%%%%%%%%%%%%%
\subsection{Revision History}

%%%%%%%%%%%%%%%%%%%%%%%%%%%%%%%%%%%%%%%%
\paragraph{v2.0:} 2018/12/30

\begin{itemize}
\item
immediate forward processing
\item
added |\childdocby| mechanism
\item
manual restructured
\end{itemize}

%%%%%%%%%%%%%%%%%%%%%%%%%%%%%%%%%%%%%%%%
\paragraph{v1.6:} 2018/01/17

\begin{itemize}
\item
application for development of include files
\item
corrections to manual
\end{itemize}

%%%%%%%%%%%%%%%%%%%%%%%%%%%%%%%%%%%%%%%%
\paragraph{v1.5:} 2017/05/21

\begin{itemize}
\item
more complete structuring introduced
\item
|\childdocof| introduced
\item
|\childdoc| renamed to |\childdocmain|
\item
|\childredirect| renamed to |\childdocforward| and |\childdocforwardprefix|
and functionality expanded
\end{itemize}

%%%%%%%%%%%%%%%%%%%%%%%%%%%%%%%%%%%%%%%%
\paragraph{v1.0:} 2017/04/27

\begin{itemize}
\item
manual and install package
\item
first version published on CTAN
\end{itemize}

%%%%%%%%%%%%%%%%%%%%%%%%%%%%%%%%%%%%%%%%
\paragraph{v0.6:} 2017/04/26

\begin{itemize}
\item
redirection mechanism added
\end{itemize}

%%%%%%%%%%%%%%%%%%%%%%%%%%%%%%%%%%%%%%%%
\paragraph{v0.5:} 2017/04/26

\begin{itemize}
\item
functionality in definition file
\end{itemize}


%%%%%%%%%%%%%%%%%%%%%%%%%%%%%%%%%%%%%%%%%%%%%%%%%%%%%%%%%%%%%%%%%%%%%%%%%%%%%%%%
%%%%%%%%%%%%%%%%%%%%%%%%%%%%%%%%%%%%%%%%%%%%%%%%%%%%%%%%%%%%%%%%%%%%%%%%%%%%%%%%
%%%%%%%%%%%%%%%%%%%%%%%%%%%%%%%%%%%%%%%%%%%%%%%%%%%%%%%%%%%%%%%%%%%%%%%%%%%%%%%%
\appendix

\settowidth\MacroIndent{\rmfamily\scriptsize 000\ }

 \DocInput{childdoc.dtx}

\end{document}
%</driver>
% \fi
%
% %%%%%%%%%%%%%%%%%%%%%%%%%%%%%%%%%%%%%%%%%%%%%%%%%%%%%%%%%%%%%%%%%%%%%%%%%%%%%%
% %%%%%%%%%%%%%%%%%%%%%%%%%%%%%%%%%%%%%%%%%%%%%%%%%%%%%%%%%%%%%%%%%%%%%%%%%%%%%%
% \section{Sample}
%\iffalse
%<*samplemain>
%\fi
%
% The following presents a sample document
% with two chapters, two parts, a title page,
% a compile flag as well as three forwarding files to set the flag.
% It consists of eight |.tex| files:
% \begin{center}
% \begin{tabular}{ll}
% |cdocsamp.tex|&main file\\
% |cdocsch1.tex|&include file for chapter 1\\
% |cdocsch2.tex|&include file for chapter 2\\
% |cdocspt3.tex|&include file for part 3\\
% |cdocspt4.tex|&include file for part 4\\
% |cdocsdrf.tex|&forwarding file for main file in draft mode\\
% |cdocsfi1.tex|&forwarding file for final version of chapter 1\\
% |cdocsfi2.tex|&forwarding file for final version of chapter 2\\
% \end{tabular}
% \end{center}
% Each of the eight files can be compiled directly by the \LaTeX{} compiler.
%
% %%%%%%%%%%%%%%%%%%%%%%%%%%%%%%%%%%%%%%
% \paragraph{Main File.}
%
% The main file is called |cdocsamp.tex|.
%
% Load the \textsf{childdoc} definitions and
% declare the filename for the main document:
%    \begin{macrocode}
\input{childdoc.def}
\childdocmain{}
%    \end{macrocode}

% Optional override for |\version| flag:
%    \begin{macrocode}
%%\ifchilddoc\else\providecommand{\version}{draft}\fi
%    \end{macrocode}

% Define the default values for the |\version| flag
% (|final| for the main file and |draft| for childs):
%    \begin{macrocode}
\ifchilddoc
\providecommand{\version}{draft}
\else
\providecommand{\version}{final}
\fi
%    \end{macrocode}

% Load the standard document class:
%    \begin{macrocode}
\documentclass[12pt]{article}
%    \end{macrocode}

% Start the document body:
%    \begin{macrocode}
\begin{document}
%    \end{macrocode}

% Declare a title page.
% Print title, part of document being processed and version flag:
%    \begin{macrocode}
\addtocounter{page}{-1}
\begin{center}
{\LARGE\bfseries{}childdoc example\par}
\vspace{1cm}
\ifchilddoc
\ifchilddocmanual part\else chapter\fi:
`\childdocname' of `\childdocjob'\par
\else
main document: `\childdocjob'\par
\fi
version: \version\par
\end{center}
\newpage
%    \end{macrocode}

% Manually include selected file,
% otherwise process as usual:
%    \begin{macrocode}
\ifchilddocmanual
\section*{part `\childdocname'}
\input{\childdocname}
\else
%    \end{macrocode}

% Include the two chapters:
%    \begin{macrocode}
\include{cdocsch1}
\include{cdocsch2}
%    \end{macrocode}

% Include the two parts unless only chapters should be displayed:
%    \begin{macrocode}
\ifchilddoc\else
\section{part three}
\input{cdocspt3}
\section{part four}
\input{cdocspt4}
\fi
%    \end{macrocode}

% Process as usual until here:
%    \begin{macrocode}
\fi
%    \end{macrocode}

% End of document body:
%    \begin{macrocode}
\end{document}
%    \end{macrocode}
%\iffalse
%</samplemain>
%\fi
%
% %%%%%%%%%%%%%%%%%%%%%%%%%%%%%%%%%%%%%%
% \paragraph{Chapter Include Files.}
%
% The include files are called |cdocsch1.tex| and |cdocsch2.tex|.
%
%\iffalse
%<*samplechap1|samplechap2>
%\fi

% Optional override for |\version| flag:
%    \begin{macrocode}
%%\providecommand{\version}{final}
%    \end{macrocode}

% Include the main document:
%    \begin{macrocode}
\input{childdoc.def}
\childdocof{cdocsamp}
%    \end{macrocode}

%\iffalse
%</samplechap1|samplechap2>
%\fi
%
%\iffalse
%<*samplechap1>
%\fi
% Some text for chapter 1:
%    \begin{macrocode}
\section{one}
some text in chapter one
%    \end{macrocode}

%\iffalse
%</samplechap1>
%\fi
% Some text for chapter 2:
%\iffalse
%<*samplechap2>
%\fi
%    \begin{macrocode}
\section{two}
more text in chapter two
%    \end{macrocode}

%\iffalse
%</samplechap2>
%\fi
%
% %%%%%%%%%%%%%%%%%%%%%%%%%%%%%%%%%%%%%%
% \paragraph{Part Include Files.}
%
% The include files are called |cdocspt3.tex| and |cdocspt4.tex|.
%
%\iffalse
%<*samplepart3|samplepart4>
%\fi

% Optional override for |\version| flag:
%    \begin{macrocode}
%%\providecommand{\version}{final}
%    \end{macrocode}

% Include the main document:
%    \begin{macrocode}
\input{childdoc.def}
\childdocby{cdocsamp}
%    \end{macrocode}

%\iffalse
%</samplepart3|samplepart4>
%\fi
%
%\iffalse
%<*samplepart3>
%\fi
% Some text for part 3:
%    \begin{macrocode}
some text in part three
%    \end{macrocode}

%\iffalse
%</samplepart3>
%\fi
% Some text for part 4:
%\iffalse
%<*samplepart4>
%\fi
%    \begin{macrocode}
more text in part four
%    \end{macrocode}

%\iffalse
%</samplepart4>
%\fi
%
% %%%%%%%%%%%%%%%%%%%%%%%%%%%%%%%%%%%%%%
% \paragraph{Forwarding for a Complete Draft.}
%
% The following forwarding file |cdocsdrf.tex|
% compiles the main document in draft mode:
%\iffalse
%<*sampledraft>
%\fi
%    \begin{macrocode}
\def\version{draft}
\input{childdoc.def}
\childdocforward{cdocsamp}
%    \end{macrocode}

%\iffalse
%</sampledraft>
%\fi
%
% %%%%%%%%%%%%%%%%%%%%%%%%%%%%%%%%%%%%%%
% \paragraph{Forwarding for Final Version of the Chapters.}
%
% The following forwarding files |cdocsfn1.tex| and |cdocsfn2.tex|
% (with identical content)
% compile the final versions of the child documents
% |cdocsch1.tex| and |cdocsch2.tex|, respectively:
%\iffalse
%<*samplefinal>
%\fi
%    \begin{macrocode}
\def\version{final}
\input{childdoc.def}
\childdocforwardprefix[cdocsamp]{cdocsfn}{cdocsch}
%    \end{macrocode}

%\iffalse
%</samplefinal>
%\fi
%
% %%%%%%%%%%%%%%%%%%%%%%%%%%%%%%%%%%%%%%
% \paragraph{Command Line Processing.}
%
% The following three command lines generate the output files
% |cdocscld|, |cdocscl1| and |cdocscl2|
% which should be identical to
% |cdocsdrf|, |cdocsch1| and |cdocsfn2|, respectively:
% \begin{center}
% \begin{tabular}{l}
% |latex -jobname cdocscld \|\\
% |  "\def\version{draft}\input{childdoc.def}\childdocforward{cdocsamp}"|\\
% |latex -jobname cdocscl1 \|\\
% |  "\input{childdoc.def}\childdocforward[cdocsamp]{cdocsch1}"|\\
% |latex -jobname cdocscl2 \|\\
% |  "\def\version{final}\input{childdoc.def}\childdocforward{cdocsch2}"|
% \end{tabular}
% \end{center}
% Note that the trailing backslash on each first line
% merely continues the input to the second line
% (for convenient cut ant paste).
% Furthermore, the command |latex| can be replaced by any
% of its alternative versions such as |pdflatex|.
%
% %%%%%%%%%%%%%%%%%%%%%%%%%%%%%%%%%%%%%%%%%%%%%%%%%%%%%%%%%%%%%%%%%%%%%%%%%%%%%%
% %%%%%%%%%%%%%%%%%%%%%%%%%%%%%%%%%%%%%%%%%%%%%%%%%%%%%%%%%%%%%%%%%%%%%%%%%%%%%%
% \section{Implementation}
%\iffalse
%<*package>
%\fi
%
% This section describes the definitions file |childdoc.def|.

% The definitions cannot be loaded using |\usepackage| or |\RequirePackage|
% which has a mechanism to prevent loading a style file more than once.
% When loading the definitions by means of |\input|
% multiple instances have to be prevented manually:
%\iffalse
%This code needs to be before the `\ProvidesFile' directive
%which is defined at the beginning of this file.
%Therefore it is also placed there and commented out here.
%</package>
%<*discard>
%\fi
%    \begin{macrocode}
\ifdefined\childdocmain\endinput\fi
%    \end{macrocode}
%\iffalse
%</discard>
%<*package>
%\fi
%
% \macro{\ifchilddoc}
% \macro{\ifchilddocmanual}
% The conditional |\ifchilddoc| tells whether a
% child (true) or main (false) document is being compiled.
% The conditional |\ifchilddocmanual| tells whether
% the |\includeonly| mechanism is used (false) or
% the selection of child files must be performed manually (true).
% The definitions initialise to false:
%    \begin{macrocode}
\newif\ifchilddoc
\newif\ifchilddocmanual
%    \end{macrocode}

% \macro{\childdocname}
% \macro{\childdocjob}
% The macro |\childdocname| stores the name of the main document
% to be compiled. The macro |\childdocjob| stores the name of
% the document on which the \LaTeX{} compiler was originally invoked.
% The content of |\jobname| cannot be compared
% to filenames specified in the source due to different catcodes.
% The following code rescans |\jobname|, stores the result
% in |\childdocname| and saves a copy in |\childdocjob|:
%    \begin{macrocode}
\edef\childdocname{\scantokens\expandafter{\jobname\noexpand}}
\let\childdocjob\childdocname
%    \end{macrocode}

% \macro{\childdocdisable}
% The macro |\childdocdisable| prevents the main file
% from being processed more than once.
% At this stage, the main document command |\childdocmain|
% is assumed to be called once again where it should do nothing.
% Any subsequent call to it should prevent
% a secondary processing of the main document
% It overwrites the forwarding commands
% |\childdocof| and |\childdocforward|
% with empty macros to prevent further inclusions of the main document:
%    \begin{macrocode}
\newcommand{\childdocdisable}
{
  \renewcommand{\childdocmain}[1]{\renewcommand{\childdocmain}[1]{\endinput}}
  \renewcommand{\childdocof}[1]{}
  \renewcommand{\childdocby}[2][]{}
  \renewcommand{\childdocforward}[2][]{}
  \renewcommand{\childdocdisable}{}
}
%    \end{macrocode}

% \macro{\childdocmain}
% The macro |\childdocmain| is to be called at the top of the main file
% with nothing or the main filename (without extension) as argument.
% First, it breaks loops.
% If the argument is not empty and does not match |\childdocname|
% (which is set by the first inclusion of |childdoc.def|),
% |\ifchilddoc| is set to true, |\includeonly| is applied to the child file
% and |\jobname| is set to the main file
% (for proper handling of |.aux| files):
%    \begin{macrocode}
\newcommand{\childdocmain}[1]
{
  \childdocdisable\childdocmain{}
  \if?#1?\else
    \begingroup
      \def\childdoctmp{#1}
      \ifx\childdoctmp\childdocname
        \def\childdoctmp{}
      \else
        \def\childdoctmp
        {
          \childdoctrue
          \includeonly{\childdocname}
          \def\childdocjob{#1}
          \def\jobname{#1}
        }
      \fi
      \expandafter
    \endgroup
    \childdoctmp
  \fi
}
%    \end{macrocode}

% \macro{\childdocof}
% The command |\childdocof| redirects
% compilation to the main file |#1|.
%    \begin{macrocode}
\newcommand{\childdocof}[1]
{
  \childdocdisable
  \childdoctrue
  \includeonly{\childdocname}
  \def\jobname{#1}
  \def\childdocjob{#1}
  \input{#1}
}
%    \end{macrocode}

% \macro{\childdocby}
% The command |\childdocby| ....
%    \begin{macrocode}
\newcommand{\childdocby}[2][]
{
  \childdocdisable
  \childdoctrue
  \childdocmanualtrue
  \if?#1?\else
    \def\jobname{#2}
  \fi
  \def\childdocjob{#2}
  \input{#2}
  \endinput
}
%    \end{macrocode}

% \macro{\childdocforward}
% The command |\childdocforward| redirects
% compilation to the main file or
% (if the optional argument is given) a child file.
% Parameters are set as if the main file
% or a child file starting with |\childdocof| was compiled.
% Then compilation is handed over to the main file:
%    \begin{macrocode}
\newcommand{\childdocforward}[2][]
{
  \begingroup
    \if?#1?
      \def\childdoctmp
      {
        \def\childdocname{#2}
        \def\childdocjob{#2}
        \def\jobname{#2}
        \input{#2}
        \endinput
      }
    \else
      \def\childdoctmp
      {
        \childdocdisable
        \def\childdocname{#2}
        \childdoctrue
        \includeonly{#2}
        \def\childdocjob{#1}
        \def\jobname{#1}
        \input{#1}
        \endinput
      }
    \fi
    \expandafter
  \endgroup
  \childdoctmp
}
%    \end{macrocode}

% \macro{\childdocforwardprefix}
% The command |\childdocforwardprefix| redirects
% compilation to the main or a child file by means of a pattern.
% The prefix |#1| in the current filename is replaced by |#2|
% and the suffix of the current filename is kept
% (it is assumed that the filename does not contain the substring `|~~~|'
% which is used as a delimiter).
% Compilation is handed over to the new file by |\childdocforward|:
%    \begin{macrocode}
\newcommand{\childdocforwardprefix}[3][]
{
  \begingroup
    \def\childdocextract #2##1~~~{\def\childdoctmp{\childdocforward[#1]{#3##1}}}
    \expandafter\childdocextract\childdocname~~~
    \expandafter
  \endgroup
  \childdoctmp
}
%    \end{macrocode}

% \macro{\childdoc}
% The deprecated macro |\childdoc| is a legacy version of |\childdocmain|:
%    \begin{macrocode}
\newcommand{\childdoc}{\childdocmain}
%    \end{macrocode}

% \macro{\childdocredirect}
% The deprecated macro |\childdocredirect| is a legacy version
% of |\childdocforward| and |\childdocforwardprefix|:
%    \begin{macrocode}
\newcommand{\childdocredirect}[2][]
{
  \begingroup
    \if?#1?
      \def\childdoctmp{\childdocforward{#2}}
    \else
      \def\childdoctmp{\childdocforwardprefix{#1}{#2}}
    \fi
    \expandafter
  \endgroup
  \childdoctmp
}
%    \end{macrocode}

%\iffalse
%</package>
%\fi
%
\endinput
|\\
|\childdocforward[|\textit{main}|]{|\textit{dest}|}|\\
\end{tabular}
\end{center}
%
The argument \textit{dest} is the destination file
(without extension).
It should be the main file or one of the child files.
Note that further \textsf{childdoc} directives
such as |\childdocof| and |\childdocforward|
in the indicated file will be processed in this form.
The optional argument \textit{main}
passes on directly to the main file \textit{main}
while pretending to compile the child \textit{dest}.
This form behaves as if \textit{dest}
issues |\childdocof{|\textit{main}|}| right away,
and no further \textsf{childdoc} directives will be processed.

%%%%%%%%%%%%%%%%%%%%%%%%%%%%%%%%%%%%%%%%
\DescribeMacro{\...prefix}
In the alternative form |\childdocforwardprefix|,
%
\begin{center}
\begin{tabular}{l}
|% \iffalse
%
% childdoc.dtx Copyright (C) 2017-2018 Niklas Beisert
%
% This work may be distributed and/or modified under the
% conditions of the LaTeX Project Public License, either version 1.3
% of this license or (at your option) any later version.
% The latest version of this license is in
%   http://www.latex-project.org/lppl.txt
% and version 1.3 or later is part of all distributions of LaTeX
% version 2005/12/01 or later.
%
% This work has the LPPL maintenance status `maintained'.
%
% The Current Maintainer of this work is Niklas Beisert.
%
% This work consists of the files childdoc.dtx and childdoc.ins
% and the derived files childdoc.def and cdocsamp.tex with
% cdocsch1.tex, cdocsch2.tex, cdocsdrf.tex, cdocsfn1.tex, cdocsfn2.tex.
%
%<package>\ifdefined\childdocmain\endinput\fi
%<package>\ProvidesFile{childdoc.def}[2018/12/30 v2.0 child document driver]
%<samplemain>\ProvidesFile{cdocsamp.tex}[2018/12/30 v2.0 sample for childdoc]
%<*driver>
%\ProvidesFile{childdoc.drv}[2018/12/30 v2.0 childdoc reference manual file]
\PassOptionsToClass{10pt,a4paper}{article}
\documentclass{ltxdoc}

\usepackage[margin=35mm]{geometry}
\usepackage{hyperref}
\usepackage{hyperxmp}
\usepackage[usenames]{color}

\hypersetup{colorlinks=true}
\hypersetup{pdfstartview=FitH}
\hypersetup{pdfpagemode=UseNone}
\hypersetup{pdfsource={}}
\hypersetup{pdflang={en-UK}}
\hypersetup{pdfcopyright={Copyright 2017-2018 Niklas Beisert.
  This work may be distributed and/or modified under the
  conditions of the LaTeX Project Public License, either version 1.3
  of this license or (at your option) any later version.}}
\hypersetup{pdflicenseurl={http://www.latex-project.org/lppl.txt}}
\hypersetup{pdfcontactaddress={ETH Zurich, ITP, HIT K,
  Wolfgang-Pauli-Strasse 27}}
\hypersetup{pdfcontactpostcode={8093}}
\hypersetup{pdfcontactcity={Zurich}}
\hypersetup{pdfcontactcountry={Switzerland}}
\hypersetup{pdfcontactemail={nbeisert@itp.phys.ethz.ch}}
\hypersetup{pdfcontacturl={http://people.phys.ethz.ch/\xmptilde nbeisert/}}

\newcommand{\secref}[1]{\hyperref[#1]{section \ref*{#1}}}

\parskip1ex
\parindent0pt
\let\olditemize\itemize
\def\itemize{\olditemize\parskip0pt}

\begin{document}

\title{The \textsf{childdoc} Package}
\hypersetup{pdftitle={The childdoc Package}}
\author{Niklas Beisert\\[2ex]
  Institut f\"ur Theoretische Physik\\
  Eidgen\"ossische Technische Hochschule Z\"urich\\
  Wolfgang-Pauli-Strasse 27, 8093 Z\"urich, Switzerland\\[1ex]
  \href{mailto:nbeisert@itp.phys.ethz.ch}
  {\texttt{nbeisert@itp.phys.ethz.ch}}}
\hypersetup{pdfauthor={Niklas Beisert}}
\hypersetup{pdfsubject={Manual for the LaTeX2e Package childdoc}}
\date{30 December 2018, \textsf{v2.0}}
\maketitle

\begin{abstract}\noindent
\textsf{childdoc} is a \LaTeXe{} package
that enables the direct compilation
of document sections included by |\include|
to individual files.
\end{abstract}

\begingroup
\parskip0ex
\tableofcontents
\endgroup

%%%%%%%%%%%%%%%%%%%%%%%%%%%%%%%%%%%%%%%%%%%%%%%%%%%%%%%%%%%%%%%%%%%%%%%%%%%%%%%%
%%%%%%%%%%%%%%%%%%%%%%%%%%%%%%%%%%%%%%%%%%%%%%%%%%%%%%%%%%%%%%%%%%%%%%%%%%%%%%%%
\section{Introduction}

\LaTeX{} provides a mechanism to structure a large document (such as a book)
into a main file and several child files (containing the chapters)
using the |\include| command.
This mechanism is beneficial for documents
which span hundreds of pages in order to
make the source file(s) more manageable.
Moreover, compilation can be restricted to
selected child files by means of the |\includeonly| command.
The latter feature can be used to reduce the compilation time while editing
(this was significantly more useful in the earlier days of \LaTeX{})
or to generate a smaller document which is easier to navigate.
Another application of |\includeonly| is to generate
documents consisting of selected parts of the complete document.

However, there are a few drawbacks of the plain |\include| mechanism:
\begin{itemize}
\item
The child files cannot be compiled on their own,
they can only be compiled via the main file.
A naive editing environment
(such as a text editor with an option
to have the current file processed by \LaTeX)
may require one to switch to the main file before compiling;
attempting to compile the child file produces errors.
\item
The main file must be modified (each time)
to adjust the |\includeonly| command
to the present needs. This easily leaves the main file in a messy state.
\item
The generated document will always carry the filename
of the main document. This is inconvenient if
several child files are to be compiled and
to be kept for distribution.
\end{itemize}

The present package provides a simple interface
to make child files individually compilable by \LaTeX{}.
Compiling a child file then has the same effect as compiling
the main file with an |\includeonly| command
to select the appropriate child.
Moreover the generated document will carry the name of the child
rather than the main file.
This resolves all three above issues.

This feature is meant to make the editing of books,
thesis documents and lecture notes somewhat more convenient.
However, the package can also be used efficiently for
composing a series of documents (such as exercise sheets)
which are typically distributed individually.
It then assists the author in generating the individual documents
(potentially in different versions)
as well as a document containing the collected series.
Another application is in developing style files
or other kinds of included material
where compilation of the style file could redirect
to a sample or test file.

%%%%%%%%%%%%%%%%%%%%%%%%%%%%%%%%%%%%%%%%%%%%%%%%%%%%%%%%%%%%%%%%%%%%%%%%%%%%%%%%
%%%%%%%%%%%%%%%%%%%%%%%%%%%%%%%%%%%%%%%%%%%%%%%%%%%%%%%%%%%%%%%%%%%%%%%%%%%%%%%%
\section{Usage}

First of all, the package \textsf{childdoc} is \emph{not} a standard
\LaTeXe{} |.sty| style file! Therefore it needs to be invoked in
a non-standard way.

%%%%%%%%%%%%%%%%%%%%%%%%%%%%%%%%%%%%%%%%%%%%%%%%%%%%%%%%%%%%%%%%%%%%%%%%%%%%%%%%
\subsection{Included Files}
\label{sec:include}

%%%%%%%%%%%%%%%%%%%%%%%%%%%%%%%%%%%%%%%%
\DescribeMacro{\childdocmain}
To use the package, add the commands
\begin{center}
\begin{tabular}{l}
|\input{childdoc.def}|\\
|\childdocmain{}|\\
\end{tabular}
\end{center}
at the very top of the main \LaTeX{} file,
in particular \emph{before} the |\documentclass| statement!
The argument of |\childdocmain| should be left empty
(but it must be present).

%%%%%%%%%%%%%%%%%%%%%%%%%%%%%%%%%%%%%%%%
\DescribeMacro{\childdocof}
Furthermore, add the commands
\begin{center}
\begin{tabular}{l}
|\input{childdoc.def}|\\
|\childdocof{|\textit{main}|}|\\
\end{tabular}
\end{center}
at the top of every child file \textit{child}
which is included by |\include{|\textit{child}|}|
from within the main file
(or at least for those files to be compiled individually).
The argument \textit{main} must be the filename of the main file.

There are a couple of
considerations in setting up the main and child documents:

%%%%%%%%%%%%%%%%%%%%%%%%%%%%%%%%%%%%%%%%
\paragraph{Restrictions.}

Please note the following restrictions:
\begin{itemize}
\item
|\childdocmain| must be called with one argument \textit{main}
to ensure compatibility with earlier version of the package.
It must either be empty (|\childdocmain{}|)
or precisely match the filename of the main file in which it is specified.
See \secref{sec:detection} for further information.
\item
The filename \textit{main} must be specified without the |.tex| extension.
\item
The filename \textit{main} is case sensitive
(even in case-insensitive file systems)
due to internal string comparison.
\item
The argument \textit{main} should be fully expanded, it cannot be a macro.
\item
Subdirectories and special characters should be avoided in filenames.
\item
The command |\childdocmain{|\textit{main}|}| must be followed by a whitespace.
It should not be followed immediately by another command
or by a comment mark `|%|'.
This is because the \TeX{} parser reads the token immediately following
the argument of |\childdocmain| and puts it
at the beginning of every child section;
however, a white\-space is ignored.
\end{itemize}

%%%%%%%%%%%%%%%%%%%%%%%%%%%%%%%%%%%%%%%%
\paragraph{Content of Main File.}

It is advisable to place all content in the child files included by |\include|.
Any output contained in the main file will appear in all child documents
unless suppressed manually;
it cannot be suppressed automatically by the |\includeonly| directive
and thus should normally be avoided.
A method to include some content in the main file
by means of conditional processing is described in \secref{sec:conditional}.

%%%%%%%%%%%%%%%%%%%%%%%%%%%%%%%%%%%%%%%%
\paragraph{Page Numbering.}

When only a part of the document is compiled,
the appropriate numbering of pages
(as well as other status parameters)
is determined from the |.aux| files.
The latter contain information from previous passes.
However this information needs to propagate through
all intermediate child documents.
Therefore the page numbering in child documents may well
be inconsistent until the complete document is compiled at least once.

A useful (if unconventional) way to always ensure a consistent
page numbering is to restart the numbering in each child document
and denote the pages by `\textit{child}|.|\textit{page}'
where \textit{child} represents the chapter/section number of the child file.
This can be achieved by the command
|\numberwithin{page}{|\textit{child}|}|
of the \textsf{amsmath} package
where \textit{child} can be |chapter| or |section|
depending on the chosen structuring.
Alternatively, one can modify the macro |\thepage| appropriately
and reset the counter |page| at the start of each child file.

%%%%%%%%%%%%%%%%%%%%%%%%%%%%%%%%%%%%%%%%%%%%%%%%%%%%%%%%%%%%%%%%%%%%%%%%%%%%%%%%
\subsection{Conditional Processing}
\label{sec:conditional}

The package provides a mechanism to compile different versions
of a document. To customise the versions further some conditional processing
can come in handy to distinguish which version is being compiled.
The package provides two macros to describe the compilation context:

%%%%%%%%%%%%%%%%%%%%%%%%%%%%%%%%%%%%%%%%
\DescribeMacro{\ifchilddoc}
The conditional |\ifchilddoc| distinguishes between the compilation of
child documents and the main document:
%
\begin{center}
|\ifchilddoc |\textit{child-code}| |[|\||else |\textit{main-code}]| \||fi|
\end{center}

%%%%%%%%%%%%%%%%%%%%%%%%%%%%%%%%%%%%%%%%
\DescribeMacro{\childdocname}
\DescribeMacro{\childdocjob}
The macro |\childdocname| contains the filename (without extension)
of the main or child file being processed.
Note that |\childdocjob| will always contain the name of the main file.

%%%%%%%%%%%%%%%%%%%%%%%%%%%%%%%%%%%%%%%%
\paragraph{Title Page.}

Conditional processing can be used to include a title or banner page
in the main document when proper precautions are taken.
Importantly, the code in the main file should ensure that the page counter
(as well as other status parameters which are stored in the |.aux| files)
takes the same value after the conditional processing.
Otherwise the page numbers may take divergent values
depending on which part is compiled.

For example, a title page could be declared by:
%
\begin{center}
\begin{tabular}{l}
|\ifchilddoc\||else|\\
|\addtocounter{page}{-1}|\\
\textit{code for title page}\\
|\newpage|\\
|\||fi|
\end{tabular}
\end{center}
%
A banner page for the child documents can be generated by:
%
\begin{center}
\begin{tabular}{l}
|\ifchilddoc|\\
|\addtocounter{page}{-1}|\\
\textit{code for banner page}\\
|\newpage|\\
|\||fi|
\end{tabular}
\end{center}
%
Here one could write a message such as:
\begin{center}
|This is the part \childdocname{} of \childdocjob{}.|
\end{center}

%%%%%%%%%%%%%%%%%%%%%%%%%%%%%%%%%%%%%%%%%%%%%%%%%%%%%%%%%%%%%%%%%%%%%%%%%%%%%%%%
\subsection{Flags}
\label{sec:flags}

The package makes it easy to generate different versions
of the main or child documents.
To this end compilation flags can be defined
and assigned different default values.
They will be particularly useful in conjunction
with the forwarding mechanism described in \secref{sec:forward}.

For example, it may be useful to have a flag |\version|
which can be set to |draft| or |final|.
The document source will contain some conditional code
depending on the value of |\version|.
Suppose further, the flag should default to |final| for the main file
and to |draft| for child files
which is a natural assignment for editing the document.
This is achieved by placing the following code
in the preamble of the main document
(below the |\childdocmain| directive):
%
\begin{center}
\begin{tabular}{l}
|\ifchilddoc|\\
|\providecommand{\version}{draft}|\\
|\||else|\\
|\providecommand{\version}{final}|\\
|\||fi|
\end{tabular}
\end{center}
%
The definition by |\providecommand| makes sure
that previous definitions are not overwritten.
Further statements |\providecommand{\version}{...}|
can thus be added before the above code to override it.

For the main file, one might add a line
(between |\childdocmain| and the above block)
%
\begin{center}
|%\ifchilddoc\||else\providecommand{\version}{draft}\||fi|
\end{center}
%
which can be uncommented to produce a draft version.
Likewise one can add a line to the very top of a child file
(above the |\childdocof{|\textit{main}|}| directive)
%
\begin{center}
|%\providecommand{\version}{final}|
\end{center}
%
which can be uncommented to produce the final version of this child document.

%%%%%%%%%%%%%%%%%%%%%%%%%%%%%%%%%%%%%%%%%%%%%%%%%%%%%%%%%%%%%%%%%%%%%%%%%%%%%%%%
\subsection{Forwarding}
\label{sec:forward}

Different versions of the main or child documents
using compilation flags as described in \secref{sec:flags}
can be (permanently) stored in different files
for convenient compilation, viewing and distribution.
To this end, the package defines a command
to pass on compilation to a different file:

%%%%%%%%%%%%%%%%%%%%%%%%%%%%%%%%%%%%%%%%
\DescribeMacro{\childdocforward}
The command |\childdocforward| redirects processing to
another source file:
%
\begin{center}
\begin{tabular}{l}
|\input{childdoc.def}|\\
|\childdocforward[|\textit{main}|]{|\textit{dest}|}|\\
\end{tabular}
\end{center}
%
The argument \textit{dest} is the destination file
(without extension).
It should be the main file or one of the child files.
Note that further \textsf{childdoc} directives
such as |\childdocof| and |\childdocforward|
in the indicated file will be processed in this form.
The optional argument \textit{main}
passes on directly to the main file \textit{main}
while pretending to compile the child \textit{dest}.
This form behaves as if \textit{dest}
issues |\childdocof{|\textit{main}|}| right away,
and no further \textsf{childdoc} directives will be processed.

%%%%%%%%%%%%%%%%%%%%%%%%%%%%%%%%%%%%%%%%
\DescribeMacro{\...prefix}
In the alternative form |\childdocforwardprefix|,
%
\begin{center}
\begin{tabular}{l}
|\input{childdoc.def}|\\
|\childdocforwardprefix[|\textit{main}|]{|\textit{prefix}|}{|\textit{dest}|}|
\end{tabular}
\end{center}
%
the destination file is determined by a pattern
depending on the current file:
To make this work, the current file must be called
`{\textit{prefix}\hspace{0.2em}\textit{suffix}}'
with \textit{prefix} matching precisely the argument.
Processing is then passed on to the file
`{\textit{dest}\hspace{0.2em}\textit{suffix}}'.
Surely, the same effect is achieved by
directly specifying the
argument `{\textit{dest}\hspace{0.2em}\textit{suffix}}'
in the first form.
However, that requires to set up a different file
for each child. With the alternative form of the command
all these files can have exactly the same content
which simplifies setting them up and maintaining them.

For example, the following file |draft.tex|
with a compilation flag |\version| as described in \secref{sec:flags}
compiles the main document as a draft:
%
\begin{center}
\begin{tabular}{l}
|\def\version{draft}|\\
|\input{childdoc.def}|\\
|\childdocforward{|\textit{main}|}|
\end{tabular}
\end{center}
%
Likewise, the following files |final|\textit{nn}|.tex|
compile the final version of the child document
|child|\textit{nn}|.tex|:
%
\begin{center}
\begin{tabular}{l}
|\def\version{final}|\\
|\input{childdoc.def}|\\
|\childdocforwardprefix{final}{child}|
\end{tabular}
\end{center}
%

Note that when several versions of a main file and/or of each child file
are to be generated, it may be convenient to set up a |Makefile| or
shell script to automatise the process.

%%%%%%%%%%%%%%%%%%%%%%%%%%%%%%%%%%%%%%%%%%%%%%%%%%%%%%%%%%%%%%%%%%%%%%%%%%%%%%%%
\subsection{Command Line Processing}
\label{sec:commandline}

The effect of redirection files can also be achieved by invoking
the \LaTeX{} compiler with a more elaborate command line.
Most conveniently this should be done as part
of a shell script or a |Makefile|.

When using \textsf{childdoc} in the main file, the following
command lines effectively perform a redirection
(note that depending on the shell being used,
backslashes may have to be doubled: `|\|' $\to$ `|\\|'):
%
\begin{center}
|... -jobname "|\textit{target}|" |\\|"|[\textit{flags}]%
|\input{childdoc.def}\childdocforward[|\textit{main}|]{|\textit{dest}|}"|
\end{center}
%
Here \textit{target} is the name of the output file,
\textit{main} is the name of the main file
and \textit{dest} is the name of the main or child file to be processed
(all filenames without extensions).
The optional argument \textit{main} can be omitted
if \textit{main} matches \textit{dest}.
Optionally, compilation \textit{flags} can be defined via |\def| commands.
This command line makes the \TeX{} engine believe
it is compiling the file \textit{target}
whose content is specified as the latter parameter.
The provided code then forwards the processing to
\textit{main} or \textit{dest} as described in \secref{sec:forward}.

%%%%%%%%%%%%%%%%%%%%%%%%%%%%%%%%%%%%%%%%%%%%%%%%%%%%%%%%%%%%%%%%%%%%%%%%%%%%%%%%
\subsection{Include by Input}
\label{sec:input}

Including child documents by |\include| has some restrictions by design.
Most notably, the content of a child document always occupies
its own set of pages; pages cannot be shared between child documents.
Usually, this behaviour makes perfect sense
because each child document contain an essential part of the document.
However, in some situations it may be desirable to compose
a document from a collection of parts
without having mandatory page breaks between then.
For this case, the package
provides a mechanism to include parts
by |\input| which can also be processed individually.
However, by construction this mechanism
requires manual handling of the content to be output.

%%%%%%%%%%%%%%%%%%%%%%%%%%%%%%%%%%%%%%%%
\DescribeMacro{\ifchilddocmanual}
The main file should be prepared as usual, see \secref{sec:include}.
However, the document body must make a distinction
between processing of an individual part and of the main document, e.g.:
%
\begin{center}
\begin{tabular}{l}
|\ifchilddocmanual|\\
|\input{\childdocname}|\\
|\||else|\\
\textit{document body with }|\input{|\textit{part}|}|\\
|\||fi|
\end{tabular}
\end{center}
%
The conditional |\ifchilddocmanual| is true whenever
a part to be included by |\input| is being compiled,
and the name of the part is stored in |\childdocname|.

%%%%%%%%%%%%%%%%%%%%%%%%%%%%%%%%%%%%%%%%
\DescribeMacro{\childdocby}
Each part to be included by |\input| should start with:
%
\begin{center}
\begin{tabular}{l}
|\input{childdoc.def}|\\
|\childdocby{|\textit{main}|}|\\
\end{tabular}
\end{center}
%
The directive |\childdocby| is similar to |\childdocof|
described in \secref{sec:include},
but the subsequent selection of content must be done manually.
To that end, both |\ifchilddoc| and |\ifchilddocmanual|
will be true upon processing of a part,
and the name of the part is stored in |\childdocname|.
Note that |\jobname| will be set to the filename of the current part
so that each part receives an individual |.aux| file
that does not interfere with the |.aux| file(s) of the main document.
This behaviour can be altered by the alternative form
|\childdocby[*]{|\textit{main}|}| (with a non-empty optional argument)
which uses the |.aux| file of the main document
by setting |\jobname| to \textit{main}.

%%%%%%%%%%%%%%%%%%%%%%%%%%%%%%%%%%%%%%%%%%%%%%%%%%%%%%%%%%%%%%%%%%%%%%%%%%%%%%%%
\subsection{Driver Development}
\label{sec:driver}

The \textsf{childdoc} mechanism can also be use for the development
of definition files such as \LaTeX{} styles or classes.
This case differs from the above setup with multiple parts
included by |\include| in that no |\includeonly| should be invoked.
This can be achieved by starting the include file
(before |\ProvidesPackage|) with:
%
\begin{center}
\begin{tabular}{l}
|\input{childdoc.def}|\\
|\childdocforward{|\textit{main}|}|\\
\end{tabular}
\end{center}
%
or alternatively with:
%
\begin{center}
\begin{tabular}{l}
|\input{childdoc.def}|\\
|\childdocby{|\textit{main}|}|\\
\end{tabular}
\end{center}
%
Both forms have slightly different effects as described above.
The main file is prepared as usual, see \secref{sec:include}.

%%%%%%%%%%%%%%%%%%%%%%%%%%%%%%%%%%%%%%%%%%%%%%%%%%%%%%%%%%%%%%%%%%%%%%%%%%%%%%%%
\subsection{Legacy Detection}
\label{sec:detection}

The directive |\childdocmain| in the main file can detect
whether the complete document or merely a child is to be compiled
even without using the directive |\childdocof|.
This method is deprecated because it is less robust
and there is no compelling reason to use it;
it is merely provided for backward compatibility
and it may be removed in future versions.

If the detection mechanism is to be used,
it is mandatory to correctly specify
the filename of the main file as the argument of |\childdocmain|:
%
\begin{center}
\begin{tabular}{l}
|\input{childdoc.def}|\\
|\childdocmain{|\textit{main}|}|\\
\end{tabular}
\end{center}
%
If |\jobname| does not match the argument \textit{main} of |\childdocmain|,
it is assumed that |\jobname| points to the child file to be compiled.
When using |\childdocmain| with the main file specified as argument,
it suffices to start a child file
with just |\input{|\textit{main}|}|
without loading of the package and using |\childdocof|.
If instead all processing is done
with the appropriate \textsf{childdoc} directives,
the argument of \textit{main} of |\childdocmain| can be empty.

An alternative version of the command line processing described
in \secref{sec:commandline} using the detection mechanism reads:
%
\begin{center}
|... -jobname "|\textit{target}|" "|[\textit{flags}]%
[|\def\jobname{|\textit{dest}|}|]|\input{|\textit{main}|}"|
\end{center}

%%%%%%%%%%%%%%%%%%%%%%%%%%%%%%%%%%%%%%%%%%%%%%%%%%%%%%%%%%%%%%%%%%%%%%%%%%%%%%%%
\subsection{Manual Code}
\label{sec:manual}

In case one cannot be certain whether the definitions file |childdoc.def|
is installed on the target \TeX{} distribution
and one prefers not to ship it,
it is conceivable to paste a few relevant commands into the sources.

To that end, drop all statements |\input{childdoc.def}|
and perform the replacements as outlined below.
Instead of |\childdocmain{|\textit{main}|}| add the following code
to the top of the main file:
%
\begin{center}
\begin{tabular}{l}
|\||ifdefined\childdocname\endinput\||fi\newif\ifchilddoc|\\
|\edef\childdocname{\scantokens\expandafter{\jobname\noexpand}}|\\
|\def\childdocmain{|\textit{main}|}\||ifx\childdocmain\childdocname\||else|\\
|\childdoctrue\includeonly{\childdocname}\let\jobname\childdocmain\||fi|\\
\end{tabular}
\end{center}
%
Instead of |\childdocof{|\textit{main}|}| just include the main file
at the top of each child file:
%
\begin{center}
|\input{|\textit{main}|}|
\end{center}
%
A simple redirection |\childdocforward{|\textit{dest}|}| is achieved by:
%
\begin{center}
|\def\jobname{|\textit{dest}|}\input{\jobname}|
\end{center}
%
The redirection with prefix
|\childdocforwardprefix[|\textit{prefix}|]{|\textit{dest}|}|
is accomplished by:
%
\begin{center}
\begin{tabular}{l}
|{\edef\jobname{\scantokens\expandafter{\jobname\noexpand}}|\\
|\def\redirectjob |\textit{prefix}|#1~~~{\gdef\jobname{|\textit{dest}|#1}}|\\
|\expandafter\redirectjob\jobname~~~}\input{\jobname}|
\end{tabular}
\end{center}

In an alternative approach,
child documents can be compiled by a specific command line
without additional code or specific definitions:
%
\begin{center}
|... -jobname "|\textit{target}|" "|[\textit{flags}]%
|\includeonly{|\textit{dest}|}\input{|\textit{main}|}"|
\end{center}
%

%%%%%%%%%%%%%%%%%%%%%%%%%%%%%%%%%%%%%%%%%%%%%%%%%%%%%%%%%%%%%%%%%%%%%%%%%%%%%%%%
%%%%%%%%%%%%%%%%%%%%%%%%%%%%%%%%%%%%%%%%%%%%%%%%%%%%%%%%%%%%%%%%%%%%%%%%%%%%%%%%
\section{Information}

%%%%%%%%%%%%%%%%%%%%%%%%%%%%%%%%%%%%%%%%%%%%%%%%%%%%%%%%%%%%%%%%%%%%%%%%%%%%%%%%
\subsection{Copyright}

Copyright \copyright{} 2017--2018 Niklas Beisert

This work may be distributed and/or modified under the
conditions of the \LaTeX{} Project Public License, either version 1.3
of this license or (at your option) any later version.
The latest version of this license is in
  \url{http://www.latex-project.org/lppl.txt}
and version 1.3 or later is part of all distributions of \LaTeX{}
version 2005/12/01 or later.

This work has the LPPL maintenance status `maintained'.

The Current Maintainer of this work is Niklas Beisert.

This work consists of the files |README.txt|, |childdoc.ins| and |childdoc.dtx|
as well as the derived files |childdoc.def|, |cdocsamp.tex|
with |cdocsch1.tex|, |cdocsch2.tex|, |cdocspt3.tex|, |cdocspt4.tex|,
|cdocsdrf.tex|, |cdocsfn1.tex|, |cdocsfn2.tex|
as well as |childdoc.pdf|.

%%%%%%%%%%%%%%%%%%%%%%%%%%%%%%%%%%%%%%%%%%%%%%%%%%%%%%%%%%%%%%%%%%%%%%%%%%%%%%%%
\subsection{Files and Installation}

The package consists of the files:
%
\begin{center}
\begin{tabular}{ll}
    |README.txt|   & readme file \\
    |childdoc.ins| & installation file \\
    |childdoc.dtx| & source file \\
    |childdoc.def| & definition file \\
    |cdocsamp.tex| & sample main file \\
    |cdocsch1.tex| & sample include file \\
    |cdocsch2.tex| & sample include file \\
    |cdocspt3.tex| & sample part file \\
    |cdocspt4.tex| & sample part file \\
    |cdocsdrf.tex| & sample redirection file \\
    |cdocsfn1.tex| & sample redirection file \\
    |cdocsfn2.tex| & sample redirection file \\
    |childdoc.pdf| & manual
\end{tabular}
\end{center}
%
The distribution consists of the files
|README.txt|, |childdoc.ins| and |childdoc.dtx|.
%
\begin{itemize}
\item
Run (pdf)\LaTeX{} on |childdoc.dtx|
to compile the manual |childdoc.pdf| (this file).
\item
Run \LaTeX{} on |childdoc.ins| to create the definitions file |childdoc.def|
and the sample |cdocsamp.tex| with include files
|cdocsch1.tex|, |cdocsch2.tex|, |cdocspt3.tex|, |cdocspt4.tex|,
|cdocsdrf.tex|, |cdocsfn1.tex|, |cdocsfn2.tex|.
Then copy the file |childdoc.def| to an appropriate directory of your \LaTeX{}
distribution, e.g.\ \textit{texmf-root}|/tex/latex/childdoc|.
\end{itemize}

%%%%%%%%%%%%%%%%%%%%%%%%%%%%%%%%%%%%%%%%%%%%%%%%%%%%%%%%%%%%%%%%%%%%%%%%%%%%%%%%
\subsection{Related CTAN Packages}

There are several other packages which offer a similar functionality:
%
\begin{itemize}
\item
The packages
\href{http://ctan.org/pkg/docmute}{\textsf{docmute}},
\href{http://ctan.org/pkg/includex}{\textsf{includex}} and
\href{http://ctan.org/pkg/standalone}{\textsf{standalone}}
provide commands to include only the document body of
a child file thus allowing both files to be compiled individually.
\item
The packages \href{http://ctan.org/pkg/subdocs}{\textsf{subdocs}}
and \href{http://ctan.org/pkg/subfiles}{\textsf{subfiles}}
provide structures in which the main and child documents can be
encapsulated and allowing them to be compiled individually.
The inclusion mechanism is different from the conventional |\include|.
\item
The package \href{http://ctan.org/pkg/combine}{\textsf{combine}}
is an elaborate solution to combine several documents into one.
\end{itemize}
%
See also the CTAN topic \href{http://ctan.org/topic/subdocs}{\textsf{subdocs}}
for further related packages.
The present package differs from the above solutions in that
a document structure constructed with the conventional |\include| mechanism
just needs two extra commands at the top of every file
such that all constituent files can be compiled individually.

%%%%%%%%%%%%%%%%%%%%%%%%%%%%%%%%%%%%%%%%%%%%%%%%%%%%%%%%%%%%%%%%%%%%%%%%%%%%%%%%
%\subsection{Feature Suggestions}
%
%The following is a list of features which may be useful for future
%versions of this package:
%%
%\begin{itemize}
%\item
%\ldots
%\end{itemize}

%%%%%%%%%%%%%%%%%%%%%%%%%%%%%%%%%%%%%%%%%%%%%%%%%%%%%%%%%%%%%%%%%%%%%%%%%%%%%%%%
\subsection{Revision History}

%%%%%%%%%%%%%%%%%%%%%%%%%%%%%%%%%%%%%%%%
\paragraph{v2.0:} 2018/12/30

\begin{itemize}
\item
immediate forward processing
\item
added |\childdocby| mechanism
\item
manual restructured
\end{itemize}

%%%%%%%%%%%%%%%%%%%%%%%%%%%%%%%%%%%%%%%%
\paragraph{v1.6:} 2018/01/17

\begin{itemize}
\item
application for development of include files
\item
corrections to manual
\end{itemize}

%%%%%%%%%%%%%%%%%%%%%%%%%%%%%%%%%%%%%%%%
\paragraph{v1.5:} 2017/05/21

\begin{itemize}
\item
more complete structuring introduced
\item
|\childdocof| introduced
\item
|\childdoc| renamed to |\childdocmain|
\item
|\childredirect| renamed to |\childdocforward| and |\childdocforwardprefix|
and functionality expanded
\end{itemize}

%%%%%%%%%%%%%%%%%%%%%%%%%%%%%%%%%%%%%%%%
\paragraph{v1.0:} 2017/04/27

\begin{itemize}
\item
manual and install package
\item
first version published on CTAN
\end{itemize}

%%%%%%%%%%%%%%%%%%%%%%%%%%%%%%%%%%%%%%%%
\paragraph{v0.6:} 2017/04/26

\begin{itemize}
\item
redirection mechanism added
\end{itemize}

%%%%%%%%%%%%%%%%%%%%%%%%%%%%%%%%%%%%%%%%
\paragraph{v0.5:} 2017/04/26

\begin{itemize}
\item
functionality in definition file
\end{itemize}


%%%%%%%%%%%%%%%%%%%%%%%%%%%%%%%%%%%%%%%%%%%%%%%%%%%%%%%%%%%%%%%%%%%%%%%%%%%%%%%%
%%%%%%%%%%%%%%%%%%%%%%%%%%%%%%%%%%%%%%%%%%%%%%%%%%%%%%%%%%%%%%%%%%%%%%%%%%%%%%%%
%%%%%%%%%%%%%%%%%%%%%%%%%%%%%%%%%%%%%%%%%%%%%%%%%%%%%%%%%%%%%%%%%%%%%%%%%%%%%%%%
\appendix

\settowidth\MacroIndent{\rmfamily\scriptsize 000\ }

 \DocInput{childdoc.dtx}

\end{document}
%</driver>
% \fi
%
% %%%%%%%%%%%%%%%%%%%%%%%%%%%%%%%%%%%%%%%%%%%%%%%%%%%%%%%%%%%%%%%%%%%%%%%%%%%%%%
% %%%%%%%%%%%%%%%%%%%%%%%%%%%%%%%%%%%%%%%%%%%%%%%%%%%%%%%%%%%%%%%%%%%%%%%%%%%%%%
% \section{Sample}
%\iffalse
%<*samplemain>
%\fi
%
% The following presents a sample document
% with two chapters, two parts, a title page,
% a compile flag as well as three forwarding files to set the flag.
% It consists of eight |.tex| files:
% \begin{center}
% \begin{tabular}{ll}
% |cdocsamp.tex|&main file\\
% |cdocsch1.tex|&include file for chapter 1\\
% |cdocsch2.tex|&include file for chapter 2\\
% |cdocspt3.tex|&include file for part 3\\
% |cdocspt4.tex|&include file for part 4\\
% |cdocsdrf.tex|&forwarding file for main file in draft mode\\
% |cdocsfi1.tex|&forwarding file for final version of chapter 1\\
% |cdocsfi2.tex|&forwarding file for final version of chapter 2\\
% \end{tabular}
% \end{center}
% Each of the eight files can be compiled directly by the \LaTeX{} compiler.
%
% %%%%%%%%%%%%%%%%%%%%%%%%%%%%%%%%%%%%%%
% \paragraph{Main File.}
%
% The main file is called |cdocsamp.tex|.
%
% Load the \textsf{childdoc} definitions and
% declare the filename for the main document:
%    \begin{macrocode}
\input{childdoc.def}
\childdocmain{}
%    \end{macrocode}

% Optional override for |\version| flag:
%    \begin{macrocode}
%%\ifchilddoc\else\providecommand{\version}{draft}\fi
%    \end{macrocode}

% Define the default values for the |\version| flag
% (|final| for the main file and |draft| for childs):
%    \begin{macrocode}
\ifchilddoc
\providecommand{\version}{draft}
\else
\providecommand{\version}{final}
\fi
%    \end{macrocode}

% Load the standard document class:
%    \begin{macrocode}
\documentclass[12pt]{article}
%    \end{macrocode}

% Start the document body:
%    \begin{macrocode}
\begin{document}
%    \end{macrocode}

% Declare a title page.
% Print title, part of document being processed and version flag:
%    \begin{macrocode}
\addtocounter{page}{-1}
\begin{center}
{\LARGE\bfseries{}childdoc example\par}
\vspace{1cm}
\ifchilddoc
\ifchilddocmanual part\else chapter\fi:
`\childdocname' of `\childdocjob'\par
\else
main document: `\childdocjob'\par
\fi
version: \version\par
\end{center}
\newpage
%    \end{macrocode}

% Manually include selected file,
% otherwise process as usual:
%    \begin{macrocode}
\ifchilddocmanual
\section*{part `\childdocname'}
\input{\childdocname}
\else
%    \end{macrocode}

% Include the two chapters:
%    \begin{macrocode}
\include{cdocsch1}
\include{cdocsch2}
%    \end{macrocode}

% Include the two parts unless only chapters should be displayed:
%    \begin{macrocode}
\ifchilddoc\else
\section{part three}
\input{cdocspt3}
\section{part four}
\input{cdocspt4}
\fi
%    \end{macrocode}

% Process as usual until here:
%    \begin{macrocode}
\fi
%    \end{macrocode}

% End of document body:
%    \begin{macrocode}
\end{document}
%    \end{macrocode}
%\iffalse
%</samplemain>
%\fi
%
% %%%%%%%%%%%%%%%%%%%%%%%%%%%%%%%%%%%%%%
% \paragraph{Chapter Include Files.}
%
% The include files are called |cdocsch1.tex| and |cdocsch2.tex|.
%
%\iffalse
%<*samplechap1|samplechap2>
%\fi

% Optional override for |\version| flag:
%    \begin{macrocode}
%%\providecommand{\version}{final}
%    \end{macrocode}

% Include the main document:
%    \begin{macrocode}
\input{childdoc.def}
\childdocof{cdocsamp}
%    \end{macrocode}

%\iffalse
%</samplechap1|samplechap2>
%\fi
%
%\iffalse
%<*samplechap1>
%\fi
% Some text for chapter 1:
%    \begin{macrocode}
\section{one}
some text in chapter one
%    \end{macrocode}

%\iffalse
%</samplechap1>
%\fi
% Some text for chapter 2:
%\iffalse
%<*samplechap2>
%\fi
%    \begin{macrocode}
\section{two}
more text in chapter two
%    \end{macrocode}

%\iffalse
%</samplechap2>
%\fi
%
% %%%%%%%%%%%%%%%%%%%%%%%%%%%%%%%%%%%%%%
% \paragraph{Part Include Files.}
%
% The include files are called |cdocspt3.tex| and |cdocspt4.tex|.
%
%\iffalse
%<*samplepart3|samplepart4>
%\fi

% Optional override for |\version| flag:
%    \begin{macrocode}
%%\providecommand{\version}{final}
%    \end{macrocode}

% Include the main document:
%    \begin{macrocode}
\input{childdoc.def}
\childdocby{cdocsamp}
%    \end{macrocode}

%\iffalse
%</samplepart3|samplepart4>
%\fi
%
%\iffalse
%<*samplepart3>
%\fi
% Some text for part 3:
%    \begin{macrocode}
some text in part three
%    \end{macrocode}

%\iffalse
%</samplepart3>
%\fi
% Some text for part 4:
%\iffalse
%<*samplepart4>
%\fi
%    \begin{macrocode}
more text in part four
%    \end{macrocode}

%\iffalse
%</samplepart4>
%\fi
%
% %%%%%%%%%%%%%%%%%%%%%%%%%%%%%%%%%%%%%%
% \paragraph{Forwarding for a Complete Draft.}
%
% The following forwarding file |cdocsdrf.tex|
% compiles the main document in draft mode:
%\iffalse
%<*sampledraft>
%\fi
%    \begin{macrocode}
\def\version{draft}
\input{childdoc.def}
\childdocforward{cdocsamp}
%    \end{macrocode}

%\iffalse
%</sampledraft>
%\fi
%
% %%%%%%%%%%%%%%%%%%%%%%%%%%%%%%%%%%%%%%
% \paragraph{Forwarding for Final Version of the Chapters.}
%
% The following forwarding files |cdocsfn1.tex| and |cdocsfn2.tex|
% (with identical content)
% compile the final versions of the child documents
% |cdocsch1.tex| and |cdocsch2.tex|, respectively:
%\iffalse
%<*samplefinal>
%\fi
%    \begin{macrocode}
\def\version{final}
\input{childdoc.def}
\childdocforwardprefix[cdocsamp]{cdocsfn}{cdocsch}
%    \end{macrocode}

%\iffalse
%</samplefinal>
%\fi
%
% %%%%%%%%%%%%%%%%%%%%%%%%%%%%%%%%%%%%%%
% \paragraph{Command Line Processing.}
%
% The following three command lines generate the output files
% |cdocscld|, |cdocscl1| and |cdocscl2|
% which should be identical to
% |cdocsdrf|, |cdocsch1| and |cdocsfn2|, respectively:
% \begin{center}
% \begin{tabular}{l}
% |latex -jobname cdocscld \|\\
% |  "\def\version{draft}\input{childdoc.def}\childdocforward{cdocsamp}"|\\
% |latex -jobname cdocscl1 \|\\
% |  "\input{childdoc.def}\childdocforward[cdocsamp]{cdocsch1}"|\\
% |latex -jobname cdocscl2 \|\\
% |  "\def\version{final}\input{childdoc.def}\childdocforward{cdocsch2}"|
% \end{tabular}
% \end{center}
% Note that the trailing backslash on each first line
% merely continues the input to the second line
% (for convenient cut ant paste).
% Furthermore, the command |latex| can be replaced by any
% of its alternative versions such as |pdflatex|.
%
% %%%%%%%%%%%%%%%%%%%%%%%%%%%%%%%%%%%%%%%%%%%%%%%%%%%%%%%%%%%%%%%%%%%%%%%%%%%%%%
% %%%%%%%%%%%%%%%%%%%%%%%%%%%%%%%%%%%%%%%%%%%%%%%%%%%%%%%%%%%%%%%%%%%%%%%%%%%%%%
% \section{Implementation}
%\iffalse
%<*package>
%\fi
%
% This section describes the definitions file |childdoc.def|.

% The definitions cannot be loaded using |\usepackage| or |\RequirePackage|
% which has a mechanism to prevent loading a style file more than once.
% When loading the definitions by means of |\input|
% multiple instances have to be prevented manually:
%\iffalse
%This code needs to be before the `\ProvidesFile' directive
%which is defined at the beginning of this file.
%Therefore it is also placed there and commented out here.
%</package>
%<*discard>
%\fi
%    \begin{macrocode}
\ifdefined\childdocmain\endinput\fi
%    \end{macrocode}
%\iffalse
%</discard>
%<*package>
%\fi
%
% \macro{\ifchilddoc}
% \macro{\ifchilddocmanual}
% The conditional |\ifchilddoc| tells whether a
% child (true) or main (false) document is being compiled.
% The conditional |\ifchilddocmanual| tells whether
% the |\includeonly| mechanism is used (false) or
% the selection of child files must be performed manually (true).
% The definitions initialise to false:
%    \begin{macrocode}
\newif\ifchilddoc
\newif\ifchilddocmanual
%    \end{macrocode}

% \macro{\childdocname}
% \macro{\childdocjob}
% The macro |\childdocname| stores the name of the main document
% to be compiled. The macro |\childdocjob| stores the name of
% the document on which the \LaTeX{} compiler was originally invoked.
% The content of |\jobname| cannot be compared
% to filenames specified in the source due to different catcodes.
% The following code rescans |\jobname|, stores the result
% in |\childdocname| and saves a copy in |\childdocjob|:
%    \begin{macrocode}
\edef\childdocname{\scantokens\expandafter{\jobname\noexpand}}
\let\childdocjob\childdocname
%    \end{macrocode}

% \macro{\childdocdisable}
% The macro |\childdocdisable| prevents the main file
% from being processed more than once.
% At this stage, the main document command |\childdocmain|
% is assumed to be called once again where it should do nothing.
% Any subsequent call to it should prevent
% a secondary processing of the main document
% It overwrites the forwarding commands
% |\childdocof| and |\childdocforward|
% with empty macros to prevent further inclusions of the main document:
%    \begin{macrocode}
\newcommand{\childdocdisable}
{
  \renewcommand{\childdocmain}[1]{\renewcommand{\childdocmain}[1]{\endinput}}
  \renewcommand{\childdocof}[1]{}
  \renewcommand{\childdocby}[2][]{}
  \renewcommand{\childdocforward}[2][]{}
  \renewcommand{\childdocdisable}{}
}
%    \end{macrocode}

% \macro{\childdocmain}
% The macro |\childdocmain| is to be called at the top of the main file
% with nothing or the main filename (without extension) as argument.
% First, it breaks loops.
% If the argument is not empty and does not match |\childdocname|
% (which is set by the first inclusion of |childdoc.def|),
% |\ifchilddoc| is set to true, |\includeonly| is applied to the child file
% and |\jobname| is set to the main file
% (for proper handling of |.aux| files):
%    \begin{macrocode}
\newcommand{\childdocmain}[1]
{
  \childdocdisable\childdocmain{}
  \if?#1?\else
    \begingroup
      \def\childdoctmp{#1}
      \ifx\childdoctmp\childdocname
        \def\childdoctmp{}
      \else
        \def\childdoctmp
        {
          \childdoctrue
          \includeonly{\childdocname}
          \def\childdocjob{#1}
          \def\jobname{#1}
        }
      \fi
      \expandafter
    \endgroup
    \childdoctmp
  \fi
}
%    \end{macrocode}

% \macro{\childdocof}
% The command |\childdocof| redirects
% compilation to the main file |#1|.
%    \begin{macrocode}
\newcommand{\childdocof}[1]
{
  \childdocdisable
  \childdoctrue
  \includeonly{\childdocname}
  \def\jobname{#1}
  \def\childdocjob{#1}
  \input{#1}
}
%    \end{macrocode}

% \macro{\childdocby}
% The command |\childdocby| ....
%    \begin{macrocode}
\newcommand{\childdocby}[2][]
{
  \childdocdisable
  \childdoctrue
  \childdocmanualtrue
  \if?#1?\else
    \def\jobname{#2}
  \fi
  \def\childdocjob{#2}
  \input{#2}
  \endinput
}
%    \end{macrocode}

% \macro{\childdocforward}
% The command |\childdocforward| redirects
% compilation to the main file or
% (if the optional argument is given) a child file.
% Parameters are set as if the main file
% or a child file starting with |\childdocof| was compiled.
% Then compilation is handed over to the main file:
%    \begin{macrocode}
\newcommand{\childdocforward}[2][]
{
  \begingroup
    \if?#1?
      \def\childdoctmp
      {
        \def\childdocname{#2}
        \def\childdocjob{#2}
        \def\jobname{#2}
        \input{#2}
        \endinput
      }
    \else
      \def\childdoctmp
      {
        \childdocdisable
        \def\childdocname{#2}
        \childdoctrue
        \includeonly{#2}
        \def\childdocjob{#1}
        \def\jobname{#1}
        \input{#1}
        \endinput
      }
    \fi
    \expandafter
  \endgroup
  \childdoctmp
}
%    \end{macrocode}

% \macro{\childdocforwardprefix}
% The command |\childdocforwardprefix| redirects
% compilation to the main or a child file by means of a pattern.
% The prefix |#1| in the current filename is replaced by |#2|
% and the suffix of the current filename is kept
% (it is assumed that the filename does not contain the substring `|~~~|'
% which is used as a delimiter).
% Compilation is handed over to the new file by |\childdocforward|:
%    \begin{macrocode}
\newcommand{\childdocforwardprefix}[3][]
{
  \begingroup
    \def\childdocextract #2##1~~~{\def\childdoctmp{\childdocforward[#1]{#3##1}}}
    \expandafter\childdocextract\childdocname~~~
    \expandafter
  \endgroup
  \childdoctmp
}
%    \end{macrocode}

% \macro{\childdoc}
% The deprecated macro |\childdoc| is a legacy version of |\childdocmain|:
%    \begin{macrocode}
\newcommand{\childdoc}{\childdocmain}
%    \end{macrocode}

% \macro{\childdocredirect}
% The deprecated macro |\childdocredirect| is a legacy version
% of |\childdocforward| and |\childdocforwardprefix|:
%    \begin{macrocode}
\newcommand{\childdocredirect}[2][]
{
  \begingroup
    \if?#1?
      \def\childdoctmp{\childdocforward{#2}}
    \else
      \def\childdoctmp{\childdocforwardprefix{#1}{#2}}
    \fi
    \expandafter
  \endgroup
  \childdoctmp
}
%    \end{macrocode}

%\iffalse
%</package>
%\fi
%
\endinput
|\\
|\childdocforwardprefix[|\textit{main}|]{|\textit{prefix}|}{|\textit{dest}|}|
\end{tabular}
\end{center}
%
the destination file is determined by a pattern
depending on the current file:
To make this work, the current file must be called
`{\textit{prefix}\hspace{0.2em}\textit{suffix}}'
with \textit{prefix} matching precisely the argument.
Processing is then passed on to the file
`{\textit{dest}\hspace{0.2em}\textit{suffix}}'.
Surely, the same effect is achieved by
directly specifying the
argument `{\textit{dest}\hspace{0.2em}\textit{suffix}}'
in the first form.
However, that requires to set up a different file
for each child. With the alternative form of the command
all these files can have exactly the same content
which simplifies setting them up and maintaining them.

For example, the following file |draft.tex|
with a compilation flag |\version| as described in \secref{sec:flags}
compiles the main document as a draft:
%
\begin{center}
\begin{tabular}{l}
|\def\version{draft}|\\
|% \iffalse
%
% childdoc.dtx Copyright (C) 2017-2018 Niklas Beisert
%
% This work may be distributed and/or modified under the
% conditions of the LaTeX Project Public License, either version 1.3
% of this license or (at your option) any later version.
% The latest version of this license is in
%   http://www.latex-project.org/lppl.txt
% and version 1.3 or later is part of all distributions of LaTeX
% version 2005/12/01 or later.
%
% This work has the LPPL maintenance status `maintained'.
%
% The Current Maintainer of this work is Niklas Beisert.
%
% This work consists of the files childdoc.dtx and childdoc.ins
% and the derived files childdoc.def and cdocsamp.tex with
% cdocsch1.tex, cdocsch2.tex, cdocsdrf.tex, cdocsfn1.tex, cdocsfn2.tex.
%
%<package>\ifdefined\childdocmain\endinput\fi
%<package>\ProvidesFile{childdoc.def}[2018/12/30 v2.0 child document driver]
%<samplemain>\ProvidesFile{cdocsamp.tex}[2018/12/30 v2.0 sample for childdoc]
%<*driver>
%\ProvidesFile{childdoc.drv}[2018/12/30 v2.0 childdoc reference manual file]
\PassOptionsToClass{10pt,a4paper}{article}
\documentclass{ltxdoc}

\usepackage[margin=35mm]{geometry}
\usepackage{hyperref}
\usepackage{hyperxmp}
\usepackage[usenames]{color}

\hypersetup{colorlinks=true}
\hypersetup{pdfstartview=FitH}
\hypersetup{pdfpagemode=UseNone}
\hypersetup{pdfsource={}}
\hypersetup{pdflang={en-UK}}
\hypersetup{pdfcopyright={Copyright 2017-2018 Niklas Beisert.
  This work may be distributed and/or modified under the
  conditions of the LaTeX Project Public License, either version 1.3
  of this license or (at your option) any later version.}}
\hypersetup{pdflicenseurl={http://www.latex-project.org/lppl.txt}}
\hypersetup{pdfcontactaddress={ETH Zurich, ITP, HIT K,
  Wolfgang-Pauli-Strasse 27}}
\hypersetup{pdfcontactpostcode={8093}}
\hypersetup{pdfcontactcity={Zurich}}
\hypersetup{pdfcontactcountry={Switzerland}}
\hypersetup{pdfcontactemail={nbeisert@itp.phys.ethz.ch}}
\hypersetup{pdfcontacturl={http://people.phys.ethz.ch/\xmptilde nbeisert/}}

\newcommand{\secref}[1]{\hyperref[#1]{section \ref*{#1}}}

\parskip1ex
\parindent0pt
\let\olditemize\itemize
\def\itemize{\olditemize\parskip0pt}

\begin{document}

\title{The \textsf{childdoc} Package}
\hypersetup{pdftitle={The childdoc Package}}
\author{Niklas Beisert\\[2ex]
  Institut f\"ur Theoretische Physik\\
  Eidgen\"ossische Technische Hochschule Z\"urich\\
  Wolfgang-Pauli-Strasse 27, 8093 Z\"urich, Switzerland\\[1ex]
  \href{mailto:nbeisert@itp.phys.ethz.ch}
  {\texttt{nbeisert@itp.phys.ethz.ch}}}
\hypersetup{pdfauthor={Niklas Beisert}}
\hypersetup{pdfsubject={Manual for the LaTeX2e Package childdoc}}
\date{30 December 2018, \textsf{v2.0}}
\maketitle

\begin{abstract}\noindent
\textsf{childdoc} is a \LaTeXe{} package
that enables the direct compilation
of document sections included by |\include|
to individual files.
\end{abstract}

\begingroup
\parskip0ex
\tableofcontents
\endgroup

%%%%%%%%%%%%%%%%%%%%%%%%%%%%%%%%%%%%%%%%%%%%%%%%%%%%%%%%%%%%%%%%%%%%%%%%%%%%%%%%
%%%%%%%%%%%%%%%%%%%%%%%%%%%%%%%%%%%%%%%%%%%%%%%%%%%%%%%%%%%%%%%%%%%%%%%%%%%%%%%%
\section{Introduction}

\LaTeX{} provides a mechanism to structure a large document (such as a book)
into a main file and several child files (containing the chapters)
using the |\include| command.
This mechanism is beneficial for documents
which span hundreds of pages in order to
make the source file(s) more manageable.
Moreover, compilation can be restricted to
selected child files by means of the |\includeonly| command.
The latter feature can be used to reduce the compilation time while editing
(this was significantly more useful in the earlier days of \LaTeX{})
or to generate a smaller document which is easier to navigate.
Another application of |\includeonly| is to generate
documents consisting of selected parts of the complete document.

However, there are a few drawbacks of the plain |\include| mechanism:
\begin{itemize}
\item
The child files cannot be compiled on their own,
they can only be compiled via the main file.
A naive editing environment
(such as a text editor with an option
to have the current file processed by \LaTeX)
may require one to switch to the main file before compiling;
attempting to compile the child file produces errors.
\item
The main file must be modified (each time)
to adjust the |\includeonly| command
to the present needs. This easily leaves the main file in a messy state.
\item
The generated document will always carry the filename
of the main document. This is inconvenient if
several child files are to be compiled and
to be kept for distribution.
\end{itemize}

The present package provides a simple interface
to make child files individually compilable by \LaTeX{}.
Compiling a child file then has the same effect as compiling
the main file with an |\includeonly| command
to select the appropriate child.
Moreover the generated document will carry the name of the child
rather than the main file.
This resolves all three above issues.

This feature is meant to make the editing of books,
thesis documents and lecture notes somewhat more convenient.
However, the package can also be used efficiently for
composing a series of documents (such as exercise sheets)
which are typically distributed individually.
It then assists the author in generating the individual documents
(potentially in different versions)
as well as a document containing the collected series.
Another application is in developing style files
or other kinds of included material
where compilation of the style file could redirect
to a sample or test file.

%%%%%%%%%%%%%%%%%%%%%%%%%%%%%%%%%%%%%%%%%%%%%%%%%%%%%%%%%%%%%%%%%%%%%%%%%%%%%%%%
%%%%%%%%%%%%%%%%%%%%%%%%%%%%%%%%%%%%%%%%%%%%%%%%%%%%%%%%%%%%%%%%%%%%%%%%%%%%%%%%
\section{Usage}

First of all, the package \textsf{childdoc} is \emph{not} a standard
\LaTeXe{} |.sty| style file! Therefore it needs to be invoked in
a non-standard way.

%%%%%%%%%%%%%%%%%%%%%%%%%%%%%%%%%%%%%%%%%%%%%%%%%%%%%%%%%%%%%%%%%%%%%%%%%%%%%%%%
\subsection{Included Files}
\label{sec:include}

%%%%%%%%%%%%%%%%%%%%%%%%%%%%%%%%%%%%%%%%
\DescribeMacro{\childdocmain}
To use the package, add the commands
\begin{center}
\begin{tabular}{l}
|\input{childdoc.def}|\\
|\childdocmain{}|\\
\end{tabular}
\end{center}
at the very top of the main \LaTeX{} file,
in particular \emph{before} the |\documentclass| statement!
The argument of |\childdocmain| should be left empty
(but it must be present).

%%%%%%%%%%%%%%%%%%%%%%%%%%%%%%%%%%%%%%%%
\DescribeMacro{\childdocof}
Furthermore, add the commands
\begin{center}
\begin{tabular}{l}
|\input{childdoc.def}|\\
|\childdocof{|\textit{main}|}|\\
\end{tabular}
\end{center}
at the top of every child file \textit{child}
which is included by |\include{|\textit{child}|}|
from within the main file
(or at least for those files to be compiled individually).
The argument \textit{main} must be the filename of the main file.

There are a couple of
considerations in setting up the main and child documents:

%%%%%%%%%%%%%%%%%%%%%%%%%%%%%%%%%%%%%%%%
\paragraph{Restrictions.}

Please note the following restrictions:
\begin{itemize}
\item
|\childdocmain| must be called with one argument \textit{main}
to ensure compatibility with earlier version of the package.
It must either be empty (|\childdocmain{}|)
or precisely match the filename of the main file in which it is specified.
See \secref{sec:detection} for further information.
\item
The filename \textit{main} must be specified without the |.tex| extension.
\item
The filename \textit{main} is case sensitive
(even in case-insensitive file systems)
due to internal string comparison.
\item
The argument \textit{main} should be fully expanded, it cannot be a macro.
\item
Subdirectories and special characters should be avoided in filenames.
\item
The command |\childdocmain{|\textit{main}|}| must be followed by a whitespace.
It should not be followed immediately by another command
or by a comment mark `|%|'.
This is because the \TeX{} parser reads the token immediately following
the argument of |\childdocmain| and puts it
at the beginning of every child section;
however, a white\-space is ignored.
\end{itemize}

%%%%%%%%%%%%%%%%%%%%%%%%%%%%%%%%%%%%%%%%
\paragraph{Content of Main File.}

It is advisable to place all content in the child files included by |\include|.
Any output contained in the main file will appear in all child documents
unless suppressed manually;
it cannot be suppressed automatically by the |\includeonly| directive
and thus should normally be avoided.
A method to include some content in the main file
by means of conditional processing is described in \secref{sec:conditional}.

%%%%%%%%%%%%%%%%%%%%%%%%%%%%%%%%%%%%%%%%
\paragraph{Page Numbering.}

When only a part of the document is compiled,
the appropriate numbering of pages
(as well as other status parameters)
is determined from the |.aux| files.
The latter contain information from previous passes.
However this information needs to propagate through
all intermediate child documents.
Therefore the page numbering in child documents may well
be inconsistent until the complete document is compiled at least once.

A useful (if unconventional) way to always ensure a consistent
page numbering is to restart the numbering in each child document
and denote the pages by `\textit{child}|.|\textit{page}'
where \textit{child} represents the chapter/section number of the child file.
This can be achieved by the command
|\numberwithin{page}{|\textit{child}|}|
of the \textsf{amsmath} package
where \textit{child} can be |chapter| or |section|
depending on the chosen structuring.
Alternatively, one can modify the macro |\thepage| appropriately
and reset the counter |page| at the start of each child file.

%%%%%%%%%%%%%%%%%%%%%%%%%%%%%%%%%%%%%%%%%%%%%%%%%%%%%%%%%%%%%%%%%%%%%%%%%%%%%%%%
\subsection{Conditional Processing}
\label{sec:conditional}

The package provides a mechanism to compile different versions
of a document. To customise the versions further some conditional processing
can come in handy to distinguish which version is being compiled.
The package provides two macros to describe the compilation context:

%%%%%%%%%%%%%%%%%%%%%%%%%%%%%%%%%%%%%%%%
\DescribeMacro{\ifchilddoc}
The conditional |\ifchilddoc| distinguishes between the compilation of
child documents and the main document:
%
\begin{center}
|\ifchilddoc |\textit{child-code}| |[|\||else |\textit{main-code}]| \||fi|
\end{center}

%%%%%%%%%%%%%%%%%%%%%%%%%%%%%%%%%%%%%%%%
\DescribeMacro{\childdocname}
\DescribeMacro{\childdocjob}
The macro |\childdocname| contains the filename (without extension)
of the main or child file being processed.
Note that |\childdocjob| will always contain the name of the main file.

%%%%%%%%%%%%%%%%%%%%%%%%%%%%%%%%%%%%%%%%
\paragraph{Title Page.}

Conditional processing can be used to include a title or banner page
in the main document when proper precautions are taken.
Importantly, the code in the main file should ensure that the page counter
(as well as other status parameters which are stored in the |.aux| files)
takes the same value after the conditional processing.
Otherwise the page numbers may take divergent values
depending on which part is compiled.

For example, a title page could be declared by:
%
\begin{center}
\begin{tabular}{l}
|\ifchilddoc\||else|\\
|\addtocounter{page}{-1}|\\
\textit{code for title page}\\
|\newpage|\\
|\||fi|
\end{tabular}
\end{center}
%
A banner page for the child documents can be generated by:
%
\begin{center}
\begin{tabular}{l}
|\ifchilddoc|\\
|\addtocounter{page}{-1}|\\
\textit{code for banner page}\\
|\newpage|\\
|\||fi|
\end{tabular}
\end{center}
%
Here one could write a message such as:
\begin{center}
|This is the part \childdocname{} of \childdocjob{}.|
\end{center}

%%%%%%%%%%%%%%%%%%%%%%%%%%%%%%%%%%%%%%%%%%%%%%%%%%%%%%%%%%%%%%%%%%%%%%%%%%%%%%%%
\subsection{Flags}
\label{sec:flags}

The package makes it easy to generate different versions
of the main or child documents.
To this end compilation flags can be defined
and assigned different default values.
They will be particularly useful in conjunction
with the forwarding mechanism described in \secref{sec:forward}.

For example, it may be useful to have a flag |\version|
which can be set to |draft| or |final|.
The document source will contain some conditional code
depending on the value of |\version|.
Suppose further, the flag should default to |final| for the main file
and to |draft| for child files
which is a natural assignment for editing the document.
This is achieved by placing the following code
in the preamble of the main document
(below the |\childdocmain| directive):
%
\begin{center}
\begin{tabular}{l}
|\ifchilddoc|\\
|\providecommand{\version}{draft}|\\
|\||else|\\
|\providecommand{\version}{final}|\\
|\||fi|
\end{tabular}
\end{center}
%
The definition by |\providecommand| makes sure
that previous definitions are not overwritten.
Further statements |\providecommand{\version}{...}|
can thus be added before the above code to override it.

For the main file, one might add a line
(between |\childdocmain| and the above block)
%
\begin{center}
|%\ifchilddoc\||else\providecommand{\version}{draft}\||fi|
\end{center}
%
which can be uncommented to produce a draft version.
Likewise one can add a line to the very top of a child file
(above the |\childdocof{|\textit{main}|}| directive)
%
\begin{center}
|%\providecommand{\version}{final}|
\end{center}
%
which can be uncommented to produce the final version of this child document.

%%%%%%%%%%%%%%%%%%%%%%%%%%%%%%%%%%%%%%%%%%%%%%%%%%%%%%%%%%%%%%%%%%%%%%%%%%%%%%%%
\subsection{Forwarding}
\label{sec:forward}

Different versions of the main or child documents
using compilation flags as described in \secref{sec:flags}
can be (permanently) stored in different files
for convenient compilation, viewing and distribution.
To this end, the package defines a command
to pass on compilation to a different file:

%%%%%%%%%%%%%%%%%%%%%%%%%%%%%%%%%%%%%%%%
\DescribeMacro{\childdocforward}
The command |\childdocforward| redirects processing to
another source file:
%
\begin{center}
\begin{tabular}{l}
|\input{childdoc.def}|\\
|\childdocforward[|\textit{main}|]{|\textit{dest}|}|\\
\end{tabular}
\end{center}
%
The argument \textit{dest} is the destination file
(without extension).
It should be the main file or one of the child files.
Note that further \textsf{childdoc} directives
such as |\childdocof| and |\childdocforward|
in the indicated file will be processed in this form.
The optional argument \textit{main}
passes on directly to the main file \textit{main}
while pretending to compile the child \textit{dest}.
This form behaves as if \textit{dest}
issues |\childdocof{|\textit{main}|}| right away,
and no further \textsf{childdoc} directives will be processed.

%%%%%%%%%%%%%%%%%%%%%%%%%%%%%%%%%%%%%%%%
\DescribeMacro{\...prefix}
In the alternative form |\childdocforwardprefix|,
%
\begin{center}
\begin{tabular}{l}
|\input{childdoc.def}|\\
|\childdocforwardprefix[|\textit{main}|]{|\textit{prefix}|}{|\textit{dest}|}|
\end{tabular}
\end{center}
%
the destination file is determined by a pattern
depending on the current file:
To make this work, the current file must be called
`{\textit{prefix}\hspace{0.2em}\textit{suffix}}'
with \textit{prefix} matching precisely the argument.
Processing is then passed on to the file
`{\textit{dest}\hspace{0.2em}\textit{suffix}}'.
Surely, the same effect is achieved by
directly specifying the
argument `{\textit{dest}\hspace{0.2em}\textit{suffix}}'
in the first form.
However, that requires to set up a different file
for each child. With the alternative form of the command
all these files can have exactly the same content
which simplifies setting them up and maintaining them.

For example, the following file |draft.tex|
with a compilation flag |\version| as described in \secref{sec:flags}
compiles the main document as a draft:
%
\begin{center}
\begin{tabular}{l}
|\def\version{draft}|\\
|\input{childdoc.def}|\\
|\childdocforward{|\textit{main}|}|
\end{tabular}
\end{center}
%
Likewise, the following files |final|\textit{nn}|.tex|
compile the final version of the child document
|child|\textit{nn}|.tex|:
%
\begin{center}
\begin{tabular}{l}
|\def\version{final}|\\
|\input{childdoc.def}|\\
|\childdocforwardprefix{final}{child}|
\end{tabular}
\end{center}
%

Note that when several versions of a main file and/or of each child file
are to be generated, it may be convenient to set up a |Makefile| or
shell script to automatise the process.

%%%%%%%%%%%%%%%%%%%%%%%%%%%%%%%%%%%%%%%%%%%%%%%%%%%%%%%%%%%%%%%%%%%%%%%%%%%%%%%%
\subsection{Command Line Processing}
\label{sec:commandline}

The effect of redirection files can also be achieved by invoking
the \LaTeX{} compiler with a more elaborate command line.
Most conveniently this should be done as part
of a shell script or a |Makefile|.

When using \textsf{childdoc} in the main file, the following
command lines effectively perform a redirection
(note that depending on the shell being used,
backslashes may have to be doubled: `|\|' $\to$ `|\\|'):
%
\begin{center}
|... -jobname "|\textit{target}|" |\\|"|[\textit{flags}]%
|\input{childdoc.def}\childdocforward[|\textit{main}|]{|\textit{dest}|}"|
\end{center}
%
Here \textit{target} is the name of the output file,
\textit{main} is the name of the main file
and \textit{dest} is the name of the main or child file to be processed
(all filenames without extensions).
The optional argument \textit{main} can be omitted
if \textit{main} matches \textit{dest}.
Optionally, compilation \textit{flags} can be defined via |\def| commands.
This command line makes the \TeX{} engine believe
it is compiling the file \textit{target}
whose content is specified as the latter parameter.
The provided code then forwards the processing to
\textit{main} or \textit{dest} as described in \secref{sec:forward}.

%%%%%%%%%%%%%%%%%%%%%%%%%%%%%%%%%%%%%%%%%%%%%%%%%%%%%%%%%%%%%%%%%%%%%%%%%%%%%%%%
\subsection{Include by Input}
\label{sec:input}

Including child documents by |\include| has some restrictions by design.
Most notably, the content of a child document always occupies
its own set of pages; pages cannot be shared between child documents.
Usually, this behaviour makes perfect sense
because each child document contain an essential part of the document.
However, in some situations it may be desirable to compose
a document from a collection of parts
without having mandatory page breaks between then.
For this case, the package
provides a mechanism to include parts
by |\input| which can also be processed individually.
However, by construction this mechanism
requires manual handling of the content to be output.

%%%%%%%%%%%%%%%%%%%%%%%%%%%%%%%%%%%%%%%%
\DescribeMacro{\ifchilddocmanual}
The main file should be prepared as usual, see \secref{sec:include}.
However, the document body must make a distinction
between processing of an individual part and of the main document, e.g.:
%
\begin{center}
\begin{tabular}{l}
|\ifchilddocmanual|\\
|\input{\childdocname}|\\
|\||else|\\
\textit{document body with }|\input{|\textit{part}|}|\\
|\||fi|
\end{tabular}
\end{center}
%
The conditional |\ifchilddocmanual| is true whenever
a part to be included by |\input| is being compiled,
and the name of the part is stored in |\childdocname|.

%%%%%%%%%%%%%%%%%%%%%%%%%%%%%%%%%%%%%%%%
\DescribeMacro{\childdocby}
Each part to be included by |\input| should start with:
%
\begin{center}
\begin{tabular}{l}
|\input{childdoc.def}|\\
|\childdocby{|\textit{main}|}|\\
\end{tabular}
\end{center}
%
The directive |\childdocby| is similar to |\childdocof|
described in \secref{sec:include},
but the subsequent selection of content must be done manually.
To that end, both |\ifchilddoc| and |\ifchilddocmanual|
will be true upon processing of a part,
and the name of the part is stored in |\childdocname|.
Note that |\jobname| will be set to the filename of the current part
so that each part receives an individual |.aux| file
that does not interfere with the |.aux| file(s) of the main document.
This behaviour can be altered by the alternative form
|\childdocby[*]{|\textit{main}|}| (with a non-empty optional argument)
which uses the |.aux| file of the main document
by setting |\jobname| to \textit{main}.

%%%%%%%%%%%%%%%%%%%%%%%%%%%%%%%%%%%%%%%%%%%%%%%%%%%%%%%%%%%%%%%%%%%%%%%%%%%%%%%%
\subsection{Driver Development}
\label{sec:driver}

The \textsf{childdoc} mechanism can also be use for the development
of definition files such as \LaTeX{} styles or classes.
This case differs from the above setup with multiple parts
included by |\include| in that no |\includeonly| should be invoked.
This can be achieved by starting the include file
(before |\ProvidesPackage|) with:
%
\begin{center}
\begin{tabular}{l}
|\input{childdoc.def}|\\
|\childdocforward{|\textit{main}|}|\\
\end{tabular}
\end{center}
%
or alternatively with:
%
\begin{center}
\begin{tabular}{l}
|\input{childdoc.def}|\\
|\childdocby{|\textit{main}|}|\\
\end{tabular}
\end{center}
%
Both forms have slightly different effects as described above.
The main file is prepared as usual, see \secref{sec:include}.

%%%%%%%%%%%%%%%%%%%%%%%%%%%%%%%%%%%%%%%%%%%%%%%%%%%%%%%%%%%%%%%%%%%%%%%%%%%%%%%%
\subsection{Legacy Detection}
\label{sec:detection}

The directive |\childdocmain| in the main file can detect
whether the complete document or merely a child is to be compiled
even without using the directive |\childdocof|.
This method is deprecated because it is less robust
and there is no compelling reason to use it;
it is merely provided for backward compatibility
and it may be removed in future versions.

If the detection mechanism is to be used,
it is mandatory to correctly specify
the filename of the main file as the argument of |\childdocmain|:
%
\begin{center}
\begin{tabular}{l}
|\input{childdoc.def}|\\
|\childdocmain{|\textit{main}|}|\\
\end{tabular}
\end{center}
%
If |\jobname| does not match the argument \textit{main} of |\childdocmain|,
it is assumed that |\jobname| points to the child file to be compiled.
When using |\childdocmain| with the main file specified as argument,
it suffices to start a child file
with just |\input{|\textit{main}|}|
without loading of the package and using |\childdocof|.
If instead all processing is done
with the appropriate \textsf{childdoc} directives,
the argument of \textit{main} of |\childdocmain| can be empty.

An alternative version of the command line processing described
in \secref{sec:commandline} using the detection mechanism reads:
%
\begin{center}
|... -jobname "|\textit{target}|" "|[\textit{flags}]%
[|\def\jobname{|\textit{dest}|}|]|\input{|\textit{main}|}"|
\end{center}

%%%%%%%%%%%%%%%%%%%%%%%%%%%%%%%%%%%%%%%%%%%%%%%%%%%%%%%%%%%%%%%%%%%%%%%%%%%%%%%%
\subsection{Manual Code}
\label{sec:manual}

In case one cannot be certain whether the definitions file |childdoc.def|
is installed on the target \TeX{} distribution
and one prefers not to ship it,
it is conceivable to paste a few relevant commands into the sources.

To that end, drop all statements |\input{childdoc.def}|
and perform the replacements as outlined below.
Instead of |\childdocmain{|\textit{main}|}| add the following code
to the top of the main file:
%
\begin{center}
\begin{tabular}{l}
|\||ifdefined\childdocname\endinput\||fi\newif\ifchilddoc|\\
|\edef\childdocname{\scantokens\expandafter{\jobname\noexpand}}|\\
|\def\childdocmain{|\textit{main}|}\||ifx\childdocmain\childdocname\||else|\\
|\childdoctrue\includeonly{\childdocname}\let\jobname\childdocmain\||fi|\\
\end{tabular}
\end{center}
%
Instead of |\childdocof{|\textit{main}|}| just include the main file
at the top of each child file:
%
\begin{center}
|\input{|\textit{main}|}|
\end{center}
%
A simple redirection |\childdocforward{|\textit{dest}|}| is achieved by:
%
\begin{center}
|\def\jobname{|\textit{dest}|}\input{\jobname}|
\end{center}
%
The redirection with prefix
|\childdocforwardprefix[|\textit{prefix}|]{|\textit{dest}|}|
is accomplished by:
%
\begin{center}
\begin{tabular}{l}
|{\edef\jobname{\scantokens\expandafter{\jobname\noexpand}}|\\
|\def\redirectjob |\textit{prefix}|#1~~~{\gdef\jobname{|\textit{dest}|#1}}|\\
|\expandafter\redirectjob\jobname~~~}\input{\jobname}|
\end{tabular}
\end{center}

In an alternative approach,
child documents can be compiled by a specific command line
without additional code or specific definitions:
%
\begin{center}
|... -jobname "|\textit{target}|" "|[\textit{flags}]%
|\includeonly{|\textit{dest}|}\input{|\textit{main}|}"|
\end{center}
%

%%%%%%%%%%%%%%%%%%%%%%%%%%%%%%%%%%%%%%%%%%%%%%%%%%%%%%%%%%%%%%%%%%%%%%%%%%%%%%%%
%%%%%%%%%%%%%%%%%%%%%%%%%%%%%%%%%%%%%%%%%%%%%%%%%%%%%%%%%%%%%%%%%%%%%%%%%%%%%%%%
\section{Information}

%%%%%%%%%%%%%%%%%%%%%%%%%%%%%%%%%%%%%%%%%%%%%%%%%%%%%%%%%%%%%%%%%%%%%%%%%%%%%%%%
\subsection{Copyright}

Copyright \copyright{} 2017--2018 Niklas Beisert

This work may be distributed and/or modified under the
conditions of the \LaTeX{} Project Public License, either version 1.3
of this license or (at your option) any later version.
The latest version of this license is in
  \url{http://www.latex-project.org/lppl.txt}
and version 1.3 or later is part of all distributions of \LaTeX{}
version 2005/12/01 or later.

This work has the LPPL maintenance status `maintained'.

The Current Maintainer of this work is Niklas Beisert.

This work consists of the files |README.txt|, |childdoc.ins| and |childdoc.dtx|
as well as the derived files |childdoc.def|, |cdocsamp.tex|
with |cdocsch1.tex|, |cdocsch2.tex|, |cdocspt3.tex|, |cdocspt4.tex|,
|cdocsdrf.tex|, |cdocsfn1.tex|, |cdocsfn2.tex|
as well as |childdoc.pdf|.

%%%%%%%%%%%%%%%%%%%%%%%%%%%%%%%%%%%%%%%%%%%%%%%%%%%%%%%%%%%%%%%%%%%%%%%%%%%%%%%%
\subsection{Files and Installation}

The package consists of the files:
%
\begin{center}
\begin{tabular}{ll}
    |README.txt|   & readme file \\
    |childdoc.ins| & installation file \\
    |childdoc.dtx| & source file \\
    |childdoc.def| & definition file \\
    |cdocsamp.tex| & sample main file \\
    |cdocsch1.tex| & sample include file \\
    |cdocsch2.tex| & sample include file \\
    |cdocspt3.tex| & sample part file \\
    |cdocspt4.tex| & sample part file \\
    |cdocsdrf.tex| & sample redirection file \\
    |cdocsfn1.tex| & sample redirection file \\
    |cdocsfn2.tex| & sample redirection file \\
    |childdoc.pdf| & manual
\end{tabular}
\end{center}
%
The distribution consists of the files
|README.txt|, |childdoc.ins| and |childdoc.dtx|.
%
\begin{itemize}
\item
Run (pdf)\LaTeX{} on |childdoc.dtx|
to compile the manual |childdoc.pdf| (this file).
\item
Run \LaTeX{} on |childdoc.ins| to create the definitions file |childdoc.def|
and the sample |cdocsamp.tex| with include files
|cdocsch1.tex|, |cdocsch2.tex|, |cdocspt3.tex|, |cdocspt4.tex|,
|cdocsdrf.tex|, |cdocsfn1.tex|, |cdocsfn2.tex|.
Then copy the file |childdoc.def| to an appropriate directory of your \LaTeX{}
distribution, e.g.\ \textit{texmf-root}|/tex/latex/childdoc|.
\end{itemize}

%%%%%%%%%%%%%%%%%%%%%%%%%%%%%%%%%%%%%%%%%%%%%%%%%%%%%%%%%%%%%%%%%%%%%%%%%%%%%%%%
\subsection{Related CTAN Packages}

There are several other packages which offer a similar functionality:
%
\begin{itemize}
\item
The packages
\href{http://ctan.org/pkg/docmute}{\textsf{docmute}},
\href{http://ctan.org/pkg/includex}{\textsf{includex}} and
\href{http://ctan.org/pkg/standalone}{\textsf{standalone}}
provide commands to include only the document body of
a child file thus allowing both files to be compiled individually.
\item
The packages \href{http://ctan.org/pkg/subdocs}{\textsf{subdocs}}
and \href{http://ctan.org/pkg/subfiles}{\textsf{subfiles}}
provide structures in which the main and child documents can be
encapsulated and allowing them to be compiled individually.
The inclusion mechanism is different from the conventional |\include|.
\item
The package \href{http://ctan.org/pkg/combine}{\textsf{combine}}
is an elaborate solution to combine several documents into one.
\end{itemize}
%
See also the CTAN topic \href{http://ctan.org/topic/subdocs}{\textsf{subdocs}}
for further related packages.
The present package differs from the above solutions in that
a document structure constructed with the conventional |\include| mechanism
just needs two extra commands at the top of every file
such that all constituent files can be compiled individually.

%%%%%%%%%%%%%%%%%%%%%%%%%%%%%%%%%%%%%%%%%%%%%%%%%%%%%%%%%%%%%%%%%%%%%%%%%%%%%%%%
%\subsection{Feature Suggestions}
%
%The following is a list of features which may be useful for future
%versions of this package:
%%
%\begin{itemize}
%\item
%\ldots
%\end{itemize}

%%%%%%%%%%%%%%%%%%%%%%%%%%%%%%%%%%%%%%%%%%%%%%%%%%%%%%%%%%%%%%%%%%%%%%%%%%%%%%%%
\subsection{Revision History}

%%%%%%%%%%%%%%%%%%%%%%%%%%%%%%%%%%%%%%%%
\paragraph{v2.0:} 2018/12/30

\begin{itemize}
\item
immediate forward processing
\item
added |\childdocby| mechanism
\item
manual restructured
\end{itemize}

%%%%%%%%%%%%%%%%%%%%%%%%%%%%%%%%%%%%%%%%
\paragraph{v1.6:} 2018/01/17

\begin{itemize}
\item
application for development of include files
\item
corrections to manual
\end{itemize}

%%%%%%%%%%%%%%%%%%%%%%%%%%%%%%%%%%%%%%%%
\paragraph{v1.5:} 2017/05/21

\begin{itemize}
\item
more complete structuring introduced
\item
|\childdocof| introduced
\item
|\childdoc| renamed to |\childdocmain|
\item
|\childredirect| renamed to |\childdocforward| and |\childdocforwardprefix|
and functionality expanded
\end{itemize}

%%%%%%%%%%%%%%%%%%%%%%%%%%%%%%%%%%%%%%%%
\paragraph{v1.0:} 2017/04/27

\begin{itemize}
\item
manual and install package
\item
first version published on CTAN
\end{itemize}

%%%%%%%%%%%%%%%%%%%%%%%%%%%%%%%%%%%%%%%%
\paragraph{v0.6:} 2017/04/26

\begin{itemize}
\item
redirection mechanism added
\end{itemize}

%%%%%%%%%%%%%%%%%%%%%%%%%%%%%%%%%%%%%%%%
\paragraph{v0.5:} 2017/04/26

\begin{itemize}
\item
functionality in definition file
\end{itemize}


%%%%%%%%%%%%%%%%%%%%%%%%%%%%%%%%%%%%%%%%%%%%%%%%%%%%%%%%%%%%%%%%%%%%%%%%%%%%%%%%
%%%%%%%%%%%%%%%%%%%%%%%%%%%%%%%%%%%%%%%%%%%%%%%%%%%%%%%%%%%%%%%%%%%%%%%%%%%%%%%%
%%%%%%%%%%%%%%%%%%%%%%%%%%%%%%%%%%%%%%%%%%%%%%%%%%%%%%%%%%%%%%%%%%%%%%%%%%%%%%%%
\appendix

\settowidth\MacroIndent{\rmfamily\scriptsize 000\ }

 \DocInput{childdoc.dtx}

\end{document}
%</driver>
% \fi
%
% %%%%%%%%%%%%%%%%%%%%%%%%%%%%%%%%%%%%%%%%%%%%%%%%%%%%%%%%%%%%%%%%%%%%%%%%%%%%%%
% %%%%%%%%%%%%%%%%%%%%%%%%%%%%%%%%%%%%%%%%%%%%%%%%%%%%%%%%%%%%%%%%%%%%%%%%%%%%%%
% \section{Sample}
%\iffalse
%<*samplemain>
%\fi
%
% The following presents a sample document
% with two chapters, two parts, a title page,
% a compile flag as well as three forwarding files to set the flag.
% It consists of eight |.tex| files:
% \begin{center}
% \begin{tabular}{ll}
% |cdocsamp.tex|&main file\\
% |cdocsch1.tex|&include file for chapter 1\\
% |cdocsch2.tex|&include file for chapter 2\\
% |cdocspt3.tex|&include file for part 3\\
% |cdocspt4.tex|&include file for part 4\\
% |cdocsdrf.tex|&forwarding file for main file in draft mode\\
% |cdocsfi1.tex|&forwarding file for final version of chapter 1\\
% |cdocsfi2.tex|&forwarding file for final version of chapter 2\\
% \end{tabular}
% \end{center}
% Each of the eight files can be compiled directly by the \LaTeX{} compiler.
%
% %%%%%%%%%%%%%%%%%%%%%%%%%%%%%%%%%%%%%%
% \paragraph{Main File.}
%
% The main file is called |cdocsamp.tex|.
%
% Load the \textsf{childdoc} definitions and
% declare the filename for the main document:
%    \begin{macrocode}
\input{childdoc.def}
\childdocmain{}
%    \end{macrocode}

% Optional override for |\version| flag:
%    \begin{macrocode}
%%\ifchilddoc\else\providecommand{\version}{draft}\fi
%    \end{macrocode}

% Define the default values for the |\version| flag
% (|final| for the main file and |draft| for childs):
%    \begin{macrocode}
\ifchilddoc
\providecommand{\version}{draft}
\else
\providecommand{\version}{final}
\fi
%    \end{macrocode}

% Load the standard document class:
%    \begin{macrocode}
\documentclass[12pt]{article}
%    \end{macrocode}

% Start the document body:
%    \begin{macrocode}
\begin{document}
%    \end{macrocode}

% Declare a title page.
% Print title, part of document being processed and version flag:
%    \begin{macrocode}
\addtocounter{page}{-1}
\begin{center}
{\LARGE\bfseries{}childdoc example\par}
\vspace{1cm}
\ifchilddoc
\ifchilddocmanual part\else chapter\fi:
`\childdocname' of `\childdocjob'\par
\else
main document: `\childdocjob'\par
\fi
version: \version\par
\end{center}
\newpage
%    \end{macrocode}

% Manually include selected file,
% otherwise process as usual:
%    \begin{macrocode}
\ifchilddocmanual
\section*{part `\childdocname'}
\input{\childdocname}
\else
%    \end{macrocode}

% Include the two chapters:
%    \begin{macrocode}
\include{cdocsch1}
\include{cdocsch2}
%    \end{macrocode}

% Include the two parts unless only chapters should be displayed:
%    \begin{macrocode}
\ifchilddoc\else
\section{part three}
\input{cdocspt3}
\section{part four}
\input{cdocspt4}
\fi
%    \end{macrocode}

% Process as usual until here:
%    \begin{macrocode}
\fi
%    \end{macrocode}

% End of document body:
%    \begin{macrocode}
\end{document}
%    \end{macrocode}
%\iffalse
%</samplemain>
%\fi
%
% %%%%%%%%%%%%%%%%%%%%%%%%%%%%%%%%%%%%%%
% \paragraph{Chapter Include Files.}
%
% The include files are called |cdocsch1.tex| and |cdocsch2.tex|.
%
%\iffalse
%<*samplechap1|samplechap2>
%\fi

% Optional override for |\version| flag:
%    \begin{macrocode}
%%\providecommand{\version}{final}
%    \end{macrocode}

% Include the main document:
%    \begin{macrocode}
\input{childdoc.def}
\childdocof{cdocsamp}
%    \end{macrocode}

%\iffalse
%</samplechap1|samplechap2>
%\fi
%
%\iffalse
%<*samplechap1>
%\fi
% Some text for chapter 1:
%    \begin{macrocode}
\section{one}
some text in chapter one
%    \end{macrocode}

%\iffalse
%</samplechap1>
%\fi
% Some text for chapter 2:
%\iffalse
%<*samplechap2>
%\fi
%    \begin{macrocode}
\section{two}
more text in chapter two
%    \end{macrocode}

%\iffalse
%</samplechap2>
%\fi
%
% %%%%%%%%%%%%%%%%%%%%%%%%%%%%%%%%%%%%%%
% \paragraph{Part Include Files.}
%
% The include files are called |cdocspt3.tex| and |cdocspt4.tex|.
%
%\iffalse
%<*samplepart3|samplepart4>
%\fi

% Optional override for |\version| flag:
%    \begin{macrocode}
%%\providecommand{\version}{final}
%    \end{macrocode}

% Include the main document:
%    \begin{macrocode}
\input{childdoc.def}
\childdocby{cdocsamp}
%    \end{macrocode}

%\iffalse
%</samplepart3|samplepart4>
%\fi
%
%\iffalse
%<*samplepart3>
%\fi
% Some text for part 3:
%    \begin{macrocode}
some text in part three
%    \end{macrocode}

%\iffalse
%</samplepart3>
%\fi
% Some text for part 4:
%\iffalse
%<*samplepart4>
%\fi
%    \begin{macrocode}
more text in part four
%    \end{macrocode}

%\iffalse
%</samplepart4>
%\fi
%
% %%%%%%%%%%%%%%%%%%%%%%%%%%%%%%%%%%%%%%
% \paragraph{Forwarding for a Complete Draft.}
%
% The following forwarding file |cdocsdrf.tex|
% compiles the main document in draft mode:
%\iffalse
%<*sampledraft>
%\fi
%    \begin{macrocode}
\def\version{draft}
\input{childdoc.def}
\childdocforward{cdocsamp}
%    \end{macrocode}

%\iffalse
%</sampledraft>
%\fi
%
% %%%%%%%%%%%%%%%%%%%%%%%%%%%%%%%%%%%%%%
% \paragraph{Forwarding for Final Version of the Chapters.}
%
% The following forwarding files |cdocsfn1.tex| and |cdocsfn2.tex|
% (with identical content)
% compile the final versions of the child documents
% |cdocsch1.tex| and |cdocsch2.tex|, respectively:
%\iffalse
%<*samplefinal>
%\fi
%    \begin{macrocode}
\def\version{final}
\input{childdoc.def}
\childdocforwardprefix[cdocsamp]{cdocsfn}{cdocsch}
%    \end{macrocode}

%\iffalse
%</samplefinal>
%\fi
%
% %%%%%%%%%%%%%%%%%%%%%%%%%%%%%%%%%%%%%%
% \paragraph{Command Line Processing.}
%
% The following three command lines generate the output files
% |cdocscld|, |cdocscl1| and |cdocscl2|
% which should be identical to
% |cdocsdrf|, |cdocsch1| and |cdocsfn2|, respectively:
% \begin{center}
% \begin{tabular}{l}
% |latex -jobname cdocscld \|\\
% |  "\def\version{draft}\input{childdoc.def}\childdocforward{cdocsamp}"|\\
% |latex -jobname cdocscl1 \|\\
% |  "\input{childdoc.def}\childdocforward[cdocsamp]{cdocsch1}"|\\
% |latex -jobname cdocscl2 \|\\
% |  "\def\version{final}\input{childdoc.def}\childdocforward{cdocsch2}"|
% \end{tabular}
% \end{center}
% Note that the trailing backslash on each first line
% merely continues the input to the second line
% (for convenient cut ant paste).
% Furthermore, the command |latex| can be replaced by any
% of its alternative versions such as |pdflatex|.
%
% %%%%%%%%%%%%%%%%%%%%%%%%%%%%%%%%%%%%%%%%%%%%%%%%%%%%%%%%%%%%%%%%%%%%%%%%%%%%%%
% %%%%%%%%%%%%%%%%%%%%%%%%%%%%%%%%%%%%%%%%%%%%%%%%%%%%%%%%%%%%%%%%%%%%%%%%%%%%%%
% \section{Implementation}
%\iffalse
%<*package>
%\fi
%
% This section describes the definitions file |childdoc.def|.

% The definitions cannot be loaded using |\usepackage| or |\RequirePackage|
% which has a mechanism to prevent loading a style file more than once.
% When loading the definitions by means of |\input|
% multiple instances have to be prevented manually:
%\iffalse
%This code needs to be before the `\ProvidesFile' directive
%which is defined at the beginning of this file.
%Therefore it is also placed there and commented out here.
%</package>
%<*discard>
%\fi
%    \begin{macrocode}
\ifdefined\childdocmain\endinput\fi
%    \end{macrocode}
%\iffalse
%</discard>
%<*package>
%\fi
%
% \macro{\ifchilddoc}
% \macro{\ifchilddocmanual}
% The conditional |\ifchilddoc| tells whether a
% child (true) or main (false) document is being compiled.
% The conditional |\ifchilddocmanual| tells whether
% the |\includeonly| mechanism is used (false) or
% the selection of child files must be performed manually (true).
% The definitions initialise to false:
%    \begin{macrocode}
\newif\ifchilddoc
\newif\ifchilddocmanual
%    \end{macrocode}

% \macro{\childdocname}
% \macro{\childdocjob}
% The macro |\childdocname| stores the name of the main document
% to be compiled. The macro |\childdocjob| stores the name of
% the document on which the \LaTeX{} compiler was originally invoked.
% The content of |\jobname| cannot be compared
% to filenames specified in the source due to different catcodes.
% The following code rescans |\jobname|, stores the result
% in |\childdocname| and saves a copy in |\childdocjob|:
%    \begin{macrocode}
\edef\childdocname{\scantokens\expandafter{\jobname\noexpand}}
\let\childdocjob\childdocname
%    \end{macrocode}

% \macro{\childdocdisable}
% The macro |\childdocdisable| prevents the main file
% from being processed more than once.
% At this stage, the main document command |\childdocmain|
% is assumed to be called once again where it should do nothing.
% Any subsequent call to it should prevent
% a secondary processing of the main document
% It overwrites the forwarding commands
% |\childdocof| and |\childdocforward|
% with empty macros to prevent further inclusions of the main document:
%    \begin{macrocode}
\newcommand{\childdocdisable}
{
  \renewcommand{\childdocmain}[1]{\renewcommand{\childdocmain}[1]{\endinput}}
  \renewcommand{\childdocof}[1]{}
  \renewcommand{\childdocby}[2][]{}
  \renewcommand{\childdocforward}[2][]{}
  \renewcommand{\childdocdisable}{}
}
%    \end{macrocode}

% \macro{\childdocmain}
% The macro |\childdocmain| is to be called at the top of the main file
% with nothing or the main filename (without extension) as argument.
% First, it breaks loops.
% If the argument is not empty and does not match |\childdocname|
% (which is set by the first inclusion of |childdoc.def|),
% |\ifchilddoc| is set to true, |\includeonly| is applied to the child file
% and |\jobname| is set to the main file
% (for proper handling of |.aux| files):
%    \begin{macrocode}
\newcommand{\childdocmain}[1]
{
  \childdocdisable\childdocmain{}
  \if?#1?\else
    \begingroup
      \def\childdoctmp{#1}
      \ifx\childdoctmp\childdocname
        \def\childdoctmp{}
      \else
        \def\childdoctmp
        {
          \childdoctrue
          \includeonly{\childdocname}
          \def\childdocjob{#1}
          \def\jobname{#1}
        }
      \fi
      \expandafter
    \endgroup
    \childdoctmp
  \fi
}
%    \end{macrocode}

% \macro{\childdocof}
% The command |\childdocof| redirects
% compilation to the main file |#1|.
%    \begin{macrocode}
\newcommand{\childdocof}[1]
{
  \childdocdisable
  \childdoctrue
  \includeonly{\childdocname}
  \def\jobname{#1}
  \def\childdocjob{#1}
  \input{#1}
}
%    \end{macrocode}

% \macro{\childdocby}
% The command |\childdocby| ....
%    \begin{macrocode}
\newcommand{\childdocby}[2][]
{
  \childdocdisable
  \childdoctrue
  \childdocmanualtrue
  \if?#1?\else
    \def\jobname{#2}
  \fi
  \def\childdocjob{#2}
  \input{#2}
  \endinput
}
%    \end{macrocode}

% \macro{\childdocforward}
% The command |\childdocforward| redirects
% compilation to the main file or
% (if the optional argument is given) a child file.
% Parameters are set as if the main file
% or a child file starting with |\childdocof| was compiled.
% Then compilation is handed over to the main file:
%    \begin{macrocode}
\newcommand{\childdocforward}[2][]
{
  \begingroup
    \if?#1?
      \def\childdoctmp
      {
        \def\childdocname{#2}
        \def\childdocjob{#2}
        \def\jobname{#2}
        \input{#2}
        \endinput
      }
    \else
      \def\childdoctmp
      {
        \childdocdisable
        \def\childdocname{#2}
        \childdoctrue
        \includeonly{#2}
        \def\childdocjob{#1}
        \def\jobname{#1}
        \input{#1}
        \endinput
      }
    \fi
    \expandafter
  \endgroup
  \childdoctmp
}
%    \end{macrocode}

% \macro{\childdocforwardprefix}
% The command |\childdocforwardprefix| redirects
% compilation to the main or a child file by means of a pattern.
% The prefix |#1| in the current filename is replaced by |#2|
% and the suffix of the current filename is kept
% (it is assumed that the filename does not contain the substring `|~~~|'
% which is used as a delimiter).
% Compilation is handed over to the new file by |\childdocforward|:
%    \begin{macrocode}
\newcommand{\childdocforwardprefix}[3][]
{
  \begingroup
    \def\childdocextract #2##1~~~{\def\childdoctmp{\childdocforward[#1]{#3##1}}}
    \expandafter\childdocextract\childdocname~~~
    \expandafter
  \endgroup
  \childdoctmp
}
%    \end{macrocode}

% \macro{\childdoc}
% The deprecated macro |\childdoc| is a legacy version of |\childdocmain|:
%    \begin{macrocode}
\newcommand{\childdoc}{\childdocmain}
%    \end{macrocode}

% \macro{\childdocredirect}
% The deprecated macro |\childdocredirect| is a legacy version
% of |\childdocforward| and |\childdocforwardprefix|:
%    \begin{macrocode}
\newcommand{\childdocredirect}[2][]
{
  \begingroup
    \if?#1?
      \def\childdoctmp{\childdocforward{#2}}
    \else
      \def\childdoctmp{\childdocforwardprefix{#1}{#2}}
    \fi
    \expandafter
  \endgroup
  \childdoctmp
}
%    \end{macrocode}

%\iffalse
%</package>
%\fi
%
\endinput
|\\
|\childdocforward{|\textit{main}|}|
\end{tabular}
\end{center}
%
Likewise, the following files |final|\textit{nn}|.tex|
compile the final version of the child document
|child|\textit{nn}|.tex|:
%
\begin{center}
\begin{tabular}{l}
|\def\version{final}|\\
|% \iffalse
%
% childdoc.dtx Copyright (C) 2017-2018 Niklas Beisert
%
% This work may be distributed and/or modified under the
% conditions of the LaTeX Project Public License, either version 1.3
% of this license or (at your option) any later version.
% The latest version of this license is in
%   http://www.latex-project.org/lppl.txt
% and version 1.3 or later is part of all distributions of LaTeX
% version 2005/12/01 or later.
%
% This work has the LPPL maintenance status `maintained'.
%
% The Current Maintainer of this work is Niklas Beisert.
%
% This work consists of the files childdoc.dtx and childdoc.ins
% and the derived files childdoc.def and cdocsamp.tex with
% cdocsch1.tex, cdocsch2.tex, cdocsdrf.tex, cdocsfn1.tex, cdocsfn2.tex.
%
%<package>\ifdefined\childdocmain\endinput\fi
%<package>\ProvidesFile{childdoc.def}[2018/12/30 v2.0 child document driver]
%<samplemain>\ProvidesFile{cdocsamp.tex}[2018/12/30 v2.0 sample for childdoc]
%<*driver>
%\ProvidesFile{childdoc.drv}[2018/12/30 v2.0 childdoc reference manual file]
\PassOptionsToClass{10pt,a4paper}{article}
\documentclass{ltxdoc}

\usepackage[margin=35mm]{geometry}
\usepackage{hyperref}
\usepackage{hyperxmp}
\usepackage[usenames]{color}

\hypersetup{colorlinks=true}
\hypersetup{pdfstartview=FitH}
\hypersetup{pdfpagemode=UseNone}
\hypersetup{pdfsource={}}
\hypersetup{pdflang={en-UK}}
\hypersetup{pdfcopyright={Copyright 2017-2018 Niklas Beisert.
  This work may be distributed and/or modified under the
  conditions of the LaTeX Project Public License, either version 1.3
  of this license or (at your option) any later version.}}
\hypersetup{pdflicenseurl={http://www.latex-project.org/lppl.txt}}
\hypersetup{pdfcontactaddress={ETH Zurich, ITP, HIT K,
  Wolfgang-Pauli-Strasse 27}}
\hypersetup{pdfcontactpostcode={8093}}
\hypersetup{pdfcontactcity={Zurich}}
\hypersetup{pdfcontactcountry={Switzerland}}
\hypersetup{pdfcontactemail={nbeisert@itp.phys.ethz.ch}}
\hypersetup{pdfcontacturl={http://people.phys.ethz.ch/\xmptilde nbeisert/}}

\newcommand{\secref}[1]{\hyperref[#1]{section \ref*{#1}}}

\parskip1ex
\parindent0pt
\let\olditemize\itemize
\def\itemize{\olditemize\parskip0pt}

\begin{document}

\title{The \textsf{childdoc} Package}
\hypersetup{pdftitle={The childdoc Package}}
\author{Niklas Beisert\\[2ex]
  Institut f\"ur Theoretische Physik\\
  Eidgen\"ossische Technische Hochschule Z\"urich\\
  Wolfgang-Pauli-Strasse 27, 8093 Z\"urich, Switzerland\\[1ex]
  \href{mailto:nbeisert@itp.phys.ethz.ch}
  {\texttt{nbeisert@itp.phys.ethz.ch}}}
\hypersetup{pdfauthor={Niklas Beisert}}
\hypersetup{pdfsubject={Manual for the LaTeX2e Package childdoc}}
\date{30 December 2018, \textsf{v2.0}}
\maketitle

\begin{abstract}\noindent
\textsf{childdoc} is a \LaTeXe{} package
that enables the direct compilation
of document sections included by |\include|
to individual files.
\end{abstract}

\begingroup
\parskip0ex
\tableofcontents
\endgroup

%%%%%%%%%%%%%%%%%%%%%%%%%%%%%%%%%%%%%%%%%%%%%%%%%%%%%%%%%%%%%%%%%%%%%%%%%%%%%%%%
%%%%%%%%%%%%%%%%%%%%%%%%%%%%%%%%%%%%%%%%%%%%%%%%%%%%%%%%%%%%%%%%%%%%%%%%%%%%%%%%
\section{Introduction}

\LaTeX{} provides a mechanism to structure a large document (such as a book)
into a main file and several child files (containing the chapters)
using the |\include| command.
This mechanism is beneficial for documents
which span hundreds of pages in order to
make the source file(s) more manageable.
Moreover, compilation can be restricted to
selected child files by means of the |\includeonly| command.
The latter feature can be used to reduce the compilation time while editing
(this was significantly more useful in the earlier days of \LaTeX{})
or to generate a smaller document which is easier to navigate.
Another application of |\includeonly| is to generate
documents consisting of selected parts of the complete document.

However, there are a few drawbacks of the plain |\include| mechanism:
\begin{itemize}
\item
The child files cannot be compiled on their own,
they can only be compiled via the main file.
A naive editing environment
(such as a text editor with an option
to have the current file processed by \LaTeX)
may require one to switch to the main file before compiling;
attempting to compile the child file produces errors.
\item
The main file must be modified (each time)
to adjust the |\includeonly| command
to the present needs. This easily leaves the main file in a messy state.
\item
The generated document will always carry the filename
of the main document. This is inconvenient if
several child files are to be compiled and
to be kept for distribution.
\end{itemize}

The present package provides a simple interface
to make child files individually compilable by \LaTeX{}.
Compiling a child file then has the same effect as compiling
the main file with an |\includeonly| command
to select the appropriate child.
Moreover the generated document will carry the name of the child
rather than the main file.
This resolves all three above issues.

This feature is meant to make the editing of books,
thesis documents and lecture notes somewhat more convenient.
However, the package can also be used efficiently for
composing a series of documents (such as exercise sheets)
which are typically distributed individually.
It then assists the author in generating the individual documents
(potentially in different versions)
as well as a document containing the collected series.
Another application is in developing style files
or other kinds of included material
where compilation of the style file could redirect
to a sample or test file.

%%%%%%%%%%%%%%%%%%%%%%%%%%%%%%%%%%%%%%%%%%%%%%%%%%%%%%%%%%%%%%%%%%%%%%%%%%%%%%%%
%%%%%%%%%%%%%%%%%%%%%%%%%%%%%%%%%%%%%%%%%%%%%%%%%%%%%%%%%%%%%%%%%%%%%%%%%%%%%%%%
\section{Usage}

First of all, the package \textsf{childdoc} is \emph{not} a standard
\LaTeXe{} |.sty| style file! Therefore it needs to be invoked in
a non-standard way.

%%%%%%%%%%%%%%%%%%%%%%%%%%%%%%%%%%%%%%%%%%%%%%%%%%%%%%%%%%%%%%%%%%%%%%%%%%%%%%%%
\subsection{Included Files}
\label{sec:include}

%%%%%%%%%%%%%%%%%%%%%%%%%%%%%%%%%%%%%%%%
\DescribeMacro{\childdocmain}
To use the package, add the commands
\begin{center}
\begin{tabular}{l}
|\input{childdoc.def}|\\
|\childdocmain{}|\\
\end{tabular}
\end{center}
at the very top of the main \LaTeX{} file,
in particular \emph{before} the |\documentclass| statement!
The argument of |\childdocmain| should be left empty
(but it must be present).

%%%%%%%%%%%%%%%%%%%%%%%%%%%%%%%%%%%%%%%%
\DescribeMacro{\childdocof}
Furthermore, add the commands
\begin{center}
\begin{tabular}{l}
|\input{childdoc.def}|\\
|\childdocof{|\textit{main}|}|\\
\end{tabular}
\end{center}
at the top of every child file \textit{child}
which is included by |\include{|\textit{child}|}|
from within the main file
(or at least for those files to be compiled individually).
The argument \textit{main} must be the filename of the main file.

There are a couple of
considerations in setting up the main and child documents:

%%%%%%%%%%%%%%%%%%%%%%%%%%%%%%%%%%%%%%%%
\paragraph{Restrictions.}

Please note the following restrictions:
\begin{itemize}
\item
|\childdocmain| must be called with one argument \textit{main}
to ensure compatibility with earlier version of the package.
It must either be empty (|\childdocmain{}|)
or precisely match the filename of the main file in which it is specified.
See \secref{sec:detection} for further information.
\item
The filename \textit{main} must be specified without the |.tex| extension.
\item
The filename \textit{main} is case sensitive
(even in case-insensitive file systems)
due to internal string comparison.
\item
The argument \textit{main} should be fully expanded, it cannot be a macro.
\item
Subdirectories and special characters should be avoided in filenames.
\item
The command |\childdocmain{|\textit{main}|}| must be followed by a whitespace.
It should not be followed immediately by another command
or by a comment mark `|%|'.
This is because the \TeX{} parser reads the token immediately following
the argument of |\childdocmain| and puts it
at the beginning of every child section;
however, a white\-space is ignored.
\end{itemize}

%%%%%%%%%%%%%%%%%%%%%%%%%%%%%%%%%%%%%%%%
\paragraph{Content of Main File.}

It is advisable to place all content in the child files included by |\include|.
Any output contained in the main file will appear in all child documents
unless suppressed manually;
it cannot be suppressed automatically by the |\includeonly| directive
and thus should normally be avoided.
A method to include some content in the main file
by means of conditional processing is described in \secref{sec:conditional}.

%%%%%%%%%%%%%%%%%%%%%%%%%%%%%%%%%%%%%%%%
\paragraph{Page Numbering.}

When only a part of the document is compiled,
the appropriate numbering of pages
(as well as other status parameters)
is determined from the |.aux| files.
The latter contain information from previous passes.
However this information needs to propagate through
all intermediate child documents.
Therefore the page numbering in child documents may well
be inconsistent until the complete document is compiled at least once.

A useful (if unconventional) way to always ensure a consistent
page numbering is to restart the numbering in each child document
and denote the pages by `\textit{child}|.|\textit{page}'
where \textit{child} represents the chapter/section number of the child file.
This can be achieved by the command
|\numberwithin{page}{|\textit{child}|}|
of the \textsf{amsmath} package
where \textit{child} can be |chapter| or |section|
depending on the chosen structuring.
Alternatively, one can modify the macro |\thepage| appropriately
and reset the counter |page| at the start of each child file.

%%%%%%%%%%%%%%%%%%%%%%%%%%%%%%%%%%%%%%%%%%%%%%%%%%%%%%%%%%%%%%%%%%%%%%%%%%%%%%%%
\subsection{Conditional Processing}
\label{sec:conditional}

The package provides a mechanism to compile different versions
of a document. To customise the versions further some conditional processing
can come in handy to distinguish which version is being compiled.
The package provides two macros to describe the compilation context:

%%%%%%%%%%%%%%%%%%%%%%%%%%%%%%%%%%%%%%%%
\DescribeMacro{\ifchilddoc}
The conditional |\ifchilddoc| distinguishes between the compilation of
child documents and the main document:
%
\begin{center}
|\ifchilddoc |\textit{child-code}| |[|\||else |\textit{main-code}]| \||fi|
\end{center}

%%%%%%%%%%%%%%%%%%%%%%%%%%%%%%%%%%%%%%%%
\DescribeMacro{\childdocname}
\DescribeMacro{\childdocjob}
The macro |\childdocname| contains the filename (without extension)
of the main or child file being processed.
Note that |\childdocjob| will always contain the name of the main file.

%%%%%%%%%%%%%%%%%%%%%%%%%%%%%%%%%%%%%%%%
\paragraph{Title Page.}

Conditional processing can be used to include a title or banner page
in the main document when proper precautions are taken.
Importantly, the code in the main file should ensure that the page counter
(as well as other status parameters which are stored in the |.aux| files)
takes the same value after the conditional processing.
Otherwise the page numbers may take divergent values
depending on which part is compiled.

For example, a title page could be declared by:
%
\begin{center}
\begin{tabular}{l}
|\ifchilddoc\||else|\\
|\addtocounter{page}{-1}|\\
\textit{code for title page}\\
|\newpage|\\
|\||fi|
\end{tabular}
\end{center}
%
A banner page for the child documents can be generated by:
%
\begin{center}
\begin{tabular}{l}
|\ifchilddoc|\\
|\addtocounter{page}{-1}|\\
\textit{code for banner page}\\
|\newpage|\\
|\||fi|
\end{tabular}
\end{center}
%
Here one could write a message such as:
\begin{center}
|This is the part \childdocname{} of \childdocjob{}.|
\end{center}

%%%%%%%%%%%%%%%%%%%%%%%%%%%%%%%%%%%%%%%%%%%%%%%%%%%%%%%%%%%%%%%%%%%%%%%%%%%%%%%%
\subsection{Flags}
\label{sec:flags}

The package makes it easy to generate different versions
of the main or child documents.
To this end compilation flags can be defined
and assigned different default values.
They will be particularly useful in conjunction
with the forwarding mechanism described in \secref{sec:forward}.

For example, it may be useful to have a flag |\version|
which can be set to |draft| or |final|.
The document source will contain some conditional code
depending on the value of |\version|.
Suppose further, the flag should default to |final| for the main file
and to |draft| for child files
which is a natural assignment for editing the document.
This is achieved by placing the following code
in the preamble of the main document
(below the |\childdocmain| directive):
%
\begin{center}
\begin{tabular}{l}
|\ifchilddoc|\\
|\providecommand{\version}{draft}|\\
|\||else|\\
|\providecommand{\version}{final}|\\
|\||fi|
\end{tabular}
\end{center}
%
The definition by |\providecommand| makes sure
that previous definitions are not overwritten.
Further statements |\providecommand{\version}{...}|
can thus be added before the above code to override it.

For the main file, one might add a line
(between |\childdocmain| and the above block)
%
\begin{center}
|%\ifchilddoc\||else\providecommand{\version}{draft}\||fi|
\end{center}
%
which can be uncommented to produce a draft version.
Likewise one can add a line to the very top of a child file
(above the |\childdocof{|\textit{main}|}| directive)
%
\begin{center}
|%\providecommand{\version}{final}|
\end{center}
%
which can be uncommented to produce the final version of this child document.

%%%%%%%%%%%%%%%%%%%%%%%%%%%%%%%%%%%%%%%%%%%%%%%%%%%%%%%%%%%%%%%%%%%%%%%%%%%%%%%%
\subsection{Forwarding}
\label{sec:forward}

Different versions of the main or child documents
using compilation flags as described in \secref{sec:flags}
can be (permanently) stored in different files
for convenient compilation, viewing and distribution.
To this end, the package defines a command
to pass on compilation to a different file:

%%%%%%%%%%%%%%%%%%%%%%%%%%%%%%%%%%%%%%%%
\DescribeMacro{\childdocforward}
The command |\childdocforward| redirects processing to
another source file:
%
\begin{center}
\begin{tabular}{l}
|\input{childdoc.def}|\\
|\childdocforward[|\textit{main}|]{|\textit{dest}|}|\\
\end{tabular}
\end{center}
%
The argument \textit{dest} is the destination file
(without extension).
It should be the main file or one of the child files.
Note that further \textsf{childdoc} directives
such as |\childdocof| and |\childdocforward|
in the indicated file will be processed in this form.
The optional argument \textit{main}
passes on directly to the main file \textit{main}
while pretending to compile the child \textit{dest}.
This form behaves as if \textit{dest}
issues |\childdocof{|\textit{main}|}| right away,
and no further \textsf{childdoc} directives will be processed.

%%%%%%%%%%%%%%%%%%%%%%%%%%%%%%%%%%%%%%%%
\DescribeMacro{\...prefix}
In the alternative form |\childdocforwardprefix|,
%
\begin{center}
\begin{tabular}{l}
|\input{childdoc.def}|\\
|\childdocforwardprefix[|\textit{main}|]{|\textit{prefix}|}{|\textit{dest}|}|
\end{tabular}
\end{center}
%
the destination file is determined by a pattern
depending on the current file:
To make this work, the current file must be called
`{\textit{prefix}\hspace{0.2em}\textit{suffix}}'
with \textit{prefix} matching precisely the argument.
Processing is then passed on to the file
`{\textit{dest}\hspace{0.2em}\textit{suffix}}'.
Surely, the same effect is achieved by
directly specifying the
argument `{\textit{dest}\hspace{0.2em}\textit{suffix}}'
in the first form.
However, that requires to set up a different file
for each child. With the alternative form of the command
all these files can have exactly the same content
which simplifies setting them up and maintaining them.

For example, the following file |draft.tex|
with a compilation flag |\version| as described in \secref{sec:flags}
compiles the main document as a draft:
%
\begin{center}
\begin{tabular}{l}
|\def\version{draft}|\\
|\input{childdoc.def}|\\
|\childdocforward{|\textit{main}|}|
\end{tabular}
\end{center}
%
Likewise, the following files |final|\textit{nn}|.tex|
compile the final version of the child document
|child|\textit{nn}|.tex|:
%
\begin{center}
\begin{tabular}{l}
|\def\version{final}|\\
|\input{childdoc.def}|\\
|\childdocforwardprefix{final}{child}|
\end{tabular}
\end{center}
%

Note that when several versions of a main file and/or of each child file
are to be generated, it may be convenient to set up a |Makefile| or
shell script to automatise the process.

%%%%%%%%%%%%%%%%%%%%%%%%%%%%%%%%%%%%%%%%%%%%%%%%%%%%%%%%%%%%%%%%%%%%%%%%%%%%%%%%
\subsection{Command Line Processing}
\label{sec:commandline}

The effect of redirection files can also be achieved by invoking
the \LaTeX{} compiler with a more elaborate command line.
Most conveniently this should be done as part
of a shell script or a |Makefile|.

When using \textsf{childdoc} in the main file, the following
command lines effectively perform a redirection
(note that depending on the shell being used,
backslashes may have to be doubled: `|\|' $\to$ `|\\|'):
%
\begin{center}
|... -jobname "|\textit{target}|" |\\|"|[\textit{flags}]%
|\input{childdoc.def}\childdocforward[|\textit{main}|]{|\textit{dest}|}"|
\end{center}
%
Here \textit{target} is the name of the output file,
\textit{main} is the name of the main file
and \textit{dest} is the name of the main or child file to be processed
(all filenames without extensions).
The optional argument \textit{main} can be omitted
if \textit{main} matches \textit{dest}.
Optionally, compilation \textit{flags} can be defined via |\def| commands.
This command line makes the \TeX{} engine believe
it is compiling the file \textit{target}
whose content is specified as the latter parameter.
The provided code then forwards the processing to
\textit{main} or \textit{dest} as described in \secref{sec:forward}.

%%%%%%%%%%%%%%%%%%%%%%%%%%%%%%%%%%%%%%%%%%%%%%%%%%%%%%%%%%%%%%%%%%%%%%%%%%%%%%%%
\subsection{Include by Input}
\label{sec:input}

Including child documents by |\include| has some restrictions by design.
Most notably, the content of a child document always occupies
its own set of pages; pages cannot be shared between child documents.
Usually, this behaviour makes perfect sense
because each child document contain an essential part of the document.
However, in some situations it may be desirable to compose
a document from a collection of parts
without having mandatory page breaks between then.
For this case, the package
provides a mechanism to include parts
by |\input| which can also be processed individually.
However, by construction this mechanism
requires manual handling of the content to be output.

%%%%%%%%%%%%%%%%%%%%%%%%%%%%%%%%%%%%%%%%
\DescribeMacro{\ifchilddocmanual}
The main file should be prepared as usual, see \secref{sec:include}.
However, the document body must make a distinction
between processing of an individual part and of the main document, e.g.:
%
\begin{center}
\begin{tabular}{l}
|\ifchilddocmanual|\\
|\input{\childdocname}|\\
|\||else|\\
\textit{document body with }|\input{|\textit{part}|}|\\
|\||fi|
\end{tabular}
\end{center}
%
The conditional |\ifchilddocmanual| is true whenever
a part to be included by |\input| is being compiled,
and the name of the part is stored in |\childdocname|.

%%%%%%%%%%%%%%%%%%%%%%%%%%%%%%%%%%%%%%%%
\DescribeMacro{\childdocby}
Each part to be included by |\input| should start with:
%
\begin{center}
\begin{tabular}{l}
|\input{childdoc.def}|\\
|\childdocby{|\textit{main}|}|\\
\end{tabular}
\end{center}
%
The directive |\childdocby| is similar to |\childdocof|
described in \secref{sec:include},
but the subsequent selection of content must be done manually.
To that end, both |\ifchilddoc| and |\ifchilddocmanual|
will be true upon processing of a part,
and the name of the part is stored in |\childdocname|.
Note that |\jobname| will be set to the filename of the current part
so that each part receives an individual |.aux| file
that does not interfere with the |.aux| file(s) of the main document.
This behaviour can be altered by the alternative form
|\childdocby[*]{|\textit{main}|}| (with a non-empty optional argument)
which uses the |.aux| file of the main document
by setting |\jobname| to \textit{main}.

%%%%%%%%%%%%%%%%%%%%%%%%%%%%%%%%%%%%%%%%%%%%%%%%%%%%%%%%%%%%%%%%%%%%%%%%%%%%%%%%
\subsection{Driver Development}
\label{sec:driver}

The \textsf{childdoc} mechanism can also be use for the development
of definition files such as \LaTeX{} styles or classes.
This case differs from the above setup with multiple parts
included by |\include| in that no |\includeonly| should be invoked.
This can be achieved by starting the include file
(before |\ProvidesPackage|) with:
%
\begin{center}
\begin{tabular}{l}
|\input{childdoc.def}|\\
|\childdocforward{|\textit{main}|}|\\
\end{tabular}
\end{center}
%
or alternatively with:
%
\begin{center}
\begin{tabular}{l}
|\input{childdoc.def}|\\
|\childdocby{|\textit{main}|}|\\
\end{tabular}
\end{center}
%
Both forms have slightly different effects as described above.
The main file is prepared as usual, see \secref{sec:include}.

%%%%%%%%%%%%%%%%%%%%%%%%%%%%%%%%%%%%%%%%%%%%%%%%%%%%%%%%%%%%%%%%%%%%%%%%%%%%%%%%
\subsection{Legacy Detection}
\label{sec:detection}

The directive |\childdocmain| in the main file can detect
whether the complete document or merely a child is to be compiled
even without using the directive |\childdocof|.
This method is deprecated because it is less robust
and there is no compelling reason to use it;
it is merely provided for backward compatibility
and it may be removed in future versions.

If the detection mechanism is to be used,
it is mandatory to correctly specify
the filename of the main file as the argument of |\childdocmain|:
%
\begin{center}
\begin{tabular}{l}
|\input{childdoc.def}|\\
|\childdocmain{|\textit{main}|}|\\
\end{tabular}
\end{center}
%
If |\jobname| does not match the argument \textit{main} of |\childdocmain|,
it is assumed that |\jobname| points to the child file to be compiled.
When using |\childdocmain| with the main file specified as argument,
it suffices to start a child file
with just |\input{|\textit{main}|}|
without loading of the package and using |\childdocof|.
If instead all processing is done
with the appropriate \textsf{childdoc} directives,
the argument of \textit{main} of |\childdocmain| can be empty.

An alternative version of the command line processing described
in \secref{sec:commandline} using the detection mechanism reads:
%
\begin{center}
|... -jobname "|\textit{target}|" "|[\textit{flags}]%
[|\def\jobname{|\textit{dest}|}|]|\input{|\textit{main}|}"|
\end{center}

%%%%%%%%%%%%%%%%%%%%%%%%%%%%%%%%%%%%%%%%%%%%%%%%%%%%%%%%%%%%%%%%%%%%%%%%%%%%%%%%
\subsection{Manual Code}
\label{sec:manual}

In case one cannot be certain whether the definitions file |childdoc.def|
is installed on the target \TeX{} distribution
and one prefers not to ship it,
it is conceivable to paste a few relevant commands into the sources.

To that end, drop all statements |\input{childdoc.def}|
and perform the replacements as outlined below.
Instead of |\childdocmain{|\textit{main}|}| add the following code
to the top of the main file:
%
\begin{center}
\begin{tabular}{l}
|\||ifdefined\childdocname\endinput\||fi\newif\ifchilddoc|\\
|\edef\childdocname{\scantokens\expandafter{\jobname\noexpand}}|\\
|\def\childdocmain{|\textit{main}|}\||ifx\childdocmain\childdocname\||else|\\
|\childdoctrue\includeonly{\childdocname}\let\jobname\childdocmain\||fi|\\
\end{tabular}
\end{center}
%
Instead of |\childdocof{|\textit{main}|}| just include the main file
at the top of each child file:
%
\begin{center}
|\input{|\textit{main}|}|
\end{center}
%
A simple redirection |\childdocforward{|\textit{dest}|}| is achieved by:
%
\begin{center}
|\def\jobname{|\textit{dest}|}\input{\jobname}|
\end{center}
%
The redirection with prefix
|\childdocforwardprefix[|\textit{prefix}|]{|\textit{dest}|}|
is accomplished by:
%
\begin{center}
\begin{tabular}{l}
|{\edef\jobname{\scantokens\expandafter{\jobname\noexpand}}|\\
|\def\redirectjob |\textit{prefix}|#1~~~{\gdef\jobname{|\textit{dest}|#1}}|\\
|\expandafter\redirectjob\jobname~~~}\input{\jobname}|
\end{tabular}
\end{center}

In an alternative approach,
child documents can be compiled by a specific command line
without additional code or specific definitions:
%
\begin{center}
|... -jobname "|\textit{target}|" "|[\textit{flags}]%
|\includeonly{|\textit{dest}|}\input{|\textit{main}|}"|
\end{center}
%

%%%%%%%%%%%%%%%%%%%%%%%%%%%%%%%%%%%%%%%%%%%%%%%%%%%%%%%%%%%%%%%%%%%%%%%%%%%%%%%%
%%%%%%%%%%%%%%%%%%%%%%%%%%%%%%%%%%%%%%%%%%%%%%%%%%%%%%%%%%%%%%%%%%%%%%%%%%%%%%%%
\section{Information}

%%%%%%%%%%%%%%%%%%%%%%%%%%%%%%%%%%%%%%%%%%%%%%%%%%%%%%%%%%%%%%%%%%%%%%%%%%%%%%%%
\subsection{Copyright}

Copyright \copyright{} 2017--2018 Niklas Beisert

This work may be distributed and/or modified under the
conditions of the \LaTeX{} Project Public License, either version 1.3
of this license or (at your option) any later version.
The latest version of this license is in
  \url{http://www.latex-project.org/lppl.txt}
and version 1.3 or later is part of all distributions of \LaTeX{}
version 2005/12/01 or later.

This work has the LPPL maintenance status `maintained'.

The Current Maintainer of this work is Niklas Beisert.

This work consists of the files |README.txt|, |childdoc.ins| and |childdoc.dtx|
as well as the derived files |childdoc.def|, |cdocsamp.tex|
with |cdocsch1.tex|, |cdocsch2.tex|, |cdocspt3.tex|, |cdocspt4.tex|,
|cdocsdrf.tex|, |cdocsfn1.tex|, |cdocsfn2.tex|
as well as |childdoc.pdf|.

%%%%%%%%%%%%%%%%%%%%%%%%%%%%%%%%%%%%%%%%%%%%%%%%%%%%%%%%%%%%%%%%%%%%%%%%%%%%%%%%
\subsection{Files and Installation}

The package consists of the files:
%
\begin{center}
\begin{tabular}{ll}
    |README.txt|   & readme file \\
    |childdoc.ins| & installation file \\
    |childdoc.dtx| & source file \\
    |childdoc.def| & definition file \\
    |cdocsamp.tex| & sample main file \\
    |cdocsch1.tex| & sample include file \\
    |cdocsch2.tex| & sample include file \\
    |cdocspt3.tex| & sample part file \\
    |cdocspt4.tex| & sample part file \\
    |cdocsdrf.tex| & sample redirection file \\
    |cdocsfn1.tex| & sample redirection file \\
    |cdocsfn2.tex| & sample redirection file \\
    |childdoc.pdf| & manual
\end{tabular}
\end{center}
%
The distribution consists of the files
|README.txt|, |childdoc.ins| and |childdoc.dtx|.
%
\begin{itemize}
\item
Run (pdf)\LaTeX{} on |childdoc.dtx|
to compile the manual |childdoc.pdf| (this file).
\item
Run \LaTeX{} on |childdoc.ins| to create the definitions file |childdoc.def|
and the sample |cdocsamp.tex| with include files
|cdocsch1.tex|, |cdocsch2.tex|, |cdocspt3.tex|, |cdocspt4.tex|,
|cdocsdrf.tex|, |cdocsfn1.tex|, |cdocsfn2.tex|.
Then copy the file |childdoc.def| to an appropriate directory of your \LaTeX{}
distribution, e.g.\ \textit{texmf-root}|/tex/latex/childdoc|.
\end{itemize}

%%%%%%%%%%%%%%%%%%%%%%%%%%%%%%%%%%%%%%%%%%%%%%%%%%%%%%%%%%%%%%%%%%%%%%%%%%%%%%%%
\subsection{Related CTAN Packages}

There are several other packages which offer a similar functionality:
%
\begin{itemize}
\item
The packages
\href{http://ctan.org/pkg/docmute}{\textsf{docmute}},
\href{http://ctan.org/pkg/includex}{\textsf{includex}} and
\href{http://ctan.org/pkg/standalone}{\textsf{standalone}}
provide commands to include only the document body of
a child file thus allowing both files to be compiled individually.
\item
The packages \href{http://ctan.org/pkg/subdocs}{\textsf{subdocs}}
and \href{http://ctan.org/pkg/subfiles}{\textsf{subfiles}}
provide structures in which the main and child documents can be
encapsulated and allowing them to be compiled individually.
The inclusion mechanism is different from the conventional |\include|.
\item
The package \href{http://ctan.org/pkg/combine}{\textsf{combine}}
is an elaborate solution to combine several documents into one.
\end{itemize}
%
See also the CTAN topic \href{http://ctan.org/topic/subdocs}{\textsf{subdocs}}
for further related packages.
The present package differs from the above solutions in that
a document structure constructed with the conventional |\include| mechanism
just needs two extra commands at the top of every file
such that all constituent files can be compiled individually.

%%%%%%%%%%%%%%%%%%%%%%%%%%%%%%%%%%%%%%%%%%%%%%%%%%%%%%%%%%%%%%%%%%%%%%%%%%%%%%%%
%\subsection{Feature Suggestions}
%
%The following is a list of features which may be useful for future
%versions of this package:
%%
%\begin{itemize}
%\item
%\ldots
%\end{itemize}

%%%%%%%%%%%%%%%%%%%%%%%%%%%%%%%%%%%%%%%%%%%%%%%%%%%%%%%%%%%%%%%%%%%%%%%%%%%%%%%%
\subsection{Revision History}

%%%%%%%%%%%%%%%%%%%%%%%%%%%%%%%%%%%%%%%%
\paragraph{v2.0:} 2018/12/30

\begin{itemize}
\item
immediate forward processing
\item
added |\childdocby| mechanism
\item
manual restructured
\end{itemize}

%%%%%%%%%%%%%%%%%%%%%%%%%%%%%%%%%%%%%%%%
\paragraph{v1.6:} 2018/01/17

\begin{itemize}
\item
application for development of include files
\item
corrections to manual
\end{itemize}

%%%%%%%%%%%%%%%%%%%%%%%%%%%%%%%%%%%%%%%%
\paragraph{v1.5:} 2017/05/21

\begin{itemize}
\item
more complete structuring introduced
\item
|\childdocof| introduced
\item
|\childdoc| renamed to |\childdocmain|
\item
|\childredirect| renamed to |\childdocforward| and |\childdocforwardprefix|
and functionality expanded
\end{itemize}

%%%%%%%%%%%%%%%%%%%%%%%%%%%%%%%%%%%%%%%%
\paragraph{v1.0:} 2017/04/27

\begin{itemize}
\item
manual and install package
\item
first version published on CTAN
\end{itemize}

%%%%%%%%%%%%%%%%%%%%%%%%%%%%%%%%%%%%%%%%
\paragraph{v0.6:} 2017/04/26

\begin{itemize}
\item
redirection mechanism added
\end{itemize}

%%%%%%%%%%%%%%%%%%%%%%%%%%%%%%%%%%%%%%%%
\paragraph{v0.5:} 2017/04/26

\begin{itemize}
\item
functionality in definition file
\end{itemize}


%%%%%%%%%%%%%%%%%%%%%%%%%%%%%%%%%%%%%%%%%%%%%%%%%%%%%%%%%%%%%%%%%%%%%%%%%%%%%%%%
%%%%%%%%%%%%%%%%%%%%%%%%%%%%%%%%%%%%%%%%%%%%%%%%%%%%%%%%%%%%%%%%%%%%%%%%%%%%%%%%
%%%%%%%%%%%%%%%%%%%%%%%%%%%%%%%%%%%%%%%%%%%%%%%%%%%%%%%%%%%%%%%%%%%%%%%%%%%%%%%%
\appendix

\settowidth\MacroIndent{\rmfamily\scriptsize 000\ }

 \DocInput{childdoc.dtx}

\end{document}
%</driver>
% \fi
%
% %%%%%%%%%%%%%%%%%%%%%%%%%%%%%%%%%%%%%%%%%%%%%%%%%%%%%%%%%%%%%%%%%%%%%%%%%%%%%%
% %%%%%%%%%%%%%%%%%%%%%%%%%%%%%%%%%%%%%%%%%%%%%%%%%%%%%%%%%%%%%%%%%%%%%%%%%%%%%%
% \section{Sample}
%\iffalse
%<*samplemain>
%\fi
%
% The following presents a sample document
% with two chapters, two parts, a title page,
% a compile flag as well as three forwarding files to set the flag.
% It consists of eight |.tex| files:
% \begin{center}
% \begin{tabular}{ll}
% |cdocsamp.tex|&main file\\
% |cdocsch1.tex|&include file for chapter 1\\
% |cdocsch2.tex|&include file for chapter 2\\
% |cdocspt3.tex|&include file for part 3\\
% |cdocspt4.tex|&include file for part 4\\
% |cdocsdrf.tex|&forwarding file for main file in draft mode\\
% |cdocsfi1.tex|&forwarding file for final version of chapter 1\\
% |cdocsfi2.tex|&forwarding file for final version of chapter 2\\
% \end{tabular}
% \end{center}
% Each of the eight files can be compiled directly by the \LaTeX{} compiler.
%
% %%%%%%%%%%%%%%%%%%%%%%%%%%%%%%%%%%%%%%
% \paragraph{Main File.}
%
% The main file is called |cdocsamp.tex|.
%
% Load the \textsf{childdoc} definitions and
% declare the filename for the main document:
%    \begin{macrocode}
\input{childdoc.def}
\childdocmain{}
%    \end{macrocode}

% Optional override for |\version| flag:
%    \begin{macrocode}
%%\ifchilddoc\else\providecommand{\version}{draft}\fi
%    \end{macrocode}

% Define the default values for the |\version| flag
% (|final| for the main file and |draft| for childs):
%    \begin{macrocode}
\ifchilddoc
\providecommand{\version}{draft}
\else
\providecommand{\version}{final}
\fi
%    \end{macrocode}

% Load the standard document class:
%    \begin{macrocode}
\documentclass[12pt]{article}
%    \end{macrocode}

% Start the document body:
%    \begin{macrocode}
\begin{document}
%    \end{macrocode}

% Declare a title page.
% Print title, part of document being processed and version flag:
%    \begin{macrocode}
\addtocounter{page}{-1}
\begin{center}
{\LARGE\bfseries{}childdoc example\par}
\vspace{1cm}
\ifchilddoc
\ifchilddocmanual part\else chapter\fi:
`\childdocname' of `\childdocjob'\par
\else
main document: `\childdocjob'\par
\fi
version: \version\par
\end{center}
\newpage
%    \end{macrocode}

% Manually include selected file,
% otherwise process as usual:
%    \begin{macrocode}
\ifchilddocmanual
\section*{part `\childdocname'}
\input{\childdocname}
\else
%    \end{macrocode}

% Include the two chapters:
%    \begin{macrocode}
\include{cdocsch1}
\include{cdocsch2}
%    \end{macrocode}

% Include the two parts unless only chapters should be displayed:
%    \begin{macrocode}
\ifchilddoc\else
\section{part three}
\input{cdocspt3}
\section{part four}
\input{cdocspt4}
\fi
%    \end{macrocode}

% Process as usual until here:
%    \begin{macrocode}
\fi
%    \end{macrocode}

% End of document body:
%    \begin{macrocode}
\end{document}
%    \end{macrocode}
%\iffalse
%</samplemain>
%\fi
%
% %%%%%%%%%%%%%%%%%%%%%%%%%%%%%%%%%%%%%%
% \paragraph{Chapter Include Files.}
%
% The include files are called |cdocsch1.tex| and |cdocsch2.tex|.
%
%\iffalse
%<*samplechap1|samplechap2>
%\fi

% Optional override for |\version| flag:
%    \begin{macrocode}
%%\providecommand{\version}{final}
%    \end{macrocode}

% Include the main document:
%    \begin{macrocode}
\input{childdoc.def}
\childdocof{cdocsamp}
%    \end{macrocode}

%\iffalse
%</samplechap1|samplechap2>
%\fi
%
%\iffalse
%<*samplechap1>
%\fi
% Some text for chapter 1:
%    \begin{macrocode}
\section{one}
some text in chapter one
%    \end{macrocode}

%\iffalse
%</samplechap1>
%\fi
% Some text for chapter 2:
%\iffalse
%<*samplechap2>
%\fi
%    \begin{macrocode}
\section{two}
more text in chapter two
%    \end{macrocode}

%\iffalse
%</samplechap2>
%\fi
%
% %%%%%%%%%%%%%%%%%%%%%%%%%%%%%%%%%%%%%%
% \paragraph{Part Include Files.}
%
% The include files are called |cdocspt3.tex| and |cdocspt4.tex|.
%
%\iffalse
%<*samplepart3|samplepart4>
%\fi

% Optional override for |\version| flag:
%    \begin{macrocode}
%%\providecommand{\version}{final}
%    \end{macrocode}

% Include the main document:
%    \begin{macrocode}
\input{childdoc.def}
\childdocby{cdocsamp}
%    \end{macrocode}

%\iffalse
%</samplepart3|samplepart4>
%\fi
%
%\iffalse
%<*samplepart3>
%\fi
% Some text for part 3:
%    \begin{macrocode}
some text in part three
%    \end{macrocode}

%\iffalse
%</samplepart3>
%\fi
% Some text for part 4:
%\iffalse
%<*samplepart4>
%\fi
%    \begin{macrocode}
more text in part four
%    \end{macrocode}

%\iffalse
%</samplepart4>
%\fi
%
% %%%%%%%%%%%%%%%%%%%%%%%%%%%%%%%%%%%%%%
% \paragraph{Forwarding for a Complete Draft.}
%
% The following forwarding file |cdocsdrf.tex|
% compiles the main document in draft mode:
%\iffalse
%<*sampledraft>
%\fi
%    \begin{macrocode}
\def\version{draft}
\input{childdoc.def}
\childdocforward{cdocsamp}
%    \end{macrocode}

%\iffalse
%</sampledraft>
%\fi
%
% %%%%%%%%%%%%%%%%%%%%%%%%%%%%%%%%%%%%%%
% \paragraph{Forwarding for Final Version of the Chapters.}
%
% The following forwarding files |cdocsfn1.tex| and |cdocsfn2.tex|
% (with identical content)
% compile the final versions of the child documents
% |cdocsch1.tex| and |cdocsch2.tex|, respectively:
%\iffalse
%<*samplefinal>
%\fi
%    \begin{macrocode}
\def\version{final}
\input{childdoc.def}
\childdocforwardprefix[cdocsamp]{cdocsfn}{cdocsch}
%    \end{macrocode}

%\iffalse
%</samplefinal>
%\fi
%
% %%%%%%%%%%%%%%%%%%%%%%%%%%%%%%%%%%%%%%
% \paragraph{Command Line Processing.}
%
% The following three command lines generate the output files
% |cdocscld|, |cdocscl1| and |cdocscl2|
% which should be identical to
% |cdocsdrf|, |cdocsch1| and |cdocsfn2|, respectively:
% \begin{center}
% \begin{tabular}{l}
% |latex -jobname cdocscld \|\\
% |  "\def\version{draft}\input{childdoc.def}\childdocforward{cdocsamp}"|\\
% |latex -jobname cdocscl1 \|\\
% |  "\input{childdoc.def}\childdocforward[cdocsamp]{cdocsch1}"|\\
% |latex -jobname cdocscl2 \|\\
% |  "\def\version{final}\input{childdoc.def}\childdocforward{cdocsch2}"|
% \end{tabular}
% \end{center}
% Note that the trailing backslash on each first line
% merely continues the input to the second line
% (for convenient cut ant paste).
% Furthermore, the command |latex| can be replaced by any
% of its alternative versions such as |pdflatex|.
%
% %%%%%%%%%%%%%%%%%%%%%%%%%%%%%%%%%%%%%%%%%%%%%%%%%%%%%%%%%%%%%%%%%%%%%%%%%%%%%%
% %%%%%%%%%%%%%%%%%%%%%%%%%%%%%%%%%%%%%%%%%%%%%%%%%%%%%%%%%%%%%%%%%%%%%%%%%%%%%%
% \section{Implementation}
%\iffalse
%<*package>
%\fi
%
% This section describes the definitions file |childdoc.def|.

% The definitions cannot be loaded using |\usepackage| or |\RequirePackage|
% which has a mechanism to prevent loading a style file more than once.
% When loading the definitions by means of |\input|
% multiple instances have to be prevented manually:
%\iffalse
%This code needs to be before the `\ProvidesFile' directive
%which is defined at the beginning of this file.
%Therefore it is also placed there and commented out here.
%</package>
%<*discard>
%\fi
%    \begin{macrocode}
\ifdefined\childdocmain\endinput\fi
%    \end{macrocode}
%\iffalse
%</discard>
%<*package>
%\fi
%
% \macro{\ifchilddoc}
% \macro{\ifchilddocmanual}
% The conditional |\ifchilddoc| tells whether a
% child (true) or main (false) document is being compiled.
% The conditional |\ifchilddocmanual| tells whether
% the |\includeonly| mechanism is used (false) or
% the selection of child files must be performed manually (true).
% The definitions initialise to false:
%    \begin{macrocode}
\newif\ifchilddoc
\newif\ifchilddocmanual
%    \end{macrocode}

% \macro{\childdocname}
% \macro{\childdocjob}
% The macro |\childdocname| stores the name of the main document
% to be compiled. The macro |\childdocjob| stores the name of
% the document on which the \LaTeX{} compiler was originally invoked.
% The content of |\jobname| cannot be compared
% to filenames specified in the source due to different catcodes.
% The following code rescans |\jobname|, stores the result
% in |\childdocname| and saves a copy in |\childdocjob|:
%    \begin{macrocode}
\edef\childdocname{\scantokens\expandafter{\jobname\noexpand}}
\let\childdocjob\childdocname
%    \end{macrocode}

% \macro{\childdocdisable}
% The macro |\childdocdisable| prevents the main file
% from being processed more than once.
% At this stage, the main document command |\childdocmain|
% is assumed to be called once again where it should do nothing.
% Any subsequent call to it should prevent
% a secondary processing of the main document
% It overwrites the forwarding commands
% |\childdocof| and |\childdocforward|
% with empty macros to prevent further inclusions of the main document:
%    \begin{macrocode}
\newcommand{\childdocdisable}
{
  \renewcommand{\childdocmain}[1]{\renewcommand{\childdocmain}[1]{\endinput}}
  \renewcommand{\childdocof}[1]{}
  \renewcommand{\childdocby}[2][]{}
  \renewcommand{\childdocforward}[2][]{}
  \renewcommand{\childdocdisable}{}
}
%    \end{macrocode}

% \macro{\childdocmain}
% The macro |\childdocmain| is to be called at the top of the main file
% with nothing or the main filename (without extension) as argument.
% First, it breaks loops.
% If the argument is not empty and does not match |\childdocname|
% (which is set by the first inclusion of |childdoc.def|),
% |\ifchilddoc| is set to true, |\includeonly| is applied to the child file
% and |\jobname| is set to the main file
% (for proper handling of |.aux| files):
%    \begin{macrocode}
\newcommand{\childdocmain}[1]
{
  \childdocdisable\childdocmain{}
  \if?#1?\else
    \begingroup
      \def\childdoctmp{#1}
      \ifx\childdoctmp\childdocname
        \def\childdoctmp{}
      \else
        \def\childdoctmp
        {
          \childdoctrue
          \includeonly{\childdocname}
          \def\childdocjob{#1}
          \def\jobname{#1}
        }
      \fi
      \expandafter
    \endgroup
    \childdoctmp
  \fi
}
%    \end{macrocode}

% \macro{\childdocof}
% The command |\childdocof| redirects
% compilation to the main file |#1|.
%    \begin{macrocode}
\newcommand{\childdocof}[1]
{
  \childdocdisable
  \childdoctrue
  \includeonly{\childdocname}
  \def\jobname{#1}
  \def\childdocjob{#1}
  \input{#1}
}
%    \end{macrocode}

% \macro{\childdocby}
% The command |\childdocby| ....
%    \begin{macrocode}
\newcommand{\childdocby}[2][]
{
  \childdocdisable
  \childdoctrue
  \childdocmanualtrue
  \if?#1?\else
    \def\jobname{#2}
  \fi
  \def\childdocjob{#2}
  \input{#2}
  \endinput
}
%    \end{macrocode}

% \macro{\childdocforward}
% The command |\childdocforward| redirects
% compilation to the main file or
% (if the optional argument is given) a child file.
% Parameters are set as if the main file
% or a child file starting with |\childdocof| was compiled.
% Then compilation is handed over to the main file:
%    \begin{macrocode}
\newcommand{\childdocforward}[2][]
{
  \begingroup
    \if?#1?
      \def\childdoctmp
      {
        \def\childdocname{#2}
        \def\childdocjob{#2}
        \def\jobname{#2}
        \input{#2}
        \endinput
      }
    \else
      \def\childdoctmp
      {
        \childdocdisable
        \def\childdocname{#2}
        \childdoctrue
        \includeonly{#2}
        \def\childdocjob{#1}
        \def\jobname{#1}
        \input{#1}
        \endinput
      }
    \fi
    \expandafter
  \endgroup
  \childdoctmp
}
%    \end{macrocode}

% \macro{\childdocforwardprefix}
% The command |\childdocforwardprefix| redirects
% compilation to the main or a child file by means of a pattern.
% The prefix |#1| in the current filename is replaced by |#2|
% and the suffix of the current filename is kept
% (it is assumed that the filename does not contain the substring `|~~~|'
% which is used as a delimiter).
% Compilation is handed over to the new file by |\childdocforward|:
%    \begin{macrocode}
\newcommand{\childdocforwardprefix}[3][]
{
  \begingroup
    \def\childdocextract #2##1~~~{\def\childdoctmp{\childdocforward[#1]{#3##1}}}
    \expandafter\childdocextract\childdocname~~~
    \expandafter
  \endgroup
  \childdoctmp
}
%    \end{macrocode}

% \macro{\childdoc}
% The deprecated macro |\childdoc| is a legacy version of |\childdocmain|:
%    \begin{macrocode}
\newcommand{\childdoc}{\childdocmain}
%    \end{macrocode}

% \macro{\childdocredirect}
% The deprecated macro |\childdocredirect| is a legacy version
% of |\childdocforward| and |\childdocforwardprefix|:
%    \begin{macrocode}
\newcommand{\childdocredirect}[2][]
{
  \begingroup
    \if?#1?
      \def\childdoctmp{\childdocforward{#2}}
    \else
      \def\childdoctmp{\childdocforwardprefix{#1}{#2}}
    \fi
    \expandafter
  \endgroup
  \childdoctmp
}
%    \end{macrocode}

%\iffalse
%</package>
%\fi
%
\endinput
|\\
|\childdocforwardprefix{final}{child}|
\end{tabular}
\end{center}
%

Note that when several versions of a main file and/or of each child file
are to be generated, it may be convenient to set up a |Makefile| or
shell script to automatise the process.

%%%%%%%%%%%%%%%%%%%%%%%%%%%%%%%%%%%%%%%%%%%%%%%%%%%%%%%%%%%%%%%%%%%%%%%%%%%%%%%%
\subsection{Command Line Processing}
\label{sec:commandline}

The effect of redirection files can also be achieved by invoking
the \LaTeX{} compiler with a more elaborate command line.
Most conveniently this should be done as part
of a shell script or a |Makefile|.

When using \textsf{childdoc} in the main file, the following
command lines effectively perform a redirection
(note that depending on the shell being used,
backslashes may have to be doubled: `|\|' $\to$ `|\\|'):
%
\begin{center}
|... -jobname "|\textit{target}|" |\\|"|[\textit{flags}]%
|% \iffalse
%
% childdoc.dtx Copyright (C) 2017-2018 Niklas Beisert
%
% This work may be distributed and/or modified under the
% conditions of the LaTeX Project Public License, either version 1.3
% of this license or (at your option) any later version.
% The latest version of this license is in
%   http://www.latex-project.org/lppl.txt
% and version 1.3 or later is part of all distributions of LaTeX
% version 2005/12/01 or later.
%
% This work has the LPPL maintenance status `maintained'.
%
% The Current Maintainer of this work is Niklas Beisert.
%
% This work consists of the files childdoc.dtx and childdoc.ins
% and the derived files childdoc.def and cdocsamp.tex with
% cdocsch1.tex, cdocsch2.tex, cdocsdrf.tex, cdocsfn1.tex, cdocsfn2.tex.
%
%<package>\ifdefined\childdocmain\endinput\fi
%<package>\ProvidesFile{childdoc.def}[2018/12/30 v2.0 child document driver]
%<samplemain>\ProvidesFile{cdocsamp.tex}[2018/12/30 v2.0 sample for childdoc]
%<*driver>
%\ProvidesFile{childdoc.drv}[2018/12/30 v2.0 childdoc reference manual file]
\PassOptionsToClass{10pt,a4paper}{article}
\documentclass{ltxdoc}

\usepackage[margin=35mm]{geometry}
\usepackage{hyperref}
\usepackage{hyperxmp}
\usepackage[usenames]{color}

\hypersetup{colorlinks=true}
\hypersetup{pdfstartview=FitH}
\hypersetup{pdfpagemode=UseNone}
\hypersetup{pdfsource={}}
\hypersetup{pdflang={en-UK}}
\hypersetup{pdfcopyright={Copyright 2017-2018 Niklas Beisert.
  This work may be distributed and/or modified under the
  conditions of the LaTeX Project Public License, either version 1.3
  of this license or (at your option) any later version.}}
\hypersetup{pdflicenseurl={http://www.latex-project.org/lppl.txt}}
\hypersetup{pdfcontactaddress={ETH Zurich, ITP, HIT K,
  Wolfgang-Pauli-Strasse 27}}
\hypersetup{pdfcontactpostcode={8093}}
\hypersetup{pdfcontactcity={Zurich}}
\hypersetup{pdfcontactcountry={Switzerland}}
\hypersetup{pdfcontactemail={nbeisert@itp.phys.ethz.ch}}
\hypersetup{pdfcontacturl={http://people.phys.ethz.ch/\xmptilde nbeisert/}}

\newcommand{\secref}[1]{\hyperref[#1]{section \ref*{#1}}}

\parskip1ex
\parindent0pt
\let\olditemize\itemize
\def\itemize{\olditemize\parskip0pt}

\begin{document}

\title{The \textsf{childdoc} Package}
\hypersetup{pdftitle={The childdoc Package}}
\author{Niklas Beisert\\[2ex]
  Institut f\"ur Theoretische Physik\\
  Eidgen\"ossische Technische Hochschule Z\"urich\\
  Wolfgang-Pauli-Strasse 27, 8093 Z\"urich, Switzerland\\[1ex]
  \href{mailto:nbeisert@itp.phys.ethz.ch}
  {\texttt{nbeisert@itp.phys.ethz.ch}}}
\hypersetup{pdfauthor={Niklas Beisert}}
\hypersetup{pdfsubject={Manual for the LaTeX2e Package childdoc}}
\date{30 December 2018, \textsf{v2.0}}
\maketitle

\begin{abstract}\noindent
\textsf{childdoc} is a \LaTeXe{} package
that enables the direct compilation
of document sections included by |\include|
to individual files.
\end{abstract}

\begingroup
\parskip0ex
\tableofcontents
\endgroup

%%%%%%%%%%%%%%%%%%%%%%%%%%%%%%%%%%%%%%%%%%%%%%%%%%%%%%%%%%%%%%%%%%%%%%%%%%%%%%%%
%%%%%%%%%%%%%%%%%%%%%%%%%%%%%%%%%%%%%%%%%%%%%%%%%%%%%%%%%%%%%%%%%%%%%%%%%%%%%%%%
\section{Introduction}

\LaTeX{} provides a mechanism to structure a large document (such as a book)
into a main file and several child files (containing the chapters)
using the |\include| command.
This mechanism is beneficial for documents
which span hundreds of pages in order to
make the source file(s) more manageable.
Moreover, compilation can be restricted to
selected child files by means of the |\includeonly| command.
The latter feature can be used to reduce the compilation time while editing
(this was significantly more useful in the earlier days of \LaTeX{})
or to generate a smaller document which is easier to navigate.
Another application of |\includeonly| is to generate
documents consisting of selected parts of the complete document.

However, there are a few drawbacks of the plain |\include| mechanism:
\begin{itemize}
\item
The child files cannot be compiled on their own,
they can only be compiled via the main file.
A naive editing environment
(such as a text editor with an option
to have the current file processed by \LaTeX)
may require one to switch to the main file before compiling;
attempting to compile the child file produces errors.
\item
The main file must be modified (each time)
to adjust the |\includeonly| command
to the present needs. This easily leaves the main file in a messy state.
\item
The generated document will always carry the filename
of the main document. This is inconvenient if
several child files are to be compiled and
to be kept for distribution.
\end{itemize}

The present package provides a simple interface
to make child files individually compilable by \LaTeX{}.
Compiling a child file then has the same effect as compiling
the main file with an |\includeonly| command
to select the appropriate child.
Moreover the generated document will carry the name of the child
rather than the main file.
This resolves all three above issues.

This feature is meant to make the editing of books,
thesis documents and lecture notes somewhat more convenient.
However, the package can also be used efficiently for
composing a series of documents (such as exercise sheets)
which are typically distributed individually.
It then assists the author in generating the individual documents
(potentially in different versions)
as well as a document containing the collected series.
Another application is in developing style files
or other kinds of included material
where compilation of the style file could redirect
to a sample or test file.

%%%%%%%%%%%%%%%%%%%%%%%%%%%%%%%%%%%%%%%%%%%%%%%%%%%%%%%%%%%%%%%%%%%%%%%%%%%%%%%%
%%%%%%%%%%%%%%%%%%%%%%%%%%%%%%%%%%%%%%%%%%%%%%%%%%%%%%%%%%%%%%%%%%%%%%%%%%%%%%%%
\section{Usage}

First of all, the package \textsf{childdoc} is \emph{not} a standard
\LaTeXe{} |.sty| style file! Therefore it needs to be invoked in
a non-standard way.

%%%%%%%%%%%%%%%%%%%%%%%%%%%%%%%%%%%%%%%%%%%%%%%%%%%%%%%%%%%%%%%%%%%%%%%%%%%%%%%%
\subsection{Included Files}
\label{sec:include}

%%%%%%%%%%%%%%%%%%%%%%%%%%%%%%%%%%%%%%%%
\DescribeMacro{\childdocmain}
To use the package, add the commands
\begin{center}
\begin{tabular}{l}
|\input{childdoc.def}|\\
|\childdocmain{}|\\
\end{tabular}
\end{center}
at the very top of the main \LaTeX{} file,
in particular \emph{before} the |\documentclass| statement!
The argument of |\childdocmain| should be left empty
(but it must be present).

%%%%%%%%%%%%%%%%%%%%%%%%%%%%%%%%%%%%%%%%
\DescribeMacro{\childdocof}
Furthermore, add the commands
\begin{center}
\begin{tabular}{l}
|\input{childdoc.def}|\\
|\childdocof{|\textit{main}|}|\\
\end{tabular}
\end{center}
at the top of every child file \textit{child}
which is included by |\include{|\textit{child}|}|
from within the main file
(or at least for those files to be compiled individually).
The argument \textit{main} must be the filename of the main file.

There are a couple of
considerations in setting up the main and child documents:

%%%%%%%%%%%%%%%%%%%%%%%%%%%%%%%%%%%%%%%%
\paragraph{Restrictions.}

Please note the following restrictions:
\begin{itemize}
\item
|\childdocmain| must be called with one argument \textit{main}
to ensure compatibility with earlier version of the package.
It must either be empty (|\childdocmain{}|)
or precisely match the filename of the main file in which it is specified.
See \secref{sec:detection} for further information.
\item
The filename \textit{main} must be specified without the |.tex| extension.
\item
The filename \textit{main} is case sensitive
(even in case-insensitive file systems)
due to internal string comparison.
\item
The argument \textit{main} should be fully expanded, it cannot be a macro.
\item
Subdirectories and special characters should be avoided in filenames.
\item
The command |\childdocmain{|\textit{main}|}| must be followed by a whitespace.
It should not be followed immediately by another command
or by a comment mark `|%|'.
This is because the \TeX{} parser reads the token immediately following
the argument of |\childdocmain| and puts it
at the beginning of every child section;
however, a white\-space is ignored.
\end{itemize}

%%%%%%%%%%%%%%%%%%%%%%%%%%%%%%%%%%%%%%%%
\paragraph{Content of Main File.}

It is advisable to place all content in the child files included by |\include|.
Any output contained in the main file will appear in all child documents
unless suppressed manually;
it cannot be suppressed automatically by the |\includeonly| directive
and thus should normally be avoided.
A method to include some content in the main file
by means of conditional processing is described in \secref{sec:conditional}.

%%%%%%%%%%%%%%%%%%%%%%%%%%%%%%%%%%%%%%%%
\paragraph{Page Numbering.}

When only a part of the document is compiled,
the appropriate numbering of pages
(as well as other status parameters)
is determined from the |.aux| files.
The latter contain information from previous passes.
However this information needs to propagate through
all intermediate child documents.
Therefore the page numbering in child documents may well
be inconsistent until the complete document is compiled at least once.

A useful (if unconventional) way to always ensure a consistent
page numbering is to restart the numbering in each child document
and denote the pages by `\textit{child}|.|\textit{page}'
where \textit{child} represents the chapter/section number of the child file.
This can be achieved by the command
|\numberwithin{page}{|\textit{child}|}|
of the \textsf{amsmath} package
where \textit{child} can be |chapter| or |section|
depending on the chosen structuring.
Alternatively, one can modify the macro |\thepage| appropriately
and reset the counter |page| at the start of each child file.

%%%%%%%%%%%%%%%%%%%%%%%%%%%%%%%%%%%%%%%%%%%%%%%%%%%%%%%%%%%%%%%%%%%%%%%%%%%%%%%%
\subsection{Conditional Processing}
\label{sec:conditional}

The package provides a mechanism to compile different versions
of a document. To customise the versions further some conditional processing
can come in handy to distinguish which version is being compiled.
The package provides two macros to describe the compilation context:

%%%%%%%%%%%%%%%%%%%%%%%%%%%%%%%%%%%%%%%%
\DescribeMacro{\ifchilddoc}
The conditional |\ifchilddoc| distinguishes between the compilation of
child documents and the main document:
%
\begin{center}
|\ifchilddoc |\textit{child-code}| |[|\||else |\textit{main-code}]| \||fi|
\end{center}

%%%%%%%%%%%%%%%%%%%%%%%%%%%%%%%%%%%%%%%%
\DescribeMacro{\childdocname}
\DescribeMacro{\childdocjob}
The macro |\childdocname| contains the filename (without extension)
of the main or child file being processed.
Note that |\childdocjob| will always contain the name of the main file.

%%%%%%%%%%%%%%%%%%%%%%%%%%%%%%%%%%%%%%%%
\paragraph{Title Page.}

Conditional processing can be used to include a title or banner page
in the main document when proper precautions are taken.
Importantly, the code in the main file should ensure that the page counter
(as well as other status parameters which are stored in the |.aux| files)
takes the same value after the conditional processing.
Otherwise the page numbers may take divergent values
depending on which part is compiled.

For example, a title page could be declared by:
%
\begin{center}
\begin{tabular}{l}
|\ifchilddoc\||else|\\
|\addtocounter{page}{-1}|\\
\textit{code for title page}\\
|\newpage|\\
|\||fi|
\end{tabular}
\end{center}
%
A banner page for the child documents can be generated by:
%
\begin{center}
\begin{tabular}{l}
|\ifchilddoc|\\
|\addtocounter{page}{-1}|\\
\textit{code for banner page}\\
|\newpage|\\
|\||fi|
\end{tabular}
\end{center}
%
Here one could write a message such as:
\begin{center}
|This is the part \childdocname{} of \childdocjob{}.|
\end{center}

%%%%%%%%%%%%%%%%%%%%%%%%%%%%%%%%%%%%%%%%%%%%%%%%%%%%%%%%%%%%%%%%%%%%%%%%%%%%%%%%
\subsection{Flags}
\label{sec:flags}

The package makes it easy to generate different versions
of the main or child documents.
To this end compilation flags can be defined
and assigned different default values.
They will be particularly useful in conjunction
with the forwarding mechanism described in \secref{sec:forward}.

For example, it may be useful to have a flag |\version|
which can be set to |draft| or |final|.
The document source will contain some conditional code
depending on the value of |\version|.
Suppose further, the flag should default to |final| for the main file
and to |draft| for child files
which is a natural assignment for editing the document.
This is achieved by placing the following code
in the preamble of the main document
(below the |\childdocmain| directive):
%
\begin{center}
\begin{tabular}{l}
|\ifchilddoc|\\
|\providecommand{\version}{draft}|\\
|\||else|\\
|\providecommand{\version}{final}|\\
|\||fi|
\end{tabular}
\end{center}
%
The definition by |\providecommand| makes sure
that previous definitions are not overwritten.
Further statements |\providecommand{\version}{...}|
can thus be added before the above code to override it.

For the main file, one might add a line
(between |\childdocmain| and the above block)
%
\begin{center}
|%\ifchilddoc\||else\providecommand{\version}{draft}\||fi|
\end{center}
%
which can be uncommented to produce a draft version.
Likewise one can add a line to the very top of a child file
(above the |\childdocof{|\textit{main}|}| directive)
%
\begin{center}
|%\providecommand{\version}{final}|
\end{center}
%
which can be uncommented to produce the final version of this child document.

%%%%%%%%%%%%%%%%%%%%%%%%%%%%%%%%%%%%%%%%%%%%%%%%%%%%%%%%%%%%%%%%%%%%%%%%%%%%%%%%
\subsection{Forwarding}
\label{sec:forward}

Different versions of the main or child documents
using compilation flags as described in \secref{sec:flags}
can be (permanently) stored in different files
for convenient compilation, viewing and distribution.
To this end, the package defines a command
to pass on compilation to a different file:

%%%%%%%%%%%%%%%%%%%%%%%%%%%%%%%%%%%%%%%%
\DescribeMacro{\childdocforward}
The command |\childdocforward| redirects processing to
another source file:
%
\begin{center}
\begin{tabular}{l}
|\input{childdoc.def}|\\
|\childdocforward[|\textit{main}|]{|\textit{dest}|}|\\
\end{tabular}
\end{center}
%
The argument \textit{dest} is the destination file
(without extension).
It should be the main file or one of the child files.
Note that further \textsf{childdoc} directives
such as |\childdocof| and |\childdocforward|
in the indicated file will be processed in this form.
The optional argument \textit{main}
passes on directly to the main file \textit{main}
while pretending to compile the child \textit{dest}.
This form behaves as if \textit{dest}
issues |\childdocof{|\textit{main}|}| right away,
and no further \textsf{childdoc} directives will be processed.

%%%%%%%%%%%%%%%%%%%%%%%%%%%%%%%%%%%%%%%%
\DescribeMacro{\...prefix}
In the alternative form |\childdocforwardprefix|,
%
\begin{center}
\begin{tabular}{l}
|\input{childdoc.def}|\\
|\childdocforwardprefix[|\textit{main}|]{|\textit{prefix}|}{|\textit{dest}|}|
\end{tabular}
\end{center}
%
the destination file is determined by a pattern
depending on the current file:
To make this work, the current file must be called
`{\textit{prefix}\hspace{0.2em}\textit{suffix}}'
with \textit{prefix} matching precisely the argument.
Processing is then passed on to the file
`{\textit{dest}\hspace{0.2em}\textit{suffix}}'.
Surely, the same effect is achieved by
directly specifying the
argument `{\textit{dest}\hspace{0.2em}\textit{suffix}}'
in the first form.
However, that requires to set up a different file
for each child. With the alternative form of the command
all these files can have exactly the same content
which simplifies setting them up and maintaining them.

For example, the following file |draft.tex|
with a compilation flag |\version| as described in \secref{sec:flags}
compiles the main document as a draft:
%
\begin{center}
\begin{tabular}{l}
|\def\version{draft}|\\
|\input{childdoc.def}|\\
|\childdocforward{|\textit{main}|}|
\end{tabular}
\end{center}
%
Likewise, the following files |final|\textit{nn}|.tex|
compile the final version of the child document
|child|\textit{nn}|.tex|:
%
\begin{center}
\begin{tabular}{l}
|\def\version{final}|\\
|\input{childdoc.def}|\\
|\childdocforwardprefix{final}{child}|
\end{tabular}
\end{center}
%

Note that when several versions of a main file and/or of each child file
are to be generated, it may be convenient to set up a |Makefile| or
shell script to automatise the process.

%%%%%%%%%%%%%%%%%%%%%%%%%%%%%%%%%%%%%%%%%%%%%%%%%%%%%%%%%%%%%%%%%%%%%%%%%%%%%%%%
\subsection{Command Line Processing}
\label{sec:commandline}

The effect of redirection files can also be achieved by invoking
the \LaTeX{} compiler with a more elaborate command line.
Most conveniently this should be done as part
of a shell script or a |Makefile|.

When using \textsf{childdoc} in the main file, the following
command lines effectively perform a redirection
(note that depending on the shell being used,
backslashes may have to be doubled: `|\|' $\to$ `|\\|'):
%
\begin{center}
|... -jobname "|\textit{target}|" |\\|"|[\textit{flags}]%
|\input{childdoc.def}\childdocforward[|\textit{main}|]{|\textit{dest}|}"|
\end{center}
%
Here \textit{target} is the name of the output file,
\textit{main} is the name of the main file
and \textit{dest} is the name of the main or child file to be processed
(all filenames without extensions).
The optional argument \textit{main} can be omitted
if \textit{main} matches \textit{dest}.
Optionally, compilation \textit{flags} can be defined via |\def| commands.
This command line makes the \TeX{} engine believe
it is compiling the file \textit{target}
whose content is specified as the latter parameter.
The provided code then forwards the processing to
\textit{main} or \textit{dest} as described in \secref{sec:forward}.

%%%%%%%%%%%%%%%%%%%%%%%%%%%%%%%%%%%%%%%%%%%%%%%%%%%%%%%%%%%%%%%%%%%%%%%%%%%%%%%%
\subsection{Include by Input}
\label{sec:input}

Including child documents by |\include| has some restrictions by design.
Most notably, the content of a child document always occupies
its own set of pages; pages cannot be shared between child documents.
Usually, this behaviour makes perfect sense
because each child document contain an essential part of the document.
However, in some situations it may be desirable to compose
a document from a collection of parts
without having mandatory page breaks between then.
For this case, the package
provides a mechanism to include parts
by |\input| which can also be processed individually.
However, by construction this mechanism
requires manual handling of the content to be output.

%%%%%%%%%%%%%%%%%%%%%%%%%%%%%%%%%%%%%%%%
\DescribeMacro{\ifchilddocmanual}
The main file should be prepared as usual, see \secref{sec:include}.
However, the document body must make a distinction
between processing of an individual part and of the main document, e.g.:
%
\begin{center}
\begin{tabular}{l}
|\ifchilddocmanual|\\
|\input{\childdocname}|\\
|\||else|\\
\textit{document body with }|\input{|\textit{part}|}|\\
|\||fi|
\end{tabular}
\end{center}
%
The conditional |\ifchilddocmanual| is true whenever
a part to be included by |\input| is being compiled,
and the name of the part is stored in |\childdocname|.

%%%%%%%%%%%%%%%%%%%%%%%%%%%%%%%%%%%%%%%%
\DescribeMacro{\childdocby}
Each part to be included by |\input| should start with:
%
\begin{center}
\begin{tabular}{l}
|\input{childdoc.def}|\\
|\childdocby{|\textit{main}|}|\\
\end{tabular}
\end{center}
%
The directive |\childdocby| is similar to |\childdocof|
described in \secref{sec:include},
but the subsequent selection of content must be done manually.
To that end, both |\ifchilddoc| and |\ifchilddocmanual|
will be true upon processing of a part,
and the name of the part is stored in |\childdocname|.
Note that |\jobname| will be set to the filename of the current part
so that each part receives an individual |.aux| file
that does not interfere with the |.aux| file(s) of the main document.
This behaviour can be altered by the alternative form
|\childdocby[*]{|\textit{main}|}| (with a non-empty optional argument)
which uses the |.aux| file of the main document
by setting |\jobname| to \textit{main}.

%%%%%%%%%%%%%%%%%%%%%%%%%%%%%%%%%%%%%%%%%%%%%%%%%%%%%%%%%%%%%%%%%%%%%%%%%%%%%%%%
\subsection{Driver Development}
\label{sec:driver}

The \textsf{childdoc} mechanism can also be use for the development
of definition files such as \LaTeX{} styles or classes.
This case differs from the above setup with multiple parts
included by |\include| in that no |\includeonly| should be invoked.
This can be achieved by starting the include file
(before |\ProvidesPackage|) with:
%
\begin{center}
\begin{tabular}{l}
|\input{childdoc.def}|\\
|\childdocforward{|\textit{main}|}|\\
\end{tabular}
\end{center}
%
or alternatively with:
%
\begin{center}
\begin{tabular}{l}
|\input{childdoc.def}|\\
|\childdocby{|\textit{main}|}|\\
\end{tabular}
\end{center}
%
Both forms have slightly different effects as described above.
The main file is prepared as usual, see \secref{sec:include}.

%%%%%%%%%%%%%%%%%%%%%%%%%%%%%%%%%%%%%%%%%%%%%%%%%%%%%%%%%%%%%%%%%%%%%%%%%%%%%%%%
\subsection{Legacy Detection}
\label{sec:detection}

The directive |\childdocmain| in the main file can detect
whether the complete document or merely a child is to be compiled
even without using the directive |\childdocof|.
This method is deprecated because it is less robust
and there is no compelling reason to use it;
it is merely provided for backward compatibility
and it may be removed in future versions.

If the detection mechanism is to be used,
it is mandatory to correctly specify
the filename of the main file as the argument of |\childdocmain|:
%
\begin{center}
\begin{tabular}{l}
|\input{childdoc.def}|\\
|\childdocmain{|\textit{main}|}|\\
\end{tabular}
\end{center}
%
If |\jobname| does not match the argument \textit{main} of |\childdocmain|,
it is assumed that |\jobname| points to the child file to be compiled.
When using |\childdocmain| with the main file specified as argument,
it suffices to start a child file
with just |\input{|\textit{main}|}|
without loading of the package and using |\childdocof|.
If instead all processing is done
with the appropriate \textsf{childdoc} directives,
the argument of \textit{main} of |\childdocmain| can be empty.

An alternative version of the command line processing described
in \secref{sec:commandline} using the detection mechanism reads:
%
\begin{center}
|... -jobname "|\textit{target}|" "|[\textit{flags}]%
[|\def\jobname{|\textit{dest}|}|]|\input{|\textit{main}|}"|
\end{center}

%%%%%%%%%%%%%%%%%%%%%%%%%%%%%%%%%%%%%%%%%%%%%%%%%%%%%%%%%%%%%%%%%%%%%%%%%%%%%%%%
\subsection{Manual Code}
\label{sec:manual}

In case one cannot be certain whether the definitions file |childdoc.def|
is installed on the target \TeX{} distribution
and one prefers not to ship it,
it is conceivable to paste a few relevant commands into the sources.

To that end, drop all statements |\input{childdoc.def}|
and perform the replacements as outlined below.
Instead of |\childdocmain{|\textit{main}|}| add the following code
to the top of the main file:
%
\begin{center}
\begin{tabular}{l}
|\||ifdefined\childdocname\endinput\||fi\newif\ifchilddoc|\\
|\edef\childdocname{\scantokens\expandafter{\jobname\noexpand}}|\\
|\def\childdocmain{|\textit{main}|}\||ifx\childdocmain\childdocname\||else|\\
|\childdoctrue\includeonly{\childdocname}\let\jobname\childdocmain\||fi|\\
\end{tabular}
\end{center}
%
Instead of |\childdocof{|\textit{main}|}| just include the main file
at the top of each child file:
%
\begin{center}
|\input{|\textit{main}|}|
\end{center}
%
A simple redirection |\childdocforward{|\textit{dest}|}| is achieved by:
%
\begin{center}
|\def\jobname{|\textit{dest}|}\input{\jobname}|
\end{center}
%
The redirection with prefix
|\childdocforwardprefix[|\textit{prefix}|]{|\textit{dest}|}|
is accomplished by:
%
\begin{center}
\begin{tabular}{l}
|{\edef\jobname{\scantokens\expandafter{\jobname\noexpand}}|\\
|\def\redirectjob |\textit{prefix}|#1~~~{\gdef\jobname{|\textit{dest}|#1}}|\\
|\expandafter\redirectjob\jobname~~~}\input{\jobname}|
\end{tabular}
\end{center}

In an alternative approach,
child documents can be compiled by a specific command line
without additional code or specific definitions:
%
\begin{center}
|... -jobname "|\textit{target}|" "|[\textit{flags}]%
|\includeonly{|\textit{dest}|}\input{|\textit{main}|}"|
\end{center}
%

%%%%%%%%%%%%%%%%%%%%%%%%%%%%%%%%%%%%%%%%%%%%%%%%%%%%%%%%%%%%%%%%%%%%%%%%%%%%%%%%
%%%%%%%%%%%%%%%%%%%%%%%%%%%%%%%%%%%%%%%%%%%%%%%%%%%%%%%%%%%%%%%%%%%%%%%%%%%%%%%%
\section{Information}

%%%%%%%%%%%%%%%%%%%%%%%%%%%%%%%%%%%%%%%%%%%%%%%%%%%%%%%%%%%%%%%%%%%%%%%%%%%%%%%%
\subsection{Copyright}

Copyright \copyright{} 2017--2018 Niklas Beisert

This work may be distributed and/or modified under the
conditions of the \LaTeX{} Project Public License, either version 1.3
of this license or (at your option) any later version.
The latest version of this license is in
  \url{http://www.latex-project.org/lppl.txt}
and version 1.3 or later is part of all distributions of \LaTeX{}
version 2005/12/01 or later.

This work has the LPPL maintenance status `maintained'.

The Current Maintainer of this work is Niklas Beisert.

This work consists of the files |README.txt|, |childdoc.ins| and |childdoc.dtx|
as well as the derived files |childdoc.def|, |cdocsamp.tex|
with |cdocsch1.tex|, |cdocsch2.tex|, |cdocspt3.tex|, |cdocspt4.tex|,
|cdocsdrf.tex|, |cdocsfn1.tex|, |cdocsfn2.tex|
as well as |childdoc.pdf|.

%%%%%%%%%%%%%%%%%%%%%%%%%%%%%%%%%%%%%%%%%%%%%%%%%%%%%%%%%%%%%%%%%%%%%%%%%%%%%%%%
\subsection{Files and Installation}

The package consists of the files:
%
\begin{center}
\begin{tabular}{ll}
    |README.txt|   & readme file \\
    |childdoc.ins| & installation file \\
    |childdoc.dtx| & source file \\
    |childdoc.def| & definition file \\
    |cdocsamp.tex| & sample main file \\
    |cdocsch1.tex| & sample include file \\
    |cdocsch2.tex| & sample include file \\
    |cdocspt3.tex| & sample part file \\
    |cdocspt4.tex| & sample part file \\
    |cdocsdrf.tex| & sample redirection file \\
    |cdocsfn1.tex| & sample redirection file \\
    |cdocsfn2.tex| & sample redirection file \\
    |childdoc.pdf| & manual
\end{tabular}
\end{center}
%
The distribution consists of the files
|README.txt|, |childdoc.ins| and |childdoc.dtx|.
%
\begin{itemize}
\item
Run (pdf)\LaTeX{} on |childdoc.dtx|
to compile the manual |childdoc.pdf| (this file).
\item
Run \LaTeX{} on |childdoc.ins| to create the definitions file |childdoc.def|
and the sample |cdocsamp.tex| with include files
|cdocsch1.tex|, |cdocsch2.tex|, |cdocspt3.tex|, |cdocspt4.tex|,
|cdocsdrf.tex|, |cdocsfn1.tex|, |cdocsfn2.tex|.
Then copy the file |childdoc.def| to an appropriate directory of your \LaTeX{}
distribution, e.g.\ \textit{texmf-root}|/tex/latex/childdoc|.
\end{itemize}

%%%%%%%%%%%%%%%%%%%%%%%%%%%%%%%%%%%%%%%%%%%%%%%%%%%%%%%%%%%%%%%%%%%%%%%%%%%%%%%%
\subsection{Related CTAN Packages}

There are several other packages which offer a similar functionality:
%
\begin{itemize}
\item
The packages
\href{http://ctan.org/pkg/docmute}{\textsf{docmute}},
\href{http://ctan.org/pkg/includex}{\textsf{includex}} and
\href{http://ctan.org/pkg/standalone}{\textsf{standalone}}
provide commands to include only the document body of
a child file thus allowing both files to be compiled individually.
\item
The packages \href{http://ctan.org/pkg/subdocs}{\textsf{subdocs}}
and \href{http://ctan.org/pkg/subfiles}{\textsf{subfiles}}
provide structures in which the main and child documents can be
encapsulated and allowing them to be compiled individually.
The inclusion mechanism is different from the conventional |\include|.
\item
The package \href{http://ctan.org/pkg/combine}{\textsf{combine}}
is an elaborate solution to combine several documents into one.
\end{itemize}
%
See also the CTAN topic \href{http://ctan.org/topic/subdocs}{\textsf{subdocs}}
for further related packages.
The present package differs from the above solutions in that
a document structure constructed with the conventional |\include| mechanism
just needs two extra commands at the top of every file
such that all constituent files can be compiled individually.

%%%%%%%%%%%%%%%%%%%%%%%%%%%%%%%%%%%%%%%%%%%%%%%%%%%%%%%%%%%%%%%%%%%%%%%%%%%%%%%%
%\subsection{Feature Suggestions}
%
%The following is a list of features which may be useful for future
%versions of this package:
%%
%\begin{itemize}
%\item
%\ldots
%\end{itemize}

%%%%%%%%%%%%%%%%%%%%%%%%%%%%%%%%%%%%%%%%%%%%%%%%%%%%%%%%%%%%%%%%%%%%%%%%%%%%%%%%
\subsection{Revision History}

%%%%%%%%%%%%%%%%%%%%%%%%%%%%%%%%%%%%%%%%
\paragraph{v2.0:} 2018/12/30

\begin{itemize}
\item
immediate forward processing
\item
added |\childdocby| mechanism
\item
manual restructured
\end{itemize}

%%%%%%%%%%%%%%%%%%%%%%%%%%%%%%%%%%%%%%%%
\paragraph{v1.6:} 2018/01/17

\begin{itemize}
\item
application for development of include files
\item
corrections to manual
\end{itemize}

%%%%%%%%%%%%%%%%%%%%%%%%%%%%%%%%%%%%%%%%
\paragraph{v1.5:} 2017/05/21

\begin{itemize}
\item
more complete structuring introduced
\item
|\childdocof| introduced
\item
|\childdoc| renamed to |\childdocmain|
\item
|\childredirect| renamed to |\childdocforward| and |\childdocforwardprefix|
and functionality expanded
\end{itemize}

%%%%%%%%%%%%%%%%%%%%%%%%%%%%%%%%%%%%%%%%
\paragraph{v1.0:} 2017/04/27

\begin{itemize}
\item
manual and install package
\item
first version published on CTAN
\end{itemize}

%%%%%%%%%%%%%%%%%%%%%%%%%%%%%%%%%%%%%%%%
\paragraph{v0.6:} 2017/04/26

\begin{itemize}
\item
redirection mechanism added
\end{itemize}

%%%%%%%%%%%%%%%%%%%%%%%%%%%%%%%%%%%%%%%%
\paragraph{v0.5:} 2017/04/26

\begin{itemize}
\item
functionality in definition file
\end{itemize}


%%%%%%%%%%%%%%%%%%%%%%%%%%%%%%%%%%%%%%%%%%%%%%%%%%%%%%%%%%%%%%%%%%%%%%%%%%%%%%%%
%%%%%%%%%%%%%%%%%%%%%%%%%%%%%%%%%%%%%%%%%%%%%%%%%%%%%%%%%%%%%%%%%%%%%%%%%%%%%%%%
%%%%%%%%%%%%%%%%%%%%%%%%%%%%%%%%%%%%%%%%%%%%%%%%%%%%%%%%%%%%%%%%%%%%%%%%%%%%%%%%
\appendix

\settowidth\MacroIndent{\rmfamily\scriptsize 000\ }

 \DocInput{childdoc.dtx}

\end{document}
%</driver>
% \fi
%
% %%%%%%%%%%%%%%%%%%%%%%%%%%%%%%%%%%%%%%%%%%%%%%%%%%%%%%%%%%%%%%%%%%%%%%%%%%%%%%
% %%%%%%%%%%%%%%%%%%%%%%%%%%%%%%%%%%%%%%%%%%%%%%%%%%%%%%%%%%%%%%%%%%%%%%%%%%%%%%
% \section{Sample}
%\iffalse
%<*samplemain>
%\fi
%
% The following presents a sample document
% with two chapters, two parts, a title page,
% a compile flag as well as three forwarding files to set the flag.
% It consists of eight |.tex| files:
% \begin{center}
% \begin{tabular}{ll}
% |cdocsamp.tex|&main file\\
% |cdocsch1.tex|&include file for chapter 1\\
% |cdocsch2.tex|&include file for chapter 2\\
% |cdocspt3.tex|&include file for part 3\\
% |cdocspt4.tex|&include file for part 4\\
% |cdocsdrf.tex|&forwarding file for main file in draft mode\\
% |cdocsfi1.tex|&forwarding file for final version of chapter 1\\
% |cdocsfi2.tex|&forwarding file for final version of chapter 2\\
% \end{tabular}
% \end{center}
% Each of the eight files can be compiled directly by the \LaTeX{} compiler.
%
% %%%%%%%%%%%%%%%%%%%%%%%%%%%%%%%%%%%%%%
% \paragraph{Main File.}
%
% The main file is called |cdocsamp.tex|.
%
% Load the \textsf{childdoc} definitions and
% declare the filename for the main document:
%    \begin{macrocode}
\input{childdoc.def}
\childdocmain{}
%    \end{macrocode}

% Optional override for |\version| flag:
%    \begin{macrocode}
%%\ifchilddoc\else\providecommand{\version}{draft}\fi
%    \end{macrocode}

% Define the default values for the |\version| flag
% (|final| for the main file and |draft| for childs):
%    \begin{macrocode}
\ifchilddoc
\providecommand{\version}{draft}
\else
\providecommand{\version}{final}
\fi
%    \end{macrocode}

% Load the standard document class:
%    \begin{macrocode}
\documentclass[12pt]{article}
%    \end{macrocode}

% Start the document body:
%    \begin{macrocode}
\begin{document}
%    \end{macrocode}

% Declare a title page.
% Print title, part of document being processed and version flag:
%    \begin{macrocode}
\addtocounter{page}{-1}
\begin{center}
{\LARGE\bfseries{}childdoc example\par}
\vspace{1cm}
\ifchilddoc
\ifchilddocmanual part\else chapter\fi:
`\childdocname' of `\childdocjob'\par
\else
main document: `\childdocjob'\par
\fi
version: \version\par
\end{center}
\newpage
%    \end{macrocode}

% Manually include selected file,
% otherwise process as usual:
%    \begin{macrocode}
\ifchilddocmanual
\section*{part `\childdocname'}
\input{\childdocname}
\else
%    \end{macrocode}

% Include the two chapters:
%    \begin{macrocode}
\include{cdocsch1}
\include{cdocsch2}
%    \end{macrocode}

% Include the two parts unless only chapters should be displayed:
%    \begin{macrocode}
\ifchilddoc\else
\section{part three}
\input{cdocspt3}
\section{part four}
\input{cdocspt4}
\fi
%    \end{macrocode}

% Process as usual until here:
%    \begin{macrocode}
\fi
%    \end{macrocode}

% End of document body:
%    \begin{macrocode}
\end{document}
%    \end{macrocode}
%\iffalse
%</samplemain>
%\fi
%
% %%%%%%%%%%%%%%%%%%%%%%%%%%%%%%%%%%%%%%
% \paragraph{Chapter Include Files.}
%
% The include files are called |cdocsch1.tex| and |cdocsch2.tex|.
%
%\iffalse
%<*samplechap1|samplechap2>
%\fi

% Optional override for |\version| flag:
%    \begin{macrocode}
%%\providecommand{\version}{final}
%    \end{macrocode}

% Include the main document:
%    \begin{macrocode}
\input{childdoc.def}
\childdocof{cdocsamp}
%    \end{macrocode}

%\iffalse
%</samplechap1|samplechap2>
%\fi
%
%\iffalse
%<*samplechap1>
%\fi
% Some text for chapter 1:
%    \begin{macrocode}
\section{one}
some text in chapter one
%    \end{macrocode}

%\iffalse
%</samplechap1>
%\fi
% Some text for chapter 2:
%\iffalse
%<*samplechap2>
%\fi
%    \begin{macrocode}
\section{two}
more text in chapter two
%    \end{macrocode}

%\iffalse
%</samplechap2>
%\fi
%
% %%%%%%%%%%%%%%%%%%%%%%%%%%%%%%%%%%%%%%
% \paragraph{Part Include Files.}
%
% The include files are called |cdocspt3.tex| and |cdocspt4.tex|.
%
%\iffalse
%<*samplepart3|samplepart4>
%\fi

% Optional override for |\version| flag:
%    \begin{macrocode}
%%\providecommand{\version}{final}
%    \end{macrocode}

% Include the main document:
%    \begin{macrocode}
\input{childdoc.def}
\childdocby{cdocsamp}
%    \end{macrocode}

%\iffalse
%</samplepart3|samplepart4>
%\fi
%
%\iffalse
%<*samplepart3>
%\fi
% Some text for part 3:
%    \begin{macrocode}
some text in part three
%    \end{macrocode}

%\iffalse
%</samplepart3>
%\fi
% Some text for part 4:
%\iffalse
%<*samplepart4>
%\fi
%    \begin{macrocode}
more text in part four
%    \end{macrocode}

%\iffalse
%</samplepart4>
%\fi
%
% %%%%%%%%%%%%%%%%%%%%%%%%%%%%%%%%%%%%%%
% \paragraph{Forwarding for a Complete Draft.}
%
% The following forwarding file |cdocsdrf.tex|
% compiles the main document in draft mode:
%\iffalse
%<*sampledraft>
%\fi
%    \begin{macrocode}
\def\version{draft}
\input{childdoc.def}
\childdocforward{cdocsamp}
%    \end{macrocode}

%\iffalse
%</sampledraft>
%\fi
%
% %%%%%%%%%%%%%%%%%%%%%%%%%%%%%%%%%%%%%%
% \paragraph{Forwarding for Final Version of the Chapters.}
%
% The following forwarding files |cdocsfn1.tex| and |cdocsfn2.tex|
% (with identical content)
% compile the final versions of the child documents
% |cdocsch1.tex| and |cdocsch2.tex|, respectively:
%\iffalse
%<*samplefinal>
%\fi
%    \begin{macrocode}
\def\version{final}
\input{childdoc.def}
\childdocforwardprefix[cdocsamp]{cdocsfn}{cdocsch}
%    \end{macrocode}

%\iffalse
%</samplefinal>
%\fi
%
% %%%%%%%%%%%%%%%%%%%%%%%%%%%%%%%%%%%%%%
% \paragraph{Command Line Processing.}
%
% The following three command lines generate the output files
% |cdocscld|, |cdocscl1| and |cdocscl2|
% which should be identical to
% |cdocsdrf|, |cdocsch1| and |cdocsfn2|, respectively:
% \begin{center}
% \begin{tabular}{l}
% |latex -jobname cdocscld \|\\
% |  "\def\version{draft}\input{childdoc.def}\childdocforward{cdocsamp}"|\\
% |latex -jobname cdocscl1 \|\\
% |  "\input{childdoc.def}\childdocforward[cdocsamp]{cdocsch1}"|\\
% |latex -jobname cdocscl2 \|\\
% |  "\def\version{final}\input{childdoc.def}\childdocforward{cdocsch2}"|
% \end{tabular}
% \end{center}
% Note that the trailing backslash on each first line
% merely continues the input to the second line
% (for convenient cut ant paste).
% Furthermore, the command |latex| can be replaced by any
% of its alternative versions such as |pdflatex|.
%
% %%%%%%%%%%%%%%%%%%%%%%%%%%%%%%%%%%%%%%%%%%%%%%%%%%%%%%%%%%%%%%%%%%%%%%%%%%%%%%
% %%%%%%%%%%%%%%%%%%%%%%%%%%%%%%%%%%%%%%%%%%%%%%%%%%%%%%%%%%%%%%%%%%%%%%%%%%%%%%
% \section{Implementation}
%\iffalse
%<*package>
%\fi
%
% This section describes the definitions file |childdoc.def|.

% The definitions cannot be loaded using |\usepackage| or |\RequirePackage|
% which has a mechanism to prevent loading a style file more than once.
% When loading the definitions by means of |\input|
% multiple instances have to be prevented manually:
%\iffalse
%This code needs to be before the `\ProvidesFile' directive
%which is defined at the beginning of this file.
%Therefore it is also placed there and commented out here.
%</package>
%<*discard>
%\fi
%    \begin{macrocode}
\ifdefined\childdocmain\endinput\fi
%    \end{macrocode}
%\iffalse
%</discard>
%<*package>
%\fi
%
% \macro{\ifchilddoc}
% \macro{\ifchilddocmanual}
% The conditional |\ifchilddoc| tells whether a
% child (true) or main (false) document is being compiled.
% The conditional |\ifchilddocmanual| tells whether
% the |\includeonly| mechanism is used (false) or
% the selection of child files must be performed manually (true).
% The definitions initialise to false:
%    \begin{macrocode}
\newif\ifchilddoc
\newif\ifchilddocmanual
%    \end{macrocode}

% \macro{\childdocname}
% \macro{\childdocjob}
% The macro |\childdocname| stores the name of the main document
% to be compiled. The macro |\childdocjob| stores the name of
% the document on which the \LaTeX{} compiler was originally invoked.
% The content of |\jobname| cannot be compared
% to filenames specified in the source due to different catcodes.
% The following code rescans |\jobname|, stores the result
% in |\childdocname| and saves a copy in |\childdocjob|:
%    \begin{macrocode}
\edef\childdocname{\scantokens\expandafter{\jobname\noexpand}}
\let\childdocjob\childdocname
%    \end{macrocode}

% \macro{\childdocdisable}
% The macro |\childdocdisable| prevents the main file
% from being processed more than once.
% At this stage, the main document command |\childdocmain|
% is assumed to be called once again where it should do nothing.
% Any subsequent call to it should prevent
% a secondary processing of the main document
% It overwrites the forwarding commands
% |\childdocof| and |\childdocforward|
% with empty macros to prevent further inclusions of the main document:
%    \begin{macrocode}
\newcommand{\childdocdisable}
{
  \renewcommand{\childdocmain}[1]{\renewcommand{\childdocmain}[1]{\endinput}}
  \renewcommand{\childdocof}[1]{}
  \renewcommand{\childdocby}[2][]{}
  \renewcommand{\childdocforward}[2][]{}
  \renewcommand{\childdocdisable}{}
}
%    \end{macrocode}

% \macro{\childdocmain}
% The macro |\childdocmain| is to be called at the top of the main file
% with nothing or the main filename (without extension) as argument.
% First, it breaks loops.
% If the argument is not empty and does not match |\childdocname|
% (which is set by the first inclusion of |childdoc.def|),
% |\ifchilddoc| is set to true, |\includeonly| is applied to the child file
% and |\jobname| is set to the main file
% (for proper handling of |.aux| files):
%    \begin{macrocode}
\newcommand{\childdocmain}[1]
{
  \childdocdisable\childdocmain{}
  \if?#1?\else
    \begingroup
      \def\childdoctmp{#1}
      \ifx\childdoctmp\childdocname
        \def\childdoctmp{}
      \else
        \def\childdoctmp
        {
          \childdoctrue
          \includeonly{\childdocname}
          \def\childdocjob{#1}
          \def\jobname{#1}
        }
      \fi
      \expandafter
    \endgroup
    \childdoctmp
  \fi
}
%    \end{macrocode}

% \macro{\childdocof}
% The command |\childdocof| redirects
% compilation to the main file |#1|.
%    \begin{macrocode}
\newcommand{\childdocof}[1]
{
  \childdocdisable
  \childdoctrue
  \includeonly{\childdocname}
  \def\jobname{#1}
  \def\childdocjob{#1}
  \input{#1}
}
%    \end{macrocode}

% \macro{\childdocby}
% The command |\childdocby| ....
%    \begin{macrocode}
\newcommand{\childdocby}[2][]
{
  \childdocdisable
  \childdoctrue
  \childdocmanualtrue
  \if?#1?\else
    \def\jobname{#2}
  \fi
  \def\childdocjob{#2}
  \input{#2}
  \endinput
}
%    \end{macrocode}

% \macro{\childdocforward}
% The command |\childdocforward| redirects
% compilation to the main file or
% (if the optional argument is given) a child file.
% Parameters are set as if the main file
% or a child file starting with |\childdocof| was compiled.
% Then compilation is handed over to the main file:
%    \begin{macrocode}
\newcommand{\childdocforward}[2][]
{
  \begingroup
    \if?#1?
      \def\childdoctmp
      {
        \def\childdocname{#2}
        \def\childdocjob{#2}
        \def\jobname{#2}
        \input{#2}
        \endinput
      }
    \else
      \def\childdoctmp
      {
        \childdocdisable
        \def\childdocname{#2}
        \childdoctrue
        \includeonly{#2}
        \def\childdocjob{#1}
        \def\jobname{#1}
        \input{#1}
        \endinput
      }
    \fi
    \expandafter
  \endgroup
  \childdoctmp
}
%    \end{macrocode}

% \macro{\childdocforwardprefix}
% The command |\childdocforwardprefix| redirects
% compilation to the main or a child file by means of a pattern.
% The prefix |#1| in the current filename is replaced by |#2|
% and the suffix of the current filename is kept
% (it is assumed that the filename does not contain the substring `|~~~|'
% which is used as a delimiter).
% Compilation is handed over to the new file by |\childdocforward|:
%    \begin{macrocode}
\newcommand{\childdocforwardprefix}[3][]
{
  \begingroup
    \def\childdocextract #2##1~~~{\def\childdoctmp{\childdocforward[#1]{#3##1}}}
    \expandafter\childdocextract\childdocname~~~
    \expandafter
  \endgroup
  \childdoctmp
}
%    \end{macrocode}

% \macro{\childdoc}
% The deprecated macro |\childdoc| is a legacy version of |\childdocmain|:
%    \begin{macrocode}
\newcommand{\childdoc}{\childdocmain}
%    \end{macrocode}

% \macro{\childdocredirect}
% The deprecated macro |\childdocredirect| is a legacy version
% of |\childdocforward| and |\childdocforwardprefix|:
%    \begin{macrocode}
\newcommand{\childdocredirect}[2][]
{
  \begingroup
    \if?#1?
      \def\childdoctmp{\childdocforward{#2}}
    \else
      \def\childdoctmp{\childdocforwardprefix{#1}{#2}}
    \fi
    \expandafter
  \endgroup
  \childdoctmp
}
%    \end{macrocode}

%\iffalse
%</package>
%\fi
%
\endinput
\childdocforward[|\textit{main}|]{|\textit{dest}|}"|
\end{center}
%
Here \textit{target} is the name of the output file,
\textit{main} is the name of the main file
and \textit{dest} is the name of the main or child file to be processed
(all filenames without extensions).
The optional argument \textit{main} can be omitted
if \textit{main} matches \textit{dest}.
Optionally, compilation \textit{flags} can be defined via |\def| commands.
This command line makes the \TeX{} engine believe
it is compiling the file \textit{target}
whose content is specified as the latter parameter.
The provided code then forwards the processing to
\textit{main} or \textit{dest} as described in \secref{sec:forward}.

%%%%%%%%%%%%%%%%%%%%%%%%%%%%%%%%%%%%%%%%%%%%%%%%%%%%%%%%%%%%%%%%%%%%%%%%%%%%%%%%
\subsection{Include by Input}
\label{sec:input}

Including child documents by |\include| has some restrictions by design.
Most notably, the content of a child document always occupies
its own set of pages; pages cannot be shared between child documents.
Usually, this behaviour makes perfect sense
because each child document contain an essential part of the document.
However, in some situations it may be desirable to compose
a document from a collection of parts
without having mandatory page breaks between then.
For this case, the package
provides a mechanism to include parts
by |\input| which can also be processed individually.
However, by construction this mechanism
requires manual handling of the content to be output.

%%%%%%%%%%%%%%%%%%%%%%%%%%%%%%%%%%%%%%%%
\DescribeMacro{\ifchilddocmanual}
The main file should be prepared as usual, see \secref{sec:include}.
However, the document body must make a distinction
between processing of an individual part and of the main document, e.g.:
%
\begin{center}
\begin{tabular}{l}
|\ifchilddocmanual|\\
|\input{\childdocname}|\\
|\||else|\\
\textit{document body with }|\input{|\textit{part}|}|\\
|\||fi|
\end{tabular}
\end{center}
%
The conditional |\ifchilddocmanual| is true whenever
a part to be included by |\input| is being compiled,
and the name of the part is stored in |\childdocname|.

%%%%%%%%%%%%%%%%%%%%%%%%%%%%%%%%%%%%%%%%
\DescribeMacro{\childdocby}
Each part to be included by |\input| should start with:
%
\begin{center}
\begin{tabular}{l}
|% \iffalse
%
% childdoc.dtx Copyright (C) 2017-2018 Niklas Beisert
%
% This work may be distributed and/or modified under the
% conditions of the LaTeX Project Public License, either version 1.3
% of this license or (at your option) any later version.
% The latest version of this license is in
%   http://www.latex-project.org/lppl.txt
% and version 1.3 or later is part of all distributions of LaTeX
% version 2005/12/01 or later.
%
% This work has the LPPL maintenance status `maintained'.
%
% The Current Maintainer of this work is Niklas Beisert.
%
% This work consists of the files childdoc.dtx and childdoc.ins
% and the derived files childdoc.def and cdocsamp.tex with
% cdocsch1.tex, cdocsch2.tex, cdocsdrf.tex, cdocsfn1.tex, cdocsfn2.tex.
%
%<package>\ifdefined\childdocmain\endinput\fi
%<package>\ProvidesFile{childdoc.def}[2018/12/30 v2.0 child document driver]
%<samplemain>\ProvidesFile{cdocsamp.tex}[2018/12/30 v2.0 sample for childdoc]
%<*driver>
%\ProvidesFile{childdoc.drv}[2018/12/30 v2.0 childdoc reference manual file]
\PassOptionsToClass{10pt,a4paper}{article}
\documentclass{ltxdoc}

\usepackage[margin=35mm]{geometry}
\usepackage{hyperref}
\usepackage{hyperxmp}
\usepackage[usenames]{color}

\hypersetup{colorlinks=true}
\hypersetup{pdfstartview=FitH}
\hypersetup{pdfpagemode=UseNone}
\hypersetup{pdfsource={}}
\hypersetup{pdflang={en-UK}}
\hypersetup{pdfcopyright={Copyright 2017-2018 Niklas Beisert.
  This work may be distributed and/or modified under the
  conditions of the LaTeX Project Public License, either version 1.3
  of this license or (at your option) any later version.}}
\hypersetup{pdflicenseurl={http://www.latex-project.org/lppl.txt}}
\hypersetup{pdfcontactaddress={ETH Zurich, ITP, HIT K,
  Wolfgang-Pauli-Strasse 27}}
\hypersetup{pdfcontactpostcode={8093}}
\hypersetup{pdfcontactcity={Zurich}}
\hypersetup{pdfcontactcountry={Switzerland}}
\hypersetup{pdfcontactemail={nbeisert@itp.phys.ethz.ch}}
\hypersetup{pdfcontacturl={http://people.phys.ethz.ch/\xmptilde nbeisert/}}

\newcommand{\secref}[1]{\hyperref[#1]{section \ref*{#1}}}

\parskip1ex
\parindent0pt
\let\olditemize\itemize
\def\itemize{\olditemize\parskip0pt}

\begin{document}

\title{The \textsf{childdoc} Package}
\hypersetup{pdftitle={The childdoc Package}}
\author{Niklas Beisert\\[2ex]
  Institut f\"ur Theoretische Physik\\
  Eidgen\"ossische Technische Hochschule Z\"urich\\
  Wolfgang-Pauli-Strasse 27, 8093 Z\"urich, Switzerland\\[1ex]
  \href{mailto:nbeisert@itp.phys.ethz.ch}
  {\texttt{nbeisert@itp.phys.ethz.ch}}}
\hypersetup{pdfauthor={Niklas Beisert}}
\hypersetup{pdfsubject={Manual for the LaTeX2e Package childdoc}}
\date{30 December 2018, \textsf{v2.0}}
\maketitle

\begin{abstract}\noindent
\textsf{childdoc} is a \LaTeXe{} package
that enables the direct compilation
of document sections included by |\include|
to individual files.
\end{abstract}

\begingroup
\parskip0ex
\tableofcontents
\endgroup

%%%%%%%%%%%%%%%%%%%%%%%%%%%%%%%%%%%%%%%%%%%%%%%%%%%%%%%%%%%%%%%%%%%%%%%%%%%%%%%%
%%%%%%%%%%%%%%%%%%%%%%%%%%%%%%%%%%%%%%%%%%%%%%%%%%%%%%%%%%%%%%%%%%%%%%%%%%%%%%%%
\section{Introduction}

\LaTeX{} provides a mechanism to structure a large document (such as a book)
into a main file and several child files (containing the chapters)
using the |\include| command.
This mechanism is beneficial for documents
which span hundreds of pages in order to
make the source file(s) more manageable.
Moreover, compilation can be restricted to
selected child files by means of the |\includeonly| command.
The latter feature can be used to reduce the compilation time while editing
(this was significantly more useful in the earlier days of \LaTeX{})
or to generate a smaller document which is easier to navigate.
Another application of |\includeonly| is to generate
documents consisting of selected parts of the complete document.

However, there are a few drawbacks of the plain |\include| mechanism:
\begin{itemize}
\item
The child files cannot be compiled on their own,
they can only be compiled via the main file.
A naive editing environment
(such as a text editor with an option
to have the current file processed by \LaTeX)
may require one to switch to the main file before compiling;
attempting to compile the child file produces errors.
\item
The main file must be modified (each time)
to adjust the |\includeonly| command
to the present needs. This easily leaves the main file in a messy state.
\item
The generated document will always carry the filename
of the main document. This is inconvenient if
several child files are to be compiled and
to be kept for distribution.
\end{itemize}

The present package provides a simple interface
to make child files individually compilable by \LaTeX{}.
Compiling a child file then has the same effect as compiling
the main file with an |\includeonly| command
to select the appropriate child.
Moreover the generated document will carry the name of the child
rather than the main file.
This resolves all three above issues.

This feature is meant to make the editing of books,
thesis documents and lecture notes somewhat more convenient.
However, the package can also be used efficiently for
composing a series of documents (such as exercise sheets)
which are typically distributed individually.
It then assists the author in generating the individual documents
(potentially in different versions)
as well as a document containing the collected series.
Another application is in developing style files
or other kinds of included material
where compilation of the style file could redirect
to a sample or test file.

%%%%%%%%%%%%%%%%%%%%%%%%%%%%%%%%%%%%%%%%%%%%%%%%%%%%%%%%%%%%%%%%%%%%%%%%%%%%%%%%
%%%%%%%%%%%%%%%%%%%%%%%%%%%%%%%%%%%%%%%%%%%%%%%%%%%%%%%%%%%%%%%%%%%%%%%%%%%%%%%%
\section{Usage}

First of all, the package \textsf{childdoc} is \emph{not} a standard
\LaTeXe{} |.sty| style file! Therefore it needs to be invoked in
a non-standard way.

%%%%%%%%%%%%%%%%%%%%%%%%%%%%%%%%%%%%%%%%%%%%%%%%%%%%%%%%%%%%%%%%%%%%%%%%%%%%%%%%
\subsection{Included Files}
\label{sec:include}

%%%%%%%%%%%%%%%%%%%%%%%%%%%%%%%%%%%%%%%%
\DescribeMacro{\childdocmain}
To use the package, add the commands
\begin{center}
\begin{tabular}{l}
|\input{childdoc.def}|\\
|\childdocmain{}|\\
\end{tabular}
\end{center}
at the very top of the main \LaTeX{} file,
in particular \emph{before} the |\documentclass| statement!
The argument of |\childdocmain| should be left empty
(but it must be present).

%%%%%%%%%%%%%%%%%%%%%%%%%%%%%%%%%%%%%%%%
\DescribeMacro{\childdocof}
Furthermore, add the commands
\begin{center}
\begin{tabular}{l}
|\input{childdoc.def}|\\
|\childdocof{|\textit{main}|}|\\
\end{tabular}
\end{center}
at the top of every child file \textit{child}
which is included by |\include{|\textit{child}|}|
from within the main file
(or at least for those files to be compiled individually).
The argument \textit{main} must be the filename of the main file.

There are a couple of
considerations in setting up the main and child documents:

%%%%%%%%%%%%%%%%%%%%%%%%%%%%%%%%%%%%%%%%
\paragraph{Restrictions.}

Please note the following restrictions:
\begin{itemize}
\item
|\childdocmain| must be called with one argument \textit{main}
to ensure compatibility with earlier version of the package.
It must either be empty (|\childdocmain{}|)
or precisely match the filename of the main file in which it is specified.
See \secref{sec:detection} for further information.
\item
The filename \textit{main} must be specified without the |.tex| extension.
\item
The filename \textit{main} is case sensitive
(even in case-insensitive file systems)
due to internal string comparison.
\item
The argument \textit{main} should be fully expanded, it cannot be a macro.
\item
Subdirectories and special characters should be avoided in filenames.
\item
The command |\childdocmain{|\textit{main}|}| must be followed by a whitespace.
It should not be followed immediately by another command
or by a comment mark `|%|'.
This is because the \TeX{} parser reads the token immediately following
the argument of |\childdocmain| and puts it
at the beginning of every child section;
however, a white\-space is ignored.
\end{itemize}

%%%%%%%%%%%%%%%%%%%%%%%%%%%%%%%%%%%%%%%%
\paragraph{Content of Main File.}

It is advisable to place all content in the child files included by |\include|.
Any output contained in the main file will appear in all child documents
unless suppressed manually;
it cannot be suppressed automatically by the |\includeonly| directive
and thus should normally be avoided.
A method to include some content in the main file
by means of conditional processing is described in \secref{sec:conditional}.

%%%%%%%%%%%%%%%%%%%%%%%%%%%%%%%%%%%%%%%%
\paragraph{Page Numbering.}

When only a part of the document is compiled,
the appropriate numbering of pages
(as well as other status parameters)
is determined from the |.aux| files.
The latter contain information from previous passes.
However this information needs to propagate through
all intermediate child documents.
Therefore the page numbering in child documents may well
be inconsistent until the complete document is compiled at least once.

A useful (if unconventional) way to always ensure a consistent
page numbering is to restart the numbering in each child document
and denote the pages by `\textit{child}|.|\textit{page}'
where \textit{child} represents the chapter/section number of the child file.
This can be achieved by the command
|\numberwithin{page}{|\textit{child}|}|
of the \textsf{amsmath} package
where \textit{child} can be |chapter| or |section|
depending on the chosen structuring.
Alternatively, one can modify the macro |\thepage| appropriately
and reset the counter |page| at the start of each child file.

%%%%%%%%%%%%%%%%%%%%%%%%%%%%%%%%%%%%%%%%%%%%%%%%%%%%%%%%%%%%%%%%%%%%%%%%%%%%%%%%
\subsection{Conditional Processing}
\label{sec:conditional}

The package provides a mechanism to compile different versions
of a document. To customise the versions further some conditional processing
can come in handy to distinguish which version is being compiled.
The package provides two macros to describe the compilation context:

%%%%%%%%%%%%%%%%%%%%%%%%%%%%%%%%%%%%%%%%
\DescribeMacro{\ifchilddoc}
The conditional |\ifchilddoc| distinguishes between the compilation of
child documents and the main document:
%
\begin{center}
|\ifchilddoc |\textit{child-code}| |[|\||else |\textit{main-code}]| \||fi|
\end{center}

%%%%%%%%%%%%%%%%%%%%%%%%%%%%%%%%%%%%%%%%
\DescribeMacro{\childdocname}
\DescribeMacro{\childdocjob}
The macro |\childdocname| contains the filename (without extension)
of the main or child file being processed.
Note that |\childdocjob| will always contain the name of the main file.

%%%%%%%%%%%%%%%%%%%%%%%%%%%%%%%%%%%%%%%%
\paragraph{Title Page.}

Conditional processing can be used to include a title or banner page
in the main document when proper precautions are taken.
Importantly, the code in the main file should ensure that the page counter
(as well as other status parameters which are stored in the |.aux| files)
takes the same value after the conditional processing.
Otherwise the page numbers may take divergent values
depending on which part is compiled.

For example, a title page could be declared by:
%
\begin{center}
\begin{tabular}{l}
|\ifchilddoc\||else|\\
|\addtocounter{page}{-1}|\\
\textit{code for title page}\\
|\newpage|\\
|\||fi|
\end{tabular}
\end{center}
%
A banner page for the child documents can be generated by:
%
\begin{center}
\begin{tabular}{l}
|\ifchilddoc|\\
|\addtocounter{page}{-1}|\\
\textit{code for banner page}\\
|\newpage|\\
|\||fi|
\end{tabular}
\end{center}
%
Here one could write a message such as:
\begin{center}
|This is the part \childdocname{} of \childdocjob{}.|
\end{center}

%%%%%%%%%%%%%%%%%%%%%%%%%%%%%%%%%%%%%%%%%%%%%%%%%%%%%%%%%%%%%%%%%%%%%%%%%%%%%%%%
\subsection{Flags}
\label{sec:flags}

The package makes it easy to generate different versions
of the main or child documents.
To this end compilation flags can be defined
and assigned different default values.
They will be particularly useful in conjunction
with the forwarding mechanism described in \secref{sec:forward}.

For example, it may be useful to have a flag |\version|
which can be set to |draft| or |final|.
The document source will contain some conditional code
depending on the value of |\version|.
Suppose further, the flag should default to |final| for the main file
and to |draft| for child files
which is a natural assignment for editing the document.
This is achieved by placing the following code
in the preamble of the main document
(below the |\childdocmain| directive):
%
\begin{center}
\begin{tabular}{l}
|\ifchilddoc|\\
|\providecommand{\version}{draft}|\\
|\||else|\\
|\providecommand{\version}{final}|\\
|\||fi|
\end{tabular}
\end{center}
%
The definition by |\providecommand| makes sure
that previous definitions are not overwritten.
Further statements |\providecommand{\version}{...}|
can thus be added before the above code to override it.

For the main file, one might add a line
(between |\childdocmain| and the above block)
%
\begin{center}
|%\ifchilddoc\||else\providecommand{\version}{draft}\||fi|
\end{center}
%
which can be uncommented to produce a draft version.
Likewise one can add a line to the very top of a child file
(above the |\childdocof{|\textit{main}|}| directive)
%
\begin{center}
|%\providecommand{\version}{final}|
\end{center}
%
which can be uncommented to produce the final version of this child document.

%%%%%%%%%%%%%%%%%%%%%%%%%%%%%%%%%%%%%%%%%%%%%%%%%%%%%%%%%%%%%%%%%%%%%%%%%%%%%%%%
\subsection{Forwarding}
\label{sec:forward}

Different versions of the main or child documents
using compilation flags as described in \secref{sec:flags}
can be (permanently) stored in different files
for convenient compilation, viewing and distribution.
To this end, the package defines a command
to pass on compilation to a different file:

%%%%%%%%%%%%%%%%%%%%%%%%%%%%%%%%%%%%%%%%
\DescribeMacro{\childdocforward}
The command |\childdocforward| redirects processing to
another source file:
%
\begin{center}
\begin{tabular}{l}
|\input{childdoc.def}|\\
|\childdocforward[|\textit{main}|]{|\textit{dest}|}|\\
\end{tabular}
\end{center}
%
The argument \textit{dest} is the destination file
(without extension).
It should be the main file or one of the child files.
Note that further \textsf{childdoc} directives
such as |\childdocof| and |\childdocforward|
in the indicated file will be processed in this form.
The optional argument \textit{main}
passes on directly to the main file \textit{main}
while pretending to compile the child \textit{dest}.
This form behaves as if \textit{dest}
issues |\childdocof{|\textit{main}|}| right away,
and no further \textsf{childdoc} directives will be processed.

%%%%%%%%%%%%%%%%%%%%%%%%%%%%%%%%%%%%%%%%
\DescribeMacro{\...prefix}
In the alternative form |\childdocforwardprefix|,
%
\begin{center}
\begin{tabular}{l}
|\input{childdoc.def}|\\
|\childdocforwardprefix[|\textit{main}|]{|\textit{prefix}|}{|\textit{dest}|}|
\end{tabular}
\end{center}
%
the destination file is determined by a pattern
depending on the current file:
To make this work, the current file must be called
`{\textit{prefix}\hspace{0.2em}\textit{suffix}}'
with \textit{prefix} matching precisely the argument.
Processing is then passed on to the file
`{\textit{dest}\hspace{0.2em}\textit{suffix}}'.
Surely, the same effect is achieved by
directly specifying the
argument `{\textit{dest}\hspace{0.2em}\textit{suffix}}'
in the first form.
However, that requires to set up a different file
for each child. With the alternative form of the command
all these files can have exactly the same content
which simplifies setting them up and maintaining them.

For example, the following file |draft.tex|
with a compilation flag |\version| as described in \secref{sec:flags}
compiles the main document as a draft:
%
\begin{center}
\begin{tabular}{l}
|\def\version{draft}|\\
|\input{childdoc.def}|\\
|\childdocforward{|\textit{main}|}|
\end{tabular}
\end{center}
%
Likewise, the following files |final|\textit{nn}|.tex|
compile the final version of the child document
|child|\textit{nn}|.tex|:
%
\begin{center}
\begin{tabular}{l}
|\def\version{final}|\\
|\input{childdoc.def}|\\
|\childdocforwardprefix{final}{child}|
\end{tabular}
\end{center}
%

Note that when several versions of a main file and/or of each child file
are to be generated, it may be convenient to set up a |Makefile| or
shell script to automatise the process.

%%%%%%%%%%%%%%%%%%%%%%%%%%%%%%%%%%%%%%%%%%%%%%%%%%%%%%%%%%%%%%%%%%%%%%%%%%%%%%%%
\subsection{Command Line Processing}
\label{sec:commandline}

The effect of redirection files can also be achieved by invoking
the \LaTeX{} compiler with a more elaborate command line.
Most conveniently this should be done as part
of a shell script or a |Makefile|.

When using \textsf{childdoc} in the main file, the following
command lines effectively perform a redirection
(note that depending on the shell being used,
backslashes may have to be doubled: `|\|' $\to$ `|\\|'):
%
\begin{center}
|... -jobname "|\textit{target}|" |\\|"|[\textit{flags}]%
|\input{childdoc.def}\childdocforward[|\textit{main}|]{|\textit{dest}|}"|
\end{center}
%
Here \textit{target} is the name of the output file,
\textit{main} is the name of the main file
and \textit{dest} is the name of the main or child file to be processed
(all filenames without extensions).
The optional argument \textit{main} can be omitted
if \textit{main} matches \textit{dest}.
Optionally, compilation \textit{flags} can be defined via |\def| commands.
This command line makes the \TeX{} engine believe
it is compiling the file \textit{target}
whose content is specified as the latter parameter.
The provided code then forwards the processing to
\textit{main} or \textit{dest} as described in \secref{sec:forward}.

%%%%%%%%%%%%%%%%%%%%%%%%%%%%%%%%%%%%%%%%%%%%%%%%%%%%%%%%%%%%%%%%%%%%%%%%%%%%%%%%
\subsection{Include by Input}
\label{sec:input}

Including child documents by |\include| has some restrictions by design.
Most notably, the content of a child document always occupies
its own set of pages; pages cannot be shared between child documents.
Usually, this behaviour makes perfect sense
because each child document contain an essential part of the document.
However, in some situations it may be desirable to compose
a document from a collection of parts
without having mandatory page breaks between then.
For this case, the package
provides a mechanism to include parts
by |\input| which can also be processed individually.
However, by construction this mechanism
requires manual handling of the content to be output.

%%%%%%%%%%%%%%%%%%%%%%%%%%%%%%%%%%%%%%%%
\DescribeMacro{\ifchilddocmanual}
The main file should be prepared as usual, see \secref{sec:include}.
However, the document body must make a distinction
between processing of an individual part and of the main document, e.g.:
%
\begin{center}
\begin{tabular}{l}
|\ifchilddocmanual|\\
|\input{\childdocname}|\\
|\||else|\\
\textit{document body with }|\input{|\textit{part}|}|\\
|\||fi|
\end{tabular}
\end{center}
%
The conditional |\ifchilddocmanual| is true whenever
a part to be included by |\input| is being compiled,
and the name of the part is stored in |\childdocname|.

%%%%%%%%%%%%%%%%%%%%%%%%%%%%%%%%%%%%%%%%
\DescribeMacro{\childdocby}
Each part to be included by |\input| should start with:
%
\begin{center}
\begin{tabular}{l}
|\input{childdoc.def}|\\
|\childdocby{|\textit{main}|}|\\
\end{tabular}
\end{center}
%
The directive |\childdocby| is similar to |\childdocof|
described in \secref{sec:include},
but the subsequent selection of content must be done manually.
To that end, both |\ifchilddoc| and |\ifchilddocmanual|
will be true upon processing of a part,
and the name of the part is stored in |\childdocname|.
Note that |\jobname| will be set to the filename of the current part
so that each part receives an individual |.aux| file
that does not interfere with the |.aux| file(s) of the main document.
This behaviour can be altered by the alternative form
|\childdocby[*]{|\textit{main}|}| (with a non-empty optional argument)
which uses the |.aux| file of the main document
by setting |\jobname| to \textit{main}.

%%%%%%%%%%%%%%%%%%%%%%%%%%%%%%%%%%%%%%%%%%%%%%%%%%%%%%%%%%%%%%%%%%%%%%%%%%%%%%%%
\subsection{Driver Development}
\label{sec:driver}

The \textsf{childdoc} mechanism can also be use for the development
of definition files such as \LaTeX{} styles or classes.
This case differs from the above setup with multiple parts
included by |\include| in that no |\includeonly| should be invoked.
This can be achieved by starting the include file
(before |\ProvidesPackage|) with:
%
\begin{center}
\begin{tabular}{l}
|\input{childdoc.def}|\\
|\childdocforward{|\textit{main}|}|\\
\end{tabular}
\end{center}
%
or alternatively with:
%
\begin{center}
\begin{tabular}{l}
|\input{childdoc.def}|\\
|\childdocby{|\textit{main}|}|\\
\end{tabular}
\end{center}
%
Both forms have slightly different effects as described above.
The main file is prepared as usual, see \secref{sec:include}.

%%%%%%%%%%%%%%%%%%%%%%%%%%%%%%%%%%%%%%%%%%%%%%%%%%%%%%%%%%%%%%%%%%%%%%%%%%%%%%%%
\subsection{Legacy Detection}
\label{sec:detection}

The directive |\childdocmain| in the main file can detect
whether the complete document or merely a child is to be compiled
even without using the directive |\childdocof|.
This method is deprecated because it is less robust
and there is no compelling reason to use it;
it is merely provided for backward compatibility
and it may be removed in future versions.

If the detection mechanism is to be used,
it is mandatory to correctly specify
the filename of the main file as the argument of |\childdocmain|:
%
\begin{center}
\begin{tabular}{l}
|\input{childdoc.def}|\\
|\childdocmain{|\textit{main}|}|\\
\end{tabular}
\end{center}
%
If |\jobname| does not match the argument \textit{main} of |\childdocmain|,
it is assumed that |\jobname| points to the child file to be compiled.
When using |\childdocmain| with the main file specified as argument,
it suffices to start a child file
with just |\input{|\textit{main}|}|
without loading of the package and using |\childdocof|.
If instead all processing is done
with the appropriate \textsf{childdoc} directives,
the argument of \textit{main} of |\childdocmain| can be empty.

An alternative version of the command line processing described
in \secref{sec:commandline} using the detection mechanism reads:
%
\begin{center}
|... -jobname "|\textit{target}|" "|[\textit{flags}]%
[|\def\jobname{|\textit{dest}|}|]|\input{|\textit{main}|}"|
\end{center}

%%%%%%%%%%%%%%%%%%%%%%%%%%%%%%%%%%%%%%%%%%%%%%%%%%%%%%%%%%%%%%%%%%%%%%%%%%%%%%%%
\subsection{Manual Code}
\label{sec:manual}

In case one cannot be certain whether the definitions file |childdoc.def|
is installed on the target \TeX{} distribution
and one prefers not to ship it,
it is conceivable to paste a few relevant commands into the sources.

To that end, drop all statements |\input{childdoc.def}|
and perform the replacements as outlined below.
Instead of |\childdocmain{|\textit{main}|}| add the following code
to the top of the main file:
%
\begin{center}
\begin{tabular}{l}
|\||ifdefined\childdocname\endinput\||fi\newif\ifchilddoc|\\
|\edef\childdocname{\scantokens\expandafter{\jobname\noexpand}}|\\
|\def\childdocmain{|\textit{main}|}\||ifx\childdocmain\childdocname\||else|\\
|\childdoctrue\includeonly{\childdocname}\let\jobname\childdocmain\||fi|\\
\end{tabular}
\end{center}
%
Instead of |\childdocof{|\textit{main}|}| just include the main file
at the top of each child file:
%
\begin{center}
|\input{|\textit{main}|}|
\end{center}
%
A simple redirection |\childdocforward{|\textit{dest}|}| is achieved by:
%
\begin{center}
|\def\jobname{|\textit{dest}|}\input{\jobname}|
\end{center}
%
The redirection with prefix
|\childdocforwardprefix[|\textit{prefix}|]{|\textit{dest}|}|
is accomplished by:
%
\begin{center}
\begin{tabular}{l}
|{\edef\jobname{\scantokens\expandafter{\jobname\noexpand}}|\\
|\def\redirectjob |\textit{prefix}|#1~~~{\gdef\jobname{|\textit{dest}|#1}}|\\
|\expandafter\redirectjob\jobname~~~}\input{\jobname}|
\end{tabular}
\end{center}

In an alternative approach,
child documents can be compiled by a specific command line
without additional code or specific definitions:
%
\begin{center}
|... -jobname "|\textit{target}|" "|[\textit{flags}]%
|\includeonly{|\textit{dest}|}\input{|\textit{main}|}"|
\end{center}
%

%%%%%%%%%%%%%%%%%%%%%%%%%%%%%%%%%%%%%%%%%%%%%%%%%%%%%%%%%%%%%%%%%%%%%%%%%%%%%%%%
%%%%%%%%%%%%%%%%%%%%%%%%%%%%%%%%%%%%%%%%%%%%%%%%%%%%%%%%%%%%%%%%%%%%%%%%%%%%%%%%
\section{Information}

%%%%%%%%%%%%%%%%%%%%%%%%%%%%%%%%%%%%%%%%%%%%%%%%%%%%%%%%%%%%%%%%%%%%%%%%%%%%%%%%
\subsection{Copyright}

Copyright \copyright{} 2017--2018 Niklas Beisert

This work may be distributed and/or modified under the
conditions of the \LaTeX{} Project Public License, either version 1.3
of this license or (at your option) any later version.
The latest version of this license is in
  \url{http://www.latex-project.org/lppl.txt}
and version 1.3 or later is part of all distributions of \LaTeX{}
version 2005/12/01 or later.

This work has the LPPL maintenance status `maintained'.

The Current Maintainer of this work is Niklas Beisert.

This work consists of the files |README.txt|, |childdoc.ins| and |childdoc.dtx|
as well as the derived files |childdoc.def|, |cdocsamp.tex|
with |cdocsch1.tex|, |cdocsch2.tex|, |cdocspt3.tex|, |cdocspt4.tex|,
|cdocsdrf.tex|, |cdocsfn1.tex|, |cdocsfn2.tex|
as well as |childdoc.pdf|.

%%%%%%%%%%%%%%%%%%%%%%%%%%%%%%%%%%%%%%%%%%%%%%%%%%%%%%%%%%%%%%%%%%%%%%%%%%%%%%%%
\subsection{Files and Installation}

The package consists of the files:
%
\begin{center}
\begin{tabular}{ll}
    |README.txt|   & readme file \\
    |childdoc.ins| & installation file \\
    |childdoc.dtx| & source file \\
    |childdoc.def| & definition file \\
    |cdocsamp.tex| & sample main file \\
    |cdocsch1.tex| & sample include file \\
    |cdocsch2.tex| & sample include file \\
    |cdocspt3.tex| & sample part file \\
    |cdocspt4.tex| & sample part file \\
    |cdocsdrf.tex| & sample redirection file \\
    |cdocsfn1.tex| & sample redirection file \\
    |cdocsfn2.tex| & sample redirection file \\
    |childdoc.pdf| & manual
\end{tabular}
\end{center}
%
The distribution consists of the files
|README.txt|, |childdoc.ins| and |childdoc.dtx|.
%
\begin{itemize}
\item
Run (pdf)\LaTeX{} on |childdoc.dtx|
to compile the manual |childdoc.pdf| (this file).
\item
Run \LaTeX{} on |childdoc.ins| to create the definitions file |childdoc.def|
and the sample |cdocsamp.tex| with include files
|cdocsch1.tex|, |cdocsch2.tex|, |cdocspt3.tex|, |cdocspt4.tex|,
|cdocsdrf.tex|, |cdocsfn1.tex|, |cdocsfn2.tex|.
Then copy the file |childdoc.def| to an appropriate directory of your \LaTeX{}
distribution, e.g.\ \textit{texmf-root}|/tex/latex/childdoc|.
\end{itemize}

%%%%%%%%%%%%%%%%%%%%%%%%%%%%%%%%%%%%%%%%%%%%%%%%%%%%%%%%%%%%%%%%%%%%%%%%%%%%%%%%
\subsection{Related CTAN Packages}

There are several other packages which offer a similar functionality:
%
\begin{itemize}
\item
The packages
\href{http://ctan.org/pkg/docmute}{\textsf{docmute}},
\href{http://ctan.org/pkg/includex}{\textsf{includex}} and
\href{http://ctan.org/pkg/standalone}{\textsf{standalone}}
provide commands to include only the document body of
a child file thus allowing both files to be compiled individually.
\item
The packages \href{http://ctan.org/pkg/subdocs}{\textsf{subdocs}}
and \href{http://ctan.org/pkg/subfiles}{\textsf{subfiles}}
provide structures in which the main and child documents can be
encapsulated and allowing them to be compiled individually.
The inclusion mechanism is different from the conventional |\include|.
\item
The package \href{http://ctan.org/pkg/combine}{\textsf{combine}}
is an elaborate solution to combine several documents into one.
\end{itemize}
%
See also the CTAN topic \href{http://ctan.org/topic/subdocs}{\textsf{subdocs}}
for further related packages.
The present package differs from the above solutions in that
a document structure constructed with the conventional |\include| mechanism
just needs two extra commands at the top of every file
such that all constituent files can be compiled individually.

%%%%%%%%%%%%%%%%%%%%%%%%%%%%%%%%%%%%%%%%%%%%%%%%%%%%%%%%%%%%%%%%%%%%%%%%%%%%%%%%
%\subsection{Feature Suggestions}
%
%The following is a list of features which may be useful for future
%versions of this package:
%%
%\begin{itemize}
%\item
%\ldots
%\end{itemize}

%%%%%%%%%%%%%%%%%%%%%%%%%%%%%%%%%%%%%%%%%%%%%%%%%%%%%%%%%%%%%%%%%%%%%%%%%%%%%%%%
\subsection{Revision History}

%%%%%%%%%%%%%%%%%%%%%%%%%%%%%%%%%%%%%%%%
\paragraph{v2.0:} 2018/12/30

\begin{itemize}
\item
immediate forward processing
\item
added |\childdocby| mechanism
\item
manual restructured
\end{itemize}

%%%%%%%%%%%%%%%%%%%%%%%%%%%%%%%%%%%%%%%%
\paragraph{v1.6:} 2018/01/17

\begin{itemize}
\item
application for development of include files
\item
corrections to manual
\end{itemize}

%%%%%%%%%%%%%%%%%%%%%%%%%%%%%%%%%%%%%%%%
\paragraph{v1.5:} 2017/05/21

\begin{itemize}
\item
more complete structuring introduced
\item
|\childdocof| introduced
\item
|\childdoc| renamed to |\childdocmain|
\item
|\childredirect| renamed to |\childdocforward| and |\childdocforwardprefix|
and functionality expanded
\end{itemize}

%%%%%%%%%%%%%%%%%%%%%%%%%%%%%%%%%%%%%%%%
\paragraph{v1.0:} 2017/04/27

\begin{itemize}
\item
manual and install package
\item
first version published on CTAN
\end{itemize}

%%%%%%%%%%%%%%%%%%%%%%%%%%%%%%%%%%%%%%%%
\paragraph{v0.6:} 2017/04/26

\begin{itemize}
\item
redirection mechanism added
\end{itemize}

%%%%%%%%%%%%%%%%%%%%%%%%%%%%%%%%%%%%%%%%
\paragraph{v0.5:} 2017/04/26

\begin{itemize}
\item
functionality in definition file
\end{itemize}


%%%%%%%%%%%%%%%%%%%%%%%%%%%%%%%%%%%%%%%%%%%%%%%%%%%%%%%%%%%%%%%%%%%%%%%%%%%%%%%%
%%%%%%%%%%%%%%%%%%%%%%%%%%%%%%%%%%%%%%%%%%%%%%%%%%%%%%%%%%%%%%%%%%%%%%%%%%%%%%%%
%%%%%%%%%%%%%%%%%%%%%%%%%%%%%%%%%%%%%%%%%%%%%%%%%%%%%%%%%%%%%%%%%%%%%%%%%%%%%%%%
\appendix

\settowidth\MacroIndent{\rmfamily\scriptsize 000\ }

 \DocInput{childdoc.dtx}

\end{document}
%</driver>
% \fi
%
% %%%%%%%%%%%%%%%%%%%%%%%%%%%%%%%%%%%%%%%%%%%%%%%%%%%%%%%%%%%%%%%%%%%%%%%%%%%%%%
% %%%%%%%%%%%%%%%%%%%%%%%%%%%%%%%%%%%%%%%%%%%%%%%%%%%%%%%%%%%%%%%%%%%%%%%%%%%%%%
% \section{Sample}
%\iffalse
%<*samplemain>
%\fi
%
% The following presents a sample document
% with two chapters, two parts, a title page,
% a compile flag as well as three forwarding files to set the flag.
% It consists of eight |.tex| files:
% \begin{center}
% \begin{tabular}{ll}
% |cdocsamp.tex|&main file\\
% |cdocsch1.tex|&include file for chapter 1\\
% |cdocsch2.tex|&include file for chapter 2\\
% |cdocspt3.tex|&include file for part 3\\
% |cdocspt4.tex|&include file for part 4\\
% |cdocsdrf.tex|&forwarding file for main file in draft mode\\
% |cdocsfi1.tex|&forwarding file for final version of chapter 1\\
% |cdocsfi2.tex|&forwarding file for final version of chapter 2\\
% \end{tabular}
% \end{center}
% Each of the eight files can be compiled directly by the \LaTeX{} compiler.
%
% %%%%%%%%%%%%%%%%%%%%%%%%%%%%%%%%%%%%%%
% \paragraph{Main File.}
%
% The main file is called |cdocsamp.tex|.
%
% Load the \textsf{childdoc} definitions and
% declare the filename for the main document:
%    \begin{macrocode}
\input{childdoc.def}
\childdocmain{}
%    \end{macrocode}

% Optional override for |\version| flag:
%    \begin{macrocode}
%%\ifchilddoc\else\providecommand{\version}{draft}\fi
%    \end{macrocode}

% Define the default values for the |\version| flag
% (|final| for the main file and |draft| for childs):
%    \begin{macrocode}
\ifchilddoc
\providecommand{\version}{draft}
\else
\providecommand{\version}{final}
\fi
%    \end{macrocode}

% Load the standard document class:
%    \begin{macrocode}
\documentclass[12pt]{article}
%    \end{macrocode}

% Start the document body:
%    \begin{macrocode}
\begin{document}
%    \end{macrocode}

% Declare a title page.
% Print title, part of document being processed and version flag:
%    \begin{macrocode}
\addtocounter{page}{-1}
\begin{center}
{\LARGE\bfseries{}childdoc example\par}
\vspace{1cm}
\ifchilddoc
\ifchilddocmanual part\else chapter\fi:
`\childdocname' of `\childdocjob'\par
\else
main document: `\childdocjob'\par
\fi
version: \version\par
\end{center}
\newpage
%    \end{macrocode}

% Manually include selected file,
% otherwise process as usual:
%    \begin{macrocode}
\ifchilddocmanual
\section*{part `\childdocname'}
\input{\childdocname}
\else
%    \end{macrocode}

% Include the two chapters:
%    \begin{macrocode}
\include{cdocsch1}
\include{cdocsch2}
%    \end{macrocode}

% Include the two parts unless only chapters should be displayed:
%    \begin{macrocode}
\ifchilddoc\else
\section{part three}
\input{cdocspt3}
\section{part four}
\input{cdocspt4}
\fi
%    \end{macrocode}

% Process as usual until here:
%    \begin{macrocode}
\fi
%    \end{macrocode}

% End of document body:
%    \begin{macrocode}
\end{document}
%    \end{macrocode}
%\iffalse
%</samplemain>
%\fi
%
% %%%%%%%%%%%%%%%%%%%%%%%%%%%%%%%%%%%%%%
% \paragraph{Chapter Include Files.}
%
% The include files are called |cdocsch1.tex| and |cdocsch2.tex|.
%
%\iffalse
%<*samplechap1|samplechap2>
%\fi

% Optional override for |\version| flag:
%    \begin{macrocode}
%%\providecommand{\version}{final}
%    \end{macrocode}

% Include the main document:
%    \begin{macrocode}
\input{childdoc.def}
\childdocof{cdocsamp}
%    \end{macrocode}

%\iffalse
%</samplechap1|samplechap2>
%\fi
%
%\iffalse
%<*samplechap1>
%\fi
% Some text for chapter 1:
%    \begin{macrocode}
\section{one}
some text in chapter one
%    \end{macrocode}

%\iffalse
%</samplechap1>
%\fi
% Some text for chapter 2:
%\iffalse
%<*samplechap2>
%\fi
%    \begin{macrocode}
\section{two}
more text in chapter two
%    \end{macrocode}

%\iffalse
%</samplechap2>
%\fi
%
% %%%%%%%%%%%%%%%%%%%%%%%%%%%%%%%%%%%%%%
% \paragraph{Part Include Files.}
%
% The include files are called |cdocspt3.tex| and |cdocspt4.tex|.
%
%\iffalse
%<*samplepart3|samplepart4>
%\fi

% Optional override for |\version| flag:
%    \begin{macrocode}
%%\providecommand{\version}{final}
%    \end{macrocode}

% Include the main document:
%    \begin{macrocode}
\input{childdoc.def}
\childdocby{cdocsamp}
%    \end{macrocode}

%\iffalse
%</samplepart3|samplepart4>
%\fi
%
%\iffalse
%<*samplepart3>
%\fi
% Some text for part 3:
%    \begin{macrocode}
some text in part three
%    \end{macrocode}

%\iffalse
%</samplepart3>
%\fi
% Some text for part 4:
%\iffalse
%<*samplepart4>
%\fi
%    \begin{macrocode}
more text in part four
%    \end{macrocode}

%\iffalse
%</samplepart4>
%\fi
%
% %%%%%%%%%%%%%%%%%%%%%%%%%%%%%%%%%%%%%%
% \paragraph{Forwarding for a Complete Draft.}
%
% The following forwarding file |cdocsdrf.tex|
% compiles the main document in draft mode:
%\iffalse
%<*sampledraft>
%\fi
%    \begin{macrocode}
\def\version{draft}
\input{childdoc.def}
\childdocforward{cdocsamp}
%    \end{macrocode}

%\iffalse
%</sampledraft>
%\fi
%
% %%%%%%%%%%%%%%%%%%%%%%%%%%%%%%%%%%%%%%
% \paragraph{Forwarding for Final Version of the Chapters.}
%
% The following forwarding files |cdocsfn1.tex| and |cdocsfn2.tex|
% (with identical content)
% compile the final versions of the child documents
% |cdocsch1.tex| and |cdocsch2.tex|, respectively:
%\iffalse
%<*samplefinal>
%\fi
%    \begin{macrocode}
\def\version{final}
\input{childdoc.def}
\childdocforwardprefix[cdocsamp]{cdocsfn}{cdocsch}
%    \end{macrocode}

%\iffalse
%</samplefinal>
%\fi
%
% %%%%%%%%%%%%%%%%%%%%%%%%%%%%%%%%%%%%%%
% \paragraph{Command Line Processing.}
%
% The following three command lines generate the output files
% |cdocscld|, |cdocscl1| and |cdocscl2|
% which should be identical to
% |cdocsdrf|, |cdocsch1| and |cdocsfn2|, respectively:
% \begin{center}
% \begin{tabular}{l}
% |latex -jobname cdocscld \|\\
% |  "\def\version{draft}\input{childdoc.def}\childdocforward{cdocsamp}"|\\
% |latex -jobname cdocscl1 \|\\
% |  "\input{childdoc.def}\childdocforward[cdocsamp]{cdocsch1}"|\\
% |latex -jobname cdocscl2 \|\\
% |  "\def\version{final}\input{childdoc.def}\childdocforward{cdocsch2}"|
% \end{tabular}
% \end{center}
% Note that the trailing backslash on each first line
% merely continues the input to the second line
% (for convenient cut ant paste).
% Furthermore, the command |latex| can be replaced by any
% of its alternative versions such as |pdflatex|.
%
% %%%%%%%%%%%%%%%%%%%%%%%%%%%%%%%%%%%%%%%%%%%%%%%%%%%%%%%%%%%%%%%%%%%%%%%%%%%%%%
% %%%%%%%%%%%%%%%%%%%%%%%%%%%%%%%%%%%%%%%%%%%%%%%%%%%%%%%%%%%%%%%%%%%%%%%%%%%%%%
% \section{Implementation}
%\iffalse
%<*package>
%\fi
%
% This section describes the definitions file |childdoc.def|.

% The definitions cannot be loaded using |\usepackage| or |\RequirePackage|
% which has a mechanism to prevent loading a style file more than once.
% When loading the definitions by means of |\input|
% multiple instances have to be prevented manually:
%\iffalse
%This code needs to be before the `\ProvidesFile' directive
%which is defined at the beginning of this file.
%Therefore it is also placed there and commented out here.
%</package>
%<*discard>
%\fi
%    \begin{macrocode}
\ifdefined\childdocmain\endinput\fi
%    \end{macrocode}
%\iffalse
%</discard>
%<*package>
%\fi
%
% \macro{\ifchilddoc}
% \macro{\ifchilddocmanual}
% The conditional |\ifchilddoc| tells whether a
% child (true) or main (false) document is being compiled.
% The conditional |\ifchilddocmanual| tells whether
% the |\includeonly| mechanism is used (false) or
% the selection of child files must be performed manually (true).
% The definitions initialise to false:
%    \begin{macrocode}
\newif\ifchilddoc
\newif\ifchilddocmanual
%    \end{macrocode}

% \macro{\childdocname}
% \macro{\childdocjob}
% The macro |\childdocname| stores the name of the main document
% to be compiled. The macro |\childdocjob| stores the name of
% the document on which the \LaTeX{} compiler was originally invoked.
% The content of |\jobname| cannot be compared
% to filenames specified in the source due to different catcodes.
% The following code rescans |\jobname|, stores the result
% in |\childdocname| and saves a copy in |\childdocjob|:
%    \begin{macrocode}
\edef\childdocname{\scantokens\expandafter{\jobname\noexpand}}
\let\childdocjob\childdocname
%    \end{macrocode}

% \macro{\childdocdisable}
% The macro |\childdocdisable| prevents the main file
% from being processed more than once.
% At this stage, the main document command |\childdocmain|
% is assumed to be called once again where it should do nothing.
% Any subsequent call to it should prevent
% a secondary processing of the main document
% It overwrites the forwarding commands
% |\childdocof| and |\childdocforward|
% with empty macros to prevent further inclusions of the main document:
%    \begin{macrocode}
\newcommand{\childdocdisable}
{
  \renewcommand{\childdocmain}[1]{\renewcommand{\childdocmain}[1]{\endinput}}
  \renewcommand{\childdocof}[1]{}
  \renewcommand{\childdocby}[2][]{}
  \renewcommand{\childdocforward}[2][]{}
  \renewcommand{\childdocdisable}{}
}
%    \end{macrocode}

% \macro{\childdocmain}
% The macro |\childdocmain| is to be called at the top of the main file
% with nothing or the main filename (without extension) as argument.
% First, it breaks loops.
% If the argument is not empty and does not match |\childdocname|
% (which is set by the first inclusion of |childdoc.def|),
% |\ifchilddoc| is set to true, |\includeonly| is applied to the child file
% and |\jobname| is set to the main file
% (for proper handling of |.aux| files):
%    \begin{macrocode}
\newcommand{\childdocmain}[1]
{
  \childdocdisable\childdocmain{}
  \if?#1?\else
    \begingroup
      \def\childdoctmp{#1}
      \ifx\childdoctmp\childdocname
        \def\childdoctmp{}
      \else
        \def\childdoctmp
        {
          \childdoctrue
          \includeonly{\childdocname}
          \def\childdocjob{#1}
          \def\jobname{#1}
        }
      \fi
      \expandafter
    \endgroup
    \childdoctmp
  \fi
}
%    \end{macrocode}

% \macro{\childdocof}
% The command |\childdocof| redirects
% compilation to the main file |#1|.
%    \begin{macrocode}
\newcommand{\childdocof}[1]
{
  \childdocdisable
  \childdoctrue
  \includeonly{\childdocname}
  \def\jobname{#1}
  \def\childdocjob{#1}
  \input{#1}
}
%    \end{macrocode}

% \macro{\childdocby}
% The command |\childdocby| ....
%    \begin{macrocode}
\newcommand{\childdocby}[2][]
{
  \childdocdisable
  \childdoctrue
  \childdocmanualtrue
  \if?#1?\else
    \def\jobname{#2}
  \fi
  \def\childdocjob{#2}
  \input{#2}
  \endinput
}
%    \end{macrocode}

% \macro{\childdocforward}
% The command |\childdocforward| redirects
% compilation to the main file or
% (if the optional argument is given) a child file.
% Parameters are set as if the main file
% or a child file starting with |\childdocof| was compiled.
% Then compilation is handed over to the main file:
%    \begin{macrocode}
\newcommand{\childdocforward}[2][]
{
  \begingroup
    \if?#1?
      \def\childdoctmp
      {
        \def\childdocname{#2}
        \def\childdocjob{#2}
        \def\jobname{#2}
        \input{#2}
        \endinput
      }
    \else
      \def\childdoctmp
      {
        \childdocdisable
        \def\childdocname{#2}
        \childdoctrue
        \includeonly{#2}
        \def\childdocjob{#1}
        \def\jobname{#1}
        \input{#1}
        \endinput
      }
    \fi
    \expandafter
  \endgroup
  \childdoctmp
}
%    \end{macrocode}

% \macro{\childdocforwardprefix}
% The command |\childdocforwardprefix| redirects
% compilation to the main or a child file by means of a pattern.
% The prefix |#1| in the current filename is replaced by |#2|
% and the suffix of the current filename is kept
% (it is assumed that the filename does not contain the substring `|~~~|'
% which is used as a delimiter).
% Compilation is handed over to the new file by |\childdocforward|:
%    \begin{macrocode}
\newcommand{\childdocforwardprefix}[3][]
{
  \begingroup
    \def\childdocextract #2##1~~~{\def\childdoctmp{\childdocforward[#1]{#3##1}}}
    \expandafter\childdocextract\childdocname~~~
    \expandafter
  \endgroup
  \childdoctmp
}
%    \end{macrocode}

% \macro{\childdoc}
% The deprecated macro |\childdoc| is a legacy version of |\childdocmain|:
%    \begin{macrocode}
\newcommand{\childdoc}{\childdocmain}
%    \end{macrocode}

% \macro{\childdocredirect}
% The deprecated macro |\childdocredirect| is a legacy version
% of |\childdocforward| and |\childdocforwardprefix|:
%    \begin{macrocode}
\newcommand{\childdocredirect}[2][]
{
  \begingroup
    \if?#1?
      \def\childdoctmp{\childdocforward{#2}}
    \else
      \def\childdoctmp{\childdocforwardprefix{#1}{#2}}
    \fi
    \expandafter
  \endgroup
  \childdoctmp
}
%    \end{macrocode}

%\iffalse
%</package>
%\fi
%
\endinput
|\\
|\childdocby{|\textit{main}|}|\\
\end{tabular}
\end{center}
%
The directive |\childdocby| is similar to |\childdocof|
described in \secref{sec:include},
but the subsequent selection of content must be done manually.
To that end, both |\ifchilddoc| and |\ifchilddocmanual|
will be true upon processing of a part,
and the name of the part is stored in |\childdocname|.
Note that |\jobname| will be set to the filename of the current part
so that each part receives an individual |.aux| file
that does not interfere with the |.aux| file(s) of the main document.
This behaviour can be altered by the alternative form
|\childdocby[*]{|\textit{main}|}| (with a non-empty optional argument)
which uses the |.aux| file of the main document
by setting |\jobname| to \textit{main}.

%%%%%%%%%%%%%%%%%%%%%%%%%%%%%%%%%%%%%%%%%%%%%%%%%%%%%%%%%%%%%%%%%%%%%%%%%%%%%%%%
\subsection{Driver Development}
\label{sec:driver}

The \textsf{childdoc} mechanism can also be use for the development
of definition files such as \LaTeX{} styles or classes.
This case differs from the above setup with multiple parts
included by |\include| in that no |\includeonly| should be invoked.
This can be achieved by starting the include file
(before |\ProvidesPackage|) with:
%
\begin{center}
\begin{tabular}{l}
|% \iffalse
%
% childdoc.dtx Copyright (C) 2017-2018 Niklas Beisert
%
% This work may be distributed and/or modified under the
% conditions of the LaTeX Project Public License, either version 1.3
% of this license or (at your option) any later version.
% The latest version of this license is in
%   http://www.latex-project.org/lppl.txt
% and version 1.3 or later is part of all distributions of LaTeX
% version 2005/12/01 or later.
%
% This work has the LPPL maintenance status `maintained'.
%
% The Current Maintainer of this work is Niklas Beisert.
%
% This work consists of the files childdoc.dtx and childdoc.ins
% and the derived files childdoc.def and cdocsamp.tex with
% cdocsch1.tex, cdocsch2.tex, cdocsdrf.tex, cdocsfn1.tex, cdocsfn2.tex.
%
%<package>\ifdefined\childdocmain\endinput\fi
%<package>\ProvidesFile{childdoc.def}[2018/12/30 v2.0 child document driver]
%<samplemain>\ProvidesFile{cdocsamp.tex}[2018/12/30 v2.0 sample for childdoc]
%<*driver>
%\ProvidesFile{childdoc.drv}[2018/12/30 v2.0 childdoc reference manual file]
\PassOptionsToClass{10pt,a4paper}{article}
\documentclass{ltxdoc}

\usepackage[margin=35mm]{geometry}
\usepackage{hyperref}
\usepackage{hyperxmp}
\usepackage[usenames]{color}

\hypersetup{colorlinks=true}
\hypersetup{pdfstartview=FitH}
\hypersetup{pdfpagemode=UseNone}
\hypersetup{pdfsource={}}
\hypersetup{pdflang={en-UK}}
\hypersetup{pdfcopyright={Copyright 2017-2018 Niklas Beisert.
  This work may be distributed and/or modified under the
  conditions of the LaTeX Project Public License, either version 1.3
  of this license or (at your option) any later version.}}
\hypersetup{pdflicenseurl={http://www.latex-project.org/lppl.txt}}
\hypersetup{pdfcontactaddress={ETH Zurich, ITP, HIT K,
  Wolfgang-Pauli-Strasse 27}}
\hypersetup{pdfcontactpostcode={8093}}
\hypersetup{pdfcontactcity={Zurich}}
\hypersetup{pdfcontactcountry={Switzerland}}
\hypersetup{pdfcontactemail={nbeisert@itp.phys.ethz.ch}}
\hypersetup{pdfcontacturl={http://people.phys.ethz.ch/\xmptilde nbeisert/}}

\newcommand{\secref}[1]{\hyperref[#1]{section \ref*{#1}}}

\parskip1ex
\parindent0pt
\let\olditemize\itemize
\def\itemize{\olditemize\parskip0pt}

\begin{document}

\title{The \textsf{childdoc} Package}
\hypersetup{pdftitle={The childdoc Package}}
\author{Niklas Beisert\\[2ex]
  Institut f\"ur Theoretische Physik\\
  Eidgen\"ossische Technische Hochschule Z\"urich\\
  Wolfgang-Pauli-Strasse 27, 8093 Z\"urich, Switzerland\\[1ex]
  \href{mailto:nbeisert@itp.phys.ethz.ch}
  {\texttt{nbeisert@itp.phys.ethz.ch}}}
\hypersetup{pdfauthor={Niklas Beisert}}
\hypersetup{pdfsubject={Manual for the LaTeX2e Package childdoc}}
\date{30 December 2018, \textsf{v2.0}}
\maketitle

\begin{abstract}\noindent
\textsf{childdoc} is a \LaTeXe{} package
that enables the direct compilation
of document sections included by |\include|
to individual files.
\end{abstract}

\begingroup
\parskip0ex
\tableofcontents
\endgroup

%%%%%%%%%%%%%%%%%%%%%%%%%%%%%%%%%%%%%%%%%%%%%%%%%%%%%%%%%%%%%%%%%%%%%%%%%%%%%%%%
%%%%%%%%%%%%%%%%%%%%%%%%%%%%%%%%%%%%%%%%%%%%%%%%%%%%%%%%%%%%%%%%%%%%%%%%%%%%%%%%
\section{Introduction}

\LaTeX{} provides a mechanism to structure a large document (such as a book)
into a main file and several child files (containing the chapters)
using the |\include| command.
This mechanism is beneficial for documents
which span hundreds of pages in order to
make the source file(s) more manageable.
Moreover, compilation can be restricted to
selected child files by means of the |\includeonly| command.
The latter feature can be used to reduce the compilation time while editing
(this was significantly more useful in the earlier days of \LaTeX{})
or to generate a smaller document which is easier to navigate.
Another application of |\includeonly| is to generate
documents consisting of selected parts of the complete document.

However, there are a few drawbacks of the plain |\include| mechanism:
\begin{itemize}
\item
The child files cannot be compiled on their own,
they can only be compiled via the main file.
A naive editing environment
(such as a text editor with an option
to have the current file processed by \LaTeX)
may require one to switch to the main file before compiling;
attempting to compile the child file produces errors.
\item
The main file must be modified (each time)
to adjust the |\includeonly| command
to the present needs. This easily leaves the main file in a messy state.
\item
The generated document will always carry the filename
of the main document. This is inconvenient if
several child files are to be compiled and
to be kept for distribution.
\end{itemize}

The present package provides a simple interface
to make child files individually compilable by \LaTeX{}.
Compiling a child file then has the same effect as compiling
the main file with an |\includeonly| command
to select the appropriate child.
Moreover the generated document will carry the name of the child
rather than the main file.
This resolves all three above issues.

This feature is meant to make the editing of books,
thesis documents and lecture notes somewhat more convenient.
However, the package can also be used efficiently for
composing a series of documents (such as exercise sheets)
which are typically distributed individually.
It then assists the author in generating the individual documents
(potentially in different versions)
as well as a document containing the collected series.
Another application is in developing style files
or other kinds of included material
where compilation of the style file could redirect
to a sample or test file.

%%%%%%%%%%%%%%%%%%%%%%%%%%%%%%%%%%%%%%%%%%%%%%%%%%%%%%%%%%%%%%%%%%%%%%%%%%%%%%%%
%%%%%%%%%%%%%%%%%%%%%%%%%%%%%%%%%%%%%%%%%%%%%%%%%%%%%%%%%%%%%%%%%%%%%%%%%%%%%%%%
\section{Usage}

First of all, the package \textsf{childdoc} is \emph{not} a standard
\LaTeXe{} |.sty| style file! Therefore it needs to be invoked in
a non-standard way.

%%%%%%%%%%%%%%%%%%%%%%%%%%%%%%%%%%%%%%%%%%%%%%%%%%%%%%%%%%%%%%%%%%%%%%%%%%%%%%%%
\subsection{Included Files}
\label{sec:include}

%%%%%%%%%%%%%%%%%%%%%%%%%%%%%%%%%%%%%%%%
\DescribeMacro{\childdocmain}
To use the package, add the commands
\begin{center}
\begin{tabular}{l}
|\input{childdoc.def}|\\
|\childdocmain{}|\\
\end{tabular}
\end{center}
at the very top of the main \LaTeX{} file,
in particular \emph{before} the |\documentclass| statement!
The argument of |\childdocmain| should be left empty
(but it must be present).

%%%%%%%%%%%%%%%%%%%%%%%%%%%%%%%%%%%%%%%%
\DescribeMacro{\childdocof}
Furthermore, add the commands
\begin{center}
\begin{tabular}{l}
|\input{childdoc.def}|\\
|\childdocof{|\textit{main}|}|\\
\end{tabular}
\end{center}
at the top of every child file \textit{child}
which is included by |\include{|\textit{child}|}|
from within the main file
(or at least for those files to be compiled individually).
The argument \textit{main} must be the filename of the main file.

There are a couple of
considerations in setting up the main and child documents:

%%%%%%%%%%%%%%%%%%%%%%%%%%%%%%%%%%%%%%%%
\paragraph{Restrictions.}

Please note the following restrictions:
\begin{itemize}
\item
|\childdocmain| must be called with one argument \textit{main}
to ensure compatibility with earlier version of the package.
It must either be empty (|\childdocmain{}|)
or precisely match the filename of the main file in which it is specified.
See \secref{sec:detection} for further information.
\item
The filename \textit{main} must be specified without the |.tex| extension.
\item
The filename \textit{main} is case sensitive
(even in case-insensitive file systems)
due to internal string comparison.
\item
The argument \textit{main} should be fully expanded, it cannot be a macro.
\item
Subdirectories and special characters should be avoided in filenames.
\item
The command |\childdocmain{|\textit{main}|}| must be followed by a whitespace.
It should not be followed immediately by another command
or by a comment mark `|%|'.
This is because the \TeX{} parser reads the token immediately following
the argument of |\childdocmain| and puts it
at the beginning of every child section;
however, a white\-space is ignored.
\end{itemize}

%%%%%%%%%%%%%%%%%%%%%%%%%%%%%%%%%%%%%%%%
\paragraph{Content of Main File.}

It is advisable to place all content in the child files included by |\include|.
Any output contained in the main file will appear in all child documents
unless suppressed manually;
it cannot be suppressed automatically by the |\includeonly| directive
and thus should normally be avoided.
A method to include some content in the main file
by means of conditional processing is described in \secref{sec:conditional}.

%%%%%%%%%%%%%%%%%%%%%%%%%%%%%%%%%%%%%%%%
\paragraph{Page Numbering.}

When only a part of the document is compiled,
the appropriate numbering of pages
(as well as other status parameters)
is determined from the |.aux| files.
The latter contain information from previous passes.
However this information needs to propagate through
all intermediate child documents.
Therefore the page numbering in child documents may well
be inconsistent until the complete document is compiled at least once.

A useful (if unconventional) way to always ensure a consistent
page numbering is to restart the numbering in each child document
and denote the pages by `\textit{child}|.|\textit{page}'
where \textit{child} represents the chapter/section number of the child file.
This can be achieved by the command
|\numberwithin{page}{|\textit{child}|}|
of the \textsf{amsmath} package
where \textit{child} can be |chapter| or |section|
depending on the chosen structuring.
Alternatively, one can modify the macro |\thepage| appropriately
and reset the counter |page| at the start of each child file.

%%%%%%%%%%%%%%%%%%%%%%%%%%%%%%%%%%%%%%%%%%%%%%%%%%%%%%%%%%%%%%%%%%%%%%%%%%%%%%%%
\subsection{Conditional Processing}
\label{sec:conditional}

The package provides a mechanism to compile different versions
of a document. To customise the versions further some conditional processing
can come in handy to distinguish which version is being compiled.
The package provides two macros to describe the compilation context:

%%%%%%%%%%%%%%%%%%%%%%%%%%%%%%%%%%%%%%%%
\DescribeMacro{\ifchilddoc}
The conditional |\ifchilddoc| distinguishes between the compilation of
child documents and the main document:
%
\begin{center}
|\ifchilddoc |\textit{child-code}| |[|\||else |\textit{main-code}]| \||fi|
\end{center}

%%%%%%%%%%%%%%%%%%%%%%%%%%%%%%%%%%%%%%%%
\DescribeMacro{\childdocname}
\DescribeMacro{\childdocjob}
The macro |\childdocname| contains the filename (without extension)
of the main or child file being processed.
Note that |\childdocjob| will always contain the name of the main file.

%%%%%%%%%%%%%%%%%%%%%%%%%%%%%%%%%%%%%%%%
\paragraph{Title Page.}

Conditional processing can be used to include a title or banner page
in the main document when proper precautions are taken.
Importantly, the code in the main file should ensure that the page counter
(as well as other status parameters which are stored in the |.aux| files)
takes the same value after the conditional processing.
Otherwise the page numbers may take divergent values
depending on which part is compiled.

For example, a title page could be declared by:
%
\begin{center}
\begin{tabular}{l}
|\ifchilddoc\||else|\\
|\addtocounter{page}{-1}|\\
\textit{code for title page}\\
|\newpage|\\
|\||fi|
\end{tabular}
\end{center}
%
A banner page for the child documents can be generated by:
%
\begin{center}
\begin{tabular}{l}
|\ifchilddoc|\\
|\addtocounter{page}{-1}|\\
\textit{code for banner page}\\
|\newpage|\\
|\||fi|
\end{tabular}
\end{center}
%
Here one could write a message such as:
\begin{center}
|This is the part \childdocname{} of \childdocjob{}.|
\end{center}

%%%%%%%%%%%%%%%%%%%%%%%%%%%%%%%%%%%%%%%%%%%%%%%%%%%%%%%%%%%%%%%%%%%%%%%%%%%%%%%%
\subsection{Flags}
\label{sec:flags}

The package makes it easy to generate different versions
of the main or child documents.
To this end compilation flags can be defined
and assigned different default values.
They will be particularly useful in conjunction
with the forwarding mechanism described in \secref{sec:forward}.

For example, it may be useful to have a flag |\version|
which can be set to |draft| or |final|.
The document source will contain some conditional code
depending on the value of |\version|.
Suppose further, the flag should default to |final| for the main file
and to |draft| for child files
which is a natural assignment for editing the document.
This is achieved by placing the following code
in the preamble of the main document
(below the |\childdocmain| directive):
%
\begin{center}
\begin{tabular}{l}
|\ifchilddoc|\\
|\providecommand{\version}{draft}|\\
|\||else|\\
|\providecommand{\version}{final}|\\
|\||fi|
\end{tabular}
\end{center}
%
The definition by |\providecommand| makes sure
that previous definitions are not overwritten.
Further statements |\providecommand{\version}{...}|
can thus be added before the above code to override it.

For the main file, one might add a line
(between |\childdocmain| and the above block)
%
\begin{center}
|%\ifchilddoc\||else\providecommand{\version}{draft}\||fi|
\end{center}
%
which can be uncommented to produce a draft version.
Likewise one can add a line to the very top of a child file
(above the |\childdocof{|\textit{main}|}| directive)
%
\begin{center}
|%\providecommand{\version}{final}|
\end{center}
%
which can be uncommented to produce the final version of this child document.

%%%%%%%%%%%%%%%%%%%%%%%%%%%%%%%%%%%%%%%%%%%%%%%%%%%%%%%%%%%%%%%%%%%%%%%%%%%%%%%%
\subsection{Forwarding}
\label{sec:forward}

Different versions of the main or child documents
using compilation flags as described in \secref{sec:flags}
can be (permanently) stored in different files
for convenient compilation, viewing and distribution.
To this end, the package defines a command
to pass on compilation to a different file:

%%%%%%%%%%%%%%%%%%%%%%%%%%%%%%%%%%%%%%%%
\DescribeMacro{\childdocforward}
The command |\childdocforward| redirects processing to
another source file:
%
\begin{center}
\begin{tabular}{l}
|\input{childdoc.def}|\\
|\childdocforward[|\textit{main}|]{|\textit{dest}|}|\\
\end{tabular}
\end{center}
%
The argument \textit{dest} is the destination file
(without extension).
It should be the main file or one of the child files.
Note that further \textsf{childdoc} directives
such as |\childdocof| and |\childdocforward|
in the indicated file will be processed in this form.
The optional argument \textit{main}
passes on directly to the main file \textit{main}
while pretending to compile the child \textit{dest}.
This form behaves as if \textit{dest}
issues |\childdocof{|\textit{main}|}| right away,
and no further \textsf{childdoc} directives will be processed.

%%%%%%%%%%%%%%%%%%%%%%%%%%%%%%%%%%%%%%%%
\DescribeMacro{\...prefix}
In the alternative form |\childdocforwardprefix|,
%
\begin{center}
\begin{tabular}{l}
|\input{childdoc.def}|\\
|\childdocforwardprefix[|\textit{main}|]{|\textit{prefix}|}{|\textit{dest}|}|
\end{tabular}
\end{center}
%
the destination file is determined by a pattern
depending on the current file:
To make this work, the current file must be called
`{\textit{prefix}\hspace{0.2em}\textit{suffix}}'
with \textit{prefix} matching precisely the argument.
Processing is then passed on to the file
`{\textit{dest}\hspace{0.2em}\textit{suffix}}'.
Surely, the same effect is achieved by
directly specifying the
argument `{\textit{dest}\hspace{0.2em}\textit{suffix}}'
in the first form.
However, that requires to set up a different file
for each child. With the alternative form of the command
all these files can have exactly the same content
which simplifies setting them up and maintaining them.

For example, the following file |draft.tex|
with a compilation flag |\version| as described in \secref{sec:flags}
compiles the main document as a draft:
%
\begin{center}
\begin{tabular}{l}
|\def\version{draft}|\\
|\input{childdoc.def}|\\
|\childdocforward{|\textit{main}|}|
\end{tabular}
\end{center}
%
Likewise, the following files |final|\textit{nn}|.tex|
compile the final version of the child document
|child|\textit{nn}|.tex|:
%
\begin{center}
\begin{tabular}{l}
|\def\version{final}|\\
|\input{childdoc.def}|\\
|\childdocforwardprefix{final}{child}|
\end{tabular}
\end{center}
%

Note that when several versions of a main file and/or of each child file
are to be generated, it may be convenient to set up a |Makefile| or
shell script to automatise the process.

%%%%%%%%%%%%%%%%%%%%%%%%%%%%%%%%%%%%%%%%%%%%%%%%%%%%%%%%%%%%%%%%%%%%%%%%%%%%%%%%
\subsection{Command Line Processing}
\label{sec:commandline}

The effect of redirection files can also be achieved by invoking
the \LaTeX{} compiler with a more elaborate command line.
Most conveniently this should be done as part
of a shell script or a |Makefile|.

When using \textsf{childdoc} in the main file, the following
command lines effectively perform a redirection
(note that depending on the shell being used,
backslashes may have to be doubled: `|\|' $\to$ `|\\|'):
%
\begin{center}
|... -jobname "|\textit{target}|" |\\|"|[\textit{flags}]%
|\input{childdoc.def}\childdocforward[|\textit{main}|]{|\textit{dest}|}"|
\end{center}
%
Here \textit{target} is the name of the output file,
\textit{main} is the name of the main file
and \textit{dest} is the name of the main or child file to be processed
(all filenames without extensions).
The optional argument \textit{main} can be omitted
if \textit{main} matches \textit{dest}.
Optionally, compilation \textit{flags} can be defined via |\def| commands.
This command line makes the \TeX{} engine believe
it is compiling the file \textit{target}
whose content is specified as the latter parameter.
The provided code then forwards the processing to
\textit{main} or \textit{dest} as described in \secref{sec:forward}.

%%%%%%%%%%%%%%%%%%%%%%%%%%%%%%%%%%%%%%%%%%%%%%%%%%%%%%%%%%%%%%%%%%%%%%%%%%%%%%%%
\subsection{Include by Input}
\label{sec:input}

Including child documents by |\include| has some restrictions by design.
Most notably, the content of a child document always occupies
its own set of pages; pages cannot be shared between child documents.
Usually, this behaviour makes perfect sense
because each child document contain an essential part of the document.
However, in some situations it may be desirable to compose
a document from a collection of parts
without having mandatory page breaks between then.
For this case, the package
provides a mechanism to include parts
by |\input| which can also be processed individually.
However, by construction this mechanism
requires manual handling of the content to be output.

%%%%%%%%%%%%%%%%%%%%%%%%%%%%%%%%%%%%%%%%
\DescribeMacro{\ifchilddocmanual}
The main file should be prepared as usual, see \secref{sec:include}.
However, the document body must make a distinction
between processing of an individual part and of the main document, e.g.:
%
\begin{center}
\begin{tabular}{l}
|\ifchilddocmanual|\\
|\input{\childdocname}|\\
|\||else|\\
\textit{document body with }|\input{|\textit{part}|}|\\
|\||fi|
\end{tabular}
\end{center}
%
The conditional |\ifchilddocmanual| is true whenever
a part to be included by |\input| is being compiled,
and the name of the part is stored in |\childdocname|.

%%%%%%%%%%%%%%%%%%%%%%%%%%%%%%%%%%%%%%%%
\DescribeMacro{\childdocby}
Each part to be included by |\input| should start with:
%
\begin{center}
\begin{tabular}{l}
|\input{childdoc.def}|\\
|\childdocby{|\textit{main}|}|\\
\end{tabular}
\end{center}
%
The directive |\childdocby| is similar to |\childdocof|
described in \secref{sec:include},
but the subsequent selection of content must be done manually.
To that end, both |\ifchilddoc| and |\ifchilddocmanual|
will be true upon processing of a part,
and the name of the part is stored in |\childdocname|.
Note that |\jobname| will be set to the filename of the current part
so that each part receives an individual |.aux| file
that does not interfere with the |.aux| file(s) of the main document.
This behaviour can be altered by the alternative form
|\childdocby[*]{|\textit{main}|}| (with a non-empty optional argument)
which uses the |.aux| file of the main document
by setting |\jobname| to \textit{main}.

%%%%%%%%%%%%%%%%%%%%%%%%%%%%%%%%%%%%%%%%%%%%%%%%%%%%%%%%%%%%%%%%%%%%%%%%%%%%%%%%
\subsection{Driver Development}
\label{sec:driver}

The \textsf{childdoc} mechanism can also be use for the development
of definition files such as \LaTeX{} styles or classes.
This case differs from the above setup with multiple parts
included by |\include| in that no |\includeonly| should be invoked.
This can be achieved by starting the include file
(before |\ProvidesPackage|) with:
%
\begin{center}
\begin{tabular}{l}
|\input{childdoc.def}|\\
|\childdocforward{|\textit{main}|}|\\
\end{tabular}
\end{center}
%
or alternatively with:
%
\begin{center}
\begin{tabular}{l}
|\input{childdoc.def}|\\
|\childdocby{|\textit{main}|}|\\
\end{tabular}
\end{center}
%
Both forms have slightly different effects as described above.
The main file is prepared as usual, see \secref{sec:include}.

%%%%%%%%%%%%%%%%%%%%%%%%%%%%%%%%%%%%%%%%%%%%%%%%%%%%%%%%%%%%%%%%%%%%%%%%%%%%%%%%
\subsection{Legacy Detection}
\label{sec:detection}

The directive |\childdocmain| in the main file can detect
whether the complete document or merely a child is to be compiled
even without using the directive |\childdocof|.
This method is deprecated because it is less robust
and there is no compelling reason to use it;
it is merely provided for backward compatibility
and it may be removed in future versions.

If the detection mechanism is to be used,
it is mandatory to correctly specify
the filename of the main file as the argument of |\childdocmain|:
%
\begin{center}
\begin{tabular}{l}
|\input{childdoc.def}|\\
|\childdocmain{|\textit{main}|}|\\
\end{tabular}
\end{center}
%
If |\jobname| does not match the argument \textit{main} of |\childdocmain|,
it is assumed that |\jobname| points to the child file to be compiled.
When using |\childdocmain| with the main file specified as argument,
it suffices to start a child file
with just |\input{|\textit{main}|}|
without loading of the package and using |\childdocof|.
If instead all processing is done
with the appropriate \textsf{childdoc} directives,
the argument of \textit{main} of |\childdocmain| can be empty.

An alternative version of the command line processing described
in \secref{sec:commandline} using the detection mechanism reads:
%
\begin{center}
|... -jobname "|\textit{target}|" "|[\textit{flags}]%
[|\def\jobname{|\textit{dest}|}|]|\input{|\textit{main}|}"|
\end{center}

%%%%%%%%%%%%%%%%%%%%%%%%%%%%%%%%%%%%%%%%%%%%%%%%%%%%%%%%%%%%%%%%%%%%%%%%%%%%%%%%
\subsection{Manual Code}
\label{sec:manual}

In case one cannot be certain whether the definitions file |childdoc.def|
is installed on the target \TeX{} distribution
and one prefers not to ship it,
it is conceivable to paste a few relevant commands into the sources.

To that end, drop all statements |\input{childdoc.def}|
and perform the replacements as outlined below.
Instead of |\childdocmain{|\textit{main}|}| add the following code
to the top of the main file:
%
\begin{center}
\begin{tabular}{l}
|\||ifdefined\childdocname\endinput\||fi\newif\ifchilddoc|\\
|\edef\childdocname{\scantokens\expandafter{\jobname\noexpand}}|\\
|\def\childdocmain{|\textit{main}|}\||ifx\childdocmain\childdocname\||else|\\
|\childdoctrue\includeonly{\childdocname}\let\jobname\childdocmain\||fi|\\
\end{tabular}
\end{center}
%
Instead of |\childdocof{|\textit{main}|}| just include the main file
at the top of each child file:
%
\begin{center}
|\input{|\textit{main}|}|
\end{center}
%
A simple redirection |\childdocforward{|\textit{dest}|}| is achieved by:
%
\begin{center}
|\def\jobname{|\textit{dest}|}\input{\jobname}|
\end{center}
%
The redirection with prefix
|\childdocforwardprefix[|\textit{prefix}|]{|\textit{dest}|}|
is accomplished by:
%
\begin{center}
\begin{tabular}{l}
|{\edef\jobname{\scantokens\expandafter{\jobname\noexpand}}|\\
|\def\redirectjob |\textit{prefix}|#1~~~{\gdef\jobname{|\textit{dest}|#1}}|\\
|\expandafter\redirectjob\jobname~~~}\input{\jobname}|
\end{tabular}
\end{center}

In an alternative approach,
child documents can be compiled by a specific command line
without additional code or specific definitions:
%
\begin{center}
|... -jobname "|\textit{target}|" "|[\textit{flags}]%
|\includeonly{|\textit{dest}|}\input{|\textit{main}|}"|
\end{center}
%

%%%%%%%%%%%%%%%%%%%%%%%%%%%%%%%%%%%%%%%%%%%%%%%%%%%%%%%%%%%%%%%%%%%%%%%%%%%%%%%%
%%%%%%%%%%%%%%%%%%%%%%%%%%%%%%%%%%%%%%%%%%%%%%%%%%%%%%%%%%%%%%%%%%%%%%%%%%%%%%%%
\section{Information}

%%%%%%%%%%%%%%%%%%%%%%%%%%%%%%%%%%%%%%%%%%%%%%%%%%%%%%%%%%%%%%%%%%%%%%%%%%%%%%%%
\subsection{Copyright}

Copyright \copyright{} 2017--2018 Niklas Beisert

This work may be distributed and/or modified under the
conditions of the \LaTeX{} Project Public License, either version 1.3
of this license or (at your option) any later version.
The latest version of this license is in
  \url{http://www.latex-project.org/lppl.txt}
and version 1.3 or later is part of all distributions of \LaTeX{}
version 2005/12/01 or later.

This work has the LPPL maintenance status `maintained'.

The Current Maintainer of this work is Niklas Beisert.

This work consists of the files |README.txt|, |childdoc.ins| and |childdoc.dtx|
as well as the derived files |childdoc.def|, |cdocsamp.tex|
with |cdocsch1.tex|, |cdocsch2.tex|, |cdocspt3.tex|, |cdocspt4.tex|,
|cdocsdrf.tex|, |cdocsfn1.tex|, |cdocsfn2.tex|
as well as |childdoc.pdf|.

%%%%%%%%%%%%%%%%%%%%%%%%%%%%%%%%%%%%%%%%%%%%%%%%%%%%%%%%%%%%%%%%%%%%%%%%%%%%%%%%
\subsection{Files and Installation}

The package consists of the files:
%
\begin{center}
\begin{tabular}{ll}
    |README.txt|   & readme file \\
    |childdoc.ins| & installation file \\
    |childdoc.dtx| & source file \\
    |childdoc.def| & definition file \\
    |cdocsamp.tex| & sample main file \\
    |cdocsch1.tex| & sample include file \\
    |cdocsch2.tex| & sample include file \\
    |cdocspt3.tex| & sample part file \\
    |cdocspt4.tex| & sample part file \\
    |cdocsdrf.tex| & sample redirection file \\
    |cdocsfn1.tex| & sample redirection file \\
    |cdocsfn2.tex| & sample redirection file \\
    |childdoc.pdf| & manual
\end{tabular}
\end{center}
%
The distribution consists of the files
|README.txt|, |childdoc.ins| and |childdoc.dtx|.
%
\begin{itemize}
\item
Run (pdf)\LaTeX{} on |childdoc.dtx|
to compile the manual |childdoc.pdf| (this file).
\item
Run \LaTeX{} on |childdoc.ins| to create the definitions file |childdoc.def|
and the sample |cdocsamp.tex| with include files
|cdocsch1.tex|, |cdocsch2.tex|, |cdocspt3.tex|, |cdocspt4.tex|,
|cdocsdrf.tex|, |cdocsfn1.tex|, |cdocsfn2.tex|.
Then copy the file |childdoc.def| to an appropriate directory of your \LaTeX{}
distribution, e.g.\ \textit{texmf-root}|/tex/latex/childdoc|.
\end{itemize}

%%%%%%%%%%%%%%%%%%%%%%%%%%%%%%%%%%%%%%%%%%%%%%%%%%%%%%%%%%%%%%%%%%%%%%%%%%%%%%%%
\subsection{Related CTAN Packages}

There are several other packages which offer a similar functionality:
%
\begin{itemize}
\item
The packages
\href{http://ctan.org/pkg/docmute}{\textsf{docmute}},
\href{http://ctan.org/pkg/includex}{\textsf{includex}} and
\href{http://ctan.org/pkg/standalone}{\textsf{standalone}}
provide commands to include only the document body of
a child file thus allowing both files to be compiled individually.
\item
The packages \href{http://ctan.org/pkg/subdocs}{\textsf{subdocs}}
and \href{http://ctan.org/pkg/subfiles}{\textsf{subfiles}}
provide structures in which the main and child documents can be
encapsulated and allowing them to be compiled individually.
The inclusion mechanism is different from the conventional |\include|.
\item
The package \href{http://ctan.org/pkg/combine}{\textsf{combine}}
is an elaborate solution to combine several documents into one.
\end{itemize}
%
See also the CTAN topic \href{http://ctan.org/topic/subdocs}{\textsf{subdocs}}
for further related packages.
The present package differs from the above solutions in that
a document structure constructed with the conventional |\include| mechanism
just needs two extra commands at the top of every file
such that all constituent files can be compiled individually.

%%%%%%%%%%%%%%%%%%%%%%%%%%%%%%%%%%%%%%%%%%%%%%%%%%%%%%%%%%%%%%%%%%%%%%%%%%%%%%%%
%\subsection{Feature Suggestions}
%
%The following is a list of features which may be useful for future
%versions of this package:
%%
%\begin{itemize}
%\item
%\ldots
%\end{itemize}

%%%%%%%%%%%%%%%%%%%%%%%%%%%%%%%%%%%%%%%%%%%%%%%%%%%%%%%%%%%%%%%%%%%%%%%%%%%%%%%%
\subsection{Revision History}

%%%%%%%%%%%%%%%%%%%%%%%%%%%%%%%%%%%%%%%%
\paragraph{v2.0:} 2018/12/30

\begin{itemize}
\item
immediate forward processing
\item
added |\childdocby| mechanism
\item
manual restructured
\end{itemize}

%%%%%%%%%%%%%%%%%%%%%%%%%%%%%%%%%%%%%%%%
\paragraph{v1.6:} 2018/01/17

\begin{itemize}
\item
application for development of include files
\item
corrections to manual
\end{itemize}

%%%%%%%%%%%%%%%%%%%%%%%%%%%%%%%%%%%%%%%%
\paragraph{v1.5:} 2017/05/21

\begin{itemize}
\item
more complete structuring introduced
\item
|\childdocof| introduced
\item
|\childdoc| renamed to |\childdocmain|
\item
|\childredirect| renamed to |\childdocforward| and |\childdocforwardprefix|
and functionality expanded
\end{itemize}

%%%%%%%%%%%%%%%%%%%%%%%%%%%%%%%%%%%%%%%%
\paragraph{v1.0:} 2017/04/27

\begin{itemize}
\item
manual and install package
\item
first version published on CTAN
\end{itemize}

%%%%%%%%%%%%%%%%%%%%%%%%%%%%%%%%%%%%%%%%
\paragraph{v0.6:} 2017/04/26

\begin{itemize}
\item
redirection mechanism added
\end{itemize}

%%%%%%%%%%%%%%%%%%%%%%%%%%%%%%%%%%%%%%%%
\paragraph{v0.5:} 2017/04/26

\begin{itemize}
\item
functionality in definition file
\end{itemize}


%%%%%%%%%%%%%%%%%%%%%%%%%%%%%%%%%%%%%%%%%%%%%%%%%%%%%%%%%%%%%%%%%%%%%%%%%%%%%%%%
%%%%%%%%%%%%%%%%%%%%%%%%%%%%%%%%%%%%%%%%%%%%%%%%%%%%%%%%%%%%%%%%%%%%%%%%%%%%%%%%
%%%%%%%%%%%%%%%%%%%%%%%%%%%%%%%%%%%%%%%%%%%%%%%%%%%%%%%%%%%%%%%%%%%%%%%%%%%%%%%%
\appendix

\settowidth\MacroIndent{\rmfamily\scriptsize 000\ }

 \DocInput{childdoc.dtx}

\end{document}
%</driver>
% \fi
%
% %%%%%%%%%%%%%%%%%%%%%%%%%%%%%%%%%%%%%%%%%%%%%%%%%%%%%%%%%%%%%%%%%%%%%%%%%%%%%%
% %%%%%%%%%%%%%%%%%%%%%%%%%%%%%%%%%%%%%%%%%%%%%%%%%%%%%%%%%%%%%%%%%%%%%%%%%%%%%%
% \section{Sample}
%\iffalse
%<*samplemain>
%\fi
%
% The following presents a sample document
% with two chapters, two parts, a title page,
% a compile flag as well as three forwarding files to set the flag.
% It consists of eight |.tex| files:
% \begin{center}
% \begin{tabular}{ll}
% |cdocsamp.tex|&main file\\
% |cdocsch1.tex|&include file for chapter 1\\
% |cdocsch2.tex|&include file for chapter 2\\
% |cdocspt3.tex|&include file for part 3\\
% |cdocspt4.tex|&include file for part 4\\
% |cdocsdrf.tex|&forwarding file for main file in draft mode\\
% |cdocsfi1.tex|&forwarding file for final version of chapter 1\\
% |cdocsfi2.tex|&forwarding file for final version of chapter 2\\
% \end{tabular}
% \end{center}
% Each of the eight files can be compiled directly by the \LaTeX{} compiler.
%
% %%%%%%%%%%%%%%%%%%%%%%%%%%%%%%%%%%%%%%
% \paragraph{Main File.}
%
% The main file is called |cdocsamp.tex|.
%
% Load the \textsf{childdoc} definitions and
% declare the filename for the main document:
%    \begin{macrocode}
\input{childdoc.def}
\childdocmain{}
%    \end{macrocode}

% Optional override for |\version| flag:
%    \begin{macrocode}
%%\ifchilddoc\else\providecommand{\version}{draft}\fi
%    \end{macrocode}

% Define the default values for the |\version| flag
% (|final| for the main file and |draft| for childs):
%    \begin{macrocode}
\ifchilddoc
\providecommand{\version}{draft}
\else
\providecommand{\version}{final}
\fi
%    \end{macrocode}

% Load the standard document class:
%    \begin{macrocode}
\documentclass[12pt]{article}
%    \end{macrocode}

% Start the document body:
%    \begin{macrocode}
\begin{document}
%    \end{macrocode}

% Declare a title page.
% Print title, part of document being processed and version flag:
%    \begin{macrocode}
\addtocounter{page}{-1}
\begin{center}
{\LARGE\bfseries{}childdoc example\par}
\vspace{1cm}
\ifchilddoc
\ifchilddocmanual part\else chapter\fi:
`\childdocname' of `\childdocjob'\par
\else
main document: `\childdocjob'\par
\fi
version: \version\par
\end{center}
\newpage
%    \end{macrocode}

% Manually include selected file,
% otherwise process as usual:
%    \begin{macrocode}
\ifchilddocmanual
\section*{part `\childdocname'}
\input{\childdocname}
\else
%    \end{macrocode}

% Include the two chapters:
%    \begin{macrocode}
\include{cdocsch1}
\include{cdocsch2}
%    \end{macrocode}

% Include the two parts unless only chapters should be displayed:
%    \begin{macrocode}
\ifchilddoc\else
\section{part three}
\input{cdocspt3}
\section{part four}
\input{cdocspt4}
\fi
%    \end{macrocode}

% Process as usual until here:
%    \begin{macrocode}
\fi
%    \end{macrocode}

% End of document body:
%    \begin{macrocode}
\end{document}
%    \end{macrocode}
%\iffalse
%</samplemain>
%\fi
%
% %%%%%%%%%%%%%%%%%%%%%%%%%%%%%%%%%%%%%%
% \paragraph{Chapter Include Files.}
%
% The include files are called |cdocsch1.tex| and |cdocsch2.tex|.
%
%\iffalse
%<*samplechap1|samplechap2>
%\fi

% Optional override for |\version| flag:
%    \begin{macrocode}
%%\providecommand{\version}{final}
%    \end{macrocode}

% Include the main document:
%    \begin{macrocode}
\input{childdoc.def}
\childdocof{cdocsamp}
%    \end{macrocode}

%\iffalse
%</samplechap1|samplechap2>
%\fi
%
%\iffalse
%<*samplechap1>
%\fi
% Some text for chapter 1:
%    \begin{macrocode}
\section{one}
some text in chapter one
%    \end{macrocode}

%\iffalse
%</samplechap1>
%\fi
% Some text for chapter 2:
%\iffalse
%<*samplechap2>
%\fi
%    \begin{macrocode}
\section{two}
more text in chapter two
%    \end{macrocode}

%\iffalse
%</samplechap2>
%\fi
%
% %%%%%%%%%%%%%%%%%%%%%%%%%%%%%%%%%%%%%%
% \paragraph{Part Include Files.}
%
% The include files are called |cdocspt3.tex| and |cdocspt4.tex|.
%
%\iffalse
%<*samplepart3|samplepart4>
%\fi

% Optional override for |\version| flag:
%    \begin{macrocode}
%%\providecommand{\version}{final}
%    \end{macrocode}

% Include the main document:
%    \begin{macrocode}
\input{childdoc.def}
\childdocby{cdocsamp}
%    \end{macrocode}

%\iffalse
%</samplepart3|samplepart4>
%\fi
%
%\iffalse
%<*samplepart3>
%\fi
% Some text for part 3:
%    \begin{macrocode}
some text in part three
%    \end{macrocode}

%\iffalse
%</samplepart3>
%\fi
% Some text for part 4:
%\iffalse
%<*samplepart4>
%\fi
%    \begin{macrocode}
more text in part four
%    \end{macrocode}

%\iffalse
%</samplepart4>
%\fi
%
% %%%%%%%%%%%%%%%%%%%%%%%%%%%%%%%%%%%%%%
% \paragraph{Forwarding for a Complete Draft.}
%
% The following forwarding file |cdocsdrf.tex|
% compiles the main document in draft mode:
%\iffalse
%<*sampledraft>
%\fi
%    \begin{macrocode}
\def\version{draft}
\input{childdoc.def}
\childdocforward{cdocsamp}
%    \end{macrocode}

%\iffalse
%</sampledraft>
%\fi
%
% %%%%%%%%%%%%%%%%%%%%%%%%%%%%%%%%%%%%%%
% \paragraph{Forwarding for Final Version of the Chapters.}
%
% The following forwarding files |cdocsfn1.tex| and |cdocsfn2.tex|
% (with identical content)
% compile the final versions of the child documents
% |cdocsch1.tex| and |cdocsch2.tex|, respectively:
%\iffalse
%<*samplefinal>
%\fi
%    \begin{macrocode}
\def\version{final}
\input{childdoc.def}
\childdocforwardprefix[cdocsamp]{cdocsfn}{cdocsch}
%    \end{macrocode}

%\iffalse
%</samplefinal>
%\fi
%
% %%%%%%%%%%%%%%%%%%%%%%%%%%%%%%%%%%%%%%
% \paragraph{Command Line Processing.}
%
% The following three command lines generate the output files
% |cdocscld|, |cdocscl1| and |cdocscl2|
% which should be identical to
% |cdocsdrf|, |cdocsch1| and |cdocsfn2|, respectively:
% \begin{center}
% \begin{tabular}{l}
% |latex -jobname cdocscld \|\\
% |  "\def\version{draft}\input{childdoc.def}\childdocforward{cdocsamp}"|\\
% |latex -jobname cdocscl1 \|\\
% |  "\input{childdoc.def}\childdocforward[cdocsamp]{cdocsch1}"|\\
% |latex -jobname cdocscl2 \|\\
% |  "\def\version{final}\input{childdoc.def}\childdocforward{cdocsch2}"|
% \end{tabular}
% \end{center}
% Note that the trailing backslash on each first line
% merely continues the input to the second line
% (for convenient cut ant paste).
% Furthermore, the command |latex| can be replaced by any
% of its alternative versions such as |pdflatex|.
%
% %%%%%%%%%%%%%%%%%%%%%%%%%%%%%%%%%%%%%%%%%%%%%%%%%%%%%%%%%%%%%%%%%%%%%%%%%%%%%%
% %%%%%%%%%%%%%%%%%%%%%%%%%%%%%%%%%%%%%%%%%%%%%%%%%%%%%%%%%%%%%%%%%%%%%%%%%%%%%%
% \section{Implementation}
%\iffalse
%<*package>
%\fi
%
% This section describes the definitions file |childdoc.def|.

% The definitions cannot be loaded using |\usepackage| or |\RequirePackage|
% which has a mechanism to prevent loading a style file more than once.
% When loading the definitions by means of |\input|
% multiple instances have to be prevented manually:
%\iffalse
%This code needs to be before the `\ProvidesFile' directive
%which is defined at the beginning of this file.
%Therefore it is also placed there and commented out here.
%</package>
%<*discard>
%\fi
%    \begin{macrocode}
\ifdefined\childdocmain\endinput\fi
%    \end{macrocode}
%\iffalse
%</discard>
%<*package>
%\fi
%
% \macro{\ifchilddoc}
% \macro{\ifchilddocmanual}
% The conditional |\ifchilddoc| tells whether a
% child (true) or main (false) document is being compiled.
% The conditional |\ifchilddocmanual| tells whether
% the |\includeonly| mechanism is used (false) or
% the selection of child files must be performed manually (true).
% The definitions initialise to false:
%    \begin{macrocode}
\newif\ifchilddoc
\newif\ifchilddocmanual
%    \end{macrocode}

% \macro{\childdocname}
% \macro{\childdocjob}
% The macro |\childdocname| stores the name of the main document
% to be compiled. The macro |\childdocjob| stores the name of
% the document on which the \LaTeX{} compiler was originally invoked.
% The content of |\jobname| cannot be compared
% to filenames specified in the source due to different catcodes.
% The following code rescans |\jobname|, stores the result
% in |\childdocname| and saves a copy in |\childdocjob|:
%    \begin{macrocode}
\edef\childdocname{\scantokens\expandafter{\jobname\noexpand}}
\let\childdocjob\childdocname
%    \end{macrocode}

% \macro{\childdocdisable}
% The macro |\childdocdisable| prevents the main file
% from being processed more than once.
% At this stage, the main document command |\childdocmain|
% is assumed to be called once again where it should do nothing.
% Any subsequent call to it should prevent
% a secondary processing of the main document
% It overwrites the forwarding commands
% |\childdocof| and |\childdocforward|
% with empty macros to prevent further inclusions of the main document:
%    \begin{macrocode}
\newcommand{\childdocdisable}
{
  \renewcommand{\childdocmain}[1]{\renewcommand{\childdocmain}[1]{\endinput}}
  \renewcommand{\childdocof}[1]{}
  \renewcommand{\childdocby}[2][]{}
  \renewcommand{\childdocforward}[2][]{}
  \renewcommand{\childdocdisable}{}
}
%    \end{macrocode}

% \macro{\childdocmain}
% The macro |\childdocmain| is to be called at the top of the main file
% with nothing or the main filename (without extension) as argument.
% First, it breaks loops.
% If the argument is not empty and does not match |\childdocname|
% (which is set by the first inclusion of |childdoc.def|),
% |\ifchilddoc| is set to true, |\includeonly| is applied to the child file
% and |\jobname| is set to the main file
% (for proper handling of |.aux| files):
%    \begin{macrocode}
\newcommand{\childdocmain}[1]
{
  \childdocdisable\childdocmain{}
  \if?#1?\else
    \begingroup
      \def\childdoctmp{#1}
      \ifx\childdoctmp\childdocname
        \def\childdoctmp{}
      \else
        \def\childdoctmp
        {
          \childdoctrue
          \includeonly{\childdocname}
          \def\childdocjob{#1}
          \def\jobname{#1}
        }
      \fi
      \expandafter
    \endgroup
    \childdoctmp
  \fi
}
%    \end{macrocode}

% \macro{\childdocof}
% The command |\childdocof| redirects
% compilation to the main file |#1|.
%    \begin{macrocode}
\newcommand{\childdocof}[1]
{
  \childdocdisable
  \childdoctrue
  \includeonly{\childdocname}
  \def\jobname{#1}
  \def\childdocjob{#1}
  \input{#1}
}
%    \end{macrocode}

% \macro{\childdocby}
% The command |\childdocby| ....
%    \begin{macrocode}
\newcommand{\childdocby}[2][]
{
  \childdocdisable
  \childdoctrue
  \childdocmanualtrue
  \if?#1?\else
    \def\jobname{#2}
  \fi
  \def\childdocjob{#2}
  \input{#2}
  \endinput
}
%    \end{macrocode}

% \macro{\childdocforward}
% The command |\childdocforward| redirects
% compilation to the main file or
% (if the optional argument is given) a child file.
% Parameters are set as if the main file
% or a child file starting with |\childdocof| was compiled.
% Then compilation is handed over to the main file:
%    \begin{macrocode}
\newcommand{\childdocforward}[2][]
{
  \begingroup
    \if?#1?
      \def\childdoctmp
      {
        \def\childdocname{#2}
        \def\childdocjob{#2}
        \def\jobname{#2}
        \input{#2}
        \endinput
      }
    \else
      \def\childdoctmp
      {
        \childdocdisable
        \def\childdocname{#2}
        \childdoctrue
        \includeonly{#2}
        \def\childdocjob{#1}
        \def\jobname{#1}
        \input{#1}
        \endinput
      }
    \fi
    \expandafter
  \endgroup
  \childdoctmp
}
%    \end{macrocode}

% \macro{\childdocforwardprefix}
% The command |\childdocforwardprefix| redirects
% compilation to the main or a child file by means of a pattern.
% The prefix |#1| in the current filename is replaced by |#2|
% and the suffix of the current filename is kept
% (it is assumed that the filename does not contain the substring `|~~~|'
% which is used as a delimiter).
% Compilation is handed over to the new file by |\childdocforward|:
%    \begin{macrocode}
\newcommand{\childdocforwardprefix}[3][]
{
  \begingroup
    \def\childdocextract #2##1~~~{\def\childdoctmp{\childdocforward[#1]{#3##1}}}
    \expandafter\childdocextract\childdocname~~~
    \expandafter
  \endgroup
  \childdoctmp
}
%    \end{macrocode}

% \macro{\childdoc}
% The deprecated macro |\childdoc| is a legacy version of |\childdocmain|:
%    \begin{macrocode}
\newcommand{\childdoc}{\childdocmain}
%    \end{macrocode}

% \macro{\childdocredirect}
% The deprecated macro |\childdocredirect| is a legacy version
% of |\childdocforward| and |\childdocforwardprefix|:
%    \begin{macrocode}
\newcommand{\childdocredirect}[2][]
{
  \begingroup
    \if?#1?
      \def\childdoctmp{\childdocforward{#2}}
    \else
      \def\childdoctmp{\childdocforwardprefix{#1}{#2}}
    \fi
    \expandafter
  \endgroup
  \childdoctmp
}
%    \end{macrocode}

%\iffalse
%</package>
%\fi
%
\endinput
|\\
|\childdocforward{|\textit{main}|}|\\
\end{tabular}
\end{center}
%
or alternatively with:
%
\begin{center}
\begin{tabular}{l}
|% \iffalse
%
% childdoc.dtx Copyright (C) 2017-2018 Niklas Beisert
%
% This work may be distributed and/or modified under the
% conditions of the LaTeX Project Public License, either version 1.3
% of this license or (at your option) any later version.
% The latest version of this license is in
%   http://www.latex-project.org/lppl.txt
% and version 1.3 or later is part of all distributions of LaTeX
% version 2005/12/01 or later.
%
% This work has the LPPL maintenance status `maintained'.
%
% The Current Maintainer of this work is Niklas Beisert.
%
% This work consists of the files childdoc.dtx and childdoc.ins
% and the derived files childdoc.def and cdocsamp.tex with
% cdocsch1.tex, cdocsch2.tex, cdocsdrf.tex, cdocsfn1.tex, cdocsfn2.tex.
%
%<package>\ifdefined\childdocmain\endinput\fi
%<package>\ProvidesFile{childdoc.def}[2018/12/30 v2.0 child document driver]
%<samplemain>\ProvidesFile{cdocsamp.tex}[2018/12/30 v2.0 sample for childdoc]
%<*driver>
%\ProvidesFile{childdoc.drv}[2018/12/30 v2.0 childdoc reference manual file]
\PassOptionsToClass{10pt,a4paper}{article}
\documentclass{ltxdoc}

\usepackage[margin=35mm]{geometry}
\usepackage{hyperref}
\usepackage{hyperxmp}
\usepackage[usenames]{color}

\hypersetup{colorlinks=true}
\hypersetup{pdfstartview=FitH}
\hypersetup{pdfpagemode=UseNone}
\hypersetup{pdfsource={}}
\hypersetup{pdflang={en-UK}}
\hypersetup{pdfcopyright={Copyright 2017-2018 Niklas Beisert.
  This work may be distributed and/or modified under the
  conditions of the LaTeX Project Public License, either version 1.3
  of this license or (at your option) any later version.}}
\hypersetup{pdflicenseurl={http://www.latex-project.org/lppl.txt}}
\hypersetup{pdfcontactaddress={ETH Zurich, ITP, HIT K,
  Wolfgang-Pauli-Strasse 27}}
\hypersetup{pdfcontactpostcode={8093}}
\hypersetup{pdfcontactcity={Zurich}}
\hypersetup{pdfcontactcountry={Switzerland}}
\hypersetup{pdfcontactemail={nbeisert@itp.phys.ethz.ch}}
\hypersetup{pdfcontacturl={http://people.phys.ethz.ch/\xmptilde nbeisert/}}

\newcommand{\secref}[1]{\hyperref[#1]{section \ref*{#1}}}

\parskip1ex
\parindent0pt
\let\olditemize\itemize
\def\itemize{\olditemize\parskip0pt}

\begin{document}

\title{The \textsf{childdoc} Package}
\hypersetup{pdftitle={The childdoc Package}}
\author{Niklas Beisert\\[2ex]
  Institut f\"ur Theoretische Physik\\
  Eidgen\"ossische Technische Hochschule Z\"urich\\
  Wolfgang-Pauli-Strasse 27, 8093 Z\"urich, Switzerland\\[1ex]
  \href{mailto:nbeisert@itp.phys.ethz.ch}
  {\texttt{nbeisert@itp.phys.ethz.ch}}}
\hypersetup{pdfauthor={Niklas Beisert}}
\hypersetup{pdfsubject={Manual for the LaTeX2e Package childdoc}}
\date{30 December 2018, \textsf{v2.0}}
\maketitle

\begin{abstract}\noindent
\textsf{childdoc} is a \LaTeXe{} package
that enables the direct compilation
of document sections included by |\include|
to individual files.
\end{abstract}

\begingroup
\parskip0ex
\tableofcontents
\endgroup

%%%%%%%%%%%%%%%%%%%%%%%%%%%%%%%%%%%%%%%%%%%%%%%%%%%%%%%%%%%%%%%%%%%%%%%%%%%%%%%%
%%%%%%%%%%%%%%%%%%%%%%%%%%%%%%%%%%%%%%%%%%%%%%%%%%%%%%%%%%%%%%%%%%%%%%%%%%%%%%%%
\section{Introduction}

\LaTeX{} provides a mechanism to structure a large document (such as a book)
into a main file and several child files (containing the chapters)
using the |\include| command.
This mechanism is beneficial for documents
which span hundreds of pages in order to
make the source file(s) more manageable.
Moreover, compilation can be restricted to
selected child files by means of the |\includeonly| command.
The latter feature can be used to reduce the compilation time while editing
(this was significantly more useful in the earlier days of \LaTeX{})
or to generate a smaller document which is easier to navigate.
Another application of |\includeonly| is to generate
documents consisting of selected parts of the complete document.

However, there are a few drawbacks of the plain |\include| mechanism:
\begin{itemize}
\item
The child files cannot be compiled on their own,
they can only be compiled via the main file.
A naive editing environment
(such as a text editor with an option
to have the current file processed by \LaTeX)
may require one to switch to the main file before compiling;
attempting to compile the child file produces errors.
\item
The main file must be modified (each time)
to adjust the |\includeonly| command
to the present needs. This easily leaves the main file in a messy state.
\item
The generated document will always carry the filename
of the main document. This is inconvenient if
several child files are to be compiled and
to be kept for distribution.
\end{itemize}

The present package provides a simple interface
to make child files individually compilable by \LaTeX{}.
Compiling a child file then has the same effect as compiling
the main file with an |\includeonly| command
to select the appropriate child.
Moreover the generated document will carry the name of the child
rather than the main file.
This resolves all three above issues.

This feature is meant to make the editing of books,
thesis documents and lecture notes somewhat more convenient.
However, the package can also be used efficiently for
composing a series of documents (such as exercise sheets)
which are typically distributed individually.
It then assists the author in generating the individual documents
(potentially in different versions)
as well as a document containing the collected series.
Another application is in developing style files
or other kinds of included material
where compilation of the style file could redirect
to a sample or test file.

%%%%%%%%%%%%%%%%%%%%%%%%%%%%%%%%%%%%%%%%%%%%%%%%%%%%%%%%%%%%%%%%%%%%%%%%%%%%%%%%
%%%%%%%%%%%%%%%%%%%%%%%%%%%%%%%%%%%%%%%%%%%%%%%%%%%%%%%%%%%%%%%%%%%%%%%%%%%%%%%%
\section{Usage}

First of all, the package \textsf{childdoc} is \emph{not} a standard
\LaTeXe{} |.sty| style file! Therefore it needs to be invoked in
a non-standard way.

%%%%%%%%%%%%%%%%%%%%%%%%%%%%%%%%%%%%%%%%%%%%%%%%%%%%%%%%%%%%%%%%%%%%%%%%%%%%%%%%
\subsection{Included Files}
\label{sec:include}

%%%%%%%%%%%%%%%%%%%%%%%%%%%%%%%%%%%%%%%%
\DescribeMacro{\childdocmain}
To use the package, add the commands
\begin{center}
\begin{tabular}{l}
|\input{childdoc.def}|\\
|\childdocmain{}|\\
\end{tabular}
\end{center}
at the very top of the main \LaTeX{} file,
in particular \emph{before} the |\documentclass| statement!
The argument of |\childdocmain| should be left empty
(but it must be present).

%%%%%%%%%%%%%%%%%%%%%%%%%%%%%%%%%%%%%%%%
\DescribeMacro{\childdocof}
Furthermore, add the commands
\begin{center}
\begin{tabular}{l}
|\input{childdoc.def}|\\
|\childdocof{|\textit{main}|}|\\
\end{tabular}
\end{center}
at the top of every child file \textit{child}
which is included by |\include{|\textit{child}|}|
from within the main file
(or at least for those files to be compiled individually).
The argument \textit{main} must be the filename of the main file.

There are a couple of
considerations in setting up the main and child documents:

%%%%%%%%%%%%%%%%%%%%%%%%%%%%%%%%%%%%%%%%
\paragraph{Restrictions.}

Please note the following restrictions:
\begin{itemize}
\item
|\childdocmain| must be called with one argument \textit{main}
to ensure compatibility with earlier version of the package.
It must either be empty (|\childdocmain{}|)
or precisely match the filename of the main file in which it is specified.
See \secref{sec:detection} for further information.
\item
The filename \textit{main} must be specified without the |.tex| extension.
\item
The filename \textit{main} is case sensitive
(even in case-insensitive file systems)
due to internal string comparison.
\item
The argument \textit{main} should be fully expanded, it cannot be a macro.
\item
Subdirectories and special characters should be avoided in filenames.
\item
The command |\childdocmain{|\textit{main}|}| must be followed by a whitespace.
It should not be followed immediately by another command
or by a comment mark `|%|'.
This is because the \TeX{} parser reads the token immediately following
the argument of |\childdocmain| and puts it
at the beginning of every child section;
however, a white\-space is ignored.
\end{itemize}

%%%%%%%%%%%%%%%%%%%%%%%%%%%%%%%%%%%%%%%%
\paragraph{Content of Main File.}

It is advisable to place all content in the child files included by |\include|.
Any output contained in the main file will appear in all child documents
unless suppressed manually;
it cannot be suppressed automatically by the |\includeonly| directive
and thus should normally be avoided.
A method to include some content in the main file
by means of conditional processing is described in \secref{sec:conditional}.

%%%%%%%%%%%%%%%%%%%%%%%%%%%%%%%%%%%%%%%%
\paragraph{Page Numbering.}

When only a part of the document is compiled,
the appropriate numbering of pages
(as well as other status parameters)
is determined from the |.aux| files.
The latter contain information from previous passes.
However this information needs to propagate through
all intermediate child documents.
Therefore the page numbering in child documents may well
be inconsistent until the complete document is compiled at least once.

A useful (if unconventional) way to always ensure a consistent
page numbering is to restart the numbering in each child document
and denote the pages by `\textit{child}|.|\textit{page}'
where \textit{child} represents the chapter/section number of the child file.
This can be achieved by the command
|\numberwithin{page}{|\textit{child}|}|
of the \textsf{amsmath} package
where \textit{child} can be |chapter| or |section|
depending on the chosen structuring.
Alternatively, one can modify the macro |\thepage| appropriately
and reset the counter |page| at the start of each child file.

%%%%%%%%%%%%%%%%%%%%%%%%%%%%%%%%%%%%%%%%%%%%%%%%%%%%%%%%%%%%%%%%%%%%%%%%%%%%%%%%
\subsection{Conditional Processing}
\label{sec:conditional}

The package provides a mechanism to compile different versions
of a document. To customise the versions further some conditional processing
can come in handy to distinguish which version is being compiled.
The package provides two macros to describe the compilation context:

%%%%%%%%%%%%%%%%%%%%%%%%%%%%%%%%%%%%%%%%
\DescribeMacro{\ifchilddoc}
The conditional |\ifchilddoc| distinguishes between the compilation of
child documents and the main document:
%
\begin{center}
|\ifchilddoc |\textit{child-code}| |[|\||else |\textit{main-code}]| \||fi|
\end{center}

%%%%%%%%%%%%%%%%%%%%%%%%%%%%%%%%%%%%%%%%
\DescribeMacro{\childdocname}
\DescribeMacro{\childdocjob}
The macro |\childdocname| contains the filename (without extension)
of the main or child file being processed.
Note that |\childdocjob| will always contain the name of the main file.

%%%%%%%%%%%%%%%%%%%%%%%%%%%%%%%%%%%%%%%%
\paragraph{Title Page.}

Conditional processing can be used to include a title or banner page
in the main document when proper precautions are taken.
Importantly, the code in the main file should ensure that the page counter
(as well as other status parameters which are stored in the |.aux| files)
takes the same value after the conditional processing.
Otherwise the page numbers may take divergent values
depending on which part is compiled.

For example, a title page could be declared by:
%
\begin{center}
\begin{tabular}{l}
|\ifchilddoc\||else|\\
|\addtocounter{page}{-1}|\\
\textit{code for title page}\\
|\newpage|\\
|\||fi|
\end{tabular}
\end{center}
%
A banner page for the child documents can be generated by:
%
\begin{center}
\begin{tabular}{l}
|\ifchilddoc|\\
|\addtocounter{page}{-1}|\\
\textit{code for banner page}\\
|\newpage|\\
|\||fi|
\end{tabular}
\end{center}
%
Here one could write a message such as:
\begin{center}
|This is the part \childdocname{} of \childdocjob{}.|
\end{center}

%%%%%%%%%%%%%%%%%%%%%%%%%%%%%%%%%%%%%%%%%%%%%%%%%%%%%%%%%%%%%%%%%%%%%%%%%%%%%%%%
\subsection{Flags}
\label{sec:flags}

The package makes it easy to generate different versions
of the main or child documents.
To this end compilation flags can be defined
and assigned different default values.
They will be particularly useful in conjunction
with the forwarding mechanism described in \secref{sec:forward}.

For example, it may be useful to have a flag |\version|
which can be set to |draft| or |final|.
The document source will contain some conditional code
depending on the value of |\version|.
Suppose further, the flag should default to |final| for the main file
and to |draft| for child files
which is a natural assignment for editing the document.
This is achieved by placing the following code
in the preamble of the main document
(below the |\childdocmain| directive):
%
\begin{center}
\begin{tabular}{l}
|\ifchilddoc|\\
|\providecommand{\version}{draft}|\\
|\||else|\\
|\providecommand{\version}{final}|\\
|\||fi|
\end{tabular}
\end{center}
%
The definition by |\providecommand| makes sure
that previous definitions are not overwritten.
Further statements |\providecommand{\version}{...}|
can thus be added before the above code to override it.

For the main file, one might add a line
(between |\childdocmain| and the above block)
%
\begin{center}
|%\ifchilddoc\||else\providecommand{\version}{draft}\||fi|
\end{center}
%
which can be uncommented to produce a draft version.
Likewise one can add a line to the very top of a child file
(above the |\childdocof{|\textit{main}|}| directive)
%
\begin{center}
|%\providecommand{\version}{final}|
\end{center}
%
which can be uncommented to produce the final version of this child document.

%%%%%%%%%%%%%%%%%%%%%%%%%%%%%%%%%%%%%%%%%%%%%%%%%%%%%%%%%%%%%%%%%%%%%%%%%%%%%%%%
\subsection{Forwarding}
\label{sec:forward}

Different versions of the main or child documents
using compilation flags as described in \secref{sec:flags}
can be (permanently) stored in different files
for convenient compilation, viewing and distribution.
To this end, the package defines a command
to pass on compilation to a different file:

%%%%%%%%%%%%%%%%%%%%%%%%%%%%%%%%%%%%%%%%
\DescribeMacro{\childdocforward}
The command |\childdocforward| redirects processing to
another source file:
%
\begin{center}
\begin{tabular}{l}
|\input{childdoc.def}|\\
|\childdocforward[|\textit{main}|]{|\textit{dest}|}|\\
\end{tabular}
\end{center}
%
The argument \textit{dest} is the destination file
(without extension).
It should be the main file or one of the child files.
Note that further \textsf{childdoc} directives
such as |\childdocof| and |\childdocforward|
in the indicated file will be processed in this form.
The optional argument \textit{main}
passes on directly to the main file \textit{main}
while pretending to compile the child \textit{dest}.
This form behaves as if \textit{dest}
issues |\childdocof{|\textit{main}|}| right away,
and no further \textsf{childdoc} directives will be processed.

%%%%%%%%%%%%%%%%%%%%%%%%%%%%%%%%%%%%%%%%
\DescribeMacro{\...prefix}
In the alternative form |\childdocforwardprefix|,
%
\begin{center}
\begin{tabular}{l}
|\input{childdoc.def}|\\
|\childdocforwardprefix[|\textit{main}|]{|\textit{prefix}|}{|\textit{dest}|}|
\end{tabular}
\end{center}
%
the destination file is determined by a pattern
depending on the current file:
To make this work, the current file must be called
`{\textit{prefix}\hspace{0.2em}\textit{suffix}}'
with \textit{prefix} matching precisely the argument.
Processing is then passed on to the file
`{\textit{dest}\hspace{0.2em}\textit{suffix}}'.
Surely, the same effect is achieved by
directly specifying the
argument `{\textit{dest}\hspace{0.2em}\textit{suffix}}'
in the first form.
However, that requires to set up a different file
for each child. With the alternative form of the command
all these files can have exactly the same content
which simplifies setting them up and maintaining them.

For example, the following file |draft.tex|
with a compilation flag |\version| as described in \secref{sec:flags}
compiles the main document as a draft:
%
\begin{center}
\begin{tabular}{l}
|\def\version{draft}|\\
|\input{childdoc.def}|\\
|\childdocforward{|\textit{main}|}|
\end{tabular}
\end{center}
%
Likewise, the following files |final|\textit{nn}|.tex|
compile the final version of the child document
|child|\textit{nn}|.tex|:
%
\begin{center}
\begin{tabular}{l}
|\def\version{final}|\\
|\input{childdoc.def}|\\
|\childdocforwardprefix{final}{child}|
\end{tabular}
\end{center}
%

Note that when several versions of a main file and/or of each child file
are to be generated, it may be convenient to set up a |Makefile| or
shell script to automatise the process.

%%%%%%%%%%%%%%%%%%%%%%%%%%%%%%%%%%%%%%%%%%%%%%%%%%%%%%%%%%%%%%%%%%%%%%%%%%%%%%%%
\subsection{Command Line Processing}
\label{sec:commandline}

The effect of redirection files can also be achieved by invoking
the \LaTeX{} compiler with a more elaborate command line.
Most conveniently this should be done as part
of a shell script or a |Makefile|.

When using \textsf{childdoc} in the main file, the following
command lines effectively perform a redirection
(note that depending on the shell being used,
backslashes may have to be doubled: `|\|' $\to$ `|\\|'):
%
\begin{center}
|... -jobname "|\textit{target}|" |\\|"|[\textit{flags}]%
|\input{childdoc.def}\childdocforward[|\textit{main}|]{|\textit{dest}|}"|
\end{center}
%
Here \textit{target} is the name of the output file,
\textit{main} is the name of the main file
and \textit{dest} is the name of the main or child file to be processed
(all filenames without extensions).
The optional argument \textit{main} can be omitted
if \textit{main} matches \textit{dest}.
Optionally, compilation \textit{flags} can be defined via |\def| commands.
This command line makes the \TeX{} engine believe
it is compiling the file \textit{target}
whose content is specified as the latter parameter.
The provided code then forwards the processing to
\textit{main} or \textit{dest} as described in \secref{sec:forward}.

%%%%%%%%%%%%%%%%%%%%%%%%%%%%%%%%%%%%%%%%%%%%%%%%%%%%%%%%%%%%%%%%%%%%%%%%%%%%%%%%
\subsection{Include by Input}
\label{sec:input}

Including child documents by |\include| has some restrictions by design.
Most notably, the content of a child document always occupies
its own set of pages; pages cannot be shared between child documents.
Usually, this behaviour makes perfect sense
because each child document contain an essential part of the document.
However, in some situations it may be desirable to compose
a document from a collection of parts
without having mandatory page breaks between then.
For this case, the package
provides a mechanism to include parts
by |\input| which can also be processed individually.
However, by construction this mechanism
requires manual handling of the content to be output.

%%%%%%%%%%%%%%%%%%%%%%%%%%%%%%%%%%%%%%%%
\DescribeMacro{\ifchilddocmanual}
The main file should be prepared as usual, see \secref{sec:include}.
However, the document body must make a distinction
between processing of an individual part and of the main document, e.g.:
%
\begin{center}
\begin{tabular}{l}
|\ifchilddocmanual|\\
|\input{\childdocname}|\\
|\||else|\\
\textit{document body with }|\input{|\textit{part}|}|\\
|\||fi|
\end{tabular}
\end{center}
%
The conditional |\ifchilddocmanual| is true whenever
a part to be included by |\input| is being compiled,
and the name of the part is stored in |\childdocname|.

%%%%%%%%%%%%%%%%%%%%%%%%%%%%%%%%%%%%%%%%
\DescribeMacro{\childdocby}
Each part to be included by |\input| should start with:
%
\begin{center}
\begin{tabular}{l}
|\input{childdoc.def}|\\
|\childdocby{|\textit{main}|}|\\
\end{tabular}
\end{center}
%
The directive |\childdocby| is similar to |\childdocof|
described in \secref{sec:include},
but the subsequent selection of content must be done manually.
To that end, both |\ifchilddoc| and |\ifchilddocmanual|
will be true upon processing of a part,
and the name of the part is stored in |\childdocname|.
Note that |\jobname| will be set to the filename of the current part
so that each part receives an individual |.aux| file
that does not interfere with the |.aux| file(s) of the main document.
This behaviour can be altered by the alternative form
|\childdocby[*]{|\textit{main}|}| (with a non-empty optional argument)
which uses the |.aux| file of the main document
by setting |\jobname| to \textit{main}.

%%%%%%%%%%%%%%%%%%%%%%%%%%%%%%%%%%%%%%%%%%%%%%%%%%%%%%%%%%%%%%%%%%%%%%%%%%%%%%%%
\subsection{Driver Development}
\label{sec:driver}

The \textsf{childdoc} mechanism can also be use for the development
of definition files such as \LaTeX{} styles or classes.
This case differs from the above setup with multiple parts
included by |\include| in that no |\includeonly| should be invoked.
This can be achieved by starting the include file
(before |\ProvidesPackage|) with:
%
\begin{center}
\begin{tabular}{l}
|\input{childdoc.def}|\\
|\childdocforward{|\textit{main}|}|\\
\end{tabular}
\end{center}
%
or alternatively with:
%
\begin{center}
\begin{tabular}{l}
|\input{childdoc.def}|\\
|\childdocby{|\textit{main}|}|\\
\end{tabular}
\end{center}
%
Both forms have slightly different effects as described above.
The main file is prepared as usual, see \secref{sec:include}.

%%%%%%%%%%%%%%%%%%%%%%%%%%%%%%%%%%%%%%%%%%%%%%%%%%%%%%%%%%%%%%%%%%%%%%%%%%%%%%%%
\subsection{Legacy Detection}
\label{sec:detection}

The directive |\childdocmain| in the main file can detect
whether the complete document or merely a child is to be compiled
even without using the directive |\childdocof|.
This method is deprecated because it is less robust
and there is no compelling reason to use it;
it is merely provided for backward compatibility
and it may be removed in future versions.

If the detection mechanism is to be used,
it is mandatory to correctly specify
the filename of the main file as the argument of |\childdocmain|:
%
\begin{center}
\begin{tabular}{l}
|\input{childdoc.def}|\\
|\childdocmain{|\textit{main}|}|\\
\end{tabular}
\end{center}
%
If |\jobname| does not match the argument \textit{main} of |\childdocmain|,
it is assumed that |\jobname| points to the child file to be compiled.
When using |\childdocmain| with the main file specified as argument,
it suffices to start a child file
with just |\input{|\textit{main}|}|
without loading of the package and using |\childdocof|.
If instead all processing is done
with the appropriate \textsf{childdoc} directives,
the argument of \textit{main} of |\childdocmain| can be empty.

An alternative version of the command line processing described
in \secref{sec:commandline} using the detection mechanism reads:
%
\begin{center}
|... -jobname "|\textit{target}|" "|[\textit{flags}]%
[|\def\jobname{|\textit{dest}|}|]|\input{|\textit{main}|}"|
\end{center}

%%%%%%%%%%%%%%%%%%%%%%%%%%%%%%%%%%%%%%%%%%%%%%%%%%%%%%%%%%%%%%%%%%%%%%%%%%%%%%%%
\subsection{Manual Code}
\label{sec:manual}

In case one cannot be certain whether the definitions file |childdoc.def|
is installed on the target \TeX{} distribution
and one prefers not to ship it,
it is conceivable to paste a few relevant commands into the sources.

To that end, drop all statements |\input{childdoc.def}|
and perform the replacements as outlined below.
Instead of |\childdocmain{|\textit{main}|}| add the following code
to the top of the main file:
%
\begin{center}
\begin{tabular}{l}
|\||ifdefined\childdocname\endinput\||fi\newif\ifchilddoc|\\
|\edef\childdocname{\scantokens\expandafter{\jobname\noexpand}}|\\
|\def\childdocmain{|\textit{main}|}\||ifx\childdocmain\childdocname\||else|\\
|\childdoctrue\includeonly{\childdocname}\let\jobname\childdocmain\||fi|\\
\end{tabular}
\end{center}
%
Instead of |\childdocof{|\textit{main}|}| just include the main file
at the top of each child file:
%
\begin{center}
|\input{|\textit{main}|}|
\end{center}
%
A simple redirection |\childdocforward{|\textit{dest}|}| is achieved by:
%
\begin{center}
|\def\jobname{|\textit{dest}|}\input{\jobname}|
\end{center}
%
The redirection with prefix
|\childdocforwardprefix[|\textit{prefix}|]{|\textit{dest}|}|
is accomplished by:
%
\begin{center}
\begin{tabular}{l}
|{\edef\jobname{\scantokens\expandafter{\jobname\noexpand}}|\\
|\def\redirectjob |\textit{prefix}|#1~~~{\gdef\jobname{|\textit{dest}|#1}}|\\
|\expandafter\redirectjob\jobname~~~}\input{\jobname}|
\end{tabular}
\end{center}

In an alternative approach,
child documents can be compiled by a specific command line
without additional code or specific definitions:
%
\begin{center}
|... -jobname "|\textit{target}|" "|[\textit{flags}]%
|\includeonly{|\textit{dest}|}\input{|\textit{main}|}"|
\end{center}
%

%%%%%%%%%%%%%%%%%%%%%%%%%%%%%%%%%%%%%%%%%%%%%%%%%%%%%%%%%%%%%%%%%%%%%%%%%%%%%%%%
%%%%%%%%%%%%%%%%%%%%%%%%%%%%%%%%%%%%%%%%%%%%%%%%%%%%%%%%%%%%%%%%%%%%%%%%%%%%%%%%
\section{Information}

%%%%%%%%%%%%%%%%%%%%%%%%%%%%%%%%%%%%%%%%%%%%%%%%%%%%%%%%%%%%%%%%%%%%%%%%%%%%%%%%
\subsection{Copyright}

Copyright \copyright{} 2017--2018 Niklas Beisert

This work may be distributed and/or modified under the
conditions of the \LaTeX{} Project Public License, either version 1.3
of this license or (at your option) any later version.
The latest version of this license is in
  \url{http://www.latex-project.org/lppl.txt}
and version 1.3 or later is part of all distributions of \LaTeX{}
version 2005/12/01 or later.

This work has the LPPL maintenance status `maintained'.

The Current Maintainer of this work is Niklas Beisert.

This work consists of the files |README.txt|, |childdoc.ins| and |childdoc.dtx|
as well as the derived files |childdoc.def|, |cdocsamp.tex|
with |cdocsch1.tex|, |cdocsch2.tex|, |cdocspt3.tex|, |cdocspt4.tex|,
|cdocsdrf.tex|, |cdocsfn1.tex|, |cdocsfn2.tex|
as well as |childdoc.pdf|.

%%%%%%%%%%%%%%%%%%%%%%%%%%%%%%%%%%%%%%%%%%%%%%%%%%%%%%%%%%%%%%%%%%%%%%%%%%%%%%%%
\subsection{Files and Installation}

The package consists of the files:
%
\begin{center}
\begin{tabular}{ll}
    |README.txt|   & readme file \\
    |childdoc.ins| & installation file \\
    |childdoc.dtx| & source file \\
    |childdoc.def| & definition file \\
    |cdocsamp.tex| & sample main file \\
    |cdocsch1.tex| & sample include file \\
    |cdocsch2.tex| & sample include file \\
    |cdocspt3.tex| & sample part file \\
    |cdocspt4.tex| & sample part file \\
    |cdocsdrf.tex| & sample redirection file \\
    |cdocsfn1.tex| & sample redirection file \\
    |cdocsfn2.tex| & sample redirection file \\
    |childdoc.pdf| & manual
\end{tabular}
\end{center}
%
The distribution consists of the files
|README.txt|, |childdoc.ins| and |childdoc.dtx|.
%
\begin{itemize}
\item
Run (pdf)\LaTeX{} on |childdoc.dtx|
to compile the manual |childdoc.pdf| (this file).
\item
Run \LaTeX{} on |childdoc.ins| to create the definitions file |childdoc.def|
and the sample |cdocsamp.tex| with include files
|cdocsch1.tex|, |cdocsch2.tex|, |cdocspt3.tex|, |cdocspt4.tex|,
|cdocsdrf.tex|, |cdocsfn1.tex|, |cdocsfn2.tex|.
Then copy the file |childdoc.def| to an appropriate directory of your \LaTeX{}
distribution, e.g.\ \textit{texmf-root}|/tex/latex/childdoc|.
\end{itemize}

%%%%%%%%%%%%%%%%%%%%%%%%%%%%%%%%%%%%%%%%%%%%%%%%%%%%%%%%%%%%%%%%%%%%%%%%%%%%%%%%
\subsection{Related CTAN Packages}

There are several other packages which offer a similar functionality:
%
\begin{itemize}
\item
The packages
\href{http://ctan.org/pkg/docmute}{\textsf{docmute}},
\href{http://ctan.org/pkg/includex}{\textsf{includex}} and
\href{http://ctan.org/pkg/standalone}{\textsf{standalone}}
provide commands to include only the document body of
a child file thus allowing both files to be compiled individually.
\item
The packages \href{http://ctan.org/pkg/subdocs}{\textsf{subdocs}}
and \href{http://ctan.org/pkg/subfiles}{\textsf{subfiles}}
provide structures in which the main and child documents can be
encapsulated and allowing them to be compiled individually.
The inclusion mechanism is different from the conventional |\include|.
\item
The package \href{http://ctan.org/pkg/combine}{\textsf{combine}}
is an elaborate solution to combine several documents into one.
\end{itemize}
%
See also the CTAN topic \href{http://ctan.org/topic/subdocs}{\textsf{subdocs}}
for further related packages.
The present package differs from the above solutions in that
a document structure constructed with the conventional |\include| mechanism
just needs two extra commands at the top of every file
such that all constituent files can be compiled individually.

%%%%%%%%%%%%%%%%%%%%%%%%%%%%%%%%%%%%%%%%%%%%%%%%%%%%%%%%%%%%%%%%%%%%%%%%%%%%%%%%
%\subsection{Feature Suggestions}
%
%The following is a list of features which may be useful for future
%versions of this package:
%%
%\begin{itemize}
%\item
%\ldots
%\end{itemize}

%%%%%%%%%%%%%%%%%%%%%%%%%%%%%%%%%%%%%%%%%%%%%%%%%%%%%%%%%%%%%%%%%%%%%%%%%%%%%%%%
\subsection{Revision History}

%%%%%%%%%%%%%%%%%%%%%%%%%%%%%%%%%%%%%%%%
\paragraph{v2.0:} 2018/12/30

\begin{itemize}
\item
immediate forward processing
\item
added |\childdocby| mechanism
\item
manual restructured
\end{itemize}

%%%%%%%%%%%%%%%%%%%%%%%%%%%%%%%%%%%%%%%%
\paragraph{v1.6:} 2018/01/17

\begin{itemize}
\item
application for development of include files
\item
corrections to manual
\end{itemize}

%%%%%%%%%%%%%%%%%%%%%%%%%%%%%%%%%%%%%%%%
\paragraph{v1.5:} 2017/05/21

\begin{itemize}
\item
more complete structuring introduced
\item
|\childdocof| introduced
\item
|\childdoc| renamed to |\childdocmain|
\item
|\childredirect| renamed to |\childdocforward| and |\childdocforwardprefix|
and functionality expanded
\end{itemize}

%%%%%%%%%%%%%%%%%%%%%%%%%%%%%%%%%%%%%%%%
\paragraph{v1.0:} 2017/04/27

\begin{itemize}
\item
manual and install package
\item
first version published on CTAN
\end{itemize}

%%%%%%%%%%%%%%%%%%%%%%%%%%%%%%%%%%%%%%%%
\paragraph{v0.6:} 2017/04/26

\begin{itemize}
\item
redirection mechanism added
\end{itemize}

%%%%%%%%%%%%%%%%%%%%%%%%%%%%%%%%%%%%%%%%
\paragraph{v0.5:} 2017/04/26

\begin{itemize}
\item
functionality in definition file
\end{itemize}


%%%%%%%%%%%%%%%%%%%%%%%%%%%%%%%%%%%%%%%%%%%%%%%%%%%%%%%%%%%%%%%%%%%%%%%%%%%%%%%%
%%%%%%%%%%%%%%%%%%%%%%%%%%%%%%%%%%%%%%%%%%%%%%%%%%%%%%%%%%%%%%%%%%%%%%%%%%%%%%%%
%%%%%%%%%%%%%%%%%%%%%%%%%%%%%%%%%%%%%%%%%%%%%%%%%%%%%%%%%%%%%%%%%%%%%%%%%%%%%%%%
\appendix

\settowidth\MacroIndent{\rmfamily\scriptsize 000\ }

 \DocInput{childdoc.dtx}

\end{document}
%</driver>
% \fi
%
% %%%%%%%%%%%%%%%%%%%%%%%%%%%%%%%%%%%%%%%%%%%%%%%%%%%%%%%%%%%%%%%%%%%%%%%%%%%%%%
% %%%%%%%%%%%%%%%%%%%%%%%%%%%%%%%%%%%%%%%%%%%%%%%%%%%%%%%%%%%%%%%%%%%%%%%%%%%%%%
% \section{Sample}
%\iffalse
%<*samplemain>
%\fi
%
% The following presents a sample document
% with two chapters, two parts, a title page,
% a compile flag as well as three forwarding files to set the flag.
% It consists of eight |.tex| files:
% \begin{center}
% \begin{tabular}{ll}
% |cdocsamp.tex|&main file\\
% |cdocsch1.tex|&include file for chapter 1\\
% |cdocsch2.tex|&include file for chapter 2\\
% |cdocspt3.tex|&include file for part 3\\
% |cdocspt4.tex|&include file for part 4\\
% |cdocsdrf.tex|&forwarding file for main file in draft mode\\
% |cdocsfi1.tex|&forwarding file for final version of chapter 1\\
% |cdocsfi2.tex|&forwarding file for final version of chapter 2\\
% \end{tabular}
% \end{center}
% Each of the eight files can be compiled directly by the \LaTeX{} compiler.
%
% %%%%%%%%%%%%%%%%%%%%%%%%%%%%%%%%%%%%%%
% \paragraph{Main File.}
%
% The main file is called |cdocsamp.tex|.
%
% Load the \textsf{childdoc} definitions and
% declare the filename for the main document:
%    \begin{macrocode}
\input{childdoc.def}
\childdocmain{}
%    \end{macrocode}

% Optional override for |\version| flag:
%    \begin{macrocode}
%%\ifchilddoc\else\providecommand{\version}{draft}\fi
%    \end{macrocode}

% Define the default values for the |\version| flag
% (|final| for the main file and |draft| for childs):
%    \begin{macrocode}
\ifchilddoc
\providecommand{\version}{draft}
\else
\providecommand{\version}{final}
\fi
%    \end{macrocode}

% Load the standard document class:
%    \begin{macrocode}
\documentclass[12pt]{article}
%    \end{macrocode}

% Start the document body:
%    \begin{macrocode}
\begin{document}
%    \end{macrocode}

% Declare a title page.
% Print title, part of document being processed and version flag:
%    \begin{macrocode}
\addtocounter{page}{-1}
\begin{center}
{\LARGE\bfseries{}childdoc example\par}
\vspace{1cm}
\ifchilddoc
\ifchilddocmanual part\else chapter\fi:
`\childdocname' of `\childdocjob'\par
\else
main document: `\childdocjob'\par
\fi
version: \version\par
\end{center}
\newpage
%    \end{macrocode}

% Manually include selected file,
% otherwise process as usual:
%    \begin{macrocode}
\ifchilddocmanual
\section*{part `\childdocname'}
\input{\childdocname}
\else
%    \end{macrocode}

% Include the two chapters:
%    \begin{macrocode}
\include{cdocsch1}
\include{cdocsch2}
%    \end{macrocode}

% Include the two parts unless only chapters should be displayed:
%    \begin{macrocode}
\ifchilddoc\else
\section{part three}
\input{cdocspt3}
\section{part four}
\input{cdocspt4}
\fi
%    \end{macrocode}

% Process as usual until here:
%    \begin{macrocode}
\fi
%    \end{macrocode}

% End of document body:
%    \begin{macrocode}
\end{document}
%    \end{macrocode}
%\iffalse
%</samplemain>
%\fi
%
% %%%%%%%%%%%%%%%%%%%%%%%%%%%%%%%%%%%%%%
% \paragraph{Chapter Include Files.}
%
% The include files are called |cdocsch1.tex| and |cdocsch2.tex|.
%
%\iffalse
%<*samplechap1|samplechap2>
%\fi

% Optional override for |\version| flag:
%    \begin{macrocode}
%%\providecommand{\version}{final}
%    \end{macrocode}

% Include the main document:
%    \begin{macrocode}
\input{childdoc.def}
\childdocof{cdocsamp}
%    \end{macrocode}

%\iffalse
%</samplechap1|samplechap2>
%\fi
%
%\iffalse
%<*samplechap1>
%\fi
% Some text for chapter 1:
%    \begin{macrocode}
\section{one}
some text in chapter one
%    \end{macrocode}

%\iffalse
%</samplechap1>
%\fi
% Some text for chapter 2:
%\iffalse
%<*samplechap2>
%\fi
%    \begin{macrocode}
\section{two}
more text in chapter two
%    \end{macrocode}

%\iffalse
%</samplechap2>
%\fi
%
% %%%%%%%%%%%%%%%%%%%%%%%%%%%%%%%%%%%%%%
% \paragraph{Part Include Files.}
%
% The include files are called |cdocspt3.tex| and |cdocspt4.tex|.
%
%\iffalse
%<*samplepart3|samplepart4>
%\fi

% Optional override for |\version| flag:
%    \begin{macrocode}
%%\providecommand{\version}{final}
%    \end{macrocode}

% Include the main document:
%    \begin{macrocode}
\input{childdoc.def}
\childdocby{cdocsamp}
%    \end{macrocode}

%\iffalse
%</samplepart3|samplepart4>
%\fi
%
%\iffalse
%<*samplepart3>
%\fi
% Some text for part 3:
%    \begin{macrocode}
some text in part three
%    \end{macrocode}

%\iffalse
%</samplepart3>
%\fi
% Some text for part 4:
%\iffalse
%<*samplepart4>
%\fi
%    \begin{macrocode}
more text in part four
%    \end{macrocode}

%\iffalse
%</samplepart4>
%\fi
%
% %%%%%%%%%%%%%%%%%%%%%%%%%%%%%%%%%%%%%%
% \paragraph{Forwarding for a Complete Draft.}
%
% The following forwarding file |cdocsdrf.tex|
% compiles the main document in draft mode:
%\iffalse
%<*sampledraft>
%\fi
%    \begin{macrocode}
\def\version{draft}
\input{childdoc.def}
\childdocforward{cdocsamp}
%    \end{macrocode}

%\iffalse
%</sampledraft>
%\fi
%
% %%%%%%%%%%%%%%%%%%%%%%%%%%%%%%%%%%%%%%
% \paragraph{Forwarding for Final Version of the Chapters.}
%
% The following forwarding files |cdocsfn1.tex| and |cdocsfn2.tex|
% (with identical content)
% compile the final versions of the child documents
% |cdocsch1.tex| and |cdocsch2.tex|, respectively:
%\iffalse
%<*samplefinal>
%\fi
%    \begin{macrocode}
\def\version{final}
\input{childdoc.def}
\childdocforwardprefix[cdocsamp]{cdocsfn}{cdocsch}
%    \end{macrocode}

%\iffalse
%</samplefinal>
%\fi
%
% %%%%%%%%%%%%%%%%%%%%%%%%%%%%%%%%%%%%%%
% \paragraph{Command Line Processing.}
%
% The following three command lines generate the output files
% |cdocscld|, |cdocscl1| and |cdocscl2|
% which should be identical to
% |cdocsdrf|, |cdocsch1| and |cdocsfn2|, respectively:
% \begin{center}
% \begin{tabular}{l}
% |latex -jobname cdocscld \|\\
% |  "\def\version{draft}\input{childdoc.def}\childdocforward{cdocsamp}"|\\
% |latex -jobname cdocscl1 \|\\
% |  "\input{childdoc.def}\childdocforward[cdocsamp]{cdocsch1}"|\\
% |latex -jobname cdocscl2 \|\\
% |  "\def\version{final}\input{childdoc.def}\childdocforward{cdocsch2}"|
% \end{tabular}
% \end{center}
% Note that the trailing backslash on each first line
% merely continues the input to the second line
% (for convenient cut ant paste).
% Furthermore, the command |latex| can be replaced by any
% of its alternative versions such as |pdflatex|.
%
% %%%%%%%%%%%%%%%%%%%%%%%%%%%%%%%%%%%%%%%%%%%%%%%%%%%%%%%%%%%%%%%%%%%%%%%%%%%%%%
% %%%%%%%%%%%%%%%%%%%%%%%%%%%%%%%%%%%%%%%%%%%%%%%%%%%%%%%%%%%%%%%%%%%%%%%%%%%%%%
% \section{Implementation}
%\iffalse
%<*package>
%\fi
%
% This section describes the definitions file |childdoc.def|.

% The definitions cannot be loaded using |\usepackage| or |\RequirePackage|
% which has a mechanism to prevent loading a style file more than once.
% When loading the definitions by means of |\input|
% multiple instances have to be prevented manually:
%\iffalse
%This code needs to be before the `\ProvidesFile' directive
%which is defined at the beginning of this file.
%Therefore it is also placed there and commented out here.
%</package>
%<*discard>
%\fi
%    \begin{macrocode}
\ifdefined\childdocmain\endinput\fi
%    \end{macrocode}
%\iffalse
%</discard>
%<*package>
%\fi
%
% \macro{\ifchilddoc}
% \macro{\ifchilddocmanual}
% The conditional |\ifchilddoc| tells whether a
% child (true) or main (false) document is being compiled.
% The conditional |\ifchilddocmanual| tells whether
% the |\includeonly| mechanism is used (false) or
% the selection of child files must be performed manually (true).
% The definitions initialise to false:
%    \begin{macrocode}
\newif\ifchilddoc
\newif\ifchilddocmanual
%    \end{macrocode}

% \macro{\childdocname}
% \macro{\childdocjob}
% The macro |\childdocname| stores the name of the main document
% to be compiled. The macro |\childdocjob| stores the name of
% the document on which the \LaTeX{} compiler was originally invoked.
% The content of |\jobname| cannot be compared
% to filenames specified in the source due to different catcodes.
% The following code rescans |\jobname|, stores the result
% in |\childdocname| and saves a copy in |\childdocjob|:
%    \begin{macrocode}
\edef\childdocname{\scantokens\expandafter{\jobname\noexpand}}
\let\childdocjob\childdocname
%    \end{macrocode}

% \macro{\childdocdisable}
% The macro |\childdocdisable| prevents the main file
% from being processed more than once.
% At this stage, the main document command |\childdocmain|
% is assumed to be called once again where it should do nothing.
% Any subsequent call to it should prevent
% a secondary processing of the main document
% It overwrites the forwarding commands
% |\childdocof| and |\childdocforward|
% with empty macros to prevent further inclusions of the main document:
%    \begin{macrocode}
\newcommand{\childdocdisable}
{
  \renewcommand{\childdocmain}[1]{\renewcommand{\childdocmain}[1]{\endinput}}
  \renewcommand{\childdocof}[1]{}
  \renewcommand{\childdocby}[2][]{}
  \renewcommand{\childdocforward}[2][]{}
  \renewcommand{\childdocdisable}{}
}
%    \end{macrocode}

% \macro{\childdocmain}
% The macro |\childdocmain| is to be called at the top of the main file
% with nothing or the main filename (without extension) as argument.
% First, it breaks loops.
% If the argument is not empty and does not match |\childdocname|
% (which is set by the first inclusion of |childdoc.def|),
% |\ifchilddoc| is set to true, |\includeonly| is applied to the child file
% and |\jobname| is set to the main file
% (for proper handling of |.aux| files):
%    \begin{macrocode}
\newcommand{\childdocmain}[1]
{
  \childdocdisable\childdocmain{}
  \if?#1?\else
    \begingroup
      \def\childdoctmp{#1}
      \ifx\childdoctmp\childdocname
        \def\childdoctmp{}
      \else
        \def\childdoctmp
        {
          \childdoctrue
          \includeonly{\childdocname}
          \def\childdocjob{#1}
          \def\jobname{#1}
        }
      \fi
      \expandafter
    \endgroup
    \childdoctmp
  \fi
}
%    \end{macrocode}

% \macro{\childdocof}
% The command |\childdocof| redirects
% compilation to the main file |#1|.
%    \begin{macrocode}
\newcommand{\childdocof}[1]
{
  \childdocdisable
  \childdoctrue
  \includeonly{\childdocname}
  \def\jobname{#1}
  \def\childdocjob{#1}
  \input{#1}
}
%    \end{macrocode}

% \macro{\childdocby}
% The command |\childdocby| ....
%    \begin{macrocode}
\newcommand{\childdocby}[2][]
{
  \childdocdisable
  \childdoctrue
  \childdocmanualtrue
  \if?#1?\else
    \def\jobname{#2}
  \fi
  \def\childdocjob{#2}
  \input{#2}
  \endinput
}
%    \end{macrocode}

% \macro{\childdocforward}
% The command |\childdocforward| redirects
% compilation to the main file or
% (if the optional argument is given) a child file.
% Parameters are set as if the main file
% or a child file starting with |\childdocof| was compiled.
% Then compilation is handed over to the main file:
%    \begin{macrocode}
\newcommand{\childdocforward}[2][]
{
  \begingroup
    \if?#1?
      \def\childdoctmp
      {
        \def\childdocname{#2}
        \def\childdocjob{#2}
        \def\jobname{#2}
        \input{#2}
        \endinput
      }
    \else
      \def\childdoctmp
      {
        \childdocdisable
        \def\childdocname{#2}
        \childdoctrue
        \includeonly{#2}
        \def\childdocjob{#1}
        \def\jobname{#1}
        \input{#1}
        \endinput
      }
    \fi
    \expandafter
  \endgroup
  \childdoctmp
}
%    \end{macrocode}

% \macro{\childdocforwardprefix}
% The command |\childdocforwardprefix| redirects
% compilation to the main or a child file by means of a pattern.
% The prefix |#1| in the current filename is replaced by |#2|
% and the suffix of the current filename is kept
% (it is assumed that the filename does not contain the substring `|~~~|'
% which is used as a delimiter).
% Compilation is handed over to the new file by |\childdocforward|:
%    \begin{macrocode}
\newcommand{\childdocforwardprefix}[3][]
{
  \begingroup
    \def\childdocextract #2##1~~~{\def\childdoctmp{\childdocforward[#1]{#3##1}}}
    \expandafter\childdocextract\childdocname~~~
    \expandafter
  \endgroup
  \childdoctmp
}
%    \end{macrocode}

% \macro{\childdoc}
% The deprecated macro |\childdoc| is a legacy version of |\childdocmain|:
%    \begin{macrocode}
\newcommand{\childdoc}{\childdocmain}
%    \end{macrocode}

% \macro{\childdocredirect}
% The deprecated macro |\childdocredirect| is a legacy version
% of |\childdocforward| and |\childdocforwardprefix|:
%    \begin{macrocode}
\newcommand{\childdocredirect}[2][]
{
  \begingroup
    \if?#1?
      \def\childdoctmp{\childdocforward{#2}}
    \else
      \def\childdoctmp{\childdocforwardprefix{#1}{#2}}
    \fi
    \expandafter
  \endgroup
  \childdoctmp
}
%    \end{macrocode}

%\iffalse
%</package>
%\fi
%
\endinput
|\\
|\childdocby{|\textit{main}|}|\\
\end{tabular}
\end{center}
%
Both forms have slightly different effects as described above.
The main file is prepared as usual, see \secref{sec:include}.

%%%%%%%%%%%%%%%%%%%%%%%%%%%%%%%%%%%%%%%%%%%%%%%%%%%%%%%%%%%%%%%%%%%%%%%%%%%%%%%%
\subsection{Legacy Detection}
\label{sec:detection}

The directive |\childdocmain| in the main file can detect
whether the complete document or merely a child is to be compiled
even without using the directive |\childdocof|.
This method is deprecated because it is less robust
and there is no compelling reason to use it;
it is merely provided for backward compatibility
and it may be removed in future versions.

If the detection mechanism is to be used,
it is mandatory to correctly specify
the filename of the main file as the argument of |\childdocmain|:
%
\begin{center}
\begin{tabular}{l}
|% \iffalse
%
% childdoc.dtx Copyright (C) 2017-2018 Niklas Beisert
%
% This work may be distributed and/or modified under the
% conditions of the LaTeX Project Public License, either version 1.3
% of this license or (at your option) any later version.
% The latest version of this license is in
%   http://www.latex-project.org/lppl.txt
% and version 1.3 or later is part of all distributions of LaTeX
% version 2005/12/01 or later.
%
% This work has the LPPL maintenance status `maintained'.
%
% The Current Maintainer of this work is Niklas Beisert.
%
% This work consists of the files childdoc.dtx and childdoc.ins
% and the derived files childdoc.def and cdocsamp.tex with
% cdocsch1.tex, cdocsch2.tex, cdocsdrf.tex, cdocsfn1.tex, cdocsfn2.tex.
%
%<package>\ifdefined\childdocmain\endinput\fi
%<package>\ProvidesFile{childdoc.def}[2018/12/30 v2.0 child document driver]
%<samplemain>\ProvidesFile{cdocsamp.tex}[2018/12/30 v2.0 sample for childdoc]
%<*driver>
%\ProvidesFile{childdoc.drv}[2018/12/30 v2.0 childdoc reference manual file]
\PassOptionsToClass{10pt,a4paper}{article}
\documentclass{ltxdoc}

\usepackage[margin=35mm]{geometry}
\usepackage{hyperref}
\usepackage{hyperxmp}
\usepackage[usenames]{color}

\hypersetup{colorlinks=true}
\hypersetup{pdfstartview=FitH}
\hypersetup{pdfpagemode=UseNone}
\hypersetup{pdfsource={}}
\hypersetup{pdflang={en-UK}}
\hypersetup{pdfcopyright={Copyright 2017-2018 Niklas Beisert.
  This work may be distributed and/or modified under the
  conditions of the LaTeX Project Public License, either version 1.3
  of this license or (at your option) any later version.}}
\hypersetup{pdflicenseurl={http://www.latex-project.org/lppl.txt}}
\hypersetup{pdfcontactaddress={ETH Zurich, ITP, HIT K,
  Wolfgang-Pauli-Strasse 27}}
\hypersetup{pdfcontactpostcode={8093}}
\hypersetup{pdfcontactcity={Zurich}}
\hypersetup{pdfcontactcountry={Switzerland}}
\hypersetup{pdfcontactemail={nbeisert@itp.phys.ethz.ch}}
\hypersetup{pdfcontacturl={http://people.phys.ethz.ch/\xmptilde nbeisert/}}

\newcommand{\secref}[1]{\hyperref[#1]{section \ref*{#1}}}

\parskip1ex
\parindent0pt
\let\olditemize\itemize
\def\itemize{\olditemize\parskip0pt}

\begin{document}

\title{The \textsf{childdoc} Package}
\hypersetup{pdftitle={The childdoc Package}}
\author{Niklas Beisert\\[2ex]
  Institut f\"ur Theoretische Physik\\
  Eidgen\"ossische Technische Hochschule Z\"urich\\
  Wolfgang-Pauli-Strasse 27, 8093 Z\"urich, Switzerland\\[1ex]
  \href{mailto:nbeisert@itp.phys.ethz.ch}
  {\texttt{nbeisert@itp.phys.ethz.ch}}}
\hypersetup{pdfauthor={Niklas Beisert}}
\hypersetup{pdfsubject={Manual for the LaTeX2e Package childdoc}}
\date{30 December 2018, \textsf{v2.0}}
\maketitle

\begin{abstract}\noindent
\textsf{childdoc} is a \LaTeXe{} package
that enables the direct compilation
of document sections included by |\include|
to individual files.
\end{abstract}

\begingroup
\parskip0ex
\tableofcontents
\endgroup

%%%%%%%%%%%%%%%%%%%%%%%%%%%%%%%%%%%%%%%%%%%%%%%%%%%%%%%%%%%%%%%%%%%%%%%%%%%%%%%%
%%%%%%%%%%%%%%%%%%%%%%%%%%%%%%%%%%%%%%%%%%%%%%%%%%%%%%%%%%%%%%%%%%%%%%%%%%%%%%%%
\section{Introduction}

\LaTeX{} provides a mechanism to structure a large document (such as a book)
into a main file and several child files (containing the chapters)
using the |\include| command.
This mechanism is beneficial for documents
which span hundreds of pages in order to
make the source file(s) more manageable.
Moreover, compilation can be restricted to
selected child files by means of the |\includeonly| command.
The latter feature can be used to reduce the compilation time while editing
(this was significantly more useful in the earlier days of \LaTeX{})
or to generate a smaller document which is easier to navigate.
Another application of |\includeonly| is to generate
documents consisting of selected parts of the complete document.

However, there are a few drawbacks of the plain |\include| mechanism:
\begin{itemize}
\item
The child files cannot be compiled on their own,
they can only be compiled via the main file.
A naive editing environment
(such as a text editor with an option
to have the current file processed by \LaTeX)
may require one to switch to the main file before compiling;
attempting to compile the child file produces errors.
\item
The main file must be modified (each time)
to adjust the |\includeonly| command
to the present needs. This easily leaves the main file in a messy state.
\item
The generated document will always carry the filename
of the main document. This is inconvenient if
several child files are to be compiled and
to be kept for distribution.
\end{itemize}

The present package provides a simple interface
to make child files individually compilable by \LaTeX{}.
Compiling a child file then has the same effect as compiling
the main file with an |\includeonly| command
to select the appropriate child.
Moreover the generated document will carry the name of the child
rather than the main file.
This resolves all three above issues.

This feature is meant to make the editing of books,
thesis documents and lecture notes somewhat more convenient.
However, the package can also be used efficiently for
composing a series of documents (such as exercise sheets)
which are typically distributed individually.
It then assists the author in generating the individual documents
(potentially in different versions)
as well as a document containing the collected series.
Another application is in developing style files
or other kinds of included material
where compilation of the style file could redirect
to a sample or test file.

%%%%%%%%%%%%%%%%%%%%%%%%%%%%%%%%%%%%%%%%%%%%%%%%%%%%%%%%%%%%%%%%%%%%%%%%%%%%%%%%
%%%%%%%%%%%%%%%%%%%%%%%%%%%%%%%%%%%%%%%%%%%%%%%%%%%%%%%%%%%%%%%%%%%%%%%%%%%%%%%%
\section{Usage}

First of all, the package \textsf{childdoc} is \emph{not} a standard
\LaTeXe{} |.sty| style file! Therefore it needs to be invoked in
a non-standard way.

%%%%%%%%%%%%%%%%%%%%%%%%%%%%%%%%%%%%%%%%%%%%%%%%%%%%%%%%%%%%%%%%%%%%%%%%%%%%%%%%
\subsection{Included Files}
\label{sec:include}

%%%%%%%%%%%%%%%%%%%%%%%%%%%%%%%%%%%%%%%%
\DescribeMacro{\childdocmain}
To use the package, add the commands
\begin{center}
\begin{tabular}{l}
|\input{childdoc.def}|\\
|\childdocmain{}|\\
\end{tabular}
\end{center}
at the very top of the main \LaTeX{} file,
in particular \emph{before} the |\documentclass| statement!
The argument of |\childdocmain| should be left empty
(but it must be present).

%%%%%%%%%%%%%%%%%%%%%%%%%%%%%%%%%%%%%%%%
\DescribeMacro{\childdocof}
Furthermore, add the commands
\begin{center}
\begin{tabular}{l}
|\input{childdoc.def}|\\
|\childdocof{|\textit{main}|}|\\
\end{tabular}
\end{center}
at the top of every child file \textit{child}
which is included by |\include{|\textit{child}|}|
from within the main file
(or at least for those files to be compiled individually).
The argument \textit{main} must be the filename of the main file.

There are a couple of
considerations in setting up the main and child documents:

%%%%%%%%%%%%%%%%%%%%%%%%%%%%%%%%%%%%%%%%
\paragraph{Restrictions.}

Please note the following restrictions:
\begin{itemize}
\item
|\childdocmain| must be called with one argument \textit{main}
to ensure compatibility with earlier version of the package.
It must either be empty (|\childdocmain{}|)
or precisely match the filename of the main file in which it is specified.
See \secref{sec:detection} for further information.
\item
The filename \textit{main} must be specified without the |.tex| extension.
\item
The filename \textit{main} is case sensitive
(even in case-insensitive file systems)
due to internal string comparison.
\item
The argument \textit{main} should be fully expanded, it cannot be a macro.
\item
Subdirectories and special characters should be avoided in filenames.
\item
The command |\childdocmain{|\textit{main}|}| must be followed by a whitespace.
It should not be followed immediately by another command
or by a comment mark `|%|'.
This is because the \TeX{} parser reads the token immediately following
the argument of |\childdocmain| and puts it
at the beginning of every child section;
however, a white\-space is ignored.
\end{itemize}

%%%%%%%%%%%%%%%%%%%%%%%%%%%%%%%%%%%%%%%%
\paragraph{Content of Main File.}

It is advisable to place all content in the child files included by |\include|.
Any output contained in the main file will appear in all child documents
unless suppressed manually;
it cannot be suppressed automatically by the |\includeonly| directive
and thus should normally be avoided.
A method to include some content in the main file
by means of conditional processing is described in \secref{sec:conditional}.

%%%%%%%%%%%%%%%%%%%%%%%%%%%%%%%%%%%%%%%%
\paragraph{Page Numbering.}

When only a part of the document is compiled,
the appropriate numbering of pages
(as well as other status parameters)
is determined from the |.aux| files.
The latter contain information from previous passes.
However this information needs to propagate through
all intermediate child documents.
Therefore the page numbering in child documents may well
be inconsistent until the complete document is compiled at least once.

A useful (if unconventional) way to always ensure a consistent
page numbering is to restart the numbering in each child document
and denote the pages by `\textit{child}|.|\textit{page}'
where \textit{child} represents the chapter/section number of the child file.
This can be achieved by the command
|\numberwithin{page}{|\textit{child}|}|
of the \textsf{amsmath} package
where \textit{child} can be |chapter| or |section|
depending on the chosen structuring.
Alternatively, one can modify the macro |\thepage| appropriately
and reset the counter |page| at the start of each child file.

%%%%%%%%%%%%%%%%%%%%%%%%%%%%%%%%%%%%%%%%%%%%%%%%%%%%%%%%%%%%%%%%%%%%%%%%%%%%%%%%
\subsection{Conditional Processing}
\label{sec:conditional}

The package provides a mechanism to compile different versions
of a document. To customise the versions further some conditional processing
can come in handy to distinguish which version is being compiled.
The package provides two macros to describe the compilation context:

%%%%%%%%%%%%%%%%%%%%%%%%%%%%%%%%%%%%%%%%
\DescribeMacro{\ifchilddoc}
The conditional |\ifchilddoc| distinguishes between the compilation of
child documents and the main document:
%
\begin{center}
|\ifchilddoc |\textit{child-code}| |[|\||else |\textit{main-code}]| \||fi|
\end{center}

%%%%%%%%%%%%%%%%%%%%%%%%%%%%%%%%%%%%%%%%
\DescribeMacro{\childdocname}
\DescribeMacro{\childdocjob}
The macro |\childdocname| contains the filename (without extension)
of the main or child file being processed.
Note that |\childdocjob| will always contain the name of the main file.

%%%%%%%%%%%%%%%%%%%%%%%%%%%%%%%%%%%%%%%%
\paragraph{Title Page.}

Conditional processing can be used to include a title or banner page
in the main document when proper precautions are taken.
Importantly, the code in the main file should ensure that the page counter
(as well as other status parameters which are stored in the |.aux| files)
takes the same value after the conditional processing.
Otherwise the page numbers may take divergent values
depending on which part is compiled.

For example, a title page could be declared by:
%
\begin{center}
\begin{tabular}{l}
|\ifchilddoc\||else|\\
|\addtocounter{page}{-1}|\\
\textit{code for title page}\\
|\newpage|\\
|\||fi|
\end{tabular}
\end{center}
%
A banner page for the child documents can be generated by:
%
\begin{center}
\begin{tabular}{l}
|\ifchilddoc|\\
|\addtocounter{page}{-1}|\\
\textit{code for banner page}\\
|\newpage|\\
|\||fi|
\end{tabular}
\end{center}
%
Here one could write a message such as:
\begin{center}
|This is the part \childdocname{} of \childdocjob{}.|
\end{center}

%%%%%%%%%%%%%%%%%%%%%%%%%%%%%%%%%%%%%%%%%%%%%%%%%%%%%%%%%%%%%%%%%%%%%%%%%%%%%%%%
\subsection{Flags}
\label{sec:flags}

The package makes it easy to generate different versions
of the main or child documents.
To this end compilation flags can be defined
and assigned different default values.
They will be particularly useful in conjunction
with the forwarding mechanism described in \secref{sec:forward}.

For example, it may be useful to have a flag |\version|
which can be set to |draft| or |final|.
The document source will contain some conditional code
depending on the value of |\version|.
Suppose further, the flag should default to |final| for the main file
and to |draft| for child files
which is a natural assignment for editing the document.
This is achieved by placing the following code
in the preamble of the main document
(below the |\childdocmain| directive):
%
\begin{center}
\begin{tabular}{l}
|\ifchilddoc|\\
|\providecommand{\version}{draft}|\\
|\||else|\\
|\providecommand{\version}{final}|\\
|\||fi|
\end{tabular}
\end{center}
%
The definition by |\providecommand| makes sure
that previous definitions are not overwritten.
Further statements |\providecommand{\version}{...}|
can thus be added before the above code to override it.

For the main file, one might add a line
(between |\childdocmain| and the above block)
%
\begin{center}
|%\ifchilddoc\||else\providecommand{\version}{draft}\||fi|
\end{center}
%
which can be uncommented to produce a draft version.
Likewise one can add a line to the very top of a child file
(above the |\childdocof{|\textit{main}|}| directive)
%
\begin{center}
|%\providecommand{\version}{final}|
\end{center}
%
which can be uncommented to produce the final version of this child document.

%%%%%%%%%%%%%%%%%%%%%%%%%%%%%%%%%%%%%%%%%%%%%%%%%%%%%%%%%%%%%%%%%%%%%%%%%%%%%%%%
\subsection{Forwarding}
\label{sec:forward}

Different versions of the main or child documents
using compilation flags as described in \secref{sec:flags}
can be (permanently) stored in different files
for convenient compilation, viewing and distribution.
To this end, the package defines a command
to pass on compilation to a different file:

%%%%%%%%%%%%%%%%%%%%%%%%%%%%%%%%%%%%%%%%
\DescribeMacro{\childdocforward}
The command |\childdocforward| redirects processing to
another source file:
%
\begin{center}
\begin{tabular}{l}
|\input{childdoc.def}|\\
|\childdocforward[|\textit{main}|]{|\textit{dest}|}|\\
\end{tabular}
\end{center}
%
The argument \textit{dest} is the destination file
(without extension).
It should be the main file or one of the child files.
Note that further \textsf{childdoc} directives
such as |\childdocof| and |\childdocforward|
in the indicated file will be processed in this form.
The optional argument \textit{main}
passes on directly to the main file \textit{main}
while pretending to compile the child \textit{dest}.
This form behaves as if \textit{dest}
issues |\childdocof{|\textit{main}|}| right away,
and no further \textsf{childdoc} directives will be processed.

%%%%%%%%%%%%%%%%%%%%%%%%%%%%%%%%%%%%%%%%
\DescribeMacro{\...prefix}
In the alternative form |\childdocforwardprefix|,
%
\begin{center}
\begin{tabular}{l}
|\input{childdoc.def}|\\
|\childdocforwardprefix[|\textit{main}|]{|\textit{prefix}|}{|\textit{dest}|}|
\end{tabular}
\end{center}
%
the destination file is determined by a pattern
depending on the current file:
To make this work, the current file must be called
`{\textit{prefix}\hspace{0.2em}\textit{suffix}}'
with \textit{prefix} matching precisely the argument.
Processing is then passed on to the file
`{\textit{dest}\hspace{0.2em}\textit{suffix}}'.
Surely, the same effect is achieved by
directly specifying the
argument `{\textit{dest}\hspace{0.2em}\textit{suffix}}'
in the first form.
However, that requires to set up a different file
for each child. With the alternative form of the command
all these files can have exactly the same content
which simplifies setting them up and maintaining them.

For example, the following file |draft.tex|
with a compilation flag |\version| as described in \secref{sec:flags}
compiles the main document as a draft:
%
\begin{center}
\begin{tabular}{l}
|\def\version{draft}|\\
|\input{childdoc.def}|\\
|\childdocforward{|\textit{main}|}|
\end{tabular}
\end{center}
%
Likewise, the following files |final|\textit{nn}|.tex|
compile the final version of the child document
|child|\textit{nn}|.tex|:
%
\begin{center}
\begin{tabular}{l}
|\def\version{final}|\\
|\input{childdoc.def}|\\
|\childdocforwardprefix{final}{child}|
\end{tabular}
\end{center}
%

Note that when several versions of a main file and/or of each child file
are to be generated, it may be convenient to set up a |Makefile| or
shell script to automatise the process.

%%%%%%%%%%%%%%%%%%%%%%%%%%%%%%%%%%%%%%%%%%%%%%%%%%%%%%%%%%%%%%%%%%%%%%%%%%%%%%%%
\subsection{Command Line Processing}
\label{sec:commandline}

The effect of redirection files can also be achieved by invoking
the \LaTeX{} compiler with a more elaborate command line.
Most conveniently this should be done as part
of a shell script or a |Makefile|.

When using \textsf{childdoc} in the main file, the following
command lines effectively perform a redirection
(note that depending on the shell being used,
backslashes may have to be doubled: `|\|' $\to$ `|\\|'):
%
\begin{center}
|... -jobname "|\textit{target}|" |\\|"|[\textit{flags}]%
|\input{childdoc.def}\childdocforward[|\textit{main}|]{|\textit{dest}|}"|
\end{center}
%
Here \textit{target} is the name of the output file,
\textit{main} is the name of the main file
and \textit{dest} is the name of the main or child file to be processed
(all filenames without extensions).
The optional argument \textit{main} can be omitted
if \textit{main} matches \textit{dest}.
Optionally, compilation \textit{flags} can be defined via |\def| commands.
This command line makes the \TeX{} engine believe
it is compiling the file \textit{target}
whose content is specified as the latter parameter.
The provided code then forwards the processing to
\textit{main} or \textit{dest} as described in \secref{sec:forward}.

%%%%%%%%%%%%%%%%%%%%%%%%%%%%%%%%%%%%%%%%%%%%%%%%%%%%%%%%%%%%%%%%%%%%%%%%%%%%%%%%
\subsection{Include by Input}
\label{sec:input}

Including child documents by |\include| has some restrictions by design.
Most notably, the content of a child document always occupies
its own set of pages; pages cannot be shared between child documents.
Usually, this behaviour makes perfect sense
because each child document contain an essential part of the document.
However, in some situations it may be desirable to compose
a document from a collection of parts
without having mandatory page breaks between then.
For this case, the package
provides a mechanism to include parts
by |\input| which can also be processed individually.
However, by construction this mechanism
requires manual handling of the content to be output.

%%%%%%%%%%%%%%%%%%%%%%%%%%%%%%%%%%%%%%%%
\DescribeMacro{\ifchilddocmanual}
The main file should be prepared as usual, see \secref{sec:include}.
However, the document body must make a distinction
between processing of an individual part and of the main document, e.g.:
%
\begin{center}
\begin{tabular}{l}
|\ifchilddocmanual|\\
|\input{\childdocname}|\\
|\||else|\\
\textit{document body with }|\input{|\textit{part}|}|\\
|\||fi|
\end{tabular}
\end{center}
%
The conditional |\ifchilddocmanual| is true whenever
a part to be included by |\input| is being compiled,
and the name of the part is stored in |\childdocname|.

%%%%%%%%%%%%%%%%%%%%%%%%%%%%%%%%%%%%%%%%
\DescribeMacro{\childdocby}
Each part to be included by |\input| should start with:
%
\begin{center}
\begin{tabular}{l}
|\input{childdoc.def}|\\
|\childdocby{|\textit{main}|}|\\
\end{tabular}
\end{center}
%
The directive |\childdocby| is similar to |\childdocof|
described in \secref{sec:include},
but the subsequent selection of content must be done manually.
To that end, both |\ifchilddoc| and |\ifchilddocmanual|
will be true upon processing of a part,
and the name of the part is stored in |\childdocname|.
Note that |\jobname| will be set to the filename of the current part
so that each part receives an individual |.aux| file
that does not interfere with the |.aux| file(s) of the main document.
This behaviour can be altered by the alternative form
|\childdocby[*]{|\textit{main}|}| (with a non-empty optional argument)
which uses the |.aux| file of the main document
by setting |\jobname| to \textit{main}.

%%%%%%%%%%%%%%%%%%%%%%%%%%%%%%%%%%%%%%%%%%%%%%%%%%%%%%%%%%%%%%%%%%%%%%%%%%%%%%%%
\subsection{Driver Development}
\label{sec:driver}

The \textsf{childdoc} mechanism can also be use for the development
of definition files such as \LaTeX{} styles or classes.
This case differs from the above setup with multiple parts
included by |\include| in that no |\includeonly| should be invoked.
This can be achieved by starting the include file
(before |\ProvidesPackage|) with:
%
\begin{center}
\begin{tabular}{l}
|\input{childdoc.def}|\\
|\childdocforward{|\textit{main}|}|\\
\end{tabular}
\end{center}
%
or alternatively with:
%
\begin{center}
\begin{tabular}{l}
|\input{childdoc.def}|\\
|\childdocby{|\textit{main}|}|\\
\end{tabular}
\end{center}
%
Both forms have slightly different effects as described above.
The main file is prepared as usual, see \secref{sec:include}.

%%%%%%%%%%%%%%%%%%%%%%%%%%%%%%%%%%%%%%%%%%%%%%%%%%%%%%%%%%%%%%%%%%%%%%%%%%%%%%%%
\subsection{Legacy Detection}
\label{sec:detection}

The directive |\childdocmain| in the main file can detect
whether the complete document or merely a child is to be compiled
even without using the directive |\childdocof|.
This method is deprecated because it is less robust
and there is no compelling reason to use it;
it is merely provided for backward compatibility
and it may be removed in future versions.

If the detection mechanism is to be used,
it is mandatory to correctly specify
the filename of the main file as the argument of |\childdocmain|:
%
\begin{center}
\begin{tabular}{l}
|\input{childdoc.def}|\\
|\childdocmain{|\textit{main}|}|\\
\end{tabular}
\end{center}
%
If |\jobname| does not match the argument \textit{main} of |\childdocmain|,
it is assumed that |\jobname| points to the child file to be compiled.
When using |\childdocmain| with the main file specified as argument,
it suffices to start a child file
with just |\input{|\textit{main}|}|
without loading of the package and using |\childdocof|.
If instead all processing is done
with the appropriate \textsf{childdoc} directives,
the argument of \textit{main} of |\childdocmain| can be empty.

An alternative version of the command line processing described
in \secref{sec:commandline} using the detection mechanism reads:
%
\begin{center}
|... -jobname "|\textit{target}|" "|[\textit{flags}]%
[|\def\jobname{|\textit{dest}|}|]|\input{|\textit{main}|}"|
\end{center}

%%%%%%%%%%%%%%%%%%%%%%%%%%%%%%%%%%%%%%%%%%%%%%%%%%%%%%%%%%%%%%%%%%%%%%%%%%%%%%%%
\subsection{Manual Code}
\label{sec:manual}

In case one cannot be certain whether the definitions file |childdoc.def|
is installed on the target \TeX{} distribution
and one prefers not to ship it,
it is conceivable to paste a few relevant commands into the sources.

To that end, drop all statements |\input{childdoc.def}|
and perform the replacements as outlined below.
Instead of |\childdocmain{|\textit{main}|}| add the following code
to the top of the main file:
%
\begin{center}
\begin{tabular}{l}
|\||ifdefined\childdocname\endinput\||fi\newif\ifchilddoc|\\
|\edef\childdocname{\scantokens\expandafter{\jobname\noexpand}}|\\
|\def\childdocmain{|\textit{main}|}\||ifx\childdocmain\childdocname\||else|\\
|\childdoctrue\includeonly{\childdocname}\let\jobname\childdocmain\||fi|\\
\end{tabular}
\end{center}
%
Instead of |\childdocof{|\textit{main}|}| just include the main file
at the top of each child file:
%
\begin{center}
|\input{|\textit{main}|}|
\end{center}
%
A simple redirection |\childdocforward{|\textit{dest}|}| is achieved by:
%
\begin{center}
|\def\jobname{|\textit{dest}|}\input{\jobname}|
\end{center}
%
The redirection with prefix
|\childdocforwardprefix[|\textit{prefix}|]{|\textit{dest}|}|
is accomplished by:
%
\begin{center}
\begin{tabular}{l}
|{\edef\jobname{\scantokens\expandafter{\jobname\noexpand}}|\\
|\def\redirectjob |\textit{prefix}|#1~~~{\gdef\jobname{|\textit{dest}|#1}}|\\
|\expandafter\redirectjob\jobname~~~}\input{\jobname}|
\end{tabular}
\end{center}

In an alternative approach,
child documents can be compiled by a specific command line
without additional code or specific definitions:
%
\begin{center}
|... -jobname "|\textit{target}|" "|[\textit{flags}]%
|\includeonly{|\textit{dest}|}\input{|\textit{main}|}"|
\end{center}
%

%%%%%%%%%%%%%%%%%%%%%%%%%%%%%%%%%%%%%%%%%%%%%%%%%%%%%%%%%%%%%%%%%%%%%%%%%%%%%%%%
%%%%%%%%%%%%%%%%%%%%%%%%%%%%%%%%%%%%%%%%%%%%%%%%%%%%%%%%%%%%%%%%%%%%%%%%%%%%%%%%
\section{Information}

%%%%%%%%%%%%%%%%%%%%%%%%%%%%%%%%%%%%%%%%%%%%%%%%%%%%%%%%%%%%%%%%%%%%%%%%%%%%%%%%
\subsection{Copyright}

Copyright \copyright{} 2017--2018 Niklas Beisert

This work may be distributed and/or modified under the
conditions of the \LaTeX{} Project Public License, either version 1.3
of this license or (at your option) any later version.
The latest version of this license is in
  \url{http://www.latex-project.org/lppl.txt}
and version 1.3 or later is part of all distributions of \LaTeX{}
version 2005/12/01 or later.

This work has the LPPL maintenance status `maintained'.

The Current Maintainer of this work is Niklas Beisert.

This work consists of the files |README.txt|, |childdoc.ins| and |childdoc.dtx|
as well as the derived files |childdoc.def|, |cdocsamp.tex|
with |cdocsch1.tex|, |cdocsch2.tex|, |cdocspt3.tex|, |cdocspt4.tex|,
|cdocsdrf.tex|, |cdocsfn1.tex|, |cdocsfn2.tex|
as well as |childdoc.pdf|.

%%%%%%%%%%%%%%%%%%%%%%%%%%%%%%%%%%%%%%%%%%%%%%%%%%%%%%%%%%%%%%%%%%%%%%%%%%%%%%%%
\subsection{Files and Installation}

The package consists of the files:
%
\begin{center}
\begin{tabular}{ll}
    |README.txt|   & readme file \\
    |childdoc.ins| & installation file \\
    |childdoc.dtx| & source file \\
    |childdoc.def| & definition file \\
    |cdocsamp.tex| & sample main file \\
    |cdocsch1.tex| & sample include file \\
    |cdocsch2.tex| & sample include file \\
    |cdocspt3.tex| & sample part file \\
    |cdocspt4.tex| & sample part file \\
    |cdocsdrf.tex| & sample redirection file \\
    |cdocsfn1.tex| & sample redirection file \\
    |cdocsfn2.tex| & sample redirection file \\
    |childdoc.pdf| & manual
\end{tabular}
\end{center}
%
The distribution consists of the files
|README.txt|, |childdoc.ins| and |childdoc.dtx|.
%
\begin{itemize}
\item
Run (pdf)\LaTeX{} on |childdoc.dtx|
to compile the manual |childdoc.pdf| (this file).
\item
Run \LaTeX{} on |childdoc.ins| to create the definitions file |childdoc.def|
and the sample |cdocsamp.tex| with include files
|cdocsch1.tex|, |cdocsch2.tex|, |cdocspt3.tex|, |cdocspt4.tex|,
|cdocsdrf.tex|, |cdocsfn1.tex|, |cdocsfn2.tex|.
Then copy the file |childdoc.def| to an appropriate directory of your \LaTeX{}
distribution, e.g.\ \textit{texmf-root}|/tex/latex/childdoc|.
\end{itemize}

%%%%%%%%%%%%%%%%%%%%%%%%%%%%%%%%%%%%%%%%%%%%%%%%%%%%%%%%%%%%%%%%%%%%%%%%%%%%%%%%
\subsection{Related CTAN Packages}

There are several other packages which offer a similar functionality:
%
\begin{itemize}
\item
The packages
\href{http://ctan.org/pkg/docmute}{\textsf{docmute}},
\href{http://ctan.org/pkg/includex}{\textsf{includex}} and
\href{http://ctan.org/pkg/standalone}{\textsf{standalone}}
provide commands to include only the document body of
a child file thus allowing both files to be compiled individually.
\item
The packages \href{http://ctan.org/pkg/subdocs}{\textsf{subdocs}}
and \href{http://ctan.org/pkg/subfiles}{\textsf{subfiles}}
provide structures in which the main and child documents can be
encapsulated and allowing them to be compiled individually.
The inclusion mechanism is different from the conventional |\include|.
\item
The package \href{http://ctan.org/pkg/combine}{\textsf{combine}}
is an elaborate solution to combine several documents into one.
\end{itemize}
%
See also the CTAN topic \href{http://ctan.org/topic/subdocs}{\textsf{subdocs}}
for further related packages.
The present package differs from the above solutions in that
a document structure constructed with the conventional |\include| mechanism
just needs two extra commands at the top of every file
such that all constituent files can be compiled individually.

%%%%%%%%%%%%%%%%%%%%%%%%%%%%%%%%%%%%%%%%%%%%%%%%%%%%%%%%%%%%%%%%%%%%%%%%%%%%%%%%
%\subsection{Feature Suggestions}
%
%The following is a list of features which may be useful for future
%versions of this package:
%%
%\begin{itemize}
%\item
%\ldots
%\end{itemize}

%%%%%%%%%%%%%%%%%%%%%%%%%%%%%%%%%%%%%%%%%%%%%%%%%%%%%%%%%%%%%%%%%%%%%%%%%%%%%%%%
\subsection{Revision History}

%%%%%%%%%%%%%%%%%%%%%%%%%%%%%%%%%%%%%%%%
\paragraph{v2.0:} 2018/12/30

\begin{itemize}
\item
immediate forward processing
\item
added |\childdocby| mechanism
\item
manual restructured
\end{itemize}

%%%%%%%%%%%%%%%%%%%%%%%%%%%%%%%%%%%%%%%%
\paragraph{v1.6:} 2018/01/17

\begin{itemize}
\item
application for development of include files
\item
corrections to manual
\end{itemize}

%%%%%%%%%%%%%%%%%%%%%%%%%%%%%%%%%%%%%%%%
\paragraph{v1.5:} 2017/05/21

\begin{itemize}
\item
more complete structuring introduced
\item
|\childdocof| introduced
\item
|\childdoc| renamed to |\childdocmain|
\item
|\childredirect| renamed to |\childdocforward| and |\childdocforwardprefix|
and functionality expanded
\end{itemize}

%%%%%%%%%%%%%%%%%%%%%%%%%%%%%%%%%%%%%%%%
\paragraph{v1.0:} 2017/04/27

\begin{itemize}
\item
manual and install package
\item
first version published on CTAN
\end{itemize}

%%%%%%%%%%%%%%%%%%%%%%%%%%%%%%%%%%%%%%%%
\paragraph{v0.6:} 2017/04/26

\begin{itemize}
\item
redirection mechanism added
\end{itemize}

%%%%%%%%%%%%%%%%%%%%%%%%%%%%%%%%%%%%%%%%
\paragraph{v0.5:} 2017/04/26

\begin{itemize}
\item
functionality in definition file
\end{itemize}


%%%%%%%%%%%%%%%%%%%%%%%%%%%%%%%%%%%%%%%%%%%%%%%%%%%%%%%%%%%%%%%%%%%%%%%%%%%%%%%%
%%%%%%%%%%%%%%%%%%%%%%%%%%%%%%%%%%%%%%%%%%%%%%%%%%%%%%%%%%%%%%%%%%%%%%%%%%%%%%%%
%%%%%%%%%%%%%%%%%%%%%%%%%%%%%%%%%%%%%%%%%%%%%%%%%%%%%%%%%%%%%%%%%%%%%%%%%%%%%%%%
\appendix

\settowidth\MacroIndent{\rmfamily\scriptsize 000\ }

 \DocInput{childdoc.dtx}

\end{document}
%</driver>
% \fi
%
% %%%%%%%%%%%%%%%%%%%%%%%%%%%%%%%%%%%%%%%%%%%%%%%%%%%%%%%%%%%%%%%%%%%%%%%%%%%%%%
% %%%%%%%%%%%%%%%%%%%%%%%%%%%%%%%%%%%%%%%%%%%%%%%%%%%%%%%%%%%%%%%%%%%%%%%%%%%%%%
% \section{Sample}
%\iffalse
%<*samplemain>
%\fi
%
% The following presents a sample document
% with two chapters, two parts, a title page,
% a compile flag as well as three forwarding files to set the flag.
% It consists of eight |.tex| files:
% \begin{center}
% \begin{tabular}{ll}
% |cdocsamp.tex|&main file\\
% |cdocsch1.tex|&include file for chapter 1\\
% |cdocsch2.tex|&include file for chapter 2\\
% |cdocspt3.tex|&include file for part 3\\
% |cdocspt4.tex|&include file for part 4\\
% |cdocsdrf.tex|&forwarding file for main file in draft mode\\
% |cdocsfi1.tex|&forwarding file for final version of chapter 1\\
% |cdocsfi2.tex|&forwarding file for final version of chapter 2\\
% \end{tabular}
% \end{center}
% Each of the eight files can be compiled directly by the \LaTeX{} compiler.
%
% %%%%%%%%%%%%%%%%%%%%%%%%%%%%%%%%%%%%%%
% \paragraph{Main File.}
%
% The main file is called |cdocsamp.tex|.
%
% Load the \textsf{childdoc} definitions and
% declare the filename for the main document:
%    \begin{macrocode}
\input{childdoc.def}
\childdocmain{}
%    \end{macrocode}

% Optional override for |\version| flag:
%    \begin{macrocode}
%%\ifchilddoc\else\providecommand{\version}{draft}\fi
%    \end{macrocode}

% Define the default values for the |\version| flag
% (|final| for the main file and |draft| for childs):
%    \begin{macrocode}
\ifchilddoc
\providecommand{\version}{draft}
\else
\providecommand{\version}{final}
\fi
%    \end{macrocode}

% Load the standard document class:
%    \begin{macrocode}
\documentclass[12pt]{article}
%    \end{macrocode}

% Start the document body:
%    \begin{macrocode}
\begin{document}
%    \end{macrocode}

% Declare a title page.
% Print title, part of document being processed and version flag:
%    \begin{macrocode}
\addtocounter{page}{-1}
\begin{center}
{\LARGE\bfseries{}childdoc example\par}
\vspace{1cm}
\ifchilddoc
\ifchilddocmanual part\else chapter\fi:
`\childdocname' of `\childdocjob'\par
\else
main document: `\childdocjob'\par
\fi
version: \version\par
\end{center}
\newpage
%    \end{macrocode}

% Manually include selected file,
% otherwise process as usual:
%    \begin{macrocode}
\ifchilddocmanual
\section*{part `\childdocname'}
\input{\childdocname}
\else
%    \end{macrocode}

% Include the two chapters:
%    \begin{macrocode}
\include{cdocsch1}
\include{cdocsch2}
%    \end{macrocode}

% Include the two parts unless only chapters should be displayed:
%    \begin{macrocode}
\ifchilddoc\else
\section{part three}
\input{cdocspt3}
\section{part four}
\input{cdocspt4}
\fi
%    \end{macrocode}

% Process as usual until here:
%    \begin{macrocode}
\fi
%    \end{macrocode}

% End of document body:
%    \begin{macrocode}
\end{document}
%    \end{macrocode}
%\iffalse
%</samplemain>
%\fi
%
% %%%%%%%%%%%%%%%%%%%%%%%%%%%%%%%%%%%%%%
% \paragraph{Chapter Include Files.}
%
% The include files are called |cdocsch1.tex| and |cdocsch2.tex|.
%
%\iffalse
%<*samplechap1|samplechap2>
%\fi

% Optional override for |\version| flag:
%    \begin{macrocode}
%%\providecommand{\version}{final}
%    \end{macrocode}

% Include the main document:
%    \begin{macrocode}
\input{childdoc.def}
\childdocof{cdocsamp}
%    \end{macrocode}

%\iffalse
%</samplechap1|samplechap2>
%\fi
%
%\iffalse
%<*samplechap1>
%\fi
% Some text for chapter 1:
%    \begin{macrocode}
\section{one}
some text in chapter one
%    \end{macrocode}

%\iffalse
%</samplechap1>
%\fi
% Some text for chapter 2:
%\iffalse
%<*samplechap2>
%\fi
%    \begin{macrocode}
\section{two}
more text in chapter two
%    \end{macrocode}

%\iffalse
%</samplechap2>
%\fi
%
% %%%%%%%%%%%%%%%%%%%%%%%%%%%%%%%%%%%%%%
% \paragraph{Part Include Files.}
%
% The include files are called |cdocspt3.tex| and |cdocspt4.tex|.
%
%\iffalse
%<*samplepart3|samplepart4>
%\fi

% Optional override for |\version| flag:
%    \begin{macrocode}
%%\providecommand{\version}{final}
%    \end{macrocode}

% Include the main document:
%    \begin{macrocode}
\input{childdoc.def}
\childdocby{cdocsamp}
%    \end{macrocode}

%\iffalse
%</samplepart3|samplepart4>
%\fi
%
%\iffalse
%<*samplepart3>
%\fi
% Some text for part 3:
%    \begin{macrocode}
some text in part three
%    \end{macrocode}

%\iffalse
%</samplepart3>
%\fi
% Some text for part 4:
%\iffalse
%<*samplepart4>
%\fi
%    \begin{macrocode}
more text in part four
%    \end{macrocode}

%\iffalse
%</samplepart4>
%\fi
%
% %%%%%%%%%%%%%%%%%%%%%%%%%%%%%%%%%%%%%%
% \paragraph{Forwarding for a Complete Draft.}
%
% The following forwarding file |cdocsdrf.tex|
% compiles the main document in draft mode:
%\iffalse
%<*sampledraft>
%\fi
%    \begin{macrocode}
\def\version{draft}
\input{childdoc.def}
\childdocforward{cdocsamp}
%    \end{macrocode}

%\iffalse
%</sampledraft>
%\fi
%
% %%%%%%%%%%%%%%%%%%%%%%%%%%%%%%%%%%%%%%
% \paragraph{Forwarding for Final Version of the Chapters.}
%
% The following forwarding files |cdocsfn1.tex| and |cdocsfn2.tex|
% (with identical content)
% compile the final versions of the child documents
% |cdocsch1.tex| and |cdocsch2.tex|, respectively:
%\iffalse
%<*samplefinal>
%\fi
%    \begin{macrocode}
\def\version{final}
\input{childdoc.def}
\childdocforwardprefix[cdocsamp]{cdocsfn}{cdocsch}
%    \end{macrocode}

%\iffalse
%</samplefinal>
%\fi
%
% %%%%%%%%%%%%%%%%%%%%%%%%%%%%%%%%%%%%%%
% \paragraph{Command Line Processing.}
%
% The following three command lines generate the output files
% |cdocscld|, |cdocscl1| and |cdocscl2|
% which should be identical to
% |cdocsdrf|, |cdocsch1| and |cdocsfn2|, respectively:
% \begin{center}
% \begin{tabular}{l}
% |latex -jobname cdocscld \|\\
% |  "\def\version{draft}\input{childdoc.def}\childdocforward{cdocsamp}"|\\
% |latex -jobname cdocscl1 \|\\
% |  "\input{childdoc.def}\childdocforward[cdocsamp]{cdocsch1}"|\\
% |latex -jobname cdocscl2 \|\\
% |  "\def\version{final}\input{childdoc.def}\childdocforward{cdocsch2}"|
% \end{tabular}
% \end{center}
% Note that the trailing backslash on each first line
% merely continues the input to the second line
% (for convenient cut ant paste).
% Furthermore, the command |latex| can be replaced by any
% of its alternative versions such as |pdflatex|.
%
% %%%%%%%%%%%%%%%%%%%%%%%%%%%%%%%%%%%%%%%%%%%%%%%%%%%%%%%%%%%%%%%%%%%%%%%%%%%%%%
% %%%%%%%%%%%%%%%%%%%%%%%%%%%%%%%%%%%%%%%%%%%%%%%%%%%%%%%%%%%%%%%%%%%%%%%%%%%%%%
% \section{Implementation}
%\iffalse
%<*package>
%\fi
%
% This section describes the definitions file |childdoc.def|.

% The definitions cannot be loaded using |\usepackage| or |\RequirePackage|
% which has a mechanism to prevent loading a style file more than once.
% When loading the definitions by means of |\input|
% multiple instances have to be prevented manually:
%\iffalse
%This code needs to be before the `\ProvidesFile' directive
%which is defined at the beginning of this file.
%Therefore it is also placed there and commented out here.
%</package>
%<*discard>
%\fi
%    \begin{macrocode}
\ifdefined\childdocmain\endinput\fi
%    \end{macrocode}
%\iffalse
%</discard>
%<*package>
%\fi
%
% \macro{\ifchilddoc}
% \macro{\ifchilddocmanual}
% The conditional |\ifchilddoc| tells whether a
% child (true) or main (false) document is being compiled.
% The conditional |\ifchilddocmanual| tells whether
% the |\includeonly| mechanism is used (false) or
% the selection of child files must be performed manually (true).
% The definitions initialise to false:
%    \begin{macrocode}
\newif\ifchilddoc
\newif\ifchilddocmanual
%    \end{macrocode}

% \macro{\childdocname}
% \macro{\childdocjob}
% The macro |\childdocname| stores the name of the main document
% to be compiled. The macro |\childdocjob| stores the name of
% the document on which the \LaTeX{} compiler was originally invoked.
% The content of |\jobname| cannot be compared
% to filenames specified in the source due to different catcodes.
% The following code rescans |\jobname|, stores the result
% in |\childdocname| and saves a copy in |\childdocjob|:
%    \begin{macrocode}
\edef\childdocname{\scantokens\expandafter{\jobname\noexpand}}
\let\childdocjob\childdocname
%    \end{macrocode}

% \macro{\childdocdisable}
% The macro |\childdocdisable| prevents the main file
% from being processed more than once.
% At this stage, the main document command |\childdocmain|
% is assumed to be called once again where it should do nothing.
% Any subsequent call to it should prevent
% a secondary processing of the main document
% It overwrites the forwarding commands
% |\childdocof| and |\childdocforward|
% with empty macros to prevent further inclusions of the main document:
%    \begin{macrocode}
\newcommand{\childdocdisable}
{
  \renewcommand{\childdocmain}[1]{\renewcommand{\childdocmain}[1]{\endinput}}
  \renewcommand{\childdocof}[1]{}
  \renewcommand{\childdocby}[2][]{}
  \renewcommand{\childdocforward}[2][]{}
  \renewcommand{\childdocdisable}{}
}
%    \end{macrocode}

% \macro{\childdocmain}
% The macro |\childdocmain| is to be called at the top of the main file
% with nothing or the main filename (without extension) as argument.
% First, it breaks loops.
% If the argument is not empty and does not match |\childdocname|
% (which is set by the first inclusion of |childdoc.def|),
% |\ifchilddoc| is set to true, |\includeonly| is applied to the child file
% and |\jobname| is set to the main file
% (for proper handling of |.aux| files):
%    \begin{macrocode}
\newcommand{\childdocmain}[1]
{
  \childdocdisable\childdocmain{}
  \if?#1?\else
    \begingroup
      \def\childdoctmp{#1}
      \ifx\childdoctmp\childdocname
        \def\childdoctmp{}
      \else
        \def\childdoctmp
        {
          \childdoctrue
          \includeonly{\childdocname}
          \def\childdocjob{#1}
          \def\jobname{#1}
        }
      \fi
      \expandafter
    \endgroup
    \childdoctmp
  \fi
}
%    \end{macrocode}

% \macro{\childdocof}
% The command |\childdocof| redirects
% compilation to the main file |#1|.
%    \begin{macrocode}
\newcommand{\childdocof}[1]
{
  \childdocdisable
  \childdoctrue
  \includeonly{\childdocname}
  \def\jobname{#1}
  \def\childdocjob{#1}
  \input{#1}
}
%    \end{macrocode}

% \macro{\childdocby}
% The command |\childdocby| ....
%    \begin{macrocode}
\newcommand{\childdocby}[2][]
{
  \childdocdisable
  \childdoctrue
  \childdocmanualtrue
  \if?#1?\else
    \def\jobname{#2}
  \fi
  \def\childdocjob{#2}
  \input{#2}
  \endinput
}
%    \end{macrocode}

% \macro{\childdocforward}
% The command |\childdocforward| redirects
% compilation to the main file or
% (if the optional argument is given) a child file.
% Parameters are set as if the main file
% or a child file starting with |\childdocof| was compiled.
% Then compilation is handed over to the main file:
%    \begin{macrocode}
\newcommand{\childdocforward}[2][]
{
  \begingroup
    \if?#1?
      \def\childdoctmp
      {
        \def\childdocname{#2}
        \def\childdocjob{#2}
        \def\jobname{#2}
        \input{#2}
        \endinput
      }
    \else
      \def\childdoctmp
      {
        \childdocdisable
        \def\childdocname{#2}
        \childdoctrue
        \includeonly{#2}
        \def\childdocjob{#1}
        \def\jobname{#1}
        \input{#1}
        \endinput
      }
    \fi
    \expandafter
  \endgroup
  \childdoctmp
}
%    \end{macrocode}

% \macro{\childdocforwardprefix}
% The command |\childdocforwardprefix| redirects
% compilation to the main or a child file by means of a pattern.
% The prefix |#1| in the current filename is replaced by |#2|
% and the suffix of the current filename is kept
% (it is assumed that the filename does not contain the substring `|~~~|'
% which is used as a delimiter).
% Compilation is handed over to the new file by |\childdocforward|:
%    \begin{macrocode}
\newcommand{\childdocforwardprefix}[3][]
{
  \begingroup
    \def\childdocextract #2##1~~~{\def\childdoctmp{\childdocforward[#1]{#3##1}}}
    \expandafter\childdocextract\childdocname~~~
    \expandafter
  \endgroup
  \childdoctmp
}
%    \end{macrocode}

% \macro{\childdoc}
% The deprecated macro |\childdoc| is a legacy version of |\childdocmain|:
%    \begin{macrocode}
\newcommand{\childdoc}{\childdocmain}
%    \end{macrocode}

% \macro{\childdocredirect}
% The deprecated macro |\childdocredirect| is a legacy version
% of |\childdocforward| and |\childdocforwardprefix|:
%    \begin{macrocode}
\newcommand{\childdocredirect}[2][]
{
  \begingroup
    \if?#1?
      \def\childdoctmp{\childdocforward{#2}}
    \else
      \def\childdoctmp{\childdocforwardprefix{#1}{#2}}
    \fi
    \expandafter
  \endgroup
  \childdoctmp
}
%    \end{macrocode}

%\iffalse
%</package>
%\fi
%
\endinput
|\\
|\childdocmain{|\textit{main}|}|\\
\end{tabular}
\end{center}
%
If |\jobname| does not match the argument \textit{main} of |\childdocmain|,
it is assumed that |\jobname| points to the child file to be compiled.
When using |\childdocmain| with the main file specified as argument,
it suffices to start a child file
with just |\input{|\textit{main}|}|
without loading of the package and using |\childdocof|.
If instead all processing is done
with the appropriate \textsf{childdoc} directives,
the argument of \textit{main} of |\childdocmain| can be empty.

An alternative version of the command line processing described
in \secref{sec:commandline} using the detection mechanism reads:
%
\begin{center}
|... -jobname "|\textit{target}|" "|[\textit{flags}]%
[|\def\jobname{|\textit{dest}|}|]|\input{|\textit{main}|}"|
\end{center}

%%%%%%%%%%%%%%%%%%%%%%%%%%%%%%%%%%%%%%%%%%%%%%%%%%%%%%%%%%%%%%%%%%%%%%%%%%%%%%%%
\subsection{Manual Code}
\label{sec:manual}

In case one cannot be certain whether the definitions file |childdoc.def|
is installed on the target \TeX{} distribution
and one prefers not to ship it,
it is conceivable to paste a few relevant commands into the sources.

To that end, drop all statements |% \iffalse
%
% childdoc.dtx Copyright (C) 2017-2018 Niklas Beisert
%
% This work may be distributed and/or modified under the
% conditions of the LaTeX Project Public License, either version 1.3
% of this license or (at your option) any later version.
% The latest version of this license is in
%   http://www.latex-project.org/lppl.txt
% and version 1.3 or later is part of all distributions of LaTeX
% version 2005/12/01 or later.
%
% This work has the LPPL maintenance status `maintained'.
%
% The Current Maintainer of this work is Niklas Beisert.
%
% This work consists of the files childdoc.dtx and childdoc.ins
% and the derived files childdoc.def and cdocsamp.tex with
% cdocsch1.tex, cdocsch2.tex, cdocsdrf.tex, cdocsfn1.tex, cdocsfn2.tex.
%
%<package>\ifdefined\childdocmain\endinput\fi
%<package>\ProvidesFile{childdoc.def}[2018/12/30 v2.0 child document driver]
%<samplemain>\ProvidesFile{cdocsamp.tex}[2018/12/30 v2.0 sample for childdoc]
%<*driver>
%\ProvidesFile{childdoc.drv}[2018/12/30 v2.0 childdoc reference manual file]
\PassOptionsToClass{10pt,a4paper}{article}
\documentclass{ltxdoc}

\usepackage[margin=35mm]{geometry}
\usepackage{hyperref}
\usepackage{hyperxmp}
\usepackage[usenames]{color}

\hypersetup{colorlinks=true}
\hypersetup{pdfstartview=FitH}
\hypersetup{pdfpagemode=UseNone}
\hypersetup{pdfsource={}}
\hypersetup{pdflang={en-UK}}
\hypersetup{pdfcopyright={Copyright 2017-2018 Niklas Beisert.
  This work may be distributed and/or modified under the
  conditions of the LaTeX Project Public License, either version 1.3
  of this license or (at your option) any later version.}}
\hypersetup{pdflicenseurl={http://www.latex-project.org/lppl.txt}}
\hypersetup{pdfcontactaddress={ETH Zurich, ITP, HIT K,
  Wolfgang-Pauli-Strasse 27}}
\hypersetup{pdfcontactpostcode={8093}}
\hypersetup{pdfcontactcity={Zurich}}
\hypersetup{pdfcontactcountry={Switzerland}}
\hypersetup{pdfcontactemail={nbeisert@itp.phys.ethz.ch}}
\hypersetup{pdfcontacturl={http://people.phys.ethz.ch/\xmptilde nbeisert/}}

\newcommand{\secref}[1]{\hyperref[#1]{section \ref*{#1}}}

\parskip1ex
\parindent0pt
\let\olditemize\itemize
\def\itemize{\olditemize\parskip0pt}

\begin{document}

\title{The \textsf{childdoc} Package}
\hypersetup{pdftitle={The childdoc Package}}
\author{Niklas Beisert\\[2ex]
  Institut f\"ur Theoretische Physik\\
  Eidgen\"ossische Technische Hochschule Z\"urich\\
  Wolfgang-Pauli-Strasse 27, 8093 Z\"urich, Switzerland\\[1ex]
  \href{mailto:nbeisert@itp.phys.ethz.ch}
  {\texttt{nbeisert@itp.phys.ethz.ch}}}
\hypersetup{pdfauthor={Niklas Beisert}}
\hypersetup{pdfsubject={Manual for the LaTeX2e Package childdoc}}
\date{30 December 2018, \textsf{v2.0}}
\maketitle

\begin{abstract}\noindent
\textsf{childdoc} is a \LaTeXe{} package
that enables the direct compilation
of document sections included by |\include|
to individual files.
\end{abstract}

\begingroup
\parskip0ex
\tableofcontents
\endgroup

%%%%%%%%%%%%%%%%%%%%%%%%%%%%%%%%%%%%%%%%%%%%%%%%%%%%%%%%%%%%%%%%%%%%%%%%%%%%%%%%
%%%%%%%%%%%%%%%%%%%%%%%%%%%%%%%%%%%%%%%%%%%%%%%%%%%%%%%%%%%%%%%%%%%%%%%%%%%%%%%%
\section{Introduction}

\LaTeX{} provides a mechanism to structure a large document (such as a book)
into a main file and several child files (containing the chapters)
using the |\include| command.
This mechanism is beneficial for documents
which span hundreds of pages in order to
make the source file(s) more manageable.
Moreover, compilation can be restricted to
selected child files by means of the |\includeonly| command.
The latter feature can be used to reduce the compilation time while editing
(this was significantly more useful in the earlier days of \LaTeX{})
or to generate a smaller document which is easier to navigate.
Another application of |\includeonly| is to generate
documents consisting of selected parts of the complete document.

However, there are a few drawbacks of the plain |\include| mechanism:
\begin{itemize}
\item
The child files cannot be compiled on their own,
they can only be compiled via the main file.
A naive editing environment
(such as a text editor with an option
to have the current file processed by \LaTeX)
may require one to switch to the main file before compiling;
attempting to compile the child file produces errors.
\item
The main file must be modified (each time)
to adjust the |\includeonly| command
to the present needs. This easily leaves the main file in a messy state.
\item
The generated document will always carry the filename
of the main document. This is inconvenient if
several child files are to be compiled and
to be kept for distribution.
\end{itemize}

The present package provides a simple interface
to make child files individually compilable by \LaTeX{}.
Compiling a child file then has the same effect as compiling
the main file with an |\includeonly| command
to select the appropriate child.
Moreover the generated document will carry the name of the child
rather than the main file.
This resolves all three above issues.

This feature is meant to make the editing of books,
thesis documents and lecture notes somewhat more convenient.
However, the package can also be used efficiently for
composing a series of documents (such as exercise sheets)
which are typically distributed individually.
It then assists the author in generating the individual documents
(potentially in different versions)
as well as a document containing the collected series.
Another application is in developing style files
or other kinds of included material
where compilation of the style file could redirect
to a sample or test file.

%%%%%%%%%%%%%%%%%%%%%%%%%%%%%%%%%%%%%%%%%%%%%%%%%%%%%%%%%%%%%%%%%%%%%%%%%%%%%%%%
%%%%%%%%%%%%%%%%%%%%%%%%%%%%%%%%%%%%%%%%%%%%%%%%%%%%%%%%%%%%%%%%%%%%%%%%%%%%%%%%
\section{Usage}

First of all, the package \textsf{childdoc} is \emph{not} a standard
\LaTeXe{} |.sty| style file! Therefore it needs to be invoked in
a non-standard way.

%%%%%%%%%%%%%%%%%%%%%%%%%%%%%%%%%%%%%%%%%%%%%%%%%%%%%%%%%%%%%%%%%%%%%%%%%%%%%%%%
\subsection{Included Files}
\label{sec:include}

%%%%%%%%%%%%%%%%%%%%%%%%%%%%%%%%%%%%%%%%
\DescribeMacro{\childdocmain}
To use the package, add the commands
\begin{center}
\begin{tabular}{l}
|\input{childdoc.def}|\\
|\childdocmain{}|\\
\end{tabular}
\end{center}
at the very top of the main \LaTeX{} file,
in particular \emph{before} the |\documentclass| statement!
The argument of |\childdocmain| should be left empty
(but it must be present).

%%%%%%%%%%%%%%%%%%%%%%%%%%%%%%%%%%%%%%%%
\DescribeMacro{\childdocof}
Furthermore, add the commands
\begin{center}
\begin{tabular}{l}
|\input{childdoc.def}|\\
|\childdocof{|\textit{main}|}|\\
\end{tabular}
\end{center}
at the top of every child file \textit{child}
which is included by |\include{|\textit{child}|}|
from within the main file
(or at least for those files to be compiled individually).
The argument \textit{main} must be the filename of the main file.

There are a couple of
considerations in setting up the main and child documents:

%%%%%%%%%%%%%%%%%%%%%%%%%%%%%%%%%%%%%%%%
\paragraph{Restrictions.}

Please note the following restrictions:
\begin{itemize}
\item
|\childdocmain| must be called with one argument \textit{main}
to ensure compatibility with earlier version of the package.
It must either be empty (|\childdocmain{}|)
or precisely match the filename of the main file in which it is specified.
See \secref{sec:detection} for further information.
\item
The filename \textit{main} must be specified without the |.tex| extension.
\item
The filename \textit{main} is case sensitive
(even in case-insensitive file systems)
due to internal string comparison.
\item
The argument \textit{main} should be fully expanded, it cannot be a macro.
\item
Subdirectories and special characters should be avoided in filenames.
\item
The command |\childdocmain{|\textit{main}|}| must be followed by a whitespace.
It should not be followed immediately by another command
or by a comment mark `|%|'.
This is because the \TeX{} parser reads the token immediately following
the argument of |\childdocmain| and puts it
at the beginning of every child section;
however, a white\-space is ignored.
\end{itemize}

%%%%%%%%%%%%%%%%%%%%%%%%%%%%%%%%%%%%%%%%
\paragraph{Content of Main File.}

It is advisable to place all content in the child files included by |\include|.
Any output contained in the main file will appear in all child documents
unless suppressed manually;
it cannot be suppressed automatically by the |\includeonly| directive
and thus should normally be avoided.
A method to include some content in the main file
by means of conditional processing is described in \secref{sec:conditional}.

%%%%%%%%%%%%%%%%%%%%%%%%%%%%%%%%%%%%%%%%
\paragraph{Page Numbering.}

When only a part of the document is compiled,
the appropriate numbering of pages
(as well as other status parameters)
is determined from the |.aux| files.
The latter contain information from previous passes.
However this information needs to propagate through
all intermediate child documents.
Therefore the page numbering in child documents may well
be inconsistent until the complete document is compiled at least once.

A useful (if unconventional) way to always ensure a consistent
page numbering is to restart the numbering in each child document
and denote the pages by `\textit{child}|.|\textit{page}'
where \textit{child} represents the chapter/section number of the child file.
This can be achieved by the command
|\numberwithin{page}{|\textit{child}|}|
of the \textsf{amsmath} package
where \textit{child} can be |chapter| or |section|
depending on the chosen structuring.
Alternatively, one can modify the macro |\thepage| appropriately
and reset the counter |page| at the start of each child file.

%%%%%%%%%%%%%%%%%%%%%%%%%%%%%%%%%%%%%%%%%%%%%%%%%%%%%%%%%%%%%%%%%%%%%%%%%%%%%%%%
\subsection{Conditional Processing}
\label{sec:conditional}

The package provides a mechanism to compile different versions
of a document. To customise the versions further some conditional processing
can come in handy to distinguish which version is being compiled.
The package provides two macros to describe the compilation context:

%%%%%%%%%%%%%%%%%%%%%%%%%%%%%%%%%%%%%%%%
\DescribeMacro{\ifchilddoc}
The conditional |\ifchilddoc| distinguishes between the compilation of
child documents and the main document:
%
\begin{center}
|\ifchilddoc |\textit{child-code}| |[|\||else |\textit{main-code}]| \||fi|
\end{center}

%%%%%%%%%%%%%%%%%%%%%%%%%%%%%%%%%%%%%%%%
\DescribeMacro{\childdocname}
\DescribeMacro{\childdocjob}
The macro |\childdocname| contains the filename (without extension)
of the main or child file being processed.
Note that |\childdocjob| will always contain the name of the main file.

%%%%%%%%%%%%%%%%%%%%%%%%%%%%%%%%%%%%%%%%
\paragraph{Title Page.}

Conditional processing can be used to include a title or banner page
in the main document when proper precautions are taken.
Importantly, the code in the main file should ensure that the page counter
(as well as other status parameters which are stored in the |.aux| files)
takes the same value after the conditional processing.
Otherwise the page numbers may take divergent values
depending on which part is compiled.

For example, a title page could be declared by:
%
\begin{center}
\begin{tabular}{l}
|\ifchilddoc\||else|\\
|\addtocounter{page}{-1}|\\
\textit{code for title page}\\
|\newpage|\\
|\||fi|
\end{tabular}
\end{center}
%
A banner page for the child documents can be generated by:
%
\begin{center}
\begin{tabular}{l}
|\ifchilddoc|\\
|\addtocounter{page}{-1}|\\
\textit{code for banner page}\\
|\newpage|\\
|\||fi|
\end{tabular}
\end{center}
%
Here one could write a message such as:
\begin{center}
|This is the part \childdocname{} of \childdocjob{}.|
\end{center}

%%%%%%%%%%%%%%%%%%%%%%%%%%%%%%%%%%%%%%%%%%%%%%%%%%%%%%%%%%%%%%%%%%%%%%%%%%%%%%%%
\subsection{Flags}
\label{sec:flags}

The package makes it easy to generate different versions
of the main or child documents.
To this end compilation flags can be defined
and assigned different default values.
They will be particularly useful in conjunction
with the forwarding mechanism described in \secref{sec:forward}.

For example, it may be useful to have a flag |\version|
which can be set to |draft| or |final|.
The document source will contain some conditional code
depending on the value of |\version|.
Suppose further, the flag should default to |final| for the main file
and to |draft| for child files
which is a natural assignment for editing the document.
This is achieved by placing the following code
in the preamble of the main document
(below the |\childdocmain| directive):
%
\begin{center}
\begin{tabular}{l}
|\ifchilddoc|\\
|\providecommand{\version}{draft}|\\
|\||else|\\
|\providecommand{\version}{final}|\\
|\||fi|
\end{tabular}
\end{center}
%
The definition by |\providecommand| makes sure
that previous definitions are not overwritten.
Further statements |\providecommand{\version}{...}|
can thus be added before the above code to override it.

For the main file, one might add a line
(between |\childdocmain| and the above block)
%
\begin{center}
|%\ifchilddoc\||else\providecommand{\version}{draft}\||fi|
\end{center}
%
which can be uncommented to produce a draft version.
Likewise one can add a line to the very top of a child file
(above the |\childdocof{|\textit{main}|}| directive)
%
\begin{center}
|%\providecommand{\version}{final}|
\end{center}
%
which can be uncommented to produce the final version of this child document.

%%%%%%%%%%%%%%%%%%%%%%%%%%%%%%%%%%%%%%%%%%%%%%%%%%%%%%%%%%%%%%%%%%%%%%%%%%%%%%%%
\subsection{Forwarding}
\label{sec:forward}

Different versions of the main or child documents
using compilation flags as described in \secref{sec:flags}
can be (permanently) stored in different files
for convenient compilation, viewing and distribution.
To this end, the package defines a command
to pass on compilation to a different file:

%%%%%%%%%%%%%%%%%%%%%%%%%%%%%%%%%%%%%%%%
\DescribeMacro{\childdocforward}
The command |\childdocforward| redirects processing to
another source file:
%
\begin{center}
\begin{tabular}{l}
|\input{childdoc.def}|\\
|\childdocforward[|\textit{main}|]{|\textit{dest}|}|\\
\end{tabular}
\end{center}
%
The argument \textit{dest} is the destination file
(without extension).
It should be the main file or one of the child files.
Note that further \textsf{childdoc} directives
such as |\childdocof| and |\childdocforward|
in the indicated file will be processed in this form.
The optional argument \textit{main}
passes on directly to the main file \textit{main}
while pretending to compile the child \textit{dest}.
This form behaves as if \textit{dest}
issues |\childdocof{|\textit{main}|}| right away,
and no further \textsf{childdoc} directives will be processed.

%%%%%%%%%%%%%%%%%%%%%%%%%%%%%%%%%%%%%%%%
\DescribeMacro{\...prefix}
In the alternative form |\childdocforwardprefix|,
%
\begin{center}
\begin{tabular}{l}
|\input{childdoc.def}|\\
|\childdocforwardprefix[|\textit{main}|]{|\textit{prefix}|}{|\textit{dest}|}|
\end{tabular}
\end{center}
%
the destination file is determined by a pattern
depending on the current file:
To make this work, the current file must be called
`{\textit{prefix}\hspace{0.2em}\textit{suffix}}'
with \textit{prefix} matching precisely the argument.
Processing is then passed on to the file
`{\textit{dest}\hspace{0.2em}\textit{suffix}}'.
Surely, the same effect is achieved by
directly specifying the
argument `{\textit{dest}\hspace{0.2em}\textit{suffix}}'
in the first form.
However, that requires to set up a different file
for each child. With the alternative form of the command
all these files can have exactly the same content
which simplifies setting them up and maintaining them.

For example, the following file |draft.tex|
with a compilation flag |\version| as described in \secref{sec:flags}
compiles the main document as a draft:
%
\begin{center}
\begin{tabular}{l}
|\def\version{draft}|\\
|\input{childdoc.def}|\\
|\childdocforward{|\textit{main}|}|
\end{tabular}
\end{center}
%
Likewise, the following files |final|\textit{nn}|.tex|
compile the final version of the child document
|child|\textit{nn}|.tex|:
%
\begin{center}
\begin{tabular}{l}
|\def\version{final}|\\
|\input{childdoc.def}|\\
|\childdocforwardprefix{final}{child}|
\end{tabular}
\end{center}
%

Note that when several versions of a main file and/or of each child file
are to be generated, it may be convenient to set up a |Makefile| or
shell script to automatise the process.

%%%%%%%%%%%%%%%%%%%%%%%%%%%%%%%%%%%%%%%%%%%%%%%%%%%%%%%%%%%%%%%%%%%%%%%%%%%%%%%%
\subsection{Command Line Processing}
\label{sec:commandline}

The effect of redirection files can also be achieved by invoking
the \LaTeX{} compiler with a more elaborate command line.
Most conveniently this should be done as part
of a shell script or a |Makefile|.

When using \textsf{childdoc} in the main file, the following
command lines effectively perform a redirection
(note that depending on the shell being used,
backslashes may have to be doubled: `|\|' $\to$ `|\\|'):
%
\begin{center}
|... -jobname "|\textit{target}|" |\\|"|[\textit{flags}]%
|\input{childdoc.def}\childdocforward[|\textit{main}|]{|\textit{dest}|}"|
\end{center}
%
Here \textit{target} is the name of the output file,
\textit{main} is the name of the main file
and \textit{dest} is the name of the main or child file to be processed
(all filenames without extensions).
The optional argument \textit{main} can be omitted
if \textit{main} matches \textit{dest}.
Optionally, compilation \textit{flags} can be defined via |\def| commands.
This command line makes the \TeX{} engine believe
it is compiling the file \textit{target}
whose content is specified as the latter parameter.
The provided code then forwards the processing to
\textit{main} or \textit{dest} as described in \secref{sec:forward}.

%%%%%%%%%%%%%%%%%%%%%%%%%%%%%%%%%%%%%%%%%%%%%%%%%%%%%%%%%%%%%%%%%%%%%%%%%%%%%%%%
\subsection{Include by Input}
\label{sec:input}

Including child documents by |\include| has some restrictions by design.
Most notably, the content of a child document always occupies
its own set of pages; pages cannot be shared between child documents.
Usually, this behaviour makes perfect sense
because each child document contain an essential part of the document.
However, in some situations it may be desirable to compose
a document from a collection of parts
without having mandatory page breaks between then.
For this case, the package
provides a mechanism to include parts
by |\input| which can also be processed individually.
However, by construction this mechanism
requires manual handling of the content to be output.

%%%%%%%%%%%%%%%%%%%%%%%%%%%%%%%%%%%%%%%%
\DescribeMacro{\ifchilddocmanual}
The main file should be prepared as usual, see \secref{sec:include}.
However, the document body must make a distinction
between processing of an individual part and of the main document, e.g.:
%
\begin{center}
\begin{tabular}{l}
|\ifchilddocmanual|\\
|\input{\childdocname}|\\
|\||else|\\
\textit{document body with }|\input{|\textit{part}|}|\\
|\||fi|
\end{tabular}
\end{center}
%
The conditional |\ifchilddocmanual| is true whenever
a part to be included by |\input| is being compiled,
and the name of the part is stored in |\childdocname|.

%%%%%%%%%%%%%%%%%%%%%%%%%%%%%%%%%%%%%%%%
\DescribeMacro{\childdocby}
Each part to be included by |\input| should start with:
%
\begin{center}
\begin{tabular}{l}
|\input{childdoc.def}|\\
|\childdocby{|\textit{main}|}|\\
\end{tabular}
\end{center}
%
The directive |\childdocby| is similar to |\childdocof|
described in \secref{sec:include},
but the subsequent selection of content must be done manually.
To that end, both |\ifchilddoc| and |\ifchilddocmanual|
will be true upon processing of a part,
and the name of the part is stored in |\childdocname|.
Note that |\jobname| will be set to the filename of the current part
so that each part receives an individual |.aux| file
that does not interfere with the |.aux| file(s) of the main document.
This behaviour can be altered by the alternative form
|\childdocby[*]{|\textit{main}|}| (with a non-empty optional argument)
which uses the |.aux| file of the main document
by setting |\jobname| to \textit{main}.

%%%%%%%%%%%%%%%%%%%%%%%%%%%%%%%%%%%%%%%%%%%%%%%%%%%%%%%%%%%%%%%%%%%%%%%%%%%%%%%%
\subsection{Driver Development}
\label{sec:driver}

The \textsf{childdoc} mechanism can also be use for the development
of definition files such as \LaTeX{} styles or classes.
This case differs from the above setup with multiple parts
included by |\include| in that no |\includeonly| should be invoked.
This can be achieved by starting the include file
(before |\ProvidesPackage|) with:
%
\begin{center}
\begin{tabular}{l}
|\input{childdoc.def}|\\
|\childdocforward{|\textit{main}|}|\\
\end{tabular}
\end{center}
%
or alternatively with:
%
\begin{center}
\begin{tabular}{l}
|\input{childdoc.def}|\\
|\childdocby{|\textit{main}|}|\\
\end{tabular}
\end{center}
%
Both forms have slightly different effects as described above.
The main file is prepared as usual, see \secref{sec:include}.

%%%%%%%%%%%%%%%%%%%%%%%%%%%%%%%%%%%%%%%%%%%%%%%%%%%%%%%%%%%%%%%%%%%%%%%%%%%%%%%%
\subsection{Legacy Detection}
\label{sec:detection}

The directive |\childdocmain| in the main file can detect
whether the complete document or merely a child is to be compiled
even without using the directive |\childdocof|.
This method is deprecated because it is less robust
and there is no compelling reason to use it;
it is merely provided for backward compatibility
and it may be removed in future versions.

If the detection mechanism is to be used,
it is mandatory to correctly specify
the filename of the main file as the argument of |\childdocmain|:
%
\begin{center}
\begin{tabular}{l}
|\input{childdoc.def}|\\
|\childdocmain{|\textit{main}|}|\\
\end{tabular}
\end{center}
%
If |\jobname| does not match the argument \textit{main} of |\childdocmain|,
it is assumed that |\jobname| points to the child file to be compiled.
When using |\childdocmain| with the main file specified as argument,
it suffices to start a child file
with just |\input{|\textit{main}|}|
without loading of the package and using |\childdocof|.
If instead all processing is done
with the appropriate \textsf{childdoc} directives,
the argument of \textit{main} of |\childdocmain| can be empty.

An alternative version of the command line processing described
in \secref{sec:commandline} using the detection mechanism reads:
%
\begin{center}
|... -jobname "|\textit{target}|" "|[\textit{flags}]%
[|\def\jobname{|\textit{dest}|}|]|\input{|\textit{main}|}"|
\end{center}

%%%%%%%%%%%%%%%%%%%%%%%%%%%%%%%%%%%%%%%%%%%%%%%%%%%%%%%%%%%%%%%%%%%%%%%%%%%%%%%%
\subsection{Manual Code}
\label{sec:manual}

In case one cannot be certain whether the definitions file |childdoc.def|
is installed on the target \TeX{} distribution
and one prefers not to ship it,
it is conceivable to paste a few relevant commands into the sources.

To that end, drop all statements |\input{childdoc.def}|
and perform the replacements as outlined below.
Instead of |\childdocmain{|\textit{main}|}| add the following code
to the top of the main file:
%
\begin{center}
\begin{tabular}{l}
|\||ifdefined\childdocname\endinput\||fi\newif\ifchilddoc|\\
|\edef\childdocname{\scantokens\expandafter{\jobname\noexpand}}|\\
|\def\childdocmain{|\textit{main}|}\||ifx\childdocmain\childdocname\||else|\\
|\childdoctrue\includeonly{\childdocname}\let\jobname\childdocmain\||fi|\\
\end{tabular}
\end{center}
%
Instead of |\childdocof{|\textit{main}|}| just include the main file
at the top of each child file:
%
\begin{center}
|\input{|\textit{main}|}|
\end{center}
%
A simple redirection |\childdocforward{|\textit{dest}|}| is achieved by:
%
\begin{center}
|\def\jobname{|\textit{dest}|}\input{\jobname}|
\end{center}
%
The redirection with prefix
|\childdocforwardprefix[|\textit{prefix}|]{|\textit{dest}|}|
is accomplished by:
%
\begin{center}
\begin{tabular}{l}
|{\edef\jobname{\scantokens\expandafter{\jobname\noexpand}}|\\
|\def\redirectjob |\textit{prefix}|#1~~~{\gdef\jobname{|\textit{dest}|#1}}|\\
|\expandafter\redirectjob\jobname~~~}\input{\jobname}|
\end{tabular}
\end{center}

In an alternative approach,
child documents can be compiled by a specific command line
without additional code or specific definitions:
%
\begin{center}
|... -jobname "|\textit{target}|" "|[\textit{flags}]%
|\includeonly{|\textit{dest}|}\input{|\textit{main}|}"|
\end{center}
%

%%%%%%%%%%%%%%%%%%%%%%%%%%%%%%%%%%%%%%%%%%%%%%%%%%%%%%%%%%%%%%%%%%%%%%%%%%%%%%%%
%%%%%%%%%%%%%%%%%%%%%%%%%%%%%%%%%%%%%%%%%%%%%%%%%%%%%%%%%%%%%%%%%%%%%%%%%%%%%%%%
\section{Information}

%%%%%%%%%%%%%%%%%%%%%%%%%%%%%%%%%%%%%%%%%%%%%%%%%%%%%%%%%%%%%%%%%%%%%%%%%%%%%%%%
\subsection{Copyright}

Copyright \copyright{} 2017--2018 Niklas Beisert

This work may be distributed and/or modified under the
conditions of the \LaTeX{} Project Public License, either version 1.3
of this license or (at your option) any later version.
The latest version of this license is in
  \url{http://www.latex-project.org/lppl.txt}
and version 1.3 or later is part of all distributions of \LaTeX{}
version 2005/12/01 or later.

This work has the LPPL maintenance status `maintained'.

The Current Maintainer of this work is Niklas Beisert.

This work consists of the files |README.txt|, |childdoc.ins| and |childdoc.dtx|
as well as the derived files |childdoc.def|, |cdocsamp.tex|
with |cdocsch1.tex|, |cdocsch2.tex|, |cdocspt3.tex|, |cdocspt4.tex|,
|cdocsdrf.tex|, |cdocsfn1.tex|, |cdocsfn2.tex|
as well as |childdoc.pdf|.

%%%%%%%%%%%%%%%%%%%%%%%%%%%%%%%%%%%%%%%%%%%%%%%%%%%%%%%%%%%%%%%%%%%%%%%%%%%%%%%%
\subsection{Files and Installation}

The package consists of the files:
%
\begin{center}
\begin{tabular}{ll}
    |README.txt|   & readme file \\
    |childdoc.ins| & installation file \\
    |childdoc.dtx| & source file \\
    |childdoc.def| & definition file \\
    |cdocsamp.tex| & sample main file \\
    |cdocsch1.tex| & sample include file \\
    |cdocsch2.tex| & sample include file \\
    |cdocspt3.tex| & sample part file \\
    |cdocspt4.tex| & sample part file \\
    |cdocsdrf.tex| & sample redirection file \\
    |cdocsfn1.tex| & sample redirection file \\
    |cdocsfn2.tex| & sample redirection file \\
    |childdoc.pdf| & manual
\end{tabular}
\end{center}
%
The distribution consists of the files
|README.txt|, |childdoc.ins| and |childdoc.dtx|.
%
\begin{itemize}
\item
Run (pdf)\LaTeX{} on |childdoc.dtx|
to compile the manual |childdoc.pdf| (this file).
\item
Run \LaTeX{} on |childdoc.ins| to create the definitions file |childdoc.def|
and the sample |cdocsamp.tex| with include files
|cdocsch1.tex|, |cdocsch2.tex|, |cdocspt3.tex|, |cdocspt4.tex|,
|cdocsdrf.tex|, |cdocsfn1.tex|, |cdocsfn2.tex|.
Then copy the file |childdoc.def| to an appropriate directory of your \LaTeX{}
distribution, e.g.\ \textit{texmf-root}|/tex/latex/childdoc|.
\end{itemize}

%%%%%%%%%%%%%%%%%%%%%%%%%%%%%%%%%%%%%%%%%%%%%%%%%%%%%%%%%%%%%%%%%%%%%%%%%%%%%%%%
\subsection{Related CTAN Packages}

There are several other packages which offer a similar functionality:
%
\begin{itemize}
\item
The packages
\href{http://ctan.org/pkg/docmute}{\textsf{docmute}},
\href{http://ctan.org/pkg/includex}{\textsf{includex}} and
\href{http://ctan.org/pkg/standalone}{\textsf{standalone}}
provide commands to include only the document body of
a child file thus allowing both files to be compiled individually.
\item
The packages \href{http://ctan.org/pkg/subdocs}{\textsf{subdocs}}
and \href{http://ctan.org/pkg/subfiles}{\textsf{subfiles}}
provide structures in which the main and child documents can be
encapsulated and allowing them to be compiled individually.
The inclusion mechanism is different from the conventional |\include|.
\item
The package \href{http://ctan.org/pkg/combine}{\textsf{combine}}
is an elaborate solution to combine several documents into one.
\end{itemize}
%
See also the CTAN topic \href{http://ctan.org/topic/subdocs}{\textsf{subdocs}}
for further related packages.
The present package differs from the above solutions in that
a document structure constructed with the conventional |\include| mechanism
just needs two extra commands at the top of every file
such that all constituent files can be compiled individually.

%%%%%%%%%%%%%%%%%%%%%%%%%%%%%%%%%%%%%%%%%%%%%%%%%%%%%%%%%%%%%%%%%%%%%%%%%%%%%%%%
%\subsection{Feature Suggestions}
%
%The following is a list of features which may be useful for future
%versions of this package:
%%
%\begin{itemize}
%\item
%\ldots
%\end{itemize}

%%%%%%%%%%%%%%%%%%%%%%%%%%%%%%%%%%%%%%%%%%%%%%%%%%%%%%%%%%%%%%%%%%%%%%%%%%%%%%%%
\subsection{Revision History}

%%%%%%%%%%%%%%%%%%%%%%%%%%%%%%%%%%%%%%%%
\paragraph{v2.0:} 2018/12/30

\begin{itemize}
\item
immediate forward processing
\item
added |\childdocby| mechanism
\item
manual restructured
\end{itemize}

%%%%%%%%%%%%%%%%%%%%%%%%%%%%%%%%%%%%%%%%
\paragraph{v1.6:} 2018/01/17

\begin{itemize}
\item
application for development of include files
\item
corrections to manual
\end{itemize}

%%%%%%%%%%%%%%%%%%%%%%%%%%%%%%%%%%%%%%%%
\paragraph{v1.5:} 2017/05/21

\begin{itemize}
\item
more complete structuring introduced
\item
|\childdocof| introduced
\item
|\childdoc| renamed to |\childdocmain|
\item
|\childredirect| renamed to |\childdocforward| and |\childdocforwardprefix|
and functionality expanded
\end{itemize}

%%%%%%%%%%%%%%%%%%%%%%%%%%%%%%%%%%%%%%%%
\paragraph{v1.0:} 2017/04/27

\begin{itemize}
\item
manual and install package
\item
first version published on CTAN
\end{itemize}

%%%%%%%%%%%%%%%%%%%%%%%%%%%%%%%%%%%%%%%%
\paragraph{v0.6:} 2017/04/26

\begin{itemize}
\item
redirection mechanism added
\end{itemize}

%%%%%%%%%%%%%%%%%%%%%%%%%%%%%%%%%%%%%%%%
\paragraph{v0.5:} 2017/04/26

\begin{itemize}
\item
functionality in definition file
\end{itemize}


%%%%%%%%%%%%%%%%%%%%%%%%%%%%%%%%%%%%%%%%%%%%%%%%%%%%%%%%%%%%%%%%%%%%%%%%%%%%%%%%
%%%%%%%%%%%%%%%%%%%%%%%%%%%%%%%%%%%%%%%%%%%%%%%%%%%%%%%%%%%%%%%%%%%%%%%%%%%%%%%%
%%%%%%%%%%%%%%%%%%%%%%%%%%%%%%%%%%%%%%%%%%%%%%%%%%%%%%%%%%%%%%%%%%%%%%%%%%%%%%%%
\appendix

\settowidth\MacroIndent{\rmfamily\scriptsize 000\ }

 \DocInput{childdoc.dtx}

\end{document}
%</driver>
% \fi
%
% %%%%%%%%%%%%%%%%%%%%%%%%%%%%%%%%%%%%%%%%%%%%%%%%%%%%%%%%%%%%%%%%%%%%%%%%%%%%%%
% %%%%%%%%%%%%%%%%%%%%%%%%%%%%%%%%%%%%%%%%%%%%%%%%%%%%%%%%%%%%%%%%%%%%%%%%%%%%%%
% \section{Sample}
%\iffalse
%<*samplemain>
%\fi
%
% The following presents a sample document
% with two chapters, two parts, a title page,
% a compile flag as well as three forwarding files to set the flag.
% It consists of eight |.tex| files:
% \begin{center}
% \begin{tabular}{ll}
% |cdocsamp.tex|&main file\\
% |cdocsch1.tex|&include file for chapter 1\\
% |cdocsch2.tex|&include file for chapter 2\\
% |cdocspt3.tex|&include file for part 3\\
% |cdocspt4.tex|&include file for part 4\\
% |cdocsdrf.tex|&forwarding file for main file in draft mode\\
% |cdocsfi1.tex|&forwarding file for final version of chapter 1\\
% |cdocsfi2.tex|&forwarding file for final version of chapter 2\\
% \end{tabular}
% \end{center}
% Each of the eight files can be compiled directly by the \LaTeX{} compiler.
%
% %%%%%%%%%%%%%%%%%%%%%%%%%%%%%%%%%%%%%%
% \paragraph{Main File.}
%
% The main file is called |cdocsamp.tex|.
%
% Load the \textsf{childdoc} definitions and
% declare the filename for the main document:
%    \begin{macrocode}
\input{childdoc.def}
\childdocmain{}
%    \end{macrocode}

% Optional override for |\version| flag:
%    \begin{macrocode}
%%\ifchilddoc\else\providecommand{\version}{draft}\fi
%    \end{macrocode}

% Define the default values for the |\version| flag
% (|final| for the main file and |draft| for childs):
%    \begin{macrocode}
\ifchilddoc
\providecommand{\version}{draft}
\else
\providecommand{\version}{final}
\fi
%    \end{macrocode}

% Load the standard document class:
%    \begin{macrocode}
\documentclass[12pt]{article}
%    \end{macrocode}

% Start the document body:
%    \begin{macrocode}
\begin{document}
%    \end{macrocode}

% Declare a title page.
% Print title, part of document being processed and version flag:
%    \begin{macrocode}
\addtocounter{page}{-1}
\begin{center}
{\LARGE\bfseries{}childdoc example\par}
\vspace{1cm}
\ifchilddoc
\ifchilddocmanual part\else chapter\fi:
`\childdocname' of `\childdocjob'\par
\else
main document: `\childdocjob'\par
\fi
version: \version\par
\end{center}
\newpage
%    \end{macrocode}

% Manually include selected file,
% otherwise process as usual:
%    \begin{macrocode}
\ifchilddocmanual
\section*{part `\childdocname'}
\input{\childdocname}
\else
%    \end{macrocode}

% Include the two chapters:
%    \begin{macrocode}
\include{cdocsch1}
\include{cdocsch2}
%    \end{macrocode}

% Include the two parts unless only chapters should be displayed:
%    \begin{macrocode}
\ifchilddoc\else
\section{part three}
\input{cdocspt3}
\section{part four}
\input{cdocspt4}
\fi
%    \end{macrocode}

% Process as usual until here:
%    \begin{macrocode}
\fi
%    \end{macrocode}

% End of document body:
%    \begin{macrocode}
\end{document}
%    \end{macrocode}
%\iffalse
%</samplemain>
%\fi
%
% %%%%%%%%%%%%%%%%%%%%%%%%%%%%%%%%%%%%%%
% \paragraph{Chapter Include Files.}
%
% The include files are called |cdocsch1.tex| and |cdocsch2.tex|.
%
%\iffalse
%<*samplechap1|samplechap2>
%\fi

% Optional override for |\version| flag:
%    \begin{macrocode}
%%\providecommand{\version}{final}
%    \end{macrocode}

% Include the main document:
%    \begin{macrocode}
\input{childdoc.def}
\childdocof{cdocsamp}
%    \end{macrocode}

%\iffalse
%</samplechap1|samplechap2>
%\fi
%
%\iffalse
%<*samplechap1>
%\fi
% Some text for chapter 1:
%    \begin{macrocode}
\section{one}
some text in chapter one
%    \end{macrocode}

%\iffalse
%</samplechap1>
%\fi
% Some text for chapter 2:
%\iffalse
%<*samplechap2>
%\fi
%    \begin{macrocode}
\section{two}
more text in chapter two
%    \end{macrocode}

%\iffalse
%</samplechap2>
%\fi
%
% %%%%%%%%%%%%%%%%%%%%%%%%%%%%%%%%%%%%%%
% \paragraph{Part Include Files.}
%
% The include files are called |cdocspt3.tex| and |cdocspt4.tex|.
%
%\iffalse
%<*samplepart3|samplepart4>
%\fi

% Optional override for |\version| flag:
%    \begin{macrocode}
%%\providecommand{\version}{final}
%    \end{macrocode}

% Include the main document:
%    \begin{macrocode}
\input{childdoc.def}
\childdocby{cdocsamp}
%    \end{macrocode}

%\iffalse
%</samplepart3|samplepart4>
%\fi
%
%\iffalse
%<*samplepart3>
%\fi
% Some text for part 3:
%    \begin{macrocode}
some text in part three
%    \end{macrocode}

%\iffalse
%</samplepart3>
%\fi
% Some text for part 4:
%\iffalse
%<*samplepart4>
%\fi
%    \begin{macrocode}
more text in part four
%    \end{macrocode}

%\iffalse
%</samplepart4>
%\fi
%
% %%%%%%%%%%%%%%%%%%%%%%%%%%%%%%%%%%%%%%
% \paragraph{Forwarding for a Complete Draft.}
%
% The following forwarding file |cdocsdrf.tex|
% compiles the main document in draft mode:
%\iffalse
%<*sampledraft>
%\fi
%    \begin{macrocode}
\def\version{draft}
\input{childdoc.def}
\childdocforward{cdocsamp}
%    \end{macrocode}

%\iffalse
%</sampledraft>
%\fi
%
% %%%%%%%%%%%%%%%%%%%%%%%%%%%%%%%%%%%%%%
% \paragraph{Forwarding for Final Version of the Chapters.}
%
% The following forwarding files |cdocsfn1.tex| and |cdocsfn2.tex|
% (with identical content)
% compile the final versions of the child documents
% |cdocsch1.tex| and |cdocsch2.tex|, respectively:
%\iffalse
%<*samplefinal>
%\fi
%    \begin{macrocode}
\def\version{final}
\input{childdoc.def}
\childdocforwardprefix[cdocsamp]{cdocsfn}{cdocsch}
%    \end{macrocode}

%\iffalse
%</samplefinal>
%\fi
%
% %%%%%%%%%%%%%%%%%%%%%%%%%%%%%%%%%%%%%%
% \paragraph{Command Line Processing.}
%
% The following three command lines generate the output files
% |cdocscld|, |cdocscl1| and |cdocscl2|
% which should be identical to
% |cdocsdrf|, |cdocsch1| and |cdocsfn2|, respectively:
% \begin{center}
% \begin{tabular}{l}
% |latex -jobname cdocscld \|\\
% |  "\def\version{draft}\input{childdoc.def}\childdocforward{cdocsamp}"|\\
% |latex -jobname cdocscl1 \|\\
% |  "\input{childdoc.def}\childdocforward[cdocsamp]{cdocsch1}"|\\
% |latex -jobname cdocscl2 \|\\
% |  "\def\version{final}\input{childdoc.def}\childdocforward{cdocsch2}"|
% \end{tabular}
% \end{center}
% Note that the trailing backslash on each first line
% merely continues the input to the second line
% (for convenient cut ant paste).
% Furthermore, the command |latex| can be replaced by any
% of its alternative versions such as |pdflatex|.
%
% %%%%%%%%%%%%%%%%%%%%%%%%%%%%%%%%%%%%%%%%%%%%%%%%%%%%%%%%%%%%%%%%%%%%%%%%%%%%%%
% %%%%%%%%%%%%%%%%%%%%%%%%%%%%%%%%%%%%%%%%%%%%%%%%%%%%%%%%%%%%%%%%%%%%%%%%%%%%%%
% \section{Implementation}
%\iffalse
%<*package>
%\fi
%
% This section describes the definitions file |childdoc.def|.

% The definitions cannot be loaded using |\usepackage| or |\RequirePackage|
% which has a mechanism to prevent loading a style file more than once.
% When loading the definitions by means of |\input|
% multiple instances have to be prevented manually:
%\iffalse
%This code needs to be before the `\ProvidesFile' directive
%which is defined at the beginning of this file.
%Therefore it is also placed there and commented out here.
%</package>
%<*discard>
%\fi
%    \begin{macrocode}
\ifdefined\childdocmain\endinput\fi
%    \end{macrocode}
%\iffalse
%</discard>
%<*package>
%\fi
%
% \macro{\ifchilddoc}
% \macro{\ifchilddocmanual}
% The conditional |\ifchilddoc| tells whether a
% child (true) or main (false) document is being compiled.
% The conditional |\ifchilddocmanual| tells whether
% the |\includeonly| mechanism is used (false) or
% the selection of child files must be performed manually (true).
% The definitions initialise to false:
%    \begin{macrocode}
\newif\ifchilddoc
\newif\ifchilddocmanual
%    \end{macrocode}

% \macro{\childdocname}
% \macro{\childdocjob}
% The macro |\childdocname| stores the name of the main document
% to be compiled. The macro |\childdocjob| stores the name of
% the document on which the \LaTeX{} compiler was originally invoked.
% The content of |\jobname| cannot be compared
% to filenames specified in the source due to different catcodes.
% The following code rescans |\jobname|, stores the result
% in |\childdocname| and saves a copy in |\childdocjob|:
%    \begin{macrocode}
\edef\childdocname{\scantokens\expandafter{\jobname\noexpand}}
\let\childdocjob\childdocname
%    \end{macrocode}

% \macro{\childdocdisable}
% The macro |\childdocdisable| prevents the main file
% from being processed more than once.
% At this stage, the main document command |\childdocmain|
% is assumed to be called once again where it should do nothing.
% Any subsequent call to it should prevent
% a secondary processing of the main document
% It overwrites the forwarding commands
% |\childdocof| and |\childdocforward|
% with empty macros to prevent further inclusions of the main document:
%    \begin{macrocode}
\newcommand{\childdocdisable}
{
  \renewcommand{\childdocmain}[1]{\renewcommand{\childdocmain}[1]{\endinput}}
  \renewcommand{\childdocof}[1]{}
  \renewcommand{\childdocby}[2][]{}
  \renewcommand{\childdocforward}[2][]{}
  \renewcommand{\childdocdisable}{}
}
%    \end{macrocode}

% \macro{\childdocmain}
% The macro |\childdocmain| is to be called at the top of the main file
% with nothing or the main filename (without extension) as argument.
% First, it breaks loops.
% If the argument is not empty and does not match |\childdocname|
% (which is set by the first inclusion of |childdoc.def|),
% |\ifchilddoc| is set to true, |\includeonly| is applied to the child file
% and |\jobname| is set to the main file
% (for proper handling of |.aux| files):
%    \begin{macrocode}
\newcommand{\childdocmain}[1]
{
  \childdocdisable\childdocmain{}
  \if?#1?\else
    \begingroup
      \def\childdoctmp{#1}
      \ifx\childdoctmp\childdocname
        \def\childdoctmp{}
      \else
        \def\childdoctmp
        {
          \childdoctrue
          \includeonly{\childdocname}
          \def\childdocjob{#1}
          \def\jobname{#1}
        }
      \fi
      \expandafter
    \endgroup
    \childdoctmp
  \fi
}
%    \end{macrocode}

% \macro{\childdocof}
% The command |\childdocof| redirects
% compilation to the main file |#1|.
%    \begin{macrocode}
\newcommand{\childdocof}[1]
{
  \childdocdisable
  \childdoctrue
  \includeonly{\childdocname}
  \def\jobname{#1}
  \def\childdocjob{#1}
  \input{#1}
}
%    \end{macrocode}

% \macro{\childdocby}
% The command |\childdocby| ....
%    \begin{macrocode}
\newcommand{\childdocby}[2][]
{
  \childdocdisable
  \childdoctrue
  \childdocmanualtrue
  \if?#1?\else
    \def\jobname{#2}
  \fi
  \def\childdocjob{#2}
  \input{#2}
  \endinput
}
%    \end{macrocode}

% \macro{\childdocforward}
% The command |\childdocforward| redirects
% compilation to the main file or
% (if the optional argument is given) a child file.
% Parameters are set as if the main file
% or a child file starting with |\childdocof| was compiled.
% Then compilation is handed over to the main file:
%    \begin{macrocode}
\newcommand{\childdocforward}[2][]
{
  \begingroup
    \if?#1?
      \def\childdoctmp
      {
        \def\childdocname{#2}
        \def\childdocjob{#2}
        \def\jobname{#2}
        \input{#2}
        \endinput
      }
    \else
      \def\childdoctmp
      {
        \childdocdisable
        \def\childdocname{#2}
        \childdoctrue
        \includeonly{#2}
        \def\childdocjob{#1}
        \def\jobname{#1}
        \input{#1}
        \endinput
      }
    \fi
    \expandafter
  \endgroup
  \childdoctmp
}
%    \end{macrocode}

% \macro{\childdocforwardprefix}
% The command |\childdocforwardprefix| redirects
% compilation to the main or a child file by means of a pattern.
% The prefix |#1| in the current filename is replaced by |#2|
% and the suffix of the current filename is kept
% (it is assumed that the filename does not contain the substring `|~~~|'
% which is used as a delimiter).
% Compilation is handed over to the new file by |\childdocforward|:
%    \begin{macrocode}
\newcommand{\childdocforwardprefix}[3][]
{
  \begingroup
    \def\childdocextract #2##1~~~{\def\childdoctmp{\childdocforward[#1]{#3##1}}}
    \expandafter\childdocextract\childdocname~~~
    \expandafter
  \endgroup
  \childdoctmp
}
%    \end{macrocode}

% \macro{\childdoc}
% The deprecated macro |\childdoc| is a legacy version of |\childdocmain|:
%    \begin{macrocode}
\newcommand{\childdoc}{\childdocmain}
%    \end{macrocode}

% \macro{\childdocredirect}
% The deprecated macro |\childdocredirect| is a legacy version
% of |\childdocforward| and |\childdocforwardprefix|:
%    \begin{macrocode}
\newcommand{\childdocredirect}[2][]
{
  \begingroup
    \if?#1?
      \def\childdoctmp{\childdocforward{#2}}
    \else
      \def\childdoctmp{\childdocforwardprefix{#1}{#2}}
    \fi
    \expandafter
  \endgroup
  \childdoctmp
}
%    \end{macrocode}

%\iffalse
%</package>
%\fi
%
\endinput
|
and perform the replacements as outlined below.
Instead of |\childdocmain{|\textit{main}|}| add the following code
to the top of the main file:
%
\begin{center}
\begin{tabular}{l}
|\||ifdefined\childdocname\endinput\||fi\newif\ifchilddoc|\\
|\edef\childdocname{\scantokens\expandafter{\jobname\noexpand}}|\\
|\def\childdocmain{|\textit{main}|}\||ifx\childdocmain\childdocname\||else|\\
|\childdoctrue\includeonly{\childdocname}\let\jobname\childdocmain\||fi|\\
\end{tabular}
\end{center}
%
Instead of |\childdocof{|\textit{main}|}| just include the main file
at the top of each child file:
%
\begin{center}
|\input{|\textit{main}|}|
\end{center}
%
A simple redirection |\childdocforward{|\textit{dest}|}| is achieved by:
%
\begin{center}
|\def\jobname{|\textit{dest}|}\input{\jobname}|
\end{center}
%
The redirection with prefix
|\childdocforwardprefix[|\textit{prefix}|]{|\textit{dest}|}|
is accomplished by:
%
\begin{center}
\begin{tabular}{l}
|{\edef\jobname{\scantokens\expandafter{\jobname\noexpand}}|\\
|\def\redirectjob |\textit{prefix}|#1~~~{\gdef\jobname{|\textit{dest}|#1}}|\\
|\expandafter\redirectjob\jobname~~~}\input{\jobname}|
\end{tabular}
\end{center}

In an alternative approach,
child documents can be compiled by a specific command line
without additional code or specific definitions:
%
\begin{center}
|... -jobname "|\textit{target}|" "|[\textit{flags}]%
|\includeonly{|\textit{dest}|}\input{|\textit{main}|}"|
\end{center}
%

%%%%%%%%%%%%%%%%%%%%%%%%%%%%%%%%%%%%%%%%%%%%%%%%%%%%%%%%%%%%%%%%%%%%%%%%%%%%%%%%
%%%%%%%%%%%%%%%%%%%%%%%%%%%%%%%%%%%%%%%%%%%%%%%%%%%%%%%%%%%%%%%%%%%%%%%%%%%%%%%%
\section{Information}

%%%%%%%%%%%%%%%%%%%%%%%%%%%%%%%%%%%%%%%%%%%%%%%%%%%%%%%%%%%%%%%%%%%%%%%%%%%%%%%%
\subsection{Copyright}

Copyright \copyright{} 2017--2018 Niklas Beisert

This work may be distributed and/or modified under the
conditions of the \LaTeX{} Project Public License, either version 1.3
of this license or (at your option) any later version.
The latest version of this license is in
  \url{http://www.latex-project.org/lppl.txt}
and version 1.3 or later is part of all distributions of \LaTeX{}
version 2005/12/01 or later.

This work has the LPPL maintenance status `maintained'.

The Current Maintainer of this work is Niklas Beisert.

This work consists of the files |README.txt|, |childdoc.ins| and |childdoc.dtx|
as well as the derived files |childdoc.def|, |cdocsamp.tex|
with |cdocsch1.tex|, |cdocsch2.tex|, |cdocspt3.tex|, |cdocspt4.tex|,
|cdocsdrf.tex|, |cdocsfn1.tex|, |cdocsfn2.tex|
as well as |childdoc.pdf|.

%%%%%%%%%%%%%%%%%%%%%%%%%%%%%%%%%%%%%%%%%%%%%%%%%%%%%%%%%%%%%%%%%%%%%%%%%%%%%%%%
\subsection{Files and Installation}

The package consists of the files:
%
\begin{center}
\begin{tabular}{ll}
    |README.txt|   & readme file \\
    |childdoc.ins| & installation file \\
    |childdoc.dtx| & source file \\
    |childdoc.def| & definition file \\
    |cdocsamp.tex| & sample main file \\
    |cdocsch1.tex| & sample include file \\
    |cdocsch2.tex| & sample include file \\
    |cdocspt3.tex| & sample part file \\
    |cdocspt4.tex| & sample part file \\
    |cdocsdrf.tex| & sample redirection file \\
    |cdocsfn1.tex| & sample redirection file \\
    |cdocsfn2.tex| & sample redirection file \\
    |childdoc.pdf| & manual
\end{tabular}
\end{center}
%
The distribution consists of the files
|README.txt|, |childdoc.ins| and |childdoc.dtx|.
%
\begin{itemize}
\item
Run (pdf)\LaTeX{} on |childdoc.dtx|
to compile the manual |childdoc.pdf| (this file).
\item
Run \LaTeX{} on |childdoc.ins| to create the definitions file |childdoc.def|
and the sample |cdocsamp.tex| with include files
|cdocsch1.tex|, |cdocsch2.tex|, |cdocspt3.tex|, |cdocspt4.tex|,
|cdocsdrf.tex|, |cdocsfn1.tex|, |cdocsfn2.tex|.
Then copy the file |childdoc.def| to an appropriate directory of your \LaTeX{}
distribution, e.g.\ \textit{texmf-root}|/tex/latex/childdoc|.
\end{itemize}

%%%%%%%%%%%%%%%%%%%%%%%%%%%%%%%%%%%%%%%%%%%%%%%%%%%%%%%%%%%%%%%%%%%%%%%%%%%%%%%%
\subsection{Related CTAN Packages}

There are several other packages which offer a similar functionality:
%
\begin{itemize}
\item
The packages
\href{http://ctan.org/pkg/docmute}{\textsf{docmute}},
\href{http://ctan.org/pkg/includex}{\textsf{includex}} and
\href{http://ctan.org/pkg/standalone}{\textsf{standalone}}
provide commands to include only the document body of
a child file thus allowing both files to be compiled individually.
\item
The packages \href{http://ctan.org/pkg/subdocs}{\textsf{subdocs}}
and \href{http://ctan.org/pkg/subfiles}{\textsf{subfiles}}
provide structures in which the main and child documents can be
encapsulated and allowing them to be compiled individually.
The inclusion mechanism is different from the conventional |\include|.
\item
The package \href{http://ctan.org/pkg/combine}{\textsf{combine}}
is an elaborate solution to combine several documents into one.
\end{itemize}
%
See also the CTAN topic \href{http://ctan.org/topic/subdocs}{\textsf{subdocs}}
for further related packages.
The present package differs from the above solutions in that
a document structure constructed with the conventional |\include| mechanism
just needs two extra commands at the top of every file
such that all constituent files can be compiled individually.

%%%%%%%%%%%%%%%%%%%%%%%%%%%%%%%%%%%%%%%%%%%%%%%%%%%%%%%%%%%%%%%%%%%%%%%%%%%%%%%%
%\subsection{Feature Suggestions}
%
%The following is a list of features which may be useful for future
%versions of this package:
%%
%\begin{itemize}
%\item
%\ldots
%\end{itemize}

%%%%%%%%%%%%%%%%%%%%%%%%%%%%%%%%%%%%%%%%%%%%%%%%%%%%%%%%%%%%%%%%%%%%%%%%%%%%%%%%
\subsection{Revision History}

%%%%%%%%%%%%%%%%%%%%%%%%%%%%%%%%%%%%%%%%
\paragraph{v2.0:} 2018/12/30

\begin{itemize}
\item
immediate forward processing
\item
added |\childdocby| mechanism
\item
manual restructured
\end{itemize}

%%%%%%%%%%%%%%%%%%%%%%%%%%%%%%%%%%%%%%%%
\paragraph{v1.6:} 2018/01/17

\begin{itemize}
\item
application for development of include files
\item
corrections to manual
\end{itemize}

%%%%%%%%%%%%%%%%%%%%%%%%%%%%%%%%%%%%%%%%
\paragraph{v1.5:} 2017/05/21

\begin{itemize}
\item
more complete structuring introduced
\item
|\childdocof| introduced
\item
|\childdoc| renamed to |\childdocmain|
\item
|\childredirect| renamed to |\childdocforward| and |\childdocforwardprefix|
and functionality expanded
\end{itemize}

%%%%%%%%%%%%%%%%%%%%%%%%%%%%%%%%%%%%%%%%
\paragraph{v1.0:} 2017/04/27

\begin{itemize}
\item
manual and install package
\item
first version published on CTAN
\end{itemize}

%%%%%%%%%%%%%%%%%%%%%%%%%%%%%%%%%%%%%%%%
\paragraph{v0.6:} 2017/04/26

\begin{itemize}
\item
redirection mechanism added
\end{itemize}

%%%%%%%%%%%%%%%%%%%%%%%%%%%%%%%%%%%%%%%%
\paragraph{v0.5:} 2017/04/26

\begin{itemize}
\item
functionality in definition file
\end{itemize}


%%%%%%%%%%%%%%%%%%%%%%%%%%%%%%%%%%%%%%%%%%%%%%%%%%%%%%%%%%%%%%%%%%%%%%%%%%%%%%%%
%%%%%%%%%%%%%%%%%%%%%%%%%%%%%%%%%%%%%%%%%%%%%%%%%%%%%%%%%%%%%%%%%%%%%%%%%%%%%%%%
%%%%%%%%%%%%%%%%%%%%%%%%%%%%%%%%%%%%%%%%%%%%%%%%%%%%%%%%%%%%%%%%%%%%%%%%%%%%%%%%
\appendix

\settowidth\MacroIndent{\rmfamily\scriptsize 000\ }

 \DocInput{childdoc.dtx}

\end{document}
%</driver>
% \fi
%
% %%%%%%%%%%%%%%%%%%%%%%%%%%%%%%%%%%%%%%%%%%%%%%%%%%%%%%%%%%%%%%%%%%%%%%%%%%%%%%
% %%%%%%%%%%%%%%%%%%%%%%%%%%%%%%%%%%%%%%%%%%%%%%%%%%%%%%%%%%%%%%%%%%%%%%%%%%%%%%
% \section{Sample}
%\iffalse
%<*samplemain>
%\fi
%
% The following presents a sample document
% with two chapters, two parts, a title page,
% a compile flag as well as three forwarding files to set the flag.
% It consists of eight |.tex| files:
% \begin{center}
% \begin{tabular}{ll}
% |cdocsamp.tex|&main file\\
% |cdocsch1.tex|&include file for chapter 1\\
% |cdocsch2.tex|&include file for chapter 2\\
% |cdocspt3.tex|&include file for part 3\\
% |cdocspt4.tex|&include file for part 4\\
% |cdocsdrf.tex|&forwarding file for main file in draft mode\\
% |cdocsfi1.tex|&forwarding file for final version of chapter 1\\
% |cdocsfi2.tex|&forwarding file for final version of chapter 2\\
% \end{tabular}
% \end{center}
% Each of the eight files can be compiled directly by the \LaTeX{} compiler.
%
% %%%%%%%%%%%%%%%%%%%%%%%%%%%%%%%%%%%%%%
% \paragraph{Main File.}
%
% The main file is called |cdocsamp.tex|.
%
% Load the \textsf{childdoc} definitions and
% declare the filename for the main document:
%    \begin{macrocode}
% \iffalse
%
% childdoc.dtx Copyright (C) 2017-2018 Niklas Beisert
%
% This work may be distributed and/or modified under the
% conditions of the LaTeX Project Public License, either version 1.3
% of this license or (at your option) any later version.
% The latest version of this license is in
%   http://www.latex-project.org/lppl.txt
% and version 1.3 or later is part of all distributions of LaTeX
% version 2005/12/01 or later.
%
% This work has the LPPL maintenance status `maintained'.
%
% The Current Maintainer of this work is Niklas Beisert.
%
% This work consists of the files childdoc.dtx and childdoc.ins
% and the derived files childdoc.def and cdocsamp.tex with
% cdocsch1.tex, cdocsch2.tex, cdocsdrf.tex, cdocsfn1.tex, cdocsfn2.tex.
%
%<package>\ifdefined\childdocmain\endinput\fi
%<package>\ProvidesFile{childdoc.def}[2018/12/30 v2.0 child document driver]
%<samplemain>\ProvidesFile{cdocsamp.tex}[2018/12/30 v2.0 sample for childdoc]
%<*driver>
%\ProvidesFile{childdoc.drv}[2018/12/30 v2.0 childdoc reference manual file]
\PassOptionsToClass{10pt,a4paper}{article}
\documentclass{ltxdoc}

\usepackage[margin=35mm]{geometry}
\usepackage{hyperref}
\usepackage{hyperxmp}
\usepackage[usenames]{color}

\hypersetup{colorlinks=true}
\hypersetup{pdfstartview=FitH}
\hypersetup{pdfpagemode=UseNone}
\hypersetup{pdfsource={}}
\hypersetup{pdflang={en-UK}}
\hypersetup{pdfcopyright={Copyright 2017-2018 Niklas Beisert.
  This work may be distributed and/or modified under the
  conditions of the LaTeX Project Public License, either version 1.3
  of this license or (at your option) any later version.}}
\hypersetup{pdflicenseurl={http://www.latex-project.org/lppl.txt}}
\hypersetup{pdfcontactaddress={ETH Zurich, ITP, HIT K,
  Wolfgang-Pauli-Strasse 27}}
\hypersetup{pdfcontactpostcode={8093}}
\hypersetup{pdfcontactcity={Zurich}}
\hypersetup{pdfcontactcountry={Switzerland}}
\hypersetup{pdfcontactemail={nbeisert@itp.phys.ethz.ch}}
\hypersetup{pdfcontacturl={http://people.phys.ethz.ch/\xmptilde nbeisert/}}

\newcommand{\secref}[1]{\hyperref[#1]{section \ref*{#1}}}

\parskip1ex
\parindent0pt
\let\olditemize\itemize
\def\itemize{\olditemize\parskip0pt}

\begin{document}

\title{The \textsf{childdoc} Package}
\hypersetup{pdftitle={The childdoc Package}}
\author{Niklas Beisert\\[2ex]
  Institut f\"ur Theoretische Physik\\
  Eidgen\"ossische Technische Hochschule Z\"urich\\
  Wolfgang-Pauli-Strasse 27, 8093 Z\"urich, Switzerland\\[1ex]
  \href{mailto:nbeisert@itp.phys.ethz.ch}
  {\texttt{nbeisert@itp.phys.ethz.ch}}}
\hypersetup{pdfauthor={Niklas Beisert}}
\hypersetup{pdfsubject={Manual for the LaTeX2e Package childdoc}}
\date{30 December 2018, \textsf{v2.0}}
\maketitle

\begin{abstract}\noindent
\textsf{childdoc} is a \LaTeXe{} package
that enables the direct compilation
of document sections included by |\include|
to individual files.
\end{abstract}

\begingroup
\parskip0ex
\tableofcontents
\endgroup

%%%%%%%%%%%%%%%%%%%%%%%%%%%%%%%%%%%%%%%%%%%%%%%%%%%%%%%%%%%%%%%%%%%%%%%%%%%%%%%%
%%%%%%%%%%%%%%%%%%%%%%%%%%%%%%%%%%%%%%%%%%%%%%%%%%%%%%%%%%%%%%%%%%%%%%%%%%%%%%%%
\section{Introduction}

\LaTeX{} provides a mechanism to structure a large document (such as a book)
into a main file and several child files (containing the chapters)
using the |\include| command.
This mechanism is beneficial for documents
which span hundreds of pages in order to
make the source file(s) more manageable.
Moreover, compilation can be restricted to
selected child files by means of the |\includeonly| command.
The latter feature can be used to reduce the compilation time while editing
(this was significantly more useful in the earlier days of \LaTeX{})
or to generate a smaller document which is easier to navigate.
Another application of |\includeonly| is to generate
documents consisting of selected parts of the complete document.

However, there are a few drawbacks of the plain |\include| mechanism:
\begin{itemize}
\item
The child files cannot be compiled on their own,
they can only be compiled via the main file.
A naive editing environment
(such as a text editor with an option
to have the current file processed by \LaTeX)
may require one to switch to the main file before compiling;
attempting to compile the child file produces errors.
\item
The main file must be modified (each time)
to adjust the |\includeonly| command
to the present needs. This easily leaves the main file in a messy state.
\item
The generated document will always carry the filename
of the main document. This is inconvenient if
several child files are to be compiled and
to be kept for distribution.
\end{itemize}

The present package provides a simple interface
to make child files individually compilable by \LaTeX{}.
Compiling a child file then has the same effect as compiling
the main file with an |\includeonly| command
to select the appropriate child.
Moreover the generated document will carry the name of the child
rather than the main file.
This resolves all three above issues.

This feature is meant to make the editing of books,
thesis documents and lecture notes somewhat more convenient.
However, the package can also be used efficiently for
composing a series of documents (such as exercise sheets)
which are typically distributed individually.
It then assists the author in generating the individual documents
(potentially in different versions)
as well as a document containing the collected series.
Another application is in developing style files
or other kinds of included material
where compilation of the style file could redirect
to a sample or test file.

%%%%%%%%%%%%%%%%%%%%%%%%%%%%%%%%%%%%%%%%%%%%%%%%%%%%%%%%%%%%%%%%%%%%%%%%%%%%%%%%
%%%%%%%%%%%%%%%%%%%%%%%%%%%%%%%%%%%%%%%%%%%%%%%%%%%%%%%%%%%%%%%%%%%%%%%%%%%%%%%%
\section{Usage}

First of all, the package \textsf{childdoc} is \emph{not} a standard
\LaTeXe{} |.sty| style file! Therefore it needs to be invoked in
a non-standard way.

%%%%%%%%%%%%%%%%%%%%%%%%%%%%%%%%%%%%%%%%%%%%%%%%%%%%%%%%%%%%%%%%%%%%%%%%%%%%%%%%
\subsection{Included Files}
\label{sec:include}

%%%%%%%%%%%%%%%%%%%%%%%%%%%%%%%%%%%%%%%%
\DescribeMacro{\childdocmain}
To use the package, add the commands
\begin{center}
\begin{tabular}{l}
|\input{childdoc.def}|\\
|\childdocmain{}|\\
\end{tabular}
\end{center}
at the very top of the main \LaTeX{} file,
in particular \emph{before} the |\documentclass| statement!
The argument of |\childdocmain| should be left empty
(but it must be present).

%%%%%%%%%%%%%%%%%%%%%%%%%%%%%%%%%%%%%%%%
\DescribeMacro{\childdocof}
Furthermore, add the commands
\begin{center}
\begin{tabular}{l}
|\input{childdoc.def}|\\
|\childdocof{|\textit{main}|}|\\
\end{tabular}
\end{center}
at the top of every child file \textit{child}
which is included by |\include{|\textit{child}|}|
from within the main file
(or at least for those files to be compiled individually).
The argument \textit{main} must be the filename of the main file.

There are a couple of
considerations in setting up the main and child documents:

%%%%%%%%%%%%%%%%%%%%%%%%%%%%%%%%%%%%%%%%
\paragraph{Restrictions.}

Please note the following restrictions:
\begin{itemize}
\item
|\childdocmain| must be called with one argument \textit{main}
to ensure compatibility with earlier version of the package.
It must either be empty (|\childdocmain{}|)
or precisely match the filename of the main file in which it is specified.
See \secref{sec:detection} for further information.
\item
The filename \textit{main} must be specified without the |.tex| extension.
\item
The filename \textit{main} is case sensitive
(even in case-insensitive file systems)
due to internal string comparison.
\item
The argument \textit{main} should be fully expanded, it cannot be a macro.
\item
Subdirectories and special characters should be avoided in filenames.
\item
The command |\childdocmain{|\textit{main}|}| must be followed by a whitespace.
It should not be followed immediately by another command
or by a comment mark `|%|'.
This is because the \TeX{} parser reads the token immediately following
the argument of |\childdocmain| and puts it
at the beginning of every child section;
however, a white\-space is ignored.
\end{itemize}

%%%%%%%%%%%%%%%%%%%%%%%%%%%%%%%%%%%%%%%%
\paragraph{Content of Main File.}

It is advisable to place all content in the child files included by |\include|.
Any output contained in the main file will appear in all child documents
unless suppressed manually;
it cannot be suppressed automatically by the |\includeonly| directive
and thus should normally be avoided.
A method to include some content in the main file
by means of conditional processing is described in \secref{sec:conditional}.

%%%%%%%%%%%%%%%%%%%%%%%%%%%%%%%%%%%%%%%%
\paragraph{Page Numbering.}

When only a part of the document is compiled,
the appropriate numbering of pages
(as well as other status parameters)
is determined from the |.aux| files.
The latter contain information from previous passes.
However this information needs to propagate through
all intermediate child documents.
Therefore the page numbering in child documents may well
be inconsistent until the complete document is compiled at least once.

A useful (if unconventional) way to always ensure a consistent
page numbering is to restart the numbering in each child document
and denote the pages by `\textit{child}|.|\textit{page}'
where \textit{child} represents the chapter/section number of the child file.
This can be achieved by the command
|\numberwithin{page}{|\textit{child}|}|
of the \textsf{amsmath} package
where \textit{child} can be |chapter| or |section|
depending on the chosen structuring.
Alternatively, one can modify the macro |\thepage| appropriately
and reset the counter |page| at the start of each child file.

%%%%%%%%%%%%%%%%%%%%%%%%%%%%%%%%%%%%%%%%%%%%%%%%%%%%%%%%%%%%%%%%%%%%%%%%%%%%%%%%
\subsection{Conditional Processing}
\label{sec:conditional}

The package provides a mechanism to compile different versions
of a document. To customise the versions further some conditional processing
can come in handy to distinguish which version is being compiled.
The package provides two macros to describe the compilation context:

%%%%%%%%%%%%%%%%%%%%%%%%%%%%%%%%%%%%%%%%
\DescribeMacro{\ifchilddoc}
The conditional |\ifchilddoc| distinguishes between the compilation of
child documents and the main document:
%
\begin{center}
|\ifchilddoc |\textit{child-code}| |[|\||else |\textit{main-code}]| \||fi|
\end{center}

%%%%%%%%%%%%%%%%%%%%%%%%%%%%%%%%%%%%%%%%
\DescribeMacro{\childdocname}
\DescribeMacro{\childdocjob}
The macro |\childdocname| contains the filename (without extension)
of the main or child file being processed.
Note that |\childdocjob| will always contain the name of the main file.

%%%%%%%%%%%%%%%%%%%%%%%%%%%%%%%%%%%%%%%%
\paragraph{Title Page.}

Conditional processing can be used to include a title or banner page
in the main document when proper precautions are taken.
Importantly, the code in the main file should ensure that the page counter
(as well as other status parameters which are stored in the |.aux| files)
takes the same value after the conditional processing.
Otherwise the page numbers may take divergent values
depending on which part is compiled.

For example, a title page could be declared by:
%
\begin{center}
\begin{tabular}{l}
|\ifchilddoc\||else|\\
|\addtocounter{page}{-1}|\\
\textit{code for title page}\\
|\newpage|\\
|\||fi|
\end{tabular}
\end{center}
%
A banner page for the child documents can be generated by:
%
\begin{center}
\begin{tabular}{l}
|\ifchilddoc|\\
|\addtocounter{page}{-1}|\\
\textit{code for banner page}\\
|\newpage|\\
|\||fi|
\end{tabular}
\end{center}
%
Here one could write a message such as:
\begin{center}
|This is the part \childdocname{} of \childdocjob{}.|
\end{center}

%%%%%%%%%%%%%%%%%%%%%%%%%%%%%%%%%%%%%%%%%%%%%%%%%%%%%%%%%%%%%%%%%%%%%%%%%%%%%%%%
\subsection{Flags}
\label{sec:flags}

The package makes it easy to generate different versions
of the main or child documents.
To this end compilation flags can be defined
and assigned different default values.
They will be particularly useful in conjunction
with the forwarding mechanism described in \secref{sec:forward}.

For example, it may be useful to have a flag |\version|
which can be set to |draft| or |final|.
The document source will contain some conditional code
depending on the value of |\version|.
Suppose further, the flag should default to |final| for the main file
and to |draft| for child files
which is a natural assignment for editing the document.
This is achieved by placing the following code
in the preamble of the main document
(below the |\childdocmain| directive):
%
\begin{center}
\begin{tabular}{l}
|\ifchilddoc|\\
|\providecommand{\version}{draft}|\\
|\||else|\\
|\providecommand{\version}{final}|\\
|\||fi|
\end{tabular}
\end{center}
%
The definition by |\providecommand| makes sure
that previous definitions are not overwritten.
Further statements |\providecommand{\version}{...}|
can thus be added before the above code to override it.

For the main file, one might add a line
(between |\childdocmain| and the above block)
%
\begin{center}
|%\ifchilddoc\||else\providecommand{\version}{draft}\||fi|
\end{center}
%
which can be uncommented to produce a draft version.
Likewise one can add a line to the very top of a child file
(above the |\childdocof{|\textit{main}|}| directive)
%
\begin{center}
|%\providecommand{\version}{final}|
\end{center}
%
which can be uncommented to produce the final version of this child document.

%%%%%%%%%%%%%%%%%%%%%%%%%%%%%%%%%%%%%%%%%%%%%%%%%%%%%%%%%%%%%%%%%%%%%%%%%%%%%%%%
\subsection{Forwarding}
\label{sec:forward}

Different versions of the main or child documents
using compilation flags as described in \secref{sec:flags}
can be (permanently) stored in different files
for convenient compilation, viewing and distribution.
To this end, the package defines a command
to pass on compilation to a different file:

%%%%%%%%%%%%%%%%%%%%%%%%%%%%%%%%%%%%%%%%
\DescribeMacro{\childdocforward}
The command |\childdocforward| redirects processing to
another source file:
%
\begin{center}
\begin{tabular}{l}
|\input{childdoc.def}|\\
|\childdocforward[|\textit{main}|]{|\textit{dest}|}|\\
\end{tabular}
\end{center}
%
The argument \textit{dest} is the destination file
(without extension).
It should be the main file or one of the child files.
Note that further \textsf{childdoc} directives
such as |\childdocof| and |\childdocforward|
in the indicated file will be processed in this form.
The optional argument \textit{main}
passes on directly to the main file \textit{main}
while pretending to compile the child \textit{dest}.
This form behaves as if \textit{dest}
issues |\childdocof{|\textit{main}|}| right away,
and no further \textsf{childdoc} directives will be processed.

%%%%%%%%%%%%%%%%%%%%%%%%%%%%%%%%%%%%%%%%
\DescribeMacro{\...prefix}
In the alternative form |\childdocforwardprefix|,
%
\begin{center}
\begin{tabular}{l}
|\input{childdoc.def}|\\
|\childdocforwardprefix[|\textit{main}|]{|\textit{prefix}|}{|\textit{dest}|}|
\end{tabular}
\end{center}
%
the destination file is determined by a pattern
depending on the current file:
To make this work, the current file must be called
`{\textit{prefix}\hspace{0.2em}\textit{suffix}}'
with \textit{prefix} matching precisely the argument.
Processing is then passed on to the file
`{\textit{dest}\hspace{0.2em}\textit{suffix}}'.
Surely, the same effect is achieved by
directly specifying the
argument `{\textit{dest}\hspace{0.2em}\textit{suffix}}'
in the first form.
However, that requires to set up a different file
for each child. With the alternative form of the command
all these files can have exactly the same content
which simplifies setting them up and maintaining them.

For example, the following file |draft.tex|
with a compilation flag |\version| as described in \secref{sec:flags}
compiles the main document as a draft:
%
\begin{center}
\begin{tabular}{l}
|\def\version{draft}|\\
|\input{childdoc.def}|\\
|\childdocforward{|\textit{main}|}|
\end{tabular}
\end{center}
%
Likewise, the following files |final|\textit{nn}|.tex|
compile the final version of the child document
|child|\textit{nn}|.tex|:
%
\begin{center}
\begin{tabular}{l}
|\def\version{final}|\\
|\input{childdoc.def}|\\
|\childdocforwardprefix{final}{child}|
\end{tabular}
\end{center}
%

Note that when several versions of a main file and/or of each child file
are to be generated, it may be convenient to set up a |Makefile| or
shell script to automatise the process.

%%%%%%%%%%%%%%%%%%%%%%%%%%%%%%%%%%%%%%%%%%%%%%%%%%%%%%%%%%%%%%%%%%%%%%%%%%%%%%%%
\subsection{Command Line Processing}
\label{sec:commandline}

The effect of redirection files can also be achieved by invoking
the \LaTeX{} compiler with a more elaborate command line.
Most conveniently this should be done as part
of a shell script or a |Makefile|.

When using \textsf{childdoc} in the main file, the following
command lines effectively perform a redirection
(note that depending on the shell being used,
backslashes may have to be doubled: `|\|' $\to$ `|\\|'):
%
\begin{center}
|... -jobname "|\textit{target}|" |\\|"|[\textit{flags}]%
|\input{childdoc.def}\childdocforward[|\textit{main}|]{|\textit{dest}|}"|
\end{center}
%
Here \textit{target} is the name of the output file,
\textit{main} is the name of the main file
and \textit{dest} is the name of the main or child file to be processed
(all filenames without extensions).
The optional argument \textit{main} can be omitted
if \textit{main} matches \textit{dest}.
Optionally, compilation \textit{flags} can be defined via |\def| commands.
This command line makes the \TeX{} engine believe
it is compiling the file \textit{target}
whose content is specified as the latter parameter.
The provided code then forwards the processing to
\textit{main} or \textit{dest} as described in \secref{sec:forward}.

%%%%%%%%%%%%%%%%%%%%%%%%%%%%%%%%%%%%%%%%%%%%%%%%%%%%%%%%%%%%%%%%%%%%%%%%%%%%%%%%
\subsection{Include by Input}
\label{sec:input}

Including child documents by |\include| has some restrictions by design.
Most notably, the content of a child document always occupies
its own set of pages; pages cannot be shared between child documents.
Usually, this behaviour makes perfect sense
because each child document contain an essential part of the document.
However, in some situations it may be desirable to compose
a document from a collection of parts
without having mandatory page breaks between then.
For this case, the package
provides a mechanism to include parts
by |\input| which can also be processed individually.
However, by construction this mechanism
requires manual handling of the content to be output.

%%%%%%%%%%%%%%%%%%%%%%%%%%%%%%%%%%%%%%%%
\DescribeMacro{\ifchilddocmanual}
The main file should be prepared as usual, see \secref{sec:include}.
However, the document body must make a distinction
between processing of an individual part and of the main document, e.g.:
%
\begin{center}
\begin{tabular}{l}
|\ifchilddocmanual|\\
|\input{\childdocname}|\\
|\||else|\\
\textit{document body with }|\input{|\textit{part}|}|\\
|\||fi|
\end{tabular}
\end{center}
%
The conditional |\ifchilddocmanual| is true whenever
a part to be included by |\input| is being compiled,
and the name of the part is stored in |\childdocname|.

%%%%%%%%%%%%%%%%%%%%%%%%%%%%%%%%%%%%%%%%
\DescribeMacro{\childdocby}
Each part to be included by |\input| should start with:
%
\begin{center}
\begin{tabular}{l}
|\input{childdoc.def}|\\
|\childdocby{|\textit{main}|}|\\
\end{tabular}
\end{center}
%
The directive |\childdocby| is similar to |\childdocof|
described in \secref{sec:include},
but the subsequent selection of content must be done manually.
To that end, both |\ifchilddoc| and |\ifchilddocmanual|
will be true upon processing of a part,
and the name of the part is stored in |\childdocname|.
Note that |\jobname| will be set to the filename of the current part
so that each part receives an individual |.aux| file
that does not interfere with the |.aux| file(s) of the main document.
This behaviour can be altered by the alternative form
|\childdocby[*]{|\textit{main}|}| (with a non-empty optional argument)
which uses the |.aux| file of the main document
by setting |\jobname| to \textit{main}.

%%%%%%%%%%%%%%%%%%%%%%%%%%%%%%%%%%%%%%%%%%%%%%%%%%%%%%%%%%%%%%%%%%%%%%%%%%%%%%%%
\subsection{Driver Development}
\label{sec:driver}

The \textsf{childdoc} mechanism can also be use for the development
of definition files such as \LaTeX{} styles or classes.
This case differs from the above setup with multiple parts
included by |\include| in that no |\includeonly| should be invoked.
This can be achieved by starting the include file
(before |\ProvidesPackage|) with:
%
\begin{center}
\begin{tabular}{l}
|\input{childdoc.def}|\\
|\childdocforward{|\textit{main}|}|\\
\end{tabular}
\end{center}
%
or alternatively with:
%
\begin{center}
\begin{tabular}{l}
|\input{childdoc.def}|\\
|\childdocby{|\textit{main}|}|\\
\end{tabular}
\end{center}
%
Both forms have slightly different effects as described above.
The main file is prepared as usual, see \secref{sec:include}.

%%%%%%%%%%%%%%%%%%%%%%%%%%%%%%%%%%%%%%%%%%%%%%%%%%%%%%%%%%%%%%%%%%%%%%%%%%%%%%%%
\subsection{Legacy Detection}
\label{sec:detection}

The directive |\childdocmain| in the main file can detect
whether the complete document or merely a child is to be compiled
even without using the directive |\childdocof|.
This method is deprecated because it is less robust
and there is no compelling reason to use it;
it is merely provided for backward compatibility
and it may be removed in future versions.

If the detection mechanism is to be used,
it is mandatory to correctly specify
the filename of the main file as the argument of |\childdocmain|:
%
\begin{center}
\begin{tabular}{l}
|\input{childdoc.def}|\\
|\childdocmain{|\textit{main}|}|\\
\end{tabular}
\end{center}
%
If |\jobname| does not match the argument \textit{main} of |\childdocmain|,
it is assumed that |\jobname| points to the child file to be compiled.
When using |\childdocmain| with the main file specified as argument,
it suffices to start a child file
with just |\input{|\textit{main}|}|
without loading of the package and using |\childdocof|.
If instead all processing is done
with the appropriate \textsf{childdoc} directives,
the argument of \textit{main} of |\childdocmain| can be empty.

An alternative version of the command line processing described
in \secref{sec:commandline} using the detection mechanism reads:
%
\begin{center}
|... -jobname "|\textit{target}|" "|[\textit{flags}]%
[|\def\jobname{|\textit{dest}|}|]|\input{|\textit{main}|}"|
\end{center}

%%%%%%%%%%%%%%%%%%%%%%%%%%%%%%%%%%%%%%%%%%%%%%%%%%%%%%%%%%%%%%%%%%%%%%%%%%%%%%%%
\subsection{Manual Code}
\label{sec:manual}

In case one cannot be certain whether the definitions file |childdoc.def|
is installed on the target \TeX{} distribution
and one prefers not to ship it,
it is conceivable to paste a few relevant commands into the sources.

To that end, drop all statements |\input{childdoc.def}|
and perform the replacements as outlined below.
Instead of |\childdocmain{|\textit{main}|}| add the following code
to the top of the main file:
%
\begin{center}
\begin{tabular}{l}
|\||ifdefined\childdocname\endinput\||fi\newif\ifchilddoc|\\
|\edef\childdocname{\scantokens\expandafter{\jobname\noexpand}}|\\
|\def\childdocmain{|\textit{main}|}\||ifx\childdocmain\childdocname\||else|\\
|\childdoctrue\includeonly{\childdocname}\let\jobname\childdocmain\||fi|\\
\end{tabular}
\end{center}
%
Instead of |\childdocof{|\textit{main}|}| just include the main file
at the top of each child file:
%
\begin{center}
|\input{|\textit{main}|}|
\end{center}
%
A simple redirection |\childdocforward{|\textit{dest}|}| is achieved by:
%
\begin{center}
|\def\jobname{|\textit{dest}|}\input{\jobname}|
\end{center}
%
The redirection with prefix
|\childdocforwardprefix[|\textit{prefix}|]{|\textit{dest}|}|
is accomplished by:
%
\begin{center}
\begin{tabular}{l}
|{\edef\jobname{\scantokens\expandafter{\jobname\noexpand}}|\\
|\def\redirectjob |\textit{prefix}|#1~~~{\gdef\jobname{|\textit{dest}|#1}}|\\
|\expandafter\redirectjob\jobname~~~}\input{\jobname}|
\end{tabular}
\end{center}

In an alternative approach,
child documents can be compiled by a specific command line
without additional code or specific definitions:
%
\begin{center}
|... -jobname "|\textit{target}|" "|[\textit{flags}]%
|\includeonly{|\textit{dest}|}\input{|\textit{main}|}"|
\end{center}
%

%%%%%%%%%%%%%%%%%%%%%%%%%%%%%%%%%%%%%%%%%%%%%%%%%%%%%%%%%%%%%%%%%%%%%%%%%%%%%%%%
%%%%%%%%%%%%%%%%%%%%%%%%%%%%%%%%%%%%%%%%%%%%%%%%%%%%%%%%%%%%%%%%%%%%%%%%%%%%%%%%
\section{Information}

%%%%%%%%%%%%%%%%%%%%%%%%%%%%%%%%%%%%%%%%%%%%%%%%%%%%%%%%%%%%%%%%%%%%%%%%%%%%%%%%
\subsection{Copyright}

Copyright \copyright{} 2017--2018 Niklas Beisert

This work may be distributed and/or modified under the
conditions of the \LaTeX{} Project Public License, either version 1.3
of this license or (at your option) any later version.
The latest version of this license is in
  \url{http://www.latex-project.org/lppl.txt}
and version 1.3 or later is part of all distributions of \LaTeX{}
version 2005/12/01 or later.

This work has the LPPL maintenance status `maintained'.

The Current Maintainer of this work is Niklas Beisert.

This work consists of the files |README.txt|, |childdoc.ins| and |childdoc.dtx|
as well as the derived files |childdoc.def|, |cdocsamp.tex|
with |cdocsch1.tex|, |cdocsch2.tex|, |cdocspt3.tex|, |cdocspt4.tex|,
|cdocsdrf.tex|, |cdocsfn1.tex|, |cdocsfn2.tex|
as well as |childdoc.pdf|.

%%%%%%%%%%%%%%%%%%%%%%%%%%%%%%%%%%%%%%%%%%%%%%%%%%%%%%%%%%%%%%%%%%%%%%%%%%%%%%%%
\subsection{Files and Installation}

The package consists of the files:
%
\begin{center}
\begin{tabular}{ll}
    |README.txt|   & readme file \\
    |childdoc.ins| & installation file \\
    |childdoc.dtx| & source file \\
    |childdoc.def| & definition file \\
    |cdocsamp.tex| & sample main file \\
    |cdocsch1.tex| & sample include file \\
    |cdocsch2.tex| & sample include file \\
    |cdocspt3.tex| & sample part file \\
    |cdocspt4.tex| & sample part file \\
    |cdocsdrf.tex| & sample redirection file \\
    |cdocsfn1.tex| & sample redirection file \\
    |cdocsfn2.tex| & sample redirection file \\
    |childdoc.pdf| & manual
\end{tabular}
\end{center}
%
The distribution consists of the files
|README.txt|, |childdoc.ins| and |childdoc.dtx|.
%
\begin{itemize}
\item
Run (pdf)\LaTeX{} on |childdoc.dtx|
to compile the manual |childdoc.pdf| (this file).
\item
Run \LaTeX{} on |childdoc.ins| to create the definitions file |childdoc.def|
and the sample |cdocsamp.tex| with include files
|cdocsch1.tex|, |cdocsch2.tex|, |cdocspt3.tex|, |cdocspt4.tex|,
|cdocsdrf.tex|, |cdocsfn1.tex|, |cdocsfn2.tex|.
Then copy the file |childdoc.def| to an appropriate directory of your \LaTeX{}
distribution, e.g.\ \textit{texmf-root}|/tex/latex/childdoc|.
\end{itemize}

%%%%%%%%%%%%%%%%%%%%%%%%%%%%%%%%%%%%%%%%%%%%%%%%%%%%%%%%%%%%%%%%%%%%%%%%%%%%%%%%
\subsection{Related CTAN Packages}

There are several other packages which offer a similar functionality:
%
\begin{itemize}
\item
The packages
\href{http://ctan.org/pkg/docmute}{\textsf{docmute}},
\href{http://ctan.org/pkg/includex}{\textsf{includex}} and
\href{http://ctan.org/pkg/standalone}{\textsf{standalone}}
provide commands to include only the document body of
a child file thus allowing both files to be compiled individually.
\item
The packages \href{http://ctan.org/pkg/subdocs}{\textsf{subdocs}}
and \href{http://ctan.org/pkg/subfiles}{\textsf{subfiles}}
provide structures in which the main and child documents can be
encapsulated and allowing them to be compiled individually.
The inclusion mechanism is different from the conventional |\include|.
\item
The package \href{http://ctan.org/pkg/combine}{\textsf{combine}}
is an elaborate solution to combine several documents into one.
\end{itemize}
%
See also the CTAN topic \href{http://ctan.org/topic/subdocs}{\textsf{subdocs}}
for further related packages.
The present package differs from the above solutions in that
a document structure constructed with the conventional |\include| mechanism
just needs two extra commands at the top of every file
such that all constituent files can be compiled individually.

%%%%%%%%%%%%%%%%%%%%%%%%%%%%%%%%%%%%%%%%%%%%%%%%%%%%%%%%%%%%%%%%%%%%%%%%%%%%%%%%
%\subsection{Feature Suggestions}
%
%The following is a list of features which may be useful for future
%versions of this package:
%%
%\begin{itemize}
%\item
%\ldots
%\end{itemize}

%%%%%%%%%%%%%%%%%%%%%%%%%%%%%%%%%%%%%%%%%%%%%%%%%%%%%%%%%%%%%%%%%%%%%%%%%%%%%%%%
\subsection{Revision History}

%%%%%%%%%%%%%%%%%%%%%%%%%%%%%%%%%%%%%%%%
\paragraph{v2.0:} 2018/12/30

\begin{itemize}
\item
immediate forward processing
\item
added |\childdocby| mechanism
\item
manual restructured
\end{itemize}

%%%%%%%%%%%%%%%%%%%%%%%%%%%%%%%%%%%%%%%%
\paragraph{v1.6:} 2018/01/17

\begin{itemize}
\item
application for development of include files
\item
corrections to manual
\end{itemize}

%%%%%%%%%%%%%%%%%%%%%%%%%%%%%%%%%%%%%%%%
\paragraph{v1.5:} 2017/05/21

\begin{itemize}
\item
more complete structuring introduced
\item
|\childdocof| introduced
\item
|\childdoc| renamed to |\childdocmain|
\item
|\childredirect| renamed to |\childdocforward| and |\childdocforwardprefix|
and functionality expanded
\end{itemize}

%%%%%%%%%%%%%%%%%%%%%%%%%%%%%%%%%%%%%%%%
\paragraph{v1.0:} 2017/04/27

\begin{itemize}
\item
manual and install package
\item
first version published on CTAN
\end{itemize}

%%%%%%%%%%%%%%%%%%%%%%%%%%%%%%%%%%%%%%%%
\paragraph{v0.6:} 2017/04/26

\begin{itemize}
\item
redirection mechanism added
\end{itemize}

%%%%%%%%%%%%%%%%%%%%%%%%%%%%%%%%%%%%%%%%
\paragraph{v0.5:} 2017/04/26

\begin{itemize}
\item
functionality in definition file
\end{itemize}


%%%%%%%%%%%%%%%%%%%%%%%%%%%%%%%%%%%%%%%%%%%%%%%%%%%%%%%%%%%%%%%%%%%%%%%%%%%%%%%%
%%%%%%%%%%%%%%%%%%%%%%%%%%%%%%%%%%%%%%%%%%%%%%%%%%%%%%%%%%%%%%%%%%%%%%%%%%%%%%%%
%%%%%%%%%%%%%%%%%%%%%%%%%%%%%%%%%%%%%%%%%%%%%%%%%%%%%%%%%%%%%%%%%%%%%%%%%%%%%%%%
\appendix

\settowidth\MacroIndent{\rmfamily\scriptsize 000\ }

 \DocInput{childdoc.dtx}

\end{document}
%</driver>
% \fi
%
% %%%%%%%%%%%%%%%%%%%%%%%%%%%%%%%%%%%%%%%%%%%%%%%%%%%%%%%%%%%%%%%%%%%%%%%%%%%%%%
% %%%%%%%%%%%%%%%%%%%%%%%%%%%%%%%%%%%%%%%%%%%%%%%%%%%%%%%%%%%%%%%%%%%%%%%%%%%%%%
% \section{Sample}
%\iffalse
%<*samplemain>
%\fi
%
% The following presents a sample document
% with two chapters, two parts, a title page,
% a compile flag as well as three forwarding files to set the flag.
% It consists of eight |.tex| files:
% \begin{center}
% \begin{tabular}{ll}
% |cdocsamp.tex|&main file\\
% |cdocsch1.tex|&include file for chapter 1\\
% |cdocsch2.tex|&include file for chapter 2\\
% |cdocspt3.tex|&include file for part 3\\
% |cdocspt4.tex|&include file for part 4\\
% |cdocsdrf.tex|&forwarding file for main file in draft mode\\
% |cdocsfi1.tex|&forwarding file for final version of chapter 1\\
% |cdocsfi2.tex|&forwarding file for final version of chapter 2\\
% \end{tabular}
% \end{center}
% Each of the eight files can be compiled directly by the \LaTeX{} compiler.
%
% %%%%%%%%%%%%%%%%%%%%%%%%%%%%%%%%%%%%%%
% \paragraph{Main File.}
%
% The main file is called |cdocsamp.tex|.
%
% Load the \textsf{childdoc} definitions and
% declare the filename for the main document:
%    \begin{macrocode}
\input{childdoc.def}
\childdocmain{}
%    \end{macrocode}

% Optional override for |\version| flag:
%    \begin{macrocode}
%%\ifchilddoc\else\providecommand{\version}{draft}\fi
%    \end{macrocode}

% Define the default values for the |\version| flag
% (|final| for the main file and |draft| for childs):
%    \begin{macrocode}
\ifchilddoc
\providecommand{\version}{draft}
\else
\providecommand{\version}{final}
\fi
%    \end{macrocode}

% Load the standard document class:
%    \begin{macrocode}
\documentclass[12pt]{article}
%    \end{macrocode}

% Start the document body:
%    \begin{macrocode}
\begin{document}
%    \end{macrocode}

% Declare a title page.
% Print title, part of document being processed and version flag:
%    \begin{macrocode}
\addtocounter{page}{-1}
\begin{center}
{\LARGE\bfseries{}childdoc example\par}
\vspace{1cm}
\ifchilddoc
\ifchilddocmanual part\else chapter\fi:
`\childdocname' of `\childdocjob'\par
\else
main document: `\childdocjob'\par
\fi
version: \version\par
\end{center}
\newpage
%    \end{macrocode}

% Manually include selected file,
% otherwise process as usual:
%    \begin{macrocode}
\ifchilddocmanual
\section*{part `\childdocname'}
\input{\childdocname}
\else
%    \end{macrocode}

% Include the two chapters:
%    \begin{macrocode}
\include{cdocsch1}
\include{cdocsch2}
%    \end{macrocode}

% Include the two parts unless only chapters should be displayed:
%    \begin{macrocode}
\ifchilddoc\else
\section{part three}
\input{cdocspt3}
\section{part four}
\input{cdocspt4}
\fi
%    \end{macrocode}

% Process as usual until here:
%    \begin{macrocode}
\fi
%    \end{macrocode}

% End of document body:
%    \begin{macrocode}
\end{document}
%    \end{macrocode}
%\iffalse
%</samplemain>
%\fi
%
% %%%%%%%%%%%%%%%%%%%%%%%%%%%%%%%%%%%%%%
% \paragraph{Chapter Include Files.}
%
% The include files are called |cdocsch1.tex| and |cdocsch2.tex|.
%
%\iffalse
%<*samplechap1|samplechap2>
%\fi

% Optional override for |\version| flag:
%    \begin{macrocode}
%%\providecommand{\version}{final}
%    \end{macrocode}

% Include the main document:
%    \begin{macrocode}
\input{childdoc.def}
\childdocof{cdocsamp}
%    \end{macrocode}

%\iffalse
%</samplechap1|samplechap2>
%\fi
%
%\iffalse
%<*samplechap1>
%\fi
% Some text for chapter 1:
%    \begin{macrocode}
\section{one}
some text in chapter one
%    \end{macrocode}

%\iffalse
%</samplechap1>
%\fi
% Some text for chapter 2:
%\iffalse
%<*samplechap2>
%\fi
%    \begin{macrocode}
\section{two}
more text in chapter two
%    \end{macrocode}

%\iffalse
%</samplechap2>
%\fi
%
% %%%%%%%%%%%%%%%%%%%%%%%%%%%%%%%%%%%%%%
% \paragraph{Part Include Files.}
%
% The include files are called |cdocspt3.tex| and |cdocspt4.tex|.
%
%\iffalse
%<*samplepart3|samplepart4>
%\fi

% Optional override for |\version| flag:
%    \begin{macrocode}
%%\providecommand{\version}{final}
%    \end{macrocode}

% Include the main document:
%    \begin{macrocode}
\input{childdoc.def}
\childdocby{cdocsamp}
%    \end{macrocode}

%\iffalse
%</samplepart3|samplepart4>
%\fi
%
%\iffalse
%<*samplepart3>
%\fi
% Some text for part 3:
%    \begin{macrocode}
some text in part three
%    \end{macrocode}

%\iffalse
%</samplepart3>
%\fi
% Some text for part 4:
%\iffalse
%<*samplepart4>
%\fi
%    \begin{macrocode}
more text in part four
%    \end{macrocode}

%\iffalse
%</samplepart4>
%\fi
%
% %%%%%%%%%%%%%%%%%%%%%%%%%%%%%%%%%%%%%%
% \paragraph{Forwarding for a Complete Draft.}
%
% The following forwarding file |cdocsdrf.tex|
% compiles the main document in draft mode:
%\iffalse
%<*sampledraft>
%\fi
%    \begin{macrocode}
\def\version{draft}
\input{childdoc.def}
\childdocforward{cdocsamp}
%    \end{macrocode}

%\iffalse
%</sampledraft>
%\fi
%
% %%%%%%%%%%%%%%%%%%%%%%%%%%%%%%%%%%%%%%
% \paragraph{Forwarding for Final Version of the Chapters.}
%
% The following forwarding files |cdocsfn1.tex| and |cdocsfn2.tex|
% (with identical content)
% compile the final versions of the child documents
% |cdocsch1.tex| and |cdocsch2.tex|, respectively:
%\iffalse
%<*samplefinal>
%\fi
%    \begin{macrocode}
\def\version{final}
\input{childdoc.def}
\childdocforwardprefix[cdocsamp]{cdocsfn}{cdocsch}
%    \end{macrocode}

%\iffalse
%</samplefinal>
%\fi
%
% %%%%%%%%%%%%%%%%%%%%%%%%%%%%%%%%%%%%%%
% \paragraph{Command Line Processing.}
%
% The following three command lines generate the output files
% |cdocscld|, |cdocscl1| and |cdocscl2|
% which should be identical to
% |cdocsdrf|, |cdocsch1| and |cdocsfn2|, respectively:
% \begin{center}
% \begin{tabular}{l}
% |latex -jobname cdocscld \|\\
% |  "\def\version{draft}\input{childdoc.def}\childdocforward{cdocsamp}"|\\
% |latex -jobname cdocscl1 \|\\
% |  "\input{childdoc.def}\childdocforward[cdocsamp]{cdocsch1}"|\\
% |latex -jobname cdocscl2 \|\\
% |  "\def\version{final}\input{childdoc.def}\childdocforward{cdocsch2}"|
% \end{tabular}
% \end{center}
% Note that the trailing backslash on each first line
% merely continues the input to the second line
% (for convenient cut ant paste).
% Furthermore, the command |latex| can be replaced by any
% of its alternative versions such as |pdflatex|.
%
% %%%%%%%%%%%%%%%%%%%%%%%%%%%%%%%%%%%%%%%%%%%%%%%%%%%%%%%%%%%%%%%%%%%%%%%%%%%%%%
% %%%%%%%%%%%%%%%%%%%%%%%%%%%%%%%%%%%%%%%%%%%%%%%%%%%%%%%%%%%%%%%%%%%%%%%%%%%%%%
% \section{Implementation}
%\iffalse
%<*package>
%\fi
%
% This section describes the definitions file |childdoc.def|.

% The definitions cannot be loaded using |\usepackage| or |\RequirePackage|
% which has a mechanism to prevent loading a style file more than once.
% When loading the definitions by means of |\input|
% multiple instances have to be prevented manually:
%\iffalse
%This code needs to be before the `\ProvidesFile' directive
%which is defined at the beginning of this file.
%Therefore it is also placed there and commented out here.
%</package>
%<*discard>
%\fi
%    \begin{macrocode}
\ifdefined\childdocmain\endinput\fi
%    \end{macrocode}
%\iffalse
%</discard>
%<*package>
%\fi
%
% \macro{\ifchilddoc}
% \macro{\ifchilddocmanual}
% The conditional |\ifchilddoc| tells whether a
% child (true) or main (false) document is being compiled.
% The conditional |\ifchilddocmanual| tells whether
% the |\includeonly| mechanism is used (false) or
% the selection of child files must be performed manually (true).
% The definitions initialise to false:
%    \begin{macrocode}
\newif\ifchilddoc
\newif\ifchilddocmanual
%    \end{macrocode}

% \macro{\childdocname}
% \macro{\childdocjob}
% The macro |\childdocname| stores the name of the main document
% to be compiled. The macro |\childdocjob| stores the name of
% the document on which the \LaTeX{} compiler was originally invoked.
% The content of |\jobname| cannot be compared
% to filenames specified in the source due to different catcodes.
% The following code rescans |\jobname|, stores the result
% in |\childdocname| and saves a copy in |\childdocjob|:
%    \begin{macrocode}
\edef\childdocname{\scantokens\expandafter{\jobname\noexpand}}
\let\childdocjob\childdocname
%    \end{macrocode}

% \macro{\childdocdisable}
% The macro |\childdocdisable| prevents the main file
% from being processed more than once.
% At this stage, the main document command |\childdocmain|
% is assumed to be called once again where it should do nothing.
% Any subsequent call to it should prevent
% a secondary processing of the main document
% It overwrites the forwarding commands
% |\childdocof| and |\childdocforward|
% with empty macros to prevent further inclusions of the main document:
%    \begin{macrocode}
\newcommand{\childdocdisable}
{
  \renewcommand{\childdocmain}[1]{\renewcommand{\childdocmain}[1]{\endinput}}
  \renewcommand{\childdocof}[1]{}
  \renewcommand{\childdocby}[2][]{}
  \renewcommand{\childdocforward}[2][]{}
  \renewcommand{\childdocdisable}{}
}
%    \end{macrocode}

% \macro{\childdocmain}
% The macro |\childdocmain| is to be called at the top of the main file
% with nothing or the main filename (without extension) as argument.
% First, it breaks loops.
% If the argument is not empty and does not match |\childdocname|
% (which is set by the first inclusion of |childdoc.def|),
% |\ifchilddoc| is set to true, |\includeonly| is applied to the child file
% and |\jobname| is set to the main file
% (for proper handling of |.aux| files):
%    \begin{macrocode}
\newcommand{\childdocmain}[1]
{
  \childdocdisable\childdocmain{}
  \if?#1?\else
    \begingroup
      \def\childdoctmp{#1}
      \ifx\childdoctmp\childdocname
        \def\childdoctmp{}
      \else
        \def\childdoctmp
        {
          \childdoctrue
          \includeonly{\childdocname}
          \def\childdocjob{#1}
          \def\jobname{#1}
        }
      \fi
      \expandafter
    \endgroup
    \childdoctmp
  \fi
}
%    \end{macrocode}

% \macro{\childdocof}
% The command |\childdocof| redirects
% compilation to the main file |#1|.
%    \begin{macrocode}
\newcommand{\childdocof}[1]
{
  \childdocdisable
  \childdoctrue
  \includeonly{\childdocname}
  \def\jobname{#1}
  \def\childdocjob{#1}
  \input{#1}
}
%    \end{macrocode}

% \macro{\childdocby}
% The command |\childdocby| ....
%    \begin{macrocode}
\newcommand{\childdocby}[2][]
{
  \childdocdisable
  \childdoctrue
  \childdocmanualtrue
  \if?#1?\else
    \def\jobname{#2}
  \fi
  \def\childdocjob{#2}
  \input{#2}
  \endinput
}
%    \end{macrocode}

% \macro{\childdocforward}
% The command |\childdocforward| redirects
% compilation to the main file or
% (if the optional argument is given) a child file.
% Parameters are set as if the main file
% or a child file starting with |\childdocof| was compiled.
% Then compilation is handed over to the main file:
%    \begin{macrocode}
\newcommand{\childdocforward}[2][]
{
  \begingroup
    \if?#1?
      \def\childdoctmp
      {
        \def\childdocname{#2}
        \def\childdocjob{#2}
        \def\jobname{#2}
        \input{#2}
        \endinput
      }
    \else
      \def\childdoctmp
      {
        \childdocdisable
        \def\childdocname{#2}
        \childdoctrue
        \includeonly{#2}
        \def\childdocjob{#1}
        \def\jobname{#1}
        \input{#1}
        \endinput
      }
    \fi
    \expandafter
  \endgroup
  \childdoctmp
}
%    \end{macrocode}

% \macro{\childdocforwardprefix}
% The command |\childdocforwardprefix| redirects
% compilation to the main or a child file by means of a pattern.
% The prefix |#1| in the current filename is replaced by |#2|
% and the suffix of the current filename is kept
% (it is assumed that the filename does not contain the substring `|~~~|'
% which is used as a delimiter).
% Compilation is handed over to the new file by |\childdocforward|:
%    \begin{macrocode}
\newcommand{\childdocforwardprefix}[3][]
{
  \begingroup
    \def\childdocextract #2##1~~~{\def\childdoctmp{\childdocforward[#1]{#3##1}}}
    \expandafter\childdocextract\childdocname~~~
    \expandafter
  \endgroup
  \childdoctmp
}
%    \end{macrocode}

% \macro{\childdoc}
% The deprecated macro |\childdoc| is a legacy version of |\childdocmain|:
%    \begin{macrocode}
\newcommand{\childdoc}{\childdocmain}
%    \end{macrocode}

% \macro{\childdocredirect}
% The deprecated macro |\childdocredirect| is a legacy version
% of |\childdocforward| and |\childdocforwardprefix|:
%    \begin{macrocode}
\newcommand{\childdocredirect}[2][]
{
  \begingroup
    \if?#1?
      \def\childdoctmp{\childdocforward{#2}}
    \else
      \def\childdoctmp{\childdocforwardprefix{#1}{#2}}
    \fi
    \expandafter
  \endgroup
  \childdoctmp
}
%    \end{macrocode}

%\iffalse
%</package>
%\fi
%
\endinput

\childdocmain{}
%    \end{macrocode}

% Optional override for |\version| flag:
%    \begin{macrocode}
%%\ifchilddoc\else\providecommand{\version}{draft}\fi
%    \end{macrocode}

% Define the default values for the |\version| flag
% (|final| for the main file and |draft| for childs):
%    \begin{macrocode}
\ifchilddoc
\providecommand{\version}{draft}
\else
\providecommand{\version}{final}
\fi
%    \end{macrocode}

% Load the standard document class:
%    \begin{macrocode}
\documentclass[12pt]{article}
%    \end{macrocode}

% Start the document body:
%    \begin{macrocode}
\begin{document}
%    \end{macrocode}

% Declare a title page.
% Print title, part of document being processed and version flag:
%    \begin{macrocode}
\addtocounter{page}{-1}
\begin{center}
{\LARGE\bfseries{}childdoc example\par}
\vspace{1cm}
\ifchilddoc
\ifchilddocmanual part\else chapter\fi:
`\childdocname' of `\childdocjob'\par
\else
main document: `\childdocjob'\par
\fi
version: \version\par
\end{center}
\newpage
%    \end{macrocode}

% Manually include selected file,
% otherwise process as usual:
%    \begin{macrocode}
\ifchilddocmanual
\section*{part `\childdocname'}
\input{\childdocname}
\else
%    \end{macrocode}

% Include the two chapters:
%    \begin{macrocode}
\include{cdocsch1}
\include{cdocsch2}
%    \end{macrocode}

% Include the two parts unless only chapters should be displayed:
%    \begin{macrocode}
\ifchilddoc\else
\section{part three}
\input{cdocspt3}
\section{part four}
\input{cdocspt4}
\fi
%    \end{macrocode}

% Process as usual until here:
%    \begin{macrocode}
\fi
%    \end{macrocode}

% End of document body:
%    \begin{macrocode}
\end{document}
%    \end{macrocode}
%\iffalse
%</samplemain>
%\fi
%
% %%%%%%%%%%%%%%%%%%%%%%%%%%%%%%%%%%%%%%
% \paragraph{Chapter Include Files.}
%
% The include files are called |cdocsch1.tex| and |cdocsch2.tex|.
%
%\iffalse
%<*samplechap1|samplechap2>
%\fi

% Optional override for |\version| flag:
%    \begin{macrocode}
%%\providecommand{\version}{final}
%    \end{macrocode}

% Include the main document:
%    \begin{macrocode}
% \iffalse
%
% childdoc.dtx Copyright (C) 2017-2018 Niklas Beisert
%
% This work may be distributed and/or modified under the
% conditions of the LaTeX Project Public License, either version 1.3
% of this license or (at your option) any later version.
% The latest version of this license is in
%   http://www.latex-project.org/lppl.txt
% and version 1.3 or later is part of all distributions of LaTeX
% version 2005/12/01 or later.
%
% This work has the LPPL maintenance status `maintained'.
%
% The Current Maintainer of this work is Niklas Beisert.
%
% This work consists of the files childdoc.dtx and childdoc.ins
% and the derived files childdoc.def and cdocsamp.tex with
% cdocsch1.tex, cdocsch2.tex, cdocsdrf.tex, cdocsfn1.tex, cdocsfn2.tex.
%
%<package>\ifdefined\childdocmain\endinput\fi
%<package>\ProvidesFile{childdoc.def}[2018/12/30 v2.0 child document driver]
%<samplemain>\ProvidesFile{cdocsamp.tex}[2018/12/30 v2.0 sample for childdoc]
%<*driver>
%\ProvidesFile{childdoc.drv}[2018/12/30 v2.0 childdoc reference manual file]
\PassOptionsToClass{10pt,a4paper}{article}
\documentclass{ltxdoc}

\usepackage[margin=35mm]{geometry}
\usepackage{hyperref}
\usepackage{hyperxmp}
\usepackage[usenames]{color}

\hypersetup{colorlinks=true}
\hypersetup{pdfstartview=FitH}
\hypersetup{pdfpagemode=UseNone}
\hypersetup{pdfsource={}}
\hypersetup{pdflang={en-UK}}
\hypersetup{pdfcopyright={Copyright 2017-2018 Niklas Beisert.
  This work may be distributed and/or modified under the
  conditions of the LaTeX Project Public License, either version 1.3
  of this license or (at your option) any later version.}}
\hypersetup{pdflicenseurl={http://www.latex-project.org/lppl.txt}}
\hypersetup{pdfcontactaddress={ETH Zurich, ITP, HIT K,
  Wolfgang-Pauli-Strasse 27}}
\hypersetup{pdfcontactpostcode={8093}}
\hypersetup{pdfcontactcity={Zurich}}
\hypersetup{pdfcontactcountry={Switzerland}}
\hypersetup{pdfcontactemail={nbeisert@itp.phys.ethz.ch}}
\hypersetup{pdfcontacturl={http://people.phys.ethz.ch/\xmptilde nbeisert/}}

\newcommand{\secref}[1]{\hyperref[#1]{section \ref*{#1}}}

\parskip1ex
\parindent0pt
\let\olditemize\itemize
\def\itemize{\olditemize\parskip0pt}

\begin{document}

\title{The \textsf{childdoc} Package}
\hypersetup{pdftitle={The childdoc Package}}
\author{Niklas Beisert\\[2ex]
  Institut f\"ur Theoretische Physik\\
  Eidgen\"ossische Technische Hochschule Z\"urich\\
  Wolfgang-Pauli-Strasse 27, 8093 Z\"urich, Switzerland\\[1ex]
  \href{mailto:nbeisert@itp.phys.ethz.ch}
  {\texttt{nbeisert@itp.phys.ethz.ch}}}
\hypersetup{pdfauthor={Niklas Beisert}}
\hypersetup{pdfsubject={Manual for the LaTeX2e Package childdoc}}
\date{30 December 2018, \textsf{v2.0}}
\maketitle

\begin{abstract}\noindent
\textsf{childdoc} is a \LaTeXe{} package
that enables the direct compilation
of document sections included by |\include|
to individual files.
\end{abstract}

\begingroup
\parskip0ex
\tableofcontents
\endgroup

%%%%%%%%%%%%%%%%%%%%%%%%%%%%%%%%%%%%%%%%%%%%%%%%%%%%%%%%%%%%%%%%%%%%%%%%%%%%%%%%
%%%%%%%%%%%%%%%%%%%%%%%%%%%%%%%%%%%%%%%%%%%%%%%%%%%%%%%%%%%%%%%%%%%%%%%%%%%%%%%%
\section{Introduction}

\LaTeX{} provides a mechanism to structure a large document (such as a book)
into a main file and several child files (containing the chapters)
using the |\include| command.
This mechanism is beneficial for documents
which span hundreds of pages in order to
make the source file(s) more manageable.
Moreover, compilation can be restricted to
selected child files by means of the |\includeonly| command.
The latter feature can be used to reduce the compilation time while editing
(this was significantly more useful in the earlier days of \LaTeX{})
or to generate a smaller document which is easier to navigate.
Another application of |\includeonly| is to generate
documents consisting of selected parts of the complete document.

However, there are a few drawbacks of the plain |\include| mechanism:
\begin{itemize}
\item
The child files cannot be compiled on their own,
they can only be compiled via the main file.
A naive editing environment
(such as a text editor with an option
to have the current file processed by \LaTeX)
may require one to switch to the main file before compiling;
attempting to compile the child file produces errors.
\item
The main file must be modified (each time)
to adjust the |\includeonly| command
to the present needs. This easily leaves the main file in a messy state.
\item
The generated document will always carry the filename
of the main document. This is inconvenient if
several child files are to be compiled and
to be kept for distribution.
\end{itemize}

The present package provides a simple interface
to make child files individually compilable by \LaTeX{}.
Compiling a child file then has the same effect as compiling
the main file with an |\includeonly| command
to select the appropriate child.
Moreover the generated document will carry the name of the child
rather than the main file.
This resolves all three above issues.

This feature is meant to make the editing of books,
thesis documents and lecture notes somewhat more convenient.
However, the package can also be used efficiently for
composing a series of documents (such as exercise sheets)
which are typically distributed individually.
It then assists the author in generating the individual documents
(potentially in different versions)
as well as a document containing the collected series.
Another application is in developing style files
or other kinds of included material
where compilation of the style file could redirect
to a sample or test file.

%%%%%%%%%%%%%%%%%%%%%%%%%%%%%%%%%%%%%%%%%%%%%%%%%%%%%%%%%%%%%%%%%%%%%%%%%%%%%%%%
%%%%%%%%%%%%%%%%%%%%%%%%%%%%%%%%%%%%%%%%%%%%%%%%%%%%%%%%%%%%%%%%%%%%%%%%%%%%%%%%
\section{Usage}

First of all, the package \textsf{childdoc} is \emph{not} a standard
\LaTeXe{} |.sty| style file! Therefore it needs to be invoked in
a non-standard way.

%%%%%%%%%%%%%%%%%%%%%%%%%%%%%%%%%%%%%%%%%%%%%%%%%%%%%%%%%%%%%%%%%%%%%%%%%%%%%%%%
\subsection{Included Files}
\label{sec:include}

%%%%%%%%%%%%%%%%%%%%%%%%%%%%%%%%%%%%%%%%
\DescribeMacro{\childdocmain}
To use the package, add the commands
\begin{center}
\begin{tabular}{l}
|\input{childdoc.def}|\\
|\childdocmain{}|\\
\end{tabular}
\end{center}
at the very top of the main \LaTeX{} file,
in particular \emph{before} the |\documentclass| statement!
The argument of |\childdocmain| should be left empty
(but it must be present).

%%%%%%%%%%%%%%%%%%%%%%%%%%%%%%%%%%%%%%%%
\DescribeMacro{\childdocof}
Furthermore, add the commands
\begin{center}
\begin{tabular}{l}
|\input{childdoc.def}|\\
|\childdocof{|\textit{main}|}|\\
\end{tabular}
\end{center}
at the top of every child file \textit{child}
which is included by |\include{|\textit{child}|}|
from within the main file
(or at least for those files to be compiled individually).
The argument \textit{main} must be the filename of the main file.

There are a couple of
considerations in setting up the main and child documents:

%%%%%%%%%%%%%%%%%%%%%%%%%%%%%%%%%%%%%%%%
\paragraph{Restrictions.}

Please note the following restrictions:
\begin{itemize}
\item
|\childdocmain| must be called with one argument \textit{main}
to ensure compatibility with earlier version of the package.
It must either be empty (|\childdocmain{}|)
or precisely match the filename of the main file in which it is specified.
See \secref{sec:detection} for further information.
\item
The filename \textit{main} must be specified without the |.tex| extension.
\item
The filename \textit{main} is case sensitive
(even in case-insensitive file systems)
due to internal string comparison.
\item
The argument \textit{main} should be fully expanded, it cannot be a macro.
\item
Subdirectories and special characters should be avoided in filenames.
\item
The command |\childdocmain{|\textit{main}|}| must be followed by a whitespace.
It should not be followed immediately by another command
or by a comment mark `|%|'.
This is because the \TeX{} parser reads the token immediately following
the argument of |\childdocmain| and puts it
at the beginning of every child section;
however, a white\-space is ignored.
\end{itemize}

%%%%%%%%%%%%%%%%%%%%%%%%%%%%%%%%%%%%%%%%
\paragraph{Content of Main File.}

It is advisable to place all content in the child files included by |\include|.
Any output contained in the main file will appear in all child documents
unless suppressed manually;
it cannot be suppressed automatically by the |\includeonly| directive
and thus should normally be avoided.
A method to include some content in the main file
by means of conditional processing is described in \secref{sec:conditional}.

%%%%%%%%%%%%%%%%%%%%%%%%%%%%%%%%%%%%%%%%
\paragraph{Page Numbering.}

When only a part of the document is compiled,
the appropriate numbering of pages
(as well as other status parameters)
is determined from the |.aux| files.
The latter contain information from previous passes.
However this information needs to propagate through
all intermediate child documents.
Therefore the page numbering in child documents may well
be inconsistent until the complete document is compiled at least once.

A useful (if unconventional) way to always ensure a consistent
page numbering is to restart the numbering in each child document
and denote the pages by `\textit{child}|.|\textit{page}'
where \textit{child} represents the chapter/section number of the child file.
This can be achieved by the command
|\numberwithin{page}{|\textit{child}|}|
of the \textsf{amsmath} package
where \textit{child} can be |chapter| or |section|
depending on the chosen structuring.
Alternatively, one can modify the macro |\thepage| appropriately
and reset the counter |page| at the start of each child file.

%%%%%%%%%%%%%%%%%%%%%%%%%%%%%%%%%%%%%%%%%%%%%%%%%%%%%%%%%%%%%%%%%%%%%%%%%%%%%%%%
\subsection{Conditional Processing}
\label{sec:conditional}

The package provides a mechanism to compile different versions
of a document. To customise the versions further some conditional processing
can come in handy to distinguish which version is being compiled.
The package provides two macros to describe the compilation context:

%%%%%%%%%%%%%%%%%%%%%%%%%%%%%%%%%%%%%%%%
\DescribeMacro{\ifchilddoc}
The conditional |\ifchilddoc| distinguishes between the compilation of
child documents and the main document:
%
\begin{center}
|\ifchilddoc |\textit{child-code}| |[|\||else |\textit{main-code}]| \||fi|
\end{center}

%%%%%%%%%%%%%%%%%%%%%%%%%%%%%%%%%%%%%%%%
\DescribeMacro{\childdocname}
\DescribeMacro{\childdocjob}
The macro |\childdocname| contains the filename (without extension)
of the main or child file being processed.
Note that |\childdocjob| will always contain the name of the main file.

%%%%%%%%%%%%%%%%%%%%%%%%%%%%%%%%%%%%%%%%
\paragraph{Title Page.}

Conditional processing can be used to include a title or banner page
in the main document when proper precautions are taken.
Importantly, the code in the main file should ensure that the page counter
(as well as other status parameters which are stored in the |.aux| files)
takes the same value after the conditional processing.
Otherwise the page numbers may take divergent values
depending on which part is compiled.

For example, a title page could be declared by:
%
\begin{center}
\begin{tabular}{l}
|\ifchilddoc\||else|\\
|\addtocounter{page}{-1}|\\
\textit{code for title page}\\
|\newpage|\\
|\||fi|
\end{tabular}
\end{center}
%
A banner page for the child documents can be generated by:
%
\begin{center}
\begin{tabular}{l}
|\ifchilddoc|\\
|\addtocounter{page}{-1}|\\
\textit{code for banner page}\\
|\newpage|\\
|\||fi|
\end{tabular}
\end{center}
%
Here one could write a message such as:
\begin{center}
|This is the part \childdocname{} of \childdocjob{}.|
\end{center}

%%%%%%%%%%%%%%%%%%%%%%%%%%%%%%%%%%%%%%%%%%%%%%%%%%%%%%%%%%%%%%%%%%%%%%%%%%%%%%%%
\subsection{Flags}
\label{sec:flags}

The package makes it easy to generate different versions
of the main or child documents.
To this end compilation flags can be defined
and assigned different default values.
They will be particularly useful in conjunction
with the forwarding mechanism described in \secref{sec:forward}.

For example, it may be useful to have a flag |\version|
which can be set to |draft| or |final|.
The document source will contain some conditional code
depending on the value of |\version|.
Suppose further, the flag should default to |final| for the main file
and to |draft| for child files
which is a natural assignment for editing the document.
This is achieved by placing the following code
in the preamble of the main document
(below the |\childdocmain| directive):
%
\begin{center}
\begin{tabular}{l}
|\ifchilddoc|\\
|\providecommand{\version}{draft}|\\
|\||else|\\
|\providecommand{\version}{final}|\\
|\||fi|
\end{tabular}
\end{center}
%
The definition by |\providecommand| makes sure
that previous definitions are not overwritten.
Further statements |\providecommand{\version}{...}|
can thus be added before the above code to override it.

For the main file, one might add a line
(between |\childdocmain| and the above block)
%
\begin{center}
|%\ifchilddoc\||else\providecommand{\version}{draft}\||fi|
\end{center}
%
which can be uncommented to produce a draft version.
Likewise one can add a line to the very top of a child file
(above the |\childdocof{|\textit{main}|}| directive)
%
\begin{center}
|%\providecommand{\version}{final}|
\end{center}
%
which can be uncommented to produce the final version of this child document.

%%%%%%%%%%%%%%%%%%%%%%%%%%%%%%%%%%%%%%%%%%%%%%%%%%%%%%%%%%%%%%%%%%%%%%%%%%%%%%%%
\subsection{Forwarding}
\label{sec:forward}

Different versions of the main or child documents
using compilation flags as described in \secref{sec:flags}
can be (permanently) stored in different files
for convenient compilation, viewing and distribution.
To this end, the package defines a command
to pass on compilation to a different file:

%%%%%%%%%%%%%%%%%%%%%%%%%%%%%%%%%%%%%%%%
\DescribeMacro{\childdocforward}
The command |\childdocforward| redirects processing to
another source file:
%
\begin{center}
\begin{tabular}{l}
|\input{childdoc.def}|\\
|\childdocforward[|\textit{main}|]{|\textit{dest}|}|\\
\end{tabular}
\end{center}
%
The argument \textit{dest} is the destination file
(without extension).
It should be the main file or one of the child files.
Note that further \textsf{childdoc} directives
such as |\childdocof| and |\childdocforward|
in the indicated file will be processed in this form.
The optional argument \textit{main}
passes on directly to the main file \textit{main}
while pretending to compile the child \textit{dest}.
This form behaves as if \textit{dest}
issues |\childdocof{|\textit{main}|}| right away,
and no further \textsf{childdoc} directives will be processed.

%%%%%%%%%%%%%%%%%%%%%%%%%%%%%%%%%%%%%%%%
\DescribeMacro{\...prefix}
In the alternative form |\childdocforwardprefix|,
%
\begin{center}
\begin{tabular}{l}
|\input{childdoc.def}|\\
|\childdocforwardprefix[|\textit{main}|]{|\textit{prefix}|}{|\textit{dest}|}|
\end{tabular}
\end{center}
%
the destination file is determined by a pattern
depending on the current file:
To make this work, the current file must be called
`{\textit{prefix}\hspace{0.2em}\textit{suffix}}'
with \textit{prefix} matching precisely the argument.
Processing is then passed on to the file
`{\textit{dest}\hspace{0.2em}\textit{suffix}}'.
Surely, the same effect is achieved by
directly specifying the
argument `{\textit{dest}\hspace{0.2em}\textit{suffix}}'
in the first form.
However, that requires to set up a different file
for each child. With the alternative form of the command
all these files can have exactly the same content
which simplifies setting them up and maintaining them.

For example, the following file |draft.tex|
with a compilation flag |\version| as described in \secref{sec:flags}
compiles the main document as a draft:
%
\begin{center}
\begin{tabular}{l}
|\def\version{draft}|\\
|\input{childdoc.def}|\\
|\childdocforward{|\textit{main}|}|
\end{tabular}
\end{center}
%
Likewise, the following files |final|\textit{nn}|.tex|
compile the final version of the child document
|child|\textit{nn}|.tex|:
%
\begin{center}
\begin{tabular}{l}
|\def\version{final}|\\
|\input{childdoc.def}|\\
|\childdocforwardprefix{final}{child}|
\end{tabular}
\end{center}
%

Note that when several versions of a main file and/or of each child file
are to be generated, it may be convenient to set up a |Makefile| or
shell script to automatise the process.

%%%%%%%%%%%%%%%%%%%%%%%%%%%%%%%%%%%%%%%%%%%%%%%%%%%%%%%%%%%%%%%%%%%%%%%%%%%%%%%%
\subsection{Command Line Processing}
\label{sec:commandline}

The effect of redirection files can also be achieved by invoking
the \LaTeX{} compiler with a more elaborate command line.
Most conveniently this should be done as part
of a shell script or a |Makefile|.

When using \textsf{childdoc} in the main file, the following
command lines effectively perform a redirection
(note that depending on the shell being used,
backslashes may have to be doubled: `|\|' $\to$ `|\\|'):
%
\begin{center}
|... -jobname "|\textit{target}|" |\\|"|[\textit{flags}]%
|\input{childdoc.def}\childdocforward[|\textit{main}|]{|\textit{dest}|}"|
\end{center}
%
Here \textit{target} is the name of the output file,
\textit{main} is the name of the main file
and \textit{dest} is the name of the main or child file to be processed
(all filenames without extensions).
The optional argument \textit{main} can be omitted
if \textit{main} matches \textit{dest}.
Optionally, compilation \textit{flags} can be defined via |\def| commands.
This command line makes the \TeX{} engine believe
it is compiling the file \textit{target}
whose content is specified as the latter parameter.
The provided code then forwards the processing to
\textit{main} or \textit{dest} as described in \secref{sec:forward}.

%%%%%%%%%%%%%%%%%%%%%%%%%%%%%%%%%%%%%%%%%%%%%%%%%%%%%%%%%%%%%%%%%%%%%%%%%%%%%%%%
\subsection{Include by Input}
\label{sec:input}

Including child documents by |\include| has some restrictions by design.
Most notably, the content of a child document always occupies
its own set of pages; pages cannot be shared between child documents.
Usually, this behaviour makes perfect sense
because each child document contain an essential part of the document.
However, in some situations it may be desirable to compose
a document from a collection of parts
without having mandatory page breaks between then.
For this case, the package
provides a mechanism to include parts
by |\input| which can also be processed individually.
However, by construction this mechanism
requires manual handling of the content to be output.

%%%%%%%%%%%%%%%%%%%%%%%%%%%%%%%%%%%%%%%%
\DescribeMacro{\ifchilddocmanual}
The main file should be prepared as usual, see \secref{sec:include}.
However, the document body must make a distinction
between processing of an individual part and of the main document, e.g.:
%
\begin{center}
\begin{tabular}{l}
|\ifchilddocmanual|\\
|\input{\childdocname}|\\
|\||else|\\
\textit{document body with }|\input{|\textit{part}|}|\\
|\||fi|
\end{tabular}
\end{center}
%
The conditional |\ifchilddocmanual| is true whenever
a part to be included by |\input| is being compiled,
and the name of the part is stored in |\childdocname|.

%%%%%%%%%%%%%%%%%%%%%%%%%%%%%%%%%%%%%%%%
\DescribeMacro{\childdocby}
Each part to be included by |\input| should start with:
%
\begin{center}
\begin{tabular}{l}
|\input{childdoc.def}|\\
|\childdocby{|\textit{main}|}|\\
\end{tabular}
\end{center}
%
The directive |\childdocby| is similar to |\childdocof|
described in \secref{sec:include},
but the subsequent selection of content must be done manually.
To that end, both |\ifchilddoc| and |\ifchilddocmanual|
will be true upon processing of a part,
and the name of the part is stored in |\childdocname|.
Note that |\jobname| will be set to the filename of the current part
so that each part receives an individual |.aux| file
that does not interfere with the |.aux| file(s) of the main document.
This behaviour can be altered by the alternative form
|\childdocby[*]{|\textit{main}|}| (with a non-empty optional argument)
which uses the |.aux| file of the main document
by setting |\jobname| to \textit{main}.

%%%%%%%%%%%%%%%%%%%%%%%%%%%%%%%%%%%%%%%%%%%%%%%%%%%%%%%%%%%%%%%%%%%%%%%%%%%%%%%%
\subsection{Driver Development}
\label{sec:driver}

The \textsf{childdoc} mechanism can also be use for the development
of definition files such as \LaTeX{} styles or classes.
This case differs from the above setup with multiple parts
included by |\include| in that no |\includeonly| should be invoked.
This can be achieved by starting the include file
(before |\ProvidesPackage|) with:
%
\begin{center}
\begin{tabular}{l}
|\input{childdoc.def}|\\
|\childdocforward{|\textit{main}|}|\\
\end{tabular}
\end{center}
%
or alternatively with:
%
\begin{center}
\begin{tabular}{l}
|\input{childdoc.def}|\\
|\childdocby{|\textit{main}|}|\\
\end{tabular}
\end{center}
%
Both forms have slightly different effects as described above.
The main file is prepared as usual, see \secref{sec:include}.

%%%%%%%%%%%%%%%%%%%%%%%%%%%%%%%%%%%%%%%%%%%%%%%%%%%%%%%%%%%%%%%%%%%%%%%%%%%%%%%%
\subsection{Legacy Detection}
\label{sec:detection}

The directive |\childdocmain| in the main file can detect
whether the complete document or merely a child is to be compiled
even without using the directive |\childdocof|.
This method is deprecated because it is less robust
and there is no compelling reason to use it;
it is merely provided for backward compatibility
and it may be removed in future versions.

If the detection mechanism is to be used,
it is mandatory to correctly specify
the filename of the main file as the argument of |\childdocmain|:
%
\begin{center}
\begin{tabular}{l}
|\input{childdoc.def}|\\
|\childdocmain{|\textit{main}|}|\\
\end{tabular}
\end{center}
%
If |\jobname| does not match the argument \textit{main} of |\childdocmain|,
it is assumed that |\jobname| points to the child file to be compiled.
When using |\childdocmain| with the main file specified as argument,
it suffices to start a child file
with just |\input{|\textit{main}|}|
without loading of the package and using |\childdocof|.
If instead all processing is done
with the appropriate \textsf{childdoc} directives,
the argument of \textit{main} of |\childdocmain| can be empty.

An alternative version of the command line processing described
in \secref{sec:commandline} using the detection mechanism reads:
%
\begin{center}
|... -jobname "|\textit{target}|" "|[\textit{flags}]%
[|\def\jobname{|\textit{dest}|}|]|\input{|\textit{main}|}"|
\end{center}

%%%%%%%%%%%%%%%%%%%%%%%%%%%%%%%%%%%%%%%%%%%%%%%%%%%%%%%%%%%%%%%%%%%%%%%%%%%%%%%%
\subsection{Manual Code}
\label{sec:manual}

In case one cannot be certain whether the definitions file |childdoc.def|
is installed on the target \TeX{} distribution
and one prefers not to ship it,
it is conceivable to paste a few relevant commands into the sources.

To that end, drop all statements |\input{childdoc.def}|
and perform the replacements as outlined below.
Instead of |\childdocmain{|\textit{main}|}| add the following code
to the top of the main file:
%
\begin{center}
\begin{tabular}{l}
|\||ifdefined\childdocname\endinput\||fi\newif\ifchilddoc|\\
|\edef\childdocname{\scantokens\expandafter{\jobname\noexpand}}|\\
|\def\childdocmain{|\textit{main}|}\||ifx\childdocmain\childdocname\||else|\\
|\childdoctrue\includeonly{\childdocname}\let\jobname\childdocmain\||fi|\\
\end{tabular}
\end{center}
%
Instead of |\childdocof{|\textit{main}|}| just include the main file
at the top of each child file:
%
\begin{center}
|\input{|\textit{main}|}|
\end{center}
%
A simple redirection |\childdocforward{|\textit{dest}|}| is achieved by:
%
\begin{center}
|\def\jobname{|\textit{dest}|}\input{\jobname}|
\end{center}
%
The redirection with prefix
|\childdocforwardprefix[|\textit{prefix}|]{|\textit{dest}|}|
is accomplished by:
%
\begin{center}
\begin{tabular}{l}
|{\edef\jobname{\scantokens\expandafter{\jobname\noexpand}}|\\
|\def\redirectjob |\textit{prefix}|#1~~~{\gdef\jobname{|\textit{dest}|#1}}|\\
|\expandafter\redirectjob\jobname~~~}\input{\jobname}|
\end{tabular}
\end{center}

In an alternative approach,
child documents can be compiled by a specific command line
without additional code or specific definitions:
%
\begin{center}
|... -jobname "|\textit{target}|" "|[\textit{flags}]%
|\includeonly{|\textit{dest}|}\input{|\textit{main}|}"|
\end{center}
%

%%%%%%%%%%%%%%%%%%%%%%%%%%%%%%%%%%%%%%%%%%%%%%%%%%%%%%%%%%%%%%%%%%%%%%%%%%%%%%%%
%%%%%%%%%%%%%%%%%%%%%%%%%%%%%%%%%%%%%%%%%%%%%%%%%%%%%%%%%%%%%%%%%%%%%%%%%%%%%%%%
\section{Information}

%%%%%%%%%%%%%%%%%%%%%%%%%%%%%%%%%%%%%%%%%%%%%%%%%%%%%%%%%%%%%%%%%%%%%%%%%%%%%%%%
\subsection{Copyright}

Copyright \copyright{} 2017--2018 Niklas Beisert

This work may be distributed and/or modified under the
conditions of the \LaTeX{} Project Public License, either version 1.3
of this license or (at your option) any later version.
The latest version of this license is in
  \url{http://www.latex-project.org/lppl.txt}
and version 1.3 or later is part of all distributions of \LaTeX{}
version 2005/12/01 or later.

This work has the LPPL maintenance status `maintained'.

The Current Maintainer of this work is Niklas Beisert.

This work consists of the files |README.txt|, |childdoc.ins| and |childdoc.dtx|
as well as the derived files |childdoc.def|, |cdocsamp.tex|
with |cdocsch1.tex|, |cdocsch2.tex|, |cdocspt3.tex|, |cdocspt4.tex|,
|cdocsdrf.tex|, |cdocsfn1.tex|, |cdocsfn2.tex|
as well as |childdoc.pdf|.

%%%%%%%%%%%%%%%%%%%%%%%%%%%%%%%%%%%%%%%%%%%%%%%%%%%%%%%%%%%%%%%%%%%%%%%%%%%%%%%%
\subsection{Files and Installation}

The package consists of the files:
%
\begin{center}
\begin{tabular}{ll}
    |README.txt|   & readme file \\
    |childdoc.ins| & installation file \\
    |childdoc.dtx| & source file \\
    |childdoc.def| & definition file \\
    |cdocsamp.tex| & sample main file \\
    |cdocsch1.tex| & sample include file \\
    |cdocsch2.tex| & sample include file \\
    |cdocspt3.tex| & sample part file \\
    |cdocspt4.tex| & sample part file \\
    |cdocsdrf.tex| & sample redirection file \\
    |cdocsfn1.tex| & sample redirection file \\
    |cdocsfn2.tex| & sample redirection file \\
    |childdoc.pdf| & manual
\end{tabular}
\end{center}
%
The distribution consists of the files
|README.txt|, |childdoc.ins| and |childdoc.dtx|.
%
\begin{itemize}
\item
Run (pdf)\LaTeX{} on |childdoc.dtx|
to compile the manual |childdoc.pdf| (this file).
\item
Run \LaTeX{} on |childdoc.ins| to create the definitions file |childdoc.def|
and the sample |cdocsamp.tex| with include files
|cdocsch1.tex|, |cdocsch2.tex|, |cdocspt3.tex|, |cdocspt4.tex|,
|cdocsdrf.tex|, |cdocsfn1.tex|, |cdocsfn2.tex|.
Then copy the file |childdoc.def| to an appropriate directory of your \LaTeX{}
distribution, e.g.\ \textit{texmf-root}|/tex/latex/childdoc|.
\end{itemize}

%%%%%%%%%%%%%%%%%%%%%%%%%%%%%%%%%%%%%%%%%%%%%%%%%%%%%%%%%%%%%%%%%%%%%%%%%%%%%%%%
\subsection{Related CTAN Packages}

There are several other packages which offer a similar functionality:
%
\begin{itemize}
\item
The packages
\href{http://ctan.org/pkg/docmute}{\textsf{docmute}},
\href{http://ctan.org/pkg/includex}{\textsf{includex}} and
\href{http://ctan.org/pkg/standalone}{\textsf{standalone}}
provide commands to include only the document body of
a child file thus allowing both files to be compiled individually.
\item
The packages \href{http://ctan.org/pkg/subdocs}{\textsf{subdocs}}
and \href{http://ctan.org/pkg/subfiles}{\textsf{subfiles}}
provide structures in which the main and child documents can be
encapsulated and allowing them to be compiled individually.
The inclusion mechanism is different from the conventional |\include|.
\item
The package \href{http://ctan.org/pkg/combine}{\textsf{combine}}
is an elaborate solution to combine several documents into one.
\end{itemize}
%
See also the CTAN topic \href{http://ctan.org/topic/subdocs}{\textsf{subdocs}}
for further related packages.
The present package differs from the above solutions in that
a document structure constructed with the conventional |\include| mechanism
just needs two extra commands at the top of every file
such that all constituent files can be compiled individually.

%%%%%%%%%%%%%%%%%%%%%%%%%%%%%%%%%%%%%%%%%%%%%%%%%%%%%%%%%%%%%%%%%%%%%%%%%%%%%%%%
%\subsection{Feature Suggestions}
%
%The following is a list of features which may be useful for future
%versions of this package:
%%
%\begin{itemize}
%\item
%\ldots
%\end{itemize}

%%%%%%%%%%%%%%%%%%%%%%%%%%%%%%%%%%%%%%%%%%%%%%%%%%%%%%%%%%%%%%%%%%%%%%%%%%%%%%%%
\subsection{Revision History}

%%%%%%%%%%%%%%%%%%%%%%%%%%%%%%%%%%%%%%%%
\paragraph{v2.0:} 2018/12/30

\begin{itemize}
\item
immediate forward processing
\item
added |\childdocby| mechanism
\item
manual restructured
\end{itemize}

%%%%%%%%%%%%%%%%%%%%%%%%%%%%%%%%%%%%%%%%
\paragraph{v1.6:} 2018/01/17

\begin{itemize}
\item
application for development of include files
\item
corrections to manual
\end{itemize}

%%%%%%%%%%%%%%%%%%%%%%%%%%%%%%%%%%%%%%%%
\paragraph{v1.5:} 2017/05/21

\begin{itemize}
\item
more complete structuring introduced
\item
|\childdocof| introduced
\item
|\childdoc| renamed to |\childdocmain|
\item
|\childredirect| renamed to |\childdocforward| and |\childdocforwardprefix|
and functionality expanded
\end{itemize}

%%%%%%%%%%%%%%%%%%%%%%%%%%%%%%%%%%%%%%%%
\paragraph{v1.0:} 2017/04/27

\begin{itemize}
\item
manual and install package
\item
first version published on CTAN
\end{itemize}

%%%%%%%%%%%%%%%%%%%%%%%%%%%%%%%%%%%%%%%%
\paragraph{v0.6:} 2017/04/26

\begin{itemize}
\item
redirection mechanism added
\end{itemize}

%%%%%%%%%%%%%%%%%%%%%%%%%%%%%%%%%%%%%%%%
\paragraph{v0.5:} 2017/04/26

\begin{itemize}
\item
functionality in definition file
\end{itemize}


%%%%%%%%%%%%%%%%%%%%%%%%%%%%%%%%%%%%%%%%%%%%%%%%%%%%%%%%%%%%%%%%%%%%%%%%%%%%%%%%
%%%%%%%%%%%%%%%%%%%%%%%%%%%%%%%%%%%%%%%%%%%%%%%%%%%%%%%%%%%%%%%%%%%%%%%%%%%%%%%%
%%%%%%%%%%%%%%%%%%%%%%%%%%%%%%%%%%%%%%%%%%%%%%%%%%%%%%%%%%%%%%%%%%%%%%%%%%%%%%%%
\appendix

\settowidth\MacroIndent{\rmfamily\scriptsize 000\ }

 \DocInput{childdoc.dtx}

\end{document}
%</driver>
% \fi
%
% %%%%%%%%%%%%%%%%%%%%%%%%%%%%%%%%%%%%%%%%%%%%%%%%%%%%%%%%%%%%%%%%%%%%%%%%%%%%%%
% %%%%%%%%%%%%%%%%%%%%%%%%%%%%%%%%%%%%%%%%%%%%%%%%%%%%%%%%%%%%%%%%%%%%%%%%%%%%%%
% \section{Sample}
%\iffalse
%<*samplemain>
%\fi
%
% The following presents a sample document
% with two chapters, two parts, a title page,
% a compile flag as well as three forwarding files to set the flag.
% It consists of eight |.tex| files:
% \begin{center}
% \begin{tabular}{ll}
% |cdocsamp.tex|&main file\\
% |cdocsch1.tex|&include file for chapter 1\\
% |cdocsch2.tex|&include file for chapter 2\\
% |cdocspt3.tex|&include file for part 3\\
% |cdocspt4.tex|&include file for part 4\\
% |cdocsdrf.tex|&forwarding file for main file in draft mode\\
% |cdocsfi1.tex|&forwarding file for final version of chapter 1\\
% |cdocsfi2.tex|&forwarding file for final version of chapter 2\\
% \end{tabular}
% \end{center}
% Each of the eight files can be compiled directly by the \LaTeX{} compiler.
%
% %%%%%%%%%%%%%%%%%%%%%%%%%%%%%%%%%%%%%%
% \paragraph{Main File.}
%
% The main file is called |cdocsamp.tex|.
%
% Load the \textsf{childdoc} definitions and
% declare the filename for the main document:
%    \begin{macrocode}
\input{childdoc.def}
\childdocmain{}
%    \end{macrocode}

% Optional override for |\version| flag:
%    \begin{macrocode}
%%\ifchilddoc\else\providecommand{\version}{draft}\fi
%    \end{macrocode}

% Define the default values for the |\version| flag
% (|final| for the main file and |draft| for childs):
%    \begin{macrocode}
\ifchilddoc
\providecommand{\version}{draft}
\else
\providecommand{\version}{final}
\fi
%    \end{macrocode}

% Load the standard document class:
%    \begin{macrocode}
\documentclass[12pt]{article}
%    \end{macrocode}

% Start the document body:
%    \begin{macrocode}
\begin{document}
%    \end{macrocode}

% Declare a title page.
% Print title, part of document being processed and version flag:
%    \begin{macrocode}
\addtocounter{page}{-1}
\begin{center}
{\LARGE\bfseries{}childdoc example\par}
\vspace{1cm}
\ifchilddoc
\ifchilddocmanual part\else chapter\fi:
`\childdocname' of `\childdocjob'\par
\else
main document: `\childdocjob'\par
\fi
version: \version\par
\end{center}
\newpage
%    \end{macrocode}

% Manually include selected file,
% otherwise process as usual:
%    \begin{macrocode}
\ifchilddocmanual
\section*{part `\childdocname'}
\input{\childdocname}
\else
%    \end{macrocode}

% Include the two chapters:
%    \begin{macrocode}
\include{cdocsch1}
\include{cdocsch2}
%    \end{macrocode}

% Include the two parts unless only chapters should be displayed:
%    \begin{macrocode}
\ifchilddoc\else
\section{part three}
\input{cdocspt3}
\section{part four}
\input{cdocspt4}
\fi
%    \end{macrocode}

% Process as usual until here:
%    \begin{macrocode}
\fi
%    \end{macrocode}

% End of document body:
%    \begin{macrocode}
\end{document}
%    \end{macrocode}
%\iffalse
%</samplemain>
%\fi
%
% %%%%%%%%%%%%%%%%%%%%%%%%%%%%%%%%%%%%%%
% \paragraph{Chapter Include Files.}
%
% The include files are called |cdocsch1.tex| and |cdocsch2.tex|.
%
%\iffalse
%<*samplechap1|samplechap2>
%\fi

% Optional override for |\version| flag:
%    \begin{macrocode}
%%\providecommand{\version}{final}
%    \end{macrocode}

% Include the main document:
%    \begin{macrocode}
\input{childdoc.def}
\childdocof{cdocsamp}
%    \end{macrocode}

%\iffalse
%</samplechap1|samplechap2>
%\fi
%
%\iffalse
%<*samplechap1>
%\fi
% Some text for chapter 1:
%    \begin{macrocode}
\section{one}
some text in chapter one
%    \end{macrocode}

%\iffalse
%</samplechap1>
%\fi
% Some text for chapter 2:
%\iffalse
%<*samplechap2>
%\fi
%    \begin{macrocode}
\section{two}
more text in chapter two
%    \end{macrocode}

%\iffalse
%</samplechap2>
%\fi
%
% %%%%%%%%%%%%%%%%%%%%%%%%%%%%%%%%%%%%%%
% \paragraph{Part Include Files.}
%
% The include files are called |cdocspt3.tex| and |cdocspt4.tex|.
%
%\iffalse
%<*samplepart3|samplepart4>
%\fi

% Optional override for |\version| flag:
%    \begin{macrocode}
%%\providecommand{\version}{final}
%    \end{macrocode}

% Include the main document:
%    \begin{macrocode}
\input{childdoc.def}
\childdocby{cdocsamp}
%    \end{macrocode}

%\iffalse
%</samplepart3|samplepart4>
%\fi
%
%\iffalse
%<*samplepart3>
%\fi
% Some text for part 3:
%    \begin{macrocode}
some text in part three
%    \end{macrocode}

%\iffalse
%</samplepart3>
%\fi
% Some text for part 4:
%\iffalse
%<*samplepart4>
%\fi
%    \begin{macrocode}
more text in part four
%    \end{macrocode}

%\iffalse
%</samplepart4>
%\fi
%
% %%%%%%%%%%%%%%%%%%%%%%%%%%%%%%%%%%%%%%
% \paragraph{Forwarding for a Complete Draft.}
%
% The following forwarding file |cdocsdrf.tex|
% compiles the main document in draft mode:
%\iffalse
%<*sampledraft>
%\fi
%    \begin{macrocode}
\def\version{draft}
\input{childdoc.def}
\childdocforward{cdocsamp}
%    \end{macrocode}

%\iffalse
%</sampledraft>
%\fi
%
% %%%%%%%%%%%%%%%%%%%%%%%%%%%%%%%%%%%%%%
% \paragraph{Forwarding for Final Version of the Chapters.}
%
% The following forwarding files |cdocsfn1.tex| and |cdocsfn2.tex|
% (with identical content)
% compile the final versions of the child documents
% |cdocsch1.tex| and |cdocsch2.tex|, respectively:
%\iffalse
%<*samplefinal>
%\fi
%    \begin{macrocode}
\def\version{final}
\input{childdoc.def}
\childdocforwardprefix[cdocsamp]{cdocsfn}{cdocsch}
%    \end{macrocode}

%\iffalse
%</samplefinal>
%\fi
%
% %%%%%%%%%%%%%%%%%%%%%%%%%%%%%%%%%%%%%%
% \paragraph{Command Line Processing.}
%
% The following three command lines generate the output files
% |cdocscld|, |cdocscl1| and |cdocscl2|
% which should be identical to
% |cdocsdrf|, |cdocsch1| and |cdocsfn2|, respectively:
% \begin{center}
% \begin{tabular}{l}
% |latex -jobname cdocscld \|\\
% |  "\def\version{draft}\input{childdoc.def}\childdocforward{cdocsamp}"|\\
% |latex -jobname cdocscl1 \|\\
% |  "\input{childdoc.def}\childdocforward[cdocsamp]{cdocsch1}"|\\
% |latex -jobname cdocscl2 \|\\
% |  "\def\version{final}\input{childdoc.def}\childdocforward{cdocsch2}"|
% \end{tabular}
% \end{center}
% Note that the trailing backslash on each first line
% merely continues the input to the second line
% (for convenient cut ant paste).
% Furthermore, the command |latex| can be replaced by any
% of its alternative versions such as |pdflatex|.
%
% %%%%%%%%%%%%%%%%%%%%%%%%%%%%%%%%%%%%%%%%%%%%%%%%%%%%%%%%%%%%%%%%%%%%%%%%%%%%%%
% %%%%%%%%%%%%%%%%%%%%%%%%%%%%%%%%%%%%%%%%%%%%%%%%%%%%%%%%%%%%%%%%%%%%%%%%%%%%%%
% \section{Implementation}
%\iffalse
%<*package>
%\fi
%
% This section describes the definitions file |childdoc.def|.

% The definitions cannot be loaded using |\usepackage| or |\RequirePackage|
% which has a mechanism to prevent loading a style file more than once.
% When loading the definitions by means of |\input|
% multiple instances have to be prevented manually:
%\iffalse
%This code needs to be before the `\ProvidesFile' directive
%which is defined at the beginning of this file.
%Therefore it is also placed there and commented out here.
%</package>
%<*discard>
%\fi
%    \begin{macrocode}
\ifdefined\childdocmain\endinput\fi
%    \end{macrocode}
%\iffalse
%</discard>
%<*package>
%\fi
%
% \macro{\ifchilddoc}
% \macro{\ifchilddocmanual}
% The conditional |\ifchilddoc| tells whether a
% child (true) or main (false) document is being compiled.
% The conditional |\ifchilddocmanual| tells whether
% the |\includeonly| mechanism is used (false) or
% the selection of child files must be performed manually (true).
% The definitions initialise to false:
%    \begin{macrocode}
\newif\ifchilddoc
\newif\ifchilddocmanual
%    \end{macrocode}

% \macro{\childdocname}
% \macro{\childdocjob}
% The macro |\childdocname| stores the name of the main document
% to be compiled. The macro |\childdocjob| stores the name of
% the document on which the \LaTeX{} compiler was originally invoked.
% The content of |\jobname| cannot be compared
% to filenames specified in the source due to different catcodes.
% The following code rescans |\jobname|, stores the result
% in |\childdocname| and saves a copy in |\childdocjob|:
%    \begin{macrocode}
\edef\childdocname{\scantokens\expandafter{\jobname\noexpand}}
\let\childdocjob\childdocname
%    \end{macrocode}

% \macro{\childdocdisable}
% The macro |\childdocdisable| prevents the main file
% from being processed more than once.
% At this stage, the main document command |\childdocmain|
% is assumed to be called once again where it should do nothing.
% Any subsequent call to it should prevent
% a secondary processing of the main document
% It overwrites the forwarding commands
% |\childdocof| and |\childdocforward|
% with empty macros to prevent further inclusions of the main document:
%    \begin{macrocode}
\newcommand{\childdocdisable}
{
  \renewcommand{\childdocmain}[1]{\renewcommand{\childdocmain}[1]{\endinput}}
  \renewcommand{\childdocof}[1]{}
  \renewcommand{\childdocby}[2][]{}
  \renewcommand{\childdocforward}[2][]{}
  \renewcommand{\childdocdisable}{}
}
%    \end{macrocode}

% \macro{\childdocmain}
% The macro |\childdocmain| is to be called at the top of the main file
% with nothing or the main filename (without extension) as argument.
% First, it breaks loops.
% If the argument is not empty and does not match |\childdocname|
% (which is set by the first inclusion of |childdoc.def|),
% |\ifchilddoc| is set to true, |\includeonly| is applied to the child file
% and |\jobname| is set to the main file
% (for proper handling of |.aux| files):
%    \begin{macrocode}
\newcommand{\childdocmain}[1]
{
  \childdocdisable\childdocmain{}
  \if?#1?\else
    \begingroup
      \def\childdoctmp{#1}
      \ifx\childdoctmp\childdocname
        \def\childdoctmp{}
      \else
        \def\childdoctmp
        {
          \childdoctrue
          \includeonly{\childdocname}
          \def\childdocjob{#1}
          \def\jobname{#1}
        }
      \fi
      \expandafter
    \endgroup
    \childdoctmp
  \fi
}
%    \end{macrocode}

% \macro{\childdocof}
% The command |\childdocof| redirects
% compilation to the main file |#1|.
%    \begin{macrocode}
\newcommand{\childdocof}[1]
{
  \childdocdisable
  \childdoctrue
  \includeonly{\childdocname}
  \def\jobname{#1}
  \def\childdocjob{#1}
  \input{#1}
}
%    \end{macrocode}

% \macro{\childdocby}
% The command |\childdocby| ....
%    \begin{macrocode}
\newcommand{\childdocby}[2][]
{
  \childdocdisable
  \childdoctrue
  \childdocmanualtrue
  \if?#1?\else
    \def\jobname{#2}
  \fi
  \def\childdocjob{#2}
  \input{#2}
  \endinput
}
%    \end{macrocode}

% \macro{\childdocforward}
% The command |\childdocforward| redirects
% compilation to the main file or
% (if the optional argument is given) a child file.
% Parameters are set as if the main file
% or a child file starting with |\childdocof| was compiled.
% Then compilation is handed over to the main file:
%    \begin{macrocode}
\newcommand{\childdocforward}[2][]
{
  \begingroup
    \if?#1?
      \def\childdoctmp
      {
        \def\childdocname{#2}
        \def\childdocjob{#2}
        \def\jobname{#2}
        \input{#2}
        \endinput
      }
    \else
      \def\childdoctmp
      {
        \childdocdisable
        \def\childdocname{#2}
        \childdoctrue
        \includeonly{#2}
        \def\childdocjob{#1}
        \def\jobname{#1}
        \input{#1}
        \endinput
      }
    \fi
    \expandafter
  \endgroup
  \childdoctmp
}
%    \end{macrocode}

% \macro{\childdocforwardprefix}
% The command |\childdocforwardprefix| redirects
% compilation to the main or a child file by means of a pattern.
% The prefix |#1| in the current filename is replaced by |#2|
% and the suffix of the current filename is kept
% (it is assumed that the filename does not contain the substring `|~~~|'
% which is used as a delimiter).
% Compilation is handed over to the new file by |\childdocforward|:
%    \begin{macrocode}
\newcommand{\childdocforwardprefix}[3][]
{
  \begingroup
    \def\childdocextract #2##1~~~{\def\childdoctmp{\childdocforward[#1]{#3##1}}}
    \expandafter\childdocextract\childdocname~~~
    \expandafter
  \endgroup
  \childdoctmp
}
%    \end{macrocode}

% \macro{\childdoc}
% The deprecated macro |\childdoc| is a legacy version of |\childdocmain|:
%    \begin{macrocode}
\newcommand{\childdoc}{\childdocmain}
%    \end{macrocode}

% \macro{\childdocredirect}
% The deprecated macro |\childdocredirect| is a legacy version
% of |\childdocforward| and |\childdocforwardprefix|:
%    \begin{macrocode}
\newcommand{\childdocredirect}[2][]
{
  \begingroup
    \if?#1?
      \def\childdoctmp{\childdocforward{#2}}
    \else
      \def\childdoctmp{\childdocforwardprefix{#1}{#2}}
    \fi
    \expandafter
  \endgroup
  \childdoctmp
}
%    \end{macrocode}

%\iffalse
%</package>
%\fi
%
\endinput

\childdocof{cdocsamp}
%    \end{macrocode}

%\iffalse
%</samplechap1|samplechap2>
%\fi
%
%\iffalse
%<*samplechap1>
%\fi
% Some text for chapter 1:
%    \begin{macrocode}
\section{one}
some text in chapter one
%    \end{macrocode}

%\iffalse
%</samplechap1>
%\fi
% Some text for chapter 2:
%\iffalse
%<*samplechap2>
%\fi
%    \begin{macrocode}
\section{two}
more text in chapter two
%    \end{macrocode}

%\iffalse
%</samplechap2>
%\fi
%
% %%%%%%%%%%%%%%%%%%%%%%%%%%%%%%%%%%%%%%
% \paragraph{Part Include Files.}
%
% The include files are called |cdocspt3.tex| and |cdocspt4.tex|.
%
%\iffalse
%<*samplepart3|samplepart4>
%\fi

% Optional override for |\version| flag:
%    \begin{macrocode}
%%\providecommand{\version}{final}
%    \end{macrocode}

% Include the main document:
%    \begin{macrocode}
% \iffalse
%
% childdoc.dtx Copyright (C) 2017-2018 Niklas Beisert
%
% This work may be distributed and/or modified under the
% conditions of the LaTeX Project Public License, either version 1.3
% of this license or (at your option) any later version.
% The latest version of this license is in
%   http://www.latex-project.org/lppl.txt
% and version 1.3 or later is part of all distributions of LaTeX
% version 2005/12/01 or later.
%
% This work has the LPPL maintenance status `maintained'.
%
% The Current Maintainer of this work is Niklas Beisert.
%
% This work consists of the files childdoc.dtx and childdoc.ins
% and the derived files childdoc.def and cdocsamp.tex with
% cdocsch1.tex, cdocsch2.tex, cdocsdrf.tex, cdocsfn1.tex, cdocsfn2.tex.
%
%<package>\ifdefined\childdocmain\endinput\fi
%<package>\ProvidesFile{childdoc.def}[2018/12/30 v2.0 child document driver]
%<samplemain>\ProvidesFile{cdocsamp.tex}[2018/12/30 v2.0 sample for childdoc]
%<*driver>
%\ProvidesFile{childdoc.drv}[2018/12/30 v2.0 childdoc reference manual file]
\PassOptionsToClass{10pt,a4paper}{article}
\documentclass{ltxdoc}

\usepackage[margin=35mm]{geometry}
\usepackage{hyperref}
\usepackage{hyperxmp}
\usepackage[usenames]{color}

\hypersetup{colorlinks=true}
\hypersetup{pdfstartview=FitH}
\hypersetup{pdfpagemode=UseNone}
\hypersetup{pdfsource={}}
\hypersetup{pdflang={en-UK}}
\hypersetup{pdfcopyright={Copyright 2017-2018 Niklas Beisert.
  This work may be distributed and/or modified under the
  conditions of the LaTeX Project Public License, either version 1.3
  of this license or (at your option) any later version.}}
\hypersetup{pdflicenseurl={http://www.latex-project.org/lppl.txt}}
\hypersetup{pdfcontactaddress={ETH Zurich, ITP, HIT K,
  Wolfgang-Pauli-Strasse 27}}
\hypersetup{pdfcontactpostcode={8093}}
\hypersetup{pdfcontactcity={Zurich}}
\hypersetup{pdfcontactcountry={Switzerland}}
\hypersetup{pdfcontactemail={nbeisert@itp.phys.ethz.ch}}
\hypersetup{pdfcontacturl={http://people.phys.ethz.ch/\xmptilde nbeisert/}}

\newcommand{\secref}[1]{\hyperref[#1]{section \ref*{#1}}}

\parskip1ex
\parindent0pt
\let\olditemize\itemize
\def\itemize{\olditemize\parskip0pt}

\begin{document}

\title{The \textsf{childdoc} Package}
\hypersetup{pdftitle={The childdoc Package}}
\author{Niklas Beisert\\[2ex]
  Institut f\"ur Theoretische Physik\\
  Eidgen\"ossische Technische Hochschule Z\"urich\\
  Wolfgang-Pauli-Strasse 27, 8093 Z\"urich, Switzerland\\[1ex]
  \href{mailto:nbeisert@itp.phys.ethz.ch}
  {\texttt{nbeisert@itp.phys.ethz.ch}}}
\hypersetup{pdfauthor={Niklas Beisert}}
\hypersetup{pdfsubject={Manual for the LaTeX2e Package childdoc}}
\date{30 December 2018, \textsf{v2.0}}
\maketitle

\begin{abstract}\noindent
\textsf{childdoc} is a \LaTeXe{} package
that enables the direct compilation
of document sections included by |\include|
to individual files.
\end{abstract}

\begingroup
\parskip0ex
\tableofcontents
\endgroup

%%%%%%%%%%%%%%%%%%%%%%%%%%%%%%%%%%%%%%%%%%%%%%%%%%%%%%%%%%%%%%%%%%%%%%%%%%%%%%%%
%%%%%%%%%%%%%%%%%%%%%%%%%%%%%%%%%%%%%%%%%%%%%%%%%%%%%%%%%%%%%%%%%%%%%%%%%%%%%%%%
\section{Introduction}

\LaTeX{} provides a mechanism to structure a large document (such as a book)
into a main file and several child files (containing the chapters)
using the |\include| command.
This mechanism is beneficial for documents
which span hundreds of pages in order to
make the source file(s) more manageable.
Moreover, compilation can be restricted to
selected child files by means of the |\includeonly| command.
The latter feature can be used to reduce the compilation time while editing
(this was significantly more useful in the earlier days of \LaTeX{})
or to generate a smaller document which is easier to navigate.
Another application of |\includeonly| is to generate
documents consisting of selected parts of the complete document.

However, there are a few drawbacks of the plain |\include| mechanism:
\begin{itemize}
\item
The child files cannot be compiled on their own,
they can only be compiled via the main file.
A naive editing environment
(such as a text editor with an option
to have the current file processed by \LaTeX)
may require one to switch to the main file before compiling;
attempting to compile the child file produces errors.
\item
The main file must be modified (each time)
to adjust the |\includeonly| command
to the present needs. This easily leaves the main file in a messy state.
\item
The generated document will always carry the filename
of the main document. This is inconvenient if
several child files are to be compiled and
to be kept for distribution.
\end{itemize}

The present package provides a simple interface
to make child files individually compilable by \LaTeX{}.
Compiling a child file then has the same effect as compiling
the main file with an |\includeonly| command
to select the appropriate child.
Moreover the generated document will carry the name of the child
rather than the main file.
This resolves all three above issues.

This feature is meant to make the editing of books,
thesis documents and lecture notes somewhat more convenient.
However, the package can also be used efficiently for
composing a series of documents (such as exercise sheets)
which are typically distributed individually.
It then assists the author in generating the individual documents
(potentially in different versions)
as well as a document containing the collected series.
Another application is in developing style files
or other kinds of included material
where compilation of the style file could redirect
to a sample or test file.

%%%%%%%%%%%%%%%%%%%%%%%%%%%%%%%%%%%%%%%%%%%%%%%%%%%%%%%%%%%%%%%%%%%%%%%%%%%%%%%%
%%%%%%%%%%%%%%%%%%%%%%%%%%%%%%%%%%%%%%%%%%%%%%%%%%%%%%%%%%%%%%%%%%%%%%%%%%%%%%%%
\section{Usage}

First of all, the package \textsf{childdoc} is \emph{not} a standard
\LaTeXe{} |.sty| style file! Therefore it needs to be invoked in
a non-standard way.

%%%%%%%%%%%%%%%%%%%%%%%%%%%%%%%%%%%%%%%%%%%%%%%%%%%%%%%%%%%%%%%%%%%%%%%%%%%%%%%%
\subsection{Included Files}
\label{sec:include}

%%%%%%%%%%%%%%%%%%%%%%%%%%%%%%%%%%%%%%%%
\DescribeMacro{\childdocmain}
To use the package, add the commands
\begin{center}
\begin{tabular}{l}
|\input{childdoc.def}|\\
|\childdocmain{}|\\
\end{tabular}
\end{center}
at the very top of the main \LaTeX{} file,
in particular \emph{before} the |\documentclass| statement!
The argument of |\childdocmain| should be left empty
(but it must be present).

%%%%%%%%%%%%%%%%%%%%%%%%%%%%%%%%%%%%%%%%
\DescribeMacro{\childdocof}
Furthermore, add the commands
\begin{center}
\begin{tabular}{l}
|\input{childdoc.def}|\\
|\childdocof{|\textit{main}|}|\\
\end{tabular}
\end{center}
at the top of every child file \textit{child}
which is included by |\include{|\textit{child}|}|
from within the main file
(or at least for those files to be compiled individually).
The argument \textit{main} must be the filename of the main file.

There are a couple of
considerations in setting up the main and child documents:

%%%%%%%%%%%%%%%%%%%%%%%%%%%%%%%%%%%%%%%%
\paragraph{Restrictions.}

Please note the following restrictions:
\begin{itemize}
\item
|\childdocmain| must be called with one argument \textit{main}
to ensure compatibility with earlier version of the package.
It must either be empty (|\childdocmain{}|)
or precisely match the filename of the main file in which it is specified.
See \secref{sec:detection} for further information.
\item
The filename \textit{main} must be specified without the |.tex| extension.
\item
The filename \textit{main} is case sensitive
(even in case-insensitive file systems)
due to internal string comparison.
\item
The argument \textit{main} should be fully expanded, it cannot be a macro.
\item
Subdirectories and special characters should be avoided in filenames.
\item
The command |\childdocmain{|\textit{main}|}| must be followed by a whitespace.
It should not be followed immediately by another command
or by a comment mark `|%|'.
This is because the \TeX{} parser reads the token immediately following
the argument of |\childdocmain| and puts it
at the beginning of every child section;
however, a white\-space is ignored.
\end{itemize}

%%%%%%%%%%%%%%%%%%%%%%%%%%%%%%%%%%%%%%%%
\paragraph{Content of Main File.}

It is advisable to place all content in the child files included by |\include|.
Any output contained in the main file will appear in all child documents
unless suppressed manually;
it cannot be suppressed automatically by the |\includeonly| directive
and thus should normally be avoided.
A method to include some content in the main file
by means of conditional processing is described in \secref{sec:conditional}.

%%%%%%%%%%%%%%%%%%%%%%%%%%%%%%%%%%%%%%%%
\paragraph{Page Numbering.}

When only a part of the document is compiled,
the appropriate numbering of pages
(as well as other status parameters)
is determined from the |.aux| files.
The latter contain information from previous passes.
However this information needs to propagate through
all intermediate child documents.
Therefore the page numbering in child documents may well
be inconsistent until the complete document is compiled at least once.

A useful (if unconventional) way to always ensure a consistent
page numbering is to restart the numbering in each child document
and denote the pages by `\textit{child}|.|\textit{page}'
where \textit{child} represents the chapter/section number of the child file.
This can be achieved by the command
|\numberwithin{page}{|\textit{child}|}|
of the \textsf{amsmath} package
where \textit{child} can be |chapter| or |section|
depending on the chosen structuring.
Alternatively, one can modify the macro |\thepage| appropriately
and reset the counter |page| at the start of each child file.

%%%%%%%%%%%%%%%%%%%%%%%%%%%%%%%%%%%%%%%%%%%%%%%%%%%%%%%%%%%%%%%%%%%%%%%%%%%%%%%%
\subsection{Conditional Processing}
\label{sec:conditional}

The package provides a mechanism to compile different versions
of a document. To customise the versions further some conditional processing
can come in handy to distinguish which version is being compiled.
The package provides two macros to describe the compilation context:

%%%%%%%%%%%%%%%%%%%%%%%%%%%%%%%%%%%%%%%%
\DescribeMacro{\ifchilddoc}
The conditional |\ifchilddoc| distinguishes between the compilation of
child documents and the main document:
%
\begin{center}
|\ifchilddoc |\textit{child-code}| |[|\||else |\textit{main-code}]| \||fi|
\end{center}

%%%%%%%%%%%%%%%%%%%%%%%%%%%%%%%%%%%%%%%%
\DescribeMacro{\childdocname}
\DescribeMacro{\childdocjob}
The macro |\childdocname| contains the filename (without extension)
of the main or child file being processed.
Note that |\childdocjob| will always contain the name of the main file.

%%%%%%%%%%%%%%%%%%%%%%%%%%%%%%%%%%%%%%%%
\paragraph{Title Page.}

Conditional processing can be used to include a title or banner page
in the main document when proper precautions are taken.
Importantly, the code in the main file should ensure that the page counter
(as well as other status parameters which are stored in the |.aux| files)
takes the same value after the conditional processing.
Otherwise the page numbers may take divergent values
depending on which part is compiled.

For example, a title page could be declared by:
%
\begin{center}
\begin{tabular}{l}
|\ifchilddoc\||else|\\
|\addtocounter{page}{-1}|\\
\textit{code for title page}\\
|\newpage|\\
|\||fi|
\end{tabular}
\end{center}
%
A banner page for the child documents can be generated by:
%
\begin{center}
\begin{tabular}{l}
|\ifchilddoc|\\
|\addtocounter{page}{-1}|\\
\textit{code for banner page}\\
|\newpage|\\
|\||fi|
\end{tabular}
\end{center}
%
Here one could write a message such as:
\begin{center}
|This is the part \childdocname{} of \childdocjob{}.|
\end{center}

%%%%%%%%%%%%%%%%%%%%%%%%%%%%%%%%%%%%%%%%%%%%%%%%%%%%%%%%%%%%%%%%%%%%%%%%%%%%%%%%
\subsection{Flags}
\label{sec:flags}

The package makes it easy to generate different versions
of the main or child documents.
To this end compilation flags can be defined
and assigned different default values.
They will be particularly useful in conjunction
with the forwarding mechanism described in \secref{sec:forward}.

For example, it may be useful to have a flag |\version|
which can be set to |draft| or |final|.
The document source will contain some conditional code
depending on the value of |\version|.
Suppose further, the flag should default to |final| for the main file
and to |draft| for child files
which is a natural assignment for editing the document.
This is achieved by placing the following code
in the preamble of the main document
(below the |\childdocmain| directive):
%
\begin{center}
\begin{tabular}{l}
|\ifchilddoc|\\
|\providecommand{\version}{draft}|\\
|\||else|\\
|\providecommand{\version}{final}|\\
|\||fi|
\end{tabular}
\end{center}
%
The definition by |\providecommand| makes sure
that previous definitions are not overwritten.
Further statements |\providecommand{\version}{...}|
can thus be added before the above code to override it.

For the main file, one might add a line
(between |\childdocmain| and the above block)
%
\begin{center}
|%\ifchilddoc\||else\providecommand{\version}{draft}\||fi|
\end{center}
%
which can be uncommented to produce a draft version.
Likewise one can add a line to the very top of a child file
(above the |\childdocof{|\textit{main}|}| directive)
%
\begin{center}
|%\providecommand{\version}{final}|
\end{center}
%
which can be uncommented to produce the final version of this child document.

%%%%%%%%%%%%%%%%%%%%%%%%%%%%%%%%%%%%%%%%%%%%%%%%%%%%%%%%%%%%%%%%%%%%%%%%%%%%%%%%
\subsection{Forwarding}
\label{sec:forward}

Different versions of the main or child documents
using compilation flags as described in \secref{sec:flags}
can be (permanently) stored in different files
for convenient compilation, viewing and distribution.
To this end, the package defines a command
to pass on compilation to a different file:

%%%%%%%%%%%%%%%%%%%%%%%%%%%%%%%%%%%%%%%%
\DescribeMacro{\childdocforward}
The command |\childdocforward| redirects processing to
another source file:
%
\begin{center}
\begin{tabular}{l}
|\input{childdoc.def}|\\
|\childdocforward[|\textit{main}|]{|\textit{dest}|}|\\
\end{tabular}
\end{center}
%
The argument \textit{dest} is the destination file
(without extension).
It should be the main file or one of the child files.
Note that further \textsf{childdoc} directives
such as |\childdocof| and |\childdocforward|
in the indicated file will be processed in this form.
The optional argument \textit{main}
passes on directly to the main file \textit{main}
while pretending to compile the child \textit{dest}.
This form behaves as if \textit{dest}
issues |\childdocof{|\textit{main}|}| right away,
and no further \textsf{childdoc} directives will be processed.

%%%%%%%%%%%%%%%%%%%%%%%%%%%%%%%%%%%%%%%%
\DescribeMacro{\...prefix}
In the alternative form |\childdocforwardprefix|,
%
\begin{center}
\begin{tabular}{l}
|\input{childdoc.def}|\\
|\childdocforwardprefix[|\textit{main}|]{|\textit{prefix}|}{|\textit{dest}|}|
\end{tabular}
\end{center}
%
the destination file is determined by a pattern
depending on the current file:
To make this work, the current file must be called
`{\textit{prefix}\hspace{0.2em}\textit{suffix}}'
with \textit{prefix} matching precisely the argument.
Processing is then passed on to the file
`{\textit{dest}\hspace{0.2em}\textit{suffix}}'.
Surely, the same effect is achieved by
directly specifying the
argument `{\textit{dest}\hspace{0.2em}\textit{suffix}}'
in the first form.
However, that requires to set up a different file
for each child. With the alternative form of the command
all these files can have exactly the same content
which simplifies setting them up and maintaining them.

For example, the following file |draft.tex|
with a compilation flag |\version| as described in \secref{sec:flags}
compiles the main document as a draft:
%
\begin{center}
\begin{tabular}{l}
|\def\version{draft}|\\
|\input{childdoc.def}|\\
|\childdocforward{|\textit{main}|}|
\end{tabular}
\end{center}
%
Likewise, the following files |final|\textit{nn}|.tex|
compile the final version of the child document
|child|\textit{nn}|.tex|:
%
\begin{center}
\begin{tabular}{l}
|\def\version{final}|\\
|\input{childdoc.def}|\\
|\childdocforwardprefix{final}{child}|
\end{tabular}
\end{center}
%

Note that when several versions of a main file and/or of each child file
are to be generated, it may be convenient to set up a |Makefile| or
shell script to automatise the process.

%%%%%%%%%%%%%%%%%%%%%%%%%%%%%%%%%%%%%%%%%%%%%%%%%%%%%%%%%%%%%%%%%%%%%%%%%%%%%%%%
\subsection{Command Line Processing}
\label{sec:commandline}

The effect of redirection files can also be achieved by invoking
the \LaTeX{} compiler with a more elaborate command line.
Most conveniently this should be done as part
of a shell script or a |Makefile|.

When using \textsf{childdoc} in the main file, the following
command lines effectively perform a redirection
(note that depending on the shell being used,
backslashes may have to be doubled: `|\|' $\to$ `|\\|'):
%
\begin{center}
|... -jobname "|\textit{target}|" |\\|"|[\textit{flags}]%
|\input{childdoc.def}\childdocforward[|\textit{main}|]{|\textit{dest}|}"|
\end{center}
%
Here \textit{target} is the name of the output file,
\textit{main} is the name of the main file
and \textit{dest} is the name of the main or child file to be processed
(all filenames without extensions).
The optional argument \textit{main} can be omitted
if \textit{main} matches \textit{dest}.
Optionally, compilation \textit{flags} can be defined via |\def| commands.
This command line makes the \TeX{} engine believe
it is compiling the file \textit{target}
whose content is specified as the latter parameter.
The provided code then forwards the processing to
\textit{main} or \textit{dest} as described in \secref{sec:forward}.

%%%%%%%%%%%%%%%%%%%%%%%%%%%%%%%%%%%%%%%%%%%%%%%%%%%%%%%%%%%%%%%%%%%%%%%%%%%%%%%%
\subsection{Include by Input}
\label{sec:input}

Including child documents by |\include| has some restrictions by design.
Most notably, the content of a child document always occupies
its own set of pages; pages cannot be shared between child documents.
Usually, this behaviour makes perfect sense
because each child document contain an essential part of the document.
However, in some situations it may be desirable to compose
a document from a collection of parts
without having mandatory page breaks between then.
For this case, the package
provides a mechanism to include parts
by |\input| which can also be processed individually.
However, by construction this mechanism
requires manual handling of the content to be output.

%%%%%%%%%%%%%%%%%%%%%%%%%%%%%%%%%%%%%%%%
\DescribeMacro{\ifchilddocmanual}
The main file should be prepared as usual, see \secref{sec:include}.
However, the document body must make a distinction
between processing of an individual part and of the main document, e.g.:
%
\begin{center}
\begin{tabular}{l}
|\ifchilddocmanual|\\
|\input{\childdocname}|\\
|\||else|\\
\textit{document body with }|\input{|\textit{part}|}|\\
|\||fi|
\end{tabular}
\end{center}
%
The conditional |\ifchilddocmanual| is true whenever
a part to be included by |\input| is being compiled,
and the name of the part is stored in |\childdocname|.

%%%%%%%%%%%%%%%%%%%%%%%%%%%%%%%%%%%%%%%%
\DescribeMacro{\childdocby}
Each part to be included by |\input| should start with:
%
\begin{center}
\begin{tabular}{l}
|\input{childdoc.def}|\\
|\childdocby{|\textit{main}|}|\\
\end{tabular}
\end{center}
%
The directive |\childdocby| is similar to |\childdocof|
described in \secref{sec:include},
but the subsequent selection of content must be done manually.
To that end, both |\ifchilddoc| and |\ifchilddocmanual|
will be true upon processing of a part,
and the name of the part is stored in |\childdocname|.
Note that |\jobname| will be set to the filename of the current part
so that each part receives an individual |.aux| file
that does not interfere with the |.aux| file(s) of the main document.
This behaviour can be altered by the alternative form
|\childdocby[*]{|\textit{main}|}| (with a non-empty optional argument)
which uses the |.aux| file of the main document
by setting |\jobname| to \textit{main}.

%%%%%%%%%%%%%%%%%%%%%%%%%%%%%%%%%%%%%%%%%%%%%%%%%%%%%%%%%%%%%%%%%%%%%%%%%%%%%%%%
\subsection{Driver Development}
\label{sec:driver}

The \textsf{childdoc} mechanism can also be use for the development
of definition files such as \LaTeX{} styles or classes.
This case differs from the above setup with multiple parts
included by |\include| in that no |\includeonly| should be invoked.
This can be achieved by starting the include file
(before |\ProvidesPackage|) with:
%
\begin{center}
\begin{tabular}{l}
|\input{childdoc.def}|\\
|\childdocforward{|\textit{main}|}|\\
\end{tabular}
\end{center}
%
or alternatively with:
%
\begin{center}
\begin{tabular}{l}
|\input{childdoc.def}|\\
|\childdocby{|\textit{main}|}|\\
\end{tabular}
\end{center}
%
Both forms have slightly different effects as described above.
The main file is prepared as usual, see \secref{sec:include}.

%%%%%%%%%%%%%%%%%%%%%%%%%%%%%%%%%%%%%%%%%%%%%%%%%%%%%%%%%%%%%%%%%%%%%%%%%%%%%%%%
\subsection{Legacy Detection}
\label{sec:detection}

The directive |\childdocmain| in the main file can detect
whether the complete document or merely a child is to be compiled
even without using the directive |\childdocof|.
This method is deprecated because it is less robust
and there is no compelling reason to use it;
it is merely provided for backward compatibility
and it may be removed in future versions.

If the detection mechanism is to be used,
it is mandatory to correctly specify
the filename of the main file as the argument of |\childdocmain|:
%
\begin{center}
\begin{tabular}{l}
|\input{childdoc.def}|\\
|\childdocmain{|\textit{main}|}|\\
\end{tabular}
\end{center}
%
If |\jobname| does not match the argument \textit{main} of |\childdocmain|,
it is assumed that |\jobname| points to the child file to be compiled.
When using |\childdocmain| with the main file specified as argument,
it suffices to start a child file
with just |\input{|\textit{main}|}|
without loading of the package and using |\childdocof|.
If instead all processing is done
with the appropriate \textsf{childdoc} directives,
the argument of \textit{main} of |\childdocmain| can be empty.

An alternative version of the command line processing described
in \secref{sec:commandline} using the detection mechanism reads:
%
\begin{center}
|... -jobname "|\textit{target}|" "|[\textit{flags}]%
[|\def\jobname{|\textit{dest}|}|]|\input{|\textit{main}|}"|
\end{center}

%%%%%%%%%%%%%%%%%%%%%%%%%%%%%%%%%%%%%%%%%%%%%%%%%%%%%%%%%%%%%%%%%%%%%%%%%%%%%%%%
\subsection{Manual Code}
\label{sec:manual}

In case one cannot be certain whether the definitions file |childdoc.def|
is installed on the target \TeX{} distribution
and one prefers not to ship it,
it is conceivable to paste a few relevant commands into the sources.

To that end, drop all statements |\input{childdoc.def}|
and perform the replacements as outlined below.
Instead of |\childdocmain{|\textit{main}|}| add the following code
to the top of the main file:
%
\begin{center}
\begin{tabular}{l}
|\||ifdefined\childdocname\endinput\||fi\newif\ifchilddoc|\\
|\edef\childdocname{\scantokens\expandafter{\jobname\noexpand}}|\\
|\def\childdocmain{|\textit{main}|}\||ifx\childdocmain\childdocname\||else|\\
|\childdoctrue\includeonly{\childdocname}\let\jobname\childdocmain\||fi|\\
\end{tabular}
\end{center}
%
Instead of |\childdocof{|\textit{main}|}| just include the main file
at the top of each child file:
%
\begin{center}
|\input{|\textit{main}|}|
\end{center}
%
A simple redirection |\childdocforward{|\textit{dest}|}| is achieved by:
%
\begin{center}
|\def\jobname{|\textit{dest}|}\input{\jobname}|
\end{center}
%
The redirection with prefix
|\childdocforwardprefix[|\textit{prefix}|]{|\textit{dest}|}|
is accomplished by:
%
\begin{center}
\begin{tabular}{l}
|{\edef\jobname{\scantokens\expandafter{\jobname\noexpand}}|\\
|\def\redirectjob |\textit{prefix}|#1~~~{\gdef\jobname{|\textit{dest}|#1}}|\\
|\expandafter\redirectjob\jobname~~~}\input{\jobname}|
\end{tabular}
\end{center}

In an alternative approach,
child documents can be compiled by a specific command line
without additional code or specific definitions:
%
\begin{center}
|... -jobname "|\textit{target}|" "|[\textit{flags}]%
|\includeonly{|\textit{dest}|}\input{|\textit{main}|}"|
\end{center}
%

%%%%%%%%%%%%%%%%%%%%%%%%%%%%%%%%%%%%%%%%%%%%%%%%%%%%%%%%%%%%%%%%%%%%%%%%%%%%%%%%
%%%%%%%%%%%%%%%%%%%%%%%%%%%%%%%%%%%%%%%%%%%%%%%%%%%%%%%%%%%%%%%%%%%%%%%%%%%%%%%%
\section{Information}

%%%%%%%%%%%%%%%%%%%%%%%%%%%%%%%%%%%%%%%%%%%%%%%%%%%%%%%%%%%%%%%%%%%%%%%%%%%%%%%%
\subsection{Copyright}

Copyright \copyright{} 2017--2018 Niklas Beisert

This work may be distributed and/or modified under the
conditions of the \LaTeX{} Project Public License, either version 1.3
of this license or (at your option) any later version.
The latest version of this license is in
  \url{http://www.latex-project.org/lppl.txt}
and version 1.3 or later is part of all distributions of \LaTeX{}
version 2005/12/01 or later.

This work has the LPPL maintenance status `maintained'.

The Current Maintainer of this work is Niklas Beisert.

This work consists of the files |README.txt|, |childdoc.ins| and |childdoc.dtx|
as well as the derived files |childdoc.def|, |cdocsamp.tex|
with |cdocsch1.tex|, |cdocsch2.tex|, |cdocspt3.tex|, |cdocspt4.tex|,
|cdocsdrf.tex|, |cdocsfn1.tex|, |cdocsfn2.tex|
as well as |childdoc.pdf|.

%%%%%%%%%%%%%%%%%%%%%%%%%%%%%%%%%%%%%%%%%%%%%%%%%%%%%%%%%%%%%%%%%%%%%%%%%%%%%%%%
\subsection{Files and Installation}

The package consists of the files:
%
\begin{center}
\begin{tabular}{ll}
    |README.txt|   & readme file \\
    |childdoc.ins| & installation file \\
    |childdoc.dtx| & source file \\
    |childdoc.def| & definition file \\
    |cdocsamp.tex| & sample main file \\
    |cdocsch1.tex| & sample include file \\
    |cdocsch2.tex| & sample include file \\
    |cdocspt3.tex| & sample part file \\
    |cdocspt4.tex| & sample part file \\
    |cdocsdrf.tex| & sample redirection file \\
    |cdocsfn1.tex| & sample redirection file \\
    |cdocsfn2.tex| & sample redirection file \\
    |childdoc.pdf| & manual
\end{tabular}
\end{center}
%
The distribution consists of the files
|README.txt|, |childdoc.ins| and |childdoc.dtx|.
%
\begin{itemize}
\item
Run (pdf)\LaTeX{} on |childdoc.dtx|
to compile the manual |childdoc.pdf| (this file).
\item
Run \LaTeX{} on |childdoc.ins| to create the definitions file |childdoc.def|
and the sample |cdocsamp.tex| with include files
|cdocsch1.tex|, |cdocsch2.tex|, |cdocspt3.tex|, |cdocspt4.tex|,
|cdocsdrf.tex|, |cdocsfn1.tex|, |cdocsfn2.tex|.
Then copy the file |childdoc.def| to an appropriate directory of your \LaTeX{}
distribution, e.g.\ \textit{texmf-root}|/tex/latex/childdoc|.
\end{itemize}

%%%%%%%%%%%%%%%%%%%%%%%%%%%%%%%%%%%%%%%%%%%%%%%%%%%%%%%%%%%%%%%%%%%%%%%%%%%%%%%%
\subsection{Related CTAN Packages}

There are several other packages which offer a similar functionality:
%
\begin{itemize}
\item
The packages
\href{http://ctan.org/pkg/docmute}{\textsf{docmute}},
\href{http://ctan.org/pkg/includex}{\textsf{includex}} and
\href{http://ctan.org/pkg/standalone}{\textsf{standalone}}
provide commands to include only the document body of
a child file thus allowing both files to be compiled individually.
\item
The packages \href{http://ctan.org/pkg/subdocs}{\textsf{subdocs}}
and \href{http://ctan.org/pkg/subfiles}{\textsf{subfiles}}
provide structures in which the main and child documents can be
encapsulated and allowing them to be compiled individually.
The inclusion mechanism is different from the conventional |\include|.
\item
The package \href{http://ctan.org/pkg/combine}{\textsf{combine}}
is an elaborate solution to combine several documents into one.
\end{itemize}
%
See also the CTAN topic \href{http://ctan.org/topic/subdocs}{\textsf{subdocs}}
for further related packages.
The present package differs from the above solutions in that
a document structure constructed with the conventional |\include| mechanism
just needs two extra commands at the top of every file
such that all constituent files can be compiled individually.

%%%%%%%%%%%%%%%%%%%%%%%%%%%%%%%%%%%%%%%%%%%%%%%%%%%%%%%%%%%%%%%%%%%%%%%%%%%%%%%%
%\subsection{Feature Suggestions}
%
%The following is a list of features which may be useful for future
%versions of this package:
%%
%\begin{itemize}
%\item
%\ldots
%\end{itemize}

%%%%%%%%%%%%%%%%%%%%%%%%%%%%%%%%%%%%%%%%%%%%%%%%%%%%%%%%%%%%%%%%%%%%%%%%%%%%%%%%
\subsection{Revision History}

%%%%%%%%%%%%%%%%%%%%%%%%%%%%%%%%%%%%%%%%
\paragraph{v2.0:} 2018/12/30

\begin{itemize}
\item
immediate forward processing
\item
added |\childdocby| mechanism
\item
manual restructured
\end{itemize}

%%%%%%%%%%%%%%%%%%%%%%%%%%%%%%%%%%%%%%%%
\paragraph{v1.6:} 2018/01/17

\begin{itemize}
\item
application for development of include files
\item
corrections to manual
\end{itemize}

%%%%%%%%%%%%%%%%%%%%%%%%%%%%%%%%%%%%%%%%
\paragraph{v1.5:} 2017/05/21

\begin{itemize}
\item
more complete structuring introduced
\item
|\childdocof| introduced
\item
|\childdoc| renamed to |\childdocmain|
\item
|\childredirect| renamed to |\childdocforward| and |\childdocforwardprefix|
and functionality expanded
\end{itemize}

%%%%%%%%%%%%%%%%%%%%%%%%%%%%%%%%%%%%%%%%
\paragraph{v1.0:} 2017/04/27

\begin{itemize}
\item
manual and install package
\item
first version published on CTAN
\end{itemize}

%%%%%%%%%%%%%%%%%%%%%%%%%%%%%%%%%%%%%%%%
\paragraph{v0.6:} 2017/04/26

\begin{itemize}
\item
redirection mechanism added
\end{itemize}

%%%%%%%%%%%%%%%%%%%%%%%%%%%%%%%%%%%%%%%%
\paragraph{v0.5:} 2017/04/26

\begin{itemize}
\item
functionality in definition file
\end{itemize}


%%%%%%%%%%%%%%%%%%%%%%%%%%%%%%%%%%%%%%%%%%%%%%%%%%%%%%%%%%%%%%%%%%%%%%%%%%%%%%%%
%%%%%%%%%%%%%%%%%%%%%%%%%%%%%%%%%%%%%%%%%%%%%%%%%%%%%%%%%%%%%%%%%%%%%%%%%%%%%%%%
%%%%%%%%%%%%%%%%%%%%%%%%%%%%%%%%%%%%%%%%%%%%%%%%%%%%%%%%%%%%%%%%%%%%%%%%%%%%%%%%
\appendix

\settowidth\MacroIndent{\rmfamily\scriptsize 000\ }

 \DocInput{childdoc.dtx}

\end{document}
%</driver>
% \fi
%
% %%%%%%%%%%%%%%%%%%%%%%%%%%%%%%%%%%%%%%%%%%%%%%%%%%%%%%%%%%%%%%%%%%%%%%%%%%%%%%
% %%%%%%%%%%%%%%%%%%%%%%%%%%%%%%%%%%%%%%%%%%%%%%%%%%%%%%%%%%%%%%%%%%%%%%%%%%%%%%
% \section{Sample}
%\iffalse
%<*samplemain>
%\fi
%
% The following presents a sample document
% with two chapters, two parts, a title page,
% a compile flag as well as three forwarding files to set the flag.
% It consists of eight |.tex| files:
% \begin{center}
% \begin{tabular}{ll}
% |cdocsamp.tex|&main file\\
% |cdocsch1.tex|&include file for chapter 1\\
% |cdocsch2.tex|&include file for chapter 2\\
% |cdocspt3.tex|&include file for part 3\\
% |cdocspt4.tex|&include file for part 4\\
% |cdocsdrf.tex|&forwarding file for main file in draft mode\\
% |cdocsfi1.tex|&forwarding file for final version of chapter 1\\
% |cdocsfi2.tex|&forwarding file for final version of chapter 2\\
% \end{tabular}
% \end{center}
% Each of the eight files can be compiled directly by the \LaTeX{} compiler.
%
% %%%%%%%%%%%%%%%%%%%%%%%%%%%%%%%%%%%%%%
% \paragraph{Main File.}
%
% The main file is called |cdocsamp.tex|.
%
% Load the \textsf{childdoc} definitions and
% declare the filename for the main document:
%    \begin{macrocode}
\input{childdoc.def}
\childdocmain{}
%    \end{macrocode}

% Optional override for |\version| flag:
%    \begin{macrocode}
%%\ifchilddoc\else\providecommand{\version}{draft}\fi
%    \end{macrocode}

% Define the default values for the |\version| flag
% (|final| for the main file and |draft| for childs):
%    \begin{macrocode}
\ifchilddoc
\providecommand{\version}{draft}
\else
\providecommand{\version}{final}
\fi
%    \end{macrocode}

% Load the standard document class:
%    \begin{macrocode}
\documentclass[12pt]{article}
%    \end{macrocode}

% Start the document body:
%    \begin{macrocode}
\begin{document}
%    \end{macrocode}

% Declare a title page.
% Print title, part of document being processed and version flag:
%    \begin{macrocode}
\addtocounter{page}{-1}
\begin{center}
{\LARGE\bfseries{}childdoc example\par}
\vspace{1cm}
\ifchilddoc
\ifchilddocmanual part\else chapter\fi:
`\childdocname' of `\childdocjob'\par
\else
main document: `\childdocjob'\par
\fi
version: \version\par
\end{center}
\newpage
%    \end{macrocode}

% Manually include selected file,
% otherwise process as usual:
%    \begin{macrocode}
\ifchilddocmanual
\section*{part `\childdocname'}
\input{\childdocname}
\else
%    \end{macrocode}

% Include the two chapters:
%    \begin{macrocode}
\include{cdocsch1}
\include{cdocsch2}
%    \end{macrocode}

% Include the two parts unless only chapters should be displayed:
%    \begin{macrocode}
\ifchilddoc\else
\section{part three}
\input{cdocspt3}
\section{part four}
\input{cdocspt4}
\fi
%    \end{macrocode}

% Process as usual until here:
%    \begin{macrocode}
\fi
%    \end{macrocode}

% End of document body:
%    \begin{macrocode}
\end{document}
%    \end{macrocode}
%\iffalse
%</samplemain>
%\fi
%
% %%%%%%%%%%%%%%%%%%%%%%%%%%%%%%%%%%%%%%
% \paragraph{Chapter Include Files.}
%
% The include files are called |cdocsch1.tex| and |cdocsch2.tex|.
%
%\iffalse
%<*samplechap1|samplechap2>
%\fi

% Optional override for |\version| flag:
%    \begin{macrocode}
%%\providecommand{\version}{final}
%    \end{macrocode}

% Include the main document:
%    \begin{macrocode}
\input{childdoc.def}
\childdocof{cdocsamp}
%    \end{macrocode}

%\iffalse
%</samplechap1|samplechap2>
%\fi
%
%\iffalse
%<*samplechap1>
%\fi
% Some text for chapter 1:
%    \begin{macrocode}
\section{one}
some text in chapter one
%    \end{macrocode}

%\iffalse
%</samplechap1>
%\fi
% Some text for chapter 2:
%\iffalse
%<*samplechap2>
%\fi
%    \begin{macrocode}
\section{two}
more text in chapter two
%    \end{macrocode}

%\iffalse
%</samplechap2>
%\fi
%
% %%%%%%%%%%%%%%%%%%%%%%%%%%%%%%%%%%%%%%
% \paragraph{Part Include Files.}
%
% The include files are called |cdocspt3.tex| and |cdocspt4.tex|.
%
%\iffalse
%<*samplepart3|samplepart4>
%\fi

% Optional override for |\version| flag:
%    \begin{macrocode}
%%\providecommand{\version}{final}
%    \end{macrocode}

% Include the main document:
%    \begin{macrocode}
\input{childdoc.def}
\childdocby{cdocsamp}
%    \end{macrocode}

%\iffalse
%</samplepart3|samplepart4>
%\fi
%
%\iffalse
%<*samplepart3>
%\fi
% Some text for part 3:
%    \begin{macrocode}
some text in part three
%    \end{macrocode}

%\iffalse
%</samplepart3>
%\fi
% Some text for part 4:
%\iffalse
%<*samplepart4>
%\fi
%    \begin{macrocode}
more text in part four
%    \end{macrocode}

%\iffalse
%</samplepart4>
%\fi
%
% %%%%%%%%%%%%%%%%%%%%%%%%%%%%%%%%%%%%%%
% \paragraph{Forwarding for a Complete Draft.}
%
% The following forwarding file |cdocsdrf.tex|
% compiles the main document in draft mode:
%\iffalse
%<*sampledraft>
%\fi
%    \begin{macrocode}
\def\version{draft}
\input{childdoc.def}
\childdocforward{cdocsamp}
%    \end{macrocode}

%\iffalse
%</sampledraft>
%\fi
%
% %%%%%%%%%%%%%%%%%%%%%%%%%%%%%%%%%%%%%%
% \paragraph{Forwarding for Final Version of the Chapters.}
%
% The following forwarding files |cdocsfn1.tex| and |cdocsfn2.tex|
% (with identical content)
% compile the final versions of the child documents
% |cdocsch1.tex| and |cdocsch2.tex|, respectively:
%\iffalse
%<*samplefinal>
%\fi
%    \begin{macrocode}
\def\version{final}
\input{childdoc.def}
\childdocforwardprefix[cdocsamp]{cdocsfn}{cdocsch}
%    \end{macrocode}

%\iffalse
%</samplefinal>
%\fi
%
% %%%%%%%%%%%%%%%%%%%%%%%%%%%%%%%%%%%%%%
% \paragraph{Command Line Processing.}
%
% The following three command lines generate the output files
% |cdocscld|, |cdocscl1| and |cdocscl2|
% which should be identical to
% |cdocsdrf|, |cdocsch1| and |cdocsfn2|, respectively:
% \begin{center}
% \begin{tabular}{l}
% |latex -jobname cdocscld \|\\
% |  "\def\version{draft}\input{childdoc.def}\childdocforward{cdocsamp}"|\\
% |latex -jobname cdocscl1 \|\\
% |  "\input{childdoc.def}\childdocforward[cdocsamp]{cdocsch1}"|\\
% |latex -jobname cdocscl2 \|\\
% |  "\def\version{final}\input{childdoc.def}\childdocforward{cdocsch2}"|
% \end{tabular}
% \end{center}
% Note that the trailing backslash on each first line
% merely continues the input to the second line
% (for convenient cut ant paste).
% Furthermore, the command |latex| can be replaced by any
% of its alternative versions such as |pdflatex|.
%
% %%%%%%%%%%%%%%%%%%%%%%%%%%%%%%%%%%%%%%%%%%%%%%%%%%%%%%%%%%%%%%%%%%%%%%%%%%%%%%
% %%%%%%%%%%%%%%%%%%%%%%%%%%%%%%%%%%%%%%%%%%%%%%%%%%%%%%%%%%%%%%%%%%%%%%%%%%%%%%
% \section{Implementation}
%\iffalse
%<*package>
%\fi
%
% This section describes the definitions file |childdoc.def|.

% The definitions cannot be loaded using |\usepackage| or |\RequirePackage|
% which has a mechanism to prevent loading a style file more than once.
% When loading the definitions by means of |\input|
% multiple instances have to be prevented manually:
%\iffalse
%This code needs to be before the `\ProvidesFile' directive
%which is defined at the beginning of this file.
%Therefore it is also placed there and commented out here.
%</package>
%<*discard>
%\fi
%    \begin{macrocode}
\ifdefined\childdocmain\endinput\fi
%    \end{macrocode}
%\iffalse
%</discard>
%<*package>
%\fi
%
% \macro{\ifchilddoc}
% \macro{\ifchilddocmanual}
% The conditional |\ifchilddoc| tells whether a
% child (true) or main (false) document is being compiled.
% The conditional |\ifchilddocmanual| tells whether
% the |\includeonly| mechanism is used (false) or
% the selection of child files must be performed manually (true).
% The definitions initialise to false:
%    \begin{macrocode}
\newif\ifchilddoc
\newif\ifchilddocmanual
%    \end{macrocode}

% \macro{\childdocname}
% \macro{\childdocjob}
% The macro |\childdocname| stores the name of the main document
% to be compiled. The macro |\childdocjob| stores the name of
% the document on which the \LaTeX{} compiler was originally invoked.
% The content of |\jobname| cannot be compared
% to filenames specified in the source due to different catcodes.
% The following code rescans |\jobname|, stores the result
% in |\childdocname| and saves a copy in |\childdocjob|:
%    \begin{macrocode}
\edef\childdocname{\scantokens\expandafter{\jobname\noexpand}}
\let\childdocjob\childdocname
%    \end{macrocode}

% \macro{\childdocdisable}
% The macro |\childdocdisable| prevents the main file
% from being processed more than once.
% At this stage, the main document command |\childdocmain|
% is assumed to be called once again where it should do nothing.
% Any subsequent call to it should prevent
% a secondary processing of the main document
% It overwrites the forwarding commands
% |\childdocof| and |\childdocforward|
% with empty macros to prevent further inclusions of the main document:
%    \begin{macrocode}
\newcommand{\childdocdisable}
{
  \renewcommand{\childdocmain}[1]{\renewcommand{\childdocmain}[1]{\endinput}}
  \renewcommand{\childdocof}[1]{}
  \renewcommand{\childdocby}[2][]{}
  \renewcommand{\childdocforward}[2][]{}
  \renewcommand{\childdocdisable}{}
}
%    \end{macrocode}

% \macro{\childdocmain}
% The macro |\childdocmain| is to be called at the top of the main file
% with nothing or the main filename (without extension) as argument.
% First, it breaks loops.
% If the argument is not empty and does not match |\childdocname|
% (which is set by the first inclusion of |childdoc.def|),
% |\ifchilddoc| is set to true, |\includeonly| is applied to the child file
% and |\jobname| is set to the main file
% (for proper handling of |.aux| files):
%    \begin{macrocode}
\newcommand{\childdocmain}[1]
{
  \childdocdisable\childdocmain{}
  \if?#1?\else
    \begingroup
      \def\childdoctmp{#1}
      \ifx\childdoctmp\childdocname
        \def\childdoctmp{}
      \else
        \def\childdoctmp
        {
          \childdoctrue
          \includeonly{\childdocname}
          \def\childdocjob{#1}
          \def\jobname{#1}
        }
      \fi
      \expandafter
    \endgroup
    \childdoctmp
  \fi
}
%    \end{macrocode}

% \macro{\childdocof}
% The command |\childdocof| redirects
% compilation to the main file |#1|.
%    \begin{macrocode}
\newcommand{\childdocof}[1]
{
  \childdocdisable
  \childdoctrue
  \includeonly{\childdocname}
  \def\jobname{#1}
  \def\childdocjob{#1}
  \input{#1}
}
%    \end{macrocode}

% \macro{\childdocby}
% The command |\childdocby| ....
%    \begin{macrocode}
\newcommand{\childdocby}[2][]
{
  \childdocdisable
  \childdoctrue
  \childdocmanualtrue
  \if?#1?\else
    \def\jobname{#2}
  \fi
  \def\childdocjob{#2}
  \input{#2}
  \endinput
}
%    \end{macrocode}

% \macro{\childdocforward}
% The command |\childdocforward| redirects
% compilation to the main file or
% (if the optional argument is given) a child file.
% Parameters are set as if the main file
% or a child file starting with |\childdocof| was compiled.
% Then compilation is handed over to the main file:
%    \begin{macrocode}
\newcommand{\childdocforward}[2][]
{
  \begingroup
    \if?#1?
      \def\childdoctmp
      {
        \def\childdocname{#2}
        \def\childdocjob{#2}
        \def\jobname{#2}
        \input{#2}
        \endinput
      }
    \else
      \def\childdoctmp
      {
        \childdocdisable
        \def\childdocname{#2}
        \childdoctrue
        \includeonly{#2}
        \def\childdocjob{#1}
        \def\jobname{#1}
        \input{#1}
        \endinput
      }
    \fi
    \expandafter
  \endgroup
  \childdoctmp
}
%    \end{macrocode}

% \macro{\childdocforwardprefix}
% The command |\childdocforwardprefix| redirects
% compilation to the main or a child file by means of a pattern.
% The prefix |#1| in the current filename is replaced by |#2|
% and the suffix of the current filename is kept
% (it is assumed that the filename does not contain the substring `|~~~|'
% which is used as a delimiter).
% Compilation is handed over to the new file by |\childdocforward|:
%    \begin{macrocode}
\newcommand{\childdocforwardprefix}[3][]
{
  \begingroup
    \def\childdocextract #2##1~~~{\def\childdoctmp{\childdocforward[#1]{#3##1}}}
    \expandafter\childdocextract\childdocname~~~
    \expandafter
  \endgroup
  \childdoctmp
}
%    \end{macrocode}

% \macro{\childdoc}
% The deprecated macro |\childdoc| is a legacy version of |\childdocmain|:
%    \begin{macrocode}
\newcommand{\childdoc}{\childdocmain}
%    \end{macrocode}

% \macro{\childdocredirect}
% The deprecated macro |\childdocredirect| is a legacy version
% of |\childdocforward| and |\childdocforwardprefix|:
%    \begin{macrocode}
\newcommand{\childdocredirect}[2][]
{
  \begingroup
    \if?#1?
      \def\childdoctmp{\childdocforward{#2}}
    \else
      \def\childdoctmp{\childdocforwardprefix{#1}{#2}}
    \fi
    \expandafter
  \endgroup
  \childdoctmp
}
%    \end{macrocode}

%\iffalse
%</package>
%\fi
%
\endinput

\childdocby{cdocsamp}
%    \end{macrocode}

%\iffalse
%</samplepart3|samplepart4>
%\fi
%
%\iffalse
%<*samplepart3>
%\fi
% Some text for part 3:
%    \begin{macrocode}
some text in part three
%    \end{macrocode}

%\iffalse
%</samplepart3>
%\fi
% Some text for part 4:
%\iffalse
%<*samplepart4>
%\fi
%    \begin{macrocode}
more text in part four
%    \end{macrocode}

%\iffalse
%</samplepart4>
%\fi
%
% %%%%%%%%%%%%%%%%%%%%%%%%%%%%%%%%%%%%%%
% \paragraph{Forwarding for a Complete Draft.}
%
% The following forwarding file |cdocsdrf.tex|
% compiles the main document in draft mode:
%\iffalse
%<*sampledraft>
%\fi
%    \begin{macrocode}
\def\version{draft}
% \iffalse
%
% childdoc.dtx Copyright (C) 2017-2018 Niklas Beisert
%
% This work may be distributed and/or modified under the
% conditions of the LaTeX Project Public License, either version 1.3
% of this license or (at your option) any later version.
% The latest version of this license is in
%   http://www.latex-project.org/lppl.txt
% and version 1.3 or later is part of all distributions of LaTeX
% version 2005/12/01 or later.
%
% This work has the LPPL maintenance status `maintained'.
%
% The Current Maintainer of this work is Niklas Beisert.
%
% This work consists of the files childdoc.dtx and childdoc.ins
% and the derived files childdoc.def and cdocsamp.tex with
% cdocsch1.tex, cdocsch2.tex, cdocsdrf.tex, cdocsfn1.tex, cdocsfn2.tex.
%
%<package>\ifdefined\childdocmain\endinput\fi
%<package>\ProvidesFile{childdoc.def}[2018/12/30 v2.0 child document driver]
%<samplemain>\ProvidesFile{cdocsamp.tex}[2018/12/30 v2.0 sample for childdoc]
%<*driver>
%\ProvidesFile{childdoc.drv}[2018/12/30 v2.0 childdoc reference manual file]
\PassOptionsToClass{10pt,a4paper}{article}
\documentclass{ltxdoc}

\usepackage[margin=35mm]{geometry}
\usepackage{hyperref}
\usepackage{hyperxmp}
\usepackage[usenames]{color}

\hypersetup{colorlinks=true}
\hypersetup{pdfstartview=FitH}
\hypersetup{pdfpagemode=UseNone}
\hypersetup{pdfsource={}}
\hypersetup{pdflang={en-UK}}
\hypersetup{pdfcopyright={Copyright 2017-2018 Niklas Beisert.
  This work may be distributed and/or modified under the
  conditions of the LaTeX Project Public License, either version 1.3
  of this license or (at your option) any later version.}}
\hypersetup{pdflicenseurl={http://www.latex-project.org/lppl.txt}}
\hypersetup{pdfcontactaddress={ETH Zurich, ITP, HIT K,
  Wolfgang-Pauli-Strasse 27}}
\hypersetup{pdfcontactpostcode={8093}}
\hypersetup{pdfcontactcity={Zurich}}
\hypersetup{pdfcontactcountry={Switzerland}}
\hypersetup{pdfcontactemail={nbeisert@itp.phys.ethz.ch}}
\hypersetup{pdfcontacturl={http://people.phys.ethz.ch/\xmptilde nbeisert/}}

\newcommand{\secref}[1]{\hyperref[#1]{section \ref*{#1}}}

\parskip1ex
\parindent0pt
\let\olditemize\itemize
\def\itemize{\olditemize\parskip0pt}

\begin{document}

\title{The \textsf{childdoc} Package}
\hypersetup{pdftitle={The childdoc Package}}
\author{Niklas Beisert\\[2ex]
  Institut f\"ur Theoretische Physik\\
  Eidgen\"ossische Technische Hochschule Z\"urich\\
  Wolfgang-Pauli-Strasse 27, 8093 Z\"urich, Switzerland\\[1ex]
  \href{mailto:nbeisert@itp.phys.ethz.ch}
  {\texttt{nbeisert@itp.phys.ethz.ch}}}
\hypersetup{pdfauthor={Niklas Beisert}}
\hypersetup{pdfsubject={Manual for the LaTeX2e Package childdoc}}
\date{30 December 2018, \textsf{v2.0}}
\maketitle

\begin{abstract}\noindent
\textsf{childdoc} is a \LaTeXe{} package
that enables the direct compilation
of document sections included by |\include|
to individual files.
\end{abstract}

\begingroup
\parskip0ex
\tableofcontents
\endgroup

%%%%%%%%%%%%%%%%%%%%%%%%%%%%%%%%%%%%%%%%%%%%%%%%%%%%%%%%%%%%%%%%%%%%%%%%%%%%%%%%
%%%%%%%%%%%%%%%%%%%%%%%%%%%%%%%%%%%%%%%%%%%%%%%%%%%%%%%%%%%%%%%%%%%%%%%%%%%%%%%%
\section{Introduction}

\LaTeX{} provides a mechanism to structure a large document (such as a book)
into a main file and several child files (containing the chapters)
using the |\include| command.
This mechanism is beneficial for documents
which span hundreds of pages in order to
make the source file(s) more manageable.
Moreover, compilation can be restricted to
selected child files by means of the |\includeonly| command.
The latter feature can be used to reduce the compilation time while editing
(this was significantly more useful in the earlier days of \LaTeX{})
or to generate a smaller document which is easier to navigate.
Another application of |\includeonly| is to generate
documents consisting of selected parts of the complete document.

However, there are a few drawbacks of the plain |\include| mechanism:
\begin{itemize}
\item
The child files cannot be compiled on their own,
they can only be compiled via the main file.
A naive editing environment
(such as a text editor with an option
to have the current file processed by \LaTeX)
may require one to switch to the main file before compiling;
attempting to compile the child file produces errors.
\item
The main file must be modified (each time)
to adjust the |\includeonly| command
to the present needs. This easily leaves the main file in a messy state.
\item
The generated document will always carry the filename
of the main document. This is inconvenient if
several child files are to be compiled and
to be kept for distribution.
\end{itemize}

The present package provides a simple interface
to make child files individually compilable by \LaTeX{}.
Compiling a child file then has the same effect as compiling
the main file with an |\includeonly| command
to select the appropriate child.
Moreover the generated document will carry the name of the child
rather than the main file.
This resolves all three above issues.

This feature is meant to make the editing of books,
thesis documents and lecture notes somewhat more convenient.
However, the package can also be used efficiently for
composing a series of documents (such as exercise sheets)
which are typically distributed individually.
It then assists the author in generating the individual documents
(potentially in different versions)
as well as a document containing the collected series.
Another application is in developing style files
or other kinds of included material
where compilation of the style file could redirect
to a sample or test file.

%%%%%%%%%%%%%%%%%%%%%%%%%%%%%%%%%%%%%%%%%%%%%%%%%%%%%%%%%%%%%%%%%%%%%%%%%%%%%%%%
%%%%%%%%%%%%%%%%%%%%%%%%%%%%%%%%%%%%%%%%%%%%%%%%%%%%%%%%%%%%%%%%%%%%%%%%%%%%%%%%
\section{Usage}

First of all, the package \textsf{childdoc} is \emph{not} a standard
\LaTeXe{} |.sty| style file! Therefore it needs to be invoked in
a non-standard way.

%%%%%%%%%%%%%%%%%%%%%%%%%%%%%%%%%%%%%%%%%%%%%%%%%%%%%%%%%%%%%%%%%%%%%%%%%%%%%%%%
\subsection{Included Files}
\label{sec:include}

%%%%%%%%%%%%%%%%%%%%%%%%%%%%%%%%%%%%%%%%
\DescribeMacro{\childdocmain}
To use the package, add the commands
\begin{center}
\begin{tabular}{l}
|\input{childdoc.def}|\\
|\childdocmain{}|\\
\end{tabular}
\end{center}
at the very top of the main \LaTeX{} file,
in particular \emph{before} the |\documentclass| statement!
The argument of |\childdocmain| should be left empty
(but it must be present).

%%%%%%%%%%%%%%%%%%%%%%%%%%%%%%%%%%%%%%%%
\DescribeMacro{\childdocof}
Furthermore, add the commands
\begin{center}
\begin{tabular}{l}
|\input{childdoc.def}|\\
|\childdocof{|\textit{main}|}|\\
\end{tabular}
\end{center}
at the top of every child file \textit{child}
which is included by |\include{|\textit{child}|}|
from within the main file
(or at least for those files to be compiled individually).
The argument \textit{main} must be the filename of the main file.

There are a couple of
considerations in setting up the main and child documents:

%%%%%%%%%%%%%%%%%%%%%%%%%%%%%%%%%%%%%%%%
\paragraph{Restrictions.}

Please note the following restrictions:
\begin{itemize}
\item
|\childdocmain| must be called with one argument \textit{main}
to ensure compatibility with earlier version of the package.
It must either be empty (|\childdocmain{}|)
or precisely match the filename of the main file in which it is specified.
See \secref{sec:detection} for further information.
\item
The filename \textit{main} must be specified without the |.tex| extension.
\item
The filename \textit{main} is case sensitive
(even in case-insensitive file systems)
due to internal string comparison.
\item
The argument \textit{main} should be fully expanded, it cannot be a macro.
\item
Subdirectories and special characters should be avoided in filenames.
\item
The command |\childdocmain{|\textit{main}|}| must be followed by a whitespace.
It should not be followed immediately by another command
or by a comment mark `|%|'.
This is because the \TeX{} parser reads the token immediately following
the argument of |\childdocmain| and puts it
at the beginning of every child section;
however, a white\-space is ignored.
\end{itemize}

%%%%%%%%%%%%%%%%%%%%%%%%%%%%%%%%%%%%%%%%
\paragraph{Content of Main File.}

It is advisable to place all content in the child files included by |\include|.
Any output contained in the main file will appear in all child documents
unless suppressed manually;
it cannot be suppressed automatically by the |\includeonly| directive
and thus should normally be avoided.
A method to include some content in the main file
by means of conditional processing is described in \secref{sec:conditional}.

%%%%%%%%%%%%%%%%%%%%%%%%%%%%%%%%%%%%%%%%
\paragraph{Page Numbering.}

When only a part of the document is compiled,
the appropriate numbering of pages
(as well as other status parameters)
is determined from the |.aux| files.
The latter contain information from previous passes.
However this information needs to propagate through
all intermediate child documents.
Therefore the page numbering in child documents may well
be inconsistent until the complete document is compiled at least once.

A useful (if unconventional) way to always ensure a consistent
page numbering is to restart the numbering in each child document
and denote the pages by `\textit{child}|.|\textit{page}'
where \textit{child} represents the chapter/section number of the child file.
This can be achieved by the command
|\numberwithin{page}{|\textit{child}|}|
of the \textsf{amsmath} package
where \textit{child} can be |chapter| or |section|
depending on the chosen structuring.
Alternatively, one can modify the macro |\thepage| appropriately
and reset the counter |page| at the start of each child file.

%%%%%%%%%%%%%%%%%%%%%%%%%%%%%%%%%%%%%%%%%%%%%%%%%%%%%%%%%%%%%%%%%%%%%%%%%%%%%%%%
\subsection{Conditional Processing}
\label{sec:conditional}

The package provides a mechanism to compile different versions
of a document. To customise the versions further some conditional processing
can come in handy to distinguish which version is being compiled.
The package provides two macros to describe the compilation context:

%%%%%%%%%%%%%%%%%%%%%%%%%%%%%%%%%%%%%%%%
\DescribeMacro{\ifchilddoc}
The conditional |\ifchilddoc| distinguishes between the compilation of
child documents and the main document:
%
\begin{center}
|\ifchilddoc |\textit{child-code}| |[|\||else |\textit{main-code}]| \||fi|
\end{center}

%%%%%%%%%%%%%%%%%%%%%%%%%%%%%%%%%%%%%%%%
\DescribeMacro{\childdocname}
\DescribeMacro{\childdocjob}
The macro |\childdocname| contains the filename (without extension)
of the main or child file being processed.
Note that |\childdocjob| will always contain the name of the main file.

%%%%%%%%%%%%%%%%%%%%%%%%%%%%%%%%%%%%%%%%
\paragraph{Title Page.}

Conditional processing can be used to include a title or banner page
in the main document when proper precautions are taken.
Importantly, the code in the main file should ensure that the page counter
(as well as other status parameters which are stored in the |.aux| files)
takes the same value after the conditional processing.
Otherwise the page numbers may take divergent values
depending on which part is compiled.

For example, a title page could be declared by:
%
\begin{center}
\begin{tabular}{l}
|\ifchilddoc\||else|\\
|\addtocounter{page}{-1}|\\
\textit{code for title page}\\
|\newpage|\\
|\||fi|
\end{tabular}
\end{center}
%
A banner page for the child documents can be generated by:
%
\begin{center}
\begin{tabular}{l}
|\ifchilddoc|\\
|\addtocounter{page}{-1}|\\
\textit{code for banner page}\\
|\newpage|\\
|\||fi|
\end{tabular}
\end{center}
%
Here one could write a message such as:
\begin{center}
|This is the part \childdocname{} of \childdocjob{}.|
\end{center}

%%%%%%%%%%%%%%%%%%%%%%%%%%%%%%%%%%%%%%%%%%%%%%%%%%%%%%%%%%%%%%%%%%%%%%%%%%%%%%%%
\subsection{Flags}
\label{sec:flags}

The package makes it easy to generate different versions
of the main or child documents.
To this end compilation flags can be defined
and assigned different default values.
They will be particularly useful in conjunction
with the forwarding mechanism described in \secref{sec:forward}.

For example, it may be useful to have a flag |\version|
which can be set to |draft| or |final|.
The document source will contain some conditional code
depending on the value of |\version|.
Suppose further, the flag should default to |final| for the main file
and to |draft| for child files
which is a natural assignment for editing the document.
This is achieved by placing the following code
in the preamble of the main document
(below the |\childdocmain| directive):
%
\begin{center}
\begin{tabular}{l}
|\ifchilddoc|\\
|\providecommand{\version}{draft}|\\
|\||else|\\
|\providecommand{\version}{final}|\\
|\||fi|
\end{tabular}
\end{center}
%
The definition by |\providecommand| makes sure
that previous definitions are not overwritten.
Further statements |\providecommand{\version}{...}|
can thus be added before the above code to override it.

For the main file, one might add a line
(between |\childdocmain| and the above block)
%
\begin{center}
|%\ifchilddoc\||else\providecommand{\version}{draft}\||fi|
\end{center}
%
which can be uncommented to produce a draft version.
Likewise one can add a line to the very top of a child file
(above the |\childdocof{|\textit{main}|}| directive)
%
\begin{center}
|%\providecommand{\version}{final}|
\end{center}
%
which can be uncommented to produce the final version of this child document.

%%%%%%%%%%%%%%%%%%%%%%%%%%%%%%%%%%%%%%%%%%%%%%%%%%%%%%%%%%%%%%%%%%%%%%%%%%%%%%%%
\subsection{Forwarding}
\label{sec:forward}

Different versions of the main or child documents
using compilation flags as described in \secref{sec:flags}
can be (permanently) stored in different files
for convenient compilation, viewing and distribution.
To this end, the package defines a command
to pass on compilation to a different file:

%%%%%%%%%%%%%%%%%%%%%%%%%%%%%%%%%%%%%%%%
\DescribeMacro{\childdocforward}
The command |\childdocforward| redirects processing to
another source file:
%
\begin{center}
\begin{tabular}{l}
|\input{childdoc.def}|\\
|\childdocforward[|\textit{main}|]{|\textit{dest}|}|\\
\end{tabular}
\end{center}
%
The argument \textit{dest} is the destination file
(without extension).
It should be the main file or one of the child files.
Note that further \textsf{childdoc} directives
such as |\childdocof| and |\childdocforward|
in the indicated file will be processed in this form.
The optional argument \textit{main}
passes on directly to the main file \textit{main}
while pretending to compile the child \textit{dest}.
This form behaves as if \textit{dest}
issues |\childdocof{|\textit{main}|}| right away,
and no further \textsf{childdoc} directives will be processed.

%%%%%%%%%%%%%%%%%%%%%%%%%%%%%%%%%%%%%%%%
\DescribeMacro{\...prefix}
In the alternative form |\childdocforwardprefix|,
%
\begin{center}
\begin{tabular}{l}
|\input{childdoc.def}|\\
|\childdocforwardprefix[|\textit{main}|]{|\textit{prefix}|}{|\textit{dest}|}|
\end{tabular}
\end{center}
%
the destination file is determined by a pattern
depending on the current file:
To make this work, the current file must be called
`{\textit{prefix}\hspace{0.2em}\textit{suffix}}'
with \textit{prefix} matching precisely the argument.
Processing is then passed on to the file
`{\textit{dest}\hspace{0.2em}\textit{suffix}}'.
Surely, the same effect is achieved by
directly specifying the
argument `{\textit{dest}\hspace{0.2em}\textit{suffix}}'
in the first form.
However, that requires to set up a different file
for each child. With the alternative form of the command
all these files can have exactly the same content
which simplifies setting them up and maintaining them.

For example, the following file |draft.tex|
with a compilation flag |\version| as described in \secref{sec:flags}
compiles the main document as a draft:
%
\begin{center}
\begin{tabular}{l}
|\def\version{draft}|\\
|\input{childdoc.def}|\\
|\childdocforward{|\textit{main}|}|
\end{tabular}
\end{center}
%
Likewise, the following files |final|\textit{nn}|.tex|
compile the final version of the child document
|child|\textit{nn}|.tex|:
%
\begin{center}
\begin{tabular}{l}
|\def\version{final}|\\
|\input{childdoc.def}|\\
|\childdocforwardprefix{final}{child}|
\end{tabular}
\end{center}
%

Note that when several versions of a main file and/or of each child file
are to be generated, it may be convenient to set up a |Makefile| or
shell script to automatise the process.

%%%%%%%%%%%%%%%%%%%%%%%%%%%%%%%%%%%%%%%%%%%%%%%%%%%%%%%%%%%%%%%%%%%%%%%%%%%%%%%%
\subsection{Command Line Processing}
\label{sec:commandline}

The effect of redirection files can also be achieved by invoking
the \LaTeX{} compiler with a more elaborate command line.
Most conveniently this should be done as part
of a shell script or a |Makefile|.

When using \textsf{childdoc} in the main file, the following
command lines effectively perform a redirection
(note that depending on the shell being used,
backslashes may have to be doubled: `|\|' $\to$ `|\\|'):
%
\begin{center}
|... -jobname "|\textit{target}|" |\\|"|[\textit{flags}]%
|\input{childdoc.def}\childdocforward[|\textit{main}|]{|\textit{dest}|}"|
\end{center}
%
Here \textit{target} is the name of the output file,
\textit{main} is the name of the main file
and \textit{dest} is the name of the main or child file to be processed
(all filenames without extensions).
The optional argument \textit{main} can be omitted
if \textit{main} matches \textit{dest}.
Optionally, compilation \textit{flags} can be defined via |\def| commands.
This command line makes the \TeX{} engine believe
it is compiling the file \textit{target}
whose content is specified as the latter parameter.
The provided code then forwards the processing to
\textit{main} or \textit{dest} as described in \secref{sec:forward}.

%%%%%%%%%%%%%%%%%%%%%%%%%%%%%%%%%%%%%%%%%%%%%%%%%%%%%%%%%%%%%%%%%%%%%%%%%%%%%%%%
\subsection{Include by Input}
\label{sec:input}

Including child documents by |\include| has some restrictions by design.
Most notably, the content of a child document always occupies
its own set of pages; pages cannot be shared between child documents.
Usually, this behaviour makes perfect sense
because each child document contain an essential part of the document.
However, in some situations it may be desirable to compose
a document from a collection of parts
without having mandatory page breaks between then.
For this case, the package
provides a mechanism to include parts
by |\input| which can also be processed individually.
However, by construction this mechanism
requires manual handling of the content to be output.

%%%%%%%%%%%%%%%%%%%%%%%%%%%%%%%%%%%%%%%%
\DescribeMacro{\ifchilddocmanual}
The main file should be prepared as usual, see \secref{sec:include}.
However, the document body must make a distinction
between processing of an individual part and of the main document, e.g.:
%
\begin{center}
\begin{tabular}{l}
|\ifchilddocmanual|\\
|\input{\childdocname}|\\
|\||else|\\
\textit{document body with }|\input{|\textit{part}|}|\\
|\||fi|
\end{tabular}
\end{center}
%
The conditional |\ifchilddocmanual| is true whenever
a part to be included by |\input| is being compiled,
and the name of the part is stored in |\childdocname|.

%%%%%%%%%%%%%%%%%%%%%%%%%%%%%%%%%%%%%%%%
\DescribeMacro{\childdocby}
Each part to be included by |\input| should start with:
%
\begin{center}
\begin{tabular}{l}
|\input{childdoc.def}|\\
|\childdocby{|\textit{main}|}|\\
\end{tabular}
\end{center}
%
The directive |\childdocby| is similar to |\childdocof|
described in \secref{sec:include},
but the subsequent selection of content must be done manually.
To that end, both |\ifchilddoc| and |\ifchilddocmanual|
will be true upon processing of a part,
and the name of the part is stored in |\childdocname|.
Note that |\jobname| will be set to the filename of the current part
so that each part receives an individual |.aux| file
that does not interfere with the |.aux| file(s) of the main document.
This behaviour can be altered by the alternative form
|\childdocby[*]{|\textit{main}|}| (with a non-empty optional argument)
which uses the |.aux| file of the main document
by setting |\jobname| to \textit{main}.

%%%%%%%%%%%%%%%%%%%%%%%%%%%%%%%%%%%%%%%%%%%%%%%%%%%%%%%%%%%%%%%%%%%%%%%%%%%%%%%%
\subsection{Driver Development}
\label{sec:driver}

The \textsf{childdoc} mechanism can also be use for the development
of definition files such as \LaTeX{} styles or classes.
This case differs from the above setup with multiple parts
included by |\include| in that no |\includeonly| should be invoked.
This can be achieved by starting the include file
(before |\ProvidesPackage|) with:
%
\begin{center}
\begin{tabular}{l}
|\input{childdoc.def}|\\
|\childdocforward{|\textit{main}|}|\\
\end{tabular}
\end{center}
%
or alternatively with:
%
\begin{center}
\begin{tabular}{l}
|\input{childdoc.def}|\\
|\childdocby{|\textit{main}|}|\\
\end{tabular}
\end{center}
%
Both forms have slightly different effects as described above.
The main file is prepared as usual, see \secref{sec:include}.

%%%%%%%%%%%%%%%%%%%%%%%%%%%%%%%%%%%%%%%%%%%%%%%%%%%%%%%%%%%%%%%%%%%%%%%%%%%%%%%%
\subsection{Legacy Detection}
\label{sec:detection}

The directive |\childdocmain| in the main file can detect
whether the complete document or merely a child is to be compiled
even without using the directive |\childdocof|.
This method is deprecated because it is less robust
and there is no compelling reason to use it;
it is merely provided for backward compatibility
and it may be removed in future versions.

If the detection mechanism is to be used,
it is mandatory to correctly specify
the filename of the main file as the argument of |\childdocmain|:
%
\begin{center}
\begin{tabular}{l}
|\input{childdoc.def}|\\
|\childdocmain{|\textit{main}|}|\\
\end{tabular}
\end{center}
%
If |\jobname| does not match the argument \textit{main} of |\childdocmain|,
it is assumed that |\jobname| points to the child file to be compiled.
When using |\childdocmain| with the main file specified as argument,
it suffices to start a child file
with just |\input{|\textit{main}|}|
without loading of the package and using |\childdocof|.
If instead all processing is done
with the appropriate \textsf{childdoc} directives,
the argument of \textit{main} of |\childdocmain| can be empty.

An alternative version of the command line processing described
in \secref{sec:commandline} using the detection mechanism reads:
%
\begin{center}
|... -jobname "|\textit{target}|" "|[\textit{flags}]%
[|\def\jobname{|\textit{dest}|}|]|\input{|\textit{main}|}"|
\end{center}

%%%%%%%%%%%%%%%%%%%%%%%%%%%%%%%%%%%%%%%%%%%%%%%%%%%%%%%%%%%%%%%%%%%%%%%%%%%%%%%%
\subsection{Manual Code}
\label{sec:manual}

In case one cannot be certain whether the definitions file |childdoc.def|
is installed on the target \TeX{} distribution
and one prefers not to ship it,
it is conceivable to paste a few relevant commands into the sources.

To that end, drop all statements |\input{childdoc.def}|
and perform the replacements as outlined below.
Instead of |\childdocmain{|\textit{main}|}| add the following code
to the top of the main file:
%
\begin{center}
\begin{tabular}{l}
|\||ifdefined\childdocname\endinput\||fi\newif\ifchilddoc|\\
|\edef\childdocname{\scantokens\expandafter{\jobname\noexpand}}|\\
|\def\childdocmain{|\textit{main}|}\||ifx\childdocmain\childdocname\||else|\\
|\childdoctrue\includeonly{\childdocname}\let\jobname\childdocmain\||fi|\\
\end{tabular}
\end{center}
%
Instead of |\childdocof{|\textit{main}|}| just include the main file
at the top of each child file:
%
\begin{center}
|\input{|\textit{main}|}|
\end{center}
%
A simple redirection |\childdocforward{|\textit{dest}|}| is achieved by:
%
\begin{center}
|\def\jobname{|\textit{dest}|}\input{\jobname}|
\end{center}
%
The redirection with prefix
|\childdocforwardprefix[|\textit{prefix}|]{|\textit{dest}|}|
is accomplished by:
%
\begin{center}
\begin{tabular}{l}
|{\edef\jobname{\scantokens\expandafter{\jobname\noexpand}}|\\
|\def\redirectjob |\textit{prefix}|#1~~~{\gdef\jobname{|\textit{dest}|#1}}|\\
|\expandafter\redirectjob\jobname~~~}\input{\jobname}|
\end{tabular}
\end{center}

In an alternative approach,
child documents can be compiled by a specific command line
without additional code or specific definitions:
%
\begin{center}
|... -jobname "|\textit{target}|" "|[\textit{flags}]%
|\includeonly{|\textit{dest}|}\input{|\textit{main}|}"|
\end{center}
%

%%%%%%%%%%%%%%%%%%%%%%%%%%%%%%%%%%%%%%%%%%%%%%%%%%%%%%%%%%%%%%%%%%%%%%%%%%%%%%%%
%%%%%%%%%%%%%%%%%%%%%%%%%%%%%%%%%%%%%%%%%%%%%%%%%%%%%%%%%%%%%%%%%%%%%%%%%%%%%%%%
\section{Information}

%%%%%%%%%%%%%%%%%%%%%%%%%%%%%%%%%%%%%%%%%%%%%%%%%%%%%%%%%%%%%%%%%%%%%%%%%%%%%%%%
\subsection{Copyright}

Copyright \copyright{} 2017--2018 Niklas Beisert

This work may be distributed and/or modified under the
conditions of the \LaTeX{} Project Public License, either version 1.3
of this license or (at your option) any later version.
The latest version of this license is in
  \url{http://www.latex-project.org/lppl.txt}
and version 1.3 or later is part of all distributions of \LaTeX{}
version 2005/12/01 or later.

This work has the LPPL maintenance status `maintained'.

The Current Maintainer of this work is Niklas Beisert.

This work consists of the files |README.txt|, |childdoc.ins| and |childdoc.dtx|
as well as the derived files |childdoc.def|, |cdocsamp.tex|
with |cdocsch1.tex|, |cdocsch2.tex|, |cdocspt3.tex|, |cdocspt4.tex|,
|cdocsdrf.tex|, |cdocsfn1.tex|, |cdocsfn2.tex|
as well as |childdoc.pdf|.

%%%%%%%%%%%%%%%%%%%%%%%%%%%%%%%%%%%%%%%%%%%%%%%%%%%%%%%%%%%%%%%%%%%%%%%%%%%%%%%%
\subsection{Files and Installation}

The package consists of the files:
%
\begin{center}
\begin{tabular}{ll}
    |README.txt|   & readme file \\
    |childdoc.ins| & installation file \\
    |childdoc.dtx| & source file \\
    |childdoc.def| & definition file \\
    |cdocsamp.tex| & sample main file \\
    |cdocsch1.tex| & sample include file \\
    |cdocsch2.tex| & sample include file \\
    |cdocspt3.tex| & sample part file \\
    |cdocspt4.tex| & sample part file \\
    |cdocsdrf.tex| & sample redirection file \\
    |cdocsfn1.tex| & sample redirection file \\
    |cdocsfn2.tex| & sample redirection file \\
    |childdoc.pdf| & manual
\end{tabular}
\end{center}
%
The distribution consists of the files
|README.txt|, |childdoc.ins| and |childdoc.dtx|.
%
\begin{itemize}
\item
Run (pdf)\LaTeX{} on |childdoc.dtx|
to compile the manual |childdoc.pdf| (this file).
\item
Run \LaTeX{} on |childdoc.ins| to create the definitions file |childdoc.def|
and the sample |cdocsamp.tex| with include files
|cdocsch1.tex|, |cdocsch2.tex|, |cdocspt3.tex|, |cdocspt4.tex|,
|cdocsdrf.tex|, |cdocsfn1.tex|, |cdocsfn2.tex|.
Then copy the file |childdoc.def| to an appropriate directory of your \LaTeX{}
distribution, e.g.\ \textit{texmf-root}|/tex/latex/childdoc|.
\end{itemize}

%%%%%%%%%%%%%%%%%%%%%%%%%%%%%%%%%%%%%%%%%%%%%%%%%%%%%%%%%%%%%%%%%%%%%%%%%%%%%%%%
\subsection{Related CTAN Packages}

There are several other packages which offer a similar functionality:
%
\begin{itemize}
\item
The packages
\href{http://ctan.org/pkg/docmute}{\textsf{docmute}},
\href{http://ctan.org/pkg/includex}{\textsf{includex}} and
\href{http://ctan.org/pkg/standalone}{\textsf{standalone}}
provide commands to include only the document body of
a child file thus allowing both files to be compiled individually.
\item
The packages \href{http://ctan.org/pkg/subdocs}{\textsf{subdocs}}
and \href{http://ctan.org/pkg/subfiles}{\textsf{subfiles}}
provide structures in which the main and child documents can be
encapsulated and allowing them to be compiled individually.
The inclusion mechanism is different from the conventional |\include|.
\item
The package \href{http://ctan.org/pkg/combine}{\textsf{combine}}
is an elaborate solution to combine several documents into one.
\end{itemize}
%
See also the CTAN topic \href{http://ctan.org/topic/subdocs}{\textsf{subdocs}}
for further related packages.
The present package differs from the above solutions in that
a document structure constructed with the conventional |\include| mechanism
just needs two extra commands at the top of every file
such that all constituent files can be compiled individually.

%%%%%%%%%%%%%%%%%%%%%%%%%%%%%%%%%%%%%%%%%%%%%%%%%%%%%%%%%%%%%%%%%%%%%%%%%%%%%%%%
%\subsection{Feature Suggestions}
%
%The following is a list of features which may be useful for future
%versions of this package:
%%
%\begin{itemize}
%\item
%\ldots
%\end{itemize}

%%%%%%%%%%%%%%%%%%%%%%%%%%%%%%%%%%%%%%%%%%%%%%%%%%%%%%%%%%%%%%%%%%%%%%%%%%%%%%%%
\subsection{Revision History}

%%%%%%%%%%%%%%%%%%%%%%%%%%%%%%%%%%%%%%%%
\paragraph{v2.0:} 2018/12/30

\begin{itemize}
\item
immediate forward processing
\item
added |\childdocby| mechanism
\item
manual restructured
\end{itemize}

%%%%%%%%%%%%%%%%%%%%%%%%%%%%%%%%%%%%%%%%
\paragraph{v1.6:} 2018/01/17

\begin{itemize}
\item
application for development of include files
\item
corrections to manual
\end{itemize}

%%%%%%%%%%%%%%%%%%%%%%%%%%%%%%%%%%%%%%%%
\paragraph{v1.5:} 2017/05/21

\begin{itemize}
\item
more complete structuring introduced
\item
|\childdocof| introduced
\item
|\childdoc| renamed to |\childdocmain|
\item
|\childredirect| renamed to |\childdocforward| and |\childdocforwardprefix|
and functionality expanded
\end{itemize}

%%%%%%%%%%%%%%%%%%%%%%%%%%%%%%%%%%%%%%%%
\paragraph{v1.0:} 2017/04/27

\begin{itemize}
\item
manual and install package
\item
first version published on CTAN
\end{itemize}

%%%%%%%%%%%%%%%%%%%%%%%%%%%%%%%%%%%%%%%%
\paragraph{v0.6:} 2017/04/26

\begin{itemize}
\item
redirection mechanism added
\end{itemize}

%%%%%%%%%%%%%%%%%%%%%%%%%%%%%%%%%%%%%%%%
\paragraph{v0.5:} 2017/04/26

\begin{itemize}
\item
functionality in definition file
\end{itemize}


%%%%%%%%%%%%%%%%%%%%%%%%%%%%%%%%%%%%%%%%%%%%%%%%%%%%%%%%%%%%%%%%%%%%%%%%%%%%%%%%
%%%%%%%%%%%%%%%%%%%%%%%%%%%%%%%%%%%%%%%%%%%%%%%%%%%%%%%%%%%%%%%%%%%%%%%%%%%%%%%%
%%%%%%%%%%%%%%%%%%%%%%%%%%%%%%%%%%%%%%%%%%%%%%%%%%%%%%%%%%%%%%%%%%%%%%%%%%%%%%%%
\appendix

\settowidth\MacroIndent{\rmfamily\scriptsize 000\ }

 \DocInput{childdoc.dtx}

\end{document}
%</driver>
% \fi
%
% %%%%%%%%%%%%%%%%%%%%%%%%%%%%%%%%%%%%%%%%%%%%%%%%%%%%%%%%%%%%%%%%%%%%%%%%%%%%%%
% %%%%%%%%%%%%%%%%%%%%%%%%%%%%%%%%%%%%%%%%%%%%%%%%%%%%%%%%%%%%%%%%%%%%%%%%%%%%%%
% \section{Sample}
%\iffalse
%<*samplemain>
%\fi
%
% The following presents a sample document
% with two chapters, two parts, a title page,
% a compile flag as well as three forwarding files to set the flag.
% It consists of eight |.tex| files:
% \begin{center}
% \begin{tabular}{ll}
% |cdocsamp.tex|&main file\\
% |cdocsch1.tex|&include file for chapter 1\\
% |cdocsch2.tex|&include file for chapter 2\\
% |cdocspt3.tex|&include file for part 3\\
% |cdocspt4.tex|&include file for part 4\\
% |cdocsdrf.tex|&forwarding file for main file in draft mode\\
% |cdocsfi1.tex|&forwarding file for final version of chapter 1\\
% |cdocsfi2.tex|&forwarding file for final version of chapter 2\\
% \end{tabular}
% \end{center}
% Each of the eight files can be compiled directly by the \LaTeX{} compiler.
%
% %%%%%%%%%%%%%%%%%%%%%%%%%%%%%%%%%%%%%%
% \paragraph{Main File.}
%
% The main file is called |cdocsamp.tex|.
%
% Load the \textsf{childdoc} definitions and
% declare the filename for the main document:
%    \begin{macrocode}
\input{childdoc.def}
\childdocmain{}
%    \end{macrocode}

% Optional override for |\version| flag:
%    \begin{macrocode}
%%\ifchilddoc\else\providecommand{\version}{draft}\fi
%    \end{macrocode}

% Define the default values for the |\version| flag
% (|final| for the main file and |draft| for childs):
%    \begin{macrocode}
\ifchilddoc
\providecommand{\version}{draft}
\else
\providecommand{\version}{final}
\fi
%    \end{macrocode}

% Load the standard document class:
%    \begin{macrocode}
\documentclass[12pt]{article}
%    \end{macrocode}

% Start the document body:
%    \begin{macrocode}
\begin{document}
%    \end{macrocode}

% Declare a title page.
% Print title, part of document being processed and version flag:
%    \begin{macrocode}
\addtocounter{page}{-1}
\begin{center}
{\LARGE\bfseries{}childdoc example\par}
\vspace{1cm}
\ifchilddoc
\ifchilddocmanual part\else chapter\fi:
`\childdocname' of `\childdocjob'\par
\else
main document: `\childdocjob'\par
\fi
version: \version\par
\end{center}
\newpage
%    \end{macrocode}

% Manually include selected file,
% otherwise process as usual:
%    \begin{macrocode}
\ifchilddocmanual
\section*{part `\childdocname'}
\input{\childdocname}
\else
%    \end{macrocode}

% Include the two chapters:
%    \begin{macrocode}
\include{cdocsch1}
\include{cdocsch2}
%    \end{macrocode}

% Include the two parts unless only chapters should be displayed:
%    \begin{macrocode}
\ifchilddoc\else
\section{part three}
\input{cdocspt3}
\section{part four}
\input{cdocspt4}
\fi
%    \end{macrocode}

% Process as usual until here:
%    \begin{macrocode}
\fi
%    \end{macrocode}

% End of document body:
%    \begin{macrocode}
\end{document}
%    \end{macrocode}
%\iffalse
%</samplemain>
%\fi
%
% %%%%%%%%%%%%%%%%%%%%%%%%%%%%%%%%%%%%%%
% \paragraph{Chapter Include Files.}
%
% The include files are called |cdocsch1.tex| and |cdocsch2.tex|.
%
%\iffalse
%<*samplechap1|samplechap2>
%\fi

% Optional override for |\version| flag:
%    \begin{macrocode}
%%\providecommand{\version}{final}
%    \end{macrocode}

% Include the main document:
%    \begin{macrocode}
\input{childdoc.def}
\childdocof{cdocsamp}
%    \end{macrocode}

%\iffalse
%</samplechap1|samplechap2>
%\fi
%
%\iffalse
%<*samplechap1>
%\fi
% Some text for chapter 1:
%    \begin{macrocode}
\section{one}
some text in chapter one
%    \end{macrocode}

%\iffalse
%</samplechap1>
%\fi
% Some text for chapter 2:
%\iffalse
%<*samplechap2>
%\fi
%    \begin{macrocode}
\section{two}
more text in chapter two
%    \end{macrocode}

%\iffalse
%</samplechap2>
%\fi
%
% %%%%%%%%%%%%%%%%%%%%%%%%%%%%%%%%%%%%%%
% \paragraph{Part Include Files.}
%
% The include files are called |cdocspt3.tex| and |cdocspt4.tex|.
%
%\iffalse
%<*samplepart3|samplepart4>
%\fi

% Optional override for |\version| flag:
%    \begin{macrocode}
%%\providecommand{\version}{final}
%    \end{macrocode}

% Include the main document:
%    \begin{macrocode}
\input{childdoc.def}
\childdocby{cdocsamp}
%    \end{macrocode}

%\iffalse
%</samplepart3|samplepart4>
%\fi
%
%\iffalse
%<*samplepart3>
%\fi
% Some text for part 3:
%    \begin{macrocode}
some text in part three
%    \end{macrocode}

%\iffalse
%</samplepart3>
%\fi
% Some text for part 4:
%\iffalse
%<*samplepart4>
%\fi
%    \begin{macrocode}
more text in part four
%    \end{macrocode}

%\iffalse
%</samplepart4>
%\fi
%
% %%%%%%%%%%%%%%%%%%%%%%%%%%%%%%%%%%%%%%
% \paragraph{Forwarding for a Complete Draft.}
%
% The following forwarding file |cdocsdrf.tex|
% compiles the main document in draft mode:
%\iffalse
%<*sampledraft>
%\fi
%    \begin{macrocode}
\def\version{draft}
\input{childdoc.def}
\childdocforward{cdocsamp}
%    \end{macrocode}

%\iffalse
%</sampledraft>
%\fi
%
% %%%%%%%%%%%%%%%%%%%%%%%%%%%%%%%%%%%%%%
% \paragraph{Forwarding for Final Version of the Chapters.}
%
% The following forwarding files |cdocsfn1.tex| and |cdocsfn2.tex|
% (with identical content)
% compile the final versions of the child documents
% |cdocsch1.tex| and |cdocsch2.tex|, respectively:
%\iffalse
%<*samplefinal>
%\fi
%    \begin{macrocode}
\def\version{final}
\input{childdoc.def}
\childdocforwardprefix[cdocsamp]{cdocsfn}{cdocsch}
%    \end{macrocode}

%\iffalse
%</samplefinal>
%\fi
%
% %%%%%%%%%%%%%%%%%%%%%%%%%%%%%%%%%%%%%%
% \paragraph{Command Line Processing.}
%
% The following three command lines generate the output files
% |cdocscld|, |cdocscl1| and |cdocscl2|
% which should be identical to
% |cdocsdrf|, |cdocsch1| and |cdocsfn2|, respectively:
% \begin{center}
% \begin{tabular}{l}
% |latex -jobname cdocscld \|\\
% |  "\def\version{draft}\input{childdoc.def}\childdocforward{cdocsamp}"|\\
% |latex -jobname cdocscl1 \|\\
% |  "\input{childdoc.def}\childdocforward[cdocsamp]{cdocsch1}"|\\
% |latex -jobname cdocscl2 \|\\
% |  "\def\version{final}\input{childdoc.def}\childdocforward{cdocsch2}"|
% \end{tabular}
% \end{center}
% Note that the trailing backslash on each first line
% merely continues the input to the second line
% (for convenient cut ant paste).
% Furthermore, the command |latex| can be replaced by any
% of its alternative versions such as |pdflatex|.
%
% %%%%%%%%%%%%%%%%%%%%%%%%%%%%%%%%%%%%%%%%%%%%%%%%%%%%%%%%%%%%%%%%%%%%%%%%%%%%%%
% %%%%%%%%%%%%%%%%%%%%%%%%%%%%%%%%%%%%%%%%%%%%%%%%%%%%%%%%%%%%%%%%%%%%%%%%%%%%%%
% \section{Implementation}
%\iffalse
%<*package>
%\fi
%
% This section describes the definitions file |childdoc.def|.

% The definitions cannot be loaded using |\usepackage| or |\RequirePackage|
% which has a mechanism to prevent loading a style file more than once.
% When loading the definitions by means of |\input|
% multiple instances have to be prevented manually:
%\iffalse
%This code needs to be before the `\ProvidesFile' directive
%which is defined at the beginning of this file.
%Therefore it is also placed there and commented out here.
%</package>
%<*discard>
%\fi
%    \begin{macrocode}
\ifdefined\childdocmain\endinput\fi
%    \end{macrocode}
%\iffalse
%</discard>
%<*package>
%\fi
%
% \macro{\ifchilddoc}
% \macro{\ifchilddocmanual}
% The conditional |\ifchilddoc| tells whether a
% child (true) or main (false) document is being compiled.
% The conditional |\ifchilddocmanual| tells whether
% the |\includeonly| mechanism is used (false) or
% the selection of child files must be performed manually (true).
% The definitions initialise to false:
%    \begin{macrocode}
\newif\ifchilddoc
\newif\ifchilddocmanual
%    \end{macrocode}

% \macro{\childdocname}
% \macro{\childdocjob}
% The macro |\childdocname| stores the name of the main document
% to be compiled. The macro |\childdocjob| stores the name of
% the document on which the \LaTeX{} compiler was originally invoked.
% The content of |\jobname| cannot be compared
% to filenames specified in the source due to different catcodes.
% The following code rescans |\jobname|, stores the result
% in |\childdocname| and saves a copy in |\childdocjob|:
%    \begin{macrocode}
\edef\childdocname{\scantokens\expandafter{\jobname\noexpand}}
\let\childdocjob\childdocname
%    \end{macrocode}

% \macro{\childdocdisable}
% The macro |\childdocdisable| prevents the main file
% from being processed more than once.
% At this stage, the main document command |\childdocmain|
% is assumed to be called once again where it should do nothing.
% Any subsequent call to it should prevent
% a secondary processing of the main document
% It overwrites the forwarding commands
% |\childdocof| and |\childdocforward|
% with empty macros to prevent further inclusions of the main document:
%    \begin{macrocode}
\newcommand{\childdocdisable}
{
  \renewcommand{\childdocmain}[1]{\renewcommand{\childdocmain}[1]{\endinput}}
  \renewcommand{\childdocof}[1]{}
  \renewcommand{\childdocby}[2][]{}
  \renewcommand{\childdocforward}[2][]{}
  \renewcommand{\childdocdisable}{}
}
%    \end{macrocode}

% \macro{\childdocmain}
% The macro |\childdocmain| is to be called at the top of the main file
% with nothing or the main filename (without extension) as argument.
% First, it breaks loops.
% If the argument is not empty and does not match |\childdocname|
% (which is set by the first inclusion of |childdoc.def|),
% |\ifchilddoc| is set to true, |\includeonly| is applied to the child file
% and |\jobname| is set to the main file
% (for proper handling of |.aux| files):
%    \begin{macrocode}
\newcommand{\childdocmain}[1]
{
  \childdocdisable\childdocmain{}
  \if?#1?\else
    \begingroup
      \def\childdoctmp{#1}
      \ifx\childdoctmp\childdocname
        \def\childdoctmp{}
      \else
        \def\childdoctmp
        {
          \childdoctrue
          \includeonly{\childdocname}
          \def\childdocjob{#1}
          \def\jobname{#1}
        }
      \fi
      \expandafter
    \endgroup
    \childdoctmp
  \fi
}
%    \end{macrocode}

% \macro{\childdocof}
% The command |\childdocof| redirects
% compilation to the main file |#1|.
%    \begin{macrocode}
\newcommand{\childdocof}[1]
{
  \childdocdisable
  \childdoctrue
  \includeonly{\childdocname}
  \def\jobname{#1}
  \def\childdocjob{#1}
  \input{#1}
}
%    \end{macrocode}

% \macro{\childdocby}
% The command |\childdocby| ....
%    \begin{macrocode}
\newcommand{\childdocby}[2][]
{
  \childdocdisable
  \childdoctrue
  \childdocmanualtrue
  \if?#1?\else
    \def\jobname{#2}
  \fi
  \def\childdocjob{#2}
  \input{#2}
  \endinput
}
%    \end{macrocode}

% \macro{\childdocforward}
% The command |\childdocforward| redirects
% compilation to the main file or
% (if the optional argument is given) a child file.
% Parameters are set as if the main file
% or a child file starting with |\childdocof| was compiled.
% Then compilation is handed over to the main file:
%    \begin{macrocode}
\newcommand{\childdocforward}[2][]
{
  \begingroup
    \if?#1?
      \def\childdoctmp
      {
        \def\childdocname{#2}
        \def\childdocjob{#2}
        \def\jobname{#2}
        \input{#2}
        \endinput
      }
    \else
      \def\childdoctmp
      {
        \childdocdisable
        \def\childdocname{#2}
        \childdoctrue
        \includeonly{#2}
        \def\childdocjob{#1}
        \def\jobname{#1}
        \input{#1}
        \endinput
      }
    \fi
    \expandafter
  \endgroup
  \childdoctmp
}
%    \end{macrocode}

% \macro{\childdocforwardprefix}
% The command |\childdocforwardprefix| redirects
% compilation to the main or a child file by means of a pattern.
% The prefix |#1| in the current filename is replaced by |#2|
% and the suffix of the current filename is kept
% (it is assumed that the filename does not contain the substring `|~~~|'
% which is used as a delimiter).
% Compilation is handed over to the new file by |\childdocforward|:
%    \begin{macrocode}
\newcommand{\childdocforwardprefix}[3][]
{
  \begingroup
    \def\childdocextract #2##1~~~{\def\childdoctmp{\childdocforward[#1]{#3##1}}}
    \expandafter\childdocextract\childdocname~~~
    \expandafter
  \endgroup
  \childdoctmp
}
%    \end{macrocode}

% \macro{\childdoc}
% The deprecated macro |\childdoc| is a legacy version of |\childdocmain|:
%    \begin{macrocode}
\newcommand{\childdoc}{\childdocmain}
%    \end{macrocode}

% \macro{\childdocredirect}
% The deprecated macro |\childdocredirect| is a legacy version
% of |\childdocforward| and |\childdocforwardprefix|:
%    \begin{macrocode}
\newcommand{\childdocredirect}[2][]
{
  \begingroup
    \if?#1?
      \def\childdoctmp{\childdocforward{#2}}
    \else
      \def\childdoctmp{\childdocforwardprefix{#1}{#2}}
    \fi
    \expandafter
  \endgroup
  \childdoctmp
}
%    \end{macrocode}

%\iffalse
%</package>
%\fi
%
\endinput

\childdocforward{cdocsamp}
%    \end{macrocode}

%\iffalse
%</sampledraft>
%\fi
%
% %%%%%%%%%%%%%%%%%%%%%%%%%%%%%%%%%%%%%%
% \paragraph{Forwarding for Final Version of the Chapters.}
%
% The following forwarding files |cdocsfn1.tex| and |cdocsfn2.tex|
% (with identical content)
% compile the final versions of the child documents
% |cdocsch1.tex| and |cdocsch2.tex|, respectively:
%\iffalse
%<*samplefinal>
%\fi
%    \begin{macrocode}
\def\version{final}
% \iffalse
%
% childdoc.dtx Copyright (C) 2017-2018 Niklas Beisert
%
% This work may be distributed and/or modified under the
% conditions of the LaTeX Project Public License, either version 1.3
% of this license or (at your option) any later version.
% The latest version of this license is in
%   http://www.latex-project.org/lppl.txt
% and version 1.3 or later is part of all distributions of LaTeX
% version 2005/12/01 or later.
%
% This work has the LPPL maintenance status `maintained'.
%
% The Current Maintainer of this work is Niklas Beisert.
%
% This work consists of the files childdoc.dtx and childdoc.ins
% and the derived files childdoc.def and cdocsamp.tex with
% cdocsch1.tex, cdocsch2.tex, cdocsdrf.tex, cdocsfn1.tex, cdocsfn2.tex.
%
%<package>\ifdefined\childdocmain\endinput\fi
%<package>\ProvidesFile{childdoc.def}[2018/12/30 v2.0 child document driver]
%<samplemain>\ProvidesFile{cdocsamp.tex}[2018/12/30 v2.0 sample for childdoc]
%<*driver>
%\ProvidesFile{childdoc.drv}[2018/12/30 v2.0 childdoc reference manual file]
\PassOptionsToClass{10pt,a4paper}{article}
\documentclass{ltxdoc}

\usepackage[margin=35mm]{geometry}
\usepackage{hyperref}
\usepackage{hyperxmp}
\usepackage[usenames]{color}

\hypersetup{colorlinks=true}
\hypersetup{pdfstartview=FitH}
\hypersetup{pdfpagemode=UseNone}
\hypersetup{pdfsource={}}
\hypersetup{pdflang={en-UK}}
\hypersetup{pdfcopyright={Copyright 2017-2018 Niklas Beisert.
  This work may be distributed and/or modified under the
  conditions of the LaTeX Project Public License, either version 1.3
  of this license or (at your option) any later version.}}
\hypersetup{pdflicenseurl={http://www.latex-project.org/lppl.txt}}
\hypersetup{pdfcontactaddress={ETH Zurich, ITP, HIT K,
  Wolfgang-Pauli-Strasse 27}}
\hypersetup{pdfcontactpostcode={8093}}
\hypersetup{pdfcontactcity={Zurich}}
\hypersetup{pdfcontactcountry={Switzerland}}
\hypersetup{pdfcontactemail={nbeisert@itp.phys.ethz.ch}}
\hypersetup{pdfcontacturl={http://people.phys.ethz.ch/\xmptilde nbeisert/}}

\newcommand{\secref}[1]{\hyperref[#1]{section \ref*{#1}}}

\parskip1ex
\parindent0pt
\let\olditemize\itemize
\def\itemize{\olditemize\parskip0pt}

\begin{document}

\title{The \textsf{childdoc} Package}
\hypersetup{pdftitle={The childdoc Package}}
\author{Niklas Beisert\\[2ex]
  Institut f\"ur Theoretische Physik\\
  Eidgen\"ossische Technische Hochschule Z\"urich\\
  Wolfgang-Pauli-Strasse 27, 8093 Z\"urich, Switzerland\\[1ex]
  \href{mailto:nbeisert@itp.phys.ethz.ch}
  {\texttt{nbeisert@itp.phys.ethz.ch}}}
\hypersetup{pdfauthor={Niklas Beisert}}
\hypersetup{pdfsubject={Manual for the LaTeX2e Package childdoc}}
\date{30 December 2018, \textsf{v2.0}}
\maketitle

\begin{abstract}\noindent
\textsf{childdoc} is a \LaTeXe{} package
that enables the direct compilation
of document sections included by |\include|
to individual files.
\end{abstract}

\begingroup
\parskip0ex
\tableofcontents
\endgroup

%%%%%%%%%%%%%%%%%%%%%%%%%%%%%%%%%%%%%%%%%%%%%%%%%%%%%%%%%%%%%%%%%%%%%%%%%%%%%%%%
%%%%%%%%%%%%%%%%%%%%%%%%%%%%%%%%%%%%%%%%%%%%%%%%%%%%%%%%%%%%%%%%%%%%%%%%%%%%%%%%
\section{Introduction}

\LaTeX{} provides a mechanism to structure a large document (such as a book)
into a main file and several child files (containing the chapters)
using the |\include| command.
This mechanism is beneficial for documents
which span hundreds of pages in order to
make the source file(s) more manageable.
Moreover, compilation can be restricted to
selected child files by means of the |\includeonly| command.
The latter feature can be used to reduce the compilation time while editing
(this was significantly more useful in the earlier days of \LaTeX{})
or to generate a smaller document which is easier to navigate.
Another application of |\includeonly| is to generate
documents consisting of selected parts of the complete document.

However, there are a few drawbacks of the plain |\include| mechanism:
\begin{itemize}
\item
The child files cannot be compiled on their own,
they can only be compiled via the main file.
A naive editing environment
(such as a text editor with an option
to have the current file processed by \LaTeX)
may require one to switch to the main file before compiling;
attempting to compile the child file produces errors.
\item
The main file must be modified (each time)
to adjust the |\includeonly| command
to the present needs. This easily leaves the main file in a messy state.
\item
The generated document will always carry the filename
of the main document. This is inconvenient if
several child files are to be compiled and
to be kept for distribution.
\end{itemize}

The present package provides a simple interface
to make child files individually compilable by \LaTeX{}.
Compiling a child file then has the same effect as compiling
the main file with an |\includeonly| command
to select the appropriate child.
Moreover the generated document will carry the name of the child
rather than the main file.
This resolves all three above issues.

This feature is meant to make the editing of books,
thesis documents and lecture notes somewhat more convenient.
However, the package can also be used efficiently for
composing a series of documents (such as exercise sheets)
which are typically distributed individually.
It then assists the author in generating the individual documents
(potentially in different versions)
as well as a document containing the collected series.
Another application is in developing style files
or other kinds of included material
where compilation of the style file could redirect
to a sample or test file.

%%%%%%%%%%%%%%%%%%%%%%%%%%%%%%%%%%%%%%%%%%%%%%%%%%%%%%%%%%%%%%%%%%%%%%%%%%%%%%%%
%%%%%%%%%%%%%%%%%%%%%%%%%%%%%%%%%%%%%%%%%%%%%%%%%%%%%%%%%%%%%%%%%%%%%%%%%%%%%%%%
\section{Usage}

First of all, the package \textsf{childdoc} is \emph{not} a standard
\LaTeXe{} |.sty| style file! Therefore it needs to be invoked in
a non-standard way.

%%%%%%%%%%%%%%%%%%%%%%%%%%%%%%%%%%%%%%%%%%%%%%%%%%%%%%%%%%%%%%%%%%%%%%%%%%%%%%%%
\subsection{Included Files}
\label{sec:include}

%%%%%%%%%%%%%%%%%%%%%%%%%%%%%%%%%%%%%%%%
\DescribeMacro{\childdocmain}
To use the package, add the commands
\begin{center}
\begin{tabular}{l}
|\input{childdoc.def}|\\
|\childdocmain{}|\\
\end{tabular}
\end{center}
at the very top of the main \LaTeX{} file,
in particular \emph{before} the |\documentclass| statement!
The argument of |\childdocmain| should be left empty
(but it must be present).

%%%%%%%%%%%%%%%%%%%%%%%%%%%%%%%%%%%%%%%%
\DescribeMacro{\childdocof}
Furthermore, add the commands
\begin{center}
\begin{tabular}{l}
|\input{childdoc.def}|\\
|\childdocof{|\textit{main}|}|\\
\end{tabular}
\end{center}
at the top of every child file \textit{child}
which is included by |\include{|\textit{child}|}|
from within the main file
(or at least for those files to be compiled individually).
The argument \textit{main} must be the filename of the main file.

There are a couple of
considerations in setting up the main and child documents:

%%%%%%%%%%%%%%%%%%%%%%%%%%%%%%%%%%%%%%%%
\paragraph{Restrictions.}

Please note the following restrictions:
\begin{itemize}
\item
|\childdocmain| must be called with one argument \textit{main}
to ensure compatibility with earlier version of the package.
It must either be empty (|\childdocmain{}|)
or precisely match the filename of the main file in which it is specified.
See \secref{sec:detection} for further information.
\item
The filename \textit{main} must be specified without the |.tex| extension.
\item
The filename \textit{main} is case sensitive
(even in case-insensitive file systems)
due to internal string comparison.
\item
The argument \textit{main} should be fully expanded, it cannot be a macro.
\item
Subdirectories and special characters should be avoided in filenames.
\item
The command |\childdocmain{|\textit{main}|}| must be followed by a whitespace.
It should not be followed immediately by another command
or by a comment mark `|%|'.
This is because the \TeX{} parser reads the token immediately following
the argument of |\childdocmain| and puts it
at the beginning of every child section;
however, a white\-space is ignored.
\end{itemize}

%%%%%%%%%%%%%%%%%%%%%%%%%%%%%%%%%%%%%%%%
\paragraph{Content of Main File.}

It is advisable to place all content in the child files included by |\include|.
Any output contained in the main file will appear in all child documents
unless suppressed manually;
it cannot be suppressed automatically by the |\includeonly| directive
and thus should normally be avoided.
A method to include some content in the main file
by means of conditional processing is described in \secref{sec:conditional}.

%%%%%%%%%%%%%%%%%%%%%%%%%%%%%%%%%%%%%%%%
\paragraph{Page Numbering.}

When only a part of the document is compiled,
the appropriate numbering of pages
(as well as other status parameters)
is determined from the |.aux| files.
The latter contain information from previous passes.
However this information needs to propagate through
all intermediate child documents.
Therefore the page numbering in child documents may well
be inconsistent until the complete document is compiled at least once.

A useful (if unconventional) way to always ensure a consistent
page numbering is to restart the numbering in each child document
and denote the pages by `\textit{child}|.|\textit{page}'
where \textit{child} represents the chapter/section number of the child file.
This can be achieved by the command
|\numberwithin{page}{|\textit{child}|}|
of the \textsf{amsmath} package
where \textit{child} can be |chapter| or |section|
depending on the chosen structuring.
Alternatively, one can modify the macro |\thepage| appropriately
and reset the counter |page| at the start of each child file.

%%%%%%%%%%%%%%%%%%%%%%%%%%%%%%%%%%%%%%%%%%%%%%%%%%%%%%%%%%%%%%%%%%%%%%%%%%%%%%%%
\subsection{Conditional Processing}
\label{sec:conditional}

The package provides a mechanism to compile different versions
of a document. To customise the versions further some conditional processing
can come in handy to distinguish which version is being compiled.
The package provides two macros to describe the compilation context:

%%%%%%%%%%%%%%%%%%%%%%%%%%%%%%%%%%%%%%%%
\DescribeMacro{\ifchilddoc}
The conditional |\ifchilddoc| distinguishes between the compilation of
child documents and the main document:
%
\begin{center}
|\ifchilddoc |\textit{child-code}| |[|\||else |\textit{main-code}]| \||fi|
\end{center}

%%%%%%%%%%%%%%%%%%%%%%%%%%%%%%%%%%%%%%%%
\DescribeMacro{\childdocname}
\DescribeMacro{\childdocjob}
The macro |\childdocname| contains the filename (without extension)
of the main or child file being processed.
Note that |\childdocjob| will always contain the name of the main file.

%%%%%%%%%%%%%%%%%%%%%%%%%%%%%%%%%%%%%%%%
\paragraph{Title Page.}

Conditional processing can be used to include a title or banner page
in the main document when proper precautions are taken.
Importantly, the code in the main file should ensure that the page counter
(as well as other status parameters which are stored in the |.aux| files)
takes the same value after the conditional processing.
Otherwise the page numbers may take divergent values
depending on which part is compiled.

For example, a title page could be declared by:
%
\begin{center}
\begin{tabular}{l}
|\ifchilddoc\||else|\\
|\addtocounter{page}{-1}|\\
\textit{code for title page}\\
|\newpage|\\
|\||fi|
\end{tabular}
\end{center}
%
A banner page for the child documents can be generated by:
%
\begin{center}
\begin{tabular}{l}
|\ifchilddoc|\\
|\addtocounter{page}{-1}|\\
\textit{code for banner page}\\
|\newpage|\\
|\||fi|
\end{tabular}
\end{center}
%
Here one could write a message such as:
\begin{center}
|This is the part \childdocname{} of \childdocjob{}.|
\end{center}

%%%%%%%%%%%%%%%%%%%%%%%%%%%%%%%%%%%%%%%%%%%%%%%%%%%%%%%%%%%%%%%%%%%%%%%%%%%%%%%%
\subsection{Flags}
\label{sec:flags}

The package makes it easy to generate different versions
of the main or child documents.
To this end compilation flags can be defined
and assigned different default values.
They will be particularly useful in conjunction
with the forwarding mechanism described in \secref{sec:forward}.

For example, it may be useful to have a flag |\version|
which can be set to |draft| or |final|.
The document source will contain some conditional code
depending on the value of |\version|.
Suppose further, the flag should default to |final| for the main file
and to |draft| for child files
which is a natural assignment for editing the document.
This is achieved by placing the following code
in the preamble of the main document
(below the |\childdocmain| directive):
%
\begin{center}
\begin{tabular}{l}
|\ifchilddoc|\\
|\providecommand{\version}{draft}|\\
|\||else|\\
|\providecommand{\version}{final}|\\
|\||fi|
\end{tabular}
\end{center}
%
The definition by |\providecommand| makes sure
that previous definitions are not overwritten.
Further statements |\providecommand{\version}{...}|
can thus be added before the above code to override it.

For the main file, one might add a line
(between |\childdocmain| and the above block)
%
\begin{center}
|%\ifchilddoc\||else\providecommand{\version}{draft}\||fi|
\end{center}
%
which can be uncommented to produce a draft version.
Likewise one can add a line to the very top of a child file
(above the |\childdocof{|\textit{main}|}| directive)
%
\begin{center}
|%\providecommand{\version}{final}|
\end{center}
%
which can be uncommented to produce the final version of this child document.

%%%%%%%%%%%%%%%%%%%%%%%%%%%%%%%%%%%%%%%%%%%%%%%%%%%%%%%%%%%%%%%%%%%%%%%%%%%%%%%%
\subsection{Forwarding}
\label{sec:forward}

Different versions of the main or child documents
using compilation flags as described in \secref{sec:flags}
can be (permanently) stored in different files
for convenient compilation, viewing and distribution.
To this end, the package defines a command
to pass on compilation to a different file:

%%%%%%%%%%%%%%%%%%%%%%%%%%%%%%%%%%%%%%%%
\DescribeMacro{\childdocforward}
The command |\childdocforward| redirects processing to
another source file:
%
\begin{center}
\begin{tabular}{l}
|\input{childdoc.def}|\\
|\childdocforward[|\textit{main}|]{|\textit{dest}|}|\\
\end{tabular}
\end{center}
%
The argument \textit{dest} is the destination file
(without extension).
It should be the main file or one of the child files.
Note that further \textsf{childdoc} directives
such as |\childdocof| and |\childdocforward|
in the indicated file will be processed in this form.
The optional argument \textit{main}
passes on directly to the main file \textit{main}
while pretending to compile the child \textit{dest}.
This form behaves as if \textit{dest}
issues |\childdocof{|\textit{main}|}| right away,
and no further \textsf{childdoc} directives will be processed.

%%%%%%%%%%%%%%%%%%%%%%%%%%%%%%%%%%%%%%%%
\DescribeMacro{\...prefix}
In the alternative form |\childdocforwardprefix|,
%
\begin{center}
\begin{tabular}{l}
|\input{childdoc.def}|\\
|\childdocforwardprefix[|\textit{main}|]{|\textit{prefix}|}{|\textit{dest}|}|
\end{tabular}
\end{center}
%
the destination file is determined by a pattern
depending on the current file:
To make this work, the current file must be called
`{\textit{prefix}\hspace{0.2em}\textit{suffix}}'
with \textit{prefix} matching precisely the argument.
Processing is then passed on to the file
`{\textit{dest}\hspace{0.2em}\textit{suffix}}'.
Surely, the same effect is achieved by
directly specifying the
argument `{\textit{dest}\hspace{0.2em}\textit{suffix}}'
in the first form.
However, that requires to set up a different file
for each child. With the alternative form of the command
all these files can have exactly the same content
which simplifies setting them up and maintaining them.

For example, the following file |draft.tex|
with a compilation flag |\version| as described in \secref{sec:flags}
compiles the main document as a draft:
%
\begin{center}
\begin{tabular}{l}
|\def\version{draft}|\\
|\input{childdoc.def}|\\
|\childdocforward{|\textit{main}|}|
\end{tabular}
\end{center}
%
Likewise, the following files |final|\textit{nn}|.tex|
compile the final version of the child document
|child|\textit{nn}|.tex|:
%
\begin{center}
\begin{tabular}{l}
|\def\version{final}|\\
|\input{childdoc.def}|\\
|\childdocforwardprefix{final}{child}|
\end{tabular}
\end{center}
%

Note that when several versions of a main file and/or of each child file
are to be generated, it may be convenient to set up a |Makefile| or
shell script to automatise the process.

%%%%%%%%%%%%%%%%%%%%%%%%%%%%%%%%%%%%%%%%%%%%%%%%%%%%%%%%%%%%%%%%%%%%%%%%%%%%%%%%
\subsection{Command Line Processing}
\label{sec:commandline}

The effect of redirection files can also be achieved by invoking
the \LaTeX{} compiler with a more elaborate command line.
Most conveniently this should be done as part
of a shell script or a |Makefile|.

When using \textsf{childdoc} in the main file, the following
command lines effectively perform a redirection
(note that depending on the shell being used,
backslashes may have to be doubled: `|\|' $\to$ `|\\|'):
%
\begin{center}
|... -jobname "|\textit{target}|" |\\|"|[\textit{flags}]%
|\input{childdoc.def}\childdocforward[|\textit{main}|]{|\textit{dest}|}"|
\end{center}
%
Here \textit{target} is the name of the output file,
\textit{main} is the name of the main file
and \textit{dest} is the name of the main or child file to be processed
(all filenames without extensions).
The optional argument \textit{main} can be omitted
if \textit{main} matches \textit{dest}.
Optionally, compilation \textit{flags} can be defined via |\def| commands.
This command line makes the \TeX{} engine believe
it is compiling the file \textit{target}
whose content is specified as the latter parameter.
The provided code then forwards the processing to
\textit{main} or \textit{dest} as described in \secref{sec:forward}.

%%%%%%%%%%%%%%%%%%%%%%%%%%%%%%%%%%%%%%%%%%%%%%%%%%%%%%%%%%%%%%%%%%%%%%%%%%%%%%%%
\subsection{Include by Input}
\label{sec:input}

Including child documents by |\include| has some restrictions by design.
Most notably, the content of a child document always occupies
its own set of pages; pages cannot be shared between child documents.
Usually, this behaviour makes perfect sense
because each child document contain an essential part of the document.
However, in some situations it may be desirable to compose
a document from a collection of parts
without having mandatory page breaks between then.
For this case, the package
provides a mechanism to include parts
by |\input| which can also be processed individually.
However, by construction this mechanism
requires manual handling of the content to be output.

%%%%%%%%%%%%%%%%%%%%%%%%%%%%%%%%%%%%%%%%
\DescribeMacro{\ifchilddocmanual}
The main file should be prepared as usual, see \secref{sec:include}.
However, the document body must make a distinction
between processing of an individual part and of the main document, e.g.:
%
\begin{center}
\begin{tabular}{l}
|\ifchilddocmanual|\\
|\input{\childdocname}|\\
|\||else|\\
\textit{document body with }|\input{|\textit{part}|}|\\
|\||fi|
\end{tabular}
\end{center}
%
The conditional |\ifchilddocmanual| is true whenever
a part to be included by |\input| is being compiled,
and the name of the part is stored in |\childdocname|.

%%%%%%%%%%%%%%%%%%%%%%%%%%%%%%%%%%%%%%%%
\DescribeMacro{\childdocby}
Each part to be included by |\input| should start with:
%
\begin{center}
\begin{tabular}{l}
|\input{childdoc.def}|\\
|\childdocby{|\textit{main}|}|\\
\end{tabular}
\end{center}
%
The directive |\childdocby| is similar to |\childdocof|
described in \secref{sec:include},
but the subsequent selection of content must be done manually.
To that end, both |\ifchilddoc| and |\ifchilddocmanual|
will be true upon processing of a part,
and the name of the part is stored in |\childdocname|.
Note that |\jobname| will be set to the filename of the current part
so that each part receives an individual |.aux| file
that does not interfere with the |.aux| file(s) of the main document.
This behaviour can be altered by the alternative form
|\childdocby[*]{|\textit{main}|}| (with a non-empty optional argument)
which uses the |.aux| file of the main document
by setting |\jobname| to \textit{main}.

%%%%%%%%%%%%%%%%%%%%%%%%%%%%%%%%%%%%%%%%%%%%%%%%%%%%%%%%%%%%%%%%%%%%%%%%%%%%%%%%
\subsection{Driver Development}
\label{sec:driver}

The \textsf{childdoc} mechanism can also be use for the development
of definition files such as \LaTeX{} styles or classes.
This case differs from the above setup with multiple parts
included by |\include| in that no |\includeonly| should be invoked.
This can be achieved by starting the include file
(before |\ProvidesPackage|) with:
%
\begin{center}
\begin{tabular}{l}
|\input{childdoc.def}|\\
|\childdocforward{|\textit{main}|}|\\
\end{tabular}
\end{center}
%
or alternatively with:
%
\begin{center}
\begin{tabular}{l}
|\input{childdoc.def}|\\
|\childdocby{|\textit{main}|}|\\
\end{tabular}
\end{center}
%
Both forms have slightly different effects as described above.
The main file is prepared as usual, see \secref{sec:include}.

%%%%%%%%%%%%%%%%%%%%%%%%%%%%%%%%%%%%%%%%%%%%%%%%%%%%%%%%%%%%%%%%%%%%%%%%%%%%%%%%
\subsection{Legacy Detection}
\label{sec:detection}

The directive |\childdocmain| in the main file can detect
whether the complete document or merely a child is to be compiled
even without using the directive |\childdocof|.
This method is deprecated because it is less robust
and there is no compelling reason to use it;
it is merely provided for backward compatibility
and it may be removed in future versions.

If the detection mechanism is to be used,
it is mandatory to correctly specify
the filename of the main file as the argument of |\childdocmain|:
%
\begin{center}
\begin{tabular}{l}
|\input{childdoc.def}|\\
|\childdocmain{|\textit{main}|}|\\
\end{tabular}
\end{center}
%
If |\jobname| does not match the argument \textit{main} of |\childdocmain|,
it is assumed that |\jobname| points to the child file to be compiled.
When using |\childdocmain| with the main file specified as argument,
it suffices to start a child file
with just |\input{|\textit{main}|}|
without loading of the package and using |\childdocof|.
If instead all processing is done
with the appropriate \textsf{childdoc} directives,
the argument of \textit{main} of |\childdocmain| can be empty.

An alternative version of the command line processing described
in \secref{sec:commandline} using the detection mechanism reads:
%
\begin{center}
|... -jobname "|\textit{target}|" "|[\textit{flags}]%
[|\def\jobname{|\textit{dest}|}|]|\input{|\textit{main}|}"|
\end{center}

%%%%%%%%%%%%%%%%%%%%%%%%%%%%%%%%%%%%%%%%%%%%%%%%%%%%%%%%%%%%%%%%%%%%%%%%%%%%%%%%
\subsection{Manual Code}
\label{sec:manual}

In case one cannot be certain whether the definitions file |childdoc.def|
is installed on the target \TeX{} distribution
and one prefers not to ship it,
it is conceivable to paste a few relevant commands into the sources.

To that end, drop all statements |\input{childdoc.def}|
and perform the replacements as outlined below.
Instead of |\childdocmain{|\textit{main}|}| add the following code
to the top of the main file:
%
\begin{center}
\begin{tabular}{l}
|\||ifdefined\childdocname\endinput\||fi\newif\ifchilddoc|\\
|\edef\childdocname{\scantokens\expandafter{\jobname\noexpand}}|\\
|\def\childdocmain{|\textit{main}|}\||ifx\childdocmain\childdocname\||else|\\
|\childdoctrue\includeonly{\childdocname}\let\jobname\childdocmain\||fi|\\
\end{tabular}
\end{center}
%
Instead of |\childdocof{|\textit{main}|}| just include the main file
at the top of each child file:
%
\begin{center}
|\input{|\textit{main}|}|
\end{center}
%
A simple redirection |\childdocforward{|\textit{dest}|}| is achieved by:
%
\begin{center}
|\def\jobname{|\textit{dest}|}\input{\jobname}|
\end{center}
%
The redirection with prefix
|\childdocforwardprefix[|\textit{prefix}|]{|\textit{dest}|}|
is accomplished by:
%
\begin{center}
\begin{tabular}{l}
|{\edef\jobname{\scantokens\expandafter{\jobname\noexpand}}|\\
|\def\redirectjob |\textit{prefix}|#1~~~{\gdef\jobname{|\textit{dest}|#1}}|\\
|\expandafter\redirectjob\jobname~~~}\input{\jobname}|
\end{tabular}
\end{center}

In an alternative approach,
child documents can be compiled by a specific command line
without additional code or specific definitions:
%
\begin{center}
|... -jobname "|\textit{target}|" "|[\textit{flags}]%
|\includeonly{|\textit{dest}|}\input{|\textit{main}|}"|
\end{center}
%

%%%%%%%%%%%%%%%%%%%%%%%%%%%%%%%%%%%%%%%%%%%%%%%%%%%%%%%%%%%%%%%%%%%%%%%%%%%%%%%%
%%%%%%%%%%%%%%%%%%%%%%%%%%%%%%%%%%%%%%%%%%%%%%%%%%%%%%%%%%%%%%%%%%%%%%%%%%%%%%%%
\section{Information}

%%%%%%%%%%%%%%%%%%%%%%%%%%%%%%%%%%%%%%%%%%%%%%%%%%%%%%%%%%%%%%%%%%%%%%%%%%%%%%%%
\subsection{Copyright}

Copyright \copyright{} 2017--2018 Niklas Beisert

This work may be distributed and/or modified under the
conditions of the \LaTeX{} Project Public License, either version 1.3
of this license or (at your option) any later version.
The latest version of this license is in
  \url{http://www.latex-project.org/lppl.txt}
and version 1.3 or later is part of all distributions of \LaTeX{}
version 2005/12/01 or later.

This work has the LPPL maintenance status `maintained'.

The Current Maintainer of this work is Niklas Beisert.

This work consists of the files |README.txt|, |childdoc.ins| and |childdoc.dtx|
as well as the derived files |childdoc.def|, |cdocsamp.tex|
with |cdocsch1.tex|, |cdocsch2.tex|, |cdocspt3.tex|, |cdocspt4.tex|,
|cdocsdrf.tex|, |cdocsfn1.tex|, |cdocsfn2.tex|
as well as |childdoc.pdf|.

%%%%%%%%%%%%%%%%%%%%%%%%%%%%%%%%%%%%%%%%%%%%%%%%%%%%%%%%%%%%%%%%%%%%%%%%%%%%%%%%
\subsection{Files and Installation}

The package consists of the files:
%
\begin{center}
\begin{tabular}{ll}
    |README.txt|   & readme file \\
    |childdoc.ins| & installation file \\
    |childdoc.dtx| & source file \\
    |childdoc.def| & definition file \\
    |cdocsamp.tex| & sample main file \\
    |cdocsch1.tex| & sample include file \\
    |cdocsch2.tex| & sample include file \\
    |cdocspt3.tex| & sample part file \\
    |cdocspt4.tex| & sample part file \\
    |cdocsdrf.tex| & sample redirection file \\
    |cdocsfn1.tex| & sample redirection file \\
    |cdocsfn2.tex| & sample redirection file \\
    |childdoc.pdf| & manual
\end{tabular}
\end{center}
%
The distribution consists of the files
|README.txt|, |childdoc.ins| and |childdoc.dtx|.
%
\begin{itemize}
\item
Run (pdf)\LaTeX{} on |childdoc.dtx|
to compile the manual |childdoc.pdf| (this file).
\item
Run \LaTeX{} on |childdoc.ins| to create the definitions file |childdoc.def|
and the sample |cdocsamp.tex| with include files
|cdocsch1.tex|, |cdocsch2.tex|, |cdocspt3.tex|, |cdocspt4.tex|,
|cdocsdrf.tex|, |cdocsfn1.tex|, |cdocsfn2.tex|.
Then copy the file |childdoc.def| to an appropriate directory of your \LaTeX{}
distribution, e.g.\ \textit{texmf-root}|/tex/latex/childdoc|.
\end{itemize}

%%%%%%%%%%%%%%%%%%%%%%%%%%%%%%%%%%%%%%%%%%%%%%%%%%%%%%%%%%%%%%%%%%%%%%%%%%%%%%%%
\subsection{Related CTAN Packages}

There are several other packages which offer a similar functionality:
%
\begin{itemize}
\item
The packages
\href{http://ctan.org/pkg/docmute}{\textsf{docmute}},
\href{http://ctan.org/pkg/includex}{\textsf{includex}} and
\href{http://ctan.org/pkg/standalone}{\textsf{standalone}}
provide commands to include only the document body of
a child file thus allowing both files to be compiled individually.
\item
The packages \href{http://ctan.org/pkg/subdocs}{\textsf{subdocs}}
and \href{http://ctan.org/pkg/subfiles}{\textsf{subfiles}}
provide structures in which the main and child documents can be
encapsulated and allowing them to be compiled individually.
The inclusion mechanism is different from the conventional |\include|.
\item
The package \href{http://ctan.org/pkg/combine}{\textsf{combine}}
is an elaborate solution to combine several documents into one.
\end{itemize}
%
See also the CTAN topic \href{http://ctan.org/topic/subdocs}{\textsf{subdocs}}
for further related packages.
The present package differs from the above solutions in that
a document structure constructed with the conventional |\include| mechanism
just needs two extra commands at the top of every file
such that all constituent files can be compiled individually.

%%%%%%%%%%%%%%%%%%%%%%%%%%%%%%%%%%%%%%%%%%%%%%%%%%%%%%%%%%%%%%%%%%%%%%%%%%%%%%%%
%\subsection{Feature Suggestions}
%
%The following is a list of features which may be useful for future
%versions of this package:
%%
%\begin{itemize}
%\item
%\ldots
%\end{itemize}

%%%%%%%%%%%%%%%%%%%%%%%%%%%%%%%%%%%%%%%%%%%%%%%%%%%%%%%%%%%%%%%%%%%%%%%%%%%%%%%%
\subsection{Revision History}

%%%%%%%%%%%%%%%%%%%%%%%%%%%%%%%%%%%%%%%%
\paragraph{v2.0:} 2018/12/30

\begin{itemize}
\item
immediate forward processing
\item
added |\childdocby| mechanism
\item
manual restructured
\end{itemize}

%%%%%%%%%%%%%%%%%%%%%%%%%%%%%%%%%%%%%%%%
\paragraph{v1.6:} 2018/01/17

\begin{itemize}
\item
application for development of include files
\item
corrections to manual
\end{itemize}

%%%%%%%%%%%%%%%%%%%%%%%%%%%%%%%%%%%%%%%%
\paragraph{v1.5:} 2017/05/21

\begin{itemize}
\item
more complete structuring introduced
\item
|\childdocof| introduced
\item
|\childdoc| renamed to |\childdocmain|
\item
|\childredirect| renamed to |\childdocforward| and |\childdocforwardprefix|
and functionality expanded
\end{itemize}

%%%%%%%%%%%%%%%%%%%%%%%%%%%%%%%%%%%%%%%%
\paragraph{v1.0:} 2017/04/27

\begin{itemize}
\item
manual and install package
\item
first version published on CTAN
\end{itemize}

%%%%%%%%%%%%%%%%%%%%%%%%%%%%%%%%%%%%%%%%
\paragraph{v0.6:} 2017/04/26

\begin{itemize}
\item
redirection mechanism added
\end{itemize}

%%%%%%%%%%%%%%%%%%%%%%%%%%%%%%%%%%%%%%%%
\paragraph{v0.5:} 2017/04/26

\begin{itemize}
\item
functionality in definition file
\end{itemize}


%%%%%%%%%%%%%%%%%%%%%%%%%%%%%%%%%%%%%%%%%%%%%%%%%%%%%%%%%%%%%%%%%%%%%%%%%%%%%%%%
%%%%%%%%%%%%%%%%%%%%%%%%%%%%%%%%%%%%%%%%%%%%%%%%%%%%%%%%%%%%%%%%%%%%%%%%%%%%%%%%
%%%%%%%%%%%%%%%%%%%%%%%%%%%%%%%%%%%%%%%%%%%%%%%%%%%%%%%%%%%%%%%%%%%%%%%%%%%%%%%%
\appendix

\settowidth\MacroIndent{\rmfamily\scriptsize 000\ }

 \DocInput{childdoc.dtx}

\end{document}
%</driver>
% \fi
%
% %%%%%%%%%%%%%%%%%%%%%%%%%%%%%%%%%%%%%%%%%%%%%%%%%%%%%%%%%%%%%%%%%%%%%%%%%%%%%%
% %%%%%%%%%%%%%%%%%%%%%%%%%%%%%%%%%%%%%%%%%%%%%%%%%%%%%%%%%%%%%%%%%%%%%%%%%%%%%%
% \section{Sample}
%\iffalse
%<*samplemain>
%\fi
%
% The following presents a sample document
% with two chapters, two parts, a title page,
% a compile flag as well as three forwarding files to set the flag.
% It consists of eight |.tex| files:
% \begin{center}
% \begin{tabular}{ll}
% |cdocsamp.tex|&main file\\
% |cdocsch1.tex|&include file for chapter 1\\
% |cdocsch2.tex|&include file for chapter 2\\
% |cdocspt3.tex|&include file for part 3\\
% |cdocspt4.tex|&include file for part 4\\
% |cdocsdrf.tex|&forwarding file for main file in draft mode\\
% |cdocsfi1.tex|&forwarding file for final version of chapter 1\\
% |cdocsfi2.tex|&forwarding file for final version of chapter 2\\
% \end{tabular}
% \end{center}
% Each of the eight files can be compiled directly by the \LaTeX{} compiler.
%
% %%%%%%%%%%%%%%%%%%%%%%%%%%%%%%%%%%%%%%
% \paragraph{Main File.}
%
% The main file is called |cdocsamp.tex|.
%
% Load the \textsf{childdoc} definitions and
% declare the filename for the main document:
%    \begin{macrocode}
\input{childdoc.def}
\childdocmain{}
%    \end{macrocode}

% Optional override for |\version| flag:
%    \begin{macrocode}
%%\ifchilddoc\else\providecommand{\version}{draft}\fi
%    \end{macrocode}

% Define the default values for the |\version| flag
% (|final| for the main file and |draft| for childs):
%    \begin{macrocode}
\ifchilddoc
\providecommand{\version}{draft}
\else
\providecommand{\version}{final}
\fi
%    \end{macrocode}

% Load the standard document class:
%    \begin{macrocode}
\documentclass[12pt]{article}
%    \end{macrocode}

% Start the document body:
%    \begin{macrocode}
\begin{document}
%    \end{macrocode}

% Declare a title page.
% Print title, part of document being processed and version flag:
%    \begin{macrocode}
\addtocounter{page}{-1}
\begin{center}
{\LARGE\bfseries{}childdoc example\par}
\vspace{1cm}
\ifchilddoc
\ifchilddocmanual part\else chapter\fi:
`\childdocname' of `\childdocjob'\par
\else
main document: `\childdocjob'\par
\fi
version: \version\par
\end{center}
\newpage
%    \end{macrocode}

% Manually include selected file,
% otherwise process as usual:
%    \begin{macrocode}
\ifchilddocmanual
\section*{part `\childdocname'}
\input{\childdocname}
\else
%    \end{macrocode}

% Include the two chapters:
%    \begin{macrocode}
\include{cdocsch1}
\include{cdocsch2}
%    \end{macrocode}

% Include the two parts unless only chapters should be displayed:
%    \begin{macrocode}
\ifchilddoc\else
\section{part three}
\input{cdocspt3}
\section{part four}
\input{cdocspt4}
\fi
%    \end{macrocode}

% Process as usual until here:
%    \begin{macrocode}
\fi
%    \end{macrocode}

% End of document body:
%    \begin{macrocode}
\end{document}
%    \end{macrocode}
%\iffalse
%</samplemain>
%\fi
%
% %%%%%%%%%%%%%%%%%%%%%%%%%%%%%%%%%%%%%%
% \paragraph{Chapter Include Files.}
%
% The include files are called |cdocsch1.tex| and |cdocsch2.tex|.
%
%\iffalse
%<*samplechap1|samplechap2>
%\fi

% Optional override for |\version| flag:
%    \begin{macrocode}
%%\providecommand{\version}{final}
%    \end{macrocode}

% Include the main document:
%    \begin{macrocode}
\input{childdoc.def}
\childdocof{cdocsamp}
%    \end{macrocode}

%\iffalse
%</samplechap1|samplechap2>
%\fi
%
%\iffalse
%<*samplechap1>
%\fi
% Some text for chapter 1:
%    \begin{macrocode}
\section{one}
some text in chapter one
%    \end{macrocode}

%\iffalse
%</samplechap1>
%\fi
% Some text for chapter 2:
%\iffalse
%<*samplechap2>
%\fi
%    \begin{macrocode}
\section{two}
more text in chapter two
%    \end{macrocode}

%\iffalse
%</samplechap2>
%\fi
%
% %%%%%%%%%%%%%%%%%%%%%%%%%%%%%%%%%%%%%%
% \paragraph{Part Include Files.}
%
% The include files are called |cdocspt3.tex| and |cdocspt4.tex|.
%
%\iffalse
%<*samplepart3|samplepart4>
%\fi

% Optional override for |\version| flag:
%    \begin{macrocode}
%%\providecommand{\version}{final}
%    \end{macrocode}

% Include the main document:
%    \begin{macrocode}
\input{childdoc.def}
\childdocby{cdocsamp}
%    \end{macrocode}

%\iffalse
%</samplepart3|samplepart4>
%\fi
%
%\iffalse
%<*samplepart3>
%\fi
% Some text for part 3:
%    \begin{macrocode}
some text in part three
%    \end{macrocode}

%\iffalse
%</samplepart3>
%\fi
% Some text for part 4:
%\iffalse
%<*samplepart4>
%\fi
%    \begin{macrocode}
more text in part four
%    \end{macrocode}

%\iffalse
%</samplepart4>
%\fi
%
% %%%%%%%%%%%%%%%%%%%%%%%%%%%%%%%%%%%%%%
% \paragraph{Forwarding for a Complete Draft.}
%
% The following forwarding file |cdocsdrf.tex|
% compiles the main document in draft mode:
%\iffalse
%<*sampledraft>
%\fi
%    \begin{macrocode}
\def\version{draft}
\input{childdoc.def}
\childdocforward{cdocsamp}
%    \end{macrocode}

%\iffalse
%</sampledraft>
%\fi
%
% %%%%%%%%%%%%%%%%%%%%%%%%%%%%%%%%%%%%%%
% \paragraph{Forwarding for Final Version of the Chapters.}
%
% The following forwarding files |cdocsfn1.tex| and |cdocsfn2.tex|
% (with identical content)
% compile the final versions of the child documents
% |cdocsch1.tex| and |cdocsch2.tex|, respectively:
%\iffalse
%<*samplefinal>
%\fi
%    \begin{macrocode}
\def\version{final}
\input{childdoc.def}
\childdocforwardprefix[cdocsamp]{cdocsfn}{cdocsch}
%    \end{macrocode}

%\iffalse
%</samplefinal>
%\fi
%
% %%%%%%%%%%%%%%%%%%%%%%%%%%%%%%%%%%%%%%
% \paragraph{Command Line Processing.}
%
% The following three command lines generate the output files
% |cdocscld|, |cdocscl1| and |cdocscl2|
% which should be identical to
% |cdocsdrf|, |cdocsch1| and |cdocsfn2|, respectively:
% \begin{center}
% \begin{tabular}{l}
% |latex -jobname cdocscld \|\\
% |  "\def\version{draft}\input{childdoc.def}\childdocforward{cdocsamp}"|\\
% |latex -jobname cdocscl1 \|\\
% |  "\input{childdoc.def}\childdocforward[cdocsamp]{cdocsch1}"|\\
% |latex -jobname cdocscl2 \|\\
% |  "\def\version{final}\input{childdoc.def}\childdocforward{cdocsch2}"|
% \end{tabular}
% \end{center}
% Note that the trailing backslash on each first line
% merely continues the input to the second line
% (for convenient cut ant paste).
% Furthermore, the command |latex| can be replaced by any
% of its alternative versions such as |pdflatex|.
%
% %%%%%%%%%%%%%%%%%%%%%%%%%%%%%%%%%%%%%%%%%%%%%%%%%%%%%%%%%%%%%%%%%%%%%%%%%%%%%%
% %%%%%%%%%%%%%%%%%%%%%%%%%%%%%%%%%%%%%%%%%%%%%%%%%%%%%%%%%%%%%%%%%%%%%%%%%%%%%%
% \section{Implementation}
%\iffalse
%<*package>
%\fi
%
% This section describes the definitions file |childdoc.def|.

% The definitions cannot be loaded using |\usepackage| or |\RequirePackage|
% which has a mechanism to prevent loading a style file more than once.
% When loading the definitions by means of |\input|
% multiple instances have to be prevented manually:
%\iffalse
%This code needs to be before the `\ProvidesFile' directive
%which is defined at the beginning of this file.
%Therefore it is also placed there and commented out here.
%</package>
%<*discard>
%\fi
%    \begin{macrocode}
\ifdefined\childdocmain\endinput\fi
%    \end{macrocode}
%\iffalse
%</discard>
%<*package>
%\fi
%
% \macro{\ifchilddoc}
% \macro{\ifchilddocmanual}
% The conditional |\ifchilddoc| tells whether a
% child (true) or main (false) document is being compiled.
% The conditional |\ifchilddocmanual| tells whether
% the |\includeonly| mechanism is used (false) or
% the selection of child files must be performed manually (true).
% The definitions initialise to false:
%    \begin{macrocode}
\newif\ifchilddoc
\newif\ifchilddocmanual
%    \end{macrocode}

% \macro{\childdocname}
% \macro{\childdocjob}
% The macro |\childdocname| stores the name of the main document
% to be compiled. The macro |\childdocjob| stores the name of
% the document on which the \LaTeX{} compiler was originally invoked.
% The content of |\jobname| cannot be compared
% to filenames specified in the source due to different catcodes.
% The following code rescans |\jobname|, stores the result
% in |\childdocname| and saves a copy in |\childdocjob|:
%    \begin{macrocode}
\edef\childdocname{\scantokens\expandafter{\jobname\noexpand}}
\let\childdocjob\childdocname
%    \end{macrocode}

% \macro{\childdocdisable}
% The macro |\childdocdisable| prevents the main file
% from being processed more than once.
% At this stage, the main document command |\childdocmain|
% is assumed to be called once again where it should do nothing.
% Any subsequent call to it should prevent
% a secondary processing of the main document
% It overwrites the forwarding commands
% |\childdocof| and |\childdocforward|
% with empty macros to prevent further inclusions of the main document:
%    \begin{macrocode}
\newcommand{\childdocdisable}
{
  \renewcommand{\childdocmain}[1]{\renewcommand{\childdocmain}[1]{\endinput}}
  \renewcommand{\childdocof}[1]{}
  \renewcommand{\childdocby}[2][]{}
  \renewcommand{\childdocforward}[2][]{}
  \renewcommand{\childdocdisable}{}
}
%    \end{macrocode}

% \macro{\childdocmain}
% The macro |\childdocmain| is to be called at the top of the main file
% with nothing or the main filename (without extension) as argument.
% First, it breaks loops.
% If the argument is not empty and does not match |\childdocname|
% (which is set by the first inclusion of |childdoc.def|),
% |\ifchilddoc| is set to true, |\includeonly| is applied to the child file
% and |\jobname| is set to the main file
% (for proper handling of |.aux| files):
%    \begin{macrocode}
\newcommand{\childdocmain}[1]
{
  \childdocdisable\childdocmain{}
  \if?#1?\else
    \begingroup
      \def\childdoctmp{#1}
      \ifx\childdoctmp\childdocname
        \def\childdoctmp{}
      \else
        \def\childdoctmp
        {
          \childdoctrue
          \includeonly{\childdocname}
          \def\childdocjob{#1}
          \def\jobname{#1}
        }
      \fi
      \expandafter
    \endgroup
    \childdoctmp
  \fi
}
%    \end{macrocode}

% \macro{\childdocof}
% The command |\childdocof| redirects
% compilation to the main file |#1|.
%    \begin{macrocode}
\newcommand{\childdocof}[1]
{
  \childdocdisable
  \childdoctrue
  \includeonly{\childdocname}
  \def\jobname{#1}
  \def\childdocjob{#1}
  \input{#1}
}
%    \end{macrocode}

% \macro{\childdocby}
% The command |\childdocby| ....
%    \begin{macrocode}
\newcommand{\childdocby}[2][]
{
  \childdocdisable
  \childdoctrue
  \childdocmanualtrue
  \if?#1?\else
    \def\jobname{#2}
  \fi
  \def\childdocjob{#2}
  \input{#2}
  \endinput
}
%    \end{macrocode}

% \macro{\childdocforward}
% The command |\childdocforward| redirects
% compilation to the main file or
% (if the optional argument is given) a child file.
% Parameters are set as if the main file
% or a child file starting with |\childdocof| was compiled.
% Then compilation is handed over to the main file:
%    \begin{macrocode}
\newcommand{\childdocforward}[2][]
{
  \begingroup
    \if?#1?
      \def\childdoctmp
      {
        \def\childdocname{#2}
        \def\childdocjob{#2}
        \def\jobname{#2}
        \input{#2}
        \endinput
      }
    \else
      \def\childdoctmp
      {
        \childdocdisable
        \def\childdocname{#2}
        \childdoctrue
        \includeonly{#2}
        \def\childdocjob{#1}
        \def\jobname{#1}
        \input{#1}
        \endinput
      }
    \fi
    \expandafter
  \endgroup
  \childdoctmp
}
%    \end{macrocode}

% \macro{\childdocforwardprefix}
% The command |\childdocforwardprefix| redirects
% compilation to the main or a child file by means of a pattern.
% The prefix |#1| in the current filename is replaced by |#2|
% and the suffix of the current filename is kept
% (it is assumed that the filename does not contain the substring `|~~~|'
% which is used as a delimiter).
% Compilation is handed over to the new file by |\childdocforward|:
%    \begin{macrocode}
\newcommand{\childdocforwardprefix}[3][]
{
  \begingroup
    \def\childdocextract #2##1~~~{\def\childdoctmp{\childdocforward[#1]{#3##1}}}
    \expandafter\childdocextract\childdocname~~~
    \expandafter
  \endgroup
  \childdoctmp
}
%    \end{macrocode}

% \macro{\childdoc}
% The deprecated macro |\childdoc| is a legacy version of |\childdocmain|:
%    \begin{macrocode}
\newcommand{\childdoc}{\childdocmain}
%    \end{macrocode}

% \macro{\childdocredirect}
% The deprecated macro |\childdocredirect| is a legacy version
% of |\childdocforward| and |\childdocforwardprefix|:
%    \begin{macrocode}
\newcommand{\childdocredirect}[2][]
{
  \begingroup
    \if?#1?
      \def\childdoctmp{\childdocforward{#2}}
    \else
      \def\childdoctmp{\childdocforwardprefix{#1}{#2}}
    \fi
    \expandafter
  \endgroup
  \childdoctmp
}
%    \end{macrocode}

%\iffalse
%</package>
%\fi
%
\endinput

\childdocforwardprefix[cdocsamp]{cdocsfn}{cdocsch}
%    \end{macrocode}

%\iffalse
%</samplefinal>
%\fi
%
% %%%%%%%%%%%%%%%%%%%%%%%%%%%%%%%%%%%%%%
% \paragraph{Command Line Processing.}
%
% The following three command lines generate the output files
% |cdocscld|, |cdocscl1| and |cdocscl2|
% which should be identical to
% |cdocsdrf|, |cdocsch1| and |cdocsfn2|, respectively:
% \begin{center}
% \begin{tabular}{l}
% |latex -jobname cdocscld \|\\
% |  "\def\version{draft}% \iffalse
%
% childdoc.dtx Copyright (C) 2017-2018 Niklas Beisert
%
% This work may be distributed and/or modified under the
% conditions of the LaTeX Project Public License, either version 1.3
% of this license or (at your option) any later version.
% The latest version of this license is in
%   http://www.latex-project.org/lppl.txt
% and version 1.3 or later is part of all distributions of LaTeX
% version 2005/12/01 or later.
%
% This work has the LPPL maintenance status `maintained'.
%
% The Current Maintainer of this work is Niklas Beisert.
%
% This work consists of the files childdoc.dtx and childdoc.ins
% and the derived files childdoc.def and cdocsamp.tex with
% cdocsch1.tex, cdocsch2.tex, cdocsdrf.tex, cdocsfn1.tex, cdocsfn2.tex.
%
%<package>\ifdefined\childdocmain\endinput\fi
%<package>\ProvidesFile{childdoc.def}[2018/12/30 v2.0 child document driver]
%<samplemain>\ProvidesFile{cdocsamp.tex}[2018/12/30 v2.0 sample for childdoc]
%<*driver>
%\ProvidesFile{childdoc.drv}[2018/12/30 v2.0 childdoc reference manual file]
\PassOptionsToClass{10pt,a4paper}{article}
\documentclass{ltxdoc}

\usepackage[margin=35mm]{geometry}
\usepackage{hyperref}
\usepackage{hyperxmp}
\usepackage[usenames]{color}

\hypersetup{colorlinks=true}
\hypersetup{pdfstartview=FitH}
\hypersetup{pdfpagemode=UseNone}
\hypersetup{pdfsource={}}
\hypersetup{pdflang={en-UK}}
\hypersetup{pdfcopyright={Copyright 2017-2018 Niklas Beisert.
  This work may be distributed and/or modified under the
  conditions of the LaTeX Project Public License, either version 1.3
  of this license or (at your option) any later version.}}
\hypersetup{pdflicenseurl={http://www.latex-project.org/lppl.txt}}
\hypersetup{pdfcontactaddress={ETH Zurich, ITP, HIT K,
  Wolfgang-Pauli-Strasse 27}}
\hypersetup{pdfcontactpostcode={8093}}
\hypersetup{pdfcontactcity={Zurich}}
\hypersetup{pdfcontactcountry={Switzerland}}
\hypersetup{pdfcontactemail={nbeisert@itp.phys.ethz.ch}}
\hypersetup{pdfcontacturl={http://people.phys.ethz.ch/\xmptilde nbeisert/}}

\newcommand{\secref}[1]{\hyperref[#1]{section \ref*{#1}}}

\parskip1ex
\parindent0pt
\let\olditemize\itemize
\def\itemize{\olditemize\parskip0pt}

\begin{document}

\title{The \textsf{childdoc} Package}
\hypersetup{pdftitle={The childdoc Package}}
\author{Niklas Beisert\\[2ex]
  Institut f\"ur Theoretische Physik\\
  Eidgen\"ossische Technische Hochschule Z\"urich\\
  Wolfgang-Pauli-Strasse 27, 8093 Z\"urich, Switzerland\\[1ex]
  \href{mailto:nbeisert@itp.phys.ethz.ch}
  {\texttt{nbeisert@itp.phys.ethz.ch}}}
\hypersetup{pdfauthor={Niklas Beisert}}
\hypersetup{pdfsubject={Manual for the LaTeX2e Package childdoc}}
\date{30 December 2018, \textsf{v2.0}}
\maketitle

\begin{abstract}\noindent
\textsf{childdoc} is a \LaTeXe{} package
that enables the direct compilation
of document sections included by |\include|
to individual files.
\end{abstract}

\begingroup
\parskip0ex
\tableofcontents
\endgroup

%%%%%%%%%%%%%%%%%%%%%%%%%%%%%%%%%%%%%%%%%%%%%%%%%%%%%%%%%%%%%%%%%%%%%%%%%%%%%%%%
%%%%%%%%%%%%%%%%%%%%%%%%%%%%%%%%%%%%%%%%%%%%%%%%%%%%%%%%%%%%%%%%%%%%%%%%%%%%%%%%
\section{Introduction}

\LaTeX{} provides a mechanism to structure a large document (such as a book)
into a main file and several child files (containing the chapters)
using the |\include| command.
This mechanism is beneficial for documents
which span hundreds of pages in order to
make the source file(s) more manageable.
Moreover, compilation can be restricted to
selected child files by means of the |\includeonly| command.
The latter feature can be used to reduce the compilation time while editing
(this was significantly more useful in the earlier days of \LaTeX{})
or to generate a smaller document which is easier to navigate.
Another application of |\includeonly| is to generate
documents consisting of selected parts of the complete document.

However, there are a few drawbacks of the plain |\include| mechanism:
\begin{itemize}
\item
The child files cannot be compiled on their own,
they can only be compiled via the main file.
A naive editing environment
(such as a text editor with an option
to have the current file processed by \LaTeX)
may require one to switch to the main file before compiling;
attempting to compile the child file produces errors.
\item
The main file must be modified (each time)
to adjust the |\includeonly| command
to the present needs. This easily leaves the main file in a messy state.
\item
The generated document will always carry the filename
of the main document. This is inconvenient if
several child files are to be compiled and
to be kept for distribution.
\end{itemize}

The present package provides a simple interface
to make child files individually compilable by \LaTeX{}.
Compiling a child file then has the same effect as compiling
the main file with an |\includeonly| command
to select the appropriate child.
Moreover the generated document will carry the name of the child
rather than the main file.
This resolves all three above issues.

This feature is meant to make the editing of books,
thesis documents and lecture notes somewhat more convenient.
However, the package can also be used efficiently for
composing a series of documents (such as exercise sheets)
which are typically distributed individually.
It then assists the author in generating the individual documents
(potentially in different versions)
as well as a document containing the collected series.
Another application is in developing style files
or other kinds of included material
where compilation of the style file could redirect
to a sample or test file.

%%%%%%%%%%%%%%%%%%%%%%%%%%%%%%%%%%%%%%%%%%%%%%%%%%%%%%%%%%%%%%%%%%%%%%%%%%%%%%%%
%%%%%%%%%%%%%%%%%%%%%%%%%%%%%%%%%%%%%%%%%%%%%%%%%%%%%%%%%%%%%%%%%%%%%%%%%%%%%%%%
\section{Usage}

First of all, the package \textsf{childdoc} is \emph{not} a standard
\LaTeXe{} |.sty| style file! Therefore it needs to be invoked in
a non-standard way.

%%%%%%%%%%%%%%%%%%%%%%%%%%%%%%%%%%%%%%%%%%%%%%%%%%%%%%%%%%%%%%%%%%%%%%%%%%%%%%%%
\subsection{Included Files}
\label{sec:include}

%%%%%%%%%%%%%%%%%%%%%%%%%%%%%%%%%%%%%%%%
\DescribeMacro{\childdocmain}
To use the package, add the commands
\begin{center}
\begin{tabular}{l}
|\input{childdoc.def}|\\
|\childdocmain{}|\\
\end{tabular}
\end{center}
at the very top of the main \LaTeX{} file,
in particular \emph{before} the |\documentclass| statement!
The argument of |\childdocmain| should be left empty
(but it must be present).

%%%%%%%%%%%%%%%%%%%%%%%%%%%%%%%%%%%%%%%%
\DescribeMacro{\childdocof}
Furthermore, add the commands
\begin{center}
\begin{tabular}{l}
|\input{childdoc.def}|\\
|\childdocof{|\textit{main}|}|\\
\end{tabular}
\end{center}
at the top of every child file \textit{child}
which is included by |\include{|\textit{child}|}|
from within the main file
(or at least for those files to be compiled individually).
The argument \textit{main} must be the filename of the main file.

There are a couple of
considerations in setting up the main and child documents:

%%%%%%%%%%%%%%%%%%%%%%%%%%%%%%%%%%%%%%%%
\paragraph{Restrictions.}

Please note the following restrictions:
\begin{itemize}
\item
|\childdocmain| must be called with one argument \textit{main}
to ensure compatibility with earlier version of the package.
It must either be empty (|\childdocmain{}|)
or precisely match the filename of the main file in which it is specified.
See \secref{sec:detection} for further information.
\item
The filename \textit{main} must be specified without the |.tex| extension.
\item
The filename \textit{main} is case sensitive
(even in case-insensitive file systems)
due to internal string comparison.
\item
The argument \textit{main} should be fully expanded, it cannot be a macro.
\item
Subdirectories and special characters should be avoided in filenames.
\item
The command |\childdocmain{|\textit{main}|}| must be followed by a whitespace.
It should not be followed immediately by another command
or by a comment mark `|%|'.
This is because the \TeX{} parser reads the token immediately following
the argument of |\childdocmain| and puts it
at the beginning of every child section;
however, a white\-space is ignored.
\end{itemize}

%%%%%%%%%%%%%%%%%%%%%%%%%%%%%%%%%%%%%%%%
\paragraph{Content of Main File.}

It is advisable to place all content in the child files included by |\include|.
Any output contained in the main file will appear in all child documents
unless suppressed manually;
it cannot be suppressed automatically by the |\includeonly| directive
and thus should normally be avoided.
A method to include some content in the main file
by means of conditional processing is described in \secref{sec:conditional}.

%%%%%%%%%%%%%%%%%%%%%%%%%%%%%%%%%%%%%%%%
\paragraph{Page Numbering.}

When only a part of the document is compiled,
the appropriate numbering of pages
(as well as other status parameters)
is determined from the |.aux| files.
The latter contain information from previous passes.
However this information needs to propagate through
all intermediate child documents.
Therefore the page numbering in child documents may well
be inconsistent until the complete document is compiled at least once.

A useful (if unconventional) way to always ensure a consistent
page numbering is to restart the numbering in each child document
and denote the pages by `\textit{child}|.|\textit{page}'
where \textit{child} represents the chapter/section number of the child file.
This can be achieved by the command
|\numberwithin{page}{|\textit{child}|}|
of the \textsf{amsmath} package
where \textit{child} can be |chapter| or |section|
depending on the chosen structuring.
Alternatively, one can modify the macro |\thepage| appropriately
and reset the counter |page| at the start of each child file.

%%%%%%%%%%%%%%%%%%%%%%%%%%%%%%%%%%%%%%%%%%%%%%%%%%%%%%%%%%%%%%%%%%%%%%%%%%%%%%%%
\subsection{Conditional Processing}
\label{sec:conditional}

The package provides a mechanism to compile different versions
of a document. To customise the versions further some conditional processing
can come in handy to distinguish which version is being compiled.
The package provides two macros to describe the compilation context:

%%%%%%%%%%%%%%%%%%%%%%%%%%%%%%%%%%%%%%%%
\DescribeMacro{\ifchilddoc}
The conditional |\ifchilddoc| distinguishes between the compilation of
child documents and the main document:
%
\begin{center}
|\ifchilddoc |\textit{child-code}| |[|\||else |\textit{main-code}]| \||fi|
\end{center}

%%%%%%%%%%%%%%%%%%%%%%%%%%%%%%%%%%%%%%%%
\DescribeMacro{\childdocname}
\DescribeMacro{\childdocjob}
The macro |\childdocname| contains the filename (without extension)
of the main or child file being processed.
Note that |\childdocjob| will always contain the name of the main file.

%%%%%%%%%%%%%%%%%%%%%%%%%%%%%%%%%%%%%%%%
\paragraph{Title Page.}

Conditional processing can be used to include a title or banner page
in the main document when proper precautions are taken.
Importantly, the code in the main file should ensure that the page counter
(as well as other status parameters which are stored in the |.aux| files)
takes the same value after the conditional processing.
Otherwise the page numbers may take divergent values
depending on which part is compiled.

For example, a title page could be declared by:
%
\begin{center}
\begin{tabular}{l}
|\ifchilddoc\||else|\\
|\addtocounter{page}{-1}|\\
\textit{code for title page}\\
|\newpage|\\
|\||fi|
\end{tabular}
\end{center}
%
A banner page for the child documents can be generated by:
%
\begin{center}
\begin{tabular}{l}
|\ifchilddoc|\\
|\addtocounter{page}{-1}|\\
\textit{code for banner page}\\
|\newpage|\\
|\||fi|
\end{tabular}
\end{center}
%
Here one could write a message such as:
\begin{center}
|This is the part \childdocname{} of \childdocjob{}.|
\end{center}

%%%%%%%%%%%%%%%%%%%%%%%%%%%%%%%%%%%%%%%%%%%%%%%%%%%%%%%%%%%%%%%%%%%%%%%%%%%%%%%%
\subsection{Flags}
\label{sec:flags}

The package makes it easy to generate different versions
of the main or child documents.
To this end compilation flags can be defined
and assigned different default values.
They will be particularly useful in conjunction
with the forwarding mechanism described in \secref{sec:forward}.

For example, it may be useful to have a flag |\version|
which can be set to |draft| or |final|.
The document source will contain some conditional code
depending on the value of |\version|.
Suppose further, the flag should default to |final| for the main file
and to |draft| for child files
which is a natural assignment for editing the document.
This is achieved by placing the following code
in the preamble of the main document
(below the |\childdocmain| directive):
%
\begin{center}
\begin{tabular}{l}
|\ifchilddoc|\\
|\providecommand{\version}{draft}|\\
|\||else|\\
|\providecommand{\version}{final}|\\
|\||fi|
\end{tabular}
\end{center}
%
The definition by |\providecommand| makes sure
that previous definitions are not overwritten.
Further statements |\providecommand{\version}{...}|
can thus be added before the above code to override it.

For the main file, one might add a line
(between |\childdocmain| and the above block)
%
\begin{center}
|%\ifchilddoc\||else\providecommand{\version}{draft}\||fi|
\end{center}
%
which can be uncommented to produce a draft version.
Likewise one can add a line to the very top of a child file
(above the |\childdocof{|\textit{main}|}| directive)
%
\begin{center}
|%\providecommand{\version}{final}|
\end{center}
%
which can be uncommented to produce the final version of this child document.

%%%%%%%%%%%%%%%%%%%%%%%%%%%%%%%%%%%%%%%%%%%%%%%%%%%%%%%%%%%%%%%%%%%%%%%%%%%%%%%%
\subsection{Forwarding}
\label{sec:forward}

Different versions of the main or child documents
using compilation flags as described in \secref{sec:flags}
can be (permanently) stored in different files
for convenient compilation, viewing and distribution.
To this end, the package defines a command
to pass on compilation to a different file:

%%%%%%%%%%%%%%%%%%%%%%%%%%%%%%%%%%%%%%%%
\DescribeMacro{\childdocforward}
The command |\childdocforward| redirects processing to
another source file:
%
\begin{center}
\begin{tabular}{l}
|\input{childdoc.def}|\\
|\childdocforward[|\textit{main}|]{|\textit{dest}|}|\\
\end{tabular}
\end{center}
%
The argument \textit{dest} is the destination file
(without extension).
It should be the main file or one of the child files.
Note that further \textsf{childdoc} directives
such as |\childdocof| and |\childdocforward|
in the indicated file will be processed in this form.
The optional argument \textit{main}
passes on directly to the main file \textit{main}
while pretending to compile the child \textit{dest}.
This form behaves as if \textit{dest}
issues |\childdocof{|\textit{main}|}| right away,
and no further \textsf{childdoc} directives will be processed.

%%%%%%%%%%%%%%%%%%%%%%%%%%%%%%%%%%%%%%%%
\DescribeMacro{\...prefix}
In the alternative form |\childdocforwardprefix|,
%
\begin{center}
\begin{tabular}{l}
|\input{childdoc.def}|\\
|\childdocforwardprefix[|\textit{main}|]{|\textit{prefix}|}{|\textit{dest}|}|
\end{tabular}
\end{center}
%
the destination file is determined by a pattern
depending on the current file:
To make this work, the current file must be called
`{\textit{prefix}\hspace{0.2em}\textit{suffix}}'
with \textit{prefix} matching precisely the argument.
Processing is then passed on to the file
`{\textit{dest}\hspace{0.2em}\textit{suffix}}'.
Surely, the same effect is achieved by
directly specifying the
argument `{\textit{dest}\hspace{0.2em}\textit{suffix}}'
in the first form.
However, that requires to set up a different file
for each child. With the alternative form of the command
all these files can have exactly the same content
which simplifies setting them up and maintaining them.

For example, the following file |draft.tex|
with a compilation flag |\version| as described in \secref{sec:flags}
compiles the main document as a draft:
%
\begin{center}
\begin{tabular}{l}
|\def\version{draft}|\\
|\input{childdoc.def}|\\
|\childdocforward{|\textit{main}|}|
\end{tabular}
\end{center}
%
Likewise, the following files |final|\textit{nn}|.tex|
compile the final version of the child document
|child|\textit{nn}|.tex|:
%
\begin{center}
\begin{tabular}{l}
|\def\version{final}|\\
|\input{childdoc.def}|\\
|\childdocforwardprefix{final}{child}|
\end{tabular}
\end{center}
%

Note that when several versions of a main file and/or of each child file
are to be generated, it may be convenient to set up a |Makefile| or
shell script to automatise the process.

%%%%%%%%%%%%%%%%%%%%%%%%%%%%%%%%%%%%%%%%%%%%%%%%%%%%%%%%%%%%%%%%%%%%%%%%%%%%%%%%
\subsection{Command Line Processing}
\label{sec:commandline}

The effect of redirection files can also be achieved by invoking
the \LaTeX{} compiler with a more elaborate command line.
Most conveniently this should be done as part
of a shell script or a |Makefile|.

When using \textsf{childdoc} in the main file, the following
command lines effectively perform a redirection
(note that depending on the shell being used,
backslashes may have to be doubled: `|\|' $\to$ `|\\|'):
%
\begin{center}
|... -jobname "|\textit{target}|" |\\|"|[\textit{flags}]%
|\input{childdoc.def}\childdocforward[|\textit{main}|]{|\textit{dest}|}"|
\end{center}
%
Here \textit{target} is the name of the output file,
\textit{main} is the name of the main file
and \textit{dest} is the name of the main or child file to be processed
(all filenames without extensions).
The optional argument \textit{main} can be omitted
if \textit{main} matches \textit{dest}.
Optionally, compilation \textit{flags} can be defined via |\def| commands.
This command line makes the \TeX{} engine believe
it is compiling the file \textit{target}
whose content is specified as the latter parameter.
The provided code then forwards the processing to
\textit{main} or \textit{dest} as described in \secref{sec:forward}.

%%%%%%%%%%%%%%%%%%%%%%%%%%%%%%%%%%%%%%%%%%%%%%%%%%%%%%%%%%%%%%%%%%%%%%%%%%%%%%%%
\subsection{Include by Input}
\label{sec:input}

Including child documents by |\include| has some restrictions by design.
Most notably, the content of a child document always occupies
its own set of pages; pages cannot be shared between child documents.
Usually, this behaviour makes perfect sense
because each child document contain an essential part of the document.
However, in some situations it may be desirable to compose
a document from a collection of parts
without having mandatory page breaks between then.
For this case, the package
provides a mechanism to include parts
by |\input| which can also be processed individually.
However, by construction this mechanism
requires manual handling of the content to be output.

%%%%%%%%%%%%%%%%%%%%%%%%%%%%%%%%%%%%%%%%
\DescribeMacro{\ifchilddocmanual}
The main file should be prepared as usual, see \secref{sec:include}.
However, the document body must make a distinction
between processing of an individual part and of the main document, e.g.:
%
\begin{center}
\begin{tabular}{l}
|\ifchilddocmanual|\\
|\input{\childdocname}|\\
|\||else|\\
\textit{document body with }|\input{|\textit{part}|}|\\
|\||fi|
\end{tabular}
\end{center}
%
The conditional |\ifchilddocmanual| is true whenever
a part to be included by |\input| is being compiled,
and the name of the part is stored in |\childdocname|.

%%%%%%%%%%%%%%%%%%%%%%%%%%%%%%%%%%%%%%%%
\DescribeMacro{\childdocby}
Each part to be included by |\input| should start with:
%
\begin{center}
\begin{tabular}{l}
|\input{childdoc.def}|\\
|\childdocby{|\textit{main}|}|\\
\end{tabular}
\end{center}
%
The directive |\childdocby| is similar to |\childdocof|
described in \secref{sec:include},
but the subsequent selection of content must be done manually.
To that end, both |\ifchilddoc| and |\ifchilddocmanual|
will be true upon processing of a part,
and the name of the part is stored in |\childdocname|.
Note that |\jobname| will be set to the filename of the current part
so that each part receives an individual |.aux| file
that does not interfere with the |.aux| file(s) of the main document.
This behaviour can be altered by the alternative form
|\childdocby[*]{|\textit{main}|}| (with a non-empty optional argument)
which uses the |.aux| file of the main document
by setting |\jobname| to \textit{main}.

%%%%%%%%%%%%%%%%%%%%%%%%%%%%%%%%%%%%%%%%%%%%%%%%%%%%%%%%%%%%%%%%%%%%%%%%%%%%%%%%
\subsection{Driver Development}
\label{sec:driver}

The \textsf{childdoc} mechanism can also be use for the development
of definition files such as \LaTeX{} styles or classes.
This case differs from the above setup with multiple parts
included by |\include| in that no |\includeonly| should be invoked.
This can be achieved by starting the include file
(before |\ProvidesPackage|) with:
%
\begin{center}
\begin{tabular}{l}
|\input{childdoc.def}|\\
|\childdocforward{|\textit{main}|}|\\
\end{tabular}
\end{center}
%
or alternatively with:
%
\begin{center}
\begin{tabular}{l}
|\input{childdoc.def}|\\
|\childdocby{|\textit{main}|}|\\
\end{tabular}
\end{center}
%
Both forms have slightly different effects as described above.
The main file is prepared as usual, see \secref{sec:include}.

%%%%%%%%%%%%%%%%%%%%%%%%%%%%%%%%%%%%%%%%%%%%%%%%%%%%%%%%%%%%%%%%%%%%%%%%%%%%%%%%
\subsection{Legacy Detection}
\label{sec:detection}

The directive |\childdocmain| in the main file can detect
whether the complete document or merely a child is to be compiled
even without using the directive |\childdocof|.
This method is deprecated because it is less robust
and there is no compelling reason to use it;
it is merely provided for backward compatibility
and it may be removed in future versions.

If the detection mechanism is to be used,
it is mandatory to correctly specify
the filename of the main file as the argument of |\childdocmain|:
%
\begin{center}
\begin{tabular}{l}
|\input{childdoc.def}|\\
|\childdocmain{|\textit{main}|}|\\
\end{tabular}
\end{center}
%
If |\jobname| does not match the argument \textit{main} of |\childdocmain|,
it is assumed that |\jobname| points to the child file to be compiled.
When using |\childdocmain| with the main file specified as argument,
it suffices to start a child file
with just |\input{|\textit{main}|}|
without loading of the package and using |\childdocof|.
If instead all processing is done
with the appropriate \textsf{childdoc} directives,
the argument of \textit{main} of |\childdocmain| can be empty.

An alternative version of the command line processing described
in \secref{sec:commandline} using the detection mechanism reads:
%
\begin{center}
|... -jobname "|\textit{target}|" "|[\textit{flags}]%
[|\def\jobname{|\textit{dest}|}|]|\input{|\textit{main}|}"|
\end{center}

%%%%%%%%%%%%%%%%%%%%%%%%%%%%%%%%%%%%%%%%%%%%%%%%%%%%%%%%%%%%%%%%%%%%%%%%%%%%%%%%
\subsection{Manual Code}
\label{sec:manual}

In case one cannot be certain whether the definitions file |childdoc.def|
is installed on the target \TeX{} distribution
and one prefers not to ship it,
it is conceivable to paste a few relevant commands into the sources.

To that end, drop all statements |\input{childdoc.def}|
and perform the replacements as outlined below.
Instead of |\childdocmain{|\textit{main}|}| add the following code
to the top of the main file:
%
\begin{center}
\begin{tabular}{l}
|\||ifdefined\childdocname\endinput\||fi\newif\ifchilddoc|\\
|\edef\childdocname{\scantokens\expandafter{\jobname\noexpand}}|\\
|\def\childdocmain{|\textit{main}|}\||ifx\childdocmain\childdocname\||else|\\
|\childdoctrue\includeonly{\childdocname}\let\jobname\childdocmain\||fi|\\
\end{tabular}
\end{center}
%
Instead of |\childdocof{|\textit{main}|}| just include the main file
at the top of each child file:
%
\begin{center}
|\input{|\textit{main}|}|
\end{center}
%
A simple redirection |\childdocforward{|\textit{dest}|}| is achieved by:
%
\begin{center}
|\def\jobname{|\textit{dest}|}\input{\jobname}|
\end{center}
%
The redirection with prefix
|\childdocforwardprefix[|\textit{prefix}|]{|\textit{dest}|}|
is accomplished by:
%
\begin{center}
\begin{tabular}{l}
|{\edef\jobname{\scantokens\expandafter{\jobname\noexpand}}|\\
|\def\redirectjob |\textit{prefix}|#1~~~{\gdef\jobname{|\textit{dest}|#1}}|\\
|\expandafter\redirectjob\jobname~~~}\input{\jobname}|
\end{tabular}
\end{center}

In an alternative approach,
child documents can be compiled by a specific command line
without additional code or specific definitions:
%
\begin{center}
|... -jobname "|\textit{target}|" "|[\textit{flags}]%
|\includeonly{|\textit{dest}|}\input{|\textit{main}|}"|
\end{center}
%

%%%%%%%%%%%%%%%%%%%%%%%%%%%%%%%%%%%%%%%%%%%%%%%%%%%%%%%%%%%%%%%%%%%%%%%%%%%%%%%%
%%%%%%%%%%%%%%%%%%%%%%%%%%%%%%%%%%%%%%%%%%%%%%%%%%%%%%%%%%%%%%%%%%%%%%%%%%%%%%%%
\section{Information}

%%%%%%%%%%%%%%%%%%%%%%%%%%%%%%%%%%%%%%%%%%%%%%%%%%%%%%%%%%%%%%%%%%%%%%%%%%%%%%%%
\subsection{Copyright}

Copyright \copyright{} 2017--2018 Niklas Beisert

This work may be distributed and/or modified under the
conditions of the \LaTeX{} Project Public License, either version 1.3
of this license or (at your option) any later version.
The latest version of this license is in
  \url{http://www.latex-project.org/lppl.txt}
and version 1.3 or later is part of all distributions of \LaTeX{}
version 2005/12/01 or later.

This work has the LPPL maintenance status `maintained'.

The Current Maintainer of this work is Niklas Beisert.

This work consists of the files |README.txt|, |childdoc.ins| and |childdoc.dtx|
as well as the derived files |childdoc.def|, |cdocsamp.tex|
with |cdocsch1.tex|, |cdocsch2.tex|, |cdocspt3.tex|, |cdocspt4.tex|,
|cdocsdrf.tex|, |cdocsfn1.tex|, |cdocsfn2.tex|
as well as |childdoc.pdf|.

%%%%%%%%%%%%%%%%%%%%%%%%%%%%%%%%%%%%%%%%%%%%%%%%%%%%%%%%%%%%%%%%%%%%%%%%%%%%%%%%
\subsection{Files and Installation}

The package consists of the files:
%
\begin{center}
\begin{tabular}{ll}
    |README.txt|   & readme file \\
    |childdoc.ins| & installation file \\
    |childdoc.dtx| & source file \\
    |childdoc.def| & definition file \\
    |cdocsamp.tex| & sample main file \\
    |cdocsch1.tex| & sample include file \\
    |cdocsch2.tex| & sample include file \\
    |cdocspt3.tex| & sample part file \\
    |cdocspt4.tex| & sample part file \\
    |cdocsdrf.tex| & sample redirection file \\
    |cdocsfn1.tex| & sample redirection file \\
    |cdocsfn2.tex| & sample redirection file \\
    |childdoc.pdf| & manual
\end{tabular}
\end{center}
%
The distribution consists of the files
|README.txt|, |childdoc.ins| and |childdoc.dtx|.
%
\begin{itemize}
\item
Run (pdf)\LaTeX{} on |childdoc.dtx|
to compile the manual |childdoc.pdf| (this file).
\item
Run \LaTeX{} on |childdoc.ins| to create the definitions file |childdoc.def|
and the sample |cdocsamp.tex| with include files
|cdocsch1.tex|, |cdocsch2.tex|, |cdocspt3.tex|, |cdocspt4.tex|,
|cdocsdrf.tex|, |cdocsfn1.tex|, |cdocsfn2.tex|.
Then copy the file |childdoc.def| to an appropriate directory of your \LaTeX{}
distribution, e.g.\ \textit{texmf-root}|/tex/latex/childdoc|.
\end{itemize}

%%%%%%%%%%%%%%%%%%%%%%%%%%%%%%%%%%%%%%%%%%%%%%%%%%%%%%%%%%%%%%%%%%%%%%%%%%%%%%%%
\subsection{Related CTAN Packages}

There are several other packages which offer a similar functionality:
%
\begin{itemize}
\item
The packages
\href{http://ctan.org/pkg/docmute}{\textsf{docmute}},
\href{http://ctan.org/pkg/includex}{\textsf{includex}} and
\href{http://ctan.org/pkg/standalone}{\textsf{standalone}}
provide commands to include only the document body of
a child file thus allowing both files to be compiled individually.
\item
The packages \href{http://ctan.org/pkg/subdocs}{\textsf{subdocs}}
and \href{http://ctan.org/pkg/subfiles}{\textsf{subfiles}}
provide structures in which the main and child documents can be
encapsulated and allowing them to be compiled individually.
The inclusion mechanism is different from the conventional |\include|.
\item
The package \href{http://ctan.org/pkg/combine}{\textsf{combine}}
is an elaborate solution to combine several documents into one.
\end{itemize}
%
See also the CTAN topic \href{http://ctan.org/topic/subdocs}{\textsf{subdocs}}
for further related packages.
The present package differs from the above solutions in that
a document structure constructed with the conventional |\include| mechanism
just needs two extra commands at the top of every file
such that all constituent files can be compiled individually.

%%%%%%%%%%%%%%%%%%%%%%%%%%%%%%%%%%%%%%%%%%%%%%%%%%%%%%%%%%%%%%%%%%%%%%%%%%%%%%%%
%\subsection{Feature Suggestions}
%
%The following is a list of features which may be useful for future
%versions of this package:
%%
%\begin{itemize}
%\item
%\ldots
%\end{itemize}

%%%%%%%%%%%%%%%%%%%%%%%%%%%%%%%%%%%%%%%%%%%%%%%%%%%%%%%%%%%%%%%%%%%%%%%%%%%%%%%%
\subsection{Revision History}

%%%%%%%%%%%%%%%%%%%%%%%%%%%%%%%%%%%%%%%%
\paragraph{v2.0:} 2018/12/30

\begin{itemize}
\item
immediate forward processing
\item
added |\childdocby| mechanism
\item
manual restructured
\end{itemize}

%%%%%%%%%%%%%%%%%%%%%%%%%%%%%%%%%%%%%%%%
\paragraph{v1.6:} 2018/01/17

\begin{itemize}
\item
application for development of include files
\item
corrections to manual
\end{itemize}

%%%%%%%%%%%%%%%%%%%%%%%%%%%%%%%%%%%%%%%%
\paragraph{v1.5:} 2017/05/21

\begin{itemize}
\item
more complete structuring introduced
\item
|\childdocof| introduced
\item
|\childdoc| renamed to |\childdocmain|
\item
|\childredirect| renamed to |\childdocforward| and |\childdocforwardprefix|
and functionality expanded
\end{itemize}

%%%%%%%%%%%%%%%%%%%%%%%%%%%%%%%%%%%%%%%%
\paragraph{v1.0:} 2017/04/27

\begin{itemize}
\item
manual and install package
\item
first version published on CTAN
\end{itemize}

%%%%%%%%%%%%%%%%%%%%%%%%%%%%%%%%%%%%%%%%
\paragraph{v0.6:} 2017/04/26

\begin{itemize}
\item
redirection mechanism added
\end{itemize}

%%%%%%%%%%%%%%%%%%%%%%%%%%%%%%%%%%%%%%%%
\paragraph{v0.5:} 2017/04/26

\begin{itemize}
\item
functionality in definition file
\end{itemize}


%%%%%%%%%%%%%%%%%%%%%%%%%%%%%%%%%%%%%%%%%%%%%%%%%%%%%%%%%%%%%%%%%%%%%%%%%%%%%%%%
%%%%%%%%%%%%%%%%%%%%%%%%%%%%%%%%%%%%%%%%%%%%%%%%%%%%%%%%%%%%%%%%%%%%%%%%%%%%%%%%
%%%%%%%%%%%%%%%%%%%%%%%%%%%%%%%%%%%%%%%%%%%%%%%%%%%%%%%%%%%%%%%%%%%%%%%%%%%%%%%%
\appendix

\settowidth\MacroIndent{\rmfamily\scriptsize 000\ }

 \DocInput{childdoc.dtx}

\end{document}
%</driver>
% \fi
%
% %%%%%%%%%%%%%%%%%%%%%%%%%%%%%%%%%%%%%%%%%%%%%%%%%%%%%%%%%%%%%%%%%%%%%%%%%%%%%%
% %%%%%%%%%%%%%%%%%%%%%%%%%%%%%%%%%%%%%%%%%%%%%%%%%%%%%%%%%%%%%%%%%%%%%%%%%%%%%%
% \section{Sample}
%\iffalse
%<*samplemain>
%\fi
%
% The following presents a sample document
% with two chapters, two parts, a title page,
% a compile flag as well as three forwarding files to set the flag.
% It consists of eight |.tex| files:
% \begin{center}
% \begin{tabular}{ll}
% |cdocsamp.tex|&main file\\
% |cdocsch1.tex|&include file for chapter 1\\
% |cdocsch2.tex|&include file for chapter 2\\
% |cdocspt3.tex|&include file for part 3\\
% |cdocspt4.tex|&include file for part 4\\
% |cdocsdrf.tex|&forwarding file for main file in draft mode\\
% |cdocsfi1.tex|&forwarding file for final version of chapter 1\\
% |cdocsfi2.tex|&forwarding file for final version of chapter 2\\
% \end{tabular}
% \end{center}
% Each of the eight files can be compiled directly by the \LaTeX{} compiler.
%
% %%%%%%%%%%%%%%%%%%%%%%%%%%%%%%%%%%%%%%
% \paragraph{Main File.}
%
% The main file is called |cdocsamp.tex|.
%
% Load the \textsf{childdoc} definitions and
% declare the filename for the main document:
%    \begin{macrocode}
\input{childdoc.def}
\childdocmain{}
%    \end{macrocode}

% Optional override for |\version| flag:
%    \begin{macrocode}
%%\ifchilddoc\else\providecommand{\version}{draft}\fi
%    \end{macrocode}

% Define the default values for the |\version| flag
% (|final| for the main file and |draft| for childs):
%    \begin{macrocode}
\ifchilddoc
\providecommand{\version}{draft}
\else
\providecommand{\version}{final}
\fi
%    \end{macrocode}

% Load the standard document class:
%    \begin{macrocode}
\documentclass[12pt]{article}
%    \end{macrocode}

% Start the document body:
%    \begin{macrocode}
\begin{document}
%    \end{macrocode}

% Declare a title page.
% Print title, part of document being processed and version flag:
%    \begin{macrocode}
\addtocounter{page}{-1}
\begin{center}
{\LARGE\bfseries{}childdoc example\par}
\vspace{1cm}
\ifchilddoc
\ifchilddocmanual part\else chapter\fi:
`\childdocname' of `\childdocjob'\par
\else
main document: `\childdocjob'\par
\fi
version: \version\par
\end{center}
\newpage
%    \end{macrocode}

% Manually include selected file,
% otherwise process as usual:
%    \begin{macrocode}
\ifchilddocmanual
\section*{part `\childdocname'}
\input{\childdocname}
\else
%    \end{macrocode}

% Include the two chapters:
%    \begin{macrocode}
\include{cdocsch1}
\include{cdocsch2}
%    \end{macrocode}

% Include the two parts unless only chapters should be displayed:
%    \begin{macrocode}
\ifchilddoc\else
\section{part three}
\input{cdocspt3}
\section{part four}
\input{cdocspt4}
\fi
%    \end{macrocode}

% Process as usual until here:
%    \begin{macrocode}
\fi
%    \end{macrocode}

% End of document body:
%    \begin{macrocode}
\end{document}
%    \end{macrocode}
%\iffalse
%</samplemain>
%\fi
%
% %%%%%%%%%%%%%%%%%%%%%%%%%%%%%%%%%%%%%%
% \paragraph{Chapter Include Files.}
%
% The include files are called |cdocsch1.tex| and |cdocsch2.tex|.
%
%\iffalse
%<*samplechap1|samplechap2>
%\fi

% Optional override for |\version| flag:
%    \begin{macrocode}
%%\providecommand{\version}{final}
%    \end{macrocode}

% Include the main document:
%    \begin{macrocode}
\input{childdoc.def}
\childdocof{cdocsamp}
%    \end{macrocode}

%\iffalse
%</samplechap1|samplechap2>
%\fi
%
%\iffalse
%<*samplechap1>
%\fi
% Some text for chapter 1:
%    \begin{macrocode}
\section{one}
some text in chapter one
%    \end{macrocode}

%\iffalse
%</samplechap1>
%\fi
% Some text for chapter 2:
%\iffalse
%<*samplechap2>
%\fi
%    \begin{macrocode}
\section{two}
more text in chapter two
%    \end{macrocode}

%\iffalse
%</samplechap2>
%\fi
%
% %%%%%%%%%%%%%%%%%%%%%%%%%%%%%%%%%%%%%%
% \paragraph{Part Include Files.}
%
% The include files are called |cdocspt3.tex| and |cdocspt4.tex|.
%
%\iffalse
%<*samplepart3|samplepart4>
%\fi

% Optional override for |\version| flag:
%    \begin{macrocode}
%%\providecommand{\version}{final}
%    \end{macrocode}

% Include the main document:
%    \begin{macrocode}
\input{childdoc.def}
\childdocby{cdocsamp}
%    \end{macrocode}

%\iffalse
%</samplepart3|samplepart4>
%\fi
%
%\iffalse
%<*samplepart3>
%\fi
% Some text for part 3:
%    \begin{macrocode}
some text in part three
%    \end{macrocode}

%\iffalse
%</samplepart3>
%\fi
% Some text for part 4:
%\iffalse
%<*samplepart4>
%\fi
%    \begin{macrocode}
more text in part four
%    \end{macrocode}

%\iffalse
%</samplepart4>
%\fi
%
% %%%%%%%%%%%%%%%%%%%%%%%%%%%%%%%%%%%%%%
% \paragraph{Forwarding for a Complete Draft.}
%
% The following forwarding file |cdocsdrf.tex|
% compiles the main document in draft mode:
%\iffalse
%<*sampledraft>
%\fi
%    \begin{macrocode}
\def\version{draft}
\input{childdoc.def}
\childdocforward{cdocsamp}
%    \end{macrocode}

%\iffalse
%</sampledraft>
%\fi
%
% %%%%%%%%%%%%%%%%%%%%%%%%%%%%%%%%%%%%%%
% \paragraph{Forwarding for Final Version of the Chapters.}
%
% The following forwarding files |cdocsfn1.tex| and |cdocsfn2.tex|
% (with identical content)
% compile the final versions of the child documents
% |cdocsch1.tex| and |cdocsch2.tex|, respectively:
%\iffalse
%<*samplefinal>
%\fi
%    \begin{macrocode}
\def\version{final}
\input{childdoc.def}
\childdocforwardprefix[cdocsamp]{cdocsfn}{cdocsch}
%    \end{macrocode}

%\iffalse
%</samplefinal>
%\fi
%
% %%%%%%%%%%%%%%%%%%%%%%%%%%%%%%%%%%%%%%
% \paragraph{Command Line Processing.}
%
% The following three command lines generate the output files
% |cdocscld|, |cdocscl1| and |cdocscl2|
% which should be identical to
% |cdocsdrf|, |cdocsch1| and |cdocsfn2|, respectively:
% \begin{center}
% \begin{tabular}{l}
% |latex -jobname cdocscld \|\\
% |  "\def\version{draft}\input{childdoc.def}\childdocforward{cdocsamp}"|\\
% |latex -jobname cdocscl1 \|\\
% |  "\input{childdoc.def}\childdocforward[cdocsamp]{cdocsch1}"|\\
% |latex -jobname cdocscl2 \|\\
% |  "\def\version{final}\input{childdoc.def}\childdocforward{cdocsch2}"|
% \end{tabular}
% \end{center}
% Note that the trailing backslash on each first line
% merely continues the input to the second line
% (for convenient cut ant paste).
% Furthermore, the command |latex| can be replaced by any
% of its alternative versions such as |pdflatex|.
%
% %%%%%%%%%%%%%%%%%%%%%%%%%%%%%%%%%%%%%%%%%%%%%%%%%%%%%%%%%%%%%%%%%%%%%%%%%%%%%%
% %%%%%%%%%%%%%%%%%%%%%%%%%%%%%%%%%%%%%%%%%%%%%%%%%%%%%%%%%%%%%%%%%%%%%%%%%%%%%%
% \section{Implementation}
%\iffalse
%<*package>
%\fi
%
% This section describes the definitions file |childdoc.def|.

% The definitions cannot be loaded using |\usepackage| or |\RequirePackage|
% which has a mechanism to prevent loading a style file more than once.
% When loading the definitions by means of |\input|
% multiple instances have to be prevented manually:
%\iffalse
%This code needs to be before the `\ProvidesFile' directive
%which is defined at the beginning of this file.
%Therefore it is also placed there and commented out here.
%</package>
%<*discard>
%\fi
%    \begin{macrocode}
\ifdefined\childdocmain\endinput\fi
%    \end{macrocode}
%\iffalse
%</discard>
%<*package>
%\fi
%
% \macro{\ifchilddoc}
% \macro{\ifchilddocmanual}
% The conditional |\ifchilddoc| tells whether a
% child (true) or main (false) document is being compiled.
% The conditional |\ifchilddocmanual| tells whether
% the |\includeonly| mechanism is used (false) or
% the selection of child files must be performed manually (true).
% The definitions initialise to false:
%    \begin{macrocode}
\newif\ifchilddoc
\newif\ifchilddocmanual
%    \end{macrocode}

% \macro{\childdocname}
% \macro{\childdocjob}
% The macro |\childdocname| stores the name of the main document
% to be compiled. The macro |\childdocjob| stores the name of
% the document on which the \LaTeX{} compiler was originally invoked.
% The content of |\jobname| cannot be compared
% to filenames specified in the source due to different catcodes.
% The following code rescans |\jobname|, stores the result
% in |\childdocname| and saves a copy in |\childdocjob|:
%    \begin{macrocode}
\edef\childdocname{\scantokens\expandafter{\jobname\noexpand}}
\let\childdocjob\childdocname
%    \end{macrocode}

% \macro{\childdocdisable}
% The macro |\childdocdisable| prevents the main file
% from being processed more than once.
% At this stage, the main document command |\childdocmain|
% is assumed to be called once again where it should do nothing.
% Any subsequent call to it should prevent
% a secondary processing of the main document
% It overwrites the forwarding commands
% |\childdocof| and |\childdocforward|
% with empty macros to prevent further inclusions of the main document:
%    \begin{macrocode}
\newcommand{\childdocdisable}
{
  \renewcommand{\childdocmain}[1]{\renewcommand{\childdocmain}[1]{\endinput}}
  \renewcommand{\childdocof}[1]{}
  \renewcommand{\childdocby}[2][]{}
  \renewcommand{\childdocforward}[2][]{}
  \renewcommand{\childdocdisable}{}
}
%    \end{macrocode}

% \macro{\childdocmain}
% The macro |\childdocmain| is to be called at the top of the main file
% with nothing or the main filename (without extension) as argument.
% First, it breaks loops.
% If the argument is not empty and does not match |\childdocname|
% (which is set by the first inclusion of |childdoc.def|),
% |\ifchilddoc| is set to true, |\includeonly| is applied to the child file
% and |\jobname| is set to the main file
% (for proper handling of |.aux| files):
%    \begin{macrocode}
\newcommand{\childdocmain}[1]
{
  \childdocdisable\childdocmain{}
  \if?#1?\else
    \begingroup
      \def\childdoctmp{#1}
      \ifx\childdoctmp\childdocname
        \def\childdoctmp{}
      \else
        \def\childdoctmp
        {
          \childdoctrue
          \includeonly{\childdocname}
          \def\childdocjob{#1}
          \def\jobname{#1}
        }
      \fi
      \expandafter
    \endgroup
    \childdoctmp
  \fi
}
%    \end{macrocode}

% \macro{\childdocof}
% The command |\childdocof| redirects
% compilation to the main file |#1|.
%    \begin{macrocode}
\newcommand{\childdocof}[1]
{
  \childdocdisable
  \childdoctrue
  \includeonly{\childdocname}
  \def\jobname{#1}
  \def\childdocjob{#1}
  \input{#1}
}
%    \end{macrocode}

% \macro{\childdocby}
% The command |\childdocby| ....
%    \begin{macrocode}
\newcommand{\childdocby}[2][]
{
  \childdocdisable
  \childdoctrue
  \childdocmanualtrue
  \if?#1?\else
    \def\jobname{#2}
  \fi
  \def\childdocjob{#2}
  \input{#2}
  \endinput
}
%    \end{macrocode}

% \macro{\childdocforward}
% The command |\childdocforward| redirects
% compilation to the main file or
% (if the optional argument is given) a child file.
% Parameters are set as if the main file
% or a child file starting with |\childdocof| was compiled.
% Then compilation is handed over to the main file:
%    \begin{macrocode}
\newcommand{\childdocforward}[2][]
{
  \begingroup
    \if?#1?
      \def\childdoctmp
      {
        \def\childdocname{#2}
        \def\childdocjob{#2}
        \def\jobname{#2}
        \input{#2}
        \endinput
      }
    \else
      \def\childdoctmp
      {
        \childdocdisable
        \def\childdocname{#2}
        \childdoctrue
        \includeonly{#2}
        \def\childdocjob{#1}
        \def\jobname{#1}
        \input{#1}
        \endinput
      }
    \fi
    \expandafter
  \endgroup
  \childdoctmp
}
%    \end{macrocode}

% \macro{\childdocforwardprefix}
% The command |\childdocforwardprefix| redirects
% compilation to the main or a child file by means of a pattern.
% The prefix |#1| in the current filename is replaced by |#2|
% and the suffix of the current filename is kept
% (it is assumed that the filename does not contain the substring `|~~~|'
% which is used as a delimiter).
% Compilation is handed over to the new file by |\childdocforward|:
%    \begin{macrocode}
\newcommand{\childdocforwardprefix}[3][]
{
  \begingroup
    \def\childdocextract #2##1~~~{\def\childdoctmp{\childdocforward[#1]{#3##1}}}
    \expandafter\childdocextract\childdocname~~~
    \expandafter
  \endgroup
  \childdoctmp
}
%    \end{macrocode}

% \macro{\childdoc}
% The deprecated macro |\childdoc| is a legacy version of |\childdocmain|:
%    \begin{macrocode}
\newcommand{\childdoc}{\childdocmain}
%    \end{macrocode}

% \macro{\childdocredirect}
% The deprecated macro |\childdocredirect| is a legacy version
% of |\childdocforward| and |\childdocforwardprefix|:
%    \begin{macrocode}
\newcommand{\childdocredirect}[2][]
{
  \begingroup
    \if?#1?
      \def\childdoctmp{\childdocforward{#2}}
    \else
      \def\childdoctmp{\childdocforwardprefix{#1}{#2}}
    \fi
    \expandafter
  \endgroup
  \childdoctmp
}
%    \end{macrocode}

%\iffalse
%</package>
%\fi
%
\endinput
\childdocforward{cdocsamp}"|\\
% |latex -jobname cdocscl1 \|\\
% |  "% \iffalse
%
% childdoc.dtx Copyright (C) 2017-2018 Niklas Beisert
%
% This work may be distributed and/or modified under the
% conditions of the LaTeX Project Public License, either version 1.3
% of this license or (at your option) any later version.
% The latest version of this license is in
%   http://www.latex-project.org/lppl.txt
% and version 1.3 or later is part of all distributions of LaTeX
% version 2005/12/01 or later.
%
% This work has the LPPL maintenance status `maintained'.
%
% The Current Maintainer of this work is Niklas Beisert.
%
% This work consists of the files childdoc.dtx and childdoc.ins
% and the derived files childdoc.def and cdocsamp.tex with
% cdocsch1.tex, cdocsch2.tex, cdocsdrf.tex, cdocsfn1.tex, cdocsfn2.tex.
%
%<package>\ifdefined\childdocmain\endinput\fi
%<package>\ProvidesFile{childdoc.def}[2018/12/30 v2.0 child document driver]
%<samplemain>\ProvidesFile{cdocsamp.tex}[2018/12/30 v2.0 sample for childdoc]
%<*driver>
%\ProvidesFile{childdoc.drv}[2018/12/30 v2.0 childdoc reference manual file]
\PassOptionsToClass{10pt,a4paper}{article}
\documentclass{ltxdoc}

\usepackage[margin=35mm]{geometry}
\usepackage{hyperref}
\usepackage{hyperxmp}
\usepackage[usenames]{color}

\hypersetup{colorlinks=true}
\hypersetup{pdfstartview=FitH}
\hypersetup{pdfpagemode=UseNone}
\hypersetup{pdfsource={}}
\hypersetup{pdflang={en-UK}}
\hypersetup{pdfcopyright={Copyright 2017-2018 Niklas Beisert.
  This work may be distributed and/or modified under the
  conditions of the LaTeX Project Public License, either version 1.3
  of this license or (at your option) any later version.}}
\hypersetup{pdflicenseurl={http://www.latex-project.org/lppl.txt}}
\hypersetup{pdfcontactaddress={ETH Zurich, ITP, HIT K,
  Wolfgang-Pauli-Strasse 27}}
\hypersetup{pdfcontactpostcode={8093}}
\hypersetup{pdfcontactcity={Zurich}}
\hypersetup{pdfcontactcountry={Switzerland}}
\hypersetup{pdfcontactemail={nbeisert@itp.phys.ethz.ch}}
\hypersetup{pdfcontacturl={http://people.phys.ethz.ch/\xmptilde nbeisert/}}

\newcommand{\secref}[1]{\hyperref[#1]{section \ref*{#1}}}

\parskip1ex
\parindent0pt
\let\olditemize\itemize
\def\itemize{\olditemize\parskip0pt}

\begin{document}

\title{The \textsf{childdoc} Package}
\hypersetup{pdftitle={The childdoc Package}}
\author{Niklas Beisert\\[2ex]
  Institut f\"ur Theoretische Physik\\
  Eidgen\"ossische Technische Hochschule Z\"urich\\
  Wolfgang-Pauli-Strasse 27, 8093 Z\"urich, Switzerland\\[1ex]
  \href{mailto:nbeisert@itp.phys.ethz.ch}
  {\texttt{nbeisert@itp.phys.ethz.ch}}}
\hypersetup{pdfauthor={Niklas Beisert}}
\hypersetup{pdfsubject={Manual for the LaTeX2e Package childdoc}}
\date{30 December 2018, \textsf{v2.0}}
\maketitle

\begin{abstract}\noindent
\textsf{childdoc} is a \LaTeXe{} package
that enables the direct compilation
of document sections included by |\include|
to individual files.
\end{abstract}

\begingroup
\parskip0ex
\tableofcontents
\endgroup

%%%%%%%%%%%%%%%%%%%%%%%%%%%%%%%%%%%%%%%%%%%%%%%%%%%%%%%%%%%%%%%%%%%%%%%%%%%%%%%%
%%%%%%%%%%%%%%%%%%%%%%%%%%%%%%%%%%%%%%%%%%%%%%%%%%%%%%%%%%%%%%%%%%%%%%%%%%%%%%%%
\section{Introduction}

\LaTeX{} provides a mechanism to structure a large document (such as a book)
into a main file and several child files (containing the chapters)
using the |\include| command.
This mechanism is beneficial for documents
which span hundreds of pages in order to
make the source file(s) more manageable.
Moreover, compilation can be restricted to
selected child files by means of the |\includeonly| command.
The latter feature can be used to reduce the compilation time while editing
(this was significantly more useful in the earlier days of \LaTeX{})
or to generate a smaller document which is easier to navigate.
Another application of |\includeonly| is to generate
documents consisting of selected parts of the complete document.

However, there are a few drawbacks of the plain |\include| mechanism:
\begin{itemize}
\item
The child files cannot be compiled on their own,
they can only be compiled via the main file.
A naive editing environment
(such as a text editor with an option
to have the current file processed by \LaTeX)
may require one to switch to the main file before compiling;
attempting to compile the child file produces errors.
\item
The main file must be modified (each time)
to adjust the |\includeonly| command
to the present needs. This easily leaves the main file in a messy state.
\item
The generated document will always carry the filename
of the main document. This is inconvenient if
several child files are to be compiled and
to be kept for distribution.
\end{itemize}

The present package provides a simple interface
to make child files individually compilable by \LaTeX{}.
Compiling a child file then has the same effect as compiling
the main file with an |\includeonly| command
to select the appropriate child.
Moreover the generated document will carry the name of the child
rather than the main file.
This resolves all three above issues.

This feature is meant to make the editing of books,
thesis documents and lecture notes somewhat more convenient.
However, the package can also be used efficiently for
composing a series of documents (such as exercise sheets)
which are typically distributed individually.
It then assists the author in generating the individual documents
(potentially in different versions)
as well as a document containing the collected series.
Another application is in developing style files
or other kinds of included material
where compilation of the style file could redirect
to a sample or test file.

%%%%%%%%%%%%%%%%%%%%%%%%%%%%%%%%%%%%%%%%%%%%%%%%%%%%%%%%%%%%%%%%%%%%%%%%%%%%%%%%
%%%%%%%%%%%%%%%%%%%%%%%%%%%%%%%%%%%%%%%%%%%%%%%%%%%%%%%%%%%%%%%%%%%%%%%%%%%%%%%%
\section{Usage}

First of all, the package \textsf{childdoc} is \emph{not} a standard
\LaTeXe{} |.sty| style file! Therefore it needs to be invoked in
a non-standard way.

%%%%%%%%%%%%%%%%%%%%%%%%%%%%%%%%%%%%%%%%%%%%%%%%%%%%%%%%%%%%%%%%%%%%%%%%%%%%%%%%
\subsection{Included Files}
\label{sec:include}

%%%%%%%%%%%%%%%%%%%%%%%%%%%%%%%%%%%%%%%%
\DescribeMacro{\childdocmain}
To use the package, add the commands
\begin{center}
\begin{tabular}{l}
|\input{childdoc.def}|\\
|\childdocmain{}|\\
\end{tabular}
\end{center}
at the very top of the main \LaTeX{} file,
in particular \emph{before} the |\documentclass| statement!
The argument of |\childdocmain| should be left empty
(but it must be present).

%%%%%%%%%%%%%%%%%%%%%%%%%%%%%%%%%%%%%%%%
\DescribeMacro{\childdocof}
Furthermore, add the commands
\begin{center}
\begin{tabular}{l}
|\input{childdoc.def}|\\
|\childdocof{|\textit{main}|}|\\
\end{tabular}
\end{center}
at the top of every child file \textit{child}
which is included by |\include{|\textit{child}|}|
from within the main file
(or at least for those files to be compiled individually).
The argument \textit{main} must be the filename of the main file.

There are a couple of
considerations in setting up the main and child documents:

%%%%%%%%%%%%%%%%%%%%%%%%%%%%%%%%%%%%%%%%
\paragraph{Restrictions.}

Please note the following restrictions:
\begin{itemize}
\item
|\childdocmain| must be called with one argument \textit{main}
to ensure compatibility with earlier version of the package.
It must either be empty (|\childdocmain{}|)
or precisely match the filename of the main file in which it is specified.
See \secref{sec:detection} for further information.
\item
The filename \textit{main} must be specified without the |.tex| extension.
\item
The filename \textit{main} is case sensitive
(even in case-insensitive file systems)
due to internal string comparison.
\item
The argument \textit{main} should be fully expanded, it cannot be a macro.
\item
Subdirectories and special characters should be avoided in filenames.
\item
The command |\childdocmain{|\textit{main}|}| must be followed by a whitespace.
It should not be followed immediately by another command
or by a comment mark `|%|'.
This is because the \TeX{} parser reads the token immediately following
the argument of |\childdocmain| and puts it
at the beginning of every child section;
however, a white\-space is ignored.
\end{itemize}

%%%%%%%%%%%%%%%%%%%%%%%%%%%%%%%%%%%%%%%%
\paragraph{Content of Main File.}

It is advisable to place all content in the child files included by |\include|.
Any output contained in the main file will appear in all child documents
unless suppressed manually;
it cannot be suppressed automatically by the |\includeonly| directive
and thus should normally be avoided.
A method to include some content in the main file
by means of conditional processing is described in \secref{sec:conditional}.

%%%%%%%%%%%%%%%%%%%%%%%%%%%%%%%%%%%%%%%%
\paragraph{Page Numbering.}

When only a part of the document is compiled,
the appropriate numbering of pages
(as well as other status parameters)
is determined from the |.aux| files.
The latter contain information from previous passes.
However this information needs to propagate through
all intermediate child documents.
Therefore the page numbering in child documents may well
be inconsistent until the complete document is compiled at least once.

A useful (if unconventional) way to always ensure a consistent
page numbering is to restart the numbering in each child document
and denote the pages by `\textit{child}|.|\textit{page}'
where \textit{child} represents the chapter/section number of the child file.
This can be achieved by the command
|\numberwithin{page}{|\textit{child}|}|
of the \textsf{amsmath} package
where \textit{child} can be |chapter| or |section|
depending on the chosen structuring.
Alternatively, one can modify the macro |\thepage| appropriately
and reset the counter |page| at the start of each child file.

%%%%%%%%%%%%%%%%%%%%%%%%%%%%%%%%%%%%%%%%%%%%%%%%%%%%%%%%%%%%%%%%%%%%%%%%%%%%%%%%
\subsection{Conditional Processing}
\label{sec:conditional}

The package provides a mechanism to compile different versions
of a document. To customise the versions further some conditional processing
can come in handy to distinguish which version is being compiled.
The package provides two macros to describe the compilation context:

%%%%%%%%%%%%%%%%%%%%%%%%%%%%%%%%%%%%%%%%
\DescribeMacro{\ifchilddoc}
The conditional |\ifchilddoc| distinguishes between the compilation of
child documents and the main document:
%
\begin{center}
|\ifchilddoc |\textit{child-code}| |[|\||else |\textit{main-code}]| \||fi|
\end{center}

%%%%%%%%%%%%%%%%%%%%%%%%%%%%%%%%%%%%%%%%
\DescribeMacro{\childdocname}
\DescribeMacro{\childdocjob}
The macro |\childdocname| contains the filename (without extension)
of the main or child file being processed.
Note that |\childdocjob| will always contain the name of the main file.

%%%%%%%%%%%%%%%%%%%%%%%%%%%%%%%%%%%%%%%%
\paragraph{Title Page.}

Conditional processing can be used to include a title or banner page
in the main document when proper precautions are taken.
Importantly, the code in the main file should ensure that the page counter
(as well as other status parameters which are stored in the |.aux| files)
takes the same value after the conditional processing.
Otherwise the page numbers may take divergent values
depending on which part is compiled.

For example, a title page could be declared by:
%
\begin{center}
\begin{tabular}{l}
|\ifchilddoc\||else|\\
|\addtocounter{page}{-1}|\\
\textit{code for title page}\\
|\newpage|\\
|\||fi|
\end{tabular}
\end{center}
%
A banner page for the child documents can be generated by:
%
\begin{center}
\begin{tabular}{l}
|\ifchilddoc|\\
|\addtocounter{page}{-1}|\\
\textit{code for banner page}\\
|\newpage|\\
|\||fi|
\end{tabular}
\end{center}
%
Here one could write a message such as:
\begin{center}
|This is the part \childdocname{} of \childdocjob{}.|
\end{center}

%%%%%%%%%%%%%%%%%%%%%%%%%%%%%%%%%%%%%%%%%%%%%%%%%%%%%%%%%%%%%%%%%%%%%%%%%%%%%%%%
\subsection{Flags}
\label{sec:flags}

The package makes it easy to generate different versions
of the main or child documents.
To this end compilation flags can be defined
and assigned different default values.
They will be particularly useful in conjunction
with the forwarding mechanism described in \secref{sec:forward}.

For example, it may be useful to have a flag |\version|
which can be set to |draft| or |final|.
The document source will contain some conditional code
depending on the value of |\version|.
Suppose further, the flag should default to |final| for the main file
and to |draft| for child files
which is a natural assignment for editing the document.
This is achieved by placing the following code
in the preamble of the main document
(below the |\childdocmain| directive):
%
\begin{center}
\begin{tabular}{l}
|\ifchilddoc|\\
|\providecommand{\version}{draft}|\\
|\||else|\\
|\providecommand{\version}{final}|\\
|\||fi|
\end{tabular}
\end{center}
%
The definition by |\providecommand| makes sure
that previous definitions are not overwritten.
Further statements |\providecommand{\version}{...}|
can thus be added before the above code to override it.

For the main file, one might add a line
(between |\childdocmain| and the above block)
%
\begin{center}
|%\ifchilddoc\||else\providecommand{\version}{draft}\||fi|
\end{center}
%
which can be uncommented to produce a draft version.
Likewise one can add a line to the very top of a child file
(above the |\childdocof{|\textit{main}|}| directive)
%
\begin{center}
|%\providecommand{\version}{final}|
\end{center}
%
which can be uncommented to produce the final version of this child document.

%%%%%%%%%%%%%%%%%%%%%%%%%%%%%%%%%%%%%%%%%%%%%%%%%%%%%%%%%%%%%%%%%%%%%%%%%%%%%%%%
\subsection{Forwarding}
\label{sec:forward}

Different versions of the main or child documents
using compilation flags as described in \secref{sec:flags}
can be (permanently) stored in different files
for convenient compilation, viewing and distribution.
To this end, the package defines a command
to pass on compilation to a different file:

%%%%%%%%%%%%%%%%%%%%%%%%%%%%%%%%%%%%%%%%
\DescribeMacro{\childdocforward}
The command |\childdocforward| redirects processing to
another source file:
%
\begin{center}
\begin{tabular}{l}
|\input{childdoc.def}|\\
|\childdocforward[|\textit{main}|]{|\textit{dest}|}|\\
\end{tabular}
\end{center}
%
The argument \textit{dest} is the destination file
(without extension).
It should be the main file or one of the child files.
Note that further \textsf{childdoc} directives
such as |\childdocof| and |\childdocforward|
in the indicated file will be processed in this form.
The optional argument \textit{main}
passes on directly to the main file \textit{main}
while pretending to compile the child \textit{dest}.
This form behaves as if \textit{dest}
issues |\childdocof{|\textit{main}|}| right away,
and no further \textsf{childdoc} directives will be processed.

%%%%%%%%%%%%%%%%%%%%%%%%%%%%%%%%%%%%%%%%
\DescribeMacro{\...prefix}
In the alternative form |\childdocforwardprefix|,
%
\begin{center}
\begin{tabular}{l}
|\input{childdoc.def}|\\
|\childdocforwardprefix[|\textit{main}|]{|\textit{prefix}|}{|\textit{dest}|}|
\end{tabular}
\end{center}
%
the destination file is determined by a pattern
depending on the current file:
To make this work, the current file must be called
`{\textit{prefix}\hspace{0.2em}\textit{suffix}}'
with \textit{prefix} matching precisely the argument.
Processing is then passed on to the file
`{\textit{dest}\hspace{0.2em}\textit{suffix}}'.
Surely, the same effect is achieved by
directly specifying the
argument `{\textit{dest}\hspace{0.2em}\textit{suffix}}'
in the first form.
However, that requires to set up a different file
for each child. With the alternative form of the command
all these files can have exactly the same content
which simplifies setting them up and maintaining them.

For example, the following file |draft.tex|
with a compilation flag |\version| as described in \secref{sec:flags}
compiles the main document as a draft:
%
\begin{center}
\begin{tabular}{l}
|\def\version{draft}|\\
|\input{childdoc.def}|\\
|\childdocforward{|\textit{main}|}|
\end{tabular}
\end{center}
%
Likewise, the following files |final|\textit{nn}|.tex|
compile the final version of the child document
|child|\textit{nn}|.tex|:
%
\begin{center}
\begin{tabular}{l}
|\def\version{final}|\\
|\input{childdoc.def}|\\
|\childdocforwardprefix{final}{child}|
\end{tabular}
\end{center}
%

Note that when several versions of a main file and/or of each child file
are to be generated, it may be convenient to set up a |Makefile| or
shell script to automatise the process.

%%%%%%%%%%%%%%%%%%%%%%%%%%%%%%%%%%%%%%%%%%%%%%%%%%%%%%%%%%%%%%%%%%%%%%%%%%%%%%%%
\subsection{Command Line Processing}
\label{sec:commandline}

The effect of redirection files can also be achieved by invoking
the \LaTeX{} compiler with a more elaborate command line.
Most conveniently this should be done as part
of a shell script or a |Makefile|.

When using \textsf{childdoc} in the main file, the following
command lines effectively perform a redirection
(note that depending on the shell being used,
backslashes may have to be doubled: `|\|' $\to$ `|\\|'):
%
\begin{center}
|... -jobname "|\textit{target}|" |\\|"|[\textit{flags}]%
|\input{childdoc.def}\childdocforward[|\textit{main}|]{|\textit{dest}|}"|
\end{center}
%
Here \textit{target} is the name of the output file,
\textit{main} is the name of the main file
and \textit{dest} is the name of the main or child file to be processed
(all filenames without extensions).
The optional argument \textit{main} can be omitted
if \textit{main} matches \textit{dest}.
Optionally, compilation \textit{flags} can be defined via |\def| commands.
This command line makes the \TeX{} engine believe
it is compiling the file \textit{target}
whose content is specified as the latter parameter.
The provided code then forwards the processing to
\textit{main} or \textit{dest} as described in \secref{sec:forward}.

%%%%%%%%%%%%%%%%%%%%%%%%%%%%%%%%%%%%%%%%%%%%%%%%%%%%%%%%%%%%%%%%%%%%%%%%%%%%%%%%
\subsection{Include by Input}
\label{sec:input}

Including child documents by |\include| has some restrictions by design.
Most notably, the content of a child document always occupies
its own set of pages; pages cannot be shared between child documents.
Usually, this behaviour makes perfect sense
because each child document contain an essential part of the document.
However, in some situations it may be desirable to compose
a document from a collection of parts
without having mandatory page breaks between then.
For this case, the package
provides a mechanism to include parts
by |\input| which can also be processed individually.
However, by construction this mechanism
requires manual handling of the content to be output.

%%%%%%%%%%%%%%%%%%%%%%%%%%%%%%%%%%%%%%%%
\DescribeMacro{\ifchilddocmanual}
The main file should be prepared as usual, see \secref{sec:include}.
However, the document body must make a distinction
between processing of an individual part and of the main document, e.g.:
%
\begin{center}
\begin{tabular}{l}
|\ifchilddocmanual|\\
|\input{\childdocname}|\\
|\||else|\\
\textit{document body with }|\input{|\textit{part}|}|\\
|\||fi|
\end{tabular}
\end{center}
%
The conditional |\ifchilddocmanual| is true whenever
a part to be included by |\input| is being compiled,
and the name of the part is stored in |\childdocname|.

%%%%%%%%%%%%%%%%%%%%%%%%%%%%%%%%%%%%%%%%
\DescribeMacro{\childdocby}
Each part to be included by |\input| should start with:
%
\begin{center}
\begin{tabular}{l}
|\input{childdoc.def}|\\
|\childdocby{|\textit{main}|}|\\
\end{tabular}
\end{center}
%
The directive |\childdocby| is similar to |\childdocof|
described in \secref{sec:include},
but the subsequent selection of content must be done manually.
To that end, both |\ifchilddoc| and |\ifchilddocmanual|
will be true upon processing of a part,
and the name of the part is stored in |\childdocname|.
Note that |\jobname| will be set to the filename of the current part
so that each part receives an individual |.aux| file
that does not interfere with the |.aux| file(s) of the main document.
This behaviour can be altered by the alternative form
|\childdocby[*]{|\textit{main}|}| (with a non-empty optional argument)
which uses the |.aux| file of the main document
by setting |\jobname| to \textit{main}.

%%%%%%%%%%%%%%%%%%%%%%%%%%%%%%%%%%%%%%%%%%%%%%%%%%%%%%%%%%%%%%%%%%%%%%%%%%%%%%%%
\subsection{Driver Development}
\label{sec:driver}

The \textsf{childdoc} mechanism can also be use for the development
of definition files such as \LaTeX{} styles or classes.
This case differs from the above setup with multiple parts
included by |\include| in that no |\includeonly| should be invoked.
This can be achieved by starting the include file
(before |\ProvidesPackage|) with:
%
\begin{center}
\begin{tabular}{l}
|\input{childdoc.def}|\\
|\childdocforward{|\textit{main}|}|\\
\end{tabular}
\end{center}
%
or alternatively with:
%
\begin{center}
\begin{tabular}{l}
|\input{childdoc.def}|\\
|\childdocby{|\textit{main}|}|\\
\end{tabular}
\end{center}
%
Both forms have slightly different effects as described above.
The main file is prepared as usual, see \secref{sec:include}.

%%%%%%%%%%%%%%%%%%%%%%%%%%%%%%%%%%%%%%%%%%%%%%%%%%%%%%%%%%%%%%%%%%%%%%%%%%%%%%%%
\subsection{Legacy Detection}
\label{sec:detection}

The directive |\childdocmain| in the main file can detect
whether the complete document or merely a child is to be compiled
even without using the directive |\childdocof|.
This method is deprecated because it is less robust
and there is no compelling reason to use it;
it is merely provided for backward compatibility
and it may be removed in future versions.

If the detection mechanism is to be used,
it is mandatory to correctly specify
the filename of the main file as the argument of |\childdocmain|:
%
\begin{center}
\begin{tabular}{l}
|\input{childdoc.def}|\\
|\childdocmain{|\textit{main}|}|\\
\end{tabular}
\end{center}
%
If |\jobname| does not match the argument \textit{main} of |\childdocmain|,
it is assumed that |\jobname| points to the child file to be compiled.
When using |\childdocmain| with the main file specified as argument,
it suffices to start a child file
with just |\input{|\textit{main}|}|
without loading of the package and using |\childdocof|.
If instead all processing is done
with the appropriate \textsf{childdoc} directives,
the argument of \textit{main} of |\childdocmain| can be empty.

An alternative version of the command line processing described
in \secref{sec:commandline} using the detection mechanism reads:
%
\begin{center}
|... -jobname "|\textit{target}|" "|[\textit{flags}]%
[|\def\jobname{|\textit{dest}|}|]|\input{|\textit{main}|}"|
\end{center}

%%%%%%%%%%%%%%%%%%%%%%%%%%%%%%%%%%%%%%%%%%%%%%%%%%%%%%%%%%%%%%%%%%%%%%%%%%%%%%%%
\subsection{Manual Code}
\label{sec:manual}

In case one cannot be certain whether the definitions file |childdoc.def|
is installed on the target \TeX{} distribution
and one prefers not to ship it,
it is conceivable to paste a few relevant commands into the sources.

To that end, drop all statements |\input{childdoc.def}|
and perform the replacements as outlined below.
Instead of |\childdocmain{|\textit{main}|}| add the following code
to the top of the main file:
%
\begin{center}
\begin{tabular}{l}
|\||ifdefined\childdocname\endinput\||fi\newif\ifchilddoc|\\
|\edef\childdocname{\scantokens\expandafter{\jobname\noexpand}}|\\
|\def\childdocmain{|\textit{main}|}\||ifx\childdocmain\childdocname\||else|\\
|\childdoctrue\includeonly{\childdocname}\let\jobname\childdocmain\||fi|\\
\end{tabular}
\end{center}
%
Instead of |\childdocof{|\textit{main}|}| just include the main file
at the top of each child file:
%
\begin{center}
|\input{|\textit{main}|}|
\end{center}
%
A simple redirection |\childdocforward{|\textit{dest}|}| is achieved by:
%
\begin{center}
|\def\jobname{|\textit{dest}|}\input{\jobname}|
\end{center}
%
The redirection with prefix
|\childdocforwardprefix[|\textit{prefix}|]{|\textit{dest}|}|
is accomplished by:
%
\begin{center}
\begin{tabular}{l}
|{\edef\jobname{\scantokens\expandafter{\jobname\noexpand}}|\\
|\def\redirectjob |\textit{prefix}|#1~~~{\gdef\jobname{|\textit{dest}|#1}}|\\
|\expandafter\redirectjob\jobname~~~}\input{\jobname}|
\end{tabular}
\end{center}

In an alternative approach,
child documents can be compiled by a specific command line
without additional code or specific definitions:
%
\begin{center}
|... -jobname "|\textit{target}|" "|[\textit{flags}]%
|\includeonly{|\textit{dest}|}\input{|\textit{main}|}"|
\end{center}
%

%%%%%%%%%%%%%%%%%%%%%%%%%%%%%%%%%%%%%%%%%%%%%%%%%%%%%%%%%%%%%%%%%%%%%%%%%%%%%%%%
%%%%%%%%%%%%%%%%%%%%%%%%%%%%%%%%%%%%%%%%%%%%%%%%%%%%%%%%%%%%%%%%%%%%%%%%%%%%%%%%
\section{Information}

%%%%%%%%%%%%%%%%%%%%%%%%%%%%%%%%%%%%%%%%%%%%%%%%%%%%%%%%%%%%%%%%%%%%%%%%%%%%%%%%
\subsection{Copyright}

Copyright \copyright{} 2017--2018 Niklas Beisert

This work may be distributed and/or modified under the
conditions of the \LaTeX{} Project Public License, either version 1.3
of this license or (at your option) any later version.
The latest version of this license is in
  \url{http://www.latex-project.org/lppl.txt}
and version 1.3 or later is part of all distributions of \LaTeX{}
version 2005/12/01 or later.

This work has the LPPL maintenance status `maintained'.

The Current Maintainer of this work is Niklas Beisert.

This work consists of the files |README.txt|, |childdoc.ins| and |childdoc.dtx|
as well as the derived files |childdoc.def|, |cdocsamp.tex|
with |cdocsch1.tex|, |cdocsch2.tex|, |cdocspt3.tex|, |cdocspt4.tex|,
|cdocsdrf.tex|, |cdocsfn1.tex|, |cdocsfn2.tex|
as well as |childdoc.pdf|.

%%%%%%%%%%%%%%%%%%%%%%%%%%%%%%%%%%%%%%%%%%%%%%%%%%%%%%%%%%%%%%%%%%%%%%%%%%%%%%%%
\subsection{Files and Installation}

The package consists of the files:
%
\begin{center}
\begin{tabular}{ll}
    |README.txt|   & readme file \\
    |childdoc.ins| & installation file \\
    |childdoc.dtx| & source file \\
    |childdoc.def| & definition file \\
    |cdocsamp.tex| & sample main file \\
    |cdocsch1.tex| & sample include file \\
    |cdocsch2.tex| & sample include file \\
    |cdocspt3.tex| & sample part file \\
    |cdocspt4.tex| & sample part file \\
    |cdocsdrf.tex| & sample redirection file \\
    |cdocsfn1.tex| & sample redirection file \\
    |cdocsfn2.tex| & sample redirection file \\
    |childdoc.pdf| & manual
\end{tabular}
\end{center}
%
The distribution consists of the files
|README.txt|, |childdoc.ins| and |childdoc.dtx|.
%
\begin{itemize}
\item
Run (pdf)\LaTeX{} on |childdoc.dtx|
to compile the manual |childdoc.pdf| (this file).
\item
Run \LaTeX{} on |childdoc.ins| to create the definitions file |childdoc.def|
and the sample |cdocsamp.tex| with include files
|cdocsch1.tex|, |cdocsch2.tex|, |cdocspt3.tex|, |cdocspt4.tex|,
|cdocsdrf.tex|, |cdocsfn1.tex|, |cdocsfn2.tex|.
Then copy the file |childdoc.def| to an appropriate directory of your \LaTeX{}
distribution, e.g.\ \textit{texmf-root}|/tex/latex/childdoc|.
\end{itemize}

%%%%%%%%%%%%%%%%%%%%%%%%%%%%%%%%%%%%%%%%%%%%%%%%%%%%%%%%%%%%%%%%%%%%%%%%%%%%%%%%
\subsection{Related CTAN Packages}

There are several other packages which offer a similar functionality:
%
\begin{itemize}
\item
The packages
\href{http://ctan.org/pkg/docmute}{\textsf{docmute}},
\href{http://ctan.org/pkg/includex}{\textsf{includex}} and
\href{http://ctan.org/pkg/standalone}{\textsf{standalone}}
provide commands to include only the document body of
a child file thus allowing both files to be compiled individually.
\item
The packages \href{http://ctan.org/pkg/subdocs}{\textsf{subdocs}}
and \href{http://ctan.org/pkg/subfiles}{\textsf{subfiles}}
provide structures in which the main and child documents can be
encapsulated and allowing them to be compiled individually.
The inclusion mechanism is different from the conventional |\include|.
\item
The package \href{http://ctan.org/pkg/combine}{\textsf{combine}}
is an elaborate solution to combine several documents into one.
\end{itemize}
%
See also the CTAN topic \href{http://ctan.org/topic/subdocs}{\textsf{subdocs}}
for further related packages.
The present package differs from the above solutions in that
a document structure constructed with the conventional |\include| mechanism
just needs two extra commands at the top of every file
such that all constituent files can be compiled individually.

%%%%%%%%%%%%%%%%%%%%%%%%%%%%%%%%%%%%%%%%%%%%%%%%%%%%%%%%%%%%%%%%%%%%%%%%%%%%%%%%
%\subsection{Feature Suggestions}
%
%The following is a list of features which may be useful for future
%versions of this package:
%%
%\begin{itemize}
%\item
%\ldots
%\end{itemize}

%%%%%%%%%%%%%%%%%%%%%%%%%%%%%%%%%%%%%%%%%%%%%%%%%%%%%%%%%%%%%%%%%%%%%%%%%%%%%%%%
\subsection{Revision History}

%%%%%%%%%%%%%%%%%%%%%%%%%%%%%%%%%%%%%%%%
\paragraph{v2.0:} 2018/12/30

\begin{itemize}
\item
immediate forward processing
\item
added |\childdocby| mechanism
\item
manual restructured
\end{itemize}

%%%%%%%%%%%%%%%%%%%%%%%%%%%%%%%%%%%%%%%%
\paragraph{v1.6:} 2018/01/17

\begin{itemize}
\item
application for development of include files
\item
corrections to manual
\end{itemize}

%%%%%%%%%%%%%%%%%%%%%%%%%%%%%%%%%%%%%%%%
\paragraph{v1.5:} 2017/05/21

\begin{itemize}
\item
more complete structuring introduced
\item
|\childdocof| introduced
\item
|\childdoc| renamed to |\childdocmain|
\item
|\childredirect| renamed to |\childdocforward| and |\childdocforwardprefix|
and functionality expanded
\end{itemize}

%%%%%%%%%%%%%%%%%%%%%%%%%%%%%%%%%%%%%%%%
\paragraph{v1.0:} 2017/04/27

\begin{itemize}
\item
manual and install package
\item
first version published on CTAN
\end{itemize}

%%%%%%%%%%%%%%%%%%%%%%%%%%%%%%%%%%%%%%%%
\paragraph{v0.6:} 2017/04/26

\begin{itemize}
\item
redirection mechanism added
\end{itemize}

%%%%%%%%%%%%%%%%%%%%%%%%%%%%%%%%%%%%%%%%
\paragraph{v0.5:} 2017/04/26

\begin{itemize}
\item
functionality in definition file
\end{itemize}


%%%%%%%%%%%%%%%%%%%%%%%%%%%%%%%%%%%%%%%%%%%%%%%%%%%%%%%%%%%%%%%%%%%%%%%%%%%%%%%%
%%%%%%%%%%%%%%%%%%%%%%%%%%%%%%%%%%%%%%%%%%%%%%%%%%%%%%%%%%%%%%%%%%%%%%%%%%%%%%%%
%%%%%%%%%%%%%%%%%%%%%%%%%%%%%%%%%%%%%%%%%%%%%%%%%%%%%%%%%%%%%%%%%%%%%%%%%%%%%%%%
\appendix

\settowidth\MacroIndent{\rmfamily\scriptsize 000\ }

 \DocInput{childdoc.dtx}

\end{document}
%</driver>
% \fi
%
% %%%%%%%%%%%%%%%%%%%%%%%%%%%%%%%%%%%%%%%%%%%%%%%%%%%%%%%%%%%%%%%%%%%%%%%%%%%%%%
% %%%%%%%%%%%%%%%%%%%%%%%%%%%%%%%%%%%%%%%%%%%%%%%%%%%%%%%%%%%%%%%%%%%%%%%%%%%%%%
% \section{Sample}
%\iffalse
%<*samplemain>
%\fi
%
% The following presents a sample document
% with two chapters, two parts, a title page,
% a compile flag as well as three forwarding files to set the flag.
% It consists of eight |.tex| files:
% \begin{center}
% \begin{tabular}{ll}
% |cdocsamp.tex|&main file\\
% |cdocsch1.tex|&include file for chapter 1\\
% |cdocsch2.tex|&include file for chapter 2\\
% |cdocspt3.tex|&include file for part 3\\
% |cdocspt4.tex|&include file for part 4\\
% |cdocsdrf.tex|&forwarding file for main file in draft mode\\
% |cdocsfi1.tex|&forwarding file for final version of chapter 1\\
% |cdocsfi2.tex|&forwarding file for final version of chapter 2\\
% \end{tabular}
% \end{center}
% Each of the eight files can be compiled directly by the \LaTeX{} compiler.
%
% %%%%%%%%%%%%%%%%%%%%%%%%%%%%%%%%%%%%%%
% \paragraph{Main File.}
%
% The main file is called |cdocsamp.tex|.
%
% Load the \textsf{childdoc} definitions and
% declare the filename for the main document:
%    \begin{macrocode}
\input{childdoc.def}
\childdocmain{}
%    \end{macrocode}

% Optional override for |\version| flag:
%    \begin{macrocode}
%%\ifchilddoc\else\providecommand{\version}{draft}\fi
%    \end{macrocode}

% Define the default values for the |\version| flag
% (|final| for the main file and |draft| for childs):
%    \begin{macrocode}
\ifchilddoc
\providecommand{\version}{draft}
\else
\providecommand{\version}{final}
\fi
%    \end{macrocode}

% Load the standard document class:
%    \begin{macrocode}
\documentclass[12pt]{article}
%    \end{macrocode}

% Start the document body:
%    \begin{macrocode}
\begin{document}
%    \end{macrocode}

% Declare a title page.
% Print title, part of document being processed and version flag:
%    \begin{macrocode}
\addtocounter{page}{-1}
\begin{center}
{\LARGE\bfseries{}childdoc example\par}
\vspace{1cm}
\ifchilddoc
\ifchilddocmanual part\else chapter\fi:
`\childdocname' of `\childdocjob'\par
\else
main document: `\childdocjob'\par
\fi
version: \version\par
\end{center}
\newpage
%    \end{macrocode}

% Manually include selected file,
% otherwise process as usual:
%    \begin{macrocode}
\ifchilddocmanual
\section*{part `\childdocname'}
\input{\childdocname}
\else
%    \end{macrocode}

% Include the two chapters:
%    \begin{macrocode}
\include{cdocsch1}
\include{cdocsch2}
%    \end{macrocode}

% Include the two parts unless only chapters should be displayed:
%    \begin{macrocode}
\ifchilddoc\else
\section{part three}
\input{cdocspt3}
\section{part four}
\input{cdocspt4}
\fi
%    \end{macrocode}

% Process as usual until here:
%    \begin{macrocode}
\fi
%    \end{macrocode}

% End of document body:
%    \begin{macrocode}
\end{document}
%    \end{macrocode}
%\iffalse
%</samplemain>
%\fi
%
% %%%%%%%%%%%%%%%%%%%%%%%%%%%%%%%%%%%%%%
% \paragraph{Chapter Include Files.}
%
% The include files are called |cdocsch1.tex| and |cdocsch2.tex|.
%
%\iffalse
%<*samplechap1|samplechap2>
%\fi

% Optional override for |\version| flag:
%    \begin{macrocode}
%%\providecommand{\version}{final}
%    \end{macrocode}

% Include the main document:
%    \begin{macrocode}
\input{childdoc.def}
\childdocof{cdocsamp}
%    \end{macrocode}

%\iffalse
%</samplechap1|samplechap2>
%\fi
%
%\iffalse
%<*samplechap1>
%\fi
% Some text for chapter 1:
%    \begin{macrocode}
\section{one}
some text in chapter one
%    \end{macrocode}

%\iffalse
%</samplechap1>
%\fi
% Some text for chapter 2:
%\iffalse
%<*samplechap2>
%\fi
%    \begin{macrocode}
\section{two}
more text in chapter two
%    \end{macrocode}

%\iffalse
%</samplechap2>
%\fi
%
% %%%%%%%%%%%%%%%%%%%%%%%%%%%%%%%%%%%%%%
% \paragraph{Part Include Files.}
%
% The include files are called |cdocspt3.tex| and |cdocspt4.tex|.
%
%\iffalse
%<*samplepart3|samplepart4>
%\fi

% Optional override for |\version| flag:
%    \begin{macrocode}
%%\providecommand{\version}{final}
%    \end{macrocode}

% Include the main document:
%    \begin{macrocode}
\input{childdoc.def}
\childdocby{cdocsamp}
%    \end{macrocode}

%\iffalse
%</samplepart3|samplepart4>
%\fi
%
%\iffalse
%<*samplepart3>
%\fi
% Some text for part 3:
%    \begin{macrocode}
some text in part three
%    \end{macrocode}

%\iffalse
%</samplepart3>
%\fi
% Some text for part 4:
%\iffalse
%<*samplepart4>
%\fi
%    \begin{macrocode}
more text in part four
%    \end{macrocode}

%\iffalse
%</samplepart4>
%\fi
%
% %%%%%%%%%%%%%%%%%%%%%%%%%%%%%%%%%%%%%%
% \paragraph{Forwarding for a Complete Draft.}
%
% The following forwarding file |cdocsdrf.tex|
% compiles the main document in draft mode:
%\iffalse
%<*sampledraft>
%\fi
%    \begin{macrocode}
\def\version{draft}
\input{childdoc.def}
\childdocforward{cdocsamp}
%    \end{macrocode}

%\iffalse
%</sampledraft>
%\fi
%
% %%%%%%%%%%%%%%%%%%%%%%%%%%%%%%%%%%%%%%
% \paragraph{Forwarding for Final Version of the Chapters.}
%
% The following forwarding files |cdocsfn1.tex| and |cdocsfn2.tex|
% (with identical content)
% compile the final versions of the child documents
% |cdocsch1.tex| and |cdocsch2.tex|, respectively:
%\iffalse
%<*samplefinal>
%\fi
%    \begin{macrocode}
\def\version{final}
\input{childdoc.def}
\childdocforwardprefix[cdocsamp]{cdocsfn}{cdocsch}
%    \end{macrocode}

%\iffalse
%</samplefinal>
%\fi
%
% %%%%%%%%%%%%%%%%%%%%%%%%%%%%%%%%%%%%%%
% \paragraph{Command Line Processing.}
%
% The following three command lines generate the output files
% |cdocscld|, |cdocscl1| and |cdocscl2|
% which should be identical to
% |cdocsdrf|, |cdocsch1| and |cdocsfn2|, respectively:
% \begin{center}
% \begin{tabular}{l}
% |latex -jobname cdocscld \|\\
% |  "\def\version{draft}\input{childdoc.def}\childdocforward{cdocsamp}"|\\
% |latex -jobname cdocscl1 \|\\
% |  "\input{childdoc.def}\childdocforward[cdocsamp]{cdocsch1}"|\\
% |latex -jobname cdocscl2 \|\\
% |  "\def\version{final}\input{childdoc.def}\childdocforward{cdocsch2}"|
% \end{tabular}
% \end{center}
% Note that the trailing backslash on each first line
% merely continues the input to the second line
% (for convenient cut ant paste).
% Furthermore, the command |latex| can be replaced by any
% of its alternative versions such as |pdflatex|.
%
% %%%%%%%%%%%%%%%%%%%%%%%%%%%%%%%%%%%%%%%%%%%%%%%%%%%%%%%%%%%%%%%%%%%%%%%%%%%%%%
% %%%%%%%%%%%%%%%%%%%%%%%%%%%%%%%%%%%%%%%%%%%%%%%%%%%%%%%%%%%%%%%%%%%%%%%%%%%%%%
% \section{Implementation}
%\iffalse
%<*package>
%\fi
%
% This section describes the definitions file |childdoc.def|.

% The definitions cannot be loaded using |\usepackage| or |\RequirePackage|
% which has a mechanism to prevent loading a style file more than once.
% When loading the definitions by means of |\input|
% multiple instances have to be prevented manually:
%\iffalse
%This code needs to be before the `\ProvidesFile' directive
%which is defined at the beginning of this file.
%Therefore it is also placed there and commented out here.
%</package>
%<*discard>
%\fi
%    \begin{macrocode}
\ifdefined\childdocmain\endinput\fi
%    \end{macrocode}
%\iffalse
%</discard>
%<*package>
%\fi
%
% \macro{\ifchilddoc}
% \macro{\ifchilddocmanual}
% The conditional |\ifchilddoc| tells whether a
% child (true) or main (false) document is being compiled.
% The conditional |\ifchilddocmanual| tells whether
% the |\includeonly| mechanism is used (false) or
% the selection of child files must be performed manually (true).
% The definitions initialise to false:
%    \begin{macrocode}
\newif\ifchilddoc
\newif\ifchilddocmanual
%    \end{macrocode}

% \macro{\childdocname}
% \macro{\childdocjob}
% The macro |\childdocname| stores the name of the main document
% to be compiled. The macro |\childdocjob| stores the name of
% the document on which the \LaTeX{} compiler was originally invoked.
% The content of |\jobname| cannot be compared
% to filenames specified in the source due to different catcodes.
% The following code rescans |\jobname|, stores the result
% in |\childdocname| and saves a copy in |\childdocjob|:
%    \begin{macrocode}
\edef\childdocname{\scantokens\expandafter{\jobname\noexpand}}
\let\childdocjob\childdocname
%    \end{macrocode}

% \macro{\childdocdisable}
% The macro |\childdocdisable| prevents the main file
% from being processed more than once.
% At this stage, the main document command |\childdocmain|
% is assumed to be called once again where it should do nothing.
% Any subsequent call to it should prevent
% a secondary processing of the main document
% It overwrites the forwarding commands
% |\childdocof| and |\childdocforward|
% with empty macros to prevent further inclusions of the main document:
%    \begin{macrocode}
\newcommand{\childdocdisable}
{
  \renewcommand{\childdocmain}[1]{\renewcommand{\childdocmain}[1]{\endinput}}
  \renewcommand{\childdocof}[1]{}
  \renewcommand{\childdocby}[2][]{}
  \renewcommand{\childdocforward}[2][]{}
  \renewcommand{\childdocdisable}{}
}
%    \end{macrocode}

% \macro{\childdocmain}
% The macro |\childdocmain| is to be called at the top of the main file
% with nothing or the main filename (without extension) as argument.
% First, it breaks loops.
% If the argument is not empty and does not match |\childdocname|
% (which is set by the first inclusion of |childdoc.def|),
% |\ifchilddoc| is set to true, |\includeonly| is applied to the child file
% and |\jobname| is set to the main file
% (for proper handling of |.aux| files):
%    \begin{macrocode}
\newcommand{\childdocmain}[1]
{
  \childdocdisable\childdocmain{}
  \if?#1?\else
    \begingroup
      \def\childdoctmp{#1}
      \ifx\childdoctmp\childdocname
        \def\childdoctmp{}
      \else
        \def\childdoctmp
        {
          \childdoctrue
          \includeonly{\childdocname}
          \def\childdocjob{#1}
          \def\jobname{#1}
        }
      \fi
      \expandafter
    \endgroup
    \childdoctmp
  \fi
}
%    \end{macrocode}

% \macro{\childdocof}
% The command |\childdocof| redirects
% compilation to the main file |#1|.
%    \begin{macrocode}
\newcommand{\childdocof}[1]
{
  \childdocdisable
  \childdoctrue
  \includeonly{\childdocname}
  \def\jobname{#1}
  \def\childdocjob{#1}
  \input{#1}
}
%    \end{macrocode}

% \macro{\childdocby}
% The command |\childdocby| ....
%    \begin{macrocode}
\newcommand{\childdocby}[2][]
{
  \childdocdisable
  \childdoctrue
  \childdocmanualtrue
  \if?#1?\else
    \def\jobname{#2}
  \fi
  \def\childdocjob{#2}
  \input{#2}
  \endinput
}
%    \end{macrocode}

% \macro{\childdocforward}
% The command |\childdocforward| redirects
% compilation to the main file or
% (if the optional argument is given) a child file.
% Parameters are set as if the main file
% or a child file starting with |\childdocof| was compiled.
% Then compilation is handed over to the main file:
%    \begin{macrocode}
\newcommand{\childdocforward}[2][]
{
  \begingroup
    \if?#1?
      \def\childdoctmp
      {
        \def\childdocname{#2}
        \def\childdocjob{#2}
        \def\jobname{#2}
        \input{#2}
        \endinput
      }
    \else
      \def\childdoctmp
      {
        \childdocdisable
        \def\childdocname{#2}
        \childdoctrue
        \includeonly{#2}
        \def\childdocjob{#1}
        \def\jobname{#1}
        \input{#1}
        \endinput
      }
    \fi
    \expandafter
  \endgroup
  \childdoctmp
}
%    \end{macrocode}

% \macro{\childdocforwardprefix}
% The command |\childdocforwardprefix| redirects
% compilation to the main or a child file by means of a pattern.
% The prefix |#1| in the current filename is replaced by |#2|
% and the suffix of the current filename is kept
% (it is assumed that the filename does not contain the substring `|~~~|'
% which is used as a delimiter).
% Compilation is handed over to the new file by |\childdocforward|:
%    \begin{macrocode}
\newcommand{\childdocforwardprefix}[3][]
{
  \begingroup
    \def\childdocextract #2##1~~~{\def\childdoctmp{\childdocforward[#1]{#3##1}}}
    \expandafter\childdocextract\childdocname~~~
    \expandafter
  \endgroup
  \childdoctmp
}
%    \end{macrocode}

% \macro{\childdoc}
% The deprecated macro |\childdoc| is a legacy version of |\childdocmain|:
%    \begin{macrocode}
\newcommand{\childdoc}{\childdocmain}
%    \end{macrocode}

% \macro{\childdocredirect}
% The deprecated macro |\childdocredirect| is a legacy version
% of |\childdocforward| and |\childdocforwardprefix|:
%    \begin{macrocode}
\newcommand{\childdocredirect}[2][]
{
  \begingroup
    \if?#1?
      \def\childdoctmp{\childdocforward{#2}}
    \else
      \def\childdoctmp{\childdocforwardprefix{#1}{#2}}
    \fi
    \expandafter
  \endgroup
  \childdoctmp
}
%    \end{macrocode}

%\iffalse
%</package>
%\fi
%
\endinput
\childdocforward[cdocsamp]{cdocsch1}"|\\
% |latex -jobname cdocscl2 \|\\
% |  "\def\version{final}% \iffalse
%
% childdoc.dtx Copyright (C) 2017-2018 Niklas Beisert
%
% This work may be distributed and/or modified under the
% conditions of the LaTeX Project Public License, either version 1.3
% of this license or (at your option) any later version.
% The latest version of this license is in
%   http://www.latex-project.org/lppl.txt
% and version 1.3 or later is part of all distributions of LaTeX
% version 2005/12/01 or later.
%
% This work has the LPPL maintenance status `maintained'.
%
% The Current Maintainer of this work is Niklas Beisert.
%
% This work consists of the files childdoc.dtx and childdoc.ins
% and the derived files childdoc.def and cdocsamp.tex with
% cdocsch1.tex, cdocsch2.tex, cdocsdrf.tex, cdocsfn1.tex, cdocsfn2.tex.
%
%<package>\ifdefined\childdocmain\endinput\fi
%<package>\ProvidesFile{childdoc.def}[2018/12/30 v2.0 child document driver]
%<samplemain>\ProvidesFile{cdocsamp.tex}[2018/12/30 v2.0 sample for childdoc]
%<*driver>
%\ProvidesFile{childdoc.drv}[2018/12/30 v2.0 childdoc reference manual file]
\PassOptionsToClass{10pt,a4paper}{article}
\documentclass{ltxdoc}

\usepackage[margin=35mm]{geometry}
\usepackage{hyperref}
\usepackage{hyperxmp}
\usepackage[usenames]{color}

\hypersetup{colorlinks=true}
\hypersetup{pdfstartview=FitH}
\hypersetup{pdfpagemode=UseNone}
\hypersetup{pdfsource={}}
\hypersetup{pdflang={en-UK}}
\hypersetup{pdfcopyright={Copyright 2017-2018 Niklas Beisert.
  This work may be distributed and/or modified under the
  conditions of the LaTeX Project Public License, either version 1.3
  of this license or (at your option) any later version.}}
\hypersetup{pdflicenseurl={http://www.latex-project.org/lppl.txt}}
\hypersetup{pdfcontactaddress={ETH Zurich, ITP, HIT K,
  Wolfgang-Pauli-Strasse 27}}
\hypersetup{pdfcontactpostcode={8093}}
\hypersetup{pdfcontactcity={Zurich}}
\hypersetup{pdfcontactcountry={Switzerland}}
\hypersetup{pdfcontactemail={nbeisert@itp.phys.ethz.ch}}
\hypersetup{pdfcontacturl={http://people.phys.ethz.ch/\xmptilde nbeisert/}}

\newcommand{\secref}[1]{\hyperref[#1]{section \ref*{#1}}}

\parskip1ex
\parindent0pt
\let\olditemize\itemize
\def\itemize{\olditemize\parskip0pt}

\begin{document}

\title{The \textsf{childdoc} Package}
\hypersetup{pdftitle={The childdoc Package}}
\author{Niklas Beisert\\[2ex]
  Institut f\"ur Theoretische Physik\\
  Eidgen\"ossische Technische Hochschule Z\"urich\\
  Wolfgang-Pauli-Strasse 27, 8093 Z\"urich, Switzerland\\[1ex]
  \href{mailto:nbeisert@itp.phys.ethz.ch}
  {\texttt{nbeisert@itp.phys.ethz.ch}}}
\hypersetup{pdfauthor={Niklas Beisert}}
\hypersetup{pdfsubject={Manual for the LaTeX2e Package childdoc}}
\date{30 December 2018, \textsf{v2.0}}
\maketitle

\begin{abstract}\noindent
\textsf{childdoc} is a \LaTeXe{} package
that enables the direct compilation
of document sections included by |\include|
to individual files.
\end{abstract}

\begingroup
\parskip0ex
\tableofcontents
\endgroup

%%%%%%%%%%%%%%%%%%%%%%%%%%%%%%%%%%%%%%%%%%%%%%%%%%%%%%%%%%%%%%%%%%%%%%%%%%%%%%%%
%%%%%%%%%%%%%%%%%%%%%%%%%%%%%%%%%%%%%%%%%%%%%%%%%%%%%%%%%%%%%%%%%%%%%%%%%%%%%%%%
\section{Introduction}

\LaTeX{} provides a mechanism to structure a large document (such as a book)
into a main file and several child files (containing the chapters)
using the |\include| command.
This mechanism is beneficial for documents
which span hundreds of pages in order to
make the source file(s) more manageable.
Moreover, compilation can be restricted to
selected child files by means of the |\includeonly| command.
The latter feature can be used to reduce the compilation time while editing
(this was significantly more useful in the earlier days of \LaTeX{})
or to generate a smaller document which is easier to navigate.
Another application of |\includeonly| is to generate
documents consisting of selected parts of the complete document.

However, there are a few drawbacks of the plain |\include| mechanism:
\begin{itemize}
\item
The child files cannot be compiled on their own,
they can only be compiled via the main file.
A naive editing environment
(such as a text editor with an option
to have the current file processed by \LaTeX)
may require one to switch to the main file before compiling;
attempting to compile the child file produces errors.
\item
The main file must be modified (each time)
to adjust the |\includeonly| command
to the present needs. This easily leaves the main file in a messy state.
\item
The generated document will always carry the filename
of the main document. This is inconvenient if
several child files are to be compiled and
to be kept for distribution.
\end{itemize}

The present package provides a simple interface
to make child files individually compilable by \LaTeX{}.
Compiling a child file then has the same effect as compiling
the main file with an |\includeonly| command
to select the appropriate child.
Moreover the generated document will carry the name of the child
rather than the main file.
This resolves all three above issues.

This feature is meant to make the editing of books,
thesis documents and lecture notes somewhat more convenient.
However, the package can also be used efficiently for
composing a series of documents (such as exercise sheets)
which are typically distributed individually.
It then assists the author in generating the individual documents
(potentially in different versions)
as well as a document containing the collected series.
Another application is in developing style files
or other kinds of included material
where compilation of the style file could redirect
to a sample or test file.

%%%%%%%%%%%%%%%%%%%%%%%%%%%%%%%%%%%%%%%%%%%%%%%%%%%%%%%%%%%%%%%%%%%%%%%%%%%%%%%%
%%%%%%%%%%%%%%%%%%%%%%%%%%%%%%%%%%%%%%%%%%%%%%%%%%%%%%%%%%%%%%%%%%%%%%%%%%%%%%%%
\section{Usage}

First of all, the package \textsf{childdoc} is \emph{not} a standard
\LaTeXe{} |.sty| style file! Therefore it needs to be invoked in
a non-standard way.

%%%%%%%%%%%%%%%%%%%%%%%%%%%%%%%%%%%%%%%%%%%%%%%%%%%%%%%%%%%%%%%%%%%%%%%%%%%%%%%%
\subsection{Included Files}
\label{sec:include}

%%%%%%%%%%%%%%%%%%%%%%%%%%%%%%%%%%%%%%%%
\DescribeMacro{\childdocmain}
To use the package, add the commands
\begin{center}
\begin{tabular}{l}
|\input{childdoc.def}|\\
|\childdocmain{}|\\
\end{tabular}
\end{center}
at the very top of the main \LaTeX{} file,
in particular \emph{before} the |\documentclass| statement!
The argument of |\childdocmain| should be left empty
(but it must be present).

%%%%%%%%%%%%%%%%%%%%%%%%%%%%%%%%%%%%%%%%
\DescribeMacro{\childdocof}
Furthermore, add the commands
\begin{center}
\begin{tabular}{l}
|\input{childdoc.def}|\\
|\childdocof{|\textit{main}|}|\\
\end{tabular}
\end{center}
at the top of every child file \textit{child}
which is included by |\include{|\textit{child}|}|
from within the main file
(or at least for those files to be compiled individually).
The argument \textit{main} must be the filename of the main file.

There are a couple of
considerations in setting up the main and child documents:

%%%%%%%%%%%%%%%%%%%%%%%%%%%%%%%%%%%%%%%%
\paragraph{Restrictions.}

Please note the following restrictions:
\begin{itemize}
\item
|\childdocmain| must be called with one argument \textit{main}
to ensure compatibility with earlier version of the package.
It must either be empty (|\childdocmain{}|)
or precisely match the filename of the main file in which it is specified.
See \secref{sec:detection} for further information.
\item
The filename \textit{main} must be specified without the |.tex| extension.
\item
The filename \textit{main} is case sensitive
(even in case-insensitive file systems)
due to internal string comparison.
\item
The argument \textit{main} should be fully expanded, it cannot be a macro.
\item
Subdirectories and special characters should be avoided in filenames.
\item
The command |\childdocmain{|\textit{main}|}| must be followed by a whitespace.
It should not be followed immediately by another command
or by a comment mark `|%|'.
This is because the \TeX{} parser reads the token immediately following
the argument of |\childdocmain| and puts it
at the beginning of every child section;
however, a white\-space is ignored.
\end{itemize}

%%%%%%%%%%%%%%%%%%%%%%%%%%%%%%%%%%%%%%%%
\paragraph{Content of Main File.}

It is advisable to place all content in the child files included by |\include|.
Any output contained in the main file will appear in all child documents
unless suppressed manually;
it cannot be suppressed automatically by the |\includeonly| directive
and thus should normally be avoided.
A method to include some content in the main file
by means of conditional processing is described in \secref{sec:conditional}.

%%%%%%%%%%%%%%%%%%%%%%%%%%%%%%%%%%%%%%%%
\paragraph{Page Numbering.}

When only a part of the document is compiled,
the appropriate numbering of pages
(as well as other status parameters)
is determined from the |.aux| files.
The latter contain information from previous passes.
However this information needs to propagate through
all intermediate child documents.
Therefore the page numbering in child documents may well
be inconsistent until the complete document is compiled at least once.

A useful (if unconventional) way to always ensure a consistent
page numbering is to restart the numbering in each child document
and denote the pages by `\textit{child}|.|\textit{page}'
where \textit{child} represents the chapter/section number of the child file.
This can be achieved by the command
|\numberwithin{page}{|\textit{child}|}|
of the \textsf{amsmath} package
where \textit{child} can be |chapter| or |section|
depending on the chosen structuring.
Alternatively, one can modify the macro |\thepage| appropriately
and reset the counter |page| at the start of each child file.

%%%%%%%%%%%%%%%%%%%%%%%%%%%%%%%%%%%%%%%%%%%%%%%%%%%%%%%%%%%%%%%%%%%%%%%%%%%%%%%%
\subsection{Conditional Processing}
\label{sec:conditional}

The package provides a mechanism to compile different versions
of a document. To customise the versions further some conditional processing
can come in handy to distinguish which version is being compiled.
The package provides two macros to describe the compilation context:

%%%%%%%%%%%%%%%%%%%%%%%%%%%%%%%%%%%%%%%%
\DescribeMacro{\ifchilddoc}
The conditional |\ifchilddoc| distinguishes between the compilation of
child documents and the main document:
%
\begin{center}
|\ifchilddoc |\textit{child-code}| |[|\||else |\textit{main-code}]| \||fi|
\end{center}

%%%%%%%%%%%%%%%%%%%%%%%%%%%%%%%%%%%%%%%%
\DescribeMacro{\childdocname}
\DescribeMacro{\childdocjob}
The macro |\childdocname| contains the filename (without extension)
of the main or child file being processed.
Note that |\childdocjob| will always contain the name of the main file.

%%%%%%%%%%%%%%%%%%%%%%%%%%%%%%%%%%%%%%%%
\paragraph{Title Page.}

Conditional processing can be used to include a title or banner page
in the main document when proper precautions are taken.
Importantly, the code in the main file should ensure that the page counter
(as well as other status parameters which are stored in the |.aux| files)
takes the same value after the conditional processing.
Otherwise the page numbers may take divergent values
depending on which part is compiled.

For example, a title page could be declared by:
%
\begin{center}
\begin{tabular}{l}
|\ifchilddoc\||else|\\
|\addtocounter{page}{-1}|\\
\textit{code for title page}\\
|\newpage|\\
|\||fi|
\end{tabular}
\end{center}
%
A banner page for the child documents can be generated by:
%
\begin{center}
\begin{tabular}{l}
|\ifchilddoc|\\
|\addtocounter{page}{-1}|\\
\textit{code for banner page}\\
|\newpage|\\
|\||fi|
\end{tabular}
\end{center}
%
Here one could write a message such as:
\begin{center}
|This is the part \childdocname{} of \childdocjob{}.|
\end{center}

%%%%%%%%%%%%%%%%%%%%%%%%%%%%%%%%%%%%%%%%%%%%%%%%%%%%%%%%%%%%%%%%%%%%%%%%%%%%%%%%
\subsection{Flags}
\label{sec:flags}

The package makes it easy to generate different versions
of the main or child documents.
To this end compilation flags can be defined
and assigned different default values.
They will be particularly useful in conjunction
with the forwarding mechanism described in \secref{sec:forward}.

For example, it may be useful to have a flag |\version|
which can be set to |draft| or |final|.
The document source will contain some conditional code
depending on the value of |\version|.
Suppose further, the flag should default to |final| for the main file
and to |draft| for child files
which is a natural assignment for editing the document.
This is achieved by placing the following code
in the preamble of the main document
(below the |\childdocmain| directive):
%
\begin{center}
\begin{tabular}{l}
|\ifchilddoc|\\
|\providecommand{\version}{draft}|\\
|\||else|\\
|\providecommand{\version}{final}|\\
|\||fi|
\end{tabular}
\end{center}
%
The definition by |\providecommand| makes sure
that previous definitions are not overwritten.
Further statements |\providecommand{\version}{...}|
can thus be added before the above code to override it.

For the main file, one might add a line
(between |\childdocmain| and the above block)
%
\begin{center}
|%\ifchilddoc\||else\providecommand{\version}{draft}\||fi|
\end{center}
%
which can be uncommented to produce a draft version.
Likewise one can add a line to the very top of a child file
(above the |\childdocof{|\textit{main}|}| directive)
%
\begin{center}
|%\providecommand{\version}{final}|
\end{center}
%
which can be uncommented to produce the final version of this child document.

%%%%%%%%%%%%%%%%%%%%%%%%%%%%%%%%%%%%%%%%%%%%%%%%%%%%%%%%%%%%%%%%%%%%%%%%%%%%%%%%
\subsection{Forwarding}
\label{sec:forward}

Different versions of the main or child documents
using compilation flags as described in \secref{sec:flags}
can be (permanently) stored in different files
for convenient compilation, viewing and distribution.
To this end, the package defines a command
to pass on compilation to a different file:

%%%%%%%%%%%%%%%%%%%%%%%%%%%%%%%%%%%%%%%%
\DescribeMacro{\childdocforward}
The command |\childdocforward| redirects processing to
another source file:
%
\begin{center}
\begin{tabular}{l}
|\input{childdoc.def}|\\
|\childdocforward[|\textit{main}|]{|\textit{dest}|}|\\
\end{tabular}
\end{center}
%
The argument \textit{dest} is the destination file
(without extension).
It should be the main file or one of the child files.
Note that further \textsf{childdoc} directives
such as |\childdocof| and |\childdocforward|
in the indicated file will be processed in this form.
The optional argument \textit{main}
passes on directly to the main file \textit{main}
while pretending to compile the child \textit{dest}.
This form behaves as if \textit{dest}
issues |\childdocof{|\textit{main}|}| right away,
and no further \textsf{childdoc} directives will be processed.

%%%%%%%%%%%%%%%%%%%%%%%%%%%%%%%%%%%%%%%%
\DescribeMacro{\...prefix}
In the alternative form |\childdocforwardprefix|,
%
\begin{center}
\begin{tabular}{l}
|\input{childdoc.def}|\\
|\childdocforwardprefix[|\textit{main}|]{|\textit{prefix}|}{|\textit{dest}|}|
\end{tabular}
\end{center}
%
the destination file is determined by a pattern
depending on the current file:
To make this work, the current file must be called
`{\textit{prefix}\hspace{0.2em}\textit{suffix}}'
with \textit{prefix} matching precisely the argument.
Processing is then passed on to the file
`{\textit{dest}\hspace{0.2em}\textit{suffix}}'.
Surely, the same effect is achieved by
directly specifying the
argument `{\textit{dest}\hspace{0.2em}\textit{suffix}}'
in the first form.
However, that requires to set up a different file
for each child. With the alternative form of the command
all these files can have exactly the same content
which simplifies setting them up and maintaining them.

For example, the following file |draft.tex|
with a compilation flag |\version| as described in \secref{sec:flags}
compiles the main document as a draft:
%
\begin{center}
\begin{tabular}{l}
|\def\version{draft}|\\
|\input{childdoc.def}|\\
|\childdocforward{|\textit{main}|}|
\end{tabular}
\end{center}
%
Likewise, the following files |final|\textit{nn}|.tex|
compile the final version of the child document
|child|\textit{nn}|.tex|:
%
\begin{center}
\begin{tabular}{l}
|\def\version{final}|\\
|\input{childdoc.def}|\\
|\childdocforwardprefix{final}{child}|
\end{tabular}
\end{center}
%

Note that when several versions of a main file and/or of each child file
are to be generated, it may be convenient to set up a |Makefile| or
shell script to automatise the process.

%%%%%%%%%%%%%%%%%%%%%%%%%%%%%%%%%%%%%%%%%%%%%%%%%%%%%%%%%%%%%%%%%%%%%%%%%%%%%%%%
\subsection{Command Line Processing}
\label{sec:commandline}

The effect of redirection files can also be achieved by invoking
the \LaTeX{} compiler with a more elaborate command line.
Most conveniently this should be done as part
of a shell script or a |Makefile|.

When using \textsf{childdoc} in the main file, the following
command lines effectively perform a redirection
(note that depending on the shell being used,
backslashes may have to be doubled: `|\|' $\to$ `|\\|'):
%
\begin{center}
|... -jobname "|\textit{target}|" |\\|"|[\textit{flags}]%
|\input{childdoc.def}\childdocforward[|\textit{main}|]{|\textit{dest}|}"|
\end{center}
%
Here \textit{target} is the name of the output file,
\textit{main} is the name of the main file
and \textit{dest} is the name of the main or child file to be processed
(all filenames without extensions).
The optional argument \textit{main} can be omitted
if \textit{main} matches \textit{dest}.
Optionally, compilation \textit{flags} can be defined via |\def| commands.
This command line makes the \TeX{} engine believe
it is compiling the file \textit{target}
whose content is specified as the latter parameter.
The provided code then forwards the processing to
\textit{main} or \textit{dest} as described in \secref{sec:forward}.

%%%%%%%%%%%%%%%%%%%%%%%%%%%%%%%%%%%%%%%%%%%%%%%%%%%%%%%%%%%%%%%%%%%%%%%%%%%%%%%%
\subsection{Include by Input}
\label{sec:input}

Including child documents by |\include| has some restrictions by design.
Most notably, the content of a child document always occupies
its own set of pages; pages cannot be shared between child documents.
Usually, this behaviour makes perfect sense
because each child document contain an essential part of the document.
However, in some situations it may be desirable to compose
a document from a collection of parts
without having mandatory page breaks between then.
For this case, the package
provides a mechanism to include parts
by |\input| which can also be processed individually.
However, by construction this mechanism
requires manual handling of the content to be output.

%%%%%%%%%%%%%%%%%%%%%%%%%%%%%%%%%%%%%%%%
\DescribeMacro{\ifchilddocmanual}
The main file should be prepared as usual, see \secref{sec:include}.
However, the document body must make a distinction
between processing of an individual part and of the main document, e.g.:
%
\begin{center}
\begin{tabular}{l}
|\ifchilddocmanual|\\
|\input{\childdocname}|\\
|\||else|\\
\textit{document body with }|\input{|\textit{part}|}|\\
|\||fi|
\end{tabular}
\end{center}
%
The conditional |\ifchilddocmanual| is true whenever
a part to be included by |\input| is being compiled,
and the name of the part is stored in |\childdocname|.

%%%%%%%%%%%%%%%%%%%%%%%%%%%%%%%%%%%%%%%%
\DescribeMacro{\childdocby}
Each part to be included by |\input| should start with:
%
\begin{center}
\begin{tabular}{l}
|\input{childdoc.def}|\\
|\childdocby{|\textit{main}|}|\\
\end{tabular}
\end{center}
%
The directive |\childdocby| is similar to |\childdocof|
described in \secref{sec:include},
but the subsequent selection of content must be done manually.
To that end, both |\ifchilddoc| and |\ifchilddocmanual|
will be true upon processing of a part,
and the name of the part is stored in |\childdocname|.
Note that |\jobname| will be set to the filename of the current part
so that each part receives an individual |.aux| file
that does not interfere with the |.aux| file(s) of the main document.
This behaviour can be altered by the alternative form
|\childdocby[*]{|\textit{main}|}| (with a non-empty optional argument)
which uses the |.aux| file of the main document
by setting |\jobname| to \textit{main}.

%%%%%%%%%%%%%%%%%%%%%%%%%%%%%%%%%%%%%%%%%%%%%%%%%%%%%%%%%%%%%%%%%%%%%%%%%%%%%%%%
\subsection{Driver Development}
\label{sec:driver}

The \textsf{childdoc} mechanism can also be use for the development
of definition files such as \LaTeX{} styles or classes.
This case differs from the above setup with multiple parts
included by |\include| in that no |\includeonly| should be invoked.
This can be achieved by starting the include file
(before |\ProvidesPackage|) with:
%
\begin{center}
\begin{tabular}{l}
|\input{childdoc.def}|\\
|\childdocforward{|\textit{main}|}|\\
\end{tabular}
\end{center}
%
or alternatively with:
%
\begin{center}
\begin{tabular}{l}
|\input{childdoc.def}|\\
|\childdocby{|\textit{main}|}|\\
\end{tabular}
\end{center}
%
Both forms have slightly different effects as described above.
The main file is prepared as usual, see \secref{sec:include}.

%%%%%%%%%%%%%%%%%%%%%%%%%%%%%%%%%%%%%%%%%%%%%%%%%%%%%%%%%%%%%%%%%%%%%%%%%%%%%%%%
\subsection{Legacy Detection}
\label{sec:detection}

The directive |\childdocmain| in the main file can detect
whether the complete document or merely a child is to be compiled
even without using the directive |\childdocof|.
This method is deprecated because it is less robust
and there is no compelling reason to use it;
it is merely provided for backward compatibility
and it may be removed in future versions.

If the detection mechanism is to be used,
it is mandatory to correctly specify
the filename of the main file as the argument of |\childdocmain|:
%
\begin{center}
\begin{tabular}{l}
|\input{childdoc.def}|\\
|\childdocmain{|\textit{main}|}|\\
\end{tabular}
\end{center}
%
If |\jobname| does not match the argument \textit{main} of |\childdocmain|,
it is assumed that |\jobname| points to the child file to be compiled.
When using |\childdocmain| with the main file specified as argument,
it suffices to start a child file
with just |\input{|\textit{main}|}|
without loading of the package and using |\childdocof|.
If instead all processing is done
with the appropriate \textsf{childdoc} directives,
the argument of \textit{main} of |\childdocmain| can be empty.

An alternative version of the command line processing described
in \secref{sec:commandline} using the detection mechanism reads:
%
\begin{center}
|... -jobname "|\textit{target}|" "|[\textit{flags}]%
[|\def\jobname{|\textit{dest}|}|]|\input{|\textit{main}|}"|
\end{center}

%%%%%%%%%%%%%%%%%%%%%%%%%%%%%%%%%%%%%%%%%%%%%%%%%%%%%%%%%%%%%%%%%%%%%%%%%%%%%%%%
\subsection{Manual Code}
\label{sec:manual}

In case one cannot be certain whether the definitions file |childdoc.def|
is installed on the target \TeX{} distribution
and one prefers not to ship it,
it is conceivable to paste a few relevant commands into the sources.

To that end, drop all statements |\input{childdoc.def}|
and perform the replacements as outlined below.
Instead of |\childdocmain{|\textit{main}|}| add the following code
to the top of the main file:
%
\begin{center}
\begin{tabular}{l}
|\||ifdefined\childdocname\endinput\||fi\newif\ifchilddoc|\\
|\edef\childdocname{\scantokens\expandafter{\jobname\noexpand}}|\\
|\def\childdocmain{|\textit{main}|}\||ifx\childdocmain\childdocname\||else|\\
|\childdoctrue\includeonly{\childdocname}\let\jobname\childdocmain\||fi|\\
\end{tabular}
\end{center}
%
Instead of |\childdocof{|\textit{main}|}| just include the main file
at the top of each child file:
%
\begin{center}
|\input{|\textit{main}|}|
\end{center}
%
A simple redirection |\childdocforward{|\textit{dest}|}| is achieved by:
%
\begin{center}
|\def\jobname{|\textit{dest}|}\input{\jobname}|
\end{center}
%
The redirection with prefix
|\childdocforwardprefix[|\textit{prefix}|]{|\textit{dest}|}|
is accomplished by:
%
\begin{center}
\begin{tabular}{l}
|{\edef\jobname{\scantokens\expandafter{\jobname\noexpand}}|\\
|\def\redirectjob |\textit{prefix}|#1~~~{\gdef\jobname{|\textit{dest}|#1}}|\\
|\expandafter\redirectjob\jobname~~~}\input{\jobname}|
\end{tabular}
\end{center}

In an alternative approach,
child documents can be compiled by a specific command line
without additional code or specific definitions:
%
\begin{center}
|... -jobname "|\textit{target}|" "|[\textit{flags}]%
|\includeonly{|\textit{dest}|}\input{|\textit{main}|}"|
\end{center}
%

%%%%%%%%%%%%%%%%%%%%%%%%%%%%%%%%%%%%%%%%%%%%%%%%%%%%%%%%%%%%%%%%%%%%%%%%%%%%%%%%
%%%%%%%%%%%%%%%%%%%%%%%%%%%%%%%%%%%%%%%%%%%%%%%%%%%%%%%%%%%%%%%%%%%%%%%%%%%%%%%%
\section{Information}

%%%%%%%%%%%%%%%%%%%%%%%%%%%%%%%%%%%%%%%%%%%%%%%%%%%%%%%%%%%%%%%%%%%%%%%%%%%%%%%%
\subsection{Copyright}

Copyright \copyright{} 2017--2018 Niklas Beisert

This work may be distributed and/or modified under the
conditions of the \LaTeX{} Project Public License, either version 1.3
of this license or (at your option) any later version.
The latest version of this license is in
  \url{http://www.latex-project.org/lppl.txt}
and version 1.3 or later is part of all distributions of \LaTeX{}
version 2005/12/01 or later.

This work has the LPPL maintenance status `maintained'.

The Current Maintainer of this work is Niklas Beisert.

This work consists of the files |README.txt|, |childdoc.ins| and |childdoc.dtx|
as well as the derived files |childdoc.def|, |cdocsamp.tex|
with |cdocsch1.tex|, |cdocsch2.tex|, |cdocspt3.tex|, |cdocspt4.tex|,
|cdocsdrf.tex|, |cdocsfn1.tex|, |cdocsfn2.tex|
as well as |childdoc.pdf|.

%%%%%%%%%%%%%%%%%%%%%%%%%%%%%%%%%%%%%%%%%%%%%%%%%%%%%%%%%%%%%%%%%%%%%%%%%%%%%%%%
\subsection{Files and Installation}

The package consists of the files:
%
\begin{center}
\begin{tabular}{ll}
    |README.txt|   & readme file \\
    |childdoc.ins| & installation file \\
    |childdoc.dtx| & source file \\
    |childdoc.def| & definition file \\
    |cdocsamp.tex| & sample main file \\
    |cdocsch1.tex| & sample include file \\
    |cdocsch2.tex| & sample include file \\
    |cdocspt3.tex| & sample part file \\
    |cdocspt4.tex| & sample part file \\
    |cdocsdrf.tex| & sample redirection file \\
    |cdocsfn1.tex| & sample redirection file \\
    |cdocsfn2.tex| & sample redirection file \\
    |childdoc.pdf| & manual
\end{tabular}
\end{center}
%
The distribution consists of the files
|README.txt|, |childdoc.ins| and |childdoc.dtx|.
%
\begin{itemize}
\item
Run (pdf)\LaTeX{} on |childdoc.dtx|
to compile the manual |childdoc.pdf| (this file).
\item
Run \LaTeX{} on |childdoc.ins| to create the definitions file |childdoc.def|
and the sample |cdocsamp.tex| with include files
|cdocsch1.tex|, |cdocsch2.tex|, |cdocspt3.tex|, |cdocspt4.tex|,
|cdocsdrf.tex|, |cdocsfn1.tex|, |cdocsfn2.tex|.
Then copy the file |childdoc.def| to an appropriate directory of your \LaTeX{}
distribution, e.g.\ \textit{texmf-root}|/tex/latex/childdoc|.
\end{itemize}

%%%%%%%%%%%%%%%%%%%%%%%%%%%%%%%%%%%%%%%%%%%%%%%%%%%%%%%%%%%%%%%%%%%%%%%%%%%%%%%%
\subsection{Related CTAN Packages}

There are several other packages which offer a similar functionality:
%
\begin{itemize}
\item
The packages
\href{http://ctan.org/pkg/docmute}{\textsf{docmute}},
\href{http://ctan.org/pkg/includex}{\textsf{includex}} and
\href{http://ctan.org/pkg/standalone}{\textsf{standalone}}
provide commands to include only the document body of
a child file thus allowing both files to be compiled individually.
\item
The packages \href{http://ctan.org/pkg/subdocs}{\textsf{subdocs}}
and \href{http://ctan.org/pkg/subfiles}{\textsf{subfiles}}
provide structures in which the main and child documents can be
encapsulated and allowing them to be compiled individually.
The inclusion mechanism is different from the conventional |\include|.
\item
The package \href{http://ctan.org/pkg/combine}{\textsf{combine}}
is an elaborate solution to combine several documents into one.
\end{itemize}
%
See also the CTAN topic \href{http://ctan.org/topic/subdocs}{\textsf{subdocs}}
for further related packages.
The present package differs from the above solutions in that
a document structure constructed with the conventional |\include| mechanism
just needs two extra commands at the top of every file
such that all constituent files can be compiled individually.

%%%%%%%%%%%%%%%%%%%%%%%%%%%%%%%%%%%%%%%%%%%%%%%%%%%%%%%%%%%%%%%%%%%%%%%%%%%%%%%%
%\subsection{Feature Suggestions}
%
%The following is a list of features which may be useful for future
%versions of this package:
%%
%\begin{itemize}
%\item
%\ldots
%\end{itemize}

%%%%%%%%%%%%%%%%%%%%%%%%%%%%%%%%%%%%%%%%%%%%%%%%%%%%%%%%%%%%%%%%%%%%%%%%%%%%%%%%
\subsection{Revision History}

%%%%%%%%%%%%%%%%%%%%%%%%%%%%%%%%%%%%%%%%
\paragraph{v2.0:} 2018/12/30

\begin{itemize}
\item
immediate forward processing
\item
added |\childdocby| mechanism
\item
manual restructured
\end{itemize}

%%%%%%%%%%%%%%%%%%%%%%%%%%%%%%%%%%%%%%%%
\paragraph{v1.6:} 2018/01/17

\begin{itemize}
\item
application for development of include files
\item
corrections to manual
\end{itemize}

%%%%%%%%%%%%%%%%%%%%%%%%%%%%%%%%%%%%%%%%
\paragraph{v1.5:} 2017/05/21

\begin{itemize}
\item
more complete structuring introduced
\item
|\childdocof| introduced
\item
|\childdoc| renamed to |\childdocmain|
\item
|\childredirect| renamed to |\childdocforward| and |\childdocforwardprefix|
and functionality expanded
\end{itemize}

%%%%%%%%%%%%%%%%%%%%%%%%%%%%%%%%%%%%%%%%
\paragraph{v1.0:} 2017/04/27

\begin{itemize}
\item
manual and install package
\item
first version published on CTAN
\end{itemize}

%%%%%%%%%%%%%%%%%%%%%%%%%%%%%%%%%%%%%%%%
\paragraph{v0.6:} 2017/04/26

\begin{itemize}
\item
redirection mechanism added
\end{itemize}

%%%%%%%%%%%%%%%%%%%%%%%%%%%%%%%%%%%%%%%%
\paragraph{v0.5:} 2017/04/26

\begin{itemize}
\item
functionality in definition file
\end{itemize}


%%%%%%%%%%%%%%%%%%%%%%%%%%%%%%%%%%%%%%%%%%%%%%%%%%%%%%%%%%%%%%%%%%%%%%%%%%%%%%%%
%%%%%%%%%%%%%%%%%%%%%%%%%%%%%%%%%%%%%%%%%%%%%%%%%%%%%%%%%%%%%%%%%%%%%%%%%%%%%%%%
%%%%%%%%%%%%%%%%%%%%%%%%%%%%%%%%%%%%%%%%%%%%%%%%%%%%%%%%%%%%%%%%%%%%%%%%%%%%%%%%
\appendix

\settowidth\MacroIndent{\rmfamily\scriptsize 000\ }

 \DocInput{childdoc.dtx}

\end{document}
%</driver>
% \fi
%
% %%%%%%%%%%%%%%%%%%%%%%%%%%%%%%%%%%%%%%%%%%%%%%%%%%%%%%%%%%%%%%%%%%%%%%%%%%%%%%
% %%%%%%%%%%%%%%%%%%%%%%%%%%%%%%%%%%%%%%%%%%%%%%%%%%%%%%%%%%%%%%%%%%%%%%%%%%%%%%
% \section{Sample}
%\iffalse
%<*samplemain>
%\fi
%
% The following presents a sample document
% with two chapters, two parts, a title page,
% a compile flag as well as three forwarding files to set the flag.
% It consists of eight |.tex| files:
% \begin{center}
% \begin{tabular}{ll}
% |cdocsamp.tex|&main file\\
% |cdocsch1.tex|&include file for chapter 1\\
% |cdocsch2.tex|&include file for chapter 2\\
% |cdocspt3.tex|&include file for part 3\\
% |cdocspt4.tex|&include file for part 4\\
% |cdocsdrf.tex|&forwarding file for main file in draft mode\\
% |cdocsfi1.tex|&forwarding file for final version of chapter 1\\
% |cdocsfi2.tex|&forwarding file for final version of chapter 2\\
% \end{tabular}
% \end{center}
% Each of the eight files can be compiled directly by the \LaTeX{} compiler.
%
% %%%%%%%%%%%%%%%%%%%%%%%%%%%%%%%%%%%%%%
% \paragraph{Main File.}
%
% The main file is called |cdocsamp.tex|.
%
% Load the \textsf{childdoc} definitions and
% declare the filename for the main document:
%    \begin{macrocode}
\input{childdoc.def}
\childdocmain{}
%    \end{macrocode}

% Optional override for |\version| flag:
%    \begin{macrocode}
%%\ifchilddoc\else\providecommand{\version}{draft}\fi
%    \end{macrocode}

% Define the default values for the |\version| flag
% (|final| for the main file and |draft| for childs):
%    \begin{macrocode}
\ifchilddoc
\providecommand{\version}{draft}
\else
\providecommand{\version}{final}
\fi
%    \end{macrocode}

% Load the standard document class:
%    \begin{macrocode}
\documentclass[12pt]{article}
%    \end{macrocode}

% Start the document body:
%    \begin{macrocode}
\begin{document}
%    \end{macrocode}

% Declare a title page.
% Print title, part of document being processed and version flag:
%    \begin{macrocode}
\addtocounter{page}{-1}
\begin{center}
{\LARGE\bfseries{}childdoc example\par}
\vspace{1cm}
\ifchilddoc
\ifchilddocmanual part\else chapter\fi:
`\childdocname' of `\childdocjob'\par
\else
main document: `\childdocjob'\par
\fi
version: \version\par
\end{center}
\newpage
%    \end{macrocode}

% Manually include selected file,
% otherwise process as usual:
%    \begin{macrocode}
\ifchilddocmanual
\section*{part `\childdocname'}
\input{\childdocname}
\else
%    \end{macrocode}

% Include the two chapters:
%    \begin{macrocode}
\include{cdocsch1}
\include{cdocsch2}
%    \end{macrocode}

% Include the two parts unless only chapters should be displayed:
%    \begin{macrocode}
\ifchilddoc\else
\section{part three}
\input{cdocspt3}
\section{part four}
\input{cdocspt4}
\fi
%    \end{macrocode}

% Process as usual until here:
%    \begin{macrocode}
\fi
%    \end{macrocode}

% End of document body:
%    \begin{macrocode}
\end{document}
%    \end{macrocode}
%\iffalse
%</samplemain>
%\fi
%
% %%%%%%%%%%%%%%%%%%%%%%%%%%%%%%%%%%%%%%
% \paragraph{Chapter Include Files.}
%
% The include files are called |cdocsch1.tex| and |cdocsch2.tex|.
%
%\iffalse
%<*samplechap1|samplechap2>
%\fi

% Optional override for |\version| flag:
%    \begin{macrocode}
%%\providecommand{\version}{final}
%    \end{macrocode}

% Include the main document:
%    \begin{macrocode}
\input{childdoc.def}
\childdocof{cdocsamp}
%    \end{macrocode}

%\iffalse
%</samplechap1|samplechap2>
%\fi
%
%\iffalse
%<*samplechap1>
%\fi
% Some text for chapter 1:
%    \begin{macrocode}
\section{one}
some text in chapter one
%    \end{macrocode}

%\iffalse
%</samplechap1>
%\fi
% Some text for chapter 2:
%\iffalse
%<*samplechap2>
%\fi
%    \begin{macrocode}
\section{two}
more text in chapter two
%    \end{macrocode}

%\iffalse
%</samplechap2>
%\fi
%
% %%%%%%%%%%%%%%%%%%%%%%%%%%%%%%%%%%%%%%
% \paragraph{Part Include Files.}
%
% The include files are called |cdocspt3.tex| and |cdocspt4.tex|.
%
%\iffalse
%<*samplepart3|samplepart4>
%\fi

% Optional override for |\version| flag:
%    \begin{macrocode}
%%\providecommand{\version}{final}
%    \end{macrocode}

% Include the main document:
%    \begin{macrocode}
\input{childdoc.def}
\childdocby{cdocsamp}
%    \end{macrocode}

%\iffalse
%</samplepart3|samplepart4>
%\fi
%
%\iffalse
%<*samplepart3>
%\fi
% Some text for part 3:
%    \begin{macrocode}
some text in part three
%    \end{macrocode}

%\iffalse
%</samplepart3>
%\fi
% Some text for part 4:
%\iffalse
%<*samplepart4>
%\fi
%    \begin{macrocode}
more text in part four
%    \end{macrocode}

%\iffalse
%</samplepart4>
%\fi
%
% %%%%%%%%%%%%%%%%%%%%%%%%%%%%%%%%%%%%%%
% \paragraph{Forwarding for a Complete Draft.}
%
% The following forwarding file |cdocsdrf.tex|
% compiles the main document in draft mode:
%\iffalse
%<*sampledraft>
%\fi
%    \begin{macrocode}
\def\version{draft}
\input{childdoc.def}
\childdocforward{cdocsamp}
%    \end{macrocode}

%\iffalse
%</sampledraft>
%\fi
%
% %%%%%%%%%%%%%%%%%%%%%%%%%%%%%%%%%%%%%%
% \paragraph{Forwarding for Final Version of the Chapters.}
%
% The following forwarding files |cdocsfn1.tex| and |cdocsfn2.tex|
% (with identical content)
% compile the final versions of the child documents
% |cdocsch1.tex| and |cdocsch2.tex|, respectively:
%\iffalse
%<*samplefinal>
%\fi
%    \begin{macrocode}
\def\version{final}
\input{childdoc.def}
\childdocforwardprefix[cdocsamp]{cdocsfn}{cdocsch}
%    \end{macrocode}

%\iffalse
%</samplefinal>
%\fi
%
% %%%%%%%%%%%%%%%%%%%%%%%%%%%%%%%%%%%%%%
% \paragraph{Command Line Processing.}
%
% The following three command lines generate the output files
% |cdocscld|, |cdocscl1| and |cdocscl2|
% which should be identical to
% |cdocsdrf|, |cdocsch1| and |cdocsfn2|, respectively:
% \begin{center}
% \begin{tabular}{l}
% |latex -jobname cdocscld \|\\
% |  "\def\version{draft}\input{childdoc.def}\childdocforward{cdocsamp}"|\\
% |latex -jobname cdocscl1 \|\\
% |  "\input{childdoc.def}\childdocforward[cdocsamp]{cdocsch1}"|\\
% |latex -jobname cdocscl2 \|\\
% |  "\def\version{final}\input{childdoc.def}\childdocforward{cdocsch2}"|
% \end{tabular}
% \end{center}
% Note that the trailing backslash on each first line
% merely continues the input to the second line
% (for convenient cut ant paste).
% Furthermore, the command |latex| can be replaced by any
% of its alternative versions such as |pdflatex|.
%
% %%%%%%%%%%%%%%%%%%%%%%%%%%%%%%%%%%%%%%%%%%%%%%%%%%%%%%%%%%%%%%%%%%%%%%%%%%%%%%
% %%%%%%%%%%%%%%%%%%%%%%%%%%%%%%%%%%%%%%%%%%%%%%%%%%%%%%%%%%%%%%%%%%%%%%%%%%%%%%
% \section{Implementation}
%\iffalse
%<*package>
%\fi
%
% This section describes the definitions file |childdoc.def|.

% The definitions cannot be loaded using |\usepackage| or |\RequirePackage|
% which has a mechanism to prevent loading a style file more than once.
% When loading the definitions by means of |\input|
% multiple instances have to be prevented manually:
%\iffalse
%This code needs to be before the `\ProvidesFile' directive
%which is defined at the beginning of this file.
%Therefore it is also placed there and commented out here.
%</package>
%<*discard>
%\fi
%    \begin{macrocode}
\ifdefined\childdocmain\endinput\fi
%    \end{macrocode}
%\iffalse
%</discard>
%<*package>
%\fi
%
% \macro{\ifchilddoc}
% \macro{\ifchilddocmanual}
% The conditional |\ifchilddoc| tells whether a
% child (true) or main (false) document is being compiled.
% The conditional |\ifchilddocmanual| tells whether
% the |\includeonly| mechanism is used (false) or
% the selection of child files must be performed manually (true).
% The definitions initialise to false:
%    \begin{macrocode}
\newif\ifchilddoc
\newif\ifchilddocmanual
%    \end{macrocode}

% \macro{\childdocname}
% \macro{\childdocjob}
% The macro |\childdocname| stores the name of the main document
% to be compiled. The macro |\childdocjob| stores the name of
% the document on which the \LaTeX{} compiler was originally invoked.
% The content of |\jobname| cannot be compared
% to filenames specified in the source due to different catcodes.
% The following code rescans |\jobname|, stores the result
% in |\childdocname| and saves a copy in |\childdocjob|:
%    \begin{macrocode}
\edef\childdocname{\scantokens\expandafter{\jobname\noexpand}}
\let\childdocjob\childdocname
%    \end{macrocode}

% \macro{\childdocdisable}
% The macro |\childdocdisable| prevents the main file
% from being processed more than once.
% At this stage, the main document command |\childdocmain|
% is assumed to be called once again where it should do nothing.
% Any subsequent call to it should prevent
% a secondary processing of the main document
% It overwrites the forwarding commands
% |\childdocof| and |\childdocforward|
% with empty macros to prevent further inclusions of the main document:
%    \begin{macrocode}
\newcommand{\childdocdisable}
{
  \renewcommand{\childdocmain}[1]{\renewcommand{\childdocmain}[1]{\endinput}}
  \renewcommand{\childdocof}[1]{}
  \renewcommand{\childdocby}[2][]{}
  \renewcommand{\childdocforward}[2][]{}
  \renewcommand{\childdocdisable}{}
}
%    \end{macrocode}

% \macro{\childdocmain}
% The macro |\childdocmain| is to be called at the top of the main file
% with nothing or the main filename (without extension) as argument.
% First, it breaks loops.
% If the argument is not empty and does not match |\childdocname|
% (which is set by the first inclusion of |childdoc.def|),
% |\ifchilddoc| is set to true, |\includeonly| is applied to the child file
% and |\jobname| is set to the main file
% (for proper handling of |.aux| files):
%    \begin{macrocode}
\newcommand{\childdocmain}[1]
{
  \childdocdisable\childdocmain{}
  \if?#1?\else
    \begingroup
      \def\childdoctmp{#1}
      \ifx\childdoctmp\childdocname
        \def\childdoctmp{}
      \else
        \def\childdoctmp
        {
          \childdoctrue
          \includeonly{\childdocname}
          \def\childdocjob{#1}
          \def\jobname{#1}
        }
      \fi
      \expandafter
    \endgroup
    \childdoctmp
  \fi
}
%    \end{macrocode}

% \macro{\childdocof}
% The command |\childdocof| redirects
% compilation to the main file |#1|.
%    \begin{macrocode}
\newcommand{\childdocof}[1]
{
  \childdocdisable
  \childdoctrue
  \includeonly{\childdocname}
  \def\jobname{#1}
  \def\childdocjob{#1}
  \input{#1}
}
%    \end{macrocode}

% \macro{\childdocby}
% The command |\childdocby| ....
%    \begin{macrocode}
\newcommand{\childdocby}[2][]
{
  \childdocdisable
  \childdoctrue
  \childdocmanualtrue
  \if?#1?\else
    \def\jobname{#2}
  \fi
  \def\childdocjob{#2}
  \input{#2}
  \endinput
}
%    \end{macrocode}

% \macro{\childdocforward}
% The command |\childdocforward| redirects
% compilation to the main file or
% (if the optional argument is given) a child file.
% Parameters are set as if the main file
% or a child file starting with |\childdocof| was compiled.
% Then compilation is handed over to the main file:
%    \begin{macrocode}
\newcommand{\childdocforward}[2][]
{
  \begingroup
    \if?#1?
      \def\childdoctmp
      {
        \def\childdocname{#2}
        \def\childdocjob{#2}
        \def\jobname{#2}
        \input{#2}
        \endinput
      }
    \else
      \def\childdoctmp
      {
        \childdocdisable
        \def\childdocname{#2}
        \childdoctrue
        \includeonly{#2}
        \def\childdocjob{#1}
        \def\jobname{#1}
        \input{#1}
        \endinput
      }
    \fi
    \expandafter
  \endgroup
  \childdoctmp
}
%    \end{macrocode}

% \macro{\childdocforwardprefix}
% The command |\childdocforwardprefix| redirects
% compilation to the main or a child file by means of a pattern.
% The prefix |#1| in the current filename is replaced by |#2|
% and the suffix of the current filename is kept
% (it is assumed that the filename does not contain the substring `|~~~|'
% which is used as a delimiter).
% Compilation is handed over to the new file by |\childdocforward|:
%    \begin{macrocode}
\newcommand{\childdocforwardprefix}[3][]
{
  \begingroup
    \def\childdocextract #2##1~~~{\def\childdoctmp{\childdocforward[#1]{#3##1}}}
    \expandafter\childdocextract\childdocname~~~
    \expandafter
  \endgroup
  \childdoctmp
}
%    \end{macrocode}

% \macro{\childdoc}
% The deprecated macro |\childdoc| is a legacy version of |\childdocmain|:
%    \begin{macrocode}
\newcommand{\childdoc}{\childdocmain}
%    \end{macrocode}

% \macro{\childdocredirect}
% The deprecated macro |\childdocredirect| is a legacy version
% of |\childdocforward| and |\childdocforwardprefix|:
%    \begin{macrocode}
\newcommand{\childdocredirect}[2][]
{
  \begingroup
    \if?#1?
      \def\childdoctmp{\childdocforward{#2}}
    \else
      \def\childdoctmp{\childdocforwardprefix{#1}{#2}}
    \fi
    \expandafter
  \endgroup
  \childdoctmp
}
%    \end{macrocode}

%\iffalse
%</package>
%\fi
%
\endinput
\childdocforward{cdocsch2}"|
% \end{tabular}
% \end{center}
% Note that the trailing backslash on each first line
% merely continues the input to the second line
% (for convenient cut ant paste).
% Furthermore, the command |latex| can be replaced by any
% of its alternative versions such as |pdflatex|.
%
% %%%%%%%%%%%%%%%%%%%%%%%%%%%%%%%%%%%%%%%%%%%%%%%%%%%%%%%%%%%%%%%%%%%%%%%%%%%%%%
% %%%%%%%%%%%%%%%%%%%%%%%%%%%%%%%%%%%%%%%%%%%%%%%%%%%%%%%%%%%%%%%%%%%%%%%%%%%%%%
% \section{Implementation}
%\iffalse
%<*package>
%\fi
%
% This section describes the definitions file |childdoc.def|.

% The definitions cannot be loaded using |\usepackage| or |\RequirePackage|
% which has a mechanism to prevent loading a style file more than once.
% When loading the definitions by means of |\input|
% multiple instances have to be prevented manually:
%\iffalse
%This code needs to be before the `\ProvidesFile' directive
%which is defined at the beginning of this file.
%Therefore it is also placed there and commented out here.
%</package>
%<*discard>
%\fi
%    \begin{macrocode}
\ifdefined\childdocmain\endinput\fi
%    \end{macrocode}
%\iffalse
%</discard>
%<*package>
%\fi
%
% \macro{\ifchilddoc}
% \macro{\ifchilddocmanual}
% The conditional |\ifchilddoc| tells whether a
% child (true) or main (false) document is being compiled.
% The conditional |\ifchilddocmanual| tells whether
% the |\includeonly| mechanism is used (false) or
% the selection of child files must be performed manually (true).
% The definitions initialise to false:
%    \begin{macrocode}
\newif\ifchilddoc
\newif\ifchilddocmanual
%    \end{macrocode}

% \macro{\childdocname}
% \macro{\childdocjob}
% The macro |\childdocname| stores the name of the main document
% to be compiled. The macro |\childdocjob| stores the name of
% the document on which the \LaTeX{} compiler was originally invoked.
% The content of |\jobname| cannot be compared
% to filenames specified in the source due to different catcodes.
% The following code rescans |\jobname|, stores the result
% in |\childdocname| and saves a copy in |\childdocjob|:
%    \begin{macrocode}
\edef\childdocname{\scantokens\expandafter{\jobname\noexpand}}
\let\childdocjob\childdocname
%    \end{macrocode}

% \macro{\childdocdisable}
% The macro |\childdocdisable| prevents the main file
% from being processed more than once.
% At this stage, the main document command |\childdocmain|
% is assumed to be called once again where it should do nothing.
% Any subsequent call to it should prevent
% a secondary processing of the main document
% It overwrites the forwarding commands
% |\childdocof| and |\childdocforward|
% with empty macros to prevent further inclusions of the main document:
%    \begin{macrocode}
\newcommand{\childdocdisable}
{
  \renewcommand{\childdocmain}[1]{\renewcommand{\childdocmain}[1]{\endinput}}
  \renewcommand{\childdocof}[1]{}
  \renewcommand{\childdocby}[2][]{}
  \renewcommand{\childdocforward}[2][]{}
  \renewcommand{\childdocdisable}{}
}
%    \end{macrocode}

% \macro{\childdocmain}
% The macro |\childdocmain| is to be called at the top of the main file
% with nothing or the main filename (without extension) as argument.
% First, it breaks loops.
% If the argument is not empty and does not match |\childdocname|
% (which is set by the first inclusion of |childdoc.def|),
% |\ifchilddoc| is set to true, |\includeonly| is applied to the child file
% and |\jobname| is set to the main file
% (for proper handling of |.aux| files):
%    \begin{macrocode}
\newcommand{\childdocmain}[1]
{
  \childdocdisable\childdocmain{}
  \if?#1?\else
    \begingroup
      \def\childdoctmp{#1}
      \ifx\childdoctmp\childdocname
        \def\childdoctmp{}
      \else
        \def\childdoctmp
        {
          \childdoctrue
          \includeonly{\childdocname}
          \def\childdocjob{#1}
          \def\jobname{#1}
        }
      \fi
      \expandafter
    \endgroup
    \childdoctmp
  \fi
}
%    \end{macrocode}

% \macro{\childdocof}
% The command |\childdocof| redirects
% compilation to the main file |#1|.
%    \begin{macrocode}
\newcommand{\childdocof}[1]
{
  \childdocdisable
  \childdoctrue
  \includeonly{\childdocname}
  \def\jobname{#1}
  \def\childdocjob{#1}
  \input{#1}
}
%    \end{macrocode}

% \macro{\childdocby}
% The command |\childdocby| ....
%    \begin{macrocode}
\newcommand{\childdocby}[2][]
{
  \childdocdisable
  \childdoctrue
  \childdocmanualtrue
  \if?#1?\else
    \def\jobname{#2}
  \fi
  \def\childdocjob{#2}
  \input{#2}
  \endinput
}
%    \end{macrocode}

% \macro{\childdocforward}
% The command |\childdocforward| redirects
% compilation to the main file or
% (if the optional argument is given) a child file.
% Parameters are set as if the main file
% or a child file starting with |\childdocof| was compiled.
% Then compilation is handed over to the main file:
%    \begin{macrocode}
\newcommand{\childdocforward}[2][]
{
  \begingroup
    \if?#1?
      \def\childdoctmp
      {
        \def\childdocname{#2}
        \def\childdocjob{#2}
        \def\jobname{#2}
        \input{#2}
        \endinput
      }
    \else
      \def\childdoctmp
      {
        \childdocdisable
        \def\childdocname{#2}
        \childdoctrue
        \includeonly{#2}
        \def\childdocjob{#1}
        \def\jobname{#1}
        \input{#1}
        \endinput
      }
    \fi
    \expandafter
  \endgroup
  \childdoctmp
}
%    \end{macrocode}

% \macro{\childdocforwardprefix}
% The command |\childdocforwardprefix| redirects
% compilation to the main or a child file by means of a pattern.
% The prefix |#1| in the current filename is replaced by |#2|
% and the suffix of the current filename is kept
% (it is assumed that the filename does not contain the substring `|~~~|'
% which is used as a delimiter).
% Compilation is handed over to the new file by |\childdocforward|:
%    \begin{macrocode}
\newcommand{\childdocforwardprefix}[3][]
{
  \begingroup
    \def\childdocextract #2##1~~~{\def\childdoctmp{\childdocforward[#1]{#3##1}}}
    \expandafter\childdocextract\childdocname~~~
    \expandafter
  \endgroup
  \childdoctmp
}
%    \end{macrocode}

% \macro{\childdoc}
% The deprecated macro |\childdoc| is a legacy version of |\childdocmain|:
%    \begin{macrocode}
\newcommand{\childdoc}{\childdocmain}
%    \end{macrocode}

% \macro{\childdocredirect}
% The deprecated macro |\childdocredirect| is a legacy version
% of |\childdocforward| and |\childdocforwardprefix|:
%    \begin{macrocode}
\newcommand{\childdocredirect}[2][]
{
  \begingroup
    \if?#1?
      \def\childdoctmp{\childdocforward{#2}}
    \else
      \def\childdoctmp{\childdocforwardprefix{#1}{#2}}
    \fi
    \expandafter
  \endgroup
  \childdoctmp
}
%    \end{macrocode}

%\iffalse
%</package>
%\fi
%
\endinput
|\\
|\childdocmain{|\textit{main}|}|\\
\end{tabular}
\end{center}
%
If |\jobname| does not match the argument \textit{main} of |\childdocmain|,
it is assumed that |\jobname| points to the child file to be compiled.
When using |\childdocmain| with the main file specified as argument,
it suffices to start a child file
with just |\input{|\textit{main}|}|
without loading of the package and using |\childdocof|.
If instead all processing is done
with the appropriate \textsf{childdoc} directives,
the argument of \textit{main} of |\childdocmain| can be empty.

An alternative version of the command line processing described
in \secref{sec:commandline} using the detection mechanism reads:
%
\begin{center}
|... -jobname "|\textit{target}|" "|[\textit{flags}]%
[|\def\jobname{|\textit{dest}|}|]|\input{|\textit{main}|}"|
\end{center}

%%%%%%%%%%%%%%%%%%%%%%%%%%%%%%%%%%%%%%%%%%%%%%%%%%%%%%%%%%%%%%%%%%%%%%%%%%%%%%%%
\subsection{Manual Code}
\label{sec:manual}

In case one cannot be certain whether the definitions file |childdoc.def|
is installed on the target \TeX{} distribution
and one prefers not to ship it,
it is conceivable to paste a few relevant commands into the sources.

To that end, drop all statements |% \iffalse
%
% childdoc.dtx Copyright (C) 2017-2018 Niklas Beisert
%
% This work may be distributed and/or modified under the
% conditions of the LaTeX Project Public License, either version 1.3
% of this license or (at your option) any later version.
% The latest version of this license is in
%   http://www.latex-project.org/lppl.txt
% and version 1.3 or later is part of all distributions of LaTeX
% version 2005/12/01 or later.
%
% This work has the LPPL maintenance status `maintained'.
%
% The Current Maintainer of this work is Niklas Beisert.
%
% This work consists of the files childdoc.dtx and childdoc.ins
% and the derived files childdoc.def and cdocsamp.tex with
% cdocsch1.tex, cdocsch2.tex, cdocsdrf.tex, cdocsfn1.tex, cdocsfn2.tex.
%
%<package>\ifdefined\childdocmain\endinput\fi
%<package>\ProvidesFile{childdoc.def}[2018/12/30 v2.0 child document driver]
%<samplemain>\ProvidesFile{cdocsamp.tex}[2018/12/30 v2.0 sample for childdoc]
%<*driver>
%\ProvidesFile{childdoc.drv}[2018/12/30 v2.0 childdoc reference manual file]
\PassOptionsToClass{10pt,a4paper}{article}
\documentclass{ltxdoc}

\usepackage[margin=35mm]{geometry}
\usepackage{hyperref}
\usepackage{hyperxmp}
\usepackage[usenames]{color}

\hypersetup{colorlinks=true}
\hypersetup{pdfstartview=FitH}
\hypersetup{pdfpagemode=UseNone}
\hypersetup{pdfsource={}}
\hypersetup{pdflang={en-UK}}
\hypersetup{pdfcopyright={Copyright 2017-2018 Niklas Beisert.
  This work may be distributed and/or modified under the
  conditions of the LaTeX Project Public License, either version 1.3
  of this license or (at your option) any later version.}}
\hypersetup{pdflicenseurl={http://www.latex-project.org/lppl.txt}}
\hypersetup{pdfcontactaddress={ETH Zurich, ITP, HIT K,
  Wolfgang-Pauli-Strasse 27}}
\hypersetup{pdfcontactpostcode={8093}}
\hypersetup{pdfcontactcity={Zurich}}
\hypersetup{pdfcontactcountry={Switzerland}}
\hypersetup{pdfcontactemail={nbeisert@itp.phys.ethz.ch}}
\hypersetup{pdfcontacturl={http://people.phys.ethz.ch/\xmptilde nbeisert/}}

\newcommand{\secref}[1]{\hyperref[#1]{section \ref*{#1}}}

\parskip1ex
\parindent0pt
\let\olditemize\itemize
\def\itemize{\olditemize\parskip0pt}

\begin{document}

\title{The \textsf{childdoc} Package}
\hypersetup{pdftitle={The childdoc Package}}
\author{Niklas Beisert\\[2ex]
  Institut f\"ur Theoretische Physik\\
  Eidgen\"ossische Technische Hochschule Z\"urich\\
  Wolfgang-Pauli-Strasse 27, 8093 Z\"urich, Switzerland\\[1ex]
  \href{mailto:nbeisert@itp.phys.ethz.ch}
  {\texttt{nbeisert@itp.phys.ethz.ch}}}
\hypersetup{pdfauthor={Niklas Beisert}}
\hypersetup{pdfsubject={Manual for the LaTeX2e Package childdoc}}
\date{30 December 2018, \textsf{v2.0}}
\maketitle

\begin{abstract}\noindent
\textsf{childdoc} is a \LaTeXe{} package
that enables the direct compilation
of document sections included by |\include|
to individual files.
\end{abstract}

\begingroup
\parskip0ex
\tableofcontents
\endgroup

%%%%%%%%%%%%%%%%%%%%%%%%%%%%%%%%%%%%%%%%%%%%%%%%%%%%%%%%%%%%%%%%%%%%%%%%%%%%%%%%
%%%%%%%%%%%%%%%%%%%%%%%%%%%%%%%%%%%%%%%%%%%%%%%%%%%%%%%%%%%%%%%%%%%%%%%%%%%%%%%%
\section{Introduction}

\LaTeX{} provides a mechanism to structure a large document (such as a book)
into a main file and several child files (containing the chapters)
using the |\include| command.
This mechanism is beneficial for documents
which span hundreds of pages in order to
make the source file(s) more manageable.
Moreover, compilation can be restricted to
selected child files by means of the |\includeonly| command.
The latter feature can be used to reduce the compilation time while editing
(this was significantly more useful in the earlier days of \LaTeX{})
or to generate a smaller document which is easier to navigate.
Another application of |\includeonly| is to generate
documents consisting of selected parts of the complete document.

However, there are a few drawbacks of the plain |\include| mechanism:
\begin{itemize}
\item
The child files cannot be compiled on their own,
they can only be compiled via the main file.
A naive editing environment
(such as a text editor with an option
to have the current file processed by \LaTeX)
may require one to switch to the main file before compiling;
attempting to compile the child file produces errors.
\item
The main file must be modified (each time)
to adjust the |\includeonly| command
to the present needs. This easily leaves the main file in a messy state.
\item
The generated document will always carry the filename
of the main document. This is inconvenient if
several child files are to be compiled and
to be kept for distribution.
\end{itemize}

The present package provides a simple interface
to make child files individually compilable by \LaTeX{}.
Compiling a child file then has the same effect as compiling
the main file with an |\includeonly| command
to select the appropriate child.
Moreover the generated document will carry the name of the child
rather than the main file.
This resolves all three above issues.

This feature is meant to make the editing of books,
thesis documents and lecture notes somewhat more convenient.
However, the package can also be used efficiently for
composing a series of documents (such as exercise sheets)
which are typically distributed individually.
It then assists the author in generating the individual documents
(potentially in different versions)
as well as a document containing the collected series.
Another application is in developing style files
or other kinds of included material
where compilation of the style file could redirect
to a sample or test file.

%%%%%%%%%%%%%%%%%%%%%%%%%%%%%%%%%%%%%%%%%%%%%%%%%%%%%%%%%%%%%%%%%%%%%%%%%%%%%%%%
%%%%%%%%%%%%%%%%%%%%%%%%%%%%%%%%%%%%%%%%%%%%%%%%%%%%%%%%%%%%%%%%%%%%%%%%%%%%%%%%
\section{Usage}

First of all, the package \textsf{childdoc} is \emph{not} a standard
\LaTeXe{} |.sty| style file! Therefore it needs to be invoked in
a non-standard way.

%%%%%%%%%%%%%%%%%%%%%%%%%%%%%%%%%%%%%%%%%%%%%%%%%%%%%%%%%%%%%%%%%%%%%%%%%%%%%%%%
\subsection{Included Files}
\label{sec:include}

%%%%%%%%%%%%%%%%%%%%%%%%%%%%%%%%%%%%%%%%
\DescribeMacro{\childdocmain}
To use the package, add the commands
\begin{center}
\begin{tabular}{l}
|% \iffalse
%
% childdoc.dtx Copyright (C) 2017-2018 Niklas Beisert
%
% This work may be distributed and/or modified under the
% conditions of the LaTeX Project Public License, either version 1.3
% of this license or (at your option) any later version.
% The latest version of this license is in
%   http://www.latex-project.org/lppl.txt
% and version 1.3 or later is part of all distributions of LaTeX
% version 2005/12/01 or later.
%
% This work has the LPPL maintenance status `maintained'.
%
% The Current Maintainer of this work is Niklas Beisert.
%
% This work consists of the files childdoc.dtx and childdoc.ins
% and the derived files childdoc.def and cdocsamp.tex with
% cdocsch1.tex, cdocsch2.tex, cdocsdrf.tex, cdocsfn1.tex, cdocsfn2.tex.
%
%<package>\ifdefined\childdocmain\endinput\fi
%<package>\ProvidesFile{childdoc.def}[2018/12/30 v2.0 child document driver]
%<samplemain>\ProvidesFile{cdocsamp.tex}[2018/12/30 v2.0 sample for childdoc]
%<*driver>
%\ProvidesFile{childdoc.drv}[2018/12/30 v2.0 childdoc reference manual file]
\PassOptionsToClass{10pt,a4paper}{article}
\documentclass{ltxdoc}

\usepackage[margin=35mm]{geometry}
\usepackage{hyperref}
\usepackage{hyperxmp}
\usepackage[usenames]{color}

\hypersetup{colorlinks=true}
\hypersetup{pdfstartview=FitH}
\hypersetup{pdfpagemode=UseNone}
\hypersetup{pdfsource={}}
\hypersetup{pdflang={en-UK}}
\hypersetup{pdfcopyright={Copyright 2017-2018 Niklas Beisert.
  This work may be distributed and/or modified under the
  conditions of the LaTeX Project Public License, either version 1.3
  of this license or (at your option) any later version.}}
\hypersetup{pdflicenseurl={http://www.latex-project.org/lppl.txt}}
\hypersetup{pdfcontactaddress={ETH Zurich, ITP, HIT K,
  Wolfgang-Pauli-Strasse 27}}
\hypersetup{pdfcontactpostcode={8093}}
\hypersetup{pdfcontactcity={Zurich}}
\hypersetup{pdfcontactcountry={Switzerland}}
\hypersetup{pdfcontactemail={nbeisert@itp.phys.ethz.ch}}
\hypersetup{pdfcontacturl={http://people.phys.ethz.ch/\xmptilde nbeisert/}}

\newcommand{\secref}[1]{\hyperref[#1]{section \ref*{#1}}}

\parskip1ex
\parindent0pt
\let\olditemize\itemize
\def\itemize{\olditemize\parskip0pt}

\begin{document}

\title{The \textsf{childdoc} Package}
\hypersetup{pdftitle={The childdoc Package}}
\author{Niklas Beisert\\[2ex]
  Institut f\"ur Theoretische Physik\\
  Eidgen\"ossische Technische Hochschule Z\"urich\\
  Wolfgang-Pauli-Strasse 27, 8093 Z\"urich, Switzerland\\[1ex]
  \href{mailto:nbeisert@itp.phys.ethz.ch}
  {\texttt{nbeisert@itp.phys.ethz.ch}}}
\hypersetup{pdfauthor={Niklas Beisert}}
\hypersetup{pdfsubject={Manual for the LaTeX2e Package childdoc}}
\date{30 December 2018, \textsf{v2.0}}
\maketitle

\begin{abstract}\noindent
\textsf{childdoc} is a \LaTeXe{} package
that enables the direct compilation
of document sections included by |\include|
to individual files.
\end{abstract}

\begingroup
\parskip0ex
\tableofcontents
\endgroup

%%%%%%%%%%%%%%%%%%%%%%%%%%%%%%%%%%%%%%%%%%%%%%%%%%%%%%%%%%%%%%%%%%%%%%%%%%%%%%%%
%%%%%%%%%%%%%%%%%%%%%%%%%%%%%%%%%%%%%%%%%%%%%%%%%%%%%%%%%%%%%%%%%%%%%%%%%%%%%%%%
\section{Introduction}

\LaTeX{} provides a mechanism to structure a large document (such as a book)
into a main file and several child files (containing the chapters)
using the |\include| command.
This mechanism is beneficial for documents
which span hundreds of pages in order to
make the source file(s) more manageable.
Moreover, compilation can be restricted to
selected child files by means of the |\includeonly| command.
The latter feature can be used to reduce the compilation time while editing
(this was significantly more useful in the earlier days of \LaTeX{})
or to generate a smaller document which is easier to navigate.
Another application of |\includeonly| is to generate
documents consisting of selected parts of the complete document.

However, there are a few drawbacks of the plain |\include| mechanism:
\begin{itemize}
\item
The child files cannot be compiled on their own,
they can only be compiled via the main file.
A naive editing environment
(such as a text editor with an option
to have the current file processed by \LaTeX)
may require one to switch to the main file before compiling;
attempting to compile the child file produces errors.
\item
The main file must be modified (each time)
to adjust the |\includeonly| command
to the present needs. This easily leaves the main file in a messy state.
\item
The generated document will always carry the filename
of the main document. This is inconvenient if
several child files are to be compiled and
to be kept for distribution.
\end{itemize}

The present package provides a simple interface
to make child files individually compilable by \LaTeX{}.
Compiling a child file then has the same effect as compiling
the main file with an |\includeonly| command
to select the appropriate child.
Moreover the generated document will carry the name of the child
rather than the main file.
This resolves all three above issues.

This feature is meant to make the editing of books,
thesis documents and lecture notes somewhat more convenient.
However, the package can also be used efficiently for
composing a series of documents (such as exercise sheets)
which are typically distributed individually.
It then assists the author in generating the individual documents
(potentially in different versions)
as well as a document containing the collected series.
Another application is in developing style files
or other kinds of included material
where compilation of the style file could redirect
to a sample or test file.

%%%%%%%%%%%%%%%%%%%%%%%%%%%%%%%%%%%%%%%%%%%%%%%%%%%%%%%%%%%%%%%%%%%%%%%%%%%%%%%%
%%%%%%%%%%%%%%%%%%%%%%%%%%%%%%%%%%%%%%%%%%%%%%%%%%%%%%%%%%%%%%%%%%%%%%%%%%%%%%%%
\section{Usage}

First of all, the package \textsf{childdoc} is \emph{not} a standard
\LaTeXe{} |.sty| style file! Therefore it needs to be invoked in
a non-standard way.

%%%%%%%%%%%%%%%%%%%%%%%%%%%%%%%%%%%%%%%%%%%%%%%%%%%%%%%%%%%%%%%%%%%%%%%%%%%%%%%%
\subsection{Included Files}
\label{sec:include}

%%%%%%%%%%%%%%%%%%%%%%%%%%%%%%%%%%%%%%%%
\DescribeMacro{\childdocmain}
To use the package, add the commands
\begin{center}
\begin{tabular}{l}
|\input{childdoc.def}|\\
|\childdocmain{}|\\
\end{tabular}
\end{center}
at the very top of the main \LaTeX{} file,
in particular \emph{before} the |\documentclass| statement!
The argument of |\childdocmain| should be left empty
(but it must be present).

%%%%%%%%%%%%%%%%%%%%%%%%%%%%%%%%%%%%%%%%
\DescribeMacro{\childdocof}
Furthermore, add the commands
\begin{center}
\begin{tabular}{l}
|\input{childdoc.def}|\\
|\childdocof{|\textit{main}|}|\\
\end{tabular}
\end{center}
at the top of every child file \textit{child}
which is included by |\include{|\textit{child}|}|
from within the main file
(or at least for those files to be compiled individually).
The argument \textit{main} must be the filename of the main file.

There are a couple of
considerations in setting up the main and child documents:

%%%%%%%%%%%%%%%%%%%%%%%%%%%%%%%%%%%%%%%%
\paragraph{Restrictions.}

Please note the following restrictions:
\begin{itemize}
\item
|\childdocmain| must be called with one argument \textit{main}
to ensure compatibility with earlier version of the package.
It must either be empty (|\childdocmain{}|)
or precisely match the filename of the main file in which it is specified.
See \secref{sec:detection} for further information.
\item
The filename \textit{main} must be specified without the |.tex| extension.
\item
The filename \textit{main} is case sensitive
(even in case-insensitive file systems)
due to internal string comparison.
\item
The argument \textit{main} should be fully expanded, it cannot be a macro.
\item
Subdirectories and special characters should be avoided in filenames.
\item
The command |\childdocmain{|\textit{main}|}| must be followed by a whitespace.
It should not be followed immediately by another command
or by a comment mark `|%|'.
This is because the \TeX{} parser reads the token immediately following
the argument of |\childdocmain| and puts it
at the beginning of every child section;
however, a white\-space is ignored.
\end{itemize}

%%%%%%%%%%%%%%%%%%%%%%%%%%%%%%%%%%%%%%%%
\paragraph{Content of Main File.}

It is advisable to place all content in the child files included by |\include|.
Any output contained in the main file will appear in all child documents
unless suppressed manually;
it cannot be suppressed automatically by the |\includeonly| directive
and thus should normally be avoided.
A method to include some content in the main file
by means of conditional processing is described in \secref{sec:conditional}.

%%%%%%%%%%%%%%%%%%%%%%%%%%%%%%%%%%%%%%%%
\paragraph{Page Numbering.}

When only a part of the document is compiled,
the appropriate numbering of pages
(as well as other status parameters)
is determined from the |.aux| files.
The latter contain information from previous passes.
However this information needs to propagate through
all intermediate child documents.
Therefore the page numbering in child documents may well
be inconsistent until the complete document is compiled at least once.

A useful (if unconventional) way to always ensure a consistent
page numbering is to restart the numbering in each child document
and denote the pages by `\textit{child}|.|\textit{page}'
where \textit{child} represents the chapter/section number of the child file.
This can be achieved by the command
|\numberwithin{page}{|\textit{child}|}|
of the \textsf{amsmath} package
where \textit{child} can be |chapter| or |section|
depending on the chosen structuring.
Alternatively, one can modify the macro |\thepage| appropriately
and reset the counter |page| at the start of each child file.

%%%%%%%%%%%%%%%%%%%%%%%%%%%%%%%%%%%%%%%%%%%%%%%%%%%%%%%%%%%%%%%%%%%%%%%%%%%%%%%%
\subsection{Conditional Processing}
\label{sec:conditional}

The package provides a mechanism to compile different versions
of a document. To customise the versions further some conditional processing
can come in handy to distinguish which version is being compiled.
The package provides two macros to describe the compilation context:

%%%%%%%%%%%%%%%%%%%%%%%%%%%%%%%%%%%%%%%%
\DescribeMacro{\ifchilddoc}
The conditional |\ifchilddoc| distinguishes between the compilation of
child documents and the main document:
%
\begin{center}
|\ifchilddoc |\textit{child-code}| |[|\||else |\textit{main-code}]| \||fi|
\end{center}

%%%%%%%%%%%%%%%%%%%%%%%%%%%%%%%%%%%%%%%%
\DescribeMacro{\childdocname}
\DescribeMacro{\childdocjob}
The macro |\childdocname| contains the filename (without extension)
of the main or child file being processed.
Note that |\childdocjob| will always contain the name of the main file.

%%%%%%%%%%%%%%%%%%%%%%%%%%%%%%%%%%%%%%%%
\paragraph{Title Page.}

Conditional processing can be used to include a title or banner page
in the main document when proper precautions are taken.
Importantly, the code in the main file should ensure that the page counter
(as well as other status parameters which are stored in the |.aux| files)
takes the same value after the conditional processing.
Otherwise the page numbers may take divergent values
depending on which part is compiled.

For example, a title page could be declared by:
%
\begin{center}
\begin{tabular}{l}
|\ifchilddoc\||else|\\
|\addtocounter{page}{-1}|\\
\textit{code for title page}\\
|\newpage|\\
|\||fi|
\end{tabular}
\end{center}
%
A banner page for the child documents can be generated by:
%
\begin{center}
\begin{tabular}{l}
|\ifchilddoc|\\
|\addtocounter{page}{-1}|\\
\textit{code for banner page}\\
|\newpage|\\
|\||fi|
\end{tabular}
\end{center}
%
Here one could write a message such as:
\begin{center}
|This is the part \childdocname{} of \childdocjob{}.|
\end{center}

%%%%%%%%%%%%%%%%%%%%%%%%%%%%%%%%%%%%%%%%%%%%%%%%%%%%%%%%%%%%%%%%%%%%%%%%%%%%%%%%
\subsection{Flags}
\label{sec:flags}

The package makes it easy to generate different versions
of the main or child documents.
To this end compilation flags can be defined
and assigned different default values.
They will be particularly useful in conjunction
with the forwarding mechanism described in \secref{sec:forward}.

For example, it may be useful to have a flag |\version|
which can be set to |draft| or |final|.
The document source will contain some conditional code
depending on the value of |\version|.
Suppose further, the flag should default to |final| for the main file
and to |draft| for child files
which is a natural assignment for editing the document.
This is achieved by placing the following code
in the preamble of the main document
(below the |\childdocmain| directive):
%
\begin{center}
\begin{tabular}{l}
|\ifchilddoc|\\
|\providecommand{\version}{draft}|\\
|\||else|\\
|\providecommand{\version}{final}|\\
|\||fi|
\end{tabular}
\end{center}
%
The definition by |\providecommand| makes sure
that previous definitions are not overwritten.
Further statements |\providecommand{\version}{...}|
can thus be added before the above code to override it.

For the main file, one might add a line
(between |\childdocmain| and the above block)
%
\begin{center}
|%\ifchilddoc\||else\providecommand{\version}{draft}\||fi|
\end{center}
%
which can be uncommented to produce a draft version.
Likewise one can add a line to the very top of a child file
(above the |\childdocof{|\textit{main}|}| directive)
%
\begin{center}
|%\providecommand{\version}{final}|
\end{center}
%
which can be uncommented to produce the final version of this child document.

%%%%%%%%%%%%%%%%%%%%%%%%%%%%%%%%%%%%%%%%%%%%%%%%%%%%%%%%%%%%%%%%%%%%%%%%%%%%%%%%
\subsection{Forwarding}
\label{sec:forward}

Different versions of the main or child documents
using compilation flags as described in \secref{sec:flags}
can be (permanently) stored in different files
for convenient compilation, viewing and distribution.
To this end, the package defines a command
to pass on compilation to a different file:

%%%%%%%%%%%%%%%%%%%%%%%%%%%%%%%%%%%%%%%%
\DescribeMacro{\childdocforward}
The command |\childdocforward| redirects processing to
another source file:
%
\begin{center}
\begin{tabular}{l}
|\input{childdoc.def}|\\
|\childdocforward[|\textit{main}|]{|\textit{dest}|}|\\
\end{tabular}
\end{center}
%
The argument \textit{dest} is the destination file
(without extension).
It should be the main file or one of the child files.
Note that further \textsf{childdoc} directives
such as |\childdocof| and |\childdocforward|
in the indicated file will be processed in this form.
The optional argument \textit{main}
passes on directly to the main file \textit{main}
while pretending to compile the child \textit{dest}.
This form behaves as if \textit{dest}
issues |\childdocof{|\textit{main}|}| right away,
and no further \textsf{childdoc} directives will be processed.

%%%%%%%%%%%%%%%%%%%%%%%%%%%%%%%%%%%%%%%%
\DescribeMacro{\...prefix}
In the alternative form |\childdocforwardprefix|,
%
\begin{center}
\begin{tabular}{l}
|\input{childdoc.def}|\\
|\childdocforwardprefix[|\textit{main}|]{|\textit{prefix}|}{|\textit{dest}|}|
\end{tabular}
\end{center}
%
the destination file is determined by a pattern
depending on the current file:
To make this work, the current file must be called
`{\textit{prefix}\hspace{0.2em}\textit{suffix}}'
with \textit{prefix} matching precisely the argument.
Processing is then passed on to the file
`{\textit{dest}\hspace{0.2em}\textit{suffix}}'.
Surely, the same effect is achieved by
directly specifying the
argument `{\textit{dest}\hspace{0.2em}\textit{suffix}}'
in the first form.
However, that requires to set up a different file
for each child. With the alternative form of the command
all these files can have exactly the same content
which simplifies setting them up and maintaining them.

For example, the following file |draft.tex|
with a compilation flag |\version| as described in \secref{sec:flags}
compiles the main document as a draft:
%
\begin{center}
\begin{tabular}{l}
|\def\version{draft}|\\
|\input{childdoc.def}|\\
|\childdocforward{|\textit{main}|}|
\end{tabular}
\end{center}
%
Likewise, the following files |final|\textit{nn}|.tex|
compile the final version of the child document
|child|\textit{nn}|.tex|:
%
\begin{center}
\begin{tabular}{l}
|\def\version{final}|\\
|\input{childdoc.def}|\\
|\childdocforwardprefix{final}{child}|
\end{tabular}
\end{center}
%

Note that when several versions of a main file and/or of each child file
are to be generated, it may be convenient to set up a |Makefile| or
shell script to automatise the process.

%%%%%%%%%%%%%%%%%%%%%%%%%%%%%%%%%%%%%%%%%%%%%%%%%%%%%%%%%%%%%%%%%%%%%%%%%%%%%%%%
\subsection{Command Line Processing}
\label{sec:commandline}

The effect of redirection files can also be achieved by invoking
the \LaTeX{} compiler with a more elaborate command line.
Most conveniently this should be done as part
of a shell script or a |Makefile|.

When using \textsf{childdoc} in the main file, the following
command lines effectively perform a redirection
(note that depending on the shell being used,
backslashes may have to be doubled: `|\|' $\to$ `|\\|'):
%
\begin{center}
|... -jobname "|\textit{target}|" |\\|"|[\textit{flags}]%
|\input{childdoc.def}\childdocforward[|\textit{main}|]{|\textit{dest}|}"|
\end{center}
%
Here \textit{target} is the name of the output file,
\textit{main} is the name of the main file
and \textit{dest} is the name of the main or child file to be processed
(all filenames without extensions).
The optional argument \textit{main} can be omitted
if \textit{main} matches \textit{dest}.
Optionally, compilation \textit{flags} can be defined via |\def| commands.
This command line makes the \TeX{} engine believe
it is compiling the file \textit{target}
whose content is specified as the latter parameter.
The provided code then forwards the processing to
\textit{main} or \textit{dest} as described in \secref{sec:forward}.

%%%%%%%%%%%%%%%%%%%%%%%%%%%%%%%%%%%%%%%%%%%%%%%%%%%%%%%%%%%%%%%%%%%%%%%%%%%%%%%%
\subsection{Include by Input}
\label{sec:input}

Including child documents by |\include| has some restrictions by design.
Most notably, the content of a child document always occupies
its own set of pages; pages cannot be shared between child documents.
Usually, this behaviour makes perfect sense
because each child document contain an essential part of the document.
However, in some situations it may be desirable to compose
a document from a collection of parts
without having mandatory page breaks between then.
For this case, the package
provides a mechanism to include parts
by |\input| which can also be processed individually.
However, by construction this mechanism
requires manual handling of the content to be output.

%%%%%%%%%%%%%%%%%%%%%%%%%%%%%%%%%%%%%%%%
\DescribeMacro{\ifchilddocmanual}
The main file should be prepared as usual, see \secref{sec:include}.
However, the document body must make a distinction
between processing of an individual part and of the main document, e.g.:
%
\begin{center}
\begin{tabular}{l}
|\ifchilddocmanual|\\
|\input{\childdocname}|\\
|\||else|\\
\textit{document body with }|\input{|\textit{part}|}|\\
|\||fi|
\end{tabular}
\end{center}
%
The conditional |\ifchilddocmanual| is true whenever
a part to be included by |\input| is being compiled,
and the name of the part is stored in |\childdocname|.

%%%%%%%%%%%%%%%%%%%%%%%%%%%%%%%%%%%%%%%%
\DescribeMacro{\childdocby}
Each part to be included by |\input| should start with:
%
\begin{center}
\begin{tabular}{l}
|\input{childdoc.def}|\\
|\childdocby{|\textit{main}|}|\\
\end{tabular}
\end{center}
%
The directive |\childdocby| is similar to |\childdocof|
described in \secref{sec:include},
but the subsequent selection of content must be done manually.
To that end, both |\ifchilddoc| and |\ifchilddocmanual|
will be true upon processing of a part,
and the name of the part is stored in |\childdocname|.
Note that |\jobname| will be set to the filename of the current part
so that each part receives an individual |.aux| file
that does not interfere with the |.aux| file(s) of the main document.
This behaviour can be altered by the alternative form
|\childdocby[*]{|\textit{main}|}| (with a non-empty optional argument)
which uses the |.aux| file of the main document
by setting |\jobname| to \textit{main}.

%%%%%%%%%%%%%%%%%%%%%%%%%%%%%%%%%%%%%%%%%%%%%%%%%%%%%%%%%%%%%%%%%%%%%%%%%%%%%%%%
\subsection{Driver Development}
\label{sec:driver}

The \textsf{childdoc} mechanism can also be use for the development
of definition files such as \LaTeX{} styles or classes.
This case differs from the above setup with multiple parts
included by |\include| in that no |\includeonly| should be invoked.
This can be achieved by starting the include file
(before |\ProvidesPackage|) with:
%
\begin{center}
\begin{tabular}{l}
|\input{childdoc.def}|\\
|\childdocforward{|\textit{main}|}|\\
\end{tabular}
\end{center}
%
or alternatively with:
%
\begin{center}
\begin{tabular}{l}
|\input{childdoc.def}|\\
|\childdocby{|\textit{main}|}|\\
\end{tabular}
\end{center}
%
Both forms have slightly different effects as described above.
The main file is prepared as usual, see \secref{sec:include}.

%%%%%%%%%%%%%%%%%%%%%%%%%%%%%%%%%%%%%%%%%%%%%%%%%%%%%%%%%%%%%%%%%%%%%%%%%%%%%%%%
\subsection{Legacy Detection}
\label{sec:detection}

The directive |\childdocmain| in the main file can detect
whether the complete document or merely a child is to be compiled
even without using the directive |\childdocof|.
This method is deprecated because it is less robust
and there is no compelling reason to use it;
it is merely provided for backward compatibility
and it may be removed in future versions.

If the detection mechanism is to be used,
it is mandatory to correctly specify
the filename of the main file as the argument of |\childdocmain|:
%
\begin{center}
\begin{tabular}{l}
|\input{childdoc.def}|\\
|\childdocmain{|\textit{main}|}|\\
\end{tabular}
\end{center}
%
If |\jobname| does not match the argument \textit{main} of |\childdocmain|,
it is assumed that |\jobname| points to the child file to be compiled.
When using |\childdocmain| with the main file specified as argument,
it suffices to start a child file
with just |\input{|\textit{main}|}|
without loading of the package and using |\childdocof|.
If instead all processing is done
with the appropriate \textsf{childdoc} directives,
the argument of \textit{main} of |\childdocmain| can be empty.

An alternative version of the command line processing described
in \secref{sec:commandline} using the detection mechanism reads:
%
\begin{center}
|... -jobname "|\textit{target}|" "|[\textit{flags}]%
[|\def\jobname{|\textit{dest}|}|]|\input{|\textit{main}|}"|
\end{center}

%%%%%%%%%%%%%%%%%%%%%%%%%%%%%%%%%%%%%%%%%%%%%%%%%%%%%%%%%%%%%%%%%%%%%%%%%%%%%%%%
\subsection{Manual Code}
\label{sec:manual}

In case one cannot be certain whether the definitions file |childdoc.def|
is installed on the target \TeX{} distribution
and one prefers not to ship it,
it is conceivable to paste a few relevant commands into the sources.

To that end, drop all statements |\input{childdoc.def}|
and perform the replacements as outlined below.
Instead of |\childdocmain{|\textit{main}|}| add the following code
to the top of the main file:
%
\begin{center}
\begin{tabular}{l}
|\||ifdefined\childdocname\endinput\||fi\newif\ifchilddoc|\\
|\edef\childdocname{\scantokens\expandafter{\jobname\noexpand}}|\\
|\def\childdocmain{|\textit{main}|}\||ifx\childdocmain\childdocname\||else|\\
|\childdoctrue\includeonly{\childdocname}\let\jobname\childdocmain\||fi|\\
\end{tabular}
\end{center}
%
Instead of |\childdocof{|\textit{main}|}| just include the main file
at the top of each child file:
%
\begin{center}
|\input{|\textit{main}|}|
\end{center}
%
A simple redirection |\childdocforward{|\textit{dest}|}| is achieved by:
%
\begin{center}
|\def\jobname{|\textit{dest}|}\input{\jobname}|
\end{center}
%
The redirection with prefix
|\childdocforwardprefix[|\textit{prefix}|]{|\textit{dest}|}|
is accomplished by:
%
\begin{center}
\begin{tabular}{l}
|{\edef\jobname{\scantokens\expandafter{\jobname\noexpand}}|\\
|\def\redirectjob |\textit{prefix}|#1~~~{\gdef\jobname{|\textit{dest}|#1}}|\\
|\expandafter\redirectjob\jobname~~~}\input{\jobname}|
\end{tabular}
\end{center}

In an alternative approach,
child documents can be compiled by a specific command line
without additional code or specific definitions:
%
\begin{center}
|... -jobname "|\textit{target}|" "|[\textit{flags}]%
|\includeonly{|\textit{dest}|}\input{|\textit{main}|}"|
\end{center}
%

%%%%%%%%%%%%%%%%%%%%%%%%%%%%%%%%%%%%%%%%%%%%%%%%%%%%%%%%%%%%%%%%%%%%%%%%%%%%%%%%
%%%%%%%%%%%%%%%%%%%%%%%%%%%%%%%%%%%%%%%%%%%%%%%%%%%%%%%%%%%%%%%%%%%%%%%%%%%%%%%%
\section{Information}

%%%%%%%%%%%%%%%%%%%%%%%%%%%%%%%%%%%%%%%%%%%%%%%%%%%%%%%%%%%%%%%%%%%%%%%%%%%%%%%%
\subsection{Copyright}

Copyright \copyright{} 2017--2018 Niklas Beisert

This work may be distributed and/or modified under the
conditions of the \LaTeX{} Project Public License, either version 1.3
of this license or (at your option) any later version.
The latest version of this license is in
  \url{http://www.latex-project.org/lppl.txt}
and version 1.3 or later is part of all distributions of \LaTeX{}
version 2005/12/01 or later.

This work has the LPPL maintenance status `maintained'.

The Current Maintainer of this work is Niklas Beisert.

This work consists of the files |README.txt|, |childdoc.ins| and |childdoc.dtx|
as well as the derived files |childdoc.def|, |cdocsamp.tex|
with |cdocsch1.tex|, |cdocsch2.tex|, |cdocspt3.tex|, |cdocspt4.tex|,
|cdocsdrf.tex|, |cdocsfn1.tex|, |cdocsfn2.tex|
as well as |childdoc.pdf|.

%%%%%%%%%%%%%%%%%%%%%%%%%%%%%%%%%%%%%%%%%%%%%%%%%%%%%%%%%%%%%%%%%%%%%%%%%%%%%%%%
\subsection{Files and Installation}

The package consists of the files:
%
\begin{center}
\begin{tabular}{ll}
    |README.txt|   & readme file \\
    |childdoc.ins| & installation file \\
    |childdoc.dtx| & source file \\
    |childdoc.def| & definition file \\
    |cdocsamp.tex| & sample main file \\
    |cdocsch1.tex| & sample include file \\
    |cdocsch2.tex| & sample include file \\
    |cdocspt3.tex| & sample part file \\
    |cdocspt4.tex| & sample part file \\
    |cdocsdrf.tex| & sample redirection file \\
    |cdocsfn1.tex| & sample redirection file \\
    |cdocsfn2.tex| & sample redirection file \\
    |childdoc.pdf| & manual
\end{tabular}
\end{center}
%
The distribution consists of the files
|README.txt|, |childdoc.ins| and |childdoc.dtx|.
%
\begin{itemize}
\item
Run (pdf)\LaTeX{} on |childdoc.dtx|
to compile the manual |childdoc.pdf| (this file).
\item
Run \LaTeX{} on |childdoc.ins| to create the definitions file |childdoc.def|
and the sample |cdocsamp.tex| with include files
|cdocsch1.tex|, |cdocsch2.tex|, |cdocspt3.tex|, |cdocspt4.tex|,
|cdocsdrf.tex|, |cdocsfn1.tex|, |cdocsfn2.tex|.
Then copy the file |childdoc.def| to an appropriate directory of your \LaTeX{}
distribution, e.g.\ \textit{texmf-root}|/tex/latex/childdoc|.
\end{itemize}

%%%%%%%%%%%%%%%%%%%%%%%%%%%%%%%%%%%%%%%%%%%%%%%%%%%%%%%%%%%%%%%%%%%%%%%%%%%%%%%%
\subsection{Related CTAN Packages}

There are several other packages which offer a similar functionality:
%
\begin{itemize}
\item
The packages
\href{http://ctan.org/pkg/docmute}{\textsf{docmute}},
\href{http://ctan.org/pkg/includex}{\textsf{includex}} and
\href{http://ctan.org/pkg/standalone}{\textsf{standalone}}
provide commands to include only the document body of
a child file thus allowing both files to be compiled individually.
\item
The packages \href{http://ctan.org/pkg/subdocs}{\textsf{subdocs}}
and \href{http://ctan.org/pkg/subfiles}{\textsf{subfiles}}
provide structures in which the main and child documents can be
encapsulated and allowing them to be compiled individually.
The inclusion mechanism is different from the conventional |\include|.
\item
The package \href{http://ctan.org/pkg/combine}{\textsf{combine}}
is an elaborate solution to combine several documents into one.
\end{itemize}
%
See also the CTAN topic \href{http://ctan.org/topic/subdocs}{\textsf{subdocs}}
for further related packages.
The present package differs from the above solutions in that
a document structure constructed with the conventional |\include| mechanism
just needs two extra commands at the top of every file
such that all constituent files can be compiled individually.

%%%%%%%%%%%%%%%%%%%%%%%%%%%%%%%%%%%%%%%%%%%%%%%%%%%%%%%%%%%%%%%%%%%%%%%%%%%%%%%%
%\subsection{Feature Suggestions}
%
%The following is a list of features which may be useful for future
%versions of this package:
%%
%\begin{itemize}
%\item
%\ldots
%\end{itemize}

%%%%%%%%%%%%%%%%%%%%%%%%%%%%%%%%%%%%%%%%%%%%%%%%%%%%%%%%%%%%%%%%%%%%%%%%%%%%%%%%
\subsection{Revision History}

%%%%%%%%%%%%%%%%%%%%%%%%%%%%%%%%%%%%%%%%
\paragraph{v2.0:} 2018/12/30

\begin{itemize}
\item
immediate forward processing
\item
added |\childdocby| mechanism
\item
manual restructured
\end{itemize}

%%%%%%%%%%%%%%%%%%%%%%%%%%%%%%%%%%%%%%%%
\paragraph{v1.6:} 2018/01/17

\begin{itemize}
\item
application for development of include files
\item
corrections to manual
\end{itemize}

%%%%%%%%%%%%%%%%%%%%%%%%%%%%%%%%%%%%%%%%
\paragraph{v1.5:} 2017/05/21

\begin{itemize}
\item
more complete structuring introduced
\item
|\childdocof| introduced
\item
|\childdoc| renamed to |\childdocmain|
\item
|\childredirect| renamed to |\childdocforward| and |\childdocforwardprefix|
and functionality expanded
\end{itemize}

%%%%%%%%%%%%%%%%%%%%%%%%%%%%%%%%%%%%%%%%
\paragraph{v1.0:} 2017/04/27

\begin{itemize}
\item
manual and install package
\item
first version published on CTAN
\end{itemize}

%%%%%%%%%%%%%%%%%%%%%%%%%%%%%%%%%%%%%%%%
\paragraph{v0.6:} 2017/04/26

\begin{itemize}
\item
redirection mechanism added
\end{itemize}

%%%%%%%%%%%%%%%%%%%%%%%%%%%%%%%%%%%%%%%%
\paragraph{v0.5:} 2017/04/26

\begin{itemize}
\item
functionality in definition file
\end{itemize}


%%%%%%%%%%%%%%%%%%%%%%%%%%%%%%%%%%%%%%%%%%%%%%%%%%%%%%%%%%%%%%%%%%%%%%%%%%%%%%%%
%%%%%%%%%%%%%%%%%%%%%%%%%%%%%%%%%%%%%%%%%%%%%%%%%%%%%%%%%%%%%%%%%%%%%%%%%%%%%%%%
%%%%%%%%%%%%%%%%%%%%%%%%%%%%%%%%%%%%%%%%%%%%%%%%%%%%%%%%%%%%%%%%%%%%%%%%%%%%%%%%
\appendix

\settowidth\MacroIndent{\rmfamily\scriptsize 000\ }

 \DocInput{childdoc.dtx}

\end{document}
%</driver>
% \fi
%
% %%%%%%%%%%%%%%%%%%%%%%%%%%%%%%%%%%%%%%%%%%%%%%%%%%%%%%%%%%%%%%%%%%%%%%%%%%%%%%
% %%%%%%%%%%%%%%%%%%%%%%%%%%%%%%%%%%%%%%%%%%%%%%%%%%%%%%%%%%%%%%%%%%%%%%%%%%%%%%
% \section{Sample}
%\iffalse
%<*samplemain>
%\fi
%
% The following presents a sample document
% with two chapters, two parts, a title page,
% a compile flag as well as three forwarding files to set the flag.
% It consists of eight |.tex| files:
% \begin{center}
% \begin{tabular}{ll}
% |cdocsamp.tex|&main file\\
% |cdocsch1.tex|&include file for chapter 1\\
% |cdocsch2.tex|&include file for chapter 2\\
% |cdocspt3.tex|&include file for part 3\\
% |cdocspt4.tex|&include file for part 4\\
% |cdocsdrf.tex|&forwarding file for main file in draft mode\\
% |cdocsfi1.tex|&forwarding file for final version of chapter 1\\
% |cdocsfi2.tex|&forwarding file for final version of chapter 2\\
% \end{tabular}
% \end{center}
% Each of the eight files can be compiled directly by the \LaTeX{} compiler.
%
% %%%%%%%%%%%%%%%%%%%%%%%%%%%%%%%%%%%%%%
% \paragraph{Main File.}
%
% The main file is called |cdocsamp.tex|.
%
% Load the \textsf{childdoc} definitions and
% declare the filename for the main document:
%    \begin{macrocode}
\input{childdoc.def}
\childdocmain{}
%    \end{macrocode}

% Optional override for |\version| flag:
%    \begin{macrocode}
%%\ifchilddoc\else\providecommand{\version}{draft}\fi
%    \end{macrocode}

% Define the default values for the |\version| flag
% (|final| for the main file and |draft| for childs):
%    \begin{macrocode}
\ifchilddoc
\providecommand{\version}{draft}
\else
\providecommand{\version}{final}
\fi
%    \end{macrocode}

% Load the standard document class:
%    \begin{macrocode}
\documentclass[12pt]{article}
%    \end{macrocode}

% Start the document body:
%    \begin{macrocode}
\begin{document}
%    \end{macrocode}

% Declare a title page.
% Print title, part of document being processed and version flag:
%    \begin{macrocode}
\addtocounter{page}{-1}
\begin{center}
{\LARGE\bfseries{}childdoc example\par}
\vspace{1cm}
\ifchilddoc
\ifchilddocmanual part\else chapter\fi:
`\childdocname' of `\childdocjob'\par
\else
main document: `\childdocjob'\par
\fi
version: \version\par
\end{center}
\newpage
%    \end{macrocode}

% Manually include selected file,
% otherwise process as usual:
%    \begin{macrocode}
\ifchilddocmanual
\section*{part `\childdocname'}
\input{\childdocname}
\else
%    \end{macrocode}

% Include the two chapters:
%    \begin{macrocode}
\include{cdocsch1}
\include{cdocsch2}
%    \end{macrocode}

% Include the two parts unless only chapters should be displayed:
%    \begin{macrocode}
\ifchilddoc\else
\section{part three}
\input{cdocspt3}
\section{part four}
\input{cdocspt4}
\fi
%    \end{macrocode}

% Process as usual until here:
%    \begin{macrocode}
\fi
%    \end{macrocode}

% End of document body:
%    \begin{macrocode}
\end{document}
%    \end{macrocode}
%\iffalse
%</samplemain>
%\fi
%
% %%%%%%%%%%%%%%%%%%%%%%%%%%%%%%%%%%%%%%
% \paragraph{Chapter Include Files.}
%
% The include files are called |cdocsch1.tex| and |cdocsch2.tex|.
%
%\iffalse
%<*samplechap1|samplechap2>
%\fi

% Optional override for |\version| flag:
%    \begin{macrocode}
%%\providecommand{\version}{final}
%    \end{macrocode}

% Include the main document:
%    \begin{macrocode}
\input{childdoc.def}
\childdocof{cdocsamp}
%    \end{macrocode}

%\iffalse
%</samplechap1|samplechap2>
%\fi
%
%\iffalse
%<*samplechap1>
%\fi
% Some text for chapter 1:
%    \begin{macrocode}
\section{one}
some text in chapter one
%    \end{macrocode}

%\iffalse
%</samplechap1>
%\fi
% Some text for chapter 2:
%\iffalse
%<*samplechap2>
%\fi
%    \begin{macrocode}
\section{two}
more text in chapter two
%    \end{macrocode}

%\iffalse
%</samplechap2>
%\fi
%
% %%%%%%%%%%%%%%%%%%%%%%%%%%%%%%%%%%%%%%
% \paragraph{Part Include Files.}
%
% The include files are called |cdocspt3.tex| and |cdocspt4.tex|.
%
%\iffalse
%<*samplepart3|samplepart4>
%\fi

% Optional override for |\version| flag:
%    \begin{macrocode}
%%\providecommand{\version}{final}
%    \end{macrocode}

% Include the main document:
%    \begin{macrocode}
\input{childdoc.def}
\childdocby{cdocsamp}
%    \end{macrocode}

%\iffalse
%</samplepart3|samplepart4>
%\fi
%
%\iffalse
%<*samplepart3>
%\fi
% Some text for part 3:
%    \begin{macrocode}
some text in part three
%    \end{macrocode}

%\iffalse
%</samplepart3>
%\fi
% Some text for part 4:
%\iffalse
%<*samplepart4>
%\fi
%    \begin{macrocode}
more text in part four
%    \end{macrocode}

%\iffalse
%</samplepart4>
%\fi
%
% %%%%%%%%%%%%%%%%%%%%%%%%%%%%%%%%%%%%%%
% \paragraph{Forwarding for a Complete Draft.}
%
% The following forwarding file |cdocsdrf.tex|
% compiles the main document in draft mode:
%\iffalse
%<*sampledraft>
%\fi
%    \begin{macrocode}
\def\version{draft}
\input{childdoc.def}
\childdocforward{cdocsamp}
%    \end{macrocode}

%\iffalse
%</sampledraft>
%\fi
%
% %%%%%%%%%%%%%%%%%%%%%%%%%%%%%%%%%%%%%%
% \paragraph{Forwarding for Final Version of the Chapters.}
%
% The following forwarding files |cdocsfn1.tex| and |cdocsfn2.tex|
% (with identical content)
% compile the final versions of the child documents
% |cdocsch1.tex| and |cdocsch2.tex|, respectively:
%\iffalse
%<*samplefinal>
%\fi
%    \begin{macrocode}
\def\version{final}
\input{childdoc.def}
\childdocforwardprefix[cdocsamp]{cdocsfn}{cdocsch}
%    \end{macrocode}

%\iffalse
%</samplefinal>
%\fi
%
% %%%%%%%%%%%%%%%%%%%%%%%%%%%%%%%%%%%%%%
% \paragraph{Command Line Processing.}
%
% The following three command lines generate the output files
% |cdocscld|, |cdocscl1| and |cdocscl2|
% which should be identical to
% |cdocsdrf|, |cdocsch1| and |cdocsfn2|, respectively:
% \begin{center}
% \begin{tabular}{l}
% |latex -jobname cdocscld \|\\
% |  "\def\version{draft}\input{childdoc.def}\childdocforward{cdocsamp}"|\\
% |latex -jobname cdocscl1 \|\\
% |  "\input{childdoc.def}\childdocforward[cdocsamp]{cdocsch1}"|\\
% |latex -jobname cdocscl2 \|\\
% |  "\def\version{final}\input{childdoc.def}\childdocforward{cdocsch2}"|
% \end{tabular}
% \end{center}
% Note that the trailing backslash on each first line
% merely continues the input to the second line
% (for convenient cut ant paste).
% Furthermore, the command |latex| can be replaced by any
% of its alternative versions such as |pdflatex|.
%
% %%%%%%%%%%%%%%%%%%%%%%%%%%%%%%%%%%%%%%%%%%%%%%%%%%%%%%%%%%%%%%%%%%%%%%%%%%%%%%
% %%%%%%%%%%%%%%%%%%%%%%%%%%%%%%%%%%%%%%%%%%%%%%%%%%%%%%%%%%%%%%%%%%%%%%%%%%%%%%
% \section{Implementation}
%\iffalse
%<*package>
%\fi
%
% This section describes the definitions file |childdoc.def|.

% The definitions cannot be loaded using |\usepackage| or |\RequirePackage|
% which has a mechanism to prevent loading a style file more than once.
% When loading the definitions by means of |\input|
% multiple instances have to be prevented manually:
%\iffalse
%This code needs to be before the `\ProvidesFile' directive
%which is defined at the beginning of this file.
%Therefore it is also placed there and commented out here.
%</package>
%<*discard>
%\fi
%    \begin{macrocode}
\ifdefined\childdocmain\endinput\fi
%    \end{macrocode}
%\iffalse
%</discard>
%<*package>
%\fi
%
% \macro{\ifchilddoc}
% \macro{\ifchilddocmanual}
% The conditional |\ifchilddoc| tells whether a
% child (true) or main (false) document is being compiled.
% The conditional |\ifchilddocmanual| tells whether
% the |\includeonly| mechanism is used (false) or
% the selection of child files must be performed manually (true).
% The definitions initialise to false:
%    \begin{macrocode}
\newif\ifchilddoc
\newif\ifchilddocmanual
%    \end{macrocode}

% \macro{\childdocname}
% \macro{\childdocjob}
% The macro |\childdocname| stores the name of the main document
% to be compiled. The macro |\childdocjob| stores the name of
% the document on which the \LaTeX{} compiler was originally invoked.
% The content of |\jobname| cannot be compared
% to filenames specified in the source due to different catcodes.
% The following code rescans |\jobname|, stores the result
% in |\childdocname| and saves a copy in |\childdocjob|:
%    \begin{macrocode}
\edef\childdocname{\scantokens\expandafter{\jobname\noexpand}}
\let\childdocjob\childdocname
%    \end{macrocode}

% \macro{\childdocdisable}
% The macro |\childdocdisable| prevents the main file
% from being processed more than once.
% At this stage, the main document command |\childdocmain|
% is assumed to be called once again where it should do nothing.
% Any subsequent call to it should prevent
% a secondary processing of the main document
% It overwrites the forwarding commands
% |\childdocof| and |\childdocforward|
% with empty macros to prevent further inclusions of the main document:
%    \begin{macrocode}
\newcommand{\childdocdisable}
{
  \renewcommand{\childdocmain}[1]{\renewcommand{\childdocmain}[1]{\endinput}}
  \renewcommand{\childdocof}[1]{}
  \renewcommand{\childdocby}[2][]{}
  \renewcommand{\childdocforward}[2][]{}
  \renewcommand{\childdocdisable}{}
}
%    \end{macrocode}

% \macro{\childdocmain}
% The macro |\childdocmain| is to be called at the top of the main file
% with nothing or the main filename (without extension) as argument.
% First, it breaks loops.
% If the argument is not empty and does not match |\childdocname|
% (which is set by the first inclusion of |childdoc.def|),
% |\ifchilddoc| is set to true, |\includeonly| is applied to the child file
% and |\jobname| is set to the main file
% (for proper handling of |.aux| files):
%    \begin{macrocode}
\newcommand{\childdocmain}[1]
{
  \childdocdisable\childdocmain{}
  \if?#1?\else
    \begingroup
      \def\childdoctmp{#1}
      \ifx\childdoctmp\childdocname
        \def\childdoctmp{}
      \else
        \def\childdoctmp
        {
          \childdoctrue
          \includeonly{\childdocname}
          \def\childdocjob{#1}
          \def\jobname{#1}
        }
      \fi
      \expandafter
    \endgroup
    \childdoctmp
  \fi
}
%    \end{macrocode}

% \macro{\childdocof}
% The command |\childdocof| redirects
% compilation to the main file |#1|.
%    \begin{macrocode}
\newcommand{\childdocof}[1]
{
  \childdocdisable
  \childdoctrue
  \includeonly{\childdocname}
  \def\jobname{#1}
  \def\childdocjob{#1}
  \input{#1}
}
%    \end{macrocode}

% \macro{\childdocby}
% The command |\childdocby| ....
%    \begin{macrocode}
\newcommand{\childdocby}[2][]
{
  \childdocdisable
  \childdoctrue
  \childdocmanualtrue
  \if?#1?\else
    \def\jobname{#2}
  \fi
  \def\childdocjob{#2}
  \input{#2}
  \endinput
}
%    \end{macrocode}

% \macro{\childdocforward}
% The command |\childdocforward| redirects
% compilation to the main file or
% (if the optional argument is given) a child file.
% Parameters are set as if the main file
% or a child file starting with |\childdocof| was compiled.
% Then compilation is handed over to the main file:
%    \begin{macrocode}
\newcommand{\childdocforward}[2][]
{
  \begingroup
    \if?#1?
      \def\childdoctmp
      {
        \def\childdocname{#2}
        \def\childdocjob{#2}
        \def\jobname{#2}
        \input{#2}
        \endinput
      }
    \else
      \def\childdoctmp
      {
        \childdocdisable
        \def\childdocname{#2}
        \childdoctrue
        \includeonly{#2}
        \def\childdocjob{#1}
        \def\jobname{#1}
        \input{#1}
        \endinput
      }
    \fi
    \expandafter
  \endgroup
  \childdoctmp
}
%    \end{macrocode}

% \macro{\childdocforwardprefix}
% The command |\childdocforwardprefix| redirects
% compilation to the main or a child file by means of a pattern.
% The prefix |#1| in the current filename is replaced by |#2|
% and the suffix of the current filename is kept
% (it is assumed that the filename does not contain the substring `|~~~|'
% which is used as a delimiter).
% Compilation is handed over to the new file by |\childdocforward|:
%    \begin{macrocode}
\newcommand{\childdocforwardprefix}[3][]
{
  \begingroup
    \def\childdocextract #2##1~~~{\def\childdoctmp{\childdocforward[#1]{#3##1}}}
    \expandafter\childdocextract\childdocname~~~
    \expandafter
  \endgroup
  \childdoctmp
}
%    \end{macrocode}

% \macro{\childdoc}
% The deprecated macro |\childdoc| is a legacy version of |\childdocmain|:
%    \begin{macrocode}
\newcommand{\childdoc}{\childdocmain}
%    \end{macrocode}

% \macro{\childdocredirect}
% The deprecated macro |\childdocredirect| is a legacy version
% of |\childdocforward| and |\childdocforwardprefix|:
%    \begin{macrocode}
\newcommand{\childdocredirect}[2][]
{
  \begingroup
    \if?#1?
      \def\childdoctmp{\childdocforward{#2}}
    \else
      \def\childdoctmp{\childdocforwardprefix{#1}{#2}}
    \fi
    \expandafter
  \endgroup
  \childdoctmp
}
%    \end{macrocode}

%\iffalse
%</package>
%\fi
%
\endinput
|\\
|\childdocmain{}|\\
\end{tabular}
\end{center}
at the very top of the main \LaTeX{} file,
in particular \emph{before} the |\documentclass| statement!
The argument of |\childdocmain| should be left empty
(but it must be present).

%%%%%%%%%%%%%%%%%%%%%%%%%%%%%%%%%%%%%%%%
\DescribeMacro{\childdocof}
Furthermore, add the commands
\begin{center}
\begin{tabular}{l}
|% \iffalse
%
% childdoc.dtx Copyright (C) 2017-2018 Niklas Beisert
%
% This work may be distributed and/or modified under the
% conditions of the LaTeX Project Public License, either version 1.3
% of this license or (at your option) any later version.
% The latest version of this license is in
%   http://www.latex-project.org/lppl.txt
% and version 1.3 or later is part of all distributions of LaTeX
% version 2005/12/01 or later.
%
% This work has the LPPL maintenance status `maintained'.
%
% The Current Maintainer of this work is Niklas Beisert.
%
% This work consists of the files childdoc.dtx and childdoc.ins
% and the derived files childdoc.def and cdocsamp.tex with
% cdocsch1.tex, cdocsch2.tex, cdocsdrf.tex, cdocsfn1.tex, cdocsfn2.tex.
%
%<package>\ifdefined\childdocmain\endinput\fi
%<package>\ProvidesFile{childdoc.def}[2018/12/30 v2.0 child document driver]
%<samplemain>\ProvidesFile{cdocsamp.tex}[2018/12/30 v2.0 sample for childdoc]
%<*driver>
%\ProvidesFile{childdoc.drv}[2018/12/30 v2.0 childdoc reference manual file]
\PassOptionsToClass{10pt,a4paper}{article}
\documentclass{ltxdoc}

\usepackage[margin=35mm]{geometry}
\usepackage{hyperref}
\usepackage{hyperxmp}
\usepackage[usenames]{color}

\hypersetup{colorlinks=true}
\hypersetup{pdfstartview=FitH}
\hypersetup{pdfpagemode=UseNone}
\hypersetup{pdfsource={}}
\hypersetup{pdflang={en-UK}}
\hypersetup{pdfcopyright={Copyright 2017-2018 Niklas Beisert.
  This work may be distributed and/or modified under the
  conditions of the LaTeX Project Public License, either version 1.3
  of this license or (at your option) any later version.}}
\hypersetup{pdflicenseurl={http://www.latex-project.org/lppl.txt}}
\hypersetup{pdfcontactaddress={ETH Zurich, ITP, HIT K,
  Wolfgang-Pauli-Strasse 27}}
\hypersetup{pdfcontactpostcode={8093}}
\hypersetup{pdfcontactcity={Zurich}}
\hypersetup{pdfcontactcountry={Switzerland}}
\hypersetup{pdfcontactemail={nbeisert@itp.phys.ethz.ch}}
\hypersetup{pdfcontacturl={http://people.phys.ethz.ch/\xmptilde nbeisert/}}

\newcommand{\secref}[1]{\hyperref[#1]{section \ref*{#1}}}

\parskip1ex
\parindent0pt
\let\olditemize\itemize
\def\itemize{\olditemize\parskip0pt}

\begin{document}

\title{The \textsf{childdoc} Package}
\hypersetup{pdftitle={The childdoc Package}}
\author{Niklas Beisert\\[2ex]
  Institut f\"ur Theoretische Physik\\
  Eidgen\"ossische Technische Hochschule Z\"urich\\
  Wolfgang-Pauli-Strasse 27, 8093 Z\"urich, Switzerland\\[1ex]
  \href{mailto:nbeisert@itp.phys.ethz.ch}
  {\texttt{nbeisert@itp.phys.ethz.ch}}}
\hypersetup{pdfauthor={Niklas Beisert}}
\hypersetup{pdfsubject={Manual for the LaTeX2e Package childdoc}}
\date{30 December 2018, \textsf{v2.0}}
\maketitle

\begin{abstract}\noindent
\textsf{childdoc} is a \LaTeXe{} package
that enables the direct compilation
of document sections included by |\include|
to individual files.
\end{abstract}

\begingroup
\parskip0ex
\tableofcontents
\endgroup

%%%%%%%%%%%%%%%%%%%%%%%%%%%%%%%%%%%%%%%%%%%%%%%%%%%%%%%%%%%%%%%%%%%%%%%%%%%%%%%%
%%%%%%%%%%%%%%%%%%%%%%%%%%%%%%%%%%%%%%%%%%%%%%%%%%%%%%%%%%%%%%%%%%%%%%%%%%%%%%%%
\section{Introduction}

\LaTeX{} provides a mechanism to structure a large document (such as a book)
into a main file and several child files (containing the chapters)
using the |\include| command.
This mechanism is beneficial for documents
which span hundreds of pages in order to
make the source file(s) more manageable.
Moreover, compilation can be restricted to
selected child files by means of the |\includeonly| command.
The latter feature can be used to reduce the compilation time while editing
(this was significantly more useful in the earlier days of \LaTeX{})
or to generate a smaller document which is easier to navigate.
Another application of |\includeonly| is to generate
documents consisting of selected parts of the complete document.

However, there are a few drawbacks of the plain |\include| mechanism:
\begin{itemize}
\item
The child files cannot be compiled on their own,
they can only be compiled via the main file.
A naive editing environment
(such as a text editor with an option
to have the current file processed by \LaTeX)
may require one to switch to the main file before compiling;
attempting to compile the child file produces errors.
\item
The main file must be modified (each time)
to adjust the |\includeonly| command
to the present needs. This easily leaves the main file in a messy state.
\item
The generated document will always carry the filename
of the main document. This is inconvenient if
several child files are to be compiled and
to be kept for distribution.
\end{itemize}

The present package provides a simple interface
to make child files individually compilable by \LaTeX{}.
Compiling a child file then has the same effect as compiling
the main file with an |\includeonly| command
to select the appropriate child.
Moreover the generated document will carry the name of the child
rather than the main file.
This resolves all three above issues.

This feature is meant to make the editing of books,
thesis documents and lecture notes somewhat more convenient.
However, the package can also be used efficiently for
composing a series of documents (such as exercise sheets)
which are typically distributed individually.
It then assists the author in generating the individual documents
(potentially in different versions)
as well as a document containing the collected series.
Another application is in developing style files
or other kinds of included material
where compilation of the style file could redirect
to a sample or test file.

%%%%%%%%%%%%%%%%%%%%%%%%%%%%%%%%%%%%%%%%%%%%%%%%%%%%%%%%%%%%%%%%%%%%%%%%%%%%%%%%
%%%%%%%%%%%%%%%%%%%%%%%%%%%%%%%%%%%%%%%%%%%%%%%%%%%%%%%%%%%%%%%%%%%%%%%%%%%%%%%%
\section{Usage}

First of all, the package \textsf{childdoc} is \emph{not} a standard
\LaTeXe{} |.sty| style file! Therefore it needs to be invoked in
a non-standard way.

%%%%%%%%%%%%%%%%%%%%%%%%%%%%%%%%%%%%%%%%%%%%%%%%%%%%%%%%%%%%%%%%%%%%%%%%%%%%%%%%
\subsection{Included Files}
\label{sec:include}

%%%%%%%%%%%%%%%%%%%%%%%%%%%%%%%%%%%%%%%%
\DescribeMacro{\childdocmain}
To use the package, add the commands
\begin{center}
\begin{tabular}{l}
|\input{childdoc.def}|\\
|\childdocmain{}|\\
\end{tabular}
\end{center}
at the very top of the main \LaTeX{} file,
in particular \emph{before} the |\documentclass| statement!
The argument of |\childdocmain| should be left empty
(but it must be present).

%%%%%%%%%%%%%%%%%%%%%%%%%%%%%%%%%%%%%%%%
\DescribeMacro{\childdocof}
Furthermore, add the commands
\begin{center}
\begin{tabular}{l}
|\input{childdoc.def}|\\
|\childdocof{|\textit{main}|}|\\
\end{tabular}
\end{center}
at the top of every child file \textit{child}
which is included by |\include{|\textit{child}|}|
from within the main file
(or at least for those files to be compiled individually).
The argument \textit{main} must be the filename of the main file.

There are a couple of
considerations in setting up the main and child documents:

%%%%%%%%%%%%%%%%%%%%%%%%%%%%%%%%%%%%%%%%
\paragraph{Restrictions.}

Please note the following restrictions:
\begin{itemize}
\item
|\childdocmain| must be called with one argument \textit{main}
to ensure compatibility with earlier version of the package.
It must either be empty (|\childdocmain{}|)
or precisely match the filename of the main file in which it is specified.
See \secref{sec:detection} for further information.
\item
The filename \textit{main} must be specified without the |.tex| extension.
\item
The filename \textit{main} is case sensitive
(even in case-insensitive file systems)
due to internal string comparison.
\item
The argument \textit{main} should be fully expanded, it cannot be a macro.
\item
Subdirectories and special characters should be avoided in filenames.
\item
The command |\childdocmain{|\textit{main}|}| must be followed by a whitespace.
It should not be followed immediately by another command
or by a comment mark `|%|'.
This is because the \TeX{} parser reads the token immediately following
the argument of |\childdocmain| and puts it
at the beginning of every child section;
however, a white\-space is ignored.
\end{itemize}

%%%%%%%%%%%%%%%%%%%%%%%%%%%%%%%%%%%%%%%%
\paragraph{Content of Main File.}

It is advisable to place all content in the child files included by |\include|.
Any output contained in the main file will appear in all child documents
unless suppressed manually;
it cannot be suppressed automatically by the |\includeonly| directive
and thus should normally be avoided.
A method to include some content in the main file
by means of conditional processing is described in \secref{sec:conditional}.

%%%%%%%%%%%%%%%%%%%%%%%%%%%%%%%%%%%%%%%%
\paragraph{Page Numbering.}

When only a part of the document is compiled,
the appropriate numbering of pages
(as well as other status parameters)
is determined from the |.aux| files.
The latter contain information from previous passes.
However this information needs to propagate through
all intermediate child documents.
Therefore the page numbering in child documents may well
be inconsistent until the complete document is compiled at least once.

A useful (if unconventional) way to always ensure a consistent
page numbering is to restart the numbering in each child document
and denote the pages by `\textit{child}|.|\textit{page}'
where \textit{child} represents the chapter/section number of the child file.
This can be achieved by the command
|\numberwithin{page}{|\textit{child}|}|
of the \textsf{amsmath} package
where \textit{child} can be |chapter| or |section|
depending on the chosen structuring.
Alternatively, one can modify the macro |\thepage| appropriately
and reset the counter |page| at the start of each child file.

%%%%%%%%%%%%%%%%%%%%%%%%%%%%%%%%%%%%%%%%%%%%%%%%%%%%%%%%%%%%%%%%%%%%%%%%%%%%%%%%
\subsection{Conditional Processing}
\label{sec:conditional}

The package provides a mechanism to compile different versions
of a document. To customise the versions further some conditional processing
can come in handy to distinguish which version is being compiled.
The package provides two macros to describe the compilation context:

%%%%%%%%%%%%%%%%%%%%%%%%%%%%%%%%%%%%%%%%
\DescribeMacro{\ifchilddoc}
The conditional |\ifchilddoc| distinguishes between the compilation of
child documents and the main document:
%
\begin{center}
|\ifchilddoc |\textit{child-code}| |[|\||else |\textit{main-code}]| \||fi|
\end{center}

%%%%%%%%%%%%%%%%%%%%%%%%%%%%%%%%%%%%%%%%
\DescribeMacro{\childdocname}
\DescribeMacro{\childdocjob}
The macro |\childdocname| contains the filename (without extension)
of the main or child file being processed.
Note that |\childdocjob| will always contain the name of the main file.

%%%%%%%%%%%%%%%%%%%%%%%%%%%%%%%%%%%%%%%%
\paragraph{Title Page.}

Conditional processing can be used to include a title or banner page
in the main document when proper precautions are taken.
Importantly, the code in the main file should ensure that the page counter
(as well as other status parameters which are stored in the |.aux| files)
takes the same value after the conditional processing.
Otherwise the page numbers may take divergent values
depending on which part is compiled.

For example, a title page could be declared by:
%
\begin{center}
\begin{tabular}{l}
|\ifchilddoc\||else|\\
|\addtocounter{page}{-1}|\\
\textit{code for title page}\\
|\newpage|\\
|\||fi|
\end{tabular}
\end{center}
%
A banner page for the child documents can be generated by:
%
\begin{center}
\begin{tabular}{l}
|\ifchilddoc|\\
|\addtocounter{page}{-1}|\\
\textit{code for banner page}\\
|\newpage|\\
|\||fi|
\end{tabular}
\end{center}
%
Here one could write a message such as:
\begin{center}
|This is the part \childdocname{} of \childdocjob{}.|
\end{center}

%%%%%%%%%%%%%%%%%%%%%%%%%%%%%%%%%%%%%%%%%%%%%%%%%%%%%%%%%%%%%%%%%%%%%%%%%%%%%%%%
\subsection{Flags}
\label{sec:flags}

The package makes it easy to generate different versions
of the main or child documents.
To this end compilation flags can be defined
and assigned different default values.
They will be particularly useful in conjunction
with the forwarding mechanism described in \secref{sec:forward}.

For example, it may be useful to have a flag |\version|
which can be set to |draft| or |final|.
The document source will contain some conditional code
depending on the value of |\version|.
Suppose further, the flag should default to |final| for the main file
and to |draft| for child files
which is a natural assignment for editing the document.
This is achieved by placing the following code
in the preamble of the main document
(below the |\childdocmain| directive):
%
\begin{center}
\begin{tabular}{l}
|\ifchilddoc|\\
|\providecommand{\version}{draft}|\\
|\||else|\\
|\providecommand{\version}{final}|\\
|\||fi|
\end{tabular}
\end{center}
%
The definition by |\providecommand| makes sure
that previous definitions are not overwritten.
Further statements |\providecommand{\version}{...}|
can thus be added before the above code to override it.

For the main file, one might add a line
(between |\childdocmain| and the above block)
%
\begin{center}
|%\ifchilddoc\||else\providecommand{\version}{draft}\||fi|
\end{center}
%
which can be uncommented to produce a draft version.
Likewise one can add a line to the very top of a child file
(above the |\childdocof{|\textit{main}|}| directive)
%
\begin{center}
|%\providecommand{\version}{final}|
\end{center}
%
which can be uncommented to produce the final version of this child document.

%%%%%%%%%%%%%%%%%%%%%%%%%%%%%%%%%%%%%%%%%%%%%%%%%%%%%%%%%%%%%%%%%%%%%%%%%%%%%%%%
\subsection{Forwarding}
\label{sec:forward}

Different versions of the main or child documents
using compilation flags as described in \secref{sec:flags}
can be (permanently) stored in different files
for convenient compilation, viewing and distribution.
To this end, the package defines a command
to pass on compilation to a different file:

%%%%%%%%%%%%%%%%%%%%%%%%%%%%%%%%%%%%%%%%
\DescribeMacro{\childdocforward}
The command |\childdocforward| redirects processing to
another source file:
%
\begin{center}
\begin{tabular}{l}
|\input{childdoc.def}|\\
|\childdocforward[|\textit{main}|]{|\textit{dest}|}|\\
\end{tabular}
\end{center}
%
The argument \textit{dest} is the destination file
(without extension).
It should be the main file or one of the child files.
Note that further \textsf{childdoc} directives
such as |\childdocof| and |\childdocforward|
in the indicated file will be processed in this form.
The optional argument \textit{main}
passes on directly to the main file \textit{main}
while pretending to compile the child \textit{dest}.
This form behaves as if \textit{dest}
issues |\childdocof{|\textit{main}|}| right away,
and no further \textsf{childdoc} directives will be processed.

%%%%%%%%%%%%%%%%%%%%%%%%%%%%%%%%%%%%%%%%
\DescribeMacro{\...prefix}
In the alternative form |\childdocforwardprefix|,
%
\begin{center}
\begin{tabular}{l}
|\input{childdoc.def}|\\
|\childdocforwardprefix[|\textit{main}|]{|\textit{prefix}|}{|\textit{dest}|}|
\end{tabular}
\end{center}
%
the destination file is determined by a pattern
depending on the current file:
To make this work, the current file must be called
`{\textit{prefix}\hspace{0.2em}\textit{suffix}}'
with \textit{prefix} matching precisely the argument.
Processing is then passed on to the file
`{\textit{dest}\hspace{0.2em}\textit{suffix}}'.
Surely, the same effect is achieved by
directly specifying the
argument `{\textit{dest}\hspace{0.2em}\textit{suffix}}'
in the first form.
However, that requires to set up a different file
for each child. With the alternative form of the command
all these files can have exactly the same content
which simplifies setting them up and maintaining them.

For example, the following file |draft.tex|
with a compilation flag |\version| as described in \secref{sec:flags}
compiles the main document as a draft:
%
\begin{center}
\begin{tabular}{l}
|\def\version{draft}|\\
|\input{childdoc.def}|\\
|\childdocforward{|\textit{main}|}|
\end{tabular}
\end{center}
%
Likewise, the following files |final|\textit{nn}|.tex|
compile the final version of the child document
|child|\textit{nn}|.tex|:
%
\begin{center}
\begin{tabular}{l}
|\def\version{final}|\\
|\input{childdoc.def}|\\
|\childdocforwardprefix{final}{child}|
\end{tabular}
\end{center}
%

Note that when several versions of a main file and/or of each child file
are to be generated, it may be convenient to set up a |Makefile| or
shell script to automatise the process.

%%%%%%%%%%%%%%%%%%%%%%%%%%%%%%%%%%%%%%%%%%%%%%%%%%%%%%%%%%%%%%%%%%%%%%%%%%%%%%%%
\subsection{Command Line Processing}
\label{sec:commandline}

The effect of redirection files can also be achieved by invoking
the \LaTeX{} compiler with a more elaborate command line.
Most conveniently this should be done as part
of a shell script or a |Makefile|.

When using \textsf{childdoc} in the main file, the following
command lines effectively perform a redirection
(note that depending on the shell being used,
backslashes may have to be doubled: `|\|' $\to$ `|\\|'):
%
\begin{center}
|... -jobname "|\textit{target}|" |\\|"|[\textit{flags}]%
|\input{childdoc.def}\childdocforward[|\textit{main}|]{|\textit{dest}|}"|
\end{center}
%
Here \textit{target} is the name of the output file,
\textit{main} is the name of the main file
and \textit{dest} is the name of the main or child file to be processed
(all filenames without extensions).
The optional argument \textit{main} can be omitted
if \textit{main} matches \textit{dest}.
Optionally, compilation \textit{flags} can be defined via |\def| commands.
This command line makes the \TeX{} engine believe
it is compiling the file \textit{target}
whose content is specified as the latter parameter.
The provided code then forwards the processing to
\textit{main} or \textit{dest} as described in \secref{sec:forward}.

%%%%%%%%%%%%%%%%%%%%%%%%%%%%%%%%%%%%%%%%%%%%%%%%%%%%%%%%%%%%%%%%%%%%%%%%%%%%%%%%
\subsection{Include by Input}
\label{sec:input}

Including child documents by |\include| has some restrictions by design.
Most notably, the content of a child document always occupies
its own set of pages; pages cannot be shared between child documents.
Usually, this behaviour makes perfect sense
because each child document contain an essential part of the document.
However, in some situations it may be desirable to compose
a document from a collection of parts
without having mandatory page breaks between then.
For this case, the package
provides a mechanism to include parts
by |\input| which can also be processed individually.
However, by construction this mechanism
requires manual handling of the content to be output.

%%%%%%%%%%%%%%%%%%%%%%%%%%%%%%%%%%%%%%%%
\DescribeMacro{\ifchilddocmanual}
The main file should be prepared as usual, see \secref{sec:include}.
However, the document body must make a distinction
between processing of an individual part and of the main document, e.g.:
%
\begin{center}
\begin{tabular}{l}
|\ifchilddocmanual|\\
|\input{\childdocname}|\\
|\||else|\\
\textit{document body with }|\input{|\textit{part}|}|\\
|\||fi|
\end{tabular}
\end{center}
%
The conditional |\ifchilddocmanual| is true whenever
a part to be included by |\input| is being compiled,
and the name of the part is stored in |\childdocname|.

%%%%%%%%%%%%%%%%%%%%%%%%%%%%%%%%%%%%%%%%
\DescribeMacro{\childdocby}
Each part to be included by |\input| should start with:
%
\begin{center}
\begin{tabular}{l}
|\input{childdoc.def}|\\
|\childdocby{|\textit{main}|}|\\
\end{tabular}
\end{center}
%
The directive |\childdocby| is similar to |\childdocof|
described in \secref{sec:include},
but the subsequent selection of content must be done manually.
To that end, both |\ifchilddoc| and |\ifchilddocmanual|
will be true upon processing of a part,
and the name of the part is stored in |\childdocname|.
Note that |\jobname| will be set to the filename of the current part
so that each part receives an individual |.aux| file
that does not interfere with the |.aux| file(s) of the main document.
This behaviour can be altered by the alternative form
|\childdocby[*]{|\textit{main}|}| (with a non-empty optional argument)
which uses the |.aux| file of the main document
by setting |\jobname| to \textit{main}.

%%%%%%%%%%%%%%%%%%%%%%%%%%%%%%%%%%%%%%%%%%%%%%%%%%%%%%%%%%%%%%%%%%%%%%%%%%%%%%%%
\subsection{Driver Development}
\label{sec:driver}

The \textsf{childdoc} mechanism can also be use for the development
of definition files such as \LaTeX{} styles or classes.
This case differs from the above setup with multiple parts
included by |\include| in that no |\includeonly| should be invoked.
This can be achieved by starting the include file
(before |\ProvidesPackage|) with:
%
\begin{center}
\begin{tabular}{l}
|\input{childdoc.def}|\\
|\childdocforward{|\textit{main}|}|\\
\end{tabular}
\end{center}
%
or alternatively with:
%
\begin{center}
\begin{tabular}{l}
|\input{childdoc.def}|\\
|\childdocby{|\textit{main}|}|\\
\end{tabular}
\end{center}
%
Both forms have slightly different effects as described above.
The main file is prepared as usual, see \secref{sec:include}.

%%%%%%%%%%%%%%%%%%%%%%%%%%%%%%%%%%%%%%%%%%%%%%%%%%%%%%%%%%%%%%%%%%%%%%%%%%%%%%%%
\subsection{Legacy Detection}
\label{sec:detection}

The directive |\childdocmain| in the main file can detect
whether the complete document or merely a child is to be compiled
even without using the directive |\childdocof|.
This method is deprecated because it is less robust
and there is no compelling reason to use it;
it is merely provided for backward compatibility
and it may be removed in future versions.

If the detection mechanism is to be used,
it is mandatory to correctly specify
the filename of the main file as the argument of |\childdocmain|:
%
\begin{center}
\begin{tabular}{l}
|\input{childdoc.def}|\\
|\childdocmain{|\textit{main}|}|\\
\end{tabular}
\end{center}
%
If |\jobname| does not match the argument \textit{main} of |\childdocmain|,
it is assumed that |\jobname| points to the child file to be compiled.
When using |\childdocmain| with the main file specified as argument,
it suffices to start a child file
with just |\input{|\textit{main}|}|
without loading of the package and using |\childdocof|.
If instead all processing is done
with the appropriate \textsf{childdoc} directives,
the argument of \textit{main} of |\childdocmain| can be empty.

An alternative version of the command line processing described
in \secref{sec:commandline} using the detection mechanism reads:
%
\begin{center}
|... -jobname "|\textit{target}|" "|[\textit{flags}]%
[|\def\jobname{|\textit{dest}|}|]|\input{|\textit{main}|}"|
\end{center}

%%%%%%%%%%%%%%%%%%%%%%%%%%%%%%%%%%%%%%%%%%%%%%%%%%%%%%%%%%%%%%%%%%%%%%%%%%%%%%%%
\subsection{Manual Code}
\label{sec:manual}

In case one cannot be certain whether the definitions file |childdoc.def|
is installed on the target \TeX{} distribution
and one prefers not to ship it,
it is conceivable to paste a few relevant commands into the sources.

To that end, drop all statements |\input{childdoc.def}|
and perform the replacements as outlined below.
Instead of |\childdocmain{|\textit{main}|}| add the following code
to the top of the main file:
%
\begin{center}
\begin{tabular}{l}
|\||ifdefined\childdocname\endinput\||fi\newif\ifchilddoc|\\
|\edef\childdocname{\scantokens\expandafter{\jobname\noexpand}}|\\
|\def\childdocmain{|\textit{main}|}\||ifx\childdocmain\childdocname\||else|\\
|\childdoctrue\includeonly{\childdocname}\let\jobname\childdocmain\||fi|\\
\end{tabular}
\end{center}
%
Instead of |\childdocof{|\textit{main}|}| just include the main file
at the top of each child file:
%
\begin{center}
|\input{|\textit{main}|}|
\end{center}
%
A simple redirection |\childdocforward{|\textit{dest}|}| is achieved by:
%
\begin{center}
|\def\jobname{|\textit{dest}|}\input{\jobname}|
\end{center}
%
The redirection with prefix
|\childdocforwardprefix[|\textit{prefix}|]{|\textit{dest}|}|
is accomplished by:
%
\begin{center}
\begin{tabular}{l}
|{\edef\jobname{\scantokens\expandafter{\jobname\noexpand}}|\\
|\def\redirectjob |\textit{prefix}|#1~~~{\gdef\jobname{|\textit{dest}|#1}}|\\
|\expandafter\redirectjob\jobname~~~}\input{\jobname}|
\end{tabular}
\end{center}

In an alternative approach,
child documents can be compiled by a specific command line
without additional code or specific definitions:
%
\begin{center}
|... -jobname "|\textit{target}|" "|[\textit{flags}]%
|\includeonly{|\textit{dest}|}\input{|\textit{main}|}"|
\end{center}
%

%%%%%%%%%%%%%%%%%%%%%%%%%%%%%%%%%%%%%%%%%%%%%%%%%%%%%%%%%%%%%%%%%%%%%%%%%%%%%%%%
%%%%%%%%%%%%%%%%%%%%%%%%%%%%%%%%%%%%%%%%%%%%%%%%%%%%%%%%%%%%%%%%%%%%%%%%%%%%%%%%
\section{Information}

%%%%%%%%%%%%%%%%%%%%%%%%%%%%%%%%%%%%%%%%%%%%%%%%%%%%%%%%%%%%%%%%%%%%%%%%%%%%%%%%
\subsection{Copyright}

Copyright \copyright{} 2017--2018 Niklas Beisert

This work may be distributed and/or modified under the
conditions of the \LaTeX{} Project Public License, either version 1.3
of this license or (at your option) any later version.
The latest version of this license is in
  \url{http://www.latex-project.org/lppl.txt}
and version 1.3 or later is part of all distributions of \LaTeX{}
version 2005/12/01 or later.

This work has the LPPL maintenance status `maintained'.

The Current Maintainer of this work is Niklas Beisert.

This work consists of the files |README.txt|, |childdoc.ins| and |childdoc.dtx|
as well as the derived files |childdoc.def|, |cdocsamp.tex|
with |cdocsch1.tex|, |cdocsch2.tex|, |cdocspt3.tex|, |cdocspt4.tex|,
|cdocsdrf.tex|, |cdocsfn1.tex|, |cdocsfn2.tex|
as well as |childdoc.pdf|.

%%%%%%%%%%%%%%%%%%%%%%%%%%%%%%%%%%%%%%%%%%%%%%%%%%%%%%%%%%%%%%%%%%%%%%%%%%%%%%%%
\subsection{Files and Installation}

The package consists of the files:
%
\begin{center}
\begin{tabular}{ll}
    |README.txt|   & readme file \\
    |childdoc.ins| & installation file \\
    |childdoc.dtx| & source file \\
    |childdoc.def| & definition file \\
    |cdocsamp.tex| & sample main file \\
    |cdocsch1.tex| & sample include file \\
    |cdocsch2.tex| & sample include file \\
    |cdocspt3.tex| & sample part file \\
    |cdocspt4.tex| & sample part file \\
    |cdocsdrf.tex| & sample redirection file \\
    |cdocsfn1.tex| & sample redirection file \\
    |cdocsfn2.tex| & sample redirection file \\
    |childdoc.pdf| & manual
\end{tabular}
\end{center}
%
The distribution consists of the files
|README.txt|, |childdoc.ins| and |childdoc.dtx|.
%
\begin{itemize}
\item
Run (pdf)\LaTeX{} on |childdoc.dtx|
to compile the manual |childdoc.pdf| (this file).
\item
Run \LaTeX{} on |childdoc.ins| to create the definitions file |childdoc.def|
and the sample |cdocsamp.tex| with include files
|cdocsch1.tex|, |cdocsch2.tex|, |cdocspt3.tex|, |cdocspt4.tex|,
|cdocsdrf.tex|, |cdocsfn1.tex|, |cdocsfn2.tex|.
Then copy the file |childdoc.def| to an appropriate directory of your \LaTeX{}
distribution, e.g.\ \textit{texmf-root}|/tex/latex/childdoc|.
\end{itemize}

%%%%%%%%%%%%%%%%%%%%%%%%%%%%%%%%%%%%%%%%%%%%%%%%%%%%%%%%%%%%%%%%%%%%%%%%%%%%%%%%
\subsection{Related CTAN Packages}

There are several other packages which offer a similar functionality:
%
\begin{itemize}
\item
The packages
\href{http://ctan.org/pkg/docmute}{\textsf{docmute}},
\href{http://ctan.org/pkg/includex}{\textsf{includex}} and
\href{http://ctan.org/pkg/standalone}{\textsf{standalone}}
provide commands to include only the document body of
a child file thus allowing both files to be compiled individually.
\item
The packages \href{http://ctan.org/pkg/subdocs}{\textsf{subdocs}}
and \href{http://ctan.org/pkg/subfiles}{\textsf{subfiles}}
provide structures in which the main and child documents can be
encapsulated and allowing them to be compiled individually.
The inclusion mechanism is different from the conventional |\include|.
\item
The package \href{http://ctan.org/pkg/combine}{\textsf{combine}}
is an elaborate solution to combine several documents into one.
\end{itemize}
%
See also the CTAN topic \href{http://ctan.org/topic/subdocs}{\textsf{subdocs}}
for further related packages.
The present package differs from the above solutions in that
a document structure constructed with the conventional |\include| mechanism
just needs two extra commands at the top of every file
such that all constituent files can be compiled individually.

%%%%%%%%%%%%%%%%%%%%%%%%%%%%%%%%%%%%%%%%%%%%%%%%%%%%%%%%%%%%%%%%%%%%%%%%%%%%%%%%
%\subsection{Feature Suggestions}
%
%The following is a list of features which may be useful for future
%versions of this package:
%%
%\begin{itemize}
%\item
%\ldots
%\end{itemize}

%%%%%%%%%%%%%%%%%%%%%%%%%%%%%%%%%%%%%%%%%%%%%%%%%%%%%%%%%%%%%%%%%%%%%%%%%%%%%%%%
\subsection{Revision History}

%%%%%%%%%%%%%%%%%%%%%%%%%%%%%%%%%%%%%%%%
\paragraph{v2.0:} 2018/12/30

\begin{itemize}
\item
immediate forward processing
\item
added |\childdocby| mechanism
\item
manual restructured
\end{itemize}

%%%%%%%%%%%%%%%%%%%%%%%%%%%%%%%%%%%%%%%%
\paragraph{v1.6:} 2018/01/17

\begin{itemize}
\item
application for development of include files
\item
corrections to manual
\end{itemize}

%%%%%%%%%%%%%%%%%%%%%%%%%%%%%%%%%%%%%%%%
\paragraph{v1.5:} 2017/05/21

\begin{itemize}
\item
more complete structuring introduced
\item
|\childdocof| introduced
\item
|\childdoc| renamed to |\childdocmain|
\item
|\childredirect| renamed to |\childdocforward| and |\childdocforwardprefix|
and functionality expanded
\end{itemize}

%%%%%%%%%%%%%%%%%%%%%%%%%%%%%%%%%%%%%%%%
\paragraph{v1.0:} 2017/04/27

\begin{itemize}
\item
manual and install package
\item
first version published on CTAN
\end{itemize}

%%%%%%%%%%%%%%%%%%%%%%%%%%%%%%%%%%%%%%%%
\paragraph{v0.6:} 2017/04/26

\begin{itemize}
\item
redirection mechanism added
\end{itemize}

%%%%%%%%%%%%%%%%%%%%%%%%%%%%%%%%%%%%%%%%
\paragraph{v0.5:} 2017/04/26

\begin{itemize}
\item
functionality in definition file
\end{itemize}


%%%%%%%%%%%%%%%%%%%%%%%%%%%%%%%%%%%%%%%%%%%%%%%%%%%%%%%%%%%%%%%%%%%%%%%%%%%%%%%%
%%%%%%%%%%%%%%%%%%%%%%%%%%%%%%%%%%%%%%%%%%%%%%%%%%%%%%%%%%%%%%%%%%%%%%%%%%%%%%%%
%%%%%%%%%%%%%%%%%%%%%%%%%%%%%%%%%%%%%%%%%%%%%%%%%%%%%%%%%%%%%%%%%%%%%%%%%%%%%%%%
\appendix

\settowidth\MacroIndent{\rmfamily\scriptsize 000\ }

 \DocInput{childdoc.dtx}

\end{document}
%</driver>
% \fi
%
% %%%%%%%%%%%%%%%%%%%%%%%%%%%%%%%%%%%%%%%%%%%%%%%%%%%%%%%%%%%%%%%%%%%%%%%%%%%%%%
% %%%%%%%%%%%%%%%%%%%%%%%%%%%%%%%%%%%%%%%%%%%%%%%%%%%%%%%%%%%%%%%%%%%%%%%%%%%%%%
% \section{Sample}
%\iffalse
%<*samplemain>
%\fi
%
% The following presents a sample document
% with two chapters, two parts, a title page,
% a compile flag as well as three forwarding files to set the flag.
% It consists of eight |.tex| files:
% \begin{center}
% \begin{tabular}{ll}
% |cdocsamp.tex|&main file\\
% |cdocsch1.tex|&include file for chapter 1\\
% |cdocsch2.tex|&include file for chapter 2\\
% |cdocspt3.tex|&include file for part 3\\
% |cdocspt4.tex|&include file for part 4\\
% |cdocsdrf.tex|&forwarding file for main file in draft mode\\
% |cdocsfi1.tex|&forwarding file for final version of chapter 1\\
% |cdocsfi2.tex|&forwarding file for final version of chapter 2\\
% \end{tabular}
% \end{center}
% Each of the eight files can be compiled directly by the \LaTeX{} compiler.
%
% %%%%%%%%%%%%%%%%%%%%%%%%%%%%%%%%%%%%%%
% \paragraph{Main File.}
%
% The main file is called |cdocsamp.tex|.
%
% Load the \textsf{childdoc} definitions and
% declare the filename for the main document:
%    \begin{macrocode}
\input{childdoc.def}
\childdocmain{}
%    \end{macrocode}

% Optional override for |\version| flag:
%    \begin{macrocode}
%%\ifchilddoc\else\providecommand{\version}{draft}\fi
%    \end{macrocode}

% Define the default values for the |\version| flag
% (|final| for the main file and |draft| for childs):
%    \begin{macrocode}
\ifchilddoc
\providecommand{\version}{draft}
\else
\providecommand{\version}{final}
\fi
%    \end{macrocode}

% Load the standard document class:
%    \begin{macrocode}
\documentclass[12pt]{article}
%    \end{macrocode}

% Start the document body:
%    \begin{macrocode}
\begin{document}
%    \end{macrocode}

% Declare a title page.
% Print title, part of document being processed and version flag:
%    \begin{macrocode}
\addtocounter{page}{-1}
\begin{center}
{\LARGE\bfseries{}childdoc example\par}
\vspace{1cm}
\ifchilddoc
\ifchilddocmanual part\else chapter\fi:
`\childdocname' of `\childdocjob'\par
\else
main document: `\childdocjob'\par
\fi
version: \version\par
\end{center}
\newpage
%    \end{macrocode}

% Manually include selected file,
% otherwise process as usual:
%    \begin{macrocode}
\ifchilddocmanual
\section*{part `\childdocname'}
\input{\childdocname}
\else
%    \end{macrocode}

% Include the two chapters:
%    \begin{macrocode}
\include{cdocsch1}
\include{cdocsch2}
%    \end{macrocode}

% Include the two parts unless only chapters should be displayed:
%    \begin{macrocode}
\ifchilddoc\else
\section{part three}
\input{cdocspt3}
\section{part four}
\input{cdocspt4}
\fi
%    \end{macrocode}

% Process as usual until here:
%    \begin{macrocode}
\fi
%    \end{macrocode}

% End of document body:
%    \begin{macrocode}
\end{document}
%    \end{macrocode}
%\iffalse
%</samplemain>
%\fi
%
% %%%%%%%%%%%%%%%%%%%%%%%%%%%%%%%%%%%%%%
% \paragraph{Chapter Include Files.}
%
% The include files are called |cdocsch1.tex| and |cdocsch2.tex|.
%
%\iffalse
%<*samplechap1|samplechap2>
%\fi

% Optional override for |\version| flag:
%    \begin{macrocode}
%%\providecommand{\version}{final}
%    \end{macrocode}

% Include the main document:
%    \begin{macrocode}
\input{childdoc.def}
\childdocof{cdocsamp}
%    \end{macrocode}

%\iffalse
%</samplechap1|samplechap2>
%\fi
%
%\iffalse
%<*samplechap1>
%\fi
% Some text for chapter 1:
%    \begin{macrocode}
\section{one}
some text in chapter one
%    \end{macrocode}

%\iffalse
%</samplechap1>
%\fi
% Some text for chapter 2:
%\iffalse
%<*samplechap2>
%\fi
%    \begin{macrocode}
\section{two}
more text in chapter two
%    \end{macrocode}

%\iffalse
%</samplechap2>
%\fi
%
% %%%%%%%%%%%%%%%%%%%%%%%%%%%%%%%%%%%%%%
% \paragraph{Part Include Files.}
%
% The include files are called |cdocspt3.tex| and |cdocspt4.tex|.
%
%\iffalse
%<*samplepart3|samplepart4>
%\fi

% Optional override for |\version| flag:
%    \begin{macrocode}
%%\providecommand{\version}{final}
%    \end{macrocode}

% Include the main document:
%    \begin{macrocode}
\input{childdoc.def}
\childdocby{cdocsamp}
%    \end{macrocode}

%\iffalse
%</samplepart3|samplepart4>
%\fi
%
%\iffalse
%<*samplepart3>
%\fi
% Some text for part 3:
%    \begin{macrocode}
some text in part three
%    \end{macrocode}

%\iffalse
%</samplepart3>
%\fi
% Some text for part 4:
%\iffalse
%<*samplepart4>
%\fi
%    \begin{macrocode}
more text in part four
%    \end{macrocode}

%\iffalse
%</samplepart4>
%\fi
%
% %%%%%%%%%%%%%%%%%%%%%%%%%%%%%%%%%%%%%%
% \paragraph{Forwarding for a Complete Draft.}
%
% The following forwarding file |cdocsdrf.tex|
% compiles the main document in draft mode:
%\iffalse
%<*sampledraft>
%\fi
%    \begin{macrocode}
\def\version{draft}
\input{childdoc.def}
\childdocforward{cdocsamp}
%    \end{macrocode}

%\iffalse
%</sampledraft>
%\fi
%
% %%%%%%%%%%%%%%%%%%%%%%%%%%%%%%%%%%%%%%
% \paragraph{Forwarding for Final Version of the Chapters.}
%
% The following forwarding files |cdocsfn1.tex| and |cdocsfn2.tex|
% (with identical content)
% compile the final versions of the child documents
% |cdocsch1.tex| and |cdocsch2.tex|, respectively:
%\iffalse
%<*samplefinal>
%\fi
%    \begin{macrocode}
\def\version{final}
\input{childdoc.def}
\childdocforwardprefix[cdocsamp]{cdocsfn}{cdocsch}
%    \end{macrocode}

%\iffalse
%</samplefinal>
%\fi
%
% %%%%%%%%%%%%%%%%%%%%%%%%%%%%%%%%%%%%%%
% \paragraph{Command Line Processing.}
%
% The following three command lines generate the output files
% |cdocscld|, |cdocscl1| and |cdocscl2|
% which should be identical to
% |cdocsdrf|, |cdocsch1| and |cdocsfn2|, respectively:
% \begin{center}
% \begin{tabular}{l}
% |latex -jobname cdocscld \|\\
% |  "\def\version{draft}\input{childdoc.def}\childdocforward{cdocsamp}"|\\
% |latex -jobname cdocscl1 \|\\
% |  "\input{childdoc.def}\childdocforward[cdocsamp]{cdocsch1}"|\\
% |latex -jobname cdocscl2 \|\\
% |  "\def\version{final}\input{childdoc.def}\childdocforward{cdocsch2}"|
% \end{tabular}
% \end{center}
% Note that the trailing backslash on each first line
% merely continues the input to the second line
% (for convenient cut ant paste).
% Furthermore, the command |latex| can be replaced by any
% of its alternative versions such as |pdflatex|.
%
% %%%%%%%%%%%%%%%%%%%%%%%%%%%%%%%%%%%%%%%%%%%%%%%%%%%%%%%%%%%%%%%%%%%%%%%%%%%%%%
% %%%%%%%%%%%%%%%%%%%%%%%%%%%%%%%%%%%%%%%%%%%%%%%%%%%%%%%%%%%%%%%%%%%%%%%%%%%%%%
% \section{Implementation}
%\iffalse
%<*package>
%\fi
%
% This section describes the definitions file |childdoc.def|.

% The definitions cannot be loaded using |\usepackage| or |\RequirePackage|
% which has a mechanism to prevent loading a style file more than once.
% When loading the definitions by means of |\input|
% multiple instances have to be prevented manually:
%\iffalse
%This code needs to be before the `\ProvidesFile' directive
%which is defined at the beginning of this file.
%Therefore it is also placed there and commented out here.
%</package>
%<*discard>
%\fi
%    \begin{macrocode}
\ifdefined\childdocmain\endinput\fi
%    \end{macrocode}
%\iffalse
%</discard>
%<*package>
%\fi
%
% \macro{\ifchilddoc}
% \macro{\ifchilddocmanual}
% The conditional |\ifchilddoc| tells whether a
% child (true) or main (false) document is being compiled.
% The conditional |\ifchilddocmanual| tells whether
% the |\includeonly| mechanism is used (false) or
% the selection of child files must be performed manually (true).
% The definitions initialise to false:
%    \begin{macrocode}
\newif\ifchilddoc
\newif\ifchilddocmanual
%    \end{macrocode}

% \macro{\childdocname}
% \macro{\childdocjob}
% The macro |\childdocname| stores the name of the main document
% to be compiled. The macro |\childdocjob| stores the name of
% the document on which the \LaTeX{} compiler was originally invoked.
% The content of |\jobname| cannot be compared
% to filenames specified in the source due to different catcodes.
% The following code rescans |\jobname|, stores the result
% in |\childdocname| and saves a copy in |\childdocjob|:
%    \begin{macrocode}
\edef\childdocname{\scantokens\expandafter{\jobname\noexpand}}
\let\childdocjob\childdocname
%    \end{macrocode}

% \macro{\childdocdisable}
% The macro |\childdocdisable| prevents the main file
% from being processed more than once.
% At this stage, the main document command |\childdocmain|
% is assumed to be called once again where it should do nothing.
% Any subsequent call to it should prevent
% a secondary processing of the main document
% It overwrites the forwarding commands
% |\childdocof| and |\childdocforward|
% with empty macros to prevent further inclusions of the main document:
%    \begin{macrocode}
\newcommand{\childdocdisable}
{
  \renewcommand{\childdocmain}[1]{\renewcommand{\childdocmain}[1]{\endinput}}
  \renewcommand{\childdocof}[1]{}
  \renewcommand{\childdocby}[2][]{}
  \renewcommand{\childdocforward}[2][]{}
  \renewcommand{\childdocdisable}{}
}
%    \end{macrocode}

% \macro{\childdocmain}
% The macro |\childdocmain| is to be called at the top of the main file
% with nothing or the main filename (without extension) as argument.
% First, it breaks loops.
% If the argument is not empty and does not match |\childdocname|
% (which is set by the first inclusion of |childdoc.def|),
% |\ifchilddoc| is set to true, |\includeonly| is applied to the child file
% and |\jobname| is set to the main file
% (for proper handling of |.aux| files):
%    \begin{macrocode}
\newcommand{\childdocmain}[1]
{
  \childdocdisable\childdocmain{}
  \if?#1?\else
    \begingroup
      \def\childdoctmp{#1}
      \ifx\childdoctmp\childdocname
        \def\childdoctmp{}
      \else
        \def\childdoctmp
        {
          \childdoctrue
          \includeonly{\childdocname}
          \def\childdocjob{#1}
          \def\jobname{#1}
        }
      \fi
      \expandafter
    \endgroup
    \childdoctmp
  \fi
}
%    \end{macrocode}

% \macro{\childdocof}
% The command |\childdocof| redirects
% compilation to the main file |#1|.
%    \begin{macrocode}
\newcommand{\childdocof}[1]
{
  \childdocdisable
  \childdoctrue
  \includeonly{\childdocname}
  \def\jobname{#1}
  \def\childdocjob{#1}
  \input{#1}
}
%    \end{macrocode}

% \macro{\childdocby}
% The command |\childdocby| ....
%    \begin{macrocode}
\newcommand{\childdocby}[2][]
{
  \childdocdisable
  \childdoctrue
  \childdocmanualtrue
  \if?#1?\else
    \def\jobname{#2}
  \fi
  \def\childdocjob{#2}
  \input{#2}
  \endinput
}
%    \end{macrocode}

% \macro{\childdocforward}
% The command |\childdocforward| redirects
% compilation to the main file or
% (if the optional argument is given) a child file.
% Parameters are set as if the main file
% or a child file starting with |\childdocof| was compiled.
% Then compilation is handed over to the main file:
%    \begin{macrocode}
\newcommand{\childdocforward}[2][]
{
  \begingroup
    \if?#1?
      \def\childdoctmp
      {
        \def\childdocname{#2}
        \def\childdocjob{#2}
        \def\jobname{#2}
        \input{#2}
        \endinput
      }
    \else
      \def\childdoctmp
      {
        \childdocdisable
        \def\childdocname{#2}
        \childdoctrue
        \includeonly{#2}
        \def\childdocjob{#1}
        \def\jobname{#1}
        \input{#1}
        \endinput
      }
    \fi
    \expandafter
  \endgroup
  \childdoctmp
}
%    \end{macrocode}

% \macro{\childdocforwardprefix}
% The command |\childdocforwardprefix| redirects
% compilation to the main or a child file by means of a pattern.
% The prefix |#1| in the current filename is replaced by |#2|
% and the suffix of the current filename is kept
% (it is assumed that the filename does not contain the substring `|~~~|'
% which is used as a delimiter).
% Compilation is handed over to the new file by |\childdocforward|:
%    \begin{macrocode}
\newcommand{\childdocforwardprefix}[3][]
{
  \begingroup
    \def\childdocextract #2##1~~~{\def\childdoctmp{\childdocforward[#1]{#3##1}}}
    \expandafter\childdocextract\childdocname~~~
    \expandafter
  \endgroup
  \childdoctmp
}
%    \end{macrocode}

% \macro{\childdoc}
% The deprecated macro |\childdoc| is a legacy version of |\childdocmain|:
%    \begin{macrocode}
\newcommand{\childdoc}{\childdocmain}
%    \end{macrocode}

% \macro{\childdocredirect}
% The deprecated macro |\childdocredirect| is a legacy version
% of |\childdocforward| and |\childdocforwardprefix|:
%    \begin{macrocode}
\newcommand{\childdocredirect}[2][]
{
  \begingroup
    \if?#1?
      \def\childdoctmp{\childdocforward{#2}}
    \else
      \def\childdoctmp{\childdocforwardprefix{#1}{#2}}
    \fi
    \expandafter
  \endgroup
  \childdoctmp
}
%    \end{macrocode}

%\iffalse
%</package>
%\fi
%
\endinput
|\\
|\childdocof{|\textit{main}|}|\\
\end{tabular}
\end{center}
at the top of every child file \textit{child}
which is included by |\include{|\textit{child}|}|
from within the main file
(or at least for those files to be compiled individually).
The argument \textit{main} must be the filename of the main file.

There are a couple of
considerations in setting up the main and child documents:

%%%%%%%%%%%%%%%%%%%%%%%%%%%%%%%%%%%%%%%%
\paragraph{Restrictions.}

Please note the following restrictions:
\begin{itemize}
\item
|\childdocmain| must be called with one argument \textit{main}
to ensure compatibility with earlier version of the package.
It must either be empty (|\childdocmain{}|)
or precisely match the filename of the main file in which it is specified.
See \secref{sec:detection} for further information.
\item
The filename \textit{main} must be specified without the |.tex| extension.
\item
The filename \textit{main} is case sensitive
(even in case-insensitive file systems)
due to internal string comparison.
\item
The argument \textit{main} should be fully expanded, it cannot be a macro.
\item
Subdirectories and special characters should be avoided in filenames.
\item
The command |\childdocmain{|\textit{main}|}| must be followed by a whitespace.
It should not be followed immediately by another command
or by a comment mark `|%|'.
This is because the \TeX{} parser reads the token immediately following
the argument of |\childdocmain| and puts it
at the beginning of every child section;
however, a white\-space is ignored.
\end{itemize}

%%%%%%%%%%%%%%%%%%%%%%%%%%%%%%%%%%%%%%%%
\paragraph{Content of Main File.}

It is advisable to place all content in the child files included by |\include|.
Any output contained in the main file will appear in all child documents
unless suppressed manually;
it cannot be suppressed automatically by the |\includeonly| directive
and thus should normally be avoided.
A method to include some content in the main file
by means of conditional processing is described in \secref{sec:conditional}.

%%%%%%%%%%%%%%%%%%%%%%%%%%%%%%%%%%%%%%%%
\paragraph{Page Numbering.}

When only a part of the document is compiled,
the appropriate numbering of pages
(as well as other status parameters)
is determined from the |.aux| files.
The latter contain information from previous passes.
However this information needs to propagate through
all intermediate child documents.
Therefore the page numbering in child documents may well
be inconsistent until the complete document is compiled at least once.

A useful (if unconventional) way to always ensure a consistent
page numbering is to restart the numbering in each child document
and denote the pages by `\textit{child}|.|\textit{page}'
where \textit{child} represents the chapter/section number of the child file.
This can be achieved by the command
|\numberwithin{page}{|\textit{child}|}|
of the \textsf{amsmath} package
where \textit{child} can be |chapter| or |section|
depending on the chosen structuring.
Alternatively, one can modify the macro |\thepage| appropriately
and reset the counter |page| at the start of each child file.

%%%%%%%%%%%%%%%%%%%%%%%%%%%%%%%%%%%%%%%%%%%%%%%%%%%%%%%%%%%%%%%%%%%%%%%%%%%%%%%%
\subsection{Conditional Processing}
\label{sec:conditional}

The package provides a mechanism to compile different versions
of a document. To customise the versions further some conditional processing
can come in handy to distinguish which version is being compiled.
The package provides two macros to describe the compilation context:

%%%%%%%%%%%%%%%%%%%%%%%%%%%%%%%%%%%%%%%%
\DescribeMacro{\ifchilddoc}
The conditional |\ifchilddoc| distinguishes between the compilation of
child documents and the main document:
%
\begin{center}
|\ifchilddoc |\textit{child-code}| |[|\||else |\textit{main-code}]| \||fi|
\end{center}

%%%%%%%%%%%%%%%%%%%%%%%%%%%%%%%%%%%%%%%%
\DescribeMacro{\childdocname}
\DescribeMacro{\childdocjob}
The macro |\childdocname| contains the filename (without extension)
of the main or child file being processed.
Note that |\childdocjob| will always contain the name of the main file.

%%%%%%%%%%%%%%%%%%%%%%%%%%%%%%%%%%%%%%%%
\paragraph{Title Page.}

Conditional processing can be used to include a title or banner page
in the main document when proper precautions are taken.
Importantly, the code in the main file should ensure that the page counter
(as well as other status parameters which are stored in the |.aux| files)
takes the same value after the conditional processing.
Otherwise the page numbers may take divergent values
depending on which part is compiled.

For example, a title page could be declared by:
%
\begin{center}
\begin{tabular}{l}
|\ifchilddoc\||else|\\
|\addtocounter{page}{-1}|\\
\textit{code for title page}\\
|\newpage|\\
|\||fi|
\end{tabular}
\end{center}
%
A banner page for the child documents can be generated by:
%
\begin{center}
\begin{tabular}{l}
|\ifchilddoc|\\
|\addtocounter{page}{-1}|\\
\textit{code for banner page}\\
|\newpage|\\
|\||fi|
\end{tabular}
\end{center}
%
Here one could write a message such as:
\begin{center}
|This is the part \childdocname{} of \childdocjob{}.|
\end{center}

%%%%%%%%%%%%%%%%%%%%%%%%%%%%%%%%%%%%%%%%%%%%%%%%%%%%%%%%%%%%%%%%%%%%%%%%%%%%%%%%
\subsection{Flags}
\label{sec:flags}

The package makes it easy to generate different versions
of the main or child documents.
To this end compilation flags can be defined
and assigned different default values.
They will be particularly useful in conjunction
with the forwarding mechanism described in \secref{sec:forward}.

For example, it may be useful to have a flag |\version|
which can be set to |draft| or |final|.
The document source will contain some conditional code
depending on the value of |\version|.
Suppose further, the flag should default to |final| for the main file
and to |draft| for child files
which is a natural assignment for editing the document.
This is achieved by placing the following code
in the preamble of the main document
(below the |\childdocmain| directive):
%
\begin{center}
\begin{tabular}{l}
|\ifchilddoc|\\
|\providecommand{\version}{draft}|\\
|\||else|\\
|\providecommand{\version}{final}|\\
|\||fi|
\end{tabular}
\end{center}
%
The definition by |\providecommand| makes sure
that previous definitions are not overwritten.
Further statements |\providecommand{\version}{...}|
can thus be added before the above code to override it.

For the main file, one might add a line
(between |\childdocmain| and the above block)
%
\begin{center}
|%\ifchilddoc\||else\providecommand{\version}{draft}\||fi|
\end{center}
%
which can be uncommented to produce a draft version.
Likewise one can add a line to the very top of a child file
(above the |\childdocof{|\textit{main}|}| directive)
%
\begin{center}
|%\providecommand{\version}{final}|
\end{center}
%
which can be uncommented to produce the final version of this child document.

%%%%%%%%%%%%%%%%%%%%%%%%%%%%%%%%%%%%%%%%%%%%%%%%%%%%%%%%%%%%%%%%%%%%%%%%%%%%%%%%
\subsection{Forwarding}
\label{sec:forward}

Different versions of the main or child documents
using compilation flags as described in \secref{sec:flags}
can be (permanently) stored in different files
for convenient compilation, viewing and distribution.
To this end, the package defines a command
to pass on compilation to a different file:

%%%%%%%%%%%%%%%%%%%%%%%%%%%%%%%%%%%%%%%%
\DescribeMacro{\childdocforward}
The command |\childdocforward| redirects processing to
another source file:
%
\begin{center}
\begin{tabular}{l}
|% \iffalse
%
% childdoc.dtx Copyright (C) 2017-2018 Niklas Beisert
%
% This work may be distributed and/or modified under the
% conditions of the LaTeX Project Public License, either version 1.3
% of this license or (at your option) any later version.
% The latest version of this license is in
%   http://www.latex-project.org/lppl.txt
% and version 1.3 or later is part of all distributions of LaTeX
% version 2005/12/01 or later.
%
% This work has the LPPL maintenance status `maintained'.
%
% The Current Maintainer of this work is Niklas Beisert.
%
% This work consists of the files childdoc.dtx and childdoc.ins
% and the derived files childdoc.def and cdocsamp.tex with
% cdocsch1.tex, cdocsch2.tex, cdocsdrf.tex, cdocsfn1.tex, cdocsfn2.tex.
%
%<package>\ifdefined\childdocmain\endinput\fi
%<package>\ProvidesFile{childdoc.def}[2018/12/30 v2.0 child document driver]
%<samplemain>\ProvidesFile{cdocsamp.tex}[2018/12/30 v2.0 sample for childdoc]
%<*driver>
%\ProvidesFile{childdoc.drv}[2018/12/30 v2.0 childdoc reference manual file]
\PassOptionsToClass{10pt,a4paper}{article}
\documentclass{ltxdoc}

\usepackage[margin=35mm]{geometry}
\usepackage{hyperref}
\usepackage{hyperxmp}
\usepackage[usenames]{color}

\hypersetup{colorlinks=true}
\hypersetup{pdfstartview=FitH}
\hypersetup{pdfpagemode=UseNone}
\hypersetup{pdfsource={}}
\hypersetup{pdflang={en-UK}}
\hypersetup{pdfcopyright={Copyright 2017-2018 Niklas Beisert.
  This work may be distributed and/or modified under the
  conditions of the LaTeX Project Public License, either version 1.3
  of this license or (at your option) any later version.}}
\hypersetup{pdflicenseurl={http://www.latex-project.org/lppl.txt}}
\hypersetup{pdfcontactaddress={ETH Zurich, ITP, HIT K,
  Wolfgang-Pauli-Strasse 27}}
\hypersetup{pdfcontactpostcode={8093}}
\hypersetup{pdfcontactcity={Zurich}}
\hypersetup{pdfcontactcountry={Switzerland}}
\hypersetup{pdfcontactemail={nbeisert@itp.phys.ethz.ch}}
\hypersetup{pdfcontacturl={http://people.phys.ethz.ch/\xmptilde nbeisert/}}

\newcommand{\secref}[1]{\hyperref[#1]{section \ref*{#1}}}

\parskip1ex
\parindent0pt
\let\olditemize\itemize
\def\itemize{\olditemize\parskip0pt}

\begin{document}

\title{The \textsf{childdoc} Package}
\hypersetup{pdftitle={The childdoc Package}}
\author{Niklas Beisert\\[2ex]
  Institut f\"ur Theoretische Physik\\
  Eidgen\"ossische Technische Hochschule Z\"urich\\
  Wolfgang-Pauli-Strasse 27, 8093 Z\"urich, Switzerland\\[1ex]
  \href{mailto:nbeisert@itp.phys.ethz.ch}
  {\texttt{nbeisert@itp.phys.ethz.ch}}}
\hypersetup{pdfauthor={Niklas Beisert}}
\hypersetup{pdfsubject={Manual for the LaTeX2e Package childdoc}}
\date{30 December 2018, \textsf{v2.0}}
\maketitle

\begin{abstract}\noindent
\textsf{childdoc} is a \LaTeXe{} package
that enables the direct compilation
of document sections included by |\include|
to individual files.
\end{abstract}

\begingroup
\parskip0ex
\tableofcontents
\endgroup

%%%%%%%%%%%%%%%%%%%%%%%%%%%%%%%%%%%%%%%%%%%%%%%%%%%%%%%%%%%%%%%%%%%%%%%%%%%%%%%%
%%%%%%%%%%%%%%%%%%%%%%%%%%%%%%%%%%%%%%%%%%%%%%%%%%%%%%%%%%%%%%%%%%%%%%%%%%%%%%%%
\section{Introduction}

\LaTeX{} provides a mechanism to structure a large document (such as a book)
into a main file and several child files (containing the chapters)
using the |\include| command.
This mechanism is beneficial for documents
which span hundreds of pages in order to
make the source file(s) more manageable.
Moreover, compilation can be restricted to
selected child files by means of the |\includeonly| command.
The latter feature can be used to reduce the compilation time while editing
(this was significantly more useful in the earlier days of \LaTeX{})
or to generate a smaller document which is easier to navigate.
Another application of |\includeonly| is to generate
documents consisting of selected parts of the complete document.

However, there are a few drawbacks of the plain |\include| mechanism:
\begin{itemize}
\item
The child files cannot be compiled on their own,
they can only be compiled via the main file.
A naive editing environment
(such as a text editor with an option
to have the current file processed by \LaTeX)
may require one to switch to the main file before compiling;
attempting to compile the child file produces errors.
\item
The main file must be modified (each time)
to adjust the |\includeonly| command
to the present needs. This easily leaves the main file in a messy state.
\item
The generated document will always carry the filename
of the main document. This is inconvenient if
several child files are to be compiled and
to be kept for distribution.
\end{itemize}

The present package provides a simple interface
to make child files individually compilable by \LaTeX{}.
Compiling a child file then has the same effect as compiling
the main file with an |\includeonly| command
to select the appropriate child.
Moreover the generated document will carry the name of the child
rather than the main file.
This resolves all three above issues.

This feature is meant to make the editing of books,
thesis documents and lecture notes somewhat more convenient.
However, the package can also be used efficiently for
composing a series of documents (such as exercise sheets)
which are typically distributed individually.
It then assists the author in generating the individual documents
(potentially in different versions)
as well as a document containing the collected series.
Another application is in developing style files
or other kinds of included material
where compilation of the style file could redirect
to a sample or test file.

%%%%%%%%%%%%%%%%%%%%%%%%%%%%%%%%%%%%%%%%%%%%%%%%%%%%%%%%%%%%%%%%%%%%%%%%%%%%%%%%
%%%%%%%%%%%%%%%%%%%%%%%%%%%%%%%%%%%%%%%%%%%%%%%%%%%%%%%%%%%%%%%%%%%%%%%%%%%%%%%%
\section{Usage}

First of all, the package \textsf{childdoc} is \emph{not} a standard
\LaTeXe{} |.sty| style file! Therefore it needs to be invoked in
a non-standard way.

%%%%%%%%%%%%%%%%%%%%%%%%%%%%%%%%%%%%%%%%%%%%%%%%%%%%%%%%%%%%%%%%%%%%%%%%%%%%%%%%
\subsection{Included Files}
\label{sec:include}

%%%%%%%%%%%%%%%%%%%%%%%%%%%%%%%%%%%%%%%%
\DescribeMacro{\childdocmain}
To use the package, add the commands
\begin{center}
\begin{tabular}{l}
|\input{childdoc.def}|\\
|\childdocmain{}|\\
\end{tabular}
\end{center}
at the very top of the main \LaTeX{} file,
in particular \emph{before} the |\documentclass| statement!
The argument of |\childdocmain| should be left empty
(but it must be present).

%%%%%%%%%%%%%%%%%%%%%%%%%%%%%%%%%%%%%%%%
\DescribeMacro{\childdocof}
Furthermore, add the commands
\begin{center}
\begin{tabular}{l}
|\input{childdoc.def}|\\
|\childdocof{|\textit{main}|}|\\
\end{tabular}
\end{center}
at the top of every child file \textit{child}
which is included by |\include{|\textit{child}|}|
from within the main file
(or at least for those files to be compiled individually).
The argument \textit{main} must be the filename of the main file.

There are a couple of
considerations in setting up the main and child documents:

%%%%%%%%%%%%%%%%%%%%%%%%%%%%%%%%%%%%%%%%
\paragraph{Restrictions.}

Please note the following restrictions:
\begin{itemize}
\item
|\childdocmain| must be called with one argument \textit{main}
to ensure compatibility with earlier version of the package.
It must either be empty (|\childdocmain{}|)
or precisely match the filename of the main file in which it is specified.
See \secref{sec:detection} for further information.
\item
The filename \textit{main} must be specified without the |.tex| extension.
\item
The filename \textit{main} is case sensitive
(even in case-insensitive file systems)
due to internal string comparison.
\item
The argument \textit{main} should be fully expanded, it cannot be a macro.
\item
Subdirectories and special characters should be avoided in filenames.
\item
The command |\childdocmain{|\textit{main}|}| must be followed by a whitespace.
It should not be followed immediately by another command
or by a comment mark `|%|'.
This is because the \TeX{} parser reads the token immediately following
the argument of |\childdocmain| and puts it
at the beginning of every child section;
however, a white\-space is ignored.
\end{itemize}

%%%%%%%%%%%%%%%%%%%%%%%%%%%%%%%%%%%%%%%%
\paragraph{Content of Main File.}

It is advisable to place all content in the child files included by |\include|.
Any output contained in the main file will appear in all child documents
unless suppressed manually;
it cannot be suppressed automatically by the |\includeonly| directive
and thus should normally be avoided.
A method to include some content in the main file
by means of conditional processing is described in \secref{sec:conditional}.

%%%%%%%%%%%%%%%%%%%%%%%%%%%%%%%%%%%%%%%%
\paragraph{Page Numbering.}

When only a part of the document is compiled,
the appropriate numbering of pages
(as well as other status parameters)
is determined from the |.aux| files.
The latter contain information from previous passes.
However this information needs to propagate through
all intermediate child documents.
Therefore the page numbering in child documents may well
be inconsistent until the complete document is compiled at least once.

A useful (if unconventional) way to always ensure a consistent
page numbering is to restart the numbering in each child document
and denote the pages by `\textit{child}|.|\textit{page}'
where \textit{child} represents the chapter/section number of the child file.
This can be achieved by the command
|\numberwithin{page}{|\textit{child}|}|
of the \textsf{amsmath} package
where \textit{child} can be |chapter| or |section|
depending on the chosen structuring.
Alternatively, one can modify the macro |\thepage| appropriately
and reset the counter |page| at the start of each child file.

%%%%%%%%%%%%%%%%%%%%%%%%%%%%%%%%%%%%%%%%%%%%%%%%%%%%%%%%%%%%%%%%%%%%%%%%%%%%%%%%
\subsection{Conditional Processing}
\label{sec:conditional}

The package provides a mechanism to compile different versions
of a document. To customise the versions further some conditional processing
can come in handy to distinguish which version is being compiled.
The package provides two macros to describe the compilation context:

%%%%%%%%%%%%%%%%%%%%%%%%%%%%%%%%%%%%%%%%
\DescribeMacro{\ifchilddoc}
The conditional |\ifchilddoc| distinguishes between the compilation of
child documents and the main document:
%
\begin{center}
|\ifchilddoc |\textit{child-code}| |[|\||else |\textit{main-code}]| \||fi|
\end{center}

%%%%%%%%%%%%%%%%%%%%%%%%%%%%%%%%%%%%%%%%
\DescribeMacro{\childdocname}
\DescribeMacro{\childdocjob}
The macro |\childdocname| contains the filename (without extension)
of the main or child file being processed.
Note that |\childdocjob| will always contain the name of the main file.

%%%%%%%%%%%%%%%%%%%%%%%%%%%%%%%%%%%%%%%%
\paragraph{Title Page.}

Conditional processing can be used to include a title or banner page
in the main document when proper precautions are taken.
Importantly, the code in the main file should ensure that the page counter
(as well as other status parameters which are stored in the |.aux| files)
takes the same value after the conditional processing.
Otherwise the page numbers may take divergent values
depending on which part is compiled.

For example, a title page could be declared by:
%
\begin{center}
\begin{tabular}{l}
|\ifchilddoc\||else|\\
|\addtocounter{page}{-1}|\\
\textit{code for title page}\\
|\newpage|\\
|\||fi|
\end{tabular}
\end{center}
%
A banner page for the child documents can be generated by:
%
\begin{center}
\begin{tabular}{l}
|\ifchilddoc|\\
|\addtocounter{page}{-1}|\\
\textit{code for banner page}\\
|\newpage|\\
|\||fi|
\end{tabular}
\end{center}
%
Here one could write a message such as:
\begin{center}
|This is the part \childdocname{} of \childdocjob{}.|
\end{center}

%%%%%%%%%%%%%%%%%%%%%%%%%%%%%%%%%%%%%%%%%%%%%%%%%%%%%%%%%%%%%%%%%%%%%%%%%%%%%%%%
\subsection{Flags}
\label{sec:flags}

The package makes it easy to generate different versions
of the main or child documents.
To this end compilation flags can be defined
and assigned different default values.
They will be particularly useful in conjunction
with the forwarding mechanism described in \secref{sec:forward}.

For example, it may be useful to have a flag |\version|
which can be set to |draft| or |final|.
The document source will contain some conditional code
depending on the value of |\version|.
Suppose further, the flag should default to |final| for the main file
and to |draft| for child files
which is a natural assignment for editing the document.
This is achieved by placing the following code
in the preamble of the main document
(below the |\childdocmain| directive):
%
\begin{center}
\begin{tabular}{l}
|\ifchilddoc|\\
|\providecommand{\version}{draft}|\\
|\||else|\\
|\providecommand{\version}{final}|\\
|\||fi|
\end{tabular}
\end{center}
%
The definition by |\providecommand| makes sure
that previous definitions are not overwritten.
Further statements |\providecommand{\version}{...}|
can thus be added before the above code to override it.

For the main file, one might add a line
(between |\childdocmain| and the above block)
%
\begin{center}
|%\ifchilddoc\||else\providecommand{\version}{draft}\||fi|
\end{center}
%
which can be uncommented to produce a draft version.
Likewise one can add a line to the very top of a child file
(above the |\childdocof{|\textit{main}|}| directive)
%
\begin{center}
|%\providecommand{\version}{final}|
\end{center}
%
which can be uncommented to produce the final version of this child document.

%%%%%%%%%%%%%%%%%%%%%%%%%%%%%%%%%%%%%%%%%%%%%%%%%%%%%%%%%%%%%%%%%%%%%%%%%%%%%%%%
\subsection{Forwarding}
\label{sec:forward}

Different versions of the main or child documents
using compilation flags as described in \secref{sec:flags}
can be (permanently) stored in different files
for convenient compilation, viewing and distribution.
To this end, the package defines a command
to pass on compilation to a different file:

%%%%%%%%%%%%%%%%%%%%%%%%%%%%%%%%%%%%%%%%
\DescribeMacro{\childdocforward}
The command |\childdocforward| redirects processing to
another source file:
%
\begin{center}
\begin{tabular}{l}
|\input{childdoc.def}|\\
|\childdocforward[|\textit{main}|]{|\textit{dest}|}|\\
\end{tabular}
\end{center}
%
The argument \textit{dest} is the destination file
(without extension).
It should be the main file or one of the child files.
Note that further \textsf{childdoc} directives
such as |\childdocof| and |\childdocforward|
in the indicated file will be processed in this form.
The optional argument \textit{main}
passes on directly to the main file \textit{main}
while pretending to compile the child \textit{dest}.
This form behaves as if \textit{dest}
issues |\childdocof{|\textit{main}|}| right away,
and no further \textsf{childdoc} directives will be processed.

%%%%%%%%%%%%%%%%%%%%%%%%%%%%%%%%%%%%%%%%
\DescribeMacro{\...prefix}
In the alternative form |\childdocforwardprefix|,
%
\begin{center}
\begin{tabular}{l}
|\input{childdoc.def}|\\
|\childdocforwardprefix[|\textit{main}|]{|\textit{prefix}|}{|\textit{dest}|}|
\end{tabular}
\end{center}
%
the destination file is determined by a pattern
depending on the current file:
To make this work, the current file must be called
`{\textit{prefix}\hspace{0.2em}\textit{suffix}}'
with \textit{prefix} matching precisely the argument.
Processing is then passed on to the file
`{\textit{dest}\hspace{0.2em}\textit{suffix}}'.
Surely, the same effect is achieved by
directly specifying the
argument `{\textit{dest}\hspace{0.2em}\textit{suffix}}'
in the first form.
However, that requires to set up a different file
for each child. With the alternative form of the command
all these files can have exactly the same content
which simplifies setting them up and maintaining them.

For example, the following file |draft.tex|
with a compilation flag |\version| as described in \secref{sec:flags}
compiles the main document as a draft:
%
\begin{center}
\begin{tabular}{l}
|\def\version{draft}|\\
|\input{childdoc.def}|\\
|\childdocforward{|\textit{main}|}|
\end{tabular}
\end{center}
%
Likewise, the following files |final|\textit{nn}|.tex|
compile the final version of the child document
|child|\textit{nn}|.tex|:
%
\begin{center}
\begin{tabular}{l}
|\def\version{final}|\\
|\input{childdoc.def}|\\
|\childdocforwardprefix{final}{child}|
\end{tabular}
\end{center}
%

Note that when several versions of a main file and/or of each child file
are to be generated, it may be convenient to set up a |Makefile| or
shell script to automatise the process.

%%%%%%%%%%%%%%%%%%%%%%%%%%%%%%%%%%%%%%%%%%%%%%%%%%%%%%%%%%%%%%%%%%%%%%%%%%%%%%%%
\subsection{Command Line Processing}
\label{sec:commandline}

The effect of redirection files can also be achieved by invoking
the \LaTeX{} compiler with a more elaborate command line.
Most conveniently this should be done as part
of a shell script or a |Makefile|.

When using \textsf{childdoc} in the main file, the following
command lines effectively perform a redirection
(note that depending on the shell being used,
backslashes may have to be doubled: `|\|' $\to$ `|\\|'):
%
\begin{center}
|... -jobname "|\textit{target}|" |\\|"|[\textit{flags}]%
|\input{childdoc.def}\childdocforward[|\textit{main}|]{|\textit{dest}|}"|
\end{center}
%
Here \textit{target} is the name of the output file,
\textit{main} is the name of the main file
and \textit{dest} is the name of the main or child file to be processed
(all filenames without extensions).
The optional argument \textit{main} can be omitted
if \textit{main} matches \textit{dest}.
Optionally, compilation \textit{flags} can be defined via |\def| commands.
This command line makes the \TeX{} engine believe
it is compiling the file \textit{target}
whose content is specified as the latter parameter.
The provided code then forwards the processing to
\textit{main} or \textit{dest} as described in \secref{sec:forward}.

%%%%%%%%%%%%%%%%%%%%%%%%%%%%%%%%%%%%%%%%%%%%%%%%%%%%%%%%%%%%%%%%%%%%%%%%%%%%%%%%
\subsection{Include by Input}
\label{sec:input}

Including child documents by |\include| has some restrictions by design.
Most notably, the content of a child document always occupies
its own set of pages; pages cannot be shared between child documents.
Usually, this behaviour makes perfect sense
because each child document contain an essential part of the document.
However, in some situations it may be desirable to compose
a document from a collection of parts
without having mandatory page breaks between then.
For this case, the package
provides a mechanism to include parts
by |\input| which can also be processed individually.
However, by construction this mechanism
requires manual handling of the content to be output.

%%%%%%%%%%%%%%%%%%%%%%%%%%%%%%%%%%%%%%%%
\DescribeMacro{\ifchilddocmanual}
The main file should be prepared as usual, see \secref{sec:include}.
However, the document body must make a distinction
between processing of an individual part and of the main document, e.g.:
%
\begin{center}
\begin{tabular}{l}
|\ifchilddocmanual|\\
|\input{\childdocname}|\\
|\||else|\\
\textit{document body with }|\input{|\textit{part}|}|\\
|\||fi|
\end{tabular}
\end{center}
%
The conditional |\ifchilddocmanual| is true whenever
a part to be included by |\input| is being compiled,
and the name of the part is stored in |\childdocname|.

%%%%%%%%%%%%%%%%%%%%%%%%%%%%%%%%%%%%%%%%
\DescribeMacro{\childdocby}
Each part to be included by |\input| should start with:
%
\begin{center}
\begin{tabular}{l}
|\input{childdoc.def}|\\
|\childdocby{|\textit{main}|}|\\
\end{tabular}
\end{center}
%
The directive |\childdocby| is similar to |\childdocof|
described in \secref{sec:include},
but the subsequent selection of content must be done manually.
To that end, both |\ifchilddoc| and |\ifchilddocmanual|
will be true upon processing of a part,
and the name of the part is stored in |\childdocname|.
Note that |\jobname| will be set to the filename of the current part
so that each part receives an individual |.aux| file
that does not interfere with the |.aux| file(s) of the main document.
This behaviour can be altered by the alternative form
|\childdocby[*]{|\textit{main}|}| (with a non-empty optional argument)
which uses the |.aux| file of the main document
by setting |\jobname| to \textit{main}.

%%%%%%%%%%%%%%%%%%%%%%%%%%%%%%%%%%%%%%%%%%%%%%%%%%%%%%%%%%%%%%%%%%%%%%%%%%%%%%%%
\subsection{Driver Development}
\label{sec:driver}

The \textsf{childdoc} mechanism can also be use for the development
of definition files such as \LaTeX{} styles or classes.
This case differs from the above setup with multiple parts
included by |\include| in that no |\includeonly| should be invoked.
This can be achieved by starting the include file
(before |\ProvidesPackage|) with:
%
\begin{center}
\begin{tabular}{l}
|\input{childdoc.def}|\\
|\childdocforward{|\textit{main}|}|\\
\end{tabular}
\end{center}
%
or alternatively with:
%
\begin{center}
\begin{tabular}{l}
|\input{childdoc.def}|\\
|\childdocby{|\textit{main}|}|\\
\end{tabular}
\end{center}
%
Both forms have slightly different effects as described above.
The main file is prepared as usual, see \secref{sec:include}.

%%%%%%%%%%%%%%%%%%%%%%%%%%%%%%%%%%%%%%%%%%%%%%%%%%%%%%%%%%%%%%%%%%%%%%%%%%%%%%%%
\subsection{Legacy Detection}
\label{sec:detection}

The directive |\childdocmain| in the main file can detect
whether the complete document or merely a child is to be compiled
even without using the directive |\childdocof|.
This method is deprecated because it is less robust
and there is no compelling reason to use it;
it is merely provided for backward compatibility
and it may be removed in future versions.

If the detection mechanism is to be used,
it is mandatory to correctly specify
the filename of the main file as the argument of |\childdocmain|:
%
\begin{center}
\begin{tabular}{l}
|\input{childdoc.def}|\\
|\childdocmain{|\textit{main}|}|\\
\end{tabular}
\end{center}
%
If |\jobname| does not match the argument \textit{main} of |\childdocmain|,
it is assumed that |\jobname| points to the child file to be compiled.
When using |\childdocmain| with the main file specified as argument,
it suffices to start a child file
with just |\input{|\textit{main}|}|
without loading of the package and using |\childdocof|.
If instead all processing is done
with the appropriate \textsf{childdoc} directives,
the argument of \textit{main} of |\childdocmain| can be empty.

An alternative version of the command line processing described
in \secref{sec:commandline} using the detection mechanism reads:
%
\begin{center}
|... -jobname "|\textit{target}|" "|[\textit{flags}]%
[|\def\jobname{|\textit{dest}|}|]|\input{|\textit{main}|}"|
\end{center}

%%%%%%%%%%%%%%%%%%%%%%%%%%%%%%%%%%%%%%%%%%%%%%%%%%%%%%%%%%%%%%%%%%%%%%%%%%%%%%%%
\subsection{Manual Code}
\label{sec:manual}

In case one cannot be certain whether the definitions file |childdoc.def|
is installed on the target \TeX{} distribution
and one prefers not to ship it,
it is conceivable to paste a few relevant commands into the sources.

To that end, drop all statements |\input{childdoc.def}|
and perform the replacements as outlined below.
Instead of |\childdocmain{|\textit{main}|}| add the following code
to the top of the main file:
%
\begin{center}
\begin{tabular}{l}
|\||ifdefined\childdocname\endinput\||fi\newif\ifchilddoc|\\
|\edef\childdocname{\scantokens\expandafter{\jobname\noexpand}}|\\
|\def\childdocmain{|\textit{main}|}\||ifx\childdocmain\childdocname\||else|\\
|\childdoctrue\includeonly{\childdocname}\let\jobname\childdocmain\||fi|\\
\end{tabular}
\end{center}
%
Instead of |\childdocof{|\textit{main}|}| just include the main file
at the top of each child file:
%
\begin{center}
|\input{|\textit{main}|}|
\end{center}
%
A simple redirection |\childdocforward{|\textit{dest}|}| is achieved by:
%
\begin{center}
|\def\jobname{|\textit{dest}|}\input{\jobname}|
\end{center}
%
The redirection with prefix
|\childdocforwardprefix[|\textit{prefix}|]{|\textit{dest}|}|
is accomplished by:
%
\begin{center}
\begin{tabular}{l}
|{\edef\jobname{\scantokens\expandafter{\jobname\noexpand}}|\\
|\def\redirectjob |\textit{prefix}|#1~~~{\gdef\jobname{|\textit{dest}|#1}}|\\
|\expandafter\redirectjob\jobname~~~}\input{\jobname}|
\end{tabular}
\end{center}

In an alternative approach,
child documents can be compiled by a specific command line
without additional code or specific definitions:
%
\begin{center}
|... -jobname "|\textit{target}|" "|[\textit{flags}]%
|\includeonly{|\textit{dest}|}\input{|\textit{main}|}"|
\end{center}
%

%%%%%%%%%%%%%%%%%%%%%%%%%%%%%%%%%%%%%%%%%%%%%%%%%%%%%%%%%%%%%%%%%%%%%%%%%%%%%%%%
%%%%%%%%%%%%%%%%%%%%%%%%%%%%%%%%%%%%%%%%%%%%%%%%%%%%%%%%%%%%%%%%%%%%%%%%%%%%%%%%
\section{Information}

%%%%%%%%%%%%%%%%%%%%%%%%%%%%%%%%%%%%%%%%%%%%%%%%%%%%%%%%%%%%%%%%%%%%%%%%%%%%%%%%
\subsection{Copyright}

Copyright \copyright{} 2017--2018 Niklas Beisert

This work may be distributed and/or modified under the
conditions of the \LaTeX{} Project Public License, either version 1.3
of this license or (at your option) any later version.
The latest version of this license is in
  \url{http://www.latex-project.org/lppl.txt}
and version 1.3 or later is part of all distributions of \LaTeX{}
version 2005/12/01 or later.

This work has the LPPL maintenance status `maintained'.

The Current Maintainer of this work is Niklas Beisert.

This work consists of the files |README.txt|, |childdoc.ins| and |childdoc.dtx|
as well as the derived files |childdoc.def|, |cdocsamp.tex|
with |cdocsch1.tex|, |cdocsch2.tex|, |cdocspt3.tex|, |cdocspt4.tex|,
|cdocsdrf.tex|, |cdocsfn1.tex|, |cdocsfn2.tex|
as well as |childdoc.pdf|.

%%%%%%%%%%%%%%%%%%%%%%%%%%%%%%%%%%%%%%%%%%%%%%%%%%%%%%%%%%%%%%%%%%%%%%%%%%%%%%%%
\subsection{Files and Installation}

The package consists of the files:
%
\begin{center}
\begin{tabular}{ll}
    |README.txt|   & readme file \\
    |childdoc.ins| & installation file \\
    |childdoc.dtx| & source file \\
    |childdoc.def| & definition file \\
    |cdocsamp.tex| & sample main file \\
    |cdocsch1.tex| & sample include file \\
    |cdocsch2.tex| & sample include file \\
    |cdocspt3.tex| & sample part file \\
    |cdocspt4.tex| & sample part file \\
    |cdocsdrf.tex| & sample redirection file \\
    |cdocsfn1.tex| & sample redirection file \\
    |cdocsfn2.tex| & sample redirection file \\
    |childdoc.pdf| & manual
\end{tabular}
\end{center}
%
The distribution consists of the files
|README.txt|, |childdoc.ins| and |childdoc.dtx|.
%
\begin{itemize}
\item
Run (pdf)\LaTeX{} on |childdoc.dtx|
to compile the manual |childdoc.pdf| (this file).
\item
Run \LaTeX{} on |childdoc.ins| to create the definitions file |childdoc.def|
and the sample |cdocsamp.tex| with include files
|cdocsch1.tex|, |cdocsch2.tex|, |cdocspt3.tex|, |cdocspt4.tex|,
|cdocsdrf.tex|, |cdocsfn1.tex|, |cdocsfn2.tex|.
Then copy the file |childdoc.def| to an appropriate directory of your \LaTeX{}
distribution, e.g.\ \textit{texmf-root}|/tex/latex/childdoc|.
\end{itemize}

%%%%%%%%%%%%%%%%%%%%%%%%%%%%%%%%%%%%%%%%%%%%%%%%%%%%%%%%%%%%%%%%%%%%%%%%%%%%%%%%
\subsection{Related CTAN Packages}

There are several other packages which offer a similar functionality:
%
\begin{itemize}
\item
The packages
\href{http://ctan.org/pkg/docmute}{\textsf{docmute}},
\href{http://ctan.org/pkg/includex}{\textsf{includex}} and
\href{http://ctan.org/pkg/standalone}{\textsf{standalone}}
provide commands to include only the document body of
a child file thus allowing both files to be compiled individually.
\item
The packages \href{http://ctan.org/pkg/subdocs}{\textsf{subdocs}}
and \href{http://ctan.org/pkg/subfiles}{\textsf{subfiles}}
provide structures in which the main and child documents can be
encapsulated and allowing them to be compiled individually.
The inclusion mechanism is different from the conventional |\include|.
\item
The package \href{http://ctan.org/pkg/combine}{\textsf{combine}}
is an elaborate solution to combine several documents into one.
\end{itemize}
%
See also the CTAN topic \href{http://ctan.org/topic/subdocs}{\textsf{subdocs}}
for further related packages.
The present package differs from the above solutions in that
a document structure constructed with the conventional |\include| mechanism
just needs two extra commands at the top of every file
such that all constituent files can be compiled individually.

%%%%%%%%%%%%%%%%%%%%%%%%%%%%%%%%%%%%%%%%%%%%%%%%%%%%%%%%%%%%%%%%%%%%%%%%%%%%%%%%
%\subsection{Feature Suggestions}
%
%The following is a list of features which may be useful for future
%versions of this package:
%%
%\begin{itemize}
%\item
%\ldots
%\end{itemize}

%%%%%%%%%%%%%%%%%%%%%%%%%%%%%%%%%%%%%%%%%%%%%%%%%%%%%%%%%%%%%%%%%%%%%%%%%%%%%%%%
\subsection{Revision History}

%%%%%%%%%%%%%%%%%%%%%%%%%%%%%%%%%%%%%%%%
\paragraph{v2.0:} 2018/12/30

\begin{itemize}
\item
immediate forward processing
\item
added |\childdocby| mechanism
\item
manual restructured
\end{itemize}

%%%%%%%%%%%%%%%%%%%%%%%%%%%%%%%%%%%%%%%%
\paragraph{v1.6:} 2018/01/17

\begin{itemize}
\item
application for development of include files
\item
corrections to manual
\end{itemize}

%%%%%%%%%%%%%%%%%%%%%%%%%%%%%%%%%%%%%%%%
\paragraph{v1.5:} 2017/05/21

\begin{itemize}
\item
more complete structuring introduced
\item
|\childdocof| introduced
\item
|\childdoc| renamed to |\childdocmain|
\item
|\childredirect| renamed to |\childdocforward| and |\childdocforwardprefix|
and functionality expanded
\end{itemize}

%%%%%%%%%%%%%%%%%%%%%%%%%%%%%%%%%%%%%%%%
\paragraph{v1.0:} 2017/04/27

\begin{itemize}
\item
manual and install package
\item
first version published on CTAN
\end{itemize}

%%%%%%%%%%%%%%%%%%%%%%%%%%%%%%%%%%%%%%%%
\paragraph{v0.6:} 2017/04/26

\begin{itemize}
\item
redirection mechanism added
\end{itemize}

%%%%%%%%%%%%%%%%%%%%%%%%%%%%%%%%%%%%%%%%
\paragraph{v0.5:} 2017/04/26

\begin{itemize}
\item
functionality in definition file
\end{itemize}


%%%%%%%%%%%%%%%%%%%%%%%%%%%%%%%%%%%%%%%%%%%%%%%%%%%%%%%%%%%%%%%%%%%%%%%%%%%%%%%%
%%%%%%%%%%%%%%%%%%%%%%%%%%%%%%%%%%%%%%%%%%%%%%%%%%%%%%%%%%%%%%%%%%%%%%%%%%%%%%%%
%%%%%%%%%%%%%%%%%%%%%%%%%%%%%%%%%%%%%%%%%%%%%%%%%%%%%%%%%%%%%%%%%%%%%%%%%%%%%%%%
\appendix

\settowidth\MacroIndent{\rmfamily\scriptsize 000\ }

 \DocInput{childdoc.dtx}

\end{document}
%</driver>
% \fi
%
% %%%%%%%%%%%%%%%%%%%%%%%%%%%%%%%%%%%%%%%%%%%%%%%%%%%%%%%%%%%%%%%%%%%%%%%%%%%%%%
% %%%%%%%%%%%%%%%%%%%%%%%%%%%%%%%%%%%%%%%%%%%%%%%%%%%%%%%%%%%%%%%%%%%%%%%%%%%%%%
% \section{Sample}
%\iffalse
%<*samplemain>
%\fi
%
% The following presents a sample document
% with two chapters, two parts, a title page,
% a compile flag as well as three forwarding files to set the flag.
% It consists of eight |.tex| files:
% \begin{center}
% \begin{tabular}{ll}
% |cdocsamp.tex|&main file\\
% |cdocsch1.tex|&include file for chapter 1\\
% |cdocsch2.tex|&include file for chapter 2\\
% |cdocspt3.tex|&include file for part 3\\
% |cdocspt4.tex|&include file for part 4\\
% |cdocsdrf.tex|&forwarding file for main file in draft mode\\
% |cdocsfi1.tex|&forwarding file for final version of chapter 1\\
% |cdocsfi2.tex|&forwarding file for final version of chapter 2\\
% \end{tabular}
% \end{center}
% Each of the eight files can be compiled directly by the \LaTeX{} compiler.
%
% %%%%%%%%%%%%%%%%%%%%%%%%%%%%%%%%%%%%%%
% \paragraph{Main File.}
%
% The main file is called |cdocsamp.tex|.
%
% Load the \textsf{childdoc} definitions and
% declare the filename for the main document:
%    \begin{macrocode}
\input{childdoc.def}
\childdocmain{}
%    \end{macrocode}

% Optional override for |\version| flag:
%    \begin{macrocode}
%%\ifchilddoc\else\providecommand{\version}{draft}\fi
%    \end{macrocode}

% Define the default values for the |\version| flag
% (|final| for the main file and |draft| for childs):
%    \begin{macrocode}
\ifchilddoc
\providecommand{\version}{draft}
\else
\providecommand{\version}{final}
\fi
%    \end{macrocode}

% Load the standard document class:
%    \begin{macrocode}
\documentclass[12pt]{article}
%    \end{macrocode}

% Start the document body:
%    \begin{macrocode}
\begin{document}
%    \end{macrocode}

% Declare a title page.
% Print title, part of document being processed and version flag:
%    \begin{macrocode}
\addtocounter{page}{-1}
\begin{center}
{\LARGE\bfseries{}childdoc example\par}
\vspace{1cm}
\ifchilddoc
\ifchilddocmanual part\else chapter\fi:
`\childdocname' of `\childdocjob'\par
\else
main document: `\childdocjob'\par
\fi
version: \version\par
\end{center}
\newpage
%    \end{macrocode}

% Manually include selected file,
% otherwise process as usual:
%    \begin{macrocode}
\ifchilddocmanual
\section*{part `\childdocname'}
\input{\childdocname}
\else
%    \end{macrocode}

% Include the two chapters:
%    \begin{macrocode}
\include{cdocsch1}
\include{cdocsch2}
%    \end{macrocode}

% Include the two parts unless only chapters should be displayed:
%    \begin{macrocode}
\ifchilddoc\else
\section{part three}
\input{cdocspt3}
\section{part four}
\input{cdocspt4}
\fi
%    \end{macrocode}

% Process as usual until here:
%    \begin{macrocode}
\fi
%    \end{macrocode}

% End of document body:
%    \begin{macrocode}
\end{document}
%    \end{macrocode}
%\iffalse
%</samplemain>
%\fi
%
% %%%%%%%%%%%%%%%%%%%%%%%%%%%%%%%%%%%%%%
% \paragraph{Chapter Include Files.}
%
% The include files are called |cdocsch1.tex| and |cdocsch2.tex|.
%
%\iffalse
%<*samplechap1|samplechap2>
%\fi

% Optional override for |\version| flag:
%    \begin{macrocode}
%%\providecommand{\version}{final}
%    \end{macrocode}

% Include the main document:
%    \begin{macrocode}
\input{childdoc.def}
\childdocof{cdocsamp}
%    \end{macrocode}

%\iffalse
%</samplechap1|samplechap2>
%\fi
%
%\iffalse
%<*samplechap1>
%\fi
% Some text for chapter 1:
%    \begin{macrocode}
\section{one}
some text in chapter one
%    \end{macrocode}

%\iffalse
%</samplechap1>
%\fi
% Some text for chapter 2:
%\iffalse
%<*samplechap2>
%\fi
%    \begin{macrocode}
\section{two}
more text in chapter two
%    \end{macrocode}

%\iffalse
%</samplechap2>
%\fi
%
% %%%%%%%%%%%%%%%%%%%%%%%%%%%%%%%%%%%%%%
% \paragraph{Part Include Files.}
%
% The include files are called |cdocspt3.tex| and |cdocspt4.tex|.
%
%\iffalse
%<*samplepart3|samplepart4>
%\fi

% Optional override for |\version| flag:
%    \begin{macrocode}
%%\providecommand{\version}{final}
%    \end{macrocode}

% Include the main document:
%    \begin{macrocode}
\input{childdoc.def}
\childdocby{cdocsamp}
%    \end{macrocode}

%\iffalse
%</samplepart3|samplepart4>
%\fi
%
%\iffalse
%<*samplepart3>
%\fi
% Some text for part 3:
%    \begin{macrocode}
some text in part three
%    \end{macrocode}

%\iffalse
%</samplepart3>
%\fi
% Some text for part 4:
%\iffalse
%<*samplepart4>
%\fi
%    \begin{macrocode}
more text in part four
%    \end{macrocode}

%\iffalse
%</samplepart4>
%\fi
%
% %%%%%%%%%%%%%%%%%%%%%%%%%%%%%%%%%%%%%%
% \paragraph{Forwarding for a Complete Draft.}
%
% The following forwarding file |cdocsdrf.tex|
% compiles the main document in draft mode:
%\iffalse
%<*sampledraft>
%\fi
%    \begin{macrocode}
\def\version{draft}
\input{childdoc.def}
\childdocforward{cdocsamp}
%    \end{macrocode}

%\iffalse
%</sampledraft>
%\fi
%
% %%%%%%%%%%%%%%%%%%%%%%%%%%%%%%%%%%%%%%
% \paragraph{Forwarding for Final Version of the Chapters.}
%
% The following forwarding files |cdocsfn1.tex| and |cdocsfn2.tex|
% (with identical content)
% compile the final versions of the child documents
% |cdocsch1.tex| and |cdocsch2.tex|, respectively:
%\iffalse
%<*samplefinal>
%\fi
%    \begin{macrocode}
\def\version{final}
\input{childdoc.def}
\childdocforwardprefix[cdocsamp]{cdocsfn}{cdocsch}
%    \end{macrocode}

%\iffalse
%</samplefinal>
%\fi
%
% %%%%%%%%%%%%%%%%%%%%%%%%%%%%%%%%%%%%%%
% \paragraph{Command Line Processing.}
%
% The following three command lines generate the output files
% |cdocscld|, |cdocscl1| and |cdocscl2|
% which should be identical to
% |cdocsdrf|, |cdocsch1| and |cdocsfn2|, respectively:
% \begin{center}
% \begin{tabular}{l}
% |latex -jobname cdocscld \|\\
% |  "\def\version{draft}\input{childdoc.def}\childdocforward{cdocsamp}"|\\
% |latex -jobname cdocscl1 \|\\
% |  "\input{childdoc.def}\childdocforward[cdocsamp]{cdocsch1}"|\\
% |latex -jobname cdocscl2 \|\\
% |  "\def\version{final}\input{childdoc.def}\childdocforward{cdocsch2}"|
% \end{tabular}
% \end{center}
% Note that the trailing backslash on each first line
% merely continues the input to the second line
% (for convenient cut ant paste).
% Furthermore, the command |latex| can be replaced by any
% of its alternative versions such as |pdflatex|.
%
% %%%%%%%%%%%%%%%%%%%%%%%%%%%%%%%%%%%%%%%%%%%%%%%%%%%%%%%%%%%%%%%%%%%%%%%%%%%%%%
% %%%%%%%%%%%%%%%%%%%%%%%%%%%%%%%%%%%%%%%%%%%%%%%%%%%%%%%%%%%%%%%%%%%%%%%%%%%%%%
% \section{Implementation}
%\iffalse
%<*package>
%\fi
%
% This section describes the definitions file |childdoc.def|.

% The definitions cannot be loaded using |\usepackage| or |\RequirePackage|
% which has a mechanism to prevent loading a style file more than once.
% When loading the definitions by means of |\input|
% multiple instances have to be prevented manually:
%\iffalse
%This code needs to be before the `\ProvidesFile' directive
%which is defined at the beginning of this file.
%Therefore it is also placed there and commented out here.
%</package>
%<*discard>
%\fi
%    \begin{macrocode}
\ifdefined\childdocmain\endinput\fi
%    \end{macrocode}
%\iffalse
%</discard>
%<*package>
%\fi
%
% \macro{\ifchilddoc}
% \macro{\ifchilddocmanual}
% The conditional |\ifchilddoc| tells whether a
% child (true) or main (false) document is being compiled.
% The conditional |\ifchilddocmanual| tells whether
% the |\includeonly| mechanism is used (false) or
% the selection of child files must be performed manually (true).
% The definitions initialise to false:
%    \begin{macrocode}
\newif\ifchilddoc
\newif\ifchilddocmanual
%    \end{macrocode}

% \macro{\childdocname}
% \macro{\childdocjob}
% The macro |\childdocname| stores the name of the main document
% to be compiled. The macro |\childdocjob| stores the name of
% the document on which the \LaTeX{} compiler was originally invoked.
% The content of |\jobname| cannot be compared
% to filenames specified in the source due to different catcodes.
% The following code rescans |\jobname|, stores the result
% in |\childdocname| and saves a copy in |\childdocjob|:
%    \begin{macrocode}
\edef\childdocname{\scantokens\expandafter{\jobname\noexpand}}
\let\childdocjob\childdocname
%    \end{macrocode}

% \macro{\childdocdisable}
% The macro |\childdocdisable| prevents the main file
% from being processed more than once.
% At this stage, the main document command |\childdocmain|
% is assumed to be called once again where it should do nothing.
% Any subsequent call to it should prevent
% a secondary processing of the main document
% It overwrites the forwarding commands
% |\childdocof| and |\childdocforward|
% with empty macros to prevent further inclusions of the main document:
%    \begin{macrocode}
\newcommand{\childdocdisable}
{
  \renewcommand{\childdocmain}[1]{\renewcommand{\childdocmain}[1]{\endinput}}
  \renewcommand{\childdocof}[1]{}
  \renewcommand{\childdocby}[2][]{}
  \renewcommand{\childdocforward}[2][]{}
  \renewcommand{\childdocdisable}{}
}
%    \end{macrocode}

% \macro{\childdocmain}
% The macro |\childdocmain| is to be called at the top of the main file
% with nothing or the main filename (without extension) as argument.
% First, it breaks loops.
% If the argument is not empty and does not match |\childdocname|
% (which is set by the first inclusion of |childdoc.def|),
% |\ifchilddoc| is set to true, |\includeonly| is applied to the child file
% and |\jobname| is set to the main file
% (for proper handling of |.aux| files):
%    \begin{macrocode}
\newcommand{\childdocmain}[1]
{
  \childdocdisable\childdocmain{}
  \if?#1?\else
    \begingroup
      \def\childdoctmp{#1}
      \ifx\childdoctmp\childdocname
        \def\childdoctmp{}
      \else
        \def\childdoctmp
        {
          \childdoctrue
          \includeonly{\childdocname}
          \def\childdocjob{#1}
          \def\jobname{#1}
        }
      \fi
      \expandafter
    \endgroup
    \childdoctmp
  \fi
}
%    \end{macrocode}

% \macro{\childdocof}
% The command |\childdocof| redirects
% compilation to the main file |#1|.
%    \begin{macrocode}
\newcommand{\childdocof}[1]
{
  \childdocdisable
  \childdoctrue
  \includeonly{\childdocname}
  \def\jobname{#1}
  \def\childdocjob{#1}
  \input{#1}
}
%    \end{macrocode}

% \macro{\childdocby}
% The command |\childdocby| ....
%    \begin{macrocode}
\newcommand{\childdocby}[2][]
{
  \childdocdisable
  \childdoctrue
  \childdocmanualtrue
  \if?#1?\else
    \def\jobname{#2}
  \fi
  \def\childdocjob{#2}
  \input{#2}
  \endinput
}
%    \end{macrocode}

% \macro{\childdocforward}
% The command |\childdocforward| redirects
% compilation to the main file or
% (if the optional argument is given) a child file.
% Parameters are set as if the main file
% or a child file starting with |\childdocof| was compiled.
% Then compilation is handed over to the main file:
%    \begin{macrocode}
\newcommand{\childdocforward}[2][]
{
  \begingroup
    \if?#1?
      \def\childdoctmp
      {
        \def\childdocname{#2}
        \def\childdocjob{#2}
        \def\jobname{#2}
        \input{#2}
        \endinput
      }
    \else
      \def\childdoctmp
      {
        \childdocdisable
        \def\childdocname{#2}
        \childdoctrue
        \includeonly{#2}
        \def\childdocjob{#1}
        \def\jobname{#1}
        \input{#1}
        \endinput
      }
    \fi
    \expandafter
  \endgroup
  \childdoctmp
}
%    \end{macrocode}

% \macro{\childdocforwardprefix}
% The command |\childdocforwardprefix| redirects
% compilation to the main or a child file by means of a pattern.
% The prefix |#1| in the current filename is replaced by |#2|
% and the suffix of the current filename is kept
% (it is assumed that the filename does not contain the substring `|~~~|'
% which is used as a delimiter).
% Compilation is handed over to the new file by |\childdocforward|:
%    \begin{macrocode}
\newcommand{\childdocforwardprefix}[3][]
{
  \begingroup
    \def\childdocextract #2##1~~~{\def\childdoctmp{\childdocforward[#1]{#3##1}}}
    \expandafter\childdocextract\childdocname~~~
    \expandafter
  \endgroup
  \childdoctmp
}
%    \end{macrocode}

% \macro{\childdoc}
% The deprecated macro |\childdoc| is a legacy version of |\childdocmain|:
%    \begin{macrocode}
\newcommand{\childdoc}{\childdocmain}
%    \end{macrocode}

% \macro{\childdocredirect}
% The deprecated macro |\childdocredirect| is a legacy version
% of |\childdocforward| and |\childdocforwardprefix|:
%    \begin{macrocode}
\newcommand{\childdocredirect}[2][]
{
  \begingroup
    \if?#1?
      \def\childdoctmp{\childdocforward{#2}}
    \else
      \def\childdoctmp{\childdocforwardprefix{#1}{#2}}
    \fi
    \expandafter
  \endgroup
  \childdoctmp
}
%    \end{macrocode}

%\iffalse
%</package>
%\fi
%
\endinput
|\\
|\childdocforward[|\textit{main}|]{|\textit{dest}|}|\\
\end{tabular}
\end{center}
%
The argument \textit{dest} is the destination file
(without extension).
It should be the main file or one of the child files.
Note that further \textsf{childdoc} directives
such as |\childdocof| and |\childdocforward|
in the indicated file will be processed in this form.
The optional argument \textit{main}
passes on directly to the main file \textit{main}
while pretending to compile the child \textit{dest}.
This form behaves as if \textit{dest}
issues |\childdocof{|\textit{main}|}| right away,
and no further \textsf{childdoc} directives will be processed.

%%%%%%%%%%%%%%%%%%%%%%%%%%%%%%%%%%%%%%%%
\DescribeMacro{\...prefix}
In the alternative form |\childdocforwardprefix|,
%
\begin{center}
\begin{tabular}{l}
|% \iffalse
%
% childdoc.dtx Copyright (C) 2017-2018 Niklas Beisert
%
% This work may be distributed and/or modified under the
% conditions of the LaTeX Project Public License, either version 1.3
% of this license or (at your option) any later version.
% The latest version of this license is in
%   http://www.latex-project.org/lppl.txt
% and version 1.3 or later is part of all distributions of LaTeX
% version 2005/12/01 or later.
%
% This work has the LPPL maintenance status `maintained'.
%
% The Current Maintainer of this work is Niklas Beisert.
%
% This work consists of the files childdoc.dtx and childdoc.ins
% and the derived files childdoc.def and cdocsamp.tex with
% cdocsch1.tex, cdocsch2.tex, cdocsdrf.tex, cdocsfn1.tex, cdocsfn2.tex.
%
%<package>\ifdefined\childdocmain\endinput\fi
%<package>\ProvidesFile{childdoc.def}[2018/12/30 v2.0 child document driver]
%<samplemain>\ProvidesFile{cdocsamp.tex}[2018/12/30 v2.0 sample for childdoc]
%<*driver>
%\ProvidesFile{childdoc.drv}[2018/12/30 v2.0 childdoc reference manual file]
\PassOptionsToClass{10pt,a4paper}{article}
\documentclass{ltxdoc}

\usepackage[margin=35mm]{geometry}
\usepackage{hyperref}
\usepackage{hyperxmp}
\usepackage[usenames]{color}

\hypersetup{colorlinks=true}
\hypersetup{pdfstartview=FitH}
\hypersetup{pdfpagemode=UseNone}
\hypersetup{pdfsource={}}
\hypersetup{pdflang={en-UK}}
\hypersetup{pdfcopyright={Copyright 2017-2018 Niklas Beisert.
  This work may be distributed and/or modified under the
  conditions of the LaTeX Project Public License, either version 1.3
  of this license or (at your option) any later version.}}
\hypersetup{pdflicenseurl={http://www.latex-project.org/lppl.txt}}
\hypersetup{pdfcontactaddress={ETH Zurich, ITP, HIT K,
  Wolfgang-Pauli-Strasse 27}}
\hypersetup{pdfcontactpostcode={8093}}
\hypersetup{pdfcontactcity={Zurich}}
\hypersetup{pdfcontactcountry={Switzerland}}
\hypersetup{pdfcontactemail={nbeisert@itp.phys.ethz.ch}}
\hypersetup{pdfcontacturl={http://people.phys.ethz.ch/\xmptilde nbeisert/}}

\newcommand{\secref}[1]{\hyperref[#1]{section \ref*{#1}}}

\parskip1ex
\parindent0pt
\let\olditemize\itemize
\def\itemize{\olditemize\parskip0pt}

\begin{document}

\title{The \textsf{childdoc} Package}
\hypersetup{pdftitle={The childdoc Package}}
\author{Niklas Beisert\\[2ex]
  Institut f\"ur Theoretische Physik\\
  Eidgen\"ossische Technische Hochschule Z\"urich\\
  Wolfgang-Pauli-Strasse 27, 8093 Z\"urich, Switzerland\\[1ex]
  \href{mailto:nbeisert@itp.phys.ethz.ch}
  {\texttt{nbeisert@itp.phys.ethz.ch}}}
\hypersetup{pdfauthor={Niklas Beisert}}
\hypersetup{pdfsubject={Manual for the LaTeX2e Package childdoc}}
\date{30 December 2018, \textsf{v2.0}}
\maketitle

\begin{abstract}\noindent
\textsf{childdoc} is a \LaTeXe{} package
that enables the direct compilation
of document sections included by |\include|
to individual files.
\end{abstract}

\begingroup
\parskip0ex
\tableofcontents
\endgroup

%%%%%%%%%%%%%%%%%%%%%%%%%%%%%%%%%%%%%%%%%%%%%%%%%%%%%%%%%%%%%%%%%%%%%%%%%%%%%%%%
%%%%%%%%%%%%%%%%%%%%%%%%%%%%%%%%%%%%%%%%%%%%%%%%%%%%%%%%%%%%%%%%%%%%%%%%%%%%%%%%
\section{Introduction}

\LaTeX{} provides a mechanism to structure a large document (such as a book)
into a main file and several child files (containing the chapters)
using the |\include| command.
This mechanism is beneficial for documents
which span hundreds of pages in order to
make the source file(s) more manageable.
Moreover, compilation can be restricted to
selected child files by means of the |\includeonly| command.
The latter feature can be used to reduce the compilation time while editing
(this was significantly more useful in the earlier days of \LaTeX{})
or to generate a smaller document which is easier to navigate.
Another application of |\includeonly| is to generate
documents consisting of selected parts of the complete document.

However, there are a few drawbacks of the plain |\include| mechanism:
\begin{itemize}
\item
The child files cannot be compiled on their own,
they can only be compiled via the main file.
A naive editing environment
(such as a text editor with an option
to have the current file processed by \LaTeX)
may require one to switch to the main file before compiling;
attempting to compile the child file produces errors.
\item
The main file must be modified (each time)
to adjust the |\includeonly| command
to the present needs. This easily leaves the main file in a messy state.
\item
The generated document will always carry the filename
of the main document. This is inconvenient if
several child files are to be compiled and
to be kept for distribution.
\end{itemize}

The present package provides a simple interface
to make child files individually compilable by \LaTeX{}.
Compiling a child file then has the same effect as compiling
the main file with an |\includeonly| command
to select the appropriate child.
Moreover the generated document will carry the name of the child
rather than the main file.
This resolves all three above issues.

This feature is meant to make the editing of books,
thesis documents and lecture notes somewhat more convenient.
However, the package can also be used efficiently for
composing a series of documents (such as exercise sheets)
which are typically distributed individually.
It then assists the author in generating the individual documents
(potentially in different versions)
as well as a document containing the collected series.
Another application is in developing style files
or other kinds of included material
where compilation of the style file could redirect
to a sample or test file.

%%%%%%%%%%%%%%%%%%%%%%%%%%%%%%%%%%%%%%%%%%%%%%%%%%%%%%%%%%%%%%%%%%%%%%%%%%%%%%%%
%%%%%%%%%%%%%%%%%%%%%%%%%%%%%%%%%%%%%%%%%%%%%%%%%%%%%%%%%%%%%%%%%%%%%%%%%%%%%%%%
\section{Usage}

First of all, the package \textsf{childdoc} is \emph{not} a standard
\LaTeXe{} |.sty| style file! Therefore it needs to be invoked in
a non-standard way.

%%%%%%%%%%%%%%%%%%%%%%%%%%%%%%%%%%%%%%%%%%%%%%%%%%%%%%%%%%%%%%%%%%%%%%%%%%%%%%%%
\subsection{Included Files}
\label{sec:include}

%%%%%%%%%%%%%%%%%%%%%%%%%%%%%%%%%%%%%%%%
\DescribeMacro{\childdocmain}
To use the package, add the commands
\begin{center}
\begin{tabular}{l}
|\input{childdoc.def}|\\
|\childdocmain{}|\\
\end{tabular}
\end{center}
at the very top of the main \LaTeX{} file,
in particular \emph{before} the |\documentclass| statement!
The argument of |\childdocmain| should be left empty
(but it must be present).

%%%%%%%%%%%%%%%%%%%%%%%%%%%%%%%%%%%%%%%%
\DescribeMacro{\childdocof}
Furthermore, add the commands
\begin{center}
\begin{tabular}{l}
|\input{childdoc.def}|\\
|\childdocof{|\textit{main}|}|\\
\end{tabular}
\end{center}
at the top of every child file \textit{child}
which is included by |\include{|\textit{child}|}|
from within the main file
(or at least for those files to be compiled individually).
The argument \textit{main} must be the filename of the main file.

There are a couple of
considerations in setting up the main and child documents:

%%%%%%%%%%%%%%%%%%%%%%%%%%%%%%%%%%%%%%%%
\paragraph{Restrictions.}

Please note the following restrictions:
\begin{itemize}
\item
|\childdocmain| must be called with one argument \textit{main}
to ensure compatibility with earlier version of the package.
It must either be empty (|\childdocmain{}|)
or precisely match the filename of the main file in which it is specified.
See \secref{sec:detection} for further information.
\item
The filename \textit{main} must be specified without the |.tex| extension.
\item
The filename \textit{main} is case sensitive
(even in case-insensitive file systems)
due to internal string comparison.
\item
The argument \textit{main} should be fully expanded, it cannot be a macro.
\item
Subdirectories and special characters should be avoided in filenames.
\item
The command |\childdocmain{|\textit{main}|}| must be followed by a whitespace.
It should not be followed immediately by another command
or by a comment mark `|%|'.
This is because the \TeX{} parser reads the token immediately following
the argument of |\childdocmain| and puts it
at the beginning of every child section;
however, a white\-space is ignored.
\end{itemize}

%%%%%%%%%%%%%%%%%%%%%%%%%%%%%%%%%%%%%%%%
\paragraph{Content of Main File.}

It is advisable to place all content in the child files included by |\include|.
Any output contained in the main file will appear in all child documents
unless suppressed manually;
it cannot be suppressed automatically by the |\includeonly| directive
and thus should normally be avoided.
A method to include some content in the main file
by means of conditional processing is described in \secref{sec:conditional}.

%%%%%%%%%%%%%%%%%%%%%%%%%%%%%%%%%%%%%%%%
\paragraph{Page Numbering.}

When only a part of the document is compiled,
the appropriate numbering of pages
(as well as other status parameters)
is determined from the |.aux| files.
The latter contain information from previous passes.
However this information needs to propagate through
all intermediate child documents.
Therefore the page numbering in child documents may well
be inconsistent until the complete document is compiled at least once.

A useful (if unconventional) way to always ensure a consistent
page numbering is to restart the numbering in each child document
and denote the pages by `\textit{child}|.|\textit{page}'
where \textit{child} represents the chapter/section number of the child file.
This can be achieved by the command
|\numberwithin{page}{|\textit{child}|}|
of the \textsf{amsmath} package
where \textit{child} can be |chapter| or |section|
depending on the chosen structuring.
Alternatively, one can modify the macro |\thepage| appropriately
and reset the counter |page| at the start of each child file.

%%%%%%%%%%%%%%%%%%%%%%%%%%%%%%%%%%%%%%%%%%%%%%%%%%%%%%%%%%%%%%%%%%%%%%%%%%%%%%%%
\subsection{Conditional Processing}
\label{sec:conditional}

The package provides a mechanism to compile different versions
of a document. To customise the versions further some conditional processing
can come in handy to distinguish which version is being compiled.
The package provides two macros to describe the compilation context:

%%%%%%%%%%%%%%%%%%%%%%%%%%%%%%%%%%%%%%%%
\DescribeMacro{\ifchilddoc}
The conditional |\ifchilddoc| distinguishes between the compilation of
child documents and the main document:
%
\begin{center}
|\ifchilddoc |\textit{child-code}| |[|\||else |\textit{main-code}]| \||fi|
\end{center}

%%%%%%%%%%%%%%%%%%%%%%%%%%%%%%%%%%%%%%%%
\DescribeMacro{\childdocname}
\DescribeMacro{\childdocjob}
The macro |\childdocname| contains the filename (without extension)
of the main or child file being processed.
Note that |\childdocjob| will always contain the name of the main file.

%%%%%%%%%%%%%%%%%%%%%%%%%%%%%%%%%%%%%%%%
\paragraph{Title Page.}

Conditional processing can be used to include a title or banner page
in the main document when proper precautions are taken.
Importantly, the code in the main file should ensure that the page counter
(as well as other status parameters which are stored in the |.aux| files)
takes the same value after the conditional processing.
Otherwise the page numbers may take divergent values
depending on which part is compiled.

For example, a title page could be declared by:
%
\begin{center}
\begin{tabular}{l}
|\ifchilddoc\||else|\\
|\addtocounter{page}{-1}|\\
\textit{code for title page}\\
|\newpage|\\
|\||fi|
\end{tabular}
\end{center}
%
A banner page for the child documents can be generated by:
%
\begin{center}
\begin{tabular}{l}
|\ifchilddoc|\\
|\addtocounter{page}{-1}|\\
\textit{code for banner page}\\
|\newpage|\\
|\||fi|
\end{tabular}
\end{center}
%
Here one could write a message such as:
\begin{center}
|This is the part \childdocname{} of \childdocjob{}.|
\end{center}

%%%%%%%%%%%%%%%%%%%%%%%%%%%%%%%%%%%%%%%%%%%%%%%%%%%%%%%%%%%%%%%%%%%%%%%%%%%%%%%%
\subsection{Flags}
\label{sec:flags}

The package makes it easy to generate different versions
of the main or child documents.
To this end compilation flags can be defined
and assigned different default values.
They will be particularly useful in conjunction
with the forwarding mechanism described in \secref{sec:forward}.

For example, it may be useful to have a flag |\version|
which can be set to |draft| or |final|.
The document source will contain some conditional code
depending on the value of |\version|.
Suppose further, the flag should default to |final| for the main file
and to |draft| for child files
which is a natural assignment for editing the document.
This is achieved by placing the following code
in the preamble of the main document
(below the |\childdocmain| directive):
%
\begin{center}
\begin{tabular}{l}
|\ifchilddoc|\\
|\providecommand{\version}{draft}|\\
|\||else|\\
|\providecommand{\version}{final}|\\
|\||fi|
\end{tabular}
\end{center}
%
The definition by |\providecommand| makes sure
that previous definitions are not overwritten.
Further statements |\providecommand{\version}{...}|
can thus be added before the above code to override it.

For the main file, one might add a line
(between |\childdocmain| and the above block)
%
\begin{center}
|%\ifchilddoc\||else\providecommand{\version}{draft}\||fi|
\end{center}
%
which can be uncommented to produce a draft version.
Likewise one can add a line to the very top of a child file
(above the |\childdocof{|\textit{main}|}| directive)
%
\begin{center}
|%\providecommand{\version}{final}|
\end{center}
%
which can be uncommented to produce the final version of this child document.

%%%%%%%%%%%%%%%%%%%%%%%%%%%%%%%%%%%%%%%%%%%%%%%%%%%%%%%%%%%%%%%%%%%%%%%%%%%%%%%%
\subsection{Forwarding}
\label{sec:forward}

Different versions of the main or child documents
using compilation flags as described in \secref{sec:flags}
can be (permanently) stored in different files
for convenient compilation, viewing and distribution.
To this end, the package defines a command
to pass on compilation to a different file:

%%%%%%%%%%%%%%%%%%%%%%%%%%%%%%%%%%%%%%%%
\DescribeMacro{\childdocforward}
The command |\childdocforward| redirects processing to
another source file:
%
\begin{center}
\begin{tabular}{l}
|\input{childdoc.def}|\\
|\childdocforward[|\textit{main}|]{|\textit{dest}|}|\\
\end{tabular}
\end{center}
%
The argument \textit{dest} is the destination file
(without extension).
It should be the main file or one of the child files.
Note that further \textsf{childdoc} directives
such as |\childdocof| and |\childdocforward|
in the indicated file will be processed in this form.
The optional argument \textit{main}
passes on directly to the main file \textit{main}
while pretending to compile the child \textit{dest}.
This form behaves as if \textit{dest}
issues |\childdocof{|\textit{main}|}| right away,
and no further \textsf{childdoc} directives will be processed.

%%%%%%%%%%%%%%%%%%%%%%%%%%%%%%%%%%%%%%%%
\DescribeMacro{\...prefix}
In the alternative form |\childdocforwardprefix|,
%
\begin{center}
\begin{tabular}{l}
|\input{childdoc.def}|\\
|\childdocforwardprefix[|\textit{main}|]{|\textit{prefix}|}{|\textit{dest}|}|
\end{tabular}
\end{center}
%
the destination file is determined by a pattern
depending on the current file:
To make this work, the current file must be called
`{\textit{prefix}\hspace{0.2em}\textit{suffix}}'
with \textit{prefix} matching precisely the argument.
Processing is then passed on to the file
`{\textit{dest}\hspace{0.2em}\textit{suffix}}'.
Surely, the same effect is achieved by
directly specifying the
argument `{\textit{dest}\hspace{0.2em}\textit{suffix}}'
in the first form.
However, that requires to set up a different file
for each child. With the alternative form of the command
all these files can have exactly the same content
which simplifies setting them up and maintaining them.

For example, the following file |draft.tex|
with a compilation flag |\version| as described in \secref{sec:flags}
compiles the main document as a draft:
%
\begin{center}
\begin{tabular}{l}
|\def\version{draft}|\\
|\input{childdoc.def}|\\
|\childdocforward{|\textit{main}|}|
\end{tabular}
\end{center}
%
Likewise, the following files |final|\textit{nn}|.tex|
compile the final version of the child document
|child|\textit{nn}|.tex|:
%
\begin{center}
\begin{tabular}{l}
|\def\version{final}|\\
|\input{childdoc.def}|\\
|\childdocforwardprefix{final}{child}|
\end{tabular}
\end{center}
%

Note that when several versions of a main file and/or of each child file
are to be generated, it may be convenient to set up a |Makefile| or
shell script to automatise the process.

%%%%%%%%%%%%%%%%%%%%%%%%%%%%%%%%%%%%%%%%%%%%%%%%%%%%%%%%%%%%%%%%%%%%%%%%%%%%%%%%
\subsection{Command Line Processing}
\label{sec:commandline}

The effect of redirection files can also be achieved by invoking
the \LaTeX{} compiler with a more elaborate command line.
Most conveniently this should be done as part
of a shell script or a |Makefile|.

When using \textsf{childdoc} in the main file, the following
command lines effectively perform a redirection
(note that depending on the shell being used,
backslashes may have to be doubled: `|\|' $\to$ `|\\|'):
%
\begin{center}
|... -jobname "|\textit{target}|" |\\|"|[\textit{flags}]%
|\input{childdoc.def}\childdocforward[|\textit{main}|]{|\textit{dest}|}"|
\end{center}
%
Here \textit{target} is the name of the output file,
\textit{main} is the name of the main file
and \textit{dest} is the name of the main or child file to be processed
(all filenames without extensions).
The optional argument \textit{main} can be omitted
if \textit{main} matches \textit{dest}.
Optionally, compilation \textit{flags} can be defined via |\def| commands.
This command line makes the \TeX{} engine believe
it is compiling the file \textit{target}
whose content is specified as the latter parameter.
The provided code then forwards the processing to
\textit{main} or \textit{dest} as described in \secref{sec:forward}.

%%%%%%%%%%%%%%%%%%%%%%%%%%%%%%%%%%%%%%%%%%%%%%%%%%%%%%%%%%%%%%%%%%%%%%%%%%%%%%%%
\subsection{Include by Input}
\label{sec:input}

Including child documents by |\include| has some restrictions by design.
Most notably, the content of a child document always occupies
its own set of pages; pages cannot be shared between child documents.
Usually, this behaviour makes perfect sense
because each child document contain an essential part of the document.
However, in some situations it may be desirable to compose
a document from a collection of parts
without having mandatory page breaks between then.
For this case, the package
provides a mechanism to include parts
by |\input| which can also be processed individually.
However, by construction this mechanism
requires manual handling of the content to be output.

%%%%%%%%%%%%%%%%%%%%%%%%%%%%%%%%%%%%%%%%
\DescribeMacro{\ifchilddocmanual}
The main file should be prepared as usual, see \secref{sec:include}.
However, the document body must make a distinction
between processing of an individual part and of the main document, e.g.:
%
\begin{center}
\begin{tabular}{l}
|\ifchilddocmanual|\\
|\input{\childdocname}|\\
|\||else|\\
\textit{document body with }|\input{|\textit{part}|}|\\
|\||fi|
\end{tabular}
\end{center}
%
The conditional |\ifchilddocmanual| is true whenever
a part to be included by |\input| is being compiled,
and the name of the part is stored in |\childdocname|.

%%%%%%%%%%%%%%%%%%%%%%%%%%%%%%%%%%%%%%%%
\DescribeMacro{\childdocby}
Each part to be included by |\input| should start with:
%
\begin{center}
\begin{tabular}{l}
|\input{childdoc.def}|\\
|\childdocby{|\textit{main}|}|\\
\end{tabular}
\end{center}
%
The directive |\childdocby| is similar to |\childdocof|
described in \secref{sec:include},
but the subsequent selection of content must be done manually.
To that end, both |\ifchilddoc| and |\ifchilddocmanual|
will be true upon processing of a part,
and the name of the part is stored in |\childdocname|.
Note that |\jobname| will be set to the filename of the current part
so that each part receives an individual |.aux| file
that does not interfere with the |.aux| file(s) of the main document.
This behaviour can be altered by the alternative form
|\childdocby[*]{|\textit{main}|}| (with a non-empty optional argument)
which uses the |.aux| file of the main document
by setting |\jobname| to \textit{main}.

%%%%%%%%%%%%%%%%%%%%%%%%%%%%%%%%%%%%%%%%%%%%%%%%%%%%%%%%%%%%%%%%%%%%%%%%%%%%%%%%
\subsection{Driver Development}
\label{sec:driver}

The \textsf{childdoc} mechanism can also be use for the development
of definition files such as \LaTeX{} styles or classes.
This case differs from the above setup with multiple parts
included by |\include| in that no |\includeonly| should be invoked.
This can be achieved by starting the include file
(before |\ProvidesPackage|) with:
%
\begin{center}
\begin{tabular}{l}
|\input{childdoc.def}|\\
|\childdocforward{|\textit{main}|}|\\
\end{tabular}
\end{center}
%
or alternatively with:
%
\begin{center}
\begin{tabular}{l}
|\input{childdoc.def}|\\
|\childdocby{|\textit{main}|}|\\
\end{tabular}
\end{center}
%
Both forms have slightly different effects as described above.
The main file is prepared as usual, see \secref{sec:include}.

%%%%%%%%%%%%%%%%%%%%%%%%%%%%%%%%%%%%%%%%%%%%%%%%%%%%%%%%%%%%%%%%%%%%%%%%%%%%%%%%
\subsection{Legacy Detection}
\label{sec:detection}

The directive |\childdocmain| in the main file can detect
whether the complete document or merely a child is to be compiled
even without using the directive |\childdocof|.
This method is deprecated because it is less robust
and there is no compelling reason to use it;
it is merely provided for backward compatibility
and it may be removed in future versions.

If the detection mechanism is to be used,
it is mandatory to correctly specify
the filename of the main file as the argument of |\childdocmain|:
%
\begin{center}
\begin{tabular}{l}
|\input{childdoc.def}|\\
|\childdocmain{|\textit{main}|}|\\
\end{tabular}
\end{center}
%
If |\jobname| does not match the argument \textit{main} of |\childdocmain|,
it is assumed that |\jobname| points to the child file to be compiled.
When using |\childdocmain| with the main file specified as argument,
it suffices to start a child file
with just |\input{|\textit{main}|}|
without loading of the package and using |\childdocof|.
If instead all processing is done
with the appropriate \textsf{childdoc} directives,
the argument of \textit{main} of |\childdocmain| can be empty.

An alternative version of the command line processing described
in \secref{sec:commandline} using the detection mechanism reads:
%
\begin{center}
|... -jobname "|\textit{target}|" "|[\textit{flags}]%
[|\def\jobname{|\textit{dest}|}|]|\input{|\textit{main}|}"|
\end{center}

%%%%%%%%%%%%%%%%%%%%%%%%%%%%%%%%%%%%%%%%%%%%%%%%%%%%%%%%%%%%%%%%%%%%%%%%%%%%%%%%
\subsection{Manual Code}
\label{sec:manual}

In case one cannot be certain whether the definitions file |childdoc.def|
is installed on the target \TeX{} distribution
and one prefers not to ship it,
it is conceivable to paste a few relevant commands into the sources.

To that end, drop all statements |\input{childdoc.def}|
and perform the replacements as outlined below.
Instead of |\childdocmain{|\textit{main}|}| add the following code
to the top of the main file:
%
\begin{center}
\begin{tabular}{l}
|\||ifdefined\childdocname\endinput\||fi\newif\ifchilddoc|\\
|\edef\childdocname{\scantokens\expandafter{\jobname\noexpand}}|\\
|\def\childdocmain{|\textit{main}|}\||ifx\childdocmain\childdocname\||else|\\
|\childdoctrue\includeonly{\childdocname}\let\jobname\childdocmain\||fi|\\
\end{tabular}
\end{center}
%
Instead of |\childdocof{|\textit{main}|}| just include the main file
at the top of each child file:
%
\begin{center}
|\input{|\textit{main}|}|
\end{center}
%
A simple redirection |\childdocforward{|\textit{dest}|}| is achieved by:
%
\begin{center}
|\def\jobname{|\textit{dest}|}\input{\jobname}|
\end{center}
%
The redirection with prefix
|\childdocforwardprefix[|\textit{prefix}|]{|\textit{dest}|}|
is accomplished by:
%
\begin{center}
\begin{tabular}{l}
|{\edef\jobname{\scantokens\expandafter{\jobname\noexpand}}|\\
|\def\redirectjob |\textit{prefix}|#1~~~{\gdef\jobname{|\textit{dest}|#1}}|\\
|\expandafter\redirectjob\jobname~~~}\input{\jobname}|
\end{tabular}
\end{center}

In an alternative approach,
child documents can be compiled by a specific command line
without additional code or specific definitions:
%
\begin{center}
|... -jobname "|\textit{target}|" "|[\textit{flags}]%
|\includeonly{|\textit{dest}|}\input{|\textit{main}|}"|
\end{center}
%

%%%%%%%%%%%%%%%%%%%%%%%%%%%%%%%%%%%%%%%%%%%%%%%%%%%%%%%%%%%%%%%%%%%%%%%%%%%%%%%%
%%%%%%%%%%%%%%%%%%%%%%%%%%%%%%%%%%%%%%%%%%%%%%%%%%%%%%%%%%%%%%%%%%%%%%%%%%%%%%%%
\section{Information}

%%%%%%%%%%%%%%%%%%%%%%%%%%%%%%%%%%%%%%%%%%%%%%%%%%%%%%%%%%%%%%%%%%%%%%%%%%%%%%%%
\subsection{Copyright}

Copyright \copyright{} 2017--2018 Niklas Beisert

This work may be distributed and/or modified under the
conditions of the \LaTeX{} Project Public License, either version 1.3
of this license or (at your option) any later version.
The latest version of this license is in
  \url{http://www.latex-project.org/lppl.txt}
and version 1.3 or later is part of all distributions of \LaTeX{}
version 2005/12/01 or later.

This work has the LPPL maintenance status `maintained'.

The Current Maintainer of this work is Niklas Beisert.

This work consists of the files |README.txt|, |childdoc.ins| and |childdoc.dtx|
as well as the derived files |childdoc.def|, |cdocsamp.tex|
with |cdocsch1.tex|, |cdocsch2.tex|, |cdocspt3.tex|, |cdocspt4.tex|,
|cdocsdrf.tex|, |cdocsfn1.tex|, |cdocsfn2.tex|
as well as |childdoc.pdf|.

%%%%%%%%%%%%%%%%%%%%%%%%%%%%%%%%%%%%%%%%%%%%%%%%%%%%%%%%%%%%%%%%%%%%%%%%%%%%%%%%
\subsection{Files and Installation}

The package consists of the files:
%
\begin{center}
\begin{tabular}{ll}
    |README.txt|   & readme file \\
    |childdoc.ins| & installation file \\
    |childdoc.dtx| & source file \\
    |childdoc.def| & definition file \\
    |cdocsamp.tex| & sample main file \\
    |cdocsch1.tex| & sample include file \\
    |cdocsch2.tex| & sample include file \\
    |cdocspt3.tex| & sample part file \\
    |cdocspt4.tex| & sample part file \\
    |cdocsdrf.tex| & sample redirection file \\
    |cdocsfn1.tex| & sample redirection file \\
    |cdocsfn2.tex| & sample redirection file \\
    |childdoc.pdf| & manual
\end{tabular}
\end{center}
%
The distribution consists of the files
|README.txt|, |childdoc.ins| and |childdoc.dtx|.
%
\begin{itemize}
\item
Run (pdf)\LaTeX{} on |childdoc.dtx|
to compile the manual |childdoc.pdf| (this file).
\item
Run \LaTeX{} on |childdoc.ins| to create the definitions file |childdoc.def|
and the sample |cdocsamp.tex| with include files
|cdocsch1.tex|, |cdocsch2.tex|, |cdocspt3.tex|, |cdocspt4.tex|,
|cdocsdrf.tex|, |cdocsfn1.tex|, |cdocsfn2.tex|.
Then copy the file |childdoc.def| to an appropriate directory of your \LaTeX{}
distribution, e.g.\ \textit{texmf-root}|/tex/latex/childdoc|.
\end{itemize}

%%%%%%%%%%%%%%%%%%%%%%%%%%%%%%%%%%%%%%%%%%%%%%%%%%%%%%%%%%%%%%%%%%%%%%%%%%%%%%%%
\subsection{Related CTAN Packages}

There are several other packages which offer a similar functionality:
%
\begin{itemize}
\item
The packages
\href{http://ctan.org/pkg/docmute}{\textsf{docmute}},
\href{http://ctan.org/pkg/includex}{\textsf{includex}} and
\href{http://ctan.org/pkg/standalone}{\textsf{standalone}}
provide commands to include only the document body of
a child file thus allowing both files to be compiled individually.
\item
The packages \href{http://ctan.org/pkg/subdocs}{\textsf{subdocs}}
and \href{http://ctan.org/pkg/subfiles}{\textsf{subfiles}}
provide structures in which the main and child documents can be
encapsulated and allowing them to be compiled individually.
The inclusion mechanism is different from the conventional |\include|.
\item
The package \href{http://ctan.org/pkg/combine}{\textsf{combine}}
is an elaborate solution to combine several documents into one.
\end{itemize}
%
See also the CTAN topic \href{http://ctan.org/topic/subdocs}{\textsf{subdocs}}
for further related packages.
The present package differs from the above solutions in that
a document structure constructed with the conventional |\include| mechanism
just needs two extra commands at the top of every file
such that all constituent files can be compiled individually.

%%%%%%%%%%%%%%%%%%%%%%%%%%%%%%%%%%%%%%%%%%%%%%%%%%%%%%%%%%%%%%%%%%%%%%%%%%%%%%%%
%\subsection{Feature Suggestions}
%
%The following is a list of features which may be useful for future
%versions of this package:
%%
%\begin{itemize}
%\item
%\ldots
%\end{itemize}

%%%%%%%%%%%%%%%%%%%%%%%%%%%%%%%%%%%%%%%%%%%%%%%%%%%%%%%%%%%%%%%%%%%%%%%%%%%%%%%%
\subsection{Revision History}

%%%%%%%%%%%%%%%%%%%%%%%%%%%%%%%%%%%%%%%%
\paragraph{v2.0:} 2018/12/30

\begin{itemize}
\item
immediate forward processing
\item
added |\childdocby| mechanism
\item
manual restructured
\end{itemize}

%%%%%%%%%%%%%%%%%%%%%%%%%%%%%%%%%%%%%%%%
\paragraph{v1.6:} 2018/01/17

\begin{itemize}
\item
application for development of include files
\item
corrections to manual
\end{itemize}

%%%%%%%%%%%%%%%%%%%%%%%%%%%%%%%%%%%%%%%%
\paragraph{v1.5:} 2017/05/21

\begin{itemize}
\item
more complete structuring introduced
\item
|\childdocof| introduced
\item
|\childdoc| renamed to |\childdocmain|
\item
|\childredirect| renamed to |\childdocforward| and |\childdocforwardprefix|
and functionality expanded
\end{itemize}

%%%%%%%%%%%%%%%%%%%%%%%%%%%%%%%%%%%%%%%%
\paragraph{v1.0:} 2017/04/27

\begin{itemize}
\item
manual and install package
\item
first version published on CTAN
\end{itemize}

%%%%%%%%%%%%%%%%%%%%%%%%%%%%%%%%%%%%%%%%
\paragraph{v0.6:} 2017/04/26

\begin{itemize}
\item
redirection mechanism added
\end{itemize}

%%%%%%%%%%%%%%%%%%%%%%%%%%%%%%%%%%%%%%%%
\paragraph{v0.5:} 2017/04/26

\begin{itemize}
\item
functionality in definition file
\end{itemize}


%%%%%%%%%%%%%%%%%%%%%%%%%%%%%%%%%%%%%%%%%%%%%%%%%%%%%%%%%%%%%%%%%%%%%%%%%%%%%%%%
%%%%%%%%%%%%%%%%%%%%%%%%%%%%%%%%%%%%%%%%%%%%%%%%%%%%%%%%%%%%%%%%%%%%%%%%%%%%%%%%
%%%%%%%%%%%%%%%%%%%%%%%%%%%%%%%%%%%%%%%%%%%%%%%%%%%%%%%%%%%%%%%%%%%%%%%%%%%%%%%%
\appendix

\settowidth\MacroIndent{\rmfamily\scriptsize 000\ }

 \DocInput{childdoc.dtx}

\end{document}
%</driver>
% \fi
%
% %%%%%%%%%%%%%%%%%%%%%%%%%%%%%%%%%%%%%%%%%%%%%%%%%%%%%%%%%%%%%%%%%%%%%%%%%%%%%%
% %%%%%%%%%%%%%%%%%%%%%%%%%%%%%%%%%%%%%%%%%%%%%%%%%%%%%%%%%%%%%%%%%%%%%%%%%%%%%%
% \section{Sample}
%\iffalse
%<*samplemain>
%\fi
%
% The following presents a sample document
% with two chapters, two parts, a title page,
% a compile flag as well as three forwarding files to set the flag.
% It consists of eight |.tex| files:
% \begin{center}
% \begin{tabular}{ll}
% |cdocsamp.tex|&main file\\
% |cdocsch1.tex|&include file for chapter 1\\
% |cdocsch2.tex|&include file for chapter 2\\
% |cdocspt3.tex|&include file for part 3\\
% |cdocspt4.tex|&include file for part 4\\
% |cdocsdrf.tex|&forwarding file for main file in draft mode\\
% |cdocsfi1.tex|&forwarding file for final version of chapter 1\\
% |cdocsfi2.tex|&forwarding file for final version of chapter 2\\
% \end{tabular}
% \end{center}
% Each of the eight files can be compiled directly by the \LaTeX{} compiler.
%
% %%%%%%%%%%%%%%%%%%%%%%%%%%%%%%%%%%%%%%
% \paragraph{Main File.}
%
% The main file is called |cdocsamp.tex|.
%
% Load the \textsf{childdoc} definitions and
% declare the filename for the main document:
%    \begin{macrocode}
\input{childdoc.def}
\childdocmain{}
%    \end{macrocode}

% Optional override for |\version| flag:
%    \begin{macrocode}
%%\ifchilddoc\else\providecommand{\version}{draft}\fi
%    \end{macrocode}

% Define the default values for the |\version| flag
% (|final| for the main file and |draft| for childs):
%    \begin{macrocode}
\ifchilddoc
\providecommand{\version}{draft}
\else
\providecommand{\version}{final}
\fi
%    \end{macrocode}

% Load the standard document class:
%    \begin{macrocode}
\documentclass[12pt]{article}
%    \end{macrocode}

% Start the document body:
%    \begin{macrocode}
\begin{document}
%    \end{macrocode}

% Declare a title page.
% Print title, part of document being processed and version flag:
%    \begin{macrocode}
\addtocounter{page}{-1}
\begin{center}
{\LARGE\bfseries{}childdoc example\par}
\vspace{1cm}
\ifchilddoc
\ifchilddocmanual part\else chapter\fi:
`\childdocname' of `\childdocjob'\par
\else
main document: `\childdocjob'\par
\fi
version: \version\par
\end{center}
\newpage
%    \end{macrocode}

% Manually include selected file,
% otherwise process as usual:
%    \begin{macrocode}
\ifchilddocmanual
\section*{part `\childdocname'}
\input{\childdocname}
\else
%    \end{macrocode}

% Include the two chapters:
%    \begin{macrocode}
\include{cdocsch1}
\include{cdocsch2}
%    \end{macrocode}

% Include the two parts unless only chapters should be displayed:
%    \begin{macrocode}
\ifchilddoc\else
\section{part three}
\input{cdocspt3}
\section{part four}
\input{cdocspt4}
\fi
%    \end{macrocode}

% Process as usual until here:
%    \begin{macrocode}
\fi
%    \end{macrocode}

% End of document body:
%    \begin{macrocode}
\end{document}
%    \end{macrocode}
%\iffalse
%</samplemain>
%\fi
%
% %%%%%%%%%%%%%%%%%%%%%%%%%%%%%%%%%%%%%%
% \paragraph{Chapter Include Files.}
%
% The include files are called |cdocsch1.tex| and |cdocsch2.tex|.
%
%\iffalse
%<*samplechap1|samplechap2>
%\fi

% Optional override for |\version| flag:
%    \begin{macrocode}
%%\providecommand{\version}{final}
%    \end{macrocode}

% Include the main document:
%    \begin{macrocode}
\input{childdoc.def}
\childdocof{cdocsamp}
%    \end{macrocode}

%\iffalse
%</samplechap1|samplechap2>
%\fi
%
%\iffalse
%<*samplechap1>
%\fi
% Some text for chapter 1:
%    \begin{macrocode}
\section{one}
some text in chapter one
%    \end{macrocode}

%\iffalse
%</samplechap1>
%\fi
% Some text for chapter 2:
%\iffalse
%<*samplechap2>
%\fi
%    \begin{macrocode}
\section{two}
more text in chapter two
%    \end{macrocode}

%\iffalse
%</samplechap2>
%\fi
%
% %%%%%%%%%%%%%%%%%%%%%%%%%%%%%%%%%%%%%%
% \paragraph{Part Include Files.}
%
% The include files are called |cdocspt3.tex| and |cdocspt4.tex|.
%
%\iffalse
%<*samplepart3|samplepart4>
%\fi

% Optional override for |\version| flag:
%    \begin{macrocode}
%%\providecommand{\version}{final}
%    \end{macrocode}

% Include the main document:
%    \begin{macrocode}
\input{childdoc.def}
\childdocby{cdocsamp}
%    \end{macrocode}

%\iffalse
%</samplepart3|samplepart4>
%\fi
%
%\iffalse
%<*samplepart3>
%\fi
% Some text for part 3:
%    \begin{macrocode}
some text in part three
%    \end{macrocode}

%\iffalse
%</samplepart3>
%\fi
% Some text for part 4:
%\iffalse
%<*samplepart4>
%\fi
%    \begin{macrocode}
more text in part four
%    \end{macrocode}

%\iffalse
%</samplepart4>
%\fi
%
% %%%%%%%%%%%%%%%%%%%%%%%%%%%%%%%%%%%%%%
% \paragraph{Forwarding for a Complete Draft.}
%
% The following forwarding file |cdocsdrf.tex|
% compiles the main document in draft mode:
%\iffalse
%<*sampledraft>
%\fi
%    \begin{macrocode}
\def\version{draft}
\input{childdoc.def}
\childdocforward{cdocsamp}
%    \end{macrocode}

%\iffalse
%</sampledraft>
%\fi
%
% %%%%%%%%%%%%%%%%%%%%%%%%%%%%%%%%%%%%%%
% \paragraph{Forwarding for Final Version of the Chapters.}
%
% The following forwarding files |cdocsfn1.tex| and |cdocsfn2.tex|
% (with identical content)
% compile the final versions of the child documents
% |cdocsch1.tex| and |cdocsch2.tex|, respectively:
%\iffalse
%<*samplefinal>
%\fi
%    \begin{macrocode}
\def\version{final}
\input{childdoc.def}
\childdocforwardprefix[cdocsamp]{cdocsfn}{cdocsch}
%    \end{macrocode}

%\iffalse
%</samplefinal>
%\fi
%
% %%%%%%%%%%%%%%%%%%%%%%%%%%%%%%%%%%%%%%
% \paragraph{Command Line Processing.}
%
% The following three command lines generate the output files
% |cdocscld|, |cdocscl1| and |cdocscl2|
% which should be identical to
% |cdocsdrf|, |cdocsch1| and |cdocsfn2|, respectively:
% \begin{center}
% \begin{tabular}{l}
% |latex -jobname cdocscld \|\\
% |  "\def\version{draft}\input{childdoc.def}\childdocforward{cdocsamp}"|\\
% |latex -jobname cdocscl1 \|\\
% |  "\input{childdoc.def}\childdocforward[cdocsamp]{cdocsch1}"|\\
% |latex -jobname cdocscl2 \|\\
% |  "\def\version{final}\input{childdoc.def}\childdocforward{cdocsch2}"|
% \end{tabular}
% \end{center}
% Note that the trailing backslash on each first line
% merely continues the input to the second line
% (for convenient cut ant paste).
% Furthermore, the command |latex| can be replaced by any
% of its alternative versions such as |pdflatex|.
%
% %%%%%%%%%%%%%%%%%%%%%%%%%%%%%%%%%%%%%%%%%%%%%%%%%%%%%%%%%%%%%%%%%%%%%%%%%%%%%%
% %%%%%%%%%%%%%%%%%%%%%%%%%%%%%%%%%%%%%%%%%%%%%%%%%%%%%%%%%%%%%%%%%%%%%%%%%%%%%%
% \section{Implementation}
%\iffalse
%<*package>
%\fi
%
% This section describes the definitions file |childdoc.def|.

% The definitions cannot be loaded using |\usepackage| or |\RequirePackage|
% which has a mechanism to prevent loading a style file more than once.
% When loading the definitions by means of |\input|
% multiple instances have to be prevented manually:
%\iffalse
%This code needs to be before the `\ProvidesFile' directive
%which is defined at the beginning of this file.
%Therefore it is also placed there and commented out here.
%</package>
%<*discard>
%\fi
%    \begin{macrocode}
\ifdefined\childdocmain\endinput\fi
%    \end{macrocode}
%\iffalse
%</discard>
%<*package>
%\fi
%
% \macro{\ifchilddoc}
% \macro{\ifchilddocmanual}
% The conditional |\ifchilddoc| tells whether a
% child (true) or main (false) document is being compiled.
% The conditional |\ifchilddocmanual| tells whether
% the |\includeonly| mechanism is used (false) or
% the selection of child files must be performed manually (true).
% The definitions initialise to false:
%    \begin{macrocode}
\newif\ifchilddoc
\newif\ifchilddocmanual
%    \end{macrocode}

% \macro{\childdocname}
% \macro{\childdocjob}
% The macro |\childdocname| stores the name of the main document
% to be compiled. The macro |\childdocjob| stores the name of
% the document on which the \LaTeX{} compiler was originally invoked.
% The content of |\jobname| cannot be compared
% to filenames specified in the source due to different catcodes.
% The following code rescans |\jobname|, stores the result
% in |\childdocname| and saves a copy in |\childdocjob|:
%    \begin{macrocode}
\edef\childdocname{\scantokens\expandafter{\jobname\noexpand}}
\let\childdocjob\childdocname
%    \end{macrocode}

% \macro{\childdocdisable}
% The macro |\childdocdisable| prevents the main file
% from being processed more than once.
% At this stage, the main document command |\childdocmain|
% is assumed to be called once again where it should do nothing.
% Any subsequent call to it should prevent
% a secondary processing of the main document
% It overwrites the forwarding commands
% |\childdocof| and |\childdocforward|
% with empty macros to prevent further inclusions of the main document:
%    \begin{macrocode}
\newcommand{\childdocdisable}
{
  \renewcommand{\childdocmain}[1]{\renewcommand{\childdocmain}[1]{\endinput}}
  \renewcommand{\childdocof}[1]{}
  \renewcommand{\childdocby}[2][]{}
  \renewcommand{\childdocforward}[2][]{}
  \renewcommand{\childdocdisable}{}
}
%    \end{macrocode}

% \macro{\childdocmain}
% The macro |\childdocmain| is to be called at the top of the main file
% with nothing or the main filename (without extension) as argument.
% First, it breaks loops.
% If the argument is not empty and does not match |\childdocname|
% (which is set by the first inclusion of |childdoc.def|),
% |\ifchilddoc| is set to true, |\includeonly| is applied to the child file
% and |\jobname| is set to the main file
% (for proper handling of |.aux| files):
%    \begin{macrocode}
\newcommand{\childdocmain}[1]
{
  \childdocdisable\childdocmain{}
  \if?#1?\else
    \begingroup
      \def\childdoctmp{#1}
      \ifx\childdoctmp\childdocname
        \def\childdoctmp{}
      \else
        \def\childdoctmp
        {
          \childdoctrue
          \includeonly{\childdocname}
          \def\childdocjob{#1}
          \def\jobname{#1}
        }
      \fi
      \expandafter
    \endgroup
    \childdoctmp
  \fi
}
%    \end{macrocode}

% \macro{\childdocof}
% The command |\childdocof| redirects
% compilation to the main file |#1|.
%    \begin{macrocode}
\newcommand{\childdocof}[1]
{
  \childdocdisable
  \childdoctrue
  \includeonly{\childdocname}
  \def\jobname{#1}
  \def\childdocjob{#1}
  \input{#1}
}
%    \end{macrocode}

% \macro{\childdocby}
% The command |\childdocby| ....
%    \begin{macrocode}
\newcommand{\childdocby}[2][]
{
  \childdocdisable
  \childdoctrue
  \childdocmanualtrue
  \if?#1?\else
    \def\jobname{#2}
  \fi
  \def\childdocjob{#2}
  \input{#2}
  \endinput
}
%    \end{macrocode}

% \macro{\childdocforward}
% The command |\childdocforward| redirects
% compilation to the main file or
% (if the optional argument is given) a child file.
% Parameters are set as if the main file
% or a child file starting with |\childdocof| was compiled.
% Then compilation is handed over to the main file:
%    \begin{macrocode}
\newcommand{\childdocforward}[2][]
{
  \begingroup
    \if?#1?
      \def\childdoctmp
      {
        \def\childdocname{#2}
        \def\childdocjob{#2}
        \def\jobname{#2}
        \input{#2}
        \endinput
      }
    \else
      \def\childdoctmp
      {
        \childdocdisable
        \def\childdocname{#2}
        \childdoctrue
        \includeonly{#2}
        \def\childdocjob{#1}
        \def\jobname{#1}
        \input{#1}
        \endinput
      }
    \fi
    \expandafter
  \endgroup
  \childdoctmp
}
%    \end{macrocode}

% \macro{\childdocforwardprefix}
% The command |\childdocforwardprefix| redirects
% compilation to the main or a child file by means of a pattern.
% The prefix |#1| in the current filename is replaced by |#2|
% and the suffix of the current filename is kept
% (it is assumed that the filename does not contain the substring `|~~~|'
% which is used as a delimiter).
% Compilation is handed over to the new file by |\childdocforward|:
%    \begin{macrocode}
\newcommand{\childdocforwardprefix}[3][]
{
  \begingroup
    \def\childdocextract #2##1~~~{\def\childdoctmp{\childdocforward[#1]{#3##1}}}
    \expandafter\childdocextract\childdocname~~~
    \expandafter
  \endgroup
  \childdoctmp
}
%    \end{macrocode}

% \macro{\childdoc}
% The deprecated macro |\childdoc| is a legacy version of |\childdocmain|:
%    \begin{macrocode}
\newcommand{\childdoc}{\childdocmain}
%    \end{macrocode}

% \macro{\childdocredirect}
% The deprecated macro |\childdocredirect| is a legacy version
% of |\childdocforward| and |\childdocforwardprefix|:
%    \begin{macrocode}
\newcommand{\childdocredirect}[2][]
{
  \begingroup
    \if?#1?
      \def\childdoctmp{\childdocforward{#2}}
    \else
      \def\childdoctmp{\childdocforwardprefix{#1}{#2}}
    \fi
    \expandafter
  \endgroup
  \childdoctmp
}
%    \end{macrocode}

%\iffalse
%</package>
%\fi
%
\endinput
|\\
|\childdocforwardprefix[|\textit{main}|]{|\textit{prefix}|}{|\textit{dest}|}|
\end{tabular}
\end{center}
%
the destination file is determined by a pattern
depending on the current file:
To make this work, the current file must be called
`{\textit{prefix}\hspace{0.2em}\textit{suffix}}'
with \textit{prefix} matching precisely the argument.
Processing is then passed on to the file
`{\textit{dest}\hspace{0.2em}\textit{suffix}}'.
Surely, the same effect is achieved by
directly specifying the
argument `{\textit{dest}\hspace{0.2em}\textit{suffix}}'
in the first form.
However, that requires to set up a different file
for each child. With the alternative form of the command
all these files can have exactly the same content
which simplifies setting them up and maintaining them.

For example, the following file |draft.tex|
with a compilation flag |\version| as described in \secref{sec:flags}
compiles the main document as a draft:
%
\begin{center}
\begin{tabular}{l}
|\def\version{draft}|\\
|% \iffalse
%
% childdoc.dtx Copyright (C) 2017-2018 Niklas Beisert
%
% This work may be distributed and/or modified under the
% conditions of the LaTeX Project Public License, either version 1.3
% of this license or (at your option) any later version.
% The latest version of this license is in
%   http://www.latex-project.org/lppl.txt
% and version 1.3 or later is part of all distributions of LaTeX
% version 2005/12/01 or later.
%
% This work has the LPPL maintenance status `maintained'.
%
% The Current Maintainer of this work is Niklas Beisert.
%
% This work consists of the files childdoc.dtx and childdoc.ins
% and the derived files childdoc.def and cdocsamp.tex with
% cdocsch1.tex, cdocsch2.tex, cdocsdrf.tex, cdocsfn1.tex, cdocsfn2.tex.
%
%<package>\ifdefined\childdocmain\endinput\fi
%<package>\ProvidesFile{childdoc.def}[2018/12/30 v2.0 child document driver]
%<samplemain>\ProvidesFile{cdocsamp.tex}[2018/12/30 v2.0 sample for childdoc]
%<*driver>
%\ProvidesFile{childdoc.drv}[2018/12/30 v2.0 childdoc reference manual file]
\PassOptionsToClass{10pt,a4paper}{article}
\documentclass{ltxdoc}

\usepackage[margin=35mm]{geometry}
\usepackage{hyperref}
\usepackage{hyperxmp}
\usepackage[usenames]{color}

\hypersetup{colorlinks=true}
\hypersetup{pdfstartview=FitH}
\hypersetup{pdfpagemode=UseNone}
\hypersetup{pdfsource={}}
\hypersetup{pdflang={en-UK}}
\hypersetup{pdfcopyright={Copyright 2017-2018 Niklas Beisert.
  This work may be distributed and/or modified under the
  conditions of the LaTeX Project Public License, either version 1.3
  of this license or (at your option) any later version.}}
\hypersetup{pdflicenseurl={http://www.latex-project.org/lppl.txt}}
\hypersetup{pdfcontactaddress={ETH Zurich, ITP, HIT K,
  Wolfgang-Pauli-Strasse 27}}
\hypersetup{pdfcontactpostcode={8093}}
\hypersetup{pdfcontactcity={Zurich}}
\hypersetup{pdfcontactcountry={Switzerland}}
\hypersetup{pdfcontactemail={nbeisert@itp.phys.ethz.ch}}
\hypersetup{pdfcontacturl={http://people.phys.ethz.ch/\xmptilde nbeisert/}}

\newcommand{\secref}[1]{\hyperref[#1]{section \ref*{#1}}}

\parskip1ex
\parindent0pt
\let\olditemize\itemize
\def\itemize{\olditemize\parskip0pt}

\begin{document}

\title{The \textsf{childdoc} Package}
\hypersetup{pdftitle={The childdoc Package}}
\author{Niklas Beisert\\[2ex]
  Institut f\"ur Theoretische Physik\\
  Eidgen\"ossische Technische Hochschule Z\"urich\\
  Wolfgang-Pauli-Strasse 27, 8093 Z\"urich, Switzerland\\[1ex]
  \href{mailto:nbeisert@itp.phys.ethz.ch}
  {\texttt{nbeisert@itp.phys.ethz.ch}}}
\hypersetup{pdfauthor={Niklas Beisert}}
\hypersetup{pdfsubject={Manual for the LaTeX2e Package childdoc}}
\date{30 December 2018, \textsf{v2.0}}
\maketitle

\begin{abstract}\noindent
\textsf{childdoc} is a \LaTeXe{} package
that enables the direct compilation
of document sections included by |\include|
to individual files.
\end{abstract}

\begingroup
\parskip0ex
\tableofcontents
\endgroup

%%%%%%%%%%%%%%%%%%%%%%%%%%%%%%%%%%%%%%%%%%%%%%%%%%%%%%%%%%%%%%%%%%%%%%%%%%%%%%%%
%%%%%%%%%%%%%%%%%%%%%%%%%%%%%%%%%%%%%%%%%%%%%%%%%%%%%%%%%%%%%%%%%%%%%%%%%%%%%%%%
\section{Introduction}

\LaTeX{} provides a mechanism to structure a large document (such as a book)
into a main file and several child files (containing the chapters)
using the |\include| command.
This mechanism is beneficial for documents
which span hundreds of pages in order to
make the source file(s) more manageable.
Moreover, compilation can be restricted to
selected child files by means of the |\includeonly| command.
The latter feature can be used to reduce the compilation time while editing
(this was significantly more useful in the earlier days of \LaTeX{})
or to generate a smaller document which is easier to navigate.
Another application of |\includeonly| is to generate
documents consisting of selected parts of the complete document.

However, there are a few drawbacks of the plain |\include| mechanism:
\begin{itemize}
\item
The child files cannot be compiled on their own,
they can only be compiled via the main file.
A naive editing environment
(such as a text editor with an option
to have the current file processed by \LaTeX)
may require one to switch to the main file before compiling;
attempting to compile the child file produces errors.
\item
The main file must be modified (each time)
to adjust the |\includeonly| command
to the present needs. This easily leaves the main file in a messy state.
\item
The generated document will always carry the filename
of the main document. This is inconvenient if
several child files are to be compiled and
to be kept for distribution.
\end{itemize}

The present package provides a simple interface
to make child files individually compilable by \LaTeX{}.
Compiling a child file then has the same effect as compiling
the main file with an |\includeonly| command
to select the appropriate child.
Moreover the generated document will carry the name of the child
rather than the main file.
This resolves all three above issues.

This feature is meant to make the editing of books,
thesis documents and lecture notes somewhat more convenient.
However, the package can also be used efficiently for
composing a series of documents (such as exercise sheets)
which are typically distributed individually.
It then assists the author in generating the individual documents
(potentially in different versions)
as well as a document containing the collected series.
Another application is in developing style files
or other kinds of included material
where compilation of the style file could redirect
to a sample or test file.

%%%%%%%%%%%%%%%%%%%%%%%%%%%%%%%%%%%%%%%%%%%%%%%%%%%%%%%%%%%%%%%%%%%%%%%%%%%%%%%%
%%%%%%%%%%%%%%%%%%%%%%%%%%%%%%%%%%%%%%%%%%%%%%%%%%%%%%%%%%%%%%%%%%%%%%%%%%%%%%%%
\section{Usage}

First of all, the package \textsf{childdoc} is \emph{not} a standard
\LaTeXe{} |.sty| style file! Therefore it needs to be invoked in
a non-standard way.

%%%%%%%%%%%%%%%%%%%%%%%%%%%%%%%%%%%%%%%%%%%%%%%%%%%%%%%%%%%%%%%%%%%%%%%%%%%%%%%%
\subsection{Included Files}
\label{sec:include}

%%%%%%%%%%%%%%%%%%%%%%%%%%%%%%%%%%%%%%%%
\DescribeMacro{\childdocmain}
To use the package, add the commands
\begin{center}
\begin{tabular}{l}
|\input{childdoc.def}|\\
|\childdocmain{}|\\
\end{tabular}
\end{center}
at the very top of the main \LaTeX{} file,
in particular \emph{before} the |\documentclass| statement!
The argument of |\childdocmain| should be left empty
(but it must be present).

%%%%%%%%%%%%%%%%%%%%%%%%%%%%%%%%%%%%%%%%
\DescribeMacro{\childdocof}
Furthermore, add the commands
\begin{center}
\begin{tabular}{l}
|\input{childdoc.def}|\\
|\childdocof{|\textit{main}|}|\\
\end{tabular}
\end{center}
at the top of every child file \textit{child}
which is included by |\include{|\textit{child}|}|
from within the main file
(or at least for those files to be compiled individually).
The argument \textit{main} must be the filename of the main file.

There are a couple of
considerations in setting up the main and child documents:

%%%%%%%%%%%%%%%%%%%%%%%%%%%%%%%%%%%%%%%%
\paragraph{Restrictions.}

Please note the following restrictions:
\begin{itemize}
\item
|\childdocmain| must be called with one argument \textit{main}
to ensure compatibility with earlier version of the package.
It must either be empty (|\childdocmain{}|)
or precisely match the filename of the main file in which it is specified.
See \secref{sec:detection} for further information.
\item
The filename \textit{main} must be specified without the |.tex| extension.
\item
The filename \textit{main} is case sensitive
(even in case-insensitive file systems)
due to internal string comparison.
\item
The argument \textit{main} should be fully expanded, it cannot be a macro.
\item
Subdirectories and special characters should be avoided in filenames.
\item
The command |\childdocmain{|\textit{main}|}| must be followed by a whitespace.
It should not be followed immediately by another command
or by a comment mark `|%|'.
This is because the \TeX{} parser reads the token immediately following
the argument of |\childdocmain| and puts it
at the beginning of every child section;
however, a white\-space is ignored.
\end{itemize}

%%%%%%%%%%%%%%%%%%%%%%%%%%%%%%%%%%%%%%%%
\paragraph{Content of Main File.}

It is advisable to place all content in the child files included by |\include|.
Any output contained in the main file will appear in all child documents
unless suppressed manually;
it cannot be suppressed automatically by the |\includeonly| directive
and thus should normally be avoided.
A method to include some content in the main file
by means of conditional processing is described in \secref{sec:conditional}.

%%%%%%%%%%%%%%%%%%%%%%%%%%%%%%%%%%%%%%%%
\paragraph{Page Numbering.}

When only a part of the document is compiled,
the appropriate numbering of pages
(as well as other status parameters)
is determined from the |.aux| files.
The latter contain information from previous passes.
However this information needs to propagate through
all intermediate child documents.
Therefore the page numbering in child documents may well
be inconsistent until the complete document is compiled at least once.

A useful (if unconventional) way to always ensure a consistent
page numbering is to restart the numbering in each child document
and denote the pages by `\textit{child}|.|\textit{page}'
where \textit{child} represents the chapter/section number of the child file.
This can be achieved by the command
|\numberwithin{page}{|\textit{child}|}|
of the \textsf{amsmath} package
where \textit{child} can be |chapter| or |section|
depending on the chosen structuring.
Alternatively, one can modify the macro |\thepage| appropriately
and reset the counter |page| at the start of each child file.

%%%%%%%%%%%%%%%%%%%%%%%%%%%%%%%%%%%%%%%%%%%%%%%%%%%%%%%%%%%%%%%%%%%%%%%%%%%%%%%%
\subsection{Conditional Processing}
\label{sec:conditional}

The package provides a mechanism to compile different versions
of a document. To customise the versions further some conditional processing
can come in handy to distinguish which version is being compiled.
The package provides two macros to describe the compilation context:

%%%%%%%%%%%%%%%%%%%%%%%%%%%%%%%%%%%%%%%%
\DescribeMacro{\ifchilddoc}
The conditional |\ifchilddoc| distinguishes between the compilation of
child documents and the main document:
%
\begin{center}
|\ifchilddoc |\textit{child-code}| |[|\||else |\textit{main-code}]| \||fi|
\end{center}

%%%%%%%%%%%%%%%%%%%%%%%%%%%%%%%%%%%%%%%%
\DescribeMacro{\childdocname}
\DescribeMacro{\childdocjob}
The macro |\childdocname| contains the filename (without extension)
of the main or child file being processed.
Note that |\childdocjob| will always contain the name of the main file.

%%%%%%%%%%%%%%%%%%%%%%%%%%%%%%%%%%%%%%%%
\paragraph{Title Page.}

Conditional processing can be used to include a title or banner page
in the main document when proper precautions are taken.
Importantly, the code in the main file should ensure that the page counter
(as well as other status parameters which are stored in the |.aux| files)
takes the same value after the conditional processing.
Otherwise the page numbers may take divergent values
depending on which part is compiled.

For example, a title page could be declared by:
%
\begin{center}
\begin{tabular}{l}
|\ifchilddoc\||else|\\
|\addtocounter{page}{-1}|\\
\textit{code for title page}\\
|\newpage|\\
|\||fi|
\end{tabular}
\end{center}
%
A banner page for the child documents can be generated by:
%
\begin{center}
\begin{tabular}{l}
|\ifchilddoc|\\
|\addtocounter{page}{-1}|\\
\textit{code for banner page}\\
|\newpage|\\
|\||fi|
\end{tabular}
\end{center}
%
Here one could write a message such as:
\begin{center}
|This is the part \childdocname{} of \childdocjob{}.|
\end{center}

%%%%%%%%%%%%%%%%%%%%%%%%%%%%%%%%%%%%%%%%%%%%%%%%%%%%%%%%%%%%%%%%%%%%%%%%%%%%%%%%
\subsection{Flags}
\label{sec:flags}

The package makes it easy to generate different versions
of the main or child documents.
To this end compilation flags can be defined
and assigned different default values.
They will be particularly useful in conjunction
with the forwarding mechanism described in \secref{sec:forward}.

For example, it may be useful to have a flag |\version|
which can be set to |draft| or |final|.
The document source will contain some conditional code
depending on the value of |\version|.
Suppose further, the flag should default to |final| for the main file
and to |draft| for child files
which is a natural assignment for editing the document.
This is achieved by placing the following code
in the preamble of the main document
(below the |\childdocmain| directive):
%
\begin{center}
\begin{tabular}{l}
|\ifchilddoc|\\
|\providecommand{\version}{draft}|\\
|\||else|\\
|\providecommand{\version}{final}|\\
|\||fi|
\end{tabular}
\end{center}
%
The definition by |\providecommand| makes sure
that previous definitions are not overwritten.
Further statements |\providecommand{\version}{...}|
can thus be added before the above code to override it.

For the main file, one might add a line
(between |\childdocmain| and the above block)
%
\begin{center}
|%\ifchilddoc\||else\providecommand{\version}{draft}\||fi|
\end{center}
%
which can be uncommented to produce a draft version.
Likewise one can add a line to the very top of a child file
(above the |\childdocof{|\textit{main}|}| directive)
%
\begin{center}
|%\providecommand{\version}{final}|
\end{center}
%
which can be uncommented to produce the final version of this child document.

%%%%%%%%%%%%%%%%%%%%%%%%%%%%%%%%%%%%%%%%%%%%%%%%%%%%%%%%%%%%%%%%%%%%%%%%%%%%%%%%
\subsection{Forwarding}
\label{sec:forward}

Different versions of the main or child documents
using compilation flags as described in \secref{sec:flags}
can be (permanently) stored in different files
for convenient compilation, viewing and distribution.
To this end, the package defines a command
to pass on compilation to a different file:

%%%%%%%%%%%%%%%%%%%%%%%%%%%%%%%%%%%%%%%%
\DescribeMacro{\childdocforward}
The command |\childdocforward| redirects processing to
another source file:
%
\begin{center}
\begin{tabular}{l}
|\input{childdoc.def}|\\
|\childdocforward[|\textit{main}|]{|\textit{dest}|}|\\
\end{tabular}
\end{center}
%
The argument \textit{dest} is the destination file
(without extension).
It should be the main file or one of the child files.
Note that further \textsf{childdoc} directives
such as |\childdocof| and |\childdocforward|
in the indicated file will be processed in this form.
The optional argument \textit{main}
passes on directly to the main file \textit{main}
while pretending to compile the child \textit{dest}.
This form behaves as if \textit{dest}
issues |\childdocof{|\textit{main}|}| right away,
and no further \textsf{childdoc} directives will be processed.

%%%%%%%%%%%%%%%%%%%%%%%%%%%%%%%%%%%%%%%%
\DescribeMacro{\...prefix}
In the alternative form |\childdocforwardprefix|,
%
\begin{center}
\begin{tabular}{l}
|\input{childdoc.def}|\\
|\childdocforwardprefix[|\textit{main}|]{|\textit{prefix}|}{|\textit{dest}|}|
\end{tabular}
\end{center}
%
the destination file is determined by a pattern
depending on the current file:
To make this work, the current file must be called
`{\textit{prefix}\hspace{0.2em}\textit{suffix}}'
with \textit{prefix} matching precisely the argument.
Processing is then passed on to the file
`{\textit{dest}\hspace{0.2em}\textit{suffix}}'.
Surely, the same effect is achieved by
directly specifying the
argument `{\textit{dest}\hspace{0.2em}\textit{suffix}}'
in the first form.
However, that requires to set up a different file
for each child. With the alternative form of the command
all these files can have exactly the same content
which simplifies setting them up and maintaining them.

For example, the following file |draft.tex|
with a compilation flag |\version| as described in \secref{sec:flags}
compiles the main document as a draft:
%
\begin{center}
\begin{tabular}{l}
|\def\version{draft}|\\
|\input{childdoc.def}|\\
|\childdocforward{|\textit{main}|}|
\end{tabular}
\end{center}
%
Likewise, the following files |final|\textit{nn}|.tex|
compile the final version of the child document
|child|\textit{nn}|.tex|:
%
\begin{center}
\begin{tabular}{l}
|\def\version{final}|\\
|\input{childdoc.def}|\\
|\childdocforwardprefix{final}{child}|
\end{tabular}
\end{center}
%

Note that when several versions of a main file and/or of each child file
are to be generated, it may be convenient to set up a |Makefile| or
shell script to automatise the process.

%%%%%%%%%%%%%%%%%%%%%%%%%%%%%%%%%%%%%%%%%%%%%%%%%%%%%%%%%%%%%%%%%%%%%%%%%%%%%%%%
\subsection{Command Line Processing}
\label{sec:commandline}

The effect of redirection files can also be achieved by invoking
the \LaTeX{} compiler with a more elaborate command line.
Most conveniently this should be done as part
of a shell script or a |Makefile|.

When using \textsf{childdoc} in the main file, the following
command lines effectively perform a redirection
(note that depending on the shell being used,
backslashes may have to be doubled: `|\|' $\to$ `|\\|'):
%
\begin{center}
|... -jobname "|\textit{target}|" |\\|"|[\textit{flags}]%
|\input{childdoc.def}\childdocforward[|\textit{main}|]{|\textit{dest}|}"|
\end{center}
%
Here \textit{target} is the name of the output file,
\textit{main} is the name of the main file
and \textit{dest} is the name of the main or child file to be processed
(all filenames without extensions).
The optional argument \textit{main} can be omitted
if \textit{main} matches \textit{dest}.
Optionally, compilation \textit{flags} can be defined via |\def| commands.
This command line makes the \TeX{} engine believe
it is compiling the file \textit{target}
whose content is specified as the latter parameter.
The provided code then forwards the processing to
\textit{main} or \textit{dest} as described in \secref{sec:forward}.

%%%%%%%%%%%%%%%%%%%%%%%%%%%%%%%%%%%%%%%%%%%%%%%%%%%%%%%%%%%%%%%%%%%%%%%%%%%%%%%%
\subsection{Include by Input}
\label{sec:input}

Including child documents by |\include| has some restrictions by design.
Most notably, the content of a child document always occupies
its own set of pages; pages cannot be shared between child documents.
Usually, this behaviour makes perfect sense
because each child document contain an essential part of the document.
However, in some situations it may be desirable to compose
a document from a collection of parts
without having mandatory page breaks between then.
For this case, the package
provides a mechanism to include parts
by |\input| which can also be processed individually.
However, by construction this mechanism
requires manual handling of the content to be output.

%%%%%%%%%%%%%%%%%%%%%%%%%%%%%%%%%%%%%%%%
\DescribeMacro{\ifchilddocmanual}
The main file should be prepared as usual, see \secref{sec:include}.
However, the document body must make a distinction
between processing of an individual part and of the main document, e.g.:
%
\begin{center}
\begin{tabular}{l}
|\ifchilddocmanual|\\
|\input{\childdocname}|\\
|\||else|\\
\textit{document body with }|\input{|\textit{part}|}|\\
|\||fi|
\end{tabular}
\end{center}
%
The conditional |\ifchilddocmanual| is true whenever
a part to be included by |\input| is being compiled,
and the name of the part is stored in |\childdocname|.

%%%%%%%%%%%%%%%%%%%%%%%%%%%%%%%%%%%%%%%%
\DescribeMacro{\childdocby}
Each part to be included by |\input| should start with:
%
\begin{center}
\begin{tabular}{l}
|\input{childdoc.def}|\\
|\childdocby{|\textit{main}|}|\\
\end{tabular}
\end{center}
%
The directive |\childdocby| is similar to |\childdocof|
described in \secref{sec:include},
but the subsequent selection of content must be done manually.
To that end, both |\ifchilddoc| and |\ifchilddocmanual|
will be true upon processing of a part,
and the name of the part is stored in |\childdocname|.
Note that |\jobname| will be set to the filename of the current part
so that each part receives an individual |.aux| file
that does not interfere with the |.aux| file(s) of the main document.
This behaviour can be altered by the alternative form
|\childdocby[*]{|\textit{main}|}| (with a non-empty optional argument)
which uses the |.aux| file of the main document
by setting |\jobname| to \textit{main}.

%%%%%%%%%%%%%%%%%%%%%%%%%%%%%%%%%%%%%%%%%%%%%%%%%%%%%%%%%%%%%%%%%%%%%%%%%%%%%%%%
\subsection{Driver Development}
\label{sec:driver}

The \textsf{childdoc} mechanism can also be use for the development
of definition files such as \LaTeX{} styles or classes.
This case differs from the above setup with multiple parts
included by |\include| in that no |\includeonly| should be invoked.
This can be achieved by starting the include file
(before |\ProvidesPackage|) with:
%
\begin{center}
\begin{tabular}{l}
|\input{childdoc.def}|\\
|\childdocforward{|\textit{main}|}|\\
\end{tabular}
\end{center}
%
or alternatively with:
%
\begin{center}
\begin{tabular}{l}
|\input{childdoc.def}|\\
|\childdocby{|\textit{main}|}|\\
\end{tabular}
\end{center}
%
Both forms have slightly different effects as described above.
The main file is prepared as usual, see \secref{sec:include}.

%%%%%%%%%%%%%%%%%%%%%%%%%%%%%%%%%%%%%%%%%%%%%%%%%%%%%%%%%%%%%%%%%%%%%%%%%%%%%%%%
\subsection{Legacy Detection}
\label{sec:detection}

The directive |\childdocmain| in the main file can detect
whether the complete document or merely a child is to be compiled
even without using the directive |\childdocof|.
This method is deprecated because it is less robust
and there is no compelling reason to use it;
it is merely provided for backward compatibility
and it may be removed in future versions.

If the detection mechanism is to be used,
it is mandatory to correctly specify
the filename of the main file as the argument of |\childdocmain|:
%
\begin{center}
\begin{tabular}{l}
|\input{childdoc.def}|\\
|\childdocmain{|\textit{main}|}|\\
\end{tabular}
\end{center}
%
If |\jobname| does not match the argument \textit{main} of |\childdocmain|,
it is assumed that |\jobname| points to the child file to be compiled.
When using |\childdocmain| with the main file specified as argument,
it suffices to start a child file
with just |\input{|\textit{main}|}|
without loading of the package and using |\childdocof|.
If instead all processing is done
with the appropriate \textsf{childdoc} directives,
the argument of \textit{main} of |\childdocmain| can be empty.

An alternative version of the command line processing described
in \secref{sec:commandline} using the detection mechanism reads:
%
\begin{center}
|... -jobname "|\textit{target}|" "|[\textit{flags}]%
[|\def\jobname{|\textit{dest}|}|]|\input{|\textit{main}|}"|
\end{center}

%%%%%%%%%%%%%%%%%%%%%%%%%%%%%%%%%%%%%%%%%%%%%%%%%%%%%%%%%%%%%%%%%%%%%%%%%%%%%%%%
\subsection{Manual Code}
\label{sec:manual}

In case one cannot be certain whether the definitions file |childdoc.def|
is installed on the target \TeX{} distribution
and one prefers not to ship it,
it is conceivable to paste a few relevant commands into the sources.

To that end, drop all statements |\input{childdoc.def}|
and perform the replacements as outlined below.
Instead of |\childdocmain{|\textit{main}|}| add the following code
to the top of the main file:
%
\begin{center}
\begin{tabular}{l}
|\||ifdefined\childdocname\endinput\||fi\newif\ifchilddoc|\\
|\edef\childdocname{\scantokens\expandafter{\jobname\noexpand}}|\\
|\def\childdocmain{|\textit{main}|}\||ifx\childdocmain\childdocname\||else|\\
|\childdoctrue\includeonly{\childdocname}\let\jobname\childdocmain\||fi|\\
\end{tabular}
\end{center}
%
Instead of |\childdocof{|\textit{main}|}| just include the main file
at the top of each child file:
%
\begin{center}
|\input{|\textit{main}|}|
\end{center}
%
A simple redirection |\childdocforward{|\textit{dest}|}| is achieved by:
%
\begin{center}
|\def\jobname{|\textit{dest}|}\input{\jobname}|
\end{center}
%
The redirection with prefix
|\childdocforwardprefix[|\textit{prefix}|]{|\textit{dest}|}|
is accomplished by:
%
\begin{center}
\begin{tabular}{l}
|{\edef\jobname{\scantokens\expandafter{\jobname\noexpand}}|\\
|\def\redirectjob |\textit{prefix}|#1~~~{\gdef\jobname{|\textit{dest}|#1}}|\\
|\expandafter\redirectjob\jobname~~~}\input{\jobname}|
\end{tabular}
\end{center}

In an alternative approach,
child documents can be compiled by a specific command line
without additional code or specific definitions:
%
\begin{center}
|... -jobname "|\textit{target}|" "|[\textit{flags}]%
|\includeonly{|\textit{dest}|}\input{|\textit{main}|}"|
\end{center}
%

%%%%%%%%%%%%%%%%%%%%%%%%%%%%%%%%%%%%%%%%%%%%%%%%%%%%%%%%%%%%%%%%%%%%%%%%%%%%%%%%
%%%%%%%%%%%%%%%%%%%%%%%%%%%%%%%%%%%%%%%%%%%%%%%%%%%%%%%%%%%%%%%%%%%%%%%%%%%%%%%%
\section{Information}

%%%%%%%%%%%%%%%%%%%%%%%%%%%%%%%%%%%%%%%%%%%%%%%%%%%%%%%%%%%%%%%%%%%%%%%%%%%%%%%%
\subsection{Copyright}

Copyright \copyright{} 2017--2018 Niklas Beisert

This work may be distributed and/or modified under the
conditions of the \LaTeX{} Project Public License, either version 1.3
of this license or (at your option) any later version.
The latest version of this license is in
  \url{http://www.latex-project.org/lppl.txt}
and version 1.3 or later is part of all distributions of \LaTeX{}
version 2005/12/01 or later.

This work has the LPPL maintenance status `maintained'.

The Current Maintainer of this work is Niklas Beisert.

This work consists of the files |README.txt|, |childdoc.ins| and |childdoc.dtx|
as well as the derived files |childdoc.def|, |cdocsamp.tex|
with |cdocsch1.tex|, |cdocsch2.tex|, |cdocspt3.tex|, |cdocspt4.tex|,
|cdocsdrf.tex|, |cdocsfn1.tex|, |cdocsfn2.tex|
as well as |childdoc.pdf|.

%%%%%%%%%%%%%%%%%%%%%%%%%%%%%%%%%%%%%%%%%%%%%%%%%%%%%%%%%%%%%%%%%%%%%%%%%%%%%%%%
\subsection{Files and Installation}

The package consists of the files:
%
\begin{center}
\begin{tabular}{ll}
    |README.txt|   & readme file \\
    |childdoc.ins| & installation file \\
    |childdoc.dtx| & source file \\
    |childdoc.def| & definition file \\
    |cdocsamp.tex| & sample main file \\
    |cdocsch1.tex| & sample include file \\
    |cdocsch2.tex| & sample include file \\
    |cdocspt3.tex| & sample part file \\
    |cdocspt4.tex| & sample part file \\
    |cdocsdrf.tex| & sample redirection file \\
    |cdocsfn1.tex| & sample redirection file \\
    |cdocsfn2.tex| & sample redirection file \\
    |childdoc.pdf| & manual
\end{tabular}
\end{center}
%
The distribution consists of the files
|README.txt|, |childdoc.ins| and |childdoc.dtx|.
%
\begin{itemize}
\item
Run (pdf)\LaTeX{} on |childdoc.dtx|
to compile the manual |childdoc.pdf| (this file).
\item
Run \LaTeX{} on |childdoc.ins| to create the definitions file |childdoc.def|
and the sample |cdocsamp.tex| with include files
|cdocsch1.tex|, |cdocsch2.tex|, |cdocspt3.tex|, |cdocspt4.tex|,
|cdocsdrf.tex|, |cdocsfn1.tex|, |cdocsfn2.tex|.
Then copy the file |childdoc.def| to an appropriate directory of your \LaTeX{}
distribution, e.g.\ \textit{texmf-root}|/tex/latex/childdoc|.
\end{itemize}

%%%%%%%%%%%%%%%%%%%%%%%%%%%%%%%%%%%%%%%%%%%%%%%%%%%%%%%%%%%%%%%%%%%%%%%%%%%%%%%%
\subsection{Related CTAN Packages}

There are several other packages which offer a similar functionality:
%
\begin{itemize}
\item
The packages
\href{http://ctan.org/pkg/docmute}{\textsf{docmute}},
\href{http://ctan.org/pkg/includex}{\textsf{includex}} and
\href{http://ctan.org/pkg/standalone}{\textsf{standalone}}
provide commands to include only the document body of
a child file thus allowing both files to be compiled individually.
\item
The packages \href{http://ctan.org/pkg/subdocs}{\textsf{subdocs}}
and \href{http://ctan.org/pkg/subfiles}{\textsf{subfiles}}
provide structures in which the main and child documents can be
encapsulated and allowing them to be compiled individually.
The inclusion mechanism is different from the conventional |\include|.
\item
The package \href{http://ctan.org/pkg/combine}{\textsf{combine}}
is an elaborate solution to combine several documents into one.
\end{itemize}
%
See also the CTAN topic \href{http://ctan.org/topic/subdocs}{\textsf{subdocs}}
for further related packages.
The present package differs from the above solutions in that
a document structure constructed with the conventional |\include| mechanism
just needs two extra commands at the top of every file
such that all constituent files can be compiled individually.

%%%%%%%%%%%%%%%%%%%%%%%%%%%%%%%%%%%%%%%%%%%%%%%%%%%%%%%%%%%%%%%%%%%%%%%%%%%%%%%%
%\subsection{Feature Suggestions}
%
%The following is a list of features which may be useful for future
%versions of this package:
%%
%\begin{itemize}
%\item
%\ldots
%\end{itemize}

%%%%%%%%%%%%%%%%%%%%%%%%%%%%%%%%%%%%%%%%%%%%%%%%%%%%%%%%%%%%%%%%%%%%%%%%%%%%%%%%
\subsection{Revision History}

%%%%%%%%%%%%%%%%%%%%%%%%%%%%%%%%%%%%%%%%
\paragraph{v2.0:} 2018/12/30

\begin{itemize}
\item
immediate forward processing
\item
added |\childdocby| mechanism
\item
manual restructured
\end{itemize}

%%%%%%%%%%%%%%%%%%%%%%%%%%%%%%%%%%%%%%%%
\paragraph{v1.6:} 2018/01/17

\begin{itemize}
\item
application for development of include files
\item
corrections to manual
\end{itemize}

%%%%%%%%%%%%%%%%%%%%%%%%%%%%%%%%%%%%%%%%
\paragraph{v1.5:} 2017/05/21

\begin{itemize}
\item
more complete structuring introduced
\item
|\childdocof| introduced
\item
|\childdoc| renamed to |\childdocmain|
\item
|\childredirect| renamed to |\childdocforward| and |\childdocforwardprefix|
and functionality expanded
\end{itemize}

%%%%%%%%%%%%%%%%%%%%%%%%%%%%%%%%%%%%%%%%
\paragraph{v1.0:} 2017/04/27

\begin{itemize}
\item
manual and install package
\item
first version published on CTAN
\end{itemize}

%%%%%%%%%%%%%%%%%%%%%%%%%%%%%%%%%%%%%%%%
\paragraph{v0.6:} 2017/04/26

\begin{itemize}
\item
redirection mechanism added
\end{itemize}

%%%%%%%%%%%%%%%%%%%%%%%%%%%%%%%%%%%%%%%%
\paragraph{v0.5:} 2017/04/26

\begin{itemize}
\item
functionality in definition file
\end{itemize}


%%%%%%%%%%%%%%%%%%%%%%%%%%%%%%%%%%%%%%%%%%%%%%%%%%%%%%%%%%%%%%%%%%%%%%%%%%%%%%%%
%%%%%%%%%%%%%%%%%%%%%%%%%%%%%%%%%%%%%%%%%%%%%%%%%%%%%%%%%%%%%%%%%%%%%%%%%%%%%%%%
%%%%%%%%%%%%%%%%%%%%%%%%%%%%%%%%%%%%%%%%%%%%%%%%%%%%%%%%%%%%%%%%%%%%%%%%%%%%%%%%
\appendix

\settowidth\MacroIndent{\rmfamily\scriptsize 000\ }

 \DocInput{childdoc.dtx}

\end{document}
%</driver>
% \fi
%
% %%%%%%%%%%%%%%%%%%%%%%%%%%%%%%%%%%%%%%%%%%%%%%%%%%%%%%%%%%%%%%%%%%%%%%%%%%%%%%
% %%%%%%%%%%%%%%%%%%%%%%%%%%%%%%%%%%%%%%%%%%%%%%%%%%%%%%%%%%%%%%%%%%%%%%%%%%%%%%
% \section{Sample}
%\iffalse
%<*samplemain>
%\fi
%
% The following presents a sample document
% with two chapters, two parts, a title page,
% a compile flag as well as three forwarding files to set the flag.
% It consists of eight |.tex| files:
% \begin{center}
% \begin{tabular}{ll}
% |cdocsamp.tex|&main file\\
% |cdocsch1.tex|&include file for chapter 1\\
% |cdocsch2.tex|&include file for chapter 2\\
% |cdocspt3.tex|&include file for part 3\\
% |cdocspt4.tex|&include file for part 4\\
% |cdocsdrf.tex|&forwarding file for main file in draft mode\\
% |cdocsfi1.tex|&forwarding file for final version of chapter 1\\
% |cdocsfi2.tex|&forwarding file for final version of chapter 2\\
% \end{tabular}
% \end{center}
% Each of the eight files can be compiled directly by the \LaTeX{} compiler.
%
% %%%%%%%%%%%%%%%%%%%%%%%%%%%%%%%%%%%%%%
% \paragraph{Main File.}
%
% The main file is called |cdocsamp.tex|.
%
% Load the \textsf{childdoc} definitions and
% declare the filename for the main document:
%    \begin{macrocode}
\input{childdoc.def}
\childdocmain{}
%    \end{macrocode}

% Optional override for |\version| flag:
%    \begin{macrocode}
%%\ifchilddoc\else\providecommand{\version}{draft}\fi
%    \end{macrocode}

% Define the default values for the |\version| flag
% (|final| for the main file and |draft| for childs):
%    \begin{macrocode}
\ifchilddoc
\providecommand{\version}{draft}
\else
\providecommand{\version}{final}
\fi
%    \end{macrocode}

% Load the standard document class:
%    \begin{macrocode}
\documentclass[12pt]{article}
%    \end{macrocode}

% Start the document body:
%    \begin{macrocode}
\begin{document}
%    \end{macrocode}

% Declare a title page.
% Print title, part of document being processed and version flag:
%    \begin{macrocode}
\addtocounter{page}{-1}
\begin{center}
{\LARGE\bfseries{}childdoc example\par}
\vspace{1cm}
\ifchilddoc
\ifchilddocmanual part\else chapter\fi:
`\childdocname' of `\childdocjob'\par
\else
main document: `\childdocjob'\par
\fi
version: \version\par
\end{center}
\newpage
%    \end{macrocode}

% Manually include selected file,
% otherwise process as usual:
%    \begin{macrocode}
\ifchilddocmanual
\section*{part `\childdocname'}
\input{\childdocname}
\else
%    \end{macrocode}

% Include the two chapters:
%    \begin{macrocode}
\include{cdocsch1}
\include{cdocsch2}
%    \end{macrocode}

% Include the two parts unless only chapters should be displayed:
%    \begin{macrocode}
\ifchilddoc\else
\section{part three}
\input{cdocspt3}
\section{part four}
\input{cdocspt4}
\fi
%    \end{macrocode}

% Process as usual until here:
%    \begin{macrocode}
\fi
%    \end{macrocode}

% End of document body:
%    \begin{macrocode}
\end{document}
%    \end{macrocode}
%\iffalse
%</samplemain>
%\fi
%
% %%%%%%%%%%%%%%%%%%%%%%%%%%%%%%%%%%%%%%
% \paragraph{Chapter Include Files.}
%
% The include files are called |cdocsch1.tex| and |cdocsch2.tex|.
%
%\iffalse
%<*samplechap1|samplechap2>
%\fi

% Optional override for |\version| flag:
%    \begin{macrocode}
%%\providecommand{\version}{final}
%    \end{macrocode}

% Include the main document:
%    \begin{macrocode}
\input{childdoc.def}
\childdocof{cdocsamp}
%    \end{macrocode}

%\iffalse
%</samplechap1|samplechap2>
%\fi
%
%\iffalse
%<*samplechap1>
%\fi
% Some text for chapter 1:
%    \begin{macrocode}
\section{one}
some text in chapter one
%    \end{macrocode}

%\iffalse
%</samplechap1>
%\fi
% Some text for chapter 2:
%\iffalse
%<*samplechap2>
%\fi
%    \begin{macrocode}
\section{two}
more text in chapter two
%    \end{macrocode}

%\iffalse
%</samplechap2>
%\fi
%
% %%%%%%%%%%%%%%%%%%%%%%%%%%%%%%%%%%%%%%
% \paragraph{Part Include Files.}
%
% The include files are called |cdocspt3.tex| and |cdocspt4.tex|.
%
%\iffalse
%<*samplepart3|samplepart4>
%\fi

% Optional override for |\version| flag:
%    \begin{macrocode}
%%\providecommand{\version}{final}
%    \end{macrocode}

% Include the main document:
%    \begin{macrocode}
\input{childdoc.def}
\childdocby{cdocsamp}
%    \end{macrocode}

%\iffalse
%</samplepart3|samplepart4>
%\fi
%
%\iffalse
%<*samplepart3>
%\fi
% Some text for part 3:
%    \begin{macrocode}
some text in part three
%    \end{macrocode}

%\iffalse
%</samplepart3>
%\fi
% Some text for part 4:
%\iffalse
%<*samplepart4>
%\fi
%    \begin{macrocode}
more text in part four
%    \end{macrocode}

%\iffalse
%</samplepart4>
%\fi
%
% %%%%%%%%%%%%%%%%%%%%%%%%%%%%%%%%%%%%%%
% \paragraph{Forwarding for a Complete Draft.}
%
% The following forwarding file |cdocsdrf.tex|
% compiles the main document in draft mode:
%\iffalse
%<*sampledraft>
%\fi
%    \begin{macrocode}
\def\version{draft}
\input{childdoc.def}
\childdocforward{cdocsamp}
%    \end{macrocode}

%\iffalse
%</sampledraft>
%\fi
%
% %%%%%%%%%%%%%%%%%%%%%%%%%%%%%%%%%%%%%%
% \paragraph{Forwarding for Final Version of the Chapters.}
%
% The following forwarding files |cdocsfn1.tex| and |cdocsfn2.tex|
% (with identical content)
% compile the final versions of the child documents
% |cdocsch1.tex| and |cdocsch2.tex|, respectively:
%\iffalse
%<*samplefinal>
%\fi
%    \begin{macrocode}
\def\version{final}
\input{childdoc.def}
\childdocforwardprefix[cdocsamp]{cdocsfn}{cdocsch}
%    \end{macrocode}

%\iffalse
%</samplefinal>
%\fi
%
% %%%%%%%%%%%%%%%%%%%%%%%%%%%%%%%%%%%%%%
% \paragraph{Command Line Processing.}
%
% The following three command lines generate the output files
% |cdocscld|, |cdocscl1| and |cdocscl2|
% which should be identical to
% |cdocsdrf|, |cdocsch1| and |cdocsfn2|, respectively:
% \begin{center}
% \begin{tabular}{l}
% |latex -jobname cdocscld \|\\
% |  "\def\version{draft}\input{childdoc.def}\childdocforward{cdocsamp}"|\\
% |latex -jobname cdocscl1 \|\\
% |  "\input{childdoc.def}\childdocforward[cdocsamp]{cdocsch1}"|\\
% |latex -jobname cdocscl2 \|\\
% |  "\def\version{final}\input{childdoc.def}\childdocforward{cdocsch2}"|
% \end{tabular}
% \end{center}
% Note that the trailing backslash on each first line
% merely continues the input to the second line
% (for convenient cut ant paste).
% Furthermore, the command |latex| can be replaced by any
% of its alternative versions such as |pdflatex|.
%
% %%%%%%%%%%%%%%%%%%%%%%%%%%%%%%%%%%%%%%%%%%%%%%%%%%%%%%%%%%%%%%%%%%%%%%%%%%%%%%
% %%%%%%%%%%%%%%%%%%%%%%%%%%%%%%%%%%%%%%%%%%%%%%%%%%%%%%%%%%%%%%%%%%%%%%%%%%%%%%
% \section{Implementation}
%\iffalse
%<*package>
%\fi
%
% This section describes the definitions file |childdoc.def|.

% The definitions cannot be loaded using |\usepackage| or |\RequirePackage|
% which has a mechanism to prevent loading a style file more than once.
% When loading the definitions by means of |\input|
% multiple instances have to be prevented manually:
%\iffalse
%This code needs to be before the `\ProvidesFile' directive
%which is defined at the beginning of this file.
%Therefore it is also placed there and commented out here.
%</package>
%<*discard>
%\fi
%    \begin{macrocode}
\ifdefined\childdocmain\endinput\fi
%    \end{macrocode}
%\iffalse
%</discard>
%<*package>
%\fi
%
% \macro{\ifchilddoc}
% \macro{\ifchilddocmanual}
% The conditional |\ifchilddoc| tells whether a
% child (true) or main (false) document is being compiled.
% The conditional |\ifchilddocmanual| tells whether
% the |\includeonly| mechanism is used (false) or
% the selection of child files must be performed manually (true).
% The definitions initialise to false:
%    \begin{macrocode}
\newif\ifchilddoc
\newif\ifchilddocmanual
%    \end{macrocode}

% \macro{\childdocname}
% \macro{\childdocjob}
% The macro |\childdocname| stores the name of the main document
% to be compiled. The macro |\childdocjob| stores the name of
% the document on which the \LaTeX{} compiler was originally invoked.
% The content of |\jobname| cannot be compared
% to filenames specified in the source due to different catcodes.
% The following code rescans |\jobname|, stores the result
% in |\childdocname| and saves a copy in |\childdocjob|:
%    \begin{macrocode}
\edef\childdocname{\scantokens\expandafter{\jobname\noexpand}}
\let\childdocjob\childdocname
%    \end{macrocode}

% \macro{\childdocdisable}
% The macro |\childdocdisable| prevents the main file
% from being processed more than once.
% At this stage, the main document command |\childdocmain|
% is assumed to be called once again where it should do nothing.
% Any subsequent call to it should prevent
% a secondary processing of the main document
% It overwrites the forwarding commands
% |\childdocof| and |\childdocforward|
% with empty macros to prevent further inclusions of the main document:
%    \begin{macrocode}
\newcommand{\childdocdisable}
{
  \renewcommand{\childdocmain}[1]{\renewcommand{\childdocmain}[1]{\endinput}}
  \renewcommand{\childdocof}[1]{}
  \renewcommand{\childdocby}[2][]{}
  \renewcommand{\childdocforward}[2][]{}
  \renewcommand{\childdocdisable}{}
}
%    \end{macrocode}

% \macro{\childdocmain}
% The macro |\childdocmain| is to be called at the top of the main file
% with nothing or the main filename (without extension) as argument.
% First, it breaks loops.
% If the argument is not empty and does not match |\childdocname|
% (which is set by the first inclusion of |childdoc.def|),
% |\ifchilddoc| is set to true, |\includeonly| is applied to the child file
% and |\jobname| is set to the main file
% (for proper handling of |.aux| files):
%    \begin{macrocode}
\newcommand{\childdocmain}[1]
{
  \childdocdisable\childdocmain{}
  \if?#1?\else
    \begingroup
      \def\childdoctmp{#1}
      \ifx\childdoctmp\childdocname
        \def\childdoctmp{}
      \else
        \def\childdoctmp
        {
          \childdoctrue
          \includeonly{\childdocname}
          \def\childdocjob{#1}
          \def\jobname{#1}
        }
      \fi
      \expandafter
    \endgroup
    \childdoctmp
  \fi
}
%    \end{macrocode}

% \macro{\childdocof}
% The command |\childdocof| redirects
% compilation to the main file |#1|.
%    \begin{macrocode}
\newcommand{\childdocof}[1]
{
  \childdocdisable
  \childdoctrue
  \includeonly{\childdocname}
  \def\jobname{#1}
  \def\childdocjob{#1}
  \input{#1}
}
%    \end{macrocode}

% \macro{\childdocby}
% The command |\childdocby| ....
%    \begin{macrocode}
\newcommand{\childdocby}[2][]
{
  \childdocdisable
  \childdoctrue
  \childdocmanualtrue
  \if?#1?\else
    \def\jobname{#2}
  \fi
  \def\childdocjob{#2}
  \input{#2}
  \endinput
}
%    \end{macrocode}

% \macro{\childdocforward}
% The command |\childdocforward| redirects
% compilation to the main file or
% (if the optional argument is given) a child file.
% Parameters are set as if the main file
% or a child file starting with |\childdocof| was compiled.
% Then compilation is handed over to the main file:
%    \begin{macrocode}
\newcommand{\childdocforward}[2][]
{
  \begingroup
    \if?#1?
      \def\childdoctmp
      {
        \def\childdocname{#2}
        \def\childdocjob{#2}
        \def\jobname{#2}
        \input{#2}
        \endinput
      }
    \else
      \def\childdoctmp
      {
        \childdocdisable
        \def\childdocname{#2}
        \childdoctrue
        \includeonly{#2}
        \def\childdocjob{#1}
        \def\jobname{#1}
        \input{#1}
        \endinput
      }
    \fi
    \expandafter
  \endgroup
  \childdoctmp
}
%    \end{macrocode}

% \macro{\childdocforwardprefix}
% The command |\childdocforwardprefix| redirects
% compilation to the main or a child file by means of a pattern.
% The prefix |#1| in the current filename is replaced by |#2|
% and the suffix of the current filename is kept
% (it is assumed that the filename does not contain the substring `|~~~|'
% which is used as a delimiter).
% Compilation is handed over to the new file by |\childdocforward|:
%    \begin{macrocode}
\newcommand{\childdocforwardprefix}[3][]
{
  \begingroup
    \def\childdocextract #2##1~~~{\def\childdoctmp{\childdocforward[#1]{#3##1}}}
    \expandafter\childdocextract\childdocname~~~
    \expandafter
  \endgroup
  \childdoctmp
}
%    \end{macrocode}

% \macro{\childdoc}
% The deprecated macro |\childdoc| is a legacy version of |\childdocmain|:
%    \begin{macrocode}
\newcommand{\childdoc}{\childdocmain}
%    \end{macrocode}

% \macro{\childdocredirect}
% The deprecated macro |\childdocredirect| is a legacy version
% of |\childdocforward| and |\childdocforwardprefix|:
%    \begin{macrocode}
\newcommand{\childdocredirect}[2][]
{
  \begingroup
    \if?#1?
      \def\childdoctmp{\childdocforward{#2}}
    \else
      \def\childdoctmp{\childdocforwardprefix{#1}{#2}}
    \fi
    \expandafter
  \endgroup
  \childdoctmp
}
%    \end{macrocode}

%\iffalse
%</package>
%\fi
%
\endinput
|\\
|\childdocforward{|\textit{main}|}|
\end{tabular}
\end{center}
%
Likewise, the following files |final|\textit{nn}|.tex|
compile the final version of the child document
|child|\textit{nn}|.tex|:
%
\begin{center}
\begin{tabular}{l}
|\def\version{final}|\\
|% \iffalse
%
% childdoc.dtx Copyright (C) 2017-2018 Niklas Beisert
%
% This work may be distributed and/or modified under the
% conditions of the LaTeX Project Public License, either version 1.3
% of this license or (at your option) any later version.
% The latest version of this license is in
%   http://www.latex-project.org/lppl.txt
% and version 1.3 or later is part of all distributions of LaTeX
% version 2005/12/01 or later.
%
% This work has the LPPL maintenance status `maintained'.
%
% The Current Maintainer of this work is Niklas Beisert.
%
% This work consists of the files childdoc.dtx and childdoc.ins
% and the derived files childdoc.def and cdocsamp.tex with
% cdocsch1.tex, cdocsch2.tex, cdocsdrf.tex, cdocsfn1.tex, cdocsfn2.tex.
%
%<package>\ifdefined\childdocmain\endinput\fi
%<package>\ProvidesFile{childdoc.def}[2018/12/30 v2.0 child document driver]
%<samplemain>\ProvidesFile{cdocsamp.tex}[2018/12/30 v2.0 sample for childdoc]
%<*driver>
%\ProvidesFile{childdoc.drv}[2018/12/30 v2.0 childdoc reference manual file]
\PassOptionsToClass{10pt,a4paper}{article}
\documentclass{ltxdoc}

\usepackage[margin=35mm]{geometry}
\usepackage{hyperref}
\usepackage{hyperxmp}
\usepackage[usenames]{color}

\hypersetup{colorlinks=true}
\hypersetup{pdfstartview=FitH}
\hypersetup{pdfpagemode=UseNone}
\hypersetup{pdfsource={}}
\hypersetup{pdflang={en-UK}}
\hypersetup{pdfcopyright={Copyright 2017-2018 Niklas Beisert.
  This work may be distributed and/or modified under the
  conditions of the LaTeX Project Public License, either version 1.3
  of this license or (at your option) any later version.}}
\hypersetup{pdflicenseurl={http://www.latex-project.org/lppl.txt}}
\hypersetup{pdfcontactaddress={ETH Zurich, ITP, HIT K,
  Wolfgang-Pauli-Strasse 27}}
\hypersetup{pdfcontactpostcode={8093}}
\hypersetup{pdfcontactcity={Zurich}}
\hypersetup{pdfcontactcountry={Switzerland}}
\hypersetup{pdfcontactemail={nbeisert@itp.phys.ethz.ch}}
\hypersetup{pdfcontacturl={http://people.phys.ethz.ch/\xmptilde nbeisert/}}

\newcommand{\secref}[1]{\hyperref[#1]{section \ref*{#1}}}

\parskip1ex
\parindent0pt
\let\olditemize\itemize
\def\itemize{\olditemize\parskip0pt}

\begin{document}

\title{The \textsf{childdoc} Package}
\hypersetup{pdftitle={The childdoc Package}}
\author{Niklas Beisert\\[2ex]
  Institut f\"ur Theoretische Physik\\
  Eidgen\"ossische Technische Hochschule Z\"urich\\
  Wolfgang-Pauli-Strasse 27, 8093 Z\"urich, Switzerland\\[1ex]
  \href{mailto:nbeisert@itp.phys.ethz.ch}
  {\texttt{nbeisert@itp.phys.ethz.ch}}}
\hypersetup{pdfauthor={Niklas Beisert}}
\hypersetup{pdfsubject={Manual for the LaTeX2e Package childdoc}}
\date{30 December 2018, \textsf{v2.0}}
\maketitle

\begin{abstract}\noindent
\textsf{childdoc} is a \LaTeXe{} package
that enables the direct compilation
of document sections included by |\include|
to individual files.
\end{abstract}

\begingroup
\parskip0ex
\tableofcontents
\endgroup

%%%%%%%%%%%%%%%%%%%%%%%%%%%%%%%%%%%%%%%%%%%%%%%%%%%%%%%%%%%%%%%%%%%%%%%%%%%%%%%%
%%%%%%%%%%%%%%%%%%%%%%%%%%%%%%%%%%%%%%%%%%%%%%%%%%%%%%%%%%%%%%%%%%%%%%%%%%%%%%%%
\section{Introduction}

\LaTeX{} provides a mechanism to structure a large document (such as a book)
into a main file and several child files (containing the chapters)
using the |\include| command.
This mechanism is beneficial for documents
which span hundreds of pages in order to
make the source file(s) more manageable.
Moreover, compilation can be restricted to
selected child files by means of the |\includeonly| command.
The latter feature can be used to reduce the compilation time while editing
(this was significantly more useful in the earlier days of \LaTeX{})
or to generate a smaller document which is easier to navigate.
Another application of |\includeonly| is to generate
documents consisting of selected parts of the complete document.

However, there are a few drawbacks of the plain |\include| mechanism:
\begin{itemize}
\item
The child files cannot be compiled on their own,
they can only be compiled via the main file.
A naive editing environment
(such as a text editor with an option
to have the current file processed by \LaTeX)
may require one to switch to the main file before compiling;
attempting to compile the child file produces errors.
\item
The main file must be modified (each time)
to adjust the |\includeonly| command
to the present needs. This easily leaves the main file in a messy state.
\item
The generated document will always carry the filename
of the main document. This is inconvenient if
several child files are to be compiled and
to be kept for distribution.
\end{itemize}

The present package provides a simple interface
to make child files individually compilable by \LaTeX{}.
Compiling a child file then has the same effect as compiling
the main file with an |\includeonly| command
to select the appropriate child.
Moreover the generated document will carry the name of the child
rather than the main file.
This resolves all three above issues.

This feature is meant to make the editing of books,
thesis documents and lecture notes somewhat more convenient.
However, the package can also be used efficiently for
composing a series of documents (such as exercise sheets)
which are typically distributed individually.
It then assists the author in generating the individual documents
(potentially in different versions)
as well as a document containing the collected series.
Another application is in developing style files
or other kinds of included material
where compilation of the style file could redirect
to a sample or test file.

%%%%%%%%%%%%%%%%%%%%%%%%%%%%%%%%%%%%%%%%%%%%%%%%%%%%%%%%%%%%%%%%%%%%%%%%%%%%%%%%
%%%%%%%%%%%%%%%%%%%%%%%%%%%%%%%%%%%%%%%%%%%%%%%%%%%%%%%%%%%%%%%%%%%%%%%%%%%%%%%%
\section{Usage}

First of all, the package \textsf{childdoc} is \emph{not} a standard
\LaTeXe{} |.sty| style file! Therefore it needs to be invoked in
a non-standard way.

%%%%%%%%%%%%%%%%%%%%%%%%%%%%%%%%%%%%%%%%%%%%%%%%%%%%%%%%%%%%%%%%%%%%%%%%%%%%%%%%
\subsection{Included Files}
\label{sec:include}

%%%%%%%%%%%%%%%%%%%%%%%%%%%%%%%%%%%%%%%%
\DescribeMacro{\childdocmain}
To use the package, add the commands
\begin{center}
\begin{tabular}{l}
|\input{childdoc.def}|\\
|\childdocmain{}|\\
\end{tabular}
\end{center}
at the very top of the main \LaTeX{} file,
in particular \emph{before} the |\documentclass| statement!
The argument of |\childdocmain| should be left empty
(but it must be present).

%%%%%%%%%%%%%%%%%%%%%%%%%%%%%%%%%%%%%%%%
\DescribeMacro{\childdocof}
Furthermore, add the commands
\begin{center}
\begin{tabular}{l}
|\input{childdoc.def}|\\
|\childdocof{|\textit{main}|}|\\
\end{tabular}
\end{center}
at the top of every child file \textit{child}
which is included by |\include{|\textit{child}|}|
from within the main file
(or at least for those files to be compiled individually).
The argument \textit{main} must be the filename of the main file.

There are a couple of
considerations in setting up the main and child documents:

%%%%%%%%%%%%%%%%%%%%%%%%%%%%%%%%%%%%%%%%
\paragraph{Restrictions.}

Please note the following restrictions:
\begin{itemize}
\item
|\childdocmain| must be called with one argument \textit{main}
to ensure compatibility with earlier version of the package.
It must either be empty (|\childdocmain{}|)
or precisely match the filename of the main file in which it is specified.
See \secref{sec:detection} for further information.
\item
The filename \textit{main} must be specified without the |.tex| extension.
\item
The filename \textit{main} is case sensitive
(even in case-insensitive file systems)
due to internal string comparison.
\item
The argument \textit{main} should be fully expanded, it cannot be a macro.
\item
Subdirectories and special characters should be avoided in filenames.
\item
The command |\childdocmain{|\textit{main}|}| must be followed by a whitespace.
It should not be followed immediately by another command
or by a comment mark `|%|'.
This is because the \TeX{} parser reads the token immediately following
the argument of |\childdocmain| and puts it
at the beginning of every child section;
however, a white\-space is ignored.
\end{itemize}

%%%%%%%%%%%%%%%%%%%%%%%%%%%%%%%%%%%%%%%%
\paragraph{Content of Main File.}

It is advisable to place all content in the child files included by |\include|.
Any output contained in the main file will appear in all child documents
unless suppressed manually;
it cannot be suppressed automatically by the |\includeonly| directive
and thus should normally be avoided.
A method to include some content in the main file
by means of conditional processing is described in \secref{sec:conditional}.

%%%%%%%%%%%%%%%%%%%%%%%%%%%%%%%%%%%%%%%%
\paragraph{Page Numbering.}

When only a part of the document is compiled,
the appropriate numbering of pages
(as well as other status parameters)
is determined from the |.aux| files.
The latter contain information from previous passes.
However this information needs to propagate through
all intermediate child documents.
Therefore the page numbering in child documents may well
be inconsistent until the complete document is compiled at least once.

A useful (if unconventional) way to always ensure a consistent
page numbering is to restart the numbering in each child document
and denote the pages by `\textit{child}|.|\textit{page}'
where \textit{child} represents the chapter/section number of the child file.
This can be achieved by the command
|\numberwithin{page}{|\textit{child}|}|
of the \textsf{amsmath} package
where \textit{child} can be |chapter| or |section|
depending on the chosen structuring.
Alternatively, one can modify the macro |\thepage| appropriately
and reset the counter |page| at the start of each child file.

%%%%%%%%%%%%%%%%%%%%%%%%%%%%%%%%%%%%%%%%%%%%%%%%%%%%%%%%%%%%%%%%%%%%%%%%%%%%%%%%
\subsection{Conditional Processing}
\label{sec:conditional}

The package provides a mechanism to compile different versions
of a document. To customise the versions further some conditional processing
can come in handy to distinguish which version is being compiled.
The package provides two macros to describe the compilation context:

%%%%%%%%%%%%%%%%%%%%%%%%%%%%%%%%%%%%%%%%
\DescribeMacro{\ifchilddoc}
The conditional |\ifchilddoc| distinguishes between the compilation of
child documents and the main document:
%
\begin{center}
|\ifchilddoc |\textit{child-code}| |[|\||else |\textit{main-code}]| \||fi|
\end{center}

%%%%%%%%%%%%%%%%%%%%%%%%%%%%%%%%%%%%%%%%
\DescribeMacro{\childdocname}
\DescribeMacro{\childdocjob}
The macro |\childdocname| contains the filename (without extension)
of the main or child file being processed.
Note that |\childdocjob| will always contain the name of the main file.

%%%%%%%%%%%%%%%%%%%%%%%%%%%%%%%%%%%%%%%%
\paragraph{Title Page.}

Conditional processing can be used to include a title or banner page
in the main document when proper precautions are taken.
Importantly, the code in the main file should ensure that the page counter
(as well as other status parameters which are stored in the |.aux| files)
takes the same value after the conditional processing.
Otherwise the page numbers may take divergent values
depending on which part is compiled.

For example, a title page could be declared by:
%
\begin{center}
\begin{tabular}{l}
|\ifchilddoc\||else|\\
|\addtocounter{page}{-1}|\\
\textit{code for title page}\\
|\newpage|\\
|\||fi|
\end{tabular}
\end{center}
%
A banner page for the child documents can be generated by:
%
\begin{center}
\begin{tabular}{l}
|\ifchilddoc|\\
|\addtocounter{page}{-1}|\\
\textit{code for banner page}\\
|\newpage|\\
|\||fi|
\end{tabular}
\end{center}
%
Here one could write a message such as:
\begin{center}
|This is the part \childdocname{} of \childdocjob{}.|
\end{center}

%%%%%%%%%%%%%%%%%%%%%%%%%%%%%%%%%%%%%%%%%%%%%%%%%%%%%%%%%%%%%%%%%%%%%%%%%%%%%%%%
\subsection{Flags}
\label{sec:flags}

The package makes it easy to generate different versions
of the main or child documents.
To this end compilation flags can be defined
and assigned different default values.
They will be particularly useful in conjunction
with the forwarding mechanism described in \secref{sec:forward}.

For example, it may be useful to have a flag |\version|
which can be set to |draft| or |final|.
The document source will contain some conditional code
depending on the value of |\version|.
Suppose further, the flag should default to |final| for the main file
and to |draft| for child files
which is a natural assignment for editing the document.
This is achieved by placing the following code
in the preamble of the main document
(below the |\childdocmain| directive):
%
\begin{center}
\begin{tabular}{l}
|\ifchilddoc|\\
|\providecommand{\version}{draft}|\\
|\||else|\\
|\providecommand{\version}{final}|\\
|\||fi|
\end{tabular}
\end{center}
%
The definition by |\providecommand| makes sure
that previous definitions are not overwritten.
Further statements |\providecommand{\version}{...}|
can thus be added before the above code to override it.

For the main file, one might add a line
(between |\childdocmain| and the above block)
%
\begin{center}
|%\ifchilddoc\||else\providecommand{\version}{draft}\||fi|
\end{center}
%
which can be uncommented to produce a draft version.
Likewise one can add a line to the very top of a child file
(above the |\childdocof{|\textit{main}|}| directive)
%
\begin{center}
|%\providecommand{\version}{final}|
\end{center}
%
which can be uncommented to produce the final version of this child document.

%%%%%%%%%%%%%%%%%%%%%%%%%%%%%%%%%%%%%%%%%%%%%%%%%%%%%%%%%%%%%%%%%%%%%%%%%%%%%%%%
\subsection{Forwarding}
\label{sec:forward}

Different versions of the main or child documents
using compilation flags as described in \secref{sec:flags}
can be (permanently) stored in different files
for convenient compilation, viewing and distribution.
To this end, the package defines a command
to pass on compilation to a different file:

%%%%%%%%%%%%%%%%%%%%%%%%%%%%%%%%%%%%%%%%
\DescribeMacro{\childdocforward}
The command |\childdocforward| redirects processing to
another source file:
%
\begin{center}
\begin{tabular}{l}
|\input{childdoc.def}|\\
|\childdocforward[|\textit{main}|]{|\textit{dest}|}|\\
\end{tabular}
\end{center}
%
The argument \textit{dest} is the destination file
(without extension).
It should be the main file or one of the child files.
Note that further \textsf{childdoc} directives
such as |\childdocof| and |\childdocforward|
in the indicated file will be processed in this form.
The optional argument \textit{main}
passes on directly to the main file \textit{main}
while pretending to compile the child \textit{dest}.
This form behaves as if \textit{dest}
issues |\childdocof{|\textit{main}|}| right away,
and no further \textsf{childdoc} directives will be processed.

%%%%%%%%%%%%%%%%%%%%%%%%%%%%%%%%%%%%%%%%
\DescribeMacro{\...prefix}
In the alternative form |\childdocforwardprefix|,
%
\begin{center}
\begin{tabular}{l}
|\input{childdoc.def}|\\
|\childdocforwardprefix[|\textit{main}|]{|\textit{prefix}|}{|\textit{dest}|}|
\end{tabular}
\end{center}
%
the destination file is determined by a pattern
depending on the current file:
To make this work, the current file must be called
`{\textit{prefix}\hspace{0.2em}\textit{suffix}}'
with \textit{prefix} matching precisely the argument.
Processing is then passed on to the file
`{\textit{dest}\hspace{0.2em}\textit{suffix}}'.
Surely, the same effect is achieved by
directly specifying the
argument `{\textit{dest}\hspace{0.2em}\textit{suffix}}'
in the first form.
However, that requires to set up a different file
for each child. With the alternative form of the command
all these files can have exactly the same content
which simplifies setting them up and maintaining them.

For example, the following file |draft.tex|
with a compilation flag |\version| as described in \secref{sec:flags}
compiles the main document as a draft:
%
\begin{center}
\begin{tabular}{l}
|\def\version{draft}|\\
|\input{childdoc.def}|\\
|\childdocforward{|\textit{main}|}|
\end{tabular}
\end{center}
%
Likewise, the following files |final|\textit{nn}|.tex|
compile the final version of the child document
|child|\textit{nn}|.tex|:
%
\begin{center}
\begin{tabular}{l}
|\def\version{final}|\\
|\input{childdoc.def}|\\
|\childdocforwardprefix{final}{child}|
\end{tabular}
\end{center}
%

Note that when several versions of a main file and/or of each child file
are to be generated, it may be convenient to set up a |Makefile| or
shell script to automatise the process.

%%%%%%%%%%%%%%%%%%%%%%%%%%%%%%%%%%%%%%%%%%%%%%%%%%%%%%%%%%%%%%%%%%%%%%%%%%%%%%%%
\subsection{Command Line Processing}
\label{sec:commandline}

The effect of redirection files can also be achieved by invoking
the \LaTeX{} compiler with a more elaborate command line.
Most conveniently this should be done as part
of a shell script or a |Makefile|.

When using \textsf{childdoc} in the main file, the following
command lines effectively perform a redirection
(note that depending on the shell being used,
backslashes may have to be doubled: `|\|' $\to$ `|\\|'):
%
\begin{center}
|... -jobname "|\textit{target}|" |\\|"|[\textit{flags}]%
|\input{childdoc.def}\childdocforward[|\textit{main}|]{|\textit{dest}|}"|
\end{center}
%
Here \textit{target} is the name of the output file,
\textit{main} is the name of the main file
and \textit{dest} is the name of the main or child file to be processed
(all filenames without extensions).
The optional argument \textit{main} can be omitted
if \textit{main} matches \textit{dest}.
Optionally, compilation \textit{flags} can be defined via |\def| commands.
This command line makes the \TeX{} engine believe
it is compiling the file \textit{target}
whose content is specified as the latter parameter.
The provided code then forwards the processing to
\textit{main} or \textit{dest} as described in \secref{sec:forward}.

%%%%%%%%%%%%%%%%%%%%%%%%%%%%%%%%%%%%%%%%%%%%%%%%%%%%%%%%%%%%%%%%%%%%%%%%%%%%%%%%
\subsection{Include by Input}
\label{sec:input}

Including child documents by |\include| has some restrictions by design.
Most notably, the content of a child document always occupies
its own set of pages; pages cannot be shared between child documents.
Usually, this behaviour makes perfect sense
because each child document contain an essential part of the document.
However, in some situations it may be desirable to compose
a document from a collection of parts
without having mandatory page breaks between then.
For this case, the package
provides a mechanism to include parts
by |\input| which can also be processed individually.
However, by construction this mechanism
requires manual handling of the content to be output.

%%%%%%%%%%%%%%%%%%%%%%%%%%%%%%%%%%%%%%%%
\DescribeMacro{\ifchilddocmanual}
The main file should be prepared as usual, see \secref{sec:include}.
However, the document body must make a distinction
between processing of an individual part and of the main document, e.g.:
%
\begin{center}
\begin{tabular}{l}
|\ifchilddocmanual|\\
|\input{\childdocname}|\\
|\||else|\\
\textit{document body with }|\input{|\textit{part}|}|\\
|\||fi|
\end{tabular}
\end{center}
%
The conditional |\ifchilddocmanual| is true whenever
a part to be included by |\input| is being compiled,
and the name of the part is stored in |\childdocname|.

%%%%%%%%%%%%%%%%%%%%%%%%%%%%%%%%%%%%%%%%
\DescribeMacro{\childdocby}
Each part to be included by |\input| should start with:
%
\begin{center}
\begin{tabular}{l}
|\input{childdoc.def}|\\
|\childdocby{|\textit{main}|}|\\
\end{tabular}
\end{center}
%
The directive |\childdocby| is similar to |\childdocof|
described in \secref{sec:include},
but the subsequent selection of content must be done manually.
To that end, both |\ifchilddoc| and |\ifchilddocmanual|
will be true upon processing of a part,
and the name of the part is stored in |\childdocname|.
Note that |\jobname| will be set to the filename of the current part
so that each part receives an individual |.aux| file
that does not interfere with the |.aux| file(s) of the main document.
This behaviour can be altered by the alternative form
|\childdocby[*]{|\textit{main}|}| (with a non-empty optional argument)
which uses the |.aux| file of the main document
by setting |\jobname| to \textit{main}.

%%%%%%%%%%%%%%%%%%%%%%%%%%%%%%%%%%%%%%%%%%%%%%%%%%%%%%%%%%%%%%%%%%%%%%%%%%%%%%%%
\subsection{Driver Development}
\label{sec:driver}

The \textsf{childdoc} mechanism can also be use for the development
of definition files such as \LaTeX{} styles or classes.
This case differs from the above setup with multiple parts
included by |\include| in that no |\includeonly| should be invoked.
This can be achieved by starting the include file
(before |\ProvidesPackage|) with:
%
\begin{center}
\begin{tabular}{l}
|\input{childdoc.def}|\\
|\childdocforward{|\textit{main}|}|\\
\end{tabular}
\end{center}
%
or alternatively with:
%
\begin{center}
\begin{tabular}{l}
|\input{childdoc.def}|\\
|\childdocby{|\textit{main}|}|\\
\end{tabular}
\end{center}
%
Both forms have slightly different effects as described above.
The main file is prepared as usual, see \secref{sec:include}.

%%%%%%%%%%%%%%%%%%%%%%%%%%%%%%%%%%%%%%%%%%%%%%%%%%%%%%%%%%%%%%%%%%%%%%%%%%%%%%%%
\subsection{Legacy Detection}
\label{sec:detection}

The directive |\childdocmain| in the main file can detect
whether the complete document or merely a child is to be compiled
even without using the directive |\childdocof|.
This method is deprecated because it is less robust
and there is no compelling reason to use it;
it is merely provided for backward compatibility
and it may be removed in future versions.

If the detection mechanism is to be used,
it is mandatory to correctly specify
the filename of the main file as the argument of |\childdocmain|:
%
\begin{center}
\begin{tabular}{l}
|\input{childdoc.def}|\\
|\childdocmain{|\textit{main}|}|\\
\end{tabular}
\end{center}
%
If |\jobname| does not match the argument \textit{main} of |\childdocmain|,
it is assumed that |\jobname| points to the child file to be compiled.
When using |\childdocmain| with the main file specified as argument,
it suffices to start a child file
with just |\input{|\textit{main}|}|
without loading of the package and using |\childdocof|.
If instead all processing is done
with the appropriate \textsf{childdoc} directives,
the argument of \textit{main} of |\childdocmain| can be empty.

An alternative version of the command line processing described
in \secref{sec:commandline} using the detection mechanism reads:
%
\begin{center}
|... -jobname "|\textit{target}|" "|[\textit{flags}]%
[|\def\jobname{|\textit{dest}|}|]|\input{|\textit{main}|}"|
\end{center}

%%%%%%%%%%%%%%%%%%%%%%%%%%%%%%%%%%%%%%%%%%%%%%%%%%%%%%%%%%%%%%%%%%%%%%%%%%%%%%%%
\subsection{Manual Code}
\label{sec:manual}

In case one cannot be certain whether the definitions file |childdoc.def|
is installed on the target \TeX{} distribution
and one prefers not to ship it,
it is conceivable to paste a few relevant commands into the sources.

To that end, drop all statements |\input{childdoc.def}|
and perform the replacements as outlined below.
Instead of |\childdocmain{|\textit{main}|}| add the following code
to the top of the main file:
%
\begin{center}
\begin{tabular}{l}
|\||ifdefined\childdocname\endinput\||fi\newif\ifchilddoc|\\
|\edef\childdocname{\scantokens\expandafter{\jobname\noexpand}}|\\
|\def\childdocmain{|\textit{main}|}\||ifx\childdocmain\childdocname\||else|\\
|\childdoctrue\includeonly{\childdocname}\let\jobname\childdocmain\||fi|\\
\end{tabular}
\end{center}
%
Instead of |\childdocof{|\textit{main}|}| just include the main file
at the top of each child file:
%
\begin{center}
|\input{|\textit{main}|}|
\end{center}
%
A simple redirection |\childdocforward{|\textit{dest}|}| is achieved by:
%
\begin{center}
|\def\jobname{|\textit{dest}|}\input{\jobname}|
\end{center}
%
The redirection with prefix
|\childdocforwardprefix[|\textit{prefix}|]{|\textit{dest}|}|
is accomplished by:
%
\begin{center}
\begin{tabular}{l}
|{\edef\jobname{\scantokens\expandafter{\jobname\noexpand}}|\\
|\def\redirectjob |\textit{prefix}|#1~~~{\gdef\jobname{|\textit{dest}|#1}}|\\
|\expandafter\redirectjob\jobname~~~}\input{\jobname}|
\end{tabular}
\end{center}

In an alternative approach,
child documents can be compiled by a specific command line
without additional code or specific definitions:
%
\begin{center}
|... -jobname "|\textit{target}|" "|[\textit{flags}]%
|\includeonly{|\textit{dest}|}\input{|\textit{main}|}"|
\end{center}
%

%%%%%%%%%%%%%%%%%%%%%%%%%%%%%%%%%%%%%%%%%%%%%%%%%%%%%%%%%%%%%%%%%%%%%%%%%%%%%%%%
%%%%%%%%%%%%%%%%%%%%%%%%%%%%%%%%%%%%%%%%%%%%%%%%%%%%%%%%%%%%%%%%%%%%%%%%%%%%%%%%
\section{Information}

%%%%%%%%%%%%%%%%%%%%%%%%%%%%%%%%%%%%%%%%%%%%%%%%%%%%%%%%%%%%%%%%%%%%%%%%%%%%%%%%
\subsection{Copyright}

Copyright \copyright{} 2017--2018 Niklas Beisert

This work may be distributed and/or modified under the
conditions of the \LaTeX{} Project Public License, either version 1.3
of this license or (at your option) any later version.
The latest version of this license is in
  \url{http://www.latex-project.org/lppl.txt}
and version 1.3 or later is part of all distributions of \LaTeX{}
version 2005/12/01 or later.

This work has the LPPL maintenance status `maintained'.

The Current Maintainer of this work is Niklas Beisert.

This work consists of the files |README.txt|, |childdoc.ins| and |childdoc.dtx|
as well as the derived files |childdoc.def|, |cdocsamp.tex|
with |cdocsch1.tex|, |cdocsch2.tex|, |cdocspt3.tex|, |cdocspt4.tex|,
|cdocsdrf.tex|, |cdocsfn1.tex|, |cdocsfn2.tex|
as well as |childdoc.pdf|.

%%%%%%%%%%%%%%%%%%%%%%%%%%%%%%%%%%%%%%%%%%%%%%%%%%%%%%%%%%%%%%%%%%%%%%%%%%%%%%%%
\subsection{Files and Installation}

The package consists of the files:
%
\begin{center}
\begin{tabular}{ll}
    |README.txt|   & readme file \\
    |childdoc.ins| & installation file \\
    |childdoc.dtx| & source file \\
    |childdoc.def| & definition file \\
    |cdocsamp.tex| & sample main file \\
    |cdocsch1.tex| & sample include file \\
    |cdocsch2.tex| & sample include file \\
    |cdocspt3.tex| & sample part file \\
    |cdocspt4.tex| & sample part file \\
    |cdocsdrf.tex| & sample redirection file \\
    |cdocsfn1.tex| & sample redirection file \\
    |cdocsfn2.tex| & sample redirection file \\
    |childdoc.pdf| & manual
\end{tabular}
\end{center}
%
The distribution consists of the files
|README.txt|, |childdoc.ins| and |childdoc.dtx|.
%
\begin{itemize}
\item
Run (pdf)\LaTeX{} on |childdoc.dtx|
to compile the manual |childdoc.pdf| (this file).
\item
Run \LaTeX{} on |childdoc.ins| to create the definitions file |childdoc.def|
and the sample |cdocsamp.tex| with include files
|cdocsch1.tex|, |cdocsch2.tex|, |cdocspt3.tex|, |cdocspt4.tex|,
|cdocsdrf.tex|, |cdocsfn1.tex|, |cdocsfn2.tex|.
Then copy the file |childdoc.def| to an appropriate directory of your \LaTeX{}
distribution, e.g.\ \textit{texmf-root}|/tex/latex/childdoc|.
\end{itemize}

%%%%%%%%%%%%%%%%%%%%%%%%%%%%%%%%%%%%%%%%%%%%%%%%%%%%%%%%%%%%%%%%%%%%%%%%%%%%%%%%
\subsection{Related CTAN Packages}

There are several other packages which offer a similar functionality:
%
\begin{itemize}
\item
The packages
\href{http://ctan.org/pkg/docmute}{\textsf{docmute}},
\href{http://ctan.org/pkg/includex}{\textsf{includex}} and
\href{http://ctan.org/pkg/standalone}{\textsf{standalone}}
provide commands to include only the document body of
a child file thus allowing both files to be compiled individually.
\item
The packages \href{http://ctan.org/pkg/subdocs}{\textsf{subdocs}}
and \href{http://ctan.org/pkg/subfiles}{\textsf{subfiles}}
provide structures in which the main and child documents can be
encapsulated and allowing them to be compiled individually.
The inclusion mechanism is different from the conventional |\include|.
\item
The package \href{http://ctan.org/pkg/combine}{\textsf{combine}}
is an elaborate solution to combine several documents into one.
\end{itemize}
%
See also the CTAN topic \href{http://ctan.org/topic/subdocs}{\textsf{subdocs}}
for further related packages.
The present package differs from the above solutions in that
a document structure constructed with the conventional |\include| mechanism
just needs two extra commands at the top of every file
such that all constituent files can be compiled individually.

%%%%%%%%%%%%%%%%%%%%%%%%%%%%%%%%%%%%%%%%%%%%%%%%%%%%%%%%%%%%%%%%%%%%%%%%%%%%%%%%
%\subsection{Feature Suggestions}
%
%The following is a list of features which may be useful for future
%versions of this package:
%%
%\begin{itemize}
%\item
%\ldots
%\end{itemize}

%%%%%%%%%%%%%%%%%%%%%%%%%%%%%%%%%%%%%%%%%%%%%%%%%%%%%%%%%%%%%%%%%%%%%%%%%%%%%%%%
\subsection{Revision History}

%%%%%%%%%%%%%%%%%%%%%%%%%%%%%%%%%%%%%%%%
\paragraph{v2.0:} 2018/12/30

\begin{itemize}
\item
immediate forward processing
\item
added |\childdocby| mechanism
\item
manual restructured
\end{itemize}

%%%%%%%%%%%%%%%%%%%%%%%%%%%%%%%%%%%%%%%%
\paragraph{v1.6:} 2018/01/17

\begin{itemize}
\item
application for development of include files
\item
corrections to manual
\end{itemize}

%%%%%%%%%%%%%%%%%%%%%%%%%%%%%%%%%%%%%%%%
\paragraph{v1.5:} 2017/05/21

\begin{itemize}
\item
more complete structuring introduced
\item
|\childdocof| introduced
\item
|\childdoc| renamed to |\childdocmain|
\item
|\childredirect| renamed to |\childdocforward| and |\childdocforwardprefix|
and functionality expanded
\end{itemize}

%%%%%%%%%%%%%%%%%%%%%%%%%%%%%%%%%%%%%%%%
\paragraph{v1.0:} 2017/04/27

\begin{itemize}
\item
manual and install package
\item
first version published on CTAN
\end{itemize}

%%%%%%%%%%%%%%%%%%%%%%%%%%%%%%%%%%%%%%%%
\paragraph{v0.6:} 2017/04/26

\begin{itemize}
\item
redirection mechanism added
\end{itemize}

%%%%%%%%%%%%%%%%%%%%%%%%%%%%%%%%%%%%%%%%
\paragraph{v0.5:} 2017/04/26

\begin{itemize}
\item
functionality in definition file
\end{itemize}


%%%%%%%%%%%%%%%%%%%%%%%%%%%%%%%%%%%%%%%%%%%%%%%%%%%%%%%%%%%%%%%%%%%%%%%%%%%%%%%%
%%%%%%%%%%%%%%%%%%%%%%%%%%%%%%%%%%%%%%%%%%%%%%%%%%%%%%%%%%%%%%%%%%%%%%%%%%%%%%%%
%%%%%%%%%%%%%%%%%%%%%%%%%%%%%%%%%%%%%%%%%%%%%%%%%%%%%%%%%%%%%%%%%%%%%%%%%%%%%%%%
\appendix

\settowidth\MacroIndent{\rmfamily\scriptsize 000\ }

 \DocInput{childdoc.dtx}

\end{document}
%</driver>
% \fi
%
% %%%%%%%%%%%%%%%%%%%%%%%%%%%%%%%%%%%%%%%%%%%%%%%%%%%%%%%%%%%%%%%%%%%%%%%%%%%%%%
% %%%%%%%%%%%%%%%%%%%%%%%%%%%%%%%%%%%%%%%%%%%%%%%%%%%%%%%%%%%%%%%%%%%%%%%%%%%%%%
% \section{Sample}
%\iffalse
%<*samplemain>
%\fi
%
% The following presents a sample document
% with two chapters, two parts, a title page,
% a compile flag as well as three forwarding files to set the flag.
% It consists of eight |.tex| files:
% \begin{center}
% \begin{tabular}{ll}
% |cdocsamp.tex|&main file\\
% |cdocsch1.tex|&include file for chapter 1\\
% |cdocsch2.tex|&include file for chapter 2\\
% |cdocspt3.tex|&include file for part 3\\
% |cdocspt4.tex|&include file for part 4\\
% |cdocsdrf.tex|&forwarding file for main file in draft mode\\
% |cdocsfi1.tex|&forwarding file for final version of chapter 1\\
% |cdocsfi2.tex|&forwarding file for final version of chapter 2\\
% \end{tabular}
% \end{center}
% Each of the eight files can be compiled directly by the \LaTeX{} compiler.
%
% %%%%%%%%%%%%%%%%%%%%%%%%%%%%%%%%%%%%%%
% \paragraph{Main File.}
%
% The main file is called |cdocsamp.tex|.
%
% Load the \textsf{childdoc} definitions and
% declare the filename for the main document:
%    \begin{macrocode}
\input{childdoc.def}
\childdocmain{}
%    \end{macrocode}

% Optional override for |\version| flag:
%    \begin{macrocode}
%%\ifchilddoc\else\providecommand{\version}{draft}\fi
%    \end{macrocode}

% Define the default values for the |\version| flag
% (|final| for the main file and |draft| for childs):
%    \begin{macrocode}
\ifchilddoc
\providecommand{\version}{draft}
\else
\providecommand{\version}{final}
\fi
%    \end{macrocode}

% Load the standard document class:
%    \begin{macrocode}
\documentclass[12pt]{article}
%    \end{macrocode}

% Start the document body:
%    \begin{macrocode}
\begin{document}
%    \end{macrocode}

% Declare a title page.
% Print title, part of document being processed and version flag:
%    \begin{macrocode}
\addtocounter{page}{-1}
\begin{center}
{\LARGE\bfseries{}childdoc example\par}
\vspace{1cm}
\ifchilddoc
\ifchilddocmanual part\else chapter\fi:
`\childdocname' of `\childdocjob'\par
\else
main document: `\childdocjob'\par
\fi
version: \version\par
\end{center}
\newpage
%    \end{macrocode}

% Manually include selected file,
% otherwise process as usual:
%    \begin{macrocode}
\ifchilddocmanual
\section*{part `\childdocname'}
\input{\childdocname}
\else
%    \end{macrocode}

% Include the two chapters:
%    \begin{macrocode}
\include{cdocsch1}
\include{cdocsch2}
%    \end{macrocode}

% Include the two parts unless only chapters should be displayed:
%    \begin{macrocode}
\ifchilddoc\else
\section{part three}
\input{cdocspt3}
\section{part four}
\input{cdocspt4}
\fi
%    \end{macrocode}

% Process as usual until here:
%    \begin{macrocode}
\fi
%    \end{macrocode}

% End of document body:
%    \begin{macrocode}
\end{document}
%    \end{macrocode}
%\iffalse
%</samplemain>
%\fi
%
% %%%%%%%%%%%%%%%%%%%%%%%%%%%%%%%%%%%%%%
% \paragraph{Chapter Include Files.}
%
% The include files are called |cdocsch1.tex| and |cdocsch2.tex|.
%
%\iffalse
%<*samplechap1|samplechap2>
%\fi

% Optional override for |\version| flag:
%    \begin{macrocode}
%%\providecommand{\version}{final}
%    \end{macrocode}

% Include the main document:
%    \begin{macrocode}
\input{childdoc.def}
\childdocof{cdocsamp}
%    \end{macrocode}

%\iffalse
%</samplechap1|samplechap2>
%\fi
%
%\iffalse
%<*samplechap1>
%\fi
% Some text for chapter 1:
%    \begin{macrocode}
\section{one}
some text in chapter one
%    \end{macrocode}

%\iffalse
%</samplechap1>
%\fi
% Some text for chapter 2:
%\iffalse
%<*samplechap2>
%\fi
%    \begin{macrocode}
\section{two}
more text in chapter two
%    \end{macrocode}

%\iffalse
%</samplechap2>
%\fi
%
% %%%%%%%%%%%%%%%%%%%%%%%%%%%%%%%%%%%%%%
% \paragraph{Part Include Files.}
%
% The include files are called |cdocspt3.tex| and |cdocspt4.tex|.
%
%\iffalse
%<*samplepart3|samplepart4>
%\fi

% Optional override for |\version| flag:
%    \begin{macrocode}
%%\providecommand{\version}{final}
%    \end{macrocode}

% Include the main document:
%    \begin{macrocode}
\input{childdoc.def}
\childdocby{cdocsamp}
%    \end{macrocode}

%\iffalse
%</samplepart3|samplepart4>
%\fi
%
%\iffalse
%<*samplepart3>
%\fi
% Some text for part 3:
%    \begin{macrocode}
some text in part three
%    \end{macrocode}

%\iffalse
%</samplepart3>
%\fi
% Some text for part 4:
%\iffalse
%<*samplepart4>
%\fi
%    \begin{macrocode}
more text in part four
%    \end{macrocode}

%\iffalse
%</samplepart4>
%\fi
%
% %%%%%%%%%%%%%%%%%%%%%%%%%%%%%%%%%%%%%%
% \paragraph{Forwarding for a Complete Draft.}
%
% The following forwarding file |cdocsdrf.tex|
% compiles the main document in draft mode:
%\iffalse
%<*sampledraft>
%\fi
%    \begin{macrocode}
\def\version{draft}
\input{childdoc.def}
\childdocforward{cdocsamp}
%    \end{macrocode}

%\iffalse
%</sampledraft>
%\fi
%
% %%%%%%%%%%%%%%%%%%%%%%%%%%%%%%%%%%%%%%
% \paragraph{Forwarding for Final Version of the Chapters.}
%
% The following forwarding files |cdocsfn1.tex| and |cdocsfn2.tex|
% (with identical content)
% compile the final versions of the child documents
% |cdocsch1.tex| and |cdocsch2.tex|, respectively:
%\iffalse
%<*samplefinal>
%\fi
%    \begin{macrocode}
\def\version{final}
\input{childdoc.def}
\childdocforwardprefix[cdocsamp]{cdocsfn}{cdocsch}
%    \end{macrocode}

%\iffalse
%</samplefinal>
%\fi
%
% %%%%%%%%%%%%%%%%%%%%%%%%%%%%%%%%%%%%%%
% \paragraph{Command Line Processing.}
%
% The following three command lines generate the output files
% |cdocscld|, |cdocscl1| and |cdocscl2|
% which should be identical to
% |cdocsdrf|, |cdocsch1| and |cdocsfn2|, respectively:
% \begin{center}
% \begin{tabular}{l}
% |latex -jobname cdocscld \|\\
% |  "\def\version{draft}\input{childdoc.def}\childdocforward{cdocsamp}"|\\
% |latex -jobname cdocscl1 \|\\
% |  "\input{childdoc.def}\childdocforward[cdocsamp]{cdocsch1}"|\\
% |latex -jobname cdocscl2 \|\\
% |  "\def\version{final}\input{childdoc.def}\childdocforward{cdocsch2}"|
% \end{tabular}
% \end{center}
% Note that the trailing backslash on each first line
% merely continues the input to the second line
% (for convenient cut ant paste).
% Furthermore, the command |latex| can be replaced by any
% of its alternative versions such as |pdflatex|.
%
% %%%%%%%%%%%%%%%%%%%%%%%%%%%%%%%%%%%%%%%%%%%%%%%%%%%%%%%%%%%%%%%%%%%%%%%%%%%%%%
% %%%%%%%%%%%%%%%%%%%%%%%%%%%%%%%%%%%%%%%%%%%%%%%%%%%%%%%%%%%%%%%%%%%%%%%%%%%%%%
% \section{Implementation}
%\iffalse
%<*package>
%\fi
%
% This section describes the definitions file |childdoc.def|.

% The definitions cannot be loaded using |\usepackage| or |\RequirePackage|
% which has a mechanism to prevent loading a style file more than once.
% When loading the definitions by means of |\input|
% multiple instances have to be prevented manually:
%\iffalse
%This code needs to be before the `\ProvidesFile' directive
%which is defined at the beginning of this file.
%Therefore it is also placed there and commented out here.
%</package>
%<*discard>
%\fi
%    \begin{macrocode}
\ifdefined\childdocmain\endinput\fi
%    \end{macrocode}
%\iffalse
%</discard>
%<*package>
%\fi
%
% \macro{\ifchilddoc}
% \macro{\ifchilddocmanual}
% The conditional |\ifchilddoc| tells whether a
% child (true) or main (false) document is being compiled.
% The conditional |\ifchilddocmanual| tells whether
% the |\includeonly| mechanism is used (false) or
% the selection of child files must be performed manually (true).
% The definitions initialise to false:
%    \begin{macrocode}
\newif\ifchilddoc
\newif\ifchilddocmanual
%    \end{macrocode}

% \macro{\childdocname}
% \macro{\childdocjob}
% The macro |\childdocname| stores the name of the main document
% to be compiled. The macro |\childdocjob| stores the name of
% the document on which the \LaTeX{} compiler was originally invoked.
% The content of |\jobname| cannot be compared
% to filenames specified in the source due to different catcodes.
% The following code rescans |\jobname|, stores the result
% in |\childdocname| and saves a copy in |\childdocjob|:
%    \begin{macrocode}
\edef\childdocname{\scantokens\expandafter{\jobname\noexpand}}
\let\childdocjob\childdocname
%    \end{macrocode}

% \macro{\childdocdisable}
% The macro |\childdocdisable| prevents the main file
% from being processed more than once.
% At this stage, the main document command |\childdocmain|
% is assumed to be called once again where it should do nothing.
% Any subsequent call to it should prevent
% a secondary processing of the main document
% It overwrites the forwarding commands
% |\childdocof| and |\childdocforward|
% with empty macros to prevent further inclusions of the main document:
%    \begin{macrocode}
\newcommand{\childdocdisable}
{
  \renewcommand{\childdocmain}[1]{\renewcommand{\childdocmain}[1]{\endinput}}
  \renewcommand{\childdocof}[1]{}
  \renewcommand{\childdocby}[2][]{}
  \renewcommand{\childdocforward}[2][]{}
  \renewcommand{\childdocdisable}{}
}
%    \end{macrocode}

% \macro{\childdocmain}
% The macro |\childdocmain| is to be called at the top of the main file
% with nothing or the main filename (without extension) as argument.
% First, it breaks loops.
% If the argument is not empty and does not match |\childdocname|
% (which is set by the first inclusion of |childdoc.def|),
% |\ifchilddoc| is set to true, |\includeonly| is applied to the child file
% and |\jobname| is set to the main file
% (for proper handling of |.aux| files):
%    \begin{macrocode}
\newcommand{\childdocmain}[1]
{
  \childdocdisable\childdocmain{}
  \if?#1?\else
    \begingroup
      \def\childdoctmp{#1}
      \ifx\childdoctmp\childdocname
        \def\childdoctmp{}
      \else
        \def\childdoctmp
        {
          \childdoctrue
          \includeonly{\childdocname}
          \def\childdocjob{#1}
          \def\jobname{#1}
        }
      \fi
      \expandafter
    \endgroup
    \childdoctmp
  \fi
}
%    \end{macrocode}

% \macro{\childdocof}
% The command |\childdocof| redirects
% compilation to the main file |#1|.
%    \begin{macrocode}
\newcommand{\childdocof}[1]
{
  \childdocdisable
  \childdoctrue
  \includeonly{\childdocname}
  \def\jobname{#1}
  \def\childdocjob{#1}
  \input{#1}
}
%    \end{macrocode}

% \macro{\childdocby}
% The command |\childdocby| ....
%    \begin{macrocode}
\newcommand{\childdocby}[2][]
{
  \childdocdisable
  \childdoctrue
  \childdocmanualtrue
  \if?#1?\else
    \def\jobname{#2}
  \fi
  \def\childdocjob{#2}
  \input{#2}
  \endinput
}
%    \end{macrocode}

% \macro{\childdocforward}
% The command |\childdocforward| redirects
% compilation to the main file or
% (if the optional argument is given) a child file.
% Parameters are set as if the main file
% or a child file starting with |\childdocof| was compiled.
% Then compilation is handed over to the main file:
%    \begin{macrocode}
\newcommand{\childdocforward}[2][]
{
  \begingroup
    \if?#1?
      \def\childdoctmp
      {
        \def\childdocname{#2}
        \def\childdocjob{#2}
        \def\jobname{#2}
        \input{#2}
        \endinput
      }
    \else
      \def\childdoctmp
      {
        \childdocdisable
        \def\childdocname{#2}
        \childdoctrue
        \includeonly{#2}
        \def\childdocjob{#1}
        \def\jobname{#1}
        \input{#1}
        \endinput
      }
    \fi
    \expandafter
  \endgroup
  \childdoctmp
}
%    \end{macrocode}

% \macro{\childdocforwardprefix}
% The command |\childdocforwardprefix| redirects
% compilation to the main or a child file by means of a pattern.
% The prefix |#1| in the current filename is replaced by |#2|
% and the suffix of the current filename is kept
% (it is assumed that the filename does not contain the substring `|~~~|'
% which is used as a delimiter).
% Compilation is handed over to the new file by |\childdocforward|:
%    \begin{macrocode}
\newcommand{\childdocforwardprefix}[3][]
{
  \begingroup
    \def\childdocextract #2##1~~~{\def\childdoctmp{\childdocforward[#1]{#3##1}}}
    \expandafter\childdocextract\childdocname~~~
    \expandafter
  \endgroup
  \childdoctmp
}
%    \end{macrocode}

% \macro{\childdoc}
% The deprecated macro |\childdoc| is a legacy version of |\childdocmain|:
%    \begin{macrocode}
\newcommand{\childdoc}{\childdocmain}
%    \end{macrocode}

% \macro{\childdocredirect}
% The deprecated macro |\childdocredirect| is a legacy version
% of |\childdocforward| and |\childdocforwardprefix|:
%    \begin{macrocode}
\newcommand{\childdocredirect}[2][]
{
  \begingroup
    \if?#1?
      \def\childdoctmp{\childdocforward{#2}}
    \else
      \def\childdoctmp{\childdocforwardprefix{#1}{#2}}
    \fi
    \expandafter
  \endgroup
  \childdoctmp
}
%    \end{macrocode}

%\iffalse
%</package>
%\fi
%
\endinput
|\\
|\childdocforwardprefix{final}{child}|
\end{tabular}
\end{center}
%

Note that when several versions of a main file and/or of each child file
are to be generated, it may be convenient to set up a |Makefile| or
shell script to automatise the process.

%%%%%%%%%%%%%%%%%%%%%%%%%%%%%%%%%%%%%%%%%%%%%%%%%%%%%%%%%%%%%%%%%%%%%%%%%%%%%%%%
\subsection{Command Line Processing}
\label{sec:commandline}

The effect of redirection files can also be achieved by invoking
the \LaTeX{} compiler with a more elaborate command line.
Most conveniently this should be done as part
of a shell script or a |Makefile|.

When using \textsf{childdoc} in the main file, the following
command lines effectively perform a redirection
(note that depending on the shell being used,
backslashes may have to be doubled: `|\|' $\to$ `|\\|'):
%
\begin{center}
|... -jobname "|\textit{target}|" |\\|"|[\textit{flags}]%
|% \iffalse
%
% childdoc.dtx Copyright (C) 2017-2018 Niklas Beisert
%
% This work may be distributed and/or modified under the
% conditions of the LaTeX Project Public License, either version 1.3
% of this license or (at your option) any later version.
% The latest version of this license is in
%   http://www.latex-project.org/lppl.txt
% and version 1.3 or later is part of all distributions of LaTeX
% version 2005/12/01 or later.
%
% This work has the LPPL maintenance status `maintained'.
%
% The Current Maintainer of this work is Niklas Beisert.
%
% This work consists of the files childdoc.dtx and childdoc.ins
% and the derived files childdoc.def and cdocsamp.tex with
% cdocsch1.tex, cdocsch2.tex, cdocsdrf.tex, cdocsfn1.tex, cdocsfn2.tex.
%
%<package>\ifdefined\childdocmain\endinput\fi
%<package>\ProvidesFile{childdoc.def}[2018/12/30 v2.0 child document driver]
%<samplemain>\ProvidesFile{cdocsamp.tex}[2018/12/30 v2.0 sample for childdoc]
%<*driver>
%\ProvidesFile{childdoc.drv}[2018/12/30 v2.0 childdoc reference manual file]
\PassOptionsToClass{10pt,a4paper}{article}
\documentclass{ltxdoc}

\usepackage[margin=35mm]{geometry}
\usepackage{hyperref}
\usepackage{hyperxmp}
\usepackage[usenames]{color}

\hypersetup{colorlinks=true}
\hypersetup{pdfstartview=FitH}
\hypersetup{pdfpagemode=UseNone}
\hypersetup{pdfsource={}}
\hypersetup{pdflang={en-UK}}
\hypersetup{pdfcopyright={Copyright 2017-2018 Niklas Beisert.
  This work may be distributed and/or modified under the
  conditions of the LaTeX Project Public License, either version 1.3
  of this license or (at your option) any later version.}}
\hypersetup{pdflicenseurl={http://www.latex-project.org/lppl.txt}}
\hypersetup{pdfcontactaddress={ETH Zurich, ITP, HIT K,
  Wolfgang-Pauli-Strasse 27}}
\hypersetup{pdfcontactpostcode={8093}}
\hypersetup{pdfcontactcity={Zurich}}
\hypersetup{pdfcontactcountry={Switzerland}}
\hypersetup{pdfcontactemail={nbeisert@itp.phys.ethz.ch}}
\hypersetup{pdfcontacturl={http://people.phys.ethz.ch/\xmptilde nbeisert/}}

\newcommand{\secref}[1]{\hyperref[#1]{section \ref*{#1}}}

\parskip1ex
\parindent0pt
\let\olditemize\itemize
\def\itemize{\olditemize\parskip0pt}

\begin{document}

\title{The \textsf{childdoc} Package}
\hypersetup{pdftitle={The childdoc Package}}
\author{Niklas Beisert\\[2ex]
  Institut f\"ur Theoretische Physik\\
  Eidgen\"ossische Technische Hochschule Z\"urich\\
  Wolfgang-Pauli-Strasse 27, 8093 Z\"urich, Switzerland\\[1ex]
  \href{mailto:nbeisert@itp.phys.ethz.ch}
  {\texttt{nbeisert@itp.phys.ethz.ch}}}
\hypersetup{pdfauthor={Niklas Beisert}}
\hypersetup{pdfsubject={Manual for the LaTeX2e Package childdoc}}
\date{30 December 2018, \textsf{v2.0}}
\maketitle

\begin{abstract}\noindent
\textsf{childdoc} is a \LaTeXe{} package
that enables the direct compilation
of document sections included by |\include|
to individual files.
\end{abstract}

\begingroup
\parskip0ex
\tableofcontents
\endgroup

%%%%%%%%%%%%%%%%%%%%%%%%%%%%%%%%%%%%%%%%%%%%%%%%%%%%%%%%%%%%%%%%%%%%%%%%%%%%%%%%
%%%%%%%%%%%%%%%%%%%%%%%%%%%%%%%%%%%%%%%%%%%%%%%%%%%%%%%%%%%%%%%%%%%%%%%%%%%%%%%%
\section{Introduction}

\LaTeX{} provides a mechanism to structure a large document (such as a book)
into a main file and several child files (containing the chapters)
using the |\include| command.
This mechanism is beneficial for documents
which span hundreds of pages in order to
make the source file(s) more manageable.
Moreover, compilation can be restricted to
selected child files by means of the |\includeonly| command.
The latter feature can be used to reduce the compilation time while editing
(this was significantly more useful in the earlier days of \LaTeX{})
or to generate a smaller document which is easier to navigate.
Another application of |\includeonly| is to generate
documents consisting of selected parts of the complete document.

However, there are a few drawbacks of the plain |\include| mechanism:
\begin{itemize}
\item
The child files cannot be compiled on their own,
they can only be compiled via the main file.
A naive editing environment
(such as a text editor with an option
to have the current file processed by \LaTeX)
may require one to switch to the main file before compiling;
attempting to compile the child file produces errors.
\item
The main file must be modified (each time)
to adjust the |\includeonly| command
to the present needs. This easily leaves the main file in a messy state.
\item
The generated document will always carry the filename
of the main document. This is inconvenient if
several child files are to be compiled and
to be kept for distribution.
\end{itemize}

The present package provides a simple interface
to make child files individually compilable by \LaTeX{}.
Compiling a child file then has the same effect as compiling
the main file with an |\includeonly| command
to select the appropriate child.
Moreover the generated document will carry the name of the child
rather than the main file.
This resolves all three above issues.

This feature is meant to make the editing of books,
thesis documents and lecture notes somewhat more convenient.
However, the package can also be used efficiently for
composing a series of documents (such as exercise sheets)
which are typically distributed individually.
It then assists the author in generating the individual documents
(potentially in different versions)
as well as a document containing the collected series.
Another application is in developing style files
or other kinds of included material
where compilation of the style file could redirect
to a sample or test file.

%%%%%%%%%%%%%%%%%%%%%%%%%%%%%%%%%%%%%%%%%%%%%%%%%%%%%%%%%%%%%%%%%%%%%%%%%%%%%%%%
%%%%%%%%%%%%%%%%%%%%%%%%%%%%%%%%%%%%%%%%%%%%%%%%%%%%%%%%%%%%%%%%%%%%%%%%%%%%%%%%
\section{Usage}

First of all, the package \textsf{childdoc} is \emph{not} a standard
\LaTeXe{} |.sty| style file! Therefore it needs to be invoked in
a non-standard way.

%%%%%%%%%%%%%%%%%%%%%%%%%%%%%%%%%%%%%%%%%%%%%%%%%%%%%%%%%%%%%%%%%%%%%%%%%%%%%%%%
\subsection{Included Files}
\label{sec:include}

%%%%%%%%%%%%%%%%%%%%%%%%%%%%%%%%%%%%%%%%
\DescribeMacro{\childdocmain}
To use the package, add the commands
\begin{center}
\begin{tabular}{l}
|\input{childdoc.def}|\\
|\childdocmain{}|\\
\end{tabular}
\end{center}
at the very top of the main \LaTeX{} file,
in particular \emph{before} the |\documentclass| statement!
The argument of |\childdocmain| should be left empty
(but it must be present).

%%%%%%%%%%%%%%%%%%%%%%%%%%%%%%%%%%%%%%%%
\DescribeMacro{\childdocof}
Furthermore, add the commands
\begin{center}
\begin{tabular}{l}
|\input{childdoc.def}|\\
|\childdocof{|\textit{main}|}|\\
\end{tabular}
\end{center}
at the top of every child file \textit{child}
which is included by |\include{|\textit{child}|}|
from within the main file
(or at least for those files to be compiled individually).
The argument \textit{main} must be the filename of the main file.

There are a couple of
considerations in setting up the main and child documents:

%%%%%%%%%%%%%%%%%%%%%%%%%%%%%%%%%%%%%%%%
\paragraph{Restrictions.}

Please note the following restrictions:
\begin{itemize}
\item
|\childdocmain| must be called with one argument \textit{main}
to ensure compatibility with earlier version of the package.
It must either be empty (|\childdocmain{}|)
or precisely match the filename of the main file in which it is specified.
See \secref{sec:detection} for further information.
\item
The filename \textit{main} must be specified without the |.tex| extension.
\item
The filename \textit{main} is case sensitive
(even in case-insensitive file systems)
due to internal string comparison.
\item
The argument \textit{main} should be fully expanded, it cannot be a macro.
\item
Subdirectories and special characters should be avoided in filenames.
\item
The command |\childdocmain{|\textit{main}|}| must be followed by a whitespace.
It should not be followed immediately by another command
or by a comment mark `|%|'.
This is because the \TeX{} parser reads the token immediately following
the argument of |\childdocmain| and puts it
at the beginning of every child section;
however, a white\-space is ignored.
\end{itemize}

%%%%%%%%%%%%%%%%%%%%%%%%%%%%%%%%%%%%%%%%
\paragraph{Content of Main File.}

It is advisable to place all content in the child files included by |\include|.
Any output contained in the main file will appear in all child documents
unless suppressed manually;
it cannot be suppressed automatically by the |\includeonly| directive
and thus should normally be avoided.
A method to include some content in the main file
by means of conditional processing is described in \secref{sec:conditional}.

%%%%%%%%%%%%%%%%%%%%%%%%%%%%%%%%%%%%%%%%
\paragraph{Page Numbering.}

When only a part of the document is compiled,
the appropriate numbering of pages
(as well as other status parameters)
is determined from the |.aux| files.
The latter contain information from previous passes.
However this information needs to propagate through
all intermediate child documents.
Therefore the page numbering in child documents may well
be inconsistent until the complete document is compiled at least once.

A useful (if unconventional) way to always ensure a consistent
page numbering is to restart the numbering in each child document
and denote the pages by `\textit{child}|.|\textit{page}'
where \textit{child} represents the chapter/section number of the child file.
This can be achieved by the command
|\numberwithin{page}{|\textit{child}|}|
of the \textsf{amsmath} package
where \textit{child} can be |chapter| or |section|
depending on the chosen structuring.
Alternatively, one can modify the macro |\thepage| appropriately
and reset the counter |page| at the start of each child file.

%%%%%%%%%%%%%%%%%%%%%%%%%%%%%%%%%%%%%%%%%%%%%%%%%%%%%%%%%%%%%%%%%%%%%%%%%%%%%%%%
\subsection{Conditional Processing}
\label{sec:conditional}

The package provides a mechanism to compile different versions
of a document. To customise the versions further some conditional processing
can come in handy to distinguish which version is being compiled.
The package provides two macros to describe the compilation context:

%%%%%%%%%%%%%%%%%%%%%%%%%%%%%%%%%%%%%%%%
\DescribeMacro{\ifchilddoc}
The conditional |\ifchilddoc| distinguishes between the compilation of
child documents and the main document:
%
\begin{center}
|\ifchilddoc |\textit{child-code}| |[|\||else |\textit{main-code}]| \||fi|
\end{center}

%%%%%%%%%%%%%%%%%%%%%%%%%%%%%%%%%%%%%%%%
\DescribeMacro{\childdocname}
\DescribeMacro{\childdocjob}
The macro |\childdocname| contains the filename (without extension)
of the main or child file being processed.
Note that |\childdocjob| will always contain the name of the main file.

%%%%%%%%%%%%%%%%%%%%%%%%%%%%%%%%%%%%%%%%
\paragraph{Title Page.}

Conditional processing can be used to include a title or banner page
in the main document when proper precautions are taken.
Importantly, the code in the main file should ensure that the page counter
(as well as other status parameters which are stored in the |.aux| files)
takes the same value after the conditional processing.
Otherwise the page numbers may take divergent values
depending on which part is compiled.

For example, a title page could be declared by:
%
\begin{center}
\begin{tabular}{l}
|\ifchilddoc\||else|\\
|\addtocounter{page}{-1}|\\
\textit{code for title page}\\
|\newpage|\\
|\||fi|
\end{tabular}
\end{center}
%
A banner page for the child documents can be generated by:
%
\begin{center}
\begin{tabular}{l}
|\ifchilddoc|\\
|\addtocounter{page}{-1}|\\
\textit{code for banner page}\\
|\newpage|\\
|\||fi|
\end{tabular}
\end{center}
%
Here one could write a message such as:
\begin{center}
|This is the part \childdocname{} of \childdocjob{}.|
\end{center}

%%%%%%%%%%%%%%%%%%%%%%%%%%%%%%%%%%%%%%%%%%%%%%%%%%%%%%%%%%%%%%%%%%%%%%%%%%%%%%%%
\subsection{Flags}
\label{sec:flags}

The package makes it easy to generate different versions
of the main or child documents.
To this end compilation flags can be defined
and assigned different default values.
They will be particularly useful in conjunction
with the forwarding mechanism described in \secref{sec:forward}.

For example, it may be useful to have a flag |\version|
which can be set to |draft| or |final|.
The document source will contain some conditional code
depending on the value of |\version|.
Suppose further, the flag should default to |final| for the main file
and to |draft| for child files
which is a natural assignment for editing the document.
This is achieved by placing the following code
in the preamble of the main document
(below the |\childdocmain| directive):
%
\begin{center}
\begin{tabular}{l}
|\ifchilddoc|\\
|\providecommand{\version}{draft}|\\
|\||else|\\
|\providecommand{\version}{final}|\\
|\||fi|
\end{tabular}
\end{center}
%
The definition by |\providecommand| makes sure
that previous definitions are not overwritten.
Further statements |\providecommand{\version}{...}|
can thus be added before the above code to override it.

For the main file, one might add a line
(between |\childdocmain| and the above block)
%
\begin{center}
|%\ifchilddoc\||else\providecommand{\version}{draft}\||fi|
\end{center}
%
which can be uncommented to produce a draft version.
Likewise one can add a line to the very top of a child file
(above the |\childdocof{|\textit{main}|}| directive)
%
\begin{center}
|%\providecommand{\version}{final}|
\end{center}
%
which can be uncommented to produce the final version of this child document.

%%%%%%%%%%%%%%%%%%%%%%%%%%%%%%%%%%%%%%%%%%%%%%%%%%%%%%%%%%%%%%%%%%%%%%%%%%%%%%%%
\subsection{Forwarding}
\label{sec:forward}

Different versions of the main or child documents
using compilation flags as described in \secref{sec:flags}
can be (permanently) stored in different files
for convenient compilation, viewing and distribution.
To this end, the package defines a command
to pass on compilation to a different file:

%%%%%%%%%%%%%%%%%%%%%%%%%%%%%%%%%%%%%%%%
\DescribeMacro{\childdocforward}
The command |\childdocforward| redirects processing to
another source file:
%
\begin{center}
\begin{tabular}{l}
|\input{childdoc.def}|\\
|\childdocforward[|\textit{main}|]{|\textit{dest}|}|\\
\end{tabular}
\end{center}
%
The argument \textit{dest} is the destination file
(without extension).
It should be the main file or one of the child files.
Note that further \textsf{childdoc} directives
such as |\childdocof| and |\childdocforward|
in the indicated file will be processed in this form.
The optional argument \textit{main}
passes on directly to the main file \textit{main}
while pretending to compile the child \textit{dest}.
This form behaves as if \textit{dest}
issues |\childdocof{|\textit{main}|}| right away,
and no further \textsf{childdoc} directives will be processed.

%%%%%%%%%%%%%%%%%%%%%%%%%%%%%%%%%%%%%%%%
\DescribeMacro{\...prefix}
In the alternative form |\childdocforwardprefix|,
%
\begin{center}
\begin{tabular}{l}
|\input{childdoc.def}|\\
|\childdocforwardprefix[|\textit{main}|]{|\textit{prefix}|}{|\textit{dest}|}|
\end{tabular}
\end{center}
%
the destination file is determined by a pattern
depending on the current file:
To make this work, the current file must be called
`{\textit{prefix}\hspace{0.2em}\textit{suffix}}'
with \textit{prefix} matching precisely the argument.
Processing is then passed on to the file
`{\textit{dest}\hspace{0.2em}\textit{suffix}}'.
Surely, the same effect is achieved by
directly specifying the
argument `{\textit{dest}\hspace{0.2em}\textit{suffix}}'
in the first form.
However, that requires to set up a different file
for each child. With the alternative form of the command
all these files can have exactly the same content
which simplifies setting them up and maintaining them.

For example, the following file |draft.tex|
with a compilation flag |\version| as described in \secref{sec:flags}
compiles the main document as a draft:
%
\begin{center}
\begin{tabular}{l}
|\def\version{draft}|\\
|\input{childdoc.def}|\\
|\childdocforward{|\textit{main}|}|
\end{tabular}
\end{center}
%
Likewise, the following files |final|\textit{nn}|.tex|
compile the final version of the child document
|child|\textit{nn}|.tex|:
%
\begin{center}
\begin{tabular}{l}
|\def\version{final}|\\
|\input{childdoc.def}|\\
|\childdocforwardprefix{final}{child}|
\end{tabular}
\end{center}
%

Note that when several versions of a main file and/or of each child file
are to be generated, it may be convenient to set up a |Makefile| or
shell script to automatise the process.

%%%%%%%%%%%%%%%%%%%%%%%%%%%%%%%%%%%%%%%%%%%%%%%%%%%%%%%%%%%%%%%%%%%%%%%%%%%%%%%%
\subsection{Command Line Processing}
\label{sec:commandline}

The effect of redirection files can also be achieved by invoking
the \LaTeX{} compiler with a more elaborate command line.
Most conveniently this should be done as part
of a shell script or a |Makefile|.

When using \textsf{childdoc} in the main file, the following
command lines effectively perform a redirection
(note that depending on the shell being used,
backslashes may have to be doubled: `|\|' $\to$ `|\\|'):
%
\begin{center}
|... -jobname "|\textit{target}|" |\\|"|[\textit{flags}]%
|\input{childdoc.def}\childdocforward[|\textit{main}|]{|\textit{dest}|}"|
\end{center}
%
Here \textit{target} is the name of the output file,
\textit{main} is the name of the main file
and \textit{dest} is the name of the main or child file to be processed
(all filenames without extensions).
The optional argument \textit{main} can be omitted
if \textit{main} matches \textit{dest}.
Optionally, compilation \textit{flags} can be defined via |\def| commands.
This command line makes the \TeX{} engine believe
it is compiling the file \textit{target}
whose content is specified as the latter parameter.
The provided code then forwards the processing to
\textit{main} or \textit{dest} as described in \secref{sec:forward}.

%%%%%%%%%%%%%%%%%%%%%%%%%%%%%%%%%%%%%%%%%%%%%%%%%%%%%%%%%%%%%%%%%%%%%%%%%%%%%%%%
\subsection{Include by Input}
\label{sec:input}

Including child documents by |\include| has some restrictions by design.
Most notably, the content of a child document always occupies
its own set of pages; pages cannot be shared between child documents.
Usually, this behaviour makes perfect sense
because each child document contain an essential part of the document.
However, in some situations it may be desirable to compose
a document from a collection of parts
without having mandatory page breaks between then.
For this case, the package
provides a mechanism to include parts
by |\input| which can also be processed individually.
However, by construction this mechanism
requires manual handling of the content to be output.

%%%%%%%%%%%%%%%%%%%%%%%%%%%%%%%%%%%%%%%%
\DescribeMacro{\ifchilddocmanual}
The main file should be prepared as usual, see \secref{sec:include}.
However, the document body must make a distinction
between processing of an individual part and of the main document, e.g.:
%
\begin{center}
\begin{tabular}{l}
|\ifchilddocmanual|\\
|\input{\childdocname}|\\
|\||else|\\
\textit{document body with }|\input{|\textit{part}|}|\\
|\||fi|
\end{tabular}
\end{center}
%
The conditional |\ifchilddocmanual| is true whenever
a part to be included by |\input| is being compiled,
and the name of the part is stored in |\childdocname|.

%%%%%%%%%%%%%%%%%%%%%%%%%%%%%%%%%%%%%%%%
\DescribeMacro{\childdocby}
Each part to be included by |\input| should start with:
%
\begin{center}
\begin{tabular}{l}
|\input{childdoc.def}|\\
|\childdocby{|\textit{main}|}|\\
\end{tabular}
\end{center}
%
The directive |\childdocby| is similar to |\childdocof|
described in \secref{sec:include},
but the subsequent selection of content must be done manually.
To that end, both |\ifchilddoc| and |\ifchilddocmanual|
will be true upon processing of a part,
and the name of the part is stored in |\childdocname|.
Note that |\jobname| will be set to the filename of the current part
so that each part receives an individual |.aux| file
that does not interfere with the |.aux| file(s) of the main document.
This behaviour can be altered by the alternative form
|\childdocby[*]{|\textit{main}|}| (with a non-empty optional argument)
which uses the |.aux| file of the main document
by setting |\jobname| to \textit{main}.

%%%%%%%%%%%%%%%%%%%%%%%%%%%%%%%%%%%%%%%%%%%%%%%%%%%%%%%%%%%%%%%%%%%%%%%%%%%%%%%%
\subsection{Driver Development}
\label{sec:driver}

The \textsf{childdoc} mechanism can also be use for the development
of definition files such as \LaTeX{} styles or classes.
This case differs from the above setup with multiple parts
included by |\include| in that no |\includeonly| should be invoked.
This can be achieved by starting the include file
(before |\ProvidesPackage|) with:
%
\begin{center}
\begin{tabular}{l}
|\input{childdoc.def}|\\
|\childdocforward{|\textit{main}|}|\\
\end{tabular}
\end{center}
%
or alternatively with:
%
\begin{center}
\begin{tabular}{l}
|\input{childdoc.def}|\\
|\childdocby{|\textit{main}|}|\\
\end{tabular}
\end{center}
%
Both forms have slightly different effects as described above.
The main file is prepared as usual, see \secref{sec:include}.

%%%%%%%%%%%%%%%%%%%%%%%%%%%%%%%%%%%%%%%%%%%%%%%%%%%%%%%%%%%%%%%%%%%%%%%%%%%%%%%%
\subsection{Legacy Detection}
\label{sec:detection}

The directive |\childdocmain| in the main file can detect
whether the complete document or merely a child is to be compiled
even without using the directive |\childdocof|.
This method is deprecated because it is less robust
and there is no compelling reason to use it;
it is merely provided for backward compatibility
and it may be removed in future versions.

If the detection mechanism is to be used,
it is mandatory to correctly specify
the filename of the main file as the argument of |\childdocmain|:
%
\begin{center}
\begin{tabular}{l}
|\input{childdoc.def}|\\
|\childdocmain{|\textit{main}|}|\\
\end{tabular}
\end{center}
%
If |\jobname| does not match the argument \textit{main} of |\childdocmain|,
it is assumed that |\jobname| points to the child file to be compiled.
When using |\childdocmain| with the main file specified as argument,
it suffices to start a child file
with just |\input{|\textit{main}|}|
without loading of the package and using |\childdocof|.
If instead all processing is done
with the appropriate \textsf{childdoc} directives,
the argument of \textit{main} of |\childdocmain| can be empty.

An alternative version of the command line processing described
in \secref{sec:commandline} using the detection mechanism reads:
%
\begin{center}
|... -jobname "|\textit{target}|" "|[\textit{flags}]%
[|\def\jobname{|\textit{dest}|}|]|\input{|\textit{main}|}"|
\end{center}

%%%%%%%%%%%%%%%%%%%%%%%%%%%%%%%%%%%%%%%%%%%%%%%%%%%%%%%%%%%%%%%%%%%%%%%%%%%%%%%%
\subsection{Manual Code}
\label{sec:manual}

In case one cannot be certain whether the definitions file |childdoc.def|
is installed on the target \TeX{} distribution
and one prefers not to ship it,
it is conceivable to paste a few relevant commands into the sources.

To that end, drop all statements |\input{childdoc.def}|
and perform the replacements as outlined below.
Instead of |\childdocmain{|\textit{main}|}| add the following code
to the top of the main file:
%
\begin{center}
\begin{tabular}{l}
|\||ifdefined\childdocname\endinput\||fi\newif\ifchilddoc|\\
|\edef\childdocname{\scantokens\expandafter{\jobname\noexpand}}|\\
|\def\childdocmain{|\textit{main}|}\||ifx\childdocmain\childdocname\||else|\\
|\childdoctrue\includeonly{\childdocname}\let\jobname\childdocmain\||fi|\\
\end{tabular}
\end{center}
%
Instead of |\childdocof{|\textit{main}|}| just include the main file
at the top of each child file:
%
\begin{center}
|\input{|\textit{main}|}|
\end{center}
%
A simple redirection |\childdocforward{|\textit{dest}|}| is achieved by:
%
\begin{center}
|\def\jobname{|\textit{dest}|}\input{\jobname}|
\end{center}
%
The redirection with prefix
|\childdocforwardprefix[|\textit{prefix}|]{|\textit{dest}|}|
is accomplished by:
%
\begin{center}
\begin{tabular}{l}
|{\edef\jobname{\scantokens\expandafter{\jobname\noexpand}}|\\
|\def\redirectjob |\textit{prefix}|#1~~~{\gdef\jobname{|\textit{dest}|#1}}|\\
|\expandafter\redirectjob\jobname~~~}\input{\jobname}|
\end{tabular}
\end{center}

In an alternative approach,
child documents can be compiled by a specific command line
without additional code or specific definitions:
%
\begin{center}
|... -jobname "|\textit{target}|" "|[\textit{flags}]%
|\includeonly{|\textit{dest}|}\input{|\textit{main}|}"|
\end{center}
%

%%%%%%%%%%%%%%%%%%%%%%%%%%%%%%%%%%%%%%%%%%%%%%%%%%%%%%%%%%%%%%%%%%%%%%%%%%%%%%%%
%%%%%%%%%%%%%%%%%%%%%%%%%%%%%%%%%%%%%%%%%%%%%%%%%%%%%%%%%%%%%%%%%%%%%%%%%%%%%%%%
\section{Information}

%%%%%%%%%%%%%%%%%%%%%%%%%%%%%%%%%%%%%%%%%%%%%%%%%%%%%%%%%%%%%%%%%%%%%%%%%%%%%%%%
\subsection{Copyright}

Copyright \copyright{} 2017--2018 Niklas Beisert

This work may be distributed and/or modified under the
conditions of the \LaTeX{} Project Public License, either version 1.3
of this license or (at your option) any later version.
The latest version of this license is in
  \url{http://www.latex-project.org/lppl.txt}
and version 1.3 or later is part of all distributions of \LaTeX{}
version 2005/12/01 or later.

This work has the LPPL maintenance status `maintained'.

The Current Maintainer of this work is Niklas Beisert.

This work consists of the files |README.txt|, |childdoc.ins| and |childdoc.dtx|
as well as the derived files |childdoc.def|, |cdocsamp.tex|
with |cdocsch1.tex|, |cdocsch2.tex|, |cdocspt3.tex|, |cdocspt4.tex|,
|cdocsdrf.tex|, |cdocsfn1.tex|, |cdocsfn2.tex|
as well as |childdoc.pdf|.

%%%%%%%%%%%%%%%%%%%%%%%%%%%%%%%%%%%%%%%%%%%%%%%%%%%%%%%%%%%%%%%%%%%%%%%%%%%%%%%%
\subsection{Files and Installation}

The package consists of the files:
%
\begin{center}
\begin{tabular}{ll}
    |README.txt|   & readme file \\
    |childdoc.ins| & installation file \\
    |childdoc.dtx| & source file \\
    |childdoc.def| & definition file \\
    |cdocsamp.tex| & sample main file \\
    |cdocsch1.tex| & sample include file \\
    |cdocsch2.tex| & sample include file \\
    |cdocspt3.tex| & sample part file \\
    |cdocspt4.tex| & sample part file \\
    |cdocsdrf.tex| & sample redirection file \\
    |cdocsfn1.tex| & sample redirection file \\
    |cdocsfn2.tex| & sample redirection file \\
    |childdoc.pdf| & manual
\end{tabular}
\end{center}
%
The distribution consists of the files
|README.txt|, |childdoc.ins| and |childdoc.dtx|.
%
\begin{itemize}
\item
Run (pdf)\LaTeX{} on |childdoc.dtx|
to compile the manual |childdoc.pdf| (this file).
\item
Run \LaTeX{} on |childdoc.ins| to create the definitions file |childdoc.def|
and the sample |cdocsamp.tex| with include files
|cdocsch1.tex|, |cdocsch2.tex|, |cdocspt3.tex|, |cdocspt4.tex|,
|cdocsdrf.tex|, |cdocsfn1.tex|, |cdocsfn2.tex|.
Then copy the file |childdoc.def| to an appropriate directory of your \LaTeX{}
distribution, e.g.\ \textit{texmf-root}|/tex/latex/childdoc|.
\end{itemize}

%%%%%%%%%%%%%%%%%%%%%%%%%%%%%%%%%%%%%%%%%%%%%%%%%%%%%%%%%%%%%%%%%%%%%%%%%%%%%%%%
\subsection{Related CTAN Packages}

There are several other packages which offer a similar functionality:
%
\begin{itemize}
\item
The packages
\href{http://ctan.org/pkg/docmute}{\textsf{docmute}},
\href{http://ctan.org/pkg/includex}{\textsf{includex}} and
\href{http://ctan.org/pkg/standalone}{\textsf{standalone}}
provide commands to include only the document body of
a child file thus allowing both files to be compiled individually.
\item
The packages \href{http://ctan.org/pkg/subdocs}{\textsf{subdocs}}
and \href{http://ctan.org/pkg/subfiles}{\textsf{subfiles}}
provide structures in which the main and child documents can be
encapsulated and allowing them to be compiled individually.
The inclusion mechanism is different from the conventional |\include|.
\item
The package \href{http://ctan.org/pkg/combine}{\textsf{combine}}
is an elaborate solution to combine several documents into one.
\end{itemize}
%
See also the CTAN topic \href{http://ctan.org/topic/subdocs}{\textsf{subdocs}}
for further related packages.
The present package differs from the above solutions in that
a document structure constructed with the conventional |\include| mechanism
just needs two extra commands at the top of every file
such that all constituent files can be compiled individually.

%%%%%%%%%%%%%%%%%%%%%%%%%%%%%%%%%%%%%%%%%%%%%%%%%%%%%%%%%%%%%%%%%%%%%%%%%%%%%%%%
%\subsection{Feature Suggestions}
%
%The following is a list of features which may be useful for future
%versions of this package:
%%
%\begin{itemize}
%\item
%\ldots
%\end{itemize}

%%%%%%%%%%%%%%%%%%%%%%%%%%%%%%%%%%%%%%%%%%%%%%%%%%%%%%%%%%%%%%%%%%%%%%%%%%%%%%%%
\subsection{Revision History}

%%%%%%%%%%%%%%%%%%%%%%%%%%%%%%%%%%%%%%%%
\paragraph{v2.0:} 2018/12/30

\begin{itemize}
\item
immediate forward processing
\item
added |\childdocby| mechanism
\item
manual restructured
\end{itemize}

%%%%%%%%%%%%%%%%%%%%%%%%%%%%%%%%%%%%%%%%
\paragraph{v1.6:} 2018/01/17

\begin{itemize}
\item
application for development of include files
\item
corrections to manual
\end{itemize}

%%%%%%%%%%%%%%%%%%%%%%%%%%%%%%%%%%%%%%%%
\paragraph{v1.5:} 2017/05/21

\begin{itemize}
\item
more complete structuring introduced
\item
|\childdocof| introduced
\item
|\childdoc| renamed to |\childdocmain|
\item
|\childredirect| renamed to |\childdocforward| and |\childdocforwardprefix|
and functionality expanded
\end{itemize}

%%%%%%%%%%%%%%%%%%%%%%%%%%%%%%%%%%%%%%%%
\paragraph{v1.0:} 2017/04/27

\begin{itemize}
\item
manual and install package
\item
first version published on CTAN
\end{itemize}

%%%%%%%%%%%%%%%%%%%%%%%%%%%%%%%%%%%%%%%%
\paragraph{v0.6:} 2017/04/26

\begin{itemize}
\item
redirection mechanism added
\end{itemize}

%%%%%%%%%%%%%%%%%%%%%%%%%%%%%%%%%%%%%%%%
\paragraph{v0.5:} 2017/04/26

\begin{itemize}
\item
functionality in definition file
\end{itemize}


%%%%%%%%%%%%%%%%%%%%%%%%%%%%%%%%%%%%%%%%%%%%%%%%%%%%%%%%%%%%%%%%%%%%%%%%%%%%%%%%
%%%%%%%%%%%%%%%%%%%%%%%%%%%%%%%%%%%%%%%%%%%%%%%%%%%%%%%%%%%%%%%%%%%%%%%%%%%%%%%%
%%%%%%%%%%%%%%%%%%%%%%%%%%%%%%%%%%%%%%%%%%%%%%%%%%%%%%%%%%%%%%%%%%%%%%%%%%%%%%%%
\appendix

\settowidth\MacroIndent{\rmfamily\scriptsize 000\ }

 \DocInput{childdoc.dtx}

\end{document}
%</driver>
% \fi
%
% %%%%%%%%%%%%%%%%%%%%%%%%%%%%%%%%%%%%%%%%%%%%%%%%%%%%%%%%%%%%%%%%%%%%%%%%%%%%%%
% %%%%%%%%%%%%%%%%%%%%%%%%%%%%%%%%%%%%%%%%%%%%%%%%%%%%%%%%%%%%%%%%%%%%%%%%%%%%%%
% \section{Sample}
%\iffalse
%<*samplemain>
%\fi
%
% The following presents a sample document
% with two chapters, two parts, a title page,
% a compile flag as well as three forwarding files to set the flag.
% It consists of eight |.tex| files:
% \begin{center}
% \begin{tabular}{ll}
% |cdocsamp.tex|&main file\\
% |cdocsch1.tex|&include file for chapter 1\\
% |cdocsch2.tex|&include file for chapter 2\\
% |cdocspt3.tex|&include file for part 3\\
% |cdocspt4.tex|&include file for part 4\\
% |cdocsdrf.tex|&forwarding file for main file in draft mode\\
% |cdocsfi1.tex|&forwarding file for final version of chapter 1\\
% |cdocsfi2.tex|&forwarding file for final version of chapter 2\\
% \end{tabular}
% \end{center}
% Each of the eight files can be compiled directly by the \LaTeX{} compiler.
%
% %%%%%%%%%%%%%%%%%%%%%%%%%%%%%%%%%%%%%%
% \paragraph{Main File.}
%
% The main file is called |cdocsamp.tex|.
%
% Load the \textsf{childdoc} definitions and
% declare the filename for the main document:
%    \begin{macrocode}
\input{childdoc.def}
\childdocmain{}
%    \end{macrocode}

% Optional override for |\version| flag:
%    \begin{macrocode}
%%\ifchilddoc\else\providecommand{\version}{draft}\fi
%    \end{macrocode}

% Define the default values for the |\version| flag
% (|final| for the main file and |draft| for childs):
%    \begin{macrocode}
\ifchilddoc
\providecommand{\version}{draft}
\else
\providecommand{\version}{final}
\fi
%    \end{macrocode}

% Load the standard document class:
%    \begin{macrocode}
\documentclass[12pt]{article}
%    \end{macrocode}

% Start the document body:
%    \begin{macrocode}
\begin{document}
%    \end{macrocode}

% Declare a title page.
% Print title, part of document being processed and version flag:
%    \begin{macrocode}
\addtocounter{page}{-1}
\begin{center}
{\LARGE\bfseries{}childdoc example\par}
\vspace{1cm}
\ifchilddoc
\ifchilddocmanual part\else chapter\fi:
`\childdocname' of `\childdocjob'\par
\else
main document: `\childdocjob'\par
\fi
version: \version\par
\end{center}
\newpage
%    \end{macrocode}

% Manually include selected file,
% otherwise process as usual:
%    \begin{macrocode}
\ifchilddocmanual
\section*{part `\childdocname'}
\input{\childdocname}
\else
%    \end{macrocode}

% Include the two chapters:
%    \begin{macrocode}
\include{cdocsch1}
\include{cdocsch2}
%    \end{macrocode}

% Include the two parts unless only chapters should be displayed:
%    \begin{macrocode}
\ifchilddoc\else
\section{part three}
\input{cdocspt3}
\section{part four}
\input{cdocspt4}
\fi
%    \end{macrocode}

% Process as usual until here:
%    \begin{macrocode}
\fi
%    \end{macrocode}

% End of document body:
%    \begin{macrocode}
\end{document}
%    \end{macrocode}
%\iffalse
%</samplemain>
%\fi
%
% %%%%%%%%%%%%%%%%%%%%%%%%%%%%%%%%%%%%%%
% \paragraph{Chapter Include Files.}
%
% The include files are called |cdocsch1.tex| and |cdocsch2.tex|.
%
%\iffalse
%<*samplechap1|samplechap2>
%\fi

% Optional override for |\version| flag:
%    \begin{macrocode}
%%\providecommand{\version}{final}
%    \end{macrocode}

% Include the main document:
%    \begin{macrocode}
\input{childdoc.def}
\childdocof{cdocsamp}
%    \end{macrocode}

%\iffalse
%</samplechap1|samplechap2>
%\fi
%
%\iffalse
%<*samplechap1>
%\fi
% Some text for chapter 1:
%    \begin{macrocode}
\section{one}
some text in chapter one
%    \end{macrocode}

%\iffalse
%</samplechap1>
%\fi
% Some text for chapter 2:
%\iffalse
%<*samplechap2>
%\fi
%    \begin{macrocode}
\section{two}
more text in chapter two
%    \end{macrocode}

%\iffalse
%</samplechap2>
%\fi
%
% %%%%%%%%%%%%%%%%%%%%%%%%%%%%%%%%%%%%%%
% \paragraph{Part Include Files.}
%
% The include files are called |cdocspt3.tex| and |cdocspt4.tex|.
%
%\iffalse
%<*samplepart3|samplepart4>
%\fi

% Optional override for |\version| flag:
%    \begin{macrocode}
%%\providecommand{\version}{final}
%    \end{macrocode}

% Include the main document:
%    \begin{macrocode}
\input{childdoc.def}
\childdocby{cdocsamp}
%    \end{macrocode}

%\iffalse
%</samplepart3|samplepart4>
%\fi
%
%\iffalse
%<*samplepart3>
%\fi
% Some text for part 3:
%    \begin{macrocode}
some text in part three
%    \end{macrocode}

%\iffalse
%</samplepart3>
%\fi
% Some text for part 4:
%\iffalse
%<*samplepart4>
%\fi
%    \begin{macrocode}
more text in part four
%    \end{macrocode}

%\iffalse
%</samplepart4>
%\fi
%
% %%%%%%%%%%%%%%%%%%%%%%%%%%%%%%%%%%%%%%
% \paragraph{Forwarding for a Complete Draft.}
%
% The following forwarding file |cdocsdrf.tex|
% compiles the main document in draft mode:
%\iffalse
%<*sampledraft>
%\fi
%    \begin{macrocode}
\def\version{draft}
\input{childdoc.def}
\childdocforward{cdocsamp}
%    \end{macrocode}

%\iffalse
%</sampledraft>
%\fi
%
% %%%%%%%%%%%%%%%%%%%%%%%%%%%%%%%%%%%%%%
% \paragraph{Forwarding for Final Version of the Chapters.}
%
% The following forwarding files |cdocsfn1.tex| and |cdocsfn2.tex|
% (with identical content)
% compile the final versions of the child documents
% |cdocsch1.tex| and |cdocsch2.tex|, respectively:
%\iffalse
%<*samplefinal>
%\fi
%    \begin{macrocode}
\def\version{final}
\input{childdoc.def}
\childdocforwardprefix[cdocsamp]{cdocsfn}{cdocsch}
%    \end{macrocode}

%\iffalse
%</samplefinal>
%\fi
%
% %%%%%%%%%%%%%%%%%%%%%%%%%%%%%%%%%%%%%%
% \paragraph{Command Line Processing.}
%
% The following three command lines generate the output files
% |cdocscld|, |cdocscl1| and |cdocscl2|
% which should be identical to
% |cdocsdrf|, |cdocsch1| and |cdocsfn2|, respectively:
% \begin{center}
% \begin{tabular}{l}
% |latex -jobname cdocscld \|\\
% |  "\def\version{draft}\input{childdoc.def}\childdocforward{cdocsamp}"|\\
% |latex -jobname cdocscl1 \|\\
% |  "\input{childdoc.def}\childdocforward[cdocsamp]{cdocsch1}"|\\
% |latex -jobname cdocscl2 \|\\
% |  "\def\version{final}\input{childdoc.def}\childdocforward{cdocsch2}"|
% \end{tabular}
% \end{center}
% Note that the trailing backslash on each first line
% merely continues the input to the second line
% (for convenient cut ant paste).
% Furthermore, the command |latex| can be replaced by any
% of its alternative versions such as |pdflatex|.
%
% %%%%%%%%%%%%%%%%%%%%%%%%%%%%%%%%%%%%%%%%%%%%%%%%%%%%%%%%%%%%%%%%%%%%%%%%%%%%%%
% %%%%%%%%%%%%%%%%%%%%%%%%%%%%%%%%%%%%%%%%%%%%%%%%%%%%%%%%%%%%%%%%%%%%%%%%%%%%%%
% \section{Implementation}
%\iffalse
%<*package>
%\fi
%
% This section describes the definitions file |childdoc.def|.

% The definitions cannot be loaded using |\usepackage| or |\RequirePackage|
% which has a mechanism to prevent loading a style file more than once.
% When loading the definitions by means of |\input|
% multiple instances have to be prevented manually:
%\iffalse
%This code needs to be before the `\ProvidesFile' directive
%which is defined at the beginning of this file.
%Therefore it is also placed there and commented out here.
%</package>
%<*discard>
%\fi
%    \begin{macrocode}
\ifdefined\childdocmain\endinput\fi
%    \end{macrocode}
%\iffalse
%</discard>
%<*package>
%\fi
%
% \macro{\ifchilddoc}
% \macro{\ifchilddocmanual}
% The conditional |\ifchilddoc| tells whether a
% child (true) or main (false) document is being compiled.
% The conditional |\ifchilddocmanual| tells whether
% the |\includeonly| mechanism is used (false) or
% the selection of child files must be performed manually (true).
% The definitions initialise to false:
%    \begin{macrocode}
\newif\ifchilddoc
\newif\ifchilddocmanual
%    \end{macrocode}

% \macro{\childdocname}
% \macro{\childdocjob}
% The macro |\childdocname| stores the name of the main document
% to be compiled. The macro |\childdocjob| stores the name of
% the document on which the \LaTeX{} compiler was originally invoked.
% The content of |\jobname| cannot be compared
% to filenames specified in the source due to different catcodes.
% The following code rescans |\jobname|, stores the result
% in |\childdocname| and saves a copy in |\childdocjob|:
%    \begin{macrocode}
\edef\childdocname{\scantokens\expandafter{\jobname\noexpand}}
\let\childdocjob\childdocname
%    \end{macrocode}

% \macro{\childdocdisable}
% The macro |\childdocdisable| prevents the main file
% from being processed more than once.
% At this stage, the main document command |\childdocmain|
% is assumed to be called once again where it should do nothing.
% Any subsequent call to it should prevent
% a secondary processing of the main document
% It overwrites the forwarding commands
% |\childdocof| and |\childdocforward|
% with empty macros to prevent further inclusions of the main document:
%    \begin{macrocode}
\newcommand{\childdocdisable}
{
  \renewcommand{\childdocmain}[1]{\renewcommand{\childdocmain}[1]{\endinput}}
  \renewcommand{\childdocof}[1]{}
  \renewcommand{\childdocby}[2][]{}
  \renewcommand{\childdocforward}[2][]{}
  \renewcommand{\childdocdisable}{}
}
%    \end{macrocode}

% \macro{\childdocmain}
% The macro |\childdocmain| is to be called at the top of the main file
% with nothing or the main filename (without extension) as argument.
% First, it breaks loops.
% If the argument is not empty and does not match |\childdocname|
% (which is set by the first inclusion of |childdoc.def|),
% |\ifchilddoc| is set to true, |\includeonly| is applied to the child file
% and |\jobname| is set to the main file
% (for proper handling of |.aux| files):
%    \begin{macrocode}
\newcommand{\childdocmain}[1]
{
  \childdocdisable\childdocmain{}
  \if?#1?\else
    \begingroup
      \def\childdoctmp{#1}
      \ifx\childdoctmp\childdocname
        \def\childdoctmp{}
      \else
        \def\childdoctmp
        {
          \childdoctrue
          \includeonly{\childdocname}
          \def\childdocjob{#1}
          \def\jobname{#1}
        }
      \fi
      \expandafter
    \endgroup
    \childdoctmp
  \fi
}
%    \end{macrocode}

% \macro{\childdocof}
% The command |\childdocof| redirects
% compilation to the main file |#1|.
%    \begin{macrocode}
\newcommand{\childdocof}[1]
{
  \childdocdisable
  \childdoctrue
  \includeonly{\childdocname}
  \def\jobname{#1}
  \def\childdocjob{#1}
  \input{#1}
}
%    \end{macrocode}

% \macro{\childdocby}
% The command |\childdocby| ....
%    \begin{macrocode}
\newcommand{\childdocby}[2][]
{
  \childdocdisable
  \childdoctrue
  \childdocmanualtrue
  \if?#1?\else
    \def\jobname{#2}
  \fi
  \def\childdocjob{#2}
  \input{#2}
  \endinput
}
%    \end{macrocode}

% \macro{\childdocforward}
% The command |\childdocforward| redirects
% compilation to the main file or
% (if the optional argument is given) a child file.
% Parameters are set as if the main file
% or a child file starting with |\childdocof| was compiled.
% Then compilation is handed over to the main file:
%    \begin{macrocode}
\newcommand{\childdocforward}[2][]
{
  \begingroup
    \if?#1?
      \def\childdoctmp
      {
        \def\childdocname{#2}
        \def\childdocjob{#2}
        \def\jobname{#2}
        \input{#2}
        \endinput
      }
    \else
      \def\childdoctmp
      {
        \childdocdisable
        \def\childdocname{#2}
        \childdoctrue
        \includeonly{#2}
        \def\childdocjob{#1}
        \def\jobname{#1}
        \input{#1}
        \endinput
      }
    \fi
    \expandafter
  \endgroup
  \childdoctmp
}
%    \end{macrocode}

% \macro{\childdocforwardprefix}
% The command |\childdocforwardprefix| redirects
% compilation to the main or a child file by means of a pattern.
% The prefix |#1| in the current filename is replaced by |#2|
% and the suffix of the current filename is kept
% (it is assumed that the filename does not contain the substring `|~~~|'
% which is used as a delimiter).
% Compilation is handed over to the new file by |\childdocforward|:
%    \begin{macrocode}
\newcommand{\childdocforwardprefix}[3][]
{
  \begingroup
    \def\childdocextract #2##1~~~{\def\childdoctmp{\childdocforward[#1]{#3##1}}}
    \expandafter\childdocextract\childdocname~~~
    \expandafter
  \endgroup
  \childdoctmp
}
%    \end{macrocode}

% \macro{\childdoc}
% The deprecated macro |\childdoc| is a legacy version of |\childdocmain|:
%    \begin{macrocode}
\newcommand{\childdoc}{\childdocmain}
%    \end{macrocode}

% \macro{\childdocredirect}
% The deprecated macro |\childdocredirect| is a legacy version
% of |\childdocforward| and |\childdocforwardprefix|:
%    \begin{macrocode}
\newcommand{\childdocredirect}[2][]
{
  \begingroup
    \if?#1?
      \def\childdoctmp{\childdocforward{#2}}
    \else
      \def\childdoctmp{\childdocforwardprefix{#1}{#2}}
    \fi
    \expandafter
  \endgroup
  \childdoctmp
}
%    \end{macrocode}

%\iffalse
%</package>
%\fi
%
\endinput
\childdocforward[|\textit{main}|]{|\textit{dest}|}"|
\end{center}
%
Here \textit{target} is the name of the output file,
\textit{main} is the name of the main file
and \textit{dest} is the name of the main or child file to be processed
(all filenames without extensions).
The optional argument \textit{main} can be omitted
if \textit{main} matches \textit{dest}.
Optionally, compilation \textit{flags} can be defined via |\def| commands.
This command line makes the \TeX{} engine believe
it is compiling the file \textit{target}
whose content is specified as the latter parameter.
The provided code then forwards the processing to
\textit{main} or \textit{dest} as described in \secref{sec:forward}.

%%%%%%%%%%%%%%%%%%%%%%%%%%%%%%%%%%%%%%%%%%%%%%%%%%%%%%%%%%%%%%%%%%%%%%%%%%%%%%%%
\subsection{Include by Input}
\label{sec:input}

Including child documents by |\include| has some restrictions by design.
Most notably, the content of a child document always occupies
its own set of pages; pages cannot be shared between child documents.
Usually, this behaviour makes perfect sense
because each child document contain an essential part of the document.
However, in some situations it may be desirable to compose
a document from a collection of parts
without having mandatory page breaks between then.
For this case, the package
provides a mechanism to include parts
by |\input| which can also be processed individually.
However, by construction this mechanism
requires manual handling of the content to be output.

%%%%%%%%%%%%%%%%%%%%%%%%%%%%%%%%%%%%%%%%
\DescribeMacro{\ifchilddocmanual}
The main file should be prepared as usual, see \secref{sec:include}.
However, the document body must make a distinction
between processing of an individual part and of the main document, e.g.:
%
\begin{center}
\begin{tabular}{l}
|\ifchilddocmanual|\\
|\input{\childdocname}|\\
|\||else|\\
\textit{document body with }|\input{|\textit{part}|}|\\
|\||fi|
\end{tabular}
\end{center}
%
The conditional |\ifchilddocmanual| is true whenever
a part to be included by |\input| is being compiled,
and the name of the part is stored in |\childdocname|.

%%%%%%%%%%%%%%%%%%%%%%%%%%%%%%%%%%%%%%%%
\DescribeMacro{\childdocby}
Each part to be included by |\input| should start with:
%
\begin{center}
\begin{tabular}{l}
|% \iffalse
%
% childdoc.dtx Copyright (C) 2017-2018 Niklas Beisert
%
% This work may be distributed and/or modified under the
% conditions of the LaTeX Project Public License, either version 1.3
% of this license or (at your option) any later version.
% The latest version of this license is in
%   http://www.latex-project.org/lppl.txt
% and version 1.3 or later is part of all distributions of LaTeX
% version 2005/12/01 or later.
%
% This work has the LPPL maintenance status `maintained'.
%
% The Current Maintainer of this work is Niklas Beisert.
%
% This work consists of the files childdoc.dtx and childdoc.ins
% and the derived files childdoc.def and cdocsamp.tex with
% cdocsch1.tex, cdocsch2.tex, cdocsdrf.tex, cdocsfn1.tex, cdocsfn2.tex.
%
%<package>\ifdefined\childdocmain\endinput\fi
%<package>\ProvidesFile{childdoc.def}[2018/12/30 v2.0 child document driver]
%<samplemain>\ProvidesFile{cdocsamp.tex}[2018/12/30 v2.0 sample for childdoc]
%<*driver>
%\ProvidesFile{childdoc.drv}[2018/12/30 v2.0 childdoc reference manual file]
\PassOptionsToClass{10pt,a4paper}{article}
\documentclass{ltxdoc}

\usepackage[margin=35mm]{geometry}
\usepackage{hyperref}
\usepackage{hyperxmp}
\usepackage[usenames]{color}

\hypersetup{colorlinks=true}
\hypersetup{pdfstartview=FitH}
\hypersetup{pdfpagemode=UseNone}
\hypersetup{pdfsource={}}
\hypersetup{pdflang={en-UK}}
\hypersetup{pdfcopyright={Copyright 2017-2018 Niklas Beisert.
  This work may be distributed and/or modified under the
  conditions of the LaTeX Project Public License, either version 1.3
  of this license or (at your option) any later version.}}
\hypersetup{pdflicenseurl={http://www.latex-project.org/lppl.txt}}
\hypersetup{pdfcontactaddress={ETH Zurich, ITP, HIT K,
  Wolfgang-Pauli-Strasse 27}}
\hypersetup{pdfcontactpostcode={8093}}
\hypersetup{pdfcontactcity={Zurich}}
\hypersetup{pdfcontactcountry={Switzerland}}
\hypersetup{pdfcontactemail={nbeisert@itp.phys.ethz.ch}}
\hypersetup{pdfcontacturl={http://people.phys.ethz.ch/\xmptilde nbeisert/}}

\newcommand{\secref}[1]{\hyperref[#1]{section \ref*{#1}}}

\parskip1ex
\parindent0pt
\let\olditemize\itemize
\def\itemize{\olditemize\parskip0pt}

\begin{document}

\title{The \textsf{childdoc} Package}
\hypersetup{pdftitle={The childdoc Package}}
\author{Niklas Beisert\\[2ex]
  Institut f\"ur Theoretische Physik\\
  Eidgen\"ossische Technische Hochschule Z\"urich\\
  Wolfgang-Pauli-Strasse 27, 8093 Z\"urich, Switzerland\\[1ex]
  \href{mailto:nbeisert@itp.phys.ethz.ch}
  {\texttt{nbeisert@itp.phys.ethz.ch}}}
\hypersetup{pdfauthor={Niklas Beisert}}
\hypersetup{pdfsubject={Manual for the LaTeX2e Package childdoc}}
\date{30 December 2018, \textsf{v2.0}}
\maketitle

\begin{abstract}\noindent
\textsf{childdoc} is a \LaTeXe{} package
that enables the direct compilation
of document sections included by |\include|
to individual files.
\end{abstract}

\begingroup
\parskip0ex
\tableofcontents
\endgroup

%%%%%%%%%%%%%%%%%%%%%%%%%%%%%%%%%%%%%%%%%%%%%%%%%%%%%%%%%%%%%%%%%%%%%%%%%%%%%%%%
%%%%%%%%%%%%%%%%%%%%%%%%%%%%%%%%%%%%%%%%%%%%%%%%%%%%%%%%%%%%%%%%%%%%%%%%%%%%%%%%
\section{Introduction}

\LaTeX{} provides a mechanism to structure a large document (such as a book)
into a main file and several child files (containing the chapters)
using the |\include| command.
This mechanism is beneficial for documents
which span hundreds of pages in order to
make the source file(s) more manageable.
Moreover, compilation can be restricted to
selected child files by means of the |\includeonly| command.
The latter feature can be used to reduce the compilation time while editing
(this was significantly more useful in the earlier days of \LaTeX{})
or to generate a smaller document which is easier to navigate.
Another application of |\includeonly| is to generate
documents consisting of selected parts of the complete document.

However, there are a few drawbacks of the plain |\include| mechanism:
\begin{itemize}
\item
The child files cannot be compiled on their own,
they can only be compiled via the main file.
A naive editing environment
(such as a text editor with an option
to have the current file processed by \LaTeX)
may require one to switch to the main file before compiling;
attempting to compile the child file produces errors.
\item
The main file must be modified (each time)
to adjust the |\includeonly| command
to the present needs. This easily leaves the main file in a messy state.
\item
The generated document will always carry the filename
of the main document. This is inconvenient if
several child files are to be compiled and
to be kept for distribution.
\end{itemize}

The present package provides a simple interface
to make child files individually compilable by \LaTeX{}.
Compiling a child file then has the same effect as compiling
the main file with an |\includeonly| command
to select the appropriate child.
Moreover the generated document will carry the name of the child
rather than the main file.
This resolves all three above issues.

This feature is meant to make the editing of books,
thesis documents and lecture notes somewhat more convenient.
However, the package can also be used efficiently for
composing a series of documents (such as exercise sheets)
which are typically distributed individually.
It then assists the author in generating the individual documents
(potentially in different versions)
as well as a document containing the collected series.
Another application is in developing style files
or other kinds of included material
where compilation of the style file could redirect
to a sample or test file.

%%%%%%%%%%%%%%%%%%%%%%%%%%%%%%%%%%%%%%%%%%%%%%%%%%%%%%%%%%%%%%%%%%%%%%%%%%%%%%%%
%%%%%%%%%%%%%%%%%%%%%%%%%%%%%%%%%%%%%%%%%%%%%%%%%%%%%%%%%%%%%%%%%%%%%%%%%%%%%%%%
\section{Usage}

First of all, the package \textsf{childdoc} is \emph{not} a standard
\LaTeXe{} |.sty| style file! Therefore it needs to be invoked in
a non-standard way.

%%%%%%%%%%%%%%%%%%%%%%%%%%%%%%%%%%%%%%%%%%%%%%%%%%%%%%%%%%%%%%%%%%%%%%%%%%%%%%%%
\subsection{Included Files}
\label{sec:include}

%%%%%%%%%%%%%%%%%%%%%%%%%%%%%%%%%%%%%%%%
\DescribeMacro{\childdocmain}
To use the package, add the commands
\begin{center}
\begin{tabular}{l}
|\input{childdoc.def}|\\
|\childdocmain{}|\\
\end{tabular}
\end{center}
at the very top of the main \LaTeX{} file,
in particular \emph{before} the |\documentclass| statement!
The argument of |\childdocmain| should be left empty
(but it must be present).

%%%%%%%%%%%%%%%%%%%%%%%%%%%%%%%%%%%%%%%%
\DescribeMacro{\childdocof}
Furthermore, add the commands
\begin{center}
\begin{tabular}{l}
|\input{childdoc.def}|\\
|\childdocof{|\textit{main}|}|\\
\end{tabular}
\end{center}
at the top of every child file \textit{child}
which is included by |\include{|\textit{child}|}|
from within the main file
(or at least for those files to be compiled individually).
The argument \textit{main} must be the filename of the main file.

There are a couple of
considerations in setting up the main and child documents:

%%%%%%%%%%%%%%%%%%%%%%%%%%%%%%%%%%%%%%%%
\paragraph{Restrictions.}

Please note the following restrictions:
\begin{itemize}
\item
|\childdocmain| must be called with one argument \textit{main}
to ensure compatibility with earlier version of the package.
It must either be empty (|\childdocmain{}|)
or precisely match the filename of the main file in which it is specified.
See \secref{sec:detection} for further information.
\item
The filename \textit{main} must be specified without the |.tex| extension.
\item
The filename \textit{main} is case sensitive
(even in case-insensitive file systems)
due to internal string comparison.
\item
The argument \textit{main} should be fully expanded, it cannot be a macro.
\item
Subdirectories and special characters should be avoided in filenames.
\item
The command |\childdocmain{|\textit{main}|}| must be followed by a whitespace.
It should not be followed immediately by another command
or by a comment mark `|%|'.
This is because the \TeX{} parser reads the token immediately following
the argument of |\childdocmain| and puts it
at the beginning of every child section;
however, a white\-space is ignored.
\end{itemize}

%%%%%%%%%%%%%%%%%%%%%%%%%%%%%%%%%%%%%%%%
\paragraph{Content of Main File.}

It is advisable to place all content in the child files included by |\include|.
Any output contained in the main file will appear in all child documents
unless suppressed manually;
it cannot be suppressed automatically by the |\includeonly| directive
and thus should normally be avoided.
A method to include some content in the main file
by means of conditional processing is described in \secref{sec:conditional}.

%%%%%%%%%%%%%%%%%%%%%%%%%%%%%%%%%%%%%%%%
\paragraph{Page Numbering.}

When only a part of the document is compiled,
the appropriate numbering of pages
(as well as other status parameters)
is determined from the |.aux| files.
The latter contain information from previous passes.
However this information needs to propagate through
all intermediate child documents.
Therefore the page numbering in child documents may well
be inconsistent until the complete document is compiled at least once.

A useful (if unconventional) way to always ensure a consistent
page numbering is to restart the numbering in each child document
and denote the pages by `\textit{child}|.|\textit{page}'
where \textit{child} represents the chapter/section number of the child file.
This can be achieved by the command
|\numberwithin{page}{|\textit{child}|}|
of the \textsf{amsmath} package
where \textit{child} can be |chapter| or |section|
depending on the chosen structuring.
Alternatively, one can modify the macro |\thepage| appropriately
and reset the counter |page| at the start of each child file.

%%%%%%%%%%%%%%%%%%%%%%%%%%%%%%%%%%%%%%%%%%%%%%%%%%%%%%%%%%%%%%%%%%%%%%%%%%%%%%%%
\subsection{Conditional Processing}
\label{sec:conditional}

The package provides a mechanism to compile different versions
of a document. To customise the versions further some conditional processing
can come in handy to distinguish which version is being compiled.
The package provides two macros to describe the compilation context:

%%%%%%%%%%%%%%%%%%%%%%%%%%%%%%%%%%%%%%%%
\DescribeMacro{\ifchilddoc}
The conditional |\ifchilddoc| distinguishes between the compilation of
child documents and the main document:
%
\begin{center}
|\ifchilddoc |\textit{child-code}| |[|\||else |\textit{main-code}]| \||fi|
\end{center}

%%%%%%%%%%%%%%%%%%%%%%%%%%%%%%%%%%%%%%%%
\DescribeMacro{\childdocname}
\DescribeMacro{\childdocjob}
The macro |\childdocname| contains the filename (without extension)
of the main or child file being processed.
Note that |\childdocjob| will always contain the name of the main file.

%%%%%%%%%%%%%%%%%%%%%%%%%%%%%%%%%%%%%%%%
\paragraph{Title Page.}

Conditional processing can be used to include a title or banner page
in the main document when proper precautions are taken.
Importantly, the code in the main file should ensure that the page counter
(as well as other status parameters which are stored in the |.aux| files)
takes the same value after the conditional processing.
Otherwise the page numbers may take divergent values
depending on which part is compiled.

For example, a title page could be declared by:
%
\begin{center}
\begin{tabular}{l}
|\ifchilddoc\||else|\\
|\addtocounter{page}{-1}|\\
\textit{code for title page}\\
|\newpage|\\
|\||fi|
\end{tabular}
\end{center}
%
A banner page for the child documents can be generated by:
%
\begin{center}
\begin{tabular}{l}
|\ifchilddoc|\\
|\addtocounter{page}{-1}|\\
\textit{code for banner page}\\
|\newpage|\\
|\||fi|
\end{tabular}
\end{center}
%
Here one could write a message such as:
\begin{center}
|This is the part \childdocname{} of \childdocjob{}.|
\end{center}

%%%%%%%%%%%%%%%%%%%%%%%%%%%%%%%%%%%%%%%%%%%%%%%%%%%%%%%%%%%%%%%%%%%%%%%%%%%%%%%%
\subsection{Flags}
\label{sec:flags}

The package makes it easy to generate different versions
of the main or child documents.
To this end compilation flags can be defined
and assigned different default values.
They will be particularly useful in conjunction
with the forwarding mechanism described in \secref{sec:forward}.

For example, it may be useful to have a flag |\version|
which can be set to |draft| or |final|.
The document source will contain some conditional code
depending on the value of |\version|.
Suppose further, the flag should default to |final| for the main file
and to |draft| for child files
which is a natural assignment for editing the document.
This is achieved by placing the following code
in the preamble of the main document
(below the |\childdocmain| directive):
%
\begin{center}
\begin{tabular}{l}
|\ifchilddoc|\\
|\providecommand{\version}{draft}|\\
|\||else|\\
|\providecommand{\version}{final}|\\
|\||fi|
\end{tabular}
\end{center}
%
The definition by |\providecommand| makes sure
that previous definitions are not overwritten.
Further statements |\providecommand{\version}{...}|
can thus be added before the above code to override it.

For the main file, one might add a line
(between |\childdocmain| and the above block)
%
\begin{center}
|%\ifchilddoc\||else\providecommand{\version}{draft}\||fi|
\end{center}
%
which can be uncommented to produce a draft version.
Likewise one can add a line to the very top of a child file
(above the |\childdocof{|\textit{main}|}| directive)
%
\begin{center}
|%\providecommand{\version}{final}|
\end{center}
%
which can be uncommented to produce the final version of this child document.

%%%%%%%%%%%%%%%%%%%%%%%%%%%%%%%%%%%%%%%%%%%%%%%%%%%%%%%%%%%%%%%%%%%%%%%%%%%%%%%%
\subsection{Forwarding}
\label{sec:forward}

Different versions of the main or child documents
using compilation flags as described in \secref{sec:flags}
can be (permanently) stored in different files
for convenient compilation, viewing and distribution.
To this end, the package defines a command
to pass on compilation to a different file:

%%%%%%%%%%%%%%%%%%%%%%%%%%%%%%%%%%%%%%%%
\DescribeMacro{\childdocforward}
The command |\childdocforward| redirects processing to
another source file:
%
\begin{center}
\begin{tabular}{l}
|\input{childdoc.def}|\\
|\childdocforward[|\textit{main}|]{|\textit{dest}|}|\\
\end{tabular}
\end{center}
%
The argument \textit{dest} is the destination file
(without extension).
It should be the main file or one of the child files.
Note that further \textsf{childdoc} directives
such as |\childdocof| and |\childdocforward|
in the indicated file will be processed in this form.
The optional argument \textit{main}
passes on directly to the main file \textit{main}
while pretending to compile the child \textit{dest}.
This form behaves as if \textit{dest}
issues |\childdocof{|\textit{main}|}| right away,
and no further \textsf{childdoc} directives will be processed.

%%%%%%%%%%%%%%%%%%%%%%%%%%%%%%%%%%%%%%%%
\DescribeMacro{\...prefix}
In the alternative form |\childdocforwardprefix|,
%
\begin{center}
\begin{tabular}{l}
|\input{childdoc.def}|\\
|\childdocforwardprefix[|\textit{main}|]{|\textit{prefix}|}{|\textit{dest}|}|
\end{tabular}
\end{center}
%
the destination file is determined by a pattern
depending on the current file:
To make this work, the current file must be called
`{\textit{prefix}\hspace{0.2em}\textit{suffix}}'
with \textit{prefix} matching precisely the argument.
Processing is then passed on to the file
`{\textit{dest}\hspace{0.2em}\textit{suffix}}'.
Surely, the same effect is achieved by
directly specifying the
argument `{\textit{dest}\hspace{0.2em}\textit{suffix}}'
in the first form.
However, that requires to set up a different file
for each child. With the alternative form of the command
all these files can have exactly the same content
which simplifies setting them up and maintaining them.

For example, the following file |draft.tex|
with a compilation flag |\version| as described in \secref{sec:flags}
compiles the main document as a draft:
%
\begin{center}
\begin{tabular}{l}
|\def\version{draft}|\\
|\input{childdoc.def}|\\
|\childdocforward{|\textit{main}|}|
\end{tabular}
\end{center}
%
Likewise, the following files |final|\textit{nn}|.tex|
compile the final version of the child document
|child|\textit{nn}|.tex|:
%
\begin{center}
\begin{tabular}{l}
|\def\version{final}|\\
|\input{childdoc.def}|\\
|\childdocforwardprefix{final}{child}|
\end{tabular}
\end{center}
%

Note that when several versions of a main file and/or of each child file
are to be generated, it may be convenient to set up a |Makefile| or
shell script to automatise the process.

%%%%%%%%%%%%%%%%%%%%%%%%%%%%%%%%%%%%%%%%%%%%%%%%%%%%%%%%%%%%%%%%%%%%%%%%%%%%%%%%
\subsection{Command Line Processing}
\label{sec:commandline}

The effect of redirection files can also be achieved by invoking
the \LaTeX{} compiler with a more elaborate command line.
Most conveniently this should be done as part
of a shell script or a |Makefile|.

When using \textsf{childdoc} in the main file, the following
command lines effectively perform a redirection
(note that depending on the shell being used,
backslashes may have to be doubled: `|\|' $\to$ `|\\|'):
%
\begin{center}
|... -jobname "|\textit{target}|" |\\|"|[\textit{flags}]%
|\input{childdoc.def}\childdocforward[|\textit{main}|]{|\textit{dest}|}"|
\end{center}
%
Here \textit{target} is the name of the output file,
\textit{main} is the name of the main file
and \textit{dest} is the name of the main or child file to be processed
(all filenames without extensions).
The optional argument \textit{main} can be omitted
if \textit{main} matches \textit{dest}.
Optionally, compilation \textit{flags} can be defined via |\def| commands.
This command line makes the \TeX{} engine believe
it is compiling the file \textit{target}
whose content is specified as the latter parameter.
The provided code then forwards the processing to
\textit{main} or \textit{dest} as described in \secref{sec:forward}.

%%%%%%%%%%%%%%%%%%%%%%%%%%%%%%%%%%%%%%%%%%%%%%%%%%%%%%%%%%%%%%%%%%%%%%%%%%%%%%%%
\subsection{Include by Input}
\label{sec:input}

Including child documents by |\include| has some restrictions by design.
Most notably, the content of a child document always occupies
its own set of pages; pages cannot be shared between child documents.
Usually, this behaviour makes perfect sense
because each child document contain an essential part of the document.
However, in some situations it may be desirable to compose
a document from a collection of parts
without having mandatory page breaks between then.
For this case, the package
provides a mechanism to include parts
by |\input| which can also be processed individually.
However, by construction this mechanism
requires manual handling of the content to be output.

%%%%%%%%%%%%%%%%%%%%%%%%%%%%%%%%%%%%%%%%
\DescribeMacro{\ifchilddocmanual}
The main file should be prepared as usual, see \secref{sec:include}.
However, the document body must make a distinction
between processing of an individual part and of the main document, e.g.:
%
\begin{center}
\begin{tabular}{l}
|\ifchilddocmanual|\\
|\input{\childdocname}|\\
|\||else|\\
\textit{document body with }|\input{|\textit{part}|}|\\
|\||fi|
\end{tabular}
\end{center}
%
The conditional |\ifchilddocmanual| is true whenever
a part to be included by |\input| is being compiled,
and the name of the part is stored in |\childdocname|.

%%%%%%%%%%%%%%%%%%%%%%%%%%%%%%%%%%%%%%%%
\DescribeMacro{\childdocby}
Each part to be included by |\input| should start with:
%
\begin{center}
\begin{tabular}{l}
|\input{childdoc.def}|\\
|\childdocby{|\textit{main}|}|\\
\end{tabular}
\end{center}
%
The directive |\childdocby| is similar to |\childdocof|
described in \secref{sec:include},
but the subsequent selection of content must be done manually.
To that end, both |\ifchilddoc| and |\ifchilddocmanual|
will be true upon processing of a part,
and the name of the part is stored in |\childdocname|.
Note that |\jobname| will be set to the filename of the current part
so that each part receives an individual |.aux| file
that does not interfere with the |.aux| file(s) of the main document.
This behaviour can be altered by the alternative form
|\childdocby[*]{|\textit{main}|}| (with a non-empty optional argument)
which uses the |.aux| file of the main document
by setting |\jobname| to \textit{main}.

%%%%%%%%%%%%%%%%%%%%%%%%%%%%%%%%%%%%%%%%%%%%%%%%%%%%%%%%%%%%%%%%%%%%%%%%%%%%%%%%
\subsection{Driver Development}
\label{sec:driver}

The \textsf{childdoc} mechanism can also be use for the development
of definition files such as \LaTeX{} styles or classes.
This case differs from the above setup with multiple parts
included by |\include| in that no |\includeonly| should be invoked.
This can be achieved by starting the include file
(before |\ProvidesPackage|) with:
%
\begin{center}
\begin{tabular}{l}
|\input{childdoc.def}|\\
|\childdocforward{|\textit{main}|}|\\
\end{tabular}
\end{center}
%
or alternatively with:
%
\begin{center}
\begin{tabular}{l}
|\input{childdoc.def}|\\
|\childdocby{|\textit{main}|}|\\
\end{tabular}
\end{center}
%
Both forms have slightly different effects as described above.
The main file is prepared as usual, see \secref{sec:include}.

%%%%%%%%%%%%%%%%%%%%%%%%%%%%%%%%%%%%%%%%%%%%%%%%%%%%%%%%%%%%%%%%%%%%%%%%%%%%%%%%
\subsection{Legacy Detection}
\label{sec:detection}

The directive |\childdocmain| in the main file can detect
whether the complete document or merely a child is to be compiled
even without using the directive |\childdocof|.
This method is deprecated because it is less robust
and there is no compelling reason to use it;
it is merely provided for backward compatibility
and it may be removed in future versions.

If the detection mechanism is to be used,
it is mandatory to correctly specify
the filename of the main file as the argument of |\childdocmain|:
%
\begin{center}
\begin{tabular}{l}
|\input{childdoc.def}|\\
|\childdocmain{|\textit{main}|}|\\
\end{tabular}
\end{center}
%
If |\jobname| does not match the argument \textit{main} of |\childdocmain|,
it is assumed that |\jobname| points to the child file to be compiled.
When using |\childdocmain| with the main file specified as argument,
it suffices to start a child file
with just |\input{|\textit{main}|}|
without loading of the package and using |\childdocof|.
If instead all processing is done
with the appropriate \textsf{childdoc} directives,
the argument of \textit{main} of |\childdocmain| can be empty.

An alternative version of the command line processing described
in \secref{sec:commandline} using the detection mechanism reads:
%
\begin{center}
|... -jobname "|\textit{target}|" "|[\textit{flags}]%
[|\def\jobname{|\textit{dest}|}|]|\input{|\textit{main}|}"|
\end{center}

%%%%%%%%%%%%%%%%%%%%%%%%%%%%%%%%%%%%%%%%%%%%%%%%%%%%%%%%%%%%%%%%%%%%%%%%%%%%%%%%
\subsection{Manual Code}
\label{sec:manual}

In case one cannot be certain whether the definitions file |childdoc.def|
is installed on the target \TeX{} distribution
and one prefers not to ship it,
it is conceivable to paste a few relevant commands into the sources.

To that end, drop all statements |\input{childdoc.def}|
and perform the replacements as outlined below.
Instead of |\childdocmain{|\textit{main}|}| add the following code
to the top of the main file:
%
\begin{center}
\begin{tabular}{l}
|\||ifdefined\childdocname\endinput\||fi\newif\ifchilddoc|\\
|\edef\childdocname{\scantokens\expandafter{\jobname\noexpand}}|\\
|\def\childdocmain{|\textit{main}|}\||ifx\childdocmain\childdocname\||else|\\
|\childdoctrue\includeonly{\childdocname}\let\jobname\childdocmain\||fi|\\
\end{tabular}
\end{center}
%
Instead of |\childdocof{|\textit{main}|}| just include the main file
at the top of each child file:
%
\begin{center}
|\input{|\textit{main}|}|
\end{center}
%
A simple redirection |\childdocforward{|\textit{dest}|}| is achieved by:
%
\begin{center}
|\def\jobname{|\textit{dest}|}\input{\jobname}|
\end{center}
%
The redirection with prefix
|\childdocforwardprefix[|\textit{prefix}|]{|\textit{dest}|}|
is accomplished by:
%
\begin{center}
\begin{tabular}{l}
|{\edef\jobname{\scantokens\expandafter{\jobname\noexpand}}|\\
|\def\redirectjob |\textit{prefix}|#1~~~{\gdef\jobname{|\textit{dest}|#1}}|\\
|\expandafter\redirectjob\jobname~~~}\input{\jobname}|
\end{tabular}
\end{center}

In an alternative approach,
child documents can be compiled by a specific command line
without additional code or specific definitions:
%
\begin{center}
|... -jobname "|\textit{target}|" "|[\textit{flags}]%
|\includeonly{|\textit{dest}|}\input{|\textit{main}|}"|
\end{center}
%

%%%%%%%%%%%%%%%%%%%%%%%%%%%%%%%%%%%%%%%%%%%%%%%%%%%%%%%%%%%%%%%%%%%%%%%%%%%%%%%%
%%%%%%%%%%%%%%%%%%%%%%%%%%%%%%%%%%%%%%%%%%%%%%%%%%%%%%%%%%%%%%%%%%%%%%%%%%%%%%%%
\section{Information}

%%%%%%%%%%%%%%%%%%%%%%%%%%%%%%%%%%%%%%%%%%%%%%%%%%%%%%%%%%%%%%%%%%%%%%%%%%%%%%%%
\subsection{Copyright}

Copyright \copyright{} 2017--2018 Niklas Beisert

This work may be distributed and/or modified under the
conditions of the \LaTeX{} Project Public License, either version 1.3
of this license or (at your option) any later version.
The latest version of this license is in
  \url{http://www.latex-project.org/lppl.txt}
and version 1.3 or later is part of all distributions of \LaTeX{}
version 2005/12/01 or later.

This work has the LPPL maintenance status `maintained'.

The Current Maintainer of this work is Niklas Beisert.

This work consists of the files |README.txt|, |childdoc.ins| and |childdoc.dtx|
as well as the derived files |childdoc.def|, |cdocsamp.tex|
with |cdocsch1.tex|, |cdocsch2.tex|, |cdocspt3.tex|, |cdocspt4.tex|,
|cdocsdrf.tex|, |cdocsfn1.tex|, |cdocsfn2.tex|
as well as |childdoc.pdf|.

%%%%%%%%%%%%%%%%%%%%%%%%%%%%%%%%%%%%%%%%%%%%%%%%%%%%%%%%%%%%%%%%%%%%%%%%%%%%%%%%
\subsection{Files and Installation}

The package consists of the files:
%
\begin{center}
\begin{tabular}{ll}
    |README.txt|   & readme file \\
    |childdoc.ins| & installation file \\
    |childdoc.dtx| & source file \\
    |childdoc.def| & definition file \\
    |cdocsamp.tex| & sample main file \\
    |cdocsch1.tex| & sample include file \\
    |cdocsch2.tex| & sample include file \\
    |cdocspt3.tex| & sample part file \\
    |cdocspt4.tex| & sample part file \\
    |cdocsdrf.tex| & sample redirection file \\
    |cdocsfn1.tex| & sample redirection file \\
    |cdocsfn2.tex| & sample redirection file \\
    |childdoc.pdf| & manual
\end{tabular}
\end{center}
%
The distribution consists of the files
|README.txt|, |childdoc.ins| and |childdoc.dtx|.
%
\begin{itemize}
\item
Run (pdf)\LaTeX{} on |childdoc.dtx|
to compile the manual |childdoc.pdf| (this file).
\item
Run \LaTeX{} on |childdoc.ins| to create the definitions file |childdoc.def|
and the sample |cdocsamp.tex| with include files
|cdocsch1.tex|, |cdocsch2.tex|, |cdocspt3.tex|, |cdocspt4.tex|,
|cdocsdrf.tex|, |cdocsfn1.tex|, |cdocsfn2.tex|.
Then copy the file |childdoc.def| to an appropriate directory of your \LaTeX{}
distribution, e.g.\ \textit{texmf-root}|/tex/latex/childdoc|.
\end{itemize}

%%%%%%%%%%%%%%%%%%%%%%%%%%%%%%%%%%%%%%%%%%%%%%%%%%%%%%%%%%%%%%%%%%%%%%%%%%%%%%%%
\subsection{Related CTAN Packages}

There are several other packages which offer a similar functionality:
%
\begin{itemize}
\item
The packages
\href{http://ctan.org/pkg/docmute}{\textsf{docmute}},
\href{http://ctan.org/pkg/includex}{\textsf{includex}} and
\href{http://ctan.org/pkg/standalone}{\textsf{standalone}}
provide commands to include only the document body of
a child file thus allowing both files to be compiled individually.
\item
The packages \href{http://ctan.org/pkg/subdocs}{\textsf{subdocs}}
and \href{http://ctan.org/pkg/subfiles}{\textsf{subfiles}}
provide structures in which the main and child documents can be
encapsulated and allowing them to be compiled individually.
The inclusion mechanism is different from the conventional |\include|.
\item
The package \href{http://ctan.org/pkg/combine}{\textsf{combine}}
is an elaborate solution to combine several documents into one.
\end{itemize}
%
See also the CTAN topic \href{http://ctan.org/topic/subdocs}{\textsf{subdocs}}
for further related packages.
The present package differs from the above solutions in that
a document structure constructed with the conventional |\include| mechanism
just needs two extra commands at the top of every file
such that all constituent files can be compiled individually.

%%%%%%%%%%%%%%%%%%%%%%%%%%%%%%%%%%%%%%%%%%%%%%%%%%%%%%%%%%%%%%%%%%%%%%%%%%%%%%%%
%\subsection{Feature Suggestions}
%
%The following is a list of features which may be useful for future
%versions of this package:
%%
%\begin{itemize}
%\item
%\ldots
%\end{itemize}

%%%%%%%%%%%%%%%%%%%%%%%%%%%%%%%%%%%%%%%%%%%%%%%%%%%%%%%%%%%%%%%%%%%%%%%%%%%%%%%%
\subsection{Revision History}

%%%%%%%%%%%%%%%%%%%%%%%%%%%%%%%%%%%%%%%%
\paragraph{v2.0:} 2018/12/30

\begin{itemize}
\item
immediate forward processing
\item
added |\childdocby| mechanism
\item
manual restructured
\end{itemize}

%%%%%%%%%%%%%%%%%%%%%%%%%%%%%%%%%%%%%%%%
\paragraph{v1.6:} 2018/01/17

\begin{itemize}
\item
application for development of include files
\item
corrections to manual
\end{itemize}

%%%%%%%%%%%%%%%%%%%%%%%%%%%%%%%%%%%%%%%%
\paragraph{v1.5:} 2017/05/21

\begin{itemize}
\item
more complete structuring introduced
\item
|\childdocof| introduced
\item
|\childdoc| renamed to |\childdocmain|
\item
|\childredirect| renamed to |\childdocforward| and |\childdocforwardprefix|
and functionality expanded
\end{itemize}

%%%%%%%%%%%%%%%%%%%%%%%%%%%%%%%%%%%%%%%%
\paragraph{v1.0:} 2017/04/27

\begin{itemize}
\item
manual and install package
\item
first version published on CTAN
\end{itemize}

%%%%%%%%%%%%%%%%%%%%%%%%%%%%%%%%%%%%%%%%
\paragraph{v0.6:} 2017/04/26

\begin{itemize}
\item
redirection mechanism added
\end{itemize}

%%%%%%%%%%%%%%%%%%%%%%%%%%%%%%%%%%%%%%%%
\paragraph{v0.5:} 2017/04/26

\begin{itemize}
\item
functionality in definition file
\end{itemize}


%%%%%%%%%%%%%%%%%%%%%%%%%%%%%%%%%%%%%%%%%%%%%%%%%%%%%%%%%%%%%%%%%%%%%%%%%%%%%%%%
%%%%%%%%%%%%%%%%%%%%%%%%%%%%%%%%%%%%%%%%%%%%%%%%%%%%%%%%%%%%%%%%%%%%%%%%%%%%%%%%
%%%%%%%%%%%%%%%%%%%%%%%%%%%%%%%%%%%%%%%%%%%%%%%%%%%%%%%%%%%%%%%%%%%%%%%%%%%%%%%%
\appendix

\settowidth\MacroIndent{\rmfamily\scriptsize 000\ }

 \DocInput{childdoc.dtx}

\end{document}
%</driver>
% \fi
%
% %%%%%%%%%%%%%%%%%%%%%%%%%%%%%%%%%%%%%%%%%%%%%%%%%%%%%%%%%%%%%%%%%%%%%%%%%%%%%%
% %%%%%%%%%%%%%%%%%%%%%%%%%%%%%%%%%%%%%%%%%%%%%%%%%%%%%%%%%%%%%%%%%%%%%%%%%%%%%%
% \section{Sample}
%\iffalse
%<*samplemain>
%\fi
%
% The following presents a sample document
% with two chapters, two parts, a title page,
% a compile flag as well as three forwarding files to set the flag.
% It consists of eight |.tex| files:
% \begin{center}
% \begin{tabular}{ll}
% |cdocsamp.tex|&main file\\
% |cdocsch1.tex|&include file for chapter 1\\
% |cdocsch2.tex|&include file for chapter 2\\
% |cdocspt3.tex|&include file for part 3\\
% |cdocspt4.tex|&include file for part 4\\
% |cdocsdrf.tex|&forwarding file for main file in draft mode\\
% |cdocsfi1.tex|&forwarding file for final version of chapter 1\\
% |cdocsfi2.tex|&forwarding file for final version of chapter 2\\
% \end{tabular}
% \end{center}
% Each of the eight files can be compiled directly by the \LaTeX{} compiler.
%
% %%%%%%%%%%%%%%%%%%%%%%%%%%%%%%%%%%%%%%
% \paragraph{Main File.}
%
% The main file is called |cdocsamp.tex|.
%
% Load the \textsf{childdoc} definitions and
% declare the filename for the main document:
%    \begin{macrocode}
\input{childdoc.def}
\childdocmain{}
%    \end{macrocode}

% Optional override for |\version| flag:
%    \begin{macrocode}
%%\ifchilddoc\else\providecommand{\version}{draft}\fi
%    \end{macrocode}

% Define the default values for the |\version| flag
% (|final| for the main file and |draft| for childs):
%    \begin{macrocode}
\ifchilddoc
\providecommand{\version}{draft}
\else
\providecommand{\version}{final}
\fi
%    \end{macrocode}

% Load the standard document class:
%    \begin{macrocode}
\documentclass[12pt]{article}
%    \end{macrocode}

% Start the document body:
%    \begin{macrocode}
\begin{document}
%    \end{macrocode}

% Declare a title page.
% Print title, part of document being processed and version flag:
%    \begin{macrocode}
\addtocounter{page}{-1}
\begin{center}
{\LARGE\bfseries{}childdoc example\par}
\vspace{1cm}
\ifchilddoc
\ifchilddocmanual part\else chapter\fi:
`\childdocname' of `\childdocjob'\par
\else
main document: `\childdocjob'\par
\fi
version: \version\par
\end{center}
\newpage
%    \end{macrocode}

% Manually include selected file,
% otherwise process as usual:
%    \begin{macrocode}
\ifchilddocmanual
\section*{part `\childdocname'}
\input{\childdocname}
\else
%    \end{macrocode}

% Include the two chapters:
%    \begin{macrocode}
\include{cdocsch1}
\include{cdocsch2}
%    \end{macrocode}

% Include the two parts unless only chapters should be displayed:
%    \begin{macrocode}
\ifchilddoc\else
\section{part three}
\input{cdocspt3}
\section{part four}
\input{cdocspt4}
\fi
%    \end{macrocode}

% Process as usual until here:
%    \begin{macrocode}
\fi
%    \end{macrocode}

% End of document body:
%    \begin{macrocode}
\end{document}
%    \end{macrocode}
%\iffalse
%</samplemain>
%\fi
%
% %%%%%%%%%%%%%%%%%%%%%%%%%%%%%%%%%%%%%%
% \paragraph{Chapter Include Files.}
%
% The include files are called |cdocsch1.tex| and |cdocsch2.tex|.
%
%\iffalse
%<*samplechap1|samplechap2>
%\fi

% Optional override for |\version| flag:
%    \begin{macrocode}
%%\providecommand{\version}{final}
%    \end{macrocode}

% Include the main document:
%    \begin{macrocode}
\input{childdoc.def}
\childdocof{cdocsamp}
%    \end{macrocode}

%\iffalse
%</samplechap1|samplechap2>
%\fi
%
%\iffalse
%<*samplechap1>
%\fi
% Some text for chapter 1:
%    \begin{macrocode}
\section{one}
some text in chapter one
%    \end{macrocode}

%\iffalse
%</samplechap1>
%\fi
% Some text for chapter 2:
%\iffalse
%<*samplechap2>
%\fi
%    \begin{macrocode}
\section{two}
more text in chapter two
%    \end{macrocode}

%\iffalse
%</samplechap2>
%\fi
%
% %%%%%%%%%%%%%%%%%%%%%%%%%%%%%%%%%%%%%%
% \paragraph{Part Include Files.}
%
% The include files are called |cdocspt3.tex| and |cdocspt4.tex|.
%
%\iffalse
%<*samplepart3|samplepart4>
%\fi

% Optional override for |\version| flag:
%    \begin{macrocode}
%%\providecommand{\version}{final}
%    \end{macrocode}

% Include the main document:
%    \begin{macrocode}
\input{childdoc.def}
\childdocby{cdocsamp}
%    \end{macrocode}

%\iffalse
%</samplepart3|samplepart4>
%\fi
%
%\iffalse
%<*samplepart3>
%\fi
% Some text for part 3:
%    \begin{macrocode}
some text in part three
%    \end{macrocode}

%\iffalse
%</samplepart3>
%\fi
% Some text for part 4:
%\iffalse
%<*samplepart4>
%\fi
%    \begin{macrocode}
more text in part four
%    \end{macrocode}

%\iffalse
%</samplepart4>
%\fi
%
% %%%%%%%%%%%%%%%%%%%%%%%%%%%%%%%%%%%%%%
% \paragraph{Forwarding for a Complete Draft.}
%
% The following forwarding file |cdocsdrf.tex|
% compiles the main document in draft mode:
%\iffalse
%<*sampledraft>
%\fi
%    \begin{macrocode}
\def\version{draft}
\input{childdoc.def}
\childdocforward{cdocsamp}
%    \end{macrocode}

%\iffalse
%</sampledraft>
%\fi
%
% %%%%%%%%%%%%%%%%%%%%%%%%%%%%%%%%%%%%%%
% \paragraph{Forwarding for Final Version of the Chapters.}
%
% The following forwarding files |cdocsfn1.tex| and |cdocsfn2.tex|
% (with identical content)
% compile the final versions of the child documents
% |cdocsch1.tex| and |cdocsch2.tex|, respectively:
%\iffalse
%<*samplefinal>
%\fi
%    \begin{macrocode}
\def\version{final}
\input{childdoc.def}
\childdocforwardprefix[cdocsamp]{cdocsfn}{cdocsch}
%    \end{macrocode}

%\iffalse
%</samplefinal>
%\fi
%
% %%%%%%%%%%%%%%%%%%%%%%%%%%%%%%%%%%%%%%
% \paragraph{Command Line Processing.}
%
% The following three command lines generate the output files
% |cdocscld|, |cdocscl1| and |cdocscl2|
% which should be identical to
% |cdocsdrf|, |cdocsch1| and |cdocsfn2|, respectively:
% \begin{center}
% \begin{tabular}{l}
% |latex -jobname cdocscld \|\\
% |  "\def\version{draft}\input{childdoc.def}\childdocforward{cdocsamp}"|\\
% |latex -jobname cdocscl1 \|\\
% |  "\input{childdoc.def}\childdocforward[cdocsamp]{cdocsch1}"|\\
% |latex -jobname cdocscl2 \|\\
% |  "\def\version{final}\input{childdoc.def}\childdocforward{cdocsch2}"|
% \end{tabular}
% \end{center}
% Note that the trailing backslash on each first line
% merely continues the input to the second line
% (for convenient cut ant paste).
% Furthermore, the command |latex| can be replaced by any
% of its alternative versions such as |pdflatex|.
%
% %%%%%%%%%%%%%%%%%%%%%%%%%%%%%%%%%%%%%%%%%%%%%%%%%%%%%%%%%%%%%%%%%%%%%%%%%%%%%%
% %%%%%%%%%%%%%%%%%%%%%%%%%%%%%%%%%%%%%%%%%%%%%%%%%%%%%%%%%%%%%%%%%%%%%%%%%%%%%%
% \section{Implementation}
%\iffalse
%<*package>
%\fi
%
% This section describes the definitions file |childdoc.def|.

% The definitions cannot be loaded using |\usepackage| or |\RequirePackage|
% which has a mechanism to prevent loading a style file more than once.
% When loading the definitions by means of |\input|
% multiple instances have to be prevented manually:
%\iffalse
%This code needs to be before the `\ProvidesFile' directive
%which is defined at the beginning of this file.
%Therefore it is also placed there and commented out here.
%</package>
%<*discard>
%\fi
%    \begin{macrocode}
\ifdefined\childdocmain\endinput\fi
%    \end{macrocode}
%\iffalse
%</discard>
%<*package>
%\fi
%
% \macro{\ifchilddoc}
% \macro{\ifchilddocmanual}
% The conditional |\ifchilddoc| tells whether a
% child (true) or main (false) document is being compiled.
% The conditional |\ifchilddocmanual| tells whether
% the |\includeonly| mechanism is used (false) or
% the selection of child files must be performed manually (true).
% The definitions initialise to false:
%    \begin{macrocode}
\newif\ifchilddoc
\newif\ifchilddocmanual
%    \end{macrocode}

% \macro{\childdocname}
% \macro{\childdocjob}
% The macro |\childdocname| stores the name of the main document
% to be compiled. The macro |\childdocjob| stores the name of
% the document on which the \LaTeX{} compiler was originally invoked.
% The content of |\jobname| cannot be compared
% to filenames specified in the source due to different catcodes.
% The following code rescans |\jobname|, stores the result
% in |\childdocname| and saves a copy in |\childdocjob|:
%    \begin{macrocode}
\edef\childdocname{\scantokens\expandafter{\jobname\noexpand}}
\let\childdocjob\childdocname
%    \end{macrocode}

% \macro{\childdocdisable}
% The macro |\childdocdisable| prevents the main file
% from being processed more than once.
% At this stage, the main document command |\childdocmain|
% is assumed to be called once again where it should do nothing.
% Any subsequent call to it should prevent
% a secondary processing of the main document
% It overwrites the forwarding commands
% |\childdocof| and |\childdocforward|
% with empty macros to prevent further inclusions of the main document:
%    \begin{macrocode}
\newcommand{\childdocdisable}
{
  \renewcommand{\childdocmain}[1]{\renewcommand{\childdocmain}[1]{\endinput}}
  \renewcommand{\childdocof}[1]{}
  \renewcommand{\childdocby}[2][]{}
  \renewcommand{\childdocforward}[2][]{}
  \renewcommand{\childdocdisable}{}
}
%    \end{macrocode}

% \macro{\childdocmain}
% The macro |\childdocmain| is to be called at the top of the main file
% with nothing or the main filename (without extension) as argument.
% First, it breaks loops.
% If the argument is not empty and does not match |\childdocname|
% (which is set by the first inclusion of |childdoc.def|),
% |\ifchilddoc| is set to true, |\includeonly| is applied to the child file
% and |\jobname| is set to the main file
% (for proper handling of |.aux| files):
%    \begin{macrocode}
\newcommand{\childdocmain}[1]
{
  \childdocdisable\childdocmain{}
  \if?#1?\else
    \begingroup
      \def\childdoctmp{#1}
      \ifx\childdoctmp\childdocname
        \def\childdoctmp{}
      \else
        \def\childdoctmp
        {
          \childdoctrue
          \includeonly{\childdocname}
          \def\childdocjob{#1}
          \def\jobname{#1}
        }
      \fi
      \expandafter
    \endgroup
    \childdoctmp
  \fi
}
%    \end{macrocode}

% \macro{\childdocof}
% The command |\childdocof| redirects
% compilation to the main file |#1|.
%    \begin{macrocode}
\newcommand{\childdocof}[1]
{
  \childdocdisable
  \childdoctrue
  \includeonly{\childdocname}
  \def\jobname{#1}
  \def\childdocjob{#1}
  \input{#1}
}
%    \end{macrocode}

% \macro{\childdocby}
% The command |\childdocby| ....
%    \begin{macrocode}
\newcommand{\childdocby}[2][]
{
  \childdocdisable
  \childdoctrue
  \childdocmanualtrue
  \if?#1?\else
    \def\jobname{#2}
  \fi
  \def\childdocjob{#2}
  \input{#2}
  \endinput
}
%    \end{macrocode}

% \macro{\childdocforward}
% The command |\childdocforward| redirects
% compilation to the main file or
% (if the optional argument is given) a child file.
% Parameters are set as if the main file
% or a child file starting with |\childdocof| was compiled.
% Then compilation is handed over to the main file:
%    \begin{macrocode}
\newcommand{\childdocforward}[2][]
{
  \begingroup
    \if?#1?
      \def\childdoctmp
      {
        \def\childdocname{#2}
        \def\childdocjob{#2}
        \def\jobname{#2}
        \input{#2}
        \endinput
      }
    \else
      \def\childdoctmp
      {
        \childdocdisable
        \def\childdocname{#2}
        \childdoctrue
        \includeonly{#2}
        \def\childdocjob{#1}
        \def\jobname{#1}
        \input{#1}
        \endinput
      }
    \fi
    \expandafter
  \endgroup
  \childdoctmp
}
%    \end{macrocode}

% \macro{\childdocforwardprefix}
% The command |\childdocforwardprefix| redirects
% compilation to the main or a child file by means of a pattern.
% The prefix |#1| in the current filename is replaced by |#2|
% and the suffix of the current filename is kept
% (it is assumed that the filename does not contain the substring `|~~~|'
% which is used as a delimiter).
% Compilation is handed over to the new file by |\childdocforward|:
%    \begin{macrocode}
\newcommand{\childdocforwardprefix}[3][]
{
  \begingroup
    \def\childdocextract #2##1~~~{\def\childdoctmp{\childdocforward[#1]{#3##1}}}
    \expandafter\childdocextract\childdocname~~~
    \expandafter
  \endgroup
  \childdoctmp
}
%    \end{macrocode}

% \macro{\childdoc}
% The deprecated macro |\childdoc| is a legacy version of |\childdocmain|:
%    \begin{macrocode}
\newcommand{\childdoc}{\childdocmain}
%    \end{macrocode}

% \macro{\childdocredirect}
% The deprecated macro |\childdocredirect| is a legacy version
% of |\childdocforward| and |\childdocforwardprefix|:
%    \begin{macrocode}
\newcommand{\childdocredirect}[2][]
{
  \begingroup
    \if?#1?
      \def\childdoctmp{\childdocforward{#2}}
    \else
      \def\childdoctmp{\childdocforwardprefix{#1}{#2}}
    \fi
    \expandafter
  \endgroup
  \childdoctmp
}
%    \end{macrocode}

%\iffalse
%</package>
%\fi
%
\endinput
|\\
|\childdocby{|\textit{main}|}|\\
\end{tabular}
\end{center}
%
The directive |\childdocby| is similar to |\childdocof|
described in \secref{sec:include},
but the subsequent selection of content must be done manually.
To that end, both |\ifchilddoc| and |\ifchilddocmanual|
will be true upon processing of a part,
and the name of the part is stored in |\childdocname|.
Note that |\jobname| will be set to the filename of the current part
so that each part receives an individual |.aux| file
that does not interfere with the |.aux| file(s) of the main document.
This behaviour can be altered by the alternative form
|\childdocby[*]{|\textit{main}|}| (with a non-empty optional argument)
which uses the |.aux| file of the main document
by setting |\jobname| to \textit{main}.

%%%%%%%%%%%%%%%%%%%%%%%%%%%%%%%%%%%%%%%%%%%%%%%%%%%%%%%%%%%%%%%%%%%%%%%%%%%%%%%%
\subsection{Driver Development}
\label{sec:driver}

The \textsf{childdoc} mechanism can also be use for the development
of definition files such as \LaTeX{} styles or classes.
This case differs from the above setup with multiple parts
included by |\include| in that no |\includeonly| should be invoked.
This can be achieved by starting the include file
(before |\ProvidesPackage|) with:
%
\begin{center}
\begin{tabular}{l}
|% \iffalse
%
% childdoc.dtx Copyright (C) 2017-2018 Niklas Beisert
%
% This work may be distributed and/or modified under the
% conditions of the LaTeX Project Public License, either version 1.3
% of this license or (at your option) any later version.
% The latest version of this license is in
%   http://www.latex-project.org/lppl.txt
% and version 1.3 or later is part of all distributions of LaTeX
% version 2005/12/01 or later.
%
% This work has the LPPL maintenance status `maintained'.
%
% The Current Maintainer of this work is Niklas Beisert.
%
% This work consists of the files childdoc.dtx and childdoc.ins
% and the derived files childdoc.def and cdocsamp.tex with
% cdocsch1.tex, cdocsch2.tex, cdocsdrf.tex, cdocsfn1.tex, cdocsfn2.tex.
%
%<package>\ifdefined\childdocmain\endinput\fi
%<package>\ProvidesFile{childdoc.def}[2018/12/30 v2.0 child document driver]
%<samplemain>\ProvidesFile{cdocsamp.tex}[2018/12/30 v2.0 sample for childdoc]
%<*driver>
%\ProvidesFile{childdoc.drv}[2018/12/30 v2.0 childdoc reference manual file]
\PassOptionsToClass{10pt,a4paper}{article}
\documentclass{ltxdoc}

\usepackage[margin=35mm]{geometry}
\usepackage{hyperref}
\usepackage{hyperxmp}
\usepackage[usenames]{color}

\hypersetup{colorlinks=true}
\hypersetup{pdfstartview=FitH}
\hypersetup{pdfpagemode=UseNone}
\hypersetup{pdfsource={}}
\hypersetup{pdflang={en-UK}}
\hypersetup{pdfcopyright={Copyright 2017-2018 Niklas Beisert.
  This work may be distributed and/or modified under the
  conditions of the LaTeX Project Public License, either version 1.3
  of this license or (at your option) any later version.}}
\hypersetup{pdflicenseurl={http://www.latex-project.org/lppl.txt}}
\hypersetup{pdfcontactaddress={ETH Zurich, ITP, HIT K,
  Wolfgang-Pauli-Strasse 27}}
\hypersetup{pdfcontactpostcode={8093}}
\hypersetup{pdfcontactcity={Zurich}}
\hypersetup{pdfcontactcountry={Switzerland}}
\hypersetup{pdfcontactemail={nbeisert@itp.phys.ethz.ch}}
\hypersetup{pdfcontacturl={http://people.phys.ethz.ch/\xmptilde nbeisert/}}

\newcommand{\secref}[1]{\hyperref[#1]{section \ref*{#1}}}

\parskip1ex
\parindent0pt
\let\olditemize\itemize
\def\itemize{\olditemize\parskip0pt}

\begin{document}

\title{The \textsf{childdoc} Package}
\hypersetup{pdftitle={The childdoc Package}}
\author{Niklas Beisert\\[2ex]
  Institut f\"ur Theoretische Physik\\
  Eidgen\"ossische Technische Hochschule Z\"urich\\
  Wolfgang-Pauli-Strasse 27, 8093 Z\"urich, Switzerland\\[1ex]
  \href{mailto:nbeisert@itp.phys.ethz.ch}
  {\texttt{nbeisert@itp.phys.ethz.ch}}}
\hypersetup{pdfauthor={Niklas Beisert}}
\hypersetup{pdfsubject={Manual for the LaTeX2e Package childdoc}}
\date{30 December 2018, \textsf{v2.0}}
\maketitle

\begin{abstract}\noindent
\textsf{childdoc} is a \LaTeXe{} package
that enables the direct compilation
of document sections included by |\include|
to individual files.
\end{abstract}

\begingroup
\parskip0ex
\tableofcontents
\endgroup

%%%%%%%%%%%%%%%%%%%%%%%%%%%%%%%%%%%%%%%%%%%%%%%%%%%%%%%%%%%%%%%%%%%%%%%%%%%%%%%%
%%%%%%%%%%%%%%%%%%%%%%%%%%%%%%%%%%%%%%%%%%%%%%%%%%%%%%%%%%%%%%%%%%%%%%%%%%%%%%%%
\section{Introduction}

\LaTeX{} provides a mechanism to structure a large document (such as a book)
into a main file and several child files (containing the chapters)
using the |\include| command.
This mechanism is beneficial for documents
which span hundreds of pages in order to
make the source file(s) more manageable.
Moreover, compilation can be restricted to
selected child files by means of the |\includeonly| command.
The latter feature can be used to reduce the compilation time while editing
(this was significantly more useful in the earlier days of \LaTeX{})
or to generate a smaller document which is easier to navigate.
Another application of |\includeonly| is to generate
documents consisting of selected parts of the complete document.

However, there are a few drawbacks of the plain |\include| mechanism:
\begin{itemize}
\item
The child files cannot be compiled on their own,
they can only be compiled via the main file.
A naive editing environment
(such as a text editor with an option
to have the current file processed by \LaTeX)
may require one to switch to the main file before compiling;
attempting to compile the child file produces errors.
\item
The main file must be modified (each time)
to adjust the |\includeonly| command
to the present needs. This easily leaves the main file in a messy state.
\item
The generated document will always carry the filename
of the main document. This is inconvenient if
several child files are to be compiled and
to be kept for distribution.
\end{itemize}

The present package provides a simple interface
to make child files individually compilable by \LaTeX{}.
Compiling a child file then has the same effect as compiling
the main file with an |\includeonly| command
to select the appropriate child.
Moreover the generated document will carry the name of the child
rather than the main file.
This resolves all three above issues.

This feature is meant to make the editing of books,
thesis documents and lecture notes somewhat more convenient.
However, the package can also be used efficiently for
composing a series of documents (such as exercise sheets)
which are typically distributed individually.
It then assists the author in generating the individual documents
(potentially in different versions)
as well as a document containing the collected series.
Another application is in developing style files
or other kinds of included material
where compilation of the style file could redirect
to a sample or test file.

%%%%%%%%%%%%%%%%%%%%%%%%%%%%%%%%%%%%%%%%%%%%%%%%%%%%%%%%%%%%%%%%%%%%%%%%%%%%%%%%
%%%%%%%%%%%%%%%%%%%%%%%%%%%%%%%%%%%%%%%%%%%%%%%%%%%%%%%%%%%%%%%%%%%%%%%%%%%%%%%%
\section{Usage}

First of all, the package \textsf{childdoc} is \emph{not} a standard
\LaTeXe{} |.sty| style file! Therefore it needs to be invoked in
a non-standard way.

%%%%%%%%%%%%%%%%%%%%%%%%%%%%%%%%%%%%%%%%%%%%%%%%%%%%%%%%%%%%%%%%%%%%%%%%%%%%%%%%
\subsection{Included Files}
\label{sec:include}

%%%%%%%%%%%%%%%%%%%%%%%%%%%%%%%%%%%%%%%%
\DescribeMacro{\childdocmain}
To use the package, add the commands
\begin{center}
\begin{tabular}{l}
|\input{childdoc.def}|\\
|\childdocmain{}|\\
\end{tabular}
\end{center}
at the very top of the main \LaTeX{} file,
in particular \emph{before} the |\documentclass| statement!
The argument of |\childdocmain| should be left empty
(but it must be present).

%%%%%%%%%%%%%%%%%%%%%%%%%%%%%%%%%%%%%%%%
\DescribeMacro{\childdocof}
Furthermore, add the commands
\begin{center}
\begin{tabular}{l}
|\input{childdoc.def}|\\
|\childdocof{|\textit{main}|}|\\
\end{tabular}
\end{center}
at the top of every child file \textit{child}
which is included by |\include{|\textit{child}|}|
from within the main file
(or at least for those files to be compiled individually).
The argument \textit{main} must be the filename of the main file.

There are a couple of
considerations in setting up the main and child documents:

%%%%%%%%%%%%%%%%%%%%%%%%%%%%%%%%%%%%%%%%
\paragraph{Restrictions.}

Please note the following restrictions:
\begin{itemize}
\item
|\childdocmain| must be called with one argument \textit{main}
to ensure compatibility with earlier version of the package.
It must either be empty (|\childdocmain{}|)
or precisely match the filename of the main file in which it is specified.
See \secref{sec:detection} for further information.
\item
The filename \textit{main} must be specified without the |.tex| extension.
\item
The filename \textit{main} is case sensitive
(even in case-insensitive file systems)
due to internal string comparison.
\item
The argument \textit{main} should be fully expanded, it cannot be a macro.
\item
Subdirectories and special characters should be avoided in filenames.
\item
The command |\childdocmain{|\textit{main}|}| must be followed by a whitespace.
It should not be followed immediately by another command
or by a comment mark `|%|'.
This is because the \TeX{} parser reads the token immediately following
the argument of |\childdocmain| and puts it
at the beginning of every child section;
however, a white\-space is ignored.
\end{itemize}

%%%%%%%%%%%%%%%%%%%%%%%%%%%%%%%%%%%%%%%%
\paragraph{Content of Main File.}

It is advisable to place all content in the child files included by |\include|.
Any output contained in the main file will appear in all child documents
unless suppressed manually;
it cannot be suppressed automatically by the |\includeonly| directive
and thus should normally be avoided.
A method to include some content in the main file
by means of conditional processing is described in \secref{sec:conditional}.

%%%%%%%%%%%%%%%%%%%%%%%%%%%%%%%%%%%%%%%%
\paragraph{Page Numbering.}

When only a part of the document is compiled,
the appropriate numbering of pages
(as well as other status parameters)
is determined from the |.aux| files.
The latter contain information from previous passes.
However this information needs to propagate through
all intermediate child documents.
Therefore the page numbering in child documents may well
be inconsistent until the complete document is compiled at least once.

A useful (if unconventional) way to always ensure a consistent
page numbering is to restart the numbering in each child document
and denote the pages by `\textit{child}|.|\textit{page}'
where \textit{child} represents the chapter/section number of the child file.
This can be achieved by the command
|\numberwithin{page}{|\textit{child}|}|
of the \textsf{amsmath} package
where \textit{child} can be |chapter| or |section|
depending on the chosen structuring.
Alternatively, one can modify the macro |\thepage| appropriately
and reset the counter |page| at the start of each child file.

%%%%%%%%%%%%%%%%%%%%%%%%%%%%%%%%%%%%%%%%%%%%%%%%%%%%%%%%%%%%%%%%%%%%%%%%%%%%%%%%
\subsection{Conditional Processing}
\label{sec:conditional}

The package provides a mechanism to compile different versions
of a document. To customise the versions further some conditional processing
can come in handy to distinguish which version is being compiled.
The package provides two macros to describe the compilation context:

%%%%%%%%%%%%%%%%%%%%%%%%%%%%%%%%%%%%%%%%
\DescribeMacro{\ifchilddoc}
The conditional |\ifchilddoc| distinguishes between the compilation of
child documents and the main document:
%
\begin{center}
|\ifchilddoc |\textit{child-code}| |[|\||else |\textit{main-code}]| \||fi|
\end{center}

%%%%%%%%%%%%%%%%%%%%%%%%%%%%%%%%%%%%%%%%
\DescribeMacro{\childdocname}
\DescribeMacro{\childdocjob}
The macro |\childdocname| contains the filename (without extension)
of the main or child file being processed.
Note that |\childdocjob| will always contain the name of the main file.

%%%%%%%%%%%%%%%%%%%%%%%%%%%%%%%%%%%%%%%%
\paragraph{Title Page.}

Conditional processing can be used to include a title or banner page
in the main document when proper precautions are taken.
Importantly, the code in the main file should ensure that the page counter
(as well as other status parameters which are stored in the |.aux| files)
takes the same value after the conditional processing.
Otherwise the page numbers may take divergent values
depending on which part is compiled.

For example, a title page could be declared by:
%
\begin{center}
\begin{tabular}{l}
|\ifchilddoc\||else|\\
|\addtocounter{page}{-1}|\\
\textit{code for title page}\\
|\newpage|\\
|\||fi|
\end{tabular}
\end{center}
%
A banner page for the child documents can be generated by:
%
\begin{center}
\begin{tabular}{l}
|\ifchilddoc|\\
|\addtocounter{page}{-1}|\\
\textit{code for banner page}\\
|\newpage|\\
|\||fi|
\end{tabular}
\end{center}
%
Here one could write a message such as:
\begin{center}
|This is the part \childdocname{} of \childdocjob{}.|
\end{center}

%%%%%%%%%%%%%%%%%%%%%%%%%%%%%%%%%%%%%%%%%%%%%%%%%%%%%%%%%%%%%%%%%%%%%%%%%%%%%%%%
\subsection{Flags}
\label{sec:flags}

The package makes it easy to generate different versions
of the main or child documents.
To this end compilation flags can be defined
and assigned different default values.
They will be particularly useful in conjunction
with the forwarding mechanism described in \secref{sec:forward}.

For example, it may be useful to have a flag |\version|
which can be set to |draft| or |final|.
The document source will contain some conditional code
depending on the value of |\version|.
Suppose further, the flag should default to |final| for the main file
and to |draft| for child files
which is a natural assignment for editing the document.
This is achieved by placing the following code
in the preamble of the main document
(below the |\childdocmain| directive):
%
\begin{center}
\begin{tabular}{l}
|\ifchilddoc|\\
|\providecommand{\version}{draft}|\\
|\||else|\\
|\providecommand{\version}{final}|\\
|\||fi|
\end{tabular}
\end{center}
%
The definition by |\providecommand| makes sure
that previous definitions are not overwritten.
Further statements |\providecommand{\version}{...}|
can thus be added before the above code to override it.

For the main file, one might add a line
(between |\childdocmain| and the above block)
%
\begin{center}
|%\ifchilddoc\||else\providecommand{\version}{draft}\||fi|
\end{center}
%
which can be uncommented to produce a draft version.
Likewise one can add a line to the very top of a child file
(above the |\childdocof{|\textit{main}|}| directive)
%
\begin{center}
|%\providecommand{\version}{final}|
\end{center}
%
which can be uncommented to produce the final version of this child document.

%%%%%%%%%%%%%%%%%%%%%%%%%%%%%%%%%%%%%%%%%%%%%%%%%%%%%%%%%%%%%%%%%%%%%%%%%%%%%%%%
\subsection{Forwarding}
\label{sec:forward}

Different versions of the main or child documents
using compilation flags as described in \secref{sec:flags}
can be (permanently) stored in different files
for convenient compilation, viewing and distribution.
To this end, the package defines a command
to pass on compilation to a different file:

%%%%%%%%%%%%%%%%%%%%%%%%%%%%%%%%%%%%%%%%
\DescribeMacro{\childdocforward}
The command |\childdocforward| redirects processing to
another source file:
%
\begin{center}
\begin{tabular}{l}
|\input{childdoc.def}|\\
|\childdocforward[|\textit{main}|]{|\textit{dest}|}|\\
\end{tabular}
\end{center}
%
The argument \textit{dest} is the destination file
(without extension).
It should be the main file or one of the child files.
Note that further \textsf{childdoc} directives
such as |\childdocof| and |\childdocforward|
in the indicated file will be processed in this form.
The optional argument \textit{main}
passes on directly to the main file \textit{main}
while pretending to compile the child \textit{dest}.
This form behaves as if \textit{dest}
issues |\childdocof{|\textit{main}|}| right away,
and no further \textsf{childdoc} directives will be processed.

%%%%%%%%%%%%%%%%%%%%%%%%%%%%%%%%%%%%%%%%
\DescribeMacro{\...prefix}
In the alternative form |\childdocforwardprefix|,
%
\begin{center}
\begin{tabular}{l}
|\input{childdoc.def}|\\
|\childdocforwardprefix[|\textit{main}|]{|\textit{prefix}|}{|\textit{dest}|}|
\end{tabular}
\end{center}
%
the destination file is determined by a pattern
depending on the current file:
To make this work, the current file must be called
`{\textit{prefix}\hspace{0.2em}\textit{suffix}}'
with \textit{prefix} matching precisely the argument.
Processing is then passed on to the file
`{\textit{dest}\hspace{0.2em}\textit{suffix}}'.
Surely, the same effect is achieved by
directly specifying the
argument `{\textit{dest}\hspace{0.2em}\textit{suffix}}'
in the first form.
However, that requires to set up a different file
for each child. With the alternative form of the command
all these files can have exactly the same content
which simplifies setting them up and maintaining them.

For example, the following file |draft.tex|
with a compilation flag |\version| as described in \secref{sec:flags}
compiles the main document as a draft:
%
\begin{center}
\begin{tabular}{l}
|\def\version{draft}|\\
|\input{childdoc.def}|\\
|\childdocforward{|\textit{main}|}|
\end{tabular}
\end{center}
%
Likewise, the following files |final|\textit{nn}|.tex|
compile the final version of the child document
|child|\textit{nn}|.tex|:
%
\begin{center}
\begin{tabular}{l}
|\def\version{final}|\\
|\input{childdoc.def}|\\
|\childdocforwardprefix{final}{child}|
\end{tabular}
\end{center}
%

Note that when several versions of a main file and/or of each child file
are to be generated, it may be convenient to set up a |Makefile| or
shell script to automatise the process.

%%%%%%%%%%%%%%%%%%%%%%%%%%%%%%%%%%%%%%%%%%%%%%%%%%%%%%%%%%%%%%%%%%%%%%%%%%%%%%%%
\subsection{Command Line Processing}
\label{sec:commandline}

The effect of redirection files can also be achieved by invoking
the \LaTeX{} compiler with a more elaborate command line.
Most conveniently this should be done as part
of a shell script or a |Makefile|.

When using \textsf{childdoc} in the main file, the following
command lines effectively perform a redirection
(note that depending on the shell being used,
backslashes may have to be doubled: `|\|' $\to$ `|\\|'):
%
\begin{center}
|... -jobname "|\textit{target}|" |\\|"|[\textit{flags}]%
|\input{childdoc.def}\childdocforward[|\textit{main}|]{|\textit{dest}|}"|
\end{center}
%
Here \textit{target} is the name of the output file,
\textit{main} is the name of the main file
and \textit{dest} is the name of the main or child file to be processed
(all filenames without extensions).
The optional argument \textit{main} can be omitted
if \textit{main} matches \textit{dest}.
Optionally, compilation \textit{flags} can be defined via |\def| commands.
This command line makes the \TeX{} engine believe
it is compiling the file \textit{target}
whose content is specified as the latter parameter.
The provided code then forwards the processing to
\textit{main} or \textit{dest} as described in \secref{sec:forward}.

%%%%%%%%%%%%%%%%%%%%%%%%%%%%%%%%%%%%%%%%%%%%%%%%%%%%%%%%%%%%%%%%%%%%%%%%%%%%%%%%
\subsection{Include by Input}
\label{sec:input}

Including child documents by |\include| has some restrictions by design.
Most notably, the content of a child document always occupies
its own set of pages; pages cannot be shared between child documents.
Usually, this behaviour makes perfect sense
because each child document contain an essential part of the document.
However, in some situations it may be desirable to compose
a document from a collection of parts
without having mandatory page breaks between then.
For this case, the package
provides a mechanism to include parts
by |\input| which can also be processed individually.
However, by construction this mechanism
requires manual handling of the content to be output.

%%%%%%%%%%%%%%%%%%%%%%%%%%%%%%%%%%%%%%%%
\DescribeMacro{\ifchilddocmanual}
The main file should be prepared as usual, see \secref{sec:include}.
However, the document body must make a distinction
between processing of an individual part and of the main document, e.g.:
%
\begin{center}
\begin{tabular}{l}
|\ifchilddocmanual|\\
|\input{\childdocname}|\\
|\||else|\\
\textit{document body with }|\input{|\textit{part}|}|\\
|\||fi|
\end{tabular}
\end{center}
%
The conditional |\ifchilddocmanual| is true whenever
a part to be included by |\input| is being compiled,
and the name of the part is stored in |\childdocname|.

%%%%%%%%%%%%%%%%%%%%%%%%%%%%%%%%%%%%%%%%
\DescribeMacro{\childdocby}
Each part to be included by |\input| should start with:
%
\begin{center}
\begin{tabular}{l}
|\input{childdoc.def}|\\
|\childdocby{|\textit{main}|}|\\
\end{tabular}
\end{center}
%
The directive |\childdocby| is similar to |\childdocof|
described in \secref{sec:include},
but the subsequent selection of content must be done manually.
To that end, both |\ifchilddoc| and |\ifchilddocmanual|
will be true upon processing of a part,
and the name of the part is stored in |\childdocname|.
Note that |\jobname| will be set to the filename of the current part
so that each part receives an individual |.aux| file
that does not interfere with the |.aux| file(s) of the main document.
This behaviour can be altered by the alternative form
|\childdocby[*]{|\textit{main}|}| (with a non-empty optional argument)
which uses the |.aux| file of the main document
by setting |\jobname| to \textit{main}.

%%%%%%%%%%%%%%%%%%%%%%%%%%%%%%%%%%%%%%%%%%%%%%%%%%%%%%%%%%%%%%%%%%%%%%%%%%%%%%%%
\subsection{Driver Development}
\label{sec:driver}

The \textsf{childdoc} mechanism can also be use for the development
of definition files such as \LaTeX{} styles or classes.
This case differs from the above setup with multiple parts
included by |\include| in that no |\includeonly| should be invoked.
This can be achieved by starting the include file
(before |\ProvidesPackage|) with:
%
\begin{center}
\begin{tabular}{l}
|\input{childdoc.def}|\\
|\childdocforward{|\textit{main}|}|\\
\end{tabular}
\end{center}
%
or alternatively with:
%
\begin{center}
\begin{tabular}{l}
|\input{childdoc.def}|\\
|\childdocby{|\textit{main}|}|\\
\end{tabular}
\end{center}
%
Both forms have slightly different effects as described above.
The main file is prepared as usual, see \secref{sec:include}.

%%%%%%%%%%%%%%%%%%%%%%%%%%%%%%%%%%%%%%%%%%%%%%%%%%%%%%%%%%%%%%%%%%%%%%%%%%%%%%%%
\subsection{Legacy Detection}
\label{sec:detection}

The directive |\childdocmain| in the main file can detect
whether the complete document or merely a child is to be compiled
even without using the directive |\childdocof|.
This method is deprecated because it is less robust
and there is no compelling reason to use it;
it is merely provided for backward compatibility
and it may be removed in future versions.

If the detection mechanism is to be used,
it is mandatory to correctly specify
the filename of the main file as the argument of |\childdocmain|:
%
\begin{center}
\begin{tabular}{l}
|\input{childdoc.def}|\\
|\childdocmain{|\textit{main}|}|\\
\end{tabular}
\end{center}
%
If |\jobname| does not match the argument \textit{main} of |\childdocmain|,
it is assumed that |\jobname| points to the child file to be compiled.
When using |\childdocmain| with the main file specified as argument,
it suffices to start a child file
with just |\input{|\textit{main}|}|
without loading of the package and using |\childdocof|.
If instead all processing is done
with the appropriate \textsf{childdoc} directives,
the argument of \textit{main} of |\childdocmain| can be empty.

An alternative version of the command line processing described
in \secref{sec:commandline} using the detection mechanism reads:
%
\begin{center}
|... -jobname "|\textit{target}|" "|[\textit{flags}]%
[|\def\jobname{|\textit{dest}|}|]|\input{|\textit{main}|}"|
\end{center}

%%%%%%%%%%%%%%%%%%%%%%%%%%%%%%%%%%%%%%%%%%%%%%%%%%%%%%%%%%%%%%%%%%%%%%%%%%%%%%%%
\subsection{Manual Code}
\label{sec:manual}

In case one cannot be certain whether the definitions file |childdoc.def|
is installed on the target \TeX{} distribution
and one prefers not to ship it,
it is conceivable to paste a few relevant commands into the sources.

To that end, drop all statements |\input{childdoc.def}|
and perform the replacements as outlined below.
Instead of |\childdocmain{|\textit{main}|}| add the following code
to the top of the main file:
%
\begin{center}
\begin{tabular}{l}
|\||ifdefined\childdocname\endinput\||fi\newif\ifchilddoc|\\
|\edef\childdocname{\scantokens\expandafter{\jobname\noexpand}}|\\
|\def\childdocmain{|\textit{main}|}\||ifx\childdocmain\childdocname\||else|\\
|\childdoctrue\includeonly{\childdocname}\let\jobname\childdocmain\||fi|\\
\end{tabular}
\end{center}
%
Instead of |\childdocof{|\textit{main}|}| just include the main file
at the top of each child file:
%
\begin{center}
|\input{|\textit{main}|}|
\end{center}
%
A simple redirection |\childdocforward{|\textit{dest}|}| is achieved by:
%
\begin{center}
|\def\jobname{|\textit{dest}|}\input{\jobname}|
\end{center}
%
The redirection with prefix
|\childdocforwardprefix[|\textit{prefix}|]{|\textit{dest}|}|
is accomplished by:
%
\begin{center}
\begin{tabular}{l}
|{\edef\jobname{\scantokens\expandafter{\jobname\noexpand}}|\\
|\def\redirectjob |\textit{prefix}|#1~~~{\gdef\jobname{|\textit{dest}|#1}}|\\
|\expandafter\redirectjob\jobname~~~}\input{\jobname}|
\end{tabular}
\end{center}

In an alternative approach,
child documents can be compiled by a specific command line
without additional code or specific definitions:
%
\begin{center}
|... -jobname "|\textit{target}|" "|[\textit{flags}]%
|\includeonly{|\textit{dest}|}\input{|\textit{main}|}"|
\end{center}
%

%%%%%%%%%%%%%%%%%%%%%%%%%%%%%%%%%%%%%%%%%%%%%%%%%%%%%%%%%%%%%%%%%%%%%%%%%%%%%%%%
%%%%%%%%%%%%%%%%%%%%%%%%%%%%%%%%%%%%%%%%%%%%%%%%%%%%%%%%%%%%%%%%%%%%%%%%%%%%%%%%
\section{Information}

%%%%%%%%%%%%%%%%%%%%%%%%%%%%%%%%%%%%%%%%%%%%%%%%%%%%%%%%%%%%%%%%%%%%%%%%%%%%%%%%
\subsection{Copyright}

Copyright \copyright{} 2017--2018 Niklas Beisert

This work may be distributed and/or modified under the
conditions of the \LaTeX{} Project Public License, either version 1.3
of this license or (at your option) any later version.
The latest version of this license is in
  \url{http://www.latex-project.org/lppl.txt}
and version 1.3 or later is part of all distributions of \LaTeX{}
version 2005/12/01 or later.

This work has the LPPL maintenance status `maintained'.

The Current Maintainer of this work is Niklas Beisert.

This work consists of the files |README.txt|, |childdoc.ins| and |childdoc.dtx|
as well as the derived files |childdoc.def|, |cdocsamp.tex|
with |cdocsch1.tex|, |cdocsch2.tex|, |cdocspt3.tex|, |cdocspt4.tex|,
|cdocsdrf.tex|, |cdocsfn1.tex|, |cdocsfn2.tex|
as well as |childdoc.pdf|.

%%%%%%%%%%%%%%%%%%%%%%%%%%%%%%%%%%%%%%%%%%%%%%%%%%%%%%%%%%%%%%%%%%%%%%%%%%%%%%%%
\subsection{Files and Installation}

The package consists of the files:
%
\begin{center}
\begin{tabular}{ll}
    |README.txt|   & readme file \\
    |childdoc.ins| & installation file \\
    |childdoc.dtx| & source file \\
    |childdoc.def| & definition file \\
    |cdocsamp.tex| & sample main file \\
    |cdocsch1.tex| & sample include file \\
    |cdocsch2.tex| & sample include file \\
    |cdocspt3.tex| & sample part file \\
    |cdocspt4.tex| & sample part file \\
    |cdocsdrf.tex| & sample redirection file \\
    |cdocsfn1.tex| & sample redirection file \\
    |cdocsfn2.tex| & sample redirection file \\
    |childdoc.pdf| & manual
\end{tabular}
\end{center}
%
The distribution consists of the files
|README.txt|, |childdoc.ins| and |childdoc.dtx|.
%
\begin{itemize}
\item
Run (pdf)\LaTeX{} on |childdoc.dtx|
to compile the manual |childdoc.pdf| (this file).
\item
Run \LaTeX{} on |childdoc.ins| to create the definitions file |childdoc.def|
and the sample |cdocsamp.tex| with include files
|cdocsch1.tex|, |cdocsch2.tex|, |cdocspt3.tex|, |cdocspt4.tex|,
|cdocsdrf.tex|, |cdocsfn1.tex|, |cdocsfn2.tex|.
Then copy the file |childdoc.def| to an appropriate directory of your \LaTeX{}
distribution, e.g.\ \textit{texmf-root}|/tex/latex/childdoc|.
\end{itemize}

%%%%%%%%%%%%%%%%%%%%%%%%%%%%%%%%%%%%%%%%%%%%%%%%%%%%%%%%%%%%%%%%%%%%%%%%%%%%%%%%
\subsection{Related CTAN Packages}

There are several other packages which offer a similar functionality:
%
\begin{itemize}
\item
The packages
\href{http://ctan.org/pkg/docmute}{\textsf{docmute}},
\href{http://ctan.org/pkg/includex}{\textsf{includex}} and
\href{http://ctan.org/pkg/standalone}{\textsf{standalone}}
provide commands to include only the document body of
a child file thus allowing both files to be compiled individually.
\item
The packages \href{http://ctan.org/pkg/subdocs}{\textsf{subdocs}}
and \href{http://ctan.org/pkg/subfiles}{\textsf{subfiles}}
provide structures in which the main and child documents can be
encapsulated and allowing them to be compiled individually.
The inclusion mechanism is different from the conventional |\include|.
\item
The package \href{http://ctan.org/pkg/combine}{\textsf{combine}}
is an elaborate solution to combine several documents into one.
\end{itemize}
%
See also the CTAN topic \href{http://ctan.org/topic/subdocs}{\textsf{subdocs}}
for further related packages.
The present package differs from the above solutions in that
a document structure constructed with the conventional |\include| mechanism
just needs two extra commands at the top of every file
such that all constituent files can be compiled individually.

%%%%%%%%%%%%%%%%%%%%%%%%%%%%%%%%%%%%%%%%%%%%%%%%%%%%%%%%%%%%%%%%%%%%%%%%%%%%%%%%
%\subsection{Feature Suggestions}
%
%The following is a list of features which may be useful for future
%versions of this package:
%%
%\begin{itemize}
%\item
%\ldots
%\end{itemize}

%%%%%%%%%%%%%%%%%%%%%%%%%%%%%%%%%%%%%%%%%%%%%%%%%%%%%%%%%%%%%%%%%%%%%%%%%%%%%%%%
\subsection{Revision History}

%%%%%%%%%%%%%%%%%%%%%%%%%%%%%%%%%%%%%%%%
\paragraph{v2.0:} 2018/12/30

\begin{itemize}
\item
immediate forward processing
\item
added |\childdocby| mechanism
\item
manual restructured
\end{itemize}

%%%%%%%%%%%%%%%%%%%%%%%%%%%%%%%%%%%%%%%%
\paragraph{v1.6:} 2018/01/17

\begin{itemize}
\item
application for development of include files
\item
corrections to manual
\end{itemize}

%%%%%%%%%%%%%%%%%%%%%%%%%%%%%%%%%%%%%%%%
\paragraph{v1.5:} 2017/05/21

\begin{itemize}
\item
more complete structuring introduced
\item
|\childdocof| introduced
\item
|\childdoc| renamed to |\childdocmain|
\item
|\childredirect| renamed to |\childdocforward| and |\childdocforwardprefix|
and functionality expanded
\end{itemize}

%%%%%%%%%%%%%%%%%%%%%%%%%%%%%%%%%%%%%%%%
\paragraph{v1.0:} 2017/04/27

\begin{itemize}
\item
manual and install package
\item
first version published on CTAN
\end{itemize}

%%%%%%%%%%%%%%%%%%%%%%%%%%%%%%%%%%%%%%%%
\paragraph{v0.6:} 2017/04/26

\begin{itemize}
\item
redirection mechanism added
\end{itemize}

%%%%%%%%%%%%%%%%%%%%%%%%%%%%%%%%%%%%%%%%
\paragraph{v0.5:} 2017/04/26

\begin{itemize}
\item
functionality in definition file
\end{itemize}


%%%%%%%%%%%%%%%%%%%%%%%%%%%%%%%%%%%%%%%%%%%%%%%%%%%%%%%%%%%%%%%%%%%%%%%%%%%%%%%%
%%%%%%%%%%%%%%%%%%%%%%%%%%%%%%%%%%%%%%%%%%%%%%%%%%%%%%%%%%%%%%%%%%%%%%%%%%%%%%%%
%%%%%%%%%%%%%%%%%%%%%%%%%%%%%%%%%%%%%%%%%%%%%%%%%%%%%%%%%%%%%%%%%%%%%%%%%%%%%%%%
\appendix

\settowidth\MacroIndent{\rmfamily\scriptsize 000\ }

 \DocInput{childdoc.dtx}

\end{document}
%</driver>
% \fi
%
% %%%%%%%%%%%%%%%%%%%%%%%%%%%%%%%%%%%%%%%%%%%%%%%%%%%%%%%%%%%%%%%%%%%%%%%%%%%%%%
% %%%%%%%%%%%%%%%%%%%%%%%%%%%%%%%%%%%%%%%%%%%%%%%%%%%%%%%%%%%%%%%%%%%%%%%%%%%%%%
% \section{Sample}
%\iffalse
%<*samplemain>
%\fi
%
% The following presents a sample document
% with two chapters, two parts, a title page,
% a compile flag as well as three forwarding files to set the flag.
% It consists of eight |.tex| files:
% \begin{center}
% \begin{tabular}{ll}
% |cdocsamp.tex|&main file\\
% |cdocsch1.tex|&include file for chapter 1\\
% |cdocsch2.tex|&include file for chapter 2\\
% |cdocspt3.tex|&include file for part 3\\
% |cdocspt4.tex|&include file for part 4\\
% |cdocsdrf.tex|&forwarding file for main file in draft mode\\
% |cdocsfi1.tex|&forwarding file for final version of chapter 1\\
% |cdocsfi2.tex|&forwarding file for final version of chapter 2\\
% \end{tabular}
% \end{center}
% Each of the eight files can be compiled directly by the \LaTeX{} compiler.
%
% %%%%%%%%%%%%%%%%%%%%%%%%%%%%%%%%%%%%%%
% \paragraph{Main File.}
%
% The main file is called |cdocsamp.tex|.
%
% Load the \textsf{childdoc} definitions and
% declare the filename for the main document:
%    \begin{macrocode}
\input{childdoc.def}
\childdocmain{}
%    \end{macrocode}

% Optional override for |\version| flag:
%    \begin{macrocode}
%%\ifchilddoc\else\providecommand{\version}{draft}\fi
%    \end{macrocode}

% Define the default values for the |\version| flag
% (|final| for the main file and |draft| for childs):
%    \begin{macrocode}
\ifchilddoc
\providecommand{\version}{draft}
\else
\providecommand{\version}{final}
\fi
%    \end{macrocode}

% Load the standard document class:
%    \begin{macrocode}
\documentclass[12pt]{article}
%    \end{macrocode}

% Start the document body:
%    \begin{macrocode}
\begin{document}
%    \end{macrocode}

% Declare a title page.
% Print title, part of document being processed and version flag:
%    \begin{macrocode}
\addtocounter{page}{-1}
\begin{center}
{\LARGE\bfseries{}childdoc example\par}
\vspace{1cm}
\ifchilddoc
\ifchilddocmanual part\else chapter\fi:
`\childdocname' of `\childdocjob'\par
\else
main document: `\childdocjob'\par
\fi
version: \version\par
\end{center}
\newpage
%    \end{macrocode}

% Manually include selected file,
% otherwise process as usual:
%    \begin{macrocode}
\ifchilddocmanual
\section*{part `\childdocname'}
\input{\childdocname}
\else
%    \end{macrocode}

% Include the two chapters:
%    \begin{macrocode}
\include{cdocsch1}
\include{cdocsch2}
%    \end{macrocode}

% Include the two parts unless only chapters should be displayed:
%    \begin{macrocode}
\ifchilddoc\else
\section{part three}
\input{cdocspt3}
\section{part four}
\input{cdocspt4}
\fi
%    \end{macrocode}

% Process as usual until here:
%    \begin{macrocode}
\fi
%    \end{macrocode}

% End of document body:
%    \begin{macrocode}
\end{document}
%    \end{macrocode}
%\iffalse
%</samplemain>
%\fi
%
% %%%%%%%%%%%%%%%%%%%%%%%%%%%%%%%%%%%%%%
% \paragraph{Chapter Include Files.}
%
% The include files are called |cdocsch1.tex| and |cdocsch2.tex|.
%
%\iffalse
%<*samplechap1|samplechap2>
%\fi

% Optional override for |\version| flag:
%    \begin{macrocode}
%%\providecommand{\version}{final}
%    \end{macrocode}

% Include the main document:
%    \begin{macrocode}
\input{childdoc.def}
\childdocof{cdocsamp}
%    \end{macrocode}

%\iffalse
%</samplechap1|samplechap2>
%\fi
%
%\iffalse
%<*samplechap1>
%\fi
% Some text for chapter 1:
%    \begin{macrocode}
\section{one}
some text in chapter one
%    \end{macrocode}

%\iffalse
%</samplechap1>
%\fi
% Some text for chapter 2:
%\iffalse
%<*samplechap2>
%\fi
%    \begin{macrocode}
\section{two}
more text in chapter two
%    \end{macrocode}

%\iffalse
%</samplechap2>
%\fi
%
% %%%%%%%%%%%%%%%%%%%%%%%%%%%%%%%%%%%%%%
% \paragraph{Part Include Files.}
%
% The include files are called |cdocspt3.tex| and |cdocspt4.tex|.
%
%\iffalse
%<*samplepart3|samplepart4>
%\fi

% Optional override for |\version| flag:
%    \begin{macrocode}
%%\providecommand{\version}{final}
%    \end{macrocode}

% Include the main document:
%    \begin{macrocode}
\input{childdoc.def}
\childdocby{cdocsamp}
%    \end{macrocode}

%\iffalse
%</samplepart3|samplepart4>
%\fi
%
%\iffalse
%<*samplepart3>
%\fi
% Some text for part 3:
%    \begin{macrocode}
some text in part three
%    \end{macrocode}

%\iffalse
%</samplepart3>
%\fi
% Some text for part 4:
%\iffalse
%<*samplepart4>
%\fi
%    \begin{macrocode}
more text in part four
%    \end{macrocode}

%\iffalse
%</samplepart4>
%\fi
%
% %%%%%%%%%%%%%%%%%%%%%%%%%%%%%%%%%%%%%%
% \paragraph{Forwarding for a Complete Draft.}
%
% The following forwarding file |cdocsdrf.tex|
% compiles the main document in draft mode:
%\iffalse
%<*sampledraft>
%\fi
%    \begin{macrocode}
\def\version{draft}
\input{childdoc.def}
\childdocforward{cdocsamp}
%    \end{macrocode}

%\iffalse
%</sampledraft>
%\fi
%
% %%%%%%%%%%%%%%%%%%%%%%%%%%%%%%%%%%%%%%
% \paragraph{Forwarding for Final Version of the Chapters.}
%
% The following forwarding files |cdocsfn1.tex| and |cdocsfn2.tex|
% (with identical content)
% compile the final versions of the child documents
% |cdocsch1.tex| and |cdocsch2.tex|, respectively:
%\iffalse
%<*samplefinal>
%\fi
%    \begin{macrocode}
\def\version{final}
\input{childdoc.def}
\childdocforwardprefix[cdocsamp]{cdocsfn}{cdocsch}
%    \end{macrocode}

%\iffalse
%</samplefinal>
%\fi
%
% %%%%%%%%%%%%%%%%%%%%%%%%%%%%%%%%%%%%%%
% \paragraph{Command Line Processing.}
%
% The following three command lines generate the output files
% |cdocscld|, |cdocscl1| and |cdocscl2|
% which should be identical to
% |cdocsdrf|, |cdocsch1| and |cdocsfn2|, respectively:
% \begin{center}
% \begin{tabular}{l}
% |latex -jobname cdocscld \|\\
% |  "\def\version{draft}\input{childdoc.def}\childdocforward{cdocsamp}"|\\
% |latex -jobname cdocscl1 \|\\
% |  "\input{childdoc.def}\childdocforward[cdocsamp]{cdocsch1}"|\\
% |latex -jobname cdocscl2 \|\\
% |  "\def\version{final}\input{childdoc.def}\childdocforward{cdocsch2}"|
% \end{tabular}
% \end{center}
% Note that the trailing backslash on each first line
% merely continues the input to the second line
% (for convenient cut ant paste).
% Furthermore, the command |latex| can be replaced by any
% of its alternative versions such as |pdflatex|.
%
% %%%%%%%%%%%%%%%%%%%%%%%%%%%%%%%%%%%%%%%%%%%%%%%%%%%%%%%%%%%%%%%%%%%%%%%%%%%%%%
% %%%%%%%%%%%%%%%%%%%%%%%%%%%%%%%%%%%%%%%%%%%%%%%%%%%%%%%%%%%%%%%%%%%%%%%%%%%%%%
% \section{Implementation}
%\iffalse
%<*package>
%\fi
%
% This section describes the definitions file |childdoc.def|.

% The definitions cannot be loaded using |\usepackage| or |\RequirePackage|
% which has a mechanism to prevent loading a style file more than once.
% When loading the definitions by means of |\input|
% multiple instances have to be prevented manually:
%\iffalse
%This code needs to be before the `\ProvidesFile' directive
%which is defined at the beginning of this file.
%Therefore it is also placed there and commented out here.
%</package>
%<*discard>
%\fi
%    \begin{macrocode}
\ifdefined\childdocmain\endinput\fi
%    \end{macrocode}
%\iffalse
%</discard>
%<*package>
%\fi
%
% \macro{\ifchilddoc}
% \macro{\ifchilddocmanual}
% The conditional |\ifchilddoc| tells whether a
% child (true) or main (false) document is being compiled.
% The conditional |\ifchilddocmanual| tells whether
% the |\includeonly| mechanism is used (false) or
% the selection of child files must be performed manually (true).
% The definitions initialise to false:
%    \begin{macrocode}
\newif\ifchilddoc
\newif\ifchilddocmanual
%    \end{macrocode}

% \macro{\childdocname}
% \macro{\childdocjob}
% The macro |\childdocname| stores the name of the main document
% to be compiled. The macro |\childdocjob| stores the name of
% the document on which the \LaTeX{} compiler was originally invoked.
% The content of |\jobname| cannot be compared
% to filenames specified in the source due to different catcodes.
% The following code rescans |\jobname|, stores the result
% in |\childdocname| and saves a copy in |\childdocjob|:
%    \begin{macrocode}
\edef\childdocname{\scantokens\expandafter{\jobname\noexpand}}
\let\childdocjob\childdocname
%    \end{macrocode}

% \macro{\childdocdisable}
% The macro |\childdocdisable| prevents the main file
% from being processed more than once.
% At this stage, the main document command |\childdocmain|
% is assumed to be called once again where it should do nothing.
% Any subsequent call to it should prevent
% a secondary processing of the main document
% It overwrites the forwarding commands
% |\childdocof| and |\childdocforward|
% with empty macros to prevent further inclusions of the main document:
%    \begin{macrocode}
\newcommand{\childdocdisable}
{
  \renewcommand{\childdocmain}[1]{\renewcommand{\childdocmain}[1]{\endinput}}
  \renewcommand{\childdocof}[1]{}
  \renewcommand{\childdocby}[2][]{}
  \renewcommand{\childdocforward}[2][]{}
  \renewcommand{\childdocdisable}{}
}
%    \end{macrocode}

% \macro{\childdocmain}
% The macro |\childdocmain| is to be called at the top of the main file
% with nothing or the main filename (without extension) as argument.
% First, it breaks loops.
% If the argument is not empty and does not match |\childdocname|
% (which is set by the first inclusion of |childdoc.def|),
% |\ifchilddoc| is set to true, |\includeonly| is applied to the child file
% and |\jobname| is set to the main file
% (for proper handling of |.aux| files):
%    \begin{macrocode}
\newcommand{\childdocmain}[1]
{
  \childdocdisable\childdocmain{}
  \if?#1?\else
    \begingroup
      \def\childdoctmp{#1}
      \ifx\childdoctmp\childdocname
        \def\childdoctmp{}
      \else
        \def\childdoctmp
        {
          \childdoctrue
          \includeonly{\childdocname}
          \def\childdocjob{#1}
          \def\jobname{#1}
        }
      \fi
      \expandafter
    \endgroup
    \childdoctmp
  \fi
}
%    \end{macrocode}

% \macro{\childdocof}
% The command |\childdocof| redirects
% compilation to the main file |#1|.
%    \begin{macrocode}
\newcommand{\childdocof}[1]
{
  \childdocdisable
  \childdoctrue
  \includeonly{\childdocname}
  \def\jobname{#1}
  \def\childdocjob{#1}
  \input{#1}
}
%    \end{macrocode}

% \macro{\childdocby}
% The command |\childdocby| ....
%    \begin{macrocode}
\newcommand{\childdocby}[2][]
{
  \childdocdisable
  \childdoctrue
  \childdocmanualtrue
  \if?#1?\else
    \def\jobname{#2}
  \fi
  \def\childdocjob{#2}
  \input{#2}
  \endinput
}
%    \end{macrocode}

% \macro{\childdocforward}
% The command |\childdocforward| redirects
% compilation to the main file or
% (if the optional argument is given) a child file.
% Parameters are set as if the main file
% or a child file starting with |\childdocof| was compiled.
% Then compilation is handed over to the main file:
%    \begin{macrocode}
\newcommand{\childdocforward}[2][]
{
  \begingroup
    \if?#1?
      \def\childdoctmp
      {
        \def\childdocname{#2}
        \def\childdocjob{#2}
        \def\jobname{#2}
        \input{#2}
        \endinput
      }
    \else
      \def\childdoctmp
      {
        \childdocdisable
        \def\childdocname{#2}
        \childdoctrue
        \includeonly{#2}
        \def\childdocjob{#1}
        \def\jobname{#1}
        \input{#1}
        \endinput
      }
    \fi
    \expandafter
  \endgroup
  \childdoctmp
}
%    \end{macrocode}

% \macro{\childdocforwardprefix}
% The command |\childdocforwardprefix| redirects
% compilation to the main or a child file by means of a pattern.
% The prefix |#1| in the current filename is replaced by |#2|
% and the suffix of the current filename is kept
% (it is assumed that the filename does not contain the substring `|~~~|'
% which is used as a delimiter).
% Compilation is handed over to the new file by |\childdocforward|:
%    \begin{macrocode}
\newcommand{\childdocforwardprefix}[3][]
{
  \begingroup
    \def\childdocextract #2##1~~~{\def\childdoctmp{\childdocforward[#1]{#3##1}}}
    \expandafter\childdocextract\childdocname~~~
    \expandafter
  \endgroup
  \childdoctmp
}
%    \end{macrocode}

% \macro{\childdoc}
% The deprecated macro |\childdoc| is a legacy version of |\childdocmain|:
%    \begin{macrocode}
\newcommand{\childdoc}{\childdocmain}
%    \end{macrocode}

% \macro{\childdocredirect}
% The deprecated macro |\childdocredirect| is a legacy version
% of |\childdocforward| and |\childdocforwardprefix|:
%    \begin{macrocode}
\newcommand{\childdocredirect}[2][]
{
  \begingroup
    \if?#1?
      \def\childdoctmp{\childdocforward{#2}}
    \else
      \def\childdoctmp{\childdocforwardprefix{#1}{#2}}
    \fi
    \expandafter
  \endgroup
  \childdoctmp
}
%    \end{macrocode}

%\iffalse
%</package>
%\fi
%
\endinput
|\\
|\childdocforward{|\textit{main}|}|\\
\end{tabular}
\end{center}
%
or alternatively with:
%
\begin{center}
\begin{tabular}{l}
|% \iffalse
%
% childdoc.dtx Copyright (C) 2017-2018 Niklas Beisert
%
% This work may be distributed and/or modified under the
% conditions of the LaTeX Project Public License, either version 1.3
% of this license or (at your option) any later version.
% The latest version of this license is in
%   http://www.latex-project.org/lppl.txt
% and version 1.3 or later is part of all distributions of LaTeX
% version 2005/12/01 or later.
%
% This work has the LPPL maintenance status `maintained'.
%
% The Current Maintainer of this work is Niklas Beisert.
%
% This work consists of the files childdoc.dtx and childdoc.ins
% and the derived files childdoc.def and cdocsamp.tex with
% cdocsch1.tex, cdocsch2.tex, cdocsdrf.tex, cdocsfn1.tex, cdocsfn2.tex.
%
%<package>\ifdefined\childdocmain\endinput\fi
%<package>\ProvidesFile{childdoc.def}[2018/12/30 v2.0 child document driver]
%<samplemain>\ProvidesFile{cdocsamp.tex}[2018/12/30 v2.0 sample for childdoc]
%<*driver>
%\ProvidesFile{childdoc.drv}[2018/12/30 v2.0 childdoc reference manual file]
\PassOptionsToClass{10pt,a4paper}{article}
\documentclass{ltxdoc}

\usepackage[margin=35mm]{geometry}
\usepackage{hyperref}
\usepackage{hyperxmp}
\usepackage[usenames]{color}

\hypersetup{colorlinks=true}
\hypersetup{pdfstartview=FitH}
\hypersetup{pdfpagemode=UseNone}
\hypersetup{pdfsource={}}
\hypersetup{pdflang={en-UK}}
\hypersetup{pdfcopyright={Copyright 2017-2018 Niklas Beisert.
  This work may be distributed and/or modified under the
  conditions of the LaTeX Project Public License, either version 1.3
  of this license or (at your option) any later version.}}
\hypersetup{pdflicenseurl={http://www.latex-project.org/lppl.txt}}
\hypersetup{pdfcontactaddress={ETH Zurich, ITP, HIT K,
  Wolfgang-Pauli-Strasse 27}}
\hypersetup{pdfcontactpostcode={8093}}
\hypersetup{pdfcontactcity={Zurich}}
\hypersetup{pdfcontactcountry={Switzerland}}
\hypersetup{pdfcontactemail={nbeisert@itp.phys.ethz.ch}}
\hypersetup{pdfcontacturl={http://people.phys.ethz.ch/\xmptilde nbeisert/}}

\newcommand{\secref}[1]{\hyperref[#1]{section \ref*{#1}}}

\parskip1ex
\parindent0pt
\let\olditemize\itemize
\def\itemize{\olditemize\parskip0pt}

\begin{document}

\title{The \textsf{childdoc} Package}
\hypersetup{pdftitle={The childdoc Package}}
\author{Niklas Beisert\\[2ex]
  Institut f\"ur Theoretische Physik\\
  Eidgen\"ossische Technische Hochschule Z\"urich\\
  Wolfgang-Pauli-Strasse 27, 8093 Z\"urich, Switzerland\\[1ex]
  \href{mailto:nbeisert@itp.phys.ethz.ch}
  {\texttt{nbeisert@itp.phys.ethz.ch}}}
\hypersetup{pdfauthor={Niklas Beisert}}
\hypersetup{pdfsubject={Manual for the LaTeX2e Package childdoc}}
\date{30 December 2018, \textsf{v2.0}}
\maketitle

\begin{abstract}\noindent
\textsf{childdoc} is a \LaTeXe{} package
that enables the direct compilation
of document sections included by |\include|
to individual files.
\end{abstract}

\begingroup
\parskip0ex
\tableofcontents
\endgroup

%%%%%%%%%%%%%%%%%%%%%%%%%%%%%%%%%%%%%%%%%%%%%%%%%%%%%%%%%%%%%%%%%%%%%%%%%%%%%%%%
%%%%%%%%%%%%%%%%%%%%%%%%%%%%%%%%%%%%%%%%%%%%%%%%%%%%%%%%%%%%%%%%%%%%%%%%%%%%%%%%
\section{Introduction}

\LaTeX{} provides a mechanism to structure a large document (such as a book)
into a main file and several child files (containing the chapters)
using the |\include| command.
This mechanism is beneficial for documents
which span hundreds of pages in order to
make the source file(s) more manageable.
Moreover, compilation can be restricted to
selected child files by means of the |\includeonly| command.
The latter feature can be used to reduce the compilation time while editing
(this was significantly more useful in the earlier days of \LaTeX{})
or to generate a smaller document which is easier to navigate.
Another application of |\includeonly| is to generate
documents consisting of selected parts of the complete document.

However, there are a few drawbacks of the plain |\include| mechanism:
\begin{itemize}
\item
The child files cannot be compiled on their own,
they can only be compiled via the main file.
A naive editing environment
(such as a text editor with an option
to have the current file processed by \LaTeX)
may require one to switch to the main file before compiling;
attempting to compile the child file produces errors.
\item
The main file must be modified (each time)
to adjust the |\includeonly| command
to the present needs. This easily leaves the main file in a messy state.
\item
The generated document will always carry the filename
of the main document. This is inconvenient if
several child files are to be compiled and
to be kept for distribution.
\end{itemize}

The present package provides a simple interface
to make child files individually compilable by \LaTeX{}.
Compiling a child file then has the same effect as compiling
the main file with an |\includeonly| command
to select the appropriate child.
Moreover the generated document will carry the name of the child
rather than the main file.
This resolves all three above issues.

This feature is meant to make the editing of books,
thesis documents and lecture notes somewhat more convenient.
However, the package can also be used efficiently for
composing a series of documents (such as exercise sheets)
which are typically distributed individually.
It then assists the author in generating the individual documents
(potentially in different versions)
as well as a document containing the collected series.
Another application is in developing style files
or other kinds of included material
where compilation of the style file could redirect
to a sample or test file.

%%%%%%%%%%%%%%%%%%%%%%%%%%%%%%%%%%%%%%%%%%%%%%%%%%%%%%%%%%%%%%%%%%%%%%%%%%%%%%%%
%%%%%%%%%%%%%%%%%%%%%%%%%%%%%%%%%%%%%%%%%%%%%%%%%%%%%%%%%%%%%%%%%%%%%%%%%%%%%%%%
\section{Usage}

First of all, the package \textsf{childdoc} is \emph{not} a standard
\LaTeXe{} |.sty| style file! Therefore it needs to be invoked in
a non-standard way.

%%%%%%%%%%%%%%%%%%%%%%%%%%%%%%%%%%%%%%%%%%%%%%%%%%%%%%%%%%%%%%%%%%%%%%%%%%%%%%%%
\subsection{Included Files}
\label{sec:include}

%%%%%%%%%%%%%%%%%%%%%%%%%%%%%%%%%%%%%%%%
\DescribeMacro{\childdocmain}
To use the package, add the commands
\begin{center}
\begin{tabular}{l}
|\input{childdoc.def}|\\
|\childdocmain{}|\\
\end{tabular}
\end{center}
at the very top of the main \LaTeX{} file,
in particular \emph{before} the |\documentclass| statement!
The argument of |\childdocmain| should be left empty
(but it must be present).

%%%%%%%%%%%%%%%%%%%%%%%%%%%%%%%%%%%%%%%%
\DescribeMacro{\childdocof}
Furthermore, add the commands
\begin{center}
\begin{tabular}{l}
|\input{childdoc.def}|\\
|\childdocof{|\textit{main}|}|\\
\end{tabular}
\end{center}
at the top of every child file \textit{child}
which is included by |\include{|\textit{child}|}|
from within the main file
(or at least for those files to be compiled individually).
The argument \textit{main} must be the filename of the main file.

There are a couple of
considerations in setting up the main and child documents:

%%%%%%%%%%%%%%%%%%%%%%%%%%%%%%%%%%%%%%%%
\paragraph{Restrictions.}

Please note the following restrictions:
\begin{itemize}
\item
|\childdocmain| must be called with one argument \textit{main}
to ensure compatibility with earlier version of the package.
It must either be empty (|\childdocmain{}|)
or precisely match the filename of the main file in which it is specified.
See \secref{sec:detection} for further information.
\item
The filename \textit{main} must be specified without the |.tex| extension.
\item
The filename \textit{main} is case sensitive
(even in case-insensitive file systems)
due to internal string comparison.
\item
The argument \textit{main} should be fully expanded, it cannot be a macro.
\item
Subdirectories and special characters should be avoided in filenames.
\item
The command |\childdocmain{|\textit{main}|}| must be followed by a whitespace.
It should not be followed immediately by another command
or by a comment mark `|%|'.
This is because the \TeX{} parser reads the token immediately following
the argument of |\childdocmain| and puts it
at the beginning of every child section;
however, a white\-space is ignored.
\end{itemize}

%%%%%%%%%%%%%%%%%%%%%%%%%%%%%%%%%%%%%%%%
\paragraph{Content of Main File.}

It is advisable to place all content in the child files included by |\include|.
Any output contained in the main file will appear in all child documents
unless suppressed manually;
it cannot be suppressed automatically by the |\includeonly| directive
and thus should normally be avoided.
A method to include some content in the main file
by means of conditional processing is described in \secref{sec:conditional}.

%%%%%%%%%%%%%%%%%%%%%%%%%%%%%%%%%%%%%%%%
\paragraph{Page Numbering.}

When only a part of the document is compiled,
the appropriate numbering of pages
(as well as other status parameters)
is determined from the |.aux| files.
The latter contain information from previous passes.
However this information needs to propagate through
all intermediate child documents.
Therefore the page numbering in child documents may well
be inconsistent until the complete document is compiled at least once.

A useful (if unconventional) way to always ensure a consistent
page numbering is to restart the numbering in each child document
and denote the pages by `\textit{child}|.|\textit{page}'
where \textit{child} represents the chapter/section number of the child file.
This can be achieved by the command
|\numberwithin{page}{|\textit{child}|}|
of the \textsf{amsmath} package
where \textit{child} can be |chapter| or |section|
depending on the chosen structuring.
Alternatively, one can modify the macro |\thepage| appropriately
and reset the counter |page| at the start of each child file.

%%%%%%%%%%%%%%%%%%%%%%%%%%%%%%%%%%%%%%%%%%%%%%%%%%%%%%%%%%%%%%%%%%%%%%%%%%%%%%%%
\subsection{Conditional Processing}
\label{sec:conditional}

The package provides a mechanism to compile different versions
of a document. To customise the versions further some conditional processing
can come in handy to distinguish which version is being compiled.
The package provides two macros to describe the compilation context:

%%%%%%%%%%%%%%%%%%%%%%%%%%%%%%%%%%%%%%%%
\DescribeMacro{\ifchilddoc}
The conditional |\ifchilddoc| distinguishes between the compilation of
child documents and the main document:
%
\begin{center}
|\ifchilddoc |\textit{child-code}| |[|\||else |\textit{main-code}]| \||fi|
\end{center}

%%%%%%%%%%%%%%%%%%%%%%%%%%%%%%%%%%%%%%%%
\DescribeMacro{\childdocname}
\DescribeMacro{\childdocjob}
The macro |\childdocname| contains the filename (without extension)
of the main or child file being processed.
Note that |\childdocjob| will always contain the name of the main file.

%%%%%%%%%%%%%%%%%%%%%%%%%%%%%%%%%%%%%%%%
\paragraph{Title Page.}

Conditional processing can be used to include a title or banner page
in the main document when proper precautions are taken.
Importantly, the code in the main file should ensure that the page counter
(as well as other status parameters which are stored in the |.aux| files)
takes the same value after the conditional processing.
Otherwise the page numbers may take divergent values
depending on which part is compiled.

For example, a title page could be declared by:
%
\begin{center}
\begin{tabular}{l}
|\ifchilddoc\||else|\\
|\addtocounter{page}{-1}|\\
\textit{code for title page}\\
|\newpage|\\
|\||fi|
\end{tabular}
\end{center}
%
A banner page for the child documents can be generated by:
%
\begin{center}
\begin{tabular}{l}
|\ifchilddoc|\\
|\addtocounter{page}{-1}|\\
\textit{code for banner page}\\
|\newpage|\\
|\||fi|
\end{tabular}
\end{center}
%
Here one could write a message such as:
\begin{center}
|This is the part \childdocname{} of \childdocjob{}.|
\end{center}

%%%%%%%%%%%%%%%%%%%%%%%%%%%%%%%%%%%%%%%%%%%%%%%%%%%%%%%%%%%%%%%%%%%%%%%%%%%%%%%%
\subsection{Flags}
\label{sec:flags}

The package makes it easy to generate different versions
of the main or child documents.
To this end compilation flags can be defined
and assigned different default values.
They will be particularly useful in conjunction
with the forwarding mechanism described in \secref{sec:forward}.

For example, it may be useful to have a flag |\version|
which can be set to |draft| or |final|.
The document source will contain some conditional code
depending on the value of |\version|.
Suppose further, the flag should default to |final| for the main file
and to |draft| for child files
which is a natural assignment for editing the document.
This is achieved by placing the following code
in the preamble of the main document
(below the |\childdocmain| directive):
%
\begin{center}
\begin{tabular}{l}
|\ifchilddoc|\\
|\providecommand{\version}{draft}|\\
|\||else|\\
|\providecommand{\version}{final}|\\
|\||fi|
\end{tabular}
\end{center}
%
The definition by |\providecommand| makes sure
that previous definitions are not overwritten.
Further statements |\providecommand{\version}{...}|
can thus be added before the above code to override it.

For the main file, one might add a line
(between |\childdocmain| and the above block)
%
\begin{center}
|%\ifchilddoc\||else\providecommand{\version}{draft}\||fi|
\end{center}
%
which can be uncommented to produce a draft version.
Likewise one can add a line to the very top of a child file
(above the |\childdocof{|\textit{main}|}| directive)
%
\begin{center}
|%\providecommand{\version}{final}|
\end{center}
%
which can be uncommented to produce the final version of this child document.

%%%%%%%%%%%%%%%%%%%%%%%%%%%%%%%%%%%%%%%%%%%%%%%%%%%%%%%%%%%%%%%%%%%%%%%%%%%%%%%%
\subsection{Forwarding}
\label{sec:forward}

Different versions of the main or child documents
using compilation flags as described in \secref{sec:flags}
can be (permanently) stored in different files
for convenient compilation, viewing and distribution.
To this end, the package defines a command
to pass on compilation to a different file:

%%%%%%%%%%%%%%%%%%%%%%%%%%%%%%%%%%%%%%%%
\DescribeMacro{\childdocforward}
The command |\childdocforward| redirects processing to
another source file:
%
\begin{center}
\begin{tabular}{l}
|\input{childdoc.def}|\\
|\childdocforward[|\textit{main}|]{|\textit{dest}|}|\\
\end{tabular}
\end{center}
%
The argument \textit{dest} is the destination file
(without extension).
It should be the main file or one of the child files.
Note that further \textsf{childdoc} directives
such as |\childdocof| and |\childdocforward|
in the indicated file will be processed in this form.
The optional argument \textit{main}
passes on directly to the main file \textit{main}
while pretending to compile the child \textit{dest}.
This form behaves as if \textit{dest}
issues |\childdocof{|\textit{main}|}| right away,
and no further \textsf{childdoc} directives will be processed.

%%%%%%%%%%%%%%%%%%%%%%%%%%%%%%%%%%%%%%%%
\DescribeMacro{\...prefix}
In the alternative form |\childdocforwardprefix|,
%
\begin{center}
\begin{tabular}{l}
|\input{childdoc.def}|\\
|\childdocforwardprefix[|\textit{main}|]{|\textit{prefix}|}{|\textit{dest}|}|
\end{tabular}
\end{center}
%
the destination file is determined by a pattern
depending on the current file:
To make this work, the current file must be called
`{\textit{prefix}\hspace{0.2em}\textit{suffix}}'
with \textit{prefix} matching precisely the argument.
Processing is then passed on to the file
`{\textit{dest}\hspace{0.2em}\textit{suffix}}'.
Surely, the same effect is achieved by
directly specifying the
argument `{\textit{dest}\hspace{0.2em}\textit{suffix}}'
in the first form.
However, that requires to set up a different file
for each child. With the alternative form of the command
all these files can have exactly the same content
which simplifies setting them up and maintaining them.

For example, the following file |draft.tex|
with a compilation flag |\version| as described in \secref{sec:flags}
compiles the main document as a draft:
%
\begin{center}
\begin{tabular}{l}
|\def\version{draft}|\\
|\input{childdoc.def}|\\
|\childdocforward{|\textit{main}|}|
\end{tabular}
\end{center}
%
Likewise, the following files |final|\textit{nn}|.tex|
compile the final version of the child document
|child|\textit{nn}|.tex|:
%
\begin{center}
\begin{tabular}{l}
|\def\version{final}|\\
|\input{childdoc.def}|\\
|\childdocforwardprefix{final}{child}|
\end{tabular}
\end{center}
%

Note that when several versions of a main file and/or of each child file
are to be generated, it may be convenient to set up a |Makefile| or
shell script to automatise the process.

%%%%%%%%%%%%%%%%%%%%%%%%%%%%%%%%%%%%%%%%%%%%%%%%%%%%%%%%%%%%%%%%%%%%%%%%%%%%%%%%
\subsection{Command Line Processing}
\label{sec:commandline}

The effect of redirection files can also be achieved by invoking
the \LaTeX{} compiler with a more elaborate command line.
Most conveniently this should be done as part
of a shell script or a |Makefile|.

When using \textsf{childdoc} in the main file, the following
command lines effectively perform a redirection
(note that depending on the shell being used,
backslashes may have to be doubled: `|\|' $\to$ `|\\|'):
%
\begin{center}
|... -jobname "|\textit{target}|" |\\|"|[\textit{flags}]%
|\input{childdoc.def}\childdocforward[|\textit{main}|]{|\textit{dest}|}"|
\end{center}
%
Here \textit{target} is the name of the output file,
\textit{main} is the name of the main file
and \textit{dest} is the name of the main or child file to be processed
(all filenames without extensions).
The optional argument \textit{main} can be omitted
if \textit{main} matches \textit{dest}.
Optionally, compilation \textit{flags} can be defined via |\def| commands.
This command line makes the \TeX{} engine believe
it is compiling the file \textit{target}
whose content is specified as the latter parameter.
The provided code then forwards the processing to
\textit{main} or \textit{dest} as described in \secref{sec:forward}.

%%%%%%%%%%%%%%%%%%%%%%%%%%%%%%%%%%%%%%%%%%%%%%%%%%%%%%%%%%%%%%%%%%%%%%%%%%%%%%%%
\subsection{Include by Input}
\label{sec:input}

Including child documents by |\include| has some restrictions by design.
Most notably, the content of a child document always occupies
its own set of pages; pages cannot be shared between child documents.
Usually, this behaviour makes perfect sense
because each child document contain an essential part of the document.
However, in some situations it may be desirable to compose
a document from a collection of parts
without having mandatory page breaks between then.
For this case, the package
provides a mechanism to include parts
by |\input| which can also be processed individually.
However, by construction this mechanism
requires manual handling of the content to be output.

%%%%%%%%%%%%%%%%%%%%%%%%%%%%%%%%%%%%%%%%
\DescribeMacro{\ifchilddocmanual}
The main file should be prepared as usual, see \secref{sec:include}.
However, the document body must make a distinction
between processing of an individual part and of the main document, e.g.:
%
\begin{center}
\begin{tabular}{l}
|\ifchilddocmanual|\\
|\input{\childdocname}|\\
|\||else|\\
\textit{document body with }|\input{|\textit{part}|}|\\
|\||fi|
\end{tabular}
\end{center}
%
The conditional |\ifchilddocmanual| is true whenever
a part to be included by |\input| is being compiled,
and the name of the part is stored in |\childdocname|.

%%%%%%%%%%%%%%%%%%%%%%%%%%%%%%%%%%%%%%%%
\DescribeMacro{\childdocby}
Each part to be included by |\input| should start with:
%
\begin{center}
\begin{tabular}{l}
|\input{childdoc.def}|\\
|\childdocby{|\textit{main}|}|\\
\end{tabular}
\end{center}
%
The directive |\childdocby| is similar to |\childdocof|
described in \secref{sec:include},
but the subsequent selection of content must be done manually.
To that end, both |\ifchilddoc| and |\ifchilddocmanual|
will be true upon processing of a part,
and the name of the part is stored in |\childdocname|.
Note that |\jobname| will be set to the filename of the current part
so that each part receives an individual |.aux| file
that does not interfere with the |.aux| file(s) of the main document.
This behaviour can be altered by the alternative form
|\childdocby[*]{|\textit{main}|}| (with a non-empty optional argument)
which uses the |.aux| file of the main document
by setting |\jobname| to \textit{main}.

%%%%%%%%%%%%%%%%%%%%%%%%%%%%%%%%%%%%%%%%%%%%%%%%%%%%%%%%%%%%%%%%%%%%%%%%%%%%%%%%
\subsection{Driver Development}
\label{sec:driver}

The \textsf{childdoc} mechanism can also be use for the development
of definition files such as \LaTeX{} styles or classes.
This case differs from the above setup with multiple parts
included by |\include| in that no |\includeonly| should be invoked.
This can be achieved by starting the include file
(before |\ProvidesPackage|) with:
%
\begin{center}
\begin{tabular}{l}
|\input{childdoc.def}|\\
|\childdocforward{|\textit{main}|}|\\
\end{tabular}
\end{center}
%
or alternatively with:
%
\begin{center}
\begin{tabular}{l}
|\input{childdoc.def}|\\
|\childdocby{|\textit{main}|}|\\
\end{tabular}
\end{center}
%
Both forms have slightly different effects as described above.
The main file is prepared as usual, see \secref{sec:include}.

%%%%%%%%%%%%%%%%%%%%%%%%%%%%%%%%%%%%%%%%%%%%%%%%%%%%%%%%%%%%%%%%%%%%%%%%%%%%%%%%
\subsection{Legacy Detection}
\label{sec:detection}

The directive |\childdocmain| in the main file can detect
whether the complete document or merely a child is to be compiled
even without using the directive |\childdocof|.
This method is deprecated because it is less robust
and there is no compelling reason to use it;
it is merely provided for backward compatibility
and it may be removed in future versions.

If the detection mechanism is to be used,
it is mandatory to correctly specify
the filename of the main file as the argument of |\childdocmain|:
%
\begin{center}
\begin{tabular}{l}
|\input{childdoc.def}|\\
|\childdocmain{|\textit{main}|}|\\
\end{tabular}
\end{center}
%
If |\jobname| does not match the argument \textit{main} of |\childdocmain|,
it is assumed that |\jobname| points to the child file to be compiled.
When using |\childdocmain| with the main file specified as argument,
it suffices to start a child file
with just |\input{|\textit{main}|}|
without loading of the package and using |\childdocof|.
If instead all processing is done
with the appropriate \textsf{childdoc} directives,
the argument of \textit{main} of |\childdocmain| can be empty.

An alternative version of the command line processing described
in \secref{sec:commandline} using the detection mechanism reads:
%
\begin{center}
|... -jobname "|\textit{target}|" "|[\textit{flags}]%
[|\def\jobname{|\textit{dest}|}|]|\input{|\textit{main}|}"|
\end{center}

%%%%%%%%%%%%%%%%%%%%%%%%%%%%%%%%%%%%%%%%%%%%%%%%%%%%%%%%%%%%%%%%%%%%%%%%%%%%%%%%
\subsection{Manual Code}
\label{sec:manual}

In case one cannot be certain whether the definitions file |childdoc.def|
is installed on the target \TeX{} distribution
and one prefers not to ship it,
it is conceivable to paste a few relevant commands into the sources.

To that end, drop all statements |\input{childdoc.def}|
and perform the replacements as outlined below.
Instead of |\childdocmain{|\textit{main}|}| add the following code
to the top of the main file:
%
\begin{center}
\begin{tabular}{l}
|\||ifdefined\childdocname\endinput\||fi\newif\ifchilddoc|\\
|\edef\childdocname{\scantokens\expandafter{\jobname\noexpand}}|\\
|\def\childdocmain{|\textit{main}|}\||ifx\childdocmain\childdocname\||else|\\
|\childdoctrue\includeonly{\childdocname}\let\jobname\childdocmain\||fi|\\
\end{tabular}
\end{center}
%
Instead of |\childdocof{|\textit{main}|}| just include the main file
at the top of each child file:
%
\begin{center}
|\input{|\textit{main}|}|
\end{center}
%
A simple redirection |\childdocforward{|\textit{dest}|}| is achieved by:
%
\begin{center}
|\def\jobname{|\textit{dest}|}\input{\jobname}|
\end{center}
%
The redirection with prefix
|\childdocforwardprefix[|\textit{prefix}|]{|\textit{dest}|}|
is accomplished by:
%
\begin{center}
\begin{tabular}{l}
|{\edef\jobname{\scantokens\expandafter{\jobname\noexpand}}|\\
|\def\redirectjob |\textit{prefix}|#1~~~{\gdef\jobname{|\textit{dest}|#1}}|\\
|\expandafter\redirectjob\jobname~~~}\input{\jobname}|
\end{tabular}
\end{center}

In an alternative approach,
child documents can be compiled by a specific command line
without additional code or specific definitions:
%
\begin{center}
|... -jobname "|\textit{target}|" "|[\textit{flags}]%
|\includeonly{|\textit{dest}|}\input{|\textit{main}|}"|
\end{center}
%

%%%%%%%%%%%%%%%%%%%%%%%%%%%%%%%%%%%%%%%%%%%%%%%%%%%%%%%%%%%%%%%%%%%%%%%%%%%%%%%%
%%%%%%%%%%%%%%%%%%%%%%%%%%%%%%%%%%%%%%%%%%%%%%%%%%%%%%%%%%%%%%%%%%%%%%%%%%%%%%%%
\section{Information}

%%%%%%%%%%%%%%%%%%%%%%%%%%%%%%%%%%%%%%%%%%%%%%%%%%%%%%%%%%%%%%%%%%%%%%%%%%%%%%%%
\subsection{Copyright}

Copyright \copyright{} 2017--2018 Niklas Beisert

This work may be distributed and/or modified under the
conditions of the \LaTeX{} Project Public License, either version 1.3
of this license or (at your option) any later version.
The latest version of this license is in
  \url{http://www.latex-project.org/lppl.txt}
and version 1.3 or later is part of all distributions of \LaTeX{}
version 2005/12/01 or later.

This work has the LPPL maintenance status `maintained'.

The Current Maintainer of this work is Niklas Beisert.

This work consists of the files |README.txt|, |childdoc.ins| and |childdoc.dtx|
as well as the derived files |childdoc.def|, |cdocsamp.tex|
with |cdocsch1.tex|, |cdocsch2.tex|, |cdocspt3.tex|, |cdocspt4.tex|,
|cdocsdrf.tex|, |cdocsfn1.tex|, |cdocsfn2.tex|
as well as |childdoc.pdf|.

%%%%%%%%%%%%%%%%%%%%%%%%%%%%%%%%%%%%%%%%%%%%%%%%%%%%%%%%%%%%%%%%%%%%%%%%%%%%%%%%
\subsection{Files and Installation}

The package consists of the files:
%
\begin{center}
\begin{tabular}{ll}
    |README.txt|   & readme file \\
    |childdoc.ins| & installation file \\
    |childdoc.dtx| & source file \\
    |childdoc.def| & definition file \\
    |cdocsamp.tex| & sample main file \\
    |cdocsch1.tex| & sample include file \\
    |cdocsch2.tex| & sample include file \\
    |cdocspt3.tex| & sample part file \\
    |cdocspt4.tex| & sample part file \\
    |cdocsdrf.tex| & sample redirection file \\
    |cdocsfn1.tex| & sample redirection file \\
    |cdocsfn2.tex| & sample redirection file \\
    |childdoc.pdf| & manual
\end{tabular}
\end{center}
%
The distribution consists of the files
|README.txt|, |childdoc.ins| and |childdoc.dtx|.
%
\begin{itemize}
\item
Run (pdf)\LaTeX{} on |childdoc.dtx|
to compile the manual |childdoc.pdf| (this file).
\item
Run \LaTeX{} on |childdoc.ins| to create the definitions file |childdoc.def|
and the sample |cdocsamp.tex| with include files
|cdocsch1.tex|, |cdocsch2.tex|, |cdocspt3.tex|, |cdocspt4.tex|,
|cdocsdrf.tex|, |cdocsfn1.tex|, |cdocsfn2.tex|.
Then copy the file |childdoc.def| to an appropriate directory of your \LaTeX{}
distribution, e.g.\ \textit{texmf-root}|/tex/latex/childdoc|.
\end{itemize}

%%%%%%%%%%%%%%%%%%%%%%%%%%%%%%%%%%%%%%%%%%%%%%%%%%%%%%%%%%%%%%%%%%%%%%%%%%%%%%%%
\subsection{Related CTAN Packages}

There are several other packages which offer a similar functionality:
%
\begin{itemize}
\item
The packages
\href{http://ctan.org/pkg/docmute}{\textsf{docmute}},
\href{http://ctan.org/pkg/includex}{\textsf{includex}} and
\href{http://ctan.org/pkg/standalone}{\textsf{standalone}}
provide commands to include only the document body of
a child file thus allowing both files to be compiled individually.
\item
The packages \href{http://ctan.org/pkg/subdocs}{\textsf{subdocs}}
and \href{http://ctan.org/pkg/subfiles}{\textsf{subfiles}}
provide structures in which the main and child documents can be
encapsulated and allowing them to be compiled individually.
The inclusion mechanism is different from the conventional |\include|.
\item
The package \href{http://ctan.org/pkg/combine}{\textsf{combine}}
is an elaborate solution to combine several documents into one.
\end{itemize}
%
See also the CTAN topic \href{http://ctan.org/topic/subdocs}{\textsf{subdocs}}
for further related packages.
The present package differs from the above solutions in that
a document structure constructed with the conventional |\include| mechanism
just needs two extra commands at the top of every file
such that all constituent files can be compiled individually.

%%%%%%%%%%%%%%%%%%%%%%%%%%%%%%%%%%%%%%%%%%%%%%%%%%%%%%%%%%%%%%%%%%%%%%%%%%%%%%%%
%\subsection{Feature Suggestions}
%
%The following is a list of features which may be useful for future
%versions of this package:
%%
%\begin{itemize}
%\item
%\ldots
%\end{itemize}

%%%%%%%%%%%%%%%%%%%%%%%%%%%%%%%%%%%%%%%%%%%%%%%%%%%%%%%%%%%%%%%%%%%%%%%%%%%%%%%%
\subsection{Revision History}

%%%%%%%%%%%%%%%%%%%%%%%%%%%%%%%%%%%%%%%%
\paragraph{v2.0:} 2018/12/30

\begin{itemize}
\item
immediate forward processing
\item
added |\childdocby| mechanism
\item
manual restructured
\end{itemize}

%%%%%%%%%%%%%%%%%%%%%%%%%%%%%%%%%%%%%%%%
\paragraph{v1.6:} 2018/01/17

\begin{itemize}
\item
application for development of include files
\item
corrections to manual
\end{itemize}

%%%%%%%%%%%%%%%%%%%%%%%%%%%%%%%%%%%%%%%%
\paragraph{v1.5:} 2017/05/21

\begin{itemize}
\item
more complete structuring introduced
\item
|\childdocof| introduced
\item
|\childdoc| renamed to |\childdocmain|
\item
|\childredirect| renamed to |\childdocforward| and |\childdocforwardprefix|
and functionality expanded
\end{itemize}

%%%%%%%%%%%%%%%%%%%%%%%%%%%%%%%%%%%%%%%%
\paragraph{v1.0:} 2017/04/27

\begin{itemize}
\item
manual and install package
\item
first version published on CTAN
\end{itemize}

%%%%%%%%%%%%%%%%%%%%%%%%%%%%%%%%%%%%%%%%
\paragraph{v0.6:} 2017/04/26

\begin{itemize}
\item
redirection mechanism added
\end{itemize}

%%%%%%%%%%%%%%%%%%%%%%%%%%%%%%%%%%%%%%%%
\paragraph{v0.5:} 2017/04/26

\begin{itemize}
\item
functionality in definition file
\end{itemize}


%%%%%%%%%%%%%%%%%%%%%%%%%%%%%%%%%%%%%%%%%%%%%%%%%%%%%%%%%%%%%%%%%%%%%%%%%%%%%%%%
%%%%%%%%%%%%%%%%%%%%%%%%%%%%%%%%%%%%%%%%%%%%%%%%%%%%%%%%%%%%%%%%%%%%%%%%%%%%%%%%
%%%%%%%%%%%%%%%%%%%%%%%%%%%%%%%%%%%%%%%%%%%%%%%%%%%%%%%%%%%%%%%%%%%%%%%%%%%%%%%%
\appendix

\settowidth\MacroIndent{\rmfamily\scriptsize 000\ }

 \DocInput{childdoc.dtx}

\end{document}
%</driver>
% \fi
%
% %%%%%%%%%%%%%%%%%%%%%%%%%%%%%%%%%%%%%%%%%%%%%%%%%%%%%%%%%%%%%%%%%%%%%%%%%%%%%%
% %%%%%%%%%%%%%%%%%%%%%%%%%%%%%%%%%%%%%%%%%%%%%%%%%%%%%%%%%%%%%%%%%%%%%%%%%%%%%%
% \section{Sample}
%\iffalse
%<*samplemain>
%\fi
%
% The following presents a sample document
% with two chapters, two parts, a title page,
% a compile flag as well as three forwarding files to set the flag.
% It consists of eight |.tex| files:
% \begin{center}
% \begin{tabular}{ll}
% |cdocsamp.tex|&main file\\
% |cdocsch1.tex|&include file for chapter 1\\
% |cdocsch2.tex|&include file for chapter 2\\
% |cdocspt3.tex|&include file for part 3\\
% |cdocspt4.tex|&include file for part 4\\
% |cdocsdrf.tex|&forwarding file for main file in draft mode\\
% |cdocsfi1.tex|&forwarding file for final version of chapter 1\\
% |cdocsfi2.tex|&forwarding file for final version of chapter 2\\
% \end{tabular}
% \end{center}
% Each of the eight files can be compiled directly by the \LaTeX{} compiler.
%
% %%%%%%%%%%%%%%%%%%%%%%%%%%%%%%%%%%%%%%
% \paragraph{Main File.}
%
% The main file is called |cdocsamp.tex|.
%
% Load the \textsf{childdoc} definitions and
% declare the filename for the main document:
%    \begin{macrocode}
\input{childdoc.def}
\childdocmain{}
%    \end{macrocode}

% Optional override for |\version| flag:
%    \begin{macrocode}
%%\ifchilddoc\else\providecommand{\version}{draft}\fi
%    \end{macrocode}

% Define the default values for the |\version| flag
% (|final| for the main file and |draft| for childs):
%    \begin{macrocode}
\ifchilddoc
\providecommand{\version}{draft}
\else
\providecommand{\version}{final}
\fi
%    \end{macrocode}

% Load the standard document class:
%    \begin{macrocode}
\documentclass[12pt]{article}
%    \end{macrocode}

% Start the document body:
%    \begin{macrocode}
\begin{document}
%    \end{macrocode}

% Declare a title page.
% Print title, part of document being processed and version flag:
%    \begin{macrocode}
\addtocounter{page}{-1}
\begin{center}
{\LARGE\bfseries{}childdoc example\par}
\vspace{1cm}
\ifchilddoc
\ifchilddocmanual part\else chapter\fi:
`\childdocname' of `\childdocjob'\par
\else
main document: `\childdocjob'\par
\fi
version: \version\par
\end{center}
\newpage
%    \end{macrocode}

% Manually include selected file,
% otherwise process as usual:
%    \begin{macrocode}
\ifchilddocmanual
\section*{part `\childdocname'}
\input{\childdocname}
\else
%    \end{macrocode}

% Include the two chapters:
%    \begin{macrocode}
\include{cdocsch1}
\include{cdocsch2}
%    \end{macrocode}

% Include the two parts unless only chapters should be displayed:
%    \begin{macrocode}
\ifchilddoc\else
\section{part three}
\input{cdocspt3}
\section{part four}
\input{cdocspt4}
\fi
%    \end{macrocode}

% Process as usual until here:
%    \begin{macrocode}
\fi
%    \end{macrocode}

% End of document body:
%    \begin{macrocode}
\end{document}
%    \end{macrocode}
%\iffalse
%</samplemain>
%\fi
%
% %%%%%%%%%%%%%%%%%%%%%%%%%%%%%%%%%%%%%%
% \paragraph{Chapter Include Files.}
%
% The include files are called |cdocsch1.tex| and |cdocsch2.tex|.
%
%\iffalse
%<*samplechap1|samplechap2>
%\fi

% Optional override for |\version| flag:
%    \begin{macrocode}
%%\providecommand{\version}{final}
%    \end{macrocode}

% Include the main document:
%    \begin{macrocode}
\input{childdoc.def}
\childdocof{cdocsamp}
%    \end{macrocode}

%\iffalse
%</samplechap1|samplechap2>
%\fi
%
%\iffalse
%<*samplechap1>
%\fi
% Some text for chapter 1:
%    \begin{macrocode}
\section{one}
some text in chapter one
%    \end{macrocode}

%\iffalse
%</samplechap1>
%\fi
% Some text for chapter 2:
%\iffalse
%<*samplechap2>
%\fi
%    \begin{macrocode}
\section{two}
more text in chapter two
%    \end{macrocode}

%\iffalse
%</samplechap2>
%\fi
%
% %%%%%%%%%%%%%%%%%%%%%%%%%%%%%%%%%%%%%%
% \paragraph{Part Include Files.}
%
% The include files are called |cdocspt3.tex| and |cdocspt4.tex|.
%
%\iffalse
%<*samplepart3|samplepart4>
%\fi

% Optional override for |\version| flag:
%    \begin{macrocode}
%%\providecommand{\version}{final}
%    \end{macrocode}

% Include the main document:
%    \begin{macrocode}
\input{childdoc.def}
\childdocby{cdocsamp}
%    \end{macrocode}

%\iffalse
%</samplepart3|samplepart4>
%\fi
%
%\iffalse
%<*samplepart3>
%\fi
% Some text for part 3:
%    \begin{macrocode}
some text in part three
%    \end{macrocode}

%\iffalse
%</samplepart3>
%\fi
% Some text for part 4:
%\iffalse
%<*samplepart4>
%\fi
%    \begin{macrocode}
more text in part four
%    \end{macrocode}

%\iffalse
%</samplepart4>
%\fi
%
% %%%%%%%%%%%%%%%%%%%%%%%%%%%%%%%%%%%%%%
% \paragraph{Forwarding for a Complete Draft.}
%
% The following forwarding file |cdocsdrf.tex|
% compiles the main document in draft mode:
%\iffalse
%<*sampledraft>
%\fi
%    \begin{macrocode}
\def\version{draft}
\input{childdoc.def}
\childdocforward{cdocsamp}
%    \end{macrocode}

%\iffalse
%</sampledraft>
%\fi
%
% %%%%%%%%%%%%%%%%%%%%%%%%%%%%%%%%%%%%%%
% \paragraph{Forwarding for Final Version of the Chapters.}
%
% The following forwarding files |cdocsfn1.tex| and |cdocsfn2.tex|
% (with identical content)
% compile the final versions of the child documents
% |cdocsch1.tex| and |cdocsch2.tex|, respectively:
%\iffalse
%<*samplefinal>
%\fi
%    \begin{macrocode}
\def\version{final}
\input{childdoc.def}
\childdocforwardprefix[cdocsamp]{cdocsfn}{cdocsch}
%    \end{macrocode}

%\iffalse
%</samplefinal>
%\fi
%
% %%%%%%%%%%%%%%%%%%%%%%%%%%%%%%%%%%%%%%
% \paragraph{Command Line Processing.}
%
% The following three command lines generate the output files
% |cdocscld|, |cdocscl1| and |cdocscl2|
% which should be identical to
% |cdocsdrf|, |cdocsch1| and |cdocsfn2|, respectively:
% \begin{center}
% \begin{tabular}{l}
% |latex -jobname cdocscld \|\\
% |  "\def\version{draft}\input{childdoc.def}\childdocforward{cdocsamp}"|\\
% |latex -jobname cdocscl1 \|\\
% |  "\input{childdoc.def}\childdocforward[cdocsamp]{cdocsch1}"|\\
% |latex -jobname cdocscl2 \|\\
% |  "\def\version{final}\input{childdoc.def}\childdocforward{cdocsch2}"|
% \end{tabular}
% \end{center}
% Note that the trailing backslash on each first line
% merely continues the input to the second line
% (for convenient cut ant paste).
% Furthermore, the command |latex| can be replaced by any
% of its alternative versions such as |pdflatex|.
%
% %%%%%%%%%%%%%%%%%%%%%%%%%%%%%%%%%%%%%%%%%%%%%%%%%%%%%%%%%%%%%%%%%%%%%%%%%%%%%%
% %%%%%%%%%%%%%%%%%%%%%%%%%%%%%%%%%%%%%%%%%%%%%%%%%%%%%%%%%%%%%%%%%%%%%%%%%%%%%%
% \section{Implementation}
%\iffalse
%<*package>
%\fi
%
% This section describes the definitions file |childdoc.def|.

% The definitions cannot be loaded using |\usepackage| or |\RequirePackage|
% which has a mechanism to prevent loading a style file more than once.
% When loading the definitions by means of |\input|
% multiple instances have to be prevented manually:
%\iffalse
%This code needs to be before the `\ProvidesFile' directive
%which is defined at the beginning of this file.
%Therefore it is also placed there and commented out here.
%</package>
%<*discard>
%\fi
%    \begin{macrocode}
\ifdefined\childdocmain\endinput\fi
%    \end{macrocode}
%\iffalse
%</discard>
%<*package>
%\fi
%
% \macro{\ifchilddoc}
% \macro{\ifchilddocmanual}
% The conditional |\ifchilddoc| tells whether a
% child (true) or main (false) document is being compiled.
% The conditional |\ifchilddocmanual| tells whether
% the |\includeonly| mechanism is used (false) or
% the selection of child files must be performed manually (true).
% The definitions initialise to false:
%    \begin{macrocode}
\newif\ifchilddoc
\newif\ifchilddocmanual
%    \end{macrocode}

% \macro{\childdocname}
% \macro{\childdocjob}
% The macro |\childdocname| stores the name of the main document
% to be compiled. The macro |\childdocjob| stores the name of
% the document on which the \LaTeX{} compiler was originally invoked.
% The content of |\jobname| cannot be compared
% to filenames specified in the source due to different catcodes.
% The following code rescans |\jobname|, stores the result
% in |\childdocname| and saves a copy in |\childdocjob|:
%    \begin{macrocode}
\edef\childdocname{\scantokens\expandafter{\jobname\noexpand}}
\let\childdocjob\childdocname
%    \end{macrocode}

% \macro{\childdocdisable}
% The macro |\childdocdisable| prevents the main file
% from being processed more than once.
% At this stage, the main document command |\childdocmain|
% is assumed to be called once again where it should do nothing.
% Any subsequent call to it should prevent
% a secondary processing of the main document
% It overwrites the forwarding commands
% |\childdocof| and |\childdocforward|
% with empty macros to prevent further inclusions of the main document:
%    \begin{macrocode}
\newcommand{\childdocdisable}
{
  \renewcommand{\childdocmain}[1]{\renewcommand{\childdocmain}[1]{\endinput}}
  \renewcommand{\childdocof}[1]{}
  \renewcommand{\childdocby}[2][]{}
  \renewcommand{\childdocforward}[2][]{}
  \renewcommand{\childdocdisable}{}
}
%    \end{macrocode}

% \macro{\childdocmain}
% The macro |\childdocmain| is to be called at the top of the main file
% with nothing or the main filename (without extension) as argument.
% First, it breaks loops.
% If the argument is not empty and does not match |\childdocname|
% (which is set by the first inclusion of |childdoc.def|),
% |\ifchilddoc| is set to true, |\includeonly| is applied to the child file
% and |\jobname| is set to the main file
% (for proper handling of |.aux| files):
%    \begin{macrocode}
\newcommand{\childdocmain}[1]
{
  \childdocdisable\childdocmain{}
  \if?#1?\else
    \begingroup
      \def\childdoctmp{#1}
      \ifx\childdoctmp\childdocname
        \def\childdoctmp{}
      \else
        \def\childdoctmp
        {
          \childdoctrue
          \includeonly{\childdocname}
          \def\childdocjob{#1}
          \def\jobname{#1}
        }
      \fi
      \expandafter
    \endgroup
    \childdoctmp
  \fi
}
%    \end{macrocode}

% \macro{\childdocof}
% The command |\childdocof| redirects
% compilation to the main file |#1|.
%    \begin{macrocode}
\newcommand{\childdocof}[1]
{
  \childdocdisable
  \childdoctrue
  \includeonly{\childdocname}
  \def\jobname{#1}
  \def\childdocjob{#1}
  \input{#1}
}
%    \end{macrocode}

% \macro{\childdocby}
% The command |\childdocby| ....
%    \begin{macrocode}
\newcommand{\childdocby}[2][]
{
  \childdocdisable
  \childdoctrue
  \childdocmanualtrue
  \if?#1?\else
    \def\jobname{#2}
  \fi
  \def\childdocjob{#2}
  \input{#2}
  \endinput
}
%    \end{macrocode}

% \macro{\childdocforward}
% The command |\childdocforward| redirects
% compilation to the main file or
% (if the optional argument is given) a child file.
% Parameters are set as if the main file
% or a child file starting with |\childdocof| was compiled.
% Then compilation is handed over to the main file:
%    \begin{macrocode}
\newcommand{\childdocforward}[2][]
{
  \begingroup
    \if?#1?
      \def\childdoctmp
      {
        \def\childdocname{#2}
        \def\childdocjob{#2}
        \def\jobname{#2}
        \input{#2}
        \endinput
      }
    \else
      \def\childdoctmp
      {
        \childdocdisable
        \def\childdocname{#2}
        \childdoctrue
        \includeonly{#2}
        \def\childdocjob{#1}
        \def\jobname{#1}
        \input{#1}
        \endinput
      }
    \fi
    \expandafter
  \endgroup
  \childdoctmp
}
%    \end{macrocode}

% \macro{\childdocforwardprefix}
% The command |\childdocforwardprefix| redirects
% compilation to the main or a child file by means of a pattern.
% The prefix |#1| in the current filename is replaced by |#2|
% and the suffix of the current filename is kept
% (it is assumed that the filename does not contain the substring `|~~~|'
% which is used as a delimiter).
% Compilation is handed over to the new file by |\childdocforward|:
%    \begin{macrocode}
\newcommand{\childdocforwardprefix}[3][]
{
  \begingroup
    \def\childdocextract #2##1~~~{\def\childdoctmp{\childdocforward[#1]{#3##1}}}
    \expandafter\childdocextract\childdocname~~~
    \expandafter
  \endgroup
  \childdoctmp
}
%    \end{macrocode}

% \macro{\childdoc}
% The deprecated macro |\childdoc| is a legacy version of |\childdocmain|:
%    \begin{macrocode}
\newcommand{\childdoc}{\childdocmain}
%    \end{macrocode}

% \macro{\childdocredirect}
% The deprecated macro |\childdocredirect| is a legacy version
% of |\childdocforward| and |\childdocforwardprefix|:
%    \begin{macrocode}
\newcommand{\childdocredirect}[2][]
{
  \begingroup
    \if?#1?
      \def\childdoctmp{\childdocforward{#2}}
    \else
      \def\childdoctmp{\childdocforwardprefix{#1}{#2}}
    \fi
    \expandafter
  \endgroup
  \childdoctmp
}
%    \end{macrocode}

%\iffalse
%</package>
%\fi
%
\endinput
|\\
|\childdocby{|\textit{main}|}|\\
\end{tabular}
\end{center}
%
Both forms have slightly different effects as described above.
The main file is prepared as usual, see \secref{sec:include}.

%%%%%%%%%%%%%%%%%%%%%%%%%%%%%%%%%%%%%%%%%%%%%%%%%%%%%%%%%%%%%%%%%%%%%%%%%%%%%%%%
\subsection{Legacy Detection}
\label{sec:detection}

The directive |\childdocmain| in the main file can detect
whether the complete document or merely a child is to be compiled
even without using the directive |\childdocof|.
This method is deprecated because it is less robust
and there is no compelling reason to use it;
it is merely provided for backward compatibility
and it may be removed in future versions.

If the detection mechanism is to be used,
it is mandatory to correctly specify
the filename of the main file as the argument of |\childdocmain|:
%
\begin{center}
\begin{tabular}{l}
|% \iffalse
%
% childdoc.dtx Copyright (C) 2017-2018 Niklas Beisert
%
% This work may be distributed and/or modified under the
% conditions of the LaTeX Project Public License, either version 1.3
% of this license or (at your option) any later version.
% The latest version of this license is in
%   http://www.latex-project.org/lppl.txt
% and version 1.3 or later is part of all distributions of LaTeX
% version 2005/12/01 or later.
%
% This work has the LPPL maintenance status `maintained'.
%
% The Current Maintainer of this work is Niklas Beisert.
%
% This work consists of the files childdoc.dtx and childdoc.ins
% and the derived files childdoc.def and cdocsamp.tex with
% cdocsch1.tex, cdocsch2.tex, cdocsdrf.tex, cdocsfn1.tex, cdocsfn2.tex.
%
%<package>\ifdefined\childdocmain\endinput\fi
%<package>\ProvidesFile{childdoc.def}[2018/12/30 v2.0 child document driver]
%<samplemain>\ProvidesFile{cdocsamp.tex}[2018/12/30 v2.0 sample for childdoc]
%<*driver>
%\ProvidesFile{childdoc.drv}[2018/12/30 v2.0 childdoc reference manual file]
\PassOptionsToClass{10pt,a4paper}{article}
\documentclass{ltxdoc}

\usepackage[margin=35mm]{geometry}
\usepackage{hyperref}
\usepackage{hyperxmp}
\usepackage[usenames]{color}

\hypersetup{colorlinks=true}
\hypersetup{pdfstartview=FitH}
\hypersetup{pdfpagemode=UseNone}
\hypersetup{pdfsource={}}
\hypersetup{pdflang={en-UK}}
\hypersetup{pdfcopyright={Copyright 2017-2018 Niklas Beisert.
  This work may be distributed and/or modified under the
  conditions of the LaTeX Project Public License, either version 1.3
  of this license or (at your option) any later version.}}
\hypersetup{pdflicenseurl={http://www.latex-project.org/lppl.txt}}
\hypersetup{pdfcontactaddress={ETH Zurich, ITP, HIT K,
  Wolfgang-Pauli-Strasse 27}}
\hypersetup{pdfcontactpostcode={8093}}
\hypersetup{pdfcontactcity={Zurich}}
\hypersetup{pdfcontactcountry={Switzerland}}
\hypersetup{pdfcontactemail={nbeisert@itp.phys.ethz.ch}}
\hypersetup{pdfcontacturl={http://people.phys.ethz.ch/\xmptilde nbeisert/}}

\newcommand{\secref}[1]{\hyperref[#1]{section \ref*{#1}}}

\parskip1ex
\parindent0pt
\let\olditemize\itemize
\def\itemize{\olditemize\parskip0pt}

\begin{document}

\title{The \textsf{childdoc} Package}
\hypersetup{pdftitle={The childdoc Package}}
\author{Niklas Beisert\\[2ex]
  Institut f\"ur Theoretische Physik\\
  Eidgen\"ossische Technische Hochschule Z\"urich\\
  Wolfgang-Pauli-Strasse 27, 8093 Z\"urich, Switzerland\\[1ex]
  \href{mailto:nbeisert@itp.phys.ethz.ch}
  {\texttt{nbeisert@itp.phys.ethz.ch}}}
\hypersetup{pdfauthor={Niklas Beisert}}
\hypersetup{pdfsubject={Manual for the LaTeX2e Package childdoc}}
\date{30 December 2018, \textsf{v2.0}}
\maketitle

\begin{abstract}\noindent
\textsf{childdoc} is a \LaTeXe{} package
that enables the direct compilation
of document sections included by |\include|
to individual files.
\end{abstract}

\begingroup
\parskip0ex
\tableofcontents
\endgroup

%%%%%%%%%%%%%%%%%%%%%%%%%%%%%%%%%%%%%%%%%%%%%%%%%%%%%%%%%%%%%%%%%%%%%%%%%%%%%%%%
%%%%%%%%%%%%%%%%%%%%%%%%%%%%%%%%%%%%%%%%%%%%%%%%%%%%%%%%%%%%%%%%%%%%%%%%%%%%%%%%
\section{Introduction}

\LaTeX{} provides a mechanism to structure a large document (such as a book)
into a main file and several child files (containing the chapters)
using the |\include| command.
This mechanism is beneficial for documents
which span hundreds of pages in order to
make the source file(s) more manageable.
Moreover, compilation can be restricted to
selected child files by means of the |\includeonly| command.
The latter feature can be used to reduce the compilation time while editing
(this was significantly more useful in the earlier days of \LaTeX{})
or to generate a smaller document which is easier to navigate.
Another application of |\includeonly| is to generate
documents consisting of selected parts of the complete document.

However, there are a few drawbacks of the plain |\include| mechanism:
\begin{itemize}
\item
The child files cannot be compiled on their own,
they can only be compiled via the main file.
A naive editing environment
(such as a text editor with an option
to have the current file processed by \LaTeX)
may require one to switch to the main file before compiling;
attempting to compile the child file produces errors.
\item
The main file must be modified (each time)
to adjust the |\includeonly| command
to the present needs. This easily leaves the main file in a messy state.
\item
The generated document will always carry the filename
of the main document. This is inconvenient if
several child files are to be compiled and
to be kept for distribution.
\end{itemize}

The present package provides a simple interface
to make child files individually compilable by \LaTeX{}.
Compiling a child file then has the same effect as compiling
the main file with an |\includeonly| command
to select the appropriate child.
Moreover the generated document will carry the name of the child
rather than the main file.
This resolves all three above issues.

This feature is meant to make the editing of books,
thesis documents and lecture notes somewhat more convenient.
However, the package can also be used efficiently for
composing a series of documents (such as exercise sheets)
which are typically distributed individually.
It then assists the author in generating the individual documents
(potentially in different versions)
as well as a document containing the collected series.
Another application is in developing style files
or other kinds of included material
where compilation of the style file could redirect
to a sample or test file.

%%%%%%%%%%%%%%%%%%%%%%%%%%%%%%%%%%%%%%%%%%%%%%%%%%%%%%%%%%%%%%%%%%%%%%%%%%%%%%%%
%%%%%%%%%%%%%%%%%%%%%%%%%%%%%%%%%%%%%%%%%%%%%%%%%%%%%%%%%%%%%%%%%%%%%%%%%%%%%%%%
\section{Usage}

First of all, the package \textsf{childdoc} is \emph{not} a standard
\LaTeXe{} |.sty| style file! Therefore it needs to be invoked in
a non-standard way.

%%%%%%%%%%%%%%%%%%%%%%%%%%%%%%%%%%%%%%%%%%%%%%%%%%%%%%%%%%%%%%%%%%%%%%%%%%%%%%%%
\subsection{Included Files}
\label{sec:include}

%%%%%%%%%%%%%%%%%%%%%%%%%%%%%%%%%%%%%%%%
\DescribeMacro{\childdocmain}
To use the package, add the commands
\begin{center}
\begin{tabular}{l}
|\input{childdoc.def}|\\
|\childdocmain{}|\\
\end{tabular}
\end{center}
at the very top of the main \LaTeX{} file,
in particular \emph{before} the |\documentclass| statement!
The argument of |\childdocmain| should be left empty
(but it must be present).

%%%%%%%%%%%%%%%%%%%%%%%%%%%%%%%%%%%%%%%%
\DescribeMacro{\childdocof}
Furthermore, add the commands
\begin{center}
\begin{tabular}{l}
|\input{childdoc.def}|\\
|\childdocof{|\textit{main}|}|\\
\end{tabular}
\end{center}
at the top of every child file \textit{child}
which is included by |\include{|\textit{child}|}|
from within the main file
(or at least for those files to be compiled individually).
The argument \textit{main} must be the filename of the main file.

There are a couple of
considerations in setting up the main and child documents:

%%%%%%%%%%%%%%%%%%%%%%%%%%%%%%%%%%%%%%%%
\paragraph{Restrictions.}

Please note the following restrictions:
\begin{itemize}
\item
|\childdocmain| must be called with one argument \textit{main}
to ensure compatibility with earlier version of the package.
It must either be empty (|\childdocmain{}|)
or precisely match the filename of the main file in which it is specified.
See \secref{sec:detection} for further information.
\item
The filename \textit{main} must be specified without the |.tex| extension.
\item
The filename \textit{main} is case sensitive
(even in case-insensitive file systems)
due to internal string comparison.
\item
The argument \textit{main} should be fully expanded, it cannot be a macro.
\item
Subdirectories and special characters should be avoided in filenames.
\item
The command |\childdocmain{|\textit{main}|}| must be followed by a whitespace.
It should not be followed immediately by another command
or by a comment mark `|%|'.
This is because the \TeX{} parser reads the token immediately following
the argument of |\childdocmain| and puts it
at the beginning of every child section;
however, a white\-space is ignored.
\end{itemize}

%%%%%%%%%%%%%%%%%%%%%%%%%%%%%%%%%%%%%%%%
\paragraph{Content of Main File.}

It is advisable to place all content in the child files included by |\include|.
Any output contained in the main file will appear in all child documents
unless suppressed manually;
it cannot be suppressed automatically by the |\includeonly| directive
and thus should normally be avoided.
A method to include some content in the main file
by means of conditional processing is described in \secref{sec:conditional}.

%%%%%%%%%%%%%%%%%%%%%%%%%%%%%%%%%%%%%%%%
\paragraph{Page Numbering.}

When only a part of the document is compiled,
the appropriate numbering of pages
(as well as other status parameters)
is determined from the |.aux| files.
The latter contain information from previous passes.
However this information needs to propagate through
all intermediate child documents.
Therefore the page numbering in child documents may well
be inconsistent until the complete document is compiled at least once.

A useful (if unconventional) way to always ensure a consistent
page numbering is to restart the numbering in each child document
and denote the pages by `\textit{child}|.|\textit{page}'
where \textit{child} represents the chapter/section number of the child file.
This can be achieved by the command
|\numberwithin{page}{|\textit{child}|}|
of the \textsf{amsmath} package
where \textit{child} can be |chapter| or |section|
depending on the chosen structuring.
Alternatively, one can modify the macro |\thepage| appropriately
and reset the counter |page| at the start of each child file.

%%%%%%%%%%%%%%%%%%%%%%%%%%%%%%%%%%%%%%%%%%%%%%%%%%%%%%%%%%%%%%%%%%%%%%%%%%%%%%%%
\subsection{Conditional Processing}
\label{sec:conditional}

The package provides a mechanism to compile different versions
of a document. To customise the versions further some conditional processing
can come in handy to distinguish which version is being compiled.
The package provides two macros to describe the compilation context:

%%%%%%%%%%%%%%%%%%%%%%%%%%%%%%%%%%%%%%%%
\DescribeMacro{\ifchilddoc}
The conditional |\ifchilddoc| distinguishes between the compilation of
child documents and the main document:
%
\begin{center}
|\ifchilddoc |\textit{child-code}| |[|\||else |\textit{main-code}]| \||fi|
\end{center}

%%%%%%%%%%%%%%%%%%%%%%%%%%%%%%%%%%%%%%%%
\DescribeMacro{\childdocname}
\DescribeMacro{\childdocjob}
The macro |\childdocname| contains the filename (without extension)
of the main or child file being processed.
Note that |\childdocjob| will always contain the name of the main file.

%%%%%%%%%%%%%%%%%%%%%%%%%%%%%%%%%%%%%%%%
\paragraph{Title Page.}

Conditional processing can be used to include a title or banner page
in the main document when proper precautions are taken.
Importantly, the code in the main file should ensure that the page counter
(as well as other status parameters which are stored in the |.aux| files)
takes the same value after the conditional processing.
Otherwise the page numbers may take divergent values
depending on which part is compiled.

For example, a title page could be declared by:
%
\begin{center}
\begin{tabular}{l}
|\ifchilddoc\||else|\\
|\addtocounter{page}{-1}|\\
\textit{code for title page}\\
|\newpage|\\
|\||fi|
\end{tabular}
\end{center}
%
A banner page for the child documents can be generated by:
%
\begin{center}
\begin{tabular}{l}
|\ifchilddoc|\\
|\addtocounter{page}{-1}|\\
\textit{code for banner page}\\
|\newpage|\\
|\||fi|
\end{tabular}
\end{center}
%
Here one could write a message such as:
\begin{center}
|This is the part \childdocname{} of \childdocjob{}.|
\end{center}

%%%%%%%%%%%%%%%%%%%%%%%%%%%%%%%%%%%%%%%%%%%%%%%%%%%%%%%%%%%%%%%%%%%%%%%%%%%%%%%%
\subsection{Flags}
\label{sec:flags}

The package makes it easy to generate different versions
of the main or child documents.
To this end compilation flags can be defined
and assigned different default values.
They will be particularly useful in conjunction
with the forwarding mechanism described in \secref{sec:forward}.

For example, it may be useful to have a flag |\version|
which can be set to |draft| or |final|.
The document source will contain some conditional code
depending on the value of |\version|.
Suppose further, the flag should default to |final| for the main file
and to |draft| for child files
which is a natural assignment for editing the document.
This is achieved by placing the following code
in the preamble of the main document
(below the |\childdocmain| directive):
%
\begin{center}
\begin{tabular}{l}
|\ifchilddoc|\\
|\providecommand{\version}{draft}|\\
|\||else|\\
|\providecommand{\version}{final}|\\
|\||fi|
\end{tabular}
\end{center}
%
The definition by |\providecommand| makes sure
that previous definitions are not overwritten.
Further statements |\providecommand{\version}{...}|
can thus be added before the above code to override it.

For the main file, one might add a line
(between |\childdocmain| and the above block)
%
\begin{center}
|%\ifchilddoc\||else\providecommand{\version}{draft}\||fi|
\end{center}
%
which can be uncommented to produce a draft version.
Likewise one can add a line to the very top of a child file
(above the |\childdocof{|\textit{main}|}| directive)
%
\begin{center}
|%\providecommand{\version}{final}|
\end{center}
%
which can be uncommented to produce the final version of this child document.

%%%%%%%%%%%%%%%%%%%%%%%%%%%%%%%%%%%%%%%%%%%%%%%%%%%%%%%%%%%%%%%%%%%%%%%%%%%%%%%%
\subsection{Forwarding}
\label{sec:forward}

Different versions of the main or child documents
using compilation flags as described in \secref{sec:flags}
can be (permanently) stored in different files
for convenient compilation, viewing and distribution.
To this end, the package defines a command
to pass on compilation to a different file:

%%%%%%%%%%%%%%%%%%%%%%%%%%%%%%%%%%%%%%%%
\DescribeMacro{\childdocforward}
The command |\childdocforward| redirects processing to
another source file:
%
\begin{center}
\begin{tabular}{l}
|\input{childdoc.def}|\\
|\childdocforward[|\textit{main}|]{|\textit{dest}|}|\\
\end{tabular}
\end{center}
%
The argument \textit{dest} is the destination file
(without extension).
It should be the main file or one of the child files.
Note that further \textsf{childdoc} directives
such as |\childdocof| and |\childdocforward|
in the indicated file will be processed in this form.
The optional argument \textit{main}
passes on directly to the main file \textit{main}
while pretending to compile the child \textit{dest}.
This form behaves as if \textit{dest}
issues |\childdocof{|\textit{main}|}| right away,
and no further \textsf{childdoc} directives will be processed.

%%%%%%%%%%%%%%%%%%%%%%%%%%%%%%%%%%%%%%%%
\DescribeMacro{\...prefix}
In the alternative form |\childdocforwardprefix|,
%
\begin{center}
\begin{tabular}{l}
|\input{childdoc.def}|\\
|\childdocforwardprefix[|\textit{main}|]{|\textit{prefix}|}{|\textit{dest}|}|
\end{tabular}
\end{center}
%
the destination file is determined by a pattern
depending on the current file:
To make this work, the current file must be called
`{\textit{prefix}\hspace{0.2em}\textit{suffix}}'
with \textit{prefix} matching precisely the argument.
Processing is then passed on to the file
`{\textit{dest}\hspace{0.2em}\textit{suffix}}'.
Surely, the same effect is achieved by
directly specifying the
argument `{\textit{dest}\hspace{0.2em}\textit{suffix}}'
in the first form.
However, that requires to set up a different file
for each child. With the alternative form of the command
all these files can have exactly the same content
which simplifies setting them up and maintaining them.

For example, the following file |draft.tex|
with a compilation flag |\version| as described in \secref{sec:flags}
compiles the main document as a draft:
%
\begin{center}
\begin{tabular}{l}
|\def\version{draft}|\\
|\input{childdoc.def}|\\
|\childdocforward{|\textit{main}|}|
\end{tabular}
\end{center}
%
Likewise, the following files |final|\textit{nn}|.tex|
compile the final version of the child document
|child|\textit{nn}|.tex|:
%
\begin{center}
\begin{tabular}{l}
|\def\version{final}|\\
|\input{childdoc.def}|\\
|\childdocforwardprefix{final}{child}|
\end{tabular}
\end{center}
%

Note that when several versions of a main file and/or of each child file
are to be generated, it may be convenient to set up a |Makefile| or
shell script to automatise the process.

%%%%%%%%%%%%%%%%%%%%%%%%%%%%%%%%%%%%%%%%%%%%%%%%%%%%%%%%%%%%%%%%%%%%%%%%%%%%%%%%
\subsection{Command Line Processing}
\label{sec:commandline}

The effect of redirection files can also be achieved by invoking
the \LaTeX{} compiler with a more elaborate command line.
Most conveniently this should be done as part
of a shell script or a |Makefile|.

When using \textsf{childdoc} in the main file, the following
command lines effectively perform a redirection
(note that depending on the shell being used,
backslashes may have to be doubled: `|\|' $\to$ `|\\|'):
%
\begin{center}
|... -jobname "|\textit{target}|" |\\|"|[\textit{flags}]%
|\input{childdoc.def}\childdocforward[|\textit{main}|]{|\textit{dest}|}"|
\end{center}
%
Here \textit{target} is the name of the output file,
\textit{main} is the name of the main file
and \textit{dest} is the name of the main or child file to be processed
(all filenames without extensions).
The optional argument \textit{main} can be omitted
if \textit{main} matches \textit{dest}.
Optionally, compilation \textit{flags} can be defined via |\def| commands.
This command line makes the \TeX{} engine believe
it is compiling the file \textit{target}
whose content is specified as the latter parameter.
The provided code then forwards the processing to
\textit{main} or \textit{dest} as described in \secref{sec:forward}.

%%%%%%%%%%%%%%%%%%%%%%%%%%%%%%%%%%%%%%%%%%%%%%%%%%%%%%%%%%%%%%%%%%%%%%%%%%%%%%%%
\subsection{Include by Input}
\label{sec:input}

Including child documents by |\include| has some restrictions by design.
Most notably, the content of a child document always occupies
its own set of pages; pages cannot be shared between child documents.
Usually, this behaviour makes perfect sense
because each child document contain an essential part of the document.
However, in some situations it may be desirable to compose
a document from a collection of parts
without having mandatory page breaks between then.
For this case, the package
provides a mechanism to include parts
by |\input| which can also be processed individually.
However, by construction this mechanism
requires manual handling of the content to be output.

%%%%%%%%%%%%%%%%%%%%%%%%%%%%%%%%%%%%%%%%
\DescribeMacro{\ifchilddocmanual}
The main file should be prepared as usual, see \secref{sec:include}.
However, the document body must make a distinction
between processing of an individual part and of the main document, e.g.:
%
\begin{center}
\begin{tabular}{l}
|\ifchilddocmanual|\\
|\input{\childdocname}|\\
|\||else|\\
\textit{document body with }|\input{|\textit{part}|}|\\
|\||fi|
\end{tabular}
\end{center}
%
The conditional |\ifchilddocmanual| is true whenever
a part to be included by |\input| is being compiled,
and the name of the part is stored in |\childdocname|.

%%%%%%%%%%%%%%%%%%%%%%%%%%%%%%%%%%%%%%%%
\DescribeMacro{\childdocby}
Each part to be included by |\input| should start with:
%
\begin{center}
\begin{tabular}{l}
|\input{childdoc.def}|\\
|\childdocby{|\textit{main}|}|\\
\end{tabular}
\end{center}
%
The directive |\childdocby| is similar to |\childdocof|
described in \secref{sec:include},
but the subsequent selection of content must be done manually.
To that end, both |\ifchilddoc| and |\ifchilddocmanual|
will be true upon processing of a part,
and the name of the part is stored in |\childdocname|.
Note that |\jobname| will be set to the filename of the current part
so that each part receives an individual |.aux| file
that does not interfere with the |.aux| file(s) of the main document.
This behaviour can be altered by the alternative form
|\childdocby[*]{|\textit{main}|}| (with a non-empty optional argument)
which uses the |.aux| file of the main document
by setting |\jobname| to \textit{main}.

%%%%%%%%%%%%%%%%%%%%%%%%%%%%%%%%%%%%%%%%%%%%%%%%%%%%%%%%%%%%%%%%%%%%%%%%%%%%%%%%
\subsection{Driver Development}
\label{sec:driver}

The \textsf{childdoc} mechanism can also be use for the development
of definition files such as \LaTeX{} styles or classes.
This case differs from the above setup with multiple parts
included by |\include| in that no |\includeonly| should be invoked.
This can be achieved by starting the include file
(before |\ProvidesPackage|) with:
%
\begin{center}
\begin{tabular}{l}
|\input{childdoc.def}|\\
|\childdocforward{|\textit{main}|}|\\
\end{tabular}
\end{center}
%
or alternatively with:
%
\begin{center}
\begin{tabular}{l}
|\input{childdoc.def}|\\
|\childdocby{|\textit{main}|}|\\
\end{tabular}
\end{center}
%
Both forms have slightly different effects as described above.
The main file is prepared as usual, see \secref{sec:include}.

%%%%%%%%%%%%%%%%%%%%%%%%%%%%%%%%%%%%%%%%%%%%%%%%%%%%%%%%%%%%%%%%%%%%%%%%%%%%%%%%
\subsection{Legacy Detection}
\label{sec:detection}

The directive |\childdocmain| in the main file can detect
whether the complete document or merely a child is to be compiled
even without using the directive |\childdocof|.
This method is deprecated because it is less robust
and there is no compelling reason to use it;
it is merely provided for backward compatibility
and it may be removed in future versions.

If the detection mechanism is to be used,
it is mandatory to correctly specify
the filename of the main file as the argument of |\childdocmain|:
%
\begin{center}
\begin{tabular}{l}
|\input{childdoc.def}|\\
|\childdocmain{|\textit{main}|}|\\
\end{tabular}
\end{center}
%
If |\jobname| does not match the argument \textit{main} of |\childdocmain|,
it is assumed that |\jobname| points to the child file to be compiled.
When using |\childdocmain| with the main file specified as argument,
it suffices to start a child file
with just |\input{|\textit{main}|}|
without loading of the package and using |\childdocof|.
If instead all processing is done
with the appropriate \textsf{childdoc} directives,
the argument of \textit{main} of |\childdocmain| can be empty.

An alternative version of the command line processing described
in \secref{sec:commandline} using the detection mechanism reads:
%
\begin{center}
|... -jobname "|\textit{target}|" "|[\textit{flags}]%
[|\def\jobname{|\textit{dest}|}|]|\input{|\textit{main}|}"|
\end{center}

%%%%%%%%%%%%%%%%%%%%%%%%%%%%%%%%%%%%%%%%%%%%%%%%%%%%%%%%%%%%%%%%%%%%%%%%%%%%%%%%
\subsection{Manual Code}
\label{sec:manual}

In case one cannot be certain whether the definitions file |childdoc.def|
is installed on the target \TeX{} distribution
and one prefers not to ship it,
it is conceivable to paste a few relevant commands into the sources.

To that end, drop all statements |\input{childdoc.def}|
and perform the replacements as outlined below.
Instead of |\childdocmain{|\textit{main}|}| add the following code
to the top of the main file:
%
\begin{center}
\begin{tabular}{l}
|\||ifdefined\childdocname\endinput\||fi\newif\ifchilddoc|\\
|\edef\childdocname{\scantokens\expandafter{\jobname\noexpand}}|\\
|\def\childdocmain{|\textit{main}|}\||ifx\childdocmain\childdocname\||else|\\
|\childdoctrue\includeonly{\childdocname}\let\jobname\childdocmain\||fi|\\
\end{tabular}
\end{center}
%
Instead of |\childdocof{|\textit{main}|}| just include the main file
at the top of each child file:
%
\begin{center}
|\input{|\textit{main}|}|
\end{center}
%
A simple redirection |\childdocforward{|\textit{dest}|}| is achieved by:
%
\begin{center}
|\def\jobname{|\textit{dest}|}\input{\jobname}|
\end{center}
%
The redirection with prefix
|\childdocforwardprefix[|\textit{prefix}|]{|\textit{dest}|}|
is accomplished by:
%
\begin{center}
\begin{tabular}{l}
|{\edef\jobname{\scantokens\expandafter{\jobname\noexpand}}|\\
|\def\redirectjob |\textit{prefix}|#1~~~{\gdef\jobname{|\textit{dest}|#1}}|\\
|\expandafter\redirectjob\jobname~~~}\input{\jobname}|
\end{tabular}
\end{center}

In an alternative approach,
child documents can be compiled by a specific command line
without additional code or specific definitions:
%
\begin{center}
|... -jobname "|\textit{target}|" "|[\textit{flags}]%
|\includeonly{|\textit{dest}|}\input{|\textit{main}|}"|
\end{center}
%

%%%%%%%%%%%%%%%%%%%%%%%%%%%%%%%%%%%%%%%%%%%%%%%%%%%%%%%%%%%%%%%%%%%%%%%%%%%%%%%%
%%%%%%%%%%%%%%%%%%%%%%%%%%%%%%%%%%%%%%%%%%%%%%%%%%%%%%%%%%%%%%%%%%%%%%%%%%%%%%%%
\section{Information}

%%%%%%%%%%%%%%%%%%%%%%%%%%%%%%%%%%%%%%%%%%%%%%%%%%%%%%%%%%%%%%%%%%%%%%%%%%%%%%%%
\subsection{Copyright}

Copyright \copyright{} 2017--2018 Niklas Beisert

This work may be distributed and/or modified under the
conditions of the \LaTeX{} Project Public License, either version 1.3
of this license or (at your option) any later version.
The latest version of this license is in
  \url{http://www.latex-project.org/lppl.txt}
and version 1.3 or later is part of all distributions of \LaTeX{}
version 2005/12/01 or later.

This work has the LPPL maintenance status `maintained'.

The Current Maintainer of this work is Niklas Beisert.

This work consists of the files |README.txt|, |childdoc.ins| and |childdoc.dtx|
as well as the derived files |childdoc.def|, |cdocsamp.tex|
with |cdocsch1.tex|, |cdocsch2.tex|, |cdocspt3.tex|, |cdocspt4.tex|,
|cdocsdrf.tex|, |cdocsfn1.tex|, |cdocsfn2.tex|
as well as |childdoc.pdf|.

%%%%%%%%%%%%%%%%%%%%%%%%%%%%%%%%%%%%%%%%%%%%%%%%%%%%%%%%%%%%%%%%%%%%%%%%%%%%%%%%
\subsection{Files and Installation}

The package consists of the files:
%
\begin{center}
\begin{tabular}{ll}
    |README.txt|   & readme file \\
    |childdoc.ins| & installation file \\
    |childdoc.dtx| & source file \\
    |childdoc.def| & definition file \\
    |cdocsamp.tex| & sample main file \\
    |cdocsch1.tex| & sample include file \\
    |cdocsch2.tex| & sample include file \\
    |cdocspt3.tex| & sample part file \\
    |cdocspt4.tex| & sample part file \\
    |cdocsdrf.tex| & sample redirection file \\
    |cdocsfn1.tex| & sample redirection file \\
    |cdocsfn2.tex| & sample redirection file \\
    |childdoc.pdf| & manual
\end{tabular}
\end{center}
%
The distribution consists of the files
|README.txt|, |childdoc.ins| and |childdoc.dtx|.
%
\begin{itemize}
\item
Run (pdf)\LaTeX{} on |childdoc.dtx|
to compile the manual |childdoc.pdf| (this file).
\item
Run \LaTeX{} on |childdoc.ins| to create the definitions file |childdoc.def|
and the sample |cdocsamp.tex| with include files
|cdocsch1.tex|, |cdocsch2.tex|, |cdocspt3.tex|, |cdocspt4.tex|,
|cdocsdrf.tex|, |cdocsfn1.tex|, |cdocsfn2.tex|.
Then copy the file |childdoc.def| to an appropriate directory of your \LaTeX{}
distribution, e.g.\ \textit{texmf-root}|/tex/latex/childdoc|.
\end{itemize}

%%%%%%%%%%%%%%%%%%%%%%%%%%%%%%%%%%%%%%%%%%%%%%%%%%%%%%%%%%%%%%%%%%%%%%%%%%%%%%%%
\subsection{Related CTAN Packages}

There are several other packages which offer a similar functionality:
%
\begin{itemize}
\item
The packages
\href{http://ctan.org/pkg/docmute}{\textsf{docmute}},
\href{http://ctan.org/pkg/includex}{\textsf{includex}} and
\href{http://ctan.org/pkg/standalone}{\textsf{standalone}}
provide commands to include only the document body of
a child file thus allowing both files to be compiled individually.
\item
The packages \href{http://ctan.org/pkg/subdocs}{\textsf{subdocs}}
and \href{http://ctan.org/pkg/subfiles}{\textsf{subfiles}}
provide structures in which the main and child documents can be
encapsulated and allowing them to be compiled individually.
The inclusion mechanism is different from the conventional |\include|.
\item
The package \href{http://ctan.org/pkg/combine}{\textsf{combine}}
is an elaborate solution to combine several documents into one.
\end{itemize}
%
See also the CTAN topic \href{http://ctan.org/topic/subdocs}{\textsf{subdocs}}
for further related packages.
The present package differs from the above solutions in that
a document structure constructed with the conventional |\include| mechanism
just needs two extra commands at the top of every file
such that all constituent files can be compiled individually.

%%%%%%%%%%%%%%%%%%%%%%%%%%%%%%%%%%%%%%%%%%%%%%%%%%%%%%%%%%%%%%%%%%%%%%%%%%%%%%%%
%\subsection{Feature Suggestions}
%
%The following is a list of features which may be useful for future
%versions of this package:
%%
%\begin{itemize}
%\item
%\ldots
%\end{itemize}

%%%%%%%%%%%%%%%%%%%%%%%%%%%%%%%%%%%%%%%%%%%%%%%%%%%%%%%%%%%%%%%%%%%%%%%%%%%%%%%%
\subsection{Revision History}

%%%%%%%%%%%%%%%%%%%%%%%%%%%%%%%%%%%%%%%%
\paragraph{v2.0:} 2018/12/30

\begin{itemize}
\item
immediate forward processing
\item
added |\childdocby| mechanism
\item
manual restructured
\end{itemize}

%%%%%%%%%%%%%%%%%%%%%%%%%%%%%%%%%%%%%%%%
\paragraph{v1.6:} 2018/01/17

\begin{itemize}
\item
application for development of include files
\item
corrections to manual
\end{itemize}

%%%%%%%%%%%%%%%%%%%%%%%%%%%%%%%%%%%%%%%%
\paragraph{v1.5:} 2017/05/21

\begin{itemize}
\item
more complete structuring introduced
\item
|\childdocof| introduced
\item
|\childdoc| renamed to |\childdocmain|
\item
|\childredirect| renamed to |\childdocforward| and |\childdocforwardprefix|
and functionality expanded
\end{itemize}

%%%%%%%%%%%%%%%%%%%%%%%%%%%%%%%%%%%%%%%%
\paragraph{v1.0:} 2017/04/27

\begin{itemize}
\item
manual and install package
\item
first version published on CTAN
\end{itemize}

%%%%%%%%%%%%%%%%%%%%%%%%%%%%%%%%%%%%%%%%
\paragraph{v0.6:} 2017/04/26

\begin{itemize}
\item
redirection mechanism added
\end{itemize}

%%%%%%%%%%%%%%%%%%%%%%%%%%%%%%%%%%%%%%%%
\paragraph{v0.5:} 2017/04/26

\begin{itemize}
\item
functionality in definition file
\end{itemize}


%%%%%%%%%%%%%%%%%%%%%%%%%%%%%%%%%%%%%%%%%%%%%%%%%%%%%%%%%%%%%%%%%%%%%%%%%%%%%%%%
%%%%%%%%%%%%%%%%%%%%%%%%%%%%%%%%%%%%%%%%%%%%%%%%%%%%%%%%%%%%%%%%%%%%%%%%%%%%%%%%
%%%%%%%%%%%%%%%%%%%%%%%%%%%%%%%%%%%%%%%%%%%%%%%%%%%%%%%%%%%%%%%%%%%%%%%%%%%%%%%%
\appendix

\settowidth\MacroIndent{\rmfamily\scriptsize 000\ }

 \DocInput{childdoc.dtx}

\end{document}
%</driver>
% \fi
%
% %%%%%%%%%%%%%%%%%%%%%%%%%%%%%%%%%%%%%%%%%%%%%%%%%%%%%%%%%%%%%%%%%%%%%%%%%%%%%%
% %%%%%%%%%%%%%%%%%%%%%%%%%%%%%%%%%%%%%%%%%%%%%%%%%%%%%%%%%%%%%%%%%%%%%%%%%%%%%%
% \section{Sample}
%\iffalse
%<*samplemain>
%\fi
%
% The following presents a sample document
% with two chapters, two parts, a title page,
% a compile flag as well as three forwarding files to set the flag.
% It consists of eight |.tex| files:
% \begin{center}
% \begin{tabular}{ll}
% |cdocsamp.tex|&main file\\
% |cdocsch1.tex|&include file for chapter 1\\
% |cdocsch2.tex|&include file for chapter 2\\
% |cdocspt3.tex|&include file for part 3\\
% |cdocspt4.tex|&include file for part 4\\
% |cdocsdrf.tex|&forwarding file for main file in draft mode\\
% |cdocsfi1.tex|&forwarding file for final version of chapter 1\\
% |cdocsfi2.tex|&forwarding file for final version of chapter 2\\
% \end{tabular}
% \end{center}
% Each of the eight files can be compiled directly by the \LaTeX{} compiler.
%
% %%%%%%%%%%%%%%%%%%%%%%%%%%%%%%%%%%%%%%
% \paragraph{Main File.}
%
% The main file is called |cdocsamp.tex|.
%
% Load the \textsf{childdoc} definitions and
% declare the filename for the main document:
%    \begin{macrocode}
\input{childdoc.def}
\childdocmain{}
%    \end{macrocode}

% Optional override for |\version| flag:
%    \begin{macrocode}
%%\ifchilddoc\else\providecommand{\version}{draft}\fi
%    \end{macrocode}

% Define the default values for the |\version| flag
% (|final| for the main file and |draft| for childs):
%    \begin{macrocode}
\ifchilddoc
\providecommand{\version}{draft}
\else
\providecommand{\version}{final}
\fi
%    \end{macrocode}

% Load the standard document class:
%    \begin{macrocode}
\documentclass[12pt]{article}
%    \end{macrocode}

% Start the document body:
%    \begin{macrocode}
\begin{document}
%    \end{macrocode}

% Declare a title page.
% Print title, part of document being processed and version flag:
%    \begin{macrocode}
\addtocounter{page}{-1}
\begin{center}
{\LARGE\bfseries{}childdoc example\par}
\vspace{1cm}
\ifchilddoc
\ifchilddocmanual part\else chapter\fi:
`\childdocname' of `\childdocjob'\par
\else
main document: `\childdocjob'\par
\fi
version: \version\par
\end{center}
\newpage
%    \end{macrocode}

% Manually include selected file,
% otherwise process as usual:
%    \begin{macrocode}
\ifchilddocmanual
\section*{part `\childdocname'}
\input{\childdocname}
\else
%    \end{macrocode}

% Include the two chapters:
%    \begin{macrocode}
\include{cdocsch1}
\include{cdocsch2}
%    \end{macrocode}

% Include the two parts unless only chapters should be displayed:
%    \begin{macrocode}
\ifchilddoc\else
\section{part three}
\input{cdocspt3}
\section{part four}
\input{cdocspt4}
\fi
%    \end{macrocode}

% Process as usual until here:
%    \begin{macrocode}
\fi
%    \end{macrocode}

% End of document body:
%    \begin{macrocode}
\end{document}
%    \end{macrocode}
%\iffalse
%</samplemain>
%\fi
%
% %%%%%%%%%%%%%%%%%%%%%%%%%%%%%%%%%%%%%%
% \paragraph{Chapter Include Files.}
%
% The include files are called |cdocsch1.tex| and |cdocsch2.tex|.
%
%\iffalse
%<*samplechap1|samplechap2>
%\fi

% Optional override for |\version| flag:
%    \begin{macrocode}
%%\providecommand{\version}{final}
%    \end{macrocode}

% Include the main document:
%    \begin{macrocode}
\input{childdoc.def}
\childdocof{cdocsamp}
%    \end{macrocode}

%\iffalse
%</samplechap1|samplechap2>
%\fi
%
%\iffalse
%<*samplechap1>
%\fi
% Some text for chapter 1:
%    \begin{macrocode}
\section{one}
some text in chapter one
%    \end{macrocode}

%\iffalse
%</samplechap1>
%\fi
% Some text for chapter 2:
%\iffalse
%<*samplechap2>
%\fi
%    \begin{macrocode}
\section{two}
more text in chapter two
%    \end{macrocode}

%\iffalse
%</samplechap2>
%\fi
%
% %%%%%%%%%%%%%%%%%%%%%%%%%%%%%%%%%%%%%%
% \paragraph{Part Include Files.}
%
% The include files are called |cdocspt3.tex| and |cdocspt4.tex|.
%
%\iffalse
%<*samplepart3|samplepart4>
%\fi

% Optional override for |\version| flag:
%    \begin{macrocode}
%%\providecommand{\version}{final}
%    \end{macrocode}

% Include the main document:
%    \begin{macrocode}
\input{childdoc.def}
\childdocby{cdocsamp}
%    \end{macrocode}

%\iffalse
%</samplepart3|samplepart4>
%\fi
%
%\iffalse
%<*samplepart3>
%\fi
% Some text for part 3:
%    \begin{macrocode}
some text in part three
%    \end{macrocode}

%\iffalse
%</samplepart3>
%\fi
% Some text for part 4:
%\iffalse
%<*samplepart4>
%\fi
%    \begin{macrocode}
more text in part four
%    \end{macrocode}

%\iffalse
%</samplepart4>
%\fi
%
% %%%%%%%%%%%%%%%%%%%%%%%%%%%%%%%%%%%%%%
% \paragraph{Forwarding for a Complete Draft.}
%
% The following forwarding file |cdocsdrf.tex|
% compiles the main document in draft mode:
%\iffalse
%<*sampledraft>
%\fi
%    \begin{macrocode}
\def\version{draft}
\input{childdoc.def}
\childdocforward{cdocsamp}
%    \end{macrocode}

%\iffalse
%</sampledraft>
%\fi
%
% %%%%%%%%%%%%%%%%%%%%%%%%%%%%%%%%%%%%%%
% \paragraph{Forwarding for Final Version of the Chapters.}
%
% The following forwarding files |cdocsfn1.tex| and |cdocsfn2.tex|
% (with identical content)
% compile the final versions of the child documents
% |cdocsch1.tex| and |cdocsch2.tex|, respectively:
%\iffalse
%<*samplefinal>
%\fi
%    \begin{macrocode}
\def\version{final}
\input{childdoc.def}
\childdocforwardprefix[cdocsamp]{cdocsfn}{cdocsch}
%    \end{macrocode}

%\iffalse
%</samplefinal>
%\fi
%
% %%%%%%%%%%%%%%%%%%%%%%%%%%%%%%%%%%%%%%
% \paragraph{Command Line Processing.}
%
% The following three command lines generate the output files
% |cdocscld|, |cdocscl1| and |cdocscl2|
% which should be identical to
% |cdocsdrf|, |cdocsch1| and |cdocsfn2|, respectively:
% \begin{center}
% \begin{tabular}{l}
% |latex -jobname cdocscld \|\\
% |  "\def\version{draft}\input{childdoc.def}\childdocforward{cdocsamp}"|\\
% |latex -jobname cdocscl1 \|\\
% |  "\input{childdoc.def}\childdocforward[cdocsamp]{cdocsch1}"|\\
% |latex -jobname cdocscl2 \|\\
% |  "\def\version{final}\input{childdoc.def}\childdocforward{cdocsch2}"|
% \end{tabular}
% \end{center}
% Note that the trailing backslash on each first line
% merely continues the input to the second line
% (for convenient cut ant paste).
% Furthermore, the command |latex| can be replaced by any
% of its alternative versions such as |pdflatex|.
%
% %%%%%%%%%%%%%%%%%%%%%%%%%%%%%%%%%%%%%%%%%%%%%%%%%%%%%%%%%%%%%%%%%%%%%%%%%%%%%%
% %%%%%%%%%%%%%%%%%%%%%%%%%%%%%%%%%%%%%%%%%%%%%%%%%%%%%%%%%%%%%%%%%%%%%%%%%%%%%%
% \section{Implementation}
%\iffalse
%<*package>
%\fi
%
% This section describes the definitions file |childdoc.def|.

% The definitions cannot be loaded using |\usepackage| or |\RequirePackage|
% which has a mechanism to prevent loading a style file more than once.
% When loading the definitions by means of |\input|
% multiple instances have to be prevented manually:
%\iffalse
%This code needs to be before the `\ProvidesFile' directive
%which is defined at the beginning of this file.
%Therefore it is also placed there and commented out here.
%</package>
%<*discard>
%\fi
%    \begin{macrocode}
\ifdefined\childdocmain\endinput\fi
%    \end{macrocode}
%\iffalse
%</discard>
%<*package>
%\fi
%
% \macro{\ifchilddoc}
% \macro{\ifchilddocmanual}
% The conditional |\ifchilddoc| tells whether a
% child (true) or main (false) document is being compiled.
% The conditional |\ifchilddocmanual| tells whether
% the |\includeonly| mechanism is used (false) or
% the selection of child files must be performed manually (true).
% The definitions initialise to false:
%    \begin{macrocode}
\newif\ifchilddoc
\newif\ifchilddocmanual
%    \end{macrocode}

% \macro{\childdocname}
% \macro{\childdocjob}
% The macro |\childdocname| stores the name of the main document
% to be compiled. The macro |\childdocjob| stores the name of
% the document on which the \LaTeX{} compiler was originally invoked.
% The content of |\jobname| cannot be compared
% to filenames specified in the source due to different catcodes.
% The following code rescans |\jobname|, stores the result
% in |\childdocname| and saves a copy in |\childdocjob|:
%    \begin{macrocode}
\edef\childdocname{\scantokens\expandafter{\jobname\noexpand}}
\let\childdocjob\childdocname
%    \end{macrocode}

% \macro{\childdocdisable}
% The macro |\childdocdisable| prevents the main file
% from being processed more than once.
% At this stage, the main document command |\childdocmain|
% is assumed to be called once again where it should do nothing.
% Any subsequent call to it should prevent
% a secondary processing of the main document
% It overwrites the forwarding commands
% |\childdocof| and |\childdocforward|
% with empty macros to prevent further inclusions of the main document:
%    \begin{macrocode}
\newcommand{\childdocdisable}
{
  \renewcommand{\childdocmain}[1]{\renewcommand{\childdocmain}[1]{\endinput}}
  \renewcommand{\childdocof}[1]{}
  \renewcommand{\childdocby}[2][]{}
  \renewcommand{\childdocforward}[2][]{}
  \renewcommand{\childdocdisable}{}
}
%    \end{macrocode}

% \macro{\childdocmain}
% The macro |\childdocmain| is to be called at the top of the main file
% with nothing or the main filename (without extension) as argument.
% First, it breaks loops.
% If the argument is not empty and does not match |\childdocname|
% (which is set by the first inclusion of |childdoc.def|),
% |\ifchilddoc| is set to true, |\includeonly| is applied to the child file
% and |\jobname| is set to the main file
% (for proper handling of |.aux| files):
%    \begin{macrocode}
\newcommand{\childdocmain}[1]
{
  \childdocdisable\childdocmain{}
  \if?#1?\else
    \begingroup
      \def\childdoctmp{#1}
      \ifx\childdoctmp\childdocname
        \def\childdoctmp{}
      \else
        \def\childdoctmp
        {
          \childdoctrue
          \includeonly{\childdocname}
          \def\childdocjob{#1}
          \def\jobname{#1}
        }
      \fi
      \expandafter
    \endgroup
    \childdoctmp
  \fi
}
%    \end{macrocode}

% \macro{\childdocof}
% The command |\childdocof| redirects
% compilation to the main file |#1|.
%    \begin{macrocode}
\newcommand{\childdocof}[1]
{
  \childdocdisable
  \childdoctrue
  \includeonly{\childdocname}
  \def\jobname{#1}
  \def\childdocjob{#1}
  \input{#1}
}
%    \end{macrocode}

% \macro{\childdocby}
% The command |\childdocby| ....
%    \begin{macrocode}
\newcommand{\childdocby}[2][]
{
  \childdocdisable
  \childdoctrue
  \childdocmanualtrue
  \if?#1?\else
    \def\jobname{#2}
  \fi
  \def\childdocjob{#2}
  \input{#2}
  \endinput
}
%    \end{macrocode}

% \macro{\childdocforward}
% The command |\childdocforward| redirects
% compilation to the main file or
% (if the optional argument is given) a child file.
% Parameters are set as if the main file
% or a child file starting with |\childdocof| was compiled.
% Then compilation is handed over to the main file:
%    \begin{macrocode}
\newcommand{\childdocforward}[2][]
{
  \begingroup
    \if?#1?
      \def\childdoctmp
      {
        \def\childdocname{#2}
        \def\childdocjob{#2}
        \def\jobname{#2}
        \input{#2}
        \endinput
      }
    \else
      \def\childdoctmp
      {
        \childdocdisable
        \def\childdocname{#2}
        \childdoctrue
        \includeonly{#2}
        \def\childdocjob{#1}
        \def\jobname{#1}
        \input{#1}
        \endinput
      }
    \fi
    \expandafter
  \endgroup
  \childdoctmp
}
%    \end{macrocode}

% \macro{\childdocforwardprefix}
% The command |\childdocforwardprefix| redirects
% compilation to the main or a child file by means of a pattern.
% The prefix |#1| in the current filename is replaced by |#2|
% and the suffix of the current filename is kept
% (it is assumed that the filename does not contain the substring `|~~~|'
% which is used as a delimiter).
% Compilation is handed over to the new file by |\childdocforward|:
%    \begin{macrocode}
\newcommand{\childdocforwardprefix}[3][]
{
  \begingroup
    \def\childdocextract #2##1~~~{\def\childdoctmp{\childdocforward[#1]{#3##1}}}
    \expandafter\childdocextract\childdocname~~~
    \expandafter
  \endgroup
  \childdoctmp
}
%    \end{macrocode}

% \macro{\childdoc}
% The deprecated macro |\childdoc| is a legacy version of |\childdocmain|:
%    \begin{macrocode}
\newcommand{\childdoc}{\childdocmain}
%    \end{macrocode}

% \macro{\childdocredirect}
% The deprecated macro |\childdocredirect| is a legacy version
% of |\childdocforward| and |\childdocforwardprefix|:
%    \begin{macrocode}
\newcommand{\childdocredirect}[2][]
{
  \begingroup
    \if?#1?
      \def\childdoctmp{\childdocforward{#2}}
    \else
      \def\childdoctmp{\childdocforwardprefix{#1}{#2}}
    \fi
    \expandafter
  \endgroup
  \childdoctmp
}
%    \end{macrocode}

%\iffalse
%</package>
%\fi
%
\endinput
|\\
|\childdocmain{|\textit{main}|}|\\
\end{tabular}
\end{center}
%
If |\jobname| does not match the argument \textit{main} of |\childdocmain|,
it is assumed that |\jobname| points to the child file to be compiled.
When using |\childdocmain| with the main file specified as argument,
it suffices to start a child file
with just |\input{|\textit{main}|}|
without loading of the package and using |\childdocof|.
If instead all processing is done
with the appropriate \textsf{childdoc} directives,
the argument of \textit{main} of |\childdocmain| can be empty.

An alternative version of the command line processing described
in \secref{sec:commandline} using the detection mechanism reads:
%
\begin{center}
|... -jobname "|\textit{target}|" "|[\textit{flags}]%
[|\def\jobname{|\textit{dest}|}|]|\input{|\textit{main}|}"|
\end{center}

%%%%%%%%%%%%%%%%%%%%%%%%%%%%%%%%%%%%%%%%%%%%%%%%%%%%%%%%%%%%%%%%%%%%%%%%%%%%%%%%
\subsection{Manual Code}
\label{sec:manual}

In case one cannot be certain whether the definitions file |childdoc.def|
is installed on the target \TeX{} distribution
and one prefers not to ship it,
it is conceivable to paste a few relevant commands into the sources.

To that end, drop all statements |% \iffalse
%
% childdoc.dtx Copyright (C) 2017-2018 Niklas Beisert
%
% This work may be distributed and/or modified under the
% conditions of the LaTeX Project Public License, either version 1.3
% of this license or (at your option) any later version.
% The latest version of this license is in
%   http://www.latex-project.org/lppl.txt
% and version 1.3 or later is part of all distributions of LaTeX
% version 2005/12/01 or later.
%
% This work has the LPPL maintenance status `maintained'.
%
% The Current Maintainer of this work is Niklas Beisert.
%
% This work consists of the files childdoc.dtx and childdoc.ins
% and the derived files childdoc.def and cdocsamp.tex with
% cdocsch1.tex, cdocsch2.tex, cdocsdrf.tex, cdocsfn1.tex, cdocsfn2.tex.
%
%<package>\ifdefined\childdocmain\endinput\fi
%<package>\ProvidesFile{childdoc.def}[2018/12/30 v2.0 child document driver]
%<samplemain>\ProvidesFile{cdocsamp.tex}[2018/12/30 v2.0 sample for childdoc]
%<*driver>
%\ProvidesFile{childdoc.drv}[2018/12/30 v2.0 childdoc reference manual file]
\PassOptionsToClass{10pt,a4paper}{article}
\documentclass{ltxdoc}

\usepackage[margin=35mm]{geometry}
\usepackage{hyperref}
\usepackage{hyperxmp}
\usepackage[usenames]{color}

\hypersetup{colorlinks=true}
\hypersetup{pdfstartview=FitH}
\hypersetup{pdfpagemode=UseNone}
\hypersetup{pdfsource={}}
\hypersetup{pdflang={en-UK}}
\hypersetup{pdfcopyright={Copyright 2017-2018 Niklas Beisert.
  This work may be distributed and/or modified under the
  conditions of the LaTeX Project Public License, either version 1.3
  of this license or (at your option) any later version.}}
\hypersetup{pdflicenseurl={http://www.latex-project.org/lppl.txt}}
\hypersetup{pdfcontactaddress={ETH Zurich, ITP, HIT K,
  Wolfgang-Pauli-Strasse 27}}
\hypersetup{pdfcontactpostcode={8093}}
\hypersetup{pdfcontactcity={Zurich}}
\hypersetup{pdfcontactcountry={Switzerland}}
\hypersetup{pdfcontactemail={nbeisert@itp.phys.ethz.ch}}
\hypersetup{pdfcontacturl={http://people.phys.ethz.ch/\xmptilde nbeisert/}}

\newcommand{\secref}[1]{\hyperref[#1]{section \ref*{#1}}}

\parskip1ex
\parindent0pt
\let\olditemize\itemize
\def\itemize{\olditemize\parskip0pt}

\begin{document}

\title{The \textsf{childdoc} Package}
\hypersetup{pdftitle={The childdoc Package}}
\author{Niklas Beisert\\[2ex]
  Institut f\"ur Theoretische Physik\\
  Eidgen\"ossische Technische Hochschule Z\"urich\\
  Wolfgang-Pauli-Strasse 27, 8093 Z\"urich, Switzerland\\[1ex]
  \href{mailto:nbeisert@itp.phys.ethz.ch}
  {\texttt{nbeisert@itp.phys.ethz.ch}}}
\hypersetup{pdfauthor={Niklas Beisert}}
\hypersetup{pdfsubject={Manual for the LaTeX2e Package childdoc}}
\date{30 December 2018, \textsf{v2.0}}
\maketitle

\begin{abstract}\noindent
\textsf{childdoc} is a \LaTeXe{} package
that enables the direct compilation
of document sections included by |\include|
to individual files.
\end{abstract}

\begingroup
\parskip0ex
\tableofcontents
\endgroup

%%%%%%%%%%%%%%%%%%%%%%%%%%%%%%%%%%%%%%%%%%%%%%%%%%%%%%%%%%%%%%%%%%%%%%%%%%%%%%%%
%%%%%%%%%%%%%%%%%%%%%%%%%%%%%%%%%%%%%%%%%%%%%%%%%%%%%%%%%%%%%%%%%%%%%%%%%%%%%%%%
\section{Introduction}

\LaTeX{} provides a mechanism to structure a large document (such as a book)
into a main file and several child files (containing the chapters)
using the |\include| command.
This mechanism is beneficial for documents
which span hundreds of pages in order to
make the source file(s) more manageable.
Moreover, compilation can be restricted to
selected child files by means of the |\includeonly| command.
The latter feature can be used to reduce the compilation time while editing
(this was significantly more useful in the earlier days of \LaTeX{})
or to generate a smaller document which is easier to navigate.
Another application of |\includeonly| is to generate
documents consisting of selected parts of the complete document.

However, there are a few drawbacks of the plain |\include| mechanism:
\begin{itemize}
\item
The child files cannot be compiled on their own,
they can only be compiled via the main file.
A naive editing environment
(such as a text editor with an option
to have the current file processed by \LaTeX)
may require one to switch to the main file before compiling;
attempting to compile the child file produces errors.
\item
The main file must be modified (each time)
to adjust the |\includeonly| command
to the present needs. This easily leaves the main file in a messy state.
\item
The generated document will always carry the filename
of the main document. This is inconvenient if
several child files are to be compiled and
to be kept for distribution.
\end{itemize}

The present package provides a simple interface
to make child files individually compilable by \LaTeX{}.
Compiling a child file then has the same effect as compiling
the main file with an |\includeonly| command
to select the appropriate child.
Moreover the generated document will carry the name of the child
rather than the main file.
This resolves all three above issues.

This feature is meant to make the editing of books,
thesis documents and lecture notes somewhat more convenient.
However, the package can also be used efficiently for
composing a series of documents (such as exercise sheets)
which are typically distributed individually.
It then assists the author in generating the individual documents
(potentially in different versions)
as well as a document containing the collected series.
Another application is in developing style files
or other kinds of included material
where compilation of the style file could redirect
to a sample or test file.

%%%%%%%%%%%%%%%%%%%%%%%%%%%%%%%%%%%%%%%%%%%%%%%%%%%%%%%%%%%%%%%%%%%%%%%%%%%%%%%%
%%%%%%%%%%%%%%%%%%%%%%%%%%%%%%%%%%%%%%%%%%%%%%%%%%%%%%%%%%%%%%%%%%%%%%%%%%%%%%%%
\section{Usage}

First of all, the package \textsf{childdoc} is \emph{not} a standard
\LaTeXe{} |.sty| style file! Therefore it needs to be invoked in
a non-standard way.

%%%%%%%%%%%%%%%%%%%%%%%%%%%%%%%%%%%%%%%%%%%%%%%%%%%%%%%%%%%%%%%%%%%%%%%%%%%%%%%%
\subsection{Included Files}
\label{sec:include}

%%%%%%%%%%%%%%%%%%%%%%%%%%%%%%%%%%%%%%%%
\DescribeMacro{\childdocmain}
To use the package, add the commands
\begin{center}
\begin{tabular}{l}
|\input{childdoc.def}|\\
|\childdocmain{}|\\
\end{tabular}
\end{center}
at the very top of the main \LaTeX{} file,
in particular \emph{before} the |\documentclass| statement!
The argument of |\childdocmain| should be left empty
(but it must be present).

%%%%%%%%%%%%%%%%%%%%%%%%%%%%%%%%%%%%%%%%
\DescribeMacro{\childdocof}
Furthermore, add the commands
\begin{center}
\begin{tabular}{l}
|\input{childdoc.def}|\\
|\childdocof{|\textit{main}|}|\\
\end{tabular}
\end{center}
at the top of every child file \textit{child}
which is included by |\include{|\textit{child}|}|
from within the main file
(or at least for those files to be compiled individually).
The argument \textit{main} must be the filename of the main file.

There are a couple of
considerations in setting up the main and child documents:

%%%%%%%%%%%%%%%%%%%%%%%%%%%%%%%%%%%%%%%%
\paragraph{Restrictions.}

Please note the following restrictions:
\begin{itemize}
\item
|\childdocmain| must be called with one argument \textit{main}
to ensure compatibility with earlier version of the package.
It must either be empty (|\childdocmain{}|)
or precisely match the filename of the main file in which it is specified.
See \secref{sec:detection} for further information.
\item
The filename \textit{main} must be specified without the |.tex| extension.
\item
The filename \textit{main} is case sensitive
(even in case-insensitive file systems)
due to internal string comparison.
\item
The argument \textit{main} should be fully expanded, it cannot be a macro.
\item
Subdirectories and special characters should be avoided in filenames.
\item
The command |\childdocmain{|\textit{main}|}| must be followed by a whitespace.
It should not be followed immediately by another command
or by a comment mark `|%|'.
This is because the \TeX{} parser reads the token immediately following
the argument of |\childdocmain| and puts it
at the beginning of every child section;
however, a white\-space is ignored.
\end{itemize}

%%%%%%%%%%%%%%%%%%%%%%%%%%%%%%%%%%%%%%%%
\paragraph{Content of Main File.}

It is advisable to place all content in the child files included by |\include|.
Any output contained in the main file will appear in all child documents
unless suppressed manually;
it cannot be suppressed automatically by the |\includeonly| directive
and thus should normally be avoided.
A method to include some content in the main file
by means of conditional processing is described in \secref{sec:conditional}.

%%%%%%%%%%%%%%%%%%%%%%%%%%%%%%%%%%%%%%%%
\paragraph{Page Numbering.}

When only a part of the document is compiled,
the appropriate numbering of pages
(as well as other status parameters)
is determined from the |.aux| files.
The latter contain information from previous passes.
However this information needs to propagate through
all intermediate child documents.
Therefore the page numbering in child documents may well
be inconsistent until the complete document is compiled at least once.

A useful (if unconventional) way to always ensure a consistent
page numbering is to restart the numbering in each child document
and denote the pages by `\textit{child}|.|\textit{page}'
where \textit{child} represents the chapter/section number of the child file.
This can be achieved by the command
|\numberwithin{page}{|\textit{child}|}|
of the \textsf{amsmath} package
where \textit{child} can be |chapter| or |section|
depending on the chosen structuring.
Alternatively, one can modify the macro |\thepage| appropriately
and reset the counter |page| at the start of each child file.

%%%%%%%%%%%%%%%%%%%%%%%%%%%%%%%%%%%%%%%%%%%%%%%%%%%%%%%%%%%%%%%%%%%%%%%%%%%%%%%%
\subsection{Conditional Processing}
\label{sec:conditional}

The package provides a mechanism to compile different versions
of a document. To customise the versions further some conditional processing
can come in handy to distinguish which version is being compiled.
The package provides two macros to describe the compilation context:

%%%%%%%%%%%%%%%%%%%%%%%%%%%%%%%%%%%%%%%%
\DescribeMacro{\ifchilddoc}
The conditional |\ifchilddoc| distinguishes between the compilation of
child documents and the main document:
%
\begin{center}
|\ifchilddoc |\textit{child-code}| |[|\||else |\textit{main-code}]| \||fi|
\end{center}

%%%%%%%%%%%%%%%%%%%%%%%%%%%%%%%%%%%%%%%%
\DescribeMacro{\childdocname}
\DescribeMacro{\childdocjob}
The macro |\childdocname| contains the filename (without extension)
of the main or child file being processed.
Note that |\childdocjob| will always contain the name of the main file.

%%%%%%%%%%%%%%%%%%%%%%%%%%%%%%%%%%%%%%%%
\paragraph{Title Page.}

Conditional processing can be used to include a title or banner page
in the main document when proper precautions are taken.
Importantly, the code in the main file should ensure that the page counter
(as well as other status parameters which are stored in the |.aux| files)
takes the same value after the conditional processing.
Otherwise the page numbers may take divergent values
depending on which part is compiled.

For example, a title page could be declared by:
%
\begin{center}
\begin{tabular}{l}
|\ifchilddoc\||else|\\
|\addtocounter{page}{-1}|\\
\textit{code for title page}\\
|\newpage|\\
|\||fi|
\end{tabular}
\end{center}
%
A banner page for the child documents can be generated by:
%
\begin{center}
\begin{tabular}{l}
|\ifchilddoc|\\
|\addtocounter{page}{-1}|\\
\textit{code for banner page}\\
|\newpage|\\
|\||fi|
\end{tabular}
\end{center}
%
Here one could write a message such as:
\begin{center}
|This is the part \childdocname{} of \childdocjob{}.|
\end{center}

%%%%%%%%%%%%%%%%%%%%%%%%%%%%%%%%%%%%%%%%%%%%%%%%%%%%%%%%%%%%%%%%%%%%%%%%%%%%%%%%
\subsection{Flags}
\label{sec:flags}

The package makes it easy to generate different versions
of the main or child documents.
To this end compilation flags can be defined
and assigned different default values.
They will be particularly useful in conjunction
with the forwarding mechanism described in \secref{sec:forward}.

For example, it may be useful to have a flag |\version|
which can be set to |draft| or |final|.
The document source will contain some conditional code
depending on the value of |\version|.
Suppose further, the flag should default to |final| for the main file
and to |draft| for child files
which is a natural assignment for editing the document.
This is achieved by placing the following code
in the preamble of the main document
(below the |\childdocmain| directive):
%
\begin{center}
\begin{tabular}{l}
|\ifchilddoc|\\
|\providecommand{\version}{draft}|\\
|\||else|\\
|\providecommand{\version}{final}|\\
|\||fi|
\end{tabular}
\end{center}
%
The definition by |\providecommand| makes sure
that previous definitions are not overwritten.
Further statements |\providecommand{\version}{...}|
can thus be added before the above code to override it.

For the main file, one might add a line
(between |\childdocmain| and the above block)
%
\begin{center}
|%\ifchilddoc\||else\providecommand{\version}{draft}\||fi|
\end{center}
%
which can be uncommented to produce a draft version.
Likewise one can add a line to the very top of a child file
(above the |\childdocof{|\textit{main}|}| directive)
%
\begin{center}
|%\providecommand{\version}{final}|
\end{center}
%
which can be uncommented to produce the final version of this child document.

%%%%%%%%%%%%%%%%%%%%%%%%%%%%%%%%%%%%%%%%%%%%%%%%%%%%%%%%%%%%%%%%%%%%%%%%%%%%%%%%
\subsection{Forwarding}
\label{sec:forward}

Different versions of the main or child documents
using compilation flags as described in \secref{sec:flags}
can be (permanently) stored in different files
for convenient compilation, viewing and distribution.
To this end, the package defines a command
to pass on compilation to a different file:

%%%%%%%%%%%%%%%%%%%%%%%%%%%%%%%%%%%%%%%%
\DescribeMacro{\childdocforward}
The command |\childdocforward| redirects processing to
another source file:
%
\begin{center}
\begin{tabular}{l}
|\input{childdoc.def}|\\
|\childdocforward[|\textit{main}|]{|\textit{dest}|}|\\
\end{tabular}
\end{center}
%
The argument \textit{dest} is the destination file
(without extension).
It should be the main file or one of the child files.
Note that further \textsf{childdoc} directives
such as |\childdocof| and |\childdocforward|
in the indicated file will be processed in this form.
The optional argument \textit{main}
passes on directly to the main file \textit{main}
while pretending to compile the child \textit{dest}.
This form behaves as if \textit{dest}
issues |\childdocof{|\textit{main}|}| right away,
and no further \textsf{childdoc} directives will be processed.

%%%%%%%%%%%%%%%%%%%%%%%%%%%%%%%%%%%%%%%%
\DescribeMacro{\...prefix}
In the alternative form |\childdocforwardprefix|,
%
\begin{center}
\begin{tabular}{l}
|\input{childdoc.def}|\\
|\childdocforwardprefix[|\textit{main}|]{|\textit{prefix}|}{|\textit{dest}|}|
\end{tabular}
\end{center}
%
the destination file is determined by a pattern
depending on the current file:
To make this work, the current file must be called
`{\textit{prefix}\hspace{0.2em}\textit{suffix}}'
with \textit{prefix} matching precisely the argument.
Processing is then passed on to the file
`{\textit{dest}\hspace{0.2em}\textit{suffix}}'.
Surely, the same effect is achieved by
directly specifying the
argument `{\textit{dest}\hspace{0.2em}\textit{suffix}}'
in the first form.
However, that requires to set up a different file
for each child. With the alternative form of the command
all these files can have exactly the same content
which simplifies setting them up and maintaining them.

For example, the following file |draft.tex|
with a compilation flag |\version| as described in \secref{sec:flags}
compiles the main document as a draft:
%
\begin{center}
\begin{tabular}{l}
|\def\version{draft}|\\
|\input{childdoc.def}|\\
|\childdocforward{|\textit{main}|}|
\end{tabular}
\end{center}
%
Likewise, the following files |final|\textit{nn}|.tex|
compile the final version of the child document
|child|\textit{nn}|.tex|:
%
\begin{center}
\begin{tabular}{l}
|\def\version{final}|\\
|\input{childdoc.def}|\\
|\childdocforwardprefix{final}{child}|
\end{tabular}
\end{center}
%

Note that when several versions of a main file and/or of each child file
are to be generated, it may be convenient to set up a |Makefile| or
shell script to automatise the process.

%%%%%%%%%%%%%%%%%%%%%%%%%%%%%%%%%%%%%%%%%%%%%%%%%%%%%%%%%%%%%%%%%%%%%%%%%%%%%%%%
\subsection{Command Line Processing}
\label{sec:commandline}

The effect of redirection files can also be achieved by invoking
the \LaTeX{} compiler with a more elaborate command line.
Most conveniently this should be done as part
of a shell script or a |Makefile|.

When using \textsf{childdoc} in the main file, the following
command lines effectively perform a redirection
(note that depending on the shell being used,
backslashes may have to be doubled: `|\|' $\to$ `|\\|'):
%
\begin{center}
|... -jobname "|\textit{target}|" |\\|"|[\textit{flags}]%
|\input{childdoc.def}\childdocforward[|\textit{main}|]{|\textit{dest}|}"|
\end{center}
%
Here \textit{target} is the name of the output file,
\textit{main} is the name of the main file
and \textit{dest} is the name of the main or child file to be processed
(all filenames without extensions).
The optional argument \textit{main} can be omitted
if \textit{main} matches \textit{dest}.
Optionally, compilation \textit{flags} can be defined via |\def| commands.
This command line makes the \TeX{} engine believe
it is compiling the file \textit{target}
whose content is specified as the latter parameter.
The provided code then forwards the processing to
\textit{main} or \textit{dest} as described in \secref{sec:forward}.

%%%%%%%%%%%%%%%%%%%%%%%%%%%%%%%%%%%%%%%%%%%%%%%%%%%%%%%%%%%%%%%%%%%%%%%%%%%%%%%%
\subsection{Include by Input}
\label{sec:input}

Including child documents by |\include| has some restrictions by design.
Most notably, the content of a child document always occupies
its own set of pages; pages cannot be shared between child documents.
Usually, this behaviour makes perfect sense
because each child document contain an essential part of the document.
However, in some situations it may be desirable to compose
a document from a collection of parts
without having mandatory page breaks between then.
For this case, the package
provides a mechanism to include parts
by |\input| which can also be processed individually.
However, by construction this mechanism
requires manual handling of the content to be output.

%%%%%%%%%%%%%%%%%%%%%%%%%%%%%%%%%%%%%%%%
\DescribeMacro{\ifchilddocmanual}
The main file should be prepared as usual, see \secref{sec:include}.
However, the document body must make a distinction
between processing of an individual part and of the main document, e.g.:
%
\begin{center}
\begin{tabular}{l}
|\ifchilddocmanual|\\
|\input{\childdocname}|\\
|\||else|\\
\textit{document body with }|\input{|\textit{part}|}|\\
|\||fi|
\end{tabular}
\end{center}
%
The conditional |\ifchilddocmanual| is true whenever
a part to be included by |\input| is being compiled,
and the name of the part is stored in |\childdocname|.

%%%%%%%%%%%%%%%%%%%%%%%%%%%%%%%%%%%%%%%%
\DescribeMacro{\childdocby}
Each part to be included by |\input| should start with:
%
\begin{center}
\begin{tabular}{l}
|\input{childdoc.def}|\\
|\childdocby{|\textit{main}|}|\\
\end{tabular}
\end{center}
%
The directive |\childdocby| is similar to |\childdocof|
described in \secref{sec:include},
but the subsequent selection of content must be done manually.
To that end, both |\ifchilddoc| and |\ifchilddocmanual|
will be true upon processing of a part,
and the name of the part is stored in |\childdocname|.
Note that |\jobname| will be set to the filename of the current part
so that each part receives an individual |.aux| file
that does not interfere with the |.aux| file(s) of the main document.
This behaviour can be altered by the alternative form
|\childdocby[*]{|\textit{main}|}| (with a non-empty optional argument)
which uses the |.aux| file of the main document
by setting |\jobname| to \textit{main}.

%%%%%%%%%%%%%%%%%%%%%%%%%%%%%%%%%%%%%%%%%%%%%%%%%%%%%%%%%%%%%%%%%%%%%%%%%%%%%%%%
\subsection{Driver Development}
\label{sec:driver}

The \textsf{childdoc} mechanism can also be use for the development
of definition files such as \LaTeX{} styles or classes.
This case differs from the above setup with multiple parts
included by |\include| in that no |\includeonly| should be invoked.
This can be achieved by starting the include file
(before |\ProvidesPackage|) with:
%
\begin{center}
\begin{tabular}{l}
|\input{childdoc.def}|\\
|\childdocforward{|\textit{main}|}|\\
\end{tabular}
\end{center}
%
or alternatively with:
%
\begin{center}
\begin{tabular}{l}
|\input{childdoc.def}|\\
|\childdocby{|\textit{main}|}|\\
\end{tabular}
\end{center}
%
Both forms have slightly different effects as described above.
The main file is prepared as usual, see \secref{sec:include}.

%%%%%%%%%%%%%%%%%%%%%%%%%%%%%%%%%%%%%%%%%%%%%%%%%%%%%%%%%%%%%%%%%%%%%%%%%%%%%%%%
\subsection{Legacy Detection}
\label{sec:detection}

The directive |\childdocmain| in the main file can detect
whether the complete document or merely a child is to be compiled
even without using the directive |\childdocof|.
This method is deprecated because it is less robust
and there is no compelling reason to use it;
it is merely provided for backward compatibility
and it may be removed in future versions.

If the detection mechanism is to be used,
it is mandatory to correctly specify
the filename of the main file as the argument of |\childdocmain|:
%
\begin{center}
\begin{tabular}{l}
|\input{childdoc.def}|\\
|\childdocmain{|\textit{main}|}|\\
\end{tabular}
\end{center}
%
If |\jobname| does not match the argument \textit{main} of |\childdocmain|,
it is assumed that |\jobname| points to the child file to be compiled.
When using |\childdocmain| with the main file specified as argument,
it suffices to start a child file
with just |\input{|\textit{main}|}|
without loading of the package and using |\childdocof|.
If instead all processing is done
with the appropriate \textsf{childdoc} directives,
the argument of \textit{main} of |\childdocmain| can be empty.

An alternative version of the command line processing described
in \secref{sec:commandline} using the detection mechanism reads:
%
\begin{center}
|... -jobname "|\textit{target}|" "|[\textit{flags}]%
[|\def\jobname{|\textit{dest}|}|]|\input{|\textit{main}|}"|
\end{center}

%%%%%%%%%%%%%%%%%%%%%%%%%%%%%%%%%%%%%%%%%%%%%%%%%%%%%%%%%%%%%%%%%%%%%%%%%%%%%%%%
\subsection{Manual Code}
\label{sec:manual}

In case one cannot be certain whether the definitions file |childdoc.def|
is installed on the target \TeX{} distribution
and one prefers not to ship it,
it is conceivable to paste a few relevant commands into the sources.

To that end, drop all statements |\input{childdoc.def}|
and perform the replacements as outlined below.
Instead of |\childdocmain{|\textit{main}|}| add the following code
to the top of the main file:
%
\begin{center}
\begin{tabular}{l}
|\||ifdefined\childdocname\endinput\||fi\newif\ifchilddoc|\\
|\edef\childdocname{\scantokens\expandafter{\jobname\noexpand}}|\\
|\def\childdocmain{|\textit{main}|}\||ifx\childdocmain\childdocname\||else|\\
|\childdoctrue\includeonly{\childdocname}\let\jobname\childdocmain\||fi|\\
\end{tabular}
\end{center}
%
Instead of |\childdocof{|\textit{main}|}| just include the main file
at the top of each child file:
%
\begin{center}
|\input{|\textit{main}|}|
\end{center}
%
A simple redirection |\childdocforward{|\textit{dest}|}| is achieved by:
%
\begin{center}
|\def\jobname{|\textit{dest}|}\input{\jobname}|
\end{center}
%
The redirection with prefix
|\childdocforwardprefix[|\textit{prefix}|]{|\textit{dest}|}|
is accomplished by:
%
\begin{center}
\begin{tabular}{l}
|{\edef\jobname{\scantokens\expandafter{\jobname\noexpand}}|\\
|\def\redirectjob |\textit{prefix}|#1~~~{\gdef\jobname{|\textit{dest}|#1}}|\\
|\expandafter\redirectjob\jobname~~~}\input{\jobname}|
\end{tabular}
\end{center}

In an alternative approach,
child documents can be compiled by a specific command line
without additional code or specific definitions:
%
\begin{center}
|... -jobname "|\textit{target}|" "|[\textit{flags}]%
|\includeonly{|\textit{dest}|}\input{|\textit{main}|}"|
\end{center}
%

%%%%%%%%%%%%%%%%%%%%%%%%%%%%%%%%%%%%%%%%%%%%%%%%%%%%%%%%%%%%%%%%%%%%%%%%%%%%%%%%
%%%%%%%%%%%%%%%%%%%%%%%%%%%%%%%%%%%%%%%%%%%%%%%%%%%%%%%%%%%%%%%%%%%%%%%%%%%%%%%%
\section{Information}

%%%%%%%%%%%%%%%%%%%%%%%%%%%%%%%%%%%%%%%%%%%%%%%%%%%%%%%%%%%%%%%%%%%%%%%%%%%%%%%%
\subsection{Copyright}

Copyright \copyright{} 2017--2018 Niklas Beisert

This work may be distributed and/or modified under the
conditions of the \LaTeX{} Project Public License, either version 1.3
of this license or (at your option) any later version.
The latest version of this license is in
  \url{http://www.latex-project.org/lppl.txt}
and version 1.3 or later is part of all distributions of \LaTeX{}
version 2005/12/01 or later.

This work has the LPPL maintenance status `maintained'.

The Current Maintainer of this work is Niklas Beisert.

This work consists of the files |README.txt|, |childdoc.ins| and |childdoc.dtx|
as well as the derived files |childdoc.def|, |cdocsamp.tex|
with |cdocsch1.tex|, |cdocsch2.tex|, |cdocspt3.tex|, |cdocspt4.tex|,
|cdocsdrf.tex|, |cdocsfn1.tex|, |cdocsfn2.tex|
as well as |childdoc.pdf|.

%%%%%%%%%%%%%%%%%%%%%%%%%%%%%%%%%%%%%%%%%%%%%%%%%%%%%%%%%%%%%%%%%%%%%%%%%%%%%%%%
\subsection{Files and Installation}

The package consists of the files:
%
\begin{center}
\begin{tabular}{ll}
    |README.txt|   & readme file \\
    |childdoc.ins| & installation file \\
    |childdoc.dtx| & source file \\
    |childdoc.def| & definition file \\
    |cdocsamp.tex| & sample main file \\
    |cdocsch1.tex| & sample include file \\
    |cdocsch2.tex| & sample include file \\
    |cdocspt3.tex| & sample part file \\
    |cdocspt4.tex| & sample part file \\
    |cdocsdrf.tex| & sample redirection file \\
    |cdocsfn1.tex| & sample redirection file \\
    |cdocsfn2.tex| & sample redirection file \\
    |childdoc.pdf| & manual
\end{tabular}
\end{center}
%
The distribution consists of the files
|README.txt|, |childdoc.ins| and |childdoc.dtx|.
%
\begin{itemize}
\item
Run (pdf)\LaTeX{} on |childdoc.dtx|
to compile the manual |childdoc.pdf| (this file).
\item
Run \LaTeX{} on |childdoc.ins| to create the definitions file |childdoc.def|
and the sample |cdocsamp.tex| with include files
|cdocsch1.tex|, |cdocsch2.tex|, |cdocspt3.tex|, |cdocspt4.tex|,
|cdocsdrf.tex|, |cdocsfn1.tex|, |cdocsfn2.tex|.
Then copy the file |childdoc.def| to an appropriate directory of your \LaTeX{}
distribution, e.g.\ \textit{texmf-root}|/tex/latex/childdoc|.
\end{itemize}

%%%%%%%%%%%%%%%%%%%%%%%%%%%%%%%%%%%%%%%%%%%%%%%%%%%%%%%%%%%%%%%%%%%%%%%%%%%%%%%%
\subsection{Related CTAN Packages}

There are several other packages which offer a similar functionality:
%
\begin{itemize}
\item
The packages
\href{http://ctan.org/pkg/docmute}{\textsf{docmute}},
\href{http://ctan.org/pkg/includex}{\textsf{includex}} and
\href{http://ctan.org/pkg/standalone}{\textsf{standalone}}
provide commands to include only the document body of
a child file thus allowing both files to be compiled individually.
\item
The packages \href{http://ctan.org/pkg/subdocs}{\textsf{subdocs}}
and \href{http://ctan.org/pkg/subfiles}{\textsf{subfiles}}
provide structures in which the main and child documents can be
encapsulated and allowing them to be compiled individually.
The inclusion mechanism is different from the conventional |\include|.
\item
The package \href{http://ctan.org/pkg/combine}{\textsf{combine}}
is an elaborate solution to combine several documents into one.
\end{itemize}
%
See also the CTAN topic \href{http://ctan.org/topic/subdocs}{\textsf{subdocs}}
for further related packages.
The present package differs from the above solutions in that
a document structure constructed with the conventional |\include| mechanism
just needs two extra commands at the top of every file
such that all constituent files can be compiled individually.

%%%%%%%%%%%%%%%%%%%%%%%%%%%%%%%%%%%%%%%%%%%%%%%%%%%%%%%%%%%%%%%%%%%%%%%%%%%%%%%%
%\subsection{Feature Suggestions}
%
%The following is a list of features which may be useful for future
%versions of this package:
%%
%\begin{itemize}
%\item
%\ldots
%\end{itemize}

%%%%%%%%%%%%%%%%%%%%%%%%%%%%%%%%%%%%%%%%%%%%%%%%%%%%%%%%%%%%%%%%%%%%%%%%%%%%%%%%
\subsection{Revision History}

%%%%%%%%%%%%%%%%%%%%%%%%%%%%%%%%%%%%%%%%
\paragraph{v2.0:} 2018/12/30

\begin{itemize}
\item
immediate forward processing
\item
added |\childdocby| mechanism
\item
manual restructured
\end{itemize}

%%%%%%%%%%%%%%%%%%%%%%%%%%%%%%%%%%%%%%%%
\paragraph{v1.6:} 2018/01/17

\begin{itemize}
\item
application for development of include files
\item
corrections to manual
\end{itemize}

%%%%%%%%%%%%%%%%%%%%%%%%%%%%%%%%%%%%%%%%
\paragraph{v1.5:} 2017/05/21

\begin{itemize}
\item
more complete structuring introduced
\item
|\childdocof| introduced
\item
|\childdoc| renamed to |\childdocmain|
\item
|\childredirect| renamed to |\childdocforward| and |\childdocforwardprefix|
and functionality expanded
\end{itemize}

%%%%%%%%%%%%%%%%%%%%%%%%%%%%%%%%%%%%%%%%
\paragraph{v1.0:} 2017/04/27

\begin{itemize}
\item
manual and install package
\item
first version published on CTAN
\end{itemize}

%%%%%%%%%%%%%%%%%%%%%%%%%%%%%%%%%%%%%%%%
\paragraph{v0.6:} 2017/04/26

\begin{itemize}
\item
redirection mechanism added
\end{itemize}

%%%%%%%%%%%%%%%%%%%%%%%%%%%%%%%%%%%%%%%%
\paragraph{v0.5:} 2017/04/26

\begin{itemize}
\item
functionality in definition file
\end{itemize}


%%%%%%%%%%%%%%%%%%%%%%%%%%%%%%%%%%%%%%%%%%%%%%%%%%%%%%%%%%%%%%%%%%%%%%%%%%%%%%%%
%%%%%%%%%%%%%%%%%%%%%%%%%%%%%%%%%%%%%%%%%%%%%%%%%%%%%%%%%%%%%%%%%%%%%%%%%%%%%%%%
%%%%%%%%%%%%%%%%%%%%%%%%%%%%%%%%%%%%%%%%%%%%%%%%%%%%%%%%%%%%%%%%%%%%%%%%%%%%%%%%
\appendix

\settowidth\MacroIndent{\rmfamily\scriptsize 000\ }

 \DocInput{childdoc.dtx}

\end{document}
%</driver>
% \fi
%
% %%%%%%%%%%%%%%%%%%%%%%%%%%%%%%%%%%%%%%%%%%%%%%%%%%%%%%%%%%%%%%%%%%%%%%%%%%%%%%
% %%%%%%%%%%%%%%%%%%%%%%%%%%%%%%%%%%%%%%%%%%%%%%%%%%%%%%%%%%%%%%%%%%%%%%%%%%%%%%
% \section{Sample}
%\iffalse
%<*samplemain>
%\fi
%
% The following presents a sample document
% with two chapters, two parts, a title page,
% a compile flag as well as three forwarding files to set the flag.
% It consists of eight |.tex| files:
% \begin{center}
% \begin{tabular}{ll}
% |cdocsamp.tex|&main file\\
% |cdocsch1.tex|&include file for chapter 1\\
% |cdocsch2.tex|&include file for chapter 2\\
% |cdocspt3.tex|&include file for part 3\\
% |cdocspt4.tex|&include file for part 4\\
% |cdocsdrf.tex|&forwarding file for main file in draft mode\\
% |cdocsfi1.tex|&forwarding file for final version of chapter 1\\
% |cdocsfi2.tex|&forwarding file for final version of chapter 2\\
% \end{tabular}
% \end{center}
% Each of the eight files can be compiled directly by the \LaTeX{} compiler.
%
% %%%%%%%%%%%%%%%%%%%%%%%%%%%%%%%%%%%%%%
% \paragraph{Main File.}
%
% The main file is called |cdocsamp.tex|.
%
% Load the \textsf{childdoc} definitions and
% declare the filename for the main document:
%    \begin{macrocode}
\input{childdoc.def}
\childdocmain{}
%    \end{macrocode}

% Optional override for |\version| flag:
%    \begin{macrocode}
%%\ifchilddoc\else\providecommand{\version}{draft}\fi
%    \end{macrocode}

% Define the default values for the |\version| flag
% (|final| for the main file and |draft| for childs):
%    \begin{macrocode}
\ifchilddoc
\providecommand{\version}{draft}
\else
\providecommand{\version}{final}
\fi
%    \end{macrocode}

% Load the standard document class:
%    \begin{macrocode}
\documentclass[12pt]{article}
%    \end{macrocode}

% Start the document body:
%    \begin{macrocode}
\begin{document}
%    \end{macrocode}

% Declare a title page.
% Print title, part of document being processed and version flag:
%    \begin{macrocode}
\addtocounter{page}{-1}
\begin{center}
{\LARGE\bfseries{}childdoc example\par}
\vspace{1cm}
\ifchilddoc
\ifchilddocmanual part\else chapter\fi:
`\childdocname' of `\childdocjob'\par
\else
main document: `\childdocjob'\par
\fi
version: \version\par
\end{center}
\newpage
%    \end{macrocode}

% Manually include selected file,
% otherwise process as usual:
%    \begin{macrocode}
\ifchilddocmanual
\section*{part `\childdocname'}
\input{\childdocname}
\else
%    \end{macrocode}

% Include the two chapters:
%    \begin{macrocode}
\include{cdocsch1}
\include{cdocsch2}
%    \end{macrocode}

% Include the two parts unless only chapters should be displayed:
%    \begin{macrocode}
\ifchilddoc\else
\section{part three}
\input{cdocspt3}
\section{part four}
\input{cdocspt4}
\fi
%    \end{macrocode}

% Process as usual until here:
%    \begin{macrocode}
\fi
%    \end{macrocode}

% End of document body:
%    \begin{macrocode}
\end{document}
%    \end{macrocode}
%\iffalse
%</samplemain>
%\fi
%
% %%%%%%%%%%%%%%%%%%%%%%%%%%%%%%%%%%%%%%
% \paragraph{Chapter Include Files.}
%
% The include files are called |cdocsch1.tex| and |cdocsch2.tex|.
%
%\iffalse
%<*samplechap1|samplechap2>
%\fi

% Optional override for |\version| flag:
%    \begin{macrocode}
%%\providecommand{\version}{final}
%    \end{macrocode}

% Include the main document:
%    \begin{macrocode}
\input{childdoc.def}
\childdocof{cdocsamp}
%    \end{macrocode}

%\iffalse
%</samplechap1|samplechap2>
%\fi
%
%\iffalse
%<*samplechap1>
%\fi
% Some text for chapter 1:
%    \begin{macrocode}
\section{one}
some text in chapter one
%    \end{macrocode}

%\iffalse
%</samplechap1>
%\fi
% Some text for chapter 2:
%\iffalse
%<*samplechap2>
%\fi
%    \begin{macrocode}
\section{two}
more text in chapter two
%    \end{macrocode}

%\iffalse
%</samplechap2>
%\fi
%
% %%%%%%%%%%%%%%%%%%%%%%%%%%%%%%%%%%%%%%
% \paragraph{Part Include Files.}
%
% The include files are called |cdocspt3.tex| and |cdocspt4.tex|.
%
%\iffalse
%<*samplepart3|samplepart4>
%\fi

% Optional override for |\version| flag:
%    \begin{macrocode}
%%\providecommand{\version}{final}
%    \end{macrocode}

% Include the main document:
%    \begin{macrocode}
\input{childdoc.def}
\childdocby{cdocsamp}
%    \end{macrocode}

%\iffalse
%</samplepart3|samplepart4>
%\fi
%
%\iffalse
%<*samplepart3>
%\fi
% Some text for part 3:
%    \begin{macrocode}
some text in part three
%    \end{macrocode}

%\iffalse
%</samplepart3>
%\fi
% Some text for part 4:
%\iffalse
%<*samplepart4>
%\fi
%    \begin{macrocode}
more text in part four
%    \end{macrocode}

%\iffalse
%</samplepart4>
%\fi
%
% %%%%%%%%%%%%%%%%%%%%%%%%%%%%%%%%%%%%%%
% \paragraph{Forwarding for a Complete Draft.}
%
% The following forwarding file |cdocsdrf.tex|
% compiles the main document in draft mode:
%\iffalse
%<*sampledraft>
%\fi
%    \begin{macrocode}
\def\version{draft}
\input{childdoc.def}
\childdocforward{cdocsamp}
%    \end{macrocode}

%\iffalse
%</sampledraft>
%\fi
%
% %%%%%%%%%%%%%%%%%%%%%%%%%%%%%%%%%%%%%%
% \paragraph{Forwarding for Final Version of the Chapters.}
%
% The following forwarding files |cdocsfn1.tex| and |cdocsfn2.tex|
% (with identical content)
% compile the final versions of the child documents
% |cdocsch1.tex| and |cdocsch2.tex|, respectively:
%\iffalse
%<*samplefinal>
%\fi
%    \begin{macrocode}
\def\version{final}
\input{childdoc.def}
\childdocforwardprefix[cdocsamp]{cdocsfn}{cdocsch}
%    \end{macrocode}

%\iffalse
%</samplefinal>
%\fi
%
% %%%%%%%%%%%%%%%%%%%%%%%%%%%%%%%%%%%%%%
% \paragraph{Command Line Processing.}
%
% The following three command lines generate the output files
% |cdocscld|, |cdocscl1| and |cdocscl2|
% which should be identical to
% |cdocsdrf|, |cdocsch1| and |cdocsfn2|, respectively:
% \begin{center}
% \begin{tabular}{l}
% |latex -jobname cdocscld \|\\
% |  "\def\version{draft}\input{childdoc.def}\childdocforward{cdocsamp}"|\\
% |latex -jobname cdocscl1 \|\\
% |  "\input{childdoc.def}\childdocforward[cdocsamp]{cdocsch1}"|\\
% |latex -jobname cdocscl2 \|\\
% |  "\def\version{final}\input{childdoc.def}\childdocforward{cdocsch2}"|
% \end{tabular}
% \end{center}
% Note that the trailing backslash on each first line
% merely continues the input to the second line
% (for convenient cut ant paste).
% Furthermore, the command |latex| can be replaced by any
% of its alternative versions such as |pdflatex|.
%
% %%%%%%%%%%%%%%%%%%%%%%%%%%%%%%%%%%%%%%%%%%%%%%%%%%%%%%%%%%%%%%%%%%%%%%%%%%%%%%
% %%%%%%%%%%%%%%%%%%%%%%%%%%%%%%%%%%%%%%%%%%%%%%%%%%%%%%%%%%%%%%%%%%%%%%%%%%%%%%
% \section{Implementation}
%\iffalse
%<*package>
%\fi
%
% This section describes the definitions file |childdoc.def|.

% The definitions cannot be loaded using |\usepackage| or |\RequirePackage|
% which has a mechanism to prevent loading a style file more than once.
% When loading the definitions by means of |\input|
% multiple instances have to be prevented manually:
%\iffalse
%This code needs to be before the `\ProvidesFile' directive
%which is defined at the beginning of this file.
%Therefore it is also placed there and commented out here.
%</package>
%<*discard>
%\fi
%    \begin{macrocode}
\ifdefined\childdocmain\endinput\fi
%    \end{macrocode}
%\iffalse
%</discard>
%<*package>
%\fi
%
% \macro{\ifchilddoc}
% \macro{\ifchilddocmanual}
% The conditional |\ifchilddoc| tells whether a
% child (true) or main (false) document is being compiled.
% The conditional |\ifchilddocmanual| tells whether
% the |\includeonly| mechanism is used (false) or
% the selection of child files must be performed manually (true).
% The definitions initialise to false:
%    \begin{macrocode}
\newif\ifchilddoc
\newif\ifchilddocmanual
%    \end{macrocode}

% \macro{\childdocname}
% \macro{\childdocjob}
% The macro |\childdocname| stores the name of the main document
% to be compiled. The macro |\childdocjob| stores the name of
% the document on which the \LaTeX{} compiler was originally invoked.
% The content of |\jobname| cannot be compared
% to filenames specified in the source due to different catcodes.
% The following code rescans |\jobname|, stores the result
% in |\childdocname| and saves a copy in |\childdocjob|:
%    \begin{macrocode}
\edef\childdocname{\scantokens\expandafter{\jobname\noexpand}}
\let\childdocjob\childdocname
%    \end{macrocode}

% \macro{\childdocdisable}
% The macro |\childdocdisable| prevents the main file
% from being processed more than once.
% At this stage, the main document command |\childdocmain|
% is assumed to be called once again where it should do nothing.
% Any subsequent call to it should prevent
% a secondary processing of the main document
% It overwrites the forwarding commands
% |\childdocof| and |\childdocforward|
% with empty macros to prevent further inclusions of the main document:
%    \begin{macrocode}
\newcommand{\childdocdisable}
{
  \renewcommand{\childdocmain}[1]{\renewcommand{\childdocmain}[1]{\endinput}}
  \renewcommand{\childdocof}[1]{}
  \renewcommand{\childdocby}[2][]{}
  \renewcommand{\childdocforward}[2][]{}
  \renewcommand{\childdocdisable}{}
}
%    \end{macrocode}

% \macro{\childdocmain}
% The macro |\childdocmain| is to be called at the top of the main file
% with nothing or the main filename (without extension) as argument.
% First, it breaks loops.
% If the argument is not empty and does not match |\childdocname|
% (which is set by the first inclusion of |childdoc.def|),
% |\ifchilddoc| is set to true, |\includeonly| is applied to the child file
% and |\jobname| is set to the main file
% (for proper handling of |.aux| files):
%    \begin{macrocode}
\newcommand{\childdocmain}[1]
{
  \childdocdisable\childdocmain{}
  \if?#1?\else
    \begingroup
      \def\childdoctmp{#1}
      \ifx\childdoctmp\childdocname
        \def\childdoctmp{}
      \else
        \def\childdoctmp
        {
          \childdoctrue
          \includeonly{\childdocname}
          \def\childdocjob{#1}
          \def\jobname{#1}
        }
      \fi
      \expandafter
    \endgroup
    \childdoctmp
  \fi
}
%    \end{macrocode}

% \macro{\childdocof}
% The command |\childdocof| redirects
% compilation to the main file |#1|.
%    \begin{macrocode}
\newcommand{\childdocof}[1]
{
  \childdocdisable
  \childdoctrue
  \includeonly{\childdocname}
  \def\jobname{#1}
  \def\childdocjob{#1}
  \input{#1}
}
%    \end{macrocode}

% \macro{\childdocby}
% The command |\childdocby| ....
%    \begin{macrocode}
\newcommand{\childdocby}[2][]
{
  \childdocdisable
  \childdoctrue
  \childdocmanualtrue
  \if?#1?\else
    \def\jobname{#2}
  \fi
  \def\childdocjob{#2}
  \input{#2}
  \endinput
}
%    \end{macrocode}

% \macro{\childdocforward}
% The command |\childdocforward| redirects
% compilation to the main file or
% (if the optional argument is given) a child file.
% Parameters are set as if the main file
% or a child file starting with |\childdocof| was compiled.
% Then compilation is handed over to the main file:
%    \begin{macrocode}
\newcommand{\childdocforward}[2][]
{
  \begingroup
    \if?#1?
      \def\childdoctmp
      {
        \def\childdocname{#2}
        \def\childdocjob{#2}
        \def\jobname{#2}
        \input{#2}
        \endinput
      }
    \else
      \def\childdoctmp
      {
        \childdocdisable
        \def\childdocname{#2}
        \childdoctrue
        \includeonly{#2}
        \def\childdocjob{#1}
        \def\jobname{#1}
        \input{#1}
        \endinput
      }
    \fi
    \expandafter
  \endgroup
  \childdoctmp
}
%    \end{macrocode}

% \macro{\childdocforwardprefix}
% The command |\childdocforwardprefix| redirects
% compilation to the main or a child file by means of a pattern.
% The prefix |#1| in the current filename is replaced by |#2|
% and the suffix of the current filename is kept
% (it is assumed that the filename does not contain the substring `|~~~|'
% which is used as a delimiter).
% Compilation is handed over to the new file by |\childdocforward|:
%    \begin{macrocode}
\newcommand{\childdocforwardprefix}[3][]
{
  \begingroup
    \def\childdocextract #2##1~~~{\def\childdoctmp{\childdocforward[#1]{#3##1}}}
    \expandafter\childdocextract\childdocname~~~
    \expandafter
  \endgroup
  \childdoctmp
}
%    \end{macrocode}

% \macro{\childdoc}
% The deprecated macro |\childdoc| is a legacy version of |\childdocmain|:
%    \begin{macrocode}
\newcommand{\childdoc}{\childdocmain}
%    \end{macrocode}

% \macro{\childdocredirect}
% The deprecated macro |\childdocredirect| is a legacy version
% of |\childdocforward| and |\childdocforwardprefix|:
%    \begin{macrocode}
\newcommand{\childdocredirect}[2][]
{
  \begingroup
    \if?#1?
      \def\childdoctmp{\childdocforward{#2}}
    \else
      \def\childdoctmp{\childdocforwardprefix{#1}{#2}}
    \fi
    \expandafter
  \endgroup
  \childdoctmp
}
%    \end{macrocode}

%\iffalse
%</package>
%\fi
%
\endinput
|
and perform the replacements as outlined below.
Instead of |\childdocmain{|\textit{main}|}| add the following code
to the top of the main file:
%
\begin{center}
\begin{tabular}{l}
|\||ifdefined\childdocname\endinput\||fi\newif\ifchilddoc|\\
|\edef\childdocname{\scantokens\expandafter{\jobname\noexpand}}|\\
|\def\childdocmain{|\textit{main}|}\||ifx\childdocmain\childdocname\||else|\\
|\childdoctrue\includeonly{\childdocname}\let\jobname\childdocmain\||fi|\\
\end{tabular}
\end{center}
%
Instead of |\childdocof{|\textit{main}|}| just include the main file
at the top of each child file:
%
\begin{center}
|\input{|\textit{main}|}|
\end{center}
%
A simple redirection |\childdocforward{|\textit{dest}|}| is achieved by:
%
\begin{center}
|\def\jobname{|\textit{dest}|}\input{\jobname}|
\end{center}
%
The redirection with prefix
|\childdocforwardprefix[|\textit{prefix}|]{|\textit{dest}|}|
is accomplished by:
%
\begin{center}
\begin{tabular}{l}
|{\edef\jobname{\scantokens\expandafter{\jobname\noexpand}}|\\
|\def\redirectjob |\textit{prefix}|#1~~~{\gdef\jobname{|\textit{dest}|#1}}|\\
|\expandafter\redirectjob\jobname~~~}\input{\jobname}|
\end{tabular}
\end{center}

In an alternative approach,
child documents can be compiled by a specific command line
without additional code or specific definitions:
%
\begin{center}
|... -jobname "|\textit{target}|" "|[\textit{flags}]%
|\includeonly{|\textit{dest}|}\input{|\textit{main}|}"|
\end{center}
%

%%%%%%%%%%%%%%%%%%%%%%%%%%%%%%%%%%%%%%%%%%%%%%%%%%%%%%%%%%%%%%%%%%%%%%%%%%%%%%%%
%%%%%%%%%%%%%%%%%%%%%%%%%%%%%%%%%%%%%%%%%%%%%%%%%%%%%%%%%%%%%%%%%%%%%%%%%%%%%%%%
\section{Information}

%%%%%%%%%%%%%%%%%%%%%%%%%%%%%%%%%%%%%%%%%%%%%%%%%%%%%%%%%%%%%%%%%%%%%%%%%%%%%%%%
\subsection{Copyright}

Copyright \copyright{} 2017--2018 Niklas Beisert

This work may be distributed and/or modified under the
conditions of the \LaTeX{} Project Public License, either version 1.3
of this license or (at your option) any later version.
The latest version of this license is in
  \url{http://www.latex-project.org/lppl.txt}
and version 1.3 or later is part of all distributions of \LaTeX{}
version 2005/12/01 or later.

This work has the LPPL maintenance status `maintained'.

The Current Maintainer of this work is Niklas Beisert.

This work consists of the files |README.txt|, |childdoc.ins| and |childdoc.dtx|
as well as the derived files |childdoc.def|, |cdocsamp.tex|
with |cdocsch1.tex|, |cdocsch2.tex|, |cdocspt3.tex|, |cdocspt4.tex|,
|cdocsdrf.tex|, |cdocsfn1.tex|, |cdocsfn2.tex|
as well as |childdoc.pdf|.

%%%%%%%%%%%%%%%%%%%%%%%%%%%%%%%%%%%%%%%%%%%%%%%%%%%%%%%%%%%%%%%%%%%%%%%%%%%%%%%%
\subsection{Files and Installation}

The package consists of the files:
%
\begin{center}
\begin{tabular}{ll}
    |README.txt|   & readme file \\
    |childdoc.ins| & installation file \\
    |childdoc.dtx| & source file \\
    |childdoc.def| & definition file \\
    |cdocsamp.tex| & sample main file \\
    |cdocsch1.tex| & sample include file \\
    |cdocsch2.tex| & sample include file \\
    |cdocspt3.tex| & sample part file \\
    |cdocspt4.tex| & sample part file \\
    |cdocsdrf.tex| & sample redirection file \\
    |cdocsfn1.tex| & sample redirection file \\
    |cdocsfn2.tex| & sample redirection file \\
    |childdoc.pdf| & manual
\end{tabular}
\end{center}
%
The distribution consists of the files
|README.txt|, |childdoc.ins| and |childdoc.dtx|.
%
\begin{itemize}
\item
Run (pdf)\LaTeX{} on |childdoc.dtx|
to compile the manual |childdoc.pdf| (this file).
\item
Run \LaTeX{} on |childdoc.ins| to create the definitions file |childdoc.def|
and the sample |cdocsamp.tex| with include files
|cdocsch1.tex|, |cdocsch2.tex|, |cdocspt3.tex|, |cdocspt4.tex|,
|cdocsdrf.tex|, |cdocsfn1.tex|, |cdocsfn2.tex|.
Then copy the file |childdoc.def| to an appropriate directory of your \LaTeX{}
distribution, e.g.\ \textit{texmf-root}|/tex/latex/childdoc|.
\end{itemize}

%%%%%%%%%%%%%%%%%%%%%%%%%%%%%%%%%%%%%%%%%%%%%%%%%%%%%%%%%%%%%%%%%%%%%%%%%%%%%%%%
\subsection{Related CTAN Packages}

There are several other packages which offer a similar functionality:
%
\begin{itemize}
\item
The packages
\href{http://ctan.org/pkg/docmute}{\textsf{docmute}},
\href{http://ctan.org/pkg/includex}{\textsf{includex}} and
\href{http://ctan.org/pkg/standalone}{\textsf{standalone}}
provide commands to include only the document body of
a child file thus allowing both files to be compiled individually.
\item
The packages \href{http://ctan.org/pkg/subdocs}{\textsf{subdocs}}
and \href{http://ctan.org/pkg/subfiles}{\textsf{subfiles}}
provide structures in which the main and child documents can be
encapsulated and allowing them to be compiled individually.
The inclusion mechanism is different from the conventional |\include|.
\item
The package \href{http://ctan.org/pkg/combine}{\textsf{combine}}
is an elaborate solution to combine several documents into one.
\end{itemize}
%
See also the CTAN topic \href{http://ctan.org/topic/subdocs}{\textsf{subdocs}}
for further related packages.
The present package differs from the above solutions in that
a document structure constructed with the conventional |\include| mechanism
just needs two extra commands at the top of every file
such that all constituent files can be compiled individually.

%%%%%%%%%%%%%%%%%%%%%%%%%%%%%%%%%%%%%%%%%%%%%%%%%%%%%%%%%%%%%%%%%%%%%%%%%%%%%%%%
%\subsection{Feature Suggestions}
%
%The following is a list of features which may be useful for future
%versions of this package:
%%
%\begin{itemize}
%\item
%\ldots
%\end{itemize}

%%%%%%%%%%%%%%%%%%%%%%%%%%%%%%%%%%%%%%%%%%%%%%%%%%%%%%%%%%%%%%%%%%%%%%%%%%%%%%%%
\subsection{Revision History}

%%%%%%%%%%%%%%%%%%%%%%%%%%%%%%%%%%%%%%%%
\paragraph{v2.0:} 2018/12/30

\begin{itemize}
\item
immediate forward processing
\item
added |\childdocby| mechanism
\item
manual restructured
\end{itemize}

%%%%%%%%%%%%%%%%%%%%%%%%%%%%%%%%%%%%%%%%
\paragraph{v1.6:} 2018/01/17

\begin{itemize}
\item
application for development of include files
\item
corrections to manual
\end{itemize}

%%%%%%%%%%%%%%%%%%%%%%%%%%%%%%%%%%%%%%%%
\paragraph{v1.5:} 2017/05/21

\begin{itemize}
\item
more complete structuring introduced
\item
|\childdocof| introduced
\item
|\childdoc| renamed to |\childdocmain|
\item
|\childredirect| renamed to |\childdocforward| and |\childdocforwardprefix|
and functionality expanded
\end{itemize}

%%%%%%%%%%%%%%%%%%%%%%%%%%%%%%%%%%%%%%%%
\paragraph{v1.0:} 2017/04/27

\begin{itemize}
\item
manual and install package
\item
first version published on CTAN
\end{itemize}

%%%%%%%%%%%%%%%%%%%%%%%%%%%%%%%%%%%%%%%%
\paragraph{v0.6:} 2017/04/26

\begin{itemize}
\item
redirection mechanism added
\end{itemize}

%%%%%%%%%%%%%%%%%%%%%%%%%%%%%%%%%%%%%%%%
\paragraph{v0.5:} 2017/04/26

\begin{itemize}
\item
functionality in definition file
\end{itemize}


%%%%%%%%%%%%%%%%%%%%%%%%%%%%%%%%%%%%%%%%%%%%%%%%%%%%%%%%%%%%%%%%%%%%%%%%%%%%%%%%
%%%%%%%%%%%%%%%%%%%%%%%%%%%%%%%%%%%%%%%%%%%%%%%%%%%%%%%%%%%%%%%%%%%%%%%%%%%%%%%%
%%%%%%%%%%%%%%%%%%%%%%%%%%%%%%%%%%%%%%%%%%%%%%%%%%%%%%%%%%%%%%%%%%%%%%%%%%%%%%%%
\appendix

\settowidth\MacroIndent{\rmfamily\scriptsize 000\ }

 \DocInput{childdoc.dtx}

\end{document}
%</driver>
% \fi
%
% %%%%%%%%%%%%%%%%%%%%%%%%%%%%%%%%%%%%%%%%%%%%%%%%%%%%%%%%%%%%%%%%%%%%%%%%%%%%%%
% %%%%%%%%%%%%%%%%%%%%%%%%%%%%%%%%%%%%%%%%%%%%%%%%%%%%%%%%%%%%%%%%%%%%%%%%%%%%%%
% \section{Sample}
%\iffalse
%<*samplemain>
%\fi
%
% The following presents a sample document
% with two chapters, two parts, a title page,
% a compile flag as well as three forwarding files to set the flag.
% It consists of eight |.tex| files:
% \begin{center}
% \begin{tabular}{ll}
% |cdocsamp.tex|&main file\\
% |cdocsch1.tex|&include file for chapter 1\\
% |cdocsch2.tex|&include file for chapter 2\\
% |cdocspt3.tex|&include file for part 3\\
% |cdocspt4.tex|&include file for part 4\\
% |cdocsdrf.tex|&forwarding file for main file in draft mode\\
% |cdocsfi1.tex|&forwarding file for final version of chapter 1\\
% |cdocsfi2.tex|&forwarding file for final version of chapter 2\\
% \end{tabular}
% \end{center}
% Each of the eight files can be compiled directly by the \LaTeX{} compiler.
%
% %%%%%%%%%%%%%%%%%%%%%%%%%%%%%%%%%%%%%%
% \paragraph{Main File.}
%
% The main file is called |cdocsamp.tex|.
%
% Load the \textsf{childdoc} definitions and
% declare the filename for the main document:
%    \begin{macrocode}
% \iffalse
%
% childdoc.dtx Copyright (C) 2017-2018 Niklas Beisert
%
% This work may be distributed and/or modified under the
% conditions of the LaTeX Project Public License, either version 1.3
% of this license or (at your option) any later version.
% The latest version of this license is in
%   http://www.latex-project.org/lppl.txt
% and version 1.3 or later is part of all distributions of LaTeX
% version 2005/12/01 or later.
%
% This work has the LPPL maintenance status `maintained'.
%
% The Current Maintainer of this work is Niklas Beisert.
%
% This work consists of the files childdoc.dtx and childdoc.ins
% and the derived files childdoc.def and cdocsamp.tex with
% cdocsch1.tex, cdocsch2.tex, cdocsdrf.tex, cdocsfn1.tex, cdocsfn2.tex.
%
%<package>\ifdefined\childdocmain\endinput\fi
%<package>\ProvidesFile{childdoc.def}[2018/12/30 v2.0 child document driver]
%<samplemain>\ProvidesFile{cdocsamp.tex}[2018/12/30 v2.0 sample for childdoc]
%<*driver>
%\ProvidesFile{childdoc.drv}[2018/12/30 v2.0 childdoc reference manual file]
\PassOptionsToClass{10pt,a4paper}{article}
\documentclass{ltxdoc}

\usepackage[margin=35mm]{geometry}
\usepackage{hyperref}
\usepackage{hyperxmp}
\usepackage[usenames]{color}

\hypersetup{colorlinks=true}
\hypersetup{pdfstartview=FitH}
\hypersetup{pdfpagemode=UseNone}
\hypersetup{pdfsource={}}
\hypersetup{pdflang={en-UK}}
\hypersetup{pdfcopyright={Copyright 2017-2018 Niklas Beisert.
  This work may be distributed and/or modified under the
  conditions of the LaTeX Project Public License, either version 1.3
  of this license or (at your option) any later version.}}
\hypersetup{pdflicenseurl={http://www.latex-project.org/lppl.txt}}
\hypersetup{pdfcontactaddress={ETH Zurich, ITP, HIT K,
  Wolfgang-Pauli-Strasse 27}}
\hypersetup{pdfcontactpostcode={8093}}
\hypersetup{pdfcontactcity={Zurich}}
\hypersetup{pdfcontactcountry={Switzerland}}
\hypersetup{pdfcontactemail={nbeisert@itp.phys.ethz.ch}}
\hypersetup{pdfcontacturl={http://people.phys.ethz.ch/\xmptilde nbeisert/}}

\newcommand{\secref}[1]{\hyperref[#1]{section \ref*{#1}}}

\parskip1ex
\parindent0pt
\let\olditemize\itemize
\def\itemize{\olditemize\parskip0pt}

\begin{document}

\title{The \textsf{childdoc} Package}
\hypersetup{pdftitle={The childdoc Package}}
\author{Niklas Beisert\\[2ex]
  Institut f\"ur Theoretische Physik\\
  Eidgen\"ossische Technische Hochschule Z\"urich\\
  Wolfgang-Pauli-Strasse 27, 8093 Z\"urich, Switzerland\\[1ex]
  \href{mailto:nbeisert@itp.phys.ethz.ch}
  {\texttt{nbeisert@itp.phys.ethz.ch}}}
\hypersetup{pdfauthor={Niklas Beisert}}
\hypersetup{pdfsubject={Manual for the LaTeX2e Package childdoc}}
\date{30 December 2018, \textsf{v2.0}}
\maketitle

\begin{abstract}\noindent
\textsf{childdoc} is a \LaTeXe{} package
that enables the direct compilation
of document sections included by |\include|
to individual files.
\end{abstract}

\begingroup
\parskip0ex
\tableofcontents
\endgroup

%%%%%%%%%%%%%%%%%%%%%%%%%%%%%%%%%%%%%%%%%%%%%%%%%%%%%%%%%%%%%%%%%%%%%%%%%%%%%%%%
%%%%%%%%%%%%%%%%%%%%%%%%%%%%%%%%%%%%%%%%%%%%%%%%%%%%%%%%%%%%%%%%%%%%%%%%%%%%%%%%
\section{Introduction}

\LaTeX{} provides a mechanism to structure a large document (such as a book)
into a main file and several child files (containing the chapters)
using the |\include| command.
This mechanism is beneficial for documents
which span hundreds of pages in order to
make the source file(s) more manageable.
Moreover, compilation can be restricted to
selected child files by means of the |\includeonly| command.
The latter feature can be used to reduce the compilation time while editing
(this was significantly more useful in the earlier days of \LaTeX{})
or to generate a smaller document which is easier to navigate.
Another application of |\includeonly| is to generate
documents consisting of selected parts of the complete document.

However, there are a few drawbacks of the plain |\include| mechanism:
\begin{itemize}
\item
The child files cannot be compiled on their own,
they can only be compiled via the main file.
A naive editing environment
(such as a text editor with an option
to have the current file processed by \LaTeX)
may require one to switch to the main file before compiling;
attempting to compile the child file produces errors.
\item
The main file must be modified (each time)
to adjust the |\includeonly| command
to the present needs. This easily leaves the main file in a messy state.
\item
The generated document will always carry the filename
of the main document. This is inconvenient if
several child files are to be compiled and
to be kept for distribution.
\end{itemize}

The present package provides a simple interface
to make child files individually compilable by \LaTeX{}.
Compiling a child file then has the same effect as compiling
the main file with an |\includeonly| command
to select the appropriate child.
Moreover the generated document will carry the name of the child
rather than the main file.
This resolves all three above issues.

This feature is meant to make the editing of books,
thesis documents and lecture notes somewhat more convenient.
However, the package can also be used efficiently for
composing a series of documents (such as exercise sheets)
which are typically distributed individually.
It then assists the author in generating the individual documents
(potentially in different versions)
as well as a document containing the collected series.
Another application is in developing style files
or other kinds of included material
where compilation of the style file could redirect
to a sample or test file.

%%%%%%%%%%%%%%%%%%%%%%%%%%%%%%%%%%%%%%%%%%%%%%%%%%%%%%%%%%%%%%%%%%%%%%%%%%%%%%%%
%%%%%%%%%%%%%%%%%%%%%%%%%%%%%%%%%%%%%%%%%%%%%%%%%%%%%%%%%%%%%%%%%%%%%%%%%%%%%%%%
\section{Usage}

First of all, the package \textsf{childdoc} is \emph{not} a standard
\LaTeXe{} |.sty| style file! Therefore it needs to be invoked in
a non-standard way.

%%%%%%%%%%%%%%%%%%%%%%%%%%%%%%%%%%%%%%%%%%%%%%%%%%%%%%%%%%%%%%%%%%%%%%%%%%%%%%%%
\subsection{Included Files}
\label{sec:include}

%%%%%%%%%%%%%%%%%%%%%%%%%%%%%%%%%%%%%%%%
\DescribeMacro{\childdocmain}
To use the package, add the commands
\begin{center}
\begin{tabular}{l}
|\input{childdoc.def}|\\
|\childdocmain{}|\\
\end{tabular}
\end{center}
at the very top of the main \LaTeX{} file,
in particular \emph{before} the |\documentclass| statement!
The argument of |\childdocmain| should be left empty
(but it must be present).

%%%%%%%%%%%%%%%%%%%%%%%%%%%%%%%%%%%%%%%%
\DescribeMacro{\childdocof}
Furthermore, add the commands
\begin{center}
\begin{tabular}{l}
|\input{childdoc.def}|\\
|\childdocof{|\textit{main}|}|\\
\end{tabular}
\end{center}
at the top of every child file \textit{child}
which is included by |\include{|\textit{child}|}|
from within the main file
(or at least for those files to be compiled individually).
The argument \textit{main} must be the filename of the main file.

There are a couple of
considerations in setting up the main and child documents:

%%%%%%%%%%%%%%%%%%%%%%%%%%%%%%%%%%%%%%%%
\paragraph{Restrictions.}

Please note the following restrictions:
\begin{itemize}
\item
|\childdocmain| must be called with one argument \textit{main}
to ensure compatibility with earlier version of the package.
It must either be empty (|\childdocmain{}|)
or precisely match the filename of the main file in which it is specified.
See \secref{sec:detection} for further information.
\item
The filename \textit{main} must be specified without the |.tex| extension.
\item
The filename \textit{main} is case sensitive
(even in case-insensitive file systems)
due to internal string comparison.
\item
The argument \textit{main} should be fully expanded, it cannot be a macro.
\item
Subdirectories and special characters should be avoided in filenames.
\item
The command |\childdocmain{|\textit{main}|}| must be followed by a whitespace.
It should not be followed immediately by another command
or by a comment mark `|%|'.
This is because the \TeX{} parser reads the token immediately following
the argument of |\childdocmain| and puts it
at the beginning of every child section;
however, a white\-space is ignored.
\end{itemize}

%%%%%%%%%%%%%%%%%%%%%%%%%%%%%%%%%%%%%%%%
\paragraph{Content of Main File.}

It is advisable to place all content in the child files included by |\include|.
Any output contained in the main file will appear in all child documents
unless suppressed manually;
it cannot be suppressed automatically by the |\includeonly| directive
and thus should normally be avoided.
A method to include some content in the main file
by means of conditional processing is described in \secref{sec:conditional}.

%%%%%%%%%%%%%%%%%%%%%%%%%%%%%%%%%%%%%%%%
\paragraph{Page Numbering.}

When only a part of the document is compiled,
the appropriate numbering of pages
(as well as other status parameters)
is determined from the |.aux| files.
The latter contain information from previous passes.
However this information needs to propagate through
all intermediate child documents.
Therefore the page numbering in child documents may well
be inconsistent until the complete document is compiled at least once.

A useful (if unconventional) way to always ensure a consistent
page numbering is to restart the numbering in each child document
and denote the pages by `\textit{child}|.|\textit{page}'
where \textit{child} represents the chapter/section number of the child file.
This can be achieved by the command
|\numberwithin{page}{|\textit{child}|}|
of the \textsf{amsmath} package
where \textit{child} can be |chapter| or |section|
depending on the chosen structuring.
Alternatively, one can modify the macro |\thepage| appropriately
and reset the counter |page| at the start of each child file.

%%%%%%%%%%%%%%%%%%%%%%%%%%%%%%%%%%%%%%%%%%%%%%%%%%%%%%%%%%%%%%%%%%%%%%%%%%%%%%%%
\subsection{Conditional Processing}
\label{sec:conditional}

The package provides a mechanism to compile different versions
of a document. To customise the versions further some conditional processing
can come in handy to distinguish which version is being compiled.
The package provides two macros to describe the compilation context:

%%%%%%%%%%%%%%%%%%%%%%%%%%%%%%%%%%%%%%%%
\DescribeMacro{\ifchilddoc}
The conditional |\ifchilddoc| distinguishes between the compilation of
child documents and the main document:
%
\begin{center}
|\ifchilddoc |\textit{child-code}| |[|\||else |\textit{main-code}]| \||fi|
\end{center}

%%%%%%%%%%%%%%%%%%%%%%%%%%%%%%%%%%%%%%%%
\DescribeMacro{\childdocname}
\DescribeMacro{\childdocjob}
The macro |\childdocname| contains the filename (without extension)
of the main or child file being processed.
Note that |\childdocjob| will always contain the name of the main file.

%%%%%%%%%%%%%%%%%%%%%%%%%%%%%%%%%%%%%%%%
\paragraph{Title Page.}

Conditional processing can be used to include a title or banner page
in the main document when proper precautions are taken.
Importantly, the code in the main file should ensure that the page counter
(as well as other status parameters which are stored in the |.aux| files)
takes the same value after the conditional processing.
Otherwise the page numbers may take divergent values
depending on which part is compiled.

For example, a title page could be declared by:
%
\begin{center}
\begin{tabular}{l}
|\ifchilddoc\||else|\\
|\addtocounter{page}{-1}|\\
\textit{code for title page}\\
|\newpage|\\
|\||fi|
\end{tabular}
\end{center}
%
A banner page for the child documents can be generated by:
%
\begin{center}
\begin{tabular}{l}
|\ifchilddoc|\\
|\addtocounter{page}{-1}|\\
\textit{code for banner page}\\
|\newpage|\\
|\||fi|
\end{tabular}
\end{center}
%
Here one could write a message such as:
\begin{center}
|This is the part \childdocname{} of \childdocjob{}.|
\end{center}

%%%%%%%%%%%%%%%%%%%%%%%%%%%%%%%%%%%%%%%%%%%%%%%%%%%%%%%%%%%%%%%%%%%%%%%%%%%%%%%%
\subsection{Flags}
\label{sec:flags}

The package makes it easy to generate different versions
of the main or child documents.
To this end compilation flags can be defined
and assigned different default values.
They will be particularly useful in conjunction
with the forwarding mechanism described in \secref{sec:forward}.

For example, it may be useful to have a flag |\version|
which can be set to |draft| or |final|.
The document source will contain some conditional code
depending on the value of |\version|.
Suppose further, the flag should default to |final| for the main file
and to |draft| for child files
which is a natural assignment for editing the document.
This is achieved by placing the following code
in the preamble of the main document
(below the |\childdocmain| directive):
%
\begin{center}
\begin{tabular}{l}
|\ifchilddoc|\\
|\providecommand{\version}{draft}|\\
|\||else|\\
|\providecommand{\version}{final}|\\
|\||fi|
\end{tabular}
\end{center}
%
The definition by |\providecommand| makes sure
that previous definitions are not overwritten.
Further statements |\providecommand{\version}{...}|
can thus be added before the above code to override it.

For the main file, one might add a line
(between |\childdocmain| and the above block)
%
\begin{center}
|%\ifchilddoc\||else\providecommand{\version}{draft}\||fi|
\end{center}
%
which can be uncommented to produce a draft version.
Likewise one can add a line to the very top of a child file
(above the |\childdocof{|\textit{main}|}| directive)
%
\begin{center}
|%\providecommand{\version}{final}|
\end{center}
%
which can be uncommented to produce the final version of this child document.

%%%%%%%%%%%%%%%%%%%%%%%%%%%%%%%%%%%%%%%%%%%%%%%%%%%%%%%%%%%%%%%%%%%%%%%%%%%%%%%%
\subsection{Forwarding}
\label{sec:forward}

Different versions of the main or child documents
using compilation flags as described in \secref{sec:flags}
can be (permanently) stored in different files
for convenient compilation, viewing and distribution.
To this end, the package defines a command
to pass on compilation to a different file:

%%%%%%%%%%%%%%%%%%%%%%%%%%%%%%%%%%%%%%%%
\DescribeMacro{\childdocforward}
The command |\childdocforward| redirects processing to
another source file:
%
\begin{center}
\begin{tabular}{l}
|\input{childdoc.def}|\\
|\childdocforward[|\textit{main}|]{|\textit{dest}|}|\\
\end{tabular}
\end{center}
%
The argument \textit{dest} is the destination file
(without extension).
It should be the main file or one of the child files.
Note that further \textsf{childdoc} directives
such as |\childdocof| and |\childdocforward|
in the indicated file will be processed in this form.
The optional argument \textit{main}
passes on directly to the main file \textit{main}
while pretending to compile the child \textit{dest}.
This form behaves as if \textit{dest}
issues |\childdocof{|\textit{main}|}| right away,
and no further \textsf{childdoc} directives will be processed.

%%%%%%%%%%%%%%%%%%%%%%%%%%%%%%%%%%%%%%%%
\DescribeMacro{\...prefix}
In the alternative form |\childdocforwardprefix|,
%
\begin{center}
\begin{tabular}{l}
|\input{childdoc.def}|\\
|\childdocforwardprefix[|\textit{main}|]{|\textit{prefix}|}{|\textit{dest}|}|
\end{tabular}
\end{center}
%
the destination file is determined by a pattern
depending on the current file:
To make this work, the current file must be called
`{\textit{prefix}\hspace{0.2em}\textit{suffix}}'
with \textit{prefix} matching precisely the argument.
Processing is then passed on to the file
`{\textit{dest}\hspace{0.2em}\textit{suffix}}'.
Surely, the same effect is achieved by
directly specifying the
argument `{\textit{dest}\hspace{0.2em}\textit{suffix}}'
in the first form.
However, that requires to set up a different file
for each child. With the alternative form of the command
all these files can have exactly the same content
which simplifies setting them up and maintaining them.

For example, the following file |draft.tex|
with a compilation flag |\version| as described in \secref{sec:flags}
compiles the main document as a draft:
%
\begin{center}
\begin{tabular}{l}
|\def\version{draft}|\\
|\input{childdoc.def}|\\
|\childdocforward{|\textit{main}|}|
\end{tabular}
\end{center}
%
Likewise, the following files |final|\textit{nn}|.tex|
compile the final version of the child document
|child|\textit{nn}|.tex|:
%
\begin{center}
\begin{tabular}{l}
|\def\version{final}|\\
|\input{childdoc.def}|\\
|\childdocforwardprefix{final}{child}|
\end{tabular}
\end{center}
%

Note that when several versions of a main file and/or of each child file
are to be generated, it may be convenient to set up a |Makefile| or
shell script to automatise the process.

%%%%%%%%%%%%%%%%%%%%%%%%%%%%%%%%%%%%%%%%%%%%%%%%%%%%%%%%%%%%%%%%%%%%%%%%%%%%%%%%
\subsection{Command Line Processing}
\label{sec:commandline}

The effect of redirection files can also be achieved by invoking
the \LaTeX{} compiler with a more elaborate command line.
Most conveniently this should be done as part
of a shell script or a |Makefile|.

When using \textsf{childdoc} in the main file, the following
command lines effectively perform a redirection
(note that depending on the shell being used,
backslashes may have to be doubled: `|\|' $\to$ `|\\|'):
%
\begin{center}
|... -jobname "|\textit{target}|" |\\|"|[\textit{flags}]%
|\input{childdoc.def}\childdocforward[|\textit{main}|]{|\textit{dest}|}"|
\end{center}
%
Here \textit{target} is the name of the output file,
\textit{main} is the name of the main file
and \textit{dest} is the name of the main or child file to be processed
(all filenames without extensions).
The optional argument \textit{main} can be omitted
if \textit{main} matches \textit{dest}.
Optionally, compilation \textit{flags} can be defined via |\def| commands.
This command line makes the \TeX{} engine believe
it is compiling the file \textit{target}
whose content is specified as the latter parameter.
The provided code then forwards the processing to
\textit{main} or \textit{dest} as described in \secref{sec:forward}.

%%%%%%%%%%%%%%%%%%%%%%%%%%%%%%%%%%%%%%%%%%%%%%%%%%%%%%%%%%%%%%%%%%%%%%%%%%%%%%%%
\subsection{Include by Input}
\label{sec:input}

Including child documents by |\include| has some restrictions by design.
Most notably, the content of a child document always occupies
its own set of pages; pages cannot be shared between child documents.
Usually, this behaviour makes perfect sense
because each child document contain an essential part of the document.
However, in some situations it may be desirable to compose
a document from a collection of parts
without having mandatory page breaks between then.
For this case, the package
provides a mechanism to include parts
by |\input| which can also be processed individually.
However, by construction this mechanism
requires manual handling of the content to be output.

%%%%%%%%%%%%%%%%%%%%%%%%%%%%%%%%%%%%%%%%
\DescribeMacro{\ifchilddocmanual}
The main file should be prepared as usual, see \secref{sec:include}.
However, the document body must make a distinction
between processing of an individual part and of the main document, e.g.:
%
\begin{center}
\begin{tabular}{l}
|\ifchilddocmanual|\\
|\input{\childdocname}|\\
|\||else|\\
\textit{document body with }|\input{|\textit{part}|}|\\
|\||fi|
\end{tabular}
\end{center}
%
The conditional |\ifchilddocmanual| is true whenever
a part to be included by |\input| is being compiled,
and the name of the part is stored in |\childdocname|.

%%%%%%%%%%%%%%%%%%%%%%%%%%%%%%%%%%%%%%%%
\DescribeMacro{\childdocby}
Each part to be included by |\input| should start with:
%
\begin{center}
\begin{tabular}{l}
|\input{childdoc.def}|\\
|\childdocby{|\textit{main}|}|\\
\end{tabular}
\end{center}
%
The directive |\childdocby| is similar to |\childdocof|
described in \secref{sec:include},
but the subsequent selection of content must be done manually.
To that end, both |\ifchilddoc| and |\ifchilddocmanual|
will be true upon processing of a part,
and the name of the part is stored in |\childdocname|.
Note that |\jobname| will be set to the filename of the current part
so that each part receives an individual |.aux| file
that does not interfere with the |.aux| file(s) of the main document.
This behaviour can be altered by the alternative form
|\childdocby[*]{|\textit{main}|}| (with a non-empty optional argument)
which uses the |.aux| file of the main document
by setting |\jobname| to \textit{main}.

%%%%%%%%%%%%%%%%%%%%%%%%%%%%%%%%%%%%%%%%%%%%%%%%%%%%%%%%%%%%%%%%%%%%%%%%%%%%%%%%
\subsection{Driver Development}
\label{sec:driver}

The \textsf{childdoc} mechanism can also be use for the development
of definition files such as \LaTeX{} styles or classes.
This case differs from the above setup with multiple parts
included by |\include| in that no |\includeonly| should be invoked.
This can be achieved by starting the include file
(before |\ProvidesPackage|) with:
%
\begin{center}
\begin{tabular}{l}
|\input{childdoc.def}|\\
|\childdocforward{|\textit{main}|}|\\
\end{tabular}
\end{center}
%
or alternatively with:
%
\begin{center}
\begin{tabular}{l}
|\input{childdoc.def}|\\
|\childdocby{|\textit{main}|}|\\
\end{tabular}
\end{center}
%
Both forms have slightly different effects as described above.
The main file is prepared as usual, see \secref{sec:include}.

%%%%%%%%%%%%%%%%%%%%%%%%%%%%%%%%%%%%%%%%%%%%%%%%%%%%%%%%%%%%%%%%%%%%%%%%%%%%%%%%
\subsection{Legacy Detection}
\label{sec:detection}

The directive |\childdocmain| in the main file can detect
whether the complete document or merely a child is to be compiled
even without using the directive |\childdocof|.
This method is deprecated because it is less robust
and there is no compelling reason to use it;
it is merely provided for backward compatibility
and it may be removed in future versions.

If the detection mechanism is to be used,
it is mandatory to correctly specify
the filename of the main file as the argument of |\childdocmain|:
%
\begin{center}
\begin{tabular}{l}
|\input{childdoc.def}|\\
|\childdocmain{|\textit{main}|}|\\
\end{tabular}
\end{center}
%
If |\jobname| does not match the argument \textit{main} of |\childdocmain|,
it is assumed that |\jobname| points to the child file to be compiled.
When using |\childdocmain| with the main file specified as argument,
it suffices to start a child file
with just |\input{|\textit{main}|}|
without loading of the package and using |\childdocof|.
If instead all processing is done
with the appropriate \textsf{childdoc} directives,
the argument of \textit{main} of |\childdocmain| can be empty.

An alternative version of the command line processing described
in \secref{sec:commandline} using the detection mechanism reads:
%
\begin{center}
|... -jobname "|\textit{target}|" "|[\textit{flags}]%
[|\def\jobname{|\textit{dest}|}|]|\input{|\textit{main}|}"|
\end{center}

%%%%%%%%%%%%%%%%%%%%%%%%%%%%%%%%%%%%%%%%%%%%%%%%%%%%%%%%%%%%%%%%%%%%%%%%%%%%%%%%
\subsection{Manual Code}
\label{sec:manual}

In case one cannot be certain whether the definitions file |childdoc.def|
is installed on the target \TeX{} distribution
and one prefers not to ship it,
it is conceivable to paste a few relevant commands into the sources.

To that end, drop all statements |\input{childdoc.def}|
and perform the replacements as outlined below.
Instead of |\childdocmain{|\textit{main}|}| add the following code
to the top of the main file:
%
\begin{center}
\begin{tabular}{l}
|\||ifdefined\childdocname\endinput\||fi\newif\ifchilddoc|\\
|\edef\childdocname{\scantokens\expandafter{\jobname\noexpand}}|\\
|\def\childdocmain{|\textit{main}|}\||ifx\childdocmain\childdocname\||else|\\
|\childdoctrue\includeonly{\childdocname}\let\jobname\childdocmain\||fi|\\
\end{tabular}
\end{center}
%
Instead of |\childdocof{|\textit{main}|}| just include the main file
at the top of each child file:
%
\begin{center}
|\input{|\textit{main}|}|
\end{center}
%
A simple redirection |\childdocforward{|\textit{dest}|}| is achieved by:
%
\begin{center}
|\def\jobname{|\textit{dest}|}\input{\jobname}|
\end{center}
%
The redirection with prefix
|\childdocforwardprefix[|\textit{prefix}|]{|\textit{dest}|}|
is accomplished by:
%
\begin{center}
\begin{tabular}{l}
|{\edef\jobname{\scantokens\expandafter{\jobname\noexpand}}|\\
|\def\redirectjob |\textit{prefix}|#1~~~{\gdef\jobname{|\textit{dest}|#1}}|\\
|\expandafter\redirectjob\jobname~~~}\input{\jobname}|
\end{tabular}
\end{center}

In an alternative approach,
child documents can be compiled by a specific command line
without additional code or specific definitions:
%
\begin{center}
|... -jobname "|\textit{target}|" "|[\textit{flags}]%
|\includeonly{|\textit{dest}|}\input{|\textit{main}|}"|
\end{center}
%

%%%%%%%%%%%%%%%%%%%%%%%%%%%%%%%%%%%%%%%%%%%%%%%%%%%%%%%%%%%%%%%%%%%%%%%%%%%%%%%%
%%%%%%%%%%%%%%%%%%%%%%%%%%%%%%%%%%%%%%%%%%%%%%%%%%%%%%%%%%%%%%%%%%%%%%%%%%%%%%%%
\section{Information}

%%%%%%%%%%%%%%%%%%%%%%%%%%%%%%%%%%%%%%%%%%%%%%%%%%%%%%%%%%%%%%%%%%%%%%%%%%%%%%%%
\subsection{Copyright}

Copyright \copyright{} 2017--2018 Niklas Beisert

This work may be distributed and/or modified under the
conditions of the \LaTeX{} Project Public License, either version 1.3
of this license or (at your option) any later version.
The latest version of this license is in
  \url{http://www.latex-project.org/lppl.txt}
and version 1.3 or later is part of all distributions of \LaTeX{}
version 2005/12/01 or later.

This work has the LPPL maintenance status `maintained'.

The Current Maintainer of this work is Niklas Beisert.

This work consists of the files |README.txt|, |childdoc.ins| and |childdoc.dtx|
as well as the derived files |childdoc.def|, |cdocsamp.tex|
with |cdocsch1.tex|, |cdocsch2.tex|, |cdocspt3.tex|, |cdocspt4.tex|,
|cdocsdrf.tex|, |cdocsfn1.tex|, |cdocsfn2.tex|
as well as |childdoc.pdf|.

%%%%%%%%%%%%%%%%%%%%%%%%%%%%%%%%%%%%%%%%%%%%%%%%%%%%%%%%%%%%%%%%%%%%%%%%%%%%%%%%
\subsection{Files and Installation}

The package consists of the files:
%
\begin{center}
\begin{tabular}{ll}
    |README.txt|   & readme file \\
    |childdoc.ins| & installation file \\
    |childdoc.dtx| & source file \\
    |childdoc.def| & definition file \\
    |cdocsamp.tex| & sample main file \\
    |cdocsch1.tex| & sample include file \\
    |cdocsch2.tex| & sample include file \\
    |cdocspt3.tex| & sample part file \\
    |cdocspt4.tex| & sample part file \\
    |cdocsdrf.tex| & sample redirection file \\
    |cdocsfn1.tex| & sample redirection file \\
    |cdocsfn2.tex| & sample redirection file \\
    |childdoc.pdf| & manual
\end{tabular}
\end{center}
%
The distribution consists of the files
|README.txt|, |childdoc.ins| and |childdoc.dtx|.
%
\begin{itemize}
\item
Run (pdf)\LaTeX{} on |childdoc.dtx|
to compile the manual |childdoc.pdf| (this file).
\item
Run \LaTeX{} on |childdoc.ins| to create the definitions file |childdoc.def|
and the sample |cdocsamp.tex| with include files
|cdocsch1.tex|, |cdocsch2.tex|, |cdocspt3.tex|, |cdocspt4.tex|,
|cdocsdrf.tex|, |cdocsfn1.tex|, |cdocsfn2.tex|.
Then copy the file |childdoc.def| to an appropriate directory of your \LaTeX{}
distribution, e.g.\ \textit{texmf-root}|/tex/latex/childdoc|.
\end{itemize}

%%%%%%%%%%%%%%%%%%%%%%%%%%%%%%%%%%%%%%%%%%%%%%%%%%%%%%%%%%%%%%%%%%%%%%%%%%%%%%%%
\subsection{Related CTAN Packages}

There are several other packages which offer a similar functionality:
%
\begin{itemize}
\item
The packages
\href{http://ctan.org/pkg/docmute}{\textsf{docmute}},
\href{http://ctan.org/pkg/includex}{\textsf{includex}} and
\href{http://ctan.org/pkg/standalone}{\textsf{standalone}}
provide commands to include only the document body of
a child file thus allowing both files to be compiled individually.
\item
The packages \href{http://ctan.org/pkg/subdocs}{\textsf{subdocs}}
and \href{http://ctan.org/pkg/subfiles}{\textsf{subfiles}}
provide structures in which the main and child documents can be
encapsulated and allowing them to be compiled individually.
The inclusion mechanism is different from the conventional |\include|.
\item
The package \href{http://ctan.org/pkg/combine}{\textsf{combine}}
is an elaborate solution to combine several documents into one.
\end{itemize}
%
See also the CTAN topic \href{http://ctan.org/topic/subdocs}{\textsf{subdocs}}
for further related packages.
The present package differs from the above solutions in that
a document structure constructed with the conventional |\include| mechanism
just needs two extra commands at the top of every file
such that all constituent files can be compiled individually.

%%%%%%%%%%%%%%%%%%%%%%%%%%%%%%%%%%%%%%%%%%%%%%%%%%%%%%%%%%%%%%%%%%%%%%%%%%%%%%%%
%\subsection{Feature Suggestions}
%
%The following is a list of features which may be useful for future
%versions of this package:
%%
%\begin{itemize}
%\item
%\ldots
%\end{itemize}

%%%%%%%%%%%%%%%%%%%%%%%%%%%%%%%%%%%%%%%%%%%%%%%%%%%%%%%%%%%%%%%%%%%%%%%%%%%%%%%%
\subsection{Revision History}

%%%%%%%%%%%%%%%%%%%%%%%%%%%%%%%%%%%%%%%%
\paragraph{v2.0:} 2018/12/30

\begin{itemize}
\item
immediate forward processing
\item
added |\childdocby| mechanism
\item
manual restructured
\end{itemize}

%%%%%%%%%%%%%%%%%%%%%%%%%%%%%%%%%%%%%%%%
\paragraph{v1.6:} 2018/01/17

\begin{itemize}
\item
application for development of include files
\item
corrections to manual
\end{itemize}

%%%%%%%%%%%%%%%%%%%%%%%%%%%%%%%%%%%%%%%%
\paragraph{v1.5:} 2017/05/21

\begin{itemize}
\item
more complete structuring introduced
\item
|\childdocof| introduced
\item
|\childdoc| renamed to |\childdocmain|
\item
|\childredirect| renamed to |\childdocforward| and |\childdocforwardprefix|
and functionality expanded
\end{itemize}

%%%%%%%%%%%%%%%%%%%%%%%%%%%%%%%%%%%%%%%%
\paragraph{v1.0:} 2017/04/27

\begin{itemize}
\item
manual and install package
\item
first version published on CTAN
\end{itemize}

%%%%%%%%%%%%%%%%%%%%%%%%%%%%%%%%%%%%%%%%
\paragraph{v0.6:} 2017/04/26

\begin{itemize}
\item
redirection mechanism added
\end{itemize}

%%%%%%%%%%%%%%%%%%%%%%%%%%%%%%%%%%%%%%%%
\paragraph{v0.5:} 2017/04/26

\begin{itemize}
\item
functionality in definition file
\end{itemize}


%%%%%%%%%%%%%%%%%%%%%%%%%%%%%%%%%%%%%%%%%%%%%%%%%%%%%%%%%%%%%%%%%%%%%%%%%%%%%%%%
%%%%%%%%%%%%%%%%%%%%%%%%%%%%%%%%%%%%%%%%%%%%%%%%%%%%%%%%%%%%%%%%%%%%%%%%%%%%%%%%
%%%%%%%%%%%%%%%%%%%%%%%%%%%%%%%%%%%%%%%%%%%%%%%%%%%%%%%%%%%%%%%%%%%%%%%%%%%%%%%%
\appendix

\settowidth\MacroIndent{\rmfamily\scriptsize 000\ }

 \DocInput{childdoc.dtx}

\end{document}
%</driver>
% \fi
%
% %%%%%%%%%%%%%%%%%%%%%%%%%%%%%%%%%%%%%%%%%%%%%%%%%%%%%%%%%%%%%%%%%%%%%%%%%%%%%%
% %%%%%%%%%%%%%%%%%%%%%%%%%%%%%%%%%%%%%%%%%%%%%%%%%%%%%%%%%%%%%%%%%%%%%%%%%%%%%%
% \section{Sample}
%\iffalse
%<*samplemain>
%\fi
%
% The following presents a sample document
% with two chapters, two parts, a title page,
% a compile flag as well as three forwarding files to set the flag.
% It consists of eight |.tex| files:
% \begin{center}
% \begin{tabular}{ll}
% |cdocsamp.tex|&main file\\
% |cdocsch1.tex|&include file for chapter 1\\
% |cdocsch2.tex|&include file for chapter 2\\
% |cdocspt3.tex|&include file for part 3\\
% |cdocspt4.tex|&include file for part 4\\
% |cdocsdrf.tex|&forwarding file for main file in draft mode\\
% |cdocsfi1.tex|&forwarding file for final version of chapter 1\\
% |cdocsfi2.tex|&forwarding file for final version of chapter 2\\
% \end{tabular}
% \end{center}
% Each of the eight files can be compiled directly by the \LaTeX{} compiler.
%
% %%%%%%%%%%%%%%%%%%%%%%%%%%%%%%%%%%%%%%
% \paragraph{Main File.}
%
% The main file is called |cdocsamp.tex|.
%
% Load the \textsf{childdoc} definitions and
% declare the filename for the main document:
%    \begin{macrocode}
\input{childdoc.def}
\childdocmain{}
%    \end{macrocode}

% Optional override for |\version| flag:
%    \begin{macrocode}
%%\ifchilddoc\else\providecommand{\version}{draft}\fi
%    \end{macrocode}

% Define the default values for the |\version| flag
% (|final| for the main file and |draft| for childs):
%    \begin{macrocode}
\ifchilddoc
\providecommand{\version}{draft}
\else
\providecommand{\version}{final}
\fi
%    \end{macrocode}

% Load the standard document class:
%    \begin{macrocode}
\documentclass[12pt]{article}
%    \end{macrocode}

% Start the document body:
%    \begin{macrocode}
\begin{document}
%    \end{macrocode}

% Declare a title page.
% Print title, part of document being processed and version flag:
%    \begin{macrocode}
\addtocounter{page}{-1}
\begin{center}
{\LARGE\bfseries{}childdoc example\par}
\vspace{1cm}
\ifchilddoc
\ifchilddocmanual part\else chapter\fi:
`\childdocname' of `\childdocjob'\par
\else
main document: `\childdocjob'\par
\fi
version: \version\par
\end{center}
\newpage
%    \end{macrocode}

% Manually include selected file,
% otherwise process as usual:
%    \begin{macrocode}
\ifchilddocmanual
\section*{part `\childdocname'}
\input{\childdocname}
\else
%    \end{macrocode}

% Include the two chapters:
%    \begin{macrocode}
\include{cdocsch1}
\include{cdocsch2}
%    \end{macrocode}

% Include the two parts unless only chapters should be displayed:
%    \begin{macrocode}
\ifchilddoc\else
\section{part three}
\input{cdocspt3}
\section{part four}
\input{cdocspt4}
\fi
%    \end{macrocode}

% Process as usual until here:
%    \begin{macrocode}
\fi
%    \end{macrocode}

% End of document body:
%    \begin{macrocode}
\end{document}
%    \end{macrocode}
%\iffalse
%</samplemain>
%\fi
%
% %%%%%%%%%%%%%%%%%%%%%%%%%%%%%%%%%%%%%%
% \paragraph{Chapter Include Files.}
%
% The include files are called |cdocsch1.tex| and |cdocsch2.tex|.
%
%\iffalse
%<*samplechap1|samplechap2>
%\fi

% Optional override for |\version| flag:
%    \begin{macrocode}
%%\providecommand{\version}{final}
%    \end{macrocode}

% Include the main document:
%    \begin{macrocode}
\input{childdoc.def}
\childdocof{cdocsamp}
%    \end{macrocode}

%\iffalse
%</samplechap1|samplechap2>
%\fi
%
%\iffalse
%<*samplechap1>
%\fi
% Some text for chapter 1:
%    \begin{macrocode}
\section{one}
some text in chapter one
%    \end{macrocode}

%\iffalse
%</samplechap1>
%\fi
% Some text for chapter 2:
%\iffalse
%<*samplechap2>
%\fi
%    \begin{macrocode}
\section{two}
more text in chapter two
%    \end{macrocode}

%\iffalse
%</samplechap2>
%\fi
%
% %%%%%%%%%%%%%%%%%%%%%%%%%%%%%%%%%%%%%%
% \paragraph{Part Include Files.}
%
% The include files are called |cdocspt3.tex| and |cdocspt4.tex|.
%
%\iffalse
%<*samplepart3|samplepart4>
%\fi

% Optional override for |\version| flag:
%    \begin{macrocode}
%%\providecommand{\version}{final}
%    \end{macrocode}

% Include the main document:
%    \begin{macrocode}
\input{childdoc.def}
\childdocby{cdocsamp}
%    \end{macrocode}

%\iffalse
%</samplepart3|samplepart4>
%\fi
%
%\iffalse
%<*samplepart3>
%\fi
% Some text for part 3:
%    \begin{macrocode}
some text in part three
%    \end{macrocode}

%\iffalse
%</samplepart3>
%\fi
% Some text for part 4:
%\iffalse
%<*samplepart4>
%\fi
%    \begin{macrocode}
more text in part four
%    \end{macrocode}

%\iffalse
%</samplepart4>
%\fi
%
% %%%%%%%%%%%%%%%%%%%%%%%%%%%%%%%%%%%%%%
% \paragraph{Forwarding for a Complete Draft.}
%
% The following forwarding file |cdocsdrf.tex|
% compiles the main document in draft mode:
%\iffalse
%<*sampledraft>
%\fi
%    \begin{macrocode}
\def\version{draft}
\input{childdoc.def}
\childdocforward{cdocsamp}
%    \end{macrocode}

%\iffalse
%</sampledraft>
%\fi
%
% %%%%%%%%%%%%%%%%%%%%%%%%%%%%%%%%%%%%%%
% \paragraph{Forwarding for Final Version of the Chapters.}
%
% The following forwarding files |cdocsfn1.tex| and |cdocsfn2.tex|
% (with identical content)
% compile the final versions of the child documents
% |cdocsch1.tex| and |cdocsch2.tex|, respectively:
%\iffalse
%<*samplefinal>
%\fi
%    \begin{macrocode}
\def\version{final}
\input{childdoc.def}
\childdocforwardprefix[cdocsamp]{cdocsfn}{cdocsch}
%    \end{macrocode}

%\iffalse
%</samplefinal>
%\fi
%
% %%%%%%%%%%%%%%%%%%%%%%%%%%%%%%%%%%%%%%
% \paragraph{Command Line Processing.}
%
% The following three command lines generate the output files
% |cdocscld|, |cdocscl1| and |cdocscl2|
% which should be identical to
% |cdocsdrf|, |cdocsch1| and |cdocsfn2|, respectively:
% \begin{center}
% \begin{tabular}{l}
% |latex -jobname cdocscld \|\\
% |  "\def\version{draft}\input{childdoc.def}\childdocforward{cdocsamp}"|\\
% |latex -jobname cdocscl1 \|\\
% |  "\input{childdoc.def}\childdocforward[cdocsamp]{cdocsch1}"|\\
% |latex -jobname cdocscl2 \|\\
% |  "\def\version{final}\input{childdoc.def}\childdocforward{cdocsch2}"|
% \end{tabular}
% \end{center}
% Note that the trailing backslash on each first line
% merely continues the input to the second line
% (for convenient cut ant paste).
% Furthermore, the command |latex| can be replaced by any
% of its alternative versions such as |pdflatex|.
%
% %%%%%%%%%%%%%%%%%%%%%%%%%%%%%%%%%%%%%%%%%%%%%%%%%%%%%%%%%%%%%%%%%%%%%%%%%%%%%%
% %%%%%%%%%%%%%%%%%%%%%%%%%%%%%%%%%%%%%%%%%%%%%%%%%%%%%%%%%%%%%%%%%%%%%%%%%%%%%%
% \section{Implementation}
%\iffalse
%<*package>
%\fi
%
% This section describes the definitions file |childdoc.def|.

% The definitions cannot be loaded using |\usepackage| or |\RequirePackage|
% which has a mechanism to prevent loading a style file more than once.
% When loading the definitions by means of |\input|
% multiple instances have to be prevented manually:
%\iffalse
%This code needs to be before the `\ProvidesFile' directive
%which is defined at the beginning of this file.
%Therefore it is also placed there and commented out here.
%</package>
%<*discard>
%\fi
%    \begin{macrocode}
\ifdefined\childdocmain\endinput\fi
%    \end{macrocode}
%\iffalse
%</discard>
%<*package>
%\fi
%
% \macro{\ifchilddoc}
% \macro{\ifchilddocmanual}
% The conditional |\ifchilddoc| tells whether a
% child (true) or main (false) document is being compiled.
% The conditional |\ifchilddocmanual| tells whether
% the |\includeonly| mechanism is used (false) or
% the selection of child files must be performed manually (true).
% The definitions initialise to false:
%    \begin{macrocode}
\newif\ifchilddoc
\newif\ifchilddocmanual
%    \end{macrocode}

% \macro{\childdocname}
% \macro{\childdocjob}
% The macro |\childdocname| stores the name of the main document
% to be compiled. The macro |\childdocjob| stores the name of
% the document on which the \LaTeX{} compiler was originally invoked.
% The content of |\jobname| cannot be compared
% to filenames specified in the source due to different catcodes.
% The following code rescans |\jobname|, stores the result
% in |\childdocname| and saves a copy in |\childdocjob|:
%    \begin{macrocode}
\edef\childdocname{\scantokens\expandafter{\jobname\noexpand}}
\let\childdocjob\childdocname
%    \end{macrocode}

% \macro{\childdocdisable}
% The macro |\childdocdisable| prevents the main file
% from being processed more than once.
% At this stage, the main document command |\childdocmain|
% is assumed to be called once again where it should do nothing.
% Any subsequent call to it should prevent
% a secondary processing of the main document
% It overwrites the forwarding commands
% |\childdocof| and |\childdocforward|
% with empty macros to prevent further inclusions of the main document:
%    \begin{macrocode}
\newcommand{\childdocdisable}
{
  \renewcommand{\childdocmain}[1]{\renewcommand{\childdocmain}[1]{\endinput}}
  \renewcommand{\childdocof}[1]{}
  \renewcommand{\childdocby}[2][]{}
  \renewcommand{\childdocforward}[2][]{}
  \renewcommand{\childdocdisable}{}
}
%    \end{macrocode}

% \macro{\childdocmain}
% The macro |\childdocmain| is to be called at the top of the main file
% with nothing or the main filename (without extension) as argument.
% First, it breaks loops.
% If the argument is not empty and does not match |\childdocname|
% (which is set by the first inclusion of |childdoc.def|),
% |\ifchilddoc| is set to true, |\includeonly| is applied to the child file
% and |\jobname| is set to the main file
% (for proper handling of |.aux| files):
%    \begin{macrocode}
\newcommand{\childdocmain}[1]
{
  \childdocdisable\childdocmain{}
  \if?#1?\else
    \begingroup
      \def\childdoctmp{#1}
      \ifx\childdoctmp\childdocname
        \def\childdoctmp{}
      \else
        \def\childdoctmp
        {
          \childdoctrue
          \includeonly{\childdocname}
          \def\childdocjob{#1}
          \def\jobname{#1}
        }
      \fi
      \expandafter
    \endgroup
    \childdoctmp
  \fi
}
%    \end{macrocode}

% \macro{\childdocof}
% The command |\childdocof| redirects
% compilation to the main file |#1|.
%    \begin{macrocode}
\newcommand{\childdocof}[1]
{
  \childdocdisable
  \childdoctrue
  \includeonly{\childdocname}
  \def\jobname{#1}
  \def\childdocjob{#1}
  \input{#1}
}
%    \end{macrocode}

% \macro{\childdocby}
% The command |\childdocby| ....
%    \begin{macrocode}
\newcommand{\childdocby}[2][]
{
  \childdocdisable
  \childdoctrue
  \childdocmanualtrue
  \if?#1?\else
    \def\jobname{#2}
  \fi
  \def\childdocjob{#2}
  \input{#2}
  \endinput
}
%    \end{macrocode}

% \macro{\childdocforward}
% The command |\childdocforward| redirects
% compilation to the main file or
% (if the optional argument is given) a child file.
% Parameters are set as if the main file
% or a child file starting with |\childdocof| was compiled.
% Then compilation is handed over to the main file:
%    \begin{macrocode}
\newcommand{\childdocforward}[2][]
{
  \begingroup
    \if?#1?
      \def\childdoctmp
      {
        \def\childdocname{#2}
        \def\childdocjob{#2}
        \def\jobname{#2}
        \input{#2}
        \endinput
      }
    \else
      \def\childdoctmp
      {
        \childdocdisable
        \def\childdocname{#2}
        \childdoctrue
        \includeonly{#2}
        \def\childdocjob{#1}
        \def\jobname{#1}
        \input{#1}
        \endinput
      }
    \fi
    \expandafter
  \endgroup
  \childdoctmp
}
%    \end{macrocode}

% \macro{\childdocforwardprefix}
% The command |\childdocforwardprefix| redirects
% compilation to the main or a child file by means of a pattern.
% The prefix |#1| in the current filename is replaced by |#2|
% and the suffix of the current filename is kept
% (it is assumed that the filename does not contain the substring `|~~~|'
% which is used as a delimiter).
% Compilation is handed over to the new file by |\childdocforward|:
%    \begin{macrocode}
\newcommand{\childdocforwardprefix}[3][]
{
  \begingroup
    \def\childdocextract #2##1~~~{\def\childdoctmp{\childdocforward[#1]{#3##1}}}
    \expandafter\childdocextract\childdocname~~~
    \expandafter
  \endgroup
  \childdoctmp
}
%    \end{macrocode}

% \macro{\childdoc}
% The deprecated macro |\childdoc| is a legacy version of |\childdocmain|:
%    \begin{macrocode}
\newcommand{\childdoc}{\childdocmain}
%    \end{macrocode}

% \macro{\childdocredirect}
% The deprecated macro |\childdocredirect| is a legacy version
% of |\childdocforward| and |\childdocforwardprefix|:
%    \begin{macrocode}
\newcommand{\childdocredirect}[2][]
{
  \begingroup
    \if?#1?
      \def\childdoctmp{\childdocforward{#2}}
    \else
      \def\childdoctmp{\childdocforwardprefix{#1}{#2}}
    \fi
    \expandafter
  \endgroup
  \childdoctmp
}
%    \end{macrocode}

%\iffalse
%</package>
%\fi
%
\endinput

\childdocmain{}
%    \end{macrocode}

% Optional override for |\version| flag:
%    \begin{macrocode}
%%\ifchilddoc\else\providecommand{\version}{draft}\fi
%    \end{macrocode}

% Define the default values for the |\version| flag
% (|final| for the main file and |draft| for childs):
%    \begin{macrocode}
\ifchilddoc
\providecommand{\version}{draft}
\else
\providecommand{\version}{final}
\fi
%    \end{macrocode}

% Load the standard document class:
%    \begin{macrocode}
\documentclass[12pt]{article}
%    \end{macrocode}

% Start the document body:
%    \begin{macrocode}
\begin{document}
%    \end{macrocode}

% Declare a title page.
% Print title, part of document being processed and version flag:
%    \begin{macrocode}
\addtocounter{page}{-1}
\begin{center}
{\LARGE\bfseries{}childdoc example\par}
\vspace{1cm}
\ifchilddoc
\ifchilddocmanual part\else chapter\fi:
`\childdocname' of `\childdocjob'\par
\else
main document: `\childdocjob'\par
\fi
version: \version\par
\end{center}
\newpage
%    \end{macrocode}

% Manually include selected file,
% otherwise process as usual:
%    \begin{macrocode}
\ifchilddocmanual
\section*{part `\childdocname'}
\input{\childdocname}
\else
%    \end{macrocode}

% Include the two chapters:
%    \begin{macrocode}
\include{cdocsch1}
\include{cdocsch2}
%    \end{macrocode}

% Include the two parts unless only chapters should be displayed:
%    \begin{macrocode}
\ifchilddoc\else
\section{part three}
\input{cdocspt3}
\section{part four}
\input{cdocspt4}
\fi
%    \end{macrocode}

% Process as usual until here:
%    \begin{macrocode}
\fi
%    \end{macrocode}

% End of document body:
%    \begin{macrocode}
\end{document}
%    \end{macrocode}
%\iffalse
%</samplemain>
%\fi
%
% %%%%%%%%%%%%%%%%%%%%%%%%%%%%%%%%%%%%%%
% \paragraph{Chapter Include Files.}
%
% The include files are called |cdocsch1.tex| and |cdocsch2.tex|.
%
%\iffalse
%<*samplechap1|samplechap2>
%\fi

% Optional override for |\version| flag:
%    \begin{macrocode}
%%\providecommand{\version}{final}
%    \end{macrocode}

% Include the main document:
%    \begin{macrocode}
% \iffalse
%
% childdoc.dtx Copyright (C) 2017-2018 Niklas Beisert
%
% This work may be distributed and/or modified under the
% conditions of the LaTeX Project Public License, either version 1.3
% of this license or (at your option) any later version.
% The latest version of this license is in
%   http://www.latex-project.org/lppl.txt
% and version 1.3 or later is part of all distributions of LaTeX
% version 2005/12/01 or later.
%
% This work has the LPPL maintenance status `maintained'.
%
% The Current Maintainer of this work is Niklas Beisert.
%
% This work consists of the files childdoc.dtx and childdoc.ins
% and the derived files childdoc.def and cdocsamp.tex with
% cdocsch1.tex, cdocsch2.tex, cdocsdrf.tex, cdocsfn1.tex, cdocsfn2.tex.
%
%<package>\ifdefined\childdocmain\endinput\fi
%<package>\ProvidesFile{childdoc.def}[2018/12/30 v2.0 child document driver]
%<samplemain>\ProvidesFile{cdocsamp.tex}[2018/12/30 v2.0 sample for childdoc]
%<*driver>
%\ProvidesFile{childdoc.drv}[2018/12/30 v2.0 childdoc reference manual file]
\PassOptionsToClass{10pt,a4paper}{article}
\documentclass{ltxdoc}

\usepackage[margin=35mm]{geometry}
\usepackage{hyperref}
\usepackage{hyperxmp}
\usepackage[usenames]{color}

\hypersetup{colorlinks=true}
\hypersetup{pdfstartview=FitH}
\hypersetup{pdfpagemode=UseNone}
\hypersetup{pdfsource={}}
\hypersetup{pdflang={en-UK}}
\hypersetup{pdfcopyright={Copyright 2017-2018 Niklas Beisert.
  This work may be distributed and/or modified under the
  conditions of the LaTeX Project Public License, either version 1.3
  of this license or (at your option) any later version.}}
\hypersetup{pdflicenseurl={http://www.latex-project.org/lppl.txt}}
\hypersetup{pdfcontactaddress={ETH Zurich, ITP, HIT K,
  Wolfgang-Pauli-Strasse 27}}
\hypersetup{pdfcontactpostcode={8093}}
\hypersetup{pdfcontactcity={Zurich}}
\hypersetup{pdfcontactcountry={Switzerland}}
\hypersetup{pdfcontactemail={nbeisert@itp.phys.ethz.ch}}
\hypersetup{pdfcontacturl={http://people.phys.ethz.ch/\xmptilde nbeisert/}}

\newcommand{\secref}[1]{\hyperref[#1]{section \ref*{#1}}}

\parskip1ex
\parindent0pt
\let\olditemize\itemize
\def\itemize{\olditemize\parskip0pt}

\begin{document}

\title{The \textsf{childdoc} Package}
\hypersetup{pdftitle={The childdoc Package}}
\author{Niklas Beisert\\[2ex]
  Institut f\"ur Theoretische Physik\\
  Eidgen\"ossische Technische Hochschule Z\"urich\\
  Wolfgang-Pauli-Strasse 27, 8093 Z\"urich, Switzerland\\[1ex]
  \href{mailto:nbeisert@itp.phys.ethz.ch}
  {\texttt{nbeisert@itp.phys.ethz.ch}}}
\hypersetup{pdfauthor={Niklas Beisert}}
\hypersetup{pdfsubject={Manual for the LaTeX2e Package childdoc}}
\date{30 December 2018, \textsf{v2.0}}
\maketitle

\begin{abstract}\noindent
\textsf{childdoc} is a \LaTeXe{} package
that enables the direct compilation
of document sections included by |\include|
to individual files.
\end{abstract}

\begingroup
\parskip0ex
\tableofcontents
\endgroup

%%%%%%%%%%%%%%%%%%%%%%%%%%%%%%%%%%%%%%%%%%%%%%%%%%%%%%%%%%%%%%%%%%%%%%%%%%%%%%%%
%%%%%%%%%%%%%%%%%%%%%%%%%%%%%%%%%%%%%%%%%%%%%%%%%%%%%%%%%%%%%%%%%%%%%%%%%%%%%%%%
\section{Introduction}

\LaTeX{} provides a mechanism to structure a large document (such as a book)
into a main file and several child files (containing the chapters)
using the |\include| command.
This mechanism is beneficial for documents
which span hundreds of pages in order to
make the source file(s) more manageable.
Moreover, compilation can be restricted to
selected child files by means of the |\includeonly| command.
The latter feature can be used to reduce the compilation time while editing
(this was significantly more useful in the earlier days of \LaTeX{})
or to generate a smaller document which is easier to navigate.
Another application of |\includeonly| is to generate
documents consisting of selected parts of the complete document.

However, there are a few drawbacks of the plain |\include| mechanism:
\begin{itemize}
\item
The child files cannot be compiled on their own,
they can only be compiled via the main file.
A naive editing environment
(such as a text editor with an option
to have the current file processed by \LaTeX)
may require one to switch to the main file before compiling;
attempting to compile the child file produces errors.
\item
The main file must be modified (each time)
to adjust the |\includeonly| command
to the present needs. This easily leaves the main file in a messy state.
\item
The generated document will always carry the filename
of the main document. This is inconvenient if
several child files are to be compiled and
to be kept for distribution.
\end{itemize}

The present package provides a simple interface
to make child files individually compilable by \LaTeX{}.
Compiling a child file then has the same effect as compiling
the main file with an |\includeonly| command
to select the appropriate child.
Moreover the generated document will carry the name of the child
rather than the main file.
This resolves all three above issues.

This feature is meant to make the editing of books,
thesis documents and lecture notes somewhat more convenient.
However, the package can also be used efficiently for
composing a series of documents (such as exercise sheets)
which are typically distributed individually.
It then assists the author in generating the individual documents
(potentially in different versions)
as well as a document containing the collected series.
Another application is in developing style files
or other kinds of included material
where compilation of the style file could redirect
to a sample or test file.

%%%%%%%%%%%%%%%%%%%%%%%%%%%%%%%%%%%%%%%%%%%%%%%%%%%%%%%%%%%%%%%%%%%%%%%%%%%%%%%%
%%%%%%%%%%%%%%%%%%%%%%%%%%%%%%%%%%%%%%%%%%%%%%%%%%%%%%%%%%%%%%%%%%%%%%%%%%%%%%%%
\section{Usage}

First of all, the package \textsf{childdoc} is \emph{not} a standard
\LaTeXe{} |.sty| style file! Therefore it needs to be invoked in
a non-standard way.

%%%%%%%%%%%%%%%%%%%%%%%%%%%%%%%%%%%%%%%%%%%%%%%%%%%%%%%%%%%%%%%%%%%%%%%%%%%%%%%%
\subsection{Included Files}
\label{sec:include}

%%%%%%%%%%%%%%%%%%%%%%%%%%%%%%%%%%%%%%%%
\DescribeMacro{\childdocmain}
To use the package, add the commands
\begin{center}
\begin{tabular}{l}
|\input{childdoc.def}|\\
|\childdocmain{}|\\
\end{tabular}
\end{center}
at the very top of the main \LaTeX{} file,
in particular \emph{before} the |\documentclass| statement!
The argument of |\childdocmain| should be left empty
(but it must be present).

%%%%%%%%%%%%%%%%%%%%%%%%%%%%%%%%%%%%%%%%
\DescribeMacro{\childdocof}
Furthermore, add the commands
\begin{center}
\begin{tabular}{l}
|\input{childdoc.def}|\\
|\childdocof{|\textit{main}|}|\\
\end{tabular}
\end{center}
at the top of every child file \textit{child}
which is included by |\include{|\textit{child}|}|
from within the main file
(or at least for those files to be compiled individually).
The argument \textit{main} must be the filename of the main file.

There are a couple of
considerations in setting up the main and child documents:

%%%%%%%%%%%%%%%%%%%%%%%%%%%%%%%%%%%%%%%%
\paragraph{Restrictions.}

Please note the following restrictions:
\begin{itemize}
\item
|\childdocmain| must be called with one argument \textit{main}
to ensure compatibility with earlier version of the package.
It must either be empty (|\childdocmain{}|)
or precisely match the filename of the main file in which it is specified.
See \secref{sec:detection} for further information.
\item
The filename \textit{main} must be specified without the |.tex| extension.
\item
The filename \textit{main} is case sensitive
(even in case-insensitive file systems)
due to internal string comparison.
\item
The argument \textit{main} should be fully expanded, it cannot be a macro.
\item
Subdirectories and special characters should be avoided in filenames.
\item
The command |\childdocmain{|\textit{main}|}| must be followed by a whitespace.
It should not be followed immediately by another command
or by a comment mark `|%|'.
This is because the \TeX{} parser reads the token immediately following
the argument of |\childdocmain| and puts it
at the beginning of every child section;
however, a white\-space is ignored.
\end{itemize}

%%%%%%%%%%%%%%%%%%%%%%%%%%%%%%%%%%%%%%%%
\paragraph{Content of Main File.}

It is advisable to place all content in the child files included by |\include|.
Any output contained in the main file will appear in all child documents
unless suppressed manually;
it cannot be suppressed automatically by the |\includeonly| directive
and thus should normally be avoided.
A method to include some content in the main file
by means of conditional processing is described in \secref{sec:conditional}.

%%%%%%%%%%%%%%%%%%%%%%%%%%%%%%%%%%%%%%%%
\paragraph{Page Numbering.}

When only a part of the document is compiled,
the appropriate numbering of pages
(as well as other status parameters)
is determined from the |.aux| files.
The latter contain information from previous passes.
However this information needs to propagate through
all intermediate child documents.
Therefore the page numbering in child documents may well
be inconsistent until the complete document is compiled at least once.

A useful (if unconventional) way to always ensure a consistent
page numbering is to restart the numbering in each child document
and denote the pages by `\textit{child}|.|\textit{page}'
where \textit{child} represents the chapter/section number of the child file.
This can be achieved by the command
|\numberwithin{page}{|\textit{child}|}|
of the \textsf{amsmath} package
where \textit{child} can be |chapter| or |section|
depending on the chosen structuring.
Alternatively, one can modify the macro |\thepage| appropriately
and reset the counter |page| at the start of each child file.

%%%%%%%%%%%%%%%%%%%%%%%%%%%%%%%%%%%%%%%%%%%%%%%%%%%%%%%%%%%%%%%%%%%%%%%%%%%%%%%%
\subsection{Conditional Processing}
\label{sec:conditional}

The package provides a mechanism to compile different versions
of a document. To customise the versions further some conditional processing
can come in handy to distinguish which version is being compiled.
The package provides two macros to describe the compilation context:

%%%%%%%%%%%%%%%%%%%%%%%%%%%%%%%%%%%%%%%%
\DescribeMacro{\ifchilddoc}
The conditional |\ifchilddoc| distinguishes between the compilation of
child documents and the main document:
%
\begin{center}
|\ifchilddoc |\textit{child-code}| |[|\||else |\textit{main-code}]| \||fi|
\end{center}

%%%%%%%%%%%%%%%%%%%%%%%%%%%%%%%%%%%%%%%%
\DescribeMacro{\childdocname}
\DescribeMacro{\childdocjob}
The macro |\childdocname| contains the filename (without extension)
of the main or child file being processed.
Note that |\childdocjob| will always contain the name of the main file.

%%%%%%%%%%%%%%%%%%%%%%%%%%%%%%%%%%%%%%%%
\paragraph{Title Page.}

Conditional processing can be used to include a title or banner page
in the main document when proper precautions are taken.
Importantly, the code in the main file should ensure that the page counter
(as well as other status parameters which are stored in the |.aux| files)
takes the same value after the conditional processing.
Otherwise the page numbers may take divergent values
depending on which part is compiled.

For example, a title page could be declared by:
%
\begin{center}
\begin{tabular}{l}
|\ifchilddoc\||else|\\
|\addtocounter{page}{-1}|\\
\textit{code for title page}\\
|\newpage|\\
|\||fi|
\end{tabular}
\end{center}
%
A banner page for the child documents can be generated by:
%
\begin{center}
\begin{tabular}{l}
|\ifchilddoc|\\
|\addtocounter{page}{-1}|\\
\textit{code for banner page}\\
|\newpage|\\
|\||fi|
\end{tabular}
\end{center}
%
Here one could write a message such as:
\begin{center}
|This is the part \childdocname{} of \childdocjob{}.|
\end{center}

%%%%%%%%%%%%%%%%%%%%%%%%%%%%%%%%%%%%%%%%%%%%%%%%%%%%%%%%%%%%%%%%%%%%%%%%%%%%%%%%
\subsection{Flags}
\label{sec:flags}

The package makes it easy to generate different versions
of the main or child documents.
To this end compilation flags can be defined
and assigned different default values.
They will be particularly useful in conjunction
with the forwarding mechanism described in \secref{sec:forward}.

For example, it may be useful to have a flag |\version|
which can be set to |draft| or |final|.
The document source will contain some conditional code
depending on the value of |\version|.
Suppose further, the flag should default to |final| for the main file
and to |draft| for child files
which is a natural assignment for editing the document.
This is achieved by placing the following code
in the preamble of the main document
(below the |\childdocmain| directive):
%
\begin{center}
\begin{tabular}{l}
|\ifchilddoc|\\
|\providecommand{\version}{draft}|\\
|\||else|\\
|\providecommand{\version}{final}|\\
|\||fi|
\end{tabular}
\end{center}
%
The definition by |\providecommand| makes sure
that previous definitions are not overwritten.
Further statements |\providecommand{\version}{...}|
can thus be added before the above code to override it.

For the main file, one might add a line
(between |\childdocmain| and the above block)
%
\begin{center}
|%\ifchilddoc\||else\providecommand{\version}{draft}\||fi|
\end{center}
%
which can be uncommented to produce a draft version.
Likewise one can add a line to the very top of a child file
(above the |\childdocof{|\textit{main}|}| directive)
%
\begin{center}
|%\providecommand{\version}{final}|
\end{center}
%
which can be uncommented to produce the final version of this child document.

%%%%%%%%%%%%%%%%%%%%%%%%%%%%%%%%%%%%%%%%%%%%%%%%%%%%%%%%%%%%%%%%%%%%%%%%%%%%%%%%
\subsection{Forwarding}
\label{sec:forward}

Different versions of the main or child documents
using compilation flags as described in \secref{sec:flags}
can be (permanently) stored in different files
for convenient compilation, viewing and distribution.
To this end, the package defines a command
to pass on compilation to a different file:

%%%%%%%%%%%%%%%%%%%%%%%%%%%%%%%%%%%%%%%%
\DescribeMacro{\childdocforward}
The command |\childdocforward| redirects processing to
another source file:
%
\begin{center}
\begin{tabular}{l}
|\input{childdoc.def}|\\
|\childdocforward[|\textit{main}|]{|\textit{dest}|}|\\
\end{tabular}
\end{center}
%
The argument \textit{dest} is the destination file
(without extension).
It should be the main file or one of the child files.
Note that further \textsf{childdoc} directives
such as |\childdocof| and |\childdocforward|
in the indicated file will be processed in this form.
The optional argument \textit{main}
passes on directly to the main file \textit{main}
while pretending to compile the child \textit{dest}.
This form behaves as if \textit{dest}
issues |\childdocof{|\textit{main}|}| right away,
and no further \textsf{childdoc} directives will be processed.

%%%%%%%%%%%%%%%%%%%%%%%%%%%%%%%%%%%%%%%%
\DescribeMacro{\...prefix}
In the alternative form |\childdocforwardprefix|,
%
\begin{center}
\begin{tabular}{l}
|\input{childdoc.def}|\\
|\childdocforwardprefix[|\textit{main}|]{|\textit{prefix}|}{|\textit{dest}|}|
\end{tabular}
\end{center}
%
the destination file is determined by a pattern
depending on the current file:
To make this work, the current file must be called
`{\textit{prefix}\hspace{0.2em}\textit{suffix}}'
with \textit{prefix} matching precisely the argument.
Processing is then passed on to the file
`{\textit{dest}\hspace{0.2em}\textit{suffix}}'.
Surely, the same effect is achieved by
directly specifying the
argument `{\textit{dest}\hspace{0.2em}\textit{suffix}}'
in the first form.
However, that requires to set up a different file
for each child. With the alternative form of the command
all these files can have exactly the same content
which simplifies setting them up and maintaining them.

For example, the following file |draft.tex|
with a compilation flag |\version| as described in \secref{sec:flags}
compiles the main document as a draft:
%
\begin{center}
\begin{tabular}{l}
|\def\version{draft}|\\
|\input{childdoc.def}|\\
|\childdocforward{|\textit{main}|}|
\end{tabular}
\end{center}
%
Likewise, the following files |final|\textit{nn}|.tex|
compile the final version of the child document
|child|\textit{nn}|.tex|:
%
\begin{center}
\begin{tabular}{l}
|\def\version{final}|\\
|\input{childdoc.def}|\\
|\childdocforwardprefix{final}{child}|
\end{tabular}
\end{center}
%

Note that when several versions of a main file and/or of each child file
are to be generated, it may be convenient to set up a |Makefile| or
shell script to automatise the process.

%%%%%%%%%%%%%%%%%%%%%%%%%%%%%%%%%%%%%%%%%%%%%%%%%%%%%%%%%%%%%%%%%%%%%%%%%%%%%%%%
\subsection{Command Line Processing}
\label{sec:commandline}

The effect of redirection files can also be achieved by invoking
the \LaTeX{} compiler with a more elaborate command line.
Most conveniently this should be done as part
of a shell script or a |Makefile|.

When using \textsf{childdoc} in the main file, the following
command lines effectively perform a redirection
(note that depending on the shell being used,
backslashes may have to be doubled: `|\|' $\to$ `|\\|'):
%
\begin{center}
|... -jobname "|\textit{target}|" |\\|"|[\textit{flags}]%
|\input{childdoc.def}\childdocforward[|\textit{main}|]{|\textit{dest}|}"|
\end{center}
%
Here \textit{target} is the name of the output file,
\textit{main} is the name of the main file
and \textit{dest} is the name of the main or child file to be processed
(all filenames without extensions).
The optional argument \textit{main} can be omitted
if \textit{main} matches \textit{dest}.
Optionally, compilation \textit{flags} can be defined via |\def| commands.
This command line makes the \TeX{} engine believe
it is compiling the file \textit{target}
whose content is specified as the latter parameter.
The provided code then forwards the processing to
\textit{main} or \textit{dest} as described in \secref{sec:forward}.

%%%%%%%%%%%%%%%%%%%%%%%%%%%%%%%%%%%%%%%%%%%%%%%%%%%%%%%%%%%%%%%%%%%%%%%%%%%%%%%%
\subsection{Include by Input}
\label{sec:input}

Including child documents by |\include| has some restrictions by design.
Most notably, the content of a child document always occupies
its own set of pages; pages cannot be shared between child documents.
Usually, this behaviour makes perfect sense
because each child document contain an essential part of the document.
However, in some situations it may be desirable to compose
a document from a collection of parts
without having mandatory page breaks between then.
For this case, the package
provides a mechanism to include parts
by |\input| which can also be processed individually.
However, by construction this mechanism
requires manual handling of the content to be output.

%%%%%%%%%%%%%%%%%%%%%%%%%%%%%%%%%%%%%%%%
\DescribeMacro{\ifchilddocmanual}
The main file should be prepared as usual, see \secref{sec:include}.
However, the document body must make a distinction
between processing of an individual part and of the main document, e.g.:
%
\begin{center}
\begin{tabular}{l}
|\ifchilddocmanual|\\
|\input{\childdocname}|\\
|\||else|\\
\textit{document body with }|\input{|\textit{part}|}|\\
|\||fi|
\end{tabular}
\end{center}
%
The conditional |\ifchilddocmanual| is true whenever
a part to be included by |\input| is being compiled,
and the name of the part is stored in |\childdocname|.

%%%%%%%%%%%%%%%%%%%%%%%%%%%%%%%%%%%%%%%%
\DescribeMacro{\childdocby}
Each part to be included by |\input| should start with:
%
\begin{center}
\begin{tabular}{l}
|\input{childdoc.def}|\\
|\childdocby{|\textit{main}|}|\\
\end{tabular}
\end{center}
%
The directive |\childdocby| is similar to |\childdocof|
described in \secref{sec:include},
but the subsequent selection of content must be done manually.
To that end, both |\ifchilddoc| and |\ifchilddocmanual|
will be true upon processing of a part,
and the name of the part is stored in |\childdocname|.
Note that |\jobname| will be set to the filename of the current part
so that each part receives an individual |.aux| file
that does not interfere with the |.aux| file(s) of the main document.
This behaviour can be altered by the alternative form
|\childdocby[*]{|\textit{main}|}| (with a non-empty optional argument)
which uses the |.aux| file of the main document
by setting |\jobname| to \textit{main}.

%%%%%%%%%%%%%%%%%%%%%%%%%%%%%%%%%%%%%%%%%%%%%%%%%%%%%%%%%%%%%%%%%%%%%%%%%%%%%%%%
\subsection{Driver Development}
\label{sec:driver}

The \textsf{childdoc} mechanism can also be use for the development
of definition files such as \LaTeX{} styles or classes.
This case differs from the above setup with multiple parts
included by |\include| in that no |\includeonly| should be invoked.
This can be achieved by starting the include file
(before |\ProvidesPackage|) with:
%
\begin{center}
\begin{tabular}{l}
|\input{childdoc.def}|\\
|\childdocforward{|\textit{main}|}|\\
\end{tabular}
\end{center}
%
or alternatively with:
%
\begin{center}
\begin{tabular}{l}
|\input{childdoc.def}|\\
|\childdocby{|\textit{main}|}|\\
\end{tabular}
\end{center}
%
Both forms have slightly different effects as described above.
The main file is prepared as usual, see \secref{sec:include}.

%%%%%%%%%%%%%%%%%%%%%%%%%%%%%%%%%%%%%%%%%%%%%%%%%%%%%%%%%%%%%%%%%%%%%%%%%%%%%%%%
\subsection{Legacy Detection}
\label{sec:detection}

The directive |\childdocmain| in the main file can detect
whether the complete document or merely a child is to be compiled
even without using the directive |\childdocof|.
This method is deprecated because it is less robust
and there is no compelling reason to use it;
it is merely provided for backward compatibility
and it may be removed in future versions.

If the detection mechanism is to be used,
it is mandatory to correctly specify
the filename of the main file as the argument of |\childdocmain|:
%
\begin{center}
\begin{tabular}{l}
|\input{childdoc.def}|\\
|\childdocmain{|\textit{main}|}|\\
\end{tabular}
\end{center}
%
If |\jobname| does not match the argument \textit{main} of |\childdocmain|,
it is assumed that |\jobname| points to the child file to be compiled.
When using |\childdocmain| with the main file specified as argument,
it suffices to start a child file
with just |\input{|\textit{main}|}|
without loading of the package and using |\childdocof|.
If instead all processing is done
with the appropriate \textsf{childdoc} directives,
the argument of \textit{main} of |\childdocmain| can be empty.

An alternative version of the command line processing described
in \secref{sec:commandline} using the detection mechanism reads:
%
\begin{center}
|... -jobname "|\textit{target}|" "|[\textit{flags}]%
[|\def\jobname{|\textit{dest}|}|]|\input{|\textit{main}|}"|
\end{center}

%%%%%%%%%%%%%%%%%%%%%%%%%%%%%%%%%%%%%%%%%%%%%%%%%%%%%%%%%%%%%%%%%%%%%%%%%%%%%%%%
\subsection{Manual Code}
\label{sec:manual}

In case one cannot be certain whether the definitions file |childdoc.def|
is installed on the target \TeX{} distribution
and one prefers not to ship it,
it is conceivable to paste a few relevant commands into the sources.

To that end, drop all statements |\input{childdoc.def}|
and perform the replacements as outlined below.
Instead of |\childdocmain{|\textit{main}|}| add the following code
to the top of the main file:
%
\begin{center}
\begin{tabular}{l}
|\||ifdefined\childdocname\endinput\||fi\newif\ifchilddoc|\\
|\edef\childdocname{\scantokens\expandafter{\jobname\noexpand}}|\\
|\def\childdocmain{|\textit{main}|}\||ifx\childdocmain\childdocname\||else|\\
|\childdoctrue\includeonly{\childdocname}\let\jobname\childdocmain\||fi|\\
\end{tabular}
\end{center}
%
Instead of |\childdocof{|\textit{main}|}| just include the main file
at the top of each child file:
%
\begin{center}
|\input{|\textit{main}|}|
\end{center}
%
A simple redirection |\childdocforward{|\textit{dest}|}| is achieved by:
%
\begin{center}
|\def\jobname{|\textit{dest}|}\input{\jobname}|
\end{center}
%
The redirection with prefix
|\childdocforwardprefix[|\textit{prefix}|]{|\textit{dest}|}|
is accomplished by:
%
\begin{center}
\begin{tabular}{l}
|{\edef\jobname{\scantokens\expandafter{\jobname\noexpand}}|\\
|\def\redirectjob |\textit{prefix}|#1~~~{\gdef\jobname{|\textit{dest}|#1}}|\\
|\expandafter\redirectjob\jobname~~~}\input{\jobname}|
\end{tabular}
\end{center}

In an alternative approach,
child documents can be compiled by a specific command line
without additional code or specific definitions:
%
\begin{center}
|... -jobname "|\textit{target}|" "|[\textit{flags}]%
|\includeonly{|\textit{dest}|}\input{|\textit{main}|}"|
\end{center}
%

%%%%%%%%%%%%%%%%%%%%%%%%%%%%%%%%%%%%%%%%%%%%%%%%%%%%%%%%%%%%%%%%%%%%%%%%%%%%%%%%
%%%%%%%%%%%%%%%%%%%%%%%%%%%%%%%%%%%%%%%%%%%%%%%%%%%%%%%%%%%%%%%%%%%%%%%%%%%%%%%%
\section{Information}

%%%%%%%%%%%%%%%%%%%%%%%%%%%%%%%%%%%%%%%%%%%%%%%%%%%%%%%%%%%%%%%%%%%%%%%%%%%%%%%%
\subsection{Copyright}

Copyright \copyright{} 2017--2018 Niklas Beisert

This work may be distributed and/or modified under the
conditions of the \LaTeX{} Project Public License, either version 1.3
of this license or (at your option) any later version.
The latest version of this license is in
  \url{http://www.latex-project.org/lppl.txt}
and version 1.3 or later is part of all distributions of \LaTeX{}
version 2005/12/01 or later.

This work has the LPPL maintenance status `maintained'.

The Current Maintainer of this work is Niklas Beisert.

This work consists of the files |README.txt|, |childdoc.ins| and |childdoc.dtx|
as well as the derived files |childdoc.def|, |cdocsamp.tex|
with |cdocsch1.tex|, |cdocsch2.tex|, |cdocspt3.tex|, |cdocspt4.tex|,
|cdocsdrf.tex|, |cdocsfn1.tex|, |cdocsfn2.tex|
as well as |childdoc.pdf|.

%%%%%%%%%%%%%%%%%%%%%%%%%%%%%%%%%%%%%%%%%%%%%%%%%%%%%%%%%%%%%%%%%%%%%%%%%%%%%%%%
\subsection{Files and Installation}

The package consists of the files:
%
\begin{center}
\begin{tabular}{ll}
    |README.txt|   & readme file \\
    |childdoc.ins| & installation file \\
    |childdoc.dtx| & source file \\
    |childdoc.def| & definition file \\
    |cdocsamp.tex| & sample main file \\
    |cdocsch1.tex| & sample include file \\
    |cdocsch2.tex| & sample include file \\
    |cdocspt3.tex| & sample part file \\
    |cdocspt4.tex| & sample part file \\
    |cdocsdrf.tex| & sample redirection file \\
    |cdocsfn1.tex| & sample redirection file \\
    |cdocsfn2.tex| & sample redirection file \\
    |childdoc.pdf| & manual
\end{tabular}
\end{center}
%
The distribution consists of the files
|README.txt|, |childdoc.ins| and |childdoc.dtx|.
%
\begin{itemize}
\item
Run (pdf)\LaTeX{} on |childdoc.dtx|
to compile the manual |childdoc.pdf| (this file).
\item
Run \LaTeX{} on |childdoc.ins| to create the definitions file |childdoc.def|
and the sample |cdocsamp.tex| with include files
|cdocsch1.tex|, |cdocsch2.tex|, |cdocspt3.tex|, |cdocspt4.tex|,
|cdocsdrf.tex|, |cdocsfn1.tex|, |cdocsfn2.tex|.
Then copy the file |childdoc.def| to an appropriate directory of your \LaTeX{}
distribution, e.g.\ \textit{texmf-root}|/tex/latex/childdoc|.
\end{itemize}

%%%%%%%%%%%%%%%%%%%%%%%%%%%%%%%%%%%%%%%%%%%%%%%%%%%%%%%%%%%%%%%%%%%%%%%%%%%%%%%%
\subsection{Related CTAN Packages}

There are several other packages which offer a similar functionality:
%
\begin{itemize}
\item
The packages
\href{http://ctan.org/pkg/docmute}{\textsf{docmute}},
\href{http://ctan.org/pkg/includex}{\textsf{includex}} and
\href{http://ctan.org/pkg/standalone}{\textsf{standalone}}
provide commands to include only the document body of
a child file thus allowing both files to be compiled individually.
\item
The packages \href{http://ctan.org/pkg/subdocs}{\textsf{subdocs}}
and \href{http://ctan.org/pkg/subfiles}{\textsf{subfiles}}
provide structures in which the main and child documents can be
encapsulated and allowing them to be compiled individually.
The inclusion mechanism is different from the conventional |\include|.
\item
The package \href{http://ctan.org/pkg/combine}{\textsf{combine}}
is an elaborate solution to combine several documents into one.
\end{itemize}
%
See also the CTAN topic \href{http://ctan.org/topic/subdocs}{\textsf{subdocs}}
for further related packages.
The present package differs from the above solutions in that
a document structure constructed with the conventional |\include| mechanism
just needs two extra commands at the top of every file
such that all constituent files can be compiled individually.

%%%%%%%%%%%%%%%%%%%%%%%%%%%%%%%%%%%%%%%%%%%%%%%%%%%%%%%%%%%%%%%%%%%%%%%%%%%%%%%%
%\subsection{Feature Suggestions}
%
%The following is a list of features which may be useful for future
%versions of this package:
%%
%\begin{itemize}
%\item
%\ldots
%\end{itemize}

%%%%%%%%%%%%%%%%%%%%%%%%%%%%%%%%%%%%%%%%%%%%%%%%%%%%%%%%%%%%%%%%%%%%%%%%%%%%%%%%
\subsection{Revision History}

%%%%%%%%%%%%%%%%%%%%%%%%%%%%%%%%%%%%%%%%
\paragraph{v2.0:} 2018/12/30

\begin{itemize}
\item
immediate forward processing
\item
added |\childdocby| mechanism
\item
manual restructured
\end{itemize}

%%%%%%%%%%%%%%%%%%%%%%%%%%%%%%%%%%%%%%%%
\paragraph{v1.6:} 2018/01/17

\begin{itemize}
\item
application for development of include files
\item
corrections to manual
\end{itemize}

%%%%%%%%%%%%%%%%%%%%%%%%%%%%%%%%%%%%%%%%
\paragraph{v1.5:} 2017/05/21

\begin{itemize}
\item
more complete structuring introduced
\item
|\childdocof| introduced
\item
|\childdoc| renamed to |\childdocmain|
\item
|\childredirect| renamed to |\childdocforward| and |\childdocforwardprefix|
and functionality expanded
\end{itemize}

%%%%%%%%%%%%%%%%%%%%%%%%%%%%%%%%%%%%%%%%
\paragraph{v1.0:} 2017/04/27

\begin{itemize}
\item
manual and install package
\item
first version published on CTAN
\end{itemize}

%%%%%%%%%%%%%%%%%%%%%%%%%%%%%%%%%%%%%%%%
\paragraph{v0.6:} 2017/04/26

\begin{itemize}
\item
redirection mechanism added
\end{itemize}

%%%%%%%%%%%%%%%%%%%%%%%%%%%%%%%%%%%%%%%%
\paragraph{v0.5:} 2017/04/26

\begin{itemize}
\item
functionality in definition file
\end{itemize}


%%%%%%%%%%%%%%%%%%%%%%%%%%%%%%%%%%%%%%%%%%%%%%%%%%%%%%%%%%%%%%%%%%%%%%%%%%%%%%%%
%%%%%%%%%%%%%%%%%%%%%%%%%%%%%%%%%%%%%%%%%%%%%%%%%%%%%%%%%%%%%%%%%%%%%%%%%%%%%%%%
%%%%%%%%%%%%%%%%%%%%%%%%%%%%%%%%%%%%%%%%%%%%%%%%%%%%%%%%%%%%%%%%%%%%%%%%%%%%%%%%
\appendix

\settowidth\MacroIndent{\rmfamily\scriptsize 000\ }

 \DocInput{childdoc.dtx}

\end{document}
%</driver>
% \fi
%
% %%%%%%%%%%%%%%%%%%%%%%%%%%%%%%%%%%%%%%%%%%%%%%%%%%%%%%%%%%%%%%%%%%%%%%%%%%%%%%
% %%%%%%%%%%%%%%%%%%%%%%%%%%%%%%%%%%%%%%%%%%%%%%%%%%%%%%%%%%%%%%%%%%%%%%%%%%%%%%
% \section{Sample}
%\iffalse
%<*samplemain>
%\fi
%
% The following presents a sample document
% with two chapters, two parts, a title page,
% a compile flag as well as three forwarding files to set the flag.
% It consists of eight |.tex| files:
% \begin{center}
% \begin{tabular}{ll}
% |cdocsamp.tex|&main file\\
% |cdocsch1.tex|&include file for chapter 1\\
% |cdocsch2.tex|&include file for chapter 2\\
% |cdocspt3.tex|&include file for part 3\\
% |cdocspt4.tex|&include file for part 4\\
% |cdocsdrf.tex|&forwarding file for main file in draft mode\\
% |cdocsfi1.tex|&forwarding file for final version of chapter 1\\
% |cdocsfi2.tex|&forwarding file for final version of chapter 2\\
% \end{tabular}
% \end{center}
% Each of the eight files can be compiled directly by the \LaTeX{} compiler.
%
% %%%%%%%%%%%%%%%%%%%%%%%%%%%%%%%%%%%%%%
% \paragraph{Main File.}
%
% The main file is called |cdocsamp.tex|.
%
% Load the \textsf{childdoc} definitions and
% declare the filename for the main document:
%    \begin{macrocode}
\input{childdoc.def}
\childdocmain{}
%    \end{macrocode}

% Optional override for |\version| flag:
%    \begin{macrocode}
%%\ifchilddoc\else\providecommand{\version}{draft}\fi
%    \end{macrocode}

% Define the default values for the |\version| flag
% (|final| for the main file and |draft| for childs):
%    \begin{macrocode}
\ifchilddoc
\providecommand{\version}{draft}
\else
\providecommand{\version}{final}
\fi
%    \end{macrocode}

% Load the standard document class:
%    \begin{macrocode}
\documentclass[12pt]{article}
%    \end{macrocode}

% Start the document body:
%    \begin{macrocode}
\begin{document}
%    \end{macrocode}

% Declare a title page.
% Print title, part of document being processed and version flag:
%    \begin{macrocode}
\addtocounter{page}{-1}
\begin{center}
{\LARGE\bfseries{}childdoc example\par}
\vspace{1cm}
\ifchilddoc
\ifchilddocmanual part\else chapter\fi:
`\childdocname' of `\childdocjob'\par
\else
main document: `\childdocjob'\par
\fi
version: \version\par
\end{center}
\newpage
%    \end{macrocode}

% Manually include selected file,
% otherwise process as usual:
%    \begin{macrocode}
\ifchilddocmanual
\section*{part `\childdocname'}
\input{\childdocname}
\else
%    \end{macrocode}

% Include the two chapters:
%    \begin{macrocode}
\include{cdocsch1}
\include{cdocsch2}
%    \end{macrocode}

% Include the two parts unless only chapters should be displayed:
%    \begin{macrocode}
\ifchilddoc\else
\section{part three}
\input{cdocspt3}
\section{part four}
\input{cdocspt4}
\fi
%    \end{macrocode}

% Process as usual until here:
%    \begin{macrocode}
\fi
%    \end{macrocode}

% End of document body:
%    \begin{macrocode}
\end{document}
%    \end{macrocode}
%\iffalse
%</samplemain>
%\fi
%
% %%%%%%%%%%%%%%%%%%%%%%%%%%%%%%%%%%%%%%
% \paragraph{Chapter Include Files.}
%
% The include files are called |cdocsch1.tex| and |cdocsch2.tex|.
%
%\iffalse
%<*samplechap1|samplechap2>
%\fi

% Optional override for |\version| flag:
%    \begin{macrocode}
%%\providecommand{\version}{final}
%    \end{macrocode}

% Include the main document:
%    \begin{macrocode}
\input{childdoc.def}
\childdocof{cdocsamp}
%    \end{macrocode}

%\iffalse
%</samplechap1|samplechap2>
%\fi
%
%\iffalse
%<*samplechap1>
%\fi
% Some text for chapter 1:
%    \begin{macrocode}
\section{one}
some text in chapter one
%    \end{macrocode}

%\iffalse
%</samplechap1>
%\fi
% Some text for chapter 2:
%\iffalse
%<*samplechap2>
%\fi
%    \begin{macrocode}
\section{two}
more text in chapter two
%    \end{macrocode}

%\iffalse
%</samplechap2>
%\fi
%
% %%%%%%%%%%%%%%%%%%%%%%%%%%%%%%%%%%%%%%
% \paragraph{Part Include Files.}
%
% The include files are called |cdocspt3.tex| and |cdocspt4.tex|.
%
%\iffalse
%<*samplepart3|samplepart4>
%\fi

% Optional override for |\version| flag:
%    \begin{macrocode}
%%\providecommand{\version}{final}
%    \end{macrocode}

% Include the main document:
%    \begin{macrocode}
\input{childdoc.def}
\childdocby{cdocsamp}
%    \end{macrocode}

%\iffalse
%</samplepart3|samplepart4>
%\fi
%
%\iffalse
%<*samplepart3>
%\fi
% Some text for part 3:
%    \begin{macrocode}
some text in part three
%    \end{macrocode}

%\iffalse
%</samplepart3>
%\fi
% Some text for part 4:
%\iffalse
%<*samplepart4>
%\fi
%    \begin{macrocode}
more text in part four
%    \end{macrocode}

%\iffalse
%</samplepart4>
%\fi
%
% %%%%%%%%%%%%%%%%%%%%%%%%%%%%%%%%%%%%%%
% \paragraph{Forwarding for a Complete Draft.}
%
% The following forwarding file |cdocsdrf.tex|
% compiles the main document in draft mode:
%\iffalse
%<*sampledraft>
%\fi
%    \begin{macrocode}
\def\version{draft}
\input{childdoc.def}
\childdocforward{cdocsamp}
%    \end{macrocode}

%\iffalse
%</sampledraft>
%\fi
%
% %%%%%%%%%%%%%%%%%%%%%%%%%%%%%%%%%%%%%%
% \paragraph{Forwarding for Final Version of the Chapters.}
%
% The following forwarding files |cdocsfn1.tex| and |cdocsfn2.tex|
% (with identical content)
% compile the final versions of the child documents
% |cdocsch1.tex| and |cdocsch2.tex|, respectively:
%\iffalse
%<*samplefinal>
%\fi
%    \begin{macrocode}
\def\version{final}
\input{childdoc.def}
\childdocforwardprefix[cdocsamp]{cdocsfn}{cdocsch}
%    \end{macrocode}

%\iffalse
%</samplefinal>
%\fi
%
% %%%%%%%%%%%%%%%%%%%%%%%%%%%%%%%%%%%%%%
% \paragraph{Command Line Processing.}
%
% The following three command lines generate the output files
% |cdocscld|, |cdocscl1| and |cdocscl2|
% which should be identical to
% |cdocsdrf|, |cdocsch1| and |cdocsfn2|, respectively:
% \begin{center}
% \begin{tabular}{l}
% |latex -jobname cdocscld \|\\
% |  "\def\version{draft}\input{childdoc.def}\childdocforward{cdocsamp}"|\\
% |latex -jobname cdocscl1 \|\\
% |  "\input{childdoc.def}\childdocforward[cdocsamp]{cdocsch1}"|\\
% |latex -jobname cdocscl2 \|\\
% |  "\def\version{final}\input{childdoc.def}\childdocforward{cdocsch2}"|
% \end{tabular}
% \end{center}
% Note that the trailing backslash on each first line
% merely continues the input to the second line
% (for convenient cut ant paste).
% Furthermore, the command |latex| can be replaced by any
% of its alternative versions such as |pdflatex|.
%
% %%%%%%%%%%%%%%%%%%%%%%%%%%%%%%%%%%%%%%%%%%%%%%%%%%%%%%%%%%%%%%%%%%%%%%%%%%%%%%
% %%%%%%%%%%%%%%%%%%%%%%%%%%%%%%%%%%%%%%%%%%%%%%%%%%%%%%%%%%%%%%%%%%%%%%%%%%%%%%
% \section{Implementation}
%\iffalse
%<*package>
%\fi
%
% This section describes the definitions file |childdoc.def|.

% The definitions cannot be loaded using |\usepackage| or |\RequirePackage|
% which has a mechanism to prevent loading a style file more than once.
% When loading the definitions by means of |\input|
% multiple instances have to be prevented manually:
%\iffalse
%This code needs to be before the `\ProvidesFile' directive
%which is defined at the beginning of this file.
%Therefore it is also placed there and commented out here.
%</package>
%<*discard>
%\fi
%    \begin{macrocode}
\ifdefined\childdocmain\endinput\fi
%    \end{macrocode}
%\iffalse
%</discard>
%<*package>
%\fi
%
% \macro{\ifchilddoc}
% \macro{\ifchilddocmanual}
% The conditional |\ifchilddoc| tells whether a
% child (true) or main (false) document is being compiled.
% The conditional |\ifchilddocmanual| tells whether
% the |\includeonly| mechanism is used (false) or
% the selection of child files must be performed manually (true).
% The definitions initialise to false:
%    \begin{macrocode}
\newif\ifchilddoc
\newif\ifchilddocmanual
%    \end{macrocode}

% \macro{\childdocname}
% \macro{\childdocjob}
% The macro |\childdocname| stores the name of the main document
% to be compiled. The macro |\childdocjob| stores the name of
% the document on which the \LaTeX{} compiler was originally invoked.
% The content of |\jobname| cannot be compared
% to filenames specified in the source due to different catcodes.
% The following code rescans |\jobname|, stores the result
% in |\childdocname| and saves a copy in |\childdocjob|:
%    \begin{macrocode}
\edef\childdocname{\scantokens\expandafter{\jobname\noexpand}}
\let\childdocjob\childdocname
%    \end{macrocode}

% \macro{\childdocdisable}
% The macro |\childdocdisable| prevents the main file
% from being processed more than once.
% At this stage, the main document command |\childdocmain|
% is assumed to be called once again where it should do nothing.
% Any subsequent call to it should prevent
% a secondary processing of the main document
% It overwrites the forwarding commands
% |\childdocof| and |\childdocforward|
% with empty macros to prevent further inclusions of the main document:
%    \begin{macrocode}
\newcommand{\childdocdisable}
{
  \renewcommand{\childdocmain}[1]{\renewcommand{\childdocmain}[1]{\endinput}}
  \renewcommand{\childdocof}[1]{}
  \renewcommand{\childdocby}[2][]{}
  \renewcommand{\childdocforward}[2][]{}
  \renewcommand{\childdocdisable}{}
}
%    \end{macrocode}

% \macro{\childdocmain}
% The macro |\childdocmain| is to be called at the top of the main file
% with nothing or the main filename (without extension) as argument.
% First, it breaks loops.
% If the argument is not empty and does not match |\childdocname|
% (which is set by the first inclusion of |childdoc.def|),
% |\ifchilddoc| is set to true, |\includeonly| is applied to the child file
% and |\jobname| is set to the main file
% (for proper handling of |.aux| files):
%    \begin{macrocode}
\newcommand{\childdocmain}[1]
{
  \childdocdisable\childdocmain{}
  \if?#1?\else
    \begingroup
      \def\childdoctmp{#1}
      \ifx\childdoctmp\childdocname
        \def\childdoctmp{}
      \else
        \def\childdoctmp
        {
          \childdoctrue
          \includeonly{\childdocname}
          \def\childdocjob{#1}
          \def\jobname{#1}
        }
      \fi
      \expandafter
    \endgroup
    \childdoctmp
  \fi
}
%    \end{macrocode}

% \macro{\childdocof}
% The command |\childdocof| redirects
% compilation to the main file |#1|.
%    \begin{macrocode}
\newcommand{\childdocof}[1]
{
  \childdocdisable
  \childdoctrue
  \includeonly{\childdocname}
  \def\jobname{#1}
  \def\childdocjob{#1}
  \input{#1}
}
%    \end{macrocode}

% \macro{\childdocby}
% The command |\childdocby| ....
%    \begin{macrocode}
\newcommand{\childdocby}[2][]
{
  \childdocdisable
  \childdoctrue
  \childdocmanualtrue
  \if?#1?\else
    \def\jobname{#2}
  \fi
  \def\childdocjob{#2}
  \input{#2}
  \endinput
}
%    \end{macrocode}

% \macro{\childdocforward}
% The command |\childdocforward| redirects
% compilation to the main file or
% (if the optional argument is given) a child file.
% Parameters are set as if the main file
% or a child file starting with |\childdocof| was compiled.
% Then compilation is handed over to the main file:
%    \begin{macrocode}
\newcommand{\childdocforward}[2][]
{
  \begingroup
    \if?#1?
      \def\childdoctmp
      {
        \def\childdocname{#2}
        \def\childdocjob{#2}
        \def\jobname{#2}
        \input{#2}
        \endinput
      }
    \else
      \def\childdoctmp
      {
        \childdocdisable
        \def\childdocname{#2}
        \childdoctrue
        \includeonly{#2}
        \def\childdocjob{#1}
        \def\jobname{#1}
        \input{#1}
        \endinput
      }
    \fi
    \expandafter
  \endgroup
  \childdoctmp
}
%    \end{macrocode}

% \macro{\childdocforwardprefix}
% The command |\childdocforwardprefix| redirects
% compilation to the main or a child file by means of a pattern.
% The prefix |#1| in the current filename is replaced by |#2|
% and the suffix of the current filename is kept
% (it is assumed that the filename does not contain the substring `|~~~|'
% which is used as a delimiter).
% Compilation is handed over to the new file by |\childdocforward|:
%    \begin{macrocode}
\newcommand{\childdocforwardprefix}[3][]
{
  \begingroup
    \def\childdocextract #2##1~~~{\def\childdoctmp{\childdocforward[#1]{#3##1}}}
    \expandafter\childdocextract\childdocname~~~
    \expandafter
  \endgroup
  \childdoctmp
}
%    \end{macrocode}

% \macro{\childdoc}
% The deprecated macro |\childdoc| is a legacy version of |\childdocmain|:
%    \begin{macrocode}
\newcommand{\childdoc}{\childdocmain}
%    \end{macrocode}

% \macro{\childdocredirect}
% The deprecated macro |\childdocredirect| is a legacy version
% of |\childdocforward| and |\childdocforwardprefix|:
%    \begin{macrocode}
\newcommand{\childdocredirect}[2][]
{
  \begingroup
    \if?#1?
      \def\childdoctmp{\childdocforward{#2}}
    \else
      \def\childdoctmp{\childdocforwardprefix{#1}{#2}}
    \fi
    \expandafter
  \endgroup
  \childdoctmp
}
%    \end{macrocode}

%\iffalse
%</package>
%\fi
%
\endinput

\childdocof{cdocsamp}
%    \end{macrocode}

%\iffalse
%</samplechap1|samplechap2>
%\fi
%
%\iffalse
%<*samplechap1>
%\fi
% Some text for chapter 1:
%    \begin{macrocode}
\section{one}
some text in chapter one
%    \end{macrocode}

%\iffalse
%</samplechap1>
%\fi
% Some text for chapter 2:
%\iffalse
%<*samplechap2>
%\fi
%    \begin{macrocode}
\section{two}
more text in chapter two
%    \end{macrocode}

%\iffalse
%</samplechap2>
%\fi
%
% %%%%%%%%%%%%%%%%%%%%%%%%%%%%%%%%%%%%%%
% \paragraph{Part Include Files.}
%
% The include files are called |cdocspt3.tex| and |cdocspt4.tex|.
%
%\iffalse
%<*samplepart3|samplepart4>
%\fi

% Optional override for |\version| flag:
%    \begin{macrocode}
%%\providecommand{\version}{final}
%    \end{macrocode}

% Include the main document:
%    \begin{macrocode}
% \iffalse
%
% childdoc.dtx Copyright (C) 2017-2018 Niklas Beisert
%
% This work may be distributed and/or modified under the
% conditions of the LaTeX Project Public License, either version 1.3
% of this license or (at your option) any later version.
% The latest version of this license is in
%   http://www.latex-project.org/lppl.txt
% and version 1.3 or later is part of all distributions of LaTeX
% version 2005/12/01 or later.
%
% This work has the LPPL maintenance status `maintained'.
%
% The Current Maintainer of this work is Niklas Beisert.
%
% This work consists of the files childdoc.dtx and childdoc.ins
% and the derived files childdoc.def and cdocsamp.tex with
% cdocsch1.tex, cdocsch2.tex, cdocsdrf.tex, cdocsfn1.tex, cdocsfn2.tex.
%
%<package>\ifdefined\childdocmain\endinput\fi
%<package>\ProvidesFile{childdoc.def}[2018/12/30 v2.0 child document driver]
%<samplemain>\ProvidesFile{cdocsamp.tex}[2018/12/30 v2.0 sample for childdoc]
%<*driver>
%\ProvidesFile{childdoc.drv}[2018/12/30 v2.0 childdoc reference manual file]
\PassOptionsToClass{10pt,a4paper}{article}
\documentclass{ltxdoc}

\usepackage[margin=35mm]{geometry}
\usepackage{hyperref}
\usepackage{hyperxmp}
\usepackage[usenames]{color}

\hypersetup{colorlinks=true}
\hypersetup{pdfstartview=FitH}
\hypersetup{pdfpagemode=UseNone}
\hypersetup{pdfsource={}}
\hypersetup{pdflang={en-UK}}
\hypersetup{pdfcopyright={Copyright 2017-2018 Niklas Beisert.
  This work may be distributed and/or modified under the
  conditions of the LaTeX Project Public License, either version 1.3
  of this license or (at your option) any later version.}}
\hypersetup{pdflicenseurl={http://www.latex-project.org/lppl.txt}}
\hypersetup{pdfcontactaddress={ETH Zurich, ITP, HIT K,
  Wolfgang-Pauli-Strasse 27}}
\hypersetup{pdfcontactpostcode={8093}}
\hypersetup{pdfcontactcity={Zurich}}
\hypersetup{pdfcontactcountry={Switzerland}}
\hypersetup{pdfcontactemail={nbeisert@itp.phys.ethz.ch}}
\hypersetup{pdfcontacturl={http://people.phys.ethz.ch/\xmptilde nbeisert/}}

\newcommand{\secref}[1]{\hyperref[#1]{section \ref*{#1}}}

\parskip1ex
\parindent0pt
\let\olditemize\itemize
\def\itemize{\olditemize\parskip0pt}

\begin{document}

\title{The \textsf{childdoc} Package}
\hypersetup{pdftitle={The childdoc Package}}
\author{Niklas Beisert\\[2ex]
  Institut f\"ur Theoretische Physik\\
  Eidgen\"ossische Technische Hochschule Z\"urich\\
  Wolfgang-Pauli-Strasse 27, 8093 Z\"urich, Switzerland\\[1ex]
  \href{mailto:nbeisert@itp.phys.ethz.ch}
  {\texttt{nbeisert@itp.phys.ethz.ch}}}
\hypersetup{pdfauthor={Niklas Beisert}}
\hypersetup{pdfsubject={Manual for the LaTeX2e Package childdoc}}
\date{30 December 2018, \textsf{v2.0}}
\maketitle

\begin{abstract}\noindent
\textsf{childdoc} is a \LaTeXe{} package
that enables the direct compilation
of document sections included by |\include|
to individual files.
\end{abstract}

\begingroup
\parskip0ex
\tableofcontents
\endgroup

%%%%%%%%%%%%%%%%%%%%%%%%%%%%%%%%%%%%%%%%%%%%%%%%%%%%%%%%%%%%%%%%%%%%%%%%%%%%%%%%
%%%%%%%%%%%%%%%%%%%%%%%%%%%%%%%%%%%%%%%%%%%%%%%%%%%%%%%%%%%%%%%%%%%%%%%%%%%%%%%%
\section{Introduction}

\LaTeX{} provides a mechanism to structure a large document (such as a book)
into a main file and several child files (containing the chapters)
using the |\include| command.
This mechanism is beneficial for documents
which span hundreds of pages in order to
make the source file(s) more manageable.
Moreover, compilation can be restricted to
selected child files by means of the |\includeonly| command.
The latter feature can be used to reduce the compilation time while editing
(this was significantly more useful in the earlier days of \LaTeX{})
or to generate a smaller document which is easier to navigate.
Another application of |\includeonly| is to generate
documents consisting of selected parts of the complete document.

However, there are a few drawbacks of the plain |\include| mechanism:
\begin{itemize}
\item
The child files cannot be compiled on their own,
they can only be compiled via the main file.
A naive editing environment
(such as a text editor with an option
to have the current file processed by \LaTeX)
may require one to switch to the main file before compiling;
attempting to compile the child file produces errors.
\item
The main file must be modified (each time)
to adjust the |\includeonly| command
to the present needs. This easily leaves the main file in a messy state.
\item
The generated document will always carry the filename
of the main document. This is inconvenient if
several child files are to be compiled and
to be kept for distribution.
\end{itemize}

The present package provides a simple interface
to make child files individually compilable by \LaTeX{}.
Compiling a child file then has the same effect as compiling
the main file with an |\includeonly| command
to select the appropriate child.
Moreover the generated document will carry the name of the child
rather than the main file.
This resolves all three above issues.

This feature is meant to make the editing of books,
thesis documents and lecture notes somewhat more convenient.
However, the package can also be used efficiently for
composing a series of documents (such as exercise sheets)
which are typically distributed individually.
It then assists the author in generating the individual documents
(potentially in different versions)
as well as a document containing the collected series.
Another application is in developing style files
or other kinds of included material
where compilation of the style file could redirect
to a sample or test file.

%%%%%%%%%%%%%%%%%%%%%%%%%%%%%%%%%%%%%%%%%%%%%%%%%%%%%%%%%%%%%%%%%%%%%%%%%%%%%%%%
%%%%%%%%%%%%%%%%%%%%%%%%%%%%%%%%%%%%%%%%%%%%%%%%%%%%%%%%%%%%%%%%%%%%%%%%%%%%%%%%
\section{Usage}

First of all, the package \textsf{childdoc} is \emph{not} a standard
\LaTeXe{} |.sty| style file! Therefore it needs to be invoked in
a non-standard way.

%%%%%%%%%%%%%%%%%%%%%%%%%%%%%%%%%%%%%%%%%%%%%%%%%%%%%%%%%%%%%%%%%%%%%%%%%%%%%%%%
\subsection{Included Files}
\label{sec:include}

%%%%%%%%%%%%%%%%%%%%%%%%%%%%%%%%%%%%%%%%
\DescribeMacro{\childdocmain}
To use the package, add the commands
\begin{center}
\begin{tabular}{l}
|\input{childdoc.def}|\\
|\childdocmain{}|\\
\end{tabular}
\end{center}
at the very top of the main \LaTeX{} file,
in particular \emph{before} the |\documentclass| statement!
The argument of |\childdocmain| should be left empty
(but it must be present).

%%%%%%%%%%%%%%%%%%%%%%%%%%%%%%%%%%%%%%%%
\DescribeMacro{\childdocof}
Furthermore, add the commands
\begin{center}
\begin{tabular}{l}
|\input{childdoc.def}|\\
|\childdocof{|\textit{main}|}|\\
\end{tabular}
\end{center}
at the top of every child file \textit{child}
which is included by |\include{|\textit{child}|}|
from within the main file
(or at least for those files to be compiled individually).
The argument \textit{main} must be the filename of the main file.

There are a couple of
considerations in setting up the main and child documents:

%%%%%%%%%%%%%%%%%%%%%%%%%%%%%%%%%%%%%%%%
\paragraph{Restrictions.}

Please note the following restrictions:
\begin{itemize}
\item
|\childdocmain| must be called with one argument \textit{main}
to ensure compatibility with earlier version of the package.
It must either be empty (|\childdocmain{}|)
or precisely match the filename of the main file in which it is specified.
See \secref{sec:detection} for further information.
\item
The filename \textit{main} must be specified without the |.tex| extension.
\item
The filename \textit{main} is case sensitive
(even in case-insensitive file systems)
due to internal string comparison.
\item
The argument \textit{main} should be fully expanded, it cannot be a macro.
\item
Subdirectories and special characters should be avoided in filenames.
\item
The command |\childdocmain{|\textit{main}|}| must be followed by a whitespace.
It should not be followed immediately by another command
or by a comment mark `|%|'.
This is because the \TeX{} parser reads the token immediately following
the argument of |\childdocmain| and puts it
at the beginning of every child section;
however, a white\-space is ignored.
\end{itemize}

%%%%%%%%%%%%%%%%%%%%%%%%%%%%%%%%%%%%%%%%
\paragraph{Content of Main File.}

It is advisable to place all content in the child files included by |\include|.
Any output contained in the main file will appear in all child documents
unless suppressed manually;
it cannot be suppressed automatically by the |\includeonly| directive
and thus should normally be avoided.
A method to include some content in the main file
by means of conditional processing is described in \secref{sec:conditional}.

%%%%%%%%%%%%%%%%%%%%%%%%%%%%%%%%%%%%%%%%
\paragraph{Page Numbering.}

When only a part of the document is compiled,
the appropriate numbering of pages
(as well as other status parameters)
is determined from the |.aux| files.
The latter contain information from previous passes.
However this information needs to propagate through
all intermediate child documents.
Therefore the page numbering in child documents may well
be inconsistent until the complete document is compiled at least once.

A useful (if unconventional) way to always ensure a consistent
page numbering is to restart the numbering in each child document
and denote the pages by `\textit{child}|.|\textit{page}'
where \textit{child} represents the chapter/section number of the child file.
This can be achieved by the command
|\numberwithin{page}{|\textit{child}|}|
of the \textsf{amsmath} package
where \textit{child} can be |chapter| or |section|
depending on the chosen structuring.
Alternatively, one can modify the macro |\thepage| appropriately
and reset the counter |page| at the start of each child file.

%%%%%%%%%%%%%%%%%%%%%%%%%%%%%%%%%%%%%%%%%%%%%%%%%%%%%%%%%%%%%%%%%%%%%%%%%%%%%%%%
\subsection{Conditional Processing}
\label{sec:conditional}

The package provides a mechanism to compile different versions
of a document. To customise the versions further some conditional processing
can come in handy to distinguish which version is being compiled.
The package provides two macros to describe the compilation context:

%%%%%%%%%%%%%%%%%%%%%%%%%%%%%%%%%%%%%%%%
\DescribeMacro{\ifchilddoc}
The conditional |\ifchilddoc| distinguishes between the compilation of
child documents and the main document:
%
\begin{center}
|\ifchilddoc |\textit{child-code}| |[|\||else |\textit{main-code}]| \||fi|
\end{center}

%%%%%%%%%%%%%%%%%%%%%%%%%%%%%%%%%%%%%%%%
\DescribeMacro{\childdocname}
\DescribeMacro{\childdocjob}
The macro |\childdocname| contains the filename (without extension)
of the main or child file being processed.
Note that |\childdocjob| will always contain the name of the main file.

%%%%%%%%%%%%%%%%%%%%%%%%%%%%%%%%%%%%%%%%
\paragraph{Title Page.}

Conditional processing can be used to include a title or banner page
in the main document when proper precautions are taken.
Importantly, the code in the main file should ensure that the page counter
(as well as other status parameters which are stored in the |.aux| files)
takes the same value after the conditional processing.
Otherwise the page numbers may take divergent values
depending on which part is compiled.

For example, a title page could be declared by:
%
\begin{center}
\begin{tabular}{l}
|\ifchilddoc\||else|\\
|\addtocounter{page}{-1}|\\
\textit{code for title page}\\
|\newpage|\\
|\||fi|
\end{tabular}
\end{center}
%
A banner page for the child documents can be generated by:
%
\begin{center}
\begin{tabular}{l}
|\ifchilddoc|\\
|\addtocounter{page}{-1}|\\
\textit{code for banner page}\\
|\newpage|\\
|\||fi|
\end{tabular}
\end{center}
%
Here one could write a message such as:
\begin{center}
|This is the part \childdocname{} of \childdocjob{}.|
\end{center}

%%%%%%%%%%%%%%%%%%%%%%%%%%%%%%%%%%%%%%%%%%%%%%%%%%%%%%%%%%%%%%%%%%%%%%%%%%%%%%%%
\subsection{Flags}
\label{sec:flags}

The package makes it easy to generate different versions
of the main or child documents.
To this end compilation flags can be defined
and assigned different default values.
They will be particularly useful in conjunction
with the forwarding mechanism described in \secref{sec:forward}.

For example, it may be useful to have a flag |\version|
which can be set to |draft| or |final|.
The document source will contain some conditional code
depending on the value of |\version|.
Suppose further, the flag should default to |final| for the main file
and to |draft| for child files
which is a natural assignment for editing the document.
This is achieved by placing the following code
in the preamble of the main document
(below the |\childdocmain| directive):
%
\begin{center}
\begin{tabular}{l}
|\ifchilddoc|\\
|\providecommand{\version}{draft}|\\
|\||else|\\
|\providecommand{\version}{final}|\\
|\||fi|
\end{tabular}
\end{center}
%
The definition by |\providecommand| makes sure
that previous definitions are not overwritten.
Further statements |\providecommand{\version}{...}|
can thus be added before the above code to override it.

For the main file, one might add a line
(between |\childdocmain| and the above block)
%
\begin{center}
|%\ifchilddoc\||else\providecommand{\version}{draft}\||fi|
\end{center}
%
which can be uncommented to produce a draft version.
Likewise one can add a line to the very top of a child file
(above the |\childdocof{|\textit{main}|}| directive)
%
\begin{center}
|%\providecommand{\version}{final}|
\end{center}
%
which can be uncommented to produce the final version of this child document.

%%%%%%%%%%%%%%%%%%%%%%%%%%%%%%%%%%%%%%%%%%%%%%%%%%%%%%%%%%%%%%%%%%%%%%%%%%%%%%%%
\subsection{Forwarding}
\label{sec:forward}

Different versions of the main or child documents
using compilation flags as described in \secref{sec:flags}
can be (permanently) stored in different files
for convenient compilation, viewing and distribution.
To this end, the package defines a command
to pass on compilation to a different file:

%%%%%%%%%%%%%%%%%%%%%%%%%%%%%%%%%%%%%%%%
\DescribeMacro{\childdocforward}
The command |\childdocforward| redirects processing to
another source file:
%
\begin{center}
\begin{tabular}{l}
|\input{childdoc.def}|\\
|\childdocforward[|\textit{main}|]{|\textit{dest}|}|\\
\end{tabular}
\end{center}
%
The argument \textit{dest} is the destination file
(without extension).
It should be the main file or one of the child files.
Note that further \textsf{childdoc} directives
such as |\childdocof| and |\childdocforward|
in the indicated file will be processed in this form.
The optional argument \textit{main}
passes on directly to the main file \textit{main}
while pretending to compile the child \textit{dest}.
This form behaves as if \textit{dest}
issues |\childdocof{|\textit{main}|}| right away,
and no further \textsf{childdoc} directives will be processed.

%%%%%%%%%%%%%%%%%%%%%%%%%%%%%%%%%%%%%%%%
\DescribeMacro{\...prefix}
In the alternative form |\childdocforwardprefix|,
%
\begin{center}
\begin{tabular}{l}
|\input{childdoc.def}|\\
|\childdocforwardprefix[|\textit{main}|]{|\textit{prefix}|}{|\textit{dest}|}|
\end{tabular}
\end{center}
%
the destination file is determined by a pattern
depending on the current file:
To make this work, the current file must be called
`{\textit{prefix}\hspace{0.2em}\textit{suffix}}'
with \textit{prefix} matching precisely the argument.
Processing is then passed on to the file
`{\textit{dest}\hspace{0.2em}\textit{suffix}}'.
Surely, the same effect is achieved by
directly specifying the
argument `{\textit{dest}\hspace{0.2em}\textit{suffix}}'
in the first form.
However, that requires to set up a different file
for each child. With the alternative form of the command
all these files can have exactly the same content
which simplifies setting them up and maintaining them.

For example, the following file |draft.tex|
with a compilation flag |\version| as described in \secref{sec:flags}
compiles the main document as a draft:
%
\begin{center}
\begin{tabular}{l}
|\def\version{draft}|\\
|\input{childdoc.def}|\\
|\childdocforward{|\textit{main}|}|
\end{tabular}
\end{center}
%
Likewise, the following files |final|\textit{nn}|.tex|
compile the final version of the child document
|child|\textit{nn}|.tex|:
%
\begin{center}
\begin{tabular}{l}
|\def\version{final}|\\
|\input{childdoc.def}|\\
|\childdocforwardprefix{final}{child}|
\end{tabular}
\end{center}
%

Note that when several versions of a main file and/or of each child file
are to be generated, it may be convenient to set up a |Makefile| or
shell script to automatise the process.

%%%%%%%%%%%%%%%%%%%%%%%%%%%%%%%%%%%%%%%%%%%%%%%%%%%%%%%%%%%%%%%%%%%%%%%%%%%%%%%%
\subsection{Command Line Processing}
\label{sec:commandline}

The effect of redirection files can also be achieved by invoking
the \LaTeX{} compiler with a more elaborate command line.
Most conveniently this should be done as part
of a shell script or a |Makefile|.

When using \textsf{childdoc} in the main file, the following
command lines effectively perform a redirection
(note that depending on the shell being used,
backslashes may have to be doubled: `|\|' $\to$ `|\\|'):
%
\begin{center}
|... -jobname "|\textit{target}|" |\\|"|[\textit{flags}]%
|\input{childdoc.def}\childdocforward[|\textit{main}|]{|\textit{dest}|}"|
\end{center}
%
Here \textit{target} is the name of the output file,
\textit{main} is the name of the main file
and \textit{dest} is the name of the main or child file to be processed
(all filenames without extensions).
The optional argument \textit{main} can be omitted
if \textit{main} matches \textit{dest}.
Optionally, compilation \textit{flags} can be defined via |\def| commands.
This command line makes the \TeX{} engine believe
it is compiling the file \textit{target}
whose content is specified as the latter parameter.
The provided code then forwards the processing to
\textit{main} or \textit{dest} as described in \secref{sec:forward}.

%%%%%%%%%%%%%%%%%%%%%%%%%%%%%%%%%%%%%%%%%%%%%%%%%%%%%%%%%%%%%%%%%%%%%%%%%%%%%%%%
\subsection{Include by Input}
\label{sec:input}

Including child documents by |\include| has some restrictions by design.
Most notably, the content of a child document always occupies
its own set of pages; pages cannot be shared between child documents.
Usually, this behaviour makes perfect sense
because each child document contain an essential part of the document.
However, in some situations it may be desirable to compose
a document from a collection of parts
without having mandatory page breaks between then.
For this case, the package
provides a mechanism to include parts
by |\input| which can also be processed individually.
However, by construction this mechanism
requires manual handling of the content to be output.

%%%%%%%%%%%%%%%%%%%%%%%%%%%%%%%%%%%%%%%%
\DescribeMacro{\ifchilddocmanual}
The main file should be prepared as usual, see \secref{sec:include}.
However, the document body must make a distinction
between processing of an individual part and of the main document, e.g.:
%
\begin{center}
\begin{tabular}{l}
|\ifchilddocmanual|\\
|\input{\childdocname}|\\
|\||else|\\
\textit{document body with }|\input{|\textit{part}|}|\\
|\||fi|
\end{tabular}
\end{center}
%
The conditional |\ifchilddocmanual| is true whenever
a part to be included by |\input| is being compiled,
and the name of the part is stored in |\childdocname|.

%%%%%%%%%%%%%%%%%%%%%%%%%%%%%%%%%%%%%%%%
\DescribeMacro{\childdocby}
Each part to be included by |\input| should start with:
%
\begin{center}
\begin{tabular}{l}
|\input{childdoc.def}|\\
|\childdocby{|\textit{main}|}|\\
\end{tabular}
\end{center}
%
The directive |\childdocby| is similar to |\childdocof|
described in \secref{sec:include},
but the subsequent selection of content must be done manually.
To that end, both |\ifchilddoc| and |\ifchilddocmanual|
will be true upon processing of a part,
and the name of the part is stored in |\childdocname|.
Note that |\jobname| will be set to the filename of the current part
so that each part receives an individual |.aux| file
that does not interfere with the |.aux| file(s) of the main document.
This behaviour can be altered by the alternative form
|\childdocby[*]{|\textit{main}|}| (with a non-empty optional argument)
which uses the |.aux| file of the main document
by setting |\jobname| to \textit{main}.

%%%%%%%%%%%%%%%%%%%%%%%%%%%%%%%%%%%%%%%%%%%%%%%%%%%%%%%%%%%%%%%%%%%%%%%%%%%%%%%%
\subsection{Driver Development}
\label{sec:driver}

The \textsf{childdoc} mechanism can also be use for the development
of definition files such as \LaTeX{} styles or classes.
This case differs from the above setup with multiple parts
included by |\include| in that no |\includeonly| should be invoked.
This can be achieved by starting the include file
(before |\ProvidesPackage|) with:
%
\begin{center}
\begin{tabular}{l}
|\input{childdoc.def}|\\
|\childdocforward{|\textit{main}|}|\\
\end{tabular}
\end{center}
%
or alternatively with:
%
\begin{center}
\begin{tabular}{l}
|\input{childdoc.def}|\\
|\childdocby{|\textit{main}|}|\\
\end{tabular}
\end{center}
%
Both forms have slightly different effects as described above.
The main file is prepared as usual, see \secref{sec:include}.

%%%%%%%%%%%%%%%%%%%%%%%%%%%%%%%%%%%%%%%%%%%%%%%%%%%%%%%%%%%%%%%%%%%%%%%%%%%%%%%%
\subsection{Legacy Detection}
\label{sec:detection}

The directive |\childdocmain| in the main file can detect
whether the complete document or merely a child is to be compiled
even without using the directive |\childdocof|.
This method is deprecated because it is less robust
and there is no compelling reason to use it;
it is merely provided for backward compatibility
and it may be removed in future versions.

If the detection mechanism is to be used,
it is mandatory to correctly specify
the filename of the main file as the argument of |\childdocmain|:
%
\begin{center}
\begin{tabular}{l}
|\input{childdoc.def}|\\
|\childdocmain{|\textit{main}|}|\\
\end{tabular}
\end{center}
%
If |\jobname| does not match the argument \textit{main} of |\childdocmain|,
it is assumed that |\jobname| points to the child file to be compiled.
When using |\childdocmain| with the main file specified as argument,
it suffices to start a child file
with just |\input{|\textit{main}|}|
without loading of the package and using |\childdocof|.
If instead all processing is done
with the appropriate \textsf{childdoc} directives,
the argument of \textit{main} of |\childdocmain| can be empty.

An alternative version of the command line processing described
in \secref{sec:commandline} using the detection mechanism reads:
%
\begin{center}
|... -jobname "|\textit{target}|" "|[\textit{flags}]%
[|\def\jobname{|\textit{dest}|}|]|\input{|\textit{main}|}"|
\end{center}

%%%%%%%%%%%%%%%%%%%%%%%%%%%%%%%%%%%%%%%%%%%%%%%%%%%%%%%%%%%%%%%%%%%%%%%%%%%%%%%%
\subsection{Manual Code}
\label{sec:manual}

In case one cannot be certain whether the definitions file |childdoc.def|
is installed on the target \TeX{} distribution
and one prefers not to ship it,
it is conceivable to paste a few relevant commands into the sources.

To that end, drop all statements |\input{childdoc.def}|
and perform the replacements as outlined below.
Instead of |\childdocmain{|\textit{main}|}| add the following code
to the top of the main file:
%
\begin{center}
\begin{tabular}{l}
|\||ifdefined\childdocname\endinput\||fi\newif\ifchilddoc|\\
|\edef\childdocname{\scantokens\expandafter{\jobname\noexpand}}|\\
|\def\childdocmain{|\textit{main}|}\||ifx\childdocmain\childdocname\||else|\\
|\childdoctrue\includeonly{\childdocname}\let\jobname\childdocmain\||fi|\\
\end{tabular}
\end{center}
%
Instead of |\childdocof{|\textit{main}|}| just include the main file
at the top of each child file:
%
\begin{center}
|\input{|\textit{main}|}|
\end{center}
%
A simple redirection |\childdocforward{|\textit{dest}|}| is achieved by:
%
\begin{center}
|\def\jobname{|\textit{dest}|}\input{\jobname}|
\end{center}
%
The redirection with prefix
|\childdocforwardprefix[|\textit{prefix}|]{|\textit{dest}|}|
is accomplished by:
%
\begin{center}
\begin{tabular}{l}
|{\edef\jobname{\scantokens\expandafter{\jobname\noexpand}}|\\
|\def\redirectjob |\textit{prefix}|#1~~~{\gdef\jobname{|\textit{dest}|#1}}|\\
|\expandafter\redirectjob\jobname~~~}\input{\jobname}|
\end{tabular}
\end{center}

In an alternative approach,
child documents can be compiled by a specific command line
without additional code or specific definitions:
%
\begin{center}
|... -jobname "|\textit{target}|" "|[\textit{flags}]%
|\includeonly{|\textit{dest}|}\input{|\textit{main}|}"|
\end{center}
%

%%%%%%%%%%%%%%%%%%%%%%%%%%%%%%%%%%%%%%%%%%%%%%%%%%%%%%%%%%%%%%%%%%%%%%%%%%%%%%%%
%%%%%%%%%%%%%%%%%%%%%%%%%%%%%%%%%%%%%%%%%%%%%%%%%%%%%%%%%%%%%%%%%%%%%%%%%%%%%%%%
\section{Information}

%%%%%%%%%%%%%%%%%%%%%%%%%%%%%%%%%%%%%%%%%%%%%%%%%%%%%%%%%%%%%%%%%%%%%%%%%%%%%%%%
\subsection{Copyright}

Copyright \copyright{} 2017--2018 Niklas Beisert

This work may be distributed and/or modified under the
conditions of the \LaTeX{} Project Public License, either version 1.3
of this license or (at your option) any later version.
The latest version of this license is in
  \url{http://www.latex-project.org/lppl.txt}
and version 1.3 or later is part of all distributions of \LaTeX{}
version 2005/12/01 or later.

This work has the LPPL maintenance status `maintained'.

The Current Maintainer of this work is Niklas Beisert.

This work consists of the files |README.txt|, |childdoc.ins| and |childdoc.dtx|
as well as the derived files |childdoc.def|, |cdocsamp.tex|
with |cdocsch1.tex|, |cdocsch2.tex|, |cdocspt3.tex|, |cdocspt4.tex|,
|cdocsdrf.tex|, |cdocsfn1.tex|, |cdocsfn2.tex|
as well as |childdoc.pdf|.

%%%%%%%%%%%%%%%%%%%%%%%%%%%%%%%%%%%%%%%%%%%%%%%%%%%%%%%%%%%%%%%%%%%%%%%%%%%%%%%%
\subsection{Files and Installation}

The package consists of the files:
%
\begin{center}
\begin{tabular}{ll}
    |README.txt|   & readme file \\
    |childdoc.ins| & installation file \\
    |childdoc.dtx| & source file \\
    |childdoc.def| & definition file \\
    |cdocsamp.tex| & sample main file \\
    |cdocsch1.tex| & sample include file \\
    |cdocsch2.tex| & sample include file \\
    |cdocspt3.tex| & sample part file \\
    |cdocspt4.tex| & sample part file \\
    |cdocsdrf.tex| & sample redirection file \\
    |cdocsfn1.tex| & sample redirection file \\
    |cdocsfn2.tex| & sample redirection file \\
    |childdoc.pdf| & manual
\end{tabular}
\end{center}
%
The distribution consists of the files
|README.txt|, |childdoc.ins| and |childdoc.dtx|.
%
\begin{itemize}
\item
Run (pdf)\LaTeX{} on |childdoc.dtx|
to compile the manual |childdoc.pdf| (this file).
\item
Run \LaTeX{} on |childdoc.ins| to create the definitions file |childdoc.def|
and the sample |cdocsamp.tex| with include files
|cdocsch1.tex|, |cdocsch2.tex|, |cdocspt3.tex|, |cdocspt4.tex|,
|cdocsdrf.tex|, |cdocsfn1.tex|, |cdocsfn2.tex|.
Then copy the file |childdoc.def| to an appropriate directory of your \LaTeX{}
distribution, e.g.\ \textit{texmf-root}|/tex/latex/childdoc|.
\end{itemize}

%%%%%%%%%%%%%%%%%%%%%%%%%%%%%%%%%%%%%%%%%%%%%%%%%%%%%%%%%%%%%%%%%%%%%%%%%%%%%%%%
\subsection{Related CTAN Packages}

There are several other packages which offer a similar functionality:
%
\begin{itemize}
\item
The packages
\href{http://ctan.org/pkg/docmute}{\textsf{docmute}},
\href{http://ctan.org/pkg/includex}{\textsf{includex}} and
\href{http://ctan.org/pkg/standalone}{\textsf{standalone}}
provide commands to include only the document body of
a child file thus allowing both files to be compiled individually.
\item
The packages \href{http://ctan.org/pkg/subdocs}{\textsf{subdocs}}
and \href{http://ctan.org/pkg/subfiles}{\textsf{subfiles}}
provide structures in which the main and child documents can be
encapsulated and allowing them to be compiled individually.
The inclusion mechanism is different from the conventional |\include|.
\item
The package \href{http://ctan.org/pkg/combine}{\textsf{combine}}
is an elaborate solution to combine several documents into one.
\end{itemize}
%
See also the CTAN topic \href{http://ctan.org/topic/subdocs}{\textsf{subdocs}}
for further related packages.
The present package differs from the above solutions in that
a document structure constructed with the conventional |\include| mechanism
just needs two extra commands at the top of every file
such that all constituent files can be compiled individually.

%%%%%%%%%%%%%%%%%%%%%%%%%%%%%%%%%%%%%%%%%%%%%%%%%%%%%%%%%%%%%%%%%%%%%%%%%%%%%%%%
%\subsection{Feature Suggestions}
%
%The following is a list of features which may be useful for future
%versions of this package:
%%
%\begin{itemize}
%\item
%\ldots
%\end{itemize}

%%%%%%%%%%%%%%%%%%%%%%%%%%%%%%%%%%%%%%%%%%%%%%%%%%%%%%%%%%%%%%%%%%%%%%%%%%%%%%%%
\subsection{Revision History}

%%%%%%%%%%%%%%%%%%%%%%%%%%%%%%%%%%%%%%%%
\paragraph{v2.0:} 2018/12/30

\begin{itemize}
\item
immediate forward processing
\item
added |\childdocby| mechanism
\item
manual restructured
\end{itemize}

%%%%%%%%%%%%%%%%%%%%%%%%%%%%%%%%%%%%%%%%
\paragraph{v1.6:} 2018/01/17

\begin{itemize}
\item
application for development of include files
\item
corrections to manual
\end{itemize}

%%%%%%%%%%%%%%%%%%%%%%%%%%%%%%%%%%%%%%%%
\paragraph{v1.5:} 2017/05/21

\begin{itemize}
\item
more complete structuring introduced
\item
|\childdocof| introduced
\item
|\childdoc| renamed to |\childdocmain|
\item
|\childredirect| renamed to |\childdocforward| and |\childdocforwardprefix|
and functionality expanded
\end{itemize}

%%%%%%%%%%%%%%%%%%%%%%%%%%%%%%%%%%%%%%%%
\paragraph{v1.0:} 2017/04/27

\begin{itemize}
\item
manual and install package
\item
first version published on CTAN
\end{itemize}

%%%%%%%%%%%%%%%%%%%%%%%%%%%%%%%%%%%%%%%%
\paragraph{v0.6:} 2017/04/26

\begin{itemize}
\item
redirection mechanism added
\end{itemize}

%%%%%%%%%%%%%%%%%%%%%%%%%%%%%%%%%%%%%%%%
\paragraph{v0.5:} 2017/04/26

\begin{itemize}
\item
functionality in definition file
\end{itemize}


%%%%%%%%%%%%%%%%%%%%%%%%%%%%%%%%%%%%%%%%%%%%%%%%%%%%%%%%%%%%%%%%%%%%%%%%%%%%%%%%
%%%%%%%%%%%%%%%%%%%%%%%%%%%%%%%%%%%%%%%%%%%%%%%%%%%%%%%%%%%%%%%%%%%%%%%%%%%%%%%%
%%%%%%%%%%%%%%%%%%%%%%%%%%%%%%%%%%%%%%%%%%%%%%%%%%%%%%%%%%%%%%%%%%%%%%%%%%%%%%%%
\appendix

\settowidth\MacroIndent{\rmfamily\scriptsize 000\ }

 \DocInput{childdoc.dtx}

\end{document}
%</driver>
% \fi
%
% %%%%%%%%%%%%%%%%%%%%%%%%%%%%%%%%%%%%%%%%%%%%%%%%%%%%%%%%%%%%%%%%%%%%%%%%%%%%%%
% %%%%%%%%%%%%%%%%%%%%%%%%%%%%%%%%%%%%%%%%%%%%%%%%%%%%%%%%%%%%%%%%%%%%%%%%%%%%%%
% \section{Sample}
%\iffalse
%<*samplemain>
%\fi
%
% The following presents a sample document
% with two chapters, two parts, a title page,
% a compile flag as well as three forwarding files to set the flag.
% It consists of eight |.tex| files:
% \begin{center}
% \begin{tabular}{ll}
% |cdocsamp.tex|&main file\\
% |cdocsch1.tex|&include file for chapter 1\\
% |cdocsch2.tex|&include file for chapter 2\\
% |cdocspt3.tex|&include file for part 3\\
% |cdocspt4.tex|&include file for part 4\\
% |cdocsdrf.tex|&forwarding file for main file in draft mode\\
% |cdocsfi1.tex|&forwarding file for final version of chapter 1\\
% |cdocsfi2.tex|&forwarding file for final version of chapter 2\\
% \end{tabular}
% \end{center}
% Each of the eight files can be compiled directly by the \LaTeX{} compiler.
%
% %%%%%%%%%%%%%%%%%%%%%%%%%%%%%%%%%%%%%%
% \paragraph{Main File.}
%
% The main file is called |cdocsamp.tex|.
%
% Load the \textsf{childdoc} definitions and
% declare the filename for the main document:
%    \begin{macrocode}
\input{childdoc.def}
\childdocmain{}
%    \end{macrocode}

% Optional override for |\version| flag:
%    \begin{macrocode}
%%\ifchilddoc\else\providecommand{\version}{draft}\fi
%    \end{macrocode}

% Define the default values for the |\version| flag
% (|final| for the main file and |draft| for childs):
%    \begin{macrocode}
\ifchilddoc
\providecommand{\version}{draft}
\else
\providecommand{\version}{final}
\fi
%    \end{macrocode}

% Load the standard document class:
%    \begin{macrocode}
\documentclass[12pt]{article}
%    \end{macrocode}

% Start the document body:
%    \begin{macrocode}
\begin{document}
%    \end{macrocode}

% Declare a title page.
% Print title, part of document being processed and version flag:
%    \begin{macrocode}
\addtocounter{page}{-1}
\begin{center}
{\LARGE\bfseries{}childdoc example\par}
\vspace{1cm}
\ifchilddoc
\ifchilddocmanual part\else chapter\fi:
`\childdocname' of `\childdocjob'\par
\else
main document: `\childdocjob'\par
\fi
version: \version\par
\end{center}
\newpage
%    \end{macrocode}

% Manually include selected file,
% otherwise process as usual:
%    \begin{macrocode}
\ifchilddocmanual
\section*{part `\childdocname'}
\input{\childdocname}
\else
%    \end{macrocode}

% Include the two chapters:
%    \begin{macrocode}
\include{cdocsch1}
\include{cdocsch2}
%    \end{macrocode}

% Include the two parts unless only chapters should be displayed:
%    \begin{macrocode}
\ifchilddoc\else
\section{part three}
\input{cdocspt3}
\section{part four}
\input{cdocspt4}
\fi
%    \end{macrocode}

% Process as usual until here:
%    \begin{macrocode}
\fi
%    \end{macrocode}

% End of document body:
%    \begin{macrocode}
\end{document}
%    \end{macrocode}
%\iffalse
%</samplemain>
%\fi
%
% %%%%%%%%%%%%%%%%%%%%%%%%%%%%%%%%%%%%%%
% \paragraph{Chapter Include Files.}
%
% The include files are called |cdocsch1.tex| and |cdocsch2.tex|.
%
%\iffalse
%<*samplechap1|samplechap2>
%\fi

% Optional override for |\version| flag:
%    \begin{macrocode}
%%\providecommand{\version}{final}
%    \end{macrocode}

% Include the main document:
%    \begin{macrocode}
\input{childdoc.def}
\childdocof{cdocsamp}
%    \end{macrocode}

%\iffalse
%</samplechap1|samplechap2>
%\fi
%
%\iffalse
%<*samplechap1>
%\fi
% Some text for chapter 1:
%    \begin{macrocode}
\section{one}
some text in chapter one
%    \end{macrocode}

%\iffalse
%</samplechap1>
%\fi
% Some text for chapter 2:
%\iffalse
%<*samplechap2>
%\fi
%    \begin{macrocode}
\section{two}
more text in chapter two
%    \end{macrocode}

%\iffalse
%</samplechap2>
%\fi
%
% %%%%%%%%%%%%%%%%%%%%%%%%%%%%%%%%%%%%%%
% \paragraph{Part Include Files.}
%
% The include files are called |cdocspt3.tex| and |cdocspt4.tex|.
%
%\iffalse
%<*samplepart3|samplepart4>
%\fi

% Optional override for |\version| flag:
%    \begin{macrocode}
%%\providecommand{\version}{final}
%    \end{macrocode}

% Include the main document:
%    \begin{macrocode}
\input{childdoc.def}
\childdocby{cdocsamp}
%    \end{macrocode}

%\iffalse
%</samplepart3|samplepart4>
%\fi
%
%\iffalse
%<*samplepart3>
%\fi
% Some text for part 3:
%    \begin{macrocode}
some text in part three
%    \end{macrocode}

%\iffalse
%</samplepart3>
%\fi
% Some text for part 4:
%\iffalse
%<*samplepart4>
%\fi
%    \begin{macrocode}
more text in part four
%    \end{macrocode}

%\iffalse
%</samplepart4>
%\fi
%
% %%%%%%%%%%%%%%%%%%%%%%%%%%%%%%%%%%%%%%
% \paragraph{Forwarding for a Complete Draft.}
%
% The following forwarding file |cdocsdrf.tex|
% compiles the main document in draft mode:
%\iffalse
%<*sampledraft>
%\fi
%    \begin{macrocode}
\def\version{draft}
\input{childdoc.def}
\childdocforward{cdocsamp}
%    \end{macrocode}

%\iffalse
%</sampledraft>
%\fi
%
% %%%%%%%%%%%%%%%%%%%%%%%%%%%%%%%%%%%%%%
% \paragraph{Forwarding for Final Version of the Chapters.}
%
% The following forwarding files |cdocsfn1.tex| and |cdocsfn2.tex|
% (with identical content)
% compile the final versions of the child documents
% |cdocsch1.tex| and |cdocsch2.tex|, respectively:
%\iffalse
%<*samplefinal>
%\fi
%    \begin{macrocode}
\def\version{final}
\input{childdoc.def}
\childdocforwardprefix[cdocsamp]{cdocsfn}{cdocsch}
%    \end{macrocode}

%\iffalse
%</samplefinal>
%\fi
%
% %%%%%%%%%%%%%%%%%%%%%%%%%%%%%%%%%%%%%%
% \paragraph{Command Line Processing.}
%
% The following three command lines generate the output files
% |cdocscld|, |cdocscl1| and |cdocscl2|
% which should be identical to
% |cdocsdrf|, |cdocsch1| and |cdocsfn2|, respectively:
% \begin{center}
% \begin{tabular}{l}
% |latex -jobname cdocscld \|\\
% |  "\def\version{draft}\input{childdoc.def}\childdocforward{cdocsamp}"|\\
% |latex -jobname cdocscl1 \|\\
% |  "\input{childdoc.def}\childdocforward[cdocsamp]{cdocsch1}"|\\
% |latex -jobname cdocscl2 \|\\
% |  "\def\version{final}\input{childdoc.def}\childdocforward{cdocsch2}"|
% \end{tabular}
% \end{center}
% Note that the trailing backslash on each first line
% merely continues the input to the second line
% (for convenient cut ant paste).
% Furthermore, the command |latex| can be replaced by any
% of its alternative versions such as |pdflatex|.
%
% %%%%%%%%%%%%%%%%%%%%%%%%%%%%%%%%%%%%%%%%%%%%%%%%%%%%%%%%%%%%%%%%%%%%%%%%%%%%%%
% %%%%%%%%%%%%%%%%%%%%%%%%%%%%%%%%%%%%%%%%%%%%%%%%%%%%%%%%%%%%%%%%%%%%%%%%%%%%%%
% \section{Implementation}
%\iffalse
%<*package>
%\fi
%
% This section describes the definitions file |childdoc.def|.

% The definitions cannot be loaded using |\usepackage| or |\RequirePackage|
% which has a mechanism to prevent loading a style file more than once.
% When loading the definitions by means of |\input|
% multiple instances have to be prevented manually:
%\iffalse
%This code needs to be before the `\ProvidesFile' directive
%which is defined at the beginning of this file.
%Therefore it is also placed there and commented out here.
%</package>
%<*discard>
%\fi
%    \begin{macrocode}
\ifdefined\childdocmain\endinput\fi
%    \end{macrocode}
%\iffalse
%</discard>
%<*package>
%\fi
%
% \macro{\ifchilddoc}
% \macro{\ifchilddocmanual}
% The conditional |\ifchilddoc| tells whether a
% child (true) or main (false) document is being compiled.
% The conditional |\ifchilddocmanual| tells whether
% the |\includeonly| mechanism is used (false) or
% the selection of child files must be performed manually (true).
% The definitions initialise to false:
%    \begin{macrocode}
\newif\ifchilddoc
\newif\ifchilddocmanual
%    \end{macrocode}

% \macro{\childdocname}
% \macro{\childdocjob}
% The macro |\childdocname| stores the name of the main document
% to be compiled. The macro |\childdocjob| stores the name of
% the document on which the \LaTeX{} compiler was originally invoked.
% The content of |\jobname| cannot be compared
% to filenames specified in the source due to different catcodes.
% The following code rescans |\jobname|, stores the result
% in |\childdocname| and saves a copy in |\childdocjob|:
%    \begin{macrocode}
\edef\childdocname{\scantokens\expandafter{\jobname\noexpand}}
\let\childdocjob\childdocname
%    \end{macrocode}

% \macro{\childdocdisable}
% The macro |\childdocdisable| prevents the main file
% from being processed more than once.
% At this stage, the main document command |\childdocmain|
% is assumed to be called once again where it should do nothing.
% Any subsequent call to it should prevent
% a secondary processing of the main document
% It overwrites the forwarding commands
% |\childdocof| and |\childdocforward|
% with empty macros to prevent further inclusions of the main document:
%    \begin{macrocode}
\newcommand{\childdocdisable}
{
  \renewcommand{\childdocmain}[1]{\renewcommand{\childdocmain}[1]{\endinput}}
  \renewcommand{\childdocof}[1]{}
  \renewcommand{\childdocby}[2][]{}
  \renewcommand{\childdocforward}[2][]{}
  \renewcommand{\childdocdisable}{}
}
%    \end{macrocode}

% \macro{\childdocmain}
% The macro |\childdocmain| is to be called at the top of the main file
% with nothing or the main filename (without extension) as argument.
% First, it breaks loops.
% If the argument is not empty and does not match |\childdocname|
% (which is set by the first inclusion of |childdoc.def|),
% |\ifchilddoc| is set to true, |\includeonly| is applied to the child file
% and |\jobname| is set to the main file
% (for proper handling of |.aux| files):
%    \begin{macrocode}
\newcommand{\childdocmain}[1]
{
  \childdocdisable\childdocmain{}
  \if?#1?\else
    \begingroup
      \def\childdoctmp{#1}
      \ifx\childdoctmp\childdocname
        \def\childdoctmp{}
      \else
        \def\childdoctmp
        {
          \childdoctrue
          \includeonly{\childdocname}
          \def\childdocjob{#1}
          \def\jobname{#1}
        }
      \fi
      \expandafter
    \endgroup
    \childdoctmp
  \fi
}
%    \end{macrocode}

% \macro{\childdocof}
% The command |\childdocof| redirects
% compilation to the main file |#1|.
%    \begin{macrocode}
\newcommand{\childdocof}[1]
{
  \childdocdisable
  \childdoctrue
  \includeonly{\childdocname}
  \def\jobname{#1}
  \def\childdocjob{#1}
  \input{#1}
}
%    \end{macrocode}

% \macro{\childdocby}
% The command |\childdocby| ....
%    \begin{macrocode}
\newcommand{\childdocby}[2][]
{
  \childdocdisable
  \childdoctrue
  \childdocmanualtrue
  \if?#1?\else
    \def\jobname{#2}
  \fi
  \def\childdocjob{#2}
  \input{#2}
  \endinput
}
%    \end{macrocode}

% \macro{\childdocforward}
% The command |\childdocforward| redirects
% compilation to the main file or
% (if the optional argument is given) a child file.
% Parameters are set as if the main file
% or a child file starting with |\childdocof| was compiled.
% Then compilation is handed over to the main file:
%    \begin{macrocode}
\newcommand{\childdocforward}[2][]
{
  \begingroup
    \if?#1?
      \def\childdoctmp
      {
        \def\childdocname{#2}
        \def\childdocjob{#2}
        \def\jobname{#2}
        \input{#2}
        \endinput
      }
    \else
      \def\childdoctmp
      {
        \childdocdisable
        \def\childdocname{#2}
        \childdoctrue
        \includeonly{#2}
        \def\childdocjob{#1}
        \def\jobname{#1}
        \input{#1}
        \endinput
      }
    \fi
    \expandafter
  \endgroup
  \childdoctmp
}
%    \end{macrocode}

% \macro{\childdocforwardprefix}
% The command |\childdocforwardprefix| redirects
% compilation to the main or a child file by means of a pattern.
% The prefix |#1| in the current filename is replaced by |#2|
% and the suffix of the current filename is kept
% (it is assumed that the filename does not contain the substring `|~~~|'
% which is used as a delimiter).
% Compilation is handed over to the new file by |\childdocforward|:
%    \begin{macrocode}
\newcommand{\childdocforwardprefix}[3][]
{
  \begingroup
    \def\childdocextract #2##1~~~{\def\childdoctmp{\childdocforward[#1]{#3##1}}}
    \expandafter\childdocextract\childdocname~~~
    \expandafter
  \endgroup
  \childdoctmp
}
%    \end{macrocode}

% \macro{\childdoc}
% The deprecated macro |\childdoc| is a legacy version of |\childdocmain|:
%    \begin{macrocode}
\newcommand{\childdoc}{\childdocmain}
%    \end{macrocode}

% \macro{\childdocredirect}
% The deprecated macro |\childdocredirect| is a legacy version
% of |\childdocforward| and |\childdocforwardprefix|:
%    \begin{macrocode}
\newcommand{\childdocredirect}[2][]
{
  \begingroup
    \if?#1?
      \def\childdoctmp{\childdocforward{#2}}
    \else
      \def\childdoctmp{\childdocforwardprefix{#1}{#2}}
    \fi
    \expandafter
  \endgroup
  \childdoctmp
}
%    \end{macrocode}

%\iffalse
%</package>
%\fi
%
\endinput

\childdocby{cdocsamp}
%    \end{macrocode}

%\iffalse
%</samplepart3|samplepart4>
%\fi
%
%\iffalse
%<*samplepart3>
%\fi
% Some text for part 3:
%    \begin{macrocode}
some text in part three
%    \end{macrocode}

%\iffalse
%</samplepart3>
%\fi
% Some text for part 4:
%\iffalse
%<*samplepart4>
%\fi
%    \begin{macrocode}
more text in part four
%    \end{macrocode}

%\iffalse
%</samplepart4>
%\fi
%
% %%%%%%%%%%%%%%%%%%%%%%%%%%%%%%%%%%%%%%
% \paragraph{Forwarding for a Complete Draft.}
%
% The following forwarding file |cdocsdrf.tex|
% compiles the main document in draft mode:
%\iffalse
%<*sampledraft>
%\fi
%    \begin{macrocode}
\def\version{draft}
% \iffalse
%
% childdoc.dtx Copyright (C) 2017-2018 Niklas Beisert
%
% This work may be distributed and/or modified under the
% conditions of the LaTeX Project Public License, either version 1.3
% of this license or (at your option) any later version.
% The latest version of this license is in
%   http://www.latex-project.org/lppl.txt
% and version 1.3 or later is part of all distributions of LaTeX
% version 2005/12/01 or later.
%
% This work has the LPPL maintenance status `maintained'.
%
% The Current Maintainer of this work is Niklas Beisert.
%
% This work consists of the files childdoc.dtx and childdoc.ins
% and the derived files childdoc.def and cdocsamp.tex with
% cdocsch1.tex, cdocsch2.tex, cdocsdrf.tex, cdocsfn1.tex, cdocsfn2.tex.
%
%<package>\ifdefined\childdocmain\endinput\fi
%<package>\ProvidesFile{childdoc.def}[2018/12/30 v2.0 child document driver]
%<samplemain>\ProvidesFile{cdocsamp.tex}[2018/12/30 v2.0 sample for childdoc]
%<*driver>
%\ProvidesFile{childdoc.drv}[2018/12/30 v2.0 childdoc reference manual file]
\PassOptionsToClass{10pt,a4paper}{article}
\documentclass{ltxdoc}

\usepackage[margin=35mm]{geometry}
\usepackage{hyperref}
\usepackage{hyperxmp}
\usepackage[usenames]{color}

\hypersetup{colorlinks=true}
\hypersetup{pdfstartview=FitH}
\hypersetup{pdfpagemode=UseNone}
\hypersetup{pdfsource={}}
\hypersetup{pdflang={en-UK}}
\hypersetup{pdfcopyright={Copyright 2017-2018 Niklas Beisert.
  This work may be distributed and/or modified under the
  conditions of the LaTeX Project Public License, either version 1.3
  of this license or (at your option) any later version.}}
\hypersetup{pdflicenseurl={http://www.latex-project.org/lppl.txt}}
\hypersetup{pdfcontactaddress={ETH Zurich, ITP, HIT K,
  Wolfgang-Pauli-Strasse 27}}
\hypersetup{pdfcontactpostcode={8093}}
\hypersetup{pdfcontactcity={Zurich}}
\hypersetup{pdfcontactcountry={Switzerland}}
\hypersetup{pdfcontactemail={nbeisert@itp.phys.ethz.ch}}
\hypersetup{pdfcontacturl={http://people.phys.ethz.ch/\xmptilde nbeisert/}}

\newcommand{\secref}[1]{\hyperref[#1]{section \ref*{#1}}}

\parskip1ex
\parindent0pt
\let\olditemize\itemize
\def\itemize{\olditemize\parskip0pt}

\begin{document}

\title{The \textsf{childdoc} Package}
\hypersetup{pdftitle={The childdoc Package}}
\author{Niklas Beisert\\[2ex]
  Institut f\"ur Theoretische Physik\\
  Eidgen\"ossische Technische Hochschule Z\"urich\\
  Wolfgang-Pauli-Strasse 27, 8093 Z\"urich, Switzerland\\[1ex]
  \href{mailto:nbeisert@itp.phys.ethz.ch}
  {\texttt{nbeisert@itp.phys.ethz.ch}}}
\hypersetup{pdfauthor={Niklas Beisert}}
\hypersetup{pdfsubject={Manual for the LaTeX2e Package childdoc}}
\date{30 December 2018, \textsf{v2.0}}
\maketitle

\begin{abstract}\noindent
\textsf{childdoc} is a \LaTeXe{} package
that enables the direct compilation
of document sections included by |\include|
to individual files.
\end{abstract}

\begingroup
\parskip0ex
\tableofcontents
\endgroup

%%%%%%%%%%%%%%%%%%%%%%%%%%%%%%%%%%%%%%%%%%%%%%%%%%%%%%%%%%%%%%%%%%%%%%%%%%%%%%%%
%%%%%%%%%%%%%%%%%%%%%%%%%%%%%%%%%%%%%%%%%%%%%%%%%%%%%%%%%%%%%%%%%%%%%%%%%%%%%%%%
\section{Introduction}

\LaTeX{} provides a mechanism to structure a large document (such as a book)
into a main file and several child files (containing the chapters)
using the |\include| command.
This mechanism is beneficial for documents
which span hundreds of pages in order to
make the source file(s) more manageable.
Moreover, compilation can be restricted to
selected child files by means of the |\includeonly| command.
The latter feature can be used to reduce the compilation time while editing
(this was significantly more useful in the earlier days of \LaTeX{})
or to generate a smaller document which is easier to navigate.
Another application of |\includeonly| is to generate
documents consisting of selected parts of the complete document.

However, there are a few drawbacks of the plain |\include| mechanism:
\begin{itemize}
\item
The child files cannot be compiled on their own,
they can only be compiled via the main file.
A naive editing environment
(such as a text editor with an option
to have the current file processed by \LaTeX)
may require one to switch to the main file before compiling;
attempting to compile the child file produces errors.
\item
The main file must be modified (each time)
to adjust the |\includeonly| command
to the present needs. This easily leaves the main file in a messy state.
\item
The generated document will always carry the filename
of the main document. This is inconvenient if
several child files are to be compiled and
to be kept for distribution.
\end{itemize}

The present package provides a simple interface
to make child files individually compilable by \LaTeX{}.
Compiling a child file then has the same effect as compiling
the main file with an |\includeonly| command
to select the appropriate child.
Moreover the generated document will carry the name of the child
rather than the main file.
This resolves all three above issues.

This feature is meant to make the editing of books,
thesis documents and lecture notes somewhat more convenient.
However, the package can also be used efficiently for
composing a series of documents (such as exercise sheets)
which are typically distributed individually.
It then assists the author in generating the individual documents
(potentially in different versions)
as well as a document containing the collected series.
Another application is in developing style files
or other kinds of included material
where compilation of the style file could redirect
to a sample or test file.

%%%%%%%%%%%%%%%%%%%%%%%%%%%%%%%%%%%%%%%%%%%%%%%%%%%%%%%%%%%%%%%%%%%%%%%%%%%%%%%%
%%%%%%%%%%%%%%%%%%%%%%%%%%%%%%%%%%%%%%%%%%%%%%%%%%%%%%%%%%%%%%%%%%%%%%%%%%%%%%%%
\section{Usage}

First of all, the package \textsf{childdoc} is \emph{not} a standard
\LaTeXe{} |.sty| style file! Therefore it needs to be invoked in
a non-standard way.

%%%%%%%%%%%%%%%%%%%%%%%%%%%%%%%%%%%%%%%%%%%%%%%%%%%%%%%%%%%%%%%%%%%%%%%%%%%%%%%%
\subsection{Included Files}
\label{sec:include}

%%%%%%%%%%%%%%%%%%%%%%%%%%%%%%%%%%%%%%%%
\DescribeMacro{\childdocmain}
To use the package, add the commands
\begin{center}
\begin{tabular}{l}
|\input{childdoc.def}|\\
|\childdocmain{}|\\
\end{tabular}
\end{center}
at the very top of the main \LaTeX{} file,
in particular \emph{before} the |\documentclass| statement!
The argument of |\childdocmain| should be left empty
(but it must be present).

%%%%%%%%%%%%%%%%%%%%%%%%%%%%%%%%%%%%%%%%
\DescribeMacro{\childdocof}
Furthermore, add the commands
\begin{center}
\begin{tabular}{l}
|\input{childdoc.def}|\\
|\childdocof{|\textit{main}|}|\\
\end{tabular}
\end{center}
at the top of every child file \textit{child}
which is included by |\include{|\textit{child}|}|
from within the main file
(or at least for those files to be compiled individually).
The argument \textit{main} must be the filename of the main file.

There are a couple of
considerations in setting up the main and child documents:

%%%%%%%%%%%%%%%%%%%%%%%%%%%%%%%%%%%%%%%%
\paragraph{Restrictions.}

Please note the following restrictions:
\begin{itemize}
\item
|\childdocmain| must be called with one argument \textit{main}
to ensure compatibility with earlier version of the package.
It must either be empty (|\childdocmain{}|)
or precisely match the filename of the main file in which it is specified.
See \secref{sec:detection} for further information.
\item
The filename \textit{main} must be specified without the |.tex| extension.
\item
The filename \textit{main} is case sensitive
(even in case-insensitive file systems)
due to internal string comparison.
\item
The argument \textit{main} should be fully expanded, it cannot be a macro.
\item
Subdirectories and special characters should be avoided in filenames.
\item
The command |\childdocmain{|\textit{main}|}| must be followed by a whitespace.
It should not be followed immediately by another command
or by a comment mark `|%|'.
This is because the \TeX{} parser reads the token immediately following
the argument of |\childdocmain| and puts it
at the beginning of every child section;
however, a white\-space is ignored.
\end{itemize}

%%%%%%%%%%%%%%%%%%%%%%%%%%%%%%%%%%%%%%%%
\paragraph{Content of Main File.}

It is advisable to place all content in the child files included by |\include|.
Any output contained in the main file will appear in all child documents
unless suppressed manually;
it cannot be suppressed automatically by the |\includeonly| directive
and thus should normally be avoided.
A method to include some content in the main file
by means of conditional processing is described in \secref{sec:conditional}.

%%%%%%%%%%%%%%%%%%%%%%%%%%%%%%%%%%%%%%%%
\paragraph{Page Numbering.}

When only a part of the document is compiled,
the appropriate numbering of pages
(as well as other status parameters)
is determined from the |.aux| files.
The latter contain information from previous passes.
However this information needs to propagate through
all intermediate child documents.
Therefore the page numbering in child documents may well
be inconsistent until the complete document is compiled at least once.

A useful (if unconventional) way to always ensure a consistent
page numbering is to restart the numbering in each child document
and denote the pages by `\textit{child}|.|\textit{page}'
where \textit{child} represents the chapter/section number of the child file.
This can be achieved by the command
|\numberwithin{page}{|\textit{child}|}|
of the \textsf{amsmath} package
where \textit{child} can be |chapter| or |section|
depending on the chosen structuring.
Alternatively, one can modify the macro |\thepage| appropriately
and reset the counter |page| at the start of each child file.

%%%%%%%%%%%%%%%%%%%%%%%%%%%%%%%%%%%%%%%%%%%%%%%%%%%%%%%%%%%%%%%%%%%%%%%%%%%%%%%%
\subsection{Conditional Processing}
\label{sec:conditional}

The package provides a mechanism to compile different versions
of a document. To customise the versions further some conditional processing
can come in handy to distinguish which version is being compiled.
The package provides two macros to describe the compilation context:

%%%%%%%%%%%%%%%%%%%%%%%%%%%%%%%%%%%%%%%%
\DescribeMacro{\ifchilddoc}
The conditional |\ifchilddoc| distinguishes between the compilation of
child documents and the main document:
%
\begin{center}
|\ifchilddoc |\textit{child-code}| |[|\||else |\textit{main-code}]| \||fi|
\end{center}

%%%%%%%%%%%%%%%%%%%%%%%%%%%%%%%%%%%%%%%%
\DescribeMacro{\childdocname}
\DescribeMacro{\childdocjob}
The macro |\childdocname| contains the filename (without extension)
of the main or child file being processed.
Note that |\childdocjob| will always contain the name of the main file.

%%%%%%%%%%%%%%%%%%%%%%%%%%%%%%%%%%%%%%%%
\paragraph{Title Page.}

Conditional processing can be used to include a title or banner page
in the main document when proper precautions are taken.
Importantly, the code in the main file should ensure that the page counter
(as well as other status parameters which are stored in the |.aux| files)
takes the same value after the conditional processing.
Otherwise the page numbers may take divergent values
depending on which part is compiled.

For example, a title page could be declared by:
%
\begin{center}
\begin{tabular}{l}
|\ifchilddoc\||else|\\
|\addtocounter{page}{-1}|\\
\textit{code for title page}\\
|\newpage|\\
|\||fi|
\end{tabular}
\end{center}
%
A banner page for the child documents can be generated by:
%
\begin{center}
\begin{tabular}{l}
|\ifchilddoc|\\
|\addtocounter{page}{-1}|\\
\textit{code for banner page}\\
|\newpage|\\
|\||fi|
\end{tabular}
\end{center}
%
Here one could write a message such as:
\begin{center}
|This is the part \childdocname{} of \childdocjob{}.|
\end{center}

%%%%%%%%%%%%%%%%%%%%%%%%%%%%%%%%%%%%%%%%%%%%%%%%%%%%%%%%%%%%%%%%%%%%%%%%%%%%%%%%
\subsection{Flags}
\label{sec:flags}

The package makes it easy to generate different versions
of the main or child documents.
To this end compilation flags can be defined
and assigned different default values.
They will be particularly useful in conjunction
with the forwarding mechanism described in \secref{sec:forward}.

For example, it may be useful to have a flag |\version|
which can be set to |draft| or |final|.
The document source will contain some conditional code
depending on the value of |\version|.
Suppose further, the flag should default to |final| for the main file
and to |draft| for child files
which is a natural assignment for editing the document.
This is achieved by placing the following code
in the preamble of the main document
(below the |\childdocmain| directive):
%
\begin{center}
\begin{tabular}{l}
|\ifchilddoc|\\
|\providecommand{\version}{draft}|\\
|\||else|\\
|\providecommand{\version}{final}|\\
|\||fi|
\end{tabular}
\end{center}
%
The definition by |\providecommand| makes sure
that previous definitions are not overwritten.
Further statements |\providecommand{\version}{...}|
can thus be added before the above code to override it.

For the main file, one might add a line
(between |\childdocmain| and the above block)
%
\begin{center}
|%\ifchilddoc\||else\providecommand{\version}{draft}\||fi|
\end{center}
%
which can be uncommented to produce a draft version.
Likewise one can add a line to the very top of a child file
(above the |\childdocof{|\textit{main}|}| directive)
%
\begin{center}
|%\providecommand{\version}{final}|
\end{center}
%
which can be uncommented to produce the final version of this child document.

%%%%%%%%%%%%%%%%%%%%%%%%%%%%%%%%%%%%%%%%%%%%%%%%%%%%%%%%%%%%%%%%%%%%%%%%%%%%%%%%
\subsection{Forwarding}
\label{sec:forward}

Different versions of the main or child documents
using compilation flags as described in \secref{sec:flags}
can be (permanently) stored in different files
for convenient compilation, viewing and distribution.
To this end, the package defines a command
to pass on compilation to a different file:

%%%%%%%%%%%%%%%%%%%%%%%%%%%%%%%%%%%%%%%%
\DescribeMacro{\childdocforward}
The command |\childdocforward| redirects processing to
another source file:
%
\begin{center}
\begin{tabular}{l}
|\input{childdoc.def}|\\
|\childdocforward[|\textit{main}|]{|\textit{dest}|}|\\
\end{tabular}
\end{center}
%
The argument \textit{dest} is the destination file
(without extension).
It should be the main file or one of the child files.
Note that further \textsf{childdoc} directives
such as |\childdocof| and |\childdocforward|
in the indicated file will be processed in this form.
The optional argument \textit{main}
passes on directly to the main file \textit{main}
while pretending to compile the child \textit{dest}.
This form behaves as if \textit{dest}
issues |\childdocof{|\textit{main}|}| right away,
and no further \textsf{childdoc} directives will be processed.

%%%%%%%%%%%%%%%%%%%%%%%%%%%%%%%%%%%%%%%%
\DescribeMacro{\...prefix}
In the alternative form |\childdocforwardprefix|,
%
\begin{center}
\begin{tabular}{l}
|\input{childdoc.def}|\\
|\childdocforwardprefix[|\textit{main}|]{|\textit{prefix}|}{|\textit{dest}|}|
\end{tabular}
\end{center}
%
the destination file is determined by a pattern
depending on the current file:
To make this work, the current file must be called
`{\textit{prefix}\hspace{0.2em}\textit{suffix}}'
with \textit{prefix} matching precisely the argument.
Processing is then passed on to the file
`{\textit{dest}\hspace{0.2em}\textit{suffix}}'.
Surely, the same effect is achieved by
directly specifying the
argument `{\textit{dest}\hspace{0.2em}\textit{suffix}}'
in the first form.
However, that requires to set up a different file
for each child. With the alternative form of the command
all these files can have exactly the same content
which simplifies setting them up and maintaining them.

For example, the following file |draft.tex|
with a compilation flag |\version| as described in \secref{sec:flags}
compiles the main document as a draft:
%
\begin{center}
\begin{tabular}{l}
|\def\version{draft}|\\
|\input{childdoc.def}|\\
|\childdocforward{|\textit{main}|}|
\end{tabular}
\end{center}
%
Likewise, the following files |final|\textit{nn}|.tex|
compile the final version of the child document
|child|\textit{nn}|.tex|:
%
\begin{center}
\begin{tabular}{l}
|\def\version{final}|\\
|\input{childdoc.def}|\\
|\childdocforwardprefix{final}{child}|
\end{tabular}
\end{center}
%

Note that when several versions of a main file and/or of each child file
are to be generated, it may be convenient to set up a |Makefile| or
shell script to automatise the process.

%%%%%%%%%%%%%%%%%%%%%%%%%%%%%%%%%%%%%%%%%%%%%%%%%%%%%%%%%%%%%%%%%%%%%%%%%%%%%%%%
\subsection{Command Line Processing}
\label{sec:commandline}

The effect of redirection files can also be achieved by invoking
the \LaTeX{} compiler with a more elaborate command line.
Most conveniently this should be done as part
of a shell script or a |Makefile|.

When using \textsf{childdoc} in the main file, the following
command lines effectively perform a redirection
(note that depending on the shell being used,
backslashes may have to be doubled: `|\|' $\to$ `|\\|'):
%
\begin{center}
|... -jobname "|\textit{target}|" |\\|"|[\textit{flags}]%
|\input{childdoc.def}\childdocforward[|\textit{main}|]{|\textit{dest}|}"|
\end{center}
%
Here \textit{target} is the name of the output file,
\textit{main} is the name of the main file
and \textit{dest} is the name of the main or child file to be processed
(all filenames without extensions).
The optional argument \textit{main} can be omitted
if \textit{main} matches \textit{dest}.
Optionally, compilation \textit{flags} can be defined via |\def| commands.
This command line makes the \TeX{} engine believe
it is compiling the file \textit{target}
whose content is specified as the latter parameter.
The provided code then forwards the processing to
\textit{main} or \textit{dest} as described in \secref{sec:forward}.

%%%%%%%%%%%%%%%%%%%%%%%%%%%%%%%%%%%%%%%%%%%%%%%%%%%%%%%%%%%%%%%%%%%%%%%%%%%%%%%%
\subsection{Include by Input}
\label{sec:input}

Including child documents by |\include| has some restrictions by design.
Most notably, the content of a child document always occupies
its own set of pages; pages cannot be shared between child documents.
Usually, this behaviour makes perfect sense
because each child document contain an essential part of the document.
However, in some situations it may be desirable to compose
a document from a collection of parts
without having mandatory page breaks between then.
For this case, the package
provides a mechanism to include parts
by |\input| which can also be processed individually.
However, by construction this mechanism
requires manual handling of the content to be output.

%%%%%%%%%%%%%%%%%%%%%%%%%%%%%%%%%%%%%%%%
\DescribeMacro{\ifchilddocmanual}
The main file should be prepared as usual, see \secref{sec:include}.
However, the document body must make a distinction
between processing of an individual part and of the main document, e.g.:
%
\begin{center}
\begin{tabular}{l}
|\ifchilddocmanual|\\
|\input{\childdocname}|\\
|\||else|\\
\textit{document body with }|\input{|\textit{part}|}|\\
|\||fi|
\end{tabular}
\end{center}
%
The conditional |\ifchilddocmanual| is true whenever
a part to be included by |\input| is being compiled,
and the name of the part is stored in |\childdocname|.

%%%%%%%%%%%%%%%%%%%%%%%%%%%%%%%%%%%%%%%%
\DescribeMacro{\childdocby}
Each part to be included by |\input| should start with:
%
\begin{center}
\begin{tabular}{l}
|\input{childdoc.def}|\\
|\childdocby{|\textit{main}|}|\\
\end{tabular}
\end{center}
%
The directive |\childdocby| is similar to |\childdocof|
described in \secref{sec:include},
but the subsequent selection of content must be done manually.
To that end, both |\ifchilddoc| and |\ifchilddocmanual|
will be true upon processing of a part,
and the name of the part is stored in |\childdocname|.
Note that |\jobname| will be set to the filename of the current part
so that each part receives an individual |.aux| file
that does not interfere with the |.aux| file(s) of the main document.
This behaviour can be altered by the alternative form
|\childdocby[*]{|\textit{main}|}| (with a non-empty optional argument)
which uses the |.aux| file of the main document
by setting |\jobname| to \textit{main}.

%%%%%%%%%%%%%%%%%%%%%%%%%%%%%%%%%%%%%%%%%%%%%%%%%%%%%%%%%%%%%%%%%%%%%%%%%%%%%%%%
\subsection{Driver Development}
\label{sec:driver}

The \textsf{childdoc} mechanism can also be use for the development
of definition files such as \LaTeX{} styles or classes.
This case differs from the above setup with multiple parts
included by |\include| in that no |\includeonly| should be invoked.
This can be achieved by starting the include file
(before |\ProvidesPackage|) with:
%
\begin{center}
\begin{tabular}{l}
|\input{childdoc.def}|\\
|\childdocforward{|\textit{main}|}|\\
\end{tabular}
\end{center}
%
or alternatively with:
%
\begin{center}
\begin{tabular}{l}
|\input{childdoc.def}|\\
|\childdocby{|\textit{main}|}|\\
\end{tabular}
\end{center}
%
Both forms have slightly different effects as described above.
The main file is prepared as usual, see \secref{sec:include}.

%%%%%%%%%%%%%%%%%%%%%%%%%%%%%%%%%%%%%%%%%%%%%%%%%%%%%%%%%%%%%%%%%%%%%%%%%%%%%%%%
\subsection{Legacy Detection}
\label{sec:detection}

The directive |\childdocmain| in the main file can detect
whether the complete document or merely a child is to be compiled
even without using the directive |\childdocof|.
This method is deprecated because it is less robust
and there is no compelling reason to use it;
it is merely provided for backward compatibility
and it may be removed in future versions.

If the detection mechanism is to be used,
it is mandatory to correctly specify
the filename of the main file as the argument of |\childdocmain|:
%
\begin{center}
\begin{tabular}{l}
|\input{childdoc.def}|\\
|\childdocmain{|\textit{main}|}|\\
\end{tabular}
\end{center}
%
If |\jobname| does not match the argument \textit{main} of |\childdocmain|,
it is assumed that |\jobname| points to the child file to be compiled.
When using |\childdocmain| with the main file specified as argument,
it suffices to start a child file
with just |\input{|\textit{main}|}|
without loading of the package and using |\childdocof|.
If instead all processing is done
with the appropriate \textsf{childdoc} directives,
the argument of \textit{main} of |\childdocmain| can be empty.

An alternative version of the command line processing described
in \secref{sec:commandline} using the detection mechanism reads:
%
\begin{center}
|... -jobname "|\textit{target}|" "|[\textit{flags}]%
[|\def\jobname{|\textit{dest}|}|]|\input{|\textit{main}|}"|
\end{center}

%%%%%%%%%%%%%%%%%%%%%%%%%%%%%%%%%%%%%%%%%%%%%%%%%%%%%%%%%%%%%%%%%%%%%%%%%%%%%%%%
\subsection{Manual Code}
\label{sec:manual}

In case one cannot be certain whether the definitions file |childdoc.def|
is installed on the target \TeX{} distribution
and one prefers not to ship it,
it is conceivable to paste a few relevant commands into the sources.

To that end, drop all statements |\input{childdoc.def}|
and perform the replacements as outlined below.
Instead of |\childdocmain{|\textit{main}|}| add the following code
to the top of the main file:
%
\begin{center}
\begin{tabular}{l}
|\||ifdefined\childdocname\endinput\||fi\newif\ifchilddoc|\\
|\edef\childdocname{\scantokens\expandafter{\jobname\noexpand}}|\\
|\def\childdocmain{|\textit{main}|}\||ifx\childdocmain\childdocname\||else|\\
|\childdoctrue\includeonly{\childdocname}\let\jobname\childdocmain\||fi|\\
\end{tabular}
\end{center}
%
Instead of |\childdocof{|\textit{main}|}| just include the main file
at the top of each child file:
%
\begin{center}
|\input{|\textit{main}|}|
\end{center}
%
A simple redirection |\childdocforward{|\textit{dest}|}| is achieved by:
%
\begin{center}
|\def\jobname{|\textit{dest}|}\input{\jobname}|
\end{center}
%
The redirection with prefix
|\childdocforwardprefix[|\textit{prefix}|]{|\textit{dest}|}|
is accomplished by:
%
\begin{center}
\begin{tabular}{l}
|{\edef\jobname{\scantokens\expandafter{\jobname\noexpand}}|\\
|\def\redirectjob |\textit{prefix}|#1~~~{\gdef\jobname{|\textit{dest}|#1}}|\\
|\expandafter\redirectjob\jobname~~~}\input{\jobname}|
\end{tabular}
\end{center}

In an alternative approach,
child documents can be compiled by a specific command line
without additional code or specific definitions:
%
\begin{center}
|... -jobname "|\textit{target}|" "|[\textit{flags}]%
|\includeonly{|\textit{dest}|}\input{|\textit{main}|}"|
\end{center}
%

%%%%%%%%%%%%%%%%%%%%%%%%%%%%%%%%%%%%%%%%%%%%%%%%%%%%%%%%%%%%%%%%%%%%%%%%%%%%%%%%
%%%%%%%%%%%%%%%%%%%%%%%%%%%%%%%%%%%%%%%%%%%%%%%%%%%%%%%%%%%%%%%%%%%%%%%%%%%%%%%%
\section{Information}

%%%%%%%%%%%%%%%%%%%%%%%%%%%%%%%%%%%%%%%%%%%%%%%%%%%%%%%%%%%%%%%%%%%%%%%%%%%%%%%%
\subsection{Copyright}

Copyright \copyright{} 2017--2018 Niklas Beisert

This work may be distributed and/or modified under the
conditions of the \LaTeX{} Project Public License, either version 1.3
of this license or (at your option) any later version.
The latest version of this license is in
  \url{http://www.latex-project.org/lppl.txt}
and version 1.3 or later is part of all distributions of \LaTeX{}
version 2005/12/01 or later.

This work has the LPPL maintenance status `maintained'.

The Current Maintainer of this work is Niklas Beisert.

This work consists of the files |README.txt|, |childdoc.ins| and |childdoc.dtx|
as well as the derived files |childdoc.def|, |cdocsamp.tex|
with |cdocsch1.tex|, |cdocsch2.tex|, |cdocspt3.tex|, |cdocspt4.tex|,
|cdocsdrf.tex|, |cdocsfn1.tex|, |cdocsfn2.tex|
as well as |childdoc.pdf|.

%%%%%%%%%%%%%%%%%%%%%%%%%%%%%%%%%%%%%%%%%%%%%%%%%%%%%%%%%%%%%%%%%%%%%%%%%%%%%%%%
\subsection{Files and Installation}

The package consists of the files:
%
\begin{center}
\begin{tabular}{ll}
    |README.txt|   & readme file \\
    |childdoc.ins| & installation file \\
    |childdoc.dtx| & source file \\
    |childdoc.def| & definition file \\
    |cdocsamp.tex| & sample main file \\
    |cdocsch1.tex| & sample include file \\
    |cdocsch2.tex| & sample include file \\
    |cdocspt3.tex| & sample part file \\
    |cdocspt4.tex| & sample part file \\
    |cdocsdrf.tex| & sample redirection file \\
    |cdocsfn1.tex| & sample redirection file \\
    |cdocsfn2.tex| & sample redirection file \\
    |childdoc.pdf| & manual
\end{tabular}
\end{center}
%
The distribution consists of the files
|README.txt|, |childdoc.ins| and |childdoc.dtx|.
%
\begin{itemize}
\item
Run (pdf)\LaTeX{} on |childdoc.dtx|
to compile the manual |childdoc.pdf| (this file).
\item
Run \LaTeX{} on |childdoc.ins| to create the definitions file |childdoc.def|
and the sample |cdocsamp.tex| with include files
|cdocsch1.tex|, |cdocsch2.tex|, |cdocspt3.tex|, |cdocspt4.tex|,
|cdocsdrf.tex|, |cdocsfn1.tex|, |cdocsfn2.tex|.
Then copy the file |childdoc.def| to an appropriate directory of your \LaTeX{}
distribution, e.g.\ \textit{texmf-root}|/tex/latex/childdoc|.
\end{itemize}

%%%%%%%%%%%%%%%%%%%%%%%%%%%%%%%%%%%%%%%%%%%%%%%%%%%%%%%%%%%%%%%%%%%%%%%%%%%%%%%%
\subsection{Related CTAN Packages}

There are several other packages which offer a similar functionality:
%
\begin{itemize}
\item
The packages
\href{http://ctan.org/pkg/docmute}{\textsf{docmute}},
\href{http://ctan.org/pkg/includex}{\textsf{includex}} and
\href{http://ctan.org/pkg/standalone}{\textsf{standalone}}
provide commands to include only the document body of
a child file thus allowing both files to be compiled individually.
\item
The packages \href{http://ctan.org/pkg/subdocs}{\textsf{subdocs}}
and \href{http://ctan.org/pkg/subfiles}{\textsf{subfiles}}
provide structures in which the main and child documents can be
encapsulated and allowing them to be compiled individually.
The inclusion mechanism is different from the conventional |\include|.
\item
The package \href{http://ctan.org/pkg/combine}{\textsf{combine}}
is an elaborate solution to combine several documents into one.
\end{itemize}
%
See also the CTAN topic \href{http://ctan.org/topic/subdocs}{\textsf{subdocs}}
for further related packages.
The present package differs from the above solutions in that
a document structure constructed with the conventional |\include| mechanism
just needs two extra commands at the top of every file
such that all constituent files can be compiled individually.

%%%%%%%%%%%%%%%%%%%%%%%%%%%%%%%%%%%%%%%%%%%%%%%%%%%%%%%%%%%%%%%%%%%%%%%%%%%%%%%%
%\subsection{Feature Suggestions}
%
%The following is a list of features which may be useful for future
%versions of this package:
%%
%\begin{itemize}
%\item
%\ldots
%\end{itemize}

%%%%%%%%%%%%%%%%%%%%%%%%%%%%%%%%%%%%%%%%%%%%%%%%%%%%%%%%%%%%%%%%%%%%%%%%%%%%%%%%
\subsection{Revision History}

%%%%%%%%%%%%%%%%%%%%%%%%%%%%%%%%%%%%%%%%
\paragraph{v2.0:} 2018/12/30

\begin{itemize}
\item
immediate forward processing
\item
added |\childdocby| mechanism
\item
manual restructured
\end{itemize}

%%%%%%%%%%%%%%%%%%%%%%%%%%%%%%%%%%%%%%%%
\paragraph{v1.6:} 2018/01/17

\begin{itemize}
\item
application for development of include files
\item
corrections to manual
\end{itemize}

%%%%%%%%%%%%%%%%%%%%%%%%%%%%%%%%%%%%%%%%
\paragraph{v1.5:} 2017/05/21

\begin{itemize}
\item
more complete structuring introduced
\item
|\childdocof| introduced
\item
|\childdoc| renamed to |\childdocmain|
\item
|\childredirect| renamed to |\childdocforward| and |\childdocforwardprefix|
and functionality expanded
\end{itemize}

%%%%%%%%%%%%%%%%%%%%%%%%%%%%%%%%%%%%%%%%
\paragraph{v1.0:} 2017/04/27

\begin{itemize}
\item
manual and install package
\item
first version published on CTAN
\end{itemize}

%%%%%%%%%%%%%%%%%%%%%%%%%%%%%%%%%%%%%%%%
\paragraph{v0.6:} 2017/04/26

\begin{itemize}
\item
redirection mechanism added
\end{itemize}

%%%%%%%%%%%%%%%%%%%%%%%%%%%%%%%%%%%%%%%%
\paragraph{v0.5:} 2017/04/26

\begin{itemize}
\item
functionality in definition file
\end{itemize}


%%%%%%%%%%%%%%%%%%%%%%%%%%%%%%%%%%%%%%%%%%%%%%%%%%%%%%%%%%%%%%%%%%%%%%%%%%%%%%%%
%%%%%%%%%%%%%%%%%%%%%%%%%%%%%%%%%%%%%%%%%%%%%%%%%%%%%%%%%%%%%%%%%%%%%%%%%%%%%%%%
%%%%%%%%%%%%%%%%%%%%%%%%%%%%%%%%%%%%%%%%%%%%%%%%%%%%%%%%%%%%%%%%%%%%%%%%%%%%%%%%
\appendix

\settowidth\MacroIndent{\rmfamily\scriptsize 000\ }

 \DocInput{childdoc.dtx}

\end{document}
%</driver>
% \fi
%
% %%%%%%%%%%%%%%%%%%%%%%%%%%%%%%%%%%%%%%%%%%%%%%%%%%%%%%%%%%%%%%%%%%%%%%%%%%%%%%
% %%%%%%%%%%%%%%%%%%%%%%%%%%%%%%%%%%%%%%%%%%%%%%%%%%%%%%%%%%%%%%%%%%%%%%%%%%%%%%
% \section{Sample}
%\iffalse
%<*samplemain>
%\fi
%
% The following presents a sample document
% with two chapters, two parts, a title page,
% a compile flag as well as three forwarding files to set the flag.
% It consists of eight |.tex| files:
% \begin{center}
% \begin{tabular}{ll}
% |cdocsamp.tex|&main file\\
% |cdocsch1.tex|&include file for chapter 1\\
% |cdocsch2.tex|&include file for chapter 2\\
% |cdocspt3.tex|&include file for part 3\\
% |cdocspt4.tex|&include file for part 4\\
% |cdocsdrf.tex|&forwarding file for main file in draft mode\\
% |cdocsfi1.tex|&forwarding file for final version of chapter 1\\
% |cdocsfi2.tex|&forwarding file for final version of chapter 2\\
% \end{tabular}
% \end{center}
% Each of the eight files can be compiled directly by the \LaTeX{} compiler.
%
% %%%%%%%%%%%%%%%%%%%%%%%%%%%%%%%%%%%%%%
% \paragraph{Main File.}
%
% The main file is called |cdocsamp.tex|.
%
% Load the \textsf{childdoc} definitions and
% declare the filename for the main document:
%    \begin{macrocode}
\input{childdoc.def}
\childdocmain{}
%    \end{macrocode}

% Optional override for |\version| flag:
%    \begin{macrocode}
%%\ifchilddoc\else\providecommand{\version}{draft}\fi
%    \end{macrocode}

% Define the default values for the |\version| flag
% (|final| for the main file and |draft| for childs):
%    \begin{macrocode}
\ifchilddoc
\providecommand{\version}{draft}
\else
\providecommand{\version}{final}
\fi
%    \end{macrocode}

% Load the standard document class:
%    \begin{macrocode}
\documentclass[12pt]{article}
%    \end{macrocode}

% Start the document body:
%    \begin{macrocode}
\begin{document}
%    \end{macrocode}

% Declare a title page.
% Print title, part of document being processed and version flag:
%    \begin{macrocode}
\addtocounter{page}{-1}
\begin{center}
{\LARGE\bfseries{}childdoc example\par}
\vspace{1cm}
\ifchilddoc
\ifchilddocmanual part\else chapter\fi:
`\childdocname' of `\childdocjob'\par
\else
main document: `\childdocjob'\par
\fi
version: \version\par
\end{center}
\newpage
%    \end{macrocode}

% Manually include selected file,
% otherwise process as usual:
%    \begin{macrocode}
\ifchilddocmanual
\section*{part `\childdocname'}
\input{\childdocname}
\else
%    \end{macrocode}

% Include the two chapters:
%    \begin{macrocode}
\include{cdocsch1}
\include{cdocsch2}
%    \end{macrocode}

% Include the two parts unless only chapters should be displayed:
%    \begin{macrocode}
\ifchilddoc\else
\section{part three}
\input{cdocspt3}
\section{part four}
\input{cdocspt4}
\fi
%    \end{macrocode}

% Process as usual until here:
%    \begin{macrocode}
\fi
%    \end{macrocode}

% End of document body:
%    \begin{macrocode}
\end{document}
%    \end{macrocode}
%\iffalse
%</samplemain>
%\fi
%
% %%%%%%%%%%%%%%%%%%%%%%%%%%%%%%%%%%%%%%
% \paragraph{Chapter Include Files.}
%
% The include files are called |cdocsch1.tex| and |cdocsch2.tex|.
%
%\iffalse
%<*samplechap1|samplechap2>
%\fi

% Optional override for |\version| flag:
%    \begin{macrocode}
%%\providecommand{\version}{final}
%    \end{macrocode}

% Include the main document:
%    \begin{macrocode}
\input{childdoc.def}
\childdocof{cdocsamp}
%    \end{macrocode}

%\iffalse
%</samplechap1|samplechap2>
%\fi
%
%\iffalse
%<*samplechap1>
%\fi
% Some text for chapter 1:
%    \begin{macrocode}
\section{one}
some text in chapter one
%    \end{macrocode}

%\iffalse
%</samplechap1>
%\fi
% Some text for chapter 2:
%\iffalse
%<*samplechap2>
%\fi
%    \begin{macrocode}
\section{two}
more text in chapter two
%    \end{macrocode}

%\iffalse
%</samplechap2>
%\fi
%
% %%%%%%%%%%%%%%%%%%%%%%%%%%%%%%%%%%%%%%
% \paragraph{Part Include Files.}
%
% The include files are called |cdocspt3.tex| and |cdocspt4.tex|.
%
%\iffalse
%<*samplepart3|samplepart4>
%\fi

% Optional override for |\version| flag:
%    \begin{macrocode}
%%\providecommand{\version}{final}
%    \end{macrocode}

% Include the main document:
%    \begin{macrocode}
\input{childdoc.def}
\childdocby{cdocsamp}
%    \end{macrocode}

%\iffalse
%</samplepart3|samplepart4>
%\fi
%
%\iffalse
%<*samplepart3>
%\fi
% Some text for part 3:
%    \begin{macrocode}
some text in part three
%    \end{macrocode}

%\iffalse
%</samplepart3>
%\fi
% Some text for part 4:
%\iffalse
%<*samplepart4>
%\fi
%    \begin{macrocode}
more text in part four
%    \end{macrocode}

%\iffalse
%</samplepart4>
%\fi
%
% %%%%%%%%%%%%%%%%%%%%%%%%%%%%%%%%%%%%%%
% \paragraph{Forwarding for a Complete Draft.}
%
% The following forwarding file |cdocsdrf.tex|
% compiles the main document in draft mode:
%\iffalse
%<*sampledraft>
%\fi
%    \begin{macrocode}
\def\version{draft}
\input{childdoc.def}
\childdocforward{cdocsamp}
%    \end{macrocode}

%\iffalse
%</sampledraft>
%\fi
%
% %%%%%%%%%%%%%%%%%%%%%%%%%%%%%%%%%%%%%%
% \paragraph{Forwarding for Final Version of the Chapters.}
%
% The following forwarding files |cdocsfn1.tex| and |cdocsfn2.tex|
% (with identical content)
% compile the final versions of the child documents
% |cdocsch1.tex| and |cdocsch2.tex|, respectively:
%\iffalse
%<*samplefinal>
%\fi
%    \begin{macrocode}
\def\version{final}
\input{childdoc.def}
\childdocforwardprefix[cdocsamp]{cdocsfn}{cdocsch}
%    \end{macrocode}

%\iffalse
%</samplefinal>
%\fi
%
% %%%%%%%%%%%%%%%%%%%%%%%%%%%%%%%%%%%%%%
% \paragraph{Command Line Processing.}
%
% The following three command lines generate the output files
% |cdocscld|, |cdocscl1| and |cdocscl2|
% which should be identical to
% |cdocsdrf|, |cdocsch1| and |cdocsfn2|, respectively:
% \begin{center}
% \begin{tabular}{l}
% |latex -jobname cdocscld \|\\
% |  "\def\version{draft}\input{childdoc.def}\childdocforward{cdocsamp}"|\\
% |latex -jobname cdocscl1 \|\\
% |  "\input{childdoc.def}\childdocforward[cdocsamp]{cdocsch1}"|\\
% |latex -jobname cdocscl2 \|\\
% |  "\def\version{final}\input{childdoc.def}\childdocforward{cdocsch2}"|
% \end{tabular}
% \end{center}
% Note that the trailing backslash on each first line
% merely continues the input to the second line
% (for convenient cut ant paste).
% Furthermore, the command |latex| can be replaced by any
% of its alternative versions such as |pdflatex|.
%
% %%%%%%%%%%%%%%%%%%%%%%%%%%%%%%%%%%%%%%%%%%%%%%%%%%%%%%%%%%%%%%%%%%%%%%%%%%%%%%
% %%%%%%%%%%%%%%%%%%%%%%%%%%%%%%%%%%%%%%%%%%%%%%%%%%%%%%%%%%%%%%%%%%%%%%%%%%%%%%
% \section{Implementation}
%\iffalse
%<*package>
%\fi
%
% This section describes the definitions file |childdoc.def|.

% The definitions cannot be loaded using |\usepackage| or |\RequirePackage|
% which has a mechanism to prevent loading a style file more than once.
% When loading the definitions by means of |\input|
% multiple instances have to be prevented manually:
%\iffalse
%This code needs to be before the `\ProvidesFile' directive
%which is defined at the beginning of this file.
%Therefore it is also placed there and commented out here.
%</package>
%<*discard>
%\fi
%    \begin{macrocode}
\ifdefined\childdocmain\endinput\fi
%    \end{macrocode}
%\iffalse
%</discard>
%<*package>
%\fi
%
% \macro{\ifchilddoc}
% \macro{\ifchilddocmanual}
% The conditional |\ifchilddoc| tells whether a
% child (true) or main (false) document is being compiled.
% The conditional |\ifchilddocmanual| tells whether
% the |\includeonly| mechanism is used (false) or
% the selection of child files must be performed manually (true).
% The definitions initialise to false:
%    \begin{macrocode}
\newif\ifchilddoc
\newif\ifchilddocmanual
%    \end{macrocode}

% \macro{\childdocname}
% \macro{\childdocjob}
% The macro |\childdocname| stores the name of the main document
% to be compiled. The macro |\childdocjob| stores the name of
% the document on which the \LaTeX{} compiler was originally invoked.
% The content of |\jobname| cannot be compared
% to filenames specified in the source due to different catcodes.
% The following code rescans |\jobname|, stores the result
% in |\childdocname| and saves a copy in |\childdocjob|:
%    \begin{macrocode}
\edef\childdocname{\scantokens\expandafter{\jobname\noexpand}}
\let\childdocjob\childdocname
%    \end{macrocode}

% \macro{\childdocdisable}
% The macro |\childdocdisable| prevents the main file
% from being processed more than once.
% At this stage, the main document command |\childdocmain|
% is assumed to be called once again where it should do nothing.
% Any subsequent call to it should prevent
% a secondary processing of the main document
% It overwrites the forwarding commands
% |\childdocof| and |\childdocforward|
% with empty macros to prevent further inclusions of the main document:
%    \begin{macrocode}
\newcommand{\childdocdisable}
{
  \renewcommand{\childdocmain}[1]{\renewcommand{\childdocmain}[1]{\endinput}}
  \renewcommand{\childdocof}[1]{}
  \renewcommand{\childdocby}[2][]{}
  \renewcommand{\childdocforward}[2][]{}
  \renewcommand{\childdocdisable}{}
}
%    \end{macrocode}

% \macro{\childdocmain}
% The macro |\childdocmain| is to be called at the top of the main file
% with nothing or the main filename (without extension) as argument.
% First, it breaks loops.
% If the argument is not empty and does not match |\childdocname|
% (which is set by the first inclusion of |childdoc.def|),
% |\ifchilddoc| is set to true, |\includeonly| is applied to the child file
% and |\jobname| is set to the main file
% (for proper handling of |.aux| files):
%    \begin{macrocode}
\newcommand{\childdocmain}[1]
{
  \childdocdisable\childdocmain{}
  \if?#1?\else
    \begingroup
      \def\childdoctmp{#1}
      \ifx\childdoctmp\childdocname
        \def\childdoctmp{}
      \else
        \def\childdoctmp
        {
          \childdoctrue
          \includeonly{\childdocname}
          \def\childdocjob{#1}
          \def\jobname{#1}
        }
      \fi
      \expandafter
    \endgroup
    \childdoctmp
  \fi
}
%    \end{macrocode}

% \macro{\childdocof}
% The command |\childdocof| redirects
% compilation to the main file |#1|.
%    \begin{macrocode}
\newcommand{\childdocof}[1]
{
  \childdocdisable
  \childdoctrue
  \includeonly{\childdocname}
  \def\jobname{#1}
  \def\childdocjob{#1}
  \input{#1}
}
%    \end{macrocode}

% \macro{\childdocby}
% The command |\childdocby| ....
%    \begin{macrocode}
\newcommand{\childdocby}[2][]
{
  \childdocdisable
  \childdoctrue
  \childdocmanualtrue
  \if?#1?\else
    \def\jobname{#2}
  \fi
  \def\childdocjob{#2}
  \input{#2}
  \endinput
}
%    \end{macrocode}

% \macro{\childdocforward}
% The command |\childdocforward| redirects
% compilation to the main file or
% (if the optional argument is given) a child file.
% Parameters are set as if the main file
% or a child file starting with |\childdocof| was compiled.
% Then compilation is handed over to the main file:
%    \begin{macrocode}
\newcommand{\childdocforward}[2][]
{
  \begingroup
    \if?#1?
      \def\childdoctmp
      {
        \def\childdocname{#2}
        \def\childdocjob{#2}
        \def\jobname{#2}
        \input{#2}
        \endinput
      }
    \else
      \def\childdoctmp
      {
        \childdocdisable
        \def\childdocname{#2}
        \childdoctrue
        \includeonly{#2}
        \def\childdocjob{#1}
        \def\jobname{#1}
        \input{#1}
        \endinput
      }
    \fi
    \expandafter
  \endgroup
  \childdoctmp
}
%    \end{macrocode}

% \macro{\childdocforwardprefix}
% The command |\childdocforwardprefix| redirects
% compilation to the main or a child file by means of a pattern.
% The prefix |#1| in the current filename is replaced by |#2|
% and the suffix of the current filename is kept
% (it is assumed that the filename does not contain the substring `|~~~|'
% which is used as a delimiter).
% Compilation is handed over to the new file by |\childdocforward|:
%    \begin{macrocode}
\newcommand{\childdocforwardprefix}[3][]
{
  \begingroup
    \def\childdocextract #2##1~~~{\def\childdoctmp{\childdocforward[#1]{#3##1}}}
    \expandafter\childdocextract\childdocname~~~
    \expandafter
  \endgroup
  \childdoctmp
}
%    \end{macrocode}

% \macro{\childdoc}
% The deprecated macro |\childdoc| is a legacy version of |\childdocmain|:
%    \begin{macrocode}
\newcommand{\childdoc}{\childdocmain}
%    \end{macrocode}

% \macro{\childdocredirect}
% The deprecated macro |\childdocredirect| is a legacy version
% of |\childdocforward| and |\childdocforwardprefix|:
%    \begin{macrocode}
\newcommand{\childdocredirect}[2][]
{
  \begingroup
    \if?#1?
      \def\childdoctmp{\childdocforward{#2}}
    \else
      \def\childdoctmp{\childdocforwardprefix{#1}{#2}}
    \fi
    \expandafter
  \endgroup
  \childdoctmp
}
%    \end{macrocode}

%\iffalse
%</package>
%\fi
%
\endinput

\childdocforward{cdocsamp}
%    \end{macrocode}

%\iffalse
%</sampledraft>
%\fi
%
% %%%%%%%%%%%%%%%%%%%%%%%%%%%%%%%%%%%%%%
% \paragraph{Forwarding for Final Version of the Chapters.}
%
% The following forwarding files |cdocsfn1.tex| and |cdocsfn2.tex|
% (with identical content)
% compile the final versions of the child documents
% |cdocsch1.tex| and |cdocsch2.tex|, respectively:
%\iffalse
%<*samplefinal>
%\fi
%    \begin{macrocode}
\def\version{final}
% \iffalse
%
% childdoc.dtx Copyright (C) 2017-2018 Niklas Beisert
%
% This work may be distributed and/or modified under the
% conditions of the LaTeX Project Public License, either version 1.3
% of this license or (at your option) any later version.
% The latest version of this license is in
%   http://www.latex-project.org/lppl.txt
% and version 1.3 or later is part of all distributions of LaTeX
% version 2005/12/01 or later.
%
% This work has the LPPL maintenance status `maintained'.
%
% The Current Maintainer of this work is Niklas Beisert.
%
% This work consists of the files childdoc.dtx and childdoc.ins
% and the derived files childdoc.def and cdocsamp.tex with
% cdocsch1.tex, cdocsch2.tex, cdocsdrf.tex, cdocsfn1.tex, cdocsfn2.tex.
%
%<package>\ifdefined\childdocmain\endinput\fi
%<package>\ProvidesFile{childdoc.def}[2018/12/30 v2.0 child document driver]
%<samplemain>\ProvidesFile{cdocsamp.tex}[2018/12/30 v2.0 sample for childdoc]
%<*driver>
%\ProvidesFile{childdoc.drv}[2018/12/30 v2.0 childdoc reference manual file]
\PassOptionsToClass{10pt,a4paper}{article}
\documentclass{ltxdoc}

\usepackage[margin=35mm]{geometry}
\usepackage{hyperref}
\usepackage{hyperxmp}
\usepackage[usenames]{color}

\hypersetup{colorlinks=true}
\hypersetup{pdfstartview=FitH}
\hypersetup{pdfpagemode=UseNone}
\hypersetup{pdfsource={}}
\hypersetup{pdflang={en-UK}}
\hypersetup{pdfcopyright={Copyright 2017-2018 Niklas Beisert.
  This work may be distributed and/or modified under the
  conditions of the LaTeX Project Public License, either version 1.3
  of this license or (at your option) any later version.}}
\hypersetup{pdflicenseurl={http://www.latex-project.org/lppl.txt}}
\hypersetup{pdfcontactaddress={ETH Zurich, ITP, HIT K,
  Wolfgang-Pauli-Strasse 27}}
\hypersetup{pdfcontactpostcode={8093}}
\hypersetup{pdfcontactcity={Zurich}}
\hypersetup{pdfcontactcountry={Switzerland}}
\hypersetup{pdfcontactemail={nbeisert@itp.phys.ethz.ch}}
\hypersetup{pdfcontacturl={http://people.phys.ethz.ch/\xmptilde nbeisert/}}

\newcommand{\secref}[1]{\hyperref[#1]{section \ref*{#1}}}

\parskip1ex
\parindent0pt
\let\olditemize\itemize
\def\itemize{\olditemize\parskip0pt}

\begin{document}

\title{The \textsf{childdoc} Package}
\hypersetup{pdftitle={The childdoc Package}}
\author{Niklas Beisert\\[2ex]
  Institut f\"ur Theoretische Physik\\
  Eidgen\"ossische Technische Hochschule Z\"urich\\
  Wolfgang-Pauli-Strasse 27, 8093 Z\"urich, Switzerland\\[1ex]
  \href{mailto:nbeisert@itp.phys.ethz.ch}
  {\texttt{nbeisert@itp.phys.ethz.ch}}}
\hypersetup{pdfauthor={Niklas Beisert}}
\hypersetup{pdfsubject={Manual for the LaTeX2e Package childdoc}}
\date{30 December 2018, \textsf{v2.0}}
\maketitle

\begin{abstract}\noindent
\textsf{childdoc} is a \LaTeXe{} package
that enables the direct compilation
of document sections included by |\include|
to individual files.
\end{abstract}

\begingroup
\parskip0ex
\tableofcontents
\endgroup

%%%%%%%%%%%%%%%%%%%%%%%%%%%%%%%%%%%%%%%%%%%%%%%%%%%%%%%%%%%%%%%%%%%%%%%%%%%%%%%%
%%%%%%%%%%%%%%%%%%%%%%%%%%%%%%%%%%%%%%%%%%%%%%%%%%%%%%%%%%%%%%%%%%%%%%%%%%%%%%%%
\section{Introduction}

\LaTeX{} provides a mechanism to structure a large document (such as a book)
into a main file and several child files (containing the chapters)
using the |\include| command.
This mechanism is beneficial for documents
which span hundreds of pages in order to
make the source file(s) more manageable.
Moreover, compilation can be restricted to
selected child files by means of the |\includeonly| command.
The latter feature can be used to reduce the compilation time while editing
(this was significantly more useful in the earlier days of \LaTeX{})
or to generate a smaller document which is easier to navigate.
Another application of |\includeonly| is to generate
documents consisting of selected parts of the complete document.

However, there are a few drawbacks of the plain |\include| mechanism:
\begin{itemize}
\item
The child files cannot be compiled on their own,
they can only be compiled via the main file.
A naive editing environment
(such as a text editor with an option
to have the current file processed by \LaTeX)
may require one to switch to the main file before compiling;
attempting to compile the child file produces errors.
\item
The main file must be modified (each time)
to adjust the |\includeonly| command
to the present needs. This easily leaves the main file in a messy state.
\item
The generated document will always carry the filename
of the main document. This is inconvenient if
several child files are to be compiled and
to be kept for distribution.
\end{itemize}

The present package provides a simple interface
to make child files individually compilable by \LaTeX{}.
Compiling a child file then has the same effect as compiling
the main file with an |\includeonly| command
to select the appropriate child.
Moreover the generated document will carry the name of the child
rather than the main file.
This resolves all three above issues.

This feature is meant to make the editing of books,
thesis documents and lecture notes somewhat more convenient.
However, the package can also be used efficiently for
composing a series of documents (such as exercise sheets)
which are typically distributed individually.
It then assists the author in generating the individual documents
(potentially in different versions)
as well as a document containing the collected series.
Another application is in developing style files
or other kinds of included material
where compilation of the style file could redirect
to a sample or test file.

%%%%%%%%%%%%%%%%%%%%%%%%%%%%%%%%%%%%%%%%%%%%%%%%%%%%%%%%%%%%%%%%%%%%%%%%%%%%%%%%
%%%%%%%%%%%%%%%%%%%%%%%%%%%%%%%%%%%%%%%%%%%%%%%%%%%%%%%%%%%%%%%%%%%%%%%%%%%%%%%%
\section{Usage}

First of all, the package \textsf{childdoc} is \emph{not} a standard
\LaTeXe{} |.sty| style file! Therefore it needs to be invoked in
a non-standard way.

%%%%%%%%%%%%%%%%%%%%%%%%%%%%%%%%%%%%%%%%%%%%%%%%%%%%%%%%%%%%%%%%%%%%%%%%%%%%%%%%
\subsection{Included Files}
\label{sec:include}

%%%%%%%%%%%%%%%%%%%%%%%%%%%%%%%%%%%%%%%%
\DescribeMacro{\childdocmain}
To use the package, add the commands
\begin{center}
\begin{tabular}{l}
|\input{childdoc.def}|\\
|\childdocmain{}|\\
\end{tabular}
\end{center}
at the very top of the main \LaTeX{} file,
in particular \emph{before} the |\documentclass| statement!
The argument of |\childdocmain| should be left empty
(but it must be present).

%%%%%%%%%%%%%%%%%%%%%%%%%%%%%%%%%%%%%%%%
\DescribeMacro{\childdocof}
Furthermore, add the commands
\begin{center}
\begin{tabular}{l}
|\input{childdoc.def}|\\
|\childdocof{|\textit{main}|}|\\
\end{tabular}
\end{center}
at the top of every child file \textit{child}
which is included by |\include{|\textit{child}|}|
from within the main file
(or at least for those files to be compiled individually).
The argument \textit{main} must be the filename of the main file.

There are a couple of
considerations in setting up the main and child documents:

%%%%%%%%%%%%%%%%%%%%%%%%%%%%%%%%%%%%%%%%
\paragraph{Restrictions.}

Please note the following restrictions:
\begin{itemize}
\item
|\childdocmain| must be called with one argument \textit{main}
to ensure compatibility with earlier version of the package.
It must either be empty (|\childdocmain{}|)
or precisely match the filename of the main file in which it is specified.
See \secref{sec:detection} for further information.
\item
The filename \textit{main} must be specified without the |.tex| extension.
\item
The filename \textit{main} is case sensitive
(even in case-insensitive file systems)
due to internal string comparison.
\item
The argument \textit{main} should be fully expanded, it cannot be a macro.
\item
Subdirectories and special characters should be avoided in filenames.
\item
The command |\childdocmain{|\textit{main}|}| must be followed by a whitespace.
It should not be followed immediately by another command
or by a comment mark `|%|'.
This is because the \TeX{} parser reads the token immediately following
the argument of |\childdocmain| and puts it
at the beginning of every child section;
however, a white\-space is ignored.
\end{itemize}

%%%%%%%%%%%%%%%%%%%%%%%%%%%%%%%%%%%%%%%%
\paragraph{Content of Main File.}

It is advisable to place all content in the child files included by |\include|.
Any output contained in the main file will appear in all child documents
unless suppressed manually;
it cannot be suppressed automatically by the |\includeonly| directive
and thus should normally be avoided.
A method to include some content in the main file
by means of conditional processing is described in \secref{sec:conditional}.

%%%%%%%%%%%%%%%%%%%%%%%%%%%%%%%%%%%%%%%%
\paragraph{Page Numbering.}

When only a part of the document is compiled,
the appropriate numbering of pages
(as well as other status parameters)
is determined from the |.aux| files.
The latter contain information from previous passes.
However this information needs to propagate through
all intermediate child documents.
Therefore the page numbering in child documents may well
be inconsistent until the complete document is compiled at least once.

A useful (if unconventional) way to always ensure a consistent
page numbering is to restart the numbering in each child document
and denote the pages by `\textit{child}|.|\textit{page}'
where \textit{child} represents the chapter/section number of the child file.
This can be achieved by the command
|\numberwithin{page}{|\textit{child}|}|
of the \textsf{amsmath} package
where \textit{child} can be |chapter| or |section|
depending on the chosen structuring.
Alternatively, one can modify the macro |\thepage| appropriately
and reset the counter |page| at the start of each child file.

%%%%%%%%%%%%%%%%%%%%%%%%%%%%%%%%%%%%%%%%%%%%%%%%%%%%%%%%%%%%%%%%%%%%%%%%%%%%%%%%
\subsection{Conditional Processing}
\label{sec:conditional}

The package provides a mechanism to compile different versions
of a document. To customise the versions further some conditional processing
can come in handy to distinguish which version is being compiled.
The package provides two macros to describe the compilation context:

%%%%%%%%%%%%%%%%%%%%%%%%%%%%%%%%%%%%%%%%
\DescribeMacro{\ifchilddoc}
The conditional |\ifchilddoc| distinguishes between the compilation of
child documents and the main document:
%
\begin{center}
|\ifchilddoc |\textit{child-code}| |[|\||else |\textit{main-code}]| \||fi|
\end{center}

%%%%%%%%%%%%%%%%%%%%%%%%%%%%%%%%%%%%%%%%
\DescribeMacro{\childdocname}
\DescribeMacro{\childdocjob}
The macro |\childdocname| contains the filename (without extension)
of the main or child file being processed.
Note that |\childdocjob| will always contain the name of the main file.

%%%%%%%%%%%%%%%%%%%%%%%%%%%%%%%%%%%%%%%%
\paragraph{Title Page.}

Conditional processing can be used to include a title or banner page
in the main document when proper precautions are taken.
Importantly, the code in the main file should ensure that the page counter
(as well as other status parameters which are stored in the |.aux| files)
takes the same value after the conditional processing.
Otherwise the page numbers may take divergent values
depending on which part is compiled.

For example, a title page could be declared by:
%
\begin{center}
\begin{tabular}{l}
|\ifchilddoc\||else|\\
|\addtocounter{page}{-1}|\\
\textit{code for title page}\\
|\newpage|\\
|\||fi|
\end{tabular}
\end{center}
%
A banner page for the child documents can be generated by:
%
\begin{center}
\begin{tabular}{l}
|\ifchilddoc|\\
|\addtocounter{page}{-1}|\\
\textit{code for banner page}\\
|\newpage|\\
|\||fi|
\end{tabular}
\end{center}
%
Here one could write a message such as:
\begin{center}
|This is the part \childdocname{} of \childdocjob{}.|
\end{center}

%%%%%%%%%%%%%%%%%%%%%%%%%%%%%%%%%%%%%%%%%%%%%%%%%%%%%%%%%%%%%%%%%%%%%%%%%%%%%%%%
\subsection{Flags}
\label{sec:flags}

The package makes it easy to generate different versions
of the main or child documents.
To this end compilation flags can be defined
and assigned different default values.
They will be particularly useful in conjunction
with the forwarding mechanism described in \secref{sec:forward}.

For example, it may be useful to have a flag |\version|
which can be set to |draft| or |final|.
The document source will contain some conditional code
depending on the value of |\version|.
Suppose further, the flag should default to |final| for the main file
and to |draft| for child files
which is a natural assignment for editing the document.
This is achieved by placing the following code
in the preamble of the main document
(below the |\childdocmain| directive):
%
\begin{center}
\begin{tabular}{l}
|\ifchilddoc|\\
|\providecommand{\version}{draft}|\\
|\||else|\\
|\providecommand{\version}{final}|\\
|\||fi|
\end{tabular}
\end{center}
%
The definition by |\providecommand| makes sure
that previous definitions are not overwritten.
Further statements |\providecommand{\version}{...}|
can thus be added before the above code to override it.

For the main file, one might add a line
(between |\childdocmain| and the above block)
%
\begin{center}
|%\ifchilddoc\||else\providecommand{\version}{draft}\||fi|
\end{center}
%
which can be uncommented to produce a draft version.
Likewise one can add a line to the very top of a child file
(above the |\childdocof{|\textit{main}|}| directive)
%
\begin{center}
|%\providecommand{\version}{final}|
\end{center}
%
which can be uncommented to produce the final version of this child document.

%%%%%%%%%%%%%%%%%%%%%%%%%%%%%%%%%%%%%%%%%%%%%%%%%%%%%%%%%%%%%%%%%%%%%%%%%%%%%%%%
\subsection{Forwarding}
\label{sec:forward}

Different versions of the main or child documents
using compilation flags as described in \secref{sec:flags}
can be (permanently) stored in different files
for convenient compilation, viewing and distribution.
To this end, the package defines a command
to pass on compilation to a different file:

%%%%%%%%%%%%%%%%%%%%%%%%%%%%%%%%%%%%%%%%
\DescribeMacro{\childdocforward}
The command |\childdocforward| redirects processing to
another source file:
%
\begin{center}
\begin{tabular}{l}
|\input{childdoc.def}|\\
|\childdocforward[|\textit{main}|]{|\textit{dest}|}|\\
\end{tabular}
\end{center}
%
The argument \textit{dest} is the destination file
(without extension).
It should be the main file or one of the child files.
Note that further \textsf{childdoc} directives
such as |\childdocof| and |\childdocforward|
in the indicated file will be processed in this form.
The optional argument \textit{main}
passes on directly to the main file \textit{main}
while pretending to compile the child \textit{dest}.
This form behaves as if \textit{dest}
issues |\childdocof{|\textit{main}|}| right away,
and no further \textsf{childdoc} directives will be processed.

%%%%%%%%%%%%%%%%%%%%%%%%%%%%%%%%%%%%%%%%
\DescribeMacro{\...prefix}
In the alternative form |\childdocforwardprefix|,
%
\begin{center}
\begin{tabular}{l}
|\input{childdoc.def}|\\
|\childdocforwardprefix[|\textit{main}|]{|\textit{prefix}|}{|\textit{dest}|}|
\end{tabular}
\end{center}
%
the destination file is determined by a pattern
depending on the current file:
To make this work, the current file must be called
`{\textit{prefix}\hspace{0.2em}\textit{suffix}}'
with \textit{prefix} matching precisely the argument.
Processing is then passed on to the file
`{\textit{dest}\hspace{0.2em}\textit{suffix}}'.
Surely, the same effect is achieved by
directly specifying the
argument `{\textit{dest}\hspace{0.2em}\textit{suffix}}'
in the first form.
However, that requires to set up a different file
for each child. With the alternative form of the command
all these files can have exactly the same content
which simplifies setting them up and maintaining them.

For example, the following file |draft.tex|
with a compilation flag |\version| as described in \secref{sec:flags}
compiles the main document as a draft:
%
\begin{center}
\begin{tabular}{l}
|\def\version{draft}|\\
|\input{childdoc.def}|\\
|\childdocforward{|\textit{main}|}|
\end{tabular}
\end{center}
%
Likewise, the following files |final|\textit{nn}|.tex|
compile the final version of the child document
|child|\textit{nn}|.tex|:
%
\begin{center}
\begin{tabular}{l}
|\def\version{final}|\\
|\input{childdoc.def}|\\
|\childdocforwardprefix{final}{child}|
\end{tabular}
\end{center}
%

Note that when several versions of a main file and/or of each child file
are to be generated, it may be convenient to set up a |Makefile| or
shell script to automatise the process.

%%%%%%%%%%%%%%%%%%%%%%%%%%%%%%%%%%%%%%%%%%%%%%%%%%%%%%%%%%%%%%%%%%%%%%%%%%%%%%%%
\subsection{Command Line Processing}
\label{sec:commandline}

The effect of redirection files can also be achieved by invoking
the \LaTeX{} compiler with a more elaborate command line.
Most conveniently this should be done as part
of a shell script or a |Makefile|.

When using \textsf{childdoc} in the main file, the following
command lines effectively perform a redirection
(note that depending on the shell being used,
backslashes may have to be doubled: `|\|' $\to$ `|\\|'):
%
\begin{center}
|... -jobname "|\textit{target}|" |\\|"|[\textit{flags}]%
|\input{childdoc.def}\childdocforward[|\textit{main}|]{|\textit{dest}|}"|
\end{center}
%
Here \textit{target} is the name of the output file,
\textit{main} is the name of the main file
and \textit{dest} is the name of the main or child file to be processed
(all filenames without extensions).
The optional argument \textit{main} can be omitted
if \textit{main} matches \textit{dest}.
Optionally, compilation \textit{flags} can be defined via |\def| commands.
This command line makes the \TeX{} engine believe
it is compiling the file \textit{target}
whose content is specified as the latter parameter.
The provided code then forwards the processing to
\textit{main} or \textit{dest} as described in \secref{sec:forward}.

%%%%%%%%%%%%%%%%%%%%%%%%%%%%%%%%%%%%%%%%%%%%%%%%%%%%%%%%%%%%%%%%%%%%%%%%%%%%%%%%
\subsection{Include by Input}
\label{sec:input}

Including child documents by |\include| has some restrictions by design.
Most notably, the content of a child document always occupies
its own set of pages; pages cannot be shared between child documents.
Usually, this behaviour makes perfect sense
because each child document contain an essential part of the document.
However, in some situations it may be desirable to compose
a document from a collection of parts
without having mandatory page breaks between then.
For this case, the package
provides a mechanism to include parts
by |\input| which can also be processed individually.
However, by construction this mechanism
requires manual handling of the content to be output.

%%%%%%%%%%%%%%%%%%%%%%%%%%%%%%%%%%%%%%%%
\DescribeMacro{\ifchilddocmanual}
The main file should be prepared as usual, see \secref{sec:include}.
However, the document body must make a distinction
between processing of an individual part and of the main document, e.g.:
%
\begin{center}
\begin{tabular}{l}
|\ifchilddocmanual|\\
|\input{\childdocname}|\\
|\||else|\\
\textit{document body with }|\input{|\textit{part}|}|\\
|\||fi|
\end{tabular}
\end{center}
%
The conditional |\ifchilddocmanual| is true whenever
a part to be included by |\input| is being compiled,
and the name of the part is stored in |\childdocname|.

%%%%%%%%%%%%%%%%%%%%%%%%%%%%%%%%%%%%%%%%
\DescribeMacro{\childdocby}
Each part to be included by |\input| should start with:
%
\begin{center}
\begin{tabular}{l}
|\input{childdoc.def}|\\
|\childdocby{|\textit{main}|}|\\
\end{tabular}
\end{center}
%
The directive |\childdocby| is similar to |\childdocof|
described in \secref{sec:include},
but the subsequent selection of content must be done manually.
To that end, both |\ifchilddoc| and |\ifchilddocmanual|
will be true upon processing of a part,
and the name of the part is stored in |\childdocname|.
Note that |\jobname| will be set to the filename of the current part
so that each part receives an individual |.aux| file
that does not interfere with the |.aux| file(s) of the main document.
This behaviour can be altered by the alternative form
|\childdocby[*]{|\textit{main}|}| (with a non-empty optional argument)
which uses the |.aux| file of the main document
by setting |\jobname| to \textit{main}.

%%%%%%%%%%%%%%%%%%%%%%%%%%%%%%%%%%%%%%%%%%%%%%%%%%%%%%%%%%%%%%%%%%%%%%%%%%%%%%%%
\subsection{Driver Development}
\label{sec:driver}

The \textsf{childdoc} mechanism can also be use for the development
of definition files such as \LaTeX{} styles or classes.
This case differs from the above setup with multiple parts
included by |\include| in that no |\includeonly| should be invoked.
This can be achieved by starting the include file
(before |\ProvidesPackage|) with:
%
\begin{center}
\begin{tabular}{l}
|\input{childdoc.def}|\\
|\childdocforward{|\textit{main}|}|\\
\end{tabular}
\end{center}
%
or alternatively with:
%
\begin{center}
\begin{tabular}{l}
|\input{childdoc.def}|\\
|\childdocby{|\textit{main}|}|\\
\end{tabular}
\end{center}
%
Both forms have slightly different effects as described above.
The main file is prepared as usual, see \secref{sec:include}.

%%%%%%%%%%%%%%%%%%%%%%%%%%%%%%%%%%%%%%%%%%%%%%%%%%%%%%%%%%%%%%%%%%%%%%%%%%%%%%%%
\subsection{Legacy Detection}
\label{sec:detection}

The directive |\childdocmain| in the main file can detect
whether the complete document or merely a child is to be compiled
even without using the directive |\childdocof|.
This method is deprecated because it is less robust
and there is no compelling reason to use it;
it is merely provided for backward compatibility
and it may be removed in future versions.

If the detection mechanism is to be used,
it is mandatory to correctly specify
the filename of the main file as the argument of |\childdocmain|:
%
\begin{center}
\begin{tabular}{l}
|\input{childdoc.def}|\\
|\childdocmain{|\textit{main}|}|\\
\end{tabular}
\end{center}
%
If |\jobname| does not match the argument \textit{main} of |\childdocmain|,
it is assumed that |\jobname| points to the child file to be compiled.
When using |\childdocmain| with the main file specified as argument,
it suffices to start a child file
with just |\input{|\textit{main}|}|
without loading of the package and using |\childdocof|.
If instead all processing is done
with the appropriate \textsf{childdoc} directives,
the argument of \textit{main} of |\childdocmain| can be empty.

An alternative version of the command line processing described
in \secref{sec:commandline} using the detection mechanism reads:
%
\begin{center}
|... -jobname "|\textit{target}|" "|[\textit{flags}]%
[|\def\jobname{|\textit{dest}|}|]|\input{|\textit{main}|}"|
\end{center}

%%%%%%%%%%%%%%%%%%%%%%%%%%%%%%%%%%%%%%%%%%%%%%%%%%%%%%%%%%%%%%%%%%%%%%%%%%%%%%%%
\subsection{Manual Code}
\label{sec:manual}

In case one cannot be certain whether the definitions file |childdoc.def|
is installed on the target \TeX{} distribution
and one prefers not to ship it,
it is conceivable to paste a few relevant commands into the sources.

To that end, drop all statements |\input{childdoc.def}|
and perform the replacements as outlined below.
Instead of |\childdocmain{|\textit{main}|}| add the following code
to the top of the main file:
%
\begin{center}
\begin{tabular}{l}
|\||ifdefined\childdocname\endinput\||fi\newif\ifchilddoc|\\
|\edef\childdocname{\scantokens\expandafter{\jobname\noexpand}}|\\
|\def\childdocmain{|\textit{main}|}\||ifx\childdocmain\childdocname\||else|\\
|\childdoctrue\includeonly{\childdocname}\let\jobname\childdocmain\||fi|\\
\end{tabular}
\end{center}
%
Instead of |\childdocof{|\textit{main}|}| just include the main file
at the top of each child file:
%
\begin{center}
|\input{|\textit{main}|}|
\end{center}
%
A simple redirection |\childdocforward{|\textit{dest}|}| is achieved by:
%
\begin{center}
|\def\jobname{|\textit{dest}|}\input{\jobname}|
\end{center}
%
The redirection with prefix
|\childdocforwardprefix[|\textit{prefix}|]{|\textit{dest}|}|
is accomplished by:
%
\begin{center}
\begin{tabular}{l}
|{\edef\jobname{\scantokens\expandafter{\jobname\noexpand}}|\\
|\def\redirectjob |\textit{prefix}|#1~~~{\gdef\jobname{|\textit{dest}|#1}}|\\
|\expandafter\redirectjob\jobname~~~}\input{\jobname}|
\end{tabular}
\end{center}

In an alternative approach,
child documents can be compiled by a specific command line
without additional code or specific definitions:
%
\begin{center}
|... -jobname "|\textit{target}|" "|[\textit{flags}]%
|\includeonly{|\textit{dest}|}\input{|\textit{main}|}"|
\end{center}
%

%%%%%%%%%%%%%%%%%%%%%%%%%%%%%%%%%%%%%%%%%%%%%%%%%%%%%%%%%%%%%%%%%%%%%%%%%%%%%%%%
%%%%%%%%%%%%%%%%%%%%%%%%%%%%%%%%%%%%%%%%%%%%%%%%%%%%%%%%%%%%%%%%%%%%%%%%%%%%%%%%
\section{Information}

%%%%%%%%%%%%%%%%%%%%%%%%%%%%%%%%%%%%%%%%%%%%%%%%%%%%%%%%%%%%%%%%%%%%%%%%%%%%%%%%
\subsection{Copyright}

Copyright \copyright{} 2017--2018 Niklas Beisert

This work may be distributed and/or modified under the
conditions of the \LaTeX{} Project Public License, either version 1.3
of this license or (at your option) any later version.
The latest version of this license is in
  \url{http://www.latex-project.org/lppl.txt}
and version 1.3 or later is part of all distributions of \LaTeX{}
version 2005/12/01 or later.

This work has the LPPL maintenance status `maintained'.

The Current Maintainer of this work is Niklas Beisert.

This work consists of the files |README.txt|, |childdoc.ins| and |childdoc.dtx|
as well as the derived files |childdoc.def|, |cdocsamp.tex|
with |cdocsch1.tex|, |cdocsch2.tex|, |cdocspt3.tex|, |cdocspt4.tex|,
|cdocsdrf.tex|, |cdocsfn1.tex|, |cdocsfn2.tex|
as well as |childdoc.pdf|.

%%%%%%%%%%%%%%%%%%%%%%%%%%%%%%%%%%%%%%%%%%%%%%%%%%%%%%%%%%%%%%%%%%%%%%%%%%%%%%%%
\subsection{Files and Installation}

The package consists of the files:
%
\begin{center}
\begin{tabular}{ll}
    |README.txt|   & readme file \\
    |childdoc.ins| & installation file \\
    |childdoc.dtx| & source file \\
    |childdoc.def| & definition file \\
    |cdocsamp.tex| & sample main file \\
    |cdocsch1.tex| & sample include file \\
    |cdocsch2.tex| & sample include file \\
    |cdocspt3.tex| & sample part file \\
    |cdocspt4.tex| & sample part file \\
    |cdocsdrf.tex| & sample redirection file \\
    |cdocsfn1.tex| & sample redirection file \\
    |cdocsfn2.tex| & sample redirection file \\
    |childdoc.pdf| & manual
\end{tabular}
\end{center}
%
The distribution consists of the files
|README.txt|, |childdoc.ins| and |childdoc.dtx|.
%
\begin{itemize}
\item
Run (pdf)\LaTeX{} on |childdoc.dtx|
to compile the manual |childdoc.pdf| (this file).
\item
Run \LaTeX{} on |childdoc.ins| to create the definitions file |childdoc.def|
and the sample |cdocsamp.tex| with include files
|cdocsch1.tex|, |cdocsch2.tex|, |cdocspt3.tex|, |cdocspt4.tex|,
|cdocsdrf.tex|, |cdocsfn1.tex|, |cdocsfn2.tex|.
Then copy the file |childdoc.def| to an appropriate directory of your \LaTeX{}
distribution, e.g.\ \textit{texmf-root}|/tex/latex/childdoc|.
\end{itemize}

%%%%%%%%%%%%%%%%%%%%%%%%%%%%%%%%%%%%%%%%%%%%%%%%%%%%%%%%%%%%%%%%%%%%%%%%%%%%%%%%
\subsection{Related CTAN Packages}

There are several other packages which offer a similar functionality:
%
\begin{itemize}
\item
The packages
\href{http://ctan.org/pkg/docmute}{\textsf{docmute}},
\href{http://ctan.org/pkg/includex}{\textsf{includex}} and
\href{http://ctan.org/pkg/standalone}{\textsf{standalone}}
provide commands to include only the document body of
a child file thus allowing both files to be compiled individually.
\item
The packages \href{http://ctan.org/pkg/subdocs}{\textsf{subdocs}}
and \href{http://ctan.org/pkg/subfiles}{\textsf{subfiles}}
provide structures in which the main and child documents can be
encapsulated and allowing them to be compiled individually.
The inclusion mechanism is different from the conventional |\include|.
\item
The package \href{http://ctan.org/pkg/combine}{\textsf{combine}}
is an elaborate solution to combine several documents into one.
\end{itemize}
%
See also the CTAN topic \href{http://ctan.org/topic/subdocs}{\textsf{subdocs}}
for further related packages.
The present package differs from the above solutions in that
a document structure constructed with the conventional |\include| mechanism
just needs two extra commands at the top of every file
such that all constituent files can be compiled individually.

%%%%%%%%%%%%%%%%%%%%%%%%%%%%%%%%%%%%%%%%%%%%%%%%%%%%%%%%%%%%%%%%%%%%%%%%%%%%%%%%
%\subsection{Feature Suggestions}
%
%The following is a list of features which may be useful for future
%versions of this package:
%%
%\begin{itemize}
%\item
%\ldots
%\end{itemize}

%%%%%%%%%%%%%%%%%%%%%%%%%%%%%%%%%%%%%%%%%%%%%%%%%%%%%%%%%%%%%%%%%%%%%%%%%%%%%%%%
\subsection{Revision History}

%%%%%%%%%%%%%%%%%%%%%%%%%%%%%%%%%%%%%%%%
\paragraph{v2.0:} 2018/12/30

\begin{itemize}
\item
immediate forward processing
\item
added |\childdocby| mechanism
\item
manual restructured
\end{itemize}

%%%%%%%%%%%%%%%%%%%%%%%%%%%%%%%%%%%%%%%%
\paragraph{v1.6:} 2018/01/17

\begin{itemize}
\item
application for development of include files
\item
corrections to manual
\end{itemize}

%%%%%%%%%%%%%%%%%%%%%%%%%%%%%%%%%%%%%%%%
\paragraph{v1.5:} 2017/05/21

\begin{itemize}
\item
more complete structuring introduced
\item
|\childdocof| introduced
\item
|\childdoc| renamed to |\childdocmain|
\item
|\childredirect| renamed to |\childdocforward| and |\childdocforwardprefix|
and functionality expanded
\end{itemize}

%%%%%%%%%%%%%%%%%%%%%%%%%%%%%%%%%%%%%%%%
\paragraph{v1.0:} 2017/04/27

\begin{itemize}
\item
manual and install package
\item
first version published on CTAN
\end{itemize}

%%%%%%%%%%%%%%%%%%%%%%%%%%%%%%%%%%%%%%%%
\paragraph{v0.6:} 2017/04/26

\begin{itemize}
\item
redirection mechanism added
\end{itemize}

%%%%%%%%%%%%%%%%%%%%%%%%%%%%%%%%%%%%%%%%
\paragraph{v0.5:} 2017/04/26

\begin{itemize}
\item
functionality in definition file
\end{itemize}


%%%%%%%%%%%%%%%%%%%%%%%%%%%%%%%%%%%%%%%%%%%%%%%%%%%%%%%%%%%%%%%%%%%%%%%%%%%%%%%%
%%%%%%%%%%%%%%%%%%%%%%%%%%%%%%%%%%%%%%%%%%%%%%%%%%%%%%%%%%%%%%%%%%%%%%%%%%%%%%%%
%%%%%%%%%%%%%%%%%%%%%%%%%%%%%%%%%%%%%%%%%%%%%%%%%%%%%%%%%%%%%%%%%%%%%%%%%%%%%%%%
\appendix

\settowidth\MacroIndent{\rmfamily\scriptsize 000\ }

 \DocInput{childdoc.dtx}

\end{document}
%</driver>
% \fi
%
% %%%%%%%%%%%%%%%%%%%%%%%%%%%%%%%%%%%%%%%%%%%%%%%%%%%%%%%%%%%%%%%%%%%%%%%%%%%%%%
% %%%%%%%%%%%%%%%%%%%%%%%%%%%%%%%%%%%%%%%%%%%%%%%%%%%%%%%%%%%%%%%%%%%%%%%%%%%%%%
% \section{Sample}
%\iffalse
%<*samplemain>
%\fi
%
% The following presents a sample document
% with two chapters, two parts, a title page,
% a compile flag as well as three forwarding files to set the flag.
% It consists of eight |.tex| files:
% \begin{center}
% \begin{tabular}{ll}
% |cdocsamp.tex|&main file\\
% |cdocsch1.tex|&include file for chapter 1\\
% |cdocsch2.tex|&include file for chapter 2\\
% |cdocspt3.tex|&include file for part 3\\
% |cdocspt4.tex|&include file for part 4\\
% |cdocsdrf.tex|&forwarding file for main file in draft mode\\
% |cdocsfi1.tex|&forwarding file for final version of chapter 1\\
% |cdocsfi2.tex|&forwarding file for final version of chapter 2\\
% \end{tabular}
% \end{center}
% Each of the eight files can be compiled directly by the \LaTeX{} compiler.
%
% %%%%%%%%%%%%%%%%%%%%%%%%%%%%%%%%%%%%%%
% \paragraph{Main File.}
%
% The main file is called |cdocsamp.tex|.
%
% Load the \textsf{childdoc} definitions and
% declare the filename for the main document:
%    \begin{macrocode}
\input{childdoc.def}
\childdocmain{}
%    \end{macrocode}

% Optional override for |\version| flag:
%    \begin{macrocode}
%%\ifchilddoc\else\providecommand{\version}{draft}\fi
%    \end{macrocode}

% Define the default values for the |\version| flag
% (|final| for the main file and |draft| for childs):
%    \begin{macrocode}
\ifchilddoc
\providecommand{\version}{draft}
\else
\providecommand{\version}{final}
\fi
%    \end{macrocode}

% Load the standard document class:
%    \begin{macrocode}
\documentclass[12pt]{article}
%    \end{macrocode}

% Start the document body:
%    \begin{macrocode}
\begin{document}
%    \end{macrocode}

% Declare a title page.
% Print title, part of document being processed and version flag:
%    \begin{macrocode}
\addtocounter{page}{-1}
\begin{center}
{\LARGE\bfseries{}childdoc example\par}
\vspace{1cm}
\ifchilddoc
\ifchilddocmanual part\else chapter\fi:
`\childdocname' of `\childdocjob'\par
\else
main document: `\childdocjob'\par
\fi
version: \version\par
\end{center}
\newpage
%    \end{macrocode}

% Manually include selected file,
% otherwise process as usual:
%    \begin{macrocode}
\ifchilddocmanual
\section*{part `\childdocname'}
\input{\childdocname}
\else
%    \end{macrocode}

% Include the two chapters:
%    \begin{macrocode}
\include{cdocsch1}
\include{cdocsch2}
%    \end{macrocode}

% Include the two parts unless only chapters should be displayed:
%    \begin{macrocode}
\ifchilddoc\else
\section{part three}
\input{cdocspt3}
\section{part four}
\input{cdocspt4}
\fi
%    \end{macrocode}

% Process as usual until here:
%    \begin{macrocode}
\fi
%    \end{macrocode}

% End of document body:
%    \begin{macrocode}
\end{document}
%    \end{macrocode}
%\iffalse
%</samplemain>
%\fi
%
% %%%%%%%%%%%%%%%%%%%%%%%%%%%%%%%%%%%%%%
% \paragraph{Chapter Include Files.}
%
% The include files are called |cdocsch1.tex| and |cdocsch2.tex|.
%
%\iffalse
%<*samplechap1|samplechap2>
%\fi

% Optional override for |\version| flag:
%    \begin{macrocode}
%%\providecommand{\version}{final}
%    \end{macrocode}

% Include the main document:
%    \begin{macrocode}
\input{childdoc.def}
\childdocof{cdocsamp}
%    \end{macrocode}

%\iffalse
%</samplechap1|samplechap2>
%\fi
%
%\iffalse
%<*samplechap1>
%\fi
% Some text for chapter 1:
%    \begin{macrocode}
\section{one}
some text in chapter one
%    \end{macrocode}

%\iffalse
%</samplechap1>
%\fi
% Some text for chapter 2:
%\iffalse
%<*samplechap2>
%\fi
%    \begin{macrocode}
\section{two}
more text in chapter two
%    \end{macrocode}

%\iffalse
%</samplechap2>
%\fi
%
% %%%%%%%%%%%%%%%%%%%%%%%%%%%%%%%%%%%%%%
% \paragraph{Part Include Files.}
%
% The include files are called |cdocspt3.tex| and |cdocspt4.tex|.
%
%\iffalse
%<*samplepart3|samplepart4>
%\fi

% Optional override for |\version| flag:
%    \begin{macrocode}
%%\providecommand{\version}{final}
%    \end{macrocode}

% Include the main document:
%    \begin{macrocode}
\input{childdoc.def}
\childdocby{cdocsamp}
%    \end{macrocode}

%\iffalse
%</samplepart3|samplepart4>
%\fi
%
%\iffalse
%<*samplepart3>
%\fi
% Some text for part 3:
%    \begin{macrocode}
some text in part three
%    \end{macrocode}

%\iffalse
%</samplepart3>
%\fi
% Some text for part 4:
%\iffalse
%<*samplepart4>
%\fi
%    \begin{macrocode}
more text in part four
%    \end{macrocode}

%\iffalse
%</samplepart4>
%\fi
%
% %%%%%%%%%%%%%%%%%%%%%%%%%%%%%%%%%%%%%%
% \paragraph{Forwarding for a Complete Draft.}
%
% The following forwarding file |cdocsdrf.tex|
% compiles the main document in draft mode:
%\iffalse
%<*sampledraft>
%\fi
%    \begin{macrocode}
\def\version{draft}
\input{childdoc.def}
\childdocforward{cdocsamp}
%    \end{macrocode}

%\iffalse
%</sampledraft>
%\fi
%
% %%%%%%%%%%%%%%%%%%%%%%%%%%%%%%%%%%%%%%
% \paragraph{Forwarding for Final Version of the Chapters.}
%
% The following forwarding files |cdocsfn1.tex| and |cdocsfn2.tex|
% (with identical content)
% compile the final versions of the child documents
% |cdocsch1.tex| and |cdocsch2.tex|, respectively:
%\iffalse
%<*samplefinal>
%\fi
%    \begin{macrocode}
\def\version{final}
\input{childdoc.def}
\childdocforwardprefix[cdocsamp]{cdocsfn}{cdocsch}
%    \end{macrocode}

%\iffalse
%</samplefinal>
%\fi
%
% %%%%%%%%%%%%%%%%%%%%%%%%%%%%%%%%%%%%%%
% \paragraph{Command Line Processing.}
%
% The following three command lines generate the output files
% |cdocscld|, |cdocscl1| and |cdocscl2|
% which should be identical to
% |cdocsdrf|, |cdocsch1| and |cdocsfn2|, respectively:
% \begin{center}
% \begin{tabular}{l}
% |latex -jobname cdocscld \|\\
% |  "\def\version{draft}\input{childdoc.def}\childdocforward{cdocsamp}"|\\
% |latex -jobname cdocscl1 \|\\
% |  "\input{childdoc.def}\childdocforward[cdocsamp]{cdocsch1}"|\\
% |latex -jobname cdocscl2 \|\\
% |  "\def\version{final}\input{childdoc.def}\childdocforward{cdocsch2}"|
% \end{tabular}
% \end{center}
% Note that the trailing backslash on each first line
% merely continues the input to the second line
% (for convenient cut ant paste).
% Furthermore, the command |latex| can be replaced by any
% of its alternative versions such as |pdflatex|.
%
% %%%%%%%%%%%%%%%%%%%%%%%%%%%%%%%%%%%%%%%%%%%%%%%%%%%%%%%%%%%%%%%%%%%%%%%%%%%%%%
% %%%%%%%%%%%%%%%%%%%%%%%%%%%%%%%%%%%%%%%%%%%%%%%%%%%%%%%%%%%%%%%%%%%%%%%%%%%%%%
% \section{Implementation}
%\iffalse
%<*package>
%\fi
%
% This section describes the definitions file |childdoc.def|.

% The definitions cannot be loaded using |\usepackage| or |\RequirePackage|
% which has a mechanism to prevent loading a style file more than once.
% When loading the definitions by means of |\input|
% multiple instances have to be prevented manually:
%\iffalse
%This code needs to be before the `\ProvidesFile' directive
%which is defined at the beginning of this file.
%Therefore it is also placed there and commented out here.
%</package>
%<*discard>
%\fi
%    \begin{macrocode}
\ifdefined\childdocmain\endinput\fi
%    \end{macrocode}
%\iffalse
%</discard>
%<*package>
%\fi
%
% \macro{\ifchilddoc}
% \macro{\ifchilddocmanual}
% The conditional |\ifchilddoc| tells whether a
% child (true) or main (false) document is being compiled.
% The conditional |\ifchilddocmanual| tells whether
% the |\includeonly| mechanism is used (false) or
% the selection of child files must be performed manually (true).
% The definitions initialise to false:
%    \begin{macrocode}
\newif\ifchilddoc
\newif\ifchilddocmanual
%    \end{macrocode}

% \macro{\childdocname}
% \macro{\childdocjob}
% The macro |\childdocname| stores the name of the main document
% to be compiled. The macro |\childdocjob| stores the name of
% the document on which the \LaTeX{} compiler was originally invoked.
% The content of |\jobname| cannot be compared
% to filenames specified in the source due to different catcodes.
% The following code rescans |\jobname|, stores the result
% in |\childdocname| and saves a copy in |\childdocjob|:
%    \begin{macrocode}
\edef\childdocname{\scantokens\expandafter{\jobname\noexpand}}
\let\childdocjob\childdocname
%    \end{macrocode}

% \macro{\childdocdisable}
% The macro |\childdocdisable| prevents the main file
% from being processed more than once.
% At this stage, the main document command |\childdocmain|
% is assumed to be called once again where it should do nothing.
% Any subsequent call to it should prevent
% a secondary processing of the main document
% It overwrites the forwarding commands
% |\childdocof| and |\childdocforward|
% with empty macros to prevent further inclusions of the main document:
%    \begin{macrocode}
\newcommand{\childdocdisable}
{
  \renewcommand{\childdocmain}[1]{\renewcommand{\childdocmain}[1]{\endinput}}
  \renewcommand{\childdocof}[1]{}
  \renewcommand{\childdocby}[2][]{}
  \renewcommand{\childdocforward}[2][]{}
  \renewcommand{\childdocdisable}{}
}
%    \end{macrocode}

% \macro{\childdocmain}
% The macro |\childdocmain| is to be called at the top of the main file
% with nothing or the main filename (without extension) as argument.
% First, it breaks loops.
% If the argument is not empty and does not match |\childdocname|
% (which is set by the first inclusion of |childdoc.def|),
% |\ifchilddoc| is set to true, |\includeonly| is applied to the child file
% and |\jobname| is set to the main file
% (for proper handling of |.aux| files):
%    \begin{macrocode}
\newcommand{\childdocmain}[1]
{
  \childdocdisable\childdocmain{}
  \if?#1?\else
    \begingroup
      \def\childdoctmp{#1}
      \ifx\childdoctmp\childdocname
        \def\childdoctmp{}
      \else
        \def\childdoctmp
        {
          \childdoctrue
          \includeonly{\childdocname}
          \def\childdocjob{#1}
          \def\jobname{#1}
        }
      \fi
      \expandafter
    \endgroup
    \childdoctmp
  \fi
}
%    \end{macrocode}

% \macro{\childdocof}
% The command |\childdocof| redirects
% compilation to the main file |#1|.
%    \begin{macrocode}
\newcommand{\childdocof}[1]
{
  \childdocdisable
  \childdoctrue
  \includeonly{\childdocname}
  \def\jobname{#1}
  \def\childdocjob{#1}
  \input{#1}
}
%    \end{macrocode}

% \macro{\childdocby}
% The command |\childdocby| ....
%    \begin{macrocode}
\newcommand{\childdocby}[2][]
{
  \childdocdisable
  \childdoctrue
  \childdocmanualtrue
  \if?#1?\else
    \def\jobname{#2}
  \fi
  \def\childdocjob{#2}
  \input{#2}
  \endinput
}
%    \end{macrocode}

% \macro{\childdocforward}
% The command |\childdocforward| redirects
% compilation to the main file or
% (if the optional argument is given) a child file.
% Parameters are set as if the main file
% or a child file starting with |\childdocof| was compiled.
% Then compilation is handed over to the main file:
%    \begin{macrocode}
\newcommand{\childdocforward}[2][]
{
  \begingroup
    \if?#1?
      \def\childdoctmp
      {
        \def\childdocname{#2}
        \def\childdocjob{#2}
        \def\jobname{#2}
        \input{#2}
        \endinput
      }
    \else
      \def\childdoctmp
      {
        \childdocdisable
        \def\childdocname{#2}
        \childdoctrue
        \includeonly{#2}
        \def\childdocjob{#1}
        \def\jobname{#1}
        \input{#1}
        \endinput
      }
    \fi
    \expandafter
  \endgroup
  \childdoctmp
}
%    \end{macrocode}

% \macro{\childdocforwardprefix}
% The command |\childdocforwardprefix| redirects
% compilation to the main or a child file by means of a pattern.
% The prefix |#1| in the current filename is replaced by |#2|
% and the suffix of the current filename is kept
% (it is assumed that the filename does not contain the substring `|~~~|'
% which is used as a delimiter).
% Compilation is handed over to the new file by |\childdocforward|:
%    \begin{macrocode}
\newcommand{\childdocforwardprefix}[3][]
{
  \begingroup
    \def\childdocextract #2##1~~~{\def\childdoctmp{\childdocforward[#1]{#3##1}}}
    \expandafter\childdocextract\childdocname~~~
    \expandafter
  \endgroup
  \childdoctmp
}
%    \end{macrocode}

% \macro{\childdoc}
% The deprecated macro |\childdoc| is a legacy version of |\childdocmain|:
%    \begin{macrocode}
\newcommand{\childdoc}{\childdocmain}
%    \end{macrocode}

% \macro{\childdocredirect}
% The deprecated macro |\childdocredirect| is a legacy version
% of |\childdocforward| and |\childdocforwardprefix|:
%    \begin{macrocode}
\newcommand{\childdocredirect}[2][]
{
  \begingroup
    \if?#1?
      \def\childdoctmp{\childdocforward{#2}}
    \else
      \def\childdoctmp{\childdocforwardprefix{#1}{#2}}
    \fi
    \expandafter
  \endgroup
  \childdoctmp
}
%    \end{macrocode}

%\iffalse
%</package>
%\fi
%
\endinput

\childdocforwardprefix[cdocsamp]{cdocsfn}{cdocsch}
%    \end{macrocode}

%\iffalse
%</samplefinal>
%\fi
%
% %%%%%%%%%%%%%%%%%%%%%%%%%%%%%%%%%%%%%%
% \paragraph{Command Line Processing.}
%
% The following three command lines generate the output files
% |cdocscld|, |cdocscl1| and |cdocscl2|
% which should be identical to
% |cdocsdrf|, |cdocsch1| and |cdocsfn2|, respectively:
% \begin{center}
% \begin{tabular}{l}
% |latex -jobname cdocscld \|\\
% |  "\def\version{draft}% \iffalse
%
% childdoc.dtx Copyright (C) 2017-2018 Niklas Beisert
%
% This work may be distributed and/or modified under the
% conditions of the LaTeX Project Public License, either version 1.3
% of this license or (at your option) any later version.
% The latest version of this license is in
%   http://www.latex-project.org/lppl.txt
% and version 1.3 or later is part of all distributions of LaTeX
% version 2005/12/01 or later.
%
% This work has the LPPL maintenance status `maintained'.
%
% The Current Maintainer of this work is Niklas Beisert.
%
% This work consists of the files childdoc.dtx and childdoc.ins
% and the derived files childdoc.def and cdocsamp.tex with
% cdocsch1.tex, cdocsch2.tex, cdocsdrf.tex, cdocsfn1.tex, cdocsfn2.tex.
%
%<package>\ifdefined\childdocmain\endinput\fi
%<package>\ProvidesFile{childdoc.def}[2018/12/30 v2.0 child document driver]
%<samplemain>\ProvidesFile{cdocsamp.tex}[2018/12/30 v2.0 sample for childdoc]
%<*driver>
%\ProvidesFile{childdoc.drv}[2018/12/30 v2.0 childdoc reference manual file]
\PassOptionsToClass{10pt,a4paper}{article}
\documentclass{ltxdoc}

\usepackage[margin=35mm]{geometry}
\usepackage{hyperref}
\usepackage{hyperxmp}
\usepackage[usenames]{color}

\hypersetup{colorlinks=true}
\hypersetup{pdfstartview=FitH}
\hypersetup{pdfpagemode=UseNone}
\hypersetup{pdfsource={}}
\hypersetup{pdflang={en-UK}}
\hypersetup{pdfcopyright={Copyright 2017-2018 Niklas Beisert.
  This work may be distributed and/or modified under the
  conditions of the LaTeX Project Public License, either version 1.3
  of this license or (at your option) any later version.}}
\hypersetup{pdflicenseurl={http://www.latex-project.org/lppl.txt}}
\hypersetup{pdfcontactaddress={ETH Zurich, ITP, HIT K,
  Wolfgang-Pauli-Strasse 27}}
\hypersetup{pdfcontactpostcode={8093}}
\hypersetup{pdfcontactcity={Zurich}}
\hypersetup{pdfcontactcountry={Switzerland}}
\hypersetup{pdfcontactemail={nbeisert@itp.phys.ethz.ch}}
\hypersetup{pdfcontacturl={http://people.phys.ethz.ch/\xmptilde nbeisert/}}

\newcommand{\secref}[1]{\hyperref[#1]{section \ref*{#1}}}

\parskip1ex
\parindent0pt
\let\olditemize\itemize
\def\itemize{\olditemize\parskip0pt}

\begin{document}

\title{The \textsf{childdoc} Package}
\hypersetup{pdftitle={The childdoc Package}}
\author{Niklas Beisert\\[2ex]
  Institut f\"ur Theoretische Physik\\
  Eidgen\"ossische Technische Hochschule Z\"urich\\
  Wolfgang-Pauli-Strasse 27, 8093 Z\"urich, Switzerland\\[1ex]
  \href{mailto:nbeisert@itp.phys.ethz.ch}
  {\texttt{nbeisert@itp.phys.ethz.ch}}}
\hypersetup{pdfauthor={Niklas Beisert}}
\hypersetup{pdfsubject={Manual for the LaTeX2e Package childdoc}}
\date{30 December 2018, \textsf{v2.0}}
\maketitle

\begin{abstract}\noindent
\textsf{childdoc} is a \LaTeXe{} package
that enables the direct compilation
of document sections included by |\include|
to individual files.
\end{abstract}

\begingroup
\parskip0ex
\tableofcontents
\endgroup

%%%%%%%%%%%%%%%%%%%%%%%%%%%%%%%%%%%%%%%%%%%%%%%%%%%%%%%%%%%%%%%%%%%%%%%%%%%%%%%%
%%%%%%%%%%%%%%%%%%%%%%%%%%%%%%%%%%%%%%%%%%%%%%%%%%%%%%%%%%%%%%%%%%%%%%%%%%%%%%%%
\section{Introduction}

\LaTeX{} provides a mechanism to structure a large document (such as a book)
into a main file and several child files (containing the chapters)
using the |\include| command.
This mechanism is beneficial for documents
which span hundreds of pages in order to
make the source file(s) more manageable.
Moreover, compilation can be restricted to
selected child files by means of the |\includeonly| command.
The latter feature can be used to reduce the compilation time while editing
(this was significantly more useful in the earlier days of \LaTeX{})
or to generate a smaller document which is easier to navigate.
Another application of |\includeonly| is to generate
documents consisting of selected parts of the complete document.

However, there are a few drawbacks of the plain |\include| mechanism:
\begin{itemize}
\item
The child files cannot be compiled on their own,
they can only be compiled via the main file.
A naive editing environment
(such as a text editor with an option
to have the current file processed by \LaTeX)
may require one to switch to the main file before compiling;
attempting to compile the child file produces errors.
\item
The main file must be modified (each time)
to adjust the |\includeonly| command
to the present needs. This easily leaves the main file in a messy state.
\item
The generated document will always carry the filename
of the main document. This is inconvenient if
several child files are to be compiled and
to be kept for distribution.
\end{itemize}

The present package provides a simple interface
to make child files individually compilable by \LaTeX{}.
Compiling a child file then has the same effect as compiling
the main file with an |\includeonly| command
to select the appropriate child.
Moreover the generated document will carry the name of the child
rather than the main file.
This resolves all three above issues.

This feature is meant to make the editing of books,
thesis documents and lecture notes somewhat more convenient.
However, the package can also be used efficiently for
composing a series of documents (such as exercise sheets)
which are typically distributed individually.
It then assists the author in generating the individual documents
(potentially in different versions)
as well as a document containing the collected series.
Another application is in developing style files
or other kinds of included material
where compilation of the style file could redirect
to a sample or test file.

%%%%%%%%%%%%%%%%%%%%%%%%%%%%%%%%%%%%%%%%%%%%%%%%%%%%%%%%%%%%%%%%%%%%%%%%%%%%%%%%
%%%%%%%%%%%%%%%%%%%%%%%%%%%%%%%%%%%%%%%%%%%%%%%%%%%%%%%%%%%%%%%%%%%%%%%%%%%%%%%%
\section{Usage}

First of all, the package \textsf{childdoc} is \emph{not} a standard
\LaTeXe{} |.sty| style file! Therefore it needs to be invoked in
a non-standard way.

%%%%%%%%%%%%%%%%%%%%%%%%%%%%%%%%%%%%%%%%%%%%%%%%%%%%%%%%%%%%%%%%%%%%%%%%%%%%%%%%
\subsection{Included Files}
\label{sec:include}

%%%%%%%%%%%%%%%%%%%%%%%%%%%%%%%%%%%%%%%%
\DescribeMacro{\childdocmain}
To use the package, add the commands
\begin{center}
\begin{tabular}{l}
|\input{childdoc.def}|\\
|\childdocmain{}|\\
\end{tabular}
\end{center}
at the very top of the main \LaTeX{} file,
in particular \emph{before} the |\documentclass| statement!
The argument of |\childdocmain| should be left empty
(but it must be present).

%%%%%%%%%%%%%%%%%%%%%%%%%%%%%%%%%%%%%%%%
\DescribeMacro{\childdocof}
Furthermore, add the commands
\begin{center}
\begin{tabular}{l}
|\input{childdoc.def}|\\
|\childdocof{|\textit{main}|}|\\
\end{tabular}
\end{center}
at the top of every child file \textit{child}
which is included by |\include{|\textit{child}|}|
from within the main file
(or at least for those files to be compiled individually).
The argument \textit{main} must be the filename of the main file.

There are a couple of
considerations in setting up the main and child documents:

%%%%%%%%%%%%%%%%%%%%%%%%%%%%%%%%%%%%%%%%
\paragraph{Restrictions.}

Please note the following restrictions:
\begin{itemize}
\item
|\childdocmain| must be called with one argument \textit{main}
to ensure compatibility with earlier version of the package.
It must either be empty (|\childdocmain{}|)
or precisely match the filename of the main file in which it is specified.
See \secref{sec:detection} for further information.
\item
The filename \textit{main} must be specified without the |.tex| extension.
\item
The filename \textit{main} is case sensitive
(even in case-insensitive file systems)
due to internal string comparison.
\item
The argument \textit{main} should be fully expanded, it cannot be a macro.
\item
Subdirectories and special characters should be avoided in filenames.
\item
The command |\childdocmain{|\textit{main}|}| must be followed by a whitespace.
It should not be followed immediately by another command
or by a comment mark `|%|'.
This is because the \TeX{} parser reads the token immediately following
the argument of |\childdocmain| and puts it
at the beginning of every child section;
however, a white\-space is ignored.
\end{itemize}

%%%%%%%%%%%%%%%%%%%%%%%%%%%%%%%%%%%%%%%%
\paragraph{Content of Main File.}

It is advisable to place all content in the child files included by |\include|.
Any output contained in the main file will appear in all child documents
unless suppressed manually;
it cannot be suppressed automatically by the |\includeonly| directive
and thus should normally be avoided.
A method to include some content in the main file
by means of conditional processing is described in \secref{sec:conditional}.

%%%%%%%%%%%%%%%%%%%%%%%%%%%%%%%%%%%%%%%%
\paragraph{Page Numbering.}

When only a part of the document is compiled,
the appropriate numbering of pages
(as well as other status parameters)
is determined from the |.aux| files.
The latter contain information from previous passes.
However this information needs to propagate through
all intermediate child documents.
Therefore the page numbering in child documents may well
be inconsistent until the complete document is compiled at least once.

A useful (if unconventional) way to always ensure a consistent
page numbering is to restart the numbering in each child document
and denote the pages by `\textit{child}|.|\textit{page}'
where \textit{child} represents the chapter/section number of the child file.
This can be achieved by the command
|\numberwithin{page}{|\textit{child}|}|
of the \textsf{amsmath} package
where \textit{child} can be |chapter| or |section|
depending on the chosen structuring.
Alternatively, one can modify the macro |\thepage| appropriately
and reset the counter |page| at the start of each child file.

%%%%%%%%%%%%%%%%%%%%%%%%%%%%%%%%%%%%%%%%%%%%%%%%%%%%%%%%%%%%%%%%%%%%%%%%%%%%%%%%
\subsection{Conditional Processing}
\label{sec:conditional}

The package provides a mechanism to compile different versions
of a document. To customise the versions further some conditional processing
can come in handy to distinguish which version is being compiled.
The package provides two macros to describe the compilation context:

%%%%%%%%%%%%%%%%%%%%%%%%%%%%%%%%%%%%%%%%
\DescribeMacro{\ifchilddoc}
The conditional |\ifchilddoc| distinguishes between the compilation of
child documents and the main document:
%
\begin{center}
|\ifchilddoc |\textit{child-code}| |[|\||else |\textit{main-code}]| \||fi|
\end{center}

%%%%%%%%%%%%%%%%%%%%%%%%%%%%%%%%%%%%%%%%
\DescribeMacro{\childdocname}
\DescribeMacro{\childdocjob}
The macro |\childdocname| contains the filename (without extension)
of the main or child file being processed.
Note that |\childdocjob| will always contain the name of the main file.

%%%%%%%%%%%%%%%%%%%%%%%%%%%%%%%%%%%%%%%%
\paragraph{Title Page.}

Conditional processing can be used to include a title or banner page
in the main document when proper precautions are taken.
Importantly, the code in the main file should ensure that the page counter
(as well as other status parameters which are stored in the |.aux| files)
takes the same value after the conditional processing.
Otherwise the page numbers may take divergent values
depending on which part is compiled.

For example, a title page could be declared by:
%
\begin{center}
\begin{tabular}{l}
|\ifchilddoc\||else|\\
|\addtocounter{page}{-1}|\\
\textit{code for title page}\\
|\newpage|\\
|\||fi|
\end{tabular}
\end{center}
%
A banner page for the child documents can be generated by:
%
\begin{center}
\begin{tabular}{l}
|\ifchilddoc|\\
|\addtocounter{page}{-1}|\\
\textit{code for banner page}\\
|\newpage|\\
|\||fi|
\end{tabular}
\end{center}
%
Here one could write a message such as:
\begin{center}
|This is the part \childdocname{} of \childdocjob{}.|
\end{center}

%%%%%%%%%%%%%%%%%%%%%%%%%%%%%%%%%%%%%%%%%%%%%%%%%%%%%%%%%%%%%%%%%%%%%%%%%%%%%%%%
\subsection{Flags}
\label{sec:flags}

The package makes it easy to generate different versions
of the main or child documents.
To this end compilation flags can be defined
and assigned different default values.
They will be particularly useful in conjunction
with the forwarding mechanism described in \secref{sec:forward}.

For example, it may be useful to have a flag |\version|
which can be set to |draft| or |final|.
The document source will contain some conditional code
depending on the value of |\version|.
Suppose further, the flag should default to |final| for the main file
and to |draft| for child files
which is a natural assignment for editing the document.
This is achieved by placing the following code
in the preamble of the main document
(below the |\childdocmain| directive):
%
\begin{center}
\begin{tabular}{l}
|\ifchilddoc|\\
|\providecommand{\version}{draft}|\\
|\||else|\\
|\providecommand{\version}{final}|\\
|\||fi|
\end{tabular}
\end{center}
%
The definition by |\providecommand| makes sure
that previous definitions are not overwritten.
Further statements |\providecommand{\version}{...}|
can thus be added before the above code to override it.

For the main file, one might add a line
(between |\childdocmain| and the above block)
%
\begin{center}
|%\ifchilddoc\||else\providecommand{\version}{draft}\||fi|
\end{center}
%
which can be uncommented to produce a draft version.
Likewise one can add a line to the very top of a child file
(above the |\childdocof{|\textit{main}|}| directive)
%
\begin{center}
|%\providecommand{\version}{final}|
\end{center}
%
which can be uncommented to produce the final version of this child document.

%%%%%%%%%%%%%%%%%%%%%%%%%%%%%%%%%%%%%%%%%%%%%%%%%%%%%%%%%%%%%%%%%%%%%%%%%%%%%%%%
\subsection{Forwarding}
\label{sec:forward}

Different versions of the main or child documents
using compilation flags as described in \secref{sec:flags}
can be (permanently) stored in different files
for convenient compilation, viewing and distribution.
To this end, the package defines a command
to pass on compilation to a different file:

%%%%%%%%%%%%%%%%%%%%%%%%%%%%%%%%%%%%%%%%
\DescribeMacro{\childdocforward}
The command |\childdocforward| redirects processing to
another source file:
%
\begin{center}
\begin{tabular}{l}
|\input{childdoc.def}|\\
|\childdocforward[|\textit{main}|]{|\textit{dest}|}|\\
\end{tabular}
\end{center}
%
The argument \textit{dest} is the destination file
(without extension).
It should be the main file or one of the child files.
Note that further \textsf{childdoc} directives
such as |\childdocof| and |\childdocforward|
in the indicated file will be processed in this form.
The optional argument \textit{main}
passes on directly to the main file \textit{main}
while pretending to compile the child \textit{dest}.
This form behaves as if \textit{dest}
issues |\childdocof{|\textit{main}|}| right away,
and no further \textsf{childdoc} directives will be processed.

%%%%%%%%%%%%%%%%%%%%%%%%%%%%%%%%%%%%%%%%
\DescribeMacro{\...prefix}
In the alternative form |\childdocforwardprefix|,
%
\begin{center}
\begin{tabular}{l}
|\input{childdoc.def}|\\
|\childdocforwardprefix[|\textit{main}|]{|\textit{prefix}|}{|\textit{dest}|}|
\end{tabular}
\end{center}
%
the destination file is determined by a pattern
depending on the current file:
To make this work, the current file must be called
`{\textit{prefix}\hspace{0.2em}\textit{suffix}}'
with \textit{prefix} matching precisely the argument.
Processing is then passed on to the file
`{\textit{dest}\hspace{0.2em}\textit{suffix}}'.
Surely, the same effect is achieved by
directly specifying the
argument `{\textit{dest}\hspace{0.2em}\textit{suffix}}'
in the first form.
However, that requires to set up a different file
for each child. With the alternative form of the command
all these files can have exactly the same content
which simplifies setting them up and maintaining them.

For example, the following file |draft.tex|
with a compilation flag |\version| as described in \secref{sec:flags}
compiles the main document as a draft:
%
\begin{center}
\begin{tabular}{l}
|\def\version{draft}|\\
|\input{childdoc.def}|\\
|\childdocforward{|\textit{main}|}|
\end{tabular}
\end{center}
%
Likewise, the following files |final|\textit{nn}|.tex|
compile the final version of the child document
|child|\textit{nn}|.tex|:
%
\begin{center}
\begin{tabular}{l}
|\def\version{final}|\\
|\input{childdoc.def}|\\
|\childdocforwardprefix{final}{child}|
\end{tabular}
\end{center}
%

Note that when several versions of a main file and/or of each child file
are to be generated, it may be convenient to set up a |Makefile| or
shell script to automatise the process.

%%%%%%%%%%%%%%%%%%%%%%%%%%%%%%%%%%%%%%%%%%%%%%%%%%%%%%%%%%%%%%%%%%%%%%%%%%%%%%%%
\subsection{Command Line Processing}
\label{sec:commandline}

The effect of redirection files can also be achieved by invoking
the \LaTeX{} compiler with a more elaborate command line.
Most conveniently this should be done as part
of a shell script or a |Makefile|.

When using \textsf{childdoc} in the main file, the following
command lines effectively perform a redirection
(note that depending on the shell being used,
backslashes may have to be doubled: `|\|' $\to$ `|\\|'):
%
\begin{center}
|... -jobname "|\textit{target}|" |\\|"|[\textit{flags}]%
|\input{childdoc.def}\childdocforward[|\textit{main}|]{|\textit{dest}|}"|
\end{center}
%
Here \textit{target} is the name of the output file,
\textit{main} is the name of the main file
and \textit{dest} is the name of the main or child file to be processed
(all filenames without extensions).
The optional argument \textit{main} can be omitted
if \textit{main} matches \textit{dest}.
Optionally, compilation \textit{flags} can be defined via |\def| commands.
This command line makes the \TeX{} engine believe
it is compiling the file \textit{target}
whose content is specified as the latter parameter.
The provided code then forwards the processing to
\textit{main} or \textit{dest} as described in \secref{sec:forward}.

%%%%%%%%%%%%%%%%%%%%%%%%%%%%%%%%%%%%%%%%%%%%%%%%%%%%%%%%%%%%%%%%%%%%%%%%%%%%%%%%
\subsection{Include by Input}
\label{sec:input}

Including child documents by |\include| has some restrictions by design.
Most notably, the content of a child document always occupies
its own set of pages; pages cannot be shared between child documents.
Usually, this behaviour makes perfect sense
because each child document contain an essential part of the document.
However, in some situations it may be desirable to compose
a document from a collection of parts
without having mandatory page breaks between then.
For this case, the package
provides a mechanism to include parts
by |\input| which can also be processed individually.
However, by construction this mechanism
requires manual handling of the content to be output.

%%%%%%%%%%%%%%%%%%%%%%%%%%%%%%%%%%%%%%%%
\DescribeMacro{\ifchilddocmanual}
The main file should be prepared as usual, see \secref{sec:include}.
However, the document body must make a distinction
between processing of an individual part and of the main document, e.g.:
%
\begin{center}
\begin{tabular}{l}
|\ifchilddocmanual|\\
|\input{\childdocname}|\\
|\||else|\\
\textit{document body with }|\input{|\textit{part}|}|\\
|\||fi|
\end{tabular}
\end{center}
%
The conditional |\ifchilddocmanual| is true whenever
a part to be included by |\input| is being compiled,
and the name of the part is stored in |\childdocname|.

%%%%%%%%%%%%%%%%%%%%%%%%%%%%%%%%%%%%%%%%
\DescribeMacro{\childdocby}
Each part to be included by |\input| should start with:
%
\begin{center}
\begin{tabular}{l}
|\input{childdoc.def}|\\
|\childdocby{|\textit{main}|}|\\
\end{tabular}
\end{center}
%
The directive |\childdocby| is similar to |\childdocof|
described in \secref{sec:include},
but the subsequent selection of content must be done manually.
To that end, both |\ifchilddoc| and |\ifchilddocmanual|
will be true upon processing of a part,
and the name of the part is stored in |\childdocname|.
Note that |\jobname| will be set to the filename of the current part
so that each part receives an individual |.aux| file
that does not interfere with the |.aux| file(s) of the main document.
This behaviour can be altered by the alternative form
|\childdocby[*]{|\textit{main}|}| (with a non-empty optional argument)
which uses the |.aux| file of the main document
by setting |\jobname| to \textit{main}.

%%%%%%%%%%%%%%%%%%%%%%%%%%%%%%%%%%%%%%%%%%%%%%%%%%%%%%%%%%%%%%%%%%%%%%%%%%%%%%%%
\subsection{Driver Development}
\label{sec:driver}

The \textsf{childdoc} mechanism can also be use for the development
of definition files such as \LaTeX{} styles or classes.
This case differs from the above setup with multiple parts
included by |\include| in that no |\includeonly| should be invoked.
This can be achieved by starting the include file
(before |\ProvidesPackage|) with:
%
\begin{center}
\begin{tabular}{l}
|\input{childdoc.def}|\\
|\childdocforward{|\textit{main}|}|\\
\end{tabular}
\end{center}
%
or alternatively with:
%
\begin{center}
\begin{tabular}{l}
|\input{childdoc.def}|\\
|\childdocby{|\textit{main}|}|\\
\end{tabular}
\end{center}
%
Both forms have slightly different effects as described above.
The main file is prepared as usual, see \secref{sec:include}.

%%%%%%%%%%%%%%%%%%%%%%%%%%%%%%%%%%%%%%%%%%%%%%%%%%%%%%%%%%%%%%%%%%%%%%%%%%%%%%%%
\subsection{Legacy Detection}
\label{sec:detection}

The directive |\childdocmain| in the main file can detect
whether the complete document or merely a child is to be compiled
even without using the directive |\childdocof|.
This method is deprecated because it is less robust
and there is no compelling reason to use it;
it is merely provided for backward compatibility
and it may be removed in future versions.

If the detection mechanism is to be used,
it is mandatory to correctly specify
the filename of the main file as the argument of |\childdocmain|:
%
\begin{center}
\begin{tabular}{l}
|\input{childdoc.def}|\\
|\childdocmain{|\textit{main}|}|\\
\end{tabular}
\end{center}
%
If |\jobname| does not match the argument \textit{main} of |\childdocmain|,
it is assumed that |\jobname| points to the child file to be compiled.
When using |\childdocmain| with the main file specified as argument,
it suffices to start a child file
with just |\input{|\textit{main}|}|
without loading of the package and using |\childdocof|.
If instead all processing is done
with the appropriate \textsf{childdoc} directives,
the argument of \textit{main} of |\childdocmain| can be empty.

An alternative version of the command line processing described
in \secref{sec:commandline} using the detection mechanism reads:
%
\begin{center}
|... -jobname "|\textit{target}|" "|[\textit{flags}]%
[|\def\jobname{|\textit{dest}|}|]|\input{|\textit{main}|}"|
\end{center}

%%%%%%%%%%%%%%%%%%%%%%%%%%%%%%%%%%%%%%%%%%%%%%%%%%%%%%%%%%%%%%%%%%%%%%%%%%%%%%%%
\subsection{Manual Code}
\label{sec:manual}

In case one cannot be certain whether the definitions file |childdoc.def|
is installed on the target \TeX{} distribution
and one prefers not to ship it,
it is conceivable to paste a few relevant commands into the sources.

To that end, drop all statements |\input{childdoc.def}|
and perform the replacements as outlined below.
Instead of |\childdocmain{|\textit{main}|}| add the following code
to the top of the main file:
%
\begin{center}
\begin{tabular}{l}
|\||ifdefined\childdocname\endinput\||fi\newif\ifchilddoc|\\
|\edef\childdocname{\scantokens\expandafter{\jobname\noexpand}}|\\
|\def\childdocmain{|\textit{main}|}\||ifx\childdocmain\childdocname\||else|\\
|\childdoctrue\includeonly{\childdocname}\let\jobname\childdocmain\||fi|\\
\end{tabular}
\end{center}
%
Instead of |\childdocof{|\textit{main}|}| just include the main file
at the top of each child file:
%
\begin{center}
|\input{|\textit{main}|}|
\end{center}
%
A simple redirection |\childdocforward{|\textit{dest}|}| is achieved by:
%
\begin{center}
|\def\jobname{|\textit{dest}|}\input{\jobname}|
\end{center}
%
The redirection with prefix
|\childdocforwardprefix[|\textit{prefix}|]{|\textit{dest}|}|
is accomplished by:
%
\begin{center}
\begin{tabular}{l}
|{\edef\jobname{\scantokens\expandafter{\jobname\noexpand}}|\\
|\def\redirectjob |\textit{prefix}|#1~~~{\gdef\jobname{|\textit{dest}|#1}}|\\
|\expandafter\redirectjob\jobname~~~}\input{\jobname}|
\end{tabular}
\end{center}

In an alternative approach,
child documents can be compiled by a specific command line
without additional code or specific definitions:
%
\begin{center}
|... -jobname "|\textit{target}|" "|[\textit{flags}]%
|\includeonly{|\textit{dest}|}\input{|\textit{main}|}"|
\end{center}
%

%%%%%%%%%%%%%%%%%%%%%%%%%%%%%%%%%%%%%%%%%%%%%%%%%%%%%%%%%%%%%%%%%%%%%%%%%%%%%%%%
%%%%%%%%%%%%%%%%%%%%%%%%%%%%%%%%%%%%%%%%%%%%%%%%%%%%%%%%%%%%%%%%%%%%%%%%%%%%%%%%
\section{Information}

%%%%%%%%%%%%%%%%%%%%%%%%%%%%%%%%%%%%%%%%%%%%%%%%%%%%%%%%%%%%%%%%%%%%%%%%%%%%%%%%
\subsection{Copyright}

Copyright \copyright{} 2017--2018 Niklas Beisert

This work may be distributed and/or modified under the
conditions of the \LaTeX{} Project Public License, either version 1.3
of this license or (at your option) any later version.
The latest version of this license is in
  \url{http://www.latex-project.org/lppl.txt}
and version 1.3 or later is part of all distributions of \LaTeX{}
version 2005/12/01 or later.

This work has the LPPL maintenance status `maintained'.

The Current Maintainer of this work is Niklas Beisert.

This work consists of the files |README.txt|, |childdoc.ins| and |childdoc.dtx|
as well as the derived files |childdoc.def|, |cdocsamp.tex|
with |cdocsch1.tex|, |cdocsch2.tex|, |cdocspt3.tex|, |cdocspt4.tex|,
|cdocsdrf.tex|, |cdocsfn1.tex|, |cdocsfn2.tex|
as well as |childdoc.pdf|.

%%%%%%%%%%%%%%%%%%%%%%%%%%%%%%%%%%%%%%%%%%%%%%%%%%%%%%%%%%%%%%%%%%%%%%%%%%%%%%%%
\subsection{Files and Installation}

The package consists of the files:
%
\begin{center}
\begin{tabular}{ll}
    |README.txt|   & readme file \\
    |childdoc.ins| & installation file \\
    |childdoc.dtx| & source file \\
    |childdoc.def| & definition file \\
    |cdocsamp.tex| & sample main file \\
    |cdocsch1.tex| & sample include file \\
    |cdocsch2.tex| & sample include file \\
    |cdocspt3.tex| & sample part file \\
    |cdocspt4.tex| & sample part file \\
    |cdocsdrf.tex| & sample redirection file \\
    |cdocsfn1.tex| & sample redirection file \\
    |cdocsfn2.tex| & sample redirection file \\
    |childdoc.pdf| & manual
\end{tabular}
\end{center}
%
The distribution consists of the files
|README.txt|, |childdoc.ins| and |childdoc.dtx|.
%
\begin{itemize}
\item
Run (pdf)\LaTeX{} on |childdoc.dtx|
to compile the manual |childdoc.pdf| (this file).
\item
Run \LaTeX{} on |childdoc.ins| to create the definitions file |childdoc.def|
and the sample |cdocsamp.tex| with include files
|cdocsch1.tex|, |cdocsch2.tex|, |cdocspt3.tex|, |cdocspt4.tex|,
|cdocsdrf.tex|, |cdocsfn1.tex|, |cdocsfn2.tex|.
Then copy the file |childdoc.def| to an appropriate directory of your \LaTeX{}
distribution, e.g.\ \textit{texmf-root}|/tex/latex/childdoc|.
\end{itemize}

%%%%%%%%%%%%%%%%%%%%%%%%%%%%%%%%%%%%%%%%%%%%%%%%%%%%%%%%%%%%%%%%%%%%%%%%%%%%%%%%
\subsection{Related CTAN Packages}

There are several other packages which offer a similar functionality:
%
\begin{itemize}
\item
The packages
\href{http://ctan.org/pkg/docmute}{\textsf{docmute}},
\href{http://ctan.org/pkg/includex}{\textsf{includex}} and
\href{http://ctan.org/pkg/standalone}{\textsf{standalone}}
provide commands to include only the document body of
a child file thus allowing both files to be compiled individually.
\item
The packages \href{http://ctan.org/pkg/subdocs}{\textsf{subdocs}}
and \href{http://ctan.org/pkg/subfiles}{\textsf{subfiles}}
provide structures in which the main and child documents can be
encapsulated and allowing them to be compiled individually.
The inclusion mechanism is different from the conventional |\include|.
\item
The package \href{http://ctan.org/pkg/combine}{\textsf{combine}}
is an elaborate solution to combine several documents into one.
\end{itemize}
%
See also the CTAN topic \href{http://ctan.org/topic/subdocs}{\textsf{subdocs}}
for further related packages.
The present package differs from the above solutions in that
a document structure constructed with the conventional |\include| mechanism
just needs two extra commands at the top of every file
such that all constituent files can be compiled individually.

%%%%%%%%%%%%%%%%%%%%%%%%%%%%%%%%%%%%%%%%%%%%%%%%%%%%%%%%%%%%%%%%%%%%%%%%%%%%%%%%
%\subsection{Feature Suggestions}
%
%The following is a list of features which may be useful for future
%versions of this package:
%%
%\begin{itemize}
%\item
%\ldots
%\end{itemize}

%%%%%%%%%%%%%%%%%%%%%%%%%%%%%%%%%%%%%%%%%%%%%%%%%%%%%%%%%%%%%%%%%%%%%%%%%%%%%%%%
\subsection{Revision History}

%%%%%%%%%%%%%%%%%%%%%%%%%%%%%%%%%%%%%%%%
\paragraph{v2.0:} 2018/12/30

\begin{itemize}
\item
immediate forward processing
\item
added |\childdocby| mechanism
\item
manual restructured
\end{itemize}

%%%%%%%%%%%%%%%%%%%%%%%%%%%%%%%%%%%%%%%%
\paragraph{v1.6:} 2018/01/17

\begin{itemize}
\item
application for development of include files
\item
corrections to manual
\end{itemize}

%%%%%%%%%%%%%%%%%%%%%%%%%%%%%%%%%%%%%%%%
\paragraph{v1.5:} 2017/05/21

\begin{itemize}
\item
more complete structuring introduced
\item
|\childdocof| introduced
\item
|\childdoc| renamed to |\childdocmain|
\item
|\childredirect| renamed to |\childdocforward| and |\childdocforwardprefix|
and functionality expanded
\end{itemize}

%%%%%%%%%%%%%%%%%%%%%%%%%%%%%%%%%%%%%%%%
\paragraph{v1.0:} 2017/04/27

\begin{itemize}
\item
manual and install package
\item
first version published on CTAN
\end{itemize}

%%%%%%%%%%%%%%%%%%%%%%%%%%%%%%%%%%%%%%%%
\paragraph{v0.6:} 2017/04/26

\begin{itemize}
\item
redirection mechanism added
\end{itemize}

%%%%%%%%%%%%%%%%%%%%%%%%%%%%%%%%%%%%%%%%
\paragraph{v0.5:} 2017/04/26

\begin{itemize}
\item
functionality in definition file
\end{itemize}


%%%%%%%%%%%%%%%%%%%%%%%%%%%%%%%%%%%%%%%%%%%%%%%%%%%%%%%%%%%%%%%%%%%%%%%%%%%%%%%%
%%%%%%%%%%%%%%%%%%%%%%%%%%%%%%%%%%%%%%%%%%%%%%%%%%%%%%%%%%%%%%%%%%%%%%%%%%%%%%%%
%%%%%%%%%%%%%%%%%%%%%%%%%%%%%%%%%%%%%%%%%%%%%%%%%%%%%%%%%%%%%%%%%%%%%%%%%%%%%%%%
\appendix

\settowidth\MacroIndent{\rmfamily\scriptsize 000\ }

 \DocInput{childdoc.dtx}

\end{document}
%</driver>
% \fi
%
% %%%%%%%%%%%%%%%%%%%%%%%%%%%%%%%%%%%%%%%%%%%%%%%%%%%%%%%%%%%%%%%%%%%%%%%%%%%%%%
% %%%%%%%%%%%%%%%%%%%%%%%%%%%%%%%%%%%%%%%%%%%%%%%%%%%%%%%%%%%%%%%%%%%%%%%%%%%%%%
% \section{Sample}
%\iffalse
%<*samplemain>
%\fi
%
% The following presents a sample document
% with two chapters, two parts, a title page,
% a compile flag as well as three forwarding files to set the flag.
% It consists of eight |.tex| files:
% \begin{center}
% \begin{tabular}{ll}
% |cdocsamp.tex|&main file\\
% |cdocsch1.tex|&include file for chapter 1\\
% |cdocsch2.tex|&include file for chapter 2\\
% |cdocspt3.tex|&include file for part 3\\
% |cdocspt4.tex|&include file for part 4\\
% |cdocsdrf.tex|&forwarding file for main file in draft mode\\
% |cdocsfi1.tex|&forwarding file for final version of chapter 1\\
% |cdocsfi2.tex|&forwarding file for final version of chapter 2\\
% \end{tabular}
% \end{center}
% Each of the eight files can be compiled directly by the \LaTeX{} compiler.
%
% %%%%%%%%%%%%%%%%%%%%%%%%%%%%%%%%%%%%%%
% \paragraph{Main File.}
%
% The main file is called |cdocsamp.tex|.
%
% Load the \textsf{childdoc} definitions and
% declare the filename for the main document:
%    \begin{macrocode}
\input{childdoc.def}
\childdocmain{}
%    \end{macrocode}

% Optional override for |\version| flag:
%    \begin{macrocode}
%%\ifchilddoc\else\providecommand{\version}{draft}\fi
%    \end{macrocode}

% Define the default values for the |\version| flag
% (|final| for the main file and |draft| for childs):
%    \begin{macrocode}
\ifchilddoc
\providecommand{\version}{draft}
\else
\providecommand{\version}{final}
\fi
%    \end{macrocode}

% Load the standard document class:
%    \begin{macrocode}
\documentclass[12pt]{article}
%    \end{macrocode}

% Start the document body:
%    \begin{macrocode}
\begin{document}
%    \end{macrocode}

% Declare a title page.
% Print title, part of document being processed and version flag:
%    \begin{macrocode}
\addtocounter{page}{-1}
\begin{center}
{\LARGE\bfseries{}childdoc example\par}
\vspace{1cm}
\ifchilddoc
\ifchilddocmanual part\else chapter\fi:
`\childdocname' of `\childdocjob'\par
\else
main document: `\childdocjob'\par
\fi
version: \version\par
\end{center}
\newpage
%    \end{macrocode}

% Manually include selected file,
% otherwise process as usual:
%    \begin{macrocode}
\ifchilddocmanual
\section*{part `\childdocname'}
\input{\childdocname}
\else
%    \end{macrocode}

% Include the two chapters:
%    \begin{macrocode}
\include{cdocsch1}
\include{cdocsch2}
%    \end{macrocode}

% Include the two parts unless only chapters should be displayed:
%    \begin{macrocode}
\ifchilddoc\else
\section{part three}
\input{cdocspt3}
\section{part four}
\input{cdocspt4}
\fi
%    \end{macrocode}

% Process as usual until here:
%    \begin{macrocode}
\fi
%    \end{macrocode}

% End of document body:
%    \begin{macrocode}
\end{document}
%    \end{macrocode}
%\iffalse
%</samplemain>
%\fi
%
% %%%%%%%%%%%%%%%%%%%%%%%%%%%%%%%%%%%%%%
% \paragraph{Chapter Include Files.}
%
% The include files are called |cdocsch1.tex| and |cdocsch2.tex|.
%
%\iffalse
%<*samplechap1|samplechap2>
%\fi

% Optional override for |\version| flag:
%    \begin{macrocode}
%%\providecommand{\version}{final}
%    \end{macrocode}

% Include the main document:
%    \begin{macrocode}
\input{childdoc.def}
\childdocof{cdocsamp}
%    \end{macrocode}

%\iffalse
%</samplechap1|samplechap2>
%\fi
%
%\iffalse
%<*samplechap1>
%\fi
% Some text for chapter 1:
%    \begin{macrocode}
\section{one}
some text in chapter one
%    \end{macrocode}

%\iffalse
%</samplechap1>
%\fi
% Some text for chapter 2:
%\iffalse
%<*samplechap2>
%\fi
%    \begin{macrocode}
\section{two}
more text in chapter two
%    \end{macrocode}

%\iffalse
%</samplechap2>
%\fi
%
% %%%%%%%%%%%%%%%%%%%%%%%%%%%%%%%%%%%%%%
% \paragraph{Part Include Files.}
%
% The include files are called |cdocspt3.tex| and |cdocspt4.tex|.
%
%\iffalse
%<*samplepart3|samplepart4>
%\fi

% Optional override for |\version| flag:
%    \begin{macrocode}
%%\providecommand{\version}{final}
%    \end{macrocode}

% Include the main document:
%    \begin{macrocode}
\input{childdoc.def}
\childdocby{cdocsamp}
%    \end{macrocode}

%\iffalse
%</samplepart3|samplepart4>
%\fi
%
%\iffalse
%<*samplepart3>
%\fi
% Some text for part 3:
%    \begin{macrocode}
some text in part three
%    \end{macrocode}

%\iffalse
%</samplepart3>
%\fi
% Some text for part 4:
%\iffalse
%<*samplepart4>
%\fi
%    \begin{macrocode}
more text in part four
%    \end{macrocode}

%\iffalse
%</samplepart4>
%\fi
%
% %%%%%%%%%%%%%%%%%%%%%%%%%%%%%%%%%%%%%%
% \paragraph{Forwarding for a Complete Draft.}
%
% The following forwarding file |cdocsdrf.tex|
% compiles the main document in draft mode:
%\iffalse
%<*sampledraft>
%\fi
%    \begin{macrocode}
\def\version{draft}
\input{childdoc.def}
\childdocforward{cdocsamp}
%    \end{macrocode}

%\iffalse
%</sampledraft>
%\fi
%
% %%%%%%%%%%%%%%%%%%%%%%%%%%%%%%%%%%%%%%
% \paragraph{Forwarding for Final Version of the Chapters.}
%
% The following forwarding files |cdocsfn1.tex| and |cdocsfn2.tex|
% (with identical content)
% compile the final versions of the child documents
% |cdocsch1.tex| and |cdocsch2.tex|, respectively:
%\iffalse
%<*samplefinal>
%\fi
%    \begin{macrocode}
\def\version{final}
\input{childdoc.def}
\childdocforwardprefix[cdocsamp]{cdocsfn}{cdocsch}
%    \end{macrocode}

%\iffalse
%</samplefinal>
%\fi
%
% %%%%%%%%%%%%%%%%%%%%%%%%%%%%%%%%%%%%%%
% \paragraph{Command Line Processing.}
%
% The following three command lines generate the output files
% |cdocscld|, |cdocscl1| and |cdocscl2|
% which should be identical to
% |cdocsdrf|, |cdocsch1| and |cdocsfn2|, respectively:
% \begin{center}
% \begin{tabular}{l}
% |latex -jobname cdocscld \|\\
% |  "\def\version{draft}\input{childdoc.def}\childdocforward{cdocsamp}"|\\
% |latex -jobname cdocscl1 \|\\
% |  "\input{childdoc.def}\childdocforward[cdocsamp]{cdocsch1}"|\\
% |latex -jobname cdocscl2 \|\\
% |  "\def\version{final}\input{childdoc.def}\childdocforward{cdocsch2}"|
% \end{tabular}
% \end{center}
% Note that the trailing backslash on each first line
% merely continues the input to the second line
% (for convenient cut ant paste).
% Furthermore, the command |latex| can be replaced by any
% of its alternative versions such as |pdflatex|.
%
% %%%%%%%%%%%%%%%%%%%%%%%%%%%%%%%%%%%%%%%%%%%%%%%%%%%%%%%%%%%%%%%%%%%%%%%%%%%%%%
% %%%%%%%%%%%%%%%%%%%%%%%%%%%%%%%%%%%%%%%%%%%%%%%%%%%%%%%%%%%%%%%%%%%%%%%%%%%%%%
% \section{Implementation}
%\iffalse
%<*package>
%\fi
%
% This section describes the definitions file |childdoc.def|.

% The definitions cannot be loaded using |\usepackage| or |\RequirePackage|
% which has a mechanism to prevent loading a style file more than once.
% When loading the definitions by means of |\input|
% multiple instances have to be prevented manually:
%\iffalse
%This code needs to be before the `\ProvidesFile' directive
%which is defined at the beginning of this file.
%Therefore it is also placed there and commented out here.
%</package>
%<*discard>
%\fi
%    \begin{macrocode}
\ifdefined\childdocmain\endinput\fi
%    \end{macrocode}
%\iffalse
%</discard>
%<*package>
%\fi
%
% \macro{\ifchilddoc}
% \macro{\ifchilddocmanual}
% The conditional |\ifchilddoc| tells whether a
% child (true) or main (false) document is being compiled.
% The conditional |\ifchilddocmanual| tells whether
% the |\includeonly| mechanism is used (false) or
% the selection of child files must be performed manually (true).
% The definitions initialise to false:
%    \begin{macrocode}
\newif\ifchilddoc
\newif\ifchilddocmanual
%    \end{macrocode}

% \macro{\childdocname}
% \macro{\childdocjob}
% The macro |\childdocname| stores the name of the main document
% to be compiled. The macro |\childdocjob| stores the name of
% the document on which the \LaTeX{} compiler was originally invoked.
% The content of |\jobname| cannot be compared
% to filenames specified in the source due to different catcodes.
% The following code rescans |\jobname|, stores the result
% in |\childdocname| and saves a copy in |\childdocjob|:
%    \begin{macrocode}
\edef\childdocname{\scantokens\expandafter{\jobname\noexpand}}
\let\childdocjob\childdocname
%    \end{macrocode}

% \macro{\childdocdisable}
% The macro |\childdocdisable| prevents the main file
% from being processed more than once.
% At this stage, the main document command |\childdocmain|
% is assumed to be called once again where it should do nothing.
% Any subsequent call to it should prevent
% a secondary processing of the main document
% It overwrites the forwarding commands
% |\childdocof| and |\childdocforward|
% with empty macros to prevent further inclusions of the main document:
%    \begin{macrocode}
\newcommand{\childdocdisable}
{
  \renewcommand{\childdocmain}[1]{\renewcommand{\childdocmain}[1]{\endinput}}
  \renewcommand{\childdocof}[1]{}
  \renewcommand{\childdocby}[2][]{}
  \renewcommand{\childdocforward}[2][]{}
  \renewcommand{\childdocdisable}{}
}
%    \end{macrocode}

% \macro{\childdocmain}
% The macro |\childdocmain| is to be called at the top of the main file
% with nothing or the main filename (without extension) as argument.
% First, it breaks loops.
% If the argument is not empty and does not match |\childdocname|
% (which is set by the first inclusion of |childdoc.def|),
% |\ifchilddoc| is set to true, |\includeonly| is applied to the child file
% and |\jobname| is set to the main file
% (for proper handling of |.aux| files):
%    \begin{macrocode}
\newcommand{\childdocmain}[1]
{
  \childdocdisable\childdocmain{}
  \if?#1?\else
    \begingroup
      \def\childdoctmp{#1}
      \ifx\childdoctmp\childdocname
        \def\childdoctmp{}
      \else
        \def\childdoctmp
        {
          \childdoctrue
          \includeonly{\childdocname}
          \def\childdocjob{#1}
          \def\jobname{#1}
        }
      \fi
      \expandafter
    \endgroup
    \childdoctmp
  \fi
}
%    \end{macrocode}

% \macro{\childdocof}
% The command |\childdocof| redirects
% compilation to the main file |#1|.
%    \begin{macrocode}
\newcommand{\childdocof}[1]
{
  \childdocdisable
  \childdoctrue
  \includeonly{\childdocname}
  \def\jobname{#1}
  \def\childdocjob{#1}
  \input{#1}
}
%    \end{macrocode}

% \macro{\childdocby}
% The command |\childdocby| ....
%    \begin{macrocode}
\newcommand{\childdocby}[2][]
{
  \childdocdisable
  \childdoctrue
  \childdocmanualtrue
  \if?#1?\else
    \def\jobname{#2}
  \fi
  \def\childdocjob{#2}
  \input{#2}
  \endinput
}
%    \end{macrocode}

% \macro{\childdocforward}
% The command |\childdocforward| redirects
% compilation to the main file or
% (if the optional argument is given) a child file.
% Parameters are set as if the main file
% or a child file starting with |\childdocof| was compiled.
% Then compilation is handed over to the main file:
%    \begin{macrocode}
\newcommand{\childdocforward}[2][]
{
  \begingroup
    \if?#1?
      \def\childdoctmp
      {
        \def\childdocname{#2}
        \def\childdocjob{#2}
        \def\jobname{#2}
        \input{#2}
        \endinput
      }
    \else
      \def\childdoctmp
      {
        \childdocdisable
        \def\childdocname{#2}
        \childdoctrue
        \includeonly{#2}
        \def\childdocjob{#1}
        \def\jobname{#1}
        \input{#1}
        \endinput
      }
    \fi
    \expandafter
  \endgroup
  \childdoctmp
}
%    \end{macrocode}

% \macro{\childdocforwardprefix}
% The command |\childdocforwardprefix| redirects
% compilation to the main or a child file by means of a pattern.
% The prefix |#1| in the current filename is replaced by |#2|
% and the suffix of the current filename is kept
% (it is assumed that the filename does not contain the substring `|~~~|'
% which is used as a delimiter).
% Compilation is handed over to the new file by |\childdocforward|:
%    \begin{macrocode}
\newcommand{\childdocforwardprefix}[3][]
{
  \begingroup
    \def\childdocextract #2##1~~~{\def\childdoctmp{\childdocforward[#1]{#3##1}}}
    \expandafter\childdocextract\childdocname~~~
    \expandafter
  \endgroup
  \childdoctmp
}
%    \end{macrocode}

% \macro{\childdoc}
% The deprecated macro |\childdoc| is a legacy version of |\childdocmain|:
%    \begin{macrocode}
\newcommand{\childdoc}{\childdocmain}
%    \end{macrocode}

% \macro{\childdocredirect}
% The deprecated macro |\childdocredirect| is a legacy version
% of |\childdocforward| and |\childdocforwardprefix|:
%    \begin{macrocode}
\newcommand{\childdocredirect}[2][]
{
  \begingroup
    \if?#1?
      \def\childdoctmp{\childdocforward{#2}}
    \else
      \def\childdoctmp{\childdocforwardprefix{#1}{#2}}
    \fi
    \expandafter
  \endgroup
  \childdoctmp
}
%    \end{macrocode}

%\iffalse
%</package>
%\fi
%
\endinput
\childdocforward{cdocsamp}"|\\
% |latex -jobname cdocscl1 \|\\
% |  "% \iffalse
%
% childdoc.dtx Copyright (C) 2017-2018 Niklas Beisert
%
% This work may be distributed and/or modified under the
% conditions of the LaTeX Project Public License, either version 1.3
% of this license or (at your option) any later version.
% The latest version of this license is in
%   http://www.latex-project.org/lppl.txt
% and version 1.3 or later is part of all distributions of LaTeX
% version 2005/12/01 or later.
%
% This work has the LPPL maintenance status `maintained'.
%
% The Current Maintainer of this work is Niklas Beisert.
%
% This work consists of the files childdoc.dtx and childdoc.ins
% and the derived files childdoc.def and cdocsamp.tex with
% cdocsch1.tex, cdocsch2.tex, cdocsdrf.tex, cdocsfn1.tex, cdocsfn2.tex.
%
%<package>\ifdefined\childdocmain\endinput\fi
%<package>\ProvidesFile{childdoc.def}[2018/12/30 v2.0 child document driver]
%<samplemain>\ProvidesFile{cdocsamp.tex}[2018/12/30 v2.0 sample for childdoc]
%<*driver>
%\ProvidesFile{childdoc.drv}[2018/12/30 v2.0 childdoc reference manual file]
\PassOptionsToClass{10pt,a4paper}{article}
\documentclass{ltxdoc}

\usepackage[margin=35mm]{geometry}
\usepackage{hyperref}
\usepackage{hyperxmp}
\usepackage[usenames]{color}

\hypersetup{colorlinks=true}
\hypersetup{pdfstartview=FitH}
\hypersetup{pdfpagemode=UseNone}
\hypersetup{pdfsource={}}
\hypersetup{pdflang={en-UK}}
\hypersetup{pdfcopyright={Copyright 2017-2018 Niklas Beisert.
  This work may be distributed and/or modified under the
  conditions of the LaTeX Project Public License, either version 1.3
  of this license or (at your option) any later version.}}
\hypersetup{pdflicenseurl={http://www.latex-project.org/lppl.txt}}
\hypersetup{pdfcontactaddress={ETH Zurich, ITP, HIT K,
  Wolfgang-Pauli-Strasse 27}}
\hypersetup{pdfcontactpostcode={8093}}
\hypersetup{pdfcontactcity={Zurich}}
\hypersetup{pdfcontactcountry={Switzerland}}
\hypersetup{pdfcontactemail={nbeisert@itp.phys.ethz.ch}}
\hypersetup{pdfcontacturl={http://people.phys.ethz.ch/\xmptilde nbeisert/}}

\newcommand{\secref}[1]{\hyperref[#1]{section \ref*{#1}}}

\parskip1ex
\parindent0pt
\let\olditemize\itemize
\def\itemize{\olditemize\parskip0pt}

\begin{document}

\title{The \textsf{childdoc} Package}
\hypersetup{pdftitle={The childdoc Package}}
\author{Niklas Beisert\\[2ex]
  Institut f\"ur Theoretische Physik\\
  Eidgen\"ossische Technische Hochschule Z\"urich\\
  Wolfgang-Pauli-Strasse 27, 8093 Z\"urich, Switzerland\\[1ex]
  \href{mailto:nbeisert@itp.phys.ethz.ch}
  {\texttt{nbeisert@itp.phys.ethz.ch}}}
\hypersetup{pdfauthor={Niklas Beisert}}
\hypersetup{pdfsubject={Manual for the LaTeX2e Package childdoc}}
\date{30 December 2018, \textsf{v2.0}}
\maketitle

\begin{abstract}\noindent
\textsf{childdoc} is a \LaTeXe{} package
that enables the direct compilation
of document sections included by |\include|
to individual files.
\end{abstract}

\begingroup
\parskip0ex
\tableofcontents
\endgroup

%%%%%%%%%%%%%%%%%%%%%%%%%%%%%%%%%%%%%%%%%%%%%%%%%%%%%%%%%%%%%%%%%%%%%%%%%%%%%%%%
%%%%%%%%%%%%%%%%%%%%%%%%%%%%%%%%%%%%%%%%%%%%%%%%%%%%%%%%%%%%%%%%%%%%%%%%%%%%%%%%
\section{Introduction}

\LaTeX{} provides a mechanism to structure a large document (such as a book)
into a main file and several child files (containing the chapters)
using the |\include| command.
This mechanism is beneficial for documents
which span hundreds of pages in order to
make the source file(s) more manageable.
Moreover, compilation can be restricted to
selected child files by means of the |\includeonly| command.
The latter feature can be used to reduce the compilation time while editing
(this was significantly more useful in the earlier days of \LaTeX{})
or to generate a smaller document which is easier to navigate.
Another application of |\includeonly| is to generate
documents consisting of selected parts of the complete document.

However, there are a few drawbacks of the plain |\include| mechanism:
\begin{itemize}
\item
The child files cannot be compiled on their own,
they can only be compiled via the main file.
A naive editing environment
(such as a text editor with an option
to have the current file processed by \LaTeX)
may require one to switch to the main file before compiling;
attempting to compile the child file produces errors.
\item
The main file must be modified (each time)
to adjust the |\includeonly| command
to the present needs. This easily leaves the main file in a messy state.
\item
The generated document will always carry the filename
of the main document. This is inconvenient if
several child files are to be compiled and
to be kept for distribution.
\end{itemize}

The present package provides a simple interface
to make child files individually compilable by \LaTeX{}.
Compiling a child file then has the same effect as compiling
the main file with an |\includeonly| command
to select the appropriate child.
Moreover the generated document will carry the name of the child
rather than the main file.
This resolves all three above issues.

This feature is meant to make the editing of books,
thesis documents and lecture notes somewhat more convenient.
However, the package can also be used efficiently for
composing a series of documents (such as exercise sheets)
which are typically distributed individually.
It then assists the author in generating the individual documents
(potentially in different versions)
as well as a document containing the collected series.
Another application is in developing style files
or other kinds of included material
where compilation of the style file could redirect
to a sample or test file.

%%%%%%%%%%%%%%%%%%%%%%%%%%%%%%%%%%%%%%%%%%%%%%%%%%%%%%%%%%%%%%%%%%%%%%%%%%%%%%%%
%%%%%%%%%%%%%%%%%%%%%%%%%%%%%%%%%%%%%%%%%%%%%%%%%%%%%%%%%%%%%%%%%%%%%%%%%%%%%%%%
\section{Usage}

First of all, the package \textsf{childdoc} is \emph{not} a standard
\LaTeXe{} |.sty| style file! Therefore it needs to be invoked in
a non-standard way.

%%%%%%%%%%%%%%%%%%%%%%%%%%%%%%%%%%%%%%%%%%%%%%%%%%%%%%%%%%%%%%%%%%%%%%%%%%%%%%%%
\subsection{Included Files}
\label{sec:include}

%%%%%%%%%%%%%%%%%%%%%%%%%%%%%%%%%%%%%%%%
\DescribeMacro{\childdocmain}
To use the package, add the commands
\begin{center}
\begin{tabular}{l}
|\input{childdoc.def}|\\
|\childdocmain{}|\\
\end{tabular}
\end{center}
at the very top of the main \LaTeX{} file,
in particular \emph{before} the |\documentclass| statement!
The argument of |\childdocmain| should be left empty
(but it must be present).

%%%%%%%%%%%%%%%%%%%%%%%%%%%%%%%%%%%%%%%%
\DescribeMacro{\childdocof}
Furthermore, add the commands
\begin{center}
\begin{tabular}{l}
|\input{childdoc.def}|\\
|\childdocof{|\textit{main}|}|\\
\end{tabular}
\end{center}
at the top of every child file \textit{child}
which is included by |\include{|\textit{child}|}|
from within the main file
(or at least for those files to be compiled individually).
The argument \textit{main} must be the filename of the main file.

There are a couple of
considerations in setting up the main and child documents:

%%%%%%%%%%%%%%%%%%%%%%%%%%%%%%%%%%%%%%%%
\paragraph{Restrictions.}

Please note the following restrictions:
\begin{itemize}
\item
|\childdocmain| must be called with one argument \textit{main}
to ensure compatibility with earlier version of the package.
It must either be empty (|\childdocmain{}|)
or precisely match the filename of the main file in which it is specified.
See \secref{sec:detection} for further information.
\item
The filename \textit{main} must be specified without the |.tex| extension.
\item
The filename \textit{main} is case sensitive
(even in case-insensitive file systems)
due to internal string comparison.
\item
The argument \textit{main} should be fully expanded, it cannot be a macro.
\item
Subdirectories and special characters should be avoided in filenames.
\item
The command |\childdocmain{|\textit{main}|}| must be followed by a whitespace.
It should not be followed immediately by another command
or by a comment mark `|%|'.
This is because the \TeX{} parser reads the token immediately following
the argument of |\childdocmain| and puts it
at the beginning of every child section;
however, a white\-space is ignored.
\end{itemize}

%%%%%%%%%%%%%%%%%%%%%%%%%%%%%%%%%%%%%%%%
\paragraph{Content of Main File.}

It is advisable to place all content in the child files included by |\include|.
Any output contained in the main file will appear in all child documents
unless suppressed manually;
it cannot be suppressed automatically by the |\includeonly| directive
and thus should normally be avoided.
A method to include some content in the main file
by means of conditional processing is described in \secref{sec:conditional}.

%%%%%%%%%%%%%%%%%%%%%%%%%%%%%%%%%%%%%%%%
\paragraph{Page Numbering.}

When only a part of the document is compiled,
the appropriate numbering of pages
(as well as other status parameters)
is determined from the |.aux| files.
The latter contain information from previous passes.
However this information needs to propagate through
all intermediate child documents.
Therefore the page numbering in child documents may well
be inconsistent until the complete document is compiled at least once.

A useful (if unconventional) way to always ensure a consistent
page numbering is to restart the numbering in each child document
and denote the pages by `\textit{child}|.|\textit{page}'
where \textit{child} represents the chapter/section number of the child file.
This can be achieved by the command
|\numberwithin{page}{|\textit{child}|}|
of the \textsf{amsmath} package
where \textit{child} can be |chapter| or |section|
depending on the chosen structuring.
Alternatively, one can modify the macro |\thepage| appropriately
and reset the counter |page| at the start of each child file.

%%%%%%%%%%%%%%%%%%%%%%%%%%%%%%%%%%%%%%%%%%%%%%%%%%%%%%%%%%%%%%%%%%%%%%%%%%%%%%%%
\subsection{Conditional Processing}
\label{sec:conditional}

The package provides a mechanism to compile different versions
of a document. To customise the versions further some conditional processing
can come in handy to distinguish which version is being compiled.
The package provides two macros to describe the compilation context:

%%%%%%%%%%%%%%%%%%%%%%%%%%%%%%%%%%%%%%%%
\DescribeMacro{\ifchilddoc}
The conditional |\ifchilddoc| distinguishes between the compilation of
child documents and the main document:
%
\begin{center}
|\ifchilddoc |\textit{child-code}| |[|\||else |\textit{main-code}]| \||fi|
\end{center}

%%%%%%%%%%%%%%%%%%%%%%%%%%%%%%%%%%%%%%%%
\DescribeMacro{\childdocname}
\DescribeMacro{\childdocjob}
The macro |\childdocname| contains the filename (without extension)
of the main or child file being processed.
Note that |\childdocjob| will always contain the name of the main file.

%%%%%%%%%%%%%%%%%%%%%%%%%%%%%%%%%%%%%%%%
\paragraph{Title Page.}

Conditional processing can be used to include a title or banner page
in the main document when proper precautions are taken.
Importantly, the code in the main file should ensure that the page counter
(as well as other status parameters which are stored in the |.aux| files)
takes the same value after the conditional processing.
Otherwise the page numbers may take divergent values
depending on which part is compiled.

For example, a title page could be declared by:
%
\begin{center}
\begin{tabular}{l}
|\ifchilddoc\||else|\\
|\addtocounter{page}{-1}|\\
\textit{code for title page}\\
|\newpage|\\
|\||fi|
\end{tabular}
\end{center}
%
A banner page for the child documents can be generated by:
%
\begin{center}
\begin{tabular}{l}
|\ifchilddoc|\\
|\addtocounter{page}{-1}|\\
\textit{code for banner page}\\
|\newpage|\\
|\||fi|
\end{tabular}
\end{center}
%
Here one could write a message such as:
\begin{center}
|This is the part \childdocname{} of \childdocjob{}.|
\end{center}

%%%%%%%%%%%%%%%%%%%%%%%%%%%%%%%%%%%%%%%%%%%%%%%%%%%%%%%%%%%%%%%%%%%%%%%%%%%%%%%%
\subsection{Flags}
\label{sec:flags}

The package makes it easy to generate different versions
of the main or child documents.
To this end compilation flags can be defined
and assigned different default values.
They will be particularly useful in conjunction
with the forwarding mechanism described in \secref{sec:forward}.

For example, it may be useful to have a flag |\version|
which can be set to |draft| or |final|.
The document source will contain some conditional code
depending on the value of |\version|.
Suppose further, the flag should default to |final| for the main file
and to |draft| for child files
which is a natural assignment for editing the document.
This is achieved by placing the following code
in the preamble of the main document
(below the |\childdocmain| directive):
%
\begin{center}
\begin{tabular}{l}
|\ifchilddoc|\\
|\providecommand{\version}{draft}|\\
|\||else|\\
|\providecommand{\version}{final}|\\
|\||fi|
\end{tabular}
\end{center}
%
The definition by |\providecommand| makes sure
that previous definitions are not overwritten.
Further statements |\providecommand{\version}{...}|
can thus be added before the above code to override it.

For the main file, one might add a line
(between |\childdocmain| and the above block)
%
\begin{center}
|%\ifchilddoc\||else\providecommand{\version}{draft}\||fi|
\end{center}
%
which can be uncommented to produce a draft version.
Likewise one can add a line to the very top of a child file
(above the |\childdocof{|\textit{main}|}| directive)
%
\begin{center}
|%\providecommand{\version}{final}|
\end{center}
%
which can be uncommented to produce the final version of this child document.

%%%%%%%%%%%%%%%%%%%%%%%%%%%%%%%%%%%%%%%%%%%%%%%%%%%%%%%%%%%%%%%%%%%%%%%%%%%%%%%%
\subsection{Forwarding}
\label{sec:forward}

Different versions of the main or child documents
using compilation flags as described in \secref{sec:flags}
can be (permanently) stored in different files
for convenient compilation, viewing and distribution.
To this end, the package defines a command
to pass on compilation to a different file:

%%%%%%%%%%%%%%%%%%%%%%%%%%%%%%%%%%%%%%%%
\DescribeMacro{\childdocforward}
The command |\childdocforward| redirects processing to
another source file:
%
\begin{center}
\begin{tabular}{l}
|\input{childdoc.def}|\\
|\childdocforward[|\textit{main}|]{|\textit{dest}|}|\\
\end{tabular}
\end{center}
%
The argument \textit{dest} is the destination file
(without extension).
It should be the main file or one of the child files.
Note that further \textsf{childdoc} directives
such as |\childdocof| and |\childdocforward|
in the indicated file will be processed in this form.
The optional argument \textit{main}
passes on directly to the main file \textit{main}
while pretending to compile the child \textit{dest}.
This form behaves as if \textit{dest}
issues |\childdocof{|\textit{main}|}| right away,
and no further \textsf{childdoc} directives will be processed.

%%%%%%%%%%%%%%%%%%%%%%%%%%%%%%%%%%%%%%%%
\DescribeMacro{\...prefix}
In the alternative form |\childdocforwardprefix|,
%
\begin{center}
\begin{tabular}{l}
|\input{childdoc.def}|\\
|\childdocforwardprefix[|\textit{main}|]{|\textit{prefix}|}{|\textit{dest}|}|
\end{tabular}
\end{center}
%
the destination file is determined by a pattern
depending on the current file:
To make this work, the current file must be called
`{\textit{prefix}\hspace{0.2em}\textit{suffix}}'
with \textit{prefix} matching precisely the argument.
Processing is then passed on to the file
`{\textit{dest}\hspace{0.2em}\textit{suffix}}'.
Surely, the same effect is achieved by
directly specifying the
argument `{\textit{dest}\hspace{0.2em}\textit{suffix}}'
in the first form.
However, that requires to set up a different file
for each child. With the alternative form of the command
all these files can have exactly the same content
which simplifies setting them up and maintaining them.

For example, the following file |draft.tex|
with a compilation flag |\version| as described in \secref{sec:flags}
compiles the main document as a draft:
%
\begin{center}
\begin{tabular}{l}
|\def\version{draft}|\\
|\input{childdoc.def}|\\
|\childdocforward{|\textit{main}|}|
\end{tabular}
\end{center}
%
Likewise, the following files |final|\textit{nn}|.tex|
compile the final version of the child document
|child|\textit{nn}|.tex|:
%
\begin{center}
\begin{tabular}{l}
|\def\version{final}|\\
|\input{childdoc.def}|\\
|\childdocforwardprefix{final}{child}|
\end{tabular}
\end{center}
%

Note that when several versions of a main file and/or of each child file
are to be generated, it may be convenient to set up a |Makefile| or
shell script to automatise the process.

%%%%%%%%%%%%%%%%%%%%%%%%%%%%%%%%%%%%%%%%%%%%%%%%%%%%%%%%%%%%%%%%%%%%%%%%%%%%%%%%
\subsection{Command Line Processing}
\label{sec:commandline}

The effect of redirection files can also be achieved by invoking
the \LaTeX{} compiler with a more elaborate command line.
Most conveniently this should be done as part
of a shell script or a |Makefile|.

When using \textsf{childdoc} in the main file, the following
command lines effectively perform a redirection
(note that depending on the shell being used,
backslashes may have to be doubled: `|\|' $\to$ `|\\|'):
%
\begin{center}
|... -jobname "|\textit{target}|" |\\|"|[\textit{flags}]%
|\input{childdoc.def}\childdocforward[|\textit{main}|]{|\textit{dest}|}"|
\end{center}
%
Here \textit{target} is the name of the output file,
\textit{main} is the name of the main file
and \textit{dest} is the name of the main or child file to be processed
(all filenames without extensions).
The optional argument \textit{main} can be omitted
if \textit{main} matches \textit{dest}.
Optionally, compilation \textit{flags} can be defined via |\def| commands.
This command line makes the \TeX{} engine believe
it is compiling the file \textit{target}
whose content is specified as the latter parameter.
The provided code then forwards the processing to
\textit{main} or \textit{dest} as described in \secref{sec:forward}.

%%%%%%%%%%%%%%%%%%%%%%%%%%%%%%%%%%%%%%%%%%%%%%%%%%%%%%%%%%%%%%%%%%%%%%%%%%%%%%%%
\subsection{Include by Input}
\label{sec:input}

Including child documents by |\include| has some restrictions by design.
Most notably, the content of a child document always occupies
its own set of pages; pages cannot be shared between child documents.
Usually, this behaviour makes perfect sense
because each child document contain an essential part of the document.
However, in some situations it may be desirable to compose
a document from a collection of parts
without having mandatory page breaks between then.
For this case, the package
provides a mechanism to include parts
by |\input| which can also be processed individually.
However, by construction this mechanism
requires manual handling of the content to be output.

%%%%%%%%%%%%%%%%%%%%%%%%%%%%%%%%%%%%%%%%
\DescribeMacro{\ifchilddocmanual}
The main file should be prepared as usual, see \secref{sec:include}.
However, the document body must make a distinction
between processing of an individual part and of the main document, e.g.:
%
\begin{center}
\begin{tabular}{l}
|\ifchilddocmanual|\\
|\input{\childdocname}|\\
|\||else|\\
\textit{document body with }|\input{|\textit{part}|}|\\
|\||fi|
\end{tabular}
\end{center}
%
The conditional |\ifchilddocmanual| is true whenever
a part to be included by |\input| is being compiled,
and the name of the part is stored in |\childdocname|.

%%%%%%%%%%%%%%%%%%%%%%%%%%%%%%%%%%%%%%%%
\DescribeMacro{\childdocby}
Each part to be included by |\input| should start with:
%
\begin{center}
\begin{tabular}{l}
|\input{childdoc.def}|\\
|\childdocby{|\textit{main}|}|\\
\end{tabular}
\end{center}
%
The directive |\childdocby| is similar to |\childdocof|
described in \secref{sec:include},
but the subsequent selection of content must be done manually.
To that end, both |\ifchilddoc| and |\ifchilddocmanual|
will be true upon processing of a part,
and the name of the part is stored in |\childdocname|.
Note that |\jobname| will be set to the filename of the current part
so that each part receives an individual |.aux| file
that does not interfere with the |.aux| file(s) of the main document.
This behaviour can be altered by the alternative form
|\childdocby[*]{|\textit{main}|}| (with a non-empty optional argument)
which uses the |.aux| file of the main document
by setting |\jobname| to \textit{main}.

%%%%%%%%%%%%%%%%%%%%%%%%%%%%%%%%%%%%%%%%%%%%%%%%%%%%%%%%%%%%%%%%%%%%%%%%%%%%%%%%
\subsection{Driver Development}
\label{sec:driver}

The \textsf{childdoc} mechanism can also be use for the development
of definition files such as \LaTeX{} styles or classes.
This case differs from the above setup with multiple parts
included by |\include| in that no |\includeonly| should be invoked.
This can be achieved by starting the include file
(before |\ProvidesPackage|) with:
%
\begin{center}
\begin{tabular}{l}
|\input{childdoc.def}|\\
|\childdocforward{|\textit{main}|}|\\
\end{tabular}
\end{center}
%
or alternatively with:
%
\begin{center}
\begin{tabular}{l}
|\input{childdoc.def}|\\
|\childdocby{|\textit{main}|}|\\
\end{tabular}
\end{center}
%
Both forms have slightly different effects as described above.
The main file is prepared as usual, see \secref{sec:include}.

%%%%%%%%%%%%%%%%%%%%%%%%%%%%%%%%%%%%%%%%%%%%%%%%%%%%%%%%%%%%%%%%%%%%%%%%%%%%%%%%
\subsection{Legacy Detection}
\label{sec:detection}

The directive |\childdocmain| in the main file can detect
whether the complete document or merely a child is to be compiled
even without using the directive |\childdocof|.
This method is deprecated because it is less robust
and there is no compelling reason to use it;
it is merely provided for backward compatibility
and it may be removed in future versions.

If the detection mechanism is to be used,
it is mandatory to correctly specify
the filename of the main file as the argument of |\childdocmain|:
%
\begin{center}
\begin{tabular}{l}
|\input{childdoc.def}|\\
|\childdocmain{|\textit{main}|}|\\
\end{tabular}
\end{center}
%
If |\jobname| does not match the argument \textit{main} of |\childdocmain|,
it is assumed that |\jobname| points to the child file to be compiled.
When using |\childdocmain| with the main file specified as argument,
it suffices to start a child file
with just |\input{|\textit{main}|}|
without loading of the package and using |\childdocof|.
If instead all processing is done
with the appropriate \textsf{childdoc} directives,
the argument of \textit{main} of |\childdocmain| can be empty.

An alternative version of the command line processing described
in \secref{sec:commandline} using the detection mechanism reads:
%
\begin{center}
|... -jobname "|\textit{target}|" "|[\textit{flags}]%
[|\def\jobname{|\textit{dest}|}|]|\input{|\textit{main}|}"|
\end{center}

%%%%%%%%%%%%%%%%%%%%%%%%%%%%%%%%%%%%%%%%%%%%%%%%%%%%%%%%%%%%%%%%%%%%%%%%%%%%%%%%
\subsection{Manual Code}
\label{sec:manual}

In case one cannot be certain whether the definitions file |childdoc.def|
is installed on the target \TeX{} distribution
and one prefers not to ship it,
it is conceivable to paste a few relevant commands into the sources.

To that end, drop all statements |\input{childdoc.def}|
and perform the replacements as outlined below.
Instead of |\childdocmain{|\textit{main}|}| add the following code
to the top of the main file:
%
\begin{center}
\begin{tabular}{l}
|\||ifdefined\childdocname\endinput\||fi\newif\ifchilddoc|\\
|\edef\childdocname{\scantokens\expandafter{\jobname\noexpand}}|\\
|\def\childdocmain{|\textit{main}|}\||ifx\childdocmain\childdocname\||else|\\
|\childdoctrue\includeonly{\childdocname}\let\jobname\childdocmain\||fi|\\
\end{tabular}
\end{center}
%
Instead of |\childdocof{|\textit{main}|}| just include the main file
at the top of each child file:
%
\begin{center}
|\input{|\textit{main}|}|
\end{center}
%
A simple redirection |\childdocforward{|\textit{dest}|}| is achieved by:
%
\begin{center}
|\def\jobname{|\textit{dest}|}\input{\jobname}|
\end{center}
%
The redirection with prefix
|\childdocforwardprefix[|\textit{prefix}|]{|\textit{dest}|}|
is accomplished by:
%
\begin{center}
\begin{tabular}{l}
|{\edef\jobname{\scantokens\expandafter{\jobname\noexpand}}|\\
|\def\redirectjob |\textit{prefix}|#1~~~{\gdef\jobname{|\textit{dest}|#1}}|\\
|\expandafter\redirectjob\jobname~~~}\input{\jobname}|
\end{tabular}
\end{center}

In an alternative approach,
child documents can be compiled by a specific command line
without additional code or specific definitions:
%
\begin{center}
|... -jobname "|\textit{target}|" "|[\textit{flags}]%
|\includeonly{|\textit{dest}|}\input{|\textit{main}|}"|
\end{center}
%

%%%%%%%%%%%%%%%%%%%%%%%%%%%%%%%%%%%%%%%%%%%%%%%%%%%%%%%%%%%%%%%%%%%%%%%%%%%%%%%%
%%%%%%%%%%%%%%%%%%%%%%%%%%%%%%%%%%%%%%%%%%%%%%%%%%%%%%%%%%%%%%%%%%%%%%%%%%%%%%%%
\section{Information}

%%%%%%%%%%%%%%%%%%%%%%%%%%%%%%%%%%%%%%%%%%%%%%%%%%%%%%%%%%%%%%%%%%%%%%%%%%%%%%%%
\subsection{Copyright}

Copyright \copyright{} 2017--2018 Niklas Beisert

This work may be distributed and/or modified under the
conditions of the \LaTeX{} Project Public License, either version 1.3
of this license or (at your option) any later version.
The latest version of this license is in
  \url{http://www.latex-project.org/lppl.txt}
and version 1.3 or later is part of all distributions of \LaTeX{}
version 2005/12/01 or later.

This work has the LPPL maintenance status `maintained'.

The Current Maintainer of this work is Niklas Beisert.

This work consists of the files |README.txt|, |childdoc.ins| and |childdoc.dtx|
as well as the derived files |childdoc.def|, |cdocsamp.tex|
with |cdocsch1.tex|, |cdocsch2.tex|, |cdocspt3.tex|, |cdocspt4.tex|,
|cdocsdrf.tex|, |cdocsfn1.tex|, |cdocsfn2.tex|
as well as |childdoc.pdf|.

%%%%%%%%%%%%%%%%%%%%%%%%%%%%%%%%%%%%%%%%%%%%%%%%%%%%%%%%%%%%%%%%%%%%%%%%%%%%%%%%
\subsection{Files and Installation}

The package consists of the files:
%
\begin{center}
\begin{tabular}{ll}
    |README.txt|   & readme file \\
    |childdoc.ins| & installation file \\
    |childdoc.dtx| & source file \\
    |childdoc.def| & definition file \\
    |cdocsamp.tex| & sample main file \\
    |cdocsch1.tex| & sample include file \\
    |cdocsch2.tex| & sample include file \\
    |cdocspt3.tex| & sample part file \\
    |cdocspt4.tex| & sample part file \\
    |cdocsdrf.tex| & sample redirection file \\
    |cdocsfn1.tex| & sample redirection file \\
    |cdocsfn2.tex| & sample redirection file \\
    |childdoc.pdf| & manual
\end{tabular}
\end{center}
%
The distribution consists of the files
|README.txt|, |childdoc.ins| and |childdoc.dtx|.
%
\begin{itemize}
\item
Run (pdf)\LaTeX{} on |childdoc.dtx|
to compile the manual |childdoc.pdf| (this file).
\item
Run \LaTeX{} on |childdoc.ins| to create the definitions file |childdoc.def|
and the sample |cdocsamp.tex| with include files
|cdocsch1.tex|, |cdocsch2.tex|, |cdocspt3.tex|, |cdocspt4.tex|,
|cdocsdrf.tex|, |cdocsfn1.tex|, |cdocsfn2.tex|.
Then copy the file |childdoc.def| to an appropriate directory of your \LaTeX{}
distribution, e.g.\ \textit{texmf-root}|/tex/latex/childdoc|.
\end{itemize}

%%%%%%%%%%%%%%%%%%%%%%%%%%%%%%%%%%%%%%%%%%%%%%%%%%%%%%%%%%%%%%%%%%%%%%%%%%%%%%%%
\subsection{Related CTAN Packages}

There are several other packages which offer a similar functionality:
%
\begin{itemize}
\item
The packages
\href{http://ctan.org/pkg/docmute}{\textsf{docmute}},
\href{http://ctan.org/pkg/includex}{\textsf{includex}} and
\href{http://ctan.org/pkg/standalone}{\textsf{standalone}}
provide commands to include only the document body of
a child file thus allowing both files to be compiled individually.
\item
The packages \href{http://ctan.org/pkg/subdocs}{\textsf{subdocs}}
and \href{http://ctan.org/pkg/subfiles}{\textsf{subfiles}}
provide structures in which the main and child documents can be
encapsulated and allowing them to be compiled individually.
The inclusion mechanism is different from the conventional |\include|.
\item
The package \href{http://ctan.org/pkg/combine}{\textsf{combine}}
is an elaborate solution to combine several documents into one.
\end{itemize}
%
See also the CTAN topic \href{http://ctan.org/topic/subdocs}{\textsf{subdocs}}
for further related packages.
The present package differs from the above solutions in that
a document structure constructed with the conventional |\include| mechanism
just needs two extra commands at the top of every file
such that all constituent files can be compiled individually.

%%%%%%%%%%%%%%%%%%%%%%%%%%%%%%%%%%%%%%%%%%%%%%%%%%%%%%%%%%%%%%%%%%%%%%%%%%%%%%%%
%\subsection{Feature Suggestions}
%
%The following is a list of features which may be useful for future
%versions of this package:
%%
%\begin{itemize}
%\item
%\ldots
%\end{itemize}

%%%%%%%%%%%%%%%%%%%%%%%%%%%%%%%%%%%%%%%%%%%%%%%%%%%%%%%%%%%%%%%%%%%%%%%%%%%%%%%%
\subsection{Revision History}

%%%%%%%%%%%%%%%%%%%%%%%%%%%%%%%%%%%%%%%%
\paragraph{v2.0:} 2018/12/30

\begin{itemize}
\item
immediate forward processing
\item
added |\childdocby| mechanism
\item
manual restructured
\end{itemize}

%%%%%%%%%%%%%%%%%%%%%%%%%%%%%%%%%%%%%%%%
\paragraph{v1.6:} 2018/01/17

\begin{itemize}
\item
application for development of include files
\item
corrections to manual
\end{itemize}

%%%%%%%%%%%%%%%%%%%%%%%%%%%%%%%%%%%%%%%%
\paragraph{v1.5:} 2017/05/21

\begin{itemize}
\item
more complete structuring introduced
\item
|\childdocof| introduced
\item
|\childdoc| renamed to |\childdocmain|
\item
|\childredirect| renamed to |\childdocforward| and |\childdocforwardprefix|
and functionality expanded
\end{itemize}

%%%%%%%%%%%%%%%%%%%%%%%%%%%%%%%%%%%%%%%%
\paragraph{v1.0:} 2017/04/27

\begin{itemize}
\item
manual and install package
\item
first version published on CTAN
\end{itemize}

%%%%%%%%%%%%%%%%%%%%%%%%%%%%%%%%%%%%%%%%
\paragraph{v0.6:} 2017/04/26

\begin{itemize}
\item
redirection mechanism added
\end{itemize}

%%%%%%%%%%%%%%%%%%%%%%%%%%%%%%%%%%%%%%%%
\paragraph{v0.5:} 2017/04/26

\begin{itemize}
\item
functionality in definition file
\end{itemize}


%%%%%%%%%%%%%%%%%%%%%%%%%%%%%%%%%%%%%%%%%%%%%%%%%%%%%%%%%%%%%%%%%%%%%%%%%%%%%%%%
%%%%%%%%%%%%%%%%%%%%%%%%%%%%%%%%%%%%%%%%%%%%%%%%%%%%%%%%%%%%%%%%%%%%%%%%%%%%%%%%
%%%%%%%%%%%%%%%%%%%%%%%%%%%%%%%%%%%%%%%%%%%%%%%%%%%%%%%%%%%%%%%%%%%%%%%%%%%%%%%%
\appendix

\settowidth\MacroIndent{\rmfamily\scriptsize 000\ }

 \DocInput{childdoc.dtx}

\end{document}
%</driver>
% \fi
%
% %%%%%%%%%%%%%%%%%%%%%%%%%%%%%%%%%%%%%%%%%%%%%%%%%%%%%%%%%%%%%%%%%%%%%%%%%%%%%%
% %%%%%%%%%%%%%%%%%%%%%%%%%%%%%%%%%%%%%%%%%%%%%%%%%%%%%%%%%%%%%%%%%%%%%%%%%%%%%%
% \section{Sample}
%\iffalse
%<*samplemain>
%\fi
%
% The following presents a sample document
% with two chapters, two parts, a title page,
% a compile flag as well as three forwarding files to set the flag.
% It consists of eight |.tex| files:
% \begin{center}
% \begin{tabular}{ll}
% |cdocsamp.tex|&main file\\
% |cdocsch1.tex|&include file for chapter 1\\
% |cdocsch2.tex|&include file for chapter 2\\
% |cdocspt3.tex|&include file for part 3\\
% |cdocspt4.tex|&include file for part 4\\
% |cdocsdrf.tex|&forwarding file for main file in draft mode\\
% |cdocsfi1.tex|&forwarding file for final version of chapter 1\\
% |cdocsfi2.tex|&forwarding file for final version of chapter 2\\
% \end{tabular}
% \end{center}
% Each of the eight files can be compiled directly by the \LaTeX{} compiler.
%
% %%%%%%%%%%%%%%%%%%%%%%%%%%%%%%%%%%%%%%
% \paragraph{Main File.}
%
% The main file is called |cdocsamp.tex|.
%
% Load the \textsf{childdoc} definitions and
% declare the filename for the main document:
%    \begin{macrocode}
\input{childdoc.def}
\childdocmain{}
%    \end{macrocode}

% Optional override for |\version| flag:
%    \begin{macrocode}
%%\ifchilddoc\else\providecommand{\version}{draft}\fi
%    \end{macrocode}

% Define the default values for the |\version| flag
% (|final| for the main file and |draft| for childs):
%    \begin{macrocode}
\ifchilddoc
\providecommand{\version}{draft}
\else
\providecommand{\version}{final}
\fi
%    \end{macrocode}

% Load the standard document class:
%    \begin{macrocode}
\documentclass[12pt]{article}
%    \end{macrocode}

% Start the document body:
%    \begin{macrocode}
\begin{document}
%    \end{macrocode}

% Declare a title page.
% Print title, part of document being processed and version flag:
%    \begin{macrocode}
\addtocounter{page}{-1}
\begin{center}
{\LARGE\bfseries{}childdoc example\par}
\vspace{1cm}
\ifchilddoc
\ifchilddocmanual part\else chapter\fi:
`\childdocname' of `\childdocjob'\par
\else
main document: `\childdocjob'\par
\fi
version: \version\par
\end{center}
\newpage
%    \end{macrocode}

% Manually include selected file,
% otherwise process as usual:
%    \begin{macrocode}
\ifchilddocmanual
\section*{part `\childdocname'}
\input{\childdocname}
\else
%    \end{macrocode}

% Include the two chapters:
%    \begin{macrocode}
\include{cdocsch1}
\include{cdocsch2}
%    \end{macrocode}

% Include the two parts unless only chapters should be displayed:
%    \begin{macrocode}
\ifchilddoc\else
\section{part three}
\input{cdocspt3}
\section{part four}
\input{cdocspt4}
\fi
%    \end{macrocode}

% Process as usual until here:
%    \begin{macrocode}
\fi
%    \end{macrocode}

% End of document body:
%    \begin{macrocode}
\end{document}
%    \end{macrocode}
%\iffalse
%</samplemain>
%\fi
%
% %%%%%%%%%%%%%%%%%%%%%%%%%%%%%%%%%%%%%%
% \paragraph{Chapter Include Files.}
%
% The include files are called |cdocsch1.tex| and |cdocsch2.tex|.
%
%\iffalse
%<*samplechap1|samplechap2>
%\fi

% Optional override for |\version| flag:
%    \begin{macrocode}
%%\providecommand{\version}{final}
%    \end{macrocode}

% Include the main document:
%    \begin{macrocode}
\input{childdoc.def}
\childdocof{cdocsamp}
%    \end{macrocode}

%\iffalse
%</samplechap1|samplechap2>
%\fi
%
%\iffalse
%<*samplechap1>
%\fi
% Some text for chapter 1:
%    \begin{macrocode}
\section{one}
some text in chapter one
%    \end{macrocode}

%\iffalse
%</samplechap1>
%\fi
% Some text for chapter 2:
%\iffalse
%<*samplechap2>
%\fi
%    \begin{macrocode}
\section{two}
more text in chapter two
%    \end{macrocode}

%\iffalse
%</samplechap2>
%\fi
%
% %%%%%%%%%%%%%%%%%%%%%%%%%%%%%%%%%%%%%%
% \paragraph{Part Include Files.}
%
% The include files are called |cdocspt3.tex| and |cdocspt4.tex|.
%
%\iffalse
%<*samplepart3|samplepart4>
%\fi

% Optional override for |\version| flag:
%    \begin{macrocode}
%%\providecommand{\version}{final}
%    \end{macrocode}

% Include the main document:
%    \begin{macrocode}
\input{childdoc.def}
\childdocby{cdocsamp}
%    \end{macrocode}

%\iffalse
%</samplepart3|samplepart4>
%\fi
%
%\iffalse
%<*samplepart3>
%\fi
% Some text for part 3:
%    \begin{macrocode}
some text in part three
%    \end{macrocode}

%\iffalse
%</samplepart3>
%\fi
% Some text for part 4:
%\iffalse
%<*samplepart4>
%\fi
%    \begin{macrocode}
more text in part four
%    \end{macrocode}

%\iffalse
%</samplepart4>
%\fi
%
% %%%%%%%%%%%%%%%%%%%%%%%%%%%%%%%%%%%%%%
% \paragraph{Forwarding for a Complete Draft.}
%
% The following forwarding file |cdocsdrf.tex|
% compiles the main document in draft mode:
%\iffalse
%<*sampledraft>
%\fi
%    \begin{macrocode}
\def\version{draft}
\input{childdoc.def}
\childdocforward{cdocsamp}
%    \end{macrocode}

%\iffalse
%</sampledraft>
%\fi
%
% %%%%%%%%%%%%%%%%%%%%%%%%%%%%%%%%%%%%%%
% \paragraph{Forwarding for Final Version of the Chapters.}
%
% The following forwarding files |cdocsfn1.tex| and |cdocsfn2.tex|
% (with identical content)
% compile the final versions of the child documents
% |cdocsch1.tex| and |cdocsch2.tex|, respectively:
%\iffalse
%<*samplefinal>
%\fi
%    \begin{macrocode}
\def\version{final}
\input{childdoc.def}
\childdocforwardprefix[cdocsamp]{cdocsfn}{cdocsch}
%    \end{macrocode}

%\iffalse
%</samplefinal>
%\fi
%
% %%%%%%%%%%%%%%%%%%%%%%%%%%%%%%%%%%%%%%
% \paragraph{Command Line Processing.}
%
% The following three command lines generate the output files
% |cdocscld|, |cdocscl1| and |cdocscl2|
% which should be identical to
% |cdocsdrf|, |cdocsch1| and |cdocsfn2|, respectively:
% \begin{center}
% \begin{tabular}{l}
% |latex -jobname cdocscld \|\\
% |  "\def\version{draft}\input{childdoc.def}\childdocforward{cdocsamp}"|\\
% |latex -jobname cdocscl1 \|\\
% |  "\input{childdoc.def}\childdocforward[cdocsamp]{cdocsch1}"|\\
% |latex -jobname cdocscl2 \|\\
% |  "\def\version{final}\input{childdoc.def}\childdocforward{cdocsch2}"|
% \end{tabular}
% \end{center}
% Note that the trailing backslash on each first line
% merely continues the input to the second line
% (for convenient cut ant paste).
% Furthermore, the command |latex| can be replaced by any
% of its alternative versions such as |pdflatex|.
%
% %%%%%%%%%%%%%%%%%%%%%%%%%%%%%%%%%%%%%%%%%%%%%%%%%%%%%%%%%%%%%%%%%%%%%%%%%%%%%%
% %%%%%%%%%%%%%%%%%%%%%%%%%%%%%%%%%%%%%%%%%%%%%%%%%%%%%%%%%%%%%%%%%%%%%%%%%%%%%%
% \section{Implementation}
%\iffalse
%<*package>
%\fi
%
% This section describes the definitions file |childdoc.def|.

% The definitions cannot be loaded using |\usepackage| or |\RequirePackage|
% which has a mechanism to prevent loading a style file more than once.
% When loading the definitions by means of |\input|
% multiple instances have to be prevented manually:
%\iffalse
%This code needs to be before the `\ProvidesFile' directive
%which is defined at the beginning of this file.
%Therefore it is also placed there and commented out here.
%</package>
%<*discard>
%\fi
%    \begin{macrocode}
\ifdefined\childdocmain\endinput\fi
%    \end{macrocode}
%\iffalse
%</discard>
%<*package>
%\fi
%
% \macro{\ifchilddoc}
% \macro{\ifchilddocmanual}
% The conditional |\ifchilddoc| tells whether a
% child (true) or main (false) document is being compiled.
% The conditional |\ifchilddocmanual| tells whether
% the |\includeonly| mechanism is used (false) or
% the selection of child files must be performed manually (true).
% The definitions initialise to false:
%    \begin{macrocode}
\newif\ifchilddoc
\newif\ifchilddocmanual
%    \end{macrocode}

% \macro{\childdocname}
% \macro{\childdocjob}
% The macro |\childdocname| stores the name of the main document
% to be compiled. The macro |\childdocjob| stores the name of
% the document on which the \LaTeX{} compiler was originally invoked.
% The content of |\jobname| cannot be compared
% to filenames specified in the source due to different catcodes.
% The following code rescans |\jobname|, stores the result
% in |\childdocname| and saves a copy in |\childdocjob|:
%    \begin{macrocode}
\edef\childdocname{\scantokens\expandafter{\jobname\noexpand}}
\let\childdocjob\childdocname
%    \end{macrocode}

% \macro{\childdocdisable}
% The macro |\childdocdisable| prevents the main file
% from being processed more than once.
% At this stage, the main document command |\childdocmain|
% is assumed to be called once again where it should do nothing.
% Any subsequent call to it should prevent
% a secondary processing of the main document
% It overwrites the forwarding commands
% |\childdocof| and |\childdocforward|
% with empty macros to prevent further inclusions of the main document:
%    \begin{macrocode}
\newcommand{\childdocdisable}
{
  \renewcommand{\childdocmain}[1]{\renewcommand{\childdocmain}[1]{\endinput}}
  \renewcommand{\childdocof}[1]{}
  \renewcommand{\childdocby}[2][]{}
  \renewcommand{\childdocforward}[2][]{}
  \renewcommand{\childdocdisable}{}
}
%    \end{macrocode}

% \macro{\childdocmain}
% The macro |\childdocmain| is to be called at the top of the main file
% with nothing or the main filename (without extension) as argument.
% First, it breaks loops.
% If the argument is not empty and does not match |\childdocname|
% (which is set by the first inclusion of |childdoc.def|),
% |\ifchilddoc| is set to true, |\includeonly| is applied to the child file
% and |\jobname| is set to the main file
% (for proper handling of |.aux| files):
%    \begin{macrocode}
\newcommand{\childdocmain}[1]
{
  \childdocdisable\childdocmain{}
  \if?#1?\else
    \begingroup
      \def\childdoctmp{#1}
      \ifx\childdoctmp\childdocname
        \def\childdoctmp{}
      \else
        \def\childdoctmp
        {
          \childdoctrue
          \includeonly{\childdocname}
          \def\childdocjob{#1}
          \def\jobname{#1}
        }
      \fi
      \expandafter
    \endgroup
    \childdoctmp
  \fi
}
%    \end{macrocode}

% \macro{\childdocof}
% The command |\childdocof| redirects
% compilation to the main file |#1|.
%    \begin{macrocode}
\newcommand{\childdocof}[1]
{
  \childdocdisable
  \childdoctrue
  \includeonly{\childdocname}
  \def\jobname{#1}
  \def\childdocjob{#1}
  \input{#1}
}
%    \end{macrocode}

% \macro{\childdocby}
% The command |\childdocby| ....
%    \begin{macrocode}
\newcommand{\childdocby}[2][]
{
  \childdocdisable
  \childdoctrue
  \childdocmanualtrue
  \if?#1?\else
    \def\jobname{#2}
  \fi
  \def\childdocjob{#2}
  \input{#2}
  \endinput
}
%    \end{macrocode}

% \macro{\childdocforward}
% The command |\childdocforward| redirects
% compilation to the main file or
% (if the optional argument is given) a child file.
% Parameters are set as if the main file
% or a child file starting with |\childdocof| was compiled.
% Then compilation is handed over to the main file:
%    \begin{macrocode}
\newcommand{\childdocforward}[2][]
{
  \begingroup
    \if?#1?
      \def\childdoctmp
      {
        \def\childdocname{#2}
        \def\childdocjob{#2}
        \def\jobname{#2}
        \input{#2}
        \endinput
      }
    \else
      \def\childdoctmp
      {
        \childdocdisable
        \def\childdocname{#2}
        \childdoctrue
        \includeonly{#2}
        \def\childdocjob{#1}
        \def\jobname{#1}
        \input{#1}
        \endinput
      }
    \fi
    \expandafter
  \endgroup
  \childdoctmp
}
%    \end{macrocode}

% \macro{\childdocforwardprefix}
% The command |\childdocforwardprefix| redirects
% compilation to the main or a child file by means of a pattern.
% The prefix |#1| in the current filename is replaced by |#2|
% and the suffix of the current filename is kept
% (it is assumed that the filename does not contain the substring `|~~~|'
% which is used as a delimiter).
% Compilation is handed over to the new file by |\childdocforward|:
%    \begin{macrocode}
\newcommand{\childdocforwardprefix}[3][]
{
  \begingroup
    \def\childdocextract #2##1~~~{\def\childdoctmp{\childdocforward[#1]{#3##1}}}
    \expandafter\childdocextract\childdocname~~~
    \expandafter
  \endgroup
  \childdoctmp
}
%    \end{macrocode}

% \macro{\childdoc}
% The deprecated macro |\childdoc| is a legacy version of |\childdocmain|:
%    \begin{macrocode}
\newcommand{\childdoc}{\childdocmain}
%    \end{macrocode}

% \macro{\childdocredirect}
% The deprecated macro |\childdocredirect| is a legacy version
% of |\childdocforward| and |\childdocforwardprefix|:
%    \begin{macrocode}
\newcommand{\childdocredirect}[2][]
{
  \begingroup
    \if?#1?
      \def\childdoctmp{\childdocforward{#2}}
    \else
      \def\childdoctmp{\childdocforwardprefix{#1}{#2}}
    \fi
    \expandafter
  \endgroup
  \childdoctmp
}
%    \end{macrocode}

%\iffalse
%</package>
%\fi
%
\endinput
\childdocforward[cdocsamp]{cdocsch1}"|\\
% |latex -jobname cdocscl2 \|\\
% |  "\def\version{final}% \iffalse
%
% childdoc.dtx Copyright (C) 2017-2018 Niklas Beisert
%
% This work may be distributed and/or modified under the
% conditions of the LaTeX Project Public License, either version 1.3
% of this license or (at your option) any later version.
% The latest version of this license is in
%   http://www.latex-project.org/lppl.txt
% and version 1.3 or later is part of all distributions of LaTeX
% version 2005/12/01 or later.
%
% This work has the LPPL maintenance status `maintained'.
%
% The Current Maintainer of this work is Niklas Beisert.
%
% This work consists of the files childdoc.dtx and childdoc.ins
% and the derived files childdoc.def and cdocsamp.tex with
% cdocsch1.tex, cdocsch2.tex, cdocsdrf.tex, cdocsfn1.tex, cdocsfn2.tex.
%
%<package>\ifdefined\childdocmain\endinput\fi
%<package>\ProvidesFile{childdoc.def}[2018/12/30 v2.0 child document driver]
%<samplemain>\ProvidesFile{cdocsamp.tex}[2018/12/30 v2.0 sample for childdoc]
%<*driver>
%\ProvidesFile{childdoc.drv}[2018/12/30 v2.0 childdoc reference manual file]
\PassOptionsToClass{10pt,a4paper}{article}
\documentclass{ltxdoc}

\usepackage[margin=35mm]{geometry}
\usepackage{hyperref}
\usepackage{hyperxmp}
\usepackage[usenames]{color}

\hypersetup{colorlinks=true}
\hypersetup{pdfstartview=FitH}
\hypersetup{pdfpagemode=UseNone}
\hypersetup{pdfsource={}}
\hypersetup{pdflang={en-UK}}
\hypersetup{pdfcopyright={Copyright 2017-2018 Niklas Beisert.
  This work may be distributed and/or modified under the
  conditions of the LaTeX Project Public License, either version 1.3
  of this license or (at your option) any later version.}}
\hypersetup{pdflicenseurl={http://www.latex-project.org/lppl.txt}}
\hypersetup{pdfcontactaddress={ETH Zurich, ITP, HIT K,
  Wolfgang-Pauli-Strasse 27}}
\hypersetup{pdfcontactpostcode={8093}}
\hypersetup{pdfcontactcity={Zurich}}
\hypersetup{pdfcontactcountry={Switzerland}}
\hypersetup{pdfcontactemail={nbeisert@itp.phys.ethz.ch}}
\hypersetup{pdfcontacturl={http://people.phys.ethz.ch/\xmptilde nbeisert/}}

\newcommand{\secref}[1]{\hyperref[#1]{section \ref*{#1}}}

\parskip1ex
\parindent0pt
\let\olditemize\itemize
\def\itemize{\olditemize\parskip0pt}

\begin{document}

\title{The \textsf{childdoc} Package}
\hypersetup{pdftitle={The childdoc Package}}
\author{Niklas Beisert\\[2ex]
  Institut f\"ur Theoretische Physik\\
  Eidgen\"ossische Technische Hochschule Z\"urich\\
  Wolfgang-Pauli-Strasse 27, 8093 Z\"urich, Switzerland\\[1ex]
  \href{mailto:nbeisert@itp.phys.ethz.ch}
  {\texttt{nbeisert@itp.phys.ethz.ch}}}
\hypersetup{pdfauthor={Niklas Beisert}}
\hypersetup{pdfsubject={Manual for the LaTeX2e Package childdoc}}
\date{30 December 2018, \textsf{v2.0}}
\maketitle

\begin{abstract}\noindent
\textsf{childdoc} is a \LaTeXe{} package
that enables the direct compilation
of document sections included by |\include|
to individual files.
\end{abstract}

\begingroup
\parskip0ex
\tableofcontents
\endgroup

%%%%%%%%%%%%%%%%%%%%%%%%%%%%%%%%%%%%%%%%%%%%%%%%%%%%%%%%%%%%%%%%%%%%%%%%%%%%%%%%
%%%%%%%%%%%%%%%%%%%%%%%%%%%%%%%%%%%%%%%%%%%%%%%%%%%%%%%%%%%%%%%%%%%%%%%%%%%%%%%%
\section{Introduction}

\LaTeX{} provides a mechanism to structure a large document (such as a book)
into a main file and several child files (containing the chapters)
using the |\include| command.
This mechanism is beneficial for documents
which span hundreds of pages in order to
make the source file(s) more manageable.
Moreover, compilation can be restricted to
selected child files by means of the |\includeonly| command.
The latter feature can be used to reduce the compilation time while editing
(this was significantly more useful in the earlier days of \LaTeX{})
or to generate a smaller document which is easier to navigate.
Another application of |\includeonly| is to generate
documents consisting of selected parts of the complete document.

However, there are a few drawbacks of the plain |\include| mechanism:
\begin{itemize}
\item
The child files cannot be compiled on their own,
they can only be compiled via the main file.
A naive editing environment
(such as a text editor with an option
to have the current file processed by \LaTeX)
may require one to switch to the main file before compiling;
attempting to compile the child file produces errors.
\item
The main file must be modified (each time)
to adjust the |\includeonly| command
to the present needs. This easily leaves the main file in a messy state.
\item
The generated document will always carry the filename
of the main document. This is inconvenient if
several child files are to be compiled and
to be kept for distribution.
\end{itemize}

The present package provides a simple interface
to make child files individually compilable by \LaTeX{}.
Compiling a child file then has the same effect as compiling
the main file with an |\includeonly| command
to select the appropriate child.
Moreover the generated document will carry the name of the child
rather than the main file.
This resolves all three above issues.

This feature is meant to make the editing of books,
thesis documents and lecture notes somewhat more convenient.
However, the package can also be used efficiently for
composing a series of documents (such as exercise sheets)
which are typically distributed individually.
It then assists the author in generating the individual documents
(potentially in different versions)
as well as a document containing the collected series.
Another application is in developing style files
or other kinds of included material
where compilation of the style file could redirect
to a sample or test file.

%%%%%%%%%%%%%%%%%%%%%%%%%%%%%%%%%%%%%%%%%%%%%%%%%%%%%%%%%%%%%%%%%%%%%%%%%%%%%%%%
%%%%%%%%%%%%%%%%%%%%%%%%%%%%%%%%%%%%%%%%%%%%%%%%%%%%%%%%%%%%%%%%%%%%%%%%%%%%%%%%
\section{Usage}

First of all, the package \textsf{childdoc} is \emph{not} a standard
\LaTeXe{} |.sty| style file! Therefore it needs to be invoked in
a non-standard way.

%%%%%%%%%%%%%%%%%%%%%%%%%%%%%%%%%%%%%%%%%%%%%%%%%%%%%%%%%%%%%%%%%%%%%%%%%%%%%%%%
\subsection{Included Files}
\label{sec:include}

%%%%%%%%%%%%%%%%%%%%%%%%%%%%%%%%%%%%%%%%
\DescribeMacro{\childdocmain}
To use the package, add the commands
\begin{center}
\begin{tabular}{l}
|\input{childdoc.def}|\\
|\childdocmain{}|\\
\end{tabular}
\end{center}
at the very top of the main \LaTeX{} file,
in particular \emph{before} the |\documentclass| statement!
The argument of |\childdocmain| should be left empty
(but it must be present).

%%%%%%%%%%%%%%%%%%%%%%%%%%%%%%%%%%%%%%%%
\DescribeMacro{\childdocof}
Furthermore, add the commands
\begin{center}
\begin{tabular}{l}
|\input{childdoc.def}|\\
|\childdocof{|\textit{main}|}|\\
\end{tabular}
\end{center}
at the top of every child file \textit{child}
which is included by |\include{|\textit{child}|}|
from within the main file
(or at least for those files to be compiled individually).
The argument \textit{main} must be the filename of the main file.

There are a couple of
considerations in setting up the main and child documents:

%%%%%%%%%%%%%%%%%%%%%%%%%%%%%%%%%%%%%%%%
\paragraph{Restrictions.}

Please note the following restrictions:
\begin{itemize}
\item
|\childdocmain| must be called with one argument \textit{main}
to ensure compatibility with earlier version of the package.
It must either be empty (|\childdocmain{}|)
or precisely match the filename of the main file in which it is specified.
See \secref{sec:detection} for further information.
\item
The filename \textit{main} must be specified without the |.tex| extension.
\item
The filename \textit{main} is case sensitive
(even in case-insensitive file systems)
due to internal string comparison.
\item
The argument \textit{main} should be fully expanded, it cannot be a macro.
\item
Subdirectories and special characters should be avoided in filenames.
\item
The command |\childdocmain{|\textit{main}|}| must be followed by a whitespace.
It should not be followed immediately by another command
or by a comment mark `|%|'.
This is because the \TeX{} parser reads the token immediately following
the argument of |\childdocmain| and puts it
at the beginning of every child section;
however, a white\-space is ignored.
\end{itemize}

%%%%%%%%%%%%%%%%%%%%%%%%%%%%%%%%%%%%%%%%
\paragraph{Content of Main File.}

It is advisable to place all content in the child files included by |\include|.
Any output contained in the main file will appear in all child documents
unless suppressed manually;
it cannot be suppressed automatically by the |\includeonly| directive
and thus should normally be avoided.
A method to include some content in the main file
by means of conditional processing is described in \secref{sec:conditional}.

%%%%%%%%%%%%%%%%%%%%%%%%%%%%%%%%%%%%%%%%
\paragraph{Page Numbering.}

When only a part of the document is compiled,
the appropriate numbering of pages
(as well as other status parameters)
is determined from the |.aux| files.
The latter contain information from previous passes.
However this information needs to propagate through
all intermediate child documents.
Therefore the page numbering in child documents may well
be inconsistent until the complete document is compiled at least once.

A useful (if unconventional) way to always ensure a consistent
page numbering is to restart the numbering in each child document
and denote the pages by `\textit{child}|.|\textit{page}'
where \textit{child} represents the chapter/section number of the child file.
This can be achieved by the command
|\numberwithin{page}{|\textit{child}|}|
of the \textsf{amsmath} package
where \textit{child} can be |chapter| or |section|
depending on the chosen structuring.
Alternatively, one can modify the macro |\thepage| appropriately
and reset the counter |page| at the start of each child file.

%%%%%%%%%%%%%%%%%%%%%%%%%%%%%%%%%%%%%%%%%%%%%%%%%%%%%%%%%%%%%%%%%%%%%%%%%%%%%%%%
\subsection{Conditional Processing}
\label{sec:conditional}

The package provides a mechanism to compile different versions
of a document. To customise the versions further some conditional processing
can come in handy to distinguish which version is being compiled.
The package provides two macros to describe the compilation context:

%%%%%%%%%%%%%%%%%%%%%%%%%%%%%%%%%%%%%%%%
\DescribeMacro{\ifchilddoc}
The conditional |\ifchilddoc| distinguishes between the compilation of
child documents and the main document:
%
\begin{center}
|\ifchilddoc |\textit{child-code}| |[|\||else |\textit{main-code}]| \||fi|
\end{center}

%%%%%%%%%%%%%%%%%%%%%%%%%%%%%%%%%%%%%%%%
\DescribeMacro{\childdocname}
\DescribeMacro{\childdocjob}
The macro |\childdocname| contains the filename (without extension)
of the main or child file being processed.
Note that |\childdocjob| will always contain the name of the main file.

%%%%%%%%%%%%%%%%%%%%%%%%%%%%%%%%%%%%%%%%
\paragraph{Title Page.}

Conditional processing can be used to include a title or banner page
in the main document when proper precautions are taken.
Importantly, the code in the main file should ensure that the page counter
(as well as other status parameters which are stored in the |.aux| files)
takes the same value after the conditional processing.
Otherwise the page numbers may take divergent values
depending on which part is compiled.

For example, a title page could be declared by:
%
\begin{center}
\begin{tabular}{l}
|\ifchilddoc\||else|\\
|\addtocounter{page}{-1}|\\
\textit{code for title page}\\
|\newpage|\\
|\||fi|
\end{tabular}
\end{center}
%
A banner page for the child documents can be generated by:
%
\begin{center}
\begin{tabular}{l}
|\ifchilddoc|\\
|\addtocounter{page}{-1}|\\
\textit{code for banner page}\\
|\newpage|\\
|\||fi|
\end{tabular}
\end{center}
%
Here one could write a message such as:
\begin{center}
|This is the part \childdocname{} of \childdocjob{}.|
\end{center}

%%%%%%%%%%%%%%%%%%%%%%%%%%%%%%%%%%%%%%%%%%%%%%%%%%%%%%%%%%%%%%%%%%%%%%%%%%%%%%%%
\subsection{Flags}
\label{sec:flags}

The package makes it easy to generate different versions
of the main or child documents.
To this end compilation flags can be defined
and assigned different default values.
They will be particularly useful in conjunction
with the forwarding mechanism described in \secref{sec:forward}.

For example, it may be useful to have a flag |\version|
which can be set to |draft| or |final|.
The document source will contain some conditional code
depending on the value of |\version|.
Suppose further, the flag should default to |final| for the main file
and to |draft| for child files
which is a natural assignment for editing the document.
This is achieved by placing the following code
in the preamble of the main document
(below the |\childdocmain| directive):
%
\begin{center}
\begin{tabular}{l}
|\ifchilddoc|\\
|\providecommand{\version}{draft}|\\
|\||else|\\
|\providecommand{\version}{final}|\\
|\||fi|
\end{tabular}
\end{center}
%
The definition by |\providecommand| makes sure
that previous definitions are not overwritten.
Further statements |\providecommand{\version}{...}|
can thus be added before the above code to override it.

For the main file, one might add a line
(between |\childdocmain| and the above block)
%
\begin{center}
|%\ifchilddoc\||else\providecommand{\version}{draft}\||fi|
\end{center}
%
which can be uncommented to produce a draft version.
Likewise one can add a line to the very top of a child file
(above the |\childdocof{|\textit{main}|}| directive)
%
\begin{center}
|%\providecommand{\version}{final}|
\end{center}
%
which can be uncommented to produce the final version of this child document.

%%%%%%%%%%%%%%%%%%%%%%%%%%%%%%%%%%%%%%%%%%%%%%%%%%%%%%%%%%%%%%%%%%%%%%%%%%%%%%%%
\subsection{Forwarding}
\label{sec:forward}

Different versions of the main or child documents
using compilation flags as described in \secref{sec:flags}
can be (permanently) stored in different files
for convenient compilation, viewing and distribution.
To this end, the package defines a command
to pass on compilation to a different file:

%%%%%%%%%%%%%%%%%%%%%%%%%%%%%%%%%%%%%%%%
\DescribeMacro{\childdocforward}
The command |\childdocforward| redirects processing to
another source file:
%
\begin{center}
\begin{tabular}{l}
|\input{childdoc.def}|\\
|\childdocforward[|\textit{main}|]{|\textit{dest}|}|\\
\end{tabular}
\end{center}
%
The argument \textit{dest} is the destination file
(without extension).
It should be the main file or one of the child files.
Note that further \textsf{childdoc} directives
such as |\childdocof| and |\childdocforward|
in the indicated file will be processed in this form.
The optional argument \textit{main}
passes on directly to the main file \textit{main}
while pretending to compile the child \textit{dest}.
This form behaves as if \textit{dest}
issues |\childdocof{|\textit{main}|}| right away,
and no further \textsf{childdoc} directives will be processed.

%%%%%%%%%%%%%%%%%%%%%%%%%%%%%%%%%%%%%%%%
\DescribeMacro{\...prefix}
In the alternative form |\childdocforwardprefix|,
%
\begin{center}
\begin{tabular}{l}
|\input{childdoc.def}|\\
|\childdocforwardprefix[|\textit{main}|]{|\textit{prefix}|}{|\textit{dest}|}|
\end{tabular}
\end{center}
%
the destination file is determined by a pattern
depending on the current file:
To make this work, the current file must be called
`{\textit{prefix}\hspace{0.2em}\textit{suffix}}'
with \textit{prefix} matching precisely the argument.
Processing is then passed on to the file
`{\textit{dest}\hspace{0.2em}\textit{suffix}}'.
Surely, the same effect is achieved by
directly specifying the
argument `{\textit{dest}\hspace{0.2em}\textit{suffix}}'
in the first form.
However, that requires to set up a different file
for each child. With the alternative form of the command
all these files can have exactly the same content
which simplifies setting them up and maintaining them.

For example, the following file |draft.tex|
with a compilation flag |\version| as described in \secref{sec:flags}
compiles the main document as a draft:
%
\begin{center}
\begin{tabular}{l}
|\def\version{draft}|\\
|\input{childdoc.def}|\\
|\childdocforward{|\textit{main}|}|
\end{tabular}
\end{center}
%
Likewise, the following files |final|\textit{nn}|.tex|
compile the final version of the child document
|child|\textit{nn}|.tex|:
%
\begin{center}
\begin{tabular}{l}
|\def\version{final}|\\
|\input{childdoc.def}|\\
|\childdocforwardprefix{final}{child}|
\end{tabular}
\end{center}
%

Note that when several versions of a main file and/or of each child file
are to be generated, it may be convenient to set up a |Makefile| or
shell script to automatise the process.

%%%%%%%%%%%%%%%%%%%%%%%%%%%%%%%%%%%%%%%%%%%%%%%%%%%%%%%%%%%%%%%%%%%%%%%%%%%%%%%%
\subsection{Command Line Processing}
\label{sec:commandline}

The effect of redirection files can also be achieved by invoking
the \LaTeX{} compiler with a more elaborate command line.
Most conveniently this should be done as part
of a shell script or a |Makefile|.

When using \textsf{childdoc} in the main file, the following
command lines effectively perform a redirection
(note that depending on the shell being used,
backslashes may have to be doubled: `|\|' $\to$ `|\\|'):
%
\begin{center}
|... -jobname "|\textit{target}|" |\\|"|[\textit{flags}]%
|\input{childdoc.def}\childdocforward[|\textit{main}|]{|\textit{dest}|}"|
\end{center}
%
Here \textit{target} is the name of the output file,
\textit{main} is the name of the main file
and \textit{dest} is the name of the main or child file to be processed
(all filenames without extensions).
The optional argument \textit{main} can be omitted
if \textit{main} matches \textit{dest}.
Optionally, compilation \textit{flags} can be defined via |\def| commands.
This command line makes the \TeX{} engine believe
it is compiling the file \textit{target}
whose content is specified as the latter parameter.
The provided code then forwards the processing to
\textit{main} or \textit{dest} as described in \secref{sec:forward}.

%%%%%%%%%%%%%%%%%%%%%%%%%%%%%%%%%%%%%%%%%%%%%%%%%%%%%%%%%%%%%%%%%%%%%%%%%%%%%%%%
\subsection{Include by Input}
\label{sec:input}

Including child documents by |\include| has some restrictions by design.
Most notably, the content of a child document always occupies
its own set of pages; pages cannot be shared between child documents.
Usually, this behaviour makes perfect sense
because each child document contain an essential part of the document.
However, in some situations it may be desirable to compose
a document from a collection of parts
without having mandatory page breaks between then.
For this case, the package
provides a mechanism to include parts
by |\input| which can also be processed individually.
However, by construction this mechanism
requires manual handling of the content to be output.

%%%%%%%%%%%%%%%%%%%%%%%%%%%%%%%%%%%%%%%%
\DescribeMacro{\ifchilddocmanual}
The main file should be prepared as usual, see \secref{sec:include}.
However, the document body must make a distinction
between processing of an individual part and of the main document, e.g.:
%
\begin{center}
\begin{tabular}{l}
|\ifchilddocmanual|\\
|\input{\childdocname}|\\
|\||else|\\
\textit{document body with }|\input{|\textit{part}|}|\\
|\||fi|
\end{tabular}
\end{center}
%
The conditional |\ifchilddocmanual| is true whenever
a part to be included by |\input| is being compiled,
and the name of the part is stored in |\childdocname|.

%%%%%%%%%%%%%%%%%%%%%%%%%%%%%%%%%%%%%%%%
\DescribeMacro{\childdocby}
Each part to be included by |\input| should start with:
%
\begin{center}
\begin{tabular}{l}
|\input{childdoc.def}|\\
|\childdocby{|\textit{main}|}|\\
\end{tabular}
\end{center}
%
The directive |\childdocby| is similar to |\childdocof|
described in \secref{sec:include},
but the subsequent selection of content must be done manually.
To that end, both |\ifchilddoc| and |\ifchilddocmanual|
will be true upon processing of a part,
and the name of the part is stored in |\childdocname|.
Note that |\jobname| will be set to the filename of the current part
so that each part receives an individual |.aux| file
that does not interfere with the |.aux| file(s) of the main document.
This behaviour can be altered by the alternative form
|\childdocby[*]{|\textit{main}|}| (with a non-empty optional argument)
which uses the |.aux| file of the main document
by setting |\jobname| to \textit{main}.

%%%%%%%%%%%%%%%%%%%%%%%%%%%%%%%%%%%%%%%%%%%%%%%%%%%%%%%%%%%%%%%%%%%%%%%%%%%%%%%%
\subsection{Driver Development}
\label{sec:driver}

The \textsf{childdoc} mechanism can also be use for the development
of definition files such as \LaTeX{} styles or classes.
This case differs from the above setup with multiple parts
included by |\include| in that no |\includeonly| should be invoked.
This can be achieved by starting the include file
(before |\ProvidesPackage|) with:
%
\begin{center}
\begin{tabular}{l}
|\input{childdoc.def}|\\
|\childdocforward{|\textit{main}|}|\\
\end{tabular}
\end{center}
%
or alternatively with:
%
\begin{center}
\begin{tabular}{l}
|\input{childdoc.def}|\\
|\childdocby{|\textit{main}|}|\\
\end{tabular}
\end{center}
%
Both forms have slightly different effects as described above.
The main file is prepared as usual, see \secref{sec:include}.

%%%%%%%%%%%%%%%%%%%%%%%%%%%%%%%%%%%%%%%%%%%%%%%%%%%%%%%%%%%%%%%%%%%%%%%%%%%%%%%%
\subsection{Legacy Detection}
\label{sec:detection}

The directive |\childdocmain| in the main file can detect
whether the complete document or merely a child is to be compiled
even without using the directive |\childdocof|.
This method is deprecated because it is less robust
and there is no compelling reason to use it;
it is merely provided for backward compatibility
and it may be removed in future versions.

If the detection mechanism is to be used,
it is mandatory to correctly specify
the filename of the main file as the argument of |\childdocmain|:
%
\begin{center}
\begin{tabular}{l}
|\input{childdoc.def}|\\
|\childdocmain{|\textit{main}|}|\\
\end{tabular}
\end{center}
%
If |\jobname| does not match the argument \textit{main} of |\childdocmain|,
it is assumed that |\jobname| points to the child file to be compiled.
When using |\childdocmain| with the main file specified as argument,
it suffices to start a child file
with just |\input{|\textit{main}|}|
without loading of the package and using |\childdocof|.
If instead all processing is done
with the appropriate \textsf{childdoc} directives,
the argument of \textit{main} of |\childdocmain| can be empty.

An alternative version of the command line processing described
in \secref{sec:commandline} using the detection mechanism reads:
%
\begin{center}
|... -jobname "|\textit{target}|" "|[\textit{flags}]%
[|\def\jobname{|\textit{dest}|}|]|\input{|\textit{main}|}"|
\end{center}

%%%%%%%%%%%%%%%%%%%%%%%%%%%%%%%%%%%%%%%%%%%%%%%%%%%%%%%%%%%%%%%%%%%%%%%%%%%%%%%%
\subsection{Manual Code}
\label{sec:manual}

In case one cannot be certain whether the definitions file |childdoc.def|
is installed on the target \TeX{} distribution
and one prefers not to ship it,
it is conceivable to paste a few relevant commands into the sources.

To that end, drop all statements |\input{childdoc.def}|
and perform the replacements as outlined below.
Instead of |\childdocmain{|\textit{main}|}| add the following code
to the top of the main file:
%
\begin{center}
\begin{tabular}{l}
|\||ifdefined\childdocname\endinput\||fi\newif\ifchilddoc|\\
|\edef\childdocname{\scantokens\expandafter{\jobname\noexpand}}|\\
|\def\childdocmain{|\textit{main}|}\||ifx\childdocmain\childdocname\||else|\\
|\childdoctrue\includeonly{\childdocname}\let\jobname\childdocmain\||fi|\\
\end{tabular}
\end{center}
%
Instead of |\childdocof{|\textit{main}|}| just include the main file
at the top of each child file:
%
\begin{center}
|\input{|\textit{main}|}|
\end{center}
%
A simple redirection |\childdocforward{|\textit{dest}|}| is achieved by:
%
\begin{center}
|\def\jobname{|\textit{dest}|}\input{\jobname}|
\end{center}
%
The redirection with prefix
|\childdocforwardprefix[|\textit{prefix}|]{|\textit{dest}|}|
is accomplished by:
%
\begin{center}
\begin{tabular}{l}
|{\edef\jobname{\scantokens\expandafter{\jobname\noexpand}}|\\
|\def\redirectjob |\textit{prefix}|#1~~~{\gdef\jobname{|\textit{dest}|#1}}|\\
|\expandafter\redirectjob\jobname~~~}\input{\jobname}|
\end{tabular}
\end{center}

In an alternative approach,
child documents can be compiled by a specific command line
without additional code or specific definitions:
%
\begin{center}
|... -jobname "|\textit{target}|" "|[\textit{flags}]%
|\includeonly{|\textit{dest}|}\input{|\textit{main}|}"|
\end{center}
%

%%%%%%%%%%%%%%%%%%%%%%%%%%%%%%%%%%%%%%%%%%%%%%%%%%%%%%%%%%%%%%%%%%%%%%%%%%%%%%%%
%%%%%%%%%%%%%%%%%%%%%%%%%%%%%%%%%%%%%%%%%%%%%%%%%%%%%%%%%%%%%%%%%%%%%%%%%%%%%%%%
\section{Information}

%%%%%%%%%%%%%%%%%%%%%%%%%%%%%%%%%%%%%%%%%%%%%%%%%%%%%%%%%%%%%%%%%%%%%%%%%%%%%%%%
\subsection{Copyright}

Copyright \copyright{} 2017--2018 Niklas Beisert

This work may be distributed and/or modified under the
conditions of the \LaTeX{} Project Public License, either version 1.3
of this license or (at your option) any later version.
The latest version of this license is in
  \url{http://www.latex-project.org/lppl.txt}
and version 1.3 or later is part of all distributions of \LaTeX{}
version 2005/12/01 or later.

This work has the LPPL maintenance status `maintained'.

The Current Maintainer of this work is Niklas Beisert.

This work consists of the files |README.txt|, |childdoc.ins| and |childdoc.dtx|
as well as the derived files |childdoc.def|, |cdocsamp.tex|
with |cdocsch1.tex|, |cdocsch2.tex|, |cdocspt3.tex|, |cdocspt4.tex|,
|cdocsdrf.tex|, |cdocsfn1.tex|, |cdocsfn2.tex|
as well as |childdoc.pdf|.

%%%%%%%%%%%%%%%%%%%%%%%%%%%%%%%%%%%%%%%%%%%%%%%%%%%%%%%%%%%%%%%%%%%%%%%%%%%%%%%%
\subsection{Files and Installation}

The package consists of the files:
%
\begin{center}
\begin{tabular}{ll}
    |README.txt|   & readme file \\
    |childdoc.ins| & installation file \\
    |childdoc.dtx| & source file \\
    |childdoc.def| & definition file \\
    |cdocsamp.tex| & sample main file \\
    |cdocsch1.tex| & sample include file \\
    |cdocsch2.tex| & sample include file \\
    |cdocspt3.tex| & sample part file \\
    |cdocspt4.tex| & sample part file \\
    |cdocsdrf.tex| & sample redirection file \\
    |cdocsfn1.tex| & sample redirection file \\
    |cdocsfn2.tex| & sample redirection file \\
    |childdoc.pdf| & manual
\end{tabular}
\end{center}
%
The distribution consists of the files
|README.txt|, |childdoc.ins| and |childdoc.dtx|.
%
\begin{itemize}
\item
Run (pdf)\LaTeX{} on |childdoc.dtx|
to compile the manual |childdoc.pdf| (this file).
\item
Run \LaTeX{} on |childdoc.ins| to create the definitions file |childdoc.def|
and the sample |cdocsamp.tex| with include files
|cdocsch1.tex|, |cdocsch2.tex|, |cdocspt3.tex|, |cdocspt4.tex|,
|cdocsdrf.tex|, |cdocsfn1.tex|, |cdocsfn2.tex|.
Then copy the file |childdoc.def| to an appropriate directory of your \LaTeX{}
distribution, e.g.\ \textit{texmf-root}|/tex/latex/childdoc|.
\end{itemize}

%%%%%%%%%%%%%%%%%%%%%%%%%%%%%%%%%%%%%%%%%%%%%%%%%%%%%%%%%%%%%%%%%%%%%%%%%%%%%%%%
\subsection{Related CTAN Packages}

There are several other packages which offer a similar functionality:
%
\begin{itemize}
\item
The packages
\href{http://ctan.org/pkg/docmute}{\textsf{docmute}},
\href{http://ctan.org/pkg/includex}{\textsf{includex}} and
\href{http://ctan.org/pkg/standalone}{\textsf{standalone}}
provide commands to include only the document body of
a child file thus allowing both files to be compiled individually.
\item
The packages \href{http://ctan.org/pkg/subdocs}{\textsf{subdocs}}
and \href{http://ctan.org/pkg/subfiles}{\textsf{subfiles}}
provide structures in which the main and child documents can be
encapsulated and allowing them to be compiled individually.
The inclusion mechanism is different from the conventional |\include|.
\item
The package \href{http://ctan.org/pkg/combine}{\textsf{combine}}
is an elaborate solution to combine several documents into one.
\end{itemize}
%
See also the CTAN topic \href{http://ctan.org/topic/subdocs}{\textsf{subdocs}}
for further related packages.
The present package differs from the above solutions in that
a document structure constructed with the conventional |\include| mechanism
just needs two extra commands at the top of every file
such that all constituent files can be compiled individually.

%%%%%%%%%%%%%%%%%%%%%%%%%%%%%%%%%%%%%%%%%%%%%%%%%%%%%%%%%%%%%%%%%%%%%%%%%%%%%%%%
%\subsection{Feature Suggestions}
%
%The following is a list of features which may be useful for future
%versions of this package:
%%
%\begin{itemize}
%\item
%\ldots
%\end{itemize}

%%%%%%%%%%%%%%%%%%%%%%%%%%%%%%%%%%%%%%%%%%%%%%%%%%%%%%%%%%%%%%%%%%%%%%%%%%%%%%%%
\subsection{Revision History}

%%%%%%%%%%%%%%%%%%%%%%%%%%%%%%%%%%%%%%%%
\paragraph{v2.0:} 2018/12/30

\begin{itemize}
\item
immediate forward processing
\item
added |\childdocby| mechanism
\item
manual restructured
\end{itemize}

%%%%%%%%%%%%%%%%%%%%%%%%%%%%%%%%%%%%%%%%
\paragraph{v1.6:} 2018/01/17

\begin{itemize}
\item
application for development of include files
\item
corrections to manual
\end{itemize}

%%%%%%%%%%%%%%%%%%%%%%%%%%%%%%%%%%%%%%%%
\paragraph{v1.5:} 2017/05/21

\begin{itemize}
\item
more complete structuring introduced
\item
|\childdocof| introduced
\item
|\childdoc| renamed to |\childdocmain|
\item
|\childredirect| renamed to |\childdocforward| and |\childdocforwardprefix|
and functionality expanded
\end{itemize}

%%%%%%%%%%%%%%%%%%%%%%%%%%%%%%%%%%%%%%%%
\paragraph{v1.0:} 2017/04/27

\begin{itemize}
\item
manual and install package
\item
first version published on CTAN
\end{itemize}

%%%%%%%%%%%%%%%%%%%%%%%%%%%%%%%%%%%%%%%%
\paragraph{v0.6:} 2017/04/26

\begin{itemize}
\item
redirection mechanism added
\end{itemize}

%%%%%%%%%%%%%%%%%%%%%%%%%%%%%%%%%%%%%%%%
\paragraph{v0.5:} 2017/04/26

\begin{itemize}
\item
functionality in definition file
\end{itemize}


%%%%%%%%%%%%%%%%%%%%%%%%%%%%%%%%%%%%%%%%%%%%%%%%%%%%%%%%%%%%%%%%%%%%%%%%%%%%%%%%
%%%%%%%%%%%%%%%%%%%%%%%%%%%%%%%%%%%%%%%%%%%%%%%%%%%%%%%%%%%%%%%%%%%%%%%%%%%%%%%%
%%%%%%%%%%%%%%%%%%%%%%%%%%%%%%%%%%%%%%%%%%%%%%%%%%%%%%%%%%%%%%%%%%%%%%%%%%%%%%%%
\appendix

\settowidth\MacroIndent{\rmfamily\scriptsize 000\ }

 \DocInput{childdoc.dtx}

\end{document}
%</driver>
% \fi
%
% %%%%%%%%%%%%%%%%%%%%%%%%%%%%%%%%%%%%%%%%%%%%%%%%%%%%%%%%%%%%%%%%%%%%%%%%%%%%%%
% %%%%%%%%%%%%%%%%%%%%%%%%%%%%%%%%%%%%%%%%%%%%%%%%%%%%%%%%%%%%%%%%%%%%%%%%%%%%%%
% \section{Sample}
%\iffalse
%<*samplemain>
%\fi
%
% The following presents a sample document
% with two chapters, two parts, a title page,
% a compile flag as well as three forwarding files to set the flag.
% It consists of eight |.tex| files:
% \begin{center}
% \begin{tabular}{ll}
% |cdocsamp.tex|&main file\\
% |cdocsch1.tex|&include file for chapter 1\\
% |cdocsch2.tex|&include file for chapter 2\\
% |cdocspt3.tex|&include file for part 3\\
% |cdocspt4.tex|&include file for part 4\\
% |cdocsdrf.tex|&forwarding file for main file in draft mode\\
% |cdocsfi1.tex|&forwarding file for final version of chapter 1\\
% |cdocsfi2.tex|&forwarding file for final version of chapter 2\\
% \end{tabular}
% \end{center}
% Each of the eight files can be compiled directly by the \LaTeX{} compiler.
%
% %%%%%%%%%%%%%%%%%%%%%%%%%%%%%%%%%%%%%%
% \paragraph{Main File.}
%
% The main file is called |cdocsamp.tex|.
%
% Load the \textsf{childdoc} definitions and
% declare the filename for the main document:
%    \begin{macrocode}
\input{childdoc.def}
\childdocmain{}
%    \end{macrocode}

% Optional override for |\version| flag:
%    \begin{macrocode}
%%\ifchilddoc\else\providecommand{\version}{draft}\fi
%    \end{macrocode}

% Define the default values for the |\version| flag
% (|final| for the main file and |draft| for childs):
%    \begin{macrocode}
\ifchilddoc
\providecommand{\version}{draft}
\else
\providecommand{\version}{final}
\fi
%    \end{macrocode}

% Load the standard document class:
%    \begin{macrocode}
\documentclass[12pt]{article}
%    \end{macrocode}

% Start the document body:
%    \begin{macrocode}
\begin{document}
%    \end{macrocode}

% Declare a title page.
% Print title, part of document being processed and version flag:
%    \begin{macrocode}
\addtocounter{page}{-1}
\begin{center}
{\LARGE\bfseries{}childdoc example\par}
\vspace{1cm}
\ifchilddoc
\ifchilddocmanual part\else chapter\fi:
`\childdocname' of `\childdocjob'\par
\else
main document: `\childdocjob'\par
\fi
version: \version\par
\end{center}
\newpage
%    \end{macrocode}

% Manually include selected file,
% otherwise process as usual:
%    \begin{macrocode}
\ifchilddocmanual
\section*{part `\childdocname'}
\input{\childdocname}
\else
%    \end{macrocode}

% Include the two chapters:
%    \begin{macrocode}
\include{cdocsch1}
\include{cdocsch2}
%    \end{macrocode}

% Include the two parts unless only chapters should be displayed:
%    \begin{macrocode}
\ifchilddoc\else
\section{part three}
\input{cdocspt3}
\section{part four}
\input{cdocspt4}
\fi
%    \end{macrocode}

% Process as usual until here:
%    \begin{macrocode}
\fi
%    \end{macrocode}

% End of document body:
%    \begin{macrocode}
\end{document}
%    \end{macrocode}
%\iffalse
%</samplemain>
%\fi
%
% %%%%%%%%%%%%%%%%%%%%%%%%%%%%%%%%%%%%%%
% \paragraph{Chapter Include Files.}
%
% The include files are called |cdocsch1.tex| and |cdocsch2.tex|.
%
%\iffalse
%<*samplechap1|samplechap2>
%\fi

% Optional override for |\version| flag:
%    \begin{macrocode}
%%\providecommand{\version}{final}
%    \end{macrocode}

% Include the main document:
%    \begin{macrocode}
\input{childdoc.def}
\childdocof{cdocsamp}
%    \end{macrocode}

%\iffalse
%</samplechap1|samplechap2>
%\fi
%
%\iffalse
%<*samplechap1>
%\fi
% Some text for chapter 1:
%    \begin{macrocode}
\section{one}
some text in chapter one
%    \end{macrocode}

%\iffalse
%</samplechap1>
%\fi
% Some text for chapter 2:
%\iffalse
%<*samplechap2>
%\fi
%    \begin{macrocode}
\section{two}
more text in chapter two
%    \end{macrocode}

%\iffalse
%</samplechap2>
%\fi
%
% %%%%%%%%%%%%%%%%%%%%%%%%%%%%%%%%%%%%%%
% \paragraph{Part Include Files.}
%
% The include files are called |cdocspt3.tex| and |cdocspt4.tex|.
%
%\iffalse
%<*samplepart3|samplepart4>
%\fi

% Optional override for |\version| flag:
%    \begin{macrocode}
%%\providecommand{\version}{final}
%    \end{macrocode}

% Include the main document:
%    \begin{macrocode}
\input{childdoc.def}
\childdocby{cdocsamp}
%    \end{macrocode}

%\iffalse
%</samplepart3|samplepart4>
%\fi
%
%\iffalse
%<*samplepart3>
%\fi
% Some text for part 3:
%    \begin{macrocode}
some text in part three
%    \end{macrocode}

%\iffalse
%</samplepart3>
%\fi
% Some text for part 4:
%\iffalse
%<*samplepart4>
%\fi
%    \begin{macrocode}
more text in part four
%    \end{macrocode}

%\iffalse
%</samplepart4>
%\fi
%
% %%%%%%%%%%%%%%%%%%%%%%%%%%%%%%%%%%%%%%
% \paragraph{Forwarding for a Complete Draft.}
%
% The following forwarding file |cdocsdrf.tex|
% compiles the main document in draft mode:
%\iffalse
%<*sampledraft>
%\fi
%    \begin{macrocode}
\def\version{draft}
\input{childdoc.def}
\childdocforward{cdocsamp}
%    \end{macrocode}

%\iffalse
%</sampledraft>
%\fi
%
% %%%%%%%%%%%%%%%%%%%%%%%%%%%%%%%%%%%%%%
% \paragraph{Forwarding for Final Version of the Chapters.}
%
% The following forwarding files |cdocsfn1.tex| and |cdocsfn2.tex|
% (with identical content)
% compile the final versions of the child documents
% |cdocsch1.tex| and |cdocsch2.tex|, respectively:
%\iffalse
%<*samplefinal>
%\fi
%    \begin{macrocode}
\def\version{final}
\input{childdoc.def}
\childdocforwardprefix[cdocsamp]{cdocsfn}{cdocsch}
%    \end{macrocode}

%\iffalse
%</samplefinal>
%\fi
%
% %%%%%%%%%%%%%%%%%%%%%%%%%%%%%%%%%%%%%%
% \paragraph{Command Line Processing.}
%
% The following three command lines generate the output files
% |cdocscld|, |cdocscl1| and |cdocscl2|
% which should be identical to
% |cdocsdrf|, |cdocsch1| and |cdocsfn2|, respectively:
% \begin{center}
% \begin{tabular}{l}
% |latex -jobname cdocscld \|\\
% |  "\def\version{draft}\input{childdoc.def}\childdocforward{cdocsamp}"|\\
% |latex -jobname cdocscl1 \|\\
% |  "\input{childdoc.def}\childdocforward[cdocsamp]{cdocsch1}"|\\
% |latex -jobname cdocscl2 \|\\
% |  "\def\version{final}\input{childdoc.def}\childdocforward{cdocsch2}"|
% \end{tabular}
% \end{center}
% Note that the trailing backslash on each first line
% merely continues the input to the second line
% (for convenient cut ant paste).
% Furthermore, the command |latex| can be replaced by any
% of its alternative versions such as |pdflatex|.
%
% %%%%%%%%%%%%%%%%%%%%%%%%%%%%%%%%%%%%%%%%%%%%%%%%%%%%%%%%%%%%%%%%%%%%%%%%%%%%%%
% %%%%%%%%%%%%%%%%%%%%%%%%%%%%%%%%%%%%%%%%%%%%%%%%%%%%%%%%%%%%%%%%%%%%%%%%%%%%%%
% \section{Implementation}
%\iffalse
%<*package>
%\fi
%
% This section describes the definitions file |childdoc.def|.

% The definitions cannot be loaded using |\usepackage| or |\RequirePackage|
% which has a mechanism to prevent loading a style file more than once.
% When loading the definitions by means of |\input|
% multiple instances have to be prevented manually:
%\iffalse
%This code needs to be before the `\ProvidesFile' directive
%which is defined at the beginning of this file.
%Therefore it is also placed there and commented out here.
%</package>
%<*discard>
%\fi
%    \begin{macrocode}
\ifdefined\childdocmain\endinput\fi
%    \end{macrocode}
%\iffalse
%</discard>
%<*package>
%\fi
%
% \macro{\ifchilddoc}
% \macro{\ifchilddocmanual}
% The conditional |\ifchilddoc| tells whether a
% child (true) or main (false) document is being compiled.
% The conditional |\ifchilddocmanual| tells whether
% the |\includeonly| mechanism is used (false) or
% the selection of child files must be performed manually (true).
% The definitions initialise to false:
%    \begin{macrocode}
\newif\ifchilddoc
\newif\ifchilddocmanual
%    \end{macrocode}

% \macro{\childdocname}
% \macro{\childdocjob}
% The macro |\childdocname| stores the name of the main document
% to be compiled. The macro |\childdocjob| stores the name of
% the document on which the \LaTeX{} compiler was originally invoked.
% The content of |\jobname| cannot be compared
% to filenames specified in the source due to different catcodes.
% The following code rescans |\jobname|, stores the result
% in |\childdocname| and saves a copy in |\childdocjob|:
%    \begin{macrocode}
\edef\childdocname{\scantokens\expandafter{\jobname\noexpand}}
\let\childdocjob\childdocname
%    \end{macrocode}

% \macro{\childdocdisable}
% The macro |\childdocdisable| prevents the main file
% from being processed more than once.
% At this stage, the main document command |\childdocmain|
% is assumed to be called once again where it should do nothing.
% Any subsequent call to it should prevent
% a secondary processing of the main document
% It overwrites the forwarding commands
% |\childdocof| and |\childdocforward|
% with empty macros to prevent further inclusions of the main document:
%    \begin{macrocode}
\newcommand{\childdocdisable}
{
  \renewcommand{\childdocmain}[1]{\renewcommand{\childdocmain}[1]{\endinput}}
  \renewcommand{\childdocof}[1]{}
  \renewcommand{\childdocby}[2][]{}
  \renewcommand{\childdocforward}[2][]{}
  \renewcommand{\childdocdisable}{}
}
%    \end{macrocode}

% \macro{\childdocmain}
% The macro |\childdocmain| is to be called at the top of the main file
% with nothing or the main filename (without extension) as argument.
% First, it breaks loops.
% If the argument is not empty and does not match |\childdocname|
% (which is set by the first inclusion of |childdoc.def|),
% |\ifchilddoc| is set to true, |\includeonly| is applied to the child file
% and |\jobname| is set to the main file
% (for proper handling of |.aux| files):
%    \begin{macrocode}
\newcommand{\childdocmain}[1]
{
  \childdocdisable\childdocmain{}
  \if?#1?\else
    \begingroup
      \def\childdoctmp{#1}
      \ifx\childdoctmp\childdocname
        \def\childdoctmp{}
      \else
        \def\childdoctmp
        {
          \childdoctrue
          \includeonly{\childdocname}
          \def\childdocjob{#1}
          \def\jobname{#1}
        }
      \fi
      \expandafter
    \endgroup
    \childdoctmp
  \fi
}
%    \end{macrocode}

% \macro{\childdocof}
% The command |\childdocof| redirects
% compilation to the main file |#1|.
%    \begin{macrocode}
\newcommand{\childdocof}[1]
{
  \childdocdisable
  \childdoctrue
  \includeonly{\childdocname}
  \def\jobname{#1}
  \def\childdocjob{#1}
  \input{#1}
}
%    \end{macrocode}

% \macro{\childdocby}
% The command |\childdocby| ....
%    \begin{macrocode}
\newcommand{\childdocby}[2][]
{
  \childdocdisable
  \childdoctrue
  \childdocmanualtrue
  \if?#1?\else
    \def\jobname{#2}
  \fi
  \def\childdocjob{#2}
  \input{#2}
  \endinput
}
%    \end{macrocode}

% \macro{\childdocforward}
% The command |\childdocforward| redirects
% compilation to the main file or
% (if the optional argument is given) a child file.
% Parameters are set as if the main file
% or a child file starting with |\childdocof| was compiled.
% Then compilation is handed over to the main file:
%    \begin{macrocode}
\newcommand{\childdocforward}[2][]
{
  \begingroup
    \if?#1?
      \def\childdoctmp
      {
        \def\childdocname{#2}
        \def\childdocjob{#2}
        \def\jobname{#2}
        \input{#2}
        \endinput
      }
    \else
      \def\childdoctmp
      {
        \childdocdisable
        \def\childdocname{#2}
        \childdoctrue
        \includeonly{#2}
        \def\childdocjob{#1}
        \def\jobname{#1}
        \input{#1}
        \endinput
      }
    \fi
    \expandafter
  \endgroup
  \childdoctmp
}
%    \end{macrocode}

% \macro{\childdocforwardprefix}
% The command |\childdocforwardprefix| redirects
% compilation to the main or a child file by means of a pattern.
% The prefix |#1| in the current filename is replaced by |#2|
% and the suffix of the current filename is kept
% (it is assumed that the filename does not contain the substring `|~~~|'
% which is used as a delimiter).
% Compilation is handed over to the new file by |\childdocforward|:
%    \begin{macrocode}
\newcommand{\childdocforwardprefix}[3][]
{
  \begingroup
    \def\childdocextract #2##1~~~{\def\childdoctmp{\childdocforward[#1]{#3##1}}}
    \expandafter\childdocextract\childdocname~~~
    \expandafter
  \endgroup
  \childdoctmp
}
%    \end{macrocode}

% \macro{\childdoc}
% The deprecated macro |\childdoc| is a legacy version of |\childdocmain|:
%    \begin{macrocode}
\newcommand{\childdoc}{\childdocmain}
%    \end{macrocode}

% \macro{\childdocredirect}
% The deprecated macro |\childdocredirect| is a legacy version
% of |\childdocforward| and |\childdocforwardprefix|:
%    \begin{macrocode}
\newcommand{\childdocredirect}[2][]
{
  \begingroup
    \if?#1?
      \def\childdoctmp{\childdocforward{#2}}
    \else
      \def\childdoctmp{\childdocforwardprefix{#1}{#2}}
    \fi
    \expandafter
  \endgroup
  \childdoctmp
}
%    \end{macrocode}

%\iffalse
%</package>
%\fi
%
\endinput
\childdocforward{cdocsch2}"|
% \end{tabular}
% \end{center}
% Note that the trailing backslash on each first line
% merely continues the input to the second line
% (for convenient cut ant paste).
% Furthermore, the command |latex| can be replaced by any
% of its alternative versions such as |pdflatex|.
%
% %%%%%%%%%%%%%%%%%%%%%%%%%%%%%%%%%%%%%%%%%%%%%%%%%%%%%%%%%%%%%%%%%%%%%%%%%%%%%%
% %%%%%%%%%%%%%%%%%%%%%%%%%%%%%%%%%%%%%%%%%%%%%%%%%%%%%%%%%%%%%%%%%%%%%%%%%%%%%%
% \section{Implementation}
%\iffalse
%<*package>
%\fi
%
% This section describes the definitions file |childdoc.def|.

% The definitions cannot be loaded using |\usepackage| or |\RequirePackage|
% which has a mechanism to prevent loading a style file more than once.
% When loading the definitions by means of |\input|
% multiple instances have to be prevented manually:
%\iffalse
%This code needs to be before the `\ProvidesFile' directive
%which is defined at the beginning of this file.
%Therefore it is also placed there and commented out here.
%</package>
%<*discard>
%\fi
%    \begin{macrocode}
\ifdefined\childdocmain\endinput\fi
%    \end{macrocode}
%\iffalse
%</discard>
%<*package>
%\fi
%
% \macro{\ifchilddoc}
% \macro{\ifchilddocmanual}
% The conditional |\ifchilddoc| tells whether a
% child (true) or main (false) document is being compiled.
% The conditional |\ifchilddocmanual| tells whether
% the |\includeonly| mechanism is used (false) or
% the selection of child files must be performed manually (true).
% The definitions initialise to false:
%    \begin{macrocode}
\newif\ifchilddoc
\newif\ifchilddocmanual
%    \end{macrocode}

% \macro{\childdocname}
% \macro{\childdocjob}
% The macro |\childdocname| stores the name of the main document
% to be compiled. The macro |\childdocjob| stores the name of
% the document on which the \LaTeX{} compiler was originally invoked.
% The content of |\jobname| cannot be compared
% to filenames specified in the source due to different catcodes.
% The following code rescans |\jobname|, stores the result
% in |\childdocname| and saves a copy in |\childdocjob|:
%    \begin{macrocode}
\edef\childdocname{\scantokens\expandafter{\jobname\noexpand}}
\let\childdocjob\childdocname
%    \end{macrocode}

% \macro{\childdocdisable}
% The macro |\childdocdisable| prevents the main file
% from being processed more than once.
% At this stage, the main document command |\childdocmain|
% is assumed to be called once again where it should do nothing.
% Any subsequent call to it should prevent
% a secondary processing of the main document
% It overwrites the forwarding commands
% |\childdocof| and |\childdocforward|
% with empty macros to prevent further inclusions of the main document:
%    \begin{macrocode}
\newcommand{\childdocdisable}
{
  \renewcommand{\childdocmain}[1]{\renewcommand{\childdocmain}[1]{\endinput}}
  \renewcommand{\childdocof}[1]{}
  \renewcommand{\childdocby}[2][]{}
  \renewcommand{\childdocforward}[2][]{}
  \renewcommand{\childdocdisable}{}
}
%    \end{macrocode}

% \macro{\childdocmain}
% The macro |\childdocmain| is to be called at the top of the main file
% with nothing or the main filename (without extension) as argument.
% First, it breaks loops.
% If the argument is not empty and does not match |\childdocname|
% (which is set by the first inclusion of |childdoc.def|),
% |\ifchilddoc| is set to true, |\includeonly| is applied to the child file
% and |\jobname| is set to the main file
% (for proper handling of |.aux| files):
%    \begin{macrocode}
\newcommand{\childdocmain}[1]
{
  \childdocdisable\childdocmain{}
  \if?#1?\else
    \begingroup
      \def\childdoctmp{#1}
      \ifx\childdoctmp\childdocname
        \def\childdoctmp{}
      \else
        \def\childdoctmp
        {
          \childdoctrue
          \includeonly{\childdocname}
          \def\childdocjob{#1}
          \def\jobname{#1}
        }
      \fi
      \expandafter
    \endgroup
    \childdoctmp
  \fi
}
%    \end{macrocode}

% \macro{\childdocof}
% The command |\childdocof| redirects
% compilation to the main file |#1|.
%    \begin{macrocode}
\newcommand{\childdocof}[1]
{
  \childdocdisable
  \childdoctrue
  \includeonly{\childdocname}
  \def\jobname{#1}
  \def\childdocjob{#1}
  \input{#1}
}
%    \end{macrocode}

% \macro{\childdocby}
% The command |\childdocby| ....
%    \begin{macrocode}
\newcommand{\childdocby}[2][]
{
  \childdocdisable
  \childdoctrue
  \childdocmanualtrue
  \if?#1?\else
    \def\jobname{#2}
  \fi
  \def\childdocjob{#2}
  \input{#2}
  \endinput
}
%    \end{macrocode}

% \macro{\childdocforward}
% The command |\childdocforward| redirects
% compilation to the main file or
% (if the optional argument is given) a child file.
% Parameters are set as if the main file
% or a child file starting with |\childdocof| was compiled.
% Then compilation is handed over to the main file:
%    \begin{macrocode}
\newcommand{\childdocforward}[2][]
{
  \begingroup
    \if?#1?
      \def\childdoctmp
      {
        \def\childdocname{#2}
        \def\childdocjob{#2}
        \def\jobname{#2}
        \input{#2}
        \endinput
      }
    \else
      \def\childdoctmp
      {
        \childdocdisable
        \def\childdocname{#2}
        \childdoctrue
        \includeonly{#2}
        \def\childdocjob{#1}
        \def\jobname{#1}
        \input{#1}
        \endinput
      }
    \fi
    \expandafter
  \endgroup
  \childdoctmp
}
%    \end{macrocode}

% \macro{\childdocforwardprefix}
% The command |\childdocforwardprefix| redirects
% compilation to the main or a child file by means of a pattern.
% The prefix |#1| in the current filename is replaced by |#2|
% and the suffix of the current filename is kept
% (it is assumed that the filename does not contain the substring `|~~~|'
% which is used as a delimiter).
% Compilation is handed over to the new file by |\childdocforward|:
%    \begin{macrocode}
\newcommand{\childdocforwardprefix}[3][]
{
  \begingroup
    \def\childdocextract #2##1~~~{\def\childdoctmp{\childdocforward[#1]{#3##1}}}
    \expandafter\childdocextract\childdocname~~~
    \expandafter
  \endgroup
  \childdoctmp
}
%    \end{macrocode}

% \macro{\childdoc}
% The deprecated macro |\childdoc| is a legacy version of |\childdocmain|:
%    \begin{macrocode}
\newcommand{\childdoc}{\childdocmain}
%    \end{macrocode}

% \macro{\childdocredirect}
% The deprecated macro |\childdocredirect| is a legacy version
% of |\childdocforward| and |\childdocforwardprefix|:
%    \begin{macrocode}
\newcommand{\childdocredirect}[2][]
{
  \begingroup
    \if?#1?
      \def\childdoctmp{\childdocforward{#2}}
    \else
      \def\childdoctmp{\childdocforwardprefix{#1}{#2}}
    \fi
    \expandafter
  \endgroup
  \childdoctmp
}
%    \end{macrocode}

%\iffalse
%</package>
%\fi
%
\endinput
|
and perform the replacements as outlined below.
Instead of |\childdocmain{|\textit{main}|}| add the following code
to the top of the main file:
%
\begin{center}
\begin{tabular}{l}
|\||ifdefined\childdocname\endinput\||fi\newif\ifchilddoc|\\
|\edef\childdocname{\scantokens\expandafter{\jobname\noexpand}}|\\
|\def\childdocmain{|\textit{main}|}\||ifx\childdocmain\childdocname\||else|\\
|\childdoctrue\includeonly{\childdocname}\let\jobname\childdocmain\||fi|\\
\end{tabular}
\end{center}
%
Instead of |\childdocof{|\textit{main}|}| just include the main file
at the top of each child file:
%
\begin{center}
|\input{|\textit{main}|}|
\end{center}
%
A simple redirection |\childdocforward{|\textit{dest}|}| is achieved by:
%
\begin{center}
|\def\jobname{|\textit{dest}|}\input{\jobname}|
\end{center}
%
The redirection with prefix
|\childdocforwardprefix[|\textit{prefix}|]{|\textit{dest}|}|
is accomplished by:
%
\begin{center}
\begin{tabular}{l}
|{\edef\jobname{\scantokens\expandafter{\jobname\noexpand}}|\\
|\def\redirectjob |\textit{prefix}|#1~~~{\gdef\jobname{|\textit{dest}|#1}}|\\
|\expandafter\redirectjob\jobname~~~}\input{\jobname}|
\end{tabular}
\end{center}

In an alternative approach,
child documents can be compiled by a specific command line
without additional code or specific definitions:
%
\begin{center}
|... -jobname "|\textit{target}|" "|[\textit{flags}]%
|\includeonly{|\textit{dest}|}\input{|\textit{main}|}"|
\end{center}
%

%%%%%%%%%%%%%%%%%%%%%%%%%%%%%%%%%%%%%%%%%%%%%%%%%%%%%%%%%%%%%%%%%%%%%%%%%%%%%%%%
%%%%%%%%%%%%%%%%%%%%%%%%%%%%%%%%%%%%%%%%%%%%%%%%%%%%%%%%%%%%%%%%%%%%%%%%%%%%%%%%
\section{Information}

%%%%%%%%%%%%%%%%%%%%%%%%%%%%%%%%%%%%%%%%%%%%%%%%%%%%%%%%%%%%%%%%%%%%%%%%%%%%%%%%
\subsection{Copyright}

Copyright \copyright{} 2017--2018 Niklas Beisert

This work may be distributed and/or modified under the
conditions of the \LaTeX{} Project Public License, either version 1.3
of this license or (at your option) any later version.
The latest version of this license is in
  \url{http://www.latex-project.org/lppl.txt}
and version 1.3 or later is part of all distributions of \LaTeX{}
version 2005/12/01 or later.

This work has the LPPL maintenance status `maintained'.

The Current Maintainer of this work is Niklas Beisert.

This work consists of the files |README.txt|, |childdoc.ins| and |childdoc.dtx|
as well as the derived files |childdoc.def|, |cdocsamp.tex|
with |cdocsch1.tex|, |cdocsch2.tex|, |cdocspt3.tex|, |cdocspt4.tex|,
|cdocsdrf.tex|, |cdocsfn1.tex|, |cdocsfn2.tex|
as well as |childdoc.pdf|.

%%%%%%%%%%%%%%%%%%%%%%%%%%%%%%%%%%%%%%%%%%%%%%%%%%%%%%%%%%%%%%%%%%%%%%%%%%%%%%%%
\subsection{Files and Installation}

The package consists of the files:
%
\begin{center}
\begin{tabular}{ll}
    |README.txt|   & readme file \\
    |childdoc.ins| & installation file \\
    |childdoc.dtx| & source file \\
    |childdoc.def| & definition file \\
    |cdocsamp.tex| & sample main file \\
    |cdocsch1.tex| & sample include file \\
    |cdocsch2.tex| & sample include file \\
    |cdocspt3.tex| & sample part file \\
    |cdocspt4.tex| & sample part file \\
    |cdocsdrf.tex| & sample redirection file \\
    |cdocsfn1.tex| & sample redirection file \\
    |cdocsfn2.tex| & sample redirection file \\
    |childdoc.pdf| & manual
\end{tabular}
\end{center}
%
The distribution consists of the files
|README.txt|, |childdoc.ins| and |childdoc.dtx|.
%
\begin{itemize}
\item
Run (pdf)\LaTeX{} on |childdoc.dtx|
to compile the manual |childdoc.pdf| (this file).
\item
Run \LaTeX{} on |childdoc.ins| to create the definitions file |childdoc.def|
and the sample |cdocsamp.tex| with include files
|cdocsch1.tex|, |cdocsch2.tex|, |cdocspt3.tex|, |cdocspt4.tex|,
|cdocsdrf.tex|, |cdocsfn1.tex|, |cdocsfn2.tex|.
Then copy the file |childdoc.def| to an appropriate directory of your \LaTeX{}
distribution, e.g.\ \textit{texmf-root}|/tex/latex/childdoc|.
\end{itemize}

%%%%%%%%%%%%%%%%%%%%%%%%%%%%%%%%%%%%%%%%%%%%%%%%%%%%%%%%%%%%%%%%%%%%%%%%%%%%%%%%
\subsection{Related CTAN Packages}

There are several other packages which offer a similar functionality:
%
\begin{itemize}
\item
The packages
\href{http://ctan.org/pkg/docmute}{\textsf{docmute}},
\href{http://ctan.org/pkg/includex}{\textsf{includex}} and
\href{http://ctan.org/pkg/standalone}{\textsf{standalone}}
provide commands to include only the document body of
a child file thus allowing both files to be compiled individually.
\item
The packages \href{http://ctan.org/pkg/subdocs}{\textsf{subdocs}}
and \href{http://ctan.org/pkg/subfiles}{\textsf{subfiles}}
provide structures in which the main and child documents can be
encapsulated and allowing them to be compiled individually.
The inclusion mechanism is different from the conventional |\include|.
\item
The package \href{http://ctan.org/pkg/combine}{\textsf{combine}}
is an elaborate solution to combine several documents into one.
\end{itemize}
%
See also the CTAN topic \href{http://ctan.org/topic/subdocs}{\textsf{subdocs}}
for further related packages.
The present package differs from the above solutions in that
a document structure constructed with the conventional |\include| mechanism
just needs two extra commands at the top of every file
such that all constituent files can be compiled individually.

%%%%%%%%%%%%%%%%%%%%%%%%%%%%%%%%%%%%%%%%%%%%%%%%%%%%%%%%%%%%%%%%%%%%%%%%%%%%%%%%
%\subsection{Feature Suggestions}
%
%The following is a list of features which may be useful for future
%versions of this package:
%%
%\begin{itemize}
%\item
%\ldots
%\end{itemize}

%%%%%%%%%%%%%%%%%%%%%%%%%%%%%%%%%%%%%%%%%%%%%%%%%%%%%%%%%%%%%%%%%%%%%%%%%%%%%%%%
\subsection{Revision History}

%%%%%%%%%%%%%%%%%%%%%%%%%%%%%%%%%%%%%%%%
\paragraph{v2.0:} 2018/12/30

\begin{itemize}
\item
immediate forward processing
\item
added |\childdocby| mechanism
\item
manual restructured
\end{itemize}

%%%%%%%%%%%%%%%%%%%%%%%%%%%%%%%%%%%%%%%%
\paragraph{v1.6:} 2018/01/17

\begin{itemize}
\item
application for development of include files
\item
corrections to manual
\end{itemize}

%%%%%%%%%%%%%%%%%%%%%%%%%%%%%%%%%%%%%%%%
\paragraph{v1.5:} 2017/05/21

\begin{itemize}
\item
more complete structuring introduced
\item
|\childdocof| introduced
\item
|\childdoc| renamed to |\childdocmain|
\item
|\childredirect| renamed to |\childdocforward| and |\childdocforwardprefix|
and functionality expanded
\end{itemize}

%%%%%%%%%%%%%%%%%%%%%%%%%%%%%%%%%%%%%%%%
\paragraph{v1.0:} 2017/04/27

\begin{itemize}
\item
manual and install package
\item
first version published on CTAN
\end{itemize}

%%%%%%%%%%%%%%%%%%%%%%%%%%%%%%%%%%%%%%%%
\paragraph{v0.6:} 2017/04/26

\begin{itemize}
\item
redirection mechanism added
\end{itemize}

%%%%%%%%%%%%%%%%%%%%%%%%%%%%%%%%%%%%%%%%
\paragraph{v0.5:} 2017/04/26

\begin{itemize}
\item
functionality in definition file
\end{itemize}


%%%%%%%%%%%%%%%%%%%%%%%%%%%%%%%%%%%%%%%%%%%%%%%%%%%%%%%%%%%%%%%%%%%%%%%%%%%%%%%%
%%%%%%%%%%%%%%%%%%%%%%%%%%%%%%%%%%%%%%%%%%%%%%%%%%%%%%%%%%%%%%%%%%%%%%%%%%%%%%%%
%%%%%%%%%%%%%%%%%%%%%%%%%%%%%%%%%%%%%%%%%%%%%%%%%%%%%%%%%%%%%%%%%%%%%%%%%%%%%%%%
\appendix

\settowidth\MacroIndent{\rmfamily\scriptsize 000\ }

 \DocInput{childdoc.dtx}

\end{document}
%</driver>
% \fi
%
% %%%%%%%%%%%%%%%%%%%%%%%%%%%%%%%%%%%%%%%%%%%%%%%%%%%%%%%%%%%%%%%%%%%%%%%%%%%%%%
% %%%%%%%%%%%%%%%%%%%%%%%%%%%%%%%%%%%%%%%%%%%%%%%%%%%%%%%%%%%%%%%%%%%%%%%%%%%%%%
% \section{Sample}
%\iffalse
%<*samplemain>
%\fi
%
% The following presents a sample document
% with two chapters, two parts, a title page,
% a compile flag as well as three forwarding files to set the flag.
% It consists of eight |.tex| files:
% \begin{center}
% \begin{tabular}{ll}
% |cdocsamp.tex|&main file\\
% |cdocsch1.tex|&include file for chapter 1\\
% |cdocsch2.tex|&include file for chapter 2\\
% |cdocspt3.tex|&include file for part 3\\
% |cdocspt4.tex|&include file for part 4\\
% |cdocsdrf.tex|&forwarding file for main file in draft mode\\
% |cdocsfi1.tex|&forwarding file for final version of chapter 1\\
% |cdocsfi2.tex|&forwarding file for final version of chapter 2\\
% \end{tabular}
% \end{center}
% Each of the eight files can be compiled directly by the \LaTeX{} compiler.
%
% %%%%%%%%%%%%%%%%%%%%%%%%%%%%%%%%%%%%%%
% \paragraph{Main File.}
%
% The main file is called |cdocsamp.tex|.
%
% Load the \textsf{childdoc} definitions and
% declare the filename for the main document:
%    \begin{macrocode}
% \iffalse
%
% childdoc.dtx Copyright (C) 2017-2018 Niklas Beisert
%
% This work may be distributed and/or modified under the
% conditions of the LaTeX Project Public License, either version 1.3
% of this license or (at your option) any later version.
% The latest version of this license is in
%   http://www.latex-project.org/lppl.txt
% and version 1.3 or later is part of all distributions of LaTeX
% version 2005/12/01 or later.
%
% This work has the LPPL maintenance status `maintained'.
%
% The Current Maintainer of this work is Niklas Beisert.
%
% This work consists of the files childdoc.dtx and childdoc.ins
% and the derived files childdoc.def and cdocsamp.tex with
% cdocsch1.tex, cdocsch2.tex, cdocsdrf.tex, cdocsfn1.tex, cdocsfn2.tex.
%
%<package>\ifdefined\childdocmain\endinput\fi
%<package>\ProvidesFile{childdoc.def}[2018/12/30 v2.0 child document driver]
%<samplemain>\ProvidesFile{cdocsamp.tex}[2018/12/30 v2.0 sample for childdoc]
%<*driver>
%\ProvidesFile{childdoc.drv}[2018/12/30 v2.0 childdoc reference manual file]
\PassOptionsToClass{10pt,a4paper}{article}
\documentclass{ltxdoc}

\usepackage[margin=35mm]{geometry}
\usepackage{hyperref}
\usepackage{hyperxmp}
\usepackage[usenames]{color}

\hypersetup{colorlinks=true}
\hypersetup{pdfstartview=FitH}
\hypersetup{pdfpagemode=UseNone}
\hypersetup{pdfsource={}}
\hypersetup{pdflang={en-UK}}
\hypersetup{pdfcopyright={Copyright 2017-2018 Niklas Beisert.
  This work may be distributed and/or modified under the
  conditions of the LaTeX Project Public License, either version 1.3
  of this license or (at your option) any later version.}}
\hypersetup{pdflicenseurl={http://www.latex-project.org/lppl.txt}}
\hypersetup{pdfcontactaddress={ETH Zurich, ITP, HIT K,
  Wolfgang-Pauli-Strasse 27}}
\hypersetup{pdfcontactpostcode={8093}}
\hypersetup{pdfcontactcity={Zurich}}
\hypersetup{pdfcontactcountry={Switzerland}}
\hypersetup{pdfcontactemail={nbeisert@itp.phys.ethz.ch}}
\hypersetup{pdfcontacturl={http://people.phys.ethz.ch/\xmptilde nbeisert/}}

\newcommand{\secref}[1]{\hyperref[#1]{section \ref*{#1}}}

\parskip1ex
\parindent0pt
\let\olditemize\itemize
\def\itemize{\olditemize\parskip0pt}

\begin{document}

\title{The \textsf{childdoc} Package}
\hypersetup{pdftitle={The childdoc Package}}
\author{Niklas Beisert\\[2ex]
  Institut f\"ur Theoretische Physik\\
  Eidgen\"ossische Technische Hochschule Z\"urich\\
  Wolfgang-Pauli-Strasse 27, 8093 Z\"urich, Switzerland\\[1ex]
  \href{mailto:nbeisert@itp.phys.ethz.ch}
  {\texttt{nbeisert@itp.phys.ethz.ch}}}
\hypersetup{pdfauthor={Niklas Beisert}}
\hypersetup{pdfsubject={Manual for the LaTeX2e Package childdoc}}
\date{30 December 2018, \textsf{v2.0}}
\maketitle

\begin{abstract}\noindent
\textsf{childdoc} is a \LaTeXe{} package
that enables the direct compilation
of document sections included by |\include|
to individual files.
\end{abstract}

\begingroup
\parskip0ex
\tableofcontents
\endgroup

%%%%%%%%%%%%%%%%%%%%%%%%%%%%%%%%%%%%%%%%%%%%%%%%%%%%%%%%%%%%%%%%%%%%%%%%%%%%%%%%
%%%%%%%%%%%%%%%%%%%%%%%%%%%%%%%%%%%%%%%%%%%%%%%%%%%%%%%%%%%%%%%%%%%%%%%%%%%%%%%%
\section{Introduction}

\LaTeX{} provides a mechanism to structure a large document (such as a book)
into a main file and several child files (containing the chapters)
using the |\include| command.
This mechanism is beneficial for documents
which span hundreds of pages in order to
make the source file(s) more manageable.
Moreover, compilation can be restricted to
selected child files by means of the |\includeonly| command.
The latter feature can be used to reduce the compilation time while editing
(this was significantly more useful in the earlier days of \LaTeX{})
or to generate a smaller document which is easier to navigate.
Another application of |\includeonly| is to generate
documents consisting of selected parts of the complete document.

However, there are a few drawbacks of the plain |\include| mechanism:
\begin{itemize}
\item
The child files cannot be compiled on their own,
they can only be compiled via the main file.
A naive editing environment
(such as a text editor with an option
to have the current file processed by \LaTeX)
may require one to switch to the main file before compiling;
attempting to compile the child file produces errors.
\item
The main file must be modified (each time)
to adjust the |\includeonly| command
to the present needs. This easily leaves the main file in a messy state.
\item
The generated document will always carry the filename
of the main document. This is inconvenient if
several child files are to be compiled and
to be kept for distribution.
\end{itemize}

The present package provides a simple interface
to make child files individually compilable by \LaTeX{}.
Compiling a child file then has the same effect as compiling
the main file with an |\includeonly| command
to select the appropriate child.
Moreover the generated document will carry the name of the child
rather than the main file.
This resolves all three above issues.

This feature is meant to make the editing of books,
thesis documents and lecture notes somewhat more convenient.
However, the package can also be used efficiently for
composing a series of documents (such as exercise sheets)
which are typically distributed individually.
It then assists the author in generating the individual documents
(potentially in different versions)
as well as a document containing the collected series.
Another application is in developing style files
or other kinds of included material
where compilation of the style file could redirect
to a sample or test file.

%%%%%%%%%%%%%%%%%%%%%%%%%%%%%%%%%%%%%%%%%%%%%%%%%%%%%%%%%%%%%%%%%%%%%%%%%%%%%%%%
%%%%%%%%%%%%%%%%%%%%%%%%%%%%%%%%%%%%%%%%%%%%%%%%%%%%%%%%%%%%%%%%%%%%%%%%%%%%%%%%
\section{Usage}

First of all, the package \textsf{childdoc} is \emph{not} a standard
\LaTeXe{} |.sty| style file! Therefore it needs to be invoked in
a non-standard way.

%%%%%%%%%%%%%%%%%%%%%%%%%%%%%%%%%%%%%%%%%%%%%%%%%%%%%%%%%%%%%%%%%%%%%%%%%%%%%%%%
\subsection{Included Files}
\label{sec:include}

%%%%%%%%%%%%%%%%%%%%%%%%%%%%%%%%%%%%%%%%
\DescribeMacro{\childdocmain}
To use the package, add the commands
\begin{center}
\begin{tabular}{l}
|% \iffalse
%
% childdoc.dtx Copyright (C) 2017-2018 Niklas Beisert
%
% This work may be distributed and/or modified under the
% conditions of the LaTeX Project Public License, either version 1.3
% of this license or (at your option) any later version.
% The latest version of this license is in
%   http://www.latex-project.org/lppl.txt
% and version 1.3 or later is part of all distributions of LaTeX
% version 2005/12/01 or later.
%
% This work has the LPPL maintenance status `maintained'.
%
% The Current Maintainer of this work is Niklas Beisert.
%
% This work consists of the files childdoc.dtx and childdoc.ins
% and the derived files childdoc.def and cdocsamp.tex with
% cdocsch1.tex, cdocsch2.tex, cdocsdrf.tex, cdocsfn1.tex, cdocsfn2.tex.
%
%<package>\ifdefined\childdocmain\endinput\fi
%<package>\ProvidesFile{childdoc.def}[2018/12/30 v2.0 child document driver]
%<samplemain>\ProvidesFile{cdocsamp.tex}[2018/12/30 v2.0 sample for childdoc]
%<*driver>
%\ProvidesFile{childdoc.drv}[2018/12/30 v2.0 childdoc reference manual file]
\PassOptionsToClass{10pt,a4paper}{article}
\documentclass{ltxdoc}

\usepackage[margin=35mm]{geometry}
\usepackage{hyperref}
\usepackage{hyperxmp}
\usepackage[usenames]{color}

\hypersetup{colorlinks=true}
\hypersetup{pdfstartview=FitH}
\hypersetup{pdfpagemode=UseNone}
\hypersetup{pdfsource={}}
\hypersetup{pdflang={en-UK}}
\hypersetup{pdfcopyright={Copyright 2017-2018 Niklas Beisert.
  This work may be distributed and/or modified under the
  conditions of the LaTeX Project Public License, either version 1.3
  of this license or (at your option) any later version.}}
\hypersetup{pdflicenseurl={http://www.latex-project.org/lppl.txt}}
\hypersetup{pdfcontactaddress={ETH Zurich, ITP, HIT K,
  Wolfgang-Pauli-Strasse 27}}
\hypersetup{pdfcontactpostcode={8093}}
\hypersetup{pdfcontactcity={Zurich}}
\hypersetup{pdfcontactcountry={Switzerland}}
\hypersetup{pdfcontactemail={nbeisert@itp.phys.ethz.ch}}
\hypersetup{pdfcontacturl={http://people.phys.ethz.ch/\xmptilde nbeisert/}}

\newcommand{\secref}[1]{\hyperref[#1]{section \ref*{#1}}}

\parskip1ex
\parindent0pt
\let\olditemize\itemize
\def\itemize{\olditemize\parskip0pt}

\begin{document}

\title{The \textsf{childdoc} Package}
\hypersetup{pdftitle={The childdoc Package}}
\author{Niklas Beisert\\[2ex]
  Institut f\"ur Theoretische Physik\\
  Eidgen\"ossische Technische Hochschule Z\"urich\\
  Wolfgang-Pauli-Strasse 27, 8093 Z\"urich, Switzerland\\[1ex]
  \href{mailto:nbeisert@itp.phys.ethz.ch}
  {\texttt{nbeisert@itp.phys.ethz.ch}}}
\hypersetup{pdfauthor={Niklas Beisert}}
\hypersetup{pdfsubject={Manual for the LaTeX2e Package childdoc}}
\date{30 December 2018, \textsf{v2.0}}
\maketitle

\begin{abstract}\noindent
\textsf{childdoc} is a \LaTeXe{} package
that enables the direct compilation
of document sections included by |\include|
to individual files.
\end{abstract}

\begingroup
\parskip0ex
\tableofcontents
\endgroup

%%%%%%%%%%%%%%%%%%%%%%%%%%%%%%%%%%%%%%%%%%%%%%%%%%%%%%%%%%%%%%%%%%%%%%%%%%%%%%%%
%%%%%%%%%%%%%%%%%%%%%%%%%%%%%%%%%%%%%%%%%%%%%%%%%%%%%%%%%%%%%%%%%%%%%%%%%%%%%%%%
\section{Introduction}

\LaTeX{} provides a mechanism to structure a large document (such as a book)
into a main file and several child files (containing the chapters)
using the |\include| command.
This mechanism is beneficial for documents
which span hundreds of pages in order to
make the source file(s) more manageable.
Moreover, compilation can be restricted to
selected child files by means of the |\includeonly| command.
The latter feature can be used to reduce the compilation time while editing
(this was significantly more useful in the earlier days of \LaTeX{})
or to generate a smaller document which is easier to navigate.
Another application of |\includeonly| is to generate
documents consisting of selected parts of the complete document.

However, there are a few drawbacks of the plain |\include| mechanism:
\begin{itemize}
\item
The child files cannot be compiled on their own,
they can only be compiled via the main file.
A naive editing environment
(such as a text editor with an option
to have the current file processed by \LaTeX)
may require one to switch to the main file before compiling;
attempting to compile the child file produces errors.
\item
The main file must be modified (each time)
to adjust the |\includeonly| command
to the present needs. This easily leaves the main file in a messy state.
\item
The generated document will always carry the filename
of the main document. This is inconvenient if
several child files are to be compiled and
to be kept for distribution.
\end{itemize}

The present package provides a simple interface
to make child files individually compilable by \LaTeX{}.
Compiling a child file then has the same effect as compiling
the main file with an |\includeonly| command
to select the appropriate child.
Moreover the generated document will carry the name of the child
rather than the main file.
This resolves all three above issues.

This feature is meant to make the editing of books,
thesis documents and lecture notes somewhat more convenient.
However, the package can also be used efficiently for
composing a series of documents (such as exercise sheets)
which are typically distributed individually.
It then assists the author in generating the individual documents
(potentially in different versions)
as well as a document containing the collected series.
Another application is in developing style files
or other kinds of included material
where compilation of the style file could redirect
to a sample or test file.

%%%%%%%%%%%%%%%%%%%%%%%%%%%%%%%%%%%%%%%%%%%%%%%%%%%%%%%%%%%%%%%%%%%%%%%%%%%%%%%%
%%%%%%%%%%%%%%%%%%%%%%%%%%%%%%%%%%%%%%%%%%%%%%%%%%%%%%%%%%%%%%%%%%%%%%%%%%%%%%%%
\section{Usage}

First of all, the package \textsf{childdoc} is \emph{not} a standard
\LaTeXe{} |.sty| style file! Therefore it needs to be invoked in
a non-standard way.

%%%%%%%%%%%%%%%%%%%%%%%%%%%%%%%%%%%%%%%%%%%%%%%%%%%%%%%%%%%%%%%%%%%%%%%%%%%%%%%%
\subsection{Included Files}
\label{sec:include}

%%%%%%%%%%%%%%%%%%%%%%%%%%%%%%%%%%%%%%%%
\DescribeMacro{\childdocmain}
To use the package, add the commands
\begin{center}
\begin{tabular}{l}
|\input{childdoc.def}|\\
|\childdocmain{}|\\
\end{tabular}
\end{center}
at the very top of the main \LaTeX{} file,
in particular \emph{before} the |\documentclass| statement!
The argument of |\childdocmain| should be left empty
(but it must be present).

%%%%%%%%%%%%%%%%%%%%%%%%%%%%%%%%%%%%%%%%
\DescribeMacro{\childdocof}
Furthermore, add the commands
\begin{center}
\begin{tabular}{l}
|\input{childdoc.def}|\\
|\childdocof{|\textit{main}|}|\\
\end{tabular}
\end{center}
at the top of every child file \textit{child}
which is included by |\include{|\textit{child}|}|
from within the main file
(or at least for those files to be compiled individually).
The argument \textit{main} must be the filename of the main file.

There are a couple of
considerations in setting up the main and child documents:

%%%%%%%%%%%%%%%%%%%%%%%%%%%%%%%%%%%%%%%%
\paragraph{Restrictions.}

Please note the following restrictions:
\begin{itemize}
\item
|\childdocmain| must be called with one argument \textit{main}
to ensure compatibility with earlier version of the package.
It must either be empty (|\childdocmain{}|)
or precisely match the filename of the main file in which it is specified.
See \secref{sec:detection} for further information.
\item
The filename \textit{main} must be specified without the |.tex| extension.
\item
The filename \textit{main} is case sensitive
(even in case-insensitive file systems)
due to internal string comparison.
\item
The argument \textit{main} should be fully expanded, it cannot be a macro.
\item
Subdirectories and special characters should be avoided in filenames.
\item
The command |\childdocmain{|\textit{main}|}| must be followed by a whitespace.
It should not be followed immediately by another command
or by a comment mark `|%|'.
This is because the \TeX{} parser reads the token immediately following
the argument of |\childdocmain| and puts it
at the beginning of every child section;
however, a white\-space is ignored.
\end{itemize}

%%%%%%%%%%%%%%%%%%%%%%%%%%%%%%%%%%%%%%%%
\paragraph{Content of Main File.}

It is advisable to place all content in the child files included by |\include|.
Any output contained in the main file will appear in all child documents
unless suppressed manually;
it cannot be suppressed automatically by the |\includeonly| directive
and thus should normally be avoided.
A method to include some content in the main file
by means of conditional processing is described in \secref{sec:conditional}.

%%%%%%%%%%%%%%%%%%%%%%%%%%%%%%%%%%%%%%%%
\paragraph{Page Numbering.}

When only a part of the document is compiled,
the appropriate numbering of pages
(as well as other status parameters)
is determined from the |.aux| files.
The latter contain information from previous passes.
However this information needs to propagate through
all intermediate child documents.
Therefore the page numbering in child documents may well
be inconsistent until the complete document is compiled at least once.

A useful (if unconventional) way to always ensure a consistent
page numbering is to restart the numbering in each child document
and denote the pages by `\textit{child}|.|\textit{page}'
where \textit{child} represents the chapter/section number of the child file.
This can be achieved by the command
|\numberwithin{page}{|\textit{child}|}|
of the \textsf{amsmath} package
where \textit{child} can be |chapter| or |section|
depending on the chosen structuring.
Alternatively, one can modify the macro |\thepage| appropriately
and reset the counter |page| at the start of each child file.

%%%%%%%%%%%%%%%%%%%%%%%%%%%%%%%%%%%%%%%%%%%%%%%%%%%%%%%%%%%%%%%%%%%%%%%%%%%%%%%%
\subsection{Conditional Processing}
\label{sec:conditional}

The package provides a mechanism to compile different versions
of a document. To customise the versions further some conditional processing
can come in handy to distinguish which version is being compiled.
The package provides two macros to describe the compilation context:

%%%%%%%%%%%%%%%%%%%%%%%%%%%%%%%%%%%%%%%%
\DescribeMacro{\ifchilddoc}
The conditional |\ifchilddoc| distinguishes between the compilation of
child documents and the main document:
%
\begin{center}
|\ifchilddoc |\textit{child-code}| |[|\||else |\textit{main-code}]| \||fi|
\end{center}

%%%%%%%%%%%%%%%%%%%%%%%%%%%%%%%%%%%%%%%%
\DescribeMacro{\childdocname}
\DescribeMacro{\childdocjob}
The macro |\childdocname| contains the filename (without extension)
of the main or child file being processed.
Note that |\childdocjob| will always contain the name of the main file.

%%%%%%%%%%%%%%%%%%%%%%%%%%%%%%%%%%%%%%%%
\paragraph{Title Page.}

Conditional processing can be used to include a title or banner page
in the main document when proper precautions are taken.
Importantly, the code in the main file should ensure that the page counter
(as well as other status parameters which are stored in the |.aux| files)
takes the same value after the conditional processing.
Otherwise the page numbers may take divergent values
depending on which part is compiled.

For example, a title page could be declared by:
%
\begin{center}
\begin{tabular}{l}
|\ifchilddoc\||else|\\
|\addtocounter{page}{-1}|\\
\textit{code for title page}\\
|\newpage|\\
|\||fi|
\end{tabular}
\end{center}
%
A banner page for the child documents can be generated by:
%
\begin{center}
\begin{tabular}{l}
|\ifchilddoc|\\
|\addtocounter{page}{-1}|\\
\textit{code for banner page}\\
|\newpage|\\
|\||fi|
\end{tabular}
\end{center}
%
Here one could write a message such as:
\begin{center}
|This is the part \childdocname{} of \childdocjob{}.|
\end{center}

%%%%%%%%%%%%%%%%%%%%%%%%%%%%%%%%%%%%%%%%%%%%%%%%%%%%%%%%%%%%%%%%%%%%%%%%%%%%%%%%
\subsection{Flags}
\label{sec:flags}

The package makes it easy to generate different versions
of the main or child documents.
To this end compilation flags can be defined
and assigned different default values.
They will be particularly useful in conjunction
with the forwarding mechanism described in \secref{sec:forward}.

For example, it may be useful to have a flag |\version|
which can be set to |draft| or |final|.
The document source will contain some conditional code
depending on the value of |\version|.
Suppose further, the flag should default to |final| for the main file
and to |draft| for child files
which is a natural assignment for editing the document.
This is achieved by placing the following code
in the preamble of the main document
(below the |\childdocmain| directive):
%
\begin{center}
\begin{tabular}{l}
|\ifchilddoc|\\
|\providecommand{\version}{draft}|\\
|\||else|\\
|\providecommand{\version}{final}|\\
|\||fi|
\end{tabular}
\end{center}
%
The definition by |\providecommand| makes sure
that previous definitions are not overwritten.
Further statements |\providecommand{\version}{...}|
can thus be added before the above code to override it.

For the main file, one might add a line
(between |\childdocmain| and the above block)
%
\begin{center}
|%\ifchilddoc\||else\providecommand{\version}{draft}\||fi|
\end{center}
%
which can be uncommented to produce a draft version.
Likewise one can add a line to the very top of a child file
(above the |\childdocof{|\textit{main}|}| directive)
%
\begin{center}
|%\providecommand{\version}{final}|
\end{center}
%
which can be uncommented to produce the final version of this child document.

%%%%%%%%%%%%%%%%%%%%%%%%%%%%%%%%%%%%%%%%%%%%%%%%%%%%%%%%%%%%%%%%%%%%%%%%%%%%%%%%
\subsection{Forwarding}
\label{sec:forward}

Different versions of the main or child documents
using compilation flags as described in \secref{sec:flags}
can be (permanently) stored in different files
for convenient compilation, viewing and distribution.
To this end, the package defines a command
to pass on compilation to a different file:

%%%%%%%%%%%%%%%%%%%%%%%%%%%%%%%%%%%%%%%%
\DescribeMacro{\childdocforward}
The command |\childdocforward| redirects processing to
another source file:
%
\begin{center}
\begin{tabular}{l}
|\input{childdoc.def}|\\
|\childdocforward[|\textit{main}|]{|\textit{dest}|}|\\
\end{tabular}
\end{center}
%
The argument \textit{dest} is the destination file
(without extension).
It should be the main file or one of the child files.
Note that further \textsf{childdoc} directives
such as |\childdocof| and |\childdocforward|
in the indicated file will be processed in this form.
The optional argument \textit{main}
passes on directly to the main file \textit{main}
while pretending to compile the child \textit{dest}.
This form behaves as if \textit{dest}
issues |\childdocof{|\textit{main}|}| right away,
and no further \textsf{childdoc} directives will be processed.

%%%%%%%%%%%%%%%%%%%%%%%%%%%%%%%%%%%%%%%%
\DescribeMacro{\...prefix}
In the alternative form |\childdocforwardprefix|,
%
\begin{center}
\begin{tabular}{l}
|\input{childdoc.def}|\\
|\childdocforwardprefix[|\textit{main}|]{|\textit{prefix}|}{|\textit{dest}|}|
\end{tabular}
\end{center}
%
the destination file is determined by a pattern
depending on the current file:
To make this work, the current file must be called
`{\textit{prefix}\hspace{0.2em}\textit{suffix}}'
with \textit{prefix} matching precisely the argument.
Processing is then passed on to the file
`{\textit{dest}\hspace{0.2em}\textit{suffix}}'.
Surely, the same effect is achieved by
directly specifying the
argument `{\textit{dest}\hspace{0.2em}\textit{suffix}}'
in the first form.
However, that requires to set up a different file
for each child. With the alternative form of the command
all these files can have exactly the same content
which simplifies setting them up and maintaining them.

For example, the following file |draft.tex|
with a compilation flag |\version| as described in \secref{sec:flags}
compiles the main document as a draft:
%
\begin{center}
\begin{tabular}{l}
|\def\version{draft}|\\
|\input{childdoc.def}|\\
|\childdocforward{|\textit{main}|}|
\end{tabular}
\end{center}
%
Likewise, the following files |final|\textit{nn}|.tex|
compile the final version of the child document
|child|\textit{nn}|.tex|:
%
\begin{center}
\begin{tabular}{l}
|\def\version{final}|\\
|\input{childdoc.def}|\\
|\childdocforwardprefix{final}{child}|
\end{tabular}
\end{center}
%

Note that when several versions of a main file and/or of each child file
are to be generated, it may be convenient to set up a |Makefile| or
shell script to automatise the process.

%%%%%%%%%%%%%%%%%%%%%%%%%%%%%%%%%%%%%%%%%%%%%%%%%%%%%%%%%%%%%%%%%%%%%%%%%%%%%%%%
\subsection{Command Line Processing}
\label{sec:commandline}

The effect of redirection files can also be achieved by invoking
the \LaTeX{} compiler with a more elaborate command line.
Most conveniently this should be done as part
of a shell script or a |Makefile|.

When using \textsf{childdoc} in the main file, the following
command lines effectively perform a redirection
(note that depending on the shell being used,
backslashes may have to be doubled: `|\|' $\to$ `|\\|'):
%
\begin{center}
|... -jobname "|\textit{target}|" |\\|"|[\textit{flags}]%
|\input{childdoc.def}\childdocforward[|\textit{main}|]{|\textit{dest}|}"|
\end{center}
%
Here \textit{target} is the name of the output file,
\textit{main} is the name of the main file
and \textit{dest} is the name of the main or child file to be processed
(all filenames without extensions).
The optional argument \textit{main} can be omitted
if \textit{main} matches \textit{dest}.
Optionally, compilation \textit{flags} can be defined via |\def| commands.
This command line makes the \TeX{} engine believe
it is compiling the file \textit{target}
whose content is specified as the latter parameter.
The provided code then forwards the processing to
\textit{main} or \textit{dest} as described in \secref{sec:forward}.

%%%%%%%%%%%%%%%%%%%%%%%%%%%%%%%%%%%%%%%%%%%%%%%%%%%%%%%%%%%%%%%%%%%%%%%%%%%%%%%%
\subsection{Include by Input}
\label{sec:input}

Including child documents by |\include| has some restrictions by design.
Most notably, the content of a child document always occupies
its own set of pages; pages cannot be shared between child documents.
Usually, this behaviour makes perfect sense
because each child document contain an essential part of the document.
However, in some situations it may be desirable to compose
a document from a collection of parts
without having mandatory page breaks between then.
For this case, the package
provides a mechanism to include parts
by |\input| which can also be processed individually.
However, by construction this mechanism
requires manual handling of the content to be output.

%%%%%%%%%%%%%%%%%%%%%%%%%%%%%%%%%%%%%%%%
\DescribeMacro{\ifchilddocmanual}
The main file should be prepared as usual, see \secref{sec:include}.
However, the document body must make a distinction
between processing of an individual part and of the main document, e.g.:
%
\begin{center}
\begin{tabular}{l}
|\ifchilddocmanual|\\
|\input{\childdocname}|\\
|\||else|\\
\textit{document body with }|\input{|\textit{part}|}|\\
|\||fi|
\end{tabular}
\end{center}
%
The conditional |\ifchilddocmanual| is true whenever
a part to be included by |\input| is being compiled,
and the name of the part is stored in |\childdocname|.

%%%%%%%%%%%%%%%%%%%%%%%%%%%%%%%%%%%%%%%%
\DescribeMacro{\childdocby}
Each part to be included by |\input| should start with:
%
\begin{center}
\begin{tabular}{l}
|\input{childdoc.def}|\\
|\childdocby{|\textit{main}|}|\\
\end{tabular}
\end{center}
%
The directive |\childdocby| is similar to |\childdocof|
described in \secref{sec:include},
but the subsequent selection of content must be done manually.
To that end, both |\ifchilddoc| and |\ifchilddocmanual|
will be true upon processing of a part,
and the name of the part is stored in |\childdocname|.
Note that |\jobname| will be set to the filename of the current part
so that each part receives an individual |.aux| file
that does not interfere with the |.aux| file(s) of the main document.
This behaviour can be altered by the alternative form
|\childdocby[*]{|\textit{main}|}| (with a non-empty optional argument)
which uses the |.aux| file of the main document
by setting |\jobname| to \textit{main}.

%%%%%%%%%%%%%%%%%%%%%%%%%%%%%%%%%%%%%%%%%%%%%%%%%%%%%%%%%%%%%%%%%%%%%%%%%%%%%%%%
\subsection{Driver Development}
\label{sec:driver}

The \textsf{childdoc} mechanism can also be use for the development
of definition files such as \LaTeX{} styles or classes.
This case differs from the above setup with multiple parts
included by |\include| in that no |\includeonly| should be invoked.
This can be achieved by starting the include file
(before |\ProvidesPackage|) with:
%
\begin{center}
\begin{tabular}{l}
|\input{childdoc.def}|\\
|\childdocforward{|\textit{main}|}|\\
\end{tabular}
\end{center}
%
or alternatively with:
%
\begin{center}
\begin{tabular}{l}
|\input{childdoc.def}|\\
|\childdocby{|\textit{main}|}|\\
\end{tabular}
\end{center}
%
Both forms have slightly different effects as described above.
The main file is prepared as usual, see \secref{sec:include}.

%%%%%%%%%%%%%%%%%%%%%%%%%%%%%%%%%%%%%%%%%%%%%%%%%%%%%%%%%%%%%%%%%%%%%%%%%%%%%%%%
\subsection{Legacy Detection}
\label{sec:detection}

The directive |\childdocmain| in the main file can detect
whether the complete document or merely a child is to be compiled
even without using the directive |\childdocof|.
This method is deprecated because it is less robust
and there is no compelling reason to use it;
it is merely provided for backward compatibility
and it may be removed in future versions.

If the detection mechanism is to be used,
it is mandatory to correctly specify
the filename of the main file as the argument of |\childdocmain|:
%
\begin{center}
\begin{tabular}{l}
|\input{childdoc.def}|\\
|\childdocmain{|\textit{main}|}|\\
\end{tabular}
\end{center}
%
If |\jobname| does not match the argument \textit{main} of |\childdocmain|,
it is assumed that |\jobname| points to the child file to be compiled.
When using |\childdocmain| with the main file specified as argument,
it suffices to start a child file
with just |\input{|\textit{main}|}|
without loading of the package and using |\childdocof|.
If instead all processing is done
with the appropriate \textsf{childdoc} directives,
the argument of \textit{main} of |\childdocmain| can be empty.

An alternative version of the command line processing described
in \secref{sec:commandline} using the detection mechanism reads:
%
\begin{center}
|... -jobname "|\textit{target}|" "|[\textit{flags}]%
[|\def\jobname{|\textit{dest}|}|]|\input{|\textit{main}|}"|
\end{center}

%%%%%%%%%%%%%%%%%%%%%%%%%%%%%%%%%%%%%%%%%%%%%%%%%%%%%%%%%%%%%%%%%%%%%%%%%%%%%%%%
\subsection{Manual Code}
\label{sec:manual}

In case one cannot be certain whether the definitions file |childdoc.def|
is installed on the target \TeX{} distribution
and one prefers not to ship it,
it is conceivable to paste a few relevant commands into the sources.

To that end, drop all statements |\input{childdoc.def}|
and perform the replacements as outlined below.
Instead of |\childdocmain{|\textit{main}|}| add the following code
to the top of the main file:
%
\begin{center}
\begin{tabular}{l}
|\||ifdefined\childdocname\endinput\||fi\newif\ifchilddoc|\\
|\edef\childdocname{\scantokens\expandafter{\jobname\noexpand}}|\\
|\def\childdocmain{|\textit{main}|}\||ifx\childdocmain\childdocname\||else|\\
|\childdoctrue\includeonly{\childdocname}\let\jobname\childdocmain\||fi|\\
\end{tabular}
\end{center}
%
Instead of |\childdocof{|\textit{main}|}| just include the main file
at the top of each child file:
%
\begin{center}
|\input{|\textit{main}|}|
\end{center}
%
A simple redirection |\childdocforward{|\textit{dest}|}| is achieved by:
%
\begin{center}
|\def\jobname{|\textit{dest}|}\input{\jobname}|
\end{center}
%
The redirection with prefix
|\childdocforwardprefix[|\textit{prefix}|]{|\textit{dest}|}|
is accomplished by:
%
\begin{center}
\begin{tabular}{l}
|{\edef\jobname{\scantokens\expandafter{\jobname\noexpand}}|\\
|\def\redirectjob |\textit{prefix}|#1~~~{\gdef\jobname{|\textit{dest}|#1}}|\\
|\expandafter\redirectjob\jobname~~~}\input{\jobname}|
\end{tabular}
\end{center}

In an alternative approach,
child documents can be compiled by a specific command line
without additional code or specific definitions:
%
\begin{center}
|... -jobname "|\textit{target}|" "|[\textit{flags}]%
|\includeonly{|\textit{dest}|}\input{|\textit{main}|}"|
\end{center}
%

%%%%%%%%%%%%%%%%%%%%%%%%%%%%%%%%%%%%%%%%%%%%%%%%%%%%%%%%%%%%%%%%%%%%%%%%%%%%%%%%
%%%%%%%%%%%%%%%%%%%%%%%%%%%%%%%%%%%%%%%%%%%%%%%%%%%%%%%%%%%%%%%%%%%%%%%%%%%%%%%%
\section{Information}

%%%%%%%%%%%%%%%%%%%%%%%%%%%%%%%%%%%%%%%%%%%%%%%%%%%%%%%%%%%%%%%%%%%%%%%%%%%%%%%%
\subsection{Copyright}

Copyright \copyright{} 2017--2018 Niklas Beisert

This work may be distributed and/or modified under the
conditions of the \LaTeX{} Project Public License, either version 1.3
of this license or (at your option) any later version.
The latest version of this license is in
  \url{http://www.latex-project.org/lppl.txt}
and version 1.3 or later is part of all distributions of \LaTeX{}
version 2005/12/01 or later.

This work has the LPPL maintenance status `maintained'.

The Current Maintainer of this work is Niklas Beisert.

This work consists of the files |README.txt|, |childdoc.ins| and |childdoc.dtx|
as well as the derived files |childdoc.def|, |cdocsamp.tex|
with |cdocsch1.tex|, |cdocsch2.tex|, |cdocspt3.tex|, |cdocspt4.tex|,
|cdocsdrf.tex|, |cdocsfn1.tex|, |cdocsfn2.tex|
as well as |childdoc.pdf|.

%%%%%%%%%%%%%%%%%%%%%%%%%%%%%%%%%%%%%%%%%%%%%%%%%%%%%%%%%%%%%%%%%%%%%%%%%%%%%%%%
\subsection{Files and Installation}

The package consists of the files:
%
\begin{center}
\begin{tabular}{ll}
    |README.txt|   & readme file \\
    |childdoc.ins| & installation file \\
    |childdoc.dtx| & source file \\
    |childdoc.def| & definition file \\
    |cdocsamp.tex| & sample main file \\
    |cdocsch1.tex| & sample include file \\
    |cdocsch2.tex| & sample include file \\
    |cdocspt3.tex| & sample part file \\
    |cdocspt4.tex| & sample part file \\
    |cdocsdrf.tex| & sample redirection file \\
    |cdocsfn1.tex| & sample redirection file \\
    |cdocsfn2.tex| & sample redirection file \\
    |childdoc.pdf| & manual
\end{tabular}
\end{center}
%
The distribution consists of the files
|README.txt|, |childdoc.ins| and |childdoc.dtx|.
%
\begin{itemize}
\item
Run (pdf)\LaTeX{} on |childdoc.dtx|
to compile the manual |childdoc.pdf| (this file).
\item
Run \LaTeX{} on |childdoc.ins| to create the definitions file |childdoc.def|
and the sample |cdocsamp.tex| with include files
|cdocsch1.tex|, |cdocsch2.tex|, |cdocspt3.tex|, |cdocspt4.tex|,
|cdocsdrf.tex|, |cdocsfn1.tex|, |cdocsfn2.tex|.
Then copy the file |childdoc.def| to an appropriate directory of your \LaTeX{}
distribution, e.g.\ \textit{texmf-root}|/tex/latex/childdoc|.
\end{itemize}

%%%%%%%%%%%%%%%%%%%%%%%%%%%%%%%%%%%%%%%%%%%%%%%%%%%%%%%%%%%%%%%%%%%%%%%%%%%%%%%%
\subsection{Related CTAN Packages}

There are several other packages which offer a similar functionality:
%
\begin{itemize}
\item
The packages
\href{http://ctan.org/pkg/docmute}{\textsf{docmute}},
\href{http://ctan.org/pkg/includex}{\textsf{includex}} and
\href{http://ctan.org/pkg/standalone}{\textsf{standalone}}
provide commands to include only the document body of
a child file thus allowing both files to be compiled individually.
\item
The packages \href{http://ctan.org/pkg/subdocs}{\textsf{subdocs}}
and \href{http://ctan.org/pkg/subfiles}{\textsf{subfiles}}
provide structures in which the main and child documents can be
encapsulated and allowing them to be compiled individually.
The inclusion mechanism is different from the conventional |\include|.
\item
The package \href{http://ctan.org/pkg/combine}{\textsf{combine}}
is an elaborate solution to combine several documents into one.
\end{itemize}
%
See also the CTAN topic \href{http://ctan.org/topic/subdocs}{\textsf{subdocs}}
for further related packages.
The present package differs from the above solutions in that
a document structure constructed with the conventional |\include| mechanism
just needs two extra commands at the top of every file
such that all constituent files can be compiled individually.

%%%%%%%%%%%%%%%%%%%%%%%%%%%%%%%%%%%%%%%%%%%%%%%%%%%%%%%%%%%%%%%%%%%%%%%%%%%%%%%%
%\subsection{Feature Suggestions}
%
%The following is a list of features which may be useful for future
%versions of this package:
%%
%\begin{itemize}
%\item
%\ldots
%\end{itemize}

%%%%%%%%%%%%%%%%%%%%%%%%%%%%%%%%%%%%%%%%%%%%%%%%%%%%%%%%%%%%%%%%%%%%%%%%%%%%%%%%
\subsection{Revision History}

%%%%%%%%%%%%%%%%%%%%%%%%%%%%%%%%%%%%%%%%
\paragraph{v2.0:} 2018/12/30

\begin{itemize}
\item
immediate forward processing
\item
added |\childdocby| mechanism
\item
manual restructured
\end{itemize}

%%%%%%%%%%%%%%%%%%%%%%%%%%%%%%%%%%%%%%%%
\paragraph{v1.6:} 2018/01/17

\begin{itemize}
\item
application for development of include files
\item
corrections to manual
\end{itemize}

%%%%%%%%%%%%%%%%%%%%%%%%%%%%%%%%%%%%%%%%
\paragraph{v1.5:} 2017/05/21

\begin{itemize}
\item
more complete structuring introduced
\item
|\childdocof| introduced
\item
|\childdoc| renamed to |\childdocmain|
\item
|\childredirect| renamed to |\childdocforward| and |\childdocforwardprefix|
and functionality expanded
\end{itemize}

%%%%%%%%%%%%%%%%%%%%%%%%%%%%%%%%%%%%%%%%
\paragraph{v1.0:} 2017/04/27

\begin{itemize}
\item
manual and install package
\item
first version published on CTAN
\end{itemize}

%%%%%%%%%%%%%%%%%%%%%%%%%%%%%%%%%%%%%%%%
\paragraph{v0.6:} 2017/04/26

\begin{itemize}
\item
redirection mechanism added
\end{itemize}

%%%%%%%%%%%%%%%%%%%%%%%%%%%%%%%%%%%%%%%%
\paragraph{v0.5:} 2017/04/26

\begin{itemize}
\item
functionality in definition file
\end{itemize}


%%%%%%%%%%%%%%%%%%%%%%%%%%%%%%%%%%%%%%%%%%%%%%%%%%%%%%%%%%%%%%%%%%%%%%%%%%%%%%%%
%%%%%%%%%%%%%%%%%%%%%%%%%%%%%%%%%%%%%%%%%%%%%%%%%%%%%%%%%%%%%%%%%%%%%%%%%%%%%%%%
%%%%%%%%%%%%%%%%%%%%%%%%%%%%%%%%%%%%%%%%%%%%%%%%%%%%%%%%%%%%%%%%%%%%%%%%%%%%%%%%
\appendix

\settowidth\MacroIndent{\rmfamily\scriptsize 000\ }

 \DocInput{childdoc.dtx}

\end{document}
%</driver>
% \fi
%
% %%%%%%%%%%%%%%%%%%%%%%%%%%%%%%%%%%%%%%%%%%%%%%%%%%%%%%%%%%%%%%%%%%%%%%%%%%%%%%
% %%%%%%%%%%%%%%%%%%%%%%%%%%%%%%%%%%%%%%%%%%%%%%%%%%%%%%%%%%%%%%%%%%%%%%%%%%%%%%
% \section{Sample}
%\iffalse
%<*samplemain>
%\fi
%
% The following presents a sample document
% with two chapters, two parts, a title page,
% a compile flag as well as three forwarding files to set the flag.
% It consists of eight |.tex| files:
% \begin{center}
% \begin{tabular}{ll}
% |cdocsamp.tex|&main file\\
% |cdocsch1.tex|&include file for chapter 1\\
% |cdocsch2.tex|&include file for chapter 2\\
% |cdocspt3.tex|&include file for part 3\\
% |cdocspt4.tex|&include file for part 4\\
% |cdocsdrf.tex|&forwarding file for main file in draft mode\\
% |cdocsfi1.tex|&forwarding file for final version of chapter 1\\
% |cdocsfi2.tex|&forwarding file for final version of chapter 2\\
% \end{tabular}
% \end{center}
% Each of the eight files can be compiled directly by the \LaTeX{} compiler.
%
% %%%%%%%%%%%%%%%%%%%%%%%%%%%%%%%%%%%%%%
% \paragraph{Main File.}
%
% The main file is called |cdocsamp.tex|.
%
% Load the \textsf{childdoc} definitions and
% declare the filename for the main document:
%    \begin{macrocode}
\input{childdoc.def}
\childdocmain{}
%    \end{macrocode}

% Optional override for |\version| flag:
%    \begin{macrocode}
%%\ifchilddoc\else\providecommand{\version}{draft}\fi
%    \end{macrocode}

% Define the default values for the |\version| flag
% (|final| for the main file and |draft| for childs):
%    \begin{macrocode}
\ifchilddoc
\providecommand{\version}{draft}
\else
\providecommand{\version}{final}
\fi
%    \end{macrocode}

% Load the standard document class:
%    \begin{macrocode}
\documentclass[12pt]{article}
%    \end{macrocode}

% Start the document body:
%    \begin{macrocode}
\begin{document}
%    \end{macrocode}

% Declare a title page.
% Print title, part of document being processed and version flag:
%    \begin{macrocode}
\addtocounter{page}{-1}
\begin{center}
{\LARGE\bfseries{}childdoc example\par}
\vspace{1cm}
\ifchilddoc
\ifchilddocmanual part\else chapter\fi:
`\childdocname' of `\childdocjob'\par
\else
main document: `\childdocjob'\par
\fi
version: \version\par
\end{center}
\newpage
%    \end{macrocode}

% Manually include selected file,
% otherwise process as usual:
%    \begin{macrocode}
\ifchilddocmanual
\section*{part `\childdocname'}
\input{\childdocname}
\else
%    \end{macrocode}

% Include the two chapters:
%    \begin{macrocode}
\include{cdocsch1}
\include{cdocsch2}
%    \end{macrocode}

% Include the two parts unless only chapters should be displayed:
%    \begin{macrocode}
\ifchilddoc\else
\section{part three}
\input{cdocspt3}
\section{part four}
\input{cdocspt4}
\fi
%    \end{macrocode}

% Process as usual until here:
%    \begin{macrocode}
\fi
%    \end{macrocode}

% End of document body:
%    \begin{macrocode}
\end{document}
%    \end{macrocode}
%\iffalse
%</samplemain>
%\fi
%
% %%%%%%%%%%%%%%%%%%%%%%%%%%%%%%%%%%%%%%
% \paragraph{Chapter Include Files.}
%
% The include files are called |cdocsch1.tex| and |cdocsch2.tex|.
%
%\iffalse
%<*samplechap1|samplechap2>
%\fi

% Optional override for |\version| flag:
%    \begin{macrocode}
%%\providecommand{\version}{final}
%    \end{macrocode}

% Include the main document:
%    \begin{macrocode}
\input{childdoc.def}
\childdocof{cdocsamp}
%    \end{macrocode}

%\iffalse
%</samplechap1|samplechap2>
%\fi
%
%\iffalse
%<*samplechap1>
%\fi
% Some text for chapter 1:
%    \begin{macrocode}
\section{one}
some text in chapter one
%    \end{macrocode}

%\iffalse
%</samplechap1>
%\fi
% Some text for chapter 2:
%\iffalse
%<*samplechap2>
%\fi
%    \begin{macrocode}
\section{two}
more text in chapter two
%    \end{macrocode}

%\iffalse
%</samplechap2>
%\fi
%
% %%%%%%%%%%%%%%%%%%%%%%%%%%%%%%%%%%%%%%
% \paragraph{Part Include Files.}
%
% The include files are called |cdocspt3.tex| and |cdocspt4.tex|.
%
%\iffalse
%<*samplepart3|samplepart4>
%\fi

% Optional override for |\version| flag:
%    \begin{macrocode}
%%\providecommand{\version}{final}
%    \end{macrocode}

% Include the main document:
%    \begin{macrocode}
\input{childdoc.def}
\childdocby{cdocsamp}
%    \end{macrocode}

%\iffalse
%</samplepart3|samplepart4>
%\fi
%
%\iffalse
%<*samplepart3>
%\fi
% Some text for part 3:
%    \begin{macrocode}
some text in part three
%    \end{macrocode}

%\iffalse
%</samplepart3>
%\fi
% Some text for part 4:
%\iffalse
%<*samplepart4>
%\fi
%    \begin{macrocode}
more text in part four
%    \end{macrocode}

%\iffalse
%</samplepart4>
%\fi
%
% %%%%%%%%%%%%%%%%%%%%%%%%%%%%%%%%%%%%%%
% \paragraph{Forwarding for a Complete Draft.}
%
% The following forwarding file |cdocsdrf.tex|
% compiles the main document in draft mode:
%\iffalse
%<*sampledraft>
%\fi
%    \begin{macrocode}
\def\version{draft}
\input{childdoc.def}
\childdocforward{cdocsamp}
%    \end{macrocode}

%\iffalse
%</sampledraft>
%\fi
%
% %%%%%%%%%%%%%%%%%%%%%%%%%%%%%%%%%%%%%%
% \paragraph{Forwarding for Final Version of the Chapters.}
%
% The following forwarding files |cdocsfn1.tex| and |cdocsfn2.tex|
% (with identical content)
% compile the final versions of the child documents
% |cdocsch1.tex| and |cdocsch2.tex|, respectively:
%\iffalse
%<*samplefinal>
%\fi
%    \begin{macrocode}
\def\version{final}
\input{childdoc.def}
\childdocforwardprefix[cdocsamp]{cdocsfn}{cdocsch}
%    \end{macrocode}

%\iffalse
%</samplefinal>
%\fi
%
% %%%%%%%%%%%%%%%%%%%%%%%%%%%%%%%%%%%%%%
% \paragraph{Command Line Processing.}
%
% The following three command lines generate the output files
% |cdocscld|, |cdocscl1| and |cdocscl2|
% which should be identical to
% |cdocsdrf|, |cdocsch1| and |cdocsfn2|, respectively:
% \begin{center}
% \begin{tabular}{l}
% |latex -jobname cdocscld \|\\
% |  "\def\version{draft}\input{childdoc.def}\childdocforward{cdocsamp}"|\\
% |latex -jobname cdocscl1 \|\\
% |  "\input{childdoc.def}\childdocforward[cdocsamp]{cdocsch1}"|\\
% |latex -jobname cdocscl2 \|\\
% |  "\def\version{final}\input{childdoc.def}\childdocforward{cdocsch2}"|
% \end{tabular}
% \end{center}
% Note that the trailing backslash on each first line
% merely continues the input to the second line
% (for convenient cut ant paste).
% Furthermore, the command |latex| can be replaced by any
% of its alternative versions such as |pdflatex|.
%
% %%%%%%%%%%%%%%%%%%%%%%%%%%%%%%%%%%%%%%%%%%%%%%%%%%%%%%%%%%%%%%%%%%%%%%%%%%%%%%
% %%%%%%%%%%%%%%%%%%%%%%%%%%%%%%%%%%%%%%%%%%%%%%%%%%%%%%%%%%%%%%%%%%%%%%%%%%%%%%
% \section{Implementation}
%\iffalse
%<*package>
%\fi
%
% This section describes the definitions file |childdoc.def|.

% The definitions cannot be loaded using |\usepackage| or |\RequirePackage|
% which has a mechanism to prevent loading a style file more than once.
% When loading the definitions by means of |\input|
% multiple instances have to be prevented manually:
%\iffalse
%This code needs to be before the `\ProvidesFile' directive
%which is defined at the beginning of this file.
%Therefore it is also placed there and commented out here.
%</package>
%<*discard>
%\fi
%    \begin{macrocode}
\ifdefined\childdocmain\endinput\fi
%    \end{macrocode}
%\iffalse
%</discard>
%<*package>
%\fi
%
% \macro{\ifchilddoc}
% \macro{\ifchilddocmanual}
% The conditional |\ifchilddoc| tells whether a
% child (true) or main (false) document is being compiled.
% The conditional |\ifchilddocmanual| tells whether
% the |\includeonly| mechanism is used (false) or
% the selection of child files must be performed manually (true).
% The definitions initialise to false:
%    \begin{macrocode}
\newif\ifchilddoc
\newif\ifchilddocmanual
%    \end{macrocode}

% \macro{\childdocname}
% \macro{\childdocjob}
% The macro |\childdocname| stores the name of the main document
% to be compiled. The macro |\childdocjob| stores the name of
% the document on which the \LaTeX{} compiler was originally invoked.
% The content of |\jobname| cannot be compared
% to filenames specified in the source due to different catcodes.
% The following code rescans |\jobname|, stores the result
% in |\childdocname| and saves a copy in |\childdocjob|:
%    \begin{macrocode}
\edef\childdocname{\scantokens\expandafter{\jobname\noexpand}}
\let\childdocjob\childdocname
%    \end{macrocode}

% \macro{\childdocdisable}
% The macro |\childdocdisable| prevents the main file
% from being processed more than once.
% At this stage, the main document command |\childdocmain|
% is assumed to be called once again where it should do nothing.
% Any subsequent call to it should prevent
% a secondary processing of the main document
% It overwrites the forwarding commands
% |\childdocof| and |\childdocforward|
% with empty macros to prevent further inclusions of the main document:
%    \begin{macrocode}
\newcommand{\childdocdisable}
{
  \renewcommand{\childdocmain}[1]{\renewcommand{\childdocmain}[1]{\endinput}}
  \renewcommand{\childdocof}[1]{}
  \renewcommand{\childdocby}[2][]{}
  \renewcommand{\childdocforward}[2][]{}
  \renewcommand{\childdocdisable}{}
}
%    \end{macrocode}

% \macro{\childdocmain}
% The macro |\childdocmain| is to be called at the top of the main file
% with nothing or the main filename (without extension) as argument.
% First, it breaks loops.
% If the argument is not empty and does not match |\childdocname|
% (which is set by the first inclusion of |childdoc.def|),
% |\ifchilddoc| is set to true, |\includeonly| is applied to the child file
% and |\jobname| is set to the main file
% (for proper handling of |.aux| files):
%    \begin{macrocode}
\newcommand{\childdocmain}[1]
{
  \childdocdisable\childdocmain{}
  \if?#1?\else
    \begingroup
      \def\childdoctmp{#1}
      \ifx\childdoctmp\childdocname
        \def\childdoctmp{}
      \else
        \def\childdoctmp
        {
          \childdoctrue
          \includeonly{\childdocname}
          \def\childdocjob{#1}
          \def\jobname{#1}
        }
      \fi
      \expandafter
    \endgroup
    \childdoctmp
  \fi
}
%    \end{macrocode}

% \macro{\childdocof}
% The command |\childdocof| redirects
% compilation to the main file |#1|.
%    \begin{macrocode}
\newcommand{\childdocof}[1]
{
  \childdocdisable
  \childdoctrue
  \includeonly{\childdocname}
  \def\jobname{#1}
  \def\childdocjob{#1}
  \input{#1}
}
%    \end{macrocode}

% \macro{\childdocby}
% The command |\childdocby| ....
%    \begin{macrocode}
\newcommand{\childdocby}[2][]
{
  \childdocdisable
  \childdoctrue
  \childdocmanualtrue
  \if?#1?\else
    \def\jobname{#2}
  \fi
  \def\childdocjob{#2}
  \input{#2}
  \endinput
}
%    \end{macrocode}

% \macro{\childdocforward}
% The command |\childdocforward| redirects
% compilation to the main file or
% (if the optional argument is given) a child file.
% Parameters are set as if the main file
% or a child file starting with |\childdocof| was compiled.
% Then compilation is handed over to the main file:
%    \begin{macrocode}
\newcommand{\childdocforward}[2][]
{
  \begingroup
    \if?#1?
      \def\childdoctmp
      {
        \def\childdocname{#2}
        \def\childdocjob{#2}
        \def\jobname{#2}
        \input{#2}
        \endinput
      }
    \else
      \def\childdoctmp
      {
        \childdocdisable
        \def\childdocname{#2}
        \childdoctrue
        \includeonly{#2}
        \def\childdocjob{#1}
        \def\jobname{#1}
        \input{#1}
        \endinput
      }
    \fi
    \expandafter
  \endgroup
  \childdoctmp
}
%    \end{macrocode}

% \macro{\childdocforwardprefix}
% The command |\childdocforwardprefix| redirects
% compilation to the main or a child file by means of a pattern.
% The prefix |#1| in the current filename is replaced by |#2|
% and the suffix of the current filename is kept
% (it is assumed that the filename does not contain the substring `|~~~|'
% which is used as a delimiter).
% Compilation is handed over to the new file by |\childdocforward|:
%    \begin{macrocode}
\newcommand{\childdocforwardprefix}[3][]
{
  \begingroup
    \def\childdocextract #2##1~~~{\def\childdoctmp{\childdocforward[#1]{#3##1}}}
    \expandafter\childdocextract\childdocname~~~
    \expandafter
  \endgroup
  \childdoctmp
}
%    \end{macrocode}

% \macro{\childdoc}
% The deprecated macro |\childdoc| is a legacy version of |\childdocmain|:
%    \begin{macrocode}
\newcommand{\childdoc}{\childdocmain}
%    \end{macrocode}

% \macro{\childdocredirect}
% The deprecated macro |\childdocredirect| is a legacy version
% of |\childdocforward| and |\childdocforwardprefix|:
%    \begin{macrocode}
\newcommand{\childdocredirect}[2][]
{
  \begingroup
    \if?#1?
      \def\childdoctmp{\childdocforward{#2}}
    \else
      \def\childdoctmp{\childdocforwardprefix{#1}{#2}}
    \fi
    \expandafter
  \endgroup
  \childdoctmp
}
%    \end{macrocode}

%\iffalse
%</package>
%\fi
%
\endinput
|\\
|\childdocmain{}|\\
\end{tabular}
\end{center}
at the very top of the main \LaTeX{} file,
in particular \emph{before} the |\documentclass| statement!
The argument of |\childdocmain| should be left empty
(but it must be present).

%%%%%%%%%%%%%%%%%%%%%%%%%%%%%%%%%%%%%%%%
\DescribeMacro{\childdocof}
Furthermore, add the commands
\begin{center}
\begin{tabular}{l}
|% \iffalse
%
% childdoc.dtx Copyright (C) 2017-2018 Niklas Beisert
%
% This work may be distributed and/or modified under the
% conditions of the LaTeX Project Public License, either version 1.3
% of this license or (at your option) any later version.
% The latest version of this license is in
%   http://www.latex-project.org/lppl.txt
% and version 1.3 or later is part of all distributions of LaTeX
% version 2005/12/01 or later.
%
% This work has the LPPL maintenance status `maintained'.
%
% The Current Maintainer of this work is Niklas Beisert.
%
% This work consists of the files childdoc.dtx and childdoc.ins
% and the derived files childdoc.def and cdocsamp.tex with
% cdocsch1.tex, cdocsch2.tex, cdocsdrf.tex, cdocsfn1.tex, cdocsfn2.tex.
%
%<package>\ifdefined\childdocmain\endinput\fi
%<package>\ProvidesFile{childdoc.def}[2018/12/30 v2.0 child document driver]
%<samplemain>\ProvidesFile{cdocsamp.tex}[2018/12/30 v2.0 sample for childdoc]
%<*driver>
%\ProvidesFile{childdoc.drv}[2018/12/30 v2.0 childdoc reference manual file]
\PassOptionsToClass{10pt,a4paper}{article}
\documentclass{ltxdoc}

\usepackage[margin=35mm]{geometry}
\usepackage{hyperref}
\usepackage{hyperxmp}
\usepackage[usenames]{color}

\hypersetup{colorlinks=true}
\hypersetup{pdfstartview=FitH}
\hypersetup{pdfpagemode=UseNone}
\hypersetup{pdfsource={}}
\hypersetup{pdflang={en-UK}}
\hypersetup{pdfcopyright={Copyright 2017-2018 Niklas Beisert.
  This work may be distributed and/or modified under the
  conditions of the LaTeX Project Public License, either version 1.3
  of this license or (at your option) any later version.}}
\hypersetup{pdflicenseurl={http://www.latex-project.org/lppl.txt}}
\hypersetup{pdfcontactaddress={ETH Zurich, ITP, HIT K,
  Wolfgang-Pauli-Strasse 27}}
\hypersetup{pdfcontactpostcode={8093}}
\hypersetup{pdfcontactcity={Zurich}}
\hypersetup{pdfcontactcountry={Switzerland}}
\hypersetup{pdfcontactemail={nbeisert@itp.phys.ethz.ch}}
\hypersetup{pdfcontacturl={http://people.phys.ethz.ch/\xmptilde nbeisert/}}

\newcommand{\secref}[1]{\hyperref[#1]{section \ref*{#1}}}

\parskip1ex
\parindent0pt
\let\olditemize\itemize
\def\itemize{\olditemize\parskip0pt}

\begin{document}

\title{The \textsf{childdoc} Package}
\hypersetup{pdftitle={The childdoc Package}}
\author{Niklas Beisert\\[2ex]
  Institut f\"ur Theoretische Physik\\
  Eidgen\"ossische Technische Hochschule Z\"urich\\
  Wolfgang-Pauli-Strasse 27, 8093 Z\"urich, Switzerland\\[1ex]
  \href{mailto:nbeisert@itp.phys.ethz.ch}
  {\texttt{nbeisert@itp.phys.ethz.ch}}}
\hypersetup{pdfauthor={Niklas Beisert}}
\hypersetup{pdfsubject={Manual for the LaTeX2e Package childdoc}}
\date{30 December 2018, \textsf{v2.0}}
\maketitle

\begin{abstract}\noindent
\textsf{childdoc} is a \LaTeXe{} package
that enables the direct compilation
of document sections included by |\include|
to individual files.
\end{abstract}

\begingroup
\parskip0ex
\tableofcontents
\endgroup

%%%%%%%%%%%%%%%%%%%%%%%%%%%%%%%%%%%%%%%%%%%%%%%%%%%%%%%%%%%%%%%%%%%%%%%%%%%%%%%%
%%%%%%%%%%%%%%%%%%%%%%%%%%%%%%%%%%%%%%%%%%%%%%%%%%%%%%%%%%%%%%%%%%%%%%%%%%%%%%%%
\section{Introduction}

\LaTeX{} provides a mechanism to structure a large document (such as a book)
into a main file and several child files (containing the chapters)
using the |\include| command.
This mechanism is beneficial for documents
which span hundreds of pages in order to
make the source file(s) more manageable.
Moreover, compilation can be restricted to
selected child files by means of the |\includeonly| command.
The latter feature can be used to reduce the compilation time while editing
(this was significantly more useful in the earlier days of \LaTeX{})
or to generate a smaller document which is easier to navigate.
Another application of |\includeonly| is to generate
documents consisting of selected parts of the complete document.

However, there are a few drawbacks of the plain |\include| mechanism:
\begin{itemize}
\item
The child files cannot be compiled on their own,
they can only be compiled via the main file.
A naive editing environment
(such as a text editor with an option
to have the current file processed by \LaTeX)
may require one to switch to the main file before compiling;
attempting to compile the child file produces errors.
\item
The main file must be modified (each time)
to adjust the |\includeonly| command
to the present needs. This easily leaves the main file in a messy state.
\item
The generated document will always carry the filename
of the main document. This is inconvenient if
several child files are to be compiled and
to be kept for distribution.
\end{itemize}

The present package provides a simple interface
to make child files individually compilable by \LaTeX{}.
Compiling a child file then has the same effect as compiling
the main file with an |\includeonly| command
to select the appropriate child.
Moreover the generated document will carry the name of the child
rather than the main file.
This resolves all three above issues.

This feature is meant to make the editing of books,
thesis documents and lecture notes somewhat more convenient.
However, the package can also be used efficiently for
composing a series of documents (such as exercise sheets)
which are typically distributed individually.
It then assists the author in generating the individual documents
(potentially in different versions)
as well as a document containing the collected series.
Another application is in developing style files
or other kinds of included material
where compilation of the style file could redirect
to a sample or test file.

%%%%%%%%%%%%%%%%%%%%%%%%%%%%%%%%%%%%%%%%%%%%%%%%%%%%%%%%%%%%%%%%%%%%%%%%%%%%%%%%
%%%%%%%%%%%%%%%%%%%%%%%%%%%%%%%%%%%%%%%%%%%%%%%%%%%%%%%%%%%%%%%%%%%%%%%%%%%%%%%%
\section{Usage}

First of all, the package \textsf{childdoc} is \emph{not} a standard
\LaTeXe{} |.sty| style file! Therefore it needs to be invoked in
a non-standard way.

%%%%%%%%%%%%%%%%%%%%%%%%%%%%%%%%%%%%%%%%%%%%%%%%%%%%%%%%%%%%%%%%%%%%%%%%%%%%%%%%
\subsection{Included Files}
\label{sec:include}

%%%%%%%%%%%%%%%%%%%%%%%%%%%%%%%%%%%%%%%%
\DescribeMacro{\childdocmain}
To use the package, add the commands
\begin{center}
\begin{tabular}{l}
|\input{childdoc.def}|\\
|\childdocmain{}|\\
\end{tabular}
\end{center}
at the very top of the main \LaTeX{} file,
in particular \emph{before} the |\documentclass| statement!
The argument of |\childdocmain| should be left empty
(but it must be present).

%%%%%%%%%%%%%%%%%%%%%%%%%%%%%%%%%%%%%%%%
\DescribeMacro{\childdocof}
Furthermore, add the commands
\begin{center}
\begin{tabular}{l}
|\input{childdoc.def}|\\
|\childdocof{|\textit{main}|}|\\
\end{tabular}
\end{center}
at the top of every child file \textit{child}
which is included by |\include{|\textit{child}|}|
from within the main file
(or at least for those files to be compiled individually).
The argument \textit{main} must be the filename of the main file.

There are a couple of
considerations in setting up the main and child documents:

%%%%%%%%%%%%%%%%%%%%%%%%%%%%%%%%%%%%%%%%
\paragraph{Restrictions.}

Please note the following restrictions:
\begin{itemize}
\item
|\childdocmain| must be called with one argument \textit{main}
to ensure compatibility with earlier version of the package.
It must either be empty (|\childdocmain{}|)
or precisely match the filename of the main file in which it is specified.
See \secref{sec:detection} for further information.
\item
The filename \textit{main} must be specified without the |.tex| extension.
\item
The filename \textit{main} is case sensitive
(even in case-insensitive file systems)
due to internal string comparison.
\item
The argument \textit{main} should be fully expanded, it cannot be a macro.
\item
Subdirectories and special characters should be avoided in filenames.
\item
The command |\childdocmain{|\textit{main}|}| must be followed by a whitespace.
It should not be followed immediately by another command
or by a comment mark `|%|'.
This is because the \TeX{} parser reads the token immediately following
the argument of |\childdocmain| and puts it
at the beginning of every child section;
however, a white\-space is ignored.
\end{itemize}

%%%%%%%%%%%%%%%%%%%%%%%%%%%%%%%%%%%%%%%%
\paragraph{Content of Main File.}

It is advisable to place all content in the child files included by |\include|.
Any output contained in the main file will appear in all child documents
unless suppressed manually;
it cannot be suppressed automatically by the |\includeonly| directive
and thus should normally be avoided.
A method to include some content in the main file
by means of conditional processing is described in \secref{sec:conditional}.

%%%%%%%%%%%%%%%%%%%%%%%%%%%%%%%%%%%%%%%%
\paragraph{Page Numbering.}

When only a part of the document is compiled,
the appropriate numbering of pages
(as well as other status parameters)
is determined from the |.aux| files.
The latter contain information from previous passes.
However this information needs to propagate through
all intermediate child documents.
Therefore the page numbering in child documents may well
be inconsistent until the complete document is compiled at least once.

A useful (if unconventional) way to always ensure a consistent
page numbering is to restart the numbering in each child document
and denote the pages by `\textit{child}|.|\textit{page}'
where \textit{child} represents the chapter/section number of the child file.
This can be achieved by the command
|\numberwithin{page}{|\textit{child}|}|
of the \textsf{amsmath} package
where \textit{child} can be |chapter| or |section|
depending on the chosen structuring.
Alternatively, one can modify the macro |\thepage| appropriately
and reset the counter |page| at the start of each child file.

%%%%%%%%%%%%%%%%%%%%%%%%%%%%%%%%%%%%%%%%%%%%%%%%%%%%%%%%%%%%%%%%%%%%%%%%%%%%%%%%
\subsection{Conditional Processing}
\label{sec:conditional}

The package provides a mechanism to compile different versions
of a document. To customise the versions further some conditional processing
can come in handy to distinguish which version is being compiled.
The package provides two macros to describe the compilation context:

%%%%%%%%%%%%%%%%%%%%%%%%%%%%%%%%%%%%%%%%
\DescribeMacro{\ifchilddoc}
The conditional |\ifchilddoc| distinguishes between the compilation of
child documents and the main document:
%
\begin{center}
|\ifchilddoc |\textit{child-code}| |[|\||else |\textit{main-code}]| \||fi|
\end{center}

%%%%%%%%%%%%%%%%%%%%%%%%%%%%%%%%%%%%%%%%
\DescribeMacro{\childdocname}
\DescribeMacro{\childdocjob}
The macro |\childdocname| contains the filename (without extension)
of the main or child file being processed.
Note that |\childdocjob| will always contain the name of the main file.

%%%%%%%%%%%%%%%%%%%%%%%%%%%%%%%%%%%%%%%%
\paragraph{Title Page.}

Conditional processing can be used to include a title or banner page
in the main document when proper precautions are taken.
Importantly, the code in the main file should ensure that the page counter
(as well as other status parameters which are stored in the |.aux| files)
takes the same value after the conditional processing.
Otherwise the page numbers may take divergent values
depending on which part is compiled.

For example, a title page could be declared by:
%
\begin{center}
\begin{tabular}{l}
|\ifchilddoc\||else|\\
|\addtocounter{page}{-1}|\\
\textit{code for title page}\\
|\newpage|\\
|\||fi|
\end{tabular}
\end{center}
%
A banner page for the child documents can be generated by:
%
\begin{center}
\begin{tabular}{l}
|\ifchilddoc|\\
|\addtocounter{page}{-1}|\\
\textit{code for banner page}\\
|\newpage|\\
|\||fi|
\end{tabular}
\end{center}
%
Here one could write a message such as:
\begin{center}
|This is the part \childdocname{} of \childdocjob{}.|
\end{center}

%%%%%%%%%%%%%%%%%%%%%%%%%%%%%%%%%%%%%%%%%%%%%%%%%%%%%%%%%%%%%%%%%%%%%%%%%%%%%%%%
\subsection{Flags}
\label{sec:flags}

The package makes it easy to generate different versions
of the main or child documents.
To this end compilation flags can be defined
and assigned different default values.
They will be particularly useful in conjunction
with the forwarding mechanism described in \secref{sec:forward}.

For example, it may be useful to have a flag |\version|
which can be set to |draft| or |final|.
The document source will contain some conditional code
depending on the value of |\version|.
Suppose further, the flag should default to |final| for the main file
and to |draft| for child files
which is a natural assignment for editing the document.
This is achieved by placing the following code
in the preamble of the main document
(below the |\childdocmain| directive):
%
\begin{center}
\begin{tabular}{l}
|\ifchilddoc|\\
|\providecommand{\version}{draft}|\\
|\||else|\\
|\providecommand{\version}{final}|\\
|\||fi|
\end{tabular}
\end{center}
%
The definition by |\providecommand| makes sure
that previous definitions are not overwritten.
Further statements |\providecommand{\version}{...}|
can thus be added before the above code to override it.

For the main file, one might add a line
(between |\childdocmain| and the above block)
%
\begin{center}
|%\ifchilddoc\||else\providecommand{\version}{draft}\||fi|
\end{center}
%
which can be uncommented to produce a draft version.
Likewise one can add a line to the very top of a child file
(above the |\childdocof{|\textit{main}|}| directive)
%
\begin{center}
|%\providecommand{\version}{final}|
\end{center}
%
which can be uncommented to produce the final version of this child document.

%%%%%%%%%%%%%%%%%%%%%%%%%%%%%%%%%%%%%%%%%%%%%%%%%%%%%%%%%%%%%%%%%%%%%%%%%%%%%%%%
\subsection{Forwarding}
\label{sec:forward}

Different versions of the main or child documents
using compilation flags as described in \secref{sec:flags}
can be (permanently) stored in different files
for convenient compilation, viewing and distribution.
To this end, the package defines a command
to pass on compilation to a different file:

%%%%%%%%%%%%%%%%%%%%%%%%%%%%%%%%%%%%%%%%
\DescribeMacro{\childdocforward}
The command |\childdocforward| redirects processing to
another source file:
%
\begin{center}
\begin{tabular}{l}
|\input{childdoc.def}|\\
|\childdocforward[|\textit{main}|]{|\textit{dest}|}|\\
\end{tabular}
\end{center}
%
The argument \textit{dest} is the destination file
(without extension).
It should be the main file or one of the child files.
Note that further \textsf{childdoc} directives
such as |\childdocof| and |\childdocforward|
in the indicated file will be processed in this form.
The optional argument \textit{main}
passes on directly to the main file \textit{main}
while pretending to compile the child \textit{dest}.
This form behaves as if \textit{dest}
issues |\childdocof{|\textit{main}|}| right away,
and no further \textsf{childdoc} directives will be processed.

%%%%%%%%%%%%%%%%%%%%%%%%%%%%%%%%%%%%%%%%
\DescribeMacro{\...prefix}
In the alternative form |\childdocforwardprefix|,
%
\begin{center}
\begin{tabular}{l}
|\input{childdoc.def}|\\
|\childdocforwardprefix[|\textit{main}|]{|\textit{prefix}|}{|\textit{dest}|}|
\end{tabular}
\end{center}
%
the destination file is determined by a pattern
depending on the current file:
To make this work, the current file must be called
`{\textit{prefix}\hspace{0.2em}\textit{suffix}}'
with \textit{prefix} matching precisely the argument.
Processing is then passed on to the file
`{\textit{dest}\hspace{0.2em}\textit{suffix}}'.
Surely, the same effect is achieved by
directly specifying the
argument `{\textit{dest}\hspace{0.2em}\textit{suffix}}'
in the first form.
However, that requires to set up a different file
for each child. With the alternative form of the command
all these files can have exactly the same content
which simplifies setting them up and maintaining them.

For example, the following file |draft.tex|
with a compilation flag |\version| as described in \secref{sec:flags}
compiles the main document as a draft:
%
\begin{center}
\begin{tabular}{l}
|\def\version{draft}|\\
|\input{childdoc.def}|\\
|\childdocforward{|\textit{main}|}|
\end{tabular}
\end{center}
%
Likewise, the following files |final|\textit{nn}|.tex|
compile the final version of the child document
|child|\textit{nn}|.tex|:
%
\begin{center}
\begin{tabular}{l}
|\def\version{final}|\\
|\input{childdoc.def}|\\
|\childdocforwardprefix{final}{child}|
\end{tabular}
\end{center}
%

Note that when several versions of a main file and/or of each child file
are to be generated, it may be convenient to set up a |Makefile| or
shell script to automatise the process.

%%%%%%%%%%%%%%%%%%%%%%%%%%%%%%%%%%%%%%%%%%%%%%%%%%%%%%%%%%%%%%%%%%%%%%%%%%%%%%%%
\subsection{Command Line Processing}
\label{sec:commandline}

The effect of redirection files can also be achieved by invoking
the \LaTeX{} compiler with a more elaborate command line.
Most conveniently this should be done as part
of a shell script or a |Makefile|.

When using \textsf{childdoc} in the main file, the following
command lines effectively perform a redirection
(note that depending on the shell being used,
backslashes may have to be doubled: `|\|' $\to$ `|\\|'):
%
\begin{center}
|... -jobname "|\textit{target}|" |\\|"|[\textit{flags}]%
|\input{childdoc.def}\childdocforward[|\textit{main}|]{|\textit{dest}|}"|
\end{center}
%
Here \textit{target} is the name of the output file,
\textit{main} is the name of the main file
and \textit{dest} is the name of the main or child file to be processed
(all filenames without extensions).
The optional argument \textit{main} can be omitted
if \textit{main} matches \textit{dest}.
Optionally, compilation \textit{flags} can be defined via |\def| commands.
This command line makes the \TeX{} engine believe
it is compiling the file \textit{target}
whose content is specified as the latter parameter.
The provided code then forwards the processing to
\textit{main} or \textit{dest} as described in \secref{sec:forward}.

%%%%%%%%%%%%%%%%%%%%%%%%%%%%%%%%%%%%%%%%%%%%%%%%%%%%%%%%%%%%%%%%%%%%%%%%%%%%%%%%
\subsection{Include by Input}
\label{sec:input}

Including child documents by |\include| has some restrictions by design.
Most notably, the content of a child document always occupies
its own set of pages; pages cannot be shared between child documents.
Usually, this behaviour makes perfect sense
because each child document contain an essential part of the document.
However, in some situations it may be desirable to compose
a document from a collection of parts
without having mandatory page breaks between then.
For this case, the package
provides a mechanism to include parts
by |\input| which can also be processed individually.
However, by construction this mechanism
requires manual handling of the content to be output.

%%%%%%%%%%%%%%%%%%%%%%%%%%%%%%%%%%%%%%%%
\DescribeMacro{\ifchilddocmanual}
The main file should be prepared as usual, see \secref{sec:include}.
However, the document body must make a distinction
between processing of an individual part and of the main document, e.g.:
%
\begin{center}
\begin{tabular}{l}
|\ifchilddocmanual|\\
|\input{\childdocname}|\\
|\||else|\\
\textit{document body with }|\input{|\textit{part}|}|\\
|\||fi|
\end{tabular}
\end{center}
%
The conditional |\ifchilddocmanual| is true whenever
a part to be included by |\input| is being compiled,
and the name of the part is stored in |\childdocname|.

%%%%%%%%%%%%%%%%%%%%%%%%%%%%%%%%%%%%%%%%
\DescribeMacro{\childdocby}
Each part to be included by |\input| should start with:
%
\begin{center}
\begin{tabular}{l}
|\input{childdoc.def}|\\
|\childdocby{|\textit{main}|}|\\
\end{tabular}
\end{center}
%
The directive |\childdocby| is similar to |\childdocof|
described in \secref{sec:include},
but the subsequent selection of content must be done manually.
To that end, both |\ifchilddoc| and |\ifchilddocmanual|
will be true upon processing of a part,
and the name of the part is stored in |\childdocname|.
Note that |\jobname| will be set to the filename of the current part
so that each part receives an individual |.aux| file
that does not interfere with the |.aux| file(s) of the main document.
This behaviour can be altered by the alternative form
|\childdocby[*]{|\textit{main}|}| (with a non-empty optional argument)
which uses the |.aux| file of the main document
by setting |\jobname| to \textit{main}.

%%%%%%%%%%%%%%%%%%%%%%%%%%%%%%%%%%%%%%%%%%%%%%%%%%%%%%%%%%%%%%%%%%%%%%%%%%%%%%%%
\subsection{Driver Development}
\label{sec:driver}

The \textsf{childdoc} mechanism can also be use for the development
of definition files such as \LaTeX{} styles or classes.
This case differs from the above setup with multiple parts
included by |\include| in that no |\includeonly| should be invoked.
This can be achieved by starting the include file
(before |\ProvidesPackage|) with:
%
\begin{center}
\begin{tabular}{l}
|\input{childdoc.def}|\\
|\childdocforward{|\textit{main}|}|\\
\end{tabular}
\end{center}
%
or alternatively with:
%
\begin{center}
\begin{tabular}{l}
|\input{childdoc.def}|\\
|\childdocby{|\textit{main}|}|\\
\end{tabular}
\end{center}
%
Both forms have slightly different effects as described above.
The main file is prepared as usual, see \secref{sec:include}.

%%%%%%%%%%%%%%%%%%%%%%%%%%%%%%%%%%%%%%%%%%%%%%%%%%%%%%%%%%%%%%%%%%%%%%%%%%%%%%%%
\subsection{Legacy Detection}
\label{sec:detection}

The directive |\childdocmain| in the main file can detect
whether the complete document or merely a child is to be compiled
even without using the directive |\childdocof|.
This method is deprecated because it is less robust
and there is no compelling reason to use it;
it is merely provided for backward compatibility
and it may be removed in future versions.

If the detection mechanism is to be used,
it is mandatory to correctly specify
the filename of the main file as the argument of |\childdocmain|:
%
\begin{center}
\begin{tabular}{l}
|\input{childdoc.def}|\\
|\childdocmain{|\textit{main}|}|\\
\end{tabular}
\end{center}
%
If |\jobname| does not match the argument \textit{main} of |\childdocmain|,
it is assumed that |\jobname| points to the child file to be compiled.
When using |\childdocmain| with the main file specified as argument,
it suffices to start a child file
with just |\input{|\textit{main}|}|
without loading of the package and using |\childdocof|.
If instead all processing is done
with the appropriate \textsf{childdoc} directives,
the argument of \textit{main} of |\childdocmain| can be empty.

An alternative version of the command line processing described
in \secref{sec:commandline} using the detection mechanism reads:
%
\begin{center}
|... -jobname "|\textit{target}|" "|[\textit{flags}]%
[|\def\jobname{|\textit{dest}|}|]|\input{|\textit{main}|}"|
\end{center}

%%%%%%%%%%%%%%%%%%%%%%%%%%%%%%%%%%%%%%%%%%%%%%%%%%%%%%%%%%%%%%%%%%%%%%%%%%%%%%%%
\subsection{Manual Code}
\label{sec:manual}

In case one cannot be certain whether the definitions file |childdoc.def|
is installed on the target \TeX{} distribution
and one prefers not to ship it,
it is conceivable to paste a few relevant commands into the sources.

To that end, drop all statements |\input{childdoc.def}|
and perform the replacements as outlined below.
Instead of |\childdocmain{|\textit{main}|}| add the following code
to the top of the main file:
%
\begin{center}
\begin{tabular}{l}
|\||ifdefined\childdocname\endinput\||fi\newif\ifchilddoc|\\
|\edef\childdocname{\scantokens\expandafter{\jobname\noexpand}}|\\
|\def\childdocmain{|\textit{main}|}\||ifx\childdocmain\childdocname\||else|\\
|\childdoctrue\includeonly{\childdocname}\let\jobname\childdocmain\||fi|\\
\end{tabular}
\end{center}
%
Instead of |\childdocof{|\textit{main}|}| just include the main file
at the top of each child file:
%
\begin{center}
|\input{|\textit{main}|}|
\end{center}
%
A simple redirection |\childdocforward{|\textit{dest}|}| is achieved by:
%
\begin{center}
|\def\jobname{|\textit{dest}|}\input{\jobname}|
\end{center}
%
The redirection with prefix
|\childdocforwardprefix[|\textit{prefix}|]{|\textit{dest}|}|
is accomplished by:
%
\begin{center}
\begin{tabular}{l}
|{\edef\jobname{\scantokens\expandafter{\jobname\noexpand}}|\\
|\def\redirectjob |\textit{prefix}|#1~~~{\gdef\jobname{|\textit{dest}|#1}}|\\
|\expandafter\redirectjob\jobname~~~}\input{\jobname}|
\end{tabular}
\end{center}

In an alternative approach,
child documents can be compiled by a specific command line
without additional code or specific definitions:
%
\begin{center}
|... -jobname "|\textit{target}|" "|[\textit{flags}]%
|\includeonly{|\textit{dest}|}\input{|\textit{main}|}"|
\end{center}
%

%%%%%%%%%%%%%%%%%%%%%%%%%%%%%%%%%%%%%%%%%%%%%%%%%%%%%%%%%%%%%%%%%%%%%%%%%%%%%%%%
%%%%%%%%%%%%%%%%%%%%%%%%%%%%%%%%%%%%%%%%%%%%%%%%%%%%%%%%%%%%%%%%%%%%%%%%%%%%%%%%
\section{Information}

%%%%%%%%%%%%%%%%%%%%%%%%%%%%%%%%%%%%%%%%%%%%%%%%%%%%%%%%%%%%%%%%%%%%%%%%%%%%%%%%
\subsection{Copyright}

Copyright \copyright{} 2017--2018 Niklas Beisert

This work may be distributed and/or modified under the
conditions of the \LaTeX{} Project Public License, either version 1.3
of this license or (at your option) any later version.
The latest version of this license is in
  \url{http://www.latex-project.org/lppl.txt}
and version 1.3 or later is part of all distributions of \LaTeX{}
version 2005/12/01 or later.

This work has the LPPL maintenance status `maintained'.

The Current Maintainer of this work is Niklas Beisert.

This work consists of the files |README.txt|, |childdoc.ins| and |childdoc.dtx|
as well as the derived files |childdoc.def|, |cdocsamp.tex|
with |cdocsch1.tex|, |cdocsch2.tex|, |cdocspt3.tex|, |cdocspt4.tex|,
|cdocsdrf.tex|, |cdocsfn1.tex|, |cdocsfn2.tex|
as well as |childdoc.pdf|.

%%%%%%%%%%%%%%%%%%%%%%%%%%%%%%%%%%%%%%%%%%%%%%%%%%%%%%%%%%%%%%%%%%%%%%%%%%%%%%%%
\subsection{Files and Installation}

The package consists of the files:
%
\begin{center}
\begin{tabular}{ll}
    |README.txt|   & readme file \\
    |childdoc.ins| & installation file \\
    |childdoc.dtx| & source file \\
    |childdoc.def| & definition file \\
    |cdocsamp.tex| & sample main file \\
    |cdocsch1.tex| & sample include file \\
    |cdocsch2.tex| & sample include file \\
    |cdocspt3.tex| & sample part file \\
    |cdocspt4.tex| & sample part file \\
    |cdocsdrf.tex| & sample redirection file \\
    |cdocsfn1.tex| & sample redirection file \\
    |cdocsfn2.tex| & sample redirection file \\
    |childdoc.pdf| & manual
\end{tabular}
\end{center}
%
The distribution consists of the files
|README.txt|, |childdoc.ins| and |childdoc.dtx|.
%
\begin{itemize}
\item
Run (pdf)\LaTeX{} on |childdoc.dtx|
to compile the manual |childdoc.pdf| (this file).
\item
Run \LaTeX{} on |childdoc.ins| to create the definitions file |childdoc.def|
and the sample |cdocsamp.tex| with include files
|cdocsch1.tex|, |cdocsch2.tex|, |cdocspt3.tex|, |cdocspt4.tex|,
|cdocsdrf.tex|, |cdocsfn1.tex|, |cdocsfn2.tex|.
Then copy the file |childdoc.def| to an appropriate directory of your \LaTeX{}
distribution, e.g.\ \textit{texmf-root}|/tex/latex/childdoc|.
\end{itemize}

%%%%%%%%%%%%%%%%%%%%%%%%%%%%%%%%%%%%%%%%%%%%%%%%%%%%%%%%%%%%%%%%%%%%%%%%%%%%%%%%
\subsection{Related CTAN Packages}

There are several other packages which offer a similar functionality:
%
\begin{itemize}
\item
The packages
\href{http://ctan.org/pkg/docmute}{\textsf{docmute}},
\href{http://ctan.org/pkg/includex}{\textsf{includex}} and
\href{http://ctan.org/pkg/standalone}{\textsf{standalone}}
provide commands to include only the document body of
a child file thus allowing both files to be compiled individually.
\item
The packages \href{http://ctan.org/pkg/subdocs}{\textsf{subdocs}}
and \href{http://ctan.org/pkg/subfiles}{\textsf{subfiles}}
provide structures in which the main and child documents can be
encapsulated and allowing them to be compiled individually.
The inclusion mechanism is different from the conventional |\include|.
\item
The package \href{http://ctan.org/pkg/combine}{\textsf{combine}}
is an elaborate solution to combine several documents into one.
\end{itemize}
%
See also the CTAN topic \href{http://ctan.org/topic/subdocs}{\textsf{subdocs}}
for further related packages.
The present package differs from the above solutions in that
a document structure constructed with the conventional |\include| mechanism
just needs two extra commands at the top of every file
such that all constituent files can be compiled individually.

%%%%%%%%%%%%%%%%%%%%%%%%%%%%%%%%%%%%%%%%%%%%%%%%%%%%%%%%%%%%%%%%%%%%%%%%%%%%%%%%
%\subsection{Feature Suggestions}
%
%The following is a list of features which may be useful for future
%versions of this package:
%%
%\begin{itemize}
%\item
%\ldots
%\end{itemize}

%%%%%%%%%%%%%%%%%%%%%%%%%%%%%%%%%%%%%%%%%%%%%%%%%%%%%%%%%%%%%%%%%%%%%%%%%%%%%%%%
\subsection{Revision History}

%%%%%%%%%%%%%%%%%%%%%%%%%%%%%%%%%%%%%%%%
\paragraph{v2.0:} 2018/12/30

\begin{itemize}
\item
immediate forward processing
\item
added |\childdocby| mechanism
\item
manual restructured
\end{itemize}

%%%%%%%%%%%%%%%%%%%%%%%%%%%%%%%%%%%%%%%%
\paragraph{v1.6:} 2018/01/17

\begin{itemize}
\item
application for development of include files
\item
corrections to manual
\end{itemize}

%%%%%%%%%%%%%%%%%%%%%%%%%%%%%%%%%%%%%%%%
\paragraph{v1.5:} 2017/05/21

\begin{itemize}
\item
more complete structuring introduced
\item
|\childdocof| introduced
\item
|\childdoc| renamed to |\childdocmain|
\item
|\childredirect| renamed to |\childdocforward| and |\childdocforwardprefix|
and functionality expanded
\end{itemize}

%%%%%%%%%%%%%%%%%%%%%%%%%%%%%%%%%%%%%%%%
\paragraph{v1.0:} 2017/04/27

\begin{itemize}
\item
manual and install package
\item
first version published on CTAN
\end{itemize}

%%%%%%%%%%%%%%%%%%%%%%%%%%%%%%%%%%%%%%%%
\paragraph{v0.6:} 2017/04/26

\begin{itemize}
\item
redirection mechanism added
\end{itemize}

%%%%%%%%%%%%%%%%%%%%%%%%%%%%%%%%%%%%%%%%
\paragraph{v0.5:} 2017/04/26

\begin{itemize}
\item
functionality in definition file
\end{itemize}


%%%%%%%%%%%%%%%%%%%%%%%%%%%%%%%%%%%%%%%%%%%%%%%%%%%%%%%%%%%%%%%%%%%%%%%%%%%%%%%%
%%%%%%%%%%%%%%%%%%%%%%%%%%%%%%%%%%%%%%%%%%%%%%%%%%%%%%%%%%%%%%%%%%%%%%%%%%%%%%%%
%%%%%%%%%%%%%%%%%%%%%%%%%%%%%%%%%%%%%%%%%%%%%%%%%%%%%%%%%%%%%%%%%%%%%%%%%%%%%%%%
\appendix

\settowidth\MacroIndent{\rmfamily\scriptsize 000\ }

 \DocInput{childdoc.dtx}

\end{document}
%</driver>
% \fi
%
% %%%%%%%%%%%%%%%%%%%%%%%%%%%%%%%%%%%%%%%%%%%%%%%%%%%%%%%%%%%%%%%%%%%%%%%%%%%%%%
% %%%%%%%%%%%%%%%%%%%%%%%%%%%%%%%%%%%%%%%%%%%%%%%%%%%%%%%%%%%%%%%%%%%%%%%%%%%%%%
% \section{Sample}
%\iffalse
%<*samplemain>
%\fi
%
% The following presents a sample document
% with two chapters, two parts, a title page,
% a compile flag as well as three forwarding files to set the flag.
% It consists of eight |.tex| files:
% \begin{center}
% \begin{tabular}{ll}
% |cdocsamp.tex|&main file\\
% |cdocsch1.tex|&include file for chapter 1\\
% |cdocsch2.tex|&include file for chapter 2\\
% |cdocspt3.tex|&include file for part 3\\
% |cdocspt4.tex|&include file for part 4\\
% |cdocsdrf.tex|&forwarding file for main file in draft mode\\
% |cdocsfi1.tex|&forwarding file for final version of chapter 1\\
% |cdocsfi2.tex|&forwarding file for final version of chapter 2\\
% \end{tabular}
% \end{center}
% Each of the eight files can be compiled directly by the \LaTeX{} compiler.
%
% %%%%%%%%%%%%%%%%%%%%%%%%%%%%%%%%%%%%%%
% \paragraph{Main File.}
%
% The main file is called |cdocsamp.tex|.
%
% Load the \textsf{childdoc} definitions and
% declare the filename for the main document:
%    \begin{macrocode}
\input{childdoc.def}
\childdocmain{}
%    \end{macrocode}

% Optional override for |\version| flag:
%    \begin{macrocode}
%%\ifchilddoc\else\providecommand{\version}{draft}\fi
%    \end{macrocode}

% Define the default values for the |\version| flag
% (|final| for the main file and |draft| for childs):
%    \begin{macrocode}
\ifchilddoc
\providecommand{\version}{draft}
\else
\providecommand{\version}{final}
\fi
%    \end{macrocode}

% Load the standard document class:
%    \begin{macrocode}
\documentclass[12pt]{article}
%    \end{macrocode}

% Start the document body:
%    \begin{macrocode}
\begin{document}
%    \end{macrocode}

% Declare a title page.
% Print title, part of document being processed and version flag:
%    \begin{macrocode}
\addtocounter{page}{-1}
\begin{center}
{\LARGE\bfseries{}childdoc example\par}
\vspace{1cm}
\ifchilddoc
\ifchilddocmanual part\else chapter\fi:
`\childdocname' of `\childdocjob'\par
\else
main document: `\childdocjob'\par
\fi
version: \version\par
\end{center}
\newpage
%    \end{macrocode}

% Manually include selected file,
% otherwise process as usual:
%    \begin{macrocode}
\ifchilddocmanual
\section*{part `\childdocname'}
\input{\childdocname}
\else
%    \end{macrocode}

% Include the two chapters:
%    \begin{macrocode}
\include{cdocsch1}
\include{cdocsch2}
%    \end{macrocode}

% Include the two parts unless only chapters should be displayed:
%    \begin{macrocode}
\ifchilddoc\else
\section{part three}
\input{cdocspt3}
\section{part four}
\input{cdocspt4}
\fi
%    \end{macrocode}

% Process as usual until here:
%    \begin{macrocode}
\fi
%    \end{macrocode}

% End of document body:
%    \begin{macrocode}
\end{document}
%    \end{macrocode}
%\iffalse
%</samplemain>
%\fi
%
% %%%%%%%%%%%%%%%%%%%%%%%%%%%%%%%%%%%%%%
% \paragraph{Chapter Include Files.}
%
% The include files are called |cdocsch1.tex| and |cdocsch2.tex|.
%
%\iffalse
%<*samplechap1|samplechap2>
%\fi

% Optional override for |\version| flag:
%    \begin{macrocode}
%%\providecommand{\version}{final}
%    \end{macrocode}

% Include the main document:
%    \begin{macrocode}
\input{childdoc.def}
\childdocof{cdocsamp}
%    \end{macrocode}

%\iffalse
%</samplechap1|samplechap2>
%\fi
%
%\iffalse
%<*samplechap1>
%\fi
% Some text for chapter 1:
%    \begin{macrocode}
\section{one}
some text in chapter one
%    \end{macrocode}

%\iffalse
%</samplechap1>
%\fi
% Some text for chapter 2:
%\iffalse
%<*samplechap2>
%\fi
%    \begin{macrocode}
\section{two}
more text in chapter two
%    \end{macrocode}

%\iffalse
%</samplechap2>
%\fi
%
% %%%%%%%%%%%%%%%%%%%%%%%%%%%%%%%%%%%%%%
% \paragraph{Part Include Files.}
%
% The include files are called |cdocspt3.tex| and |cdocspt4.tex|.
%
%\iffalse
%<*samplepart3|samplepart4>
%\fi

% Optional override for |\version| flag:
%    \begin{macrocode}
%%\providecommand{\version}{final}
%    \end{macrocode}

% Include the main document:
%    \begin{macrocode}
\input{childdoc.def}
\childdocby{cdocsamp}
%    \end{macrocode}

%\iffalse
%</samplepart3|samplepart4>
%\fi
%
%\iffalse
%<*samplepart3>
%\fi
% Some text for part 3:
%    \begin{macrocode}
some text in part three
%    \end{macrocode}

%\iffalse
%</samplepart3>
%\fi
% Some text for part 4:
%\iffalse
%<*samplepart4>
%\fi
%    \begin{macrocode}
more text in part four
%    \end{macrocode}

%\iffalse
%</samplepart4>
%\fi
%
% %%%%%%%%%%%%%%%%%%%%%%%%%%%%%%%%%%%%%%
% \paragraph{Forwarding for a Complete Draft.}
%
% The following forwarding file |cdocsdrf.tex|
% compiles the main document in draft mode:
%\iffalse
%<*sampledraft>
%\fi
%    \begin{macrocode}
\def\version{draft}
\input{childdoc.def}
\childdocforward{cdocsamp}
%    \end{macrocode}

%\iffalse
%</sampledraft>
%\fi
%
% %%%%%%%%%%%%%%%%%%%%%%%%%%%%%%%%%%%%%%
% \paragraph{Forwarding for Final Version of the Chapters.}
%
% The following forwarding files |cdocsfn1.tex| and |cdocsfn2.tex|
% (with identical content)
% compile the final versions of the child documents
% |cdocsch1.tex| and |cdocsch2.tex|, respectively:
%\iffalse
%<*samplefinal>
%\fi
%    \begin{macrocode}
\def\version{final}
\input{childdoc.def}
\childdocforwardprefix[cdocsamp]{cdocsfn}{cdocsch}
%    \end{macrocode}

%\iffalse
%</samplefinal>
%\fi
%
% %%%%%%%%%%%%%%%%%%%%%%%%%%%%%%%%%%%%%%
% \paragraph{Command Line Processing.}
%
% The following three command lines generate the output files
% |cdocscld|, |cdocscl1| and |cdocscl2|
% which should be identical to
% |cdocsdrf|, |cdocsch1| and |cdocsfn2|, respectively:
% \begin{center}
% \begin{tabular}{l}
% |latex -jobname cdocscld \|\\
% |  "\def\version{draft}\input{childdoc.def}\childdocforward{cdocsamp}"|\\
% |latex -jobname cdocscl1 \|\\
% |  "\input{childdoc.def}\childdocforward[cdocsamp]{cdocsch1}"|\\
% |latex -jobname cdocscl2 \|\\
% |  "\def\version{final}\input{childdoc.def}\childdocforward{cdocsch2}"|
% \end{tabular}
% \end{center}
% Note that the trailing backslash on each first line
% merely continues the input to the second line
% (for convenient cut ant paste).
% Furthermore, the command |latex| can be replaced by any
% of its alternative versions such as |pdflatex|.
%
% %%%%%%%%%%%%%%%%%%%%%%%%%%%%%%%%%%%%%%%%%%%%%%%%%%%%%%%%%%%%%%%%%%%%%%%%%%%%%%
% %%%%%%%%%%%%%%%%%%%%%%%%%%%%%%%%%%%%%%%%%%%%%%%%%%%%%%%%%%%%%%%%%%%%%%%%%%%%%%
% \section{Implementation}
%\iffalse
%<*package>
%\fi
%
% This section describes the definitions file |childdoc.def|.

% The definitions cannot be loaded using |\usepackage| or |\RequirePackage|
% which has a mechanism to prevent loading a style file more than once.
% When loading the definitions by means of |\input|
% multiple instances have to be prevented manually:
%\iffalse
%This code needs to be before the `\ProvidesFile' directive
%which is defined at the beginning of this file.
%Therefore it is also placed there and commented out here.
%</package>
%<*discard>
%\fi
%    \begin{macrocode}
\ifdefined\childdocmain\endinput\fi
%    \end{macrocode}
%\iffalse
%</discard>
%<*package>
%\fi
%
% \macro{\ifchilddoc}
% \macro{\ifchilddocmanual}
% The conditional |\ifchilddoc| tells whether a
% child (true) or main (false) document is being compiled.
% The conditional |\ifchilddocmanual| tells whether
% the |\includeonly| mechanism is used (false) or
% the selection of child files must be performed manually (true).
% The definitions initialise to false:
%    \begin{macrocode}
\newif\ifchilddoc
\newif\ifchilddocmanual
%    \end{macrocode}

% \macro{\childdocname}
% \macro{\childdocjob}
% The macro |\childdocname| stores the name of the main document
% to be compiled. The macro |\childdocjob| stores the name of
% the document on which the \LaTeX{} compiler was originally invoked.
% The content of |\jobname| cannot be compared
% to filenames specified in the source due to different catcodes.
% The following code rescans |\jobname|, stores the result
% in |\childdocname| and saves a copy in |\childdocjob|:
%    \begin{macrocode}
\edef\childdocname{\scantokens\expandafter{\jobname\noexpand}}
\let\childdocjob\childdocname
%    \end{macrocode}

% \macro{\childdocdisable}
% The macro |\childdocdisable| prevents the main file
% from being processed more than once.
% At this stage, the main document command |\childdocmain|
% is assumed to be called once again where it should do nothing.
% Any subsequent call to it should prevent
% a secondary processing of the main document
% It overwrites the forwarding commands
% |\childdocof| and |\childdocforward|
% with empty macros to prevent further inclusions of the main document:
%    \begin{macrocode}
\newcommand{\childdocdisable}
{
  \renewcommand{\childdocmain}[1]{\renewcommand{\childdocmain}[1]{\endinput}}
  \renewcommand{\childdocof}[1]{}
  \renewcommand{\childdocby}[2][]{}
  \renewcommand{\childdocforward}[2][]{}
  \renewcommand{\childdocdisable}{}
}
%    \end{macrocode}

% \macro{\childdocmain}
% The macro |\childdocmain| is to be called at the top of the main file
% with nothing or the main filename (without extension) as argument.
% First, it breaks loops.
% If the argument is not empty and does not match |\childdocname|
% (which is set by the first inclusion of |childdoc.def|),
% |\ifchilddoc| is set to true, |\includeonly| is applied to the child file
% and |\jobname| is set to the main file
% (for proper handling of |.aux| files):
%    \begin{macrocode}
\newcommand{\childdocmain}[1]
{
  \childdocdisable\childdocmain{}
  \if?#1?\else
    \begingroup
      \def\childdoctmp{#1}
      \ifx\childdoctmp\childdocname
        \def\childdoctmp{}
      \else
        \def\childdoctmp
        {
          \childdoctrue
          \includeonly{\childdocname}
          \def\childdocjob{#1}
          \def\jobname{#1}
        }
      \fi
      \expandafter
    \endgroup
    \childdoctmp
  \fi
}
%    \end{macrocode}

% \macro{\childdocof}
% The command |\childdocof| redirects
% compilation to the main file |#1|.
%    \begin{macrocode}
\newcommand{\childdocof}[1]
{
  \childdocdisable
  \childdoctrue
  \includeonly{\childdocname}
  \def\jobname{#1}
  \def\childdocjob{#1}
  \input{#1}
}
%    \end{macrocode}

% \macro{\childdocby}
% The command |\childdocby| ....
%    \begin{macrocode}
\newcommand{\childdocby}[2][]
{
  \childdocdisable
  \childdoctrue
  \childdocmanualtrue
  \if?#1?\else
    \def\jobname{#2}
  \fi
  \def\childdocjob{#2}
  \input{#2}
  \endinput
}
%    \end{macrocode}

% \macro{\childdocforward}
% The command |\childdocforward| redirects
% compilation to the main file or
% (if the optional argument is given) a child file.
% Parameters are set as if the main file
% or a child file starting with |\childdocof| was compiled.
% Then compilation is handed over to the main file:
%    \begin{macrocode}
\newcommand{\childdocforward}[2][]
{
  \begingroup
    \if?#1?
      \def\childdoctmp
      {
        \def\childdocname{#2}
        \def\childdocjob{#2}
        \def\jobname{#2}
        \input{#2}
        \endinput
      }
    \else
      \def\childdoctmp
      {
        \childdocdisable
        \def\childdocname{#2}
        \childdoctrue
        \includeonly{#2}
        \def\childdocjob{#1}
        \def\jobname{#1}
        \input{#1}
        \endinput
      }
    \fi
    \expandafter
  \endgroup
  \childdoctmp
}
%    \end{macrocode}

% \macro{\childdocforwardprefix}
% The command |\childdocforwardprefix| redirects
% compilation to the main or a child file by means of a pattern.
% The prefix |#1| in the current filename is replaced by |#2|
% and the suffix of the current filename is kept
% (it is assumed that the filename does not contain the substring `|~~~|'
% which is used as a delimiter).
% Compilation is handed over to the new file by |\childdocforward|:
%    \begin{macrocode}
\newcommand{\childdocforwardprefix}[3][]
{
  \begingroup
    \def\childdocextract #2##1~~~{\def\childdoctmp{\childdocforward[#1]{#3##1}}}
    \expandafter\childdocextract\childdocname~~~
    \expandafter
  \endgroup
  \childdoctmp
}
%    \end{macrocode}

% \macro{\childdoc}
% The deprecated macro |\childdoc| is a legacy version of |\childdocmain|:
%    \begin{macrocode}
\newcommand{\childdoc}{\childdocmain}
%    \end{macrocode}

% \macro{\childdocredirect}
% The deprecated macro |\childdocredirect| is a legacy version
% of |\childdocforward| and |\childdocforwardprefix|:
%    \begin{macrocode}
\newcommand{\childdocredirect}[2][]
{
  \begingroup
    \if?#1?
      \def\childdoctmp{\childdocforward{#2}}
    \else
      \def\childdoctmp{\childdocforwardprefix{#1}{#2}}
    \fi
    \expandafter
  \endgroup
  \childdoctmp
}
%    \end{macrocode}

%\iffalse
%</package>
%\fi
%
\endinput
|\\
|\childdocof{|\textit{main}|}|\\
\end{tabular}
\end{center}
at the top of every child file \textit{child}
which is included by |\include{|\textit{child}|}|
from within the main file
(or at least for those files to be compiled individually).
The argument \textit{main} must be the filename of the main file.

There are a couple of
considerations in setting up the main and child documents:

%%%%%%%%%%%%%%%%%%%%%%%%%%%%%%%%%%%%%%%%
\paragraph{Restrictions.}

Please note the following restrictions:
\begin{itemize}
\item
|\childdocmain| must be called with one argument \textit{main}
to ensure compatibility with earlier version of the package.
It must either be empty (|\childdocmain{}|)
or precisely match the filename of the main file in which it is specified.
See \secref{sec:detection} for further information.
\item
The filename \textit{main} must be specified without the |.tex| extension.
\item
The filename \textit{main} is case sensitive
(even in case-insensitive file systems)
due to internal string comparison.
\item
The argument \textit{main} should be fully expanded, it cannot be a macro.
\item
Subdirectories and special characters should be avoided in filenames.
\item
The command |\childdocmain{|\textit{main}|}| must be followed by a whitespace.
It should not be followed immediately by another command
or by a comment mark `|%|'.
This is because the \TeX{} parser reads the token immediately following
the argument of |\childdocmain| and puts it
at the beginning of every child section;
however, a white\-space is ignored.
\end{itemize}

%%%%%%%%%%%%%%%%%%%%%%%%%%%%%%%%%%%%%%%%
\paragraph{Content of Main File.}

It is advisable to place all content in the child files included by |\include|.
Any output contained in the main file will appear in all child documents
unless suppressed manually;
it cannot be suppressed automatically by the |\includeonly| directive
and thus should normally be avoided.
A method to include some content in the main file
by means of conditional processing is described in \secref{sec:conditional}.

%%%%%%%%%%%%%%%%%%%%%%%%%%%%%%%%%%%%%%%%
\paragraph{Page Numbering.}

When only a part of the document is compiled,
the appropriate numbering of pages
(as well as other status parameters)
is determined from the |.aux| files.
The latter contain information from previous passes.
However this information needs to propagate through
all intermediate child documents.
Therefore the page numbering in child documents may well
be inconsistent until the complete document is compiled at least once.

A useful (if unconventional) way to always ensure a consistent
page numbering is to restart the numbering in each child document
and denote the pages by `\textit{child}|.|\textit{page}'
where \textit{child} represents the chapter/section number of the child file.
This can be achieved by the command
|\numberwithin{page}{|\textit{child}|}|
of the \textsf{amsmath} package
where \textit{child} can be |chapter| or |section|
depending on the chosen structuring.
Alternatively, one can modify the macro |\thepage| appropriately
and reset the counter |page| at the start of each child file.

%%%%%%%%%%%%%%%%%%%%%%%%%%%%%%%%%%%%%%%%%%%%%%%%%%%%%%%%%%%%%%%%%%%%%%%%%%%%%%%%
\subsection{Conditional Processing}
\label{sec:conditional}

The package provides a mechanism to compile different versions
of a document. To customise the versions further some conditional processing
can come in handy to distinguish which version is being compiled.
The package provides two macros to describe the compilation context:

%%%%%%%%%%%%%%%%%%%%%%%%%%%%%%%%%%%%%%%%
\DescribeMacro{\ifchilddoc}
The conditional |\ifchilddoc| distinguishes between the compilation of
child documents and the main document:
%
\begin{center}
|\ifchilddoc |\textit{child-code}| |[|\||else |\textit{main-code}]| \||fi|
\end{center}

%%%%%%%%%%%%%%%%%%%%%%%%%%%%%%%%%%%%%%%%
\DescribeMacro{\childdocname}
\DescribeMacro{\childdocjob}
The macro |\childdocname| contains the filename (without extension)
of the main or child file being processed.
Note that |\childdocjob| will always contain the name of the main file.

%%%%%%%%%%%%%%%%%%%%%%%%%%%%%%%%%%%%%%%%
\paragraph{Title Page.}

Conditional processing can be used to include a title or banner page
in the main document when proper precautions are taken.
Importantly, the code in the main file should ensure that the page counter
(as well as other status parameters which are stored in the |.aux| files)
takes the same value after the conditional processing.
Otherwise the page numbers may take divergent values
depending on which part is compiled.

For example, a title page could be declared by:
%
\begin{center}
\begin{tabular}{l}
|\ifchilddoc\||else|\\
|\addtocounter{page}{-1}|\\
\textit{code for title page}\\
|\newpage|\\
|\||fi|
\end{tabular}
\end{center}
%
A banner page for the child documents can be generated by:
%
\begin{center}
\begin{tabular}{l}
|\ifchilddoc|\\
|\addtocounter{page}{-1}|\\
\textit{code for banner page}\\
|\newpage|\\
|\||fi|
\end{tabular}
\end{center}
%
Here one could write a message such as:
\begin{center}
|This is the part \childdocname{} of \childdocjob{}.|
\end{center}

%%%%%%%%%%%%%%%%%%%%%%%%%%%%%%%%%%%%%%%%%%%%%%%%%%%%%%%%%%%%%%%%%%%%%%%%%%%%%%%%
\subsection{Flags}
\label{sec:flags}

The package makes it easy to generate different versions
of the main or child documents.
To this end compilation flags can be defined
and assigned different default values.
They will be particularly useful in conjunction
with the forwarding mechanism described in \secref{sec:forward}.

For example, it may be useful to have a flag |\version|
which can be set to |draft| or |final|.
The document source will contain some conditional code
depending on the value of |\version|.
Suppose further, the flag should default to |final| for the main file
and to |draft| for child files
which is a natural assignment for editing the document.
This is achieved by placing the following code
in the preamble of the main document
(below the |\childdocmain| directive):
%
\begin{center}
\begin{tabular}{l}
|\ifchilddoc|\\
|\providecommand{\version}{draft}|\\
|\||else|\\
|\providecommand{\version}{final}|\\
|\||fi|
\end{tabular}
\end{center}
%
The definition by |\providecommand| makes sure
that previous definitions are not overwritten.
Further statements |\providecommand{\version}{...}|
can thus be added before the above code to override it.

For the main file, one might add a line
(between |\childdocmain| and the above block)
%
\begin{center}
|%\ifchilddoc\||else\providecommand{\version}{draft}\||fi|
\end{center}
%
which can be uncommented to produce a draft version.
Likewise one can add a line to the very top of a child file
(above the |\childdocof{|\textit{main}|}| directive)
%
\begin{center}
|%\providecommand{\version}{final}|
\end{center}
%
which can be uncommented to produce the final version of this child document.

%%%%%%%%%%%%%%%%%%%%%%%%%%%%%%%%%%%%%%%%%%%%%%%%%%%%%%%%%%%%%%%%%%%%%%%%%%%%%%%%
\subsection{Forwarding}
\label{sec:forward}

Different versions of the main or child documents
using compilation flags as described in \secref{sec:flags}
can be (permanently) stored in different files
for convenient compilation, viewing and distribution.
To this end, the package defines a command
to pass on compilation to a different file:

%%%%%%%%%%%%%%%%%%%%%%%%%%%%%%%%%%%%%%%%
\DescribeMacro{\childdocforward}
The command |\childdocforward| redirects processing to
another source file:
%
\begin{center}
\begin{tabular}{l}
|% \iffalse
%
% childdoc.dtx Copyright (C) 2017-2018 Niklas Beisert
%
% This work may be distributed and/or modified under the
% conditions of the LaTeX Project Public License, either version 1.3
% of this license or (at your option) any later version.
% The latest version of this license is in
%   http://www.latex-project.org/lppl.txt
% and version 1.3 or later is part of all distributions of LaTeX
% version 2005/12/01 or later.
%
% This work has the LPPL maintenance status `maintained'.
%
% The Current Maintainer of this work is Niklas Beisert.
%
% This work consists of the files childdoc.dtx and childdoc.ins
% and the derived files childdoc.def and cdocsamp.tex with
% cdocsch1.tex, cdocsch2.tex, cdocsdrf.tex, cdocsfn1.tex, cdocsfn2.tex.
%
%<package>\ifdefined\childdocmain\endinput\fi
%<package>\ProvidesFile{childdoc.def}[2018/12/30 v2.0 child document driver]
%<samplemain>\ProvidesFile{cdocsamp.tex}[2018/12/30 v2.0 sample for childdoc]
%<*driver>
%\ProvidesFile{childdoc.drv}[2018/12/30 v2.0 childdoc reference manual file]
\PassOptionsToClass{10pt,a4paper}{article}
\documentclass{ltxdoc}

\usepackage[margin=35mm]{geometry}
\usepackage{hyperref}
\usepackage{hyperxmp}
\usepackage[usenames]{color}

\hypersetup{colorlinks=true}
\hypersetup{pdfstartview=FitH}
\hypersetup{pdfpagemode=UseNone}
\hypersetup{pdfsource={}}
\hypersetup{pdflang={en-UK}}
\hypersetup{pdfcopyright={Copyright 2017-2018 Niklas Beisert.
  This work may be distributed and/or modified under the
  conditions of the LaTeX Project Public License, either version 1.3
  of this license or (at your option) any later version.}}
\hypersetup{pdflicenseurl={http://www.latex-project.org/lppl.txt}}
\hypersetup{pdfcontactaddress={ETH Zurich, ITP, HIT K,
  Wolfgang-Pauli-Strasse 27}}
\hypersetup{pdfcontactpostcode={8093}}
\hypersetup{pdfcontactcity={Zurich}}
\hypersetup{pdfcontactcountry={Switzerland}}
\hypersetup{pdfcontactemail={nbeisert@itp.phys.ethz.ch}}
\hypersetup{pdfcontacturl={http://people.phys.ethz.ch/\xmptilde nbeisert/}}

\newcommand{\secref}[1]{\hyperref[#1]{section \ref*{#1}}}

\parskip1ex
\parindent0pt
\let\olditemize\itemize
\def\itemize{\olditemize\parskip0pt}

\begin{document}

\title{The \textsf{childdoc} Package}
\hypersetup{pdftitle={The childdoc Package}}
\author{Niklas Beisert\\[2ex]
  Institut f\"ur Theoretische Physik\\
  Eidgen\"ossische Technische Hochschule Z\"urich\\
  Wolfgang-Pauli-Strasse 27, 8093 Z\"urich, Switzerland\\[1ex]
  \href{mailto:nbeisert@itp.phys.ethz.ch}
  {\texttt{nbeisert@itp.phys.ethz.ch}}}
\hypersetup{pdfauthor={Niklas Beisert}}
\hypersetup{pdfsubject={Manual for the LaTeX2e Package childdoc}}
\date{30 December 2018, \textsf{v2.0}}
\maketitle

\begin{abstract}\noindent
\textsf{childdoc} is a \LaTeXe{} package
that enables the direct compilation
of document sections included by |\include|
to individual files.
\end{abstract}

\begingroup
\parskip0ex
\tableofcontents
\endgroup

%%%%%%%%%%%%%%%%%%%%%%%%%%%%%%%%%%%%%%%%%%%%%%%%%%%%%%%%%%%%%%%%%%%%%%%%%%%%%%%%
%%%%%%%%%%%%%%%%%%%%%%%%%%%%%%%%%%%%%%%%%%%%%%%%%%%%%%%%%%%%%%%%%%%%%%%%%%%%%%%%
\section{Introduction}

\LaTeX{} provides a mechanism to structure a large document (such as a book)
into a main file and several child files (containing the chapters)
using the |\include| command.
This mechanism is beneficial for documents
which span hundreds of pages in order to
make the source file(s) more manageable.
Moreover, compilation can be restricted to
selected child files by means of the |\includeonly| command.
The latter feature can be used to reduce the compilation time while editing
(this was significantly more useful in the earlier days of \LaTeX{})
or to generate a smaller document which is easier to navigate.
Another application of |\includeonly| is to generate
documents consisting of selected parts of the complete document.

However, there are a few drawbacks of the plain |\include| mechanism:
\begin{itemize}
\item
The child files cannot be compiled on their own,
they can only be compiled via the main file.
A naive editing environment
(such as a text editor with an option
to have the current file processed by \LaTeX)
may require one to switch to the main file before compiling;
attempting to compile the child file produces errors.
\item
The main file must be modified (each time)
to adjust the |\includeonly| command
to the present needs. This easily leaves the main file in a messy state.
\item
The generated document will always carry the filename
of the main document. This is inconvenient if
several child files are to be compiled and
to be kept for distribution.
\end{itemize}

The present package provides a simple interface
to make child files individually compilable by \LaTeX{}.
Compiling a child file then has the same effect as compiling
the main file with an |\includeonly| command
to select the appropriate child.
Moreover the generated document will carry the name of the child
rather than the main file.
This resolves all three above issues.

This feature is meant to make the editing of books,
thesis documents and lecture notes somewhat more convenient.
However, the package can also be used efficiently for
composing a series of documents (such as exercise sheets)
which are typically distributed individually.
It then assists the author in generating the individual documents
(potentially in different versions)
as well as a document containing the collected series.
Another application is in developing style files
or other kinds of included material
where compilation of the style file could redirect
to a sample or test file.

%%%%%%%%%%%%%%%%%%%%%%%%%%%%%%%%%%%%%%%%%%%%%%%%%%%%%%%%%%%%%%%%%%%%%%%%%%%%%%%%
%%%%%%%%%%%%%%%%%%%%%%%%%%%%%%%%%%%%%%%%%%%%%%%%%%%%%%%%%%%%%%%%%%%%%%%%%%%%%%%%
\section{Usage}

First of all, the package \textsf{childdoc} is \emph{not} a standard
\LaTeXe{} |.sty| style file! Therefore it needs to be invoked in
a non-standard way.

%%%%%%%%%%%%%%%%%%%%%%%%%%%%%%%%%%%%%%%%%%%%%%%%%%%%%%%%%%%%%%%%%%%%%%%%%%%%%%%%
\subsection{Included Files}
\label{sec:include}

%%%%%%%%%%%%%%%%%%%%%%%%%%%%%%%%%%%%%%%%
\DescribeMacro{\childdocmain}
To use the package, add the commands
\begin{center}
\begin{tabular}{l}
|\input{childdoc.def}|\\
|\childdocmain{}|\\
\end{tabular}
\end{center}
at the very top of the main \LaTeX{} file,
in particular \emph{before} the |\documentclass| statement!
The argument of |\childdocmain| should be left empty
(but it must be present).

%%%%%%%%%%%%%%%%%%%%%%%%%%%%%%%%%%%%%%%%
\DescribeMacro{\childdocof}
Furthermore, add the commands
\begin{center}
\begin{tabular}{l}
|\input{childdoc.def}|\\
|\childdocof{|\textit{main}|}|\\
\end{tabular}
\end{center}
at the top of every child file \textit{child}
which is included by |\include{|\textit{child}|}|
from within the main file
(or at least for those files to be compiled individually).
The argument \textit{main} must be the filename of the main file.

There are a couple of
considerations in setting up the main and child documents:

%%%%%%%%%%%%%%%%%%%%%%%%%%%%%%%%%%%%%%%%
\paragraph{Restrictions.}

Please note the following restrictions:
\begin{itemize}
\item
|\childdocmain| must be called with one argument \textit{main}
to ensure compatibility with earlier version of the package.
It must either be empty (|\childdocmain{}|)
or precisely match the filename of the main file in which it is specified.
See \secref{sec:detection} for further information.
\item
The filename \textit{main} must be specified without the |.tex| extension.
\item
The filename \textit{main} is case sensitive
(even in case-insensitive file systems)
due to internal string comparison.
\item
The argument \textit{main} should be fully expanded, it cannot be a macro.
\item
Subdirectories and special characters should be avoided in filenames.
\item
The command |\childdocmain{|\textit{main}|}| must be followed by a whitespace.
It should not be followed immediately by another command
or by a comment mark `|%|'.
This is because the \TeX{} parser reads the token immediately following
the argument of |\childdocmain| and puts it
at the beginning of every child section;
however, a white\-space is ignored.
\end{itemize}

%%%%%%%%%%%%%%%%%%%%%%%%%%%%%%%%%%%%%%%%
\paragraph{Content of Main File.}

It is advisable to place all content in the child files included by |\include|.
Any output contained in the main file will appear in all child documents
unless suppressed manually;
it cannot be suppressed automatically by the |\includeonly| directive
and thus should normally be avoided.
A method to include some content in the main file
by means of conditional processing is described in \secref{sec:conditional}.

%%%%%%%%%%%%%%%%%%%%%%%%%%%%%%%%%%%%%%%%
\paragraph{Page Numbering.}

When only a part of the document is compiled,
the appropriate numbering of pages
(as well as other status parameters)
is determined from the |.aux| files.
The latter contain information from previous passes.
However this information needs to propagate through
all intermediate child documents.
Therefore the page numbering in child documents may well
be inconsistent until the complete document is compiled at least once.

A useful (if unconventional) way to always ensure a consistent
page numbering is to restart the numbering in each child document
and denote the pages by `\textit{child}|.|\textit{page}'
where \textit{child} represents the chapter/section number of the child file.
This can be achieved by the command
|\numberwithin{page}{|\textit{child}|}|
of the \textsf{amsmath} package
where \textit{child} can be |chapter| or |section|
depending on the chosen structuring.
Alternatively, one can modify the macro |\thepage| appropriately
and reset the counter |page| at the start of each child file.

%%%%%%%%%%%%%%%%%%%%%%%%%%%%%%%%%%%%%%%%%%%%%%%%%%%%%%%%%%%%%%%%%%%%%%%%%%%%%%%%
\subsection{Conditional Processing}
\label{sec:conditional}

The package provides a mechanism to compile different versions
of a document. To customise the versions further some conditional processing
can come in handy to distinguish which version is being compiled.
The package provides two macros to describe the compilation context:

%%%%%%%%%%%%%%%%%%%%%%%%%%%%%%%%%%%%%%%%
\DescribeMacro{\ifchilddoc}
The conditional |\ifchilddoc| distinguishes between the compilation of
child documents and the main document:
%
\begin{center}
|\ifchilddoc |\textit{child-code}| |[|\||else |\textit{main-code}]| \||fi|
\end{center}

%%%%%%%%%%%%%%%%%%%%%%%%%%%%%%%%%%%%%%%%
\DescribeMacro{\childdocname}
\DescribeMacro{\childdocjob}
The macro |\childdocname| contains the filename (without extension)
of the main or child file being processed.
Note that |\childdocjob| will always contain the name of the main file.

%%%%%%%%%%%%%%%%%%%%%%%%%%%%%%%%%%%%%%%%
\paragraph{Title Page.}

Conditional processing can be used to include a title or banner page
in the main document when proper precautions are taken.
Importantly, the code in the main file should ensure that the page counter
(as well as other status parameters which are stored in the |.aux| files)
takes the same value after the conditional processing.
Otherwise the page numbers may take divergent values
depending on which part is compiled.

For example, a title page could be declared by:
%
\begin{center}
\begin{tabular}{l}
|\ifchilddoc\||else|\\
|\addtocounter{page}{-1}|\\
\textit{code for title page}\\
|\newpage|\\
|\||fi|
\end{tabular}
\end{center}
%
A banner page for the child documents can be generated by:
%
\begin{center}
\begin{tabular}{l}
|\ifchilddoc|\\
|\addtocounter{page}{-1}|\\
\textit{code for banner page}\\
|\newpage|\\
|\||fi|
\end{tabular}
\end{center}
%
Here one could write a message such as:
\begin{center}
|This is the part \childdocname{} of \childdocjob{}.|
\end{center}

%%%%%%%%%%%%%%%%%%%%%%%%%%%%%%%%%%%%%%%%%%%%%%%%%%%%%%%%%%%%%%%%%%%%%%%%%%%%%%%%
\subsection{Flags}
\label{sec:flags}

The package makes it easy to generate different versions
of the main or child documents.
To this end compilation flags can be defined
and assigned different default values.
They will be particularly useful in conjunction
with the forwarding mechanism described in \secref{sec:forward}.

For example, it may be useful to have a flag |\version|
which can be set to |draft| or |final|.
The document source will contain some conditional code
depending on the value of |\version|.
Suppose further, the flag should default to |final| for the main file
and to |draft| for child files
which is a natural assignment for editing the document.
This is achieved by placing the following code
in the preamble of the main document
(below the |\childdocmain| directive):
%
\begin{center}
\begin{tabular}{l}
|\ifchilddoc|\\
|\providecommand{\version}{draft}|\\
|\||else|\\
|\providecommand{\version}{final}|\\
|\||fi|
\end{tabular}
\end{center}
%
The definition by |\providecommand| makes sure
that previous definitions are not overwritten.
Further statements |\providecommand{\version}{...}|
can thus be added before the above code to override it.

For the main file, one might add a line
(between |\childdocmain| and the above block)
%
\begin{center}
|%\ifchilddoc\||else\providecommand{\version}{draft}\||fi|
\end{center}
%
which can be uncommented to produce a draft version.
Likewise one can add a line to the very top of a child file
(above the |\childdocof{|\textit{main}|}| directive)
%
\begin{center}
|%\providecommand{\version}{final}|
\end{center}
%
which can be uncommented to produce the final version of this child document.

%%%%%%%%%%%%%%%%%%%%%%%%%%%%%%%%%%%%%%%%%%%%%%%%%%%%%%%%%%%%%%%%%%%%%%%%%%%%%%%%
\subsection{Forwarding}
\label{sec:forward}

Different versions of the main or child documents
using compilation flags as described in \secref{sec:flags}
can be (permanently) stored in different files
for convenient compilation, viewing and distribution.
To this end, the package defines a command
to pass on compilation to a different file:

%%%%%%%%%%%%%%%%%%%%%%%%%%%%%%%%%%%%%%%%
\DescribeMacro{\childdocforward}
The command |\childdocforward| redirects processing to
another source file:
%
\begin{center}
\begin{tabular}{l}
|\input{childdoc.def}|\\
|\childdocforward[|\textit{main}|]{|\textit{dest}|}|\\
\end{tabular}
\end{center}
%
The argument \textit{dest} is the destination file
(without extension).
It should be the main file or one of the child files.
Note that further \textsf{childdoc} directives
such as |\childdocof| and |\childdocforward|
in the indicated file will be processed in this form.
The optional argument \textit{main}
passes on directly to the main file \textit{main}
while pretending to compile the child \textit{dest}.
This form behaves as if \textit{dest}
issues |\childdocof{|\textit{main}|}| right away,
and no further \textsf{childdoc} directives will be processed.

%%%%%%%%%%%%%%%%%%%%%%%%%%%%%%%%%%%%%%%%
\DescribeMacro{\...prefix}
In the alternative form |\childdocforwardprefix|,
%
\begin{center}
\begin{tabular}{l}
|\input{childdoc.def}|\\
|\childdocforwardprefix[|\textit{main}|]{|\textit{prefix}|}{|\textit{dest}|}|
\end{tabular}
\end{center}
%
the destination file is determined by a pattern
depending on the current file:
To make this work, the current file must be called
`{\textit{prefix}\hspace{0.2em}\textit{suffix}}'
with \textit{prefix} matching precisely the argument.
Processing is then passed on to the file
`{\textit{dest}\hspace{0.2em}\textit{suffix}}'.
Surely, the same effect is achieved by
directly specifying the
argument `{\textit{dest}\hspace{0.2em}\textit{suffix}}'
in the first form.
However, that requires to set up a different file
for each child. With the alternative form of the command
all these files can have exactly the same content
which simplifies setting them up and maintaining them.

For example, the following file |draft.tex|
with a compilation flag |\version| as described in \secref{sec:flags}
compiles the main document as a draft:
%
\begin{center}
\begin{tabular}{l}
|\def\version{draft}|\\
|\input{childdoc.def}|\\
|\childdocforward{|\textit{main}|}|
\end{tabular}
\end{center}
%
Likewise, the following files |final|\textit{nn}|.tex|
compile the final version of the child document
|child|\textit{nn}|.tex|:
%
\begin{center}
\begin{tabular}{l}
|\def\version{final}|\\
|\input{childdoc.def}|\\
|\childdocforwardprefix{final}{child}|
\end{tabular}
\end{center}
%

Note that when several versions of a main file and/or of each child file
are to be generated, it may be convenient to set up a |Makefile| or
shell script to automatise the process.

%%%%%%%%%%%%%%%%%%%%%%%%%%%%%%%%%%%%%%%%%%%%%%%%%%%%%%%%%%%%%%%%%%%%%%%%%%%%%%%%
\subsection{Command Line Processing}
\label{sec:commandline}

The effect of redirection files can also be achieved by invoking
the \LaTeX{} compiler with a more elaborate command line.
Most conveniently this should be done as part
of a shell script or a |Makefile|.

When using \textsf{childdoc} in the main file, the following
command lines effectively perform a redirection
(note that depending on the shell being used,
backslashes may have to be doubled: `|\|' $\to$ `|\\|'):
%
\begin{center}
|... -jobname "|\textit{target}|" |\\|"|[\textit{flags}]%
|\input{childdoc.def}\childdocforward[|\textit{main}|]{|\textit{dest}|}"|
\end{center}
%
Here \textit{target} is the name of the output file,
\textit{main} is the name of the main file
and \textit{dest} is the name of the main or child file to be processed
(all filenames without extensions).
The optional argument \textit{main} can be omitted
if \textit{main} matches \textit{dest}.
Optionally, compilation \textit{flags} can be defined via |\def| commands.
This command line makes the \TeX{} engine believe
it is compiling the file \textit{target}
whose content is specified as the latter parameter.
The provided code then forwards the processing to
\textit{main} or \textit{dest} as described in \secref{sec:forward}.

%%%%%%%%%%%%%%%%%%%%%%%%%%%%%%%%%%%%%%%%%%%%%%%%%%%%%%%%%%%%%%%%%%%%%%%%%%%%%%%%
\subsection{Include by Input}
\label{sec:input}

Including child documents by |\include| has some restrictions by design.
Most notably, the content of a child document always occupies
its own set of pages; pages cannot be shared between child documents.
Usually, this behaviour makes perfect sense
because each child document contain an essential part of the document.
However, in some situations it may be desirable to compose
a document from a collection of parts
without having mandatory page breaks between then.
For this case, the package
provides a mechanism to include parts
by |\input| which can also be processed individually.
However, by construction this mechanism
requires manual handling of the content to be output.

%%%%%%%%%%%%%%%%%%%%%%%%%%%%%%%%%%%%%%%%
\DescribeMacro{\ifchilddocmanual}
The main file should be prepared as usual, see \secref{sec:include}.
However, the document body must make a distinction
between processing of an individual part and of the main document, e.g.:
%
\begin{center}
\begin{tabular}{l}
|\ifchilddocmanual|\\
|\input{\childdocname}|\\
|\||else|\\
\textit{document body with }|\input{|\textit{part}|}|\\
|\||fi|
\end{tabular}
\end{center}
%
The conditional |\ifchilddocmanual| is true whenever
a part to be included by |\input| is being compiled,
and the name of the part is stored in |\childdocname|.

%%%%%%%%%%%%%%%%%%%%%%%%%%%%%%%%%%%%%%%%
\DescribeMacro{\childdocby}
Each part to be included by |\input| should start with:
%
\begin{center}
\begin{tabular}{l}
|\input{childdoc.def}|\\
|\childdocby{|\textit{main}|}|\\
\end{tabular}
\end{center}
%
The directive |\childdocby| is similar to |\childdocof|
described in \secref{sec:include},
but the subsequent selection of content must be done manually.
To that end, both |\ifchilddoc| and |\ifchilddocmanual|
will be true upon processing of a part,
and the name of the part is stored in |\childdocname|.
Note that |\jobname| will be set to the filename of the current part
so that each part receives an individual |.aux| file
that does not interfere with the |.aux| file(s) of the main document.
This behaviour can be altered by the alternative form
|\childdocby[*]{|\textit{main}|}| (with a non-empty optional argument)
which uses the |.aux| file of the main document
by setting |\jobname| to \textit{main}.

%%%%%%%%%%%%%%%%%%%%%%%%%%%%%%%%%%%%%%%%%%%%%%%%%%%%%%%%%%%%%%%%%%%%%%%%%%%%%%%%
\subsection{Driver Development}
\label{sec:driver}

The \textsf{childdoc} mechanism can also be use for the development
of definition files such as \LaTeX{} styles or classes.
This case differs from the above setup with multiple parts
included by |\include| in that no |\includeonly| should be invoked.
This can be achieved by starting the include file
(before |\ProvidesPackage|) with:
%
\begin{center}
\begin{tabular}{l}
|\input{childdoc.def}|\\
|\childdocforward{|\textit{main}|}|\\
\end{tabular}
\end{center}
%
or alternatively with:
%
\begin{center}
\begin{tabular}{l}
|\input{childdoc.def}|\\
|\childdocby{|\textit{main}|}|\\
\end{tabular}
\end{center}
%
Both forms have slightly different effects as described above.
The main file is prepared as usual, see \secref{sec:include}.

%%%%%%%%%%%%%%%%%%%%%%%%%%%%%%%%%%%%%%%%%%%%%%%%%%%%%%%%%%%%%%%%%%%%%%%%%%%%%%%%
\subsection{Legacy Detection}
\label{sec:detection}

The directive |\childdocmain| in the main file can detect
whether the complete document or merely a child is to be compiled
even without using the directive |\childdocof|.
This method is deprecated because it is less robust
and there is no compelling reason to use it;
it is merely provided for backward compatibility
and it may be removed in future versions.

If the detection mechanism is to be used,
it is mandatory to correctly specify
the filename of the main file as the argument of |\childdocmain|:
%
\begin{center}
\begin{tabular}{l}
|\input{childdoc.def}|\\
|\childdocmain{|\textit{main}|}|\\
\end{tabular}
\end{center}
%
If |\jobname| does not match the argument \textit{main} of |\childdocmain|,
it is assumed that |\jobname| points to the child file to be compiled.
When using |\childdocmain| with the main file specified as argument,
it suffices to start a child file
with just |\input{|\textit{main}|}|
without loading of the package and using |\childdocof|.
If instead all processing is done
with the appropriate \textsf{childdoc} directives,
the argument of \textit{main} of |\childdocmain| can be empty.

An alternative version of the command line processing described
in \secref{sec:commandline} using the detection mechanism reads:
%
\begin{center}
|... -jobname "|\textit{target}|" "|[\textit{flags}]%
[|\def\jobname{|\textit{dest}|}|]|\input{|\textit{main}|}"|
\end{center}

%%%%%%%%%%%%%%%%%%%%%%%%%%%%%%%%%%%%%%%%%%%%%%%%%%%%%%%%%%%%%%%%%%%%%%%%%%%%%%%%
\subsection{Manual Code}
\label{sec:manual}

In case one cannot be certain whether the definitions file |childdoc.def|
is installed on the target \TeX{} distribution
and one prefers not to ship it,
it is conceivable to paste a few relevant commands into the sources.

To that end, drop all statements |\input{childdoc.def}|
and perform the replacements as outlined below.
Instead of |\childdocmain{|\textit{main}|}| add the following code
to the top of the main file:
%
\begin{center}
\begin{tabular}{l}
|\||ifdefined\childdocname\endinput\||fi\newif\ifchilddoc|\\
|\edef\childdocname{\scantokens\expandafter{\jobname\noexpand}}|\\
|\def\childdocmain{|\textit{main}|}\||ifx\childdocmain\childdocname\||else|\\
|\childdoctrue\includeonly{\childdocname}\let\jobname\childdocmain\||fi|\\
\end{tabular}
\end{center}
%
Instead of |\childdocof{|\textit{main}|}| just include the main file
at the top of each child file:
%
\begin{center}
|\input{|\textit{main}|}|
\end{center}
%
A simple redirection |\childdocforward{|\textit{dest}|}| is achieved by:
%
\begin{center}
|\def\jobname{|\textit{dest}|}\input{\jobname}|
\end{center}
%
The redirection with prefix
|\childdocforwardprefix[|\textit{prefix}|]{|\textit{dest}|}|
is accomplished by:
%
\begin{center}
\begin{tabular}{l}
|{\edef\jobname{\scantokens\expandafter{\jobname\noexpand}}|\\
|\def\redirectjob |\textit{prefix}|#1~~~{\gdef\jobname{|\textit{dest}|#1}}|\\
|\expandafter\redirectjob\jobname~~~}\input{\jobname}|
\end{tabular}
\end{center}

In an alternative approach,
child documents can be compiled by a specific command line
without additional code or specific definitions:
%
\begin{center}
|... -jobname "|\textit{target}|" "|[\textit{flags}]%
|\includeonly{|\textit{dest}|}\input{|\textit{main}|}"|
\end{center}
%

%%%%%%%%%%%%%%%%%%%%%%%%%%%%%%%%%%%%%%%%%%%%%%%%%%%%%%%%%%%%%%%%%%%%%%%%%%%%%%%%
%%%%%%%%%%%%%%%%%%%%%%%%%%%%%%%%%%%%%%%%%%%%%%%%%%%%%%%%%%%%%%%%%%%%%%%%%%%%%%%%
\section{Information}

%%%%%%%%%%%%%%%%%%%%%%%%%%%%%%%%%%%%%%%%%%%%%%%%%%%%%%%%%%%%%%%%%%%%%%%%%%%%%%%%
\subsection{Copyright}

Copyright \copyright{} 2017--2018 Niklas Beisert

This work may be distributed and/or modified under the
conditions of the \LaTeX{} Project Public License, either version 1.3
of this license or (at your option) any later version.
The latest version of this license is in
  \url{http://www.latex-project.org/lppl.txt}
and version 1.3 or later is part of all distributions of \LaTeX{}
version 2005/12/01 or later.

This work has the LPPL maintenance status `maintained'.

The Current Maintainer of this work is Niklas Beisert.

This work consists of the files |README.txt|, |childdoc.ins| and |childdoc.dtx|
as well as the derived files |childdoc.def|, |cdocsamp.tex|
with |cdocsch1.tex|, |cdocsch2.tex|, |cdocspt3.tex|, |cdocspt4.tex|,
|cdocsdrf.tex|, |cdocsfn1.tex|, |cdocsfn2.tex|
as well as |childdoc.pdf|.

%%%%%%%%%%%%%%%%%%%%%%%%%%%%%%%%%%%%%%%%%%%%%%%%%%%%%%%%%%%%%%%%%%%%%%%%%%%%%%%%
\subsection{Files and Installation}

The package consists of the files:
%
\begin{center}
\begin{tabular}{ll}
    |README.txt|   & readme file \\
    |childdoc.ins| & installation file \\
    |childdoc.dtx| & source file \\
    |childdoc.def| & definition file \\
    |cdocsamp.tex| & sample main file \\
    |cdocsch1.tex| & sample include file \\
    |cdocsch2.tex| & sample include file \\
    |cdocspt3.tex| & sample part file \\
    |cdocspt4.tex| & sample part file \\
    |cdocsdrf.tex| & sample redirection file \\
    |cdocsfn1.tex| & sample redirection file \\
    |cdocsfn2.tex| & sample redirection file \\
    |childdoc.pdf| & manual
\end{tabular}
\end{center}
%
The distribution consists of the files
|README.txt|, |childdoc.ins| and |childdoc.dtx|.
%
\begin{itemize}
\item
Run (pdf)\LaTeX{} on |childdoc.dtx|
to compile the manual |childdoc.pdf| (this file).
\item
Run \LaTeX{} on |childdoc.ins| to create the definitions file |childdoc.def|
and the sample |cdocsamp.tex| with include files
|cdocsch1.tex|, |cdocsch2.tex|, |cdocspt3.tex|, |cdocspt4.tex|,
|cdocsdrf.tex|, |cdocsfn1.tex|, |cdocsfn2.tex|.
Then copy the file |childdoc.def| to an appropriate directory of your \LaTeX{}
distribution, e.g.\ \textit{texmf-root}|/tex/latex/childdoc|.
\end{itemize}

%%%%%%%%%%%%%%%%%%%%%%%%%%%%%%%%%%%%%%%%%%%%%%%%%%%%%%%%%%%%%%%%%%%%%%%%%%%%%%%%
\subsection{Related CTAN Packages}

There are several other packages which offer a similar functionality:
%
\begin{itemize}
\item
The packages
\href{http://ctan.org/pkg/docmute}{\textsf{docmute}},
\href{http://ctan.org/pkg/includex}{\textsf{includex}} and
\href{http://ctan.org/pkg/standalone}{\textsf{standalone}}
provide commands to include only the document body of
a child file thus allowing both files to be compiled individually.
\item
The packages \href{http://ctan.org/pkg/subdocs}{\textsf{subdocs}}
and \href{http://ctan.org/pkg/subfiles}{\textsf{subfiles}}
provide structures in which the main and child documents can be
encapsulated and allowing them to be compiled individually.
The inclusion mechanism is different from the conventional |\include|.
\item
The package \href{http://ctan.org/pkg/combine}{\textsf{combine}}
is an elaborate solution to combine several documents into one.
\end{itemize}
%
See also the CTAN topic \href{http://ctan.org/topic/subdocs}{\textsf{subdocs}}
for further related packages.
The present package differs from the above solutions in that
a document structure constructed with the conventional |\include| mechanism
just needs two extra commands at the top of every file
such that all constituent files can be compiled individually.

%%%%%%%%%%%%%%%%%%%%%%%%%%%%%%%%%%%%%%%%%%%%%%%%%%%%%%%%%%%%%%%%%%%%%%%%%%%%%%%%
%\subsection{Feature Suggestions}
%
%The following is a list of features which may be useful for future
%versions of this package:
%%
%\begin{itemize}
%\item
%\ldots
%\end{itemize}

%%%%%%%%%%%%%%%%%%%%%%%%%%%%%%%%%%%%%%%%%%%%%%%%%%%%%%%%%%%%%%%%%%%%%%%%%%%%%%%%
\subsection{Revision History}

%%%%%%%%%%%%%%%%%%%%%%%%%%%%%%%%%%%%%%%%
\paragraph{v2.0:} 2018/12/30

\begin{itemize}
\item
immediate forward processing
\item
added |\childdocby| mechanism
\item
manual restructured
\end{itemize}

%%%%%%%%%%%%%%%%%%%%%%%%%%%%%%%%%%%%%%%%
\paragraph{v1.6:} 2018/01/17

\begin{itemize}
\item
application for development of include files
\item
corrections to manual
\end{itemize}

%%%%%%%%%%%%%%%%%%%%%%%%%%%%%%%%%%%%%%%%
\paragraph{v1.5:} 2017/05/21

\begin{itemize}
\item
more complete structuring introduced
\item
|\childdocof| introduced
\item
|\childdoc| renamed to |\childdocmain|
\item
|\childredirect| renamed to |\childdocforward| and |\childdocforwardprefix|
and functionality expanded
\end{itemize}

%%%%%%%%%%%%%%%%%%%%%%%%%%%%%%%%%%%%%%%%
\paragraph{v1.0:} 2017/04/27

\begin{itemize}
\item
manual and install package
\item
first version published on CTAN
\end{itemize}

%%%%%%%%%%%%%%%%%%%%%%%%%%%%%%%%%%%%%%%%
\paragraph{v0.6:} 2017/04/26

\begin{itemize}
\item
redirection mechanism added
\end{itemize}

%%%%%%%%%%%%%%%%%%%%%%%%%%%%%%%%%%%%%%%%
\paragraph{v0.5:} 2017/04/26

\begin{itemize}
\item
functionality in definition file
\end{itemize}


%%%%%%%%%%%%%%%%%%%%%%%%%%%%%%%%%%%%%%%%%%%%%%%%%%%%%%%%%%%%%%%%%%%%%%%%%%%%%%%%
%%%%%%%%%%%%%%%%%%%%%%%%%%%%%%%%%%%%%%%%%%%%%%%%%%%%%%%%%%%%%%%%%%%%%%%%%%%%%%%%
%%%%%%%%%%%%%%%%%%%%%%%%%%%%%%%%%%%%%%%%%%%%%%%%%%%%%%%%%%%%%%%%%%%%%%%%%%%%%%%%
\appendix

\settowidth\MacroIndent{\rmfamily\scriptsize 000\ }

 \DocInput{childdoc.dtx}

\end{document}
%</driver>
% \fi
%
% %%%%%%%%%%%%%%%%%%%%%%%%%%%%%%%%%%%%%%%%%%%%%%%%%%%%%%%%%%%%%%%%%%%%%%%%%%%%%%
% %%%%%%%%%%%%%%%%%%%%%%%%%%%%%%%%%%%%%%%%%%%%%%%%%%%%%%%%%%%%%%%%%%%%%%%%%%%%%%
% \section{Sample}
%\iffalse
%<*samplemain>
%\fi
%
% The following presents a sample document
% with two chapters, two parts, a title page,
% a compile flag as well as three forwarding files to set the flag.
% It consists of eight |.tex| files:
% \begin{center}
% \begin{tabular}{ll}
% |cdocsamp.tex|&main file\\
% |cdocsch1.tex|&include file for chapter 1\\
% |cdocsch2.tex|&include file for chapter 2\\
% |cdocspt3.tex|&include file for part 3\\
% |cdocspt4.tex|&include file for part 4\\
% |cdocsdrf.tex|&forwarding file for main file in draft mode\\
% |cdocsfi1.tex|&forwarding file for final version of chapter 1\\
% |cdocsfi2.tex|&forwarding file for final version of chapter 2\\
% \end{tabular}
% \end{center}
% Each of the eight files can be compiled directly by the \LaTeX{} compiler.
%
% %%%%%%%%%%%%%%%%%%%%%%%%%%%%%%%%%%%%%%
% \paragraph{Main File.}
%
% The main file is called |cdocsamp.tex|.
%
% Load the \textsf{childdoc} definitions and
% declare the filename for the main document:
%    \begin{macrocode}
\input{childdoc.def}
\childdocmain{}
%    \end{macrocode}

% Optional override for |\version| flag:
%    \begin{macrocode}
%%\ifchilddoc\else\providecommand{\version}{draft}\fi
%    \end{macrocode}

% Define the default values for the |\version| flag
% (|final| for the main file and |draft| for childs):
%    \begin{macrocode}
\ifchilddoc
\providecommand{\version}{draft}
\else
\providecommand{\version}{final}
\fi
%    \end{macrocode}

% Load the standard document class:
%    \begin{macrocode}
\documentclass[12pt]{article}
%    \end{macrocode}

% Start the document body:
%    \begin{macrocode}
\begin{document}
%    \end{macrocode}

% Declare a title page.
% Print title, part of document being processed and version flag:
%    \begin{macrocode}
\addtocounter{page}{-1}
\begin{center}
{\LARGE\bfseries{}childdoc example\par}
\vspace{1cm}
\ifchilddoc
\ifchilddocmanual part\else chapter\fi:
`\childdocname' of `\childdocjob'\par
\else
main document: `\childdocjob'\par
\fi
version: \version\par
\end{center}
\newpage
%    \end{macrocode}

% Manually include selected file,
% otherwise process as usual:
%    \begin{macrocode}
\ifchilddocmanual
\section*{part `\childdocname'}
\input{\childdocname}
\else
%    \end{macrocode}

% Include the two chapters:
%    \begin{macrocode}
\include{cdocsch1}
\include{cdocsch2}
%    \end{macrocode}

% Include the two parts unless only chapters should be displayed:
%    \begin{macrocode}
\ifchilddoc\else
\section{part three}
\input{cdocspt3}
\section{part four}
\input{cdocspt4}
\fi
%    \end{macrocode}

% Process as usual until here:
%    \begin{macrocode}
\fi
%    \end{macrocode}

% End of document body:
%    \begin{macrocode}
\end{document}
%    \end{macrocode}
%\iffalse
%</samplemain>
%\fi
%
% %%%%%%%%%%%%%%%%%%%%%%%%%%%%%%%%%%%%%%
% \paragraph{Chapter Include Files.}
%
% The include files are called |cdocsch1.tex| and |cdocsch2.tex|.
%
%\iffalse
%<*samplechap1|samplechap2>
%\fi

% Optional override for |\version| flag:
%    \begin{macrocode}
%%\providecommand{\version}{final}
%    \end{macrocode}

% Include the main document:
%    \begin{macrocode}
\input{childdoc.def}
\childdocof{cdocsamp}
%    \end{macrocode}

%\iffalse
%</samplechap1|samplechap2>
%\fi
%
%\iffalse
%<*samplechap1>
%\fi
% Some text for chapter 1:
%    \begin{macrocode}
\section{one}
some text in chapter one
%    \end{macrocode}

%\iffalse
%</samplechap1>
%\fi
% Some text for chapter 2:
%\iffalse
%<*samplechap2>
%\fi
%    \begin{macrocode}
\section{two}
more text in chapter two
%    \end{macrocode}

%\iffalse
%</samplechap2>
%\fi
%
% %%%%%%%%%%%%%%%%%%%%%%%%%%%%%%%%%%%%%%
% \paragraph{Part Include Files.}
%
% The include files are called |cdocspt3.tex| and |cdocspt4.tex|.
%
%\iffalse
%<*samplepart3|samplepart4>
%\fi

% Optional override for |\version| flag:
%    \begin{macrocode}
%%\providecommand{\version}{final}
%    \end{macrocode}

% Include the main document:
%    \begin{macrocode}
\input{childdoc.def}
\childdocby{cdocsamp}
%    \end{macrocode}

%\iffalse
%</samplepart3|samplepart4>
%\fi
%
%\iffalse
%<*samplepart3>
%\fi
% Some text for part 3:
%    \begin{macrocode}
some text in part three
%    \end{macrocode}

%\iffalse
%</samplepart3>
%\fi
% Some text for part 4:
%\iffalse
%<*samplepart4>
%\fi
%    \begin{macrocode}
more text in part four
%    \end{macrocode}

%\iffalse
%</samplepart4>
%\fi
%
% %%%%%%%%%%%%%%%%%%%%%%%%%%%%%%%%%%%%%%
% \paragraph{Forwarding for a Complete Draft.}
%
% The following forwarding file |cdocsdrf.tex|
% compiles the main document in draft mode:
%\iffalse
%<*sampledraft>
%\fi
%    \begin{macrocode}
\def\version{draft}
\input{childdoc.def}
\childdocforward{cdocsamp}
%    \end{macrocode}

%\iffalse
%</sampledraft>
%\fi
%
% %%%%%%%%%%%%%%%%%%%%%%%%%%%%%%%%%%%%%%
% \paragraph{Forwarding for Final Version of the Chapters.}
%
% The following forwarding files |cdocsfn1.tex| and |cdocsfn2.tex|
% (with identical content)
% compile the final versions of the child documents
% |cdocsch1.tex| and |cdocsch2.tex|, respectively:
%\iffalse
%<*samplefinal>
%\fi
%    \begin{macrocode}
\def\version{final}
\input{childdoc.def}
\childdocforwardprefix[cdocsamp]{cdocsfn}{cdocsch}
%    \end{macrocode}

%\iffalse
%</samplefinal>
%\fi
%
% %%%%%%%%%%%%%%%%%%%%%%%%%%%%%%%%%%%%%%
% \paragraph{Command Line Processing.}
%
% The following three command lines generate the output files
% |cdocscld|, |cdocscl1| and |cdocscl2|
% which should be identical to
% |cdocsdrf|, |cdocsch1| and |cdocsfn2|, respectively:
% \begin{center}
% \begin{tabular}{l}
% |latex -jobname cdocscld \|\\
% |  "\def\version{draft}\input{childdoc.def}\childdocforward{cdocsamp}"|\\
% |latex -jobname cdocscl1 \|\\
% |  "\input{childdoc.def}\childdocforward[cdocsamp]{cdocsch1}"|\\
% |latex -jobname cdocscl2 \|\\
% |  "\def\version{final}\input{childdoc.def}\childdocforward{cdocsch2}"|
% \end{tabular}
% \end{center}
% Note that the trailing backslash on each first line
% merely continues the input to the second line
% (for convenient cut ant paste).
% Furthermore, the command |latex| can be replaced by any
% of its alternative versions such as |pdflatex|.
%
% %%%%%%%%%%%%%%%%%%%%%%%%%%%%%%%%%%%%%%%%%%%%%%%%%%%%%%%%%%%%%%%%%%%%%%%%%%%%%%
% %%%%%%%%%%%%%%%%%%%%%%%%%%%%%%%%%%%%%%%%%%%%%%%%%%%%%%%%%%%%%%%%%%%%%%%%%%%%%%
% \section{Implementation}
%\iffalse
%<*package>
%\fi
%
% This section describes the definitions file |childdoc.def|.

% The definitions cannot be loaded using |\usepackage| or |\RequirePackage|
% which has a mechanism to prevent loading a style file more than once.
% When loading the definitions by means of |\input|
% multiple instances have to be prevented manually:
%\iffalse
%This code needs to be before the `\ProvidesFile' directive
%which is defined at the beginning of this file.
%Therefore it is also placed there and commented out here.
%</package>
%<*discard>
%\fi
%    \begin{macrocode}
\ifdefined\childdocmain\endinput\fi
%    \end{macrocode}
%\iffalse
%</discard>
%<*package>
%\fi
%
% \macro{\ifchilddoc}
% \macro{\ifchilddocmanual}
% The conditional |\ifchilddoc| tells whether a
% child (true) or main (false) document is being compiled.
% The conditional |\ifchilddocmanual| tells whether
% the |\includeonly| mechanism is used (false) or
% the selection of child files must be performed manually (true).
% The definitions initialise to false:
%    \begin{macrocode}
\newif\ifchilddoc
\newif\ifchilddocmanual
%    \end{macrocode}

% \macro{\childdocname}
% \macro{\childdocjob}
% The macro |\childdocname| stores the name of the main document
% to be compiled. The macro |\childdocjob| stores the name of
% the document on which the \LaTeX{} compiler was originally invoked.
% The content of |\jobname| cannot be compared
% to filenames specified in the source due to different catcodes.
% The following code rescans |\jobname|, stores the result
% in |\childdocname| and saves a copy in |\childdocjob|:
%    \begin{macrocode}
\edef\childdocname{\scantokens\expandafter{\jobname\noexpand}}
\let\childdocjob\childdocname
%    \end{macrocode}

% \macro{\childdocdisable}
% The macro |\childdocdisable| prevents the main file
% from being processed more than once.
% At this stage, the main document command |\childdocmain|
% is assumed to be called once again where it should do nothing.
% Any subsequent call to it should prevent
% a secondary processing of the main document
% It overwrites the forwarding commands
% |\childdocof| and |\childdocforward|
% with empty macros to prevent further inclusions of the main document:
%    \begin{macrocode}
\newcommand{\childdocdisable}
{
  \renewcommand{\childdocmain}[1]{\renewcommand{\childdocmain}[1]{\endinput}}
  \renewcommand{\childdocof}[1]{}
  \renewcommand{\childdocby}[2][]{}
  \renewcommand{\childdocforward}[2][]{}
  \renewcommand{\childdocdisable}{}
}
%    \end{macrocode}

% \macro{\childdocmain}
% The macro |\childdocmain| is to be called at the top of the main file
% with nothing or the main filename (without extension) as argument.
% First, it breaks loops.
% If the argument is not empty and does not match |\childdocname|
% (which is set by the first inclusion of |childdoc.def|),
% |\ifchilddoc| is set to true, |\includeonly| is applied to the child file
% and |\jobname| is set to the main file
% (for proper handling of |.aux| files):
%    \begin{macrocode}
\newcommand{\childdocmain}[1]
{
  \childdocdisable\childdocmain{}
  \if?#1?\else
    \begingroup
      \def\childdoctmp{#1}
      \ifx\childdoctmp\childdocname
        \def\childdoctmp{}
      \else
        \def\childdoctmp
        {
          \childdoctrue
          \includeonly{\childdocname}
          \def\childdocjob{#1}
          \def\jobname{#1}
        }
      \fi
      \expandafter
    \endgroup
    \childdoctmp
  \fi
}
%    \end{macrocode}

% \macro{\childdocof}
% The command |\childdocof| redirects
% compilation to the main file |#1|.
%    \begin{macrocode}
\newcommand{\childdocof}[1]
{
  \childdocdisable
  \childdoctrue
  \includeonly{\childdocname}
  \def\jobname{#1}
  \def\childdocjob{#1}
  \input{#1}
}
%    \end{macrocode}

% \macro{\childdocby}
% The command |\childdocby| ....
%    \begin{macrocode}
\newcommand{\childdocby}[2][]
{
  \childdocdisable
  \childdoctrue
  \childdocmanualtrue
  \if?#1?\else
    \def\jobname{#2}
  \fi
  \def\childdocjob{#2}
  \input{#2}
  \endinput
}
%    \end{macrocode}

% \macro{\childdocforward}
% The command |\childdocforward| redirects
% compilation to the main file or
% (if the optional argument is given) a child file.
% Parameters are set as if the main file
% or a child file starting with |\childdocof| was compiled.
% Then compilation is handed over to the main file:
%    \begin{macrocode}
\newcommand{\childdocforward}[2][]
{
  \begingroup
    \if?#1?
      \def\childdoctmp
      {
        \def\childdocname{#2}
        \def\childdocjob{#2}
        \def\jobname{#2}
        \input{#2}
        \endinput
      }
    \else
      \def\childdoctmp
      {
        \childdocdisable
        \def\childdocname{#2}
        \childdoctrue
        \includeonly{#2}
        \def\childdocjob{#1}
        \def\jobname{#1}
        \input{#1}
        \endinput
      }
    \fi
    \expandafter
  \endgroup
  \childdoctmp
}
%    \end{macrocode}

% \macro{\childdocforwardprefix}
% The command |\childdocforwardprefix| redirects
% compilation to the main or a child file by means of a pattern.
% The prefix |#1| in the current filename is replaced by |#2|
% and the suffix of the current filename is kept
% (it is assumed that the filename does not contain the substring `|~~~|'
% which is used as a delimiter).
% Compilation is handed over to the new file by |\childdocforward|:
%    \begin{macrocode}
\newcommand{\childdocforwardprefix}[3][]
{
  \begingroup
    \def\childdocextract #2##1~~~{\def\childdoctmp{\childdocforward[#1]{#3##1}}}
    \expandafter\childdocextract\childdocname~~~
    \expandafter
  \endgroup
  \childdoctmp
}
%    \end{macrocode}

% \macro{\childdoc}
% The deprecated macro |\childdoc| is a legacy version of |\childdocmain|:
%    \begin{macrocode}
\newcommand{\childdoc}{\childdocmain}
%    \end{macrocode}

% \macro{\childdocredirect}
% The deprecated macro |\childdocredirect| is a legacy version
% of |\childdocforward| and |\childdocforwardprefix|:
%    \begin{macrocode}
\newcommand{\childdocredirect}[2][]
{
  \begingroup
    \if?#1?
      \def\childdoctmp{\childdocforward{#2}}
    \else
      \def\childdoctmp{\childdocforwardprefix{#1}{#2}}
    \fi
    \expandafter
  \endgroup
  \childdoctmp
}
%    \end{macrocode}

%\iffalse
%</package>
%\fi
%
\endinput
|\\
|\childdocforward[|\textit{main}|]{|\textit{dest}|}|\\
\end{tabular}
\end{center}
%
The argument \textit{dest} is the destination file
(without extension).
It should be the main file or one of the child files.
Note that further \textsf{childdoc} directives
such as |\childdocof| and |\childdocforward|
in the indicated file will be processed in this form.
The optional argument \textit{main}
passes on directly to the main file \textit{main}
while pretending to compile the child \textit{dest}.
This form behaves as if \textit{dest}
issues |\childdocof{|\textit{main}|}| right away,
and no further \textsf{childdoc} directives will be processed.

%%%%%%%%%%%%%%%%%%%%%%%%%%%%%%%%%%%%%%%%
\DescribeMacro{\...prefix}
In the alternative form |\childdocforwardprefix|,
%
\begin{center}
\begin{tabular}{l}
|% \iffalse
%
% childdoc.dtx Copyright (C) 2017-2018 Niklas Beisert
%
% This work may be distributed and/or modified under the
% conditions of the LaTeX Project Public License, either version 1.3
% of this license or (at your option) any later version.
% The latest version of this license is in
%   http://www.latex-project.org/lppl.txt
% and version 1.3 or later is part of all distributions of LaTeX
% version 2005/12/01 or later.
%
% This work has the LPPL maintenance status `maintained'.
%
% The Current Maintainer of this work is Niklas Beisert.
%
% This work consists of the files childdoc.dtx and childdoc.ins
% and the derived files childdoc.def and cdocsamp.tex with
% cdocsch1.tex, cdocsch2.tex, cdocsdrf.tex, cdocsfn1.tex, cdocsfn2.tex.
%
%<package>\ifdefined\childdocmain\endinput\fi
%<package>\ProvidesFile{childdoc.def}[2018/12/30 v2.0 child document driver]
%<samplemain>\ProvidesFile{cdocsamp.tex}[2018/12/30 v2.0 sample for childdoc]
%<*driver>
%\ProvidesFile{childdoc.drv}[2018/12/30 v2.0 childdoc reference manual file]
\PassOptionsToClass{10pt,a4paper}{article}
\documentclass{ltxdoc}

\usepackage[margin=35mm]{geometry}
\usepackage{hyperref}
\usepackage{hyperxmp}
\usepackage[usenames]{color}

\hypersetup{colorlinks=true}
\hypersetup{pdfstartview=FitH}
\hypersetup{pdfpagemode=UseNone}
\hypersetup{pdfsource={}}
\hypersetup{pdflang={en-UK}}
\hypersetup{pdfcopyright={Copyright 2017-2018 Niklas Beisert.
  This work may be distributed and/or modified under the
  conditions of the LaTeX Project Public License, either version 1.3
  of this license or (at your option) any later version.}}
\hypersetup{pdflicenseurl={http://www.latex-project.org/lppl.txt}}
\hypersetup{pdfcontactaddress={ETH Zurich, ITP, HIT K,
  Wolfgang-Pauli-Strasse 27}}
\hypersetup{pdfcontactpostcode={8093}}
\hypersetup{pdfcontactcity={Zurich}}
\hypersetup{pdfcontactcountry={Switzerland}}
\hypersetup{pdfcontactemail={nbeisert@itp.phys.ethz.ch}}
\hypersetup{pdfcontacturl={http://people.phys.ethz.ch/\xmptilde nbeisert/}}

\newcommand{\secref}[1]{\hyperref[#1]{section \ref*{#1}}}

\parskip1ex
\parindent0pt
\let\olditemize\itemize
\def\itemize{\olditemize\parskip0pt}

\begin{document}

\title{The \textsf{childdoc} Package}
\hypersetup{pdftitle={The childdoc Package}}
\author{Niklas Beisert\\[2ex]
  Institut f\"ur Theoretische Physik\\
  Eidgen\"ossische Technische Hochschule Z\"urich\\
  Wolfgang-Pauli-Strasse 27, 8093 Z\"urich, Switzerland\\[1ex]
  \href{mailto:nbeisert@itp.phys.ethz.ch}
  {\texttt{nbeisert@itp.phys.ethz.ch}}}
\hypersetup{pdfauthor={Niklas Beisert}}
\hypersetup{pdfsubject={Manual for the LaTeX2e Package childdoc}}
\date{30 December 2018, \textsf{v2.0}}
\maketitle

\begin{abstract}\noindent
\textsf{childdoc} is a \LaTeXe{} package
that enables the direct compilation
of document sections included by |\include|
to individual files.
\end{abstract}

\begingroup
\parskip0ex
\tableofcontents
\endgroup

%%%%%%%%%%%%%%%%%%%%%%%%%%%%%%%%%%%%%%%%%%%%%%%%%%%%%%%%%%%%%%%%%%%%%%%%%%%%%%%%
%%%%%%%%%%%%%%%%%%%%%%%%%%%%%%%%%%%%%%%%%%%%%%%%%%%%%%%%%%%%%%%%%%%%%%%%%%%%%%%%
\section{Introduction}

\LaTeX{} provides a mechanism to structure a large document (such as a book)
into a main file and several child files (containing the chapters)
using the |\include| command.
This mechanism is beneficial for documents
which span hundreds of pages in order to
make the source file(s) more manageable.
Moreover, compilation can be restricted to
selected child files by means of the |\includeonly| command.
The latter feature can be used to reduce the compilation time while editing
(this was significantly more useful in the earlier days of \LaTeX{})
or to generate a smaller document which is easier to navigate.
Another application of |\includeonly| is to generate
documents consisting of selected parts of the complete document.

However, there are a few drawbacks of the plain |\include| mechanism:
\begin{itemize}
\item
The child files cannot be compiled on their own,
they can only be compiled via the main file.
A naive editing environment
(such as a text editor with an option
to have the current file processed by \LaTeX)
may require one to switch to the main file before compiling;
attempting to compile the child file produces errors.
\item
The main file must be modified (each time)
to adjust the |\includeonly| command
to the present needs. This easily leaves the main file in a messy state.
\item
The generated document will always carry the filename
of the main document. This is inconvenient if
several child files are to be compiled and
to be kept for distribution.
\end{itemize}

The present package provides a simple interface
to make child files individually compilable by \LaTeX{}.
Compiling a child file then has the same effect as compiling
the main file with an |\includeonly| command
to select the appropriate child.
Moreover the generated document will carry the name of the child
rather than the main file.
This resolves all three above issues.

This feature is meant to make the editing of books,
thesis documents and lecture notes somewhat more convenient.
However, the package can also be used efficiently for
composing a series of documents (such as exercise sheets)
which are typically distributed individually.
It then assists the author in generating the individual documents
(potentially in different versions)
as well as a document containing the collected series.
Another application is in developing style files
or other kinds of included material
where compilation of the style file could redirect
to a sample or test file.

%%%%%%%%%%%%%%%%%%%%%%%%%%%%%%%%%%%%%%%%%%%%%%%%%%%%%%%%%%%%%%%%%%%%%%%%%%%%%%%%
%%%%%%%%%%%%%%%%%%%%%%%%%%%%%%%%%%%%%%%%%%%%%%%%%%%%%%%%%%%%%%%%%%%%%%%%%%%%%%%%
\section{Usage}

First of all, the package \textsf{childdoc} is \emph{not} a standard
\LaTeXe{} |.sty| style file! Therefore it needs to be invoked in
a non-standard way.

%%%%%%%%%%%%%%%%%%%%%%%%%%%%%%%%%%%%%%%%%%%%%%%%%%%%%%%%%%%%%%%%%%%%%%%%%%%%%%%%
\subsection{Included Files}
\label{sec:include}

%%%%%%%%%%%%%%%%%%%%%%%%%%%%%%%%%%%%%%%%
\DescribeMacro{\childdocmain}
To use the package, add the commands
\begin{center}
\begin{tabular}{l}
|\input{childdoc.def}|\\
|\childdocmain{}|\\
\end{tabular}
\end{center}
at the very top of the main \LaTeX{} file,
in particular \emph{before} the |\documentclass| statement!
The argument of |\childdocmain| should be left empty
(but it must be present).

%%%%%%%%%%%%%%%%%%%%%%%%%%%%%%%%%%%%%%%%
\DescribeMacro{\childdocof}
Furthermore, add the commands
\begin{center}
\begin{tabular}{l}
|\input{childdoc.def}|\\
|\childdocof{|\textit{main}|}|\\
\end{tabular}
\end{center}
at the top of every child file \textit{child}
which is included by |\include{|\textit{child}|}|
from within the main file
(or at least for those files to be compiled individually).
The argument \textit{main} must be the filename of the main file.

There are a couple of
considerations in setting up the main and child documents:

%%%%%%%%%%%%%%%%%%%%%%%%%%%%%%%%%%%%%%%%
\paragraph{Restrictions.}

Please note the following restrictions:
\begin{itemize}
\item
|\childdocmain| must be called with one argument \textit{main}
to ensure compatibility with earlier version of the package.
It must either be empty (|\childdocmain{}|)
or precisely match the filename of the main file in which it is specified.
See \secref{sec:detection} for further information.
\item
The filename \textit{main} must be specified without the |.tex| extension.
\item
The filename \textit{main} is case sensitive
(even in case-insensitive file systems)
due to internal string comparison.
\item
The argument \textit{main} should be fully expanded, it cannot be a macro.
\item
Subdirectories and special characters should be avoided in filenames.
\item
The command |\childdocmain{|\textit{main}|}| must be followed by a whitespace.
It should not be followed immediately by another command
or by a comment mark `|%|'.
This is because the \TeX{} parser reads the token immediately following
the argument of |\childdocmain| and puts it
at the beginning of every child section;
however, a white\-space is ignored.
\end{itemize}

%%%%%%%%%%%%%%%%%%%%%%%%%%%%%%%%%%%%%%%%
\paragraph{Content of Main File.}

It is advisable to place all content in the child files included by |\include|.
Any output contained in the main file will appear in all child documents
unless suppressed manually;
it cannot be suppressed automatically by the |\includeonly| directive
and thus should normally be avoided.
A method to include some content in the main file
by means of conditional processing is described in \secref{sec:conditional}.

%%%%%%%%%%%%%%%%%%%%%%%%%%%%%%%%%%%%%%%%
\paragraph{Page Numbering.}

When only a part of the document is compiled,
the appropriate numbering of pages
(as well as other status parameters)
is determined from the |.aux| files.
The latter contain information from previous passes.
However this information needs to propagate through
all intermediate child documents.
Therefore the page numbering in child documents may well
be inconsistent until the complete document is compiled at least once.

A useful (if unconventional) way to always ensure a consistent
page numbering is to restart the numbering in each child document
and denote the pages by `\textit{child}|.|\textit{page}'
where \textit{child} represents the chapter/section number of the child file.
This can be achieved by the command
|\numberwithin{page}{|\textit{child}|}|
of the \textsf{amsmath} package
where \textit{child} can be |chapter| or |section|
depending on the chosen structuring.
Alternatively, one can modify the macro |\thepage| appropriately
and reset the counter |page| at the start of each child file.

%%%%%%%%%%%%%%%%%%%%%%%%%%%%%%%%%%%%%%%%%%%%%%%%%%%%%%%%%%%%%%%%%%%%%%%%%%%%%%%%
\subsection{Conditional Processing}
\label{sec:conditional}

The package provides a mechanism to compile different versions
of a document. To customise the versions further some conditional processing
can come in handy to distinguish which version is being compiled.
The package provides two macros to describe the compilation context:

%%%%%%%%%%%%%%%%%%%%%%%%%%%%%%%%%%%%%%%%
\DescribeMacro{\ifchilddoc}
The conditional |\ifchilddoc| distinguishes between the compilation of
child documents and the main document:
%
\begin{center}
|\ifchilddoc |\textit{child-code}| |[|\||else |\textit{main-code}]| \||fi|
\end{center}

%%%%%%%%%%%%%%%%%%%%%%%%%%%%%%%%%%%%%%%%
\DescribeMacro{\childdocname}
\DescribeMacro{\childdocjob}
The macro |\childdocname| contains the filename (without extension)
of the main or child file being processed.
Note that |\childdocjob| will always contain the name of the main file.

%%%%%%%%%%%%%%%%%%%%%%%%%%%%%%%%%%%%%%%%
\paragraph{Title Page.}

Conditional processing can be used to include a title or banner page
in the main document when proper precautions are taken.
Importantly, the code in the main file should ensure that the page counter
(as well as other status parameters which are stored in the |.aux| files)
takes the same value after the conditional processing.
Otherwise the page numbers may take divergent values
depending on which part is compiled.

For example, a title page could be declared by:
%
\begin{center}
\begin{tabular}{l}
|\ifchilddoc\||else|\\
|\addtocounter{page}{-1}|\\
\textit{code for title page}\\
|\newpage|\\
|\||fi|
\end{tabular}
\end{center}
%
A banner page for the child documents can be generated by:
%
\begin{center}
\begin{tabular}{l}
|\ifchilddoc|\\
|\addtocounter{page}{-1}|\\
\textit{code for banner page}\\
|\newpage|\\
|\||fi|
\end{tabular}
\end{center}
%
Here one could write a message such as:
\begin{center}
|This is the part \childdocname{} of \childdocjob{}.|
\end{center}

%%%%%%%%%%%%%%%%%%%%%%%%%%%%%%%%%%%%%%%%%%%%%%%%%%%%%%%%%%%%%%%%%%%%%%%%%%%%%%%%
\subsection{Flags}
\label{sec:flags}

The package makes it easy to generate different versions
of the main or child documents.
To this end compilation flags can be defined
and assigned different default values.
They will be particularly useful in conjunction
with the forwarding mechanism described in \secref{sec:forward}.

For example, it may be useful to have a flag |\version|
which can be set to |draft| or |final|.
The document source will contain some conditional code
depending on the value of |\version|.
Suppose further, the flag should default to |final| for the main file
and to |draft| for child files
which is a natural assignment for editing the document.
This is achieved by placing the following code
in the preamble of the main document
(below the |\childdocmain| directive):
%
\begin{center}
\begin{tabular}{l}
|\ifchilddoc|\\
|\providecommand{\version}{draft}|\\
|\||else|\\
|\providecommand{\version}{final}|\\
|\||fi|
\end{tabular}
\end{center}
%
The definition by |\providecommand| makes sure
that previous definitions are not overwritten.
Further statements |\providecommand{\version}{...}|
can thus be added before the above code to override it.

For the main file, one might add a line
(between |\childdocmain| and the above block)
%
\begin{center}
|%\ifchilddoc\||else\providecommand{\version}{draft}\||fi|
\end{center}
%
which can be uncommented to produce a draft version.
Likewise one can add a line to the very top of a child file
(above the |\childdocof{|\textit{main}|}| directive)
%
\begin{center}
|%\providecommand{\version}{final}|
\end{center}
%
which can be uncommented to produce the final version of this child document.

%%%%%%%%%%%%%%%%%%%%%%%%%%%%%%%%%%%%%%%%%%%%%%%%%%%%%%%%%%%%%%%%%%%%%%%%%%%%%%%%
\subsection{Forwarding}
\label{sec:forward}

Different versions of the main or child documents
using compilation flags as described in \secref{sec:flags}
can be (permanently) stored in different files
for convenient compilation, viewing and distribution.
To this end, the package defines a command
to pass on compilation to a different file:

%%%%%%%%%%%%%%%%%%%%%%%%%%%%%%%%%%%%%%%%
\DescribeMacro{\childdocforward}
The command |\childdocforward| redirects processing to
another source file:
%
\begin{center}
\begin{tabular}{l}
|\input{childdoc.def}|\\
|\childdocforward[|\textit{main}|]{|\textit{dest}|}|\\
\end{tabular}
\end{center}
%
The argument \textit{dest} is the destination file
(without extension).
It should be the main file or one of the child files.
Note that further \textsf{childdoc} directives
such as |\childdocof| and |\childdocforward|
in the indicated file will be processed in this form.
The optional argument \textit{main}
passes on directly to the main file \textit{main}
while pretending to compile the child \textit{dest}.
This form behaves as if \textit{dest}
issues |\childdocof{|\textit{main}|}| right away,
and no further \textsf{childdoc} directives will be processed.

%%%%%%%%%%%%%%%%%%%%%%%%%%%%%%%%%%%%%%%%
\DescribeMacro{\...prefix}
In the alternative form |\childdocforwardprefix|,
%
\begin{center}
\begin{tabular}{l}
|\input{childdoc.def}|\\
|\childdocforwardprefix[|\textit{main}|]{|\textit{prefix}|}{|\textit{dest}|}|
\end{tabular}
\end{center}
%
the destination file is determined by a pattern
depending on the current file:
To make this work, the current file must be called
`{\textit{prefix}\hspace{0.2em}\textit{suffix}}'
with \textit{prefix} matching precisely the argument.
Processing is then passed on to the file
`{\textit{dest}\hspace{0.2em}\textit{suffix}}'.
Surely, the same effect is achieved by
directly specifying the
argument `{\textit{dest}\hspace{0.2em}\textit{suffix}}'
in the first form.
However, that requires to set up a different file
for each child. With the alternative form of the command
all these files can have exactly the same content
which simplifies setting them up and maintaining them.

For example, the following file |draft.tex|
with a compilation flag |\version| as described in \secref{sec:flags}
compiles the main document as a draft:
%
\begin{center}
\begin{tabular}{l}
|\def\version{draft}|\\
|\input{childdoc.def}|\\
|\childdocforward{|\textit{main}|}|
\end{tabular}
\end{center}
%
Likewise, the following files |final|\textit{nn}|.tex|
compile the final version of the child document
|child|\textit{nn}|.tex|:
%
\begin{center}
\begin{tabular}{l}
|\def\version{final}|\\
|\input{childdoc.def}|\\
|\childdocforwardprefix{final}{child}|
\end{tabular}
\end{center}
%

Note that when several versions of a main file and/or of each child file
are to be generated, it may be convenient to set up a |Makefile| or
shell script to automatise the process.

%%%%%%%%%%%%%%%%%%%%%%%%%%%%%%%%%%%%%%%%%%%%%%%%%%%%%%%%%%%%%%%%%%%%%%%%%%%%%%%%
\subsection{Command Line Processing}
\label{sec:commandline}

The effect of redirection files can also be achieved by invoking
the \LaTeX{} compiler with a more elaborate command line.
Most conveniently this should be done as part
of a shell script or a |Makefile|.

When using \textsf{childdoc} in the main file, the following
command lines effectively perform a redirection
(note that depending on the shell being used,
backslashes may have to be doubled: `|\|' $\to$ `|\\|'):
%
\begin{center}
|... -jobname "|\textit{target}|" |\\|"|[\textit{flags}]%
|\input{childdoc.def}\childdocforward[|\textit{main}|]{|\textit{dest}|}"|
\end{center}
%
Here \textit{target} is the name of the output file,
\textit{main} is the name of the main file
and \textit{dest} is the name of the main or child file to be processed
(all filenames without extensions).
The optional argument \textit{main} can be omitted
if \textit{main} matches \textit{dest}.
Optionally, compilation \textit{flags} can be defined via |\def| commands.
This command line makes the \TeX{} engine believe
it is compiling the file \textit{target}
whose content is specified as the latter parameter.
The provided code then forwards the processing to
\textit{main} or \textit{dest} as described in \secref{sec:forward}.

%%%%%%%%%%%%%%%%%%%%%%%%%%%%%%%%%%%%%%%%%%%%%%%%%%%%%%%%%%%%%%%%%%%%%%%%%%%%%%%%
\subsection{Include by Input}
\label{sec:input}

Including child documents by |\include| has some restrictions by design.
Most notably, the content of a child document always occupies
its own set of pages; pages cannot be shared between child documents.
Usually, this behaviour makes perfect sense
because each child document contain an essential part of the document.
However, in some situations it may be desirable to compose
a document from a collection of parts
without having mandatory page breaks between then.
For this case, the package
provides a mechanism to include parts
by |\input| which can also be processed individually.
However, by construction this mechanism
requires manual handling of the content to be output.

%%%%%%%%%%%%%%%%%%%%%%%%%%%%%%%%%%%%%%%%
\DescribeMacro{\ifchilddocmanual}
The main file should be prepared as usual, see \secref{sec:include}.
However, the document body must make a distinction
between processing of an individual part and of the main document, e.g.:
%
\begin{center}
\begin{tabular}{l}
|\ifchilddocmanual|\\
|\input{\childdocname}|\\
|\||else|\\
\textit{document body with }|\input{|\textit{part}|}|\\
|\||fi|
\end{tabular}
\end{center}
%
The conditional |\ifchilddocmanual| is true whenever
a part to be included by |\input| is being compiled,
and the name of the part is stored in |\childdocname|.

%%%%%%%%%%%%%%%%%%%%%%%%%%%%%%%%%%%%%%%%
\DescribeMacro{\childdocby}
Each part to be included by |\input| should start with:
%
\begin{center}
\begin{tabular}{l}
|\input{childdoc.def}|\\
|\childdocby{|\textit{main}|}|\\
\end{tabular}
\end{center}
%
The directive |\childdocby| is similar to |\childdocof|
described in \secref{sec:include},
but the subsequent selection of content must be done manually.
To that end, both |\ifchilddoc| and |\ifchilddocmanual|
will be true upon processing of a part,
and the name of the part is stored in |\childdocname|.
Note that |\jobname| will be set to the filename of the current part
so that each part receives an individual |.aux| file
that does not interfere with the |.aux| file(s) of the main document.
This behaviour can be altered by the alternative form
|\childdocby[*]{|\textit{main}|}| (with a non-empty optional argument)
which uses the |.aux| file of the main document
by setting |\jobname| to \textit{main}.

%%%%%%%%%%%%%%%%%%%%%%%%%%%%%%%%%%%%%%%%%%%%%%%%%%%%%%%%%%%%%%%%%%%%%%%%%%%%%%%%
\subsection{Driver Development}
\label{sec:driver}

The \textsf{childdoc} mechanism can also be use for the development
of definition files such as \LaTeX{} styles or classes.
This case differs from the above setup with multiple parts
included by |\include| in that no |\includeonly| should be invoked.
This can be achieved by starting the include file
(before |\ProvidesPackage|) with:
%
\begin{center}
\begin{tabular}{l}
|\input{childdoc.def}|\\
|\childdocforward{|\textit{main}|}|\\
\end{tabular}
\end{center}
%
or alternatively with:
%
\begin{center}
\begin{tabular}{l}
|\input{childdoc.def}|\\
|\childdocby{|\textit{main}|}|\\
\end{tabular}
\end{center}
%
Both forms have slightly different effects as described above.
The main file is prepared as usual, see \secref{sec:include}.

%%%%%%%%%%%%%%%%%%%%%%%%%%%%%%%%%%%%%%%%%%%%%%%%%%%%%%%%%%%%%%%%%%%%%%%%%%%%%%%%
\subsection{Legacy Detection}
\label{sec:detection}

The directive |\childdocmain| in the main file can detect
whether the complete document or merely a child is to be compiled
even without using the directive |\childdocof|.
This method is deprecated because it is less robust
and there is no compelling reason to use it;
it is merely provided for backward compatibility
and it may be removed in future versions.

If the detection mechanism is to be used,
it is mandatory to correctly specify
the filename of the main file as the argument of |\childdocmain|:
%
\begin{center}
\begin{tabular}{l}
|\input{childdoc.def}|\\
|\childdocmain{|\textit{main}|}|\\
\end{tabular}
\end{center}
%
If |\jobname| does not match the argument \textit{main} of |\childdocmain|,
it is assumed that |\jobname| points to the child file to be compiled.
When using |\childdocmain| with the main file specified as argument,
it suffices to start a child file
with just |\input{|\textit{main}|}|
without loading of the package and using |\childdocof|.
If instead all processing is done
with the appropriate \textsf{childdoc} directives,
the argument of \textit{main} of |\childdocmain| can be empty.

An alternative version of the command line processing described
in \secref{sec:commandline} using the detection mechanism reads:
%
\begin{center}
|... -jobname "|\textit{target}|" "|[\textit{flags}]%
[|\def\jobname{|\textit{dest}|}|]|\input{|\textit{main}|}"|
\end{center}

%%%%%%%%%%%%%%%%%%%%%%%%%%%%%%%%%%%%%%%%%%%%%%%%%%%%%%%%%%%%%%%%%%%%%%%%%%%%%%%%
\subsection{Manual Code}
\label{sec:manual}

In case one cannot be certain whether the definitions file |childdoc.def|
is installed on the target \TeX{} distribution
and one prefers not to ship it,
it is conceivable to paste a few relevant commands into the sources.

To that end, drop all statements |\input{childdoc.def}|
and perform the replacements as outlined below.
Instead of |\childdocmain{|\textit{main}|}| add the following code
to the top of the main file:
%
\begin{center}
\begin{tabular}{l}
|\||ifdefined\childdocname\endinput\||fi\newif\ifchilddoc|\\
|\edef\childdocname{\scantokens\expandafter{\jobname\noexpand}}|\\
|\def\childdocmain{|\textit{main}|}\||ifx\childdocmain\childdocname\||else|\\
|\childdoctrue\includeonly{\childdocname}\let\jobname\childdocmain\||fi|\\
\end{tabular}
\end{center}
%
Instead of |\childdocof{|\textit{main}|}| just include the main file
at the top of each child file:
%
\begin{center}
|\input{|\textit{main}|}|
\end{center}
%
A simple redirection |\childdocforward{|\textit{dest}|}| is achieved by:
%
\begin{center}
|\def\jobname{|\textit{dest}|}\input{\jobname}|
\end{center}
%
The redirection with prefix
|\childdocforwardprefix[|\textit{prefix}|]{|\textit{dest}|}|
is accomplished by:
%
\begin{center}
\begin{tabular}{l}
|{\edef\jobname{\scantokens\expandafter{\jobname\noexpand}}|\\
|\def\redirectjob |\textit{prefix}|#1~~~{\gdef\jobname{|\textit{dest}|#1}}|\\
|\expandafter\redirectjob\jobname~~~}\input{\jobname}|
\end{tabular}
\end{center}

In an alternative approach,
child documents can be compiled by a specific command line
without additional code or specific definitions:
%
\begin{center}
|... -jobname "|\textit{target}|" "|[\textit{flags}]%
|\includeonly{|\textit{dest}|}\input{|\textit{main}|}"|
\end{center}
%

%%%%%%%%%%%%%%%%%%%%%%%%%%%%%%%%%%%%%%%%%%%%%%%%%%%%%%%%%%%%%%%%%%%%%%%%%%%%%%%%
%%%%%%%%%%%%%%%%%%%%%%%%%%%%%%%%%%%%%%%%%%%%%%%%%%%%%%%%%%%%%%%%%%%%%%%%%%%%%%%%
\section{Information}

%%%%%%%%%%%%%%%%%%%%%%%%%%%%%%%%%%%%%%%%%%%%%%%%%%%%%%%%%%%%%%%%%%%%%%%%%%%%%%%%
\subsection{Copyright}

Copyright \copyright{} 2017--2018 Niklas Beisert

This work may be distributed and/or modified under the
conditions of the \LaTeX{} Project Public License, either version 1.3
of this license or (at your option) any later version.
The latest version of this license is in
  \url{http://www.latex-project.org/lppl.txt}
and version 1.3 or later is part of all distributions of \LaTeX{}
version 2005/12/01 or later.

This work has the LPPL maintenance status `maintained'.

The Current Maintainer of this work is Niklas Beisert.

This work consists of the files |README.txt|, |childdoc.ins| and |childdoc.dtx|
as well as the derived files |childdoc.def|, |cdocsamp.tex|
with |cdocsch1.tex|, |cdocsch2.tex|, |cdocspt3.tex|, |cdocspt4.tex|,
|cdocsdrf.tex|, |cdocsfn1.tex|, |cdocsfn2.tex|
as well as |childdoc.pdf|.

%%%%%%%%%%%%%%%%%%%%%%%%%%%%%%%%%%%%%%%%%%%%%%%%%%%%%%%%%%%%%%%%%%%%%%%%%%%%%%%%
\subsection{Files and Installation}

The package consists of the files:
%
\begin{center}
\begin{tabular}{ll}
    |README.txt|   & readme file \\
    |childdoc.ins| & installation file \\
    |childdoc.dtx| & source file \\
    |childdoc.def| & definition file \\
    |cdocsamp.tex| & sample main file \\
    |cdocsch1.tex| & sample include file \\
    |cdocsch2.tex| & sample include file \\
    |cdocspt3.tex| & sample part file \\
    |cdocspt4.tex| & sample part file \\
    |cdocsdrf.tex| & sample redirection file \\
    |cdocsfn1.tex| & sample redirection file \\
    |cdocsfn2.tex| & sample redirection file \\
    |childdoc.pdf| & manual
\end{tabular}
\end{center}
%
The distribution consists of the files
|README.txt|, |childdoc.ins| and |childdoc.dtx|.
%
\begin{itemize}
\item
Run (pdf)\LaTeX{} on |childdoc.dtx|
to compile the manual |childdoc.pdf| (this file).
\item
Run \LaTeX{} on |childdoc.ins| to create the definitions file |childdoc.def|
and the sample |cdocsamp.tex| with include files
|cdocsch1.tex|, |cdocsch2.tex|, |cdocspt3.tex|, |cdocspt4.tex|,
|cdocsdrf.tex|, |cdocsfn1.tex|, |cdocsfn2.tex|.
Then copy the file |childdoc.def| to an appropriate directory of your \LaTeX{}
distribution, e.g.\ \textit{texmf-root}|/tex/latex/childdoc|.
\end{itemize}

%%%%%%%%%%%%%%%%%%%%%%%%%%%%%%%%%%%%%%%%%%%%%%%%%%%%%%%%%%%%%%%%%%%%%%%%%%%%%%%%
\subsection{Related CTAN Packages}

There are several other packages which offer a similar functionality:
%
\begin{itemize}
\item
The packages
\href{http://ctan.org/pkg/docmute}{\textsf{docmute}},
\href{http://ctan.org/pkg/includex}{\textsf{includex}} and
\href{http://ctan.org/pkg/standalone}{\textsf{standalone}}
provide commands to include only the document body of
a child file thus allowing both files to be compiled individually.
\item
The packages \href{http://ctan.org/pkg/subdocs}{\textsf{subdocs}}
and \href{http://ctan.org/pkg/subfiles}{\textsf{subfiles}}
provide structures in which the main and child documents can be
encapsulated and allowing them to be compiled individually.
The inclusion mechanism is different from the conventional |\include|.
\item
The package \href{http://ctan.org/pkg/combine}{\textsf{combine}}
is an elaborate solution to combine several documents into one.
\end{itemize}
%
See also the CTAN topic \href{http://ctan.org/topic/subdocs}{\textsf{subdocs}}
for further related packages.
The present package differs from the above solutions in that
a document structure constructed with the conventional |\include| mechanism
just needs two extra commands at the top of every file
such that all constituent files can be compiled individually.

%%%%%%%%%%%%%%%%%%%%%%%%%%%%%%%%%%%%%%%%%%%%%%%%%%%%%%%%%%%%%%%%%%%%%%%%%%%%%%%%
%\subsection{Feature Suggestions}
%
%The following is a list of features which may be useful for future
%versions of this package:
%%
%\begin{itemize}
%\item
%\ldots
%\end{itemize}

%%%%%%%%%%%%%%%%%%%%%%%%%%%%%%%%%%%%%%%%%%%%%%%%%%%%%%%%%%%%%%%%%%%%%%%%%%%%%%%%
\subsection{Revision History}

%%%%%%%%%%%%%%%%%%%%%%%%%%%%%%%%%%%%%%%%
\paragraph{v2.0:} 2018/12/30

\begin{itemize}
\item
immediate forward processing
\item
added |\childdocby| mechanism
\item
manual restructured
\end{itemize}

%%%%%%%%%%%%%%%%%%%%%%%%%%%%%%%%%%%%%%%%
\paragraph{v1.6:} 2018/01/17

\begin{itemize}
\item
application for development of include files
\item
corrections to manual
\end{itemize}

%%%%%%%%%%%%%%%%%%%%%%%%%%%%%%%%%%%%%%%%
\paragraph{v1.5:} 2017/05/21

\begin{itemize}
\item
more complete structuring introduced
\item
|\childdocof| introduced
\item
|\childdoc| renamed to |\childdocmain|
\item
|\childredirect| renamed to |\childdocforward| and |\childdocforwardprefix|
and functionality expanded
\end{itemize}

%%%%%%%%%%%%%%%%%%%%%%%%%%%%%%%%%%%%%%%%
\paragraph{v1.0:} 2017/04/27

\begin{itemize}
\item
manual and install package
\item
first version published on CTAN
\end{itemize}

%%%%%%%%%%%%%%%%%%%%%%%%%%%%%%%%%%%%%%%%
\paragraph{v0.6:} 2017/04/26

\begin{itemize}
\item
redirection mechanism added
\end{itemize}

%%%%%%%%%%%%%%%%%%%%%%%%%%%%%%%%%%%%%%%%
\paragraph{v0.5:} 2017/04/26

\begin{itemize}
\item
functionality in definition file
\end{itemize}


%%%%%%%%%%%%%%%%%%%%%%%%%%%%%%%%%%%%%%%%%%%%%%%%%%%%%%%%%%%%%%%%%%%%%%%%%%%%%%%%
%%%%%%%%%%%%%%%%%%%%%%%%%%%%%%%%%%%%%%%%%%%%%%%%%%%%%%%%%%%%%%%%%%%%%%%%%%%%%%%%
%%%%%%%%%%%%%%%%%%%%%%%%%%%%%%%%%%%%%%%%%%%%%%%%%%%%%%%%%%%%%%%%%%%%%%%%%%%%%%%%
\appendix

\settowidth\MacroIndent{\rmfamily\scriptsize 000\ }

 \DocInput{childdoc.dtx}

\end{document}
%</driver>
% \fi
%
% %%%%%%%%%%%%%%%%%%%%%%%%%%%%%%%%%%%%%%%%%%%%%%%%%%%%%%%%%%%%%%%%%%%%%%%%%%%%%%
% %%%%%%%%%%%%%%%%%%%%%%%%%%%%%%%%%%%%%%%%%%%%%%%%%%%%%%%%%%%%%%%%%%%%%%%%%%%%%%
% \section{Sample}
%\iffalse
%<*samplemain>
%\fi
%
% The following presents a sample document
% with two chapters, two parts, a title page,
% a compile flag as well as three forwarding files to set the flag.
% It consists of eight |.tex| files:
% \begin{center}
% \begin{tabular}{ll}
% |cdocsamp.tex|&main file\\
% |cdocsch1.tex|&include file for chapter 1\\
% |cdocsch2.tex|&include file for chapter 2\\
% |cdocspt3.tex|&include file for part 3\\
% |cdocspt4.tex|&include file for part 4\\
% |cdocsdrf.tex|&forwarding file for main file in draft mode\\
% |cdocsfi1.tex|&forwarding file for final version of chapter 1\\
% |cdocsfi2.tex|&forwarding file for final version of chapter 2\\
% \end{tabular}
% \end{center}
% Each of the eight files can be compiled directly by the \LaTeX{} compiler.
%
% %%%%%%%%%%%%%%%%%%%%%%%%%%%%%%%%%%%%%%
% \paragraph{Main File.}
%
% The main file is called |cdocsamp.tex|.
%
% Load the \textsf{childdoc} definitions and
% declare the filename for the main document:
%    \begin{macrocode}
\input{childdoc.def}
\childdocmain{}
%    \end{macrocode}

% Optional override for |\version| flag:
%    \begin{macrocode}
%%\ifchilddoc\else\providecommand{\version}{draft}\fi
%    \end{macrocode}

% Define the default values for the |\version| flag
% (|final| for the main file and |draft| for childs):
%    \begin{macrocode}
\ifchilddoc
\providecommand{\version}{draft}
\else
\providecommand{\version}{final}
\fi
%    \end{macrocode}

% Load the standard document class:
%    \begin{macrocode}
\documentclass[12pt]{article}
%    \end{macrocode}

% Start the document body:
%    \begin{macrocode}
\begin{document}
%    \end{macrocode}

% Declare a title page.
% Print title, part of document being processed and version flag:
%    \begin{macrocode}
\addtocounter{page}{-1}
\begin{center}
{\LARGE\bfseries{}childdoc example\par}
\vspace{1cm}
\ifchilddoc
\ifchilddocmanual part\else chapter\fi:
`\childdocname' of `\childdocjob'\par
\else
main document: `\childdocjob'\par
\fi
version: \version\par
\end{center}
\newpage
%    \end{macrocode}

% Manually include selected file,
% otherwise process as usual:
%    \begin{macrocode}
\ifchilddocmanual
\section*{part `\childdocname'}
\input{\childdocname}
\else
%    \end{macrocode}

% Include the two chapters:
%    \begin{macrocode}
\include{cdocsch1}
\include{cdocsch2}
%    \end{macrocode}

% Include the two parts unless only chapters should be displayed:
%    \begin{macrocode}
\ifchilddoc\else
\section{part three}
\input{cdocspt3}
\section{part four}
\input{cdocspt4}
\fi
%    \end{macrocode}

% Process as usual until here:
%    \begin{macrocode}
\fi
%    \end{macrocode}

% End of document body:
%    \begin{macrocode}
\end{document}
%    \end{macrocode}
%\iffalse
%</samplemain>
%\fi
%
% %%%%%%%%%%%%%%%%%%%%%%%%%%%%%%%%%%%%%%
% \paragraph{Chapter Include Files.}
%
% The include files are called |cdocsch1.tex| and |cdocsch2.tex|.
%
%\iffalse
%<*samplechap1|samplechap2>
%\fi

% Optional override for |\version| flag:
%    \begin{macrocode}
%%\providecommand{\version}{final}
%    \end{macrocode}

% Include the main document:
%    \begin{macrocode}
\input{childdoc.def}
\childdocof{cdocsamp}
%    \end{macrocode}

%\iffalse
%</samplechap1|samplechap2>
%\fi
%
%\iffalse
%<*samplechap1>
%\fi
% Some text for chapter 1:
%    \begin{macrocode}
\section{one}
some text in chapter one
%    \end{macrocode}

%\iffalse
%</samplechap1>
%\fi
% Some text for chapter 2:
%\iffalse
%<*samplechap2>
%\fi
%    \begin{macrocode}
\section{two}
more text in chapter two
%    \end{macrocode}

%\iffalse
%</samplechap2>
%\fi
%
% %%%%%%%%%%%%%%%%%%%%%%%%%%%%%%%%%%%%%%
% \paragraph{Part Include Files.}
%
% The include files are called |cdocspt3.tex| and |cdocspt4.tex|.
%
%\iffalse
%<*samplepart3|samplepart4>
%\fi

% Optional override for |\version| flag:
%    \begin{macrocode}
%%\providecommand{\version}{final}
%    \end{macrocode}

% Include the main document:
%    \begin{macrocode}
\input{childdoc.def}
\childdocby{cdocsamp}
%    \end{macrocode}

%\iffalse
%</samplepart3|samplepart4>
%\fi
%
%\iffalse
%<*samplepart3>
%\fi
% Some text for part 3:
%    \begin{macrocode}
some text in part three
%    \end{macrocode}

%\iffalse
%</samplepart3>
%\fi
% Some text for part 4:
%\iffalse
%<*samplepart4>
%\fi
%    \begin{macrocode}
more text in part four
%    \end{macrocode}

%\iffalse
%</samplepart4>
%\fi
%
% %%%%%%%%%%%%%%%%%%%%%%%%%%%%%%%%%%%%%%
% \paragraph{Forwarding for a Complete Draft.}
%
% The following forwarding file |cdocsdrf.tex|
% compiles the main document in draft mode:
%\iffalse
%<*sampledraft>
%\fi
%    \begin{macrocode}
\def\version{draft}
\input{childdoc.def}
\childdocforward{cdocsamp}
%    \end{macrocode}

%\iffalse
%</sampledraft>
%\fi
%
% %%%%%%%%%%%%%%%%%%%%%%%%%%%%%%%%%%%%%%
% \paragraph{Forwarding for Final Version of the Chapters.}
%
% The following forwarding files |cdocsfn1.tex| and |cdocsfn2.tex|
% (with identical content)
% compile the final versions of the child documents
% |cdocsch1.tex| and |cdocsch2.tex|, respectively:
%\iffalse
%<*samplefinal>
%\fi
%    \begin{macrocode}
\def\version{final}
\input{childdoc.def}
\childdocforwardprefix[cdocsamp]{cdocsfn}{cdocsch}
%    \end{macrocode}

%\iffalse
%</samplefinal>
%\fi
%
% %%%%%%%%%%%%%%%%%%%%%%%%%%%%%%%%%%%%%%
% \paragraph{Command Line Processing.}
%
% The following three command lines generate the output files
% |cdocscld|, |cdocscl1| and |cdocscl2|
% which should be identical to
% |cdocsdrf|, |cdocsch1| and |cdocsfn2|, respectively:
% \begin{center}
% \begin{tabular}{l}
% |latex -jobname cdocscld \|\\
% |  "\def\version{draft}\input{childdoc.def}\childdocforward{cdocsamp}"|\\
% |latex -jobname cdocscl1 \|\\
% |  "\input{childdoc.def}\childdocforward[cdocsamp]{cdocsch1}"|\\
% |latex -jobname cdocscl2 \|\\
% |  "\def\version{final}\input{childdoc.def}\childdocforward{cdocsch2}"|
% \end{tabular}
% \end{center}
% Note that the trailing backslash on each first line
% merely continues the input to the second line
% (for convenient cut ant paste).
% Furthermore, the command |latex| can be replaced by any
% of its alternative versions such as |pdflatex|.
%
% %%%%%%%%%%%%%%%%%%%%%%%%%%%%%%%%%%%%%%%%%%%%%%%%%%%%%%%%%%%%%%%%%%%%%%%%%%%%%%
% %%%%%%%%%%%%%%%%%%%%%%%%%%%%%%%%%%%%%%%%%%%%%%%%%%%%%%%%%%%%%%%%%%%%%%%%%%%%%%
% \section{Implementation}
%\iffalse
%<*package>
%\fi
%
% This section describes the definitions file |childdoc.def|.

% The definitions cannot be loaded using |\usepackage| or |\RequirePackage|
% which has a mechanism to prevent loading a style file more than once.
% When loading the definitions by means of |\input|
% multiple instances have to be prevented manually:
%\iffalse
%This code needs to be before the `\ProvidesFile' directive
%which is defined at the beginning of this file.
%Therefore it is also placed there and commented out here.
%</package>
%<*discard>
%\fi
%    \begin{macrocode}
\ifdefined\childdocmain\endinput\fi
%    \end{macrocode}
%\iffalse
%</discard>
%<*package>
%\fi
%
% \macro{\ifchilddoc}
% \macro{\ifchilddocmanual}
% The conditional |\ifchilddoc| tells whether a
% child (true) or main (false) document is being compiled.
% The conditional |\ifchilddocmanual| tells whether
% the |\includeonly| mechanism is used (false) or
% the selection of child files must be performed manually (true).
% The definitions initialise to false:
%    \begin{macrocode}
\newif\ifchilddoc
\newif\ifchilddocmanual
%    \end{macrocode}

% \macro{\childdocname}
% \macro{\childdocjob}
% The macro |\childdocname| stores the name of the main document
% to be compiled. The macro |\childdocjob| stores the name of
% the document on which the \LaTeX{} compiler was originally invoked.
% The content of |\jobname| cannot be compared
% to filenames specified in the source due to different catcodes.
% The following code rescans |\jobname|, stores the result
% in |\childdocname| and saves a copy in |\childdocjob|:
%    \begin{macrocode}
\edef\childdocname{\scantokens\expandafter{\jobname\noexpand}}
\let\childdocjob\childdocname
%    \end{macrocode}

% \macro{\childdocdisable}
% The macro |\childdocdisable| prevents the main file
% from being processed more than once.
% At this stage, the main document command |\childdocmain|
% is assumed to be called once again where it should do nothing.
% Any subsequent call to it should prevent
% a secondary processing of the main document
% It overwrites the forwarding commands
% |\childdocof| and |\childdocforward|
% with empty macros to prevent further inclusions of the main document:
%    \begin{macrocode}
\newcommand{\childdocdisable}
{
  \renewcommand{\childdocmain}[1]{\renewcommand{\childdocmain}[1]{\endinput}}
  \renewcommand{\childdocof}[1]{}
  \renewcommand{\childdocby}[2][]{}
  \renewcommand{\childdocforward}[2][]{}
  \renewcommand{\childdocdisable}{}
}
%    \end{macrocode}

% \macro{\childdocmain}
% The macro |\childdocmain| is to be called at the top of the main file
% with nothing or the main filename (without extension) as argument.
% First, it breaks loops.
% If the argument is not empty and does not match |\childdocname|
% (which is set by the first inclusion of |childdoc.def|),
% |\ifchilddoc| is set to true, |\includeonly| is applied to the child file
% and |\jobname| is set to the main file
% (for proper handling of |.aux| files):
%    \begin{macrocode}
\newcommand{\childdocmain}[1]
{
  \childdocdisable\childdocmain{}
  \if?#1?\else
    \begingroup
      \def\childdoctmp{#1}
      \ifx\childdoctmp\childdocname
        \def\childdoctmp{}
      \else
        \def\childdoctmp
        {
          \childdoctrue
          \includeonly{\childdocname}
          \def\childdocjob{#1}
          \def\jobname{#1}
        }
      \fi
      \expandafter
    \endgroup
    \childdoctmp
  \fi
}
%    \end{macrocode}

% \macro{\childdocof}
% The command |\childdocof| redirects
% compilation to the main file |#1|.
%    \begin{macrocode}
\newcommand{\childdocof}[1]
{
  \childdocdisable
  \childdoctrue
  \includeonly{\childdocname}
  \def\jobname{#1}
  \def\childdocjob{#1}
  \input{#1}
}
%    \end{macrocode}

% \macro{\childdocby}
% The command |\childdocby| ....
%    \begin{macrocode}
\newcommand{\childdocby}[2][]
{
  \childdocdisable
  \childdoctrue
  \childdocmanualtrue
  \if?#1?\else
    \def\jobname{#2}
  \fi
  \def\childdocjob{#2}
  \input{#2}
  \endinput
}
%    \end{macrocode}

% \macro{\childdocforward}
% The command |\childdocforward| redirects
% compilation to the main file or
% (if the optional argument is given) a child file.
% Parameters are set as if the main file
% or a child file starting with |\childdocof| was compiled.
% Then compilation is handed over to the main file:
%    \begin{macrocode}
\newcommand{\childdocforward}[2][]
{
  \begingroup
    \if?#1?
      \def\childdoctmp
      {
        \def\childdocname{#2}
        \def\childdocjob{#2}
        \def\jobname{#2}
        \input{#2}
        \endinput
      }
    \else
      \def\childdoctmp
      {
        \childdocdisable
        \def\childdocname{#2}
        \childdoctrue
        \includeonly{#2}
        \def\childdocjob{#1}
        \def\jobname{#1}
        \input{#1}
        \endinput
      }
    \fi
    \expandafter
  \endgroup
  \childdoctmp
}
%    \end{macrocode}

% \macro{\childdocforwardprefix}
% The command |\childdocforwardprefix| redirects
% compilation to the main or a child file by means of a pattern.
% The prefix |#1| in the current filename is replaced by |#2|
% and the suffix of the current filename is kept
% (it is assumed that the filename does not contain the substring `|~~~|'
% which is used as a delimiter).
% Compilation is handed over to the new file by |\childdocforward|:
%    \begin{macrocode}
\newcommand{\childdocforwardprefix}[3][]
{
  \begingroup
    \def\childdocextract #2##1~~~{\def\childdoctmp{\childdocforward[#1]{#3##1}}}
    \expandafter\childdocextract\childdocname~~~
    \expandafter
  \endgroup
  \childdoctmp
}
%    \end{macrocode}

% \macro{\childdoc}
% The deprecated macro |\childdoc| is a legacy version of |\childdocmain|:
%    \begin{macrocode}
\newcommand{\childdoc}{\childdocmain}
%    \end{macrocode}

% \macro{\childdocredirect}
% The deprecated macro |\childdocredirect| is a legacy version
% of |\childdocforward| and |\childdocforwardprefix|:
%    \begin{macrocode}
\newcommand{\childdocredirect}[2][]
{
  \begingroup
    \if?#1?
      \def\childdoctmp{\childdocforward{#2}}
    \else
      \def\childdoctmp{\childdocforwardprefix{#1}{#2}}
    \fi
    \expandafter
  \endgroup
  \childdoctmp
}
%    \end{macrocode}

%\iffalse
%</package>
%\fi
%
\endinput
|\\
|\childdocforwardprefix[|\textit{main}|]{|\textit{prefix}|}{|\textit{dest}|}|
\end{tabular}
\end{center}
%
the destination file is determined by a pattern
depending on the current file:
To make this work, the current file must be called
`{\textit{prefix}\hspace{0.2em}\textit{suffix}}'
with \textit{prefix} matching precisely the argument.
Processing is then passed on to the file
`{\textit{dest}\hspace{0.2em}\textit{suffix}}'.
Surely, the same effect is achieved by
directly specifying the
argument `{\textit{dest}\hspace{0.2em}\textit{suffix}}'
in the first form.
However, that requires to set up a different file
for each child. With the alternative form of the command
all these files can have exactly the same content
which simplifies setting them up and maintaining them.

For example, the following file |draft.tex|
with a compilation flag |\version| as described in \secref{sec:flags}
compiles the main document as a draft:
%
\begin{center}
\begin{tabular}{l}
|\def\version{draft}|\\
|% \iffalse
%
% childdoc.dtx Copyright (C) 2017-2018 Niklas Beisert
%
% This work may be distributed and/or modified under the
% conditions of the LaTeX Project Public License, either version 1.3
% of this license or (at your option) any later version.
% The latest version of this license is in
%   http://www.latex-project.org/lppl.txt
% and version 1.3 or later is part of all distributions of LaTeX
% version 2005/12/01 or later.
%
% This work has the LPPL maintenance status `maintained'.
%
% The Current Maintainer of this work is Niklas Beisert.
%
% This work consists of the files childdoc.dtx and childdoc.ins
% and the derived files childdoc.def and cdocsamp.tex with
% cdocsch1.tex, cdocsch2.tex, cdocsdrf.tex, cdocsfn1.tex, cdocsfn2.tex.
%
%<package>\ifdefined\childdocmain\endinput\fi
%<package>\ProvidesFile{childdoc.def}[2018/12/30 v2.0 child document driver]
%<samplemain>\ProvidesFile{cdocsamp.tex}[2018/12/30 v2.0 sample for childdoc]
%<*driver>
%\ProvidesFile{childdoc.drv}[2018/12/30 v2.0 childdoc reference manual file]
\PassOptionsToClass{10pt,a4paper}{article}
\documentclass{ltxdoc}

\usepackage[margin=35mm]{geometry}
\usepackage{hyperref}
\usepackage{hyperxmp}
\usepackage[usenames]{color}

\hypersetup{colorlinks=true}
\hypersetup{pdfstartview=FitH}
\hypersetup{pdfpagemode=UseNone}
\hypersetup{pdfsource={}}
\hypersetup{pdflang={en-UK}}
\hypersetup{pdfcopyright={Copyright 2017-2018 Niklas Beisert.
  This work may be distributed and/or modified under the
  conditions of the LaTeX Project Public License, either version 1.3
  of this license or (at your option) any later version.}}
\hypersetup{pdflicenseurl={http://www.latex-project.org/lppl.txt}}
\hypersetup{pdfcontactaddress={ETH Zurich, ITP, HIT K,
  Wolfgang-Pauli-Strasse 27}}
\hypersetup{pdfcontactpostcode={8093}}
\hypersetup{pdfcontactcity={Zurich}}
\hypersetup{pdfcontactcountry={Switzerland}}
\hypersetup{pdfcontactemail={nbeisert@itp.phys.ethz.ch}}
\hypersetup{pdfcontacturl={http://people.phys.ethz.ch/\xmptilde nbeisert/}}

\newcommand{\secref}[1]{\hyperref[#1]{section \ref*{#1}}}

\parskip1ex
\parindent0pt
\let\olditemize\itemize
\def\itemize{\olditemize\parskip0pt}

\begin{document}

\title{The \textsf{childdoc} Package}
\hypersetup{pdftitle={The childdoc Package}}
\author{Niklas Beisert\\[2ex]
  Institut f\"ur Theoretische Physik\\
  Eidgen\"ossische Technische Hochschule Z\"urich\\
  Wolfgang-Pauli-Strasse 27, 8093 Z\"urich, Switzerland\\[1ex]
  \href{mailto:nbeisert@itp.phys.ethz.ch}
  {\texttt{nbeisert@itp.phys.ethz.ch}}}
\hypersetup{pdfauthor={Niklas Beisert}}
\hypersetup{pdfsubject={Manual for the LaTeX2e Package childdoc}}
\date{30 December 2018, \textsf{v2.0}}
\maketitle

\begin{abstract}\noindent
\textsf{childdoc} is a \LaTeXe{} package
that enables the direct compilation
of document sections included by |\include|
to individual files.
\end{abstract}

\begingroup
\parskip0ex
\tableofcontents
\endgroup

%%%%%%%%%%%%%%%%%%%%%%%%%%%%%%%%%%%%%%%%%%%%%%%%%%%%%%%%%%%%%%%%%%%%%%%%%%%%%%%%
%%%%%%%%%%%%%%%%%%%%%%%%%%%%%%%%%%%%%%%%%%%%%%%%%%%%%%%%%%%%%%%%%%%%%%%%%%%%%%%%
\section{Introduction}

\LaTeX{} provides a mechanism to structure a large document (such as a book)
into a main file and several child files (containing the chapters)
using the |\include| command.
This mechanism is beneficial for documents
which span hundreds of pages in order to
make the source file(s) more manageable.
Moreover, compilation can be restricted to
selected child files by means of the |\includeonly| command.
The latter feature can be used to reduce the compilation time while editing
(this was significantly more useful in the earlier days of \LaTeX{})
or to generate a smaller document which is easier to navigate.
Another application of |\includeonly| is to generate
documents consisting of selected parts of the complete document.

However, there are a few drawbacks of the plain |\include| mechanism:
\begin{itemize}
\item
The child files cannot be compiled on their own,
they can only be compiled via the main file.
A naive editing environment
(such as a text editor with an option
to have the current file processed by \LaTeX)
may require one to switch to the main file before compiling;
attempting to compile the child file produces errors.
\item
The main file must be modified (each time)
to adjust the |\includeonly| command
to the present needs. This easily leaves the main file in a messy state.
\item
The generated document will always carry the filename
of the main document. This is inconvenient if
several child files are to be compiled and
to be kept for distribution.
\end{itemize}

The present package provides a simple interface
to make child files individually compilable by \LaTeX{}.
Compiling a child file then has the same effect as compiling
the main file with an |\includeonly| command
to select the appropriate child.
Moreover the generated document will carry the name of the child
rather than the main file.
This resolves all three above issues.

This feature is meant to make the editing of books,
thesis documents and lecture notes somewhat more convenient.
However, the package can also be used efficiently for
composing a series of documents (such as exercise sheets)
which are typically distributed individually.
It then assists the author in generating the individual documents
(potentially in different versions)
as well as a document containing the collected series.
Another application is in developing style files
or other kinds of included material
where compilation of the style file could redirect
to a sample or test file.

%%%%%%%%%%%%%%%%%%%%%%%%%%%%%%%%%%%%%%%%%%%%%%%%%%%%%%%%%%%%%%%%%%%%%%%%%%%%%%%%
%%%%%%%%%%%%%%%%%%%%%%%%%%%%%%%%%%%%%%%%%%%%%%%%%%%%%%%%%%%%%%%%%%%%%%%%%%%%%%%%
\section{Usage}

First of all, the package \textsf{childdoc} is \emph{not} a standard
\LaTeXe{} |.sty| style file! Therefore it needs to be invoked in
a non-standard way.

%%%%%%%%%%%%%%%%%%%%%%%%%%%%%%%%%%%%%%%%%%%%%%%%%%%%%%%%%%%%%%%%%%%%%%%%%%%%%%%%
\subsection{Included Files}
\label{sec:include}

%%%%%%%%%%%%%%%%%%%%%%%%%%%%%%%%%%%%%%%%
\DescribeMacro{\childdocmain}
To use the package, add the commands
\begin{center}
\begin{tabular}{l}
|\input{childdoc.def}|\\
|\childdocmain{}|\\
\end{tabular}
\end{center}
at the very top of the main \LaTeX{} file,
in particular \emph{before} the |\documentclass| statement!
The argument of |\childdocmain| should be left empty
(but it must be present).

%%%%%%%%%%%%%%%%%%%%%%%%%%%%%%%%%%%%%%%%
\DescribeMacro{\childdocof}
Furthermore, add the commands
\begin{center}
\begin{tabular}{l}
|\input{childdoc.def}|\\
|\childdocof{|\textit{main}|}|\\
\end{tabular}
\end{center}
at the top of every child file \textit{child}
which is included by |\include{|\textit{child}|}|
from within the main file
(or at least for those files to be compiled individually).
The argument \textit{main} must be the filename of the main file.

There are a couple of
considerations in setting up the main and child documents:

%%%%%%%%%%%%%%%%%%%%%%%%%%%%%%%%%%%%%%%%
\paragraph{Restrictions.}

Please note the following restrictions:
\begin{itemize}
\item
|\childdocmain| must be called with one argument \textit{main}
to ensure compatibility with earlier version of the package.
It must either be empty (|\childdocmain{}|)
or precisely match the filename of the main file in which it is specified.
See \secref{sec:detection} for further information.
\item
The filename \textit{main} must be specified without the |.tex| extension.
\item
The filename \textit{main} is case sensitive
(even in case-insensitive file systems)
due to internal string comparison.
\item
The argument \textit{main} should be fully expanded, it cannot be a macro.
\item
Subdirectories and special characters should be avoided in filenames.
\item
The command |\childdocmain{|\textit{main}|}| must be followed by a whitespace.
It should not be followed immediately by another command
or by a comment mark `|%|'.
This is because the \TeX{} parser reads the token immediately following
the argument of |\childdocmain| and puts it
at the beginning of every child section;
however, a white\-space is ignored.
\end{itemize}

%%%%%%%%%%%%%%%%%%%%%%%%%%%%%%%%%%%%%%%%
\paragraph{Content of Main File.}

It is advisable to place all content in the child files included by |\include|.
Any output contained in the main file will appear in all child documents
unless suppressed manually;
it cannot be suppressed automatically by the |\includeonly| directive
and thus should normally be avoided.
A method to include some content in the main file
by means of conditional processing is described in \secref{sec:conditional}.

%%%%%%%%%%%%%%%%%%%%%%%%%%%%%%%%%%%%%%%%
\paragraph{Page Numbering.}

When only a part of the document is compiled,
the appropriate numbering of pages
(as well as other status parameters)
is determined from the |.aux| files.
The latter contain information from previous passes.
However this information needs to propagate through
all intermediate child documents.
Therefore the page numbering in child documents may well
be inconsistent until the complete document is compiled at least once.

A useful (if unconventional) way to always ensure a consistent
page numbering is to restart the numbering in each child document
and denote the pages by `\textit{child}|.|\textit{page}'
where \textit{child} represents the chapter/section number of the child file.
This can be achieved by the command
|\numberwithin{page}{|\textit{child}|}|
of the \textsf{amsmath} package
where \textit{child} can be |chapter| or |section|
depending on the chosen structuring.
Alternatively, one can modify the macro |\thepage| appropriately
and reset the counter |page| at the start of each child file.

%%%%%%%%%%%%%%%%%%%%%%%%%%%%%%%%%%%%%%%%%%%%%%%%%%%%%%%%%%%%%%%%%%%%%%%%%%%%%%%%
\subsection{Conditional Processing}
\label{sec:conditional}

The package provides a mechanism to compile different versions
of a document. To customise the versions further some conditional processing
can come in handy to distinguish which version is being compiled.
The package provides two macros to describe the compilation context:

%%%%%%%%%%%%%%%%%%%%%%%%%%%%%%%%%%%%%%%%
\DescribeMacro{\ifchilddoc}
The conditional |\ifchilddoc| distinguishes between the compilation of
child documents and the main document:
%
\begin{center}
|\ifchilddoc |\textit{child-code}| |[|\||else |\textit{main-code}]| \||fi|
\end{center}

%%%%%%%%%%%%%%%%%%%%%%%%%%%%%%%%%%%%%%%%
\DescribeMacro{\childdocname}
\DescribeMacro{\childdocjob}
The macro |\childdocname| contains the filename (without extension)
of the main or child file being processed.
Note that |\childdocjob| will always contain the name of the main file.

%%%%%%%%%%%%%%%%%%%%%%%%%%%%%%%%%%%%%%%%
\paragraph{Title Page.}

Conditional processing can be used to include a title or banner page
in the main document when proper precautions are taken.
Importantly, the code in the main file should ensure that the page counter
(as well as other status parameters which are stored in the |.aux| files)
takes the same value after the conditional processing.
Otherwise the page numbers may take divergent values
depending on which part is compiled.

For example, a title page could be declared by:
%
\begin{center}
\begin{tabular}{l}
|\ifchilddoc\||else|\\
|\addtocounter{page}{-1}|\\
\textit{code for title page}\\
|\newpage|\\
|\||fi|
\end{tabular}
\end{center}
%
A banner page for the child documents can be generated by:
%
\begin{center}
\begin{tabular}{l}
|\ifchilddoc|\\
|\addtocounter{page}{-1}|\\
\textit{code for banner page}\\
|\newpage|\\
|\||fi|
\end{tabular}
\end{center}
%
Here one could write a message such as:
\begin{center}
|This is the part \childdocname{} of \childdocjob{}.|
\end{center}

%%%%%%%%%%%%%%%%%%%%%%%%%%%%%%%%%%%%%%%%%%%%%%%%%%%%%%%%%%%%%%%%%%%%%%%%%%%%%%%%
\subsection{Flags}
\label{sec:flags}

The package makes it easy to generate different versions
of the main or child documents.
To this end compilation flags can be defined
and assigned different default values.
They will be particularly useful in conjunction
with the forwarding mechanism described in \secref{sec:forward}.

For example, it may be useful to have a flag |\version|
which can be set to |draft| or |final|.
The document source will contain some conditional code
depending on the value of |\version|.
Suppose further, the flag should default to |final| for the main file
and to |draft| for child files
which is a natural assignment for editing the document.
This is achieved by placing the following code
in the preamble of the main document
(below the |\childdocmain| directive):
%
\begin{center}
\begin{tabular}{l}
|\ifchilddoc|\\
|\providecommand{\version}{draft}|\\
|\||else|\\
|\providecommand{\version}{final}|\\
|\||fi|
\end{tabular}
\end{center}
%
The definition by |\providecommand| makes sure
that previous definitions are not overwritten.
Further statements |\providecommand{\version}{...}|
can thus be added before the above code to override it.

For the main file, one might add a line
(between |\childdocmain| and the above block)
%
\begin{center}
|%\ifchilddoc\||else\providecommand{\version}{draft}\||fi|
\end{center}
%
which can be uncommented to produce a draft version.
Likewise one can add a line to the very top of a child file
(above the |\childdocof{|\textit{main}|}| directive)
%
\begin{center}
|%\providecommand{\version}{final}|
\end{center}
%
which can be uncommented to produce the final version of this child document.

%%%%%%%%%%%%%%%%%%%%%%%%%%%%%%%%%%%%%%%%%%%%%%%%%%%%%%%%%%%%%%%%%%%%%%%%%%%%%%%%
\subsection{Forwarding}
\label{sec:forward}

Different versions of the main or child documents
using compilation flags as described in \secref{sec:flags}
can be (permanently) stored in different files
for convenient compilation, viewing and distribution.
To this end, the package defines a command
to pass on compilation to a different file:

%%%%%%%%%%%%%%%%%%%%%%%%%%%%%%%%%%%%%%%%
\DescribeMacro{\childdocforward}
The command |\childdocforward| redirects processing to
another source file:
%
\begin{center}
\begin{tabular}{l}
|\input{childdoc.def}|\\
|\childdocforward[|\textit{main}|]{|\textit{dest}|}|\\
\end{tabular}
\end{center}
%
The argument \textit{dest} is the destination file
(without extension).
It should be the main file or one of the child files.
Note that further \textsf{childdoc} directives
such as |\childdocof| and |\childdocforward|
in the indicated file will be processed in this form.
The optional argument \textit{main}
passes on directly to the main file \textit{main}
while pretending to compile the child \textit{dest}.
This form behaves as if \textit{dest}
issues |\childdocof{|\textit{main}|}| right away,
and no further \textsf{childdoc} directives will be processed.

%%%%%%%%%%%%%%%%%%%%%%%%%%%%%%%%%%%%%%%%
\DescribeMacro{\...prefix}
In the alternative form |\childdocforwardprefix|,
%
\begin{center}
\begin{tabular}{l}
|\input{childdoc.def}|\\
|\childdocforwardprefix[|\textit{main}|]{|\textit{prefix}|}{|\textit{dest}|}|
\end{tabular}
\end{center}
%
the destination file is determined by a pattern
depending on the current file:
To make this work, the current file must be called
`{\textit{prefix}\hspace{0.2em}\textit{suffix}}'
with \textit{prefix} matching precisely the argument.
Processing is then passed on to the file
`{\textit{dest}\hspace{0.2em}\textit{suffix}}'.
Surely, the same effect is achieved by
directly specifying the
argument `{\textit{dest}\hspace{0.2em}\textit{suffix}}'
in the first form.
However, that requires to set up a different file
for each child. With the alternative form of the command
all these files can have exactly the same content
which simplifies setting them up and maintaining them.

For example, the following file |draft.tex|
with a compilation flag |\version| as described in \secref{sec:flags}
compiles the main document as a draft:
%
\begin{center}
\begin{tabular}{l}
|\def\version{draft}|\\
|\input{childdoc.def}|\\
|\childdocforward{|\textit{main}|}|
\end{tabular}
\end{center}
%
Likewise, the following files |final|\textit{nn}|.tex|
compile the final version of the child document
|child|\textit{nn}|.tex|:
%
\begin{center}
\begin{tabular}{l}
|\def\version{final}|\\
|\input{childdoc.def}|\\
|\childdocforwardprefix{final}{child}|
\end{tabular}
\end{center}
%

Note that when several versions of a main file and/or of each child file
are to be generated, it may be convenient to set up a |Makefile| or
shell script to automatise the process.

%%%%%%%%%%%%%%%%%%%%%%%%%%%%%%%%%%%%%%%%%%%%%%%%%%%%%%%%%%%%%%%%%%%%%%%%%%%%%%%%
\subsection{Command Line Processing}
\label{sec:commandline}

The effect of redirection files can also be achieved by invoking
the \LaTeX{} compiler with a more elaborate command line.
Most conveniently this should be done as part
of a shell script or a |Makefile|.

When using \textsf{childdoc} in the main file, the following
command lines effectively perform a redirection
(note that depending on the shell being used,
backslashes may have to be doubled: `|\|' $\to$ `|\\|'):
%
\begin{center}
|... -jobname "|\textit{target}|" |\\|"|[\textit{flags}]%
|\input{childdoc.def}\childdocforward[|\textit{main}|]{|\textit{dest}|}"|
\end{center}
%
Here \textit{target} is the name of the output file,
\textit{main} is the name of the main file
and \textit{dest} is the name of the main or child file to be processed
(all filenames without extensions).
The optional argument \textit{main} can be omitted
if \textit{main} matches \textit{dest}.
Optionally, compilation \textit{flags} can be defined via |\def| commands.
This command line makes the \TeX{} engine believe
it is compiling the file \textit{target}
whose content is specified as the latter parameter.
The provided code then forwards the processing to
\textit{main} or \textit{dest} as described in \secref{sec:forward}.

%%%%%%%%%%%%%%%%%%%%%%%%%%%%%%%%%%%%%%%%%%%%%%%%%%%%%%%%%%%%%%%%%%%%%%%%%%%%%%%%
\subsection{Include by Input}
\label{sec:input}

Including child documents by |\include| has some restrictions by design.
Most notably, the content of a child document always occupies
its own set of pages; pages cannot be shared between child documents.
Usually, this behaviour makes perfect sense
because each child document contain an essential part of the document.
However, in some situations it may be desirable to compose
a document from a collection of parts
without having mandatory page breaks between then.
For this case, the package
provides a mechanism to include parts
by |\input| which can also be processed individually.
However, by construction this mechanism
requires manual handling of the content to be output.

%%%%%%%%%%%%%%%%%%%%%%%%%%%%%%%%%%%%%%%%
\DescribeMacro{\ifchilddocmanual}
The main file should be prepared as usual, see \secref{sec:include}.
However, the document body must make a distinction
between processing of an individual part and of the main document, e.g.:
%
\begin{center}
\begin{tabular}{l}
|\ifchilddocmanual|\\
|\input{\childdocname}|\\
|\||else|\\
\textit{document body with }|\input{|\textit{part}|}|\\
|\||fi|
\end{tabular}
\end{center}
%
The conditional |\ifchilddocmanual| is true whenever
a part to be included by |\input| is being compiled,
and the name of the part is stored in |\childdocname|.

%%%%%%%%%%%%%%%%%%%%%%%%%%%%%%%%%%%%%%%%
\DescribeMacro{\childdocby}
Each part to be included by |\input| should start with:
%
\begin{center}
\begin{tabular}{l}
|\input{childdoc.def}|\\
|\childdocby{|\textit{main}|}|\\
\end{tabular}
\end{center}
%
The directive |\childdocby| is similar to |\childdocof|
described in \secref{sec:include},
but the subsequent selection of content must be done manually.
To that end, both |\ifchilddoc| and |\ifchilddocmanual|
will be true upon processing of a part,
and the name of the part is stored in |\childdocname|.
Note that |\jobname| will be set to the filename of the current part
so that each part receives an individual |.aux| file
that does not interfere with the |.aux| file(s) of the main document.
This behaviour can be altered by the alternative form
|\childdocby[*]{|\textit{main}|}| (with a non-empty optional argument)
which uses the |.aux| file of the main document
by setting |\jobname| to \textit{main}.

%%%%%%%%%%%%%%%%%%%%%%%%%%%%%%%%%%%%%%%%%%%%%%%%%%%%%%%%%%%%%%%%%%%%%%%%%%%%%%%%
\subsection{Driver Development}
\label{sec:driver}

The \textsf{childdoc} mechanism can also be use for the development
of definition files such as \LaTeX{} styles or classes.
This case differs from the above setup with multiple parts
included by |\include| in that no |\includeonly| should be invoked.
This can be achieved by starting the include file
(before |\ProvidesPackage|) with:
%
\begin{center}
\begin{tabular}{l}
|\input{childdoc.def}|\\
|\childdocforward{|\textit{main}|}|\\
\end{tabular}
\end{center}
%
or alternatively with:
%
\begin{center}
\begin{tabular}{l}
|\input{childdoc.def}|\\
|\childdocby{|\textit{main}|}|\\
\end{tabular}
\end{center}
%
Both forms have slightly different effects as described above.
The main file is prepared as usual, see \secref{sec:include}.

%%%%%%%%%%%%%%%%%%%%%%%%%%%%%%%%%%%%%%%%%%%%%%%%%%%%%%%%%%%%%%%%%%%%%%%%%%%%%%%%
\subsection{Legacy Detection}
\label{sec:detection}

The directive |\childdocmain| in the main file can detect
whether the complete document or merely a child is to be compiled
even without using the directive |\childdocof|.
This method is deprecated because it is less robust
and there is no compelling reason to use it;
it is merely provided for backward compatibility
and it may be removed in future versions.

If the detection mechanism is to be used,
it is mandatory to correctly specify
the filename of the main file as the argument of |\childdocmain|:
%
\begin{center}
\begin{tabular}{l}
|\input{childdoc.def}|\\
|\childdocmain{|\textit{main}|}|\\
\end{tabular}
\end{center}
%
If |\jobname| does not match the argument \textit{main} of |\childdocmain|,
it is assumed that |\jobname| points to the child file to be compiled.
When using |\childdocmain| with the main file specified as argument,
it suffices to start a child file
with just |\input{|\textit{main}|}|
without loading of the package and using |\childdocof|.
If instead all processing is done
with the appropriate \textsf{childdoc} directives,
the argument of \textit{main} of |\childdocmain| can be empty.

An alternative version of the command line processing described
in \secref{sec:commandline} using the detection mechanism reads:
%
\begin{center}
|... -jobname "|\textit{target}|" "|[\textit{flags}]%
[|\def\jobname{|\textit{dest}|}|]|\input{|\textit{main}|}"|
\end{center}

%%%%%%%%%%%%%%%%%%%%%%%%%%%%%%%%%%%%%%%%%%%%%%%%%%%%%%%%%%%%%%%%%%%%%%%%%%%%%%%%
\subsection{Manual Code}
\label{sec:manual}

In case one cannot be certain whether the definitions file |childdoc.def|
is installed on the target \TeX{} distribution
and one prefers not to ship it,
it is conceivable to paste a few relevant commands into the sources.

To that end, drop all statements |\input{childdoc.def}|
and perform the replacements as outlined below.
Instead of |\childdocmain{|\textit{main}|}| add the following code
to the top of the main file:
%
\begin{center}
\begin{tabular}{l}
|\||ifdefined\childdocname\endinput\||fi\newif\ifchilddoc|\\
|\edef\childdocname{\scantokens\expandafter{\jobname\noexpand}}|\\
|\def\childdocmain{|\textit{main}|}\||ifx\childdocmain\childdocname\||else|\\
|\childdoctrue\includeonly{\childdocname}\let\jobname\childdocmain\||fi|\\
\end{tabular}
\end{center}
%
Instead of |\childdocof{|\textit{main}|}| just include the main file
at the top of each child file:
%
\begin{center}
|\input{|\textit{main}|}|
\end{center}
%
A simple redirection |\childdocforward{|\textit{dest}|}| is achieved by:
%
\begin{center}
|\def\jobname{|\textit{dest}|}\input{\jobname}|
\end{center}
%
The redirection with prefix
|\childdocforwardprefix[|\textit{prefix}|]{|\textit{dest}|}|
is accomplished by:
%
\begin{center}
\begin{tabular}{l}
|{\edef\jobname{\scantokens\expandafter{\jobname\noexpand}}|\\
|\def\redirectjob |\textit{prefix}|#1~~~{\gdef\jobname{|\textit{dest}|#1}}|\\
|\expandafter\redirectjob\jobname~~~}\input{\jobname}|
\end{tabular}
\end{center}

In an alternative approach,
child documents can be compiled by a specific command line
without additional code or specific definitions:
%
\begin{center}
|... -jobname "|\textit{target}|" "|[\textit{flags}]%
|\includeonly{|\textit{dest}|}\input{|\textit{main}|}"|
\end{center}
%

%%%%%%%%%%%%%%%%%%%%%%%%%%%%%%%%%%%%%%%%%%%%%%%%%%%%%%%%%%%%%%%%%%%%%%%%%%%%%%%%
%%%%%%%%%%%%%%%%%%%%%%%%%%%%%%%%%%%%%%%%%%%%%%%%%%%%%%%%%%%%%%%%%%%%%%%%%%%%%%%%
\section{Information}

%%%%%%%%%%%%%%%%%%%%%%%%%%%%%%%%%%%%%%%%%%%%%%%%%%%%%%%%%%%%%%%%%%%%%%%%%%%%%%%%
\subsection{Copyright}

Copyright \copyright{} 2017--2018 Niklas Beisert

This work may be distributed and/or modified under the
conditions of the \LaTeX{} Project Public License, either version 1.3
of this license or (at your option) any later version.
The latest version of this license is in
  \url{http://www.latex-project.org/lppl.txt}
and version 1.3 or later is part of all distributions of \LaTeX{}
version 2005/12/01 or later.

This work has the LPPL maintenance status `maintained'.

The Current Maintainer of this work is Niklas Beisert.

This work consists of the files |README.txt|, |childdoc.ins| and |childdoc.dtx|
as well as the derived files |childdoc.def|, |cdocsamp.tex|
with |cdocsch1.tex|, |cdocsch2.tex|, |cdocspt3.tex|, |cdocspt4.tex|,
|cdocsdrf.tex|, |cdocsfn1.tex|, |cdocsfn2.tex|
as well as |childdoc.pdf|.

%%%%%%%%%%%%%%%%%%%%%%%%%%%%%%%%%%%%%%%%%%%%%%%%%%%%%%%%%%%%%%%%%%%%%%%%%%%%%%%%
\subsection{Files and Installation}

The package consists of the files:
%
\begin{center}
\begin{tabular}{ll}
    |README.txt|   & readme file \\
    |childdoc.ins| & installation file \\
    |childdoc.dtx| & source file \\
    |childdoc.def| & definition file \\
    |cdocsamp.tex| & sample main file \\
    |cdocsch1.tex| & sample include file \\
    |cdocsch2.tex| & sample include file \\
    |cdocspt3.tex| & sample part file \\
    |cdocspt4.tex| & sample part file \\
    |cdocsdrf.tex| & sample redirection file \\
    |cdocsfn1.tex| & sample redirection file \\
    |cdocsfn2.tex| & sample redirection file \\
    |childdoc.pdf| & manual
\end{tabular}
\end{center}
%
The distribution consists of the files
|README.txt|, |childdoc.ins| and |childdoc.dtx|.
%
\begin{itemize}
\item
Run (pdf)\LaTeX{} on |childdoc.dtx|
to compile the manual |childdoc.pdf| (this file).
\item
Run \LaTeX{} on |childdoc.ins| to create the definitions file |childdoc.def|
and the sample |cdocsamp.tex| with include files
|cdocsch1.tex|, |cdocsch2.tex|, |cdocspt3.tex|, |cdocspt4.tex|,
|cdocsdrf.tex|, |cdocsfn1.tex|, |cdocsfn2.tex|.
Then copy the file |childdoc.def| to an appropriate directory of your \LaTeX{}
distribution, e.g.\ \textit{texmf-root}|/tex/latex/childdoc|.
\end{itemize}

%%%%%%%%%%%%%%%%%%%%%%%%%%%%%%%%%%%%%%%%%%%%%%%%%%%%%%%%%%%%%%%%%%%%%%%%%%%%%%%%
\subsection{Related CTAN Packages}

There are several other packages which offer a similar functionality:
%
\begin{itemize}
\item
The packages
\href{http://ctan.org/pkg/docmute}{\textsf{docmute}},
\href{http://ctan.org/pkg/includex}{\textsf{includex}} and
\href{http://ctan.org/pkg/standalone}{\textsf{standalone}}
provide commands to include only the document body of
a child file thus allowing both files to be compiled individually.
\item
The packages \href{http://ctan.org/pkg/subdocs}{\textsf{subdocs}}
and \href{http://ctan.org/pkg/subfiles}{\textsf{subfiles}}
provide structures in which the main and child documents can be
encapsulated and allowing them to be compiled individually.
The inclusion mechanism is different from the conventional |\include|.
\item
The package \href{http://ctan.org/pkg/combine}{\textsf{combine}}
is an elaborate solution to combine several documents into one.
\end{itemize}
%
See also the CTAN topic \href{http://ctan.org/topic/subdocs}{\textsf{subdocs}}
for further related packages.
The present package differs from the above solutions in that
a document structure constructed with the conventional |\include| mechanism
just needs two extra commands at the top of every file
such that all constituent files can be compiled individually.

%%%%%%%%%%%%%%%%%%%%%%%%%%%%%%%%%%%%%%%%%%%%%%%%%%%%%%%%%%%%%%%%%%%%%%%%%%%%%%%%
%\subsection{Feature Suggestions}
%
%The following is a list of features which may be useful for future
%versions of this package:
%%
%\begin{itemize}
%\item
%\ldots
%\end{itemize}

%%%%%%%%%%%%%%%%%%%%%%%%%%%%%%%%%%%%%%%%%%%%%%%%%%%%%%%%%%%%%%%%%%%%%%%%%%%%%%%%
\subsection{Revision History}

%%%%%%%%%%%%%%%%%%%%%%%%%%%%%%%%%%%%%%%%
\paragraph{v2.0:} 2018/12/30

\begin{itemize}
\item
immediate forward processing
\item
added |\childdocby| mechanism
\item
manual restructured
\end{itemize}

%%%%%%%%%%%%%%%%%%%%%%%%%%%%%%%%%%%%%%%%
\paragraph{v1.6:} 2018/01/17

\begin{itemize}
\item
application for development of include files
\item
corrections to manual
\end{itemize}

%%%%%%%%%%%%%%%%%%%%%%%%%%%%%%%%%%%%%%%%
\paragraph{v1.5:} 2017/05/21

\begin{itemize}
\item
more complete structuring introduced
\item
|\childdocof| introduced
\item
|\childdoc| renamed to |\childdocmain|
\item
|\childredirect| renamed to |\childdocforward| and |\childdocforwardprefix|
and functionality expanded
\end{itemize}

%%%%%%%%%%%%%%%%%%%%%%%%%%%%%%%%%%%%%%%%
\paragraph{v1.0:} 2017/04/27

\begin{itemize}
\item
manual and install package
\item
first version published on CTAN
\end{itemize}

%%%%%%%%%%%%%%%%%%%%%%%%%%%%%%%%%%%%%%%%
\paragraph{v0.6:} 2017/04/26

\begin{itemize}
\item
redirection mechanism added
\end{itemize}

%%%%%%%%%%%%%%%%%%%%%%%%%%%%%%%%%%%%%%%%
\paragraph{v0.5:} 2017/04/26

\begin{itemize}
\item
functionality in definition file
\end{itemize}


%%%%%%%%%%%%%%%%%%%%%%%%%%%%%%%%%%%%%%%%%%%%%%%%%%%%%%%%%%%%%%%%%%%%%%%%%%%%%%%%
%%%%%%%%%%%%%%%%%%%%%%%%%%%%%%%%%%%%%%%%%%%%%%%%%%%%%%%%%%%%%%%%%%%%%%%%%%%%%%%%
%%%%%%%%%%%%%%%%%%%%%%%%%%%%%%%%%%%%%%%%%%%%%%%%%%%%%%%%%%%%%%%%%%%%%%%%%%%%%%%%
\appendix

\settowidth\MacroIndent{\rmfamily\scriptsize 000\ }

 \DocInput{childdoc.dtx}

\end{document}
%</driver>
% \fi
%
% %%%%%%%%%%%%%%%%%%%%%%%%%%%%%%%%%%%%%%%%%%%%%%%%%%%%%%%%%%%%%%%%%%%%%%%%%%%%%%
% %%%%%%%%%%%%%%%%%%%%%%%%%%%%%%%%%%%%%%%%%%%%%%%%%%%%%%%%%%%%%%%%%%%%%%%%%%%%%%
% \section{Sample}
%\iffalse
%<*samplemain>
%\fi
%
% The following presents a sample document
% with two chapters, two parts, a title page,
% a compile flag as well as three forwarding files to set the flag.
% It consists of eight |.tex| files:
% \begin{center}
% \begin{tabular}{ll}
% |cdocsamp.tex|&main file\\
% |cdocsch1.tex|&include file for chapter 1\\
% |cdocsch2.tex|&include file for chapter 2\\
% |cdocspt3.tex|&include file for part 3\\
% |cdocspt4.tex|&include file for part 4\\
% |cdocsdrf.tex|&forwarding file for main file in draft mode\\
% |cdocsfi1.tex|&forwarding file for final version of chapter 1\\
% |cdocsfi2.tex|&forwarding file for final version of chapter 2\\
% \end{tabular}
% \end{center}
% Each of the eight files can be compiled directly by the \LaTeX{} compiler.
%
% %%%%%%%%%%%%%%%%%%%%%%%%%%%%%%%%%%%%%%
% \paragraph{Main File.}
%
% The main file is called |cdocsamp.tex|.
%
% Load the \textsf{childdoc} definitions and
% declare the filename for the main document:
%    \begin{macrocode}
\input{childdoc.def}
\childdocmain{}
%    \end{macrocode}

% Optional override for |\version| flag:
%    \begin{macrocode}
%%\ifchilddoc\else\providecommand{\version}{draft}\fi
%    \end{macrocode}

% Define the default values for the |\version| flag
% (|final| for the main file and |draft| for childs):
%    \begin{macrocode}
\ifchilddoc
\providecommand{\version}{draft}
\else
\providecommand{\version}{final}
\fi
%    \end{macrocode}

% Load the standard document class:
%    \begin{macrocode}
\documentclass[12pt]{article}
%    \end{macrocode}

% Start the document body:
%    \begin{macrocode}
\begin{document}
%    \end{macrocode}

% Declare a title page.
% Print title, part of document being processed and version flag:
%    \begin{macrocode}
\addtocounter{page}{-1}
\begin{center}
{\LARGE\bfseries{}childdoc example\par}
\vspace{1cm}
\ifchilddoc
\ifchilddocmanual part\else chapter\fi:
`\childdocname' of `\childdocjob'\par
\else
main document: `\childdocjob'\par
\fi
version: \version\par
\end{center}
\newpage
%    \end{macrocode}

% Manually include selected file,
% otherwise process as usual:
%    \begin{macrocode}
\ifchilddocmanual
\section*{part `\childdocname'}
\input{\childdocname}
\else
%    \end{macrocode}

% Include the two chapters:
%    \begin{macrocode}
\include{cdocsch1}
\include{cdocsch2}
%    \end{macrocode}

% Include the two parts unless only chapters should be displayed:
%    \begin{macrocode}
\ifchilddoc\else
\section{part three}
\input{cdocspt3}
\section{part four}
\input{cdocspt4}
\fi
%    \end{macrocode}

% Process as usual until here:
%    \begin{macrocode}
\fi
%    \end{macrocode}

% End of document body:
%    \begin{macrocode}
\end{document}
%    \end{macrocode}
%\iffalse
%</samplemain>
%\fi
%
% %%%%%%%%%%%%%%%%%%%%%%%%%%%%%%%%%%%%%%
% \paragraph{Chapter Include Files.}
%
% The include files are called |cdocsch1.tex| and |cdocsch2.tex|.
%
%\iffalse
%<*samplechap1|samplechap2>
%\fi

% Optional override for |\version| flag:
%    \begin{macrocode}
%%\providecommand{\version}{final}
%    \end{macrocode}

% Include the main document:
%    \begin{macrocode}
\input{childdoc.def}
\childdocof{cdocsamp}
%    \end{macrocode}

%\iffalse
%</samplechap1|samplechap2>
%\fi
%
%\iffalse
%<*samplechap1>
%\fi
% Some text for chapter 1:
%    \begin{macrocode}
\section{one}
some text in chapter one
%    \end{macrocode}

%\iffalse
%</samplechap1>
%\fi
% Some text for chapter 2:
%\iffalse
%<*samplechap2>
%\fi
%    \begin{macrocode}
\section{two}
more text in chapter two
%    \end{macrocode}

%\iffalse
%</samplechap2>
%\fi
%
% %%%%%%%%%%%%%%%%%%%%%%%%%%%%%%%%%%%%%%
% \paragraph{Part Include Files.}
%
% The include files are called |cdocspt3.tex| and |cdocspt4.tex|.
%
%\iffalse
%<*samplepart3|samplepart4>
%\fi

% Optional override for |\version| flag:
%    \begin{macrocode}
%%\providecommand{\version}{final}
%    \end{macrocode}

% Include the main document:
%    \begin{macrocode}
\input{childdoc.def}
\childdocby{cdocsamp}
%    \end{macrocode}

%\iffalse
%</samplepart3|samplepart4>
%\fi
%
%\iffalse
%<*samplepart3>
%\fi
% Some text for part 3:
%    \begin{macrocode}
some text in part three
%    \end{macrocode}

%\iffalse
%</samplepart3>
%\fi
% Some text for part 4:
%\iffalse
%<*samplepart4>
%\fi
%    \begin{macrocode}
more text in part four
%    \end{macrocode}

%\iffalse
%</samplepart4>
%\fi
%
% %%%%%%%%%%%%%%%%%%%%%%%%%%%%%%%%%%%%%%
% \paragraph{Forwarding for a Complete Draft.}
%
% The following forwarding file |cdocsdrf.tex|
% compiles the main document in draft mode:
%\iffalse
%<*sampledraft>
%\fi
%    \begin{macrocode}
\def\version{draft}
\input{childdoc.def}
\childdocforward{cdocsamp}
%    \end{macrocode}

%\iffalse
%</sampledraft>
%\fi
%
% %%%%%%%%%%%%%%%%%%%%%%%%%%%%%%%%%%%%%%
% \paragraph{Forwarding for Final Version of the Chapters.}
%
% The following forwarding files |cdocsfn1.tex| and |cdocsfn2.tex|
% (with identical content)
% compile the final versions of the child documents
% |cdocsch1.tex| and |cdocsch2.tex|, respectively:
%\iffalse
%<*samplefinal>
%\fi
%    \begin{macrocode}
\def\version{final}
\input{childdoc.def}
\childdocforwardprefix[cdocsamp]{cdocsfn}{cdocsch}
%    \end{macrocode}

%\iffalse
%</samplefinal>
%\fi
%
% %%%%%%%%%%%%%%%%%%%%%%%%%%%%%%%%%%%%%%
% \paragraph{Command Line Processing.}
%
% The following three command lines generate the output files
% |cdocscld|, |cdocscl1| and |cdocscl2|
% which should be identical to
% |cdocsdrf|, |cdocsch1| and |cdocsfn2|, respectively:
% \begin{center}
% \begin{tabular}{l}
% |latex -jobname cdocscld \|\\
% |  "\def\version{draft}\input{childdoc.def}\childdocforward{cdocsamp}"|\\
% |latex -jobname cdocscl1 \|\\
% |  "\input{childdoc.def}\childdocforward[cdocsamp]{cdocsch1}"|\\
% |latex -jobname cdocscl2 \|\\
% |  "\def\version{final}\input{childdoc.def}\childdocforward{cdocsch2}"|
% \end{tabular}
% \end{center}
% Note that the trailing backslash on each first line
% merely continues the input to the second line
% (for convenient cut ant paste).
% Furthermore, the command |latex| can be replaced by any
% of its alternative versions such as |pdflatex|.
%
% %%%%%%%%%%%%%%%%%%%%%%%%%%%%%%%%%%%%%%%%%%%%%%%%%%%%%%%%%%%%%%%%%%%%%%%%%%%%%%
% %%%%%%%%%%%%%%%%%%%%%%%%%%%%%%%%%%%%%%%%%%%%%%%%%%%%%%%%%%%%%%%%%%%%%%%%%%%%%%
% \section{Implementation}
%\iffalse
%<*package>
%\fi
%
% This section describes the definitions file |childdoc.def|.

% The definitions cannot be loaded using |\usepackage| or |\RequirePackage|
% which has a mechanism to prevent loading a style file more than once.
% When loading the definitions by means of |\input|
% multiple instances have to be prevented manually:
%\iffalse
%This code needs to be before the `\ProvidesFile' directive
%which is defined at the beginning of this file.
%Therefore it is also placed there and commented out here.
%</package>
%<*discard>
%\fi
%    \begin{macrocode}
\ifdefined\childdocmain\endinput\fi
%    \end{macrocode}
%\iffalse
%</discard>
%<*package>
%\fi
%
% \macro{\ifchilddoc}
% \macro{\ifchilddocmanual}
% The conditional |\ifchilddoc| tells whether a
% child (true) or main (false) document is being compiled.
% The conditional |\ifchilddocmanual| tells whether
% the |\includeonly| mechanism is used (false) or
% the selection of child files must be performed manually (true).
% The definitions initialise to false:
%    \begin{macrocode}
\newif\ifchilddoc
\newif\ifchilddocmanual
%    \end{macrocode}

% \macro{\childdocname}
% \macro{\childdocjob}
% The macro |\childdocname| stores the name of the main document
% to be compiled. The macro |\childdocjob| stores the name of
% the document on which the \LaTeX{} compiler was originally invoked.
% The content of |\jobname| cannot be compared
% to filenames specified in the source due to different catcodes.
% The following code rescans |\jobname|, stores the result
% in |\childdocname| and saves a copy in |\childdocjob|:
%    \begin{macrocode}
\edef\childdocname{\scantokens\expandafter{\jobname\noexpand}}
\let\childdocjob\childdocname
%    \end{macrocode}

% \macro{\childdocdisable}
% The macro |\childdocdisable| prevents the main file
% from being processed more than once.
% At this stage, the main document command |\childdocmain|
% is assumed to be called once again where it should do nothing.
% Any subsequent call to it should prevent
% a secondary processing of the main document
% It overwrites the forwarding commands
% |\childdocof| and |\childdocforward|
% with empty macros to prevent further inclusions of the main document:
%    \begin{macrocode}
\newcommand{\childdocdisable}
{
  \renewcommand{\childdocmain}[1]{\renewcommand{\childdocmain}[1]{\endinput}}
  \renewcommand{\childdocof}[1]{}
  \renewcommand{\childdocby}[2][]{}
  \renewcommand{\childdocforward}[2][]{}
  \renewcommand{\childdocdisable}{}
}
%    \end{macrocode}

% \macro{\childdocmain}
% The macro |\childdocmain| is to be called at the top of the main file
% with nothing or the main filename (without extension) as argument.
% First, it breaks loops.
% If the argument is not empty and does not match |\childdocname|
% (which is set by the first inclusion of |childdoc.def|),
% |\ifchilddoc| is set to true, |\includeonly| is applied to the child file
% and |\jobname| is set to the main file
% (for proper handling of |.aux| files):
%    \begin{macrocode}
\newcommand{\childdocmain}[1]
{
  \childdocdisable\childdocmain{}
  \if?#1?\else
    \begingroup
      \def\childdoctmp{#1}
      \ifx\childdoctmp\childdocname
        \def\childdoctmp{}
      \else
        \def\childdoctmp
        {
          \childdoctrue
          \includeonly{\childdocname}
          \def\childdocjob{#1}
          \def\jobname{#1}
        }
      \fi
      \expandafter
    \endgroup
    \childdoctmp
  \fi
}
%    \end{macrocode}

% \macro{\childdocof}
% The command |\childdocof| redirects
% compilation to the main file |#1|.
%    \begin{macrocode}
\newcommand{\childdocof}[1]
{
  \childdocdisable
  \childdoctrue
  \includeonly{\childdocname}
  \def\jobname{#1}
  \def\childdocjob{#1}
  \input{#1}
}
%    \end{macrocode}

% \macro{\childdocby}
% The command |\childdocby| ....
%    \begin{macrocode}
\newcommand{\childdocby}[2][]
{
  \childdocdisable
  \childdoctrue
  \childdocmanualtrue
  \if?#1?\else
    \def\jobname{#2}
  \fi
  \def\childdocjob{#2}
  \input{#2}
  \endinput
}
%    \end{macrocode}

% \macro{\childdocforward}
% The command |\childdocforward| redirects
% compilation to the main file or
% (if the optional argument is given) a child file.
% Parameters are set as if the main file
% or a child file starting with |\childdocof| was compiled.
% Then compilation is handed over to the main file:
%    \begin{macrocode}
\newcommand{\childdocforward}[2][]
{
  \begingroup
    \if?#1?
      \def\childdoctmp
      {
        \def\childdocname{#2}
        \def\childdocjob{#2}
        \def\jobname{#2}
        \input{#2}
        \endinput
      }
    \else
      \def\childdoctmp
      {
        \childdocdisable
        \def\childdocname{#2}
        \childdoctrue
        \includeonly{#2}
        \def\childdocjob{#1}
        \def\jobname{#1}
        \input{#1}
        \endinput
      }
    \fi
    \expandafter
  \endgroup
  \childdoctmp
}
%    \end{macrocode}

% \macro{\childdocforwardprefix}
% The command |\childdocforwardprefix| redirects
% compilation to the main or a child file by means of a pattern.
% The prefix |#1| in the current filename is replaced by |#2|
% and the suffix of the current filename is kept
% (it is assumed that the filename does not contain the substring `|~~~|'
% which is used as a delimiter).
% Compilation is handed over to the new file by |\childdocforward|:
%    \begin{macrocode}
\newcommand{\childdocforwardprefix}[3][]
{
  \begingroup
    \def\childdocextract #2##1~~~{\def\childdoctmp{\childdocforward[#1]{#3##1}}}
    \expandafter\childdocextract\childdocname~~~
    \expandafter
  \endgroup
  \childdoctmp
}
%    \end{macrocode}

% \macro{\childdoc}
% The deprecated macro |\childdoc| is a legacy version of |\childdocmain|:
%    \begin{macrocode}
\newcommand{\childdoc}{\childdocmain}
%    \end{macrocode}

% \macro{\childdocredirect}
% The deprecated macro |\childdocredirect| is a legacy version
% of |\childdocforward| and |\childdocforwardprefix|:
%    \begin{macrocode}
\newcommand{\childdocredirect}[2][]
{
  \begingroup
    \if?#1?
      \def\childdoctmp{\childdocforward{#2}}
    \else
      \def\childdoctmp{\childdocforwardprefix{#1}{#2}}
    \fi
    \expandafter
  \endgroup
  \childdoctmp
}
%    \end{macrocode}

%\iffalse
%</package>
%\fi
%
\endinput
|\\
|\childdocforward{|\textit{main}|}|
\end{tabular}
\end{center}
%
Likewise, the following files |final|\textit{nn}|.tex|
compile the final version of the child document
|child|\textit{nn}|.tex|:
%
\begin{center}
\begin{tabular}{l}
|\def\version{final}|\\
|% \iffalse
%
% childdoc.dtx Copyright (C) 2017-2018 Niklas Beisert
%
% This work may be distributed and/or modified under the
% conditions of the LaTeX Project Public License, either version 1.3
% of this license or (at your option) any later version.
% The latest version of this license is in
%   http://www.latex-project.org/lppl.txt
% and version 1.3 or later is part of all distributions of LaTeX
% version 2005/12/01 or later.
%
% This work has the LPPL maintenance status `maintained'.
%
% The Current Maintainer of this work is Niklas Beisert.
%
% This work consists of the files childdoc.dtx and childdoc.ins
% and the derived files childdoc.def and cdocsamp.tex with
% cdocsch1.tex, cdocsch2.tex, cdocsdrf.tex, cdocsfn1.tex, cdocsfn2.tex.
%
%<package>\ifdefined\childdocmain\endinput\fi
%<package>\ProvidesFile{childdoc.def}[2018/12/30 v2.0 child document driver]
%<samplemain>\ProvidesFile{cdocsamp.tex}[2018/12/30 v2.0 sample for childdoc]
%<*driver>
%\ProvidesFile{childdoc.drv}[2018/12/30 v2.0 childdoc reference manual file]
\PassOptionsToClass{10pt,a4paper}{article}
\documentclass{ltxdoc}

\usepackage[margin=35mm]{geometry}
\usepackage{hyperref}
\usepackage{hyperxmp}
\usepackage[usenames]{color}

\hypersetup{colorlinks=true}
\hypersetup{pdfstartview=FitH}
\hypersetup{pdfpagemode=UseNone}
\hypersetup{pdfsource={}}
\hypersetup{pdflang={en-UK}}
\hypersetup{pdfcopyright={Copyright 2017-2018 Niklas Beisert.
  This work may be distributed and/or modified under the
  conditions of the LaTeX Project Public License, either version 1.3
  of this license or (at your option) any later version.}}
\hypersetup{pdflicenseurl={http://www.latex-project.org/lppl.txt}}
\hypersetup{pdfcontactaddress={ETH Zurich, ITP, HIT K,
  Wolfgang-Pauli-Strasse 27}}
\hypersetup{pdfcontactpostcode={8093}}
\hypersetup{pdfcontactcity={Zurich}}
\hypersetup{pdfcontactcountry={Switzerland}}
\hypersetup{pdfcontactemail={nbeisert@itp.phys.ethz.ch}}
\hypersetup{pdfcontacturl={http://people.phys.ethz.ch/\xmptilde nbeisert/}}

\newcommand{\secref}[1]{\hyperref[#1]{section \ref*{#1}}}

\parskip1ex
\parindent0pt
\let\olditemize\itemize
\def\itemize{\olditemize\parskip0pt}

\begin{document}

\title{The \textsf{childdoc} Package}
\hypersetup{pdftitle={The childdoc Package}}
\author{Niklas Beisert\\[2ex]
  Institut f\"ur Theoretische Physik\\
  Eidgen\"ossische Technische Hochschule Z\"urich\\
  Wolfgang-Pauli-Strasse 27, 8093 Z\"urich, Switzerland\\[1ex]
  \href{mailto:nbeisert@itp.phys.ethz.ch}
  {\texttt{nbeisert@itp.phys.ethz.ch}}}
\hypersetup{pdfauthor={Niklas Beisert}}
\hypersetup{pdfsubject={Manual for the LaTeX2e Package childdoc}}
\date{30 December 2018, \textsf{v2.0}}
\maketitle

\begin{abstract}\noindent
\textsf{childdoc} is a \LaTeXe{} package
that enables the direct compilation
of document sections included by |\include|
to individual files.
\end{abstract}

\begingroup
\parskip0ex
\tableofcontents
\endgroup

%%%%%%%%%%%%%%%%%%%%%%%%%%%%%%%%%%%%%%%%%%%%%%%%%%%%%%%%%%%%%%%%%%%%%%%%%%%%%%%%
%%%%%%%%%%%%%%%%%%%%%%%%%%%%%%%%%%%%%%%%%%%%%%%%%%%%%%%%%%%%%%%%%%%%%%%%%%%%%%%%
\section{Introduction}

\LaTeX{} provides a mechanism to structure a large document (such as a book)
into a main file and several child files (containing the chapters)
using the |\include| command.
This mechanism is beneficial for documents
which span hundreds of pages in order to
make the source file(s) more manageable.
Moreover, compilation can be restricted to
selected child files by means of the |\includeonly| command.
The latter feature can be used to reduce the compilation time while editing
(this was significantly more useful in the earlier days of \LaTeX{})
or to generate a smaller document which is easier to navigate.
Another application of |\includeonly| is to generate
documents consisting of selected parts of the complete document.

However, there are a few drawbacks of the plain |\include| mechanism:
\begin{itemize}
\item
The child files cannot be compiled on their own,
they can only be compiled via the main file.
A naive editing environment
(such as a text editor with an option
to have the current file processed by \LaTeX)
may require one to switch to the main file before compiling;
attempting to compile the child file produces errors.
\item
The main file must be modified (each time)
to adjust the |\includeonly| command
to the present needs. This easily leaves the main file in a messy state.
\item
The generated document will always carry the filename
of the main document. This is inconvenient if
several child files are to be compiled and
to be kept for distribution.
\end{itemize}

The present package provides a simple interface
to make child files individually compilable by \LaTeX{}.
Compiling a child file then has the same effect as compiling
the main file with an |\includeonly| command
to select the appropriate child.
Moreover the generated document will carry the name of the child
rather than the main file.
This resolves all three above issues.

This feature is meant to make the editing of books,
thesis documents and lecture notes somewhat more convenient.
However, the package can also be used efficiently for
composing a series of documents (such as exercise sheets)
which are typically distributed individually.
It then assists the author in generating the individual documents
(potentially in different versions)
as well as a document containing the collected series.
Another application is in developing style files
or other kinds of included material
where compilation of the style file could redirect
to a sample or test file.

%%%%%%%%%%%%%%%%%%%%%%%%%%%%%%%%%%%%%%%%%%%%%%%%%%%%%%%%%%%%%%%%%%%%%%%%%%%%%%%%
%%%%%%%%%%%%%%%%%%%%%%%%%%%%%%%%%%%%%%%%%%%%%%%%%%%%%%%%%%%%%%%%%%%%%%%%%%%%%%%%
\section{Usage}

First of all, the package \textsf{childdoc} is \emph{not} a standard
\LaTeXe{} |.sty| style file! Therefore it needs to be invoked in
a non-standard way.

%%%%%%%%%%%%%%%%%%%%%%%%%%%%%%%%%%%%%%%%%%%%%%%%%%%%%%%%%%%%%%%%%%%%%%%%%%%%%%%%
\subsection{Included Files}
\label{sec:include}

%%%%%%%%%%%%%%%%%%%%%%%%%%%%%%%%%%%%%%%%
\DescribeMacro{\childdocmain}
To use the package, add the commands
\begin{center}
\begin{tabular}{l}
|\input{childdoc.def}|\\
|\childdocmain{}|\\
\end{tabular}
\end{center}
at the very top of the main \LaTeX{} file,
in particular \emph{before} the |\documentclass| statement!
The argument of |\childdocmain| should be left empty
(but it must be present).

%%%%%%%%%%%%%%%%%%%%%%%%%%%%%%%%%%%%%%%%
\DescribeMacro{\childdocof}
Furthermore, add the commands
\begin{center}
\begin{tabular}{l}
|\input{childdoc.def}|\\
|\childdocof{|\textit{main}|}|\\
\end{tabular}
\end{center}
at the top of every child file \textit{child}
which is included by |\include{|\textit{child}|}|
from within the main file
(or at least for those files to be compiled individually).
The argument \textit{main} must be the filename of the main file.

There are a couple of
considerations in setting up the main and child documents:

%%%%%%%%%%%%%%%%%%%%%%%%%%%%%%%%%%%%%%%%
\paragraph{Restrictions.}

Please note the following restrictions:
\begin{itemize}
\item
|\childdocmain| must be called with one argument \textit{main}
to ensure compatibility with earlier version of the package.
It must either be empty (|\childdocmain{}|)
or precisely match the filename of the main file in which it is specified.
See \secref{sec:detection} for further information.
\item
The filename \textit{main} must be specified without the |.tex| extension.
\item
The filename \textit{main} is case sensitive
(even in case-insensitive file systems)
due to internal string comparison.
\item
The argument \textit{main} should be fully expanded, it cannot be a macro.
\item
Subdirectories and special characters should be avoided in filenames.
\item
The command |\childdocmain{|\textit{main}|}| must be followed by a whitespace.
It should not be followed immediately by another command
or by a comment mark `|%|'.
This is because the \TeX{} parser reads the token immediately following
the argument of |\childdocmain| and puts it
at the beginning of every child section;
however, a white\-space is ignored.
\end{itemize}

%%%%%%%%%%%%%%%%%%%%%%%%%%%%%%%%%%%%%%%%
\paragraph{Content of Main File.}

It is advisable to place all content in the child files included by |\include|.
Any output contained in the main file will appear in all child documents
unless suppressed manually;
it cannot be suppressed automatically by the |\includeonly| directive
and thus should normally be avoided.
A method to include some content in the main file
by means of conditional processing is described in \secref{sec:conditional}.

%%%%%%%%%%%%%%%%%%%%%%%%%%%%%%%%%%%%%%%%
\paragraph{Page Numbering.}

When only a part of the document is compiled,
the appropriate numbering of pages
(as well as other status parameters)
is determined from the |.aux| files.
The latter contain information from previous passes.
However this information needs to propagate through
all intermediate child documents.
Therefore the page numbering in child documents may well
be inconsistent until the complete document is compiled at least once.

A useful (if unconventional) way to always ensure a consistent
page numbering is to restart the numbering in each child document
and denote the pages by `\textit{child}|.|\textit{page}'
where \textit{child} represents the chapter/section number of the child file.
This can be achieved by the command
|\numberwithin{page}{|\textit{child}|}|
of the \textsf{amsmath} package
where \textit{child} can be |chapter| or |section|
depending on the chosen structuring.
Alternatively, one can modify the macro |\thepage| appropriately
and reset the counter |page| at the start of each child file.

%%%%%%%%%%%%%%%%%%%%%%%%%%%%%%%%%%%%%%%%%%%%%%%%%%%%%%%%%%%%%%%%%%%%%%%%%%%%%%%%
\subsection{Conditional Processing}
\label{sec:conditional}

The package provides a mechanism to compile different versions
of a document. To customise the versions further some conditional processing
can come in handy to distinguish which version is being compiled.
The package provides two macros to describe the compilation context:

%%%%%%%%%%%%%%%%%%%%%%%%%%%%%%%%%%%%%%%%
\DescribeMacro{\ifchilddoc}
The conditional |\ifchilddoc| distinguishes between the compilation of
child documents and the main document:
%
\begin{center}
|\ifchilddoc |\textit{child-code}| |[|\||else |\textit{main-code}]| \||fi|
\end{center}

%%%%%%%%%%%%%%%%%%%%%%%%%%%%%%%%%%%%%%%%
\DescribeMacro{\childdocname}
\DescribeMacro{\childdocjob}
The macro |\childdocname| contains the filename (without extension)
of the main or child file being processed.
Note that |\childdocjob| will always contain the name of the main file.

%%%%%%%%%%%%%%%%%%%%%%%%%%%%%%%%%%%%%%%%
\paragraph{Title Page.}

Conditional processing can be used to include a title or banner page
in the main document when proper precautions are taken.
Importantly, the code in the main file should ensure that the page counter
(as well as other status parameters which are stored in the |.aux| files)
takes the same value after the conditional processing.
Otherwise the page numbers may take divergent values
depending on which part is compiled.

For example, a title page could be declared by:
%
\begin{center}
\begin{tabular}{l}
|\ifchilddoc\||else|\\
|\addtocounter{page}{-1}|\\
\textit{code for title page}\\
|\newpage|\\
|\||fi|
\end{tabular}
\end{center}
%
A banner page for the child documents can be generated by:
%
\begin{center}
\begin{tabular}{l}
|\ifchilddoc|\\
|\addtocounter{page}{-1}|\\
\textit{code for banner page}\\
|\newpage|\\
|\||fi|
\end{tabular}
\end{center}
%
Here one could write a message such as:
\begin{center}
|This is the part \childdocname{} of \childdocjob{}.|
\end{center}

%%%%%%%%%%%%%%%%%%%%%%%%%%%%%%%%%%%%%%%%%%%%%%%%%%%%%%%%%%%%%%%%%%%%%%%%%%%%%%%%
\subsection{Flags}
\label{sec:flags}

The package makes it easy to generate different versions
of the main or child documents.
To this end compilation flags can be defined
and assigned different default values.
They will be particularly useful in conjunction
with the forwarding mechanism described in \secref{sec:forward}.

For example, it may be useful to have a flag |\version|
which can be set to |draft| or |final|.
The document source will contain some conditional code
depending on the value of |\version|.
Suppose further, the flag should default to |final| for the main file
and to |draft| for child files
which is a natural assignment for editing the document.
This is achieved by placing the following code
in the preamble of the main document
(below the |\childdocmain| directive):
%
\begin{center}
\begin{tabular}{l}
|\ifchilddoc|\\
|\providecommand{\version}{draft}|\\
|\||else|\\
|\providecommand{\version}{final}|\\
|\||fi|
\end{tabular}
\end{center}
%
The definition by |\providecommand| makes sure
that previous definitions are not overwritten.
Further statements |\providecommand{\version}{...}|
can thus be added before the above code to override it.

For the main file, one might add a line
(between |\childdocmain| and the above block)
%
\begin{center}
|%\ifchilddoc\||else\providecommand{\version}{draft}\||fi|
\end{center}
%
which can be uncommented to produce a draft version.
Likewise one can add a line to the very top of a child file
(above the |\childdocof{|\textit{main}|}| directive)
%
\begin{center}
|%\providecommand{\version}{final}|
\end{center}
%
which can be uncommented to produce the final version of this child document.

%%%%%%%%%%%%%%%%%%%%%%%%%%%%%%%%%%%%%%%%%%%%%%%%%%%%%%%%%%%%%%%%%%%%%%%%%%%%%%%%
\subsection{Forwarding}
\label{sec:forward}

Different versions of the main or child documents
using compilation flags as described in \secref{sec:flags}
can be (permanently) stored in different files
for convenient compilation, viewing and distribution.
To this end, the package defines a command
to pass on compilation to a different file:

%%%%%%%%%%%%%%%%%%%%%%%%%%%%%%%%%%%%%%%%
\DescribeMacro{\childdocforward}
The command |\childdocforward| redirects processing to
another source file:
%
\begin{center}
\begin{tabular}{l}
|\input{childdoc.def}|\\
|\childdocforward[|\textit{main}|]{|\textit{dest}|}|\\
\end{tabular}
\end{center}
%
The argument \textit{dest} is the destination file
(without extension).
It should be the main file or one of the child files.
Note that further \textsf{childdoc} directives
such as |\childdocof| and |\childdocforward|
in the indicated file will be processed in this form.
The optional argument \textit{main}
passes on directly to the main file \textit{main}
while pretending to compile the child \textit{dest}.
This form behaves as if \textit{dest}
issues |\childdocof{|\textit{main}|}| right away,
and no further \textsf{childdoc} directives will be processed.

%%%%%%%%%%%%%%%%%%%%%%%%%%%%%%%%%%%%%%%%
\DescribeMacro{\...prefix}
In the alternative form |\childdocforwardprefix|,
%
\begin{center}
\begin{tabular}{l}
|\input{childdoc.def}|\\
|\childdocforwardprefix[|\textit{main}|]{|\textit{prefix}|}{|\textit{dest}|}|
\end{tabular}
\end{center}
%
the destination file is determined by a pattern
depending on the current file:
To make this work, the current file must be called
`{\textit{prefix}\hspace{0.2em}\textit{suffix}}'
with \textit{prefix} matching precisely the argument.
Processing is then passed on to the file
`{\textit{dest}\hspace{0.2em}\textit{suffix}}'.
Surely, the same effect is achieved by
directly specifying the
argument `{\textit{dest}\hspace{0.2em}\textit{suffix}}'
in the first form.
However, that requires to set up a different file
for each child. With the alternative form of the command
all these files can have exactly the same content
which simplifies setting them up and maintaining them.

For example, the following file |draft.tex|
with a compilation flag |\version| as described in \secref{sec:flags}
compiles the main document as a draft:
%
\begin{center}
\begin{tabular}{l}
|\def\version{draft}|\\
|\input{childdoc.def}|\\
|\childdocforward{|\textit{main}|}|
\end{tabular}
\end{center}
%
Likewise, the following files |final|\textit{nn}|.tex|
compile the final version of the child document
|child|\textit{nn}|.tex|:
%
\begin{center}
\begin{tabular}{l}
|\def\version{final}|\\
|\input{childdoc.def}|\\
|\childdocforwardprefix{final}{child}|
\end{tabular}
\end{center}
%

Note that when several versions of a main file and/or of each child file
are to be generated, it may be convenient to set up a |Makefile| or
shell script to automatise the process.

%%%%%%%%%%%%%%%%%%%%%%%%%%%%%%%%%%%%%%%%%%%%%%%%%%%%%%%%%%%%%%%%%%%%%%%%%%%%%%%%
\subsection{Command Line Processing}
\label{sec:commandline}

The effect of redirection files can also be achieved by invoking
the \LaTeX{} compiler with a more elaborate command line.
Most conveniently this should be done as part
of a shell script or a |Makefile|.

When using \textsf{childdoc} in the main file, the following
command lines effectively perform a redirection
(note that depending on the shell being used,
backslashes may have to be doubled: `|\|' $\to$ `|\\|'):
%
\begin{center}
|... -jobname "|\textit{target}|" |\\|"|[\textit{flags}]%
|\input{childdoc.def}\childdocforward[|\textit{main}|]{|\textit{dest}|}"|
\end{center}
%
Here \textit{target} is the name of the output file,
\textit{main} is the name of the main file
and \textit{dest} is the name of the main or child file to be processed
(all filenames without extensions).
The optional argument \textit{main} can be omitted
if \textit{main} matches \textit{dest}.
Optionally, compilation \textit{flags} can be defined via |\def| commands.
This command line makes the \TeX{} engine believe
it is compiling the file \textit{target}
whose content is specified as the latter parameter.
The provided code then forwards the processing to
\textit{main} or \textit{dest} as described in \secref{sec:forward}.

%%%%%%%%%%%%%%%%%%%%%%%%%%%%%%%%%%%%%%%%%%%%%%%%%%%%%%%%%%%%%%%%%%%%%%%%%%%%%%%%
\subsection{Include by Input}
\label{sec:input}

Including child documents by |\include| has some restrictions by design.
Most notably, the content of a child document always occupies
its own set of pages; pages cannot be shared between child documents.
Usually, this behaviour makes perfect sense
because each child document contain an essential part of the document.
However, in some situations it may be desirable to compose
a document from a collection of parts
without having mandatory page breaks between then.
For this case, the package
provides a mechanism to include parts
by |\input| which can also be processed individually.
However, by construction this mechanism
requires manual handling of the content to be output.

%%%%%%%%%%%%%%%%%%%%%%%%%%%%%%%%%%%%%%%%
\DescribeMacro{\ifchilddocmanual}
The main file should be prepared as usual, see \secref{sec:include}.
However, the document body must make a distinction
between processing of an individual part and of the main document, e.g.:
%
\begin{center}
\begin{tabular}{l}
|\ifchilddocmanual|\\
|\input{\childdocname}|\\
|\||else|\\
\textit{document body with }|\input{|\textit{part}|}|\\
|\||fi|
\end{tabular}
\end{center}
%
The conditional |\ifchilddocmanual| is true whenever
a part to be included by |\input| is being compiled,
and the name of the part is stored in |\childdocname|.

%%%%%%%%%%%%%%%%%%%%%%%%%%%%%%%%%%%%%%%%
\DescribeMacro{\childdocby}
Each part to be included by |\input| should start with:
%
\begin{center}
\begin{tabular}{l}
|\input{childdoc.def}|\\
|\childdocby{|\textit{main}|}|\\
\end{tabular}
\end{center}
%
The directive |\childdocby| is similar to |\childdocof|
described in \secref{sec:include},
but the subsequent selection of content must be done manually.
To that end, both |\ifchilddoc| and |\ifchilddocmanual|
will be true upon processing of a part,
and the name of the part is stored in |\childdocname|.
Note that |\jobname| will be set to the filename of the current part
so that each part receives an individual |.aux| file
that does not interfere with the |.aux| file(s) of the main document.
This behaviour can be altered by the alternative form
|\childdocby[*]{|\textit{main}|}| (with a non-empty optional argument)
which uses the |.aux| file of the main document
by setting |\jobname| to \textit{main}.

%%%%%%%%%%%%%%%%%%%%%%%%%%%%%%%%%%%%%%%%%%%%%%%%%%%%%%%%%%%%%%%%%%%%%%%%%%%%%%%%
\subsection{Driver Development}
\label{sec:driver}

The \textsf{childdoc} mechanism can also be use for the development
of definition files such as \LaTeX{} styles or classes.
This case differs from the above setup with multiple parts
included by |\include| in that no |\includeonly| should be invoked.
This can be achieved by starting the include file
(before |\ProvidesPackage|) with:
%
\begin{center}
\begin{tabular}{l}
|\input{childdoc.def}|\\
|\childdocforward{|\textit{main}|}|\\
\end{tabular}
\end{center}
%
or alternatively with:
%
\begin{center}
\begin{tabular}{l}
|\input{childdoc.def}|\\
|\childdocby{|\textit{main}|}|\\
\end{tabular}
\end{center}
%
Both forms have slightly different effects as described above.
The main file is prepared as usual, see \secref{sec:include}.

%%%%%%%%%%%%%%%%%%%%%%%%%%%%%%%%%%%%%%%%%%%%%%%%%%%%%%%%%%%%%%%%%%%%%%%%%%%%%%%%
\subsection{Legacy Detection}
\label{sec:detection}

The directive |\childdocmain| in the main file can detect
whether the complete document or merely a child is to be compiled
even without using the directive |\childdocof|.
This method is deprecated because it is less robust
and there is no compelling reason to use it;
it is merely provided for backward compatibility
and it may be removed in future versions.

If the detection mechanism is to be used,
it is mandatory to correctly specify
the filename of the main file as the argument of |\childdocmain|:
%
\begin{center}
\begin{tabular}{l}
|\input{childdoc.def}|\\
|\childdocmain{|\textit{main}|}|\\
\end{tabular}
\end{center}
%
If |\jobname| does not match the argument \textit{main} of |\childdocmain|,
it is assumed that |\jobname| points to the child file to be compiled.
When using |\childdocmain| with the main file specified as argument,
it suffices to start a child file
with just |\input{|\textit{main}|}|
without loading of the package and using |\childdocof|.
If instead all processing is done
with the appropriate \textsf{childdoc} directives,
the argument of \textit{main} of |\childdocmain| can be empty.

An alternative version of the command line processing described
in \secref{sec:commandline} using the detection mechanism reads:
%
\begin{center}
|... -jobname "|\textit{target}|" "|[\textit{flags}]%
[|\def\jobname{|\textit{dest}|}|]|\input{|\textit{main}|}"|
\end{center}

%%%%%%%%%%%%%%%%%%%%%%%%%%%%%%%%%%%%%%%%%%%%%%%%%%%%%%%%%%%%%%%%%%%%%%%%%%%%%%%%
\subsection{Manual Code}
\label{sec:manual}

In case one cannot be certain whether the definitions file |childdoc.def|
is installed on the target \TeX{} distribution
and one prefers not to ship it,
it is conceivable to paste a few relevant commands into the sources.

To that end, drop all statements |\input{childdoc.def}|
and perform the replacements as outlined below.
Instead of |\childdocmain{|\textit{main}|}| add the following code
to the top of the main file:
%
\begin{center}
\begin{tabular}{l}
|\||ifdefined\childdocname\endinput\||fi\newif\ifchilddoc|\\
|\edef\childdocname{\scantokens\expandafter{\jobname\noexpand}}|\\
|\def\childdocmain{|\textit{main}|}\||ifx\childdocmain\childdocname\||else|\\
|\childdoctrue\includeonly{\childdocname}\let\jobname\childdocmain\||fi|\\
\end{tabular}
\end{center}
%
Instead of |\childdocof{|\textit{main}|}| just include the main file
at the top of each child file:
%
\begin{center}
|\input{|\textit{main}|}|
\end{center}
%
A simple redirection |\childdocforward{|\textit{dest}|}| is achieved by:
%
\begin{center}
|\def\jobname{|\textit{dest}|}\input{\jobname}|
\end{center}
%
The redirection with prefix
|\childdocforwardprefix[|\textit{prefix}|]{|\textit{dest}|}|
is accomplished by:
%
\begin{center}
\begin{tabular}{l}
|{\edef\jobname{\scantokens\expandafter{\jobname\noexpand}}|\\
|\def\redirectjob |\textit{prefix}|#1~~~{\gdef\jobname{|\textit{dest}|#1}}|\\
|\expandafter\redirectjob\jobname~~~}\input{\jobname}|
\end{tabular}
\end{center}

In an alternative approach,
child documents can be compiled by a specific command line
without additional code or specific definitions:
%
\begin{center}
|... -jobname "|\textit{target}|" "|[\textit{flags}]%
|\includeonly{|\textit{dest}|}\input{|\textit{main}|}"|
\end{center}
%

%%%%%%%%%%%%%%%%%%%%%%%%%%%%%%%%%%%%%%%%%%%%%%%%%%%%%%%%%%%%%%%%%%%%%%%%%%%%%%%%
%%%%%%%%%%%%%%%%%%%%%%%%%%%%%%%%%%%%%%%%%%%%%%%%%%%%%%%%%%%%%%%%%%%%%%%%%%%%%%%%
\section{Information}

%%%%%%%%%%%%%%%%%%%%%%%%%%%%%%%%%%%%%%%%%%%%%%%%%%%%%%%%%%%%%%%%%%%%%%%%%%%%%%%%
\subsection{Copyright}

Copyright \copyright{} 2017--2018 Niklas Beisert

This work may be distributed and/or modified under the
conditions of the \LaTeX{} Project Public License, either version 1.3
of this license or (at your option) any later version.
The latest version of this license is in
  \url{http://www.latex-project.org/lppl.txt}
and version 1.3 or later is part of all distributions of \LaTeX{}
version 2005/12/01 or later.

This work has the LPPL maintenance status `maintained'.

The Current Maintainer of this work is Niklas Beisert.

This work consists of the files |README.txt|, |childdoc.ins| and |childdoc.dtx|
as well as the derived files |childdoc.def|, |cdocsamp.tex|
with |cdocsch1.tex|, |cdocsch2.tex|, |cdocspt3.tex|, |cdocspt4.tex|,
|cdocsdrf.tex|, |cdocsfn1.tex|, |cdocsfn2.tex|
as well as |childdoc.pdf|.

%%%%%%%%%%%%%%%%%%%%%%%%%%%%%%%%%%%%%%%%%%%%%%%%%%%%%%%%%%%%%%%%%%%%%%%%%%%%%%%%
\subsection{Files and Installation}

The package consists of the files:
%
\begin{center}
\begin{tabular}{ll}
    |README.txt|   & readme file \\
    |childdoc.ins| & installation file \\
    |childdoc.dtx| & source file \\
    |childdoc.def| & definition file \\
    |cdocsamp.tex| & sample main file \\
    |cdocsch1.tex| & sample include file \\
    |cdocsch2.tex| & sample include file \\
    |cdocspt3.tex| & sample part file \\
    |cdocspt4.tex| & sample part file \\
    |cdocsdrf.tex| & sample redirection file \\
    |cdocsfn1.tex| & sample redirection file \\
    |cdocsfn2.tex| & sample redirection file \\
    |childdoc.pdf| & manual
\end{tabular}
\end{center}
%
The distribution consists of the files
|README.txt|, |childdoc.ins| and |childdoc.dtx|.
%
\begin{itemize}
\item
Run (pdf)\LaTeX{} on |childdoc.dtx|
to compile the manual |childdoc.pdf| (this file).
\item
Run \LaTeX{} on |childdoc.ins| to create the definitions file |childdoc.def|
and the sample |cdocsamp.tex| with include files
|cdocsch1.tex|, |cdocsch2.tex|, |cdocspt3.tex|, |cdocspt4.tex|,
|cdocsdrf.tex|, |cdocsfn1.tex|, |cdocsfn2.tex|.
Then copy the file |childdoc.def| to an appropriate directory of your \LaTeX{}
distribution, e.g.\ \textit{texmf-root}|/tex/latex/childdoc|.
\end{itemize}

%%%%%%%%%%%%%%%%%%%%%%%%%%%%%%%%%%%%%%%%%%%%%%%%%%%%%%%%%%%%%%%%%%%%%%%%%%%%%%%%
\subsection{Related CTAN Packages}

There are several other packages which offer a similar functionality:
%
\begin{itemize}
\item
The packages
\href{http://ctan.org/pkg/docmute}{\textsf{docmute}},
\href{http://ctan.org/pkg/includex}{\textsf{includex}} and
\href{http://ctan.org/pkg/standalone}{\textsf{standalone}}
provide commands to include only the document body of
a child file thus allowing both files to be compiled individually.
\item
The packages \href{http://ctan.org/pkg/subdocs}{\textsf{subdocs}}
and \href{http://ctan.org/pkg/subfiles}{\textsf{subfiles}}
provide structures in which the main and child documents can be
encapsulated and allowing them to be compiled individually.
The inclusion mechanism is different from the conventional |\include|.
\item
The package \href{http://ctan.org/pkg/combine}{\textsf{combine}}
is an elaborate solution to combine several documents into one.
\end{itemize}
%
See also the CTAN topic \href{http://ctan.org/topic/subdocs}{\textsf{subdocs}}
for further related packages.
The present package differs from the above solutions in that
a document structure constructed with the conventional |\include| mechanism
just needs two extra commands at the top of every file
such that all constituent files can be compiled individually.

%%%%%%%%%%%%%%%%%%%%%%%%%%%%%%%%%%%%%%%%%%%%%%%%%%%%%%%%%%%%%%%%%%%%%%%%%%%%%%%%
%\subsection{Feature Suggestions}
%
%The following is a list of features which may be useful for future
%versions of this package:
%%
%\begin{itemize}
%\item
%\ldots
%\end{itemize}

%%%%%%%%%%%%%%%%%%%%%%%%%%%%%%%%%%%%%%%%%%%%%%%%%%%%%%%%%%%%%%%%%%%%%%%%%%%%%%%%
\subsection{Revision History}

%%%%%%%%%%%%%%%%%%%%%%%%%%%%%%%%%%%%%%%%
\paragraph{v2.0:} 2018/12/30

\begin{itemize}
\item
immediate forward processing
\item
added |\childdocby| mechanism
\item
manual restructured
\end{itemize}

%%%%%%%%%%%%%%%%%%%%%%%%%%%%%%%%%%%%%%%%
\paragraph{v1.6:} 2018/01/17

\begin{itemize}
\item
application for development of include files
\item
corrections to manual
\end{itemize}

%%%%%%%%%%%%%%%%%%%%%%%%%%%%%%%%%%%%%%%%
\paragraph{v1.5:} 2017/05/21

\begin{itemize}
\item
more complete structuring introduced
\item
|\childdocof| introduced
\item
|\childdoc| renamed to |\childdocmain|
\item
|\childredirect| renamed to |\childdocforward| and |\childdocforwardprefix|
and functionality expanded
\end{itemize}

%%%%%%%%%%%%%%%%%%%%%%%%%%%%%%%%%%%%%%%%
\paragraph{v1.0:} 2017/04/27

\begin{itemize}
\item
manual and install package
\item
first version published on CTAN
\end{itemize}

%%%%%%%%%%%%%%%%%%%%%%%%%%%%%%%%%%%%%%%%
\paragraph{v0.6:} 2017/04/26

\begin{itemize}
\item
redirection mechanism added
\end{itemize}

%%%%%%%%%%%%%%%%%%%%%%%%%%%%%%%%%%%%%%%%
\paragraph{v0.5:} 2017/04/26

\begin{itemize}
\item
functionality in definition file
\end{itemize}


%%%%%%%%%%%%%%%%%%%%%%%%%%%%%%%%%%%%%%%%%%%%%%%%%%%%%%%%%%%%%%%%%%%%%%%%%%%%%%%%
%%%%%%%%%%%%%%%%%%%%%%%%%%%%%%%%%%%%%%%%%%%%%%%%%%%%%%%%%%%%%%%%%%%%%%%%%%%%%%%%
%%%%%%%%%%%%%%%%%%%%%%%%%%%%%%%%%%%%%%%%%%%%%%%%%%%%%%%%%%%%%%%%%%%%%%%%%%%%%%%%
\appendix

\settowidth\MacroIndent{\rmfamily\scriptsize 000\ }

 \DocInput{childdoc.dtx}

\end{document}
%</driver>
% \fi
%
% %%%%%%%%%%%%%%%%%%%%%%%%%%%%%%%%%%%%%%%%%%%%%%%%%%%%%%%%%%%%%%%%%%%%%%%%%%%%%%
% %%%%%%%%%%%%%%%%%%%%%%%%%%%%%%%%%%%%%%%%%%%%%%%%%%%%%%%%%%%%%%%%%%%%%%%%%%%%%%
% \section{Sample}
%\iffalse
%<*samplemain>
%\fi
%
% The following presents a sample document
% with two chapters, two parts, a title page,
% a compile flag as well as three forwarding files to set the flag.
% It consists of eight |.tex| files:
% \begin{center}
% \begin{tabular}{ll}
% |cdocsamp.tex|&main file\\
% |cdocsch1.tex|&include file for chapter 1\\
% |cdocsch2.tex|&include file for chapter 2\\
% |cdocspt3.tex|&include file for part 3\\
% |cdocspt4.tex|&include file for part 4\\
% |cdocsdrf.tex|&forwarding file for main file in draft mode\\
% |cdocsfi1.tex|&forwarding file for final version of chapter 1\\
% |cdocsfi2.tex|&forwarding file for final version of chapter 2\\
% \end{tabular}
% \end{center}
% Each of the eight files can be compiled directly by the \LaTeX{} compiler.
%
% %%%%%%%%%%%%%%%%%%%%%%%%%%%%%%%%%%%%%%
% \paragraph{Main File.}
%
% The main file is called |cdocsamp.tex|.
%
% Load the \textsf{childdoc} definitions and
% declare the filename for the main document:
%    \begin{macrocode}
\input{childdoc.def}
\childdocmain{}
%    \end{macrocode}

% Optional override for |\version| flag:
%    \begin{macrocode}
%%\ifchilddoc\else\providecommand{\version}{draft}\fi
%    \end{macrocode}

% Define the default values for the |\version| flag
% (|final| for the main file and |draft| for childs):
%    \begin{macrocode}
\ifchilddoc
\providecommand{\version}{draft}
\else
\providecommand{\version}{final}
\fi
%    \end{macrocode}

% Load the standard document class:
%    \begin{macrocode}
\documentclass[12pt]{article}
%    \end{macrocode}

% Start the document body:
%    \begin{macrocode}
\begin{document}
%    \end{macrocode}

% Declare a title page.
% Print title, part of document being processed and version flag:
%    \begin{macrocode}
\addtocounter{page}{-1}
\begin{center}
{\LARGE\bfseries{}childdoc example\par}
\vspace{1cm}
\ifchilddoc
\ifchilddocmanual part\else chapter\fi:
`\childdocname' of `\childdocjob'\par
\else
main document: `\childdocjob'\par
\fi
version: \version\par
\end{center}
\newpage
%    \end{macrocode}

% Manually include selected file,
% otherwise process as usual:
%    \begin{macrocode}
\ifchilddocmanual
\section*{part `\childdocname'}
\input{\childdocname}
\else
%    \end{macrocode}

% Include the two chapters:
%    \begin{macrocode}
\include{cdocsch1}
\include{cdocsch2}
%    \end{macrocode}

% Include the two parts unless only chapters should be displayed:
%    \begin{macrocode}
\ifchilddoc\else
\section{part three}
\input{cdocspt3}
\section{part four}
\input{cdocspt4}
\fi
%    \end{macrocode}

% Process as usual until here:
%    \begin{macrocode}
\fi
%    \end{macrocode}

% End of document body:
%    \begin{macrocode}
\end{document}
%    \end{macrocode}
%\iffalse
%</samplemain>
%\fi
%
% %%%%%%%%%%%%%%%%%%%%%%%%%%%%%%%%%%%%%%
% \paragraph{Chapter Include Files.}
%
% The include files are called |cdocsch1.tex| and |cdocsch2.tex|.
%
%\iffalse
%<*samplechap1|samplechap2>
%\fi

% Optional override for |\version| flag:
%    \begin{macrocode}
%%\providecommand{\version}{final}
%    \end{macrocode}

% Include the main document:
%    \begin{macrocode}
\input{childdoc.def}
\childdocof{cdocsamp}
%    \end{macrocode}

%\iffalse
%</samplechap1|samplechap2>
%\fi
%
%\iffalse
%<*samplechap1>
%\fi
% Some text for chapter 1:
%    \begin{macrocode}
\section{one}
some text in chapter one
%    \end{macrocode}

%\iffalse
%</samplechap1>
%\fi
% Some text for chapter 2:
%\iffalse
%<*samplechap2>
%\fi
%    \begin{macrocode}
\section{two}
more text in chapter two
%    \end{macrocode}

%\iffalse
%</samplechap2>
%\fi
%
% %%%%%%%%%%%%%%%%%%%%%%%%%%%%%%%%%%%%%%
% \paragraph{Part Include Files.}
%
% The include files are called |cdocspt3.tex| and |cdocspt4.tex|.
%
%\iffalse
%<*samplepart3|samplepart4>
%\fi

% Optional override for |\version| flag:
%    \begin{macrocode}
%%\providecommand{\version}{final}
%    \end{macrocode}

% Include the main document:
%    \begin{macrocode}
\input{childdoc.def}
\childdocby{cdocsamp}
%    \end{macrocode}

%\iffalse
%</samplepart3|samplepart4>
%\fi
%
%\iffalse
%<*samplepart3>
%\fi
% Some text for part 3:
%    \begin{macrocode}
some text in part three
%    \end{macrocode}

%\iffalse
%</samplepart3>
%\fi
% Some text for part 4:
%\iffalse
%<*samplepart4>
%\fi
%    \begin{macrocode}
more text in part four
%    \end{macrocode}

%\iffalse
%</samplepart4>
%\fi
%
% %%%%%%%%%%%%%%%%%%%%%%%%%%%%%%%%%%%%%%
% \paragraph{Forwarding for a Complete Draft.}
%
% The following forwarding file |cdocsdrf.tex|
% compiles the main document in draft mode:
%\iffalse
%<*sampledraft>
%\fi
%    \begin{macrocode}
\def\version{draft}
\input{childdoc.def}
\childdocforward{cdocsamp}
%    \end{macrocode}

%\iffalse
%</sampledraft>
%\fi
%
% %%%%%%%%%%%%%%%%%%%%%%%%%%%%%%%%%%%%%%
% \paragraph{Forwarding for Final Version of the Chapters.}
%
% The following forwarding files |cdocsfn1.tex| and |cdocsfn2.tex|
% (with identical content)
% compile the final versions of the child documents
% |cdocsch1.tex| and |cdocsch2.tex|, respectively:
%\iffalse
%<*samplefinal>
%\fi
%    \begin{macrocode}
\def\version{final}
\input{childdoc.def}
\childdocforwardprefix[cdocsamp]{cdocsfn}{cdocsch}
%    \end{macrocode}

%\iffalse
%</samplefinal>
%\fi
%
% %%%%%%%%%%%%%%%%%%%%%%%%%%%%%%%%%%%%%%
% \paragraph{Command Line Processing.}
%
% The following three command lines generate the output files
% |cdocscld|, |cdocscl1| and |cdocscl2|
% which should be identical to
% |cdocsdrf|, |cdocsch1| and |cdocsfn2|, respectively:
% \begin{center}
% \begin{tabular}{l}
% |latex -jobname cdocscld \|\\
% |  "\def\version{draft}\input{childdoc.def}\childdocforward{cdocsamp}"|\\
% |latex -jobname cdocscl1 \|\\
% |  "\input{childdoc.def}\childdocforward[cdocsamp]{cdocsch1}"|\\
% |latex -jobname cdocscl2 \|\\
% |  "\def\version{final}\input{childdoc.def}\childdocforward{cdocsch2}"|
% \end{tabular}
% \end{center}
% Note that the trailing backslash on each first line
% merely continues the input to the second line
% (for convenient cut ant paste).
% Furthermore, the command |latex| can be replaced by any
% of its alternative versions such as |pdflatex|.
%
% %%%%%%%%%%%%%%%%%%%%%%%%%%%%%%%%%%%%%%%%%%%%%%%%%%%%%%%%%%%%%%%%%%%%%%%%%%%%%%
% %%%%%%%%%%%%%%%%%%%%%%%%%%%%%%%%%%%%%%%%%%%%%%%%%%%%%%%%%%%%%%%%%%%%%%%%%%%%%%
% \section{Implementation}
%\iffalse
%<*package>
%\fi
%
% This section describes the definitions file |childdoc.def|.

% The definitions cannot be loaded using |\usepackage| or |\RequirePackage|
% which has a mechanism to prevent loading a style file more than once.
% When loading the definitions by means of |\input|
% multiple instances have to be prevented manually:
%\iffalse
%This code needs to be before the `\ProvidesFile' directive
%which is defined at the beginning of this file.
%Therefore it is also placed there and commented out here.
%</package>
%<*discard>
%\fi
%    \begin{macrocode}
\ifdefined\childdocmain\endinput\fi
%    \end{macrocode}
%\iffalse
%</discard>
%<*package>
%\fi
%
% \macro{\ifchilddoc}
% \macro{\ifchilddocmanual}
% The conditional |\ifchilddoc| tells whether a
% child (true) or main (false) document is being compiled.
% The conditional |\ifchilddocmanual| tells whether
% the |\includeonly| mechanism is used (false) or
% the selection of child files must be performed manually (true).
% The definitions initialise to false:
%    \begin{macrocode}
\newif\ifchilddoc
\newif\ifchilddocmanual
%    \end{macrocode}

% \macro{\childdocname}
% \macro{\childdocjob}
% The macro |\childdocname| stores the name of the main document
% to be compiled. The macro |\childdocjob| stores the name of
% the document on which the \LaTeX{} compiler was originally invoked.
% The content of |\jobname| cannot be compared
% to filenames specified in the source due to different catcodes.
% The following code rescans |\jobname|, stores the result
% in |\childdocname| and saves a copy in |\childdocjob|:
%    \begin{macrocode}
\edef\childdocname{\scantokens\expandafter{\jobname\noexpand}}
\let\childdocjob\childdocname
%    \end{macrocode}

% \macro{\childdocdisable}
% The macro |\childdocdisable| prevents the main file
% from being processed more than once.
% At this stage, the main document command |\childdocmain|
% is assumed to be called once again where it should do nothing.
% Any subsequent call to it should prevent
% a secondary processing of the main document
% It overwrites the forwarding commands
% |\childdocof| and |\childdocforward|
% with empty macros to prevent further inclusions of the main document:
%    \begin{macrocode}
\newcommand{\childdocdisable}
{
  \renewcommand{\childdocmain}[1]{\renewcommand{\childdocmain}[1]{\endinput}}
  \renewcommand{\childdocof}[1]{}
  \renewcommand{\childdocby}[2][]{}
  \renewcommand{\childdocforward}[2][]{}
  \renewcommand{\childdocdisable}{}
}
%    \end{macrocode}

% \macro{\childdocmain}
% The macro |\childdocmain| is to be called at the top of the main file
% with nothing or the main filename (without extension) as argument.
% First, it breaks loops.
% If the argument is not empty and does not match |\childdocname|
% (which is set by the first inclusion of |childdoc.def|),
% |\ifchilddoc| is set to true, |\includeonly| is applied to the child file
% and |\jobname| is set to the main file
% (for proper handling of |.aux| files):
%    \begin{macrocode}
\newcommand{\childdocmain}[1]
{
  \childdocdisable\childdocmain{}
  \if?#1?\else
    \begingroup
      \def\childdoctmp{#1}
      \ifx\childdoctmp\childdocname
        \def\childdoctmp{}
      \else
        \def\childdoctmp
        {
          \childdoctrue
          \includeonly{\childdocname}
          \def\childdocjob{#1}
          \def\jobname{#1}
        }
      \fi
      \expandafter
    \endgroup
    \childdoctmp
  \fi
}
%    \end{macrocode}

% \macro{\childdocof}
% The command |\childdocof| redirects
% compilation to the main file |#1|.
%    \begin{macrocode}
\newcommand{\childdocof}[1]
{
  \childdocdisable
  \childdoctrue
  \includeonly{\childdocname}
  \def\jobname{#1}
  \def\childdocjob{#1}
  \input{#1}
}
%    \end{macrocode}

% \macro{\childdocby}
% The command |\childdocby| ....
%    \begin{macrocode}
\newcommand{\childdocby}[2][]
{
  \childdocdisable
  \childdoctrue
  \childdocmanualtrue
  \if?#1?\else
    \def\jobname{#2}
  \fi
  \def\childdocjob{#2}
  \input{#2}
  \endinput
}
%    \end{macrocode}

% \macro{\childdocforward}
% The command |\childdocforward| redirects
% compilation to the main file or
% (if the optional argument is given) a child file.
% Parameters are set as if the main file
% or a child file starting with |\childdocof| was compiled.
% Then compilation is handed over to the main file:
%    \begin{macrocode}
\newcommand{\childdocforward}[2][]
{
  \begingroup
    \if?#1?
      \def\childdoctmp
      {
        \def\childdocname{#2}
        \def\childdocjob{#2}
        \def\jobname{#2}
        \input{#2}
        \endinput
      }
    \else
      \def\childdoctmp
      {
        \childdocdisable
        \def\childdocname{#2}
        \childdoctrue
        \includeonly{#2}
        \def\childdocjob{#1}
        \def\jobname{#1}
        \input{#1}
        \endinput
      }
    \fi
    \expandafter
  \endgroup
  \childdoctmp
}
%    \end{macrocode}

% \macro{\childdocforwardprefix}
% The command |\childdocforwardprefix| redirects
% compilation to the main or a child file by means of a pattern.
% The prefix |#1| in the current filename is replaced by |#2|
% and the suffix of the current filename is kept
% (it is assumed that the filename does not contain the substring `|~~~|'
% which is used as a delimiter).
% Compilation is handed over to the new file by |\childdocforward|:
%    \begin{macrocode}
\newcommand{\childdocforwardprefix}[3][]
{
  \begingroup
    \def\childdocextract #2##1~~~{\def\childdoctmp{\childdocforward[#1]{#3##1}}}
    \expandafter\childdocextract\childdocname~~~
    \expandafter
  \endgroup
  \childdoctmp
}
%    \end{macrocode}

% \macro{\childdoc}
% The deprecated macro |\childdoc| is a legacy version of |\childdocmain|:
%    \begin{macrocode}
\newcommand{\childdoc}{\childdocmain}
%    \end{macrocode}

% \macro{\childdocredirect}
% The deprecated macro |\childdocredirect| is a legacy version
% of |\childdocforward| and |\childdocforwardprefix|:
%    \begin{macrocode}
\newcommand{\childdocredirect}[2][]
{
  \begingroup
    \if?#1?
      \def\childdoctmp{\childdocforward{#2}}
    \else
      \def\childdoctmp{\childdocforwardprefix{#1}{#2}}
    \fi
    \expandafter
  \endgroup
  \childdoctmp
}
%    \end{macrocode}

%\iffalse
%</package>
%\fi
%
\endinput
|\\
|\childdocforwardprefix{final}{child}|
\end{tabular}
\end{center}
%

Note that when several versions of a main file and/or of each child file
are to be generated, it may be convenient to set up a |Makefile| or
shell script to automatise the process.

%%%%%%%%%%%%%%%%%%%%%%%%%%%%%%%%%%%%%%%%%%%%%%%%%%%%%%%%%%%%%%%%%%%%%%%%%%%%%%%%
\subsection{Command Line Processing}
\label{sec:commandline}

The effect of redirection files can also be achieved by invoking
the \LaTeX{} compiler with a more elaborate command line.
Most conveniently this should be done as part
of a shell script or a |Makefile|.

When using \textsf{childdoc} in the main file, the following
command lines effectively perform a redirection
(note that depending on the shell being used,
backslashes may have to be doubled: `|\|' $\to$ `|\\|'):
%
\begin{center}
|... -jobname "|\textit{target}|" |\\|"|[\textit{flags}]%
|% \iffalse
%
% childdoc.dtx Copyright (C) 2017-2018 Niklas Beisert
%
% This work may be distributed and/or modified under the
% conditions of the LaTeX Project Public License, either version 1.3
% of this license or (at your option) any later version.
% The latest version of this license is in
%   http://www.latex-project.org/lppl.txt
% and version 1.3 or later is part of all distributions of LaTeX
% version 2005/12/01 or later.
%
% This work has the LPPL maintenance status `maintained'.
%
% The Current Maintainer of this work is Niklas Beisert.
%
% This work consists of the files childdoc.dtx and childdoc.ins
% and the derived files childdoc.def and cdocsamp.tex with
% cdocsch1.tex, cdocsch2.tex, cdocsdrf.tex, cdocsfn1.tex, cdocsfn2.tex.
%
%<package>\ifdefined\childdocmain\endinput\fi
%<package>\ProvidesFile{childdoc.def}[2018/12/30 v2.0 child document driver]
%<samplemain>\ProvidesFile{cdocsamp.tex}[2018/12/30 v2.0 sample for childdoc]
%<*driver>
%\ProvidesFile{childdoc.drv}[2018/12/30 v2.0 childdoc reference manual file]
\PassOptionsToClass{10pt,a4paper}{article}
\documentclass{ltxdoc}

\usepackage[margin=35mm]{geometry}
\usepackage{hyperref}
\usepackage{hyperxmp}
\usepackage[usenames]{color}

\hypersetup{colorlinks=true}
\hypersetup{pdfstartview=FitH}
\hypersetup{pdfpagemode=UseNone}
\hypersetup{pdfsource={}}
\hypersetup{pdflang={en-UK}}
\hypersetup{pdfcopyright={Copyright 2017-2018 Niklas Beisert.
  This work may be distributed and/or modified under the
  conditions of the LaTeX Project Public License, either version 1.3
  of this license or (at your option) any later version.}}
\hypersetup{pdflicenseurl={http://www.latex-project.org/lppl.txt}}
\hypersetup{pdfcontactaddress={ETH Zurich, ITP, HIT K,
  Wolfgang-Pauli-Strasse 27}}
\hypersetup{pdfcontactpostcode={8093}}
\hypersetup{pdfcontactcity={Zurich}}
\hypersetup{pdfcontactcountry={Switzerland}}
\hypersetup{pdfcontactemail={nbeisert@itp.phys.ethz.ch}}
\hypersetup{pdfcontacturl={http://people.phys.ethz.ch/\xmptilde nbeisert/}}

\newcommand{\secref}[1]{\hyperref[#1]{section \ref*{#1}}}

\parskip1ex
\parindent0pt
\let\olditemize\itemize
\def\itemize{\olditemize\parskip0pt}

\begin{document}

\title{The \textsf{childdoc} Package}
\hypersetup{pdftitle={The childdoc Package}}
\author{Niklas Beisert\\[2ex]
  Institut f\"ur Theoretische Physik\\
  Eidgen\"ossische Technische Hochschule Z\"urich\\
  Wolfgang-Pauli-Strasse 27, 8093 Z\"urich, Switzerland\\[1ex]
  \href{mailto:nbeisert@itp.phys.ethz.ch}
  {\texttt{nbeisert@itp.phys.ethz.ch}}}
\hypersetup{pdfauthor={Niklas Beisert}}
\hypersetup{pdfsubject={Manual for the LaTeX2e Package childdoc}}
\date{30 December 2018, \textsf{v2.0}}
\maketitle

\begin{abstract}\noindent
\textsf{childdoc} is a \LaTeXe{} package
that enables the direct compilation
of document sections included by |\include|
to individual files.
\end{abstract}

\begingroup
\parskip0ex
\tableofcontents
\endgroup

%%%%%%%%%%%%%%%%%%%%%%%%%%%%%%%%%%%%%%%%%%%%%%%%%%%%%%%%%%%%%%%%%%%%%%%%%%%%%%%%
%%%%%%%%%%%%%%%%%%%%%%%%%%%%%%%%%%%%%%%%%%%%%%%%%%%%%%%%%%%%%%%%%%%%%%%%%%%%%%%%
\section{Introduction}

\LaTeX{} provides a mechanism to structure a large document (such as a book)
into a main file and several child files (containing the chapters)
using the |\include| command.
This mechanism is beneficial for documents
which span hundreds of pages in order to
make the source file(s) more manageable.
Moreover, compilation can be restricted to
selected child files by means of the |\includeonly| command.
The latter feature can be used to reduce the compilation time while editing
(this was significantly more useful in the earlier days of \LaTeX{})
or to generate a smaller document which is easier to navigate.
Another application of |\includeonly| is to generate
documents consisting of selected parts of the complete document.

However, there are a few drawbacks of the plain |\include| mechanism:
\begin{itemize}
\item
The child files cannot be compiled on their own,
they can only be compiled via the main file.
A naive editing environment
(such as a text editor with an option
to have the current file processed by \LaTeX)
may require one to switch to the main file before compiling;
attempting to compile the child file produces errors.
\item
The main file must be modified (each time)
to adjust the |\includeonly| command
to the present needs. This easily leaves the main file in a messy state.
\item
The generated document will always carry the filename
of the main document. This is inconvenient if
several child files are to be compiled and
to be kept for distribution.
\end{itemize}

The present package provides a simple interface
to make child files individually compilable by \LaTeX{}.
Compiling a child file then has the same effect as compiling
the main file with an |\includeonly| command
to select the appropriate child.
Moreover the generated document will carry the name of the child
rather than the main file.
This resolves all three above issues.

This feature is meant to make the editing of books,
thesis documents and lecture notes somewhat more convenient.
However, the package can also be used efficiently for
composing a series of documents (such as exercise sheets)
which are typically distributed individually.
It then assists the author in generating the individual documents
(potentially in different versions)
as well as a document containing the collected series.
Another application is in developing style files
or other kinds of included material
where compilation of the style file could redirect
to a sample or test file.

%%%%%%%%%%%%%%%%%%%%%%%%%%%%%%%%%%%%%%%%%%%%%%%%%%%%%%%%%%%%%%%%%%%%%%%%%%%%%%%%
%%%%%%%%%%%%%%%%%%%%%%%%%%%%%%%%%%%%%%%%%%%%%%%%%%%%%%%%%%%%%%%%%%%%%%%%%%%%%%%%
\section{Usage}

First of all, the package \textsf{childdoc} is \emph{not} a standard
\LaTeXe{} |.sty| style file! Therefore it needs to be invoked in
a non-standard way.

%%%%%%%%%%%%%%%%%%%%%%%%%%%%%%%%%%%%%%%%%%%%%%%%%%%%%%%%%%%%%%%%%%%%%%%%%%%%%%%%
\subsection{Included Files}
\label{sec:include}

%%%%%%%%%%%%%%%%%%%%%%%%%%%%%%%%%%%%%%%%
\DescribeMacro{\childdocmain}
To use the package, add the commands
\begin{center}
\begin{tabular}{l}
|\input{childdoc.def}|\\
|\childdocmain{}|\\
\end{tabular}
\end{center}
at the very top of the main \LaTeX{} file,
in particular \emph{before} the |\documentclass| statement!
The argument of |\childdocmain| should be left empty
(but it must be present).

%%%%%%%%%%%%%%%%%%%%%%%%%%%%%%%%%%%%%%%%
\DescribeMacro{\childdocof}
Furthermore, add the commands
\begin{center}
\begin{tabular}{l}
|\input{childdoc.def}|\\
|\childdocof{|\textit{main}|}|\\
\end{tabular}
\end{center}
at the top of every child file \textit{child}
which is included by |\include{|\textit{child}|}|
from within the main file
(or at least for those files to be compiled individually).
The argument \textit{main} must be the filename of the main file.

There are a couple of
considerations in setting up the main and child documents:

%%%%%%%%%%%%%%%%%%%%%%%%%%%%%%%%%%%%%%%%
\paragraph{Restrictions.}

Please note the following restrictions:
\begin{itemize}
\item
|\childdocmain| must be called with one argument \textit{main}
to ensure compatibility with earlier version of the package.
It must either be empty (|\childdocmain{}|)
or precisely match the filename of the main file in which it is specified.
See \secref{sec:detection} for further information.
\item
The filename \textit{main} must be specified without the |.tex| extension.
\item
The filename \textit{main} is case sensitive
(even in case-insensitive file systems)
due to internal string comparison.
\item
The argument \textit{main} should be fully expanded, it cannot be a macro.
\item
Subdirectories and special characters should be avoided in filenames.
\item
The command |\childdocmain{|\textit{main}|}| must be followed by a whitespace.
It should not be followed immediately by another command
or by a comment mark `|%|'.
This is because the \TeX{} parser reads the token immediately following
the argument of |\childdocmain| and puts it
at the beginning of every child section;
however, a white\-space is ignored.
\end{itemize}

%%%%%%%%%%%%%%%%%%%%%%%%%%%%%%%%%%%%%%%%
\paragraph{Content of Main File.}

It is advisable to place all content in the child files included by |\include|.
Any output contained in the main file will appear in all child documents
unless suppressed manually;
it cannot be suppressed automatically by the |\includeonly| directive
and thus should normally be avoided.
A method to include some content in the main file
by means of conditional processing is described in \secref{sec:conditional}.

%%%%%%%%%%%%%%%%%%%%%%%%%%%%%%%%%%%%%%%%
\paragraph{Page Numbering.}

When only a part of the document is compiled,
the appropriate numbering of pages
(as well as other status parameters)
is determined from the |.aux| files.
The latter contain information from previous passes.
However this information needs to propagate through
all intermediate child documents.
Therefore the page numbering in child documents may well
be inconsistent until the complete document is compiled at least once.

A useful (if unconventional) way to always ensure a consistent
page numbering is to restart the numbering in each child document
and denote the pages by `\textit{child}|.|\textit{page}'
where \textit{child} represents the chapter/section number of the child file.
This can be achieved by the command
|\numberwithin{page}{|\textit{child}|}|
of the \textsf{amsmath} package
where \textit{child} can be |chapter| or |section|
depending on the chosen structuring.
Alternatively, one can modify the macro |\thepage| appropriately
and reset the counter |page| at the start of each child file.

%%%%%%%%%%%%%%%%%%%%%%%%%%%%%%%%%%%%%%%%%%%%%%%%%%%%%%%%%%%%%%%%%%%%%%%%%%%%%%%%
\subsection{Conditional Processing}
\label{sec:conditional}

The package provides a mechanism to compile different versions
of a document. To customise the versions further some conditional processing
can come in handy to distinguish which version is being compiled.
The package provides two macros to describe the compilation context:

%%%%%%%%%%%%%%%%%%%%%%%%%%%%%%%%%%%%%%%%
\DescribeMacro{\ifchilddoc}
The conditional |\ifchilddoc| distinguishes between the compilation of
child documents and the main document:
%
\begin{center}
|\ifchilddoc |\textit{child-code}| |[|\||else |\textit{main-code}]| \||fi|
\end{center}

%%%%%%%%%%%%%%%%%%%%%%%%%%%%%%%%%%%%%%%%
\DescribeMacro{\childdocname}
\DescribeMacro{\childdocjob}
The macro |\childdocname| contains the filename (without extension)
of the main or child file being processed.
Note that |\childdocjob| will always contain the name of the main file.

%%%%%%%%%%%%%%%%%%%%%%%%%%%%%%%%%%%%%%%%
\paragraph{Title Page.}

Conditional processing can be used to include a title or banner page
in the main document when proper precautions are taken.
Importantly, the code in the main file should ensure that the page counter
(as well as other status parameters which are stored in the |.aux| files)
takes the same value after the conditional processing.
Otherwise the page numbers may take divergent values
depending on which part is compiled.

For example, a title page could be declared by:
%
\begin{center}
\begin{tabular}{l}
|\ifchilddoc\||else|\\
|\addtocounter{page}{-1}|\\
\textit{code for title page}\\
|\newpage|\\
|\||fi|
\end{tabular}
\end{center}
%
A banner page for the child documents can be generated by:
%
\begin{center}
\begin{tabular}{l}
|\ifchilddoc|\\
|\addtocounter{page}{-1}|\\
\textit{code for banner page}\\
|\newpage|\\
|\||fi|
\end{tabular}
\end{center}
%
Here one could write a message such as:
\begin{center}
|This is the part \childdocname{} of \childdocjob{}.|
\end{center}

%%%%%%%%%%%%%%%%%%%%%%%%%%%%%%%%%%%%%%%%%%%%%%%%%%%%%%%%%%%%%%%%%%%%%%%%%%%%%%%%
\subsection{Flags}
\label{sec:flags}

The package makes it easy to generate different versions
of the main or child documents.
To this end compilation flags can be defined
and assigned different default values.
They will be particularly useful in conjunction
with the forwarding mechanism described in \secref{sec:forward}.

For example, it may be useful to have a flag |\version|
which can be set to |draft| or |final|.
The document source will contain some conditional code
depending on the value of |\version|.
Suppose further, the flag should default to |final| for the main file
and to |draft| for child files
which is a natural assignment for editing the document.
This is achieved by placing the following code
in the preamble of the main document
(below the |\childdocmain| directive):
%
\begin{center}
\begin{tabular}{l}
|\ifchilddoc|\\
|\providecommand{\version}{draft}|\\
|\||else|\\
|\providecommand{\version}{final}|\\
|\||fi|
\end{tabular}
\end{center}
%
The definition by |\providecommand| makes sure
that previous definitions are not overwritten.
Further statements |\providecommand{\version}{...}|
can thus be added before the above code to override it.

For the main file, one might add a line
(between |\childdocmain| and the above block)
%
\begin{center}
|%\ifchilddoc\||else\providecommand{\version}{draft}\||fi|
\end{center}
%
which can be uncommented to produce a draft version.
Likewise one can add a line to the very top of a child file
(above the |\childdocof{|\textit{main}|}| directive)
%
\begin{center}
|%\providecommand{\version}{final}|
\end{center}
%
which can be uncommented to produce the final version of this child document.

%%%%%%%%%%%%%%%%%%%%%%%%%%%%%%%%%%%%%%%%%%%%%%%%%%%%%%%%%%%%%%%%%%%%%%%%%%%%%%%%
\subsection{Forwarding}
\label{sec:forward}

Different versions of the main or child documents
using compilation flags as described in \secref{sec:flags}
can be (permanently) stored in different files
for convenient compilation, viewing and distribution.
To this end, the package defines a command
to pass on compilation to a different file:

%%%%%%%%%%%%%%%%%%%%%%%%%%%%%%%%%%%%%%%%
\DescribeMacro{\childdocforward}
The command |\childdocforward| redirects processing to
another source file:
%
\begin{center}
\begin{tabular}{l}
|\input{childdoc.def}|\\
|\childdocforward[|\textit{main}|]{|\textit{dest}|}|\\
\end{tabular}
\end{center}
%
The argument \textit{dest} is the destination file
(without extension).
It should be the main file or one of the child files.
Note that further \textsf{childdoc} directives
such as |\childdocof| and |\childdocforward|
in the indicated file will be processed in this form.
The optional argument \textit{main}
passes on directly to the main file \textit{main}
while pretending to compile the child \textit{dest}.
This form behaves as if \textit{dest}
issues |\childdocof{|\textit{main}|}| right away,
and no further \textsf{childdoc} directives will be processed.

%%%%%%%%%%%%%%%%%%%%%%%%%%%%%%%%%%%%%%%%
\DescribeMacro{\...prefix}
In the alternative form |\childdocforwardprefix|,
%
\begin{center}
\begin{tabular}{l}
|\input{childdoc.def}|\\
|\childdocforwardprefix[|\textit{main}|]{|\textit{prefix}|}{|\textit{dest}|}|
\end{tabular}
\end{center}
%
the destination file is determined by a pattern
depending on the current file:
To make this work, the current file must be called
`{\textit{prefix}\hspace{0.2em}\textit{suffix}}'
with \textit{prefix} matching precisely the argument.
Processing is then passed on to the file
`{\textit{dest}\hspace{0.2em}\textit{suffix}}'.
Surely, the same effect is achieved by
directly specifying the
argument `{\textit{dest}\hspace{0.2em}\textit{suffix}}'
in the first form.
However, that requires to set up a different file
for each child. With the alternative form of the command
all these files can have exactly the same content
which simplifies setting them up and maintaining them.

For example, the following file |draft.tex|
with a compilation flag |\version| as described in \secref{sec:flags}
compiles the main document as a draft:
%
\begin{center}
\begin{tabular}{l}
|\def\version{draft}|\\
|\input{childdoc.def}|\\
|\childdocforward{|\textit{main}|}|
\end{tabular}
\end{center}
%
Likewise, the following files |final|\textit{nn}|.tex|
compile the final version of the child document
|child|\textit{nn}|.tex|:
%
\begin{center}
\begin{tabular}{l}
|\def\version{final}|\\
|\input{childdoc.def}|\\
|\childdocforwardprefix{final}{child}|
\end{tabular}
\end{center}
%

Note that when several versions of a main file and/or of each child file
are to be generated, it may be convenient to set up a |Makefile| or
shell script to automatise the process.

%%%%%%%%%%%%%%%%%%%%%%%%%%%%%%%%%%%%%%%%%%%%%%%%%%%%%%%%%%%%%%%%%%%%%%%%%%%%%%%%
\subsection{Command Line Processing}
\label{sec:commandline}

The effect of redirection files can also be achieved by invoking
the \LaTeX{} compiler with a more elaborate command line.
Most conveniently this should be done as part
of a shell script or a |Makefile|.

When using \textsf{childdoc} in the main file, the following
command lines effectively perform a redirection
(note that depending on the shell being used,
backslashes may have to be doubled: `|\|' $\to$ `|\\|'):
%
\begin{center}
|... -jobname "|\textit{target}|" |\\|"|[\textit{flags}]%
|\input{childdoc.def}\childdocforward[|\textit{main}|]{|\textit{dest}|}"|
\end{center}
%
Here \textit{target} is the name of the output file,
\textit{main} is the name of the main file
and \textit{dest} is the name of the main or child file to be processed
(all filenames without extensions).
The optional argument \textit{main} can be omitted
if \textit{main} matches \textit{dest}.
Optionally, compilation \textit{flags} can be defined via |\def| commands.
This command line makes the \TeX{} engine believe
it is compiling the file \textit{target}
whose content is specified as the latter parameter.
The provided code then forwards the processing to
\textit{main} or \textit{dest} as described in \secref{sec:forward}.

%%%%%%%%%%%%%%%%%%%%%%%%%%%%%%%%%%%%%%%%%%%%%%%%%%%%%%%%%%%%%%%%%%%%%%%%%%%%%%%%
\subsection{Include by Input}
\label{sec:input}

Including child documents by |\include| has some restrictions by design.
Most notably, the content of a child document always occupies
its own set of pages; pages cannot be shared between child documents.
Usually, this behaviour makes perfect sense
because each child document contain an essential part of the document.
However, in some situations it may be desirable to compose
a document from a collection of parts
without having mandatory page breaks between then.
For this case, the package
provides a mechanism to include parts
by |\input| which can also be processed individually.
However, by construction this mechanism
requires manual handling of the content to be output.

%%%%%%%%%%%%%%%%%%%%%%%%%%%%%%%%%%%%%%%%
\DescribeMacro{\ifchilddocmanual}
The main file should be prepared as usual, see \secref{sec:include}.
However, the document body must make a distinction
between processing of an individual part and of the main document, e.g.:
%
\begin{center}
\begin{tabular}{l}
|\ifchilddocmanual|\\
|\input{\childdocname}|\\
|\||else|\\
\textit{document body with }|\input{|\textit{part}|}|\\
|\||fi|
\end{tabular}
\end{center}
%
The conditional |\ifchilddocmanual| is true whenever
a part to be included by |\input| is being compiled,
and the name of the part is stored in |\childdocname|.

%%%%%%%%%%%%%%%%%%%%%%%%%%%%%%%%%%%%%%%%
\DescribeMacro{\childdocby}
Each part to be included by |\input| should start with:
%
\begin{center}
\begin{tabular}{l}
|\input{childdoc.def}|\\
|\childdocby{|\textit{main}|}|\\
\end{tabular}
\end{center}
%
The directive |\childdocby| is similar to |\childdocof|
described in \secref{sec:include},
but the subsequent selection of content must be done manually.
To that end, both |\ifchilddoc| and |\ifchilddocmanual|
will be true upon processing of a part,
and the name of the part is stored in |\childdocname|.
Note that |\jobname| will be set to the filename of the current part
so that each part receives an individual |.aux| file
that does not interfere with the |.aux| file(s) of the main document.
This behaviour can be altered by the alternative form
|\childdocby[*]{|\textit{main}|}| (with a non-empty optional argument)
which uses the |.aux| file of the main document
by setting |\jobname| to \textit{main}.

%%%%%%%%%%%%%%%%%%%%%%%%%%%%%%%%%%%%%%%%%%%%%%%%%%%%%%%%%%%%%%%%%%%%%%%%%%%%%%%%
\subsection{Driver Development}
\label{sec:driver}

The \textsf{childdoc} mechanism can also be use for the development
of definition files such as \LaTeX{} styles or classes.
This case differs from the above setup with multiple parts
included by |\include| in that no |\includeonly| should be invoked.
This can be achieved by starting the include file
(before |\ProvidesPackage|) with:
%
\begin{center}
\begin{tabular}{l}
|\input{childdoc.def}|\\
|\childdocforward{|\textit{main}|}|\\
\end{tabular}
\end{center}
%
or alternatively with:
%
\begin{center}
\begin{tabular}{l}
|\input{childdoc.def}|\\
|\childdocby{|\textit{main}|}|\\
\end{tabular}
\end{center}
%
Both forms have slightly different effects as described above.
The main file is prepared as usual, see \secref{sec:include}.

%%%%%%%%%%%%%%%%%%%%%%%%%%%%%%%%%%%%%%%%%%%%%%%%%%%%%%%%%%%%%%%%%%%%%%%%%%%%%%%%
\subsection{Legacy Detection}
\label{sec:detection}

The directive |\childdocmain| in the main file can detect
whether the complete document or merely a child is to be compiled
even without using the directive |\childdocof|.
This method is deprecated because it is less robust
and there is no compelling reason to use it;
it is merely provided for backward compatibility
and it may be removed in future versions.

If the detection mechanism is to be used,
it is mandatory to correctly specify
the filename of the main file as the argument of |\childdocmain|:
%
\begin{center}
\begin{tabular}{l}
|\input{childdoc.def}|\\
|\childdocmain{|\textit{main}|}|\\
\end{tabular}
\end{center}
%
If |\jobname| does not match the argument \textit{main} of |\childdocmain|,
it is assumed that |\jobname| points to the child file to be compiled.
When using |\childdocmain| with the main file specified as argument,
it suffices to start a child file
with just |\input{|\textit{main}|}|
without loading of the package and using |\childdocof|.
If instead all processing is done
with the appropriate \textsf{childdoc} directives,
the argument of \textit{main} of |\childdocmain| can be empty.

An alternative version of the command line processing described
in \secref{sec:commandline} using the detection mechanism reads:
%
\begin{center}
|... -jobname "|\textit{target}|" "|[\textit{flags}]%
[|\def\jobname{|\textit{dest}|}|]|\input{|\textit{main}|}"|
\end{center}

%%%%%%%%%%%%%%%%%%%%%%%%%%%%%%%%%%%%%%%%%%%%%%%%%%%%%%%%%%%%%%%%%%%%%%%%%%%%%%%%
\subsection{Manual Code}
\label{sec:manual}

In case one cannot be certain whether the definitions file |childdoc.def|
is installed on the target \TeX{} distribution
and one prefers not to ship it,
it is conceivable to paste a few relevant commands into the sources.

To that end, drop all statements |\input{childdoc.def}|
and perform the replacements as outlined below.
Instead of |\childdocmain{|\textit{main}|}| add the following code
to the top of the main file:
%
\begin{center}
\begin{tabular}{l}
|\||ifdefined\childdocname\endinput\||fi\newif\ifchilddoc|\\
|\edef\childdocname{\scantokens\expandafter{\jobname\noexpand}}|\\
|\def\childdocmain{|\textit{main}|}\||ifx\childdocmain\childdocname\||else|\\
|\childdoctrue\includeonly{\childdocname}\let\jobname\childdocmain\||fi|\\
\end{tabular}
\end{center}
%
Instead of |\childdocof{|\textit{main}|}| just include the main file
at the top of each child file:
%
\begin{center}
|\input{|\textit{main}|}|
\end{center}
%
A simple redirection |\childdocforward{|\textit{dest}|}| is achieved by:
%
\begin{center}
|\def\jobname{|\textit{dest}|}\input{\jobname}|
\end{center}
%
The redirection with prefix
|\childdocforwardprefix[|\textit{prefix}|]{|\textit{dest}|}|
is accomplished by:
%
\begin{center}
\begin{tabular}{l}
|{\edef\jobname{\scantokens\expandafter{\jobname\noexpand}}|\\
|\def\redirectjob |\textit{prefix}|#1~~~{\gdef\jobname{|\textit{dest}|#1}}|\\
|\expandafter\redirectjob\jobname~~~}\input{\jobname}|
\end{tabular}
\end{center}

In an alternative approach,
child documents can be compiled by a specific command line
without additional code or specific definitions:
%
\begin{center}
|... -jobname "|\textit{target}|" "|[\textit{flags}]%
|\includeonly{|\textit{dest}|}\input{|\textit{main}|}"|
\end{center}
%

%%%%%%%%%%%%%%%%%%%%%%%%%%%%%%%%%%%%%%%%%%%%%%%%%%%%%%%%%%%%%%%%%%%%%%%%%%%%%%%%
%%%%%%%%%%%%%%%%%%%%%%%%%%%%%%%%%%%%%%%%%%%%%%%%%%%%%%%%%%%%%%%%%%%%%%%%%%%%%%%%
\section{Information}

%%%%%%%%%%%%%%%%%%%%%%%%%%%%%%%%%%%%%%%%%%%%%%%%%%%%%%%%%%%%%%%%%%%%%%%%%%%%%%%%
\subsection{Copyright}

Copyright \copyright{} 2017--2018 Niklas Beisert

This work may be distributed and/or modified under the
conditions of the \LaTeX{} Project Public License, either version 1.3
of this license or (at your option) any later version.
The latest version of this license is in
  \url{http://www.latex-project.org/lppl.txt}
and version 1.3 or later is part of all distributions of \LaTeX{}
version 2005/12/01 or later.

This work has the LPPL maintenance status `maintained'.

The Current Maintainer of this work is Niklas Beisert.

This work consists of the files |README.txt|, |childdoc.ins| and |childdoc.dtx|
as well as the derived files |childdoc.def|, |cdocsamp.tex|
with |cdocsch1.tex|, |cdocsch2.tex|, |cdocspt3.tex|, |cdocspt4.tex|,
|cdocsdrf.tex|, |cdocsfn1.tex|, |cdocsfn2.tex|
as well as |childdoc.pdf|.

%%%%%%%%%%%%%%%%%%%%%%%%%%%%%%%%%%%%%%%%%%%%%%%%%%%%%%%%%%%%%%%%%%%%%%%%%%%%%%%%
\subsection{Files and Installation}

The package consists of the files:
%
\begin{center}
\begin{tabular}{ll}
    |README.txt|   & readme file \\
    |childdoc.ins| & installation file \\
    |childdoc.dtx| & source file \\
    |childdoc.def| & definition file \\
    |cdocsamp.tex| & sample main file \\
    |cdocsch1.tex| & sample include file \\
    |cdocsch2.tex| & sample include file \\
    |cdocspt3.tex| & sample part file \\
    |cdocspt4.tex| & sample part file \\
    |cdocsdrf.tex| & sample redirection file \\
    |cdocsfn1.tex| & sample redirection file \\
    |cdocsfn2.tex| & sample redirection file \\
    |childdoc.pdf| & manual
\end{tabular}
\end{center}
%
The distribution consists of the files
|README.txt|, |childdoc.ins| and |childdoc.dtx|.
%
\begin{itemize}
\item
Run (pdf)\LaTeX{} on |childdoc.dtx|
to compile the manual |childdoc.pdf| (this file).
\item
Run \LaTeX{} on |childdoc.ins| to create the definitions file |childdoc.def|
and the sample |cdocsamp.tex| with include files
|cdocsch1.tex|, |cdocsch2.tex|, |cdocspt3.tex|, |cdocspt4.tex|,
|cdocsdrf.tex|, |cdocsfn1.tex|, |cdocsfn2.tex|.
Then copy the file |childdoc.def| to an appropriate directory of your \LaTeX{}
distribution, e.g.\ \textit{texmf-root}|/tex/latex/childdoc|.
\end{itemize}

%%%%%%%%%%%%%%%%%%%%%%%%%%%%%%%%%%%%%%%%%%%%%%%%%%%%%%%%%%%%%%%%%%%%%%%%%%%%%%%%
\subsection{Related CTAN Packages}

There are several other packages which offer a similar functionality:
%
\begin{itemize}
\item
The packages
\href{http://ctan.org/pkg/docmute}{\textsf{docmute}},
\href{http://ctan.org/pkg/includex}{\textsf{includex}} and
\href{http://ctan.org/pkg/standalone}{\textsf{standalone}}
provide commands to include only the document body of
a child file thus allowing both files to be compiled individually.
\item
The packages \href{http://ctan.org/pkg/subdocs}{\textsf{subdocs}}
and \href{http://ctan.org/pkg/subfiles}{\textsf{subfiles}}
provide structures in which the main and child documents can be
encapsulated and allowing them to be compiled individually.
The inclusion mechanism is different from the conventional |\include|.
\item
The package \href{http://ctan.org/pkg/combine}{\textsf{combine}}
is an elaborate solution to combine several documents into one.
\end{itemize}
%
See also the CTAN topic \href{http://ctan.org/topic/subdocs}{\textsf{subdocs}}
for further related packages.
The present package differs from the above solutions in that
a document structure constructed with the conventional |\include| mechanism
just needs two extra commands at the top of every file
such that all constituent files can be compiled individually.

%%%%%%%%%%%%%%%%%%%%%%%%%%%%%%%%%%%%%%%%%%%%%%%%%%%%%%%%%%%%%%%%%%%%%%%%%%%%%%%%
%\subsection{Feature Suggestions}
%
%The following is a list of features which may be useful for future
%versions of this package:
%%
%\begin{itemize}
%\item
%\ldots
%\end{itemize}

%%%%%%%%%%%%%%%%%%%%%%%%%%%%%%%%%%%%%%%%%%%%%%%%%%%%%%%%%%%%%%%%%%%%%%%%%%%%%%%%
\subsection{Revision History}

%%%%%%%%%%%%%%%%%%%%%%%%%%%%%%%%%%%%%%%%
\paragraph{v2.0:} 2018/12/30

\begin{itemize}
\item
immediate forward processing
\item
added |\childdocby| mechanism
\item
manual restructured
\end{itemize}

%%%%%%%%%%%%%%%%%%%%%%%%%%%%%%%%%%%%%%%%
\paragraph{v1.6:} 2018/01/17

\begin{itemize}
\item
application for development of include files
\item
corrections to manual
\end{itemize}

%%%%%%%%%%%%%%%%%%%%%%%%%%%%%%%%%%%%%%%%
\paragraph{v1.5:} 2017/05/21

\begin{itemize}
\item
more complete structuring introduced
\item
|\childdocof| introduced
\item
|\childdoc| renamed to |\childdocmain|
\item
|\childredirect| renamed to |\childdocforward| and |\childdocforwardprefix|
and functionality expanded
\end{itemize}

%%%%%%%%%%%%%%%%%%%%%%%%%%%%%%%%%%%%%%%%
\paragraph{v1.0:} 2017/04/27

\begin{itemize}
\item
manual and install package
\item
first version published on CTAN
\end{itemize}

%%%%%%%%%%%%%%%%%%%%%%%%%%%%%%%%%%%%%%%%
\paragraph{v0.6:} 2017/04/26

\begin{itemize}
\item
redirection mechanism added
\end{itemize}

%%%%%%%%%%%%%%%%%%%%%%%%%%%%%%%%%%%%%%%%
\paragraph{v0.5:} 2017/04/26

\begin{itemize}
\item
functionality in definition file
\end{itemize}


%%%%%%%%%%%%%%%%%%%%%%%%%%%%%%%%%%%%%%%%%%%%%%%%%%%%%%%%%%%%%%%%%%%%%%%%%%%%%%%%
%%%%%%%%%%%%%%%%%%%%%%%%%%%%%%%%%%%%%%%%%%%%%%%%%%%%%%%%%%%%%%%%%%%%%%%%%%%%%%%%
%%%%%%%%%%%%%%%%%%%%%%%%%%%%%%%%%%%%%%%%%%%%%%%%%%%%%%%%%%%%%%%%%%%%%%%%%%%%%%%%
\appendix

\settowidth\MacroIndent{\rmfamily\scriptsize 000\ }

 \DocInput{childdoc.dtx}

\end{document}
%</driver>
% \fi
%
% %%%%%%%%%%%%%%%%%%%%%%%%%%%%%%%%%%%%%%%%%%%%%%%%%%%%%%%%%%%%%%%%%%%%%%%%%%%%%%
% %%%%%%%%%%%%%%%%%%%%%%%%%%%%%%%%%%%%%%%%%%%%%%%%%%%%%%%%%%%%%%%%%%%%%%%%%%%%%%
% \section{Sample}
%\iffalse
%<*samplemain>
%\fi
%
% The following presents a sample document
% with two chapters, two parts, a title page,
% a compile flag as well as three forwarding files to set the flag.
% It consists of eight |.tex| files:
% \begin{center}
% \begin{tabular}{ll}
% |cdocsamp.tex|&main file\\
% |cdocsch1.tex|&include file for chapter 1\\
% |cdocsch2.tex|&include file for chapter 2\\
% |cdocspt3.tex|&include file for part 3\\
% |cdocspt4.tex|&include file for part 4\\
% |cdocsdrf.tex|&forwarding file for main file in draft mode\\
% |cdocsfi1.tex|&forwarding file for final version of chapter 1\\
% |cdocsfi2.tex|&forwarding file for final version of chapter 2\\
% \end{tabular}
% \end{center}
% Each of the eight files can be compiled directly by the \LaTeX{} compiler.
%
% %%%%%%%%%%%%%%%%%%%%%%%%%%%%%%%%%%%%%%
% \paragraph{Main File.}
%
% The main file is called |cdocsamp.tex|.
%
% Load the \textsf{childdoc} definitions and
% declare the filename for the main document:
%    \begin{macrocode}
\input{childdoc.def}
\childdocmain{}
%    \end{macrocode}

% Optional override for |\version| flag:
%    \begin{macrocode}
%%\ifchilddoc\else\providecommand{\version}{draft}\fi
%    \end{macrocode}

% Define the default values for the |\version| flag
% (|final| for the main file and |draft| for childs):
%    \begin{macrocode}
\ifchilddoc
\providecommand{\version}{draft}
\else
\providecommand{\version}{final}
\fi
%    \end{macrocode}

% Load the standard document class:
%    \begin{macrocode}
\documentclass[12pt]{article}
%    \end{macrocode}

% Start the document body:
%    \begin{macrocode}
\begin{document}
%    \end{macrocode}

% Declare a title page.
% Print title, part of document being processed and version flag:
%    \begin{macrocode}
\addtocounter{page}{-1}
\begin{center}
{\LARGE\bfseries{}childdoc example\par}
\vspace{1cm}
\ifchilddoc
\ifchilddocmanual part\else chapter\fi:
`\childdocname' of `\childdocjob'\par
\else
main document: `\childdocjob'\par
\fi
version: \version\par
\end{center}
\newpage
%    \end{macrocode}

% Manually include selected file,
% otherwise process as usual:
%    \begin{macrocode}
\ifchilddocmanual
\section*{part `\childdocname'}
\input{\childdocname}
\else
%    \end{macrocode}

% Include the two chapters:
%    \begin{macrocode}
\include{cdocsch1}
\include{cdocsch2}
%    \end{macrocode}

% Include the two parts unless only chapters should be displayed:
%    \begin{macrocode}
\ifchilddoc\else
\section{part three}
\input{cdocspt3}
\section{part four}
\input{cdocspt4}
\fi
%    \end{macrocode}

% Process as usual until here:
%    \begin{macrocode}
\fi
%    \end{macrocode}

% End of document body:
%    \begin{macrocode}
\end{document}
%    \end{macrocode}
%\iffalse
%</samplemain>
%\fi
%
% %%%%%%%%%%%%%%%%%%%%%%%%%%%%%%%%%%%%%%
% \paragraph{Chapter Include Files.}
%
% The include files are called |cdocsch1.tex| and |cdocsch2.tex|.
%
%\iffalse
%<*samplechap1|samplechap2>
%\fi

% Optional override for |\version| flag:
%    \begin{macrocode}
%%\providecommand{\version}{final}
%    \end{macrocode}

% Include the main document:
%    \begin{macrocode}
\input{childdoc.def}
\childdocof{cdocsamp}
%    \end{macrocode}

%\iffalse
%</samplechap1|samplechap2>
%\fi
%
%\iffalse
%<*samplechap1>
%\fi
% Some text for chapter 1:
%    \begin{macrocode}
\section{one}
some text in chapter one
%    \end{macrocode}

%\iffalse
%</samplechap1>
%\fi
% Some text for chapter 2:
%\iffalse
%<*samplechap2>
%\fi
%    \begin{macrocode}
\section{two}
more text in chapter two
%    \end{macrocode}

%\iffalse
%</samplechap2>
%\fi
%
% %%%%%%%%%%%%%%%%%%%%%%%%%%%%%%%%%%%%%%
% \paragraph{Part Include Files.}
%
% The include files are called |cdocspt3.tex| and |cdocspt4.tex|.
%
%\iffalse
%<*samplepart3|samplepart4>
%\fi

% Optional override for |\version| flag:
%    \begin{macrocode}
%%\providecommand{\version}{final}
%    \end{macrocode}

% Include the main document:
%    \begin{macrocode}
\input{childdoc.def}
\childdocby{cdocsamp}
%    \end{macrocode}

%\iffalse
%</samplepart3|samplepart4>
%\fi
%
%\iffalse
%<*samplepart3>
%\fi
% Some text for part 3:
%    \begin{macrocode}
some text in part three
%    \end{macrocode}

%\iffalse
%</samplepart3>
%\fi
% Some text for part 4:
%\iffalse
%<*samplepart4>
%\fi
%    \begin{macrocode}
more text in part four
%    \end{macrocode}

%\iffalse
%</samplepart4>
%\fi
%
% %%%%%%%%%%%%%%%%%%%%%%%%%%%%%%%%%%%%%%
% \paragraph{Forwarding for a Complete Draft.}
%
% The following forwarding file |cdocsdrf.tex|
% compiles the main document in draft mode:
%\iffalse
%<*sampledraft>
%\fi
%    \begin{macrocode}
\def\version{draft}
\input{childdoc.def}
\childdocforward{cdocsamp}
%    \end{macrocode}

%\iffalse
%</sampledraft>
%\fi
%
% %%%%%%%%%%%%%%%%%%%%%%%%%%%%%%%%%%%%%%
% \paragraph{Forwarding for Final Version of the Chapters.}
%
% The following forwarding files |cdocsfn1.tex| and |cdocsfn2.tex|
% (with identical content)
% compile the final versions of the child documents
% |cdocsch1.tex| and |cdocsch2.tex|, respectively:
%\iffalse
%<*samplefinal>
%\fi
%    \begin{macrocode}
\def\version{final}
\input{childdoc.def}
\childdocforwardprefix[cdocsamp]{cdocsfn}{cdocsch}
%    \end{macrocode}

%\iffalse
%</samplefinal>
%\fi
%
% %%%%%%%%%%%%%%%%%%%%%%%%%%%%%%%%%%%%%%
% \paragraph{Command Line Processing.}
%
% The following three command lines generate the output files
% |cdocscld|, |cdocscl1| and |cdocscl2|
% which should be identical to
% |cdocsdrf|, |cdocsch1| and |cdocsfn2|, respectively:
% \begin{center}
% \begin{tabular}{l}
% |latex -jobname cdocscld \|\\
% |  "\def\version{draft}\input{childdoc.def}\childdocforward{cdocsamp}"|\\
% |latex -jobname cdocscl1 \|\\
% |  "\input{childdoc.def}\childdocforward[cdocsamp]{cdocsch1}"|\\
% |latex -jobname cdocscl2 \|\\
% |  "\def\version{final}\input{childdoc.def}\childdocforward{cdocsch2}"|
% \end{tabular}
% \end{center}
% Note that the trailing backslash on each first line
% merely continues the input to the second line
% (for convenient cut ant paste).
% Furthermore, the command |latex| can be replaced by any
% of its alternative versions such as |pdflatex|.
%
% %%%%%%%%%%%%%%%%%%%%%%%%%%%%%%%%%%%%%%%%%%%%%%%%%%%%%%%%%%%%%%%%%%%%%%%%%%%%%%
% %%%%%%%%%%%%%%%%%%%%%%%%%%%%%%%%%%%%%%%%%%%%%%%%%%%%%%%%%%%%%%%%%%%%%%%%%%%%%%
% \section{Implementation}
%\iffalse
%<*package>
%\fi
%
% This section describes the definitions file |childdoc.def|.

% The definitions cannot be loaded using |\usepackage| or |\RequirePackage|
% which has a mechanism to prevent loading a style file more than once.
% When loading the definitions by means of |\input|
% multiple instances have to be prevented manually:
%\iffalse
%This code needs to be before the `\ProvidesFile' directive
%which is defined at the beginning of this file.
%Therefore it is also placed there and commented out here.
%</package>
%<*discard>
%\fi
%    \begin{macrocode}
\ifdefined\childdocmain\endinput\fi
%    \end{macrocode}
%\iffalse
%</discard>
%<*package>
%\fi
%
% \macro{\ifchilddoc}
% \macro{\ifchilddocmanual}
% The conditional |\ifchilddoc| tells whether a
% child (true) or main (false) document is being compiled.
% The conditional |\ifchilddocmanual| tells whether
% the |\includeonly| mechanism is used (false) or
% the selection of child files must be performed manually (true).
% The definitions initialise to false:
%    \begin{macrocode}
\newif\ifchilddoc
\newif\ifchilddocmanual
%    \end{macrocode}

% \macro{\childdocname}
% \macro{\childdocjob}
% The macro |\childdocname| stores the name of the main document
% to be compiled. The macro |\childdocjob| stores the name of
% the document on which the \LaTeX{} compiler was originally invoked.
% The content of |\jobname| cannot be compared
% to filenames specified in the source due to different catcodes.
% The following code rescans |\jobname|, stores the result
% in |\childdocname| and saves a copy in |\childdocjob|:
%    \begin{macrocode}
\edef\childdocname{\scantokens\expandafter{\jobname\noexpand}}
\let\childdocjob\childdocname
%    \end{macrocode}

% \macro{\childdocdisable}
% The macro |\childdocdisable| prevents the main file
% from being processed more than once.
% At this stage, the main document command |\childdocmain|
% is assumed to be called once again where it should do nothing.
% Any subsequent call to it should prevent
% a secondary processing of the main document
% It overwrites the forwarding commands
% |\childdocof| and |\childdocforward|
% with empty macros to prevent further inclusions of the main document:
%    \begin{macrocode}
\newcommand{\childdocdisable}
{
  \renewcommand{\childdocmain}[1]{\renewcommand{\childdocmain}[1]{\endinput}}
  \renewcommand{\childdocof}[1]{}
  \renewcommand{\childdocby}[2][]{}
  \renewcommand{\childdocforward}[2][]{}
  \renewcommand{\childdocdisable}{}
}
%    \end{macrocode}

% \macro{\childdocmain}
% The macro |\childdocmain| is to be called at the top of the main file
% with nothing or the main filename (without extension) as argument.
% First, it breaks loops.
% If the argument is not empty and does not match |\childdocname|
% (which is set by the first inclusion of |childdoc.def|),
% |\ifchilddoc| is set to true, |\includeonly| is applied to the child file
% and |\jobname| is set to the main file
% (for proper handling of |.aux| files):
%    \begin{macrocode}
\newcommand{\childdocmain}[1]
{
  \childdocdisable\childdocmain{}
  \if?#1?\else
    \begingroup
      \def\childdoctmp{#1}
      \ifx\childdoctmp\childdocname
        \def\childdoctmp{}
      \else
        \def\childdoctmp
        {
          \childdoctrue
          \includeonly{\childdocname}
          \def\childdocjob{#1}
          \def\jobname{#1}
        }
      \fi
      \expandafter
    \endgroup
    \childdoctmp
  \fi
}
%    \end{macrocode}

% \macro{\childdocof}
% The command |\childdocof| redirects
% compilation to the main file |#1|.
%    \begin{macrocode}
\newcommand{\childdocof}[1]
{
  \childdocdisable
  \childdoctrue
  \includeonly{\childdocname}
  \def\jobname{#1}
  \def\childdocjob{#1}
  \input{#1}
}
%    \end{macrocode}

% \macro{\childdocby}
% The command |\childdocby| ....
%    \begin{macrocode}
\newcommand{\childdocby}[2][]
{
  \childdocdisable
  \childdoctrue
  \childdocmanualtrue
  \if?#1?\else
    \def\jobname{#2}
  \fi
  \def\childdocjob{#2}
  \input{#2}
  \endinput
}
%    \end{macrocode}

% \macro{\childdocforward}
% The command |\childdocforward| redirects
% compilation to the main file or
% (if the optional argument is given) a child file.
% Parameters are set as if the main file
% or a child file starting with |\childdocof| was compiled.
% Then compilation is handed over to the main file:
%    \begin{macrocode}
\newcommand{\childdocforward}[2][]
{
  \begingroup
    \if?#1?
      \def\childdoctmp
      {
        \def\childdocname{#2}
        \def\childdocjob{#2}
        \def\jobname{#2}
        \input{#2}
        \endinput
      }
    \else
      \def\childdoctmp
      {
        \childdocdisable
        \def\childdocname{#2}
        \childdoctrue
        \includeonly{#2}
        \def\childdocjob{#1}
        \def\jobname{#1}
        \input{#1}
        \endinput
      }
    \fi
    \expandafter
  \endgroup
  \childdoctmp
}
%    \end{macrocode}

% \macro{\childdocforwardprefix}
% The command |\childdocforwardprefix| redirects
% compilation to the main or a child file by means of a pattern.
% The prefix |#1| in the current filename is replaced by |#2|
% and the suffix of the current filename is kept
% (it is assumed that the filename does not contain the substring `|~~~|'
% which is used as a delimiter).
% Compilation is handed over to the new file by |\childdocforward|:
%    \begin{macrocode}
\newcommand{\childdocforwardprefix}[3][]
{
  \begingroup
    \def\childdocextract #2##1~~~{\def\childdoctmp{\childdocforward[#1]{#3##1}}}
    \expandafter\childdocextract\childdocname~~~
    \expandafter
  \endgroup
  \childdoctmp
}
%    \end{macrocode}

% \macro{\childdoc}
% The deprecated macro |\childdoc| is a legacy version of |\childdocmain|:
%    \begin{macrocode}
\newcommand{\childdoc}{\childdocmain}
%    \end{macrocode}

% \macro{\childdocredirect}
% The deprecated macro |\childdocredirect| is a legacy version
% of |\childdocforward| and |\childdocforwardprefix|:
%    \begin{macrocode}
\newcommand{\childdocredirect}[2][]
{
  \begingroup
    \if?#1?
      \def\childdoctmp{\childdocforward{#2}}
    \else
      \def\childdoctmp{\childdocforwardprefix{#1}{#2}}
    \fi
    \expandafter
  \endgroup
  \childdoctmp
}
%    \end{macrocode}

%\iffalse
%</package>
%\fi
%
\endinput
\childdocforward[|\textit{main}|]{|\textit{dest}|}"|
\end{center}
%
Here \textit{target} is the name of the output file,
\textit{main} is the name of the main file
and \textit{dest} is the name of the main or child file to be processed
(all filenames without extensions).
The optional argument \textit{main} can be omitted
if \textit{main} matches \textit{dest}.
Optionally, compilation \textit{flags} can be defined via |\def| commands.
This command line makes the \TeX{} engine believe
it is compiling the file \textit{target}
whose content is specified as the latter parameter.
The provided code then forwards the processing to
\textit{main} or \textit{dest} as described in \secref{sec:forward}.

%%%%%%%%%%%%%%%%%%%%%%%%%%%%%%%%%%%%%%%%%%%%%%%%%%%%%%%%%%%%%%%%%%%%%%%%%%%%%%%%
\subsection{Include by Input}
\label{sec:input}

Including child documents by |\include| has some restrictions by design.
Most notably, the content of a child document always occupies
its own set of pages; pages cannot be shared between child documents.
Usually, this behaviour makes perfect sense
because each child document contain an essential part of the document.
However, in some situations it may be desirable to compose
a document from a collection of parts
without having mandatory page breaks between then.
For this case, the package
provides a mechanism to include parts
by |\input| which can also be processed individually.
However, by construction this mechanism
requires manual handling of the content to be output.

%%%%%%%%%%%%%%%%%%%%%%%%%%%%%%%%%%%%%%%%
\DescribeMacro{\ifchilddocmanual}
The main file should be prepared as usual, see \secref{sec:include}.
However, the document body must make a distinction
between processing of an individual part and of the main document, e.g.:
%
\begin{center}
\begin{tabular}{l}
|\ifchilddocmanual|\\
|\input{\childdocname}|\\
|\||else|\\
\textit{document body with }|\input{|\textit{part}|}|\\
|\||fi|
\end{tabular}
\end{center}
%
The conditional |\ifchilddocmanual| is true whenever
a part to be included by |\input| is being compiled,
and the name of the part is stored in |\childdocname|.

%%%%%%%%%%%%%%%%%%%%%%%%%%%%%%%%%%%%%%%%
\DescribeMacro{\childdocby}
Each part to be included by |\input| should start with:
%
\begin{center}
\begin{tabular}{l}
|% \iffalse
%
% childdoc.dtx Copyright (C) 2017-2018 Niklas Beisert
%
% This work may be distributed and/or modified under the
% conditions of the LaTeX Project Public License, either version 1.3
% of this license or (at your option) any later version.
% The latest version of this license is in
%   http://www.latex-project.org/lppl.txt
% and version 1.3 or later is part of all distributions of LaTeX
% version 2005/12/01 or later.
%
% This work has the LPPL maintenance status `maintained'.
%
% The Current Maintainer of this work is Niklas Beisert.
%
% This work consists of the files childdoc.dtx and childdoc.ins
% and the derived files childdoc.def and cdocsamp.tex with
% cdocsch1.tex, cdocsch2.tex, cdocsdrf.tex, cdocsfn1.tex, cdocsfn2.tex.
%
%<package>\ifdefined\childdocmain\endinput\fi
%<package>\ProvidesFile{childdoc.def}[2018/12/30 v2.0 child document driver]
%<samplemain>\ProvidesFile{cdocsamp.tex}[2018/12/30 v2.0 sample for childdoc]
%<*driver>
%\ProvidesFile{childdoc.drv}[2018/12/30 v2.0 childdoc reference manual file]
\PassOptionsToClass{10pt,a4paper}{article}
\documentclass{ltxdoc}

\usepackage[margin=35mm]{geometry}
\usepackage{hyperref}
\usepackage{hyperxmp}
\usepackage[usenames]{color}

\hypersetup{colorlinks=true}
\hypersetup{pdfstartview=FitH}
\hypersetup{pdfpagemode=UseNone}
\hypersetup{pdfsource={}}
\hypersetup{pdflang={en-UK}}
\hypersetup{pdfcopyright={Copyright 2017-2018 Niklas Beisert.
  This work may be distributed and/or modified under the
  conditions of the LaTeX Project Public License, either version 1.3
  of this license or (at your option) any later version.}}
\hypersetup{pdflicenseurl={http://www.latex-project.org/lppl.txt}}
\hypersetup{pdfcontactaddress={ETH Zurich, ITP, HIT K,
  Wolfgang-Pauli-Strasse 27}}
\hypersetup{pdfcontactpostcode={8093}}
\hypersetup{pdfcontactcity={Zurich}}
\hypersetup{pdfcontactcountry={Switzerland}}
\hypersetup{pdfcontactemail={nbeisert@itp.phys.ethz.ch}}
\hypersetup{pdfcontacturl={http://people.phys.ethz.ch/\xmptilde nbeisert/}}

\newcommand{\secref}[1]{\hyperref[#1]{section \ref*{#1}}}

\parskip1ex
\parindent0pt
\let\olditemize\itemize
\def\itemize{\olditemize\parskip0pt}

\begin{document}

\title{The \textsf{childdoc} Package}
\hypersetup{pdftitle={The childdoc Package}}
\author{Niklas Beisert\\[2ex]
  Institut f\"ur Theoretische Physik\\
  Eidgen\"ossische Technische Hochschule Z\"urich\\
  Wolfgang-Pauli-Strasse 27, 8093 Z\"urich, Switzerland\\[1ex]
  \href{mailto:nbeisert@itp.phys.ethz.ch}
  {\texttt{nbeisert@itp.phys.ethz.ch}}}
\hypersetup{pdfauthor={Niklas Beisert}}
\hypersetup{pdfsubject={Manual for the LaTeX2e Package childdoc}}
\date{30 December 2018, \textsf{v2.0}}
\maketitle

\begin{abstract}\noindent
\textsf{childdoc} is a \LaTeXe{} package
that enables the direct compilation
of document sections included by |\include|
to individual files.
\end{abstract}

\begingroup
\parskip0ex
\tableofcontents
\endgroup

%%%%%%%%%%%%%%%%%%%%%%%%%%%%%%%%%%%%%%%%%%%%%%%%%%%%%%%%%%%%%%%%%%%%%%%%%%%%%%%%
%%%%%%%%%%%%%%%%%%%%%%%%%%%%%%%%%%%%%%%%%%%%%%%%%%%%%%%%%%%%%%%%%%%%%%%%%%%%%%%%
\section{Introduction}

\LaTeX{} provides a mechanism to structure a large document (such as a book)
into a main file and several child files (containing the chapters)
using the |\include| command.
This mechanism is beneficial for documents
which span hundreds of pages in order to
make the source file(s) more manageable.
Moreover, compilation can be restricted to
selected child files by means of the |\includeonly| command.
The latter feature can be used to reduce the compilation time while editing
(this was significantly more useful in the earlier days of \LaTeX{})
or to generate a smaller document which is easier to navigate.
Another application of |\includeonly| is to generate
documents consisting of selected parts of the complete document.

However, there are a few drawbacks of the plain |\include| mechanism:
\begin{itemize}
\item
The child files cannot be compiled on their own,
they can only be compiled via the main file.
A naive editing environment
(such as a text editor with an option
to have the current file processed by \LaTeX)
may require one to switch to the main file before compiling;
attempting to compile the child file produces errors.
\item
The main file must be modified (each time)
to adjust the |\includeonly| command
to the present needs. This easily leaves the main file in a messy state.
\item
The generated document will always carry the filename
of the main document. This is inconvenient if
several child files are to be compiled and
to be kept for distribution.
\end{itemize}

The present package provides a simple interface
to make child files individually compilable by \LaTeX{}.
Compiling a child file then has the same effect as compiling
the main file with an |\includeonly| command
to select the appropriate child.
Moreover the generated document will carry the name of the child
rather than the main file.
This resolves all three above issues.

This feature is meant to make the editing of books,
thesis documents and lecture notes somewhat more convenient.
However, the package can also be used efficiently for
composing a series of documents (such as exercise sheets)
which are typically distributed individually.
It then assists the author in generating the individual documents
(potentially in different versions)
as well as a document containing the collected series.
Another application is in developing style files
or other kinds of included material
where compilation of the style file could redirect
to a sample or test file.

%%%%%%%%%%%%%%%%%%%%%%%%%%%%%%%%%%%%%%%%%%%%%%%%%%%%%%%%%%%%%%%%%%%%%%%%%%%%%%%%
%%%%%%%%%%%%%%%%%%%%%%%%%%%%%%%%%%%%%%%%%%%%%%%%%%%%%%%%%%%%%%%%%%%%%%%%%%%%%%%%
\section{Usage}

First of all, the package \textsf{childdoc} is \emph{not} a standard
\LaTeXe{} |.sty| style file! Therefore it needs to be invoked in
a non-standard way.

%%%%%%%%%%%%%%%%%%%%%%%%%%%%%%%%%%%%%%%%%%%%%%%%%%%%%%%%%%%%%%%%%%%%%%%%%%%%%%%%
\subsection{Included Files}
\label{sec:include}

%%%%%%%%%%%%%%%%%%%%%%%%%%%%%%%%%%%%%%%%
\DescribeMacro{\childdocmain}
To use the package, add the commands
\begin{center}
\begin{tabular}{l}
|\input{childdoc.def}|\\
|\childdocmain{}|\\
\end{tabular}
\end{center}
at the very top of the main \LaTeX{} file,
in particular \emph{before} the |\documentclass| statement!
The argument of |\childdocmain| should be left empty
(but it must be present).

%%%%%%%%%%%%%%%%%%%%%%%%%%%%%%%%%%%%%%%%
\DescribeMacro{\childdocof}
Furthermore, add the commands
\begin{center}
\begin{tabular}{l}
|\input{childdoc.def}|\\
|\childdocof{|\textit{main}|}|\\
\end{tabular}
\end{center}
at the top of every child file \textit{child}
which is included by |\include{|\textit{child}|}|
from within the main file
(or at least for those files to be compiled individually).
The argument \textit{main} must be the filename of the main file.

There are a couple of
considerations in setting up the main and child documents:

%%%%%%%%%%%%%%%%%%%%%%%%%%%%%%%%%%%%%%%%
\paragraph{Restrictions.}

Please note the following restrictions:
\begin{itemize}
\item
|\childdocmain| must be called with one argument \textit{main}
to ensure compatibility with earlier version of the package.
It must either be empty (|\childdocmain{}|)
or precisely match the filename of the main file in which it is specified.
See \secref{sec:detection} for further information.
\item
The filename \textit{main} must be specified without the |.tex| extension.
\item
The filename \textit{main} is case sensitive
(even in case-insensitive file systems)
due to internal string comparison.
\item
The argument \textit{main} should be fully expanded, it cannot be a macro.
\item
Subdirectories and special characters should be avoided in filenames.
\item
The command |\childdocmain{|\textit{main}|}| must be followed by a whitespace.
It should not be followed immediately by another command
or by a comment mark `|%|'.
This is because the \TeX{} parser reads the token immediately following
the argument of |\childdocmain| and puts it
at the beginning of every child section;
however, a white\-space is ignored.
\end{itemize}

%%%%%%%%%%%%%%%%%%%%%%%%%%%%%%%%%%%%%%%%
\paragraph{Content of Main File.}

It is advisable to place all content in the child files included by |\include|.
Any output contained in the main file will appear in all child documents
unless suppressed manually;
it cannot be suppressed automatically by the |\includeonly| directive
and thus should normally be avoided.
A method to include some content in the main file
by means of conditional processing is described in \secref{sec:conditional}.

%%%%%%%%%%%%%%%%%%%%%%%%%%%%%%%%%%%%%%%%
\paragraph{Page Numbering.}

When only a part of the document is compiled,
the appropriate numbering of pages
(as well as other status parameters)
is determined from the |.aux| files.
The latter contain information from previous passes.
However this information needs to propagate through
all intermediate child documents.
Therefore the page numbering in child documents may well
be inconsistent until the complete document is compiled at least once.

A useful (if unconventional) way to always ensure a consistent
page numbering is to restart the numbering in each child document
and denote the pages by `\textit{child}|.|\textit{page}'
where \textit{child} represents the chapter/section number of the child file.
This can be achieved by the command
|\numberwithin{page}{|\textit{child}|}|
of the \textsf{amsmath} package
where \textit{child} can be |chapter| or |section|
depending on the chosen structuring.
Alternatively, one can modify the macro |\thepage| appropriately
and reset the counter |page| at the start of each child file.

%%%%%%%%%%%%%%%%%%%%%%%%%%%%%%%%%%%%%%%%%%%%%%%%%%%%%%%%%%%%%%%%%%%%%%%%%%%%%%%%
\subsection{Conditional Processing}
\label{sec:conditional}

The package provides a mechanism to compile different versions
of a document. To customise the versions further some conditional processing
can come in handy to distinguish which version is being compiled.
The package provides two macros to describe the compilation context:

%%%%%%%%%%%%%%%%%%%%%%%%%%%%%%%%%%%%%%%%
\DescribeMacro{\ifchilddoc}
The conditional |\ifchilddoc| distinguishes between the compilation of
child documents and the main document:
%
\begin{center}
|\ifchilddoc |\textit{child-code}| |[|\||else |\textit{main-code}]| \||fi|
\end{center}

%%%%%%%%%%%%%%%%%%%%%%%%%%%%%%%%%%%%%%%%
\DescribeMacro{\childdocname}
\DescribeMacro{\childdocjob}
The macro |\childdocname| contains the filename (without extension)
of the main or child file being processed.
Note that |\childdocjob| will always contain the name of the main file.

%%%%%%%%%%%%%%%%%%%%%%%%%%%%%%%%%%%%%%%%
\paragraph{Title Page.}

Conditional processing can be used to include a title or banner page
in the main document when proper precautions are taken.
Importantly, the code in the main file should ensure that the page counter
(as well as other status parameters which are stored in the |.aux| files)
takes the same value after the conditional processing.
Otherwise the page numbers may take divergent values
depending on which part is compiled.

For example, a title page could be declared by:
%
\begin{center}
\begin{tabular}{l}
|\ifchilddoc\||else|\\
|\addtocounter{page}{-1}|\\
\textit{code for title page}\\
|\newpage|\\
|\||fi|
\end{tabular}
\end{center}
%
A banner page for the child documents can be generated by:
%
\begin{center}
\begin{tabular}{l}
|\ifchilddoc|\\
|\addtocounter{page}{-1}|\\
\textit{code for banner page}\\
|\newpage|\\
|\||fi|
\end{tabular}
\end{center}
%
Here one could write a message such as:
\begin{center}
|This is the part \childdocname{} of \childdocjob{}.|
\end{center}

%%%%%%%%%%%%%%%%%%%%%%%%%%%%%%%%%%%%%%%%%%%%%%%%%%%%%%%%%%%%%%%%%%%%%%%%%%%%%%%%
\subsection{Flags}
\label{sec:flags}

The package makes it easy to generate different versions
of the main or child documents.
To this end compilation flags can be defined
and assigned different default values.
They will be particularly useful in conjunction
with the forwarding mechanism described in \secref{sec:forward}.

For example, it may be useful to have a flag |\version|
which can be set to |draft| or |final|.
The document source will contain some conditional code
depending on the value of |\version|.
Suppose further, the flag should default to |final| for the main file
and to |draft| for child files
which is a natural assignment for editing the document.
This is achieved by placing the following code
in the preamble of the main document
(below the |\childdocmain| directive):
%
\begin{center}
\begin{tabular}{l}
|\ifchilddoc|\\
|\providecommand{\version}{draft}|\\
|\||else|\\
|\providecommand{\version}{final}|\\
|\||fi|
\end{tabular}
\end{center}
%
The definition by |\providecommand| makes sure
that previous definitions are not overwritten.
Further statements |\providecommand{\version}{...}|
can thus be added before the above code to override it.

For the main file, one might add a line
(between |\childdocmain| and the above block)
%
\begin{center}
|%\ifchilddoc\||else\providecommand{\version}{draft}\||fi|
\end{center}
%
which can be uncommented to produce a draft version.
Likewise one can add a line to the very top of a child file
(above the |\childdocof{|\textit{main}|}| directive)
%
\begin{center}
|%\providecommand{\version}{final}|
\end{center}
%
which can be uncommented to produce the final version of this child document.

%%%%%%%%%%%%%%%%%%%%%%%%%%%%%%%%%%%%%%%%%%%%%%%%%%%%%%%%%%%%%%%%%%%%%%%%%%%%%%%%
\subsection{Forwarding}
\label{sec:forward}

Different versions of the main or child documents
using compilation flags as described in \secref{sec:flags}
can be (permanently) stored in different files
for convenient compilation, viewing and distribution.
To this end, the package defines a command
to pass on compilation to a different file:

%%%%%%%%%%%%%%%%%%%%%%%%%%%%%%%%%%%%%%%%
\DescribeMacro{\childdocforward}
The command |\childdocforward| redirects processing to
another source file:
%
\begin{center}
\begin{tabular}{l}
|\input{childdoc.def}|\\
|\childdocforward[|\textit{main}|]{|\textit{dest}|}|\\
\end{tabular}
\end{center}
%
The argument \textit{dest} is the destination file
(without extension).
It should be the main file or one of the child files.
Note that further \textsf{childdoc} directives
such as |\childdocof| and |\childdocforward|
in the indicated file will be processed in this form.
The optional argument \textit{main}
passes on directly to the main file \textit{main}
while pretending to compile the child \textit{dest}.
This form behaves as if \textit{dest}
issues |\childdocof{|\textit{main}|}| right away,
and no further \textsf{childdoc} directives will be processed.

%%%%%%%%%%%%%%%%%%%%%%%%%%%%%%%%%%%%%%%%
\DescribeMacro{\...prefix}
In the alternative form |\childdocforwardprefix|,
%
\begin{center}
\begin{tabular}{l}
|\input{childdoc.def}|\\
|\childdocforwardprefix[|\textit{main}|]{|\textit{prefix}|}{|\textit{dest}|}|
\end{tabular}
\end{center}
%
the destination file is determined by a pattern
depending on the current file:
To make this work, the current file must be called
`{\textit{prefix}\hspace{0.2em}\textit{suffix}}'
with \textit{prefix} matching precisely the argument.
Processing is then passed on to the file
`{\textit{dest}\hspace{0.2em}\textit{suffix}}'.
Surely, the same effect is achieved by
directly specifying the
argument `{\textit{dest}\hspace{0.2em}\textit{suffix}}'
in the first form.
However, that requires to set up a different file
for each child. With the alternative form of the command
all these files can have exactly the same content
which simplifies setting them up and maintaining them.

For example, the following file |draft.tex|
with a compilation flag |\version| as described in \secref{sec:flags}
compiles the main document as a draft:
%
\begin{center}
\begin{tabular}{l}
|\def\version{draft}|\\
|\input{childdoc.def}|\\
|\childdocforward{|\textit{main}|}|
\end{tabular}
\end{center}
%
Likewise, the following files |final|\textit{nn}|.tex|
compile the final version of the child document
|child|\textit{nn}|.tex|:
%
\begin{center}
\begin{tabular}{l}
|\def\version{final}|\\
|\input{childdoc.def}|\\
|\childdocforwardprefix{final}{child}|
\end{tabular}
\end{center}
%

Note that when several versions of a main file and/or of each child file
are to be generated, it may be convenient to set up a |Makefile| or
shell script to automatise the process.

%%%%%%%%%%%%%%%%%%%%%%%%%%%%%%%%%%%%%%%%%%%%%%%%%%%%%%%%%%%%%%%%%%%%%%%%%%%%%%%%
\subsection{Command Line Processing}
\label{sec:commandline}

The effect of redirection files can also be achieved by invoking
the \LaTeX{} compiler with a more elaborate command line.
Most conveniently this should be done as part
of a shell script or a |Makefile|.

When using \textsf{childdoc} in the main file, the following
command lines effectively perform a redirection
(note that depending on the shell being used,
backslashes may have to be doubled: `|\|' $\to$ `|\\|'):
%
\begin{center}
|... -jobname "|\textit{target}|" |\\|"|[\textit{flags}]%
|\input{childdoc.def}\childdocforward[|\textit{main}|]{|\textit{dest}|}"|
\end{center}
%
Here \textit{target} is the name of the output file,
\textit{main} is the name of the main file
and \textit{dest} is the name of the main or child file to be processed
(all filenames without extensions).
The optional argument \textit{main} can be omitted
if \textit{main} matches \textit{dest}.
Optionally, compilation \textit{flags} can be defined via |\def| commands.
This command line makes the \TeX{} engine believe
it is compiling the file \textit{target}
whose content is specified as the latter parameter.
The provided code then forwards the processing to
\textit{main} or \textit{dest} as described in \secref{sec:forward}.

%%%%%%%%%%%%%%%%%%%%%%%%%%%%%%%%%%%%%%%%%%%%%%%%%%%%%%%%%%%%%%%%%%%%%%%%%%%%%%%%
\subsection{Include by Input}
\label{sec:input}

Including child documents by |\include| has some restrictions by design.
Most notably, the content of a child document always occupies
its own set of pages; pages cannot be shared between child documents.
Usually, this behaviour makes perfect sense
because each child document contain an essential part of the document.
However, in some situations it may be desirable to compose
a document from a collection of parts
without having mandatory page breaks between then.
For this case, the package
provides a mechanism to include parts
by |\input| which can also be processed individually.
However, by construction this mechanism
requires manual handling of the content to be output.

%%%%%%%%%%%%%%%%%%%%%%%%%%%%%%%%%%%%%%%%
\DescribeMacro{\ifchilddocmanual}
The main file should be prepared as usual, see \secref{sec:include}.
However, the document body must make a distinction
between processing of an individual part and of the main document, e.g.:
%
\begin{center}
\begin{tabular}{l}
|\ifchilddocmanual|\\
|\input{\childdocname}|\\
|\||else|\\
\textit{document body with }|\input{|\textit{part}|}|\\
|\||fi|
\end{tabular}
\end{center}
%
The conditional |\ifchilddocmanual| is true whenever
a part to be included by |\input| is being compiled,
and the name of the part is stored in |\childdocname|.

%%%%%%%%%%%%%%%%%%%%%%%%%%%%%%%%%%%%%%%%
\DescribeMacro{\childdocby}
Each part to be included by |\input| should start with:
%
\begin{center}
\begin{tabular}{l}
|\input{childdoc.def}|\\
|\childdocby{|\textit{main}|}|\\
\end{tabular}
\end{center}
%
The directive |\childdocby| is similar to |\childdocof|
described in \secref{sec:include},
but the subsequent selection of content must be done manually.
To that end, both |\ifchilddoc| and |\ifchilddocmanual|
will be true upon processing of a part,
and the name of the part is stored in |\childdocname|.
Note that |\jobname| will be set to the filename of the current part
so that each part receives an individual |.aux| file
that does not interfere with the |.aux| file(s) of the main document.
This behaviour can be altered by the alternative form
|\childdocby[*]{|\textit{main}|}| (with a non-empty optional argument)
which uses the |.aux| file of the main document
by setting |\jobname| to \textit{main}.

%%%%%%%%%%%%%%%%%%%%%%%%%%%%%%%%%%%%%%%%%%%%%%%%%%%%%%%%%%%%%%%%%%%%%%%%%%%%%%%%
\subsection{Driver Development}
\label{sec:driver}

The \textsf{childdoc} mechanism can also be use for the development
of definition files such as \LaTeX{} styles or classes.
This case differs from the above setup with multiple parts
included by |\include| in that no |\includeonly| should be invoked.
This can be achieved by starting the include file
(before |\ProvidesPackage|) with:
%
\begin{center}
\begin{tabular}{l}
|\input{childdoc.def}|\\
|\childdocforward{|\textit{main}|}|\\
\end{tabular}
\end{center}
%
or alternatively with:
%
\begin{center}
\begin{tabular}{l}
|\input{childdoc.def}|\\
|\childdocby{|\textit{main}|}|\\
\end{tabular}
\end{center}
%
Both forms have slightly different effects as described above.
The main file is prepared as usual, see \secref{sec:include}.

%%%%%%%%%%%%%%%%%%%%%%%%%%%%%%%%%%%%%%%%%%%%%%%%%%%%%%%%%%%%%%%%%%%%%%%%%%%%%%%%
\subsection{Legacy Detection}
\label{sec:detection}

The directive |\childdocmain| in the main file can detect
whether the complete document or merely a child is to be compiled
even without using the directive |\childdocof|.
This method is deprecated because it is less robust
and there is no compelling reason to use it;
it is merely provided for backward compatibility
and it may be removed in future versions.

If the detection mechanism is to be used,
it is mandatory to correctly specify
the filename of the main file as the argument of |\childdocmain|:
%
\begin{center}
\begin{tabular}{l}
|\input{childdoc.def}|\\
|\childdocmain{|\textit{main}|}|\\
\end{tabular}
\end{center}
%
If |\jobname| does not match the argument \textit{main} of |\childdocmain|,
it is assumed that |\jobname| points to the child file to be compiled.
When using |\childdocmain| with the main file specified as argument,
it suffices to start a child file
with just |\input{|\textit{main}|}|
without loading of the package and using |\childdocof|.
If instead all processing is done
with the appropriate \textsf{childdoc} directives,
the argument of \textit{main} of |\childdocmain| can be empty.

An alternative version of the command line processing described
in \secref{sec:commandline} using the detection mechanism reads:
%
\begin{center}
|... -jobname "|\textit{target}|" "|[\textit{flags}]%
[|\def\jobname{|\textit{dest}|}|]|\input{|\textit{main}|}"|
\end{center}

%%%%%%%%%%%%%%%%%%%%%%%%%%%%%%%%%%%%%%%%%%%%%%%%%%%%%%%%%%%%%%%%%%%%%%%%%%%%%%%%
\subsection{Manual Code}
\label{sec:manual}

In case one cannot be certain whether the definitions file |childdoc.def|
is installed on the target \TeX{} distribution
and one prefers not to ship it,
it is conceivable to paste a few relevant commands into the sources.

To that end, drop all statements |\input{childdoc.def}|
and perform the replacements as outlined below.
Instead of |\childdocmain{|\textit{main}|}| add the following code
to the top of the main file:
%
\begin{center}
\begin{tabular}{l}
|\||ifdefined\childdocname\endinput\||fi\newif\ifchilddoc|\\
|\edef\childdocname{\scantokens\expandafter{\jobname\noexpand}}|\\
|\def\childdocmain{|\textit{main}|}\||ifx\childdocmain\childdocname\||else|\\
|\childdoctrue\includeonly{\childdocname}\let\jobname\childdocmain\||fi|\\
\end{tabular}
\end{center}
%
Instead of |\childdocof{|\textit{main}|}| just include the main file
at the top of each child file:
%
\begin{center}
|\input{|\textit{main}|}|
\end{center}
%
A simple redirection |\childdocforward{|\textit{dest}|}| is achieved by:
%
\begin{center}
|\def\jobname{|\textit{dest}|}\input{\jobname}|
\end{center}
%
The redirection with prefix
|\childdocforwardprefix[|\textit{prefix}|]{|\textit{dest}|}|
is accomplished by:
%
\begin{center}
\begin{tabular}{l}
|{\edef\jobname{\scantokens\expandafter{\jobname\noexpand}}|\\
|\def\redirectjob |\textit{prefix}|#1~~~{\gdef\jobname{|\textit{dest}|#1}}|\\
|\expandafter\redirectjob\jobname~~~}\input{\jobname}|
\end{tabular}
\end{center}

In an alternative approach,
child documents can be compiled by a specific command line
without additional code or specific definitions:
%
\begin{center}
|... -jobname "|\textit{target}|" "|[\textit{flags}]%
|\includeonly{|\textit{dest}|}\input{|\textit{main}|}"|
\end{center}
%

%%%%%%%%%%%%%%%%%%%%%%%%%%%%%%%%%%%%%%%%%%%%%%%%%%%%%%%%%%%%%%%%%%%%%%%%%%%%%%%%
%%%%%%%%%%%%%%%%%%%%%%%%%%%%%%%%%%%%%%%%%%%%%%%%%%%%%%%%%%%%%%%%%%%%%%%%%%%%%%%%
\section{Information}

%%%%%%%%%%%%%%%%%%%%%%%%%%%%%%%%%%%%%%%%%%%%%%%%%%%%%%%%%%%%%%%%%%%%%%%%%%%%%%%%
\subsection{Copyright}

Copyright \copyright{} 2017--2018 Niklas Beisert

This work may be distributed and/or modified under the
conditions of the \LaTeX{} Project Public License, either version 1.3
of this license or (at your option) any later version.
The latest version of this license is in
  \url{http://www.latex-project.org/lppl.txt}
and version 1.3 or later is part of all distributions of \LaTeX{}
version 2005/12/01 or later.

This work has the LPPL maintenance status `maintained'.

The Current Maintainer of this work is Niklas Beisert.

This work consists of the files |README.txt|, |childdoc.ins| and |childdoc.dtx|
as well as the derived files |childdoc.def|, |cdocsamp.tex|
with |cdocsch1.tex|, |cdocsch2.tex|, |cdocspt3.tex|, |cdocspt4.tex|,
|cdocsdrf.tex|, |cdocsfn1.tex|, |cdocsfn2.tex|
as well as |childdoc.pdf|.

%%%%%%%%%%%%%%%%%%%%%%%%%%%%%%%%%%%%%%%%%%%%%%%%%%%%%%%%%%%%%%%%%%%%%%%%%%%%%%%%
\subsection{Files and Installation}

The package consists of the files:
%
\begin{center}
\begin{tabular}{ll}
    |README.txt|   & readme file \\
    |childdoc.ins| & installation file \\
    |childdoc.dtx| & source file \\
    |childdoc.def| & definition file \\
    |cdocsamp.tex| & sample main file \\
    |cdocsch1.tex| & sample include file \\
    |cdocsch2.tex| & sample include file \\
    |cdocspt3.tex| & sample part file \\
    |cdocspt4.tex| & sample part file \\
    |cdocsdrf.tex| & sample redirection file \\
    |cdocsfn1.tex| & sample redirection file \\
    |cdocsfn2.tex| & sample redirection file \\
    |childdoc.pdf| & manual
\end{tabular}
\end{center}
%
The distribution consists of the files
|README.txt|, |childdoc.ins| and |childdoc.dtx|.
%
\begin{itemize}
\item
Run (pdf)\LaTeX{} on |childdoc.dtx|
to compile the manual |childdoc.pdf| (this file).
\item
Run \LaTeX{} on |childdoc.ins| to create the definitions file |childdoc.def|
and the sample |cdocsamp.tex| with include files
|cdocsch1.tex|, |cdocsch2.tex|, |cdocspt3.tex|, |cdocspt4.tex|,
|cdocsdrf.tex|, |cdocsfn1.tex|, |cdocsfn2.tex|.
Then copy the file |childdoc.def| to an appropriate directory of your \LaTeX{}
distribution, e.g.\ \textit{texmf-root}|/tex/latex/childdoc|.
\end{itemize}

%%%%%%%%%%%%%%%%%%%%%%%%%%%%%%%%%%%%%%%%%%%%%%%%%%%%%%%%%%%%%%%%%%%%%%%%%%%%%%%%
\subsection{Related CTAN Packages}

There are several other packages which offer a similar functionality:
%
\begin{itemize}
\item
The packages
\href{http://ctan.org/pkg/docmute}{\textsf{docmute}},
\href{http://ctan.org/pkg/includex}{\textsf{includex}} and
\href{http://ctan.org/pkg/standalone}{\textsf{standalone}}
provide commands to include only the document body of
a child file thus allowing both files to be compiled individually.
\item
The packages \href{http://ctan.org/pkg/subdocs}{\textsf{subdocs}}
and \href{http://ctan.org/pkg/subfiles}{\textsf{subfiles}}
provide structures in which the main and child documents can be
encapsulated and allowing them to be compiled individually.
The inclusion mechanism is different from the conventional |\include|.
\item
The package \href{http://ctan.org/pkg/combine}{\textsf{combine}}
is an elaborate solution to combine several documents into one.
\end{itemize}
%
See also the CTAN topic \href{http://ctan.org/topic/subdocs}{\textsf{subdocs}}
for further related packages.
The present package differs from the above solutions in that
a document structure constructed with the conventional |\include| mechanism
just needs two extra commands at the top of every file
such that all constituent files can be compiled individually.

%%%%%%%%%%%%%%%%%%%%%%%%%%%%%%%%%%%%%%%%%%%%%%%%%%%%%%%%%%%%%%%%%%%%%%%%%%%%%%%%
%\subsection{Feature Suggestions}
%
%The following is a list of features which may be useful for future
%versions of this package:
%%
%\begin{itemize}
%\item
%\ldots
%\end{itemize}

%%%%%%%%%%%%%%%%%%%%%%%%%%%%%%%%%%%%%%%%%%%%%%%%%%%%%%%%%%%%%%%%%%%%%%%%%%%%%%%%
\subsection{Revision History}

%%%%%%%%%%%%%%%%%%%%%%%%%%%%%%%%%%%%%%%%
\paragraph{v2.0:} 2018/12/30

\begin{itemize}
\item
immediate forward processing
\item
added |\childdocby| mechanism
\item
manual restructured
\end{itemize}

%%%%%%%%%%%%%%%%%%%%%%%%%%%%%%%%%%%%%%%%
\paragraph{v1.6:} 2018/01/17

\begin{itemize}
\item
application for development of include files
\item
corrections to manual
\end{itemize}

%%%%%%%%%%%%%%%%%%%%%%%%%%%%%%%%%%%%%%%%
\paragraph{v1.5:} 2017/05/21

\begin{itemize}
\item
more complete structuring introduced
\item
|\childdocof| introduced
\item
|\childdoc| renamed to |\childdocmain|
\item
|\childredirect| renamed to |\childdocforward| and |\childdocforwardprefix|
and functionality expanded
\end{itemize}

%%%%%%%%%%%%%%%%%%%%%%%%%%%%%%%%%%%%%%%%
\paragraph{v1.0:} 2017/04/27

\begin{itemize}
\item
manual and install package
\item
first version published on CTAN
\end{itemize}

%%%%%%%%%%%%%%%%%%%%%%%%%%%%%%%%%%%%%%%%
\paragraph{v0.6:} 2017/04/26

\begin{itemize}
\item
redirection mechanism added
\end{itemize}

%%%%%%%%%%%%%%%%%%%%%%%%%%%%%%%%%%%%%%%%
\paragraph{v0.5:} 2017/04/26

\begin{itemize}
\item
functionality in definition file
\end{itemize}


%%%%%%%%%%%%%%%%%%%%%%%%%%%%%%%%%%%%%%%%%%%%%%%%%%%%%%%%%%%%%%%%%%%%%%%%%%%%%%%%
%%%%%%%%%%%%%%%%%%%%%%%%%%%%%%%%%%%%%%%%%%%%%%%%%%%%%%%%%%%%%%%%%%%%%%%%%%%%%%%%
%%%%%%%%%%%%%%%%%%%%%%%%%%%%%%%%%%%%%%%%%%%%%%%%%%%%%%%%%%%%%%%%%%%%%%%%%%%%%%%%
\appendix

\settowidth\MacroIndent{\rmfamily\scriptsize 000\ }

 \DocInput{childdoc.dtx}

\end{document}
%</driver>
% \fi
%
% %%%%%%%%%%%%%%%%%%%%%%%%%%%%%%%%%%%%%%%%%%%%%%%%%%%%%%%%%%%%%%%%%%%%%%%%%%%%%%
% %%%%%%%%%%%%%%%%%%%%%%%%%%%%%%%%%%%%%%%%%%%%%%%%%%%%%%%%%%%%%%%%%%%%%%%%%%%%%%
% \section{Sample}
%\iffalse
%<*samplemain>
%\fi
%
% The following presents a sample document
% with two chapters, two parts, a title page,
% a compile flag as well as three forwarding files to set the flag.
% It consists of eight |.tex| files:
% \begin{center}
% \begin{tabular}{ll}
% |cdocsamp.tex|&main file\\
% |cdocsch1.tex|&include file for chapter 1\\
% |cdocsch2.tex|&include file for chapter 2\\
% |cdocspt3.tex|&include file for part 3\\
% |cdocspt4.tex|&include file for part 4\\
% |cdocsdrf.tex|&forwarding file for main file in draft mode\\
% |cdocsfi1.tex|&forwarding file for final version of chapter 1\\
% |cdocsfi2.tex|&forwarding file for final version of chapter 2\\
% \end{tabular}
% \end{center}
% Each of the eight files can be compiled directly by the \LaTeX{} compiler.
%
% %%%%%%%%%%%%%%%%%%%%%%%%%%%%%%%%%%%%%%
% \paragraph{Main File.}
%
% The main file is called |cdocsamp.tex|.
%
% Load the \textsf{childdoc} definitions and
% declare the filename for the main document:
%    \begin{macrocode}
\input{childdoc.def}
\childdocmain{}
%    \end{macrocode}

% Optional override for |\version| flag:
%    \begin{macrocode}
%%\ifchilddoc\else\providecommand{\version}{draft}\fi
%    \end{macrocode}

% Define the default values for the |\version| flag
% (|final| for the main file and |draft| for childs):
%    \begin{macrocode}
\ifchilddoc
\providecommand{\version}{draft}
\else
\providecommand{\version}{final}
\fi
%    \end{macrocode}

% Load the standard document class:
%    \begin{macrocode}
\documentclass[12pt]{article}
%    \end{macrocode}

% Start the document body:
%    \begin{macrocode}
\begin{document}
%    \end{macrocode}

% Declare a title page.
% Print title, part of document being processed and version flag:
%    \begin{macrocode}
\addtocounter{page}{-1}
\begin{center}
{\LARGE\bfseries{}childdoc example\par}
\vspace{1cm}
\ifchilddoc
\ifchilddocmanual part\else chapter\fi:
`\childdocname' of `\childdocjob'\par
\else
main document: `\childdocjob'\par
\fi
version: \version\par
\end{center}
\newpage
%    \end{macrocode}

% Manually include selected file,
% otherwise process as usual:
%    \begin{macrocode}
\ifchilddocmanual
\section*{part `\childdocname'}
\input{\childdocname}
\else
%    \end{macrocode}

% Include the two chapters:
%    \begin{macrocode}
\include{cdocsch1}
\include{cdocsch2}
%    \end{macrocode}

% Include the two parts unless only chapters should be displayed:
%    \begin{macrocode}
\ifchilddoc\else
\section{part three}
\input{cdocspt3}
\section{part four}
\input{cdocspt4}
\fi
%    \end{macrocode}

% Process as usual until here:
%    \begin{macrocode}
\fi
%    \end{macrocode}

% End of document body:
%    \begin{macrocode}
\end{document}
%    \end{macrocode}
%\iffalse
%</samplemain>
%\fi
%
% %%%%%%%%%%%%%%%%%%%%%%%%%%%%%%%%%%%%%%
% \paragraph{Chapter Include Files.}
%
% The include files are called |cdocsch1.tex| and |cdocsch2.tex|.
%
%\iffalse
%<*samplechap1|samplechap2>
%\fi

% Optional override for |\version| flag:
%    \begin{macrocode}
%%\providecommand{\version}{final}
%    \end{macrocode}

% Include the main document:
%    \begin{macrocode}
\input{childdoc.def}
\childdocof{cdocsamp}
%    \end{macrocode}

%\iffalse
%</samplechap1|samplechap2>
%\fi
%
%\iffalse
%<*samplechap1>
%\fi
% Some text for chapter 1:
%    \begin{macrocode}
\section{one}
some text in chapter one
%    \end{macrocode}

%\iffalse
%</samplechap1>
%\fi
% Some text for chapter 2:
%\iffalse
%<*samplechap2>
%\fi
%    \begin{macrocode}
\section{two}
more text in chapter two
%    \end{macrocode}

%\iffalse
%</samplechap2>
%\fi
%
% %%%%%%%%%%%%%%%%%%%%%%%%%%%%%%%%%%%%%%
% \paragraph{Part Include Files.}
%
% The include files are called |cdocspt3.tex| and |cdocspt4.tex|.
%
%\iffalse
%<*samplepart3|samplepart4>
%\fi

% Optional override for |\version| flag:
%    \begin{macrocode}
%%\providecommand{\version}{final}
%    \end{macrocode}

% Include the main document:
%    \begin{macrocode}
\input{childdoc.def}
\childdocby{cdocsamp}
%    \end{macrocode}

%\iffalse
%</samplepart3|samplepart4>
%\fi
%
%\iffalse
%<*samplepart3>
%\fi
% Some text for part 3:
%    \begin{macrocode}
some text in part three
%    \end{macrocode}

%\iffalse
%</samplepart3>
%\fi
% Some text for part 4:
%\iffalse
%<*samplepart4>
%\fi
%    \begin{macrocode}
more text in part four
%    \end{macrocode}

%\iffalse
%</samplepart4>
%\fi
%
% %%%%%%%%%%%%%%%%%%%%%%%%%%%%%%%%%%%%%%
% \paragraph{Forwarding for a Complete Draft.}
%
% The following forwarding file |cdocsdrf.tex|
% compiles the main document in draft mode:
%\iffalse
%<*sampledraft>
%\fi
%    \begin{macrocode}
\def\version{draft}
\input{childdoc.def}
\childdocforward{cdocsamp}
%    \end{macrocode}

%\iffalse
%</sampledraft>
%\fi
%
% %%%%%%%%%%%%%%%%%%%%%%%%%%%%%%%%%%%%%%
% \paragraph{Forwarding for Final Version of the Chapters.}
%
% The following forwarding files |cdocsfn1.tex| and |cdocsfn2.tex|
% (with identical content)
% compile the final versions of the child documents
% |cdocsch1.tex| and |cdocsch2.tex|, respectively:
%\iffalse
%<*samplefinal>
%\fi
%    \begin{macrocode}
\def\version{final}
\input{childdoc.def}
\childdocforwardprefix[cdocsamp]{cdocsfn}{cdocsch}
%    \end{macrocode}

%\iffalse
%</samplefinal>
%\fi
%
% %%%%%%%%%%%%%%%%%%%%%%%%%%%%%%%%%%%%%%
% \paragraph{Command Line Processing.}
%
% The following three command lines generate the output files
% |cdocscld|, |cdocscl1| and |cdocscl2|
% which should be identical to
% |cdocsdrf|, |cdocsch1| and |cdocsfn2|, respectively:
% \begin{center}
% \begin{tabular}{l}
% |latex -jobname cdocscld \|\\
% |  "\def\version{draft}\input{childdoc.def}\childdocforward{cdocsamp}"|\\
% |latex -jobname cdocscl1 \|\\
% |  "\input{childdoc.def}\childdocforward[cdocsamp]{cdocsch1}"|\\
% |latex -jobname cdocscl2 \|\\
% |  "\def\version{final}\input{childdoc.def}\childdocforward{cdocsch2}"|
% \end{tabular}
% \end{center}
% Note that the trailing backslash on each first line
% merely continues the input to the second line
% (for convenient cut ant paste).
% Furthermore, the command |latex| can be replaced by any
% of its alternative versions such as |pdflatex|.
%
% %%%%%%%%%%%%%%%%%%%%%%%%%%%%%%%%%%%%%%%%%%%%%%%%%%%%%%%%%%%%%%%%%%%%%%%%%%%%%%
% %%%%%%%%%%%%%%%%%%%%%%%%%%%%%%%%%%%%%%%%%%%%%%%%%%%%%%%%%%%%%%%%%%%%%%%%%%%%%%
% \section{Implementation}
%\iffalse
%<*package>
%\fi
%
% This section describes the definitions file |childdoc.def|.

% The definitions cannot be loaded using |\usepackage| or |\RequirePackage|
% which has a mechanism to prevent loading a style file more than once.
% When loading the definitions by means of |\input|
% multiple instances have to be prevented manually:
%\iffalse
%This code needs to be before the `\ProvidesFile' directive
%which is defined at the beginning of this file.
%Therefore it is also placed there and commented out here.
%</package>
%<*discard>
%\fi
%    \begin{macrocode}
\ifdefined\childdocmain\endinput\fi
%    \end{macrocode}
%\iffalse
%</discard>
%<*package>
%\fi
%
% \macro{\ifchilddoc}
% \macro{\ifchilddocmanual}
% The conditional |\ifchilddoc| tells whether a
% child (true) or main (false) document is being compiled.
% The conditional |\ifchilddocmanual| tells whether
% the |\includeonly| mechanism is used (false) or
% the selection of child files must be performed manually (true).
% The definitions initialise to false:
%    \begin{macrocode}
\newif\ifchilddoc
\newif\ifchilddocmanual
%    \end{macrocode}

% \macro{\childdocname}
% \macro{\childdocjob}
% The macro |\childdocname| stores the name of the main document
% to be compiled. The macro |\childdocjob| stores the name of
% the document on which the \LaTeX{} compiler was originally invoked.
% The content of |\jobname| cannot be compared
% to filenames specified in the source due to different catcodes.
% The following code rescans |\jobname|, stores the result
% in |\childdocname| and saves a copy in |\childdocjob|:
%    \begin{macrocode}
\edef\childdocname{\scantokens\expandafter{\jobname\noexpand}}
\let\childdocjob\childdocname
%    \end{macrocode}

% \macro{\childdocdisable}
% The macro |\childdocdisable| prevents the main file
% from being processed more than once.
% At this stage, the main document command |\childdocmain|
% is assumed to be called once again where it should do nothing.
% Any subsequent call to it should prevent
% a secondary processing of the main document
% It overwrites the forwarding commands
% |\childdocof| and |\childdocforward|
% with empty macros to prevent further inclusions of the main document:
%    \begin{macrocode}
\newcommand{\childdocdisable}
{
  \renewcommand{\childdocmain}[1]{\renewcommand{\childdocmain}[1]{\endinput}}
  \renewcommand{\childdocof}[1]{}
  \renewcommand{\childdocby}[2][]{}
  \renewcommand{\childdocforward}[2][]{}
  \renewcommand{\childdocdisable}{}
}
%    \end{macrocode}

% \macro{\childdocmain}
% The macro |\childdocmain| is to be called at the top of the main file
% with nothing or the main filename (without extension) as argument.
% First, it breaks loops.
% If the argument is not empty and does not match |\childdocname|
% (which is set by the first inclusion of |childdoc.def|),
% |\ifchilddoc| is set to true, |\includeonly| is applied to the child file
% and |\jobname| is set to the main file
% (for proper handling of |.aux| files):
%    \begin{macrocode}
\newcommand{\childdocmain}[1]
{
  \childdocdisable\childdocmain{}
  \if?#1?\else
    \begingroup
      \def\childdoctmp{#1}
      \ifx\childdoctmp\childdocname
        \def\childdoctmp{}
      \else
        \def\childdoctmp
        {
          \childdoctrue
          \includeonly{\childdocname}
          \def\childdocjob{#1}
          \def\jobname{#1}
        }
      \fi
      \expandafter
    \endgroup
    \childdoctmp
  \fi
}
%    \end{macrocode}

% \macro{\childdocof}
% The command |\childdocof| redirects
% compilation to the main file |#1|.
%    \begin{macrocode}
\newcommand{\childdocof}[1]
{
  \childdocdisable
  \childdoctrue
  \includeonly{\childdocname}
  \def\jobname{#1}
  \def\childdocjob{#1}
  \input{#1}
}
%    \end{macrocode}

% \macro{\childdocby}
% The command |\childdocby| ....
%    \begin{macrocode}
\newcommand{\childdocby}[2][]
{
  \childdocdisable
  \childdoctrue
  \childdocmanualtrue
  \if?#1?\else
    \def\jobname{#2}
  \fi
  \def\childdocjob{#2}
  \input{#2}
  \endinput
}
%    \end{macrocode}

% \macro{\childdocforward}
% The command |\childdocforward| redirects
% compilation to the main file or
% (if the optional argument is given) a child file.
% Parameters are set as if the main file
% or a child file starting with |\childdocof| was compiled.
% Then compilation is handed over to the main file:
%    \begin{macrocode}
\newcommand{\childdocforward}[2][]
{
  \begingroup
    \if?#1?
      \def\childdoctmp
      {
        \def\childdocname{#2}
        \def\childdocjob{#2}
        \def\jobname{#2}
        \input{#2}
        \endinput
      }
    \else
      \def\childdoctmp
      {
        \childdocdisable
        \def\childdocname{#2}
        \childdoctrue
        \includeonly{#2}
        \def\childdocjob{#1}
        \def\jobname{#1}
        \input{#1}
        \endinput
      }
    \fi
    \expandafter
  \endgroup
  \childdoctmp
}
%    \end{macrocode}

% \macro{\childdocforwardprefix}
% The command |\childdocforwardprefix| redirects
% compilation to the main or a child file by means of a pattern.
% The prefix |#1| in the current filename is replaced by |#2|
% and the suffix of the current filename is kept
% (it is assumed that the filename does not contain the substring `|~~~|'
% which is used as a delimiter).
% Compilation is handed over to the new file by |\childdocforward|:
%    \begin{macrocode}
\newcommand{\childdocforwardprefix}[3][]
{
  \begingroup
    \def\childdocextract #2##1~~~{\def\childdoctmp{\childdocforward[#1]{#3##1}}}
    \expandafter\childdocextract\childdocname~~~
    \expandafter
  \endgroup
  \childdoctmp
}
%    \end{macrocode}

% \macro{\childdoc}
% The deprecated macro |\childdoc| is a legacy version of |\childdocmain|:
%    \begin{macrocode}
\newcommand{\childdoc}{\childdocmain}
%    \end{macrocode}

% \macro{\childdocredirect}
% The deprecated macro |\childdocredirect| is a legacy version
% of |\childdocforward| and |\childdocforwardprefix|:
%    \begin{macrocode}
\newcommand{\childdocredirect}[2][]
{
  \begingroup
    \if?#1?
      \def\childdoctmp{\childdocforward{#2}}
    \else
      \def\childdoctmp{\childdocforwardprefix{#1}{#2}}
    \fi
    \expandafter
  \endgroup
  \childdoctmp
}
%    \end{macrocode}

%\iffalse
%</package>
%\fi
%
\endinput
|\\
|\childdocby{|\textit{main}|}|\\
\end{tabular}
\end{center}
%
The directive |\childdocby| is similar to |\childdocof|
described in \secref{sec:include},
but the subsequent selection of content must be done manually.
To that end, both |\ifchilddoc| and |\ifchilddocmanual|
will be true upon processing of a part,
and the name of the part is stored in |\childdocname|.
Note that |\jobname| will be set to the filename of the current part
so that each part receives an individual |.aux| file
that does not interfere with the |.aux| file(s) of the main document.
This behaviour can be altered by the alternative form
|\childdocby[*]{|\textit{main}|}| (with a non-empty optional argument)
which uses the |.aux| file of the main document
by setting |\jobname| to \textit{main}.

%%%%%%%%%%%%%%%%%%%%%%%%%%%%%%%%%%%%%%%%%%%%%%%%%%%%%%%%%%%%%%%%%%%%%%%%%%%%%%%%
\subsection{Driver Development}
\label{sec:driver}

The \textsf{childdoc} mechanism can also be use for the development
of definition files such as \LaTeX{} styles or classes.
This case differs from the above setup with multiple parts
included by |\include| in that no |\includeonly| should be invoked.
This can be achieved by starting the include file
(before |\ProvidesPackage|) with:
%
\begin{center}
\begin{tabular}{l}
|% \iffalse
%
% childdoc.dtx Copyright (C) 2017-2018 Niklas Beisert
%
% This work may be distributed and/or modified under the
% conditions of the LaTeX Project Public License, either version 1.3
% of this license or (at your option) any later version.
% The latest version of this license is in
%   http://www.latex-project.org/lppl.txt
% and version 1.3 or later is part of all distributions of LaTeX
% version 2005/12/01 or later.
%
% This work has the LPPL maintenance status `maintained'.
%
% The Current Maintainer of this work is Niklas Beisert.
%
% This work consists of the files childdoc.dtx and childdoc.ins
% and the derived files childdoc.def and cdocsamp.tex with
% cdocsch1.tex, cdocsch2.tex, cdocsdrf.tex, cdocsfn1.tex, cdocsfn2.tex.
%
%<package>\ifdefined\childdocmain\endinput\fi
%<package>\ProvidesFile{childdoc.def}[2018/12/30 v2.0 child document driver]
%<samplemain>\ProvidesFile{cdocsamp.tex}[2018/12/30 v2.0 sample for childdoc]
%<*driver>
%\ProvidesFile{childdoc.drv}[2018/12/30 v2.0 childdoc reference manual file]
\PassOptionsToClass{10pt,a4paper}{article}
\documentclass{ltxdoc}

\usepackage[margin=35mm]{geometry}
\usepackage{hyperref}
\usepackage{hyperxmp}
\usepackage[usenames]{color}

\hypersetup{colorlinks=true}
\hypersetup{pdfstartview=FitH}
\hypersetup{pdfpagemode=UseNone}
\hypersetup{pdfsource={}}
\hypersetup{pdflang={en-UK}}
\hypersetup{pdfcopyright={Copyright 2017-2018 Niklas Beisert.
  This work may be distributed and/or modified under the
  conditions of the LaTeX Project Public License, either version 1.3
  of this license or (at your option) any later version.}}
\hypersetup{pdflicenseurl={http://www.latex-project.org/lppl.txt}}
\hypersetup{pdfcontactaddress={ETH Zurich, ITP, HIT K,
  Wolfgang-Pauli-Strasse 27}}
\hypersetup{pdfcontactpostcode={8093}}
\hypersetup{pdfcontactcity={Zurich}}
\hypersetup{pdfcontactcountry={Switzerland}}
\hypersetup{pdfcontactemail={nbeisert@itp.phys.ethz.ch}}
\hypersetup{pdfcontacturl={http://people.phys.ethz.ch/\xmptilde nbeisert/}}

\newcommand{\secref}[1]{\hyperref[#1]{section \ref*{#1}}}

\parskip1ex
\parindent0pt
\let\olditemize\itemize
\def\itemize{\olditemize\parskip0pt}

\begin{document}

\title{The \textsf{childdoc} Package}
\hypersetup{pdftitle={The childdoc Package}}
\author{Niklas Beisert\\[2ex]
  Institut f\"ur Theoretische Physik\\
  Eidgen\"ossische Technische Hochschule Z\"urich\\
  Wolfgang-Pauli-Strasse 27, 8093 Z\"urich, Switzerland\\[1ex]
  \href{mailto:nbeisert@itp.phys.ethz.ch}
  {\texttt{nbeisert@itp.phys.ethz.ch}}}
\hypersetup{pdfauthor={Niklas Beisert}}
\hypersetup{pdfsubject={Manual for the LaTeX2e Package childdoc}}
\date{30 December 2018, \textsf{v2.0}}
\maketitle

\begin{abstract}\noindent
\textsf{childdoc} is a \LaTeXe{} package
that enables the direct compilation
of document sections included by |\include|
to individual files.
\end{abstract}

\begingroup
\parskip0ex
\tableofcontents
\endgroup

%%%%%%%%%%%%%%%%%%%%%%%%%%%%%%%%%%%%%%%%%%%%%%%%%%%%%%%%%%%%%%%%%%%%%%%%%%%%%%%%
%%%%%%%%%%%%%%%%%%%%%%%%%%%%%%%%%%%%%%%%%%%%%%%%%%%%%%%%%%%%%%%%%%%%%%%%%%%%%%%%
\section{Introduction}

\LaTeX{} provides a mechanism to structure a large document (such as a book)
into a main file and several child files (containing the chapters)
using the |\include| command.
This mechanism is beneficial for documents
which span hundreds of pages in order to
make the source file(s) more manageable.
Moreover, compilation can be restricted to
selected child files by means of the |\includeonly| command.
The latter feature can be used to reduce the compilation time while editing
(this was significantly more useful in the earlier days of \LaTeX{})
or to generate a smaller document which is easier to navigate.
Another application of |\includeonly| is to generate
documents consisting of selected parts of the complete document.

However, there are a few drawbacks of the plain |\include| mechanism:
\begin{itemize}
\item
The child files cannot be compiled on their own,
they can only be compiled via the main file.
A naive editing environment
(such as a text editor with an option
to have the current file processed by \LaTeX)
may require one to switch to the main file before compiling;
attempting to compile the child file produces errors.
\item
The main file must be modified (each time)
to adjust the |\includeonly| command
to the present needs. This easily leaves the main file in a messy state.
\item
The generated document will always carry the filename
of the main document. This is inconvenient if
several child files are to be compiled and
to be kept for distribution.
\end{itemize}

The present package provides a simple interface
to make child files individually compilable by \LaTeX{}.
Compiling a child file then has the same effect as compiling
the main file with an |\includeonly| command
to select the appropriate child.
Moreover the generated document will carry the name of the child
rather than the main file.
This resolves all three above issues.

This feature is meant to make the editing of books,
thesis documents and lecture notes somewhat more convenient.
However, the package can also be used efficiently for
composing a series of documents (such as exercise sheets)
which are typically distributed individually.
It then assists the author in generating the individual documents
(potentially in different versions)
as well as a document containing the collected series.
Another application is in developing style files
or other kinds of included material
where compilation of the style file could redirect
to a sample or test file.

%%%%%%%%%%%%%%%%%%%%%%%%%%%%%%%%%%%%%%%%%%%%%%%%%%%%%%%%%%%%%%%%%%%%%%%%%%%%%%%%
%%%%%%%%%%%%%%%%%%%%%%%%%%%%%%%%%%%%%%%%%%%%%%%%%%%%%%%%%%%%%%%%%%%%%%%%%%%%%%%%
\section{Usage}

First of all, the package \textsf{childdoc} is \emph{not} a standard
\LaTeXe{} |.sty| style file! Therefore it needs to be invoked in
a non-standard way.

%%%%%%%%%%%%%%%%%%%%%%%%%%%%%%%%%%%%%%%%%%%%%%%%%%%%%%%%%%%%%%%%%%%%%%%%%%%%%%%%
\subsection{Included Files}
\label{sec:include}

%%%%%%%%%%%%%%%%%%%%%%%%%%%%%%%%%%%%%%%%
\DescribeMacro{\childdocmain}
To use the package, add the commands
\begin{center}
\begin{tabular}{l}
|\input{childdoc.def}|\\
|\childdocmain{}|\\
\end{tabular}
\end{center}
at the very top of the main \LaTeX{} file,
in particular \emph{before} the |\documentclass| statement!
The argument of |\childdocmain| should be left empty
(but it must be present).

%%%%%%%%%%%%%%%%%%%%%%%%%%%%%%%%%%%%%%%%
\DescribeMacro{\childdocof}
Furthermore, add the commands
\begin{center}
\begin{tabular}{l}
|\input{childdoc.def}|\\
|\childdocof{|\textit{main}|}|\\
\end{tabular}
\end{center}
at the top of every child file \textit{child}
which is included by |\include{|\textit{child}|}|
from within the main file
(or at least for those files to be compiled individually).
The argument \textit{main} must be the filename of the main file.

There are a couple of
considerations in setting up the main and child documents:

%%%%%%%%%%%%%%%%%%%%%%%%%%%%%%%%%%%%%%%%
\paragraph{Restrictions.}

Please note the following restrictions:
\begin{itemize}
\item
|\childdocmain| must be called with one argument \textit{main}
to ensure compatibility with earlier version of the package.
It must either be empty (|\childdocmain{}|)
or precisely match the filename of the main file in which it is specified.
See \secref{sec:detection} for further information.
\item
The filename \textit{main} must be specified without the |.tex| extension.
\item
The filename \textit{main} is case sensitive
(even in case-insensitive file systems)
due to internal string comparison.
\item
The argument \textit{main} should be fully expanded, it cannot be a macro.
\item
Subdirectories and special characters should be avoided in filenames.
\item
The command |\childdocmain{|\textit{main}|}| must be followed by a whitespace.
It should not be followed immediately by another command
or by a comment mark `|%|'.
This is because the \TeX{} parser reads the token immediately following
the argument of |\childdocmain| and puts it
at the beginning of every child section;
however, a white\-space is ignored.
\end{itemize}

%%%%%%%%%%%%%%%%%%%%%%%%%%%%%%%%%%%%%%%%
\paragraph{Content of Main File.}

It is advisable to place all content in the child files included by |\include|.
Any output contained in the main file will appear in all child documents
unless suppressed manually;
it cannot be suppressed automatically by the |\includeonly| directive
and thus should normally be avoided.
A method to include some content in the main file
by means of conditional processing is described in \secref{sec:conditional}.

%%%%%%%%%%%%%%%%%%%%%%%%%%%%%%%%%%%%%%%%
\paragraph{Page Numbering.}

When only a part of the document is compiled,
the appropriate numbering of pages
(as well as other status parameters)
is determined from the |.aux| files.
The latter contain information from previous passes.
However this information needs to propagate through
all intermediate child documents.
Therefore the page numbering in child documents may well
be inconsistent until the complete document is compiled at least once.

A useful (if unconventional) way to always ensure a consistent
page numbering is to restart the numbering in each child document
and denote the pages by `\textit{child}|.|\textit{page}'
where \textit{child} represents the chapter/section number of the child file.
This can be achieved by the command
|\numberwithin{page}{|\textit{child}|}|
of the \textsf{amsmath} package
where \textit{child} can be |chapter| or |section|
depending on the chosen structuring.
Alternatively, one can modify the macro |\thepage| appropriately
and reset the counter |page| at the start of each child file.

%%%%%%%%%%%%%%%%%%%%%%%%%%%%%%%%%%%%%%%%%%%%%%%%%%%%%%%%%%%%%%%%%%%%%%%%%%%%%%%%
\subsection{Conditional Processing}
\label{sec:conditional}

The package provides a mechanism to compile different versions
of a document. To customise the versions further some conditional processing
can come in handy to distinguish which version is being compiled.
The package provides two macros to describe the compilation context:

%%%%%%%%%%%%%%%%%%%%%%%%%%%%%%%%%%%%%%%%
\DescribeMacro{\ifchilddoc}
The conditional |\ifchilddoc| distinguishes between the compilation of
child documents and the main document:
%
\begin{center}
|\ifchilddoc |\textit{child-code}| |[|\||else |\textit{main-code}]| \||fi|
\end{center}

%%%%%%%%%%%%%%%%%%%%%%%%%%%%%%%%%%%%%%%%
\DescribeMacro{\childdocname}
\DescribeMacro{\childdocjob}
The macro |\childdocname| contains the filename (without extension)
of the main or child file being processed.
Note that |\childdocjob| will always contain the name of the main file.

%%%%%%%%%%%%%%%%%%%%%%%%%%%%%%%%%%%%%%%%
\paragraph{Title Page.}

Conditional processing can be used to include a title or banner page
in the main document when proper precautions are taken.
Importantly, the code in the main file should ensure that the page counter
(as well as other status parameters which are stored in the |.aux| files)
takes the same value after the conditional processing.
Otherwise the page numbers may take divergent values
depending on which part is compiled.

For example, a title page could be declared by:
%
\begin{center}
\begin{tabular}{l}
|\ifchilddoc\||else|\\
|\addtocounter{page}{-1}|\\
\textit{code for title page}\\
|\newpage|\\
|\||fi|
\end{tabular}
\end{center}
%
A banner page for the child documents can be generated by:
%
\begin{center}
\begin{tabular}{l}
|\ifchilddoc|\\
|\addtocounter{page}{-1}|\\
\textit{code for banner page}\\
|\newpage|\\
|\||fi|
\end{tabular}
\end{center}
%
Here one could write a message such as:
\begin{center}
|This is the part \childdocname{} of \childdocjob{}.|
\end{center}

%%%%%%%%%%%%%%%%%%%%%%%%%%%%%%%%%%%%%%%%%%%%%%%%%%%%%%%%%%%%%%%%%%%%%%%%%%%%%%%%
\subsection{Flags}
\label{sec:flags}

The package makes it easy to generate different versions
of the main or child documents.
To this end compilation flags can be defined
and assigned different default values.
They will be particularly useful in conjunction
with the forwarding mechanism described in \secref{sec:forward}.

For example, it may be useful to have a flag |\version|
which can be set to |draft| or |final|.
The document source will contain some conditional code
depending on the value of |\version|.
Suppose further, the flag should default to |final| for the main file
and to |draft| for child files
which is a natural assignment for editing the document.
This is achieved by placing the following code
in the preamble of the main document
(below the |\childdocmain| directive):
%
\begin{center}
\begin{tabular}{l}
|\ifchilddoc|\\
|\providecommand{\version}{draft}|\\
|\||else|\\
|\providecommand{\version}{final}|\\
|\||fi|
\end{tabular}
\end{center}
%
The definition by |\providecommand| makes sure
that previous definitions are not overwritten.
Further statements |\providecommand{\version}{...}|
can thus be added before the above code to override it.

For the main file, one might add a line
(between |\childdocmain| and the above block)
%
\begin{center}
|%\ifchilddoc\||else\providecommand{\version}{draft}\||fi|
\end{center}
%
which can be uncommented to produce a draft version.
Likewise one can add a line to the very top of a child file
(above the |\childdocof{|\textit{main}|}| directive)
%
\begin{center}
|%\providecommand{\version}{final}|
\end{center}
%
which can be uncommented to produce the final version of this child document.

%%%%%%%%%%%%%%%%%%%%%%%%%%%%%%%%%%%%%%%%%%%%%%%%%%%%%%%%%%%%%%%%%%%%%%%%%%%%%%%%
\subsection{Forwarding}
\label{sec:forward}

Different versions of the main or child documents
using compilation flags as described in \secref{sec:flags}
can be (permanently) stored in different files
for convenient compilation, viewing and distribution.
To this end, the package defines a command
to pass on compilation to a different file:

%%%%%%%%%%%%%%%%%%%%%%%%%%%%%%%%%%%%%%%%
\DescribeMacro{\childdocforward}
The command |\childdocforward| redirects processing to
another source file:
%
\begin{center}
\begin{tabular}{l}
|\input{childdoc.def}|\\
|\childdocforward[|\textit{main}|]{|\textit{dest}|}|\\
\end{tabular}
\end{center}
%
The argument \textit{dest} is the destination file
(without extension).
It should be the main file or one of the child files.
Note that further \textsf{childdoc} directives
such as |\childdocof| and |\childdocforward|
in the indicated file will be processed in this form.
The optional argument \textit{main}
passes on directly to the main file \textit{main}
while pretending to compile the child \textit{dest}.
This form behaves as if \textit{dest}
issues |\childdocof{|\textit{main}|}| right away,
and no further \textsf{childdoc} directives will be processed.

%%%%%%%%%%%%%%%%%%%%%%%%%%%%%%%%%%%%%%%%
\DescribeMacro{\...prefix}
In the alternative form |\childdocforwardprefix|,
%
\begin{center}
\begin{tabular}{l}
|\input{childdoc.def}|\\
|\childdocforwardprefix[|\textit{main}|]{|\textit{prefix}|}{|\textit{dest}|}|
\end{tabular}
\end{center}
%
the destination file is determined by a pattern
depending on the current file:
To make this work, the current file must be called
`{\textit{prefix}\hspace{0.2em}\textit{suffix}}'
with \textit{prefix} matching precisely the argument.
Processing is then passed on to the file
`{\textit{dest}\hspace{0.2em}\textit{suffix}}'.
Surely, the same effect is achieved by
directly specifying the
argument `{\textit{dest}\hspace{0.2em}\textit{suffix}}'
in the first form.
However, that requires to set up a different file
for each child. With the alternative form of the command
all these files can have exactly the same content
which simplifies setting them up and maintaining them.

For example, the following file |draft.tex|
with a compilation flag |\version| as described in \secref{sec:flags}
compiles the main document as a draft:
%
\begin{center}
\begin{tabular}{l}
|\def\version{draft}|\\
|\input{childdoc.def}|\\
|\childdocforward{|\textit{main}|}|
\end{tabular}
\end{center}
%
Likewise, the following files |final|\textit{nn}|.tex|
compile the final version of the child document
|child|\textit{nn}|.tex|:
%
\begin{center}
\begin{tabular}{l}
|\def\version{final}|\\
|\input{childdoc.def}|\\
|\childdocforwardprefix{final}{child}|
\end{tabular}
\end{center}
%

Note that when several versions of a main file and/or of each child file
are to be generated, it may be convenient to set up a |Makefile| or
shell script to automatise the process.

%%%%%%%%%%%%%%%%%%%%%%%%%%%%%%%%%%%%%%%%%%%%%%%%%%%%%%%%%%%%%%%%%%%%%%%%%%%%%%%%
\subsection{Command Line Processing}
\label{sec:commandline}

The effect of redirection files can also be achieved by invoking
the \LaTeX{} compiler with a more elaborate command line.
Most conveniently this should be done as part
of a shell script or a |Makefile|.

When using \textsf{childdoc} in the main file, the following
command lines effectively perform a redirection
(note that depending on the shell being used,
backslashes may have to be doubled: `|\|' $\to$ `|\\|'):
%
\begin{center}
|... -jobname "|\textit{target}|" |\\|"|[\textit{flags}]%
|\input{childdoc.def}\childdocforward[|\textit{main}|]{|\textit{dest}|}"|
\end{center}
%
Here \textit{target} is the name of the output file,
\textit{main} is the name of the main file
and \textit{dest} is the name of the main or child file to be processed
(all filenames without extensions).
The optional argument \textit{main} can be omitted
if \textit{main} matches \textit{dest}.
Optionally, compilation \textit{flags} can be defined via |\def| commands.
This command line makes the \TeX{} engine believe
it is compiling the file \textit{target}
whose content is specified as the latter parameter.
The provided code then forwards the processing to
\textit{main} or \textit{dest} as described in \secref{sec:forward}.

%%%%%%%%%%%%%%%%%%%%%%%%%%%%%%%%%%%%%%%%%%%%%%%%%%%%%%%%%%%%%%%%%%%%%%%%%%%%%%%%
\subsection{Include by Input}
\label{sec:input}

Including child documents by |\include| has some restrictions by design.
Most notably, the content of a child document always occupies
its own set of pages; pages cannot be shared between child documents.
Usually, this behaviour makes perfect sense
because each child document contain an essential part of the document.
However, in some situations it may be desirable to compose
a document from a collection of parts
without having mandatory page breaks between then.
For this case, the package
provides a mechanism to include parts
by |\input| which can also be processed individually.
However, by construction this mechanism
requires manual handling of the content to be output.

%%%%%%%%%%%%%%%%%%%%%%%%%%%%%%%%%%%%%%%%
\DescribeMacro{\ifchilddocmanual}
The main file should be prepared as usual, see \secref{sec:include}.
However, the document body must make a distinction
between processing of an individual part and of the main document, e.g.:
%
\begin{center}
\begin{tabular}{l}
|\ifchilddocmanual|\\
|\input{\childdocname}|\\
|\||else|\\
\textit{document body with }|\input{|\textit{part}|}|\\
|\||fi|
\end{tabular}
\end{center}
%
The conditional |\ifchilddocmanual| is true whenever
a part to be included by |\input| is being compiled,
and the name of the part is stored in |\childdocname|.

%%%%%%%%%%%%%%%%%%%%%%%%%%%%%%%%%%%%%%%%
\DescribeMacro{\childdocby}
Each part to be included by |\input| should start with:
%
\begin{center}
\begin{tabular}{l}
|\input{childdoc.def}|\\
|\childdocby{|\textit{main}|}|\\
\end{tabular}
\end{center}
%
The directive |\childdocby| is similar to |\childdocof|
described in \secref{sec:include},
but the subsequent selection of content must be done manually.
To that end, both |\ifchilddoc| and |\ifchilddocmanual|
will be true upon processing of a part,
and the name of the part is stored in |\childdocname|.
Note that |\jobname| will be set to the filename of the current part
so that each part receives an individual |.aux| file
that does not interfere with the |.aux| file(s) of the main document.
This behaviour can be altered by the alternative form
|\childdocby[*]{|\textit{main}|}| (with a non-empty optional argument)
which uses the |.aux| file of the main document
by setting |\jobname| to \textit{main}.

%%%%%%%%%%%%%%%%%%%%%%%%%%%%%%%%%%%%%%%%%%%%%%%%%%%%%%%%%%%%%%%%%%%%%%%%%%%%%%%%
\subsection{Driver Development}
\label{sec:driver}

The \textsf{childdoc} mechanism can also be use for the development
of definition files such as \LaTeX{} styles or classes.
This case differs from the above setup with multiple parts
included by |\include| in that no |\includeonly| should be invoked.
This can be achieved by starting the include file
(before |\ProvidesPackage|) with:
%
\begin{center}
\begin{tabular}{l}
|\input{childdoc.def}|\\
|\childdocforward{|\textit{main}|}|\\
\end{tabular}
\end{center}
%
or alternatively with:
%
\begin{center}
\begin{tabular}{l}
|\input{childdoc.def}|\\
|\childdocby{|\textit{main}|}|\\
\end{tabular}
\end{center}
%
Both forms have slightly different effects as described above.
The main file is prepared as usual, see \secref{sec:include}.

%%%%%%%%%%%%%%%%%%%%%%%%%%%%%%%%%%%%%%%%%%%%%%%%%%%%%%%%%%%%%%%%%%%%%%%%%%%%%%%%
\subsection{Legacy Detection}
\label{sec:detection}

The directive |\childdocmain| in the main file can detect
whether the complete document or merely a child is to be compiled
even without using the directive |\childdocof|.
This method is deprecated because it is less robust
and there is no compelling reason to use it;
it is merely provided for backward compatibility
and it may be removed in future versions.

If the detection mechanism is to be used,
it is mandatory to correctly specify
the filename of the main file as the argument of |\childdocmain|:
%
\begin{center}
\begin{tabular}{l}
|\input{childdoc.def}|\\
|\childdocmain{|\textit{main}|}|\\
\end{tabular}
\end{center}
%
If |\jobname| does not match the argument \textit{main} of |\childdocmain|,
it is assumed that |\jobname| points to the child file to be compiled.
When using |\childdocmain| with the main file specified as argument,
it suffices to start a child file
with just |\input{|\textit{main}|}|
without loading of the package and using |\childdocof|.
If instead all processing is done
with the appropriate \textsf{childdoc} directives,
the argument of \textit{main} of |\childdocmain| can be empty.

An alternative version of the command line processing described
in \secref{sec:commandline} using the detection mechanism reads:
%
\begin{center}
|... -jobname "|\textit{target}|" "|[\textit{flags}]%
[|\def\jobname{|\textit{dest}|}|]|\input{|\textit{main}|}"|
\end{center}

%%%%%%%%%%%%%%%%%%%%%%%%%%%%%%%%%%%%%%%%%%%%%%%%%%%%%%%%%%%%%%%%%%%%%%%%%%%%%%%%
\subsection{Manual Code}
\label{sec:manual}

In case one cannot be certain whether the definitions file |childdoc.def|
is installed on the target \TeX{} distribution
and one prefers not to ship it,
it is conceivable to paste a few relevant commands into the sources.

To that end, drop all statements |\input{childdoc.def}|
and perform the replacements as outlined below.
Instead of |\childdocmain{|\textit{main}|}| add the following code
to the top of the main file:
%
\begin{center}
\begin{tabular}{l}
|\||ifdefined\childdocname\endinput\||fi\newif\ifchilddoc|\\
|\edef\childdocname{\scantokens\expandafter{\jobname\noexpand}}|\\
|\def\childdocmain{|\textit{main}|}\||ifx\childdocmain\childdocname\||else|\\
|\childdoctrue\includeonly{\childdocname}\let\jobname\childdocmain\||fi|\\
\end{tabular}
\end{center}
%
Instead of |\childdocof{|\textit{main}|}| just include the main file
at the top of each child file:
%
\begin{center}
|\input{|\textit{main}|}|
\end{center}
%
A simple redirection |\childdocforward{|\textit{dest}|}| is achieved by:
%
\begin{center}
|\def\jobname{|\textit{dest}|}\input{\jobname}|
\end{center}
%
The redirection with prefix
|\childdocforwardprefix[|\textit{prefix}|]{|\textit{dest}|}|
is accomplished by:
%
\begin{center}
\begin{tabular}{l}
|{\edef\jobname{\scantokens\expandafter{\jobname\noexpand}}|\\
|\def\redirectjob |\textit{prefix}|#1~~~{\gdef\jobname{|\textit{dest}|#1}}|\\
|\expandafter\redirectjob\jobname~~~}\input{\jobname}|
\end{tabular}
\end{center}

In an alternative approach,
child documents can be compiled by a specific command line
without additional code or specific definitions:
%
\begin{center}
|... -jobname "|\textit{target}|" "|[\textit{flags}]%
|\includeonly{|\textit{dest}|}\input{|\textit{main}|}"|
\end{center}
%

%%%%%%%%%%%%%%%%%%%%%%%%%%%%%%%%%%%%%%%%%%%%%%%%%%%%%%%%%%%%%%%%%%%%%%%%%%%%%%%%
%%%%%%%%%%%%%%%%%%%%%%%%%%%%%%%%%%%%%%%%%%%%%%%%%%%%%%%%%%%%%%%%%%%%%%%%%%%%%%%%
\section{Information}

%%%%%%%%%%%%%%%%%%%%%%%%%%%%%%%%%%%%%%%%%%%%%%%%%%%%%%%%%%%%%%%%%%%%%%%%%%%%%%%%
\subsection{Copyright}

Copyright \copyright{} 2017--2018 Niklas Beisert

This work may be distributed and/or modified under the
conditions of the \LaTeX{} Project Public License, either version 1.3
of this license or (at your option) any later version.
The latest version of this license is in
  \url{http://www.latex-project.org/lppl.txt}
and version 1.3 or later is part of all distributions of \LaTeX{}
version 2005/12/01 or later.

This work has the LPPL maintenance status `maintained'.

The Current Maintainer of this work is Niklas Beisert.

This work consists of the files |README.txt|, |childdoc.ins| and |childdoc.dtx|
as well as the derived files |childdoc.def|, |cdocsamp.tex|
with |cdocsch1.tex|, |cdocsch2.tex|, |cdocspt3.tex|, |cdocspt4.tex|,
|cdocsdrf.tex|, |cdocsfn1.tex|, |cdocsfn2.tex|
as well as |childdoc.pdf|.

%%%%%%%%%%%%%%%%%%%%%%%%%%%%%%%%%%%%%%%%%%%%%%%%%%%%%%%%%%%%%%%%%%%%%%%%%%%%%%%%
\subsection{Files and Installation}

The package consists of the files:
%
\begin{center}
\begin{tabular}{ll}
    |README.txt|   & readme file \\
    |childdoc.ins| & installation file \\
    |childdoc.dtx| & source file \\
    |childdoc.def| & definition file \\
    |cdocsamp.tex| & sample main file \\
    |cdocsch1.tex| & sample include file \\
    |cdocsch2.tex| & sample include file \\
    |cdocspt3.tex| & sample part file \\
    |cdocspt4.tex| & sample part file \\
    |cdocsdrf.tex| & sample redirection file \\
    |cdocsfn1.tex| & sample redirection file \\
    |cdocsfn2.tex| & sample redirection file \\
    |childdoc.pdf| & manual
\end{tabular}
\end{center}
%
The distribution consists of the files
|README.txt|, |childdoc.ins| and |childdoc.dtx|.
%
\begin{itemize}
\item
Run (pdf)\LaTeX{} on |childdoc.dtx|
to compile the manual |childdoc.pdf| (this file).
\item
Run \LaTeX{} on |childdoc.ins| to create the definitions file |childdoc.def|
and the sample |cdocsamp.tex| with include files
|cdocsch1.tex|, |cdocsch2.tex|, |cdocspt3.tex|, |cdocspt4.tex|,
|cdocsdrf.tex|, |cdocsfn1.tex|, |cdocsfn2.tex|.
Then copy the file |childdoc.def| to an appropriate directory of your \LaTeX{}
distribution, e.g.\ \textit{texmf-root}|/tex/latex/childdoc|.
\end{itemize}

%%%%%%%%%%%%%%%%%%%%%%%%%%%%%%%%%%%%%%%%%%%%%%%%%%%%%%%%%%%%%%%%%%%%%%%%%%%%%%%%
\subsection{Related CTAN Packages}

There are several other packages which offer a similar functionality:
%
\begin{itemize}
\item
The packages
\href{http://ctan.org/pkg/docmute}{\textsf{docmute}},
\href{http://ctan.org/pkg/includex}{\textsf{includex}} and
\href{http://ctan.org/pkg/standalone}{\textsf{standalone}}
provide commands to include only the document body of
a child file thus allowing both files to be compiled individually.
\item
The packages \href{http://ctan.org/pkg/subdocs}{\textsf{subdocs}}
and \href{http://ctan.org/pkg/subfiles}{\textsf{subfiles}}
provide structures in which the main and child documents can be
encapsulated and allowing them to be compiled individually.
The inclusion mechanism is different from the conventional |\include|.
\item
The package \href{http://ctan.org/pkg/combine}{\textsf{combine}}
is an elaborate solution to combine several documents into one.
\end{itemize}
%
See also the CTAN topic \href{http://ctan.org/topic/subdocs}{\textsf{subdocs}}
for further related packages.
The present package differs from the above solutions in that
a document structure constructed with the conventional |\include| mechanism
just needs two extra commands at the top of every file
such that all constituent files can be compiled individually.

%%%%%%%%%%%%%%%%%%%%%%%%%%%%%%%%%%%%%%%%%%%%%%%%%%%%%%%%%%%%%%%%%%%%%%%%%%%%%%%%
%\subsection{Feature Suggestions}
%
%The following is a list of features which may be useful for future
%versions of this package:
%%
%\begin{itemize}
%\item
%\ldots
%\end{itemize}

%%%%%%%%%%%%%%%%%%%%%%%%%%%%%%%%%%%%%%%%%%%%%%%%%%%%%%%%%%%%%%%%%%%%%%%%%%%%%%%%
\subsection{Revision History}

%%%%%%%%%%%%%%%%%%%%%%%%%%%%%%%%%%%%%%%%
\paragraph{v2.0:} 2018/12/30

\begin{itemize}
\item
immediate forward processing
\item
added |\childdocby| mechanism
\item
manual restructured
\end{itemize}

%%%%%%%%%%%%%%%%%%%%%%%%%%%%%%%%%%%%%%%%
\paragraph{v1.6:} 2018/01/17

\begin{itemize}
\item
application for development of include files
\item
corrections to manual
\end{itemize}

%%%%%%%%%%%%%%%%%%%%%%%%%%%%%%%%%%%%%%%%
\paragraph{v1.5:} 2017/05/21

\begin{itemize}
\item
more complete structuring introduced
\item
|\childdocof| introduced
\item
|\childdoc| renamed to |\childdocmain|
\item
|\childredirect| renamed to |\childdocforward| and |\childdocforwardprefix|
and functionality expanded
\end{itemize}

%%%%%%%%%%%%%%%%%%%%%%%%%%%%%%%%%%%%%%%%
\paragraph{v1.0:} 2017/04/27

\begin{itemize}
\item
manual and install package
\item
first version published on CTAN
\end{itemize}

%%%%%%%%%%%%%%%%%%%%%%%%%%%%%%%%%%%%%%%%
\paragraph{v0.6:} 2017/04/26

\begin{itemize}
\item
redirection mechanism added
\end{itemize}

%%%%%%%%%%%%%%%%%%%%%%%%%%%%%%%%%%%%%%%%
\paragraph{v0.5:} 2017/04/26

\begin{itemize}
\item
functionality in definition file
\end{itemize}


%%%%%%%%%%%%%%%%%%%%%%%%%%%%%%%%%%%%%%%%%%%%%%%%%%%%%%%%%%%%%%%%%%%%%%%%%%%%%%%%
%%%%%%%%%%%%%%%%%%%%%%%%%%%%%%%%%%%%%%%%%%%%%%%%%%%%%%%%%%%%%%%%%%%%%%%%%%%%%%%%
%%%%%%%%%%%%%%%%%%%%%%%%%%%%%%%%%%%%%%%%%%%%%%%%%%%%%%%%%%%%%%%%%%%%%%%%%%%%%%%%
\appendix

\settowidth\MacroIndent{\rmfamily\scriptsize 000\ }

 \DocInput{childdoc.dtx}

\end{document}
%</driver>
% \fi
%
% %%%%%%%%%%%%%%%%%%%%%%%%%%%%%%%%%%%%%%%%%%%%%%%%%%%%%%%%%%%%%%%%%%%%%%%%%%%%%%
% %%%%%%%%%%%%%%%%%%%%%%%%%%%%%%%%%%%%%%%%%%%%%%%%%%%%%%%%%%%%%%%%%%%%%%%%%%%%%%
% \section{Sample}
%\iffalse
%<*samplemain>
%\fi
%
% The following presents a sample document
% with two chapters, two parts, a title page,
% a compile flag as well as three forwarding files to set the flag.
% It consists of eight |.tex| files:
% \begin{center}
% \begin{tabular}{ll}
% |cdocsamp.tex|&main file\\
% |cdocsch1.tex|&include file for chapter 1\\
% |cdocsch2.tex|&include file for chapter 2\\
% |cdocspt3.tex|&include file for part 3\\
% |cdocspt4.tex|&include file for part 4\\
% |cdocsdrf.tex|&forwarding file for main file in draft mode\\
% |cdocsfi1.tex|&forwarding file for final version of chapter 1\\
% |cdocsfi2.tex|&forwarding file for final version of chapter 2\\
% \end{tabular}
% \end{center}
% Each of the eight files can be compiled directly by the \LaTeX{} compiler.
%
% %%%%%%%%%%%%%%%%%%%%%%%%%%%%%%%%%%%%%%
% \paragraph{Main File.}
%
% The main file is called |cdocsamp.tex|.
%
% Load the \textsf{childdoc} definitions and
% declare the filename for the main document:
%    \begin{macrocode}
\input{childdoc.def}
\childdocmain{}
%    \end{macrocode}

% Optional override for |\version| flag:
%    \begin{macrocode}
%%\ifchilddoc\else\providecommand{\version}{draft}\fi
%    \end{macrocode}

% Define the default values for the |\version| flag
% (|final| for the main file and |draft| for childs):
%    \begin{macrocode}
\ifchilddoc
\providecommand{\version}{draft}
\else
\providecommand{\version}{final}
\fi
%    \end{macrocode}

% Load the standard document class:
%    \begin{macrocode}
\documentclass[12pt]{article}
%    \end{macrocode}

% Start the document body:
%    \begin{macrocode}
\begin{document}
%    \end{macrocode}

% Declare a title page.
% Print title, part of document being processed and version flag:
%    \begin{macrocode}
\addtocounter{page}{-1}
\begin{center}
{\LARGE\bfseries{}childdoc example\par}
\vspace{1cm}
\ifchilddoc
\ifchilddocmanual part\else chapter\fi:
`\childdocname' of `\childdocjob'\par
\else
main document: `\childdocjob'\par
\fi
version: \version\par
\end{center}
\newpage
%    \end{macrocode}

% Manually include selected file,
% otherwise process as usual:
%    \begin{macrocode}
\ifchilddocmanual
\section*{part `\childdocname'}
\input{\childdocname}
\else
%    \end{macrocode}

% Include the two chapters:
%    \begin{macrocode}
\include{cdocsch1}
\include{cdocsch2}
%    \end{macrocode}

% Include the two parts unless only chapters should be displayed:
%    \begin{macrocode}
\ifchilddoc\else
\section{part three}
\input{cdocspt3}
\section{part four}
\input{cdocspt4}
\fi
%    \end{macrocode}

% Process as usual until here:
%    \begin{macrocode}
\fi
%    \end{macrocode}

% End of document body:
%    \begin{macrocode}
\end{document}
%    \end{macrocode}
%\iffalse
%</samplemain>
%\fi
%
% %%%%%%%%%%%%%%%%%%%%%%%%%%%%%%%%%%%%%%
% \paragraph{Chapter Include Files.}
%
% The include files are called |cdocsch1.tex| and |cdocsch2.tex|.
%
%\iffalse
%<*samplechap1|samplechap2>
%\fi

% Optional override for |\version| flag:
%    \begin{macrocode}
%%\providecommand{\version}{final}
%    \end{macrocode}

% Include the main document:
%    \begin{macrocode}
\input{childdoc.def}
\childdocof{cdocsamp}
%    \end{macrocode}

%\iffalse
%</samplechap1|samplechap2>
%\fi
%
%\iffalse
%<*samplechap1>
%\fi
% Some text for chapter 1:
%    \begin{macrocode}
\section{one}
some text in chapter one
%    \end{macrocode}

%\iffalse
%</samplechap1>
%\fi
% Some text for chapter 2:
%\iffalse
%<*samplechap2>
%\fi
%    \begin{macrocode}
\section{two}
more text in chapter two
%    \end{macrocode}

%\iffalse
%</samplechap2>
%\fi
%
% %%%%%%%%%%%%%%%%%%%%%%%%%%%%%%%%%%%%%%
% \paragraph{Part Include Files.}
%
% The include files are called |cdocspt3.tex| and |cdocspt4.tex|.
%
%\iffalse
%<*samplepart3|samplepart4>
%\fi

% Optional override for |\version| flag:
%    \begin{macrocode}
%%\providecommand{\version}{final}
%    \end{macrocode}

% Include the main document:
%    \begin{macrocode}
\input{childdoc.def}
\childdocby{cdocsamp}
%    \end{macrocode}

%\iffalse
%</samplepart3|samplepart4>
%\fi
%
%\iffalse
%<*samplepart3>
%\fi
% Some text for part 3:
%    \begin{macrocode}
some text in part three
%    \end{macrocode}

%\iffalse
%</samplepart3>
%\fi
% Some text for part 4:
%\iffalse
%<*samplepart4>
%\fi
%    \begin{macrocode}
more text in part four
%    \end{macrocode}

%\iffalse
%</samplepart4>
%\fi
%
% %%%%%%%%%%%%%%%%%%%%%%%%%%%%%%%%%%%%%%
% \paragraph{Forwarding for a Complete Draft.}
%
% The following forwarding file |cdocsdrf.tex|
% compiles the main document in draft mode:
%\iffalse
%<*sampledraft>
%\fi
%    \begin{macrocode}
\def\version{draft}
\input{childdoc.def}
\childdocforward{cdocsamp}
%    \end{macrocode}

%\iffalse
%</sampledraft>
%\fi
%
% %%%%%%%%%%%%%%%%%%%%%%%%%%%%%%%%%%%%%%
% \paragraph{Forwarding for Final Version of the Chapters.}
%
% The following forwarding files |cdocsfn1.tex| and |cdocsfn2.tex|
% (with identical content)
% compile the final versions of the child documents
% |cdocsch1.tex| and |cdocsch2.tex|, respectively:
%\iffalse
%<*samplefinal>
%\fi
%    \begin{macrocode}
\def\version{final}
\input{childdoc.def}
\childdocforwardprefix[cdocsamp]{cdocsfn}{cdocsch}
%    \end{macrocode}

%\iffalse
%</samplefinal>
%\fi
%
% %%%%%%%%%%%%%%%%%%%%%%%%%%%%%%%%%%%%%%
% \paragraph{Command Line Processing.}
%
% The following three command lines generate the output files
% |cdocscld|, |cdocscl1| and |cdocscl2|
% which should be identical to
% |cdocsdrf|, |cdocsch1| and |cdocsfn2|, respectively:
% \begin{center}
% \begin{tabular}{l}
% |latex -jobname cdocscld \|\\
% |  "\def\version{draft}\input{childdoc.def}\childdocforward{cdocsamp}"|\\
% |latex -jobname cdocscl1 \|\\
% |  "\input{childdoc.def}\childdocforward[cdocsamp]{cdocsch1}"|\\
% |latex -jobname cdocscl2 \|\\
% |  "\def\version{final}\input{childdoc.def}\childdocforward{cdocsch2}"|
% \end{tabular}
% \end{center}
% Note that the trailing backslash on each first line
% merely continues the input to the second line
% (for convenient cut ant paste).
% Furthermore, the command |latex| can be replaced by any
% of its alternative versions such as |pdflatex|.
%
% %%%%%%%%%%%%%%%%%%%%%%%%%%%%%%%%%%%%%%%%%%%%%%%%%%%%%%%%%%%%%%%%%%%%%%%%%%%%%%
% %%%%%%%%%%%%%%%%%%%%%%%%%%%%%%%%%%%%%%%%%%%%%%%%%%%%%%%%%%%%%%%%%%%%%%%%%%%%%%
% \section{Implementation}
%\iffalse
%<*package>
%\fi
%
% This section describes the definitions file |childdoc.def|.

% The definitions cannot be loaded using |\usepackage| or |\RequirePackage|
% which has a mechanism to prevent loading a style file more than once.
% When loading the definitions by means of |\input|
% multiple instances have to be prevented manually:
%\iffalse
%This code needs to be before the `\ProvidesFile' directive
%which is defined at the beginning of this file.
%Therefore it is also placed there and commented out here.
%</package>
%<*discard>
%\fi
%    \begin{macrocode}
\ifdefined\childdocmain\endinput\fi
%    \end{macrocode}
%\iffalse
%</discard>
%<*package>
%\fi
%
% \macro{\ifchilddoc}
% \macro{\ifchilddocmanual}
% The conditional |\ifchilddoc| tells whether a
% child (true) or main (false) document is being compiled.
% The conditional |\ifchilddocmanual| tells whether
% the |\includeonly| mechanism is used (false) or
% the selection of child files must be performed manually (true).
% The definitions initialise to false:
%    \begin{macrocode}
\newif\ifchilddoc
\newif\ifchilddocmanual
%    \end{macrocode}

% \macro{\childdocname}
% \macro{\childdocjob}
% The macro |\childdocname| stores the name of the main document
% to be compiled. The macro |\childdocjob| stores the name of
% the document on which the \LaTeX{} compiler was originally invoked.
% The content of |\jobname| cannot be compared
% to filenames specified in the source due to different catcodes.
% The following code rescans |\jobname|, stores the result
% in |\childdocname| and saves a copy in |\childdocjob|:
%    \begin{macrocode}
\edef\childdocname{\scantokens\expandafter{\jobname\noexpand}}
\let\childdocjob\childdocname
%    \end{macrocode}

% \macro{\childdocdisable}
% The macro |\childdocdisable| prevents the main file
% from being processed more than once.
% At this stage, the main document command |\childdocmain|
% is assumed to be called once again where it should do nothing.
% Any subsequent call to it should prevent
% a secondary processing of the main document
% It overwrites the forwarding commands
% |\childdocof| and |\childdocforward|
% with empty macros to prevent further inclusions of the main document:
%    \begin{macrocode}
\newcommand{\childdocdisable}
{
  \renewcommand{\childdocmain}[1]{\renewcommand{\childdocmain}[1]{\endinput}}
  \renewcommand{\childdocof}[1]{}
  \renewcommand{\childdocby}[2][]{}
  \renewcommand{\childdocforward}[2][]{}
  \renewcommand{\childdocdisable}{}
}
%    \end{macrocode}

% \macro{\childdocmain}
% The macro |\childdocmain| is to be called at the top of the main file
% with nothing or the main filename (without extension) as argument.
% First, it breaks loops.
% If the argument is not empty and does not match |\childdocname|
% (which is set by the first inclusion of |childdoc.def|),
% |\ifchilddoc| is set to true, |\includeonly| is applied to the child file
% and |\jobname| is set to the main file
% (for proper handling of |.aux| files):
%    \begin{macrocode}
\newcommand{\childdocmain}[1]
{
  \childdocdisable\childdocmain{}
  \if?#1?\else
    \begingroup
      \def\childdoctmp{#1}
      \ifx\childdoctmp\childdocname
        \def\childdoctmp{}
      \else
        \def\childdoctmp
        {
          \childdoctrue
          \includeonly{\childdocname}
          \def\childdocjob{#1}
          \def\jobname{#1}
        }
      \fi
      \expandafter
    \endgroup
    \childdoctmp
  \fi
}
%    \end{macrocode}

% \macro{\childdocof}
% The command |\childdocof| redirects
% compilation to the main file |#1|.
%    \begin{macrocode}
\newcommand{\childdocof}[1]
{
  \childdocdisable
  \childdoctrue
  \includeonly{\childdocname}
  \def\jobname{#1}
  \def\childdocjob{#1}
  \input{#1}
}
%    \end{macrocode}

% \macro{\childdocby}
% The command |\childdocby| ....
%    \begin{macrocode}
\newcommand{\childdocby}[2][]
{
  \childdocdisable
  \childdoctrue
  \childdocmanualtrue
  \if?#1?\else
    \def\jobname{#2}
  \fi
  \def\childdocjob{#2}
  \input{#2}
  \endinput
}
%    \end{macrocode}

% \macro{\childdocforward}
% The command |\childdocforward| redirects
% compilation to the main file or
% (if the optional argument is given) a child file.
% Parameters are set as if the main file
% or a child file starting with |\childdocof| was compiled.
% Then compilation is handed over to the main file:
%    \begin{macrocode}
\newcommand{\childdocforward}[2][]
{
  \begingroup
    \if?#1?
      \def\childdoctmp
      {
        \def\childdocname{#2}
        \def\childdocjob{#2}
        \def\jobname{#2}
        \input{#2}
        \endinput
      }
    \else
      \def\childdoctmp
      {
        \childdocdisable
        \def\childdocname{#2}
        \childdoctrue
        \includeonly{#2}
        \def\childdocjob{#1}
        \def\jobname{#1}
        \input{#1}
        \endinput
      }
    \fi
    \expandafter
  \endgroup
  \childdoctmp
}
%    \end{macrocode}

% \macro{\childdocforwardprefix}
% The command |\childdocforwardprefix| redirects
% compilation to the main or a child file by means of a pattern.
% The prefix |#1| in the current filename is replaced by |#2|
% and the suffix of the current filename is kept
% (it is assumed that the filename does not contain the substring `|~~~|'
% which is used as a delimiter).
% Compilation is handed over to the new file by |\childdocforward|:
%    \begin{macrocode}
\newcommand{\childdocforwardprefix}[3][]
{
  \begingroup
    \def\childdocextract #2##1~~~{\def\childdoctmp{\childdocforward[#1]{#3##1}}}
    \expandafter\childdocextract\childdocname~~~
    \expandafter
  \endgroup
  \childdoctmp
}
%    \end{macrocode}

% \macro{\childdoc}
% The deprecated macro |\childdoc| is a legacy version of |\childdocmain|:
%    \begin{macrocode}
\newcommand{\childdoc}{\childdocmain}
%    \end{macrocode}

% \macro{\childdocredirect}
% The deprecated macro |\childdocredirect| is a legacy version
% of |\childdocforward| and |\childdocforwardprefix|:
%    \begin{macrocode}
\newcommand{\childdocredirect}[2][]
{
  \begingroup
    \if?#1?
      \def\childdoctmp{\childdocforward{#2}}
    \else
      \def\childdoctmp{\childdocforwardprefix{#1}{#2}}
    \fi
    \expandafter
  \endgroup
  \childdoctmp
}
%    \end{macrocode}

%\iffalse
%</package>
%\fi
%
\endinput
|\\
|\childdocforward{|\textit{main}|}|\\
\end{tabular}
\end{center}
%
or alternatively with:
%
\begin{center}
\begin{tabular}{l}
|% \iffalse
%
% childdoc.dtx Copyright (C) 2017-2018 Niklas Beisert
%
% This work may be distributed and/or modified under the
% conditions of the LaTeX Project Public License, either version 1.3
% of this license or (at your option) any later version.
% The latest version of this license is in
%   http://www.latex-project.org/lppl.txt
% and version 1.3 or later is part of all distributions of LaTeX
% version 2005/12/01 or later.
%
% This work has the LPPL maintenance status `maintained'.
%
% The Current Maintainer of this work is Niklas Beisert.
%
% This work consists of the files childdoc.dtx and childdoc.ins
% and the derived files childdoc.def and cdocsamp.tex with
% cdocsch1.tex, cdocsch2.tex, cdocsdrf.tex, cdocsfn1.tex, cdocsfn2.tex.
%
%<package>\ifdefined\childdocmain\endinput\fi
%<package>\ProvidesFile{childdoc.def}[2018/12/30 v2.0 child document driver]
%<samplemain>\ProvidesFile{cdocsamp.tex}[2018/12/30 v2.0 sample for childdoc]
%<*driver>
%\ProvidesFile{childdoc.drv}[2018/12/30 v2.0 childdoc reference manual file]
\PassOptionsToClass{10pt,a4paper}{article}
\documentclass{ltxdoc}

\usepackage[margin=35mm]{geometry}
\usepackage{hyperref}
\usepackage{hyperxmp}
\usepackage[usenames]{color}

\hypersetup{colorlinks=true}
\hypersetup{pdfstartview=FitH}
\hypersetup{pdfpagemode=UseNone}
\hypersetup{pdfsource={}}
\hypersetup{pdflang={en-UK}}
\hypersetup{pdfcopyright={Copyright 2017-2018 Niklas Beisert.
  This work may be distributed and/or modified under the
  conditions of the LaTeX Project Public License, either version 1.3
  of this license or (at your option) any later version.}}
\hypersetup{pdflicenseurl={http://www.latex-project.org/lppl.txt}}
\hypersetup{pdfcontactaddress={ETH Zurich, ITP, HIT K,
  Wolfgang-Pauli-Strasse 27}}
\hypersetup{pdfcontactpostcode={8093}}
\hypersetup{pdfcontactcity={Zurich}}
\hypersetup{pdfcontactcountry={Switzerland}}
\hypersetup{pdfcontactemail={nbeisert@itp.phys.ethz.ch}}
\hypersetup{pdfcontacturl={http://people.phys.ethz.ch/\xmptilde nbeisert/}}

\newcommand{\secref}[1]{\hyperref[#1]{section \ref*{#1}}}

\parskip1ex
\parindent0pt
\let\olditemize\itemize
\def\itemize{\olditemize\parskip0pt}

\begin{document}

\title{The \textsf{childdoc} Package}
\hypersetup{pdftitle={The childdoc Package}}
\author{Niklas Beisert\\[2ex]
  Institut f\"ur Theoretische Physik\\
  Eidgen\"ossische Technische Hochschule Z\"urich\\
  Wolfgang-Pauli-Strasse 27, 8093 Z\"urich, Switzerland\\[1ex]
  \href{mailto:nbeisert@itp.phys.ethz.ch}
  {\texttt{nbeisert@itp.phys.ethz.ch}}}
\hypersetup{pdfauthor={Niklas Beisert}}
\hypersetup{pdfsubject={Manual for the LaTeX2e Package childdoc}}
\date{30 December 2018, \textsf{v2.0}}
\maketitle

\begin{abstract}\noindent
\textsf{childdoc} is a \LaTeXe{} package
that enables the direct compilation
of document sections included by |\include|
to individual files.
\end{abstract}

\begingroup
\parskip0ex
\tableofcontents
\endgroup

%%%%%%%%%%%%%%%%%%%%%%%%%%%%%%%%%%%%%%%%%%%%%%%%%%%%%%%%%%%%%%%%%%%%%%%%%%%%%%%%
%%%%%%%%%%%%%%%%%%%%%%%%%%%%%%%%%%%%%%%%%%%%%%%%%%%%%%%%%%%%%%%%%%%%%%%%%%%%%%%%
\section{Introduction}

\LaTeX{} provides a mechanism to structure a large document (such as a book)
into a main file and several child files (containing the chapters)
using the |\include| command.
This mechanism is beneficial for documents
which span hundreds of pages in order to
make the source file(s) more manageable.
Moreover, compilation can be restricted to
selected child files by means of the |\includeonly| command.
The latter feature can be used to reduce the compilation time while editing
(this was significantly more useful in the earlier days of \LaTeX{})
or to generate a smaller document which is easier to navigate.
Another application of |\includeonly| is to generate
documents consisting of selected parts of the complete document.

However, there are a few drawbacks of the plain |\include| mechanism:
\begin{itemize}
\item
The child files cannot be compiled on their own,
they can only be compiled via the main file.
A naive editing environment
(such as a text editor with an option
to have the current file processed by \LaTeX)
may require one to switch to the main file before compiling;
attempting to compile the child file produces errors.
\item
The main file must be modified (each time)
to adjust the |\includeonly| command
to the present needs. This easily leaves the main file in a messy state.
\item
The generated document will always carry the filename
of the main document. This is inconvenient if
several child files are to be compiled and
to be kept for distribution.
\end{itemize}

The present package provides a simple interface
to make child files individually compilable by \LaTeX{}.
Compiling a child file then has the same effect as compiling
the main file with an |\includeonly| command
to select the appropriate child.
Moreover the generated document will carry the name of the child
rather than the main file.
This resolves all three above issues.

This feature is meant to make the editing of books,
thesis documents and lecture notes somewhat more convenient.
However, the package can also be used efficiently for
composing a series of documents (such as exercise sheets)
which are typically distributed individually.
It then assists the author in generating the individual documents
(potentially in different versions)
as well as a document containing the collected series.
Another application is in developing style files
or other kinds of included material
where compilation of the style file could redirect
to a sample or test file.

%%%%%%%%%%%%%%%%%%%%%%%%%%%%%%%%%%%%%%%%%%%%%%%%%%%%%%%%%%%%%%%%%%%%%%%%%%%%%%%%
%%%%%%%%%%%%%%%%%%%%%%%%%%%%%%%%%%%%%%%%%%%%%%%%%%%%%%%%%%%%%%%%%%%%%%%%%%%%%%%%
\section{Usage}

First of all, the package \textsf{childdoc} is \emph{not} a standard
\LaTeXe{} |.sty| style file! Therefore it needs to be invoked in
a non-standard way.

%%%%%%%%%%%%%%%%%%%%%%%%%%%%%%%%%%%%%%%%%%%%%%%%%%%%%%%%%%%%%%%%%%%%%%%%%%%%%%%%
\subsection{Included Files}
\label{sec:include}

%%%%%%%%%%%%%%%%%%%%%%%%%%%%%%%%%%%%%%%%
\DescribeMacro{\childdocmain}
To use the package, add the commands
\begin{center}
\begin{tabular}{l}
|\input{childdoc.def}|\\
|\childdocmain{}|\\
\end{tabular}
\end{center}
at the very top of the main \LaTeX{} file,
in particular \emph{before} the |\documentclass| statement!
The argument of |\childdocmain| should be left empty
(but it must be present).

%%%%%%%%%%%%%%%%%%%%%%%%%%%%%%%%%%%%%%%%
\DescribeMacro{\childdocof}
Furthermore, add the commands
\begin{center}
\begin{tabular}{l}
|\input{childdoc.def}|\\
|\childdocof{|\textit{main}|}|\\
\end{tabular}
\end{center}
at the top of every child file \textit{child}
which is included by |\include{|\textit{child}|}|
from within the main file
(or at least for those files to be compiled individually).
The argument \textit{main} must be the filename of the main file.

There are a couple of
considerations in setting up the main and child documents:

%%%%%%%%%%%%%%%%%%%%%%%%%%%%%%%%%%%%%%%%
\paragraph{Restrictions.}

Please note the following restrictions:
\begin{itemize}
\item
|\childdocmain| must be called with one argument \textit{main}
to ensure compatibility with earlier version of the package.
It must either be empty (|\childdocmain{}|)
or precisely match the filename of the main file in which it is specified.
See \secref{sec:detection} for further information.
\item
The filename \textit{main} must be specified without the |.tex| extension.
\item
The filename \textit{main} is case sensitive
(even in case-insensitive file systems)
due to internal string comparison.
\item
The argument \textit{main} should be fully expanded, it cannot be a macro.
\item
Subdirectories and special characters should be avoided in filenames.
\item
The command |\childdocmain{|\textit{main}|}| must be followed by a whitespace.
It should not be followed immediately by another command
or by a comment mark `|%|'.
This is because the \TeX{} parser reads the token immediately following
the argument of |\childdocmain| and puts it
at the beginning of every child section;
however, a white\-space is ignored.
\end{itemize}

%%%%%%%%%%%%%%%%%%%%%%%%%%%%%%%%%%%%%%%%
\paragraph{Content of Main File.}

It is advisable to place all content in the child files included by |\include|.
Any output contained in the main file will appear in all child documents
unless suppressed manually;
it cannot be suppressed automatically by the |\includeonly| directive
and thus should normally be avoided.
A method to include some content in the main file
by means of conditional processing is described in \secref{sec:conditional}.

%%%%%%%%%%%%%%%%%%%%%%%%%%%%%%%%%%%%%%%%
\paragraph{Page Numbering.}

When only a part of the document is compiled,
the appropriate numbering of pages
(as well as other status parameters)
is determined from the |.aux| files.
The latter contain information from previous passes.
However this information needs to propagate through
all intermediate child documents.
Therefore the page numbering in child documents may well
be inconsistent until the complete document is compiled at least once.

A useful (if unconventional) way to always ensure a consistent
page numbering is to restart the numbering in each child document
and denote the pages by `\textit{child}|.|\textit{page}'
where \textit{child} represents the chapter/section number of the child file.
This can be achieved by the command
|\numberwithin{page}{|\textit{child}|}|
of the \textsf{amsmath} package
where \textit{child} can be |chapter| or |section|
depending on the chosen structuring.
Alternatively, one can modify the macro |\thepage| appropriately
and reset the counter |page| at the start of each child file.

%%%%%%%%%%%%%%%%%%%%%%%%%%%%%%%%%%%%%%%%%%%%%%%%%%%%%%%%%%%%%%%%%%%%%%%%%%%%%%%%
\subsection{Conditional Processing}
\label{sec:conditional}

The package provides a mechanism to compile different versions
of a document. To customise the versions further some conditional processing
can come in handy to distinguish which version is being compiled.
The package provides two macros to describe the compilation context:

%%%%%%%%%%%%%%%%%%%%%%%%%%%%%%%%%%%%%%%%
\DescribeMacro{\ifchilddoc}
The conditional |\ifchilddoc| distinguishes between the compilation of
child documents and the main document:
%
\begin{center}
|\ifchilddoc |\textit{child-code}| |[|\||else |\textit{main-code}]| \||fi|
\end{center}

%%%%%%%%%%%%%%%%%%%%%%%%%%%%%%%%%%%%%%%%
\DescribeMacro{\childdocname}
\DescribeMacro{\childdocjob}
The macro |\childdocname| contains the filename (without extension)
of the main or child file being processed.
Note that |\childdocjob| will always contain the name of the main file.

%%%%%%%%%%%%%%%%%%%%%%%%%%%%%%%%%%%%%%%%
\paragraph{Title Page.}

Conditional processing can be used to include a title or banner page
in the main document when proper precautions are taken.
Importantly, the code in the main file should ensure that the page counter
(as well as other status parameters which are stored in the |.aux| files)
takes the same value after the conditional processing.
Otherwise the page numbers may take divergent values
depending on which part is compiled.

For example, a title page could be declared by:
%
\begin{center}
\begin{tabular}{l}
|\ifchilddoc\||else|\\
|\addtocounter{page}{-1}|\\
\textit{code for title page}\\
|\newpage|\\
|\||fi|
\end{tabular}
\end{center}
%
A banner page for the child documents can be generated by:
%
\begin{center}
\begin{tabular}{l}
|\ifchilddoc|\\
|\addtocounter{page}{-1}|\\
\textit{code for banner page}\\
|\newpage|\\
|\||fi|
\end{tabular}
\end{center}
%
Here one could write a message such as:
\begin{center}
|This is the part \childdocname{} of \childdocjob{}.|
\end{center}

%%%%%%%%%%%%%%%%%%%%%%%%%%%%%%%%%%%%%%%%%%%%%%%%%%%%%%%%%%%%%%%%%%%%%%%%%%%%%%%%
\subsection{Flags}
\label{sec:flags}

The package makes it easy to generate different versions
of the main or child documents.
To this end compilation flags can be defined
and assigned different default values.
They will be particularly useful in conjunction
with the forwarding mechanism described in \secref{sec:forward}.

For example, it may be useful to have a flag |\version|
which can be set to |draft| or |final|.
The document source will contain some conditional code
depending on the value of |\version|.
Suppose further, the flag should default to |final| for the main file
and to |draft| for child files
which is a natural assignment for editing the document.
This is achieved by placing the following code
in the preamble of the main document
(below the |\childdocmain| directive):
%
\begin{center}
\begin{tabular}{l}
|\ifchilddoc|\\
|\providecommand{\version}{draft}|\\
|\||else|\\
|\providecommand{\version}{final}|\\
|\||fi|
\end{tabular}
\end{center}
%
The definition by |\providecommand| makes sure
that previous definitions are not overwritten.
Further statements |\providecommand{\version}{...}|
can thus be added before the above code to override it.

For the main file, one might add a line
(between |\childdocmain| and the above block)
%
\begin{center}
|%\ifchilddoc\||else\providecommand{\version}{draft}\||fi|
\end{center}
%
which can be uncommented to produce a draft version.
Likewise one can add a line to the very top of a child file
(above the |\childdocof{|\textit{main}|}| directive)
%
\begin{center}
|%\providecommand{\version}{final}|
\end{center}
%
which can be uncommented to produce the final version of this child document.

%%%%%%%%%%%%%%%%%%%%%%%%%%%%%%%%%%%%%%%%%%%%%%%%%%%%%%%%%%%%%%%%%%%%%%%%%%%%%%%%
\subsection{Forwarding}
\label{sec:forward}

Different versions of the main or child documents
using compilation flags as described in \secref{sec:flags}
can be (permanently) stored in different files
for convenient compilation, viewing and distribution.
To this end, the package defines a command
to pass on compilation to a different file:

%%%%%%%%%%%%%%%%%%%%%%%%%%%%%%%%%%%%%%%%
\DescribeMacro{\childdocforward}
The command |\childdocforward| redirects processing to
another source file:
%
\begin{center}
\begin{tabular}{l}
|\input{childdoc.def}|\\
|\childdocforward[|\textit{main}|]{|\textit{dest}|}|\\
\end{tabular}
\end{center}
%
The argument \textit{dest} is the destination file
(without extension).
It should be the main file or one of the child files.
Note that further \textsf{childdoc} directives
such as |\childdocof| and |\childdocforward|
in the indicated file will be processed in this form.
The optional argument \textit{main}
passes on directly to the main file \textit{main}
while pretending to compile the child \textit{dest}.
This form behaves as if \textit{dest}
issues |\childdocof{|\textit{main}|}| right away,
and no further \textsf{childdoc} directives will be processed.

%%%%%%%%%%%%%%%%%%%%%%%%%%%%%%%%%%%%%%%%
\DescribeMacro{\...prefix}
In the alternative form |\childdocforwardprefix|,
%
\begin{center}
\begin{tabular}{l}
|\input{childdoc.def}|\\
|\childdocforwardprefix[|\textit{main}|]{|\textit{prefix}|}{|\textit{dest}|}|
\end{tabular}
\end{center}
%
the destination file is determined by a pattern
depending on the current file:
To make this work, the current file must be called
`{\textit{prefix}\hspace{0.2em}\textit{suffix}}'
with \textit{prefix} matching precisely the argument.
Processing is then passed on to the file
`{\textit{dest}\hspace{0.2em}\textit{suffix}}'.
Surely, the same effect is achieved by
directly specifying the
argument `{\textit{dest}\hspace{0.2em}\textit{suffix}}'
in the first form.
However, that requires to set up a different file
for each child. With the alternative form of the command
all these files can have exactly the same content
which simplifies setting them up and maintaining them.

For example, the following file |draft.tex|
with a compilation flag |\version| as described in \secref{sec:flags}
compiles the main document as a draft:
%
\begin{center}
\begin{tabular}{l}
|\def\version{draft}|\\
|\input{childdoc.def}|\\
|\childdocforward{|\textit{main}|}|
\end{tabular}
\end{center}
%
Likewise, the following files |final|\textit{nn}|.tex|
compile the final version of the child document
|child|\textit{nn}|.tex|:
%
\begin{center}
\begin{tabular}{l}
|\def\version{final}|\\
|\input{childdoc.def}|\\
|\childdocforwardprefix{final}{child}|
\end{tabular}
\end{center}
%

Note that when several versions of a main file and/or of each child file
are to be generated, it may be convenient to set up a |Makefile| or
shell script to automatise the process.

%%%%%%%%%%%%%%%%%%%%%%%%%%%%%%%%%%%%%%%%%%%%%%%%%%%%%%%%%%%%%%%%%%%%%%%%%%%%%%%%
\subsection{Command Line Processing}
\label{sec:commandline}

The effect of redirection files can also be achieved by invoking
the \LaTeX{} compiler with a more elaborate command line.
Most conveniently this should be done as part
of a shell script or a |Makefile|.

When using \textsf{childdoc} in the main file, the following
command lines effectively perform a redirection
(note that depending on the shell being used,
backslashes may have to be doubled: `|\|' $\to$ `|\\|'):
%
\begin{center}
|... -jobname "|\textit{target}|" |\\|"|[\textit{flags}]%
|\input{childdoc.def}\childdocforward[|\textit{main}|]{|\textit{dest}|}"|
\end{center}
%
Here \textit{target} is the name of the output file,
\textit{main} is the name of the main file
and \textit{dest} is the name of the main or child file to be processed
(all filenames without extensions).
The optional argument \textit{main} can be omitted
if \textit{main} matches \textit{dest}.
Optionally, compilation \textit{flags} can be defined via |\def| commands.
This command line makes the \TeX{} engine believe
it is compiling the file \textit{target}
whose content is specified as the latter parameter.
The provided code then forwards the processing to
\textit{main} or \textit{dest} as described in \secref{sec:forward}.

%%%%%%%%%%%%%%%%%%%%%%%%%%%%%%%%%%%%%%%%%%%%%%%%%%%%%%%%%%%%%%%%%%%%%%%%%%%%%%%%
\subsection{Include by Input}
\label{sec:input}

Including child documents by |\include| has some restrictions by design.
Most notably, the content of a child document always occupies
its own set of pages; pages cannot be shared between child documents.
Usually, this behaviour makes perfect sense
because each child document contain an essential part of the document.
However, in some situations it may be desirable to compose
a document from a collection of parts
without having mandatory page breaks between then.
For this case, the package
provides a mechanism to include parts
by |\input| which can also be processed individually.
However, by construction this mechanism
requires manual handling of the content to be output.

%%%%%%%%%%%%%%%%%%%%%%%%%%%%%%%%%%%%%%%%
\DescribeMacro{\ifchilddocmanual}
The main file should be prepared as usual, see \secref{sec:include}.
However, the document body must make a distinction
between processing of an individual part and of the main document, e.g.:
%
\begin{center}
\begin{tabular}{l}
|\ifchilddocmanual|\\
|\input{\childdocname}|\\
|\||else|\\
\textit{document body with }|\input{|\textit{part}|}|\\
|\||fi|
\end{tabular}
\end{center}
%
The conditional |\ifchilddocmanual| is true whenever
a part to be included by |\input| is being compiled,
and the name of the part is stored in |\childdocname|.

%%%%%%%%%%%%%%%%%%%%%%%%%%%%%%%%%%%%%%%%
\DescribeMacro{\childdocby}
Each part to be included by |\input| should start with:
%
\begin{center}
\begin{tabular}{l}
|\input{childdoc.def}|\\
|\childdocby{|\textit{main}|}|\\
\end{tabular}
\end{center}
%
The directive |\childdocby| is similar to |\childdocof|
described in \secref{sec:include},
but the subsequent selection of content must be done manually.
To that end, both |\ifchilddoc| and |\ifchilddocmanual|
will be true upon processing of a part,
and the name of the part is stored in |\childdocname|.
Note that |\jobname| will be set to the filename of the current part
so that each part receives an individual |.aux| file
that does not interfere with the |.aux| file(s) of the main document.
This behaviour can be altered by the alternative form
|\childdocby[*]{|\textit{main}|}| (with a non-empty optional argument)
which uses the |.aux| file of the main document
by setting |\jobname| to \textit{main}.

%%%%%%%%%%%%%%%%%%%%%%%%%%%%%%%%%%%%%%%%%%%%%%%%%%%%%%%%%%%%%%%%%%%%%%%%%%%%%%%%
\subsection{Driver Development}
\label{sec:driver}

The \textsf{childdoc} mechanism can also be use for the development
of definition files such as \LaTeX{} styles or classes.
This case differs from the above setup with multiple parts
included by |\include| in that no |\includeonly| should be invoked.
This can be achieved by starting the include file
(before |\ProvidesPackage|) with:
%
\begin{center}
\begin{tabular}{l}
|\input{childdoc.def}|\\
|\childdocforward{|\textit{main}|}|\\
\end{tabular}
\end{center}
%
or alternatively with:
%
\begin{center}
\begin{tabular}{l}
|\input{childdoc.def}|\\
|\childdocby{|\textit{main}|}|\\
\end{tabular}
\end{center}
%
Both forms have slightly different effects as described above.
The main file is prepared as usual, see \secref{sec:include}.

%%%%%%%%%%%%%%%%%%%%%%%%%%%%%%%%%%%%%%%%%%%%%%%%%%%%%%%%%%%%%%%%%%%%%%%%%%%%%%%%
\subsection{Legacy Detection}
\label{sec:detection}

The directive |\childdocmain| in the main file can detect
whether the complete document or merely a child is to be compiled
even without using the directive |\childdocof|.
This method is deprecated because it is less robust
and there is no compelling reason to use it;
it is merely provided for backward compatibility
and it may be removed in future versions.

If the detection mechanism is to be used,
it is mandatory to correctly specify
the filename of the main file as the argument of |\childdocmain|:
%
\begin{center}
\begin{tabular}{l}
|\input{childdoc.def}|\\
|\childdocmain{|\textit{main}|}|\\
\end{tabular}
\end{center}
%
If |\jobname| does not match the argument \textit{main} of |\childdocmain|,
it is assumed that |\jobname| points to the child file to be compiled.
When using |\childdocmain| with the main file specified as argument,
it suffices to start a child file
with just |\input{|\textit{main}|}|
without loading of the package and using |\childdocof|.
If instead all processing is done
with the appropriate \textsf{childdoc} directives,
the argument of \textit{main} of |\childdocmain| can be empty.

An alternative version of the command line processing described
in \secref{sec:commandline} using the detection mechanism reads:
%
\begin{center}
|... -jobname "|\textit{target}|" "|[\textit{flags}]%
[|\def\jobname{|\textit{dest}|}|]|\input{|\textit{main}|}"|
\end{center}

%%%%%%%%%%%%%%%%%%%%%%%%%%%%%%%%%%%%%%%%%%%%%%%%%%%%%%%%%%%%%%%%%%%%%%%%%%%%%%%%
\subsection{Manual Code}
\label{sec:manual}

In case one cannot be certain whether the definitions file |childdoc.def|
is installed on the target \TeX{} distribution
and one prefers not to ship it,
it is conceivable to paste a few relevant commands into the sources.

To that end, drop all statements |\input{childdoc.def}|
and perform the replacements as outlined below.
Instead of |\childdocmain{|\textit{main}|}| add the following code
to the top of the main file:
%
\begin{center}
\begin{tabular}{l}
|\||ifdefined\childdocname\endinput\||fi\newif\ifchilddoc|\\
|\edef\childdocname{\scantokens\expandafter{\jobname\noexpand}}|\\
|\def\childdocmain{|\textit{main}|}\||ifx\childdocmain\childdocname\||else|\\
|\childdoctrue\includeonly{\childdocname}\let\jobname\childdocmain\||fi|\\
\end{tabular}
\end{center}
%
Instead of |\childdocof{|\textit{main}|}| just include the main file
at the top of each child file:
%
\begin{center}
|\input{|\textit{main}|}|
\end{center}
%
A simple redirection |\childdocforward{|\textit{dest}|}| is achieved by:
%
\begin{center}
|\def\jobname{|\textit{dest}|}\input{\jobname}|
\end{center}
%
The redirection with prefix
|\childdocforwardprefix[|\textit{prefix}|]{|\textit{dest}|}|
is accomplished by:
%
\begin{center}
\begin{tabular}{l}
|{\edef\jobname{\scantokens\expandafter{\jobname\noexpand}}|\\
|\def\redirectjob |\textit{prefix}|#1~~~{\gdef\jobname{|\textit{dest}|#1}}|\\
|\expandafter\redirectjob\jobname~~~}\input{\jobname}|
\end{tabular}
\end{center}

In an alternative approach,
child documents can be compiled by a specific command line
without additional code or specific definitions:
%
\begin{center}
|... -jobname "|\textit{target}|" "|[\textit{flags}]%
|\includeonly{|\textit{dest}|}\input{|\textit{main}|}"|
\end{center}
%

%%%%%%%%%%%%%%%%%%%%%%%%%%%%%%%%%%%%%%%%%%%%%%%%%%%%%%%%%%%%%%%%%%%%%%%%%%%%%%%%
%%%%%%%%%%%%%%%%%%%%%%%%%%%%%%%%%%%%%%%%%%%%%%%%%%%%%%%%%%%%%%%%%%%%%%%%%%%%%%%%
\section{Information}

%%%%%%%%%%%%%%%%%%%%%%%%%%%%%%%%%%%%%%%%%%%%%%%%%%%%%%%%%%%%%%%%%%%%%%%%%%%%%%%%
\subsection{Copyright}

Copyright \copyright{} 2017--2018 Niklas Beisert

This work may be distributed and/or modified under the
conditions of the \LaTeX{} Project Public License, either version 1.3
of this license or (at your option) any later version.
The latest version of this license is in
  \url{http://www.latex-project.org/lppl.txt}
and version 1.3 or later is part of all distributions of \LaTeX{}
version 2005/12/01 or later.

This work has the LPPL maintenance status `maintained'.

The Current Maintainer of this work is Niklas Beisert.

This work consists of the files |README.txt|, |childdoc.ins| and |childdoc.dtx|
as well as the derived files |childdoc.def|, |cdocsamp.tex|
with |cdocsch1.tex|, |cdocsch2.tex|, |cdocspt3.tex|, |cdocspt4.tex|,
|cdocsdrf.tex|, |cdocsfn1.tex|, |cdocsfn2.tex|
as well as |childdoc.pdf|.

%%%%%%%%%%%%%%%%%%%%%%%%%%%%%%%%%%%%%%%%%%%%%%%%%%%%%%%%%%%%%%%%%%%%%%%%%%%%%%%%
\subsection{Files and Installation}

The package consists of the files:
%
\begin{center}
\begin{tabular}{ll}
    |README.txt|   & readme file \\
    |childdoc.ins| & installation file \\
    |childdoc.dtx| & source file \\
    |childdoc.def| & definition file \\
    |cdocsamp.tex| & sample main file \\
    |cdocsch1.tex| & sample include file \\
    |cdocsch2.tex| & sample include file \\
    |cdocspt3.tex| & sample part file \\
    |cdocspt4.tex| & sample part file \\
    |cdocsdrf.tex| & sample redirection file \\
    |cdocsfn1.tex| & sample redirection file \\
    |cdocsfn2.tex| & sample redirection file \\
    |childdoc.pdf| & manual
\end{tabular}
\end{center}
%
The distribution consists of the files
|README.txt|, |childdoc.ins| and |childdoc.dtx|.
%
\begin{itemize}
\item
Run (pdf)\LaTeX{} on |childdoc.dtx|
to compile the manual |childdoc.pdf| (this file).
\item
Run \LaTeX{} on |childdoc.ins| to create the definitions file |childdoc.def|
and the sample |cdocsamp.tex| with include files
|cdocsch1.tex|, |cdocsch2.tex|, |cdocspt3.tex|, |cdocspt4.tex|,
|cdocsdrf.tex|, |cdocsfn1.tex|, |cdocsfn2.tex|.
Then copy the file |childdoc.def| to an appropriate directory of your \LaTeX{}
distribution, e.g.\ \textit{texmf-root}|/tex/latex/childdoc|.
\end{itemize}

%%%%%%%%%%%%%%%%%%%%%%%%%%%%%%%%%%%%%%%%%%%%%%%%%%%%%%%%%%%%%%%%%%%%%%%%%%%%%%%%
\subsection{Related CTAN Packages}

There are several other packages which offer a similar functionality:
%
\begin{itemize}
\item
The packages
\href{http://ctan.org/pkg/docmute}{\textsf{docmute}},
\href{http://ctan.org/pkg/includex}{\textsf{includex}} and
\href{http://ctan.org/pkg/standalone}{\textsf{standalone}}
provide commands to include only the document body of
a child file thus allowing both files to be compiled individually.
\item
The packages \href{http://ctan.org/pkg/subdocs}{\textsf{subdocs}}
and \href{http://ctan.org/pkg/subfiles}{\textsf{subfiles}}
provide structures in which the main and child documents can be
encapsulated and allowing them to be compiled individually.
The inclusion mechanism is different from the conventional |\include|.
\item
The package \href{http://ctan.org/pkg/combine}{\textsf{combine}}
is an elaborate solution to combine several documents into one.
\end{itemize}
%
See also the CTAN topic \href{http://ctan.org/topic/subdocs}{\textsf{subdocs}}
for further related packages.
The present package differs from the above solutions in that
a document structure constructed with the conventional |\include| mechanism
just needs two extra commands at the top of every file
such that all constituent files can be compiled individually.

%%%%%%%%%%%%%%%%%%%%%%%%%%%%%%%%%%%%%%%%%%%%%%%%%%%%%%%%%%%%%%%%%%%%%%%%%%%%%%%%
%\subsection{Feature Suggestions}
%
%The following is a list of features which may be useful for future
%versions of this package:
%%
%\begin{itemize}
%\item
%\ldots
%\end{itemize}

%%%%%%%%%%%%%%%%%%%%%%%%%%%%%%%%%%%%%%%%%%%%%%%%%%%%%%%%%%%%%%%%%%%%%%%%%%%%%%%%
\subsection{Revision History}

%%%%%%%%%%%%%%%%%%%%%%%%%%%%%%%%%%%%%%%%
\paragraph{v2.0:} 2018/12/30

\begin{itemize}
\item
immediate forward processing
\item
added |\childdocby| mechanism
\item
manual restructured
\end{itemize}

%%%%%%%%%%%%%%%%%%%%%%%%%%%%%%%%%%%%%%%%
\paragraph{v1.6:} 2018/01/17

\begin{itemize}
\item
application for development of include files
\item
corrections to manual
\end{itemize}

%%%%%%%%%%%%%%%%%%%%%%%%%%%%%%%%%%%%%%%%
\paragraph{v1.5:} 2017/05/21

\begin{itemize}
\item
more complete structuring introduced
\item
|\childdocof| introduced
\item
|\childdoc| renamed to |\childdocmain|
\item
|\childredirect| renamed to |\childdocforward| and |\childdocforwardprefix|
and functionality expanded
\end{itemize}

%%%%%%%%%%%%%%%%%%%%%%%%%%%%%%%%%%%%%%%%
\paragraph{v1.0:} 2017/04/27

\begin{itemize}
\item
manual and install package
\item
first version published on CTAN
\end{itemize}

%%%%%%%%%%%%%%%%%%%%%%%%%%%%%%%%%%%%%%%%
\paragraph{v0.6:} 2017/04/26

\begin{itemize}
\item
redirection mechanism added
\end{itemize}

%%%%%%%%%%%%%%%%%%%%%%%%%%%%%%%%%%%%%%%%
\paragraph{v0.5:} 2017/04/26

\begin{itemize}
\item
functionality in definition file
\end{itemize}


%%%%%%%%%%%%%%%%%%%%%%%%%%%%%%%%%%%%%%%%%%%%%%%%%%%%%%%%%%%%%%%%%%%%%%%%%%%%%%%%
%%%%%%%%%%%%%%%%%%%%%%%%%%%%%%%%%%%%%%%%%%%%%%%%%%%%%%%%%%%%%%%%%%%%%%%%%%%%%%%%
%%%%%%%%%%%%%%%%%%%%%%%%%%%%%%%%%%%%%%%%%%%%%%%%%%%%%%%%%%%%%%%%%%%%%%%%%%%%%%%%
\appendix

\settowidth\MacroIndent{\rmfamily\scriptsize 000\ }

 \DocInput{childdoc.dtx}

\end{document}
%</driver>
% \fi
%
% %%%%%%%%%%%%%%%%%%%%%%%%%%%%%%%%%%%%%%%%%%%%%%%%%%%%%%%%%%%%%%%%%%%%%%%%%%%%%%
% %%%%%%%%%%%%%%%%%%%%%%%%%%%%%%%%%%%%%%%%%%%%%%%%%%%%%%%%%%%%%%%%%%%%%%%%%%%%%%
% \section{Sample}
%\iffalse
%<*samplemain>
%\fi
%
% The following presents a sample document
% with two chapters, two parts, a title page,
% a compile flag as well as three forwarding files to set the flag.
% It consists of eight |.tex| files:
% \begin{center}
% \begin{tabular}{ll}
% |cdocsamp.tex|&main file\\
% |cdocsch1.tex|&include file for chapter 1\\
% |cdocsch2.tex|&include file for chapter 2\\
% |cdocspt3.tex|&include file for part 3\\
% |cdocspt4.tex|&include file for part 4\\
% |cdocsdrf.tex|&forwarding file for main file in draft mode\\
% |cdocsfi1.tex|&forwarding file for final version of chapter 1\\
% |cdocsfi2.tex|&forwarding file for final version of chapter 2\\
% \end{tabular}
% \end{center}
% Each of the eight files can be compiled directly by the \LaTeX{} compiler.
%
% %%%%%%%%%%%%%%%%%%%%%%%%%%%%%%%%%%%%%%
% \paragraph{Main File.}
%
% The main file is called |cdocsamp.tex|.
%
% Load the \textsf{childdoc} definitions and
% declare the filename for the main document:
%    \begin{macrocode}
\input{childdoc.def}
\childdocmain{}
%    \end{macrocode}

% Optional override for |\version| flag:
%    \begin{macrocode}
%%\ifchilddoc\else\providecommand{\version}{draft}\fi
%    \end{macrocode}

% Define the default values for the |\version| flag
% (|final| for the main file and |draft| for childs):
%    \begin{macrocode}
\ifchilddoc
\providecommand{\version}{draft}
\else
\providecommand{\version}{final}
\fi
%    \end{macrocode}

% Load the standard document class:
%    \begin{macrocode}
\documentclass[12pt]{article}
%    \end{macrocode}

% Start the document body:
%    \begin{macrocode}
\begin{document}
%    \end{macrocode}

% Declare a title page.
% Print title, part of document being processed and version flag:
%    \begin{macrocode}
\addtocounter{page}{-1}
\begin{center}
{\LARGE\bfseries{}childdoc example\par}
\vspace{1cm}
\ifchilddoc
\ifchilddocmanual part\else chapter\fi:
`\childdocname' of `\childdocjob'\par
\else
main document: `\childdocjob'\par
\fi
version: \version\par
\end{center}
\newpage
%    \end{macrocode}

% Manually include selected file,
% otherwise process as usual:
%    \begin{macrocode}
\ifchilddocmanual
\section*{part `\childdocname'}
\input{\childdocname}
\else
%    \end{macrocode}

% Include the two chapters:
%    \begin{macrocode}
\include{cdocsch1}
\include{cdocsch2}
%    \end{macrocode}

% Include the two parts unless only chapters should be displayed:
%    \begin{macrocode}
\ifchilddoc\else
\section{part three}
\input{cdocspt3}
\section{part four}
\input{cdocspt4}
\fi
%    \end{macrocode}

% Process as usual until here:
%    \begin{macrocode}
\fi
%    \end{macrocode}

% End of document body:
%    \begin{macrocode}
\end{document}
%    \end{macrocode}
%\iffalse
%</samplemain>
%\fi
%
% %%%%%%%%%%%%%%%%%%%%%%%%%%%%%%%%%%%%%%
% \paragraph{Chapter Include Files.}
%
% The include files are called |cdocsch1.tex| and |cdocsch2.tex|.
%
%\iffalse
%<*samplechap1|samplechap2>
%\fi

% Optional override for |\version| flag:
%    \begin{macrocode}
%%\providecommand{\version}{final}
%    \end{macrocode}

% Include the main document:
%    \begin{macrocode}
\input{childdoc.def}
\childdocof{cdocsamp}
%    \end{macrocode}

%\iffalse
%</samplechap1|samplechap2>
%\fi
%
%\iffalse
%<*samplechap1>
%\fi
% Some text for chapter 1:
%    \begin{macrocode}
\section{one}
some text in chapter one
%    \end{macrocode}

%\iffalse
%</samplechap1>
%\fi
% Some text for chapter 2:
%\iffalse
%<*samplechap2>
%\fi
%    \begin{macrocode}
\section{two}
more text in chapter two
%    \end{macrocode}

%\iffalse
%</samplechap2>
%\fi
%
% %%%%%%%%%%%%%%%%%%%%%%%%%%%%%%%%%%%%%%
% \paragraph{Part Include Files.}
%
% The include files are called |cdocspt3.tex| and |cdocspt4.tex|.
%
%\iffalse
%<*samplepart3|samplepart4>
%\fi

% Optional override for |\version| flag:
%    \begin{macrocode}
%%\providecommand{\version}{final}
%    \end{macrocode}

% Include the main document:
%    \begin{macrocode}
\input{childdoc.def}
\childdocby{cdocsamp}
%    \end{macrocode}

%\iffalse
%</samplepart3|samplepart4>
%\fi
%
%\iffalse
%<*samplepart3>
%\fi
% Some text for part 3:
%    \begin{macrocode}
some text in part three
%    \end{macrocode}

%\iffalse
%</samplepart3>
%\fi
% Some text for part 4:
%\iffalse
%<*samplepart4>
%\fi
%    \begin{macrocode}
more text in part four
%    \end{macrocode}

%\iffalse
%</samplepart4>
%\fi
%
% %%%%%%%%%%%%%%%%%%%%%%%%%%%%%%%%%%%%%%
% \paragraph{Forwarding for a Complete Draft.}
%
% The following forwarding file |cdocsdrf.tex|
% compiles the main document in draft mode:
%\iffalse
%<*sampledraft>
%\fi
%    \begin{macrocode}
\def\version{draft}
\input{childdoc.def}
\childdocforward{cdocsamp}
%    \end{macrocode}

%\iffalse
%</sampledraft>
%\fi
%
% %%%%%%%%%%%%%%%%%%%%%%%%%%%%%%%%%%%%%%
% \paragraph{Forwarding for Final Version of the Chapters.}
%
% The following forwarding files |cdocsfn1.tex| and |cdocsfn2.tex|
% (with identical content)
% compile the final versions of the child documents
% |cdocsch1.tex| and |cdocsch2.tex|, respectively:
%\iffalse
%<*samplefinal>
%\fi
%    \begin{macrocode}
\def\version{final}
\input{childdoc.def}
\childdocforwardprefix[cdocsamp]{cdocsfn}{cdocsch}
%    \end{macrocode}

%\iffalse
%</samplefinal>
%\fi
%
% %%%%%%%%%%%%%%%%%%%%%%%%%%%%%%%%%%%%%%
% \paragraph{Command Line Processing.}
%
% The following three command lines generate the output files
% |cdocscld|, |cdocscl1| and |cdocscl2|
% which should be identical to
% |cdocsdrf|, |cdocsch1| and |cdocsfn2|, respectively:
% \begin{center}
% \begin{tabular}{l}
% |latex -jobname cdocscld \|\\
% |  "\def\version{draft}\input{childdoc.def}\childdocforward{cdocsamp}"|\\
% |latex -jobname cdocscl1 \|\\
% |  "\input{childdoc.def}\childdocforward[cdocsamp]{cdocsch1}"|\\
% |latex -jobname cdocscl2 \|\\
% |  "\def\version{final}\input{childdoc.def}\childdocforward{cdocsch2}"|
% \end{tabular}
% \end{center}
% Note that the trailing backslash on each first line
% merely continues the input to the second line
% (for convenient cut ant paste).
% Furthermore, the command |latex| can be replaced by any
% of its alternative versions such as |pdflatex|.
%
% %%%%%%%%%%%%%%%%%%%%%%%%%%%%%%%%%%%%%%%%%%%%%%%%%%%%%%%%%%%%%%%%%%%%%%%%%%%%%%
% %%%%%%%%%%%%%%%%%%%%%%%%%%%%%%%%%%%%%%%%%%%%%%%%%%%%%%%%%%%%%%%%%%%%%%%%%%%%%%
% \section{Implementation}
%\iffalse
%<*package>
%\fi
%
% This section describes the definitions file |childdoc.def|.

% The definitions cannot be loaded using |\usepackage| or |\RequirePackage|
% which has a mechanism to prevent loading a style file more than once.
% When loading the definitions by means of |\input|
% multiple instances have to be prevented manually:
%\iffalse
%This code needs to be before the `\ProvidesFile' directive
%which is defined at the beginning of this file.
%Therefore it is also placed there and commented out here.
%</package>
%<*discard>
%\fi
%    \begin{macrocode}
\ifdefined\childdocmain\endinput\fi
%    \end{macrocode}
%\iffalse
%</discard>
%<*package>
%\fi
%
% \macro{\ifchilddoc}
% \macro{\ifchilddocmanual}
% The conditional |\ifchilddoc| tells whether a
% child (true) or main (false) document is being compiled.
% The conditional |\ifchilddocmanual| tells whether
% the |\includeonly| mechanism is used (false) or
% the selection of child files must be performed manually (true).
% The definitions initialise to false:
%    \begin{macrocode}
\newif\ifchilddoc
\newif\ifchilddocmanual
%    \end{macrocode}

% \macro{\childdocname}
% \macro{\childdocjob}
% The macro |\childdocname| stores the name of the main document
% to be compiled. The macro |\childdocjob| stores the name of
% the document on which the \LaTeX{} compiler was originally invoked.
% The content of |\jobname| cannot be compared
% to filenames specified in the source due to different catcodes.
% The following code rescans |\jobname|, stores the result
% in |\childdocname| and saves a copy in |\childdocjob|:
%    \begin{macrocode}
\edef\childdocname{\scantokens\expandafter{\jobname\noexpand}}
\let\childdocjob\childdocname
%    \end{macrocode}

% \macro{\childdocdisable}
% The macro |\childdocdisable| prevents the main file
% from being processed more than once.
% At this stage, the main document command |\childdocmain|
% is assumed to be called once again where it should do nothing.
% Any subsequent call to it should prevent
% a secondary processing of the main document
% It overwrites the forwarding commands
% |\childdocof| and |\childdocforward|
% with empty macros to prevent further inclusions of the main document:
%    \begin{macrocode}
\newcommand{\childdocdisable}
{
  \renewcommand{\childdocmain}[1]{\renewcommand{\childdocmain}[1]{\endinput}}
  \renewcommand{\childdocof}[1]{}
  \renewcommand{\childdocby}[2][]{}
  \renewcommand{\childdocforward}[2][]{}
  \renewcommand{\childdocdisable}{}
}
%    \end{macrocode}

% \macro{\childdocmain}
% The macro |\childdocmain| is to be called at the top of the main file
% with nothing or the main filename (without extension) as argument.
% First, it breaks loops.
% If the argument is not empty and does not match |\childdocname|
% (which is set by the first inclusion of |childdoc.def|),
% |\ifchilddoc| is set to true, |\includeonly| is applied to the child file
% and |\jobname| is set to the main file
% (for proper handling of |.aux| files):
%    \begin{macrocode}
\newcommand{\childdocmain}[1]
{
  \childdocdisable\childdocmain{}
  \if?#1?\else
    \begingroup
      \def\childdoctmp{#1}
      \ifx\childdoctmp\childdocname
        \def\childdoctmp{}
      \else
        \def\childdoctmp
        {
          \childdoctrue
          \includeonly{\childdocname}
          \def\childdocjob{#1}
          \def\jobname{#1}
        }
      \fi
      \expandafter
    \endgroup
    \childdoctmp
  \fi
}
%    \end{macrocode}

% \macro{\childdocof}
% The command |\childdocof| redirects
% compilation to the main file |#1|.
%    \begin{macrocode}
\newcommand{\childdocof}[1]
{
  \childdocdisable
  \childdoctrue
  \includeonly{\childdocname}
  \def\jobname{#1}
  \def\childdocjob{#1}
  \input{#1}
}
%    \end{macrocode}

% \macro{\childdocby}
% The command |\childdocby| ....
%    \begin{macrocode}
\newcommand{\childdocby}[2][]
{
  \childdocdisable
  \childdoctrue
  \childdocmanualtrue
  \if?#1?\else
    \def\jobname{#2}
  \fi
  \def\childdocjob{#2}
  \input{#2}
  \endinput
}
%    \end{macrocode}

% \macro{\childdocforward}
% The command |\childdocforward| redirects
% compilation to the main file or
% (if the optional argument is given) a child file.
% Parameters are set as if the main file
% or a child file starting with |\childdocof| was compiled.
% Then compilation is handed over to the main file:
%    \begin{macrocode}
\newcommand{\childdocforward}[2][]
{
  \begingroup
    \if?#1?
      \def\childdoctmp
      {
        \def\childdocname{#2}
        \def\childdocjob{#2}
        \def\jobname{#2}
        \input{#2}
        \endinput
      }
    \else
      \def\childdoctmp
      {
        \childdocdisable
        \def\childdocname{#2}
        \childdoctrue
        \includeonly{#2}
        \def\childdocjob{#1}
        \def\jobname{#1}
        \input{#1}
        \endinput
      }
    \fi
    \expandafter
  \endgroup
  \childdoctmp
}
%    \end{macrocode}

% \macro{\childdocforwardprefix}
% The command |\childdocforwardprefix| redirects
% compilation to the main or a child file by means of a pattern.
% The prefix |#1| in the current filename is replaced by |#2|
% and the suffix of the current filename is kept
% (it is assumed that the filename does not contain the substring `|~~~|'
% which is used as a delimiter).
% Compilation is handed over to the new file by |\childdocforward|:
%    \begin{macrocode}
\newcommand{\childdocforwardprefix}[3][]
{
  \begingroup
    \def\childdocextract #2##1~~~{\def\childdoctmp{\childdocforward[#1]{#3##1}}}
    \expandafter\childdocextract\childdocname~~~
    \expandafter
  \endgroup
  \childdoctmp
}
%    \end{macrocode}

% \macro{\childdoc}
% The deprecated macro |\childdoc| is a legacy version of |\childdocmain|:
%    \begin{macrocode}
\newcommand{\childdoc}{\childdocmain}
%    \end{macrocode}

% \macro{\childdocredirect}
% The deprecated macro |\childdocredirect| is a legacy version
% of |\childdocforward| and |\childdocforwardprefix|:
%    \begin{macrocode}
\newcommand{\childdocredirect}[2][]
{
  \begingroup
    \if?#1?
      \def\childdoctmp{\childdocforward{#2}}
    \else
      \def\childdoctmp{\childdocforwardprefix{#1}{#2}}
    \fi
    \expandafter
  \endgroup
  \childdoctmp
}
%    \end{macrocode}

%\iffalse
%</package>
%\fi
%
\endinput
|\\
|\childdocby{|\textit{main}|}|\\
\end{tabular}
\end{center}
%
Both forms have slightly different effects as described above.
The main file is prepared as usual, see \secref{sec:include}.

%%%%%%%%%%%%%%%%%%%%%%%%%%%%%%%%%%%%%%%%%%%%%%%%%%%%%%%%%%%%%%%%%%%%%%%%%%%%%%%%
\subsection{Legacy Detection}
\label{sec:detection}

The directive |\childdocmain| in the main file can detect
whether the complete document or merely a child is to be compiled
even without using the directive |\childdocof|.
This method is deprecated because it is less robust
and there is no compelling reason to use it;
it is merely provided for backward compatibility
and it may be removed in future versions.

If the detection mechanism is to be used,
it is mandatory to correctly specify
the filename of the main file as the argument of |\childdocmain|:
%
\begin{center}
\begin{tabular}{l}
|% \iffalse
%
% childdoc.dtx Copyright (C) 2017-2018 Niklas Beisert
%
% This work may be distributed and/or modified under the
% conditions of the LaTeX Project Public License, either version 1.3
% of this license or (at your option) any later version.
% The latest version of this license is in
%   http://www.latex-project.org/lppl.txt
% and version 1.3 or later is part of all distributions of LaTeX
% version 2005/12/01 or later.
%
% This work has the LPPL maintenance status `maintained'.
%
% The Current Maintainer of this work is Niklas Beisert.
%
% This work consists of the files childdoc.dtx and childdoc.ins
% and the derived files childdoc.def and cdocsamp.tex with
% cdocsch1.tex, cdocsch2.tex, cdocsdrf.tex, cdocsfn1.tex, cdocsfn2.tex.
%
%<package>\ifdefined\childdocmain\endinput\fi
%<package>\ProvidesFile{childdoc.def}[2018/12/30 v2.0 child document driver]
%<samplemain>\ProvidesFile{cdocsamp.tex}[2018/12/30 v2.0 sample for childdoc]
%<*driver>
%\ProvidesFile{childdoc.drv}[2018/12/30 v2.0 childdoc reference manual file]
\PassOptionsToClass{10pt,a4paper}{article}
\documentclass{ltxdoc}

\usepackage[margin=35mm]{geometry}
\usepackage{hyperref}
\usepackage{hyperxmp}
\usepackage[usenames]{color}

\hypersetup{colorlinks=true}
\hypersetup{pdfstartview=FitH}
\hypersetup{pdfpagemode=UseNone}
\hypersetup{pdfsource={}}
\hypersetup{pdflang={en-UK}}
\hypersetup{pdfcopyright={Copyright 2017-2018 Niklas Beisert.
  This work may be distributed and/or modified under the
  conditions of the LaTeX Project Public License, either version 1.3
  of this license or (at your option) any later version.}}
\hypersetup{pdflicenseurl={http://www.latex-project.org/lppl.txt}}
\hypersetup{pdfcontactaddress={ETH Zurich, ITP, HIT K,
  Wolfgang-Pauli-Strasse 27}}
\hypersetup{pdfcontactpostcode={8093}}
\hypersetup{pdfcontactcity={Zurich}}
\hypersetup{pdfcontactcountry={Switzerland}}
\hypersetup{pdfcontactemail={nbeisert@itp.phys.ethz.ch}}
\hypersetup{pdfcontacturl={http://people.phys.ethz.ch/\xmptilde nbeisert/}}

\newcommand{\secref}[1]{\hyperref[#1]{section \ref*{#1}}}

\parskip1ex
\parindent0pt
\let\olditemize\itemize
\def\itemize{\olditemize\parskip0pt}

\begin{document}

\title{The \textsf{childdoc} Package}
\hypersetup{pdftitle={The childdoc Package}}
\author{Niklas Beisert\\[2ex]
  Institut f\"ur Theoretische Physik\\
  Eidgen\"ossische Technische Hochschule Z\"urich\\
  Wolfgang-Pauli-Strasse 27, 8093 Z\"urich, Switzerland\\[1ex]
  \href{mailto:nbeisert@itp.phys.ethz.ch}
  {\texttt{nbeisert@itp.phys.ethz.ch}}}
\hypersetup{pdfauthor={Niklas Beisert}}
\hypersetup{pdfsubject={Manual for the LaTeX2e Package childdoc}}
\date{30 December 2018, \textsf{v2.0}}
\maketitle

\begin{abstract}\noindent
\textsf{childdoc} is a \LaTeXe{} package
that enables the direct compilation
of document sections included by |\include|
to individual files.
\end{abstract}

\begingroup
\parskip0ex
\tableofcontents
\endgroup

%%%%%%%%%%%%%%%%%%%%%%%%%%%%%%%%%%%%%%%%%%%%%%%%%%%%%%%%%%%%%%%%%%%%%%%%%%%%%%%%
%%%%%%%%%%%%%%%%%%%%%%%%%%%%%%%%%%%%%%%%%%%%%%%%%%%%%%%%%%%%%%%%%%%%%%%%%%%%%%%%
\section{Introduction}

\LaTeX{} provides a mechanism to structure a large document (such as a book)
into a main file and several child files (containing the chapters)
using the |\include| command.
This mechanism is beneficial for documents
which span hundreds of pages in order to
make the source file(s) more manageable.
Moreover, compilation can be restricted to
selected child files by means of the |\includeonly| command.
The latter feature can be used to reduce the compilation time while editing
(this was significantly more useful in the earlier days of \LaTeX{})
or to generate a smaller document which is easier to navigate.
Another application of |\includeonly| is to generate
documents consisting of selected parts of the complete document.

However, there are a few drawbacks of the plain |\include| mechanism:
\begin{itemize}
\item
The child files cannot be compiled on their own,
they can only be compiled via the main file.
A naive editing environment
(such as a text editor with an option
to have the current file processed by \LaTeX)
may require one to switch to the main file before compiling;
attempting to compile the child file produces errors.
\item
The main file must be modified (each time)
to adjust the |\includeonly| command
to the present needs. This easily leaves the main file in a messy state.
\item
The generated document will always carry the filename
of the main document. This is inconvenient if
several child files are to be compiled and
to be kept for distribution.
\end{itemize}

The present package provides a simple interface
to make child files individually compilable by \LaTeX{}.
Compiling a child file then has the same effect as compiling
the main file with an |\includeonly| command
to select the appropriate child.
Moreover the generated document will carry the name of the child
rather than the main file.
This resolves all three above issues.

This feature is meant to make the editing of books,
thesis documents and lecture notes somewhat more convenient.
However, the package can also be used efficiently for
composing a series of documents (such as exercise sheets)
which are typically distributed individually.
It then assists the author in generating the individual documents
(potentially in different versions)
as well as a document containing the collected series.
Another application is in developing style files
or other kinds of included material
where compilation of the style file could redirect
to a sample or test file.

%%%%%%%%%%%%%%%%%%%%%%%%%%%%%%%%%%%%%%%%%%%%%%%%%%%%%%%%%%%%%%%%%%%%%%%%%%%%%%%%
%%%%%%%%%%%%%%%%%%%%%%%%%%%%%%%%%%%%%%%%%%%%%%%%%%%%%%%%%%%%%%%%%%%%%%%%%%%%%%%%
\section{Usage}

First of all, the package \textsf{childdoc} is \emph{not} a standard
\LaTeXe{} |.sty| style file! Therefore it needs to be invoked in
a non-standard way.

%%%%%%%%%%%%%%%%%%%%%%%%%%%%%%%%%%%%%%%%%%%%%%%%%%%%%%%%%%%%%%%%%%%%%%%%%%%%%%%%
\subsection{Included Files}
\label{sec:include}

%%%%%%%%%%%%%%%%%%%%%%%%%%%%%%%%%%%%%%%%
\DescribeMacro{\childdocmain}
To use the package, add the commands
\begin{center}
\begin{tabular}{l}
|\input{childdoc.def}|\\
|\childdocmain{}|\\
\end{tabular}
\end{center}
at the very top of the main \LaTeX{} file,
in particular \emph{before} the |\documentclass| statement!
The argument of |\childdocmain| should be left empty
(but it must be present).

%%%%%%%%%%%%%%%%%%%%%%%%%%%%%%%%%%%%%%%%
\DescribeMacro{\childdocof}
Furthermore, add the commands
\begin{center}
\begin{tabular}{l}
|\input{childdoc.def}|\\
|\childdocof{|\textit{main}|}|\\
\end{tabular}
\end{center}
at the top of every child file \textit{child}
which is included by |\include{|\textit{child}|}|
from within the main file
(or at least for those files to be compiled individually).
The argument \textit{main} must be the filename of the main file.

There are a couple of
considerations in setting up the main and child documents:

%%%%%%%%%%%%%%%%%%%%%%%%%%%%%%%%%%%%%%%%
\paragraph{Restrictions.}

Please note the following restrictions:
\begin{itemize}
\item
|\childdocmain| must be called with one argument \textit{main}
to ensure compatibility with earlier version of the package.
It must either be empty (|\childdocmain{}|)
or precisely match the filename of the main file in which it is specified.
See \secref{sec:detection} for further information.
\item
The filename \textit{main} must be specified without the |.tex| extension.
\item
The filename \textit{main} is case sensitive
(even in case-insensitive file systems)
due to internal string comparison.
\item
The argument \textit{main} should be fully expanded, it cannot be a macro.
\item
Subdirectories and special characters should be avoided in filenames.
\item
The command |\childdocmain{|\textit{main}|}| must be followed by a whitespace.
It should not be followed immediately by another command
or by a comment mark `|%|'.
This is because the \TeX{} parser reads the token immediately following
the argument of |\childdocmain| and puts it
at the beginning of every child section;
however, a white\-space is ignored.
\end{itemize}

%%%%%%%%%%%%%%%%%%%%%%%%%%%%%%%%%%%%%%%%
\paragraph{Content of Main File.}

It is advisable to place all content in the child files included by |\include|.
Any output contained in the main file will appear in all child documents
unless suppressed manually;
it cannot be suppressed automatically by the |\includeonly| directive
and thus should normally be avoided.
A method to include some content in the main file
by means of conditional processing is described in \secref{sec:conditional}.

%%%%%%%%%%%%%%%%%%%%%%%%%%%%%%%%%%%%%%%%
\paragraph{Page Numbering.}

When only a part of the document is compiled,
the appropriate numbering of pages
(as well as other status parameters)
is determined from the |.aux| files.
The latter contain information from previous passes.
However this information needs to propagate through
all intermediate child documents.
Therefore the page numbering in child documents may well
be inconsistent until the complete document is compiled at least once.

A useful (if unconventional) way to always ensure a consistent
page numbering is to restart the numbering in each child document
and denote the pages by `\textit{child}|.|\textit{page}'
where \textit{child} represents the chapter/section number of the child file.
This can be achieved by the command
|\numberwithin{page}{|\textit{child}|}|
of the \textsf{amsmath} package
where \textit{child} can be |chapter| or |section|
depending on the chosen structuring.
Alternatively, one can modify the macro |\thepage| appropriately
and reset the counter |page| at the start of each child file.

%%%%%%%%%%%%%%%%%%%%%%%%%%%%%%%%%%%%%%%%%%%%%%%%%%%%%%%%%%%%%%%%%%%%%%%%%%%%%%%%
\subsection{Conditional Processing}
\label{sec:conditional}

The package provides a mechanism to compile different versions
of a document. To customise the versions further some conditional processing
can come in handy to distinguish which version is being compiled.
The package provides two macros to describe the compilation context:

%%%%%%%%%%%%%%%%%%%%%%%%%%%%%%%%%%%%%%%%
\DescribeMacro{\ifchilddoc}
The conditional |\ifchilddoc| distinguishes between the compilation of
child documents and the main document:
%
\begin{center}
|\ifchilddoc |\textit{child-code}| |[|\||else |\textit{main-code}]| \||fi|
\end{center}

%%%%%%%%%%%%%%%%%%%%%%%%%%%%%%%%%%%%%%%%
\DescribeMacro{\childdocname}
\DescribeMacro{\childdocjob}
The macro |\childdocname| contains the filename (without extension)
of the main or child file being processed.
Note that |\childdocjob| will always contain the name of the main file.

%%%%%%%%%%%%%%%%%%%%%%%%%%%%%%%%%%%%%%%%
\paragraph{Title Page.}

Conditional processing can be used to include a title or banner page
in the main document when proper precautions are taken.
Importantly, the code in the main file should ensure that the page counter
(as well as other status parameters which are stored in the |.aux| files)
takes the same value after the conditional processing.
Otherwise the page numbers may take divergent values
depending on which part is compiled.

For example, a title page could be declared by:
%
\begin{center}
\begin{tabular}{l}
|\ifchilddoc\||else|\\
|\addtocounter{page}{-1}|\\
\textit{code for title page}\\
|\newpage|\\
|\||fi|
\end{tabular}
\end{center}
%
A banner page for the child documents can be generated by:
%
\begin{center}
\begin{tabular}{l}
|\ifchilddoc|\\
|\addtocounter{page}{-1}|\\
\textit{code for banner page}\\
|\newpage|\\
|\||fi|
\end{tabular}
\end{center}
%
Here one could write a message such as:
\begin{center}
|This is the part \childdocname{} of \childdocjob{}.|
\end{center}

%%%%%%%%%%%%%%%%%%%%%%%%%%%%%%%%%%%%%%%%%%%%%%%%%%%%%%%%%%%%%%%%%%%%%%%%%%%%%%%%
\subsection{Flags}
\label{sec:flags}

The package makes it easy to generate different versions
of the main or child documents.
To this end compilation flags can be defined
and assigned different default values.
They will be particularly useful in conjunction
with the forwarding mechanism described in \secref{sec:forward}.

For example, it may be useful to have a flag |\version|
which can be set to |draft| or |final|.
The document source will contain some conditional code
depending on the value of |\version|.
Suppose further, the flag should default to |final| for the main file
and to |draft| for child files
which is a natural assignment for editing the document.
This is achieved by placing the following code
in the preamble of the main document
(below the |\childdocmain| directive):
%
\begin{center}
\begin{tabular}{l}
|\ifchilddoc|\\
|\providecommand{\version}{draft}|\\
|\||else|\\
|\providecommand{\version}{final}|\\
|\||fi|
\end{tabular}
\end{center}
%
The definition by |\providecommand| makes sure
that previous definitions are not overwritten.
Further statements |\providecommand{\version}{...}|
can thus be added before the above code to override it.

For the main file, one might add a line
(between |\childdocmain| and the above block)
%
\begin{center}
|%\ifchilddoc\||else\providecommand{\version}{draft}\||fi|
\end{center}
%
which can be uncommented to produce a draft version.
Likewise one can add a line to the very top of a child file
(above the |\childdocof{|\textit{main}|}| directive)
%
\begin{center}
|%\providecommand{\version}{final}|
\end{center}
%
which can be uncommented to produce the final version of this child document.

%%%%%%%%%%%%%%%%%%%%%%%%%%%%%%%%%%%%%%%%%%%%%%%%%%%%%%%%%%%%%%%%%%%%%%%%%%%%%%%%
\subsection{Forwarding}
\label{sec:forward}

Different versions of the main or child documents
using compilation flags as described in \secref{sec:flags}
can be (permanently) stored in different files
for convenient compilation, viewing and distribution.
To this end, the package defines a command
to pass on compilation to a different file:

%%%%%%%%%%%%%%%%%%%%%%%%%%%%%%%%%%%%%%%%
\DescribeMacro{\childdocforward}
The command |\childdocforward| redirects processing to
another source file:
%
\begin{center}
\begin{tabular}{l}
|\input{childdoc.def}|\\
|\childdocforward[|\textit{main}|]{|\textit{dest}|}|\\
\end{tabular}
\end{center}
%
The argument \textit{dest} is the destination file
(without extension).
It should be the main file or one of the child files.
Note that further \textsf{childdoc} directives
such as |\childdocof| and |\childdocforward|
in the indicated file will be processed in this form.
The optional argument \textit{main}
passes on directly to the main file \textit{main}
while pretending to compile the child \textit{dest}.
This form behaves as if \textit{dest}
issues |\childdocof{|\textit{main}|}| right away,
and no further \textsf{childdoc} directives will be processed.

%%%%%%%%%%%%%%%%%%%%%%%%%%%%%%%%%%%%%%%%
\DescribeMacro{\...prefix}
In the alternative form |\childdocforwardprefix|,
%
\begin{center}
\begin{tabular}{l}
|\input{childdoc.def}|\\
|\childdocforwardprefix[|\textit{main}|]{|\textit{prefix}|}{|\textit{dest}|}|
\end{tabular}
\end{center}
%
the destination file is determined by a pattern
depending on the current file:
To make this work, the current file must be called
`{\textit{prefix}\hspace{0.2em}\textit{suffix}}'
with \textit{prefix} matching precisely the argument.
Processing is then passed on to the file
`{\textit{dest}\hspace{0.2em}\textit{suffix}}'.
Surely, the same effect is achieved by
directly specifying the
argument `{\textit{dest}\hspace{0.2em}\textit{suffix}}'
in the first form.
However, that requires to set up a different file
for each child. With the alternative form of the command
all these files can have exactly the same content
which simplifies setting them up and maintaining them.

For example, the following file |draft.tex|
with a compilation flag |\version| as described in \secref{sec:flags}
compiles the main document as a draft:
%
\begin{center}
\begin{tabular}{l}
|\def\version{draft}|\\
|\input{childdoc.def}|\\
|\childdocforward{|\textit{main}|}|
\end{tabular}
\end{center}
%
Likewise, the following files |final|\textit{nn}|.tex|
compile the final version of the child document
|child|\textit{nn}|.tex|:
%
\begin{center}
\begin{tabular}{l}
|\def\version{final}|\\
|\input{childdoc.def}|\\
|\childdocforwardprefix{final}{child}|
\end{tabular}
\end{center}
%

Note that when several versions of a main file and/or of each child file
are to be generated, it may be convenient to set up a |Makefile| or
shell script to automatise the process.

%%%%%%%%%%%%%%%%%%%%%%%%%%%%%%%%%%%%%%%%%%%%%%%%%%%%%%%%%%%%%%%%%%%%%%%%%%%%%%%%
\subsection{Command Line Processing}
\label{sec:commandline}

The effect of redirection files can also be achieved by invoking
the \LaTeX{} compiler with a more elaborate command line.
Most conveniently this should be done as part
of a shell script or a |Makefile|.

When using \textsf{childdoc} in the main file, the following
command lines effectively perform a redirection
(note that depending on the shell being used,
backslashes may have to be doubled: `|\|' $\to$ `|\\|'):
%
\begin{center}
|... -jobname "|\textit{target}|" |\\|"|[\textit{flags}]%
|\input{childdoc.def}\childdocforward[|\textit{main}|]{|\textit{dest}|}"|
\end{center}
%
Here \textit{target} is the name of the output file,
\textit{main} is the name of the main file
and \textit{dest} is the name of the main or child file to be processed
(all filenames without extensions).
The optional argument \textit{main} can be omitted
if \textit{main} matches \textit{dest}.
Optionally, compilation \textit{flags} can be defined via |\def| commands.
This command line makes the \TeX{} engine believe
it is compiling the file \textit{target}
whose content is specified as the latter parameter.
The provided code then forwards the processing to
\textit{main} or \textit{dest} as described in \secref{sec:forward}.

%%%%%%%%%%%%%%%%%%%%%%%%%%%%%%%%%%%%%%%%%%%%%%%%%%%%%%%%%%%%%%%%%%%%%%%%%%%%%%%%
\subsection{Include by Input}
\label{sec:input}

Including child documents by |\include| has some restrictions by design.
Most notably, the content of a child document always occupies
its own set of pages; pages cannot be shared between child documents.
Usually, this behaviour makes perfect sense
because each child document contain an essential part of the document.
However, in some situations it may be desirable to compose
a document from a collection of parts
without having mandatory page breaks between then.
For this case, the package
provides a mechanism to include parts
by |\input| which can also be processed individually.
However, by construction this mechanism
requires manual handling of the content to be output.

%%%%%%%%%%%%%%%%%%%%%%%%%%%%%%%%%%%%%%%%
\DescribeMacro{\ifchilddocmanual}
The main file should be prepared as usual, see \secref{sec:include}.
However, the document body must make a distinction
between processing of an individual part and of the main document, e.g.:
%
\begin{center}
\begin{tabular}{l}
|\ifchilddocmanual|\\
|\input{\childdocname}|\\
|\||else|\\
\textit{document body with }|\input{|\textit{part}|}|\\
|\||fi|
\end{tabular}
\end{center}
%
The conditional |\ifchilddocmanual| is true whenever
a part to be included by |\input| is being compiled,
and the name of the part is stored in |\childdocname|.

%%%%%%%%%%%%%%%%%%%%%%%%%%%%%%%%%%%%%%%%
\DescribeMacro{\childdocby}
Each part to be included by |\input| should start with:
%
\begin{center}
\begin{tabular}{l}
|\input{childdoc.def}|\\
|\childdocby{|\textit{main}|}|\\
\end{tabular}
\end{center}
%
The directive |\childdocby| is similar to |\childdocof|
described in \secref{sec:include},
but the subsequent selection of content must be done manually.
To that end, both |\ifchilddoc| and |\ifchilddocmanual|
will be true upon processing of a part,
and the name of the part is stored in |\childdocname|.
Note that |\jobname| will be set to the filename of the current part
so that each part receives an individual |.aux| file
that does not interfere with the |.aux| file(s) of the main document.
This behaviour can be altered by the alternative form
|\childdocby[*]{|\textit{main}|}| (with a non-empty optional argument)
which uses the |.aux| file of the main document
by setting |\jobname| to \textit{main}.

%%%%%%%%%%%%%%%%%%%%%%%%%%%%%%%%%%%%%%%%%%%%%%%%%%%%%%%%%%%%%%%%%%%%%%%%%%%%%%%%
\subsection{Driver Development}
\label{sec:driver}

The \textsf{childdoc} mechanism can also be use for the development
of definition files such as \LaTeX{} styles or classes.
This case differs from the above setup with multiple parts
included by |\include| in that no |\includeonly| should be invoked.
This can be achieved by starting the include file
(before |\ProvidesPackage|) with:
%
\begin{center}
\begin{tabular}{l}
|\input{childdoc.def}|\\
|\childdocforward{|\textit{main}|}|\\
\end{tabular}
\end{center}
%
or alternatively with:
%
\begin{center}
\begin{tabular}{l}
|\input{childdoc.def}|\\
|\childdocby{|\textit{main}|}|\\
\end{tabular}
\end{center}
%
Both forms have slightly different effects as described above.
The main file is prepared as usual, see \secref{sec:include}.

%%%%%%%%%%%%%%%%%%%%%%%%%%%%%%%%%%%%%%%%%%%%%%%%%%%%%%%%%%%%%%%%%%%%%%%%%%%%%%%%
\subsection{Legacy Detection}
\label{sec:detection}

The directive |\childdocmain| in the main file can detect
whether the complete document or merely a child is to be compiled
even without using the directive |\childdocof|.
This method is deprecated because it is less robust
and there is no compelling reason to use it;
it is merely provided for backward compatibility
and it may be removed in future versions.

If the detection mechanism is to be used,
it is mandatory to correctly specify
the filename of the main file as the argument of |\childdocmain|:
%
\begin{center}
\begin{tabular}{l}
|\input{childdoc.def}|\\
|\childdocmain{|\textit{main}|}|\\
\end{tabular}
\end{center}
%
If |\jobname| does not match the argument \textit{main} of |\childdocmain|,
it is assumed that |\jobname| points to the child file to be compiled.
When using |\childdocmain| with the main file specified as argument,
it suffices to start a child file
with just |\input{|\textit{main}|}|
without loading of the package and using |\childdocof|.
If instead all processing is done
with the appropriate \textsf{childdoc} directives,
the argument of \textit{main} of |\childdocmain| can be empty.

An alternative version of the command line processing described
in \secref{sec:commandline} using the detection mechanism reads:
%
\begin{center}
|... -jobname "|\textit{target}|" "|[\textit{flags}]%
[|\def\jobname{|\textit{dest}|}|]|\input{|\textit{main}|}"|
\end{center}

%%%%%%%%%%%%%%%%%%%%%%%%%%%%%%%%%%%%%%%%%%%%%%%%%%%%%%%%%%%%%%%%%%%%%%%%%%%%%%%%
\subsection{Manual Code}
\label{sec:manual}

In case one cannot be certain whether the definitions file |childdoc.def|
is installed on the target \TeX{} distribution
and one prefers not to ship it,
it is conceivable to paste a few relevant commands into the sources.

To that end, drop all statements |\input{childdoc.def}|
and perform the replacements as outlined below.
Instead of |\childdocmain{|\textit{main}|}| add the following code
to the top of the main file:
%
\begin{center}
\begin{tabular}{l}
|\||ifdefined\childdocname\endinput\||fi\newif\ifchilddoc|\\
|\edef\childdocname{\scantokens\expandafter{\jobname\noexpand}}|\\
|\def\childdocmain{|\textit{main}|}\||ifx\childdocmain\childdocname\||else|\\
|\childdoctrue\includeonly{\childdocname}\let\jobname\childdocmain\||fi|\\
\end{tabular}
\end{center}
%
Instead of |\childdocof{|\textit{main}|}| just include the main file
at the top of each child file:
%
\begin{center}
|\input{|\textit{main}|}|
\end{center}
%
A simple redirection |\childdocforward{|\textit{dest}|}| is achieved by:
%
\begin{center}
|\def\jobname{|\textit{dest}|}\input{\jobname}|
\end{center}
%
The redirection with prefix
|\childdocforwardprefix[|\textit{prefix}|]{|\textit{dest}|}|
is accomplished by:
%
\begin{center}
\begin{tabular}{l}
|{\edef\jobname{\scantokens\expandafter{\jobname\noexpand}}|\\
|\def\redirectjob |\textit{prefix}|#1~~~{\gdef\jobname{|\textit{dest}|#1}}|\\
|\expandafter\redirectjob\jobname~~~}\input{\jobname}|
\end{tabular}
\end{center}

In an alternative approach,
child documents can be compiled by a specific command line
without additional code or specific definitions:
%
\begin{center}
|... -jobname "|\textit{target}|" "|[\textit{flags}]%
|\includeonly{|\textit{dest}|}\input{|\textit{main}|}"|
\end{center}
%

%%%%%%%%%%%%%%%%%%%%%%%%%%%%%%%%%%%%%%%%%%%%%%%%%%%%%%%%%%%%%%%%%%%%%%%%%%%%%%%%
%%%%%%%%%%%%%%%%%%%%%%%%%%%%%%%%%%%%%%%%%%%%%%%%%%%%%%%%%%%%%%%%%%%%%%%%%%%%%%%%
\section{Information}

%%%%%%%%%%%%%%%%%%%%%%%%%%%%%%%%%%%%%%%%%%%%%%%%%%%%%%%%%%%%%%%%%%%%%%%%%%%%%%%%
\subsection{Copyright}

Copyright \copyright{} 2017--2018 Niklas Beisert

This work may be distributed and/or modified under the
conditions of the \LaTeX{} Project Public License, either version 1.3
of this license or (at your option) any later version.
The latest version of this license is in
  \url{http://www.latex-project.org/lppl.txt}
and version 1.3 or later is part of all distributions of \LaTeX{}
version 2005/12/01 or later.

This work has the LPPL maintenance status `maintained'.

The Current Maintainer of this work is Niklas Beisert.

This work consists of the files |README.txt|, |childdoc.ins| and |childdoc.dtx|
as well as the derived files |childdoc.def|, |cdocsamp.tex|
with |cdocsch1.tex|, |cdocsch2.tex|, |cdocspt3.tex|, |cdocspt4.tex|,
|cdocsdrf.tex|, |cdocsfn1.tex|, |cdocsfn2.tex|
as well as |childdoc.pdf|.

%%%%%%%%%%%%%%%%%%%%%%%%%%%%%%%%%%%%%%%%%%%%%%%%%%%%%%%%%%%%%%%%%%%%%%%%%%%%%%%%
\subsection{Files and Installation}

The package consists of the files:
%
\begin{center}
\begin{tabular}{ll}
    |README.txt|   & readme file \\
    |childdoc.ins| & installation file \\
    |childdoc.dtx| & source file \\
    |childdoc.def| & definition file \\
    |cdocsamp.tex| & sample main file \\
    |cdocsch1.tex| & sample include file \\
    |cdocsch2.tex| & sample include file \\
    |cdocspt3.tex| & sample part file \\
    |cdocspt4.tex| & sample part file \\
    |cdocsdrf.tex| & sample redirection file \\
    |cdocsfn1.tex| & sample redirection file \\
    |cdocsfn2.tex| & sample redirection file \\
    |childdoc.pdf| & manual
\end{tabular}
\end{center}
%
The distribution consists of the files
|README.txt|, |childdoc.ins| and |childdoc.dtx|.
%
\begin{itemize}
\item
Run (pdf)\LaTeX{} on |childdoc.dtx|
to compile the manual |childdoc.pdf| (this file).
\item
Run \LaTeX{} on |childdoc.ins| to create the definitions file |childdoc.def|
and the sample |cdocsamp.tex| with include files
|cdocsch1.tex|, |cdocsch2.tex|, |cdocspt3.tex|, |cdocspt4.tex|,
|cdocsdrf.tex|, |cdocsfn1.tex|, |cdocsfn2.tex|.
Then copy the file |childdoc.def| to an appropriate directory of your \LaTeX{}
distribution, e.g.\ \textit{texmf-root}|/tex/latex/childdoc|.
\end{itemize}

%%%%%%%%%%%%%%%%%%%%%%%%%%%%%%%%%%%%%%%%%%%%%%%%%%%%%%%%%%%%%%%%%%%%%%%%%%%%%%%%
\subsection{Related CTAN Packages}

There are several other packages which offer a similar functionality:
%
\begin{itemize}
\item
The packages
\href{http://ctan.org/pkg/docmute}{\textsf{docmute}},
\href{http://ctan.org/pkg/includex}{\textsf{includex}} and
\href{http://ctan.org/pkg/standalone}{\textsf{standalone}}
provide commands to include only the document body of
a child file thus allowing both files to be compiled individually.
\item
The packages \href{http://ctan.org/pkg/subdocs}{\textsf{subdocs}}
and \href{http://ctan.org/pkg/subfiles}{\textsf{subfiles}}
provide structures in which the main and child documents can be
encapsulated and allowing them to be compiled individually.
The inclusion mechanism is different from the conventional |\include|.
\item
The package \href{http://ctan.org/pkg/combine}{\textsf{combine}}
is an elaborate solution to combine several documents into one.
\end{itemize}
%
See also the CTAN topic \href{http://ctan.org/topic/subdocs}{\textsf{subdocs}}
for further related packages.
The present package differs from the above solutions in that
a document structure constructed with the conventional |\include| mechanism
just needs two extra commands at the top of every file
such that all constituent files can be compiled individually.

%%%%%%%%%%%%%%%%%%%%%%%%%%%%%%%%%%%%%%%%%%%%%%%%%%%%%%%%%%%%%%%%%%%%%%%%%%%%%%%%
%\subsection{Feature Suggestions}
%
%The following is a list of features which may be useful for future
%versions of this package:
%%
%\begin{itemize}
%\item
%\ldots
%\end{itemize}

%%%%%%%%%%%%%%%%%%%%%%%%%%%%%%%%%%%%%%%%%%%%%%%%%%%%%%%%%%%%%%%%%%%%%%%%%%%%%%%%
\subsection{Revision History}

%%%%%%%%%%%%%%%%%%%%%%%%%%%%%%%%%%%%%%%%
\paragraph{v2.0:} 2018/12/30

\begin{itemize}
\item
immediate forward processing
\item
added |\childdocby| mechanism
\item
manual restructured
\end{itemize}

%%%%%%%%%%%%%%%%%%%%%%%%%%%%%%%%%%%%%%%%
\paragraph{v1.6:} 2018/01/17

\begin{itemize}
\item
application for development of include files
\item
corrections to manual
\end{itemize}

%%%%%%%%%%%%%%%%%%%%%%%%%%%%%%%%%%%%%%%%
\paragraph{v1.5:} 2017/05/21

\begin{itemize}
\item
more complete structuring introduced
\item
|\childdocof| introduced
\item
|\childdoc| renamed to |\childdocmain|
\item
|\childredirect| renamed to |\childdocforward| and |\childdocforwardprefix|
and functionality expanded
\end{itemize}

%%%%%%%%%%%%%%%%%%%%%%%%%%%%%%%%%%%%%%%%
\paragraph{v1.0:} 2017/04/27

\begin{itemize}
\item
manual and install package
\item
first version published on CTAN
\end{itemize}

%%%%%%%%%%%%%%%%%%%%%%%%%%%%%%%%%%%%%%%%
\paragraph{v0.6:} 2017/04/26

\begin{itemize}
\item
redirection mechanism added
\end{itemize}

%%%%%%%%%%%%%%%%%%%%%%%%%%%%%%%%%%%%%%%%
\paragraph{v0.5:} 2017/04/26

\begin{itemize}
\item
functionality in definition file
\end{itemize}


%%%%%%%%%%%%%%%%%%%%%%%%%%%%%%%%%%%%%%%%%%%%%%%%%%%%%%%%%%%%%%%%%%%%%%%%%%%%%%%%
%%%%%%%%%%%%%%%%%%%%%%%%%%%%%%%%%%%%%%%%%%%%%%%%%%%%%%%%%%%%%%%%%%%%%%%%%%%%%%%%
%%%%%%%%%%%%%%%%%%%%%%%%%%%%%%%%%%%%%%%%%%%%%%%%%%%%%%%%%%%%%%%%%%%%%%%%%%%%%%%%
\appendix

\settowidth\MacroIndent{\rmfamily\scriptsize 000\ }

 \DocInput{childdoc.dtx}

\end{document}
%</driver>
% \fi
%
% %%%%%%%%%%%%%%%%%%%%%%%%%%%%%%%%%%%%%%%%%%%%%%%%%%%%%%%%%%%%%%%%%%%%%%%%%%%%%%
% %%%%%%%%%%%%%%%%%%%%%%%%%%%%%%%%%%%%%%%%%%%%%%%%%%%%%%%%%%%%%%%%%%%%%%%%%%%%%%
% \section{Sample}
%\iffalse
%<*samplemain>
%\fi
%
% The following presents a sample document
% with two chapters, two parts, a title page,
% a compile flag as well as three forwarding files to set the flag.
% It consists of eight |.tex| files:
% \begin{center}
% \begin{tabular}{ll}
% |cdocsamp.tex|&main file\\
% |cdocsch1.tex|&include file for chapter 1\\
% |cdocsch2.tex|&include file for chapter 2\\
% |cdocspt3.tex|&include file for part 3\\
% |cdocspt4.tex|&include file for part 4\\
% |cdocsdrf.tex|&forwarding file for main file in draft mode\\
% |cdocsfi1.tex|&forwarding file for final version of chapter 1\\
% |cdocsfi2.tex|&forwarding file for final version of chapter 2\\
% \end{tabular}
% \end{center}
% Each of the eight files can be compiled directly by the \LaTeX{} compiler.
%
% %%%%%%%%%%%%%%%%%%%%%%%%%%%%%%%%%%%%%%
% \paragraph{Main File.}
%
% The main file is called |cdocsamp.tex|.
%
% Load the \textsf{childdoc} definitions and
% declare the filename for the main document:
%    \begin{macrocode}
\input{childdoc.def}
\childdocmain{}
%    \end{macrocode}

% Optional override for |\version| flag:
%    \begin{macrocode}
%%\ifchilddoc\else\providecommand{\version}{draft}\fi
%    \end{macrocode}

% Define the default values for the |\version| flag
% (|final| for the main file and |draft| for childs):
%    \begin{macrocode}
\ifchilddoc
\providecommand{\version}{draft}
\else
\providecommand{\version}{final}
\fi
%    \end{macrocode}

% Load the standard document class:
%    \begin{macrocode}
\documentclass[12pt]{article}
%    \end{macrocode}

% Start the document body:
%    \begin{macrocode}
\begin{document}
%    \end{macrocode}

% Declare a title page.
% Print title, part of document being processed and version flag:
%    \begin{macrocode}
\addtocounter{page}{-1}
\begin{center}
{\LARGE\bfseries{}childdoc example\par}
\vspace{1cm}
\ifchilddoc
\ifchilddocmanual part\else chapter\fi:
`\childdocname' of `\childdocjob'\par
\else
main document: `\childdocjob'\par
\fi
version: \version\par
\end{center}
\newpage
%    \end{macrocode}

% Manually include selected file,
% otherwise process as usual:
%    \begin{macrocode}
\ifchilddocmanual
\section*{part `\childdocname'}
\input{\childdocname}
\else
%    \end{macrocode}

% Include the two chapters:
%    \begin{macrocode}
\include{cdocsch1}
\include{cdocsch2}
%    \end{macrocode}

% Include the two parts unless only chapters should be displayed:
%    \begin{macrocode}
\ifchilddoc\else
\section{part three}
\input{cdocspt3}
\section{part four}
\input{cdocspt4}
\fi
%    \end{macrocode}

% Process as usual until here:
%    \begin{macrocode}
\fi
%    \end{macrocode}

% End of document body:
%    \begin{macrocode}
\end{document}
%    \end{macrocode}
%\iffalse
%</samplemain>
%\fi
%
% %%%%%%%%%%%%%%%%%%%%%%%%%%%%%%%%%%%%%%
% \paragraph{Chapter Include Files.}
%
% The include files are called |cdocsch1.tex| and |cdocsch2.tex|.
%
%\iffalse
%<*samplechap1|samplechap2>
%\fi

% Optional override for |\version| flag:
%    \begin{macrocode}
%%\providecommand{\version}{final}
%    \end{macrocode}

% Include the main document:
%    \begin{macrocode}
\input{childdoc.def}
\childdocof{cdocsamp}
%    \end{macrocode}

%\iffalse
%</samplechap1|samplechap2>
%\fi
%
%\iffalse
%<*samplechap1>
%\fi
% Some text for chapter 1:
%    \begin{macrocode}
\section{one}
some text in chapter one
%    \end{macrocode}

%\iffalse
%</samplechap1>
%\fi
% Some text for chapter 2:
%\iffalse
%<*samplechap2>
%\fi
%    \begin{macrocode}
\section{two}
more text in chapter two
%    \end{macrocode}

%\iffalse
%</samplechap2>
%\fi
%
% %%%%%%%%%%%%%%%%%%%%%%%%%%%%%%%%%%%%%%
% \paragraph{Part Include Files.}
%
% The include files are called |cdocspt3.tex| and |cdocspt4.tex|.
%
%\iffalse
%<*samplepart3|samplepart4>
%\fi

% Optional override for |\version| flag:
%    \begin{macrocode}
%%\providecommand{\version}{final}
%    \end{macrocode}

% Include the main document:
%    \begin{macrocode}
\input{childdoc.def}
\childdocby{cdocsamp}
%    \end{macrocode}

%\iffalse
%</samplepart3|samplepart4>
%\fi
%
%\iffalse
%<*samplepart3>
%\fi
% Some text for part 3:
%    \begin{macrocode}
some text in part three
%    \end{macrocode}

%\iffalse
%</samplepart3>
%\fi
% Some text for part 4:
%\iffalse
%<*samplepart4>
%\fi
%    \begin{macrocode}
more text in part four
%    \end{macrocode}

%\iffalse
%</samplepart4>
%\fi
%
% %%%%%%%%%%%%%%%%%%%%%%%%%%%%%%%%%%%%%%
% \paragraph{Forwarding for a Complete Draft.}
%
% The following forwarding file |cdocsdrf.tex|
% compiles the main document in draft mode:
%\iffalse
%<*sampledraft>
%\fi
%    \begin{macrocode}
\def\version{draft}
\input{childdoc.def}
\childdocforward{cdocsamp}
%    \end{macrocode}

%\iffalse
%</sampledraft>
%\fi
%
% %%%%%%%%%%%%%%%%%%%%%%%%%%%%%%%%%%%%%%
% \paragraph{Forwarding for Final Version of the Chapters.}
%
% The following forwarding files |cdocsfn1.tex| and |cdocsfn2.tex|
% (with identical content)
% compile the final versions of the child documents
% |cdocsch1.tex| and |cdocsch2.tex|, respectively:
%\iffalse
%<*samplefinal>
%\fi
%    \begin{macrocode}
\def\version{final}
\input{childdoc.def}
\childdocforwardprefix[cdocsamp]{cdocsfn}{cdocsch}
%    \end{macrocode}

%\iffalse
%</samplefinal>
%\fi
%
% %%%%%%%%%%%%%%%%%%%%%%%%%%%%%%%%%%%%%%
% \paragraph{Command Line Processing.}
%
% The following three command lines generate the output files
% |cdocscld|, |cdocscl1| and |cdocscl2|
% which should be identical to
% |cdocsdrf|, |cdocsch1| and |cdocsfn2|, respectively:
% \begin{center}
% \begin{tabular}{l}
% |latex -jobname cdocscld \|\\
% |  "\def\version{draft}\input{childdoc.def}\childdocforward{cdocsamp}"|\\
% |latex -jobname cdocscl1 \|\\
% |  "\input{childdoc.def}\childdocforward[cdocsamp]{cdocsch1}"|\\
% |latex -jobname cdocscl2 \|\\
% |  "\def\version{final}\input{childdoc.def}\childdocforward{cdocsch2}"|
% \end{tabular}
% \end{center}
% Note that the trailing backslash on each first line
% merely continues the input to the second line
% (for convenient cut ant paste).
% Furthermore, the command |latex| can be replaced by any
% of its alternative versions such as |pdflatex|.
%
% %%%%%%%%%%%%%%%%%%%%%%%%%%%%%%%%%%%%%%%%%%%%%%%%%%%%%%%%%%%%%%%%%%%%%%%%%%%%%%
% %%%%%%%%%%%%%%%%%%%%%%%%%%%%%%%%%%%%%%%%%%%%%%%%%%%%%%%%%%%%%%%%%%%%%%%%%%%%%%
% \section{Implementation}
%\iffalse
%<*package>
%\fi
%
% This section describes the definitions file |childdoc.def|.

% The definitions cannot be loaded using |\usepackage| or |\RequirePackage|
% which has a mechanism to prevent loading a style file more than once.
% When loading the definitions by means of |\input|
% multiple instances have to be prevented manually:
%\iffalse
%This code needs to be before the `\ProvidesFile' directive
%which is defined at the beginning of this file.
%Therefore it is also placed there and commented out here.
%</package>
%<*discard>
%\fi
%    \begin{macrocode}
\ifdefined\childdocmain\endinput\fi
%    \end{macrocode}
%\iffalse
%</discard>
%<*package>
%\fi
%
% \macro{\ifchilddoc}
% \macro{\ifchilddocmanual}
% The conditional |\ifchilddoc| tells whether a
% child (true) or main (false) document is being compiled.
% The conditional |\ifchilddocmanual| tells whether
% the |\includeonly| mechanism is used (false) or
% the selection of child files must be performed manually (true).
% The definitions initialise to false:
%    \begin{macrocode}
\newif\ifchilddoc
\newif\ifchilddocmanual
%    \end{macrocode}

% \macro{\childdocname}
% \macro{\childdocjob}
% The macro |\childdocname| stores the name of the main document
% to be compiled. The macro |\childdocjob| stores the name of
% the document on which the \LaTeX{} compiler was originally invoked.
% The content of |\jobname| cannot be compared
% to filenames specified in the source due to different catcodes.
% The following code rescans |\jobname|, stores the result
% in |\childdocname| and saves a copy in |\childdocjob|:
%    \begin{macrocode}
\edef\childdocname{\scantokens\expandafter{\jobname\noexpand}}
\let\childdocjob\childdocname
%    \end{macrocode}

% \macro{\childdocdisable}
% The macro |\childdocdisable| prevents the main file
% from being processed more than once.
% At this stage, the main document command |\childdocmain|
% is assumed to be called once again where it should do nothing.
% Any subsequent call to it should prevent
% a secondary processing of the main document
% It overwrites the forwarding commands
% |\childdocof| and |\childdocforward|
% with empty macros to prevent further inclusions of the main document:
%    \begin{macrocode}
\newcommand{\childdocdisable}
{
  \renewcommand{\childdocmain}[1]{\renewcommand{\childdocmain}[1]{\endinput}}
  \renewcommand{\childdocof}[1]{}
  \renewcommand{\childdocby}[2][]{}
  \renewcommand{\childdocforward}[2][]{}
  \renewcommand{\childdocdisable}{}
}
%    \end{macrocode}

% \macro{\childdocmain}
% The macro |\childdocmain| is to be called at the top of the main file
% with nothing or the main filename (without extension) as argument.
% First, it breaks loops.
% If the argument is not empty and does not match |\childdocname|
% (which is set by the first inclusion of |childdoc.def|),
% |\ifchilddoc| is set to true, |\includeonly| is applied to the child file
% and |\jobname| is set to the main file
% (for proper handling of |.aux| files):
%    \begin{macrocode}
\newcommand{\childdocmain}[1]
{
  \childdocdisable\childdocmain{}
  \if?#1?\else
    \begingroup
      \def\childdoctmp{#1}
      \ifx\childdoctmp\childdocname
        \def\childdoctmp{}
      \else
        \def\childdoctmp
        {
          \childdoctrue
          \includeonly{\childdocname}
          \def\childdocjob{#1}
          \def\jobname{#1}
        }
      \fi
      \expandafter
    \endgroup
    \childdoctmp
  \fi
}
%    \end{macrocode}

% \macro{\childdocof}
% The command |\childdocof| redirects
% compilation to the main file |#1|.
%    \begin{macrocode}
\newcommand{\childdocof}[1]
{
  \childdocdisable
  \childdoctrue
  \includeonly{\childdocname}
  \def\jobname{#1}
  \def\childdocjob{#1}
  \input{#1}
}
%    \end{macrocode}

% \macro{\childdocby}
% The command |\childdocby| ....
%    \begin{macrocode}
\newcommand{\childdocby}[2][]
{
  \childdocdisable
  \childdoctrue
  \childdocmanualtrue
  \if?#1?\else
    \def\jobname{#2}
  \fi
  \def\childdocjob{#2}
  \input{#2}
  \endinput
}
%    \end{macrocode}

% \macro{\childdocforward}
% The command |\childdocforward| redirects
% compilation to the main file or
% (if the optional argument is given) a child file.
% Parameters are set as if the main file
% or a child file starting with |\childdocof| was compiled.
% Then compilation is handed over to the main file:
%    \begin{macrocode}
\newcommand{\childdocforward}[2][]
{
  \begingroup
    \if?#1?
      \def\childdoctmp
      {
        \def\childdocname{#2}
        \def\childdocjob{#2}
        \def\jobname{#2}
        \input{#2}
        \endinput
      }
    \else
      \def\childdoctmp
      {
        \childdocdisable
        \def\childdocname{#2}
        \childdoctrue
        \includeonly{#2}
        \def\childdocjob{#1}
        \def\jobname{#1}
        \input{#1}
        \endinput
      }
    \fi
    \expandafter
  \endgroup
  \childdoctmp
}
%    \end{macrocode}

% \macro{\childdocforwardprefix}
% The command |\childdocforwardprefix| redirects
% compilation to the main or a child file by means of a pattern.
% The prefix |#1| in the current filename is replaced by |#2|
% and the suffix of the current filename is kept
% (it is assumed that the filename does not contain the substring `|~~~|'
% which is used as a delimiter).
% Compilation is handed over to the new file by |\childdocforward|:
%    \begin{macrocode}
\newcommand{\childdocforwardprefix}[3][]
{
  \begingroup
    \def\childdocextract #2##1~~~{\def\childdoctmp{\childdocforward[#1]{#3##1}}}
    \expandafter\childdocextract\childdocname~~~
    \expandafter
  \endgroup
  \childdoctmp
}
%    \end{macrocode}

% \macro{\childdoc}
% The deprecated macro |\childdoc| is a legacy version of |\childdocmain|:
%    \begin{macrocode}
\newcommand{\childdoc}{\childdocmain}
%    \end{macrocode}

% \macro{\childdocredirect}
% The deprecated macro |\childdocredirect| is a legacy version
% of |\childdocforward| and |\childdocforwardprefix|:
%    \begin{macrocode}
\newcommand{\childdocredirect}[2][]
{
  \begingroup
    \if?#1?
      \def\childdoctmp{\childdocforward{#2}}
    \else
      \def\childdoctmp{\childdocforwardprefix{#1}{#2}}
    \fi
    \expandafter
  \endgroup
  \childdoctmp
}
%    \end{macrocode}

%\iffalse
%</package>
%\fi
%
\endinput
|\\
|\childdocmain{|\textit{main}|}|\\
\end{tabular}
\end{center}
%
If |\jobname| does not match the argument \textit{main} of |\childdocmain|,
it is assumed that |\jobname| points to the child file to be compiled.
When using |\childdocmain| with the main file specified as argument,
it suffices to start a child file
with just |\input{|\textit{main}|}|
without loading of the package and using |\childdocof|.
If instead all processing is done
with the appropriate \textsf{childdoc} directives,
the argument of \textit{main} of |\childdocmain| can be empty.

An alternative version of the command line processing described
in \secref{sec:commandline} using the detection mechanism reads:
%
\begin{center}
|... -jobname "|\textit{target}|" "|[\textit{flags}]%
[|\def\jobname{|\textit{dest}|}|]|\input{|\textit{main}|}"|
\end{center}

%%%%%%%%%%%%%%%%%%%%%%%%%%%%%%%%%%%%%%%%%%%%%%%%%%%%%%%%%%%%%%%%%%%%%%%%%%%%%%%%
\subsection{Manual Code}
\label{sec:manual}

In case one cannot be certain whether the definitions file |childdoc.def|
is installed on the target \TeX{} distribution
and one prefers not to ship it,
it is conceivable to paste a few relevant commands into the sources.

To that end, drop all statements |% \iffalse
%
% childdoc.dtx Copyright (C) 2017-2018 Niklas Beisert
%
% This work may be distributed and/or modified under the
% conditions of the LaTeX Project Public License, either version 1.3
% of this license or (at your option) any later version.
% The latest version of this license is in
%   http://www.latex-project.org/lppl.txt
% and version 1.3 or later is part of all distributions of LaTeX
% version 2005/12/01 or later.
%
% This work has the LPPL maintenance status `maintained'.
%
% The Current Maintainer of this work is Niklas Beisert.
%
% This work consists of the files childdoc.dtx and childdoc.ins
% and the derived files childdoc.def and cdocsamp.tex with
% cdocsch1.tex, cdocsch2.tex, cdocsdrf.tex, cdocsfn1.tex, cdocsfn2.tex.
%
%<package>\ifdefined\childdocmain\endinput\fi
%<package>\ProvidesFile{childdoc.def}[2018/12/30 v2.0 child document driver]
%<samplemain>\ProvidesFile{cdocsamp.tex}[2018/12/30 v2.0 sample for childdoc]
%<*driver>
%\ProvidesFile{childdoc.drv}[2018/12/30 v2.0 childdoc reference manual file]
\PassOptionsToClass{10pt,a4paper}{article}
\documentclass{ltxdoc}

\usepackage[margin=35mm]{geometry}
\usepackage{hyperref}
\usepackage{hyperxmp}
\usepackage[usenames]{color}

\hypersetup{colorlinks=true}
\hypersetup{pdfstartview=FitH}
\hypersetup{pdfpagemode=UseNone}
\hypersetup{pdfsource={}}
\hypersetup{pdflang={en-UK}}
\hypersetup{pdfcopyright={Copyright 2017-2018 Niklas Beisert.
  This work may be distributed and/or modified under the
  conditions of the LaTeX Project Public License, either version 1.3
  of this license or (at your option) any later version.}}
\hypersetup{pdflicenseurl={http://www.latex-project.org/lppl.txt}}
\hypersetup{pdfcontactaddress={ETH Zurich, ITP, HIT K,
  Wolfgang-Pauli-Strasse 27}}
\hypersetup{pdfcontactpostcode={8093}}
\hypersetup{pdfcontactcity={Zurich}}
\hypersetup{pdfcontactcountry={Switzerland}}
\hypersetup{pdfcontactemail={nbeisert@itp.phys.ethz.ch}}
\hypersetup{pdfcontacturl={http://people.phys.ethz.ch/\xmptilde nbeisert/}}

\newcommand{\secref}[1]{\hyperref[#1]{section \ref*{#1}}}

\parskip1ex
\parindent0pt
\let\olditemize\itemize
\def\itemize{\olditemize\parskip0pt}

\begin{document}

\title{The \textsf{childdoc} Package}
\hypersetup{pdftitle={The childdoc Package}}
\author{Niklas Beisert\\[2ex]
  Institut f\"ur Theoretische Physik\\
  Eidgen\"ossische Technische Hochschule Z\"urich\\
  Wolfgang-Pauli-Strasse 27, 8093 Z\"urich, Switzerland\\[1ex]
  \href{mailto:nbeisert@itp.phys.ethz.ch}
  {\texttt{nbeisert@itp.phys.ethz.ch}}}
\hypersetup{pdfauthor={Niklas Beisert}}
\hypersetup{pdfsubject={Manual for the LaTeX2e Package childdoc}}
\date{30 December 2018, \textsf{v2.0}}
\maketitle

\begin{abstract}\noindent
\textsf{childdoc} is a \LaTeXe{} package
that enables the direct compilation
of document sections included by |\include|
to individual files.
\end{abstract}

\begingroup
\parskip0ex
\tableofcontents
\endgroup

%%%%%%%%%%%%%%%%%%%%%%%%%%%%%%%%%%%%%%%%%%%%%%%%%%%%%%%%%%%%%%%%%%%%%%%%%%%%%%%%
%%%%%%%%%%%%%%%%%%%%%%%%%%%%%%%%%%%%%%%%%%%%%%%%%%%%%%%%%%%%%%%%%%%%%%%%%%%%%%%%
\section{Introduction}

\LaTeX{} provides a mechanism to structure a large document (such as a book)
into a main file and several child files (containing the chapters)
using the |\include| command.
This mechanism is beneficial for documents
which span hundreds of pages in order to
make the source file(s) more manageable.
Moreover, compilation can be restricted to
selected child files by means of the |\includeonly| command.
The latter feature can be used to reduce the compilation time while editing
(this was significantly more useful in the earlier days of \LaTeX{})
or to generate a smaller document which is easier to navigate.
Another application of |\includeonly| is to generate
documents consisting of selected parts of the complete document.

However, there are a few drawbacks of the plain |\include| mechanism:
\begin{itemize}
\item
The child files cannot be compiled on their own,
they can only be compiled via the main file.
A naive editing environment
(such as a text editor with an option
to have the current file processed by \LaTeX)
may require one to switch to the main file before compiling;
attempting to compile the child file produces errors.
\item
The main file must be modified (each time)
to adjust the |\includeonly| command
to the present needs. This easily leaves the main file in a messy state.
\item
The generated document will always carry the filename
of the main document. This is inconvenient if
several child files are to be compiled and
to be kept for distribution.
\end{itemize}

The present package provides a simple interface
to make child files individually compilable by \LaTeX{}.
Compiling a child file then has the same effect as compiling
the main file with an |\includeonly| command
to select the appropriate child.
Moreover the generated document will carry the name of the child
rather than the main file.
This resolves all three above issues.

This feature is meant to make the editing of books,
thesis documents and lecture notes somewhat more convenient.
However, the package can also be used efficiently for
composing a series of documents (such as exercise sheets)
which are typically distributed individually.
It then assists the author in generating the individual documents
(potentially in different versions)
as well as a document containing the collected series.
Another application is in developing style files
or other kinds of included material
where compilation of the style file could redirect
to a sample or test file.

%%%%%%%%%%%%%%%%%%%%%%%%%%%%%%%%%%%%%%%%%%%%%%%%%%%%%%%%%%%%%%%%%%%%%%%%%%%%%%%%
%%%%%%%%%%%%%%%%%%%%%%%%%%%%%%%%%%%%%%%%%%%%%%%%%%%%%%%%%%%%%%%%%%%%%%%%%%%%%%%%
\section{Usage}

First of all, the package \textsf{childdoc} is \emph{not} a standard
\LaTeXe{} |.sty| style file! Therefore it needs to be invoked in
a non-standard way.

%%%%%%%%%%%%%%%%%%%%%%%%%%%%%%%%%%%%%%%%%%%%%%%%%%%%%%%%%%%%%%%%%%%%%%%%%%%%%%%%
\subsection{Included Files}
\label{sec:include}

%%%%%%%%%%%%%%%%%%%%%%%%%%%%%%%%%%%%%%%%
\DescribeMacro{\childdocmain}
To use the package, add the commands
\begin{center}
\begin{tabular}{l}
|\input{childdoc.def}|\\
|\childdocmain{}|\\
\end{tabular}
\end{center}
at the very top of the main \LaTeX{} file,
in particular \emph{before} the |\documentclass| statement!
The argument of |\childdocmain| should be left empty
(but it must be present).

%%%%%%%%%%%%%%%%%%%%%%%%%%%%%%%%%%%%%%%%
\DescribeMacro{\childdocof}
Furthermore, add the commands
\begin{center}
\begin{tabular}{l}
|\input{childdoc.def}|\\
|\childdocof{|\textit{main}|}|\\
\end{tabular}
\end{center}
at the top of every child file \textit{child}
which is included by |\include{|\textit{child}|}|
from within the main file
(or at least for those files to be compiled individually).
The argument \textit{main} must be the filename of the main file.

There are a couple of
considerations in setting up the main and child documents:

%%%%%%%%%%%%%%%%%%%%%%%%%%%%%%%%%%%%%%%%
\paragraph{Restrictions.}

Please note the following restrictions:
\begin{itemize}
\item
|\childdocmain| must be called with one argument \textit{main}
to ensure compatibility with earlier version of the package.
It must either be empty (|\childdocmain{}|)
or precisely match the filename of the main file in which it is specified.
See \secref{sec:detection} for further information.
\item
The filename \textit{main} must be specified without the |.tex| extension.
\item
The filename \textit{main} is case sensitive
(even in case-insensitive file systems)
due to internal string comparison.
\item
The argument \textit{main} should be fully expanded, it cannot be a macro.
\item
Subdirectories and special characters should be avoided in filenames.
\item
The command |\childdocmain{|\textit{main}|}| must be followed by a whitespace.
It should not be followed immediately by another command
or by a comment mark `|%|'.
This is because the \TeX{} parser reads the token immediately following
the argument of |\childdocmain| and puts it
at the beginning of every child section;
however, a white\-space is ignored.
\end{itemize}

%%%%%%%%%%%%%%%%%%%%%%%%%%%%%%%%%%%%%%%%
\paragraph{Content of Main File.}

It is advisable to place all content in the child files included by |\include|.
Any output contained in the main file will appear in all child documents
unless suppressed manually;
it cannot be suppressed automatically by the |\includeonly| directive
and thus should normally be avoided.
A method to include some content in the main file
by means of conditional processing is described in \secref{sec:conditional}.

%%%%%%%%%%%%%%%%%%%%%%%%%%%%%%%%%%%%%%%%
\paragraph{Page Numbering.}

When only a part of the document is compiled,
the appropriate numbering of pages
(as well as other status parameters)
is determined from the |.aux| files.
The latter contain information from previous passes.
However this information needs to propagate through
all intermediate child documents.
Therefore the page numbering in child documents may well
be inconsistent until the complete document is compiled at least once.

A useful (if unconventional) way to always ensure a consistent
page numbering is to restart the numbering in each child document
and denote the pages by `\textit{child}|.|\textit{page}'
where \textit{child} represents the chapter/section number of the child file.
This can be achieved by the command
|\numberwithin{page}{|\textit{child}|}|
of the \textsf{amsmath} package
where \textit{child} can be |chapter| or |section|
depending on the chosen structuring.
Alternatively, one can modify the macro |\thepage| appropriately
and reset the counter |page| at the start of each child file.

%%%%%%%%%%%%%%%%%%%%%%%%%%%%%%%%%%%%%%%%%%%%%%%%%%%%%%%%%%%%%%%%%%%%%%%%%%%%%%%%
\subsection{Conditional Processing}
\label{sec:conditional}

The package provides a mechanism to compile different versions
of a document. To customise the versions further some conditional processing
can come in handy to distinguish which version is being compiled.
The package provides two macros to describe the compilation context:

%%%%%%%%%%%%%%%%%%%%%%%%%%%%%%%%%%%%%%%%
\DescribeMacro{\ifchilddoc}
The conditional |\ifchilddoc| distinguishes between the compilation of
child documents and the main document:
%
\begin{center}
|\ifchilddoc |\textit{child-code}| |[|\||else |\textit{main-code}]| \||fi|
\end{center}

%%%%%%%%%%%%%%%%%%%%%%%%%%%%%%%%%%%%%%%%
\DescribeMacro{\childdocname}
\DescribeMacro{\childdocjob}
The macro |\childdocname| contains the filename (without extension)
of the main or child file being processed.
Note that |\childdocjob| will always contain the name of the main file.

%%%%%%%%%%%%%%%%%%%%%%%%%%%%%%%%%%%%%%%%
\paragraph{Title Page.}

Conditional processing can be used to include a title or banner page
in the main document when proper precautions are taken.
Importantly, the code in the main file should ensure that the page counter
(as well as other status parameters which are stored in the |.aux| files)
takes the same value after the conditional processing.
Otherwise the page numbers may take divergent values
depending on which part is compiled.

For example, a title page could be declared by:
%
\begin{center}
\begin{tabular}{l}
|\ifchilddoc\||else|\\
|\addtocounter{page}{-1}|\\
\textit{code for title page}\\
|\newpage|\\
|\||fi|
\end{tabular}
\end{center}
%
A banner page for the child documents can be generated by:
%
\begin{center}
\begin{tabular}{l}
|\ifchilddoc|\\
|\addtocounter{page}{-1}|\\
\textit{code for banner page}\\
|\newpage|\\
|\||fi|
\end{tabular}
\end{center}
%
Here one could write a message such as:
\begin{center}
|This is the part \childdocname{} of \childdocjob{}.|
\end{center}

%%%%%%%%%%%%%%%%%%%%%%%%%%%%%%%%%%%%%%%%%%%%%%%%%%%%%%%%%%%%%%%%%%%%%%%%%%%%%%%%
\subsection{Flags}
\label{sec:flags}

The package makes it easy to generate different versions
of the main or child documents.
To this end compilation flags can be defined
and assigned different default values.
They will be particularly useful in conjunction
with the forwarding mechanism described in \secref{sec:forward}.

For example, it may be useful to have a flag |\version|
which can be set to |draft| or |final|.
The document source will contain some conditional code
depending on the value of |\version|.
Suppose further, the flag should default to |final| for the main file
and to |draft| for child files
which is a natural assignment for editing the document.
This is achieved by placing the following code
in the preamble of the main document
(below the |\childdocmain| directive):
%
\begin{center}
\begin{tabular}{l}
|\ifchilddoc|\\
|\providecommand{\version}{draft}|\\
|\||else|\\
|\providecommand{\version}{final}|\\
|\||fi|
\end{tabular}
\end{center}
%
The definition by |\providecommand| makes sure
that previous definitions are not overwritten.
Further statements |\providecommand{\version}{...}|
can thus be added before the above code to override it.

For the main file, one might add a line
(between |\childdocmain| and the above block)
%
\begin{center}
|%\ifchilddoc\||else\providecommand{\version}{draft}\||fi|
\end{center}
%
which can be uncommented to produce a draft version.
Likewise one can add a line to the very top of a child file
(above the |\childdocof{|\textit{main}|}| directive)
%
\begin{center}
|%\providecommand{\version}{final}|
\end{center}
%
which can be uncommented to produce the final version of this child document.

%%%%%%%%%%%%%%%%%%%%%%%%%%%%%%%%%%%%%%%%%%%%%%%%%%%%%%%%%%%%%%%%%%%%%%%%%%%%%%%%
\subsection{Forwarding}
\label{sec:forward}

Different versions of the main or child documents
using compilation flags as described in \secref{sec:flags}
can be (permanently) stored in different files
for convenient compilation, viewing and distribution.
To this end, the package defines a command
to pass on compilation to a different file:

%%%%%%%%%%%%%%%%%%%%%%%%%%%%%%%%%%%%%%%%
\DescribeMacro{\childdocforward}
The command |\childdocforward| redirects processing to
another source file:
%
\begin{center}
\begin{tabular}{l}
|\input{childdoc.def}|\\
|\childdocforward[|\textit{main}|]{|\textit{dest}|}|\\
\end{tabular}
\end{center}
%
The argument \textit{dest} is the destination file
(without extension).
It should be the main file or one of the child files.
Note that further \textsf{childdoc} directives
such as |\childdocof| and |\childdocforward|
in the indicated file will be processed in this form.
The optional argument \textit{main}
passes on directly to the main file \textit{main}
while pretending to compile the child \textit{dest}.
This form behaves as if \textit{dest}
issues |\childdocof{|\textit{main}|}| right away,
and no further \textsf{childdoc} directives will be processed.

%%%%%%%%%%%%%%%%%%%%%%%%%%%%%%%%%%%%%%%%
\DescribeMacro{\...prefix}
In the alternative form |\childdocforwardprefix|,
%
\begin{center}
\begin{tabular}{l}
|\input{childdoc.def}|\\
|\childdocforwardprefix[|\textit{main}|]{|\textit{prefix}|}{|\textit{dest}|}|
\end{tabular}
\end{center}
%
the destination file is determined by a pattern
depending on the current file:
To make this work, the current file must be called
`{\textit{prefix}\hspace{0.2em}\textit{suffix}}'
with \textit{prefix} matching precisely the argument.
Processing is then passed on to the file
`{\textit{dest}\hspace{0.2em}\textit{suffix}}'.
Surely, the same effect is achieved by
directly specifying the
argument `{\textit{dest}\hspace{0.2em}\textit{suffix}}'
in the first form.
However, that requires to set up a different file
for each child. With the alternative form of the command
all these files can have exactly the same content
which simplifies setting them up and maintaining them.

For example, the following file |draft.tex|
with a compilation flag |\version| as described in \secref{sec:flags}
compiles the main document as a draft:
%
\begin{center}
\begin{tabular}{l}
|\def\version{draft}|\\
|\input{childdoc.def}|\\
|\childdocforward{|\textit{main}|}|
\end{tabular}
\end{center}
%
Likewise, the following files |final|\textit{nn}|.tex|
compile the final version of the child document
|child|\textit{nn}|.tex|:
%
\begin{center}
\begin{tabular}{l}
|\def\version{final}|\\
|\input{childdoc.def}|\\
|\childdocforwardprefix{final}{child}|
\end{tabular}
\end{center}
%

Note that when several versions of a main file and/or of each child file
are to be generated, it may be convenient to set up a |Makefile| or
shell script to automatise the process.

%%%%%%%%%%%%%%%%%%%%%%%%%%%%%%%%%%%%%%%%%%%%%%%%%%%%%%%%%%%%%%%%%%%%%%%%%%%%%%%%
\subsection{Command Line Processing}
\label{sec:commandline}

The effect of redirection files can also be achieved by invoking
the \LaTeX{} compiler with a more elaborate command line.
Most conveniently this should be done as part
of a shell script or a |Makefile|.

When using \textsf{childdoc} in the main file, the following
command lines effectively perform a redirection
(note that depending on the shell being used,
backslashes may have to be doubled: `|\|' $\to$ `|\\|'):
%
\begin{center}
|... -jobname "|\textit{target}|" |\\|"|[\textit{flags}]%
|\input{childdoc.def}\childdocforward[|\textit{main}|]{|\textit{dest}|}"|
\end{center}
%
Here \textit{target} is the name of the output file,
\textit{main} is the name of the main file
and \textit{dest} is the name of the main or child file to be processed
(all filenames without extensions).
The optional argument \textit{main} can be omitted
if \textit{main} matches \textit{dest}.
Optionally, compilation \textit{flags} can be defined via |\def| commands.
This command line makes the \TeX{} engine believe
it is compiling the file \textit{target}
whose content is specified as the latter parameter.
The provided code then forwards the processing to
\textit{main} or \textit{dest} as described in \secref{sec:forward}.

%%%%%%%%%%%%%%%%%%%%%%%%%%%%%%%%%%%%%%%%%%%%%%%%%%%%%%%%%%%%%%%%%%%%%%%%%%%%%%%%
\subsection{Include by Input}
\label{sec:input}

Including child documents by |\include| has some restrictions by design.
Most notably, the content of a child document always occupies
its own set of pages; pages cannot be shared between child documents.
Usually, this behaviour makes perfect sense
because each child document contain an essential part of the document.
However, in some situations it may be desirable to compose
a document from a collection of parts
without having mandatory page breaks between then.
For this case, the package
provides a mechanism to include parts
by |\input| which can also be processed individually.
However, by construction this mechanism
requires manual handling of the content to be output.

%%%%%%%%%%%%%%%%%%%%%%%%%%%%%%%%%%%%%%%%
\DescribeMacro{\ifchilddocmanual}
The main file should be prepared as usual, see \secref{sec:include}.
However, the document body must make a distinction
between processing of an individual part and of the main document, e.g.:
%
\begin{center}
\begin{tabular}{l}
|\ifchilddocmanual|\\
|\input{\childdocname}|\\
|\||else|\\
\textit{document body with }|\input{|\textit{part}|}|\\
|\||fi|
\end{tabular}
\end{center}
%
The conditional |\ifchilddocmanual| is true whenever
a part to be included by |\input| is being compiled,
and the name of the part is stored in |\childdocname|.

%%%%%%%%%%%%%%%%%%%%%%%%%%%%%%%%%%%%%%%%
\DescribeMacro{\childdocby}
Each part to be included by |\input| should start with:
%
\begin{center}
\begin{tabular}{l}
|\input{childdoc.def}|\\
|\childdocby{|\textit{main}|}|\\
\end{tabular}
\end{center}
%
The directive |\childdocby| is similar to |\childdocof|
described in \secref{sec:include},
but the subsequent selection of content must be done manually.
To that end, both |\ifchilddoc| and |\ifchilddocmanual|
will be true upon processing of a part,
and the name of the part is stored in |\childdocname|.
Note that |\jobname| will be set to the filename of the current part
so that each part receives an individual |.aux| file
that does not interfere with the |.aux| file(s) of the main document.
This behaviour can be altered by the alternative form
|\childdocby[*]{|\textit{main}|}| (with a non-empty optional argument)
which uses the |.aux| file of the main document
by setting |\jobname| to \textit{main}.

%%%%%%%%%%%%%%%%%%%%%%%%%%%%%%%%%%%%%%%%%%%%%%%%%%%%%%%%%%%%%%%%%%%%%%%%%%%%%%%%
\subsection{Driver Development}
\label{sec:driver}

The \textsf{childdoc} mechanism can also be use for the development
of definition files such as \LaTeX{} styles or classes.
This case differs from the above setup with multiple parts
included by |\include| in that no |\includeonly| should be invoked.
This can be achieved by starting the include file
(before |\ProvidesPackage|) with:
%
\begin{center}
\begin{tabular}{l}
|\input{childdoc.def}|\\
|\childdocforward{|\textit{main}|}|\\
\end{tabular}
\end{center}
%
or alternatively with:
%
\begin{center}
\begin{tabular}{l}
|\input{childdoc.def}|\\
|\childdocby{|\textit{main}|}|\\
\end{tabular}
\end{center}
%
Both forms have slightly different effects as described above.
The main file is prepared as usual, see \secref{sec:include}.

%%%%%%%%%%%%%%%%%%%%%%%%%%%%%%%%%%%%%%%%%%%%%%%%%%%%%%%%%%%%%%%%%%%%%%%%%%%%%%%%
\subsection{Legacy Detection}
\label{sec:detection}

The directive |\childdocmain| in the main file can detect
whether the complete document or merely a child is to be compiled
even without using the directive |\childdocof|.
This method is deprecated because it is less robust
and there is no compelling reason to use it;
it is merely provided for backward compatibility
and it may be removed in future versions.

If the detection mechanism is to be used,
it is mandatory to correctly specify
the filename of the main file as the argument of |\childdocmain|:
%
\begin{center}
\begin{tabular}{l}
|\input{childdoc.def}|\\
|\childdocmain{|\textit{main}|}|\\
\end{tabular}
\end{center}
%
If |\jobname| does not match the argument \textit{main} of |\childdocmain|,
it is assumed that |\jobname| points to the child file to be compiled.
When using |\childdocmain| with the main file specified as argument,
it suffices to start a child file
with just |\input{|\textit{main}|}|
without loading of the package and using |\childdocof|.
If instead all processing is done
with the appropriate \textsf{childdoc} directives,
the argument of \textit{main} of |\childdocmain| can be empty.

An alternative version of the command line processing described
in \secref{sec:commandline} using the detection mechanism reads:
%
\begin{center}
|... -jobname "|\textit{target}|" "|[\textit{flags}]%
[|\def\jobname{|\textit{dest}|}|]|\input{|\textit{main}|}"|
\end{center}

%%%%%%%%%%%%%%%%%%%%%%%%%%%%%%%%%%%%%%%%%%%%%%%%%%%%%%%%%%%%%%%%%%%%%%%%%%%%%%%%
\subsection{Manual Code}
\label{sec:manual}

In case one cannot be certain whether the definitions file |childdoc.def|
is installed on the target \TeX{} distribution
and one prefers not to ship it,
it is conceivable to paste a few relevant commands into the sources.

To that end, drop all statements |\input{childdoc.def}|
and perform the replacements as outlined below.
Instead of |\childdocmain{|\textit{main}|}| add the following code
to the top of the main file:
%
\begin{center}
\begin{tabular}{l}
|\||ifdefined\childdocname\endinput\||fi\newif\ifchilddoc|\\
|\edef\childdocname{\scantokens\expandafter{\jobname\noexpand}}|\\
|\def\childdocmain{|\textit{main}|}\||ifx\childdocmain\childdocname\||else|\\
|\childdoctrue\includeonly{\childdocname}\let\jobname\childdocmain\||fi|\\
\end{tabular}
\end{center}
%
Instead of |\childdocof{|\textit{main}|}| just include the main file
at the top of each child file:
%
\begin{center}
|\input{|\textit{main}|}|
\end{center}
%
A simple redirection |\childdocforward{|\textit{dest}|}| is achieved by:
%
\begin{center}
|\def\jobname{|\textit{dest}|}\input{\jobname}|
\end{center}
%
The redirection with prefix
|\childdocforwardprefix[|\textit{prefix}|]{|\textit{dest}|}|
is accomplished by:
%
\begin{center}
\begin{tabular}{l}
|{\edef\jobname{\scantokens\expandafter{\jobname\noexpand}}|\\
|\def\redirectjob |\textit{prefix}|#1~~~{\gdef\jobname{|\textit{dest}|#1}}|\\
|\expandafter\redirectjob\jobname~~~}\input{\jobname}|
\end{tabular}
\end{center}

In an alternative approach,
child documents can be compiled by a specific command line
without additional code or specific definitions:
%
\begin{center}
|... -jobname "|\textit{target}|" "|[\textit{flags}]%
|\includeonly{|\textit{dest}|}\input{|\textit{main}|}"|
\end{center}
%

%%%%%%%%%%%%%%%%%%%%%%%%%%%%%%%%%%%%%%%%%%%%%%%%%%%%%%%%%%%%%%%%%%%%%%%%%%%%%%%%
%%%%%%%%%%%%%%%%%%%%%%%%%%%%%%%%%%%%%%%%%%%%%%%%%%%%%%%%%%%%%%%%%%%%%%%%%%%%%%%%
\section{Information}

%%%%%%%%%%%%%%%%%%%%%%%%%%%%%%%%%%%%%%%%%%%%%%%%%%%%%%%%%%%%%%%%%%%%%%%%%%%%%%%%
\subsection{Copyright}

Copyright \copyright{} 2017--2018 Niklas Beisert

This work may be distributed and/or modified under the
conditions of the \LaTeX{} Project Public License, either version 1.3
of this license or (at your option) any later version.
The latest version of this license is in
  \url{http://www.latex-project.org/lppl.txt}
and version 1.3 or later is part of all distributions of \LaTeX{}
version 2005/12/01 or later.

This work has the LPPL maintenance status `maintained'.

The Current Maintainer of this work is Niklas Beisert.

This work consists of the files |README.txt|, |childdoc.ins| and |childdoc.dtx|
as well as the derived files |childdoc.def|, |cdocsamp.tex|
with |cdocsch1.tex|, |cdocsch2.tex|, |cdocspt3.tex|, |cdocspt4.tex|,
|cdocsdrf.tex|, |cdocsfn1.tex|, |cdocsfn2.tex|
as well as |childdoc.pdf|.

%%%%%%%%%%%%%%%%%%%%%%%%%%%%%%%%%%%%%%%%%%%%%%%%%%%%%%%%%%%%%%%%%%%%%%%%%%%%%%%%
\subsection{Files and Installation}

The package consists of the files:
%
\begin{center}
\begin{tabular}{ll}
    |README.txt|   & readme file \\
    |childdoc.ins| & installation file \\
    |childdoc.dtx| & source file \\
    |childdoc.def| & definition file \\
    |cdocsamp.tex| & sample main file \\
    |cdocsch1.tex| & sample include file \\
    |cdocsch2.tex| & sample include file \\
    |cdocspt3.tex| & sample part file \\
    |cdocspt4.tex| & sample part file \\
    |cdocsdrf.tex| & sample redirection file \\
    |cdocsfn1.tex| & sample redirection file \\
    |cdocsfn2.tex| & sample redirection file \\
    |childdoc.pdf| & manual
\end{tabular}
\end{center}
%
The distribution consists of the files
|README.txt|, |childdoc.ins| and |childdoc.dtx|.
%
\begin{itemize}
\item
Run (pdf)\LaTeX{} on |childdoc.dtx|
to compile the manual |childdoc.pdf| (this file).
\item
Run \LaTeX{} on |childdoc.ins| to create the definitions file |childdoc.def|
and the sample |cdocsamp.tex| with include files
|cdocsch1.tex|, |cdocsch2.tex|, |cdocspt3.tex|, |cdocspt4.tex|,
|cdocsdrf.tex|, |cdocsfn1.tex|, |cdocsfn2.tex|.
Then copy the file |childdoc.def| to an appropriate directory of your \LaTeX{}
distribution, e.g.\ \textit{texmf-root}|/tex/latex/childdoc|.
\end{itemize}

%%%%%%%%%%%%%%%%%%%%%%%%%%%%%%%%%%%%%%%%%%%%%%%%%%%%%%%%%%%%%%%%%%%%%%%%%%%%%%%%
\subsection{Related CTAN Packages}

There are several other packages which offer a similar functionality:
%
\begin{itemize}
\item
The packages
\href{http://ctan.org/pkg/docmute}{\textsf{docmute}},
\href{http://ctan.org/pkg/includex}{\textsf{includex}} and
\href{http://ctan.org/pkg/standalone}{\textsf{standalone}}
provide commands to include only the document body of
a child file thus allowing both files to be compiled individually.
\item
The packages \href{http://ctan.org/pkg/subdocs}{\textsf{subdocs}}
and \href{http://ctan.org/pkg/subfiles}{\textsf{subfiles}}
provide structures in which the main and child documents can be
encapsulated and allowing them to be compiled individually.
The inclusion mechanism is different from the conventional |\include|.
\item
The package \href{http://ctan.org/pkg/combine}{\textsf{combine}}
is an elaborate solution to combine several documents into one.
\end{itemize}
%
See also the CTAN topic \href{http://ctan.org/topic/subdocs}{\textsf{subdocs}}
for further related packages.
The present package differs from the above solutions in that
a document structure constructed with the conventional |\include| mechanism
just needs two extra commands at the top of every file
such that all constituent files can be compiled individually.

%%%%%%%%%%%%%%%%%%%%%%%%%%%%%%%%%%%%%%%%%%%%%%%%%%%%%%%%%%%%%%%%%%%%%%%%%%%%%%%%
%\subsection{Feature Suggestions}
%
%The following is a list of features which may be useful for future
%versions of this package:
%%
%\begin{itemize}
%\item
%\ldots
%\end{itemize}

%%%%%%%%%%%%%%%%%%%%%%%%%%%%%%%%%%%%%%%%%%%%%%%%%%%%%%%%%%%%%%%%%%%%%%%%%%%%%%%%
\subsection{Revision History}

%%%%%%%%%%%%%%%%%%%%%%%%%%%%%%%%%%%%%%%%
\paragraph{v2.0:} 2018/12/30

\begin{itemize}
\item
immediate forward processing
\item
added |\childdocby| mechanism
\item
manual restructured
\end{itemize}

%%%%%%%%%%%%%%%%%%%%%%%%%%%%%%%%%%%%%%%%
\paragraph{v1.6:} 2018/01/17

\begin{itemize}
\item
application for development of include files
\item
corrections to manual
\end{itemize}

%%%%%%%%%%%%%%%%%%%%%%%%%%%%%%%%%%%%%%%%
\paragraph{v1.5:} 2017/05/21

\begin{itemize}
\item
more complete structuring introduced
\item
|\childdocof| introduced
\item
|\childdoc| renamed to |\childdocmain|
\item
|\childredirect| renamed to |\childdocforward| and |\childdocforwardprefix|
and functionality expanded
\end{itemize}

%%%%%%%%%%%%%%%%%%%%%%%%%%%%%%%%%%%%%%%%
\paragraph{v1.0:} 2017/04/27

\begin{itemize}
\item
manual and install package
\item
first version published on CTAN
\end{itemize}

%%%%%%%%%%%%%%%%%%%%%%%%%%%%%%%%%%%%%%%%
\paragraph{v0.6:} 2017/04/26

\begin{itemize}
\item
redirection mechanism added
\end{itemize}

%%%%%%%%%%%%%%%%%%%%%%%%%%%%%%%%%%%%%%%%
\paragraph{v0.5:} 2017/04/26

\begin{itemize}
\item
functionality in definition file
\end{itemize}


%%%%%%%%%%%%%%%%%%%%%%%%%%%%%%%%%%%%%%%%%%%%%%%%%%%%%%%%%%%%%%%%%%%%%%%%%%%%%%%%
%%%%%%%%%%%%%%%%%%%%%%%%%%%%%%%%%%%%%%%%%%%%%%%%%%%%%%%%%%%%%%%%%%%%%%%%%%%%%%%%
%%%%%%%%%%%%%%%%%%%%%%%%%%%%%%%%%%%%%%%%%%%%%%%%%%%%%%%%%%%%%%%%%%%%%%%%%%%%%%%%
\appendix

\settowidth\MacroIndent{\rmfamily\scriptsize 000\ }

 \DocInput{childdoc.dtx}

\end{document}
%</driver>
% \fi
%
% %%%%%%%%%%%%%%%%%%%%%%%%%%%%%%%%%%%%%%%%%%%%%%%%%%%%%%%%%%%%%%%%%%%%%%%%%%%%%%
% %%%%%%%%%%%%%%%%%%%%%%%%%%%%%%%%%%%%%%%%%%%%%%%%%%%%%%%%%%%%%%%%%%%%%%%%%%%%%%
% \section{Sample}
%\iffalse
%<*samplemain>
%\fi
%
% The following presents a sample document
% with two chapters, two parts, a title page,
% a compile flag as well as three forwarding files to set the flag.
% It consists of eight |.tex| files:
% \begin{center}
% \begin{tabular}{ll}
% |cdocsamp.tex|&main file\\
% |cdocsch1.tex|&include file for chapter 1\\
% |cdocsch2.tex|&include file for chapter 2\\
% |cdocspt3.tex|&include file for part 3\\
% |cdocspt4.tex|&include file for part 4\\
% |cdocsdrf.tex|&forwarding file for main file in draft mode\\
% |cdocsfi1.tex|&forwarding file for final version of chapter 1\\
% |cdocsfi2.tex|&forwarding file for final version of chapter 2\\
% \end{tabular}
% \end{center}
% Each of the eight files can be compiled directly by the \LaTeX{} compiler.
%
% %%%%%%%%%%%%%%%%%%%%%%%%%%%%%%%%%%%%%%
% \paragraph{Main File.}
%
% The main file is called |cdocsamp.tex|.
%
% Load the \textsf{childdoc} definitions and
% declare the filename for the main document:
%    \begin{macrocode}
\input{childdoc.def}
\childdocmain{}
%    \end{macrocode}

% Optional override for |\version| flag:
%    \begin{macrocode}
%%\ifchilddoc\else\providecommand{\version}{draft}\fi
%    \end{macrocode}

% Define the default values for the |\version| flag
% (|final| for the main file and |draft| for childs):
%    \begin{macrocode}
\ifchilddoc
\providecommand{\version}{draft}
\else
\providecommand{\version}{final}
\fi
%    \end{macrocode}

% Load the standard document class:
%    \begin{macrocode}
\documentclass[12pt]{article}
%    \end{macrocode}

% Start the document body:
%    \begin{macrocode}
\begin{document}
%    \end{macrocode}

% Declare a title page.
% Print title, part of document being processed and version flag:
%    \begin{macrocode}
\addtocounter{page}{-1}
\begin{center}
{\LARGE\bfseries{}childdoc example\par}
\vspace{1cm}
\ifchilddoc
\ifchilddocmanual part\else chapter\fi:
`\childdocname' of `\childdocjob'\par
\else
main document: `\childdocjob'\par
\fi
version: \version\par
\end{center}
\newpage
%    \end{macrocode}

% Manually include selected file,
% otherwise process as usual:
%    \begin{macrocode}
\ifchilddocmanual
\section*{part `\childdocname'}
\input{\childdocname}
\else
%    \end{macrocode}

% Include the two chapters:
%    \begin{macrocode}
\include{cdocsch1}
\include{cdocsch2}
%    \end{macrocode}

% Include the two parts unless only chapters should be displayed:
%    \begin{macrocode}
\ifchilddoc\else
\section{part three}
\input{cdocspt3}
\section{part four}
\input{cdocspt4}
\fi
%    \end{macrocode}

% Process as usual until here:
%    \begin{macrocode}
\fi
%    \end{macrocode}

% End of document body:
%    \begin{macrocode}
\end{document}
%    \end{macrocode}
%\iffalse
%</samplemain>
%\fi
%
% %%%%%%%%%%%%%%%%%%%%%%%%%%%%%%%%%%%%%%
% \paragraph{Chapter Include Files.}
%
% The include files are called |cdocsch1.tex| and |cdocsch2.tex|.
%
%\iffalse
%<*samplechap1|samplechap2>
%\fi

% Optional override for |\version| flag:
%    \begin{macrocode}
%%\providecommand{\version}{final}
%    \end{macrocode}

% Include the main document:
%    \begin{macrocode}
\input{childdoc.def}
\childdocof{cdocsamp}
%    \end{macrocode}

%\iffalse
%</samplechap1|samplechap2>
%\fi
%
%\iffalse
%<*samplechap1>
%\fi
% Some text for chapter 1:
%    \begin{macrocode}
\section{one}
some text in chapter one
%    \end{macrocode}

%\iffalse
%</samplechap1>
%\fi
% Some text for chapter 2:
%\iffalse
%<*samplechap2>
%\fi
%    \begin{macrocode}
\section{two}
more text in chapter two
%    \end{macrocode}

%\iffalse
%</samplechap2>
%\fi
%
% %%%%%%%%%%%%%%%%%%%%%%%%%%%%%%%%%%%%%%
% \paragraph{Part Include Files.}
%
% The include files are called |cdocspt3.tex| and |cdocspt4.tex|.
%
%\iffalse
%<*samplepart3|samplepart4>
%\fi

% Optional override for |\version| flag:
%    \begin{macrocode}
%%\providecommand{\version}{final}
%    \end{macrocode}

% Include the main document:
%    \begin{macrocode}
\input{childdoc.def}
\childdocby{cdocsamp}
%    \end{macrocode}

%\iffalse
%</samplepart3|samplepart4>
%\fi
%
%\iffalse
%<*samplepart3>
%\fi
% Some text for part 3:
%    \begin{macrocode}
some text in part three
%    \end{macrocode}

%\iffalse
%</samplepart3>
%\fi
% Some text for part 4:
%\iffalse
%<*samplepart4>
%\fi
%    \begin{macrocode}
more text in part four
%    \end{macrocode}

%\iffalse
%</samplepart4>
%\fi
%
% %%%%%%%%%%%%%%%%%%%%%%%%%%%%%%%%%%%%%%
% \paragraph{Forwarding for a Complete Draft.}
%
% The following forwarding file |cdocsdrf.tex|
% compiles the main document in draft mode:
%\iffalse
%<*sampledraft>
%\fi
%    \begin{macrocode}
\def\version{draft}
\input{childdoc.def}
\childdocforward{cdocsamp}
%    \end{macrocode}

%\iffalse
%</sampledraft>
%\fi
%
% %%%%%%%%%%%%%%%%%%%%%%%%%%%%%%%%%%%%%%
% \paragraph{Forwarding for Final Version of the Chapters.}
%
% The following forwarding files |cdocsfn1.tex| and |cdocsfn2.tex|
% (with identical content)
% compile the final versions of the child documents
% |cdocsch1.tex| and |cdocsch2.tex|, respectively:
%\iffalse
%<*samplefinal>
%\fi
%    \begin{macrocode}
\def\version{final}
\input{childdoc.def}
\childdocforwardprefix[cdocsamp]{cdocsfn}{cdocsch}
%    \end{macrocode}

%\iffalse
%</samplefinal>
%\fi
%
% %%%%%%%%%%%%%%%%%%%%%%%%%%%%%%%%%%%%%%
% \paragraph{Command Line Processing.}
%
% The following three command lines generate the output files
% |cdocscld|, |cdocscl1| and |cdocscl2|
% which should be identical to
% |cdocsdrf|, |cdocsch1| and |cdocsfn2|, respectively:
% \begin{center}
% \begin{tabular}{l}
% |latex -jobname cdocscld \|\\
% |  "\def\version{draft}\input{childdoc.def}\childdocforward{cdocsamp}"|\\
% |latex -jobname cdocscl1 \|\\
% |  "\input{childdoc.def}\childdocforward[cdocsamp]{cdocsch1}"|\\
% |latex -jobname cdocscl2 \|\\
% |  "\def\version{final}\input{childdoc.def}\childdocforward{cdocsch2}"|
% \end{tabular}
% \end{center}
% Note that the trailing backslash on each first line
% merely continues the input to the second line
% (for convenient cut ant paste).
% Furthermore, the command |latex| can be replaced by any
% of its alternative versions such as |pdflatex|.
%
% %%%%%%%%%%%%%%%%%%%%%%%%%%%%%%%%%%%%%%%%%%%%%%%%%%%%%%%%%%%%%%%%%%%%%%%%%%%%%%
% %%%%%%%%%%%%%%%%%%%%%%%%%%%%%%%%%%%%%%%%%%%%%%%%%%%%%%%%%%%%%%%%%%%%%%%%%%%%%%
% \section{Implementation}
%\iffalse
%<*package>
%\fi
%
% This section describes the definitions file |childdoc.def|.

% The definitions cannot be loaded using |\usepackage| or |\RequirePackage|
% which has a mechanism to prevent loading a style file more than once.
% When loading the definitions by means of |\input|
% multiple instances have to be prevented manually:
%\iffalse
%This code needs to be before the `\ProvidesFile' directive
%which is defined at the beginning of this file.
%Therefore it is also placed there and commented out here.
%</package>
%<*discard>
%\fi
%    \begin{macrocode}
\ifdefined\childdocmain\endinput\fi
%    \end{macrocode}
%\iffalse
%</discard>
%<*package>
%\fi
%
% \macro{\ifchilddoc}
% \macro{\ifchilddocmanual}
% The conditional |\ifchilddoc| tells whether a
% child (true) or main (false) document is being compiled.
% The conditional |\ifchilddocmanual| tells whether
% the |\includeonly| mechanism is used (false) or
% the selection of child files must be performed manually (true).
% The definitions initialise to false:
%    \begin{macrocode}
\newif\ifchilddoc
\newif\ifchilddocmanual
%    \end{macrocode}

% \macro{\childdocname}
% \macro{\childdocjob}
% The macro |\childdocname| stores the name of the main document
% to be compiled. The macro |\childdocjob| stores the name of
% the document on which the \LaTeX{} compiler was originally invoked.
% The content of |\jobname| cannot be compared
% to filenames specified in the source due to different catcodes.
% The following code rescans |\jobname|, stores the result
% in |\childdocname| and saves a copy in |\childdocjob|:
%    \begin{macrocode}
\edef\childdocname{\scantokens\expandafter{\jobname\noexpand}}
\let\childdocjob\childdocname
%    \end{macrocode}

% \macro{\childdocdisable}
% The macro |\childdocdisable| prevents the main file
% from being processed more than once.
% At this stage, the main document command |\childdocmain|
% is assumed to be called once again where it should do nothing.
% Any subsequent call to it should prevent
% a secondary processing of the main document
% It overwrites the forwarding commands
% |\childdocof| and |\childdocforward|
% with empty macros to prevent further inclusions of the main document:
%    \begin{macrocode}
\newcommand{\childdocdisable}
{
  \renewcommand{\childdocmain}[1]{\renewcommand{\childdocmain}[1]{\endinput}}
  \renewcommand{\childdocof}[1]{}
  \renewcommand{\childdocby}[2][]{}
  \renewcommand{\childdocforward}[2][]{}
  \renewcommand{\childdocdisable}{}
}
%    \end{macrocode}

% \macro{\childdocmain}
% The macro |\childdocmain| is to be called at the top of the main file
% with nothing or the main filename (without extension) as argument.
% First, it breaks loops.
% If the argument is not empty and does not match |\childdocname|
% (which is set by the first inclusion of |childdoc.def|),
% |\ifchilddoc| is set to true, |\includeonly| is applied to the child file
% and |\jobname| is set to the main file
% (for proper handling of |.aux| files):
%    \begin{macrocode}
\newcommand{\childdocmain}[1]
{
  \childdocdisable\childdocmain{}
  \if?#1?\else
    \begingroup
      \def\childdoctmp{#1}
      \ifx\childdoctmp\childdocname
        \def\childdoctmp{}
      \else
        \def\childdoctmp
        {
          \childdoctrue
          \includeonly{\childdocname}
          \def\childdocjob{#1}
          \def\jobname{#1}
        }
      \fi
      \expandafter
    \endgroup
    \childdoctmp
  \fi
}
%    \end{macrocode}

% \macro{\childdocof}
% The command |\childdocof| redirects
% compilation to the main file |#1|.
%    \begin{macrocode}
\newcommand{\childdocof}[1]
{
  \childdocdisable
  \childdoctrue
  \includeonly{\childdocname}
  \def\jobname{#1}
  \def\childdocjob{#1}
  \input{#1}
}
%    \end{macrocode}

% \macro{\childdocby}
% The command |\childdocby| ....
%    \begin{macrocode}
\newcommand{\childdocby}[2][]
{
  \childdocdisable
  \childdoctrue
  \childdocmanualtrue
  \if?#1?\else
    \def\jobname{#2}
  \fi
  \def\childdocjob{#2}
  \input{#2}
  \endinput
}
%    \end{macrocode}

% \macro{\childdocforward}
% The command |\childdocforward| redirects
% compilation to the main file or
% (if the optional argument is given) a child file.
% Parameters are set as if the main file
% or a child file starting with |\childdocof| was compiled.
% Then compilation is handed over to the main file:
%    \begin{macrocode}
\newcommand{\childdocforward}[2][]
{
  \begingroup
    \if?#1?
      \def\childdoctmp
      {
        \def\childdocname{#2}
        \def\childdocjob{#2}
        \def\jobname{#2}
        \input{#2}
        \endinput
      }
    \else
      \def\childdoctmp
      {
        \childdocdisable
        \def\childdocname{#2}
        \childdoctrue
        \includeonly{#2}
        \def\childdocjob{#1}
        \def\jobname{#1}
        \input{#1}
        \endinput
      }
    \fi
    \expandafter
  \endgroup
  \childdoctmp
}
%    \end{macrocode}

% \macro{\childdocforwardprefix}
% The command |\childdocforwardprefix| redirects
% compilation to the main or a child file by means of a pattern.
% The prefix |#1| in the current filename is replaced by |#2|
% and the suffix of the current filename is kept
% (it is assumed that the filename does not contain the substring `|~~~|'
% which is used as a delimiter).
% Compilation is handed over to the new file by |\childdocforward|:
%    \begin{macrocode}
\newcommand{\childdocforwardprefix}[3][]
{
  \begingroup
    \def\childdocextract #2##1~~~{\def\childdoctmp{\childdocforward[#1]{#3##1}}}
    \expandafter\childdocextract\childdocname~~~
    \expandafter
  \endgroup
  \childdoctmp
}
%    \end{macrocode}

% \macro{\childdoc}
% The deprecated macro |\childdoc| is a legacy version of |\childdocmain|:
%    \begin{macrocode}
\newcommand{\childdoc}{\childdocmain}
%    \end{macrocode}

% \macro{\childdocredirect}
% The deprecated macro |\childdocredirect| is a legacy version
% of |\childdocforward| and |\childdocforwardprefix|:
%    \begin{macrocode}
\newcommand{\childdocredirect}[2][]
{
  \begingroup
    \if?#1?
      \def\childdoctmp{\childdocforward{#2}}
    \else
      \def\childdoctmp{\childdocforwardprefix{#1}{#2}}
    \fi
    \expandafter
  \endgroup
  \childdoctmp
}
%    \end{macrocode}

%\iffalse
%</package>
%\fi
%
\endinput
|
and perform the replacements as outlined below.
Instead of |\childdocmain{|\textit{main}|}| add the following code
to the top of the main file:
%
\begin{center}
\begin{tabular}{l}
|\||ifdefined\childdocname\endinput\||fi\newif\ifchilddoc|\\
|\edef\childdocname{\scantokens\expandafter{\jobname\noexpand}}|\\
|\def\childdocmain{|\textit{main}|}\||ifx\childdocmain\childdocname\||else|\\
|\childdoctrue\includeonly{\childdocname}\let\jobname\childdocmain\||fi|\\
\end{tabular}
\end{center}
%
Instead of |\childdocof{|\textit{main}|}| just include the main file
at the top of each child file:
%
\begin{center}
|\input{|\textit{main}|}|
\end{center}
%
A simple redirection |\childdocforward{|\textit{dest}|}| is achieved by:
%
\begin{center}
|\def\jobname{|\textit{dest}|}\input{\jobname}|
\end{center}
%
The redirection with prefix
|\childdocforwardprefix[|\textit{prefix}|]{|\textit{dest}|}|
is accomplished by:
%
\begin{center}
\begin{tabular}{l}
|{\edef\jobname{\scantokens\expandafter{\jobname\noexpand}}|\\
|\def\redirectjob |\textit{prefix}|#1~~~{\gdef\jobname{|\textit{dest}|#1}}|\\
|\expandafter\redirectjob\jobname~~~}\input{\jobname}|
\end{tabular}
\end{center}

In an alternative approach,
child documents can be compiled by a specific command line
without additional code or specific definitions:
%
\begin{center}
|... -jobname "|\textit{target}|" "|[\textit{flags}]%
|\includeonly{|\textit{dest}|}\input{|\textit{main}|}"|
\end{center}
%

%%%%%%%%%%%%%%%%%%%%%%%%%%%%%%%%%%%%%%%%%%%%%%%%%%%%%%%%%%%%%%%%%%%%%%%%%%%%%%%%
%%%%%%%%%%%%%%%%%%%%%%%%%%%%%%%%%%%%%%%%%%%%%%%%%%%%%%%%%%%%%%%%%%%%%%%%%%%%%%%%
\section{Information}

%%%%%%%%%%%%%%%%%%%%%%%%%%%%%%%%%%%%%%%%%%%%%%%%%%%%%%%%%%%%%%%%%%%%%%%%%%%%%%%%
\subsection{Copyright}

Copyright \copyright{} 2017--2018 Niklas Beisert

This work may be distributed and/or modified under the
conditions of the \LaTeX{} Project Public License, either version 1.3
of this license or (at your option) any later version.
The latest version of this license is in
  \url{http://www.latex-project.org/lppl.txt}
and version 1.3 or later is part of all distributions of \LaTeX{}
version 2005/12/01 or later.

This work has the LPPL maintenance status `maintained'.

The Current Maintainer of this work is Niklas Beisert.

This work consists of the files |README.txt|, |childdoc.ins| and |childdoc.dtx|
as well as the derived files |childdoc.def|, |cdocsamp.tex|
with |cdocsch1.tex|, |cdocsch2.tex|, |cdocspt3.tex|, |cdocspt4.tex|,
|cdocsdrf.tex|, |cdocsfn1.tex|, |cdocsfn2.tex|
as well as |childdoc.pdf|.

%%%%%%%%%%%%%%%%%%%%%%%%%%%%%%%%%%%%%%%%%%%%%%%%%%%%%%%%%%%%%%%%%%%%%%%%%%%%%%%%
\subsection{Files and Installation}

The package consists of the files:
%
\begin{center}
\begin{tabular}{ll}
    |README.txt|   & readme file \\
    |childdoc.ins| & installation file \\
    |childdoc.dtx| & source file \\
    |childdoc.def| & definition file \\
    |cdocsamp.tex| & sample main file \\
    |cdocsch1.tex| & sample include file \\
    |cdocsch2.tex| & sample include file \\
    |cdocspt3.tex| & sample part file \\
    |cdocspt4.tex| & sample part file \\
    |cdocsdrf.tex| & sample redirection file \\
    |cdocsfn1.tex| & sample redirection file \\
    |cdocsfn2.tex| & sample redirection file \\
    |childdoc.pdf| & manual
\end{tabular}
\end{center}
%
The distribution consists of the files
|README.txt|, |childdoc.ins| and |childdoc.dtx|.
%
\begin{itemize}
\item
Run (pdf)\LaTeX{} on |childdoc.dtx|
to compile the manual |childdoc.pdf| (this file).
\item
Run \LaTeX{} on |childdoc.ins| to create the definitions file |childdoc.def|
and the sample |cdocsamp.tex| with include files
|cdocsch1.tex|, |cdocsch2.tex|, |cdocspt3.tex|, |cdocspt4.tex|,
|cdocsdrf.tex|, |cdocsfn1.tex|, |cdocsfn2.tex|.
Then copy the file |childdoc.def| to an appropriate directory of your \LaTeX{}
distribution, e.g.\ \textit{texmf-root}|/tex/latex/childdoc|.
\end{itemize}

%%%%%%%%%%%%%%%%%%%%%%%%%%%%%%%%%%%%%%%%%%%%%%%%%%%%%%%%%%%%%%%%%%%%%%%%%%%%%%%%
\subsection{Related CTAN Packages}

There are several other packages which offer a similar functionality:
%
\begin{itemize}
\item
The packages
\href{http://ctan.org/pkg/docmute}{\textsf{docmute}},
\href{http://ctan.org/pkg/includex}{\textsf{includex}} and
\href{http://ctan.org/pkg/standalone}{\textsf{standalone}}
provide commands to include only the document body of
a child file thus allowing both files to be compiled individually.
\item
The packages \href{http://ctan.org/pkg/subdocs}{\textsf{subdocs}}
and \href{http://ctan.org/pkg/subfiles}{\textsf{subfiles}}
provide structures in which the main and child documents can be
encapsulated and allowing them to be compiled individually.
The inclusion mechanism is different from the conventional |\include|.
\item
The package \href{http://ctan.org/pkg/combine}{\textsf{combine}}
is an elaborate solution to combine several documents into one.
\end{itemize}
%
See also the CTAN topic \href{http://ctan.org/topic/subdocs}{\textsf{subdocs}}
for further related packages.
The present package differs from the above solutions in that
a document structure constructed with the conventional |\include| mechanism
just needs two extra commands at the top of every file
such that all constituent files can be compiled individually.

%%%%%%%%%%%%%%%%%%%%%%%%%%%%%%%%%%%%%%%%%%%%%%%%%%%%%%%%%%%%%%%%%%%%%%%%%%%%%%%%
%\subsection{Feature Suggestions}
%
%The following is a list of features which may be useful for future
%versions of this package:
%%
%\begin{itemize}
%\item
%\ldots
%\end{itemize}

%%%%%%%%%%%%%%%%%%%%%%%%%%%%%%%%%%%%%%%%%%%%%%%%%%%%%%%%%%%%%%%%%%%%%%%%%%%%%%%%
\subsection{Revision History}

%%%%%%%%%%%%%%%%%%%%%%%%%%%%%%%%%%%%%%%%
\paragraph{v2.0:} 2018/12/30

\begin{itemize}
\item
immediate forward processing
\item
added |\childdocby| mechanism
\item
manual restructured
\end{itemize}

%%%%%%%%%%%%%%%%%%%%%%%%%%%%%%%%%%%%%%%%
\paragraph{v1.6:} 2018/01/17

\begin{itemize}
\item
application for development of include files
\item
corrections to manual
\end{itemize}

%%%%%%%%%%%%%%%%%%%%%%%%%%%%%%%%%%%%%%%%
\paragraph{v1.5:} 2017/05/21

\begin{itemize}
\item
more complete structuring introduced
\item
|\childdocof| introduced
\item
|\childdoc| renamed to |\childdocmain|
\item
|\childredirect| renamed to |\childdocforward| and |\childdocforwardprefix|
and functionality expanded
\end{itemize}

%%%%%%%%%%%%%%%%%%%%%%%%%%%%%%%%%%%%%%%%
\paragraph{v1.0:} 2017/04/27

\begin{itemize}
\item
manual and install package
\item
first version published on CTAN
\end{itemize}

%%%%%%%%%%%%%%%%%%%%%%%%%%%%%%%%%%%%%%%%
\paragraph{v0.6:} 2017/04/26

\begin{itemize}
\item
redirection mechanism added
\end{itemize}

%%%%%%%%%%%%%%%%%%%%%%%%%%%%%%%%%%%%%%%%
\paragraph{v0.5:} 2017/04/26

\begin{itemize}
\item
functionality in definition file
\end{itemize}


%%%%%%%%%%%%%%%%%%%%%%%%%%%%%%%%%%%%%%%%%%%%%%%%%%%%%%%%%%%%%%%%%%%%%%%%%%%%%%%%
%%%%%%%%%%%%%%%%%%%%%%%%%%%%%%%%%%%%%%%%%%%%%%%%%%%%%%%%%%%%%%%%%%%%%%%%%%%%%%%%
%%%%%%%%%%%%%%%%%%%%%%%%%%%%%%%%%%%%%%%%%%%%%%%%%%%%%%%%%%%%%%%%%%%%%%%%%%%%%%%%
\appendix

\settowidth\MacroIndent{\rmfamily\scriptsize 000\ }

 \DocInput{childdoc.dtx}

\end{document}
%</driver>
% \fi
%
% %%%%%%%%%%%%%%%%%%%%%%%%%%%%%%%%%%%%%%%%%%%%%%%%%%%%%%%%%%%%%%%%%%%%%%%%%%%%%%
% %%%%%%%%%%%%%%%%%%%%%%%%%%%%%%%%%%%%%%%%%%%%%%%%%%%%%%%%%%%%%%%%%%%%%%%%%%%%%%
% \section{Sample}
%\iffalse
%<*samplemain>
%\fi
%
% The following presents a sample document
% with two chapters, two parts, a title page,
% a compile flag as well as three forwarding files to set the flag.
% It consists of eight |.tex| files:
% \begin{center}
% \begin{tabular}{ll}
% |cdocsamp.tex|&main file\\
% |cdocsch1.tex|&include file for chapter 1\\
% |cdocsch2.tex|&include file for chapter 2\\
% |cdocspt3.tex|&include file for part 3\\
% |cdocspt4.tex|&include file for part 4\\
% |cdocsdrf.tex|&forwarding file for main file in draft mode\\
% |cdocsfi1.tex|&forwarding file for final version of chapter 1\\
% |cdocsfi2.tex|&forwarding file for final version of chapter 2\\
% \end{tabular}
% \end{center}
% Each of the eight files can be compiled directly by the \LaTeX{} compiler.
%
% %%%%%%%%%%%%%%%%%%%%%%%%%%%%%%%%%%%%%%
% \paragraph{Main File.}
%
% The main file is called |cdocsamp.tex|.
%
% Load the \textsf{childdoc} definitions and
% declare the filename for the main document:
%    \begin{macrocode}
% \iffalse
%
% childdoc.dtx Copyright (C) 2017-2018 Niklas Beisert
%
% This work may be distributed and/or modified under the
% conditions of the LaTeX Project Public License, either version 1.3
% of this license or (at your option) any later version.
% The latest version of this license is in
%   http://www.latex-project.org/lppl.txt
% and version 1.3 or later is part of all distributions of LaTeX
% version 2005/12/01 or later.
%
% This work has the LPPL maintenance status `maintained'.
%
% The Current Maintainer of this work is Niklas Beisert.
%
% This work consists of the files childdoc.dtx and childdoc.ins
% and the derived files childdoc.def and cdocsamp.tex with
% cdocsch1.tex, cdocsch2.tex, cdocsdrf.tex, cdocsfn1.tex, cdocsfn2.tex.
%
%<package>\ifdefined\childdocmain\endinput\fi
%<package>\ProvidesFile{childdoc.def}[2018/12/30 v2.0 child document driver]
%<samplemain>\ProvidesFile{cdocsamp.tex}[2018/12/30 v2.0 sample for childdoc]
%<*driver>
%\ProvidesFile{childdoc.drv}[2018/12/30 v2.0 childdoc reference manual file]
\PassOptionsToClass{10pt,a4paper}{article}
\documentclass{ltxdoc}

\usepackage[margin=35mm]{geometry}
\usepackage{hyperref}
\usepackage{hyperxmp}
\usepackage[usenames]{color}

\hypersetup{colorlinks=true}
\hypersetup{pdfstartview=FitH}
\hypersetup{pdfpagemode=UseNone}
\hypersetup{pdfsource={}}
\hypersetup{pdflang={en-UK}}
\hypersetup{pdfcopyright={Copyright 2017-2018 Niklas Beisert.
  This work may be distributed and/or modified under the
  conditions of the LaTeX Project Public License, either version 1.3
  of this license or (at your option) any later version.}}
\hypersetup{pdflicenseurl={http://www.latex-project.org/lppl.txt}}
\hypersetup{pdfcontactaddress={ETH Zurich, ITP, HIT K,
  Wolfgang-Pauli-Strasse 27}}
\hypersetup{pdfcontactpostcode={8093}}
\hypersetup{pdfcontactcity={Zurich}}
\hypersetup{pdfcontactcountry={Switzerland}}
\hypersetup{pdfcontactemail={nbeisert@itp.phys.ethz.ch}}
\hypersetup{pdfcontacturl={http://people.phys.ethz.ch/\xmptilde nbeisert/}}

\newcommand{\secref}[1]{\hyperref[#1]{section \ref*{#1}}}

\parskip1ex
\parindent0pt
\let\olditemize\itemize
\def\itemize{\olditemize\parskip0pt}

\begin{document}

\title{The \textsf{childdoc} Package}
\hypersetup{pdftitle={The childdoc Package}}
\author{Niklas Beisert\\[2ex]
  Institut f\"ur Theoretische Physik\\
  Eidgen\"ossische Technische Hochschule Z\"urich\\
  Wolfgang-Pauli-Strasse 27, 8093 Z\"urich, Switzerland\\[1ex]
  \href{mailto:nbeisert@itp.phys.ethz.ch}
  {\texttt{nbeisert@itp.phys.ethz.ch}}}
\hypersetup{pdfauthor={Niklas Beisert}}
\hypersetup{pdfsubject={Manual for the LaTeX2e Package childdoc}}
\date{30 December 2018, \textsf{v2.0}}
\maketitle

\begin{abstract}\noindent
\textsf{childdoc} is a \LaTeXe{} package
that enables the direct compilation
of document sections included by |\include|
to individual files.
\end{abstract}

\begingroup
\parskip0ex
\tableofcontents
\endgroup

%%%%%%%%%%%%%%%%%%%%%%%%%%%%%%%%%%%%%%%%%%%%%%%%%%%%%%%%%%%%%%%%%%%%%%%%%%%%%%%%
%%%%%%%%%%%%%%%%%%%%%%%%%%%%%%%%%%%%%%%%%%%%%%%%%%%%%%%%%%%%%%%%%%%%%%%%%%%%%%%%
\section{Introduction}

\LaTeX{} provides a mechanism to structure a large document (such as a book)
into a main file and several child files (containing the chapters)
using the |\include| command.
This mechanism is beneficial for documents
which span hundreds of pages in order to
make the source file(s) more manageable.
Moreover, compilation can be restricted to
selected child files by means of the |\includeonly| command.
The latter feature can be used to reduce the compilation time while editing
(this was significantly more useful in the earlier days of \LaTeX{})
or to generate a smaller document which is easier to navigate.
Another application of |\includeonly| is to generate
documents consisting of selected parts of the complete document.

However, there are a few drawbacks of the plain |\include| mechanism:
\begin{itemize}
\item
The child files cannot be compiled on their own,
they can only be compiled via the main file.
A naive editing environment
(such as a text editor with an option
to have the current file processed by \LaTeX)
may require one to switch to the main file before compiling;
attempting to compile the child file produces errors.
\item
The main file must be modified (each time)
to adjust the |\includeonly| command
to the present needs. This easily leaves the main file in a messy state.
\item
The generated document will always carry the filename
of the main document. This is inconvenient if
several child files are to be compiled and
to be kept for distribution.
\end{itemize}

The present package provides a simple interface
to make child files individually compilable by \LaTeX{}.
Compiling a child file then has the same effect as compiling
the main file with an |\includeonly| command
to select the appropriate child.
Moreover the generated document will carry the name of the child
rather than the main file.
This resolves all three above issues.

This feature is meant to make the editing of books,
thesis documents and lecture notes somewhat more convenient.
However, the package can also be used efficiently for
composing a series of documents (such as exercise sheets)
which are typically distributed individually.
It then assists the author in generating the individual documents
(potentially in different versions)
as well as a document containing the collected series.
Another application is in developing style files
or other kinds of included material
where compilation of the style file could redirect
to a sample or test file.

%%%%%%%%%%%%%%%%%%%%%%%%%%%%%%%%%%%%%%%%%%%%%%%%%%%%%%%%%%%%%%%%%%%%%%%%%%%%%%%%
%%%%%%%%%%%%%%%%%%%%%%%%%%%%%%%%%%%%%%%%%%%%%%%%%%%%%%%%%%%%%%%%%%%%%%%%%%%%%%%%
\section{Usage}

First of all, the package \textsf{childdoc} is \emph{not} a standard
\LaTeXe{} |.sty| style file! Therefore it needs to be invoked in
a non-standard way.

%%%%%%%%%%%%%%%%%%%%%%%%%%%%%%%%%%%%%%%%%%%%%%%%%%%%%%%%%%%%%%%%%%%%%%%%%%%%%%%%
\subsection{Included Files}
\label{sec:include}

%%%%%%%%%%%%%%%%%%%%%%%%%%%%%%%%%%%%%%%%
\DescribeMacro{\childdocmain}
To use the package, add the commands
\begin{center}
\begin{tabular}{l}
|\input{childdoc.def}|\\
|\childdocmain{}|\\
\end{tabular}
\end{center}
at the very top of the main \LaTeX{} file,
in particular \emph{before} the |\documentclass| statement!
The argument of |\childdocmain| should be left empty
(but it must be present).

%%%%%%%%%%%%%%%%%%%%%%%%%%%%%%%%%%%%%%%%
\DescribeMacro{\childdocof}
Furthermore, add the commands
\begin{center}
\begin{tabular}{l}
|\input{childdoc.def}|\\
|\childdocof{|\textit{main}|}|\\
\end{tabular}
\end{center}
at the top of every child file \textit{child}
which is included by |\include{|\textit{child}|}|
from within the main file
(or at least for those files to be compiled individually).
The argument \textit{main} must be the filename of the main file.

There are a couple of
considerations in setting up the main and child documents:

%%%%%%%%%%%%%%%%%%%%%%%%%%%%%%%%%%%%%%%%
\paragraph{Restrictions.}

Please note the following restrictions:
\begin{itemize}
\item
|\childdocmain| must be called with one argument \textit{main}
to ensure compatibility with earlier version of the package.
It must either be empty (|\childdocmain{}|)
or precisely match the filename of the main file in which it is specified.
See \secref{sec:detection} for further information.
\item
The filename \textit{main} must be specified without the |.tex| extension.
\item
The filename \textit{main} is case sensitive
(even in case-insensitive file systems)
due to internal string comparison.
\item
The argument \textit{main} should be fully expanded, it cannot be a macro.
\item
Subdirectories and special characters should be avoided in filenames.
\item
The command |\childdocmain{|\textit{main}|}| must be followed by a whitespace.
It should not be followed immediately by another command
or by a comment mark `|%|'.
This is because the \TeX{} parser reads the token immediately following
the argument of |\childdocmain| and puts it
at the beginning of every child section;
however, a white\-space is ignored.
\end{itemize}

%%%%%%%%%%%%%%%%%%%%%%%%%%%%%%%%%%%%%%%%
\paragraph{Content of Main File.}

It is advisable to place all content in the child files included by |\include|.
Any output contained in the main file will appear in all child documents
unless suppressed manually;
it cannot be suppressed automatically by the |\includeonly| directive
and thus should normally be avoided.
A method to include some content in the main file
by means of conditional processing is described in \secref{sec:conditional}.

%%%%%%%%%%%%%%%%%%%%%%%%%%%%%%%%%%%%%%%%
\paragraph{Page Numbering.}

When only a part of the document is compiled,
the appropriate numbering of pages
(as well as other status parameters)
is determined from the |.aux| files.
The latter contain information from previous passes.
However this information needs to propagate through
all intermediate child documents.
Therefore the page numbering in child documents may well
be inconsistent until the complete document is compiled at least once.

A useful (if unconventional) way to always ensure a consistent
page numbering is to restart the numbering in each child document
and denote the pages by `\textit{child}|.|\textit{page}'
where \textit{child} represents the chapter/section number of the child file.
This can be achieved by the command
|\numberwithin{page}{|\textit{child}|}|
of the \textsf{amsmath} package
where \textit{child} can be |chapter| or |section|
depending on the chosen structuring.
Alternatively, one can modify the macro |\thepage| appropriately
and reset the counter |page| at the start of each child file.

%%%%%%%%%%%%%%%%%%%%%%%%%%%%%%%%%%%%%%%%%%%%%%%%%%%%%%%%%%%%%%%%%%%%%%%%%%%%%%%%
\subsection{Conditional Processing}
\label{sec:conditional}

The package provides a mechanism to compile different versions
of a document. To customise the versions further some conditional processing
can come in handy to distinguish which version is being compiled.
The package provides two macros to describe the compilation context:

%%%%%%%%%%%%%%%%%%%%%%%%%%%%%%%%%%%%%%%%
\DescribeMacro{\ifchilddoc}
The conditional |\ifchilddoc| distinguishes between the compilation of
child documents and the main document:
%
\begin{center}
|\ifchilddoc |\textit{child-code}| |[|\||else |\textit{main-code}]| \||fi|
\end{center}

%%%%%%%%%%%%%%%%%%%%%%%%%%%%%%%%%%%%%%%%
\DescribeMacro{\childdocname}
\DescribeMacro{\childdocjob}
The macro |\childdocname| contains the filename (without extension)
of the main or child file being processed.
Note that |\childdocjob| will always contain the name of the main file.

%%%%%%%%%%%%%%%%%%%%%%%%%%%%%%%%%%%%%%%%
\paragraph{Title Page.}

Conditional processing can be used to include a title or banner page
in the main document when proper precautions are taken.
Importantly, the code in the main file should ensure that the page counter
(as well as other status parameters which are stored in the |.aux| files)
takes the same value after the conditional processing.
Otherwise the page numbers may take divergent values
depending on which part is compiled.

For example, a title page could be declared by:
%
\begin{center}
\begin{tabular}{l}
|\ifchilddoc\||else|\\
|\addtocounter{page}{-1}|\\
\textit{code for title page}\\
|\newpage|\\
|\||fi|
\end{tabular}
\end{center}
%
A banner page for the child documents can be generated by:
%
\begin{center}
\begin{tabular}{l}
|\ifchilddoc|\\
|\addtocounter{page}{-1}|\\
\textit{code for banner page}\\
|\newpage|\\
|\||fi|
\end{tabular}
\end{center}
%
Here one could write a message such as:
\begin{center}
|This is the part \childdocname{} of \childdocjob{}.|
\end{center}

%%%%%%%%%%%%%%%%%%%%%%%%%%%%%%%%%%%%%%%%%%%%%%%%%%%%%%%%%%%%%%%%%%%%%%%%%%%%%%%%
\subsection{Flags}
\label{sec:flags}

The package makes it easy to generate different versions
of the main or child documents.
To this end compilation flags can be defined
and assigned different default values.
They will be particularly useful in conjunction
with the forwarding mechanism described in \secref{sec:forward}.

For example, it may be useful to have a flag |\version|
which can be set to |draft| or |final|.
The document source will contain some conditional code
depending on the value of |\version|.
Suppose further, the flag should default to |final| for the main file
and to |draft| for child files
which is a natural assignment for editing the document.
This is achieved by placing the following code
in the preamble of the main document
(below the |\childdocmain| directive):
%
\begin{center}
\begin{tabular}{l}
|\ifchilddoc|\\
|\providecommand{\version}{draft}|\\
|\||else|\\
|\providecommand{\version}{final}|\\
|\||fi|
\end{tabular}
\end{center}
%
The definition by |\providecommand| makes sure
that previous definitions are not overwritten.
Further statements |\providecommand{\version}{...}|
can thus be added before the above code to override it.

For the main file, one might add a line
(between |\childdocmain| and the above block)
%
\begin{center}
|%\ifchilddoc\||else\providecommand{\version}{draft}\||fi|
\end{center}
%
which can be uncommented to produce a draft version.
Likewise one can add a line to the very top of a child file
(above the |\childdocof{|\textit{main}|}| directive)
%
\begin{center}
|%\providecommand{\version}{final}|
\end{center}
%
which can be uncommented to produce the final version of this child document.

%%%%%%%%%%%%%%%%%%%%%%%%%%%%%%%%%%%%%%%%%%%%%%%%%%%%%%%%%%%%%%%%%%%%%%%%%%%%%%%%
\subsection{Forwarding}
\label{sec:forward}

Different versions of the main or child documents
using compilation flags as described in \secref{sec:flags}
can be (permanently) stored in different files
for convenient compilation, viewing and distribution.
To this end, the package defines a command
to pass on compilation to a different file:

%%%%%%%%%%%%%%%%%%%%%%%%%%%%%%%%%%%%%%%%
\DescribeMacro{\childdocforward}
The command |\childdocforward| redirects processing to
another source file:
%
\begin{center}
\begin{tabular}{l}
|\input{childdoc.def}|\\
|\childdocforward[|\textit{main}|]{|\textit{dest}|}|\\
\end{tabular}
\end{center}
%
The argument \textit{dest} is the destination file
(without extension).
It should be the main file or one of the child files.
Note that further \textsf{childdoc} directives
such as |\childdocof| and |\childdocforward|
in the indicated file will be processed in this form.
The optional argument \textit{main}
passes on directly to the main file \textit{main}
while pretending to compile the child \textit{dest}.
This form behaves as if \textit{dest}
issues |\childdocof{|\textit{main}|}| right away,
and no further \textsf{childdoc} directives will be processed.

%%%%%%%%%%%%%%%%%%%%%%%%%%%%%%%%%%%%%%%%
\DescribeMacro{\...prefix}
In the alternative form |\childdocforwardprefix|,
%
\begin{center}
\begin{tabular}{l}
|\input{childdoc.def}|\\
|\childdocforwardprefix[|\textit{main}|]{|\textit{prefix}|}{|\textit{dest}|}|
\end{tabular}
\end{center}
%
the destination file is determined by a pattern
depending on the current file:
To make this work, the current file must be called
`{\textit{prefix}\hspace{0.2em}\textit{suffix}}'
with \textit{prefix} matching precisely the argument.
Processing is then passed on to the file
`{\textit{dest}\hspace{0.2em}\textit{suffix}}'.
Surely, the same effect is achieved by
directly specifying the
argument `{\textit{dest}\hspace{0.2em}\textit{suffix}}'
in the first form.
However, that requires to set up a different file
for each child. With the alternative form of the command
all these files can have exactly the same content
which simplifies setting them up and maintaining them.

For example, the following file |draft.tex|
with a compilation flag |\version| as described in \secref{sec:flags}
compiles the main document as a draft:
%
\begin{center}
\begin{tabular}{l}
|\def\version{draft}|\\
|\input{childdoc.def}|\\
|\childdocforward{|\textit{main}|}|
\end{tabular}
\end{center}
%
Likewise, the following files |final|\textit{nn}|.tex|
compile the final version of the child document
|child|\textit{nn}|.tex|:
%
\begin{center}
\begin{tabular}{l}
|\def\version{final}|\\
|\input{childdoc.def}|\\
|\childdocforwardprefix{final}{child}|
\end{tabular}
\end{center}
%

Note that when several versions of a main file and/or of each child file
are to be generated, it may be convenient to set up a |Makefile| or
shell script to automatise the process.

%%%%%%%%%%%%%%%%%%%%%%%%%%%%%%%%%%%%%%%%%%%%%%%%%%%%%%%%%%%%%%%%%%%%%%%%%%%%%%%%
\subsection{Command Line Processing}
\label{sec:commandline}

The effect of redirection files can also be achieved by invoking
the \LaTeX{} compiler with a more elaborate command line.
Most conveniently this should be done as part
of a shell script or a |Makefile|.

When using \textsf{childdoc} in the main file, the following
command lines effectively perform a redirection
(note that depending on the shell being used,
backslashes may have to be doubled: `|\|' $\to$ `|\\|'):
%
\begin{center}
|... -jobname "|\textit{target}|" |\\|"|[\textit{flags}]%
|\input{childdoc.def}\childdocforward[|\textit{main}|]{|\textit{dest}|}"|
\end{center}
%
Here \textit{target} is the name of the output file,
\textit{main} is the name of the main file
and \textit{dest} is the name of the main or child file to be processed
(all filenames without extensions).
The optional argument \textit{main} can be omitted
if \textit{main} matches \textit{dest}.
Optionally, compilation \textit{flags} can be defined via |\def| commands.
This command line makes the \TeX{} engine believe
it is compiling the file \textit{target}
whose content is specified as the latter parameter.
The provided code then forwards the processing to
\textit{main} or \textit{dest} as described in \secref{sec:forward}.

%%%%%%%%%%%%%%%%%%%%%%%%%%%%%%%%%%%%%%%%%%%%%%%%%%%%%%%%%%%%%%%%%%%%%%%%%%%%%%%%
\subsection{Include by Input}
\label{sec:input}

Including child documents by |\include| has some restrictions by design.
Most notably, the content of a child document always occupies
its own set of pages; pages cannot be shared between child documents.
Usually, this behaviour makes perfect sense
because each child document contain an essential part of the document.
However, in some situations it may be desirable to compose
a document from a collection of parts
without having mandatory page breaks between then.
For this case, the package
provides a mechanism to include parts
by |\input| which can also be processed individually.
However, by construction this mechanism
requires manual handling of the content to be output.

%%%%%%%%%%%%%%%%%%%%%%%%%%%%%%%%%%%%%%%%
\DescribeMacro{\ifchilddocmanual}
The main file should be prepared as usual, see \secref{sec:include}.
However, the document body must make a distinction
between processing of an individual part and of the main document, e.g.:
%
\begin{center}
\begin{tabular}{l}
|\ifchilddocmanual|\\
|\input{\childdocname}|\\
|\||else|\\
\textit{document body with }|\input{|\textit{part}|}|\\
|\||fi|
\end{tabular}
\end{center}
%
The conditional |\ifchilddocmanual| is true whenever
a part to be included by |\input| is being compiled,
and the name of the part is stored in |\childdocname|.

%%%%%%%%%%%%%%%%%%%%%%%%%%%%%%%%%%%%%%%%
\DescribeMacro{\childdocby}
Each part to be included by |\input| should start with:
%
\begin{center}
\begin{tabular}{l}
|\input{childdoc.def}|\\
|\childdocby{|\textit{main}|}|\\
\end{tabular}
\end{center}
%
The directive |\childdocby| is similar to |\childdocof|
described in \secref{sec:include},
but the subsequent selection of content must be done manually.
To that end, both |\ifchilddoc| and |\ifchilddocmanual|
will be true upon processing of a part,
and the name of the part is stored in |\childdocname|.
Note that |\jobname| will be set to the filename of the current part
so that each part receives an individual |.aux| file
that does not interfere with the |.aux| file(s) of the main document.
This behaviour can be altered by the alternative form
|\childdocby[*]{|\textit{main}|}| (with a non-empty optional argument)
which uses the |.aux| file of the main document
by setting |\jobname| to \textit{main}.

%%%%%%%%%%%%%%%%%%%%%%%%%%%%%%%%%%%%%%%%%%%%%%%%%%%%%%%%%%%%%%%%%%%%%%%%%%%%%%%%
\subsection{Driver Development}
\label{sec:driver}

The \textsf{childdoc} mechanism can also be use for the development
of definition files such as \LaTeX{} styles or classes.
This case differs from the above setup with multiple parts
included by |\include| in that no |\includeonly| should be invoked.
This can be achieved by starting the include file
(before |\ProvidesPackage|) with:
%
\begin{center}
\begin{tabular}{l}
|\input{childdoc.def}|\\
|\childdocforward{|\textit{main}|}|\\
\end{tabular}
\end{center}
%
or alternatively with:
%
\begin{center}
\begin{tabular}{l}
|\input{childdoc.def}|\\
|\childdocby{|\textit{main}|}|\\
\end{tabular}
\end{center}
%
Both forms have slightly different effects as described above.
The main file is prepared as usual, see \secref{sec:include}.

%%%%%%%%%%%%%%%%%%%%%%%%%%%%%%%%%%%%%%%%%%%%%%%%%%%%%%%%%%%%%%%%%%%%%%%%%%%%%%%%
\subsection{Legacy Detection}
\label{sec:detection}

The directive |\childdocmain| in the main file can detect
whether the complete document or merely a child is to be compiled
even without using the directive |\childdocof|.
This method is deprecated because it is less robust
and there is no compelling reason to use it;
it is merely provided for backward compatibility
and it may be removed in future versions.

If the detection mechanism is to be used,
it is mandatory to correctly specify
the filename of the main file as the argument of |\childdocmain|:
%
\begin{center}
\begin{tabular}{l}
|\input{childdoc.def}|\\
|\childdocmain{|\textit{main}|}|\\
\end{tabular}
\end{center}
%
If |\jobname| does not match the argument \textit{main} of |\childdocmain|,
it is assumed that |\jobname| points to the child file to be compiled.
When using |\childdocmain| with the main file specified as argument,
it suffices to start a child file
with just |\input{|\textit{main}|}|
without loading of the package and using |\childdocof|.
If instead all processing is done
with the appropriate \textsf{childdoc} directives,
the argument of \textit{main} of |\childdocmain| can be empty.

An alternative version of the command line processing described
in \secref{sec:commandline} using the detection mechanism reads:
%
\begin{center}
|... -jobname "|\textit{target}|" "|[\textit{flags}]%
[|\def\jobname{|\textit{dest}|}|]|\input{|\textit{main}|}"|
\end{center}

%%%%%%%%%%%%%%%%%%%%%%%%%%%%%%%%%%%%%%%%%%%%%%%%%%%%%%%%%%%%%%%%%%%%%%%%%%%%%%%%
\subsection{Manual Code}
\label{sec:manual}

In case one cannot be certain whether the definitions file |childdoc.def|
is installed on the target \TeX{} distribution
and one prefers not to ship it,
it is conceivable to paste a few relevant commands into the sources.

To that end, drop all statements |\input{childdoc.def}|
and perform the replacements as outlined below.
Instead of |\childdocmain{|\textit{main}|}| add the following code
to the top of the main file:
%
\begin{center}
\begin{tabular}{l}
|\||ifdefined\childdocname\endinput\||fi\newif\ifchilddoc|\\
|\edef\childdocname{\scantokens\expandafter{\jobname\noexpand}}|\\
|\def\childdocmain{|\textit{main}|}\||ifx\childdocmain\childdocname\||else|\\
|\childdoctrue\includeonly{\childdocname}\let\jobname\childdocmain\||fi|\\
\end{tabular}
\end{center}
%
Instead of |\childdocof{|\textit{main}|}| just include the main file
at the top of each child file:
%
\begin{center}
|\input{|\textit{main}|}|
\end{center}
%
A simple redirection |\childdocforward{|\textit{dest}|}| is achieved by:
%
\begin{center}
|\def\jobname{|\textit{dest}|}\input{\jobname}|
\end{center}
%
The redirection with prefix
|\childdocforwardprefix[|\textit{prefix}|]{|\textit{dest}|}|
is accomplished by:
%
\begin{center}
\begin{tabular}{l}
|{\edef\jobname{\scantokens\expandafter{\jobname\noexpand}}|\\
|\def\redirectjob |\textit{prefix}|#1~~~{\gdef\jobname{|\textit{dest}|#1}}|\\
|\expandafter\redirectjob\jobname~~~}\input{\jobname}|
\end{tabular}
\end{center}

In an alternative approach,
child documents can be compiled by a specific command line
without additional code or specific definitions:
%
\begin{center}
|... -jobname "|\textit{target}|" "|[\textit{flags}]%
|\includeonly{|\textit{dest}|}\input{|\textit{main}|}"|
\end{center}
%

%%%%%%%%%%%%%%%%%%%%%%%%%%%%%%%%%%%%%%%%%%%%%%%%%%%%%%%%%%%%%%%%%%%%%%%%%%%%%%%%
%%%%%%%%%%%%%%%%%%%%%%%%%%%%%%%%%%%%%%%%%%%%%%%%%%%%%%%%%%%%%%%%%%%%%%%%%%%%%%%%
\section{Information}

%%%%%%%%%%%%%%%%%%%%%%%%%%%%%%%%%%%%%%%%%%%%%%%%%%%%%%%%%%%%%%%%%%%%%%%%%%%%%%%%
\subsection{Copyright}

Copyright \copyright{} 2017--2018 Niklas Beisert

This work may be distributed and/or modified under the
conditions of the \LaTeX{} Project Public License, either version 1.3
of this license or (at your option) any later version.
The latest version of this license is in
  \url{http://www.latex-project.org/lppl.txt}
and version 1.3 or later is part of all distributions of \LaTeX{}
version 2005/12/01 or later.

This work has the LPPL maintenance status `maintained'.

The Current Maintainer of this work is Niklas Beisert.

This work consists of the files |README.txt|, |childdoc.ins| and |childdoc.dtx|
as well as the derived files |childdoc.def|, |cdocsamp.tex|
with |cdocsch1.tex|, |cdocsch2.tex|, |cdocspt3.tex|, |cdocspt4.tex|,
|cdocsdrf.tex|, |cdocsfn1.tex|, |cdocsfn2.tex|
as well as |childdoc.pdf|.

%%%%%%%%%%%%%%%%%%%%%%%%%%%%%%%%%%%%%%%%%%%%%%%%%%%%%%%%%%%%%%%%%%%%%%%%%%%%%%%%
\subsection{Files and Installation}

The package consists of the files:
%
\begin{center}
\begin{tabular}{ll}
    |README.txt|   & readme file \\
    |childdoc.ins| & installation file \\
    |childdoc.dtx| & source file \\
    |childdoc.def| & definition file \\
    |cdocsamp.tex| & sample main file \\
    |cdocsch1.tex| & sample include file \\
    |cdocsch2.tex| & sample include file \\
    |cdocspt3.tex| & sample part file \\
    |cdocspt4.tex| & sample part file \\
    |cdocsdrf.tex| & sample redirection file \\
    |cdocsfn1.tex| & sample redirection file \\
    |cdocsfn2.tex| & sample redirection file \\
    |childdoc.pdf| & manual
\end{tabular}
\end{center}
%
The distribution consists of the files
|README.txt|, |childdoc.ins| and |childdoc.dtx|.
%
\begin{itemize}
\item
Run (pdf)\LaTeX{} on |childdoc.dtx|
to compile the manual |childdoc.pdf| (this file).
\item
Run \LaTeX{} on |childdoc.ins| to create the definitions file |childdoc.def|
and the sample |cdocsamp.tex| with include files
|cdocsch1.tex|, |cdocsch2.tex|, |cdocspt3.tex|, |cdocspt4.tex|,
|cdocsdrf.tex|, |cdocsfn1.tex|, |cdocsfn2.tex|.
Then copy the file |childdoc.def| to an appropriate directory of your \LaTeX{}
distribution, e.g.\ \textit{texmf-root}|/tex/latex/childdoc|.
\end{itemize}

%%%%%%%%%%%%%%%%%%%%%%%%%%%%%%%%%%%%%%%%%%%%%%%%%%%%%%%%%%%%%%%%%%%%%%%%%%%%%%%%
\subsection{Related CTAN Packages}

There are several other packages which offer a similar functionality:
%
\begin{itemize}
\item
The packages
\href{http://ctan.org/pkg/docmute}{\textsf{docmute}},
\href{http://ctan.org/pkg/includex}{\textsf{includex}} and
\href{http://ctan.org/pkg/standalone}{\textsf{standalone}}
provide commands to include only the document body of
a child file thus allowing both files to be compiled individually.
\item
The packages \href{http://ctan.org/pkg/subdocs}{\textsf{subdocs}}
and \href{http://ctan.org/pkg/subfiles}{\textsf{subfiles}}
provide structures in which the main and child documents can be
encapsulated and allowing them to be compiled individually.
The inclusion mechanism is different from the conventional |\include|.
\item
The package \href{http://ctan.org/pkg/combine}{\textsf{combine}}
is an elaborate solution to combine several documents into one.
\end{itemize}
%
See also the CTAN topic \href{http://ctan.org/topic/subdocs}{\textsf{subdocs}}
for further related packages.
The present package differs from the above solutions in that
a document structure constructed with the conventional |\include| mechanism
just needs two extra commands at the top of every file
such that all constituent files can be compiled individually.

%%%%%%%%%%%%%%%%%%%%%%%%%%%%%%%%%%%%%%%%%%%%%%%%%%%%%%%%%%%%%%%%%%%%%%%%%%%%%%%%
%\subsection{Feature Suggestions}
%
%The following is a list of features which may be useful for future
%versions of this package:
%%
%\begin{itemize}
%\item
%\ldots
%\end{itemize}

%%%%%%%%%%%%%%%%%%%%%%%%%%%%%%%%%%%%%%%%%%%%%%%%%%%%%%%%%%%%%%%%%%%%%%%%%%%%%%%%
\subsection{Revision History}

%%%%%%%%%%%%%%%%%%%%%%%%%%%%%%%%%%%%%%%%
\paragraph{v2.0:} 2018/12/30

\begin{itemize}
\item
immediate forward processing
\item
added |\childdocby| mechanism
\item
manual restructured
\end{itemize}

%%%%%%%%%%%%%%%%%%%%%%%%%%%%%%%%%%%%%%%%
\paragraph{v1.6:} 2018/01/17

\begin{itemize}
\item
application for development of include files
\item
corrections to manual
\end{itemize}

%%%%%%%%%%%%%%%%%%%%%%%%%%%%%%%%%%%%%%%%
\paragraph{v1.5:} 2017/05/21

\begin{itemize}
\item
more complete structuring introduced
\item
|\childdocof| introduced
\item
|\childdoc| renamed to |\childdocmain|
\item
|\childredirect| renamed to |\childdocforward| and |\childdocforwardprefix|
and functionality expanded
\end{itemize}

%%%%%%%%%%%%%%%%%%%%%%%%%%%%%%%%%%%%%%%%
\paragraph{v1.0:} 2017/04/27

\begin{itemize}
\item
manual and install package
\item
first version published on CTAN
\end{itemize}

%%%%%%%%%%%%%%%%%%%%%%%%%%%%%%%%%%%%%%%%
\paragraph{v0.6:} 2017/04/26

\begin{itemize}
\item
redirection mechanism added
\end{itemize}

%%%%%%%%%%%%%%%%%%%%%%%%%%%%%%%%%%%%%%%%
\paragraph{v0.5:} 2017/04/26

\begin{itemize}
\item
functionality in definition file
\end{itemize}


%%%%%%%%%%%%%%%%%%%%%%%%%%%%%%%%%%%%%%%%%%%%%%%%%%%%%%%%%%%%%%%%%%%%%%%%%%%%%%%%
%%%%%%%%%%%%%%%%%%%%%%%%%%%%%%%%%%%%%%%%%%%%%%%%%%%%%%%%%%%%%%%%%%%%%%%%%%%%%%%%
%%%%%%%%%%%%%%%%%%%%%%%%%%%%%%%%%%%%%%%%%%%%%%%%%%%%%%%%%%%%%%%%%%%%%%%%%%%%%%%%
\appendix

\settowidth\MacroIndent{\rmfamily\scriptsize 000\ }

 \DocInput{childdoc.dtx}

\end{document}
%</driver>
% \fi
%
% %%%%%%%%%%%%%%%%%%%%%%%%%%%%%%%%%%%%%%%%%%%%%%%%%%%%%%%%%%%%%%%%%%%%%%%%%%%%%%
% %%%%%%%%%%%%%%%%%%%%%%%%%%%%%%%%%%%%%%%%%%%%%%%%%%%%%%%%%%%%%%%%%%%%%%%%%%%%%%
% \section{Sample}
%\iffalse
%<*samplemain>
%\fi
%
% The following presents a sample document
% with two chapters, two parts, a title page,
% a compile flag as well as three forwarding files to set the flag.
% It consists of eight |.tex| files:
% \begin{center}
% \begin{tabular}{ll}
% |cdocsamp.tex|&main file\\
% |cdocsch1.tex|&include file for chapter 1\\
% |cdocsch2.tex|&include file for chapter 2\\
% |cdocspt3.tex|&include file for part 3\\
% |cdocspt4.tex|&include file for part 4\\
% |cdocsdrf.tex|&forwarding file for main file in draft mode\\
% |cdocsfi1.tex|&forwarding file for final version of chapter 1\\
% |cdocsfi2.tex|&forwarding file for final version of chapter 2\\
% \end{tabular}
% \end{center}
% Each of the eight files can be compiled directly by the \LaTeX{} compiler.
%
% %%%%%%%%%%%%%%%%%%%%%%%%%%%%%%%%%%%%%%
% \paragraph{Main File.}
%
% The main file is called |cdocsamp.tex|.
%
% Load the \textsf{childdoc} definitions and
% declare the filename for the main document:
%    \begin{macrocode}
\input{childdoc.def}
\childdocmain{}
%    \end{macrocode}

% Optional override for |\version| flag:
%    \begin{macrocode}
%%\ifchilddoc\else\providecommand{\version}{draft}\fi
%    \end{macrocode}

% Define the default values for the |\version| flag
% (|final| for the main file and |draft| for childs):
%    \begin{macrocode}
\ifchilddoc
\providecommand{\version}{draft}
\else
\providecommand{\version}{final}
\fi
%    \end{macrocode}

% Load the standard document class:
%    \begin{macrocode}
\documentclass[12pt]{article}
%    \end{macrocode}

% Start the document body:
%    \begin{macrocode}
\begin{document}
%    \end{macrocode}

% Declare a title page.
% Print title, part of document being processed and version flag:
%    \begin{macrocode}
\addtocounter{page}{-1}
\begin{center}
{\LARGE\bfseries{}childdoc example\par}
\vspace{1cm}
\ifchilddoc
\ifchilddocmanual part\else chapter\fi:
`\childdocname' of `\childdocjob'\par
\else
main document: `\childdocjob'\par
\fi
version: \version\par
\end{center}
\newpage
%    \end{macrocode}

% Manually include selected file,
% otherwise process as usual:
%    \begin{macrocode}
\ifchilddocmanual
\section*{part `\childdocname'}
\input{\childdocname}
\else
%    \end{macrocode}

% Include the two chapters:
%    \begin{macrocode}
\include{cdocsch1}
\include{cdocsch2}
%    \end{macrocode}

% Include the two parts unless only chapters should be displayed:
%    \begin{macrocode}
\ifchilddoc\else
\section{part three}
\input{cdocspt3}
\section{part four}
\input{cdocspt4}
\fi
%    \end{macrocode}

% Process as usual until here:
%    \begin{macrocode}
\fi
%    \end{macrocode}

% End of document body:
%    \begin{macrocode}
\end{document}
%    \end{macrocode}
%\iffalse
%</samplemain>
%\fi
%
% %%%%%%%%%%%%%%%%%%%%%%%%%%%%%%%%%%%%%%
% \paragraph{Chapter Include Files.}
%
% The include files are called |cdocsch1.tex| and |cdocsch2.tex|.
%
%\iffalse
%<*samplechap1|samplechap2>
%\fi

% Optional override for |\version| flag:
%    \begin{macrocode}
%%\providecommand{\version}{final}
%    \end{macrocode}

% Include the main document:
%    \begin{macrocode}
\input{childdoc.def}
\childdocof{cdocsamp}
%    \end{macrocode}

%\iffalse
%</samplechap1|samplechap2>
%\fi
%
%\iffalse
%<*samplechap1>
%\fi
% Some text for chapter 1:
%    \begin{macrocode}
\section{one}
some text in chapter one
%    \end{macrocode}

%\iffalse
%</samplechap1>
%\fi
% Some text for chapter 2:
%\iffalse
%<*samplechap2>
%\fi
%    \begin{macrocode}
\section{two}
more text in chapter two
%    \end{macrocode}

%\iffalse
%</samplechap2>
%\fi
%
% %%%%%%%%%%%%%%%%%%%%%%%%%%%%%%%%%%%%%%
% \paragraph{Part Include Files.}
%
% The include files are called |cdocspt3.tex| and |cdocspt4.tex|.
%
%\iffalse
%<*samplepart3|samplepart4>
%\fi

% Optional override for |\version| flag:
%    \begin{macrocode}
%%\providecommand{\version}{final}
%    \end{macrocode}

% Include the main document:
%    \begin{macrocode}
\input{childdoc.def}
\childdocby{cdocsamp}
%    \end{macrocode}

%\iffalse
%</samplepart3|samplepart4>
%\fi
%
%\iffalse
%<*samplepart3>
%\fi
% Some text for part 3:
%    \begin{macrocode}
some text in part three
%    \end{macrocode}

%\iffalse
%</samplepart3>
%\fi
% Some text for part 4:
%\iffalse
%<*samplepart4>
%\fi
%    \begin{macrocode}
more text in part four
%    \end{macrocode}

%\iffalse
%</samplepart4>
%\fi
%
% %%%%%%%%%%%%%%%%%%%%%%%%%%%%%%%%%%%%%%
% \paragraph{Forwarding for a Complete Draft.}
%
% The following forwarding file |cdocsdrf.tex|
% compiles the main document in draft mode:
%\iffalse
%<*sampledraft>
%\fi
%    \begin{macrocode}
\def\version{draft}
\input{childdoc.def}
\childdocforward{cdocsamp}
%    \end{macrocode}

%\iffalse
%</sampledraft>
%\fi
%
% %%%%%%%%%%%%%%%%%%%%%%%%%%%%%%%%%%%%%%
% \paragraph{Forwarding for Final Version of the Chapters.}
%
% The following forwarding files |cdocsfn1.tex| and |cdocsfn2.tex|
% (with identical content)
% compile the final versions of the child documents
% |cdocsch1.tex| and |cdocsch2.tex|, respectively:
%\iffalse
%<*samplefinal>
%\fi
%    \begin{macrocode}
\def\version{final}
\input{childdoc.def}
\childdocforwardprefix[cdocsamp]{cdocsfn}{cdocsch}
%    \end{macrocode}

%\iffalse
%</samplefinal>
%\fi
%
% %%%%%%%%%%%%%%%%%%%%%%%%%%%%%%%%%%%%%%
% \paragraph{Command Line Processing.}
%
% The following three command lines generate the output files
% |cdocscld|, |cdocscl1| and |cdocscl2|
% which should be identical to
% |cdocsdrf|, |cdocsch1| and |cdocsfn2|, respectively:
% \begin{center}
% \begin{tabular}{l}
% |latex -jobname cdocscld \|\\
% |  "\def\version{draft}\input{childdoc.def}\childdocforward{cdocsamp}"|\\
% |latex -jobname cdocscl1 \|\\
% |  "\input{childdoc.def}\childdocforward[cdocsamp]{cdocsch1}"|\\
% |latex -jobname cdocscl2 \|\\
% |  "\def\version{final}\input{childdoc.def}\childdocforward{cdocsch2}"|
% \end{tabular}
% \end{center}
% Note that the trailing backslash on each first line
% merely continues the input to the second line
% (for convenient cut ant paste).
% Furthermore, the command |latex| can be replaced by any
% of its alternative versions such as |pdflatex|.
%
% %%%%%%%%%%%%%%%%%%%%%%%%%%%%%%%%%%%%%%%%%%%%%%%%%%%%%%%%%%%%%%%%%%%%%%%%%%%%%%
% %%%%%%%%%%%%%%%%%%%%%%%%%%%%%%%%%%%%%%%%%%%%%%%%%%%%%%%%%%%%%%%%%%%%%%%%%%%%%%
% \section{Implementation}
%\iffalse
%<*package>
%\fi
%
% This section describes the definitions file |childdoc.def|.

% The definitions cannot be loaded using |\usepackage| or |\RequirePackage|
% which has a mechanism to prevent loading a style file more than once.
% When loading the definitions by means of |\input|
% multiple instances have to be prevented manually:
%\iffalse
%This code needs to be before the `\ProvidesFile' directive
%which is defined at the beginning of this file.
%Therefore it is also placed there and commented out here.
%</package>
%<*discard>
%\fi
%    \begin{macrocode}
\ifdefined\childdocmain\endinput\fi
%    \end{macrocode}
%\iffalse
%</discard>
%<*package>
%\fi
%
% \macro{\ifchilddoc}
% \macro{\ifchilddocmanual}
% The conditional |\ifchilddoc| tells whether a
% child (true) or main (false) document is being compiled.
% The conditional |\ifchilddocmanual| tells whether
% the |\includeonly| mechanism is used (false) or
% the selection of child files must be performed manually (true).
% The definitions initialise to false:
%    \begin{macrocode}
\newif\ifchilddoc
\newif\ifchilddocmanual
%    \end{macrocode}

% \macro{\childdocname}
% \macro{\childdocjob}
% The macro |\childdocname| stores the name of the main document
% to be compiled. The macro |\childdocjob| stores the name of
% the document on which the \LaTeX{} compiler was originally invoked.
% The content of |\jobname| cannot be compared
% to filenames specified in the source due to different catcodes.
% The following code rescans |\jobname|, stores the result
% in |\childdocname| and saves a copy in |\childdocjob|:
%    \begin{macrocode}
\edef\childdocname{\scantokens\expandafter{\jobname\noexpand}}
\let\childdocjob\childdocname
%    \end{macrocode}

% \macro{\childdocdisable}
% The macro |\childdocdisable| prevents the main file
% from being processed more than once.
% At this stage, the main document command |\childdocmain|
% is assumed to be called once again where it should do nothing.
% Any subsequent call to it should prevent
% a secondary processing of the main document
% It overwrites the forwarding commands
% |\childdocof| and |\childdocforward|
% with empty macros to prevent further inclusions of the main document:
%    \begin{macrocode}
\newcommand{\childdocdisable}
{
  \renewcommand{\childdocmain}[1]{\renewcommand{\childdocmain}[1]{\endinput}}
  \renewcommand{\childdocof}[1]{}
  \renewcommand{\childdocby}[2][]{}
  \renewcommand{\childdocforward}[2][]{}
  \renewcommand{\childdocdisable}{}
}
%    \end{macrocode}

% \macro{\childdocmain}
% The macro |\childdocmain| is to be called at the top of the main file
% with nothing or the main filename (without extension) as argument.
% First, it breaks loops.
% If the argument is not empty and does not match |\childdocname|
% (which is set by the first inclusion of |childdoc.def|),
% |\ifchilddoc| is set to true, |\includeonly| is applied to the child file
% and |\jobname| is set to the main file
% (for proper handling of |.aux| files):
%    \begin{macrocode}
\newcommand{\childdocmain}[1]
{
  \childdocdisable\childdocmain{}
  \if?#1?\else
    \begingroup
      \def\childdoctmp{#1}
      \ifx\childdoctmp\childdocname
        \def\childdoctmp{}
      \else
        \def\childdoctmp
        {
          \childdoctrue
          \includeonly{\childdocname}
          \def\childdocjob{#1}
          \def\jobname{#1}
        }
      \fi
      \expandafter
    \endgroup
    \childdoctmp
  \fi
}
%    \end{macrocode}

% \macro{\childdocof}
% The command |\childdocof| redirects
% compilation to the main file |#1|.
%    \begin{macrocode}
\newcommand{\childdocof}[1]
{
  \childdocdisable
  \childdoctrue
  \includeonly{\childdocname}
  \def\jobname{#1}
  \def\childdocjob{#1}
  \input{#1}
}
%    \end{macrocode}

% \macro{\childdocby}
% The command |\childdocby| ....
%    \begin{macrocode}
\newcommand{\childdocby}[2][]
{
  \childdocdisable
  \childdoctrue
  \childdocmanualtrue
  \if?#1?\else
    \def\jobname{#2}
  \fi
  \def\childdocjob{#2}
  \input{#2}
  \endinput
}
%    \end{macrocode}

% \macro{\childdocforward}
% The command |\childdocforward| redirects
% compilation to the main file or
% (if the optional argument is given) a child file.
% Parameters are set as if the main file
% or a child file starting with |\childdocof| was compiled.
% Then compilation is handed over to the main file:
%    \begin{macrocode}
\newcommand{\childdocforward}[2][]
{
  \begingroup
    \if?#1?
      \def\childdoctmp
      {
        \def\childdocname{#2}
        \def\childdocjob{#2}
        \def\jobname{#2}
        \input{#2}
        \endinput
      }
    \else
      \def\childdoctmp
      {
        \childdocdisable
        \def\childdocname{#2}
        \childdoctrue
        \includeonly{#2}
        \def\childdocjob{#1}
        \def\jobname{#1}
        \input{#1}
        \endinput
      }
    \fi
    \expandafter
  \endgroup
  \childdoctmp
}
%    \end{macrocode}

% \macro{\childdocforwardprefix}
% The command |\childdocforwardprefix| redirects
% compilation to the main or a child file by means of a pattern.
% The prefix |#1| in the current filename is replaced by |#2|
% and the suffix of the current filename is kept
% (it is assumed that the filename does not contain the substring `|~~~|'
% which is used as a delimiter).
% Compilation is handed over to the new file by |\childdocforward|:
%    \begin{macrocode}
\newcommand{\childdocforwardprefix}[3][]
{
  \begingroup
    \def\childdocextract #2##1~~~{\def\childdoctmp{\childdocforward[#1]{#3##1}}}
    \expandafter\childdocextract\childdocname~~~
    \expandafter
  \endgroup
  \childdoctmp
}
%    \end{macrocode}

% \macro{\childdoc}
% The deprecated macro |\childdoc| is a legacy version of |\childdocmain|:
%    \begin{macrocode}
\newcommand{\childdoc}{\childdocmain}
%    \end{macrocode}

% \macro{\childdocredirect}
% The deprecated macro |\childdocredirect| is a legacy version
% of |\childdocforward| and |\childdocforwardprefix|:
%    \begin{macrocode}
\newcommand{\childdocredirect}[2][]
{
  \begingroup
    \if?#1?
      \def\childdoctmp{\childdocforward{#2}}
    \else
      \def\childdoctmp{\childdocforwardprefix{#1}{#2}}
    \fi
    \expandafter
  \endgroup
  \childdoctmp
}
%    \end{macrocode}

%\iffalse
%</package>
%\fi
%
\endinput

\childdocmain{}
%    \end{macrocode}

% Optional override for |\version| flag:
%    \begin{macrocode}
%%\ifchilddoc\else\providecommand{\version}{draft}\fi
%    \end{macrocode}

% Define the default values for the |\version| flag
% (|final| for the main file and |draft| for childs):
%    \begin{macrocode}
\ifchilddoc
\providecommand{\version}{draft}
\else
\providecommand{\version}{final}
\fi
%    \end{macrocode}

% Load the standard document class:
%    \begin{macrocode}
\documentclass[12pt]{article}
%    \end{macrocode}

% Start the document body:
%    \begin{macrocode}
\begin{document}
%    \end{macrocode}

% Declare a title page.
% Print title, part of document being processed and version flag:
%    \begin{macrocode}
\addtocounter{page}{-1}
\begin{center}
{\LARGE\bfseries{}childdoc example\par}
\vspace{1cm}
\ifchilddoc
\ifchilddocmanual part\else chapter\fi:
`\childdocname' of `\childdocjob'\par
\else
main document: `\childdocjob'\par
\fi
version: \version\par
\end{center}
\newpage
%    \end{macrocode}

% Manually include selected file,
% otherwise process as usual:
%    \begin{macrocode}
\ifchilddocmanual
\section*{part `\childdocname'}
\input{\childdocname}
\else
%    \end{macrocode}

% Include the two chapters:
%    \begin{macrocode}
\include{cdocsch1}
\include{cdocsch2}
%    \end{macrocode}

% Include the two parts unless only chapters should be displayed:
%    \begin{macrocode}
\ifchilddoc\else
\section{part three}
\input{cdocspt3}
\section{part four}
\input{cdocspt4}
\fi
%    \end{macrocode}

% Process as usual until here:
%    \begin{macrocode}
\fi
%    \end{macrocode}

% End of document body:
%    \begin{macrocode}
\end{document}
%    \end{macrocode}
%\iffalse
%</samplemain>
%\fi
%
% %%%%%%%%%%%%%%%%%%%%%%%%%%%%%%%%%%%%%%
% \paragraph{Chapter Include Files.}
%
% The include files are called |cdocsch1.tex| and |cdocsch2.tex|.
%
%\iffalse
%<*samplechap1|samplechap2>
%\fi

% Optional override for |\version| flag:
%    \begin{macrocode}
%%\providecommand{\version}{final}
%    \end{macrocode}

% Include the main document:
%    \begin{macrocode}
% \iffalse
%
% childdoc.dtx Copyright (C) 2017-2018 Niklas Beisert
%
% This work may be distributed and/or modified under the
% conditions of the LaTeX Project Public License, either version 1.3
% of this license or (at your option) any later version.
% The latest version of this license is in
%   http://www.latex-project.org/lppl.txt
% and version 1.3 or later is part of all distributions of LaTeX
% version 2005/12/01 or later.
%
% This work has the LPPL maintenance status `maintained'.
%
% The Current Maintainer of this work is Niklas Beisert.
%
% This work consists of the files childdoc.dtx and childdoc.ins
% and the derived files childdoc.def and cdocsamp.tex with
% cdocsch1.tex, cdocsch2.tex, cdocsdrf.tex, cdocsfn1.tex, cdocsfn2.tex.
%
%<package>\ifdefined\childdocmain\endinput\fi
%<package>\ProvidesFile{childdoc.def}[2018/12/30 v2.0 child document driver]
%<samplemain>\ProvidesFile{cdocsamp.tex}[2018/12/30 v2.0 sample for childdoc]
%<*driver>
%\ProvidesFile{childdoc.drv}[2018/12/30 v2.0 childdoc reference manual file]
\PassOptionsToClass{10pt,a4paper}{article}
\documentclass{ltxdoc}

\usepackage[margin=35mm]{geometry}
\usepackage{hyperref}
\usepackage{hyperxmp}
\usepackage[usenames]{color}

\hypersetup{colorlinks=true}
\hypersetup{pdfstartview=FitH}
\hypersetup{pdfpagemode=UseNone}
\hypersetup{pdfsource={}}
\hypersetup{pdflang={en-UK}}
\hypersetup{pdfcopyright={Copyright 2017-2018 Niklas Beisert.
  This work may be distributed and/or modified under the
  conditions of the LaTeX Project Public License, either version 1.3
  of this license or (at your option) any later version.}}
\hypersetup{pdflicenseurl={http://www.latex-project.org/lppl.txt}}
\hypersetup{pdfcontactaddress={ETH Zurich, ITP, HIT K,
  Wolfgang-Pauli-Strasse 27}}
\hypersetup{pdfcontactpostcode={8093}}
\hypersetup{pdfcontactcity={Zurich}}
\hypersetup{pdfcontactcountry={Switzerland}}
\hypersetup{pdfcontactemail={nbeisert@itp.phys.ethz.ch}}
\hypersetup{pdfcontacturl={http://people.phys.ethz.ch/\xmptilde nbeisert/}}

\newcommand{\secref}[1]{\hyperref[#1]{section \ref*{#1}}}

\parskip1ex
\parindent0pt
\let\olditemize\itemize
\def\itemize{\olditemize\parskip0pt}

\begin{document}

\title{The \textsf{childdoc} Package}
\hypersetup{pdftitle={The childdoc Package}}
\author{Niklas Beisert\\[2ex]
  Institut f\"ur Theoretische Physik\\
  Eidgen\"ossische Technische Hochschule Z\"urich\\
  Wolfgang-Pauli-Strasse 27, 8093 Z\"urich, Switzerland\\[1ex]
  \href{mailto:nbeisert@itp.phys.ethz.ch}
  {\texttt{nbeisert@itp.phys.ethz.ch}}}
\hypersetup{pdfauthor={Niklas Beisert}}
\hypersetup{pdfsubject={Manual for the LaTeX2e Package childdoc}}
\date{30 December 2018, \textsf{v2.0}}
\maketitle

\begin{abstract}\noindent
\textsf{childdoc} is a \LaTeXe{} package
that enables the direct compilation
of document sections included by |\include|
to individual files.
\end{abstract}

\begingroup
\parskip0ex
\tableofcontents
\endgroup

%%%%%%%%%%%%%%%%%%%%%%%%%%%%%%%%%%%%%%%%%%%%%%%%%%%%%%%%%%%%%%%%%%%%%%%%%%%%%%%%
%%%%%%%%%%%%%%%%%%%%%%%%%%%%%%%%%%%%%%%%%%%%%%%%%%%%%%%%%%%%%%%%%%%%%%%%%%%%%%%%
\section{Introduction}

\LaTeX{} provides a mechanism to structure a large document (such as a book)
into a main file and several child files (containing the chapters)
using the |\include| command.
This mechanism is beneficial for documents
which span hundreds of pages in order to
make the source file(s) more manageable.
Moreover, compilation can be restricted to
selected child files by means of the |\includeonly| command.
The latter feature can be used to reduce the compilation time while editing
(this was significantly more useful in the earlier days of \LaTeX{})
or to generate a smaller document which is easier to navigate.
Another application of |\includeonly| is to generate
documents consisting of selected parts of the complete document.

However, there are a few drawbacks of the plain |\include| mechanism:
\begin{itemize}
\item
The child files cannot be compiled on their own,
they can only be compiled via the main file.
A naive editing environment
(such as a text editor with an option
to have the current file processed by \LaTeX)
may require one to switch to the main file before compiling;
attempting to compile the child file produces errors.
\item
The main file must be modified (each time)
to adjust the |\includeonly| command
to the present needs. This easily leaves the main file in a messy state.
\item
The generated document will always carry the filename
of the main document. This is inconvenient if
several child files are to be compiled and
to be kept for distribution.
\end{itemize}

The present package provides a simple interface
to make child files individually compilable by \LaTeX{}.
Compiling a child file then has the same effect as compiling
the main file with an |\includeonly| command
to select the appropriate child.
Moreover the generated document will carry the name of the child
rather than the main file.
This resolves all three above issues.

This feature is meant to make the editing of books,
thesis documents and lecture notes somewhat more convenient.
However, the package can also be used efficiently for
composing a series of documents (such as exercise sheets)
which are typically distributed individually.
It then assists the author in generating the individual documents
(potentially in different versions)
as well as a document containing the collected series.
Another application is in developing style files
or other kinds of included material
where compilation of the style file could redirect
to a sample or test file.

%%%%%%%%%%%%%%%%%%%%%%%%%%%%%%%%%%%%%%%%%%%%%%%%%%%%%%%%%%%%%%%%%%%%%%%%%%%%%%%%
%%%%%%%%%%%%%%%%%%%%%%%%%%%%%%%%%%%%%%%%%%%%%%%%%%%%%%%%%%%%%%%%%%%%%%%%%%%%%%%%
\section{Usage}

First of all, the package \textsf{childdoc} is \emph{not} a standard
\LaTeXe{} |.sty| style file! Therefore it needs to be invoked in
a non-standard way.

%%%%%%%%%%%%%%%%%%%%%%%%%%%%%%%%%%%%%%%%%%%%%%%%%%%%%%%%%%%%%%%%%%%%%%%%%%%%%%%%
\subsection{Included Files}
\label{sec:include}

%%%%%%%%%%%%%%%%%%%%%%%%%%%%%%%%%%%%%%%%
\DescribeMacro{\childdocmain}
To use the package, add the commands
\begin{center}
\begin{tabular}{l}
|\input{childdoc.def}|\\
|\childdocmain{}|\\
\end{tabular}
\end{center}
at the very top of the main \LaTeX{} file,
in particular \emph{before} the |\documentclass| statement!
The argument of |\childdocmain| should be left empty
(but it must be present).

%%%%%%%%%%%%%%%%%%%%%%%%%%%%%%%%%%%%%%%%
\DescribeMacro{\childdocof}
Furthermore, add the commands
\begin{center}
\begin{tabular}{l}
|\input{childdoc.def}|\\
|\childdocof{|\textit{main}|}|\\
\end{tabular}
\end{center}
at the top of every child file \textit{child}
which is included by |\include{|\textit{child}|}|
from within the main file
(or at least for those files to be compiled individually).
The argument \textit{main} must be the filename of the main file.

There are a couple of
considerations in setting up the main and child documents:

%%%%%%%%%%%%%%%%%%%%%%%%%%%%%%%%%%%%%%%%
\paragraph{Restrictions.}

Please note the following restrictions:
\begin{itemize}
\item
|\childdocmain| must be called with one argument \textit{main}
to ensure compatibility with earlier version of the package.
It must either be empty (|\childdocmain{}|)
or precisely match the filename of the main file in which it is specified.
See \secref{sec:detection} for further information.
\item
The filename \textit{main} must be specified without the |.tex| extension.
\item
The filename \textit{main} is case sensitive
(even in case-insensitive file systems)
due to internal string comparison.
\item
The argument \textit{main} should be fully expanded, it cannot be a macro.
\item
Subdirectories and special characters should be avoided in filenames.
\item
The command |\childdocmain{|\textit{main}|}| must be followed by a whitespace.
It should not be followed immediately by another command
or by a comment mark `|%|'.
This is because the \TeX{} parser reads the token immediately following
the argument of |\childdocmain| and puts it
at the beginning of every child section;
however, a white\-space is ignored.
\end{itemize}

%%%%%%%%%%%%%%%%%%%%%%%%%%%%%%%%%%%%%%%%
\paragraph{Content of Main File.}

It is advisable to place all content in the child files included by |\include|.
Any output contained in the main file will appear in all child documents
unless suppressed manually;
it cannot be suppressed automatically by the |\includeonly| directive
and thus should normally be avoided.
A method to include some content in the main file
by means of conditional processing is described in \secref{sec:conditional}.

%%%%%%%%%%%%%%%%%%%%%%%%%%%%%%%%%%%%%%%%
\paragraph{Page Numbering.}

When only a part of the document is compiled,
the appropriate numbering of pages
(as well as other status parameters)
is determined from the |.aux| files.
The latter contain information from previous passes.
However this information needs to propagate through
all intermediate child documents.
Therefore the page numbering in child documents may well
be inconsistent until the complete document is compiled at least once.

A useful (if unconventional) way to always ensure a consistent
page numbering is to restart the numbering in each child document
and denote the pages by `\textit{child}|.|\textit{page}'
where \textit{child} represents the chapter/section number of the child file.
This can be achieved by the command
|\numberwithin{page}{|\textit{child}|}|
of the \textsf{amsmath} package
where \textit{child} can be |chapter| or |section|
depending on the chosen structuring.
Alternatively, one can modify the macro |\thepage| appropriately
and reset the counter |page| at the start of each child file.

%%%%%%%%%%%%%%%%%%%%%%%%%%%%%%%%%%%%%%%%%%%%%%%%%%%%%%%%%%%%%%%%%%%%%%%%%%%%%%%%
\subsection{Conditional Processing}
\label{sec:conditional}

The package provides a mechanism to compile different versions
of a document. To customise the versions further some conditional processing
can come in handy to distinguish which version is being compiled.
The package provides two macros to describe the compilation context:

%%%%%%%%%%%%%%%%%%%%%%%%%%%%%%%%%%%%%%%%
\DescribeMacro{\ifchilddoc}
The conditional |\ifchilddoc| distinguishes between the compilation of
child documents and the main document:
%
\begin{center}
|\ifchilddoc |\textit{child-code}| |[|\||else |\textit{main-code}]| \||fi|
\end{center}

%%%%%%%%%%%%%%%%%%%%%%%%%%%%%%%%%%%%%%%%
\DescribeMacro{\childdocname}
\DescribeMacro{\childdocjob}
The macro |\childdocname| contains the filename (without extension)
of the main or child file being processed.
Note that |\childdocjob| will always contain the name of the main file.

%%%%%%%%%%%%%%%%%%%%%%%%%%%%%%%%%%%%%%%%
\paragraph{Title Page.}

Conditional processing can be used to include a title or banner page
in the main document when proper precautions are taken.
Importantly, the code in the main file should ensure that the page counter
(as well as other status parameters which are stored in the |.aux| files)
takes the same value after the conditional processing.
Otherwise the page numbers may take divergent values
depending on which part is compiled.

For example, a title page could be declared by:
%
\begin{center}
\begin{tabular}{l}
|\ifchilddoc\||else|\\
|\addtocounter{page}{-1}|\\
\textit{code for title page}\\
|\newpage|\\
|\||fi|
\end{tabular}
\end{center}
%
A banner page for the child documents can be generated by:
%
\begin{center}
\begin{tabular}{l}
|\ifchilddoc|\\
|\addtocounter{page}{-1}|\\
\textit{code for banner page}\\
|\newpage|\\
|\||fi|
\end{tabular}
\end{center}
%
Here one could write a message such as:
\begin{center}
|This is the part \childdocname{} of \childdocjob{}.|
\end{center}

%%%%%%%%%%%%%%%%%%%%%%%%%%%%%%%%%%%%%%%%%%%%%%%%%%%%%%%%%%%%%%%%%%%%%%%%%%%%%%%%
\subsection{Flags}
\label{sec:flags}

The package makes it easy to generate different versions
of the main or child documents.
To this end compilation flags can be defined
and assigned different default values.
They will be particularly useful in conjunction
with the forwarding mechanism described in \secref{sec:forward}.

For example, it may be useful to have a flag |\version|
which can be set to |draft| or |final|.
The document source will contain some conditional code
depending on the value of |\version|.
Suppose further, the flag should default to |final| for the main file
and to |draft| for child files
which is a natural assignment for editing the document.
This is achieved by placing the following code
in the preamble of the main document
(below the |\childdocmain| directive):
%
\begin{center}
\begin{tabular}{l}
|\ifchilddoc|\\
|\providecommand{\version}{draft}|\\
|\||else|\\
|\providecommand{\version}{final}|\\
|\||fi|
\end{tabular}
\end{center}
%
The definition by |\providecommand| makes sure
that previous definitions are not overwritten.
Further statements |\providecommand{\version}{...}|
can thus be added before the above code to override it.

For the main file, one might add a line
(between |\childdocmain| and the above block)
%
\begin{center}
|%\ifchilddoc\||else\providecommand{\version}{draft}\||fi|
\end{center}
%
which can be uncommented to produce a draft version.
Likewise one can add a line to the very top of a child file
(above the |\childdocof{|\textit{main}|}| directive)
%
\begin{center}
|%\providecommand{\version}{final}|
\end{center}
%
which can be uncommented to produce the final version of this child document.

%%%%%%%%%%%%%%%%%%%%%%%%%%%%%%%%%%%%%%%%%%%%%%%%%%%%%%%%%%%%%%%%%%%%%%%%%%%%%%%%
\subsection{Forwarding}
\label{sec:forward}

Different versions of the main or child documents
using compilation flags as described in \secref{sec:flags}
can be (permanently) stored in different files
for convenient compilation, viewing and distribution.
To this end, the package defines a command
to pass on compilation to a different file:

%%%%%%%%%%%%%%%%%%%%%%%%%%%%%%%%%%%%%%%%
\DescribeMacro{\childdocforward}
The command |\childdocforward| redirects processing to
another source file:
%
\begin{center}
\begin{tabular}{l}
|\input{childdoc.def}|\\
|\childdocforward[|\textit{main}|]{|\textit{dest}|}|\\
\end{tabular}
\end{center}
%
The argument \textit{dest} is the destination file
(without extension).
It should be the main file or one of the child files.
Note that further \textsf{childdoc} directives
such as |\childdocof| and |\childdocforward|
in the indicated file will be processed in this form.
The optional argument \textit{main}
passes on directly to the main file \textit{main}
while pretending to compile the child \textit{dest}.
This form behaves as if \textit{dest}
issues |\childdocof{|\textit{main}|}| right away,
and no further \textsf{childdoc} directives will be processed.

%%%%%%%%%%%%%%%%%%%%%%%%%%%%%%%%%%%%%%%%
\DescribeMacro{\...prefix}
In the alternative form |\childdocforwardprefix|,
%
\begin{center}
\begin{tabular}{l}
|\input{childdoc.def}|\\
|\childdocforwardprefix[|\textit{main}|]{|\textit{prefix}|}{|\textit{dest}|}|
\end{tabular}
\end{center}
%
the destination file is determined by a pattern
depending on the current file:
To make this work, the current file must be called
`{\textit{prefix}\hspace{0.2em}\textit{suffix}}'
with \textit{prefix} matching precisely the argument.
Processing is then passed on to the file
`{\textit{dest}\hspace{0.2em}\textit{suffix}}'.
Surely, the same effect is achieved by
directly specifying the
argument `{\textit{dest}\hspace{0.2em}\textit{suffix}}'
in the first form.
However, that requires to set up a different file
for each child. With the alternative form of the command
all these files can have exactly the same content
which simplifies setting them up and maintaining them.

For example, the following file |draft.tex|
with a compilation flag |\version| as described in \secref{sec:flags}
compiles the main document as a draft:
%
\begin{center}
\begin{tabular}{l}
|\def\version{draft}|\\
|\input{childdoc.def}|\\
|\childdocforward{|\textit{main}|}|
\end{tabular}
\end{center}
%
Likewise, the following files |final|\textit{nn}|.tex|
compile the final version of the child document
|child|\textit{nn}|.tex|:
%
\begin{center}
\begin{tabular}{l}
|\def\version{final}|\\
|\input{childdoc.def}|\\
|\childdocforwardprefix{final}{child}|
\end{tabular}
\end{center}
%

Note that when several versions of a main file and/or of each child file
are to be generated, it may be convenient to set up a |Makefile| or
shell script to automatise the process.

%%%%%%%%%%%%%%%%%%%%%%%%%%%%%%%%%%%%%%%%%%%%%%%%%%%%%%%%%%%%%%%%%%%%%%%%%%%%%%%%
\subsection{Command Line Processing}
\label{sec:commandline}

The effect of redirection files can also be achieved by invoking
the \LaTeX{} compiler with a more elaborate command line.
Most conveniently this should be done as part
of a shell script or a |Makefile|.

When using \textsf{childdoc} in the main file, the following
command lines effectively perform a redirection
(note that depending on the shell being used,
backslashes may have to be doubled: `|\|' $\to$ `|\\|'):
%
\begin{center}
|... -jobname "|\textit{target}|" |\\|"|[\textit{flags}]%
|\input{childdoc.def}\childdocforward[|\textit{main}|]{|\textit{dest}|}"|
\end{center}
%
Here \textit{target} is the name of the output file,
\textit{main} is the name of the main file
and \textit{dest} is the name of the main or child file to be processed
(all filenames without extensions).
The optional argument \textit{main} can be omitted
if \textit{main} matches \textit{dest}.
Optionally, compilation \textit{flags} can be defined via |\def| commands.
This command line makes the \TeX{} engine believe
it is compiling the file \textit{target}
whose content is specified as the latter parameter.
The provided code then forwards the processing to
\textit{main} or \textit{dest} as described in \secref{sec:forward}.

%%%%%%%%%%%%%%%%%%%%%%%%%%%%%%%%%%%%%%%%%%%%%%%%%%%%%%%%%%%%%%%%%%%%%%%%%%%%%%%%
\subsection{Include by Input}
\label{sec:input}

Including child documents by |\include| has some restrictions by design.
Most notably, the content of a child document always occupies
its own set of pages; pages cannot be shared between child documents.
Usually, this behaviour makes perfect sense
because each child document contain an essential part of the document.
However, in some situations it may be desirable to compose
a document from a collection of parts
without having mandatory page breaks between then.
For this case, the package
provides a mechanism to include parts
by |\input| which can also be processed individually.
However, by construction this mechanism
requires manual handling of the content to be output.

%%%%%%%%%%%%%%%%%%%%%%%%%%%%%%%%%%%%%%%%
\DescribeMacro{\ifchilddocmanual}
The main file should be prepared as usual, see \secref{sec:include}.
However, the document body must make a distinction
between processing of an individual part and of the main document, e.g.:
%
\begin{center}
\begin{tabular}{l}
|\ifchilddocmanual|\\
|\input{\childdocname}|\\
|\||else|\\
\textit{document body with }|\input{|\textit{part}|}|\\
|\||fi|
\end{tabular}
\end{center}
%
The conditional |\ifchilddocmanual| is true whenever
a part to be included by |\input| is being compiled,
and the name of the part is stored in |\childdocname|.

%%%%%%%%%%%%%%%%%%%%%%%%%%%%%%%%%%%%%%%%
\DescribeMacro{\childdocby}
Each part to be included by |\input| should start with:
%
\begin{center}
\begin{tabular}{l}
|\input{childdoc.def}|\\
|\childdocby{|\textit{main}|}|\\
\end{tabular}
\end{center}
%
The directive |\childdocby| is similar to |\childdocof|
described in \secref{sec:include},
but the subsequent selection of content must be done manually.
To that end, both |\ifchilddoc| and |\ifchilddocmanual|
will be true upon processing of a part,
and the name of the part is stored in |\childdocname|.
Note that |\jobname| will be set to the filename of the current part
so that each part receives an individual |.aux| file
that does not interfere with the |.aux| file(s) of the main document.
This behaviour can be altered by the alternative form
|\childdocby[*]{|\textit{main}|}| (with a non-empty optional argument)
which uses the |.aux| file of the main document
by setting |\jobname| to \textit{main}.

%%%%%%%%%%%%%%%%%%%%%%%%%%%%%%%%%%%%%%%%%%%%%%%%%%%%%%%%%%%%%%%%%%%%%%%%%%%%%%%%
\subsection{Driver Development}
\label{sec:driver}

The \textsf{childdoc} mechanism can also be use for the development
of definition files such as \LaTeX{} styles or classes.
This case differs from the above setup with multiple parts
included by |\include| in that no |\includeonly| should be invoked.
This can be achieved by starting the include file
(before |\ProvidesPackage|) with:
%
\begin{center}
\begin{tabular}{l}
|\input{childdoc.def}|\\
|\childdocforward{|\textit{main}|}|\\
\end{tabular}
\end{center}
%
or alternatively with:
%
\begin{center}
\begin{tabular}{l}
|\input{childdoc.def}|\\
|\childdocby{|\textit{main}|}|\\
\end{tabular}
\end{center}
%
Both forms have slightly different effects as described above.
The main file is prepared as usual, see \secref{sec:include}.

%%%%%%%%%%%%%%%%%%%%%%%%%%%%%%%%%%%%%%%%%%%%%%%%%%%%%%%%%%%%%%%%%%%%%%%%%%%%%%%%
\subsection{Legacy Detection}
\label{sec:detection}

The directive |\childdocmain| in the main file can detect
whether the complete document or merely a child is to be compiled
even without using the directive |\childdocof|.
This method is deprecated because it is less robust
and there is no compelling reason to use it;
it is merely provided for backward compatibility
and it may be removed in future versions.

If the detection mechanism is to be used,
it is mandatory to correctly specify
the filename of the main file as the argument of |\childdocmain|:
%
\begin{center}
\begin{tabular}{l}
|\input{childdoc.def}|\\
|\childdocmain{|\textit{main}|}|\\
\end{tabular}
\end{center}
%
If |\jobname| does not match the argument \textit{main} of |\childdocmain|,
it is assumed that |\jobname| points to the child file to be compiled.
When using |\childdocmain| with the main file specified as argument,
it suffices to start a child file
with just |\input{|\textit{main}|}|
without loading of the package and using |\childdocof|.
If instead all processing is done
with the appropriate \textsf{childdoc} directives,
the argument of \textit{main} of |\childdocmain| can be empty.

An alternative version of the command line processing described
in \secref{sec:commandline} using the detection mechanism reads:
%
\begin{center}
|... -jobname "|\textit{target}|" "|[\textit{flags}]%
[|\def\jobname{|\textit{dest}|}|]|\input{|\textit{main}|}"|
\end{center}

%%%%%%%%%%%%%%%%%%%%%%%%%%%%%%%%%%%%%%%%%%%%%%%%%%%%%%%%%%%%%%%%%%%%%%%%%%%%%%%%
\subsection{Manual Code}
\label{sec:manual}

In case one cannot be certain whether the definitions file |childdoc.def|
is installed on the target \TeX{} distribution
and one prefers not to ship it,
it is conceivable to paste a few relevant commands into the sources.

To that end, drop all statements |\input{childdoc.def}|
and perform the replacements as outlined below.
Instead of |\childdocmain{|\textit{main}|}| add the following code
to the top of the main file:
%
\begin{center}
\begin{tabular}{l}
|\||ifdefined\childdocname\endinput\||fi\newif\ifchilddoc|\\
|\edef\childdocname{\scantokens\expandafter{\jobname\noexpand}}|\\
|\def\childdocmain{|\textit{main}|}\||ifx\childdocmain\childdocname\||else|\\
|\childdoctrue\includeonly{\childdocname}\let\jobname\childdocmain\||fi|\\
\end{tabular}
\end{center}
%
Instead of |\childdocof{|\textit{main}|}| just include the main file
at the top of each child file:
%
\begin{center}
|\input{|\textit{main}|}|
\end{center}
%
A simple redirection |\childdocforward{|\textit{dest}|}| is achieved by:
%
\begin{center}
|\def\jobname{|\textit{dest}|}\input{\jobname}|
\end{center}
%
The redirection with prefix
|\childdocforwardprefix[|\textit{prefix}|]{|\textit{dest}|}|
is accomplished by:
%
\begin{center}
\begin{tabular}{l}
|{\edef\jobname{\scantokens\expandafter{\jobname\noexpand}}|\\
|\def\redirectjob |\textit{prefix}|#1~~~{\gdef\jobname{|\textit{dest}|#1}}|\\
|\expandafter\redirectjob\jobname~~~}\input{\jobname}|
\end{tabular}
\end{center}

In an alternative approach,
child documents can be compiled by a specific command line
without additional code or specific definitions:
%
\begin{center}
|... -jobname "|\textit{target}|" "|[\textit{flags}]%
|\includeonly{|\textit{dest}|}\input{|\textit{main}|}"|
\end{center}
%

%%%%%%%%%%%%%%%%%%%%%%%%%%%%%%%%%%%%%%%%%%%%%%%%%%%%%%%%%%%%%%%%%%%%%%%%%%%%%%%%
%%%%%%%%%%%%%%%%%%%%%%%%%%%%%%%%%%%%%%%%%%%%%%%%%%%%%%%%%%%%%%%%%%%%%%%%%%%%%%%%
\section{Information}

%%%%%%%%%%%%%%%%%%%%%%%%%%%%%%%%%%%%%%%%%%%%%%%%%%%%%%%%%%%%%%%%%%%%%%%%%%%%%%%%
\subsection{Copyright}

Copyright \copyright{} 2017--2018 Niklas Beisert

This work may be distributed and/or modified under the
conditions of the \LaTeX{} Project Public License, either version 1.3
of this license or (at your option) any later version.
The latest version of this license is in
  \url{http://www.latex-project.org/lppl.txt}
and version 1.3 or later is part of all distributions of \LaTeX{}
version 2005/12/01 or later.

This work has the LPPL maintenance status `maintained'.

The Current Maintainer of this work is Niklas Beisert.

This work consists of the files |README.txt|, |childdoc.ins| and |childdoc.dtx|
as well as the derived files |childdoc.def|, |cdocsamp.tex|
with |cdocsch1.tex|, |cdocsch2.tex|, |cdocspt3.tex|, |cdocspt4.tex|,
|cdocsdrf.tex|, |cdocsfn1.tex|, |cdocsfn2.tex|
as well as |childdoc.pdf|.

%%%%%%%%%%%%%%%%%%%%%%%%%%%%%%%%%%%%%%%%%%%%%%%%%%%%%%%%%%%%%%%%%%%%%%%%%%%%%%%%
\subsection{Files and Installation}

The package consists of the files:
%
\begin{center}
\begin{tabular}{ll}
    |README.txt|   & readme file \\
    |childdoc.ins| & installation file \\
    |childdoc.dtx| & source file \\
    |childdoc.def| & definition file \\
    |cdocsamp.tex| & sample main file \\
    |cdocsch1.tex| & sample include file \\
    |cdocsch2.tex| & sample include file \\
    |cdocspt3.tex| & sample part file \\
    |cdocspt4.tex| & sample part file \\
    |cdocsdrf.tex| & sample redirection file \\
    |cdocsfn1.tex| & sample redirection file \\
    |cdocsfn2.tex| & sample redirection file \\
    |childdoc.pdf| & manual
\end{tabular}
\end{center}
%
The distribution consists of the files
|README.txt|, |childdoc.ins| and |childdoc.dtx|.
%
\begin{itemize}
\item
Run (pdf)\LaTeX{} on |childdoc.dtx|
to compile the manual |childdoc.pdf| (this file).
\item
Run \LaTeX{} on |childdoc.ins| to create the definitions file |childdoc.def|
and the sample |cdocsamp.tex| with include files
|cdocsch1.tex|, |cdocsch2.tex|, |cdocspt3.tex|, |cdocspt4.tex|,
|cdocsdrf.tex|, |cdocsfn1.tex|, |cdocsfn2.tex|.
Then copy the file |childdoc.def| to an appropriate directory of your \LaTeX{}
distribution, e.g.\ \textit{texmf-root}|/tex/latex/childdoc|.
\end{itemize}

%%%%%%%%%%%%%%%%%%%%%%%%%%%%%%%%%%%%%%%%%%%%%%%%%%%%%%%%%%%%%%%%%%%%%%%%%%%%%%%%
\subsection{Related CTAN Packages}

There are several other packages which offer a similar functionality:
%
\begin{itemize}
\item
The packages
\href{http://ctan.org/pkg/docmute}{\textsf{docmute}},
\href{http://ctan.org/pkg/includex}{\textsf{includex}} and
\href{http://ctan.org/pkg/standalone}{\textsf{standalone}}
provide commands to include only the document body of
a child file thus allowing both files to be compiled individually.
\item
The packages \href{http://ctan.org/pkg/subdocs}{\textsf{subdocs}}
and \href{http://ctan.org/pkg/subfiles}{\textsf{subfiles}}
provide structures in which the main and child documents can be
encapsulated and allowing them to be compiled individually.
The inclusion mechanism is different from the conventional |\include|.
\item
The package \href{http://ctan.org/pkg/combine}{\textsf{combine}}
is an elaborate solution to combine several documents into one.
\end{itemize}
%
See also the CTAN topic \href{http://ctan.org/topic/subdocs}{\textsf{subdocs}}
for further related packages.
The present package differs from the above solutions in that
a document structure constructed with the conventional |\include| mechanism
just needs two extra commands at the top of every file
such that all constituent files can be compiled individually.

%%%%%%%%%%%%%%%%%%%%%%%%%%%%%%%%%%%%%%%%%%%%%%%%%%%%%%%%%%%%%%%%%%%%%%%%%%%%%%%%
%\subsection{Feature Suggestions}
%
%The following is a list of features which may be useful for future
%versions of this package:
%%
%\begin{itemize}
%\item
%\ldots
%\end{itemize}

%%%%%%%%%%%%%%%%%%%%%%%%%%%%%%%%%%%%%%%%%%%%%%%%%%%%%%%%%%%%%%%%%%%%%%%%%%%%%%%%
\subsection{Revision History}

%%%%%%%%%%%%%%%%%%%%%%%%%%%%%%%%%%%%%%%%
\paragraph{v2.0:} 2018/12/30

\begin{itemize}
\item
immediate forward processing
\item
added |\childdocby| mechanism
\item
manual restructured
\end{itemize}

%%%%%%%%%%%%%%%%%%%%%%%%%%%%%%%%%%%%%%%%
\paragraph{v1.6:} 2018/01/17

\begin{itemize}
\item
application for development of include files
\item
corrections to manual
\end{itemize}

%%%%%%%%%%%%%%%%%%%%%%%%%%%%%%%%%%%%%%%%
\paragraph{v1.5:} 2017/05/21

\begin{itemize}
\item
more complete structuring introduced
\item
|\childdocof| introduced
\item
|\childdoc| renamed to |\childdocmain|
\item
|\childredirect| renamed to |\childdocforward| and |\childdocforwardprefix|
and functionality expanded
\end{itemize}

%%%%%%%%%%%%%%%%%%%%%%%%%%%%%%%%%%%%%%%%
\paragraph{v1.0:} 2017/04/27

\begin{itemize}
\item
manual and install package
\item
first version published on CTAN
\end{itemize}

%%%%%%%%%%%%%%%%%%%%%%%%%%%%%%%%%%%%%%%%
\paragraph{v0.6:} 2017/04/26

\begin{itemize}
\item
redirection mechanism added
\end{itemize}

%%%%%%%%%%%%%%%%%%%%%%%%%%%%%%%%%%%%%%%%
\paragraph{v0.5:} 2017/04/26

\begin{itemize}
\item
functionality in definition file
\end{itemize}


%%%%%%%%%%%%%%%%%%%%%%%%%%%%%%%%%%%%%%%%%%%%%%%%%%%%%%%%%%%%%%%%%%%%%%%%%%%%%%%%
%%%%%%%%%%%%%%%%%%%%%%%%%%%%%%%%%%%%%%%%%%%%%%%%%%%%%%%%%%%%%%%%%%%%%%%%%%%%%%%%
%%%%%%%%%%%%%%%%%%%%%%%%%%%%%%%%%%%%%%%%%%%%%%%%%%%%%%%%%%%%%%%%%%%%%%%%%%%%%%%%
\appendix

\settowidth\MacroIndent{\rmfamily\scriptsize 000\ }

 \DocInput{childdoc.dtx}

\end{document}
%</driver>
% \fi
%
% %%%%%%%%%%%%%%%%%%%%%%%%%%%%%%%%%%%%%%%%%%%%%%%%%%%%%%%%%%%%%%%%%%%%%%%%%%%%%%
% %%%%%%%%%%%%%%%%%%%%%%%%%%%%%%%%%%%%%%%%%%%%%%%%%%%%%%%%%%%%%%%%%%%%%%%%%%%%%%
% \section{Sample}
%\iffalse
%<*samplemain>
%\fi
%
% The following presents a sample document
% with two chapters, two parts, a title page,
% a compile flag as well as three forwarding files to set the flag.
% It consists of eight |.tex| files:
% \begin{center}
% \begin{tabular}{ll}
% |cdocsamp.tex|&main file\\
% |cdocsch1.tex|&include file for chapter 1\\
% |cdocsch2.tex|&include file for chapter 2\\
% |cdocspt3.tex|&include file for part 3\\
% |cdocspt4.tex|&include file for part 4\\
% |cdocsdrf.tex|&forwarding file for main file in draft mode\\
% |cdocsfi1.tex|&forwarding file for final version of chapter 1\\
% |cdocsfi2.tex|&forwarding file for final version of chapter 2\\
% \end{tabular}
% \end{center}
% Each of the eight files can be compiled directly by the \LaTeX{} compiler.
%
% %%%%%%%%%%%%%%%%%%%%%%%%%%%%%%%%%%%%%%
% \paragraph{Main File.}
%
% The main file is called |cdocsamp.tex|.
%
% Load the \textsf{childdoc} definitions and
% declare the filename for the main document:
%    \begin{macrocode}
\input{childdoc.def}
\childdocmain{}
%    \end{macrocode}

% Optional override for |\version| flag:
%    \begin{macrocode}
%%\ifchilddoc\else\providecommand{\version}{draft}\fi
%    \end{macrocode}

% Define the default values for the |\version| flag
% (|final| for the main file and |draft| for childs):
%    \begin{macrocode}
\ifchilddoc
\providecommand{\version}{draft}
\else
\providecommand{\version}{final}
\fi
%    \end{macrocode}

% Load the standard document class:
%    \begin{macrocode}
\documentclass[12pt]{article}
%    \end{macrocode}

% Start the document body:
%    \begin{macrocode}
\begin{document}
%    \end{macrocode}

% Declare a title page.
% Print title, part of document being processed and version flag:
%    \begin{macrocode}
\addtocounter{page}{-1}
\begin{center}
{\LARGE\bfseries{}childdoc example\par}
\vspace{1cm}
\ifchilddoc
\ifchilddocmanual part\else chapter\fi:
`\childdocname' of `\childdocjob'\par
\else
main document: `\childdocjob'\par
\fi
version: \version\par
\end{center}
\newpage
%    \end{macrocode}

% Manually include selected file,
% otherwise process as usual:
%    \begin{macrocode}
\ifchilddocmanual
\section*{part `\childdocname'}
\input{\childdocname}
\else
%    \end{macrocode}

% Include the two chapters:
%    \begin{macrocode}
\include{cdocsch1}
\include{cdocsch2}
%    \end{macrocode}

% Include the two parts unless only chapters should be displayed:
%    \begin{macrocode}
\ifchilddoc\else
\section{part three}
\input{cdocspt3}
\section{part four}
\input{cdocspt4}
\fi
%    \end{macrocode}

% Process as usual until here:
%    \begin{macrocode}
\fi
%    \end{macrocode}

% End of document body:
%    \begin{macrocode}
\end{document}
%    \end{macrocode}
%\iffalse
%</samplemain>
%\fi
%
% %%%%%%%%%%%%%%%%%%%%%%%%%%%%%%%%%%%%%%
% \paragraph{Chapter Include Files.}
%
% The include files are called |cdocsch1.tex| and |cdocsch2.tex|.
%
%\iffalse
%<*samplechap1|samplechap2>
%\fi

% Optional override for |\version| flag:
%    \begin{macrocode}
%%\providecommand{\version}{final}
%    \end{macrocode}

% Include the main document:
%    \begin{macrocode}
\input{childdoc.def}
\childdocof{cdocsamp}
%    \end{macrocode}

%\iffalse
%</samplechap1|samplechap2>
%\fi
%
%\iffalse
%<*samplechap1>
%\fi
% Some text for chapter 1:
%    \begin{macrocode}
\section{one}
some text in chapter one
%    \end{macrocode}

%\iffalse
%</samplechap1>
%\fi
% Some text for chapter 2:
%\iffalse
%<*samplechap2>
%\fi
%    \begin{macrocode}
\section{two}
more text in chapter two
%    \end{macrocode}

%\iffalse
%</samplechap2>
%\fi
%
% %%%%%%%%%%%%%%%%%%%%%%%%%%%%%%%%%%%%%%
% \paragraph{Part Include Files.}
%
% The include files are called |cdocspt3.tex| and |cdocspt4.tex|.
%
%\iffalse
%<*samplepart3|samplepart4>
%\fi

% Optional override for |\version| flag:
%    \begin{macrocode}
%%\providecommand{\version}{final}
%    \end{macrocode}

% Include the main document:
%    \begin{macrocode}
\input{childdoc.def}
\childdocby{cdocsamp}
%    \end{macrocode}

%\iffalse
%</samplepart3|samplepart4>
%\fi
%
%\iffalse
%<*samplepart3>
%\fi
% Some text for part 3:
%    \begin{macrocode}
some text in part three
%    \end{macrocode}

%\iffalse
%</samplepart3>
%\fi
% Some text for part 4:
%\iffalse
%<*samplepart4>
%\fi
%    \begin{macrocode}
more text in part four
%    \end{macrocode}

%\iffalse
%</samplepart4>
%\fi
%
% %%%%%%%%%%%%%%%%%%%%%%%%%%%%%%%%%%%%%%
% \paragraph{Forwarding for a Complete Draft.}
%
% The following forwarding file |cdocsdrf.tex|
% compiles the main document in draft mode:
%\iffalse
%<*sampledraft>
%\fi
%    \begin{macrocode}
\def\version{draft}
\input{childdoc.def}
\childdocforward{cdocsamp}
%    \end{macrocode}

%\iffalse
%</sampledraft>
%\fi
%
% %%%%%%%%%%%%%%%%%%%%%%%%%%%%%%%%%%%%%%
% \paragraph{Forwarding for Final Version of the Chapters.}
%
% The following forwarding files |cdocsfn1.tex| and |cdocsfn2.tex|
% (with identical content)
% compile the final versions of the child documents
% |cdocsch1.tex| and |cdocsch2.tex|, respectively:
%\iffalse
%<*samplefinal>
%\fi
%    \begin{macrocode}
\def\version{final}
\input{childdoc.def}
\childdocforwardprefix[cdocsamp]{cdocsfn}{cdocsch}
%    \end{macrocode}

%\iffalse
%</samplefinal>
%\fi
%
% %%%%%%%%%%%%%%%%%%%%%%%%%%%%%%%%%%%%%%
% \paragraph{Command Line Processing.}
%
% The following three command lines generate the output files
% |cdocscld|, |cdocscl1| and |cdocscl2|
% which should be identical to
% |cdocsdrf|, |cdocsch1| and |cdocsfn2|, respectively:
% \begin{center}
% \begin{tabular}{l}
% |latex -jobname cdocscld \|\\
% |  "\def\version{draft}\input{childdoc.def}\childdocforward{cdocsamp}"|\\
% |latex -jobname cdocscl1 \|\\
% |  "\input{childdoc.def}\childdocforward[cdocsamp]{cdocsch1}"|\\
% |latex -jobname cdocscl2 \|\\
% |  "\def\version{final}\input{childdoc.def}\childdocforward{cdocsch2}"|
% \end{tabular}
% \end{center}
% Note that the trailing backslash on each first line
% merely continues the input to the second line
% (for convenient cut ant paste).
% Furthermore, the command |latex| can be replaced by any
% of its alternative versions such as |pdflatex|.
%
% %%%%%%%%%%%%%%%%%%%%%%%%%%%%%%%%%%%%%%%%%%%%%%%%%%%%%%%%%%%%%%%%%%%%%%%%%%%%%%
% %%%%%%%%%%%%%%%%%%%%%%%%%%%%%%%%%%%%%%%%%%%%%%%%%%%%%%%%%%%%%%%%%%%%%%%%%%%%%%
% \section{Implementation}
%\iffalse
%<*package>
%\fi
%
% This section describes the definitions file |childdoc.def|.

% The definitions cannot be loaded using |\usepackage| or |\RequirePackage|
% which has a mechanism to prevent loading a style file more than once.
% When loading the definitions by means of |\input|
% multiple instances have to be prevented manually:
%\iffalse
%This code needs to be before the `\ProvidesFile' directive
%which is defined at the beginning of this file.
%Therefore it is also placed there and commented out here.
%</package>
%<*discard>
%\fi
%    \begin{macrocode}
\ifdefined\childdocmain\endinput\fi
%    \end{macrocode}
%\iffalse
%</discard>
%<*package>
%\fi
%
% \macro{\ifchilddoc}
% \macro{\ifchilddocmanual}
% The conditional |\ifchilddoc| tells whether a
% child (true) or main (false) document is being compiled.
% The conditional |\ifchilddocmanual| tells whether
% the |\includeonly| mechanism is used (false) or
% the selection of child files must be performed manually (true).
% The definitions initialise to false:
%    \begin{macrocode}
\newif\ifchilddoc
\newif\ifchilddocmanual
%    \end{macrocode}

% \macro{\childdocname}
% \macro{\childdocjob}
% The macro |\childdocname| stores the name of the main document
% to be compiled. The macro |\childdocjob| stores the name of
% the document on which the \LaTeX{} compiler was originally invoked.
% The content of |\jobname| cannot be compared
% to filenames specified in the source due to different catcodes.
% The following code rescans |\jobname|, stores the result
% in |\childdocname| and saves a copy in |\childdocjob|:
%    \begin{macrocode}
\edef\childdocname{\scantokens\expandafter{\jobname\noexpand}}
\let\childdocjob\childdocname
%    \end{macrocode}

% \macro{\childdocdisable}
% The macro |\childdocdisable| prevents the main file
% from being processed more than once.
% At this stage, the main document command |\childdocmain|
% is assumed to be called once again where it should do nothing.
% Any subsequent call to it should prevent
% a secondary processing of the main document
% It overwrites the forwarding commands
% |\childdocof| and |\childdocforward|
% with empty macros to prevent further inclusions of the main document:
%    \begin{macrocode}
\newcommand{\childdocdisable}
{
  \renewcommand{\childdocmain}[1]{\renewcommand{\childdocmain}[1]{\endinput}}
  \renewcommand{\childdocof}[1]{}
  \renewcommand{\childdocby}[2][]{}
  \renewcommand{\childdocforward}[2][]{}
  \renewcommand{\childdocdisable}{}
}
%    \end{macrocode}

% \macro{\childdocmain}
% The macro |\childdocmain| is to be called at the top of the main file
% with nothing or the main filename (without extension) as argument.
% First, it breaks loops.
% If the argument is not empty and does not match |\childdocname|
% (which is set by the first inclusion of |childdoc.def|),
% |\ifchilddoc| is set to true, |\includeonly| is applied to the child file
% and |\jobname| is set to the main file
% (for proper handling of |.aux| files):
%    \begin{macrocode}
\newcommand{\childdocmain}[1]
{
  \childdocdisable\childdocmain{}
  \if?#1?\else
    \begingroup
      \def\childdoctmp{#1}
      \ifx\childdoctmp\childdocname
        \def\childdoctmp{}
      \else
        \def\childdoctmp
        {
          \childdoctrue
          \includeonly{\childdocname}
          \def\childdocjob{#1}
          \def\jobname{#1}
        }
      \fi
      \expandafter
    \endgroup
    \childdoctmp
  \fi
}
%    \end{macrocode}

% \macro{\childdocof}
% The command |\childdocof| redirects
% compilation to the main file |#1|.
%    \begin{macrocode}
\newcommand{\childdocof}[1]
{
  \childdocdisable
  \childdoctrue
  \includeonly{\childdocname}
  \def\jobname{#1}
  \def\childdocjob{#1}
  \input{#1}
}
%    \end{macrocode}

% \macro{\childdocby}
% The command |\childdocby| ....
%    \begin{macrocode}
\newcommand{\childdocby}[2][]
{
  \childdocdisable
  \childdoctrue
  \childdocmanualtrue
  \if?#1?\else
    \def\jobname{#2}
  \fi
  \def\childdocjob{#2}
  \input{#2}
  \endinput
}
%    \end{macrocode}

% \macro{\childdocforward}
% The command |\childdocforward| redirects
% compilation to the main file or
% (if the optional argument is given) a child file.
% Parameters are set as if the main file
% or a child file starting with |\childdocof| was compiled.
% Then compilation is handed over to the main file:
%    \begin{macrocode}
\newcommand{\childdocforward}[2][]
{
  \begingroup
    \if?#1?
      \def\childdoctmp
      {
        \def\childdocname{#2}
        \def\childdocjob{#2}
        \def\jobname{#2}
        \input{#2}
        \endinput
      }
    \else
      \def\childdoctmp
      {
        \childdocdisable
        \def\childdocname{#2}
        \childdoctrue
        \includeonly{#2}
        \def\childdocjob{#1}
        \def\jobname{#1}
        \input{#1}
        \endinput
      }
    \fi
    \expandafter
  \endgroup
  \childdoctmp
}
%    \end{macrocode}

% \macro{\childdocforwardprefix}
% The command |\childdocforwardprefix| redirects
% compilation to the main or a child file by means of a pattern.
% The prefix |#1| in the current filename is replaced by |#2|
% and the suffix of the current filename is kept
% (it is assumed that the filename does not contain the substring `|~~~|'
% which is used as a delimiter).
% Compilation is handed over to the new file by |\childdocforward|:
%    \begin{macrocode}
\newcommand{\childdocforwardprefix}[3][]
{
  \begingroup
    \def\childdocextract #2##1~~~{\def\childdoctmp{\childdocforward[#1]{#3##1}}}
    \expandafter\childdocextract\childdocname~~~
    \expandafter
  \endgroup
  \childdoctmp
}
%    \end{macrocode}

% \macro{\childdoc}
% The deprecated macro |\childdoc| is a legacy version of |\childdocmain|:
%    \begin{macrocode}
\newcommand{\childdoc}{\childdocmain}
%    \end{macrocode}

% \macro{\childdocredirect}
% The deprecated macro |\childdocredirect| is a legacy version
% of |\childdocforward| and |\childdocforwardprefix|:
%    \begin{macrocode}
\newcommand{\childdocredirect}[2][]
{
  \begingroup
    \if?#1?
      \def\childdoctmp{\childdocforward{#2}}
    \else
      \def\childdoctmp{\childdocforwardprefix{#1}{#2}}
    \fi
    \expandafter
  \endgroup
  \childdoctmp
}
%    \end{macrocode}

%\iffalse
%</package>
%\fi
%
\endinput

\childdocof{cdocsamp}
%    \end{macrocode}

%\iffalse
%</samplechap1|samplechap2>
%\fi
%
%\iffalse
%<*samplechap1>
%\fi
% Some text for chapter 1:
%    \begin{macrocode}
\section{one}
some text in chapter one
%    \end{macrocode}

%\iffalse
%</samplechap1>
%\fi
% Some text for chapter 2:
%\iffalse
%<*samplechap2>
%\fi
%    \begin{macrocode}
\section{two}
more text in chapter two
%    \end{macrocode}

%\iffalse
%</samplechap2>
%\fi
%
% %%%%%%%%%%%%%%%%%%%%%%%%%%%%%%%%%%%%%%
% \paragraph{Part Include Files.}
%
% The include files are called |cdocspt3.tex| and |cdocspt4.tex|.
%
%\iffalse
%<*samplepart3|samplepart4>
%\fi

% Optional override for |\version| flag:
%    \begin{macrocode}
%%\providecommand{\version}{final}
%    \end{macrocode}

% Include the main document:
%    \begin{macrocode}
% \iffalse
%
% childdoc.dtx Copyright (C) 2017-2018 Niklas Beisert
%
% This work may be distributed and/or modified under the
% conditions of the LaTeX Project Public License, either version 1.3
% of this license or (at your option) any later version.
% The latest version of this license is in
%   http://www.latex-project.org/lppl.txt
% and version 1.3 or later is part of all distributions of LaTeX
% version 2005/12/01 or later.
%
% This work has the LPPL maintenance status `maintained'.
%
% The Current Maintainer of this work is Niklas Beisert.
%
% This work consists of the files childdoc.dtx and childdoc.ins
% and the derived files childdoc.def and cdocsamp.tex with
% cdocsch1.tex, cdocsch2.tex, cdocsdrf.tex, cdocsfn1.tex, cdocsfn2.tex.
%
%<package>\ifdefined\childdocmain\endinput\fi
%<package>\ProvidesFile{childdoc.def}[2018/12/30 v2.0 child document driver]
%<samplemain>\ProvidesFile{cdocsamp.tex}[2018/12/30 v2.0 sample for childdoc]
%<*driver>
%\ProvidesFile{childdoc.drv}[2018/12/30 v2.0 childdoc reference manual file]
\PassOptionsToClass{10pt,a4paper}{article}
\documentclass{ltxdoc}

\usepackage[margin=35mm]{geometry}
\usepackage{hyperref}
\usepackage{hyperxmp}
\usepackage[usenames]{color}

\hypersetup{colorlinks=true}
\hypersetup{pdfstartview=FitH}
\hypersetup{pdfpagemode=UseNone}
\hypersetup{pdfsource={}}
\hypersetup{pdflang={en-UK}}
\hypersetup{pdfcopyright={Copyright 2017-2018 Niklas Beisert.
  This work may be distributed and/or modified under the
  conditions of the LaTeX Project Public License, either version 1.3
  of this license or (at your option) any later version.}}
\hypersetup{pdflicenseurl={http://www.latex-project.org/lppl.txt}}
\hypersetup{pdfcontactaddress={ETH Zurich, ITP, HIT K,
  Wolfgang-Pauli-Strasse 27}}
\hypersetup{pdfcontactpostcode={8093}}
\hypersetup{pdfcontactcity={Zurich}}
\hypersetup{pdfcontactcountry={Switzerland}}
\hypersetup{pdfcontactemail={nbeisert@itp.phys.ethz.ch}}
\hypersetup{pdfcontacturl={http://people.phys.ethz.ch/\xmptilde nbeisert/}}

\newcommand{\secref}[1]{\hyperref[#1]{section \ref*{#1}}}

\parskip1ex
\parindent0pt
\let\olditemize\itemize
\def\itemize{\olditemize\parskip0pt}

\begin{document}

\title{The \textsf{childdoc} Package}
\hypersetup{pdftitle={The childdoc Package}}
\author{Niklas Beisert\\[2ex]
  Institut f\"ur Theoretische Physik\\
  Eidgen\"ossische Technische Hochschule Z\"urich\\
  Wolfgang-Pauli-Strasse 27, 8093 Z\"urich, Switzerland\\[1ex]
  \href{mailto:nbeisert@itp.phys.ethz.ch}
  {\texttt{nbeisert@itp.phys.ethz.ch}}}
\hypersetup{pdfauthor={Niklas Beisert}}
\hypersetup{pdfsubject={Manual for the LaTeX2e Package childdoc}}
\date{30 December 2018, \textsf{v2.0}}
\maketitle

\begin{abstract}\noindent
\textsf{childdoc} is a \LaTeXe{} package
that enables the direct compilation
of document sections included by |\include|
to individual files.
\end{abstract}

\begingroup
\parskip0ex
\tableofcontents
\endgroup

%%%%%%%%%%%%%%%%%%%%%%%%%%%%%%%%%%%%%%%%%%%%%%%%%%%%%%%%%%%%%%%%%%%%%%%%%%%%%%%%
%%%%%%%%%%%%%%%%%%%%%%%%%%%%%%%%%%%%%%%%%%%%%%%%%%%%%%%%%%%%%%%%%%%%%%%%%%%%%%%%
\section{Introduction}

\LaTeX{} provides a mechanism to structure a large document (such as a book)
into a main file and several child files (containing the chapters)
using the |\include| command.
This mechanism is beneficial for documents
which span hundreds of pages in order to
make the source file(s) more manageable.
Moreover, compilation can be restricted to
selected child files by means of the |\includeonly| command.
The latter feature can be used to reduce the compilation time while editing
(this was significantly more useful in the earlier days of \LaTeX{})
or to generate a smaller document which is easier to navigate.
Another application of |\includeonly| is to generate
documents consisting of selected parts of the complete document.

However, there are a few drawbacks of the plain |\include| mechanism:
\begin{itemize}
\item
The child files cannot be compiled on their own,
they can only be compiled via the main file.
A naive editing environment
(such as a text editor with an option
to have the current file processed by \LaTeX)
may require one to switch to the main file before compiling;
attempting to compile the child file produces errors.
\item
The main file must be modified (each time)
to adjust the |\includeonly| command
to the present needs. This easily leaves the main file in a messy state.
\item
The generated document will always carry the filename
of the main document. This is inconvenient if
several child files are to be compiled and
to be kept for distribution.
\end{itemize}

The present package provides a simple interface
to make child files individually compilable by \LaTeX{}.
Compiling a child file then has the same effect as compiling
the main file with an |\includeonly| command
to select the appropriate child.
Moreover the generated document will carry the name of the child
rather than the main file.
This resolves all three above issues.

This feature is meant to make the editing of books,
thesis documents and lecture notes somewhat more convenient.
However, the package can also be used efficiently for
composing a series of documents (such as exercise sheets)
which are typically distributed individually.
It then assists the author in generating the individual documents
(potentially in different versions)
as well as a document containing the collected series.
Another application is in developing style files
or other kinds of included material
where compilation of the style file could redirect
to a sample or test file.

%%%%%%%%%%%%%%%%%%%%%%%%%%%%%%%%%%%%%%%%%%%%%%%%%%%%%%%%%%%%%%%%%%%%%%%%%%%%%%%%
%%%%%%%%%%%%%%%%%%%%%%%%%%%%%%%%%%%%%%%%%%%%%%%%%%%%%%%%%%%%%%%%%%%%%%%%%%%%%%%%
\section{Usage}

First of all, the package \textsf{childdoc} is \emph{not} a standard
\LaTeXe{} |.sty| style file! Therefore it needs to be invoked in
a non-standard way.

%%%%%%%%%%%%%%%%%%%%%%%%%%%%%%%%%%%%%%%%%%%%%%%%%%%%%%%%%%%%%%%%%%%%%%%%%%%%%%%%
\subsection{Included Files}
\label{sec:include}

%%%%%%%%%%%%%%%%%%%%%%%%%%%%%%%%%%%%%%%%
\DescribeMacro{\childdocmain}
To use the package, add the commands
\begin{center}
\begin{tabular}{l}
|\input{childdoc.def}|\\
|\childdocmain{}|\\
\end{tabular}
\end{center}
at the very top of the main \LaTeX{} file,
in particular \emph{before} the |\documentclass| statement!
The argument of |\childdocmain| should be left empty
(but it must be present).

%%%%%%%%%%%%%%%%%%%%%%%%%%%%%%%%%%%%%%%%
\DescribeMacro{\childdocof}
Furthermore, add the commands
\begin{center}
\begin{tabular}{l}
|\input{childdoc.def}|\\
|\childdocof{|\textit{main}|}|\\
\end{tabular}
\end{center}
at the top of every child file \textit{child}
which is included by |\include{|\textit{child}|}|
from within the main file
(or at least for those files to be compiled individually).
The argument \textit{main} must be the filename of the main file.

There are a couple of
considerations in setting up the main and child documents:

%%%%%%%%%%%%%%%%%%%%%%%%%%%%%%%%%%%%%%%%
\paragraph{Restrictions.}

Please note the following restrictions:
\begin{itemize}
\item
|\childdocmain| must be called with one argument \textit{main}
to ensure compatibility with earlier version of the package.
It must either be empty (|\childdocmain{}|)
or precisely match the filename of the main file in which it is specified.
See \secref{sec:detection} for further information.
\item
The filename \textit{main} must be specified without the |.tex| extension.
\item
The filename \textit{main} is case sensitive
(even in case-insensitive file systems)
due to internal string comparison.
\item
The argument \textit{main} should be fully expanded, it cannot be a macro.
\item
Subdirectories and special characters should be avoided in filenames.
\item
The command |\childdocmain{|\textit{main}|}| must be followed by a whitespace.
It should not be followed immediately by another command
or by a comment mark `|%|'.
This is because the \TeX{} parser reads the token immediately following
the argument of |\childdocmain| and puts it
at the beginning of every child section;
however, a white\-space is ignored.
\end{itemize}

%%%%%%%%%%%%%%%%%%%%%%%%%%%%%%%%%%%%%%%%
\paragraph{Content of Main File.}

It is advisable to place all content in the child files included by |\include|.
Any output contained in the main file will appear in all child documents
unless suppressed manually;
it cannot be suppressed automatically by the |\includeonly| directive
and thus should normally be avoided.
A method to include some content in the main file
by means of conditional processing is described in \secref{sec:conditional}.

%%%%%%%%%%%%%%%%%%%%%%%%%%%%%%%%%%%%%%%%
\paragraph{Page Numbering.}

When only a part of the document is compiled,
the appropriate numbering of pages
(as well as other status parameters)
is determined from the |.aux| files.
The latter contain information from previous passes.
However this information needs to propagate through
all intermediate child documents.
Therefore the page numbering in child documents may well
be inconsistent until the complete document is compiled at least once.

A useful (if unconventional) way to always ensure a consistent
page numbering is to restart the numbering in each child document
and denote the pages by `\textit{child}|.|\textit{page}'
where \textit{child} represents the chapter/section number of the child file.
This can be achieved by the command
|\numberwithin{page}{|\textit{child}|}|
of the \textsf{amsmath} package
where \textit{child} can be |chapter| or |section|
depending on the chosen structuring.
Alternatively, one can modify the macro |\thepage| appropriately
and reset the counter |page| at the start of each child file.

%%%%%%%%%%%%%%%%%%%%%%%%%%%%%%%%%%%%%%%%%%%%%%%%%%%%%%%%%%%%%%%%%%%%%%%%%%%%%%%%
\subsection{Conditional Processing}
\label{sec:conditional}

The package provides a mechanism to compile different versions
of a document. To customise the versions further some conditional processing
can come in handy to distinguish which version is being compiled.
The package provides two macros to describe the compilation context:

%%%%%%%%%%%%%%%%%%%%%%%%%%%%%%%%%%%%%%%%
\DescribeMacro{\ifchilddoc}
The conditional |\ifchilddoc| distinguishes between the compilation of
child documents and the main document:
%
\begin{center}
|\ifchilddoc |\textit{child-code}| |[|\||else |\textit{main-code}]| \||fi|
\end{center}

%%%%%%%%%%%%%%%%%%%%%%%%%%%%%%%%%%%%%%%%
\DescribeMacro{\childdocname}
\DescribeMacro{\childdocjob}
The macro |\childdocname| contains the filename (without extension)
of the main or child file being processed.
Note that |\childdocjob| will always contain the name of the main file.

%%%%%%%%%%%%%%%%%%%%%%%%%%%%%%%%%%%%%%%%
\paragraph{Title Page.}

Conditional processing can be used to include a title or banner page
in the main document when proper precautions are taken.
Importantly, the code in the main file should ensure that the page counter
(as well as other status parameters which are stored in the |.aux| files)
takes the same value after the conditional processing.
Otherwise the page numbers may take divergent values
depending on which part is compiled.

For example, a title page could be declared by:
%
\begin{center}
\begin{tabular}{l}
|\ifchilddoc\||else|\\
|\addtocounter{page}{-1}|\\
\textit{code for title page}\\
|\newpage|\\
|\||fi|
\end{tabular}
\end{center}
%
A banner page for the child documents can be generated by:
%
\begin{center}
\begin{tabular}{l}
|\ifchilddoc|\\
|\addtocounter{page}{-1}|\\
\textit{code for banner page}\\
|\newpage|\\
|\||fi|
\end{tabular}
\end{center}
%
Here one could write a message such as:
\begin{center}
|This is the part \childdocname{} of \childdocjob{}.|
\end{center}

%%%%%%%%%%%%%%%%%%%%%%%%%%%%%%%%%%%%%%%%%%%%%%%%%%%%%%%%%%%%%%%%%%%%%%%%%%%%%%%%
\subsection{Flags}
\label{sec:flags}

The package makes it easy to generate different versions
of the main or child documents.
To this end compilation flags can be defined
and assigned different default values.
They will be particularly useful in conjunction
with the forwarding mechanism described in \secref{sec:forward}.

For example, it may be useful to have a flag |\version|
which can be set to |draft| or |final|.
The document source will contain some conditional code
depending on the value of |\version|.
Suppose further, the flag should default to |final| for the main file
and to |draft| for child files
which is a natural assignment for editing the document.
This is achieved by placing the following code
in the preamble of the main document
(below the |\childdocmain| directive):
%
\begin{center}
\begin{tabular}{l}
|\ifchilddoc|\\
|\providecommand{\version}{draft}|\\
|\||else|\\
|\providecommand{\version}{final}|\\
|\||fi|
\end{tabular}
\end{center}
%
The definition by |\providecommand| makes sure
that previous definitions are not overwritten.
Further statements |\providecommand{\version}{...}|
can thus be added before the above code to override it.

For the main file, one might add a line
(between |\childdocmain| and the above block)
%
\begin{center}
|%\ifchilddoc\||else\providecommand{\version}{draft}\||fi|
\end{center}
%
which can be uncommented to produce a draft version.
Likewise one can add a line to the very top of a child file
(above the |\childdocof{|\textit{main}|}| directive)
%
\begin{center}
|%\providecommand{\version}{final}|
\end{center}
%
which can be uncommented to produce the final version of this child document.

%%%%%%%%%%%%%%%%%%%%%%%%%%%%%%%%%%%%%%%%%%%%%%%%%%%%%%%%%%%%%%%%%%%%%%%%%%%%%%%%
\subsection{Forwarding}
\label{sec:forward}

Different versions of the main or child documents
using compilation flags as described in \secref{sec:flags}
can be (permanently) stored in different files
for convenient compilation, viewing and distribution.
To this end, the package defines a command
to pass on compilation to a different file:

%%%%%%%%%%%%%%%%%%%%%%%%%%%%%%%%%%%%%%%%
\DescribeMacro{\childdocforward}
The command |\childdocforward| redirects processing to
another source file:
%
\begin{center}
\begin{tabular}{l}
|\input{childdoc.def}|\\
|\childdocforward[|\textit{main}|]{|\textit{dest}|}|\\
\end{tabular}
\end{center}
%
The argument \textit{dest} is the destination file
(without extension).
It should be the main file or one of the child files.
Note that further \textsf{childdoc} directives
such as |\childdocof| and |\childdocforward|
in the indicated file will be processed in this form.
The optional argument \textit{main}
passes on directly to the main file \textit{main}
while pretending to compile the child \textit{dest}.
This form behaves as if \textit{dest}
issues |\childdocof{|\textit{main}|}| right away,
and no further \textsf{childdoc} directives will be processed.

%%%%%%%%%%%%%%%%%%%%%%%%%%%%%%%%%%%%%%%%
\DescribeMacro{\...prefix}
In the alternative form |\childdocforwardprefix|,
%
\begin{center}
\begin{tabular}{l}
|\input{childdoc.def}|\\
|\childdocforwardprefix[|\textit{main}|]{|\textit{prefix}|}{|\textit{dest}|}|
\end{tabular}
\end{center}
%
the destination file is determined by a pattern
depending on the current file:
To make this work, the current file must be called
`{\textit{prefix}\hspace{0.2em}\textit{suffix}}'
with \textit{prefix} matching precisely the argument.
Processing is then passed on to the file
`{\textit{dest}\hspace{0.2em}\textit{suffix}}'.
Surely, the same effect is achieved by
directly specifying the
argument `{\textit{dest}\hspace{0.2em}\textit{suffix}}'
in the first form.
However, that requires to set up a different file
for each child. With the alternative form of the command
all these files can have exactly the same content
which simplifies setting them up and maintaining them.

For example, the following file |draft.tex|
with a compilation flag |\version| as described in \secref{sec:flags}
compiles the main document as a draft:
%
\begin{center}
\begin{tabular}{l}
|\def\version{draft}|\\
|\input{childdoc.def}|\\
|\childdocforward{|\textit{main}|}|
\end{tabular}
\end{center}
%
Likewise, the following files |final|\textit{nn}|.tex|
compile the final version of the child document
|child|\textit{nn}|.tex|:
%
\begin{center}
\begin{tabular}{l}
|\def\version{final}|\\
|\input{childdoc.def}|\\
|\childdocforwardprefix{final}{child}|
\end{tabular}
\end{center}
%

Note that when several versions of a main file and/or of each child file
are to be generated, it may be convenient to set up a |Makefile| or
shell script to automatise the process.

%%%%%%%%%%%%%%%%%%%%%%%%%%%%%%%%%%%%%%%%%%%%%%%%%%%%%%%%%%%%%%%%%%%%%%%%%%%%%%%%
\subsection{Command Line Processing}
\label{sec:commandline}

The effect of redirection files can also be achieved by invoking
the \LaTeX{} compiler with a more elaborate command line.
Most conveniently this should be done as part
of a shell script or a |Makefile|.

When using \textsf{childdoc} in the main file, the following
command lines effectively perform a redirection
(note that depending on the shell being used,
backslashes may have to be doubled: `|\|' $\to$ `|\\|'):
%
\begin{center}
|... -jobname "|\textit{target}|" |\\|"|[\textit{flags}]%
|\input{childdoc.def}\childdocforward[|\textit{main}|]{|\textit{dest}|}"|
\end{center}
%
Here \textit{target} is the name of the output file,
\textit{main} is the name of the main file
and \textit{dest} is the name of the main or child file to be processed
(all filenames without extensions).
The optional argument \textit{main} can be omitted
if \textit{main} matches \textit{dest}.
Optionally, compilation \textit{flags} can be defined via |\def| commands.
This command line makes the \TeX{} engine believe
it is compiling the file \textit{target}
whose content is specified as the latter parameter.
The provided code then forwards the processing to
\textit{main} or \textit{dest} as described in \secref{sec:forward}.

%%%%%%%%%%%%%%%%%%%%%%%%%%%%%%%%%%%%%%%%%%%%%%%%%%%%%%%%%%%%%%%%%%%%%%%%%%%%%%%%
\subsection{Include by Input}
\label{sec:input}

Including child documents by |\include| has some restrictions by design.
Most notably, the content of a child document always occupies
its own set of pages; pages cannot be shared between child documents.
Usually, this behaviour makes perfect sense
because each child document contain an essential part of the document.
However, in some situations it may be desirable to compose
a document from a collection of parts
without having mandatory page breaks between then.
For this case, the package
provides a mechanism to include parts
by |\input| which can also be processed individually.
However, by construction this mechanism
requires manual handling of the content to be output.

%%%%%%%%%%%%%%%%%%%%%%%%%%%%%%%%%%%%%%%%
\DescribeMacro{\ifchilddocmanual}
The main file should be prepared as usual, see \secref{sec:include}.
However, the document body must make a distinction
between processing of an individual part and of the main document, e.g.:
%
\begin{center}
\begin{tabular}{l}
|\ifchilddocmanual|\\
|\input{\childdocname}|\\
|\||else|\\
\textit{document body with }|\input{|\textit{part}|}|\\
|\||fi|
\end{tabular}
\end{center}
%
The conditional |\ifchilddocmanual| is true whenever
a part to be included by |\input| is being compiled,
and the name of the part is stored in |\childdocname|.

%%%%%%%%%%%%%%%%%%%%%%%%%%%%%%%%%%%%%%%%
\DescribeMacro{\childdocby}
Each part to be included by |\input| should start with:
%
\begin{center}
\begin{tabular}{l}
|\input{childdoc.def}|\\
|\childdocby{|\textit{main}|}|\\
\end{tabular}
\end{center}
%
The directive |\childdocby| is similar to |\childdocof|
described in \secref{sec:include},
but the subsequent selection of content must be done manually.
To that end, both |\ifchilddoc| and |\ifchilddocmanual|
will be true upon processing of a part,
and the name of the part is stored in |\childdocname|.
Note that |\jobname| will be set to the filename of the current part
so that each part receives an individual |.aux| file
that does not interfere with the |.aux| file(s) of the main document.
This behaviour can be altered by the alternative form
|\childdocby[*]{|\textit{main}|}| (with a non-empty optional argument)
which uses the |.aux| file of the main document
by setting |\jobname| to \textit{main}.

%%%%%%%%%%%%%%%%%%%%%%%%%%%%%%%%%%%%%%%%%%%%%%%%%%%%%%%%%%%%%%%%%%%%%%%%%%%%%%%%
\subsection{Driver Development}
\label{sec:driver}

The \textsf{childdoc} mechanism can also be use for the development
of definition files such as \LaTeX{} styles or classes.
This case differs from the above setup with multiple parts
included by |\include| in that no |\includeonly| should be invoked.
This can be achieved by starting the include file
(before |\ProvidesPackage|) with:
%
\begin{center}
\begin{tabular}{l}
|\input{childdoc.def}|\\
|\childdocforward{|\textit{main}|}|\\
\end{tabular}
\end{center}
%
or alternatively with:
%
\begin{center}
\begin{tabular}{l}
|\input{childdoc.def}|\\
|\childdocby{|\textit{main}|}|\\
\end{tabular}
\end{center}
%
Both forms have slightly different effects as described above.
The main file is prepared as usual, see \secref{sec:include}.

%%%%%%%%%%%%%%%%%%%%%%%%%%%%%%%%%%%%%%%%%%%%%%%%%%%%%%%%%%%%%%%%%%%%%%%%%%%%%%%%
\subsection{Legacy Detection}
\label{sec:detection}

The directive |\childdocmain| in the main file can detect
whether the complete document or merely a child is to be compiled
even without using the directive |\childdocof|.
This method is deprecated because it is less robust
and there is no compelling reason to use it;
it is merely provided for backward compatibility
and it may be removed in future versions.

If the detection mechanism is to be used,
it is mandatory to correctly specify
the filename of the main file as the argument of |\childdocmain|:
%
\begin{center}
\begin{tabular}{l}
|\input{childdoc.def}|\\
|\childdocmain{|\textit{main}|}|\\
\end{tabular}
\end{center}
%
If |\jobname| does not match the argument \textit{main} of |\childdocmain|,
it is assumed that |\jobname| points to the child file to be compiled.
When using |\childdocmain| with the main file specified as argument,
it suffices to start a child file
with just |\input{|\textit{main}|}|
without loading of the package and using |\childdocof|.
If instead all processing is done
with the appropriate \textsf{childdoc} directives,
the argument of \textit{main} of |\childdocmain| can be empty.

An alternative version of the command line processing described
in \secref{sec:commandline} using the detection mechanism reads:
%
\begin{center}
|... -jobname "|\textit{target}|" "|[\textit{flags}]%
[|\def\jobname{|\textit{dest}|}|]|\input{|\textit{main}|}"|
\end{center}

%%%%%%%%%%%%%%%%%%%%%%%%%%%%%%%%%%%%%%%%%%%%%%%%%%%%%%%%%%%%%%%%%%%%%%%%%%%%%%%%
\subsection{Manual Code}
\label{sec:manual}

In case one cannot be certain whether the definitions file |childdoc.def|
is installed on the target \TeX{} distribution
and one prefers not to ship it,
it is conceivable to paste a few relevant commands into the sources.

To that end, drop all statements |\input{childdoc.def}|
and perform the replacements as outlined below.
Instead of |\childdocmain{|\textit{main}|}| add the following code
to the top of the main file:
%
\begin{center}
\begin{tabular}{l}
|\||ifdefined\childdocname\endinput\||fi\newif\ifchilddoc|\\
|\edef\childdocname{\scantokens\expandafter{\jobname\noexpand}}|\\
|\def\childdocmain{|\textit{main}|}\||ifx\childdocmain\childdocname\||else|\\
|\childdoctrue\includeonly{\childdocname}\let\jobname\childdocmain\||fi|\\
\end{tabular}
\end{center}
%
Instead of |\childdocof{|\textit{main}|}| just include the main file
at the top of each child file:
%
\begin{center}
|\input{|\textit{main}|}|
\end{center}
%
A simple redirection |\childdocforward{|\textit{dest}|}| is achieved by:
%
\begin{center}
|\def\jobname{|\textit{dest}|}\input{\jobname}|
\end{center}
%
The redirection with prefix
|\childdocforwardprefix[|\textit{prefix}|]{|\textit{dest}|}|
is accomplished by:
%
\begin{center}
\begin{tabular}{l}
|{\edef\jobname{\scantokens\expandafter{\jobname\noexpand}}|\\
|\def\redirectjob |\textit{prefix}|#1~~~{\gdef\jobname{|\textit{dest}|#1}}|\\
|\expandafter\redirectjob\jobname~~~}\input{\jobname}|
\end{tabular}
\end{center}

In an alternative approach,
child documents can be compiled by a specific command line
without additional code or specific definitions:
%
\begin{center}
|... -jobname "|\textit{target}|" "|[\textit{flags}]%
|\includeonly{|\textit{dest}|}\input{|\textit{main}|}"|
\end{center}
%

%%%%%%%%%%%%%%%%%%%%%%%%%%%%%%%%%%%%%%%%%%%%%%%%%%%%%%%%%%%%%%%%%%%%%%%%%%%%%%%%
%%%%%%%%%%%%%%%%%%%%%%%%%%%%%%%%%%%%%%%%%%%%%%%%%%%%%%%%%%%%%%%%%%%%%%%%%%%%%%%%
\section{Information}

%%%%%%%%%%%%%%%%%%%%%%%%%%%%%%%%%%%%%%%%%%%%%%%%%%%%%%%%%%%%%%%%%%%%%%%%%%%%%%%%
\subsection{Copyright}

Copyright \copyright{} 2017--2018 Niklas Beisert

This work may be distributed and/or modified under the
conditions of the \LaTeX{} Project Public License, either version 1.3
of this license or (at your option) any later version.
The latest version of this license is in
  \url{http://www.latex-project.org/lppl.txt}
and version 1.3 or later is part of all distributions of \LaTeX{}
version 2005/12/01 or later.

This work has the LPPL maintenance status `maintained'.

The Current Maintainer of this work is Niklas Beisert.

This work consists of the files |README.txt|, |childdoc.ins| and |childdoc.dtx|
as well as the derived files |childdoc.def|, |cdocsamp.tex|
with |cdocsch1.tex|, |cdocsch2.tex|, |cdocspt3.tex|, |cdocspt4.tex|,
|cdocsdrf.tex|, |cdocsfn1.tex|, |cdocsfn2.tex|
as well as |childdoc.pdf|.

%%%%%%%%%%%%%%%%%%%%%%%%%%%%%%%%%%%%%%%%%%%%%%%%%%%%%%%%%%%%%%%%%%%%%%%%%%%%%%%%
\subsection{Files and Installation}

The package consists of the files:
%
\begin{center}
\begin{tabular}{ll}
    |README.txt|   & readme file \\
    |childdoc.ins| & installation file \\
    |childdoc.dtx| & source file \\
    |childdoc.def| & definition file \\
    |cdocsamp.tex| & sample main file \\
    |cdocsch1.tex| & sample include file \\
    |cdocsch2.tex| & sample include file \\
    |cdocspt3.tex| & sample part file \\
    |cdocspt4.tex| & sample part file \\
    |cdocsdrf.tex| & sample redirection file \\
    |cdocsfn1.tex| & sample redirection file \\
    |cdocsfn2.tex| & sample redirection file \\
    |childdoc.pdf| & manual
\end{tabular}
\end{center}
%
The distribution consists of the files
|README.txt|, |childdoc.ins| and |childdoc.dtx|.
%
\begin{itemize}
\item
Run (pdf)\LaTeX{} on |childdoc.dtx|
to compile the manual |childdoc.pdf| (this file).
\item
Run \LaTeX{} on |childdoc.ins| to create the definitions file |childdoc.def|
and the sample |cdocsamp.tex| with include files
|cdocsch1.tex|, |cdocsch2.tex|, |cdocspt3.tex|, |cdocspt4.tex|,
|cdocsdrf.tex|, |cdocsfn1.tex|, |cdocsfn2.tex|.
Then copy the file |childdoc.def| to an appropriate directory of your \LaTeX{}
distribution, e.g.\ \textit{texmf-root}|/tex/latex/childdoc|.
\end{itemize}

%%%%%%%%%%%%%%%%%%%%%%%%%%%%%%%%%%%%%%%%%%%%%%%%%%%%%%%%%%%%%%%%%%%%%%%%%%%%%%%%
\subsection{Related CTAN Packages}

There are several other packages which offer a similar functionality:
%
\begin{itemize}
\item
The packages
\href{http://ctan.org/pkg/docmute}{\textsf{docmute}},
\href{http://ctan.org/pkg/includex}{\textsf{includex}} and
\href{http://ctan.org/pkg/standalone}{\textsf{standalone}}
provide commands to include only the document body of
a child file thus allowing both files to be compiled individually.
\item
The packages \href{http://ctan.org/pkg/subdocs}{\textsf{subdocs}}
and \href{http://ctan.org/pkg/subfiles}{\textsf{subfiles}}
provide structures in which the main and child documents can be
encapsulated and allowing them to be compiled individually.
The inclusion mechanism is different from the conventional |\include|.
\item
The package \href{http://ctan.org/pkg/combine}{\textsf{combine}}
is an elaborate solution to combine several documents into one.
\end{itemize}
%
See also the CTAN topic \href{http://ctan.org/topic/subdocs}{\textsf{subdocs}}
for further related packages.
The present package differs from the above solutions in that
a document structure constructed with the conventional |\include| mechanism
just needs two extra commands at the top of every file
such that all constituent files can be compiled individually.

%%%%%%%%%%%%%%%%%%%%%%%%%%%%%%%%%%%%%%%%%%%%%%%%%%%%%%%%%%%%%%%%%%%%%%%%%%%%%%%%
%\subsection{Feature Suggestions}
%
%The following is a list of features which may be useful for future
%versions of this package:
%%
%\begin{itemize}
%\item
%\ldots
%\end{itemize}

%%%%%%%%%%%%%%%%%%%%%%%%%%%%%%%%%%%%%%%%%%%%%%%%%%%%%%%%%%%%%%%%%%%%%%%%%%%%%%%%
\subsection{Revision History}

%%%%%%%%%%%%%%%%%%%%%%%%%%%%%%%%%%%%%%%%
\paragraph{v2.0:} 2018/12/30

\begin{itemize}
\item
immediate forward processing
\item
added |\childdocby| mechanism
\item
manual restructured
\end{itemize}

%%%%%%%%%%%%%%%%%%%%%%%%%%%%%%%%%%%%%%%%
\paragraph{v1.6:} 2018/01/17

\begin{itemize}
\item
application for development of include files
\item
corrections to manual
\end{itemize}

%%%%%%%%%%%%%%%%%%%%%%%%%%%%%%%%%%%%%%%%
\paragraph{v1.5:} 2017/05/21

\begin{itemize}
\item
more complete structuring introduced
\item
|\childdocof| introduced
\item
|\childdoc| renamed to |\childdocmain|
\item
|\childredirect| renamed to |\childdocforward| and |\childdocforwardprefix|
and functionality expanded
\end{itemize}

%%%%%%%%%%%%%%%%%%%%%%%%%%%%%%%%%%%%%%%%
\paragraph{v1.0:} 2017/04/27

\begin{itemize}
\item
manual and install package
\item
first version published on CTAN
\end{itemize}

%%%%%%%%%%%%%%%%%%%%%%%%%%%%%%%%%%%%%%%%
\paragraph{v0.6:} 2017/04/26

\begin{itemize}
\item
redirection mechanism added
\end{itemize}

%%%%%%%%%%%%%%%%%%%%%%%%%%%%%%%%%%%%%%%%
\paragraph{v0.5:} 2017/04/26

\begin{itemize}
\item
functionality in definition file
\end{itemize}


%%%%%%%%%%%%%%%%%%%%%%%%%%%%%%%%%%%%%%%%%%%%%%%%%%%%%%%%%%%%%%%%%%%%%%%%%%%%%%%%
%%%%%%%%%%%%%%%%%%%%%%%%%%%%%%%%%%%%%%%%%%%%%%%%%%%%%%%%%%%%%%%%%%%%%%%%%%%%%%%%
%%%%%%%%%%%%%%%%%%%%%%%%%%%%%%%%%%%%%%%%%%%%%%%%%%%%%%%%%%%%%%%%%%%%%%%%%%%%%%%%
\appendix

\settowidth\MacroIndent{\rmfamily\scriptsize 000\ }

 \DocInput{childdoc.dtx}

\end{document}
%</driver>
% \fi
%
% %%%%%%%%%%%%%%%%%%%%%%%%%%%%%%%%%%%%%%%%%%%%%%%%%%%%%%%%%%%%%%%%%%%%%%%%%%%%%%
% %%%%%%%%%%%%%%%%%%%%%%%%%%%%%%%%%%%%%%%%%%%%%%%%%%%%%%%%%%%%%%%%%%%%%%%%%%%%%%
% \section{Sample}
%\iffalse
%<*samplemain>
%\fi
%
% The following presents a sample document
% with two chapters, two parts, a title page,
% a compile flag as well as three forwarding files to set the flag.
% It consists of eight |.tex| files:
% \begin{center}
% \begin{tabular}{ll}
% |cdocsamp.tex|&main file\\
% |cdocsch1.tex|&include file for chapter 1\\
% |cdocsch2.tex|&include file for chapter 2\\
% |cdocspt3.tex|&include file for part 3\\
% |cdocspt4.tex|&include file for part 4\\
% |cdocsdrf.tex|&forwarding file for main file in draft mode\\
% |cdocsfi1.tex|&forwarding file for final version of chapter 1\\
% |cdocsfi2.tex|&forwarding file for final version of chapter 2\\
% \end{tabular}
% \end{center}
% Each of the eight files can be compiled directly by the \LaTeX{} compiler.
%
% %%%%%%%%%%%%%%%%%%%%%%%%%%%%%%%%%%%%%%
% \paragraph{Main File.}
%
% The main file is called |cdocsamp.tex|.
%
% Load the \textsf{childdoc} definitions and
% declare the filename for the main document:
%    \begin{macrocode}
\input{childdoc.def}
\childdocmain{}
%    \end{macrocode}

% Optional override for |\version| flag:
%    \begin{macrocode}
%%\ifchilddoc\else\providecommand{\version}{draft}\fi
%    \end{macrocode}

% Define the default values for the |\version| flag
% (|final| for the main file and |draft| for childs):
%    \begin{macrocode}
\ifchilddoc
\providecommand{\version}{draft}
\else
\providecommand{\version}{final}
\fi
%    \end{macrocode}

% Load the standard document class:
%    \begin{macrocode}
\documentclass[12pt]{article}
%    \end{macrocode}

% Start the document body:
%    \begin{macrocode}
\begin{document}
%    \end{macrocode}

% Declare a title page.
% Print title, part of document being processed and version flag:
%    \begin{macrocode}
\addtocounter{page}{-1}
\begin{center}
{\LARGE\bfseries{}childdoc example\par}
\vspace{1cm}
\ifchilddoc
\ifchilddocmanual part\else chapter\fi:
`\childdocname' of `\childdocjob'\par
\else
main document: `\childdocjob'\par
\fi
version: \version\par
\end{center}
\newpage
%    \end{macrocode}

% Manually include selected file,
% otherwise process as usual:
%    \begin{macrocode}
\ifchilddocmanual
\section*{part `\childdocname'}
\input{\childdocname}
\else
%    \end{macrocode}

% Include the two chapters:
%    \begin{macrocode}
\include{cdocsch1}
\include{cdocsch2}
%    \end{macrocode}

% Include the two parts unless only chapters should be displayed:
%    \begin{macrocode}
\ifchilddoc\else
\section{part three}
\input{cdocspt3}
\section{part four}
\input{cdocspt4}
\fi
%    \end{macrocode}

% Process as usual until here:
%    \begin{macrocode}
\fi
%    \end{macrocode}

% End of document body:
%    \begin{macrocode}
\end{document}
%    \end{macrocode}
%\iffalse
%</samplemain>
%\fi
%
% %%%%%%%%%%%%%%%%%%%%%%%%%%%%%%%%%%%%%%
% \paragraph{Chapter Include Files.}
%
% The include files are called |cdocsch1.tex| and |cdocsch2.tex|.
%
%\iffalse
%<*samplechap1|samplechap2>
%\fi

% Optional override for |\version| flag:
%    \begin{macrocode}
%%\providecommand{\version}{final}
%    \end{macrocode}

% Include the main document:
%    \begin{macrocode}
\input{childdoc.def}
\childdocof{cdocsamp}
%    \end{macrocode}

%\iffalse
%</samplechap1|samplechap2>
%\fi
%
%\iffalse
%<*samplechap1>
%\fi
% Some text for chapter 1:
%    \begin{macrocode}
\section{one}
some text in chapter one
%    \end{macrocode}

%\iffalse
%</samplechap1>
%\fi
% Some text for chapter 2:
%\iffalse
%<*samplechap2>
%\fi
%    \begin{macrocode}
\section{two}
more text in chapter two
%    \end{macrocode}

%\iffalse
%</samplechap2>
%\fi
%
% %%%%%%%%%%%%%%%%%%%%%%%%%%%%%%%%%%%%%%
% \paragraph{Part Include Files.}
%
% The include files are called |cdocspt3.tex| and |cdocspt4.tex|.
%
%\iffalse
%<*samplepart3|samplepart4>
%\fi

% Optional override for |\version| flag:
%    \begin{macrocode}
%%\providecommand{\version}{final}
%    \end{macrocode}

% Include the main document:
%    \begin{macrocode}
\input{childdoc.def}
\childdocby{cdocsamp}
%    \end{macrocode}

%\iffalse
%</samplepart3|samplepart4>
%\fi
%
%\iffalse
%<*samplepart3>
%\fi
% Some text for part 3:
%    \begin{macrocode}
some text in part three
%    \end{macrocode}

%\iffalse
%</samplepart3>
%\fi
% Some text for part 4:
%\iffalse
%<*samplepart4>
%\fi
%    \begin{macrocode}
more text in part four
%    \end{macrocode}

%\iffalse
%</samplepart4>
%\fi
%
% %%%%%%%%%%%%%%%%%%%%%%%%%%%%%%%%%%%%%%
% \paragraph{Forwarding for a Complete Draft.}
%
% The following forwarding file |cdocsdrf.tex|
% compiles the main document in draft mode:
%\iffalse
%<*sampledraft>
%\fi
%    \begin{macrocode}
\def\version{draft}
\input{childdoc.def}
\childdocforward{cdocsamp}
%    \end{macrocode}

%\iffalse
%</sampledraft>
%\fi
%
% %%%%%%%%%%%%%%%%%%%%%%%%%%%%%%%%%%%%%%
% \paragraph{Forwarding for Final Version of the Chapters.}
%
% The following forwarding files |cdocsfn1.tex| and |cdocsfn2.tex|
% (with identical content)
% compile the final versions of the child documents
% |cdocsch1.tex| and |cdocsch2.tex|, respectively:
%\iffalse
%<*samplefinal>
%\fi
%    \begin{macrocode}
\def\version{final}
\input{childdoc.def}
\childdocforwardprefix[cdocsamp]{cdocsfn}{cdocsch}
%    \end{macrocode}

%\iffalse
%</samplefinal>
%\fi
%
% %%%%%%%%%%%%%%%%%%%%%%%%%%%%%%%%%%%%%%
% \paragraph{Command Line Processing.}
%
% The following three command lines generate the output files
% |cdocscld|, |cdocscl1| and |cdocscl2|
% which should be identical to
% |cdocsdrf|, |cdocsch1| and |cdocsfn2|, respectively:
% \begin{center}
% \begin{tabular}{l}
% |latex -jobname cdocscld \|\\
% |  "\def\version{draft}\input{childdoc.def}\childdocforward{cdocsamp}"|\\
% |latex -jobname cdocscl1 \|\\
% |  "\input{childdoc.def}\childdocforward[cdocsamp]{cdocsch1}"|\\
% |latex -jobname cdocscl2 \|\\
% |  "\def\version{final}\input{childdoc.def}\childdocforward{cdocsch2}"|
% \end{tabular}
% \end{center}
% Note that the trailing backslash on each first line
% merely continues the input to the second line
% (for convenient cut ant paste).
% Furthermore, the command |latex| can be replaced by any
% of its alternative versions such as |pdflatex|.
%
% %%%%%%%%%%%%%%%%%%%%%%%%%%%%%%%%%%%%%%%%%%%%%%%%%%%%%%%%%%%%%%%%%%%%%%%%%%%%%%
% %%%%%%%%%%%%%%%%%%%%%%%%%%%%%%%%%%%%%%%%%%%%%%%%%%%%%%%%%%%%%%%%%%%%%%%%%%%%%%
% \section{Implementation}
%\iffalse
%<*package>
%\fi
%
% This section describes the definitions file |childdoc.def|.

% The definitions cannot be loaded using |\usepackage| or |\RequirePackage|
% which has a mechanism to prevent loading a style file more than once.
% When loading the definitions by means of |\input|
% multiple instances have to be prevented manually:
%\iffalse
%This code needs to be before the `\ProvidesFile' directive
%which is defined at the beginning of this file.
%Therefore it is also placed there and commented out here.
%</package>
%<*discard>
%\fi
%    \begin{macrocode}
\ifdefined\childdocmain\endinput\fi
%    \end{macrocode}
%\iffalse
%</discard>
%<*package>
%\fi
%
% \macro{\ifchilddoc}
% \macro{\ifchilddocmanual}
% The conditional |\ifchilddoc| tells whether a
% child (true) or main (false) document is being compiled.
% The conditional |\ifchilddocmanual| tells whether
% the |\includeonly| mechanism is used (false) or
% the selection of child files must be performed manually (true).
% The definitions initialise to false:
%    \begin{macrocode}
\newif\ifchilddoc
\newif\ifchilddocmanual
%    \end{macrocode}

% \macro{\childdocname}
% \macro{\childdocjob}
% The macro |\childdocname| stores the name of the main document
% to be compiled. The macro |\childdocjob| stores the name of
% the document on which the \LaTeX{} compiler was originally invoked.
% The content of |\jobname| cannot be compared
% to filenames specified in the source due to different catcodes.
% The following code rescans |\jobname|, stores the result
% in |\childdocname| and saves a copy in |\childdocjob|:
%    \begin{macrocode}
\edef\childdocname{\scantokens\expandafter{\jobname\noexpand}}
\let\childdocjob\childdocname
%    \end{macrocode}

% \macro{\childdocdisable}
% The macro |\childdocdisable| prevents the main file
% from being processed more than once.
% At this stage, the main document command |\childdocmain|
% is assumed to be called once again where it should do nothing.
% Any subsequent call to it should prevent
% a secondary processing of the main document
% It overwrites the forwarding commands
% |\childdocof| and |\childdocforward|
% with empty macros to prevent further inclusions of the main document:
%    \begin{macrocode}
\newcommand{\childdocdisable}
{
  \renewcommand{\childdocmain}[1]{\renewcommand{\childdocmain}[1]{\endinput}}
  \renewcommand{\childdocof}[1]{}
  \renewcommand{\childdocby}[2][]{}
  \renewcommand{\childdocforward}[2][]{}
  \renewcommand{\childdocdisable}{}
}
%    \end{macrocode}

% \macro{\childdocmain}
% The macro |\childdocmain| is to be called at the top of the main file
% with nothing or the main filename (without extension) as argument.
% First, it breaks loops.
% If the argument is not empty and does not match |\childdocname|
% (which is set by the first inclusion of |childdoc.def|),
% |\ifchilddoc| is set to true, |\includeonly| is applied to the child file
% and |\jobname| is set to the main file
% (for proper handling of |.aux| files):
%    \begin{macrocode}
\newcommand{\childdocmain}[1]
{
  \childdocdisable\childdocmain{}
  \if?#1?\else
    \begingroup
      \def\childdoctmp{#1}
      \ifx\childdoctmp\childdocname
        \def\childdoctmp{}
      \else
        \def\childdoctmp
        {
          \childdoctrue
          \includeonly{\childdocname}
          \def\childdocjob{#1}
          \def\jobname{#1}
        }
      \fi
      \expandafter
    \endgroup
    \childdoctmp
  \fi
}
%    \end{macrocode}

% \macro{\childdocof}
% The command |\childdocof| redirects
% compilation to the main file |#1|.
%    \begin{macrocode}
\newcommand{\childdocof}[1]
{
  \childdocdisable
  \childdoctrue
  \includeonly{\childdocname}
  \def\jobname{#1}
  \def\childdocjob{#1}
  \input{#1}
}
%    \end{macrocode}

% \macro{\childdocby}
% The command |\childdocby| ....
%    \begin{macrocode}
\newcommand{\childdocby}[2][]
{
  \childdocdisable
  \childdoctrue
  \childdocmanualtrue
  \if?#1?\else
    \def\jobname{#2}
  \fi
  \def\childdocjob{#2}
  \input{#2}
  \endinput
}
%    \end{macrocode}

% \macro{\childdocforward}
% The command |\childdocforward| redirects
% compilation to the main file or
% (if the optional argument is given) a child file.
% Parameters are set as if the main file
% or a child file starting with |\childdocof| was compiled.
% Then compilation is handed over to the main file:
%    \begin{macrocode}
\newcommand{\childdocforward}[2][]
{
  \begingroup
    \if?#1?
      \def\childdoctmp
      {
        \def\childdocname{#2}
        \def\childdocjob{#2}
        \def\jobname{#2}
        \input{#2}
        \endinput
      }
    \else
      \def\childdoctmp
      {
        \childdocdisable
        \def\childdocname{#2}
        \childdoctrue
        \includeonly{#2}
        \def\childdocjob{#1}
        \def\jobname{#1}
        \input{#1}
        \endinput
      }
    \fi
    \expandafter
  \endgroup
  \childdoctmp
}
%    \end{macrocode}

% \macro{\childdocforwardprefix}
% The command |\childdocforwardprefix| redirects
% compilation to the main or a child file by means of a pattern.
% The prefix |#1| in the current filename is replaced by |#2|
% and the suffix of the current filename is kept
% (it is assumed that the filename does not contain the substring `|~~~|'
% which is used as a delimiter).
% Compilation is handed over to the new file by |\childdocforward|:
%    \begin{macrocode}
\newcommand{\childdocforwardprefix}[3][]
{
  \begingroup
    \def\childdocextract #2##1~~~{\def\childdoctmp{\childdocforward[#1]{#3##1}}}
    \expandafter\childdocextract\childdocname~~~
    \expandafter
  \endgroup
  \childdoctmp
}
%    \end{macrocode}

% \macro{\childdoc}
% The deprecated macro |\childdoc| is a legacy version of |\childdocmain|:
%    \begin{macrocode}
\newcommand{\childdoc}{\childdocmain}
%    \end{macrocode}

% \macro{\childdocredirect}
% The deprecated macro |\childdocredirect| is a legacy version
% of |\childdocforward| and |\childdocforwardprefix|:
%    \begin{macrocode}
\newcommand{\childdocredirect}[2][]
{
  \begingroup
    \if?#1?
      \def\childdoctmp{\childdocforward{#2}}
    \else
      \def\childdoctmp{\childdocforwardprefix{#1}{#2}}
    \fi
    \expandafter
  \endgroup
  \childdoctmp
}
%    \end{macrocode}

%\iffalse
%</package>
%\fi
%
\endinput

\childdocby{cdocsamp}
%    \end{macrocode}

%\iffalse
%</samplepart3|samplepart4>
%\fi
%
%\iffalse
%<*samplepart3>
%\fi
% Some text for part 3:
%    \begin{macrocode}
some text in part three
%    \end{macrocode}

%\iffalse
%</samplepart3>
%\fi
% Some text for part 4:
%\iffalse
%<*samplepart4>
%\fi
%    \begin{macrocode}
more text in part four
%    \end{macrocode}

%\iffalse
%</samplepart4>
%\fi
%
% %%%%%%%%%%%%%%%%%%%%%%%%%%%%%%%%%%%%%%
% \paragraph{Forwarding for a Complete Draft.}
%
% The following forwarding file |cdocsdrf.tex|
% compiles the main document in draft mode:
%\iffalse
%<*sampledraft>
%\fi
%    \begin{macrocode}
\def\version{draft}
% \iffalse
%
% childdoc.dtx Copyright (C) 2017-2018 Niklas Beisert
%
% This work may be distributed and/or modified under the
% conditions of the LaTeX Project Public License, either version 1.3
% of this license or (at your option) any later version.
% The latest version of this license is in
%   http://www.latex-project.org/lppl.txt
% and version 1.3 or later is part of all distributions of LaTeX
% version 2005/12/01 or later.
%
% This work has the LPPL maintenance status `maintained'.
%
% The Current Maintainer of this work is Niklas Beisert.
%
% This work consists of the files childdoc.dtx and childdoc.ins
% and the derived files childdoc.def and cdocsamp.tex with
% cdocsch1.tex, cdocsch2.tex, cdocsdrf.tex, cdocsfn1.tex, cdocsfn2.tex.
%
%<package>\ifdefined\childdocmain\endinput\fi
%<package>\ProvidesFile{childdoc.def}[2018/12/30 v2.0 child document driver]
%<samplemain>\ProvidesFile{cdocsamp.tex}[2018/12/30 v2.0 sample for childdoc]
%<*driver>
%\ProvidesFile{childdoc.drv}[2018/12/30 v2.0 childdoc reference manual file]
\PassOptionsToClass{10pt,a4paper}{article}
\documentclass{ltxdoc}

\usepackage[margin=35mm]{geometry}
\usepackage{hyperref}
\usepackage{hyperxmp}
\usepackage[usenames]{color}

\hypersetup{colorlinks=true}
\hypersetup{pdfstartview=FitH}
\hypersetup{pdfpagemode=UseNone}
\hypersetup{pdfsource={}}
\hypersetup{pdflang={en-UK}}
\hypersetup{pdfcopyright={Copyright 2017-2018 Niklas Beisert.
  This work may be distributed and/or modified under the
  conditions of the LaTeX Project Public License, either version 1.3
  of this license or (at your option) any later version.}}
\hypersetup{pdflicenseurl={http://www.latex-project.org/lppl.txt}}
\hypersetup{pdfcontactaddress={ETH Zurich, ITP, HIT K,
  Wolfgang-Pauli-Strasse 27}}
\hypersetup{pdfcontactpostcode={8093}}
\hypersetup{pdfcontactcity={Zurich}}
\hypersetup{pdfcontactcountry={Switzerland}}
\hypersetup{pdfcontactemail={nbeisert@itp.phys.ethz.ch}}
\hypersetup{pdfcontacturl={http://people.phys.ethz.ch/\xmptilde nbeisert/}}

\newcommand{\secref}[1]{\hyperref[#1]{section \ref*{#1}}}

\parskip1ex
\parindent0pt
\let\olditemize\itemize
\def\itemize{\olditemize\parskip0pt}

\begin{document}

\title{The \textsf{childdoc} Package}
\hypersetup{pdftitle={The childdoc Package}}
\author{Niklas Beisert\\[2ex]
  Institut f\"ur Theoretische Physik\\
  Eidgen\"ossische Technische Hochschule Z\"urich\\
  Wolfgang-Pauli-Strasse 27, 8093 Z\"urich, Switzerland\\[1ex]
  \href{mailto:nbeisert@itp.phys.ethz.ch}
  {\texttt{nbeisert@itp.phys.ethz.ch}}}
\hypersetup{pdfauthor={Niklas Beisert}}
\hypersetup{pdfsubject={Manual for the LaTeX2e Package childdoc}}
\date{30 December 2018, \textsf{v2.0}}
\maketitle

\begin{abstract}\noindent
\textsf{childdoc} is a \LaTeXe{} package
that enables the direct compilation
of document sections included by |\include|
to individual files.
\end{abstract}

\begingroup
\parskip0ex
\tableofcontents
\endgroup

%%%%%%%%%%%%%%%%%%%%%%%%%%%%%%%%%%%%%%%%%%%%%%%%%%%%%%%%%%%%%%%%%%%%%%%%%%%%%%%%
%%%%%%%%%%%%%%%%%%%%%%%%%%%%%%%%%%%%%%%%%%%%%%%%%%%%%%%%%%%%%%%%%%%%%%%%%%%%%%%%
\section{Introduction}

\LaTeX{} provides a mechanism to structure a large document (such as a book)
into a main file and several child files (containing the chapters)
using the |\include| command.
This mechanism is beneficial for documents
which span hundreds of pages in order to
make the source file(s) more manageable.
Moreover, compilation can be restricted to
selected child files by means of the |\includeonly| command.
The latter feature can be used to reduce the compilation time while editing
(this was significantly more useful in the earlier days of \LaTeX{})
or to generate a smaller document which is easier to navigate.
Another application of |\includeonly| is to generate
documents consisting of selected parts of the complete document.

However, there are a few drawbacks of the plain |\include| mechanism:
\begin{itemize}
\item
The child files cannot be compiled on their own,
they can only be compiled via the main file.
A naive editing environment
(such as a text editor with an option
to have the current file processed by \LaTeX)
may require one to switch to the main file before compiling;
attempting to compile the child file produces errors.
\item
The main file must be modified (each time)
to adjust the |\includeonly| command
to the present needs. This easily leaves the main file in a messy state.
\item
The generated document will always carry the filename
of the main document. This is inconvenient if
several child files are to be compiled and
to be kept for distribution.
\end{itemize}

The present package provides a simple interface
to make child files individually compilable by \LaTeX{}.
Compiling a child file then has the same effect as compiling
the main file with an |\includeonly| command
to select the appropriate child.
Moreover the generated document will carry the name of the child
rather than the main file.
This resolves all three above issues.

This feature is meant to make the editing of books,
thesis documents and lecture notes somewhat more convenient.
However, the package can also be used efficiently for
composing a series of documents (such as exercise sheets)
which are typically distributed individually.
It then assists the author in generating the individual documents
(potentially in different versions)
as well as a document containing the collected series.
Another application is in developing style files
or other kinds of included material
where compilation of the style file could redirect
to a sample or test file.

%%%%%%%%%%%%%%%%%%%%%%%%%%%%%%%%%%%%%%%%%%%%%%%%%%%%%%%%%%%%%%%%%%%%%%%%%%%%%%%%
%%%%%%%%%%%%%%%%%%%%%%%%%%%%%%%%%%%%%%%%%%%%%%%%%%%%%%%%%%%%%%%%%%%%%%%%%%%%%%%%
\section{Usage}

First of all, the package \textsf{childdoc} is \emph{not} a standard
\LaTeXe{} |.sty| style file! Therefore it needs to be invoked in
a non-standard way.

%%%%%%%%%%%%%%%%%%%%%%%%%%%%%%%%%%%%%%%%%%%%%%%%%%%%%%%%%%%%%%%%%%%%%%%%%%%%%%%%
\subsection{Included Files}
\label{sec:include}

%%%%%%%%%%%%%%%%%%%%%%%%%%%%%%%%%%%%%%%%
\DescribeMacro{\childdocmain}
To use the package, add the commands
\begin{center}
\begin{tabular}{l}
|\input{childdoc.def}|\\
|\childdocmain{}|\\
\end{tabular}
\end{center}
at the very top of the main \LaTeX{} file,
in particular \emph{before} the |\documentclass| statement!
The argument of |\childdocmain| should be left empty
(but it must be present).

%%%%%%%%%%%%%%%%%%%%%%%%%%%%%%%%%%%%%%%%
\DescribeMacro{\childdocof}
Furthermore, add the commands
\begin{center}
\begin{tabular}{l}
|\input{childdoc.def}|\\
|\childdocof{|\textit{main}|}|\\
\end{tabular}
\end{center}
at the top of every child file \textit{child}
which is included by |\include{|\textit{child}|}|
from within the main file
(or at least for those files to be compiled individually).
The argument \textit{main} must be the filename of the main file.

There are a couple of
considerations in setting up the main and child documents:

%%%%%%%%%%%%%%%%%%%%%%%%%%%%%%%%%%%%%%%%
\paragraph{Restrictions.}

Please note the following restrictions:
\begin{itemize}
\item
|\childdocmain| must be called with one argument \textit{main}
to ensure compatibility with earlier version of the package.
It must either be empty (|\childdocmain{}|)
or precisely match the filename of the main file in which it is specified.
See \secref{sec:detection} for further information.
\item
The filename \textit{main} must be specified without the |.tex| extension.
\item
The filename \textit{main} is case sensitive
(even in case-insensitive file systems)
due to internal string comparison.
\item
The argument \textit{main} should be fully expanded, it cannot be a macro.
\item
Subdirectories and special characters should be avoided in filenames.
\item
The command |\childdocmain{|\textit{main}|}| must be followed by a whitespace.
It should not be followed immediately by another command
or by a comment mark `|%|'.
This is because the \TeX{} parser reads the token immediately following
the argument of |\childdocmain| and puts it
at the beginning of every child section;
however, a white\-space is ignored.
\end{itemize}

%%%%%%%%%%%%%%%%%%%%%%%%%%%%%%%%%%%%%%%%
\paragraph{Content of Main File.}

It is advisable to place all content in the child files included by |\include|.
Any output contained in the main file will appear in all child documents
unless suppressed manually;
it cannot be suppressed automatically by the |\includeonly| directive
and thus should normally be avoided.
A method to include some content in the main file
by means of conditional processing is described in \secref{sec:conditional}.

%%%%%%%%%%%%%%%%%%%%%%%%%%%%%%%%%%%%%%%%
\paragraph{Page Numbering.}

When only a part of the document is compiled,
the appropriate numbering of pages
(as well as other status parameters)
is determined from the |.aux| files.
The latter contain information from previous passes.
However this information needs to propagate through
all intermediate child documents.
Therefore the page numbering in child documents may well
be inconsistent until the complete document is compiled at least once.

A useful (if unconventional) way to always ensure a consistent
page numbering is to restart the numbering in each child document
and denote the pages by `\textit{child}|.|\textit{page}'
where \textit{child} represents the chapter/section number of the child file.
This can be achieved by the command
|\numberwithin{page}{|\textit{child}|}|
of the \textsf{amsmath} package
where \textit{child} can be |chapter| or |section|
depending on the chosen structuring.
Alternatively, one can modify the macro |\thepage| appropriately
and reset the counter |page| at the start of each child file.

%%%%%%%%%%%%%%%%%%%%%%%%%%%%%%%%%%%%%%%%%%%%%%%%%%%%%%%%%%%%%%%%%%%%%%%%%%%%%%%%
\subsection{Conditional Processing}
\label{sec:conditional}

The package provides a mechanism to compile different versions
of a document. To customise the versions further some conditional processing
can come in handy to distinguish which version is being compiled.
The package provides two macros to describe the compilation context:

%%%%%%%%%%%%%%%%%%%%%%%%%%%%%%%%%%%%%%%%
\DescribeMacro{\ifchilddoc}
The conditional |\ifchilddoc| distinguishes between the compilation of
child documents and the main document:
%
\begin{center}
|\ifchilddoc |\textit{child-code}| |[|\||else |\textit{main-code}]| \||fi|
\end{center}

%%%%%%%%%%%%%%%%%%%%%%%%%%%%%%%%%%%%%%%%
\DescribeMacro{\childdocname}
\DescribeMacro{\childdocjob}
The macro |\childdocname| contains the filename (without extension)
of the main or child file being processed.
Note that |\childdocjob| will always contain the name of the main file.

%%%%%%%%%%%%%%%%%%%%%%%%%%%%%%%%%%%%%%%%
\paragraph{Title Page.}

Conditional processing can be used to include a title or banner page
in the main document when proper precautions are taken.
Importantly, the code in the main file should ensure that the page counter
(as well as other status parameters which are stored in the |.aux| files)
takes the same value after the conditional processing.
Otherwise the page numbers may take divergent values
depending on which part is compiled.

For example, a title page could be declared by:
%
\begin{center}
\begin{tabular}{l}
|\ifchilddoc\||else|\\
|\addtocounter{page}{-1}|\\
\textit{code for title page}\\
|\newpage|\\
|\||fi|
\end{tabular}
\end{center}
%
A banner page for the child documents can be generated by:
%
\begin{center}
\begin{tabular}{l}
|\ifchilddoc|\\
|\addtocounter{page}{-1}|\\
\textit{code for banner page}\\
|\newpage|\\
|\||fi|
\end{tabular}
\end{center}
%
Here one could write a message such as:
\begin{center}
|This is the part \childdocname{} of \childdocjob{}.|
\end{center}

%%%%%%%%%%%%%%%%%%%%%%%%%%%%%%%%%%%%%%%%%%%%%%%%%%%%%%%%%%%%%%%%%%%%%%%%%%%%%%%%
\subsection{Flags}
\label{sec:flags}

The package makes it easy to generate different versions
of the main or child documents.
To this end compilation flags can be defined
and assigned different default values.
They will be particularly useful in conjunction
with the forwarding mechanism described in \secref{sec:forward}.

For example, it may be useful to have a flag |\version|
which can be set to |draft| or |final|.
The document source will contain some conditional code
depending on the value of |\version|.
Suppose further, the flag should default to |final| for the main file
and to |draft| for child files
which is a natural assignment for editing the document.
This is achieved by placing the following code
in the preamble of the main document
(below the |\childdocmain| directive):
%
\begin{center}
\begin{tabular}{l}
|\ifchilddoc|\\
|\providecommand{\version}{draft}|\\
|\||else|\\
|\providecommand{\version}{final}|\\
|\||fi|
\end{tabular}
\end{center}
%
The definition by |\providecommand| makes sure
that previous definitions are not overwritten.
Further statements |\providecommand{\version}{...}|
can thus be added before the above code to override it.

For the main file, one might add a line
(between |\childdocmain| and the above block)
%
\begin{center}
|%\ifchilddoc\||else\providecommand{\version}{draft}\||fi|
\end{center}
%
which can be uncommented to produce a draft version.
Likewise one can add a line to the very top of a child file
(above the |\childdocof{|\textit{main}|}| directive)
%
\begin{center}
|%\providecommand{\version}{final}|
\end{center}
%
which can be uncommented to produce the final version of this child document.

%%%%%%%%%%%%%%%%%%%%%%%%%%%%%%%%%%%%%%%%%%%%%%%%%%%%%%%%%%%%%%%%%%%%%%%%%%%%%%%%
\subsection{Forwarding}
\label{sec:forward}

Different versions of the main or child documents
using compilation flags as described in \secref{sec:flags}
can be (permanently) stored in different files
for convenient compilation, viewing and distribution.
To this end, the package defines a command
to pass on compilation to a different file:

%%%%%%%%%%%%%%%%%%%%%%%%%%%%%%%%%%%%%%%%
\DescribeMacro{\childdocforward}
The command |\childdocforward| redirects processing to
another source file:
%
\begin{center}
\begin{tabular}{l}
|\input{childdoc.def}|\\
|\childdocforward[|\textit{main}|]{|\textit{dest}|}|\\
\end{tabular}
\end{center}
%
The argument \textit{dest} is the destination file
(without extension).
It should be the main file or one of the child files.
Note that further \textsf{childdoc} directives
such as |\childdocof| and |\childdocforward|
in the indicated file will be processed in this form.
The optional argument \textit{main}
passes on directly to the main file \textit{main}
while pretending to compile the child \textit{dest}.
This form behaves as if \textit{dest}
issues |\childdocof{|\textit{main}|}| right away,
and no further \textsf{childdoc} directives will be processed.

%%%%%%%%%%%%%%%%%%%%%%%%%%%%%%%%%%%%%%%%
\DescribeMacro{\...prefix}
In the alternative form |\childdocforwardprefix|,
%
\begin{center}
\begin{tabular}{l}
|\input{childdoc.def}|\\
|\childdocforwardprefix[|\textit{main}|]{|\textit{prefix}|}{|\textit{dest}|}|
\end{tabular}
\end{center}
%
the destination file is determined by a pattern
depending on the current file:
To make this work, the current file must be called
`{\textit{prefix}\hspace{0.2em}\textit{suffix}}'
with \textit{prefix} matching precisely the argument.
Processing is then passed on to the file
`{\textit{dest}\hspace{0.2em}\textit{suffix}}'.
Surely, the same effect is achieved by
directly specifying the
argument `{\textit{dest}\hspace{0.2em}\textit{suffix}}'
in the first form.
However, that requires to set up a different file
for each child. With the alternative form of the command
all these files can have exactly the same content
which simplifies setting them up and maintaining them.

For example, the following file |draft.tex|
with a compilation flag |\version| as described in \secref{sec:flags}
compiles the main document as a draft:
%
\begin{center}
\begin{tabular}{l}
|\def\version{draft}|\\
|\input{childdoc.def}|\\
|\childdocforward{|\textit{main}|}|
\end{tabular}
\end{center}
%
Likewise, the following files |final|\textit{nn}|.tex|
compile the final version of the child document
|child|\textit{nn}|.tex|:
%
\begin{center}
\begin{tabular}{l}
|\def\version{final}|\\
|\input{childdoc.def}|\\
|\childdocforwardprefix{final}{child}|
\end{tabular}
\end{center}
%

Note that when several versions of a main file and/or of each child file
are to be generated, it may be convenient to set up a |Makefile| or
shell script to automatise the process.

%%%%%%%%%%%%%%%%%%%%%%%%%%%%%%%%%%%%%%%%%%%%%%%%%%%%%%%%%%%%%%%%%%%%%%%%%%%%%%%%
\subsection{Command Line Processing}
\label{sec:commandline}

The effect of redirection files can also be achieved by invoking
the \LaTeX{} compiler with a more elaborate command line.
Most conveniently this should be done as part
of a shell script or a |Makefile|.

When using \textsf{childdoc} in the main file, the following
command lines effectively perform a redirection
(note that depending on the shell being used,
backslashes may have to be doubled: `|\|' $\to$ `|\\|'):
%
\begin{center}
|... -jobname "|\textit{target}|" |\\|"|[\textit{flags}]%
|\input{childdoc.def}\childdocforward[|\textit{main}|]{|\textit{dest}|}"|
\end{center}
%
Here \textit{target} is the name of the output file,
\textit{main} is the name of the main file
and \textit{dest} is the name of the main or child file to be processed
(all filenames without extensions).
The optional argument \textit{main} can be omitted
if \textit{main} matches \textit{dest}.
Optionally, compilation \textit{flags} can be defined via |\def| commands.
This command line makes the \TeX{} engine believe
it is compiling the file \textit{target}
whose content is specified as the latter parameter.
The provided code then forwards the processing to
\textit{main} or \textit{dest} as described in \secref{sec:forward}.

%%%%%%%%%%%%%%%%%%%%%%%%%%%%%%%%%%%%%%%%%%%%%%%%%%%%%%%%%%%%%%%%%%%%%%%%%%%%%%%%
\subsection{Include by Input}
\label{sec:input}

Including child documents by |\include| has some restrictions by design.
Most notably, the content of a child document always occupies
its own set of pages; pages cannot be shared between child documents.
Usually, this behaviour makes perfect sense
because each child document contain an essential part of the document.
However, in some situations it may be desirable to compose
a document from a collection of parts
without having mandatory page breaks between then.
For this case, the package
provides a mechanism to include parts
by |\input| which can also be processed individually.
However, by construction this mechanism
requires manual handling of the content to be output.

%%%%%%%%%%%%%%%%%%%%%%%%%%%%%%%%%%%%%%%%
\DescribeMacro{\ifchilddocmanual}
The main file should be prepared as usual, see \secref{sec:include}.
However, the document body must make a distinction
between processing of an individual part and of the main document, e.g.:
%
\begin{center}
\begin{tabular}{l}
|\ifchilddocmanual|\\
|\input{\childdocname}|\\
|\||else|\\
\textit{document body with }|\input{|\textit{part}|}|\\
|\||fi|
\end{tabular}
\end{center}
%
The conditional |\ifchilddocmanual| is true whenever
a part to be included by |\input| is being compiled,
and the name of the part is stored in |\childdocname|.

%%%%%%%%%%%%%%%%%%%%%%%%%%%%%%%%%%%%%%%%
\DescribeMacro{\childdocby}
Each part to be included by |\input| should start with:
%
\begin{center}
\begin{tabular}{l}
|\input{childdoc.def}|\\
|\childdocby{|\textit{main}|}|\\
\end{tabular}
\end{center}
%
The directive |\childdocby| is similar to |\childdocof|
described in \secref{sec:include},
but the subsequent selection of content must be done manually.
To that end, both |\ifchilddoc| and |\ifchilddocmanual|
will be true upon processing of a part,
and the name of the part is stored in |\childdocname|.
Note that |\jobname| will be set to the filename of the current part
so that each part receives an individual |.aux| file
that does not interfere with the |.aux| file(s) of the main document.
This behaviour can be altered by the alternative form
|\childdocby[*]{|\textit{main}|}| (with a non-empty optional argument)
which uses the |.aux| file of the main document
by setting |\jobname| to \textit{main}.

%%%%%%%%%%%%%%%%%%%%%%%%%%%%%%%%%%%%%%%%%%%%%%%%%%%%%%%%%%%%%%%%%%%%%%%%%%%%%%%%
\subsection{Driver Development}
\label{sec:driver}

The \textsf{childdoc} mechanism can also be use for the development
of definition files such as \LaTeX{} styles or classes.
This case differs from the above setup with multiple parts
included by |\include| in that no |\includeonly| should be invoked.
This can be achieved by starting the include file
(before |\ProvidesPackage|) with:
%
\begin{center}
\begin{tabular}{l}
|\input{childdoc.def}|\\
|\childdocforward{|\textit{main}|}|\\
\end{tabular}
\end{center}
%
or alternatively with:
%
\begin{center}
\begin{tabular}{l}
|\input{childdoc.def}|\\
|\childdocby{|\textit{main}|}|\\
\end{tabular}
\end{center}
%
Both forms have slightly different effects as described above.
The main file is prepared as usual, see \secref{sec:include}.

%%%%%%%%%%%%%%%%%%%%%%%%%%%%%%%%%%%%%%%%%%%%%%%%%%%%%%%%%%%%%%%%%%%%%%%%%%%%%%%%
\subsection{Legacy Detection}
\label{sec:detection}

The directive |\childdocmain| in the main file can detect
whether the complete document or merely a child is to be compiled
even without using the directive |\childdocof|.
This method is deprecated because it is less robust
and there is no compelling reason to use it;
it is merely provided for backward compatibility
and it may be removed in future versions.

If the detection mechanism is to be used,
it is mandatory to correctly specify
the filename of the main file as the argument of |\childdocmain|:
%
\begin{center}
\begin{tabular}{l}
|\input{childdoc.def}|\\
|\childdocmain{|\textit{main}|}|\\
\end{tabular}
\end{center}
%
If |\jobname| does not match the argument \textit{main} of |\childdocmain|,
it is assumed that |\jobname| points to the child file to be compiled.
When using |\childdocmain| with the main file specified as argument,
it suffices to start a child file
with just |\input{|\textit{main}|}|
without loading of the package and using |\childdocof|.
If instead all processing is done
with the appropriate \textsf{childdoc} directives,
the argument of \textit{main} of |\childdocmain| can be empty.

An alternative version of the command line processing described
in \secref{sec:commandline} using the detection mechanism reads:
%
\begin{center}
|... -jobname "|\textit{target}|" "|[\textit{flags}]%
[|\def\jobname{|\textit{dest}|}|]|\input{|\textit{main}|}"|
\end{center}

%%%%%%%%%%%%%%%%%%%%%%%%%%%%%%%%%%%%%%%%%%%%%%%%%%%%%%%%%%%%%%%%%%%%%%%%%%%%%%%%
\subsection{Manual Code}
\label{sec:manual}

In case one cannot be certain whether the definitions file |childdoc.def|
is installed on the target \TeX{} distribution
and one prefers not to ship it,
it is conceivable to paste a few relevant commands into the sources.

To that end, drop all statements |\input{childdoc.def}|
and perform the replacements as outlined below.
Instead of |\childdocmain{|\textit{main}|}| add the following code
to the top of the main file:
%
\begin{center}
\begin{tabular}{l}
|\||ifdefined\childdocname\endinput\||fi\newif\ifchilddoc|\\
|\edef\childdocname{\scantokens\expandafter{\jobname\noexpand}}|\\
|\def\childdocmain{|\textit{main}|}\||ifx\childdocmain\childdocname\||else|\\
|\childdoctrue\includeonly{\childdocname}\let\jobname\childdocmain\||fi|\\
\end{tabular}
\end{center}
%
Instead of |\childdocof{|\textit{main}|}| just include the main file
at the top of each child file:
%
\begin{center}
|\input{|\textit{main}|}|
\end{center}
%
A simple redirection |\childdocforward{|\textit{dest}|}| is achieved by:
%
\begin{center}
|\def\jobname{|\textit{dest}|}\input{\jobname}|
\end{center}
%
The redirection with prefix
|\childdocforwardprefix[|\textit{prefix}|]{|\textit{dest}|}|
is accomplished by:
%
\begin{center}
\begin{tabular}{l}
|{\edef\jobname{\scantokens\expandafter{\jobname\noexpand}}|\\
|\def\redirectjob |\textit{prefix}|#1~~~{\gdef\jobname{|\textit{dest}|#1}}|\\
|\expandafter\redirectjob\jobname~~~}\input{\jobname}|
\end{tabular}
\end{center}

In an alternative approach,
child documents can be compiled by a specific command line
without additional code or specific definitions:
%
\begin{center}
|... -jobname "|\textit{target}|" "|[\textit{flags}]%
|\includeonly{|\textit{dest}|}\input{|\textit{main}|}"|
\end{center}
%

%%%%%%%%%%%%%%%%%%%%%%%%%%%%%%%%%%%%%%%%%%%%%%%%%%%%%%%%%%%%%%%%%%%%%%%%%%%%%%%%
%%%%%%%%%%%%%%%%%%%%%%%%%%%%%%%%%%%%%%%%%%%%%%%%%%%%%%%%%%%%%%%%%%%%%%%%%%%%%%%%
\section{Information}

%%%%%%%%%%%%%%%%%%%%%%%%%%%%%%%%%%%%%%%%%%%%%%%%%%%%%%%%%%%%%%%%%%%%%%%%%%%%%%%%
\subsection{Copyright}

Copyright \copyright{} 2017--2018 Niklas Beisert

This work may be distributed and/or modified under the
conditions of the \LaTeX{} Project Public License, either version 1.3
of this license or (at your option) any later version.
The latest version of this license is in
  \url{http://www.latex-project.org/lppl.txt}
and version 1.3 or later is part of all distributions of \LaTeX{}
version 2005/12/01 or later.

This work has the LPPL maintenance status `maintained'.

The Current Maintainer of this work is Niklas Beisert.

This work consists of the files |README.txt|, |childdoc.ins| and |childdoc.dtx|
as well as the derived files |childdoc.def|, |cdocsamp.tex|
with |cdocsch1.tex|, |cdocsch2.tex|, |cdocspt3.tex|, |cdocspt4.tex|,
|cdocsdrf.tex|, |cdocsfn1.tex|, |cdocsfn2.tex|
as well as |childdoc.pdf|.

%%%%%%%%%%%%%%%%%%%%%%%%%%%%%%%%%%%%%%%%%%%%%%%%%%%%%%%%%%%%%%%%%%%%%%%%%%%%%%%%
\subsection{Files and Installation}

The package consists of the files:
%
\begin{center}
\begin{tabular}{ll}
    |README.txt|   & readme file \\
    |childdoc.ins| & installation file \\
    |childdoc.dtx| & source file \\
    |childdoc.def| & definition file \\
    |cdocsamp.tex| & sample main file \\
    |cdocsch1.tex| & sample include file \\
    |cdocsch2.tex| & sample include file \\
    |cdocspt3.tex| & sample part file \\
    |cdocspt4.tex| & sample part file \\
    |cdocsdrf.tex| & sample redirection file \\
    |cdocsfn1.tex| & sample redirection file \\
    |cdocsfn2.tex| & sample redirection file \\
    |childdoc.pdf| & manual
\end{tabular}
\end{center}
%
The distribution consists of the files
|README.txt|, |childdoc.ins| and |childdoc.dtx|.
%
\begin{itemize}
\item
Run (pdf)\LaTeX{} on |childdoc.dtx|
to compile the manual |childdoc.pdf| (this file).
\item
Run \LaTeX{} on |childdoc.ins| to create the definitions file |childdoc.def|
and the sample |cdocsamp.tex| with include files
|cdocsch1.tex|, |cdocsch2.tex|, |cdocspt3.tex|, |cdocspt4.tex|,
|cdocsdrf.tex|, |cdocsfn1.tex|, |cdocsfn2.tex|.
Then copy the file |childdoc.def| to an appropriate directory of your \LaTeX{}
distribution, e.g.\ \textit{texmf-root}|/tex/latex/childdoc|.
\end{itemize}

%%%%%%%%%%%%%%%%%%%%%%%%%%%%%%%%%%%%%%%%%%%%%%%%%%%%%%%%%%%%%%%%%%%%%%%%%%%%%%%%
\subsection{Related CTAN Packages}

There are several other packages which offer a similar functionality:
%
\begin{itemize}
\item
The packages
\href{http://ctan.org/pkg/docmute}{\textsf{docmute}},
\href{http://ctan.org/pkg/includex}{\textsf{includex}} and
\href{http://ctan.org/pkg/standalone}{\textsf{standalone}}
provide commands to include only the document body of
a child file thus allowing both files to be compiled individually.
\item
The packages \href{http://ctan.org/pkg/subdocs}{\textsf{subdocs}}
and \href{http://ctan.org/pkg/subfiles}{\textsf{subfiles}}
provide structures in which the main and child documents can be
encapsulated and allowing them to be compiled individually.
The inclusion mechanism is different from the conventional |\include|.
\item
The package \href{http://ctan.org/pkg/combine}{\textsf{combine}}
is an elaborate solution to combine several documents into one.
\end{itemize}
%
See also the CTAN topic \href{http://ctan.org/topic/subdocs}{\textsf{subdocs}}
for further related packages.
The present package differs from the above solutions in that
a document structure constructed with the conventional |\include| mechanism
just needs two extra commands at the top of every file
such that all constituent files can be compiled individually.

%%%%%%%%%%%%%%%%%%%%%%%%%%%%%%%%%%%%%%%%%%%%%%%%%%%%%%%%%%%%%%%%%%%%%%%%%%%%%%%%
%\subsection{Feature Suggestions}
%
%The following is a list of features which may be useful for future
%versions of this package:
%%
%\begin{itemize}
%\item
%\ldots
%\end{itemize}

%%%%%%%%%%%%%%%%%%%%%%%%%%%%%%%%%%%%%%%%%%%%%%%%%%%%%%%%%%%%%%%%%%%%%%%%%%%%%%%%
\subsection{Revision History}

%%%%%%%%%%%%%%%%%%%%%%%%%%%%%%%%%%%%%%%%
\paragraph{v2.0:} 2018/12/30

\begin{itemize}
\item
immediate forward processing
\item
added |\childdocby| mechanism
\item
manual restructured
\end{itemize}

%%%%%%%%%%%%%%%%%%%%%%%%%%%%%%%%%%%%%%%%
\paragraph{v1.6:} 2018/01/17

\begin{itemize}
\item
application for development of include files
\item
corrections to manual
\end{itemize}

%%%%%%%%%%%%%%%%%%%%%%%%%%%%%%%%%%%%%%%%
\paragraph{v1.5:} 2017/05/21

\begin{itemize}
\item
more complete structuring introduced
\item
|\childdocof| introduced
\item
|\childdoc| renamed to |\childdocmain|
\item
|\childredirect| renamed to |\childdocforward| and |\childdocforwardprefix|
and functionality expanded
\end{itemize}

%%%%%%%%%%%%%%%%%%%%%%%%%%%%%%%%%%%%%%%%
\paragraph{v1.0:} 2017/04/27

\begin{itemize}
\item
manual and install package
\item
first version published on CTAN
\end{itemize}

%%%%%%%%%%%%%%%%%%%%%%%%%%%%%%%%%%%%%%%%
\paragraph{v0.6:} 2017/04/26

\begin{itemize}
\item
redirection mechanism added
\end{itemize}

%%%%%%%%%%%%%%%%%%%%%%%%%%%%%%%%%%%%%%%%
\paragraph{v0.5:} 2017/04/26

\begin{itemize}
\item
functionality in definition file
\end{itemize}


%%%%%%%%%%%%%%%%%%%%%%%%%%%%%%%%%%%%%%%%%%%%%%%%%%%%%%%%%%%%%%%%%%%%%%%%%%%%%%%%
%%%%%%%%%%%%%%%%%%%%%%%%%%%%%%%%%%%%%%%%%%%%%%%%%%%%%%%%%%%%%%%%%%%%%%%%%%%%%%%%
%%%%%%%%%%%%%%%%%%%%%%%%%%%%%%%%%%%%%%%%%%%%%%%%%%%%%%%%%%%%%%%%%%%%%%%%%%%%%%%%
\appendix

\settowidth\MacroIndent{\rmfamily\scriptsize 000\ }

 \DocInput{childdoc.dtx}

\end{document}
%</driver>
% \fi
%
% %%%%%%%%%%%%%%%%%%%%%%%%%%%%%%%%%%%%%%%%%%%%%%%%%%%%%%%%%%%%%%%%%%%%%%%%%%%%%%
% %%%%%%%%%%%%%%%%%%%%%%%%%%%%%%%%%%%%%%%%%%%%%%%%%%%%%%%%%%%%%%%%%%%%%%%%%%%%%%
% \section{Sample}
%\iffalse
%<*samplemain>
%\fi
%
% The following presents a sample document
% with two chapters, two parts, a title page,
% a compile flag as well as three forwarding files to set the flag.
% It consists of eight |.tex| files:
% \begin{center}
% \begin{tabular}{ll}
% |cdocsamp.tex|&main file\\
% |cdocsch1.tex|&include file for chapter 1\\
% |cdocsch2.tex|&include file for chapter 2\\
% |cdocspt3.tex|&include file for part 3\\
% |cdocspt4.tex|&include file for part 4\\
% |cdocsdrf.tex|&forwarding file for main file in draft mode\\
% |cdocsfi1.tex|&forwarding file for final version of chapter 1\\
% |cdocsfi2.tex|&forwarding file for final version of chapter 2\\
% \end{tabular}
% \end{center}
% Each of the eight files can be compiled directly by the \LaTeX{} compiler.
%
% %%%%%%%%%%%%%%%%%%%%%%%%%%%%%%%%%%%%%%
% \paragraph{Main File.}
%
% The main file is called |cdocsamp.tex|.
%
% Load the \textsf{childdoc} definitions and
% declare the filename for the main document:
%    \begin{macrocode}
\input{childdoc.def}
\childdocmain{}
%    \end{macrocode}

% Optional override for |\version| flag:
%    \begin{macrocode}
%%\ifchilddoc\else\providecommand{\version}{draft}\fi
%    \end{macrocode}

% Define the default values for the |\version| flag
% (|final| for the main file and |draft| for childs):
%    \begin{macrocode}
\ifchilddoc
\providecommand{\version}{draft}
\else
\providecommand{\version}{final}
\fi
%    \end{macrocode}

% Load the standard document class:
%    \begin{macrocode}
\documentclass[12pt]{article}
%    \end{macrocode}

% Start the document body:
%    \begin{macrocode}
\begin{document}
%    \end{macrocode}

% Declare a title page.
% Print title, part of document being processed and version flag:
%    \begin{macrocode}
\addtocounter{page}{-1}
\begin{center}
{\LARGE\bfseries{}childdoc example\par}
\vspace{1cm}
\ifchilddoc
\ifchilddocmanual part\else chapter\fi:
`\childdocname' of `\childdocjob'\par
\else
main document: `\childdocjob'\par
\fi
version: \version\par
\end{center}
\newpage
%    \end{macrocode}

% Manually include selected file,
% otherwise process as usual:
%    \begin{macrocode}
\ifchilddocmanual
\section*{part `\childdocname'}
\input{\childdocname}
\else
%    \end{macrocode}

% Include the two chapters:
%    \begin{macrocode}
\include{cdocsch1}
\include{cdocsch2}
%    \end{macrocode}

% Include the two parts unless only chapters should be displayed:
%    \begin{macrocode}
\ifchilddoc\else
\section{part three}
\input{cdocspt3}
\section{part four}
\input{cdocspt4}
\fi
%    \end{macrocode}

% Process as usual until here:
%    \begin{macrocode}
\fi
%    \end{macrocode}

% End of document body:
%    \begin{macrocode}
\end{document}
%    \end{macrocode}
%\iffalse
%</samplemain>
%\fi
%
% %%%%%%%%%%%%%%%%%%%%%%%%%%%%%%%%%%%%%%
% \paragraph{Chapter Include Files.}
%
% The include files are called |cdocsch1.tex| and |cdocsch2.tex|.
%
%\iffalse
%<*samplechap1|samplechap2>
%\fi

% Optional override for |\version| flag:
%    \begin{macrocode}
%%\providecommand{\version}{final}
%    \end{macrocode}

% Include the main document:
%    \begin{macrocode}
\input{childdoc.def}
\childdocof{cdocsamp}
%    \end{macrocode}

%\iffalse
%</samplechap1|samplechap2>
%\fi
%
%\iffalse
%<*samplechap1>
%\fi
% Some text for chapter 1:
%    \begin{macrocode}
\section{one}
some text in chapter one
%    \end{macrocode}

%\iffalse
%</samplechap1>
%\fi
% Some text for chapter 2:
%\iffalse
%<*samplechap2>
%\fi
%    \begin{macrocode}
\section{two}
more text in chapter two
%    \end{macrocode}

%\iffalse
%</samplechap2>
%\fi
%
% %%%%%%%%%%%%%%%%%%%%%%%%%%%%%%%%%%%%%%
% \paragraph{Part Include Files.}
%
% The include files are called |cdocspt3.tex| and |cdocspt4.tex|.
%
%\iffalse
%<*samplepart3|samplepart4>
%\fi

% Optional override for |\version| flag:
%    \begin{macrocode}
%%\providecommand{\version}{final}
%    \end{macrocode}

% Include the main document:
%    \begin{macrocode}
\input{childdoc.def}
\childdocby{cdocsamp}
%    \end{macrocode}

%\iffalse
%</samplepart3|samplepart4>
%\fi
%
%\iffalse
%<*samplepart3>
%\fi
% Some text for part 3:
%    \begin{macrocode}
some text in part three
%    \end{macrocode}

%\iffalse
%</samplepart3>
%\fi
% Some text for part 4:
%\iffalse
%<*samplepart4>
%\fi
%    \begin{macrocode}
more text in part four
%    \end{macrocode}

%\iffalse
%</samplepart4>
%\fi
%
% %%%%%%%%%%%%%%%%%%%%%%%%%%%%%%%%%%%%%%
% \paragraph{Forwarding for a Complete Draft.}
%
% The following forwarding file |cdocsdrf.tex|
% compiles the main document in draft mode:
%\iffalse
%<*sampledraft>
%\fi
%    \begin{macrocode}
\def\version{draft}
\input{childdoc.def}
\childdocforward{cdocsamp}
%    \end{macrocode}

%\iffalse
%</sampledraft>
%\fi
%
% %%%%%%%%%%%%%%%%%%%%%%%%%%%%%%%%%%%%%%
% \paragraph{Forwarding for Final Version of the Chapters.}
%
% The following forwarding files |cdocsfn1.tex| and |cdocsfn2.tex|
% (with identical content)
% compile the final versions of the child documents
% |cdocsch1.tex| and |cdocsch2.tex|, respectively:
%\iffalse
%<*samplefinal>
%\fi
%    \begin{macrocode}
\def\version{final}
\input{childdoc.def}
\childdocforwardprefix[cdocsamp]{cdocsfn}{cdocsch}
%    \end{macrocode}

%\iffalse
%</samplefinal>
%\fi
%
% %%%%%%%%%%%%%%%%%%%%%%%%%%%%%%%%%%%%%%
% \paragraph{Command Line Processing.}
%
% The following three command lines generate the output files
% |cdocscld|, |cdocscl1| and |cdocscl2|
% which should be identical to
% |cdocsdrf|, |cdocsch1| and |cdocsfn2|, respectively:
% \begin{center}
% \begin{tabular}{l}
% |latex -jobname cdocscld \|\\
% |  "\def\version{draft}\input{childdoc.def}\childdocforward{cdocsamp}"|\\
% |latex -jobname cdocscl1 \|\\
% |  "\input{childdoc.def}\childdocforward[cdocsamp]{cdocsch1}"|\\
% |latex -jobname cdocscl2 \|\\
% |  "\def\version{final}\input{childdoc.def}\childdocforward{cdocsch2}"|
% \end{tabular}
% \end{center}
% Note that the trailing backslash on each first line
% merely continues the input to the second line
% (for convenient cut ant paste).
% Furthermore, the command |latex| can be replaced by any
% of its alternative versions such as |pdflatex|.
%
% %%%%%%%%%%%%%%%%%%%%%%%%%%%%%%%%%%%%%%%%%%%%%%%%%%%%%%%%%%%%%%%%%%%%%%%%%%%%%%
% %%%%%%%%%%%%%%%%%%%%%%%%%%%%%%%%%%%%%%%%%%%%%%%%%%%%%%%%%%%%%%%%%%%%%%%%%%%%%%
% \section{Implementation}
%\iffalse
%<*package>
%\fi
%
% This section describes the definitions file |childdoc.def|.

% The definitions cannot be loaded using |\usepackage| or |\RequirePackage|
% which has a mechanism to prevent loading a style file more than once.
% When loading the definitions by means of |\input|
% multiple instances have to be prevented manually:
%\iffalse
%This code needs to be before the `\ProvidesFile' directive
%which is defined at the beginning of this file.
%Therefore it is also placed there and commented out here.
%</package>
%<*discard>
%\fi
%    \begin{macrocode}
\ifdefined\childdocmain\endinput\fi
%    \end{macrocode}
%\iffalse
%</discard>
%<*package>
%\fi
%
% \macro{\ifchilddoc}
% \macro{\ifchilddocmanual}
% The conditional |\ifchilddoc| tells whether a
% child (true) or main (false) document is being compiled.
% The conditional |\ifchilddocmanual| tells whether
% the |\includeonly| mechanism is used (false) or
% the selection of child files must be performed manually (true).
% The definitions initialise to false:
%    \begin{macrocode}
\newif\ifchilddoc
\newif\ifchilddocmanual
%    \end{macrocode}

% \macro{\childdocname}
% \macro{\childdocjob}
% The macro |\childdocname| stores the name of the main document
% to be compiled. The macro |\childdocjob| stores the name of
% the document on which the \LaTeX{} compiler was originally invoked.
% The content of |\jobname| cannot be compared
% to filenames specified in the source due to different catcodes.
% The following code rescans |\jobname|, stores the result
% in |\childdocname| and saves a copy in |\childdocjob|:
%    \begin{macrocode}
\edef\childdocname{\scantokens\expandafter{\jobname\noexpand}}
\let\childdocjob\childdocname
%    \end{macrocode}

% \macro{\childdocdisable}
% The macro |\childdocdisable| prevents the main file
% from being processed more than once.
% At this stage, the main document command |\childdocmain|
% is assumed to be called once again where it should do nothing.
% Any subsequent call to it should prevent
% a secondary processing of the main document
% It overwrites the forwarding commands
% |\childdocof| and |\childdocforward|
% with empty macros to prevent further inclusions of the main document:
%    \begin{macrocode}
\newcommand{\childdocdisable}
{
  \renewcommand{\childdocmain}[1]{\renewcommand{\childdocmain}[1]{\endinput}}
  \renewcommand{\childdocof}[1]{}
  \renewcommand{\childdocby}[2][]{}
  \renewcommand{\childdocforward}[2][]{}
  \renewcommand{\childdocdisable}{}
}
%    \end{macrocode}

% \macro{\childdocmain}
% The macro |\childdocmain| is to be called at the top of the main file
% with nothing or the main filename (without extension) as argument.
% First, it breaks loops.
% If the argument is not empty and does not match |\childdocname|
% (which is set by the first inclusion of |childdoc.def|),
% |\ifchilddoc| is set to true, |\includeonly| is applied to the child file
% and |\jobname| is set to the main file
% (for proper handling of |.aux| files):
%    \begin{macrocode}
\newcommand{\childdocmain}[1]
{
  \childdocdisable\childdocmain{}
  \if?#1?\else
    \begingroup
      \def\childdoctmp{#1}
      \ifx\childdoctmp\childdocname
        \def\childdoctmp{}
      \else
        \def\childdoctmp
        {
          \childdoctrue
          \includeonly{\childdocname}
          \def\childdocjob{#1}
          \def\jobname{#1}
        }
      \fi
      \expandafter
    \endgroup
    \childdoctmp
  \fi
}
%    \end{macrocode}

% \macro{\childdocof}
% The command |\childdocof| redirects
% compilation to the main file |#1|.
%    \begin{macrocode}
\newcommand{\childdocof}[1]
{
  \childdocdisable
  \childdoctrue
  \includeonly{\childdocname}
  \def\jobname{#1}
  \def\childdocjob{#1}
  \input{#1}
}
%    \end{macrocode}

% \macro{\childdocby}
% The command |\childdocby| ....
%    \begin{macrocode}
\newcommand{\childdocby}[2][]
{
  \childdocdisable
  \childdoctrue
  \childdocmanualtrue
  \if?#1?\else
    \def\jobname{#2}
  \fi
  \def\childdocjob{#2}
  \input{#2}
  \endinput
}
%    \end{macrocode}

% \macro{\childdocforward}
% The command |\childdocforward| redirects
% compilation to the main file or
% (if the optional argument is given) a child file.
% Parameters are set as if the main file
% or a child file starting with |\childdocof| was compiled.
% Then compilation is handed over to the main file:
%    \begin{macrocode}
\newcommand{\childdocforward}[2][]
{
  \begingroup
    \if?#1?
      \def\childdoctmp
      {
        \def\childdocname{#2}
        \def\childdocjob{#2}
        \def\jobname{#2}
        \input{#2}
        \endinput
      }
    \else
      \def\childdoctmp
      {
        \childdocdisable
        \def\childdocname{#2}
        \childdoctrue
        \includeonly{#2}
        \def\childdocjob{#1}
        \def\jobname{#1}
        \input{#1}
        \endinput
      }
    \fi
    \expandafter
  \endgroup
  \childdoctmp
}
%    \end{macrocode}

% \macro{\childdocforwardprefix}
% The command |\childdocforwardprefix| redirects
% compilation to the main or a child file by means of a pattern.
% The prefix |#1| in the current filename is replaced by |#2|
% and the suffix of the current filename is kept
% (it is assumed that the filename does not contain the substring `|~~~|'
% which is used as a delimiter).
% Compilation is handed over to the new file by |\childdocforward|:
%    \begin{macrocode}
\newcommand{\childdocforwardprefix}[3][]
{
  \begingroup
    \def\childdocextract #2##1~~~{\def\childdoctmp{\childdocforward[#1]{#3##1}}}
    \expandafter\childdocextract\childdocname~~~
    \expandafter
  \endgroup
  \childdoctmp
}
%    \end{macrocode}

% \macro{\childdoc}
% The deprecated macro |\childdoc| is a legacy version of |\childdocmain|:
%    \begin{macrocode}
\newcommand{\childdoc}{\childdocmain}
%    \end{macrocode}

% \macro{\childdocredirect}
% The deprecated macro |\childdocredirect| is a legacy version
% of |\childdocforward| and |\childdocforwardprefix|:
%    \begin{macrocode}
\newcommand{\childdocredirect}[2][]
{
  \begingroup
    \if?#1?
      \def\childdoctmp{\childdocforward{#2}}
    \else
      \def\childdoctmp{\childdocforwardprefix{#1}{#2}}
    \fi
    \expandafter
  \endgroup
  \childdoctmp
}
%    \end{macrocode}

%\iffalse
%</package>
%\fi
%
\endinput

\childdocforward{cdocsamp}
%    \end{macrocode}

%\iffalse
%</sampledraft>
%\fi
%
% %%%%%%%%%%%%%%%%%%%%%%%%%%%%%%%%%%%%%%
% \paragraph{Forwarding for Final Version of the Chapters.}
%
% The following forwarding files |cdocsfn1.tex| and |cdocsfn2.tex|
% (with identical content)
% compile the final versions of the child documents
% |cdocsch1.tex| and |cdocsch2.tex|, respectively:
%\iffalse
%<*samplefinal>
%\fi
%    \begin{macrocode}
\def\version{final}
% \iffalse
%
% childdoc.dtx Copyright (C) 2017-2018 Niklas Beisert
%
% This work may be distributed and/or modified under the
% conditions of the LaTeX Project Public License, either version 1.3
% of this license or (at your option) any later version.
% The latest version of this license is in
%   http://www.latex-project.org/lppl.txt
% and version 1.3 or later is part of all distributions of LaTeX
% version 2005/12/01 or later.
%
% This work has the LPPL maintenance status `maintained'.
%
% The Current Maintainer of this work is Niklas Beisert.
%
% This work consists of the files childdoc.dtx and childdoc.ins
% and the derived files childdoc.def and cdocsamp.tex with
% cdocsch1.tex, cdocsch2.tex, cdocsdrf.tex, cdocsfn1.tex, cdocsfn2.tex.
%
%<package>\ifdefined\childdocmain\endinput\fi
%<package>\ProvidesFile{childdoc.def}[2018/12/30 v2.0 child document driver]
%<samplemain>\ProvidesFile{cdocsamp.tex}[2018/12/30 v2.0 sample for childdoc]
%<*driver>
%\ProvidesFile{childdoc.drv}[2018/12/30 v2.0 childdoc reference manual file]
\PassOptionsToClass{10pt,a4paper}{article}
\documentclass{ltxdoc}

\usepackage[margin=35mm]{geometry}
\usepackage{hyperref}
\usepackage{hyperxmp}
\usepackage[usenames]{color}

\hypersetup{colorlinks=true}
\hypersetup{pdfstartview=FitH}
\hypersetup{pdfpagemode=UseNone}
\hypersetup{pdfsource={}}
\hypersetup{pdflang={en-UK}}
\hypersetup{pdfcopyright={Copyright 2017-2018 Niklas Beisert.
  This work may be distributed and/or modified under the
  conditions of the LaTeX Project Public License, either version 1.3
  of this license or (at your option) any later version.}}
\hypersetup{pdflicenseurl={http://www.latex-project.org/lppl.txt}}
\hypersetup{pdfcontactaddress={ETH Zurich, ITP, HIT K,
  Wolfgang-Pauli-Strasse 27}}
\hypersetup{pdfcontactpostcode={8093}}
\hypersetup{pdfcontactcity={Zurich}}
\hypersetup{pdfcontactcountry={Switzerland}}
\hypersetup{pdfcontactemail={nbeisert@itp.phys.ethz.ch}}
\hypersetup{pdfcontacturl={http://people.phys.ethz.ch/\xmptilde nbeisert/}}

\newcommand{\secref}[1]{\hyperref[#1]{section \ref*{#1}}}

\parskip1ex
\parindent0pt
\let\olditemize\itemize
\def\itemize{\olditemize\parskip0pt}

\begin{document}

\title{The \textsf{childdoc} Package}
\hypersetup{pdftitle={The childdoc Package}}
\author{Niklas Beisert\\[2ex]
  Institut f\"ur Theoretische Physik\\
  Eidgen\"ossische Technische Hochschule Z\"urich\\
  Wolfgang-Pauli-Strasse 27, 8093 Z\"urich, Switzerland\\[1ex]
  \href{mailto:nbeisert@itp.phys.ethz.ch}
  {\texttt{nbeisert@itp.phys.ethz.ch}}}
\hypersetup{pdfauthor={Niklas Beisert}}
\hypersetup{pdfsubject={Manual for the LaTeX2e Package childdoc}}
\date{30 December 2018, \textsf{v2.0}}
\maketitle

\begin{abstract}\noindent
\textsf{childdoc} is a \LaTeXe{} package
that enables the direct compilation
of document sections included by |\include|
to individual files.
\end{abstract}

\begingroup
\parskip0ex
\tableofcontents
\endgroup

%%%%%%%%%%%%%%%%%%%%%%%%%%%%%%%%%%%%%%%%%%%%%%%%%%%%%%%%%%%%%%%%%%%%%%%%%%%%%%%%
%%%%%%%%%%%%%%%%%%%%%%%%%%%%%%%%%%%%%%%%%%%%%%%%%%%%%%%%%%%%%%%%%%%%%%%%%%%%%%%%
\section{Introduction}

\LaTeX{} provides a mechanism to structure a large document (such as a book)
into a main file and several child files (containing the chapters)
using the |\include| command.
This mechanism is beneficial for documents
which span hundreds of pages in order to
make the source file(s) more manageable.
Moreover, compilation can be restricted to
selected child files by means of the |\includeonly| command.
The latter feature can be used to reduce the compilation time while editing
(this was significantly more useful in the earlier days of \LaTeX{})
or to generate a smaller document which is easier to navigate.
Another application of |\includeonly| is to generate
documents consisting of selected parts of the complete document.

However, there are a few drawbacks of the plain |\include| mechanism:
\begin{itemize}
\item
The child files cannot be compiled on their own,
they can only be compiled via the main file.
A naive editing environment
(such as a text editor with an option
to have the current file processed by \LaTeX)
may require one to switch to the main file before compiling;
attempting to compile the child file produces errors.
\item
The main file must be modified (each time)
to adjust the |\includeonly| command
to the present needs. This easily leaves the main file in a messy state.
\item
The generated document will always carry the filename
of the main document. This is inconvenient if
several child files are to be compiled and
to be kept for distribution.
\end{itemize}

The present package provides a simple interface
to make child files individually compilable by \LaTeX{}.
Compiling a child file then has the same effect as compiling
the main file with an |\includeonly| command
to select the appropriate child.
Moreover the generated document will carry the name of the child
rather than the main file.
This resolves all three above issues.

This feature is meant to make the editing of books,
thesis documents and lecture notes somewhat more convenient.
However, the package can also be used efficiently for
composing a series of documents (such as exercise sheets)
which are typically distributed individually.
It then assists the author in generating the individual documents
(potentially in different versions)
as well as a document containing the collected series.
Another application is in developing style files
or other kinds of included material
where compilation of the style file could redirect
to a sample or test file.

%%%%%%%%%%%%%%%%%%%%%%%%%%%%%%%%%%%%%%%%%%%%%%%%%%%%%%%%%%%%%%%%%%%%%%%%%%%%%%%%
%%%%%%%%%%%%%%%%%%%%%%%%%%%%%%%%%%%%%%%%%%%%%%%%%%%%%%%%%%%%%%%%%%%%%%%%%%%%%%%%
\section{Usage}

First of all, the package \textsf{childdoc} is \emph{not} a standard
\LaTeXe{} |.sty| style file! Therefore it needs to be invoked in
a non-standard way.

%%%%%%%%%%%%%%%%%%%%%%%%%%%%%%%%%%%%%%%%%%%%%%%%%%%%%%%%%%%%%%%%%%%%%%%%%%%%%%%%
\subsection{Included Files}
\label{sec:include}

%%%%%%%%%%%%%%%%%%%%%%%%%%%%%%%%%%%%%%%%
\DescribeMacro{\childdocmain}
To use the package, add the commands
\begin{center}
\begin{tabular}{l}
|\input{childdoc.def}|\\
|\childdocmain{}|\\
\end{tabular}
\end{center}
at the very top of the main \LaTeX{} file,
in particular \emph{before} the |\documentclass| statement!
The argument of |\childdocmain| should be left empty
(but it must be present).

%%%%%%%%%%%%%%%%%%%%%%%%%%%%%%%%%%%%%%%%
\DescribeMacro{\childdocof}
Furthermore, add the commands
\begin{center}
\begin{tabular}{l}
|\input{childdoc.def}|\\
|\childdocof{|\textit{main}|}|\\
\end{tabular}
\end{center}
at the top of every child file \textit{child}
which is included by |\include{|\textit{child}|}|
from within the main file
(or at least for those files to be compiled individually).
The argument \textit{main} must be the filename of the main file.

There are a couple of
considerations in setting up the main and child documents:

%%%%%%%%%%%%%%%%%%%%%%%%%%%%%%%%%%%%%%%%
\paragraph{Restrictions.}

Please note the following restrictions:
\begin{itemize}
\item
|\childdocmain| must be called with one argument \textit{main}
to ensure compatibility with earlier version of the package.
It must either be empty (|\childdocmain{}|)
or precisely match the filename of the main file in which it is specified.
See \secref{sec:detection} for further information.
\item
The filename \textit{main} must be specified without the |.tex| extension.
\item
The filename \textit{main} is case sensitive
(even in case-insensitive file systems)
due to internal string comparison.
\item
The argument \textit{main} should be fully expanded, it cannot be a macro.
\item
Subdirectories and special characters should be avoided in filenames.
\item
The command |\childdocmain{|\textit{main}|}| must be followed by a whitespace.
It should not be followed immediately by another command
or by a comment mark `|%|'.
This is because the \TeX{} parser reads the token immediately following
the argument of |\childdocmain| and puts it
at the beginning of every child section;
however, a white\-space is ignored.
\end{itemize}

%%%%%%%%%%%%%%%%%%%%%%%%%%%%%%%%%%%%%%%%
\paragraph{Content of Main File.}

It is advisable to place all content in the child files included by |\include|.
Any output contained in the main file will appear in all child documents
unless suppressed manually;
it cannot be suppressed automatically by the |\includeonly| directive
and thus should normally be avoided.
A method to include some content in the main file
by means of conditional processing is described in \secref{sec:conditional}.

%%%%%%%%%%%%%%%%%%%%%%%%%%%%%%%%%%%%%%%%
\paragraph{Page Numbering.}

When only a part of the document is compiled,
the appropriate numbering of pages
(as well as other status parameters)
is determined from the |.aux| files.
The latter contain information from previous passes.
However this information needs to propagate through
all intermediate child documents.
Therefore the page numbering in child documents may well
be inconsistent until the complete document is compiled at least once.

A useful (if unconventional) way to always ensure a consistent
page numbering is to restart the numbering in each child document
and denote the pages by `\textit{child}|.|\textit{page}'
where \textit{child} represents the chapter/section number of the child file.
This can be achieved by the command
|\numberwithin{page}{|\textit{child}|}|
of the \textsf{amsmath} package
where \textit{child} can be |chapter| or |section|
depending on the chosen structuring.
Alternatively, one can modify the macro |\thepage| appropriately
and reset the counter |page| at the start of each child file.

%%%%%%%%%%%%%%%%%%%%%%%%%%%%%%%%%%%%%%%%%%%%%%%%%%%%%%%%%%%%%%%%%%%%%%%%%%%%%%%%
\subsection{Conditional Processing}
\label{sec:conditional}

The package provides a mechanism to compile different versions
of a document. To customise the versions further some conditional processing
can come in handy to distinguish which version is being compiled.
The package provides two macros to describe the compilation context:

%%%%%%%%%%%%%%%%%%%%%%%%%%%%%%%%%%%%%%%%
\DescribeMacro{\ifchilddoc}
The conditional |\ifchilddoc| distinguishes between the compilation of
child documents and the main document:
%
\begin{center}
|\ifchilddoc |\textit{child-code}| |[|\||else |\textit{main-code}]| \||fi|
\end{center}

%%%%%%%%%%%%%%%%%%%%%%%%%%%%%%%%%%%%%%%%
\DescribeMacro{\childdocname}
\DescribeMacro{\childdocjob}
The macro |\childdocname| contains the filename (without extension)
of the main or child file being processed.
Note that |\childdocjob| will always contain the name of the main file.

%%%%%%%%%%%%%%%%%%%%%%%%%%%%%%%%%%%%%%%%
\paragraph{Title Page.}

Conditional processing can be used to include a title or banner page
in the main document when proper precautions are taken.
Importantly, the code in the main file should ensure that the page counter
(as well as other status parameters which are stored in the |.aux| files)
takes the same value after the conditional processing.
Otherwise the page numbers may take divergent values
depending on which part is compiled.

For example, a title page could be declared by:
%
\begin{center}
\begin{tabular}{l}
|\ifchilddoc\||else|\\
|\addtocounter{page}{-1}|\\
\textit{code for title page}\\
|\newpage|\\
|\||fi|
\end{tabular}
\end{center}
%
A banner page for the child documents can be generated by:
%
\begin{center}
\begin{tabular}{l}
|\ifchilddoc|\\
|\addtocounter{page}{-1}|\\
\textit{code for banner page}\\
|\newpage|\\
|\||fi|
\end{tabular}
\end{center}
%
Here one could write a message such as:
\begin{center}
|This is the part \childdocname{} of \childdocjob{}.|
\end{center}

%%%%%%%%%%%%%%%%%%%%%%%%%%%%%%%%%%%%%%%%%%%%%%%%%%%%%%%%%%%%%%%%%%%%%%%%%%%%%%%%
\subsection{Flags}
\label{sec:flags}

The package makes it easy to generate different versions
of the main or child documents.
To this end compilation flags can be defined
and assigned different default values.
They will be particularly useful in conjunction
with the forwarding mechanism described in \secref{sec:forward}.

For example, it may be useful to have a flag |\version|
which can be set to |draft| or |final|.
The document source will contain some conditional code
depending on the value of |\version|.
Suppose further, the flag should default to |final| for the main file
and to |draft| for child files
which is a natural assignment for editing the document.
This is achieved by placing the following code
in the preamble of the main document
(below the |\childdocmain| directive):
%
\begin{center}
\begin{tabular}{l}
|\ifchilddoc|\\
|\providecommand{\version}{draft}|\\
|\||else|\\
|\providecommand{\version}{final}|\\
|\||fi|
\end{tabular}
\end{center}
%
The definition by |\providecommand| makes sure
that previous definitions are not overwritten.
Further statements |\providecommand{\version}{...}|
can thus be added before the above code to override it.

For the main file, one might add a line
(between |\childdocmain| and the above block)
%
\begin{center}
|%\ifchilddoc\||else\providecommand{\version}{draft}\||fi|
\end{center}
%
which can be uncommented to produce a draft version.
Likewise one can add a line to the very top of a child file
(above the |\childdocof{|\textit{main}|}| directive)
%
\begin{center}
|%\providecommand{\version}{final}|
\end{center}
%
which can be uncommented to produce the final version of this child document.

%%%%%%%%%%%%%%%%%%%%%%%%%%%%%%%%%%%%%%%%%%%%%%%%%%%%%%%%%%%%%%%%%%%%%%%%%%%%%%%%
\subsection{Forwarding}
\label{sec:forward}

Different versions of the main or child documents
using compilation flags as described in \secref{sec:flags}
can be (permanently) stored in different files
for convenient compilation, viewing and distribution.
To this end, the package defines a command
to pass on compilation to a different file:

%%%%%%%%%%%%%%%%%%%%%%%%%%%%%%%%%%%%%%%%
\DescribeMacro{\childdocforward}
The command |\childdocforward| redirects processing to
another source file:
%
\begin{center}
\begin{tabular}{l}
|\input{childdoc.def}|\\
|\childdocforward[|\textit{main}|]{|\textit{dest}|}|\\
\end{tabular}
\end{center}
%
The argument \textit{dest} is the destination file
(without extension).
It should be the main file or one of the child files.
Note that further \textsf{childdoc} directives
such as |\childdocof| and |\childdocforward|
in the indicated file will be processed in this form.
The optional argument \textit{main}
passes on directly to the main file \textit{main}
while pretending to compile the child \textit{dest}.
This form behaves as if \textit{dest}
issues |\childdocof{|\textit{main}|}| right away,
and no further \textsf{childdoc} directives will be processed.

%%%%%%%%%%%%%%%%%%%%%%%%%%%%%%%%%%%%%%%%
\DescribeMacro{\...prefix}
In the alternative form |\childdocforwardprefix|,
%
\begin{center}
\begin{tabular}{l}
|\input{childdoc.def}|\\
|\childdocforwardprefix[|\textit{main}|]{|\textit{prefix}|}{|\textit{dest}|}|
\end{tabular}
\end{center}
%
the destination file is determined by a pattern
depending on the current file:
To make this work, the current file must be called
`{\textit{prefix}\hspace{0.2em}\textit{suffix}}'
with \textit{prefix} matching precisely the argument.
Processing is then passed on to the file
`{\textit{dest}\hspace{0.2em}\textit{suffix}}'.
Surely, the same effect is achieved by
directly specifying the
argument `{\textit{dest}\hspace{0.2em}\textit{suffix}}'
in the first form.
However, that requires to set up a different file
for each child. With the alternative form of the command
all these files can have exactly the same content
which simplifies setting them up and maintaining them.

For example, the following file |draft.tex|
with a compilation flag |\version| as described in \secref{sec:flags}
compiles the main document as a draft:
%
\begin{center}
\begin{tabular}{l}
|\def\version{draft}|\\
|\input{childdoc.def}|\\
|\childdocforward{|\textit{main}|}|
\end{tabular}
\end{center}
%
Likewise, the following files |final|\textit{nn}|.tex|
compile the final version of the child document
|child|\textit{nn}|.tex|:
%
\begin{center}
\begin{tabular}{l}
|\def\version{final}|\\
|\input{childdoc.def}|\\
|\childdocforwardprefix{final}{child}|
\end{tabular}
\end{center}
%

Note that when several versions of a main file and/or of each child file
are to be generated, it may be convenient to set up a |Makefile| or
shell script to automatise the process.

%%%%%%%%%%%%%%%%%%%%%%%%%%%%%%%%%%%%%%%%%%%%%%%%%%%%%%%%%%%%%%%%%%%%%%%%%%%%%%%%
\subsection{Command Line Processing}
\label{sec:commandline}

The effect of redirection files can also be achieved by invoking
the \LaTeX{} compiler with a more elaborate command line.
Most conveniently this should be done as part
of a shell script or a |Makefile|.

When using \textsf{childdoc} in the main file, the following
command lines effectively perform a redirection
(note that depending on the shell being used,
backslashes may have to be doubled: `|\|' $\to$ `|\\|'):
%
\begin{center}
|... -jobname "|\textit{target}|" |\\|"|[\textit{flags}]%
|\input{childdoc.def}\childdocforward[|\textit{main}|]{|\textit{dest}|}"|
\end{center}
%
Here \textit{target} is the name of the output file,
\textit{main} is the name of the main file
and \textit{dest} is the name of the main or child file to be processed
(all filenames without extensions).
The optional argument \textit{main} can be omitted
if \textit{main} matches \textit{dest}.
Optionally, compilation \textit{flags} can be defined via |\def| commands.
This command line makes the \TeX{} engine believe
it is compiling the file \textit{target}
whose content is specified as the latter parameter.
The provided code then forwards the processing to
\textit{main} or \textit{dest} as described in \secref{sec:forward}.

%%%%%%%%%%%%%%%%%%%%%%%%%%%%%%%%%%%%%%%%%%%%%%%%%%%%%%%%%%%%%%%%%%%%%%%%%%%%%%%%
\subsection{Include by Input}
\label{sec:input}

Including child documents by |\include| has some restrictions by design.
Most notably, the content of a child document always occupies
its own set of pages; pages cannot be shared between child documents.
Usually, this behaviour makes perfect sense
because each child document contain an essential part of the document.
However, in some situations it may be desirable to compose
a document from a collection of parts
without having mandatory page breaks between then.
For this case, the package
provides a mechanism to include parts
by |\input| which can also be processed individually.
However, by construction this mechanism
requires manual handling of the content to be output.

%%%%%%%%%%%%%%%%%%%%%%%%%%%%%%%%%%%%%%%%
\DescribeMacro{\ifchilddocmanual}
The main file should be prepared as usual, see \secref{sec:include}.
However, the document body must make a distinction
between processing of an individual part and of the main document, e.g.:
%
\begin{center}
\begin{tabular}{l}
|\ifchilddocmanual|\\
|\input{\childdocname}|\\
|\||else|\\
\textit{document body with }|\input{|\textit{part}|}|\\
|\||fi|
\end{tabular}
\end{center}
%
The conditional |\ifchilddocmanual| is true whenever
a part to be included by |\input| is being compiled,
and the name of the part is stored in |\childdocname|.

%%%%%%%%%%%%%%%%%%%%%%%%%%%%%%%%%%%%%%%%
\DescribeMacro{\childdocby}
Each part to be included by |\input| should start with:
%
\begin{center}
\begin{tabular}{l}
|\input{childdoc.def}|\\
|\childdocby{|\textit{main}|}|\\
\end{tabular}
\end{center}
%
The directive |\childdocby| is similar to |\childdocof|
described in \secref{sec:include},
but the subsequent selection of content must be done manually.
To that end, both |\ifchilddoc| and |\ifchilddocmanual|
will be true upon processing of a part,
and the name of the part is stored in |\childdocname|.
Note that |\jobname| will be set to the filename of the current part
so that each part receives an individual |.aux| file
that does not interfere with the |.aux| file(s) of the main document.
This behaviour can be altered by the alternative form
|\childdocby[*]{|\textit{main}|}| (with a non-empty optional argument)
which uses the |.aux| file of the main document
by setting |\jobname| to \textit{main}.

%%%%%%%%%%%%%%%%%%%%%%%%%%%%%%%%%%%%%%%%%%%%%%%%%%%%%%%%%%%%%%%%%%%%%%%%%%%%%%%%
\subsection{Driver Development}
\label{sec:driver}

The \textsf{childdoc} mechanism can also be use for the development
of definition files such as \LaTeX{} styles or classes.
This case differs from the above setup with multiple parts
included by |\include| in that no |\includeonly| should be invoked.
This can be achieved by starting the include file
(before |\ProvidesPackage|) with:
%
\begin{center}
\begin{tabular}{l}
|\input{childdoc.def}|\\
|\childdocforward{|\textit{main}|}|\\
\end{tabular}
\end{center}
%
or alternatively with:
%
\begin{center}
\begin{tabular}{l}
|\input{childdoc.def}|\\
|\childdocby{|\textit{main}|}|\\
\end{tabular}
\end{center}
%
Both forms have slightly different effects as described above.
The main file is prepared as usual, see \secref{sec:include}.

%%%%%%%%%%%%%%%%%%%%%%%%%%%%%%%%%%%%%%%%%%%%%%%%%%%%%%%%%%%%%%%%%%%%%%%%%%%%%%%%
\subsection{Legacy Detection}
\label{sec:detection}

The directive |\childdocmain| in the main file can detect
whether the complete document or merely a child is to be compiled
even without using the directive |\childdocof|.
This method is deprecated because it is less robust
and there is no compelling reason to use it;
it is merely provided for backward compatibility
and it may be removed in future versions.

If the detection mechanism is to be used,
it is mandatory to correctly specify
the filename of the main file as the argument of |\childdocmain|:
%
\begin{center}
\begin{tabular}{l}
|\input{childdoc.def}|\\
|\childdocmain{|\textit{main}|}|\\
\end{tabular}
\end{center}
%
If |\jobname| does not match the argument \textit{main} of |\childdocmain|,
it is assumed that |\jobname| points to the child file to be compiled.
When using |\childdocmain| with the main file specified as argument,
it suffices to start a child file
with just |\input{|\textit{main}|}|
without loading of the package and using |\childdocof|.
If instead all processing is done
with the appropriate \textsf{childdoc} directives,
the argument of \textit{main} of |\childdocmain| can be empty.

An alternative version of the command line processing described
in \secref{sec:commandline} using the detection mechanism reads:
%
\begin{center}
|... -jobname "|\textit{target}|" "|[\textit{flags}]%
[|\def\jobname{|\textit{dest}|}|]|\input{|\textit{main}|}"|
\end{center}

%%%%%%%%%%%%%%%%%%%%%%%%%%%%%%%%%%%%%%%%%%%%%%%%%%%%%%%%%%%%%%%%%%%%%%%%%%%%%%%%
\subsection{Manual Code}
\label{sec:manual}

In case one cannot be certain whether the definitions file |childdoc.def|
is installed on the target \TeX{} distribution
and one prefers not to ship it,
it is conceivable to paste a few relevant commands into the sources.

To that end, drop all statements |\input{childdoc.def}|
and perform the replacements as outlined below.
Instead of |\childdocmain{|\textit{main}|}| add the following code
to the top of the main file:
%
\begin{center}
\begin{tabular}{l}
|\||ifdefined\childdocname\endinput\||fi\newif\ifchilddoc|\\
|\edef\childdocname{\scantokens\expandafter{\jobname\noexpand}}|\\
|\def\childdocmain{|\textit{main}|}\||ifx\childdocmain\childdocname\||else|\\
|\childdoctrue\includeonly{\childdocname}\let\jobname\childdocmain\||fi|\\
\end{tabular}
\end{center}
%
Instead of |\childdocof{|\textit{main}|}| just include the main file
at the top of each child file:
%
\begin{center}
|\input{|\textit{main}|}|
\end{center}
%
A simple redirection |\childdocforward{|\textit{dest}|}| is achieved by:
%
\begin{center}
|\def\jobname{|\textit{dest}|}\input{\jobname}|
\end{center}
%
The redirection with prefix
|\childdocforwardprefix[|\textit{prefix}|]{|\textit{dest}|}|
is accomplished by:
%
\begin{center}
\begin{tabular}{l}
|{\edef\jobname{\scantokens\expandafter{\jobname\noexpand}}|\\
|\def\redirectjob |\textit{prefix}|#1~~~{\gdef\jobname{|\textit{dest}|#1}}|\\
|\expandafter\redirectjob\jobname~~~}\input{\jobname}|
\end{tabular}
\end{center}

In an alternative approach,
child documents can be compiled by a specific command line
without additional code or specific definitions:
%
\begin{center}
|... -jobname "|\textit{target}|" "|[\textit{flags}]%
|\includeonly{|\textit{dest}|}\input{|\textit{main}|}"|
\end{center}
%

%%%%%%%%%%%%%%%%%%%%%%%%%%%%%%%%%%%%%%%%%%%%%%%%%%%%%%%%%%%%%%%%%%%%%%%%%%%%%%%%
%%%%%%%%%%%%%%%%%%%%%%%%%%%%%%%%%%%%%%%%%%%%%%%%%%%%%%%%%%%%%%%%%%%%%%%%%%%%%%%%
\section{Information}

%%%%%%%%%%%%%%%%%%%%%%%%%%%%%%%%%%%%%%%%%%%%%%%%%%%%%%%%%%%%%%%%%%%%%%%%%%%%%%%%
\subsection{Copyright}

Copyright \copyright{} 2017--2018 Niklas Beisert

This work may be distributed and/or modified under the
conditions of the \LaTeX{} Project Public License, either version 1.3
of this license or (at your option) any later version.
The latest version of this license is in
  \url{http://www.latex-project.org/lppl.txt}
and version 1.3 or later is part of all distributions of \LaTeX{}
version 2005/12/01 or later.

This work has the LPPL maintenance status `maintained'.

The Current Maintainer of this work is Niklas Beisert.

This work consists of the files |README.txt|, |childdoc.ins| and |childdoc.dtx|
as well as the derived files |childdoc.def|, |cdocsamp.tex|
with |cdocsch1.tex|, |cdocsch2.tex|, |cdocspt3.tex|, |cdocspt4.tex|,
|cdocsdrf.tex|, |cdocsfn1.tex|, |cdocsfn2.tex|
as well as |childdoc.pdf|.

%%%%%%%%%%%%%%%%%%%%%%%%%%%%%%%%%%%%%%%%%%%%%%%%%%%%%%%%%%%%%%%%%%%%%%%%%%%%%%%%
\subsection{Files and Installation}

The package consists of the files:
%
\begin{center}
\begin{tabular}{ll}
    |README.txt|   & readme file \\
    |childdoc.ins| & installation file \\
    |childdoc.dtx| & source file \\
    |childdoc.def| & definition file \\
    |cdocsamp.tex| & sample main file \\
    |cdocsch1.tex| & sample include file \\
    |cdocsch2.tex| & sample include file \\
    |cdocspt3.tex| & sample part file \\
    |cdocspt4.tex| & sample part file \\
    |cdocsdrf.tex| & sample redirection file \\
    |cdocsfn1.tex| & sample redirection file \\
    |cdocsfn2.tex| & sample redirection file \\
    |childdoc.pdf| & manual
\end{tabular}
\end{center}
%
The distribution consists of the files
|README.txt|, |childdoc.ins| and |childdoc.dtx|.
%
\begin{itemize}
\item
Run (pdf)\LaTeX{} on |childdoc.dtx|
to compile the manual |childdoc.pdf| (this file).
\item
Run \LaTeX{} on |childdoc.ins| to create the definitions file |childdoc.def|
and the sample |cdocsamp.tex| with include files
|cdocsch1.tex|, |cdocsch2.tex|, |cdocspt3.tex|, |cdocspt4.tex|,
|cdocsdrf.tex|, |cdocsfn1.tex|, |cdocsfn2.tex|.
Then copy the file |childdoc.def| to an appropriate directory of your \LaTeX{}
distribution, e.g.\ \textit{texmf-root}|/tex/latex/childdoc|.
\end{itemize}

%%%%%%%%%%%%%%%%%%%%%%%%%%%%%%%%%%%%%%%%%%%%%%%%%%%%%%%%%%%%%%%%%%%%%%%%%%%%%%%%
\subsection{Related CTAN Packages}

There are several other packages which offer a similar functionality:
%
\begin{itemize}
\item
The packages
\href{http://ctan.org/pkg/docmute}{\textsf{docmute}},
\href{http://ctan.org/pkg/includex}{\textsf{includex}} and
\href{http://ctan.org/pkg/standalone}{\textsf{standalone}}
provide commands to include only the document body of
a child file thus allowing both files to be compiled individually.
\item
The packages \href{http://ctan.org/pkg/subdocs}{\textsf{subdocs}}
and \href{http://ctan.org/pkg/subfiles}{\textsf{subfiles}}
provide structures in which the main and child documents can be
encapsulated and allowing them to be compiled individually.
The inclusion mechanism is different from the conventional |\include|.
\item
The package \href{http://ctan.org/pkg/combine}{\textsf{combine}}
is an elaborate solution to combine several documents into one.
\end{itemize}
%
See also the CTAN topic \href{http://ctan.org/topic/subdocs}{\textsf{subdocs}}
for further related packages.
The present package differs from the above solutions in that
a document structure constructed with the conventional |\include| mechanism
just needs two extra commands at the top of every file
such that all constituent files can be compiled individually.

%%%%%%%%%%%%%%%%%%%%%%%%%%%%%%%%%%%%%%%%%%%%%%%%%%%%%%%%%%%%%%%%%%%%%%%%%%%%%%%%
%\subsection{Feature Suggestions}
%
%The following is a list of features which may be useful for future
%versions of this package:
%%
%\begin{itemize}
%\item
%\ldots
%\end{itemize}

%%%%%%%%%%%%%%%%%%%%%%%%%%%%%%%%%%%%%%%%%%%%%%%%%%%%%%%%%%%%%%%%%%%%%%%%%%%%%%%%
\subsection{Revision History}

%%%%%%%%%%%%%%%%%%%%%%%%%%%%%%%%%%%%%%%%
\paragraph{v2.0:} 2018/12/30

\begin{itemize}
\item
immediate forward processing
\item
added |\childdocby| mechanism
\item
manual restructured
\end{itemize}

%%%%%%%%%%%%%%%%%%%%%%%%%%%%%%%%%%%%%%%%
\paragraph{v1.6:} 2018/01/17

\begin{itemize}
\item
application for development of include files
\item
corrections to manual
\end{itemize}

%%%%%%%%%%%%%%%%%%%%%%%%%%%%%%%%%%%%%%%%
\paragraph{v1.5:} 2017/05/21

\begin{itemize}
\item
more complete structuring introduced
\item
|\childdocof| introduced
\item
|\childdoc| renamed to |\childdocmain|
\item
|\childredirect| renamed to |\childdocforward| and |\childdocforwardprefix|
and functionality expanded
\end{itemize}

%%%%%%%%%%%%%%%%%%%%%%%%%%%%%%%%%%%%%%%%
\paragraph{v1.0:} 2017/04/27

\begin{itemize}
\item
manual and install package
\item
first version published on CTAN
\end{itemize}

%%%%%%%%%%%%%%%%%%%%%%%%%%%%%%%%%%%%%%%%
\paragraph{v0.6:} 2017/04/26

\begin{itemize}
\item
redirection mechanism added
\end{itemize}

%%%%%%%%%%%%%%%%%%%%%%%%%%%%%%%%%%%%%%%%
\paragraph{v0.5:} 2017/04/26

\begin{itemize}
\item
functionality in definition file
\end{itemize}


%%%%%%%%%%%%%%%%%%%%%%%%%%%%%%%%%%%%%%%%%%%%%%%%%%%%%%%%%%%%%%%%%%%%%%%%%%%%%%%%
%%%%%%%%%%%%%%%%%%%%%%%%%%%%%%%%%%%%%%%%%%%%%%%%%%%%%%%%%%%%%%%%%%%%%%%%%%%%%%%%
%%%%%%%%%%%%%%%%%%%%%%%%%%%%%%%%%%%%%%%%%%%%%%%%%%%%%%%%%%%%%%%%%%%%%%%%%%%%%%%%
\appendix

\settowidth\MacroIndent{\rmfamily\scriptsize 000\ }

 \DocInput{childdoc.dtx}

\end{document}
%</driver>
% \fi
%
% %%%%%%%%%%%%%%%%%%%%%%%%%%%%%%%%%%%%%%%%%%%%%%%%%%%%%%%%%%%%%%%%%%%%%%%%%%%%%%
% %%%%%%%%%%%%%%%%%%%%%%%%%%%%%%%%%%%%%%%%%%%%%%%%%%%%%%%%%%%%%%%%%%%%%%%%%%%%%%
% \section{Sample}
%\iffalse
%<*samplemain>
%\fi
%
% The following presents a sample document
% with two chapters, two parts, a title page,
% a compile flag as well as three forwarding files to set the flag.
% It consists of eight |.tex| files:
% \begin{center}
% \begin{tabular}{ll}
% |cdocsamp.tex|&main file\\
% |cdocsch1.tex|&include file for chapter 1\\
% |cdocsch2.tex|&include file for chapter 2\\
% |cdocspt3.tex|&include file for part 3\\
% |cdocspt4.tex|&include file for part 4\\
% |cdocsdrf.tex|&forwarding file for main file in draft mode\\
% |cdocsfi1.tex|&forwarding file for final version of chapter 1\\
% |cdocsfi2.tex|&forwarding file for final version of chapter 2\\
% \end{tabular}
% \end{center}
% Each of the eight files can be compiled directly by the \LaTeX{} compiler.
%
% %%%%%%%%%%%%%%%%%%%%%%%%%%%%%%%%%%%%%%
% \paragraph{Main File.}
%
% The main file is called |cdocsamp.tex|.
%
% Load the \textsf{childdoc} definitions and
% declare the filename for the main document:
%    \begin{macrocode}
\input{childdoc.def}
\childdocmain{}
%    \end{macrocode}

% Optional override for |\version| flag:
%    \begin{macrocode}
%%\ifchilddoc\else\providecommand{\version}{draft}\fi
%    \end{macrocode}

% Define the default values for the |\version| flag
% (|final| for the main file and |draft| for childs):
%    \begin{macrocode}
\ifchilddoc
\providecommand{\version}{draft}
\else
\providecommand{\version}{final}
\fi
%    \end{macrocode}

% Load the standard document class:
%    \begin{macrocode}
\documentclass[12pt]{article}
%    \end{macrocode}

% Start the document body:
%    \begin{macrocode}
\begin{document}
%    \end{macrocode}

% Declare a title page.
% Print title, part of document being processed and version flag:
%    \begin{macrocode}
\addtocounter{page}{-1}
\begin{center}
{\LARGE\bfseries{}childdoc example\par}
\vspace{1cm}
\ifchilddoc
\ifchilddocmanual part\else chapter\fi:
`\childdocname' of `\childdocjob'\par
\else
main document: `\childdocjob'\par
\fi
version: \version\par
\end{center}
\newpage
%    \end{macrocode}

% Manually include selected file,
% otherwise process as usual:
%    \begin{macrocode}
\ifchilddocmanual
\section*{part `\childdocname'}
\input{\childdocname}
\else
%    \end{macrocode}

% Include the two chapters:
%    \begin{macrocode}
\include{cdocsch1}
\include{cdocsch2}
%    \end{macrocode}

% Include the two parts unless only chapters should be displayed:
%    \begin{macrocode}
\ifchilddoc\else
\section{part three}
\input{cdocspt3}
\section{part four}
\input{cdocspt4}
\fi
%    \end{macrocode}

% Process as usual until here:
%    \begin{macrocode}
\fi
%    \end{macrocode}

% End of document body:
%    \begin{macrocode}
\end{document}
%    \end{macrocode}
%\iffalse
%</samplemain>
%\fi
%
% %%%%%%%%%%%%%%%%%%%%%%%%%%%%%%%%%%%%%%
% \paragraph{Chapter Include Files.}
%
% The include files are called |cdocsch1.tex| and |cdocsch2.tex|.
%
%\iffalse
%<*samplechap1|samplechap2>
%\fi

% Optional override for |\version| flag:
%    \begin{macrocode}
%%\providecommand{\version}{final}
%    \end{macrocode}

% Include the main document:
%    \begin{macrocode}
\input{childdoc.def}
\childdocof{cdocsamp}
%    \end{macrocode}

%\iffalse
%</samplechap1|samplechap2>
%\fi
%
%\iffalse
%<*samplechap1>
%\fi
% Some text for chapter 1:
%    \begin{macrocode}
\section{one}
some text in chapter one
%    \end{macrocode}

%\iffalse
%</samplechap1>
%\fi
% Some text for chapter 2:
%\iffalse
%<*samplechap2>
%\fi
%    \begin{macrocode}
\section{two}
more text in chapter two
%    \end{macrocode}

%\iffalse
%</samplechap2>
%\fi
%
% %%%%%%%%%%%%%%%%%%%%%%%%%%%%%%%%%%%%%%
% \paragraph{Part Include Files.}
%
% The include files are called |cdocspt3.tex| and |cdocspt4.tex|.
%
%\iffalse
%<*samplepart3|samplepart4>
%\fi

% Optional override for |\version| flag:
%    \begin{macrocode}
%%\providecommand{\version}{final}
%    \end{macrocode}

% Include the main document:
%    \begin{macrocode}
\input{childdoc.def}
\childdocby{cdocsamp}
%    \end{macrocode}

%\iffalse
%</samplepart3|samplepart4>
%\fi
%
%\iffalse
%<*samplepart3>
%\fi
% Some text for part 3:
%    \begin{macrocode}
some text in part three
%    \end{macrocode}

%\iffalse
%</samplepart3>
%\fi
% Some text for part 4:
%\iffalse
%<*samplepart4>
%\fi
%    \begin{macrocode}
more text in part four
%    \end{macrocode}

%\iffalse
%</samplepart4>
%\fi
%
% %%%%%%%%%%%%%%%%%%%%%%%%%%%%%%%%%%%%%%
% \paragraph{Forwarding for a Complete Draft.}
%
% The following forwarding file |cdocsdrf.tex|
% compiles the main document in draft mode:
%\iffalse
%<*sampledraft>
%\fi
%    \begin{macrocode}
\def\version{draft}
\input{childdoc.def}
\childdocforward{cdocsamp}
%    \end{macrocode}

%\iffalse
%</sampledraft>
%\fi
%
% %%%%%%%%%%%%%%%%%%%%%%%%%%%%%%%%%%%%%%
% \paragraph{Forwarding for Final Version of the Chapters.}
%
% The following forwarding files |cdocsfn1.tex| and |cdocsfn2.tex|
% (with identical content)
% compile the final versions of the child documents
% |cdocsch1.tex| and |cdocsch2.tex|, respectively:
%\iffalse
%<*samplefinal>
%\fi
%    \begin{macrocode}
\def\version{final}
\input{childdoc.def}
\childdocforwardprefix[cdocsamp]{cdocsfn}{cdocsch}
%    \end{macrocode}

%\iffalse
%</samplefinal>
%\fi
%
% %%%%%%%%%%%%%%%%%%%%%%%%%%%%%%%%%%%%%%
% \paragraph{Command Line Processing.}
%
% The following three command lines generate the output files
% |cdocscld|, |cdocscl1| and |cdocscl2|
% which should be identical to
% |cdocsdrf|, |cdocsch1| and |cdocsfn2|, respectively:
% \begin{center}
% \begin{tabular}{l}
% |latex -jobname cdocscld \|\\
% |  "\def\version{draft}\input{childdoc.def}\childdocforward{cdocsamp}"|\\
% |latex -jobname cdocscl1 \|\\
% |  "\input{childdoc.def}\childdocforward[cdocsamp]{cdocsch1}"|\\
% |latex -jobname cdocscl2 \|\\
% |  "\def\version{final}\input{childdoc.def}\childdocforward{cdocsch2}"|
% \end{tabular}
% \end{center}
% Note that the trailing backslash on each first line
% merely continues the input to the second line
% (for convenient cut ant paste).
% Furthermore, the command |latex| can be replaced by any
% of its alternative versions such as |pdflatex|.
%
% %%%%%%%%%%%%%%%%%%%%%%%%%%%%%%%%%%%%%%%%%%%%%%%%%%%%%%%%%%%%%%%%%%%%%%%%%%%%%%
% %%%%%%%%%%%%%%%%%%%%%%%%%%%%%%%%%%%%%%%%%%%%%%%%%%%%%%%%%%%%%%%%%%%%%%%%%%%%%%
% \section{Implementation}
%\iffalse
%<*package>
%\fi
%
% This section describes the definitions file |childdoc.def|.

% The definitions cannot be loaded using |\usepackage| or |\RequirePackage|
% which has a mechanism to prevent loading a style file more than once.
% When loading the definitions by means of |\input|
% multiple instances have to be prevented manually:
%\iffalse
%This code needs to be before the `\ProvidesFile' directive
%which is defined at the beginning of this file.
%Therefore it is also placed there and commented out here.
%</package>
%<*discard>
%\fi
%    \begin{macrocode}
\ifdefined\childdocmain\endinput\fi
%    \end{macrocode}
%\iffalse
%</discard>
%<*package>
%\fi
%
% \macro{\ifchilddoc}
% \macro{\ifchilddocmanual}
% The conditional |\ifchilddoc| tells whether a
% child (true) or main (false) document is being compiled.
% The conditional |\ifchilddocmanual| tells whether
% the |\includeonly| mechanism is used (false) or
% the selection of child files must be performed manually (true).
% The definitions initialise to false:
%    \begin{macrocode}
\newif\ifchilddoc
\newif\ifchilddocmanual
%    \end{macrocode}

% \macro{\childdocname}
% \macro{\childdocjob}
% The macro |\childdocname| stores the name of the main document
% to be compiled. The macro |\childdocjob| stores the name of
% the document on which the \LaTeX{} compiler was originally invoked.
% The content of |\jobname| cannot be compared
% to filenames specified in the source due to different catcodes.
% The following code rescans |\jobname|, stores the result
% in |\childdocname| and saves a copy in |\childdocjob|:
%    \begin{macrocode}
\edef\childdocname{\scantokens\expandafter{\jobname\noexpand}}
\let\childdocjob\childdocname
%    \end{macrocode}

% \macro{\childdocdisable}
% The macro |\childdocdisable| prevents the main file
% from being processed more than once.
% At this stage, the main document command |\childdocmain|
% is assumed to be called once again where it should do nothing.
% Any subsequent call to it should prevent
% a secondary processing of the main document
% It overwrites the forwarding commands
% |\childdocof| and |\childdocforward|
% with empty macros to prevent further inclusions of the main document:
%    \begin{macrocode}
\newcommand{\childdocdisable}
{
  \renewcommand{\childdocmain}[1]{\renewcommand{\childdocmain}[1]{\endinput}}
  \renewcommand{\childdocof}[1]{}
  \renewcommand{\childdocby}[2][]{}
  \renewcommand{\childdocforward}[2][]{}
  \renewcommand{\childdocdisable}{}
}
%    \end{macrocode}

% \macro{\childdocmain}
% The macro |\childdocmain| is to be called at the top of the main file
% with nothing or the main filename (without extension) as argument.
% First, it breaks loops.
% If the argument is not empty and does not match |\childdocname|
% (which is set by the first inclusion of |childdoc.def|),
% |\ifchilddoc| is set to true, |\includeonly| is applied to the child file
% and |\jobname| is set to the main file
% (for proper handling of |.aux| files):
%    \begin{macrocode}
\newcommand{\childdocmain}[1]
{
  \childdocdisable\childdocmain{}
  \if?#1?\else
    \begingroup
      \def\childdoctmp{#1}
      \ifx\childdoctmp\childdocname
        \def\childdoctmp{}
      \else
        \def\childdoctmp
        {
          \childdoctrue
          \includeonly{\childdocname}
          \def\childdocjob{#1}
          \def\jobname{#1}
        }
      \fi
      \expandafter
    \endgroup
    \childdoctmp
  \fi
}
%    \end{macrocode}

% \macro{\childdocof}
% The command |\childdocof| redirects
% compilation to the main file |#1|.
%    \begin{macrocode}
\newcommand{\childdocof}[1]
{
  \childdocdisable
  \childdoctrue
  \includeonly{\childdocname}
  \def\jobname{#1}
  \def\childdocjob{#1}
  \input{#1}
}
%    \end{macrocode}

% \macro{\childdocby}
% The command |\childdocby| ....
%    \begin{macrocode}
\newcommand{\childdocby}[2][]
{
  \childdocdisable
  \childdoctrue
  \childdocmanualtrue
  \if?#1?\else
    \def\jobname{#2}
  \fi
  \def\childdocjob{#2}
  \input{#2}
  \endinput
}
%    \end{macrocode}

% \macro{\childdocforward}
% The command |\childdocforward| redirects
% compilation to the main file or
% (if the optional argument is given) a child file.
% Parameters are set as if the main file
% or a child file starting with |\childdocof| was compiled.
% Then compilation is handed over to the main file:
%    \begin{macrocode}
\newcommand{\childdocforward}[2][]
{
  \begingroup
    \if?#1?
      \def\childdoctmp
      {
        \def\childdocname{#2}
        \def\childdocjob{#2}
        \def\jobname{#2}
        \input{#2}
        \endinput
      }
    \else
      \def\childdoctmp
      {
        \childdocdisable
        \def\childdocname{#2}
        \childdoctrue
        \includeonly{#2}
        \def\childdocjob{#1}
        \def\jobname{#1}
        \input{#1}
        \endinput
      }
    \fi
    \expandafter
  \endgroup
  \childdoctmp
}
%    \end{macrocode}

% \macro{\childdocforwardprefix}
% The command |\childdocforwardprefix| redirects
% compilation to the main or a child file by means of a pattern.
% The prefix |#1| in the current filename is replaced by |#2|
% and the suffix of the current filename is kept
% (it is assumed that the filename does not contain the substring `|~~~|'
% which is used as a delimiter).
% Compilation is handed over to the new file by |\childdocforward|:
%    \begin{macrocode}
\newcommand{\childdocforwardprefix}[3][]
{
  \begingroup
    \def\childdocextract #2##1~~~{\def\childdoctmp{\childdocforward[#1]{#3##1}}}
    \expandafter\childdocextract\childdocname~~~
    \expandafter
  \endgroup
  \childdoctmp
}
%    \end{macrocode}

% \macro{\childdoc}
% The deprecated macro |\childdoc| is a legacy version of |\childdocmain|:
%    \begin{macrocode}
\newcommand{\childdoc}{\childdocmain}
%    \end{macrocode}

% \macro{\childdocredirect}
% The deprecated macro |\childdocredirect| is a legacy version
% of |\childdocforward| and |\childdocforwardprefix|:
%    \begin{macrocode}
\newcommand{\childdocredirect}[2][]
{
  \begingroup
    \if?#1?
      \def\childdoctmp{\childdocforward{#2}}
    \else
      \def\childdoctmp{\childdocforwardprefix{#1}{#2}}
    \fi
    \expandafter
  \endgroup
  \childdoctmp
}
%    \end{macrocode}

%\iffalse
%</package>
%\fi
%
\endinput

\childdocforwardprefix[cdocsamp]{cdocsfn}{cdocsch}
%    \end{macrocode}

%\iffalse
%</samplefinal>
%\fi
%
% %%%%%%%%%%%%%%%%%%%%%%%%%%%%%%%%%%%%%%
% \paragraph{Command Line Processing.}
%
% The following three command lines generate the output files
% |cdocscld|, |cdocscl1| and |cdocscl2|
% which should be identical to
% |cdocsdrf|, |cdocsch1| and |cdocsfn2|, respectively:
% \begin{center}
% \begin{tabular}{l}
% |latex -jobname cdocscld \|\\
% |  "\def\version{draft}% \iffalse
%
% childdoc.dtx Copyright (C) 2017-2018 Niklas Beisert
%
% This work may be distributed and/or modified under the
% conditions of the LaTeX Project Public License, either version 1.3
% of this license or (at your option) any later version.
% The latest version of this license is in
%   http://www.latex-project.org/lppl.txt
% and version 1.3 or later is part of all distributions of LaTeX
% version 2005/12/01 or later.
%
% This work has the LPPL maintenance status `maintained'.
%
% The Current Maintainer of this work is Niklas Beisert.
%
% This work consists of the files childdoc.dtx and childdoc.ins
% and the derived files childdoc.def and cdocsamp.tex with
% cdocsch1.tex, cdocsch2.tex, cdocsdrf.tex, cdocsfn1.tex, cdocsfn2.tex.
%
%<package>\ifdefined\childdocmain\endinput\fi
%<package>\ProvidesFile{childdoc.def}[2018/12/30 v2.0 child document driver]
%<samplemain>\ProvidesFile{cdocsamp.tex}[2018/12/30 v2.0 sample for childdoc]
%<*driver>
%\ProvidesFile{childdoc.drv}[2018/12/30 v2.0 childdoc reference manual file]
\PassOptionsToClass{10pt,a4paper}{article}
\documentclass{ltxdoc}

\usepackage[margin=35mm]{geometry}
\usepackage{hyperref}
\usepackage{hyperxmp}
\usepackage[usenames]{color}

\hypersetup{colorlinks=true}
\hypersetup{pdfstartview=FitH}
\hypersetup{pdfpagemode=UseNone}
\hypersetup{pdfsource={}}
\hypersetup{pdflang={en-UK}}
\hypersetup{pdfcopyright={Copyright 2017-2018 Niklas Beisert.
  This work may be distributed and/or modified under the
  conditions of the LaTeX Project Public License, either version 1.3
  of this license or (at your option) any later version.}}
\hypersetup{pdflicenseurl={http://www.latex-project.org/lppl.txt}}
\hypersetup{pdfcontactaddress={ETH Zurich, ITP, HIT K,
  Wolfgang-Pauli-Strasse 27}}
\hypersetup{pdfcontactpostcode={8093}}
\hypersetup{pdfcontactcity={Zurich}}
\hypersetup{pdfcontactcountry={Switzerland}}
\hypersetup{pdfcontactemail={nbeisert@itp.phys.ethz.ch}}
\hypersetup{pdfcontacturl={http://people.phys.ethz.ch/\xmptilde nbeisert/}}

\newcommand{\secref}[1]{\hyperref[#1]{section \ref*{#1}}}

\parskip1ex
\parindent0pt
\let\olditemize\itemize
\def\itemize{\olditemize\parskip0pt}

\begin{document}

\title{The \textsf{childdoc} Package}
\hypersetup{pdftitle={The childdoc Package}}
\author{Niklas Beisert\\[2ex]
  Institut f\"ur Theoretische Physik\\
  Eidgen\"ossische Technische Hochschule Z\"urich\\
  Wolfgang-Pauli-Strasse 27, 8093 Z\"urich, Switzerland\\[1ex]
  \href{mailto:nbeisert@itp.phys.ethz.ch}
  {\texttt{nbeisert@itp.phys.ethz.ch}}}
\hypersetup{pdfauthor={Niklas Beisert}}
\hypersetup{pdfsubject={Manual for the LaTeX2e Package childdoc}}
\date{30 December 2018, \textsf{v2.0}}
\maketitle

\begin{abstract}\noindent
\textsf{childdoc} is a \LaTeXe{} package
that enables the direct compilation
of document sections included by |\include|
to individual files.
\end{abstract}

\begingroup
\parskip0ex
\tableofcontents
\endgroup

%%%%%%%%%%%%%%%%%%%%%%%%%%%%%%%%%%%%%%%%%%%%%%%%%%%%%%%%%%%%%%%%%%%%%%%%%%%%%%%%
%%%%%%%%%%%%%%%%%%%%%%%%%%%%%%%%%%%%%%%%%%%%%%%%%%%%%%%%%%%%%%%%%%%%%%%%%%%%%%%%
\section{Introduction}

\LaTeX{} provides a mechanism to structure a large document (such as a book)
into a main file and several child files (containing the chapters)
using the |\include| command.
This mechanism is beneficial for documents
which span hundreds of pages in order to
make the source file(s) more manageable.
Moreover, compilation can be restricted to
selected child files by means of the |\includeonly| command.
The latter feature can be used to reduce the compilation time while editing
(this was significantly more useful in the earlier days of \LaTeX{})
or to generate a smaller document which is easier to navigate.
Another application of |\includeonly| is to generate
documents consisting of selected parts of the complete document.

However, there are a few drawbacks of the plain |\include| mechanism:
\begin{itemize}
\item
The child files cannot be compiled on their own,
they can only be compiled via the main file.
A naive editing environment
(such as a text editor with an option
to have the current file processed by \LaTeX)
may require one to switch to the main file before compiling;
attempting to compile the child file produces errors.
\item
The main file must be modified (each time)
to adjust the |\includeonly| command
to the present needs. This easily leaves the main file in a messy state.
\item
The generated document will always carry the filename
of the main document. This is inconvenient if
several child files are to be compiled and
to be kept for distribution.
\end{itemize}

The present package provides a simple interface
to make child files individually compilable by \LaTeX{}.
Compiling a child file then has the same effect as compiling
the main file with an |\includeonly| command
to select the appropriate child.
Moreover the generated document will carry the name of the child
rather than the main file.
This resolves all three above issues.

This feature is meant to make the editing of books,
thesis documents and lecture notes somewhat more convenient.
However, the package can also be used efficiently for
composing a series of documents (such as exercise sheets)
which are typically distributed individually.
It then assists the author in generating the individual documents
(potentially in different versions)
as well as a document containing the collected series.
Another application is in developing style files
or other kinds of included material
where compilation of the style file could redirect
to a sample or test file.

%%%%%%%%%%%%%%%%%%%%%%%%%%%%%%%%%%%%%%%%%%%%%%%%%%%%%%%%%%%%%%%%%%%%%%%%%%%%%%%%
%%%%%%%%%%%%%%%%%%%%%%%%%%%%%%%%%%%%%%%%%%%%%%%%%%%%%%%%%%%%%%%%%%%%%%%%%%%%%%%%
\section{Usage}

First of all, the package \textsf{childdoc} is \emph{not} a standard
\LaTeXe{} |.sty| style file! Therefore it needs to be invoked in
a non-standard way.

%%%%%%%%%%%%%%%%%%%%%%%%%%%%%%%%%%%%%%%%%%%%%%%%%%%%%%%%%%%%%%%%%%%%%%%%%%%%%%%%
\subsection{Included Files}
\label{sec:include}

%%%%%%%%%%%%%%%%%%%%%%%%%%%%%%%%%%%%%%%%
\DescribeMacro{\childdocmain}
To use the package, add the commands
\begin{center}
\begin{tabular}{l}
|\input{childdoc.def}|\\
|\childdocmain{}|\\
\end{tabular}
\end{center}
at the very top of the main \LaTeX{} file,
in particular \emph{before} the |\documentclass| statement!
The argument of |\childdocmain| should be left empty
(but it must be present).

%%%%%%%%%%%%%%%%%%%%%%%%%%%%%%%%%%%%%%%%
\DescribeMacro{\childdocof}
Furthermore, add the commands
\begin{center}
\begin{tabular}{l}
|\input{childdoc.def}|\\
|\childdocof{|\textit{main}|}|\\
\end{tabular}
\end{center}
at the top of every child file \textit{child}
which is included by |\include{|\textit{child}|}|
from within the main file
(or at least for those files to be compiled individually).
The argument \textit{main} must be the filename of the main file.

There are a couple of
considerations in setting up the main and child documents:

%%%%%%%%%%%%%%%%%%%%%%%%%%%%%%%%%%%%%%%%
\paragraph{Restrictions.}

Please note the following restrictions:
\begin{itemize}
\item
|\childdocmain| must be called with one argument \textit{main}
to ensure compatibility with earlier version of the package.
It must either be empty (|\childdocmain{}|)
or precisely match the filename of the main file in which it is specified.
See \secref{sec:detection} for further information.
\item
The filename \textit{main} must be specified without the |.tex| extension.
\item
The filename \textit{main} is case sensitive
(even in case-insensitive file systems)
due to internal string comparison.
\item
The argument \textit{main} should be fully expanded, it cannot be a macro.
\item
Subdirectories and special characters should be avoided in filenames.
\item
The command |\childdocmain{|\textit{main}|}| must be followed by a whitespace.
It should not be followed immediately by another command
or by a comment mark `|%|'.
This is because the \TeX{} parser reads the token immediately following
the argument of |\childdocmain| and puts it
at the beginning of every child section;
however, a white\-space is ignored.
\end{itemize}

%%%%%%%%%%%%%%%%%%%%%%%%%%%%%%%%%%%%%%%%
\paragraph{Content of Main File.}

It is advisable to place all content in the child files included by |\include|.
Any output contained in the main file will appear in all child documents
unless suppressed manually;
it cannot be suppressed automatically by the |\includeonly| directive
and thus should normally be avoided.
A method to include some content in the main file
by means of conditional processing is described in \secref{sec:conditional}.

%%%%%%%%%%%%%%%%%%%%%%%%%%%%%%%%%%%%%%%%
\paragraph{Page Numbering.}

When only a part of the document is compiled,
the appropriate numbering of pages
(as well as other status parameters)
is determined from the |.aux| files.
The latter contain information from previous passes.
However this information needs to propagate through
all intermediate child documents.
Therefore the page numbering in child documents may well
be inconsistent until the complete document is compiled at least once.

A useful (if unconventional) way to always ensure a consistent
page numbering is to restart the numbering in each child document
and denote the pages by `\textit{child}|.|\textit{page}'
where \textit{child} represents the chapter/section number of the child file.
This can be achieved by the command
|\numberwithin{page}{|\textit{child}|}|
of the \textsf{amsmath} package
where \textit{child} can be |chapter| or |section|
depending on the chosen structuring.
Alternatively, one can modify the macro |\thepage| appropriately
and reset the counter |page| at the start of each child file.

%%%%%%%%%%%%%%%%%%%%%%%%%%%%%%%%%%%%%%%%%%%%%%%%%%%%%%%%%%%%%%%%%%%%%%%%%%%%%%%%
\subsection{Conditional Processing}
\label{sec:conditional}

The package provides a mechanism to compile different versions
of a document. To customise the versions further some conditional processing
can come in handy to distinguish which version is being compiled.
The package provides two macros to describe the compilation context:

%%%%%%%%%%%%%%%%%%%%%%%%%%%%%%%%%%%%%%%%
\DescribeMacro{\ifchilddoc}
The conditional |\ifchilddoc| distinguishes between the compilation of
child documents and the main document:
%
\begin{center}
|\ifchilddoc |\textit{child-code}| |[|\||else |\textit{main-code}]| \||fi|
\end{center}

%%%%%%%%%%%%%%%%%%%%%%%%%%%%%%%%%%%%%%%%
\DescribeMacro{\childdocname}
\DescribeMacro{\childdocjob}
The macro |\childdocname| contains the filename (without extension)
of the main or child file being processed.
Note that |\childdocjob| will always contain the name of the main file.

%%%%%%%%%%%%%%%%%%%%%%%%%%%%%%%%%%%%%%%%
\paragraph{Title Page.}

Conditional processing can be used to include a title or banner page
in the main document when proper precautions are taken.
Importantly, the code in the main file should ensure that the page counter
(as well as other status parameters which are stored in the |.aux| files)
takes the same value after the conditional processing.
Otherwise the page numbers may take divergent values
depending on which part is compiled.

For example, a title page could be declared by:
%
\begin{center}
\begin{tabular}{l}
|\ifchilddoc\||else|\\
|\addtocounter{page}{-1}|\\
\textit{code for title page}\\
|\newpage|\\
|\||fi|
\end{tabular}
\end{center}
%
A banner page for the child documents can be generated by:
%
\begin{center}
\begin{tabular}{l}
|\ifchilddoc|\\
|\addtocounter{page}{-1}|\\
\textit{code for banner page}\\
|\newpage|\\
|\||fi|
\end{tabular}
\end{center}
%
Here one could write a message such as:
\begin{center}
|This is the part \childdocname{} of \childdocjob{}.|
\end{center}

%%%%%%%%%%%%%%%%%%%%%%%%%%%%%%%%%%%%%%%%%%%%%%%%%%%%%%%%%%%%%%%%%%%%%%%%%%%%%%%%
\subsection{Flags}
\label{sec:flags}

The package makes it easy to generate different versions
of the main or child documents.
To this end compilation flags can be defined
and assigned different default values.
They will be particularly useful in conjunction
with the forwarding mechanism described in \secref{sec:forward}.

For example, it may be useful to have a flag |\version|
which can be set to |draft| or |final|.
The document source will contain some conditional code
depending on the value of |\version|.
Suppose further, the flag should default to |final| for the main file
and to |draft| for child files
which is a natural assignment for editing the document.
This is achieved by placing the following code
in the preamble of the main document
(below the |\childdocmain| directive):
%
\begin{center}
\begin{tabular}{l}
|\ifchilddoc|\\
|\providecommand{\version}{draft}|\\
|\||else|\\
|\providecommand{\version}{final}|\\
|\||fi|
\end{tabular}
\end{center}
%
The definition by |\providecommand| makes sure
that previous definitions are not overwritten.
Further statements |\providecommand{\version}{...}|
can thus be added before the above code to override it.

For the main file, one might add a line
(between |\childdocmain| and the above block)
%
\begin{center}
|%\ifchilddoc\||else\providecommand{\version}{draft}\||fi|
\end{center}
%
which can be uncommented to produce a draft version.
Likewise one can add a line to the very top of a child file
(above the |\childdocof{|\textit{main}|}| directive)
%
\begin{center}
|%\providecommand{\version}{final}|
\end{center}
%
which can be uncommented to produce the final version of this child document.

%%%%%%%%%%%%%%%%%%%%%%%%%%%%%%%%%%%%%%%%%%%%%%%%%%%%%%%%%%%%%%%%%%%%%%%%%%%%%%%%
\subsection{Forwarding}
\label{sec:forward}

Different versions of the main or child documents
using compilation flags as described in \secref{sec:flags}
can be (permanently) stored in different files
for convenient compilation, viewing and distribution.
To this end, the package defines a command
to pass on compilation to a different file:

%%%%%%%%%%%%%%%%%%%%%%%%%%%%%%%%%%%%%%%%
\DescribeMacro{\childdocforward}
The command |\childdocforward| redirects processing to
another source file:
%
\begin{center}
\begin{tabular}{l}
|\input{childdoc.def}|\\
|\childdocforward[|\textit{main}|]{|\textit{dest}|}|\\
\end{tabular}
\end{center}
%
The argument \textit{dest} is the destination file
(without extension).
It should be the main file or one of the child files.
Note that further \textsf{childdoc} directives
such as |\childdocof| and |\childdocforward|
in the indicated file will be processed in this form.
The optional argument \textit{main}
passes on directly to the main file \textit{main}
while pretending to compile the child \textit{dest}.
This form behaves as if \textit{dest}
issues |\childdocof{|\textit{main}|}| right away,
and no further \textsf{childdoc} directives will be processed.

%%%%%%%%%%%%%%%%%%%%%%%%%%%%%%%%%%%%%%%%
\DescribeMacro{\...prefix}
In the alternative form |\childdocforwardprefix|,
%
\begin{center}
\begin{tabular}{l}
|\input{childdoc.def}|\\
|\childdocforwardprefix[|\textit{main}|]{|\textit{prefix}|}{|\textit{dest}|}|
\end{tabular}
\end{center}
%
the destination file is determined by a pattern
depending on the current file:
To make this work, the current file must be called
`{\textit{prefix}\hspace{0.2em}\textit{suffix}}'
with \textit{prefix} matching precisely the argument.
Processing is then passed on to the file
`{\textit{dest}\hspace{0.2em}\textit{suffix}}'.
Surely, the same effect is achieved by
directly specifying the
argument `{\textit{dest}\hspace{0.2em}\textit{suffix}}'
in the first form.
However, that requires to set up a different file
for each child. With the alternative form of the command
all these files can have exactly the same content
which simplifies setting them up and maintaining them.

For example, the following file |draft.tex|
with a compilation flag |\version| as described in \secref{sec:flags}
compiles the main document as a draft:
%
\begin{center}
\begin{tabular}{l}
|\def\version{draft}|\\
|\input{childdoc.def}|\\
|\childdocforward{|\textit{main}|}|
\end{tabular}
\end{center}
%
Likewise, the following files |final|\textit{nn}|.tex|
compile the final version of the child document
|child|\textit{nn}|.tex|:
%
\begin{center}
\begin{tabular}{l}
|\def\version{final}|\\
|\input{childdoc.def}|\\
|\childdocforwardprefix{final}{child}|
\end{tabular}
\end{center}
%

Note that when several versions of a main file and/or of each child file
are to be generated, it may be convenient to set up a |Makefile| or
shell script to automatise the process.

%%%%%%%%%%%%%%%%%%%%%%%%%%%%%%%%%%%%%%%%%%%%%%%%%%%%%%%%%%%%%%%%%%%%%%%%%%%%%%%%
\subsection{Command Line Processing}
\label{sec:commandline}

The effect of redirection files can also be achieved by invoking
the \LaTeX{} compiler with a more elaborate command line.
Most conveniently this should be done as part
of a shell script or a |Makefile|.

When using \textsf{childdoc} in the main file, the following
command lines effectively perform a redirection
(note that depending on the shell being used,
backslashes may have to be doubled: `|\|' $\to$ `|\\|'):
%
\begin{center}
|... -jobname "|\textit{target}|" |\\|"|[\textit{flags}]%
|\input{childdoc.def}\childdocforward[|\textit{main}|]{|\textit{dest}|}"|
\end{center}
%
Here \textit{target} is the name of the output file,
\textit{main} is the name of the main file
and \textit{dest} is the name of the main or child file to be processed
(all filenames without extensions).
The optional argument \textit{main} can be omitted
if \textit{main} matches \textit{dest}.
Optionally, compilation \textit{flags} can be defined via |\def| commands.
This command line makes the \TeX{} engine believe
it is compiling the file \textit{target}
whose content is specified as the latter parameter.
The provided code then forwards the processing to
\textit{main} or \textit{dest} as described in \secref{sec:forward}.

%%%%%%%%%%%%%%%%%%%%%%%%%%%%%%%%%%%%%%%%%%%%%%%%%%%%%%%%%%%%%%%%%%%%%%%%%%%%%%%%
\subsection{Include by Input}
\label{sec:input}

Including child documents by |\include| has some restrictions by design.
Most notably, the content of a child document always occupies
its own set of pages; pages cannot be shared between child documents.
Usually, this behaviour makes perfect sense
because each child document contain an essential part of the document.
However, in some situations it may be desirable to compose
a document from a collection of parts
without having mandatory page breaks between then.
For this case, the package
provides a mechanism to include parts
by |\input| which can also be processed individually.
However, by construction this mechanism
requires manual handling of the content to be output.

%%%%%%%%%%%%%%%%%%%%%%%%%%%%%%%%%%%%%%%%
\DescribeMacro{\ifchilddocmanual}
The main file should be prepared as usual, see \secref{sec:include}.
However, the document body must make a distinction
between processing of an individual part and of the main document, e.g.:
%
\begin{center}
\begin{tabular}{l}
|\ifchilddocmanual|\\
|\input{\childdocname}|\\
|\||else|\\
\textit{document body with }|\input{|\textit{part}|}|\\
|\||fi|
\end{tabular}
\end{center}
%
The conditional |\ifchilddocmanual| is true whenever
a part to be included by |\input| is being compiled,
and the name of the part is stored in |\childdocname|.

%%%%%%%%%%%%%%%%%%%%%%%%%%%%%%%%%%%%%%%%
\DescribeMacro{\childdocby}
Each part to be included by |\input| should start with:
%
\begin{center}
\begin{tabular}{l}
|\input{childdoc.def}|\\
|\childdocby{|\textit{main}|}|\\
\end{tabular}
\end{center}
%
The directive |\childdocby| is similar to |\childdocof|
described in \secref{sec:include},
but the subsequent selection of content must be done manually.
To that end, both |\ifchilddoc| and |\ifchilddocmanual|
will be true upon processing of a part,
and the name of the part is stored in |\childdocname|.
Note that |\jobname| will be set to the filename of the current part
so that each part receives an individual |.aux| file
that does not interfere with the |.aux| file(s) of the main document.
This behaviour can be altered by the alternative form
|\childdocby[*]{|\textit{main}|}| (with a non-empty optional argument)
which uses the |.aux| file of the main document
by setting |\jobname| to \textit{main}.

%%%%%%%%%%%%%%%%%%%%%%%%%%%%%%%%%%%%%%%%%%%%%%%%%%%%%%%%%%%%%%%%%%%%%%%%%%%%%%%%
\subsection{Driver Development}
\label{sec:driver}

The \textsf{childdoc} mechanism can also be use for the development
of definition files such as \LaTeX{} styles or classes.
This case differs from the above setup with multiple parts
included by |\include| in that no |\includeonly| should be invoked.
This can be achieved by starting the include file
(before |\ProvidesPackage|) with:
%
\begin{center}
\begin{tabular}{l}
|\input{childdoc.def}|\\
|\childdocforward{|\textit{main}|}|\\
\end{tabular}
\end{center}
%
or alternatively with:
%
\begin{center}
\begin{tabular}{l}
|\input{childdoc.def}|\\
|\childdocby{|\textit{main}|}|\\
\end{tabular}
\end{center}
%
Both forms have slightly different effects as described above.
The main file is prepared as usual, see \secref{sec:include}.

%%%%%%%%%%%%%%%%%%%%%%%%%%%%%%%%%%%%%%%%%%%%%%%%%%%%%%%%%%%%%%%%%%%%%%%%%%%%%%%%
\subsection{Legacy Detection}
\label{sec:detection}

The directive |\childdocmain| in the main file can detect
whether the complete document or merely a child is to be compiled
even without using the directive |\childdocof|.
This method is deprecated because it is less robust
and there is no compelling reason to use it;
it is merely provided for backward compatibility
and it may be removed in future versions.

If the detection mechanism is to be used,
it is mandatory to correctly specify
the filename of the main file as the argument of |\childdocmain|:
%
\begin{center}
\begin{tabular}{l}
|\input{childdoc.def}|\\
|\childdocmain{|\textit{main}|}|\\
\end{tabular}
\end{center}
%
If |\jobname| does not match the argument \textit{main} of |\childdocmain|,
it is assumed that |\jobname| points to the child file to be compiled.
When using |\childdocmain| with the main file specified as argument,
it suffices to start a child file
with just |\input{|\textit{main}|}|
without loading of the package and using |\childdocof|.
If instead all processing is done
with the appropriate \textsf{childdoc} directives,
the argument of \textit{main} of |\childdocmain| can be empty.

An alternative version of the command line processing described
in \secref{sec:commandline} using the detection mechanism reads:
%
\begin{center}
|... -jobname "|\textit{target}|" "|[\textit{flags}]%
[|\def\jobname{|\textit{dest}|}|]|\input{|\textit{main}|}"|
\end{center}

%%%%%%%%%%%%%%%%%%%%%%%%%%%%%%%%%%%%%%%%%%%%%%%%%%%%%%%%%%%%%%%%%%%%%%%%%%%%%%%%
\subsection{Manual Code}
\label{sec:manual}

In case one cannot be certain whether the definitions file |childdoc.def|
is installed on the target \TeX{} distribution
and one prefers not to ship it,
it is conceivable to paste a few relevant commands into the sources.

To that end, drop all statements |\input{childdoc.def}|
and perform the replacements as outlined below.
Instead of |\childdocmain{|\textit{main}|}| add the following code
to the top of the main file:
%
\begin{center}
\begin{tabular}{l}
|\||ifdefined\childdocname\endinput\||fi\newif\ifchilddoc|\\
|\edef\childdocname{\scantokens\expandafter{\jobname\noexpand}}|\\
|\def\childdocmain{|\textit{main}|}\||ifx\childdocmain\childdocname\||else|\\
|\childdoctrue\includeonly{\childdocname}\let\jobname\childdocmain\||fi|\\
\end{tabular}
\end{center}
%
Instead of |\childdocof{|\textit{main}|}| just include the main file
at the top of each child file:
%
\begin{center}
|\input{|\textit{main}|}|
\end{center}
%
A simple redirection |\childdocforward{|\textit{dest}|}| is achieved by:
%
\begin{center}
|\def\jobname{|\textit{dest}|}\input{\jobname}|
\end{center}
%
The redirection with prefix
|\childdocforwardprefix[|\textit{prefix}|]{|\textit{dest}|}|
is accomplished by:
%
\begin{center}
\begin{tabular}{l}
|{\edef\jobname{\scantokens\expandafter{\jobname\noexpand}}|\\
|\def\redirectjob |\textit{prefix}|#1~~~{\gdef\jobname{|\textit{dest}|#1}}|\\
|\expandafter\redirectjob\jobname~~~}\input{\jobname}|
\end{tabular}
\end{center}

In an alternative approach,
child documents can be compiled by a specific command line
without additional code or specific definitions:
%
\begin{center}
|... -jobname "|\textit{target}|" "|[\textit{flags}]%
|\includeonly{|\textit{dest}|}\input{|\textit{main}|}"|
\end{center}
%

%%%%%%%%%%%%%%%%%%%%%%%%%%%%%%%%%%%%%%%%%%%%%%%%%%%%%%%%%%%%%%%%%%%%%%%%%%%%%%%%
%%%%%%%%%%%%%%%%%%%%%%%%%%%%%%%%%%%%%%%%%%%%%%%%%%%%%%%%%%%%%%%%%%%%%%%%%%%%%%%%
\section{Information}

%%%%%%%%%%%%%%%%%%%%%%%%%%%%%%%%%%%%%%%%%%%%%%%%%%%%%%%%%%%%%%%%%%%%%%%%%%%%%%%%
\subsection{Copyright}

Copyright \copyright{} 2017--2018 Niklas Beisert

This work may be distributed and/or modified under the
conditions of the \LaTeX{} Project Public License, either version 1.3
of this license or (at your option) any later version.
The latest version of this license is in
  \url{http://www.latex-project.org/lppl.txt}
and version 1.3 or later is part of all distributions of \LaTeX{}
version 2005/12/01 or later.

This work has the LPPL maintenance status `maintained'.

The Current Maintainer of this work is Niklas Beisert.

This work consists of the files |README.txt|, |childdoc.ins| and |childdoc.dtx|
as well as the derived files |childdoc.def|, |cdocsamp.tex|
with |cdocsch1.tex|, |cdocsch2.tex|, |cdocspt3.tex|, |cdocspt4.tex|,
|cdocsdrf.tex|, |cdocsfn1.tex|, |cdocsfn2.tex|
as well as |childdoc.pdf|.

%%%%%%%%%%%%%%%%%%%%%%%%%%%%%%%%%%%%%%%%%%%%%%%%%%%%%%%%%%%%%%%%%%%%%%%%%%%%%%%%
\subsection{Files and Installation}

The package consists of the files:
%
\begin{center}
\begin{tabular}{ll}
    |README.txt|   & readme file \\
    |childdoc.ins| & installation file \\
    |childdoc.dtx| & source file \\
    |childdoc.def| & definition file \\
    |cdocsamp.tex| & sample main file \\
    |cdocsch1.tex| & sample include file \\
    |cdocsch2.tex| & sample include file \\
    |cdocspt3.tex| & sample part file \\
    |cdocspt4.tex| & sample part file \\
    |cdocsdrf.tex| & sample redirection file \\
    |cdocsfn1.tex| & sample redirection file \\
    |cdocsfn2.tex| & sample redirection file \\
    |childdoc.pdf| & manual
\end{tabular}
\end{center}
%
The distribution consists of the files
|README.txt|, |childdoc.ins| and |childdoc.dtx|.
%
\begin{itemize}
\item
Run (pdf)\LaTeX{} on |childdoc.dtx|
to compile the manual |childdoc.pdf| (this file).
\item
Run \LaTeX{} on |childdoc.ins| to create the definitions file |childdoc.def|
and the sample |cdocsamp.tex| with include files
|cdocsch1.tex|, |cdocsch2.tex|, |cdocspt3.tex|, |cdocspt4.tex|,
|cdocsdrf.tex|, |cdocsfn1.tex|, |cdocsfn2.tex|.
Then copy the file |childdoc.def| to an appropriate directory of your \LaTeX{}
distribution, e.g.\ \textit{texmf-root}|/tex/latex/childdoc|.
\end{itemize}

%%%%%%%%%%%%%%%%%%%%%%%%%%%%%%%%%%%%%%%%%%%%%%%%%%%%%%%%%%%%%%%%%%%%%%%%%%%%%%%%
\subsection{Related CTAN Packages}

There are several other packages which offer a similar functionality:
%
\begin{itemize}
\item
The packages
\href{http://ctan.org/pkg/docmute}{\textsf{docmute}},
\href{http://ctan.org/pkg/includex}{\textsf{includex}} and
\href{http://ctan.org/pkg/standalone}{\textsf{standalone}}
provide commands to include only the document body of
a child file thus allowing both files to be compiled individually.
\item
The packages \href{http://ctan.org/pkg/subdocs}{\textsf{subdocs}}
and \href{http://ctan.org/pkg/subfiles}{\textsf{subfiles}}
provide structures in which the main and child documents can be
encapsulated and allowing them to be compiled individually.
The inclusion mechanism is different from the conventional |\include|.
\item
The package \href{http://ctan.org/pkg/combine}{\textsf{combine}}
is an elaborate solution to combine several documents into one.
\end{itemize}
%
See also the CTAN topic \href{http://ctan.org/topic/subdocs}{\textsf{subdocs}}
for further related packages.
The present package differs from the above solutions in that
a document structure constructed with the conventional |\include| mechanism
just needs two extra commands at the top of every file
such that all constituent files can be compiled individually.

%%%%%%%%%%%%%%%%%%%%%%%%%%%%%%%%%%%%%%%%%%%%%%%%%%%%%%%%%%%%%%%%%%%%%%%%%%%%%%%%
%\subsection{Feature Suggestions}
%
%The following is a list of features which may be useful for future
%versions of this package:
%%
%\begin{itemize}
%\item
%\ldots
%\end{itemize}

%%%%%%%%%%%%%%%%%%%%%%%%%%%%%%%%%%%%%%%%%%%%%%%%%%%%%%%%%%%%%%%%%%%%%%%%%%%%%%%%
\subsection{Revision History}

%%%%%%%%%%%%%%%%%%%%%%%%%%%%%%%%%%%%%%%%
\paragraph{v2.0:} 2018/12/30

\begin{itemize}
\item
immediate forward processing
\item
added |\childdocby| mechanism
\item
manual restructured
\end{itemize}

%%%%%%%%%%%%%%%%%%%%%%%%%%%%%%%%%%%%%%%%
\paragraph{v1.6:} 2018/01/17

\begin{itemize}
\item
application for development of include files
\item
corrections to manual
\end{itemize}

%%%%%%%%%%%%%%%%%%%%%%%%%%%%%%%%%%%%%%%%
\paragraph{v1.5:} 2017/05/21

\begin{itemize}
\item
more complete structuring introduced
\item
|\childdocof| introduced
\item
|\childdoc| renamed to |\childdocmain|
\item
|\childredirect| renamed to |\childdocforward| and |\childdocforwardprefix|
and functionality expanded
\end{itemize}

%%%%%%%%%%%%%%%%%%%%%%%%%%%%%%%%%%%%%%%%
\paragraph{v1.0:} 2017/04/27

\begin{itemize}
\item
manual and install package
\item
first version published on CTAN
\end{itemize}

%%%%%%%%%%%%%%%%%%%%%%%%%%%%%%%%%%%%%%%%
\paragraph{v0.6:} 2017/04/26

\begin{itemize}
\item
redirection mechanism added
\end{itemize}

%%%%%%%%%%%%%%%%%%%%%%%%%%%%%%%%%%%%%%%%
\paragraph{v0.5:} 2017/04/26

\begin{itemize}
\item
functionality in definition file
\end{itemize}


%%%%%%%%%%%%%%%%%%%%%%%%%%%%%%%%%%%%%%%%%%%%%%%%%%%%%%%%%%%%%%%%%%%%%%%%%%%%%%%%
%%%%%%%%%%%%%%%%%%%%%%%%%%%%%%%%%%%%%%%%%%%%%%%%%%%%%%%%%%%%%%%%%%%%%%%%%%%%%%%%
%%%%%%%%%%%%%%%%%%%%%%%%%%%%%%%%%%%%%%%%%%%%%%%%%%%%%%%%%%%%%%%%%%%%%%%%%%%%%%%%
\appendix

\settowidth\MacroIndent{\rmfamily\scriptsize 000\ }

 \DocInput{childdoc.dtx}

\end{document}
%</driver>
% \fi
%
% %%%%%%%%%%%%%%%%%%%%%%%%%%%%%%%%%%%%%%%%%%%%%%%%%%%%%%%%%%%%%%%%%%%%%%%%%%%%%%
% %%%%%%%%%%%%%%%%%%%%%%%%%%%%%%%%%%%%%%%%%%%%%%%%%%%%%%%%%%%%%%%%%%%%%%%%%%%%%%
% \section{Sample}
%\iffalse
%<*samplemain>
%\fi
%
% The following presents a sample document
% with two chapters, two parts, a title page,
% a compile flag as well as three forwarding files to set the flag.
% It consists of eight |.tex| files:
% \begin{center}
% \begin{tabular}{ll}
% |cdocsamp.tex|&main file\\
% |cdocsch1.tex|&include file for chapter 1\\
% |cdocsch2.tex|&include file for chapter 2\\
% |cdocspt3.tex|&include file for part 3\\
% |cdocspt4.tex|&include file for part 4\\
% |cdocsdrf.tex|&forwarding file for main file in draft mode\\
% |cdocsfi1.tex|&forwarding file for final version of chapter 1\\
% |cdocsfi2.tex|&forwarding file for final version of chapter 2\\
% \end{tabular}
% \end{center}
% Each of the eight files can be compiled directly by the \LaTeX{} compiler.
%
% %%%%%%%%%%%%%%%%%%%%%%%%%%%%%%%%%%%%%%
% \paragraph{Main File.}
%
% The main file is called |cdocsamp.tex|.
%
% Load the \textsf{childdoc} definitions and
% declare the filename for the main document:
%    \begin{macrocode}
\input{childdoc.def}
\childdocmain{}
%    \end{macrocode}

% Optional override for |\version| flag:
%    \begin{macrocode}
%%\ifchilddoc\else\providecommand{\version}{draft}\fi
%    \end{macrocode}

% Define the default values for the |\version| flag
% (|final| for the main file and |draft| for childs):
%    \begin{macrocode}
\ifchilddoc
\providecommand{\version}{draft}
\else
\providecommand{\version}{final}
\fi
%    \end{macrocode}

% Load the standard document class:
%    \begin{macrocode}
\documentclass[12pt]{article}
%    \end{macrocode}

% Start the document body:
%    \begin{macrocode}
\begin{document}
%    \end{macrocode}

% Declare a title page.
% Print title, part of document being processed and version flag:
%    \begin{macrocode}
\addtocounter{page}{-1}
\begin{center}
{\LARGE\bfseries{}childdoc example\par}
\vspace{1cm}
\ifchilddoc
\ifchilddocmanual part\else chapter\fi:
`\childdocname' of `\childdocjob'\par
\else
main document: `\childdocjob'\par
\fi
version: \version\par
\end{center}
\newpage
%    \end{macrocode}

% Manually include selected file,
% otherwise process as usual:
%    \begin{macrocode}
\ifchilddocmanual
\section*{part `\childdocname'}
\input{\childdocname}
\else
%    \end{macrocode}

% Include the two chapters:
%    \begin{macrocode}
\include{cdocsch1}
\include{cdocsch2}
%    \end{macrocode}

% Include the two parts unless only chapters should be displayed:
%    \begin{macrocode}
\ifchilddoc\else
\section{part three}
\input{cdocspt3}
\section{part four}
\input{cdocspt4}
\fi
%    \end{macrocode}

% Process as usual until here:
%    \begin{macrocode}
\fi
%    \end{macrocode}

% End of document body:
%    \begin{macrocode}
\end{document}
%    \end{macrocode}
%\iffalse
%</samplemain>
%\fi
%
% %%%%%%%%%%%%%%%%%%%%%%%%%%%%%%%%%%%%%%
% \paragraph{Chapter Include Files.}
%
% The include files are called |cdocsch1.tex| and |cdocsch2.tex|.
%
%\iffalse
%<*samplechap1|samplechap2>
%\fi

% Optional override for |\version| flag:
%    \begin{macrocode}
%%\providecommand{\version}{final}
%    \end{macrocode}

% Include the main document:
%    \begin{macrocode}
\input{childdoc.def}
\childdocof{cdocsamp}
%    \end{macrocode}

%\iffalse
%</samplechap1|samplechap2>
%\fi
%
%\iffalse
%<*samplechap1>
%\fi
% Some text for chapter 1:
%    \begin{macrocode}
\section{one}
some text in chapter one
%    \end{macrocode}

%\iffalse
%</samplechap1>
%\fi
% Some text for chapter 2:
%\iffalse
%<*samplechap2>
%\fi
%    \begin{macrocode}
\section{two}
more text in chapter two
%    \end{macrocode}

%\iffalse
%</samplechap2>
%\fi
%
% %%%%%%%%%%%%%%%%%%%%%%%%%%%%%%%%%%%%%%
% \paragraph{Part Include Files.}
%
% The include files are called |cdocspt3.tex| and |cdocspt4.tex|.
%
%\iffalse
%<*samplepart3|samplepart4>
%\fi

% Optional override for |\version| flag:
%    \begin{macrocode}
%%\providecommand{\version}{final}
%    \end{macrocode}

% Include the main document:
%    \begin{macrocode}
\input{childdoc.def}
\childdocby{cdocsamp}
%    \end{macrocode}

%\iffalse
%</samplepart3|samplepart4>
%\fi
%
%\iffalse
%<*samplepart3>
%\fi
% Some text for part 3:
%    \begin{macrocode}
some text in part three
%    \end{macrocode}

%\iffalse
%</samplepart3>
%\fi
% Some text for part 4:
%\iffalse
%<*samplepart4>
%\fi
%    \begin{macrocode}
more text in part four
%    \end{macrocode}

%\iffalse
%</samplepart4>
%\fi
%
% %%%%%%%%%%%%%%%%%%%%%%%%%%%%%%%%%%%%%%
% \paragraph{Forwarding for a Complete Draft.}
%
% The following forwarding file |cdocsdrf.tex|
% compiles the main document in draft mode:
%\iffalse
%<*sampledraft>
%\fi
%    \begin{macrocode}
\def\version{draft}
\input{childdoc.def}
\childdocforward{cdocsamp}
%    \end{macrocode}

%\iffalse
%</sampledraft>
%\fi
%
% %%%%%%%%%%%%%%%%%%%%%%%%%%%%%%%%%%%%%%
% \paragraph{Forwarding for Final Version of the Chapters.}
%
% The following forwarding files |cdocsfn1.tex| and |cdocsfn2.tex|
% (with identical content)
% compile the final versions of the child documents
% |cdocsch1.tex| and |cdocsch2.tex|, respectively:
%\iffalse
%<*samplefinal>
%\fi
%    \begin{macrocode}
\def\version{final}
\input{childdoc.def}
\childdocforwardprefix[cdocsamp]{cdocsfn}{cdocsch}
%    \end{macrocode}

%\iffalse
%</samplefinal>
%\fi
%
% %%%%%%%%%%%%%%%%%%%%%%%%%%%%%%%%%%%%%%
% \paragraph{Command Line Processing.}
%
% The following three command lines generate the output files
% |cdocscld|, |cdocscl1| and |cdocscl2|
% which should be identical to
% |cdocsdrf|, |cdocsch1| and |cdocsfn2|, respectively:
% \begin{center}
% \begin{tabular}{l}
% |latex -jobname cdocscld \|\\
% |  "\def\version{draft}\input{childdoc.def}\childdocforward{cdocsamp}"|\\
% |latex -jobname cdocscl1 \|\\
% |  "\input{childdoc.def}\childdocforward[cdocsamp]{cdocsch1}"|\\
% |latex -jobname cdocscl2 \|\\
% |  "\def\version{final}\input{childdoc.def}\childdocforward{cdocsch2}"|
% \end{tabular}
% \end{center}
% Note that the trailing backslash on each first line
% merely continues the input to the second line
% (for convenient cut ant paste).
% Furthermore, the command |latex| can be replaced by any
% of its alternative versions such as |pdflatex|.
%
% %%%%%%%%%%%%%%%%%%%%%%%%%%%%%%%%%%%%%%%%%%%%%%%%%%%%%%%%%%%%%%%%%%%%%%%%%%%%%%
% %%%%%%%%%%%%%%%%%%%%%%%%%%%%%%%%%%%%%%%%%%%%%%%%%%%%%%%%%%%%%%%%%%%%%%%%%%%%%%
% \section{Implementation}
%\iffalse
%<*package>
%\fi
%
% This section describes the definitions file |childdoc.def|.

% The definitions cannot be loaded using |\usepackage| or |\RequirePackage|
% which has a mechanism to prevent loading a style file more than once.
% When loading the definitions by means of |\input|
% multiple instances have to be prevented manually:
%\iffalse
%This code needs to be before the `\ProvidesFile' directive
%which is defined at the beginning of this file.
%Therefore it is also placed there and commented out here.
%</package>
%<*discard>
%\fi
%    \begin{macrocode}
\ifdefined\childdocmain\endinput\fi
%    \end{macrocode}
%\iffalse
%</discard>
%<*package>
%\fi
%
% \macro{\ifchilddoc}
% \macro{\ifchilddocmanual}
% The conditional |\ifchilddoc| tells whether a
% child (true) or main (false) document is being compiled.
% The conditional |\ifchilddocmanual| tells whether
% the |\includeonly| mechanism is used (false) or
% the selection of child files must be performed manually (true).
% The definitions initialise to false:
%    \begin{macrocode}
\newif\ifchilddoc
\newif\ifchilddocmanual
%    \end{macrocode}

% \macro{\childdocname}
% \macro{\childdocjob}
% The macro |\childdocname| stores the name of the main document
% to be compiled. The macro |\childdocjob| stores the name of
% the document on which the \LaTeX{} compiler was originally invoked.
% The content of |\jobname| cannot be compared
% to filenames specified in the source due to different catcodes.
% The following code rescans |\jobname|, stores the result
% in |\childdocname| and saves a copy in |\childdocjob|:
%    \begin{macrocode}
\edef\childdocname{\scantokens\expandafter{\jobname\noexpand}}
\let\childdocjob\childdocname
%    \end{macrocode}

% \macro{\childdocdisable}
% The macro |\childdocdisable| prevents the main file
% from being processed more than once.
% At this stage, the main document command |\childdocmain|
% is assumed to be called once again where it should do nothing.
% Any subsequent call to it should prevent
% a secondary processing of the main document
% It overwrites the forwarding commands
% |\childdocof| and |\childdocforward|
% with empty macros to prevent further inclusions of the main document:
%    \begin{macrocode}
\newcommand{\childdocdisable}
{
  \renewcommand{\childdocmain}[1]{\renewcommand{\childdocmain}[1]{\endinput}}
  \renewcommand{\childdocof}[1]{}
  \renewcommand{\childdocby}[2][]{}
  \renewcommand{\childdocforward}[2][]{}
  \renewcommand{\childdocdisable}{}
}
%    \end{macrocode}

% \macro{\childdocmain}
% The macro |\childdocmain| is to be called at the top of the main file
% with nothing or the main filename (without extension) as argument.
% First, it breaks loops.
% If the argument is not empty and does not match |\childdocname|
% (which is set by the first inclusion of |childdoc.def|),
% |\ifchilddoc| is set to true, |\includeonly| is applied to the child file
% and |\jobname| is set to the main file
% (for proper handling of |.aux| files):
%    \begin{macrocode}
\newcommand{\childdocmain}[1]
{
  \childdocdisable\childdocmain{}
  \if?#1?\else
    \begingroup
      \def\childdoctmp{#1}
      \ifx\childdoctmp\childdocname
        \def\childdoctmp{}
      \else
        \def\childdoctmp
        {
          \childdoctrue
          \includeonly{\childdocname}
          \def\childdocjob{#1}
          \def\jobname{#1}
        }
      \fi
      \expandafter
    \endgroup
    \childdoctmp
  \fi
}
%    \end{macrocode}

% \macro{\childdocof}
% The command |\childdocof| redirects
% compilation to the main file |#1|.
%    \begin{macrocode}
\newcommand{\childdocof}[1]
{
  \childdocdisable
  \childdoctrue
  \includeonly{\childdocname}
  \def\jobname{#1}
  \def\childdocjob{#1}
  \input{#1}
}
%    \end{macrocode}

% \macro{\childdocby}
% The command |\childdocby| ....
%    \begin{macrocode}
\newcommand{\childdocby}[2][]
{
  \childdocdisable
  \childdoctrue
  \childdocmanualtrue
  \if?#1?\else
    \def\jobname{#2}
  \fi
  \def\childdocjob{#2}
  \input{#2}
  \endinput
}
%    \end{macrocode}

% \macro{\childdocforward}
% The command |\childdocforward| redirects
% compilation to the main file or
% (if the optional argument is given) a child file.
% Parameters are set as if the main file
% or a child file starting with |\childdocof| was compiled.
% Then compilation is handed over to the main file:
%    \begin{macrocode}
\newcommand{\childdocforward}[2][]
{
  \begingroup
    \if?#1?
      \def\childdoctmp
      {
        \def\childdocname{#2}
        \def\childdocjob{#2}
        \def\jobname{#2}
        \input{#2}
        \endinput
      }
    \else
      \def\childdoctmp
      {
        \childdocdisable
        \def\childdocname{#2}
        \childdoctrue
        \includeonly{#2}
        \def\childdocjob{#1}
        \def\jobname{#1}
        \input{#1}
        \endinput
      }
    \fi
    \expandafter
  \endgroup
  \childdoctmp
}
%    \end{macrocode}

% \macro{\childdocforwardprefix}
% The command |\childdocforwardprefix| redirects
% compilation to the main or a child file by means of a pattern.
% The prefix |#1| in the current filename is replaced by |#2|
% and the suffix of the current filename is kept
% (it is assumed that the filename does not contain the substring `|~~~|'
% which is used as a delimiter).
% Compilation is handed over to the new file by |\childdocforward|:
%    \begin{macrocode}
\newcommand{\childdocforwardprefix}[3][]
{
  \begingroup
    \def\childdocextract #2##1~~~{\def\childdoctmp{\childdocforward[#1]{#3##1}}}
    \expandafter\childdocextract\childdocname~~~
    \expandafter
  \endgroup
  \childdoctmp
}
%    \end{macrocode}

% \macro{\childdoc}
% The deprecated macro |\childdoc| is a legacy version of |\childdocmain|:
%    \begin{macrocode}
\newcommand{\childdoc}{\childdocmain}
%    \end{macrocode}

% \macro{\childdocredirect}
% The deprecated macro |\childdocredirect| is a legacy version
% of |\childdocforward| and |\childdocforwardprefix|:
%    \begin{macrocode}
\newcommand{\childdocredirect}[2][]
{
  \begingroup
    \if?#1?
      \def\childdoctmp{\childdocforward{#2}}
    \else
      \def\childdoctmp{\childdocforwardprefix{#1}{#2}}
    \fi
    \expandafter
  \endgroup
  \childdoctmp
}
%    \end{macrocode}

%\iffalse
%</package>
%\fi
%
\endinput
\childdocforward{cdocsamp}"|\\
% |latex -jobname cdocscl1 \|\\
% |  "% \iffalse
%
% childdoc.dtx Copyright (C) 2017-2018 Niklas Beisert
%
% This work may be distributed and/or modified under the
% conditions of the LaTeX Project Public License, either version 1.3
% of this license or (at your option) any later version.
% The latest version of this license is in
%   http://www.latex-project.org/lppl.txt
% and version 1.3 or later is part of all distributions of LaTeX
% version 2005/12/01 or later.
%
% This work has the LPPL maintenance status `maintained'.
%
% The Current Maintainer of this work is Niklas Beisert.
%
% This work consists of the files childdoc.dtx and childdoc.ins
% and the derived files childdoc.def and cdocsamp.tex with
% cdocsch1.tex, cdocsch2.tex, cdocsdrf.tex, cdocsfn1.tex, cdocsfn2.tex.
%
%<package>\ifdefined\childdocmain\endinput\fi
%<package>\ProvidesFile{childdoc.def}[2018/12/30 v2.0 child document driver]
%<samplemain>\ProvidesFile{cdocsamp.tex}[2018/12/30 v2.0 sample for childdoc]
%<*driver>
%\ProvidesFile{childdoc.drv}[2018/12/30 v2.0 childdoc reference manual file]
\PassOptionsToClass{10pt,a4paper}{article}
\documentclass{ltxdoc}

\usepackage[margin=35mm]{geometry}
\usepackage{hyperref}
\usepackage{hyperxmp}
\usepackage[usenames]{color}

\hypersetup{colorlinks=true}
\hypersetup{pdfstartview=FitH}
\hypersetup{pdfpagemode=UseNone}
\hypersetup{pdfsource={}}
\hypersetup{pdflang={en-UK}}
\hypersetup{pdfcopyright={Copyright 2017-2018 Niklas Beisert.
  This work may be distributed and/or modified under the
  conditions of the LaTeX Project Public License, either version 1.3
  of this license or (at your option) any later version.}}
\hypersetup{pdflicenseurl={http://www.latex-project.org/lppl.txt}}
\hypersetup{pdfcontactaddress={ETH Zurich, ITP, HIT K,
  Wolfgang-Pauli-Strasse 27}}
\hypersetup{pdfcontactpostcode={8093}}
\hypersetup{pdfcontactcity={Zurich}}
\hypersetup{pdfcontactcountry={Switzerland}}
\hypersetup{pdfcontactemail={nbeisert@itp.phys.ethz.ch}}
\hypersetup{pdfcontacturl={http://people.phys.ethz.ch/\xmptilde nbeisert/}}

\newcommand{\secref}[1]{\hyperref[#1]{section \ref*{#1}}}

\parskip1ex
\parindent0pt
\let\olditemize\itemize
\def\itemize{\olditemize\parskip0pt}

\begin{document}

\title{The \textsf{childdoc} Package}
\hypersetup{pdftitle={The childdoc Package}}
\author{Niklas Beisert\\[2ex]
  Institut f\"ur Theoretische Physik\\
  Eidgen\"ossische Technische Hochschule Z\"urich\\
  Wolfgang-Pauli-Strasse 27, 8093 Z\"urich, Switzerland\\[1ex]
  \href{mailto:nbeisert@itp.phys.ethz.ch}
  {\texttt{nbeisert@itp.phys.ethz.ch}}}
\hypersetup{pdfauthor={Niklas Beisert}}
\hypersetup{pdfsubject={Manual for the LaTeX2e Package childdoc}}
\date{30 December 2018, \textsf{v2.0}}
\maketitle

\begin{abstract}\noindent
\textsf{childdoc} is a \LaTeXe{} package
that enables the direct compilation
of document sections included by |\include|
to individual files.
\end{abstract}

\begingroup
\parskip0ex
\tableofcontents
\endgroup

%%%%%%%%%%%%%%%%%%%%%%%%%%%%%%%%%%%%%%%%%%%%%%%%%%%%%%%%%%%%%%%%%%%%%%%%%%%%%%%%
%%%%%%%%%%%%%%%%%%%%%%%%%%%%%%%%%%%%%%%%%%%%%%%%%%%%%%%%%%%%%%%%%%%%%%%%%%%%%%%%
\section{Introduction}

\LaTeX{} provides a mechanism to structure a large document (such as a book)
into a main file and several child files (containing the chapters)
using the |\include| command.
This mechanism is beneficial for documents
which span hundreds of pages in order to
make the source file(s) more manageable.
Moreover, compilation can be restricted to
selected child files by means of the |\includeonly| command.
The latter feature can be used to reduce the compilation time while editing
(this was significantly more useful in the earlier days of \LaTeX{})
or to generate a smaller document which is easier to navigate.
Another application of |\includeonly| is to generate
documents consisting of selected parts of the complete document.

However, there are a few drawbacks of the plain |\include| mechanism:
\begin{itemize}
\item
The child files cannot be compiled on their own,
they can only be compiled via the main file.
A naive editing environment
(such as a text editor with an option
to have the current file processed by \LaTeX)
may require one to switch to the main file before compiling;
attempting to compile the child file produces errors.
\item
The main file must be modified (each time)
to adjust the |\includeonly| command
to the present needs. This easily leaves the main file in a messy state.
\item
The generated document will always carry the filename
of the main document. This is inconvenient if
several child files are to be compiled and
to be kept for distribution.
\end{itemize}

The present package provides a simple interface
to make child files individually compilable by \LaTeX{}.
Compiling a child file then has the same effect as compiling
the main file with an |\includeonly| command
to select the appropriate child.
Moreover the generated document will carry the name of the child
rather than the main file.
This resolves all three above issues.

This feature is meant to make the editing of books,
thesis documents and lecture notes somewhat more convenient.
However, the package can also be used efficiently for
composing a series of documents (such as exercise sheets)
which are typically distributed individually.
It then assists the author in generating the individual documents
(potentially in different versions)
as well as a document containing the collected series.
Another application is in developing style files
or other kinds of included material
where compilation of the style file could redirect
to a sample or test file.

%%%%%%%%%%%%%%%%%%%%%%%%%%%%%%%%%%%%%%%%%%%%%%%%%%%%%%%%%%%%%%%%%%%%%%%%%%%%%%%%
%%%%%%%%%%%%%%%%%%%%%%%%%%%%%%%%%%%%%%%%%%%%%%%%%%%%%%%%%%%%%%%%%%%%%%%%%%%%%%%%
\section{Usage}

First of all, the package \textsf{childdoc} is \emph{not} a standard
\LaTeXe{} |.sty| style file! Therefore it needs to be invoked in
a non-standard way.

%%%%%%%%%%%%%%%%%%%%%%%%%%%%%%%%%%%%%%%%%%%%%%%%%%%%%%%%%%%%%%%%%%%%%%%%%%%%%%%%
\subsection{Included Files}
\label{sec:include}

%%%%%%%%%%%%%%%%%%%%%%%%%%%%%%%%%%%%%%%%
\DescribeMacro{\childdocmain}
To use the package, add the commands
\begin{center}
\begin{tabular}{l}
|\input{childdoc.def}|\\
|\childdocmain{}|\\
\end{tabular}
\end{center}
at the very top of the main \LaTeX{} file,
in particular \emph{before} the |\documentclass| statement!
The argument of |\childdocmain| should be left empty
(but it must be present).

%%%%%%%%%%%%%%%%%%%%%%%%%%%%%%%%%%%%%%%%
\DescribeMacro{\childdocof}
Furthermore, add the commands
\begin{center}
\begin{tabular}{l}
|\input{childdoc.def}|\\
|\childdocof{|\textit{main}|}|\\
\end{tabular}
\end{center}
at the top of every child file \textit{child}
which is included by |\include{|\textit{child}|}|
from within the main file
(or at least for those files to be compiled individually).
The argument \textit{main} must be the filename of the main file.

There are a couple of
considerations in setting up the main and child documents:

%%%%%%%%%%%%%%%%%%%%%%%%%%%%%%%%%%%%%%%%
\paragraph{Restrictions.}

Please note the following restrictions:
\begin{itemize}
\item
|\childdocmain| must be called with one argument \textit{main}
to ensure compatibility with earlier version of the package.
It must either be empty (|\childdocmain{}|)
or precisely match the filename of the main file in which it is specified.
See \secref{sec:detection} for further information.
\item
The filename \textit{main} must be specified without the |.tex| extension.
\item
The filename \textit{main} is case sensitive
(even in case-insensitive file systems)
due to internal string comparison.
\item
The argument \textit{main} should be fully expanded, it cannot be a macro.
\item
Subdirectories and special characters should be avoided in filenames.
\item
The command |\childdocmain{|\textit{main}|}| must be followed by a whitespace.
It should not be followed immediately by another command
or by a comment mark `|%|'.
This is because the \TeX{} parser reads the token immediately following
the argument of |\childdocmain| and puts it
at the beginning of every child section;
however, a white\-space is ignored.
\end{itemize}

%%%%%%%%%%%%%%%%%%%%%%%%%%%%%%%%%%%%%%%%
\paragraph{Content of Main File.}

It is advisable to place all content in the child files included by |\include|.
Any output contained in the main file will appear in all child documents
unless suppressed manually;
it cannot be suppressed automatically by the |\includeonly| directive
and thus should normally be avoided.
A method to include some content in the main file
by means of conditional processing is described in \secref{sec:conditional}.

%%%%%%%%%%%%%%%%%%%%%%%%%%%%%%%%%%%%%%%%
\paragraph{Page Numbering.}

When only a part of the document is compiled,
the appropriate numbering of pages
(as well as other status parameters)
is determined from the |.aux| files.
The latter contain information from previous passes.
However this information needs to propagate through
all intermediate child documents.
Therefore the page numbering in child documents may well
be inconsistent until the complete document is compiled at least once.

A useful (if unconventional) way to always ensure a consistent
page numbering is to restart the numbering in each child document
and denote the pages by `\textit{child}|.|\textit{page}'
where \textit{child} represents the chapter/section number of the child file.
This can be achieved by the command
|\numberwithin{page}{|\textit{child}|}|
of the \textsf{amsmath} package
where \textit{child} can be |chapter| or |section|
depending on the chosen structuring.
Alternatively, one can modify the macro |\thepage| appropriately
and reset the counter |page| at the start of each child file.

%%%%%%%%%%%%%%%%%%%%%%%%%%%%%%%%%%%%%%%%%%%%%%%%%%%%%%%%%%%%%%%%%%%%%%%%%%%%%%%%
\subsection{Conditional Processing}
\label{sec:conditional}

The package provides a mechanism to compile different versions
of a document. To customise the versions further some conditional processing
can come in handy to distinguish which version is being compiled.
The package provides two macros to describe the compilation context:

%%%%%%%%%%%%%%%%%%%%%%%%%%%%%%%%%%%%%%%%
\DescribeMacro{\ifchilddoc}
The conditional |\ifchilddoc| distinguishes between the compilation of
child documents and the main document:
%
\begin{center}
|\ifchilddoc |\textit{child-code}| |[|\||else |\textit{main-code}]| \||fi|
\end{center}

%%%%%%%%%%%%%%%%%%%%%%%%%%%%%%%%%%%%%%%%
\DescribeMacro{\childdocname}
\DescribeMacro{\childdocjob}
The macro |\childdocname| contains the filename (without extension)
of the main or child file being processed.
Note that |\childdocjob| will always contain the name of the main file.

%%%%%%%%%%%%%%%%%%%%%%%%%%%%%%%%%%%%%%%%
\paragraph{Title Page.}

Conditional processing can be used to include a title or banner page
in the main document when proper precautions are taken.
Importantly, the code in the main file should ensure that the page counter
(as well as other status parameters which are stored in the |.aux| files)
takes the same value after the conditional processing.
Otherwise the page numbers may take divergent values
depending on which part is compiled.

For example, a title page could be declared by:
%
\begin{center}
\begin{tabular}{l}
|\ifchilddoc\||else|\\
|\addtocounter{page}{-1}|\\
\textit{code for title page}\\
|\newpage|\\
|\||fi|
\end{tabular}
\end{center}
%
A banner page for the child documents can be generated by:
%
\begin{center}
\begin{tabular}{l}
|\ifchilddoc|\\
|\addtocounter{page}{-1}|\\
\textit{code for banner page}\\
|\newpage|\\
|\||fi|
\end{tabular}
\end{center}
%
Here one could write a message such as:
\begin{center}
|This is the part \childdocname{} of \childdocjob{}.|
\end{center}

%%%%%%%%%%%%%%%%%%%%%%%%%%%%%%%%%%%%%%%%%%%%%%%%%%%%%%%%%%%%%%%%%%%%%%%%%%%%%%%%
\subsection{Flags}
\label{sec:flags}

The package makes it easy to generate different versions
of the main or child documents.
To this end compilation flags can be defined
and assigned different default values.
They will be particularly useful in conjunction
with the forwarding mechanism described in \secref{sec:forward}.

For example, it may be useful to have a flag |\version|
which can be set to |draft| or |final|.
The document source will contain some conditional code
depending on the value of |\version|.
Suppose further, the flag should default to |final| for the main file
and to |draft| for child files
which is a natural assignment for editing the document.
This is achieved by placing the following code
in the preamble of the main document
(below the |\childdocmain| directive):
%
\begin{center}
\begin{tabular}{l}
|\ifchilddoc|\\
|\providecommand{\version}{draft}|\\
|\||else|\\
|\providecommand{\version}{final}|\\
|\||fi|
\end{tabular}
\end{center}
%
The definition by |\providecommand| makes sure
that previous definitions are not overwritten.
Further statements |\providecommand{\version}{...}|
can thus be added before the above code to override it.

For the main file, one might add a line
(between |\childdocmain| and the above block)
%
\begin{center}
|%\ifchilddoc\||else\providecommand{\version}{draft}\||fi|
\end{center}
%
which can be uncommented to produce a draft version.
Likewise one can add a line to the very top of a child file
(above the |\childdocof{|\textit{main}|}| directive)
%
\begin{center}
|%\providecommand{\version}{final}|
\end{center}
%
which can be uncommented to produce the final version of this child document.

%%%%%%%%%%%%%%%%%%%%%%%%%%%%%%%%%%%%%%%%%%%%%%%%%%%%%%%%%%%%%%%%%%%%%%%%%%%%%%%%
\subsection{Forwarding}
\label{sec:forward}

Different versions of the main or child documents
using compilation flags as described in \secref{sec:flags}
can be (permanently) stored in different files
for convenient compilation, viewing and distribution.
To this end, the package defines a command
to pass on compilation to a different file:

%%%%%%%%%%%%%%%%%%%%%%%%%%%%%%%%%%%%%%%%
\DescribeMacro{\childdocforward}
The command |\childdocforward| redirects processing to
another source file:
%
\begin{center}
\begin{tabular}{l}
|\input{childdoc.def}|\\
|\childdocforward[|\textit{main}|]{|\textit{dest}|}|\\
\end{tabular}
\end{center}
%
The argument \textit{dest} is the destination file
(without extension).
It should be the main file or one of the child files.
Note that further \textsf{childdoc} directives
such as |\childdocof| and |\childdocforward|
in the indicated file will be processed in this form.
The optional argument \textit{main}
passes on directly to the main file \textit{main}
while pretending to compile the child \textit{dest}.
This form behaves as if \textit{dest}
issues |\childdocof{|\textit{main}|}| right away,
and no further \textsf{childdoc} directives will be processed.

%%%%%%%%%%%%%%%%%%%%%%%%%%%%%%%%%%%%%%%%
\DescribeMacro{\...prefix}
In the alternative form |\childdocforwardprefix|,
%
\begin{center}
\begin{tabular}{l}
|\input{childdoc.def}|\\
|\childdocforwardprefix[|\textit{main}|]{|\textit{prefix}|}{|\textit{dest}|}|
\end{tabular}
\end{center}
%
the destination file is determined by a pattern
depending on the current file:
To make this work, the current file must be called
`{\textit{prefix}\hspace{0.2em}\textit{suffix}}'
with \textit{prefix} matching precisely the argument.
Processing is then passed on to the file
`{\textit{dest}\hspace{0.2em}\textit{suffix}}'.
Surely, the same effect is achieved by
directly specifying the
argument `{\textit{dest}\hspace{0.2em}\textit{suffix}}'
in the first form.
However, that requires to set up a different file
for each child. With the alternative form of the command
all these files can have exactly the same content
which simplifies setting them up and maintaining them.

For example, the following file |draft.tex|
with a compilation flag |\version| as described in \secref{sec:flags}
compiles the main document as a draft:
%
\begin{center}
\begin{tabular}{l}
|\def\version{draft}|\\
|\input{childdoc.def}|\\
|\childdocforward{|\textit{main}|}|
\end{tabular}
\end{center}
%
Likewise, the following files |final|\textit{nn}|.tex|
compile the final version of the child document
|child|\textit{nn}|.tex|:
%
\begin{center}
\begin{tabular}{l}
|\def\version{final}|\\
|\input{childdoc.def}|\\
|\childdocforwardprefix{final}{child}|
\end{tabular}
\end{center}
%

Note that when several versions of a main file and/or of each child file
are to be generated, it may be convenient to set up a |Makefile| or
shell script to automatise the process.

%%%%%%%%%%%%%%%%%%%%%%%%%%%%%%%%%%%%%%%%%%%%%%%%%%%%%%%%%%%%%%%%%%%%%%%%%%%%%%%%
\subsection{Command Line Processing}
\label{sec:commandline}

The effect of redirection files can also be achieved by invoking
the \LaTeX{} compiler with a more elaborate command line.
Most conveniently this should be done as part
of a shell script or a |Makefile|.

When using \textsf{childdoc} in the main file, the following
command lines effectively perform a redirection
(note that depending on the shell being used,
backslashes may have to be doubled: `|\|' $\to$ `|\\|'):
%
\begin{center}
|... -jobname "|\textit{target}|" |\\|"|[\textit{flags}]%
|\input{childdoc.def}\childdocforward[|\textit{main}|]{|\textit{dest}|}"|
\end{center}
%
Here \textit{target} is the name of the output file,
\textit{main} is the name of the main file
and \textit{dest} is the name of the main or child file to be processed
(all filenames without extensions).
The optional argument \textit{main} can be omitted
if \textit{main} matches \textit{dest}.
Optionally, compilation \textit{flags} can be defined via |\def| commands.
This command line makes the \TeX{} engine believe
it is compiling the file \textit{target}
whose content is specified as the latter parameter.
The provided code then forwards the processing to
\textit{main} or \textit{dest} as described in \secref{sec:forward}.

%%%%%%%%%%%%%%%%%%%%%%%%%%%%%%%%%%%%%%%%%%%%%%%%%%%%%%%%%%%%%%%%%%%%%%%%%%%%%%%%
\subsection{Include by Input}
\label{sec:input}

Including child documents by |\include| has some restrictions by design.
Most notably, the content of a child document always occupies
its own set of pages; pages cannot be shared between child documents.
Usually, this behaviour makes perfect sense
because each child document contain an essential part of the document.
However, in some situations it may be desirable to compose
a document from a collection of parts
without having mandatory page breaks between then.
For this case, the package
provides a mechanism to include parts
by |\input| which can also be processed individually.
However, by construction this mechanism
requires manual handling of the content to be output.

%%%%%%%%%%%%%%%%%%%%%%%%%%%%%%%%%%%%%%%%
\DescribeMacro{\ifchilddocmanual}
The main file should be prepared as usual, see \secref{sec:include}.
However, the document body must make a distinction
between processing of an individual part and of the main document, e.g.:
%
\begin{center}
\begin{tabular}{l}
|\ifchilddocmanual|\\
|\input{\childdocname}|\\
|\||else|\\
\textit{document body with }|\input{|\textit{part}|}|\\
|\||fi|
\end{tabular}
\end{center}
%
The conditional |\ifchilddocmanual| is true whenever
a part to be included by |\input| is being compiled,
and the name of the part is stored in |\childdocname|.

%%%%%%%%%%%%%%%%%%%%%%%%%%%%%%%%%%%%%%%%
\DescribeMacro{\childdocby}
Each part to be included by |\input| should start with:
%
\begin{center}
\begin{tabular}{l}
|\input{childdoc.def}|\\
|\childdocby{|\textit{main}|}|\\
\end{tabular}
\end{center}
%
The directive |\childdocby| is similar to |\childdocof|
described in \secref{sec:include},
but the subsequent selection of content must be done manually.
To that end, both |\ifchilddoc| and |\ifchilddocmanual|
will be true upon processing of a part,
and the name of the part is stored in |\childdocname|.
Note that |\jobname| will be set to the filename of the current part
so that each part receives an individual |.aux| file
that does not interfere with the |.aux| file(s) of the main document.
This behaviour can be altered by the alternative form
|\childdocby[*]{|\textit{main}|}| (with a non-empty optional argument)
which uses the |.aux| file of the main document
by setting |\jobname| to \textit{main}.

%%%%%%%%%%%%%%%%%%%%%%%%%%%%%%%%%%%%%%%%%%%%%%%%%%%%%%%%%%%%%%%%%%%%%%%%%%%%%%%%
\subsection{Driver Development}
\label{sec:driver}

The \textsf{childdoc} mechanism can also be use for the development
of definition files such as \LaTeX{} styles or classes.
This case differs from the above setup with multiple parts
included by |\include| in that no |\includeonly| should be invoked.
This can be achieved by starting the include file
(before |\ProvidesPackage|) with:
%
\begin{center}
\begin{tabular}{l}
|\input{childdoc.def}|\\
|\childdocforward{|\textit{main}|}|\\
\end{tabular}
\end{center}
%
or alternatively with:
%
\begin{center}
\begin{tabular}{l}
|\input{childdoc.def}|\\
|\childdocby{|\textit{main}|}|\\
\end{tabular}
\end{center}
%
Both forms have slightly different effects as described above.
The main file is prepared as usual, see \secref{sec:include}.

%%%%%%%%%%%%%%%%%%%%%%%%%%%%%%%%%%%%%%%%%%%%%%%%%%%%%%%%%%%%%%%%%%%%%%%%%%%%%%%%
\subsection{Legacy Detection}
\label{sec:detection}

The directive |\childdocmain| in the main file can detect
whether the complete document or merely a child is to be compiled
even without using the directive |\childdocof|.
This method is deprecated because it is less robust
and there is no compelling reason to use it;
it is merely provided for backward compatibility
and it may be removed in future versions.

If the detection mechanism is to be used,
it is mandatory to correctly specify
the filename of the main file as the argument of |\childdocmain|:
%
\begin{center}
\begin{tabular}{l}
|\input{childdoc.def}|\\
|\childdocmain{|\textit{main}|}|\\
\end{tabular}
\end{center}
%
If |\jobname| does not match the argument \textit{main} of |\childdocmain|,
it is assumed that |\jobname| points to the child file to be compiled.
When using |\childdocmain| with the main file specified as argument,
it suffices to start a child file
with just |\input{|\textit{main}|}|
without loading of the package and using |\childdocof|.
If instead all processing is done
with the appropriate \textsf{childdoc} directives,
the argument of \textit{main} of |\childdocmain| can be empty.

An alternative version of the command line processing described
in \secref{sec:commandline} using the detection mechanism reads:
%
\begin{center}
|... -jobname "|\textit{target}|" "|[\textit{flags}]%
[|\def\jobname{|\textit{dest}|}|]|\input{|\textit{main}|}"|
\end{center}

%%%%%%%%%%%%%%%%%%%%%%%%%%%%%%%%%%%%%%%%%%%%%%%%%%%%%%%%%%%%%%%%%%%%%%%%%%%%%%%%
\subsection{Manual Code}
\label{sec:manual}

In case one cannot be certain whether the definitions file |childdoc.def|
is installed on the target \TeX{} distribution
and one prefers not to ship it,
it is conceivable to paste a few relevant commands into the sources.

To that end, drop all statements |\input{childdoc.def}|
and perform the replacements as outlined below.
Instead of |\childdocmain{|\textit{main}|}| add the following code
to the top of the main file:
%
\begin{center}
\begin{tabular}{l}
|\||ifdefined\childdocname\endinput\||fi\newif\ifchilddoc|\\
|\edef\childdocname{\scantokens\expandafter{\jobname\noexpand}}|\\
|\def\childdocmain{|\textit{main}|}\||ifx\childdocmain\childdocname\||else|\\
|\childdoctrue\includeonly{\childdocname}\let\jobname\childdocmain\||fi|\\
\end{tabular}
\end{center}
%
Instead of |\childdocof{|\textit{main}|}| just include the main file
at the top of each child file:
%
\begin{center}
|\input{|\textit{main}|}|
\end{center}
%
A simple redirection |\childdocforward{|\textit{dest}|}| is achieved by:
%
\begin{center}
|\def\jobname{|\textit{dest}|}\input{\jobname}|
\end{center}
%
The redirection with prefix
|\childdocforwardprefix[|\textit{prefix}|]{|\textit{dest}|}|
is accomplished by:
%
\begin{center}
\begin{tabular}{l}
|{\edef\jobname{\scantokens\expandafter{\jobname\noexpand}}|\\
|\def\redirectjob |\textit{prefix}|#1~~~{\gdef\jobname{|\textit{dest}|#1}}|\\
|\expandafter\redirectjob\jobname~~~}\input{\jobname}|
\end{tabular}
\end{center}

In an alternative approach,
child documents can be compiled by a specific command line
without additional code or specific definitions:
%
\begin{center}
|... -jobname "|\textit{target}|" "|[\textit{flags}]%
|\includeonly{|\textit{dest}|}\input{|\textit{main}|}"|
\end{center}
%

%%%%%%%%%%%%%%%%%%%%%%%%%%%%%%%%%%%%%%%%%%%%%%%%%%%%%%%%%%%%%%%%%%%%%%%%%%%%%%%%
%%%%%%%%%%%%%%%%%%%%%%%%%%%%%%%%%%%%%%%%%%%%%%%%%%%%%%%%%%%%%%%%%%%%%%%%%%%%%%%%
\section{Information}

%%%%%%%%%%%%%%%%%%%%%%%%%%%%%%%%%%%%%%%%%%%%%%%%%%%%%%%%%%%%%%%%%%%%%%%%%%%%%%%%
\subsection{Copyright}

Copyright \copyright{} 2017--2018 Niklas Beisert

This work may be distributed and/or modified under the
conditions of the \LaTeX{} Project Public License, either version 1.3
of this license or (at your option) any later version.
The latest version of this license is in
  \url{http://www.latex-project.org/lppl.txt}
and version 1.3 or later is part of all distributions of \LaTeX{}
version 2005/12/01 or later.

This work has the LPPL maintenance status `maintained'.

The Current Maintainer of this work is Niklas Beisert.

This work consists of the files |README.txt|, |childdoc.ins| and |childdoc.dtx|
as well as the derived files |childdoc.def|, |cdocsamp.tex|
with |cdocsch1.tex|, |cdocsch2.tex|, |cdocspt3.tex|, |cdocspt4.tex|,
|cdocsdrf.tex|, |cdocsfn1.tex|, |cdocsfn2.tex|
as well as |childdoc.pdf|.

%%%%%%%%%%%%%%%%%%%%%%%%%%%%%%%%%%%%%%%%%%%%%%%%%%%%%%%%%%%%%%%%%%%%%%%%%%%%%%%%
\subsection{Files and Installation}

The package consists of the files:
%
\begin{center}
\begin{tabular}{ll}
    |README.txt|   & readme file \\
    |childdoc.ins| & installation file \\
    |childdoc.dtx| & source file \\
    |childdoc.def| & definition file \\
    |cdocsamp.tex| & sample main file \\
    |cdocsch1.tex| & sample include file \\
    |cdocsch2.tex| & sample include file \\
    |cdocspt3.tex| & sample part file \\
    |cdocspt4.tex| & sample part file \\
    |cdocsdrf.tex| & sample redirection file \\
    |cdocsfn1.tex| & sample redirection file \\
    |cdocsfn2.tex| & sample redirection file \\
    |childdoc.pdf| & manual
\end{tabular}
\end{center}
%
The distribution consists of the files
|README.txt|, |childdoc.ins| and |childdoc.dtx|.
%
\begin{itemize}
\item
Run (pdf)\LaTeX{} on |childdoc.dtx|
to compile the manual |childdoc.pdf| (this file).
\item
Run \LaTeX{} on |childdoc.ins| to create the definitions file |childdoc.def|
and the sample |cdocsamp.tex| with include files
|cdocsch1.tex|, |cdocsch2.tex|, |cdocspt3.tex|, |cdocspt4.tex|,
|cdocsdrf.tex|, |cdocsfn1.tex|, |cdocsfn2.tex|.
Then copy the file |childdoc.def| to an appropriate directory of your \LaTeX{}
distribution, e.g.\ \textit{texmf-root}|/tex/latex/childdoc|.
\end{itemize}

%%%%%%%%%%%%%%%%%%%%%%%%%%%%%%%%%%%%%%%%%%%%%%%%%%%%%%%%%%%%%%%%%%%%%%%%%%%%%%%%
\subsection{Related CTAN Packages}

There are several other packages which offer a similar functionality:
%
\begin{itemize}
\item
The packages
\href{http://ctan.org/pkg/docmute}{\textsf{docmute}},
\href{http://ctan.org/pkg/includex}{\textsf{includex}} and
\href{http://ctan.org/pkg/standalone}{\textsf{standalone}}
provide commands to include only the document body of
a child file thus allowing both files to be compiled individually.
\item
The packages \href{http://ctan.org/pkg/subdocs}{\textsf{subdocs}}
and \href{http://ctan.org/pkg/subfiles}{\textsf{subfiles}}
provide structures in which the main and child documents can be
encapsulated and allowing them to be compiled individually.
The inclusion mechanism is different from the conventional |\include|.
\item
The package \href{http://ctan.org/pkg/combine}{\textsf{combine}}
is an elaborate solution to combine several documents into one.
\end{itemize}
%
See also the CTAN topic \href{http://ctan.org/topic/subdocs}{\textsf{subdocs}}
for further related packages.
The present package differs from the above solutions in that
a document structure constructed with the conventional |\include| mechanism
just needs two extra commands at the top of every file
such that all constituent files can be compiled individually.

%%%%%%%%%%%%%%%%%%%%%%%%%%%%%%%%%%%%%%%%%%%%%%%%%%%%%%%%%%%%%%%%%%%%%%%%%%%%%%%%
%\subsection{Feature Suggestions}
%
%The following is a list of features which may be useful for future
%versions of this package:
%%
%\begin{itemize}
%\item
%\ldots
%\end{itemize}

%%%%%%%%%%%%%%%%%%%%%%%%%%%%%%%%%%%%%%%%%%%%%%%%%%%%%%%%%%%%%%%%%%%%%%%%%%%%%%%%
\subsection{Revision History}

%%%%%%%%%%%%%%%%%%%%%%%%%%%%%%%%%%%%%%%%
\paragraph{v2.0:} 2018/12/30

\begin{itemize}
\item
immediate forward processing
\item
added |\childdocby| mechanism
\item
manual restructured
\end{itemize}

%%%%%%%%%%%%%%%%%%%%%%%%%%%%%%%%%%%%%%%%
\paragraph{v1.6:} 2018/01/17

\begin{itemize}
\item
application for development of include files
\item
corrections to manual
\end{itemize}

%%%%%%%%%%%%%%%%%%%%%%%%%%%%%%%%%%%%%%%%
\paragraph{v1.5:} 2017/05/21

\begin{itemize}
\item
more complete structuring introduced
\item
|\childdocof| introduced
\item
|\childdoc| renamed to |\childdocmain|
\item
|\childredirect| renamed to |\childdocforward| and |\childdocforwardprefix|
and functionality expanded
\end{itemize}

%%%%%%%%%%%%%%%%%%%%%%%%%%%%%%%%%%%%%%%%
\paragraph{v1.0:} 2017/04/27

\begin{itemize}
\item
manual and install package
\item
first version published on CTAN
\end{itemize}

%%%%%%%%%%%%%%%%%%%%%%%%%%%%%%%%%%%%%%%%
\paragraph{v0.6:} 2017/04/26

\begin{itemize}
\item
redirection mechanism added
\end{itemize}

%%%%%%%%%%%%%%%%%%%%%%%%%%%%%%%%%%%%%%%%
\paragraph{v0.5:} 2017/04/26

\begin{itemize}
\item
functionality in definition file
\end{itemize}


%%%%%%%%%%%%%%%%%%%%%%%%%%%%%%%%%%%%%%%%%%%%%%%%%%%%%%%%%%%%%%%%%%%%%%%%%%%%%%%%
%%%%%%%%%%%%%%%%%%%%%%%%%%%%%%%%%%%%%%%%%%%%%%%%%%%%%%%%%%%%%%%%%%%%%%%%%%%%%%%%
%%%%%%%%%%%%%%%%%%%%%%%%%%%%%%%%%%%%%%%%%%%%%%%%%%%%%%%%%%%%%%%%%%%%%%%%%%%%%%%%
\appendix

\settowidth\MacroIndent{\rmfamily\scriptsize 000\ }

 \DocInput{childdoc.dtx}

\end{document}
%</driver>
% \fi
%
% %%%%%%%%%%%%%%%%%%%%%%%%%%%%%%%%%%%%%%%%%%%%%%%%%%%%%%%%%%%%%%%%%%%%%%%%%%%%%%
% %%%%%%%%%%%%%%%%%%%%%%%%%%%%%%%%%%%%%%%%%%%%%%%%%%%%%%%%%%%%%%%%%%%%%%%%%%%%%%
% \section{Sample}
%\iffalse
%<*samplemain>
%\fi
%
% The following presents a sample document
% with two chapters, two parts, a title page,
% a compile flag as well as three forwarding files to set the flag.
% It consists of eight |.tex| files:
% \begin{center}
% \begin{tabular}{ll}
% |cdocsamp.tex|&main file\\
% |cdocsch1.tex|&include file for chapter 1\\
% |cdocsch2.tex|&include file for chapter 2\\
% |cdocspt3.tex|&include file for part 3\\
% |cdocspt4.tex|&include file for part 4\\
% |cdocsdrf.tex|&forwarding file for main file in draft mode\\
% |cdocsfi1.tex|&forwarding file for final version of chapter 1\\
% |cdocsfi2.tex|&forwarding file for final version of chapter 2\\
% \end{tabular}
% \end{center}
% Each of the eight files can be compiled directly by the \LaTeX{} compiler.
%
% %%%%%%%%%%%%%%%%%%%%%%%%%%%%%%%%%%%%%%
% \paragraph{Main File.}
%
% The main file is called |cdocsamp.tex|.
%
% Load the \textsf{childdoc} definitions and
% declare the filename for the main document:
%    \begin{macrocode}
\input{childdoc.def}
\childdocmain{}
%    \end{macrocode}

% Optional override for |\version| flag:
%    \begin{macrocode}
%%\ifchilddoc\else\providecommand{\version}{draft}\fi
%    \end{macrocode}

% Define the default values for the |\version| flag
% (|final| for the main file and |draft| for childs):
%    \begin{macrocode}
\ifchilddoc
\providecommand{\version}{draft}
\else
\providecommand{\version}{final}
\fi
%    \end{macrocode}

% Load the standard document class:
%    \begin{macrocode}
\documentclass[12pt]{article}
%    \end{macrocode}

% Start the document body:
%    \begin{macrocode}
\begin{document}
%    \end{macrocode}

% Declare a title page.
% Print title, part of document being processed and version flag:
%    \begin{macrocode}
\addtocounter{page}{-1}
\begin{center}
{\LARGE\bfseries{}childdoc example\par}
\vspace{1cm}
\ifchilddoc
\ifchilddocmanual part\else chapter\fi:
`\childdocname' of `\childdocjob'\par
\else
main document: `\childdocjob'\par
\fi
version: \version\par
\end{center}
\newpage
%    \end{macrocode}

% Manually include selected file,
% otherwise process as usual:
%    \begin{macrocode}
\ifchilddocmanual
\section*{part `\childdocname'}
\input{\childdocname}
\else
%    \end{macrocode}

% Include the two chapters:
%    \begin{macrocode}
\include{cdocsch1}
\include{cdocsch2}
%    \end{macrocode}

% Include the two parts unless only chapters should be displayed:
%    \begin{macrocode}
\ifchilddoc\else
\section{part three}
\input{cdocspt3}
\section{part four}
\input{cdocspt4}
\fi
%    \end{macrocode}

% Process as usual until here:
%    \begin{macrocode}
\fi
%    \end{macrocode}

% End of document body:
%    \begin{macrocode}
\end{document}
%    \end{macrocode}
%\iffalse
%</samplemain>
%\fi
%
% %%%%%%%%%%%%%%%%%%%%%%%%%%%%%%%%%%%%%%
% \paragraph{Chapter Include Files.}
%
% The include files are called |cdocsch1.tex| and |cdocsch2.tex|.
%
%\iffalse
%<*samplechap1|samplechap2>
%\fi

% Optional override for |\version| flag:
%    \begin{macrocode}
%%\providecommand{\version}{final}
%    \end{macrocode}

% Include the main document:
%    \begin{macrocode}
\input{childdoc.def}
\childdocof{cdocsamp}
%    \end{macrocode}

%\iffalse
%</samplechap1|samplechap2>
%\fi
%
%\iffalse
%<*samplechap1>
%\fi
% Some text for chapter 1:
%    \begin{macrocode}
\section{one}
some text in chapter one
%    \end{macrocode}

%\iffalse
%</samplechap1>
%\fi
% Some text for chapter 2:
%\iffalse
%<*samplechap2>
%\fi
%    \begin{macrocode}
\section{two}
more text in chapter two
%    \end{macrocode}

%\iffalse
%</samplechap2>
%\fi
%
% %%%%%%%%%%%%%%%%%%%%%%%%%%%%%%%%%%%%%%
% \paragraph{Part Include Files.}
%
% The include files are called |cdocspt3.tex| and |cdocspt4.tex|.
%
%\iffalse
%<*samplepart3|samplepart4>
%\fi

% Optional override for |\version| flag:
%    \begin{macrocode}
%%\providecommand{\version}{final}
%    \end{macrocode}

% Include the main document:
%    \begin{macrocode}
\input{childdoc.def}
\childdocby{cdocsamp}
%    \end{macrocode}

%\iffalse
%</samplepart3|samplepart4>
%\fi
%
%\iffalse
%<*samplepart3>
%\fi
% Some text for part 3:
%    \begin{macrocode}
some text in part three
%    \end{macrocode}

%\iffalse
%</samplepart3>
%\fi
% Some text for part 4:
%\iffalse
%<*samplepart4>
%\fi
%    \begin{macrocode}
more text in part four
%    \end{macrocode}

%\iffalse
%</samplepart4>
%\fi
%
% %%%%%%%%%%%%%%%%%%%%%%%%%%%%%%%%%%%%%%
% \paragraph{Forwarding for a Complete Draft.}
%
% The following forwarding file |cdocsdrf.tex|
% compiles the main document in draft mode:
%\iffalse
%<*sampledraft>
%\fi
%    \begin{macrocode}
\def\version{draft}
\input{childdoc.def}
\childdocforward{cdocsamp}
%    \end{macrocode}

%\iffalse
%</sampledraft>
%\fi
%
% %%%%%%%%%%%%%%%%%%%%%%%%%%%%%%%%%%%%%%
% \paragraph{Forwarding for Final Version of the Chapters.}
%
% The following forwarding files |cdocsfn1.tex| and |cdocsfn2.tex|
% (with identical content)
% compile the final versions of the child documents
% |cdocsch1.tex| and |cdocsch2.tex|, respectively:
%\iffalse
%<*samplefinal>
%\fi
%    \begin{macrocode}
\def\version{final}
\input{childdoc.def}
\childdocforwardprefix[cdocsamp]{cdocsfn}{cdocsch}
%    \end{macrocode}

%\iffalse
%</samplefinal>
%\fi
%
% %%%%%%%%%%%%%%%%%%%%%%%%%%%%%%%%%%%%%%
% \paragraph{Command Line Processing.}
%
% The following three command lines generate the output files
% |cdocscld|, |cdocscl1| and |cdocscl2|
% which should be identical to
% |cdocsdrf|, |cdocsch1| and |cdocsfn2|, respectively:
% \begin{center}
% \begin{tabular}{l}
% |latex -jobname cdocscld \|\\
% |  "\def\version{draft}\input{childdoc.def}\childdocforward{cdocsamp}"|\\
% |latex -jobname cdocscl1 \|\\
% |  "\input{childdoc.def}\childdocforward[cdocsamp]{cdocsch1}"|\\
% |latex -jobname cdocscl2 \|\\
% |  "\def\version{final}\input{childdoc.def}\childdocforward{cdocsch2}"|
% \end{tabular}
% \end{center}
% Note that the trailing backslash on each first line
% merely continues the input to the second line
% (for convenient cut ant paste).
% Furthermore, the command |latex| can be replaced by any
% of its alternative versions such as |pdflatex|.
%
% %%%%%%%%%%%%%%%%%%%%%%%%%%%%%%%%%%%%%%%%%%%%%%%%%%%%%%%%%%%%%%%%%%%%%%%%%%%%%%
% %%%%%%%%%%%%%%%%%%%%%%%%%%%%%%%%%%%%%%%%%%%%%%%%%%%%%%%%%%%%%%%%%%%%%%%%%%%%%%
% \section{Implementation}
%\iffalse
%<*package>
%\fi
%
% This section describes the definitions file |childdoc.def|.

% The definitions cannot be loaded using |\usepackage| or |\RequirePackage|
% which has a mechanism to prevent loading a style file more than once.
% When loading the definitions by means of |\input|
% multiple instances have to be prevented manually:
%\iffalse
%This code needs to be before the `\ProvidesFile' directive
%which is defined at the beginning of this file.
%Therefore it is also placed there and commented out here.
%</package>
%<*discard>
%\fi
%    \begin{macrocode}
\ifdefined\childdocmain\endinput\fi
%    \end{macrocode}
%\iffalse
%</discard>
%<*package>
%\fi
%
% \macro{\ifchilddoc}
% \macro{\ifchilddocmanual}
% The conditional |\ifchilddoc| tells whether a
% child (true) or main (false) document is being compiled.
% The conditional |\ifchilddocmanual| tells whether
% the |\includeonly| mechanism is used (false) or
% the selection of child files must be performed manually (true).
% The definitions initialise to false:
%    \begin{macrocode}
\newif\ifchilddoc
\newif\ifchilddocmanual
%    \end{macrocode}

% \macro{\childdocname}
% \macro{\childdocjob}
% The macro |\childdocname| stores the name of the main document
% to be compiled. The macro |\childdocjob| stores the name of
% the document on which the \LaTeX{} compiler was originally invoked.
% The content of |\jobname| cannot be compared
% to filenames specified in the source due to different catcodes.
% The following code rescans |\jobname|, stores the result
% in |\childdocname| and saves a copy in |\childdocjob|:
%    \begin{macrocode}
\edef\childdocname{\scantokens\expandafter{\jobname\noexpand}}
\let\childdocjob\childdocname
%    \end{macrocode}

% \macro{\childdocdisable}
% The macro |\childdocdisable| prevents the main file
% from being processed more than once.
% At this stage, the main document command |\childdocmain|
% is assumed to be called once again where it should do nothing.
% Any subsequent call to it should prevent
% a secondary processing of the main document
% It overwrites the forwarding commands
% |\childdocof| and |\childdocforward|
% with empty macros to prevent further inclusions of the main document:
%    \begin{macrocode}
\newcommand{\childdocdisable}
{
  \renewcommand{\childdocmain}[1]{\renewcommand{\childdocmain}[1]{\endinput}}
  \renewcommand{\childdocof}[1]{}
  \renewcommand{\childdocby}[2][]{}
  \renewcommand{\childdocforward}[2][]{}
  \renewcommand{\childdocdisable}{}
}
%    \end{macrocode}

% \macro{\childdocmain}
% The macro |\childdocmain| is to be called at the top of the main file
% with nothing or the main filename (without extension) as argument.
% First, it breaks loops.
% If the argument is not empty and does not match |\childdocname|
% (which is set by the first inclusion of |childdoc.def|),
% |\ifchilddoc| is set to true, |\includeonly| is applied to the child file
% and |\jobname| is set to the main file
% (for proper handling of |.aux| files):
%    \begin{macrocode}
\newcommand{\childdocmain}[1]
{
  \childdocdisable\childdocmain{}
  \if?#1?\else
    \begingroup
      \def\childdoctmp{#1}
      \ifx\childdoctmp\childdocname
        \def\childdoctmp{}
      \else
        \def\childdoctmp
        {
          \childdoctrue
          \includeonly{\childdocname}
          \def\childdocjob{#1}
          \def\jobname{#1}
        }
      \fi
      \expandafter
    \endgroup
    \childdoctmp
  \fi
}
%    \end{macrocode}

% \macro{\childdocof}
% The command |\childdocof| redirects
% compilation to the main file |#1|.
%    \begin{macrocode}
\newcommand{\childdocof}[1]
{
  \childdocdisable
  \childdoctrue
  \includeonly{\childdocname}
  \def\jobname{#1}
  \def\childdocjob{#1}
  \input{#1}
}
%    \end{macrocode}

% \macro{\childdocby}
% The command |\childdocby| ....
%    \begin{macrocode}
\newcommand{\childdocby}[2][]
{
  \childdocdisable
  \childdoctrue
  \childdocmanualtrue
  \if?#1?\else
    \def\jobname{#2}
  \fi
  \def\childdocjob{#2}
  \input{#2}
  \endinput
}
%    \end{macrocode}

% \macro{\childdocforward}
% The command |\childdocforward| redirects
% compilation to the main file or
% (if the optional argument is given) a child file.
% Parameters are set as if the main file
% or a child file starting with |\childdocof| was compiled.
% Then compilation is handed over to the main file:
%    \begin{macrocode}
\newcommand{\childdocforward}[2][]
{
  \begingroup
    \if?#1?
      \def\childdoctmp
      {
        \def\childdocname{#2}
        \def\childdocjob{#2}
        \def\jobname{#2}
        \input{#2}
        \endinput
      }
    \else
      \def\childdoctmp
      {
        \childdocdisable
        \def\childdocname{#2}
        \childdoctrue
        \includeonly{#2}
        \def\childdocjob{#1}
        \def\jobname{#1}
        \input{#1}
        \endinput
      }
    \fi
    \expandafter
  \endgroup
  \childdoctmp
}
%    \end{macrocode}

% \macro{\childdocforwardprefix}
% The command |\childdocforwardprefix| redirects
% compilation to the main or a child file by means of a pattern.
% The prefix |#1| in the current filename is replaced by |#2|
% and the suffix of the current filename is kept
% (it is assumed that the filename does not contain the substring `|~~~|'
% which is used as a delimiter).
% Compilation is handed over to the new file by |\childdocforward|:
%    \begin{macrocode}
\newcommand{\childdocforwardprefix}[3][]
{
  \begingroup
    \def\childdocextract #2##1~~~{\def\childdoctmp{\childdocforward[#1]{#3##1}}}
    \expandafter\childdocextract\childdocname~~~
    \expandafter
  \endgroup
  \childdoctmp
}
%    \end{macrocode}

% \macro{\childdoc}
% The deprecated macro |\childdoc| is a legacy version of |\childdocmain|:
%    \begin{macrocode}
\newcommand{\childdoc}{\childdocmain}
%    \end{macrocode}

% \macro{\childdocredirect}
% The deprecated macro |\childdocredirect| is a legacy version
% of |\childdocforward| and |\childdocforwardprefix|:
%    \begin{macrocode}
\newcommand{\childdocredirect}[2][]
{
  \begingroup
    \if?#1?
      \def\childdoctmp{\childdocforward{#2}}
    \else
      \def\childdoctmp{\childdocforwardprefix{#1}{#2}}
    \fi
    \expandafter
  \endgroup
  \childdoctmp
}
%    \end{macrocode}

%\iffalse
%</package>
%\fi
%
\endinput
\childdocforward[cdocsamp]{cdocsch1}"|\\
% |latex -jobname cdocscl2 \|\\
% |  "\def\version{final}% \iffalse
%
% childdoc.dtx Copyright (C) 2017-2018 Niklas Beisert
%
% This work may be distributed and/or modified under the
% conditions of the LaTeX Project Public License, either version 1.3
% of this license or (at your option) any later version.
% The latest version of this license is in
%   http://www.latex-project.org/lppl.txt
% and version 1.3 or later is part of all distributions of LaTeX
% version 2005/12/01 or later.
%
% This work has the LPPL maintenance status `maintained'.
%
% The Current Maintainer of this work is Niklas Beisert.
%
% This work consists of the files childdoc.dtx and childdoc.ins
% and the derived files childdoc.def and cdocsamp.tex with
% cdocsch1.tex, cdocsch2.tex, cdocsdrf.tex, cdocsfn1.tex, cdocsfn2.tex.
%
%<package>\ifdefined\childdocmain\endinput\fi
%<package>\ProvidesFile{childdoc.def}[2018/12/30 v2.0 child document driver]
%<samplemain>\ProvidesFile{cdocsamp.tex}[2018/12/30 v2.0 sample for childdoc]
%<*driver>
%\ProvidesFile{childdoc.drv}[2018/12/30 v2.0 childdoc reference manual file]
\PassOptionsToClass{10pt,a4paper}{article}
\documentclass{ltxdoc}

\usepackage[margin=35mm]{geometry}
\usepackage{hyperref}
\usepackage{hyperxmp}
\usepackage[usenames]{color}

\hypersetup{colorlinks=true}
\hypersetup{pdfstartview=FitH}
\hypersetup{pdfpagemode=UseNone}
\hypersetup{pdfsource={}}
\hypersetup{pdflang={en-UK}}
\hypersetup{pdfcopyright={Copyright 2017-2018 Niklas Beisert.
  This work may be distributed and/or modified under the
  conditions of the LaTeX Project Public License, either version 1.3
  of this license or (at your option) any later version.}}
\hypersetup{pdflicenseurl={http://www.latex-project.org/lppl.txt}}
\hypersetup{pdfcontactaddress={ETH Zurich, ITP, HIT K,
  Wolfgang-Pauli-Strasse 27}}
\hypersetup{pdfcontactpostcode={8093}}
\hypersetup{pdfcontactcity={Zurich}}
\hypersetup{pdfcontactcountry={Switzerland}}
\hypersetup{pdfcontactemail={nbeisert@itp.phys.ethz.ch}}
\hypersetup{pdfcontacturl={http://people.phys.ethz.ch/\xmptilde nbeisert/}}

\newcommand{\secref}[1]{\hyperref[#1]{section \ref*{#1}}}

\parskip1ex
\parindent0pt
\let\olditemize\itemize
\def\itemize{\olditemize\parskip0pt}

\begin{document}

\title{The \textsf{childdoc} Package}
\hypersetup{pdftitle={The childdoc Package}}
\author{Niklas Beisert\\[2ex]
  Institut f\"ur Theoretische Physik\\
  Eidgen\"ossische Technische Hochschule Z\"urich\\
  Wolfgang-Pauli-Strasse 27, 8093 Z\"urich, Switzerland\\[1ex]
  \href{mailto:nbeisert@itp.phys.ethz.ch}
  {\texttt{nbeisert@itp.phys.ethz.ch}}}
\hypersetup{pdfauthor={Niklas Beisert}}
\hypersetup{pdfsubject={Manual for the LaTeX2e Package childdoc}}
\date{30 December 2018, \textsf{v2.0}}
\maketitle

\begin{abstract}\noindent
\textsf{childdoc} is a \LaTeXe{} package
that enables the direct compilation
of document sections included by |\include|
to individual files.
\end{abstract}

\begingroup
\parskip0ex
\tableofcontents
\endgroup

%%%%%%%%%%%%%%%%%%%%%%%%%%%%%%%%%%%%%%%%%%%%%%%%%%%%%%%%%%%%%%%%%%%%%%%%%%%%%%%%
%%%%%%%%%%%%%%%%%%%%%%%%%%%%%%%%%%%%%%%%%%%%%%%%%%%%%%%%%%%%%%%%%%%%%%%%%%%%%%%%
\section{Introduction}

\LaTeX{} provides a mechanism to structure a large document (such as a book)
into a main file and several child files (containing the chapters)
using the |\include| command.
This mechanism is beneficial for documents
which span hundreds of pages in order to
make the source file(s) more manageable.
Moreover, compilation can be restricted to
selected child files by means of the |\includeonly| command.
The latter feature can be used to reduce the compilation time while editing
(this was significantly more useful in the earlier days of \LaTeX{})
or to generate a smaller document which is easier to navigate.
Another application of |\includeonly| is to generate
documents consisting of selected parts of the complete document.

However, there are a few drawbacks of the plain |\include| mechanism:
\begin{itemize}
\item
The child files cannot be compiled on their own,
they can only be compiled via the main file.
A naive editing environment
(such as a text editor with an option
to have the current file processed by \LaTeX)
may require one to switch to the main file before compiling;
attempting to compile the child file produces errors.
\item
The main file must be modified (each time)
to adjust the |\includeonly| command
to the present needs. This easily leaves the main file in a messy state.
\item
The generated document will always carry the filename
of the main document. This is inconvenient if
several child files are to be compiled and
to be kept for distribution.
\end{itemize}

The present package provides a simple interface
to make child files individually compilable by \LaTeX{}.
Compiling a child file then has the same effect as compiling
the main file with an |\includeonly| command
to select the appropriate child.
Moreover the generated document will carry the name of the child
rather than the main file.
This resolves all three above issues.

This feature is meant to make the editing of books,
thesis documents and lecture notes somewhat more convenient.
However, the package can also be used efficiently for
composing a series of documents (such as exercise sheets)
which are typically distributed individually.
It then assists the author in generating the individual documents
(potentially in different versions)
as well as a document containing the collected series.
Another application is in developing style files
or other kinds of included material
where compilation of the style file could redirect
to a sample or test file.

%%%%%%%%%%%%%%%%%%%%%%%%%%%%%%%%%%%%%%%%%%%%%%%%%%%%%%%%%%%%%%%%%%%%%%%%%%%%%%%%
%%%%%%%%%%%%%%%%%%%%%%%%%%%%%%%%%%%%%%%%%%%%%%%%%%%%%%%%%%%%%%%%%%%%%%%%%%%%%%%%
\section{Usage}

First of all, the package \textsf{childdoc} is \emph{not} a standard
\LaTeXe{} |.sty| style file! Therefore it needs to be invoked in
a non-standard way.

%%%%%%%%%%%%%%%%%%%%%%%%%%%%%%%%%%%%%%%%%%%%%%%%%%%%%%%%%%%%%%%%%%%%%%%%%%%%%%%%
\subsection{Included Files}
\label{sec:include}

%%%%%%%%%%%%%%%%%%%%%%%%%%%%%%%%%%%%%%%%
\DescribeMacro{\childdocmain}
To use the package, add the commands
\begin{center}
\begin{tabular}{l}
|\input{childdoc.def}|\\
|\childdocmain{}|\\
\end{tabular}
\end{center}
at the very top of the main \LaTeX{} file,
in particular \emph{before} the |\documentclass| statement!
The argument of |\childdocmain| should be left empty
(but it must be present).

%%%%%%%%%%%%%%%%%%%%%%%%%%%%%%%%%%%%%%%%
\DescribeMacro{\childdocof}
Furthermore, add the commands
\begin{center}
\begin{tabular}{l}
|\input{childdoc.def}|\\
|\childdocof{|\textit{main}|}|\\
\end{tabular}
\end{center}
at the top of every child file \textit{child}
which is included by |\include{|\textit{child}|}|
from within the main file
(or at least for those files to be compiled individually).
The argument \textit{main} must be the filename of the main file.

There are a couple of
considerations in setting up the main and child documents:

%%%%%%%%%%%%%%%%%%%%%%%%%%%%%%%%%%%%%%%%
\paragraph{Restrictions.}

Please note the following restrictions:
\begin{itemize}
\item
|\childdocmain| must be called with one argument \textit{main}
to ensure compatibility with earlier version of the package.
It must either be empty (|\childdocmain{}|)
or precisely match the filename of the main file in which it is specified.
See \secref{sec:detection} for further information.
\item
The filename \textit{main} must be specified without the |.tex| extension.
\item
The filename \textit{main} is case sensitive
(even in case-insensitive file systems)
due to internal string comparison.
\item
The argument \textit{main} should be fully expanded, it cannot be a macro.
\item
Subdirectories and special characters should be avoided in filenames.
\item
The command |\childdocmain{|\textit{main}|}| must be followed by a whitespace.
It should not be followed immediately by another command
or by a comment mark `|%|'.
This is because the \TeX{} parser reads the token immediately following
the argument of |\childdocmain| and puts it
at the beginning of every child section;
however, a white\-space is ignored.
\end{itemize}

%%%%%%%%%%%%%%%%%%%%%%%%%%%%%%%%%%%%%%%%
\paragraph{Content of Main File.}

It is advisable to place all content in the child files included by |\include|.
Any output contained in the main file will appear in all child documents
unless suppressed manually;
it cannot be suppressed automatically by the |\includeonly| directive
and thus should normally be avoided.
A method to include some content in the main file
by means of conditional processing is described in \secref{sec:conditional}.

%%%%%%%%%%%%%%%%%%%%%%%%%%%%%%%%%%%%%%%%
\paragraph{Page Numbering.}

When only a part of the document is compiled,
the appropriate numbering of pages
(as well as other status parameters)
is determined from the |.aux| files.
The latter contain information from previous passes.
However this information needs to propagate through
all intermediate child documents.
Therefore the page numbering in child documents may well
be inconsistent until the complete document is compiled at least once.

A useful (if unconventional) way to always ensure a consistent
page numbering is to restart the numbering in each child document
and denote the pages by `\textit{child}|.|\textit{page}'
where \textit{child} represents the chapter/section number of the child file.
This can be achieved by the command
|\numberwithin{page}{|\textit{child}|}|
of the \textsf{amsmath} package
where \textit{child} can be |chapter| or |section|
depending on the chosen structuring.
Alternatively, one can modify the macro |\thepage| appropriately
and reset the counter |page| at the start of each child file.

%%%%%%%%%%%%%%%%%%%%%%%%%%%%%%%%%%%%%%%%%%%%%%%%%%%%%%%%%%%%%%%%%%%%%%%%%%%%%%%%
\subsection{Conditional Processing}
\label{sec:conditional}

The package provides a mechanism to compile different versions
of a document. To customise the versions further some conditional processing
can come in handy to distinguish which version is being compiled.
The package provides two macros to describe the compilation context:

%%%%%%%%%%%%%%%%%%%%%%%%%%%%%%%%%%%%%%%%
\DescribeMacro{\ifchilddoc}
The conditional |\ifchilddoc| distinguishes between the compilation of
child documents and the main document:
%
\begin{center}
|\ifchilddoc |\textit{child-code}| |[|\||else |\textit{main-code}]| \||fi|
\end{center}

%%%%%%%%%%%%%%%%%%%%%%%%%%%%%%%%%%%%%%%%
\DescribeMacro{\childdocname}
\DescribeMacro{\childdocjob}
The macro |\childdocname| contains the filename (without extension)
of the main or child file being processed.
Note that |\childdocjob| will always contain the name of the main file.

%%%%%%%%%%%%%%%%%%%%%%%%%%%%%%%%%%%%%%%%
\paragraph{Title Page.}

Conditional processing can be used to include a title or banner page
in the main document when proper precautions are taken.
Importantly, the code in the main file should ensure that the page counter
(as well as other status parameters which are stored in the |.aux| files)
takes the same value after the conditional processing.
Otherwise the page numbers may take divergent values
depending on which part is compiled.

For example, a title page could be declared by:
%
\begin{center}
\begin{tabular}{l}
|\ifchilddoc\||else|\\
|\addtocounter{page}{-1}|\\
\textit{code for title page}\\
|\newpage|\\
|\||fi|
\end{tabular}
\end{center}
%
A banner page for the child documents can be generated by:
%
\begin{center}
\begin{tabular}{l}
|\ifchilddoc|\\
|\addtocounter{page}{-1}|\\
\textit{code for banner page}\\
|\newpage|\\
|\||fi|
\end{tabular}
\end{center}
%
Here one could write a message such as:
\begin{center}
|This is the part \childdocname{} of \childdocjob{}.|
\end{center}

%%%%%%%%%%%%%%%%%%%%%%%%%%%%%%%%%%%%%%%%%%%%%%%%%%%%%%%%%%%%%%%%%%%%%%%%%%%%%%%%
\subsection{Flags}
\label{sec:flags}

The package makes it easy to generate different versions
of the main or child documents.
To this end compilation flags can be defined
and assigned different default values.
They will be particularly useful in conjunction
with the forwarding mechanism described in \secref{sec:forward}.

For example, it may be useful to have a flag |\version|
which can be set to |draft| or |final|.
The document source will contain some conditional code
depending on the value of |\version|.
Suppose further, the flag should default to |final| for the main file
and to |draft| for child files
which is a natural assignment for editing the document.
This is achieved by placing the following code
in the preamble of the main document
(below the |\childdocmain| directive):
%
\begin{center}
\begin{tabular}{l}
|\ifchilddoc|\\
|\providecommand{\version}{draft}|\\
|\||else|\\
|\providecommand{\version}{final}|\\
|\||fi|
\end{tabular}
\end{center}
%
The definition by |\providecommand| makes sure
that previous definitions are not overwritten.
Further statements |\providecommand{\version}{...}|
can thus be added before the above code to override it.

For the main file, one might add a line
(between |\childdocmain| and the above block)
%
\begin{center}
|%\ifchilddoc\||else\providecommand{\version}{draft}\||fi|
\end{center}
%
which can be uncommented to produce a draft version.
Likewise one can add a line to the very top of a child file
(above the |\childdocof{|\textit{main}|}| directive)
%
\begin{center}
|%\providecommand{\version}{final}|
\end{center}
%
which can be uncommented to produce the final version of this child document.

%%%%%%%%%%%%%%%%%%%%%%%%%%%%%%%%%%%%%%%%%%%%%%%%%%%%%%%%%%%%%%%%%%%%%%%%%%%%%%%%
\subsection{Forwarding}
\label{sec:forward}

Different versions of the main or child documents
using compilation flags as described in \secref{sec:flags}
can be (permanently) stored in different files
for convenient compilation, viewing and distribution.
To this end, the package defines a command
to pass on compilation to a different file:

%%%%%%%%%%%%%%%%%%%%%%%%%%%%%%%%%%%%%%%%
\DescribeMacro{\childdocforward}
The command |\childdocforward| redirects processing to
another source file:
%
\begin{center}
\begin{tabular}{l}
|\input{childdoc.def}|\\
|\childdocforward[|\textit{main}|]{|\textit{dest}|}|\\
\end{tabular}
\end{center}
%
The argument \textit{dest} is the destination file
(without extension).
It should be the main file or one of the child files.
Note that further \textsf{childdoc} directives
such as |\childdocof| and |\childdocforward|
in the indicated file will be processed in this form.
The optional argument \textit{main}
passes on directly to the main file \textit{main}
while pretending to compile the child \textit{dest}.
This form behaves as if \textit{dest}
issues |\childdocof{|\textit{main}|}| right away,
and no further \textsf{childdoc} directives will be processed.

%%%%%%%%%%%%%%%%%%%%%%%%%%%%%%%%%%%%%%%%
\DescribeMacro{\...prefix}
In the alternative form |\childdocforwardprefix|,
%
\begin{center}
\begin{tabular}{l}
|\input{childdoc.def}|\\
|\childdocforwardprefix[|\textit{main}|]{|\textit{prefix}|}{|\textit{dest}|}|
\end{tabular}
\end{center}
%
the destination file is determined by a pattern
depending on the current file:
To make this work, the current file must be called
`{\textit{prefix}\hspace{0.2em}\textit{suffix}}'
with \textit{prefix} matching precisely the argument.
Processing is then passed on to the file
`{\textit{dest}\hspace{0.2em}\textit{suffix}}'.
Surely, the same effect is achieved by
directly specifying the
argument `{\textit{dest}\hspace{0.2em}\textit{suffix}}'
in the first form.
However, that requires to set up a different file
for each child. With the alternative form of the command
all these files can have exactly the same content
which simplifies setting them up and maintaining them.

For example, the following file |draft.tex|
with a compilation flag |\version| as described in \secref{sec:flags}
compiles the main document as a draft:
%
\begin{center}
\begin{tabular}{l}
|\def\version{draft}|\\
|\input{childdoc.def}|\\
|\childdocforward{|\textit{main}|}|
\end{tabular}
\end{center}
%
Likewise, the following files |final|\textit{nn}|.tex|
compile the final version of the child document
|child|\textit{nn}|.tex|:
%
\begin{center}
\begin{tabular}{l}
|\def\version{final}|\\
|\input{childdoc.def}|\\
|\childdocforwardprefix{final}{child}|
\end{tabular}
\end{center}
%

Note that when several versions of a main file and/or of each child file
are to be generated, it may be convenient to set up a |Makefile| or
shell script to automatise the process.

%%%%%%%%%%%%%%%%%%%%%%%%%%%%%%%%%%%%%%%%%%%%%%%%%%%%%%%%%%%%%%%%%%%%%%%%%%%%%%%%
\subsection{Command Line Processing}
\label{sec:commandline}

The effect of redirection files can also be achieved by invoking
the \LaTeX{} compiler with a more elaborate command line.
Most conveniently this should be done as part
of a shell script or a |Makefile|.

When using \textsf{childdoc} in the main file, the following
command lines effectively perform a redirection
(note that depending on the shell being used,
backslashes may have to be doubled: `|\|' $\to$ `|\\|'):
%
\begin{center}
|... -jobname "|\textit{target}|" |\\|"|[\textit{flags}]%
|\input{childdoc.def}\childdocforward[|\textit{main}|]{|\textit{dest}|}"|
\end{center}
%
Here \textit{target} is the name of the output file,
\textit{main} is the name of the main file
and \textit{dest} is the name of the main or child file to be processed
(all filenames without extensions).
The optional argument \textit{main} can be omitted
if \textit{main} matches \textit{dest}.
Optionally, compilation \textit{flags} can be defined via |\def| commands.
This command line makes the \TeX{} engine believe
it is compiling the file \textit{target}
whose content is specified as the latter parameter.
The provided code then forwards the processing to
\textit{main} or \textit{dest} as described in \secref{sec:forward}.

%%%%%%%%%%%%%%%%%%%%%%%%%%%%%%%%%%%%%%%%%%%%%%%%%%%%%%%%%%%%%%%%%%%%%%%%%%%%%%%%
\subsection{Include by Input}
\label{sec:input}

Including child documents by |\include| has some restrictions by design.
Most notably, the content of a child document always occupies
its own set of pages; pages cannot be shared between child documents.
Usually, this behaviour makes perfect sense
because each child document contain an essential part of the document.
However, in some situations it may be desirable to compose
a document from a collection of parts
without having mandatory page breaks between then.
For this case, the package
provides a mechanism to include parts
by |\input| which can also be processed individually.
However, by construction this mechanism
requires manual handling of the content to be output.

%%%%%%%%%%%%%%%%%%%%%%%%%%%%%%%%%%%%%%%%
\DescribeMacro{\ifchilddocmanual}
The main file should be prepared as usual, see \secref{sec:include}.
However, the document body must make a distinction
between processing of an individual part and of the main document, e.g.:
%
\begin{center}
\begin{tabular}{l}
|\ifchilddocmanual|\\
|\input{\childdocname}|\\
|\||else|\\
\textit{document body with }|\input{|\textit{part}|}|\\
|\||fi|
\end{tabular}
\end{center}
%
The conditional |\ifchilddocmanual| is true whenever
a part to be included by |\input| is being compiled,
and the name of the part is stored in |\childdocname|.

%%%%%%%%%%%%%%%%%%%%%%%%%%%%%%%%%%%%%%%%
\DescribeMacro{\childdocby}
Each part to be included by |\input| should start with:
%
\begin{center}
\begin{tabular}{l}
|\input{childdoc.def}|\\
|\childdocby{|\textit{main}|}|\\
\end{tabular}
\end{center}
%
The directive |\childdocby| is similar to |\childdocof|
described in \secref{sec:include},
but the subsequent selection of content must be done manually.
To that end, both |\ifchilddoc| and |\ifchilddocmanual|
will be true upon processing of a part,
and the name of the part is stored in |\childdocname|.
Note that |\jobname| will be set to the filename of the current part
so that each part receives an individual |.aux| file
that does not interfere with the |.aux| file(s) of the main document.
This behaviour can be altered by the alternative form
|\childdocby[*]{|\textit{main}|}| (with a non-empty optional argument)
which uses the |.aux| file of the main document
by setting |\jobname| to \textit{main}.

%%%%%%%%%%%%%%%%%%%%%%%%%%%%%%%%%%%%%%%%%%%%%%%%%%%%%%%%%%%%%%%%%%%%%%%%%%%%%%%%
\subsection{Driver Development}
\label{sec:driver}

The \textsf{childdoc} mechanism can also be use for the development
of definition files such as \LaTeX{} styles or classes.
This case differs from the above setup with multiple parts
included by |\include| in that no |\includeonly| should be invoked.
This can be achieved by starting the include file
(before |\ProvidesPackage|) with:
%
\begin{center}
\begin{tabular}{l}
|\input{childdoc.def}|\\
|\childdocforward{|\textit{main}|}|\\
\end{tabular}
\end{center}
%
or alternatively with:
%
\begin{center}
\begin{tabular}{l}
|\input{childdoc.def}|\\
|\childdocby{|\textit{main}|}|\\
\end{tabular}
\end{center}
%
Both forms have slightly different effects as described above.
The main file is prepared as usual, see \secref{sec:include}.

%%%%%%%%%%%%%%%%%%%%%%%%%%%%%%%%%%%%%%%%%%%%%%%%%%%%%%%%%%%%%%%%%%%%%%%%%%%%%%%%
\subsection{Legacy Detection}
\label{sec:detection}

The directive |\childdocmain| in the main file can detect
whether the complete document or merely a child is to be compiled
even without using the directive |\childdocof|.
This method is deprecated because it is less robust
and there is no compelling reason to use it;
it is merely provided for backward compatibility
and it may be removed in future versions.

If the detection mechanism is to be used,
it is mandatory to correctly specify
the filename of the main file as the argument of |\childdocmain|:
%
\begin{center}
\begin{tabular}{l}
|\input{childdoc.def}|\\
|\childdocmain{|\textit{main}|}|\\
\end{tabular}
\end{center}
%
If |\jobname| does not match the argument \textit{main} of |\childdocmain|,
it is assumed that |\jobname| points to the child file to be compiled.
When using |\childdocmain| with the main file specified as argument,
it suffices to start a child file
with just |\input{|\textit{main}|}|
without loading of the package and using |\childdocof|.
If instead all processing is done
with the appropriate \textsf{childdoc} directives,
the argument of \textit{main} of |\childdocmain| can be empty.

An alternative version of the command line processing described
in \secref{sec:commandline} using the detection mechanism reads:
%
\begin{center}
|... -jobname "|\textit{target}|" "|[\textit{flags}]%
[|\def\jobname{|\textit{dest}|}|]|\input{|\textit{main}|}"|
\end{center}

%%%%%%%%%%%%%%%%%%%%%%%%%%%%%%%%%%%%%%%%%%%%%%%%%%%%%%%%%%%%%%%%%%%%%%%%%%%%%%%%
\subsection{Manual Code}
\label{sec:manual}

In case one cannot be certain whether the definitions file |childdoc.def|
is installed on the target \TeX{} distribution
and one prefers not to ship it,
it is conceivable to paste a few relevant commands into the sources.

To that end, drop all statements |\input{childdoc.def}|
and perform the replacements as outlined below.
Instead of |\childdocmain{|\textit{main}|}| add the following code
to the top of the main file:
%
\begin{center}
\begin{tabular}{l}
|\||ifdefined\childdocname\endinput\||fi\newif\ifchilddoc|\\
|\edef\childdocname{\scantokens\expandafter{\jobname\noexpand}}|\\
|\def\childdocmain{|\textit{main}|}\||ifx\childdocmain\childdocname\||else|\\
|\childdoctrue\includeonly{\childdocname}\let\jobname\childdocmain\||fi|\\
\end{tabular}
\end{center}
%
Instead of |\childdocof{|\textit{main}|}| just include the main file
at the top of each child file:
%
\begin{center}
|\input{|\textit{main}|}|
\end{center}
%
A simple redirection |\childdocforward{|\textit{dest}|}| is achieved by:
%
\begin{center}
|\def\jobname{|\textit{dest}|}\input{\jobname}|
\end{center}
%
The redirection with prefix
|\childdocforwardprefix[|\textit{prefix}|]{|\textit{dest}|}|
is accomplished by:
%
\begin{center}
\begin{tabular}{l}
|{\edef\jobname{\scantokens\expandafter{\jobname\noexpand}}|\\
|\def\redirectjob |\textit{prefix}|#1~~~{\gdef\jobname{|\textit{dest}|#1}}|\\
|\expandafter\redirectjob\jobname~~~}\input{\jobname}|
\end{tabular}
\end{center}

In an alternative approach,
child documents can be compiled by a specific command line
without additional code or specific definitions:
%
\begin{center}
|... -jobname "|\textit{target}|" "|[\textit{flags}]%
|\includeonly{|\textit{dest}|}\input{|\textit{main}|}"|
\end{center}
%

%%%%%%%%%%%%%%%%%%%%%%%%%%%%%%%%%%%%%%%%%%%%%%%%%%%%%%%%%%%%%%%%%%%%%%%%%%%%%%%%
%%%%%%%%%%%%%%%%%%%%%%%%%%%%%%%%%%%%%%%%%%%%%%%%%%%%%%%%%%%%%%%%%%%%%%%%%%%%%%%%
\section{Information}

%%%%%%%%%%%%%%%%%%%%%%%%%%%%%%%%%%%%%%%%%%%%%%%%%%%%%%%%%%%%%%%%%%%%%%%%%%%%%%%%
\subsection{Copyright}

Copyright \copyright{} 2017--2018 Niklas Beisert

This work may be distributed and/or modified under the
conditions of the \LaTeX{} Project Public License, either version 1.3
of this license or (at your option) any later version.
The latest version of this license is in
  \url{http://www.latex-project.org/lppl.txt}
and version 1.3 or later is part of all distributions of \LaTeX{}
version 2005/12/01 or later.

This work has the LPPL maintenance status `maintained'.

The Current Maintainer of this work is Niklas Beisert.

This work consists of the files |README.txt|, |childdoc.ins| and |childdoc.dtx|
as well as the derived files |childdoc.def|, |cdocsamp.tex|
with |cdocsch1.tex|, |cdocsch2.tex|, |cdocspt3.tex|, |cdocspt4.tex|,
|cdocsdrf.tex|, |cdocsfn1.tex|, |cdocsfn2.tex|
as well as |childdoc.pdf|.

%%%%%%%%%%%%%%%%%%%%%%%%%%%%%%%%%%%%%%%%%%%%%%%%%%%%%%%%%%%%%%%%%%%%%%%%%%%%%%%%
\subsection{Files and Installation}

The package consists of the files:
%
\begin{center}
\begin{tabular}{ll}
    |README.txt|   & readme file \\
    |childdoc.ins| & installation file \\
    |childdoc.dtx| & source file \\
    |childdoc.def| & definition file \\
    |cdocsamp.tex| & sample main file \\
    |cdocsch1.tex| & sample include file \\
    |cdocsch2.tex| & sample include file \\
    |cdocspt3.tex| & sample part file \\
    |cdocspt4.tex| & sample part file \\
    |cdocsdrf.tex| & sample redirection file \\
    |cdocsfn1.tex| & sample redirection file \\
    |cdocsfn2.tex| & sample redirection file \\
    |childdoc.pdf| & manual
\end{tabular}
\end{center}
%
The distribution consists of the files
|README.txt|, |childdoc.ins| and |childdoc.dtx|.
%
\begin{itemize}
\item
Run (pdf)\LaTeX{} on |childdoc.dtx|
to compile the manual |childdoc.pdf| (this file).
\item
Run \LaTeX{} on |childdoc.ins| to create the definitions file |childdoc.def|
and the sample |cdocsamp.tex| with include files
|cdocsch1.tex|, |cdocsch2.tex|, |cdocspt3.tex|, |cdocspt4.tex|,
|cdocsdrf.tex|, |cdocsfn1.tex|, |cdocsfn2.tex|.
Then copy the file |childdoc.def| to an appropriate directory of your \LaTeX{}
distribution, e.g.\ \textit{texmf-root}|/tex/latex/childdoc|.
\end{itemize}

%%%%%%%%%%%%%%%%%%%%%%%%%%%%%%%%%%%%%%%%%%%%%%%%%%%%%%%%%%%%%%%%%%%%%%%%%%%%%%%%
\subsection{Related CTAN Packages}

There are several other packages which offer a similar functionality:
%
\begin{itemize}
\item
The packages
\href{http://ctan.org/pkg/docmute}{\textsf{docmute}},
\href{http://ctan.org/pkg/includex}{\textsf{includex}} and
\href{http://ctan.org/pkg/standalone}{\textsf{standalone}}
provide commands to include only the document body of
a child file thus allowing both files to be compiled individually.
\item
The packages \href{http://ctan.org/pkg/subdocs}{\textsf{subdocs}}
and \href{http://ctan.org/pkg/subfiles}{\textsf{subfiles}}
provide structures in which the main and child documents can be
encapsulated and allowing them to be compiled individually.
The inclusion mechanism is different from the conventional |\include|.
\item
The package \href{http://ctan.org/pkg/combine}{\textsf{combine}}
is an elaborate solution to combine several documents into one.
\end{itemize}
%
See also the CTAN topic \href{http://ctan.org/topic/subdocs}{\textsf{subdocs}}
for further related packages.
The present package differs from the above solutions in that
a document structure constructed with the conventional |\include| mechanism
just needs two extra commands at the top of every file
such that all constituent files can be compiled individually.

%%%%%%%%%%%%%%%%%%%%%%%%%%%%%%%%%%%%%%%%%%%%%%%%%%%%%%%%%%%%%%%%%%%%%%%%%%%%%%%%
%\subsection{Feature Suggestions}
%
%The following is a list of features which may be useful for future
%versions of this package:
%%
%\begin{itemize}
%\item
%\ldots
%\end{itemize}

%%%%%%%%%%%%%%%%%%%%%%%%%%%%%%%%%%%%%%%%%%%%%%%%%%%%%%%%%%%%%%%%%%%%%%%%%%%%%%%%
\subsection{Revision History}

%%%%%%%%%%%%%%%%%%%%%%%%%%%%%%%%%%%%%%%%
\paragraph{v2.0:} 2018/12/30

\begin{itemize}
\item
immediate forward processing
\item
added |\childdocby| mechanism
\item
manual restructured
\end{itemize}

%%%%%%%%%%%%%%%%%%%%%%%%%%%%%%%%%%%%%%%%
\paragraph{v1.6:} 2018/01/17

\begin{itemize}
\item
application for development of include files
\item
corrections to manual
\end{itemize}

%%%%%%%%%%%%%%%%%%%%%%%%%%%%%%%%%%%%%%%%
\paragraph{v1.5:} 2017/05/21

\begin{itemize}
\item
more complete structuring introduced
\item
|\childdocof| introduced
\item
|\childdoc| renamed to |\childdocmain|
\item
|\childredirect| renamed to |\childdocforward| and |\childdocforwardprefix|
and functionality expanded
\end{itemize}

%%%%%%%%%%%%%%%%%%%%%%%%%%%%%%%%%%%%%%%%
\paragraph{v1.0:} 2017/04/27

\begin{itemize}
\item
manual and install package
\item
first version published on CTAN
\end{itemize}

%%%%%%%%%%%%%%%%%%%%%%%%%%%%%%%%%%%%%%%%
\paragraph{v0.6:} 2017/04/26

\begin{itemize}
\item
redirection mechanism added
\end{itemize}

%%%%%%%%%%%%%%%%%%%%%%%%%%%%%%%%%%%%%%%%
\paragraph{v0.5:} 2017/04/26

\begin{itemize}
\item
functionality in definition file
\end{itemize}


%%%%%%%%%%%%%%%%%%%%%%%%%%%%%%%%%%%%%%%%%%%%%%%%%%%%%%%%%%%%%%%%%%%%%%%%%%%%%%%%
%%%%%%%%%%%%%%%%%%%%%%%%%%%%%%%%%%%%%%%%%%%%%%%%%%%%%%%%%%%%%%%%%%%%%%%%%%%%%%%%
%%%%%%%%%%%%%%%%%%%%%%%%%%%%%%%%%%%%%%%%%%%%%%%%%%%%%%%%%%%%%%%%%%%%%%%%%%%%%%%%
\appendix

\settowidth\MacroIndent{\rmfamily\scriptsize 000\ }

 \DocInput{childdoc.dtx}

\end{document}
%</driver>
% \fi
%
% %%%%%%%%%%%%%%%%%%%%%%%%%%%%%%%%%%%%%%%%%%%%%%%%%%%%%%%%%%%%%%%%%%%%%%%%%%%%%%
% %%%%%%%%%%%%%%%%%%%%%%%%%%%%%%%%%%%%%%%%%%%%%%%%%%%%%%%%%%%%%%%%%%%%%%%%%%%%%%
% \section{Sample}
%\iffalse
%<*samplemain>
%\fi
%
% The following presents a sample document
% with two chapters, two parts, a title page,
% a compile flag as well as three forwarding files to set the flag.
% It consists of eight |.tex| files:
% \begin{center}
% \begin{tabular}{ll}
% |cdocsamp.tex|&main file\\
% |cdocsch1.tex|&include file for chapter 1\\
% |cdocsch2.tex|&include file for chapter 2\\
% |cdocspt3.tex|&include file for part 3\\
% |cdocspt4.tex|&include file for part 4\\
% |cdocsdrf.tex|&forwarding file for main file in draft mode\\
% |cdocsfi1.tex|&forwarding file for final version of chapter 1\\
% |cdocsfi2.tex|&forwarding file for final version of chapter 2\\
% \end{tabular}
% \end{center}
% Each of the eight files can be compiled directly by the \LaTeX{} compiler.
%
% %%%%%%%%%%%%%%%%%%%%%%%%%%%%%%%%%%%%%%
% \paragraph{Main File.}
%
% The main file is called |cdocsamp.tex|.
%
% Load the \textsf{childdoc} definitions and
% declare the filename for the main document:
%    \begin{macrocode}
\input{childdoc.def}
\childdocmain{}
%    \end{macrocode}

% Optional override for |\version| flag:
%    \begin{macrocode}
%%\ifchilddoc\else\providecommand{\version}{draft}\fi
%    \end{macrocode}

% Define the default values for the |\version| flag
% (|final| for the main file and |draft| for childs):
%    \begin{macrocode}
\ifchilddoc
\providecommand{\version}{draft}
\else
\providecommand{\version}{final}
\fi
%    \end{macrocode}

% Load the standard document class:
%    \begin{macrocode}
\documentclass[12pt]{article}
%    \end{macrocode}

% Start the document body:
%    \begin{macrocode}
\begin{document}
%    \end{macrocode}

% Declare a title page.
% Print title, part of document being processed and version flag:
%    \begin{macrocode}
\addtocounter{page}{-1}
\begin{center}
{\LARGE\bfseries{}childdoc example\par}
\vspace{1cm}
\ifchilddoc
\ifchilddocmanual part\else chapter\fi:
`\childdocname' of `\childdocjob'\par
\else
main document: `\childdocjob'\par
\fi
version: \version\par
\end{center}
\newpage
%    \end{macrocode}

% Manually include selected file,
% otherwise process as usual:
%    \begin{macrocode}
\ifchilddocmanual
\section*{part `\childdocname'}
\input{\childdocname}
\else
%    \end{macrocode}

% Include the two chapters:
%    \begin{macrocode}
\include{cdocsch1}
\include{cdocsch2}
%    \end{macrocode}

% Include the two parts unless only chapters should be displayed:
%    \begin{macrocode}
\ifchilddoc\else
\section{part three}
\input{cdocspt3}
\section{part four}
\input{cdocspt4}
\fi
%    \end{macrocode}

% Process as usual until here:
%    \begin{macrocode}
\fi
%    \end{macrocode}

% End of document body:
%    \begin{macrocode}
\end{document}
%    \end{macrocode}
%\iffalse
%</samplemain>
%\fi
%
% %%%%%%%%%%%%%%%%%%%%%%%%%%%%%%%%%%%%%%
% \paragraph{Chapter Include Files.}
%
% The include files are called |cdocsch1.tex| and |cdocsch2.tex|.
%
%\iffalse
%<*samplechap1|samplechap2>
%\fi

% Optional override for |\version| flag:
%    \begin{macrocode}
%%\providecommand{\version}{final}
%    \end{macrocode}

% Include the main document:
%    \begin{macrocode}
\input{childdoc.def}
\childdocof{cdocsamp}
%    \end{macrocode}

%\iffalse
%</samplechap1|samplechap2>
%\fi
%
%\iffalse
%<*samplechap1>
%\fi
% Some text for chapter 1:
%    \begin{macrocode}
\section{one}
some text in chapter one
%    \end{macrocode}

%\iffalse
%</samplechap1>
%\fi
% Some text for chapter 2:
%\iffalse
%<*samplechap2>
%\fi
%    \begin{macrocode}
\section{two}
more text in chapter two
%    \end{macrocode}

%\iffalse
%</samplechap2>
%\fi
%
% %%%%%%%%%%%%%%%%%%%%%%%%%%%%%%%%%%%%%%
% \paragraph{Part Include Files.}
%
% The include files are called |cdocspt3.tex| and |cdocspt4.tex|.
%
%\iffalse
%<*samplepart3|samplepart4>
%\fi

% Optional override for |\version| flag:
%    \begin{macrocode}
%%\providecommand{\version}{final}
%    \end{macrocode}

% Include the main document:
%    \begin{macrocode}
\input{childdoc.def}
\childdocby{cdocsamp}
%    \end{macrocode}

%\iffalse
%</samplepart3|samplepart4>
%\fi
%
%\iffalse
%<*samplepart3>
%\fi
% Some text for part 3:
%    \begin{macrocode}
some text in part three
%    \end{macrocode}

%\iffalse
%</samplepart3>
%\fi
% Some text for part 4:
%\iffalse
%<*samplepart4>
%\fi
%    \begin{macrocode}
more text in part four
%    \end{macrocode}

%\iffalse
%</samplepart4>
%\fi
%
% %%%%%%%%%%%%%%%%%%%%%%%%%%%%%%%%%%%%%%
% \paragraph{Forwarding for a Complete Draft.}
%
% The following forwarding file |cdocsdrf.tex|
% compiles the main document in draft mode:
%\iffalse
%<*sampledraft>
%\fi
%    \begin{macrocode}
\def\version{draft}
\input{childdoc.def}
\childdocforward{cdocsamp}
%    \end{macrocode}

%\iffalse
%</sampledraft>
%\fi
%
% %%%%%%%%%%%%%%%%%%%%%%%%%%%%%%%%%%%%%%
% \paragraph{Forwarding for Final Version of the Chapters.}
%
% The following forwarding files |cdocsfn1.tex| and |cdocsfn2.tex|
% (with identical content)
% compile the final versions of the child documents
% |cdocsch1.tex| and |cdocsch2.tex|, respectively:
%\iffalse
%<*samplefinal>
%\fi
%    \begin{macrocode}
\def\version{final}
\input{childdoc.def}
\childdocforwardprefix[cdocsamp]{cdocsfn}{cdocsch}
%    \end{macrocode}

%\iffalse
%</samplefinal>
%\fi
%
% %%%%%%%%%%%%%%%%%%%%%%%%%%%%%%%%%%%%%%
% \paragraph{Command Line Processing.}
%
% The following three command lines generate the output files
% |cdocscld|, |cdocscl1| and |cdocscl2|
% which should be identical to
% |cdocsdrf|, |cdocsch1| and |cdocsfn2|, respectively:
% \begin{center}
% \begin{tabular}{l}
% |latex -jobname cdocscld \|\\
% |  "\def\version{draft}\input{childdoc.def}\childdocforward{cdocsamp}"|\\
% |latex -jobname cdocscl1 \|\\
% |  "\input{childdoc.def}\childdocforward[cdocsamp]{cdocsch1}"|\\
% |latex -jobname cdocscl2 \|\\
% |  "\def\version{final}\input{childdoc.def}\childdocforward{cdocsch2}"|
% \end{tabular}
% \end{center}
% Note that the trailing backslash on each first line
% merely continues the input to the second line
% (for convenient cut ant paste).
% Furthermore, the command |latex| can be replaced by any
% of its alternative versions such as |pdflatex|.
%
% %%%%%%%%%%%%%%%%%%%%%%%%%%%%%%%%%%%%%%%%%%%%%%%%%%%%%%%%%%%%%%%%%%%%%%%%%%%%%%
% %%%%%%%%%%%%%%%%%%%%%%%%%%%%%%%%%%%%%%%%%%%%%%%%%%%%%%%%%%%%%%%%%%%%%%%%%%%%%%
% \section{Implementation}
%\iffalse
%<*package>
%\fi
%
% This section describes the definitions file |childdoc.def|.

% The definitions cannot be loaded using |\usepackage| or |\RequirePackage|
% which has a mechanism to prevent loading a style file more than once.
% When loading the definitions by means of |\input|
% multiple instances have to be prevented manually:
%\iffalse
%This code needs to be before the `\ProvidesFile' directive
%which is defined at the beginning of this file.
%Therefore it is also placed there and commented out here.
%</package>
%<*discard>
%\fi
%    \begin{macrocode}
\ifdefined\childdocmain\endinput\fi
%    \end{macrocode}
%\iffalse
%</discard>
%<*package>
%\fi
%
% \macro{\ifchilddoc}
% \macro{\ifchilddocmanual}
% The conditional |\ifchilddoc| tells whether a
% child (true) or main (false) document is being compiled.
% The conditional |\ifchilddocmanual| tells whether
% the |\includeonly| mechanism is used (false) or
% the selection of child files must be performed manually (true).
% The definitions initialise to false:
%    \begin{macrocode}
\newif\ifchilddoc
\newif\ifchilddocmanual
%    \end{macrocode}

% \macro{\childdocname}
% \macro{\childdocjob}
% The macro |\childdocname| stores the name of the main document
% to be compiled. The macro |\childdocjob| stores the name of
% the document on which the \LaTeX{} compiler was originally invoked.
% The content of |\jobname| cannot be compared
% to filenames specified in the source due to different catcodes.
% The following code rescans |\jobname|, stores the result
% in |\childdocname| and saves a copy in |\childdocjob|:
%    \begin{macrocode}
\edef\childdocname{\scantokens\expandafter{\jobname\noexpand}}
\let\childdocjob\childdocname
%    \end{macrocode}

% \macro{\childdocdisable}
% The macro |\childdocdisable| prevents the main file
% from being processed more than once.
% At this stage, the main document command |\childdocmain|
% is assumed to be called once again where it should do nothing.
% Any subsequent call to it should prevent
% a secondary processing of the main document
% It overwrites the forwarding commands
% |\childdocof| and |\childdocforward|
% with empty macros to prevent further inclusions of the main document:
%    \begin{macrocode}
\newcommand{\childdocdisable}
{
  \renewcommand{\childdocmain}[1]{\renewcommand{\childdocmain}[1]{\endinput}}
  \renewcommand{\childdocof}[1]{}
  \renewcommand{\childdocby}[2][]{}
  \renewcommand{\childdocforward}[2][]{}
  \renewcommand{\childdocdisable}{}
}
%    \end{macrocode}

% \macro{\childdocmain}
% The macro |\childdocmain| is to be called at the top of the main file
% with nothing or the main filename (without extension) as argument.
% First, it breaks loops.
% If the argument is not empty and does not match |\childdocname|
% (which is set by the first inclusion of |childdoc.def|),
% |\ifchilddoc| is set to true, |\includeonly| is applied to the child file
% and |\jobname| is set to the main file
% (for proper handling of |.aux| files):
%    \begin{macrocode}
\newcommand{\childdocmain}[1]
{
  \childdocdisable\childdocmain{}
  \if?#1?\else
    \begingroup
      \def\childdoctmp{#1}
      \ifx\childdoctmp\childdocname
        \def\childdoctmp{}
      \else
        \def\childdoctmp
        {
          \childdoctrue
          \includeonly{\childdocname}
          \def\childdocjob{#1}
          \def\jobname{#1}
        }
      \fi
      \expandafter
    \endgroup
    \childdoctmp
  \fi
}
%    \end{macrocode}

% \macro{\childdocof}
% The command |\childdocof| redirects
% compilation to the main file |#1|.
%    \begin{macrocode}
\newcommand{\childdocof}[1]
{
  \childdocdisable
  \childdoctrue
  \includeonly{\childdocname}
  \def\jobname{#1}
  \def\childdocjob{#1}
  \input{#1}
}
%    \end{macrocode}

% \macro{\childdocby}
% The command |\childdocby| ....
%    \begin{macrocode}
\newcommand{\childdocby}[2][]
{
  \childdocdisable
  \childdoctrue
  \childdocmanualtrue
  \if?#1?\else
    \def\jobname{#2}
  \fi
  \def\childdocjob{#2}
  \input{#2}
  \endinput
}
%    \end{macrocode}

% \macro{\childdocforward}
% The command |\childdocforward| redirects
% compilation to the main file or
% (if the optional argument is given) a child file.
% Parameters are set as if the main file
% or a child file starting with |\childdocof| was compiled.
% Then compilation is handed over to the main file:
%    \begin{macrocode}
\newcommand{\childdocforward}[2][]
{
  \begingroup
    \if?#1?
      \def\childdoctmp
      {
        \def\childdocname{#2}
        \def\childdocjob{#2}
        \def\jobname{#2}
        \input{#2}
        \endinput
      }
    \else
      \def\childdoctmp
      {
        \childdocdisable
        \def\childdocname{#2}
        \childdoctrue
        \includeonly{#2}
        \def\childdocjob{#1}
        \def\jobname{#1}
        \input{#1}
        \endinput
      }
    \fi
    \expandafter
  \endgroup
  \childdoctmp
}
%    \end{macrocode}

% \macro{\childdocforwardprefix}
% The command |\childdocforwardprefix| redirects
% compilation to the main or a child file by means of a pattern.
% The prefix |#1| in the current filename is replaced by |#2|
% and the suffix of the current filename is kept
% (it is assumed that the filename does not contain the substring `|~~~|'
% which is used as a delimiter).
% Compilation is handed over to the new file by |\childdocforward|:
%    \begin{macrocode}
\newcommand{\childdocforwardprefix}[3][]
{
  \begingroup
    \def\childdocextract #2##1~~~{\def\childdoctmp{\childdocforward[#1]{#3##1}}}
    \expandafter\childdocextract\childdocname~~~
    \expandafter
  \endgroup
  \childdoctmp
}
%    \end{macrocode}

% \macro{\childdoc}
% The deprecated macro |\childdoc| is a legacy version of |\childdocmain|:
%    \begin{macrocode}
\newcommand{\childdoc}{\childdocmain}
%    \end{macrocode}

% \macro{\childdocredirect}
% The deprecated macro |\childdocredirect| is a legacy version
% of |\childdocforward| and |\childdocforwardprefix|:
%    \begin{macrocode}
\newcommand{\childdocredirect}[2][]
{
  \begingroup
    \if?#1?
      \def\childdoctmp{\childdocforward{#2}}
    \else
      \def\childdoctmp{\childdocforwardprefix{#1}{#2}}
    \fi
    \expandafter
  \endgroup
  \childdoctmp
}
%    \end{macrocode}

%\iffalse
%</package>
%\fi
%
\endinput
\childdocforward{cdocsch2}"|
% \end{tabular}
% \end{center}
% Note that the trailing backslash on each first line
% merely continues the input to the second line
% (for convenient cut ant paste).
% Furthermore, the command |latex| can be replaced by any
% of its alternative versions such as |pdflatex|.
%
% %%%%%%%%%%%%%%%%%%%%%%%%%%%%%%%%%%%%%%%%%%%%%%%%%%%%%%%%%%%%%%%%%%%%%%%%%%%%%%
% %%%%%%%%%%%%%%%%%%%%%%%%%%%%%%%%%%%%%%%%%%%%%%%%%%%%%%%%%%%%%%%%%%%%%%%%%%%%%%
% \section{Implementation}
%\iffalse
%<*package>
%\fi
%
% This section describes the definitions file |childdoc.def|.

% The definitions cannot be loaded using |\usepackage| or |\RequirePackage|
% which has a mechanism to prevent loading a style file more than once.
% When loading the definitions by means of |\input|
% multiple instances have to be prevented manually:
%\iffalse
%This code needs to be before the `\ProvidesFile' directive
%which is defined at the beginning of this file.
%Therefore it is also placed there and commented out here.
%</package>
%<*discard>
%\fi
%    \begin{macrocode}
\ifdefined\childdocmain\endinput\fi
%    \end{macrocode}
%\iffalse
%</discard>
%<*package>
%\fi
%
% \macro{\ifchilddoc}
% \macro{\ifchilddocmanual}
% The conditional |\ifchilddoc| tells whether a
% child (true) or main (false) document is being compiled.
% The conditional |\ifchilddocmanual| tells whether
% the |\includeonly| mechanism is used (false) or
% the selection of child files must be performed manually (true).
% The definitions initialise to false:
%    \begin{macrocode}
\newif\ifchilddoc
\newif\ifchilddocmanual
%    \end{macrocode}

% \macro{\childdocname}
% \macro{\childdocjob}
% The macro |\childdocname| stores the name of the main document
% to be compiled. The macro |\childdocjob| stores the name of
% the document on which the \LaTeX{} compiler was originally invoked.
% The content of |\jobname| cannot be compared
% to filenames specified in the source due to different catcodes.
% The following code rescans |\jobname|, stores the result
% in |\childdocname| and saves a copy in |\childdocjob|:
%    \begin{macrocode}
\edef\childdocname{\scantokens\expandafter{\jobname\noexpand}}
\let\childdocjob\childdocname
%    \end{macrocode}

% \macro{\childdocdisable}
% The macro |\childdocdisable| prevents the main file
% from being processed more than once.
% At this stage, the main document command |\childdocmain|
% is assumed to be called once again where it should do nothing.
% Any subsequent call to it should prevent
% a secondary processing of the main document
% It overwrites the forwarding commands
% |\childdocof| and |\childdocforward|
% with empty macros to prevent further inclusions of the main document:
%    \begin{macrocode}
\newcommand{\childdocdisable}
{
  \renewcommand{\childdocmain}[1]{\renewcommand{\childdocmain}[1]{\endinput}}
  \renewcommand{\childdocof}[1]{}
  \renewcommand{\childdocby}[2][]{}
  \renewcommand{\childdocforward}[2][]{}
  \renewcommand{\childdocdisable}{}
}
%    \end{macrocode}

% \macro{\childdocmain}
% The macro |\childdocmain| is to be called at the top of the main file
% with nothing or the main filename (without extension) as argument.
% First, it breaks loops.
% If the argument is not empty and does not match |\childdocname|
% (which is set by the first inclusion of |childdoc.def|),
% |\ifchilddoc| is set to true, |\includeonly| is applied to the child file
% and |\jobname| is set to the main file
% (for proper handling of |.aux| files):
%    \begin{macrocode}
\newcommand{\childdocmain}[1]
{
  \childdocdisable\childdocmain{}
  \if?#1?\else
    \begingroup
      \def\childdoctmp{#1}
      \ifx\childdoctmp\childdocname
        \def\childdoctmp{}
      \else
        \def\childdoctmp
        {
          \childdoctrue
          \includeonly{\childdocname}
          \def\childdocjob{#1}
          \def\jobname{#1}
        }
      \fi
      \expandafter
    \endgroup
    \childdoctmp
  \fi
}
%    \end{macrocode}

% \macro{\childdocof}
% The command |\childdocof| redirects
% compilation to the main file |#1|.
%    \begin{macrocode}
\newcommand{\childdocof}[1]
{
  \childdocdisable
  \childdoctrue
  \includeonly{\childdocname}
  \def\jobname{#1}
  \def\childdocjob{#1}
  \input{#1}
}
%    \end{macrocode}

% \macro{\childdocby}
% The command |\childdocby| ....
%    \begin{macrocode}
\newcommand{\childdocby}[2][]
{
  \childdocdisable
  \childdoctrue
  \childdocmanualtrue
  \if?#1?\else
    \def\jobname{#2}
  \fi
  \def\childdocjob{#2}
  \input{#2}
  \endinput
}
%    \end{macrocode}

% \macro{\childdocforward}
% The command |\childdocforward| redirects
% compilation to the main file or
% (if the optional argument is given) a child file.
% Parameters are set as if the main file
% or a child file starting with |\childdocof| was compiled.
% Then compilation is handed over to the main file:
%    \begin{macrocode}
\newcommand{\childdocforward}[2][]
{
  \begingroup
    \if?#1?
      \def\childdoctmp
      {
        \def\childdocname{#2}
        \def\childdocjob{#2}
        \def\jobname{#2}
        \input{#2}
        \endinput
      }
    \else
      \def\childdoctmp
      {
        \childdocdisable
        \def\childdocname{#2}
        \childdoctrue
        \includeonly{#2}
        \def\childdocjob{#1}
        \def\jobname{#1}
        \input{#1}
        \endinput
      }
    \fi
    \expandafter
  \endgroup
  \childdoctmp
}
%    \end{macrocode}

% \macro{\childdocforwardprefix}
% The command |\childdocforwardprefix| redirects
% compilation to the main or a child file by means of a pattern.
% The prefix |#1| in the current filename is replaced by |#2|
% and the suffix of the current filename is kept
% (it is assumed that the filename does not contain the substring `|~~~|'
% which is used as a delimiter).
% Compilation is handed over to the new file by |\childdocforward|:
%    \begin{macrocode}
\newcommand{\childdocforwardprefix}[3][]
{
  \begingroup
    \def\childdocextract #2##1~~~{\def\childdoctmp{\childdocforward[#1]{#3##1}}}
    \expandafter\childdocextract\childdocname~~~
    \expandafter
  \endgroup
  \childdoctmp
}
%    \end{macrocode}

% \macro{\childdoc}
% The deprecated macro |\childdoc| is a legacy version of |\childdocmain|:
%    \begin{macrocode}
\newcommand{\childdoc}{\childdocmain}
%    \end{macrocode}

% \macro{\childdocredirect}
% The deprecated macro |\childdocredirect| is a legacy version
% of |\childdocforward| and |\childdocforwardprefix|:
%    \begin{macrocode}
\newcommand{\childdocredirect}[2][]
{
  \begingroup
    \if?#1?
      \def\childdoctmp{\childdocforward{#2}}
    \else
      \def\childdoctmp{\childdocforwardprefix{#1}{#2}}
    \fi
    \expandafter
  \endgroup
  \childdoctmp
}
%    \end{macrocode}

%\iffalse
%</package>
%\fi
%
\endinput

\childdocmain{}
%    \end{macrocode}

% Optional override for |\version| flag:
%    \begin{macrocode}
%%\ifchilddoc\else\providecommand{\version}{draft}\fi
%    \end{macrocode}

% Define the default values for the |\version| flag
% (|final| for the main file and |draft| for childs):
%    \begin{macrocode}
\ifchilddoc
\providecommand{\version}{draft}
\else
\providecommand{\version}{final}
\fi
%    \end{macrocode}

% Load the standard document class:
%    \begin{macrocode}
\documentclass[12pt]{article}
%    \end{macrocode}

% Start the document body:
%    \begin{macrocode}
\begin{document}
%    \end{macrocode}

% Declare a title page.
% Print title, part of document being processed and version flag:
%    \begin{macrocode}
\addtocounter{page}{-1}
\begin{center}
{\LARGE\bfseries{}childdoc example\par}
\vspace{1cm}
\ifchilddoc
\ifchilddocmanual part\else chapter\fi:
`\childdocname' of `\childdocjob'\par
\else
main document: `\childdocjob'\par
\fi
version: \version\par
\end{center}
\newpage
%    \end{macrocode}

% Manually include selected file,
% otherwise process as usual:
%    \begin{macrocode}
\ifchilddocmanual
\section*{part `\childdocname'}
\input{\childdocname}
\else
%    \end{macrocode}

% Include the two chapters:
%    \begin{macrocode}
\include{cdocsch1}
\include{cdocsch2}
%    \end{macrocode}

% Include the two parts unless only chapters should be displayed:
%    \begin{macrocode}
\ifchilddoc\else
\section{part three}
\input{cdocspt3}
\section{part four}
\input{cdocspt4}
\fi
%    \end{macrocode}

% Process as usual until here:
%    \begin{macrocode}
\fi
%    \end{macrocode}

% End of document body:
%    \begin{macrocode}
\end{document}
%    \end{macrocode}
%\iffalse
%</samplemain>
%\fi
%
% %%%%%%%%%%%%%%%%%%%%%%%%%%%%%%%%%%%%%%
% \paragraph{Chapter Include Files.}
%
% The include files are called |cdocsch1.tex| and |cdocsch2.tex|.
%
%\iffalse
%<*samplechap1|samplechap2>
%\fi

% Optional override for |\version| flag:
%    \begin{macrocode}
%%\providecommand{\version}{final}
%    \end{macrocode}

% Include the main document:
%    \begin{macrocode}
% \iffalse
%
% childdoc.dtx Copyright (C) 2017-2018 Niklas Beisert
%
% This work may be distributed and/or modified under the
% conditions of the LaTeX Project Public License, either version 1.3
% of this license or (at your option) any later version.
% The latest version of this license is in
%   http://www.latex-project.org/lppl.txt
% and version 1.3 or later is part of all distributions of LaTeX
% version 2005/12/01 or later.
%
% This work has the LPPL maintenance status `maintained'.
%
% The Current Maintainer of this work is Niklas Beisert.
%
% This work consists of the files childdoc.dtx and childdoc.ins
% and the derived files childdoc.def and cdocsamp.tex with
% cdocsch1.tex, cdocsch2.tex, cdocsdrf.tex, cdocsfn1.tex, cdocsfn2.tex.
%
%<package>\ifdefined\childdocmain\endinput\fi
%<package>\ProvidesFile{childdoc.def}[2018/12/30 v2.0 child document driver]
%<samplemain>\ProvidesFile{cdocsamp.tex}[2018/12/30 v2.0 sample for childdoc]
%<*driver>
%\ProvidesFile{childdoc.drv}[2018/12/30 v2.0 childdoc reference manual file]
\PassOptionsToClass{10pt,a4paper}{article}
\documentclass{ltxdoc}

\usepackage[margin=35mm]{geometry}
\usepackage{hyperref}
\usepackage{hyperxmp}
\usepackage[usenames]{color}

\hypersetup{colorlinks=true}
\hypersetup{pdfstartview=FitH}
\hypersetup{pdfpagemode=UseNone}
\hypersetup{pdfsource={}}
\hypersetup{pdflang={en-UK}}
\hypersetup{pdfcopyright={Copyright 2017-2018 Niklas Beisert.
  This work may be distributed and/or modified under the
  conditions of the LaTeX Project Public License, either version 1.3
  of this license or (at your option) any later version.}}
\hypersetup{pdflicenseurl={http://www.latex-project.org/lppl.txt}}
\hypersetup{pdfcontactaddress={ETH Zurich, ITP, HIT K,
  Wolfgang-Pauli-Strasse 27}}
\hypersetup{pdfcontactpostcode={8093}}
\hypersetup{pdfcontactcity={Zurich}}
\hypersetup{pdfcontactcountry={Switzerland}}
\hypersetup{pdfcontactemail={nbeisert@itp.phys.ethz.ch}}
\hypersetup{pdfcontacturl={http://people.phys.ethz.ch/\xmptilde nbeisert/}}

\newcommand{\secref}[1]{\hyperref[#1]{section \ref*{#1}}}

\parskip1ex
\parindent0pt
\let\olditemize\itemize
\def\itemize{\olditemize\parskip0pt}

\begin{document}

\title{The \textsf{childdoc} Package}
\hypersetup{pdftitle={The childdoc Package}}
\author{Niklas Beisert\\[2ex]
  Institut f\"ur Theoretische Physik\\
  Eidgen\"ossische Technische Hochschule Z\"urich\\
  Wolfgang-Pauli-Strasse 27, 8093 Z\"urich, Switzerland\\[1ex]
  \href{mailto:nbeisert@itp.phys.ethz.ch}
  {\texttt{nbeisert@itp.phys.ethz.ch}}}
\hypersetup{pdfauthor={Niklas Beisert}}
\hypersetup{pdfsubject={Manual for the LaTeX2e Package childdoc}}
\date{30 December 2018, \textsf{v2.0}}
\maketitle

\begin{abstract}\noindent
\textsf{childdoc} is a \LaTeXe{} package
that enables the direct compilation
of document sections included by |\include|
to individual files.
\end{abstract}

\begingroup
\parskip0ex
\tableofcontents
\endgroup

%%%%%%%%%%%%%%%%%%%%%%%%%%%%%%%%%%%%%%%%%%%%%%%%%%%%%%%%%%%%%%%%%%%%%%%%%%%%%%%%
%%%%%%%%%%%%%%%%%%%%%%%%%%%%%%%%%%%%%%%%%%%%%%%%%%%%%%%%%%%%%%%%%%%%%%%%%%%%%%%%
\section{Introduction}

\LaTeX{} provides a mechanism to structure a large document (such as a book)
into a main file and several child files (containing the chapters)
using the |\include| command.
This mechanism is beneficial for documents
which span hundreds of pages in order to
make the source file(s) more manageable.
Moreover, compilation can be restricted to
selected child files by means of the |\includeonly| command.
The latter feature can be used to reduce the compilation time while editing
(this was significantly more useful in the earlier days of \LaTeX{})
or to generate a smaller document which is easier to navigate.
Another application of |\includeonly| is to generate
documents consisting of selected parts of the complete document.

However, there are a few drawbacks of the plain |\include| mechanism:
\begin{itemize}
\item
The child files cannot be compiled on their own,
they can only be compiled via the main file.
A naive editing environment
(such as a text editor with an option
to have the current file processed by \LaTeX)
may require one to switch to the main file before compiling;
attempting to compile the child file produces errors.
\item
The main file must be modified (each time)
to adjust the |\includeonly| command
to the present needs. This easily leaves the main file in a messy state.
\item
The generated document will always carry the filename
of the main document. This is inconvenient if
several child files are to be compiled and
to be kept for distribution.
\end{itemize}

The present package provides a simple interface
to make child files individually compilable by \LaTeX{}.
Compiling a child file then has the same effect as compiling
the main file with an |\includeonly| command
to select the appropriate child.
Moreover the generated document will carry the name of the child
rather than the main file.
This resolves all three above issues.

This feature is meant to make the editing of books,
thesis documents and lecture notes somewhat more convenient.
However, the package can also be used efficiently for
composing a series of documents (such as exercise sheets)
which are typically distributed individually.
It then assists the author in generating the individual documents
(potentially in different versions)
as well as a document containing the collected series.
Another application is in developing style files
or other kinds of included material
where compilation of the style file could redirect
to a sample or test file.

%%%%%%%%%%%%%%%%%%%%%%%%%%%%%%%%%%%%%%%%%%%%%%%%%%%%%%%%%%%%%%%%%%%%%%%%%%%%%%%%
%%%%%%%%%%%%%%%%%%%%%%%%%%%%%%%%%%%%%%%%%%%%%%%%%%%%%%%%%%%%%%%%%%%%%%%%%%%%%%%%
\section{Usage}

First of all, the package \textsf{childdoc} is \emph{not} a standard
\LaTeXe{} |.sty| style file! Therefore it needs to be invoked in
a non-standard way.

%%%%%%%%%%%%%%%%%%%%%%%%%%%%%%%%%%%%%%%%%%%%%%%%%%%%%%%%%%%%%%%%%%%%%%%%%%%%%%%%
\subsection{Included Files}
\label{sec:include}

%%%%%%%%%%%%%%%%%%%%%%%%%%%%%%%%%%%%%%%%
\DescribeMacro{\childdocmain}
To use the package, add the commands
\begin{center}
\begin{tabular}{l}
|% \iffalse
%
% childdoc.dtx Copyright (C) 2017-2018 Niklas Beisert
%
% This work may be distributed and/or modified under the
% conditions of the LaTeX Project Public License, either version 1.3
% of this license or (at your option) any later version.
% The latest version of this license is in
%   http://www.latex-project.org/lppl.txt
% and version 1.3 or later is part of all distributions of LaTeX
% version 2005/12/01 or later.
%
% This work has the LPPL maintenance status `maintained'.
%
% The Current Maintainer of this work is Niklas Beisert.
%
% This work consists of the files childdoc.dtx and childdoc.ins
% and the derived files childdoc.def and cdocsamp.tex with
% cdocsch1.tex, cdocsch2.tex, cdocsdrf.tex, cdocsfn1.tex, cdocsfn2.tex.
%
%<package>\ifdefined\childdocmain\endinput\fi
%<package>\ProvidesFile{childdoc.def}[2018/12/30 v2.0 child document driver]
%<samplemain>\ProvidesFile{cdocsamp.tex}[2018/12/30 v2.0 sample for childdoc]
%<*driver>
%\ProvidesFile{childdoc.drv}[2018/12/30 v2.0 childdoc reference manual file]
\PassOptionsToClass{10pt,a4paper}{article}
\documentclass{ltxdoc}

\usepackage[margin=35mm]{geometry}
\usepackage{hyperref}
\usepackage{hyperxmp}
\usepackage[usenames]{color}

\hypersetup{colorlinks=true}
\hypersetup{pdfstartview=FitH}
\hypersetup{pdfpagemode=UseNone}
\hypersetup{pdfsource={}}
\hypersetup{pdflang={en-UK}}
\hypersetup{pdfcopyright={Copyright 2017-2018 Niklas Beisert.
  This work may be distributed and/or modified under the
  conditions of the LaTeX Project Public License, either version 1.3
  of this license or (at your option) any later version.}}
\hypersetup{pdflicenseurl={http://www.latex-project.org/lppl.txt}}
\hypersetup{pdfcontactaddress={ETH Zurich, ITP, HIT K,
  Wolfgang-Pauli-Strasse 27}}
\hypersetup{pdfcontactpostcode={8093}}
\hypersetup{pdfcontactcity={Zurich}}
\hypersetup{pdfcontactcountry={Switzerland}}
\hypersetup{pdfcontactemail={nbeisert@itp.phys.ethz.ch}}
\hypersetup{pdfcontacturl={http://people.phys.ethz.ch/\xmptilde nbeisert/}}

\newcommand{\secref}[1]{\hyperref[#1]{section \ref*{#1}}}

\parskip1ex
\parindent0pt
\let\olditemize\itemize
\def\itemize{\olditemize\parskip0pt}

\begin{document}

\title{The \textsf{childdoc} Package}
\hypersetup{pdftitle={The childdoc Package}}
\author{Niklas Beisert\\[2ex]
  Institut f\"ur Theoretische Physik\\
  Eidgen\"ossische Technische Hochschule Z\"urich\\
  Wolfgang-Pauli-Strasse 27, 8093 Z\"urich, Switzerland\\[1ex]
  \href{mailto:nbeisert@itp.phys.ethz.ch}
  {\texttt{nbeisert@itp.phys.ethz.ch}}}
\hypersetup{pdfauthor={Niklas Beisert}}
\hypersetup{pdfsubject={Manual for the LaTeX2e Package childdoc}}
\date{30 December 2018, \textsf{v2.0}}
\maketitle

\begin{abstract}\noindent
\textsf{childdoc} is a \LaTeXe{} package
that enables the direct compilation
of document sections included by |\include|
to individual files.
\end{abstract}

\begingroup
\parskip0ex
\tableofcontents
\endgroup

%%%%%%%%%%%%%%%%%%%%%%%%%%%%%%%%%%%%%%%%%%%%%%%%%%%%%%%%%%%%%%%%%%%%%%%%%%%%%%%%
%%%%%%%%%%%%%%%%%%%%%%%%%%%%%%%%%%%%%%%%%%%%%%%%%%%%%%%%%%%%%%%%%%%%%%%%%%%%%%%%
\section{Introduction}

\LaTeX{} provides a mechanism to structure a large document (such as a book)
into a main file and several child files (containing the chapters)
using the |\include| command.
This mechanism is beneficial for documents
which span hundreds of pages in order to
make the source file(s) more manageable.
Moreover, compilation can be restricted to
selected child files by means of the |\includeonly| command.
The latter feature can be used to reduce the compilation time while editing
(this was significantly more useful in the earlier days of \LaTeX{})
or to generate a smaller document which is easier to navigate.
Another application of |\includeonly| is to generate
documents consisting of selected parts of the complete document.

However, there are a few drawbacks of the plain |\include| mechanism:
\begin{itemize}
\item
The child files cannot be compiled on their own,
they can only be compiled via the main file.
A naive editing environment
(such as a text editor with an option
to have the current file processed by \LaTeX)
may require one to switch to the main file before compiling;
attempting to compile the child file produces errors.
\item
The main file must be modified (each time)
to adjust the |\includeonly| command
to the present needs. This easily leaves the main file in a messy state.
\item
The generated document will always carry the filename
of the main document. This is inconvenient if
several child files are to be compiled and
to be kept for distribution.
\end{itemize}

The present package provides a simple interface
to make child files individually compilable by \LaTeX{}.
Compiling a child file then has the same effect as compiling
the main file with an |\includeonly| command
to select the appropriate child.
Moreover the generated document will carry the name of the child
rather than the main file.
This resolves all three above issues.

This feature is meant to make the editing of books,
thesis documents and lecture notes somewhat more convenient.
However, the package can also be used efficiently for
composing a series of documents (such as exercise sheets)
which are typically distributed individually.
It then assists the author in generating the individual documents
(potentially in different versions)
as well as a document containing the collected series.
Another application is in developing style files
or other kinds of included material
where compilation of the style file could redirect
to a sample or test file.

%%%%%%%%%%%%%%%%%%%%%%%%%%%%%%%%%%%%%%%%%%%%%%%%%%%%%%%%%%%%%%%%%%%%%%%%%%%%%%%%
%%%%%%%%%%%%%%%%%%%%%%%%%%%%%%%%%%%%%%%%%%%%%%%%%%%%%%%%%%%%%%%%%%%%%%%%%%%%%%%%
\section{Usage}

First of all, the package \textsf{childdoc} is \emph{not} a standard
\LaTeXe{} |.sty| style file! Therefore it needs to be invoked in
a non-standard way.

%%%%%%%%%%%%%%%%%%%%%%%%%%%%%%%%%%%%%%%%%%%%%%%%%%%%%%%%%%%%%%%%%%%%%%%%%%%%%%%%
\subsection{Included Files}
\label{sec:include}

%%%%%%%%%%%%%%%%%%%%%%%%%%%%%%%%%%%%%%%%
\DescribeMacro{\childdocmain}
To use the package, add the commands
\begin{center}
\begin{tabular}{l}
|\input{childdoc.def}|\\
|\childdocmain{}|\\
\end{tabular}
\end{center}
at the very top of the main \LaTeX{} file,
in particular \emph{before} the |\documentclass| statement!
The argument of |\childdocmain| should be left empty
(but it must be present).

%%%%%%%%%%%%%%%%%%%%%%%%%%%%%%%%%%%%%%%%
\DescribeMacro{\childdocof}
Furthermore, add the commands
\begin{center}
\begin{tabular}{l}
|\input{childdoc.def}|\\
|\childdocof{|\textit{main}|}|\\
\end{tabular}
\end{center}
at the top of every child file \textit{child}
which is included by |\include{|\textit{child}|}|
from within the main file
(or at least for those files to be compiled individually).
The argument \textit{main} must be the filename of the main file.

There are a couple of
considerations in setting up the main and child documents:

%%%%%%%%%%%%%%%%%%%%%%%%%%%%%%%%%%%%%%%%
\paragraph{Restrictions.}

Please note the following restrictions:
\begin{itemize}
\item
|\childdocmain| must be called with one argument \textit{main}
to ensure compatibility with earlier version of the package.
It must either be empty (|\childdocmain{}|)
or precisely match the filename of the main file in which it is specified.
See \secref{sec:detection} for further information.
\item
The filename \textit{main} must be specified without the |.tex| extension.
\item
The filename \textit{main} is case sensitive
(even in case-insensitive file systems)
due to internal string comparison.
\item
The argument \textit{main} should be fully expanded, it cannot be a macro.
\item
Subdirectories and special characters should be avoided in filenames.
\item
The command |\childdocmain{|\textit{main}|}| must be followed by a whitespace.
It should not be followed immediately by another command
or by a comment mark `|%|'.
This is because the \TeX{} parser reads the token immediately following
the argument of |\childdocmain| and puts it
at the beginning of every child section;
however, a white\-space is ignored.
\end{itemize}

%%%%%%%%%%%%%%%%%%%%%%%%%%%%%%%%%%%%%%%%
\paragraph{Content of Main File.}

It is advisable to place all content in the child files included by |\include|.
Any output contained in the main file will appear in all child documents
unless suppressed manually;
it cannot be suppressed automatically by the |\includeonly| directive
and thus should normally be avoided.
A method to include some content in the main file
by means of conditional processing is described in \secref{sec:conditional}.

%%%%%%%%%%%%%%%%%%%%%%%%%%%%%%%%%%%%%%%%
\paragraph{Page Numbering.}

When only a part of the document is compiled,
the appropriate numbering of pages
(as well as other status parameters)
is determined from the |.aux| files.
The latter contain information from previous passes.
However this information needs to propagate through
all intermediate child documents.
Therefore the page numbering in child documents may well
be inconsistent until the complete document is compiled at least once.

A useful (if unconventional) way to always ensure a consistent
page numbering is to restart the numbering in each child document
and denote the pages by `\textit{child}|.|\textit{page}'
where \textit{child} represents the chapter/section number of the child file.
This can be achieved by the command
|\numberwithin{page}{|\textit{child}|}|
of the \textsf{amsmath} package
where \textit{child} can be |chapter| or |section|
depending on the chosen structuring.
Alternatively, one can modify the macro |\thepage| appropriately
and reset the counter |page| at the start of each child file.

%%%%%%%%%%%%%%%%%%%%%%%%%%%%%%%%%%%%%%%%%%%%%%%%%%%%%%%%%%%%%%%%%%%%%%%%%%%%%%%%
\subsection{Conditional Processing}
\label{sec:conditional}

The package provides a mechanism to compile different versions
of a document. To customise the versions further some conditional processing
can come in handy to distinguish which version is being compiled.
The package provides two macros to describe the compilation context:

%%%%%%%%%%%%%%%%%%%%%%%%%%%%%%%%%%%%%%%%
\DescribeMacro{\ifchilddoc}
The conditional |\ifchilddoc| distinguishes between the compilation of
child documents and the main document:
%
\begin{center}
|\ifchilddoc |\textit{child-code}| |[|\||else |\textit{main-code}]| \||fi|
\end{center}

%%%%%%%%%%%%%%%%%%%%%%%%%%%%%%%%%%%%%%%%
\DescribeMacro{\childdocname}
\DescribeMacro{\childdocjob}
The macro |\childdocname| contains the filename (without extension)
of the main or child file being processed.
Note that |\childdocjob| will always contain the name of the main file.

%%%%%%%%%%%%%%%%%%%%%%%%%%%%%%%%%%%%%%%%
\paragraph{Title Page.}

Conditional processing can be used to include a title or banner page
in the main document when proper precautions are taken.
Importantly, the code in the main file should ensure that the page counter
(as well as other status parameters which are stored in the |.aux| files)
takes the same value after the conditional processing.
Otherwise the page numbers may take divergent values
depending on which part is compiled.

For example, a title page could be declared by:
%
\begin{center}
\begin{tabular}{l}
|\ifchilddoc\||else|\\
|\addtocounter{page}{-1}|\\
\textit{code for title page}\\
|\newpage|\\
|\||fi|
\end{tabular}
\end{center}
%
A banner page for the child documents can be generated by:
%
\begin{center}
\begin{tabular}{l}
|\ifchilddoc|\\
|\addtocounter{page}{-1}|\\
\textit{code for banner page}\\
|\newpage|\\
|\||fi|
\end{tabular}
\end{center}
%
Here one could write a message such as:
\begin{center}
|This is the part \childdocname{} of \childdocjob{}.|
\end{center}

%%%%%%%%%%%%%%%%%%%%%%%%%%%%%%%%%%%%%%%%%%%%%%%%%%%%%%%%%%%%%%%%%%%%%%%%%%%%%%%%
\subsection{Flags}
\label{sec:flags}

The package makes it easy to generate different versions
of the main or child documents.
To this end compilation flags can be defined
and assigned different default values.
They will be particularly useful in conjunction
with the forwarding mechanism described in \secref{sec:forward}.

For example, it may be useful to have a flag |\version|
which can be set to |draft| or |final|.
The document source will contain some conditional code
depending on the value of |\version|.
Suppose further, the flag should default to |final| for the main file
and to |draft| for child files
which is a natural assignment for editing the document.
This is achieved by placing the following code
in the preamble of the main document
(below the |\childdocmain| directive):
%
\begin{center}
\begin{tabular}{l}
|\ifchilddoc|\\
|\providecommand{\version}{draft}|\\
|\||else|\\
|\providecommand{\version}{final}|\\
|\||fi|
\end{tabular}
\end{center}
%
The definition by |\providecommand| makes sure
that previous definitions are not overwritten.
Further statements |\providecommand{\version}{...}|
can thus be added before the above code to override it.

For the main file, one might add a line
(between |\childdocmain| and the above block)
%
\begin{center}
|%\ifchilddoc\||else\providecommand{\version}{draft}\||fi|
\end{center}
%
which can be uncommented to produce a draft version.
Likewise one can add a line to the very top of a child file
(above the |\childdocof{|\textit{main}|}| directive)
%
\begin{center}
|%\providecommand{\version}{final}|
\end{center}
%
which can be uncommented to produce the final version of this child document.

%%%%%%%%%%%%%%%%%%%%%%%%%%%%%%%%%%%%%%%%%%%%%%%%%%%%%%%%%%%%%%%%%%%%%%%%%%%%%%%%
\subsection{Forwarding}
\label{sec:forward}

Different versions of the main or child documents
using compilation flags as described in \secref{sec:flags}
can be (permanently) stored in different files
for convenient compilation, viewing and distribution.
To this end, the package defines a command
to pass on compilation to a different file:

%%%%%%%%%%%%%%%%%%%%%%%%%%%%%%%%%%%%%%%%
\DescribeMacro{\childdocforward}
The command |\childdocforward| redirects processing to
another source file:
%
\begin{center}
\begin{tabular}{l}
|\input{childdoc.def}|\\
|\childdocforward[|\textit{main}|]{|\textit{dest}|}|\\
\end{tabular}
\end{center}
%
The argument \textit{dest} is the destination file
(without extension).
It should be the main file or one of the child files.
Note that further \textsf{childdoc} directives
such as |\childdocof| and |\childdocforward|
in the indicated file will be processed in this form.
The optional argument \textit{main}
passes on directly to the main file \textit{main}
while pretending to compile the child \textit{dest}.
This form behaves as if \textit{dest}
issues |\childdocof{|\textit{main}|}| right away,
and no further \textsf{childdoc} directives will be processed.

%%%%%%%%%%%%%%%%%%%%%%%%%%%%%%%%%%%%%%%%
\DescribeMacro{\...prefix}
In the alternative form |\childdocforwardprefix|,
%
\begin{center}
\begin{tabular}{l}
|\input{childdoc.def}|\\
|\childdocforwardprefix[|\textit{main}|]{|\textit{prefix}|}{|\textit{dest}|}|
\end{tabular}
\end{center}
%
the destination file is determined by a pattern
depending on the current file:
To make this work, the current file must be called
`{\textit{prefix}\hspace{0.2em}\textit{suffix}}'
with \textit{prefix} matching precisely the argument.
Processing is then passed on to the file
`{\textit{dest}\hspace{0.2em}\textit{suffix}}'.
Surely, the same effect is achieved by
directly specifying the
argument `{\textit{dest}\hspace{0.2em}\textit{suffix}}'
in the first form.
However, that requires to set up a different file
for each child. With the alternative form of the command
all these files can have exactly the same content
which simplifies setting them up and maintaining them.

For example, the following file |draft.tex|
with a compilation flag |\version| as described in \secref{sec:flags}
compiles the main document as a draft:
%
\begin{center}
\begin{tabular}{l}
|\def\version{draft}|\\
|\input{childdoc.def}|\\
|\childdocforward{|\textit{main}|}|
\end{tabular}
\end{center}
%
Likewise, the following files |final|\textit{nn}|.tex|
compile the final version of the child document
|child|\textit{nn}|.tex|:
%
\begin{center}
\begin{tabular}{l}
|\def\version{final}|\\
|\input{childdoc.def}|\\
|\childdocforwardprefix{final}{child}|
\end{tabular}
\end{center}
%

Note that when several versions of a main file and/or of each child file
are to be generated, it may be convenient to set up a |Makefile| or
shell script to automatise the process.

%%%%%%%%%%%%%%%%%%%%%%%%%%%%%%%%%%%%%%%%%%%%%%%%%%%%%%%%%%%%%%%%%%%%%%%%%%%%%%%%
\subsection{Command Line Processing}
\label{sec:commandline}

The effect of redirection files can also be achieved by invoking
the \LaTeX{} compiler with a more elaborate command line.
Most conveniently this should be done as part
of a shell script or a |Makefile|.

When using \textsf{childdoc} in the main file, the following
command lines effectively perform a redirection
(note that depending on the shell being used,
backslashes may have to be doubled: `|\|' $\to$ `|\\|'):
%
\begin{center}
|... -jobname "|\textit{target}|" |\\|"|[\textit{flags}]%
|\input{childdoc.def}\childdocforward[|\textit{main}|]{|\textit{dest}|}"|
\end{center}
%
Here \textit{target} is the name of the output file,
\textit{main} is the name of the main file
and \textit{dest} is the name of the main or child file to be processed
(all filenames without extensions).
The optional argument \textit{main} can be omitted
if \textit{main} matches \textit{dest}.
Optionally, compilation \textit{flags} can be defined via |\def| commands.
This command line makes the \TeX{} engine believe
it is compiling the file \textit{target}
whose content is specified as the latter parameter.
The provided code then forwards the processing to
\textit{main} or \textit{dest} as described in \secref{sec:forward}.

%%%%%%%%%%%%%%%%%%%%%%%%%%%%%%%%%%%%%%%%%%%%%%%%%%%%%%%%%%%%%%%%%%%%%%%%%%%%%%%%
\subsection{Include by Input}
\label{sec:input}

Including child documents by |\include| has some restrictions by design.
Most notably, the content of a child document always occupies
its own set of pages; pages cannot be shared between child documents.
Usually, this behaviour makes perfect sense
because each child document contain an essential part of the document.
However, in some situations it may be desirable to compose
a document from a collection of parts
without having mandatory page breaks between then.
For this case, the package
provides a mechanism to include parts
by |\input| which can also be processed individually.
However, by construction this mechanism
requires manual handling of the content to be output.

%%%%%%%%%%%%%%%%%%%%%%%%%%%%%%%%%%%%%%%%
\DescribeMacro{\ifchilddocmanual}
The main file should be prepared as usual, see \secref{sec:include}.
However, the document body must make a distinction
between processing of an individual part and of the main document, e.g.:
%
\begin{center}
\begin{tabular}{l}
|\ifchilddocmanual|\\
|\input{\childdocname}|\\
|\||else|\\
\textit{document body with }|\input{|\textit{part}|}|\\
|\||fi|
\end{tabular}
\end{center}
%
The conditional |\ifchilddocmanual| is true whenever
a part to be included by |\input| is being compiled,
and the name of the part is stored in |\childdocname|.

%%%%%%%%%%%%%%%%%%%%%%%%%%%%%%%%%%%%%%%%
\DescribeMacro{\childdocby}
Each part to be included by |\input| should start with:
%
\begin{center}
\begin{tabular}{l}
|\input{childdoc.def}|\\
|\childdocby{|\textit{main}|}|\\
\end{tabular}
\end{center}
%
The directive |\childdocby| is similar to |\childdocof|
described in \secref{sec:include},
but the subsequent selection of content must be done manually.
To that end, both |\ifchilddoc| and |\ifchilddocmanual|
will be true upon processing of a part,
and the name of the part is stored in |\childdocname|.
Note that |\jobname| will be set to the filename of the current part
so that each part receives an individual |.aux| file
that does not interfere with the |.aux| file(s) of the main document.
This behaviour can be altered by the alternative form
|\childdocby[*]{|\textit{main}|}| (with a non-empty optional argument)
which uses the |.aux| file of the main document
by setting |\jobname| to \textit{main}.

%%%%%%%%%%%%%%%%%%%%%%%%%%%%%%%%%%%%%%%%%%%%%%%%%%%%%%%%%%%%%%%%%%%%%%%%%%%%%%%%
\subsection{Driver Development}
\label{sec:driver}

The \textsf{childdoc} mechanism can also be use for the development
of definition files such as \LaTeX{} styles or classes.
This case differs from the above setup with multiple parts
included by |\include| in that no |\includeonly| should be invoked.
This can be achieved by starting the include file
(before |\ProvidesPackage|) with:
%
\begin{center}
\begin{tabular}{l}
|\input{childdoc.def}|\\
|\childdocforward{|\textit{main}|}|\\
\end{tabular}
\end{center}
%
or alternatively with:
%
\begin{center}
\begin{tabular}{l}
|\input{childdoc.def}|\\
|\childdocby{|\textit{main}|}|\\
\end{tabular}
\end{center}
%
Both forms have slightly different effects as described above.
The main file is prepared as usual, see \secref{sec:include}.

%%%%%%%%%%%%%%%%%%%%%%%%%%%%%%%%%%%%%%%%%%%%%%%%%%%%%%%%%%%%%%%%%%%%%%%%%%%%%%%%
\subsection{Legacy Detection}
\label{sec:detection}

The directive |\childdocmain| in the main file can detect
whether the complete document or merely a child is to be compiled
even without using the directive |\childdocof|.
This method is deprecated because it is less robust
and there is no compelling reason to use it;
it is merely provided for backward compatibility
and it may be removed in future versions.

If the detection mechanism is to be used,
it is mandatory to correctly specify
the filename of the main file as the argument of |\childdocmain|:
%
\begin{center}
\begin{tabular}{l}
|\input{childdoc.def}|\\
|\childdocmain{|\textit{main}|}|\\
\end{tabular}
\end{center}
%
If |\jobname| does not match the argument \textit{main} of |\childdocmain|,
it is assumed that |\jobname| points to the child file to be compiled.
When using |\childdocmain| with the main file specified as argument,
it suffices to start a child file
with just |\input{|\textit{main}|}|
without loading of the package and using |\childdocof|.
If instead all processing is done
with the appropriate \textsf{childdoc} directives,
the argument of \textit{main} of |\childdocmain| can be empty.

An alternative version of the command line processing described
in \secref{sec:commandline} using the detection mechanism reads:
%
\begin{center}
|... -jobname "|\textit{target}|" "|[\textit{flags}]%
[|\def\jobname{|\textit{dest}|}|]|\input{|\textit{main}|}"|
\end{center}

%%%%%%%%%%%%%%%%%%%%%%%%%%%%%%%%%%%%%%%%%%%%%%%%%%%%%%%%%%%%%%%%%%%%%%%%%%%%%%%%
\subsection{Manual Code}
\label{sec:manual}

In case one cannot be certain whether the definitions file |childdoc.def|
is installed on the target \TeX{} distribution
and one prefers not to ship it,
it is conceivable to paste a few relevant commands into the sources.

To that end, drop all statements |\input{childdoc.def}|
and perform the replacements as outlined below.
Instead of |\childdocmain{|\textit{main}|}| add the following code
to the top of the main file:
%
\begin{center}
\begin{tabular}{l}
|\||ifdefined\childdocname\endinput\||fi\newif\ifchilddoc|\\
|\edef\childdocname{\scantokens\expandafter{\jobname\noexpand}}|\\
|\def\childdocmain{|\textit{main}|}\||ifx\childdocmain\childdocname\||else|\\
|\childdoctrue\includeonly{\childdocname}\let\jobname\childdocmain\||fi|\\
\end{tabular}
\end{center}
%
Instead of |\childdocof{|\textit{main}|}| just include the main file
at the top of each child file:
%
\begin{center}
|\input{|\textit{main}|}|
\end{center}
%
A simple redirection |\childdocforward{|\textit{dest}|}| is achieved by:
%
\begin{center}
|\def\jobname{|\textit{dest}|}\input{\jobname}|
\end{center}
%
The redirection with prefix
|\childdocforwardprefix[|\textit{prefix}|]{|\textit{dest}|}|
is accomplished by:
%
\begin{center}
\begin{tabular}{l}
|{\edef\jobname{\scantokens\expandafter{\jobname\noexpand}}|\\
|\def\redirectjob |\textit{prefix}|#1~~~{\gdef\jobname{|\textit{dest}|#1}}|\\
|\expandafter\redirectjob\jobname~~~}\input{\jobname}|
\end{tabular}
\end{center}

In an alternative approach,
child documents can be compiled by a specific command line
without additional code or specific definitions:
%
\begin{center}
|... -jobname "|\textit{target}|" "|[\textit{flags}]%
|\includeonly{|\textit{dest}|}\input{|\textit{main}|}"|
\end{center}
%

%%%%%%%%%%%%%%%%%%%%%%%%%%%%%%%%%%%%%%%%%%%%%%%%%%%%%%%%%%%%%%%%%%%%%%%%%%%%%%%%
%%%%%%%%%%%%%%%%%%%%%%%%%%%%%%%%%%%%%%%%%%%%%%%%%%%%%%%%%%%%%%%%%%%%%%%%%%%%%%%%
\section{Information}

%%%%%%%%%%%%%%%%%%%%%%%%%%%%%%%%%%%%%%%%%%%%%%%%%%%%%%%%%%%%%%%%%%%%%%%%%%%%%%%%
\subsection{Copyright}

Copyright \copyright{} 2017--2018 Niklas Beisert

This work may be distributed and/or modified under the
conditions of the \LaTeX{} Project Public License, either version 1.3
of this license or (at your option) any later version.
The latest version of this license is in
  \url{http://www.latex-project.org/lppl.txt}
and version 1.3 or later is part of all distributions of \LaTeX{}
version 2005/12/01 or later.

This work has the LPPL maintenance status `maintained'.

The Current Maintainer of this work is Niklas Beisert.

This work consists of the files |README.txt|, |childdoc.ins| and |childdoc.dtx|
as well as the derived files |childdoc.def|, |cdocsamp.tex|
with |cdocsch1.tex|, |cdocsch2.tex|, |cdocspt3.tex|, |cdocspt4.tex|,
|cdocsdrf.tex|, |cdocsfn1.tex|, |cdocsfn2.tex|
as well as |childdoc.pdf|.

%%%%%%%%%%%%%%%%%%%%%%%%%%%%%%%%%%%%%%%%%%%%%%%%%%%%%%%%%%%%%%%%%%%%%%%%%%%%%%%%
\subsection{Files and Installation}

The package consists of the files:
%
\begin{center}
\begin{tabular}{ll}
    |README.txt|   & readme file \\
    |childdoc.ins| & installation file \\
    |childdoc.dtx| & source file \\
    |childdoc.def| & definition file \\
    |cdocsamp.tex| & sample main file \\
    |cdocsch1.tex| & sample include file \\
    |cdocsch2.tex| & sample include file \\
    |cdocspt3.tex| & sample part file \\
    |cdocspt4.tex| & sample part file \\
    |cdocsdrf.tex| & sample redirection file \\
    |cdocsfn1.tex| & sample redirection file \\
    |cdocsfn2.tex| & sample redirection file \\
    |childdoc.pdf| & manual
\end{tabular}
\end{center}
%
The distribution consists of the files
|README.txt|, |childdoc.ins| and |childdoc.dtx|.
%
\begin{itemize}
\item
Run (pdf)\LaTeX{} on |childdoc.dtx|
to compile the manual |childdoc.pdf| (this file).
\item
Run \LaTeX{} on |childdoc.ins| to create the definitions file |childdoc.def|
and the sample |cdocsamp.tex| with include files
|cdocsch1.tex|, |cdocsch2.tex|, |cdocspt3.tex|, |cdocspt4.tex|,
|cdocsdrf.tex|, |cdocsfn1.tex|, |cdocsfn2.tex|.
Then copy the file |childdoc.def| to an appropriate directory of your \LaTeX{}
distribution, e.g.\ \textit{texmf-root}|/tex/latex/childdoc|.
\end{itemize}

%%%%%%%%%%%%%%%%%%%%%%%%%%%%%%%%%%%%%%%%%%%%%%%%%%%%%%%%%%%%%%%%%%%%%%%%%%%%%%%%
\subsection{Related CTAN Packages}

There are several other packages which offer a similar functionality:
%
\begin{itemize}
\item
The packages
\href{http://ctan.org/pkg/docmute}{\textsf{docmute}},
\href{http://ctan.org/pkg/includex}{\textsf{includex}} and
\href{http://ctan.org/pkg/standalone}{\textsf{standalone}}
provide commands to include only the document body of
a child file thus allowing both files to be compiled individually.
\item
The packages \href{http://ctan.org/pkg/subdocs}{\textsf{subdocs}}
and \href{http://ctan.org/pkg/subfiles}{\textsf{subfiles}}
provide structures in which the main and child documents can be
encapsulated and allowing them to be compiled individually.
The inclusion mechanism is different from the conventional |\include|.
\item
The package \href{http://ctan.org/pkg/combine}{\textsf{combine}}
is an elaborate solution to combine several documents into one.
\end{itemize}
%
See also the CTAN topic \href{http://ctan.org/topic/subdocs}{\textsf{subdocs}}
for further related packages.
The present package differs from the above solutions in that
a document structure constructed with the conventional |\include| mechanism
just needs two extra commands at the top of every file
such that all constituent files can be compiled individually.

%%%%%%%%%%%%%%%%%%%%%%%%%%%%%%%%%%%%%%%%%%%%%%%%%%%%%%%%%%%%%%%%%%%%%%%%%%%%%%%%
%\subsection{Feature Suggestions}
%
%The following is a list of features which may be useful for future
%versions of this package:
%%
%\begin{itemize}
%\item
%\ldots
%\end{itemize}

%%%%%%%%%%%%%%%%%%%%%%%%%%%%%%%%%%%%%%%%%%%%%%%%%%%%%%%%%%%%%%%%%%%%%%%%%%%%%%%%
\subsection{Revision History}

%%%%%%%%%%%%%%%%%%%%%%%%%%%%%%%%%%%%%%%%
\paragraph{v2.0:} 2018/12/30

\begin{itemize}
\item
immediate forward processing
\item
added |\childdocby| mechanism
\item
manual restructured
\end{itemize}

%%%%%%%%%%%%%%%%%%%%%%%%%%%%%%%%%%%%%%%%
\paragraph{v1.6:} 2018/01/17

\begin{itemize}
\item
application for development of include files
\item
corrections to manual
\end{itemize}

%%%%%%%%%%%%%%%%%%%%%%%%%%%%%%%%%%%%%%%%
\paragraph{v1.5:} 2017/05/21

\begin{itemize}
\item
more complete structuring introduced
\item
|\childdocof| introduced
\item
|\childdoc| renamed to |\childdocmain|
\item
|\childredirect| renamed to |\childdocforward| and |\childdocforwardprefix|
and functionality expanded
\end{itemize}

%%%%%%%%%%%%%%%%%%%%%%%%%%%%%%%%%%%%%%%%
\paragraph{v1.0:} 2017/04/27

\begin{itemize}
\item
manual and install package
\item
first version published on CTAN
\end{itemize}

%%%%%%%%%%%%%%%%%%%%%%%%%%%%%%%%%%%%%%%%
\paragraph{v0.6:} 2017/04/26

\begin{itemize}
\item
redirection mechanism added
\end{itemize}

%%%%%%%%%%%%%%%%%%%%%%%%%%%%%%%%%%%%%%%%
\paragraph{v0.5:} 2017/04/26

\begin{itemize}
\item
functionality in definition file
\end{itemize}


%%%%%%%%%%%%%%%%%%%%%%%%%%%%%%%%%%%%%%%%%%%%%%%%%%%%%%%%%%%%%%%%%%%%%%%%%%%%%%%%
%%%%%%%%%%%%%%%%%%%%%%%%%%%%%%%%%%%%%%%%%%%%%%%%%%%%%%%%%%%%%%%%%%%%%%%%%%%%%%%%
%%%%%%%%%%%%%%%%%%%%%%%%%%%%%%%%%%%%%%%%%%%%%%%%%%%%%%%%%%%%%%%%%%%%%%%%%%%%%%%%
\appendix

\settowidth\MacroIndent{\rmfamily\scriptsize 000\ }

 \DocInput{childdoc.dtx}

\end{document}
%</driver>
% \fi
%
% %%%%%%%%%%%%%%%%%%%%%%%%%%%%%%%%%%%%%%%%%%%%%%%%%%%%%%%%%%%%%%%%%%%%%%%%%%%%%%
% %%%%%%%%%%%%%%%%%%%%%%%%%%%%%%%%%%%%%%%%%%%%%%%%%%%%%%%%%%%%%%%%%%%%%%%%%%%%%%
% \section{Sample}
%\iffalse
%<*samplemain>
%\fi
%
% The following presents a sample document
% with two chapters, two parts, a title page,
% a compile flag as well as three forwarding files to set the flag.
% It consists of eight |.tex| files:
% \begin{center}
% \begin{tabular}{ll}
% |cdocsamp.tex|&main file\\
% |cdocsch1.tex|&include file for chapter 1\\
% |cdocsch2.tex|&include file for chapter 2\\
% |cdocspt3.tex|&include file for part 3\\
% |cdocspt4.tex|&include file for part 4\\
% |cdocsdrf.tex|&forwarding file for main file in draft mode\\
% |cdocsfi1.tex|&forwarding file for final version of chapter 1\\
% |cdocsfi2.tex|&forwarding file for final version of chapter 2\\
% \end{tabular}
% \end{center}
% Each of the eight files can be compiled directly by the \LaTeX{} compiler.
%
% %%%%%%%%%%%%%%%%%%%%%%%%%%%%%%%%%%%%%%
% \paragraph{Main File.}
%
% The main file is called |cdocsamp.tex|.
%
% Load the \textsf{childdoc} definitions and
% declare the filename for the main document:
%    \begin{macrocode}
\input{childdoc.def}
\childdocmain{}
%    \end{macrocode}

% Optional override for |\version| flag:
%    \begin{macrocode}
%%\ifchilddoc\else\providecommand{\version}{draft}\fi
%    \end{macrocode}

% Define the default values for the |\version| flag
% (|final| for the main file and |draft| for childs):
%    \begin{macrocode}
\ifchilddoc
\providecommand{\version}{draft}
\else
\providecommand{\version}{final}
\fi
%    \end{macrocode}

% Load the standard document class:
%    \begin{macrocode}
\documentclass[12pt]{article}
%    \end{macrocode}

% Start the document body:
%    \begin{macrocode}
\begin{document}
%    \end{macrocode}

% Declare a title page.
% Print title, part of document being processed and version flag:
%    \begin{macrocode}
\addtocounter{page}{-1}
\begin{center}
{\LARGE\bfseries{}childdoc example\par}
\vspace{1cm}
\ifchilddoc
\ifchilddocmanual part\else chapter\fi:
`\childdocname' of `\childdocjob'\par
\else
main document: `\childdocjob'\par
\fi
version: \version\par
\end{center}
\newpage
%    \end{macrocode}

% Manually include selected file,
% otherwise process as usual:
%    \begin{macrocode}
\ifchilddocmanual
\section*{part `\childdocname'}
\input{\childdocname}
\else
%    \end{macrocode}

% Include the two chapters:
%    \begin{macrocode}
\include{cdocsch1}
\include{cdocsch2}
%    \end{macrocode}

% Include the two parts unless only chapters should be displayed:
%    \begin{macrocode}
\ifchilddoc\else
\section{part three}
\input{cdocspt3}
\section{part four}
\input{cdocspt4}
\fi
%    \end{macrocode}

% Process as usual until here:
%    \begin{macrocode}
\fi
%    \end{macrocode}

% End of document body:
%    \begin{macrocode}
\end{document}
%    \end{macrocode}
%\iffalse
%</samplemain>
%\fi
%
% %%%%%%%%%%%%%%%%%%%%%%%%%%%%%%%%%%%%%%
% \paragraph{Chapter Include Files.}
%
% The include files are called |cdocsch1.tex| and |cdocsch2.tex|.
%
%\iffalse
%<*samplechap1|samplechap2>
%\fi

% Optional override for |\version| flag:
%    \begin{macrocode}
%%\providecommand{\version}{final}
%    \end{macrocode}

% Include the main document:
%    \begin{macrocode}
\input{childdoc.def}
\childdocof{cdocsamp}
%    \end{macrocode}

%\iffalse
%</samplechap1|samplechap2>
%\fi
%
%\iffalse
%<*samplechap1>
%\fi
% Some text for chapter 1:
%    \begin{macrocode}
\section{one}
some text in chapter one
%    \end{macrocode}

%\iffalse
%</samplechap1>
%\fi
% Some text for chapter 2:
%\iffalse
%<*samplechap2>
%\fi
%    \begin{macrocode}
\section{two}
more text in chapter two
%    \end{macrocode}

%\iffalse
%</samplechap2>
%\fi
%
% %%%%%%%%%%%%%%%%%%%%%%%%%%%%%%%%%%%%%%
% \paragraph{Part Include Files.}
%
% The include files are called |cdocspt3.tex| and |cdocspt4.tex|.
%
%\iffalse
%<*samplepart3|samplepart4>
%\fi

% Optional override for |\version| flag:
%    \begin{macrocode}
%%\providecommand{\version}{final}
%    \end{macrocode}

% Include the main document:
%    \begin{macrocode}
\input{childdoc.def}
\childdocby{cdocsamp}
%    \end{macrocode}

%\iffalse
%</samplepart3|samplepart4>
%\fi
%
%\iffalse
%<*samplepart3>
%\fi
% Some text for part 3:
%    \begin{macrocode}
some text in part three
%    \end{macrocode}

%\iffalse
%</samplepart3>
%\fi
% Some text for part 4:
%\iffalse
%<*samplepart4>
%\fi
%    \begin{macrocode}
more text in part four
%    \end{macrocode}

%\iffalse
%</samplepart4>
%\fi
%
% %%%%%%%%%%%%%%%%%%%%%%%%%%%%%%%%%%%%%%
% \paragraph{Forwarding for a Complete Draft.}
%
% The following forwarding file |cdocsdrf.tex|
% compiles the main document in draft mode:
%\iffalse
%<*sampledraft>
%\fi
%    \begin{macrocode}
\def\version{draft}
\input{childdoc.def}
\childdocforward{cdocsamp}
%    \end{macrocode}

%\iffalse
%</sampledraft>
%\fi
%
% %%%%%%%%%%%%%%%%%%%%%%%%%%%%%%%%%%%%%%
% \paragraph{Forwarding for Final Version of the Chapters.}
%
% The following forwarding files |cdocsfn1.tex| and |cdocsfn2.tex|
% (with identical content)
% compile the final versions of the child documents
% |cdocsch1.tex| and |cdocsch2.tex|, respectively:
%\iffalse
%<*samplefinal>
%\fi
%    \begin{macrocode}
\def\version{final}
\input{childdoc.def}
\childdocforwardprefix[cdocsamp]{cdocsfn}{cdocsch}
%    \end{macrocode}

%\iffalse
%</samplefinal>
%\fi
%
% %%%%%%%%%%%%%%%%%%%%%%%%%%%%%%%%%%%%%%
% \paragraph{Command Line Processing.}
%
% The following three command lines generate the output files
% |cdocscld|, |cdocscl1| and |cdocscl2|
% which should be identical to
% |cdocsdrf|, |cdocsch1| and |cdocsfn2|, respectively:
% \begin{center}
% \begin{tabular}{l}
% |latex -jobname cdocscld \|\\
% |  "\def\version{draft}\input{childdoc.def}\childdocforward{cdocsamp}"|\\
% |latex -jobname cdocscl1 \|\\
% |  "\input{childdoc.def}\childdocforward[cdocsamp]{cdocsch1}"|\\
% |latex -jobname cdocscl2 \|\\
% |  "\def\version{final}\input{childdoc.def}\childdocforward{cdocsch2}"|
% \end{tabular}
% \end{center}
% Note that the trailing backslash on each first line
% merely continues the input to the second line
% (for convenient cut ant paste).
% Furthermore, the command |latex| can be replaced by any
% of its alternative versions such as |pdflatex|.
%
% %%%%%%%%%%%%%%%%%%%%%%%%%%%%%%%%%%%%%%%%%%%%%%%%%%%%%%%%%%%%%%%%%%%%%%%%%%%%%%
% %%%%%%%%%%%%%%%%%%%%%%%%%%%%%%%%%%%%%%%%%%%%%%%%%%%%%%%%%%%%%%%%%%%%%%%%%%%%%%
% \section{Implementation}
%\iffalse
%<*package>
%\fi
%
% This section describes the definitions file |childdoc.def|.

% The definitions cannot be loaded using |\usepackage| or |\RequirePackage|
% which has a mechanism to prevent loading a style file more than once.
% When loading the definitions by means of |\input|
% multiple instances have to be prevented manually:
%\iffalse
%This code needs to be before the `\ProvidesFile' directive
%which is defined at the beginning of this file.
%Therefore it is also placed there and commented out here.
%</package>
%<*discard>
%\fi
%    \begin{macrocode}
\ifdefined\childdocmain\endinput\fi
%    \end{macrocode}
%\iffalse
%</discard>
%<*package>
%\fi
%
% \macro{\ifchilddoc}
% \macro{\ifchilddocmanual}
% The conditional |\ifchilddoc| tells whether a
% child (true) or main (false) document is being compiled.
% The conditional |\ifchilddocmanual| tells whether
% the |\includeonly| mechanism is used (false) or
% the selection of child files must be performed manually (true).
% The definitions initialise to false:
%    \begin{macrocode}
\newif\ifchilddoc
\newif\ifchilddocmanual
%    \end{macrocode}

% \macro{\childdocname}
% \macro{\childdocjob}
% The macro |\childdocname| stores the name of the main document
% to be compiled. The macro |\childdocjob| stores the name of
% the document on which the \LaTeX{} compiler was originally invoked.
% The content of |\jobname| cannot be compared
% to filenames specified in the source due to different catcodes.
% The following code rescans |\jobname|, stores the result
% in |\childdocname| and saves a copy in |\childdocjob|:
%    \begin{macrocode}
\edef\childdocname{\scantokens\expandafter{\jobname\noexpand}}
\let\childdocjob\childdocname
%    \end{macrocode}

% \macro{\childdocdisable}
% The macro |\childdocdisable| prevents the main file
% from being processed more than once.
% At this stage, the main document command |\childdocmain|
% is assumed to be called once again where it should do nothing.
% Any subsequent call to it should prevent
% a secondary processing of the main document
% It overwrites the forwarding commands
% |\childdocof| and |\childdocforward|
% with empty macros to prevent further inclusions of the main document:
%    \begin{macrocode}
\newcommand{\childdocdisable}
{
  \renewcommand{\childdocmain}[1]{\renewcommand{\childdocmain}[1]{\endinput}}
  \renewcommand{\childdocof}[1]{}
  \renewcommand{\childdocby}[2][]{}
  \renewcommand{\childdocforward}[2][]{}
  \renewcommand{\childdocdisable}{}
}
%    \end{macrocode}

% \macro{\childdocmain}
% The macro |\childdocmain| is to be called at the top of the main file
% with nothing or the main filename (without extension) as argument.
% First, it breaks loops.
% If the argument is not empty and does not match |\childdocname|
% (which is set by the first inclusion of |childdoc.def|),
% |\ifchilddoc| is set to true, |\includeonly| is applied to the child file
% and |\jobname| is set to the main file
% (for proper handling of |.aux| files):
%    \begin{macrocode}
\newcommand{\childdocmain}[1]
{
  \childdocdisable\childdocmain{}
  \if?#1?\else
    \begingroup
      \def\childdoctmp{#1}
      \ifx\childdoctmp\childdocname
        \def\childdoctmp{}
      \else
        \def\childdoctmp
        {
          \childdoctrue
          \includeonly{\childdocname}
          \def\childdocjob{#1}
          \def\jobname{#1}
        }
      \fi
      \expandafter
    \endgroup
    \childdoctmp
  \fi
}
%    \end{macrocode}

% \macro{\childdocof}
% The command |\childdocof| redirects
% compilation to the main file |#1|.
%    \begin{macrocode}
\newcommand{\childdocof}[1]
{
  \childdocdisable
  \childdoctrue
  \includeonly{\childdocname}
  \def\jobname{#1}
  \def\childdocjob{#1}
  \input{#1}
}
%    \end{macrocode}

% \macro{\childdocby}
% The command |\childdocby| ....
%    \begin{macrocode}
\newcommand{\childdocby}[2][]
{
  \childdocdisable
  \childdoctrue
  \childdocmanualtrue
  \if?#1?\else
    \def\jobname{#2}
  \fi
  \def\childdocjob{#2}
  \input{#2}
  \endinput
}
%    \end{macrocode}

% \macro{\childdocforward}
% The command |\childdocforward| redirects
% compilation to the main file or
% (if the optional argument is given) a child file.
% Parameters are set as if the main file
% or a child file starting with |\childdocof| was compiled.
% Then compilation is handed over to the main file:
%    \begin{macrocode}
\newcommand{\childdocforward}[2][]
{
  \begingroup
    \if?#1?
      \def\childdoctmp
      {
        \def\childdocname{#2}
        \def\childdocjob{#2}
        \def\jobname{#2}
        \input{#2}
        \endinput
      }
    \else
      \def\childdoctmp
      {
        \childdocdisable
        \def\childdocname{#2}
        \childdoctrue
        \includeonly{#2}
        \def\childdocjob{#1}
        \def\jobname{#1}
        \input{#1}
        \endinput
      }
    \fi
    \expandafter
  \endgroup
  \childdoctmp
}
%    \end{macrocode}

% \macro{\childdocforwardprefix}
% The command |\childdocforwardprefix| redirects
% compilation to the main or a child file by means of a pattern.
% The prefix |#1| in the current filename is replaced by |#2|
% and the suffix of the current filename is kept
% (it is assumed that the filename does not contain the substring `|~~~|'
% which is used as a delimiter).
% Compilation is handed over to the new file by |\childdocforward|:
%    \begin{macrocode}
\newcommand{\childdocforwardprefix}[3][]
{
  \begingroup
    \def\childdocextract #2##1~~~{\def\childdoctmp{\childdocforward[#1]{#3##1}}}
    \expandafter\childdocextract\childdocname~~~
    \expandafter
  \endgroup
  \childdoctmp
}
%    \end{macrocode}

% \macro{\childdoc}
% The deprecated macro |\childdoc| is a legacy version of |\childdocmain|:
%    \begin{macrocode}
\newcommand{\childdoc}{\childdocmain}
%    \end{macrocode}

% \macro{\childdocredirect}
% The deprecated macro |\childdocredirect| is a legacy version
% of |\childdocforward| and |\childdocforwardprefix|:
%    \begin{macrocode}
\newcommand{\childdocredirect}[2][]
{
  \begingroup
    \if?#1?
      \def\childdoctmp{\childdocforward{#2}}
    \else
      \def\childdoctmp{\childdocforwardprefix{#1}{#2}}
    \fi
    \expandafter
  \endgroup
  \childdoctmp
}
%    \end{macrocode}

%\iffalse
%</package>
%\fi
%
\endinput
|\\
|\childdocmain{}|\\
\end{tabular}
\end{center}
at the very top of the main \LaTeX{} file,
in particular \emph{before} the |\documentclass| statement!
The argument of |\childdocmain| should be left empty
(but it must be present).

%%%%%%%%%%%%%%%%%%%%%%%%%%%%%%%%%%%%%%%%
\DescribeMacro{\childdocof}
Furthermore, add the commands
\begin{center}
\begin{tabular}{l}
|% \iffalse
%
% childdoc.dtx Copyright (C) 2017-2018 Niklas Beisert
%
% This work may be distributed and/or modified under the
% conditions of the LaTeX Project Public License, either version 1.3
% of this license or (at your option) any later version.
% The latest version of this license is in
%   http://www.latex-project.org/lppl.txt
% and version 1.3 or later is part of all distributions of LaTeX
% version 2005/12/01 or later.
%
% This work has the LPPL maintenance status `maintained'.
%
% The Current Maintainer of this work is Niklas Beisert.
%
% This work consists of the files childdoc.dtx and childdoc.ins
% and the derived files childdoc.def and cdocsamp.tex with
% cdocsch1.tex, cdocsch2.tex, cdocsdrf.tex, cdocsfn1.tex, cdocsfn2.tex.
%
%<package>\ifdefined\childdocmain\endinput\fi
%<package>\ProvidesFile{childdoc.def}[2018/12/30 v2.0 child document driver]
%<samplemain>\ProvidesFile{cdocsamp.tex}[2018/12/30 v2.0 sample for childdoc]
%<*driver>
%\ProvidesFile{childdoc.drv}[2018/12/30 v2.0 childdoc reference manual file]
\PassOptionsToClass{10pt,a4paper}{article}
\documentclass{ltxdoc}

\usepackage[margin=35mm]{geometry}
\usepackage{hyperref}
\usepackage{hyperxmp}
\usepackage[usenames]{color}

\hypersetup{colorlinks=true}
\hypersetup{pdfstartview=FitH}
\hypersetup{pdfpagemode=UseNone}
\hypersetup{pdfsource={}}
\hypersetup{pdflang={en-UK}}
\hypersetup{pdfcopyright={Copyright 2017-2018 Niklas Beisert.
  This work may be distributed and/or modified under the
  conditions of the LaTeX Project Public License, either version 1.3
  of this license or (at your option) any later version.}}
\hypersetup{pdflicenseurl={http://www.latex-project.org/lppl.txt}}
\hypersetup{pdfcontactaddress={ETH Zurich, ITP, HIT K,
  Wolfgang-Pauli-Strasse 27}}
\hypersetup{pdfcontactpostcode={8093}}
\hypersetup{pdfcontactcity={Zurich}}
\hypersetup{pdfcontactcountry={Switzerland}}
\hypersetup{pdfcontactemail={nbeisert@itp.phys.ethz.ch}}
\hypersetup{pdfcontacturl={http://people.phys.ethz.ch/\xmptilde nbeisert/}}

\newcommand{\secref}[1]{\hyperref[#1]{section \ref*{#1}}}

\parskip1ex
\parindent0pt
\let\olditemize\itemize
\def\itemize{\olditemize\parskip0pt}

\begin{document}

\title{The \textsf{childdoc} Package}
\hypersetup{pdftitle={The childdoc Package}}
\author{Niklas Beisert\\[2ex]
  Institut f\"ur Theoretische Physik\\
  Eidgen\"ossische Technische Hochschule Z\"urich\\
  Wolfgang-Pauli-Strasse 27, 8093 Z\"urich, Switzerland\\[1ex]
  \href{mailto:nbeisert@itp.phys.ethz.ch}
  {\texttt{nbeisert@itp.phys.ethz.ch}}}
\hypersetup{pdfauthor={Niklas Beisert}}
\hypersetup{pdfsubject={Manual for the LaTeX2e Package childdoc}}
\date{30 December 2018, \textsf{v2.0}}
\maketitle

\begin{abstract}\noindent
\textsf{childdoc} is a \LaTeXe{} package
that enables the direct compilation
of document sections included by |\include|
to individual files.
\end{abstract}

\begingroup
\parskip0ex
\tableofcontents
\endgroup

%%%%%%%%%%%%%%%%%%%%%%%%%%%%%%%%%%%%%%%%%%%%%%%%%%%%%%%%%%%%%%%%%%%%%%%%%%%%%%%%
%%%%%%%%%%%%%%%%%%%%%%%%%%%%%%%%%%%%%%%%%%%%%%%%%%%%%%%%%%%%%%%%%%%%%%%%%%%%%%%%
\section{Introduction}

\LaTeX{} provides a mechanism to structure a large document (such as a book)
into a main file and several child files (containing the chapters)
using the |\include| command.
This mechanism is beneficial for documents
which span hundreds of pages in order to
make the source file(s) more manageable.
Moreover, compilation can be restricted to
selected child files by means of the |\includeonly| command.
The latter feature can be used to reduce the compilation time while editing
(this was significantly more useful in the earlier days of \LaTeX{})
or to generate a smaller document which is easier to navigate.
Another application of |\includeonly| is to generate
documents consisting of selected parts of the complete document.

However, there are a few drawbacks of the plain |\include| mechanism:
\begin{itemize}
\item
The child files cannot be compiled on their own,
they can only be compiled via the main file.
A naive editing environment
(such as a text editor with an option
to have the current file processed by \LaTeX)
may require one to switch to the main file before compiling;
attempting to compile the child file produces errors.
\item
The main file must be modified (each time)
to adjust the |\includeonly| command
to the present needs. This easily leaves the main file in a messy state.
\item
The generated document will always carry the filename
of the main document. This is inconvenient if
several child files are to be compiled and
to be kept for distribution.
\end{itemize}

The present package provides a simple interface
to make child files individually compilable by \LaTeX{}.
Compiling a child file then has the same effect as compiling
the main file with an |\includeonly| command
to select the appropriate child.
Moreover the generated document will carry the name of the child
rather than the main file.
This resolves all three above issues.

This feature is meant to make the editing of books,
thesis documents and lecture notes somewhat more convenient.
However, the package can also be used efficiently for
composing a series of documents (such as exercise sheets)
which are typically distributed individually.
It then assists the author in generating the individual documents
(potentially in different versions)
as well as a document containing the collected series.
Another application is in developing style files
or other kinds of included material
where compilation of the style file could redirect
to a sample or test file.

%%%%%%%%%%%%%%%%%%%%%%%%%%%%%%%%%%%%%%%%%%%%%%%%%%%%%%%%%%%%%%%%%%%%%%%%%%%%%%%%
%%%%%%%%%%%%%%%%%%%%%%%%%%%%%%%%%%%%%%%%%%%%%%%%%%%%%%%%%%%%%%%%%%%%%%%%%%%%%%%%
\section{Usage}

First of all, the package \textsf{childdoc} is \emph{not} a standard
\LaTeXe{} |.sty| style file! Therefore it needs to be invoked in
a non-standard way.

%%%%%%%%%%%%%%%%%%%%%%%%%%%%%%%%%%%%%%%%%%%%%%%%%%%%%%%%%%%%%%%%%%%%%%%%%%%%%%%%
\subsection{Included Files}
\label{sec:include}

%%%%%%%%%%%%%%%%%%%%%%%%%%%%%%%%%%%%%%%%
\DescribeMacro{\childdocmain}
To use the package, add the commands
\begin{center}
\begin{tabular}{l}
|\input{childdoc.def}|\\
|\childdocmain{}|\\
\end{tabular}
\end{center}
at the very top of the main \LaTeX{} file,
in particular \emph{before} the |\documentclass| statement!
The argument of |\childdocmain| should be left empty
(but it must be present).

%%%%%%%%%%%%%%%%%%%%%%%%%%%%%%%%%%%%%%%%
\DescribeMacro{\childdocof}
Furthermore, add the commands
\begin{center}
\begin{tabular}{l}
|\input{childdoc.def}|\\
|\childdocof{|\textit{main}|}|\\
\end{tabular}
\end{center}
at the top of every child file \textit{child}
which is included by |\include{|\textit{child}|}|
from within the main file
(or at least for those files to be compiled individually).
The argument \textit{main} must be the filename of the main file.

There are a couple of
considerations in setting up the main and child documents:

%%%%%%%%%%%%%%%%%%%%%%%%%%%%%%%%%%%%%%%%
\paragraph{Restrictions.}

Please note the following restrictions:
\begin{itemize}
\item
|\childdocmain| must be called with one argument \textit{main}
to ensure compatibility with earlier version of the package.
It must either be empty (|\childdocmain{}|)
or precisely match the filename of the main file in which it is specified.
See \secref{sec:detection} for further information.
\item
The filename \textit{main} must be specified without the |.tex| extension.
\item
The filename \textit{main} is case sensitive
(even in case-insensitive file systems)
due to internal string comparison.
\item
The argument \textit{main} should be fully expanded, it cannot be a macro.
\item
Subdirectories and special characters should be avoided in filenames.
\item
The command |\childdocmain{|\textit{main}|}| must be followed by a whitespace.
It should not be followed immediately by another command
or by a comment mark `|%|'.
This is because the \TeX{} parser reads the token immediately following
the argument of |\childdocmain| and puts it
at the beginning of every child section;
however, a white\-space is ignored.
\end{itemize}

%%%%%%%%%%%%%%%%%%%%%%%%%%%%%%%%%%%%%%%%
\paragraph{Content of Main File.}

It is advisable to place all content in the child files included by |\include|.
Any output contained in the main file will appear in all child documents
unless suppressed manually;
it cannot be suppressed automatically by the |\includeonly| directive
and thus should normally be avoided.
A method to include some content in the main file
by means of conditional processing is described in \secref{sec:conditional}.

%%%%%%%%%%%%%%%%%%%%%%%%%%%%%%%%%%%%%%%%
\paragraph{Page Numbering.}

When only a part of the document is compiled,
the appropriate numbering of pages
(as well as other status parameters)
is determined from the |.aux| files.
The latter contain information from previous passes.
However this information needs to propagate through
all intermediate child documents.
Therefore the page numbering in child documents may well
be inconsistent until the complete document is compiled at least once.

A useful (if unconventional) way to always ensure a consistent
page numbering is to restart the numbering in each child document
and denote the pages by `\textit{child}|.|\textit{page}'
where \textit{child} represents the chapter/section number of the child file.
This can be achieved by the command
|\numberwithin{page}{|\textit{child}|}|
of the \textsf{amsmath} package
where \textit{child} can be |chapter| or |section|
depending on the chosen structuring.
Alternatively, one can modify the macro |\thepage| appropriately
and reset the counter |page| at the start of each child file.

%%%%%%%%%%%%%%%%%%%%%%%%%%%%%%%%%%%%%%%%%%%%%%%%%%%%%%%%%%%%%%%%%%%%%%%%%%%%%%%%
\subsection{Conditional Processing}
\label{sec:conditional}

The package provides a mechanism to compile different versions
of a document. To customise the versions further some conditional processing
can come in handy to distinguish which version is being compiled.
The package provides two macros to describe the compilation context:

%%%%%%%%%%%%%%%%%%%%%%%%%%%%%%%%%%%%%%%%
\DescribeMacro{\ifchilddoc}
The conditional |\ifchilddoc| distinguishes between the compilation of
child documents and the main document:
%
\begin{center}
|\ifchilddoc |\textit{child-code}| |[|\||else |\textit{main-code}]| \||fi|
\end{center}

%%%%%%%%%%%%%%%%%%%%%%%%%%%%%%%%%%%%%%%%
\DescribeMacro{\childdocname}
\DescribeMacro{\childdocjob}
The macro |\childdocname| contains the filename (without extension)
of the main or child file being processed.
Note that |\childdocjob| will always contain the name of the main file.

%%%%%%%%%%%%%%%%%%%%%%%%%%%%%%%%%%%%%%%%
\paragraph{Title Page.}

Conditional processing can be used to include a title or banner page
in the main document when proper precautions are taken.
Importantly, the code in the main file should ensure that the page counter
(as well as other status parameters which are stored in the |.aux| files)
takes the same value after the conditional processing.
Otherwise the page numbers may take divergent values
depending on which part is compiled.

For example, a title page could be declared by:
%
\begin{center}
\begin{tabular}{l}
|\ifchilddoc\||else|\\
|\addtocounter{page}{-1}|\\
\textit{code for title page}\\
|\newpage|\\
|\||fi|
\end{tabular}
\end{center}
%
A banner page for the child documents can be generated by:
%
\begin{center}
\begin{tabular}{l}
|\ifchilddoc|\\
|\addtocounter{page}{-1}|\\
\textit{code for banner page}\\
|\newpage|\\
|\||fi|
\end{tabular}
\end{center}
%
Here one could write a message such as:
\begin{center}
|This is the part \childdocname{} of \childdocjob{}.|
\end{center}

%%%%%%%%%%%%%%%%%%%%%%%%%%%%%%%%%%%%%%%%%%%%%%%%%%%%%%%%%%%%%%%%%%%%%%%%%%%%%%%%
\subsection{Flags}
\label{sec:flags}

The package makes it easy to generate different versions
of the main or child documents.
To this end compilation flags can be defined
and assigned different default values.
They will be particularly useful in conjunction
with the forwarding mechanism described in \secref{sec:forward}.

For example, it may be useful to have a flag |\version|
which can be set to |draft| or |final|.
The document source will contain some conditional code
depending on the value of |\version|.
Suppose further, the flag should default to |final| for the main file
and to |draft| for child files
which is a natural assignment for editing the document.
This is achieved by placing the following code
in the preamble of the main document
(below the |\childdocmain| directive):
%
\begin{center}
\begin{tabular}{l}
|\ifchilddoc|\\
|\providecommand{\version}{draft}|\\
|\||else|\\
|\providecommand{\version}{final}|\\
|\||fi|
\end{tabular}
\end{center}
%
The definition by |\providecommand| makes sure
that previous definitions are not overwritten.
Further statements |\providecommand{\version}{...}|
can thus be added before the above code to override it.

For the main file, one might add a line
(between |\childdocmain| and the above block)
%
\begin{center}
|%\ifchilddoc\||else\providecommand{\version}{draft}\||fi|
\end{center}
%
which can be uncommented to produce a draft version.
Likewise one can add a line to the very top of a child file
(above the |\childdocof{|\textit{main}|}| directive)
%
\begin{center}
|%\providecommand{\version}{final}|
\end{center}
%
which can be uncommented to produce the final version of this child document.

%%%%%%%%%%%%%%%%%%%%%%%%%%%%%%%%%%%%%%%%%%%%%%%%%%%%%%%%%%%%%%%%%%%%%%%%%%%%%%%%
\subsection{Forwarding}
\label{sec:forward}

Different versions of the main or child documents
using compilation flags as described in \secref{sec:flags}
can be (permanently) stored in different files
for convenient compilation, viewing and distribution.
To this end, the package defines a command
to pass on compilation to a different file:

%%%%%%%%%%%%%%%%%%%%%%%%%%%%%%%%%%%%%%%%
\DescribeMacro{\childdocforward}
The command |\childdocforward| redirects processing to
another source file:
%
\begin{center}
\begin{tabular}{l}
|\input{childdoc.def}|\\
|\childdocforward[|\textit{main}|]{|\textit{dest}|}|\\
\end{tabular}
\end{center}
%
The argument \textit{dest} is the destination file
(without extension).
It should be the main file or one of the child files.
Note that further \textsf{childdoc} directives
such as |\childdocof| and |\childdocforward|
in the indicated file will be processed in this form.
The optional argument \textit{main}
passes on directly to the main file \textit{main}
while pretending to compile the child \textit{dest}.
This form behaves as if \textit{dest}
issues |\childdocof{|\textit{main}|}| right away,
and no further \textsf{childdoc} directives will be processed.

%%%%%%%%%%%%%%%%%%%%%%%%%%%%%%%%%%%%%%%%
\DescribeMacro{\...prefix}
In the alternative form |\childdocforwardprefix|,
%
\begin{center}
\begin{tabular}{l}
|\input{childdoc.def}|\\
|\childdocforwardprefix[|\textit{main}|]{|\textit{prefix}|}{|\textit{dest}|}|
\end{tabular}
\end{center}
%
the destination file is determined by a pattern
depending on the current file:
To make this work, the current file must be called
`{\textit{prefix}\hspace{0.2em}\textit{suffix}}'
with \textit{prefix} matching precisely the argument.
Processing is then passed on to the file
`{\textit{dest}\hspace{0.2em}\textit{suffix}}'.
Surely, the same effect is achieved by
directly specifying the
argument `{\textit{dest}\hspace{0.2em}\textit{suffix}}'
in the first form.
However, that requires to set up a different file
for each child. With the alternative form of the command
all these files can have exactly the same content
which simplifies setting them up and maintaining them.

For example, the following file |draft.tex|
with a compilation flag |\version| as described in \secref{sec:flags}
compiles the main document as a draft:
%
\begin{center}
\begin{tabular}{l}
|\def\version{draft}|\\
|\input{childdoc.def}|\\
|\childdocforward{|\textit{main}|}|
\end{tabular}
\end{center}
%
Likewise, the following files |final|\textit{nn}|.tex|
compile the final version of the child document
|child|\textit{nn}|.tex|:
%
\begin{center}
\begin{tabular}{l}
|\def\version{final}|\\
|\input{childdoc.def}|\\
|\childdocforwardprefix{final}{child}|
\end{tabular}
\end{center}
%

Note that when several versions of a main file and/or of each child file
are to be generated, it may be convenient to set up a |Makefile| or
shell script to automatise the process.

%%%%%%%%%%%%%%%%%%%%%%%%%%%%%%%%%%%%%%%%%%%%%%%%%%%%%%%%%%%%%%%%%%%%%%%%%%%%%%%%
\subsection{Command Line Processing}
\label{sec:commandline}

The effect of redirection files can also be achieved by invoking
the \LaTeX{} compiler with a more elaborate command line.
Most conveniently this should be done as part
of a shell script or a |Makefile|.

When using \textsf{childdoc} in the main file, the following
command lines effectively perform a redirection
(note that depending on the shell being used,
backslashes may have to be doubled: `|\|' $\to$ `|\\|'):
%
\begin{center}
|... -jobname "|\textit{target}|" |\\|"|[\textit{flags}]%
|\input{childdoc.def}\childdocforward[|\textit{main}|]{|\textit{dest}|}"|
\end{center}
%
Here \textit{target} is the name of the output file,
\textit{main} is the name of the main file
and \textit{dest} is the name of the main or child file to be processed
(all filenames without extensions).
The optional argument \textit{main} can be omitted
if \textit{main} matches \textit{dest}.
Optionally, compilation \textit{flags} can be defined via |\def| commands.
This command line makes the \TeX{} engine believe
it is compiling the file \textit{target}
whose content is specified as the latter parameter.
The provided code then forwards the processing to
\textit{main} or \textit{dest} as described in \secref{sec:forward}.

%%%%%%%%%%%%%%%%%%%%%%%%%%%%%%%%%%%%%%%%%%%%%%%%%%%%%%%%%%%%%%%%%%%%%%%%%%%%%%%%
\subsection{Include by Input}
\label{sec:input}

Including child documents by |\include| has some restrictions by design.
Most notably, the content of a child document always occupies
its own set of pages; pages cannot be shared between child documents.
Usually, this behaviour makes perfect sense
because each child document contain an essential part of the document.
However, in some situations it may be desirable to compose
a document from a collection of parts
without having mandatory page breaks between then.
For this case, the package
provides a mechanism to include parts
by |\input| which can also be processed individually.
However, by construction this mechanism
requires manual handling of the content to be output.

%%%%%%%%%%%%%%%%%%%%%%%%%%%%%%%%%%%%%%%%
\DescribeMacro{\ifchilddocmanual}
The main file should be prepared as usual, see \secref{sec:include}.
However, the document body must make a distinction
between processing of an individual part and of the main document, e.g.:
%
\begin{center}
\begin{tabular}{l}
|\ifchilddocmanual|\\
|\input{\childdocname}|\\
|\||else|\\
\textit{document body with }|\input{|\textit{part}|}|\\
|\||fi|
\end{tabular}
\end{center}
%
The conditional |\ifchilddocmanual| is true whenever
a part to be included by |\input| is being compiled,
and the name of the part is stored in |\childdocname|.

%%%%%%%%%%%%%%%%%%%%%%%%%%%%%%%%%%%%%%%%
\DescribeMacro{\childdocby}
Each part to be included by |\input| should start with:
%
\begin{center}
\begin{tabular}{l}
|\input{childdoc.def}|\\
|\childdocby{|\textit{main}|}|\\
\end{tabular}
\end{center}
%
The directive |\childdocby| is similar to |\childdocof|
described in \secref{sec:include},
but the subsequent selection of content must be done manually.
To that end, both |\ifchilddoc| and |\ifchilddocmanual|
will be true upon processing of a part,
and the name of the part is stored in |\childdocname|.
Note that |\jobname| will be set to the filename of the current part
so that each part receives an individual |.aux| file
that does not interfere with the |.aux| file(s) of the main document.
This behaviour can be altered by the alternative form
|\childdocby[*]{|\textit{main}|}| (with a non-empty optional argument)
which uses the |.aux| file of the main document
by setting |\jobname| to \textit{main}.

%%%%%%%%%%%%%%%%%%%%%%%%%%%%%%%%%%%%%%%%%%%%%%%%%%%%%%%%%%%%%%%%%%%%%%%%%%%%%%%%
\subsection{Driver Development}
\label{sec:driver}

The \textsf{childdoc} mechanism can also be use for the development
of definition files such as \LaTeX{} styles or classes.
This case differs from the above setup with multiple parts
included by |\include| in that no |\includeonly| should be invoked.
This can be achieved by starting the include file
(before |\ProvidesPackage|) with:
%
\begin{center}
\begin{tabular}{l}
|\input{childdoc.def}|\\
|\childdocforward{|\textit{main}|}|\\
\end{tabular}
\end{center}
%
or alternatively with:
%
\begin{center}
\begin{tabular}{l}
|\input{childdoc.def}|\\
|\childdocby{|\textit{main}|}|\\
\end{tabular}
\end{center}
%
Both forms have slightly different effects as described above.
The main file is prepared as usual, see \secref{sec:include}.

%%%%%%%%%%%%%%%%%%%%%%%%%%%%%%%%%%%%%%%%%%%%%%%%%%%%%%%%%%%%%%%%%%%%%%%%%%%%%%%%
\subsection{Legacy Detection}
\label{sec:detection}

The directive |\childdocmain| in the main file can detect
whether the complete document or merely a child is to be compiled
even without using the directive |\childdocof|.
This method is deprecated because it is less robust
and there is no compelling reason to use it;
it is merely provided for backward compatibility
and it may be removed in future versions.

If the detection mechanism is to be used,
it is mandatory to correctly specify
the filename of the main file as the argument of |\childdocmain|:
%
\begin{center}
\begin{tabular}{l}
|\input{childdoc.def}|\\
|\childdocmain{|\textit{main}|}|\\
\end{tabular}
\end{center}
%
If |\jobname| does not match the argument \textit{main} of |\childdocmain|,
it is assumed that |\jobname| points to the child file to be compiled.
When using |\childdocmain| with the main file specified as argument,
it suffices to start a child file
with just |\input{|\textit{main}|}|
without loading of the package and using |\childdocof|.
If instead all processing is done
with the appropriate \textsf{childdoc} directives,
the argument of \textit{main} of |\childdocmain| can be empty.

An alternative version of the command line processing described
in \secref{sec:commandline} using the detection mechanism reads:
%
\begin{center}
|... -jobname "|\textit{target}|" "|[\textit{flags}]%
[|\def\jobname{|\textit{dest}|}|]|\input{|\textit{main}|}"|
\end{center}

%%%%%%%%%%%%%%%%%%%%%%%%%%%%%%%%%%%%%%%%%%%%%%%%%%%%%%%%%%%%%%%%%%%%%%%%%%%%%%%%
\subsection{Manual Code}
\label{sec:manual}

In case one cannot be certain whether the definitions file |childdoc.def|
is installed on the target \TeX{} distribution
and one prefers not to ship it,
it is conceivable to paste a few relevant commands into the sources.

To that end, drop all statements |\input{childdoc.def}|
and perform the replacements as outlined below.
Instead of |\childdocmain{|\textit{main}|}| add the following code
to the top of the main file:
%
\begin{center}
\begin{tabular}{l}
|\||ifdefined\childdocname\endinput\||fi\newif\ifchilddoc|\\
|\edef\childdocname{\scantokens\expandafter{\jobname\noexpand}}|\\
|\def\childdocmain{|\textit{main}|}\||ifx\childdocmain\childdocname\||else|\\
|\childdoctrue\includeonly{\childdocname}\let\jobname\childdocmain\||fi|\\
\end{tabular}
\end{center}
%
Instead of |\childdocof{|\textit{main}|}| just include the main file
at the top of each child file:
%
\begin{center}
|\input{|\textit{main}|}|
\end{center}
%
A simple redirection |\childdocforward{|\textit{dest}|}| is achieved by:
%
\begin{center}
|\def\jobname{|\textit{dest}|}\input{\jobname}|
\end{center}
%
The redirection with prefix
|\childdocforwardprefix[|\textit{prefix}|]{|\textit{dest}|}|
is accomplished by:
%
\begin{center}
\begin{tabular}{l}
|{\edef\jobname{\scantokens\expandafter{\jobname\noexpand}}|\\
|\def\redirectjob |\textit{prefix}|#1~~~{\gdef\jobname{|\textit{dest}|#1}}|\\
|\expandafter\redirectjob\jobname~~~}\input{\jobname}|
\end{tabular}
\end{center}

In an alternative approach,
child documents can be compiled by a specific command line
without additional code or specific definitions:
%
\begin{center}
|... -jobname "|\textit{target}|" "|[\textit{flags}]%
|\includeonly{|\textit{dest}|}\input{|\textit{main}|}"|
\end{center}
%

%%%%%%%%%%%%%%%%%%%%%%%%%%%%%%%%%%%%%%%%%%%%%%%%%%%%%%%%%%%%%%%%%%%%%%%%%%%%%%%%
%%%%%%%%%%%%%%%%%%%%%%%%%%%%%%%%%%%%%%%%%%%%%%%%%%%%%%%%%%%%%%%%%%%%%%%%%%%%%%%%
\section{Information}

%%%%%%%%%%%%%%%%%%%%%%%%%%%%%%%%%%%%%%%%%%%%%%%%%%%%%%%%%%%%%%%%%%%%%%%%%%%%%%%%
\subsection{Copyright}

Copyright \copyright{} 2017--2018 Niklas Beisert

This work may be distributed and/or modified under the
conditions of the \LaTeX{} Project Public License, either version 1.3
of this license or (at your option) any later version.
The latest version of this license is in
  \url{http://www.latex-project.org/lppl.txt}
and version 1.3 or later is part of all distributions of \LaTeX{}
version 2005/12/01 or later.

This work has the LPPL maintenance status `maintained'.

The Current Maintainer of this work is Niklas Beisert.

This work consists of the files |README.txt|, |childdoc.ins| and |childdoc.dtx|
as well as the derived files |childdoc.def|, |cdocsamp.tex|
with |cdocsch1.tex|, |cdocsch2.tex|, |cdocspt3.tex|, |cdocspt4.tex|,
|cdocsdrf.tex|, |cdocsfn1.tex|, |cdocsfn2.tex|
as well as |childdoc.pdf|.

%%%%%%%%%%%%%%%%%%%%%%%%%%%%%%%%%%%%%%%%%%%%%%%%%%%%%%%%%%%%%%%%%%%%%%%%%%%%%%%%
\subsection{Files and Installation}

The package consists of the files:
%
\begin{center}
\begin{tabular}{ll}
    |README.txt|   & readme file \\
    |childdoc.ins| & installation file \\
    |childdoc.dtx| & source file \\
    |childdoc.def| & definition file \\
    |cdocsamp.tex| & sample main file \\
    |cdocsch1.tex| & sample include file \\
    |cdocsch2.tex| & sample include file \\
    |cdocspt3.tex| & sample part file \\
    |cdocspt4.tex| & sample part file \\
    |cdocsdrf.tex| & sample redirection file \\
    |cdocsfn1.tex| & sample redirection file \\
    |cdocsfn2.tex| & sample redirection file \\
    |childdoc.pdf| & manual
\end{tabular}
\end{center}
%
The distribution consists of the files
|README.txt|, |childdoc.ins| and |childdoc.dtx|.
%
\begin{itemize}
\item
Run (pdf)\LaTeX{} on |childdoc.dtx|
to compile the manual |childdoc.pdf| (this file).
\item
Run \LaTeX{} on |childdoc.ins| to create the definitions file |childdoc.def|
and the sample |cdocsamp.tex| with include files
|cdocsch1.tex|, |cdocsch2.tex|, |cdocspt3.tex|, |cdocspt4.tex|,
|cdocsdrf.tex|, |cdocsfn1.tex|, |cdocsfn2.tex|.
Then copy the file |childdoc.def| to an appropriate directory of your \LaTeX{}
distribution, e.g.\ \textit{texmf-root}|/tex/latex/childdoc|.
\end{itemize}

%%%%%%%%%%%%%%%%%%%%%%%%%%%%%%%%%%%%%%%%%%%%%%%%%%%%%%%%%%%%%%%%%%%%%%%%%%%%%%%%
\subsection{Related CTAN Packages}

There are several other packages which offer a similar functionality:
%
\begin{itemize}
\item
The packages
\href{http://ctan.org/pkg/docmute}{\textsf{docmute}},
\href{http://ctan.org/pkg/includex}{\textsf{includex}} and
\href{http://ctan.org/pkg/standalone}{\textsf{standalone}}
provide commands to include only the document body of
a child file thus allowing both files to be compiled individually.
\item
The packages \href{http://ctan.org/pkg/subdocs}{\textsf{subdocs}}
and \href{http://ctan.org/pkg/subfiles}{\textsf{subfiles}}
provide structures in which the main and child documents can be
encapsulated and allowing them to be compiled individually.
The inclusion mechanism is different from the conventional |\include|.
\item
The package \href{http://ctan.org/pkg/combine}{\textsf{combine}}
is an elaborate solution to combine several documents into one.
\end{itemize}
%
See also the CTAN topic \href{http://ctan.org/topic/subdocs}{\textsf{subdocs}}
for further related packages.
The present package differs from the above solutions in that
a document structure constructed with the conventional |\include| mechanism
just needs two extra commands at the top of every file
such that all constituent files can be compiled individually.

%%%%%%%%%%%%%%%%%%%%%%%%%%%%%%%%%%%%%%%%%%%%%%%%%%%%%%%%%%%%%%%%%%%%%%%%%%%%%%%%
%\subsection{Feature Suggestions}
%
%The following is a list of features which may be useful for future
%versions of this package:
%%
%\begin{itemize}
%\item
%\ldots
%\end{itemize}

%%%%%%%%%%%%%%%%%%%%%%%%%%%%%%%%%%%%%%%%%%%%%%%%%%%%%%%%%%%%%%%%%%%%%%%%%%%%%%%%
\subsection{Revision History}

%%%%%%%%%%%%%%%%%%%%%%%%%%%%%%%%%%%%%%%%
\paragraph{v2.0:} 2018/12/30

\begin{itemize}
\item
immediate forward processing
\item
added |\childdocby| mechanism
\item
manual restructured
\end{itemize}

%%%%%%%%%%%%%%%%%%%%%%%%%%%%%%%%%%%%%%%%
\paragraph{v1.6:} 2018/01/17

\begin{itemize}
\item
application for development of include files
\item
corrections to manual
\end{itemize}

%%%%%%%%%%%%%%%%%%%%%%%%%%%%%%%%%%%%%%%%
\paragraph{v1.5:} 2017/05/21

\begin{itemize}
\item
more complete structuring introduced
\item
|\childdocof| introduced
\item
|\childdoc| renamed to |\childdocmain|
\item
|\childredirect| renamed to |\childdocforward| and |\childdocforwardprefix|
and functionality expanded
\end{itemize}

%%%%%%%%%%%%%%%%%%%%%%%%%%%%%%%%%%%%%%%%
\paragraph{v1.0:} 2017/04/27

\begin{itemize}
\item
manual and install package
\item
first version published on CTAN
\end{itemize}

%%%%%%%%%%%%%%%%%%%%%%%%%%%%%%%%%%%%%%%%
\paragraph{v0.6:} 2017/04/26

\begin{itemize}
\item
redirection mechanism added
\end{itemize}

%%%%%%%%%%%%%%%%%%%%%%%%%%%%%%%%%%%%%%%%
\paragraph{v0.5:} 2017/04/26

\begin{itemize}
\item
functionality in definition file
\end{itemize}


%%%%%%%%%%%%%%%%%%%%%%%%%%%%%%%%%%%%%%%%%%%%%%%%%%%%%%%%%%%%%%%%%%%%%%%%%%%%%%%%
%%%%%%%%%%%%%%%%%%%%%%%%%%%%%%%%%%%%%%%%%%%%%%%%%%%%%%%%%%%%%%%%%%%%%%%%%%%%%%%%
%%%%%%%%%%%%%%%%%%%%%%%%%%%%%%%%%%%%%%%%%%%%%%%%%%%%%%%%%%%%%%%%%%%%%%%%%%%%%%%%
\appendix

\settowidth\MacroIndent{\rmfamily\scriptsize 000\ }

 \DocInput{childdoc.dtx}

\end{document}
%</driver>
% \fi
%
% %%%%%%%%%%%%%%%%%%%%%%%%%%%%%%%%%%%%%%%%%%%%%%%%%%%%%%%%%%%%%%%%%%%%%%%%%%%%%%
% %%%%%%%%%%%%%%%%%%%%%%%%%%%%%%%%%%%%%%%%%%%%%%%%%%%%%%%%%%%%%%%%%%%%%%%%%%%%%%
% \section{Sample}
%\iffalse
%<*samplemain>
%\fi
%
% The following presents a sample document
% with two chapters, two parts, a title page,
% a compile flag as well as three forwarding files to set the flag.
% It consists of eight |.tex| files:
% \begin{center}
% \begin{tabular}{ll}
% |cdocsamp.tex|&main file\\
% |cdocsch1.tex|&include file for chapter 1\\
% |cdocsch2.tex|&include file for chapter 2\\
% |cdocspt3.tex|&include file for part 3\\
% |cdocspt4.tex|&include file for part 4\\
% |cdocsdrf.tex|&forwarding file for main file in draft mode\\
% |cdocsfi1.tex|&forwarding file for final version of chapter 1\\
% |cdocsfi2.tex|&forwarding file for final version of chapter 2\\
% \end{tabular}
% \end{center}
% Each of the eight files can be compiled directly by the \LaTeX{} compiler.
%
% %%%%%%%%%%%%%%%%%%%%%%%%%%%%%%%%%%%%%%
% \paragraph{Main File.}
%
% The main file is called |cdocsamp.tex|.
%
% Load the \textsf{childdoc} definitions and
% declare the filename for the main document:
%    \begin{macrocode}
\input{childdoc.def}
\childdocmain{}
%    \end{macrocode}

% Optional override for |\version| flag:
%    \begin{macrocode}
%%\ifchilddoc\else\providecommand{\version}{draft}\fi
%    \end{macrocode}

% Define the default values for the |\version| flag
% (|final| for the main file and |draft| for childs):
%    \begin{macrocode}
\ifchilddoc
\providecommand{\version}{draft}
\else
\providecommand{\version}{final}
\fi
%    \end{macrocode}

% Load the standard document class:
%    \begin{macrocode}
\documentclass[12pt]{article}
%    \end{macrocode}

% Start the document body:
%    \begin{macrocode}
\begin{document}
%    \end{macrocode}

% Declare a title page.
% Print title, part of document being processed and version flag:
%    \begin{macrocode}
\addtocounter{page}{-1}
\begin{center}
{\LARGE\bfseries{}childdoc example\par}
\vspace{1cm}
\ifchilddoc
\ifchilddocmanual part\else chapter\fi:
`\childdocname' of `\childdocjob'\par
\else
main document: `\childdocjob'\par
\fi
version: \version\par
\end{center}
\newpage
%    \end{macrocode}

% Manually include selected file,
% otherwise process as usual:
%    \begin{macrocode}
\ifchilddocmanual
\section*{part `\childdocname'}
\input{\childdocname}
\else
%    \end{macrocode}

% Include the two chapters:
%    \begin{macrocode}
\include{cdocsch1}
\include{cdocsch2}
%    \end{macrocode}

% Include the two parts unless only chapters should be displayed:
%    \begin{macrocode}
\ifchilddoc\else
\section{part three}
\input{cdocspt3}
\section{part four}
\input{cdocspt4}
\fi
%    \end{macrocode}

% Process as usual until here:
%    \begin{macrocode}
\fi
%    \end{macrocode}

% End of document body:
%    \begin{macrocode}
\end{document}
%    \end{macrocode}
%\iffalse
%</samplemain>
%\fi
%
% %%%%%%%%%%%%%%%%%%%%%%%%%%%%%%%%%%%%%%
% \paragraph{Chapter Include Files.}
%
% The include files are called |cdocsch1.tex| and |cdocsch2.tex|.
%
%\iffalse
%<*samplechap1|samplechap2>
%\fi

% Optional override for |\version| flag:
%    \begin{macrocode}
%%\providecommand{\version}{final}
%    \end{macrocode}

% Include the main document:
%    \begin{macrocode}
\input{childdoc.def}
\childdocof{cdocsamp}
%    \end{macrocode}

%\iffalse
%</samplechap1|samplechap2>
%\fi
%
%\iffalse
%<*samplechap1>
%\fi
% Some text for chapter 1:
%    \begin{macrocode}
\section{one}
some text in chapter one
%    \end{macrocode}

%\iffalse
%</samplechap1>
%\fi
% Some text for chapter 2:
%\iffalse
%<*samplechap2>
%\fi
%    \begin{macrocode}
\section{two}
more text in chapter two
%    \end{macrocode}

%\iffalse
%</samplechap2>
%\fi
%
% %%%%%%%%%%%%%%%%%%%%%%%%%%%%%%%%%%%%%%
% \paragraph{Part Include Files.}
%
% The include files are called |cdocspt3.tex| and |cdocspt4.tex|.
%
%\iffalse
%<*samplepart3|samplepart4>
%\fi

% Optional override for |\version| flag:
%    \begin{macrocode}
%%\providecommand{\version}{final}
%    \end{macrocode}

% Include the main document:
%    \begin{macrocode}
\input{childdoc.def}
\childdocby{cdocsamp}
%    \end{macrocode}

%\iffalse
%</samplepart3|samplepart4>
%\fi
%
%\iffalse
%<*samplepart3>
%\fi
% Some text for part 3:
%    \begin{macrocode}
some text in part three
%    \end{macrocode}

%\iffalse
%</samplepart3>
%\fi
% Some text for part 4:
%\iffalse
%<*samplepart4>
%\fi
%    \begin{macrocode}
more text in part four
%    \end{macrocode}

%\iffalse
%</samplepart4>
%\fi
%
% %%%%%%%%%%%%%%%%%%%%%%%%%%%%%%%%%%%%%%
% \paragraph{Forwarding for a Complete Draft.}
%
% The following forwarding file |cdocsdrf.tex|
% compiles the main document in draft mode:
%\iffalse
%<*sampledraft>
%\fi
%    \begin{macrocode}
\def\version{draft}
\input{childdoc.def}
\childdocforward{cdocsamp}
%    \end{macrocode}

%\iffalse
%</sampledraft>
%\fi
%
% %%%%%%%%%%%%%%%%%%%%%%%%%%%%%%%%%%%%%%
% \paragraph{Forwarding for Final Version of the Chapters.}
%
% The following forwarding files |cdocsfn1.tex| and |cdocsfn2.tex|
% (with identical content)
% compile the final versions of the child documents
% |cdocsch1.tex| and |cdocsch2.tex|, respectively:
%\iffalse
%<*samplefinal>
%\fi
%    \begin{macrocode}
\def\version{final}
\input{childdoc.def}
\childdocforwardprefix[cdocsamp]{cdocsfn}{cdocsch}
%    \end{macrocode}

%\iffalse
%</samplefinal>
%\fi
%
% %%%%%%%%%%%%%%%%%%%%%%%%%%%%%%%%%%%%%%
% \paragraph{Command Line Processing.}
%
% The following three command lines generate the output files
% |cdocscld|, |cdocscl1| and |cdocscl2|
% which should be identical to
% |cdocsdrf|, |cdocsch1| and |cdocsfn2|, respectively:
% \begin{center}
% \begin{tabular}{l}
% |latex -jobname cdocscld \|\\
% |  "\def\version{draft}\input{childdoc.def}\childdocforward{cdocsamp}"|\\
% |latex -jobname cdocscl1 \|\\
% |  "\input{childdoc.def}\childdocforward[cdocsamp]{cdocsch1}"|\\
% |latex -jobname cdocscl2 \|\\
% |  "\def\version{final}\input{childdoc.def}\childdocforward{cdocsch2}"|
% \end{tabular}
% \end{center}
% Note that the trailing backslash on each first line
% merely continues the input to the second line
% (for convenient cut ant paste).
% Furthermore, the command |latex| can be replaced by any
% of its alternative versions such as |pdflatex|.
%
% %%%%%%%%%%%%%%%%%%%%%%%%%%%%%%%%%%%%%%%%%%%%%%%%%%%%%%%%%%%%%%%%%%%%%%%%%%%%%%
% %%%%%%%%%%%%%%%%%%%%%%%%%%%%%%%%%%%%%%%%%%%%%%%%%%%%%%%%%%%%%%%%%%%%%%%%%%%%%%
% \section{Implementation}
%\iffalse
%<*package>
%\fi
%
% This section describes the definitions file |childdoc.def|.

% The definitions cannot be loaded using |\usepackage| or |\RequirePackage|
% which has a mechanism to prevent loading a style file more than once.
% When loading the definitions by means of |\input|
% multiple instances have to be prevented manually:
%\iffalse
%This code needs to be before the `\ProvidesFile' directive
%which is defined at the beginning of this file.
%Therefore it is also placed there and commented out here.
%</package>
%<*discard>
%\fi
%    \begin{macrocode}
\ifdefined\childdocmain\endinput\fi
%    \end{macrocode}
%\iffalse
%</discard>
%<*package>
%\fi
%
% \macro{\ifchilddoc}
% \macro{\ifchilddocmanual}
% The conditional |\ifchilddoc| tells whether a
% child (true) or main (false) document is being compiled.
% The conditional |\ifchilddocmanual| tells whether
% the |\includeonly| mechanism is used (false) or
% the selection of child files must be performed manually (true).
% The definitions initialise to false:
%    \begin{macrocode}
\newif\ifchilddoc
\newif\ifchilddocmanual
%    \end{macrocode}

% \macro{\childdocname}
% \macro{\childdocjob}
% The macro |\childdocname| stores the name of the main document
% to be compiled. The macro |\childdocjob| stores the name of
% the document on which the \LaTeX{} compiler was originally invoked.
% The content of |\jobname| cannot be compared
% to filenames specified in the source due to different catcodes.
% The following code rescans |\jobname|, stores the result
% in |\childdocname| and saves a copy in |\childdocjob|:
%    \begin{macrocode}
\edef\childdocname{\scantokens\expandafter{\jobname\noexpand}}
\let\childdocjob\childdocname
%    \end{macrocode}

% \macro{\childdocdisable}
% The macro |\childdocdisable| prevents the main file
% from being processed more than once.
% At this stage, the main document command |\childdocmain|
% is assumed to be called once again where it should do nothing.
% Any subsequent call to it should prevent
% a secondary processing of the main document
% It overwrites the forwarding commands
% |\childdocof| and |\childdocforward|
% with empty macros to prevent further inclusions of the main document:
%    \begin{macrocode}
\newcommand{\childdocdisable}
{
  \renewcommand{\childdocmain}[1]{\renewcommand{\childdocmain}[1]{\endinput}}
  \renewcommand{\childdocof}[1]{}
  \renewcommand{\childdocby}[2][]{}
  \renewcommand{\childdocforward}[2][]{}
  \renewcommand{\childdocdisable}{}
}
%    \end{macrocode}

% \macro{\childdocmain}
% The macro |\childdocmain| is to be called at the top of the main file
% with nothing or the main filename (without extension) as argument.
% First, it breaks loops.
% If the argument is not empty and does not match |\childdocname|
% (which is set by the first inclusion of |childdoc.def|),
% |\ifchilddoc| is set to true, |\includeonly| is applied to the child file
% and |\jobname| is set to the main file
% (for proper handling of |.aux| files):
%    \begin{macrocode}
\newcommand{\childdocmain}[1]
{
  \childdocdisable\childdocmain{}
  \if?#1?\else
    \begingroup
      \def\childdoctmp{#1}
      \ifx\childdoctmp\childdocname
        \def\childdoctmp{}
      \else
        \def\childdoctmp
        {
          \childdoctrue
          \includeonly{\childdocname}
          \def\childdocjob{#1}
          \def\jobname{#1}
        }
      \fi
      \expandafter
    \endgroup
    \childdoctmp
  \fi
}
%    \end{macrocode}

% \macro{\childdocof}
% The command |\childdocof| redirects
% compilation to the main file |#1|.
%    \begin{macrocode}
\newcommand{\childdocof}[1]
{
  \childdocdisable
  \childdoctrue
  \includeonly{\childdocname}
  \def\jobname{#1}
  \def\childdocjob{#1}
  \input{#1}
}
%    \end{macrocode}

% \macro{\childdocby}
% The command |\childdocby| ....
%    \begin{macrocode}
\newcommand{\childdocby}[2][]
{
  \childdocdisable
  \childdoctrue
  \childdocmanualtrue
  \if?#1?\else
    \def\jobname{#2}
  \fi
  \def\childdocjob{#2}
  \input{#2}
  \endinput
}
%    \end{macrocode}

% \macro{\childdocforward}
% The command |\childdocforward| redirects
% compilation to the main file or
% (if the optional argument is given) a child file.
% Parameters are set as if the main file
% or a child file starting with |\childdocof| was compiled.
% Then compilation is handed over to the main file:
%    \begin{macrocode}
\newcommand{\childdocforward}[2][]
{
  \begingroup
    \if?#1?
      \def\childdoctmp
      {
        \def\childdocname{#2}
        \def\childdocjob{#2}
        \def\jobname{#2}
        \input{#2}
        \endinput
      }
    \else
      \def\childdoctmp
      {
        \childdocdisable
        \def\childdocname{#2}
        \childdoctrue
        \includeonly{#2}
        \def\childdocjob{#1}
        \def\jobname{#1}
        \input{#1}
        \endinput
      }
    \fi
    \expandafter
  \endgroup
  \childdoctmp
}
%    \end{macrocode}

% \macro{\childdocforwardprefix}
% The command |\childdocforwardprefix| redirects
% compilation to the main or a child file by means of a pattern.
% The prefix |#1| in the current filename is replaced by |#2|
% and the suffix of the current filename is kept
% (it is assumed that the filename does not contain the substring `|~~~|'
% which is used as a delimiter).
% Compilation is handed over to the new file by |\childdocforward|:
%    \begin{macrocode}
\newcommand{\childdocforwardprefix}[3][]
{
  \begingroup
    \def\childdocextract #2##1~~~{\def\childdoctmp{\childdocforward[#1]{#3##1}}}
    \expandafter\childdocextract\childdocname~~~
    \expandafter
  \endgroup
  \childdoctmp
}
%    \end{macrocode}

% \macro{\childdoc}
% The deprecated macro |\childdoc| is a legacy version of |\childdocmain|:
%    \begin{macrocode}
\newcommand{\childdoc}{\childdocmain}
%    \end{macrocode}

% \macro{\childdocredirect}
% The deprecated macro |\childdocredirect| is a legacy version
% of |\childdocforward| and |\childdocforwardprefix|:
%    \begin{macrocode}
\newcommand{\childdocredirect}[2][]
{
  \begingroup
    \if?#1?
      \def\childdoctmp{\childdocforward{#2}}
    \else
      \def\childdoctmp{\childdocforwardprefix{#1}{#2}}
    \fi
    \expandafter
  \endgroup
  \childdoctmp
}
%    \end{macrocode}

%\iffalse
%</package>
%\fi
%
\endinput
|\\
|\childdocof{|\textit{main}|}|\\
\end{tabular}
\end{center}
at the top of every child file \textit{child}
which is included by |\include{|\textit{child}|}|
from within the main file
(or at least for those files to be compiled individually).
The argument \textit{main} must be the filename of the main file.

There are a couple of
considerations in setting up the main and child documents:

%%%%%%%%%%%%%%%%%%%%%%%%%%%%%%%%%%%%%%%%
\paragraph{Restrictions.}

Please note the following restrictions:
\begin{itemize}
\item
|\childdocmain| must be called with one argument \textit{main}
to ensure compatibility with earlier version of the package.
It must either be empty (|\childdocmain{}|)
or precisely match the filename of the main file in which it is specified.
See \secref{sec:detection} for further information.
\item
The filename \textit{main} must be specified without the |.tex| extension.
\item
The filename \textit{main} is case sensitive
(even in case-insensitive file systems)
due to internal string comparison.
\item
The argument \textit{main} should be fully expanded, it cannot be a macro.
\item
Subdirectories and special characters should be avoided in filenames.
\item
The command |\childdocmain{|\textit{main}|}| must be followed by a whitespace.
It should not be followed immediately by another command
or by a comment mark `|%|'.
This is because the \TeX{} parser reads the token immediately following
the argument of |\childdocmain| and puts it
at the beginning of every child section;
however, a white\-space is ignored.
\end{itemize}

%%%%%%%%%%%%%%%%%%%%%%%%%%%%%%%%%%%%%%%%
\paragraph{Content of Main File.}

It is advisable to place all content in the child files included by |\include|.
Any output contained in the main file will appear in all child documents
unless suppressed manually;
it cannot be suppressed automatically by the |\includeonly| directive
and thus should normally be avoided.
A method to include some content in the main file
by means of conditional processing is described in \secref{sec:conditional}.

%%%%%%%%%%%%%%%%%%%%%%%%%%%%%%%%%%%%%%%%
\paragraph{Page Numbering.}

When only a part of the document is compiled,
the appropriate numbering of pages
(as well as other status parameters)
is determined from the |.aux| files.
The latter contain information from previous passes.
However this information needs to propagate through
all intermediate child documents.
Therefore the page numbering in child documents may well
be inconsistent until the complete document is compiled at least once.

A useful (if unconventional) way to always ensure a consistent
page numbering is to restart the numbering in each child document
and denote the pages by `\textit{child}|.|\textit{page}'
where \textit{child} represents the chapter/section number of the child file.
This can be achieved by the command
|\numberwithin{page}{|\textit{child}|}|
of the \textsf{amsmath} package
where \textit{child} can be |chapter| or |section|
depending on the chosen structuring.
Alternatively, one can modify the macro |\thepage| appropriately
and reset the counter |page| at the start of each child file.

%%%%%%%%%%%%%%%%%%%%%%%%%%%%%%%%%%%%%%%%%%%%%%%%%%%%%%%%%%%%%%%%%%%%%%%%%%%%%%%%
\subsection{Conditional Processing}
\label{sec:conditional}

The package provides a mechanism to compile different versions
of a document. To customise the versions further some conditional processing
can come in handy to distinguish which version is being compiled.
The package provides two macros to describe the compilation context:

%%%%%%%%%%%%%%%%%%%%%%%%%%%%%%%%%%%%%%%%
\DescribeMacro{\ifchilddoc}
The conditional |\ifchilddoc| distinguishes between the compilation of
child documents and the main document:
%
\begin{center}
|\ifchilddoc |\textit{child-code}| |[|\||else |\textit{main-code}]| \||fi|
\end{center}

%%%%%%%%%%%%%%%%%%%%%%%%%%%%%%%%%%%%%%%%
\DescribeMacro{\childdocname}
\DescribeMacro{\childdocjob}
The macro |\childdocname| contains the filename (without extension)
of the main or child file being processed.
Note that |\childdocjob| will always contain the name of the main file.

%%%%%%%%%%%%%%%%%%%%%%%%%%%%%%%%%%%%%%%%
\paragraph{Title Page.}

Conditional processing can be used to include a title or banner page
in the main document when proper precautions are taken.
Importantly, the code in the main file should ensure that the page counter
(as well as other status parameters which are stored in the |.aux| files)
takes the same value after the conditional processing.
Otherwise the page numbers may take divergent values
depending on which part is compiled.

For example, a title page could be declared by:
%
\begin{center}
\begin{tabular}{l}
|\ifchilddoc\||else|\\
|\addtocounter{page}{-1}|\\
\textit{code for title page}\\
|\newpage|\\
|\||fi|
\end{tabular}
\end{center}
%
A banner page for the child documents can be generated by:
%
\begin{center}
\begin{tabular}{l}
|\ifchilddoc|\\
|\addtocounter{page}{-1}|\\
\textit{code for banner page}\\
|\newpage|\\
|\||fi|
\end{tabular}
\end{center}
%
Here one could write a message such as:
\begin{center}
|This is the part \childdocname{} of \childdocjob{}.|
\end{center}

%%%%%%%%%%%%%%%%%%%%%%%%%%%%%%%%%%%%%%%%%%%%%%%%%%%%%%%%%%%%%%%%%%%%%%%%%%%%%%%%
\subsection{Flags}
\label{sec:flags}

The package makes it easy to generate different versions
of the main or child documents.
To this end compilation flags can be defined
and assigned different default values.
They will be particularly useful in conjunction
with the forwarding mechanism described in \secref{sec:forward}.

For example, it may be useful to have a flag |\version|
which can be set to |draft| or |final|.
The document source will contain some conditional code
depending on the value of |\version|.
Suppose further, the flag should default to |final| for the main file
and to |draft| for child files
which is a natural assignment for editing the document.
This is achieved by placing the following code
in the preamble of the main document
(below the |\childdocmain| directive):
%
\begin{center}
\begin{tabular}{l}
|\ifchilddoc|\\
|\providecommand{\version}{draft}|\\
|\||else|\\
|\providecommand{\version}{final}|\\
|\||fi|
\end{tabular}
\end{center}
%
The definition by |\providecommand| makes sure
that previous definitions are not overwritten.
Further statements |\providecommand{\version}{...}|
can thus be added before the above code to override it.

For the main file, one might add a line
(between |\childdocmain| and the above block)
%
\begin{center}
|%\ifchilddoc\||else\providecommand{\version}{draft}\||fi|
\end{center}
%
which can be uncommented to produce a draft version.
Likewise one can add a line to the very top of a child file
(above the |\childdocof{|\textit{main}|}| directive)
%
\begin{center}
|%\providecommand{\version}{final}|
\end{center}
%
which can be uncommented to produce the final version of this child document.

%%%%%%%%%%%%%%%%%%%%%%%%%%%%%%%%%%%%%%%%%%%%%%%%%%%%%%%%%%%%%%%%%%%%%%%%%%%%%%%%
\subsection{Forwarding}
\label{sec:forward}

Different versions of the main or child documents
using compilation flags as described in \secref{sec:flags}
can be (permanently) stored in different files
for convenient compilation, viewing and distribution.
To this end, the package defines a command
to pass on compilation to a different file:

%%%%%%%%%%%%%%%%%%%%%%%%%%%%%%%%%%%%%%%%
\DescribeMacro{\childdocforward}
The command |\childdocforward| redirects processing to
another source file:
%
\begin{center}
\begin{tabular}{l}
|% \iffalse
%
% childdoc.dtx Copyright (C) 2017-2018 Niklas Beisert
%
% This work may be distributed and/or modified under the
% conditions of the LaTeX Project Public License, either version 1.3
% of this license or (at your option) any later version.
% The latest version of this license is in
%   http://www.latex-project.org/lppl.txt
% and version 1.3 or later is part of all distributions of LaTeX
% version 2005/12/01 or later.
%
% This work has the LPPL maintenance status `maintained'.
%
% The Current Maintainer of this work is Niklas Beisert.
%
% This work consists of the files childdoc.dtx and childdoc.ins
% and the derived files childdoc.def and cdocsamp.tex with
% cdocsch1.tex, cdocsch2.tex, cdocsdrf.tex, cdocsfn1.tex, cdocsfn2.tex.
%
%<package>\ifdefined\childdocmain\endinput\fi
%<package>\ProvidesFile{childdoc.def}[2018/12/30 v2.0 child document driver]
%<samplemain>\ProvidesFile{cdocsamp.tex}[2018/12/30 v2.0 sample for childdoc]
%<*driver>
%\ProvidesFile{childdoc.drv}[2018/12/30 v2.0 childdoc reference manual file]
\PassOptionsToClass{10pt,a4paper}{article}
\documentclass{ltxdoc}

\usepackage[margin=35mm]{geometry}
\usepackage{hyperref}
\usepackage{hyperxmp}
\usepackage[usenames]{color}

\hypersetup{colorlinks=true}
\hypersetup{pdfstartview=FitH}
\hypersetup{pdfpagemode=UseNone}
\hypersetup{pdfsource={}}
\hypersetup{pdflang={en-UK}}
\hypersetup{pdfcopyright={Copyright 2017-2018 Niklas Beisert.
  This work may be distributed and/or modified under the
  conditions of the LaTeX Project Public License, either version 1.3
  of this license or (at your option) any later version.}}
\hypersetup{pdflicenseurl={http://www.latex-project.org/lppl.txt}}
\hypersetup{pdfcontactaddress={ETH Zurich, ITP, HIT K,
  Wolfgang-Pauli-Strasse 27}}
\hypersetup{pdfcontactpostcode={8093}}
\hypersetup{pdfcontactcity={Zurich}}
\hypersetup{pdfcontactcountry={Switzerland}}
\hypersetup{pdfcontactemail={nbeisert@itp.phys.ethz.ch}}
\hypersetup{pdfcontacturl={http://people.phys.ethz.ch/\xmptilde nbeisert/}}

\newcommand{\secref}[1]{\hyperref[#1]{section \ref*{#1}}}

\parskip1ex
\parindent0pt
\let\olditemize\itemize
\def\itemize{\olditemize\parskip0pt}

\begin{document}

\title{The \textsf{childdoc} Package}
\hypersetup{pdftitle={The childdoc Package}}
\author{Niklas Beisert\\[2ex]
  Institut f\"ur Theoretische Physik\\
  Eidgen\"ossische Technische Hochschule Z\"urich\\
  Wolfgang-Pauli-Strasse 27, 8093 Z\"urich, Switzerland\\[1ex]
  \href{mailto:nbeisert@itp.phys.ethz.ch}
  {\texttt{nbeisert@itp.phys.ethz.ch}}}
\hypersetup{pdfauthor={Niklas Beisert}}
\hypersetup{pdfsubject={Manual for the LaTeX2e Package childdoc}}
\date{30 December 2018, \textsf{v2.0}}
\maketitle

\begin{abstract}\noindent
\textsf{childdoc} is a \LaTeXe{} package
that enables the direct compilation
of document sections included by |\include|
to individual files.
\end{abstract}

\begingroup
\parskip0ex
\tableofcontents
\endgroup

%%%%%%%%%%%%%%%%%%%%%%%%%%%%%%%%%%%%%%%%%%%%%%%%%%%%%%%%%%%%%%%%%%%%%%%%%%%%%%%%
%%%%%%%%%%%%%%%%%%%%%%%%%%%%%%%%%%%%%%%%%%%%%%%%%%%%%%%%%%%%%%%%%%%%%%%%%%%%%%%%
\section{Introduction}

\LaTeX{} provides a mechanism to structure a large document (such as a book)
into a main file and several child files (containing the chapters)
using the |\include| command.
This mechanism is beneficial for documents
which span hundreds of pages in order to
make the source file(s) more manageable.
Moreover, compilation can be restricted to
selected child files by means of the |\includeonly| command.
The latter feature can be used to reduce the compilation time while editing
(this was significantly more useful in the earlier days of \LaTeX{})
or to generate a smaller document which is easier to navigate.
Another application of |\includeonly| is to generate
documents consisting of selected parts of the complete document.

However, there are a few drawbacks of the plain |\include| mechanism:
\begin{itemize}
\item
The child files cannot be compiled on their own,
they can only be compiled via the main file.
A naive editing environment
(such as a text editor with an option
to have the current file processed by \LaTeX)
may require one to switch to the main file before compiling;
attempting to compile the child file produces errors.
\item
The main file must be modified (each time)
to adjust the |\includeonly| command
to the present needs. This easily leaves the main file in a messy state.
\item
The generated document will always carry the filename
of the main document. This is inconvenient if
several child files are to be compiled and
to be kept for distribution.
\end{itemize}

The present package provides a simple interface
to make child files individually compilable by \LaTeX{}.
Compiling a child file then has the same effect as compiling
the main file with an |\includeonly| command
to select the appropriate child.
Moreover the generated document will carry the name of the child
rather than the main file.
This resolves all three above issues.

This feature is meant to make the editing of books,
thesis documents and lecture notes somewhat more convenient.
However, the package can also be used efficiently for
composing a series of documents (such as exercise sheets)
which are typically distributed individually.
It then assists the author in generating the individual documents
(potentially in different versions)
as well as a document containing the collected series.
Another application is in developing style files
or other kinds of included material
where compilation of the style file could redirect
to a sample or test file.

%%%%%%%%%%%%%%%%%%%%%%%%%%%%%%%%%%%%%%%%%%%%%%%%%%%%%%%%%%%%%%%%%%%%%%%%%%%%%%%%
%%%%%%%%%%%%%%%%%%%%%%%%%%%%%%%%%%%%%%%%%%%%%%%%%%%%%%%%%%%%%%%%%%%%%%%%%%%%%%%%
\section{Usage}

First of all, the package \textsf{childdoc} is \emph{not} a standard
\LaTeXe{} |.sty| style file! Therefore it needs to be invoked in
a non-standard way.

%%%%%%%%%%%%%%%%%%%%%%%%%%%%%%%%%%%%%%%%%%%%%%%%%%%%%%%%%%%%%%%%%%%%%%%%%%%%%%%%
\subsection{Included Files}
\label{sec:include}

%%%%%%%%%%%%%%%%%%%%%%%%%%%%%%%%%%%%%%%%
\DescribeMacro{\childdocmain}
To use the package, add the commands
\begin{center}
\begin{tabular}{l}
|\input{childdoc.def}|\\
|\childdocmain{}|\\
\end{tabular}
\end{center}
at the very top of the main \LaTeX{} file,
in particular \emph{before} the |\documentclass| statement!
The argument of |\childdocmain| should be left empty
(but it must be present).

%%%%%%%%%%%%%%%%%%%%%%%%%%%%%%%%%%%%%%%%
\DescribeMacro{\childdocof}
Furthermore, add the commands
\begin{center}
\begin{tabular}{l}
|\input{childdoc.def}|\\
|\childdocof{|\textit{main}|}|\\
\end{tabular}
\end{center}
at the top of every child file \textit{child}
which is included by |\include{|\textit{child}|}|
from within the main file
(or at least for those files to be compiled individually).
The argument \textit{main} must be the filename of the main file.

There are a couple of
considerations in setting up the main and child documents:

%%%%%%%%%%%%%%%%%%%%%%%%%%%%%%%%%%%%%%%%
\paragraph{Restrictions.}

Please note the following restrictions:
\begin{itemize}
\item
|\childdocmain| must be called with one argument \textit{main}
to ensure compatibility with earlier version of the package.
It must either be empty (|\childdocmain{}|)
or precisely match the filename of the main file in which it is specified.
See \secref{sec:detection} for further information.
\item
The filename \textit{main} must be specified without the |.tex| extension.
\item
The filename \textit{main} is case sensitive
(even in case-insensitive file systems)
due to internal string comparison.
\item
The argument \textit{main} should be fully expanded, it cannot be a macro.
\item
Subdirectories and special characters should be avoided in filenames.
\item
The command |\childdocmain{|\textit{main}|}| must be followed by a whitespace.
It should not be followed immediately by another command
or by a comment mark `|%|'.
This is because the \TeX{} parser reads the token immediately following
the argument of |\childdocmain| and puts it
at the beginning of every child section;
however, a white\-space is ignored.
\end{itemize}

%%%%%%%%%%%%%%%%%%%%%%%%%%%%%%%%%%%%%%%%
\paragraph{Content of Main File.}

It is advisable to place all content in the child files included by |\include|.
Any output contained in the main file will appear in all child documents
unless suppressed manually;
it cannot be suppressed automatically by the |\includeonly| directive
and thus should normally be avoided.
A method to include some content in the main file
by means of conditional processing is described in \secref{sec:conditional}.

%%%%%%%%%%%%%%%%%%%%%%%%%%%%%%%%%%%%%%%%
\paragraph{Page Numbering.}

When only a part of the document is compiled,
the appropriate numbering of pages
(as well as other status parameters)
is determined from the |.aux| files.
The latter contain information from previous passes.
However this information needs to propagate through
all intermediate child documents.
Therefore the page numbering in child documents may well
be inconsistent until the complete document is compiled at least once.

A useful (if unconventional) way to always ensure a consistent
page numbering is to restart the numbering in each child document
and denote the pages by `\textit{child}|.|\textit{page}'
where \textit{child} represents the chapter/section number of the child file.
This can be achieved by the command
|\numberwithin{page}{|\textit{child}|}|
of the \textsf{amsmath} package
where \textit{child} can be |chapter| or |section|
depending on the chosen structuring.
Alternatively, one can modify the macro |\thepage| appropriately
and reset the counter |page| at the start of each child file.

%%%%%%%%%%%%%%%%%%%%%%%%%%%%%%%%%%%%%%%%%%%%%%%%%%%%%%%%%%%%%%%%%%%%%%%%%%%%%%%%
\subsection{Conditional Processing}
\label{sec:conditional}

The package provides a mechanism to compile different versions
of a document. To customise the versions further some conditional processing
can come in handy to distinguish which version is being compiled.
The package provides two macros to describe the compilation context:

%%%%%%%%%%%%%%%%%%%%%%%%%%%%%%%%%%%%%%%%
\DescribeMacro{\ifchilddoc}
The conditional |\ifchilddoc| distinguishes between the compilation of
child documents and the main document:
%
\begin{center}
|\ifchilddoc |\textit{child-code}| |[|\||else |\textit{main-code}]| \||fi|
\end{center}

%%%%%%%%%%%%%%%%%%%%%%%%%%%%%%%%%%%%%%%%
\DescribeMacro{\childdocname}
\DescribeMacro{\childdocjob}
The macro |\childdocname| contains the filename (without extension)
of the main or child file being processed.
Note that |\childdocjob| will always contain the name of the main file.

%%%%%%%%%%%%%%%%%%%%%%%%%%%%%%%%%%%%%%%%
\paragraph{Title Page.}

Conditional processing can be used to include a title or banner page
in the main document when proper precautions are taken.
Importantly, the code in the main file should ensure that the page counter
(as well as other status parameters which are stored in the |.aux| files)
takes the same value after the conditional processing.
Otherwise the page numbers may take divergent values
depending on which part is compiled.

For example, a title page could be declared by:
%
\begin{center}
\begin{tabular}{l}
|\ifchilddoc\||else|\\
|\addtocounter{page}{-1}|\\
\textit{code for title page}\\
|\newpage|\\
|\||fi|
\end{tabular}
\end{center}
%
A banner page for the child documents can be generated by:
%
\begin{center}
\begin{tabular}{l}
|\ifchilddoc|\\
|\addtocounter{page}{-1}|\\
\textit{code for banner page}\\
|\newpage|\\
|\||fi|
\end{tabular}
\end{center}
%
Here one could write a message such as:
\begin{center}
|This is the part \childdocname{} of \childdocjob{}.|
\end{center}

%%%%%%%%%%%%%%%%%%%%%%%%%%%%%%%%%%%%%%%%%%%%%%%%%%%%%%%%%%%%%%%%%%%%%%%%%%%%%%%%
\subsection{Flags}
\label{sec:flags}

The package makes it easy to generate different versions
of the main or child documents.
To this end compilation flags can be defined
and assigned different default values.
They will be particularly useful in conjunction
with the forwarding mechanism described in \secref{sec:forward}.

For example, it may be useful to have a flag |\version|
which can be set to |draft| or |final|.
The document source will contain some conditional code
depending on the value of |\version|.
Suppose further, the flag should default to |final| for the main file
and to |draft| for child files
which is a natural assignment for editing the document.
This is achieved by placing the following code
in the preamble of the main document
(below the |\childdocmain| directive):
%
\begin{center}
\begin{tabular}{l}
|\ifchilddoc|\\
|\providecommand{\version}{draft}|\\
|\||else|\\
|\providecommand{\version}{final}|\\
|\||fi|
\end{tabular}
\end{center}
%
The definition by |\providecommand| makes sure
that previous definitions are not overwritten.
Further statements |\providecommand{\version}{...}|
can thus be added before the above code to override it.

For the main file, one might add a line
(between |\childdocmain| and the above block)
%
\begin{center}
|%\ifchilddoc\||else\providecommand{\version}{draft}\||fi|
\end{center}
%
which can be uncommented to produce a draft version.
Likewise one can add a line to the very top of a child file
(above the |\childdocof{|\textit{main}|}| directive)
%
\begin{center}
|%\providecommand{\version}{final}|
\end{center}
%
which can be uncommented to produce the final version of this child document.

%%%%%%%%%%%%%%%%%%%%%%%%%%%%%%%%%%%%%%%%%%%%%%%%%%%%%%%%%%%%%%%%%%%%%%%%%%%%%%%%
\subsection{Forwarding}
\label{sec:forward}

Different versions of the main or child documents
using compilation flags as described in \secref{sec:flags}
can be (permanently) stored in different files
for convenient compilation, viewing and distribution.
To this end, the package defines a command
to pass on compilation to a different file:

%%%%%%%%%%%%%%%%%%%%%%%%%%%%%%%%%%%%%%%%
\DescribeMacro{\childdocforward}
The command |\childdocforward| redirects processing to
another source file:
%
\begin{center}
\begin{tabular}{l}
|\input{childdoc.def}|\\
|\childdocforward[|\textit{main}|]{|\textit{dest}|}|\\
\end{tabular}
\end{center}
%
The argument \textit{dest} is the destination file
(without extension).
It should be the main file or one of the child files.
Note that further \textsf{childdoc} directives
such as |\childdocof| and |\childdocforward|
in the indicated file will be processed in this form.
The optional argument \textit{main}
passes on directly to the main file \textit{main}
while pretending to compile the child \textit{dest}.
This form behaves as if \textit{dest}
issues |\childdocof{|\textit{main}|}| right away,
and no further \textsf{childdoc} directives will be processed.

%%%%%%%%%%%%%%%%%%%%%%%%%%%%%%%%%%%%%%%%
\DescribeMacro{\...prefix}
In the alternative form |\childdocforwardprefix|,
%
\begin{center}
\begin{tabular}{l}
|\input{childdoc.def}|\\
|\childdocforwardprefix[|\textit{main}|]{|\textit{prefix}|}{|\textit{dest}|}|
\end{tabular}
\end{center}
%
the destination file is determined by a pattern
depending on the current file:
To make this work, the current file must be called
`{\textit{prefix}\hspace{0.2em}\textit{suffix}}'
with \textit{prefix} matching precisely the argument.
Processing is then passed on to the file
`{\textit{dest}\hspace{0.2em}\textit{suffix}}'.
Surely, the same effect is achieved by
directly specifying the
argument `{\textit{dest}\hspace{0.2em}\textit{suffix}}'
in the first form.
However, that requires to set up a different file
for each child. With the alternative form of the command
all these files can have exactly the same content
which simplifies setting them up and maintaining them.

For example, the following file |draft.tex|
with a compilation flag |\version| as described in \secref{sec:flags}
compiles the main document as a draft:
%
\begin{center}
\begin{tabular}{l}
|\def\version{draft}|\\
|\input{childdoc.def}|\\
|\childdocforward{|\textit{main}|}|
\end{tabular}
\end{center}
%
Likewise, the following files |final|\textit{nn}|.tex|
compile the final version of the child document
|child|\textit{nn}|.tex|:
%
\begin{center}
\begin{tabular}{l}
|\def\version{final}|\\
|\input{childdoc.def}|\\
|\childdocforwardprefix{final}{child}|
\end{tabular}
\end{center}
%

Note that when several versions of a main file and/or of each child file
are to be generated, it may be convenient to set up a |Makefile| or
shell script to automatise the process.

%%%%%%%%%%%%%%%%%%%%%%%%%%%%%%%%%%%%%%%%%%%%%%%%%%%%%%%%%%%%%%%%%%%%%%%%%%%%%%%%
\subsection{Command Line Processing}
\label{sec:commandline}

The effect of redirection files can also be achieved by invoking
the \LaTeX{} compiler with a more elaborate command line.
Most conveniently this should be done as part
of a shell script or a |Makefile|.

When using \textsf{childdoc} in the main file, the following
command lines effectively perform a redirection
(note that depending on the shell being used,
backslashes may have to be doubled: `|\|' $\to$ `|\\|'):
%
\begin{center}
|... -jobname "|\textit{target}|" |\\|"|[\textit{flags}]%
|\input{childdoc.def}\childdocforward[|\textit{main}|]{|\textit{dest}|}"|
\end{center}
%
Here \textit{target} is the name of the output file,
\textit{main} is the name of the main file
and \textit{dest} is the name of the main or child file to be processed
(all filenames without extensions).
The optional argument \textit{main} can be omitted
if \textit{main} matches \textit{dest}.
Optionally, compilation \textit{flags} can be defined via |\def| commands.
This command line makes the \TeX{} engine believe
it is compiling the file \textit{target}
whose content is specified as the latter parameter.
The provided code then forwards the processing to
\textit{main} or \textit{dest} as described in \secref{sec:forward}.

%%%%%%%%%%%%%%%%%%%%%%%%%%%%%%%%%%%%%%%%%%%%%%%%%%%%%%%%%%%%%%%%%%%%%%%%%%%%%%%%
\subsection{Include by Input}
\label{sec:input}

Including child documents by |\include| has some restrictions by design.
Most notably, the content of a child document always occupies
its own set of pages; pages cannot be shared between child documents.
Usually, this behaviour makes perfect sense
because each child document contain an essential part of the document.
However, in some situations it may be desirable to compose
a document from a collection of parts
without having mandatory page breaks between then.
For this case, the package
provides a mechanism to include parts
by |\input| which can also be processed individually.
However, by construction this mechanism
requires manual handling of the content to be output.

%%%%%%%%%%%%%%%%%%%%%%%%%%%%%%%%%%%%%%%%
\DescribeMacro{\ifchilddocmanual}
The main file should be prepared as usual, see \secref{sec:include}.
However, the document body must make a distinction
between processing of an individual part and of the main document, e.g.:
%
\begin{center}
\begin{tabular}{l}
|\ifchilddocmanual|\\
|\input{\childdocname}|\\
|\||else|\\
\textit{document body with }|\input{|\textit{part}|}|\\
|\||fi|
\end{tabular}
\end{center}
%
The conditional |\ifchilddocmanual| is true whenever
a part to be included by |\input| is being compiled,
and the name of the part is stored in |\childdocname|.

%%%%%%%%%%%%%%%%%%%%%%%%%%%%%%%%%%%%%%%%
\DescribeMacro{\childdocby}
Each part to be included by |\input| should start with:
%
\begin{center}
\begin{tabular}{l}
|\input{childdoc.def}|\\
|\childdocby{|\textit{main}|}|\\
\end{tabular}
\end{center}
%
The directive |\childdocby| is similar to |\childdocof|
described in \secref{sec:include},
but the subsequent selection of content must be done manually.
To that end, both |\ifchilddoc| and |\ifchilddocmanual|
will be true upon processing of a part,
and the name of the part is stored in |\childdocname|.
Note that |\jobname| will be set to the filename of the current part
so that each part receives an individual |.aux| file
that does not interfere with the |.aux| file(s) of the main document.
This behaviour can be altered by the alternative form
|\childdocby[*]{|\textit{main}|}| (with a non-empty optional argument)
which uses the |.aux| file of the main document
by setting |\jobname| to \textit{main}.

%%%%%%%%%%%%%%%%%%%%%%%%%%%%%%%%%%%%%%%%%%%%%%%%%%%%%%%%%%%%%%%%%%%%%%%%%%%%%%%%
\subsection{Driver Development}
\label{sec:driver}

The \textsf{childdoc} mechanism can also be use for the development
of definition files such as \LaTeX{} styles or classes.
This case differs from the above setup with multiple parts
included by |\include| in that no |\includeonly| should be invoked.
This can be achieved by starting the include file
(before |\ProvidesPackage|) with:
%
\begin{center}
\begin{tabular}{l}
|\input{childdoc.def}|\\
|\childdocforward{|\textit{main}|}|\\
\end{tabular}
\end{center}
%
or alternatively with:
%
\begin{center}
\begin{tabular}{l}
|\input{childdoc.def}|\\
|\childdocby{|\textit{main}|}|\\
\end{tabular}
\end{center}
%
Both forms have slightly different effects as described above.
The main file is prepared as usual, see \secref{sec:include}.

%%%%%%%%%%%%%%%%%%%%%%%%%%%%%%%%%%%%%%%%%%%%%%%%%%%%%%%%%%%%%%%%%%%%%%%%%%%%%%%%
\subsection{Legacy Detection}
\label{sec:detection}

The directive |\childdocmain| in the main file can detect
whether the complete document or merely a child is to be compiled
even without using the directive |\childdocof|.
This method is deprecated because it is less robust
and there is no compelling reason to use it;
it is merely provided for backward compatibility
and it may be removed in future versions.

If the detection mechanism is to be used,
it is mandatory to correctly specify
the filename of the main file as the argument of |\childdocmain|:
%
\begin{center}
\begin{tabular}{l}
|\input{childdoc.def}|\\
|\childdocmain{|\textit{main}|}|\\
\end{tabular}
\end{center}
%
If |\jobname| does not match the argument \textit{main} of |\childdocmain|,
it is assumed that |\jobname| points to the child file to be compiled.
When using |\childdocmain| with the main file specified as argument,
it suffices to start a child file
with just |\input{|\textit{main}|}|
without loading of the package and using |\childdocof|.
If instead all processing is done
with the appropriate \textsf{childdoc} directives,
the argument of \textit{main} of |\childdocmain| can be empty.

An alternative version of the command line processing described
in \secref{sec:commandline} using the detection mechanism reads:
%
\begin{center}
|... -jobname "|\textit{target}|" "|[\textit{flags}]%
[|\def\jobname{|\textit{dest}|}|]|\input{|\textit{main}|}"|
\end{center}

%%%%%%%%%%%%%%%%%%%%%%%%%%%%%%%%%%%%%%%%%%%%%%%%%%%%%%%%%%%%%%%%%%%%%%%%%%%%%%%%
\subsection{Manual Code}
\label{sec:manual}

In case one cannot be certain whether the definitions file |childdoc.def|
is installed on the target \TeX{} distribution
and one prefers not to ship it,
it is conceivable to paste a few relevant commands into the sources.

To that end, drop all statements |\input{childdoc.def}|
and perform the replacements as outlined below.
Instead of |\childdocmain{|\textit{main}|}| add the following code
to the top of the main file:
%
\begin{center}
\begin{tabular}{l}
|\||ifdefined\childdocname\endinput\||fi\newif\ifchilddoc|\\
|\edef\childdocname{\scantokens\expandafter{\jobname\noexpand}}|\\
|\def\childdocmain{|\textit{main}|}\||ifx\childdocmain\childdocname\||else|\\
|\childdoctrue\includeonly{\childdocname}\let\jobname\childdocmain\||fi|\\
\end{tabular}
\end{center}
%
Instead of |\childdocof{|\textit{main}|}| just include the main file
at the top of each child file:
%
\begin{center}
|\input{|\textit{main}|}|
\end{center}
%
A simple redirection |\childdocforward{|\textit{dest}|}| is achieved by:
%
\begin{center}
|\def\jobname{|\textit{dest}|}\input{\jobname}|
\end{center}
%
The redirection with prefix
|\childdocforwardprefix[|\textit{prefix}|]{|\textit{dest}|}|
is accomplished by:
%
\begin{center}
\begin{tabular}{l}
|{\edef\jobname{\scantokens\expandafter{\jobname\noexpand}}|\\
|\def\redirectjob |\textit{prefix}|#1~~~{\gdef\jobname{|\textit{dest}|#1}}|\\
|\expandafter\redirectjob\jobname~~~}\input{\jobname}|
\end{tabular}
\end{center}

In an alternative approach,
child documents can be compiled by a specific command line
without additional code or specific definitions:
%
\begin{center}
|... -jobname "|\textit{target}|" "|[\textit{flags}]%
|\includeonly{|\textit{dest}|}\input{|\textit{main}|}"|
\end{center}
%

%%%%%%%%%%%%%%%%%%%%%%%%%%%%%%%%%%%%%%%%%%%%%%%%%%%%%%%%%%%%%%%%%%%%%%%%%%%%%%%%
%%%%%%%%%%%%%%%%%%%%%%%%%%%%%%%%%%%%%%%%%%%%%%%%%%%%%%%%%%%%%%%%%%%%%%%%%%%%%%%%
\section{Information}

%%%%%%%%%%%%%%%%%%%%%%%%%%%%%%%%%%%%%%%%%%%%%%%%%%%%%%%%%%%%%%%%%%%%%%%%%%%%%%%%
\subsection{Copyright}

Copyright \copyright{} 2017--2018 Niklas Beisert

This work may be distributed and/or modified under the
conditions of the \LaTeX{} Project Public License, either version 1.3
of this license or (at your option) any later version.
The latest version of this license is in
  \url{http://www.latex-project.org/lppl.txt}
and version 1.3 or later is part of all distributions of \LaTeX{}
version 2005/12/01 or later.

This work has the LPPL maintenance status `maintained'.

The Current Maintainer of this work is Niklas Beisert.

This work consists of the files |README.txt|, |childdoc.ins| and |childdoc.dtx|
as well as the derived files |childdoc.def|, |cdocsamp.tex|
with |cdocsch1.tex|, |cdocsch2.tex|, |cdocspt3.tex|, |cdocspt4.tex|,
|cdocsdrf.tex|, |cdocsfn1.tex|, |cdocsfn2.tex|
as well as |childdoc.pdf|.

%%%%%%%%%%%%%%%%%%%%%%%%%%%%%%%%%%%%%%%%%%%%%%%%%%%%%%%%%%%%%%%%%%%%%%%%%%%%%%%%
\subsection{Files and Installation}

The package consists of the files:
%
\begin{center}
\begin{tabular}{ll}
    |README.txt|   & readme file \\
    |childdoc.ins| & installation file \\
    |childdoc.dtx| & source file \\
    |childdoc.def| & definition file \\
    |cdocsamp.tex| & sample main file \\
    |cdocsch1.tex| & sample include file \\
    |cdocsch2.tex| & sample include file \\
    |cdocspt3.tex| & sample part file \\
    |cdocspt4.tex| & sample part file \\
    |cdocsdrf.tex| & sample redirection file \\
    |cdocsfn1.tex| & sample redirection file \\
    |cdocsfn2.tex| & sample redirection file \\
    |childdoc.pdf| & manual
\end{tabular}
\end{center}
%
The distribution consists of the files
|README.txt|, |childdoc.ins| and |childdoc.dtx|.
%
\begin{itemize}
\item
Run (pdf)\LaTeX{} on |childdoc.dtx|
to compile the manual |childdoc.pdf| (this file).
\item
Run \LaTeX{} on |childdoc.ins| to create the definitions file |childdoc.def|
and the sample |cdocsamp.tex| with include files
|cdocsch1.tex|, |cdocsch2.tex|, |cdocspt3.tex|, |cdocspt4.tex|,
|cdocsdrf.tex|, |cdocsfn1.tex|, |cdocsfn2.tex|.
Then copy the file |childdoc.def| to an appropriate directory of your \LaTeX{}
distribution, e.g.\ \textit{texmf-root}|/tex/latex/childdoc|.
\end{itemize}

%%%%%%%%%%%%%%%%%%%%%%%%%%%%%%%%%%%%%%%%%%%%%%%%%%%%%%%%%%%%%%%%%%%%%%%%%%%%%%%%
\subsection{Related CTAN Packages}

There are several other packages which offer a similar functionality:
%
\begin{itemize}
\item
The packages
\href{http://ctan.org/pkg/docmute}{\textsf{docmute}},
\href{http://ctan.org/pkg/includex}{\textsf{includex}} and
\href{http://ctan.org/pkg/standalone}{\textsf{standalone}}
provide commands to include only the document body of
a child file thus allowing both files to be compiled individually.
\item
The packages \href{http://ctan.org/pkg/subdocs}{\textsf{subdocs}}
and \href{http://ctan.org/pkg/subfiles}{\textsf{subfiles}}
provide structures in which the main and child documents can be
encapsulated and allowing them to be compiled individually.
The inclusion mechanism is different from the conventional |\include|.
\item
The package \href{http://ctan.org/pkg/combine}{\textsf{combine}}
is an elaborate solution to combine several documents into one.
\end{itemize}
%
See also the CTAN topic \href{http://ctan.org/topic/subdocs}{\textsf{subdocs}}
for further related packages.
The present package differs from the above solutions in that
a document structure constructed with the conventional |\include| mechanism
just needs two extra commands at the top of every file
such that all constituent files can be compiled individually.

%%%%%%%%%%%%%%%%%%%%%%%%%%%%%%%%%%%%%%%%%%%%%%%%%%%%%%%%%%%%%%%%%%%%%%%%%%%%%%%%
%\subsection{Feature Suggestions}
%
%The following is a list of features which may be useful for future
%versions of this package:
%%
%\begin{itemize}
%\item
%\ldots
%\end{itemize}

%%%%%%%%%%%%%%%%%%%%%%%%%%%%%%%%%%%%%%%%%%%%%%%%%%%%%%%%%%%%%%%%%%%%%%%%%%%%%%%%
\subsection{Revision History}

%%%%%%%%%%%%%%%%%%%%%%%%%%%%%%%%%%%%%%%%
\paragraph{v2.0:} 2018/12/30

\begin{itemize}
\item
immediate forward processing
\item
added |\childdocby| mechanism
\item
manual restructured
\end{itemize}

%%%%%%%%%%%%%%%%%%%%%%%%%%%%%%%%%%%%%%%%
\paragraph{v1.6:} 2018/01/17

\begin{itemize}
\item
application for development of include files
\item
corrections to manual
\end{itemize}

%%%%%%%%%%%%%%%%%%%%%%%%%%%%%%%%%%%%%%%%
\paragraph{v1.5:} 2017/05/21

\begin{itemize}
\item
more complete structuring introduced
\item
|\childdocof| introduced
\item
|\childdoc| renamed to |\childdocmain|
\item
|\childredirect| renamed to |\childdocforward| and |\childdocforwardprefix|
and functionality expanded
\end{itemize}

%%%%%%%%%%%%%%%%%%%%%%%%%%%%%%%%%%%%%%%%
\paragraph{v1.0:} 2017/04/27

\begin{itemize}
\item
manual and install package
\item
first version published on CTAN
\end{itemize}

%%%%%%%%%%%%%%%%%%%%%%%%%%%%%%%%%%%%%%%%
\paragraph{v0.6:} 2017/04/26

\begin{itemize}
\item
redirection mechanism added
\end{itemize}

%%%%%%%%%%%%%%%%%%%%%%%%%%%%%%%%%%%%%%%%
\paragraph{v0.5:} 2017/04/26

\begin{itemize}
\item
functionality in definition file
\end{itemize}


%%%%%%%%%%%%%%%%%%%%%%%%%%%%%%%%%%%%%%%%%%%%%%%%%%%%%%%%%%%%%%%%%%%%%%%%%%%%%%%%
%%%%%%%%%%%%%%%%%%%%%%%%%%%%%%%%%%%%%%%%%%%%%%%%%%%%%%%%%%%%%%%%%%%%%%%%%%%%%%%%
%%%%%%%%%%%%%%%%%%%%%%%%%%%%%%%%%%%%%%%%%%%%%%%%%%%%%%%%%%%%%%%%%%%%%%%%%%%%%%%%
\appendix

\settowidth\MacroIndent{\rmfamily\scriptsize 000\ }

 \DocInput{childdoc.dtx}

\end{document}
%</driver>
% \fi
%
% %%%%%%%%%%%%%%%%%%%%%%%%%%%%%%%%%%%%%%%%%%%%%%%%%%%%%%%%%%%%%%%%%%%%%%%%%%%%%%
% %%%%%%%%%%%%%%%%%%%%%%%%%%%%%%%%%%%%%%%%%%%%%%%%%%%%%%%%%%%%%%%%%%%%%%%%%%%%%%
% \section{Sample}
%\iffalse
%<*samplemain>
%\fi
%
% The following presents a sample document
% with two chapters, two parts, a title page,
% a compile flag as well as three forwarding files to set the flag.
% It consists of eight |.tex| files:
% \begin{center}
% \begin{tabular}{ll}
% |cdocsamp.tex|&main file\\
% |cdocsch1.tex|&include file for chapter 1\\
% |cdocsch2.tex|&include file for chapter 2\\
% |cdocspt3.tex|&include file for part 3\\
% |cdocspt4.tex|&include file for part 4\\
% |cdocsdrf.tex|&forwarding file for main file in draft mode\\
% |cdocsfi1.tex|&forwarding file for final version of chapter 1\\
% |cdocsfi2.tex|&forwarding file for final version of chapter 2\\
% \end{tabular}
% \end{center}
% Each of the eight files can be compiled directly by the \LaTeX{} compiler.
%
% %%%%%%%%%%%%%%%%%%%%%%%%%%%%%%%%%%%%%%
% \paragraph{Main File.}
%
% The main file is called |cdocsamp.tex|.
%
% Load the \textsf{childdoc} definitions and
% declare the filename for the main document:
%    \begin{macrocode}
\input{childdoc.def}
\childdocmain{}
%    \end{macrocode}

% Optional override for |\version| flag:
%    \begin{macrocode}
%%\ifchilddoc\else\providecommand{\version}{draft}\fi
%    \end{macrocode}

% Define the default values for the |\version| flag
% (|final| for the main file and |draft| for childs):
%    \begin{macrocode}
\ifchilddoc
\providecommand{\version}{draft}
\else
\providecommand{\version}{final}
\fi
%    \end{macrocode}

% Load the standard document class:
%    \begin{macrocode}
\documentclass[12pt]{article}
%    \end{macrocode}

% Start the document body:
%    \begin{macrocode}
\begin{document}
%    \end{macrocode}

% Declare a title page.
% Print title, part of document being processed and version flag:
%    \begin{macrocode}
\addtocounter{page}{-1}
\begin{center}
{\LARGE\bfseries{}childdoc example\par}
\vspace{1cm}
\ifchilddoc
\ifchilddocmanual part\else chapter\fi:
`\childdocname' of `\childdocjob'\par
\else
main document: `\childdocjob'\par
\fi
version: \version\par
\end{center}
\newpage
%    \end{macrocode}

% Manually include selected file,
% otherwise process as usual:
%    \begin{macrocode}
\ifchilddocmanual
\section*{part `\childdocname'}
\input{\childdocname}
\else
%    \end{macrocode}

% Include the two chapters:
%    \begin{macrocode}
\include{cdocsch1}
\include{cdocsch2}
%    \end{macrocode}

% Include the two parts unless only chapters should be displayed:
%    \begin{macrocode}
\ifchilddoc\else
\section{part three}
\input{cdocspt3}
\section{part four}
\input{cdocspt4}
\fi
%    \end{macrocode}

% Process as usual until here:
%    \begin{macrocode}
\fi
%    \end{macrocode}

% End of document body:
%    \begin{macrocode}
\end{document}
%    \end{macrocode}
%\iffalse
%</samplemain>
%\fi
%
% %%%%%%%%%%%%%%%%%%%%%%%%%%%%%%%%%%%%%%
% \paragraph{Chapter Include Files.}
%
% The include files are called |cdocsch1.tex| and |cdocsch2.tex|.
%
%\iffalse
%<*samplechap1|samplechap2>
%\fi

% Optional override for |\version| flag:
%    \begin{macrocode}
%%\providecommand{\version}{final}
%    \end{macrocode}

% Include the main document:
%    \begin{macrocode}
\input{childdoc.def}
\childdocof{cdocsamp}
%    \end{macrocode}

%\iffalse
%</samplechap1|samplechap2>
%\fi
%
%\iffalse
%<*samplechap1>
%\fi
% Some text for chapter 1:
%    \begin{macrocode}
\section{one}
some text in chapter one
%    \end{macrocode}

%\iffalse
%</samplechap1>
%\fi
% Some text for chapter 2:
%\iffalse
%<*samplechap2>
%\fi
%    \begin{macrocode}
\section{two}
more text in chapter two
%    \end{macrocode}

%\iffalse
%</samplechap2>
%\fi
%
% %%%%%%%%%%%%%%%%%%%%%%%%%%%%%%%%%%%%%%
% \paragraph{Part Include Files.}
%
% The include files are called |cdocspt3.tex| and |cdocspt4.tex|.
%
%\iffalse
%<*samplepart3|samplepart4>
%\fi

% Optional override for |\version| flag:
%    \begin{macrocode}
%%\providecommand{\version}{final}
%    \end{macrocode}

% Include the main document:
%    \begin{macrocode}
\input{childdoc.def}
\childdocby{cdocsamp}
%    \end{macrocode}

%\iffalse
%</samplepart3|samplepart4>
%\fi
%
%\iffalse
%<*samplepart3>
%\fi
% Some text for part 3:
%    \begin{macrocode}
some text in part three
%    \end{macrocode}

%\iffalse
%</samplepart3>
%\fi
% Some text for part 4:
%\iffalse
%<*samplepart4>
%\fi
%    \begin{macrocode}
more text in part four
%    \end{macrocode}

%\iffalse
%</samplepart4>
%\fi
%
% %%%%%%%%%%%%%%%%%%%%%%%%%%%%%%%%%%%%%%
% \paragraph{Forwarding for a Complete Draft.}
%
% The following forwarding file |cdocsdrf.tex|
% compiles the main document in draft mode:
%\iffalse
%<*sampledraft>
%\fi
%    \begin{macrocode}
\def\version{draft}
\input{childdoc.def}
\childdocforward{cdocsamp}
%    \end{macrocode}

%\iffalse
%</sampledraft>
%\fi
%
% %%%%%%%%%%%%%%%%%%%%%%%%%%%%%%%%%%%%%%
% \paragraph{Forwarding for Final Version of the Chapters.}
%
% The following forwarding files |cdocsfn1.tex| and |cdocsfn2.tex|
% (with identical content)
% compile the final versions of the child documents
% |cdocsch1.tex| and |cdocsch2.tex|, respectively:
%\iffalse
%<*samplefinal>
%\fi
%    \begin{macrocode}
\def\version{final}
\input{childdoc.def}
\childdocforwardprefix[cdocsamp]{cdocsfn}{cdocsch}
%    \end{macrocode}

%\iffalse
%</samplefinal>
%\fi
%
% %%%%%%%%%%%%%%%%%%%%%%%%%%%%%%%%%%%%%%
% \paragraph{Command Line Processing.}
%
% The following three command lines generate the output files
% |cdocscld|, |cdocscl1| and |cdocscl2|
% which should be identical to
% |cdocsdrf|, |cdocsch1| and |cdocsfn2|, respectively:
% \begin{center}
% \begin{tabular}{l}
% |latex -jobname cdocscld \|\\
% |  "\def\version{draft}\input{childdoc.def}\childdocforward{cdocsamp}"|\\
% |latex -jobname cdocscl1 \|\\
% |  "\input{childdoc.def}\childdocforward[cdocsamp]{cdocsch1}"|\\
% |latex -jobname cdocscl2 \|\\
% |  "\def\version{final}\input{childdoc.def}\childdocforward{cdocsch2}"|
% \end{tabular}
% \end{center}
% Note that the trailing backslash on each first line
% merely continues the input to the second line
% (for convenient cut ant paste).
% Furthermore, the command |latex| can be replaced by any
% of its alternative versions such as |pdflatex|.
%
% %%%%%%%%%%%%%%%%%%%%%%%%%%%%%%%%%%%%%%%%%%%%%%%%%%%%%%%%%%%%%%%%%%%%%%%%%%%%%%
% %%%%%%%%%%%%%%%%%%%%%%%%%%%%%%%%%%%%%%%%%%%%%%%%%%%%%%%%%%%%%%%%%%%%%%%%%%%%%%
% \section{Implementation}
%\iffalse
%<*package>
%\fi
%
% This section describes the definitions file |childdoc.def|.

% The definitions cannot be loaded using |\usepackage| or |\RequirePackage|
% which has a mechanism to prevent loading a style file more than once.
% When loading the definitions by means of |\input|
% multiple instances have to be prevented manually:
%\iffalse
%This code needs to be before the `\ProvidesFile' directive
%which is defined at the beginning of this file.
%Therefore it is also placed there and commented out here.
%</package>
%<*discard>
%\fi
%    \begin{macrocode}
\ifdefined\childdocmain\endinput\fi
%    \end{macrocode}
%\iffalse
%</discard>
%<*package>
%\fi
%
% \macro{\ifchilddoc}
% \macro{\ifchilddocmanual}
% The conditional |\ifchilddoc| tells whether a
% child (true) or main (false) document is being compiled.
% The conditional |\ifchilddocmanual| tells whether
% the |\includeonly| mechanism is used (false) or
% the selection of child files must be performed manually (true).
% The definitions initialise to false:
%    \begin{macrocode}
\newif\ifchilddoc
\newif\ifchilddocmanual
%    \end{macrocode}

% \macro{\childdocname}
% \macro{\childdocjob}
% The macro |\childdocname| stores the name of the main document
% to be compiled. The macro |\childdocjob| stores the name of
% the document on which the \LaTeX{} compiler was originally invoked.
% The content of |\jobname| cannot be compared
% to filenames specified in the source due to different catcodes.
% The following code rescans |\jobname|, stores the result
% in |\childdocname| and saves a copy in |\childdocjob|:
%    \begin{macrocode}
\edef\childdocname{\scantokens\expandafter{\jobname\noexpand}}
\let\childdocjob\childdocname
%    \end{macrocode}

% \macro{\childdocdisable}
% The macro |\childdocdisable| prevents the main file
% from being processed more than once.
% At this stage, the main document command |\childdocmain|
% is assumed to be called once again where it should do nothing.
% Any subsequent call to it should prevent
% a secondary processing of the main document
% It overwrites the forwarding commands
% |\childdocof| and |\childdocforward|
% with empty macros to prevent further inclusions of the main document:
%    \begin{macrocode}
\newcommand{\childdocdisable}
{
  \renewcommand{\childdocmain}[1]{\renewcommand{\childdocmain}[1]{\endinput}}
  \renewcommand{\childdocof}[1]{}
  \renewcommand{\childdocby}[2][]{}
  \renewcommand{\childdocforward}[2][]{}
  \renewcommand{\childdocdisable}{}
}
%    \end{macrocode}

% \macro{\childdocmain}
% The macro |\childdocmain| is to be called at the top of the main file
% with nothing or the main filename (without extension) as argument.
% First, it breaks loops.
% If the argument is not empty and does not match |\childdocname|
% (which is set by the first inclusion of |childdoc.def|),
% |\ifchilddoc| is set to true, |\includeonly| is applied to the child file
% and |\jobname| is set to the main file
% (for proper handling of |.aux| files):
%    \begin{macrocode}
\newcommand{\childdocmain}[1]
{
  \childdocdisable\childdocmain{}
  \if?#1?\else
    \begingroup
      \def\childdoctmp{#1}
      \ifx\childdoctmp\childdocname
        \def\childdoctmp{}
      \else
        \def\childdoctmp
        {
          \childdoctrue
          \includeonly{\childdocname}
          \def\childdocjob{#1}
          \def\jobname{#1}
        }
      \fi
      \expandafter
    \endgroup
    \childdoctmp
  \fi
}
%    \end{macrocode}

% \macro{\childdocof}
% The command |\childdocof| redirects
% compilation to the main file |#1|.
%    \begin{macrocode}
\newcommand{\childdocof}[1]
{
  \childdocdisable
  \childdoctrue
  \includeonly{\childdocname}
  \def\jobname{#1}
  \def\childdocjob{#1}
  \input{#1}
}
%    \end{macrocode}

% \macro{\childdocby}
% The command |\childdocby| ....
%    \begin{macrocode}
\newcommand{\childdocby}[2][]
{
  \childdocdisable
  \childdoctrue
  \childdocmanualtrue
  \if?#1?\else
    \def\jobname{#2}
  \fi
  \def\childdocjob{#2}
  \input{#2}
  \endinput
}
%    \end{macrocode}

% \macro{\childdocforward}
% The command |\childdocforward| redirects
% compilation to the main file or
% (if the optional argument is given) a child file.
% Parameters are set as if the main file
% or a child file starting with |\childdocof| was compiled.
% Then compilation is handed over to the main file:
%    \begin{macrocode}
\newcommand{\childdocforward}[2][]
{
  \begingroup
    \if?#1?
      \def\childdoctmp
      {
        \def\childdocname{#2}
        \def\childdocjob{#2}
        \def\jobname{#2}
        \input{#2}
        \endinput
      }
    \else
      \def\childdoctmp
      {
        \childdocdisable
        \def\childdocname{#2}
        \childdoctrue
        \includeonly{#2}
        \def\childdocjob{#1}
        \def\jobname{#1}
        \input{#1}
        \endinput
      }
    \fi
    \expandafter
  \endgroup
  \childdoctmp
}
%    \end{macrocode}

% \macro{\childdocforwardprefix}
% The command |\childdocforwardprefix| redirects
% compilation to the main or a child file by means of a pattern.
% The prefix |#1| in the current filename is replaced by |#2|
% and the suffix of the current filename is kept
% (it is assumed that the filename does not contain the substring `|~~~|'
% which is used as a delimiter).
% Compilation is handed over to the new file by |\childdocforward|:
%    \begin{macrocode}
\newcommand{\childdocforwardprefix}[3][]
{
  \begingroup
    \def\childdocextract #2##1~~~{\def\childdoctmp{\childdocforward[#1]{#3##1}}}
    \expandafter\childdocextract\childdocname~~~
    \expandafter
  \endgroup
  \childdoctmp
}
%    \end{macrocode}

% \macro{\childdoc}
% The deprecated macro |\childdoc| is a legacy version of |\childdocmain|:
%    \begin{macrocode}
\newcommand{\childdoc}{\childdocmain}
%    \end{macrocode}

% \macro{\childdocredirect}
% The deprecated macro |\childdocredirect| is a legacy version
% of |\childdocforward| and |\childdocforwardprefix|:
%    \begin{macrocode}
\newcommand{\childdocredirect}[2][]
{
  \begingroup
    \if?#1?
      \def\childdoctmp{\childdocforward{#2}}
    \else
      \def\childdoctmp{\childdocforwardprefix{#1}{#2}}
    \fi
    \expandafter
  \endgroup
  \childdoctmp
}
%    \end{macrocode}

%\iffalse
%</package>
%\fi
%
\endinput
|\\
|\childdocforward[|\textit{main}|]{|\textit{dest}|}|\\
\end{tabular}
\end{center}
%
The argument \textit{dest} is the destination file
(without extension).
It should be the main file or one of the child files.
Note that further \textsf{childdoc} directives
such as |\childdocof| and |\childdocforward|
in the indicated file will be processed in this form.
The optional argument \textit{main}
passes on directly to the main file \textit{main}
while pretending to compile the child \textit{dest}.
This form behaves as if \textit{dest}
issues |\childdocof{|\textit{main}|}| right away,
and no further \textsf{childdoc} directives will be processed.

%%%%%%%%%%%%%%%%%%%%%%%%%%%%%%%%%%%%%%%%
\DescribeMacro{\...prefix}
In the alternative form |\childdocforwardprefix|,
%
\begin{center}
\begin{tabular}{l}
|% \iffalse
%
% childdoc.dtx Copyright (C) 2017-2018 Niklas Beisert
%
% This work may be distributed and/or modified under the
% conditions of the LaTeX Project Public License, either version 1.3
% of this license or (at your option) any later version.
% The latest version of this license is in
%   http://www.latex-project.org/lppl.txt
% and version 1.3 or later is part of all distributions of LaTeX
% version 2005/12/01 or later.
%
% This work has the LPPL maintenance status `maintained'.
%
% The Current Maintainer of this work is Niklas Beisert.
%
% This work consists of the files childdoc.dtx and childdoc.ins
% and the derived files childdoc.def and cdocsamp.tex with
% cdocsch1.tex, cdocsch2.tex, cdocsdrf.tex, cdocsfn1.tex, cdocsfn2.tex.
%
%<package>\ifdefined\childdocmain\endinput\fi
%<package>\ProvidesFile{childdoc.def}[2018/12/30 v2.0 child document driver]
%<samplemain>\ProvidesFile{cdocsamp.tex}[2018/12/30 v2.0 sample for childdoc]
%<*driver>
%\ProvidesFile{childdoc.drv}[2018/12/30 v2.0 childdoc reference manual file]
\PassOptionsToClass{10pt,a4paper}{article}
\documentclass{ltxdoc}

\usepackage[margin=35mm]{geometry}
\usepackage{hyperref}
\usepackage{hyperxmp}
\usepackage[usenames]{color}

\hypersetup{colorlinks=true}
\hypersetup{pdfstartview=FitH}
\hypersetup{pdfpagemode=UseNone}
\hypersetup{pdfsource={}}
\hypersetup{pdflang={en-UK}}
\hypersetup{pdfcopyright={Copyright 2017-2018 Niklas Beisert.
  This work may be distributed and/or modified under the
  conditions of the LaTeX Project Public License, either version 1.3
  of this license or (at your option) any later version.}}
\hypersetup{pdflicenseurl={http://www.latex-project.org/lppl.txt}}
\hypersetup{pdfcontactaddress={ETH Zurich, ITP, HIT K,
  Wolfgang-Pauli-Strasse 27}}
\hypersetup{pdfcontactpostcode={8093}}
\hypersetup{pdfcontactcity={Zurich}}
\hypersetup{pdfcontactcountry={Switzerland}}
\hypersetup{pdfcontactemail={nbeisert@itp.phys.ethz.ch}}
\hypersetup{pdfcontacturl={http://people.phys.ethz.ch/\xmptilde nbeisert/}}

\newcommand{\secref}[1]{\hyperref[#1]{section \ref*{#1}}}

\parskip1ex
\parindent0pt
\let\olditemize\itemize
\def\itemize{\olditemize\parskip0pt}

\begin{document}

\title{The \textsf{childdoc} Package}
\hypersetup{pdftitle={The childdoc Package}}
\author{Niklas Beisert\\[2ex]
  Institut f\"ur Theoretische Physik\\
  Eidgen\"ossische Technische Hochschule Z\"urich\\
  Wolfgang-Pauli-Strasse 27, 8093 Z\"urich, Switzerland\\[1ex]
  \href{mailto:nbeisert@itp.phys.ethz.ch}
  {\texttt{nbeisert@itp.phys.ethz.ch}}}
\hypersetup{pdfauthor={Niklas Beisert}}
\hypersetup{pdfsubject={Manual for the LaTeX2e Package childdoc}}
\date{30 December 2018, \textsf{v2.0}}
\maketitle

\begin{abstract}\noindent
\textsf{childdoc} is a \LaTeXe{} package
that enables the direct compilation
of document sections included by |\include|
to individual files.
\end{abstract}

\begingroup
\parskip0ex
\tableofcontents
\endgroup

%%%%%%%%%%%%%%%%%%%%%%%%%%%%%%%%%%%%%%%%%%%%%%%%%%%%%%%%%%%%%%%%%%%%%%%%%%%%%%%%
%%%%%%%%%%%%%%%%%%%%%%%%%%%%%%%%%%%%%%%%%%%%%%%%%%%%%%%%%%%%%%%%%%%%%%%%%%%%%%%%
\section{Introduction}

\LaTeX{} provides a mechanism to structure a large document (such as a book)
into a main file and several child files (containing the chapters)
using the |\include| command.
This mechanism is beneficial for documents
which span hundreds of pages in order to
make the source file(s) more manageable.
Moreover, compilation can be restricted to
selected child files by means of the |\includeonly| command.
The latter feature can be used to reduce the compilation time while editing
(this was significantly more useful in the earlier days of \LaTeX{})
or to generate a smaller document which is easier to navigate.
Another application of |\includeonly| is to generate
documents consisting of selected parts of the complete document.

However, there are a few drawbacks of the plain |\include| mechanism:
\begin{itemize}
\item
The child files cannot be compiled on their own,
they can only be compiled via the main file.
A naive editing environment
(such as a text editor with an option
to have the current file processed by \LaTeX)
may require one to switch to the main file before compiling;
attempting to compile the child file produces errors.
\item
The main file must be modified (each time)
to adjust the |\includeonly| command
to the present needs. This easily leaves the main file in a messy state.
\item
The generated document will always carry the filename
of the main document. This is inconvenient if
several child files are to be compiled and
to be kept for distribution.
\end{itemize}

The present package provides a simple interface
to make child files individually compilable by \LaTeX{}.
Compiling a child file then has the same effect as compiling
the main file with an |\includeonly| command
to select the appropriate child.
Moreover the generated document will carry the name of the child
rather than the main file.
This resolves all three above issues.

This feature is meant to make the editing of books,
thesis documents and lecture notes somewhat more convenient.
However, the package can also be used efficiently for
composing a series of documents (such as exercise sheets)
which are typically distributed individually.
It then assists the author in generating the individual documents
(potentially in different versions)
as well as a document containing the collected series.
Another application is in developing style files
or other kinds of included material
where compilation of the style file could redirect
to a sample or test file.

%%%%%%%%%%%%%%%%%%%%%%%%%%%%%%%%%%%%%%%%%%%%%%%%%%%%%%%%%%%%%%%%%%%%%%%%%%%%%%%%
%%%%%%%%%%%%%%%%%%%%%%%%%%%%%%%%%%%%%%%%%%%%%%%%%%%%%%%%%%%%%%%%%%%%%%%%%%%%%%%%
\section{Usage}

First of all, the package \textsf{childdoc} is \emph{not} a standard
\LaTeXe{} |.sty| style file! Therefore it needs to be invoked in
a non-standard way.

%%%%%%%%%%%%%%%%%%%%%%%%%%%%%%%%%%%%%%%%%%%%%%%%%%%%%%%%%%%%%%%%%%%%%%%%%%%%%%%%
\subsection{Included Files}
\label{sec:include}

%%%%%%%%%%%%%%%%%%%%%%%%%%%%%%%%%%%%%%%%
\DescribeMacro{\childdocmain}
To use the package, add the commands
\begin{center}
\begin{tabular}{l}
|\input{childdoc.def}|\\
|\childdocmain{}|\\
\end{tabular}
\end{center}
at the very top of the main \LaTeX{} file,
in particular \emph{before} the |\documentclass| statement!
The argument of |\childdocmain| should be left empty
(but it must be present).

%%%%%%%%%%%%%%%%%%%%%%%%%%%%%%%%%%%%%%%%
\DescribeMacro{\childdocof}
Furthermore, add the commands
\begin{center}
\begin{tabular}{l}
|\input{childdoc.def}|\\
|\childdocof{|\textit{main}|}|\\
\end{tabular}
\end{center}
at the top of every child file \textit{child}
which is included by |\include{|\textit{child}|}|
from within the main file
(or at least for those files to be compiled individually).
The argument \textit{main} must be the filename of the main file.

There are a couple of
considerations in setting up the main and child documents:

%%%%%%%%%%%%%%%%%%%%%%%%%%%%%%%%%%%%%%%%
\paragraph{Restrictions.}

Please note the following restrictions:
\begin{itemize}
\item
|\childdocmain| must be called with one argument \textit{main}
to ensure compatibility with earlier version of the package.
It must either be empty (|\childdocmain{}|)
or precisely match the filename of the main file in which it is specified.
See \secref{sec:detection} for further information.
\item
The filename \textit{main} must be specified without the |.tex| extension.
\item
The filename \textit{main} is case sensitive
(even in case-insensitive file systems)
due to internal string comparison.
\item
The argument \textit{main} should be fully expanded, it cannot be a macro.
\item
Subdirectories and special characters should be avoided in filenames.
\item
The command |\childdocmain{|\textit{main}|}| must be followed by a whitespace.
It should not be followed immediately by another command
or by a comment mark `|%|'.
This is because the \TeX{} parser reads the token immediately following
the argument of |\childdocmain| and puts it
at the beginning of every child section;
however, a white\-space is ignored.
\end{itemize}

%%%%%%%%%%%%%%%%%%%%%%%%%%%%%%%%%%%%%%%%
\paragraph{Content of Main File.}

It is advisable to place all content in the child files included by |\include|.
Any output contained in the main file will appear in all child documents
unless suppressed manually;
it cannot be suppressed automatically by the |\includeonly| directive
and thus should normally be avoided.
A method to include some content in the main file
by means of conditional processing is described in \secref{sec:conditional}.

%%%%%%%%%%%%%%%%%%%%%%%%%%%%%%%%%%%%%%%%
\paragraph{Page Numbering.}

When only a part of the document is compiled,
the appropriate numbering of pages
(as well as other status parameters)
is determined from the |.aux| files.
The latter contain information from previous passes.
However this information needs to propagate through
all intermediate child documents.
Therefore the page numbering in child documents may well
be inconsistent until the complete document is compiled at least once.

A useful (if unconventional) way to always ensure a consistent
page numbering is to restart the numbering in each child document
and denote the pages by `\textit{child}|.|\textit{page}'
where \textit{child} represents the chapter/section number of the child file.
This can be achieved by the command
|\numberwithin{page}{|\textit{child}|}|
of the \textsf{amsmath} package
where \textit{child} can be |chapter| or |section|
depending on the chosen structuring.
Alternatively, one can modify the macro |\thepage| appropriately
and reset the counter |page| at the start of each child file.

%%%%%%%%%%%%%%%%%%%%%%%%%%%%%%%%%%%%%%%%%%%%%%%%%%%%%%%%%%%%%%%%%%%%%%%%%%%%%%%%
\subsection{Conditional Processing}
\label{sec:conditional}

The package provides a mechanism to compile different versions
of a document. To customise the versions further some conditional processing
can come in handy to distinguish which version is being compiled.
The package provides two macros to describe the compilation context:

%%%%%%%%%%%%%%%%%%%%%%%%%%%%%%%%%%%%%%%%
\DescribeMacro{\ifchilddoc}
The conditional |\ifchilddoc| distinguishes between the compilation of
child documents and the main document:
%
\begin{center}
|\ifchilddoc |\textit{child-code}| |[|\||else |\textit{main-code}]| \||fi|
\end{center}

%%%%%%%%%%%%%%%%%%%%%%%%%%%%%%%%%%%%%%%%
\DescribeMacro{\childdocname}
\DescribeMacro{\childdocjob}
The macro |\childdocname| contains the filename (without extension)
of the main or child file being processed.
Note that |\childdocjob| will always contain the name of the main file.

%%%%%%%%%%%%%%%%%%%%%%%%%%%%%%%%%%%%%%%%
\paragraph{Title Page.}

Conditional processing can be used to include a title or banner page
in the main document when proper precautions are taken.
Importantly, the code in the main file should ensure that the page counter
(as well as other status parameters which are stored in the |.aux| files)
takes the same value after the conditional processing.
Otherwise the page numbers may take divergent values
depending on which part is compiled.

For example, a title page could be declared by:
%
\begin{center}
\begin{tabular}{l}
|\ifchilddoc\||else|\\
|\addtocounter{page}{-1}|\\
\textit{code for title page}\\
|\newpage|\\
|\||fi|
\end{tabular}
\end{center}
%
A banner page for the child documents can be generated by:
%
\begin{center}
\begin{tabular}{l}
|\ifchilddoc|\\
|\addtocounter{page}{-1}|\\
\textit{code for banner page}\\
|\newpage|\\
|\||fi|
\end{tabular}
\end{center}
%
Here one could write a message such as:
\begin{center}
|This is the part \childdocname{} of \childdocjob{}.|
\end{center}

%%%%%%%%%%%%%%%%%%%%%%%%%%%%%%%%%%%%%%%%%%%%%%%%%%%%%%%%%%%%%%%%%%%%%%%%%%%%%%%%
\subsection{Flags}
\label{sec:flags}

The package makes it easy to generate different versions
of the main or child documents.
To this end compilation flags can be defined
and assigned different default values.
They will be particularly useful in conjunction
with the forwarding mechanism described in \secref{sec:forward}.

For example, it may be useful to have a flag |\version|
which can be set to |draft| or |final|.
The document source will contain some conditional code
depending on the value of |\version|.
Suppose further, the flag should default to |final| for the main file
and to |draft| for child files
which is a natural assignment for editing the document.
This is achieved by placing the following code
in the preamble of the main document
(below the |\childdocmain| directive):
%
\begin{center}
\begin{tabular}{l}
|\ifchilddoc|\\
|\providecommand{\version}{draft}|\\
|\||else|\\
|\providecommand{\version}{final}|\\
|\||fi|
\end{tabular}
\end{center}
%
The definition by |\providecommand| makes sure
that previous definitions are not overwritten.
Further statements |\providecommand{\version}{...}|
can thus be added before the above code to override it.

For the main file, one might add a line
(between |\childdocmain| and the above block)
%
\begin{center}
|%\ifchilddoc\||else\providecommand{\version}{draft}\||fi|
\end{center}
%
which can be uncommented to produce a draft version.
Likewise one can add a line to the very top of a child file
(above the |\childdocof{|\textit{main}|}| directive)
%
\begin{center}
|%\providecommand{\version}{final}|
\end{center}
%
which can be uncommented to produce the final version of this child document.

%%%%%%%%%%%%%%%%%%%%%%%%%%%%%%%%%%%%%%%%%%%%%%%%%%%%%%%%%%%%%%%%%%%%%%%%%%%%%%%%
\subsection{Forwarding}
\label{sec:forward}

Different versions of the main or child documents
using compilation flags as described in \secref{sec:flags}
can be (permanently) stored in different files
for convenient compilation, viewing and distribution.
To this end, the package defines a command
to pass on compilation to a different file:

%%%%%%%%%%%%%%%%%%%%%%%%%%%%%%%%%%%%%%%%
\DescribeMacro{\childdocforward}
The command |\childdocforward| redirects processing to
another source file:
%
\begin{center}
\begin{tabular}{l}
|\input{childdoc.def}|\\
|\childdocforward[|\textit{main}|]{|\textit{dest}|}|\\
\end{tabular}
\end{center}
%
The argument \textit{dest} is the destination file
(without extension).
It should be the main file or one of the child files.
Note that further \textsf{childdoc} directives
such as |\childdocof| and |\childdocforward|
in the indicated file will be processed in this form.
The optional argument \textit{main}
passes on directly to the main file \textit{main}
while pretending to compile the child \textit{dest}.
This form behaves as if \textit{dest}
issues |\childdocof{|\textit{main}|}| right away,
and no further \textsf{childdoc} directives will be processed.

%%%%%%%%%%%%%%%%%%%%%%%%%%%%%%%%%%%%%%%%
\DescribeMacro{\...prefix}
In the alternative form |\childdocforwardprefix|,
%
\begin{center}
\begin{tabular}{l}
|\input{childdoc.def}|\\
|\childdocforwardprefix[|\textit{main}|]{|\textit{prefix}|}{|\textit{dest}|}|
\end{tabular}
\end{center}
%
the destination file is determined by a pattern
depending on the current file:
To make this work, the current file must be called
`{\textit{prefix}\hspace{0.2em}\textit{suffix}}'
with \textit{prefix} matching precisely the argument.
Processing is then passed on to the file
`{\textit{dest}\hspace{0.2em}\textit{suffix}}'.
Surely, the same effect is achieved by
directly specifying the
argument `{\textit{dest}\hspace{0.2em}\textit{suffix}}'
in the first form.
However, that requires to set up a different file
for each child. With the alternative form of the command
all these files can have exactly the same content
which simplifies setting them up and maintaining them.

For example, the following file |draft.tex|
with a compilation flag |\version| as described in \secref{sec:flags}
compiles the main document as a draft:
%
\begin{center}
\begin{tabular}{l}
|\def\version{draft}|\\
|\input{childdoc.def}|\\
|\childdocforward{|\textit{main}|}|
\end{tabular}
\end{center}
%
Likewise, the following files |final|\textit{nn}|.tex|
compile the final version of the child document
|child|\textit{nn}|.tex|:
%
\begin{center}
\begin{tabular}{l}
|\def\version{final}|\\
|\input{childdoc.def}|\\
|\childdocforwardprefix{final}{child}|
\end{tabular}
\end{center}
%

Note that when several versions of a main file and/or of each child file
are to be generated, it may be convenient to set up a |Makefile| or
shell script to automatise the process.

%%%%%%%%%%%%%%%%%%%%%%%%%%%%%%%%%%%%%%%%%%%%%%%%%%%%%%%%%%%%%%%%%%%%%%%%%%%%%%%%
\subsection{Command Line Processing}
\label{sec:commandline}

The effect of redirection files can also be achieved by invoking
the \LaTeX{} compiler with a more elaborate command line.
Most conveniently this should be done as part
of a shell script or a |Makefile|.

When using \textsf{childdoc} in the main file, the following
command lines effectively perform a redirection
(note that depending on the shell being used,
backslashes may have to be doubled: `|\|' $\to$ `|\\|'):
%
\begin{center}
|... -jobname "|\textit{target}|" |\\|"|[\textit{flags}]%
|\input{childdoc.def}\childdocforward[|\textit{main}|]{|\textit{dest}|}"|
\end{center}
%
Here \textit{target} is the name of the output file,
\textit{main} is the name of the main file
and \textit{dest} is the name of the main or child file to be processed
(all filenames without extensions).
The optional argument \textit{main} can be omitted
if \textit{main} matches \textit{dest}.
Optionally, compilation \textit{flags} can be defined via |\def| commands.
This command line makes the \TeX{} engine believe
it is compiling the file \textit{target}
whose content is specified as the latter parameter.
The provided code then forwards the processing to
\textit{main} or \textit{dest} as described in \secref{sec:forward}.

%%%%%%%%%%%%%%%%%%%%%%%%%%%%%%%%%%%%%%%%%%%%%%%%%%%%%%%%%%%%%%%%%%%%%%%%%%%%%%%%
\subsection{Include by Input}
\label{sec:input}

Including child documents by |\include| has some restrictions by design.
Most notably, the content of a child document always occupies
its own set of pages; pages cannot be shared between child documents.
Usually, this behaviour makes perfect sense
because each child document contain an essential part of the document.
However, in some situations it may be desirable to compose
a document from a collection of parts
without having mandatory page breaks between then.
For this case, the package
provides a mechanism to include parts
by |\input| which can also be processed individually.
However, by construction this mechanism
requires manual handling of the content to be output.

%%%%%%%%%%%%%%%%%%%%%%%%%%%%%%%%%%%%%%%%
\DescribeMacro{\ifchilddocmanual}
The main file should be prepared as usual, see \secref{sec:include}.
However, the document body must make a distinction
between processing of an individual part and of the main document, e.g.:
%
\begin{center}
\begin{tabular}{l}
|\ifchilddocmanual|\\
|\input{\childdocname}|\\
|\||else|\\
\textit{document body with }|\input{|\textit{part}|}|\\
|\||fi|
\end{tabular}
\end{center}
%
The conditional |\ifchilddocmanual| is true whenever
a part to be included by |\input| is being compiled,
and the name of the part is stored in |\childdocname|.

%%%%%%%%%%%%%%%%%%%%%%%%%%%%%%%%%%%%%%%%
\DescribeMacro{\childdocby}
Each part to be included by |\input| should start with:
%
\begin{center}
\begin{tabular}{l}
|\input{childdoc.def}|\\
|\childdocby{|\textit{main}|}|\\
\end{tabular}
\end{center}
%
The directive |\childdocby| is similar to |\childdocof|
described in \secref{sec:include},
but the subsequent selection of content must be done manually.
To that end, both |\ifchilddoc| and |\ifchilddocmanual|
will be true upon processing of a part,
and the name of the part is stored in |\childdocname|.
Note that |\jobname| will be set to the filename of the current part
so that each part receives an individual |.aux| file
that does not interfere with the |.aux| file(s) of the main document.
This behaviour can be altered by the alternative form
|\childdocby[*]{|\textit{main}|}| (with a non-empty optional argument)
which uses the |.aux| file of the main document
by setting |\jobname| to \textit{main}.

%%%%%%%%%%%%%%%%%%%%%%%%%%%%%%%%%%%%%%%%%%%%%%%%%%%%%%%%%%%%%%%%%%%%%%%%%%%%%%%%
\subsection{Driver Development}
\label{sec:driver}

The \textsf{childdoc} mechanism can also be use for the development
of definition files such as \LaTeX{} styles or classes.
This case differs from the above setup with multiple parts
included by |\include| in that no |\includeonly| should be invoked.
This can be achieved by starting the include file
(before |\ProvidesPackage|) with:
%
\begin{center}
\begin{tabular}{l}
|\input{childdoc.def}|\\
|\childdocforward{|\textit{main}|}|\\
\end{tabular}
\end{center}
%
or alternatively with:
%
\begin{center}
\begin{tabular}{l}
|\input{childdoc.def}|\\
|\childdocby{|\textit{main}|}|\\
\end{tabular}
\end{center}
%
Both forms have slightly different effects as described above.
The main file is prepared as usual, see \secref{sec:include}.

%%%%%%%%%%%%%%%%%%%%%%%%%%%%%%%%%%%%%%%%%%%%%%%%%%%%%%%%%%%%%%%%%%%%%%%%%%%%%%%%
\subsection{Legacy Detection}
\label{sec:detection}

The directive |\childdocmain| in the main file can detect
whether the complete document or merely a child is to be compiled
even without using the directive |\childdocof|.
This method is deprecated because it is less robust
and there is no compelling reason to use it;
it is merely provided for backward compatibility
and it may be removed in future versions.

If the detection mechanism is to be used,
it is mandatory to correctly specify
the filename of the main file as the argument of |\childdocmain|:
%
\begin{center}
\begin{tabular}{l}
|\input{childdoc.def}|\\
|\childdocmain{|\textit{main}|}|\\
\end{tabular}
\end{center}
%
If |\jobname| does not match the argument \textit{main} of |\childdocmain|,
it is assumed that |\jobname| points to the child file to be compiled.
When using |\childdocmain| with the main file specified as argument,
it suffices to start a child file
with just |\input{|\textit{main}|}|
without loading of the package and using |\childdocof|.
If instead all processing is done
with the appropriate \textsf{childdoc} directives,
the argument of \textit{main} of |\childdocmain| can be empty.

An alternative version of the command line processing described
in \secref{sec:commandline} using the detection mechanism reads:
%
\begin{center}
|... -jobname "|\textit{target}|" "|[\textit{flags}]%
[|\def\jobname{|\textit{dest}|}|]|\input{|\textit{main}|}"|
\end{center}

%%%%%%%%%%%%%%%%%%%%%%%%%%%%%%%%%%%%%%%%%%%%%%%%%%%%%%%%%%%%%%%%%%%%%%%%%%%%%%%%
\subsection{Manual Code}
\label{sec:manual}

In case one cannot be certain whether the definitions file |childdoc.def|
is installed on the target \TeX{} distribution
and one prefers not to ship it,
it is conceivable to paste a few relevant commands into the sources.

To that end, drop all statements |\input{childdoc.def}|
and perform the replacements as outlined below.
Instead of |\childdocmain{|\textit{main}|}| add the following code
to the top of the main file:
%
\begin{center}
\begin{tabular}{l}
|\||ifdefined\childdocname\endinput\||fi\newif\ifchilddoc|\\
|\edef\childdocname{\scantokens\expandafter{\jobname\noexpand}}|\\
|\def\childdocmain{|\textit{main}|}\||ifx\childdocmain\childdocname\||else|\\
|\childdoctrue\includeonly{\childdocname}\let\jobname\childdocmain\||fi|\\
\end{tabular}
\end{center}
%
Instead of |\childdocof{|\textit{main}|}| just include the main file
at the top of each child file:
%
\begin{center}
|\input{|\textit{main}|}|
\end{center}
%
A simple redirection |\childdocforward{|\textit{dest}|}| is achieved by:
%
\begin{center}
|\def\jobname{|\textit{dest}|}\input{\jobname}|
\end{center}
%
The redirection with prefix
|\childdocforwardprefix[|\textit{prefix}|]{|\textit{dest}|}|
is accomplished by:
%
\begin{center}
\begin{tabular}{l}
|{\edef\jobname{\scantokens\expandafter{\jobname\noexpand}}|\\
|\def\redirectjob |\textit{prefix}|#1~~~{\gdef\jobname{|\textit{dest}|#1}}|\\
|\expandafter\redirectjob\jobname~~~}\input{\jobname}|
\end{tabular}
\end{center}

In an alternative approach,
child documents can be compiled by a specific command line
without additional code or specific definitions:
%
\begin{center}
|... -jobname "|\textit{target}|" "|[\textit{flags}]%
|\includeonly{|\textit{dest}|}\input{|\textit{main}|}"|
\end{center}
%

%%%%%%%%%%%%%%%%%%%%%%%%%%%%%%%%%%%%%%%%%%%%%%%%%%%%%%%%%%%%%%%%%%%%%%%%%%%%%%%%
%%%%%%%%%%%%%%%%%%%%%%%%%%%%%%%%%%%%%%%%%%%%%%%%%%%%%%%%%%%%%%%%%%%%%%%%%%%%%%%%
\section{Information}

%%%%%%%%%%%%%%%%%%%%%%%%%%%%%%%%%%%%%%%%%%%%%%%%%%%%%%%%%%%%%%%%%%%%%%%%%%%%%%%%
\subsection{Copyright}

Copyright \copyright{} 2017--2018 Niklas Beisert

This work may be distributed and/or modified under the
conditions of the \LaTeX{} Project Public License, either version 1.3
of this license or (at your option) any later version.
The latest version of this license is in
  \url{http://www.latex-project.org/lppl.txt}
and version 1.3 or later is part of all distributions of \LaTeX{}
version 2005/12/01 or later.

This work has the LPPL maintenance status `maintained'.

The Current Maintainer of this work is Niklas Beisert.

This work consists of the files |README.txt|, |childdoc.ins| and |childdoc.dtx|
as well as the derived files |childdoc.def|, |cdocsamp.tex|
with |cdocsch1.tex|, |cdocsch2.tex|, |cdocspt3.tex|, |cdocspt4.tex|,
|cdocsdrf.tex|, |cdocsfn1.tex|, |cdocsfn2.tex|
as well as |childdoc.pdf|.

%%%%%%%%%%%%%%%%%%%%%%%%%%%%%%%%%%%%%%%%%%%%%%%%%%%%%%%%%%%%%%%%%%%%%%%%%%%%%%%%
\subsection{Files and Installation}

The package consists of the files:
%
\begin{center}
\begin{tabular}{ll}
    |README.txt|   & readme file \\
    |childdoc.ins| & installation file \\
    |childdoc.dtx| & source file \\
    |childdoc.def| & definition file \\
    |cdocsamp.tex| & sample main file \\
    |cdocsch1.tex| & sample include file \\
    |cdocsch2.tex| & sample include file \\
    |cdocspt3.tex| & sample part file \\
    |cdocspt4.tex| & sample part file \\
    |cdocsdrf.tex| & sample redirection file \\
    |cdocsfn1.tex| & sample redirection file \\
    |cdocsfn2.tex| & sample redirection file \\
    |childdoc.pdf| & manual
\end{tabular}
\end{center}
%
The distribution consists of the files
|README.txt|, |childdoc.ins| and |childdoc.dtx|.
%
\begin{itemize}
\item
Run (pdf)\LaTeX{} on |childdoc.dtx|
to compile the manual |childdoc.pdf| (this file).
\item
Run \LaTeX{} on |childdoc.ins| to create the definitions file |childdoc.def|
and the sample |cdocsamp.tex| with include files
|cdocsch1.tex|, |cdocsch2.tex|, |cdocspt3.tex|, |cdocspt4.tex|,
|cdocsdrf.tex|, |cdocsfn1.tex|, |cdocsfn2.tex|.
Then copy the file |childdoc.def| to an appropriate directory of your \LaTeX{}
distribution, e.g.\ \textit{texmf-root}|/tex/latex/childdoc|.
\end{itemize}

%%%%%%%%%%%%%%%%%%%%%%%%%%%%%%%%%%%%%%%%%%%%%%%%%%%%%%%%%%%%%%%%%%%%%%%%%%%%%%%%
\subsection{Related CTAN Packages}

There are several other packages which offer a similar functionality:
%
\begin{itemize}
\item
The packages
\href{http://ctan.org/pkg/docmute}{\textsf{docmute}},
\href{http://ctan.org/pkg/includex}{\textsf{includex}} and
\href{http://ctan.org/pkg/standalone}{\textsf{standalone}}
provide commands to include only the document body of
a child file thus allowing both files to be compiled individually.
\item
The packages \href{http://ctan.org/pkg/subdocs}{\textsf{subdocs}}
and \href{http://ctan.org/pkg/subfiles}{\textsf{subfiles}}
provide structures in which the main and child documents can be
encapsulated and allowing them to be compiled individually.
The inclusion mechanism is different from the conventional |\include|.
\item
The package \href{http://ctan.org/pkg/combine}{\textsf{combine}}
is an elaborate solution to combine several documents into one.
\end{itemize}
%
See also the CTAN topic \href{http://ctan.org/topic/subdocs}{\textsf{subdocs}}
for further related packages.
The present package differs from the above solutions in that
a document structure constructed with the conventional |\include| mechanism
just needs two extra commands at the top of every file
such that all constituent files can be compiled individually.

%%%%%%%%%%%%%%%%%%%%%%%%%%%%%%%%%%%%%%%%%%%%%%%%%%%%%%%%%%%%%%%%%%%%%%%%%%%%%%%%
%\subsection{Feature Suggestions}
%
%The following is a list of features which may be useful for future
%versions of this package:
%%
%\begin{itemize}
%\item
%\ldots
%\end{itemize}

%%%%%%%%%%%%%%%%%%%%%%%%%%%%%%%%%%%%%%%%%%%%%%%%%%%%%%%%%%%%%%%%%%%%%%%%%%%%%%%%
\subsection{Revision History}

%%%%%%%%%%%%%%%%%%%%%%%%%%%%%%%%%%%%%%%%
\paragraph{v2.0:} 2018/12/30

\begin{itemize}
\item
immediate forward processing
\item
added |\childdocby| mechanism
\item
manual restructured
\end{itemize}

%%%%%%%%%%%%%%%%%%%%%%%%%%%%%%%%%%%%%%%%
\paragraph{v1.6:} 2018/01/17

\begin{itemize}
\item
application for development of include files
\item
corrections to manual
\end{itemize}

%%%%%%%%%%%%%%%%%%%%%%%%%%%%%%%%%%%%%%%%
\paragraph{v1.5:} 2017/05/21

\begin{itemize}
\item
more complete structuring introduced
\item
|\childdocof| introduced
\item
|\childdoc| renamed to |\childdocmain|
\item
|\childredirect| renamed to |\childdocforward| and |\childdocforwardprefix|
and functionality expanded
\end{itemize}

%%%%%%%%%%%%%%%%%%%%%%%%%%%%%%%%%%%%%%%%
\paragraph{v1.0:} 2017/04/27

\begin{itemize}
\item
manual and install package
\item
first version published on CTAN
\end{itemize}

%%%%%%%%%%%%%%%%%%%%%%%%%%%%%%%%%%%%%%%%
\paragraph{v0.6:} 2017/04/26

\begin{itemize}
\item
redirection mechanism added
\end{itemize}

%%%%%%%%%%%%%%%%%%%%%%%%%%%%%%%%%%%%%%%%
\paragraph{v0.5:} 2017/04/26

\begin{itemize}
\item
functionality in definition file
\end{itemize}


%%%%%%%%%%%%%%%%%%%%%%%%%%%%%%%%%%%%%%%%%%%%%%%%%%%%%%%%%%%%%%%%%%%%%%%%%%%%%%%%
%%%%%%%%%%%%%%%%%%%%%%%%%%%%%%%%%%%%%%%%%%%%%%%%%%%%%%%%%%%%%%%%%%%%%%%%%%%%%%%%
%%%%%%%%%%%%%%%%%%%%%%%%%%%%%%%%%%%%%%%%%%%%%%%%%%%%%%%%%%%%%%%%%%%%%%%%%%%%%%%%
\appendix

\settowidth\MacroIndent{\rmfamily\scriptsize 000\ }

 \DocInput{childdoc.dtx}

\end{document}
%</driver>
% \fi
%
% %%%%%%%%%%%%%%%%%%%%%%%%%%%%%%%%%%%%%%%%%%%%%%%%%%%%%%%%%%%%%%%%%%%%%%%%%%%%%%
% %%%%%%%%%%%%%%%%%%%%%%%%%%%%%%%%%%%%%%%%%%%%%%%%%%%%%%%%%%%%%%%%%%%%%%%%%%%%%%
% \section{Sample}
%\iffalse
%<*samplemain>
%\fi
%
% The following presents a sample document
% with two chapters, two parts, a title page,
% a compile flag as well as three forwarding files to set the flag.
% It consists of eight |.tex| files:
% \begin{center}
% \begin{tabular}{ll}
% |cdocsamp.tex|&main file\\
% |cdocsch1.tex|&include file for chapter 1\\
% |cdocsch2.tex|&include file for chapter 2\\
% |cdocspt3.tex|&include file for part 3\\
% |cdocspt4.tex|&include file for part 4\\
% |cdocsdrf.tex|&forwarding file for main file in draft mode\\
% |cdocsfi1.tex|&forwarding file for final version of chapter 1\\
% |cdocsfi2.tex|&forwarding file for final version of chapter 2\\
% \end{tabular}
% \end{center}
% Each of the eight files can be compiled directly by the \LaTeX{} compiler.
%
% %%%%%%%%%%%%%%%%%%%%%%%%%%%%%%%%%%%%%%
% \paragraph{Main File.}
%
% The main file is called |cdocsamp.tex|.
%
% Load the \textsf{childdoc} definitions and
% declare the filename for the main document:
%    \begin{macrocode}
\input{childdoc.def}
\childdocmain{}
%    \end{macrocode}

% Optional override for |\version| flag:
%    \begin{macrocode}
%%\ifchilddoc\else\providecommand{\version}{draft}\fi
%    \end{macrocode}

% Define the default values for the |\version| flag
% (|final| for the main file and |draft| for childs):
%    \begin{macrocode}
\ifchilddoc
\providecommand{\version}{draft}
\else
\providecommand{\version}{final}
\fi
%    \end{macrocode}

% Load the standard document class:
%    \begin{macrocode}
\documentclass[12pt]{article}
%    \end{macrocode}

% Start the document body:
%    \begin{macrocode}
\begin{document}
%    \end{macrocode}

% Declare a title page.
% Print title, part of document being processed and version flag:
%    \begin{macrocode}
\addtocounter{page}{-1}
\begin{center}
{\LARGE\bfseries{}childdoc example\par}
\vspace{1cm}
\ifchilddoc
\ifchilddocmanual part\else chapter\fi:
`\childdocname' of `\childdocjob'\par
\else
main document: `\childdocjob'\par
\fi
version: \version\par
\end{center}
\newpage
%    \end{macrocode}

% Manually include selected file,
% otherwise process as usual:
%    \begin{macrocode}
\ifchilddocmanual
\section*{part `\childdocname'}
\input{\childdocname}
\else
%    \end{macrocode}

% Include the two chapters:
%    \begin{macrocode}
\include{cdocsch1}
\include{cdocsch2}
%    \end{macrocode}

% Include the two parts unless only chapters should be displayed:
%    \begin{macrocode}
\ifchilddoc\else
\section{part three}
\input{cdocspt3}
\section{part four}
\input{cdocspt4}
\fi
%    \end{macrocode}

% Process as usual until here:
%    \begin{macrocode}
\fi
%    \end{macrocode}

% End of document body:
%    \begin{macrocode}
\end{document}
%    \end{macrocode}
%\iffalse
%</samplemain>
%\fi
%
% %%%%%%%%%%%%%%%%%%%%%%%%%%%%%%%%%%%%%%
% \paragraph{Chapter Include Files.}
%
% The include files are called |cdocsch1.tex| and |cdocsch2.tex|.
%
%\iffalse
%<*samplechap1|samplechap2>
%\fi

% Optional override for |\version| flag:
%    \begin{macrocode}
%%\providecommand{\version}{final}
%    \end{macrocode}

% Include the main document:
%    \begin{macrocode}
\input{childdoc.def}
\childdocof{cdocsamp}
%    \end{macrocode}

%\iffalse
%</samplechap1|samplechap2>
%\fi
%
%\iffalse
%<*samplechap1>
%\fi
% Some text for chapter 1:
%    \begin{macrocode}
\section{one}
some text in chapter one
%    \end{macrocode}

%\iffalse
%</samplechap1>
%\fi
% Some text for chapter 2:
%\iffalse
%<*samplechap2>
%\fi
%    \begin{macrocode}
\section{two}
more text in chapter two
%    \end{macrocode}

%\iffalse
%</samplechap2>
%\fi
%
% %%%%%%%%%%%%%%%%%%%%%%%%%%%%%%%%%%%%%%
% \paragraph{Part Include Files.}
%
% The include files are called |cdocspt3.tex| and |cdocspt4.tex|.
%
%\iffalse
%<*samplepart3|samplepart4>
%\fi

% Optional override for |\version| flag:
%    \begin{macrocode}
%%\providecommand{\version}{final}
%    \end{macrocode}

% Include the main document:
%    \begin{macrocode}
\input{childdoc.def}
\childdocby{cdocsamp}
%    \end{macrocode}

%\iffalse
%</samplepart3|samplepart4>
%\fi
%
%\iffalse
%<*samplepart3>
%\fi
% Some text for part 3:
%    \begin{macrocode}
some text in part three
%    \end{macrocode}

%\iffalse
%</samplepart3>
%\fi
% Some text for part 4:
%\iffalse
%<*samplepart4>
%\fi
%    \begin{macrocode}
more text in part four
%    \end{macrocode}

%\iffalse
%</samplepart4>
%\fi
%
% %%%%%%%%%%%%%%%%%%%%%%%%%%%%%%%%%%%%%%
% \paragraph{Forwarding for a Complete Draft.}
%
% The following forwarding file |cdocsdrf.tex|
% compiles the main document in draft mode:
%\iffalse
%<*sampledraft>
%\fi
%    \begin{macrocode}
\def\version{draft}
\input{childdoc.def}
\childdocforward{cdocsamp}
%    \end{macrocode}

%\iffalse
%</sampledraft>
%\fi
%
% %%%%%%%%%%%%%%%%%%%%%%%%%%%%%%%%%%%%%%
% \paragraph{Forwarding for Final Version of the Chapters.}
%
% The following forwarding files |cdocsfn1.tex| and |cdocsfn2.tex|
% (with identical content)
% compile the final versions of the child documents
% |cdocsch1.tex| and |cdocsch2.tex|, respectively:
%\iffalse
%<*samplefinal>
%\fi
%    \begin{macrocode}
\def\version{final}
\input{childdoc.def}
\childdocforwardprefix[cdocsamp]{cdocsfn}{cdocsch}
%    \end{macrocode}

%\iffalse
%</samplefinal>
%\fi
%
% %%%%%%%%%%%%%%%%%%%%%%%%%%%%%%%%%%%%%%
% \paragraph{Command Line Processing.}
%
% The following three command lines generate the output files
% |cdocscld|, |cdocscl1| and |cdocscl2|
% which should be identical to
% |cdocsdrf|, |cdocsch1| and |cdocsfn2|, respectively:
% \begin{center}
% \begin{tabular}{l}
% |latex -jobname cdocscld \|\\
% |  "\def\version{draft}\input{childdoc.def}\childdocforward{cdocsamp}"|\\
% |latex -jobname cdocscl1 \|\\
% |  "\input{childdoc.def}\childdocforward[cdocsamp]{cdocsch1}"|\\
% |latex -jobname cdocscl2 \|\\
% |  "\def\version{final}\input{childdoc.def}\childdocforward{cdocsch2}"|
% \end{tabular}
% \end{center}
% Note that the trailing backslash on each first line
% merely continues the input to the second line
% (for convenient cut ant paste).
% Furthermore, the command |latex| can be replaced by any
% of its alternative versions such as |pdflatex|.
%
% %%%%%%%%%%%%%%%%%%%%%%%%%%%%%%%%%%%%%%%%%%%%%%%%%%%%%%%%%%%%%%%%%%%%%%%%%%%%%%
% %%%%%%%%%%%%%%%%%%%%%%%%%%%%%%%%%%%%%%%%%%%%%%%%%%%%%%%%%%%%%%%%%%%%%%%%%%%%%%
% \section{Implementation}
%\iffalse
%<*package>
%\fi
%
% This section describes the definitions file |childdoc.def|.

% The definitions cannot be loaded using |\usepackage| or |\RequirePackage|
% which has a mechanism to prevent loading a style file more than once.
% When loading the definitions by means of |\input|
% multiple instances have to be prevented manually:
%\iffalse
%This code needs to be before the `\ProvidesFile' directive
%which is defined at the beginning of this file.
%Therefore it is also placed there and commented out here.
%</package>
%<*discard>
%\fi
%    \begin{macrocode}
\ifdefined\childdocmain\endinput\fi
%    \end{macrocode}
%\iffalse
%</discard>
%<*package>
%\fi
%
% \macro{\ifchilddoc}
% \macro{\ifchilddocmanual}
% The conditional |\ifchilddoc| tells whether a
% child (true) or main (false) document is being compiled.
% The conditional |\ifchilddocmanual| tells whether
% the |\includeonly| mechanism is used (false) or
% the selection of child files must be performed manually (true).
% The definitions initialise to false:
%    \begin{macrocode}
\newif\ifchilddoc
\newif\ifchilddocmanual
%    \end{macrocode}

% \macro{\childdocname}
% \macro{\childdocjob}
% The macro |\childdocname| stores the name of the main document
% to be compiled. The macro |\childdocjob| stores the name of
% the document on which the \LaTeX{} compiler was originally invoked.
% The content of |\jobname| cannot be compared
% to filenames specified in the source due to different catcodes.
% The following code rescans |\jobname|, stores the result
% in |\childdocname| and saves a copy in |\childdocjob|:
%    \begin{macrocode}
\edef\childdocname{\scantokens\expandafter{\jobname\noexpand}}
\let\childdocjob\childdocname
%    \end{macrocode}

% \macro{\childdocdisable}
% The macro |\childdocdisable| prevents the main file
% from being processed more than once.
% At this stage, the main document command |\childdocmain|
% is assumed to be called once again where it should do nothing.
% Any subsequent call to it should prevent
% a secondary processing of the main document
% It overwrites the forwarding commands
% |\childdocof| and |\childdocforward|
% with empty macros to prevent further inclusions of the main document:
%    \begin{macrocode}
\newcommand{\childdocdisable}
{
  \renewcommand{\childdocmain}[1]{\renewcommand{\childdocmain}[1]{\endinput}}
  \renewcommand{\childdocof}[1]{}
  \renewcommand{\childdocby}[2][]{}
  \renewcommand{\childdocforward}[2][]{}
  \renewcommand{\childdocdisable}{}
}
%    \end{macrocode}

% \macro{\childdocmain}
% The macro |\childdocmain| is to be called at the top of the main file
% with nothing or the main filename (without extension) as argument.
% First, it breaks loops.
% If the argument is not empty and does not match |\childdocname|
% (which is set by the first inclusion of |childdoc.def|),
% |\ifchilddoc| is set to true, |\includeonly| is applied to the child file
% and |\jobname| is set to the main file
% (for proper handling of |.aux| files):
%    \begin{macrocode}
\newcommand{\childdocmain}[1]
{
  \childdocdisable\childdocmain{}
  \if?#1?\else
    \begingroup
      \def\childdoctmp{#1}
      \ifx\childdoctmp\childdocname
        \def\childdoctmp{}
      \else
        \def\childdoctmp
        {
          \childdoctrue
          \includeonly{\childdocname}
          \def\childdocjob{#1}
          \def\jobname{#1}
        }
      \fi
      \expandafter
    \endgroup
    \childdoctmp
  \fi
}
%    \end{macrocode}

% \macro{\childdocof}
% The command |\childdocof| redirects
% compilation to the main file |#1|.
%    \begin{macrocode}
\newcommand{\childdocof}[1]
{
  \childdocdisable
  \childdoctrue
  \includeonly{\childdocname}
  \def\jobname{#1}
  \def\childdocjob{#1}
  \input{#1}
}
%    \end{macrocode}

% \macro{\childdocby}
% The command |\childdocby| ....
%    \begin{macrocode}
\newcommand{\childdocby}[2][]
{
  \childdocdisable
  \childdoctrue
  \childdocmanualtrue
  \if?#1?\else
    \def\jobname{#2}
  \fi
  \def\childdocjob{#2}
  \input{#2}
  \endinput
}
%    \end{macrocode}

% \macro{\childdocforward}
% The command |\childdocforward| redirects
% compilation to the main file or
% (if the optional argument is given) a child file.
% Parameters are set as if the main file
% or a child file starting with |\childdocof| was compiled.
% Then compilation is handed over to the main file:
%    \begin{macrocode}
\newcommand{\childdocforward}[2][]
{
  \begingroup
    \if?#1?
      \def\childdoctmp
      {
        \def\childdocname{#2}
        \def\childdocjob{#2}
        \def\jobname{#2}
        \input{#2}
        \endinput
      }
    \else
      \def\childdoctmp
      {
        \childdocdisable
        \def\childdocname{#2}
        \childdoctrue
        \includeonly{#2}
        \def\childdocjob{#1}
        \def\jobname{#1}
        \input{#1}
        \endinput
      }
    \fi
    \expandafter
  \endgroup
  \childdoctmp
}
%    \end{macrocode}

% \macro{\childdocforwardprefix}
% The command |\childdocforwardprefix| redirects
% compilation to the main or a child file by means of a pattern.
% The prefix |#1| in the current filename is replaced by |#2|
% and the suffix of the current filename is kept
% (it is assumed that the filename does not contain the substring `|~~~|'
% which is used as a delimiter).
% Compilation is handed over to the new file by |\childdocforward|:
%    \begin{macrocode}
\newcommand{\childdocforwardprefix}[3][]
{
  \begingroup
    \def\childdocextract #2##1~~~{\def\childdoctmp{\childdocforward[#1]{#3##1}}}
    \expandafter\childdocextract\childdocname~~~
    \expandafter
  \endgroup
  \childdoctmp
}
%    \end{macrocode}

% \macro{\childdoc}
% The deprecated macro |\childdoc| is a legacy version of |\childdocmain|:
%    \begin{macrocode}
\newcommand{\childdoc}{\childdocmain}
%    \end{macrocode}

% \macro{\childdocredirect}
% The deprecated macro |\childdocredirect| is a legacy version
% of |\childdocforward| and |\childdocforwardprefix|:
%    \begin{macrocode}
\newcommand{\childdocredirect}[2][]
{
  \begingroup
    \if?#1?
      \def\childdoctmp{\childdocforward{#2}}
    \else
      \def\childdoctmp{\childdocforwardprefix{#1}{#2}}
    \fi
    \expandafter
  \endgroup
  \childdoctmp
}
%    \end{macrocode}

%\iffalse
%</package>
%\fi
%
\endinput
|\\
|\childdocforwardprefix[|\textit{main}|]{|\textit{prefix}|}{|\textit{dest}|}|
\end{tabular}
\end{center}
%
the destination file is determined by a pattern
depending on the current file:
To make this work, the current file must be called
`{\textit{prefix}\hspace{0.2em}\textit{suffix}}'
with \textit{prefix} matching precisely the argument.
Processing is then passed on to the file
`{\textit{dest}\hspace{0.2em}\textit{suffix}}'.
Surely, the same effect is achieved by
directly specifying the
argument `{\textit{dest}\hspace{0.2em}\textit{suffix}}'
in the first form.
However, that requires to set up a different file
for each child. With the alternative form of the command
all these files can have exactly the same content
which simplifies setting them up and maintaining them.

For example, the following file |draft.tex|
with a compilation flag |\version| as described in \secref{sec:flags}
compiles the main document as a draft:
%
\begin{center}
\begin{tabular}{l}
|\def\version{draft}|\\
|% \iffalse
%
% childdoc.dtx Copyright (C) 2017-2018 Niklas Beisert
%
% This work may be distributed and/or modified under the
% conditions of the LaTeX Project Public License, either version 1.3
% of this license or (at your option) any later version.
% The latest version of this license is in
%   http://www.latex-project.org/lppl.txt
% and version 1.3 or later is part of all distributions of LaTeX
% version 2005/12/01 or later.
%
% This work has the LPPL maintenance status `maintained'.
%
% The Current Maintainer of this work is Niklas Beisert.
%
% This work consists of the files childdoc.dtx and childdoc.ins
% and the derived files childdoc.def and cdocsamp.tex with
% cdocsch1.tex, cdocsch2.tex, cdocsdrf.tex, cdocsfn1.tex, cdocsfn2.tex.
%
%<package>\ifdefined\childdocmain\endinput\fi
%<package>\ProvidesFile{childdoc.def}[2018/12/30 v2.0 child document driver]
%<samplemain>\ProvidesFile{cdocsamp.tex}[2018/12/30 v2.0 sample for childdoc]
%<*driver>
%\ProvidesFile{childdoc.drv}[2018/12/30 v2.0 childdoc reference manual file]
\PassOptionsToClass{10pt,a4paper}{article}
\documentclass{ltxdoc}

\usepackage[margin=35mm]{geometry}
\usepackage{hyperref}
\usepackage{hyperxmp}
\usepackage[usenames]{color}

\hypersetup{colorlinks=true}
\hypersetup{pdfstartview=FitH}
\hypersetup{pdfpagemode=UseNone}
\hypersetup{pdfsource={}}
\hypersetup{pdflang={en-UK}}
\hypersetup{pdfcopyright={Copyright 2017-2018 Niklas Beisert.
  This work may be distributed and/or modified under the
  conditions of the LaTeX Project Public License, either version 1.3
  of this license or (at your option) any later version.}}
\hypersetup{pdflicenseurl={http://www.latex-project.org/lppl.txt}}
\hypersetup{pdfcontactaddress={ETH Zurich, ITP, HIT K,
  Wolfgang-Pauli-Strasse 27}}
\hypersetup{pdfcontactpostcode={8093}}
\hypersetup{pdfcontactcity={Zurich}}
\hypersetup{pdfcontactcountry={Switzerland}}
\hypersetup{pdfcontactemail={nbeisert@itp.phys.ethz.ch}}
\hypersetup{pdfcontacturl={http://people.phys.ethz.ch/\xmptilde nbeisert/}}

\newcommand{\secref}[1]{\hyperref[#1]{section \ref*{#1}}}

\parskip1ex
\parindent0pt
\let\olditemize\itemize
\def\itemize{\olditemize\parskip0pt}

\begin{document}

\title{The \textsf{childdoc} Package}
\hypersetup{pdftitle={The childdoc Package}}
\author{Niklas Beisert\\[2ex]
  Institut f\"ur Theoretische Physik\\
  Eidgen\"ossische Technische Hochschule Z\"urich\\
  Wolfgang-Pauli-Strasse 27, 8093 Z\"urich, Switzerland\\[1ex]
  \href{mailto:nbeisert@itp.phys.ethz.ch}
  {\texttt{nbeisert@itp.phys.ethz.ch}}}
\hypersetup{pdfauthor={Niklas Beisert}}
\hypersetup{pdfsubject={Manual for the LaTeX2e Package childdoc}}
\date{30 December 2018, \textsf{v2.0}}
\maketitle

\begin{abstract}\noindent
\textsf{childdoc} is a \LaTeXe{} package
that enables the direct compilation
of document sections included by |\include|
to individual files.
\end{abstract}

\begingroup
\parskip0ex
\tableofcontents
\endgroup

%%%%%%%%%%%%%%%%%%%%%%%%%%%%%%%%%%%%%%%%%%%%%%%%%%%%%%%%%%%%%%%%%%%%%%%%%%%%%%%%
%%%%%%%%%%%%%%%%%%%%%%%%%%%%%%%%%%%%%%%%%%%%%%%%%%%%%%%%%%%%%%%%%%%%%%%%%%%%%%%%
\section{Introduction}

\LaTeX{} provides a mechanism to structure a large document (such as a book)
into a main file and several child files (containing the chapters)
using the |\include| command.
This mechanism is beneficial for documents
which span hundreds of pages in order to
make the source file(s) more manageable.
Moreover, compilation can be restricted to
selected child files by means of the |\includeonly| command.
The latter feature can be used to reduce the compilation time while editing
(this was significantly more useful in the earlier days of \LaTeX{})
or to generate a smaller document which is easier to navigate.
Another application of |\includeonly| is to generate
documents consisting of selected parts of the complete document.

However, there are a few drawbacks of the plain |\include| mechanism:
\begin{itemize}
\item
The child files cannot be compiled on their own,
they can only be compiled via the main file.
A naive editing environment
(such as a text editor with an option
to have the current file processed by \LaTeX)
may require one to switch to the main file before compiling;
attempting to compile the child file produces errors.
\item
The main file must be modified (each time)
to adjust the |\includeonly| command
to the present needs. This easily leaves the main file in a messy state.
\item
The generated document will always carry the filename
of the main document. This is inconvenient if
several child files are to be compiled and
to be kept for distribution.
\end{itemize}

The present package provides a simple interface
to make child files individually compilable by \LaTeX{}.
Compiling a child file then has the same effect as compiling
the main file with an |\includeonly| command
to select the appropriate child.
Moreover the generated document will carry the name of the child
rather than the main file.
This resolves all three above issues.

This feature is meant to make the editing of books,
thesis documents and lecture notes somewhat more convenient.
However, the package can also be used efficiently for
composing a series of documents (such as exercise sheets)
which are typically distributed individually.
It then assists the author in generating the individual documents
(potentially in different versions)
as well as a document containing the collected series.
Another application is in developing style files
or other kinds of included material
where compilation of the style file could redirect
to a sample or test file.

%%%%%%%%%%%%%%%%%%%%%%%%%%%%%%%%%%%%%%%%%%%%%%%%%%%%%%%%%%%%%%%%%%%%%%%%%%%%%%%%
%%%%%%%%%%%%%%%%%%%%%%%%%%%%%%%%%%%%%%%%%%%%%%%%%%%%%%%%%%%%%%%%%%%%%%%%%%%%%%%%
\section{Usage}

First of all, the package \textsf{childdoc} is \emph{not} a standard
\LaTeXe{} |.sty| style file! Therefore it needs to be invoked in
a non-standard way.

%%%%%%%%%%%%%%%%%%%%%%%%%%%%%%%%%%%%%%%%%%%%%%%%%%%%%%%%%%%%%%%%%%%%%%%%%%%%%%%%
\subsection{Included Files}
\label{sec:include}

%%%%%%%%%%%%%%%%%%%%%%%%%%%%%%%%%%%%%%%%
\DescribeMacro{\childdocmain}
To use the package, add the commands
\begin{center}
\begin{tabular}{l}
|\input{childdoc.def}|\\
|\childdocmain{}|\\
\end{tabular}
\end{center}
at the very top of the main \LaTeX{} file,
in particular \emph{before} the |\documentclass| statement!
The argument of |\childdocmain| should be left empty
(but it must be present).

%%%%%%%%%%%%%%%%%%%%%%%%%%%%%%%%%%%%%%%%
\DescribeMacro{\childdocof}
Furthermore, add the commands
\begin{center}
\begin{tabular}{l}
|\input{childdoc.def}|\\
|\childdocof{|\textit{main}|}|\\
\end{tabular}
\end{center}
at the top of every child file \textit{child}
which is included by |\include{|\textit{child}|}|
from within the main file
(or at least for those files to be compiled individually).
The argument \textit{main} must be the filename of the main file.

There are a couple of
considerations in setting up the main and child documents:

%%%%%%%%%%%%%%%%%%%%%%%%%%%%%%%%%%%%%%%%
\paragraph{Restrictions.}

Please note the following restrictions:
\begin{itemize}
\item
|\childdocmain| must be called with one argument \textit{main}
to ensure compatibility with earlier version of the package.
It must either be empty (|\childdocmain{}|)
or precisely match the filename of the main file in which it is specified.
See \secref{sec:detection} for further information.
\item
The filename \textit{main} must be specified without the |.tex| extension.
\item
The filename \textit{main} is case sensitive
(even in case-insensitive file systems)
due to internal string comparison.
\item
The argument \textit{main} should be fully expanded, it cannot be a macro.
\item
Subdirectories and special characters should be avoided in filenames.
\item
The command |\childdocmain{|\textit{main}|}| must be followed by a whitespace.
It should not be followed immediately by another command
or by a comment mark `|%|'.
This is because the \TeX{} parser reads the token immediately following
the argument of |\childdocmain| and puts it
at the beginning of every child section;
however, a white\-space is ignored.
\end{itemize}

%%%%%%%%%%%%%%%%%%%%%%%%%%%%%%%%%%%%%%%%
\paragraph{Content of Main File.}

It is advisable to place all content in the child files included by |\include|.
Any output contained in the main file will appear in all child documents
unless suppressed manually;
it cannot be suppressed automatically by the |\includeonly| directive
and thus should normally be avoided.
A method to include some content in the main file
by means of conditional processing is described in \secref{sec:conditional}.

%%%%%%%%%%%%%%%%%%%%%%%%%%%%%%%%%%%%%%%%
\paragraph{Page Numbering.}

When only a part of the document is compiled,
the appropriate numbering of pages
(as well as other status parameters)
is determined from the |.aux| files.
The latter contain information from previous passes.
However this information needs to propagate through
all intermediate child documents.
Therefore the page numbering in child documents may well
be inconsistent until the complete document is compiled at least once.

A useful (if unconventional) way to always ensure a consistent
page numbering is to restart the numbering in each child document
and denote the pages by `\textit{child}|.|\textit{page}'
where \textit{child} represents the chapter/section number of the child file.
This can be achieved by the command
|\numberwithin{page}{|\textit{child}|}|
of the \textsf{amsmath} package
where \textit{child} can be |chapter| or |section|
depending on the chosen structuring.
Alternatively, one can modify the macro |\thepage| appropriately
and reset the counter |page| at the start of each child file.

%%%%%%%%%%%%%%%%%%%%%%%%%%%%%%%%%%%%%%%%%%%%%%%%%%%%%%%%%%%%%%%%%%%%%%%%%%%%%%%%
\subsection{Conditional Processing}
\label{sec:conditional}

The package provides a mechanism to compile different versions
of a document. To customise the versions further some conditional processing
can come in handy to distinguish which version is being compiled.
The package provides two macros to describe the compilation context:

%%%%%%%%%%%%%%%%%%%%%%%%%%%%%%%%%%%%%%%%
\DescribeMacro{\ifchilddoc}
The conditional |\ifchilddoc| distinguishes between the compilation of
child documents and the main document:
%
\begin{center}
|\ifchilddoc |\textit{child-code}| |[|\||else |\textit{main-code}]| \||fi|
\end{center}

%%%%%%%%%%%%%%%%%%%%%%%%%%%%%%%%%%%%%%%%
\DescribeMacro{\childdocname}
\DescribeMacro{\childdocjob}
The macro |\childdocname| contains the filename (without extension)
of the main or child file being processed.
Note that |\childdocjob| will always contain the name of the main file.

%%%%%%%%%%%%%%%%%%%%%%%%%%%%%%%%%%%%%%%%
\paragraph{Title Page.}

Conditional processing can be used to include a title or banner page
in the main document when proper precautions are taken.
Importantly, the code in the main file should ensure that the page counter
(as well as other status parameters which are stored in the |.aux| files)
takes the same value after the conditional processing.
Otherwise the page numbers may take divergent values
depending on which part is compiled.

For example, a title page could be declared by:
%
\begin{center}
\begin{tabular}{l}
|\ifchilddoc\||else|\\
|\addtocounter{page}{-1}|\\
\textit{code for title page}\\
|\newpage|\\
|\||fi|
\end{tabular}
\end{center}
%
A banner page for the child documents can be generated by:
%
\begin{center}
\begin{tabular}{l}
|\ifchilddoc|\\
|\addtocounter{page}{-1}|\\
\textit{code for banner page}\\
|\newpage|\\
|\||fi|
\end{tabular}
\end{center}
%
Here one could write a message such as:
\begin{center}
|This is the part \childdocname{} of \childdocjob{}.|
\end{center}

%%%%%%%%%%%%%%%%%%%%%%%%%%%%%%%%%%%%%%%%%%%%%%%%%%%%%%%%%%%%%%%%%%%%%%%%%%%%%%%%
\subsection{Flags}
\label{sec:flags}

The package makes it easy to generate different versions
of the main or child documents.
To this end compilation flags can be defined
and assigned different default values.
They will be particularly useful in conjunction
with the forwarding mechanism described in \secref{sec:forward}.

For example, it may be useful to have a flag |\version|
which can be set to |draft| or |final|.
The document source will contain some conditional code
depending on the value of |\version|.
Suppose further, the flag should default to |final| for the main file
and to |draft| for child files
which is a natural assignment for editing the document.
This is achieved by placing the following code
in the preamble of the main document
(below the |\childdocmain| directive):
%
\begin{center}
\begin{tabular}{l}
|\ifchilddoc|\\
|\providecommand{\version}{draft}|\\
|\||else|\\
|\providecommand{\version}{final}|\\
|\||fi|
\end{tabular}
\end{center}
%
The definition by |\providecommand| makes sure
that previous definitions are not overwritten.
Further statements |\providecommand{\version}{...}|
can thus be added before the above code to override it.

For the main file, one might add a line
(between |\childdocmain| and the above block)
%
\begin{center}
|%\ifchilddoc\||else\providecommand{\version}{draft}\||fi|
\end{center}
%
which can be uncommented to produce a draft version.
Likewise one can add a line to the very top of a child file
(above the |\childdocof{|\textit{main}|}| directive)
%
\begin{center}
|%\providecommand{\version}{final}|
\end{center}
%
which can be uncommented to produce the final version of this child document.

%%%%%%%%%%%%%%%%%%%%%%%%%%%%%%%%%%%%%%%%%%%%%%%%%%%%%%%%%%%%%%%%%%%%%%%%%%%%%%%%
\subsection{Forwarding}
\label{sec:forward}

Different versions of the main or child documents
using compilation flags as described in \secref{sec:flags}
can be (permanently) stored in different files
for convenient compilation, viewing and distribution.
To this end, the package defines a command
to pass on compilation to a different file:

%%%%%%%%%%%%%%%%%%%%%%%%%%%%%%%%%%%%%%%%
\DescribeMacro{\childdocforward}
The command |\childdocforward| redirects processing to
another source file:
%
\begin{center}
\begin{tabular}{l}
|\input{childdoc.def}|\\
|\childdocforward[|\textit{main}|]{|\textit{dest}|}|\\
\end{tabular}
\end{center}
%
The argument \textit{dest} is the destination file
(without extension).
It should be the main file or one of the child files.
Note that further \textsf{childdoc} directives
such as |\childdocof| and |\childdocforward|
in the indicated file will be processed in this form.
The optional argument \textit{main}
passes on directly to the main file \textit{main}
while pretending to compile the child \textit{dest}.
This form behaves as if \textit{dest}
issues |\childdocof{|\textit{main}|}| right away,
and no further \textsf{childdoc} directives will be processed.

%%%%%%%%%%%%%%%%%%%%%%%%%%%%%%%%%%%%%%%%
\DescribeMacro{\...prefix}
In the alternative form |\childdocforwardprefix|,
%
\begin{center}
\begin{tabular}{l}
|\input{childdoc.def}|\\
|\childdocforwardprefix[|\textit{main}|]{|\textit{prefix}|}{|\textit{dest}|}|
\end{tabular}
\end{center}
%
the destination file is determined by a pattern
depending on the current file:
To make this work, the current file must be called
`{\textit{prefix}\hspace{0.2em}\textit{suffix}}'
with \textit{prefix} matching precisely the argument.
Processing is then passed on to the file
`{\textit{dest}\hspace{0.2em}\textit{suffix}}'.
Surely, the same effect is achieved by
directly specifying the
argument `{\textit{dest}\hspace{0.2em}\textit{suffix}}'
in the first form.
However, that requires to set up a different file
for each child. With the alternative form of the command
all these files can have exactly the same content
which simplifies setting them up and maintaining them.

For example, the following file |draft.tex|
with a compilation flag |\version| as described in \secref{sec:flags}
compiles the main document as a draft:
%
\begin{center}
\begin{tabular}{l}
|\def\version{draft}|\\
|\input{childdoc.def}|\\
|\childdocforward{|\textit{main}|}|
\end{tabular}
\end{center}
%
Likewise, the following files |final|\textit{nn}|.tex|
compile the final version of the child document
|child|\textit{nn}|.tex|:
%
\begin{center}
\begin{tabular}{l}
|\def\version{final}|\\
|\input{childdoc.def}|\\
|\childdocforwardprefix{final}{child}|
\end{tabular}
\end{center}
%

Note that when several versions of a main file and/or of each child file
are to be generated, it may be convenient to set up a |Makefile| or
shell script to automatise the process.

%%%%%%%%%%%%%%%%%%%%%%%%%%%%%%%%%%%%%%%%%%%%%%%%%%%%%%%%%%%%%%%%%%%%%%%%%%%%%%%%
\subsection{Command Line Processing}
\label{sec:commandline}

The effect of redirection files can also be achieved by invoking
the \LaTeX{} compiler with a more elaborate command line.
Most conveniently this should be done as part
of a shell script or a |Makefile|.

When using \textsf{childdoc} in the main file, the following
command lines effectively perform a redirection
(note that depending on the shell being used,
backslashes may have to be doubled: `|\|' $\to$ `|\\|'):
%
\begin{center}
|... -jobname "|\textit{target}|" |\\|"|[\textit{flags}]%
|\input{childdoc.def}\childdocforward[|\textit{main}|]{|\textit{dest}|}"|
\end{center}
%
Here \textit{target} is the name of the output file,
\textit{main} is the name of the main file
and \textit{dest} is the name of the main or child file to be processed
(all filenames without extensions).
The optional argument \textit{main} can be omitted
if \textit{main} matches \textit{dest}.
Optionally, compilation \textit{flags} can be defined via |\def| commands.
This command line makes the \TeX{} engine believe
it is compiling the file \textit{target}
whose content is specified as the latter parameter.
The provided code then forwards the processing to
\textit{main} or \textit{dest} as described in \secref{sec:forward}.

%%%%%%%%%%%%%%%%%%%%%%%%%%%%%%%%%%%%%%%%%%%%%%%%%%%%%%%%%%%%%%%%%%%%%%%%%%%%%%%%
\subsection{Include by Input}
\label{sec:input}

Including child documents by |\include| has some restrictions by design.
Most notably, the content of a child document always occupies
its own set of pages; pages cannot be shared between child documents.
Usually, this behaviour makes perfect sense
because each child document contain an essential part of the document.
However, in some situations it may be desirable to compose
a document from a collection of parts
without having mandatory page breaks between then.
For this case, the package
provides a mechanism to include parts
by |\input| which can also be processed individually.
However, by construction this mechanism
requires manual handling of the content to be output.

%%%%%%%%%%%%%%%%%%%%%%%%%%%%%%%%%%%%%%%%
\DescribeMacro{\ifchilddocmanual}
The main file should be prepared as usual, see \secref{sec:include}.
However, the document body must make a distinction
between processing of an individual part and of the main document, e.g.:
%
\begin{center}
\begin{tabular}{l}
|\ifchilddocmanual|\\
|\input{\childdocname}|\\
|\||else|\\
\textit{document body with }|\input{|\textit{part}|}|\\
|\||fi|
\end{tabular}
\end{center}
%
The conditional |\ifchilddocmanual| is true whenever
a part to be included by |\input| is being compiled,
and the name of the part is stored in |\childdocname|.

%%%%%%%%%%%%%%%%%%%%%%%%%%%%%%%%%%%%%%%%
\DescribeMacro{\childdocby}
Each part to be included by |\input| should start with:
%
\begin{center}
\begin{tabular}{l}
|\input{childdoc.def}|\\
|\childdocby{|\textit{main}|}|\\
\end{tabular}
\end{center}
%
The directive |\childdocby| is similar to |\childdocof|
described in \secref{sec:include},
but the subsequent selection of content must be done manually.
To that end, both |\ifchilddoc| and |\ifchilddocmanual|
will be true upon processing of a part,
and the name of the part is stored in |\childdocname|.
Note that |\jobname| will be set to the filename of the current part
so that each part receives an individual |.aux| file
that does not interfere with the |.aux| file(s) of the main document.
This behaviour can be altered by the alternative form
|\childdocby[*]{|\textit{main}|}| (with a non-empty optional argument)
which uses the |.aux| file of the main document
by setting |\jobname| to \textit{main}.

%%%%%%%%%%%%%%%%%%%%%%%%%%%%%%%%%%%%%%%%%%%%%%%%%%%%%%%%%%%%%%%%%%%%%%%%%%%%%%%%
\subsection{Driver Development}
\label{sec:driver}

The \textsf{childdoc} mechanism can also be use for the development
of definition files such as \LaTeX{} styles or classes.
This case differs from the above setup with multiple parts
included by |\include| in that no |\includeonly| should be invoked.
This can be achieved by starting the include file
(before |\ProvidesPackage|) with:
%
\begin{center}
\begin{tabular}{l}
|\input{childdoc.def}|\\
|\childdocforward{|\textit{main}|}|\\
\end{tabular}
\end{center}
%
or alternatively with:
%
\begin{center}
\begin{tabular}{l}
|\input{childdoc.def}|\\
|\childdocby{|\textit{main}|}|\\
\end{tabular}
\end{center}
%
Both forms have slightly different effects as described above.
The main file is prepared as usual, see \secref{sec:include}.

%%%%%%%%%%%%%%%%%%%%%%%%%%%%%%%%%%%%%%%%%%%%%%%%%%%%%%%%%%%%%%%%%%%%%%%%%%%%%%%%
\subsection{Legacy Detection}
\label{sec:detection}

The directive |\childdocmain| in the main file can detect
whether the complete document or merely a child is to be compiled
even without using the directive |\childdocof|.
This method is deprecated because it is less robust
and there is no compelling reason to use it;
it is merely provided for backward compatibility
and it may be removed in future versions.

If the detection mechanism is to be used,
it is mandatory to correctly specify
the filename of the main file as the argument of |\childdocmain|:
%
\begin{center}
\begin{tabular}{l}
|\input{childdoc.def}|\\
|\childdocmain{|\textit{main}|}|\\
\end{tabular}
\end{center}
%
If |\jobname| does not match the argument \textit{main} of |\childdocmain|,
it is assumed that |\jobname| points to the child file to be compiled.
When using |\childdocmain| with the main file specified as argument,
it suffices to start a child file
with just |\input{|\textit{main}|}|
without loading of the package and using |\childdocof|.
If instead all processing is done
with the appropriate \textsf{childdoc} directives,
the argument of \textit{main} of |\childdocmain| can be empty.

An alternative version of the command line processing described
in \secref{sec:commandline} using the detection mechanism reads:
%
\begin{center}
|... -jobname "|\textit{target}|" "|[\textit{flags}]%
[|\def\jobname{|\textit{dest}|}|]|\input{|\textit{main}|}"|
\end{center}

%%%%%%%%%%%%%%%%%%%%%%%%%%%%%%%%%%%%%%%%%%%%%%%%%%%%%%%%%%%%%%%%%%%%%%%%%%%%%%%%
\subsection{Manual Code}
\label{sec:manual}

In case one cannot be certain whether the definitions file |childdoc.def|
is installed on the target \TeX{} distribution
and one prefers not to ship it,
it is conceivable to paste a few relevant commands into the sources.

To that end, drop all statements |\input{childdoc.def}|
and perform the replacements as outlined below.
Instead of |\childdocmain{|\textit{main}|}| add the following code
to the top of the main file:
%
\begin{center}
\begin{tabular}{l}
|\||ifdefined\childdocname\endinput\||fi\newif\ifchilddoc|\\
|\edef\childdocname{\scantokens\expandafter{\jobname\noexpand}}|\\
|\def\childdocmain{|\textit{main}|}\||ifx\childdocmain\childdocname\||else|\\
|\childdoctrue\includeonly{\childdocname}\let\jobname\childdocmain\||fi|\\
\end{tabular}
\end{center}
%
Instead of |\childdocof{|\textit{main}|}| just include the main file
at the top of each child file:
%
\begin{center}
|\input{|\textit{main}|}|
\end{center}
%
A simple redirection |\childdocforward{|\textit{dest}|}| is achieved by:
%
\begin{center}
|\def\jobname{|\textit{dest}|}\input{\jobname}|
\end{center}
%
The redirection with prefix
|\childdocforwardprefix[|\textit{prefix}|]{|\textit{dest}|}|
is accomplished by:
%
\begin{center}
\begin{tabular}{l}
|{\edef\jobname{\scantokens\expandafter{\jobname\noexpand}}|\\
|\def\redirectjob |\textit{prefix}|#1~~~{\gdef\jobname{|\textit{dest}|#1}}|\\
|\expandafter\redirectjob\jobname~~~}\input{\jobname}|
\end{tabular}
\end{center}

In an alternative approach,
child documents can be compiled by a specific command line
without additional code or specific definitions:
%
\begin{center}
|... -jobname "|\textit{target}|" "|[\textit{flags}]%
|\includeonly{|\textit{dest}|}\input{|\textit{main}|}"|
\end{center}
%

%%%%%%%%%%%%%%%%%%%%%%%%%%%%%%%%%%%%%%%%%%%%%%%%%%%%%%%%%%%%%%%%%%%%%%%%%%%%%%%%
%%%%%%%%%%%%%%%%%%%%%%%%%%%%%%%%%%%%%%%%%%%%%%%%%%%%%%%%%%%%%%%%%%%%%%%%%%%%%%%%
\section{Information}

%%%%%%%%%%%%%%%%%%%%%%%%%%%%%%%%%%%%%%%%%%%%%%%%%%%%%%%%%%%%%%%%%%%%%%%%%%%%%%%%
\subsection{Copyright}

Copyright \copyright{} 2017--2018 Niklas Beisert

This work may be distributed and/or modified under the
conditions of the \LaTeX{} Project Public License, either version 1.3
of this license or (at your option) any later version.
The latest version of this license is in
  \url{http://www.latex-project.org/lppl.txt}
and version 1.3 or later is part of all distributions of \LaTeX{}
version 2005/12/01 or later.

This work has the LPPL maintenance status `maintained'.

The Current Maintainer of this work is Niklas Beisert.

This work consists of the files |README.txt|, |childdoc.ins| and |childdoc.dtx|
as well as the derived files |childdoc.def|, |cdocsamp.tex|
with |cdocsch1.tex|, |cdocsch2.tex|, |cdocspt3.tex|, |cdocspt4.tex|,
|cdocsdrf.tex|, |cdocsfn1.tex|, |cdocsfn2.tex|
as well as |childdoc.pdf|.

%%%%%%%%%%%%%%%%%%%%%%%%%%%%%%%%%%%%%%%%%%%%%%%%%%%%%%%%%%%%%%%%%%%%%%%%%%%%%%%%
\subsection{Files and Installation}

The package consists of the files:
%
\begin{center}
\begin{tabular}{ll}
    |README.txt|   & readme file \\
    |childdoc.ins| & installation file \\
    |childdoc.dtx| & source file \\
    |childdoc.def| & definition file \\
    |cdocsamp.tex| & sample main file \\
    |cdocsch1.tex| & sample include file \\
    |cdocsch2.tex| & sample include file \\
    |cdocspt3.tex| & sample part file \\
    |cdocspt4.tex| & sample part file \\
    |cdocsdrf.tex| & sample redirection file \\
    |cdocsfn1.tex| & sample redirection file \\
    |cdocsfn2.tex| & sample redirection file \\
    |childdoc.pdf| & manual
\end{tabular}
\end{center}
%
The distribution consists of the files
|README.txt|, |childdoc.ins| and |childdoc.dtx|.
%
\begin{itemize}
\item
Run (pdf)\LaTeX{} on |childdoc.dtx|
to compile the manual |childdoc.pdf| (this file).
\item
Run \LaTeX{} on |childdoc.ins| to create the definitions file |childdoc.def|
and the sample |cdocsamp.tex| with include files
|cdocsch1.tex|, |cdocsch2.tex|, |cdocspt3.tex|, |cdocspt4.tex|,
|cdocsdrf.tex|, |cdocsfn1.tex|, |cdocsfn2.tex|.
Then copy the file |childdoc.def| to an appropriate directory of your \LaTeX{}
distribution, e.g.\ \textit{texmf-root}|/tex/latex/childdoc|.
\end{itemize}

%%%%%%%%%%%%%%%%%%%%%%%%%%%%%%%%%%%%%%%%%%%%%%%%%%%%%%%%%%%%%%%%%%%%%%%%%%%%%%%%
\subsection{Related CTAN Packages}

There are several other packages which offer a similar functionality:
%
\begin{itemize}
\item
The packages
\href{http://ctan.org/pkg/docmute}{\textsf{docmute}},
\href{http://ctan.org/pkg/includex}{\textsf{includex}} and
\href{http://ctan.org/pkg/standalone}{\textsf{standalone}}
provide commands to include only the document body of
a child file thus allowing both files to be compiled individually.
\item
The packages \href{http://ctan.org/pkg/subdocs}{\textsf{subdocs}}
and \href{http://ctan.org/pkg/subfiles}{\textsf{subfiles}}
provide structures in which the main and child documents can be
encapsulated and allowing them to be compiled individually.
The inclusion mechanism is different from the conventional |\include|.
\item
The package \href{http://ctan.org/pkg/combine}{\textsf{combine}}
is an elaborate solution to combine several documents into one.
\end{itemize}
%
See also the CTAN topic \href{http://ctan.org/topic/subdocs}{\textsf{subdocs}}
for further related packages.
The present package differs from the above solutions in that
a document structure constructed with the conventional |\include| mechanism
just needs two extra commands at the top of every file
such that all constituent files can be compiled individually.

%%%%%%%%%%%%%%%%%%%%%%%%%%%%%%%%%%%%%%%%%%%%%%%%%%%%%%%%%%%%%%%%%%%%%%%%%%%%%%%%
%\subsection{Feature Suggestions}
%
%The following is a list of features which may be useful for future
%versions of this package:
%%
%\begin{itemize}
%\item
%\ldots
%\end{itemize}

%%%%%%%%%%%%%%%%%%%%%%%%%%%%%%%%%%%%%%%%%%%%%%%%%%%%%%%%%%%%%%%%%%%%%%%%%%%%%%%%
\subsection{Revision History}

%%%%%%%%%%%%%%%%%%%%%%%%%%%%%%%%%%%%%%%%
\paragraph{v2.0:} 2018/12/30

\begin{itemize}
\item
immediate forward processing
\item
added |\childdocby| mechanism
\item
manual restructured
\end{itemize}

%%%%%%%%%%%%%%%%%%%%%%%%%%%%%%%%%%%%%%%%
\paragraph{v1.6:} 2018/01/17

\begin{itemize}
\item
application for development of include files
\item
corrections to manual
\end{itemize}

%%%%%%%%%%%%%%%%%%%%%%%%%%%%%%%%%%%%%%%%
\paragraph{v1.5:} 2017/05/21

\begin{itemize}
\item
more complete structuring introduced
\item
|\childdocof| introduced
\item
|\childdoc| renamed to |\childdocmain|
\item
|\childredirect| renamed to |\childdocforward| and |\childdocforwardprefix|
and functionality expanded
\end{itemize}

%%%%%%%%%%%%%%%%%%%%%%%%%%%%%%%%%%%%%%%%
\paragraph{v1.0:} 2017/04/27

\begin{itemize}
\item
manual and install package
\item
first version published on CTAN
\end{itemize}

%%%%%%%%%%%%%%%%%%%%%%%%%%%%%%%%%%%%%%%%
\paragraph{v0.6:} 2017/04/26

\begin{itemize}
\item
redirection mechanism added
\end{itemize}

%%%%%%%%%%%%%%%%%%%%%%%%%%%%%%%%%%%%%%%%
\paragraph{v0.5:} 2017/04/26

\begin{itemize}
\item
functionality in definition file
\end{itemize}


%%%%%%%%%%%%%%%%%%%%%%%%%%%%%%%%%%%%%%%%%%%%%%%%%%%%%%%%%%%%%%%%%%%%%%%%%%%%%%%%
%%%%%%%%%%%%%%%%%%%%%%%%%%%%%%%%%%%%%%%%%%%%%%%%%%%%%%%%%%%%%%%%%%%%%%%%%%%%%%%%
%%%%%%%%%%%%%%%%%%%%%%%%%%%%%%%%%%%%%%%%%%%%%%%%%%%%%%%%%%%%%%%%%%%%%%%%%%%%%%%%
\appendix

\settowidth\MacroIndent{\rmfamily\scriptsize 000\ }

 \DocInput{childdoc.dtx}

\end{document}
%</driver>
% \fi
%
% %%%%%%%%%%%%%%%%%%%%%%%%%%%%%%%%%%%%%%%%%%%%%%%%%%%%%%%%%%%%%%%%%%%%%%%%%%%%%%
% %%%%%%%%%%%%%%%%%%%%%%%%%%%%%%%%%%%%%%%%%%%%%%%%%%%%%%%%%%%%%%%%%%%%%%%%%%%%%%
% \section{Sample}
%\iffalse
%<*samplemain>
%\fi
%
% The following presents a sample document
% with two chapters, two parts, a title page,
% a compile flag as well as three forwarding files to set the flag.
% It consists of eight |.tex| files:
% \begin{center}
% \begin{tabular}{ll}
% |cdocsamp.tex|&main file\\
% |cdocsch1.tex|&include file for chapter 1\\
% |cdocsch2.tex|&include file for chapter 2\\
% |cdocspt3.tex|&include file for part 3\\
% |cdocspt4.tex|&include file for part 4\\
% |cdocsdrf.tex|&forwarding file for main file in draft mode\\
% |cdocsfi1.tex|&forwarding file for final version of chapter 1\\
% |cdocsfi2.tex|&forwarding file for final version of chapter 2\\
% \end{tabular}
% \end{center}
% Each of the eight files can be compiled directly by the \LaTeX{} compiler.
%
% %%%%%%%%%%%%%%%%%%%%%%%%%%%%%%%%%%%%%%
% \paragraph{Main File.}
%
% The main file is called |cdocsamp.tex|.
%
% Load the \textsf{childdoc} definitions and
% declare the filename for the main document:
%    \begin{macrocode}
\input{childdoc.def}
\childdocmain{}
%    \end{macrocode}

% Optional override for |\version| flag:
%    \begin{macrocode}
%%\ifchilddoc\else\providecommand{\version}{draft}\fi
%    \end{macrocode}

% Define the default values for the |\version| flag
% (|final| for the main file and |draft| for childs):
%    \begin{macrocode}
\ifchilddoc
\providecommand{\version}{draft}
\else
\providecommand{\version}{final}
\fi
%    \end{macrocode}

% Load the standard document class:
%    \begin{macrocode}
\documentclass[12pt]{article}
%    \end{macrocode}

% Start the document body:
%    \begin{macrocode}
\begin{document}
%    \end{macrocode}

% Declare a title page.
% Print title, part of document being processed and version flag:
%    \begin{macrocode}
\addtocounter{page}{-1}
\begin{center}
{\LARGE\bfseries{}childdoc example\par}
\vspace{1cm}
\ifchilddoc
\ifchilddocmanual part\else chapter\fi:
`\childdocname' of `\childdocjob'\par
\else
main document: `\childdocjob'\par
\fi
version: \version\par
\end{center}
\newpage
%    \end{macrocode}

% Manually include selected file,
% otherwise process as usual:
%    \begin{macrocode}
\ifchilddocmanual
\section*{part `\childdocname'}
\input{\childdocname}
\else
%    \end{macrocode}

% Include the two chapters:
%    \begin{macrocode}
\include{cdocsch1}
\include{cdocsch2}
%    \end{macrocode}

% Include the two parts unless only chapters should be displayed:
%    \begin{macrocode}
\ifchilddoc\else
\section{part three}
\input{cdocspt3}
\section{part four}
\input{cdocspt4}
\fi
%    \end{macrocode}

% Process as usual until here:
%    \begin{macrocode}
\fi
%    \end{macrocode}

% End of document body:
%    \begin{macrocode}
\end{document}
%    \end{macrocode}
%\iffalse
%</samplemain>
%\fi
%
% %%%%%%%%%%%%%%%%%%%%%%%%%%%%%%%%%%%%%%
% \paragraph{Chapter Include Files.}
%
% The include files are called |cdocsch1.tex| and |cdocsch2.tex|.
%
%\iffalse
%<*samplechap1|samplechap2>
%\fi

% Optional override for |\version| flag:
%    \begin{macrocode}
%%\providecommand{\version}{final}
%    \end{macrocode}

% Include the main document:
%    \begin{macrocode}
\input{childdoc.def}
\childdocof{cdocsamp}
%    \end{macrocode}

%\iffalse
%</samplechap1|samplechap2>
%\fi
%
%\iffalse
%<*samplechap1>
%\fi
% Some text for chapter 1:
%    \begin{macrocode}
\section{one}
some text in chapter one
%    \end{macrocode}

%\iffalse
%</samplechap1>
%\fi
% Some text for chapter 2:
%\iffalse
%<*samplechap2>
%\fi
%    \begin{macrocode}
\section{two}
more text in chapter two
%    \end{macrocode}

%\iffalse
%</samplechap2>
%\fi
%
% %%%%%%%%%%%%%%%%%%%%%%%%%%%%%%%%%%%%%%
% \paragraph{Part Include Files.}
%
% The include files are called |cdocspt3.tex| and |cdocspt4.tex|.
%
%\iffalse
%<*samplepart3|samplepart4>
%\fi

% Optional override for |\version| flag:
%    \begin{macrocode}
%%\providecommand{\version}{final}
%    \end{macrocode}

% Include the main document:
%    \begin{macrocode}
\input{childdoc.def}
\childdocby{cdocsamp}
%    \end{macrocode}

%\iffalse
%</samplepart3|samplepart4>
%\fi
%
%\iffalse
%<*samplepart3>
%\fi
% Some text for part 3:
%    \begin{macrocode}
some text in part three
%    \end{macrocode}

%\iffalse
%</samplepart3>
%\fi
% Some text for part 4:
%\iffalse
%<*samplepart4>
%\fi
%    \begin{macrocode}
more text in part four
%    \end{macrocode}

%\iffalse
%</samplepart4>
%\fi
%
% %%%%%%%%%%%%%%%%%%%%%%%%%%%%%%%%%%%%%%
% \paragraph{Forwarding for a Complete Draft.}
%
% The following forwarding file |cdocsdrf.tex|
% compiles the main document in draft mode:
%\iffalse
%<*sampledraft>
%\fi
%    \begin{macrocode}
\def\version{draft}
\input{childdoc.def}
\childdocforward{cdocsamp}
%    \end{macrocode}

%\iffalse
%</sampledraft>
%\fi
%
% %%%%%%%%%%%%%%%%%%%%%%%%%%%%%%%%%%%%%%
% \paragraph{Forwarding for Final Version of the Chapters.}
%
% The following forwarding files |cdocsfn1.tex| and |cdocsfn2.tex|
% (with identical content)
% compile the final versions of the child documents
% |cdocsch1.tex| and |cdocsch2.tex|, respectively:
%\iffalse
%<*samplefinal>
%\fi
%    \begin{macrocode}
\def\version{final}
\input{childdoc.def}
\childdocforwardprefix[cdocsamp]{cdocsfn}{cdocsch}
%    \end{macrocode}

%\iffalse
%</samplefinal>
%\fi
%
% %%%%%%%%%%%%%%%%%%%%%%%%%%%%%%%%%%%%%%
% \paragraph{Command Line Processing.}
%
% The following three command lines generate the output files
% |cdocscld|, |cdocscl1| and |cdocscl2|
% which should be identical to
% |cdocsdrf|, |cdocsch1| and |cdocsfn2|, respectively:
% \begin{center}
% \begin{tabular}{l}
% |latex -jobname cdocscld \|\\
% |  "\def\version{draft}\input{childdoc.def}\childdocforward{cdocsamp}"|\\
% |latex -jobname cdocscl1 \|\\
% |  "\input{childdoc.def}\childdocforward[cdocsamp]{cdocsch1}"|\\
% |latex -jobname cdocscl2 \|\\
% |  "\def\version{final}\input{childdoc.def}\childdocforward{cdocsch2}"|
% \end{tabular}
% \end{center}
% Note that the trailing backslash on each first line
% merely continues the input to the second line
% (for convenient cut ant paste).
% Furthermore, the command |latex| can be replaced by any
% of its alternative versions such as |pdflatex|.
%
% %%%%%%%%%%%%%%%%%%%%%%%%%%%%%%%%%%%%%%%%%%%%%%%%%%%%%%%%%%%%%%%%%%%%%%%%%%%%%%
% %%%%%%%%%%%%%%%%%%%%%%%%%%%%%%%%%%%%%%%%%%%%%%%%%%%%%%%%%%%%%%%%%%%%%%%%%%%%%%
% \section{Implementation}
%\iffalse
%<*package>
%\fi
%
% This section describes the definitions file |childdoc.def|.

% The definitions cannot be loaded using |\usepackage| or |\RequirePackage|
% which has a mechanism to prevent loading a style file more than once.
% When loading the definitions by means of |\input|
% multiple instances have to be prevented manually:
%\iffalse
%This code needs to be before the `\ProvidesFile' directive
%which is defined at the beginning of this file.
%Therefore it is also placed there and commented out here.
%</package>
%<*discard>
%\fi
%    \begin{macrocode}
\ifdefined\childdocmain\endinput\fi
%    \end{macrocode}
%\iffalse
%</discard>
%<*package>
%\fi
%
% \macro{\ifchilddoc}
% \macro{\ifchilddocmanual}
% The conditional |\ifchilddoc| tells whether a
% child (true) or main (false) document is being compiled.
% The conditional |\ifchilddocmanual| tells whether
% the |\includeonly| mechanism is used (false) or
% the selection of child files must be performed manually (true).
% The definitions initialise to false:
%    \begin{macrocode}
\newif\ifchilddoc
\newif\ifchilddocmanual
%    \end{macrocode}

% \macro{\childdocname}
% \macro{\childdocjob}
% The macro |\childdocname| stores the name of the main document
% to be compiled. The macro |\childdocjob| stores the name of
% the document on which the \LaTeX{} compiler was originally invoked.
% The content of |\jobname| cannot be compared
% to filenames specified in the source due to different catcodes.
% The following code rescans |\jobname|, stores the result
% in |\childdocname| and saves a copy in |\childdocjob|:
%    \begin{macrocode}
\edef\childdocname{\scantokens\expandafter{\jobname\noexpand}}
\let\childdocjob\childdocname
%    \end{macrocode}

% \macro{\childdocdisable}
% The macro |\childdocdisable| prevents the main file
% from being processed more than once.
% At this stage, the main document command |\childdocmain|
% is assumed to be called once again where it should do nothing.
% Any subsequent call to it should prevent
% a secondary processing of the main document
% It overwrites the forwarding commands
% |\childdocof| and |\childdocforward|
% with empty macros to prevent further inclusions of the main document:
%    \begin{macrocode}
\newcommand{\childdocdisable}
{
  \renewcommand{\childdocmain}[1]{\renewcommand{\childdocmain}[1]{\endinput}}
  \renewcommand{\childdocof}[1]{}
  \renewcommand{\childdocby}[2][]{}
  \renewcommand{\childdocforward}[2][]{}
  \renewcommand{\childdocdisable}{}
}
%    \end{macrocode}

% \macro{\childdocmain}
% The macro |\childdocmain| is to be called at the top of the main file
% with nothing or the main filename (without extension) as argument.
% First, it breaks loops.
% If the argument is not empty and does not match |\childdocname|
% (which is set by the first inclusion of |childdoc.def|),
% |\ifchilddoc| is set to true, |\includeonly| is applied to the child file
% and |\jobname| is set to the main file
% (for proper handling of |.aux| files):
%    \begin{macrocode}
\newcommand{\childdocmain}[1]
{
  \childdocdisable\childdocmain{}
  \if?#1?\else
    \begingroup
      \def\childdoctmp{#1}
      \ifx\childdoctmp\childdocname
        \def\childdoctmp{}
      \else
        \def\childdoctmp
        {
          \childdoctrue
          \includeonly{\childdocname}
          \def\childdocjob{#1}
          \def\jobname{#1}
        }
      \fi
      \expandafter
    \endgroup
    \childdoctmp
  \fi
}
%    \end{macrocode}

% \macro{\childdocof}
% The command |\childdocof| redirects
% compilation to the main file |#1|.
%    \begin{macrocode}
\newcommand{\childdocof}[1]
{
  \childdocdisable
  \childdoctrue
  \includeonly{\childdocname}
  \def\jobname{#1}
  \def\childdocjob{#1}
  \input{#1}
}
%    \end{macrocode}

% \macro{\childdocby}
% The command |\childdocby| ....
%    \begin{macrocode}
\newcommand{\childdocby}[2][]
{
  \childdocdisable
  \childdoctrue
  \childdocmanualtrue
  \if?#1?\else
    \def\jobname{#2}
  \fi
  \def\childdocjob{#2}
  \input{#2}
  \endinput
}
%    \end{macrocode}

% \macro{\childdocforward}
% The command |\childdocforward| redirects
% compilation to the main file or
% (if the optional argument is given) a child file.
% Parameters are set as if the main file
% or a child file starting with |\childdocof| was compiled.
% Then compilation is handed over to the main file:
%    \begin{macrocode}
\newcommand{\childdocforward}[2][]
{
  \begingroup
    \if?#1?
      \def\childdoctmp
      {
        \def\childdocname{#2}
        \def\childdocjob{#2}
        \def\jobname{#2}
        \input{#2}
        \endinput
      }
    \else
      \def\childdoctmp
      {
        \childdocdisable
        \def\childdocname{#2}
        \childdoctrue
        \includeonly{#2}
        \def\childdocjob{#1}
        \def\jobname{#1}
        \input{#1}
        \endinput
      }
    \fi
    \expandafter
  \endgroup
  \childdoctmp
}
%    \end{macrocode}

% \macro{\childdocforwardprefix}
% The command |\childdocforwardprefix| redirects
% compilation to the main or a child file by means of a pattern.
% The prefix |#1| in the current filename is replaced by |#2|
% and the suffix of the current filename is kept
% (it is assumed that the filename does not contain the substring `|~~~|'
% which is used as a delimiter).
% Compilation is handed over to the new file by |\childdocforward|:
%    \begin{macrocode}
\newcommand{\childdocforwardprefix}[3][]
{
  \begingroup
    \def\childdocextract #2##1~~~{\def\childdoctmp{\childdocforward[#1]{#3##1}}}
    \expandafter\childdocextract\childdocname~~~
    \expandafter
  \endgroup
  \childdoctmp
}
%    \end{macrocode}

% \macro{\childdoc}
% The deprecated macro |\childdoc| is a legacy version of |\childdocmain|:
%    \begin{macrocode}
\newcommand{\childdoc}{\childdocmain}
%    \end{macrocode}

% \macro{\childdocredirect}
% The deprecated macro |\childdocredirect| is a legacy version
% of |\childdocforward| and |\childdocforwardprefix|:
%    \begin{macrocode}
\newcommand{\childdocredirect}[2][]
{
  \begingroup
    \if?#1?
      \def\childdoctmp{\childdocforward{#2}}
    \else
      \def\childdoctmp{\childdocforwardprefix{#1}{#2}}
    \fi
    \expandafter
  \endgroup
  \childdoctmp
}
%    \end{macrocode}

%\iffalse
%</package>
%\fi
%
\endinput
|\\
|\childdocforward{|\textit{main}|}|
\end{tabular}
\end{center}
%
Likewise, the following files |final|\textit{nn}|.tex|
compile the final version of the child document
|child|\textit{nn}|.tex|:
%
\begin{center}
\begin{tabular}{l}
|\def\version{final}|\\
|% \iffalse
%
% childdoc.dtx Copyright (C) 2017-2018 Niklas Beisert
%
% This work may be distributed and/or modified under the
% conditions of the LaTeX Project Public License, either version 1.3
% of this license or (at your option) any later version.
% The latest version of this license is in
%   http://www.latex-project.org/lppl.txt
% and version 1.3 or later is part of all distributions of LaTeX
% version 2005/12/01 or later.
%
% This work has the LPPL maintenance status `maintained'.
%
% The Current Maintainer of this work is Niklas Beisert.
%
% This work consists of the files childdoc.dtx and childdoc.ins
% and the derived files childdoc.def and cdocsamp.tex with
% cdocsch1.tex, cdocsch2.tex, cdocsdrf.tex, cdocsfn1.tex, cdocsfn2.tex.
%
%<package>\ifdefined\childdocmain\endinput\fi
%<package>\ProvidesFile{childdoc.def}[2018/12/30 v2.0 child document driver]
%<samplemain>\ProvidesFile{cdocsamp.tex}[2018/12/30 v2.0 sample for childdoc]
%<*driver>
%\ProvidesFile{childdoc.drv}[2018/12/30 v2.0 childdoc reference manual file]
\PassOptionsToClass{10pt,a4paper}{article}
\documentclass{ltxdoc}

\usepackage[margin=35mm]{geometry}
\usepackage{hyperref}
\usepackage{hyperxmp}
\usepackage[usenames]{color}

\hypersetup{colorlinks=true}
\hypersetup{pdfstartview=FitH}
\hypersetup{pdfpagemode=UseNone}
\hypersetup{pdfsource={}}
\hypersetup{pdflang={en-UK}}
\hypersetup{pdfcopyright={Copyright 2017-2018 Niklas Beisert.
  This work may be distributed and/or modified under the
  conditions of the LaTeX Project Public License, either version 1.3
  of this license or (at your option) any later version.}}
\hypersetup{pdflicenseurl={http://www.latex-project.org/lppl.txt}}
\hypersetup{pdfcontactaddress={ETH Zurich, ITP, HIT K,
  Wolfgang-Pauli-Strasse 27}}
\hypersetup{pdfcontactpostcode={8093}}
\hypersetup{pdfcontactcity={Zurich}}
\hypersetup{pdfcontactcountry={Switzerland}}
\hypersetup{pdfcontactemail={nbeisert@itp.phys.ethz.ch}}
\hypersetup{pdfcontacturl={http://people.phys.ethz.ch/\xmptilde nbeisert/}}

\newcommand{\secref}[1]{\hyperref[#1]{section \ref*{#1}}}

\parskip1ex
\parindent0pt
\let\olditemize\itemize
\def\itemize{\olditemize\parskip0pt}

\begin{document}

\title{The \textsf{childdoc} Package}
\hypersetup{pdftitle={The childdoc Package}}
\author{Niklas Beisert\\[2ex]
  Institut f\"ur Theoretische Physik\\
  Eidgen\"ossische Technische Hochschule Z\"urich\\
  Wolfgang-Pauli-Strasse 27, 8093 Z\"urich, Switzerland\\[1ex]
  \href{mailto:nbeisert@itp.phys.ethz.ch}
  {\texttt{nbeisert@itp.phys.ethz.ch}}}
\hypersetup{pdfauthor={Niklas Beisert}}
\hypersetup{pdfsubject={Manual for the LaTeX2e Package childdoc}}
\date{30 December 2018, \textsf{v2.0}}
\maketitle

\begin{abstract}\noindent
\textsf{childdoc} is a \LaTeXe{} package
that enables the direct compilation
of document sections included by |\include|
to individual files.
\end{abstract}

\begingroup
\parskip0ex
\tableofcontents
\endgroup

%%%%%%%%%%%%%%%%%%%%%%%%%%%%%%%%%%%%%%%%%%%%%%%%%%%%%%%%%%%%%%%%%%%%%%%%%%%%%%%%
%%%%%%%%%%%%%%%%%%%%%%%%%%%%%%%%%%%%%%%%%%%%%%%%%%%%%%%%%%%%%%%%%%%%%%%%%%%%%%%%
\section{Introduction}

\LaTeX{} provides a mechanism to structure a large document (such as a book)
into a main file and several child files (containing the chapters)
using the |\include| command.
This mechanism is beneficial for documents
which span hundreds of pages in order to
make the source file(s) more manageable.
Moreover, compilation can be restricted to
selected child files by means of the |\includeonly| command.
The latter feature can be used to reduce the compilation time while editing
(this was significantly more useful in the earlier days of \LaTeX{})
or to generate a smaller document which is easier to navigate.
Another application of |\includeonly| is to generate
documents consisting of selected parts of the complete document.

However, there are a few drawbacks of the plain |\include| mechanism:
\begin{itemize}
\item
The child files cannot be compiled on their own,
they can only be compiled via the main file.
A naive editing environment
(such as a text editor with an option
to have the current file processed by \LaTeX)
may require one to switch to the main file before compiling;
attempting to compile the child file produces errors.
\item
The main file must be modified (each time)
to adjust the |\includeonly| command
to the present needs. This easily leaves the main file in a messy state.
\item
The generated document will always carry the filename
of the main document. This is inconvenient if
several child files are to be compiled and
to be kept for distribution.
\end{itemize}

The present package provides a simple interface
to make child files individually compilable by \LaTeX{}.
Compiling a child file then has the same effect as compiling
the main file with an |\includeonly| command
to select the appropriate child.
Moreover the generated document will carry the name of the child
rather than the main file.
This resolves all three above issues.

This feature is meant to make the editing of books,
thesis documents and lecture notes somewhat more convenient.
However, the package can also be used efficiently for
composing a series of documents (such as exercise sheets)
which are typically distributed individually.
It then assists the author in generating the individual documents
(potentially in different versions)
as well as a document containing the collected series.
Another application is in developing style files
or other kinds of included material
where compilation of the style file could redirect
to a sample or test file.

%%%%%%%%%%%%%%%%%%%%%%%%%%%%%%%%%%%%%%%%%%%%%%%%%%%%%%%%%%%%%%%%%%%%%%%%%%%%%%%%
%%%%%%%%%%%%%%%%%%%%%%%%%%%%%%%%%%%%%%%%%%%%%%%%%%%%%%%%%%%%%%%%%%%%%%%%%%%%%%%%
\section{Usage}

First of all, the package \textsf{childdoc} is \emph{not} a standard
\LaTeXe{} |.sty| style file! Therefore it needs to be invoked in
a non-standard way.

%%%%%%%%%%%%%%%%%%%%%%%%%%%%%%%%%%%%%%%%%%%%%%%%%%%%%%%%%%%%%%%%%%%%%%%%%%%%%%%%
\subsection{Included Files}
\label{sec:include}

%%%%%%%%%%%%%%%%%%%%%%%%%%%%%%%%%%%%%%%%
\DescribeMacro{\childdocmain}
To use the package, add the commands
\begin{center}
\begin{tabular}{l}
|\input{childdoc.def}|\\
|\childdocmain{}|\\
\end{tabular}
\end{center}
at the very top of the main \LaTeX{} file,
in particular \emph{before} the |\documentclass| statement!
The argument of |\childdocmain| should be left empty
(but it must be present).

%%%%%%%%%%%%%%%%%%%%%%%%%%%%%%%%%%%%%%%%
\DescribeMacro{\childdocof}
Furthermore, add the commands
\begin{center}
\begin{tabular}{l}
|\input{childdoc.def}|\\
|\childdocof{|\textit{main}|}|\\
\end{tabular}
\end{center}
at the top of every child file \textit{child}
which is included by |\include{|\textit{child}|}|
from within the main file
(or at least for those files to be compiled individually).
The argument \textit{main} must be the filename of the main file.

There are a couple of
considerations in setting up the main and child documents:

%%%%%%%%%%%%%%%%%%%%%%%%%%%%%%%%%%%%%%%%
\paragraph{Restrictions.}

Please note the following restrictions:
\begin{itemize}
\item
|\childdocmain| must be called with one argument \textit{main}
to ensure compatibility with earlier version of the package.
It must either be empty (|\childdocmain{}|)
or precisely match the filename of the main file in which it is specified.
See \secref{sec:detection} for further information.
\item
The filename \textit{main} must be specified without the |.tex| extension.
\item
The filename \textit{main} is case sensitive
(even in case-insensitive file systems)
due to internal string comparison.
\item
The argument \textit{main} should be fully expanded, it cannot be a macro.
\item
Subdirectories and special characters should be avoided in filenames.
\item
The command |\childdocmain{|\textit{main}|}| must be followed by a whitespace.
It should not be followed immediately by another command
or by a comment mark `|%|'.
This is because the \TeX{} parser reads the token immediately following
the argument of |\childdocmain| and puts it
at the beginning of every child section;
however, a white\-space is ignored.
\end{itemize}

%%%%%%%%%%%%%%%%%%%%%%%%%%%%%%%%%%%%%%%%
\paragraph{Content of Main File.}

It is advisable to place all content in the child files included by |\include|.
Any output contained in the main file will appear in all child documents
unless suppressed manually;
it cannot be suppressed automatically by the |\includeonly| directive
and thus should normally be avoided.
A method to include some content in the main file
by means of conditional processing is described in \secref{sec:conditional}.

%%%%%%%%%%%%%%%%%%%%%%%%%%%%%%%%%%%%%%%%
\paragraph{Page Numbering.}

When only a part of the document is compiled,
the appropriate numbering of pages
(as well as other status parameters)
is determined from the |.aux| files.
The latter contain information from previous passes.
However this information needs to propagate through
all intermediate child documents.
Therefore the page numbering in child documents may well
be inconsistent until the complete document is compiled at least once.

A useful (if unconventional) way to always ensure a consistent
page numbering is to restart the numbering in each child document
and denote the pages by `\textit{child}|.|\textit{page}'
where \textit{child} represents the chapter/section number of the child file.
This can be achieved by the command
|\numberwithin{page}{|\textit{child}|}|
of the \textsf{amsmath} package
where \textit{child} can be |chapter| or |section|
depending on the chosen structuring.
Alternatively, one can modify the macro |\thepage| appropriately
and reset the counter |page| at the start of each child file.

%%%%%%%%%%%%%%%%%%%%%%%%%%%%%%%%%%%%%%%%%%%%%%%%%%%%%%%%%%%%%%%%%%%%%%%%%%%%%%%%
\subsection{Conditional Processing}
\label{sec:conditional}

The package provides a mechanism to compile different versions
of a document. To customise the versions further some conditional processing
can come in handy to distinguish which version is being compiled.
The package provides two macros to describe the compilation context:

%%%%%%%%%%%%%%%%%%%%%%%%%%%%%%%%%%%%%%%%
\DescribeMacro{\ifchilddoc}
The conditional |\ifchilddoc| distinguishes between the compilation of
child documents and the main document:
%
\begin{center}
|\ifchilddoc |\textit{child-code}| |[|\||else |\textit{main-code}]| \||fi|
\end{center}

%%%%%%%%%%%%%%%%%%%%%%%%%%%%%%%%%%%%%%%%
\DescribeMacro{\childdocname}
\DescribeMacro{\childdocjob}
The macro |\childdocname| contains the filename (without extension)
of the main or child file being processed.
Note that |\childdocjob| will always contain the name of the main file.

%%%%%%%%%%%%%%%%%%%%%%%%%%%%%%%%%%%%%%%%
\paragraph{Title Page.}

Conditional processing can be used to include a title or banner page
in the main document when proper precautions are taken.
Importantly, the code in the main file should ensure that the page counter
(as well as other status parameters which are stored in the |.aux| files)
takes the same value after the conditional processing.
Otherwise the page numbers may take divergent values
depending on which part is compiled.

For example, a title page could be declared by:
%
\begin{center}
\begin{tabular}{l}
|\ifchilddoc\||else|\\
|\addtocounter{page}{-1}|\\
\textit{code for title page}\\
|\newpage|\\
|\||fi|
\end{tabular}
\end{center}
%
A banner page for the child documents can be generated by:
%
\begin{center}
\begin{tabular}{l}
|\ifchilddoc|\\
|\addtocounter{page}{-1}|\\
\textit{code for banner page}\\
|\newpage|\\
|\||fi|
\end{tabular}
\end{center}
%
Here one could write a message such as:
\begin{center}
|This is the part \childdocname{} of \childdocjob{}.|
\end{center}

%%%%%%%%%%%%%%%%%%%%%%%%%%%%%%%%%%%%%%%%%%%%%%%%%%%%%%%%%%%%%%%%%%%%%%%%%%%%%%%%
\subsection{Flags}
\label{sec:flags}

The package makes it easy to generate different versions
of the main or child documents.
To this end compilation flags can be defined
and assigned different default values.
They will be particularly useful in conjunction
with the forwarding mechanism described in \secref{sec:forward}.

For example, it may be useful to have a flag |\version|
which can be set to |draft| or |final|.
The document source will contain some conditional code
depending on the value of |\version|.
Suppose further, the flag should default to |final| for the main file
and to |draft| for child files
which is a natural assignment for editing the document.
This is achieved by placing the following code
in the preamble of the main document
(below the |\childdocmain| directive):
%
\begin{center}
\begin{tabular}{l}
|\ifchilddoc|\\
|\providecommand{\version}{draft}|\\
|\||else|\\
|\providecommand{\version}{final}|\\
|\||fi|
\end{tabular}
\end{center}
%
The definition by |\providecommand| makes sure
that previous definitions are not overwritten.
Further statements |\providecommand{\version}{...}|
can thus be added before the above code to override it.

For the main file, one might add a line
(between |\childdocmain| and the above block)
%
\begin{center}
|%\ifchilddoc\||else\providecommand{\version}{draft}\||fi|
\end{center}
%
which can be uncommented to produce a draft version.
Likewise one can add a line to the very top of a child file
(above the |\childdocof{|\textit{main}|}| directive)
%
\begin{center}
|%\providecommand{\version}{final}|
\end{center}
%
which can be uncommented to produce the final version of this child document.

%%%%%%%%%%%%%%%%%%%%%%%%%%%%%%%%%%%%%%%%%%%%%%%%%%%%%%%%%%%%%%%%%%%%%%%%%%%%%%%%
\subsection{Forwarding}
\label{sec:forward}

Different versions of the main or child documents
using compilation flags as described in \secref{sec:flags}
can be (permanently) stored in different files
for convenient compilation, viewing and distribution.
To this end, the package defines a command
to pass on compilation to a different file:

%%%%%%%%%%%%%%%%%%%%%%%%%%%%%%%%%%%%%%%%
\DescribeMacro{\childdocforward}
The command |\childdocforward| redirects processing to
another source file:
%
\begin{center}
\begin{tabular}{l}
|\input{childdoc.def}|\\
|\childdocforward[|\textit{main}|]{|\textit{dest}|}|\\
\end{tabular}
\end{center}
%
The argument \textit{dest} is the destination file
(without extension).
It should be the main file or one of the child files.
Note that further \textsf{childdoc} directives
such as |\childdocof| and |\childdocforward|
in the indicated file will be processed in this form.
The optional argument \textit{main}
passes on directly to the main file \textit{main}
while pretending to compile the child \textit{dest}.
This form behaves as if \textit{dest}
issues |\childdocof{|\textit{main}|}| right away,
and no further \textsf{childdoc} directives will be processed.

%%%%%%%%%%%%%%%%%%%%%%%%%%%%%%%%%%%%%%%%
\DescribeMacro{\...prefix}
In the alternative form |\childdocforwardprefix|,
%
\begin{center}
\begin{tabular}{l}
|\input{childdoc.def}|\\
|\childdocforwardprefix[|\textit{main}|]{|\textit{prefix}|}{|\textit{dest}|}|
\end{tabular}
\end{center}
%
the destination file is determined by a pattern
depending on the current file:
To make this work, the current file must be called
`{\textit{prefix}\hspace{0.2em}\textit{suffix}}'
with \textit{prefix} matching precisely the argument.
Processing is then passed on to the file
`{\textit{dest}\hspace{0.2em}\textit{suffix}}'.
Surely, the same effect is achieved by
directly specifying the
argument `{\textit{dest}\hspace{0.2em}\textit{suffix}}'
in the first form.
However, that requires to set up a different file
for each child. With the alternative form of the command
all these files can have exactly the same content
which simplifies setting them up and maintaining them.

For example, the following file |draft.tex|
with a compilation flag |\version| as described in \secref{sec:flags}
compiles the main document as a draft:
%
\begin{center}
\begin{tabular}{l}
|\def\version{draft}|\\
|\input{childdoc.def}|\\
|\childdocforward{|\textit{main}|}|
\end{tabular}
\end{center}
%
Likewise, the following files |final|\textit{nn}|.tex|
compile the final version of the child document
|child|\textit{nn}|.tex|:
%
\begin{center}
\begin{tabular}{l}
|\def\version{final}|\\
|\input{childdoc.def}|\\
|\childdocforwardprefix{final}{child}|
\end{tabular}
\end{center}
%

Note that when several versions of a main file and/or of each child file
are to be generated, it may be convenient to set up a |Makefile| or
shell script to automatise the process.

%%%%%%%%%%%%%%%%%%%%%%%%%%%%%%%%%%%%%%%%%%%%%%%%%%%%%%%%%%%%%%%%%%%%%%%%%%%%%%%%
\subsection{Command Line Processing}
\label{sec:commandline}

The effect of redirection files can also be achieved by invoking
the \LaTeX{} compiler with a more elaborate command line.
Most conveniently this should be done as part
of a shell script or a |Makefile|.

When using \textsf{childdoc} in the main file, the following
command lines effectively perform a redirection
(note that depending on the shell being used,
backslashes may have to be doubled: `|\|' $\to$ `|\\|'):
%
\begin{center}
|... -jobname "|\textit{target}|" |\\|"|[\textit{flags}]%
|\input{childdoc.def}\childdocforward[|\textit{main}|]{|\textit{dest}|}"|
\end{center}
%
Here \textit{target} is the name of the output file,
\textit{main} is the name of the main file
and \textit{dest} is the name of the main or child file to be processed
(all filenames without extensions).
The optional argument \textit{main} can be omitted
if \textit{main} matches \textit{dest}.
Optionally, compilation \textit{flags} can be defined via |\def| commands.
This command line makes the \TeX{} engine believe
it is compiling the file \textit{target}
whose content is specified as the latter parameter.
The provided code then forwards the processing to
\textit{main} or \textit{dest} as described in \secref{sec:forward}.

%%%%%%%%%%%%%%%%%%%%%%%%%%%%%%%%%%%%%%%%%%%%%%%%%%%%%%%%%%%%%%%%%%%%%%%%%%%%%%%%
\subsection{Include by Input}
\label{sec:input}

Including child documents by |\include| has some restrictions by design.
Most notably, the content of a child document always occupies
its own set of pages; pages cannot be shared between child documents.
Usually, this behaviour makes perfect sense
because each child document contain an essential part of the document.
However, in some situations it may be desirable to compose
a document from a collection of parts
without having mandatory page breaks between then.
For this case, the package
provides a mechanism to include parts
by |\input| which can also be processed individually.
However, by construction this mechanism
requires manual handling of the content to be output.

%%%%%%%%%%%%%%%%%%%%%%%%%%%%%%%%%%%%%%%%
\DescribeMacro{\ifchilddocmanual}
The main file should be prepared as usual, see \secref{sec:include}.
However, the document body must make a distinction
between processing of an individual part and of the main document, e.g.:
%
\begin{center}
\begin{tabular}{l}
|\ifchilddocmanual|\\
|\input{\childdocname}|\\
|\||else|\\
\textit{document body with }|\input{|\textit{part}|}|\\
|\||fi|
\end{tabular}
\end{center}
%
The conditional |\ifchilddocmanual| is true whenever
a part to be included by |\input| is being compiled,
and the name of the part is stored in |\childdocname|.

%%%%%%%%%%%%%%%%%%%%%%%%%%%%%%%%%%%%%%%%
\DescribeMacro{\childdocby}
Each part to be included by |\input| should start with:
%
\begin{center}
\begin{tabular}{l}
|\input{childdoc.def}|\\
|\childdocby{|\textit{main}|}|\\
\end{tabular}
\end{center}
%
The directive |\childdocby| is similar to |\childdocof|
described in \secref{sec:include},
but the subsequent selection of content must be done manually.
To that end, both |\ifchilddoc| and |\ifchilddocmanual|
will be true upon processing of a part,
and the name of the part is stored in |\childdocname|.
Note that |\jobname| will be set to the filename of the current part
so that each part receives an individual |.aux| file
that does not interfere with the |.aux| file(s) of the main document.
This behaviour can be altered by the alternative form
|\childdocby[*]{|\textit{main}|}| (with a non-empty optional argument)
which uses the |.aux| file of the main document
by setting |\jobname| to \textit{main}.

%%%%%%%%%%%%%%%%%%%%%%%%%%%%%%%%%%%%%%%%%%%%%%%%%%%%%%%%%%%%%%%%%%%%%%%%%%%%%%%%
\subsection{Driver Development}
\label{sec:driver}

The \textsf{childdoc} mechanism can also be use for the development
of definition files such as \LaTeX{} styles or classes.
This case differs from the above setup with multiple parts
included by |\include| in that no |\includeonly| should be invoked.
This can be achieved by starting the include file
(before |\ProvidesPackage|) with:
%
\begin{center}
\begin{tabular}{l}
|\input{childdoc.def}|\\
|\childdocforward{|\textit{main}|}|\\
\end{tabular}
\end{center}
%
or alternatively with:
%
\begin{center}
\begin{tabular}{l}
|\input{childdoc.def}|\\
|\childdocby{|\textit{main}|}|\\
\end{tabular}
\end{center}
%
Both forms have slightly different effects as described above.
The main file is prepared as usual, see \secref{sec:include}.

%%%%%%%%%%%%%%%%%%%%%%%%%%%%%%%%%%%%%%%%%%%%%%%%%%%%%%%%%%%%%%%%%%%%%%%%%%%%%%%%
\subsection{Legacy Detection}
\label{sec:detection}

The directive |\childdocmain| in the main file can detect
whether the complete document or merely a child is to be compiled
even without using the directive |\childdocof|.
This method is deprecated because it is less robust
and there is no compelling reason to use it;
it is merely provided for backward compatibility
and it may be removed in future versions.

If the detection mechanism is to be used,
it is mandatory to correctly specify
the filename of the main file as the argument of |\childdocmain|:
%
\begin{center}
\begin{tabular}{l}
|\input{childdoc.def}|\\
|\childdocmain{|\textit{main}|}|\\
\end{tabular}
\end{center}
%
If |\jobname| does not match the argument \textit{main} of |\childdocmain|,
it is assumed that |\jobname| points to the child file to be compiled.
When using |\childdocmain| with the main file specified as argument,
it suffices to start a child file
with just |\input{|\textit{main}|}|
without loading of the package and using |\childdocof|.
If instead all processing is done
with the appropriate \textsf{childdoc} directives,
the argument of \textit{main} of |\childdocmain| can be empty.

An alternative version of the command line processing described
in \secref{sec:commandline} using the detection mechanism reads:
%
\begin{center}
|... -jobname "|\textit{target}|" "|[\textit{flags}]%
[|\def\jobname{|\textit{dest}|}|]|\input{|\textit{main}|}"|
\end{center}

%%%%%%%%%%%%%%%%%%%%%%%%%%%%%%%%%%%%%%%%%%%%%%%%%%%%%%%%%%%%%%%%%%%%%%%%%%%%%%%%
\subsection{Manual Code}
\label{sec:manual}

In case one cannot be certain whether the definitions file |childdoc.def|
is installed on the target \TeX{} distribution
and one prefers not to ship it,
it is conceivable to paste a few relevant commands into the sources.

To that end, drop all statements |\input{childdoc.def}|
and perform the replacements as outlined below.
Instead of |\childdocmain{|\textit{main}|}| add the following code
to the top of the main file:
%
\begin{center}
\begin{tabular}{l}
|\||ifdefined\childdocname\endinput\||fi\newif\ifchilddoc|\\
|\edef\childdocname{\scantokens\expandafter{\jobname\noexpand}}|\\
|\def\childdocmain{|\textit{main}|}\||ifx\childdocmain\childdocname\||else|\\
|\childdoctrue\includeonly{\childdocname}\let\jobname\childdocmain\||fi|\\
\end{tabular}
\end{center}
%
Instead of |\childdocof{|\textit{main}|}| just include the main file
at the top of each child file:
%
\begin{center}
|\input{|\textit{main}|}|
\end{center}
%
A simple redirection |\childdocforward{|\textit{dest}|}| is achieved by:
%
\begin{center}
|\def\jobname{|\textit{dest}|}\input{\jobname}|
\end{center}
%
The redirection with prefix
|\childdocforwardprefix[|\textit{prefix}|]{|\textit{dest}|}|
is accomplished by:
%
\begin{center}
\begin{tabular}{l}
|{\edef\jobname{\scantokens\expandafter{\jobname\noexpand}}|\\
|\def\redirectjob |\textit{prefix}|#1~~~{\gdef\jobname{|\textit{dest}|#1}}|\\
|\expandafter\redirectjob\jobname~~~}\input{\jobname}|
\end{tabular}
\end{center}

In an alternative approach,
child documents can be compiled by a specific command line
without additional code or specific definitions:
%
\begin{center}
|... -jobname "|\textit{target}|" "|[\textit{flags}]%
|\includeonly{|\textit{dest}|}\input{|\textit{main}|}"|
\end{center}
%

%%%%%%%%%%%%%%%%%%%%%%%%%%%%%%%%%%%%%%%%%%%%%%%%%%%%%%%%%%%%%%%%%%%%%%%%%%%%%%%%
%%%%%%%%%%%%%%%%%%%%%%%%%%%%%%%%%%%%%%%%%%%%%%%%%%%%%%%%%%%%%%%%%%%%%%%%%%%%%%%%
\section{Information}

%%%%%%%%%%%%%%%%%%%%%%%%%%%%%%%%%%%%%%%%%%%%%%%%%%%%%%%%%%%%%%%%%%%%%%%%%%%%%%%%
\subsection{Copyright}

Copyright \copyright{} 2017--2018 Niklas Beisert

This work may be distributed and/or modified under the
conditions of the \LaTeX{} Project Public License, either version 1.3
of this license or (at your option) any later version.
The latest version of this license is in
  \url{http://www.latex-project.org/lppl.txt}
and version 1.3 or later is part of all distributions of \LaTeX{}
version 2005/12/01 or later.

This work has the LPPL maintenance status `maintained'.

The Current Maintainer of this work is Niklas Beisert.

This work consists of the files |README.txt|, |childdoc.ins| and |childdoc.dtx|
as well as the derived files |childdoc.def|, |cdocsamp.tex|
with |cdocsch1.tex|, |cdocsch2.tex|, |cdocspt3.tex|, |cdocspt4.tex|,
|cdocsdrf.tex|, |cdocsfn1.tex|, |cdocsfn2.tex|
as well as |childdoc.pdf|.

%%%%%%%%%%%%%%%%%%%%%%%%%%%%%%%%%%%%%%%%%%%%%%%%%%%%%%%%%%%%%%%%%%%%%%%%%%%%%%%%
\subsection{Files and Installation}

The package consists of the files:
%
\begin{center}
\begin{tabular}{ll}
    |README.txt|   & readme file \\
    |childdoc.ins| & installation file \\
    |childdoc.dtx| & source file \\
    |childdoc.def| & definition file \\
    |cdocsamp.tex| & sample main file \\
    |cdocsch1.tex| & sample include file \\
    |cdocsch2.tex| & sample include file \\
    |cdocspt3.tex| & sample part file \\
    |cdocspt4.tex| & sample part file \\
    |cdocsdrf.tex| & sample redirection file \\
    |cdocsfn1.tex| & sample redirection file \\
    |cdocsfn2.tex| & sample redirection file \\
    |childdoc.pdf| & manual
\end{tabular}
\end{center}
%
The distribution consists of the files
|README.txt|, |childdoc.ins| and |childdoc.dtx|.
%
\begin{itemize}
\item
Run (pdf)\LaTeX{} on |childdoc.dtx|
to compile the manual |childdoc.pdf| (this file).
\item
Run \LaTeX{} on |childdoc.ins| to create the definitions file |childdoc.def|
and the sample |cdocsamp.tex| with include files
|cdocsch1.tex|, |cdocsch2.tex|, |cdocspt3.tex|, |cdocspt4.tex|,
|cdocsdrf.tex|, |cdocsfn1.tex|, |cdocsfn2.tex|.
Then copy the file |childdoc.def| to an appropriate directory of your \LaTeX{}
distribution, e.g.\ \textit{texmf-root}|/tex/latex/childdoc|.
\end{itemize}

%%%%%%%%%%%%%%%%%%%%%%%%%%%%%%%%%%%%%%%%%%%%%%%%%%%%%%%%%%%%%%%%%%%%%%%%%%%%%%%%
\subsection{Related CTAN Packages}

There are several other packages which offer a similar functionality:
%
\begin{itemize}
\item
The packages
\href{http://ctan.org/pkg/docmute}{\textsf{docmute}},
\href{http://ctan.org/pkg/includex}{\textsf{includex}} and
\href{http://ctan.org/pkg/standalone}{\textsf{standalone}}
provide commands to include only the document body of
a child file thus allowing both files to be compiled individually.
\item
The packages \href{http://ctan.org/pkg/subdocs}{\textsf{subdocs}}
and \href{http://ctan.org/pkg/subfiles}{\textsf{subfiles}}
provide structures in which the main and child documents can be
encapsulated and allowing them to be compiled individually.
The inclusion mechanism is different from the conventional |\include|.
\item
The package \href{http://ctan.org/pkg/combine}{\textsf{combine}}
is an elaborate solution to combine several documents into one.
\end{itemize}
%
See also the CTAN topic \href{http://ctan.org/topic/subdocs}{\textsf{subdocs}}
for further related packages.
The present package differs from the above solutions in that
a document structure constructed with the conventional |\include| mechanism
just needs two extra commands at the top of every file
such that all constituent files can be compiled individually.

%%%%%%%%%%%%%%%%%%%%%%%%%%%%%%%%%%%%%%%%%%%%%%%%%%%%%%%%%%%%%%%%%%%%%%%%%%%%%%%%
%\subsection{Feature Suggestions}
%
%The following is a list of features which may be useful for future
%versions of this package:
%%
%\begin{itemize}
%\item
%\ldots
%\end{itemize}

%%%%%%%%%%%%%%%%%%%%%%%%%%%%%%%%%%%%%%%%%%%%%%%%%%%%%%%%%%%%%%%%%%%%%%%%%%%%%%%%
\subsection{Revision History}

%%%%%%%%%%%%%%%%%%%%%%%%%%%%%%%%%%%%%%%%
\paragraph{v2.0:} 2018/12/30

\begin{itemize}
\item
immediate forward processing
\item
added |\childdocby| mechanism
\item
manual restructured
\end{itemize}

%%%%%%%%%%%%%%%%%%%%%%%%%%%%%%%%%%%%%%%%
\paragraph{v1.6:} 2018/01/17

\begin{itemize}
\item
application for development of include files
\item
corrections to manual
\end{itemize}

%%%%%%%%%%%%%%%%%%%%%%%%%%%%%%%%%%%%%%%%
\paragraph{v1.5:} 2017/05/21

\begin{itemize}
\item
more complete structuring introduced
\item
|\childdocof| introduced
\item
|\childdoc| renamed to |\childdocmain|
\item
|\childredirect| renamed to |\childdocforward| and |\childdocforwardprefix|
and functionality expanded
\end{itemize}

%%%%%%%%%%%%%%%%%%%%%%%%%%%%%%%%%%%%%%%%
\paragraph{v1.0:} 2017/04/27

\begin{itemize}
\item
manual and install package
\item
first version published on CTAN
\end{itemize}

%%%%%%%%%%%%%%%%%%%%%%%%%%%%%%%%%%%%%%%%
\paragraph{v0.6:} 2017/04/26

\begin{itemize}
\item
redirection mechanism added
\end{itemize}

%%%%%%%%%%%%%%%%%%%%%%%%%%%%%%%%%%%%%%%%
\paragraph{v0.5:} 2017/04/26

\begin{itemize}
\item
functionality in definition file
\end{itemize}


%%%%%%%%%%%%%%%%%%%%%%%%%%%%%%%%%%%%%%%%%%%%%%%%%%%%%%%%%%%%%%%%%%%%%%%%%%%%%%%%
%%%%%%%%%%%%%%%%%%%%%%%%%%%%%%%%%%%%%%%%%%%%%%%%%%%%%%%%%%%%%%%%%%%%%%%%%%%%%%%%
%%%%%%%%%%%%%%%%%%%%%%%%%%%%%%%%%%%%%%%%%%%%%%%%%%%%%%%%%%%%%%%%%%%%%%%%%%%%%%%%
\appendix

\settowidth\MacroIndent{\rmfamily\scriptsize 000\ }

 \DocInput{childdoc.dtx}

\end{document}
%</driver>
% \fi
%
% %%%%%%%%%%%%%%%%%%%%%%%%%%%%%%%%%%%%%%%%%%%%%%%%%%%%%%%%%%%%%%%%%%%%%%%%%%%%%%
% %%%%%%%%%%%%%%%%%%%%%%%%%%%%%%%%%%%%%%%%%%%%%%%%%%%%%%%%%%%%%%%%%%%%%%%%%%%%%%
% \section{Sample}
%\iffalse
%<*samplemain>
%\fi
%
% The following presents a sample document
% with two chapters, two parts, a title page,
% a compile flag as well as three forwarding files to set the flag.
% It consists of eight |.tex| files:
% \begin{center}
% \begin{tabular}{ll}
% |cdocsamp.tex|&main file\\
% |cdocsch1.tex|&include file for chapter 1\\
% |cdocsch2.tex|&include file for chapter 2\\
% |cdocspt3.tex|&include file for part 3\\
% |cdocspt4.tex|&include file for part 4\\
% |cdocsdrf.tex|&forwarding file for main file in draft mode\\
% |cdocsfi1.tex|&forwarding file for final version of chapter 1\\
% |cdocsfi2.tex|&forwarding file for final version of chapter 2\\
% \end{tabular}
% \end{center}
% Each of the eight files can be compiled directly by the \LaTeX{} compiler.
%
% %%%%%%%%%%%%%%%%%%%%%%%%%%%%%%%%%%%%%%
% \paragraph{Main File.}
%
% The main file is called |cdocsamp.tex|.
%
% Load the \textsf{childdoc} definitions and
% declare the filename for the main document:
%    \begin{macrocode}
\input{childdoc.def}
\childdocmain{}
%    \end{macrocode}

% Optional override for |\version| flag:
%    \begin{macrocode}
%%\ifchilddoc\else\providecommand{\version}{draft}\fi
%    \end{macrocode}

% Define the default values for the |\version| flag
% (|final| for the main file and |draft| for childs):
%    \begin{macrocode}
\ifchilddoc
\providecommand{\version}{draft}
\else
\providecommand{\version}{final}
\fi
%    \end{macrocode}

% Load the standard document class:
%    \begin{macrocode}
\documentclass[12pt]{article}
%    \end{macrocode}

% Start the document body:
%    \begin{macrocode}
\begin{document}
%    \end{macrocode}

% Declare a title page.
% Print title, part of document being processed and version flag:
%    \begin{macrocode}
\addtocounter{page}{-1}
\begin{center}
{\LARGE\bfseries{}childdoc example\par}
\vspace{1cm}
\ifchilddoc
\ifchilddocmanual part\else chapter\fi:
`\childdocname' of `\childdocjob'\par
\else
main document: `\childdocjob'\par
\fi
version: \version\par
\end{center}
\newpage
%    \end{macrocode}

% Manually include selected file,
% otherwise process as usual:
%    \begin{macrocode}
\ifchilddocmanual
\section*{part `\childdocname'}
\input{\childdocname}
\else
%    \end{macrocode}

% Include the two chapters:
%    \begin{macrocode}
\include{cdocsch1}
\include{cdocsch2}
%    \end{macrocode}

% Include the two parts unless only chapters should be displayed:
%    \begin{macrocode}
\ifchilddoc\else
\section{part three}
\input{cdocspt3}
\section{part four}
\input{cdocspt4}
\fi
%    \end{macrocode}

% Process as usual until here:
%    \begin{macrocode}
\fi
%    \end{macrocode}

% End of document body:
%    \begin{macrocode}
\end{document}
%    \end{macrocode}
%\iffalse
%</samplemain>
%\fi
%
% %%%%%%%%%%%%%%%%%%%%%%%%%%%%%%%%%%%%%%
% \paragraph{Chapter Include Files.}
%
% The include files are called |cdocsch1.tex| and |cdocsch2.tex|.
%
%\iffalse
%<*samplechap1|samplechap2>
%\fi

% Optional override for |\version| flag:
%    \begin{macrocode}
%%\providecommand{\version}{final}
%    \end{macrocode}

% Include the main document:
%    \begin{macrocode}
\input{childdoc.def}
\childdocof{cdocsamp}
%    \end{macrocode}

%\iffalse
%</samplechap1|samplechap2>
%\fi
%
%\iffalse
%<*samplechap1>
%\fi
% Some text for chapter 1:
%    \begin{macrocode}
\section{one}
some text in chapter one
%    \end{macrocode}

%\iffalse
%</samplechap1>
%\fi
% Some text for chapter 2:
%\iffalse
%<*samplechap2>
%\fi
%    \begin{macrocode}
\section{two}
more text in chapter two
%    \end{macrocode}

%\iffalse
%</samplechap2>
%\fi
%
% %%%%%%%%%%%%%%%%%%%%%%%%%%%%%%%%%%%%%%
% \paragraph{Part Include Files.}
%
% The include files are called |cdocspt3.tex| and |cdocspt4.tex|.
%
%\iffalse
%<*samplepart3|samplepart4>
%\fi

% Optional override for |\version| flag:
%    \begin{macrocode}
%%\providecommand{\version}{final}
%    \end{macrocode}

% Include the main document:
%    \begin{macrocode}
\input{childdoc.def}
\childdocby{cdocsamp}
%    \end{macrocode}

%\iffalse
%</samplepart3|samplepart4>
%\fi
%
%\iffalse
%<*samplepart3>
%\fi
% Some text for part 3:
%    \begin{macrocode}
some text in part three
%    \end{macrocode}

%\iffalse
%</samplepart3>
%\fi
% Some text for part 4:
%\iffalse
%<*samplepart4>
%\fi
%    \begin{macrocode}
more text in part four
%    \end{macrocode}

%\iffalse
%</samplepart4>
%\fi
%
% %%%%%%%%%%%%%%%%%%%%%%%%%%%%%%%%%%%%%%
% \paragraph{Forwarding for a Complete Draft.}
%
% The following forwarding file |cdocsdrf.tex|
% compiles the main document in draft mode:
%\iffalse
%<*sampledraft>
%\fi
%    \begin{macrocode}
\def\version{draft}
\input{childdoc.def}
\childdocforward{cdocsamp}
%    \end{macrocode}

%\iffalse
%</sampledraft>
%\fi
%
% %%%%%%%%%%%%%%%%%%%%%%%%%%%%%%%%%%%%%%
% \paragraph{Forwarding for Final Version of the Chapters.}
%
% The following forwarding files |cdocsfn1.tex| and |cdocsfn2.tex|
% (with identical content)
% compile the final versions of the child documents
% |cdocsch1.tex| and |cdocsch2.tex|, respectively:
%\iffalse
%<*samplefinal>
%\fi
%    \begin{macrocode}
\def\version{final}
\input{childdoc.def}
\childdocforwardprefix[cdocsamp]{cdocsfn}{cdocsch}
%    \end{macrocode}

%\iffalse
%</samplefinal>
%\fi
%
% %%%%%%%%%%%%%%%%%%%%%%%%%%%%%%%%%%%%%%
% \paragraph{Command Line Processing.}
%
% The following three command lines generate the output files
% |cdocscld|, |cdocscl1| and |cdocscl2|
% which should be identical to
% |cdocsdrf|, |cdocsch1| and |cdocsfn2|, respectively:
% \begin{center}
% \begin{tabular}{l}
% |latex -jobname cdocscld \|\\
% |  "\def\version{draft}\input{childdoc.def}\childdocforward{cdocsamp}"|\\
% |latex -jobname cdocscl1 \|\\
% |  "\input{childdoc.def}\childdocforward[cdocsamp]{cdocsch1}"|\\
% |latex -jobname cdocscl2 \|\\
% |  "\def\version{final}\input{childdoc.def}\childdocforward{cdocsch2}"|
% \end{tabular}
% \end{center}
% Note that the trailing backslash on each first line
% merely continues the input to the second line
% (for convenient cut ant paste).
% Furthermore, the command |latex| can be replaced by any
% of its alternative versions such as |pdflatex|.
%
% %%%%%%%%%%%%%%%%%%%%%%%%%%%%%%%%%%%%%%%%%%%%%%%%%%%%%%%%%%%%%%%%%%%%%%%%%%%%%%
% %%%%%%%%%%%%%%%%%%%%%%%%%%%%%%%%%%%%%%%%%%%%%%%%%%%%%%%%%%%%%%%%%%%%%%%%%%%%%%
% \section{Implementation}
%\iffalse
%<*package>
%\fi
%
% This section describes the definitions file |childdoc.def|.

% The definitions cannot be loaded using |\usepackage| or |\RequirePackage|
% which has a mechanism to prevent loading a style file more than once.
% When loading the definitions by means of |\input|
% multiple instances have to be prevented manually:
%\iffalse
%This code needs to be before the `\ProvidesFile' directive
%which is defined at the beginning of this file.
%Therefore it is also placed there and commented out here.
%</package>
%<*discard>
%\fi
%    \begin{macrocode}
\ifdefined\childdocmain\endinput\fi
%    \end{macrocode}
%\iffalse
%</discard>
%<*package>
%\fi
%
% \macro{\ifchilddoc}
% \macro{\ifchilddocmanual}
% The conditional |\ifchilddoc| tells whether a
% child (true) or main (false) document is being compiled.
% The conditional |\ifchilddocmanual| tells whether
% the |\includeonly| mechanism is used (false) or
% the selection of child files must be performed manually (true).
% The definitions initialise to false:
%    \begin{macrocode}
\newif\ifchilddoc
\newif\ifchilddocmanual
%    \end{macrocode}

% \macro{\childdocname}
% \macro{\childdocjob}
% The macro |\childdocname| stores the name of the main document
% to be compiled. The macro |\childdocjob| stores the name of
% the document on which the \LaTeX{} compiler was originally invoked.
% The content of |\jobname| cannot be compared
% to filenames specified in the source due to different catcodes.
% The following code rescans |\jobname|, stores the result
% in |\childdocname| and saves a copy in |\childdocjob|:
%    \begin{macrocode}
\edef\childdocname{\scantokens\expandafter{\jobname\noexpand}}
\let\childdocjob\childdocname
%    \end{macrocode}

% \macro{\childdocdisable}
% The macro |\childdocdisable| prevents the main file
% from being processed more than once.
% At this stage, the main document command |\childdocmain|
% is assumed to be called once again where it should do nothing.
% Any subsequent call to it should prevent
% a secondary processing of the main document
% It overwrites the forwarding commands
% |\childdocof| and |\childdocforward|
% with empty macros to prevent further inclusions of the main document:
%    \begin{macrocode}
\newcommand{\childdocdisable}
{
  \renewcommand{\childdocmain}[1]{\renewcommand{\childdocmain}[1]{\endinput}}
  \renewcommand{\childdocof}[1]{}
  \renewcommand{\childdocby}[2][]{}
  \renewcommand{\childdocforward}[2][]{}
  \renewcommand{\childdocdisable}{}
}
%    \end{macrocode}

% \macro{\childdocmain}
% The macro |\childdocmain| is to be called at the top of the main file
% with nothing or the main filename (without extension) as argument.
% First, it breaks loops.
% If the argument is not empty and does not match |\childdocname|
% (which is set by the first inclusion of |childdoc.def|),
% |\ifchilddoc| is set to true, |\includeonly| is applied to the child file
% and |\jobname| is set to the main file
% (for proper handling of |.aux| files):
%    \begin{macrocode}
\newcommand{\childdocmain}[1]
{
  \childdocdisable\childdocmain{}
  \if?#1?\else
    \begingroup
      \def\childdoctmp{#1}
      \ifx\childdoctmp\childdocname
        \def\childdoctmp{}
      \else
        \def\childdoctmp
        {
          \childdoctrue
          \includeonly{\childdocname}
          \def\childdocjob{#1}
          \def\jobname{#1}
        }
      \fi
      \expandafter
    \endgroup
    \childdoctmp
  \fi
}
%    \end{macrocode}

% \macro{\childdocof}
% The command |\childdocof| redirects
% compilation to the main file |#1|.
%    \begin{macrocode}
\newcommand{\childdocof}[1]
{
  \childdocdisable
  \childdoctrue
  \includeonly{\childdocname}
  \def\jobname{#1}
  \def\childdocjob{#1}
  \input{#1}
}
%    \end{macrocode}

% \macro{\childdocby}
% The command |\childdocby| ....
%    \begin{macrocode}
\newcommand{\childdocby}[2][]
{
  \childdocdisable
  \childdoctrue
  \childdocmanualtrue
  \if?#1?\else
    \def\jobname{#2}
  \fi
  \def\childdocjob{#2}
  \input{#2}
  \endinput
}
%    \end{macrocode}

% \macro{\childdocforward}
% The command |\childdocforward| redirects
% compilation to the main file or
% (if the optional argument is given) a child file.
% Parameters are set as if the main file
% or a child file starting with |\childdocof| was compiled.
% Then compilation is handed over to the main file:
%    \begin{macrocode}
\newcommand{\childdocforward}[2][]
{
  \begingroup
    \if?#1?
      \def\childdoctmp
      {
        \def\childdocname{#2}
        \def\childdocjob{#2}
        \def\jobname{#2}
        \input{#2}
        \endinput
      }
    \else
      \def\childdoctmp
      {
        \childdocdisable
        \def\childdocname{#2}
        \childdoctrue
        \includeonly{#2}
        \def\childdocjob{#1}
        \def\jobname{#1}
        \input{#1}
        \endinput
      }
    \fi
    \expandafter
  \endgroup
  \childdoctmp
}
%    \end{macrocode}

% \macro{\childdocforwardprefix}
% The command |\childdocforwardprefix| redirects
% compilation to the main or a child file by means of a pattern.
% The prefix |#1| in the current filename is replaced by |#2|
% and the suffix of the current filename is kept
% (it is assumed that the filename does not contain the substring `|~~~|'
% which is used as a delimiter).
% Compilation is handed over to the new file by |\childdocforward|:
%    \begin{macrocode}
\newcommand{\childdocforwardprefix}[3][]
{
  \begingroup
    \def\childdocextract #2##1~~~{\def\childdoctmp{\childdocforward[#1]{#3##1}}}
    \expandafter\childdocextract\childdocname~~~
    \expandafter
  \endgroup
  \childdoctmp
}
%    \end{macrocode}

% \macro{\childdoc}
% The deprecated macro |\childdoc| is a legacy version of |\childdocmain|:
%    \begin{macrocode}
\newcommand{\childdoc}{\childdocmain}
%    \end{macrocode}

% \macro{\childdocredirect}
% The deprecated macro |\childdocredirect| is a legacy version
% of |\childdocforward| and |\childdocforwardprefix|:
%    \begin{macrocode}
\newcommand{\childdocredirect}[2][]
{
  \begingroup
    \if?#1?
      \def\childdoctmp{\childdocforward{#2}}
    \else
      \def\childdoctmp{\childdocforwardprefix{#1}{#2}}
    \fi
    \expandafter
  \endgroup
  \childdoctmp
}
%    \end{macrocode}

%\iffalse
%</package>
%\fi
%
\endinput
|\\
|\childdocforwardprefix{final}{child}|
\end{tabular}
\end{center}
%

Note that when several versions of a main file and/or of each child file
are to be generated, it may be convenient to set up a |Makefile| or
shell script to automatise the process.

%%%%%%%%%%%%%%%%%%%%%%%%%%%%%%%%%%%%%%%%%%%%%%%%%%%%%%%%%%%%%%%%%%%%%%%%%%%%%%%%
\subsection{Command Line Processing}
\label{sec:commandline}

The effect of redirection files can also be achieved by invoking
the \LaTeX{} compiler with a more elaborate command line.
Most conveniently this should be done as part
of a shell script or a |Makefile|.

When using \textsf{childdoc} in the main file, the following
command lines effectively perform a redirection
(note that depending on the shell being used,
backslashes may have to be doubled: `|\|' $\to$ `|\\|'):
%
\begin{center}
|... -jobname "|\textit{target}|" |\\|"|[\textit{flags}]%
|% \iffalse
%
% childdoc.dtx Copyright (C) 2017-2018 Niklas Beisert
%
% This work may be distributed and/or modified under the
% conditions of the LaTeX Project Public License, either version 1.3
% of this license or (at your option) any later version.
% The latest version of this license is in
%   http://www.latex-project.org/lppl.txt
% and version 1.3 or later is part of all distributions of LaTeX
% version 2005/12/01 or later.
%
% This work has the LPPL maintenance status `maintained'.
%
% The Current Maintainer of this work is Niklas Beisert.
%
% This work consists of the files childdoc.dtx and childdoc.ins
% and the derived files childdoc.def and cdocsamp.tex with
% cdocsch1.tex, cdocsch2.tex, cdocsdrf.tex, cdocsfn1.tex, cdocsfn2.tex.
%
%<package>\ifdefined\childdocmain\endinput\fi
%<package>\ProvidesFile{childdoc.def}[2018/12/30 v2.0 child document driver]
%<samplemain>\ProvidesFile{cdocsamp.tex}[2018/12/30 v2.0 sample for childdoc]
%<*driver>
%\ProvidesFile{childdoc.drv}[2018/12/30 v2.0 childdoc reference manual file]
\PassOptionsToClass{10pt,a4paper}{article}
\documentclass{ltxdoc}

\usepackage[margin=35mm]{geometry}
\usepackage{hyperref}
\usepackage{hyperxmp}
\usepackage[usenames]{color}

\hypersetup{colorlinks=true}
\hypersetup{pdfstartview=FitH}
\hypersetup{pdfpagemode=UseNone}
\hypersetup{pdfsource={}}
\hypersetup{pdflang={en-UK}}
\hypersetup{pdfcopyright={Copyright 2017-2018 Niklas Beisert.
  This work may be distributed and/or modified under the
  conditions of the LaTeX Project Public License, either version 1.3
  of this license or (at your option) any later version.}}
\hypersetup{pdflicenseurl={http://www.latex-project.org/lppl.txt}}
\hypersetup{pdfcontactaddress={ETH Zurich, ITP, HIT K,
  Wolfgang-Pauli-Strasse 27}}
\hypersetup{pdfcontactpostcode={8093}}
\hypersetup{pdfcontactcity={Zurich}}
\hypersetup{pdfcontactcountry={Switzerland}}
\hypersetup{pdfcontactemail={nbeisert@itp.phys.ethz.ch}}
\hypersetup{pdfcontacturl={http://people.phys.ethz.ch/\xmptilde nbeisert/}}

\newcommand{\secref}[1]{\hyperref[#1]{section \ref*{#1}}}

\parskip1ex
\parindent0pt
\let\olditemize\itemize
\def\itemize{\olditemize\parskip0pt}

\begin{document}

\title{The \textsf{childdoc} Package}
\hypersetup{pdftitle={The childdoc Package}}
\author{Niklas Beisert\\[2ex]
  Institut f\"ur Theoretische Physik\\
  Eidgen\"ossische Technische Hochschule Z\"urich\\
  Wolfgang-Pauli-Strasse 27, 8093 Z\"urich, Switzerland\\[1ex]
  \href{mailto:nbeisert@itp.phys.ethz.ch}
  {\texttt{nbeisert@itp.phys.ethz.ch}}}
\hypersetup{pdfauthor={Niklas Beisert}}
\hypersetup{pdfsubject={Manual for the LaTeX2e Package childdoc}}
\date{30 December 2018, \textsf{v2.0}}
\maketitle

\begin{abstract}\noindent
\textsf{childdoc} is a \LaTeXe{} package
that enables the direct compilation
of document sections included by |\include|
to individual files.
\end{abstract}

\begingroup
\parskip0ex
\tableofcontents
\endgroup

%%%%%%%%%%%%%%%%%%%%%%%%%%%%%%%%%%%%%%%%%%%%%%%%%%%%%%%%%%%%%%%%%%%%%%%%%%%%%%%%
%%%%%%%%%%%%%%%%%%%%%%%%%%%%%%%%%%%%%%%%%%%%%%%%%%%%%%%%%%%%%%%%%%%%%%%%%%%%%%%%
\section{Introduction}

\LaTeX{} provides a mechanism to structure a large document (such as a book)
into a main file and several child files (containing the chapters)
using the |\include| command.
This mechanism is beneficial for documents
which span hundreds of pages in order to
make the source file(s) more manageable.
Moreover, compilation can be restricted to
selected child files by means of the |\includeonly| command.
The latter feature can be used to reduce the compilation time while editing
(this was significantly more useful in the earlier days of \LaTeX{})
or to generate a smaller document which is easier to navigate.
Another application of |\includeonly| is to generate
documents consisting of selected parts of the complete document.

However, there are a few drawbacks of the plain |\include| mechanism:
\begin{itemize}
\item
The child files cannot be compiled on their own,
they can only be compiled via the main file.
A naive editing environment
(such as a text editor with an option
to have the current file processed by \LaTeX)
may require one to switch to the main file before compiling;
attempting to compile the child file produces errors.
\item
The main file must be modified (each time)
to adjust the |\includeonly| command
to the present needs. This easily leaves the main file in a messy state.
\item
The generated document will always carry the filename
of the main document. This is inconvenient if
several child files are to be compiled and
to be kept for distribution.
\end{itemize}

The present package provides a simple interface
to make child files individually compilable by \LaTeX{}.
Compiling a child file then has the same effect as compiling
the main file with an |\includeonly| command
to select the appropriate child.
Moreover the generated document will carry the name of the child
rather than the main file.
This resolves all three above issues.

This feature is meant to make the editing of books,
thesis documents and lecture notes somewhat more convenient.
However, the package can also be used efficiently for
composing a series of documents (such as exercise sheets)
which are typically distributed individually.
It then assists the author in generating the individual documents
(potentially in different versions)
as well as a document containing the collected series.
Another application is in developing style files
or other kinds of included material
where compilation of the style file could redirect
to a sample or test file.

%%%%%%%%%%%%%%%%%%%%%%%%%%%%%%%%%%%%%%%%%%%%%%%%%%%%%%%%%%%%%%%%%%%%%%%%%%%%%%%%
%%%%%%%%%%%%%%%%%%%%%%%%%%%%%%%%%%%%%%%%%%%%%%%%%%%%%%%%%%%%%%%%%%%%%%%%%%%%%%%%
\section{Usage}

First of all, the package \textsf{childdoc} is \emph{not} a standard
\LaTeXe{} |.sty| style file! Therefore it needs to be invoked in
a non-standard way.

%%%%%%%%%%%%%%%%%%%%%%%%%%%%%%%%%%%%%%%%%%%%%%%%%%%%%%%%%%%%%%%%%%%%%%%%%%%%%%%%
\subsection{Included Files}
\label{sec:include}

%%%%%%%%%%%%%%%%%%%%%%%%%%%%%%%%%%%%%%%%
\DescribeMacro{\childdocmain}
To use the package, add the commands
\begin{center}
\begin{tabular}{l}
|\input{childdoc.def}|\\
|\childdocmain{}|\\
\end{tabular}
\end{center}
at the very top of the main \LaTeX{} file,
in particular \emph{before} the |\documentclass| statement!
The argument of |\childdocmain| should be left empty
(but it must be present).

%%%%%%%%%%%%%%%%%%%%%%%%%%%%%%%%%%%%%%%%
\DescribeMacro{\childdocof}
Furthermore, add the commands
\begin{center}
\begin{tabular}{l}
|\input{childdoc.def}|\\
|\childdocof{|\textit{main}|}|\\
\end{tabular}
\end{center}
at the top of every child file \textit{child}
which is included by |\include{|\textit{child}|}|
from within the main file
(or at least for those files to be compiled individually).
The argument \textit{main} must be the filename of the main file.

There are a couple of
considerations in setting up the main and child documents:

%%%%%%%%%%%%%%%%%%%%%%%%%%%%%%%%%%%%%%%%
\paragraph{Restrictions.}

Please note the following restrictions:
\begin{itemize}
\item
|\childdocmain| must be called with one argument \textit{main}
to ensure compatibility with earlier version of the package.
It must either be empty (|\childdocmain{}|)
or precisely match the filename of the main file in which it is specified.
See \secref{sec:detection} for further information.
\item
The filename \textit{main} must be specified without the |.tex| extension.
\item
The filename \textit{main} is case sensitive
(even in case-insensitive file systems)
due to internal string comparison.
\item
The argument \textit{main} should be fully expanded, it cannot be a macro.
\item
Subdirectories and special characters should be avoided in filenames.
\item
The command |\childdocmain{|\textit{main}|}| must be followed by a whitespace.
It should not be followed immediately by another command
or by a comment mark `|%|'.
This is because the \TeX{} parser reads the token immediately following
the argument of |\childdocmain| and puts it
at the beginning of every child section;
however, a white\-space is ignored.
\end{itemize}

%%%%%%%%%%%%%%%%%%%%%%%%%%%%%%%%%%%%%%%%
\paragraph{Content of Main File.}

It is advisable to place all content in the child files included by |\include|.
Any output contained in the main file will appear in all child documents
unless suppressed manually;
it cannot be suppressed automatically by the |\includeonly| directive
and thus should normally be avoided.
A method to include some content in the main file
by means of conditional processing is described in \secref{sec:conditional}.

%%%%%%%%%%%%%%%%%%%%%%%%%%%%%%%%%%%%%%%%
\paragraph{Page Numbering.}

When only a part of the document is compiled,
the appropriate numbering of pages
(as well as other status parameters)
is determined from the |.aux| files.
The latter contain information from previous passes.
However this information needs to propagate through
all intermediate child documents.
Therefore the page numbering in child documents may well
be inconsistent until the complete document is compiled at least once.

A useful (if unconventional) way to always ensure a consistent
page numbering is to restart the numbering in each child document
and denote the pages by `\textit{child}|.|\textit{page}'
where \textit{child} represents the chapter/section number of the child file.
This can be achieved by the command
|\numberwithin{page}{|\textit{child}|}|
of the \textsf{amsmath} package
where \textit{child} can be |chapter| or |section|
depending on the chosen structuring.
Alternatively, one can modify the macro |\thepage| appropriately
and reset the counter |page| at the start of each child file.

%%%%%%%%%%%%%%%%%%%%%%%%%%%%%%%%%%%%%%%%%%%%%%%%%%%%%%%%%%%%%%%%%%%%%%%%%%%%%%%%
\subsection{Conditional Processing}
\label{sec:conditional}

The package provides a mechanism to compile different versions
of a document. To customise the versions further some conditional processing
can come in handy to distinguish which version is being compiled.
The package provides two macros to describe the compilation context:

%%%%%%%%%%%%%%%%%%%%%%%%%%%%%%%%%%%%%%%%
\DescribeMacro{\ifchilddoc}
The conditional |\ifchilddoc| distinguishes between the compilation of
child documents and the main document:
%
\begin{center}
|\ifchilddoc |\textit{child-code}| |[|\||else |\textit{main-code}]| \||fi|
\end{center}

%%%%%%%%%%%%%%%%%%%%%%%%%%%%%%%%%%%%%%%%
\DescribeMacro{\childdocname}
\DescribeMacro{\childdocjob}
The macro |\childdocname| contains the filename (without extension)
of the main or child file being processed.
Note that |\childdocjob| will always contain the name of the main file.

%%%%%%%%%%%%%%%%%%%%%%%%%%%%%%%%%%%%%%%%
\paragraph{Title Page.}

Conditional processing can be used to include a title or banner page
in the main document when proper precautions are taken.
Importantly, the code in the main file should ensure that the page counter
(as well as other status parameters which are stored in the |.aux| files)
takes the same value after the conditional processing.
Otherwise the page numbers may take divergent values
depending on which part is compiled.

For example, a title page could be declared by:
%
\begin{center}
\begin{tabular}{l}
|\ifchilddoc\||else|\\
|\addtocounter{page}{-1}|\\
\textit{code for title page}\\
|\newpage|\\
|\||fi|
\end{tabular}
\end{center}
%
A banner page for the child documents can be generated by:
%
\begin{center}
\begin{tabular}{l}
|\ifchilddoc|\\
|\addtocounter{page}{-1}|\\
\textit{code for banner page}\\
|\newpage|\\
|\||fi|
\end{tabular}
\end{center}
%
Here one could write a message such as:
\begin{center}
|This is the part \childdocname{} of \childdocjob{}.|
\end{center}

%%%%%%%%%%%%%%%%%%%%%%%%%%%%%%%%%%%%%%%%%%%%%%%%%%%%%%%%%%%%%%%%%%%%%%%%%%%%%%%%
\subsection{Flags}
\label{sec:flags}

The package makes it easy to generate different versions
of the main or child documents.
To this end compilation flags can be defined
and assigned different default values.
They will be particularly useful in conjunction
with the forwarding mechanism described in \secref{sec:forward}.

For example, it may be useful to have a flag |\version|
which can be set to |draft| or |final|.
The document source will contain some conditional code
depending on the value of |\version|.
Suppose further, the flag should default to |final| for the main file
and to |draft| for child files
which is a natural assignment for editing the document.
This is achieved by placing the following code
in the preamble of the main document
(below the |\childdocmain| directive):
%
\begin{center}
\begin{tabular}{l}
|\ifchilddoc|\\
|\providecommand{\version}{draft}|\\
|\||else|\\
|\providecommand{\version}{final}|\\
|\||fi|
\end{tabular}
\end{center}
%
The definition by |\providecommand| makes sure
that previous definitions are not overwritten.
Further statements |\providecommand{\version}{...}|
can thus be added before the above code to override it.

For the main file, one might add a line
(between |\childdocmain| and the above block)
%
\begin{center}
|%\ifchilddoc\||else\providecommand{\version}{draft}\||fi|
\end{center}
%
which can be uncommented to produce a draft version.
Likewise one can add a line to the very top of a child file
(above the |\childdocof{|\textit{main}|}| directive)
%
\begin{center}
|%\providecommand{\version}{final}|
\end{center}
%
which can be uncommented to produce the final version of this child document.

%%%%%%%%%%%%%%%%%%%%%%%%%%%%%%%%%%%%%%%%%%%%%%%%%%%%%%%%%%%%%%%%%%%%%%%%%%%%%%%%
\subsection{Forwarding}
\label{sec:forward}

Different versions of the main or child documents
using compilation flags as described in \secref{sec:flags}
can be (permanently) stored in different files
for convenient compilation, viewing and distribution.
To this end, the package defines a command
to pass on compilation to a different file:

%%%%%%%%%%%%%%%%%%%%%%%%%%%%%%%%%%%%%%%%
\DescribeMacro{\childdocforward}
The command |\childdocforward| redirects processing to
another source file:
%
\begin{center}
\begin{tabular}{l}
|\input{childdoc.def}|\\
|\childdocforward[|\textit{main}|]{|\textit{dest}|}|\\
\end{tabular}
\end{center}
%
The argument \textit{dest} is the destination file
(without extension).
It should be the main file or one of the child files.
Note that further \textsf{childdoc} directives
such as |\childdocof| and |\childdocforward|
in the indicated file will be processed in this form.
The optional argument \textit{main}
passes on directly to the main file \textit{main}
while pretending to compile the child \textit{dest}.
This form behaves as if \textit{dest}
issues |\childdocof{|\textit{main}|}| right away,
and no further \textsf{childdoc} directives will be processed.

%%%%%%%%%%%%%%%%%%%%%%%%%%%%%%%%%%%%%%%%
\DescribeMacro{\...prefix}
In the alternative form |\childdocforwardprefix|,
%
\begin{center}
\begin{tabular}{l}
|\input{childdoc.def}|\\
|\childdocforwardprefix[|\textit{main}|]{|\textit{prefix}|}{|\textit{dest}|}|
\end{tabular}
\end{center}
%
the destination file is determined by a pattern
depending on the current file:
To make this work, the current file must be called
`{\textit{prefix}\hspace{0.2em}\textit{suffix}}'
with \textit{prefix} matching precisely the argument.
Processing is then passed on to the file
`{\textit{dest}\hspace{0.2em}\textit{suffix}}'.
Surely, the same effect is achieved by
directly specifying the
argument `{\textit{dest}\hspace{0.2em}\textit{suffix}}'
in the first form.
However, that requires to set up a different file
for each child. With the alternative form of the command
all these files can have exactly the same content
which simplifies setting them up and maintaining them.

For example, the following file |draft.tex|
with a compilation flag |\version| as described in \secref{sec:flags}
compiles the main document as a draft:
%
\begin{center}
\begin{tabular}{l}
|\def\version{draft}|\\
|\input{childdoc.def}|\\
|\childdocforward{|\textit{main}|}|
\end{tabular}
\end{center}
%
Likewise, the following files |final|\textit{nn}|.tex|
compile the final version of the child document
|child|\textit{nn}|.tex|:
%
\begin{center}
\begin{tabular}{l}
|\def\version{final}|\\
|\input{childdoc.def}|\\
|\childdocforwardprefix{final}{child}|
\end{tabular}
\end{center}
%

Note that when several versions of a main file and/or of each child file
are to be generated, it may be convenient to set up a |Makefile| or
shell script to automatise the process.

%%%%%%%%%%%%%%%%%%%%%%%%%%%%%%%%%%%%%%%%%%%%%%%%%%%%%%%%%%%%%%%%%%%%%%%%%%%%%%%%
\subsection{Command Line Processing}
\label{sec:commandline}

The effect of redirection files can also be achieved by invoking
the \LaTeX{} compiler with a more elaborate command line.
Most conveniently this should be done as part
of a shell script or a |Makefile|.

When using \textsf{childdoc} in the main file, the following
command lines effectively perform a redirection
(note that depending on the shell being used,
backslashes may have to be doubled: `|\|' $\to$ `|\\|'):
%
\begin{center}
|... -jobname "|\textit{target}|" |\\|"|[\textit{flags}]%
|\input{childdoc.def}\childdocforward[|\textit{main}|]{|\textit{dest}|}"|
\end{center}
%
Here \textit{target} is the name of the output file,
\textit{main} is the name of the main file
and \textit{dest} is the name of the main or child file to be processed
(all filenames without extensions).
The optional argument \textit{main} can be omitted
if \textit{main} matches \textit{dest}.
Optionally, compilation \textit{flags} can be defined via |\def| commands.
This command line makes the \TeX{} engine believe
it is compiling the file \textit{target}
whose content is specified as the latter parameter.
The provided code then forwards the processing to
\textit{main} or \textit{dest} as described in \secref{sec:forward}.

%%%%%%%%%%%%%%%%%%%%%%%%%%%%%%%%%%%%%%%%%%%%%%%%%%%%%%%%%%%%%%%%%%%%%%%%%%%%%%%%
\subsection{Include by Input}
\label{sec:input}

Including child documents by |\include| has some restrictions by design.
Most notably, the content of a child document always occupies
its own set of pages; pages cannot be shared between child documents.
Usually, this behaviour makes perfect sense
because each child document contain an essential part of the document.
However, in some situations it may be desirable to compose
a document from a collection of parts
without having mandatory page breaks between then.
For this case, the package
provides a mechanism to include parts
by |\input| which can also be processed individually.
However, by construction this mechanism
requires manual handling of the content to be output.

%%%%%%%%%%%%%%%%%%%%%%%%%%%%%%%%%%%%%%%%
\DescribeMacro{\ifchilddocmanual}
The main file should be prepared as usual, see \secref{sec:include}.
However, the document body must make a distinction
between processing of an individual part and of the main document, e.g.:
%
\begin{center}
\begin{tabular}{l}
|\ifchilddocmanual|\\
|\input{\childdocname}|\\
|\||else|\\
\textit{document body with }|\input{|\textit{part}|}|\\
|\||fi|
\end{tabular}
\end{center}
%
The conditional |\ifchilddocmanual| is true whenever
a part to be included by |\input| is being compiled,
and the name of the part is stored in |\childdocname|.

%%%%%%%%%%%%%%%%%%%%%%%%%%%%%%%%%%%%%%%%
\DescribeMacro{\childdocby}
Each part to be included by |\input| should start with:
%
\begin{center}
\begin{tabular}{l}
|\input{childdoc.def}|\\
|\childdocby{|\textit{main}|}|\\
\end{tabular}
\end{center}
%
The directive |\childdocby| is similar to |\childdocof|
described in \secref{sec:include},
but the subsequent selection of content must be done manually.
To that end, both |\ifchilddoc| and |\ifchilddocmanual|
will be true upon processing of a part,
and the name of the part is stored in |\childdocname|.
Note that |\jobname| will be set to the filename of the current part
so that each part receives an individual |.aux| file
that does not interfere with the |.aux| file(s) of the main document.
This behaviour can be altered by the alternative form
|\childdocby[*]{|\textit{main}|}| (with a non-empty optional argument)
which uses the |.aux| file of the main document
by setting |\jobname| to \textit{main}.

%%%%%%%%%%%%%%%%%%%%%%%%%%%%%%%%%%%%%%%%%%%%%%%%%%%%%%%%%%%%%%%%%%%%%%%%%%%%%%%%
\subsection{Driver Development}
\label{sec:driver}

The \textsf{childdoc} mechanism can also be use for the development
of definition files such as \LaTeX{} styles or classes.
This case differs from the above setup with multiple parts
included by |\include| in that no |\includeonly| should be invoked.
This can be achieved by starting the include file
(before |\ProvidesPackage|) with:
%
\begin{center}
\begin{tabular}{l}
|\input{childdoc.def}|\\
|\childdocforward{|\textit{main}|}|\\
\end{tabular}
\end{center}
%
or alternatively with:
%
\begin{center}
\begin{tabular}{l}
|\input{childdoc.def}|\\
|\childdocby{|\textit{main}|}|\\
\end{tabular}
\end{center}
%
Both forms have slightly different effects as described above.
The main file is prepared as usual, see \secref{sec:include}.

%%%%%%%%%%%%%%%%%%%%%%%%%%%%%%%%%%%%%%%%%%%%%%%%%%%%%%%%%%%%%%%%%%%%%%%%%%%%%%%%
\subsection{Legacy Detection}
\label{sec:detection}

The directive |\childdocmain| in the main file can detect
whether the complete document or merely a child is to be compiled
even without using the directive |\childdocof|.
This method is deprecated because it is less robust
and there is no compelling reason to use it;
it is merely provided for backward compatibility
and it may be removed in future versions.

If the detection mechanism is to be used,
it is mandatory to correctly specify
the filename of the main file as the argument of |\childdocmain|:
%
\begin{center}
\begin{tabular}{l}
|\input{childdoc.def}|\\
|\childdocmain{|\textit{main}|}|\\
\end{tabular}
\end{center}
%
If |\jobname| does not match the argument \textit{main} of |\childdocmain|,
it is assumed that |\jobname| points to the child file to be compiled.
When using |\childdocmain| with the main file specified as argument,
it suffices to start a child file
with just |\input{|\textit{main}|}|
without loading of the package and using |\childdocof|.
If instead all processing is done
with the appropriate \textsf{childdoc} directives,
the argument of \textit{main} of |\childdocmain| can be empty.

An alternative version of the command line processing described
in \secref{sec:commandline} using the detection mechanism reads:
%
\begin{center}
|... -jobname "|\textit{target}|" "|[\textit{flags}]%
[|\def\jobname{|\textit{dest}|}|]|\input{|\textit{main}|}"|
\end{center}

%%%%%%%%%%%%%%%%%%%%%%%%%%%%%%%%%%%%%%%%%%%%%%%%%%%%%%%%%%%%%%%%%%%%%%%%%%%%%%%%
\subsection{Manual Code}
\label{sec:manual}

In case one cannot be certain whether the definitions file |childdoc.def|
is installed on the target \TeX{} distribution
and one prefers not to ship it,
it is conceivable to paste a few relevant commands into the sources.

To that end, drop all statements |\input{childdoc.def}|
and perform the replacements as outlined below.
Instead of |\childdocmain{|\textit{main}|}| add the following code
to the top of the main file:
%
\begin{center}
\begin{tabular}{l}
|\||ifdefined\childdocname\endinput\||fi\newif\ifchilddoc|\\
|\edef\childdocname{\scantokens\expandafter{\jobname\noexpand}}|\\
|\def\childdocmain{|\textit{main}|}\||ifx\childdocmain\childdocname\||else|\\
|\childdoctrue\includeonly{\childdocname}\let\jobname\childdocmain\||fi|\\
\end{tabular}
\end{center}
%
Instead of |\childdocof{|\textit{main}|}| just include the main file
at the top of each child file:
%
\begin{center}
|\input{|\textit{main}|}|
\end{center}
%
A simple redirection |\childdocforward{|\textit{dest}|}| is achieved by:
%
\begin{center}
|\def\jobname{|\textit{dest}|}\input{\jobname}|
\end{center}
%
The redirection with prefix
|\childdocforwardprefix[|\textit{prefix}|]{|\textit{dest}|}|
is accomplished by:
%
\begin{center}
\begin{tabular}{l}
|{\edef\jobname{\scantokens\expandafter{\jobname\noexpand}}|\\
|\def\redirectjob |\textit{prefix}|#1~~~{\gdef\jobname{|\textit{dest}|#1}}|\\
|\expandafter\redirectjob\jobname~~~}\input{\jobname}|
\end{tabular}
\end{center}

In an alternative approach,
child documents can be compiled by a specific command line
without additional code or specific definitions:
%
\begin{center}
|... -jobname "|\textit{target}|" "|[\textit{flags}]%
|\includeonly{|\textit{dest}|}\input{|\textit{main}|}"|
\end{center}
%

%%%%%%%%%%%%%%%%%%%%%%%%%%%%%%%%%%%%%%%%%%%%%%%%%%%%%%%%%%%%%%%%%%%%%%%%%%%%%%%%
%%%%%%%%%%%%%%%%%%%%%%%%%%%%%%%%%%%%%%%%%%%%%%%%%%%%%%%%%%%%%%%%%%%%%%%%%%%%%%%%
\section{Information}

%%%%%%%%%%%%%%%%%%%%%%%%%%%%%%%%%%%%%%%%%%%%%%%%%%%%%%%%%%%%%%%%%%%%%%%%%%%%%%%%
\subsection{Copyright}

Copyright \copyright{} 2017--2018 Niklas Beisert

This work may be distributed and/or modified under the
conditions of the \LaTeX{} Project Public License, either version 1.3
of this license or (at your option) any later version.
The latest version of this license is in
  \url{http://www.latex-project.org/lppl.txt}
and version 1.3 or later is part of all distributions of \LaTeX{}
version 2005/12/01 or later.

This work has the LPPL maintenance status `maintained'.

The Current Maintainer of this work is Niklas Beisert.

This work consists of the files |README.txt|, |childdoc.ins| and |childdoc.dtx|
as well as the derived files |childdoc.def|, |cdocsamp.tex|
with |cdocsch1.tex|, |cdocsch2.tex|, |cdocspt3.tex|, |cdocspt4.tex|,
|cdocsdrf.tex|, |cdocsfn1.tex|, |cdocsfn2.tex|
as well as |childdoc.pdf|.

%%%%%%%%%%%%%%%%%%%%%%%%%%%%%%%%%%%%%%%%%%%%%%%%%%%%%%%%%%%%%%%%%%%%%%%%%%%%%%%%
\subsection{Files and Installation}

The package consists of the files:
%
\begin{center}
\begin{tabular}{ll}
    |README.txt|   & readme file \\
    |childdoc.ins| & installation file \\
    |childdoc.dtx| & source file \\
    |childdoc.def| & definition file \\
    |cdocsamp.tex| & sample main file \\
    |cdocsch1.tex| & sample include file \\
    |cdocsch2.tex| & sample include file \\
    |cdocspt3.tex| & sample part file \\
    |cdocspt4.tex| & sample part file \\
    |cdocsdrf.tex| & sample redirection file \\
    |cdocsfn1.tex| & sample redirection file \\
    |cdocsfn2.tex| & sample redirection file \\
    |childdoc.pdf| & manual
\end{tabular}
\end{center}
%
The distribution consists of the files
|README.txt|, |childdoc.ins| and |childdoc.dtx|.
%
\begin{itemize}
\item
Run (pdf)\LaTeX{} on |childdoc.dtx|
to compile the manual |childdoc.pdf| (this file).
\item
Run \LaTeX{} on |childdoc.ins| to create the definitions file |childdoc.def|
and the sample |cdocsamp.tex| with include files
|cdocsch1.tex|, |cdocsch2.tex|, |cdocspt3.tex|, |cdocspt4.tex|,
|cdocsdrf.tex|, |cdocsfn1.tex|, |cdocsfn2.tex|.
Then copy the file |childdoc.def| to an appropriate directory of your \LaTeX{}
distribution, e.g.\ \textit{texmf-root}|/tex/latex/childdoc|.
\end{itemize}

%%%%%%%%%%%%%%%%%%%%%%%%%%%%%%%%%%%%%%%%%%%%%%%%%%%%%%%%%%%%%%%%%%%%%%%%%%%%%%%%
\subsection{Related CTAN Packages}

There are several other packages which offer a similar functionality:
%
\begin{itemize}
\item
The packages
\href{http://ctan.org/pkg/docmute}{\textsf{docmute}},
\href{http://ctan.org/pkg/includex}{\textsf{includex}} and
\href{http://ctan.org/pkg/standalone}{\textsf{standalone}}
provide commands to include only the document body of
a child file thus allowing both files to be compiled individually.
\item
The packages \href{http://ctan.org/pkg/subdocs}{\textsf{subdocs}}
and \href{http://ctan.org/pkg/subfiles}{\textsf{subfiles}}
provide structures in which the main and child documents can be
encapsulated and allowing them to be compiled individually.
The inclusion mechanism is different from the conventional |\include|.
\item
The package \href{http://ctan.org/pkg/combine}{\textsf{combine}}
is an elaborate solution to combine several documents into one.
\end{itemize}
%
See also the CTAN topic \href{http://ctan.org/topic/subdocs}{\textsf{subdocs}}
for further related packages.
The present package differs from the above solutions in that
a document structure constructed with the conventional |\include| mechanism
just needs two extra commands at the top of every file
such that all constituent files can be compiled individually.

%%%%%%%%%%%%%%%%%%%%%%%%%%%%%%%%%%%%%%%%%%%%%%%%%%%%%%%%%%%%%%%%%%%%%%%%%%%%%%%%
%\subsection{Feature Suggestions}
%
%The following is a list of features which may be useful for future
%versions of this package:
%%
%\begin{itemize}
%\item
%\ldots
%\end{itemize}

%%%%%%%%%%%%%%%%%%%%%%%%%%%%%%%%%%%%%%%%%%%%%%%%%%%%%%%%%%%%%%%%%%%%%%%%%%%%%%%%
\subsection{Revision History}

%%%%%%%%%%%%%%%%%%%%%%%%%%%%%%%%%%%%%%%%
\paragraph{v2.0:} 2018/12/30

\begin{itemize}
\item
immediate forward processing
\item
added |\childdocby| mechanism
\item
manual restructured
\end{itemize}

%%%%%%%%%%%%%%%%%%%%%%%%%%%%%%%%%%%%%%%%
\paragraph{v1.6:} 2018/01/17

\begin{itemize}
\item
application for development of include files
\item
corrections to manual
\end{itemize}

%%%%%%%%%%%%%%%%%%%%%%%%%%%%%%%%%%%%%%%%
\paragraph{v1.5:} 2017/05/21

\begin{itemize}
\item
more complete structuring introduced
\item
|\childdocof| introduced
\item
|\childdoc| renamed to |\childdocmain|
\item
|\childredirect| renamed to |\childdocforward| and |\childdocforwardprefix|
and functionality expanded
\end{itemize}

%%%%%%%%%%%%%%%%%%%%%%%%%%%%%%%%%%%%%%%%
\paragraph{v1.0:} 2017/04/27

\begin{itemize}
\item
manual and install package
\item
first version published on CTAN
\end{itemize}

%%%%%%%%%%%%%%%%%%%%%%%%%%%%%%%%%%%%%%%%
\paragraph{v0.6:} 2017/04/26

\begin{itemize}
\item
redirection mechanism added
\end{itemize}

%%%%%%%%%%%%%%%%%%%%%%%%%%%%%%%%%%%%%%%%
\paragraph{v0.5:} 2017/04/26

\begin{itemize}
\item
functionality in definition file
\end{itemize}


%%%%%%%%%%%%%%%%%%%%%%%%%%%%%%%%%%%%%%%%%%%%%%%%%%%%%%%%%%%%%%%%%%%%%%%%%%%%%%%%
%%%%%%%%%%%%%%%%%%%%%%%%%%%%%%%%%%%%%%%%%%%%%%%%%%%%%%%%%%%%%%%%%%%%%%%%%%%%%%%%
%%%%%%%%%%%%%%%%%%%%%%%%%%%%%%%%%%%%%%%%%%%%%%%%%%%%%%%%%%%%%%%%%%%%%%%%%%%%%%%%
\appendix

\settowidth\MacroIndent{\rmfamily\scriptsize 000\ }

 \DocInput{childdoc.dtx}

\end{document}
%</driver>
% \fi
%
% %%%%%%%%%%%%%%%%%%%%%%%%%%%%%%%%%%%%%%%%%%%%%%%%%%%%%%%%%%%%%%%%%%%%%%%%%%%%%%
% %%%%%%%%%%%%%%%%%%%%%%%%%%%%%%%%%%%%%%%%%%%%%%%%%%%%%%%%%%%%%%%%%%%%%%%%%%%%%%
% \section{Sample}
%\iffalse
%<*samplemain>
%\fi
%
% The following presents a sample document
% with two chapters, two parts, a title page,
% a compile flag as well as three forwarding files to set the flag.
% It consists of eight |.tex| files:
% \begin{center}
% \begin{tabular}{ll}
% |cdocsamp.tex|&main file\\
% |cdocsch1.tex|&include file for chapter 1\\
% |cdocsch2.tex|&include file for chapter 2\\
% |cdocspt3.tex|&include file for part 3\\
% |cdocspt4.tex|&include file for part 4\\
% |cdocsdrf.tex|&forwarding file for main file in draft mode\\
% |cdocsfi1.tex|&forwarding file for final version of chapter 1\\
% |cdocsfi2.tex|&forwarding file for final version of chapter 2\\
% \end{tabular}
% \end{center}
% Each of the eight files can be compiled directly by the \LaTeX{} compiler.
%
% %%%%%%%%%%%%%%%%%%%%%%%%%%%%%%%%%%%%%%
% \paragraph{Main File.}
%
% The main file is called |cdocsamp.tex|.
%
% Load the \textsf{childdoc} definitions and
% declare the filename for the main document:
%    \begin{macrocode}
\input{childdoc.def}
\childdocmain{}
%    \end{macrocode}

% Optional override for |\version| flag:
%    \begin{macrocode}
%%\ifchilddoc\else\providecommand{\version}{draft}\fi
%    \end{macrocode}

% Define the default values for the |\version| flag
% (|final| for the main file and |draft| for childs):
%    \begin{macrocode}
\ifchilddoc
\providecommand{\version}{draft}
\else
\providecommand{\version}{final}
\fi
%    \end{macrocode}

% Load the standard document class:
%    \begin{macrocode}
\documentclass[12pt]{article}
%    \end{macrocode}

% Start the document body:
%    \begin{macrocode}
\begin{document}
%    \end{macrocode}

% Declare a title page.
% Print title, part of document being processed and version flag:
%    \begin{macrocode}
\addtocounter{page}{-1}
\begin{center}
{\LARGE\bfseries{}childdoc example\par}
\vspace{1cm}
\ifchilddoc
\ifchilddocmanual part\else chapter\fi:
`\childdocname' of `\childdocjob'\par
\else
main document: `\childdocjob'\par
\fi
version: \version\par
\end{center}
\newpage
%    \end{macrocode}

% Manually include selected file,
% otherwise process as usual:
%    \begin{macrocode}
\ifchilddocmanual
\section*{part `\childdocname'}
\input{\childdocname}
\else
%    \end{macrocode}

% Include the two chapters:
%    \begin{macrocode}
\include{cdocsch1}
\include{cdocsch2}
%    \end{macrocode}

% Include the two parts unless only chapters should be displayed:
%    \begin{macrocode}
\ifchilddoc\else
\section{part three}
\input{cdocspt3}
\section{part four}
\input{cdocspt4}
\fi
%    \end{macrocode}

% Process as usual until here:
%    \begin{macrocode}
\fi
%    \end{macrocode}

% End of document body:
%    \begin{macrocode}
\end{document}
%    \end{macrocode}
%\iffalse
%</samplemain>
%\fi
%
% %%%%%%%%%%%%%%%%%%%%%%%%%%%%%%%%%%%%%%
% \paragraph{Chapter Include Files.}
%
% The include files are called |cdocsch1.tex| and |cdocsch2.tex|.
%
%\iffalse
%<*samplechap1|samplechap2>
%\fi

% Optional override for |\version| flag:
%    \begin{macrocode}
%%\providecommand{\version}{final}
%    \end{macrocode}

% Include the main document:
%    \begin{macrocode}
\input{childdoc.def}
\childdocof{cdocsamp}
%    \end{macrocode}

%\iffalse
%</samplechap1|samplechap2>
%\fi
%
%\iffalse
%<*samplechap1>
%\fi
% Some text for chapter 1:
%    \begin{macrocode}
\section{one}
some text in chapter one
%    \end{macrocode}

%\iffalse
%</samplechap1>
%\fi
% Some text for chapter 2:
%\iffalse
%<*samplechap2>
%\fi
%    \begin{macrocode}
\section{two}
more text in chapter two
%    \end{macrocode}

%\iffalse
%</samplechap2>
%\fi
%
% %%%%%%%%%%%%%%%%%%%%%%%%%%%%%%%%%%%%%%
% \paragraph{Part Include Files.}
%
% The include files are called |cdocspt3.tex| and |cdocspt4.tex|.
%
%\iffalse
%<*samplepart3|samplepart4>
%\fi

% Optional override for |\version| flag:
%    \begin{macrocode}
%%\providecommand{\version}{final}
%    \end{macrocode}

% Include the main document:
%    \begin{macrocode}
\input{childdoc.def}
\childdocby{cdocsamp}
%    \end{macrocode}

%\iffalse
%</samplepart3|samplepart4>
%\fi
%
%\iffalse
%<*samplepart3>
%\fi
% Some text for part 3:
%    \begin{macrocode}
some text in part three
%    \end{macrocode}

%\iffalse
%</samplepart3>
%\fi
% Some text for part 4:
%\iffalse
%<*samplepart4>
%\fi
%    \begin{macrocode}
more text in part four
%    \end{macrocode}

%\iffalse
%</samplepart4>
%\fi
%
% %%%%%%%%%%%%%%%%%%%%%%%%%%%%%%%%%%%%%%
% \paragraph{Forwarding for a Complete Draft.}
%
% The following forwarding file |cdocsdrf.tex|
% compiles the main document in draft mode:
%\iffalse
%<*sampledraft>
%\fi
%    \begin{macrocode}
\def\version{draft}
\input{childdoc.def}
\childdocforward{cdocsamp}
%    \end{macrocode}

%\iffalse
%</sampledraft>
%\fi
%
% %%%%%%%%%%%%%%%%%%%%%%%%%%%%%%%%%%%%%%
% \paragraph{Forwarding for Final Version of the Chapters.}
%
% The following forwarding files |cdocsfn1.tex| and |cdocsfn2.tex|
% (with identical content)
% compile the final versions of the child documents
% |cdocsch1.tex| and |cdocsch2.tex|, respectively:
%\iffalse
%<*samplefinal>
%\fi
%    \begin{macrocode}
\def\version{final}
\input{childdoc.def}
\childdocforwardprefix[cdocsamp]{cdocsfn}{cdocsch}
%    \end{macrocode}

%\iffalse
%</samplefinal>
%\fi
%
% %%%%%%%%%%%%%%%%%%%%%%%%%%%%%%%%%%%%%%
% \paragraph{Command Line Processing.}
%
% The following three command lines generate the output files
% |cdocscld|, |cdocscl1| and |cdocscl2|
% which should be identical to
% |cdocsdrf|, |cdocsch1| and |cdocsfn2|, respectively:
% \begin{center}
% \begin{tabular}{l}
% |latex -jobname cdocscld \|\\
% |  "\def\version{draft}\input{childdoc.def}\childdocforward{cdocsamp}"|\\
% |latex -jobname cdocscl1 \|\\
% |  "\input{childdoc.def}\childdocforward[cdocsamp]{cdocsch1}"|\\
% |latex -jobname cdocscl2 \|\\
% |  "\def\version{final}\input{childdoc.def}\childdocforward{cdocsch2}"|
% \end{tabular}
% \end{center}
% Note that the trailing backslash on each first line
% merely continues the input to the second line
% (for convenient cut ant paste).
% Furthermore, the command |latex| can be replaced by any
% of its alternative versions such as |pdflatex|.
%
% %%%%%%%%%%%%%%%%%%%%%%%%%%%%%%%%%%%%%%%%%%%%%%%%%%%%%%%%%%%%%%%%%%%%%%%%%%%%%%
% %%%%%%%%%%%%%%%%%%%%%%%%%%%%%%%%%%%%%%%%%%%%%%%%%%%%%%%%%%%%%%%%%%%%%%%%%%%%%%
% \section{Implementation}
%\iffalse
%<*package>
%\fi
%
% This section describes the definitions file |childdoc.def|.

% The definitions cannot be loaded using |\usepackage| or |\RequirePackage|
% which has a mechanism to prevent loading a style file more than once.
% When loading the definitions by means of |\input|
% multiple instances have to be prevented manually:
%\iffalse
%This code needs to be before the `\ProvidesFile' directive
%which is defined at the beginning of this file.
%Therefore it is also placed there and commented out here.
%</package>
%<*discard>
%\fi
%    \begin{macrocode}
\ifdefined\childdocmain\endinput\fi
%    \end{macrocode}
%\iffalse
%</discard>
%<*package>
%\fi
%
% \macro{\ifchilddoc}
% \macro{\ifchilddocmanual}
% The conditional |\ifchilddoc| tells whether a
% child (true) or main (false) document is being compiled.
% The conditional |\ifchilddocmanual| tells whether
% the |\includeonly| mechanism is used (false) or
% the selection of child files must be performed manually (true).
% The definitions initialise to false:
%    \begin{macrocode}
\newif\ifchilddoc
\newif\ifchilddocmanual
%    \end{macrocode}

% \macro{\childdocname}
% \macro{\childdocjob}
% The macro |\childdocname| stores the name of the main document
% to be compiled. The macro |\childdocjob| stores the name of
% the document on which the \LaTeX{} compiler was originally invoked.
% The content of |\jobname| cannot be compared
% to filenames specified in the source due to different catcodes.
% The following code rescans |\jobname|, stores the result
% in |\childdocname| and saves a copy in |\childdocjob|:
%    \begin{macrocode}
\edef\childdocname{\scantokens\expandafter{\jobname\noexpand}}
\let\childdocjob\childdocname
%    \end{macrocode}

% \macro{\childdocdisable}
% The macro |\childdocdisable| prevents the main file
% from being processed more than once.
% At this stage, the main document command |\childdocmain|
% is assumed to be called once again where it should do nothing.
% Any subsequent call to it should prevent
% a secondary processing of the main document
% It overwrites the forwarding commands
% |\childdocof| and |\childdocforward|
% with empty macros to prevent further inclusions of the main document:
%    \begin{macrocode}
\newcommand{\childdocdisable}
{
  \renewcommand{\childdocmain}[1]{\renewcommand{\childdocmain}[1]{\endinput}}
  \renewcommand{\childdocof}[1]{}
  \renewcommand{\childdocby}[2][]{}
  \renewcommand{\childdocforward}[2][]{}
  \renewcommand{\childdocdisable}{}
}
%    \end{macrocode}

% \macro{\childdocmain}
% The macro |\childdocmain| is to be called at the top of the main file
% with nothing or the main filename (without extension) as argument.
% First, it breaks loops.
% If the argument is not empty and does not match |\childdocname|
% (which is set by the first inclusion of |childdoc.def|),
% |\ifchilddoc| is set to true, |\includeonly| is applied to the child file
% and |\jobname| is set to the main file
% (for proper handling of |.aux| files):
%    \begin{macrocode}
\newcommand{\childdocmain}[1]
{
  \childdocdisable\childdocmain{}
  \if?#1?\else
    \begingroup
      \def\childdoctmp{#1}
      \ifx\childdoctmp\childdocname
        \def\childdoctmp{}
      \else
        \def\childdoctmp
        {
          \childdoctrue
          \includeonly{\childdocname}
          \def\childdocjob{#1}
          \def\jobname{#1}
        }
      \fi
      \expandafter
    \endgroup
    \childdoctmp
  \fi
}
%    \end{macrocode}

% \macro{\childdocof}
% The command |\childdocof| redirects
% compilation to the main file |#1|.
%    \begin{macrocode}
\newcommand{\childdocof}[1]
{
  \childdocdisable
  \childdoctrue
  \includeonly{\childdocname}
  \def\jobname{#1}
  \def\childdocjob{#1}
  \input{#1}
}
%    \end{macrocode}

% \macro{\childdocby}
% The command |\childdocby| ....
%    \begin{macrocode}
\newcommand{\childdocby}[2][]
{
  \childdocdisable
  \childdoctrue
  \childdocmanualtrue
  \if?#1?\else
    \def\jobname{#2}
  \fi
  \def\childdocjob{#2}
  \input{#2}
  \endinput
}
%    \end{macrocode}

% \macro{\childdocforward}
% The command |\childdocforward| redirects
% compilation to the main file or
% (if the optional argument is given) a child file.
% Parameters are set as if the main file
% or a child file starting with |\childdocof| was compiled.
% Then compilation is handed over to the main file:
%    \begin{macrocode}
\newcommand{\childdocforward}[2][]
{
  \begingroup
    \if?#1?
      \def\childdoctmp
      {
        \def\childdocname{#2}
        \def\childdocjob{#2}
        \def\jobname{#2}
        \input{#2}
        \endinput
      }
    \else
      \def\childdoctmp
      {
        \childdocdisable
        \def\childdocname{#2}
        \childdoctrue
        \includeonly{#2}
        \def\childdocjob{#1}
        \def\jobname{#1}
        \input{#1}
        \endinput
      }
    \fi
    \expandafter
  \endgroup
  \childdoctmp
}
%    \end{macrocode}

% \macro{\childdocforwardprefix}
% The command |\childdocforwardprefix| redirects
% compilation to the main or a child file by means of a pattern.
% The prefix |#1| in the current filename is replaced by |#2|
% and the suffix of the current filename is kept
% (it is assumed that the filename does not contain the substring `|~~~|'
% which is used as a delimiter).
% Compilation is handed over to the new file by |\childdocforward|:
%    \begin{macrocode}
\newcommand{\childdocforwardprefix}[3][]
{
  \begingroup
    \def\childdocextract #2##1~~~{\def\childdoctmp{\childdocforward[#1]{#3##1}}}
    \expandafter\childdocextract\childdocname~~~
    \expandafter
  \endgroup
  \childdoctmp
}
%    \end{macrocode}

% \macro{\childdoc}
% The deprecated macro |\childdoc| is a legacy version of |\childdocmain|:
%    \begin{macrocode}
\newcommand{\childdoc}{\childdocmain}
%    \end{macrocode}

% \macro{\childdocredirect}
% The deprecated macro |\childdocredirect| is a legacy version
% of |\childdocforward| and |\childdocforwardprefix|:
%    \begin{macrocode}
\newcommand{\childdocredirect}[2][]
{
  \begingroup
    \if?#1?
      \def\childdoctmp{\childdocforward{#2}}
    \else
      \def\childdoctmp{\childdocforwardprefix{#1}{#2}}
    \fi
    \expandafter
  \endgroup
  \childdoctmp
}
%    \end{macrocode}

%\iffalse
%</package>
%\fi
%
\endinput
\childdocforward[|\textit{main}|]{|\textit{dest}|}"|
\end{center}
%
Here \textit{target} is the name of the output file,
\textit{main} is the name of the main file
and \textit{dest} is the name of the main or child file to be processed
(all filenames without extensions).
The optional argument \textit{main} can be omitted
if \textit{main} matches \textit{dest}.
Optionally, compilation \textit{flags} can be defined via |\def| commands.
This command line makes the \TeX{} engine believe
it is compiling the file \textit{target}
whose content is specified as the latter parameter.
The provided code then forwards the processing to
\textit{main} or \textit{dest} as described in \secref{sec:forward}.

%%%%%%%%%%%%%%%%%%%%%%%%%%%%%%%%%%%%%%%%%%%%%%%%%%%%%%%%%%%%%%%%%%%%%%%%%%%%%%%%
\subsection{Include by Input}
\label{sec:input}

Including child documents by |\include| has some restrictions by design.
Most notably, the content of a child document always occupies
its own set of pages; pages cannot be shared between child documents.
Usually, this behaviour makes perfect sense
because each child document contain an essential part of the document.
However, in some situations it may be desirable to compose
a document from a collection of parts
without having mandatory page breaks between then.
For this case, the package
provides a mechanism to include parts
by |\input| which can also be processed individually.
However, by construction this mechanism
requires manual handling of the content to be output.

%%%%%%%%%%%%%%%%%%%%%%%%%%%%%%%%%%%%%%%%
\DescribeMacro{\ifchilddocmanual}
The main file should be prepared as usual, see \secref{sec:include}.
However, the document body must make a distinction
between processing of an individual part and of the main document, e.g.:
%
\begin{center}
\begin{tabular}{l}
|\ifchilddocmanual|\\
|\input{\childdocname}|\\
|\||else|\\
\textit{document body with }|\input{|\textit{part}|}|\\
|\||fi|
\end{tabular}
\end{center}
%
The conditional |\ifchilddocmanual| is true whenever
a part to be included by |\input| is being compiled,
and the name of the part is stored in |\childdocname|.

%%%%%%%%%%%%%%%%%%%%%%%%%%%%%%%%%%%%%%%%
\DescribeMacro{\childdocby}
Each part to be included by |\input| should start with:
%
\begin{center}
\begin{tabular}{l}
|% \iffalse
%
% childdoc.dtx Copyright (C) 2017-2018 Niklas Beisert
%
% This work may be distributed and/or modified under the
% conditions of the LaTeX Project Public License, either version 1.3
% of this license or (at your option) any later version.
% The latest version of this license is in
%   http://www.latex-project.org/lppl.txt
% and version 1.3 or later is part of all distributions of LaTeX
% version 2005/12/01 or later.
%
% This work has the LPPL maintenance status `maintained'.
%
% The Current Maintainer of this work is Niklas Beisert.
%
% This work consists of the files childdoc.dtx and childdoc.ins
% and the derived files childdoc.def and cdocsamp.tex with
% cdocsch1.tex, cdocsch2.tex, cdocsdrf.tex, cdocsfn1.tex, cdocsfn2.tex.
%
%<package>\ifdefined\childdocmain\endinput\fi
%<package>\ProvidesFile{childdoc.def}[2018/12/30 v2.0 child document driver]
%<samplemain>\ProvidesFile{cdocsamp.tex}[2018/12/30 v2.0 sample for childdoc]
%<*driver>
%\ProvidesFile{childdoc.drv}[2018/12/30 v2.0 childdoc reference manual file]
\PassOptionsToClass{10pt,a4paper}{article}
\documentclass{ltxdoc}

\usepackage[margin=35mm]{geometry}
\usepackage{hyperref}
\usepackage{hyperxmp}
\usepackage[usenames]{color}

\hypersetup{colorlinks=true}
\hypersetup{pdfstartview=FitH}
\hypersetup{pdfpagemode=UseNone}
\hypersetup{pdfsource={}}
\hypersetup{pdflang={en-UK}}
\hypersetup{pdfcopyright={Copyright 2017-2018 Niklas Beisert.
  This work may be distributed and/or modified under the
  conditions of the LaTeX Project Public License, either version 1.3
  of this license or (at your option) any later version.}}
\hypersetup{pdflicenseurl={http://www.latex-project.org/lppl.txt}}
\hypersetup{pdfcontactaddress={ETH Zurich, ITP, HIT K,
  Wolfgang-Pauli-Strasse 27}}
\hypersetup{pdfcontactpostcode={8093}}
\hypersetup{pdfcontactcity={Zurich}}
\hypersetup{pdfcontactcountry={Switzerland}}
\hypersetup{pdfcontactemail={nbeisert@itp.phys.ethz.ch}}
\hypersetup{pdfcontacturl={http://people.phys.ethz.ch/\xmptilde nbeisert/}}

\newcommand{\secref}[1]{\hyperref[#1]{section \ref*{#1}}}

\parskip1ex
\parindent0pt
\let\olditemize\itemize
\def\itemize{\olditemize\parskip0pt}

\begin{document}

\title{The \textsf{childdoc} Package}
\hypersetup{pdftitle={The childdoc Package}}
\author{Niklas Beisert\\[2ex]
  Institut f\"ur Theoretische Physik\\
  Eidgen\"ossische Technische Hochschule Z\"urich\\
  Wolfgang-Pauli-Strasse 27, 8093 Z\"urich, Switzerland\\[1ex]
  \href{mailto:nbeisert@itp.phys.ethz.ch}
  {\texttt{nbeisert@itp.phys.ethz.ch}}}
\hypersetup{pdfauthor={Niklas Beisert}}
\hypersetup{pdfsubject={Manual for the LaTeX2e Package childdoc}}
\date{30 December 2018, \textsf{v2.0}}
\maketitle

\begin{abstract}\noindent
\textsf{childdoc} is a \LaTeXe{} package
that enables the direct compilation
of document sections included by |\include|
to individual files.
\end{abstract}

\begingroup
\parskip0ex
\tableofcontents
\endgroup

%%%%%%%%%%%%%%%%%%%%%%%%%%%%%%%%%%%%%%%%%%%%%%%%%%%%%%%%%%%%%%%%%%%%%%%%%%%%%%%%
%%%%%%%%%%%%%%%%%%%%%%%%%%%%%%%%%%%%%%%%%%%%%%%%%%%%%%%%%%%%%%%%%%%%%%%%%%%%%%%%
\section{Introduction}

\LaTeX{} provides a mechanism to structure a large document (such as a book)
into a main file and several child files (containing the chapters)
using the |\include| command.
This mechanism is beneficial for documents
which span hundreds of pages in order to
make the source file(s) more manageable.
Moreover, compilation can be restricted to
selected child files by means of the |\includeonly| command.
The latter feature can be used to reduce the compilation time while editing
(this was significantly more useful in the earlier days of \LaTeX{})
or to generate a smaller document which is easier to navigate.
Another application of |\includeonly| is to generate
documents consisting of selected parts of the complete document.

However, there are a few drawbacks of the plain |\include| mechanism:
\begin{itemize}
\item
The child files cannot be compiled on their own,
they can only be compiled via the main file.
A naive editing environment
(such as a text editor with an option
to have the current file processed by \LaTeX)
may require one to switch to the main file before compiling;
attempting to compile the child file produces errors.
\item
The main file must be modified (each time)
to adjust the |\includeonly| command
to the present needs. This easily leaves the main file in a messy state.
\item
The generated document will always carry the filename
of the main document. This is inconvenient if
several child files are to be compiled and
to be kept for distribution.
\end{itemize}

The present package provides a simple interface
to make child files individually compilable by \LaTeX{}.
Compiling a child file then has the same effect as compiling
the main file with an |\includeonly| command
to select the appropriate child.
Moreover the generated document will carry the name of the child
rather than the main file.
This resolves all three above issues.

This feature is meant to make the editing of books,
thesis documents and lecture notes somewhat more convenient.
However, the package can also be used efficiently for
composing a series of documents (such as exercise sheets)
which are typically distributed individually.
It then assists the author in generating the individual documents
(potentially in different versions)
as well as a document containing the collected series.
Another application is in developing style files
or other kinds of included material
where compilation of the style file could redirect
to a sample or test file.

%%%%%%%%%%%%%%%%%%%%%%%%%%%%%%%%%%%%%%%%%%%%%%%%%%%%%%%%%%%%%%%%%%%%%%%%%%%%%%%%
%%%%%%%%%%%%%%%%%%%%%%%%%%%%%%%%%%%%%%%%%%%%%%%%%%%%%%%%%%%%%%%%%%%%%%%%%%%%%%%%
\section{Usage}

First of all, the package \textsf{childdoc} is \emph{not} a standard
\LaTeXe{} |.sty| style file! Therefore it needs to be invoked in
a non-standard way.

%%%%%%%%%%%%%%%%%%%%%%%%%%%%%%%%%%%%%%%%%%%%%%%%%%%%%%%%%%%%%%%%%%%%%%%%%%%%%%%%
\subsection{Included Files}
\label{sec:include}

%%%%%%%%%%%%%%%%%%%%%%%%%%%%%%%%%%%%%%%%
\DescribeMacro{\childdocmain}
To use the package, add the commands
\begin{center}
\begin{tabular}{l}
|\input{childdoc.def}|\\
|\childdocmain{}|\\
\end{tabular}
\end{center}
at the very top of the main \LaTeX{} file,
in particular \emph{before} the |\documentclass| statement!
The argument of |\childdocmain| should be left empty
(but it must be present).

%%%%%%%%%%%%%%%%%%%%%%%%%%%%%%%%%%%%%%%%
\DescribeMacro{\childdocof}
Furthermore, add the commands
\begin{center}
\begin{tabular}{l}
|\input{childdoc.def}|\\
|\childdocof{|\textit{main}|}|\\
\end{tabular}
\end{center}
at the top of every child file \textit{child}
which is included by |\include{|\textit{child}|}|
from within the main file
(or at least for those files to be compiled individually).
The argument \textit{main} must be the filename of the main file.

There are a couple of
considerations in setting up the main and child documents:

%%%%%%%%%%%%%%%%%%%%%%%%%%%%%%%%%%%%%%%%
\paragraph{Restrictions.}

Please note the following restrictions:
\begin{itemize}
\item
|\childdocmain| must be called with one argument \textit{main}
to ensure compatibility with earlier version of the package.
It must either be empty (|\childdocmain{}|)
or precisely match the filename of the main file in which it is specified.
See \secref{sec:detection} for further information.
\item
The filename \textit{main} must be specified without the |.tex| extension.
\item
The filename \textit{main} is case sensitive
(even in case-insensitive file systems)
due to internal string comparison.
\item
The argument \textit{main} should be fully expanded, it cannot be a macro.
\item
Subdirectories and special characters should be avoided in filenames.
\item
The command |\childdocmain{|\textit{main}|}| must be followed by a whitespace.
It should not be followed immediately by another command
or by a comment mark `|%|'.
This is because the \TeX{} parser reads the token immediately following
the argument of |\childdocmain| and puts it
at the beginning of every child section;
however, a white\-space is ignored.
\end{itemize}

%%%%%%%%%%%%%%%%%%%%%%%%%%%%%%%%%%%%%%%%
\paragraph{Content of Main File.}

It is advisable to place all content in the child files included by |\include|.
Any output contained in the main file will appear in all child documents
unless suppressed manually;
it cannot be suppressed automatically by the |\includeonly| directive
and thus should normally be avoided.
A method to include some content in the main file
by means of conditional processing is described in \secref{sec:conditional}.

%%%%%%%%%%%%%%%%%%%%%%%%%%%%%%%%%%%%%%%%
\paragraph{Page Numbering.}

When only a part of the document is compiled,
the appropriate numbering of pages
(as well as other status parameters)
is determined from the |.aux| files.
The latter contain information from previous passes.
However this information needs to propagate through
all intermediate child documents.
Therefore the page numbering in child documents may well
be inconsistent until the complete document is compiled at least once.

A useful (if unconventional) way to always ensure a consistent
page numbering is to restart the numbering in each child document
and denote the pages by `\textit{child}|.|\textit{page}'
where \textit{child} represents the chapter/section number of the child file.
This can be achieved by the command
|\numberwithin{page}{|\textit{child}|}|
of the \textsf{amsmath} package
where \textit{child} can be |chapter| or |section|
depending on the chosen structuring.
Alternatively, one can modify the macro |\thepage| appropriately
and reset the counter |page| at the start of each child file.

%%%%%%%%%%%%%%%%%%%%%%%%%%%%%%%%%%%%%%%%%%%%%%%%%%%%%%%%%%%%%%%%%%%%%%%%%%%%%%%%
\subsection{Conditional Processing}
\label{sec:conditional}

The package provides a mechanism to compile different versions
of a document. To customise the versions further some conditional processing
can come in handy to distinguish which version is being compiled.
The package provides two macros to describe the compilation context:

%%%%%%%%%%%%%%%%%%%%%%%%%%%%%%%%%%%%%%%%
\DescribeMacro{\ifchilddoc}
The conditional |\ifchilddoc| distinguishes between the compilation of
child documents and the main document:
%
\begin{center}
|\ifchilddoc |\textit{child-code}| |[|\||else |\textit{main-code}]| \||fi|
\end{center}

%%%%%%%%%%%%%%%%%%%%%%%%%%%%%%%%%%%%%%%%
\DescribeMacro{\childdocname}
\DescribeMacro{\childdocjob}
The macro |\childdocname| contains the filename (without extension)
of the main or child file being processed.
Note that |\childdocjob| will always contain the name of the main file.

%%%%%%%%%%%%%%%%%%%%%%%%%%%%%%%%%%%%%%%%
\paragraph{Title Page.}

Conditional processing can be used to include a title or banner page
in the main document when proper precautions are taken.
Importantly, the code in the main file should ensure that the page counter
(as well as other status parameters which are stored in the |.aux| files)
takes the same value after the conditional processing.
Otherwise the page numbers may take divergent values
depending on which part is compiled.

For example, a title page could be declared by:
%
\begin{center}
\begin{tabular}{l}
|\ifchilddoc\||else|\\
|\addtocounter{page}{-1}|\\
\textit{code for title page}\\
|\newpage|\\
|\||fi|
\end{tabular}
\end{center}
%
A banner page for the child documents can be generated by:
%
\begin{center}
\begin{tabular}{l}
|\ifchilddoc|\\
|\addtocounter{page}{-1}|\\
\textit{code for banner page}\\
|\newpage|\\
|\||fi|
\end{tabular}
\end{center}
%
Here one could write a message such as:
\begin{center}
|This is the part \childdocname{} of \childdocjob{}.|
\end{center}

%%%%%%%%%%%%%%%%%%%%%%%%%%%%%%%%%%%%%%%%%%%%%%%%%%%%%%%%%%%%%%%%%%%%%%%%%%%%%%%%
\subsection{Flags}
\label{sec:flags}

The package makes it easy to generate different versions
of the main or child documents.
To this end compilation flags can be defined
and assigned different default values.
They will be particularly useful in conjunction
with the forwarding mechanism described in \secref{sec:forward}.

For example, it may be useful to have a flag |\version|
which can be set to |draft| or |final|.
The document source will contain some conditional code
depending on the value of |\version|.
Suppose further, the flag should default to |final| for the main file
and to |draft| for child files
which is a natural assignment for editing the document.
This is achieved by placing the following code
in the preamble of the main document
(below the |\childdocmain| directive):
%
\begin{center}
\begin{tabular}{l}
|\ifchilddoc|\\
|\providecommand{\version}{draft}|\\
|\||else|\\
|\providecommand{\version}{final}|\\
|\||fi|
\end{tabular}
\end{center}
%
The definition by |\providecommand| makes sure
that previous definitions are not overwritten.
Further statements |\providecommand{\version}{...}|
can thus be added before the above code to override it.

For the main file, one might add a line
(between |\childdocmain| and the above block)
%
\begin{center}
|%\ifchilddoc\||else\providecommand{\version}{draft}\||fi|
\end{center}
%
which can be uncommented to produce a draft version.
Likewise one can add a line to the very top of a child file
(above the |\childdocof{|\textit{main}|}| directive)
%
\begin{center}
|%\providecommand{\version}{final}|
\end{center}
%
which can be uncommented to produce the final version of this child document.

%%%%%%%%%%%%%%%%%%%%%%%%%%%%%%%%%%%%%%%%%%%%%%%%%%%%%%%%%%%%%%%%%%%%%%%%%%%%%%%%
\subsection{Forwarding}
\label{sec:forward}

Different versions of the main or child documents
using compilation flags as described in \secref{sec:flags}
can be (permanently) stored in different files
for convenient compilation, viewing and distribution.
To this end, the package defines a command
to pass on compilation to a different file:

%%%%%%%%%%%%%%%%%%%%%%%%%%%%%%%%%%%%%%%%
\DescribeMacro{\childdocforward}
The command |\childdocforward| redirects processing to
another source file:
%
\begin{center}
\begin{tabular}{l}
|\input{childdoc.def}|\\
|\childdocforward[|\textit{main}|]{|\textit{dest}|}|\\
\end{tabular}
\end{center}
%
The argument \textit{dest} is the destination file
(without extension).
It should be the main file or one of the child files.
Note that further \textsf{childdoc} directives
such as |\childdocof| and |\childdocforward|
in the indicated file will be processed in this form.
The optional argument \textit{main}
passes on directly to the main file \textit{main}
while pretending to compile the child \textit{dest}.
This form behaves as if \textit{dest}
issues |\childdocof{|\textit{main}|}| right away,
and no further \textsf{childdoc} directives will be processed.

%%%%%%%%%%%%%%%%%%%%%%%%%%%%%%%%%%%%%%%%
\DescribeMacro{\...prefix}
In the alternative form |\childdocforwardprefix|,
%
\begin{center}
\begin{tabular}{l}
|\input{childdoc.def}|\\
|\childdocforwardprefix[|\textit{main}|]{|\textit{prefix}|}{|\textit{dest}|}|
\end{tabular}
\end{center}
%
the destination file is determined by a pattern
depending on the current file:
To make this work, the current file must be called
`{\textit{prefix}\hspace{0.2em}\textit{suffix}}'
with \textit{prefix} matching precisely the argument.
Processing is then passed on to the file
`{\textit{dest}\hspace{0.2em}\textit{suffix}}'.
Surely, the same effect is achieved by
directly specifying the
argument `{\textit{dest}\hspace{0.2em}\textit{suffix}}'
in the first form.
However, that requires to set up a different file
for each child. With the alternative form of the command
all these files can have exactly the same content
which simplifies setting them up and maintaining them.

For example, the following file |draft.tex|
with a compilation flag |\version| as described in \secref{sec:flags}
compiles the main document as a draft:
%
\begin{center}
\begin{tabular}{l}
|\def\version{draft}|\\
|\input{childdoc.def}|\\
|\childdocforward{|\textit{main}|}|
\end{tabular}
\end{center}
%
Likewise, the following files |final|\textit{nn}|.tex|
compile the final version of the child document
|child|\textit{nn}|.tex|:
%
\begin{center}
\begin{tabular}{l}
|\def\version{final}|\\
|\input{childdoc.def}|\\
|\childdocforwardprefix{final}{child}|
\end{tabular}
\end{center}
%

Note that when several versions of a main file and/or of each child file
are to be generated, it may be convenient to set up a |Makefile| or
shell script to automatise the process.

%%%%%%%%%%%%%%%%%%%%%%%%%%%%%%%%%%%%%%%%%%%%%%%%%%%%%%%%%%%%%%%%%%%%%%%%%%%%%%%%
\subsection{Command Line Processing}
\label{sec:commandline}

The effect of redirection files can also be achieved by invoking
the \LaTeX{} compiler with a more elaborate command line.
Most conveniently this should be done as part
of a shell script or a |Makefile|.

When using \textsf{childdoc} in the main file, the following
command lines effectively perform a redirection
(note that depending on the shell being used,
backslashes may have to be doubled: `|\|' $\to$ `|\\|'):
%
\begin{center}
|... -jobname "|\textit{target}|" |\\|"|[\textit{flags}]%
|\input{childdoc.def}\childdocforward[|\textit{main}|]{|\textit{dest}|}"|
\end{center}
%
Here \textit{target} is the name of the output file,
\textit{main} is the name of the main file
and \textit{dest} is the name of the main or child file to be processed
(all filenames without extensions).
The optional argument \textit{main} can be omitted
if \textit{main} matches \textit{dest}.
Optionally, compilation \textit{flags} can be defined via |\def| commands.
This command line makes the \TeX{} engine believe
it is compiling the file \textit{target}
whose content is specified as the latter parameter.
The provided code then forwards the processing to
\textit{main} or \textit{dest} as described in \secref{sec:forward}.

%%%%%%%%%%%%%%%%%%%%%%%%%%%%%%%%%%%%%%%%%%%%%%%%%%%%%%%%%%%%%%%%%%%%%%%%%%%%%%%%
\subsection{Include by Input}
\label{sec:input}

Including child documents by |\include| has some restrictions by design.
Most notably, the content of a child document always occupies
its own set of pages; pages cannot be shared between child documents.
Usually, this behaviour makes perfect sense
because each child document contain an essential part of the document.
However, in some situations it may be desirable to compose
a document from a collection of parts
without having mandatory page breaks between then.
For this case, the package
provides a mechanism to include parts
by |\input| which can also be processed individually.
However, by construction this mechanism
requires manual handling of the content to be output.

%%%%%%%%%%%%%%%%%%%%%%%%%%%%%%%%%%%%%%%%
\DescribeMacro{\ifchilddocmanual}
The main file should be prepared as usual, see \secref{sec:include}.
However, the document body must make a distinction
between processing of an individual part and of the main document, e.g.:
%
\begin{center}
\begin{tabular}{l}
|\ifchilddocmanual|\\
|\input{\childdocname}|\\
|\||else|\\
\textit{document body with }|\input{|\textit{part}|}|\\
|\||fi|
\end{tabular}
\end{center}
%
The conditional |\ifchilddocmanual| is true whenever
a part to be included by |\input| is being compiled,
and the name of the part is stored in |\childdocname|.

%%%%%%%%%%%%%%%%%%%%%%%%%%%%%%%%%%%%%%%%
\DescribeMacro{\childdocby}
Each part to be included by |\input| should start with:
%
\begin{center}
\begin{tabular}{l}
|\input{childdoc.def}|\\
|\childdocby{|\textit{main}|}|\\
\end{tabular}
\end{center}
%
The directive |\childdocby| is similar to |\childdocof|
described in \secref{sec:include},
but the subsequent selection of content must be done manually.
To that end, both |\ifchilddoc| and |\ifchilddocmanual|
will be true upon processing of a part,
and the name of the part is stored in |\childdocname|.
Note that |\jobname| will be set to the filename of the current part
so that each part receives an individual |.aux| file
that does not interfere with the |.aux| file(s) of the main document.
This behaviour can be altered by the alternative form
|\childdocby[*]{|\textit{main}|}| (with a non-empty optional argument)
which uses the |.aux| file of the main document
by setting |\jobname| to \textit{main}.

%%%%%%%%%%%%%%%%%%%%%%%%%%%%%%%%%%%%%%%%%%%%%%%%%%%%%%%%%%%%%%%%%%%%%%%%%%%%%%%%
\subsection{Driver Development}
\label{sec:driver}

The \textsf{childdoc} mechanism can also be use for the development
of definition files such as \LaTeX{} styles or classes.
This case differs from the above setup with multiple parts
included by |\include| in that no |\includeonly| should be invoked.
This can be achieved by starting the include file
(before |\ProvidesPackage|) with:
%
\begin{center}
\begin{tabular}{l}
|\input{childdoc.def}|\\
|\childdocforward{|\textit{main}|}|\\
\end{tabular}
\end{center}
%
or alternatively with:
%
\begin{center}
\begin{tabular}{l}
|\input{childdoc.def}|\\
|\childdocby{|\textit{main}|}|\\
\end{tabular}
\end{center}
%
Both forms have slightly different effects as described above.
The main file is prepared as usual, see \secref{sec:include}.

%%%%%%%%%%%%%%%%%%%%%%%%%%%%%%%%%%%%%%%%%%%%%%%%%%%%%%%%%%%%%%%%%%%%%%%%%%%%%%%%
\subsection{Legacy Detection}
\label{sec:detection}

The directive |\childdocmain| in the main file can detect
whether the complete document or merely a child is to be compiled
even without using the directive |\childdocof|.
This method is deprecated because it is less robust
and there is no compelling reason to use it;
it is merely provided for backward compatibility
and it may be removed in future versions.

If the detection mechanism is to be used,
it is mandatory to correctly specify
the filename of the main file as the argument of |\childdocmain|:
%
\begin{center}
\begin{tabular}{l}
|\input{childdoc.def}|\\
|\childdocmain{|\textit{main}|}|\\
\end{tabular}
\end{center}
%
If |\jobname| does not match the argument \textit{main} of |\childdocmain|,
it is assumed that |\jobname| points to the child file to be compiled.
When using |\childdocmain| with the main file specified as argument,
it suffices to start a child file
with just |\input{|\textit{main}|}|
without loading of the package and using |\childdocof|.
If instead all processing is done
with the appropriate \textsf{childdoc} directives,
the argument of \textit{main} of |\childdocmain| can be empty.

An alternative version of the command line processing described
in \secref{sec:commandline} using the detection mechanism reads:
%
\begin{center}
|... -jobname "|\textit{target}|" "|[\textit{flags}]%
[|\def\jobname{|\textit{dest}|}|]|\input{|\textit{main}|}"|
\end{center}

%%%%%%%%%%%%%%%%%%%%%%%%%%%%%%%%%%%%%%%%%%%%%%%%%%%%%%%%%%%%%%%%%%%%%%%%%%%%%%%%
\subsection{Manual Code}
\label{sec:manual}

In case one cannot be certain whether the definitions file |childdoc.def|
is installed on the target \TeX{} distribution
and one prefers not to ship it,
it is conceivable to paste a few relevant commands into the sources.

To that end, drop all statements |\input{childdoc.def}|
and perform the replacements as outlined below.
Instead of |\childdocmain{|\textit{main}|}| add the following code
to the top of the main file:
%
\begin{center}
\begin{tabular}{l}
|\||ifdefined\childdocname\endinput\||fi\newif\ifchilddoc|\\
|\edef\childdocname{\scantokens\expandafter{\jobname\noexpand}}|\\
|\def\childdocmain{|\textit{main}|}\||ifx\childdocmain\childdocname\||else|\\
|\childdoctrue\includeonly{\childdocname}\let\jobname\childdocmain\||fi|\\
\end{tabular}
\end{center}
%
Instead of |\childdocof{|\textit{main}|}| just include the main file
at the top of each child file:
%
\begin{center}
|\input{|\textit{main}|}|
\end{center}
%
A simple redirection |\childdocforward{|\textit{dest}|}| is achieved by:
%
\begin{center}
|\def\jobname{|\textit{dest}|}\input{\jobname}|
\end{center}
%
The redirection with prefix
|\childdocforwardprefix[|\textit{prefix}|]{|\textit{dest}|}|
is accomplished by:
%
\begin{center}
\begin{tabular}{l}
|{\edef\jobname{\scantokens\expandafter{\jobname\noexpand}}|\\
|\def\redirectjob |\textit{prefix}|#1~~~{\gdef\jobname{|\textit{dest}|#1}}|\\
|\expandafter\redirectjob\jobname~~~}\input{\jobname}|
\end{tabular}
\end{center}

In an alternative approach,
child documents can be compiled by a specific command line
without additional code or specific definitions:
%
\begin{center}
|... -jobname "|\textit{target}|" "|[\textit{flags}]%
|\includeonly{|\textit{dest}|}\input{|\textit{main}|}"|
\end{center}
%

%%%%%%%%%%%%%%%%%%%%%%%%%%%%%%%%%%%%%%%%%%%%%%%%%%%%%%%%%%%%%%%%%%%%%%%%%%%%%%%%
%%%%%%%%%%%%%%%%%%%%%%%%%%%%%%%%%%%%%%%%%%%%%%%%%%%%%%%%%%%%%%%%%%%%%%%%%%%%%%%%
\section{Information}

%%%%%%%%%%%%%%%%%%%%%%%%%%%%%%%%%%%%%%%%%%%%%%%%%%%%%%%%%%%%%%%%%%%%%%%%%%%%%%%%
\subsection{Copyright}

Copyright \copyright{} 2017--2018 Niklas Beisert

This work may be distributed and/or modified under the
conditions of the \LaTeX{} Project Public License, either version 1.3
of this license or (at your option) any later version.
The latest version of this license is in
  \url{http://www.latex-project.org/lppl.txt}
and version 1.3 or later is part of all distributions of \LaTeX{}
version 2005/12/01 or later.

This work has the LPPL maintenance status `maintained'.

The Current Maintainer of this work is Niklas Beisert.

This work consists of the files |README.txt|, |childdoc.ins| and |childdoc.dtx|
as well as the derived files |childdoc.def|, |cdocsamp.tex|
with |cdocsch1.tex|, |cdocsch2.tex|, |cdocspt3.tex|, |cdocspt4.tex|,
|cdocsdrf.tex|, |cdocsfn1.tex|, |cdocsfn2.tex|
as well as |childdoc.pdf|.

%%%%%%%%%%%%%%%%%%%%%%%%%%%%%%%%%%%%%%%%%%%%%%%%%%%%%%%%%%%%%%%%%%%%%%%%%%%%%%%%
\subsection{Files and Installation}

The package consists of the files:
%
\begin{center}
\begin{tabular}{ll}
    |README.txt|   & readme file \\
    |childdoc.ins| & installation file \\
    |childdoc.dtx| & source file \\
    |childdoc.def| & definition file \\
    |cdocsamp.tex| & sample main file \\
    |cdocsch1.tex| & sample include file \\
    |cdocsch2.tex| & sample include file \\
    |cdocspt3.tex| & sample part file \\
    |cdocspt4.tex| & sample part file \\
    |cdocsdrf.tex| & sample redirection file \\
    |cdocsfn1.tex| & sample redirection file \\
    |cdocsfn2.tex| & sample redirection file \\
    |childdoc.pdf| & manual
\end{tabular}
\end{center}
%
The distribution consists of the files
|README.txt|, |childdoc.ins| and |childdoc.dtx|.
%
\begin{itemize}
\item
Run (pdf)\LaTeX{} on |childdoc.dtx|
to compile the manual |childdoc.pdf| (this file).
\item
Run \LaTeX{} on |childdoc.ins| to create the definitions file |childdoc.def|
and the sample |cdocsamp.tex| with include files
|cdocsch1.tex|, |cdocsch2.tex|, |cdocspt3.tex|, |cdocspt4.tex|,
|cdocsdrf.tex|, |cdocsfn1.tex|, |cdocsfn2.tex|.
Then copy the file |childdoc.def| to an appropriate directory of your \LaTeX{}
distribution, e.g.\ \textit{texmf-root}|/tex/latex/childdoc|.
\end{itemize}

%%%%%%%%%%%%%%%%%%%%%%%%%%%%%%%%%%%%%%%%%%%%%%%%%%%%%%%%%%%%%%%%%%%%%%%%%%%%%%%%
\subsection{Related CTAN Packages}

There are several other packages which offer a similar functionality:
%
\begin{itemize}
\item
The packages
\href{http://ctan.org/pkg/docmute}{\textsf{docmute}},
\href{http://ctan.org/pkg/includex}{\textsf{includex}} and
\href{http://ctan.org/pkg/standalone}{\textsf{standalone}}
provide commands to include only the document body of
a child file thus allowing both files to be compiled individually.
\item
The packages \href{http://ctan.org/pkg/subdocs}{\textsf{subdocs}}
and \href{http://ctan.org/pkg/subfiles}{\textsf{subfiles}}
provide structures in which the main and child documents can be
encapsulated and allowing them to be compiled individually.
The inclusion mechanism is different from the conventional |\include|.
\item
The package \href{http://ctan.org/pkg/combine}{\textsf{combine}}
is an elaborate solution to combine several documents into one.
\end{itemize}
%
See also the CTAN topic \href{http://ctan.org/topic/subdocs}{\textsf{subdocs}}
for further related packages.
The present package differs from the above solutions in that
a document structure constructed with the conventional |\include| mechanism
just needs two extra commands at the top of every file
such that all constituent files can be compiled individually.

%%%%%%%%%%%%%%%%%%%%%%%%%%%%%%%%%%%%%%%%%%%%%%%%%%%%%%%%%%%%%%%%%%%%%%%%%%%%%%%%
%\subsection{Feature Suggestions}
%
%The following is a list of features which may be useful for future
%versions of this package:
%%
%\begin{itemize}
%\item
%\ldots
%\end{itemize}

%%%%%%%%%%%%%%%%%%%%%%%%%%%%%%%%%%%%%%%%%%%%%%%%%%%%%%%%%%%%%%%%%%%%%%%%%%%%%%%%
\subsection{Revision History}

%%%%%%%%%%%%%%%%%%%%%%%%%%%%%%%%%%%%%%%%
\paragraph{v2.0:} 2018/12/30

\begin{itemize}
\item
immediate forward processing
\item
added |\childdocby| mechanism
\item
manual restructured
\end{itemize}

%%%%%%%%%%%%%%%%%%%%%%%%%%%%%%%%%%%%%%%%
\paragraph{v1.6:} 2018/01/17

\begin{itemize}
\item
application for development of include files
\item
corrections to manual
\end{itemize}

%%%%%%%%%%%%%%%%%%%%%%%%%%%%%%%%%%%%%%%%
\paragraph{v1.5:} 2017/05/21

\begin{itemize}
\item
more complete structuring introduced
\item
|\childdocof| introduced
\item
|\childdoc| renamed to |\childdocmain|
\item
|\childredirect| renamed to |\childdocforward| and |\childdocforwardprefix|
and functionality expanded
\end{itemize}

%%%%%%%%%%%%%%%%%%%%%%%%%%%%%%%%%%%%%%%%
\paragraph{v1.0:} 2017/04/27

\begin{itemize}
\item
manual and install package
\item
first version published on CTAN
\end{itemize}

%%%%%%%%%%%%%%%%%%%%%%%%%%%%%%%%%%%%%%%%
\paragraph{v0.6:} 2017/04/26

\begin{itemize}
\item
redirection mechanism added
\end{itemize}

%%%%%%%%%%%%%%%%%%%%%%%%%%%%%%%%%%%%%%%%
\paragraph{v0.5:} 2017/04/26

\begin{itemize}
\item
functionality in definition file
\end{itemize}


%%%%%%%%%%%%%%%%%%%%%%%%%%%%%%%%%%%%%%%%%%%%%%%%%%%%%%%%%%%%%%%%%%%%%%%%%%%%%%%%
%%%%%%%%%%%%%%%%%%%%%%%%%%%%%%%%%%%%%%%%%%%%%%%%%%%%%%%%%%%%%%%%%%%%%%%%%%%%%%%%
%%%%%%%%%%%%%%%%%%%%%%%%%%%%%%%%%%%%%%%%%%%%%%%%%%%%%%%%%%%%%%%%%%%%%%%%%%%%%%%%
\appendix

\settowidth\MacroIndent{\rmfamily\scriptsize 000\ }

 \DocInput{childdoc.dtx}

\end{document}
%</driver>
% \fi
%
% %%%%%%%%%%%%%%%%%%%%%%%%%%%%%%%%%%%%%%%%%%%%%%%%%%%%%%%%%%%%%%%%%%%%%%%%%%%%%%
% %%%%%%%%%%%%%%%%%%%%%%%%%%%%%%%%%%%%%%%%%%%%%%%%%%%%%%%%%%%%%%%%%%%%%%%%%%%%%%
% \section{Sample}
%\iffalse
%<*samplemain>
%\fi
%
% The following presents a sample document
% with two chapters, two parts, a title page,
% a compile flag as well as three forwarding files to set the flag.
% It consists of eight |.tex| files:
% \begin{center}
% \begin{tabular}{ll}
% |cdocsamp.tex|&main file\\
% |cdocsch1.tex|&include file for chapter 1\\
% |cdocsch2.tex|&include file for chapter 2\\
% |cdocspt3.tex|&include file for part 3\\
% |cdocspt4.tex|&include file for part 4\\
% |cdocsdrf.tex|&forwarding file for main file in draft mode\\
% |cdocsfi1.tex|&forwarding file for final version of chapter 1\\
% |cdocsfi2.tex|&forwarding file for final version of chapter 2\\
% \end{tabular}
% \end{center}
% Each of the eight files can be compiled directly by the \LaTeX{} compiler.
%
% %%%%%%%%%%%%%%%%%%%%%%%%%%%%%%%%%%%%%%
% \paragraph{Main File.}
%
% The main file is called |cdocsamp.tex|.
%
% Load the \textsf{childdoc} definitions and
% declare the filename for the main document:
%    \begin{macrocode}
\input{childdoc.def}
\childdocmain{}
%    \end{macrocode}

% Optional override for |\version| flag:
%    \begin{macrocode}
%%\ifchilddoc\else\providecommand{\version}{draft}\fi
%    \end{macrocode}

% Define the default values for the |\version| flag
% (|final| for the main file and |draft| for childs):
%    \begin{macrocode}
\ifchilddoc
\providecommand{\version}{draft}
\else
\providecommand{\version}{final}
\fi
%    \end{macrocode}

% Load the standard document class:
%    \begin{macrocode}
\documentclass[12pt]{article}
%    \end{macrocode}

% Start the document body:
%    \begin{macrocode}
\begin{document}
%    \end{macrocode}

% Declare a title page.
% Print title, part of document being processed and version flag:
%    \begin{macrocode}
\addtocounter{page}{-1}
\begin{center}
{\LARGE\bfseries{}childdoc example\par}
\vspace{1cm}
\ifchilddoc
\ifchilddocmanual part\else chapter\fi:
`\childdocname' of `\childdocjob'\par
\else
main document: `\childdocjob'\par
\fi
version: \version\par
\end{center}
\newpage
%    \end{macrocode}

% Manually include selected file,
% otherwise process as usual:
%    \begin{macrocode}
\ifchilddocmanual
\section*{part `\childdocname'}
\input{\childdocname}
\else
%    \end{macrocode}

% Include the two chapters:
%    \begin{macrocode}
\include{cdocsch1}
\include{cdocsch2}
%    \end{macrocode}

% Include the two parts unless only chapters should be displayed:
%    \begin{macrocode}
\ifchilddoc\else
\section{part three}
\input{cdocspt3}
\section{part four}
\input{cdocspt4}
\fi
%    \end{macrocode}

% Process as usual until here:
%    \begin{macrocode}
\fi
%    \end{macrocode}

% End of document body:
%    \begin{macrocode}
\end{document}
%    \end{macrocode}
%\iffalse
%</samplemain>
%\fi
%
% %%%%%%%%%%%%%%%%%%%%%%%%%%%%%%%%%%%%%%
% \paragraph{Chapter Include Files.}
%
% The include files are called |cdocsch1.tex| and |cdocsch2.tex|.
%
%\iffalse
%<*samplechap1|samplechap2>
%\fi

% Optional override for |\version| flag:
%    \begin{macrocode}
%%\providecommand{\version}{final}
%    \end{macrocode}

% Include the main document:
%    \begin{macrocode}
\input{childdoc.def}
\childdocof{cdocsamp}
%    \end{macrocode}

%\iffalse
%</samplechap1|samplechap2>
%\fi
%
%\iffalse
%<*samplechap1>
%\fi
% Some text for chapter 1:
%    \begin{macrocode}
\section{one}
some text in chapter one
%    \end{macrocode}

%\iffalse
%</samplechap1>
%\fi
% Some text for chapter 2:
%\iffalse
%<*samplechap2>
%\fi
%    \begin{macrocode}
\section{two}
more text in chapter two
%    \end{macrocode}

%\iffalse
%</samplechap2>
%\fi
%
% %%%%%%%%%%%%%%%%%%%%%%%%%%%%%%%%%%%%%%
% \paragraph{Part Include Files.}
%
% The include files are called |cdocspt3.tex| and |cdocspt4.tex|.
%
%\iffalse
%<*samplepart3|samplepart4>
%\fi

% Optional override for |\version| flag:
%    \begin{macrocode}
%%\providecommand{\version}{final}
%    \end{macrocode}

% Include the main document:
%    \begin{macrocode}
\input{childdoc.def}
\childdocby{cdocsamp}
%    \end{macrocode}

%\iffalse
%</samplepart3|samplepart4>
%\fi
%
%\iffalse
%<*samplepart3>
%\fi
% Some text for part 3:
%    \begin{macrocode}
some text in part three
%    \end{macrocode}

%\iffalse
%</samplepart3>
%\fi
% Some text for part 4:
%\iffalse
%<*samplepart4>
%\fi
%    \begin{macrocode}
more text in part four
%    \end{macrocode}

%\iffalse
%</samplepart4>
%\fi
%
% %%%%%%%%%%%%%%%%%%%%%%%%%%%%%%%%%%%%%%
% \paragraph{Forwarding for a Complete Draft.}
%
% The following forwarding file |cdocsdrf.tex|
% compiles the main document in draft mode:
%\iffalse
%<*sampledraft>
%\fi
%    \begin{macrocode}
\def\version{draft}
\input{childdoc.def}
\childdocforward{cdocsamp}
%    \end{macrocode}

%\iffalse
%</sampledraft>
%\fi
%
% %%%%%%%%%%%%%%%%%%%%%%%%%%%%%%%%%%%%%%
% \paragraph{Forwarding for Final Version of the Chapters.}
%
% The following forwarding files |cdocsfn1.tex| and |cdocsfn2.tex|
% (with identical content)
% compile the final versions of the child documents
% |cdocsch1.tex| and |cdocsch2.tex|, respectively:
%\iffalse
%<*samplefinal>
%\fi
%    \begin{macrocode}
\def\version{final}
\input{childdoc.def}
\childdocforwardprefix[cdocsamp]{cdocsfn}{cdocsch}
%    \end{macrocode}

%\iffalse
%</samplefinal>
%\fi
%
% %%%%%%%%%%%%%%%%%%%%%%%%%%%%%%%%%%%%%%
% \paragraph{Command Line Processing.}
%
% The following three command lines generate the output files
% |cdocscld|, |cdocscl1| and |cdocscl2|
% which should be identical to
% |cdocsdrf|, |cdocsch1| and |cdocsfn2|, respectively:
% \begin{center}
% \begin{tabular}{l}
% |latex -jobname cdocscld \|\\
% |  "\def\version{draft}\input{childdoc.def}\childdocforward{cdocsamp}"|\\
% |latex -jobname cdocscl1 \|\\
% |  "\input{childdoc.def}\childdocforward[cdocsamp]{cdocsch1}"|\\
% |latex -jobname cdocscl2 \|\\
% |  "\def\version{final}\input{childdoc.def}\childdocforward{cdocsch2}"|
% \end{tabular}
% \end{center}
% Note that the trailing backslash on each first line
% merely continues the input to the second line
% (for convenient cut ant paste).
% Furthermore, the command |latex| can be replaced by any
% of its alternative versions such as |pdflatex|.
%
% %%%%%%%%%%%%%%%%%%%%%%%%%%%%%%%%%%%%%%%%%%%%%%%%%%%%%%%%%%%%%%%%%%%%%%%%%%%%%%
% %%%%%%%%%%%%%%%%%%%%%%%%%%%%%%%%%%%%%%%%%%%%%%%%%%%%%%%%%%%%%%%%%%%%%%%%%%%%%%
% \section{Implementation}
%\iffalse
%<*package>
%\fi
%
% This section describes the definitions file |childdoc.def|.

% The definitions cannot be loaded using |\usepackage| or |\RequirePackage|
% which has a mechanism to prevent loading a style file more than once.
% When loading the definitions by means of |\input|
% multiple instances have to be prevented manually:
%\iffalse
%This code needs to be before the `\ProvidesFile' directive
%which is defined at the beginning of this file.
%Therefore it is also placed there and commented out here.
%</package>
%<*discard>
%\fi
%    \begin{macrocode}
\ifdefined\childdocmain\endinput\fi
%    \end{macrocode}
%\iffalse
%</discard>
%<*package>
%\fi
%
% \macro{\ifchilddoc}
% \macro{\ifchilddocmanual}
% The conditional |\ifchilddoc| tells whether a
% child (true) or main (false) document is being compiled.
% The conditional |\ifchilddocmanual| tells whether
% the |\includeonly| mechanism is used (false) or
% the selection of child files must be performed manually (true).
% The definitions initialise to false:
%    \begin{macrocode}
\newif\ifchilddoc
\newif\ifchilddocmanual
%    \end{macrocode}

% \macro{\childdocname}
% \macro{\childdocjob}
% The macro |\childdocname| stores the name of the main document
% to be compiled. The macro |\childdocjob| stores the name of
% the document on which the \LaTeX{} compiler was originally invoked.
% The content of |\jobname| cannot be compared
% to filenames specified in the source due to different catcodes.
% The following code rescans |\jobname|, stores the result
% in |\childdocname| and saves a copy in |\childdocjob|:
%    \begin{macrocode}
\edef\childdocname{\scantokens\expandafter{\jobname\noexpand}}
\let\childdocjob\childdocname
%    \end{macrocode}

% \macro{\childdocdisable}
% The macro |\childdocdisable| prevents the main file
% from being processed more than once.
% At this stage, the main document command |\childdocmain|
% is assumed to be called once again where it should do nothing.
% Any subsequent call to it should prevent
% a secondary processing of the main document
% It overwrites the forwarding commands
% |\childdocof| and |\childdocforward|
% with empty macros to prevent further inclusions of the main document:
%    \begin{macrocode}
\newcommand{\childdocdisable}
{
  \renewcommand{\childdocmain}[1]{\renewcommand{\childdocmain}[1]{\endinput}}
  \renewcommand{\childdocof}[1]{}
  \renewcommand{\childdocby}[2][]{}
  \renewcommand{\childdocforward}[2][]{}
  \renewcommand{\childdocdisable}{}
}
%    \end{macrocode}

% \macro{\childdocmain}
% The macro |\childdocmain| is to be called at the top of the main file
% with nothing or the main filename (without extension) as argument.
% First, it breaks loops.
% If the argument is not empty and does not match |\childdocname|
% (which is set by the first inclusion of |childdoc.def|),
% |\ifchilddoc| is set to true, |\includeonly| is applied to the child file
% and |\jobname| is set to the main file
% (for proper handling of |.aux| files):
%    \begin{macrocode}
\newcommand{\childdocmain}[1]
{
  \childdocdisable\childdocmain{}
  \if?#1?\else
    \begingroup
      \def\childdoctmp{#1}
      \ifx\childdoctmp\childdocname
        \def\childdoctmp{}
      \else
        \def\childdoctmp
        {
          \childdoctrue
          \includeonly{\childdocname}
          \def\childdocjob{#1}
          \def\jobname{#1}
        }
      \fi
      \expandafter
    \endgroup
    \childdoctmp
  \fi
}
%    \end{macrocode}

% \macro{\childdocof}
% The command |\childdocof| redirects
% compilation to the main file |#1|.
%    \begin{macrocode}
\newcommand{\childdocof}[1]
{
  \childdocdisable
  \childdoctrue
  \includeonly{\childdocname}
  \def\jobname{#1}
  \def\childdocjob{#1}
  \input{#1}
}
%    \end{macrocode}

% \macro{\childdocby}
% The command |\childdocby| ....
%    \begin{macrocode}
\newcommand{\childdocby}[2][]
{
  \childdocdisable
  \childdoctrue
  \childdocmanualtrue
  \if?#1?\else
    \def\jobname{#2}
  \fi
  \def\childdocjob{#2}
  \input{#2}
  \endinput
}
%    \end{macrocode}

% \macro{\childdocforward}
% The command |\childdocforward| redirects
% compilation to the main file or
% (if the optional argument is given) a child file.
% Parameters are set as if the main file
% or a child file starting with |\childdocof| was compiled.
% Then compilation is handed over to the main file:
%    \begin{macrocode}
\newcommand{\childdocforward}[2][]
{
  \begingroup
    \if?#1?
      \def\childdoctmp
      {
        \def\childdocname{#2}
        \def\childdocjob{#2}
        \def\jobname{#2}
        \input{#2}
        \endinput
      }
    \else
      \def\childdoctmp
      {
        \childdocdisable
        \def\childdocname{#2}
        \childdoctrue
        \includeonly{#2}
        \def\childdocjob{#1}
        \def\jobname{#1}
        \input{#1}
        \endinput
      }
    \fi
    \expandafter
  \endgroup
  \childdoctmp
}
%    \end{macrocode}

% \macro{\childdocforwardprefix}
% The command |\childdocforwardprefix| redirects
% compilation to the main or a child file by means of a pattern.
% The prefix |#1| in the current filename is replaced by |#2|
% and the suffix of the current filename is kept
% (it is assumed that the filename does not contain the substring `|~~~|'
% which is used as a delimiter).
% Compilation is handed over to the new file by |\childdocforward|:
%    \begin{macrocode}
\newcommand{\childdocforwardprefix}[3][]
{
  \begingroup
    \def\childdocextract #2##1~~~{\def\childdoctmp{\childdocforward[#1]{#3##1}}}
    \expandafter\childdocextract\childdocname~~~
    \expandafter
  \endgroup
  \childdoctmp
}
%    \end{macrocode}

% \macro{\childdoc}
% The deprecated macro |\childdoc| is a legacy version of |\childdocmain|:
%    \begin{macrocode}
\newcommand{\childdoc}{\childdocmain}
%    \end{macrocode}

% \macro{\childdocredirect}
% The deprecated macro |\childdocredirect| is a legacy version
% of |\childdocforward| and |\childdocforwardprefix|:
%    \begin{macrocode}
\newcommand{\childdocredirect}[2][]
{
  \begingroup
    \if?#1?
      \def\childdoctmp{\childdocforward{#2}}
    \else
      \def\childdoctmp{\childdocforwardprefix{#1}{#2}}
    \fi
    \expandafter
  \endgroup
  \childdoctmp
}
%    \end{macrocode}

%\iffalse
%</package>
%\fi
%
\endinput
|\\
|\childdocby{|\textit{main}|}|\\
\end{tabular}
\end{center}
%
The directive |\childdocby| is similar to |\childdocof|
described in \secref{sec:include},
but the subsequent selection of content must be done manually.
To that end, both |\ifchilddoc| and |\ifchilddocmanual|
will be true upon processing of a part,
and the name of the part is stored in |\childdocname|.
Note that |\jobname| will be set to the filename of the current part
so that each part receives an individual |.aux| file
that does not interfere with the |.aux| file(s) of the main document.
This behaviour can be altered by the alternative form
|\childdocby[*]{|\textit{main}|}| (with a non-empty optional argument)
which uses the |.aux| file of the main document
by setting |\jobname| to \textit{main}.

%%%%%%%%%%%%%%%%%%%%%%%%%%%%%%%%%%%%%%%%%%%%%%%%%%%%%%%%%%%%%%%%%%%%%%%%%%%%%%%%
\subsection{Driver Development}
\label{sec:driver}

The \textsf{childdoc} mechanism can also be use for the development
of definition files such as \LaTeX{} styles or classes.
This case differs from the above setup with multiple parts
included by |\include| in that no |\includeonly| should be invoked.
This can be achieved by starting the include file
(before |\ProvidesPackage|) with:
%
\begin{center}
\begin{tabular}{l}
|% \iffalse
%
% childdoc.dtx Copyright (C) 2017-2018 Niklas Beisert
%
% This work may be distributed and/or modified under the
% conditions of the LaTeX Project Public License, either version 1.3
% of this license or (at your option) any later version.
% The latest version of this license is in
%   http://www.latex-project.org/lppl.txt
% and version 1.3 or later is part of all distributions of LaTeX
% version 2005/12/01 or later.
%
% This work has the LPPL maintenance status `maintained'.
%
% The Current Maintainer of this work is Niklas Beisert.
%
% This work consists of the files childdoc.dtx and childdoc.ins
% and the derived files childdoc.def and cdocsamp.tex with
% cdocsch1.tex, cdocsch2.tex, cdocsdrf.tex, cdocsfn1.tex, cdocsfn2.tex.
%
%<package>\ifdefined\childdocmain\endinput\fi
%<package>\ProvidesFile{childdoc.def}[2018/12/30 v2.0 child document driver]
%<samplemain>\ProvidesFile{cdocsamp.tex}[2018/12/30 v2.0 sample for childdoc]
%<*driver>
%\ProvidesFile{childdoc.drv}[2018/12/30 v2.0 childdoc reference manual file]
\PassOptionsToClass{10pt,a4paper}{article}
\documentclass{ltxdoc}

\usepackage[margin=35mm]{geometry}
\usepackage{hyperref}
\usepackage{hyperxmp}
\usepackage[usenames]{color}

\hypersetup{colorlinks=true}
\hypersetup{pdfstartview=FitH}
\hypersetup{pdfpagemode=UseNone}
\hypersetup{pdfsource={}}
\hypersetup{pdflang={en-UK}}
\hypersetup{pdfcopyright={Copyright 2017-2018 Niklas Beisert.
  This work may be distributed and/or modified under the
  conditions of the LaTeX Project Public License, either version 1.3
  of this license or (at your option) any later version.}}
\hypersetup{pdflicenseurl={http://www.latex-project.org/lppl.txt}}
\hypersetup{pdfcontactaddress={ETH Zurich, ITP, HIT K,
  Wolfgang-Pauli-Strasse 27}}
\hypersetup{pdfcontactpostcode={8093}}
\hypersetup{pdfcontactcity={Zurich}}
\hypersetup{pdfcontactcountry={Switzerland}}
\hypersetup{pdfcontactemail={nbeisert@itp.phys.ethz.ch}}
\hypersetup{pdfcontacturl={http://people.phys.ethz.ch/\xmptilde nbeisert/}}

\newcommand{\secref}[1]{\hyperref[#1]{section \ref*{#1}}}

\parskip1ex
\parindent0pt
\let\olditemize\itemize
\def\itemize{\olditemize\parskip0pt}

\begin{document}

\title{The \textsf{childdoc} Package}
\hypersetup{pdftitle={The childdoc Package}}
\author{Niklas Beisert\\[2ex]
  Institut f\"ur Theoretische Physik\\
  Eidgen\"ossische Technische Hochschule Z\"urich\\
  Wolfgang-Pauli-Strasse 27, 8093 Z\"urich, Switzerland\\[1ex]
  \href{mailto:nbeisert@itp.phys.ethz.ch}
  {\texttt{nbeisert@itp.phys.ethz.ch}}}
\hypersetup{pdfauthor={Niklas Beisert}}
\hypersetup{pdfsubject={Manual for the LaTeX2e Package childdoc}}
\date{30 December 2018, \textsf{v2.0}}
\maketitle

\begin{abstract}\noindent
\textsf{childdoc} is a \LaTeXe{} package
that enables the direct compilation
of document sections included by |\include|
to individual files.
\end{abstract}

\begingroup
\parskip0ex
\tableofcontents
\endgroup

%%%%%%%%%%%%%%%%%%%%%%%%%%%%%%%%%%%%%%%%%%%%%%%%%%%%%%%%%%%%%%%%%%%%%%%%%%%%%%%%
%%%%%%%%%%%%%%%%%%%%%%%%%%%%%%%%%%%%%%%%%%%%%%%%%%%%%%%%%%%%%%%%%%%%%%%%%%%%%%%%
\section{Introduction}

\LaTeX{} provides a mechanism to structure a large document (such as a book)
into a main file and several child files (containing the chapters)
using the |\include| command.
This mechanism is beneficial for documents
which span hundreds of pages in order to
make the source file(s) more manageable.
Moreover, compilation can be restricted to
selected child files by means of the |\includeonly| command.
The latter feature can be used to reduce the compilation time while editing
(this was significantly more useful in the earlier days of \LaTeX{})
or to generate a smaller document which is easier to navigate.
Another application of |\includeonly| is to generate
documents consisting of selected parts of the complete document.

However, there are a few drawbacks of the plain |\include| mechanism:
\begin{itemize}
\item
The child files cannot be compiled on their own,
they can only be compiled via the main file.
A naive editing environment
(such as a text editor with an option
to have the current file processed by \LaTeX)
may require one to switch to the main file before compiling;
attempting to compile the child file produces errors.
\item
The main file must be modified (each time)
to adjust the |\includeonly| command
to the present needs. This easily leaves the main file in a messy state.
\item
The generated document will always carry the filename
of the main document. This is inconvenient if
several child files are to be compiled and
to be kept for distribution.
\end{itemize}

The present package provides a simple interface
to make child files individually compilable by \LaTeX{}.
Compiling a child file then has the same effect as compiling
the main file with an |\includeonly| command
to select the appropriate child.
Moreover the generated document will carry the name of the child
rather than the main file.
This resolves all three above issues.

This feature is meant to make the editing of books,
thesis documents and lecture notes somewhat more convenient.
However, the package can also be used efficiently for
composing a series of documents (such as exercise sheets)
which are typically distributed individually.
It then assists the author in generating the individual documents
(potentially in different versions)
as well as a document containing the collected series.
Another application is in developing style files
or other kinds of included material
where compilation of the style file could redirect
to a sample or test file.

%%%%%%%%%%%%%%%%%%%%%%%%%%%%%%%%%%%%%%%%%%%%%%%%%%%%%%%%%%%%%%%%%%%%%%%%%%%%%%%%
%%%%%%%%%%%%%%%%%%%%%%%%%%%%%%%%%%%%%%%%%%%%%%%%%%%%%%%%%%%%%%%%%%%%%%%%%%%%%%%%
\section{Usage}

First of all, the package \textsf{childdoc} is \emph{not} a standard
\LaTeXe{} |.sty| style file! Therefore it needs to be invoked in
a non-standard way.

%%%%%%%%%%%%%%%%%%%%%%%%%%%%%%%%%%%%%%%%%%%%%%%%%%%%%%%%%%%%%%%%%%%%%%%%%%%%%%%%
\subsection{Included Files}
\label{sec:include}

%%%%%%%%%%%%%%%%%%%%%%%%%%%%%%%%%%%%%%%%
\DescribeMacro{\childdocmain}
To use the package, add the commands
\begin{center}
\begin{tabular}{l}
|\input{childdoc.def}|\\
|\childdocmain{}|\\
\end{tabular}
\end{center}
at the very top of the main \LaTeX{} file,
in particular \emph{before} the |\documentclass| statement!
The argument of |\childdocmain| should be left empty
(but it must be present).

%%%%%%%%%%%%%%%%%%%%%%%%%%%%%%%%%%%%%%%%
\DescribeMacro{\childdocof}
Furthermore, add the commands
\begin{center}
\begin{tabular}{l}
|\input{childdoc.def}|\\
|\childdocof{|\textit{main}|}|\\
\end{tabular}
\end{center}
at the top of every child file \textit{child}
which is included by |\include{|\textit{child}|}|
from within the main file
(or at least for those files to be compiled individually).
The argument \textit{main} must be the filename of the main file.

There are a couple of
considerations in setting up the main and child documents:

%%%%%%%%%%%%%%%%%%%%%%%%%%%%%%%%%%%%%%%%
\paragraph{Restrictions.}

Please note the following restrictions:
\begin{itemize}
\item
|\childdocmain| must be called with one argument \textit{main}
to ensure compatibility with earlier version of the package.
It must either be empty (|\childdocmain{}|)
or precisely match the filename of the main file in which it is specified.
See \secref{sec:detection} for further information.
\item
The filename \textit{main} must be specified without the |.tex| extension.
\item
The filename \textit{main} is case sensitive
(even in case-insensitive file systems)
due to internal string comparison.
\item
The argument \textit{main} should be fully expanded, it cannot be a macro.
\item
Subdirectories and special characters should be avoided in filenames.
\item
The command |\childdocmain{|\textit{main}|}| must be followed by a whitespace.
It should not be followed immediately by another command
or by a comment mark `|%|'.
This is because the \TeX{} parser reads the token immediately following
the argument of |\childdocmain| and puts it
at the beginning of every child section;
however, a white\-space is ignored.
\end{itemize}

%%%%%%%%%%%%%%%%%%%%%%%%%%%%%%%%%%%%%%%%
\paragraph{Content of Main File.}

It is advisable to place all content in the child files included by |\include|.
Any output contained in the main file will appear in all child documents
unless suppressed manually;
it cannot be suppressed automatically by the |\includeonly| directive
and thus should normally be avoided.
A method to include some content in the main file
by means of conditional processing is described in \secref{sec:conditional}.

%%%%%%%%%%%%%%%%%%%%%%%%%%%%%%%%%%%%%%%%
\paragraph{Page Numbering.}

When only a part of the document is compiled,
the appropriate numbering of pages
(as well as other status parameters)
is determined from the |.aux| files.
The latter contain information from previous passes.
However this information needs to propagate through
all intermediate child documents.
Therefore the page numbering in child documents may well
be inconsistent until the complete document is compiled at least once.

A useful (if unconventional) way to always ensure a consistent
page numbering is to restart the numbering in each child document
and denote the pages by `\textit{child}|.|\textit{page}'
where \textit{child} represents the chapter/section number of the child file.
This can be achieved by the command
|\numberwithin{page}{|\textit{child}|}|
of the \textsf{amsmath} package
where \textit{child} can be |chapter| or |section|
depending on the chosen structuring.
Alternatively, one can modify the macro |\thepage| appropriately
and reset the counter |page| at the start of each child file.

%%%%%%%%%%%%%%%%%%%%%%%%%%%%%%%%%%%%%%%%%%%%%%%%%%%%%%%%%%%%%%%%%%%%%%%%%%%%%%%%
\subsection{Conditional Processing}
\label{sec:conditional}

The package provides a mechanism to compile different versions
of a document. To customise the versions further some conditional processing
can come in handy to distinguish which version is being compiled.
The package provides two macros to describe the compilation context:

%%%%%%%%%%%%%%%%%%%%%%%%%%%%%%%%%%%%%%%%
\DescribeMacro{\ifchilddoc}
The conditional |\ifchilddoc| distinguishes between the compilation of
child documents and the main document:
%
\begin{center}
|\ifchilddoc |\textit{child-code}| |[|\||else |\textit{main-code}]| \||fi|
\end{center}

%%%%%%%%%%%%%%%%%%%%%%%%%%%%%%%%%%%%%%%%
\DescribeMacro{\childdocname}
\DescribeMacro{\childdocjob}
The macro |\childdocname| contains the filename (without extension)
of the main or child file being processed.
Note that |\childdocjob| will always contain the name of the main file.

%%%%%%%%%%%%%%%%%%%%%%%%%%%%%%%%%%%%%%%%
\paragraph{Title Page.}

Conditional processing can be used to include a title or banner page
in the main document when proper precautions are taken.
Importantly, the code in the main file should ensure that the page counter
(as well as other status parameters which are stored in the |.aux| files)
takes the same value after the conditional processing.
Otherwise the page numbers may take divergent values
depending on which part is compiled.

For example, a title page could be declared by:
%
\begin{center}
\begin{tabular}{l}
|\ifchilddoc\||else|\\
|\addtocounter{page}{-1}|\\
\textit{code for title page}\\
|\newpage|\\
|\||fi|
\end{tabular}
\end{center}
%
A banner page for the child documents can be generated by:
%
\begin{center}
\begin{tabular}{l}
|\ifchilddoc|\\
|\addtocounter{page}{-1}|\\
\textit{code for banner page}\\
|\newpage|\\
|\||fi|
\end{tabular}
\end{center}
%
Here one could write a message such as:
\begin{center}
|This is the part \childdocname{} of \childdocjob{}.|
\end{center}

%%%%%%%%%%%%%%%%%%%%%%%%%%%%%%%%%%%%%%%%%%%%%%%%%%%%%%%%%%%%%%%%%%%%%%%%%%%%%%%%
\subsection{Flags}
\label{sec:flags}

The package makes it easy to generate different versions
of the main or child documents.
To this end compilation flags can be defined
and assigned different default values.
They will be particularly useful in conjunction
with the forwarding mechanism described in \secref{sec:forward}.

For example, it may be useful to have a flag |\version|
which can be set to |draft| or |final|.
The document source will contain some conditional code
depending on the value of |\version|.
Suppose further, the flag should default to |final| for the main file
and to |draft| for child files
which is a natural assignment for editing the document.
This is achieved by placing the following code
in the preamble of the main document
(below the |\childdocmain| directive):
%
\begin{center}
\begin{tabular}{l}
|\ifchilddoc|\\
|\providecommand{\version}{draft}|\\
|\||else|\\
|\providecommand{\version}{final}|\\
|\||fi|
\end{tabular}
\end{center}
%
The definition by |\providecommand| makes sure
that previous definitions are not overwritten.
Further statements |\providecommand{\version}{...}|
can thus be added before the above code to override it.

For the main file, one might add a line
(between |\childdocmain| and the above block)
%
\begin{center}
|%\ifchilddoc\||else\providecommand{\version}{draft}\||fi|
\end{center}
%
which can be uncommented to produce a draft version.
Likewise one can add a line to the very top of a child file
(above the |\childdocof{|\textit{main}|}| directive)
%
\begin{center}
|%\providecommand{\version}{final}|
\end{center}
%
which can be uncommented to produce the final version of this child document.

%%%%%%%%%%%%%%%%%%%%%%%%%%%%%%%%%%%%%%%%%%%%%%%%%%%%%%%%%%%%%%%%%%%%%%%%%%%%%%%%
\subsection{Forwarding}
\label{sec:forward}

Different versions of the main or child documents
using compilation flags as described in \secref{sec:flags}
can be (permanently) stored in different files
for convenient compilation, viewing and distribution.
To this end, the package defines a command
to pass on compilation to a different file:

%%%%%%%%%%%%%%%%%%%%%%%%%%%%%%%%%%%%%%%%
\DescribeMacro{\childdocforward}
The command |\childdocforward| redirects processing to
another source file:
%
\begin{center}
\begin{tabular}{l}
|\input{childdoc.def}|\\
|\childdocforward[|\textit{main}|]{|\textit{dest}|}|\\
\end{tabular}
\end{center}
%
The argument \textit{dest} is the destination file
(without extension).
It should be the main file or one of the child files.
Note that further \textsf{childdoc} directives
such as |\childdocof| and |\childdocforward|
in the indicated file will be processed in this form.
The optional argument \textit{main}
passes on directly to the main file \textit{main}
while pretending to compile the child \textit{dest}.
This form behaves as if \textit{dest}
issues |\childdocof{|\textit{main}|}| right away,
and no further \textsf{childdoc} directives will be processed.

%%%%%%%%%%%%%%%%%%%%%%%%%%%%%%%%%%%%%%%%
\DescribeMacro{\...prefix}
In the alternative form |\childdocforwardprefix|,
%
\begin{center}
\begin{tabular}{l}
|\input{childdoc.def}|\\
|\childdocforwardprefix[|\textit{main}|]{|\textit{prefix}|}{|\textit{dest}|}|
\end{tabular}
\end{center}
%
the destination file is determined by a pattern
depending on the current file:
To make this work, the current file must be called
`{\textit{prefix}\hspace{0.2em}\textit{suffix}}'
with \textit{prefix} matching precisely the argument.
Processing is then passed on to the file
`{\textit{dest}\hspace{0.2em}\textit{suffix}}'.
Surely, the same effect is achieved by
directly specifying the
argument `{\textit{dest}\hspace{0.2em}\textit{suffix}}'
in the first form.
However, that requires to set up a different file
for each child. With the alternative form of the command
all these files can have exactly the same content
which simplifies setting them up and maintaining them.

For example, the following file |draft.tex|
with a compilation flag |\version| as described in \secref{sec:flags}
compiles the main document as a draft:
%
\begin{center}
\begin{tabular}{l}
|\def\version{draft}|\\
|\input{childdoc.def}|\\
|\childdocforward{|\textit{main}|}|
\end{tabular}
\end{center}
%
Likewise, the following files |final|\textit{nn}|.tex|
compile the final version of the child document
|child|\textit{nn}|.tex|:
%
\begin{center}
\begin{tabular}{l}
|\def\version{final}|\\
|\input{childdoc.def}|\\
|\childdocforwardprefix{final}{child}|
\end{tabular}
\end{center}
%

Note that when several versions of a main file and/or of each child file
are to be generated, it may be convenient to set up a |Makefile| or
shell script to automatise the process.

%%%%%%%%%%%%%%%%%%%%%%%%%%%%%%%%%%%%%%%%%%%%%%%%%%%%%%%%%%%%%%%%%%%%%%%%%%%%%%%%
\subsection{Command Line Processing}
\label{sec:commandline}

The effect of redirection files can also be achieved by invoking
the \LaTeX{} compiler with a more elaborate command line.
Most conveniently this should be done as part
of a shell script or a |Makefile|.

When using \textsf{childdoc} in the main file, the following
command lines effectively perform a redirection
(note that depending on the shell being used,
backslashes may have to be doubled: `|\|' $\to$ `|\\|'):
%
\begin{center}
|... -jobname "|\textit{target}|" |\\|"|[\textit{flags}]%
|\input{childdoc.def}\childdocforward[|\textit{main}|]{|\textit{dest}|}"|
\end{center}
%
Here \textit{target} is the name of the output file,
\textit{main} is the name of the main file
and \textit{dest} is the name of the main or child file to be processed
(all filenames without extensions).
The optional argument \textit{main} can be omitted
if \textit{main} matches \textit{dest}.
Optionally, compilation \textit{flags} can be defined via |\def| commands.
This command line makes the \TeX{} engine believe
it is compiling the file \textit{target}
whose content is specified as the latter parameter.
The provided code then forwards the processing to
\textit{main} or \textit{dest} as described in \secref{sec:forward}.

%%%%%%%%%%%%%%%%%%%%%%%%%%%%%%%%%%%%%%%%%%%%%%%%%%%%%%%%%%%%%%%%%%%%%%%%%%%%%%%%
\subsection{Include by Input}
\label{sec:input}

Including child documents by |\include| has some restrictions by design.
Most notably, the content of a child document always occupies
its own set of pages; pages cannot be shared between child documents.
Usually, this behaviour makes perfect sense
because each child document contain an essential part of the document.
However, in some situations it may be desirable to compose
a document from a collection of parts
without having mandatory page breaks between then.
For this case, the package
provides a mechanism to include parts
by |\input| which can also be processed individually.
However, by construction this mechanism
requires manual handling of the content to be output.

%%%%%%%%%%%%%%%%%%%%%%%%%%%%%%%%%%%%%%%%
\DescribeMacro{\ifchilddocmanual}
The main file should be prepared as usual, see \secref{sec:include}.
However, the document body must make a distinction
between processing of an individual part and of the main document, e.g.:
%
\begin{center}
\begin{tabular}{l}
|\ifchilddocmanual|\\
|\input{\childdocname}|\\
|\||else|\\
\textit{document body with }|\input{|\textit{part}|}|\\
|\||fi|
\end{tabular}
\end{center}
%
The conditional |\ifchilddocmanual| is true whenever
a part to be included by |\input| is being compiled,
and the name of the part is stored in |\childdocname|.

%%%%%%%%%%%%%%%%%%%%%%%%%%%%%%%%%%%%%%%%
\DescribeMacro{\childdocby}
Each part to be included by |\input| should start with:
%
\begin{center}
\begin{tabular}{l}
|\input{childdoc.def}|\\
|\childdocby{|\textit{main}|}|\\
\end{tabular}
\end{center}
%
The directive |\childdocby| is similar to |\childdocof|
described in \secref{sec:include},
but the subsequent selection of content must be done manually.
To that end, both |\ifchilddoc| and |\ifchilddocmanual|
will be true upon processing of a part,
and the name of the part is stored in |\childdocname|.
Note that |\jobname| will be set to the filename of the current part
so that each part receives an individual |.aux| file
that does not interfere with the |.aux| file(s) of the main document.
This behaviour can be altered by the alternative form
|\childdocby[*]{|\textit{main}|}| (with a non-empty optional argument)
which uses the |.aux| file of the main document
by setting |\jobname| to \textit{main}.

%%%%%%%%%%%%%%%%%%%%%%%%%%%%%%%%%%%%%%%%%%%%%%%%%%%%%%%%%%%%%%%%%%%%%%%%%%%%%%%%
\subsection{Driver Development}
\label{sec:driver}

The \textsf{childdoc} mechanism can also be use for the development
of definition files such as \LaTeX{} styles or classes.
This case differs from the above setup with multiple parts
included by |\include| in that no |\includeonly| should be invoked.
This can be achieved by starting the include file
(before |\ProvidesPackage|) with:
%
\begin{center}
\begin{tabular}{l}
|\input{childdoc.def}|\\
|\childdocforward{|\textit{main}|}|\\
\end{tabular}
\end{center}
%
or alternatively with:
%
\begin{center}
\begin{tabular}{l}
|\input{childdoc.def}|\\
|\childdocby{|\textit{main}|}|\\
\end{tabular}
\end{center}
%
Both forms have slightly different effects as described above.
The main file is prepared as usual, see \secref{sec:include}.

%%%%%%%%%%%%%%%%%%%%%%%%%%%%%%%%%%%%%%%%%%%%%%%%%%%%%%%%%%%%%%%%%%%%%%%%%%%%%%%%
\subsection{Legacy Detection}
\label{sec:detection}

The directive |\childdocmain| in the main file can detect
whether the complete document or merely a child is to be compiled
even without using the directive |\childdocof|.
This method is deprecated because it is less robust
and there is no compelling reason to use it;
it is merely provided for backward compatibility
and it may be removed in future versions.

If the detection mechanism is to be used,
it is mandatory to correctly specify
the filename of the main file as the argument of |\childdocmain|:
%
\begin{center}
\begin{tabular}{l}
|\input{childdoc.def}|\\
|\childdocmain{|\textit{main}|}|\\
\end{tabular}
\end{center}
%
If |\jobname| does not match the argument \textit{main} of |\childdocmain|,
it is assumed that |\jobname| points to the child file to be compiled.
When using |\childdocmain| with the main file specified as argument,
it suffices to start a child file
with just |\input{|\textit{main}|}|
without loading of the package and using |\childdocof|.
If instead all processing is done
with the appropriate \textsf{childdoc} directives,
the argument of \textit{main} of |\childdocmain| can be empty.

An alternative version of the command line processing described
in \secref{sec:commandline} using the detection mechanism reads:
%
\begin{center}
|... -jobname "|\textit{target}|" "|[\textit{flags}]%
[|\def\jobname{|\textit{dest}|}|]|\input{|\textit{main}|}"|
\end{center}

%%%%%%%%%%%%%%%%%%%%%%%%%%%%%%%%%%%%%%%%%%%%%%%%%%%%%%%%%%%%%%%%%%%%%%%%%%%%%%%%
\subsection{Manual Code}
\label{sec:manual}

In case one cannot be certain whether the definitions file |childdoc.def|
is installed on the target \TeX{} distribution
and one prefers not to ship it,
it is conceivable to paste a few relevant commands into the sources.

To that end, drop all statements |\input{childdoc.def}|
and perform the replacements as outlined below.
Instead of |\childdocmain{|\textit{main}|}| add the following code
to the top of the main file:
%
\begin{center}
\begin{tabular}{l}
|\||ifdefined\childdocname\endinput\||fi\newif\ifchilddoc|\\
|\edef\childdocname{\scantokens\expandafter{\jobname\noexpand}}|\\
|\def\childdocmain{|\textit{main}|}\||ifx\childdocmain\childdocname\||else|\\
|\childdoctrue\includeonly{\childdocname}\let\jobname\childdocmain\||fi|\\
\end{tabular}
\end{center}
%
Instead of |\childdocof{|\textit{main}|}| just include the main file
at the top of each child file:
%
\begin{center}
|\input{|\textit{main}|}|
\end{center}
%
A simple redirection |\childdocforward{|\textit{dest}|}| is achieved by:
%
\begin{center}
|\def\jobname{|\textit{dest}|}\input{\jobname}|
\end{center}
%
The redirection with prefix
|\childdocforwardprefix[|\textit{prefix}|]{|\textit{dest}|}|
is accomplished by:
%
\begin{center}
\begin{tabular}{l}
|{\edef\jobname{\scantokens\expandafter{\jobname\noexpand}}|\\
|\def\redirectjob |\textit{prefix}|#1~~~{\gdef\jobname{|\textit{dest}|#1}}|\\
|\expandafter\redirectjob\jobname~~~}\input{\jobname}|
\end{tabular}
\end{center}

In an alternative approach,
child documents can be compiled by a specific command line
without additional code or specific definitions:
%
\begin{center}
|... -jobname "|\textit{target}|" "|[\textit{flags}]%
|\includeonly{|\textit{dest}|}\input{|\textit{main}|}"|
\end{center}
%

%%%%%%%%%%%%%%%%%%%%%%%%%%%%%%%%%%%%%%%%%%%%%%%%%%%%%%%%%%%%%%%%%%%%%%%%%%%%%%%%
%%%%%%%%%%%%%%%%%%%%%%%%%%%%%%%%%%%%%%%%%%%%%%%%%%%%%%%%%%%%%%%%%%%%%%%%%%%%%%%%
\section{Information}

%%%%%%%%%%%%%%%%%%%%%%%%%%%%%%%%%%%%%%%%%%%%%%%%%%%%%%%%%%%%%%%%%%%%%%%%%%%%%%%%
\subsection{Copyright}

Copyright \copyright{} 2017--2018 Niklas Beisert

This work may be distributed and/or modified under the
conditions of the \LaTeX{} Project Public License, either version 1.3
of this license or (at your option) any later version.
The latest version of this license is in
  \url{http://www.latex-project.org/lppl.txt}
and version 1.3 or later is part of all distributions of \LaTeX{}
version 2005/12/01 or later.

This work has the LPPL maintenance status `maintained'.

The Current Maintainer of this work is Niklas Beisert.

This work consists of the files |README.txt|, |childdoc.ins| and |childdoc.dtx|
as well as the derived files |childdoc.def|, |cdocsamp.tex|
with |cdocsch1.tex|, |cdocsch2.tex|, |cdocspt3.tex|, |cdocspt4.tex|,
|cdocsdrf.tex|, |cdocsfn1.tex|, |cdocsfn2.tex|
as well as |childdoc.pdf|.

%%%%%%%%%%%%%%%%%%%%%%%%%%%%%%%%%%%%%%%%%%%%%%%%%%%%%%%%%%%%%%%%%%%%%%%%%%%%%%%%
\subsection{Files and Installation}

The package consists of the files:
%
\begin{center}
\begin{tabular}{ll}
    |README.txt|   & readme file \\
    |childdoc.ins| & installation file \\
    |childdoc.dtx| & source file \\
    |childdoc.def| & definition file \\
    |cdocsamp.tex| & sample main file \\
    |cdocsch1.tex| & sample include file \\
    |cdocsch2.tex| & sample include file \\
    |cdocspt3.tex| & sample part file \\
    |cdocspt4.tex| & sample part file \\
    |cdocsdrf.tex| & sample redirection file \\
    |cdocsfn1.tex| & sample redirection file \\
    |cdocsfn2.tex| & sample redirection file \\
    |childdoc.pdf| & manual
\end{tabular}
\end{center}
%
The distribution consists of the files
|README.txt|, |childdoc.ins| and |childdoc.dtx|.
%
\begin{itemize}
\item
Run (pdf)\LaTeX{} on |childdoc.dtx|
to compile the manual |childdoc.pdf| (this file).
\item
Run \LaTeX{} on |childdoc.ins| to create the definitions file |childdoc.def|
and the sample |cdocsamp.tex| with include files
|cdocsch1.tex|, |cdocsch2.tex|, |cdocspt3.tex|, |cdocspt4.tex|,
|cdocsdrf.tex|, |cdocsfn1.tex|, |cdocsfn2.tex|.
Then copy the file |childdoc.def| to an appropriate directory of your \LaTeX{}
distribution, e.g.\ \textit{texmf-root}|/tex/latex/childdoc|.
\end{itemize}

%%%%%%%%%%%%%%%%%%%%%%%%%%%%%%%%%%%%%%%%%%%%%%%%%%%%%%%%%%%%%%%%%%%%%%%%%%%%%%%%
\subsection{Related CTAN Packages}

There are several other packages which offer a similar functionality:
%
\begin{itemize}
\item
The packages
\href{http://ctan.org/pkg/docmute}{\textsf{docmute}},
\href{http://ctan.org/pkg/includex}{\textsf{includex}} and
\href{http://ctan.org/pkg/standalone}{\textsf{standalone}}
provide commands to include only the document body of
a child file thus allowing both files to be compiled individually.
\item
The packages \href{http://ctan.org/pkg/subdocs}{\textsf{subdocs}}
and \href{http://ctan.org/pkg/subfiles}{\textsf{subfiles}}
provide structures in which the main and child documents can be
encapsulated and allowing them to be compiled individually.
The inclusion mechanism is different from the conventional |\include|.
\item
The package \href{http://ctan.org/pkg/combine}{\textsf{combine}}
is an elaborate solution to combine several documents into one.
\end{itemize}
%
See also the CTAN topic \href{http://ctan.org/topic/subdocs}{\textsf{subdocs}}
for further related packages.
The present package differs from the above solutions in that
a document structure constructed with the conventional |\include| mechanism
just needs two extra commands at the top of every file
such that all constituent files can be compiled individually.

%%%%%%%%%%%%%%%%%%%%%%%%%%%%%%%%%%%%%%%%%%%%%%%%%%%%%%%%%%%%%%%%%%%%%%%%%%%%%%%%
%\subsection{Feature Suggestions}
%
%The following is a list of features which may be useful for future
%versions of this package:
%%
%\begin{itemize}
%\item
%\ldots
%\end{itemize}

%%%%%%%%%%%%%%%%%%%%%%%%%%%%%%%%%%%%%%%%%%%%%%%%%%%%%%%%%%%%%%%%%%%%%%%%%%%%%%%%
\subsection{Revision History}

%%%%%%%%%%%%%%%%%%%%%%%%%%%%%%%%%%%%%%%%
\paragraph{v2.0:} 2018/12/30

\begin{itemize}
\item
immediate forward processing
\item
added |\childdocby| mechanism
\item
manual restructured
\end{itemize}

%%%%%%%%%%%%%%%%%%%%%%%%%%%%%%%%%%%%%%%%
\paragraph{v1.6:} 2018/01/17

\begin{itemize}
\item
application for development of include files
\item
corrections to manual
\end{itemize}

%%%%%%%%%%%%%%%%%%%%%%%%%%%%%%%%%%%%%%%%
\paragraph{v1.5:} 2017/05/21

\begin{itemize}
\item
more complete structuring introduced
\item
|\childdocof| introduced
\item
|\childdoc| renamed to |\childdocmain|
\item
|\childredirect| renamed to |\childdocforward| and |\childdocforwardprefix|
and functionality expanded
\end{itemize}

%%%%%%%%%%%%%%%%%%%%%%%%%%%%%%%%%%%%%%%%
\paragraph{v1.0:} 2017/04/27

\begin{itemize}
\item
manual and install package
\item
first version published on CTAN
\end{itemize}

%%%%%%%%%%%%%%%%%%%%%%%%%%%%%%%%%%%%%%%%
\paragraph{v0.6:} 2017/04/26

\begin{itemize}
\item
redirection mechanism added
\end{itemize}

%%%%%%%%%%%%%%%%%%%%%%%%%%%%%%%%%%%%%%%%
\paragraph{v0.5:} 2017/04/26

\begin{itemize}
\item
functionality in definition file
\end{itemize}


%%%%%%%%%%%%%%%%%%%%%%%%%%%%%%%%%%%%%%%%%%%%%%%%%%%%%%%%%%%%%%%%%%%%%%%%%%%%%%%%
%%%%%%%%%%%%%%%%%%%%%%%%%%%%%%%%%%%%%%%%%%%%%%%%%%%%%%%%%%%%%%%%%%%%%%%%%%%%%%%%
%%%%%%%%%%%%%%%%%%%%%%%%%%%%%%%%%%%%%%%%%%%%%%%%%%%%%%%%%%%%%%%%%%%%%%%%%%%%%%%%
\appendix

\settowidth\MacroIndent{\rmfamily\scriptsize 000\ }

 \DocInput{childdoc.dtx}

\end{document}
%</driver>
% \fi
%
% %%%%%%%%%%%%%%%%%%%%%%%%%%%%%%%%%%%%%%%%%%%%%%%%%%%%%%%%%%%%%%%%%%%%%%%%%%%%%%
% %%%%%%%%%%%%%%%%%%%%%%%%%%%%%%%%%%%%%%%%%%%%%%%%%%%%%%%%%%%%%%%%%%%%%%%%%%%%%%
% \section{Sample}
%\iffalse
%<*samplemain>
%\fi
%
% The following presents a sample document
% with two chapters, two parts, a title page,
% a compile flag as well as three forwarding files to set the flag.
% It consists of eight |.tex| files:
% \begin{center}
% \begin{tabular}{ll}
% |cdocsamp.tex|&main file\\
% |cdocsch1.tex|&include file for chapter 1\\
% |cdocsch2.tex|&include file for chapter 2\\
% |cdocspt3.tex|&include file for part 3\\
% |cdocspt4.tex|&include file for part 4\\
% |cdocsdrf.tex|&forwarding file for main file in draft mode\\
% |cdocsfi1.tex|&forwarding file for final version of chapter 1\\
% |cdocsfi2.tex|&forwarding file for final version of chapter 2\\
% \end{tabular}
% \end{center}
% Each of the eight files can be compiled directly by the \LaTeX{} compiler.
%
% %%%%%%%%%%%%%%%%%%%%%%%%%%%%%%%%%%%%%%
% \paragraph{Main File.}
%
% The main file is called |cdocsamp.tex|.
%
% Load the \textsf{childdoc} definitions and
% declare the filename for the main document:
%    \begin{macrocode}
\input{childdoc.def}
\childdocmain{}
%    \end{macrocode}

% Optional override for |\version| flag:
%    \begin{macrocode}
%%\ifchilddoc\else\providecommand{\version}{draft}\fi
%    \end{macrocode}

% Define the default values for the |\version| flag
% (|final| for the main file and |draft| for childs):
%    \begin{macrocode}
\ifchilddoc
\providecommand{\version}{draft}
\else
\providecommand{\version}{final}
\fi
%    \end{macrocode}

% Load the standard document class:
%    \begin{macrocode}
\documentclass[12pt]{article}
%    \end{macrocode}

% Start the document body:
%    \begin{macrocode}
\begin{document}
%    \end{macrocode}

% Declare a title page.
% Print title, part of document being processed and version flag:
%    \begin{macrocode}
\addtocounter{page}{-1}
\begin{center}
{\LARGE\bfseries{}childdoc example\par}
\vspace{1cm}
\ifchilddoc
\ifchilddocmanual part\else chapter\fi:
`\childdocname' of `\childdocjob'\par
\else
main document: `\childdocjob'\par
\fi
version: \version\par
\end{center}
\newpage
%    \end{macrocode}

% Manually include selected file,
% otherwise process as usual:
%    \begin{macrocode}
\ifchilddocmanual
\section*{part `\childdocname'}
\input{\childdocname}
\else
%    \end{macrocode}

% Include the two chapters:
%    \begin{macrocode}
\include{cdocsch1}
\include{cdocsch2}
%    \end{macrocode}

% Include the two parts unless only chapters should be displayed:
%    \begin{macrocode}
\ifchilddoc\else
\section{part three}
\input{cdocspt3}
\section{part four}
\input{cdocspt4}
\fi
%    \end{macrocode}

% Process as usual until here:
%    \begin{macrocode}
\fi
%    \end{macrocode}

% End of document body:
%    \begin{macrocode}
\end{document}
%    \end{macrocode}
%\iffalse
%</samplemain>
%\fi
%
% %%%%%%%%%%%%%%%%%%%%%%%%%%%%%%%%%%%%%%
% \paragraph{Chapter Include Files.}
%
% The include files are called |cdocsch1.tex| and |cdocsch2.tex|.
%
%\iffalse
%<*samplechap1|samplechap2>
%\fi

% Optional override for |\version| flag:
%    \begin{macrocode}
%%\providecommand{\version}{final}
%    \end{macrocode}

% Include the main document:
%    \begin{macrocode}
\input{childdoc.def}
\childdocof{cdocsamp}
%    \end{macrocode}

%\iffalse
%</samplechap1|samplechap2>
%\fi
%
%\iffalse
%<*samplechap1>
%\fi
% Some text for chapter 1:
%    \begin{macrocode}
\section{one}
some text in chapter one
%    \end{macrocode}

%\iffalse
%</samplechap1>
%\fi
% Some text for chapter 2:
%\iffalse
%<*samplechap2>
%\fi
%    \begin{macrocode}
\section{two}
more text in chapter two
%    \end{macrocode}

%\iffalse
%</samplechap2>
%\fi
%
% %%%%%%%%%%%%%%%%%%%%%%%%%%%%%%%%%%%%%%
% \paragraph{Part Include Files.}
%
% The include files are called |cdocspt3.tex| and |cdocspt4.tex|.
%
%\iffalse
%<*samplepart3|samplepart4>
%\fi

% Optional override for |\version| flag:
%    \begin{macrocode}
%%\providecommand{\version}{final}
%    \end{macrocode}

% Include the main document:
%    \begin{macrocode}
\input{childdoc.def}
\childdocby{cdocsamp}
%    \end{macrocode}

%\iffalse
%</samplepart3|samplepart4>
%\fi
%
%\iffalse
%<*samplepart3>
%\fi
% Some text for part 3:
%    \begin{macrocode}
some text in part three
%    \end{macrocode}

%\iffalse
%</samplepart3>
%\fi
% Some text for part 4:
%\iffalse
%<*samplepart4>
%\fi
%    \begin{macrocode}
more text in part four
%    \end{macrocode}

%\iffalse
%</samplepart4>
%\fi
%
% %%%%%%%%%%%%%%%%%%%%%%%%%%%%%%%%%%%%%%
% \paragraph{Forwarding for a Complete Draft.}
%
% The following forwarding file |cdocsdrf.tex|
% compiles the main document in draft mode:
%\iffalse
%<*sampledraft>
%\fi
%    \begin{macrocode}
\def\version{draft}
\input{childdoc.def}
\childdocforward{cdocsamp}
%    \end{macrocode}

%\iffalse
%</sampledraft>
%\fi
%
% %%%%%%%%%%%%%%%%%%%%%%%%%%%%%%%%%%%%%%
% \paragraph{Forwarding for Final Version of the Chapters.}
%
% The following forwarding files |cdocsfn1.tex| and |cdocsfn2.tex|
% (with identical content)
% compile the final versions of the child documents
% |cdocsch1.tex| and |cdocsch2.tex|, respectively:
%\iffalse
%<*samplefinal>
%\fi
%    \begin{macrocode}
\def\version{final}
\input{childdoc.def}
\childdocforwardprefix[cdocsamp]{cdocsfn}{cdocsch}
%    \end{macrocode}

%\iffalse
%</samplefinal>
%\fi
%
% %%%%%%%%%%%%%%%%%%%%%%%%%%%%%%%%%%%%%%
% \paragraph{Command Line Processing.}
%
% The following three command lines generate the output files
% |cdocscld|, |cdocscl1| and |cdocscl2|
% which should be identical to
% |cdocsdrf|, |cdocsch1| and |cdocsfn2|, respectively:
% \begin{center}
% \begin{tabular}{l}
% |latex -jobname cdocscld \|\\
% |  "\def\version{draft}\input{childdoc.def}\childdocforward{cdocsamp}"|\\
% |latex -jobname cdocscl1 \|\\
% |  "\input{childdoc.def}\childdocforward[cdocsamp]{cdocsch1}"|\\
% |latex -jobname cdocscl2 \|\\
% |  "\def\version{final}\input{childdoc.def}\childdocforward{cdocsch2}"|
% \end{tabular}
% \end{center}
% Note that the trailing backslash on each first line
% merely continues the input to the second line
% (for convenient cut ant paste).
% Furthermore, the command |latex| can be replaced by any
% of its alternative versions such as |pdflatex|.
%
% %%%%%%%%%%%%%%%%%%%%%%%%%%%%%%%%%%%%%%%%%%%%%%%%%%%%%%%%%%%%%%%%%%%%%%%%%%%%%%
% %%%%%%%%%%%%%%%%%%%%%%%%%%%%%%%%%%%%%%%%%%%%%%%%%%%%%%%%%%%%%%%%%%%%%%%%%%%%%%
% \section{Implementation}
%\iffalse
%<*package>
%\fi
%
% This section describes the definitions file |childdoc.def|.

% The definitions cannot be loaded using |\usepackage| or |\RequirePackage|
% which has a mechanism to prevent loading a style file more than once.
% When loading the definitions by means of |\input|
% multiple instances have to be prevented manually:
%\iffalse
%This code needs to be before the `\ProvidesFile' directive
%which is defined at the beginning of this file.
%Therefore it is also placed there and commented out here.
%</package>
%<*discard>
%\fi
%    \begin{macrocode}
\ifdefined\childdocmain\endinput\fi
%    \end{macrocode}
%\iffalse
%</discard>
%<*package>
%\fi
%
% \macro{\ifchilddoc}
% \macro{\ifchilddocmanual}
% The conditional |\ifchilddoc| tells whether a
% child (true) or main (false) document is being compiled.
% The conditional |\ifchilddocmanual| tells whether
% the |\includeonly| mechanism is used (false) or
% the selection of child files must be performed manually (true).
% The definitions initialise to false:
%    \begin{macrocode}
\newif\ifchilddoc
\newif\ifchilddocmanual
%    \end{macrocode}

% \macro{\childdocname}
% \macro{\childdocjob}
% The macro |\childdocname| stores the name of the main document
% to be compiled. The macro |\childdocjob| stores the name of
% the document on which the \LaTeX{} compiler was originally invoked.
% The content of |\jobname| cannot be compared
% to filenames specified in the source due to different catcodes.
% The following code rescans |\jobname|, stores the result
% in |\childdocname| and saves a copy in |\childdocjob|:
%    \begin{macrocode}
\edef\childdocname{\scantokens\expandafter{\jobname\noexpand}}
\let\childdocjob\childdocname
%    \end{macrocode}

% \macro{\childdocdisable}
% The macro |\childdocdisable| prevents the main file
% from being processed more than once.
% At this stage, the main document command |\childdocmain|
% is assumed to be called once again where it should do nothing.
% Any subsequent call to it should prevent
% a secondary processing of the main document
% It overwrites the forwarding commands
% |\childdocof| and |\childdocforward|
% with empty macros to prevent further inclusions of the main document:
%    \begin{macrocode}
\newcommand{\childdocdisable}
{
  \renewcommand{\childdocmain}[1]{\renewcommand{\childdocmain}[1]{\endinput}}
  \renewcommand{\childdocof}[1]{}
  \renewcommand{\childdocby}[2][]{}
  \renewcommand{\childdocforward}[2][]{}
  \renewcommand{\childdocdisable}{}
}
%    \end{macrocode}

% \macro{\childdocmain}
% The macro |\childdocmain| is to be called at the top of the main file
% with nothing or the main filename (without extension) as argument.
% First, it breaks loops.
% If the argument is not empty and does not match |\childdocname|
% (which is set by the first inclusion of |childdoc.def|),
% |\ifchilddoc| is set to true, |\includeonly| is applied to the child file
% and |\jobname| is set to the main file
% (for proper handling of |.aux| files):
%    \begin{macrocode}
\newcommand{\childdocmain}[1]
{
  \childdocdisable\childdocmain{}
  \if?#1?\else
    \begingroup
      \def\childdoctmp{#1}
      \ifx\childdoctmp\childdocname
        \def\childdoctmp{}
      \else
        \def\childdoctmp
        {
          \childdoctrue
          \includeonly{\childdocname}
          \def\childdocjob{#1}
          \def\jobname{#1}
        }
      \fi
      \expandafter
    \endgroup
    \childdoctmp
  \fi
}
%    \end{macrocode}

% \macro{\childdocof}
% The command |\childdocof| redirects
% compilation to the main file |#1|.
%    \begin{macrocode}
\newcommand{\childdocof}[1]
{
  \childdocdisable
  \childdoctrue
  \includeonly{\childdocname}
  \def\jobname{#1}
  \def\childdocjob{#1}
  \input{#1}
}
%    \end{macrocode}

% \macro{\childdocby}
% The command |\childdocby| ....
%    \begin{macrocode}
\newcommand{\childdocby}[2][]
{
  \childdocdisable
  \childdoctrue
  \childdocmanualtrue
  \if?#1?\else
    \def\jobname{#2}
  \fi
  \def\childdocjob{#2}
  \input{#2}
  \endinput
}
%    \end{macrocode}

% \macro{\childdocforward}
% The command |\childdocforward| redirects
% compilation to the main file or
% (if the optional argument is given) a child file.
% Parameters are set as if the main file
% or a child file starting with |\childdocof| was compiled.
% Then compilation is handed over to the main file:
%    \begin{macrocode}
\newcommand{\childdocforward}[2][]
{
  \begingroup
    \if?#1?
      \def\childdoctmp
      {
        \def\childdocname{#2}
        \def\childdocjob{#2}
        \def\jobname{#2}
        \input{#2}
        \endinput
      }
    \else
      \def\childdoctmp
      {
        \childdocdisable
        \def\childdocname{#2}
        \childdoctrue
        \includeonly{#2}
        \def\childdocjob{#1}
        \def\jobname{#1}
        \input{#1}
        \endinput
      }
    \fi
    \expandafter
  \endgroup
  \childdoctmp
}
%    \end{macrocode}

% \macro{\childdocforwardprefix}
% The command |\childdocforwardprefix| redirects
% compilation to the main or a child file by means of a pattern.
% The prefix |#1| in the current filename is replaced by |#2|
% and the suffix of the current filename is kept
% (it is assumed that the filename does not contain the substring `|~~~|'
% which is used as a delimiter).
% Compilation is handed over to the new file by |\childdocforward|:
%    \begin{macrocode}
\newcommand{\childdocforwardprefix}[3][]
{
  \begingroup
    \def\childdocextract #2##1~~~{\def\childdoctmp{\childdocforward[#1]{#3##1}}}
    \expandafter\childdocextract\childdocname~~~
    \expandafter
  \endgroup
  \childdoctmp
}
%    \end{macrocode}

% \macro{\childdoc}
% The deprecated macro |\childdoc| is a legacy version of |\childdocmain|:
%    \begin{macrocode}
\newcommand{\childdoc}{\childdocmain}
%    \end{macrocode}

% \macro{\childdocredirect}
% The deprecated macro |\childdocredirect| is a legacy version
% of |\childdocforward| and |\childdocforwardprefix|:
%    \begin{macrocode}
\newcommand{\childdocredirect}[2][]
{
  \begingroup
    \if?#1?
      \def\childdoctmp{\childdocforward{#2}}
    \else
      \def\childdoctmp{\childdocforwardprefix{#1}{#2}}
    \fi
    \expandafter
  \endgroup
  \childdoctmp
}
%    \end{macrocode}

%\iffalse
%</package>
%\fi
%
\endinput
|\\
|\childdocforward{|\textit{main}|}|\\
\end{tabular}
\end{center}
%
or alternatively with:
%
\begin{center}
\begin{tabular}{l}
|% \iffalse
%
% childdoc.dtx Copyright (C) 2017-2018 Niklas Beisert
%
% This work may be distributed and/or modified under the
% conditions of the LaTeX Project Public License, either version 1.3
% of this license or (at your option) any later version.
% The latest version of this license is in
%   http://www.latex-project.org/lppl.txt
% and version 1.3 or later is part of all distributions of LaTeX
% version 2005/12/01 or later.
%
% This work has the LPPL maintenance status `maintained'.
%
% The Current Maintainer of this work is Niklas Beisert.
%
% This work consists of the files childdoc.dtx and childdoc.ins
% and the derived files childdoc.def and cdocsamp.tex with
% cdocsch1.tex, cdocsch2.tex, cdocsdrf.tex, cdocsfn1.tex, cdocsfn2.tex.
%
%<package>\ifdefined\childdocmain\endinput\fi
%<package>\ProvidesFile{childdoc.def}[2018/12/30 v2.0 child document driver]
%<samplemain>\ProvidesFile{cdocsamp.tex}[2018/12/30 v2.0 sample for childdoc]
%<*driver>
%\ProvidesFile{childdoc.drv}[2018/12/30 v2.0 childdoc reference manual file]
\PassOptionsToClass{10pt,a4paper}{article}
\documentclass{ltxdoc}

\usepackage[margin=35mm]{geometry}
\usepackage{hyperref}
\usepackage{hyperxmp}
\usepackage[usenames]{color}

\hypersetup{colorlinks=true}
\hypersetup{pdfstartview=FitH}
\hypersetup{pdfpagemode=UseNone}
\hypersetup{pdfsource={}}
\hypersetup{pdflang={en-UK}}
\hypersetup{pdfcopyright={Copyright 2017-2018 Niklas Beisert.
  This work may be distributed and/or modified under the
  conditions of the LaTeX Project Public License, either version 1.3
  of this license or (at your option) any later version.}}
\hypersetup{pdflicenseurl={http://www.latex-project.org/lppl.txt}}
\hypersetup{pdfcontactaddress={ETH Zurich, ITP, HIT K,
  Wolfgang-Pauli-Strasse 27}}
\hypersetup{pdfcontactpostcode={8093}}
\hypersetup{pdfcontactcity={Zurich}}
\hypersetup{pdfcontactcountry={Switzerland}}
\hypersetup{pdfcontactemail={nbeisert@itp.phys.ethz.ch}}
\hypersetup{pdfcontacturl={http://people.phys.ethz.ch/\xmptilde nbeisert/}}

\newcommand{\secref}[1]{\hyperref[#1]{section \ref*{#1}}}

\parskip1ex
\parindent0pt
\let\olditemize\itemize
\def\itemize{\olditemize\parskip0pt}

\begin{document}

\title{The \textsf{childdoc} Package}
\hypersetup{pdftitle={The childdoc Package}}
\author{Niklas Beisert\\[2ex]
  Institut f\"ur Theoretische Physik\\
  Eidgen\"ossische Technische Hochschule Z\"urich\\
  Wolfgang-Pauli-Strasse 27, 8093 Z\"urich, Switzerland\\[1ex]
  \href{mailto:nbeisert@itp.phys.ethz.ch}
  {\texttt{nbeisert@itp.phys.ethz.ch}}}
\hypersetup{pdfauthor={Niklas Beisert}}
\hypersetup{pdfsubject={Manual for the LaTeX2e Package childdoc}}
\date{30 December 2018, \textsf{v2.0}}
\maketitle

\begin{abstract}\noindent
\textsf{childdoc} is a \LaTeXe{} package
that enables the direct compilation
of document sections included by |\include|
to individual files.
\end{abstract}

\begingroup
\parskip0ex
\tableofcontents
\endgroup

%%%%%%%%%%%%%%%%%%%%%%%%%%%%%%%%%%%%%%%%%%%%%%%%%%%%%%%%%%%%%%%%%%%%%%%%%%%%%%%%
%%%%%%%%%%%%%%%%%%%%%%%%%%%%%%%%%%%%%%%%%%%%%%%%%%%%%%%%%%%%%%%%%%%%%%%%%%%%%%%%
\section{Introduction}

\LaTeX{} provides a mechanism to structure a large document (such as a book)
into a main file and several child files (containing the chapters)
using the |\include| command.
This mechanism is beneficial for documents
which span hundreds of pages in order to
make the source file(s) more manageable.
Moreover, compilation can be restricted to
selected child files by means of the |\includeonly| command.
The latter feature can be used to reduce the compilation time while editing
(this was significantly more useful in the earlier days of \LaTeX{})
or to generate a smaller document which is easier to navigate.
Another application of |\includeonly| is to generate
documents consisting of selected parts of the complete document.

However, there are a few drawbacks of the plain |\include| mechanism:
\begin{itemize}
\item
The child files cannot be compiled on their own,
they can only be compiled via the main file.
A naive editing environment
(such as a text editor with an option
to have the current file processed by \LaTeX)
may require one to switch to the main file before compiling;
attempting to compile the child file produces errors.
\item
The main file must be modified (each time)
to adjust the |\includeonly| command
to the present needs. This easily leaves the main file in a messy state.
\item
The generated document will always carry the filename
of the main document. This is inconvenient if
several child files are to be compiled and
to be kept for distribution.
\end{itemize}

The present package provides a simple interface
to make child files individually compilable by \LaTeX{}.
Compiling a child file then has the same effect as compiling
the main file with an |\includeonly| command
to select the appropriate child.
Moreover the generated document will carry the name of the child
rather than the main file.
This resolves all three above issues.

This feature is meant to make the editing of books,
thesis documents and lecture notes somewhat more convenient.
However, the package can also be used efficiently for
composing a series of documents (such as exercise sheets)
which are typically distributed individually.
It then assists the author in generating the individual documents
(potentially in different versions)
as well as a document containing the collected series.
Another application is in developing style files
or other kinds of included material
where compilation of the style file could redirect
to a sample or test file.

%%%%%%%%%%%%%%%%%%%%%%%%%%%%%%%%%%%%%%%%%%%%%%%%%%%%%%%%%%%%%%%%%%%%%%%%%%%%%%%%
%%%%%%%%%%%%%%%%%%%%%%%%%%%%%%%%%%%%%%%%%%%%%%%%%%%%%%%%%%%%%%%%%%%%%%%%%%%%%%%%
\section{Usage}

First of all, the package \textsf{childdoc} is \emph{not} a standard
\LaTeXe{} |.sty| style file! Therefore it needs to be invoked in
a non-standard way.

%%%%%%%%%%%%%%%%%%%%%%%%%%%%%%%%%%%%%%%%%%%%%%%%%%%%%%%%%%%%%%%%%%%%%%%%%%%%%%%%
\subsection{Included Files}
\label{sec:include}

%%%%%%%%%%%%%%%%%%%%%%%%%%%%%%%%%%%%%%%%
\DescribeMacro{\childdocmain}
To use the package, add the commands
\begin{center}
\begin{tabular}{l}
|\input{childdoc.def}|\\
|\childdocmain{}|\\
\end{tabular}
\end{center}
at the very top of the main \LaTeX{} file,
in particular \emph{before} the |\documentclass| statement!
The argument of |\childdocmain| should be left empty
(but it must be present).

%%%%%%%%%%%%%%%%%%%%%%%%%%%%%%%%%%%%%%%%
\DescribeMacro{\childdocof}
Furthermore, add the commands
\begin{center}
\begin{tabular}{l}
|\input{childdoc.def}|\\
|\childdocof{|\textit{main}|}|\\
\end{tabular}
\end{center}
at the top of every child file \textit{child}
which is included by |\include{|\textit{child}|}|
from within the main file
(or at least for those files to be compiled individually).
The argument \textit{main} must be the filename of the main file.

There are a couple of
considerations in setting up the main and child documents:

%%%%%%%%%%%%%%%%%%%%%%%%%%%%%%%%%%%%%%%%
\paragraph{Restrictions.}

Please note the following restrictions:
\begin{itemize}
\item
|\childdocmain| must be called with one argument \textit{main}
to ensure compatibility with earlier version of the package.
It must either be empty (|\childdocmain{}|)
or precisely match the filename of the main file in which it is specified.
See \secref{sec:detection} for further information.
\item
The filename \textit{main} must be specified without the |.tex| extension.
\item
The filename \textit{main} is case sensitive
(even in case-insensitive file systems)
due to internal string comparison.
\item
The argument \textit{main} should be fully expanded, it cannot be a macro.
\item
Subdirectories and special characters should be avoided in filenames.
\item
The command |\childdocmain{|\textit{main}|}| must be followed by a whitespace.
It should not be followed immediately by another command
or by a comment mark `|%|'.
This is because the \TeX{} parser reads the token immediately following
the argument of |\childdocmain| and puts it
at the beginning of every child section;
however, a white\-space is ignored.
\end{itemize}

%%%%%%%%%%%%%%%%%%%%%%%%%%%%%%%%%%%%%%%%
\paragraph{Content of Main File.}

It is advisable to place all content in the child files included by |\include|.
Any output contained in the main file will appear in all child documents
unless suppressed manually;
it cannot be suppressed automatically by the |\includeonly| directive
and thus should normally be avoided.
A method to include some content in the main file
by means of conditional processing is described in \secref{sec:conditional}.

%%%%%%%%%%%%%%%%%%%%%%%%%%%%%%%%%%%%%%%%
\paragraph{Page Numbering.}

When only a part of the document is compiled,
the appropriate numbering of pages
(as well as other status parameters)
is determined from the |.aux| files.
The latter contain information from previous passes.
However this information needs to propagate through
all intermediate child documents.
Therefore the page numbering in child documents may well
be inconsistent until the complete document is compiled at least once.

A useful (if unconventional) way to always ensure a consistent
page numbering is to restart the numbering in each child document
and denote the pages by `\textit{child}|.|\textit{page}'
where \textit{child} represents the chapter/section number of the child file.
This can be achieved by the command
|\numberwithin{page}{|\textit{child}|}|
of the \textsf{amsmath} package
where \textit{child} can be |chapter| or |section|
depending on the chosen structuring.
Alternatively, one can modify the macro |\thepage| appropriately
and reset the counter |page| at the start of each child file.

%%%%%%%%%%%%%%%%%%%%%%%%%%%%%%%%%%%%%%%%%%%%%%%%%%%%%%%%%%%%%%%%%%%%%%%%%%%%%%%%
\subsection{Conditional Processing}
\label{sec:conditional}

The package provides a mechanism to compile different versions
of a document. To customise the versions further some conditional processing
can come in handy to distinguish which version is being compiled.
The package provides two macros to describe the compilation context:

%%%%%%%%%%%%%%%%%%%%%%%%%%%%%%%%%%%%%%%%
\DescribeMacro{\ifchilddoc}
The conditional |\ifchilddoc| distinguishes between the compilation of
child documents and the main document:
%
\begin{center}
|\ifchilddoc |\textit{child-code}| |[|\||else |\textit{main-code}]| \||fi|
\end{center}

%%%%%%%%%%%%%%%%%%%%%%%%%%%%%%%%%%%%%%%%
\DescribeMacro{\childdocname}
\DescribeMacro{\childdocjob}
The macro |\childdocname| contains the filename (without extension)
of the main or child file being processed.
Note that |\childdocjob| will always contain the name of the main file.

%%%%%%%%%%%%%%%%%%%%%%%%%%%%%%%%%%%%%%%%
\paragraph{Title Page.}

Conditional processing can be used to include a title or banner page
in the main document when proper precautions are taken.
Importantly, the code in the main file should ensure that the page counter
(as well as other status parameters which are stored in the |.aux| files)
takes the same value after the conditional processing.
Otherwise the page numbers may take divergent values
depending on which part is compiled.

For example, a title page could be declared by:
%
\begin{center}
\begin{tabular}{l}
|\ifchilddoc\||else|\\
|\addtocounter{page}{-1}|\\
\textit{code for title page}\\
|\newpage|\\
|\||fi|
\end{tabular}
\end{center}
%
A banner page for the child documents can be generated by:
%
\begin{center}
\begin{tabular}{l}
|\ifchilddoc|\\
|\addtocounter{page}{-1}|\\
\textit{code for banner page}\\
|\newpage|\\
|\||fi|
\end{tabular}
\end{center}
%
Here one could write a message such as:
\begin{center}
|This is the part \childdocname{} of \childdocjob{}.|
\end{center}

%%%%%%%%%%%%%%%%%%%%%%%%%%%%%%%%%%%%%%%%%%%%%%%%%%%%%%%%%%%%%%%%%%%%%%%%%%%%%%%%
\subsection{Flags}
\label{sec:flags}

The package makes it easy to generate different versions
of the main or child documents.
To this end compilation flags can be defined
and assigned different default values.
They will be particularly useful in conjunction
with the forwarding mechanism described in \secref{sec:forward}.

For example, it may be useful to have a flag |\version|
which can be set to |draft| or |final|.
The document source will contain some conditional code
depending on the value of |\version|.
Suppose further, the flag should default to |final| for the main file
and to |draft| for child files
which is a natural assignment for editing the document.
This is achieved by placing the following code
in the preamble of the main document
(below the |\childdocmain| directive):
%
\begin{center}
\begin{tabular}{l}
|\ifchilddoc|\\
|\providecommand{\version}{draft}|\\
|\||else|\\
|\providecommand{\version}{final}|\\
|\||fi|
\end{tabular}
\end{center}
%
The definition by |\providecommand| makes sure
that previous definitions are not overwritten.
Further statements |\providecommand{\version}{...}|
can thus be added before the above code to override it.

For the main file, one might add a line
(between |\childdocmain| and the above block)
%
\begin{center}
|%\ifchilddoc\||else\providecommand{\version}{draft}\||fi|
\end{center}
%
which can be uncommented to produce a draft version.
Likewise one can add a line to the very top of a child file
(above the |\childdocof{|\textit{main}|}| directive)
%
\begin{center}
|%\providecommand{\version}{final}|
\end{center}
%
which can be uncommented to produce the final version of this child document.

%%%%%%%%%%%%%%%%%%%%%%%%%%%%%%%%%%%%%%%%%%%%%%%%%%%%%%%%%%%%%%%%%%%%%%%%%%%%%%%%
\subsection{Forwarding}
\label{sec:forward}

Different versions of the main or child documents
using compilation flags as described in \secref{sec:flags}
can be (permanently) stored in different files
for convenient compilation, viewing and distribution.
To this end, the package defines a command
to pass on compilation to a different file:

%%%%%%%%%%%%%%%%%%%%%%%%%%%%%%%%%%%%%%%%
\DescribeMacro{\childdocforward}
The command |\childdocforward| redirects processing to
another source file:
%
\begin{center}
\begin{tabular}{l}
|\input{childdoc.def}|\\
|\childdocforward[|\textit{main}|]{|\textit{dest}|}|\\
\end{tabular}
\end{center}
%
The argument \textit{dest} is the destination file
(without extension).
It should be the main file or one of the child files.
Note that further \textsf{childdoc} directives
such as |\childdocof| and |\childdocforward|
in the indicated file will be processed in this form.
The optional argument \textit{main}
passes on directly to the main file \textit{main}
while pretending to compile the child \textit{dest}.
This form behaves as if \textit{dest}
issues |\childdocof{|\textit{main}|}| right away,
and no further \textsf{childdoc} directives will be processed.

%%%%%%%%%%%%%%%%%%%%%%%%%%%%%%%%%%%%%%%%
\DescribeMacro{\...prefix}
In the alternative form |\childdocforwardprefix|,
%
\begin{center}
\begin{tabular}{l}
|\input{childdoc.def}|\\
|\childdocforwardprefix[|\textit{main}|]{|\textit{prefix}|}{|\textit{dest}|}|
\end{tabular}
\end{center}
%
the destination file is determined by a pattern
depending on the current file:
To make this work, the current file must be called
`{\textit{prefix}\hspace{0.2em}\textit{suffix}}'
with \textit{prefix} matching precisely the argument.
Processing is then passed on to the file
`{\textit{dest}\hspace{0.2em}\textit{suffix}}'.
Surely, the same effect is achieved by
directly specifying the
argument `{\textit{dest}\hspace{0.2em}\textit{suffix}}'
in the first form.
However, that requires to set up a different file
for each child. With the alternative form of the command
all these files can have exactly the same content
which simplifies setting them up and maintaining them.

For example, the following file |draft.tex|
with a compilation flag |\version| as described in \secref{sec:flags}
compiles the main document as a draft:
%
\begin{center}
\begin{tabular}{l}
|\def\version{draft}|\\
|\input{childdoc.def}|\\
|\childdocforward{|\textit{main}|}|
\end{tabular}
\end{center}
%
Likewise, the following files |final|\textit{nn}|.tex|
compile the final version of the child document
|child|\textit{nn}|.tex|:
%
\begin{center}
\begin{tabular}{l}
|\def\version{final}|\\
|\input{childdoc.def}|\\
|\childdocforwardprefix{final}{child}|
\end{tabular}
\end{center}
%

Note that when several versions of a main file and/or of each child file
are to be generated, it may be convenient to set up a |Makefile| or
shell script to automatise the process.

%%%%%%%%%%%%%%%%%%%%%%%%%%%%%%%%%%%%%%%%%%%%%%%%%%%%%%%%%%%%%%%%%%%%%%%%%%%%%%%%
\subsection{Command Line Processing}
\label{sec:commandline}

The effect of redirection files can also be achieved by invoking
the \LaTeX{} compiler with a more elaborate command line.
Most conveniently this should be done as part
of a shell script or a |Makefile|.

When using \textsf{childdoc} in the main file, the following
command lines effectively perform a redirection
(note that depending on the shell being used,
backslashes may have to be doubled: `|\|' $\to$ `|\\|'):
%
\begin{center}
|... -jobname "|\textit{target}|" |\\|"|[\textit{flags}]%
|\input{childdoc.def}\childdocforward[|\textit{main}|]{|\textit{dest}|}"|
\end{center}
%
Here \textit{target} is the name of the output file,
\textit{main} is the name of the main file
and \textit{dest} is the name of the main or child file to be processed
(all filenames without extensions).
The optional argument \textit{main} can be omitted
if \textit{main} matches \textit{dest}.
Optionally, compilation \textit{flags} can be defined via |\def| commands.
This command line makes the \TeX{} engine believe
it is compiling the file \textit{target}
whose content is specified as the latter parameter.
The provided code then forwards the processing to
\textit{main} or \textit{dest} as described in \secref{sec:forward}.

%%%%%%%%%%%%%%%%%%%%%%%%%%%%%%%%%%%%%%%%%%%%%%%%%%%%%%%%%%%%%%%%%%%%%%%%%%%%%%%%
\subsection{Include by Input}
\label{sec:input}

Including child documents by |\include| has some restrictions by design.
Most notably, the content of a child document always occupies
its own set of pages; pages cannot be shared between child documents.
Usually, this behaviour makes perfect sense
because each child document contain an essential part of the document.
However, in some situations it may be desirable to compose
a document from a collection of parts
without having mandatory page breaks between then.
For this case, the package
provides a mechanism to include parts
by |\input| which can also be processed individually.
However, by construction this mechanism
requires manual handling of the content to be output.

%%%%%%%%%%%%%%%%%%%%%%%%%%%%%%%%%%%%%%%%
\DescribeMacro{\ifchilddocmanual}
The main file should be prepared as usual, see \secref{sec:include}.
However, the document body must make a distinction
between processing of an individual part and of the main document, e.g.:
%
\begin{center}
\begin{tabular}{l}
|\ifchilddocmanual|\\
|\input{\childdocname}|\\
|\||else|\\
\textit{document body with }|\input{|\textit{part}|}|\\
|\||fi|
\end{tabular}
\end{center}
%
The conditional |\ifchilddocmanual| is true whenever
a part to be included by |\input| is being compiled,
and the name of the part is stored in |\childdocname|.

%%%%%%%%%%%%%%%%%%%%%%%%%%%%%%%%%%%%%%%%
\DescribeMacro{\childdocby}
Each part to be included by |\input| should start with:
%
\begin{center}
\begin{tabular}{l}
|\input{childdoc.def}|\\
|\childdocby{|\textit{main}|}|\\
\end{tabular}
\end{center}
%
The directive |\childdocby| is similar to |\childdocof|
described in \secref{sec:include},
but the subsequent selection of content must be done manually.
To that end, both |\ifchilddoc| and |\ifchilddocmanual|
will be true upon processing of a part,
and the name of the part is stored in |\childdocname|.
Note that |\jobname| will be set to the filename of the current part
so that each part receives an individual |.aux| file
that does not interfere with the |.aux| file(s) of the main document.
This behaviour can be altered by the alternative form
|\childdocby[*]{|\textit{main}|}| (with a non-empty optional argument)
which uses the |.aux| file of the main document
by setting |\jobname| to \textit{main}.

%%%%%%%%%%%%%%%%%%%%%%%%%%%%%%%%%%%%%%%%%%%%%%%%%%%%%%%%%%%%%%%%%%%%%%%%%%%%%%%%
\subsection{Driver Development}
\label{sec:driver}

The \textsf{childdoc} mechanism can also be use for the development
of definition files such as \LaTeX{} styles or classes.
This case differs from the above setup with multiple parts
included by |\include| in that no |\includeonly| should be invoked.
This can be achieved by starting the include file
(before |\ProvidesPackage|) with:
%
\begin{center}
\begin{tabular}{l}
|\input{childdoc.def}|\\
|\childdocforward{|\textit{main}|}|\\
\end{tabular}
\end{center}
%
or alternatively with:
%
\begin{center}
\begin{tabular}{l}
|\input{childdoc.def}|\\
|\childdocby{|\textit{main}|}|\\
\end{tabular}
\end{center}
%
Both forms have slightly different effects as described above.
The main file is prepared as usual, see \secref{sec:include}.

%%%%%%%%%%%%%%%%%%%%%%%%%%%%%%%%%%%%%%%%%%%%%%%%%%%%%%%%%%%%%%%%%%%%%%%%%%%%%%%%
\subsection{Legacy Detection}
\label{sec:detection}

The directive |\childdocmain| in the main file can detect
whether the complete document or merely a child is to be compiled
even without using the directive |\childdocof|.
This method is deprecated because it is less robust
and there is no compelling reason to use it;
it is merely provided for backward compatibility
and it may be removed in future versions.

If the detection mechanism is to be used,
it is mandatory to correctly specify
the filename of the main file as the argument of |\childdocmain|:
%
\begin{center}
\begin{tabular}{l}
|\input{childdoc.def}|\\
|\childdocmain{|\textit{main}|}|\\
\end{tabular}
\end{center}
%
If |\jobname| does not match the argument \textit{main} of |\childdocmain|,
it is assumed that |\jobname| points to the child file to be compiled.
When using |\childdocmain| with the main file specified as argument,
it suffices to start a child file
with just |\input{|\textit{main}|}|
without loading of the package and using |\childdocof|.
If instead all processing is done
with the appropriate \textsf{childdoc} directives,
the argument of \textit{main} of |\childdocmain| can be empty.

An alternative version of the command line processing described
in \secref{sec:commandline} using the detection mechanism reads:
%
\begin{center}
|... -jobname "|\textit{target}|" "|[\textit{flags}]%
[|\def\jobname{|\textit{dest}|}|]|\input{|\textit{main}|}"|
\end{center}

%%%%%%%%%%%%%%%%%%%%%%%%%%%%%%%%%%%%%%%%%%%%%%%%%%%%%%%%%%%%%%%%%%%%%%%%%%%%%%%%
\subsection{Manual Code}
\label{sec:manual}

In case one cannot be certain whether the definitions file |childdoc.def|
is installed on the target \TeX{} distribution
and one prefers not to ship it,
it is conceivable to paste a few relevant commands into the sources.

To that end, drop all statements |\input{childdoc.def}|
and perform the replacements as outlined below.
Instead of |\childdocmain{|\textit{main}|}| add the following code
to the top of the main file:
%
\begin{center}
\begin{tabular}{l}
|\||ifdefined\childdocname\endinput\||fi\newif\ifchilddoc|\\
|\edef\childdocname{\scantokens\expandafter{\jobname\noexpand}}|\\
|\def\childdocmain{|\textit{main}|}\||ifx\childdocmain\childdocname\||else|\\
|\childdoctrue\includeonly{\childdocname}\let\jobname\childdocmain\||fi|\\
\end{tabular}
\end{center}
%
Instead of |\childdocof{|\textit{main}|}| just include the main file
at the top of each child file:
%
\begin{center}
|\input{|\textit{main}|}|
\end{center}
%
A simple redirection |\childdocforward{|\textit{dest}|}| is achieved by:
%
\begin{center}
|\def\jobname{|\textit{dest}|}\input{\jobname}|
\end{center}
%
The redirection with prefix
|\childdocforwardprefix[|\textit{prefix}|]{|\textit{dest}|}|
is accomplished by:
%
\begin{center}
\begin{tabular}{l}
|{\edef\jobname{\scantokens\expandafter{\jobname\noexpand}}|\\
|\def\redirectjob |\textit{prefix}|#1~~~{\gdef\jobname{|\textit{dest}|#1}}|\\
|\expandafter\redirectjob\jobname~~~}\input{\jobname}|
\end{tabular}
\end{center}

In an alternative approach,
child documents can be compiled by a specific command line
without additional code or specific definitions:
%
\begin{center}
|... -jobname "|\textit{target}|" "|[\textit{flags}]%
|\includeonly{|\textit{dest}|}\input{|\textit{main}|}"|
\end{center}
%

%%%%%%%%%%%%%%%%%%%%%%%%%%%%%%%%%%%%%%%%%%%%%%%%%%%%%%%%%%%%%%%%%%%%%%%%%%%%%%%%
%%%%%%%%%%%%%%%%%%%%%%%%%%%%%%%%%%%%%%%%%%%%%%%%%%%%%%%%%%%%%%%%%%%%%%%%%%%%%%%%
\section{Information}

%%%%%%%%%%%%%%%%%%%%%%%%%%%%%%%%%%%%%%%%%%%%%%%%%%%%%%%%%%%%%%%%%%%%%%%%%%%%%%%%
\subsection{Copyright}

Copyright \copyright{} 2017--2018 Niklas Beisert

This work may be distributed and/or modified under the
conditions of the \LaTeX{} Project Public License, either version 1.3
of this license or (at your option) any later version.
The latest version of this license is in
  \url{http://www.latex-project.org/lppl.txt}
and version 1.3 or later is part of all distributions of \LaTeX{}
version 2005/12/01 or later.

This work has the LPPL maintenance status `maintained'.

The Current Maintainer of this work is Niklas Beisert.

This work consists of the files |README.txt|, |childdoc.ins| and |childdoc.dtx|
as well as the derived files |childdoc.def|, |cdocsamp.tex|
with |cdocsch1.tex|, |cdocsch2.tex|, |cdocspt3.tex|, |cdocspt4.tex|,
|cdocsdrf.tex|, |cdocsfn1.tex|, |cdocsfn2.tex|
as well as |childdoc.pdf|.

%%%%%%%%%%%%%%%%%%%%%%%%%%%%%%%%%%%%%%%%%%%%%%%%%%%%%%%%%%%%%%%%%%%%%%%%%%%%%%%%
\subsection{Files and Installation}

The package consists of the files:
%
\begin{center}
\begin{tabular}{ll}
    |README.txt|   & readme file \\
    |childdoc.ins| & installation file \\
    |childdoc.dtx| & source file \\
    |childdoc.def| & definition file \\
    |cdocsamp.tex| & sample main file \\
    |cdocsch1.tex| & sample include file \\
    |cdocsch2.tex| & sample include file \\
    |cdocspt3.tex| & sample part file \\
    |cdocspt4.tex| & sample part file \\
    |cdocsdrf.tex| & sample redirection file \\
    |cdocsfn1.tex| & sample redirection file \\
    |cdocsfn2.tex| & sample redirection file \\
    |childdoc.pdf| & manual
\end{tabular}
\end{center}
%
The distribution consists of the files
|README.txt|, |childdoc.ins| and |childdoc.dtx|.
%
\begin{itemize}
\item
Run (pdf)\LaTeX{} on |childdoc.dtx|
to compile the manual |childdoc.pdf| (this file).
\item
Run \LaTeX{} on |childdoc.ins| to create the definitions file |childdoc.def|
and the sample |cdocsamp.tex| with include files
|cdocsch1.tex|, |cdocsch2.tex|, |cdocspt3.tex|, |cdocspt4.tex|,
|cdocsdrf.tex|, |cdocsfn1.tex|, |cdocsfn2.tex|.
Then copy the file |childdoc.def| to an appropriate directory of your \LaTeX{}
distribution, e.g.\ \textit{texmf-root}|/tex/latex/childdoc|.
\end{itemize}

%%%%%%%%%%%%%%%%%%%%%%%%%%%%%%%%%%%%%%%%%%%%%%%%%%%%%%%%%%%%%%%%%%%%%%%%%%%%%%%%
\subsection{Related CTAN Packages}

There are several other packages which offer a similar functionality:
%
\begin{itemize}
\item
The packages
\href{http://ctan.org/pkg/docmute}{\textsf{docmute}},
\href{http://ctan.org/pkg/includex}{\textsf{includex}} and
\href{http://ctan.org/pkg/standalone}{\textsf{standalone}}
provide commands to include only the document body of
a child file thus allowing both files to be compiled individually.
\item
The packages \href{http://ctan.org/pkg/subdocs}{\textsf{subdocs}}
and \href{http://ctan.org/pkg/subfiles}{\textsf{subfiles}}
provide structures in which the main and child documents can be
encapsulated and allowing them to be compiled individually.
The inclusion mechanism is different from the conventional |\include|.
\item
The package \href{http://ctan.org/pkg/combine}{\textsf{combine}}
is an elaborate solution to combine several documents into one.
\end{itemize}
%
See also the CTAN topic \href{http://ctan.org/topic/subdocs}{\textsf{subdocs}}
for further related packages.
The present package differs from the above solutions in that
a document structure constructed with the conventional |\include| mechanism
just needs two extra commands at the top of every file
such that all constituent files can be compiled individually.

%%%%%%%%%%%%%%%%%%%%%%%%%%%%%%%%%%%%%%%%%%%%%%%%%%%%%%%%%%%%%%%%%%%%%%%%%%%%%%%%
%\subsection{Feature Suggestions}
%
%The following is a list of features which may be useful for future
%versions of this package:
%%
%\begin{itemize}
%\item
%\ldots
%\end{itemize}

%%%%%%%%%%%%%%%%%%%%%%%%%%%%%%%%%%%%%%%%%%%%%%%%%%%%%%%%%%%%%%%%%%%%%%%%%%%%%%%%
\subsection{Revision History}

%%%%%%%%%%%%%%%%%%%%%%%%%%%%%%%%%%%%%%%%
\paragraph{v2.0:} 2018/12/30

\begin{itemize}
\item
immediate forward processing
\item
added |\childdocby| mechanism
\item
manual restructured
\end{itemize}

%%%%%%%%%%%%%%%%%%%%%%%%%%%%%%%%%%%%%%%%
\paragraph{v1.6:} 2018/01/17

\begin{itemize}
\item
application for development of include files
\item
corrections to manual
\end{itemize}

%%%%%%%%%%%%%%%%%%%%%%%%%%%%%%%%%%%%%%%%
\paragraph{v1.5:} 2017/05/21

\begin{itemize}
\item
more complete structuring introduced
\item
|\childdocof| introduced
\item
|\childdoc| renamed to |\childdocmain|
\item
|\childredirect| renamed to |\childdocforward| and |\childdocforwardprefix|
and functionality expanded
\end{itemize}

%%%%%%%%%%%%%%%%%%%%%%%%%%%%%%%%%%%%%%%%
\paragraph{v1.0:} 2017/04/27

\begin{itemize}
\item
manual and install package
\item
first version published on CTAN
\end{itemize}

%%%%%%%%%%%%%%%%%%%%%%%%%%%%%%%%%%%%%%%%
\paragraph{v0.6:} 2017/04/26

\begin{itemize}
\item
redirection mechanism added
\end{itemize}

%%%%%%%%%%%%%%%%%%%%%%%%%%%%%%%%%%%%%%%%
\paragraph{v0.5:} 2017/04/26

\begin{itemize}
\item
functionality in definition file
\end{itemize}


%%%%%%%%%%%%%%%%%%%%%%%%%%%%%%%%%%%%%%%%%%%%%%%%%%%%%%%%%%%%%%%%%%%%%%%%%%%%%%%%
%%%%%%%%%%%%%%%%%%%%%%%%%%%%%%%%%%%%%%%%%%%%%%%%%%%%%%%%%%%%%%%%%%%%%%%%%%%%%%%%
%%%%%%%%%%%%%%%%%%%%%%%%%%%%%%%%%%%%%%%%%%%%%%%%%%%%%%%%%%%%%%%%%%%%%%%%%%%%%%%%
\appendix

\settowidth\MacroIndent{\rmfamily\scriptsize 000\ }

 \DocInput{childdoc.dtx}

\end{document}
%</driver>
% \fi
%
% %%%%%%%%%%%%%%%%%%%%%%%%%%%%%%%%%%%%%%%%%%%%%%%%%%%%%%%%%%%%%%%%%%%%%%%%%%%%%%
% %%%%%%%%%%%%%%%%%%%%%%%%%%%%%%%%%%%%%%%%%%%%%%%%%%%%%%%%%%%%%%%%%%%%%%%%%%%%%%
% \section{Sample}
%\iffalse
%<*samplemain>
%\fi
%
% The following presents a sample document
% with two chapters, two parts, a title page,
% a compile flag as well as three forwarding files to set the flag.
% It consists of eight |.tex| files:
% \begin{center}
% \begin{tabular}{ll}
% |cdocsamp.tex|&main file\\
% |cdocsch1.tex|&include file for chapter 1\\
% |cdocsch2.tex|&include file for chapter 2\\
% |cdocspt3.tex|&include file for part 3\\
% |cdocspt4.tex|&include file for part 4\\
% |cdocsdrf.tex|&forwarding file for main file in draft mode\\
% |cdocsfi1.tex|&forwarding file for final version of chapter 1\\
% |cdocsfi2.tex|&forwarding file for final version of chapter 2\\
% \end{tabular}
% \end{center}
% Each of the eight files can be compiled directly by the \LaTeX{} compiler.
%
% %%%%%%%%%%%%%%%%%%%%%%%%%%%%%%%%%%%%%%
% \paragraph{Main File.}
%
% The main file is called |cdocsamp.tex|.
%
% Load the \textsf{childdoc} definitions and
% declare the filename for the main document:
%    \begin{macrocode}
\input{childdoc.def}
\childdocmain{}
%    \end{macrocode}

% Optional override for |\version| flag:
%    \begin{macrocode}
%%\ifchilddoc\else\providecommand{\version}{draft}\fi
%    \end{macrocode}

% Define the default values for the |\version| flag
% (|final| for the main file and |draft| for childs):
%    \begin{macrocode}
\ifchilddoc
\providecommand{\version}{draft}
\else
\providecommand{\version}{final}
\fi
%    \end{macrocode}

% Load the standard document class:
%    \begin{macrocode}
\documentclass[12pt]{article}
%    \end{macrocode}

% Start the document body:
%    \begin{macrocode}
\begin{document}
%    \end{macrocode}

% Declare a title page.
% Print title, part of document being processed and version flag:
%    \begin{macrocode}
\addtocounter{page}{-1}
\begin{center}
{\LARGE\bfseries{}childdoc example\par}
\vspace{1cm}
\ifchilddoc
\ifchilddocmanual part\else chapter\fi:
`\childdocname' of `\childdocjob'\par
\else
main document: `\childdocjob'\par
\fi
version: \version\par
\end{center}
\newpage
%    \end{macrocode}

% Manually include selected file,
% otherwise process as usual:
%    \begin{macrocode}
\ifchilddocmanual
\section*{part `\childdocname'}
\input{\childdocname}
\else
%    \end{macrocode}

% Include the two chapters:
%    \begin{macrocode}
\include{cdocsch1}
\include{cdocsch2}
%    \end{macrocode}

% Include the two parts unless only chapters should be displayed:
%    \begin{macrocode}
\ifchilddoc\else
\section{part three}
\input{cdocspt3}
\section{part four}
\input{cdocspt4}
\fi
%    \end{macrocode}

% Process as usual until here:
%    \begin{macrocode}
\fi
%    \end{macrocode}

% End of document body:
%    \begin{macrocode}
\end{document}
%    \end{macrocode}
%\iffalse
%</samplemain>
%\fi
%
% %%%%%%%%%%%%%%%%%%%%%%%%%%%%%%%%%%%%%%
% \paragraph{Chapter Include Files.}
%
% The include files are called |cdocsch1.tex| and |cdocsch2.tex|.
%
%\iffalse
%<*samplechap1|samplechap2>
%\fi

% Optional override for |\version| flag:
%    \begin{macrocode}
%%\providecommand{\version}{final}
%    \end{macrocode}

% Include the main document:
%    \begin{macrocode}
\input{childdoc.def}
\childdocof{cdocsamp}
%    \end{macrocode}

%\iffalse
%</samplechap1|samplechap2>
%\fi
%
%\iffalse
%<*samplechap1>
%\fi
% Some text for chapter 1:
%    \begin{macrocode}
\section{one}
some text in chapter one
%    \end{macrocode}

%\iffalse
%</samplechap1>
%\fi
% Some text for chapter 2:
%\iffalse
%<*samplechap2>
%\fi
%    \begin{macrocode}
\section{two}
more text in chapter two
%    \end{macrocode}

%\iffalse
%</samplechap2>
%\fi
%
% %%%%%%%%%%%%%%%%%%%%%%%%%%%%%%%%%%%%%%
% \paragraph{Part Include Files.}
%
% The include files are called |cdocspt3.tex| and |cdocspt4.tex|.
%
%\iffalse
%<*samplepart3|samplepart4>
%\fi

% Optional override for |\version| flag:
%    \begin{macrocode}
%%\providecommand{\version}{final}
%    \end{macrocode}

% Include the main document:
%    \begin{macrocode}
\input{childdoc.def}
\childdocby{cdocsamp}
%    \end{macrocode}

%\iffalse
%</samplepart3|samplepart4>
%\fi
%
%\iffalse
%<*samplepart3>
%\fi
% Some text for part 3:
%    \begin{macrocode}
some text in part three
%    \end{macrocode}

%\iffalse
%</samplepart3>
%\fi
% Some text for part 4:
%\iffalse
%<*samplepart4>
%\fi
%    \begin{macrocode}
more text in part four
%    \end{macrocode}

%\iffalse
%</samplepart4>
%\fi
%
% %%%%%%%%%%%%%%%%%%%%%%%%%%%%%%%%%%%%%%
% \paragraph{Forwarding for a Complete Draft.}
%
% The following forwarding file |cdocsdrf.tex|
% compiles the main document in draft mode:
%\iffalse
%<*sampledraft>
%\fi
%    \begin{macrocode}
\def\version{draft}
\input{childdoc.def}
\childdocforward{cdocsamp}
%    \end{macrocode}

%\iffalse
%</sampledraft>
%\fi
%
% %%%%%%%%%%%%%%%%%%%%%%%%%%%%%%%%%%%%%%
% \paragraph{Forwarding for Final Version of the Chapters.}
%
% The following forwarding files |cdocsfn1.tex| and |cdocsfn2.tex|
% (with identical content)
% compile the final versions of the child documents
% |cdocsch1.tex| and |cdocsch2.tex|, respectively:
%\iffalse
%<*samplefinal>
%\fi
%    \begin{macrocode}
\def\version{final}
\input{childdoc.def}
\childdocforwardprefix[cdocsamp]{cdocsfn}{cdocsch}
%    \end{macrocode}

%\iffalse
%</samplefinal>
%\fi
%
% %%%%%%%%%%%%%%%%%%%%%%%%%%%%%%%%%%%%%%
% \paragraph{Command Line Processing.}
%
% The following three command lines generate the output files
% |cdocscld|, |cdocscl1| and |cdocscl2|
% which should be identical to
% |cdocsdrf|, |cdocsch1| and |cdocsfn2|, respectively:
% \begin{center}
% \begin{tabular}{l}
% |latex -jobname cdocscld \|\\
% |  "\def\version{draft}\input{childdoc.def}\childdocforward{cdocsamp}"|\\
% |latex -jobname cdocscl1 \|\\
% |  "\input{childdoc.def}\childdocforward[cdocsamp]{cdocsch1}"|\\
% |latex -jobname cdocscl2 \|\\
% |  "\def\version{final}\input{childdoc.def}\childdocforward{cdocsch2}"|
% \end{tabular}
% \end{center}
% Note that the trailing backslash on each first line
% merely continues the input to the second line
% (for convenient cut ant paste).
% Furthermore, the command |latex| can be replaced by any
% of its alternative versions such as |pdflatex|.
%
% %%%%%%%%%%%%%%%%%%%%%%%%%%%%%%%%%%%%%%%%%%%%%%%%%%%%%%%%%%%%%%%%%%%%%%%%%%%%%%
% %%%%%%%%%%%%%%%%%%%%%%%%%%%%%%%%%%%%%%%%%%%%%%%%%%%%%%%%%%%%%%%%%%%%%%%%%%%%%%
% \section{Implementation}
%\iffalse
%<*package>
%\fi
%
% This section describes the definitions file |childdoc.def|.

% The definitions cannot be loaded using |\usepackage| or |\RequirePackage|
% which has a mechanism to prevent loading a style file more than once.
% When loading the definitions by means of |\input|
% multiple instances have to be prevented manually:
%\iffalse
%This code needs to be before the `\ProvidesFile' directive
%which is defined at the beginning of this file.
%Therefore it is also placed there and commented out here.
%</package>
%<*discard>
%\fi
%    \begin{macrocode}
\ifdefined\childdocmain\endinput\fi
%    \end{macrocode}
%\iffalse
%</discard>
%<*package>
%\fi
%
% \macro{\ifchilddoc}
% \macro{\ifchilddocmanual}
% The conditional |\ifchilddoc| tells whether a
% child (true) or main (false) document is being compiled.
% The conditional |\ifchilddocmanual| tells whether
% the |\includeonly| mechanism is used (false) or
% the selection of child files must be performed manually (true).
% The definitions initialise to false:
%    \begin{macrocode}
\newif\ifchilddoc
\newif\ifchilddocmanual
%    \end{macrocode}

% \macro{\childdocname}
% \macro{\childdocjob}
% The macro |\childdocname| stores the name of the main document
% to be compiled. The macro |\childdocjob| stores the name of
% the document on which the \LaTeX{} compiler was originally invoked.
% The content of |\jobname| cannot be compared
% to filenames specified in the source due to different catcodes.
% The following code rescans |\jobname|, stores the result
% in |\childdocname| and saves a copy in |\childdocjob|:
%    \begin{macrocode}
\edef\childdocname{\scantokens\expandafter{\jobname\noexpand}}
\let\childdocjob\childdocname
%    \end{macrocode}

% \macro{\childdocdisable}
% The macro |\childdocdisable| prevents the main file
% from being processed more than once.
% At this stage, the main document command |\childdocmain|
% is assumed to be called once again where it should do nothing.
% Any subsequent call to it should prevent
% a secondary processing of the main document
% It overwrites the forwarding commands
% |\childdocof| and |\childdocforward|
% with empty macros to prevent further inclusions of the main document:
%    \begin{macrocode}
\newcommand{\childdocdisable}
{
  \renewcommand{\childdocmain}[1]{\renewcommand{\childdocmain}[1]{\endinput}}
  \renewcommand{\childdocof}[1]{}
  \renewcommand{\childdocby}[2][]{}
  \renewcommand{\childdocforward}[2][]{}
  \renewcommand{\childdocdisable}{}
}
%    \end{macrocode}

% \macro{\childdocmain}
% The macro |\childdocmain| is to be called at the top of the main file
% with nothing or the main filename (without extension) as argument.
% First, it breaks loops.
% If the argument is not empty and does not match |\childdocname|
% (which is set by the first inclusion of |childdoc.def|),
% |\ifchilddoc| is set to true, |\includeonly| is applied to the child file
% and |\jobname| is set to the main file
% (for proper handling of |.aux| files):
%    \begin{macrocode}
\newcommand{\childdocmain}[1]
{
  \childdocdisable\childdocmain{}
  \if?#1?\else
    \begingroup
      \def\childdoctmp{#1}
      \ifx\childdoctmp\childdocname
        \def\childdoctmp{}
      \else
        \def\childdoctmp
        {
          \childdoctrue
          \includeonly{\childdocname}
          \def\childdocjob{#1}
          \def\jobname{#1}
        }
      \fi
      \expandafter
    \endgroup
    \childdoctmp
  \fi
}
%    \end{macrocode}

% \macro{\childdocof}
% The command |\childdocof| redirects
% compilation to the main file |#1|.
%    \begin{macrocode}
\newcommand{\childdocof}[1]
{
  \childdocdisable
  \childdoctrue
  \includeonly{\childdocname}
  \def\jobname{#1}
  \def\childdocjob{#1}
  \input{#1}
}
%    \end{macrocode}

% \macro{\childdocby}
% The command |\childdocby| ....
%    \begin{macrocode}
\newcommand{\childdocby}[2][]
{
  \childdocdisable
  \childdoctrue
  \childdocmanualtrue
  \if?#1?\else
    \def\jobname{#2}
  \fi
  \def\childdocjob{#2}
  \input{#2}
  \endinput
}
%    \end{macrocode}

% \macro{\childdocforward}
% The command |\childdocforward| redirects
% compilation to the main file or
% (if the optional argument is given) a child file.
% Parameters are set as if the main file
% or a child file starting with |\childdocof| was compiled.
% Then compilation is handed over to the main file:
%    \begin{macrocode}
\newcommand{\childdocforward}[2][]
{
  \begingroup
    \if?#1?
      \def\childdoctmp
      {
        \def\childdocname{#2}
        \def\childdocjob{#2}
        \def\jobname{#2}
        \input{#2}
        \endinput
      }
    \else
      \def\childdoctmp
      {
        \childdocdisable
        \def\childdocname{#2}
        \childdoctrue
        \includeonly{#2}
        \def\childdocjob{#1}
        \def\jobname{#1}
        \input{#1}
        \endinput
      }
    \fi
    \expandafter
  \endgroup
  \childdoctmp
}
%    \end{macrocode}

% \macro{\childdocforwardprefix}
% The command |\childdocforwardprefix| redirects
% compilation to the main or a child file by means of a pattern.
% The prefix |#1| in the current filename is replaced by |#2|
% and the suffix of the current filename is kept
% (it is assumed that the filename does not contain the substring `|~~~|'
% which is used as a delimiter).
% Compilation is handed over to the new file by |\childdocforward|:
%    \begin{macrocode}
\newcommand{\childdocforwardprefix}[3][]
{
  \begingroup
    \def\childdocextract #2##1~~~{\def\childdoctmp{\childdocforward[#1]{#3##1}}}
    \expandafter\childdocextract\childdocname~~~
    \expandafter
  \endgroup
  \childdoctmp
}
%    \end{macrocode}

% \macro{\childdoc}
% The deprecated macro |\childdoc| is a legacy version of |\childdocmain|:
%    \begin{macrocode}
\newcommand{\childdoc}{\childdocmain}
%    \end{macrocode}

% \macro{\childdocredirect}
% The deprecated macro |\childdocredirect| is a legacy version
% of |\childdocforward| and |\childdocforwardprefix|:
%    \begin{macrocode}
\newcommand{\childdocredirect}[2][]
{
  \begingroup
    \if?#1?
      \def\childdoctmp{\childdocforward{#2}}
    \else
      \def\childdoctmp{\childdocforwardprefix{#1}{#2}}
    \fi
    \expandafter
  \endgroup
  \childdoctmp
}
%    \end{macrocode}

%\iffalse
%</package>
%\fi
%
\endinput
|\\
|\childdocby{|\textit{main}|}|\\
\end{tabular}
\end{center}
%
Both forms have slightly different effects as described above.
The main file is prepared as usual, see \secref{sec:include}.

%%%%%%%%%%%%%%%%%%%%%%%%%%%%%%%%%%%%%%%%%%%%%%%%%%%%%%%%%%%%%%%%%%%%%%%%%%%%%%%%
\subsection{Legacy Detection}
\label{sec:detection}

The directive |\childdocmain| in the main file can detect
whether the complete document or merely a child is to be compiled
even without using the directive |\childdocof|.
This method is deprecated because it is less robust
and there is no compelling reason to use it;
it is merely provided for backward compatibility
and it may be removed in future versions.

If the detection mechanism is to be used,
it is mandatory to correctly specify
the filename of the main file as the argument of |\childdocmain|:
%
\begin{center}
\begin{tabular}{l}
|% \iffalse
%
% childdoc.dtx Copyright (C) 2017-2018 Niklas Beisert
%
% This work may be distributed and/or modified under the
% conditions of the LaTeX Project Public License, either version 1.3
% of this license or (at your option) any later version.
% The latest version of this license is in
%   http://www.latex-project.org/lppl.txt
% and version 1.3 or later is part of all distributions of LaTeX
% version 2005/12/01 or later.
%
% This work has the LPPL maintenance status `maintained'.
%
% The Current Maintainer of this work is Niklas Beisert.
%
% This work consists of the files childdoc.dtx and childdoc.ins
% and the derived files childdoc.def and cdocsamp.tex with
% cdocsch1.tex, cdocsch2.tex, cdocsdrf.tex, cdocsfn1.tex, cdocsfn2.tex.
%
%<package>\ifdefined\childdocmain\endinput\fi
%<package>\ProvidesFile{childdoc.def}[2018/12/30 v2.0 child document driver]
%<samplemain>\ProvidesFile{cdocsamp.tex}[2018/12/30 v2.0 sample for childdoc]
%<*driver>
%\ProvidesFile{childdoc.drv}[2018/12/30 v2.0 childdoc reference manual file]
\PassOptionsToClass{10pt,a4paper}{article}
\documentclass{ltxdoc}

\usepackage[margin=35mm]{geometry}
\usepackage{hyperref}
\usepackage{hyperxmp}
\usepackage[usenames]{color}

\hypersetup{colorlinks=true}
\hypersetup{pdfstartview=FitH}
\hypersetup{pdfpagemode=UseNone}
\hypersetup{pdfsource={}}
\hypersetup{pdflang={en-UK}}
\hypersetup{pdfcopyright={Copyright 2017-2018 Niklas Beisert.
  This work may be distributed and/or modified under the
  conditions of the LaTeX Project Public License, either version 1.3
  of this license or (at your option) any later version.}}
\hypersetup{pdflicenseurl={http://www.latex-project.org/lppl.txt}}
\hypersetup{pdfcontactaddress={ETH Zurich, ITP, HIT K,
  Wolfgang-Pauli-Strasse 27}}
\hypersetup{pdfcontactpostcode={8093}}
\hypersetup{pdfcontactcity={Zurich}}
\hypersetup{pdfcontactcountry={Switzerland}}
\hypersetup{pdfcontactemail={nbeisert@itp.phys.ethz.ch}}
\hypersetup{pdfcontacturl={http://people.phys.ethz.ch/\xmptilde nbeisert/}}

\newcommand{\secref}[1]{\hyperref[#1]{section \ref*{#1}}}

\parskip1ex
\parindent0pt
\let\olditemize\itemize
\def\itemize{\olditemize\parskip0pt}

\begin{document}

\title{The \textsf{childdoc} Package}
\hypersetup{pdftitle={The childdoc Package}}
\author{Niklas Beisert\\[2ex]
  Institut f\"ur Theoretische Physik\\
  Eidgen\"ossische Technische Hochschule Z\"urich\\
  Wolfgang-Pauli-Strasse 27, 8093 Z\"urich, Switzerland\\[1ex]
  \href{mailto:nbeisert@itp.phys.ethz.ch}
  {\texttt{nbeisert@itp.phys.ethz.ch}}}
\hypersetup{pdfauthor={Niklas Beisert}}
\hypersetup{pdfsubject={Manual for the LaTeX2e Package childdoc}}
\date{30 December 2018, \textsf{v2.0}}
\maketitle

\begin{abstract}\noindent
\textsf{childdoc} is a \LaTeXe{} package
that enables the direct compilation
of document sections included by |\include|
to individual files.
\end{abstract}

\begingroup
\parskip0ex
\tableofcontents
\endgroup

%%%%%%%%%%%%%%%%%%%%%%%%%%%%%%%%%%%%%%%%%%%%%%%%%%%%%%%%%%%%%%%%%%%%%%%%%%%%%%%%
%%%%%%%%%%%%%%%%%%%%%%%%%%%%%%%%%%%%%%%%%%%%%%%%%%%%%%%%%%%%%%%%%%%%%%%%%%%%%%%%
\section{Introduction}

\LaTeX{} provides a mechanism to structure a large document (such as a book)
into a main file and several child files (containing the chapters)
using the |\include| command.
This mechanism is beneficial for documents
which span hundreds of pages in order to
make the source file(s) more manageable.
Moreover, compilation can be restricted to
selected child files by means of the |\includeonly| command.
The latter feature can be used to reduce the compilation time while editing
(this was significantly more useful in the earlier days of \LaTeX{})
or to generate a smaller document which is easier to navigate.
Another application of |\includeonly| is to generate
documents consisting of selected parts of the complete document.

However, there are a few drawbacks of the plain |\include| mechanism:
\begin{itemize}
\item
The child files cannot be compiled on their own,
they can only be compiled via the main file.
A naive editing environment
(such as a text editor with an option
to have the current file processed by \LaTeX)
may require one to switch to the main file before compiling;
attempting to compile the child file produces errors.
\item
The main file must be modified (each time)
to adjust the |\includeonly| command
to the present needs. This easily leaves the main file in a messy state.
\item
The generated document will always carry the filename
of the main document. This is inconvenient if
several child files are to be compiled and
to be kept for distribution.
\end{itemize}

The present package provides a simple interface
to make child files individually compilable by \LaTeX{}.
Compiling a child file then has the same effect as compiling
the main file with an |\includeonly| command
to select the appropriate child.
Moreover the generated document will carry the name of the child
rather than the main file.
This resolves all three above issues.

This feature is meant to make the editing of books,
thesis documents and lecture notes somewhat more convenient.
However, the package can also be used efficiently for
composing a series of documents (such as exercise sheets)
which are typically distributed individually.
It then assists the author in generating the individual documents
(potentially in different versions)
as well as a document containing the collected series.
Another application is in developing style files
or other kinds of included material
where compilation of the style file could redirect
to a sample or test file.

%%%%%%%%%%%%%%%%%%%%%%%%%%%%%%%%%%%%%%%%%%%%%%%%%%%%%%%%%%%%%%%%%%%%%%%%%%%%%%%%
%%%%%%%%%%%%%%%%%%%%%%%%%%%%%%%%%%%%%%%%%%%%%%%%%%%%%%%%%%%%%%%%%%%%%%%%%%%%%%%%
\section{Usage}

First of all, the package \textsf{childdoc} is \emph{not} a standard
\LaTeXe{} |.sty| style file! Therefore it needs to be invoked in
a non-standard way.

%%%%%%%%%%%%%%%%%%%%%%%%%%%%%%%%%%%%%%%%%%%%%%%%%%%%%%%%%%%%%%%%%%%%%%%%%%%%%%%%
\subsection{Included Files}
\label{sec:include}

%%%%%%%%%%%%%%%%%%%%%%%%%%%%%%%%%%%%%%%%
\DescribeMacro{\childdocmain}
To use the package, add the commands
\begin{center}
\begin{tabular}{l}
|\input{childdoc.def}|\\
|\childdocmain{}|\\
\end{tabular}
\end{center}
at the very top of the main \LaTeX{} file,
in particular \emph{before} the |\documentclass| statement!
The argument of |\childdocmain| should be left empty
(but it must be present).

%%%%%%%%%%%%%%%%%%%%%%%%%%%%%%%%%%%%%%%%
\DescribeMacro{\childdocof}
Furthermore, add the commands
\begin{center}
\begin{tabular}{l}
|\input{childdoc.def}|\\
|\childdocof{|\textit{main}|}|\\
\end{tabular}
\end{center}
at the top of every child file \textit{child}
which is included by |\include{|\textit{child}|}|
from within the main file
(or at least for those files to be compiled individually).
The argument \textit{main} must be the filename of the main file.

There are a couple of
considerations in setting up the main and child documents:

%%%%%%%%%%%%%%%%%%%%%%%%%%%%%%%%%%%%%%%%
\paragraph{Restrictions.}

Please note the following restrictions:
\begin{itemize}
\item
|\childdocmain| must be called with one argument \textit{main}
to ensure compatibility with earlier version of the package.
It must either be empty (|\childdocmain{}|)
or precisely match the filename of the main file in which it is specified.
See \secref{sec:detection} for further information.
\item
The filename \textit{main} must be specified without the |.tex| extension.
\item
The filename \textit{main} is case sensitive
(even in case-insensitive file systems)
due to internal string comparison.
\item
The argument \textit{main} should be fully expanded, it cannot be a macro.
\item
Subdirectories and special characters should be avoided in filenames.
\item
The command |\childdocmain{|\textit{main}|}| must be followed by a whitespace.
It should not be followed immediately by another command
or by a comment mark `|%|'.
This is because the \TeX{} parser reads the token immediately following
the argument of |\childdocmain| and puts it
at the beginning of every child section;
however, a white\-space is ignored.
\end{itemize}

%%%%%%%%%%%%%%%%%%%%%%%%%%%%%%%%%%%%%%%%
\paragraph{Content of Main File.}

It is advisable to place all content in the child files included by |\include|.
Any output contained in the main file will appear in all child documents
unless suppressed manually;
it cannot be suppressed automatically by the |\includeonly| directive
and thus should normally be avoided.
A method to include some content in the main file
by means of conditional processing is described in \secref{sec:conditional}.

%%%%%%%%%%%%%%%%%%%%%%%%%%%%%%%%%%%%%%%%
\paragraph{Page Numbering.}

When only a part of the document is compiled,
the appropriate numbering of pages
(as well as other status parameters)
is determined from the |.aux| files.
The latter contain information from previous passes.
However this information needs to propagate through
all intermediate child documents.
Therefore the page numbering in child documents may well
be inconsistent until the complete document is compiled at least once.

A useful (if unconventional) way to always ensure a consistent
page numbering is to restart the numbering in each child document
and denote the pages by `\textit{child}|.|\textit{page}'
where \textit{child} represents the chapter/section number of the child file.
This can be achieved by the command
|\numberwithin{page}{|\textit{child}|}|
of the \textsf{amsmath} package
where \textit{child} can be |chapter| or |section|
depending on the chosen structuring.
Alternatively, one can modify the macro |\thepage| appropriately
and reset the counter |page| at the start of each child file.

%%%%%%%%%%%%%%%%%%%%%%%%%%%%%%%%%%%%%%%%%%%%%%%%%%%%%%%%%%%%%%%%%%%%%%%%%%%%%%%%
\subsection{Conditional Processing}
\label{sec:conditional}

The package provides a mechanism to compile different versions
of a document. To customise the versions further some conditional processing
can come in handy to distinguish which version is being compiled.
The package provides two macros to describe the compilation context:

%%%%%%%%%%%%%%%%%%%%%%%%%%%%%%%%%%%%%%%%
\DescribeMacro{\ifchilddoc}
The conditional |\ifchilddoc| distinguishes between the compilation of
child documents and the main document:
%
\begin{center}
|\ifchilddoc |\textit{child-code}| |[|\||else |\textit{main-code}]| \||fi|
\end{center}

%%%%%%%%%%%%%%%%%%%%%%%%%%%%%%%%%%%%%%%%
\DescribeMacro{\childdocname}
\DescribeMacro{\childdocjob}
The macro |\childdocname| contains the filename (without extension)
of the main or child file being processed.
Note that |\childdocjob| will always contain the name of the main file.

%%%%%%%%%%%%%%%%%%%%%%%%%%%%%%%%%%%%%%%%
\paragraph{Title Page.}

Conditional processing can be used to include a title or banner page
in the main document when proper precautions are taken.
Importantly, the code in the main file should ensure that the page counter
(as well as other status parameters which are stored in the |.aux| files)
takes the same value after the conditional processing.
Otherwise the page numbers may take divergent values
depending on which part is compiled.

For example, a title page could be declared by:
%
\begin{center}
\begin{tabular}{l}
|\ifchilddoc\||else|\\
|\addtocounter{page}{-1}|\\
\textit{code for title page}\\
|\newpage|\\
|\||fi|
\end{tabular}
\end{center}
%
A banner page for the child documents can be generated by:
%
\begin{center}
\begin{tabular}{l}
|\ifchilddoc|\\
|\addtocounter{page}{-1}|\\
\textit{code for banner page}\\
|\newpage|\\
|\||fi|
\end{tabular}
\end{center}
%
Here one could write a message such as:
\begin{center}
|This is the part \childdocname{} of \childdocjob{}.|
\end{center}

%%%%%%%%%%%%%%%%%%%%%%%%%%%%%%%%%%%%%%%%%%%%%%%%%%%%%%%%%%%%%%%%%%%%%%%%%%%%%%%%
\subsection{Flags}
\label{sec:flags}

The package makes it easy to generate different versions
of the main or child documents.
To this end compilation flags can be defined
and assigned different default values.
They will be particularly useful in conjunction
with the forwarding mechanism described in \secref{sec:forward}.

For example, it may be useful to have a flag |\version|
which can be set to |draft| or |final|.
The document source will contain some conditional code
depending on the value of |\version|.
Suppose further, the flag should default to |final| for the main file
and to |draft| for child files
which is a natural assignment for editing the document.
This is achieved by placing the following code
in the preamble of the main document
(below the |\childdocmain| directive):
%
\begin{center}
\begin{tabular}{l}
|\ifchilddoc|\\
|\providecommand{\version}{draft}|\\
|\||else|\\
|\providecommand{\version}{final}|\\
|\||fi|
\end{tabular}
\end{center}
%
The definition by |\providecommand| makes sure
that previous definitions are not overwritten.
Further statements |\providecommand{\version}{...}|
can thus be added before the above code to override it.

For the main file, one might add a line
(between |\childdocmain| and the above block)
%
\begin{center}
|%\ifchilddoc\||else\providecommand{\version}{draft}\||fi|
\end{center}
%
which can be uncommented to produce a draft version.
Likewise one can add a line to the very top of a child file
(above the |\childdocof{|\textit{main}|}| directive)
%
\begin{center}
|%\providecommand{\version}{final}|
\end{center}
%
which can be uncommented to produce the final version of this child document.

%%%%%%%%%%%%%%%%%%%%%%%%%%%%%%%%%%%%%%%%%%%%%%%%%%%%%%%%%%%%%%%%%%%%%%%%%%%%%%%%
\subsection{Forwarding}
\label{sec:forward}

Different versions of the main or child documents
using compilation flags as described in \secref{sec:flags}
can be (permanently) stored in different files
for convenient compilation, viewing and distribution.
To this end, the package defines a command
to pass on compilation to a different file:

%%%%%%%%%%%%%%%%%%%%%%%%%%%%%%%%%%%%%%%%
\DescribeMacro{\childdocforward}
The command |\childdocforward| redirects processing to
another source file:
%
\begin{center}
\begin{tabular}{l}
|\input{childdoc.def}|\\
|\childdocforward[|\textit{main}|]{|\textit{dest}|}|\\
\end{tabular}
\end{center}
%
The argument \textit{dest} is the destination file
(without extension).
It should be the main file or one of the child files.
Note that further \textsf{childdoc} directives
such as |\childdocof| and |\childdocforward|
in the indicated file will be processed in this form.
The optional argument \textit{main}
passes on directly to the main file \textit{main}
while pretending to compile the child \textit{dest}.
This form behaves as if \textit{dest}
issues |\childdocof{|\textit{main}|}| right away,
and no further \textsf{childdoc} directives will be processed.

%%%%%%%%%%%%%%%%%%%%%%%%%%%%%%%%%%%%%%%%
\DescribeMacro{\...prefix}
In the alternative form |\childdocforwardprefix|,
%
\begin{center}
\begin{tabular}{l}
|\input{childdoc.def}|\\
|\childdocforwardprefix[|\textit{main}|]{|\textit{prefix}|}{|\textit{dest}|}|
\end{tabular}
\end{center}
%
the destination file is determined by a pattern
depending on the current file:
To make this work, the current file must be called
`{\textit{prefix}\hspace{0.2em}\textit{suffix}}'
with \textit{prefix} matching precisely the argument.
Processing is then passed on to the file
`{\textit{dest}\hspace{0.2em}\textit{suffix}}'.
Surely, the same effect is achieved by
directly specifying the
argument `{\textit{dest}\hspace{0.2em}\textit{suffix}}'
in the first form.
However, that requires to set up a different file
for each child. With the alternative form of the command
all these files can have exactly the same content
which simplifies setting them up and maintaining them.

For example, the following file |draft.tex|
with a compilation flag |\version| as described in \secref{sec:flags}
compiles the main document as a draft:
%
\begin{center}
\begin{tabular}{l}
|\def\version{draft}|\\
|\input{childdoc.def}|\\
|\childdocforward{|\textit{main}|}|
\end{tabular}
\end{center}
%
Likewise, the following files |final|\textit{nn}|.tex|
compile the final version of the child document
|child|\textit{nn}|.tex|:
%
\begin{center}
\begin{tabular}{l}
|\def\version{final}|\\
|\input{childdoc.def}|\\
|\childdocforwardprefix{final}{child}|
\end{tabular}
\end{center}
%

Note that when several versions of a main file and/or of each child file
are to be generated, it may be convenient to set up a |Makefile| or
shell script to automatise the process.

%%%%%%%%%%%%%%%%%%%%%%%%%%%%%%%%%%%%%%%%%%%%%%%%%%%%%%%%%%%%%%%%%%%%%%%%%%%%%%%%
\subsection{Command Line Processing}
\label{sec:commandline}

The effect of redirection files can also be achieved by invoking
the \LaTeX{} compiler with a more elaborate command line.
Most conveniently this should be done as part
of a shell script or a |Makefile|.

When using \textsf{childdoc} in the main file, the following
command lines effectively perform a redirection
(note that depending on the shell being used,
backslashes may have to be doubled: `|\|' $\to$ `|\\|'):
%
\begin{center}
|... -jobname "|\textit{target}|" |\\|"|[\textit{flags}]%
|\input{childdoc.def}\childdocforward[|\textit{main}|]{|\textit{dest}|}"|
\end{center}
%
Here \textit{target} is the name of the output file,
\textit{main} is the name of the main file
and \textit{dest} is the name of the main or child file to be processed
(all filenames without extensions).
The optional argument \textit{main} can be omitted
if \textit{main} matches \textit{dest}.
Optionally, compilation \textit{flags} can be defined via |\def| commands.
This command line makes the \TeX{} engine believe
it is compiling the file \textit{target}
whose content is specified as the latter parameter.
The provided code then forwards the processing to
\textit{main} or \textit{dest} as described in \secref{sec:forward}.

%%%%%%%%%%%%%%%%%%%%%%%%%%%%%%%%%%%%%%%%%%%%%%%%%%%%%%%%%%%%%%%%%%%%%%%%%%%%%%%%
\subsection{Include by Input}
\label{sec:input}

Including child documents by |\include| has some restrictions by design.
Most notably, the content of a child document always occupies
its own set of pages; pages cannot be shared between child documents.
Usually, this behaviour makes perfect sense
because each child document contain an essential part of the document.
However, in some situations it may be desirable to compose
a document from a collection of parts
without having mandatory page breaks between then.
For this case, the package
provides a mechanism to include parts
by |\input| which can also be processed individually.
However, by construction this mechanism
requires manual handling of the content to be output.

%%%%%%%%%%%%%%%%%%%%%%%%%%%%%%%%%%%%%%%%
\DescribeMacro{\ifchilddocmanual}
The main file should be prepared as usual, see \secref{sec:include}.
However, the document body must make a distinction
between processing of an individual part and of the main document, e.g.:
%
\begin{center}
\begin{tabular}{l}
|\ifchilddocmanual|\\
|\input{\childdocname}|\\
|\||else|\\
\textit{document body with }|\input{|\textit{part}|}|\\
|\||fi|
\end{tabular}
\end{center}
%
The conditional |\ifchilddocmanual| is true whenever
a part to be included by |\input| is being compiled,
and the name of the part is stored in |\childdocname|.

%%%%%%%%%%%%%%%%%%%%%%%%%%%%%%%%%%%%%%%%
\DescribeMacro{\childdocby}
Each part to be included by |\input| should start with:
%
\begin{center}
\begin{tabular}{l}
|\input{childdoc.def}|\\
|\childdocby{|\textit{main}|}|\\
\end{tabular}
\end{center}
%
The directive |\childdocby| is similar to |\childdocof|
described in \secref{sec:include},
but the subsequent selection of content must be done manually.
To that end, both |\ifchilddoc| and |\ifchilddocmanual|
will be true upon processing of a part,
and the name of the part is stored in |\childdocname|.
Note that |\jobname| will be set to the filename of the current part
so that each part receives an individual |.aux| file
that does not interfere with the |.aux| file(s) of the main document.
This behaviour can be altered by the alternative form
|\childdocby[*]{|\textit{main}|}| (with a non-empty optional argument)
which uses the |.aux| file of the main document
by setting |\jobname| to \textit{main}.

%%%%%%%%%%%%%%%%%%%%%%%%%%%%%%%%%%%%%%%%%%%%%%%%%%%%%%%%%%%%%%%%%%%%%%%%%%%%%%%%
\subsection{Driver Development}
\label{sec:driver}

The \textsf{childdoc} mechanism can also be use for the development
of definition files such as \LaTeX{} styles or classes.
This case differs from the above setup with multiple parts
included by |\include| in that no |\includeonly| should be invoked.
This can be achieved by starting the include file
(before |\ProvidesPackage|) with:
%
\begin{center}
\begin{tabular}{l}
|\input{childdoc.def}|\\
|\childdocforward{|\textit{main}|}|\\
\end{tabular}
\end{center}
%
or alternatively with:
%
\begin{center}
\begin{tabular}{l}
|\input{childdoc.def}|\\
|\childdocby{|\textit{main}|}|\\
\end{tabular}
\end{center}
%
Both forms have slightly different effects as described above.
The main file is prepared as usual, see \secref{sec:include}.

%%%%%%%%%%%%%%%%%%%%%%%%%%%%%%%%%%%%%%%%%%%%%%%%%%%%%%%%%%%%%%%%%%%%%%%%%%%%%%%%
\subsection{Legacy Detection}
\label{sec:detection}

The directive |\childdocmain| in the main file can detect
whether the complete document or merely a child is to be compiled
even without using the directive |\childdocof|.
This method is deprecated because it is less robust
and there is no compelling reason to use it;
it is merely provided for backward compatibility
and it may be removed in future versions.

If the detection mechanism is to be used,
it is mandatory to correctly specify
the filename of the main file as the argument of |\childdocmain|:
%
\begin{center}
\begin{tabular}{l}
|\input{childdoc.def}|\\
|\childdocmain{|\textit{main}|}|\\
\end{tabular}
\end{center}
%
If |\jobname| does not match the argument \textit{main} of |\childdocmain|,
it is assumed that |\jobname| points to the child file to be compiled.
When using |\childdocmain| with the main file specified as argument,
it suffices to start a child file
with just |\input{|\textit{main}|}|
without loading of the package and using |\childdocof|.
If instead all processing is done
with the appropriate \textsf{childdoc} directives,
the argument of \textit{main} of |\childdocmain| can be empty.

An alternative version of the command line processing described
in \secref{sec:commandline} using the detection mechanism reads:
%
\begin{center}
|... -jobname "|\textit{target}|" "|[\textit{flags}]%
[|\def\jobname{|\textit{dest}|}|]|\input{|\textit{main}|}"|
\end{center}

%%%%%%%%%%%%%%%%%%%%%%%%%%%%%%%%%%%%%%%%%%%%%%%%%%%%%%%%%%%%%%%%%%%%%%%%%%%%%%%%
\subsection{Manual Code}
\label{sec:manual}

In case one cannot be certain whether the definitions file |childdoc.def|
is installed on the target \TeX{} distribution
and one prefers not to ship it,
it is conceivable to paste a few relevant commands into the sources.

To that end, drop all statements |\input{childdoc.def}|
and perform the replacements as outlined below.
Instead of |\childdocmain{|\textit{main}|}| add the following code
to the top of the main file:
%
\begin{center}
\begin{tabular}{l}
|\||ifdefined\childdocname\endinput\||fi\newif\ifchilddoc|\\
|\edef\childdocname{\scantokens\expandafter{\jobname\noexpand}}|\\
|\def\childdocmain{|\textit{main}|}\||ifx\childdocmain\childdocname\||else|\\
|\childdoctrue\includeonly{\childdocname}\let\jobname\childdocmain\||fi|\\
\end{tabular}
\end{center}
%
Instead of |\childdocof{|\textit{main}|}| just include the main file
at the top of each child file:
%
\begin{center}
|\input{|\textit{main}|}|
\end{center}
%
A simple redirection |\childdocforward{|\textit{dest}|}| is achieved by:
%
\begin{center}
|\def\jobname{|\textit{dest}|}\input{\jobname}|
\end{center}
%
The redirection with prefix
|\childdocforwardprefix[|\textit{prefix}|]{|\textit{dest}|}|
is accomplished by:
%
\begin{center}
\begin{tabular}{l}
|{\edef\jobname{\scantokens\expandafter{\jobname\noexpand}}|\\
|\def\redirectjob |\textit{prefix}|#1~~~{\gdef\jobname{|\textit{dest}|#1}}|\\
|\expandafter\redirectjob\jobname~~~}\input{\jobname}|
\end{tabular}
\end{center}

In an alternative approach,
child documents can be compiled by a specific command line
without additional code or specific definitions:
%
\begin{center}
|... -jobname "|\textit{target}|" "|[\textit{flags}]%
|\includeonly{|\textit{dest}|}\input{|\textit{main}|}"|
\end{center}
%

%%%%%%%%%%%%%%%%%%%%%%%%%%%%%%%%%%%%%%%%%%%%%%%%%%%%%%%%%%%%%%%%%%%%%%%%%%%%%%%%
%%%%%%%%%%%%%%%%%%%%%%%%%%%%%%%%%%%%%%%%%%%%%%%%%%%%%%%%%%%%%%%%%%%%%%%%%%%%%%%%
\section{Information}

%%%%%%%%%%%%%%%%%%%%%%%%%%%%%%%%%%%%%%%%%%%%%%%%%%%%%%%%%%%%%%%%%%%%%%%%%%%%%%%%
\subsection{Copyright}

Copyright \copyright{} 2017--2018 Niklas Beisert

This work may be distributed and/or modified under the
conditions of the \LaTeX{} Project Public License, either version 1.3
of this license or (at your option) any later version.
The latest version of this license is in
  \url{http://www.latex-project.org/lppl.txt}
and version 1.3 or later is part of all distributions of \LaTeX{}
version 2005/12/01 or later.

This work has the LPPL maintenance status `maintained'.

The Current Maintainer of this work is Niklas Beisert.

This work consists of the files |README.txt|, |childdoc.ins| and |childdoc.dtx|
as well as the derived files |childdoc.def|, |cdocsamp.tex|
with |cdocsch1.tex|, |cdocsch2.tex|, |cdocspt3.tex|, |cdocspt4.tex|,
|cdocsdrf.tex|, |cdocsfn1.tex|, |cdocsfn2.tex|
as well as |childdoc.pdf|.

%%%%%%%%%%%%%%%%%%%%%%%%%%%%%%%%%%%%%%%%%%%%%%%%%%%%%%%%%%%%%%%%%%%%%%%%%%%%%%%%
\subsection{Files and Installation}

The package consists of the files:
%
\begin{center}
\begin{tabular}{ll}
    |README.txt|   & readme file \\
    |childdoc.ins| & installation file \\
    |childdoc.dtx| & source file \\
    |childdoc.def| & definition file \\
    |cdocsamp.tex| & sample main file \\
    |cdocsch1.tex| & sample include file \\
    |cdocsch2.tex| & sample include file \\
    |cdocspt3.tex| & sample part file \\
    |cdocspt4.tex| & sample part file \\
    |cdocsdrf.tex| & sample redirection file \\
    |cdocsfn1.tex| & sample redirection file \\
    |cdocsfn2.tex| & sample redirection file \\
    |childdoc.pdf| & manual
\end{tabular}
\end{center}
%
The distribution consists of the files
|README.txt|, |childdoc.ins| and |childdoc.dtx|.
%
\begin{itemize}
\item
Run (pdf)\LaTeX{} on |childdoc.dtx|
to compile the manual |childdoc.pdf| (this file).
\item
Run \LaTeX{} on |childdoc.ins| to create the definitions file |childdoc.def|
and the sample |cdocsamp.tex| with include files
|cdocsch1.tex|, |cdocsch2.tex|, |cdocspt3.tex|, |cdocspt4.tex|,
|cdocsdrf.tex|, |cdocsfn1.tex|, |cdocsfn2.tex|.
Then copy the file |childdoc.def| to an appropriate directory of your \LaTeX{}
distribution, e.g.\ \textit{texmf-root}|/tex/latex/childdoc|.
\end{itemize}

%%%%%%%%%%%%%%%%%%%%%%%%%%%%%%%%%%%%%%%%%%%%%%%%%%%%%%%%%%%%%%%%%%%%%%%%%%%%%%%%
\subsection{Related CTAN Packages}

There are several other packages which offer a similar functionality:
%
\begin{itemize}
\item
The packages
\href{http://ctan.org/pkg/docmute}{\textsf{docmute}},
\href{http://ctan.org/pkg/includex}{\textsf{includex}} and
\href{http://ctan.org/pkg/standalone}{\textsf{standalone}}
provide commands to include only the document body of
a child file thus allowing both files to be compiled individually.
\item
The packages \href{http://ctan.org/pkg/subdocs}{\textsf{subdocs}}
and \href{http://ctan.org/pkg/subfiles}{\textsf{subfiles}}
provide structures in which the main and child documents can be
encapsulated and allowing them to be compiled individually.
The inclusion mechanism is different from the conventional |\include|.
\item
The package \href{http://ctan.org/pkg/combine}{\textsf{combine}}
is an elaborate solution to combine several documents into one.
\end{itemize}
%
See also the CTAN topic \href{http://ctan.org/topic/subdocs}{\textsf{subdocs}}
for further related packages.
The present package differs from the above solutions in that
a document structure constructed with the conventional |\include| mechanism
just needs two extra commands at the top of every file
such that all constituent files can be compiled individually.

%%%%%%%%%%%%%%%%%%%%%%%%%%%%%%%%%%%%%%%%%%%%%%%%%%%%%%%%%%%%%%%%%%%%%%%%%%%%%%%%
%\subsection{Feature Suggestions}
%
%The following is a list of features which may be useful for future
%versions of this package:
%%
%\begin{itemize}
%\item
%\ldots
%\end{itemize}

%%%%%%%%%%%%%%%%%%%%%%%%%%%%%%%%%%%%%%%%%%%%%%%%%%%%%%%%%%%%%%%%%%%%%%%%%%%%%%%%
\subsection{Revision History}

%%%%%%%%%%%%%%%%%%%%%%%%%%%%%%%%%%%%%%%%
\paragraph{v2.0:} 2018/12/30

\begin{itemize}
\item
immediate forward processing
\item
added |\childdocby| mechanism
\item
manual restructured
\end{itemize}

%%%%%%%%%%%%%%%%%%%%%%%%%%%%%%%%%%%%%%%%
\paragraph{v1.6:} 2018/01/17

\begin{itemize}
\item
application for development of include files
\item
corrections to manual
\end{itemize}

%%%%%%%%%%%%%%%%%%%%%%%%%%%%%%%%%%%%%%%%
\paragraph{v1.5:} 2017/05/21

\begin{itemize}
\item
more complete structuring introduced
\item
|\childdocof| introduced
\item
|\childdoc| renamed to |\childdocmain|
\item
|\childredirect| renamed to |\childdocforward| and |\childdocforwardprefix|
and functionality expanded
\end{itemize}

%%%%%%%%%%%%%%%%%%%%%%%%%%%%%%%%%%%%%%%%
\paragraph{v1.0:} 2017/04/27

\begin{itemize}
\item
manual and install package
\item
first version published on CTAN
\end{itemize}

%%%%%%%%%%%%%%%%%%%%%%%%%%%%%%%%%%%%%%%%
\paragraph{v0.6:} 2017/04/26

\begin{itemize}
\item
redirection mechanism added
\end{itemize}

%%%%%%%%%%%%%%%%%%%%%%%%%%%%%%%%%%%%%%%%
\paragraph{v0.5:} 2017/04/26

\begin{itemize}
\item
functionality in definition file
\end{itemize}


%%%%%%%%%%%%%%%%%%%%%%%%%%%%%%%%%%%%%%%%%%%%%%%%%%%%%%%%%%%%%%%%%%%%%%%%%%%%%%%%
%%%%%%%%%%%%%%%%%%%%%%%%%%%%%%%%%%%%%%%%%%%%%%%%%%%%%%%%%%%%%%%%%%%%%%%%%%%%%%%%
%%%%%%%%%%%%%%%%%%%%%%%%%%%%%%%%%%%%%%%%%%%%%%%%%%%%%%%%%%%%%%%%%%%%%%%%%%%%%%%%
\appendix

\settowidth\MacroIndent{\rmfamily\scriptsize 000\ }

 \DocInput{childdoc.dtx}

\end{document}
%</driver>
% \fi
%
% %%%%%%%%%%%%%%%%%%%%%%%%%%%%%%%%%%%%%%%%%%%%%%%%%%%%%%%%%%%%%%%%%%%%%%%%%%%%%%
% %%%%%%%%%%%%%%%%%%%%%%%%%%%%%%%%%%%%%%%%%%%%%%%%%%%%%%%%%%%%%%%%%%%%%%%%%%%%%%
% \section{Sample}
%\iffalse
%<*samplemain>
%\fi
%
% The following presents a sample document
% with two chapters, two parts, a title page,
% a compile flag as well as three forwarding files to set the flag.
% It consists of eight |.tex| files:
% \begin{center}
% \begin{tabular}{ll}
% |cdocsamp.tex|&main file\\
% |cdocsch1.tex|&include file for chapter 1\\
% |cdocsch2.tex|&include file for chapter 2\\
% |cdocspt3.tex|&include file for part 3\\
% |cdocspt4.tex|&include file for part 4\\
% |cdocsdrf.tex|&forwarding file for main file in draft mode\\
% |cdocsfi1.tex|&forwarding file for final version of chapter 1\\
% |cdocsfi2.tex|&forwarding file for final version of chapter 2\\
% \end{tabular}
% \end{center}
% Each of the eight files can be compiled directly by the \LaTeX{} compiler.
%
% %%%%%%%%%%%%%%%%%%%%%%%%%%%%%%%%%%%%%%
% \paragraph{Main File.}
%
% The main file is called |cdocsamp.tex|.
%
% Load the \textsf{childdoc} definitions and
% declare the filename for the main document:
%    \begin{macrocode}
\input{childdoc.def}
\childdocmain{}
%    \end{macrocode}

% Optional override for |\version| flag:
%    \begin{macrocode}
%%\ifchilddoc\else\providecommand{\version}{draft}\fi
%    \end{macrocode}

% Define the default values for the |\version| flag
% (|final| for the main file and |draft| for childs):
%    \begin{macrocode}
\ifchilddoc
\providecommand{\version}{draft}
\else
\providecommand{\version}{final}
\fi
%    \end{macrocode}

% Load the standard document class:
%    \begin{macrocode}
\documentclass[12pt]{article}
%    \end{macrocode}

% Start the document body:
%    \begin{macrocode}
\begin{document}
%    \end{macrocode}

% Declare a title page.
% Print title, part of document being processed and version flag:
%    \begin{macrocode}
\addtocounter{page}{-1}
\begin{center}
{\LARGE\bfseries{}childdoc example\par}
\vspace{1cm}
\ifchilddoc
\ifchilddocmanual part\else chapter\fi:
`\childdocname' of `\childdocjob'\par
\else
main document: `\childdocjob'\par
\fi
version: \version\par
\end{center}
\newpage
%    \end{macrocode}

% Manually include selected file,
% otherwise process as usual:
%    \begin{macrocode}
\ifchilddocmanual
\section*{part `\childdocname'}
\input{\childdocname}
\else
%    \end{macrocode}

% Include the two chapters:
%    \begin{macrocode}
\include{cdocsch1}
\include{cdocsch2}
%    \end{macrocode}

% Include the two parts unless only chapters should be displayed:
%    \begin{macrocode}
\ifchilddoc\else
\section{part three}
\input{cdocspt3}
\section{part four}
\input{cdocspt4}
\fi
%    \end{macrocode}

% Process as usual until here:
%    \begin{macrocode}
\fi
%    \end{macrocode}

% End of document body:
%    \begin{macrocode}
\end{document}
%    \end{macrocode}
%\iffalse
%</samplemain>
%\fi
%
% %%%%%%%%%%%%%%%%%%%%%%%%%%%%%%%%%%%%%%
% \paragraph{Chapter Include Files.}
%
% The include files are called |cdocsch1.tex| and |cdocsch2.tex|.
%
%\iffalse
%<*samplechap1|samplechap2>
%\fi

% Optional override for |\version| flag:
%    \begin{macrocode}
%%\providecommand{\version}{final}
%    \end{macrocode}

% Include the main document:
%    \begin{macrocode}
\input{childdoc.def}
\childdocof{cdocsamp}
%    \end{macrocode}

%\iffalse
%</samplechap1|samplechap2>
%\fi
%
%\iffalse
%<*samplechap1>
%\fi
% Some text for chapter 1:
%    \begin{macrocode}
\section{one}
some text in chapter one
%    \end{macrocode}

%\iffalse
%</samplechap1>
%\fi
% Some text for chapter 2:
%\iffalse
%<*samplechap2>
%\fi
%    \begin{macrocode}
\section{two}
more text in chapter two
%    \end{macrocode}

%\iffalse
%</samplechap2>
%\fi
%
% %%%%%%%%%%%%%%%%%%%%%%%%%%%%%%%%%%%%%%
% \paragraph{Part Include Files.}
%
% The include files are called |cdocspt3.tex| and |cdocspt4.tex|.
%
%\iffalse
%<*samplepart3|samplepart4>
%\fi

% Optional override for |\version| flag:
%    \begin{macrocode}
%%\providecommand{\version}{final}
%    \end{macrocode}

% Include the main document:
%    \begin{macrocode}
\input{childdoc.def}
\childdocby{cdocsamp}
%    \end{macrocode}

%\iffalse
%</samplepart3|samplepart4>
%\fi
%
%\iffalse
%<*samplepart3>
%\fi
% Some text for part 3:
%    \begin{macrocode}
some text in part three
%    \end{macrocode}

%\iffalse
%</samplepart3>
%\fi
% Some text for part 4:
%\iffalse
%<*samplepart4>
%\fi
%    \begin{macrocode}
more text in part four
%    \end{macrocode}

%\iffalse
%</samplepart4>
%\fi
%
% %%%%%%%%%%%%%%%%%%%%%%%%%%%%%%%%%%%%%%
% \paragraph{Forwarding for a Complete Draft.}
%
% The following forwarding file |cdocsdrf.tex|
% compiles the main document in draft mode:
%\iffalse
%<*sampledraft>
%\fi
%    \begin{macrocode}
\def\version{draft}
\input{childdoc.def}
\childdocforward{cdocsamp}
%    \end{macrocode}

%\iffalse
%</sampledraft>
%\fi
%
% %%%%%%%%%%%%%%%%%%%%%%%%%%%%%%%%%%%%%%
% \paragraph{Forwarding for Final Version of the Chapters.}
%
% The following forwarding files |cdocsfn1.tex| and |cdocsfn2.tex|
% (with identical content)
% compile the final versions of the child documents
% |cdocsch1.tex| and |cdocsch2.tex|, respectively:
%\iffalse
%<*samplefinal>
%\fi
%    \begin{macrocode}
\def\version{final}
\input{childdoc.def}
\childdocforwardprefix[cdocsamp]{cdocsfn}{cdocsch}
%    \end{macrocode}

%\iffalse
%</samplefinal>
%\fi
%
% %%%%%%%%%%%%%%%%%%%%%%%%%%%%%%%%%%%%%%
% \paragraph{Command Line Processing.}
%
% The following three command lines generate the output files
% |cdocscld|, |cdocscl1| and |cdocscl2|
% which should be identical to
% |cdocsdrf|, |cdocsch1| and |cdocsfn2|, respectively:
% \begin{center}
% \begin{tabular}{l}
% |latex -jobname cdocscld \|\\
% |  "\def\version{draft}\input{childdoc.def}\childdocforward{cdocsamp}"|\\
% |latex -jobname cdocscl1 \|\\
% |  "\input{childdoc.def}\childdocforward[cdocsamp]{cdocsch1}"|\\
% |latex -jobname cdocscl2 \|\\
% |  "\def\version{final}\input{childdoc.def}\childdocforward{cdocsch2}"|
% \end{tabular}
% \end{center}
% Note that the trailing backslash on each first line
% merely continues the input to the second line
% (for convenient cut ant paste).
% Furthermore, the command |latex| can be replaced by any
% of its alternative versions such as |pdflatex|.
%
% %%%%%%%%%%%%%%%%%%%%%%%%%%%%%%%%%%%%%%%%%%%%%%%%%%%%%%%%%%%%%%%%%%%%%%%%%%%%%%
% %%%%%%%%%%%%%%%%%%%%%%%%%%%%%%%%%%%%%%%%%%%%%%%%%%%%%%%%%%%%%%%%%%%%%%%%%%%%%%
% \section{Implementation}
%\iffalse
%<*package>
%\fi
%
% This section describes the definitions file |childdoc.def|.

% The definitions cannot be loaded using |\usepackage| or |\RequirePackage|
% which has a mechanism to prevent loading a style file more than once.
% When loading the definitions by means of |\input|
% multiple instances have to be prevented manually:
%\iffalse
%This code needs to be before the `\ProvidesFile' directive
%which is defined at the beginning of this file.
%Therefore it is also placed there and commented out here.
%</package>
%<*discard>
%\fi
%    \begin{macrocode}
\ifdefined\childdocmain\endinput\fi
%    \end{macrocode}
%\iffalse
%</discard>
%<*package>
%\fi
%
% \macro{\ifchilddoc}
% \macro{\ifchilddocmanual}
% The conditional |\ifchilddoc| tells whether a
% child (true) or main (false) document is being compiled.
% The conditional |\ifchilddocmanual| tells whether
% the |\includeonly| mechanism is used (false) or
% the selection of child files must be performed manually (true).
% The definitions initialise to false:
%    \begin{macrocode}
\newif\ifchilddoc
\newif\ifchilddocmanual
%    \end{macrocode}

% \macro{\childdocname}
% \macro{\childdocjob}
% The macro |\childdocname| stores the name of the main document
% to be compiled. The macro |\childdocjob| stores the name of
% the document on which the \LaTeX{} compiler was originally invoked.
% The content of |\jobname| cannot be compared
% to filenames specified in the source due to different catcodes.
% The following code rescans |\jobname|, stores the result
% in |\childdocname| and saves a copy in |\childdocjob|:
%    \begin{macrocode}
\edef\childdocname{\scantokens\expandafter{\jobname\noexpand}}
\let\childdocjob\childdocname
%    \end{macrocode}

% \macro{\childdocdisable}
% The macro |\childdocdisable| prevents the main file
% from being processed more than once.
% At this stage, the main document command |\childdocmain|
% is assumed to be called once again where it should do nothing.
% Any subsequent call to it should prevent
% a secondary processing of the main document
% It overwrites the forwarding commands
% |\childdocof| and |\childdocforward|
% with empty macros to prevent further inclusions of the main document:
%    \begin{macrocode}
\newcommand{\childdocdisable}
{
  \renewcommand{\childdocmain}[1]{\renewcommand{\childdocmain}[1]{\endinput}}
  \renewcommand{\childdocof}[1]{}
  \renewcommand{\childdocby}[2][]{}
  \renewcommand{\childdocforward}[2][]{}
  \renewcommand{\childdocdisable}{}
}
%    \end{macrocode}

% \macro{\childdocmain}
% The macro |\childdocmain| is to be called at the top of the main file
% with nothing or the main filename (without extension) as argument.
% First, it breaks loops.
% If the argument is not empty and does not match |\childdocname|
% (which is set by the first inclusion of |childdoc.def|),
% |\ifchilddoc| is set to true, |\includeonly| is applied to the child file
% and |\jobname| is set to the main file
% (for proper handling of |.aux| files):
%    \begin{macrocode}
\newcommand{\childdocmain}[1]
{
  \childdocdisable\childdocmain{}
  \if?#1?\else
    \begingroup
      \def\childdoctmp{#1}
      \ifx\childdoctmp\childdocname
        \def\childdoctmp{}
      \else
        \def\childdoctmp
        {
          \childdoctrue
          \includeonly{\childdocname}
          \def\childdocjob{#1}
          \def\jobname{#1}
        }
      \fi
      \expandafter
    \endgroup
    \childdoctmp
  \fi
}
%    \end{macrocode}

% \macro{\childdocof}
% The command |\childdocof| redirects
% compilation to the main file |#1|.
%    \begin{macrocode}
\newcommand{\childdocof}[1]
{
  \childdocdisable
  \childdoctrue
  \includeonly{\childdocname}
  \def\jobname{#1}
  \def\childdocjob{#1}
  \input{#1}
}
%    \end{macrocode}

% \macro{\childdocby}
% The command |\childdocby| ....
%    \begin{macrocode}
\newcommand{\childdocby}[2][]
{
  \childdocdisable
  \childdoctrue
  \childdocmanualtrue
  \if?#1?\else
    \def\jobname{#2}
  \fi
  \def\childdocjob{#2}
  \input{#2}
  \endinput
}
%    \end{macrocode}

% \macro{\childdocforward}
% The command |\childdocforward| redirects
% compilation to the main file or
% (if the optional argument is given) a child file.
% Parameters are set as if the main file
% or a child file starting with |\childdocof| was compiled.
% Then compilation is handed over to the main file:
%    \begin{macrocode}
\newcommand{\childdocforward}[2][]
{
  \begingroup
    \if?#1?
      \def\childdoctmp
      {
        \def\childdocname{#2}
        \def\childdocjob{#2}
        \def\jobname{#2}
        \input{#2}
        \endinput
      }
    \else
      \def\childdoctmp
      {
        \childdocdisable
        \def\childdocname{#2}
        \childdoctrue
        \includeonly{#2}
        \def\childdocjob{#1}
        \def\jobname{#1}
        \input{#1}
        \endinput
      }
    \fi
    \expandafter
  \endgroup
  \childdoctmp
}
%    \end{macrocode}

% \macro{\childdocforwardprefix}
% The command |\childdocforwardprefix| redirects
% compilation to the main or a child file by means of a pattern.
% The prefix |#1| in the current filename is replaced by |#2|
% and the suffix of the current filename is kept
% (it is assumed that the filename does not contain the substring `|~~~|'
% which is used as a delimiter).
% Compilation is handed over to the new file by |\childdocforward|:
%    \begin{macrocode}
\newcommand{\childdocforwardprefix}[3][]
{
  \begingroup
    \def\childdocextract #2##1~~~{\def\childdoctmp{\childdocforward[#1]{#3##1}}}
    \expandafter\childdocextract\childdocname~~~
    \expandafter
  \endgroup
  \childdoctmp
}
%    \end{macrocode}

% \macro{\childdoc}
% The deprecated macro |\childdoc| is a legacy version of |\childdocmain|:
%    \begin{macrocode}
\newcommand{\childdoc}{\childdocmain}
%    \end{macrocode}

% \macro{\childdocredirect}
% The deprecated macro |\childdocredirect| is a legacy version
% of |\childdocforward| and |\childdocforwardprefix|:
%    \begin{macrocode}
\newcommand{\childdocredirect}[2][]
{
  \begingroup
    \if?#1?
      \def\childdoctmp{\childdocforward{#2}}
    \else
      \def\childdoctmp{\childdocforwardprefix{#1}{#2}}
    \fi
    \expandafter
  \endgroup
  \childdoctmp
}
%    \end{macrocode}

%\iffalse
%</package>
%\fi
%
\endinput
|\\
|\childdocmain{|\textit{main}|}|\\
\end{tabular}
\end{center}
%
If |\jobname| does not match the argument \textit{main} of |\childdocmain|,
it is assumed that |\jobname| points to the child file to be compiled.
When using |\childdocmain| with the main file specified as argument,
it suffices to start a child file
with just |\input{|\textit{main}|}|
without loading of the package and using |\childdocof|.
If instead all processing is done
with the appropriate \textsf{childdoc} directives,
the argument of \textit{main} of |\childdocmain| can be empty.

An alternative version of the command line processing described
in \secref{sec:commandline} using the detection mechanism reads:
%
\begin{center}
|... -jobname "|\textit{target}|" "|[\textit{flags}]%
[|\def\jobname{|\textit{dest}|}|]|\input{|\textit{main}|}"|
\end{center}

%%%%%%%%%%%%%%%%%%%%%%%%%%%%%%%%%%%%%%%%%%%%%%%%%%%%%%%%%%%%%%%%%%%%%%%%%%%%%%%%
\subsection{Manual Code}
\label{sec:manual}

In case one cannot be certain whether the definitions file |childdoc.def|
is installed on the target \TeX{} distribution
and one prefers not to ship it,
it is conceivable to paste a few relevant commands into the sources.

To that end, drop all statements |% \iffalse
%
% childdoc.dtx Copyright (C) 2017-2018 Niklas Beisert
%
% This work may be distributed and/or modified under the
% conditions of the LaTeX Project Public License, either version 1.3
% of this license or (at your option) any later version.
% The latest version of this license is in
%   http://www.latex-project.org/lppl.txt
% and version 1.3 or later is part of all distributions of LaTeX
% version 2005/12/01 or later.
%
% This work has the LPPL maintenance status `maintained'.
%
% The Current Maintainer of this work is Niklas Beisert.
%
% This work consists of the files childdoc.dtx and childdoc.ins
% and the derived files childdoc.def and cdocsamp.tex with
% cdocsch1.tex, cdocsch2.tex, cdocsdrf.tex, cdocsfn1.tex, cdocsfn2.tex.
%
%<package>\ifdefined\childdocmain\endinput\fi
%<package>\ProvidesFile{childdoc.def}[2018/12/30 v2.0 child document driver]
%<samplemain>\ProvidesFile{cdocsamp.tex}[2018/12/30 v2.0 sample for childdoc]
%<*driver>
%\ProvidesFile{childdoc.drv}[2018/12/30 v2.0 childdoc reference manual file]
\PassOptionsToClass{10pt,a4paper}{article}
\documentclass{ltxdoc}

\usepackage[margin=35mm]{geometry}
\usepackage{hyperref}
\usepackage{hyperxmp}
\usepackage[usenames]{color}

\hypersetup{colorlinks=true}
\hypersetup{pdfstartview=FitH}
\hypersetup{pdfpagemode=UseNone}
\hypersetup{pdfsource={}}
\hypersetup{pdflang={en-UK}}
\hypersetup{pdfcopyright={Copyright 2017-2018 Niklas Beisert.
  This work may be distributed and/or modified under the
  conditions of the LaTeX Project Public License, either version 1.3
  of this license or (at your option) any later version.}}
\hypersetup{pdflicenseurl={http://www.latex-project.org/lppl.txt}}
\hypersetup{pdfcontactaddress={ETH Zurich, ITP, HIT K,
  Wolfgang-Pauli-Strasse 27}}
\hypersetup{pdfcontactpostcode={8093}}
\hypersetup{pdfcontactcity={Zurich}}
\hypersetup{pdfcontactcountry={Switzerland}}
\hypersetup{pdfcontactemail={nbeisert@itp.phys.ethz.ch}}
\hypersetup{pdfcontacturl={http://people.phys.ethz.ch/\xmptilde nbeisert/}}

\newcommand{\secref}[1]{\hyperref[#1]{section \ref*{#1}}}

\parskip1ex
\parindent0pt
\let\olditemize\itemize
\def\itemize{\olditemize\parskip0pt}

\begin{document}

\title{The \textsf{childdoc} Package}
\hypersetup{pdftitle={The childdoc Package}}
\author{Niklas Beisert\\[2ex]
  Institut f\"ur Theoretische Physik\\
  Eidgen\"ossische Technische Hochschule Z\"urich\\
  Wolfgang-Pauli-Strasse 27, 8093 Z\"urich, Switzerland\\[1ex]
  \href{mailto:nbeisert@itp.phys.ethz.ch}
  {\texttt{nbeisert@itp.phys.ethz.ch}}}
\hypersetup{pdfauthor={Niklas Beisert}}
\hypersetup{pdfsubject={Manual for the LaTeX2e Package childdoc}}
\date{30 December 2018, \textsf{v2.0}}
\maketitle

\begin{abstract}\noindent
\textsf{childdoc} is a \LaTeXe{} package
that enables the direct compilation
of document sections included by |\include|
to individual files.
\end{abstract}

\begingroup
\parskip0ex
\tableofcontents
\endgroup

%%%%%%%%%%%%%%%%%%%%%%%%%%%%%%%%%%%%%%%%%%%%%%%%%%%%%%%%%%%%%%%%%%%%%%%%%%%%%%%%
%%%%%%%%%%%%%%%%%%%%%%%%%%%%%%%%%%%%%%%%%%%%%%%%%%%%%%%%%%%%%%%%%%%%%%%%%%%%%%%%
\section{Introduction}

\LaTeX{} provides a mechanism to structure a large document (such as a book)
into a main file and several child files (containing the chapters)
using the |\include| command.
This mechanism is beneficial for documents
which span hundreds of pages in order to
make the source file(s) more manageable.
Moreover, compilation can be restricted to
selected child files by means of the |\includeonly| command.
The latter feature can be used to reduce the compilation time while editing
(this was significantly more useful in the earlier days of \LaTeX{})
or to generate a smaller document which is easier to navigate.
Another application of |\includeonly| is to generate
documents consisting of selected parts of the complete document.

However, there are a few drawbacks of the plain |\include| mechanism:
\begin{itemize}
\item
The child files cannot be compiled on their own,
they can only be compiled via the main file.
A naive editing environment
(such as a text editor with an option
to have the current file processed by \LaTeX)
may require one to switch to the main file before compiling;
attempting to compile the child file produces errors.
\item
The main file must be modified (each time)
to adjust the |\includeonly| command
to the present needs. This easily leaves the main file in a messy state.
\item
The generated document will always carry the filename
of the main document. This is inconvenient if
several child files are to be compiled and
to be kept for distribution.
\end{itemize}

The present package provides a simple interface
to make child files individually compilable by \LaTeX{}.
Compiling a child file then has the same effect as compiling
the main file with an |\includeonly| command
to select the appropriate child.
Moreover the generated document will carry the name of the child
rather than the main file.
This resolves all three above issues.

This feature is meant to make the editing of books,
thesis documents and lecture notes somewhat more convenient.
However, the package can also be used efficiently for
composing a series of documents (such as exercise sheets)
which are typically distributed individually.
It then assists the author in generating the individual documents
(potentially in different versions)
as well as a document containing the collected series.
Another application is in developing style files
or other kinds of included material
where compilation of the style file could redirect
to a sample or test file.

%%%%%%%%%%%%%%%%%%%%%%%%%%%%%%%%%%%%%%%%%%%%%%%%%%%%%%%%%%%%%%%%%%%%%%%%%%%%%%%%
%%%%%%%%%%%%%%%%%%%%%%%%%%%%%%%%%%%%%%%%%%%%%%%%%%%%%%%%%%%%%%%%%%%%%%%%%%%%%%%%
\section{Usage}

First of all, the package \textsf{childdoc} is \emph{not} a standard
\LaTeXe{} |.sty| style file! Therefore it needs to be invoked in
a non-standard way.

%%%%%%%%%%%%%%%%%%%%%%%%%%%%%%%%%%%%%%%%%%%%%%%%%%%%%%%%%%%%%%%%%%%%%%%%%%%%%%%%
\subsection{Included Files}
\label{sec:include}

%%%%%%%%%%%%%%%%%%%%%%%%%%%%%%%%%%%%%%%%
\DescribeMacro{\childdocmain}
To use the package, add the commands
\begin{center}
\begin{tabular}{l}
|\input{childdoc.def}|\\
|\childdocmain{}|\\
\end{tabular}
\end{center}
at the very top of the main \LaTeX{} file,
in particular \emph{before} the |\documentclass| statement!
The argument of |\childdocmain| should be left empty
(but it must be present).

%%%%%%%%%%%%%%%%%%%%%%%%%%%%%%%%%%%%%%%%
\DescribeMacro{\childdocof}
Furthermore, add the commands
\begin{center}
\begin{tabular}{l}
|\input{childdoc.def}|\\
|\childdocof{|\textit{main}|}|\\
\end{tabular}
\end{center}
at the top of every child file \textit{child}
which is included by |\include{|\textit{child}|}|
from within the main file
(or at least for those files to be compiled individually).
The argument \textit{main} must be the filename of the main file.

There are a couple of
considerations in setting up the main and child documents:

%%%%%%%%%%%%%%%%%%%%%%%%%%%%%%%%%%%%%%%%
\paragraph{Restrictions.}

Please note the following restrictions:
\begin{itemize}
\item
|\childdocmain| must be called with one argument \textit{main}
to ensure compatibility with earlier version of the package.
It must either be empty (|\childdocmain{}|)
or precisely match the filename of the main file in which it is specified.
See \secref{sec:detection} for further information.
\item
The filename \textit{main} must be specified without the |.tex| extension.
\item
The filename \textit{main} is case sensitive
(even in case-insensitive file systems)
due to internal string comparison.
\item
The argument \textit{main} should be fully expanded, it cannot be a macro.
\item
Subdirectories and special characters should be avoided in filenames.
\item
The command |\childdocmain{|\textit{main}|}| must be followed by a whitespace.
It should not be followed immediately by another command
or by a comment mark `|%|'.
This is because the \TeX{} parser reads the token immediately following
the argument of |\childdocmain| and puts it
at the beginning of every child section;
however, a white\-space is ignored.
\end{itemize}

%%%%%%%%%%%%%%%%%%%%%%%%%%%%%%%%%%%%%%%%
\paragraph{Content of Main File.}

It is advisable to place all content in the child files included by |\include|.
Any output contained in the main file will appear in all child documents
unless suppressed manually;
it cannot be suppressed automatically by the |\includeonly| directive
and thus should normally be avoided.
A method to include some content in the main file
by means of conditional processing is described in \secref{sec:conditional}.

%%%%%%%%%%%%%%%%%%%%%%%%%%%%%%%%%%%%%%%%
\paragraph{Page Numbering.}

When only a part of the document is compiled,
the appropriate numbering of pages
(as well as other status parameters)
is determined from the |.aux| files.
The latter contain information from previous passes.
However this information needs to propagate through
all intermediate child documents.
Therefore the page numbering in child documents may well
be inconsistent until the complete document is compiled at least once.

A useful (if unconventional) way to always ensure a consistent
page numbering is to restart the numbering in each child document
and denote the pages by `\textit{child}|.|\textit{page}'
where \textit{child} represents the chapter/section number of the child file.
This can be achieved by the command
|\numberwithin{page}{|\textit{child}|}|
of the \textsf{amsmath} package
where \textit{child} can be |chapter| or |section|
depending on the chosen structuring.
Alternatively, one can modify the macro |\thepage| appropriately
and reset the counter |page| at the start of each child file.

%%%%%%%%%%%%%%%%%%%%%%%%%%%%%%%%%%%%%%%%%%%%%%%%%%%%%%%%%%%%%%%%%%%%%%%%%%%%%%%%
\subsection{Conditional Processing}
\label{sec:conditional}

The package provides a mechanism to compile different versions
of a document. To customise the versions further some conditional processing
can come in handy to distinguish which version is being compiled.
The package provides two macros to describe the compilation context:

%%%%%%%%%%%%%%%%%%%%%%%%%%%%%%%%%%%%%%%%
\DescribeMacro{\ifchilddoc}
The conditional |\ifchilddoc| distinguishes between the compilation of
child documents and the main document:
%
\begin{center}
|\ifchilddoc |\textit{child-code}| |[|\||else |\textit{main-code}]| \||fi|
\end{center}

%%%%%%%%%%%%%%%%%%%%%%%%%%%%%%%%%%%%%%%%
\DescribeMacro{\childdocname}
\DescribeMacro{\childdocjob}
The macro |\childdocname| contains the filename (without extension)
of the main or child file being processed.
Note that |\childdocjob| will always contain the name of the main file.

%%%%%%%%%%%%%%%%%%%%%%%%%%%%%%%%%%%%%%%%
\paragraph{Title Page.}

Conditional processing can be used to include a title or banner page
in the main document when proper precautions are taken.
Importantly, the code in the main file should ensure that the page counter
(as well as other status parameters which are stored in the |.aux| files)
takes the same value after the conditional processing.
Otherwise the page numbers may take divergent values
depending on which part is compiled.

For example, a title page could be declared by:
%
\begin{center}
\begin{tabular}{l}
|\ifchilddoc\||else|\\
|\addtocounter{page}{-1}|\\
\textit{code for title page}\\
|\newpage|\\
|\||fi|
\end{tabular}
\end{center}
%
A banner page for the child documents can be generated by:
%
\begin{center}
\begin{tabular}{l}
|\ifchilddoc|\\
|\addtocounter{page}{-1}|\\
\textit{code for banner page}\\
|\newpage|\\
|\||fi|
\end{tabular}
\end{center}
%
Here one could write a message such as:
\begin{center}
|This is the part \childdocname{} of \childdocjob{}.|
\end{center}

%%%%%%%%%%%%%%%%%%%%%%%%%%%%%%%%%%%%%%%%%%%%%%%%%%%%%%%%%%%%%%%%%%%%%%%%%%%%%%%%
\subsection{Flags}
\label{sec:flags}

The package makes it easy to generate different versions
of the main or child documents.
To this end compilation flags can be defined
and assigned different default values.
They will be particularly useful in conjunction
with the forwarding mechanism described in \secref{sec:forward}.

For example, it may be useful to have a flag |\version|
which can be set to |draft| or |final|.
The document source will contain some conditional code
depending on the value of |\version|.
Suppose further, the flag should default to |final| for the main file
and to |draft| for child files
which is a natural assignment for editing the document.
This is achieved by placing the following code
in the preamble of the main document
(below the |\childdocmain| directive):
%
\begin{center}
\begin{tabular}{l}
|\ifchilddoc|\\
|\providecommand{\version}{draft}|\\
|\||else|\\
|\providecommand{\version}{final}|\\
|\||fi|
\end{tabular}
\end{center}
%
The definition by |\providecommand| makes sure
that previous definitions are not overwritten.
Further statements |\providecommand{\version}{...}|
can thus be added before the above code to override it.

For the main file, one might add a line
(between |\childdocmain| and the above block)
%
\begin{center}
|%\ifchilddoc\||else\providecommand{\version}{draft}\||fi|
\end{center}
%
which can be uncommented to produce a draft version.
Likewise one can add a line to the very top of a child file
(above the |\childdocof{|\textit{main}|}| directive)
%
\begin{center}
|%\providecommand{\version}{final}|
\end{center}
%
which can be uncommented to produce the final version of this child document.

%%%%%%%%%%%%%%%%%%%%%%%%%%%%%%%%%%%%%%%%%%%%%%%%%%%%%%%%%%%%%%%%%%%%%%%%%%%%%%%%
\subsection{Forwarding}
\label{sec:forward}

Different versions of the main or child documents
using compilation flags as described in \secref{sec:flags}
can be (permanently) stored in different files
for convenient compilation, viewing and distribution.
To this end, the package defines a command
to pass on compilation to a different file:

%%%%%%%%%%%%%%%%%%%%%%%%%%%%%%%%%%%%%%%%
\DescribeMacro{\childdocforward}
The command |\childdocforward| redirects processing to
another source file:
%
\begin{center}
\begin{tabular}{l}
|\input{childdoc.def}|\\
|\childdocforward[|\textit{main}|]{|\textit{dest}|}|\\
\end{tabular}
\end{center}
%
The argument \textit{dest} is the destination file
(without extension).
It should be the main file or one of the child files.
Note that further \textsf{childdoc} directives
such as |\childdocof| and |\childdocforward|
in the indicated file will be processed in this form.
The optional argument \textit{main}
passes on directly to the main file \textit{main}
while pretending to compile the child \textit{dest}.
This form behaves as if \textit{dest}
issues |\childdocof{|\textit{main}|}| right away,
and no further \textsf{childdoc} directives will be processed.

%%%%%%%%%%%%%%%%%%%%%%%%%%%%%%%%%%%%%%%%
\DescribeMacro{\...prefix}
In the alternative form |\childdocforwardprefix|,
%
\begin{center}
\begin{tabular}{l}
|\input{childdoc.def}|\\
|\childdocforwardprefix[|\textit{main}|]{|\textit{prefix}|}{|\textit{dest}|}|
\end{tabular}
\end{center}
%
the destination file is determined by a pattern
depending on the current file:
To make this work, the current file must be called
`{\textit{prefix}\hspace{0.2em}\textit{suffix}}'
with \textit{prefix} matching precisely the argument.
Processing is then passed on to the file
`{\textit{dest}\hspace{0.2em}\textit{suffix}}'.
Surely, the same effect is achieved by
directly specifying the
argument `{\textit{dest}\hspace{0.2em}\textit{suffix}}'
in the first form.
However, that requires to set up a different file
for each child. With the alternative form of the command
all these files can have exactly the same content
which simplifies setting them up and maintaining them.

For example, the following file |draft.tex|
with a compilation flag |\version| as described in \secref{sec:flags}
compiles the main document as a draft:
%
\begin{center}
\begin{tabular}{l}
|\def\version{draft}|\\
|\input{childdoc.def}|\\
|\childdocforward{|\textit{main}|}|
\end{tabular}
\end{center}
%
Likewise, the following files |final|\textit{nn}|.tex|
compile the final version of the child document
|child|\textit{nn}|.tex|:
%
\begin{center}
\begin{tabular}{l}
|\def\version{final}|\\
|\input{childdoc.def}|\\
|\childdocforwardprefix{final}{child}|
\end{tabular}
\end{center}
%

Note that when several versions of a main file and/or of each child file
are to be generated, it may be convenient to set up a |Makefile| or
shell script to automatise the process.

%%%%%%%%%%%%%%%%%%%%%%%%%%%%%%%%%%%%%%%%%%%%%%%%%%%%%%%%%%%%%%%%%%%%%%%%%%%%%%%%
\subsection{Command Line Processing}
\label{sec:commandline}

The effect of redirection files can also be achieved by invoking
the \LaTeX{} compiler with a more elaborate command line.
Most conveniently this should be done as part
of a shell script or a |Makefile|.

When using \textsf{childdoc} in the main file, the following
command lines effectively perform a redirection
(note that depending on the shell being used,
backslashes may have to be doubled: `|\|' $\to$ `|\\|'):
%
\begin{center}
|... -jobname "|\textit{target}|" |\\|"|[\textit{flags}]%
|\input{childdoc.def}\childdocforward[|\textit{main}|]{|\textit{dest}|}"|
\end{center}
%
Here \textit{target} is the name of the output file,
\textit{main} is the name of the main file
and \textit{dest} is the name of the main or child file to be processed
(all filenames without extensions).
The optional argument \textit{main} can be omitted
if \textit{main} matches \textit{dest}.
Optionally, compilation \textit{flags} can be defined via |\def| commands.
This command line makes the \TeX{} engine believe
it is compiling the file \textit{target}
whose content is specified as the latter parameter.
The provided code then forwards the processing to
\textit{main} or \textit{dest} as described in \secref{sec:forward}.

%%%%%%%%%%%%%%%%%%%%%%%%%%%%%%%%%%%%%%%%%%%%%%%%%%%%%%%%%%%%%%%%%%%%%%%%%%%%%%%%
\subsection{Include by Input}
\label{sec:input}

Including child documents by |\include| has some restrictions by design.
Most notably, the content of a child document always occupies
its own set of pages; pages cannot be shared between child documents.
Usually, this behaviour makes perfect sense
because each child document contain an essential part of the document.
However, in some situations it may be desirable to compose
a document from a collection of parts
without having mandatory page breaks between then.
For this case, the package
provides a mechanism to include parts
by |\input| which can also be processed individually.
However, by construction this mechanism
requires manual handling of the content to be output.

%%%%%%%%%%%%%%%%%%%%%%%%%%%%%%%%%%%%%%%%
\DescribeMacro{\ifchilddocmanual}
The main file should be prepared as usual, see \secref{sec:include}.
However, the document body must make a distinction
between processing of an individual part and of the main document, e.g.:
%
\begin{center}
\begin{tabular}{l}
|\ifchilddocmanual|\\
|\input{\childdocname}|\\
|\||else|\\
\textit{document body with }|\input{|\textit{part}|}|\\
|\||fi|
\end{tabular}
\end{center}
%
The conditional |\ifchilddocmanual| is true whenever
a part to be included by |\input| is being compiled,
and the name of the part is stored in |\childdocname|.

%%%%%%%%%%%%%%%%%%%%%%%%%%%%%%%%%%%%%%%%
\DescribeMacro{\childdocby}
Each part to be included by |\input| should start with:
%
\begin{center}
\begin{tabular}{l}
|\input{childdoc.def}|\\
|\childdocby{|\textit{main}|}|\\
\end{tabular}
\end{center}
%
The directive |\childdocby| is similar to |\childdocof|
described in \secref{sec:include},
but the subsequent selection of content must be done manually.
To that end, both |\ifchilddoc| and |\ifchilddocmanual|
will be true upon processing of a part,
and the name of the part is stored in |\childdocname|.
Note that |\jobname| will be set to the filename of the current part
so that each part receives an individual |.aux| file
that does not interfere with the |.aux| file(s) of the main document.
This behaviour can be altered by the alternative form
|\childdocby[*]{|\textit{main}|}| (with a non-empty optional argument)
which uses the |.aux| file of the main document
by setting |\jobname| to \textit{main}.

%%%%%%%%%%%%%%%%%%%%%%%%%%%%%%%%%%%%%%%%%%%%%%%%%%%%%%%%%%%%%%%%%%%%%%%%%%%%%%%%
\subsection{Driver Development}
\label{sec:driver}

The \textsf{childdoc} mechanism can also be use for the development
of definition files such as \LaTeX{} styles or classes.
This case differs from the above setup with multiple parts
included by |\include| in that no |\includeonly| should be invoked.
This can be achieved by starting the include file
(before |\ProvidesPackage|) with:
%
\begin{center}
\begin{tabular}{l}
|\input{childdoc.def}|\\
|\childdocforward{|\textit{main}|}|\\
\end{tabular}
\end{center}
%
or alternatively with:
%
\begin{center}
\begin{tabular}{l}
|\input{childdoc.def}|\\
|\childdocby{|\textit{main}|}|\\
\end{tabular}
\end{center}
%
Both forms have slightly different effects as described above.
The main file is prepared as usual, see \secref{sec:include}.

%%%%%%%%%%%%%%%%%%%%%%%%%%%%%%%%%%%%%%%%%%%%%%%%%%%%%%%%%%%%%%%%%%%%%%%%%%%%%%%%
\subsection{Legacy Detection}
\label{sec:detection}

The directive |\childdocmain| in the main file can detect
whether the complete document or merely a child is to be compiled
even without using the directive |\childdocof|.
This method is deprecated because it is less robust
and there is no compelling reason to use it;
it is merely provided for backward compatibility
and it may be removed in future versions.

If the detection mechanism is to be used,
it is mandatory to correctly specify
the filename of the main file as the argument of |\childdocmain|:
%
\begin{center}
\begin{tabular}{l}
|\input{childdoc.def}|\\
|\childdocmain{|\textit{main}|}|\\
\end{tabular}
\end{center}
%
If |\jobname| does not match the argument \textit{main} of |\childdocmain|,
it is assumed that |\jobname| points to the child file to be compiled.
When using |\childdocmain| with the main file specified as argument,
it suffices to start a child file
with just |\input{|\textit{main}|}|
without loading of the package and using |\childdocof|.
If instead all processing is done
with the appropriate \textsf{childdoc} directives,
the argument of \textit{main} of |\childdocmain| can be empty.

An alternative version of the command line processing described
in \secref{sec:commandline} using the detection mechanism reads:
%
\begin{center}
|... -jobname "|\textit{target}|" "|[\textit{flags}]%
[|\def\jobname{|\textit{dest}|}|]|\input{|\textit{main}|}"|
\end{center}

%%%%%%%%%%%%%%%%%%%%%%%%%%%%%%%%%%%%%%%%%%%%%%%%%%%%%%%%%%%%%%%%%%%%%%%%%%%%%%%%
\subsection{Manual Code}
\label{sec:manual}

In case one cannot be certain whether the definitions file |childdoc.def|
is installed on the target \TeX{} distribution
and one prefers not to ship it,
it is conceivable to paste a few relevant commands into the sources.

To that end, drop all statements |\input{childdoc.def}|
and perform the replacements as outlined below.
Instead of |\childdocmain{|\textit{main}|}| add the following code
to the top of the main file:
%
\begin{center}
\begin{tabular}{l}
|\||ifdefined\childdocname\endinput\||fi\newif\ifchilddoc|\\
|\edef\childdocname{\scantokens\expandafter{\jobname\noexpand}}|\\
|\def\childdocmain{|\textit{main}|}\||ifx\childdocmain\childdocname\||else|\\
|\childdoctrue\includeonly{\childdocname}\let\jobname\childdocmain\||fi|\\
\end{tabular}
\end{center}
%
Instead of |\childdocof{|\textit{main}|}| just include the main file
at the top of each child file:
%
\begin{center}
|\input{|\textit{main}|}|
\end{center}
%
A simple redirection |\childdocforward{|\textit{dest}|}| is achieved by:
%
\begin{center}
|\def\jobname{|\textit{dest}|}\input{\jobname}|
\end{center}
%
The redirection with prefix
|\childdocforwardprefix[|\textit{prefix}|]{|\textit{dest}|}|
is accomplished by:
%
\begin{center}
\begin{tabular}{l}
|{\edef\jobname{\scantokens\expandafter{\jobname\noexpand}}|\\
|\def\redirectjob |\textit{prefix}|#1~~~{\gdef\jobname{|\textit{dest}|#1}}|\\
|\expandafter\redirectjob\jobname~~~}\input{\jobname}|
\end{tabular}
\end{center}

In an alternative approach,
child documents can be compiled by a specific command line
without additional code or specific definitions:
%
\begin{center}
|... -jobname "|\textit{target}|" "|[\textit{flags}]%
|\includeonly{|\textit{dest}|}\input{|\textit{main}|}"|
\end{center}
%

%%%%%%%%%%%%%%%%%%%%%%%%%%%%%%%%%%%%%%%%%%%%%%%%%%%%%%%%%%%%%%%%%%%%%%%%%%%%%%%%
%%%%%%%%%%%%%%%%%%%%%%%%%%%%%%%%%%%%%%%%%%%%%%%%%%%%%%%%%%%%%%%%%%%%%%%%%%%%%%%%
\section{Information}

%%%%%%%%%%%%%%%%%%%%%%%%%%%%%%%%%%%%%%%%%%%%%%%%%%%%%%%%%%%%%%%%%%%%%%%%%%%%%%%%
\subsection{Copyright}

Copyright \copyright{} 2017--2018 Niklas Beisert

This work may be distributed and/or modified under the
conditions of the \LaTeX{} Project Public License, either version 1.3
of this license or (at your option) any later version.
The latest version of this license is in
  \url{http://www.latex-project.org/lppl.txt}
and version 1.3 or later is part of all distributions of \LaTeX{}
version 2005/12/01 or later.

This work has the LPPL maintenance status `maintained'.

The Current Maintainer of this work is Niklas Beisert.

This work consists of the files |README.txt|, |childdoc.ins| and |childdoc.dtx|
as well as the derived files |childdoc.def|, |cdocsamp.tex|
with |cdocsch1.tex|, |cdocsch2.tex|, |cdocspt3.tex|, |cdocspt4.tex|,
|cdocsdrf.tex|, |cdocsfn1.tex|, |cdocsfn2.tex|
as well as |childdoc.pdf|.

%%%%%%%%%%%%%%%%%%%%%%%%%%%%%%%%%%%%%%%%%%%%%%%%%%%%%%%%%%%%%%%%%%%%%%%%%%%%%%%%
\subsection{Files and Installation}

The package consists of the files:
%
\begin{center}
\begin{tabular}{ll}
    |README.txt|   & readme file \\
    |childdoc.ins| & installation file \\
    |childdoc.dtx| & source file \\
    |childdoc.def| & definition file \\
    |cdocsamp.tex| & sample main file \\
    |cdocsch1.tex| & sample include file \\
    |cdocsch2.tex| & sample include file \\
    |cdocspt3.tex| & sample part file \\
    |cdocspt4.tex| & sample part file \\
    |cdocsdrf.tex| & sample redirection file \\
    |cdocsfn1.tex| & sample redirection file \\
    |cdocsfn2.tex| & sample redirection file \\
    |childdoc.pdf| & manual
\end{tabular}
\end{center}
%
The distribution consists of the files
|README.txt|, |childdoc.ins| and |childdoc.dtx|.
%
\begin{itemize}
\item
Run (pdf)\LaTeX{} on |childdoc.dtx|
to compile the manual |childdoc.pdf| (this file).
\item
Run \LaTeX{} on |childdoc.ins| to create the definitions file |childdoc.def|
and the sample |cdocsamp.tex| with include files
|cdocsch1.tex|, |cdocsch2.tex|, |cdocspt3.tex|, |cdocspt4.tex|,
|cdocsdrf.tex|, |cdocsfn1.tex|, |cdocsfn2.tex|.
Then copy the file |childdoc.def| to an appropriate directory of your \LaTeX{}
distribution, e.g.\ \textit{texmf-root}|/tex/latex/childdoc|.
\end{itemize}

%%%%%%%%%%%%%%%%%%%%%%%%%%%%%%%%%%%%%%%%%%%%%%%%%%%%%%%%%%%%%%%%%%%%%%%%%%%%%%%%
\subsection{Related CTAN Packages}

There are several other packages which offer a similar functionality:
%
\begin{itemize}
\item
The packages
\href{http://ctan.org/pkg/docmute}{\textsf{docmute}},
\href{http://ctan.org/pkg/includex}{\textsf{includex}} and
\href{http://ctan.org/pkg/standalone}{\textsf{standalone}}
provide commands to include only the document body of
a child file thus allowing both files to be compiled individually.
\item
The packages \href{http://ctan.org/pkg/subdocs}{\textsf{subdocs}}
and \href{http://ctan.org/pkg/subfiles}{\textsf{subfiles}}
provide structures in which the main and child documents can be
encapsulated and allowing them to be compiled individually.
The inclusion mechanism is different from the conventional |\include|.
\item
The package \href{http://ctan.org/pkg/combine}{\textsf{combine}}
is an elaborate solution to combine several documents into one.
\end{itemize}
%
See also the CTAN topic \href{http://ctan.org/topic/subdocs}{\textsf{subdocs}}
for further related packages.
The present package differs from the above solutions in that
a document structure constructed with the conventional |\include| mechanism
just needs two extra commands at the top of every file
such that all constituent files can be compiled individually.

%%%%%%%%%%%%%%%%%%%%%%%%%%%%%%%%%%%%%%%%%%%%%%%%%%%%%%%%%%%%%%%%%%%%%%%%%%%%%%%%
%\subsection{Feature Suggestions}
%
%The following is a list of features which may be useful for future
%versions of this package:
%%
%\begin{itemize}
%\item
%\ldots
%\end{itemize}

%%%%%%%%%%%%%%%%%%%%%%%%%%%%%%%%%%%%%%%%%%%%%%%%%%%%%%%%%%%%%%%%%%%%%%%%%%%%%%%%
\subsection{Revision History}

%%%%%%%%%%%%%%%%%%%%%%%%%%%%%%%%%%%%%%%%
\paragraph{v2.0:} 2018/12/30

\begin{itemize}
\item
immediate forward processing
\item
added |\childdocby| mechanism
\item
manual restructured
\end{itemize}

%%%%%%%%%%%%%%%%%%%%%%%%%%%%%%%%%%%%%%%%
\paragraph{v1.6:} 2018/01/17

\begin{itemize}
\item
application for development of include files
\item
corrections to manual
\end{itemize}

%%%%%%%%%%%%%%%%%%%%%%%%%%%%%%%%%%%%%%%%
\paragraph{v1.5:} 2017/05/21

\begin{itemize}
\item
more complete structuring introduced
\item
|\childdocof| introduced
\item
|\childdoc| renamed to |\childdocmain|
\item
|\childredirect| renamed to |\childdocforward| and |\childdocforwardprefix|
and functionality expanded
\end{itemize}

%%%%%%%%%%%%%%%%%%%%%%%%%%%%%%%%%%%%%%%%
\paragraph{v1.0:} 2017/04/27

\begin{itemize}
\item
manual and install package
\item
first version published on CTAN
\end{itemize}

%%%%%%%%%%%%%%%%%%%%%%%%%%%%%%%%%%%%%%%%
\paragraph{v0.6:} 2017/04/26

\begin{itemize}
\item
redirection mechanism added
\end{itemize}

%%%%%%%%%%%%%%%%%%%%%%%%%%%%%%%%%%%%%%%%
\paragraph{v0.5:} 2017/04/26

\begin{itemize}
\item
functionality in definition file
\end{itemize}


%%%%%%%%%%%%%%%%%%%%%%%%%%%%%%%%%%%%%%%%%%%%%%%%%%%%%%%%%%%%%%%%%%%%%%%%%%%%%%%%
%%%%%%%%%%%%%%%%%%%%%%%%%%%%%%%%%%%%%%%%%%%%%%%%%%%%%%%%%%%%%%%%%%%%%%%%%%%%%%%%
%%%%%%%%%%%%%%%%%%%%%%%%%%%%%%%%%%%%%%%%%%%%%%%%%%%%%%%%%%%%%%%%%%%%%%%%%%%%%%%%
\appendix

\settowidth\MacroIndent{\rmfamily\scriptsize 000\ }

 \DocInput{childdoc.dtx}

\end{document}
%</driver>
% \fi
%
% %%%%%%%%%%%%%%%%%%%%%%%%%%%%%%%%%%%%%%%%%%%%%%%%%%%%%%%%%%%%%%%%%%%%%%%%%%%%%%
% %%%%%%%%%%%%%%%%%%%%%%%%%%%%%%%%%%%%%%%%%%%%%%%%%%%%%%%%%%%%%%%%%%%%%%%%%%%%%%
% \section{Sample}
%\iffalse
%<*samplemain>
%\fi
%
% The following presents a sample document
% with two chapters, two parts, a title page,
% a compile flag as well as three forwarding files to set the flag.
% It consists of eight |.tex| files:
% \begin{center}
% \begin{tabular}{ll}
% |cdocsamp.tex|&main file\\
% |cdocsch1.tex|&include file for chapter 1\\
% |cdocsch2.tex|&include file for chapter 2\\
% |cdocspt3.tex|&include file for part 3\\
% |cdocspt4.tex|&include file for part 4\\
% |cdocsdrf.tex|&forwarding file for main file in draft mode\\
% |cdocsfi1.tex|&forwarding file for final version of chapter 1\\
% |cdocsfi2.tex|&forwarding file for final version of chapter 2\\
% \end{tabular}
% \end{center}
% Each of the eight files can be compiled directly by the \LaTeX{} compiler.
%
% %%%%%%%%%%%%%%%%%%%%%%%%%%%%%%%%%%%%%%
% \paragraph{Main File.}
%
% The main file is called |cdocsamp.tex|.
%
% Load the \textsf{childdoc} definitions and
% declare the filename for the main document:
%    \begin{macrocode}
\input{childdoc.def}
\childdocmain{}
%    \end{macrocode}

% Optional override for |\version| flag:
%    \begin{macrocode}
%%\ifchilddoc\else\providecommand{\version}{draft}\fi
%    \end{macrocode}

% Define the default values for the |\version| flag
% (|final| for the main file and |draft| for childs):
%    \begin{macrocode}
\ifchilddoc
\providecommand{\version}{draft}
\else
\providecommand{\version}{final}
\fi
%    \end{macrocode}

% Load the standard document class:
%    \begin{macrocode}
\documentclass[12pt]{article}
%    \end{macrocode}

% Start the document body:
%    \begin{macrocode}
\begin{document}
%    \end{macrocode}

% Declare a title page.
% Print title, part of document being processed and version flag:
%    \begin{macrocode}
\addtocounter{page}{-1}
\begin{center}
{\LARGE\bfseries{}childdoc example\par}
\vspace{1cm}
\ifchilddoc
\ifchilddocmanual part\else chapter\fi:
`\childdocname' of `\childdocjob'\par
\else
main document: `\childdocjob'\par
\fi
version: \version\par
\end{center}
\newpage
%    \end{macrocode}

% Manually include selected file,
% otherwise process as usual:
%    \begin{macrocode}
\ifchilddocmanual
\section*{part `\childdocname'}
\input{\childdocname}
\else
%    \end{macrocode}

% Include the two chapters:
%    \begin{macrocode}
\include{cdocsch1}
\include{cdocsch2}
%    \end{macrocode}

% Include the two parts unless only chapters should be displayed:
%    \begin{macrocode}
\ifchilddoc\else
\section{part three}
\input{cdocspt3}
\section{part four}
\input{cdocspt4}
\fi
%    \end{macrocode}

% Process as usual until here:
%    \begin{macrocode}
\fi
%    \end{macrocode}

% End of document body:
%    \begin{macrocode}
\end{document}
%    \end{macrocode}
%\iffalse
%</samplemain>
%\fi
%
% %%%%%%%%%%%%%%%%%%%%%%%%%%%%%%%%%%%%%%
% \paragraph{Chapter Include Files.}
%
% The include files are called |cdocsch1.tex| and |cdocsch2.tex|.
%
%\iffalse
%<*samplechap1|samplechap2>
%\fi

% Optional override for |\version| flag:
%    \begin{macrocode}
%%\providecommand{\version}{final}
%    \end{macrocode}

% Include the main document:
%    \begin{macrocode}
\input{childdoc.def}
\childdocof{cdocsamp}
%    \end{macrocode}

%\iffalse
%</samplechap1|samplechap2>
%\fi
%
%\iffalse
%<*samplechap1>
%\fi
% Some text for chapter 1:
%    \begin{macrocode}
\section{one}
some text in chapter one
%    \end{macrocode}

%\iffalse
%</samplechap1>
%\fi
% Some text for chapter 2:
%\iffalse
%<*samplechap2>
%\fi
%    \begin{macrocode}
\section{two}
more text in chapter two
%    \end{macrocode}

%\iffalse
%</samplechap2>
%\fi
%
% %%%%%%%%%%%%%%%%%%%%%%%%%%%%%%%%%%%%%%
% \paragraph{Part Include Files.}
%
% The include files are called |cdocspt3.tex| and |cdocspt4.tex|.
%
%\iffalse
%<*samplepart3|samplepart4>
%\fi

% Optional override for |\version| flag:
%    \begin{macrocode}
%%\providecommand{\version}{final}
%    \end{macrocode}

% Include the main document:
%    \begin{macrocode}
\input{childdoc.def}
\childdocby{cdocsamp}
%    \end{macrocode}

%\iffalse
%</samplepart3|samplepart4>
%\fi
%
%\iffalse
%<*samplepart3>
%\fi
% Some text for part 3:
%    \begin{macrocode}
some text in part three
%    \end{macrocode}

%\iffalse
%</samplepart3>
%\fi
% Some text for part 4:
%\iffalse
%<*samplepart4>
%\fi
%    \begin{macrocode}
more text in part four
%    \end{macrocode}

%\iffalse
%</samplepart4>
%\fi
%
% %%%%%%%%%%%%%%%%%%%%%%%%%%%%%%%%%%%%%%
% \paragraph{Forwarding for a Complete Draft.}
%
% The following forwarding file |cdocsdrf.tex|
% compiles the main document in draft mode:
%\iffalse
%<*sampledraft>
%\fi
%    \begin{macrocode}
\def\version{draft}
\input{childdoc.def}
\childdocforward{cdocsamp}
%    \end{macrocode}

%\iffalse
%</sampledraft>
%\fi
%
% %%%%%%%%%%%%%%%%%%%%%%%%%%%%%%%%%%%%%%
% \paragraph{Forwarding for Final Version of the Chapters.}
%
% The following forwarding files |cdocsfn1.tex| and |cdocsfn2.tex|
% (with identical content)
% compile the final versions of the child documents
% |cdocsch1.tex| and |cdocsch2.tex|, respectively:
%\iffalse
%<*samplefinal>
%\fi
%    \begin{macrocode}
\def\version{final}
\input{childdoc.def}
\childdocforwardprefix[cdocsamp]{cdocsfn}{cdocsch}
%    \end{macrocode}

%\iffalse
%</samplefinal>
%\fi
%
% %%%%%%%%%%%%%%%%%%%%%%%%%%%%%%%%%%%%%%
% \paragraph{Command Line Processing.}
%
% The following three command lines generate the output files
% |cdocscld|, |cdocscl1| and |cdocscl2|
% which should be identical to
% |cdocsdrf|, |cdocsch1| and |cdocsfn2|, respectively:
% \begin{center}
% \begin{tabular}{l}
% |latex -jobname cdocscld \|\\
% |  "\def\version{draft}\input{childdoc.def}\childdocforward{cdocsamp}"|\\
% |latex -jobname cdocscl1 \|\\
% |  "\input{childdoc.def}\childdocforward[cdocsamp]{cdocsch1}"|\\
% |latex -jobname cdocscl2 \|\\
% |  "\def\version{final}\input{childdoc.def}\childdocforward{cdocsch2}"|
% \end{tabular}
% \end{center}
% Note that the trailing backslash on each first line
% merely continues the input to the second line
% (for convenient cut ant paste).
% Furthermore, the command |latex| can be replaced by any
% of its alternative versions such as |pdflatex|.
%
% %%%%%%%%%%%%%%%%%%%%%%%%%%%%%%%%%%%%%%%%%%%%%%%%%%%%%%%%%%%%%%%%%%%%%%%%%%%%%%
% %%%%%%%%%%%%%%%%%%%%%%%%%%%%%%%%%%%%%%%%%%%%%%%%%%%%%%%%%%%%%%%%%%%%%%%%%%%%%%
% \section{Implementation}
%\iffalse
%<*package>
%\fi
%
% This section describes the definitions file |childdoc.def|.

% The definitions cannot be loaded using |\usepackage| or |\RequirePackage|
% which has a mechanism to prevent loading a style file more than once.
% When loading the definitions by means of |\input|
% multiple instances have to be prevented manually:
%\iffalse
%This code needs to be before the `\ProvidesFile' directive
%which is defined at the beginning of this file.
%Therefore it is also placed there and commented out here.
%</package>
%<*discard>
%\fi
%    \begin{macrocode}
\ifdefined\childdocmain\endinput\fi
%    \end{macrocode}
%\iffalse
%</discard>
%<*package>
%\fi
%
% \macro{\ifchilddoc}
% \macro{\ifchilddocmanual}
% The conditional |\ifchilddoc| tells whether a
% child (true) or main (false) document is being compiled.
% The conditional |\ifchilddocmanual| tells whether
% the |\includeonly| mechanism is used (false) or
% the selection of child files must be performed manually (true).
% The definitions initialise to false:
%    \begin{macrocode}
\newif\ifchilddoc
\newif\ifchilddocmanual
%    \end{macrocode}

% \macro{\childdocname}
% \macro{\childdocjob}
% The macro |\childdocname| stores the name of the main document
% to be compiled. The macro |\childdocjob| stores the name of
% the document on which the \LaTeX{} compiler was originally invoked.
% The content of |\jobname| cannot be compared
% to filenames specified in the source due to different catcodes.
% The following code rescans |\jobname|, stores the result
% in |\childdocname| and saves a copy in |\childdocjob|:
%    \begin{macrocode}
\edef\childdocname{\scantokens\expandafter{\jobname\noexpand}}
\let\childdocjob\childdocname
%    \end{macrocode}

% \macro{\childdocdisable}
% The macro |\childdocdisable| prevents the main file
% from being processed more than once.
% At this stage, the main document command |\childdocmain|
% is assumed to be called once again where it should do nothing.
% Any subsequent call to it should prevent
% a secondary processing of the main document
% It overwrites the forwarding commands
% |\childdocof| and |\childdocforward|
% with empty macros to prevent further inclusions of the main document:
%    \begin{macrocode}
\newcommand{\childdocdisable}
{
  \renewcommand{\childdocmain}[1]{\renewcommand{\childdocmain}[1]{\endinput}}
  \renewcommand{\childdocof}[1]{}
  \renewcommand{\childdocby}[2][]{}
  \renewcommand{\childdocforward}[2][]{}
  \renewcommand{\childdocdisable}{}
}
%    \end{macrocode}

% \macro{\childdocmain}
% The macro |\childdocmain| is to be called at the top of the main file
% with nothing or the main filename (without extension) as argument.
% First, it breaks loops.
% If the argument is not empty and does not match |\childdocname|
% (which is set by the first inclusion of |childdoc.def|),
% |\ifchilddoc| is set to true, |\includeonly| is applied to the child file
% and |\jobname| is set to the main file
% (for proper handling of |.aux| files):
%    \begin{macrocode}
\newcommand{\childdocmain}[1]
{
  \childdocdisable\childdocmain{}
  \if?#1?\else
    \begingroup
      \def\childdoctmp{#1}
      \ifx\childdoctmp\childdocname
        \def\childdoctmp{}
      \else
        \def\childdoctmp
        {
          \childdoctrue
          \includeonly{\childdocname}
          \def\childdocjob{#1}
          \def\jobname{#1}
        }
      \fi
      \expandafter
    \endgroup
    \childdoctmp
  \fi
}
%    \end{macrocode}

% \macro{\childdocof}
% The command |\childdocof| redirects
% compilation to the main file |#1|.
%    \begin{macrocode}
\newcommand{\childdocof}[1]
{
  \childdocdisable
  \childdoctrue
  \includeonly{\childdocname}
  \def\jobname{#1}
  \def\childdocjob{#1}
  \input{#1}
}
%    \end{macrocode}

% \macro{\childdocby}
% The command |\childdocby| ....
%    \begin{macrocode}
\newcommand{\childdocby}[2][]
{
  \childdocdisable
  \childdoctrue
  \childdocmanualtrue
  \if?#1?\else
    \def\jobname{#2}
  \fi
  \def\childdocjob{#2}
  \input{#2}
  \endinput
}
%    \end{macrocode}

% \macro{\childdocforward}
% The command |\childdocforward| redirects
% compilation to the main file or
% (if the optional argument is given) a child file.
% Parameters are set as if the main file
% or a child file starting with |\childdocof| was compiled.
% Then compilation is handed over to the main file:
%    \begin{macrocode}
\newcommand{\childdocforward}[2][]
{
  \begingroup
    \if?#1?
      \def\childdoctmp
      {
        \def\childdocname{#2}
        \def\childdocjob{#2}
        \def\jobname{#2}
        \input{#2}
        \endinput
      }
    \else
      \def\childdoctmp
      {
        \childdocdisable
        \def\childdocname{#2}
        \childdoctrue
        \includeonly{#2}
        \def\childdocjob{#1}
        \def\jobname{#1}
        \input{#1}
        \endinput
      }
    \fi
    \expandafter
  \endgroup
  \childdoctmp
}
%    \end{macrocode}

% \macro{\childdocforwardprefix}
% The command |\childdocforwardprefix| redirects
% compilation to the main or a child file by means of a pattern.
% The prefix |#1| in the current filename is replaced by |#2|
% and the suffix of the current filename is kept
% (it is assumed that the filename does not contain the substring `|~~~|'
% which is used as a delimiter).
% Compilation is handed over to the new file by |\childdocforward|:
%    \begin{macrocode}
\newcommand{\childdocforwardprefix}[3][]
{
  \begingroup
    \def\childdocextract #2##1~~~{\def\childdoctmp{\childdocforward[#1]{#3##1}}}
    \expandafter\childdocextract\childdocname~~~
    \expandafter
  \endgroup
  \childdoctmp
}
%    \end{macrocode}

% \macro{\childdoc}
% The deprecated macro |\childdoc| is a legacy version of |\childdocmain|:
%    \begin{macrocode}
\newcommand{\childdoc}{\childdocmain}
%    \end{macrocode}

% \macro{\childdocredirect}
% The deprecated macro |\childdocredirect| is a legacy version
% of |\childdocforward| and |\childdocforwardprefix|:
%    \begin{macrocode}
\newcommand{\childdocredirect}[2][]
{
  \begingroup
    \if?#1?
      \def\childdoctmp{\childdocforward{#2}}
    \else
      \def\childdoctmp{\childdocforwardprefix{#1}{#2}}
    \fi
    \expandafter
  \endgroup
  \childdoctmp
}
%    \end{macrocode}

%\iffalse
%</package>
%\fi
%
\endinput
|
and perform the replacements as outlined below.
Instead of |\childdocmain{|\textit{main}|}| add the following code
to the top of the main file:
%
\begin{center}
\begin{tabular}{l}
|\||ifdefined\childdocname\endinput\||fi\newif\ifchilddoc|\\
|\edef\childdocname{\scantokens\expandafter{\jobname\noexpand}}|\\
|\def\childdocmain{|\textit{main}|}\||ifx\childdocmain\childdocname\||else|\\
|\childdoctrue\includeonly{\childdocname}\let\jobname\childdocmain\||fi|\\
\end{tabular}
\end{center}
%
Instead of |\childdocof{|\textit{main}|}| just include the main file
at the top of each child file:
%
\begin{center}
|\input{|\textit{main}|}|
\end{center}
%
A simple redirection |\childdocforward{|\textit{dest}|}| is achieved by:
%
\begin{center}
|\def\jobname{|\textit{dest}|}\input{\jobname}|
\end{center}
%
The redirection with prefix
|\childdocforwardprefix[|\textit{prefix}|]{|\textit{dest}|}|
is accomplished by:
%
\begin{center}
\begin{tabular}{l}
|{\edef\jobname{\scantokens\expandafter{\jobname\noexpand}}|\\
|\def\redirectjob |\textit{prefix}|#1~~~{\gdef\jobname{|\textit{dest}|#1}}|\\
|\expandafter\redirectjob\jobname~~~}\input{\jobname}|
\end{tabular}
\end{center}

In an alternative approach,
child documents can be compiled by a specific command line
without additional code or specific definitions:
%
\begin{center}
|... -jobname "|\textit{target}|" "|[\textit{flags}]%
|\includeonly{|\textit{dest}|}\input{|\textit{main}|}"|
\end{center}
%

%%%%%%%%%%%%%%%%%%%%%%%%%%%%%%%%%%%%%%%%%%%%%%%%%%%%%%%%%%%%%%%%%%%%%%%%%%%%%%%%
%%%%%%%%%%%%%%%%%%%%%%%%%%%%%%%%%%%%%%%%%%%%%%%%%%%%%%%%%%%%%%%%%%%%%%%%%%%%%%%%
\section{Information}

%%%%%%%%%%%%%%%%%%%%%%%%%%%%%%%%%%%%%%%%%%%%%%%%%%%%%%%%%%%%%%%%%%%%%%%%%%%%%%%%
\subsection{Copyright}

Copyright \copyright{} 2017--2018 Niklas Beisert

This work may be distributed and/or modified under the
conditions of the \LaTeX{} Project Public License, either version 1.3
of this license or (at your option) any later version.
The latest version of this license is in
  \url{http://www.latex-project.org/lppl.txt}
and version 1.3 or later is part of all distributions of \LaTeX{}
version 2005/12/01 or later.

This work has the LPPL maintenance status `maintained'.

The Current Maintainer of this work is Niklas Beisert.

This work consists of the files |README.txt|, |childdoc.ins| and |childdoc.dtx|
as well as the derived files |childdoc.def|, |cdocsamp.tex|
with |cdocsch1.tex|, |cdocsch2.tex|, |cdocspt3.tex|, |cdocspt4.tex|,
|cdocsdrf.tex|, |cdocsfn1.tex|, |cdocsfn2.tex|
as well as |childdoc.pdf|.

%%%%%%%%%%%%%%%%%%%%%%%%%%%%%%%%%%%%%%%%%%%%%%%%%%%%%%%%%%%%%%%%%%%%%%%%%%%%%%%%
\subsection{Files and Installation}

The package consists of the files:
%
\begin{center}
\begin{tabular}{ll}
    |README.txt|   & readme file \\
    |childdoc.ins| & installation file \\
    |childdoc.dtx| & source file \\
    |childdoc.def| & definition file \\
    |cdocsamp.tex| & sample main file \\
    |cdocsch1.tex| & sample include file \\
    |cdocsch2.tex| & sample include file \\
    |cdocspt3.tex| & sample part file \\
    |cdocspt4.tex| & sample part file \\
    |cdocsdrf.tex| & sample redirection file \\
    |cdocsfn1.tex| & sample redirection file \\
    |cdocsfn2.tex| & sample redirection file \\
    |childdoc.pdf| & manual
\end{tabular}
\end{center}
%
The distribution consists of the files
|README.txt|, |childdoc.ins| and |childdoc.dtx|.
%
\begin{itemize}
\item
Run (pdf)\LaTeX{} on |childdoc.dtx|
to compile the manual |childdoc.pdf| (this file).
\item
Run \LaTeX{} on |childdoc.ins| to create the definitions file |childdoc.def|
and the sample |cdocsamp.tex| with include files
|cdocsch1.tex|, |cdocsch2.tex|, |cdocspt3.tex|, |cdocspt4.tex|,
|cdocsdrf.tex|, |cdocsfn1.tex|, |cdocsfn2.tex|.
Then copy the file |childdoc.def| to an appropriate directory of your \LaTeX{}
distribution, e.g.\ \textit{texmf-root}|/tex/latex/childdoc|.
\end{itemize}

%%%%%%%%%%%%%%%%%%%%%%%%%%%%%%%%%%%%%%%%%%%%%%%%%%%%%%%%%%%%%%%%%%%%%%%%%%%%%%%%
\subsection{Related CTAN Packages}

There are several other packages which offer a similar functionality:
%
\begin{itemize}
\item
The packages
\href{http://ctan.org/pkg/docmute}{\textsf{docmute}},
\href{http://ctan.org/pkg/includex}{\textsf{includex}} and
\href{http://ctan.org/pkg/standalone}{\textsf{standalone}}
provide commands to include only the document body of
a child file thus allowing both files to be compiled individually.
\item
The packages \href{http://ctan.org/pkg/subdocs}{\textsf{subdocs}}
and \href{http://ctan.org/pkg/subfiles}{\textsf{subfiles}}
provide structures in which the main and child documents can be
encapsulated and allowing them to be compiled individually.
The inclusion mechanism is different from the conventional |\include|.
\item
The package \href{http://ctan.org/pkg/combine}{\textsf{combine}}
is an elaborate solution to combine several documents into one.
\end{itemize}
%
See also the CTAN topic \href{http://ctan.org/topic/subdocs}{\textsf{subdocs}}
for further related packages.
The present package differs from the above solutions in that
a document structure constructed with the conventional |\include| mechanism
just needs two extra commands at the top of every file
such that all constituent files can be compiled individually.

%%%%%%%%%%%%%%%%%%%%%%%%%%%%%%%%%%%%%%%%%%%%%%%%%%%%%%%%%%%%%%%%%%%%%%%%%%%%%%%%
%\subsection{Feature Suggestions}
%
%The following is a list of features which may be useful for future
%versions of this package:
%%
%\begin{itemize}
%\item
%\ldots
%\end{itemize}

%%%%%%%%%%%%%%%%%%%%%%%%%%%%%%%%%%%%%%%%%%%%%%%%%%%%%%%%%%%%%%%%%%%%%%%%%%%%%%%%
\subsection{Revision History}

%%%%%%%%%%%%%%%%%%%%%%%%%%%%%%%%%%%%%%%%
\paragraph{v2.0:} 2018/12/30

\begin{itemize}
\item
immediate forward processing
\item
added |\childdocby| mechanism
\item
manual restructured
\end{itemize}

%%%%%%%%%%%%%%%%%%%%%%%%%%%%%%%%%%%%%%%%
\paragraph{v1.6:} 2018/01/17

\begin{itemize}
\item
application for development of include files
\item
corrections to manual
\end{itemize}

%%%%%%%%%%%%%%%%%%%%%%%%%%%%%%%%%%%%%%%%
\paragraph{v1.5:} 2017/05/21

\begin{itemize}
\item
more complete structuring introduced
\item
|\childdocof| introduced
\item
|\childdoc| renamed to |\childdocmain|
\item
|\childredirect| renamed to |\childdocforward| and |\childdocforwardprefix|
and functionality expanded
\end{itemize}

%%%%%%%%%%%%%%%%%%%%%%%%%%%%%%%%%%%%%%%%
\paragraph{v1.0:} 2017/04/27

\begin{itemize}
\item
manual and install package
\item
first version published on CTAN
\end{itemize}

%%%%%%%%%%%%%%%%%%%%%%%%%%%%%%%%%%%%%%%%
\paragraph{v0.6:} 2017/04/26

\begin{itemize}
\item
redirection mechanism added
\end{itemize}

%%%%%%%%%%%%%%%%%%%%%%%%%%%%%%%%%%%%%%%%
\paragraph{v0.5:} 2017/04/26

\begin{itemize}
\item
functionality in definition file
\end{itemize}


%%%%%%%%%%%%%%%%%%%%%%%%%%%%%%%%%%%%%%%%%%%%%%%%%%%%%%%%%%%%%%%%%%%%%%%%%%%%%%%%
%%%%%%%%%%%%%%%%%%%%%%%%%%%%%%%%%%%%%%%%%%%%%%%%%%%%%%%%%%%%%%%%%%%%%%%%%%%%%%%%
%%%%%%%%%%%%%%%%%%%%%%%%%%%%%%%%%%%%%%%%%%%%%%%%%%%%%%%%%%%%%%%%%%%%%%%%%%%%%%%%
\appendix

\settowidth\MacroIndent{\rmfamily\scriptsize 000\ }

 \DocInput{childdoc.dtx}

\end{document}
%</driver>
% \fi
%
% %%%%%%%%%%%%%%%%%%%%%%%%%%%%%%%%%%%%%%%%%%%%%%%%%%%%%%%%%%%%%%%%%%%%%%%%%%%%%%
% %%%%%%%%%%%%%%%%%%%%%%%%%%%%%%%%%%%%%%%%%%%%%%%%%%%%%%%%%%%%%%%%%%%%%%%%%%%%%%
% \section{Sample}
%\iffalse
%<*samplemain>
%\fi
%
% The following presents a sample document
% with two chapters, two parts, a title page,
% a compile flag as well as three forwarding files to set the flag.
% It consists of eight |.tex| files:
% \begin{center}
% \begin{tabular}{ll}
% |cdocsamp.tex|&main file\\
% |cdocsch1.tex|&include file for chapter 1\\
% |cdocsch2.tex|&include file for chapter 2\\
% |cdocspt3.tex|&include file for part 3\\
% |cdocspt4.tex|&include file for part 4\\
% |cdocsdrf.tex|&forwarding file for main file in draft mode\\
% |cdocsfi1.tex|&forwarding file for final version of chapter 1\\
% |cdocsfi2.tex|&forwarding file for final version of chapter 2\\
% \end{tabular}
% \end{center}
% Each of the eight files can be compiled directly by the \LaTeX{} compiler.
%
% %%%%%%%%%%%%%%%%%%%%%%%%%%%%%%%%%%%%%%
% \paragraph{Main File.}
%
% The main file is called |cdocsamp.tex|.
%
% Load the \textsf{childdoc} definitions and
% declare the filename for the main document:
%    \begin{macrocode}
% \iffalse
%
% childdoc.dtx Copyright (C) 2017-2018 Niklas Beisert
%
% This work may be distributed and/or modified under the
% conditions of the LaTeX Project Public License, either version 1.3
% of this license or (at your option) any later version.
% The latest version of this license is in
%   http://www.latex-project.org/lppl.txt
% and version 1.3 or later is part of all distributions of LaTeX
% version 2005/12/01 or later.
%
% This work has the LPPL maintenance status `maintained'.
%
% The Current Maintainer of this work is Niklas Beisert.
%
% This work consists of the files childdoc.dtx and childdoc.ins
% and the derived files childdoc.def and cdocsamp.tex with
% cdocsch1.tex, cdocsch2.tex, cdocsdrf.tex, cdocsfn1.tex, cdocsfn2.tex.
%
%<package>\ifdefined\childdocmain\endinput\fi
%<package>\ProvidesFile{childdoc.def}[2018/12/30 v2.0 child document driver]
%<samplemain>\ProvidesFile{cdocsamp.tex}[2018/12/30 v2.0 sample for childdoc]
%<*driver>
%\ProvidesFile{childdoc.drv}[2018/12/30 v2.0 childdoc reference manual file]
\PassOptionsToClass{10pt,a4paper}{article}
\documentclass{ltxdoc}

\usepackage[margin=35mm]{geometry}
\usepackage{hyperref}
\usepackage{hyperxmp}
\usepackage[usenames]{color}

\hypersetup{colorlinks=true}
\hypersetup{pdfstartview=FitH}
\hypersetup{pdfpagemode=UseNone}
\hypersetup{pdfsource={}}
\hypersetup{pdflang={en-UK}}
\hypersetup{pdfcopyright={Copyright 2017-2018 Niklas Beisert.
  This work may be distributed and/or modified under the
  conditions of the LaTeX Project Public License, either version 1.3
  of this license or (at your option) any later version.}}
\hypersetup{pdflicenseurl={http://www.latex-project.org/lppl.txt}}
\hypersetup{pdfcontactaddress={ETH Zurich, ITP, HIT K,
  Wolfgang-Pauli-Strasse 27}}
\hypersetup{pdfcontactpostcode={8093}}
\hypersetup{pdfcontactcity={Zurich}}
\hypersetup{pdfcontactcountry={Switzerland}}
\hypersetup{pdfcontactemail={nbeisert@itp.phys.ethz.ch}}
\hypersetup{pdfcontacturl={http://people.phys.ethz.ch/\xmptilde nbeisert/}}

\newcommand{\secref}[1]{\hyperref[#1]{section \ref*{#1}}}

\parskip1ex
\parindent0pt
\let\olditemize\itemize
\def\itemize{\olditemize\parskip0pt}

\begin{document}

\title{The \textsf{childdoc} Package}
\hypersetup{pdftitle={The childdoc Package}}
\author{Niklas Beisert\\[2ex]
  Institut f\"ur Theoretische Physik\\
  Eidgen\"ossische Technische Hochschule Z\"urich\\
  Wolfgang-Pauli-Strasse 27, 8093 Z\"urich, Switzerland\\[1ex]
  \href{mailto:nbeisert@itp.phys.ethz.ch}
  {\texttt{nbeisert@itp.phys.ethz.ch}}}
\hypersetup{pdfauthor={Niklas Beisert}}
\hypersetup{pdfsubject={Manual for the LaTeX2e Package childdoc}}
\date{30 December 2018, \textsf{v2.0}}
\maketitle

\begin{abstract}\noindent
\textsf{childdoc} is a \LaTeXe{} package
that enables the direct compilation
of document sections included by |\include|
to individual files.
\end{abstract}

\begingroup
\parskip0ex
\tableofcontents
\endgroup

%%%%%%%%%%%%%%%%%%%%%%%%%%%%%%%%%%%%%%%%%%%%%%%%%%%%%%%%%%%%%%%%%%%%%%%%%%%%%%%%
%%%%%%%%%%%%%%%%%%%%%%%%%%%%%%%%%%%%%%%%%%%%%%%%%%%%%%%%%%%%%%%%%%%%%%%%%%%%%%%%
\section{Introduction}

\LaTeX{} provides a mechanism to structure a large document (such as a book)
into a main file and several child files (containing the chapters)
using the |\include| command.
This mechanism is beneficial for documents
which span hundreds of pages in order to
make the source file(s) more manageable.
Moreover, compilation can be restricted to
selected child files by means of the |\includeonly| command.
The latter feature can be used to reduce the compilation time while editing
(this was significantly more useful in the earlier days of \LaTeX{})
or to generate a smaller document which is easier to navigate.
Another application of |\includeonly| is to generate
documents consisting of selected parts of the complete document.

However, there are a few drawbacks of the plain |\include| mechanism:
\begin{itemize}
\item
The child files cannot be compiled on their own,
they can only be compiled via the main file.
A naive editing environment
(such as a text editor with an option
to have the current file processed by \LaTeX)
may require one to switch to the main file before compiling;
attempting to compile the child file produces errors.
\item
The main file must be modified (each time)
to adjust the |\includeonly| command
to the present needs. This easily leaves the main file in a messy state.
\item
The generated document will always carry the filename
of the main document. This is inconvenient if
several child files are to be compiled and
to be kept for distribution.
\end{itemize}

The present package provides a simple interface
to make child files individually compilable by \LaTeX{}.
Compiling a child file then has the same effect as compiling
the main file with an |\includeonly| command
to select the appropriate child.
Moreover the generated document will carry the name of the child
rather than the main file.
This resolves all three above issues.

This feature is meant to make the editing of books,
thesis documents and lecture notes somewhat more convenient.
However, the package can also be used efficiently for
composing a series of documents (such as exercise sheets)
which are typically distributed individually.
It then assists the author in generating the individual documents
(potentially in different versions)
as well as a document containing the collected series.
Another application is in developing style files
or other kinds of included material
where compilation of the style file could redirect
to a sample or test file.

%%%%%%%%%%%%%%%%%%%%%%%%%%%%%%%%%%%%%%%%%%%%%%%%%%%%%%%%%%%%%%%%%%%%%%%%%%%%%%%%
%%%%%%%%%%%%%%%%%%%%%%%%%%%%%%%%%%%%%%%%%%%%%%%%%%%%%%%%%%%%%%%%%%%%%%%%%%%%%%%%
\section{Usage}

First of all, the package \textsf{childdoc} is \emph{not} a standard
\LaTeXe{} |.sty| style file! Therefore it needs to be invoked in
a non-standard way.

%%%%%%%%%%%%%%%%%%%%%%%%%%%%%%%%%%%%%%%%%%%%%%%%%%%%%%%%%%%%%%%%%%%%%%%%%%%%%%%%
\subsection{Included Files}
\label{sec:include}

%%%%%%%%%%%%%%%%%%%%%%%%%%%%%%%%%%%%%%%%
\DescribeMacro{\childdocmain}
To use the package, add the commands
\begin{center}
\begin{tabular}{l}
|\input{childdoc.def}|\\
|\childdocmain{}|\\
\end{tabular}
\end{center}
at the very top of the main \LaTeX{} file,
in particular \emph{before} the |\documentclass| statement!
The argument of |\childdocmain| should be left empty
(but it must be present).

%%%%%%%%%%%%%%%%%%%%%%%%%%%%%%%%%%%%%%%%
\DescribeMacro{\childdocof}
Furthermore, add the commands
\begin{center}
\begin{tabular}{l}
|\input{childdoc.def}|\\
|\childdocof{|\textit{main}|}|\\
\end{tabular}
\end{center}
at the top of every child file \textit{child}
which is included by |\include{|\textit{child}|}|
from within the main file
(or at least for those files to be compiled individually).
The argument \textit{main} must be the filename of the main file.

There are a couple of
considerations in setting up the main and child documents:

%%%%%%%%%%%%%%%%%%%%%%%%%%%%%%%%%%%%%%%%
\paragraph{Restrictions.}

Please note the following restrictions:
\begin{itemize}
\item
|\childdocmain| must be called with one argument \textit{main}
to ensure compatibility with earlier version of the package.
It must either be empty (|\childdocmain{}|)
or precisely match the filename of the main file in which it is specified.
See \secref{sec:detection} for further information.
\item
The filename \textit{main} must be specified without the |.tex| extension.
\item
The filename \textit{main} is case sensitive
(even in case-insensitive file systems)
due to internal string comparison.
\item
The argument \textit{main} should be fully expanded, it cannot be a macro.
\item
Subdirectories and special characters should be avoided in filenames.
\item
The command |\childdocmain{|\textit{main}|}| must be followed by a whitespace.
It should not be followed immediately by another command
or by a comment mark `|%|'.
This is because the \TeX{} parser reads the token immediately following
the argument of |\childdocmain| and puts it
at the beginning of every child section;
however, a white\-space is ignored.
\end{itemize}

%%%%%%%%%%%%%%%%%%%%%%%%%%%%%%%%%%%%%%%%
\paragraph{Content of Main File.}

It is advisable to place all content in the child files included by |\include|.
Any output contained in the main file will appear in all child documents
unless suppressed manually;
it cannot be suppressed automatically by the |\includeonly| directive
and thus should normally be avoided.
A method to include some content in the main file
by means of conditional processing is described in \secref{sec:conditional}.

%%%%%%%%%%%%%%%%%%%%%%%%%%%%%%%%%%%%%%%%
\paragraph{Page Numbering.}

When only a part of the document is compiled,
the appropriate numbering of pages
(as well as other status parameters)
is determined from the |.aux| files.
The latter contain information from previous passes.
However this information needs to propagate through
all intermediate child documents.
Therefore the page numbering in child documents may well
be inconsistent until the complete document is compiled at least once.

A useful (if unconventional) way to always ensure a consistent
page numbering is to restart the numbering in each child document
and denote the pages by `\textit{child}|.|\textit{page}'
where \textit{child} represents the chapter/section number of the child file.
This can be achieved by the command
|\numberwithin{page}{|\textit{child}|}|
of the \textsf{amsmath} package
where \textit{child} can be |chapter| or |section|
depending on the chosen structuring.
Alternatively, one can modify the macro |\thepage| appropriately
and reset the counter |page| at the start of each child file.

%%%%%%%%%%%%%%%%%%%%%%%%%%%%%%%%%%%%%%%%%%%%%%%%%%%%%%%%%%%%%%%%%%%%%%%%%%%%%%%%
\subsection{Conditional Processing}
\label{sec:conditional}

The package provides a mechanism to compile different versions
of a document. To customise the versions further some conditional processing
can come in handy to distinguish which version is being compiled.
The package provides two macros to describe the compilation context:

%%%%%%%%%%%%%%%%%%%%%%%%%%%%%%%%%%%%%%%%
\DescribeMacro{\ifchilddoc}
The conditional |\ifchilddoc| distinguishes between the compilation of
child documents and the main document:
%
\begin{center}
|\ifchilddoc |\textit{child-code}| |[|\||else |\textit{main-code}]| \||fi|
\end{center}

%%%%%%%%%%%%%%%%%%%%%%%%%%%%%%%%%%%%%%%%
\DescribeMacro{\childdocname}
\DescribeMacro{\childdocjob}
The macro |\childdocname| contains the filename (without extension)
of the main or child file being processed.
Note that |\childdocjob| will always contain the name of the main file.

%%%%%%%%%%%%%%%%%%%%%%%%%%%%%%%%%%%%%%%%
\paragraph{Title Page.}

Conditional processing can be used to include a title or banner page
in the main document when proper precautions are taken.
Importantly, the code in the main file should ensure that the page counter
(as well as other status parameters which are stored in the |.aux| files)
takes the same value after the conditional processing.
Otherwise the page numbers may take divergent values
depending on which part is compiled.

For example, a title page could be declared by:
%
\begin{center}
\begin{tabular}{l}
|\ifchilddoc\||else|\\
|\addtocounter{page}{-1}|\\
\textit{code for title page}\\
|\newpage|\\
|\||fi|
\end{tabular}
\end{center}
%
A banner page for the child documents can be generated by:
%
\begin{center}
\begin{tabular}{l}
|\ifchilddoc|\\
|\addtocounter{page}{-1}|\\
\textit{code for banner page}\\
|\newpage|\\
|\||fi|
\end{tabular}
\end{center}
%
Here one could write a message such as:
\begin{center}
|This is the part \childdocname{} of \childdocjob{}.|
\end{center}

%%%%%%%%%%%%%%%%%%%%%%%%%%%%%%%%%%%%%%%%%%%%%%%%%%%%%%%%%%%%%%%%%%%%%%%%%%%%%%%%
\subsection{Flags}
\label{sec:flags}

The package makes it easy to generate different versions
of the main or child documents.
To this end compilation flags can be defined
and assigned different default values.
They will be particularly useful in conjunction
with the forwarding mechanism described in \secref{sec:forward}.

For example, it may be useful to have a flag |\version|
which can be set to |draft| or |final|.
The document source will contain some conditional code
depending on the value of |\version|.
Suppose further, the flag should default to |final| for the main file
and to |draft| for child files
which is a natural assignment for editing the document.
This is achieved by placing the following code
in the preamble of the main document
(below the |\childdocmain| directive):
%
\begin{center}
\begin{tabular}{l}
|\ifchilddoc|\\
|\providecommand{\version}{draft}|\\
|\||else|\\
|\providecommand{\version}{final}|\\
|\||fi|
\end{tabular}
\end{center}
%
The definition by |\providecommand| makes sure
that previous definitions are not overwritten.
Further statements |\providecommand{\version}{...}|
can thus be added before the above code to override it.

For the main file, one might add a line
(between |\childdocmain| and the above block)
%
\begin{center}
|%\ifchilddoc\||else\providecommand{\version}{draft}\||fi|
\end{center}
%
which can be uncommented to produce a draft version.
Likewise one can add a line to the very top of a child file
(above the |\childdocof{|\textit{main}|}| directive)
%
\begin{center}
|%\providecommand{\version}{final}|
\end{center}
%
which can be uncommented to produce the final version of this child document.

%%%%%%%%%%%%%%%%%%%%%%%%%%%%%%%%%%%%%%%%%%%%%%%%%%%%%%%%%%%%%%%%%%%%%%%%%%%%%%%%
\subsection{Forwarding}
\label{sec:forward}

Different versions of the main or child documents
using compilation flags as described in \secref{sec:flags}
can be (permanently) stored in different files
for convenient compilation, viewing and distribution.
To this end, the package defines a command
to pass on compilation to a different file:

%%%%%%%%%%%%%%%%%%%%%%%%%%%%%%%%%%%%%%%%
\DescribeMacro{\childdocforward}
The command |\childdocforward| redirects processing to
another source file:
%
\begin{center}
\begin{tabular}{l}
|\input{childdoc.def}|\\
|\childdocforward[|\textit{main}|]{|\textit{dest}|}|\\
\end{tabular}
\end{center}
%
The argument \textit{dest} is the destination file
(without extension).
It should be the main file or one of the child files.
Note that further \textsf{childdoc} directives
such as |\childdocof| and |\childdocforward|
in the indicated file will be processed in this form.
The optional argument \textit{main}
passes on directly to the main file \textit{main}
while pretending to compile the child \textit{dest}.
This form behaves as if \textit{dest}
issues |\childdocof{|\textit{main}|}| right away,
and no further \textsf{childdoc} directives will be processed.

%%%%%%%%%%%%%%%%%%%%%%%%%%%%%%%%%%%%%%%%
\DescribeMacro{\...prefix}
In the alternative form |\childdocforwardprefix|,
%
\begin{center}
\begin{tabular}{l}
|\input{childdoc.def}|\\
|\childdocforwardprefix[|\textit{main}|]{|\textit{prefix}|}{|\textit{dest}|}|
\end{tabular}
\end{center}
%
the destination file is determined by a pattern
depending on the current file:
To make this work, the current file must be called
`{\textit{prefix}\hspace{0.2em}\textit{suffix}}'
with \textit{prefix} matching precisely the argument.
Processing is then passed on to the file
`{\textit{dest}\hspace{0.2em}\textit{suffix}}'.
Surely, the same effect is achieved by
directly specifying the
argument `{\textit{dest}\hspace{0.2em}\textit{suffix}}'
in the first form.
However, that requires to set up a different file
for each child. With the alternative form of the command
all these files can have exactly the same content
which simplifies setting them up and maintaining them.

For example, the following file |draft.tex|
with a compilation flag |\version| as described in \secref{sec:flags}
compiles the main document as a draft:
%
\begin{center}
\begin{tabular}{l}
|\def\version{draft}|\\
|\input{childdoc.def}|\\
|\childdocforward{|\textit{main}|}|
\end{tabular}
\end{center}
%
Likewise, the following files |final|\textit{nn}|.tex|
compile the final version of the child document
|child|\textit{nn}|.tex|:
%
\begin{center}
\begin{tabular}{l}
|\def\version{final}|\\
|\input{childdoc.def}|\\
|\childdocforwardprefix{final}{child}|
\end{tabular}
\end{center}
%

Note that when several versions of a main file and/or of each child file
are to be generated, it may be convenient to set up a |Makefile| or
shell script to automatise the process.

%%%%%%%%%%%%%%%%%%%%%%%%%%%%%%%%%%%%%%%%%%%%%%%%%%%%%%%%%%%%%%%%%%%%%%%%%%%%%%%%
\subsection{Command Line Processing}
\label{sec:commandline}

The effect of redirection files can also be achieved by invoking
the \LaTeX{} compiler with a more elaborate command line.
Most conveniently this should be done as part
of a shell script or a |Makefile|.

When using \textsf{childdoc} in the main file, the following
command lines effectively perform a redirection
(note that depending on the shell being used,
backslashes may have to be doubled: `|\|' $\to$ `|\\|'):
%
\begin{center}
|... -jobname "|\textit{target}|" |\\|"|[\textit{flags}]%
|\input{childdoc.def}\childdocforward[|\textit{main}|]{|\textit{dest}|}"|
\end{center}
%
Here \textit{target} is the name of the output file,
\textit{main} is the name of the main file
and \textit{dest} is the name of the main or child file to be processed
(all filenames without extensions).
The optional argument \textit{main} can be omitted
if \textit{main} matches \textit{dest}.
Optionally, compilation \textit{flags} can be defined via |\def| commands.
This command line makes the \TeX{} engine believe
it is compiling the file \textit{target}
whose content is specified as the latter parameter.
The provided code then forwards the processing to
\textit{main} or \textit{dest} as described in \secref{sec:forward}.

%%%%%%%%%%%%%%%%%%%%%%%%%%%%%%%%%%%%%%%%%%%%%%%%%%%%%%%%%%%%%%%%%%%%%%%%%%%%%%%%
\subsection{Include by Input}
\label{sec:input}

Including child documents by |\include| has some restrictions by design.
Most notably, the content of a child document always occupies
its own set of pages; pages cannot be shared between child documents.
Usually, this behaviour makes perfect sense
because each child document contain an essential part of the document.
However, in some situations it may be desirable to compose
a document from a collection of parts
without having mandatory page breaks between then.
For this case, the package
provides a mechanism to include parts
by |\input| which can also be processed individually.
However, by construction this mechanism
requires manual handling of the content to be output.

%%%%%%%%%%%%%%%%%%%%%%%%%%%%%%%%%%%%%%%%
\DescribeMacro{\ifchilddocmanual}
The main file should be prepared as usual, see \secref{sec:include}.
However, the document body must make a distinction
between processing of an individual part and of the main document, e.g.:
%
\begin{center}
\begin{tabular}{l}
|\ifchilddocmanual|\\
|\input{\childdocname}|\\
|\||else|\\
\textit{document body with }|\input{|\textit{part}|}|\\
|\||fi|
\end{tabular}
\end{center}
%
The conditional |\ifchilddocmanual| is true whenever
a part to be included by |\input| is being compiled,
and the name of the part is stored in |\childdocname|.

%%%%%%%%%%%%%%%%%%%%%%%%%%%%%%%%%%%%%%%%
\DescribeMacro{\childdocby}
Each part to be included by |\input| should start with:
%
\begin{center}
\begin{tabular}{l}
|\input{childdoc.def}|\\
|\childdocby{|\textit{main}|}|\\
\end{tabular}
\end{center}
%
The directive |\childdocby| is similar to |\childdocof|
described in \secref{sec:include},
but the subsequent selection of content must be done manually.
To that end, both |\ifchilddoc| and |\ifchilddocmanual|
will be true upon processing of a part,
and the name of the part is stored in |\childdocname|.
Note that |\jobname| will be set to the filename of the current part
so that each part receives an individual |.aux| file
that does not interfere with the |.aux| file(s) of the main document.
This behaviour can be altered by the alternative form
|\childdocby[*]{|\textit{main}|}| (with a non-empty optional argument)
which uses the |.aux| file of the main document
by setting |\jobname| to \textit{main}.

%%%%%%%%%%%%%%%%%%%%%%%%%%%%%%%%%%%%%%%%%%%%%%%%%%%%%%%%%%%%%%%%%%%%%%%%%%%%%%%%
\subsection{Driver Development}
\label{sec:driver}

The \textsf{childdoc} mechanism can also be use for the development
of definition files such as \LaTeX{} styles or classes.
This case differs from the above setup with multiple parts
included by |\include| in that no |\includeonly| should be invoked.
This can be achieved by starting the include file
(before |\ProvidesPackage|) with:
%
\begin{center}
\begin{tabular}{l}
|\input{childdoc.def}|\\
|\childdocforward{|\textit{main}|}|\\
\end{tabular}
\end{center}
%
or alternatively with:
%
\begin{center}
\begin{tabular}{l}
|\input{childdoc.def}|\\
|\childdocby{|\textit{main}|}|\\
\end{tabular}
\end{center}
%
Both forms have slightly different effects as described above.
The main file is prepared as usual, see \secref{sec:include}.

%%%%%%%%%%%%%%%%%%%%%%%%%%%%%%%%%%%%%%%%%%%%%%%%%%%%%%%%%%%%%%%%%%%%%%%%%%%%%%%%
\subsection{Legacy Detection}
\label{sec:detection}

The directive |\childdocmain| in the main file can detect
whether the complete document or merely a child is to be compiled
even without using the directive |\childdocof|.
This method is deprecated because it is less robust
and there is no compelling reason to use it;
it is merely provided for backward compatibility
and it may be removed in future versions.

If the detection mechanism is to be used,
it is mandatory to correctly specify
the filename of the main file as the argument of |\childdocmain|:
%
\begin{center}
\begin{tabular}{l}
|\input{childdoc.def}|\\
|\childdocmain{|\textit{main}|}|\\
\end{tabular}
\end{center}
%
If |\jobname| does not match the argument \textit{main} of |\childdocmain|,
it is assumed that |\jobname| points to the child file to be compiled.
When using |\childdocmain| with the main file specified as argument,
it suffices to start a child file
with just |\input{|\textit{main}|}|
without loading of the package and using |\childdocof|.
If instead all processing is done
with the appropriate \textsf{childdoc} directives,
the argument of \textit{main} of |\childdocmain| can be empty.

An alternative version of the command line processing described
in \secref{sec:commandline} using the detection mechanism reads:
%
\begin{center}
|... -jobname "|\textit{target}|" "|[\textit{flags}]%
[|\def\jobname{|\textit{dest}|}|]|\input{|\textit{main}|}"|
\end{center}

%%%%%%%%%%%%%%%%%%%%%%%%%%%%%%%%%%%%%%%%%%%%%%%%%%%%%%%%%%%%%%%%%%%%%%%%%%%%%%%%
\subsection{Manual Code}
\label{sec:manual}

In case one cannot be certain whether the definitions file |childdoc.def|
is installed on the target \TeX{} distribution
and one prefers not to ship it,
it is conceivable to paste a few relevant commands into the sources.

To that end, drop all statements |\input{childdoc.def}|
and perform the replacements as outlined below.
Instead of |\childdocmain{|\textit{main}|}| add the following code
to the top of the main file:
%
\begin{center}
\begin{tabular}{l}
|\||ifdefined\childdocname\endinput\||fi\newif\ifchilddoc|\\
|\edef\childdocname{\scantokens\expandafter{\jobname\noexpand}}|\\
|\def\childdocmain{|\textit{main}|}\||ifx\childdocmain\childdocname\||else|\\
|\childdoctrue\includeonly{\childdocname}\let\jobname\childdocmain\||fi|\\
\end{tabular}
\end{center}
%
Instead of |\childdocof{|\textit{main}|}| just include the main file
at the top of each child file:
%
\begin{center}
|\input{|\textit{main}|}|
\end{center}
%
A simple redirection |\childdocforward{|\textit{dest}|}| is achieved by:
%
\begin{center}
|\def\jobname{|\textit{dest}|}\input{\jobname}|
\end{center}
%
The redirection with prefix
|\childdocforwardprefix[|\textit{prefix}|]{|\textit{dest}|}|
is accomplished by:
%
\begin{center}
\begin{tabular}{l}
|{\edef\jobname{\scantokens\expandafter{\jobname\noexpand}}|\\
|\def\redirectjob |\textit{prefix}|#1~~~{\gdef\jobname{|\textit{dest}|#1}}|\\
|\expandafter\redirectjob\jobname~~~}\input{\jobname}|
\end{tabular}
\end{center}

In an alternative approach,
child documents can be compiled by a specific command line
without additional code or specific definitions:
%
\begin{center}
|... -jobname "|\textit{target}|" "|[\textit{flags}]%
|\includeonly{|\textit{dest}|}\input{|\textit{main}|}"|
\end{center}
%

%%%%%%%%%%%%%%%%%%%%%%%%%%%%%%%%%%%%%%%%%%%%%%%%%%%%%%%%%%%%%%%%%%%%%%%%%%%%%%%%
%%%%%%%%%%%%%%%%%%%%%%%%%%%%%%%%%%%%%%%%%%%%%%%%%%%%%%%%%%%%%%%%%%%%%%%%%%%%%%%%
\section{Information}

%%%%%%%%%%%%%%%%%%%%%%%%%%%%%%%%%%%%%%%%%%%%%%%%%%%%%%%%%%%%%%%%%%%%%%%%%%%%%%%%
\subsection{Copyright}

Copyright \copyright{} 2017--2018 Niklas Beisert

This work may be distributed and/or modified under the
conditions of the \LaTeX{} Project Public License, either version 1.3
of this license or (at your option) any later version.
The latest version of this license is in
  \url{http://www.latex-project.org/lppl.txt}
and version 1.3 or later is part of all distributions of \LaTeX{}
version 2005/12/01 or later.

This work has the LPPL maintenance status `maintained'.

The Current Maintainer of this work is Niklas Beisert.

This work consists of the files |README.txt|, |childdoc.ins| and |childdoc.dtx|
as well as the derived files |childdoc.def|, |cdocsamp.tex|
with |cdocsch1.tex|, |cdocsch2.tex|, |cdocspt3.tex|, |cdocspt4.tex|,
|cdocsdrf.tex|, |cdocsfn1.tex|, |cdocsfn2.tex|
as well as |childdoc.pdf|.

%%%%%%%%%%%%%%%%%%%%%%%%%%%%%%%%%%%%%%%%%%%%%%%%%%%%%%%%%%%%%%%%%%%%%%%%%%%%%%%%
\subsection{Files and Installation}

The package consists of the files:
%
\begin{center}
\begin{tabular}{ll}
    |README.txt|   & readme file \\
    |childdoc.ins| & installation file \\
    |childdoc.dtx| & source file \\
    |childdoc.def| & definition file \\
    |cdocsamp.tex| & sample main file \\
    |cdocsch1.tex| & sample include file \\
    |cdocsch2.tex| & sample include file \\
    |cdocspt3.tex| & sample part file \\
    |cdocspt4.tex| & sample part file \\
    |cdocsdrf.tex| & sample redirection file \\
    |cdocsfn1.tex| & sample redirection file \\
    |cdocsfn2.tex| & sample redirection file \\
    |childdoc.pdf| & manual
\end{tabular}
\end{center}
%
The distribution consists of the files
|README.txt|, |childdoc.ins| and |childdoc.dtx|.
%
\begin{itemize}
\item
Run (pdf)\LaTeX{} on |childdoc.dtx|
to compile the manual |childdoc.pdf| (this file).
\item
Run \LaTeX{} on |childdoc.ins| to create the definitions file |childdoc.def|
and the sample |cdocsamp.tex| with include files
|cdocsch1.tex|, |cdocsch2.tex|, |cdocspt3.tex|, |cdocspt4.tex|,
|cdocsdrf.tex|, |cdocsfn1.tex|, |cdocsfn2.tex|.
Then copy the file |childdoc.def| to an appropriate directory of your \LaTeX{}
distribution, e.g.\ \textit{texmf-root}|/tex/latex/childdoc|.
\end{itemize}

%%%%%%%%%%%%%%%%%%%%%%%%%%%%%%%%%%%%%%%%%%%%%%%%%%%%%%%%%%%%%%%%%%%%%%%%%%%%%%%%
\subsection{Related CTAN Packages}

There are several other packages which offer a similar functionality:
%
\begin{itemize}
\item
The packages
\href{http://ctan.org/pkg/docmute}{\textsf{docmute}},
\href{http://ctan.org/pkg/includex}{\textsf{includex}} and
\href{http://ctan.org/pkg/standalone}{\textsf{standalone}}
provide commands to include only the document body of
a child file thus allowing both files to be compiled individually.
\item
The packages \href{http://ctan.org/pkg/subdocs}{\textsf{subdocs}}
and \href{http://ctan.org/pkg/subfiles}{\textsf{subfiles}}
provide structures in which the main and child documents can be
encapsulated and allowing them to be compiled individually.
The inclusion mechanism is different from the conventional |\include|.
\item
The package \href{http://ctan.org/pkg/combine}{\textsf{combine}}
is an elaborate solution to combine several documents into one.
\end{itemize}
%
See also the CTAN topic \href{http://ctan.org/topic/subdocs}{\textsf{subdocs}}
for further related packages.
The present package differs from the above solutions in that
a document structure constructed with the conventional |\include| mechanism
just needs two extra commands at the top of every file
such that all constituent files can be compiled individually.

%%%%%%%%%%%%%%%%%%%%%%%%%%%%%%%%%%%%%%%%%%%%%%%%%%%%%%%%%%%%%%%%%%%%%%%%%%%%%%%%
%\subsection{Feature Suggestions}
%
%The following is a list of features which may be useful for future
%versions of this package:
%%
%\begin{itemize}
%\item
%\ldots
%\end{itemize}

%%%%%%%%%%%%%%%%%%%%%%%%%%%%%%%%%%%%%%%%%%%%%%%%%%%%%%%%%%%%%%%%%%%%%%%%%%%%%%%%
\subsection{Revision History}

%%%%%%%%%%%%%%%%%%%%%%%%%%%%%%%%%%%%%%%%
\paragraph{v2.0:} 2018/12/30

\begin{itemize}
\item
immediate forward processing
\item
added |\childdocby| mechanism
\item
manual restructured
\end{itemize}

%%%%%%%%%%%%%%%%%%%%%%%%%%%%%%%%%%%%%%%%
\paragraph{v1.6:} 2018/01/17

\begin{itemize}
\item
application for development of include files
\item
corrections to manual
\end{itemize}

%%%%%%%%%%%%%%%%%%%%%%%%%%%%%%%%%%%%%%%%
\paragraph{v1.5:} 2017/05/21

\begin{itemize}
\item
more complete structuring introduced
\item
|\childdocof| introduced
\item
|\childdoc| renamed to |\childdocmain|
\item
|\childredirect| renamed to |\childdocforward| and |\childdocforwardprefix|
and functionality expanded
\end{itemize}

%%%%%%%%%%%%%%%%%%%%%%%%%%%%%%%%%%%%%%%%
\paragraph{v1.0:} 2017/04/27

\begin{itemize}
\item
manual and install package
\item
first version published on CTAN
\end{itemize}

%%%%%%%%%%%%%%%%%%%%%%%%%%%%%%%%%%%%%%%%
\paragraph{v0.6:} 2017/04/26

\begin{itemize}
\item
redirection mechanism added
\end{itemize}

%%%%%%%%%%%%%%%%%%%%%%%%%%%%%%%%%%%%%%%%
\paragraph{v0.5:} 2017/04/26

\begin{itemize}
\item
functionality in definition file
\end{itemize}


%%%%%%%%%%%%%%%%%%%%%%%%%%%%%%%%%%%%%%%%%%%%%%%%%%%%%%%%%%%%%%%%%%%%%%%%%%%%%%%%
%%%%%%%%%%%%%%%%%%%%%%%%%%%%%%%%%%%%%%%%%%%%%%%%%%%%%%%%%%%%%%%%%%%%%%%%%%%%%%%%
%%%%%%%%%%%%%%%%%%%%%%%%%%%%%%%%%%%%%%%%%%%%%%%%%%%%%%%%%%%%%%%%%%%%%%%%%%%%%%%%
\appendix

\settowidth\MacroIndent{\rmfamily\scriptsize 000\ }

 \DocInput{childdoc.dtx}

\end{document}
%</driver>
% \fi
%
% %%%%%%%%%%%%%%%%%%%%%%%%%%%%%%%%%%%%%%%%%%%%%%%%%%%%%%%%%%%%%%%%%%%%%%%%%%%%%%
% %%%%%%%%%%%%%%%%%%%%%%%%%%%%%%%%%%%%%%%%%%%%%%%%%%%%%%%%%%%%%%%%%%%%%%%%%%%%%%
% \section{Sample}
%\iffalse
%<*samplemain>
%\fi
%
% The following presents a sample document
% with two chapters, two parts, a title page,
% a compile flag as well as three forwarding files to set the flag.
% It consists of eight |.tex| files:
% \begin{center}
% \begin{tabular}{ll}
% |cdocsamp.tex|&main file\\
% |cdocsch1.tex|&include file for chapter 1\\
% |cdocsch2.tex|&include file for chapter 2\\
% |cdocspt3.tex|&include file for part 3\\
% |cdocspt4.tex|&include file for part 4\\
% |cdocsdrf.tex|&forwarding file for main file in draft mode\\
% |cdocsfi1.tex|&forwarding file for final version of chapter 1\\
% |cdocsfi2.tex|&forwarding file for final version of chapter 2\\
% \end{tabular}
% \end{center}
% Each of the eight files can be compiled directly by the \LaTeX{} compiler.
%
% %%%%%%%%%%%%%%%%%%%%%%%%%%%%%%%%%%%%%%
% \paragraph{Main File.}
%
% The main file is called |cdocsamp.tex|.
%
% Load the \textsf{childdoc} definitions and
% declare the filename for the main document:
%    \begin{macrocode}
\input{childdoc.def}
\childdocmain{}
%    \end{macrocode}

% Optional override for |\version| flag:
%    \begin{macrocode}
%%\ifchilddoc\else\providecommand{\version}{draft}\fi
%    \end{macrocode}

% Define the default values for the |\version| flag
% (|final| for the main file and |draft| for childs):
%    \begin{macrocode}
\ifchilddoc
\providecommand{\version}{draft}
\else
\providecommand{\version}{final}
\fi
%    \end{macrocode}

% Load the standard document class:
%    \begin{macrocode}
\documentclass[12pt]{article}
%    \end{macrocode}

% Start the document body:
%    \begin{macrocode}
\begin{document}
%    \end{macrocode}

% Declare a title page.
% Print title, part of document being processed and version flag:
%    \begin{macrocode}
\addtocounter{page}{-1}
\begin{center}
{\LARGE\bfseries{}childdoc example\par}
\vspace{1cm}
\ifchilddoc
\ifchilddocmanual part\else chapter\fi:
`\childdocname' of `\childdocjob'\par
\else
main document: `\childdocjob'\par
\fi
version: \version\par
\end{center}
\newpage
%    \end{macrocode}

% Manually include selected file,
% otherwise process as usual:
%    \begin{macrocode}
\ifchilddocmanual
\section*{part `\childdocname'}
\input{\childdocname}
\else
%    \end{macrocode}

% Include the two chapters:
%    \begin{macrocode}
\include{cdocsch1}
\include{cdocsch2}
%    \end{macrocode}

% Include the two parts unless only chapters should be displayed:
%    \begin{macrocode}
\ifchilddoc\else
\section{part three}
\input{cdocspt3}
\section{part four}
\input{cdocspt4}
\fi
%    \end{macrocode}

% Process as usual until here:
%    \begin{macrocode}
\fi
%    \end{macrocode}

% End of document body:
%    \begin{macrocode}
\end{document}
%    \end{macrocode}
%\iffalse
%</samplemain>
%\fi
%
% %%%%%%%%%%%%%%%%%%%%%%%%%%%%%%%%%%%%%%
% \paragraph{Chapter Include Files.}
%
% The include files are called |cdocsch1.tex| and |cdocsch2.tex|.
%
%\iffalse
%<*samplechap1|samplechap2>
%\fi

% Optional override for |\version| flag:
%    \begin{macrocode}
%%\providecommand{\version}{final}
%    \end{macrocode}

% Include the main document:
%    \begin{macrocode}
\input{childdoc.def}
\childdocof{cdocsamp}
%    \end{macrocode}

%\iffalse
%</samplechap1|samplechap2>
%\fi
%
%\iffalse
%<*samplechap1>
%\fi
% Some text for chapter 1:
%    \begin{macrocode}
\section{one}
some text in chapter one
%    \end{macrocode}

%\iffalse
%</samplechap1>
%\fi
% Some text for chapter 2:
%\iffalse
%<*samplechap2>
%\fi
%    \begin{macrocode}
\section{two}
more text in chapter two
%    \end{macrocode}

%\iffalse
%</samplechap2>
%\fi
%
% %%%%%%%%%%%%%%%%%%%%%%%%%%%%%%%%%%%%%%
% \paragraph{Part Include Files.}
%
% The include files are called |cdocspt3.tex| and |cdocspt4.tex|.
%
%\iffalse
%<*samplepart3|samplepart4>
%\fi

% Optional override for |\version| flag:
%    \begin{macrocode}
%%\providecommand{\version}{final}
%    \end{macrocode}

% Include the main document:
%    \begin{macrocode}
\input{childdoc.def}
\childdocby{cdocsamp}
%    \end{macrocode}

%\iffalse
%</samplepart3|samplepart4>
%\fi
%
%\iffalse
%<*samplepart3>
%\fi
% Some text for part 3:
%    \begin{macrocode}
some text in part three
%    \end{macrocode}

%\iffalse
%</samplepart3>
%\fi
% Some text for part 4:
%\iffalse
%<*samplepart4>
%\fi
%    \begin{macrocode}
more text in part four
%    \end{macrocode}

%\iffalse
%</samplepart4>
%\fi
%
% %%%%%%%%%%%%%%%%%%%%%%%%%%%%%%%%%%%%%%
% \paragraph{Forwarding for a Complete Draft.}
%
% The following forwarding file |cdocsdrf.tex|
% compiles the main document in draft mode:
%\iffalse
%<*sampledraft>
%\fi
%    \begin{macrocode}
\def\version{draft}
\input{childdoc.def}
\childdocforward{cdocsamp}
%    \end{macrocode}

%\iffalse
%</sampledraft>
%\fi
%
% %%%%%%%%%%%%%%%%%%%%%%%%%%%%%%%%%%%%%%
% \paragraph{Forwarding for Final Version of the Chapters.}
%
% The following forwarding files |cdocsfn1.tex| and |cdocsfn2.tex|
% (with identical content)
% compile the final versions of the child documents
% |cdocsch1.tex| and |cdocsch2.tex|, respectively:
%\iffalse
%<*samplefinal>
%\fi
%    \begin{macrocode}
\def\version{final}
\input{childdoc.def}
\childdocforwardprefix[cdocsamp]{cdocsfn}{cdocsch}
%    \end{macrocode}

%\iffalse
%</samplefinal>
%\fi
%
% %%%%%%%%%%%%%%%%%%%%%%%%%%%%%%%%%%%%%%
% \paragraph{Command Line Processing.}
%
% The following three command lines generate the output files
% |cdocscld|, |cdocscl1| and |cdocscl2|
% which should be identical to
% |cdocsdrf|, |cdocsch1| and |cdocsfn2|, respectively:
% \begin{center}
% \begin{tabular}{l}
% |latex -jobname cdocscld \|\\
% |  "\def\version{draft}\input{childdoc.def}\childdocforward{cdocsamp}"|\\
% |latex -jobname cdocscl1 \|\\
% |  "\input{childdoc.def}\childdocforward[cdocsamp]{cdocsch1}"|\\
% |latex -jobname cdocscl2 \|\\
% |  "\def\version{final}\input{childdoc.def}\childdocforward{cdocsch2}"|
% \end{tabular}
% \end{center}
% Note that the trailing backslash on each first line
% merely continues the input to the second line
% (for convenient cut ant paste).
% Furthermore, the command |latex| can be replaced by any
% of its alternative versions such as |pdflatex|.
%
% %%%%%%%%%%%%%%%%%%%%%%%%%%%%%%%%%%%%%%%%%%%%%%%%%%%%%%%%%%%%%%%%%%%%%%%%%%%%%%
% %%%%%%%%%%%%%%%%%%%%%%%%%%%%%%%%%%%%%%%%%%%%%%%%%%%%%%%%%%%%%%%%%%%%%%%%%%%%%%
% \section{Implementation}
%\iffalse
%<*package>
%\fi
%
% This section describes the definitions file |childdoc.def|.

% The definitions cannot be loaded using |\usepackage| or |\RequirePackage|
% which has a mechanism to prevent loading a style file more than once.
% When loading the definitions by means of |\input|
% multiple instances have to be prevented manually:
%\iffalse
%This code needs to be before the `\ProvidesFile' directive
%which is defined at the beginning of this file.
%Therefore it is also placed there and commented out here.
%</package>
%<*discard>
%\fi
%    \begin{macrocode}
\ifdefined\childdocmain\endinput\fi
%    \end{macrocode}
%\iffalse
%</discard>
%<*package>
%\fi
%
% \macro{\ifchilddoc}
% \macro{\ifchilddocmanual}
% The conditional |\ifchilddoc| tells whether a
% child (true) or main (false) document is being compiled.
% The conditional |\ifchilddocmanual| tells whether
% the |\includeonly| mechanism is used (false) or
% the selection of child files must be performed manually (true).
% The definitions initialise to false:
%    \begin{macrocode}
\newif\ifchilddoc
\newif\ifchilddocmanual
%    \end{macrocode}

% \macro{\childdocname}
% \macro{\childdocjob}
% The macro |\childdocname| stores the name of the main document
% to be compiled. The macro |\childdocjob| stores the name of
% the document on which the \LaTeX{} compiler was originally invoked.
% The content of |\jobname| cannot be compared
% to filenames specified in the source due to different catcodes.
% The following code rescans |\jobname|, stores the result
% in |\childdocname| and saves a copy in |\childdocjob|:
%    \begin{macrocode}
\edef\childdocname{\scantokens\expandafter{\jobname\noexpand}}
\let\childdocjob\childdocname
%    \end{macrocode}

% \macro{\childdocdisable}
% The macro |\childdocdisable| prevents the main file
% from being processed more than once.
% At this stage, the main document command |\childdocmain|
% is assumed to be called once again where it should do nothing.
% Any subsequent call to it should prevent
% a secondary processing of the main document
% It overwrites the forwarding commands
% |\childdocof| and |\childdocforward|
% with empty macros to prevent further inclusions of the main document:
%    \begin{macrocode}
\newcommand{\childdocdisable}
{
  \renewcommand{\childdocmain}[1]{\renewcommand{\childdocmain}[1]{\endinput}}
  \renewcommand{\childdocof}[1]{}
  \renewcommand{\childdocby}[2][]{}
  \renewcommand{\childdocforward}[2][]{}
  \renewcommand{\childdocdisable}{}
}
%    \end{macrocode}

% \macro{\childdocmain}
% The macro |\childdocmain| is to be called at the top of the main file
% with nothing or the main filename (without extension) as argument.
% First, it breaks loops.
% If the argument is not empty and does not match |\childdocname|
% (which is set by the first inclusion of |childdoc.def|),
% |\ifchilddoc| is set to true, |\includeonly| is applied to the child file
% and |\jobname| is set to the main file
% (for proper handling of |.aux| files):
%    \begin{macrocode}
\newcommand{\childdocmain}[1]
{
  \childdocdisable\childdocmain{}
  \if?#1?\else
    \begingroup
      \def\childdoctmp{#1}
      \ifx\childdoctmp\childdocname
        \def\childdoctmp{}
      \else
        \def\childdoctmp
        {
          \childdoctrue
          \includeonly{\childdocname}
          \def\childdocjob{#1}
          \def\jobname{#1}
        }
      \fi
      \expandafter
    \endgroup
    \childdoctmp
  \fi
}
%    \end{macrocode}

% \macro{\childdocof}
% The command |\childdocof| redirects
% compilation to the main file |#1|.
%    \begin{macrocode}
\newcommand{\childdocof}[1]
{
  \childdocdisable
  \childdoctrue
  \includeonly{\childdocname}
  \def\jobname{#1}
  \def\childdocjob{#1}
  \input{#1}
}
%    \end{macrocode}

% \macro{\childdocby}
% The command |\childdocby| ....
%    \begin{macrocode}
\newcommand{\childdocby}[2][]
{
  \childdocdisable
  \childdoctrue
  \childdocmanualtrue
  \if?#1?\else
    \def\jobname{#2}
  \fi
  \def\childdocjob{#2}
  \input{#2}
  \endinput
}
%    \end{macrocode}

% \macro{\childdocforward}
% The command |\childdocforward| redirects
% compilation to the main file or
% (if the optional argument is given) a child file.
% Parameters are set as if the main file
% or a child file starting with |\childdocof| was compiled.
% Then compilation is handed over to the main file:
%    \begin{macrocode}
\newcommand{\childdocforward}[2][]
{
  \begingroup
    \if?#1?
      \def\childdoctmp
      {
        \def\childdocname{#2}
        \def\childdocjob{#2}
        \def\jobname{#2}
        \input{#2}
        \endinput
      }
    \else
      \def\childdoctmp
      {
        \childdocdisable
        \def\childdocname{#2}
        \childdoctrue
        \includeonly{#2}
        \def\childdocjob{#1}
        \def\jobname{#1}
        \input{#1}
        \endinput
      }
    \fi
    \expandafter
  \endgroup
  \childdoctmp
}
%    \end{macrocode}

% \macro{\childdocforwardprefix}
% The command |\childdocforwardprefix| redirects
% compilation to the main or a child file by means of a pattern.
% The prefix |#1| in the current filename is replaced by |#2|
% and the suffix of the current filename is kept
% (it is assumed that the filename does not contain the substring `|~~~|'
% which is used as a delimiter).
% Compilation is handed over to the new file by |\childdocforward|:
%    \begin{macrocode}
\newcommand{\childdocforwardprefix}[3][]
{
  \begingroup
    \def\childdocextract #2##1~~~{\def\childdoctmp{\childdocforward[#1]{#3##1}}}
    \expandafter\childdocextract\childdocname~~~
    \expandafter
  \endgroup
  \childdoctmp
}
%    \end{macrocode}

% \macro{\childdoc}
% The deprecated macro |\childdoc| is a legacy version of |\childdocmain|:
%    \begin{macrocode}
\newcommand{\childdoc}{\childdocmain}
%    \end{macrocode}

% \macro{\childdocredirect}
% The deprecated macro |\childdocredirect| is a legacy version
% of |\childdocforward| and |\childdocforwardprefix|:
%    \begin{macrocode}
\newcommand{\childdocredirect}[2][]
{
  \begingroup
    \if?#1?
      \def\childdoctmp{\childdocforward{#2}}
    \else
      \def\childdoctmp{\childdocforwardprefix{#1}{#2}}
    \fi
    \expandafter
  \endgroup
  \childdoctmp
}
%    \end{macrocode}

%\iffalse
%</package>
%\fi
%
\endinput

\childdocmain{}
%    \end{macrocode}

% Optional override for |\version| flag:
%    \begin{macrocode}
%%\ifchilddoc\else\providecommand{\version}{draft}\fi
%    \end{macrocode}

% Define the default values for the |\version| flag
% (|final| for the main file and |draft| for childs):
%    \begin{macrocode}
\ifchilddoc
\providecommand{\version}{draft}
\else
\providecommand{\version}{final}
\fi
%    \end{macrocode}

% Load the standard document class:
%    \begin{macrocode}
\documentclass[12pt]{article}
%    \end{macrocode}

% Start the document body:
%    \begin{macrocode}
\begin{document}
%    \end{macrocode}

% Declare a title page.
% Print title, part of document being processed and version flag:
%    \begin{macrocode}
\addtocounter{page}{-1}
\begin{center}
{\LARGE\bfseries{}childdoc example\par}
\vspace{1cm}
\ifchilddoc
\ifchilddocmanual part\else chapter\fi:
`\childdocname' of `\childdocjob'\par
\else
main document: `\childdocjob'\par
\fi
version: \version\par
\end{center}
\newpage
%    \end{macrocode}

% Manually include selected file,
% otherwise process as usual:
%    \begin{macrocode}
\ifchilddocmanual
\section*{part `\childdocname'}
\input{\childdocname}
\else
%    \end{macrocode}

% Include the two chapters:
%    \begin{macrocode}
\include{cdocsch1}
\include{cdocsch2}
%    \end{macrocode}

% Include the two parts unless only chapters should be displayed:
%    \begin{macrocode}
\ifchilddoc\else
\section{part three}
\input{cdocspt3}
\section{part four}
\input{cdocspt4}
\fi
%    \end{macrocode}

% Process as usual until here:
%    \begin{macrocode}
\fi
%    \end{macrocode}

% End of document body:
%    \begin{macrocode}
\end{document}
%    \end{macrocode}
%\iffalse
%</samplemain>
%\fi
%
% %%%%%%%%%%%%%%%%%%%%%%%%%%%%%%%%%%%%%%
% \paragraph{Chapter Include Files.}
%
% The include files are called |cdocsch1.tex| and |cdocsch2.tex|.
%
%\iffalse
%<*samplechap1|samplechap2>
%\fi

% Optional override for |\version| flag:
%    \begin{macrocode}
%%\providecommand{\version}{final}
%    \end{macrocode}

% Include the main document:
%    \begin{macrocode}
% \iffalse
%
% childdoc.dtx Copyright (C) 2017-2018 Niklas Beisert
%
% This work may be distributed and/or modified under the
% conditions of the LaTeX Project Public License, either version 1.3
% of this license or (at your option) any later version.
% The latest version of this license is in
%   http://www.latex-project.org/lppl.txt
% and version 1.3 or later is part of all distributions of LaTeX
% version 2005/12/01 or later.
%
% This work has the LPPL maintenance status `maintained'.
%
% The Current Maintainer of this work is Niklas Beisert.
%
% This work consists of the files childdoc.dtx and childdoc.ins
% and the derived files childdoc.def and cdocsamp.tex with
% cdocsch1.tex, cdocsch2.tex, cdocsdrf.tex, cdocsfn1.tex, cdocsfn2.tex.
%
%<package>\ifdefined\childdocmain\endinput\fi
%<package>\ProvidesFile{childdoc.def}[2018/12/30 v2.0 child document driver]
%<samplemain>\ProvidesFile{cdocsamp.tex}[2018/12/30 v2.0 sample for childdoc]
%<*driver>
%\ProvidesFile{childdoc.drv}[2018/12/30 v2.0 childdoc reference manual file]
\PassOptionsToClass{10pt,a4paper}{article}
\documentclass{ltxdoc}

\usepackage[margin=35mm]{geometry}
\usepackage{hyperref}
\usepackage{hyperxmp}
\usepackage[usenames]{color}

\hypersetup{colorlinks=true}
\hypersetup{pdfstartview=FitH}
\hypersetup{pdfpagemode=UseNone}
\hypersetup{pdfsource={}}
\hypersetup{pdflang={en-UK}}
\hypersetup{pdfcopyright={Copyright 2017-2018 Niklas Beisert.
  This work may be distributed and/or modified under the
  conditions of the LaTeX Project Public License, either version 1.3
  of this license or (at your option) any later version.}}
\hypersetup{pdflicenseurl={http://www.latex-project.org/lppl.txt}}
\hypersetup{pdfcontactaddress={ETH Zurich, ITP, HIT K,
  Wolfgang-Pauli-Strasse 27}}
\hypersetup{pdfcontactpostcode={8093}}
\hypersetup{pdfcontactcity={Zurich}}
\hypersetup{pdfcontactcountry={Switzerland}}
\hypersetup{pdfcontactemail={nbeisert@itp.phys.ethz.ch}}
\hypersetup{pdfcontacturl={http://people.phys.ethz.ch/\xmptilde nbeisert/}}

\newcommand{\secref}[1]{\hyperref[#1]{section \ref*{#1}}}

\parskip1ex
\parindent0pt
\let\olditemize\itemize
\def\itemize{\olditemize\parskip0pt}

\begin{document}

\title{The \textsf{childdoc} Package}
\hypersetup{pdftitle={The childdoc Package}}
\author{Niklas Beisert\\[2ex]
  Institut f\"ur Theoretische Physik\\
  Eidgen\"ossische Technische Hochschule Z\"urich\\
  Wolfgang-Pauli-Strasse 27, 8093 Z\"urich, Switzerland\\[1ex]
  \href{mailto:nbeisert@itp.phys.ethz.ch}
  {\texttt{nbeisert@itp.phys.ethz.ch}}}
\hypersetup{pdfauthor={Niklas Beisert}}
\hypersetup{pdfsubject={Manual for the LaTeX2e Package childdoc}}
\date{30 December 2018, \textsf{v2.0}}
\maketitle

\begin{abstract}\noindent
\textsf{childdoc} is a \LaTeXe{} package
that enables the direct compilation
of document sections included by |\include|
to individual files.
\end{abstract}

\begingroup
\parskip0ex
\tableofcontents
\endgroup

%%%%%%%%%%%%%%%%%%%%%%%%%%%%%%%%%%%%%%%%%%%%%%%%%%%%%%%%%%%%%%%%%%%%%%%%%%%%%%%%
%%%%%%%%%%%%%%%%%%%%%%%%%%%%%%%%%%%%%%%%%%%%%%%%%%%%%%%%%%%%%%%%%%%%%%%%%%%%%%%%
\section{Introduction}

\LaTeX{} provides a mechanism to structure a large document (such as a book)
into a main file and several child files (containing the chapters)
using the |\include| command.
This mechanism is beneficial for documents
which span hundreds of pages in order to
make the source file(s) more manageable.
Moreover, compilation can be restricted to
selected child files by means of the |\includeonly| command.
The latter feature can be used to reduce the compilation time while editing
(this was significantly more useful in the earlier days of \LaTeX{})
or to generate a smaller document which is easier to navigate.
Another application of |\includeonly| is to generate
documents consisting of selected parts of the complete document.

However, there are a few drawbacks of the plain |\include| mechanism:
\begin{itemize}
\item
The child files cannot be compiled on their own,
they can only be compiled via the main file.
A naive editing environment
(such as a text editor with an option
to have the current file processed by \LaTeX)
may require one to switch to the main file before compiling;
attempting to compile the child file produces errors.
\item
The main file must be modified (each time)
to adjust the |\includeonly| command
to the present needs. This easily leaves the main file in a messy state.
\item
The generated document will always carry the filename
of the main document. This is inconvenient if
several child files are to be compiled and
to be kept for distribution.
\end{itemize}

The present package provides a simple interface
to make child files individually compilable by \LaTeX{}.
Compiling a child file then has the same effect as compiling
the main file with an |\includeonly| command
to select the appropriate child.
Moreover the generated document will carry the name of the child
rather than the main file.
This resolves all three above issues.

This feature is meant to make the editing of books,
thesis documents and lecture notes somewhat more convenient.
However, the package can also be used efficiently for
composing a series of documents (such as exercise sheets)
which are typically distributed individually.
It then assists the author in generating the individual documents
(potentially in different versions)
as well as a document containing the collected series.
Another application is in developing style files
or other kinds of included material
where compilation of the style file could redirect
to a sample or test file.

%%%%%%%%%%%%%%%%%%%%%%%%%%%%%%%%%%%%%%%%%%%%%%%%%%%%%%%%%%%%%%%%%%%%%%%%%%%%%%%%
%%%%%%%%%%%%%%%%%%%%%%%%%%%%%%%%%%%%%%%%%%%%%%%%%%%%%%%%%%%%%%%%%%%%%%%%%%%%%%%%
\section{Usage}

First of all, the package \textsf{childdoc} is \emph{not} a standard
\LaTeXe{} |.sty| style file! Therefore it needs to be invoked in
a non-standard way.

%%%%%%%%%%%%%%%%%%%%%%%%%%%%%%%%%%%%%%%%%%%%%%%%%%%%%%%%%%%%%%%%%%%%%%%%%%%%%%%%
\subsection{Included Files}
\label{sec:include}

%%%%%%%%%%%%%%%%%%%%%%%%%%%%%%%%%%%%%%%%
\DescribeMacro{\childdocmain}
To use the package, add the commands
\begin{center}
\begin{tabular}{l}
|\input{childdoc.def}|\\
|\childdocmain{}|\\
\end{tabular}
\end{center}
at the very top of the main \LaTeX{} file,
in particular \emph{before} the |\documentclass| statement!
The argument of |\childdocmain| should be left empty
(but it must be present).

%%%%%%%%%%%%%%%%%%%%%%%%%%%%%%%%%%%%%%%%
\DescribeMacro{\childdocof}
Furthermore, add the commands
\begin{center}
\begin{tabular}{l}
|\input{childdoc.def}|\\
|\childdocof{|\textit{main}|}|\\
\end{tabular}
\end{center}
at the top of every child file \textit{child}
which is included by |\include{|\textit{child}|}|
from within the main file
(or at least for those files to be compiled individually).
The argument \textit{main} must be the filename of the main file.

There are a couple of
considerations in setting up the main and child documents:

%%%%%%%%%%%%%%%%%%%%%%%%%%%%%%%%%%%%%%%%
\paragraph{Restrictions.}

Please note the following restrictions:
\begin{itemize}
\item
|\childdocmain| must be called with one argument \textit{main}
to ensure compatibility with earlier version of the package.
It must either be empty (|\childdocmain{}|)
or precisely match the filename of the main file in which it is specified.
See \secref{sec:detection} for further information.
\item
The filename \textit{main} must be specified without the |.tex| extension.
\item
The filename \textit{main} is case sensitive
(even in case-insensitive file systems)
due to internal string comparison.
\item
The argument \textit{main} should be fully expanded, it cannot be a macro.
\item
Subdirectories and special characters should be avoided in filenames.
\item
The command |\childdocmain{|\textit{main}|}| must be followed by a whitespace.
It should not be followed immediately by another command
or by a comment mark `|%|'.
This is because the \TeX{} parser reads the token immediately following
the argument of |\childdocmain| and puts it
at the beginning of every child section;
however, a white\-space is ignored.
\end{itemize}

%%%%%%%%%%%%%%%%%%%%%%%%%%%%%%%%%%%%%%%%
\paragraph{Content of Main File.}

It is advisable to place all content in the child files included by |\include|.
Any output contained in the main file will appear in all child documents
unless suppressed manually;
it cannot be suppressed automatically by the |\includeonly| directive
and thus should normally be avoided.
A method to include some content in the main file
by means of conditional processing is described in \secref{sec:conditional}.

%%%%%%%%%%%%%%%%%%%%%%%%%%%%%%%%%%%%%%%%
\paragraph{Page Numbering.}

When only a part of the document is compiled,
the appropriate numbering of pages
(as well as other status parameters)
is determined from the |.aux| files.
The latter contain information from previous passes.
However this information needs to propagate through
all intermediate child documents.
Therefore the page numbering in child documents may well
be inconsistent until the complete document is compiled at least once.

A useful (if unconventional) way to always ensure a consistent
page numbering is to restart the numbering in each child document
and denote the pages by `\textit{child}|.|\textit{page}'
where \textit{child} represents the chapter/section number of the child file.
This can be achieved by the command
|\numberwithin{page}{|\textit{child}|}|
of the \textsf{amsmath} package
where \textit{child} can be |chapter| or |section|
depending on the chosen structuring.
Alternatively, one can modify the macro |\thepage| appropriately
and reset the counter |page| at the start of each child file.

%%%%%%%%%%%%%%%%%%%%%%%%%%%%%%%%%%%%%%%%%%%%%%%%%%%%%%%%%%%%%%%%%%%%%%%%%%%%%%%%
\subsection{Conditional Processing}
\label{sec:conditional}

The package provides a mechanism to compile different versions
of a document. To customise the versions further some conditional processing
can come in handy to distinguish which version is being compiled.
The package provides two macros to describe the compilation context:

%%%%%%%%%%%%%%%%%%%%%%%%%%%%%%%%%%%%%%%%
\DescribeMacro{\ifchilddoc}
The conditional |\ifchilddoc| distinguishes between the compilation of
child documents and the main document:
%
\begin{center}
|\ifchilddoc |\textit{child-code}| |[|\||else |\textit{main-code}]| \||fi|
\end{center}

%%%%%%%%%%%%%%%%%%%%%%%%%%%%%%%%%%%%%%%%
\DescribeMacro{\childdocname}
\DescribeMacro{\childdocjob}
The macro |\childdocname| contains the filename (without extension)
of the main or child file being processed.
Note that |\childdocjob| will always contain the name of the main file.

%%%%%%%%%%%%%%%%%%%%%%%%%%%%%%%%%%%%%%%%
\paragraph{Title Page.}

Conditional processing can be used to include a title or banner page
in the main document when proper precautions are taken.
Importantly, the code in the main file should ensure that the page counter
(as well as other status parameters which are stored in the |.aux| files)
takes the same value after the conditional processing.
Otherwise the page numbers may take divergent values
depending on which part is compiled.

For example, a title page could be declared by:
%
\begin{center}
\begin{tabular}{l}
|\ifchilddoc\||else|\\
|\addtocounter{page}{-1}|\\
\textit{code for title page}\\
|\newpage|\\
|\||fi|
\end{tabular}
\end{center}
%
A banner page for the child documents can be generated by:
%
\begin{center}
\begin{tabular}{l}
|\ifchilddoc|\\
|\addtocounter{page}{-1}|\\
\textit{code for banner page}\\
|\newpage|\\
|\||fi|
\end{tabular}
\end{center}
%
Here one could write a message such as:
\begin{center}
|This is the part \childdocname{} of \childdocjob{}.|
\end{center}

%%%%%%%%%%%%%%%%%%%%%%%%%%%%%%%%%%%%%%%%%%%%%%%%%%%%%%%%%%%%%%%%%%%%%%%%%%%%%%%%
\subsection{Flags}
\label{sec:flags}

The package makes it easy to generate different versions
of the main or child documents.
To this end compilation flags can be defined
and assigned different default values.
They will be particularly useful in conjunction
with the forwarding mechanism described in \secref{sec:forward}.

For example, it may be useful to have a flag |\version|
which can be set to |draft| or |final|.
The document source will contain some conditional code
depending on the value of |\version|.
Suppose further, the flag should default to |final| for the main file
and to |draft| for child files
which is a natural assignment for editing the document.
This is achieved by placing the following code
in the preamble of the main document
(below the |\childdocmain| directive):
%
\begin{center}
\begin{tabular}{l}
|\ifchilddoc|\\
|\providecommand{\version}{draft}|\\
|\||else|\\
|\providecommand{\version}{final}|\\
|\||fi|
\end{tabular}
\end{center}
%
The definition by |\providecommand| makes sure
that previous definitions are not overwritten.
Further statements |\providecommand{\version}{...}|
can thus be added before the above code to override it.

For the main file, one might add a line
(between |\childdocmain| and the above block)
%
\begin{center}
|%\ifchilddoc\||else\providecommand{\version}{draft}\||fi|
\end{center}
%
which can be uncommented to produce a draft version.
Likewise one can add a line to the very top of a child file
(above the |\childdocof{|\textit{main}|}| directive)
%
\begin{center}
|%\providecommand{\version}{final}|
\end{center}
%
which can be uncommented to produce the final version of this child document.

%%%%%%%%%%%%%%%%%%%%%%%%%%%%%%%%%%%%%%%%%%%%%%%%%%%%%%%%%%%%%%%%%%%%%%%%%%%%%%%%
\subsection{Forwarding}
\label{sec:forward}

Different versions of the main or child documents
using compilation flags as described in \secref{sec:flags}
can be (permanently) stored in different files
for convenient compilation, viewing and distribution.
To this end, the package defines a command
to pass on compilation to a different file:

%%%%%%%%%%%%%%%%%%%%%%%%%%%%%%%%%%%%%%%%
\DescribeMacro{\childdocforward}
The command |\childdocforward| redirects processing to
another source file:
%
\begin{center}
\begin{tabular}{l}
|\input{childdoc.def}|\\
|\childdocforward[|\textit{main}|]{|\textit{dest}|}|\\
\end{tabular}
\end{center}
%
The argument \textit{dest} is the destination file
(without extension).
It should be the main file or one of the child files.
Note that further \textsf{childdoc} directives
such as |\childdocof| and |\childdocforward|
in the indicated file will be processed in this form.
The optional argument \textit{main}
passes on directly to the main file \textit{main}
while pretending to compile the child \textit{dest}.
This form behaves as if \textit{dest}
issues |\childdocof{|\textit{main}|}| right away,
and no further \textsf{childdoc} directives will be processed.

%%%%%%%%%%%%%%%%%%%%%%%%%%%%%%%%%%%%%%%%
\DescribeMacro{\...prefix}
In the alternative form |\childdocforwardprefix|,
%
\begin{center}
\begin{tabular}{l}
|\input{childdoc.def}|\\
|\childdocforwardprefix[|\textit{main}|]{|\textit{prefix}|}{|\textit{dest}|}|
\end{tabular}
\end{center}
%
the destination file is determined by a pattern
depending on the current file:
To make this work, the current file must be called
`{\textit{prefix}\hspace{0.2em}\textit{suffix}}'
with \textit{prefix} matching precisely the argument.
Processing is then passed on to the file
`{\textit{dest}\hspace{0.2em}\textit{suffix}}'.
Surely, the same effect is achieved by
directly specifying the
argument `{\textit{dest}\hspace{0.2em}\textit{suffix}}'
in the first form.
However, that requires to set up a different file
for each child. With the alternative form of the command
all these files can have exactly the same content
which simplifies setting them up and maintaining them.

For example, the following file |draft.tex|
with a compilation flag |\version| as described in \secref{sec:flags}
compiles the main document as a draft:
%
\begin{center}
\begin{tabular}{l}
|\def\version{draft}|\\
|\input{childdoc.def}|\\
|\childdocforward{|\textit{main}|}|
\end{tabular}
\end{center}
%
Likewise, the following files |final|\textit{nn}|.tex|
compile the final version of the child document
|child|\textit{nn}|.tex|:
%
\begin{center}
\begin{tabular}{l}
|\def\version{final}|\\
|\input{childdoc.def}|\\
|\childdocforwardprefix{final}{child}|
\end{tabular}
\end{center}
%

Note that when several versions of a main file and/or of each child file
are to be generated, it may be convenient to set up a |Makefile| or
shell script to automatise the process.

%%%%%%%%%%%%%%%%%%%%%%%%%%%%%%%%%%%%%%%%%%%%%%%%%%%%%%%%%%%%%%%%%%%%%%%%%%%%%%%%
\subsection{Command Line Processing}
\label{sec:commandline}

The effect of redirection files can also be achieved by invoking
the \LaTeX{} compiler with a more elaborate command line.
Most conveniently this should be done as part
of a shell script or a |Makefile|.

When using \textsf{childdoc} in the main file, the following
command lines effectively perform a redirection
(note that depending on the shell being used,
backslashes may have to be doubled: `|\|' $\to$ `|\\|'):
%
\begin{center}
|... -jobname "|\textit{target}|" |\\|"|[\textit{flags}]%
|\input{childdoc.def}\childdocforward[|\textit{main}|]{|\textit{dest}|}"|
\end{center}
%
Here \textit{target} is the name of the output file,
\textit{main} is the name of the main file
and \textit{dest} is the name of the main or child file to be processed
(all filenames without extensions).
The optional argument \textit{main} can be omitted
if \textit{main} matches \textit{dest}.
Optionally, compilation \textit{flags} can be defined via |\def| commands.
This command line makes the \TeX{} engine believe
it is compiling the file \textit{target}
whose content is specified as the latter parameter.
The provided code then forwards the processing to
\textit{main} or \textit{dest} as described in \secref{sec:forward}.

%%%%%%%%%%%%%%%%%%%%%%%%%%%%%%%%%%%%%%%%%%%%%%%%%%%%%%%%%%%%%%%%%%%%%%%%%%%%%%%%
\subsection{Include by Input}
\label{sec:input}

Including child documents by |\include| has some restrictions by design.
Most notably, the content of a child document always occupies
its own set of pages; pages cannot be shared between child documents.
Usually, this behaviour makes perfect sense
because each child document contain an essential part of the document.
However, in some situations it may be desirable to compose
a document from a collection of parts
without having mandatory page breaks between then.
For this case, the package
provides a mechanism to include parts
by |\input| which can also be processed individually.
However, by construction this mechanism
requires manual handling of the content to be output.

%%%%%%%%%%%%%%%%%%%%%%%%%%%%%%%%%%%%%%%%
\DescribeMacro{\ifchilddocmanual}
The main file should be prepared as usual, see \secref{sec:include}.
However, the document body must make a distinction
between processing of an individual part and of the main document, e.g.:
%
\begin{center}
\begin{tabular}{l}
|\ifchilddocmanual|\\
|\input{\childdocname}|\\
|\||else|\\
\textit{document body with }|\input{|\textit{part}|}|\\
|\||fi|
\end{tabular}
\end{center}
%
The conditional |\ifchilddocmanual| is true whenever
a part to be included by |\input| is being compiled,
and the name of the part is stored in |\childdocname|.

%%%%%%%%%%%%%%%%%%%%%%%%%%%%%%%%%%%%%%%%
\DescribeMacro{\childdocby}
Each part to be included by |\input| should start with:
%
\begin{center}
\begin{tabular}{l}
|\input{childdoc.def}|\\
|\childdocby{|\textit{main}|}|\\
\end{tabular}
\end{center}
%
The directive |\childdocby| is similar to |\childdocof|
described in \secref{sec:include},
but the subsequent selection of content must be done manually.
To that end, both |\ifchilddoc| and |\ifchilddocmanual|
will be true upon processing of a part,
and the name of the part is stored in |\childdocname|.
Note that |\jobname| will be set to the filename of the current part
so that each part receives an individual |.aux| file
that does not interfere with the |.aux| file(s) of the main document.
This behaviour can be altered by the alternative form
|\childdocby[*]{|\textit{main}|}| (with a non-empty optional argument)
which uses the |.aux| file of the main document
by setting |\jobname| to \textit{main}.

%%%%%%%%%%%%%%%%%%%%%%%%%%%%%%%%%%%%%%%%%%%%%%%%%%%%%%%%%%%%%%%%%%%%%%%%%%%%%%%%
\subsection{Driver Development}
\label{sec:driver}

The \textsf{childdoc} mechanism can also be use for the development
of definition files such as \LaTeX{} styles or classes.
This case differs from the above setup with multiple parts
included by |\include| in that no |\includeonly| should be invoked.
This can be achieved by starting the include file
(before |\ProvidesPackage|) with:
%
\begin{center}
\begin{tabular}{l}
|\input{childdoc.def}|\\
|\childdocforward{|\textit{main}|}|\\
\end{tabular}
\end{center}
%
or alternatively with:
%
\begin{center}
\begin{tabular}{l}
|\input{childdoc.def}|\\
|\childdocby{|\textit{main}|}|\\
\end{tabular}
\end{center}
%
Both forms have slightly different effects as described above.
The main file is prepared as usual, see \secref{sec:include}.

%%%%%%%%%%%%%%%%%%%%%%%%%%%%%%%%%%%%%%%%%%%%%%%%%%%%%%%%%%%%%%%%%%%%%%%%%%%%%%%%
\subsection{Legacy Detection}
\label{sec:detection}

The directive |\childdocmain| in the main file can detect
whether the complete document or merely a child is to be compiled
even without using the directive |\childdocof|.
This method is deprecated because it is less robust
and there is no compelling reason to use it;
it is merely provided for backward compatibility
and it may be removed in future versions.

If the detection mechanism is to be used,
it is mandatory to correctly specify
the filename of the main file as the argument of |\childdocmain|:
%
\begin{center}
\begin{tabular}{l}
|\input{childdoc.def}|\\
|\childdocmain{|\textit{main}|}|\\
\end{tabular}
\end{center}
%
If |\jobname| does not match the argument \textit{main} of |\childdocmain|,
it is assumed that |\jobname| points to the child file to be compiled.
When using |\childdocmain| with the main file specified as argument,
it suffices to start a child file
with just |\input{|\textit{main}|}|
without loading of the package and using |\childdocof|.
If instead all processing is done
with the appropriate \textsf{childdoc} directives,
the argument of \textit{main} of |\childdocmain| can be empty.

An alternative version of the command line processing described
in \secref{sec:commandline} using the detection mechanism reads:
%
\begin{center}
|... -jobname "|\textit{target}|" "|[\textit{flags}]%
[|\def\jobname{|\textit{dest}|}|]|\input{|\textit{main}|}"|
\end{center}

%%%%%%%%%%%%%%%%%%%%%%%%%%%%%%%%%%%%%%%%%%%%%%%%%%%%%%%%%%%%%%%%%%%%%%%%%%%%%%%%
\subsection{Manual Code}
\label{sec:manual}

In case one cannot be certain whether the definitions file |childdoc.def|
is installed on the target \TeX{} distribution
and one prefers not to ship it,
it is conceivable to paste a few relevant commands into the sources.

To that end, drop all statements |\input{childdoc.def}|
and perform the replacements as outlined below.
Instead of |\childdocmain{|\textit{main}|}| add the following code
to the top of the main file:
%
\begin{center}
\begin{tabular}{l}
|\||ifdefined\childdocname\endinput\||fi\newif\ifchilddoc|\\
|\edef\childdocname{\scantokens\expandafter{\jobname\noexpand}}|\\
|\def\childdocmain{|\textit{main}|}\||ifx\childdocmain\childdocname\||else|\\
|\childdoctrue\includeonly{\childdocname}\let\jobname\childdocmain\||fi|\\
\end{tabular}
\end{center}
%
Instead of |\childdocof{|\textit{main}|}| just include the main file
at the top of each child file:
%
\begin{center}
|\input{|\textit{main}|}|
\end{center}
%
A simple redirection |\childdocforward{|\textit{dest}|}| is achieved by:
%
\begin{center}
|\def\jobname{|\textit{dest}|}\input{\jobname}|
\end{center}
%
The redirection with prefix
|\childdocforwardprefix[|\textit{prefix}|]{|\textit{dest}|}|
is accomplished by:
%
\begin{center}
\begin{tabular}{l}
|{\edef\jobname{\scantokens\expandafter{\jobname\noexpand}}|\\
|\def\redirectjob |\textit{prefix}|#1~~~{\gdef\jobname{|\textit{dest}|#1}}|\\
|\expandafter\redirectjob\jobname~~~}\input{\jobname}|
\end{tabular}
\end{center}

In an alternative approach,
child documents can be compiled by a specific command line
without additional code or specific definitions:
%
\begin{center}
|... -jobname "|\textit{target}|" "|[\textit{flags}]%
|\includeonly{|\textit{dest}|}\input{|\textit{main}|}"|
\end{center}
%

%%%%%%%%%%%%%%%%%%%%%%%%%%%%%%%%%%%%%%%%%%%%%%%%%%%%%%%%%%%%%%%%%%%%%%%%%%%%%%%%
%%%%%%%%%%%%%%%%%%%%%%%%%%%%%%%%%%%%%%%%%%%%%%%%%%%%%%%%%%%%%%%%%%%%%%%%%%%%%%%%
\section{Information}

%%%%%%%%%%%%%%%%%%%%%%%%%%%%%%%%%%%%%%%%%%%%%%%%%%%%%%%%%%%%%%%%%%%%%%%%%%%%%%%%
\subsection{Copyright}

Copyright \copyright{} 2017--2018 Niklas Beisert

This work may be distributed and/or modified under the
conditions of the \LaTeX{} Project Public License, either version 1.3
of this license or (at your option) any later version.
The latest version of this license is in
  \url{http://www.latex-project.org/lppl.txt}
and version 1.3 or later is part of all distributions of \LaTeX{}
version 2005/12/01 or later.

This work has the LPPL maintenance status `maintained'.

The Current Maintainer of this work is Niklas Beisert.

This work consists of the files |README.txt|, |childdoc.ins| and |childdoc.dtx|
as well as the derived files |childdoc.def|, |cdocsamp.tex|
with |cdocsch1.tex|, |cdocsch2.tex|, |cdocspt3.tex|, |cdocspt4.tex|,
|cdocsdrf.tex|, |cdocsfn1.tex|, |cdocsfn2.tex|
as well as |childdoc.pdf|.

%%%%%%%%%%%%%%%%%%%%%%%%%%%%%%%%%%%%%%%%%%%%%%%%%%%%%%%%%%%%%%%%%%%%%%%%%%%%%%%%
\subsection{Files and Installation}

The package consists of the files:
%
\begin{center}
\begin{tabular}{ll}
    |README.txt|   & readme file \\
    |childdoc.ins| & installation file \\
    |childdoc.dtx| & source file \\
    |childdoc.def| & definition file \\
    |cdocsamp.tex| & sample main file \\
    |cdocsch1.tex| & sample include file \\
    |cdocsch2.tex| & sample include file \\
    |cdocspt3.tex| & sample part file \\
    |cdocspt4.tex| & sample part file \\
    |cdocsdrf.tex| & sample redirection file \\
    |cdocsfn1.tex| & sample redirection file \\
    |cdocsfn2.tex| & sample redirection file \\
    |childdoc.pdf| & manual
\end{tabular}
\end{center}
%
The distribution consists of the files
|README.txt|, |childdoc.ins| and |childdoc.dtx|.
%
\begin{itemize}
\item
Run (pdf)\LaTeX{} on |childdoc.dtx|
to compile the manual |childdoc.pdf| (this file).
\item
Run \LaTeX{} on |childdoc.ins| to create the definitions file |childdoc.def|
and the sample |cdocsamp.tex| with include files
|cdocsch1.tex|, |cdocsch2.tex|, |cdocspt3.tex|, |cdocspt4.tex|,
|cdocsdrf.tex|, |cdocsfn1.tex|, |cdocsfn2.tex|.
Then copy the file |childdoc.def| to an appropriate directory of your \LaTeX{}
distribution, e.g.\ \textit{texmf-root}|/tex/latex/childdoc|.
\end{itemize}

%%%%%%%%%%%%%%%%%%%%%%%%%%%%%%%%%%%%%%%%%%%%%%%%%%%%%%%%%%%%%%%%%%%%%%%%%%%%%%%%
\subsection{Related CTAN Packages}

There are several other packages which offer a similar functionality:
%
\begin{itemize}
\item
The packages
\href{http://ctan.org/pkg/docmute}{\textsf{docmute}},
\href{http://ctan.org/pkg/includex}{\textsf{includex}} and
\href{http://ctan.org/pkg/standalone}{\textsf{standalone}}
provide commands to include only the document body of
a child file thus allowing both files to be compiled individually.
\item
The packages \href{http://ctan.org/pkg/subdocs}{\textsf{subdocs}}
and \href{http://ctan.org/pkg/subfiles}{\textsf{subfiles}}
provide structures in which the main and child documents can be
encapsulated and allowing them to be compiled individually.
The inclusion mechanism is different from the conventional |\include|.
\item
The package \href{http://ctan.org/pkg/combine}{\textsf{combine}}
is an elaborate solution to combine several documents into one.
\end{itemize}
%
See also the CTAN topic \href{http://ctan.org/topic/subdocs}{\textsf{subdocs}}
for further related packages.
The present package differs from the above solutions in that
a document structure constructed with the conventional |\include| mechanism
just needs two extra commands at the top of every file
such that all constituent files can be compiled individually.

%%%%%%%%%%%%%%%%%%%%%%%%%%%%%%%%%%%%%%%%%%%%%%%%%%%%%%%%%%%%%%%%%%%%%%%%%%%%%%%%
%\subsection{Feature Suggestions}
%
%The following is a list of features which may be useful for future
%versions of this package:
%%
%\begin{itemize}
%\item
%\ldots
%\end{itemize}

%%%%%%%%%%%%%%%%%%%%%%%%%%%%%%%%%%%%%%%%%%%%%%%%%%%%%%%%%%%%%%%%%%%%%%%%%%%%%%%%
\subsection{Revision History}

%%%%%%%%%%%%%%%%%%%%%%%%%%%%%%%%%%%%%%%%
\paragraph{v2.0:} 2018/12/30

\begin{itemize}
\item
immediate forward processing
\item
added |\childdocby| mechanism
\item
manual restructured
\end{itemize}

%%%%%%%%%%%%%%%%%%%%%%%%%%%%%%%%%%%%%%%%
\paragraph{v1.6:} 2018/01/17

\begin{itemize}
\item
application for development of include files
\item
corrections to manual
\end{itemize}

%%%%%%%%%%%%%%%%%%%%%%%%%%%%%%%%%%%%%%%%
\paragraph{v1.5:} 2017/05/21

\begin{itemize}
\item
more complete structuring introduced
\item
|\childdocof| introduced
\item
|\childdoc| renamed to |\childdocmain|
\item
|\childredirect| renamed to |\childdocforward| and |\childdocforwardprefix|
and functionality expanded
\end{itemize}

%%%%%%%%%%%%%%%%%%%%%%%%%%%%%%%%%%%%%%%%
\paragraph{v1.0:} 2017/04/27

\begin{itemize}
\item
manual and install package
\item
first version published on CTAN
\end{itemize}

%%%%%%%%%%%%%%%%%%%%%%%%%%%%%%%%%%%%%%%%
\paragraph{v0.6:} 2017/04/26

\begin{itemize}
\item
redirection mechanism added
\end{itemize}

%%%%%%%%%%%%%%%%%%%%%%%%%%%%%%%%%%%%%%%%
\paragraph{v0.5:} 2017/04/26

\begin{itemize}
\item
functionality in definition file
\end{itemize}


%%%%%%%%%%%%%%%%%%%%%%%%%%%%%%%%%%%%%%%%%%%%%%%%%%%%%%%%%%%%%%%%%%%%%%%%%%%%%%%%
%%%%%%%%%%%%%%%%%%%%%%%%%%%%%%%%%%%%%%%%%%%%%%%%%%%%%%%%%%%%%%%%%%%%%%%%%%%%%%%%
%%%%%%%%%%%%%%%%%%%%%%%%%%%%%%%%%%%%%%%%%%%%%%%%%%%%%%%%%%%%%%%%%%%%%%%%%%%%%%%%
\appendix

\settowidth\MacroIndent{\rmfamily\scriptsize 000\ }

 \DocInput{childdoc.dtx}

\end{document}
%</driver>
% \fi
%
% %%%%%%%%%%%%%%%%%%%%%%%%%%%%%%%%%%%%%%%%%%%%%%%%%%%%%%%%%%%%%%%%%%%%%%%%%%%%%%
% %%%%%%%%%%%%%%%%%%%%%%%%%%%%%%%%%%%%%%%%%%%%%%%%%%%%%%%%%%%%%%%%%%%%%%%%%%%%%%
% \section{Sample}
%\iffalse
%<*samplemain>
%\fi
%
% The following presents a sample document
% with two chapters, two parts, a title page,
% a compile flag as well as three forwarding files to set the flag.
% It consists of eight |.tex| files:
% \begin{center}
% \begin{tabular}{ll}
% |cdocsamp.tex|&main file\\
% |cdocsch1.tex|&include file for chapter 1\\
% |cdocsch2.tex|&include file for chapter 2\\
% |cdocspt3.tex|&include file for part 3\\
% |cdocspt4.tex|&include file for part 4\\
% |cdocsdrf.tex|&forwarding file for main file in draft mode\\
% |cdocsfi1.tex|&forwarding file for final version of chapter 1\\
% |cdocsfi2.tex|&forwarding file for final version of chapter 2\\
% \end{tabular}
% \end{center}
% Each of the eight files can be compiled directly by the \LaTeX{} compiler.
%
% %%%%%%%%%%%%%%%%%%%%%%%%%%%%%%%%%%%%%%
% \paragraph{Main File.}
%
% The main file is called |cdocsamp.tex|.
%
% Load the \textsf{childdoc} definitions and
% declare the filename for the main document:
%    \begin{macrocode}
\input{childdoc.def}
\childdocmain{}
%    \end{macrocode}

% Optional override for |\version| flag:
%    \begin{macrocode}
%%\ifchilddoc\else\providecommand{\version}{draft}\fi
%    \end{macrocode}

% Define the default values for the |\version| flag
% (|final| for the main file and |draft| for childs):
%    \begin{macrocode}
\ifchilddoc
\providecommand{\version}{draft}
\else
\providecommand{\version}{final}
\fi
%    \end{macrocode}

% Load the standard document class:
%    \begin{macrocode}
\documentclass[12pt]{article}
%    \end{macrocode}

% Start the document body:
%    \begin{macrocode}
\begin{document}
%    \end{macrocode}

% Declare a title page.
% Print title, part of document being processed and version flag:
%    \begin{macrocode}
\addtocounter{page}{-1}
\begin{center}
{\LARGE\bfseries{}childdoc example\par}
\vspace{1cm}
\ifchilddoc
\ifchilddocmanual part\else chapter\fi:
`\childdocname' of `\childdocjob'\par
\else
main document: `\childdocjob'\par
\fi
version: \version\par
\end{center}
\newpage
%    \end{macrocode}

% Manually include selected file,
% otherwise process as usual:
%    \begin{macrocode}
\ifchilddocmanual
\section*{part `\childdocname'}
\input{\childdocname}
\else
%    \end{macrocode}

% Include the two chapters:
%    \begin{macrocode}
\include{cdocsch1}
\include{cdocsch2}
%    \end{macrocode}

% Include the two parts unless only chapters should be displayed:
%    \begin{macrocode}
\ifchilddoc\else
\section{part three}
\input{cdocspt3}
\section{part four}
\input{cdocspt4}
\fi
%    \end{macrocode}

% Process as usual until here:
%    \begin{macrocode}
\fi
%    \end{macrocode}

% End of document body:
%    \begin{macrocode}
\end{document}
%    \end{macrocode}
%\iffalse
%</samplemain>
%\fi
%
% %%%%%%%%%%%%%%%%%%%%%%%%%%%%%%%%%%%%%%
% \paragraph{Chapter Include Files.}
%
% The include files are called |cdocsch1.tex| and |cdocsch2.tex|.
%
%\iffalse
%<*samplechap1|samplechap2>
%\fi

% Optional override for |\version| flag:
%    \begin{macrocode}
%%\providecommand{\version}{final}
%    \end{macrocode}

% Include the main document:
%    \begin{macrocode}
\input{childdoc.def}
\childdocof{cdocsamp}
%    \end{macrocode}

%\iffalse
%</samplechap1|samplechap2>
%\fi
%
%\iffalse
%<*samplechap1>
%\fi
% Some text for chapter 1:
%    \begin{macrocode}
\section{one}
some text in chapter one
%    \end{macrocode}

%\iffalse
%</samplechap1>
%\fi
% Some text for chapter 2:
%\iffalse
%<*samplechap2>
%\fi
%    \begin{macrocode}
\section{two}
more text in chapter two
%    \end{macrocode}

%\iffalse
%</samplechap2>
%\fi
%
% %%%%%%%%%%%%%%%%%%%%%%%%%%%%%%%%%%%%%%
% \paragraph{Part Include Files.}
%
% The include files are called |cdocspt3.tex| and |cdocspt4.tex|.
%
%\iffalse
%<*samplepart3|samplepart4>
%\fi

% Optional override for |\version| flag:
%    \begin{macrocode}
%%\providecommand{\version}{final}
%    \end{macrocode}

% Include the main document:
%    \begin{macrocode}
\input{childdoc.def}
\childdocby{cdocsamp}
%    \end{macrocode}

%\iffalse
%</samplepart3|samplepart4>
%\fi
%
%\iffalse
%<*samplepart3>
%\fi
% Some text for part 3:
%    \begin{macrocode}
some text in part three
%    \end{macrocode}

%\iffalse
%</samplepart3>
%\fi
% Some text for part 4:
%\iffalse
%<*samplepart4>
%\fi
%    \begin{macrocode}
more text in part four
%    \end{macrocode}

%\iffalse
%</samplepart4>
%\fi
%
% %%%%%%%%%%%%%%%%%%%%%%%%%%%%%%%%%%%%%%
% \paragraph{Forwarding for a Complete Draft.}
%
% The following forwarding file |cdocsdrf.tex|
% compiles the main document in draft mode:
%\iffalse
%<*sampledraft>
%\fi
%    \begin{macrocode}
\def\version{draft}
\input{childdoc.def}
\childdocforward{cdocsamp}
%    \end{macrocode}

%\iffalse
%</sampledraft>
%\fi
%
% %%%%%%%%%%%%%%%%%%%%%%%%%%%%%%%%%%%%%%
% \paragraph{Forwarding for Final Version of the Chapters.}
%
% The following forwarding files |cdocsfn1.tex| and |cdocsfn2.tex|
% (with identical content)
% compile the final versions of the child documents
% |cdocsch1.tex| and |cdocsch2.tex|, respectively:
%\iffalse
%<*samplefinal>
%\fi
%    \begin{macrocode}
\def\version{final}
\input{childdoc.def}
\childdocforwardprefix[cdocsamp]{cdocsfn}{cdocsch}
%    \end{macrocode}

%\iffalse
%</samplefinal>
%\fi
%
% %%%%%%%%%%%%%%%%%%%%%%%%%%%%%%%%%%%%%%
% \paragraph{Command Line Processing.}
%
% The following three command lines generate the output files
% |cdocscld|, |cdocscl1| and |cdocscl2|
% which should be identical to
% |cdocsdrf|, |cdocsch1| and |cdocsfn2|, respectively:
% \begin{center}
% \begin{tabular}{l}
% |latex -jobname cdocscld \|\\
% |  "\def\version{draft}\input{childdoc.def}\childdocforward{cdocsamp}"|\\
% |latex -jobname cdocscl1 \|\\
% |  "\input{childdoc.def}\childdocforward[cdocsamp]{cdocsch1}"|\\
% |latex -jobname cdocscl2 \|\\
% |  "\def\version{final}\input{childdoc.def}\childdocforward{cdocsch2}"|
% \end{tabular}
% \end{center}
% Note that the trailing backslash on each first line
% merely continues the input to the second line
% (for convenient cut ant paste).
% Furthermore, the command |latex| can be replaced by any
% of its alternative versions such as |pdflatex|.
%
% %%%%%%%%%%%%%%%%%%%%%%%%%%%%%%%%%%%%%%%%%%%%%%%%%%%%%%%%%%%%%%%%%%%%%%%%%%%%%%
% %%%%%%%%%%%%%%%%%%%%%%%%%%%%%%%%%%%%%%%%%%%%%%%%%%%%%%%%%%%%%%%%%%%%%%%%%%%%%%
% \section{Implementation}
%\iffalse
%<*package>
%\fi
%
% This section describes the definitions file |childdoc.def|.

% The definitions cannot be loaded using |\usepackage| or |\RequirePackage|
% which has a mechanism to prevent loading a style file more than once.
% When loading the definitions by means of |\input|
% multiple instances have to be prevented manually:
%\iffalse
%This code needs to be before the `\ProvidesFile' directive
%which is defined at the beginning of this file.
%Therefore it is also placed there and commented out here.
%</package>
%<*discard>
%\fi
%    \begin{macrocode}
\ifdefined\childdocmain\endinput\fi
%    \end{macrocode}
%\iffalse
%</discard>
%<*package>
%\fi
%
% \macro{\ifchilddoc}
% \macro{\ifchilddocmanual}
% The conditional |\ifchilddoc| tells whether a
% child (true) or main (false) document is being compiled.
% The conditional |\ifchilddocmanual| tells whether
% the |\includeonly| mechanism is used (false) or
% the selection of child files must be performed manually (true).
% The definitions initialise to false:
%    \begin{macrocode}
\newif\ifchilddoc
\newif\ifchilddocmanual
%    \end{macrocode}

% \macro{\childdocname}
% \macro{\childdocjob}
% The macro |\childdocname| stores the name of the main document
% to be compiled. The macro |\childdocjob| stores the name of
% the document on which the \LaTeX{} compiler was originally invoked.
% The content of |\jobname| cannot be compared
% to filenames specified in the source due to different catcodes.
% The following code rescans |\jobname|, stores the result
% in |\childdocname| and saves a copy in |\childdocjob|:
%    \begin{macrocode}
\edef\childdocname{\scantokens\expandafter{\jobname\noexpand}}
\let\childdocjob\childdocname
%    \end{macrocode}

% \macro{\childdocdisable}
% The macro |\childdocdisable| prevents the main file
% from being processed more than once.
% At this stage, the main document command |\childdocmain|
% is assumed to be called once again where it should do nothing.
% Any subsequent call to it should prevent
% a secondary processing of the main document
% It overwrites the forwarding commands
% |\childdocof| and |\childdocforward|
% with empty macros to prevent further inclusions of the main document:
%    \begin{macrocode}
\newcommand{\childdocdisable}
{
  \renewcommand{\childdocmain}[1]{\renewcommand{\childdocmain}[1]{\endinput}}
  \renewcommand{\childdocof}[1]{}
  \renewcommand{\childdocby}[2][]{}
  \renewcommand{\childdocforward}[2][]{}
  \renewcommand{\childdocdisable}{}
}
%    \end{macrocode}

% \macro{\childdocmain}
% The macro |\childdocmain| is to be called at the top of the main file
% with nothing or the main filename (without extension) as argument.
% First, it breaks loops.
% If the argument is not empty and does not match |\childdocname|
% (which is set by the first inclusion of |childdoc.def|),
% |\ifchilddoc| is set to true, |\includeonly| is applied to the child file
% and |\jobname| is set to the main file
% (for proper handling of |.aux| files):
%    \begin{macrocode}
\newcommand{\childdocmain}[1]
{
  \childdocdisable\childdocmain{}
  \if?#1?\else
    \begingroup
      \def\childdoctmp{#1}
      \ifx\childdoctmp\childdocname
        \def\childdoctmp{}
      \else
        \def\childdoctmp
        {
          \childdoctrue
          \includeonly{\childdocname}
          \def\childdocjob{#1}
          \def\jobname{#1}
        }
      \fi
      \expandafter
    \endgroup
    \childdoctmp
  \fi
}
%    \end{macrocode}

% \macro{\childdocof}
% The command |\childdocof| redirects
% compilation to the main file |#1|.
%    \begin{macrocode}
\newcommand{\childdocof}[1]
{
  \childdocdisable
  \childdoctrue
  \includeonly{\childdocname}
  \def\jobname{#1}
  \def\childdocjob{#1}
  \input{#1}
}
%    \end{macrocode}

% \macro{\childdocby}
% The command |\childdocby| ....
%    \begin{macrocode}
\newcommand{\childdocby}[2][]
{
  \childdocdisable
  \childdoctrue
  \childdocmanualtrue
  \if?#1?\else
    \def\jobname{#2}
  \fi
  \def\childdocjob{#2}
  \input{#2}
  \endinput
}
%    \end{macrocode}

% \macro{\childdocforward}
% The command |\childdocforward| redirects
% compilation to the main file or
% (if the optional argument is given) a child file.
% Parameters are set as if the main file
% or a child file starting with |\childdocof| was compiled.
% Then compilation is handed over to the main file:
%    \begin{macrocode}
\newcommand{\childdocforward}[2][]
{
  \begingroup
    \if?#1?
      \def\childdoctmp
      {
        \def\childdocname{#2}
        \def\childdocjob{#2}
        \def\jobname{#2}
        \input{#2}
        \endinput
      }
    \else
      \def\childdoctmp
      {
        \childdocdisable
        \def\childdocname{#2}
        \childdoctrue
        \includeonly{#2}
        \def\childdocjob{#1}
        \def\jobname{#1}
        \input{#1}
        \endinput
      }
    \fi
    \expandafter
  \endgroup
  \childdoctmp
}
%    \end{macrocode}

% \macro{\childdocforwardprefix}
% The command |\childdocforwardprefix| redirects
% compilation to the main or a child file by means of a pattern.
% The prefix |#1| in the current filename is replaced by |#2|
% and the suffix of the current filename is kept
% (it is assumed that the filename does not contain the substring `|~~~|'
% which is used as a delimiter).
% Compilation is handed over to the new file by |\childdocforward|:
%    \begin{macrocode}
\newcommand{\childdocforwardprefix}[3][]
{
  \begingroup
    \def\childdocextract #2##1~~~{\def\childdoctmp{\childdocforward[#1]{#3##1}}}
    \expandafter\childdocextract\childdocname~~~
    \expandafter
  \endgroup
  \childdoctmp
}
%    \end{macrocode}

% \macro{\childdoc}
% The deprecated macro |\childdoc| is a legacy version of |\childdocmain|:
%    \begin{macrocode}
\newcommand{\childdoc}{\childdocmain}
%    \end{macrocode}

% \macro{\childdocredirect}
% The deprecated macro |\childdocredirect| is a legacy version
% of |\childdocforward| and |\childdocforwardprefix|:
%    \begin{macrocode}
\newcommand{\childdocredirect}[2][]
{
  \begingroup
    \if?#1?
      \def\childdoctmp{\childdocforward{#2}}
    \else
      \def\childdoctmp{\childdocforwardprefix{#1}{#2}}
    \fi
    \expandafter
  \endgroup
  \childdoctmp
}
%    \end{macrocode}

%\iffalse
%</package>
%\fi
%
\endinput

\childdocof{cdocsamp}
%    \end{macrocode}

%\iffalse
%</samplechap1|samplechap2>
%\fi
%
%\iffalse
%<*samplechap1>
%\fi
% Some text for chapter 1:
%    \begin{macrocode}
\section{one}
some text in chapter one
%    \end{macrocode}

%\iffalse
%</samplechap1>
%\fi
% Some text for chapter 2:
%\iffalse
%<*samplechap2>
%\fi
%    \begin{macrocode}
\section{two}
more text in chapter two
%    \end{macrocode}

%\iffalse
%</samplechap2>
%\fi
%
% %%%%%%%%%%%%%%%%%%%%%%%%%%%%%%%%%%%%%%
% \paragraph{Part Include Files.}
%
% The include files are called |cdocspt3.tex| and |cdocspt4.tex|.
%
%\iffalse
%<*samplepart3|samplepart4>
%\fi

% Optional override for |\version| flag:
%    \begin{macrocode}
%%\providecommand{\version}{final}
%    \end{macrocode}

% Include the main document:
%    \begin{macrocode}
% \iffalse
%
% childdoc.dtx Copyright (C) 2017-2018 Niklas Beisert
%
% This work may be distributed and/or modified under the
% conditions of the LaTeX Project Public License, either version 1.3
% of this license or (at your option) any later version.
% The latest version of this license is in
%   http://www.latex-project.org/lppl.txt
% and version 1.3 or later is part of all distributions of LaTeX
% version 2005/12/01 or later.
%
% This work has the LPPL maintenance status `maintained'.
%
% The Current Maintainer of this work is Niklas Beisert.
%
% This work consists of the files childdoc.dtx and childdoc.ins
% and the derived files childdoc.def and cdocsamp.tex with
% cdocsch1.tex, cdocsch2.tex, cdocsdrf.tex, cdocsfn1.tex, cdocsfn2.tex.
%
%<package>\ifdefined\childdocmain\endinput\fi
%<package>\ProvidesFile{childdoc.def}[2018/12/30 v2.0 child document driver]
%<samplemain>\ProvidesFile{cdocsamp.tex}[2018/12/30 v2.0 sample for childdoc]
%<*driver>
%\ProvidesFile{childdoc.drv}[2018/12/30 v2.0 childdoc reference manual file]
\PassOptionsToClass{10pt,a4paper}{article}
\documentclass{ltxdoc}

\usepackage[margin=35mm]{geometry}
\usepackage{hyperref}
\usepackage{hyperxmp}
\usepackage[usenames]{color}

\hypersetup{colorlinks=true}
\hypersetup{pdfstartview=FitH}
\hypersetup{pdfpagemode=UseNone}
\hypersetup{pdfsource={}}
\hypersetup{pdflang={en-UK}}
\hypersetup{pdfcopyright={Copyright 2017-2018 Niklas Beisert.
  This work may be distributed and/or modified under the
  conditions of the LaTeX Project Public License, either version 1.3
  of this license or (at your option) any later version.}}
\hypersetup{pdflicenseurl={http://www.latex-project.org/lppl.txt}}
\hypersetup{pdfcontactaddress={ETH Zurich, ITP, HIT K,
  Wolfgang-Pauli-Strasse 27}}
\hypersetup{pdfcontactpostcode={8093}}
\hypersetup{pdfcontactcity={Zurich}}
\hypersetup{pdfcontactcountry={Switzerland}}
\hypersetup{pdfcontactemail={nbeisert@itp.phys.ethz.ch}}
\hypersetup{pdfcontacturl={http://people.phys.ethz.ch/\xmptilde nbeisert/}}

\newcommand{\secref}[1]{\hyperref[#1]{section \ref*{#1}}}

\parskip1ex
\parindent0pt
\let\olditemize\itemize
\def\itemize{\olditemize\parskip0pt}

\begin{document}

\title{The \textsf{childdoc} Package}
\hypersetup{pdftitle={The childdoc Package}}
\author{Niklas Beisert\\[2ex]
  Institut f\"ur Theoretische Physik\\
  Eidgen\"ossische Technische Hochschule Z\"urich\\
  Wolfgang-Pauli-Strasse 27, 8093 Z\"urich, Switzerland\\[1ex]
  \href{mailto:nbeisert@itp.phys.ethz.ch}
  {\texttt{nbeisert@itp.phys.ethz.ch}}}
\hypersetup{pdfauthor={Niklas Beisert}}
\hypersetup{pdfsubject={Manual for the LaTeX2e Package childdoc}}
\date{30 December 2018, \textsf{v2.0}}
\maketitle

\begin{abstract}\noindent
\textsf{childdoc} is a \LaTeXe{} package
that enables the direct compilation
of document sections included by |\include|
to individual files.
\end{abstract}

\begingroup
\parskip0ex
\tableofcontents
\endgroup

%%%%%%%%%%%%%%%%%%%%%%%%%%%%%%%%%%%%%%%%%%%%%%%%%%%%%%%%%%%%%%%%%%%%%%%%%%%%%%%%
%%%%%%%%%%%%%%%%%%%%%%%%%%%%%%%%%%%%%%%%%%%%%%%%%%%%%%%%%%%%%%%%%%%%%%%%%%%%%%%%
\section{Introduction}

\LaTeX{} provides a mechanism to structure a large document (such as a book)
into a main file and several child files (containing the chapters)
using the |\include| command.
This mechanism is beneficial for documents
which span hundreds of pages in order to
make the source file(s) more manageable.
Moreover, compilation can be restricted to
selected child files by means of the |\includeonly| command.
The latter feature can be used to reduce the compilation time while editing
(this was significantly more useful in the earlier days of \LaTeX{})
or to generate a smaller document which is easier to navigate.
Another application of |\includeonly| is to generate
documents consisting of selected parts of the complete document.

However, there are a few drawbacks of the plain |\include| mechanism:
\begin{itemize}
\item
The child files cannot be compiled on their own,
they can only be compiled via the main file.
A naive editing environment
(such as a text editor with an option
to have the current file processed by \LaTeX)
may require one to switch to the main file before compiling;
attempting to compile the child file produces errors.
\item
The main file must be modified (each time)
to adjust the |\includeonly| command
to the present needs. This easily leaves the main file in a messy state.
\item
The generated document will always carry the filename
of the main document. This is inconvenient if
several child files are to be compiled and
to be kept for distribution.
\end{itemize}

The present package provides a simple interface
to make child files individually compilable by \LaTeX{}.
Compiling a child file then has the same effect as compiling
the main file with an |\includeonly| command
to select the appropriate child.
Moreover the generated document will carry the name of the child
rather than the main file.
This resolves all three above issues.

This feature is meant to make the editing of books,
thesis documents and lecture notes somewhat more convenient.
However, the package can also be used efficiently for
composing a series of documents (such as exercise sheets)
which are typically distributed individually.
It then assists the author in generating the individual documents
(potentially in different versions)
as well as a document containing the collected series.
Another application is in developing style files
or other kinds of included material
where compilation of the style file could redirect
to a sample or test file.

%%%%%%%%%%%%%%%%%%%%%%%%%%%%%%%%%%%%%%%%%%%%%%%%%%%%%%%%%%%%%%%%%%%%%%%%%%%%%%%%
%%%%%%%%%%%%%%%%%%%%%%%%%%%%%%%%%%%%%%%%%%%%%%%%%%%%%%%%%%%%%%%%%%%%%%%%%%%%%%%%
\section{Usage}

First of all, the package \textsf{childdoc} is \emph{not} a standard
\LaTeXe{} |.sty| style file! Therefore it needs to be invoked in
a non-standard way.

%%%%%%%%%%%%%%%%%%%%%%%%%%%%%%%%%%%%%%%%%%%%%%%%%%%%%%%%%%%%%%%%%%%%%%%%%%%%%%%%
\subsection{Included Files}
\label{sec:include}

%%%%%%%%%%%%%%%%%%%%%%%%%%%%%%%%%%%%%%%%
\DescribeMacro{\childdocmain}
To use the package, add the commands
\begin{center}
\begin{tabular}{l}
|\input{childdoc.def}|\\
|\childdocmain{}|\\
\end{tabular}
\end{center}
at the very top of the main \LaTeX{} file,
in particular \emph{before} the |\documentclass| statement!
The argument of |\childdocmain| should be left empty
(but it must be present).

%%%%%%%%%%%%%%%%%%%%%%%%%%%%%%%%%%%%%%%%
\DescribeMacro{\childdocof}
Furthermore, add the commands
\begin{center}
\begin{tabular}{l}
|\input{childdoc.def}|\\
|\childdocof{|\textit{main}|}|\\
\end{tabular}
\end{center}
at the top of every child file \textit{child}
which is included by |\include{|\textit{child}|}|
from within the main file
(or at least for those files to be compiled individually).
The argument \textit{main} must be the filename of the main file.

There are a couple of
considerations in setting up the main and child documents:

%%%%%%%%%%%%%%%%%%%%%%%%%%%%%%%%%%%%%%%%
\paragraph{Restrictions.}

Please note the following restrictions:
\begin{itemize}
\item
|\childdocmain| must be called with one argument \textit{main}
to ensure compatibility with earlier version of the package.
It must either be empty (|\childdocmain{}|)
or precisely match the filename of the main file in which it is specified.
See \secref{sec:detection} for further information.
\item
The filename \textit{main} must be specified without the |.tex| extension.
\item
The filename \textit{main} is case sensitive
(even in case-insensitive file systems)
due to internal string comparison.
\item
The argument \textit{main} should be fully expanded, it cannot be a macro.
\item
Subdirectories and special characters should be avoided in filenames.
\item
The command |\childdocmain{|\textit{main}|}| must be followed by a whitespace.
It should not be followed immediately by another command
or by a comment mark `|%|'.
This is because the \TeX{} parser reads the token immediately following
the argument of |\childdocmain| and puts it
at the beginning of every child section;
however, a white\-space is ignored.
\end{itemize}

%%%%%%%%%%%%%%%%%%%%%%%%%%%%%%%%%%%%%%%%
\paragraph{Content of Main File.}

It is advisable to place all content in the child files included by |\include|.
Any output contained in the main file will appear in all child documents
unless suppressed manually;
it cannot be suppressed automatically by the |\includeonly| directive
and thus should normally be avoided.
A method to include some content in the main file
by means of conditional processing is described in \secref{sec:conditional}.

%%%%%%%%%%%%%%%%%%%%%%%%%%%%%%%%%%%%%%%%
\paragraph{Page Numbering.}

When only a part of the document is compiled,
the appropriate numbering of pages
(as well as other status parameters)
is determined from the |.aux| files.
The latter contain information from previous passes.
However this information needs to propagate through
all intermediate child documents.
Therefore the page numbering in child documents may well
be inconsistent until the complete document is compiled at least once.

A useful (if unconventional) way to always ensure a consistent
page numbering is to restart the numbering in each child document
and denote the pages by `\textit{child}|.|\textit{page}'
where \textit{child} represents the chapter/section number of the child file.
This can be achieved by the command
|\numberwithin{page}{|\textit{child}|}|
of the \textsf{amsmath} package
where \textit{child} can be |chapter| or |section|
depending on the chosen structuring.
Alternatively, one can modify the macro |\thepage| appropriately
and reset the counter |page| at the start of each child file.

%%%%%%%%%%%%%%%%%%%%%%%%%%%%%%%%%%%%%%%%%%%%%%%%%%%%%%%%%%%%%%%%%%%%%%%%%%%%%%%%
\subsection{Conditional Processing}
\label{sec:conditional}

The package provides a mechanism to compile different versions
of a document. To customise the versions further some conditional processing
can come in handy to distinguish which version is being compiled.
The package provides two macros to describe the compilation context:

%%%%%%%%%%%%%%%%%%%%%%%%%%%%%%%%%%%%%%%%
\DescribeMacro{\ifchilddoc}
The conditional |\ifchilddoc| distinguishes between the compilation of
child documents and the main document:
%
\begin{center}
|\ifchilddoc |\textit{child-code}| |[|\||else |\textit{main-code}]| \||fi|
\end{center}

%%%%%%%%%%%%%%%%%%%%%%%%%%%%%%%%%%%%%%%%
\DescribeMacro{\childdocname}
\DescribeMacro{\childdocjob}
The macro |\childdocname| contains the filename (without extension)
of the main or child file being processed.
Note that |\childdocjob| will always contain the name of the main file.

%%%%%%%%%%%%%%%%%%%%%%%%%%%%%%%%%%%%%%%%
\paragraph{Title Page.}

Conditional processing can be used to include a title or banner page
in the main document when proper precautions are taken.
Importantly, the code in the main file should ensure that the page counter
(as well as other status parameters which are stored in the |.aux| files)
takes the same value after the conditional processing.
Otherwise the page numbers may take divergent values
depending on which part is compiled.

For example, a title page could be declared by:
%
\begin{center}
\begin{tabular}{l}
|\ifchilddoc\||else|\\
|\addtocounter{page}{-1}|\\
\textit{code for title page}\\
|\newpage|\\
|\||fi|
\end{tabular}
\end{center}
%
A banner page for the child documents can be generated by:
%
\begin{center}
\begin{tabular}{l}
|\ifchilddoc|\\
|\addtocounter{page}{-1}|\\
\textit{code for banner page}\\
|\newpage|\\
|\||fi|
\end{tabular}
\end{center}
%
Here one could write a message such as:
\begin{center}
|This is the part \childdocname{} of \childdocjob{}.|
\end{center}

%%%%%%%%%%%%%%%%%%%%%%%%%%%%%%%%%%%%%%%%%%%%%%%%%%%%%%%%%%%%%%%%%%%%%%%%%%%%%%%%
\subsection{Flags}
\label{sec:flags}

The package makes it easy to generate different versions
of the main or child documents.
To this end compilation flags can be defined
and assigned different default values.
They will be particularly useful in conjunction
with the forwarding mechanism described in \secref{sec:forward}.

For example, it may be useful to have a flag |\version|
which can be set to |draft| or |final|.
The document source will contain some conditional code
depending on the value of |\version|.
Suppose further, the flag should default to |final| for the main file
and to |draft| for child files
which is a natural assignment for editing the document.
This is achieved by placing the following code
in the preamble of the main document
(below the |\childdocmain| directive):
%
\begin{center}
\begin{tabular}{l}
|\ifchilddoc|\\
|\providecommand{\version}{draft}|\\
|\||else|\\
|\providecommand{\version}{final}|\\
|\||fi|
\end{tabular}
\end{center}
%
The definition by |\providecommand| makes sure
that previous definitions are not overwritten.
Further statements |\providecommand{\version}{...}|
can thus be added before the above code to override it.

For the main file, one might add a line
(between |\childdocmain| and the above block)
%
\begin{center}
|%\ifchilddoc\||else\providecommand{\version}{draft}\||fi|
\end{center}
%
which can be uncommented to produce a draft version.
Likewise one can add a line to the very top of a child file
(above the |\childdocof{|\textit{main}|}| directive)
%
\begin{center}
|%\providecommand{\version}{final}|
\end{center}
%
which can be uncommented to produce the final version of this child document.

%%%%%%%%%%%%%%%%%%%%%%%%%%%%%%%%%%%%%%%%%%%%%%%%%%%%%%%%%%%%%%%%%%%%%%%%%%%%%%%%
\subsection{Forwarding}
\label{sec:forward}

Different versions of the main or child documents
using compilation flags as described in \secref{sec:flags}
can be (permanently) stored in different files
for convenient compilation, viewing and distribution.
To this end, the package defines a command
to pass on compilation to a different file:

%%%%%%%%%%%%%%%%%%%%%%%%%%%%%%%%%%%%%%%%
\DescribeMacro{\childdocforward}
The command |\childdocforward| redirects processing to
another source file:
%
\begin{center}
\begin{tabular}{l}
|\input{childdoc.def}|\\
|\childdocforward[|\textit{main}|]{|\textit{dest}|}|\\
\end{tabular}
\end{center}
%
The argument \textit{dest} is the destination file
(without extension).
It should be the main file or one of the child files.
Note that further \textsf{childdoc} directives
such as |\childdocof| and |\childdocforward|
in the indicated file will be processed in this form.
The optional argument \textit{main}
passes on directly to the main file \textit{main}
while pretending to compile the child \textit{dest}.
This form behaves as if \textit{dest}
issues |\childdocof{|\textit{main}|}| right away,
and no further \textsf{childdoc} directives will be processed.

%%%%%%%%%%%%%%%%%%%%%%%%%%%%%%%%%%%%%%%%
\DescribeMacro{\...prefix}
In the alternative form |\childdocforwardprefix|,
%
\begin{center}
\begin{tabular}{l}
|\input{childdoc.def}|\\
|\childdocforwardprefix[|\textit{main}|]{|\textit{prefix}|}{|\textit{dest}|}|
\end{tabular}
\end{center}
%
the destination file is determined by a pattern
depending on the current file:
To make this work, the current file must be called
`{\textit{prefix}\hspace{0.2em}\textit{suffix}}'
with \textit{prefix} matching precisely the argument.
Processing is then passed on to the file
`{\textit{dest}\hspace{0.2em}\textit{suffix}}'.
Surely, the same effect is achieved by
directly specifying the
argument `{\textit{dest}\hspace{0.2em}\textit{suffix}}'
in the first form.
However, that requires to set up a different file
for each child. With the alternative form of the command
all these files can have exactly the same content
which simplifies setting them up and maintaining them.

For example, the following file |draft.tex|
with a compilation flag |\version| as described in \secref{sec:flags}
compiles the main document as a draft:
%
\begin{center}
\begin{tabular}{l}
|\def\version{draft}|\\
|\input{childdoc.def}|\\
|\childdocforward{|\textit{main}|}|
\end{tabular}
\end{center}
%
Likewise, the following files |final|\textit{nn}|.tex|
compile the final version of the child document
|child|\textit{nn}|.tex|:
%
\begin{center}
\begin{tabular}{l}
|\def\version{final}|\\
|\input{childdoc.def}|\\
|\childdocforwardprefix{final}{child}|
\end{tabular}
\end{center}
%

Note that when several versions of a main file and/or of each child file
are to be generated, it may be convenient to set up a |Makefile| or
shell script to automatise the process.

%%%%%%%%%%%%%%%%%%%%%%%%%%%%%%%%%%%%%%%%%%%%%%%%%%%%%%%%%%%%%%%%%%%%%%%%%%%%%%%%
\subsection{Command Line Processing}
\label{sec:commandline}

The effect of redirection files can also be achieved by invoking
the \LaTeX{} compiler with a more elaborate command line.
Most conveniently this should be done as part
of a shell script or a |Makefile|.

When using \textsf{childdoc} in the main file, the following
command lines effectively perform a redirection
(note that depending on the shell being used,
backslashes may have to be doubled: `|\|' $\to$ `|\\|'):
%
\begin{center}
|... -jobname "|\textit{target}|" |\\|"|[\textit{flags}]%
|\input{childdoc.def}\childdocforward[|\textit{main}|]{|\textit{dest}|}"|
\end{center}
%
Here \textit{target} is the name of the output file,
\textit{main} is the name of the main file
and \textit{dest} is the name of the main or child file to be processed
(all filenames without extensions).
The optional argument \textit{main} can be omitted
if \textit{main} matches \textit{dest}.
Optionally, compilation \textit{flags} can be defined via |\def| commands.
This command line makes the \TeX{} engine believe
it is compiling the file \textit{target}
whose content is specified as the latter parameter.
The provided code then forwards the processing to
\textit{main} or \textit{dest} as described in \secref{sec:forward}.

%%%%%%%%%%%%%%%%%%%%%%%%%%%%%%%%%%%%%%%%%%%%%%%%%%%%%%%%%%%%%%%%%%%%%%%%%%%%%%%%
\subsection{Include by Input}
\label{sec:input}

Including child documents by |\include| has some restrictions by design.
Most notably, the content of a child document always occupies
its own set of pages; pages cannot be shared between child documents.
Usually, this behaviour makes perfect sense
because each child document contain an essential part of the document.
However, in some situations it may be desirable to compose
a document from a collection of parts
without having mandatory page breaks between then.
For this case, the package
provides a mechanism to include parts
by |\input| which can also be processed individually.
However, by construction this mechanism
requires manual handling of the content to be output.

%%%%%%%%%%%%%%%%%%%%%%%%%%%%%%%%%%%%%%%%
\DescribeMacro{\ifchilddocmanual}
The main file should be prepared as usual, see \secref{sec:include}.
However, the document body must make a distinction
between processing of an individual part and of the main document, e.g.:
%
\begin{center}
\begin{tabular}{l}
|\ifchilddocmanual|\\
|\input{\childdocname}|\\
|\||else|\\
\textit{document body with }|\input{|\textit{part}|}|\\
|\||fi|
\end{tabular}
\end{center}
%
The conditional |\ifchilddocmanual| is true whenever
a part to be included by |\input| is being compiled,
and the name of the part is stored in |\childdocname|.

%%%%%%%%%%%%%%%%%%%%%%%%%%%%%%%%%%%%%%%%
\DescribeMacro{\childdocby}
Each part to be included by |\input| should start with:
%
\begin{center}
\begin{tabular}{l}
|\input{childdoc.def}|\\
|\childdocby{|\textit{main}|}|\\
\end{tabular}
\end{center}
%
The directive |\childdocby| is similar to |\childdocof|
described in \secref{sec:include},
but the subsequent selection of content must be done manually.
To that end, both |\ifchilddoc| and |\ifchilddocmanual|
will be true upon processing of a part,
and the name of the part is stored in |\childdocname|.
Note that |\jobname| will be set to the filename of the current part
so that each part receives an individual |.aux| file
that does not interfere with the |.aux| file(s) of the main document.
This behaviour can be altered by the alternative form
|\childdocby[*]{|\textit{main}|}| (with a non-empty optional argument)
which uses the |.aux| file of the main document
by setting |\jobname| to \textit{main}.

%%%%%%%%%%%%%%%%%%%%%%%%%%%%%%%%%%%%%%%%%%%%%%%%%%%%%%%%%%%%%%%%%%%%%%%%%%%%%%%%
\subsection{Driver Development}
\label{sec:driver}

The \textsf{childdoc} mechanism can also be use for the development
of definition files such as \LaTeX{} styles or classes.
This case differs from the above setup with multiple parts
included by |\include| in that no |\includeonly| should be invoked.
This can be achieved by starting the include file
(before |\ProvidesPackage|) with:
%
\begin{center}
\begin{tabular}{l}
|\input{childdoc.def}|\\
|\childdocforward{|\textit{main}|}|\\
\end{tabular}
\end{center}
%
or alternatively with:
%
\begin{center}
\begin{tabular}{l}
|\input{childdoc.def}|\\
|\childdocby{|\textit{main}|}|\\
\end{tabular}
\end{center}
%
Both forms have slightly different effects as described above.
The main file is prepared as usual, see \secref{sec:include}.

%%%%%%%%%%%%%%%%%%%%%%%%%%%%%%%%%%%%%%%%%%%%%%%%%%%%%%%%%%%%%%%%%%%%%%%%%%%%%%%%
\subsection{Legacy Detection}
\label{sec:detection}

The directive |\childdocmain| in the main file can detect
whether the complete document or merely a child is to be compiled
even without using the directive |\childdocof|.
This method is deprecated because it is less robust
and there is no compelling reason to use it;
it is merely provided for backward compatibility
and it may be removed in future versions.

If the detection mechanism is to be used,
it is mandatory to correctly specify
the filename of the main file as the argument of |\childdocmain|:
%
\begin{center}
\begin{tabular}{l}
|\input{childdoc.def}|\\
|\childdocmain{|\textit{main}|}|\\
\end{tabular}
\end{center}
%
If |\jobname| does not match the argument \textit{main} of |\childdocmain|,
it is assumed that |\jobname| points to the child file to be compiled.
When using |\childdocmain| with the main file specified as argument,
it suffices to start a child file
with just |\input{|\textit{main}|}|
without loading of the package and using |\childdocof|.
If instead all processing is done
with the appropriate \textsf{childdoc} directives,
the argument of \textit{main} of |\childdocmain| can be empty.

An alternative version of the command line processing described
in \secref{sec:commandline} using the detection mechanism reads:
%
\begin{center}
|... -jobname "|\textit{target}|" "|[\textit{flags}]%
[|\def\jobname{|\textit{dest}|}|]|\input{|\textit{main}|}"|
\end{center}

%%%%%%%%%%%%%%%%%%%%%%%%%%%%%%%%%%%%%%%%%%%%%%%%%%%%%%%%%%%%%%%%%%%%%%%%%%%%%%%%
\subsection{Manual Code}
\label{sec:manual}

In case one cannot be certain whether the definitions file |childdoc.def|
is installed on the target \TeX{} distribution
and one prefers not to ship it,
it is conceivable to paste a few relevant commands into the sources.

To that end, drop all statements |\input{childdoc.def}|
and perform the replacements as outlined below.
Instead of |\childdocmain{|\textit{main}|}| add the following code
to the top of the main file:
%
\begin{center}
\begin{tabular}{l}
|\||ifdefined\childdocname\endinput\||fi\newif\ifchilddoc|\\
|\edef\childdocname{\scantokens\expandafter{\jobname\noexpand}}|\\
|\def\childdocmain{|\textit{main}|}\||ifx\childdocmain\childdocname\||else|\\
|\childdoctrue\includeonly{\childdocname}\let\jobname\childdocmain\||fi|\\
\end{tabular}
\end{center}
%
Instead of |\childdocof{|\textit{main}|}| just include the main file
at the top of each child file:
%
\begin{center}
|\input{|\textit{main}|}|
\end{center}
%
A simple redirection |\childdocforward{|\textit{dest}|}| is achieved by:
%
\begin{center}
|\def\jobname{|\textit{dest}|}\input{\jobname}|
\end{center}
%
The redirection with prefix
|\childdocforwardprefix[|\textit{prefix}|]{|\textit{dest}|}|
is accomplished by:
%
\begin{center}
\begin{tabular}{l}
|{\edef\jobname{\scantokens\expandafter{\jobname\noexpand}}|\\
|\def\redirectjob |\textit{prefix}|#1~~~{\gdef\jobname{|\textit{dest}|#1}}|\\
|\expandafter\redirectjob\jobname~~~}\input{\jobname}|
\end{tabular}
\end{center}

In an alternative approach,
child documents can be compiled by a specific command line
without additional code or specific definitions:
%
\begin{center}
|... -jobname "|\textit{target}|" "|[\textit{flags}]%
|\includeonly{|\textit{dest}|}\input{|\textit{main}|}"|
\end{center}
%

%%%%%%%%%%%%%%%%%%%%%%%%%%%%%%%%%%%%%%%%%%%%%%%%%%%%%%%%%%%%%%%%%%%%%%%%%%%%%%%%
%%%%%%%%%%%%%%%%%%%%%%%%%%%%%%%%%%%%%%%%%%%%%%%%%%%%%%%%%%%%%%%%%%%%%%%%%%%%%%%%
\section{Information}

%%%%%%%%%%%%%%%%%%%%%%%%%%%%%%%%%%%%%%%%%%%%%%%%%%%%%%%%%%%%%%%%%%%%%%%%%%%%%%%%
\subsection{Copyright}

Copyright \copyright{} 2017--2018 Niklas Beisert

This work may be distributed and/or modified under the
conditions of the \LaTeX{} Project Public License, either version 1.3
of this license or (at your option) any later version.
The latest version of this license is in
  \url{http://www.latex-project.org/lppl.txt}
and version 1.3 or later is part of all distributions of \LaTeX{}
version 2005/12/01 or later.

This work has the LPPL maintenance status `maintained'.

The Current Maintainer of this work is Niklas Beisert.

This work consists of the files |README.txt|, |childdoc.ins| and |childdoc.dtx|
as well as the derived files |childdoc.def|, |cdocsamp.tex|
with |cdocsch1.tex|, |cdocsch2.tex|, |cdocspt3.tex|, |cdocspt4.tex|,
|cdocsdrf.tex|, |cdocsfn1.tex|, |cdocsfn2.tex|
as well as |childdoc.pdf|.

%%%%%%%%%%%%%%%%%%%%%%%%%%%%%%%%%%%%%%%%%%%%%%%%%%%%%%%%%%%%%%%%%%%%%%%%%%%%%%%%
\subsection{Files and Installation}

The package consists of the files:
%
\begin{center}
\begin{tabular}{ll}
    |README.txt|   & readme file \\
    |childdoc.ins| & installation file \\
    |childdoc.dtx| & source file \\
    |childdoc.def| & definition file \\
    |cdocsamp.tex| & sample main file \\
    |cdocsch1.tex| & sample include file \\
    |cdocsch2.tex| & sample include file \\
    |cdocspt3.tex| & sample part file \\
    |cdocspt4.tex| & sample part file \\
    |cdocsdrf.tex| & sample redirection file \\
    |cdocsfn1.tex| & sample redirection file \\
    |cdocsfn2.tex| & sample redirection file \\
    |childdoc.pdf| & manual
\end{tabular}
\end{center}
%
The distribution consists of the files
|README.txt|, |childdoc.ins| and |childdoc.dtx|.
%
\begin{itemize}
\item
Run (pdf)\LaTeX{} on |childdoc.dtx|
to compile the manual |childdoc.pdf| (this file).
\item
Run \LaTeX{} on |childdoc.ins| to create the definitions file |childdoc.def|
and the sample |cdocsamp.tex| with include files
|cdocsch1.tex|, |cdocsch2.tex|, |cdocspt3.tex|, |cdocspt4.tex|,
|cdocsdrf.tex|, |cdocsfn1.tex|, |cdocsfn2.tex|.
Then copy the file |childdoc.def| to an appropriate directory of your \LaTeX{}
distribution, e.g.\ \textit{texmf-root}|/tex/latex/childdoc|.
\end{itemize}

%%%%%%%%%%%%%%%%%%%%%%%%%%%%%%%%%%%%%%%%%%%%%%%%%%%%%%%%%%%%%%%%%%%%%%%%%%%%%%%%
\subsection{Related CTAN Packages}

There are several other packages which offer a similar functionality:
%
\begin{itemize}
\item
The packages
\href{http://ctan.org/pkg/docmute}{\textsf{docmute}},
\href{http://ctan.org/pkg/includex}{\textsf{includex}} and
\href{http://ctan.org/pkg/standalone}{\textsf{standalone}}
provide commands to include only the document body of
a child file thus allowing both files to be compiled individually.
\item
The packages \href{http://ctan.org/pkg/subdocs}{\textsf{subdocs}}
and \href{http://ctan.org/pkg/subfiles}{\textsf{subfiles}}
provide structures in which the main and child documents can be
encapsulated and allowing them to be compiled individually.
The inclusion mechanism is different from the conventional |\include|.
\item
The package \href{http://ctan.org/pkg/combine}{\textsf{combine}}
is an elaborate solution to combine several documents into one.
\end{itemize}
%
See also the CTAN topic \href{http://ctan.org/topic/subdocs}{\textsf{subdocs}}
for further related packages.
The present package differs from the above solutions in that
a document structure constructed with the conventional |\include| mechanism
just needs two extra commands at the top of every file
such that all constituent files can be compiled individually.

%%%%%%%%%%%%%%%%%%%%%%%%%%%%%%%%%%%%%%%%%%%%%%%%%%%%%%%%%%%%%%%%%%%%%%%%%%%%%%%%
%\subsection{Feature Suggestions}
%
%The following is a list of features which may be useful for future
%versions of this package:
%%
%\begin{itemize}
%\item
%\ldots
%\end{itemize}

%%%%%%%%%%%%%%%%%%%%%%%%%%%%%%%%%%%%%%%%%%%%%%%%%%%%%%%%%%%%%%%%%%%%%%%%%%%%%%%%
\subsection{Revision History}

%%%%%%%%%%%%%%%%%%%%%%%%%%%%%%%%%%%%%%%%
\paragraph{v2.0:} 2018/12/30

\begin{itemize}
\item
immediate forward processing
\item
added |\childdocby| mechanism
\item
manual restructured
\end{itemize}

%%%%%%%%%%%%%%%%%%%%%%%%%%%%%%%%%%%%%%%%
\paragraph{v1.6:} 2018/01/17

\begin{itemize}
\item
application for development of include files
\item
corrections to manual
\end{itemize}

%%%%%%%%%%%%%%%%%%%%%%%%%%%%%%%%%%%%%%%%
\paragraph{v1.5:} 2017/05/21

\begin{itemize}
\item
more complete structuring introduced
\item
|\childdocof| introduced
\item
|\childdoc| renamed to |\childdocmain|
\item
|\childredirect| renamed to |\childdocforward| and |\childdocforwardprefix|
and functionality expanded
\end{itemize}

%%%%%%%%%%%%%%%%%%%%%%%%%%%%%%%%%%%%%%%%
\paragraph{v1.0:} 2017/04/27

\begin{itemize}
\item
manual and install package
\item
first version published on CTAN
\end{itemize}

%%%%%%%%%%%%%%%%%%%%%%%%%%%%%%%%%%%%%%%%
\paragraph{v0.6:} 2017/04/26

\begin{itemize}
\item
redirection mechanism added
\end{itemize}

%%%%%%%%%%%%%%%%%%%%%%%%%%%%%%%%%%%%%%%%
\paragraph{v0.5:} 2017/04/26

\begin{itemize}
\item
functionality in definition file
\end{itemize}


%%%%%%%%%%%%%%%%%%%%%%%%%%%%%%%%%%%%%%%%%%%%%%%%%%%%%%%%%%%%%%%%%%%%%%%%%%%%%%%%
%%%%%%%%%%%%%%%%%%%%%%%%%%%%%%%%%%%%%%%%%%%%%%%%%%%%%%%%%%%%%%%%%%%%%%%%%%%%%%%%
%%%%%%%%%%%%%%%%%%%%%%%%%%%%%%%%%%%%%%%%%%%%%%%%%%%%%%%%%%%%%%%%%%%%%%%%%%%%%%%%
\appendix

\settowidth\MacroIndent{\rmfamily\scriptsize 000\ }

 \DocInput{childdoc.dtx}

\end{document}
%</driver>
% \fi
%
% %%%%%%%%%%%%%%%%%%%%%%%%%%%%%%%%%%%%%%%%%%%%%%%%%%%%%%%%%%%%%%%%%%%%%%%%%%%%%%
% %%%%%%%%%%%%%%%%%%%%%%%%%%%%%%%%%%%%%%%%%%%%%%%%%%%%%%%%%%%%%%%%%%%%%%%%%%%%%%
% \section{Sample}
%\iffalse
%<*samplemain>
%\fi
%
% The following presents a sample document
% with two chapters, two parts, a title page,
% a compile flag as well as three forwarding files to set the flag.
% It consists of eight |.tex| files:
% \begin{center}
% \begin{tabular}{ll}
% |cdocsamp.tex|&main file\\
% |cdocsch1.tex|&include file for chapter 1\\
% |cdocsch2.tex|&include file for chapter 2\\
% |cdocspt3.tex|&include file for part 3\\
% |cdocspt4.tex|&include file for part 4\\
% |cdocsdrf.tex|&forwarding file for main file in draft mode\\
% |cdocsfi1.tex|&forwarding file for final version of chapter 1\\
% |cdocsfi2.tex|&forwarding file for final version of chapter 2\\
% \end{tabular}
% \end{center}
% Each of the eight files can be compiled directly by the \LaTeX{} compiler.
%
% %%%%%%%%%%%%%%%%%%%%%%%%%%%%%%%%%%%%%%
% \paragraph{Main File.}
%
% The main file is called |cdocsamp.tex|.
%
% Load the \textsf{childdoc} definitions and
% declare the filename for the main document:
%    \begin{macrocode}
\input{childdoc.def}
\childdocmain{}
%    \end{macrocode}

% Optional override for |\version| flag:
%    \begin{macrocode}
%%\ifchilddoc\else\providecommand{\version}{draft}\fi
%    \end{macrocode}

% Define the default values for the |\version| flag
% (|final| for the main file and |draft| for childs):
%    \begin{macrocode}
\ifchilddoc
\providecommand{\version}{draft}
\else
\providecommand{\version}{final}
\fi
%    \end{macrocode}

% Load the standard document class:
%    \begin{macrocode}
\documentclass[12pt]{article}
%    \end{macrocode}

% Start the document body:
%    \begin{macrocode}
\begin{document}
%    \end{macrocode}

% Declare a title page.
% Print title, part of document being processed and version flag:
%    \begin{macrocode}
\addtocounter{page}{-1}
\begin{center}
{\LARGE\bfseries{}childdoc example\par}
\vspace{1cm}
\ifchilddoc
\ifchilddocmanual part\else chapter\fi:
`\childdocname' of `\childdocjob'\par
\else
main document: `\childdocjob'\par
\fi
version: \version\par
\end{center}
\newpage
%    \end{macrocode}

% Manually include selected file,
% otherwise process as usual:
%    \begin{macrocode}
\ifchilddocmanual
\section*{part `\childdocname'}
\input{\childdocname}
\else
%    \end{macrocode}

% Include the two chapters:
%    \begin{macrocode}
\include{cdocsch1}
\include{cdocsch2}
%    \end{macrocode}

% Include the two parts unless only chapters should be displayed:
%    \begin{macrocode}
\ifchilddoc\else
\section{part three}
\input{cdocspt3}
\section{part four}
\input{cdocspt4}
\fi
%    \end{macrocode}

% Process as usual until here:
%    \begin{macrocode}
\fi
%    \end{macrocode}

% End of document body:
%    \begin{macrocode}
\end{document}
%    \end{macrocode}
%\iffalse
%</samplemain>
%\fi
%
% %%%%%%%%%%%%%%%%%%%%%%%%%%%%%%%%%%%%%%
% \paragraph{Chapter Include Files.}
%
% The include files are called |cdocsch1.tex| and |cdocsch2.tex|.
%
%\iffalse
%<*samplechap1|samplechap2>
%\fi

% Optional override for |\version| flag:
%    \begin{macrocode}
%%\providecommand{\version}{final}
%    \end{macrocode}

% Include the main document:
%    \begin{macrocode}
\input{childdoc.def}
\childdocof{cdocsamp}
%    \end{macrocode}

%\iffalse
%</samplechap1|samplechap2>
%\fi
%
%\iffalse
%<*samplechap1>
%\fi
% Some text for chapter 1:
%    \begin{macrocode}
\section{one}
some text in chapter one
%    \end{macrocode}

%\iffalse
%</samplechap1>
%\fi
% Some text for chapter 2:
%\iffalse
%<*samplechap2>
%\fi
%    \begin{macrocode}
\section{two}
more text in chapter two
%    \end{macrocode}

%\iffalse
%</samplechap2>
%\fi
%
% %%%%%%%%%%%%%%%%%%%%%%%%%%%%%%%%%%%%%%
% \paragraph{Part Include Files.}
%
% The include files are called |cdocspt3.tex| and |cdocspt4.tex|.
%
%\iffalse
%<*samplepart3|samplepart4>
%\fi

% Optional override for |\version| flag:
%    \begin{macrocode}
%%\providecommand{\version}{final}
%    \end{macrocode}

% Include the main document:
%    \begin{macrocode}
\input{childdoc.def}
\childdocby{cdocsamp}
%    \end{macrocode}

%\iffalse
%</samplepart3|samplepart4>
%\fi
%
%\iffalse
%<*samplepart3>
%\fi
% Some text for part 3:
%    \begin{macrocode}
some text in part three
%    \end{macrocode}

%\iffalse
%</samplepart3>
%\fi
% Some text for part 4:
%\iffalse
%<*samplepart4>
%\fi
%    \begin{macrocode}
more text in part four
%    \end{macrocode}

%\iffalse
%</samplepart4>
%\fi
%
% %%%%%%%%%%%%%%%%%%%%%%%%%%%%%%%%%%%%%%
% \paragraph{Forwarding for a Complete Draft.}
%
% The following forwarding file |cdocsdrf.tex|
% compiles the main document in draft mode:
%\iffalse
%<*sampledraft>
%\fi
%    \begin{macrocode}
\def\version{draft}
\input{childdoc.def}
\childdocforward{cdocsamp}
%    \end{macrocode}

%\iffalse
%</sampledraft>
%\fi
%
% %%%%%%%%%%%%%%%%%%%%%%%%%%%%%%%%%%%%%%
% \paragraph{Forwarding for Final Version of the Chapters.}
%
% The following forwarding files |cdocsfn1.tex| and |cdocsfn2.tex|
% (with identical content)
% compile the final versions of the child documents
% |cdocsch1.tex| and |cdocsch2.tex|, respectively:
%\iffalse
%<*samplefinal>
%\fi
%    \begin{macrocode}
\def\version{final}
\input{childdoc.def}
\childdocforwardprefix[cdocsamp]{cdocsfn}{cdocsch}
%    \end{macrocode}

%\iffalse
%</samplefinal>
%\fi
%
% %%%%%%%%%%%%%%%%%%%%%%%%%%%%%%%%%%%%%%
% \paragraph{Command Line Processing.}
%
% The following three command lines generate the output files
% |cdocscld|, |cdocscl1| and |cdocscl2|
% which should be identical to
% |cdocsdrf|, |cdocsch1| and |cdocsfn2|, respectively:
% \begin{center}
% \begin{tabular}{l}
% |latex -jobname cdocscld \|\\
% |  "\def\version{draft}\input{childdoc.def}\childdocforward{cdocsamp}"|\\
% |latex -jobname cdocscl1 \|\\
% |  "\input{childdoc.def}\childdocforward[cdocsamp]{cdocsch1}"|\\
% |latex -jobname cdocscl2 \|\\
% |  "\def\version{final}\input{childdoc.def}\childdocforward{cdocsch2}"|
% \end{tabular}
% \end{center}
% Note that the trailing backslash on each first line
% merely continues the input to the second line
% (for convenient cut ant paste).
% Furthermore, the command |latex| can be replaced by any
% of its alternative versions such as |pdflatex|.
%
% %%%%%%%%%%%%%%%%%%%%%%%%%%%%%%%%%%%%%%%%%%%%%%%%%%%%%%%%%%%%%%%%%%%%%%%%%%%%%%
% %%%%%%%%%%%%%%%%%%%%%%%%%%%%%%%%%%%%%%%%%%%%%%%%%%%%%%%%%%%%%%%%%%%%%%%%%%%%%%
% \section{Implementation}
%\iffalse
%<*package>
%\fi
%
% This section describes the definitions file |childdoc.def|.

% The definitions cannot be loaded using |\usepackage| or |\RequirePackage|
% which has a mechanism to prevent loading a style file more than once.
% When loading the definitions by means of |\input|
% multiple instances have to be prevented manually:
%\iffalse
%This code needs to be before the `\ProvidesFile' directive
%which is defined at the beginning of this file.
%Therefore it is also placed there and commented out here.
%</package>
%<*discard>
%\fi
%    \begin{macrocode}
\ifdefined\childdocmain\endinput\fi
%    \end{macrocode}
%\iffalse
%</discard>
%<*package>
%\fi
%
% \macro{\ifchilddoc}
% \macro{\ifchilddocmanual}
% The conditional |\ifchilddoc| tells whether a
% child (true) or main (false) document is being compiled.
% The conditional |\ifchilddocmanual| tells whether
% the |\includeonly| mechanism is used (false) or
% the selection of child files must be performed manually (true).
% The definitions initialise to false:
%    \begin{macrocode}
\newif\ifchilddoc
\newif\ifchilddocmanual
%    \end{macrocode}

% \macro{\childdocname}
% \macro{\childdocjob}
% The macro |\childdocname| stores the name of the main document
% to be compiled. The macro |\childdocjob| stores the name of
% the document on which the \LaTeX{} compiler was originally invoked.
% The content of |\jobname| cannot be compared
% to filenames specified in the source due to different catcodes.
% The following code rescans |\jobname|, stores the result
% in |\childdocname| and saves a copy in |\childdocjob|:
%    \begin{macrocode}
\edef\childdocname{\scantokens\expandafter{\jobname\noexpand}}
\let\childdocjob\childdocname
%    \end{macrocode}

% \macro{\childdocdisable}
% The macro |\childdocdisable| prevents the main file
% from being processed more than once.
% At this stage, the main document command |\childdocmain|
% is assumed to be called once again where it should do nothing.
% Any subsequent call to it should prevent
% a secondary processing of the main document
% It overwrites the forwarding commands
% |\childdocof| and |\childdocforward|
% with empty macros to prevent further inclusions of the main document:
%    \begin{macrocode}
\newcommand{\childdocdisable}
{
  \renewcommand{\childdocmain}[1]{\renewcommand{\childdocmain}[1]{\endinput}}
  \renewcommand{\childdocof}[1]{}
  \renewcommand{\childdocby}[2][]{}
  \renewcommand{\childdocforward}[2][]{}
  \renewcommand{\childdocdisable}{}
}
%    \end{macrocode}

% \macro{\childdocmain}
% The macro |\childdocmain| is to be called at the top of the main file
% with nothing or the main filename (without extension) as argument.
% First, it breaks loops.
% If the argument is not empty and does not match |\childdocname|
% (which is set by the first inclusion of |childdoc.def|),
% |\ifchilddoc| is set to true, |\includeonly| is applied to the child file
% and |\jobname| is set to the main file
% (for proper handling of |.aux| files):
%    \begin{macrocode}
\newcommand{\childdocmain}[1]
{
  \childdocdisable\childdocmain{}
  \if?#1?\else
    \begingroup
      \def\childdoctmp{#1}
      \ifx\childdoctmp\childdocname
        \def\childdoctmp{}
      \else
        \def\childdoctmp
        {
          \childdoctrue
          \includeonly{\childdocname}
          \def\childdocjob{#1}
          \def\jobname{#1}
        }
      \fi
      \expandafter
    \endgroup
    \childdoctmp
  \fi
}
%    \end{macrocode}

% \macro{\childdocof}
% The command |\childdocof| redirects
% compilation to the main file |#1|.
%    \begin{macrocode}
\newcommand{\childdocof}[1]
{
  \childdocdisable
  \childdoctrue
  \includeonly{\childdocname}
  \def\jobname{#1}
  \def\childdocjob{#1}
  \input{#1}
}
%    \end{macrocode}

% \macro{\childdocby}
% The command |\childdocby| ....
%    \begin{macrocode}
\newcommand{\childdocby}[2][]
{
  \childdocdisable
  \childdoctrue
  \childdocmanualtrue
  \if?#1?\else
    \def\jobname{#2}
  \fi
  \def\childdocjob{#2}
  \input{#2}
  \endinput
}
%    \end{macrocode}

% \macro{\childdocforward}
% The command |\childdocforward| redirects
% compilation to the main file or
% (if the optional argument is given) a child file.
% Parameters are set as if the main file
% or a child file starting with |\childdocof| was compiled.
% Then compilation is handed over to the main file:
%    \begin{macrocode}
\newcommand{\childdocforward}[2][]
{
  \begingroup
    \if?#1?
      \def\childdoctmp
      {
        \def\childdocname{#2}
        \def\childdocjob{#2}
        \def\jobname{#2}
        \input{#2}
        \endinput
      }
    \else
      \def\childdoctmp
      {
        \childdocdisable
        \def\childdocname{#2}
        \childdoctrue
        \includeonly{#2}
        \def\childdocjob{#1}
        \def\jobname{#1}
        \input{#1}
        \endinput
      }
    \fi
    \expandafter
  \endgroup
  \childdoctmp
}
%    \end{macrocode}

% \macro{\childdocforwardprefix}
% The command |\childdocforwardprefix| redirects
% compilation to the main or a child file by means of a pattern.
% The prefix |#1| in the current filename is replaced by |#2|
% and the suffix of the current filename is kept
% (it is assumed that the filename does not contain the substring `|~~~|'
% which is used as a delimiter).
% Compilation is handed over to the new file by |\childdocforward|:
%    \begin{macrocode}
\newcommand{\childdocforwardprefix}[3][]
{
  \begingroup
    \def\childdocextract #2##1~~~{\def\childdoctmp{\childdocforward[#1]{#3##1}}}
    \expandafter\childdocextract\childdocname~~~
    \expandafter
  \endgroup
  \childdoctmp
}
%    \end{macrocode}

% \macro{\childdoc}
% The deprecated macro |\childdoc| is a legacy version of |\childdocmain|:
%    \begin{macrocode}
\newcommand{\childdoc}{\childdocmain}
%    \end{macrocode}

% \macro{\childdocredirect}
% The deprecated macro |\childdocredirect| is a legacy version
% of |\childdocforward| and |\childdocforwardprefix|:
%    \begin{macrocode}
\newcommand{\childdocredirect}[2][]
{
  \begingroup
    \if?#1?
      \def\childdoctmp{\childdocforward{#2}}
    \else
      \def\childdoctmp{\childdocforwardprefix{#1}{#2}}
    \fi
    \expandafter
  \endgroup
  \childdoctmp
}
%    \end{macrocode}

%\iffalse
%</package>
%\fi
%
\endinput

\childdocby{cdocsamp}
%    \end{macrocode}

%\iffalse
%</samplepart3|samplepart4>
%\fi
%
%\iffalse
%<*samplepart3>
%\fi
% Some text for part 3:
%    \begin{macrocode}
some text in part three
%    \end{macrocode}

%\iffalse
%</samplepart3>
%\fi
% Some text for part 4:
%\iffalse
%<*samplepart4>
%\fi
%    \begin{macrocode}
more text in part four
%    \end{macrocode}

%\iffalse
%</samplepart4>
%\fi
%
% %%%%%%%%%%%%%%%%%%%%%%%%%%%%%%%%%%%%%%
% \paragraph{Forwarding for a Complete Draft.}
%
% The following forwarding file |cdocsdrf.tex|
% compiles the main document in draft mode:
%\iffalse
%<*sampledraft>
%\fi
%    \begin{macrocode}
\def\version{draft}
% \iffalse
%
% childdoc.dtx Copyright (C) 2017-2018 Niklas Beisert
%
% This work may be distributed and/or modified under the
% conditions of the LaTeX Project Public License, either version 1.3
% of this license or (at your option) any later version.
% The latest version of this license is in
%   http://www.latex-project.org/lppl.txt
% and version 1.3 or later is part of all distributions of LaTeX
% version 2005/12/01 or later.
%
% This work has the LPPL maintenance status `maintained'.
%
% The Current Maintainer of this work is Niklas Beisert.
%
% This work consists of the files childdoc.dtx and childdoc.ins
% and the derived files childdoc.def and cdocsamp.tex with
% cdocsch1.tex, cdocsch2.tex, cdocsdrf.tex, cdocsfn1.tex, cdocsfn2.tex.
%
%<package>\ifdefined\childdocmain\endinput\fi
%<package>\ProvidesFile{childdoc.def}[2018/12/30 v2.0 child document driver]
%<samplemain>\ProvidesFile{cdocsamp.tex}[2018/12/30 v2.0 sample for childdoc]
%<*driver>
%\ProvidesFile{childdoc.drv}[2018/12/30 v2.0 childdoc reference manual file]
\PassOptionsToClass{10pt,a4paper}{article}
\documentclass{ltxdoc}

\usepackage[margin=35mm]{geometry}
\usepackage{hyperref}
\usepackage{hyperxmp}
\usepackage[usenames]{color}

\hypersetup{colorlinks=true}
\hypersetup{pdfstartview=FitH}
\hypersetup{pdfpagemode=UseNone}
\hypersetup{pdfsource={}}
\hypersetup{pdflang={en-UK}}
\hypersetup{pdfcopyright={Copyright 2017-2018 Niklas Beisert.
  This work may be distributed and/or modified under the
  conditions of the LaTeX Project Public License, either version 1.3
  of this license or (at your option) any later version.}}
\hypersetup{pdflicenseurl={http://www.latex-project.org/lppl.txt}}
\hypersetup{pdfcontactaddress={ETH Zurich, ITP, HIT K,
  Wolfgang-Pauli-Strasse 27}}
\hypersetup{pdfcontactpostcode={8093}}
\hypersetup{pdfcontactcity={Zurich}}
\hypersetup{pdfcontactcountry={Switzerland}}
\hypersetup{pdfcontactemail={nbeisert@itp.phys.ethz.ch}}
\hypersetup{pdfcontacturl={http://people.phys.ethz.ch/\xmptilde nbeisert/}}

\newcommand{\secref}[1]{\hyperref[#1]{section \ref*{#1}}}

\parskip1ex
\parindent0pt
\let\olditemize\itemize
\def\itemize{\olditemize\parskip0pt}

\begin{document}

\title{The \textsf{childdoc} Package}
\hypersetup{pdftitle={The childdoc Package}}
\author{Niklas Beisert\\[2ex]
  Institut f\"ur Theoretische Physik\\
  Eidgen\"ossische Technische Hochschule Z\"urich\\
  Wolfgang-Pauli-Strasse 27, 8093 Z\"urich, Switzerland\\[1ex]
  \href{mailto:nbeisert@itp.phys.ethz.ch}
  {\texttt{nbeisert@itp.phys.ethz.ch}}}
\hypersetup{pdfauthor={Niklas Beisert}}
\hypersetup{pdfsubject={Manual for the LaTeX2e Package childdoc}}
\date{30 December 2018, \textsf{v2.0}}
\maketitle

\begin{abstract}\noindent
\textsf{childdoc} is a \LaTeXe{} package
that enables the direct compilation
of document sections included by |\include|
to individual files.
\end{abstract}

\begingroup
\parskip0ex
\tableofcontents
\endgroup

%%%%%%%%%%%%%%%%%%%%%%%%%%%%%%%%%%%%%%%%%%%%%%%%%%%%%%%%%%%%%%%%%%%%%%%%%%%%%%%%
%%%%%%%%%%%%%%%%%%%%%%%%%%%%%%%%%%%%%%%%%%%%%%%%%%%%%%%%%%%%%%%%%%%%%%%%%%%%%%%%
\section{Introduction}

\LaTeX{} provides a mechanism to structure a large document (such as a book)
into a main file and several child files (containing the chapters)
using the |\include| command.
This mechanism is beneficial for documents
which span hundreds of pages in order to
make the source file(s) more manageable.
Moreover, compilation can be restricted to
selected child files by means of the |\includeonly| command.
The latter feature can be used to reduce the compilation time while editing
(this was significantly more useful in the earlier days of \LaTeX{})
or to generate a smaller document which is easier to navigate.
Another application of |\includeonly| is to generate
documents consisting of selected parts of the complete document.

However, there are a few drawbacks of the plain |\include| mechanism:
\begin{itemize}
\item
The child files cannot be compiled on their own,
they can only be compiled via the main file.
A naive editing environment
(such as a text editor with an option
to have the current file processed by \LaTeX)
may require one to switch to the main file before compiling;
attempting to compile the child file produces errors.
\item
The main file must be modified (each time)
to adjust the |\includeonly| command
to the present needs. This easily leaves the main file in a messy state.
\item
The generated document will always carry the filename
of the main document. This is inconvenient if
several child files are to be compiled and
to be kept for distribution.
\end{itemize}

The present package provides a simple interface
to make child files individually compilable by \LaTeX{}.
Compiling a child file then has the same effect as compiling
the main file with an |\includeonly| command
to select the appropriate child.
Moreover the generated document will carry the name of the child
rather than the main file.
This resolves all three above issues.

This feature is meant to make the editing of books,
thesis documents and lecture notes somewhat more convenient.
However, the package can also be used efficiently for
composing a series of documents (such as exercise sheets)
which are typically distributed individually.
It then assists the author in generating the individual documents
(potentially in different versions)
as well as a document containing the collected series.
Another application is in developing style files
or other kinds of included material
where compilation of the style file could redirect
to a sample or test file.

%%%%%%%%%%%%%%%%%%%%%%%%%%%%%%%%%%%%%%%%%%%%%%%%%%%%%%%%%%%%%%%%%%%%%%%%%%%%%%%%
%%%%%%%%%%%%%%%%%%%%%%%%%%%%%%%%%%%%%%%%%%%%%%%%%%%%%%%%%%%%%%%%%%%%%%%%%%%%%%%%
\section{Usage}

First of all, the package \textsf{childdoc} is \emph{not} a standard
\LaTeXe{} |.sty| style file! Therefore it needs to be invoked in
a non-standard way.

%%%%%%%%%%%%%%%%%%%%%%%%%%%%%%%%%%%%%%%%%%%%%%%%%%%%%%%%%%%%%%%%%%%%%%%%%%%%%%%%
\subsection{Included Files}
\label{sec:include}

%%%%%%%%%%%%%%%%%%%%%%%%%%%%%%%%%%%%%%%%
\DescribeMacro{\childdocmain}
To use the package, add the commands
\begin{center}
\begin{tabular}{l}
|\input{childdoc.def}|\\
|\childdocmain{}|\\
\end{tabular}
\end{center}
at the very top of the main \LaTeX{} file,
in particular \emph{before} the |\documentclass| statement!
The argument of |\childdocmain| should be left empty
(but it must be present).

%%%%%%%%%%%%%%%%%%%%%%%%%%%%%%%%%%%%%%%%
\DescribeMacro{\childdocof}
Furthermore, add the commands
\begin{center}
\begin{tabular}{l}
|\input{childdoc.def}|\\
|\childdocof{|\textit{main}|}|\\
\end{tabular}
\end{center}
at the top of every child file \textit{child}
which is included by |\include{|\textit{child}|}|
from within the main file
(or at least for those files to be compiled individually).
The argument \textit{main} must be the filename of the main file.

There are a couple of
considerations in setting up the main and child documents:

%%%%%%%%%%%%%%%%%%%%%%%%%%%%%%%%%%%%%%%%
\paragraph{Restrictions.}

Please note the following restrictions:
\begin{itemize}
\item
|\childdocmain| must be called with one argument \textit{main}
to ensure compatibility with earlier version of the package.
It must either be empty (|\childdocmain{}|)
or precisely match the filename of the main file in which it is specified.
See \secref{sec:detection} for further information.
\item
The filename \textit{main} must be specified without the |.tex| extension.
\item
The filename \textit{main} is case sensitive
(even in case-insensitive file systems)
due to internal string comparison.
\item
The argument \textit{main} should be fully expanded, it cannot be a macro.
\item
Subdirectories and special characters should be avoided in filenames.
\item
The command |\childdocmain{|\textit{main}|}| must be followed by a whitespace.
It should not be followed immediately by another command
or by a comment mark `|%|'.
This is because the \TeX{} parser reads the token immediately following
the argument of |\childdocmain| and puts it
at the beginning of every child section;
however, a white\-space is ignored.
\end{itemize}

%%%%%%%%%%%%%%%%%%%%%%%%%%%%%%%%%%%%%%%%
\paragraph{Content of Main File.}

It is advisable to place all content in the child files included by |\include|.
Any output contained in the main file will appear in all child documents
unless suppressed manually;
it cannot be suppressed automatically by the |\includeonly| directive
and thus should normally be avoided.
A method to include some content in the main file
by means of conditional processing is described in \secref{sec:conditional}.

%%%%%%%%%%%%%%%%%%%%%%%%%%%%%%%%%%%%%%%%
\paragraph{Page Numbering.}

When only a part of the document is compiled,
the appropriate numbering of pages
(as well as other status parameters)
is determined from the |.aux| files.
The latter contain information from previous passes.
However this information needs to propagate through
all intermediate child documents.
Therefore the page numbering in child documents may well
be inconsistent until the complete document is compiled at least once.

A useful (if unconventional) way to always ensure a consistent
page numbering is to restart the numbering in each child document
and denote the pages by `\textit{child}|.|\textit{page}'
where \textit{child} represents the chapter/section number of the child file.
This can be achieved by the command
|\numberwithin{page}{|\textit{child}|}|
of the \textsf{amsmath} package
where \textit{child} can be |chapter| or |section|
depending on the chosen structuring.
Alternatively, one can modify the macro |\thepage| appropriately
and reset the counter |page| at the start of each child file.

%%%%%%%%%%%%%%%%%%%%%%%%%%%%%%%%%%%%%%%%%%%%%%%%%%%%%%%%%%%%%%%%%%%%%%%%%%%%%%%%
\subsection{Conditional Processing}
\label{sec:conditional}

The package provides a mechanism to compile different versions
of a document. To customise the versions further some conditional processing
can come in handy to distinguish which version is being compiled.
The package provides two macros to describe the compilation context:

%%%%%%%%%%%%%%%%%%%%%%%%%%%%%%%%%%%%%%%%
\DescribeMacro{\ifchilddoc}
The conditional |\ifchilddoc| distinguishes between the compilation of
child documents and the main document:
%
\begin{center}
|\ifchilddoc |\textit{child-code}| |[|\||else |\textit{main-code}]| \||fi|
\end{center}

%%%%%%%%%%%%%%%%%%%%%%%%%%%%%%%%%%%%%%%%
\DescribeMacro{\childdocname}
\DescribeMacro{\childdocjob}
The macro |\childdocname| contains the filename (without extension)
of the main or child file being processed.
Note that |\childdocjob| will always contain the name of the main file.

%%%%%%%%%%%%%%%%%%%%%%%%%%%%%%%%%%%%%%%%
\paragraph{Title Page.}

Conditional processing can be used to include a title or banner page
in the main document when proper precautions are taken.
Importantly, the code in the main file should ensure that the page counter
(as well as other status parameters which are stored in the |.aux| files)
takes the same value after the conditional processing.
Otherwise the page numbers may take divergent values
depending on which part is compiled.

For example, a title page could be declared by:
%
\begin{center}
\begin{tabular}{l}
|\ifchilddoc\||else|\\
|\addtocounter{page}{-1}|\\
\textit{code for title page}\\
|\newpage|\\
|\||fi|
\end{tabular}
\end{center}
%
A banner page for the child documents can be generated by:
%
\begin{center}
\begin{tabular}{l}
|\ifchilddoc|\\
|\addtocounter{page}{-1}|\\
\textit{code for banner page}\\
|\newpage|\\
|\||fi|
\end{tabular}
\end{center}
%
Here one could write a message such as:
\begin{center}
|This is the part \childdocname{} of \childdocjob{}.|
\end{center}

%%%%%%%%%%%%%%%%%%%%%%%%%%%%%%%%%%%%%%%%%%%%%%%%%%%%%%%%%%%%%%%%%%%%%%%%%%%%%%%%
\subsection{Flags}
\label{sec:flags}

The package makes it easy to generate different versions
of the main or child documents.
To this end compilation flags can be defined
and assigned different default values.
They will be particularly useful in conjunction
with the forwarding mechanism described in \secref{sec:forward}.

For example, it may be useful to have a flag |\version|
which can be set to |draft| or |final|.
The document source will contain some conditional code
depending on the value of |\version|.
Suppose further, the flag should default to |final| for the main file
and to |draft| for child files
which is a natural assignment for editing the document.
This is achieved by placing the following code
in the preamble of the main document
(below the |\childdocmain| directive):
%
\begin{center}
\begin{tabular}{l}
|\ifchilddoc|\\
|\providecommand{\version}{draft}|\\
|\||else|\\
|\providecommand{\version}{final}|\\
|\||fi|
\end{tabular}
\end{center}
%
The definition by |\providecommand| makes sure
that previous definitions are not overwritten.
Further statements |\providecommand{\version}{...}|
can thus be added before the above code to override it.

For the main file, one might add a line
(between |\childdocmain| and the above block)
%
\begin{center}
|%\ifchilddoc\||else\providecommand{\version}{draft}\||fi|
\end{center}
%
which can be uncommented to produce a draft version.
Likewise one can add a line to the very top of a child file
(above the |\childdocof{|\textit{main}|}| directive)
%
\begin{center}
|%\providecommand{\version}{final}|
\end{center}
%
which can be uncommented to produce the final version of this child document.

%%%%%%%%%%%%%%%%%%%%%%%%%%%%%%%%%%%%%%%%%%%%%%%%%%%%%%%%%%%%%%%%%%%%%%%%%%%%%%%%
\subsection{Forwarding}
\label{sec:forward}

Different versions of the main or child documents
using compilation flags as described in \secref{sec:flags}
can be (permanently) stored in different files
for convenient compilation, viewing and distribution.
To this end, the package defines a command
to pass on compilation to a different file:

%%%%%%%%%%%%%%%%%%%%%%%%%%%%%%%%%%%%%%%%
\DescribeMacro{\childdocforward}
The command |\childdocforward| redirects processing to
another source file:
%
\begin{center}
\begin{tabular}{l}
|\input{childdoc.def}|\\
|\childdocforward[|\textit{main}|]{|\textit{dest}|}|\\
\end{tabular}
\end{center}
%
The argument \textit{dest} is the destination file
(without extension).
It should be the main file or one of the child files.
Note that further \textsf{childdoc} directives
such as |\childdocof| and |\childdocforward|
in the indicated file will be processed in this form.
The optional argument \textit{main}
passes on directly to the main file \textit{main}
while pretending to compile the child \textit{dest}.
This form behaves as if \textit{dest}
issues |\childdocof{|\textit{main}|}| right away,
and no further \textsf{childdoc} directives will be processed.

%%%%%%%%%%%%%%%%%%%%%%%%%%%%%%%%%%%%%%%%
\DescribeMacro{\...prefix}
In the alternative form |\childdocforwardprefix|,
%
\begin{center}
\begin{tabular}{l}
|\input{childdoc.def}|\\
|\childdocforwardprefix[|\textit{main}|]{|\textit{prefix}|}{|\textit{dest}|}|
\end{tabular}
\end{center}
%
the destination file is determined by a pattern
depending on the current file:
To make this work, the current file must be called
`{\textit{prefix}\hspace{0.2em}\textit{suffix}}'
with \textit{prefix} matching precisely the argument.
Processing is then passed on to the file
`{\textit{dest}\hspace{0.2em}\textit{suffix}}'.
Surely, the same effect is achieved by
directly specifying the
argument `{\textit{dest}\hspace{0.2em}\textit{suffix}}'
in the first form.
However, that requires to set up a different file
for each child. With the alternative form of the command
all these files can have exactly the same content
which simplifies setting them up and maintaining them.

For example, the following file |draft.tex|
with a compilation flag |\version| as described in \secref{sec:flags}
compiles the main document as a draft:
%
\begin{center}
\begin{tabular}{l}
|\def\version{draft}|\\
|\input{childdoc.def}|\\
|\childdocforward{|\textit{main}|}|
\end{tabular}
\end{center}
%
Likewise, the following files |final|\textit{nn}|.tex|
compile the final version of the child document
|child|\textit{nn}|.tex|:
%
\begin{center}
\begin{tabular}{l}
|\def\version{final}|\\
|\input{childdoc.def}|\\
|\childdocforwardprefix{final}{child}|
\end{tabular}
\end{center}
%

Note that when several versions of a main file and/or of each child file
are to be generated, it may be convenient to set up a |Makefile| or
shell script to automatise the process.

%%%%%%%%%%%%%%%%%%%%%%%%%%%%%%%%%%%%%%%%%%%%%%%%%%%%%%%%%%%%%%%%%%%%%%%%%%%%%%%%
\subsection{Command Line Processing}
\label{sec:commandline}

The effect of redirection files can also be achieved by invoking
the \LaTeX{} compiler with a more elaborate command line.
Most conveniently this should be done as part
of a shell script or a |Makefile|.

When using \textsf{childdoc} in the main file, the following
command lines effectively perform a redirection
(note that depending on the shell being used,
backslashes may have to be doubled: `|\|' $\to$ `|\\|'):
%
\begin{center}
|... -jobname "|\textit{target}|" |\\|"|[\textit{flags}]%
|\input{childdoc.def}\childdocforward[|\textit{main}|]{|\textit{dest}|}"|
\end{center}
%
Here \textit{target} is the name of the output file,
\textit{main} is the name of the main file
and \textit{dest} is the name of the main or child file to be processed
(all filenames without extensions).
The optional argument \textit{main} can be omitted
if \textit{main} matches \textit{dest}.
Optionally, compilation \textit{flags} can be defined via |\def| commands.
This command line makes the \TeX{} engine believe
it is compiling the file \textit{target}
whose content is specified as the latter parameter.
The provided code then forwards the processing to
\textit{main} or \textit{dest} as described in \secref{sec:forward}.

%%%%%%%%%%%%%%%%%%%%%%%%%%%%%%%%%%%%%%%%%%%%%%%%%%%%%%%%%%%%%%%%%%%%%%%%%%%%%%%%
\subsection{Include by Input}
\label{sec:input}

Including child documents by |\include| has some restrictions by design.
Most notably, the content of a child document always occupies
its own set of pages; pages cannot be shared between child documents.
Usually, this behaviour makes perfect sense
because each child document contain an essential part of the document.
However, in some situations it may be desirable to compose
a document from a collection of parts
without having mandatory page breaks between then.
For this case, the package
provides a mechanism to include parts
by |\input| which can also be processed individually.
However, by construction this mechanism
requires manual handling of the content to be output.

%%%%%%%%%%%%%%%%%%%%%%%%%%%%%%%%%%%%%%%%
\DescribeMacro{\ifchilddocmanual}
The main file should be prepared as usual, see \secref{sec:include}.
However, the document body must make a distinction
between processing of an individual part and of the main document, e.g.:
%
\begin{center}
\begin{tabular}{l}
|\ifchilddocmanual|\\
|\input{\childdocname}|\\
|\||else|\\
\textit{document body with }|\input{|\textit{part}|}|\\
|\||fi|
\end{tabular}
\end{center}
%
The conditional |\ifchilddocmanual| is true whenever
a part to be included by |\input| is being compiled,
and the name of the part is stored in |\childdocname|.

%%%%%%%%%%%%%%%%%%%%%%%%%%%%%%%%%%%%%%%%
\DescribeMacro{\childdocby}
Each part to be included by |\input| should start with:
%
\begin{center}
\begin{tabular}{l}
|\input{childdoc.def}|\\
|\childdocby{|\textit{main}|}|\\
\end{tabular}
\end{center}
%
The directive |\childdocby| is similar to |\childdocof|
described in \secref{sec:include},
but the subsequent selection of content must be done manually.
To that end, both |\ifchilddoc| and |\ifchilddocmanual|
will be true upon processing of a part,
and the name of the part is stored in |\childdocname|.
Note that |\jobname| will be set to the filename of the current part
so that each part receives an individual |.aux| file
that does not interfere with the |.aux| file(s) of the main document.
This behaviour can be altered by the alternative form
|\childdocby[*]{|\textit{main}|}| (with a non-empty optional argument)
which uses the |.aux| file of the main document
by setting |\jobname| to \textit{main}.

%%%%%%%%%%%%%%%%%%%%%%%%%%%%%%%%%%%%%%%%%%%%%%%%%%%%%%%%%%%%%%%%%%%%%%%%%%%%%%%%
\subsection{Driver Development}
\label{sec:driver}

The \textsf{childdoc} mechanism can also be use for the development
of definition files such as \LaTeX{} styles or classes.
This case differs from the above setup with multiple parts
included by |\include| in that no |\includeonly| should be invoked.
This can be achieved by starting the include file
(before |\ProvidesPackage|) with:
%
\begin{center}
\begin{tabular}{l}
|\input{childdoc.def}|\\
|\childdocforward{|\textit{main}|}|\\
\end{tabular}
\end{center}
%
or alternatively with:
%
\begin{center}
\begin{tabular}{l}
|\input{childdoc.def}|\\
|\childdocby{|\textit{main}|}|\\
\end{tabular}
\end{center}
%
Both forms have slightly different effects as described above.
The main file is prepared as usual, see \secref{sec:include}.

%%%%%%%%%%%%%%%%%%%%%%%%%%%%%%%%%%%%%%%%%%%%%%%%%%%%%%%%%%%%%%%%%%%%%%%%%%%%%%%%
\subsection{Legacy Detection}
\label{sec:detection}

The directive |\childdocmain| in the main file can detect
whether the complete document or merely a child is to be compiled
even without using the directive |\childdocof|.
This method is deprecated because it is less robust
and there is no compelling reason to use it;
it is merely provided for backward compatibility
and it may be removed in future versions.

If the detection mechanism is to be used,
it is mandatory to correctly specify
the filename of the main file as the argument of |\childdocmain|:
%
\begin{center}
\begin{tabular}{l}
|\input{childdoc.def}|\\
|\childdocmain{|\textit{main}|}|\\
\end{tabular}
\end{center}
%
If |\jobname| does not match the argument \textit{main} of |\childdocmain|,
it is assumed that |\jobname| points to the child file to be compiled.
When using |\childdocmain| with the main file specified as argument,
it suffices to start a child file
with just |\input{|\textit{main}|}|
without loading of the package and using |\childdocof|.
If instead all processing is done
with the appropriate \textsf{childdoc} directives,
the argument of \textit{main} of |\childdocmain| can be empty.

An alternative version of the command line processing described
in \secref{sec:commandline} using the detection mechanism reads:
%
\begin{center}
|... -jobname "|\textit{target}|" "|[\textit{flags}]%
[|\def\jobname{|\textit{dest}|}|]|\input{|\textit{main}|}"|
\end{center}

%%%%%%%%%%%%%%%%%%%%%%%%%%%%%%%%%%%%%%%%%%%%%%%%%%%%%%%%%%%%%%%%%%%%%%%%%%%%%%%%
\subsection{Manual Code}
\label{sec:manual}

In case one cannot be certain whether the definitions file |childdoc.def|
is installed on the target \TeX{} distribution
and one prefers not to ship it,
it is conceivable to paste a few relevant commands into the sources.

To that end, drop all statements |\input{childdoc.def}|
and perform the replacements as outlined below.
Instead of |\childdocmain{|\textit{main}|}| add the following code
to the top of the main file:
%
\begin{center}
\begin{tabular}{l}
|\||ifdefined\childdocname\endinput\||fi\newif\ifchilddoc|\\
|\edef\childdocname{\scantokens\expandafter{\jobname\noexpand}}|\\
|\def\childdocmain{|\textit{main}|}\||ifx\childdocmain\childdocname\||else|\\
|\childdoctrue\includeonly{\childdocname}\let\jobname\childdocmain\||fi|\\
\end{tabular}
\end{center}
%
Instead of |\childdocof{|\textit{main}|}| just include the main file
at the top of each child file:
%
\begin{center}
|\input{|\textit{main}|}|
\end{center}
%
A simple redirection |\childdocforward{|\textit{dest}|}| is achieved by:
%
\begin{center}
|\def\jobname{|\textit{dest}|}\input{\jobname}|
\end{center}
%
The redirection with prefix
|\childdocforwardprefix[|\textit{prefix}|]{|\textit{dest}|}|
is accomplished by:
%
\begin{center}
\begin{tabular}{l}
|{\edef\jobname{\scantokens\expandafter{\jobname\noexpand}}|\\
|\def\redirectjob |\textit{prefix}|#1~~~{\gdef\jobname{|\textit{dest}|#1}}|\\
|\expandafter\redirectjob\jobname~~~}\input{\jobname}|
\end{tabular}
\end{center}

In an alternative approach,
child documents can be compiled by a specific command line
without additional code or specific definitions:
%
\begin{center}
|... -jobname "|\textit{target}|" "|[\textit{flags}]%
|\includeonly{|\textit{dest}|}\input{|\textit{main}|}"|
\end{center}
%

%%%%%%%%%%%%%%%%%%%%%%%%%%%%%%%%%%%%%%%%%%%%%%%%%%%%%%%%%%%%%%%%%%%%%%%%%%%%%%%%
%%%%%%%%%%%%%%%%%%%%%%%%%%%%%%%%%%%%%%%%%%%%%%%%%%%%%%%%%%%%%%%%%%%%%%%%%%%%%%%%
\section{Information}

%%%%%%%%%%%%%%%%%%%%%%%%%%%%%%%%%%%%%%%%%%%%%%%%%%%%%%%%%%%%%%%%%%%%%%%%%%%%%%%%
\subsection{Copyright}

Copyright \copyright{} 2017--2018 Niklas Beisert

This work may be distributed and/or modified under the
conditions of the \LaTeX{} Project Public License, either version 1.3
of this license or (at your option) any later version.
The latest version of this license is in
  \url{http://www.latex-project.org/lppl.txt}
and version 1.3 or later is part of all distributions of \LaTeX{}
version 2005/12/01 or later.

This work has the LPPL maintenance status `maintained'.

The Current Maintainer of this work is Niklas Beisert.

This work consists of the files |README.txt|, |childdoc.ins| and |childdoc.dtx|
as well as the derived files |childdoc.def|, |cdocsamp.tex|
with |cdocsch1.tex|, |cdocsch2.tex|, |cdocspt3.tex|, |cdocspt4.tex|,
|cdocsdrf.tex|, |cdocsfn1.tex|, |cdocsfn2.tex|
as well as |childdoc.pdf|.

%%%%%%%%%%%%%%%%%%%%%%%%%%%%%%%%%%%%%%%%%%%%%%%%%%%%%%%%%%%%%%%%%%%%%%%%%%%%%%%%
\subsection{Files and Installation}

The package consists of the files:
%
\begin{center}
\begin{tabular}{ll}
    |README.txt|   & readme file \\
    |childdoc.ins| & installation file \\
    |childdoc.dtx| & source file \\
    |childdoc.def| & definition file \\
    |cdocsamp.tex| & sample main file \\
    |cdocsch1.tex| & sample include file \\
    |cdocsch2.tex| & sample include file \\
    |cdocspt3.tex| & sample part file \\
    |cdocspt4.tex| & sample part file \\
    |cdocsdrf.tex| & sample redirection file \\
    |cdocsfn1.tex| & sample redirection file \\
    |cdocsfn2.tex| & sample redirection file \\
    |childdoc.pdf| & manual
\end{tabular}
\end{center}
%
The distribution consists of the files
|README.txt|, |childdoc.ins| and |childdoc.dtx|.
%
\begin{itemize}
\item
Run (pdf)\LaTeX{} on |childdoc.dtx|
to compile the manual |childdoc.pdf| (this file).
\item
Run \LaTeX{} on |childdoc.ins| to create the definitions file |childdoc.def|
and the sample |cdocsamp.tex| with include files
|cdocsch1.tex|, |cdocsch2.tex|, |cdocspt3.tex|, |cdocspt4.tex|,
|cdocsdrf.tex|, |cdocsfn1.tex|, |cdocsfn2.tex|.
Then copy the file |childdoc.def| to an appropriate directory of your \LaTeX{}
distribution, e.g.\ \textit{texmf-root}|/tex/latex/childdoc|.
\end{itemize}

%%%%%%%%%%%%%%%%%%%%%%%%%%%%%%%%%%%%%%%%%%%%%%%%%%%%%%%%%%%%%%%%%%%%%%%%%%%%%%%%
\subsection{Related CTAN Packages}

There are several other packages which offer a similar functionality:
%
\begin{itemize}
\item
The packages
\href{http://ctan.org/pkg/docmute}{\textsf{docmute}},
\href{http://ctan.org/pkg/includex}{\textsf{includex}} and
\href{http://ctan.org/pkg/standalone}{\textsf{standalone}}
provide commands to include only the document body of
a child file thus allowing both files to be compiled individually.
\item
The packages \href{http://ctan.org/pkg/subdocs}{\textsf{subdocs}}
and \href{http://ctan.org/pkg/subfiles}{\textsf{subfiles}}
provide structures in which the main and child documents can be
encapsulated and allowing them to be compiled individually.
The inclusion mechanism is different from the conventional |\include|.
\item
The package \href{http://ctan.org/pkg/combine}{\textsf{combine}}
is an elaborate solution to combine several documents into one.
\end{itemize}
%
See also the CTAN topic \href{http://ctan.org/topic/subdocs}{\textsf{subdocs}}
for further related packages.
The present package differs from the above solutions in that
a document structure constructed with the conventional |\include| mechanism
just needs two extra commands at the top of every file
such that all constituent files can be compiled individually.

%%%%%%%%%%%%%%%%%%%%%%%%%%%%%%%%%%%%%%%%%%%%%%%%%%%%%%%%%%%%%%%%%%%%%%%%%%%%%%%%
%\subsection{Feature Suggestions}
%
%The following is a list of features which may be useful for future
%versions of this package:
%%
%\begin{itemize}
%\item
%\ldots
%\end{itemize}

%%%%%%%%%%%%%%%%%%%%%%%%%%%%%%%%%%%%%%%%%%%%%%%%%%%%%%%%%%%%%%%%%%%%%%%%%%%%%%%%
\subsection{Revision History}

%%%%%%%%%%%%%%%%%%%%%%%%%%%%%%%%%%%%%%%%
\paragraph{v2.0:} 2018/12/30

\begin{itemize}
\item
immediate forward processing
\item
added |\childdocby| mechanism
\item
manual restructured
\end{itemize}

%%%%%%%%%%%%%%%%%%%%%%%%%%%%%%%%%%%%%%%%
\paragraph{v1.6:} 2018/01/17

\begin{itemize}
\item
application for development of include files
\item
corrections to manual
\end{itemize}

%%%%%%%%%%%%%%%%%%%%%%%%%%%%%%%%%%%%%%%%
\paragraph{v1.5:} 2017/05/21

\begin{itemize}
\item
more complete structuring introduced
\item
|\childdocof| introduced
\item
|\childdoc| renamed to |\childdocmain|
\item
|\childredirect| renamed to |\childdocforward| and |\childdocforwardprefix|
and functionality expanded
\end{itemize}

%%%%%%%%%%%%%%%%%%%%%%%%%%%%%%%%%%%%%%%%
\paragraph{v1.0:} 2017/04/27

\begin{itemize}
\item
manual and install package
\item
first version published on CTAN
\end{itemize}

%%%%%%%%%%%%%%%%%%%%%%%%%%%%%%%%%%%%%%%%
\paragraph{v0.6:} 2017/04/26

\begin{itemize}
\item
redirection mechanism added
\end{itemize}

%%%%%%%%%%%%%%%%%%%%%%%%%%%%%%%%%%%%%%%%
\paragraph{v0.5:} 2017/04/26

\begin{itemize}
\item
functionality in definition file
\end{itemize}


%%%%%%%%%%%%%%%%%%%%%%%%%%%%%%%%%%%%%%%%%%%%%%%%%%%%%%%%%%%%%%%%%%%%%%%%%%%%%%%%
%%%%%%%%%%%%%%%%%%%%%%%%%%%%%%%%%%%%%%%%%%%%%%%%%%%%%%%%%%%%%%%%%%%%%%%%%%%%%%%%
%%%%%%%%%%%%%%%%%%%%%%%%%%%%%%%%%%%%%%%%%%%%%%%%%%%%%%%%%%%%%%%%%%%%%%%%%%%%%%%%
\appendix

\settowidth\MacroIndent{\rmfamily\scriptsize 000\ }

 \DocInput{childdoc.dtx}

\end{document}
%</driver>
% \fi
%
% %%%%%%%%%%%%%%%%%%%%%%%%%%%%%%%%%%%%%%%%%%%%%%%%%%%%%%%%%%%%%%%%%%%%%%%%%%%%%%
% %%%%%%%%%%%%%%%%%%%%%%%%%%%%%%%%%%%%%%%%%%%%%%%%%%%%%%%%%%%%%%%%%%%%%%%%%%%%%%
% \section{Sample}
%\iffalse
%<*samplemain>
%\fi
%
% The following presents a sample document
% with two chapters, two parts, a title page,
% a compile flag as well as three forwarding files to set the flag.
% It consists of eight |.tex| files:
% \begin{center}
% \begin{tabular}{ll}
% |cdocsamp.tex|&main file\\
% |cdocsch1.tex|&include file for chapter 1\\
% |cdocsch2.tex|&include file for chapter 2\\
% |cdocspt3.tex|&include file for part 3\\
% |cdocspt4.tex|&include file for part 4\\
% |cdocsdrf.tex|&forwarding file for main file in draft mode\\
% |cdocsfi1.tex|&forwarding file for final version of chapter 1\\
% |cdocsfi2.tex|&forwarding file for final version of chapter 2\\
% \end{tabular}
% \end{center}
% Each of the eight files can be compiled directly by the \LaTeX{} compiler.
%
% %%%%%%%%%%%%%%%%%%%%%%%%%%%%%%%%%%%%%%
% \paragraph{Main File.}
%
% The main file is called |cdocsamp.tex|.
%
% Load the \textsf{childdoc} definitions and
% declare the filename for the main document:
%    \begin{macrocode}
\input{childdoc.def}
\childdocmain{}
%    \end{macrocode}

% Optional override for |\version| flag:
%    \begin{macrocode}
%%\ifchilddoc\else\providecommand{\version}{draft}\fi
%    \end{macrocode}

% Define the default values for the |\version| flag
% (|final| for the main file and |draft| for childs):
%    \begin{macrocode}
\ifchilddoc
\providecommand{\version}{draft}
\else
\providecommand{\version}{final}
\fi
%    \end{macrocode}

% Load the standard document class:
%    \begin{macrocode}
\documentclass[12pt]{article}
%    \end{macrocode}

% Start the document body:
%    \begin{macrocode}
\begin{document}
%    \end{macrocode}

% Declare a title page.
% Print title, part of document being processed and version flag:
%    \begin{macrocode}
\addtocounter{page}{-1}
\begin{center}
{\LARGE\bfseries{}childdoc example\par}
\vspace{1cm}
\ifchilddoc
\ifchilddocmanual part\else chapter\fi:
`\childdocname' of `\childdocjob'\par
\else
main document: `\childdocjob'\par
\fi
version: \version\par
\end{center}
\newpage
%    \end{macrocode}

% Manually include selected file,
% otherwise process as usual:
%    \begin{macrocode}
\ifchilddocmanual
\section*{part `\childdocname'}
\input{\childdocname}
\else
%    \end{macrocode}

% Include the two chapters:
%    \begin{macrocode}
\include{cdocsch1}
\include{cdocsch2}
%    \end{macrocode}

% Include the two parts unless only chapters should be displayed:
%    \begin{macrocode}
\ifchilddoc\else
\section{part three}
\input{cdocspt3}
\section{part four}
\input{cdocspt4}
\fi
%    \end{macrocode}

% Process as usual until here:
%    \begin{macrocode}
\fi
%    \end{macrocode}

% End of document body:
%    \begin{macrocode}
\end{document}
%    \end{macrocode}
%\iffalse
%</samplemain>
%\fi
%
% %%%%%%%%%%%%%%%%%%%%%%%%%%%%%%%%%%%%%%
% \paragraph{Chapter Include Files.}
%
% The include files are called |cdocsch1.tex| and |cdocsch2.tex|.
%
%\iffalse
%<*samplechap1|samplechap2>
%\fi

% Optional override for |\version| flag:
%    \begin{macrocode}
%%\providecommand{\version}{final}
%    \end{macrocode}

% Include the main document:
%    \begin{macrocode}
\input{childdoc.def}
\childdocof{cdocsamp}
%    \end{macrocode}

%\iffalse
%</samplechap1|samplechap2>
%\fi
%
%\iffalse
%<*samplechap1>
%\fi
% Some text for chapter 1:
%    \begin{macrocode}
\section{one}
some text in chapter one
%    \end{macrocode}

%\iffalse
%</samplechap1>
%\fi
% Some text for chapter 2:
%\iffalse
%<*samplechap2>
%\fi
%    \begin{macrocode}
\section{two}
more text in chapter two
%    \end{macrocode}

%\iffalse
%</samplechap2>
%\fi
%
% %%%%%%%%%%%%%%%%%%%%%%%%%%%%%%%%%%%%%%
% \paragraph{Part Include Files.}
%
% The include files are called |cdocspt3.tex| and |cdocspt4.tex|.
%
%\iffalse
%<*samplepart3|samplepart4>
%\fi

% Optional override for |\version| flag:
%    \begin{macrocode}
%%\providecommand{\version}{final}
%    \end{macrocode}

% Include the main document:
%    \begin{macrocode}
\input{childdoc.def}
\childdocby{cdocsamp}
%    \end{macrocode}

%\iffalse
%</samplepart3|samplepart4>
%\fi
%
%\iffalse
%<*samplepart3>
%\fi
% Some text for part 3:
%    \begin{macrocode}
some text in part three
%    \end{macrocode}

%\iffalse
%</samplepart3>
%\fi
% Some text for part 4:
%\iffalse
%<*samplepart4>
%\fi
%    \begin{macrocode}
more text in part four
%    \end{macrocode}

%\iffalse
%</samplepart4>
%\fi
%
% %%%%%%%%%%%%%%%%%%%%%%%%%%%%%%%%%%%%%%
% \paragraph{Forwarding for a Complete Draft.}
%
% The following forwarding file |cdocsdrf.tex|
% compiles the main document in draft mode:
%\iffalse
%<*sampledraft>
%\fi
%    \begin{macrocode}
\def\version{draft}
\input{childdoc.def}
\childdocforward{cdocsamp}
%    \end{macrocode}

%\iffalse
%</sampledraft>
%\fi
%
% %%%%%%%%%%%%%%%%%%%%%%%%%%%%%%%%%%%%%%
% \paragraph{Forwarding for Final Version of the Chapters.}
%
% The following forwarding files |cdocsfn1.tex| and |cdocsfn2.tex|
% (with identical content)
% compile the final versions of the child documents
% |cdocsch1.tex| and |cdocsch2.tex|, respectively:
%\iffalse
%<*samplefinal>
%\fi
%    \begin{macrocode}
\def\version{final}
\input{childdoc.def}
\childdocforwardprefix[cdocsamp]{cdocsfn}{cdocsch}
%    \end{macrocode}

%\iffalse
%</samplefinal>
%\fi
%
% %%%%%%%%%%%%%%%%%%%%%%%%%%%%%%%%%%%%%%
% \paragraph{Command Line Processing.}
%
% The following three command lines generate the output files
% |cdocscld|, |cdocscl1| and |cdocscl2|
% which should be identical to
% |cdocsdrf|, |cdocsch1| and |cdocsfn2|, respectively:
% \begin{center}
% \begin{tabular}{l}
% |latex -jobname cdocscld \|\\
% |  "\def\version{draft}\input{childdoc.def}\childdocforward{cdocsamp}"|\\
% |latex -jobname cdocscl1 \|\\
% |  "\input{childdoc.def}\childdocforward[cdocsamp]{cdocsch1}"|\\
% |latex -jobname cdocscl2 \|\\
% |  "\def\version{final}\input{childdoc.def}\childdocforward{cdocsch2}"|
% \end{tabular}
% \end{center}
% Note that the trailing backslash on each first line
% merely continues the input to the second line
% (for convenient cut ant paste).
% Furthermore, the command |latex| can be replaced by any
% of its alternative versions such as |pdflatex|.
%
% %%%%%%%%%%%%%%%%%%%%%%%%%%%%%%%%%%%%%%%%%%%%%%%%%%%%%%%%%%%%%%%%%%%%%%%%%%%%%%
% %%%%%%%%%%%%%%%%%%%%%%%%%%%%%%%%%%%%%%%%%%%%%%%%%%%%%%%%%%%%%%%%%%%%%%%%%%%%%%
% \section{Implementation}
%\iffalse
%<*package>
%\fi
%
% This section describes the definitions file |childdoc.def|.

% The definitions cannot be loaded using |\usepackage| or |\RequirePackage|
% which has a mechanism to prevent loading a style file more than once.
% When loading the definitions by means of |\input|
% multiple instances have to be prevented manually:
%\iffalse
%This code needs to be before the `\ProvidesFile' directive
%which is defined at the beginning of this file.
%Therefore it is also placed there and commented out here.
%</package>
%<*discard>
%\fi
%    \begin{macrocode}
\ifdefined\childdocmain\endinput\fi
%    \end{macrocode}
%\iffalse
%</discard>
%<*package>
%\fi
%
% \macro{\ifchilddoc}
% \macro{\ifchilddocmanual}
% The conditional |\ifchilddoc| tells whether a
% child (true) or main (false) document is being compiled.
% The conditional |\ifchilddocmanual| tells whether
% the |\includeonly| mechanism is used (false) or
% the selection of child files must be performed manually (true).
% The definitions initialise to false:
%    \begin{macrocode}
\newif\ifchilddoc
\newif\ifchilddocmanual
%    \end{macrocode}

% \macro{\childdocname}
% \macro{\childdocjob}
% The macro |\childdocname| stores the name of the main document
% to be compiled. The macro |\childdocjob| stores the name of
% the document on which the \LaTeX{} compiler was originally invoked.
% The content of |\jobname| cannot be compared
% to filenames specified in the source due to different catcodes.
% The following code rescans |\jobname|, stores the result
% in |\childdocname| and saves a copy in |\childdocjob|:
%    \begin{macrocode}
\edef\childdocname{\scantokens\expandafter{\jobname\noexpand}}
\let\childdocjob\childdocname
%    \end{macrocode}

% \macro{\childdocdisable}
% The macro |\childdocdisable| prevents the main file
% from being processed more than once.
% At this stage, the main document command |\childdocmain|
% is assumed to be called once again where it should do nothing.
% Any subsequent call to it should prevent
% a secondary processing of the main document
% It overwrites the forwarding commands
% |\childdocof| and |\childdocforward|
% with empty macros to prevent further inclusions of the main document:
%    \begin{macrocode}
\newcommand{\childdocdisable}
{
  \renewcommand{\childdocmain}[1]{\renewcommand{\childdocmain}[1]{\endinput}}
  \renewcommand{\childdocof}[1]{}
  \renewcommand{\childdocby}[2][]{}
  \renewcommand{\childdocforward}[2][]{}
  \renewcommand{\childdocdisable}{}
}
%    \end{macrocode}

% \macro{\childdocmain}
% The macro |\childdocmain| is to be called at the top of the main file
% with nothing or the main filename (without extension) as argument.
% First, it breaks loops.
% If the argument is not empty and does not match |\childdocname|
% (which is set by the first inclusion of |childdoc.def|),
% |\ifchilddoc| is set to true, |\includeonly| is applied to the child file
% and |\jobname| is set to the main file
% (for proper handling of |.aux| files):
%    \begin{macrocode}
\newcommand{\childdocmain}[1]
{
  \childdocdisable\childdocmain{}
  \if?#1?\else
    \begingroup
      \def\childdoctmp{#1}
      \ifx\childdoctmp\childdocname
        \def\childdoctmp{}
      \else
        \def\childdoctmp
        {
          \childdoctrue
          \includeonly{\childdocname}
          \def\childdocjob{#1}
          \def\jobname{#1}
        }
      \fi
      \expandafter
    \endgroup
    \childdoctmp
  \fi
}
%    \end{macrocode}

% \macro{\childdocof}
% The command |\childdocof| redirects
% compilation to the main file |#1|.
%    \begin{macrocode}
\newcommand{\childdocof}[1]
{
  \childdocdisable
  \childdoctrue
  \includeonly{\childdocname}
  \def\jobname{#1}
  \def\childdocjob{#1}
  \input{#1}
}
%    \end{macrocode}

% \macro{\childdocby}
% The command |\childdocby| ....
%    \begin{macrocode}
\newcommand{\childdocby}[2][]
{
  \childdocdisable
  \childdoctrue
  \childdocmanualtrue
  \if?#1?\else
    \def\jobname{#2}
  \fi
  \def\childdocjob{#2}
  \input{#2}
  \endinput
}
%    \end{macrocode}

% \macro{\childdocforward}
% The command |\childdocforward| redirects
% compilation to the main file or
% (if the optional argument is given) a child file.
% Parameters are set as if the main file
% or a child file starting with |\childdocof| was compiled.
% Then compilation is handed over to the main file:
%    \begin{macrocode}
\newcommand{\childdocforward}[2][]
{
  \begingroup
    \if?#1?
      \def\childdoctmp
      {
        \def\childdocname{#2}
        \def\childdocjob{#2}
        \def\jobname{#2}
        \input{#2}
        \endinput
      }
    \else
      \def\childdoctmp
      {
        \childdocdisable
        \def\childdocname{#2}
        \childdoctrue
        \includeonly{#2}
        \def\childdocjob{#1}
        \def\jobname{#1}
        \input{#1}
        \endinput
      }
    \fi
    \expandafter
  \endgroup
  \childdoctmp
}
%    \end{macrocode}

% \macro{\childdocforwardprefix}
% The command |\childdocforwardprefix| redirects
% compilation to the main or a child file by means of a pattern.
% The prefix |#1| in the current filename is replaced by |#2|
% and the suffix of the current filename is kept
% (it is assumed that the filename does not contain the substring `|~~~|'
% which is used as a delimiter).
% Compilation is handed over to the new file by |\childdocforward|:
%    \begin{macrocode}
\newcommand{\childdocforwardprefix}[3][]
{
  \begingroup
    \def\childdocextract #2##1~~~{\def\childdoctmp{\childdocforward[#1]{#3##1}}}
    \expandafter\childdocextract\childdocname~~~
    \expandafter
  \endgroup
  \childdoctmp
}
%    \end{macrocode}

% \macro{\childdoc}
% The deprecated macro |\childdoc| is a legacy version of |\childdocmain|:
%    \begin{macrocode}
\newcommand{\childdoc}{\childdocmain}
%    \end{macrocode}

% \macro{\childdocredirect}
% The deprecated macro |\childdocredirect| is a legacy version
% of |\childdocforward| and |\childdocforwardprefix|:
%    \begin{macrocode}
\newcommand{\childdocredirect}[2][]
{
  \begingroup
    \if?#1?
      \def\childdoctmp{\childdocforward{#2}}
    \else
      \def\childdoctmp{\childdocforwardprefix{#1}{#2}}
    \fi
    \expandafter
  \endgroup
  \childdoctmp
}
%    \end{macrocode}

%\iffalse
%</package>
%\fi
%
\endinput

\childdocforward{cdocsamp}
%    \end{macrocode}

%\iffalse
%</sampledraft>
%\fi
%
% %%%%%%%%%%%%%%%%%%%%%%%%%%%%%%%%%%%%%%
% \paragraph{Forwarding for Final Version of the Chapters.}
%
% The following forwarding files |cdocsfn1.tex| and |cdocsfn2.tex|
% (with identical content)
% compile the final versions of the child documents
% |cdocsch1.tex| and |cdocsch2.tex|, respectively:
%\iffalse
%<*samplefinal>
%\fi
%    \begin{macrocode}
\def\version{final}
% \iffalse
%
% childdoc.dtx Copyright (C) 2017-2018 Niklas Beisert
%
% This work may be distributed and/or modified under the
% conditions of the LaTeX Project Public License, either version 1.3
% of this license or (at your option) any later version.
% The latest version of this license is in
%   http://www.latex-project.org/lppl.txt
% and version 1.3 or later is part of all distributions of LaTeX
% version 2005/12/01 or later.
%
% This work has the LPPL maintenance status `maintained'.
%
% The Current Maintainer of this work is Niklas Beisert.
%
% This work consists of the files childdoc.dtx and childdoc.ins
% and the derived files childdoc.def and cdocsamp.tex with
% cdocsch1.tex, cdocsch2.tex, cdocsdrf.tex, cdocsfn1.tex, cdocsfn2.tex.
%
%<package>\ifdefined\childdocmain\endinput\fi
%<package>\ProvidesFile{childdoc.def}[2018/12/30 v2.0 child document driver]
%<samplemain>\ProvidesFile{cdocsamp.tex}[2018/12/30 v2.0 sample for childdoc]
%<*driver>
%\ProvidesFile{childdoc.drv}[2018/12/30 v2.0 childdoc reference manual file]
\PassOptionsToClass{10pt,a4paper}{article}
\documentclass{ltxdoc}

\usepackage[margin=35mm]{geometry}
\usepackage{hyperref}
\usepackage{hyperxmp}
\usepackage[usenames]{color}

\hypersetup{colorlinks=true}
\hypersetup{pdfstartview=FitH}
\hypersetup{pdfpagemode=UseNone}
\hypersetup{pdfsource={}}
\hypersetup{pdflang={en-UK}}
\hypersetup{pdfcopyright={Copyright 2017-2018 Niklas Beisert.
  This work may be distributed and/or modified under the
  conditions of the LaTeX Project Public License, either version 1.3
  of this license or (at your option) any later version.}}
\hypersetup{pdflicenseurl={http://www.latex-project.org/lppl.txt}}
\hypersetup{pdfcontactaddress={ETH Zurich, ITP, HIT K,
  Wolfgang-Pauli-Strasse 27}}
\hypersetup{pdfcontactpostcode={8093}}
\hypersetup{pdfcontactcity={Zurich}}
\hypersetup{pdfcontactcountry={Switzerland}}
\hypersetup{pdfcontactemail={nbeisert@itp.phys.ethz.ch}}
\hypersetup{pdfcontacturl={http://people.phys.ethz.ch/\xmptilde nbeisert/}}

\newcommand{\secref}[1]{\hyperref[#1]{section \ref*{#1}}}

\parskip1ex
\parindent0pt
\let\olditemize\itemize
\def\itemize{\olditemize\parskip0pt}

\begin{document}

\title{The \textsf{childdoc} Package}
\hypersetup{pdftitle={The childdoc Package}}
\author{Niklas Beisert\\[2ex]
  Institut f\"ur Theoretische Physik\\
  Eidgen\"ossische Technische Hochschule Z\"urich\\
  Wolfgang-Pauli-Strasse 27, 8093 Z\"urich, Switzerland\\[1ex]
  \href{mailto:nbeisert@itp.phys.ethz.ch}
  {\texttt{nbeisert@itp.phys.ethz.ch}}}
\hypersetup{pdfauthor={Niklas Beisert}}
\hypersetup{pdfsubject={Manual for the LaTeX2e Package childdoc}}
\date{30 December 2018, \textsf{v2.0}}
\maketitle

\begin{abstract}\noindent
\textsf{childdoc} is a \LaTeXe{} package
that enables the direct compilation
of document sections included by |\include|
to individual files.
\end{abstract}

\begingroup
\parskip0ex
\tableofcontents
\endgroup

%%%%%%%%%%%%%%%%%%%%%%%%%%%%%%%%%%%%%%%%%%%%%%%%%%%%%%%%%%%%%%%%%%%%%%%%%%%%%%%%
%%%%%%%%%%%%%%%%%%%%%%%%%%%%%%%%%%%%%%%%%%%%%%%%%%%%%%%%%%%%%%%%%%%%%%%%%%%%%%%%
\section{Introduction}

\LaTeX{} provides a mechanism to structure a large document (such as a book)
into a main file and several child files (containing the chapters)
using the |\include| command.
This mechanism is beneficial for documents
which span hundreds of pages in order to
make the source file(s) more manageable.
Moreover, compilation can be restricted to
selected child files by means of the |\includeonly| command.
The latter feature can be used to reduce the compilation time while editing
(this was significantly more useful in the earlier days of \LaTeX{})
or to generate a smaller document which is easier to navigate.
Another application of |\includeonly| is to generate
documents consisting of selected parts of the complete document.

However, there are a few drawbacks of the plain |\include| mechanism:
\begin{itemize}
\item
The child files cannot be compiled on their own,
they can only be compiled via the main file.
A naive editing environment
(such as a text editor with an option
to have the current file processed by \LaTeX)
may require one to switch to the main file before compiling;
attempting to compile the child file produces errors.
\item
The main file must be modified (each time)
to adjust the |\includeonly| command
to the present needs. This easily leaves the main file in a messy state.
\item
The generated document will always carry the filename
of the main document. This is inconvenient if
several child files are to be compiled and
to be kept for distribution.
\end{itemize}

The present package provides a simple interface
to make child files individually compilable by \LaTeX{}.
Compiling a child file then has the same effect as compiling
the main file with an |\includeonly| command
to select the appropriate child.
Moreover the generated document will carry the name of the child
rather than the main file.
This resolves all three above issues.

This feature is meant to make the editing of books,
thesis documents and lecture notes somewhat more convenient.
However, the package can also be used efficiently for
composing a series of documents (such as exercise sheets)
which are typically distributed individually.
It then assists the author in generating the individual documents
(potentially in different versions)
as well as a document containing the collected series.
Another application is in developing style files
or other kinds of included material
where compilation of the style file could redirect
to a sample or test file.

%%%%%%%%%%%%%%%%%%%%%%%%%%%%%%%%%%%%%%%%%%%%%%%%%%%%%%%%%%%%%%%%%%%%%%%%%%%%%%%%
%%%%%%%%%%%%%%%%%%%%%%%%%%%%%%%%%%%%%%%%%%%%%%%%%%%%%%%%%%%%%%%%%%%%%%%%%%%%%%%%
\section{Usage}

First of all, the package \textsf{childdoc} is \emph{not} a standard
\LaTeXe{} |.sty| style file! Therefore it needs to be invoked in
a non-standard way.

%%%%%%%%%%%%%%%%%%%%%%%%%%%%%%%%%%%%%%%%%%%%%%%%%%%%%%%%%%%%%%%%%%%%%%%%%%%%%%%%
\subsection{Included Files}
\label{sec:include}

%%%%%%%%%%%%%%%%%%%%%%%%%%%%%%%%%%%%%%%%
\DescribeMacro{\childdocmain}
To use the package, add the commands
\begin{center}
\begin{tabular}{l}
|\input{childdoc.def}|\\
|\childdocmain{}|\\
\end{tabular}
\end{center}
at the very top of the main \LaTeX{} file,
in particular \emph{before} the |\documentclass| statement!
The argument of |\childdocmain| should be left empty
(but it must be present).

%%%%%%%%%%%%%%%%%%%%%%%%%%%%%%%%%%%%%%%%
\DescribeMacro{\childdocof}
Furthermore, add the commands
\begin{center}
\begin{tabular}{l}
|\input{childdoc.def}|\\
|\childdocof{|\textit{main}|}|\\
\end{tabular}
\end{center}
at the top of every child file \textit{child}
which is included by |\include{|\textit{child}|}|
from within the main file
(or at least for those files to be compiled individually).
The argument \textit{main} must be the filename of the main file.

There are a couple of
considerations in setting up the main and child documents:

%%%%%%%%%%%%%%%%%%%%%%%%%%%%%%%%%%%%%%%%
\paragraph{Restrictions.}

Please note the following restrictions:
\begin{itemize}
\item
|\childdocmain| must be called with one argument \textit{main}
to ensure compatibility with earlier version of the package.
It must either be empty (|\childdocmain{}|)
or precisely match the filename of the main file in which it is specified.
See \secref{sec:detection} for further information.
\item
The filename \textit{main} must be specified without the |.tex| extension.
\item
The filename \textit{main} is case sensitive
(even in case-insensitive file systems)
due to internal string comparison.
\item
The argument \textit{main} should be fully expanded, it cannot be a macro.
\item
Subdirectories and special characters should be avoided in filenames.
\item
The command |\childdocmain{|\textit{main}|}| must be followed by a whitespace.
It should not be followed immediately by another command
or by a comment mark `|%|'.
This is because the \TeX{} parser reads the token immediately following
the argument of |\childdocmain| and puts it
at the beginning of every child section;
however, a white\-space is ignored.
\end{itemize}

%%%%%%%%%%%%%%%%%%%%%%%%%%%%%%%%%%%%%%%%
\paragraph{Content of Main File.}

It is advisable to place all content in the child files included by |\include|.
Any output contained in the main file will appear in all child documents
unless suppressed manually;
it cannot be suppressed automatically by the |\includeonly| directive
and thus should normally be avoided.
A method to include some content in the main file
by means of conditional processing is described in \secref{sec:conditional}.

%%%%%%%%%%%%%%%%%%%%%%%%%%%%%%%%%%%%%%%%
\paragraph{Page Numbering.}

When only a part of the document is compiled,
the appropriate numbering of pages
(as well as other status parameters)
is determined from the |.aux| files.
The latter contain information from previous passes.
However this information needs to propagate through
all intermediate child documents.
Therefore the page numbering in child documents may well
be inconsistent until the complete document is compiled at least once.

A useful (if unconventional) way to always ensure a consistent
page numbering is to restart the numbering in each child document
and denote the pages by `\textit{child}|.|\textit{page}'
where \textit{child} represents the chapter/section number of the child file.
This can be achieved by the command
|\numberwithin{page}{|\textit{child}|}|
of the \textsf{amsmath} package
where \textit{child} can be |chapter| or |section|
depending on the chosen structuring.
Alternatively, one can modify the macro |\thepage| appropriately
and reset the counter |page| at the start of each child file.

%%%%%%%%%%%%%%%%%%%%%%%%%%%%%%%%%%%%%%%%%%%%%%%%%%%%%%%%%%%%%%%%%%%%%%%%%%%%%%%%
\subsection{Conditional Processing}
\label{sec:conditional}

The package provides a mechanism to compile different versions
of a document. To customise the versions further some conditional processing
can come in handy to distinguish which version is being compiled.
The package provides two macros to describe the compilation context:

%%%%%%%%%%%%%%%%%%%%%%%%%%%%%%%%%%%%%%%%
\DescribeMacro{\ifchilddoc}
The conditional |\ifchilddoc| distinguishes between the compilation of
child documents and the main document:
%
\begin{center}
|\ifchilddoc |\textit{child-code}| |[|\||else |\textit{main-code}]| \||fi|
\end{center}

%%%%%%%%%%%%%%%%%%%%%%%%%%%%%%%%%%%%%%%%
\DescribeMacro{\childdocname}
\DescribeMacro{\childdocjob}
The macro |\childdocname| contains the filename (without extension)
of the main or child file being processed.
Note that |\childdocjob| will always contain the name of the main file.

%%%%%%%%%%%%%%%%%%%%%%%%%%%%%%%%%%%%%%%%
\paragraph{Title Page.}

Conditional processing can be used to include a title or banner page
in the main document when proper precautions are taken.
Importantly, the code in the main file should ensure that the page counter
(as well as other status parameters which are stored in the |.aux| files)
takes the same value after the conditional processing.
Otherwise the page numbers may take divergent values
depending on which part is compiled.

For example, a title page could be declared by:
%
\begin{center}
\begin{tabular}{l}
|\ifchilddoc\||else|\\
|\addtocounter{page}{-1}|\\
\textit{code for title page}\\
|\newpage|\\
|\||fi|
\end{tabular}
\end{center}
%
A banner page for the child documents can be generated by:
%
\begin{center}
\begin{tabular}{l}
|\ifchilddoc|\\
|\addtocounter{page}{-1}|\\
\textit{code for banner page}\\
|\newpage|\\
|\||fi|
\end{tabular}
\end{center}
%
Here one could write a message such as:
\begin{center}
|This is the part \childdocname{} of \childdocjob{}.|
\end{center}

%%%%%%%%%%%%%%%%%%%%%%%%%%%%%%%%%%%%%%%%%%%%%%%%%%%%%%%%%%%%%%%%%%%%%%%%%%%%%%%%
\subsection{Flags}
\label{sec:flags}

The package makes it easy to generate different versions
of the main or child documents.
To this end compilation flags can be defined
and assigned different default values.
They will be particularly useful in conjunction
with the forwarding mechanism described in \secref{sec:forward}.

For example, it may be useful to have a flag |\version|
which can be set to |draft| or |final|.
The document source will contain some conditional code
depending on the value of |\version|.
Suppose further, the flag should default to |final| for the main file
and to |draft| for child files
which is a natural assignment for editing the document.
This is achieved by placing the following code
in the preamble of the main document
(below the |\childdocmain| directive):
%
\begin{center}
\begin{tabular}{l}
|\ifchilddoc|\\
|\providecommand{\version}{draft}|\\
|\||else|\\
|\providecommand{\version}{final}|\\
|\||fi|
\end{tabular}
\end{center}
%
The definition by |\providecommand| makes sure
that previous definitions are not overwritten.
Further statements |\providecommand{\version}{...}|
can thus be added before the above code to override it.

For the main file, one might add a line
(between |\childdocmain| and the above block)
%
\begin{center}
|%\ifchilddoc\||else\providecommand{\version}{draft}\||fi|
\end{center}
%
which can be uncommented to produce a draft version.
Likewise one can add a line to the very top of a child file
(above the |\childdocof{|\textit{main}|}| directive)
%
\begin{center}
|%\providecommand{\version}{final}|
\end{center}
%
which can be uncommented to produce the final version of this child document.

%%%%%%%%%%%%%%%%%%%%%%%%%%%%%%%%%%%%%%%%%%%%%%%%%%%%%%%%%%%%%%%%%%%%%%%%%%%%%%%%
\subsection{Forwarding}
\label{sec:forward}

Different versions of the main or child documents
using compilation flags as described in \secref{sec:flags}
can be (permanently) stored in different files
for convenient compilation, viewing and distribution.
To this end, the package defines a command
to pass on compilation to a different file:

%%%%%%%%%%%%%%%%%%%%%%%%%%%%%%%%%%%%%%%%
\DescribeMacro{\childdocforward}
The command |\childdocforward| redirects processing to
another source file:
%
\begin{center}
\begin{tabular}{l}
|\input{childdoc.def}|\\
|\childdocforward[|\textit{main}|]{|\textit{dest}|}|\\
\end{tabular}
\end{center}
%
The argument \textit{dest} is the destination file
(without extension).
It should be the main file or one of the child files.
Note that further \textsf{childdoc} directives
such as |\childdocof| and |\childdocforward|
in the indicated file will be processed in this form.
The optional argument \textit{main}
passes on directly to the main file \textit{main}
while pretending to compile the child \textit{dest}.
This form behaves as if \textit{dest}
issues |\childdocof{|\textit{main}|}| right away,
and no further \textsf{childdoc} directives will be processed.

%%%%%%%%%%%%%%%%%%%%%%%%%%%%%%%%%%%%%%%%
\DescribeMacro{\...prefix}
In the alternative form |\childdocforwardprefix|,
%
\begin{center}
\begin{tabular}{l}
|\input{childdoc.def}|\\
|\childdocforwardprefix[|\textit{main}|]{|\textit{prefix}|}{|\textit{dest}|}|
\end{tabular}
\end{center}
%
the destination file is determined by a pattern
depending on the current file:
To make this work, the current file must be called
`{\textit{prefix}\hspace{0.2em}\textit{suffix}}'
with \textit{prefix} matching precisely the argument.
Processing is then passed on to the file
`{\textit{dest}\hspace{0.2em}\textit{suffix}}'.
Surely, the same effect is achieved by
directly specifying the
argument `{\textit{dest}\hspace{0.2em}\textit{suffix}}'
in the first form.
However, that requires to set up a different file
for each child. With the alternative form of the command
all these files can have exactly the same content
which simplifies setting them up and maintaining them.

For example, the following file |draft.tex|
with a compilation flag |\version| as described in \secref{sec:flags}
compiles the main document as a draft:
%
\begin{center}
\begin{tabular}{l}
|\def\version{draft}|\\
|\input{childdoc.def}|\\
|\childdocforward{|\textit{main}|}|
\end{tabular}
\end{center}
%
Likewise, the following files |final|\textit{nn}|.tex|
compile the final version of the child document
|child|\textit{nn}|.tex|:
%
\begin{center}
\begin{tabular}{l}
|\def\version{final}|\\
|\input{childdoc.def}|\\
|\childdocforwardprefix{final}{child}|
\end{tabular}
\end{center}
%

Note that when several versions of a main file and/or of each child file
are to be generated, it may be convenient to set up a |Makefile| or
shell script to automatise the process.

%%%%%%%%%%%%%%%%%%%%%%%%%%%%%%%%%%%%%%%%%%%%%%%%%%%%%%%%%%%%%%%%%%%%%%%%%%%%%%%%
\subsection{Command Line Processing}
\label{sec:commandline}

The effect of redirection files can also be achieved by invoking
the \LaTeX{} compiler with a more elaborate command line.
Most conveniently this should be done as part
of a shell script or a |Makefile|.

When using \textsf{childdoc} in the main file, the following
command lines effectively perform a redirection
(note that depending on the shell being used,
backslashes may have to be doubled: `|\|' $\to$ `|\\|'):
%
\begin{center}
|... -jobname "|\textit{target}|" |\\|"|[\textit{flags}]%
|\input{childdoc.def}\childdocforward[|\textit{main}|]{|\textit{dest}|}"|
\end{center}
%
Here \textit{target} is the name of the output file,
\textit{main} is the name of the main file
and \textit{dest} is the name of the main or child file to be processed
(all filenames without extensions).
The optional argument \textit{main} can be omitted
if \textit{main} matches \textit{dest}.
Optionally, compilation \textit{flags} can be defined via |\def| commands.
This command line makes the \TeX{} engine believe
it is compiling the file \textit{target}
whose content is specified as the latter parameter.
The provided code then forwards the processing to
\textit{main} or \textit{dest} as described in \secref{sec:forward}.

%%%%%%%%%%%%%%%%%%%%%%%%%%%%%%%%%%%%%%%%%%%%%%%%%%%%%%%%%%%%%%%%%%%%%%%%%%%%%%%%
\subsection{Include by Input}
\label{sec:input}

Including child documents by |\include| has some restrictions by design.
Most notably, the content of a child document always occupies
its own set of pages; pages cannot be shared between child documents.
Usually, this behaviour makes perfect sense
because each child document contain an essential part of the document.
However, in some situations it may be desirable to compose
a document from a collection of parts
without having mandatory page breaks between then.
For this case, the package
provides a mechanism to include parts
by |\input| which can also be processed individually.
However, by construction this mechanism
requires manual handling of the content to be output.

%%%%%%%%%%%%%%%%%%%%%%%%%%%%%%%%%%%%%%%%
\DescribeMacro{\ifchilddocmanual}
The main file should be prepared as usual, see \secref{sec:include}.
However, the document body must make a distinction
between processing of an individual part and of the main document, e.g.:
%
\begin{center}
\begin{tabular}{l}
|\ifchilddocmanual|\\
|\input{\childdocname}|\\
|\||else|\\
\textit{document body with }|\input{|\textit{part}|}|\\
|\||fi|
\end{tabular}
\end{center}
%
The conditional |\ifchilddocmanual| is true whenever
a part to be included by |\input| is being compiled,
and the name of the part is stored in |\childdocname|.

%%%%%%%%%%%%%%%%%%%%%%%%%%%%%%%%%%%%%%%%
\DescribeMacro{\childdocby}
Each part to be included by |\input| should start with:
%
\begin{center}
\begin{tabular}{l}
|\input{childdoc.def}|\\
|\childdocby{|\textit{main}|}|\\
\end{tabular}
\end{center}
%
The directive |\childdocby| is similar to |\childdocof|
described in \secref{sec:include},
but the subsequent selection of content must be done manually.
To that end, both |\ifchilddoc| and |\ifchilddocmanual|
will be true upon processing of a part,
and the name of the part is stored in |\childdocname|.
Note that |\jobname| will be set to the filename of the current part
so that each part receives an individual |.aux| file
that does not interfere with the |.aux| file(s) of the main document.
This behaviour can be altered by the alternative form
|\childdocby[*]{|\textit{main}|}| (with a non-empty optional argument)
which uses the |.aux| file of the main document
by setting |\jobname| to \textit{main}.

%%%%%%%%%%%%%%%%%%%%%%%%%%%%%%%%%%%%%%%%%%%%%%%%%%%%%%%%%%%%%%%%%%%%%%%%%%%%%%%%
\subsection{Driver Development}
\label{sec:driver}

The \textsf{childdoc} mechanism can also be use for the development
of definition files such as \LaTeX{} styles or classes.
This case differs from the above setup with multiple parts
included by |\include| in that no |\includeonly| should be invoked.
This can be achieved by starting the include file
(before |\ProvidesPackage|) with:
%
\begin{center}
\begin{tabular}{l}
|\input{childdoc.def}|\\
|\childdocforward{|\textit{main}|}|\\
\end{tabular}
\end{center}
%
or alternatively with:
%
\begin{center}
\begin{tabular}{l}
|\input{childdoc.def}|\\
|\childdocby{|\textit{main}|}|\\
\end{tabular}
\end{center}
%
Both forms have slightly different effects as described above.
The main file is prepared as usual, see \secref{sec:include}.

%%%%%%%%%%%%%%%%%%%%%%%%%%%%%%%%%%%%%%%%%%%%%%%%%%%%%%%%%%%%%%%%%%%%%%%%%%%%%%%%
\subsection{Legacy Detection}
\label{sec:detection}

The directive |\childdocmain| in the main file can detect
whether the complete document or merely a child is to be compiled
even without using the directive |\childdocof|.
This method is deprecated because it is less robust
and there is no compelling reason to use it;
it is merely provided for backward compatibility
and it may be removed in future versions.

If the detection mechanism is to be used,
it is mandatory to correctly specify
the filename of the main file as the argument of |\childdocmain|:
%
\begin{center}
\begin{tabular}{l}
|\input{childdoc.def}|\\
|\childdocmain{|\textit{main}|}|\\
\end{tabular}
\end{center}
%
If |\jobname| does not match the argument \textit{main} of |\childdocmain|,
it is assumed that |\jobname| points to the child file to be compiled.
When using |\childdocmain| with the main file specified as argument,
it suffices to start a child file
with just |\input{|\textit{main}|}|
without loading of the package and using |\childdocof|.
If instead all processing is done
with the appropriate \textsf{childdoc} directives,
the argument of \textit{main} of |\childdocmain| can be empty.

An alternative version of the command line processing described
in \secref{sec:commandline} using the detection mechanism reads:
%
\begin{center}
|... -jobname "|\textit{target}|" "|[\textit{flags}]%
[|\def\jobname{|\textit{dest}|}|]|\input{|\textit{main}|}"|
\end{center}

%%%%%%%%%%%%%%%%%%%%%%%%%%%%%%%%%%%%%%%%%%%%%%%%%%%%%%%%%%%%%%%%%%%%%%%%%%%%%%%%
\subsection{Manual Code}
\label{sec:manual}

In case one cannot be certain whether the definitions file |childdoc.def|
is installed on the target \TeX{} distribution
and one prefers not to ship it,
it is conceivable to paste a few relevant commands into the sources.

To that end, drop all statements |\input{childdoc.def}|
and perform the replacements as outlined below.
Instead of |\childdocmain{|\textit{main}|}| add the following code
to the top of the main file:
%
\begin{center}
\begin{tabular}{l}
|\||ifdefined\childdocname\endinput\||fi\newif\ifchilddoc|\\
|\edef\childdocname{\scantokens\expandafter{\jobname\noexpand}}|\\
|\def\childdocmain{|\textit{main}|}\||ifx\childdocmain\childdocname\||else|\\
|\childdoctrue\includeonly{\childdocname}\let\jobname\childdocmain\||fi|\\
\end{tabular}
\end{center}
%
Instead of |\childdocof{|\textit{main}|}| just include the main file
at the top of each child file:
%
\begin{center}
|\input{|\textit{main}|}|
\end{center}
%
A simple redirection |\childdocforward{|\textit{dest}|}| is achieved by:
%
\begin{center}
|\def\jobname{|\textit{dest}|}\input{\jobname}|
\end{center}
%
The redirection with prefix
|\childdocforwardprefix[|\textit{prefix}|]{|\textit{dest}|}|
is accomplished by:
%
\begin{center}
\begin{tabular}{l}
|{\edef\jobname{\scantokens\expandafter{\jobname\noexpand}}|\\
|\def\redirectjob |\textit{prefix}|#1~~~{\gdef\jobname{|\textit{dest}|#1}}|\\
|\expandafter\redirectjob\jobname~~~}\input{\jobname}|
\end{tabular}
\end{center}

In an alternative approach,
child documents can be compiled by a specific command line
without additional code or specific definitions:
%
\begin{center}
|... -jobname "|\textit{target}|" "|[\textit{flags}]%
|\includeonly{|\textit{dest}|}\input{|\textit{main}|}"|
\end{center}
%

%%%%%%%%%%%%%%%%%%%%%%%%%%%%%%%%%%%%%%%%%%%%%%%%%%%%%%%%%%%%%%%%%%%%%%%%%%%%%%%%
%%%%%%%%%%%%%%%%%%%%%%%%%%%%%%%%%%%%%%%%%%%%%%%%%%%%%%%%%%%%%%%%%%%%%%%%%%%%%%%%
\section{Information}

%%%%%%%%%%%%%%%%%%%%%%%%%%%%%%%%%%%%%%%%%%%%%%%%%%%%%%%%%%%%%%%%%%%%%%%%%%%%%%%%
\subsection{Copyright}

Copyright \copyright{} 2017--2018 Niklas Beisert

This work may be distributed and/or modified under the
conditions of the \LaTeX{} Project Public License, either version 1.3
of this license or (at your option) any later version.
The latest version of this license is in
  \url{http://www.latex-project.org/lppl.txt}
and version 1.3 or later is part of all distributions of \LaTeX{}
version 2005/12/01 or later.

This work has the LPPL maintenance status `maintained'.

The Current Maintainer of this work is Niklas Beisert.

This work consists of the files |README.txt|, |childdoc.ins| and |childdoc.dtx|
as well as the derived files |childdoc.def|, |cdocsamp.tex|
with |cdocsch1.tex|, |cdocsch2.tex|, |cdocspt3.tex|, |cdocspt4.tex|,
|cdocsdrf.tex|, |cdocsfn1.tex|, |cdocsfn2.tex|
as well as |childdoc.pdf|.

%%%%%%%%%%%%%%%%%%%%%%%%%%%%%%%%%%%%%%%%%%%%%%%%%%%%%%%%%%%%%%%%%%%%%%%%%%%%%%%%
\subsection{Files and Installation}

The package consists of the files:
%
\begin{center}
\begin{tabular}{ll}
    |README.txt|   & readme file \\
    |childdoc.ins| & installation file \\
    |childdoc.dtx| & source file \\
    |childdoc.def| & definition file \\
    |cdocsamp.tex| & sample main file \\
    |cdocsch1.tex| & sample include file \\
    |cdocsch2.tex| & sample include file \\
    |cdocspt3.tex| & sample part file \\
    |cdocspt4.tex| & sample part file \\
    |cdocsdrf.tex| & sample redirection file \\
    |cdocsfn1.tex| & sample redirection file \\
    |cdocsfn2.tex| & sample redirection file \\
    |childdoc.pdf| & manual
\end{tabular}
\end{center}
%
The distribution consists of the files
|README.txt|, |childdoc.ins| and |childdoc.dtx|.
%
\begin{itemize}
\item
Run (pdf)\LaTeX{} on |childdoc.dtx|
to compile the manual |childdoc.pdf| (this file).
\item
Run \LaTeX{} on |childdoc.ins| to create the definitions file |childdoc.def|
and the sample |cdocsamp.tex| with include files
|cdocsch1.tex|, |cdocsch2.tex|, |cdocspt3.tex|, |cdocspt4.tex|,
|cdocsdrf.tex|, |cdocsfn1.tex|, |cdocsfn2.tex|.
Then copy the file |childdoc.def| to an appropriate directory of your \LaTeX{}
distribution, e.g.\ \textit{texmf-root}|/tex/latex/childdoc|.
\end{itemize}

%%%%%%%%%%%%%%%%%%%%%%%%%%%%%%%%%%%%%%%%%%%%%%%%%%%%%%%%%%%%%%%%%%%%%%%%%%%%%%%%
\subsection{Related CTAN Packages}

There are several other packages which offer a similar functionality:
%
\begin{itemize}
\item
The packages
\href{http://ctan.org/pkg/docmute}{\textsf{docmute}},
\href{http://ctan.org/pkg/includex}{\textsf{includex}} and
\href{http://ctan.org/pkg/standalone}{\textsf{standalone}}
provide commands to include only the document body of
a child file thus allowing both files to be compiled individually.
\item
The packages \href{http://ctan.org/pkg/subdocs}{\textsf{subdocs}}
and \href{http://ctan.org/pkg/subfiles}{\textsf{subfiles}}
provide structures in which the main and child documents can be
encapsulated and allowing them to be compiled individually.
The inclusion mechanism is different from the conventional |\include|.
\item
The package \href{http://ctan.org/pkg/combine}{\textsf{combine}}
is an elaborate solution to combine several documents into one.
\end{itemize}
%
See also the CTAN topic \href{http://ctan.org/topic/subdocs}{\textsf{subdocs}}
for further related packages.
The present package differs from the above solutions in that
a document structure constructed with the conventional |\include| mechanism
just needs two extra commands at the top of every file
such that all constituent files can be compiled individually.

%%%%%%%%%%%%%%%%%%%%%%%%%%%%%%%%%%%%%%%%%%%%%%%%%%%%%%%%%%%%%%%%%%%%%%%%%%%%%%%%
%\subsection{Feature Suggestions}
%
%The following is a list of features which may be useful for future
%versions of this package:
%%
%\begin{itemize}
%\item
%\ldots
%\end{itemize}

%%%%%%%%%%%%%%%%%%%%%%%%%%%%%%%%%%%%%%%%%%%%%%%%%%%%%%%%%%%%%%%%%%%%%%%%%%%%%%%%
\subsection{Revision History}

%%%%%%%%%%%%%%%%%%%%%%%%%%%%%%%%%%%%%%%%
\paragraph{v2.0:} 2018/12/30

\begin{itemize}
\item
immediate forward processing
\item
added |\childdocby| mechanism
\item
manual restructured
\end{itemize}

%%%%%%%%%%%%%%%%%%%%%%%%%%%%%%%%%%%%%%%%
\paragraph{v1.6:} 2018/01/17

\begin{itemize}
\item
application for development of include files
\item
corrections to manual
\end{itemize}

%%%%%%%%%%%%%%%%%%%%%%%%%%%%%%%%%%%%%%%%
\paragraph{v1.5:} 2017/05/21

\begin{itemize}
\item
more complete structuring introduced
\item
|\childdocof| introduced
\item
|\childdoc| renamed to |\childdocmain|
\item
|\childredirect| renamed to |\childdocforward| and |\childdocforwardprefix|
and functionality expanded
\end{itemize}

%%%%%%%%%%%%%%%%%%%%%%%%%%%%%%%%%%%%%%%%
\paragraph{v1.0:} 2017/04/27

\begin{itemize}
\item
manual and install package
\item
first version published on CTAN
\end{itemize}

%%%%%%%%%%%%%%%%%%%%%%%%%%%%%%%%%%%%%%%%
\paragraph{v0.6:} 2017/04/26

\begin{itemize}
\item
redirection mechanism added
\end{itemize}

%%%%%%%%%%%%%%%%%%%%%%%%%%%%%%%%%%%%%%%%
\paragraph{v0.5:} 2017/04/26

\begin{itemize}
\item
functionality in definition file
\end{itemize}


%%%%%%%%%%%%%%%%%%%%%%%%%%%%%%%%%%%%%%%%%%%%%%%%%%%%%%%%%%%%%%%%%%%%%%%%%%%%%%%%
%%%%%%%%%%%%%%%%%%%%%%%%%%%%%%%%%%%%%%%%%%%%%%%%%%%%%%%%%%%%%%%%%%%%%%%%%%%%%%%%
%%%%%%%%%%%%%%%%%%%%%%%%%%%%%%%%%%%%%%%%%%%%%%%%%%%%%%%%%%%%%%%%%%%%%%%%%%%%%%%%
\appendix

\settowidth\MacroIndent{\rmfamily\scriptsize 000\ }

 \DocInput{childdoc.dtx}

\end{document}
%</driver>
% \fi
%
% %%%%%%%%%%%%%%%%%%%%%%%%%%%%%%%%%%%%%%%%%%%%%%%%%%%%%%%%%%%%%%%%%%%%%%%%%%%%%%
% %%%%%%%%%%%%%%%%%%%%%%%%%%%%%%%%%%%%%%%%%%%%%%%%%%%%%%%%%%%%%%%%%%%%%%%%%%%%%%
% \section{Sample}
%\iffalse
%<*samplemain>
%\fi
%
% The following presents a sample document
% with two chapters, two parts, a title page,
% a compile flag as well as three forwarding files to set the flag.
% It consists of eight |.tex| files:
% \begin{center}
% \begin{tabular}{ll}
% |cdocsamp.tex|&main file\\
% |cdocsch1.tex|&include file for chapter 1\\
% |cdocsch2.tex|&include file for chapter 2\\
% |cdocspt3.tex|&include file for part 3\\
% |cdocspt4.tex|&include file for part 4\\
% |cdocsdrf.tex|&forwarding file for main file in draft mode\\
% |cdocsfi1.tex|&forwarding file for final version of chapter 1\\
% |cdocsfi2.tex|&forwarding file for final version of chapter 2\\
% \end{tabular}
% \end{center}
% Each of the eight files can be compiled directly by the \LaTeX{} compiler.
%
% %%%%%%%%%%%%%%%%%%%%%%%%%%%%%%%%%%%%%%
% \paragraph{Main File.}
%
% The main file is called |cdocsamp.tex|.
%
% Load the \textsf{childdoc} definitions and
% declare the filename for the main document:
%    \begin{macrocode}
\input{childdoc.def}
\childdocmain{}
%    \end{macrocode}

% Optional override for |\version| flag:
%    \begin{macrocode}
%%\ifchilddoc\else\providecommand{\version}{draft}\fi
%    \end{macrocode}

% Define the default values for the |\version| flag
% (|final| for the main file and |draft| for childs):
%    \begin{macrocode}
\ifchilddoc
\providecommand{\version}{draft}
\else
\providecommand{\version}{final}
\fi
%    \end{macrocode}

% Load the standard document class:
%    \begin{macrocode}
\documentclass[12pt]{article}
%    \end{macrocode}

% Start the document body:
%    \begin{macrocode}
\begin{document}
%    \end{macrocode}

% Declare a title page.
% Print title, part of document being processed and version flag:
%    \begin{macrocode}
\addtocounter{page}{-1}
\begin{center}
{\LARGE\bfseries{}childdoc example\par}
\vspace{1cm}
\ifchilddoc
\ifchilddocmanual part\else chapter\fi:
`\childdocname' of `\childdocjob'\par
\else
main document: `\childdocjob'\par
\fi
version: \version\par
\end{center}
\newpage
%    \end{macrocode}

% Manually include selected file,
% otherwise process as usual:
%    \begin{macrocode}
\ifchilddocmanual
\section*{part `\childdocname'}
\input{\childdocname}
\else
%    \end{macrocode}

% Include the two chapters:
%    \begin{macrocode}
\include{cdocsch1}
\include{cdocsch2}
%    \end{macrocode}

% Include the two parts unless only chapters should be displayed:
%    \begin{macrocode}
\ifchilddoc\else
\section{part three}
\input{cdocspt3}
\section{part four}
\input{cdocspt4}
\fi
%    \end{macrocode}

% Process as usual until here:
%    \begin{macrocode}
\fi
%    \end{macrocode}

% End of document body:
%    \begin{macrocode}
\end{document}
%    \end{macrocode}
%\iffalse
%</samplemain>
%\fi
%
% %%%%%%%%%%%%%%%%%%%%%%%%%%%%%%%%%%%%%%
% \paragraph{Chapter Include Files.}
%
% The include files are called |cdocsch1.tex| and |cdocsch2.tex|.
%
%\iffalse
%<*samplechap1|samplechap2>
%\fi

% Optional override for |\version| flag:
%    \begin{macrocode}
%%\providecommand{\version}{final}
%    \end{macrocode}

% Include the main document:
%    \begin{macrocode}
\input{childdoc.def}
\childdocof{cdocsamp}
%    \end{macrocode}

%\iffalse
%</samplechap1|samplechap2>
%\fi
%
%\iffalse
%<*samplechap1>
%\fi
% Some text for chapter 1:
%    \begin{macrocode}
\section{one}
some text in chapter one
%    \end{macrocode}

%\iffalse
%</samplechap1>
%\fi
% Some text for chapter 2:
%\iffalse
%<*samplechap2>
%\fi
%    \begin{macrocode}
\section{two}
more text in chapter two
%    \end{macrocode}

%\iffalse
%</samplechap2>
%\fi
%
% %%%%%%%%%%%%%%%%%%%%%%%%%%%%%%%%%%%%%%
% \paragraph{Part Include Files.}
%
% The include files are called |cdocspt3.tex| and |cdocspt4.tex|.
%
%\iffalse
%<*samplepart3|samplepart4>
%\fi

% Optional override for |\version| flag:
%    \begin{macrocode}
%%\providecommand{\version}{final}
%    \end{macrocode}

% Include the main document:
%    \begin{macrocode}
\input{childdoc.def}
\childdocby{cdocsamp}
%    \end{macrocode}

%\iffalse
%</samplepart3|samplepart4>
%\fi
%
%\iffalse
%<*samplepart3>
%\fi
% Some text for part 3:
%    \begin{macrocode}
some text in part three
%    \end{macrocode}

%\iffalse
%</samplepart3>
%\fi
% Some text for part 4:
%\iffalse
%<*samplepart4>
%\fi
%    \begin{macrocode}
more text in part four
%    \end{macrocode}

%\iffalse
%</samplepart4>
%\fi
%
% %%%%%%%%%%%%%%%%%%%%%%%%%%%%%%%%%%%%%%
% \paragraph{Forwarding for a Complete Draft.}
%
% The following forwarding file |cdocsdrf.tex|
% compiles the main document in draft mode:
%\iffalse
%<*sampledraft>
%\fi
%    \begin{macrocode}
\def\version{draft}
\input{childdoc.def}
\childdocforward{cdocsamp}
%    \end{macrocode}

%\iffalse
%</sampledraft>
%\fi
%
% %%%%%%%%%%%%%%%%%%%%%%%%%%%%%%%%%%%%%%
% \paragraph{Forwarding for Final Version of the Chapters.}
%
% The following forwarding files |cdocsfn1.tex| and |cdocsfn2.tex|
% (with identical content)
% compile the final versions of the child documents
% |cdocsch1.tex| and |cdocsch2.tex|, respectively:
%\iffalse
%<*samplefinal>
%\fi
%    \begin{macrocode}
\def\version{final}
\input{childdoc.def}
\childdocforwardprefix[cdocsamp]{cdocsfn}{cdocsch}
%    \end{macrocode}

%\iffalse
%</samplefinal>
%\fi
%
% %%%%%%%%%%%%%%%%%%%%%%%%%%%%%%%%%%%%%%
% \paragraph{Command Line Processing.}
%
% The following three command lines generate the output files
% |cdocscld|, |cdocscl1| and |cdocscl2|
% which should be identical to
% |cdocsdrf|, |cdocsch1| and |cdocsfn2|, respectively:
% \begin{center}
% \begin{tabular}{l}
% |latex -jobname cdocscld \|\\
% |  "\def\version{draft}\input{childdoc.def}\childdocforward{cdocsamp}"|\\
% |latex -jobname cdocscl1 \|\\
% |  "\input{childdoc.def}\childdocforward[cdocsamp]{cdocsch1}"|\\
% |latex -jobname cdocscl2 \|\\
% |  "\def\version{final}\input{childdoc.def}\childdocforward{cdocsch2}"|
% \end{tabular}
% \end{center}
% Note that the trailing backslash on each first line
% merely continues the input to the second line
% (for convenient cut ant paste).
% Furthermore, the command |latex| can be replaced by any
% of its alternative versions such as |pdflatex|.
%
% %%%%%%%%%%%%%%%%%%%%%%%%%%%%%%%%%%%%%%%%%%%%%%%%%%%%%%%%%%%%%%%%%%%%%%%%%%%%%%
% %%%%%%%%%%%%%%%%%%%%%%%%%%%%%%%%%%%%%%%%%%%%%%%%%%%%%%%%%%%%%%%%%%%%%%%%%%%%%%
% \section{Implementation}
%\iffalse
%<*package>
%\fi
%
% This section describes the definitions file |childdoc.def|.

% The definitions cannot be loaded using |\usepackage| or |\RequirePackage|
% which has a mechanism to prevent loading a style file more than once.
% When loading the definitions by means of |\input|
% multiple instances have to be prevented manually:
%\iffalse
%This code needs to be before the `\ProvidesFile' directive
%which is defined at the beginning of this file.
%Therefore it is also placed there and commented out here.
%</package>
%<*discard>
%\fi
%    \begin{macrocode}
\ifdefined\childdocmain\endinput\fi
%    \end{macrocode}
%\iffalse
%</discard>
%<*package>
%\fi
%
% \macro{\ifchilddoc}
% \macro{\ifchilddocmanual}
% The conditional |\ifchilddoc| tells whether a
% child (true) or main (false) document is being compiled.
% The conditional |\ifchilddocmanual| tells whether
% the |\includeonly| mechanism is used (false) or
% the selection of child files must be performed manually (true).
% The definitions initialise to false:
%    \begin{macrocode}
\newif\ifchilddoc
\newif\ifchilddocmanual
%    \end{macrocode}

% \macro{\childdocname}
% \macro{\childdocjob}
% The macro |\childdocname| stores the name of the main document
% to be compiled. The macro |\childdocjob| stores the name of
% the document on which the \LaTeX{} compiler was originally invoked.
% The content of |\jobname| cannot be compared
% to filenames specified in the source due to different catcodes.
% The following code rescans |\jobname|, stores the result
% in |\childdocname| and saves a copy in |\childdocjob|:
%    \begin{macrocode}
\edef\childdocname{\scantokens\expandafter{\jobname\noexpand}}
\let\childdocjob\childdocname
%    \end{macrocode}

% \macro{\childdocdisable}
% The macro |\childdocdisable| prevents the main file
% from being processed more than once.
% At this stage, the main document command |\childdocmain|
% is assumed to be called once again where it should do nothing.
% Any subsequent call to it should prevent
% a secondary processing of the main document
% It overwrites the forwarding commands
% |\childdocof| and |\childdocforward|
% with empty macros to prevent further inclusions of the main document:
%    \begin{macrocode}
\newcommand{\childdocdisable}
{
  \renewcommand{\childdocmain}[1]{\renewcommand{\childdocmain}[1]{\endinput}}
  \renewcommand{\childdocof}[1]{}
  \renewcommand{\childdocby}[2][]{}
  \renewcommand{\childdocforward}[2][]{}
  \renewcommand{\childdocdisable}{}
}
%    \end{macrocode}

% \macro{\childdocmain}
% The macro |\childdocmain| is to be called at the top of the main file
% with nothing or the main filename (without extension) as argument.
% First, it breaks loops.
% If the argument is not empty and does not match |\childdocname|
% (which is set by the first inclusion of |childdoc.def|),
% |\ifchilddoc| is set to true, |\includeonly| is applied to the child file
% and |\jobname| is set to the main file
% (for proper handling of |.aux| files):
%    \begin{macrocode}
\newcommand{\childdocmain}[1]
{
  \childdocdisable\childdocmain{}
  \if?#1?\else
    \begingroup
      \def\childdoctmp{#1}
      \ifx\childdoctmp\childdocname
        \def\childdoctmp{}
      \else
        \def\childdoctmp
        {
          \childdoctrue
          \includeonly{\childdocname}
          \def\childdocjob{#1}
          \def\jobname{#1}
        }
      \fi
      \expandafter
    \endgroup
    \childdoctmp
  \fi
}
%    \end{macrocode}

% \macro{\childdocof}
% The command |\childdocof| redirects
% compilation to the main file |#1|.
%    \begin{macrocode}
\newcommand{\childdocof}[1]
{
  \childdocdisable
  \childdoctrue
  \includeonly{\childdocname}
  \def\jobname{#1}
  \def\childdocjob{#1}
  \input{#1}
}
%    \end{macrocode}

% \macro{\childdocby}
% The command |\childdocby| ....
%    \begin{macrocode}
\newcommand{\childdocby}[2][]
{
  \childdocdisable
  \childdoctrue
  \childdocmanualtrue
  \if?#1?\else
    \def\jobname{#2}
  \fi
  \def\childdocjob{#2}
  \input{#2}
  \endinput
}
%    \end{macrocode}

% \macro{\childdocforward}
% The command |\childdocforward| redirects
% compilation to the main file or
% (if the optional argument is given) a child file.
% Parameters are set as if the main file
% or a child file starting with |\childdocof| was compiled.
% Then compilation is handed over to the main file:
%    \begin{macrocode}
\newcommand{\childdocforward}[2][]
{
  \begingroup
    \if?#1?
      \def\childdoctmp
      {
        \def\childdocname{#2}
        \def\childdocjob{#2}
        \def\jobname{#2}
        \input{#2}
        \endinput
      }
    \else
      \def\childdoctmp
      {
        \childdocdisable
        \def\childdocname{#2}
        \childdoctrue
        \includeonly{#2}
        \def\childdocjob{#1}
        \def\jobname{#1}
        \input{#1}
        \endinput
      }
    \fi
    \expandafter
  \endgroup
  \childdoctmp
}
%    \end{macrocode}

% \macro{\childdocforwardprefix}
% The command |\childdocforwardprefix| redirects
% compilation to the main or a child file by means of a pattern.
% The prefix |#1| in the current filename is replaced by |#2|
% and the suffix of the current filename is kept
% (it is assumed that the filename does not contain the substring `|~~~|'
% which is used as a delimiter).
% Compilation is handed over to the new file by |\childdocforward|:
%    \begin{macrocode}
\newcommand{\childdocforwardprefix}[3][]
{
  \begingroup
    \def\childdocextract #2##1~~~{\def\childdoctmp{\childdocforward[#1]{#3##1}}}
    \expandafter\childdocextract\childdocname~~~
    \expandafter
  \endgroup
  \childdoctmp
}
%    \end{macrocode}

% \macro{\childdoc}
% The deprecated macro |\childdoc| is a legacy version of |\childdocmain|:
%    \begin{macrocode}
\newcommand{\childdoc}{\childdocmain}
%    \end{macrocode}

% \macro{\childdocredirect}
% The deprecated macro |\childdocredirect| is a legacy version
% of |\childdocforward| and |\childdocforwardprefix|:
%    \begin{macrocode}
\newcommand{\childdocredirect}[2][]
{
  \begingroup
    \if?#1?
      \def\childdoctmp{\childdocforward{#2}}
    \else
      \def\childdoctmp{\childdocforwardprefix{#1}{#2}}
    \fi
    \expandafter
  \endgroup
  \childdoctmp
}
%    \end{macrocode}

%\iffalse
%</package>
%\fi
%
\endinput

\childdocforwardprefix[cdocsamp]{cdocsfn}{cdocsch}
%    \end{macrocode}

%\iffalse
%</samplefinal>
%\fi
%
% %%%%%%%%%%%%%%%%%%%%%%%%%%%%%%%%%%%%%%
% \paragraph{Command Line Processing.}
%
% The following three command lines generate the output files
% |cdocscld|, |cdocscl1| and |cdocscl2|
% which should be identical to
% |cdocsdrf|, |cdocsch1| and |cdocsfn2|, respectively:
% \begin{center}
% \begin{tabular}{l}
% |latex -jobname cdocscld \|\\
% |  "\def\version{draft}% \iffalse
%
% childdoc.dtx Copyright (C) 2017-2018 Niklas Beisert
%
% This work may be distributed and/or modified under the
% conditions of the LaTeX Project Public License, either version 1.3
% of this license or (at your option) any later version.
% The latest version of this license is in
%   http://www.latex-project.org/lppl.txt
% and version 1.3 or later is part of all distributions of LaTeX
% version 2005/12/01 or later.
%
% This work has the LPPL maintenance status `maintained'.
%
% The Current Maintainer of this work is Niklas Beisert.
%
% This work consists of the files childdoc.dtx and childdoc.ins
% and the derived files childdoc.def and cdocsamp.tex with
% cdocsch1.tex, cdocsch2.tex, cdocsdrf.tex, cdocsfn1.tex, cdocsfn2.tex.
%
%<package>\ifdefined\childdocmain\endinput\fi
%<package>\ProvidesFile{childdoc.def}[2018/12/30 v2.0 child document driver]
%<samplemain>\ProvidesFile{cdocsamp.tex}[2018/12/30 v2.0 sample for childdoc]
%<*driver>
%\ProvidesFile{childdoc.drv}[2018/12/30 v2.0 childdoc reference manual file]
\PassOptionsToClass{10pt,a4paper}{article}
\documentclass{ltxdoc}

\usepackage[margin=35mm]{geometry}
\usepackage{hyperref}
\usepackage{hyperxmp}
\usepackage[usenames]{color}

\hypersetup{colorlinks=true}
\hypersetup{pdfstartview=FitH}
\hypersetup{pdfpagemode=UseNone}
\hypersetup{pdfsource={}}
\hypersetup{pdflang={en-UK}}
\hypersetup{pdfcopyright={Copyright 2017-2018 Niklas Beisert.
  This work may be distributed and/or modified under the
  conditions of the LaTeX Project Public License, either version 1.3
  of this license or (at your option) any later version.}}
\hypersetup{pdflicenseurl={http://www.latex-project.org/lppl.txt}}
\hypersetup{pdfcontactaddress={ETH Zurich, ITP, HIT K,
  Wolfgang-Pauli-Strasse 27}}
\hypersetup{pdfcontactpostcode={8093}}
\hypersetup{pdfcontactcity={Zurich}}
\hypersetup{pdfcontactcountry={Switzerland}}
\hypersetup{pdfcontactemail={nbeisert@itp.phys.ethz.ch}}
\hypersetup{pdfcontacturl={http://people.phys.ethz.ch/\xmptilde nbeisert/}}

\newcommand{\secref}[1]{\hyperref[#1]{section \ref*{#1}}}

\parskip1ex
\parindent0pt
\let\olditemize\itemize
\def\itemize{\olditemize\parskip0pt}

\begin{document}

\title{The \textsf{childdoc} Package}
\hypersetup{pdftitle={The childdoc Package}}
\author{Niklas Beisert\\[2ex]
  Institut f\"ur Theoretische Physik\\
  Eidgen\"ossische Technische Hochschule Z\"urich\\
  Wolfgang-Pauli-Strasse 27, 8093 Z\"urich, Switzerland\\[1ex]
  \href{mailto:nbeisert@itp.phys.ethz.ch}
  {\texttt{nbeisert@itp.phys.ethz.ch}}}
\hypersetup{pdfauthor={Niklas Beisert}}
\hypersetup{pdfsubject={Manual for the LaTeX2e Package childdoc}}
\date{30 December 2018, \textsf{v2.0}}
\maketitle

\begin{abstract}\noindent
\textsf{childdoc} is a \LaTeXe{} package
that enables the direct compilation
of document sections included by |\include|
to individual files.
\end{abstract}

\begingroup
\parskip0ex
\tableofcontents
\endgroup

%%%%%%%%%%%%%%%%%%%%%%%%%%%%%%%%%%%%%%%%%%%%%%%%%%%%%%%%%%%%%%%%%%%%%%%%%%%%%%%%
%%%%%%%%%%%%%%%%%%%%%%%%%%%%%%%%%%%%%%%%%%%%%%%%%%%%%%%%%%%%%%%%%%%%%%%%%%%%%%%%
\section{Introduction}

\LaTeX{} provides a mechanism to structure a large document (such as a book)
into a main file and several child files (containing the chapters)
using the |\include| command.
This mechanism is beneficial for documents
which span hundreds of pages in order to
make the source file(s) more manageable.
Moreover, compilation can be restricted to
selected child files by means of the |\includeonly| command.
The latter feature can be used to reduce the compilation time while editing
(this was significantly more useful in the earlier days of \LaTeX{})
or to generate a smaller document which is easier to navigate.
Another application of |\includeonly| is to generate
documents consisting of selected parts of the complete document.

However, there are a few drawbacks of the plain |\include| mechanism:
\begin{itemize}
\item
The child files cannot be compiled on their own,
they can only be compiled via the main file.
A naive editing environment
(such as a text editor with an option
to have the current file processed by \LaTeX)
may require one to switch to the main file before compiling;
attempting to compile the child file produces errors.
\item
The main file must be modified (each time)
to adjust the |\includeonly| command
to the present needs. This easily leaves the main file in a messy state.
\item
The generated document will always carry the filename
of the main document. This is inconvenient if
several child files are to be compiled and
to be kept for distribution.
\end{itemize}

The present package provides a simple interface
to make child files individually compilable by \LaTeX{}.
Compiling a child file then has the same effect as compiling
the main file with an |\includeonly| command
to select the appropriate child.
Moreover the generated document will carry the name of the child
rather than the main file.
This resolves all three above issues.

This feature is meant to make the editing of books,
thesis documents and lecture notes somewhat more convenient.
However, the package can also be used efficiently for
composing a series of documents (such as exercise sheets)
which are typically distributed individually.
It then assists the author in generating the individual documents
(potentially in different versions)
as well as a document containing the collected series.
Another application is in developing style files
or other kinds of included material
where compilation of the style file could redirect
to a sample or test file.

%%%%%%%%%%%%%%%%%%%%%%%%%%%%%%%%%%%%%%%%%%%%%%%%%%%%%%%%%%%%%%%%%%%%%%%%%%%%%%%%
%%%%%%%%%%%%%%%%%%%%%%%%%%%%%%%%%%%%%%%%%%%%%%%%%%%%%%%%%%%%%%%%%%%%%%%%%%%%%%%%
\section{Usage}

First of all, the package \textsf{childdoc} is \emph{not} a standard
\LaTeXe{} |.sty| style file! Therefore it needs to be invoked in
a non-standard way.

%%%%%%%%%%%%%%%%%%%%%%%%%%%%%%%%%%%%%%%%%%%%%%%%%%%%%%%%%%%%%%%%%%%%%%%%%%%%%%%%
\subsection{Included Files}
\label{sec:include}

%%%%%%%%%%%%%%%%%%%%%%%%%%%%%%%%%%%%%%%%
\DescribeMacro{\childdocmain}
To use the package, add the commands
\begin{center}
\begin{tabular}{l}
|\input{childdoc.def}|\\
|\childdocmain{}|\\
\end{tabular}
\end{center}
at the very top of the main \LaTeX{} file,
in particular \emph{before} the |\documentclass| statement!
The argument of |\childdocmain| should be left empty
(but it must be present).

%%%%%%%%%%%%%%%%%%%%%%%%%%%%%%%%%%%%%%%%
\DescribeMacro{\childdocof}
Furthermore, add the commands
\begin{center}
\begin{tabular}{l}
|\input{childdoc.def}|\\
|\childdocof{|\textit{main}|}|\\
\end{tabular}
\end{center}
at the top of every child file \textit{child}
which is included by |\include{|\textit{child}|}|
from within the main file
(or at least for those files to be compiled individually).
The argument \textit{main} must be the filename of the main file.

There are a couple of
considerations in setting up the main and child documents:

%%%%%%%%%%%%%%%%%%%%%%%%%%%%%%%%%%%%%%%%
\paragraph{Restrictions.}

Please note the following restrictions:
\begin{itemize}
\item
|\childdocmain| must be called with one argument \textit{main}
to ensure compatibility with earlier version of the package.
It must either be empty (|\childdocmain{}|)
or precisely match the filename of the main file in which it is specified.
See \secref{sec:detection} for further information.
\item
The filename \textit{main} must be specified without the |.tex| extension.
\item
The filename \textit{main} is case sensitive
(even in case-insensitive file systems)
due to internal string comparison.
\item
The argument \textit{main} should be fully expanded, it cannot be a macro.
\item
Subdirectories and special characters should be avoided in filenames.
\item
The command |\childdocmain{|\textit{main}|}| must be followed by a whitespace.
It should not be followed immediately by another command
or by a comment mark `|%|'.
This is because the \TeX{} parser reads the token immediately following
the argument of |\childdocmain| and puts it
at the beginning of every child section;
however, a white\-space is ignored.
\end{itemize}

%%%%%%%%%%%%%%%%%%%%%%%%%%%%%%%%%%%%%%%%
\paragraph{Content of Main File.}

It is advisable to place all content in the child files included by |\include|.
Any output contained in the main file will appear in all child documents
unless suppressed manually;
it cannot be suppressed automatically by the |\includeonly| directive
and thus should normally be avoided.
A method to include some content in the main file
by means of conditional processing is described in \secref{sec:conditional}.

%%%%%%%%%%%%%%%%%%%%%%%%%%%%%%%%%%%%%%%%
\paragraph{Page Numbering.}

When only a part of the document is compiled,
the appropriate numbering of pages
(as well as other status parameters)
is determined from the |.aux| files.
The latter contain information from previous passes.
However this information needs to propagate through
all intermediate child documents.
Therefore the page numbering in child documents may well
be inconsistent until the complete document is compiled at least once.

A useful (if unconventional) way to always ensure a consistent
page numbering is to restart the numbering in each child document
and denote the pages by `\textit{child}|.|\textit{page}'
where \textit{child} represents the chapter/section number of the child file.
This can be achieved by the command
|\numberwithin{page}{|\textit{child}|}|
of the \textsf{amsmath} package
where \textit{child} can be |chapter| or |section|
depending on the chosen structuring.
Alternatively, one can modify the macro |\thepage| appropriately
and reset the counter |page| at the start of each child file.

%%%%%%%%%%%%%%%%%%%%%%%%%%%%%%%%%%%%%%%%%%%%%%%%%%%%%%%%%%%%%%%%%%%%%%%%%%%%%%%%
\subsection{Conditional Processing}
\label{sec:conditional}

The package provides a mechanism to compile different versions
of a document. To customise the versions further some conditional processing
can come in handy to distinguish which version is being compiled.
The package provides two macros to describe the compilation context:

%%%%%%%%%%%%%%%%%%%%%%%%%%%%%%%%%%%%%%%%
\DescribeMacro{\ifchilddoc}
The conditional |\ifchilddoc| distinguishes between the compilation of
child documents and the main document:
%
\begin{center}
|\ifchilddoc |\textit{child-code}| |[|\||else |\textit{main-code}]| \||fi|
\end{center}

%%%%%%%%%%%%%%%%%%%%%%%%%%%%%%%%%%%%%%%%
\DescribeMacro{\childdocname}
\DescribeMacro{\childdocjob}
The macro |\childdocname| contains the filename (without extension)
of the main or child file being processed.
Note that |\childdocjob| will always contain the name of the main file.

%%%%%%%%%%%%%%%%%%%%%%%%%%%%%%%%%%%%%%%%
\paragraph{Title Page.}

Conditional processing can be used to include a title or banner page
in the main document when proper precautions are taken.
Importantly, the code in the main file should ensure that the page counter
(as well as other status parameters which are stored in the |.aux| files)
takes the same value after the conditional processing.
Otherwise the page numbers may take divergent values
depending on which part is compiled.

For example, a title page could be declared by:
%
\begin{center}
\begin{tabular}{l}
|\ifchilddoc\||else|\\
|\addtocounter{page}{-1}|\\
\textit{code for title page}\\
|\newpage|\\
|\||fi|
\end{tabular}
\end{center}
%
A banner page for the child documents can be generated by:
%
\begin{center}
\begin{tabular}{l}
|\ifchilddoc|\\
|\addtocounter{page}{-1}|\\
\textit{code for banner page}\\
|\newpage|\\
|\||fi|
\end{tabular}
\end{center}
%
Here one could write a message such as:
\begin{center}
|This is the part \childdocname{} of \childdocjob{}.|
\end{center}

%%%%%%%%%%%%%%%%%%%%%%%%%%%%%%%%%%%%%%%%%%%%%%%%%%%%%%%%%%%%%%%%%%%%%%%%%%%%%%%%
\subsection{Flags}
\label{sec:flags}

The package makes it easy to generate different versions
of the main or child documents.
To this end compilation flags can be defined
and assigned different default values.
They will be particularly useful in conjunction
with the forwarding mechanism described in \secref{sec:forward}.

For example, it may be useful to have a flag |\version|
which can be set to |draft| or |final|.
The document source will contain some conditional code
depending on the value of |\version|.
Suppose further, the flag should default to |final| for the main file
and to |draft| for child files
which is a natural assignment for editing the document.
This is achieved by placing the following code
in the preamble of the main document
(below the |\childdocmain| directive):
%
\begin{center}
\begin{tabular}{l}
|\ifchilddoc|\\
|\providecommand{\version}{draft}|\\
|\||else|\\
|\providecommand{\version}{final}|\\
|\||fi|
\end{tabular}
\end{center}
%
The definition by |\providecommand| makes sure
that previous definitions are not overwritten.
Further statements |\providecommand{\version}{...}|
can thus be added before the above code to override it.

For the main file, one might add a line
(between |\childdocmain| and the above block)
%
\begin{center}
|%\ifchilddoc\||else\providecommand{\version}{draft}\||fi|
\end{center}
%
which can be uncommented to produce a draft version.
Likewise one can add a line to the very top of a child file
(above the |\childdocof{|\textit{main}|}| directive)
%
\begin{center}
|%\providecommand{\version}{final}|
\end{center}
%
which can be uncommented to produce the final version of this child document.

%%%%%%%%%%%%%%%%%%%%%%%%%%%%%%%%%%%%%%%%%%%%%%%%%%%%%%%%%%%%%%%%%%%%%%%%%%%%%%%%
\subsection{Forwarding}
\label{sec:forward}

Different versions of the main or child documents
using compilation flags as described in \secref{sec:flags}
can be (permanently) stored in different files
for convenient compilation, viewing and distribution.
To this end, the package defines a command
to pass on compilation to a different file:

%%%%%%%%%%%%%%%%%%%%%%%%%%%%%%%%%%%%%%%%
\DescribeMacro{\childdocforward}
The command |\childdocforward| redirects processing to
another source file:
%
\begin{center}
\begin{tabular}{l}
|\input{childdoc.def}|\\
|\childdocforward[|\textit{main}|]{|\textit{dest}|}|\\
\end{tabular}
\end{center}
%
The argument \textit{dest} is the destination file
(without extension).
It should be the main file or one of the child files.
Note that further \textsf{childdoc} directives
such as |\childdocof| and |\childdocforward|
in the indicated file will be processed in this form.
The optional argument \textit{main}
passes on directly to the main file \textit{main}
while pretending to compile the child \textit{dest}.
This form behaves as if \textit{dest}
issues |\childdocof{|\textit{main}|}| right away,
and no further \textsf{childdoc} directives will be processed.

%%%%%%%%%%%%%%%%%%%%%%%%%%%%%%%%%%%%%%%%
\DescribeMacro{\...prefix}
In the alternative form |\childdocforwardprefix|,
%
\begin{center}
\begin{tabular}{l}
|\input{childdoc.def}|\\
|\childdocforwardprefix[|\textit{main}|]{|\textit{prefix}|}{|\textit{dest}|}|
\end{tabular}
\end{center}
%
the destination file is determined by a pattern
depending on the current file:
To make this work, the current file must be called
`{\textit{prefix}\hspace{0.2em}\textit{suffix}}'
with \textit{prefix} matching precisely the argument.
Processing is then passed on to the file
`{\textit{dest}\hspace{0.2em}\textit{suffix}}'.
Surely, the same effect is achieved by
directly specifying the
argument `{\textit{dest}\hspace{0.2em}\textit{suffix}}'
in the first form.
However, that requires to set up a different file
for each child. With the alternative form of the command
all these files can have exactly the same content
which simplifies setting them up and maintaining them.

For example, the following file |draft.tex|
with a compilation flag |\version| as described in \secref{sec:flags}
compiles the main document as a draft:
%
\begin{center}
\begin{tabular}{l}
|\def\version{draft}|\\
|\input{childdoc.def}|\\
|\childdocforward{|\textit{main}|}|
\end{tabular}
\end{center}
%
Likewise, the following files |final|\textit{nn}|.tex|
compile the final version of the child document
|child|\textit{nn}|.tex|:
%
\begin{center}
\begin{tabular}{l}
|\def\version{final}|\\
|\input{childdoc.def}|\\
|\childdocforwardprefix{final}{child}|
\end{tabular}
\end{center}
%

Note that when several versions of a main file and/or of each child file
are to be generated, it may be convenient to set up a |Makefile| or
shell script to automatise the process.

%%%%%%%%%%%%%%%%%%%%%%%%%%%%%%%%%%%%%%%%%%%%%%%%%%%%%%%%%%%%%%%%%%%%%%%%%%%%%%%%
\subsection{Command Line Processing}
\label{sec:commandline}

The effect of redirection files can also be achieved by invoking
the \LaTeX{} compiler with a more elaborate command line.
Most conveniently this should be done as part
of a shell script or a |Makefile|.

When using \textsf{childdoc} in the main file, the following
command lines effectively perform a redirection
(note that depending on the shell being used,
backslashes may have to be doubled: `|\|' $\to$ `|\\|'):
%
\begin{center}
|... -jobname "|\textit{target}|" |\\|"|[\textit{flags}]%
|\input{childdoc.def}\childdocforward[|\textit{main}|]{|\textit{dest}|}"|
\end{center}
%
Here \textit{target} is the name of the output file,
\textit{main} is the name of the main file
and \textit{dest} is the name of the main or child file to be processed
(all filenames without extensions).
The optional argument \textit{main} can be omitted
if \textit{main} matches \textit{dest}.
Optionally, compilation \textit{flags} can be defined via |\def| commands.
This command line makes the \TeX{} engine believe
it is compiling the file \textit{target}
whose content is specified as the latter parameter.
The provided code then forwards the processing to
\textit{main} or \textit{dest} as described in \secref{sec:forward}.

%%%%%%%%%%%%%%%%%%%%%%%%%%%%%%%%%%%%%%%%%%%%%%%%%%%%%%%%%%%%%%%%%%%%%%%%%%%%%%%%
\subsection{Include by Input}
\label{sec:input}

Including child documents by |\include| has some restrictions by design.
Most notably, the content of a child document always occupies
its own set of pages; pages cannot be shared between child documents.
Usually, this behaviour makes perfect sense
because each child document contain an essential part of the document.
However, in some situations it may be desirable to compose
a document from a collection of parts
without having mandatory page breaks between then.
For this case, the package
provides a mechanism to include parts
by |\input| which can also be processed individually.
However, by construction this mechanism
requires manual handling of the content to be output.

%%%%%%%%%%%%%%%%%%%%%%%%%%%%%%%%%%%%%%%%
\DescribeMacro{\ifchilddocmanual}
The main file should be prepared as usual, see \secref{sec:include}.
However, the document body must make a distinction
between processing of an individual part and of the main document, e.g.:
%
\begin{center}
\begin{tabular}{l}
|\ifchilddocmanual|\\
|\input{\childdocname}|\\
|\||else|\\
\textit{document body with }|\input{|\textit{part}|}|\\
|\||fi|
\end{tabular}
\end{center}
%
The conditional |\ifchilddocmanual| is true whenever
a part to be included by |\input| is being compiled,
and the name of the part is stored in |\childdocname|.

%%%%%%%%%%%%%%%%%%%%%%%%%%%%%%%%%%%%%%%%
\DescribeMacro{\childdocby}
Each part to be included by |\input| should start with:
%
\begin{center}
\begin{tabular}{l}
|\input{childdoc.def}|\\
|\childdocby{|\textit{main}|}|\\
\end{tabular}
\end{center}
%
The directive |\childdocby| is similar to |\childdocof|
described in \secref{sec:include},
but the subsequent selection of content must be done manually.
To that end, both |\ifchilddoc| and |\ifchilddocmanual|
will be true upon processing of a part,
and the name of the part is stored in |\childdocname|.
Note that |\jobname| will be set to the filename of the current part
so that each part receives an individual |.aux| file
that does not interfere with the |.aux| file(s) of the main document.
This behaviour can be altered by the alternative form
|\childdocby[*]{|\textit{main}|}| (with a non-empty optional argument)
which uses the |.aux| file of the main document
by setting |\jobname| to \textit{main}.

%%%%%%%%%%%%%%%%%%%%%%%%%%%%%%%%%%%%%%%%%%%%%%%%%%%%%%%%%%%%%%%%%%%%%%%%%%%%%%%%
\subsection{Driver Development}
\label{sec:driver}

The \textsf{childdoc} mechanism can also be use for the development
of definition files such as \LaTeX{} styles or classes.
This case differs from the above setup with multiple parts
included by |\include| in that no |\includeonly| should be invoked.
This can be achieved by starting the include file
(before |\ProvidesPackage|) with:
%
\begin{center}
\begin{tabular}{l}
|\input{childdoc.def}|\\
|\childdocforward{|\textit{main}|}|\\
\end{tabular}
\end{center}
%
or alternatively with:
%
\begin{center}
\begin{tabular}{l}
|\input{childdoc.def}|\\
|\childdocby{|\textit{main}|}|\\
\end{tabular}
\end{center}
%
Both forms have slightly different effects as described above.
The main file is prepared as usual, see \secref{sec:include}.

%%%%%%%%%%%%%%%%%%%%%%%%%%%%%%%%%%%%%%%%%%%%%%%%%%%%%%%%%%%%%%%%%%%%%%%%%%%%%%%%
\subsection{Legacy Detection}
\label{sec:detection}

The directive |\childdocmain| in the main file can detect
whether the complete document or merely a child is to be compiled
even without using the directive |\childdocof|.
This method is deprecated because it is less robust
and there is no compelling reason to use it;
it is merely provided for backward compatibility
and it may be removed in future versions.

If the detection mechanism is to be used,
it is mandatory to correctly specify
the filename of the main file as the argument of |\childdocmain|:
%
\begin{center}
\begin{tabular}{l}
|\input{childdoc.def}|\\
|\childdocmain{|\textit{main}|}|\\
\end{tabular}
\end{center}
%
If |\jobname| does not match the argument \textit{main} of |\childdocmain|,
it is assumed that |\jobname| points to the child file to be compiled.
When using |\childdocmain| with the main file specified as argument,
it suffices to start a child file
with just |\input{|\textit{main}|}|
without loading of the package and using |\childdocof|.
If instead all processing is done
with the appropriate \textsf{childdoc} directives,
the argument of \textit{main} of |\childdocmain| can be empty.

An alternative version of the command line processing described
in \secref{sec:commandline} using the detection mechanism reads:
%
\begin{center}
|... -jobname "|\textit{target}|" "|[\textit{flags}]%
[|\def\jobname{|\textit{dest}|}|]|\input{|\textit{main}|}"|
\end{center}

%%%%%%%%%%%%%%%%%%%%%%%%%%%%%%%%%%%%%%%%%%%%%%%%%%%%%%%%%%%%%%%%%%%%%%%%%%%%%%%%
\subsection{Manual Code}
\label{sec:manual}

In case one cannot be certain whether the definitions file |childdoc.def|
is installed on the target \TeX{} distribution
and one prefers not to ship it,
it is conceivable to paste a few relevant commands into the sources.

To that end, drop all statements |\input{childdoc.def}|
and perform the replacements as outlined below.
Instead of |\childdocmain{|\textit{main}|}| add the following code
to the top of the main file:
%
\begin{center}
\begin{tabular}{l}
|\||ifdefined\childdocname\endinput\||fi\newif\ifchilddoc|\\
|\edef\childdocname{\scantokens\expandafter{\jobname\noexpand}}|\\
|\def\childdocmain{|\textit{main}|}\||ifx\childdocmain\childdocname\||else|\\
|\childdoctrue\includeonly{\childdocname}\let\jobname\childdocmain\||fi|\\
\end{tabular}
\end{center}
%
Instead of |\childdocof{|\textit{main}|}| just include the main file
at the top of each child file:
%
\begin{center}
|\input{|\textit{main}|}|
\end{center}
%
A simple redirection |\childdocforward{|\textit{dest}|}| is achieved by:
%
\begin{center}
|\def\jobname{|\textit{dest}|}\input{\jobname}|
\end{center}
%
The redirection with prefix
|\childdocforwardprefix[|\textit{prefix}|]{|\textit{dest}|}|
is accomplished by:
%
\begin{center}
\begin{tabular}{l}
|{\edef\jobname{\scantokens\expandafter{\jobname\noexpand}}|\\
|\def\redirectjob |\textit{prefix}|#1~~~{\gdef\jobname{|\textit{dest}|#1}}|\\
|\expandafter\redirectjob\jobname~~~}\input{\jobname}|
\end{tabular}
\end{center}

In an alternative approach,
child documents can be compiled by a specific command line
without additional code or specific definitions:
%
\begin{center}
|... -jobname "|\textit{target}|" "|[\textit{flags}]%
|\includeonly{|\textit{dest}|}\input{|\textit{main}|}"|
\end{center}
%

%%%%%%%%%%%%%%%%%%%%%%%%%%%%%%%%%%%%%%%%%%%%%%%%%%%%%%%%%%%%%%%%%%%%%%%%%%%%%%%%
%%%%%%%%%%%%%%%%%%%%%%%%%%%%%%%%%%%%%%%%%%%%%%%%%%%%%%%%%%%%%%%%%%%%%%%%%%%%%%%%
\section{Information}

%%%%%%%%%%%%%%%%%%%%%%%%%%%%%%%%%%%%%%%%%%%%%%%%%%%%%%%%%%%%%%%%%%%%%%%%%%%%%%%%
\subsection{Copyright}

Copyright \copyright{} 2017--2018 Niklas Beisert

This work may be distributed and/or modified under the
conditions of the \LaTeX{} Project Public License, either version 1.3
of this license or (at your option) any later version.
The latest version of this license is in
  \url{http://www.latex-project.org/lppl.txt}
and version 1.3 or later is part of all distributions of \LaTeX{}
version 2005/12/01 or later.

This work has the LPPL maintenance status `maintained'.

The Current Maintainer of this work is Niklas Beisert.

This work consists of the files |README.txt|, |childdoc.ins| and |childdoc.dtx|
as well as the derived files |childdoc.def|, |cdocsamp.tex|
with |cdocsch1.tex|, |cdocsch2.tex|, |cdocspt3.tex|, |cdocspt4.tex|,
|cdocsdrf.tex|, |cdocsfn1.tex|, |cdocsfn2.tex|
as well as |childdoc.pdf|.

%%%%%%%%%%%%%%%%%%%%%%%%%%%%%%%%%%%%%%%%%%%%%%%%%%%%%%%%%%%%%%%%%%%%%%%%%%%%%%%%
\subsection{Files and Installation}

The package consists of the files:
%
\begin{center}
\begin{tabular}{ll}
    |README.txt|   & readme file \\
    |childdoc.ins| & installation file \\
    |childdoc.dtx| & source file \\
    |childdoc.def| & definition file \\
    |cdocsamp.tex| & sample main file \\
    |cdocsch1.tex| & sample include file \\
    |cdocsch2.tex| & sample include file \\
    |cdocspt3.tex| & sample part file \\
    |cdocspt4.tex| & sample part file \\
    |cdocsdrf.tex| & sample redirection file \\
    |cdocsfn1.tex| & sample redirection file \\
    |cdocsfn2.tex| & sample redirection file \\
    |childdoc.pdf| & manual
\end{tabular}
\end{center}
%
The distribution consists of the files
|README.txt|, |childdoc.ins| and |childdoc.dtx|.
%
\begin{itemize}
\item
Run (pdf)\LaTeX{} on |childdoc.dtx|
to compile the manual |childdoc.pdf| (this file).
\item
Run \LaTeX{} on |childdoc.ins| to create the definitions file |childdoc.def|
and the sample |cdocsamp.tex| with include files
|cdocsch1.tex|, |cdocsch2.tex|, |cdocspt3.tex|, |cdocspt4.tex|,
|cdocsdrf.tex|, |cdocsfn1.tex|, |cdocsfn2.tex|.
Then copy the file |childdoc.def| to an appropriate directory of your \LaTeX{}
distribution, e.g.\ \textit{texmf-root}|/tex/latex/childdoc|.
\end{itemize}

%%%%%%%%%%%%%%%%%%%%%%%%%%%%%%%%%%%%%%%%%%%%%%%%%%%%%%%%%%%%%%%%%%%%%%%%%%%%%%%%
\subsection{Related CTAN Packages}

There are several other packages which offer a similar functionality:
%
\begin{itemize}
\item
The packages
\href{http://ctan.org/pkg/docmute}{\textsf{docmute}},
\href{http://ctan.org/pkg/includex}{\textsf{includex}} and
\href{http://ctan.org/pkg/standalone}{\textsf{standalone}}
provide commands to include only the document body of
a child file thus allowing both files to be compiled individually.
\item
The packages \href{http://ctan.org/pkg/subdocs}{\textsf{subdocs}}
and \href{http://ctan.org/pkg/subfiles}{\textsf{subfiles}}
provide structures in which the main and child documents can be
encapsulated and allowing them to be compiled individually.
The inclusion mechanism is different from the conventional |\include|.
\item
The package \href{http://ctan.org/pkg/combine}{\textsf{combine}}
is an elaborate solution to combine several documents into one.
\end{itemize}
%
See also the CTAN topic \href{http://ctan.org/topic/subdocs}{\textsf{subdocs}}
for further related packages.
The present package differs from the above solutions in that
a document structure constructed with the conventional |\include| mechanism
just needs two extra commands at the top of every file
such that all constituent files can be compiled individually.

%%%%%%%%%%%%%%%%%%%%%%%%%%%%%%%%%%%%%%%%%%%%%%%%%%%%%%%%%%%%%%%%%%%%%%%%%%%%%%%%
%\subsection{Feature Suggestions}
%
%The following is a list of features which may be useful for future
%versions of this package:
%%
%\begin{itemize}
%\item
%\ldots
%\end{itemize}

%%%%%%%%%%%%%%%%%%%%%%%%%%%%%%%%%%%%%%%%%%%%%%%%%%%%%%%%%%%%%%%%%%%%%%%%%%%%%%%%
\subsection{Revision History}

%%%%%%%%%%%%%%%%%%%%%%%%%%%%%%%%%%%%%%%%
\paragraph{v2.0:} 2018/12/30

\begin{itemize}
\item
immediate forward processing
\item
added |\childdocby| mechanism
\item
manual restructured
\end{itemize}

%%%%%%%%%%%%%%%%%%%%%%%%%%%%%%%%%%%%%%%%
\paragraph{v1.6:} 2018/01/17

\begin{itemize}
\item
application for development of include files
\item
corrections to manual
\end{itemize}

%%%%%%%%%%%%%%%%%%%%%%%%%%%%%%%%%%%%%%%%
\paragraph{v1.5:} 2017/05/21

\begin{itemize}
\item
more complete structuring introduced
\item
|\childdocof| introduced
\item
|\childdoc| renamed to |\childdocmain|
\item
|\childredirect| renamed to |\childdocforward| and |\childdocforwardprefix|
and functionality expanded
\end{itemize}

%%%%%%%%%%%%%%%%%%%%%%%%%%%%%%%%%%%%%%%%
\paragraph{v1.0:} 2017/04/27

\begin{itemize}
\item
manual and install package
\item
first version published on CTAN
\end{itemize}

%%%%%%%%%%%%%%%%%%%%%%%%%%%%%%%%%%%%%%%%
\paragraph{v0.6:} 2017/04/26

\begin{itemize}
\item
redirection mechanism added
\end{itemize}

%%%%%%%%%%%%%%%%%%%%%%%%%%%%%%%%%%%%%%%%
\paragraph{v0.5:} 2017/04/26

\begin{itemize}
\item
functionality in definition file
\end{itemize}


%%%%%%%%%%%%%%%%%%%%%%%%%%%%%%%%%%%%%%%%%%%%%%%%%%%%%%%%%%%%%%%%%%%%%%%%%%%%%%%%
%%%%%%%%%%%%%%%%%%%%%%%%%%%%%%%%%%%%%%%%%%%%%%%%%%%%%%%%%%%%%%%%%%%%%%%%%%%%%%%%
%%%%%%%%%%%%%%%%%%%%%%%%%%%%%%%%%%%%%%%%%%%%%%%%%%%%%%%%%%%%%%%%%%%%%%%%%%%%%%%%
\appendix

\settowidth\MacroIndent{\rmfamily\scriptsize 000\ }

 \DocInput{childdoc.dtx}

\end{document}
%</driver>
% \fi
%
% %%%%%%%%%%%%%%%%%%%%%%%%%%%%%%%%%%%%%%%%%%%%%%%%%%%%%%%%%%%%%%%%%%%%%%%%%%%%%%
% %%%%%%%%%%%%%%%%%%%%%%%%%%%%%%%%%%%%%%%%%%%%%%%%%%%%%%%%%%%%%%%%%%%%%%%%%%%%%%
% \section{Sample}
%\iffalse
%<*samplemain>
%\fi
%
% The following presents a sample document
% with two chapters, two parts, a title page,
% a compile flag as well as three forwarding files to set the flag.
% It consists of eight |.tex| files:
% \begin{center}
% \begin{tabular}{ll}
% |cdocsamp.tex|&main file\\
% |cdocsch1.tex|&include file for chapter 1\\
% |cdocsch2.tex|&include file for chapter 2\\
% |cdocspt3.tex|&include file for part 3\\
% |cdocspt4.tex|&include file for part 4\\
% |cdocsdrf.tex|&forwarding file for main file in draft mode\\
% |cdocsfi1.tex|&forwarding file for final version of chapter 1\\
% |cdocsfi2.tex|&forwarding file for final version of chapter 2\\
% \end{tabular}
% \end{center}
% Each of the eight files can be compiled directly by the \LaTeX{} compiler.
%
% %%%%%%%%%%%%%%%%%%%%%%%%%%%%%%%%%%%%%%
% \paragraph{Main File.}
%
% The main file is called |cdocsamp.tex|.
%
% Load the \textsf{childdoc} definitions and
% declare the filename for the main document:
%    \begin{macrocode}
\input{childdoc.def}
\childdocmain{}
%    \end{macrocode}

% Optional override for |\version| flag:
%    \begin{macrocode}
%%\ifchilddoc\else\providecommand{\version}{draft}\fi
%    \end{macrocode}

% Define the default values for the |\version| flag
% (|final| for the main file and |draft| for childs):
%    \begin{macrocode}
\ifchilddoc
\providecommand{\version}{draft}
\else
\providecommand{\version}{final}
\fi
%    \end{macrocode}

% Load the standard document class:
%    \begin{macrocode}
\documentclass[12pt]{article}
%    \end{macrocode}

% Start the document body:
%    \begin{macrocode}
\begin{document}
%    \end{macrocode}

% Declare a title page.
% Print title, part of document being processed and version flag:
%    \begin{macrocode}
\addtocounter{page}{-1}
\begin{center}
{\LARGE\bfseries{}childdoc example\par}
\vspace{1cm}
\ifchilddoc
\ifchilddocmanual part\else chapter\fi:
`\childdocname' of `\childdocjob'\par
\else
main document: `\childdocjob'\par
\fi
version: \version\par
\end{center}
\newpage
%    \end{macrocode}

% Manually include selected file,
% otherwise process as usual:
%    \begin{macrocode}
\ifchilddocmanual
\section*{part `\childdocname'}
\input{\childdocname}
\else
%    \end{macrocode}

% Include the two chapters:
%    \begin{macrocode}
\include{cdocsch1}
\include{cdocsch2}
%    \end{macrocode}

% Include the two parts unless only chapters should be displayed:
%    \begin{macrocode}
\ifchilddoc\else
\section{part three}
\input{cdocspt3}
\section{part four}
\input{cdocspt4}
\fi
%    \end{macrocode}

% Process as usual until here:
%    \begin{macrocode}
\fi
%    \end{macrocode}

% End of document body:
%    \begin{macrocode}
\end{document}
%    \end{macrocode}
%\iffalse
%</samplemain>
%\fi
%
% %%%%%%%%%%%%%%%%%%%%%%%%%%%%%%%%%%%%%%
% \paragraph{Chapter Include Files.}
%
% The include files are called |cdocsch1.tex| and |cdocsch2.tex|.
%
%\iffalse
%<*samplechap1|samplechap2>
%\fi

% Optional override for |\version| flag:
%    \begin{macrocode}
%%\providecommand{\version}{final}
%    \end{macrocode}

% Include the main document:
%    \begin{macrocode}
\input{childdoc.def}
\childdocof{cdocsamp}
%    \end{macrocode}

%\iffalse
%</samplechap1|samplechap2>
%\fi
%
%\iffalse
%<*samplechap1>
%\fi
% Some text for chapter 1:
%    \begin{macrocode}
\section{one}
some text in chapter one
%    \end{macrocode}

%\iffalse
%</samplechap1>
%\fi
% Some text for chapter 2:
%\iffalse
%<*samplechap2>
%\fi
%    \begin{macrocode}
\section{two}
more text in chapter two
%    \end{macrocode}

%\iffalse
%</samplechap2>
%\fi
%
% %%%%%%%%%%%%%%%%%%%%%%%%%%%%%%%%%%%%%%
% \paragraph{Part Include Files.}
%
% The include files are called |cdocspt3.tex| and |cdocspt4.tex|.
%
%\iffalse
%<*samplepart3|samplepart4>
%\fi

% Optional override for |\version| flag:
%    \begin{macrocode}
%%\providecommand{\version}{final}
%    \end{macrocode}

% Include the main document:
%    \begin{macrocode}
\input{childdoc.def}
\childdocby{cdocsamp}
%    \end{macrocode}

%\iffalse
%</samplepart3|samplepart4>
%\fi
%
%\iffalse
%<*samplepart3>
%\fi
% Some text for part 3:
%    \begin{macrocode}
some text in part three
%    \end{macrocode}

%\iffalse
%</samplepart3>
%\fi
% Some text for part 4:
%\iffalse
%<*samplepart4>
%\fi
%    \begin{macrocode}
more text in part four
%    \end{macrocode}

%\iffalse
%</samplepart4>
%\fi
%
% %%%%%%%%%%%%%%%%%%%%%%%%%%%%%%%%%%%%%%
% \paragraph{Forwarding for a Complete Draft.}
%
% The following forwarding file |cdocsdrf.tex|
% compiles the main document in draft mode:
%\iffalse
%<*sampledraft>
%\fi
%    \begin{macrocode}
\def\version{draft}
\input{childdoc.def}
\childdocforward{cdocsamp}
%    \end{macrocode}

%\iffalse
%</sampledraft>
%\fi
%
% %%%%%%%%%%%%%%%%%%%%%%%%%%%%%%%%%%%%%%
% \paragraph{Forwarding for Final Version of the Chapters.}
%
% The following forwarding files |cdocsfn1.tex| and |cdocsfn2.tex|
% (with identical content)
% compile the final versions of the child documents
% |cdocsch1.tex| and |cdocsch2.tex|, respectively:
%\iffalse
%<*samplefinal>
%\fi
%    \begin{macrocode}
\def\version{final}
\input{childdoc.def}
\childdocforwardprefix[cdocsamp]{cdocsfn}{cdocsch}
%    \end{macrocode}

%\iffalse
%</samplefinal>
%\fi
%
% %%%%%%%%%%%%%%%%%%%%%%%%%%%%%%%%%%%%%%
% \paragraph{Command Line Processing.}
%
% The following three command lines generate the output files
% |cdocscld|, |cdocscl1| and |cdocscl2|
% which should be identical to
% |cdocsdrf|, |cdocsch1| and |cdocsfn2|, respectively:
% \begin{center}
% \begin{tabular}{l}
% |latex -jobname cdocscld \|\\
% |  "\def\version{draft}\input{childdoc.def}\childdocforward{cdocsamp}"|\\
% |latex -jobname cdocscl1 \|\\
% |  "\input{childdoc.def}\childdocforward[cdocsamp]{cdocsch1}"|\\
% |latex -jobname cdocscl2 \|\\
% |  "\def\version{final}\input{childdoc.def}\childdocforward{cdocsch2}"|
% \end{tabular}
% \end{center}
% Note that the trailing backslash on each first line
% merely continues the input to the second line
% (for convenient cut ant paste).
% Furthermore, the command |latex| can be replaced by any
% of its alternative versions such as |pdflatex|.
%
% %%%%%%%%%%%%%%%%%%%%%%%%%%%%%%%%%%%%%%%%%%%%%%%%%%%%%%%%%%%%%%%%%%%%%%%%%%%%%%
% %%%%%%%%%%%%%%%%%%%%%%%%%%%%%%%%%%%%%%%%%%%%%%%%%%%%%%%%%%%%%%%%%%%%%%%%%%%%%%
% \section{Implementation}
%\iffalse
%<*package>
%\fi
%
% This section describes the definitions file |childdoc.def|.

% The definitions cannot be loaded using |\usepackage| or |\RequirePackage|
% which has a mechanism to prevent loading a style file more than once.
% When loading the definitions by means of |\input|
% multiple instances have to be prevented manually:
%\iffalse
%This code needs to be before the `\ProvidesFile' directive
%which is defined at the beginning of this file.
%Therefore it is also placed there and commented out here.
%</package>
%<*discard>
%\fi
%    \begin{macrocode}
\ifdefined\childdocmain\endinput\fi
%    \end{macrocode}
%\iffalse
%</discard>
%<*package>
%\fi
%
% \macro{\ifchilddoc}
% \macro{\ifchilddocmanual}
% The conditional |\ifchilddoc| tells whether a
% child (true) or main (false) document is being compiled.
% The conditional |\ifchilddocmanual| tells whether
% the |\includeonly| mechanism is used (false) or
% the selection of child files must be performed manually (true).
% The definitions initialise to false:
%    \begin{macrocode}
\newif\ifchilddoc
\newif\ifchilddocmanual
%    \end{macrocode}

% \macro{\childdocname}
% \macro{\childdocjob}
% The macro |\childdocname| stores the name of the main document
% to be compiled. The macro |\childdocjob| stores the name of
% the document on which the \LaTeX{} compiler was originally invoked.
% The content of |\jobname| cannot be compared
% to filenames specified in the source due to different catcodes.
% The following code rescans |\jobname|, stores the result
% in |\childdocname| and saves a copy in |\childdocjob|:
%    \begin{macrocode}
\edef\childdocname{\scantokens\expandafter{\jobname\noexpand}}
\let\childdocjob\childdocname
%    \end{macrocode}

% \macro{\childdocdisable}
% The macro |\childdocdisable| prevents the main file
% from being processed more than once.
% At this stage, the main document command |\childdocmain|
% is assumed to be called once again where it should do nothing.
% Any subsequent call to it should prevent
% a secondary processing of the main document
% It overwrites the forwarding commands
% |\childdocof| and |\childdocforward|
% with empty macros to prevent further inclusions of the main document:
%    \begin{macrocode}
\newcommand{\childdocdisable}
{
  \renewcommand{\childdocmain}[1]{\renewcommand{\childdocmain}[1]{\endinput}}
  \renewcommand{\childdocof}[1]{}
  \renewcommand{\childdocby}[2][]{}
  \renewcommand{\childdocforward}[2][]{}
  \renewcommand{\childdocdisable}{}
}
%    \end{macrocode}

% \macro{\childdocmain}
% The macro |\childdocmain| is to be called at the top of the main file
% with nothing or the main filename (without extension) as argument.
% First, it breaks loops.
% If the argument is not empty and does not match |\childdocname|
% (which is set by the first inclusion of |childdoc.def|),
% |\ifchilddoc| is set to true, |\includeonly| is applied to the child file
% and |\jobname| is set to the main file
% (for proper handling of |.aux| files):
%    \begin{macrocode}
\newcommand{\childdocmain}[1]
{
  \childdocdisable\childdocmain{}
  \if?#1?\else
    \begingroup
      \def\childdoctmp{#1}
      \ifx\childdoctmp\childdocname
        \def\childdoctmp{}
      \else
        \def\childdoctmp
        {
          \childdoctrue
          \includeonly{\childdocname}
          \def\childdocjob{#1}
          \def\jobname{#1}
        }
      \fi
      \expandafter
    \endgroup
    \childdoctmp
  \fi
}
%    \end{macrocode}

% \macro{\childdocof}
% The command |\childdocof| redirects
% compilation to the main file |#1|.
%    \begin{macrocode}
\newcommand{\childdocof}[1]
{
  \childdocdisable
  \childdoctrue
  \includeonly{\childdocname}
  \def\jobname{#1}
  \def\childdocjob{#1}
  \input{#1}
}
%    \end{macrocode}

% \macro{\childdocby}
% The command |\childdocby| ....
%    \begin{macrocode}
\newcommand{\childdocby}[2][]
{
  \childdocdisable
  \childdoctrue
  \childdocmanualtrue
  \if?#1?\else
    \def\jobname{#2}
  \fi
  \def\childdocjob{#2}
  \input{#2}
  \endinput
}
%    \end{macrocode}

% \macro{\childdocforward}
% The command |\childdocforward| redirects
% compilation to the main file or
% (if the optional argument is given) a child file.
% Parameters are set as if the main file
% or a child file starting with |\childdocof| was compiled.
% Then compilation is handed over to the main file:
%    \begin{macrocode}
\newcommand{\childdocforward}[2][]
{
  \begingroup
    \if?#1?
      \def\childdoctmp
      {
        \def\childdocname{#2}
        \def\childdocjob{#2}
        \def\jobname{#2}
        \input{#2}
        \endinput
      }
    \else
      \def\childdoctmp
      {
        \childdocdisable
        \def\childdocname{#2}
        \childdoctrue
        \includeonly{#2}
        \def\childdocjob{#1}
        \def\jobname{#1}
        \input{#1}
        \endinput
      }
    \fi
    \expandafter
  \endgroup
  \childdoctmp
}
%    \end{macrocode}

% \macro{\childdocforwardprefix}
% The command |\childdocforwardprefix| redirects
% compilation to the main or a child file by means of a pattern.
% The prefix |#1| in the current filename is replaced by |#2|
% and the suffix of the current filename is kept
% (it is assumed that the filename does not contain the substring `|~~~|'
% which is used as a delimiter).
% Compilation is handed over to the new file by |\childdocforward|:
%    \begin{macrocode}
\newcommand{\childdocforwardprefix}[3][]
{
  \begingroup
    \def\childdocextract #2##1~~~{\def\childdoctmp{\childdocforward[#1]{#3##1}}}
    \expandafter\childdocextract\childdocname~~~
    \expandafter
  \endgroup
  \childdoctmp
}
%    \end{macrocode}

% \macro{\childdoc}
% The deprecated macro |\childdoc| is a legacy version of |\childdocmain|:
%    \begin{macrocode}
\newcommand{\childdoc}{\childdocmain}
%    \end{macrocode}

% \macro{\childdocredirect}
% The deprecated macro |\childdocredirect| is a legacy version
% of |\childdocforward| and |\childdocforwardprefix|:
%    \begin{macrocode}
\newcommand{\childdocredirect}[2][]
{
  \begingroup
    \if?#1?
      \def\childdoctmp{\childdocforward{#2}}
    \else
      \def\childdoctmp{\childdocforwardprefix{#1}{#2}}
    \fi
    \expandafter
  \endgroup
  \childdoctmp
}
%    \end{macrocode}

%\iffalse
%</package>
%\fi
%
\endinput
\childdocforward{cdocsamp}"|\\
% |latex -jobname cdocscl1 \|\\
% |  "% \iffalse
%
% childdoc.dtx Copyright (C) 2017-2018 Niklas Beisert
%
% This work may be distributed and/or modified under the
% conditions of the LaTeX Project Public License, either version 1.3
% of this license or (at your option) any later version.
% The latest version of this license is in
%   http://www.latex-project.org/lppl.txt
% and version 1.3 or later is part of all distributions of LaTeX
% version 2005/12/01 or later.
%
% This work has the LPPL maintenance status `maintained'.
%
% The Current Maintainer of this work is Niklas Beisert.
%
% This work consists of the files childdoc.dtx and childdoc.ins
% and the derived files childdoc.def and cdocsamp.tex with
% cdocsch1.tex, cdocsch2.tex, cdocsdrf.tex, cdocsfn1.tex, cdocsfn2.tex.
%
%<package>\ifdefined\childdocmain\endinput\fi
%<package>\ProvidesFile{childdoc.def}[2018/12/30 v2.0 child document driver]
%<samplemain>\ProvidesFile{cdocsamp.tex}[2018/12/30 v2.0 sample for childdoc]
%<*driver>
%\ProvidesFile{childdoc.drv}[2018/12/30 v2.0 childdoc reference manual file]
\PassOptionsToClass{10pt,a4paper}{article}
\documentclass{ltxdoc}

\usepackage[margin=35mm]{geometry}
\usepackage{hyperref}
\usepackage{hyperxmp}
\usepackage[usenames]{color}

\hypersetup{colorlinks=true}
\hypersetup{pdfstartview=FitH}
\hypersetup{pdfpagemode=UseNone}
\hypersetup{pdfsource={}}
\hypersetup{pdflang={en-UK}}
\hypersetup{pdfcopyright={Copyright 2017-2018 Niklas Beisert.
  This work may be distributed and/or modified under the
  conditions of the LaTeX Project Public License, either version 1.3
  of this license or (at your option) any later version.}}
\hypersetup{pdflicenseurl={http://www.latex-project.org/lppl.txt}}
\hypersetup{pdfcontactaddress={ETH Zurich, ITP, HIT K,
  Wolfgang-Pauli-Strasse 27}}
\hypersetup{pdfcontactpostcode={8093}}
\hypersetup{pdfcontactcity={Zurich}}
\hypersetup{pdfcontactcountry={Switzerland}}
\hypersetup{pdfcontactemail={nbeisert@itp.phys.ethz.ch}}
\hypersetup{pdfcontacturl={http://people.phys.ethz.ch/\xmptilde nbeisert/}}

\newcommand{\secref}[1]{\hyperref[#1]{section \ref*{#1}}}

\parskip1ex
\parindent0pt
\let\olditemize\itemize
\def\itemize{\olditemize\parskip0pt}

\begin{document}

\title{The \textsf{childdoc} Package}
\hypersetup{pdftitle={The childdoc Package}}
\author{Niklas Beisert\\[2ex]
  Institut f\"ur Theoretische Physik\\
  Eidgen\"ossische Technische Hochschule Z\"urich\\
  Wolfgang-Pauli-Strasse 27, 8093 Z\"urich, Switzerland\\[1ex]
  \href{mailto:nbeisert@itp.phys.ethz.ch}
  {\texttt{nbeisert@itp.phys.ethz.ch}}}
\hypersetup{pdfauthor={Niklas Beisert}}
\hypersetup{pdfsubject={Manual for the LaTeX2e Package childdoc}}
\date{30 December 2018, \textsf{v2.0}}
\maketitle

\begin{abstract}\noindent
\textsf{childdoc} is a \LaTeXe{} package
that enables the direct compilation
of document sections included by |\include|
to individual files.
\end{abstract}

\begingroup
\parskip0ex
\tableofcontents
\endgroup

%%%%%%%%%%%%%%%%%%%%%%%%%%%%%%%%%%%%%%%%%%%%%%%%%%%%%%%%%%%%%%%%%%%%%%%%%%%%%%%%
%%%%%%%%%%%%%%%%%%%%%%%%%%%%%%%%%%%%%%%%%%%%%%%%%%%%%%%%%%%%%%%%%%%%%%%%%%%%%%%%
\section{Introduction}

\LaTeX{} provides a mechanism to structure a large document (such as a book)
into a main file and several child files (containing the chapters)
using the |\include| command.
This mechanism is beneficial for documents
which span hundreds of pages in order to
make the source file(s) more manageable.
Moreover, compilation can be restricted to
selected child files by means of the |\includeonly| command.
The latter feature can be used to reduce the compilation time while editing
(this was significantly more useful in the earlier days of \LaTeX{})
or to generate a smaller document which is easier to navigate.
Another application of |\includeonly| is to generate
documents consisting of selected parts of the complete document.

However, there are a few drawbacks of the plain |\include| mechanism:
\begin{itemize}
\item
The child files cannot be compiled on their own,
they can only be compiled via the main file.
A naive editing environment
(such as a text editor with an option
to have the current file processed by \LaTeX)
may require one to switch to the main file before compiling;
attempting to compile the child file produces errors.
\item
The main file must be modified (each time)
to adjust the |\includeonly| command
to the present needs. This easily leaves the main file in a messy state.
\item
The generated document will always carry the filename
of the main document. This is inconvenient if
several child files are to be compiled and
to be kept for distribution.
\end{itemize}

The present package provides a simple interface
to make child files individually compilable by \LaTeX{}.
Compiling a child file then has the same effect as compiling
the main file with an |\includeonly| command
to select the appropriate child.
Moreover the generated document will carry the name of the child
rather than the main file.
This resolves all three above issues.

This feature is meant to make the editing of books,
thesis documents and lecture notes somewhat more convenient.
However, the package can also be used efficiently for
composing a series of documents (such as exercise sheets)
which are typically distributed individually.
It then assists the author in generating the individual documents
(potentially in different versions)
as well as a document containing the collected series.
Another application is in developing style files
or other kinds of included material
where compilation of the style file could redirect
to a sample or test file.

%%%%%%%%%%%%%%%%%%%%%%%%%%%%%%%%%%%%%%%%%%%%%%%%%%%%%%%%%%%%%%%%%%%%%%%%%%%%%%%%
%%%%%%%%%%%%%%%%%%%%%%%%%%%%%%%%%%%%%%%%%%%%%%%%%%%%%%%%%%%%%%%%%%%%%%%%%%%%%%%%
\section{Usage}

First of all, the package \textsf{childdoc} is \emph{not} a standard
\LaTeXe{} |.sty| style file! Therefore it needs to be invoked in
a non-standard way.

%%%%%%%%%%%%%%%%%%%%%%%%%%%%%%%%%%%%%%%%%%%%%%%%%%%%%%%%%%%%%%%%%%%%%%%%%%%%%%%%
\subsection{Included Files}
\label{sec:include}

%%%%%%%%%%%%%%%%%%%%%%%%%%%%%%%%%%%%%%%%
\DescribeMacro{\childdocmain}
To use the package, add the commands
\begin{center}
\begin{tabular}{l}
|\input{childdoc.def}|\\
|\childdocmain{}|\\
\end{tabular}
\end{center}
at the very top of the main \LaTeX{} file,
in particular \emph{before} the |\documentclass| statement!
The argument of |\childdocmain| should be left empty
(but it must be present).

%%%%%%%%%%%%%%%%%%%%%%%%%%%%%%%%%%%%%%%%
\DescribeMacro{\childdocof}
Furthermore, add the commands
\begin{center}
\begin{tabular}{l}
|\input{childdoc.def}|\\
|\childdocof{|\textit{main}|}|\\
\end{tabular}
\end{center}
at the top of every child file \textit{child}
which is included by |\include{|\textit{child}|}|
from within the main file
(or at least for those files to be compiled individually).
The argument \textit{main} must be the filename of the main file.

There are a couple of
considerations in setting up the main and child documents:

%%%%%%%%%%%%%%%%%%%%%%%%%%%%%%%%%%%%%%%%
\paragraph{Restrictions.}

Please note the following restrictions:
\begin{itemize}
\item
|\childdocmain| must be called with one argument \textit{main}
to ensure compatibility with earlier version of the package.
It must either be empty (|\childdocmain{}|)
or precisely match the filename of the main file in which it is specified.
See \secref{sec:detection} for further information.
\item
The filename \textit{main} must be specified without the |.tex| extension.
\item
The filename \textit{main} is case sensitive
(even in case-insensitive file systems)
due to internal string comparison.
\item
The argument \textit{main} should be fully expanded, it cannot be a macro.
\item
Subdirectories and special characters should be avoided in filenames.
\item
The command |\childdocmain{|\textit{main}|}| must be followed by a whitespace.
It should not be followed immediately by another command
or by a comment mark `|%|'.
This is because the \TeX{} parser reads the token immediately following
the argument of |\childdocmain| and puts it
at the beginning of every child section;
however, a white\-space is ignored.
\end{itemize}

%%%%%%%%%%%%%%%%%%%%%%%%%%%%%%%%%%%%%%%%
\paragraph{Content of Main File.}

It is advisable to place all content in the child files included by |\include|.
Any output contained in the main file will appear in all child documents
unless suppressed manually;
it cannot be suppressed automatically by the |\includeonly| directive
and thus should normally be avoided.
A method to include some content in the main file
by means of conditional processing is described in \secref{sec:conditional}.

%%%%%%%%%%%%%%%%%%%%%%%%%%%%%%%%%%%%%%%%
\paragraph{Page Numbering.}

When only a part of the document is compiled,
the appropriate numbering of pages
(as well as other status parameters)
is determined from the |.aux| files.
The latter contain information from previous passes.
However this information needs to propagate through
all intermediate child documents.
Therefore the page numbering in child documents may well
be inconsistent until the complete document is compiled at least once.

A useful (if unconventional) way to always ensure a consistent
page numbering is to restart the numbering in each child document
and denote the pages by `\textit{child}|.|\textit{page}'
where \textit{child} represents the chapter/section number of the child file.
This can be achieved by the command
|\numberwithin{page}{|\textit{child}|}|
of the \textsf{amsmath} package
where \textit{child} can be |chapter| or |section|
depending on the chosen structuring.
Alternatively, one can modify the macro |\thepage| appropriately
and reset the counter |page| at the start of each child file.

%%%%%%%%%%%%%%%%%%%%%%%%%%%%%%%%%%%%%%%%%%%%%%%%%%%%%%%%%%%%%%%%%%%%%%%%%%%%%%%%
\subsection{Conditional Processing}
\label{sec:conditional}

The package provides a mechanism to compile different versions
of a document. To customise the versions further some conditional processing
can come in handy to distinguish which version is being compiled.
The package provides two macros to describe the compilation context:

%%%%%%%%%%%%%%%%%%%%%%%%%%%%%%%%%%%%%%%%
\DescribeMacro{\ifchilddoc}
The conditional |\ifchilddoc| distinguishes between the compilation of
child documents and the main document:
%
\begin{center}
|\ifchilddoc |\textit{child-code}| |[|\||else |\textit{main-code}]| \||fi|
\end{center}

%%%%%%%%%%%%%%%%%%%%%%%%%%%%%%%%%%%%%%%%
\DescribeMacro{\childdocname}
\DescribeMacro{\childdocjob}
The macro |\childdocname| contains the filename (without extension)
of the main or child file being processed.
Note that |\childdocjob| will always contain the name of the main file.

%%%%%%%%%%%%%%%%%%%%%%%%%%%%%%%%%%%%%%%%
\paragraph{Title Page.}

Conditional processing can be used to include a title or banner page
in the main document when proper precautions are taken.
Importantly, the code in the main file should ensure that the page counter
(as well as other status parameters which are stored in the |.aux| files)
takes the same value after the conditional processing.
Otherwise the page numbers may take divergent values
depending on which part is compiled.

For example, a title page could be declared by:
%
\begin{center}
\begin{tabular}{l}
|\ifchilddoc\||else|\\
|\addtocounter{page}{-1}|\\
\textit{code for title page}\\
|\newpage|\\
|\||fi|
\end{tabular}
\end{center}
%
A banner page for the child documents can be generated by:
%
\begin{center}
\begin{tabular}{l}
|\ifchilddoc|\\
|\addtocounter{page}{-1}|\\
\textit{code for banner page}\\
|\newpage|\\
|\||fi|
\end{tabular}
\end{center}
%
Here one could write a message such as:
\begin{center}
|This is the part \childdocname{} of \childdocjob{}.|
\end{center}

%%%%%%%%%%%%%%%%%%%%%%%%%%%%%%%%%%%%%%%%%%%%%%%%%%%%%%%%%%%%%%%%%%%%%%%%%%%%%%%%
\subsection{Flags}
\label{sec:flags}

The package makes it easy to generate different versions
of the main or child documents.
To this end compilation flags can be defined
and assigned different default values.
They will be particularly useful in conjunction
with the forwarding mechanism described in \secref{sec:forward}.

For example, it may be useful to have a flag |\version|
which can be set to |draft| or |final|.
The document source will contain some conditional code
depending on the value of |\version|.
Suppose further, the flag should default to |final| for the main file
and to |draft| for child files
which is a natural assignment for editing the document.
This is achieved by placing the following code
in the preamble of the main document
(below the |\childdocmain| directive):
%
\begin{center}
\begin{tabular}{l}
|\ifchilddoc|\\
|\providecommand{\version}{draft}|\\
|\||else|\\
|\providecommand{\version}{final}|\\
|\||fi|
\end{tabular}
\end{center}
%
The definition by |\providecommand| makes sure
that previous definitions are not overwritten.
Further statements |\providecommand{\version}{...}|
can thus be added before the above code to override it.

For the main file, one might add a line
(between |\childdocmain| and the above block)
%
\begin{center}
|%\ifchilddoc\||else\providecommand{\version}{draft}\||fi|
\end{center}
%
which can be uncommented to produce a draft version.
Likewise one can add a line to the very top of a child file
(above the |\childdocof{|\textit{main}|}| directive)
%
\begin{center}
|%\providecommand{\version}{final}|
\end{center}
%
which can be uncommented to produce the final version of this child document.

%%%%%%%%%%%%%%%%%%%%%%%%%%%%%%%%%%%%%%%%%%%%%%%%%%%%%%%%%%%%%%%%%%%%%%%%%%%%%%%%
\subsection{Forwarding}
\label{sec:forward}

Different versions of the main or child documents
using compilation flags as described in \secref{sec:flags}
can be (permanently) stored in different files
for convenient compilation, viewing and distribution.
To this end, the package defines a command
to pass on compilation to a different file:

%%%%%%%%%%%%%%%%%%%%%%%%%%%%%%%%%%%%%%%%
\DescribeMacro{\childdocforward}
The command |\childdocforward| redirects processing to
another source file:
%
\begin{center}
\begin{tabular}{l}
|\input{childdoc.def}|\\
|\childdocforward[|\textit{main}|]{|\textit{dest}|}|\\
\end{tabular}
\end{center}
%
The argument \textit{dest} is the destination file
(without extension).
It should be the main file or one of the child files.
Note that further \textsf{childdoc} directives
such as |\childdocof| and |\childdocforward|
in the indicated file will be processed in this form.
The optional argument \textit{main}
passes on directly to the main file \textit{main}
while pretending to compile the child \textit{dest}.
This form behaves as if \textit{dest}
issues |\childdocof{|\textit{main}|}| right away,
and no further \textsf{childdoc} directives will be processed.

%%%%%%%%%%%%%%%%%%%%%%%%%%%%%%%%%%%%%%%%
\DescribeMacro{\...prefix}
In the alternative form |\childdocforwardprefix|,
%
\begin{center}
\begin{tabular}{l}
|\input{childdoc.def}|\\
|\childdocforwardprefix[|\textit{main}|]{|\textit{prefix}|}{|\textit{dest}|}|
\end{tabular}
\end{center}
%
the destination file is determined by a pattern
depending on the current file:
To make this work, the current file must be called
`{\textit{prefix}\hspace{0.2em}\textit{suffix}}'
with \textit{prefix} matching precisely the argument.
Processing is then passed on to the file
`{\textit{dest}\hspace{0.2em}\textit{suffix}}'.
Surely, the same effect is achieved by
directly specifying the
argument `{\textit{dest}\hspace{0.2em}\textit{suffix}}'
in the first form.
However, that requires to set up a different file
for each child. With the alternative form of the command
all these files can have exactly the same content
which simplifies setting them up and maintaining them.

For example, the following file |draft.tex|
with a compilation flag |\version| as described in \secref{sec:flags}
compiles the main document as a draft:
%
\begin{center}
\begin{tabular}{l}
|\def\version{draft}|\\
|\input{childdoc.def}|\\
|\childdocforward{|\textit{main}|}|
\end{tabular}
\end{center}
%
Likewise, the following files |final|\textit{nn}|.tex|
compile the final version of the child document
|child|\textit{nn}|.tex|:
%
\begin{center}
\begin{tabular}{l}
|\def\version{final}|\\
|\input{childdoc.def}|\\
|\childdocforwardprefix{final}{child}|
\end{tabular}
\end{center}
%

Note that when several versions of a main file and/or of each child file
are to be generated, it may be convenient to set up a |Makefile| or
shell script to automatise the process.

%%%%%%%%%%%%%%%%%%%%%%%%%%%%%%%%%%%%%%%%%%%%%%%%%%%%%%%%%%%%%%%%%%%%%%%%%%%%%%%%
\subsection{Command Line Processing}
\label{sec:commandline}

The effect of redirection files can also be achieved by invoking
the \LaTeX{} compiler with a more elaborate command line.
Most conveniently this should be done as part
of a shell script or a |Makefile|.

When using \textsf{childdoc} in the main file, the following
command lines effectively perform a redirection
(note that depending on the shell being used,
backslashes may have to be doubled: `|\|' $\to$ `|\\|'):
%
\begin{center}
|... -jobname "|\textit{target}|" |\\|"|[\textit{flags}]%
|\input{childdoc.def}\childdocforward[|\textit{main}|]{|\textit{dest}|}"|
\end{center}
%
Here \textit{target} is the name of the output file,
\textit{main} is the name of the main file
and \textit{dest} is the name of the main or child file to be processed
(all filenames without extensions).
The optional argument \textit{main} can be omitted
if \textit{main} matches \textit{dest}.
Optionally, compilation \textit{flags} can be defined via |\def| commands.
This command line makes the \TeX{} engine believe
it is compiling the file \textit{target}
whose content is specified as the latter parameter.
The provided code then forwards the processing to
\textit{main} or \textit{dest} as described in \secref{sec:forward}.

%%%%%%%%%%%%%%%%%%%%%%%%%%%%%%%%%%%%%%%%%%%%%%%%%%%%%%%%%%%%%%%%%%%%%%%%%%%%%%%%
\subsection{Include by Input}
\label{sec:input}

Including child documents by |\include| has some restrictions by design.
Most notably, the content of a child document always occupies
its own set of pages; pages cannot be shared between child documents.
Usually, this behaviour makes perfect sense
because each child document contain an essential part of the document.
However, in some situations it may be desirable to compose
a document from a collection of parts
without having mandatory page breaks between then.
For this case, the package
provides a mechanism to include parts
by |\input| which can also be processed individually.
However, by construction this mechanism
requires manual handling of the content to be output.

%%%%%%%%%%%%%%%%%%%%%%%%%%%%%%%%%%%%%%%%
\DescribeMacro{\ifchilddocmanual}
The main file should be prepared as usual, see \secref{sec:include}.
However, the document body must make a distinction
between processing of an individual part and of the main document, e.g.:
%
\begin{center}
\begin{tabular}{l}
|\ifchilddocmanual|\\
|\input{\childdocname}|\\
|\||else|\\
\textit{document body with }|\input{|\textit{part}|}|\\
|\||fi|
\end{tabular}
\end{center}
%
The conditional |\ifchilddocmanual| is true whenever
a part to be included by |\input| is being compiled,
and the name of the part is stored in |\childdocname|.

%%%%%%%%%%%%%%%%%%%%%%%%%%%%%%%%%%%%%%%%
\DescribeMacro{\childdocby}
Each part to be included by |\input| should start with:
%
\begin{center}
\begin{tabular}{l}
|\input{childdoc.def}|\\
|\childdocby{|\textit{main}|}|\\
\end{tabular}
\end{center}
%
The directive |\childdocby| is similar to |\childdocof|
described in \secref{sec:include},
but the subsequent selection of content must be done manually.
To that end, both |\ifchilddoc| and |\ifchilddocmanual|
will be true upon processing of a part,
and the name of the part is stored in |\childdocname|.
Note that |\jobname| will be set to the filename of the current part
so that each part receives an individual |.aux| file
that does not interfere with the |.aux| file(s) of the main document.
This behaviour can be altered by the alternative form
|\childdocby[*]{|\textit{main}|}| (with a non-empty optional argument)
which uses the |.aux| file of the main document
by setting |\jobname| to \textit{main}.

%%%%%%%%%%%%%%%%%%%%%%%%%%%%%%%%%%%%%%%%%%%%%%%%%%%%%%%%%%%%%%%%%%%%%%%%%%%%%%%%
\subsection{Driver Development}
\label{sec:driver}

The \textsf{childdoc} mechanism can also be use for the development
of definition files such as \LaTeX{} styles or classes.
This case differs from the above setup with multiple parts
included by |\include| in that no |\includeonly| should be invoked.
This can be achieved by starting the include file
(before |\ProvidesPackage|) with:
%
\begin{center}
\begin{tabular}{l}
|\input{childdoc.def}|\\
|\childdocforward{|\textit{main}|}|\\
\end{tabular}
\end{center}
%
or alternatively with:
%
\begin{center}
\begin{tabular}{l}
|\input{childdoc.def}|\\
|\childdocby{|\textit{main}|}|\\
\end{tabular}
\end{center}
%
Both forms have slightly different effects as described above.
The main file is prepared as usual, see \secref{sec:include}.

%%%%%%%%%%%%%%%%%%%%%%%%%%%%%%%%%%%%%%%%%%%%%%%%%%%%%%%%%%%%%%%%%%%%%%%%%%%%%%%%
\subsection{Legacy Detection}
\label{sec:detection}

The directive |\childdocmain| in the main file can detect
whether the complete document or merely a child is to be compiled
even without using the directive |\childdocof|.
This method is deprecated because it is less robust
and there is no compelling reason to use it;
it is merely provided for backward compatibility
and it may be removed in future versions.

If the detection mechanism is to be used,
it is mandatory to correctly specify
the filename of the main file as the argument of |\childdocmain|:
%
\begin{center}
\begin{tabular}{l}
|\input{childdoc.def}|\\
|\childdocmain{|\textit{main}|}|\\
\end{tabular}
\end{center}
%
If |\jobname| does not match the argument \textit{main} of |\childdocmain|,
it is assumed that |\jobname| points to the child file to be compiled.
When using |\childdocmain| with the main file specified as argument,
it suffices to start a child file
with just |\input{|\textit{main}|}|
without loading of the package and using |\childdocof|.
If instead all processing is done
with the appropriate \textsf{childdoc} directives,
the argument of \textit{main} of |\childdocmain| can be empty.

An alternative version of the command line processing described
in \secref{sec:commandline} using the detection mechanism reads:
%
\begin{center}
|... -jobname "|\textit{target}|" "|[\textit{flags}]%
[|\def\jobname{|\textit{dest}|}|]|\input{|\textit{main}|}"|
\end{center}

%%%%%%%%%%%%%%%%%%%%%%%%%%%%%%%%%%%%%%%%%%%%%%%%%%%%%%%%%%%%%%%%%%%%%%%%%%%%%%%%
\subsection{Manual Code}
\label{sec:manual}

In case one cannot be certain whether the definitions file |childdoc.def|
is installed on the target \TeX{} distribution
and one prefers not to ship it,
it is conceivable to paste a few relevant commands into the sources.

To that end, drop all statements |\input{childdoc.def}|
and perform the replacements as outlined below.
Instead of |\childdocmain{|\textit{main}|}| add the following code
to the top of the main file:
%
\begin{center}
\begin{tabular}{l}
|\||ifdefined\childdocname\endinput\||fi\newif\ifchilddoc|\\
|\edef\childdocname{\scantokens\expandafter{\jobname\noexpand}}|\\
|\def\childdocmain{|\textit{main}|}\||ifx\childdocmain\childdocname\||else|\\
|\childdoctrue\includeonly{\childdocname}\let\jobname\childdocmain\||fi|\\
\end{tabular}
\end{center}
%
Instead of |\childdocof{|\textit{main}|}| just include the main file
at the top of each child file:
%
\begin{center}
|\input{|\textit{main}|}|
\end{center}
%
A simple redirection |\childdocforward{|\textit{dest}|}| is achieved by:
%
\begin{center}
|\def\jobname{|\textit{dest}|}\input{\jobname}|
\end{center}
%
The redirection with prefix
|\childdocforwardprefix[|\textit{prefix}|]{|\textit{dest}|}|
is accomplished by:
%
\begin{center}
\begin{tabular}{l}
|{\edef\jobname{\scantokens\expandafter{\jobname\noexpand}}|\\
|\def\redirectjob |\textit{prefix}|#1~~~{\gdef\jobname{|\textit{dest}|#1}}|\\
|\expandafter\redirectjob\jobname~~~}\input{\jobname}|
\end{tabular}
\end{center}

In an alternative approach,
child documents can be compiled by a specific command line
without additional code or specific definitions:
%
\begin{center}
|... -jobname "|\textit{target}|" "|[\textit{flags}]%
|\includeonly{|\textit{dest}|}\input{|\textit{main}|}"|
\end{center}
%

%%%%%%%%%%%%%%%%%%%%%%%%%%%%%%%%%%%%%%%%%%%%%%%%%%%%%%%%%%%%%%%%%%%%%%%%%%%%%%%%
%%%%%%%%%%%%%%%%%%%%%%%%%%%%%%%%%%%%%%%%%%%%%%%%%%%%%%%%%%%%%%%%%%%%%%%%%%%%%%%%
\section{Information}

%%%%%%%%%%%%%%%%%%%%%%%%%%%%%%%%%%%%%%%%%%%%%%%%%%%%%%%%%%%%%%%%%%%%%%%%%%%%%%%%
\subsection{Copyright}

Copyright \copyright{} 2017--2018 Niklas Beisert

This work may be distributed and/or modified under the
conditions of the \LaTeX{} Project Public License, either version 1.3
of this license or (at your option) any later version.
The latest version of this license is in
  \url{http://www.latex-project.org/lppl.txt}
and version 1.3 or later is part of all distributions of \LaTeX{}
version 2005/12/01 or later.

This work has the LPPL maintenance status `maintained'.

The Current Maintainer of this work is Niklas Beisert.

This work consists of the files |README.txt|, |childdoc.ins| and |childdoc.dtx|
as well as the derived files |childdoc.def|, |cdocsamp.tex|
with |cdocsch1.tex|, |cdocsch2.tex|, |cdocspt3.tex|, |cdocspt4.tex|,
|cdocsdrf.tex|, |cdocsfn1.tex|, |cdocsfn2.tex|
as well as |childdoc.pdf|.

%%%%%%%%%%%%%%%%%%%%%%%%%%%%%%%%%%%%%%%%%%%%%%%%%%%%%%%%%%%%%%%%%%%%%%%%%%%%%%%%
\subsection{Files and Installation}

The package consists of the files:
%
\begin{center}
\begin{tabular}{ll}
    |README.txt|   & readme file \\
    |childdoc.ins| & installation file \\
    |childdoc.dtx| & source file \\
    |childdoc.def| & definition file \\
    |cdocsamp.tex| & sample main file \\
    |cdocsch1.tex| & sample include file \\
    |cdocsch2.tex| & sample include file \\
    |cdocspt3.tex| & sample part file \\
    |cdocspt4.tex| & sample part file \\
    |cdocsdrf.tex| & sample redirection file \\
    |cdocsfn1.tex| & sample redirection file \\
    |cdocsfn2.tex| & sample redirection file \\
    |childdoc.pdf| & manual
\end{tabular}
\end{center}
%
The distribution consists of the files
|README.txt|, |childdoc.ins| and |childdoc.dtx|.
%
\begin{itemize}
\item
Run (pdf)\LaTeX{} on |childdoc.dtx|
to compile the manual |childdoc.pdf| (this file).
\item
Run \LaTeX{} on |childdoc.ins| to create the definitions file |childdoc.def|
and the sample |cdocsamp.tex| with include files
|cdocsch1.tex|, |cdocsch2.tex|, |cdocspt3.tex|, |cdocspt4.tex|,
|cdocsdrf.tex|, |cdocsfn1.tex|, |cdocsfn2.tex|.
Then copy the file |childdoc.def| to an appropriate directory of your \LaTeX{}
distribution, e.g.\ \textit{texmf-root}|/tex/latex/childdoc|.
\end{itemize}

%%%%%%%%%%%%%%%%%%%%%%%%%%%%%%%%%%%%%%%%%%%%%%%%%%%%%%%%%%%%%%%%%%%%%%%%%%%%%%%%
\subsection{Related CTAN Packages}

There are several other packages which offer a similar functionality:
%
\begin{itemize}
\item
The packages
\href{http://ctan.org/pkg/docmute}{\textsf{docmute}},
\href{http://ctan.org/pkg/includex}{\textsf{includex}} and
\href{http://ctan.org/pkg/standalone}{\textsf{standalone}}
provide commands to include only the document body of
a child file thus allowing both files to be compiled individually.
\item
The packages \href{http://ctan.org/pkg/subdocs}{\textsf{subdocs}}
and \href{http://ctan.org/pkg/subfiles}{\textsf{subfiles}}
provide structures in which the main and child documents can be
encapsulated and allowing them to be compiled individually.
The inclusion mechanism is different from the conventional |\include|.
\item
The package \href{http://ctan.org/pkg/combine}{\textsf{combine}}
is an elaborate solution to combine several documents into one.
\end{itemize}
%
See also the CTAN topic \href{http://ctan.org/topic/subdocs}{\textsf{subdocs}}
for further related packages.
The present package differs from the above solutions in that
a document structure constructed with the conventional |\include| mechanism
just needs two extra commands at the top of every file
such that all constituent files can be compiled individually.

%%%%%%%%%%%%%%%%%%%%%%%%%%%%%%%%%%%%%%%%%%%%%%%%%%%%%%%%%%%%%%%%%%%%%%%%%%%%%%%%
%\subsection{Feature Suggestions}
%
%The following is a list of features which may be useful for future
%versions of this package:
%%
%\begin{itemize}
%\item
%\ldots
%\end{itemize}

%%%%%%%%%%%%%%%%%%%%%%%%%%%%%%%%%%%%%%%%%%%%%%%%%%%%%%%%%%%%%%%%%%%%%%%%%%%%%%%%
\subsection{Revision History}

%%%%%%%%%%%%%%%%%%%%%%%%%%%%%%%%%%%%%%%%
\paragraph{v2.0:} 2018/12/30

\begin{itemize}
\item
immediate forward processing
\item
added |\childdocby| mechanism
\item
manual restructured
\end{itemize}

%%%%%%%%%%%%%%%%%%%%%%%%%%%%%%%%%%%%%%%%
\paragraph{v1.6:} 2018/01/17

\begin{itemize}
\item
application for development of include files
\item
corrections to manual
\end{itemize}

%%%%%%%%%%%%%%%%%%%%%%%%%%%%%%%%%%%%%%%%
\paragraph{v1.5:} 2017/05/21

\begin{itemize}
\item
more complete structuring introduced
\item
|\childdocof| introduced
\item
|\childdoc| renamed to |\childdocmain|
\item
|\childredirect| renamed to |\childdocforward| and |\childdocforwardprefix|
and functionality expanded
\end{itemize}

%%%%%%%%%%%%%%%%%%%%%%%%%%%%%%%%%%%%%%%%
\paragraph{v1.0:} 2017/04/27

\begin{itemize}
\item
manual and install package
\item
first version published on CTAN
\end{itemize}

%%%%%%%%%%%%%%%%%%%%%%%%%%%%%%%%%%%%%%%%
\paragraph{v0.6:} 2017/04/26

\begin{itemize}
\item
redirection mechanism added
\end{itemize}

%%%%%%%%%%%%%%%%%%%%%%%%%%%%%%%%%%%%%%%%
\paragraph{v0.5:} 2017/04/26

\begin{itemize}
\item
functionality in definition file
\end{itemize}


%%%%%%%%%%%%%%%%%%%%%%%%%%%%%%%%%%%%%%%%%%%%%%%%%%%%%%%%%%%%%%%%%%%%%%%%%%%%%%%%
%%%%%%%%%%%%%%%%%%%%%%%%%%%%%%%%%%%%%%%%%%%%%%%%%%%%%%%%%%%%%%%%%%%%%%%%%%%%%%%%
%%%%%%%%%%%%%%%%%%%%%%%%%%%%%%%%%%%%%%%%%%%%%%%%%%%%%%%%%%%%%%%%%%%%%%%%%%%%%%%%
\appendix

\settowidth\MacroIndent{\rmfamily\scriptsize 000\ }

 \DocInput{childdoc.dtx}

\end{document}
%</driver>
% \fi
%
% %%%%%%%%%%%%%%%%%%%%%%%%%%%%%%%%%%%%%%%%%%%%%%%%%%%%%%%%%%%%%%%%%%%%%%%%%%%%%%
% %%%%%%%%%%%%%%%%%%%%%%%%%%%%%%%%%%%%%%%%%%%%%%%%%%%%%%%%%%%%%%%%%%%%%%%%%%%%%%
% \section{Sample}
%\iffalse
%<*samplemain>
%\fi
%
% The following presents a sample document
% with two chapters, two parts, a title page,
% a compile flag as well as three forwarding files to set the flag.
% It consists of eight |.tex| files:
% \begin{center}
% \begin{tabular}{ll}
% |cdocsamp.tex|&main file\\
% |cdocsch1.tex|&include file for chapter 1\\
% |cdocsch2.tex|&include file for chapter 2\\
% |cdocspt3.tex|&include file for part 3\\
% |cdocspt4.tex|&include file for part 4\\
% |cdocsdrf.tex|&forwarding file for main file in draft mode\\
% |cdocsfi1.tex|&forwarding file for final version of chapter 1\\
% |cdocsfi2.tex|&forwarding file for final version of chapter 2\\
% \end{tabular}
% \end{center}
% Each of the eight files can be compiled directly by the \LaTeX{} compiler.
%
% %%%%%%%%%%%%%%%%%%%%%%%%%%%%%%%%%%%%%%
% \paragraph{Main File.}
%
% The main file is called |cdocsamp.tex|.
%
% Load the \textsf{childdoc} definitions and
% declare the filename for the main document:
%    \begin{macrocode}
\input{childdoc.def}
\childdocmain{}
%    \end{macrocode}

% Optional override for |\version| flag:
%    \begin{macrocode}
%%\ifchilddoc\else\providecommand{\version}{draft}\fi
%    \end{macrocode}

% Define the default values for the |\version| flag
% (|final| for the main file and |draft| for childs):
%    \begin{macrocode}
\ifchilddoc
\providecommand{\version}{draft}
\else
\providecommand{\version}{final}
\fi
%    \end{macrocode}

% Load the standard document class:
%    \begin{macrocode}
\documentclass[12pt]{article}
%    \end{macrocode}

% Start the document body:
%    \begin{macrocode}
\begin{document}
%    \end{macrocode}

% Declare a title page.
% Print title, part of document being processed and version flag:
%    \begin{macrocode}
\addtocounter{page}{-1}
\begin{center}
{\LARGE\bfseries{}childdoc example\par}
\vspace{1cm}
\ifchilddoc
\ifchilddocmanual part\else chapter\fi:
`\childdocname' of `\childdocjob'\par
\else
main document: `\childdocjob'\par
\fi
version: \version\par
\end{center}
\newpage
%    \end{macrocode}

% Manually include selected file,
% otherwise process as usual:
%    \begin{macrocode}
\ifchilddocmanual
\section*{part `\childdocname'}
\input{\childdocname}
\else
%    \end{macrocode}

% Include the two chapters:
%    \begin{macrocode}
\include{cdocsch1}
\include{cdocsch2}
%    \end{macrocode}

% Include the two parts unless only chapters should be displayed:
%    \begin{macrocode}
\ifchilddoc\else
\section{part three}
\input{cdocspt3}
\section{part four}
\input{cdocspt4}
\fi
%    \end{macrocode}

% Process as usual until here:
%    \begin{macrocode}
\fi
%    \end{macrocode}

% End of document body:
%    \begin{macrocode}
\end{document}
%    \end{macrocode}
%\iffalse
%</samplemain>
%\fi
%
% %%%%%%%%%%%%%%%%%%%%%%%%%%%%%%%%%%%%%%
% \paragraph{Chapter Include Files.}
%
% The include files are called |cdocsch1.tex| and |cdocsch2.tex|.
%
%\iffalse
%<*samplechap1|samplechap2>
%\fi

% Optional override for |\version| flag:
%    \begin{macrocode}
%%\providecommand{\version}{final}
%    \end{macrocode}

% Include the main document:
%    \begin{macrocode}
\input{childdoc.def}
\childdocof{cdocsamp}
%    \end{macrocode}

%\iffalse
%</samplechap1|samplechap2>
%\fi
%
%\iffalse
%<*samplechap1>
%\fi
% Some text for chapter 1:
%    \begin{macrocode}
\section{one}
some text in chapter one
%    \end{macrocode}

%\iffalse
%</samplechap1>
%\fi
% Some text for chapter 2:
%\iffalse
%<*samplechap2>
%\fi
%    \begin{macrocode}
\section{two}
more text in chapter two
%    \end{macrocode}

%\iffalse
%</samplechap2>
%\fi
%
% %%%%%%%%%%%%%%%%%%%%%%%%%%%%%%%%%%%%%%
% \paragraph{Part Include Files.}
%
% The include files are called |cdocspt3.tex| and |cdocspt4.tex|.
%
%\iffalse
%<*samplepart3|samplepart4>
%\fi

% Optional override for |\version| flag:
%    \begin{macrocode}
%%\providecommand{\version}{final}
%    \end{macrocode}

% Include the main document:
%    \begin{macrocode}
\input{childdoc.def}
\childdocby{cdocsamp}
%    \end{macrocode}

%\iffalse
%</samplepart3|samplepart4>
%\fi
%
%\iffalse
%<*samplepart3>
%\fi
% Some text for part 3:
%    \begin{macrocode}
some text in part three
%    \end{macrocode}

%\iffalse
%</samplepart3>
%\fi
% Some text for part 4:
%\iffalse
%<*samplepart4>
%\fi
%    \begin{macrocode}
more text in part four
%    \end{macrocode}

%\iffalse
%</samplepart4>
%\fi
%
% %%%%%%%%%%%%%%%%%%%%%%%%%%%%%%%%%%%%%%
% \paragraph{Forwarding for a Complete Draft.}
%
% The following forwarding file |cdocsdrf.tex|
% compiles the main document in draft mode:
%\iffalse
%<*sampledraft>
%\fi
%    \begin{macrocode}
\def\version{draft}
\input{childdoc.def}
\childdocforward{cdocsamp}
%    \end{macrocode}

%\iffalse
%</sampledraft>
%\fi
%
% %%%%%%%%%%%%%%%%%%%%%%%%%%%%%%%%%%%%%%
% \paragraph{Forwarding for Final Version of the Chapters.}
%
% The following forwarding files |cdocsfn1.tex| and |cdocsfn2.tex|
% (with identical content)
% compile the final versions of the child documents
% |cdocsch1.tex| and |cdocsch2.tex|, respectively:
%\iffalse
%<*samplefinal>
%\fi
%    \begin{macrocode}
\def\version{final}
\input{childdoc.def}
\childdocforwardprefix[cdocsamp]{cdocsfn}{cdocsch}
%    \end{macrocode}

%\iffalse
%</samplefinal>
%\fi
%
% %%%%%%%%%%%%%%%%%%%%%%%%%%%%%%%%%%%%%%
% \paragraph{Command Line Processing.}
%
% The following three command lines generate the output files
% |cdocscld|, |cdocscl1| and |cdocscl2|
% which should be identical to
% |cdocsdrf|, |cdocsch1| and |cdocsfn2|, respectively:
% \begin{center}
% \begin{tabular}{l}
% |latex -jobname cdocscld \|\\
% |  "\def\version{draft}\input{childdoc.def}\childdocforward{cdocsamp}"|\\
% |latex -jobname cdocscl1 \|\\
% |  "\input{childdoc.def}\childdocforward[cdocsamp]{cdocsch1}"|\\
% |latex -jobname cdocscl2 \|\\
% |  "\def\version{final}\input{childdoc.def}\childdocforward{cdocsch2}"|
% \end{tabular}
% \end{center}
% Note that the trailing backslash on each first line
% merely continues the input to the second line
% (for convenient cut ant paste).
% Furthermore, the command |latex| can be replaced by any
% of its alternative versions such as |pdflatex|.
%
% %%%%%%%%%%%%%%%%%%%%%%%%%%%%%%%%%%%%%%%%%%%%%%%%%%%%%%%%%%%%%%%%%%%%%%%%%%%%%%
% %%%%%%%%%%%%%%%%%%%%%%%%%%%%%%%%%%%%%%%%%%%%%%%%%%%%%%%%%%%%%%%%%%%%%%%%%%%%%%
% \section{Implementation}
%\iffalse
%<*package>
%\fi
%
% This section describes the definitions file |childdoc.def|.

% The definitions cannot be loaded using |\usepackage| or |\RequirePackage|
% which has a mechanism to prevent loading a style file more than once.
% When loading the definitions by means of |\input|
% multiple instances have to be prevented manually:
%\iffalse
%This code needs to be before the `\ProvidesFile' directive
%which is defined at the beginning of this file.
%Therefore it is also placed there and commented out here.
%</package>
%<*discard>
%\fi
%    \begin{macrocode}
\ifdefined\childdocmain\endinput\fi
%    \end{macrocode}
%\iffalse
%</discard>
%<*package>
%\fi
%
% \macro{\ifchilddoc}
% \macro{\ifchilddocmanual}
% The conditional |\ifchilddoc| tells whether a
% child (true) or main (false) document is being compiled.
% The conditional |\ifchilddocmanual| tells whether
% the |\includeonly| mechanism is used (false) or
% the selection of child files must be performed manually (true).
% The definitions initialise to false:
%    \begin{macrocode}
\newif\ifchilddoc
\newif\ifchilddocmanual
%    \end{macrocode}

% \macro{\childdocname}
% \macro{\childdocjob}
% The macro |\childdocname| stores the name of the main document
% to be compiled. The macro |\childdocjob| stores the name of
% the document on which the \LaTeX{} compiler was originally invoked.
% The content of |\jobname| cannot be compared
% to filenames specified in the source due to different catcodes.
% The following code rescans |\jobname|, stores the result
% in |\childdocname| and saves a copy in |\childdocjob|:
%    \begin{macrocode}
\edef\childdocname{\scantokens\expandafter{\jobname\noexpand}}
\let\childdocjob\childdocname
%    \end{macrocode}

% \macro{\childdocdisable}
% The macro |\childdocdisable| prevents the main file
% from being processed more than once.
% At this stage, the main document command |\childdocmain|
% is assumed to be called once again where it should do nothing.
% Any subsequent call to it should prevent
% a secondary processing of the main document
% It overwrites the forwarding commands
% |\childdocof| and |\childdocforward|
% with empty macros to prevent further inclusions of the main document:
%    \begin{macrocode}
\newcommand{\childdocdisable}
{
  \renewcommand{\childdocmain}[1]{\renewcommand{\childdocmain}[1]{\endinput}}
  \renewcommand{\childdocof}[1]{}
  \renewcommand{\childdocby}[2][]{}
  \renewcommand{\childdocforward}[2][]{}
  \renewcommand{\childdocdisable}{}
}
%    \end{macrocode}

% \macro{\childdocmain}
% The macro |\childdocmain| is to be called at the top of the main file
% with nothing or the main filename (without extension) as argument.
% First, it breaks loops.
% If the argument is not empty and does not match |\childdocname|
% (which is set by the first inclusion of |childdoc.def|),
% |\ifchilddoc| is set to true, |\includeonly| is applied to the child file
% and |\jobname| is set to the main file
% (for proper handling of |.aux| files):
%    \begin{macrocode}
\newcommand{\childdocmain}[1]
{
  \childdocdisable\childdocmain{}
  \if?#1?\else
    \begingroup
      \def\childdoctmp{#1}
      \ifx\childdoctmp\childdocname
        \def\childdoctmp{}
      \else
        \def\childdoctmp
        {
          \childdoctrue
          \includeonly{\childdocname}
          \def\childdocjob{#1}
          \def\jobname{#1}
        }
      \fi
      \expandafter
    \endgroup
    \childdoctmp
  \fi
}
%    \end{macrocode}

% \macro{\childdocof}
% The command |\childdocof| redirects
% compilation to the main file |#1|.
%    \begin{macrocode}
\newcommand{\childdocof}[1]
{
  \childdocdisable
  \childdoctrue
  \includeonly{\childdocname}
  \def\jobname{#1}
  \def\childdocjob{#1}
  \input{#1}
}
%    \end{macrocode}

% \macro{\childdocby}
% The command |\childdocby| ....
%    \begin{macrocode}
\newcommand{\childdocby}[2][]
{
  \childdocdisable
  \childdoctrue
  \childdocmanualtrue
  \if?#1?\else
    \def\jobname{#2}
  \fi
  \def\childdocjob{#2}
  \input{#2}
  \endinput
}
%    \end{macrocode}

% \macro{\childdocforward}
% The command |\childdocforward| redirects
% compilation to the main file or
% (if the optional argument is given) a child file.
% Parameters are set as if the main file
% or a child file starting with |\childdocof| was compiled.
% Then compilation is handed over to the main file:
%    \begin{macrocode}
\newcommand{\childdocforward}[2][]
{
  \begingroup
    \if?#1?
      \def\childdoctmp
      {
        \def\childdocname{#2}
        \def\childdocjob{#2}
        \def\jobname{#2}
        \input{#2}
        \endinput
      }
    \else
      \def\childdoctmp
      {
        \childdocdisable
        \def\childdocname{#2}
        \childdoctrue
        \includeonly{#2}
        \def\childdocjob{#1}
        \def\jobname{#1}
        \input{#1}
        \endinput
      }
    \fi
    \expandafter
  \endgroup
  \childdoctmp
}
%    \end{macrocode}

% \macro{\childdocforwardprefix}
% The command |\childdocforwardprefix| redirects
% compilation to the main or a child file by means of a pattern.
% The prefix |#1| in the current filename is replaced by |#2|
% and the suffix of the current filename is kept
% (it is assumed that the filename does not contain the substring `|~~~|'
% which is used as a delimiter).
% Compilation is handed over to the new file by |\childdocforward|:
%    \begin{macrocode}
\newcommand{\childdocforwardprefix}[3][]
{
  \begingroup
    \def\childdocextract #2##1~~~{\def\childdoctmp{\childdocforward[#1]{#3##1}}}
    \expandafter\childdocextract\childdocname~~~
    \expandafter
  \endgroup
  \childdoctmp
}
%    \end{macrocode}

% \macro{\childdoc}
% The deprecated macro |\childdoc| is a legacy version of |\childdocmain|:
%    \begin{macrocode}
\newcommand{\childdoc}{\childdocmain}
%    \end{macrocode}

% \macro{\childdocredirect}
% The deprecated macro |\childdocredirect| is a legacy version
% of |\childdocforward| and |\childdocforwardprefix|:
%    \begin{macrocode}
\newcommand{\childdocredirect}[2][]
{
  \begingroup
    \if?#1?
      \def\childdoctmp{\childdocforward{#2}}
    \else
      \def\childdoctmp{\childdocforwardprefix{#1}{#2}}
    \fi
    \expandafter
  \endgroup
  \childdoctmp
}
%    \end{macrocode}

%\iffalse
%</package>
%\fi
%
\endinput
\childdocforward[cdocsamp]{cdocsch1}"|\\
% |latex -jobname cdocscl2 \|\\
% |  "\def\version{final}% \iffalse
%
% childdoc.dtx Copyright (C) 2017-2018 Niklas Beisert
%
% This work may be distributed and/or modified under the
% conditions of the LaTeX Project Public License, either version 1.3
% of this license or (at your option) any later version.
% The latest version of this license is in
%   http://www.latex-project.org/lppl.txt
% and version 1.3 or later is part of all distributions of LaTeX
% version 2005/12/01 or later.
%
% This work has the LPPL maintenance status `maintained'.
%
% The Current Maintainer of this work is Niklas Beisert.
%
% This work consists of the files childdoc.dtx and childdoc.ins
% and the derived files childdoc.def and cdocsamp.tex with
% cdocsch1.tex, cdocsch2.tex, cdocsdrf.tex, cdocsfn1.tex, cdocsfn2.tex.
%
%<package>\ifdefined\childdocmain\endinput\fi
%<package>\ProvidesFile{childdoc.def}[2018/12/30 v2.0 child document driver]
%<samplemain>\ProvidesFile{cdocsamp.tex}[2018/12/30 v2.0 sample for childdoc]
%<*driver>
%\ProvidesFile{childdoc.drv}[2018/12/30 v2.0 childdoc reference manual file]
\PassOptionsToClass{10pt,a4paper}{article}
\documentclass{ltxdoc}

\usepackage[margin=35mm]{geometry}
\usepackage{hyperref}
\usepackage{hyperxmp}
\usepackage[usenames]{color}

\hypersetup{colorlinks=true}
\hypersetup{pdfstartview=FitH}
\hypersetup{pdfpagemode=UseNone}
\hypersetup{pdfsource={}}
\hypersetup{pdflang={en-UK}}
\hypersetup{pdfcopyright={Copyright 2017-2018 Niklas Beisert.
  This work may be distributed and/or modified under the
  conditions of the LaTeX Project Public License, either version 1.3
  of this license or (at your option) any later version.}}
\hypersetup{pdflicenseurl={http://www.latex-project.org/lppl.txt}}
\hypersetup{pdfcontactaddress={ETH Zurich, ITP, HIT K,
  Wolfgang-Pauli-Strasse 27}}
\hypersetup{pdfcontactpostcode={8093}}
\hypersetup{pdfcontactcity={Zurich}}
\hypersetup{pdfcontactcountry={Switzerland}}
\hypersetup{pdfcontactemail={nbeisert@itp.phys.ethz.ch}}
\hypersetup{pdfcontacturl={http://people.phys.ethz.ch/\xmptilde nbeisert/}}

\newcommand{\secref}[1]{\hyperref[#1]{section \ref*{#1}}}

\parskip1ex
\parindent0pt
\let\olditemize\itemize
\def\itemize{\olditemize\parskip0pt}

\begin{document}

\title{The \textsf{childdoc} Package}
\hypersetup{pdftitle={The childdoc Package}}
\author{Niklas Beisert\\[2ex]
  Institut f\"ur Theoretische Physik\\
  Eidgen\"ossische Technische Hochschule Z\"urich\\
  Wolfgang-Pauli-Strasse 27, 8093 Z\"urich, Switzerland\\[1ex]
  \href{mailto:nbeisert@itp.phys.ethz.ch}
  {\texttt{nbeisert@itp.phys.ethz.ch}}}
\hypersetup{pdfauthor={Niklas Beisert}}
\hypersetup{pdfsubject={Manual for the LaTeX2e Package childdoc}}
\date{30 December 2018, \textsf{v2.0}}
\maketitle

\begin{abstract}\noindent
\textsf{childdoc} is a \LaTeXe{} package
that enables the direct compilation
of document sections included by |\include|
to individual files.
\end{abstract}

\begingroup
\parskip0ex
\tableofcontents
\endgroup

%%%%%%%%%%%%%%%%%%%%%%%%%%%%%%%%%%%%%%%%%%%%%%%%%%%%%%%%%%%%%%%%%%%%%%%%%%%%%%%%
%%%%%%%%%%%%%%%%%%%%%%%%%%%%%%%%%%%%%%%%%%%%%%%%%%%%%%%%%%%%%%%%%%%%%%%%%%%%%%%%
\section{Introduction}

\LaTeX{} provides a mechanism to structure a large document (such as a book)
into a main file and several child files (containing the chapters)
using the |\include| command.
This mechanism is beneficial for documents
which span hundreds of pages in order to
make the source file(s) more manageable.
Moreover, compilation can be restricted to
selected child files by means of the |\includeonly| command.
The latter feature can be used to reduce the compilation time while editing
(this was significantly more useful in the earlier days of \LaTeX{})
or to generate a smaller document which is easier to navigate.
Another application of |\includeonly| is to generate
documents consisting of selected parts of the complete document.

However, there are a few drawbacks of the plain |\include| mechanism:
\begin{itemize}
\item
The child files cannot be compiled on their own,
they can only be compiled via the main file.
A naive editing environment
(such as a text editor with an option
to have the current file processed by \LaTeX)
may require one to switch to the main file before compiling;
attempting to compile the child file produces errors.
\item
The main file must be modified (each time)
to adjust the |\includeonly| command
to the present needs. This easily leaves the main file in a messy state.
\item
The generated document will always carry the filename
of the main document. This is inconvenient if
several child files are to be compiled and
to be kept for distribution.
\end{itemize}

The present package provides a simple interface
to make child files individually compilable by \LaTeX{}.
Compiling a child file then has the same effect as compiling
the main file with an |\includeonly| command
to select the appropriate child.
Moreover the generated document will carry the name of the child
rather than the main file.
This resolves all three above issues.

This feature is meant to make the editing of books,
thesis documents and lecture notes somewhat more convenient.
However, the package can also be used efficiently for
composing a series of documents (such as exercise sheets)
which are typically distributed individually.
It then assists the author in generating the individual documents
(potentially in different versions)
as well as a document containing the collected series.
Another application is in developing style files
or other kinds of included material
where compilation of the style file could redirect
to a sample or test file.

%%%%%%%%%%%%%%%%%%%%%%%%%%%%%%%%%%%%%%%%%%%%%%%%%%%%%%%%%%%%%%%%%%%%%%%%%%%%%%%%
%%%%%%%%%%%%%%%%%%%%%%%%%%%%%%%%%%%%%%%%%%%%%%%%%%%%%%%%%%%%%%%%%%%%%%%%%%%%%%%%
\section{Usage}

First of all, the package \textsf{childdoc} is \emph{not} a standard
\LaTeXe{} |.sty| style file! Therefore it needs to be invoked in
a non-standard way.

%%%%%%%%%%%%%%%%%%%%%%%%%%%%%%%%%%%%%%%%%%%%%%%%%%%%%%%%%%%%%%%%%%%%%%%%%%%%%%%%
\subsection{Included Files}
\label{sec:include}

%%%%%%%%%%%%%%%%%%%%%%%%%%%%%%%%%%%%%%%%
\DescribeMacro{\childdocmain}
To use the package, add the commands
\begin{center}
\begin{tabular}{l}
|\input{childdoc.def}|\\
|\childdocmain{}|\\
\end{tabular}
\end{center}
at the very top of the main \LaTeX{} file,
in particular \emph{before} the |\documentclass| statement!
The argument of |\childdocmain| should be left empty
(but it must be present).

%%%%%%%%%%%%%%%%%%%%%%%%%%%%%%%%%%%%%%%%
\DescribeMacro{\childdocof}
Furthermore, add the commands
\begin{center}
\begin{tabular}{l}
|\input{childdoc.def}|\\
|\childdocof{|\textit{main}|}|\\
\end{tabular}
\end{center}
at the top of every child file \textit{child}
which is included by |\include{|\textit{child}|}|
from within the main file
(or at least for those files to be compiled individually).
The argument \textit{main} must be the filename of the main file.

There are a couple of
considerations in setting up the main and child documents:

%%%%%%%%%%%%%%%%%%%%%%%%%%%%%%%%%%%%%%%%
\paragraph{Restrictions.}

Please note the following restrictions:
\begin{itemize}
\item
|\childdocmain| must be called with one argument \textit{main}
to ensure compatibility with earlier version of the package.
It must either be empty (|\childdocmain{}|)
or precisely match the filename of the main file in which it is specified.
See \secref{sec:detection} for further information.
\item
The filename \textit{main} must be specified without the |.tex| extension.
\item
The filename \textit{main} is case sensitive
(even in case-insensitive file systems)
due to internal string comparison.
\item
The argument \textit{main} should be fully expanded, it cannot be a macro.
\item
Subdirectories and special characters should be avoided in filenames.
\item
The command |\childdocmain{|\textit{main}|}| must be followed by a whitespace.
It should not be followed immediately by another command
or by a comment mark `|%|'.
This is because the \TeX{} parser reads the token immediately following
the argument of |\childdocmain| and puts it
at the beginning of every child section;
however, a white\-space is ignored.
\end{itemize}

%%%%%%%%%%%%%%%%%%%%%%%%%%%%%%%%%%%%%%%%
\paragraph{Content of Main File.}

It is advisable to place all content in the child files included by |\include|.
Any output contained in the main file will appear in all child documents
unless suppressed manually;
it cannot be suppressed automatically by the |\includeonly| directive
and thus should normally be avoided.
A method to include some content in the main file
by means of conditional processing is described in \secref{sec:conditional}.

%%%%%%%%%%%%%%%%%%%%%%%%%%%%%%%%%%%%%%%%
\paragraph{Page Numbering.}

When only a part of the document is compiled,
the appropriate numbering of pages
(as well as other status parameters)
is determined from the |.aux| files.
The latter contain information from previous passes.
However this information needs to propagate through
all intermediate child documents.
Therefore the page numbering in child documents may well
be inconsistent until the complete document is compiled at least once.

A useful (if unconventional) way to always ensure a consistent
page numbering is to restart the numbering in each child document
and denote the pages by `\textit{child}|.|\textit{page}'
where \textit{child} represents the chapter/section number of the child file.
This can be achieved by the command
|\numberwithin{page}{|\textit{child}|}|
of the \textsf{amsmath} package
where \textit{child} can be |chapter| or |section|
depending on the chosen structuring.
Alternatively, one can modify the macro |\thepage| appropriately
and reset the counter |page| at the start of each child file.

%%%%%%%%%%%%%%%%%%%%%%%%%%%%%%%%%%%%%%%%%%%%%%%%%%%%%%%%%%%%%%%%%%%%%%%%%%%%%%%%
\subsection{Conditional Processing}
\label{sec:conditional}

The package provides a mechanism to compile different versions
of a document. To customise the versions further some conditional processing
can come in handy to distinguish which version is being compiled.
The package provides two macros to describe the compilation context:

%%%%%%%%%%%%%%%%%%%%%%%%%%%%%%%%%%%%%%%%
\DescribeMacro{\ifchilddoc}
The conditional |\ifchilddoc| distinguishes between the compilation of
child documents and the main document:
%
\begin{center}
|\ifchilddoc |\textit{child-code}| |[|\||else |\textit{main-code}]| \||fi|
\end{center}

%%%%%%%%%%%%%%%%%%%%%%%%%%%%%%%%%%%%%%%%
\DescribeMacro{\childdocname}
\DescribeMacro{\childdocjob}
The macro |\childdocname| contains the filename (without extension)
of the main or child file being processed.
Note that |\childdocjob| will always contain the name of the main file.

%%%%%%%%%%%%%%%%%%%%%%%%%%%%%%%%%%%%%%%%
\paragraph{Title Page.}

Conditional processing can be used to include a title or banner page
in the main document when proper precautions are taken.
Importantly, the code in the main file should ensure that the page counter
(as well as other status parameters which are stored in the |.aux| files)
takes the same value after the conditional processing.
Otherwise the page numbers may take divergent values
depending on which part is compiled.

For example, a title page could be declared by:
%
\begin{center}
\begin{tabular}{l}
|\ifchilddoc\||else|\\
|\addtocounter{page}{-1}|\\
\textit{code for title page}\\
|\newpage|\\
|\||fi|
\end{tabular}
\end{center}
%
A banner page for the child documents can be generated by:
%
\begin{center}
\begin{tabular}{l}
|\ifchilddoc|\\
|\addtocounter{page}{-1}|\\
\textit{code for banner page}\\
|\newpage|\\
|\||fi|
\end{tabular}
\end{center}
%
Here one could write a message such as:
\begin{center}
|This is the part \childdocname{} of \childdocjob{}.|
\end{center}

%%%%%%%%%%%%%%%%%%%%%%%%%%%%%%%%%%%%%%%%%%%%%%%%%%%%%%%%%%%%%%%%%%%%%%%%%%%%%%%%
\subsection{Flags}
\label{sec:flags}

The package makes it easy to generate different versions
of the main or child documents.
To this end compilation flags can be defined
and assigned different default values.
They will be particularly useful in conjunction
with the forwarding mechanism described in \secref{sec:forward}.

For example, it may be useful to have a flag |\version|
which can be set to |draft| or |final|.
The document source will contain some conditional code
depending on the value of |\version|.
Suppose further, the flag should default to |final| for the main file
and to |draft| for child files
which is a natural assignment for editing the document.
This is achieved by placing the following code
in the preamble of the main document
(below the |\childdocmain| directive):
%
\begin{center}
\begin{tabular}{l}
|\ifchilddoc|\\
|\providecommand{\version}{draft}|\\
|\||else|\\
|\providecommand{\version}{final}|\\
|\||fi|
\end{tabular}
\end{center}
%
The definition by |\providecommand| makes sure
that previous definitions are not overwritten.
Further statements |\providecommand{\version}{...}|
can thus be added before the above code to override it.

For the main file, one might add a line
(between |\childdocmain| and the above block)
%
\begin{center}
|%\ifchilddoc\||else\providecommand{\version}{draft}\||fi|
\end{center}
%
which can be uncommented to produce a draft version.
Likewise one can add a line to the very top of a child file
(above the |\childdocof{|\textit{main}|}| directive)
%
\begin{center}
|%\providecommand{\version}{final}|
\end{center}
%
which can be uncommented to produce the final version of this child document.

%%%%%%%%%%%%%%%%%%%%%%%%%%%%%%%%%%%%%%%%%%%%%%%%%%%%%%%%%%%%%%%%%%%%%%%%%%%%%%%%
\subsection{Forwarding}
\label{sec:forward}

Different versions of the main or child documents
using compilation flags as described in \secref{sec:flags}
can be (permanently) stored in different files
for convenient compilation, viewing and distribution.
To this end, the package defines a command
to pass on compilation to a different file:

%%%%%%%%%%%%%%%%%%%%%%%%%%%%%%%%%%%%%%%%
\DescribeMacro{\childdocforward}
The command |\childdocforward| redirects processing to
another source file:
%
\begin{center}
\begin{tabular}{l}
|\input{childdoc.def}|\\
|\childdocforward[|\textit{main}|]{|\textit{dest}|}|\\
\end{tabular}
\end{center}
%
The argument \textit{dest} is the destination file
(without extension).
It should be the main file or one of the child files.
Note that further \textsf{childdoc} directives
such as |\childdocof| and |\childdocforward|
in the indicated file will be processed in this form.
The optional argument \textit{main}
passes on directly to the main file \textit{main}
while pretending to compile the child \textit{dest}.
This form behaves as if \textit{dest}
issues |\childdocof{|\textit{main}|}| right away,
and no further \textsf{childdoc} directives will be processed.

%%%%%%%%%%%%%%%%%%%%%%%%%%%%%%%%%%%%%%%%
\DescribeMacro{\...prefix}
In the alternative form |\childdocforwardprefix|,
%
\begin{center}
\begin{tabular}{l}
|\input{childdoc.def}|\\
|\childdocforwardprefix[|\textit{main}|]{|\textit{prefix}|}{|\textit{dest}|}|
\end{tabular}
\end{center}
%
the destination file is determined by a pattern
depending on the current file:
To make this work, the current file must be called
`{\textit{prefix}\hspace{0.2em}\textit{suffix}}'
with \textit{prefix} matching precisely the argument.
Processing is then passed on to the file
`{\textit{dest}\hspace{0.2em}\textit{suffix}}'.
Surely, the same effect is achieved by
directly specifying the
argument `{\textit{dest}\hspace{0.2em}\textit{suffix}}'
in the first form.
However, that requires to set up a different file
for each child. With the alternative form of the command
all these files can have exactly the same content
which simplifies setting them up and maintaining them.

For example, the following file |draft.tex|
with a compilation flag |\version| as described in \secref{sec:flags}
compiles the main document as a draft:
%
\begin{center}
\begin{tabular}{l}
|\def\version{draft}|\\
|\input{childdoc.def}|\\
|\childdocforward{|\textit{main}|}|
\end{tabular}
\end{center}
%
Likewise, the following files |final|\textit{nn}|.tex|
compile the final version of the child document
|child|\textit{nn}|.tex|:
%
\begin{center}
\begin{tabular}{l}
|\def\version{final}|\\
|\input{childdoc.def}|\\
|\childdocforwardprefix{final}{child}|
\end{tabular}
\end{center}
%

Note that when several versions of a main file and/or of each child file
are to be generated, it may be convenient to set up a |Makefile| or
shell script to automatise the process.

%%%%%%%%%%%%%%%%%%%%%%%%%%%%%%%%%%%%%%%%%%%%%%%%%%%%%%%%%%%%%%%%%%%%%%%%%%%%%%%%
\subsection{Command Line Processing}
\label{sec:commandline}

The effect of redirection files can also be achieved by invoking
the \LaTeX{} compiler with a more elaborate command line.
Most conveniently this should be done as part
of a shell script or a |Makefile|.

When using \textsf{childdoc} in the main file, the following
command lines effectively perform a redirection
(note that depending on the shell being used,
backslashes may have to be doubled: `|\|' $\to$ `|\\|'):
%
\begin{center}
|... -jobname "|\textit{target}|" |\\|"|[\textit{flags}]%
|\input{childdoc.def}\childdocforward[|\textit{main}|]{|\textit{dest}|}"|
\end{center}
%
Here \textit{target} is the name of the output file,
\textit{main} is the name of the main file
and \textit{dest} is the name of the main or child file to be processed
(all filenames without extensions).
The optional argument \textit{main} can be omitted
if \textit{main} matches \textit{dest}.
Optionally, compilation \textit{flags} can be defined via |\def| commands.
This command line makes the \TeX{} engine believe
it is compiling the file \textit{target}
whose content is specified as the latter parameter.
The provided code then forwards the processing to
\textit{main} or \textit{dest} as described in \secref{sec:forward}.

%%%%%%%%%%%%%%%%%%%%%%%%%%%%%%%%%%%%%%%%%%%%%%%%%%%%%%%%%%%%%%%%%%%%%%%%%%%%%%%%
\subsection{Include by Input}
\label{sec:input}

Including child documents by |\include| has some restrictions by design.
Most notably, the content of a child document always occupies
its own set of pages; pages cannot be shared between child documents.
Usually, this behaviour makes perfect sense
because each child document contain an essential part of the document.
However, in some situations it may be desirable to compose
a document from a collection of parts
without having mandatory page breaks between then.
For this case, the package
provides a mechanism to include parts
by |\input| which can also be processed individually.
However, by construction this mechanism
requires manual handling of the content to be output.

%%%%%%%%%%%%%%%%%%%%%%%%%%%%%%%%%%%%%%%%
\DescribeMacro{\ifchilddocmanual}
The main file should be prepared as usual, see \secref{sec:include}.
However, the document body must make a distinction
between processing of an individual part and of the main document, e.g.:
%
\begin{center}
\begin{tabular}{l}
|\ifchilddocmanual|\\
|\input{\childdocname}|\\
|\||else|\\
\textit{document body with }|\input{|\textit{part}|}|\\
|\||fi|
\end{tabular}
\end{center}
%
The conditional |\ifchilddocmanual| is true whenever
a part to be included by |\input| is being compiled,
and the name of the part is stored in |\childdocname|.

%%%%%%%%%%%%%%%%%%%%%%%%%%%%%%%%%%%%%%%%
\DescribeMacro{\childdocby}
Each part to be included by |\input| should start with:
%
\begin{center}
\begin{tabular}{l}
|\input{childdoc.def}|\\
|\childdocby{|\textit{main}|}|\\
\end{tabular}
\end{center}
%
The directive |\childdocby| is similar to |\childdocof|
described in \secref{sec:include},
but the subsequent selection of content must be done manually.
To that end, both |\ifchilddoc| and |\ifchilddocmanual|
will be true upon processing of a part,
and the name of the part is stored in |\childdocname|.
Note that |\jobname| will be set to the filename of the current part
so that each part receives an individual |.aux| file
that does not interfere with the |.aux| file(s) of the main document.
This behaviour can be altered by the alternative form
|\childdocby[*]{|\textit{main}|}| (with a non-empty optional argument)
which uses the |.aux| file of the main document
by setting |\jobname| to \textit{main}.

%%%%%%%%%%%%%%%%%%%%%%%%%%%%%%%%%%%%%%%%%%%%%%%%%%%%%%%%%%%%%%%%%%%%%%%%%%%%%%%%
\subsection{Driver Development}
\label{sec:driver}

The \textsf{childdoc} mechanism can also be use for the development
of definition files such as \LaTeX{} styles or classes.
This case differs from the above setup with multiple parts
included by |\include| in that no |\includeonly| should be invoked.
This can be achieved by starting the include file
(before |\ProvidesPackage|) with:
%
\begin{center}
\begin{tabular}{l}
|\input{childdoc.def}|\\
|\childdocforward{|\textit{main}|}|\\
\end{tabular}
\end{center}
%
or alternatively with:
%
\begin{center}
\begin{tabular}{l}
|\input{childdoc.def}|\\
|\childdocby{|\textit{main}|}|\\
\end{tabular}
\end{center}
%
Both forms have slightly different effects as described above.
The main file is prepared as usual, see \secref{sec:include}.

%%%%%%%%%%%%%%%%%%%%%%%%%%%%%%%%%%%%%%%%%%%%%%%%%%%%%%%%%%%%%%%%%%%%%%%%%%%%%%%%
\subsection{Legacy Detection}
\label{sec:detection}

The directive |\childdocmain| in the main file can detect
whether the complete document or merely a child is to be compiled
even without using the directive |\childdocof|.
This method is deprecated because it is less robust
and there is no compelling reason to use it;
it is merely provided for backward compatibility
and it may be removed in future versions.

If the detection mechanism is to be used,
it is mandatory to correctly specify
the filename of the main file as the argument of |\childdocmain|:
%
\begin{center}
\begin{tabular}{l}
|\input{childdoc.def}|\\
|\childdocmain{|\textit{main}|}|\\
\end{tabular}
\end{center}
%
If |\jobname| does not match the argument \textit{main} of |\childdocmain|,
it is assumed that |\jobname| points to the child file to be compiled.
When using |\childdocmain| with the main file specified as argument,
it suffices to start a child file
with just |\input{|\textit{main}|}|
without loading of the package and using |\childdocof|.
If instead all processing is done
with the appropriate \textsf{childdoc} directives,
the argument of \textit{main} of |\childdocmain| can be empty.

An alternative version of the command line processing described
in \secref{sec:commandline} using the detection mechanism reads:
%
\begin{center}
|... -jobname "|\textit{target}|" "|[\textit{flags}]%
[|\def\jobname{|\textit{dest}|}|]|\input{|\textit{main}|}"|
\end{center}

%%%%%%%%%%%%%%%%%%%%%%%%%%%%%%%%%%%%%%%%%%%%%%%%%%%%%%%%%%%%%%%%%%%%%%%%%%%%%%%%
\subsection{Manual Code}
\label{sec:manual}

In case one cannot be certain whether the definitions file |childdoc.def|
is installed on the target \TeX{} distribution
and one prefers not to ship it,
it is conceivable to paste a few relevant commands into the sources.

To that end, drop all statements |\input{childdoc.def}|
and perform the replacements as outlined below.
Instead of |\childdocmain{|\textit{main}|}| add the following code
to the top of the main file:
%
\begin{center}
\begin{tabular}{l}
|\||ifdefined\childdocname\endinput\||fi\newif\ifchilddoc|\\
|\edef\childdocname{\scantokens\expandafter{\jobname\noexpand}}|\\
|\def\childdocmain{|\textit{main}|}\||ifx\childdocmain\childdocname\||else|\\
|\childdoctrue\includeonly{\childdocname}\let\jobname\childdocmain\||fi|\\
\end{tabular}
\end{center}
%
Instead of |\childdocof{|\textit{main}|}| just include the main file
at the top of each child file:
%
\begin{center}
|\input{|\textit{main}|}|
\end{center}
%
A simple redirection |\childdocforward{|\textit{dest}|}| is achieved by:
%
\begin{center}
|\def\jobname{|\textit{dest}|}\input{\jobname}|
\end{center}
%
The redirection with prefix
|\childdocforwardprefix[|\textit{prefix}|]{|\textit{dest}|}|
is accomplished by:
%
\begin{center}
\begin{tabular}{l}
|{\edef\jobname{\scantokens\expandafter{\jobname\noexpand}}|\\
|\def\redirectjob |\textit{prefix}|#1~~~{\gdef\jobname{|\textit{dest}|#1}}|\\
|\expandafter\redirectjob\jobname~~~}\input{\jobname}|
\end{tabular}
\end{center}

In an alternative approach,
child documents can be compiled by a specific command line
without additional code or specific definitions:
%
\begin{center}
|... -jobname "|\textit{target}|" "|[\textit{flags}]%
|\includeonly{|\textit{dest}|}\input{|\textit{main}|}"|
\end{center}
%

%%%%%%%%%%%%%%%%%%%%%%%%%%%%%%%%%%%%%%%%%%%%%%%%%%%%%%%%%%%%%%%%%%%%%%%%%%%%%%%%
%%%%%%%%%%%%%%%%%%%%%%%%%%%%%%%%%%%%%%%%%%%%%%%%%%%%%%%%%%%%%%%%%%%%%%%%%%%%%%%%
\section{Information}

%%%%%%%%%%%%%%%%%%%%%%%%%%%%%%%%%%%%%%%%%%%%%%%%%%%%%%%%%%%%%%%%%%%%%%%%%%%%%%%%
\subsection{Copyright}

Copyright \copyright{} 2017--2018 Niklas Beisert

This work may be distributed and/or modified under the
conditions of the \LaTeX{} Project Public License, either version 1.3
of this license or (at your option) any later version.
The latest version of this license is in
  \url{http://www.latex-project.org/lppl.txt}
and version 1.3 or later is part of all distributions of \LaTeX{}
version 2005/12/01 or later.

This work has the LPPL maintenance status `maintained'.

The Current Maintainer of this work is Niklas Beisert.

This work consists of the files |README.txt|, |childdoc.ins| and |childdoc.dtx|
as well as the derived files |childdoc.def|, |cdocsamp.tex|
with |cdocsch1.tex|, |cdocsch2.tex|, |cdocspt3.tex|, |cdocspt4.tex|,
|cdocsdrf.tex|, |cdocsfn1.tex|, |cdocsfn2.tex|
as well as |childdoc.pdf|.

%%%%%%%%%%%%%%%%%%%%%%%%%%%%%%%%%%%%%%%%%%%%%%%%%%%%%%%%%%%%%%%%%%%%%%%%%%%%%%%%
\subsection{Files and Installation}

The package consists of the files:
%
\begin{center}
\begin{tabular}{ll}
    |README.txt|   & readme file \\
    |childdoc.ins| & installation file \\
    |childdoc.dtx| & source file \\
    |childdoc.def| & definition file \\
    |cdocsamp.tex| & sample main file \\
    |cdocsch1.tex| & sample include file \\
    |cdocsch2.tex| & sample include file \\
    |cdocspt3.tex| & sample part file \\
    |cdocspt4.tex| & sample part file \\
    |cdocsdrf.tex| & sample redirection file \\
    |cdocsfn1.tex| & sample redirection file \\
    |cdocsfn2.tex| & sample redirection file \\
    |childdoc.pdf| & manual
\end{tabular}
\end{center}
%
The distribution consists of the files
|README.txt|, |childdoc.ins| and |childdoc.dtx|.
%
\begin{itemize}
\item
Run (pdf)\LaTeX{} on |childdoc.dtx|
to compile the manual |childdoc.pdf| (this file).
\item
Run \LaTeX{} on |childdoc.ins| to create the definitions file |childdoc.def|
and the sample |cdocsamp.tex| with include files
|cdocsch1.tex|, |cdocsch2.tex|, |cdocspt3.tex|, |cdocspt4.tex|,
|cdocsdrf.tex|, |cdocsfn1.tex|, |cdocsfn2.tex|.
Then copy the file |childdoc.def| to an appropriate directory of your \LaTeX{}
distribution, e.g.\ \textit{texmf-root}|/tex/latex/childdoc|.
\end{itemize}

%%%%%%%%%%%%%%%%%%%%%%%%%%%%%%%%%%%%%%%%%%%%%%%%%%%%%%%%%%%%%%%%%%%%%%%%%%%%%%%%
\subsection{Related CTAN Packages}

There are several other packages which offer a similar functionality:
%
\begin{itemize}
\item
The packages
\href{http://ctan.org/pkg/docmute}{\textsf{docmute}},
\href{http://ctan.org/pkg/includex}{\textsf{includex}} and
\href{http://ctan.org/pkg/standalone}{\textsf{standalone}}
provide commands to include only the document body of
a child file thus allowing both files to be compiled individually.
\item
The packages \href{http://ctan.org/pkg/subdocs}{\textsf{subdocs}}
and \href{http://ctan.org/pkg/subfiles}{\textsf{subfiles}}
provide structures in which the main and child documents can be
encapsulated and allowing them to be compiled individually.
The inclusion mechanism is different from the conventional |\include|.
\item
The package \href{http://ctan.org/pkg/combine}{\textsf{combine}}
is an elaborate solution to combine several documents into one.
\end{itemize}
%
See also the CTAN topic \href{http://ctan.org/topic/subdocs}{\textsf{subdocs}}
for further related packages.
The present package differs from the above solutions in that
a document structure constructed with the conventional |\include| mechanism
just needs two extra commands at the top of every file
such that all constituent files can be compiled individually.

%%%%%%%%%%%%%%%%%%%%%%%%%%%%%%%%%%%%%%%%%%%%%%%%%%%%%%%%%%%%%%%%%%%%%%%%%%%%%%%%
%\subsection{Feature Suggestions}
%
%The following is a list of features which may be useful for future
%versions of this package:
%%
%\begin{itemize}
%\item
%\ldots
%\end{itemize}

%%%%%%%%%%%%%%%%%%%%%%%%%%%%%%%%%%%%%%%%%%%%%%%%%%%%%%%%%%%%%%%%%%%%%%%%%%%%%%%%
\subsection{Revision History}

%%%%%%%%%%%%%%%%%%%%%%%%%%%%%%%%%%%%%%%%
\paragraph{v2.0:} 2018/12/30

\begin{itemize}
\item
immediate forward processing
\item
added |\childdocby| mechanism
\item
manual restructured
\end{itemize}

%%%%%%%%%%%%%%%%%%%%%%%%%%%%%%%%%%%%%%%%
\paragraph{v1.6:} 2018/01/17

\begin{itemize}
\item
application for development of include files
\item
corrections to manual
\end{itemize}

%%%%%%%%%%%%%%%%%%%%%%%%%%%%%%%%%%%%%%%%
\paragraph{v1.5:} 2017/05/21

\begin{itemize}
\item
more complete structuring introduced
\item
|\childdocof| introduced
\item
|\childdoc| renamed to |\childdocmain|
\item
|\childredirect| renamed to |\childdocforward| and |\childdocforwardprefix|
and functionality expanded
\end{itemize}

%%%%%%%%%%%%%%%%%%%%%%%%%%%%%%%%%%%%%%%%
\paragraph{v1.0:} 2017/04/27

\begin{itemize}
\item
manual and install package
\item
first version published on CTAN
\end{itemize}

%%%%%%%%%%%%%%%%%%%%%%%%%%%%%%%%%%%%%%%%
\paragraph{v0.6:} 2017/04/26

\begin{itemize}
\item
redirection mechanism added
\end{itemize}

%%%%%%%%%%%%%%%%%%%%%%%%%%%%%%%%%%%%%%%%
\paragraph{v0.5:} 2017/04/26

\begin{itemize}
\item
functionality in definition file
\end{itemize}


%%%%%%%%%%%%%%%%%%%%%%%%%%%%%%%%%%%%%%%%%%%%%%%%%%%%%%%%%%%%%%%%%%%%%%%%%%%%%%%%
%%%%%%%%%%%%%%%%%%%%%%%%%%%%%%%%%%%%%%%%%%%%%%%%%%%%%%%%%%%%%%%%%%%%%%%%%%%%%%%%
%%%%%%%%%%%%%%%%%%%%%%%%%%%%%%%%%%%%%%%%%%%%%%%%%%%%%%%%%%%%%%%%%%%%%%%%%%%%%%%%
\appendix

\settowidth\MacroIndent{\rmfamily\scriptsize 000\ }

 \DocInput{childdoc.dtx}

\end{document}
%</driver>
% \fi
%
% %%%%%%%%%%%%%%%%%%%%%%%%%%%%%%%%%%%%%%%%%%%%%%%%%%%%%%%%%%%%%%%%%%%%%%%%%%%%%%
% %%%%%%%%%%%%%%%%%%%%%%%%%%%%%%%%%%%%%%%%%%%%%%%%%%%%%%%%%%%%%%%%%%%%%%%%%%%%%%
% \section{Sample}
%\iffalse
%<*samplemain>
%\fi
%
% The following presents a sample document
% with two chapters, two parts, a title page,
% a compile flag as well as three forwarding files to set the flag.
% It consists of eight |.tex| files:
% \begin{center}
% \begin{tabular}{ll}
% |cdocsamp.tex|&main file\\
% |cdocsch1.tex|&include file for chapter 1\\
% |cdocsch2.tex|&include file for chapter 2\\
% |cdocspt3.tex|&include file for part 3\\
% |cdocspt4.tex|&include file for part 4\\
% |cdocsdrf.tex|&forwarding file for main file in draft mode\\
% |cdocsfi1.tex|&forwarding file for final version of chapter 1\\
% |cdocsfi2.tex|&forwarding file for final version of chapter 2\\
% \end{tabular}
% \end{center}
% Each of the eight files can be compiled directly by the \LaTeX{} compiler.
%
% %%%%%%%%%%%%%%%%%%%%%%%%%%%%%%%%%%%%%%
% \paragraph{Main File.}
%
% The main file is called |cdocsamp.tex|.
%
% Load the \textsf{childdoc} definitions and
% declare the filename for the main document:
%    \begin{macrocode}
\input{childdoc.def}
\childdocmain{}
%    \end{macrocode}

% Optional override for |\version| flag:
%    \begin{macrocode}
%%\ifchilddoc\else\providecommand{\version}{draft}\fi
%    \end{macrocode}

% Define the default values for the |\version| flag
% (|final| for the main file and |draft| for childs):
%    \begin{macrocode}
\ifchilddoc
\providecommand{\version}{draft}
\else
\providecommand{\version}{final}
\fi
%    \end{macrocode}

% Load the standard document class:
%    \begin{macrocode}
\documentclass[12pt]{article}
%    \end{macrocode}

% Start the document body:
%    \begin{macrocode}
\begin{document}
%    \end{macrocode}

% Declare a title page.
% Print title, part of document being processed and version flag:
%    \begin{macrocode}
\addtocounter{page}{-1}
\begin{center}
{\LARGE\bfseries{}childdoc example\par}
\vspace{1cm}
\ifchilddoc
\ifchilddocmanual part\else chapter\fi:
`\childdocname' of `\childdocjob'\par
\else
main document: `\childdocjob'\par
\fi
version: \version\par
\end{center}
\newpage
%    \end{macrocode}

% Manually include selected file,
% otherwise process as usual:
%    \begin{macrocode}
\ifchilddocmanual
\section*{part `\childdocname'}
\input{\childdocname}
\else
%    \end{macrocode}

% Include the two chapters:
%    \begin{macrocode}
\include{cdocsch1}
\include{cdocsch2}
%    \end{macrocode}

% Include the two parts unless only chapters should be displayed:
%    \begin{macrocode}
\ifchilddoc\else
\section{part three}
\input{cdocspt3}
\section{part four}
\input{cdocspt4}
\fi
%    \end{macrocode}

% Process as usual until here:
%    \begin{macrocode}
\fi
%    \end{macrocode}

% End of document body:
%    \begin{macrocode}
\end{document}
%    \end{macrocode}
%\iffalse
%</samplemain>
%\fi
%
% %%%%%%%%%%%%%%%%%%%%%%%%%%%%%%%%%%%%%%
% \paragraph{Chapter Include Files.}
%
% The include files are called |cdocsch1.tex| and |cdocsch2.tex|.
%
%\iffalse
%<*samplechap1|samplechap2>
%\fi

% Optional override for |\version| flag:
%    \begin{macrocode}
%%\providecommand{\version}{final}
%    \end{macrocode}

% Include the main document:
%    \begin{macrocode}
\input{childdoc.def}
\childdocof{cdocsamp}
%    \end{macrocode}

%\iffalse
%</samplechap1|samplechap2>
%\fi
%
%\iffalse
%<*samplechap1>
%\fi
% Some text for chapter 1:
%    \begin{macrocode}
\section{one}
some text in chapter one
%    \end{macrocode}

%\iffalse
%</samplechap1>
%\fi
% Some text for chapter 2:
%\iffalse
%<*samplechap2>
%\fi
%    \begin{macrocode}
\section{two}
more text in chapter two
%    \end{macrocode}

%\iffalse
%</samplechap2>
%\fi
%
% %%%%%%%%%%%%%%%%%%%%%%%%%%%%%%%%%%%%%%
% \paragraph{Part Include Files.}
%
% The include files are called |cdocspt3.tex| and |cdocspt4.tex|.
%
%\iffalse
%<*samplepart3|samplepart4>
%\fi

% Optional override for |\version| flag:
%    \begin{macrocode}
%%\providecommand{\version}{final}
%    \end{macrocode}

% Include the main document:
%    \begin{macrocode}
\input{childdoc.def}
\childdocby{cdocsamp}
%    \end{macrocode}

%\iffalse
%</samplepart3|samplepart4>
%\fi
%
%\iffalse
%<*samplepart3>
%\fi
% Some text for part 3:
%    \begin{macrocode}
some text in part three
%    \end{macrocode}

%\iffalse
%</samplepart3>
%\fi
% Some text for part 4:
%\iffalse
%<*samplepart4>
%\fi
%    \begin{macrocode}
more text in part four
%    \end{macrocode}

%\iffalse
%</samplepart4>
%\fi
%
% %%%%%%%%%%%%%%%%%%%%%%%%%%%%%%%%%%%%%%
% \paragraph{Forwarding for a Complete Draft.}
%
% The following forwarding file |cdocsdrf.tex|
% compiles the main document in draft mode:
%\iffalse
%<*sampledraft>
%\fi
%    \begin{macrocode}
\def\version{draft}
\input{childdoc.def}
\childdocforward{cdocsamp}
%    \end{macrocode}

%\iffalse
%</sampledraft>
%\fi
%
% %%%%%%%%%%%%%%%%%%%%%%%%%%%%%%%%%%%%%%
% \paragraph{Forwarding for Final Version of the Chapters.}
%
% The following forwarding files |cdocsfn1.tex| and |cdocsfn2.tex|
% (with identical content)
% compile the final versions of the child documents
% |cdocsch1.tex| and |cdocsch2.tex|, respectively:
%\iffalse
%<*samplefinal>
%\fi
%    \begin{macrocode}
\def\version{final}
\input{childdoc.def}
\childdocforwardprefix[cdocsamp]{cdocsfn}{cdocsch}
%    \end{macrocode}

%\iffalse
%</samplefinal>
%\fi
%
% %%%%%%%%%%%%%%%%%%%%%%%%%%%%%%%%%%%%%%
% \paragraph{Command Line Processing.}
%
% The following three command lines generate the output files
% |cdocscld|, |cdocscl1| and |cdocscl2|
% which should be identical to
% |cdocsdrf|, |cdocsch1| and |cdocsfn2|, respectively:
% \begin{center}
% \begin{tabular}{l}
% |latex -jobname cdocscld \|\\
% |  "\def\version{draft}\input{childdoc.def}\childdocforward{cdocsamp}"|\\
% |latex -jobname cdocscl1 \|\\
% |  "\input{childdoc.def}\childdocforward[cdocsamp]{cdocsch1}"|\\
% |latex -jobname cdocscl2 \|\\
% |  "\def\version{final}\input{childdoc.def}\childdocforward{cdocsch2}"|
% \end{tabular}
% \end{center}
% Note that the trailing backslash on each first line
% merely continues the input to the second line
% (for convenient cut ant paste).
% Furthermore, the command |latex| can be replaced by any
% of its alternative versions such as |pdflatex|.
%
% %%%%%%%%%%%%%%%%%%%%%%%%%%%%%%%%%%%%%%%%%%%%%%%%%%%%%%%%%%%%%%%%%%%%%%%%%%%%%%
% %%%%%%%%%%%%%%%%%%%%%%%%%%%%%%%%%%%%%%%%%%%%%%%%%%%%%%%%%%%%%%%%%%%%%%%%%%%%%%
% \section{Implementation}
%\iffalse
%<*package>
%\fi
%
% This section describes the definitions file |childdoc.def|.

% The definitions cannot be loaded using |\usepackage| or |\RequirePackage|
% which has a mechanism to prevent loading a style file more than once.
% When loading the definitions by means of |\input|
% multiple instances have to be prevented manually:
%\iffalse
%This code needs to be before the `\ProvidesFile' directive
%which is defined at the beginning of this file.
%Therefore it is also placed there and commented out here.
%</package>
%<*discard>
%\fi
%    \begin{macrocode}
\ifdefined\childdocmain\endinput\fi
%    \end{macrocode}
%\iffalse
%</discard>
%<*package>
%\fi
%
% \macro{\ifchilddoc}
% \macro{\ifchilddocmanual}
% The conditional |\ifchilddoc| tells whether a
% child (true) or main (false) document is being compiled.
% The conditional |\ifchilddocmanual| tells whether
% the |\includeonly| mechanism is used (false) or
% the selection of child files must be performed manually (true).
% The definitions initialise to false:
%    \begin{macrocode}
\newif\ifchilddoc
\newif\ifchilddocmanual
%    \end{macrocode}

% \macro{\childdocname}
% \macro{\childdocjob}
% The macro |\childdocname| stores the name of the main document
% to be compiled. The macro |\childdocjob| stores the name of
% the document on which the \LaTeX{} compiler was originally invoked.
% The content of |\jobname| cannot be compared
% to filenames specified in the source due to different catcodes.
% The following code rescans |\jobname|, stores the result
% in |\childdocname| and saves a copy in |\childdocjob|:
%    \begin{macrocode}
\edef\childdocname{\scantokens\expandafter{\jobname\noexpand}}
\let\childdocjob\childdocname
%    \end{macrocode}

% \macro{\childdocdisable}
% The macro |\childdocdisable| prevents the main file
% from being processed more than once.
% At this stage, the main document command |\childdocmain|
% is assumed to be called once again where it should do nothing.
% Any subsequent call to it should prevent
% a secondary processing of the main document
% It overwrites the forwarding commands
% |\childdocof| and |\childdocforward|
% with empty macros to prevent further inclusions of the main document:
%    \begin{macrocode}
\newcommand{\childdocdisable}
{
  \renewcommand{\childdocmain}[1]{\renewcommand{\childdocmain}[1]{\endinput}}
  \renewcommand{\childdocof}[1]{}
  \renewcommand{\childdocby}[2][]{}
  \renewcommand{\childdocforward}[2][]{}
  \renewcommand{\childdocdisable}{}
}
%    \end{macrocode}

% \macro{\childdocmain}
% The macro |\childdocmain| is to be called at the top of the main file
% with nothing or the main filename (without extension) as argument.
% First, it breaks loops.
% If the argument is not empty and does not match |\childdocname|
% (which is set by the first inclusion of |childdoc.def|),
% |\ifchilddoc| is set to true, |\includeonly| is applied to the child file
% and |\jobname| is set to the main file
% (for proper handling of |.aux| files):
%    \begin{macrocode}
\newcommand{\childdocmain}[1]
{
  \childdocdisable\childdocmain{}
  \if?#1?\else
    \begingroup
      \def\childdoctmp{#1}
      \ifx\childdoctmp\childdocname
        \def\childdoctmp{}
      \else
        \def\childdoctmp
        {
          \childdoctrue
          \includeonly{\childdocname}
          \def\childdocjob{#1}
          \def\jobname{#1}
        }
      \fi
      \expandafter
    \endgroup
    \childdoctmp
  \fi
}
%    \end{macrocode}

% \macro{\childdocof}
% The command |\childdocof| redirects
% compilation to the main file |#1|.
%    \begin{macrocode}
\newcommand{\childdocof}[1]
{
  \childdocdisable
  \childdoctrue
  \includeonly{\childdocname}
  \def\jobname{#1}
  \def\childdocjob{#1}
  \input{#1}
}
%    \end{macrocode}

% \macro{\childdocby}
% The command |\childdocby| ....
%    \begin{macrocode}
\newcommand{\childdocby}[2][]
{
  \childdocdisable
  \childdoctrue
  \childdocmanualtrue
  \if?#1?\else
    \def\jobname{#2}
  \fi
  \def\childdocjob{#2}
  \input{#2}
  \endinput
}
%    \end{macrocode}

% \macro{\childdocforward}
% The command |\childdocforward| redirects
% compilation to the main file or
% (if the optional argument is given) a child file.
% Parameters are set as if the main file
% or a child file starting with |\childdocof| was compiled.
% Then compilation is handed over to the main file:
%    \begin{macrocode}
\newcommand{\childdocforward}[2][]
{
  \begingroup
    \if?#1?
      \def\childdoctmp
      {
        \def\childdocname{#2}
        \def\childdocjob{#2}
        \def\jobname{#2}
        \input{#2}
        \endinput
      }
    \else
      \def\childdoctmp
      {
        \childdocdisable
        \def\childdocname{#2}
        \childdoctrue
        \includeonly{#2}
        \def\childdocjob{#1}
        \def\jobname{#1}
        \input{#1}
        \endinput
      }
    \fi
    \expandafter
  \endgroup
  \childdoctmp
}
%    \end{macrocode}

% \macro{\childdocforwardprefix}
% The command |\childdocforwardprefix| redirects
% compilation to the main or a child file by means of a pattern.
% The prefix |#1| in the current filename is replaced by |#2|
% and the suffix of the current filename is kept
% (it is assumed that the filename does not contain the substring `|~~~|'
% which is used as a delimiter).
% Compilation is handed over to the new file by |\childdocforward|:
%    \begin{macrocode}
\newcommand{\childdocforwardprefix}[3][]
{
  \begingroup
    \def\childdocextract #2##1~~~{\def\childdoctmp{\childdocforward[#1]{#3##1}}}
    \expandafter\childdocextract\childdocname~~~
    \expandafter
  \endgroup
  \childdoctmp
}
%    \end{macrocode}

% \macro{\childdoc}
% The deprecated macro |\childdoc| is a legacy version of |\childdocmain|:
%    \begin{macrocode}
\newcommand{\childdoc}{\childdocmain}
%    \end{macrocode}

% \macro{\childdocredirect}
% The deprecated macro |\childdocredirect| is a legacy version
% of |\childdocforward| and |\childdocforwardprefix|:
%    \begin{macrocode}
\newcommand{\childdocredirect}[2][]
{
  \begingroup
    \if?#1?
      \def\childdoctmp{\childdocforward{#2}}
    \else
      \def\childdoctmp{\childdocforwardprefix{#1}{#2}}
    \fi
    \expandafter
  \endgroup
  \childdoctmp
}
%    \end{macrocode}

%\iffalse
%</package>
%\fi
%
\endinput
\childdocforward{cdocsch2}"|
% \end{tabular}
% \end{center}
% Note that the trailing backslash on each first line
% merely continues the input to the second line
% (for convenient cut ant paste).
% Furthermore, the command |latex| can be replaced by any
% of its alternative versions such as |pdflatex|.
%
% %%%%%%%%%%%%%%%%%%%%%%%%%%%%%%%%%%%%%%%%%%%%%%%%%%%%%%%%%%%%%%%%%%%%%%%%%%%%%%
% %%%%%%%%%%%%%%%%%%%%%%%%%%%%%%%%%%%%%%%%%%%%%%%%%%%%%%%%%%%%%%%%%%%%%%%%%%%%%%
% \section{Implementation}
%\iffalse
%<*package>
%\fi
%
% This section describes the definitions file |childdoc.def|.

% The definitions cannot be loaded using |\usepackage| or |\RequirePackage|
% which has a mechanism to prevent loading a style file more than once.
% When loading the definitions by means of |\input|
% multiple instances have to be prevented manually:
%\iffalse
%This code needs to be before the `\ProvidesFile' directive
%which is defined at the beginning of this file.
%Therefore it is also placed there and commented out here.
%</package>
%<*discard>
%\fi
%    \begin{macrocode}
\ifdefined\childdocmain\endinput\fi
%    \end{macrocode}
%\iffalse
%</discard>
%<*package>
%\fi
%
% \macro{\ifchilddoc}
% \macro{\ifchilddocmanual}
% The conditional |\ifchilddoc| tells whether a
% child (true) or main (false) document is being compiled.
% The conditional |\ifchilddocmanual| tells whether
% the |\includeonly| mechanism is used (false) or
% the selection of child files must be performed manually (true).
% The definitions initialise to false:
%    \begin{macrocode}
\newif\ifchilddoc
\newif\ifchilddocmanual
%    \end{macrocode}

% \macro{\childdocname}
% \macro{\childdocjob}
% The macro |\childdocname| stores the name of the main document
% to be compiled. The macro |\childdocjob| stores the name of
% the document on which the \LaTeX{} compiler was originally invoked.
% The content of |\jobname| cannot be compared
% to filenames specified in the source due to different catcodes.
% The following code rescans |\jobname|, stores the result
% in |\childdocname| and saves a copy in |\childdocjob|:
%    \begin{macrocode}
\edef\childdocname{\scantokens\expandafter{\jobname\noexpand}}
\let\childdocjob\childdocname
%    \end{macrocode}

% \macro{\childdocdisable}
% The macro |\childdocdisable| prevents the main file
% from being processed more than once.
% At this stage, the main document command |\childdocmain|
% is assumed to be called once again where it should do nothing.
% Any subsequent call to it should prevent
% a secondary processing of the main document
% It overwrites the forwarding commands
% |\childdocof| and |\childdocforward|
% with empty macros to prevent further inclusions of the main document:
%    \begin{macrocode}
\newcommand{\childdocdisable}
{
  \renewcommand{\childdocmain}[1]{\renewcommand{\childdocmain}[1]{\endinput}}
  \renewcommand{\childdocof}[1]{}
  \renewcommand{\childdocby}[2][]{}
  \renewcommand{\childdocforward}[2][]{}
  \renewcommand{\childdocdisable}{}
}
%    \end{macrocode}

% \macro{\childdocmain}
% The macro |\childdocmain| is to be called at the top of the main file
% with nothing or the main filename (without extension) as argument.
% First, it breaks loops.
% If the argument is not empty and does not match |\childdocname|
% (which is set by the first inclusion of |childdoc.def|),
% |\ifchilddoc| is set to true, |\includeonly| is applied to the child file
% and |\jobname| is set to the main file
% (for proper handling of |.aux| files):
%    \begin{macrocode}
\newcommand{\childdocmain}[1]
{
  \childdocdisable\childdocmain{}
  \if?#1?\else
    \begingroup
      \def\childdoctmp{#1}
      \ifx\childdoctmp\childdocname
        \def\childdoctmp{}
      \else
        \def\childdoctmp
        {
          \childdoctrue
          \includeonly{\childdocname}
          \def\childdocjob{#1}
          \def\jobname{#1}
        }
      \fi
      \expandafter
    \endgroup
    \childdoctmp
  \fi
}
%    \end{macrocode}

% \macro{\childdocof}
% The command |\childdocof| redirects
% compilation to the main file |#1|.
%    \begin{macrocode}
\newcommand{\childdocof}[1]
{
  \childdocdisable
  \childdoctrue
  \includeonly{\childdocname}
  \def\jobname{#1}
  \def\childdocjob{#1}
  \input{#1}
}
%    \end{macrocode}

% \macro{\childdocby}
% The command |\childdocby| ....
%    \begin{macrocode}
\newcommand{\childdocby}[2][]
{
  \childdocdisable
  \childdoctrue
  \childdocmanualtrue
  \if?#1?\else
    \def\jobname{#2}
  \fi
  \def\childdocjob{#2}
  \input{#2}
  \endinput
}
%    \end{macrocode}

% \macro{\childdocforward}
% The command |\childdocforward| redirects
% compilation to the main file or
% (if the optional argument is given) a child file.
% Parameters are set as if the main file
% or a child file starting with |\childdocof| was compiled.
% Then compilation is handed over to the main file:
%    \begin{macrocode}
\newcommand{\childdocforward}[2][]
{
  \begingroup
    \if?#1?
      \def\childdoctmp
      {
        \def\childdocname{#2}
        \def\childdocjob{#2}
        \def\jobname{#2}
        \input{#2}
        \endinput
      }
    \else
      \def\childdoctmp
      {
        \childdocdisable
        \def\childdocname{#2}
        \childdoctrue
        \includeonly{#2}
        \def\childdocjob{#1}
        \def\jobname{#1}
        \input{#1}
        \endinput
      }
    \fi
    \expandafter
  \endgroup
  \childdoctmp
}
%    \end{macrocode}

% \macro{\childdocforwardprefix}
% The command |\childdocforwardprefix| redirects
% compilation to the main or a child file by means of a pattern.
% The prefix |#1| in the current filename is replaced by |#2|
% and the suffix of the current filename is kept
% (it is assumed that the filename does not contain the substring `|~~~|'
% which is used as a delimiter).
% Compilation is handed over to the new file by |\childdocforward|:
%    \begin{macrocode}
\newcommand{\childdocforwardprefix}[3][]
{
  \begingroup
    \def\childdocextract #2##1~~~{\def\childdoctmp{\childdocforward[#1]{#3##1}}}
    \expandafter\childdocextract\childdocname~~~
    \expandafter
  \endgroup
  \childdoctmp
}
%    \end{macrocode}

% \macro{\childdoc}
% The deprecated macro |\childdoc| is a legacy version of |\childdocmain|:
%    \begin{macrocode}
\newcommand{\childdoc}{\childdocmain}
%    \end{macrocode}

% \macro{\childdocredirect}
% The deprecated macro |\childdocredirect| is a legacy version
% of |\childdocforward| and |\childdocforwardprefix|:
%    \begin{macrocode}
\newcommand{\childdocredirect}[2][]
{
  \begingroup
    \if?#1?
      \def\childdoctmp{\childdocforward{#2}}
    \else
      \def\childdoctmp{\childdocforwardprefix{#1}{#2}}
    \fi
    \expandafter
  \endgroup
  \childdoctmp
}
%    \end{macrocode}

%\iffalse
%</package>
%\fi
%
\endinput

\childdocof{cdocsamp}
%    \end{macrocode}

%\iffalse
%</samplechap1|samplechap2>
%\fi
%
%\iffalse
%<*samplechap1>
%\fi
% Some text for chapter 1:
%    \begin{macrocode}
\section{one}
some text in chapter one
%    \end{macrocode}

%\iffalse
%</samplechap1>
%\fi
% Some text for chapter 2:
%\iffalse
%<*samplechap2>
%\fi
%    \begin{macrocode}
\section{two}
more text in chapter two
%    \end{macrocode}

%\iffalse
%</samplechap2>
%\fi
%
% %%%%%%%%%%%%%%%%%%%%%%%%%%%%%%%%%%%%%%
% \paragraph{Part Include Files.}
%
% The include files are called |cdocspt3.tex| and |cdocspt4.tex|.
%
%\iffalse
%<*samplepart3|samplepart4>
%\fi

% Optional override for |\version| flag:
%    \begin{macrocode}
%%\providecommand{\version}{final}
%    \end{macrocode}

% Include the main document:
%    \begin{macrocode}
% \iffalse
%
% childdoc.dtx Copyright (C) 2017-2018 Niklas Beisert
%
% This work may be distributed and/or modified under the
% conditions of the LaTeX Project Public License, either version 1.3
% of this license or (at your option) any later version.
% The latest version of this license is in
%   http://www.latex-project.org/lppl.txt
% and version 1.3 or later is part of all distributions of LaTeX
% version 2005/12/01 or later.
%
% This work has the LPPL maintenance status `maintained'.
%
% The Current Maintainer of this work is Niklas Beisert.
%
% This work consists of the files childdoc.dtx and childdoc.ins
% and the derived files childdoc.def and cdocsamp.tex with
% cdocsch1.tex, cdocsch2.tex, cdocsdrf.tex, cdocsfn1.tex, cdocsfn2.tex.
%
%<package>\ifdefined\childdocmain\endinput\fi
%<package>\ProvidesFile{childdoc.def}[2018/12/30 v2.0 child document driver]
%<samplemain>\ProvidesFile{cdocsamp.tex}[2018/12/30 v2.0 sample for childdoc]
%<*driver>
%\ProvidesFile{childdoc.drv}[2018/12/30 v2.0 childdoc reference manual file]
\PassOptionsToClass{10pt,a4paper}{article}
\documentclass{ltxdoc}

\usepackage[margin=35mm]{geometry}
\usepackage{hyperref}
\usepackage{hyperxmp}
\usepackage[usenames]{color}

\hypersetup{colorlinks=true}
\hypersetup{pdfstartview=FitH}
\hypersetup{pdfpagemode=UseNone}
\hypersetup{pdfsource={}}
\hypersetup{pdflang={en-UK}}
\hypersetup{pdfcopyright={Copyright 2017-2018 Niklas Beisert.
  This work may be distributed and/or modified under the
  conditions of the LaTeX Project Public License, either version 1.3
  of this license or (at your option) any later version.}}
\hypersetup{pdflicenseurl={http://www.latex-project.org/lppl.txt}}
\hypersetup{pdfcontactaddress={ETH Zurich, ITP, HIT K,
  Wolfgang-Pauli-Strasse 27}}
\hypersetup{pdfcontactpostcode={8093}}
\hypersetup{pdfcontactcity={Zurich}}
\hypersetup{pdfcontactcountry={Switzerland}}
\hypersetup{pdfcontactemail={nbeisert@itp.phys.ethz.ch}}
\hypersetup{pdfcontacturl={http://people.phys.ethz.ch/\xmptilde nbeisert/}}

\newcommand{\secref}[1]{\hyperref[#1]{section \ref*{#1}}}

\parskip1ex
\parindent0pt
\let\olditemize\itemize
\def\itemize{\olditemize\parskip0pt}

\begin{document}

\title{The \textsf{childdoc} Package}
\hypersetup{pdftitle={The childdoc Package}}
\author{Niklas Beisert\\[2ex]
  Institut f\"ur Theoretische Physik\\
  Eidgen\"ossische Technische Hochschule Z\"urich\\
  Wolfgang-Pauli-Strasse 27, 8093 Z\"urich, Switzerland\\[1ex]
  \href{mailto:nbeisert@itp.phys.ethz.ch}
  {\texttt{nbeisert@itp.phys.ethz.ch}}}
\hypersetup{pdfauthor={Niklas Beisert}}
\hypersetup{pdfsubject={Manual for the LaTeX2e Package childdoc}}
\date{30 December 2018, \textsf{v2.0}}
\maketitle

\begin{abstract}\noindent
\textsf{childdoc} is a \LaTeXe{} package
that enables the direct compilation
of document sections included by |\include|
to individual files.
\end{abstract}

\begingroup
\parskip0ex
\tableofcontents
\endgroup

%%%%%%%%%%%%%%%%%%%%%%%%%%%%%%%%%%%%%%%%%%%%%%%%%%%%%%%%%%%%%%%%%%%%%%%%%%%%%%%%
%%%%%%%%%%%%%%%%%%%%%%%%%%%%%%%%%%%%%%%%%%%%%%%%%%%%%%%%%%%%%%%%%%%%%%%%%%%%%%%%
\section{Introduction}

\LaTeX{} provides a mechanism to structure a large document (such as a book)
into a main file and several child files (containing the chapters)
using the |\include| command.
This mechanism is beneficial for documents
which span hundreds of pages in order to
make the source file(s) more manageable.
Moreover, compilation can be restricted to
selected child files by means of the |\includeonly| command.
The latter feature can be used to reduce the compilation time while editing
(this was significantly more useful in the earlier days of \LaTeX{})
or to generate a smaller document which is easier to navigate.
Another application of |\includeonly| is to generate
documents consisting of selected parts of the complete document.

However, there are a few drawbacks of the plain |\include| mechanism:
\begin{itemize}
\item
The child files cannot be compiled on their own,
they can only be compiled via the main file.
A naive editing environment
(such as a text editor with an option
to have the current file processed by \LaTeX)
may require one to switch to the main file before compiling;
attempting to compile the child file produces errors.
\item
The main file must be modified (each time)
to adjust the |\includeonly| command
to the present needs. This easily leaves the main file in a messy state.
\item
The generated document will always carry the filename
of the main document. This is inconvenient if
several child files are to be compiled and
to be kept for distribution.
\end{itemize}

The present package provides a simple interface
to make child files individually compilable by \LaTeX{}.
Compiling a child file then has the same effect as compiling
the main file with an |\includeonly| command
to select the appropriate child.
Moreover the generated document will carry the name of the child
rather than the main file.
This resolves all three above issues.

This feature is meant to make the editing of books,
thesis documents and lecture notes somewhat more convenient.
However, the package can also be used efficiently for
composing a series of documents (such as exercise sheets)
which are typically distributed individually.
It then assists the author in generating the individual documents
(potentially in different versions)
as well as a document containing the collected series.
Another application is in developing style files
or other kinds of included material
where compilation of the style file could redirect
to a sample or test file.

%%%%%%%%%%%%%%%%%%%%%%%%%%%%%%%%%%%%%%%%%%%%%%%%%%%%%%%%%%%%%%%%%%%%%%%%%%%%%%%%
%%%%%%%%%%%%%%%%%%%%%%%%%%%%%%%%%%%%%%%%%%%%%%%%%%%%%%%%%%%%%%%%%%%%%%%%%%%%%%%%
\section{Usage}

First of all, the package \textsf{childdoc} is \emph{not} a standard
\LaTeXe{} |.sty| style file! Therefore it needs to be invoked in
a non-standard way.

%%%%%%%%%%%%%%%%%%%%%%%%%%%%%%%%%%%%%%%%%%%%%%%%%%%%%%%%%%%%%%%%%%%%%%%%%%%%%%%%
\subsection{Included Files}
\label{sec:include}

%%%%%%%%%%%%%%%%%%%%%%%%%%%%%%%%%%%%%%%%
\DescribeMacro{\childdocmain}
To use the package, add the commands
\begin{center}
\begin{tabular}{l}
|% \iffalse
%
% childdoc.dtx Copyright (C) 2017-2018 Niklas Beisert
%
% This work may be distributed and/or modified under the
% conditions of the LaTeX Project Public License, either version 1.3
% of this license or (at your option) any later version.
% The latest version of this license is in
%   http://www.latex-project.org/lppl.txt
% and version 1.3 or later is part of all distributions of LaTeX
% version 2005/12/01 or later.
%
% This work has the LPPL maintenance status `maintained'.
%
% The Current Maintainer of this work is Niklas Beisert.
%
% This work consists of the files childdoc.dtx and childdoc.ins
% and the derived files childdoc.def and cdocsamp.tex with
% cdocsch1.tex, cdocsch2.tex, cdocsdrf.tex, cdocsfn1.tex, cdocsfn2.tex.
%
%<package>\ifdefined\childdocmain\endinput\fi
%<package>\ProvidesFile{childdoc.def}[2018/12/30 v2.0 child document driver]
%<samplemain>\ProvidesFile{cdocsamp.tex}[2018/12/30 v2.0 sample for childdoc]
%<*driver>
%\ProvidesFile{childdoc.drv}[2018/12/30 v2.0 childdoc reference manual file]
\PassOptionsToClass{10pt,a4paper}{article}
\documentclass{ltxdoc}

\usepackage[margin=35mm]{geometry}
\usepackage{hyperref}
\usepackage{hyperxmp}
\usepackage[usenames]{color}

\hypersetup{colorlinks=true}
\hypersetup{pdfstartview=FitH}
\hypersetup{pdfpagemode=UseNone}
\hypersetup{pdfsource={}}
\hypersetup{pdflang={en-UK}}
\hypersetup{pdfcopyright={Copyright 2017-2018 Niklas Beisert.
  This work may be distributed and/or modified under the
  conditions of the LaTeX Project Public License, either version 1.3
  of this license or (at your option) any later version.}}
\hypersetup{pdflicenseurl={http://www.latex-project.org/lppl.txt}}
\hypersetup{pdfcontactaddress={ETH Zurich, ITP, HIT K,
  Wolfgang-Pauli-Strasse 27}}
\hypersetup{pdfcontactpostcode={8093}}
\hypersetup{pdfcontactcity={Zurich}}
\hypersetup{pdfcontactcountry={Switzerland}}
\hypersetup{pdfcontactemail={nbeisert@itp.phys.ethz.ch}}
\hypersetup{pdfcontacturl={http://people.phys.ethz.ch/\xmptilde nbeisert/}}

\newcommand{\secref}[1]{\hyperref[#1]{section \ref*{#1}}}

\parskip1ex
\parindent0pt
\let\olditemize\itemize
\def\itemize{\olditemize\parskip0pt}

\begin{document}

\title{The \textsf{childdoc} Package}
\hypersetup{pdftitle={The childdoc Package}}
\author{Niklas Beisert\\[2ex]
  Institut f\"ur Theoretische Physik\\
  Eidgen\"ossische Technische Hochschule Z\"urich\\
  Wolfgang-Pauli-Strasse 27, 8093 Z\"urich, Switzerland\\[1ex]
  \href{mailto:nbeisert@itp.phys.ethz.ch}
  {\texttt{nbeisert@itp.phys.ethz.ch}}}
\hypersetup{pdfauthor={Niklas Beisert}}
\hypersetup{pdfsubject={Manual for the LaTeX2e Package childdoc}}
\date{30 December 2018, \textsf{v2.0}}
\maketitle

\begin{abstract}\noindent
\textsf{childdoc} is a \LaTeXe{} package
that enables the direct compilation
of document sections included by |\include|
to individual files.
\end{abstract}

\begingroup
\parskip0ex
\tableofcontents
\endgroup

%%%%%%%%%%%%%%%%%%%%%%%%%%%%%%%%%%%%%%%%%%%%%%%%%%%%%%%%%%%%%%%%%%%%%%%%%%%%%%%%
%%%%%%%%%%%%%%%%%%%%%%%%%%%%%%%%%%%%%%%%%%%%%%%%%%%%%%%%%%%%%%%%%%%%%%%%%%%%%%%%
\section{Introduction}

\LaTeX{} provides a mechanism to structure a large document (such as a book)
into a main file and several child files (containing the chapters)
using the |\include| command.
This mechanism is beneficial for documents
which span hundreds of pages in order to
make the source file(s) more manageable.
Moreover, compilation can be restricted to
selected child files by means of the |\includeonly| command.
The latter feature can be used to reduce the compilation time while editing
(this was significantly more useful in the earlier days of \LaTeX{})
or to generate a smaller document which is easier to navigate.
Another application of |\includeonly| is to generate
documents consisting of selected parts of the complete document.

However, there are a few drawbacks of the plain |\include| mechanism:
\begin{itemize}
\item
The child files cannot be compiled on their own,
they can only be compiled via the main file.
A naive editing environment
(such as a text editor with an option
to have the current file processed by \LaTeX)
may require one to switch to the main file before compiling;
attempting to compile the child file produces errors.
\item
The main file must be modified (each time)
to adjust the |\includeonly| command
to the present needs. This easily leaves the main file in a messy state.
\item
The generated document will always carry the filename
of the main document. This is inconvenient if
several child files are to be compiled and
to be kept for distribution.
\end{itemize}

The present package provides a simple interface
to make child files individually compilable by \LaTeX{}.
Compiling a child file then has the same effect as compiling
the main file with an |\includeonly| command
to select the appropriate child.
Moreover the generated document will carry the name of the child
rather than the main file.
This resolves all three above issues.

This feature is meant to make the editing of books,
thesis documents and lecture notes somewhat more convenient.
However, the package can also be used efficiently for
composing a series of documents (such as exercise sheets)
which are typically distributed individually.
It then assists the author in generating the individual documents
(potentially in different versions)
as well as a document containing the collected series.
Another application is in developing style files
or other kinds of included material
where compilation of the style file could redirect
to a sample or test file.

%%%%%%%%%%%%%%%%%%%%%%%%%%%%%%%%%%%%%%%%%%%%%%%%%%%%%%%%%%%%%%%%%%%%%%%%%%%%%%%%
%%%%%%%%%%%%%%%%%%%%%%%%%%%%%%%%%%%%%%%%%%%%%%%%%%%%%%%%%%%%%%%%%%%%%%%%%%%%%%%%
\section{Usage}

First of all, the package \textsf{childdoc} is \emph{not} a standard
\LaTeXe{} |.sty| style file! Therefore it needs to be invoked in
a non-standard way.

%%%%%%%%%%%%%%%%%%%%%%%%%%%%%%%%%%%%%%%%%%%%%%%%%%%%%%%%%%%%%%%%%%%%%%%%%%%%%%%%
\subsection{Included Files}
\label{sec:include}

%%%%%%%%%%%%%%%%%%%%%%%%%%%%%%%%%%%%%%%%
\DescribeMacro{\childdocmain}
To use the package, add the commands
\begin{center}
\begin{tabular}{l}
|\input{childdoc.def}|\\
|\childdocmain{}|\\
\end{tabular}
\end{center}
at the very top of the main \LaTeX{} file,
in particular \emph{before} the |\documentclass| statement!
The argument of |\childdocmain| should be left empty
(but it must be present).

%%%%%%%%%%%%%%%%%%%%%%%%%%%%%%%%%%%%%%%%
\DescribeMacro{\childdocof}
Furthermore, add the commands
\begin{center}
\begin{tabular}{l}
|\input{childdoc.def}|\\
|\childdocof{|\textit{main}|}|\\
\end{tabular}
\end{center}
at the top of every child file \textit{child}
which is included by |\include{|\textit{child}|}|
from within the main file
(or at least for those files to be compiled individually).
The argument \textit{main} must be the filename of the main file.

There are a couple of
considerations in setting up the main and child documents:

%%%%%%%%%%%%%%%%%%%%%%%%%%%%%%%%%%%%%%%%
\paragraph{Restrictions.}

Please note the following restrictions:
\begin{itemize}
\item
|\childdocmain| must be called with one argument \textit{main}
to ensure compatibility with earlier version of the package.
It must either be empty (|\childdocmain{}|)
or precisely match the filename of the main file in which it is specified.
See \secref{sec:detection} for further information.
\item
The filename \textit{main} must be specified without the |.tex| extension.
\item
The filename \textit{main} is case sensitive
(even in case-insensitive file systems)
due to internal string comparison.
\item
The argument \textit{main} should be fully expanded, it cannot be a macro.
\item
Subdirectories and special characters should be avoided in filenames.
\item
The command |\childdocmain{|\textit{main}|}| must be followed by a whitespace.
It should not be followed immediately by another command
or by a comment mark `|%|'.
This is because the \TeX{} parser reads the token immediately following
the argument of |\childdocmain| and puts it
at the beginning of every child section;
however, a white\-space is ignored.
\end{itemize}

%%%%%%%%%%%%%%%%%%%%%%%%%%%%%%%%%%%%%%%%
\paragraph{Content of Main File.}

It is advisable to place all content in the child files included by |\include|.
Any output contained in the main file will appear in all child documents
unless suppressed manually;
it cannot be suppressed automatically by the |\includeonly| directive
and thus should normally be avoided.
A method to include some content in the main file
by means of conditional processing is described in \secref{sec:conditional}.

%%%%%%%%%%%%%%%%%%%%%%%%%%%%%%%%%%%%%%%%
\paragraph{Page Numbering.}

When only a part of the document is compiled,
the appropriate numbering of pages
(as well as other status parameters)
is determined from the |.aux| files.
The latter contain information from previous passes.
However this information needs to propagate through
all intermediate child documents.
Therefore the page numbering in child documents may well
be inconsistent until the complete document is compiled at least once.

A useful (if unconventional) way to always ensure a consistent
page numbering is to restart the numbering in each child document
and denote the pages by `\textit{child}|.|\textit{page}'
where \textit{child} represents the chapter/section number of the child file.
This can be achieved by the command
|\numberwithin{page}{|\textit{child}|}|
of the \textsf{amsmath} package
where \textit{child} can be |chapter| or |section|
depending on the chosen structuring.
Alternatively, one can modify the macro |\thepage| appropriately
and reset the counter |page| at the start of each child file.

%%%%%%%%%%%%%%%%%%%%%%%%%%%%%%%%%%%%%%%%%%%%%%%%%%%%%%%%%%%%%%%%%%%%%%%%%%%%%%%%
\subsection{Conditional Processing}
\label{sec:conditional}

The package provides a mechanism to compile different versions
of a document. To customise the versions further some conditional processing
can come in handy to distinguish which version is being compiled.
The package provides two macros to describe the compilation context:

%%%%%%%%%%%%%%%%%%%%%%%%%%%%%%%%%%%%%%%%
\DescribeMacro{\ifchilddoc}
The conditional |\ifchilddoc| distinguishes between the compilation of
child documents and the main document:
%
\begin{center}
|\ifchilddoc |\textit{child-code}| |[|\||else |\textit{main-code}]| \||fi|
\end{center}

%%%%%%%%%%%%%%%%%%%%%%%%%%%%%%%%%%%%%%%%
\DescribeMacro{\childdocname}
\DescribeMacro{\childdocjob}
The macro |\childdocname| contains the filename (without extension)
of the main or child file being processed.
Note that |\childdocjob| will always contain the name of the main file.

%%%%%%%%%%%%%%%%%%%%%%%%%%%%%%%%%%%%%%%%
\paragraph{Title Page.}

Conditional processing can be used to include a title or banner page
in the main document when proper precautions are taken.
Importantly, the code in the main file should ensure that the page counter
(as well as other status parameters which are stored in the |.aux| files)
takes the same value after the conditional processing.
Otherwise the page numbers may take divergent values
depending on which part is compiled.

For example, a title page could be declared by:
%
\begin{center}
\begin{tabular}{l}
|\ifchilddoc\||else|\\
|\addtocounter{page}{-1}|\\
\textit{code for title page}\\
|\newpage|\\
|\||fi|
\end{tabular}
\end{center}
%
A banner page for the child documents can be generated by:
%
\begin{center}
\begin{tabular}{l}
|\ifchilddoc|\\
|\addtocounter{page}{-1}|\\
\textit{code for banner page}\\
|\newpage|\\
|\||fi|
\end{tabular}
\end{center}
%
Here one could write a message such as:
\begin{center}
|This is the part \childdocname{} of \childdocjob{}.|
\end{center}

%%%%%%%%%%%%%%%%%%%%%%%%%%%%%%%%%%%%%%%%%%%%%%%%%%%%%%%%%%%%%%%%%%%%%%%%%%%%%%%%
\subsection{Flags}
\label{sec:flags}

The package makes it easy to generate different versions
of the main or child documents.
To this end compilation flags can be defined
and assigned different default values.
They will be particularly useful in conjunction
with the forwarding mechanism described in \secref{sec:forward}.

For example, it may be useful to have a flag |\version|
which can be set to |draft| or |final|.
The document source will contain some conditional code
depending on the value of |\version|.
Suppose further, the flag should default to |final| for the main file
and to |draft| for child files
which is a natural assignment for editing the document.
This is achieved by placing the following code
in the preamble of the main document
(below the |\childdocmain| directive):
%
\begin{center}
\begin{tabular}{l}
|\ifchilddoc|\\
|\providecommand{\version}{draft}|\\
|\||else|\\
|\providecommand{\version}{final}|\\
|\||fi|
\end{tabular}
\end{center}
%
The definition by |\providecommand| makes sure
that previous definitions are not overwritten.
Further statements |\providecommand{\version}{...}|
can thus be added before the above code to override it.

For the main file, one might add a line
(between |\childdocmain| and the above block)
%
\begin{center}
|%\ifchilddoc\||else\providecommand{\version}{draft}\||fi|
\end{center}
%
which can be uncommented to produce a draft version.
Likewise one can add a line to the very top of a child file
(above the |\childdocof{|\textit{main}|}| directive)
%
\begin{center}
|%\providecommand{\version}{final}|
\end{center}
%
which can be uncommented to produce the final version of this child document.

%%%%%%%%%%%%%%%%%%%%%%%%%%%%%%%%%%%%%%%%%%%%%%%%%%%%%%%%%%%%%%%%%%%%%%%%%%%%%%%%
\subsection{Forwarding}
\label{sec:forward}

Different versions of the main or child documents
using compilation flags as described in \secref{sec:flags}
can be (permanently) stored in different files
for convenient compilation, viewing and distribution.
To this end, the package defines a command
to pass on compilation to a different file:

%%%%%%%%%%%%%%%%%%%%%%%%%%%%%%%%%%%%%%%%
\DescribeMacro{\childdocforward}
The command |\childdocforward| redirects processing to
another source file:
%
\begin{center}
\begin{tabular}{l}
|\input{childdoc.def}|\\
|\childdocforward[|\textit{main}|]{|\textit{dest}|}|\\
\end{tabular}
\end{center}
%
The argument \textit{dest} is the destination file
(without extension).
It should be the main file or one of the child files.
Note that further \textsf{childdoc} directives
such as |\childdocof| and |\childdocforward|
in the indicated file will be processed in this form.
The optional argument \textit{main}
passes on directly to the main file \textit{main}
while pretending to compile the child \textit{dest}.
This form behaves as if \textit{dest}
issues |\childdocof{|\textit{main}|}| right away,
and no further \textsf{childdoc} directives will be processed.

%%%%%%%%%%%%%%%%%%%%%%%%%%%%%%%%%%%%%%%%
\DescribeMacro{\...prefix}
In the alternative form |\childdocforwardprefix|,
%
\begin{center}
\begin{tabular}{l}
|\input{childdoc.def}|\\
|\childdocforwardprefix[|\textit{main}|]{|\textit{prefix}|}{|\textit{dest}|}|
\end{tabular}
\end{center}
%
the destination file is determined by a pattern
depending on the current file:
To make this work, the current file must be called
`{\textit{prefix}\hspace{0.2em}\textit{suffix}}'
with \textit{prefix} matching precisely the argument.
Processing is then passed on to the file
`{\textit{dest}\hspace{0.2em}\textit{suffix}}'.
Surely, the same effect is achieved by
directly specifying the
argument `{\textit{dest}\hspace{0.2em}\textit{suffix}}'
in the first form.
However, that requires to set up a different file
for each child. With the alternative form of the command
all these files can have exactly the same content
which simplifies setting them up and maintaining them.

For example, the following file |draft.tex|
with a compilation flag |\version| as described in \secref{sec:flags}
compiles the main document as a draft:
%
\begin{center}
\begin{tabular}{l}
|\def\version{draft}|\\
|\input{childdoc.def}|\\
|\childdocforward{|\textit{main}|}|
\end{tabular}
\end{center}
%
Likewise, the following files |final|\textit{nn}|.tex|
compile the final version of the child document
|child|\textit{nn}|.tex|:
%
\begin{center}
\begin{tabular}{l}
|\def\version{final}|\\
|\input{childdoc.def}|\\
|\childdocforwardprefix{final}{child}|
\end{tabular}
\end{center}
%

Note that when several versions of a main file and/or of each child file
are to be generated, it may be convenient to set up a |Makefile| or
shell script to automatise the process.

%%%%%%%%%%%%%%%%%%%%%%%%%%%%%%%%%%%%%%%%%%%%%%%%%%%%%%%%%%%%%%%%%%%%%%%%%%%%%%%%
\subsection{Command Line Processing}
\label{sec:commandline}

The effect of redirection files can also be achieved by invoking
the \LaTeX{} compiler with a more elaborate command line.
Most conveniently this should be done as part
of a shell script or a |Makefile|.

When using \textsf{childdoc} in the main file, the following
command lines effectively perform a redirection
(note that depending on the shell being used,
backslashes may have to be doubled: `|\|' $\to$ `|\\|'):
%
\begin{center}
|... -jobname "|\textit{target}|" |\\|"|[\textit{flags}]%
|\input{childdoc.def}\childdocforward[|\textit{main}|]{|\textit{dest}|}"|
\end{center}
%
Here \textit{target} is the name of the output file,
\textit{main} is the name of the main file
and \textit{dest} is the name of the main or child file to be processed
(all filenames without extensions).
The optional argument \textit{main} can be omitted
if \textit{main} matches \textit{dest}.
Optionally, compilation \textit{flags} can be defined via |\def| commands.
This command line makes the \TeX{} engine believe
it is compiling the file \textit{target}
whose content is specified as the latter parameter.
The provided code then forwards the processing to
\textit{main} or \textit{dest} as described in \secref{sec:forward}.

%%%%%%%%%%%%%%%%%%%%%%%%%%%%%%%%%%%%%%%%%%%%%%%%%%%%%%%%%%%%%%%%%%%%%%%%%%%%%%%%
\subsection{Include by Input}
\label{sec:input}

Including child documents by |\include| has some restrictions by design.
Most notably, the content of a child document always occupies
its own set of pages; pages cannot be shared between child documents.
Usually, this behaviour makes perfect sense
because each child document contain an essential part of the document.
However, in some situations it may be desirable to compose
a document from a collection of parts
without having mandatory page breaks between then.
For this case, the package
provides a mechanism to include parts
by |\input| which can also be processed individually.
However, by construction this mechanism
requires manual handling of the content to be output.

%%%%%%%%%%%%%%%%%%%%%%%%%%%%%%%%%%%%%%%%
\DescribeMacro{\ifchilddocmanual}
The main file should be prepared as usual, see \secref{sec:include}.
However, the document body must make a distinction
between processing of an individual part and of the main document, e.g.:
%
\begin{center}
\begin{tabular}{l}
|\ifchilddocmanual|\\
|\input{\childdocname}|\\
|\||else|\\
\textit{document body with }|\input{|\textit{part}|}|\\
|\||fi|
\end{tabular}
\end{center}
%
The conditional |\ifchilddocmanual| is true whenever
a part to be included by |\input| is being compiled,
and the name of the part is stored in |\childdocname|.

%%%%%%%%%%%%%%%%%%%%%%%%%%%%%%%%%%%%%%%%
\DescribeMacro{\childdocby}
Each part to be included by |\input| should start with:
%
\begin{center}
\begin{tabular}{l}
|\input{childdoc.def}|\\
|\childdocby{|\textit{main}|}|\\
\end{tabular}
\end{center}
%
The directive |\childdocby| is similar to |\childdocof|
described in \secref{sec:include},
but the subsequent selection of content must be done manually.
To that end, both |\ifchilddoc| and |\ifchilddocmanual|
will be true upon processing of a part,
and the name of the part is stored in |\childdocname|.
Note that |\jobname| will be set to the filename of the current part
so that each part receives an individual |.aux| file
that does not interfere with the |.aux| file(s) of the main document.
This behaviour can be altered by the alternative form
|\childdocby[*]{|\textit{main}|}| (with a non-empty optional argument)
which uses the |.aux| file of the main document
by setting |\jobname| to \textit{main}.

%%%%%%%%%%%%%%%%%%%%%%%%%%%%%%%%%%%%%%%%%%%%%%%%%%%%%%%%%%%%%%%%%%%%%%%%%%%%%%%%
\subsection{Driver Development}
\label{sec:driver}

The \textsf{childdoc} mechanism can also be use for the development
of definition files such as \LaTeX{} styles or classes.
This case differs from the above setup with multiple parts
included by |\include| in that no |\includeonly| should be invoked.
This can be achieved by starting the include file
(before |\ProvidesPackage|) with:
%
\begin{center}
\begin{tabular}{l}
|\input{childdoc.def}|\\
|\childdocforward{|\textit{main}|}|\\
\end{tabular}
\end{center}
%
or alternatively with:
%
\begin{center}
\begin{tabular}{l}
|\input{childdoc.def}|\\
|\childdocby{|\textit{main}|}|\\
\end{tabular}
\end{center}
%
Both forms have slightly different effects as described above.
The main file is prepared as usual, see \secref{sec:include}.

%%%%%%%%%%%%%%%%%%%%%%%%%%%%%%%%%%%%%%%%%%%%%%%%%%%%%%%%%%%%%%%%%%%%%%%%%%%%%%%%
\subsection{Legacy Detection}
\label{sec:detection}

The directive |\childdocmain| in the main file can detect
whether the complete document or merely a child is to be compiled
even without using the directive |\childdocof|.
This method is deprecated because it is less robust
and there is no compelling reason to use it;
it is merely provided for backward compatibility
and it may be removed in future versions.

If the detection mechanism is to be used,
it is mandatory to correctly specify
the filename of the main file as the argument of |\childdocmain|:
%
\begin{center}
\begin{tabular}{l}
|\input{childdoc.def}|\\
|\childdocmain{|\textit{main}|}|\\
\end{tabular}
\end{center}
%
If |\jobname| does not match the argument \textit{main} of |\childdocmain|,
it is assumed that |\jobname| points to the child file to be compiled.
When using |\childdocmain| with the main file specified as argument,
it suffices to start a child file
with just |\input{|\textit{main}|}|
without loading of the package and using |\childdocof|.
If instead all processing is done
with the appropriate \textsf{childdoc} directives,
the argument of \textit{main} of |\childdocmain| can be empty.

An alternative version of the command line processing described
in \secref{sec:commandline} using the detection mechanism reads:
%
\begin{center}
|... -jobname "|\textit{target}|" "|[\textit{flags}]%
[|\def\jobname{|\textit{dest}|}|]|\input{|\textit{main}|}"|
\end{center}

%%%%%%%%%%%%%%%%%%%%%%%%%%%%%%%%%%%%%%%%%%%%%%%%%%%%%%%%%%%%%%%%%%%%%%%%%%%%%%%%
\subsection{Manual Code}
\label{sec:manual}

In case one cannot be certain whether the definitions file |childdoc.def|
is installed on the target \TeX{} distribution
and one prefers not to ship it,
it is conceivable to paste a few relevant commands into the sources.

To that end, drop all statements |\input{childdoc.def}|
and perform the replacements as outlined below.
Instead of |\childdocmain{|\textit{main}|}| add the following code
to the top of the main file:
%
\begin{center}
\begin{tabular}{l}
|\||ifdefined\childdocname\endinput\||fi\newif\ifchilddoc|\\
|\edef\childdocname{\scantokens\expandafter{\jobname\noexpand}}|\\
|\def\childdocmain{|\textit{main}|}\||ifx\childdocmain\childdocname\||else|\\
|\childdoctrue\includeonly{\childdocname}\let\jobname\childdocmain\||fi|\\
\end{tabular}
\end{center}
%
Instead of |\childdocof{|\textit{main}|}| just include the main file
at the top of each child file:
%
\begin{center}
|\input{|\textit{main}|}|
\end{center}
%
A simple redirection |\childdocforward{|\textit{dest}|}| is achieved by:
%
\begin{center}
|\def\jobname{|\textit{dest}|}\input{\jobname}|
\end{center}
%
The redirection with prefix
|\childdocforwardprefix[|\textit{prefix}|]{|\textit{dest}|}|
is accomplished by:
%
\begin{center}
\begin{tabular}{l}
|{\edef\jobname{\scantokens\expandafter{\jobname\noexpand}}|\\
|\def\redirectjob |\textit{prefix}|#1~~~{\gdef\jobname{|\textit{dest}|#1}}|\\
|\expandafter\redirectjob\jobname~~~}\input{\jobname}|
\end{tabular}
\end{center}

In an alternative approach,
child documents can be compiled by a specific command line
without additional code or specific definitions:
%
\begin{center}
|... -jobname "|\textit{target}|" "|[\textit{flags}]%
|\includeonly{|\textit{dest}|}\input{|\textit{main}|}"|
\end{center}
%

%%%%%%%%%%%%%%%%%%%%%%%%%%%%%%%%%%%%%%%%%%%%%%%%%%%%%%%%%%%%%%%%%%%%%%%%%%%%%%%%
%%%%%%%%%%%%%%%%%%%%%%%%%%%%%%%%%%%%%%%%%%%%%%%%%%%%%%%%%%%%%%%%%%%%%%%%%%%%%%%%
\section{Information}

%%%%%%%%%%%%%%%%%%%%%%%%%%%%%%%%%%%%%%%%%%%%%%%%%%%%%%%%%%%%%%%%%%%%%%%%%%%%%%%%
\subsection{Copyright}

Copyright \copyright{} 2017--2018 Niklas Beisert

This work may be distributed and/or modified under the
conditions of the \LaTeX{} Project Public License, either version 1.3
of this license or (at your option) any later version.
The latest version of this license is in
  \url{http://www.latex-project.org/lppl.txt}
and version 1.3 or later is part of all distributions of \LaTeX{}
version 2005/12/01 or later.

This work has the LPPL maintenance status `maintained'.

The Current Maintainer of this work is Niklas Beisert.

This work consists of the files |README.txt|, |childdoc.ins| and |childdoc.dtx|
as well as the derived files |childdoc.def|, |cdocsamp.tex|
with |cdocsch1.tex|, |cdocsch2.tex|, |cdocspt3.tex|, |cdocspt4.tex|,
|cdocsdrf.tex|, |cdocsfn1.tex|, |cdocsfn2.tex|
as well as |childdoc.pdf|.

%%%%%%%%%%%%%%%%%%%%%%%%%%%%%%%%%%%%%%%%%%%%%%%%%%%%%%%%%%%%%%%%%%%%%%%%%%%%%%%%
\subsection{Files and Installation}

The package consists of the files:
%
\begin{center}
\begin{tabular}{ll}
    |README.txt|   & readme file \\
    |childdoc.ins| & installation file \\
    |childdoc.dtx| & source file \\
    |childdoc.def| & definition file \\
    |cdocsamp.tex| & sample main file \\
    |cdocsch1.tex| & sample include file \\
    |cdocsch2.tex| & sample include file \\
    |cdocspt3.tex| & sample part file \\
    |cdocspt4.tex| & sample part file \\
    |cdocsdrf.tex| & sample redirection file \\
    |cdocsfn1.tex| & sample redirection file \\
    |cdocsfn2.tex| & sample redirection file \\
    |childdoc.pdf| & manual
\end{tabular}
\end{center}
%
The distribution consists of the files
|README.txt|, |childdoc.ins| and |childdoc.dtx|.
%
\begin{itemize}
\item
Run (pdf)\LaTeX{} on |childdoc.dtx|
to compile the manual |childdoc.pdf| (this file).
\item
Run \LaTeX{} on |childdoc.ins| to create the definitions file |childdoc.def|
and the sample |cdocsamp.tex| with include files
|cdocsch1.tex|, |cdocsch2.tex|, |cdocspt3.tex|, |cdocspt4.tex|,
|cdocsdrf.tex|, |cdocsfn1.tex|, |cdocsfn2.tex|.
Then copy the file |childdoc.def| to an appropriate directory of your \LaTeX{}
distribution, e.g.\ \textit{texmf-root}|/tex/latex/childdoc|.
\end{itemize}

%%%%%%%%%%%%%%%%%%%%%%%%%%%%%%%%%%%%%%%%%%%%%%%%%%%%%%%%%%%%%%%%%%%%%%%%%%%%%%%%
\subsection{Related CTAN Packages}

There are several other packages which offer a similar functionality:
%
\begin{itemize}
\item
The packages
\href{http://ctan.org/pkg/docmute}{\textsf{docmute}},
\href{http://ctan.org/pkg/includex}{\textsf{includex}} and
\href{http://ctan.org/pkg/standalone}{\textsf{standalone}}
provide commands to include only the document body of
a child file thus allowing both files to be compiled individually.
\item
The packages \href{http://ctan.org/pkg/subdocs}{\textsf{subdocs}}
and \href{http://ctan.org/pkg/subfiles}{\textsf{subfiles}}
provide structures in which the main and child documents can be
encapsulated and allowing them to be compiled individually.
The inclusion mechanism is different from the conventional |\include|.
\item
The package \href{http://ctan.org/pkg/combine}{\textsf{combine}}
is an elaborate solution to combine several documents into one.
\end{itemize}
%
See also the CTAN topic \href{http://ctan.org/topic/subdocs}{\textsf{subdocs}}
for further related packages.
The present package differs from the above solutions in that
a document structure constructed with the conventional |\include| mechanism
just needs two extra commands at the top of every file
such that all constituent files can be compiled individually.

%%%%%%%%%%%%%%%%%%%%%%%%%%%%%%%%%%%%%%%%%%%%%%%%%%%%%%%%%%%%%%%%%%%%%%%%%%%%%%%%
%\subsection{Feature Suggestions}
%
%The following is a list of features which may be useful for future
%versions of this package:
%%
%\begin{itemize}
%\item
%\ldots
%\end{itemize}

%%%%%%%%%%%%%%%%%%%%%%%%%%%%%%%%%%%%%%%%%%%%%%%%%%%%%%%%%%%%%%%%%%%%%%%%%%%%%%%%
\subsection{Revision History}

%%%%%%%%%%%%%%%%%%%%%%%%%%%%%%%%%%%%%%%%
\paragraph{v2.0:} 2018/12/30

\begin{itemize}
\item
immediate forward processing
\item
added |\childdocby| mechanism
\item
manual restructured
\end{itemize}

%%%%%%%%%%%%%%%%%%%%%%%%%%%%%%%%%%%%%%%%
\paragraph{v1.6:} 2018/01/17

\begin{itemize}
\item
application for development of include files
\item
corrections to manual
\end{itemize}

%%%%%%%%%%%%%%%%%%%%%%%%%%%%%%%%%%%%%%%%
\paragraph{v1.5:} 2017/05/21

\begin{itemize}
\item
more complete structuring introduced
\item
|\childdocof| introduced
\item
|\childdoc| renamed to |\childdocmain|
\item
|\childredirect| renamed to |\childdocforward| and |\childdocforwardprefix|
and functionality expanded
\end{itemize}

%%%%%%%%%%%%%%%%%%%%%%%%%%%%%%%%%%%%%%%%
\paragraph{v1.0:} 2017/04/27

\begin{itemize}
\item
manual and install package
\item
first version published on CTAN
\end{itemize}

%%%%%%%%%%%%%%%%%%%%%%%%%%%%%%%%%%%%%%%%
\paragraph{v0.6:} 2017/04/26

\begin{itemize}
\item
redirection mechanism added
\end{itemize}

%%%%%%%%%%%%%%%%%%%%%%%%%%%%%%%%%%%%%%%%
\paragraph{v0.5:} 2017/04/26

\begin{itemize}
\item
functionality in definition file
\end{itemize}


%%%%%%%%%%%%%%%%%%%%%%%%%%%%%%%%%%%%%%%%%%%%%%%%%%%%%%%%%%%%%%%%%%%%%%%%%%%%%%%%
%%%%%%%%%%%%%%%%%%%%%%%%%%%%%%%%%%%%%%%%%%%%%%%%%%%%%%%%%%%%%%%%%%%%%%%%%%%%%%%%
%%%%%%%%%%%%%%%%%%%%%%%%%%%%%%%%%%%%%%%%%%%%%%%%%%%%%%%%%%%%%%%%%%%%%%%%%%%%%%%%
\appendix

\settowidth\MacroIndent{\rmfamily\scriptsize 000\ }

 \DocInput{childdoc.dtx}

\end{document}
%</driver>
% \fi
%
% %%%%%%%%%%%%%%%%%%%%%%%%%%%%%%%%%%%%%%%%%%%%%%%%%%%%%%%%%%%%%%%%%%%%%%%%%%%%%%
% %%%%%%%%%%%%%%%%%%%%%%%%%%%%%%%%%%%%%%%%%%%%%%%%%%%%%%%%%%%%%%%%%%%%%%%%%%%%%%
% \section{Sample}
%\iffalse
%<*samplemain>
%\fi
%
% The following presents a sample document
% with two chapters, two parts, a title page,
% a compile flag as well as three forwarding files to set the flag.
% It consists of eight |.tex| files:
% \begin{center}
% \begin{tabular}{ll}
% |cdocsamp.tex|&main file\\
% |cdocsch1.tex|&include file for chapter 1\\
% |cdocsch2.tex|&include file for chapter 2\\
% |cdocspt3.tex|&include file for part 3\\
% |cdocspt4.tex|&include file for part 4\\
% |cdocsdrf.tex|&forwarding file for main file in draft mode\\
% |cdocsfi1.tex|&forwarding file for final version of chapter 1\\
% |cdocsfi2.tex|&forwarding file for final version of chapter 2\\
% \end{tabular}
% \end{center}
% Each of the eight files can be compiled directly by the \LaTeX{} compiler.
%
% %%%%%%%%%%%%%%%%%%%%%%%%%%%%%%%%%%%%%%
% \paragraph{Main File.}
%
% The main file is called |cdocsamp.tex|.
%
% Load the \textsf{childdoc} definitions and
% declare the filename for the main document:
%    \begin{macrocode}
\input{childdoc.def}
\childdocmain{}
%    \end{macrocode}

% Optional override for |\version| flag:
%    \begin{macrocode}
%%\ifchilddoc\else\providecommand{\version}{draft}\fi
%    \end{macrocode}

% Define the default values for the |\version| flag
% (|final| for the main file and |draft| for childs):
%    \begin{macrocode}
\ifchilddoc
\providecommand{\version}{draft}
\else
\providecommand{\version}{final}
\fi
%    \end{macrocode}

% Load the standard document class:
%    \begin{macrocode}
\documentclass[12pt]{article}
%    \end{macrocode}

% Start the document body:
%    \begin{macrocode}
\begin{document}
%    \end{macrocode}

% Declare a title page.
% Print title, part of document being processed and version flag:
%    \begin{macrocode}
\addtocounter{page}{-1}
\begin{center}
{\LARGE\bfseries{}childdoc example\par}
\vspace{1cm}
\ifchilddoc
\ifchilddocmanual part\else chapter\fi:
`\childdocname' of `\childdocjob'\par
\else
main document: `\childdocjob'\par
\fi
version: \version\par
\end{center}
\newpage
%    \end{macrocode}

% Manually include selected file,
% otherwise process as usual:
%    \begin{macrocode}
\ifchilddocmanual
\section*{part `\childdocname'}
\input{\childdocname}
\else
%    \end{macrocode}

% Include the two chapters:
%    \begin{macrocode}
\include{cdocsch1}
\include{cdocsch2}
%    \end{macrocode}

% Include the two parts unless only chapters should be displayed:
%    \begin{macrocode}
\ifchilddoc\else
\section{part three}
\input{cdocspt3}
\section{part four}
\input{cdocspt4}
\fi
%    \end{macrocode}

% Process as usual until here:
%    \begin{macrocode}
\fi
%    \end{macrocode}

% End of document body:
%    \begin{macrocode}
\end{document}
%    \end{macrocode}
%\iffalse
%</samplemain>
%\fi
%
% %%%%%%%%%%%%%%%%%%%%%%%%%%%%%%%%%%%%%%
% \paragraph{Chapter Include Files.}
%
% The include files are called |cdocsch1.tex| and |cdocsch2.tex|.
%
%\iffalse
%<*samplechap1|samplechap2>
%\fi

% Optional override for |\version| flag:
%    \begin{macrocode}
%%\providecommand{\version}{final}
%    \end{macrocode}

% Include the main document:
%    \begin{macrocode}
\input{childdoc.def}
\childdocof{cdocsamp}
%    \end{macrocode}

%\iffalse
%</samplechap1|samplechap2>
%\fi
%
%\iffalse
%<*samplechap1>
%\fi
% Some text for chapter 1:
%    \begin{macrocode}
\section{one}
some text in chapter one
%    \end{macrocode}

%\iffalse
%</samplechap1>
%\fi
% Some text for chapter 2:
%\iffalse
%<*samplechap2>
%\fi
%    \begin{macrocode}
\section{two}
more text in chapter two
%    \end{macrocode}

%\iffalse
%</samplechap2>
%\fi
%
% %%%%%%%%%%%%%%%%%%%%%%%%%%%%%%%%%%%%%%
% \paragraph{Part Include Files.}
%
% The include files are called |cdocspt3.tex| and |cdocspt4.tex|.
%
%\iffalse
%<*samplepart3|samplepart4>
%\fi

% Optional override for |\version| flag:
%    \begin{macrocode}
%%\providecommand{\version}{final}
%    \end{macrocode}

% Include the main document:
%    \begin{macrocode}
\input{childdoc.def}
\childdocby{cdocsamp}
%    \end{macrocode}

%\iffalse
%</samplepart3|samplepart4>
%\fi
%
%\iffalse
%<*samplepart3>
%\fi
% Some text for part 3:
%    \begin{macrocode}
some text in part three
%    \end{macrocode}

%\iffalse
%</samplepart3>
%\fi
% Some text for part 4:
%\iffalse
%<*samplepart4>
%\fi
%    \begin{macrocode}
more text in part four
%    \end{macrocode}

%\iffalse
%</samplepart4>
%\fi
%
% %%%%%%%%%%%%%%%%%%%%%%%%%%%%%%%%%%%%%%
% \paragraph{Forwarding for a Complete Draft.}
%
% The following forwarding file |cdocsdrf.tex|
% compiles the main document in draft mode:
%\iffalse
%<*sampledraft>
%\fi
%    \begin{macrocode}
\def\version{draft}
\input{childdoc.def}
\childdocforward{cdocsamp}
%    \end{macrocode}

%\iffalse
%</sampledraft>
%\fi
%
% %%%%%%%%%%%%%%%%%%%%%%%%%%%%%%%%%%%%%%
% \paragraph{Forwarding for Final Version of the Chapters.}
%
% The following forwarding files |cdocsfn1.tex| and |cdocsfn2.tex|
% (with identical content)
% compile the final versions of the child documents
% |cdocsch1.tex| and |cdocsch2.tex|, respectively:
%\iffalse
%<*samplefinal>
%\fi
%    \begin{macrocode}
\def\version{final}
\input{childdoc.def}
\childdocforwardprefix[cdocsamp]{cdocsfn}{cdocsch}
%    \end{macrocode}

%\iffalse
%</samplefinal>
%\fi
%
% %%%%%%%%%%%%%%%%%%%%%%%%%%%%%%%%%%%%%%
% \paragraph{Command Line Processing.}
%
% The following three command lines generate the output files
% |cdocscld|, |cdocscl1| and |cdocscl2|
% which should be identical to
% |cdocsdrf|, |cdocsch1| and |cdocsfn2|, respectively:
% \begin{center}
% \begin{tabular}{l}
% |latex -jobname cdocscld \|\\
% |  "\def\version{draft}\input{childdoc.def}\childdocforward{cdocsamp}"|\\
% |latex -jobname cdocscl1 \|\\
% |  "\input{childdoc.def}\childdocforward[cdocsamp]{cdocsch1}"|\\
% |latex -jobname cdocscl2 \|\\
% |  "\def\version{final}\input{childdoc.def}\childdocforward{cdocsch2}"|
% \end{tabular}
% \end{center}
% Note that the trailing backslash on each first line
% merely continues the input to the second line
% (for convenient cut ant paste).
% Furthermore, the command |latex| can be replaced by any
% of its alternative versions such as |pdflatex|.
%
% %%%%%%%%%%%%%%%%%%%%%%%%%%%%%%%%%%%%%%%%%%%%%%%%%%%%%%%%%%%%%%%%%%%%%%%%%%%%%%
% %%%%%%%%%%%%%%%%%%%%%%%%%%%%%%%%%%%%%%%%%%%%%%%%%%%%%%%%%%%%%%%%%%%%%%%%%%%%%%
% \section{Implementation}
%\iffalse
%<*package>
%\fi
%
% This section describes the definitions file |childdoc.def|.

% The definitions cannot be loaded using |\usepackage| or |\RequirePackage|
% which has a mechanism to prevent loading a style file more than once.
% When loading the definitions by means of |\input|
% multiple instances have to be prevented manually:
%\iffalse
%This code needs to be before the `\ProvidesFile' directive
%which is defined at the beginning of this file.
%Therefore it is also placed there and commented out here.
%</package>
%<*discard>
%\fi
%    \begin{macrocode}
\ifdefined\childdocmain\endinput\fi
%    \end{macrocode}
%\iffalse
%</discard>
%<*package>
%\fi
%
% \macro{\ifchilddoc}
% \macro{\ifchilddocmanual}
% The conditional |\ifchilddoc| tells whether a
% child (true) or main (false) document is being compiled.
% The conditional |\ifchilddocmanual| tells whether
% the |\includeonly| mechanism is used (false) or
% the selection of child files must be performed manually (true).
% The definitions initialise to false:
%    \begin{macrocode}
\newif\ifchilddoc
\newif\ifchilddocmanual
%    \end{macrocode}

% \macro{\childdocname}
% \macro{\childdocjob}
% The macro |\childdocname| stores the name of the main document
% to be compiled. The macro |\childdocjob| stores the name of
% the document on which the \LaTeX{} compiler was originally invoked.
% The content of |\jobname| cannot be compared
% to filenames specified in the source due to different catcodes.
% The following code rescans |\jobname|, stores the result
% in |\childdocname| and saves a copy in |\childdocjob|:
%    \begin{macrocode}
\edef\childdocname{\scantokens\expandafter{\jobname\noexpand}}
\let\childdocjob\childdocname
%    \end{macrocode}

% \macro{\childdocdisable}
% The macro |\childdocdisable| prevents the main file
% from being processed more than once.
% At this stage, the main document command |\childdocmain|
% is assumed to be called once again where it should do nothing.
% Any subsequent call to it should prevent
% a secondary processing of the main document
% It overwrites the forwarding commands
% |\childdocof| and |\childdocforward|
% with empty macros to prevent further inclusions of the main document:
%    \begin{macrocode}
\newcommand{\childdocdisable}
{
  \renewcommand{\childdocmain}[1]{\renewcommand{\childdocmain}[1]{\endinput}}
  \renewcommand{\childdocof}[1]{}
  \renewcommand{\childdocby}[2][]{}
  \renewcommand{\childdocforward}[2][]{}
  \renewcommand{\childdocdisable}{}
}
%    \end{macrocode}

% \macro{\childdocmain}
% The macro |\childdocmain| is to be called at the top of the main file
% with nothing or the main filename (without extension) as argument.
% First, it breaks loops.
% If the argument is not empty and does not match |\childdocname|
% (which is set by the first inclusion of |childdoc.def|),
% |\ifchilddoc| is set to true, |\includeonly| is applied to the child file
% and |\jobname| is set to the main file
% (for proper handling of |.aux| files):
%    \begin{macrocode}
\newcommand{\childdocmain}[1]
{
  \childdocdisable\childdocmain{}
  \if?#1?\else
    \begingroup
      \def\childdoctmp{#1}
      \ifx\childdoctmp\childdocname
        \def\childdoctmp{}
      \else
        \def\childdoctmp
        {
          \childdoctrue
          \includeonly{\childdocname}
          \def\childdocjob{#1}
          \def\jobname{#1}
        }
      \fi
      \expandafter
    \endgroup
    \childdoctmp
  \fi
}
%    \end{macrocode}

% \macro{\childdocof}
% The command |\childdocof| redirects
% compilation to the main file |#1|.
%    \begin{macrocode}
\newcommand{\childdocof}[1]
{
  \childdocdisable
  \childdoctrue
  \includeonly{\childdocname}
  \def\jobname{#1}
  \def\childdocjob{#1}
  \input{#1}
}
%    \end{macrocode}

% \macro{\childdocby}
% The command |\childdocby| ....
%    \begin{macrocode}
\newcommand{\childdocby}[2][]
{
  \childdocdisable
  \childdoctrue
  \childdocmanualtrue
  \if?#1?\else
    \def\jobname{#2}
  \fi
  \def\childdocjob{#2}
  \input{#2}
  \endinput
}
%    \end{macrocode}

% \macro{\childdocforward}
% The command |\childdocforward| redirects
% compilation to the main file or
% (if the optional argument is given) a child file.
% Parameters are set as if the main file
% or a child file starting with |\childdocof| was compiled.
% Then compilation is handed over to the main file:
%    \begin{macrocode}
\newcommand{\childdocforward}[2][]
{
  \begingroup
    \if?#1?
      \def\childdoctmp
      {
        \def\childdocname{#2}
        \def\childdocjob{#2}
        \def\jobname{#2}
        \input{#2}
        \endinput
      }
    \else
      \def\childdoctmp
      {
        \childdocdisable
        \def\childdocname{#2}
        \childdoctrue
        \includeonly{#2}
        \def\childdocjob{#1}
        \def\jobname{#1}
        \input{#1}
        \endinput
      }
    \fi
    \expandafter
  \endgroup
  \childdoctmp
}
%    \end{macrocode}

% \macro{\childdocforwardprefix}
% The command |\childdocforwardprefix| redirects
% compilation to the main or a child file by means of a pattern.
% The prefix |#1| in the current filename is replaced by |#2|
% and the suffix of the current filename is kept
% (it is assumed that the filename does not contain the substring `|~~~|'
% which is used as a delimiter).
% Compilation is handed over to the new file by |\childdocforward|:
%    \begin{macrocode}
\newcommand{\childdocforwardprefix}[3][]
{
  \begingroup
    \def\childdocextract #2##1~~~{\def\childdoctmp{\childdocforward[#1]{#3##1}}}
    \expandafter\childdocextract\childdocname~~~
    \expandafter
  \endgroup
  \childdoctmp
}
%    \end{macrocode}

% \macro{\childdoc}
% The deprecated macro |\childdoc| is a legacy version of |\childdocmain|:
%    \begin{macrocode}
\newcommand{\childdoc}{\childdocmain}
%    \end{macrocode}

% \macro{\childdocredirect}
% The deprecated macro |\childdocredirect| is a legacy version
% of |\childdocforward| and |\childdocforwardprefix|:
%    \begin{macrocode}
\newcommand{\childdocredirect}[2][]
{
  \begingroup
    \if?#1?
      \def\childdoctmp{\childdocforward{#2}}
    \else
      \def\childdoctmp{\childdocforwardprefix{#1}{#2}}
    \fi
    \expandafter
  \endgroup
  \childdoctmp
}
%    \end{macrocode}

%\iffalse
%</package>
%\fi
%
\endinput
|\\
|\childdocmain{}|\\
\end{tabular}
\end{center}
at the very top of the main \LaTeX{} file,
in particular \emph{before} the |\documentclass| statement!
The argument of |\childdocmain| should be left empty
(but it must be present).

%%%%%%%%%%%%%%%%%%%%%%%%%%%%%%%%%%%%%%%%
\DescribeMacro{\childdocof}
Furthermore, add the commands
\begin{center}
\begin{tabular}{l}
|% \iffalse
%
% childdoc.dtx Copyright (C) 2017-2018 Niklas Beisert
%
% This work may be distributed and/or modified under the
% conditions of the LaTeX Project Public License, either version 1.3
% of this license or (at your option) any later version.
% The latest version of this license is in
%   http://www.latex-project.org/lppl.txt
% and version 1.3 or later is part of all distributions of LaTeX
% version 2005/12/01 or later.
%
% This work has the LPPL maintenance status `maintained'.
%
% The Current Maintainer of this work is Niklas Beisert.
%
% This work consists of the files childdoc.dtx and childdoc.ins
% and the derived files childdoc.def and cdocsamp.tex with
% cdocsch1.tex, cdocsch2.tex, cdocsdrf.tex, cdocsfn1.tex, cdocsfn2.tex.
%
%<package>\ifdefined\childdocmain\endinput\fi
%<package>\ProvidesFile{childdoc.def}[2018/12/30 v2.0 child document driver]
%<samplemain>\ProvidesFile{cdocsamp.tex}[2018/12/30 v2.0 sample for childdoc]
%<*driver>
%\ProvidesFile{childdoc.drv}[2018/12/30 v2.0 childdoc reference manual file]
\PassOptionsToClass{10pt,a4paper}{article}
\documentclass{ltxdoc}

\usepackage[margin=35mm]{geometry}
\usepackage{hyperref}
\usepackage{hyperxmp}
\usepackage[usenames]{color}

\hypersetup{colorlinks=true}
\hypersetup{pdfstartview=FitH}
\hypersetup{pdfpagemode=UseNone}
\hypersetup{pdfsource={}}
\hypersetup{pdflang={en-UK}}
\hypersetup{pdfcopyright={Copyright 2017-2018 Niklas Beisert.
  This work may be distributed and/or modified under the
  conditions of the LaTeX Project Public License, either version 1.3
  of this license or (at your option) any later version.}}
\hypersetup{pdflicenseurl={http://www.latex-project.org/lppl.txt}}
\hypersetup{pdfcontactaddress={ETH Zurich, ITP, HIT K,
  Wolfgang-Pauli-Strasse 27}}
\hypersetup{pdfcontactpostcode={8093}}
\hypersetup{pdfcontactcity={Zurich}}
\hypersetup{pdfcontactcountry={Switzerland}}
\hypersetup{pdfcontactemail={nbeisert@itp.phys.ethz.ch}}
\hypersetup{pdfcontacturl={http://people.phys.ethz.ch/\xmptilde nbeisert/}}

\newcommand{\secref}[1]{\hyperref[#1]{section \ref*{#1}}}

\parskip1ex
\parindent0pt
\let\olditemize\itemize
\def\itemize{\olditemize\parskip0pt}

\begin{document}

\title{The \textsf{childdoc} Package}
\hypersetup{pdftitle={The childdoc Package}}
\author{Niklas Beisert\\[2ex]
  Institut f\"ur Theoretische Physik\\
  Eidgen\"ossische Technische Hochschule Z\"urich\\
  Wolfgang-Pauli-Strasse 27, 8093 Z\"urich, Switzerland\\[1ex]
  \href{mailto:nbeisert@itp.phys.ethz.ch}
  {\texttt{nbeisert@itp.phys.ethz.ch}}}
\hypersetup{pdfauthor={Niklas Beisert}}
\hypersetup{pdfsubject={Manual for the LaTeX2e Package childdoc}}
\date{30 December 2018, \textsf{v2.0}}
\maketitle

\begin{abstract}\noindent
\textsf{childdoc} is a \LaTeXe{} package
that enables the direct compilation
of document sections included by |\include|
to individual files.
\end{abstract}

\begingroup
\parskip0ex
\tableofcontents
\endgroup

%%%%%%%%%%%%%%%%%%%%%%%%%%%%%%%%%%%%%%%%%%%%%%%%%%%%%%%%%%%%%%%%%%%%%%%%%%%%%%%%
%%%%%%%%%%%%%%%%%%%%%%%%%%%%%%%%%%%%%%%%%%%%%%%%%%%%%%%%%%%%%%%%%%%%%%%%%%%%%%%%
\section{Introduction}

\LaTeX{} provides a mechanism to structure a large document (such as a book)
into a main file and several child files (containing the chapters)
using the |\include| command.
This mechanism is beneficial for documents
which span hundreds of pages in order to
make the source file(s) more manageable.
Moreover, compilation can be restricted to
selected child files by means of the |\includeonly| command.
The latter feature can be used to reduce the compilation time while editing
(this was significantly more useful in the earlier days of \LaTeX{})
or to generate a smaller document which is easier to navigate.
Another application of |\includeonly| is to generate
documents consisting of selected parts of the complete document.

However, there are a few drawbacks of the plain |\include| mechanism:
\begin{itemize}
\item
The child files cannot be compiled on their own,
they can only be compiled via the main file.
A naive editing environment
(such as a text editor with an option
to have the current file processed by \LaTeX)
may require one to switch to the main file before compiling;
attempting to compile the child file produces errors.
\item
The main file must be modified (each time)
to adjust the |\includeonly| command
to the present needs. This easily leaves the main file in a messy state.
\item
The generated document will always carry the filename
of the main document. This is inconvenient if
several child files are to be compiled and
to be kept for distribution.
\end{itemize}

The present package provides a simple interface
to make child files individually compilable by \LaTeX{}.
Compiling a child file then has the same effect as compiling
the main file with an |\includeonly| command
to select the appropriate child.
Moreover the generated document will carry the name of the child
rather than the main file.
This resolves all three above issues.

This feature is meant to make the editing of books,
thesis documents and lecture notes somewhat more convenient.
However, the package can also be used efficiently for
composing a series of documents (such as exercise sheets)
which are typically distributed individually.
It then assists the author in generating the individual documents
(potentially in different versions)
as well as a document containing the collected series.
Another application is in developing style files
or other kinds of included material
where compilation of the style file could redirect
to a sample or test file.

%%%%%%%%%%%%%%%%%%%%%%%%%%%%%%%%%%%%%%%%%%%%%%%%%%%%%%%%%%%%%%%%%%%%%%%%%%%%%%%%
%%%%%%%%%%%%%%%%%%%%%%%%%%%%%%%%%%%%%%%%%%%%%%%%%%%%%%%%%%%%%%%%%%%%%%%%%%%%%%%%
\section{Usage}

First of all, the package \textsf{childdoc} is \emph{not} a standard
\LaTeXe{} |.sty| style file! Therefore it needs to be invoked in
a non-standard way.

%%%%%%%%%%%%%%%%%%%%%%%%%%%%%%%%%%%%%%%%%%%%%%%%%%%%%%%%%%%%%%%%%%%%%%%%%%%%%%%%
\subsection{Included Files}
\label{sec:include}

%%%%%%%%%%%%%%%%%%%%%%%%%%%%%%%%%%%%%%%%
\DescribeMacro{\childdocmain}
To use the package, add the commands
\begin{center}
\begin{tabular}{l}
|\input{childdoc.def}|\\
|\childdocmain{}|\\
\end{tabular}
\end{center}
at the very top of the main \LaTeX{} file,
in particular \emph{before} the |\documentclass| statement!
The argument of |\childdocmain| should be left empty
(but it must be present).

%%%%%%%%%%%%%%%%%%%%%%%%%%%%%%%%%%%%%%%%
\DescribeMacro{\childdocof}
Furthermore, add the commands
\begin{center}
\begin{tabular}{l}
|\input{childdoc.def}|\\
|\childdocof{|\textit{main}|}|\\
\end{tabular}
\end{center}
at the top of every child file \textit{child}
which is included by |\include{|\textit{child}|}|
from within the main file
(or at least for those files to be compiled individually).
The argument \textit{main} must be the filename of the main file.

There are a couple of
considerations in setting up the main and child documents:

%%%%%%%%%%%%%%%%%%%%%%%%%%%%%%%%%%%%%%%%
\paragraph{Restrictions.}

Please note the following restrictions:
\begin{itemize}
\item
|\childdocmain| must be called with one argument \textit{main}
to ensure compatibility with earlier version of the package.
It must either be empty (|\childdocmain{}|)
or precisely match the filename of the main file in which it is specified.
See \secref{sec:detection} for further information.
\item
The filename \textit{main} must be specified without the |.tex| extension.
\item
The filename \textit{main} is case sensitive
(even in case-insensitive file systems)
due to internal string comparison.
\item
The argument \textit{main} should be fully expanded, it cannot be a macro.
\item
Subdirectories and special characters should be avoided in filenames.
\item
The command |\childdocmain{|\textit{main}|}| must be followed by a whitespace.
It should not be followed immediately by another command
or by a comment mark `|%|'.
This is because the \TeX{} parser reads the token immediately following
the argument of |\childdocmain| and puts it
at the beginning of every child section;
however, a white\-space is ignored.
\end{itemize}

%%%%%%%%%%%%%%%%%%%%%%%%%%%%%%%%%%%%%%%%
\paragraph{Content of Main File.}

It is advisable to place all content in the child files included by |\include|.
Any output contained in the main file will appear in all child documents
unless suppressed manually;
it cannot be suppressed automatically by the |\includeonly| directive
and thus should normally be avoided.
A method to include some content in the main file
by means of conditional processing is described in \secref{sec:conditional}.

%%%%%%%%%%%%%%%%%%%%%%%%%%%%%%%%%%%%%%%%
\paragraph{Page Numbering.}

When only a part of the document is compiled,
the appropriate numbering of pages
(as well as other status parameters)
is determined from the |.aux| files.
The latter contain information from previous passes.
However this information needs to propagate through
all intermediate child documents.
Therefore the page numbering in child documents may well
be inconsistent until the complete document is compiled at least once.

A useful (if unconventional) way to always ensure a consistent
page numbering is to restart the numbering in each child document
and denote the pages by `\textit{child}|.|\textit{page}'
where \textit{child} represents the chapter/section number of the child file.
This can be achieved by the command
|\numberwithin{page}{|\textit{child}|}|
of the \textsf{amsmath} package
where \textit{child} can be |chapter| or |section|
depending on the chosen structuring.
Alternatively, one can modify the macro |\thepage| appropriately
and reset the counter |page| at the start of each child file.

%%%%%%%%%%%%%%%%%%%%%%%%%%%%%%%%%%%%%%%%%%%%%%%%%%%%%%%%%%%%%%%%%%%%%%%%%%%%%%%%
\subsection{Conditional Processing}
\label{sec:conditional}

The package provides a mechanism to compile different versions
of a document. To customise the versions further some conditional processing
can come in handy to distinguish which version is being compiled.
The package provides two macros to describe the compilation context:

%%%%%%%%%%%%%%%%%%%%%%%%%%%%%%%%%%%%%%%%
\DescribeMacro{\ifchilddoc}
The conditional |\ifchilddoc| distinguishes between the compilation of
child documents and the main document:
%
\begin{center}
|\ifchilddoc |\textit{child-code}| |[|\||else |\textit{main-code}]| \||fi|
\end{center}

%%%%%%%%%%%%%%%%%%%%%%%%%%%%%%%%%%%%%%%%
\DescribeMacro{\childdocname}
\DescribeMacro{\childdocjob}
The macro |\childdocname| contains the filename (without extension)
of the main or child file being processed.
Note that |\childdocjob| will always contain the name of the main file.

%%%%%%%%%%%%%%%%%%%%%%%%%%%%%%%%%%%%%%%%
\paragraph{Title Page.}

Conditional processing can be used to include a title or banner page
in the main document when proper precautions are taken.
Importantly, the code in the main file should ensure that the page counter
(as well as other status parameters which are stored in the |.aux| files)
takes the same value after the conditional processing.
Otherwise the page numbers may take divergent values
depending on which part is compiled.

For example, a title page could be declared by:
%
\begin{center}
\begin{tabular}{l}
|\ifchilddoc\||else|\\
|\addtocounter{page}{-1}|\\
\textit{code for title page}\\
|\newpage|\\
|\||fi|
\end{tabular}
\end{center}
%
A banner page for the child documents can be generated by:
%
\begin{center}
\begin{tabular}{l}
|\ifchilddoc|\\
|\addtocounter{page}{-1}|\\
\textit{code for banner page}\\
|\newpage|\\
|\||fi|
\end{tabular}
\end{center}
%
Here one could write a message such as:
\begin{center}
|This is the part \childdocname{} of \childdocjob{}.|
\end{center}

%%%%%%%%%%%%%%%%%%%%%%%%%%%%%%%%%%%%%%%%%%%%%%%%%%%%%%%%%%%%%%%%%%%%%%%%%%%%%%%%
\subsection{Flags}
\label{sec:flags}

The package makes it easy to generate different versions
of the main or child documents.
To this end compilation flags can be defined
and assigned different default values.
They will be particularly useful in conjunction
with the forwarding mechanism described in \secref{sec:forward}.

For example, it may be useful to have a flag |\version|
which can be set to |draft| or |final|.
The document source will contain some conditional code
depending on the value of |\version|.
Suppose further, the flag should default to |final| for the main file
and to |draft| for child files
which is a natural assignment for editing the document.
This is achieved by placing the following code
in the preamble of the main document
(below the |\childdocmain| directive):
%
\begin{center}
\begin{tabular}{l}
|\ifchilddoc|\\
|\providecommand{\version}{draft}|\\
|\||else|\\
|\providecommand{\version}{final}|\\
|\||fi|
\end{tabular}
\end{center}
%
The definition by |\providecommand| makes sure
that previous definitions are not overwritten.
Further statements |\providecommand{\version}{...}|
can thus be added before the above code to override it.

For the main file, one might add a line
(between |\childdocmain| and the above block)
%
\begin{center}
|%\ifchilddoc\||else\providecommand{\version}{draft}\||fi|
\end{center}
%
which can be uncommented to produce a draft version.
Likewise one can add a line to the very top of a child file
(above the |\childdocof{|\textit{main}|}| directive)
%
\begin{center}
|%\providecommand{\version}{final}|
\end{center}
%
which can be uncommented to produce the final version of this child document.

%%%%%%%%%%%%%%%%%%%%%%%%%%%%%%%%%%%%%%%%%%%%%%%%%%%%%%%%%%%%%%%%%%%%%%%%%%%%%%%%
\subsection{Forwarding}
\label{sec:forward}

Different versions of the main or child documents
using compilation flags as described in \secref{sec:flags}
can be (permanently) stored in different files
for convenient compilation, viewing and distribution.
To this end, the package defines a command
to pass on compilation to a different file:

%%%%%%%%%%%%%%%%%%%%%%%%%%%%%%%%%%%%%%%%
\DescribeMacro{\childdocforward}
The command |\childdocforward| redirects processing to
another source file:
%
\begin{center}
\begin{tabular}{l}
|\input{childdoc.def}|\\
|\childdocforward[|\textit{main}|]{|\textit{dest}|}|\\
\end{tabular}
\end{center}
%
The argument \textit{dest} is the destination file
(without extension).
It should be the main file or one of the child files.
Note that further \textsf{childdoc} directives
such as |\childdocof| and |\childdocforward|
in the indicated file will be processed in this form.
The optional argument \textit{main}
passes on directly to the main file \textit{main}
while pretending to compile the child \textit{dest}.
This form behaves as if \textit{dest}
issues |\childdocof{|\textit{main}|}| right away,
and no further \textsf{childdoc} directives will be processed.

%%%%%%%%%%%%%%%%%%%%%%%%%%%%%%%%%%%%%%%%
\DescribeMacro{\...prefix}
In the alternative form |\childdocforwardprefix|,
%
\begin{center}
\begin{tabular}{l}
|\input{childdoc.def}|\\
|\childdocforwardprefix[|\textit{main}|]{|\textit{prefix}|}{|\textit{dest}|}|
\end{tabular}
\end{center}
%
the destination file is determined by a pattern
depending on the current file:
To make this work, the current file must be called
`{\textit{prefix}\hspace{0.2em}\textit{suffix}}'
with \textit{prefix} matching precisely the argument.
Processing is then passed on to the file
`{\textit{dest}\hspace{0.2em}\textit{suffix}}'.
Surely, the same effect is achieved by
directly specifying the
argument `{\textit{dest}\hspace{0.2em}\textit{suffix}}'
in the first form.
However, that requires to set up a different file
for each child. With the alternative form of the command
all these files can have exactly the same content
which simplifies setting them up and maintaining them.

For example, the following file |draft.tex|
with a compilation flag |\version| as described in \secref{sec:flags}
compiles the main document as a draft:
%
\begin{center}
\begin{tabular}{l}
|\def\version{draft}|\\
|\input{childdoc.def}|\\
|\childdocforward{|\textit{main}|}|
\end{tabular}
\end{center}
%
Likewise, the following files |final|\textit{nn}|.tex|
compile the final version of the child document
|child|\textit{nn}|.tex|:
%
\begin{center}
\begin{tabular}{l}
|\def\version{final}|\\
|\input{childdoc.def}|\\
|\childdocforwardprefix{final}{child}|
\end{tabular}
\end{center}
%

Note that when several versions of a main file and/or of each child file
are to be generated, it may be convenient to set up a |Makefile| or
shell script to automatise the process.

%%%%%%%%%%%%%%%%%%%%%%%%%%%%%%%%%%%%%%%%%%%%%%%%%%%%%%%%%%%%%%%%%%%%%%%%%%%%%%%%
\subsection{Command Line Processing}
\label{sec:commandline}

The effect of redirection files can also be achieved by invoking
the \LaTeX{} compiler with a more elaborate command line.
Most conveniently this should be done as part
of a shell script or a |Makefile|.

When using \textsf{childdoc} in the main file, the following
command lines effectively perform a redirection
(note that depending on the shell being used,
backslashes may have to be doubled: `|\|' $\to$ `|\\|'):
%
\begin{center}
|... -jobname "|\textit{target}|" |\\|"|[\textit{flags}]%
|\input{childdoc.def}\childdocforward[|\textit{main}|]{|\textit{dest}|}"|
\end{center}
%
Here \textit{target} is the name of the output file,
\textit{main} is the name of the main file
and \textit{dest} is the name of the main or child file to be processed
(all filenames without extensions).
The optional argument \textit{main} can be omitted
if \textit{main} matches \textit{dest}.
Optionally, compilation \textit{flags} can be defined via |\def| commands.
This command line makes the \TeX{} engine believe
it is compiling the file \textit{target}
whose content is specified as the latter parameter.
The provided code then forwards the processing to
\textit{main} or \textit{dest} as described in \secref{sec:forward}.

%%%%%%%%%%%%%%%%%%%%%%%%%%%%%%%%%%%%%%%%%%%%%%%%%%%%%%%%%%%%%%%%%%%%%%%%%%%%%%%%
\subsection{Include by Input}
\label{sec:input}

Including child documents by |\include| has some restrictions by design.
Most notably, the content of a child document always occupies
its own set of pages; pages cannot be shared between child documents.
Usually, this behaviour makes perfect sense
because each child document contain an essential part of the document.
However, in some situations it may be desirable to compose
a document from a collection of parts
without having mandatory page breaks between then.
For this case, the package
provides a mechanism to include parts
by |\input| which can also be processed individually.
However, by construction this mechanism
requires manual handling of the content to be output.

%%%%%%%%%%%%%%%%%%%%%%%%%%%%%%%%%%%%%%%%
\DescribeMacro{\ifchilddocmanual}
The main file should be prepared as usual, see \secref{sec:include}.
However, the document body must make a distinction
between processing of an individual part and of the main document, e.g.:
%
\begin{center}
\begin{tabular}{l}
|\ifchilddocmanual|\\
|\input{\childdocname}|\\
|\||else|\\
\textit{document body with }|\input{|\textit{part}|}|\\
|\||fi|
\end{tabular}
\end{center}
%
The conditional |\ifchilddocmanual| is true whenever
a part to be included by |\input| is being compiled,
and the name of the part is stored in |\childdocname|.

%%%%%%%%%%%%%%%%%%%%%%%%%%%%%%%%%%%%%%%%
\DescribeMacro{\childdocby}
Each part to be included by |\input| should start with:
%
\begin{center}
\begin{tabular}{l}
|\input{childdoc.def}|\\
|\childdocby{|\textit{main}|}|\\
\end{tabular}
\end{center}
%
The directive |\childdocby| is similar to |\childdocof|
described in \secref{sec:include},
but the subsequent selection of content must be done manually.
To that end, both |\ifchilddoc| and |\ifchilddocmanual|
will be true upon processing of a part,
and the name of the part is stored in |\childdocname|.
Note that |\jobname| will be set to the filename of the current part
so that each part receives an individual |.aux| file
that does not interfere with the |.aux| file(s) of the main document.
This behaviour can be altered by the alternative form
|\childdocby[*]{|\textit{main}|}| (with a non-empty optional argument)
which uses the |.aux| file of the main document
by setting |\jobname| to \textit{main}.

%%%%%%%%%%%%%%%%%%%%%%%%%%%%%%%%%%%%%%%%%%%%%%%%%%%%%%%%%%%%%%%%%%%%%%%%%%%%%%%%
\subsection{Driver Development}
\label{sec:driver}

The \textsf{childdoc} mechanism can also be use for the development
of definition files such as \LaTeX{} styles or classes.
This case differs from the above setup with multiple parts
included by |\include| in that no |\includeonly| should be invoked.
This can be achieved by starting the include file
(before |\ProvidesPackage|) with:
%
\begin{center}
\begin{tabular}{l}
|\input{childdoc.def}|\\
|\childdocforward{|\textit{main}|}|\\
\end{tabular}
\end{center}
%
or alternatively with:
%
\begin{center}
\begin{tabular}{l}
|\input{childdoc.def}|\\
|\childdocby{|\textit{main}|}|\\
\end{tabular}
\end{center}
%
Both forms have slightly different effects as described above.
The main file is prepared as usual, see \secref{sec:include}.

%%%%%%%%%%%%%%%%%%%%%%%%%%%%%%%%%%%%%%%%%%%%%%%%%%%%%%%%%%%%%%%%%%%%%%%%%%%%%%%%
\subsection{Legacy Detection}
\label{sec:detection}

The directive |\childdocmain| in the main file can detect
whether the complete document or merely a child is to be compiled
even without using the directive |\childdocof|.
This method is deprecated because it is less robust
and there is no compelling reason to use it;
it is merely provided for backward compatibility
and it may be removed in future versions.

If the detection mechanism is to be used,
it is mandatory to correctly specify
the filename of the main file as the argument of |\childdocmain|:
%
\begin{center}
\begin{tabular}{l}
|\input{childdoc.def}|\\
|\childdocmain{|\textit{main}|}|\\
\end{tabular}
\end{center}
%
If |\jobname| does not match the argument \textit{main} of |\childdocmain|,
it is assumed that |\jobname| points to the child file to be compiled.
When using |\childdocmain| with the main file specified as argument,
it suffices to start a child file
with just |\input{|\textit{main}|}|
without loading of the package and using |\childdocof|.
If instead all processing is done
with the appropriate \textsf{childdoc} directives,
the argument of \textit{main} of |\childdocmain| can be empty.

An alternative version of the command line processing described
in \secref{sec:commandline} using the detection mechanism reads:
%
\begin{center}
|... -jobname "|\textit{target}|" "|[\textit{flags}]%
[|\def\jobname{|\textit{dest}|}|]|\input{|\textit{main}|}"|
\end{center}

%%%%%%%%%%%%%%%%%%%%%%%%%%%%%%%%%%%%%%%%%%%%%%%%%%%%%%%%%%%%%%%%%%%%%%%%%%%%%%%%
\subsection{Manual Code}
\label{sec:manual}

In case one cannot be certain whether the definitions file |childdoc.def|
is installed on the target \TeX{} distribution
and one prefers not to ship it,
it is conceivable to paste a few relevant commands into the sources.

To that end, drop all statements |\input{childdoc.def}|
and perform the replacements as outlined below.
Instead of |\childdocmain{|\textit{main}|}| add the following code
to the top of the main file:
%
\begin{center}
\begin{tabular}{l}
|\||ifdefined\childdocname\endinput\||fi\newif\ifchilddoc|\\
|\edef\childdocname{\scantokens\expandafter{\jobname\noexpand}}|\\
|\def\childdocmain{|\textit{main}|}\||ifx\childdocmain\childdocname\||else|\\
|\childdoctrue\includeonly{\childdocname}\let\jobname\childdocmain\||fi|\\
\end{tabular}
\end{center}
%
Instead of |\childdocof{|\textit{main}|}| just include the main file
at the top of each child file:
%
\begin{center}
|\input{|\textit{main}|}|
\end{center}
%
A simple redirection |\childdocforward{|\textit{dest}|}| is achieved by:
%
\begin{center}
|\def\jobname{|\textit{dest}|}\input{\jobname}|
\end{center}
%
The redirection with prefix
|\childdocforwardprefix[|\textit{prefix}|]{|\textit{dest}|}|
is accomplished by:
%
\begin{center}
\begin{tabular}{l}
|{\edef\jobname{\scantokens\expandafter{\jobname\noexpand}}|\\
|\def\redirectjob |\textit{prefix}|#1~~~{\gdef\jobname{|\textit{dest}|#1}}|\\
|\expandafter\redirectjob\jobname~~~}\input{\jobname}|
\end{tabular}
\end{center}

In an alternative approach,
child documents can be compiled by a specific command line
without additional code or specific definitions:
%
\begin{center}
|... -jobname "|\textit{target}|" "|[\textit{flags}]%
|\includeonly{|\textit{dest}|}\input{|\textit{main}|}"|
\end{center}
%

%%%%%%%%%%%%%%%%%%%%%%%%%%%%%%%%%%%%%%%%%%%%%%%%%%%%%%%%%%%%%%%%%%%%%%%%%%%%%%%%
%%%%%%%%%%%%%%%%%%%%%%%%%%%%%%%%%%%%%%%%%%%%%%%%%%%%%%%%%%%%%%%%%%%%%%%%%%%%%%%%
\section{Information}

%%%%%%%%%%%%%%%%%%%%%%%%%%%%%%%%%%%%%%%%%%%%%%%%%%%%%%%%%%%%%%%%%%%%%%%%%%%%%%%%
\subsection{Copyright}

Copyright \copyright{} 2017--2018 Niklas Beisert

This work may be distributed and/or modified under the
conditions of the \LaTeX{} Project Public License, either version 1.3
of this license or (at your option) any later version.
The latest version of this license is in
  \url{http://www.latex-project.org/lppl.txt}
and version 1.3 or later is part of all distributions of \LaTeX{}
version 2005/12/01 or later.

This work has the LPPL maintenance status `maintained'.

The Current Maintainer of this work is Niklas Beisert.

This work consists of the files |README.txt|, |childdoc.ins| and |childdoc.dtx|
as well as the derived files |childdoc.def|, |cdocsamp.tex|
with |cdocsch1.tex|, |cdocsch2.tex|, |cdocspt3.tex|, |cdocspt4.tex|,
|cdocsdrf.tex|, |cdocsfn1.tex|, |cdocsfn2.tex|
as well as |childdoc.pdf|.

%%%%%%%%%%%%%%%%%%%%%%%%%%%%%%%%%%%%%%%%%%%%%%%%%%%%%%%%%%%%%%%%%%%%%%%%%%%%%%%%
\subsection{Files and Installation}

The package consists of the files:
%
\begin{center}
\begin{tabular}{ll}
    |README.txt|   & readme file \\
    |childdoc.ins| & installation file \\
    |childdoc.dtx| & source file \\
    |childdoc.def| & definition file \\
    |cdocsamp.tex| & sample main file \\
    |cdocsch1.tex| & sample include file \\
    |cdocsch2.tex| & sample include file \\
    |cdocspt3.tex| & sample part file \\
    |cdocspt4.tex| & sample part file \\
    |cdocsdrf.tex| & sample redirection file \\
    |cdocsfn1.tex| & sample redirection file \\
    |cdocsfn2.tex| & sample redirection file \\
    |childdoc.pdf| & manual
\end{tabular}
\end{center}
%
The distribution consists of the files
|README.txt|, |childdoc.ins| and |childdoc.dtx|.
%
\begin{itemize}
\item
Run (pdf)\LaTeX{} on |childdoc.dtx|
to compile the manual |childdoc.pdf| (this file).
\item
Run \LaTeX{} on |childdoc.ins| to create the definitions file |childdoc.def|
and the sample |cdocsamp.tex| with include files
|cdocsch1.tex|, |cdocsch2.tex|, |cdocspt3.tex|, |cdocspt4.tex|,
|cdocsdrf.tex|, |cdocsfn1.tex|, |cdocsfn2.tex|.
Then copy the file |childdoc.def| to an appropriate directory of your \LaTeX{}
distribution, e.g.\ \textit{texmf-root}|/tex/latex/childdoc|.
\end{itemize}

%%%%%%%%%%%%%%%%%%%%%%%%%%%%%%%%%%%%%%%%%%%%%%%%%%%%%%%%%%%%%%%%%%%%%%%%%%%%%%%%
\subsection{Related CTAN Packages}

There are several other packages which offer a similar functionality:
%
\begin{itemize}
\item
The packages
\href{http://ctan.org/pkg/docmute}{\textsf{docmute}},
\href{http://ctan.org/pkg/includex}{\textsf{includex}} and
\href{http://ctan.org/pkg/standalone}{\textsf{standalone}}
provide commands to include only the document body of
a child file thus allowing both files to be compiled individually.
\item
The packages \href{http://ctan.org/pkg/subdocs}{\textsf{subdocs}}
and \href{http://ctan.org/pkg/subfiles}{\textsf{subfiles}}
provide structures in which the main and child documents can be
encapsulated and allowing them to be compiled individually.
The inclusion mechanism is different from the conventional |\include|.
\item
The package \href{http://ctan.org/pkg/combine}{\textsf{combine}}
is an elaborate solution to combine several documents into one.
\end{itemize}
%
See also the CTAN topic \href{http://ctan.org/topic/subdocs}{\textsf{subdocs}}
for further related packages.
The present package differs from the above solutions in that
a document structure constructed with the conventional |\include| mechanism
just needs two extra commands at the top of every file
such that all constituent files can be compiled individually.

%%%%%%%%%%%%%%%%%%%%%%%%%%%%%%%%%%%%%%%%%%%%%%%%%%%%%%%%%%%%%%%%%%%%%%%%%%%%%%%%
%\subsection{Feature Suggestions}
%
%The following is a list of features which may be useful for future
%versions of this package:
%%
%\begin{itemize}
%\item
%\ldots
%\end{itemize}

%%%%%%%%%%%%%%%%%%%%%%%%%%%%%%%%%%%%%%%%%%%%%%%%%%%%%%%%%%%%%%%%%%%%%%%%%%%%%%%%
\subsection{Revision History}

%%%%%%%%%%%%%%%%%%%%%%%%%%%%%%%%%%%%%%%%
\paragraph{v2.0:} 2018/12/30

\begin{itemize}
\item
immediate forward processing
\item
added |\childdocby| mechanism
\item
manual restructured
\end{itemize}

%%%%%%%%%%%%%%%%%%%%%%%%%%%%%%%%%%%%%%%%
\paragraph{v1.6:} 2018/01/17

\begin{itemize}
\item
application for development of include files
\item
corrections to manual
\end{itemize}

%%%%%%%%%%%%%%%%%%%%%%%%%%%%%%%%%%%%%%%%
\paragraph{v1.5:} 2017/05/21

\begin{itemize}
\item
more complete structuring introduced
\item
|\childdocof| introduced
\item
|\childdoc| renamed to |\childdocmain|
\item
|\childredirect| renamed to |\childdocforward| and |\childdocforwardprefix|
and functionality expanded
\end{itemize}

%%%%%%%%%%%%%%%%%%%%%%%%%%%%%%%%%%%%%%%%
\paragraph{v1.0:} 2017/04/27

\begin{itemize}
\item
manual and install package
\item
first version published on CTAN
\end{itemize}

%%%%%%%%%%%%%%%%%%%%%%%%%%%%%%%%%%%%%%%%
\paragraph{v0.6:} 2017/04/26

\begin{itemize}
\item
redirection mechanism added
\end{itemize}

%%%%%%%%%%%%%%%%%%%%%%%%%%%%%%%%%%%%%%%%
\paragraph{v0.5:} 2017/04/26

\begin{itemize}
\item
functionality in definition file
\end{itemize}


%%%%%%%%%%%%%%%%%%%%%%%%%%%%%%%%%%%%%%%%%%%%%%%%%%%%%%%%%%%%%%%%%%%%%%%%%%%%%%%%
%%%%%%%%%%%%%%%%%%%%%%%%%%%%%%%%%%%%%%%%%%%%%%%%%%%%%%%%%%%%%%%%%%%%%%%%%%%%%%%%
%%%%%%%%%%%%%%%%%%%%%%%%%%%%%%%%%%%%%%%%%%%%%%%%%%%%%%%%%%%%%%%%%%%%%%%%%%%%%%%%
\appendix

\settowidth\MacroIndent{\rmfamily\scriptsize 000\ }

 \DocInput{childdoc.dtx}

\end{document}
%</driver>
% \fi
%
% %%%%%%%%%%%%%%%%%%%%%%%%%%%%%%%%%%%%%%%%%%%%%%%%%%%%%%%%%%%%%%%%%%%%%%%%%%%%%%
% %%%%%%%%%%%%%%%%%%%%%%%%%%%%%%%%%%%%%%%%%%%%%%%%%%%%%%%%%%%%%%%%%%%%%%%%%%%%%%
% \section{Sample}
%\iffalse
%<*samplemain>
%\fi
%
% The following presents a sample document
% with two chapters, two parts, a title page,
% a compile flag as well as three forwarding files to set the flag.
% It consists of eight |.tex| files:
% \begin{center}
% \begin{tabular}{ll}
% |cdocsamp.tex|&main file\\
% |cdocsch1.tex|&include file for chapter 1\\
% |cdocsch2.tex|&include file for chapter 2\\
% |cdocspt3.tex|&include file for part 3\\
% |cdocspt4.tex|&include file for part 4\\
% |cdocsdrf.tex|&forwarding file for main file in draft mode\\
% |cdocsfi1.tex|&forwarding file for final version of chapter 1\\
% |cdocsfi2.tex|&forwarding file for final version of chapter 2\\
% \end{tabular}
% \end{center}
% Each of the eight files can be compiled directly by the \LaTeX{} compiler.
%
% %%%%%%%%%%%%%%%%%%%%%%%%%%%%%%%%%%%%%%
% \paragraph{Main File.}
%
% The main file is called |cdocsamp.tex|.
%
% Load the \textsf{childdoc} definitions and
% declare the filename for the main document:
%    \begin{macrocode}
\input{childdoc.def}
\childdocmain{}
%    \end{macrocode}

% Optional override for |\version| flag:
%    \begin{macrocode}
%%\ifchilddoc\else\providecommand{\version}{draft}\fi
%    \end{macrocode}

% Define the default values for the |\version| flag
% (|final| for the main file and |draft| for childs):
%    \begin{macrocode}
\ifchilddoc
\providecommand{\version}{draft}
\else
\providecommand{\version}{final}
\fi
%    \end{macrocode}

% Load the standard document class:
%    \begin{macrocode}
\documentclass[12pt]{article}
%    \end{macrocode}

% Start the document body:
%    \begin{macrocode}
\begin{document}
%    \end{macrocode}

% Declare a title page.
% Print title, part of document being processed and version flag:
%    \begin{macrocode}
\addtocounter{page}{-1}
\begin{center}
{\LARGE\bfseries{}childdoc example\par}
\vspace{1cm}
\ifchilddoc
\ifchilddocmanual part\else chapter\fi:
`\childdocname' of `\childdocjob'\par
\else
main document: `\childdocjob'\par
\fi
version: \version\par
\end{center}
\newpage
%    \end{macrocode}

% Manually include selected file,
% otherwise process as usual:
%    \begin{macrocode}
\ifchilddocmanual
\section*{part `\childdocname'}
\input{\childdocname}
\else
%    \end{macrocode}

% Include the two chapters:
%    \begin{macrocode}
\include{cdocsch1}
\include{cdocsch2}
%    \end{macrocode}

% Include the two parts unless only chapters should be displayed:
%    \begin{macrocode}
\ifchilddoc\else
\section{part three}
\input{cdocspt3}
\section{part four}
\input{cdocspt4}
\fi
%    \end{macrocode}

% Process as usual until here:
%    \begin{macrocode}
\fi
%    \end{macrocode}

% End of document body:
%    \begin{macrocode}
\end{document}
%    \end{macrocode}
%\iffalse
%</samplemain>
%\fi
%
% %%%%%%%%%%%%%%%%%%%%%%%%%%%%%%%%%%%%%%
% \paragraph{Chapter Include Files.}
%
% The include files are called |cdocsch1.tex| and |cdocsch2.tex|.
%
%\iffalse
%<*samplechap1|samplechap2>
%\fi

% Optional override for |\version| flag:
%    \begin{macrocode}
%%\providecommand{\version}{final}
%    \end{macrocode}

% Include the main document:
%    \begin{macrocode}
\input{childdoc.def}
\childdocof{cdocsamp}
%    \end{macrocode}

%\iffalse
%</samplechap1|samplechap2>
%\fi
%
%\iffalse
%<*samplechap1>
%\fi
% Some text for chapter 1:
%    \begin{macrocode}
\section{one}
some text in chapter one
%    \end{macrocode}

%\iffalse
%</samplechap1>
%\fi
% Some text for chapter 2:
%\iffalse
%<*samplechap2>
%\fi
%    \begin{macrocode}
\section{two}
more text in chapter two
%    \end{macrocode}

%\iffalse
%</samplechap2>
%\fi
%
% %%%%%%%%%%%%%%%%%%%%%%%%%%%%%%%%%%%%%%
% \paragraph{Part Include Files.}
%
% The include files are called |cdocspt3.tex| and |cdocspt4.tex|.
%
%\iffalse
%<*samplepart3|samplepart4>
%\fi

% Optional override for |\version| flag:
%    \begin{macrocode}
%%\providecommand{\version}{final}
%    \end{macrocode}

% Include the main document:
%    \begin{macrocode}
\input{childdoc.def}
\childdocby{cdocsamp}
%    \end{macrocode}

%\iffalse
%</samplepart3|samplepart4>
%\fi
%
%\iffalse
%<*samplepart3>
%\fi
% Some text for part 3:
%    \begin{macrocode}
some text in part three
%    \end{macrocode}

%\iffalse
%</samplepart3>
%\fi
% Some text for part 4:
%\iffalse
%<*samplepart4>
%\fi
%    \begin{macrocode}
more text in part four
%    \end{macrocode}

%\iffalse
%</samplepart4>
%\fi
%
% %%%%%%%%%%%%%%%%%%%%%%%%%%%%%%%%%%%%%%
% \paragraph{Forwarding for a Complete Draft.}
%
% The following forwarding file |cdocsdrf.tex|
% compiles the main document in draft mode:
%\iffalse
%<*sampledraft>
%\fi
%    \begin{macrocode}
\def\version{draft}
\input{childdoc.def}
\childdocforward{cdocsamp}
%    \end{macrocode}

%\iffalse
%</sampledraft>
%\fi
%
% %%%%%%%%%%%%%%%%%%%%%%%%%%%%%%%%%%%%%%
% \paragraph{Forwarding for Final Version of the Chapters.}
%
% The following forwarding files |cdocsfn1.tex| and |cdocsfn2.tex|
% (with identical content)
% compile the final versions of the child documents
% |cdocsch1.tex| and |cdocsch2.tex|, respectively:
%\iffalse
%<*samplefinal>
%\fi
%    \begin{macrocode}
\def\version{final}
\input{childdoc.def}
\childdocforwardprefix[cdocsamp]{cdocsfn}{cdocsch}
%    \end{macrocode}

%\iffalse
%</samplefinal>
%\fi
%
% %%%%%%%%%%%%%%%%%%%%%%%%%%%%%%%%%%%%%%
% \paragraph{Command Line Processing.}
%
% The following three command lines generate the output files
% |cdocscld|, |cdocscl1| and |cdocscl2|
% which should be identical to
% |cdocsdrf|, |cdocsch1| and |cdocsfn2|, respectively:
% \begin{center}
% \begin{tabular}{l}
% |latex -jobname cdocscld \|\\
% |  "\def\version{draft}\input{childdoc.def}\childdocforward{cdocsamp}"|\\
% |latex -jobname cdocscl1 \|\\
% |  "\input{childdoc.def}\childdocforward[cdocsamp]{cdocsch1}"|\\
% |latex -jobname cdocscl2 \|\\
% |  "\def\version{final}\input{childdoc.def}\childdocforward{cdocsch2}"|
% \end{tabular}
% \end{center}
% Note that the trailing backslash on each first line
% merely continues the input to the second line
% (for convenient cut ant paste).
% Furthermore, the command |latex| can be replaced by any
% of its alternative versions such as |pdflatex|.
%
% %%%%%%%%%%%%%%%%%%%%%%%%%%%%%%%%%%%%%%%%%%%%%%%%%%%%%%%%%%%%%%%%%%%%%%%%%%%%%%
% %%%%%%%%%%%%%%%%%%%%%%%%%%%%%%%%%%%%%%%%%%%%%%%%%%%%%%%%%%%%%%%%%%%%%%%%%%%%%%
% \section{Implementation}
%\iffalse
%<*package>
%\fi
%
% This section describes the definitions file |childdoc.def|.

% The definitions cannot be loaded using |\usepackage| or |\RequirePackage|
% which has a mechanism to prevent loading a style file more than once.
% When loading the definitions by means of |\input|
% multiple instances have to be prevented manually:
%\iffalse
%This code needs to be before the `\ProvidesFile' directive
%which is defined at the beginning of this file.
%Therefore it is also placed there and commented out here.
%</package>
%<*discard>
%\fi
%    \begin{macrocode}
\ifdefined\childdocmain\endinput\fi
%    \end{macrocode}
%\iffalse
%</discard>
%<*package>
%\fi
%
% \macro{\ifchilddoc}
% \macro{\ifchilddocmanual}
% The conditional |\ifchilddoc| tells whether a
% child (true) or main (false) document is being compiled.
% The conditional |\ifchilddocmanual| tells whether
% the |\includeonly| mechanism is used (false) or
% the selection of child files must be performed manually (true).
% The definitions initialise to false:
%    \begin{macrocode}
\newif\ifchilddoc
\newif\ifchilddocmanual
%    \end{macrocode}

% \macro{\childdocname}
% \macro{\childdocjob}
% The macro |\childdocname| stores the name of the main document
% to be compiled. The macro |\childdocjob| stores the name of
% the document on which the \LaTeX{} compiler was originally invoked.
% The content of |\jobname| cannot be compared
% to filenames specified in the source due to different catcodes.
% The following code rescans |\jobname|, stores the result
% in |\childdocname| and saves a copy in |\childdocjob|:
%    \begin{macrocode}
\edef\childdocname{\scantokens\expandafter{\jobname\noexpand}}
\let\childdocjob\childdocname
%    \end{macrocode}

% \macro{\childdocdisable}
% The macro |\childdocdisable| prevents the main file
% from being processed more than once.
% At this stage, the main document command |\childdocmain|
% is assumed to be called once again where it should do nothing.
% Any subsequent call to it should prevent
% a secondary processing of the main document
% It overwrites the forwarding commands
% |\childdocof| and |\childdocforward|
% with empty macros to prevent further inclusions of the main document:
%    \begin{macrocode}
\newcommand{\childdocdisable}
{
  \renewcommand{\childdocmain}[1]{\renewcommand{\childdocmain}[1]{\endinput}}
  \renewcommand{\childdocof}[1]{}
  \renewcommand{\childdocby}[2][]{}
  \renewcommand{\childdocforward}[2][]{}
  \renewcommand{\childdocdisable}{}
}
%    \end{macrocode}

% \macro{\childdocmain}
% The macro |\childdocmain| is to be called at the top of the main file
% with nothing or the main filename (without extension) as argument.
% First, it breaks loops.
% If the argument is not empty and does not match |\childdocname|
% (which is set by the first inclusion of |childdoc.def|),
% |\ifchilddoc| is set to true, |\includeonly| is applied to the child file
% and |\jobname| is set to the main file
% (for proper handling of |.aux| files):
%    \begin{macrocode}
\newcommand{\childdocmain}[1]
{
  \childdocdisable\childdocmain{}
  \if?#1?\else
    \begingroup
      \def\childdoctmp{#1}
      \ifx\childdoctmp\childdocname
        \def\childdoctmp{}
      \else
        \def\childdoctmp
        {
          \childdoctrue
          \includeonly{\childdocname}
          \def\childdocjob{#1}
          \def\jobname{#1}
        }
      \fi
      \expandafter
    \endgroup
    \childdoctmp
  \fi
}
%    \end{macrocode}

% \macro{\childdocof}
% The command |\childdocof| redirects
% compilation to the main file |#1|.
%    \begin{macrocode}
\newcommand{\childdocof}[1]
{
  \childdocdisable
  \childdoctrue
  \includeonly{\childdocname}
  \def\jobname{#1}
  \def\childdocjob{#1}
  \input{#1}
}
%    \end{macrocode}

% \macro{\childdocby}
% The command |\childdocby| ....
%    \begin{macrocode}
\newcommand{\childdocby}[2][]
{
  \childdocdisable
  \childdoctrue
  \childdocmanualtrue
  \if?#1?\else
    \def\jobname{#2}
  \fi
  \def\childdocjob{#2}
  \input{#2}
  \endinput
}
%    \end{macrocode}

% \macro{\childdocforward}
% The command |\childdocforward| redirects
% compilation to the main file or
% (if the optional argument is given) a child file.
% Parameters are set as if the main file
% or a child file starting with |\childdocof| was compiled.
% Then compilation is handed over to the main file:
%    \begin{macrocode}
\newcommand{\childdocforward}[2][]
{
  \begingroup
    \if?#1?
      \def\childdoctmp
      {
        \def\childdocname{#2}
        \def\childdocjob{#2}
        \def\jobname{#2}
        \input{#2}
        \endinput
      }
    \else
      \def\childdoctmp
      {
        \childdocdisable
        \def\childdocname{#2}
        \childdoctrue
        \includeonly{#2}
        \def\childdocjob{#1}
        \def\jobname{#1}
        \input{#1}
        \endinput
      }
    \fi
    \expandafter
  \endgroup
  \childdoctmp
}
%    \end{macrocode}

% \macro{\childdocforwardprefix}
% The command |\childdocforwardprefix| redirects
% compilation to the main or a child file by means of a pattern.
% The prefix |#1| in the current filename is replaced by |#2|
% and the suffix of the current filename is kept
% (it is assumed that the filename does not contain the substring `|~~~|'
% which is used as a delimiter).
% Compilation is handed over to the new file by |\childdocforward|:
%    \begin{macrocode}
\newcommand{\childdocforwardprefix}[3][]
{
  \begingroup
    \def\childdocextract #2##1~~~{\def\childdoctmp{\childdocforward[#1]{#3##1}}}
    \expandafter\childdocextract\childdocname~~~
    \expandafter
  \endgroup
  \childdoctmp
}
%    \end{macrocode}

% \macro{\childdoc}
% The deprecated macro |\childdoc| is a legacy version of |\childdocmain|:
%    \begin{macrocode}
\newcommand{\childdoc}{\childdocmain}
%    \end{macrocode}

% \macro{\childdocredirect}
% The deprecated macro |\childdocredirect| is a legacy version
% of |\childdocforward| and |\childdocforwardprefix|:
%    \begin{macrocode}
\newcommand{\childdocredirect}[2][]
{
  \begingroup
    \if?#1?
      \def\childdoctmp{\childdocforward{#2}}
    \else
      \def\childdoctmp{\childdocforwardprefix{#1}{#2}}
    \fi
    \expandafter
  \endgroup
  \childdoctmp
}
%    \end{macrocode}

%\iffalse
%</package>
%\fi
%
\endinput
|\\
|\childdocof{|\textit{main}|}|\\
\end{tabular}
\end{center}
at the top of every child file \textit{child}
which is included by |\include{|\textit{child}|}|
from within the main file
(or at least for those files to be compiled individually).
The argument \textit{main} must be the filename of the main file.

There are a couple of
considerations in setting up the main and child documents:

%%%%%%%%%%%%%%%%%%%%%%%%%%%%%%%%%%%%%%%%
\paragraph{Restrictions.}

Please note the following restrictions:
\begin{itemize}
\item
|\childdocmain| must be called with one argument \textit{main}
to ensure compatibility with earlier version of the package.
It must either be empty (|\childdocmain{}|)
or precisely match the filename of the main file in which it is specified.
See \secref{sec:detection} for further information.
\item
The filename \textit{main} must be specified without the |.tex| extension.
\item
The filename \textit{main} is case sensitive
(even in case-insensitive file systems)
due to internal string comparison.
\item
The argument \textit{main} should be fully expanded, it cannot be a macro.
\item
Subdirectories and special characters should be avoided in filenames.
\item
The command |\childdocmain{|\textit{main}|}| must be followed by a whitespace.
It should not be followed immediately by another command
or by a comment mark `|%|'.
This is because the \TeX{} parser reads the token immediately following
the argument of |\childdocmain| and puts it
at the beginning of every child section;
however, a white\-space is ignored.
\end{itemize}

%%%%%%%%%%%%%%%%%%%%%%%%%%%%%%%%%%%%%%%%
\paragraph{Content of Main File.}

It is advisable to place all content in the child files included by |\include|.
Any output contained in the main file will appear in all child documents
unless suppressed manually;
it cannot be suppressed automatically by the |\includeonly| directive
and thus should normally be avoided.
A method to include some content in the main file
by means of conditional processing is described in \secref{sec:conditional}.

%%%%%%%%%%%%%%%%%%%%%%%%%%%%%%%%%%%%%%%%
\paragraph{Page Numbering.}

When only a part of the document is compiled,
the appropriate numbering of pages
(as well as other status parameters)
is determined from the |.aux| files.
The latter contain information from previous passes.
However this information needs to propagate through
all intermediate child documents.
Therefore the page numbering in child documents may well
be inconsistent until the complete document is compiled at least once.

A useful (if unconventional) way to always ensure a consistent
page numbering is to restart the numbering in each child document
and denote the pages by `\textit{child}|.|\textit{page}'
where \textit{child} represents the chapter/section number of the child file.
This can be achieved by the command
|\numberwithin{page}{|\textit{child}|}|
of the \textsf{amsmath} package
where \textit{child} can be |chapter| or |section|
depending on the chosen structuring.
Alternatively, one can modify the macro |\thepage| appropriately
and reset the counter |page| at the start of each child file.

%%%%%%%%%%%%%%%%%%%%%%%%%%%%%%%%%%%%%%%%%%%%%%%%%%%%%%%%%%%%%%%%%%%%%%%%%%%%%%%%
\subsection{Conditional Processing}
\label{sec:conditional}

The package provides a mechanism to compile different versions
of a document. To customise the versions further some conditional processing
can come in handy to distinguish which version is being compiled.
The package provides two macros to describe the compilation context:

%%%%%%%%%%%%%%%%%%%%%%%%%%%%%%%%%%%%%%%%
\DescribeMacro{\ifchilddoc}
The conditional |\ifchilddoc| distinguishes between the compilation of
child documents and the main document:
%
\begin{center}
|\ifchilddoc |\textit{child-code}| |[|\||else |\textit{main-code}]| \||fi|
\end{center}

%%%%%%%%%%%%%%%%%%%%%%%%%%%%%%%%%%%%%%%%
\DescribeMacro{\childdocname}
\DescribeMacro{\childdocjob}
The macro |\childdocname| contains the filename (without extension)
of the main or child file being processed.
Note that |\childdocjob| will always contain the name of the main file.

%%%%%%%%%%%%%%%%%%%%%%%%%%%%%%%%%%%%%%%%
\paragraph{Title Page.}

Conditional processing can be used to include a title or banner page
in the main document when proper precautions are taken.
Importantly, the code in the main file should ensure that the page counter
(as well as other status parameters which are stored in the |.aux| files)
takes the same value after the conditional processing.
Otherwise the page numbers may take divergent values
depending on which part is compiled.

For example, a title page could be declared by:
%
\begin{center}
\begin{tabular}{l}
|\ifchilddoc\||else|\\
|\addtocounter{page}{-1}|\\
\textit{code for title page}\\
|\newpage|\\
|\||fi|
\end{tabular}
\end{center}
%
A banner page for the child documents can be generated by:
%
\begin{center}
\begin{tabular}{l}
|\ifchilddoc|\\
|\addtocounter{page}{-1}|\\
\textit{code for banner page}\\
|\newpage|\\
|\||fi|
\end{tabular}
\end{center}
%
Here one could write a message such as:
\begin{center}
|This is the part \childdocname{} of \childdocjob{}.|
\end{center}

%%%%%%%%%%%%%%%%%%%%%%%%%%%%%%%%%%%%%%%%%%%%%%%%%%%%%%%%%%%%%%%%%%%%%%%%%%%%%%%%
\subsection{Flags}
\label{sec:flags}

The package makes it easy to generate different versions
of the main or child documents.
To this end compilation flags can be defined
and assigned different default values.
They will be particularly useful in conjunction
with the forwarding mechanism described in \secref{sec:forward}.

For example, it may be useful to have a flag |\version|
which can be set to |draft| or |final|.
The document source will contain some conditional code
depending on the value of |\version|.
Suppose further, the flag should default to |final| for the main file
and to |draft| for child files
which is a natural assignment for editing the document.
This is achieved by placing the following code
in the preamble of the main document
(below the |\childdocmain| directive):
%
\begin{center}
\begin{tabular}{l}
|\ifchilddoc|\\
|\providecommand{\version}{draft}|\\
|\||else|\\
|\providecommand{\version}{final}|\\
|\||fi|
\end{tabular}
\end{center}
%
The definition by |\providecommand| makes sure
that previous definitions are not overwritten.
Further statements |\providecommand{\version}{...}|
can thus be added before the above code to override it.

For the main file, one might add a line
(between |\childdocmain| and the above block)
%
\begin{center}
|%\ifchilddoc\||else\providecommand{\version}{draft}\||fi|
\end{center}
%
which can be uncommented to produce a draft version.
Likewise one can add a line to the very top of a child file
(above the |\childdocof{|\textit{main}|}| directive)
%
\begin{center}
|%\providecommand{\version}{final}|
\end{center}
%
which can be uncommented to produce the final version of this child document.

%%%%%%%%%%%%%%%%%%%%%%%%%%%%%%%%%%%%%%%%%%%%%%%%%%%%%%%%%%%%%%%%%%%%%%%%%%%%%%%%
\subsection{Forwarding}
\label{sec:forward}

Different versions of the main or child documents
using compilation flags as described in \secref{sec:flags}
can be (permanently) stored in different files
for convenient compilation, viewing and distribution.
To this end, the package defines a command
to pass on compilation to a different file:

%%%%%%%%%%%%%%%%%%%%%%%%%%%%%%%%%%%%%%%%
\DescribeMacro{\childdocforward}
The command |\childdocforward| redirects processing to
another source file:
%
\begin{center}
\begin{tabular}{l}
|% \iffalse
%
% childdoc.dtx Copyright (C) 2017-2018 Niklas Beisert
%
% This work may be distributed and/or modified under the
% conditions of the LaTeX Project Public License, either version 1.3
% of this license or (at your option) any later version.
% The latest version of this license is in
%   http://www.latex-project.org/lppl.txt
% and version 1.3 or later is part of all distributions of LaTeX
% version 2005/12/01 or later.
%
% This work has the LPPL maintenance status `maintained'.
%
% The Current Maintainer of this work is Niklas Beisert.
%
% This work consists of the files childdoc.dtx and childdoc.ins
% and the derived files childdoc.def and cdocsamp.tex with
% cdocsch1.tex, cdocsch2.tex, cdocsdrf.tex, cdocsfn1.tex, cdocsfn2.tex.
%
%<package>\ifdefined\childdocmain\endinput\fi
%<package>\ProvidesFile{childdoc.def}[2018/12/30 v2.0 child document driver]
%<samplemain>\ProvidesFile{cdocsamp.tex}[2018/12/30 v2.0 sample for childdoc]
%<*driver>
%\ProvidesFile{childdoc.drv}[2018/12/30 v2.0 childdoc reference manual file]
\PassOptionsToClass{10pt,a4paper}{article}
\documentclass{ltxdoc}

\usepackage[margin=35mm]{geometry}
\usepackage{hyperref}
\usepackage{hyperxmp}
\usepackage[usenames]{color}

\hypersetup{colorlinks=true}
\hypersetup{pdfstartview=FitH}
\hypersetup{pdfpagemode=UseNone}
\hypersetup{pdfsource={}}
\hypersetup{pdflang={en-UK}}
\hypersetup{pdfcopyright={Copyright 2017-2018 Niklas Beisert.
  This work may be distributed and/or modified under the
  conditions of the LaTeX Project Public License, either version 1.3
  of this license or (at your option) any later version.}}
\hypersetup{pdflicenseurl={http://www.latex-project.org/lppl.txt}}
\hypersetup{pdfcontactaddress={ETH Zurich, ITP, HIT K,
  Wolfgang-Pauli-Strasse 27}}
\hypersetup{pdfcontactpostcode={8093}}
\hypersetup{pdfcontactcity={Zurich}}
\hypersetup{pdfcontactcountry={Switzerland}}
\hypersetup{pdfcontactemail={nbeisert@itp.phys.ethz.ch}}
\hypersetup{pdfcontacturl={http://people.phys.ethz.ch/\xmptilde nbeisert/}}

\newcommand{\secref}[1]{\hyperref[#1]{section \ref*{#1}}}

\parskip1ex
\parindent0pt
\let\olditemize\itemize
\def\itemize{\olditemize\parskip0pt}

\begin{document}

\title{The \textsf{childdoc} Package}
\hypersetup{pdftitle={The childdoc Package}}
\author{Niklas Beisert\\[2ex]
  Institut f\"ur Theoretische Physik\\
  Eidgen\"ossische Technische Hochschule Z\"urich\\
  Wolfgang-Pauli-Strasse 27, 8093 Z\"urich, Switzerland\\[1ex]
  \href{mailto:nbeisert@itp.phys.ethz.ch}
  {\texttt{nbeisert@itp.phys.ethz.ch}}}
\hypersetup{pdfauthor={Niklas Beisert}}
\hypersetup{pdfsubject={Manual for the LaTeX2e Package childdoc}}
\date{30 December 2018, \textsf{v2.0}}
\maketitle

\begin{abstract}\noindent
\textsf{childdoc} is a \LaTeXe{} package
that enables the direct compilation
of document sections included by |\include|
to individual files.
\end{abstract}

\begingroup
\parskip0ex
\tableofcontents
\endgroup

%%%%%%%%%%%%%%%%%%%%%%%%%%%%%%%%%%%%%%%%%%%%%%%%%%%%%%%%%%%%%%%%%%%%%%%%%%%%%%%%
%%%%%%%%%%%%%%%%%%%%%%%%%%%%%%%%%%%%%%%%%%%%%%%%%%%%%%%%%%%%%%%%%%%%%%%%%%%%%%%%
\section{Introduction}

\LaTeX{} provides a mechanism to structure a large document (such as a book)
into a main file and several child files (containing the chapters)
using the |\include| command.
This mechanism is beneficial for documents
which span hundreds of pages in order to
make the source file(s) more manageable.
Moreover, compilation can be restricted to
selected child files by means of the |\includeonly| command.
The latter feature can be used to reduce the compilation time while editing
(this was significantly more useful in the earlier days of \LaTeX{})
or to generate a smaller document which is easier to navigate.
Another application of |\includeonly| is to generate
documents consisting of selected parts of the complete document.

However, there are a few drawbacks of the plain |\include| mechanism:
\begin{itemize}
\item
The child files cannot be compiled on their own,
they can only be compiled via the main file.
A naive editing environment
(such as a text editor with an option
to have the current file processed by \LaTeX)
may require one to switch to the main file before compiling;
attempting to compile the child file produces errors.
\item
The main file must be modified (each time)
to adjust the |\includeonly| command
to the present needs. This easily leaves the main file in a messy state.
\item
The generated document will always carry the filename
of the main document. This is inconvenient if
several child files are to be compiled and
to be kept for distribution.
\end{itemize}

The present package provides a simple interface
to make child files individually compilable by \LaTeX{}.
Compiling a child file then has the same effect as compiling
the main file with an |\includeonly| command
to select the appropriate child.
Moreover the generated document will carry the name of the child
rather than the main file.
This resolves all three above issues.

This feature is meant to make the editing of books,
thesis documents and lecture notes somewhat more convenient.
However, the package can also be used efficiently for
composing a series of documents (such as exercise sheets)
which are typically distributed individually.
It then assists the author in generating the individual documents
(potentially in different versions)
as well as a document containing the collected series.
Another application is in developing style files
or other kinds of included material
where compilation of the style file could redirect
to a sample or test file.

%%%%%%%%%%%%%%%%%%%%%%%%%%%%%%%%%%%%%%%%%%%%%%%%%%%%%%%%%%%%%%%%%%%%%%%%%%%%%%%%
%%%%%%%%%%%%%%%%%%%%%%%%%%%%%%%%%%%%%%%%%%%%%%%%%%%%%%%%%%%%%%%%%%%%%%%%%%%%%%%%
\section{Usage}

First of all, the package \textsf{childdoc} is \emph{not} a standard
\LaTeXe{} |.sty| style file! Therefore it needs to be invoked in
a non-standard way.

%%%%%%%%%%%%%%%%%%%%%%%%%%%%%%%%%%%%%%%%%%%%%%%%%%%%%%%%%%%%%%%%%%%%%%%%%%%%%%%%
\subsection{Included Files}
\label{sec:include}

%%%%%%%%%%%%%%%%%%%%%%%%%%%%%%%%%%%%%%%%
\DescribeMacro{\childdocmain}
To use the package, add the commands
\begin{center}
\begin{tabular}{l}
|\input{childdoc.def}|\\
|\childdocmain{}|\\
\end{tabular}
\end{center}
at the very top of the main \LaTeX{} file,
in particular \emph{before} the |\documentclass| statement!
The argument of |\childdocmain| should be left empty
(but it must be present).

%%%%%%%%%%%%%%%%%%%%%%%%%%%%%%%%%%%%%%%%
\DescribeMacro{\childdocof}
Furthermore, add the commands
\begin{center}
\begin{tabular}{l}
|\input{childdoc.def}|\\
|\childdocof{|\textit{main}|}|\\
\end{tabular}
\end{center}
at the top of every child file \textit{child}
which is included by |\include{|\textit{child}|}|
from within the main file
(or at least for those files to be compiled individually).
The argument \textit{main} must be the filename of the main file.

There are a couple of
considerations in setting up the main and child documents:

%%%%%%%%%%%%%%%%%%%%%%%%%%%%%%%%%%%%%%%%
\paragraph{Restrictions.}

Please note the following restrictions:
\begin{itemize}
\item
|\childdocmain| must be called with one argument \textit{main}
to ensure compatibility with earlier version of the package.
It must either be empty (|\childdocmain{}|)
or precisely match the filename of the main file in which it is specified.
See \secref{sec:detection} for further information.
\item
The filename \textit{main} must be specified without the |.tex| extension.
\item
The filename \textit{main} is case sensitive
(even in case-insensitive file systems)
due to internal string comparison.
\item
The argument \textit{main} should be fully expanded, it cannot be a macro.
\item
Subdirectories and special characters should be avoided in filenames.
\item
The command |\childdocmain{|\textit{main}|}| must be followed by a whitespace.
It should not be followed immediately by another command
or by a comment mark `|%|'.
This is because the \TeX{} parser reads the token immediately following
the argument of |\childdocmain| and puts it
at the beginning of every child section;
however, a white\-space is ignored.
\end{itemize}

%%%%%%%%%%%%%%%%%%%%%%%%%%%%%%%%%%%%%%%%
\paragraph{Content of Main File.}

It is advisable to place all content in the child files included by |\include|.
Any output contained in the main file will appear in all child documents
unless suppressed manually;
it cannot be suppressed automatically by the |\includeonly| directive
and thus should normally be avoided.
A method to include some content in the main file
by means of conditional processing is described in \secref{sec:conditional}.

%%%%%%%%%%%%%%%%%%%%%%%%%%%%%%%%%%%%%%%%
\paragraph{Page Numbering.}

When only a part of the document is compiled,
the appropriate numbering of pages
(as well as other status parameters)
is determined from the |.aux| files.
The latter contain information from previous passes.
However this information needs to propagate through
all intermediate child documents.
Therefore the page numbering in child documents may well
be inconsistent until the complete document is compiled at least once.

A useful (if unconventional) way to always ensure a consistent
page numbering is to restart the numbering in each child document
and denote the pages by `\textit{child}|.|\textit{page}'
where \textit{child} represents the chapter/section number of the child file.
This can be achieved by the command
|\numberwithin{page}{|\textit{child}|}|
of the \textsf{amsmath} package
where \textit{child} can be |chapter| or |section|
depending on the chosen structuring.
Alternatively, one can modify the macro |\thepage| appropriately
and reset the counter |page| at the start of each child file.

%%%%%%%%%%%%%%%%%%%%%%%%%%%%%%%%%%%%%%%%%%%%%%%%%%%%%%%%%%%%%%%%%%%%%%%%%%%%%%%%
\subsection{Conditional Processing}
\label{sec:conditional}

The package provides a mechanism to compile different versions
of a document. To customise the versions further some conditional processing
can come in handy to distinguish which version is being compiled.
The package provides two macros to describe the compilation context:

%%%%%%%%%%%%%%%%%%%%%%%%%%%%%%%%%%%%%%%%
\DescribeMacro{\ifchilddoc}
The conditional |\ifchilddoc| distinguishes between the compilation of
child documents and the main document:
%
\begin{center}
|\ifchilddoc |\textit{child-code}| |[|\||else |\textit{main-code}]| \||fi|
\end{center}

%%%%%%%%%%%%%%%%%%%%%%%%%%%%%%%%%%%%%%%%
\DescribeMacro{\childdocname}
\DescribeMacro{\childdocjob}
The macro |\childdocname| contains the filename (without extension)
of the main or child file being processed.
Note that |\childdocjob| will always contain the name of the main file.

%%%%%%%%%%%%%%%%%%%%%%%%%%%%%%%%%%%%%%%%
\paragraph{Title Page.}

Conditional processing can be used to include a title or banner page
in the main document when proper precautions are taken.
Importantly, the code in the main file should ensure that the page counter
(as well as other status parameters which are stored in the |.aux| files)
takes the same value after the conditional processing.
Otherwise the page numbers may take divergent values
depending on which part is compiled.

For example, a title page could be declared by:
%
\begin{center}
\begin{tabular}{l}
|\ifchilddoc\||else|\\
|\addtocounter{page}{-1}|\\
\textit{code for title page}\\
|\newpage|\\
|\||fi|
\end{tabular}
\end{center}
%
A banner page for the child documents can be generated by:
%
\begin{center}
\begin{tabular}{l}
|\ifchilddoc|\\
|\addtocounter{page}{-1}|\\
\textit{code for banner page}\\
|\newpage|\\
|\||fi|
\end{tabular}
\end{center}
%
Here one could write a message such as:
\begin{center}
|This is the part \childdocname{} of \childdocjob{}.|
\end{center}

%%%%%%%%%%%%%%%%%%%%%%%%%%%%%%%%%%%%%%%%%%%%%%%%%%%%%%%%%%%%%%%%%%%%%%%%%%%%%%%%
\subsection{Flags}
\label{sec:flags}

The package makes it easy to generate different versions
of the main or child documents.
To this end compilation flags can be defined
and assigned different default values.
They will be particularly useful in conjunction
with the forwarding mechanism described in \secref{sec:forward}.

For example, it may be useful to have a flag |\version|
which can be set to |draft| or |final|.
The document source will contain some conditional code
depending on the value of |\version|.
Suppose further, the flag should default to |final| for the main file
and to |draft| for child files
which is a natural assignment for editing the document.
This is achieved by placing the following code
in the preamble of the main document
(below the |\childdocmain| directive):
%
\begin{center}
\begin{tabular}{l}
|\ifchilddoc|\\
|\providecommand{\version}{draft}|\\
|\||else|\\
|\providecommand{\version}{final}|\\
|\||fi|
\end{tabular}
\end{center}
%
The definition by |\providecommand| makes sure
that previous definitions are not overwritten.
Further statements |\providecommand{\version}{...}|
can thus be added before the above code to override it.

For the main file, one might add a line
(between |\childdocmain| and the above block)
%
\begin{center}
|%\ifchilddoc\||else\providecommand{\version}{draft}\||fi|
\end{center}
%
which can be uncommented to produce a draft version.
Likewise one can add a line to the very top of a child file
(above the |\childdocof{|\textit{main}|}| directive)
%
\begin{center}
|%\providecommand{\version}{final}|
\end{center}
%
which can be uncommented to produce the final version of this child document.

%%%%%%%%%%%%%%%%%%%%%%%%%%%%%%%%%%%%%%%%%%%%%%%%%%%%%%%%%%%%%%%%%%%%%%%%%%%%%%%%
\subsection{Forwarding}
\label{sec:forward}

Different versions of the main or child documents
using compilation flags as described in \secref{sec:flags}
can be (permanently) stored in different files
for convenient compilation, viewing and distribution.
To this end, the package defines a command
to pass on compilation to a different file:

%%%%%%%%%%%%%%%%%%%%%%%%%%%%%%%%%%%%%%%%
\DescribeMacro{\childdocforward}
The command |\childdocforward| redirects processing to
another source file:
%
\begin{center}
\begin{tabular}{l}
|\input{childdoc.def}|\\
|\childdocforward[|\textit{main}|]{|\textit{dest}|}|\\
\end{tabular}
\end{center}
%
The argument \textit{dest} is the destination file
(without extension).
It should be the main file or one of the child files.
Note that further \textsf{childdoc} directives
such as |\childdocof| and |\childdocforward|
in the indicated file will be processed in this form.
The optional argument \textit{main}
passes on directly to the main file \textit{main}
while pretending to compile the child \textit{dest}.
This form behaves as if \textit{dest}
issues |\childdocof{|\textit{main}|}| right away,
and no further \textsf{childdoc} directives will be processed.

%%%%%%%%%%%%%%%%%%%%%%%%%%%%%%%%%%%%%%%%
\DescribeMacro{\...prefix}
In the alternative form |\childdocforwardprefix|,
%
\begin{center}
\begin{tabular}{l}
|\input{childdoc.def}|\\
|\childdocforwardprefix[|\textit{main}|]{|\textit{prefix}|}{|\textit{dest}|}|
\end{tabular}
\end{center}
%
the destination file is determined by a pattern
depending on the current file:
To make this work, the current file must be called
`{\textit{prefix}\hspace{0.2em}\textit{suffix}}'
with \textit{prefix} matching precisely the argument.
Processing is then passed on to the file
`{\textit{dest}\hspace{0.2em}\textit{suffix}}'.
Surely, the same effect is achieved by
directly specifying the
argument `{\textit{dest}\hspace{0.2em}\textit{suffix}}'
in the first form.
However, that requires to set up a different file
for each child. With the alternative form of the command
all these files can have exactly the same content
which simplifies setting them up and maintaining them.

For example, the following file |draft.tex|
with a compilation flag |\version| as described in \secref{sec:flags}
compiles the main document as a draft:
%
\begin{center}
\begin{tabular}{l}
|\def\version{draft}|\\
|\input{childdoc.def}|\\
|\childdocforward{|\textit{main}|}|
\end{tabular}
\end{center}
%
Likewise, the following files |final|\textit{nn}|.tex|
compile the final version of the child document
|child|\textit{nn}|.tex|:
%
\begin{center}
\begin{tabular}{l}
|\def\version{final}|\\
|\input{childdoc.def}|\\
|\childdocforwardprefix{final}{child}|
\end{tabular}
\end{center}
%

Note that when several versions of a main file and/or of each child file
are to be generated, it may be convenient to set up a |Makefile| or
shell script to automatise the process.

%%%%%%%%%%%%%%%%%%%%%%%%%%%%%%%%%%%%%%%%%%%%%%%%%%%%%%%%%%%%%%%%%%%%%%%%%%%%%%%%
\subsection{Command Line Processing}
\label{sec:commandline}

The effect of redirection files can also be achieved by invoking
the \LaTeX{} compiler with a more elaborate command line.
Most conveniently this should be done as part
of a shell script or a |Makefile|.

When using \textsf{childdoc} in the main file, the following
command lines effectively perform a redirection
(note that depending on the shell being used,
backslashes may have to be doubled: `|\|' $\to$ `|\\|'):
%
\begin{center}
|... -jobname "|\textit{target}|" |\\|"|[\textit{flags}]%
|\input{childdoc.def}\childdocforward[|\textit{main}|]{|\textit{dest}|}"|
\end{center}
%
Here \textit{target} is the name of the output file,
\textit{main} is the name of the main file
and \textit{dest} is the name of the main or child file to be processed
(all filenames without extensions).
The optional argument \textit{main} can be omitted
if \textit{main} matches \textit{dest}.
Optionally, compilation \textit{flags} can be defined via |\def| commands.
This command line makes the \TeX{} engine believe
it is compiling the file \textit{target}
whose content is specified as the latter parameter.
The provided code then forwards the processing to
\textit{main} or \textit{dest} as described in \secref{sec:forward}.

%%%%%%%%%%%%%%%%%%%%%%%%%%%%%%%%%%%%%%%%%%%%%%%%%%%%%%%%%%%%%%%%%%%%%%%%%%%%%%%%
\subsection{Include by Input}
\label{sec:input}

Including child documents by |\include| has some restrictions by design.
Most notably, the content of a child document always occupies
its own set of pages; pages cannot be shared between child documents.
Usually, this behaviour makes perfect sense
because each child document contain an essential part of the document.
However, in some situations it may be desirable to compose
a document from a collection of parts
without having mandatory page breaks between then.
For this case, the package
provides a mechanism to include parts
by |\input| which can also be processed individually.
However, by construction this mechanism
requires manual handling of the content to be output.

%%%%%%%%%%%%%%%%%%%%%%%%%%%%%%%%%%%%%%%%
\DescribeMacro{\ifchilddocmanual}
The main file should be prepared as usual, see \secref{sec:include}.
However, the document body must make a distinction
between processing of an individual part and of the main document, e.g.:
%
\begin{center}
\begin{tabular}{l}
|\ifchilddocmanual|\\
|\input{\childdocname}|\\
|\||else|\\
\textit{document body with }|\input{|\textit{part}|}|\\
|\||fi|
\end{tabular}
\end{center}
%
The conditional |\ifchilddocmanual| is true whenever
a part to be included by |\input| is being compiled,
and the name of the part is stored in |\childdocname|.

%%%%%%%%%%%%%%%%%%%%%%%%%%%%%%%%%%%%%%%%
\DescribeMacro{\childdocby}
Each part to be included by |\input| should start with:
%
\begin{center}
\begin{tabular}{l}
|\input{childdoc.def}|\\
|\childdocby{|\textit{main}|}|\\
\end{tabular}
\end{center}
%
The directive |\childdocby| is similar to |\childdocof|
described in \secref{sec:include},
but the subsequent selection of content must be done manually.
To that end, both |\ifchilddoc| and |\ifchilddocmanual|
will be true upon processing of a part,
and the name of the part is stored in |\childdocname|.
Note that |\jobname| will be set to the filename of the current part
so that each part receives an individual |.aux| file
that does not interfere with the |.aux| file(s) of the main document.
This behaviour can be altered by the alternative form
|\childdocby[*]{|\textit{main}|}| (with a non-empty optional argument)
which uses the |.aux| file of the main document
by setting |\jobname| to \textit{main}.

%%%%%%%%%%%%%%%%%%%%%%%%%%%%%%%%%%%%%%%%%%%%%%%%%%%%%%%%%%%%%%%%%%%%%%%%%%%%%%%%
\subsection{Driver Development}
\label{sec:driver}

The \textsf{childdoc} mechanism can also be use for the development
of definition files such as \LaTeX{} styles or classes.
This case differs from the above setup with multiple parts
included by |\include| in that no |\includeonly| should be invoked.
This can be achieved by starting the include file
(before |\ProvidesPackage|) with:
%
\begin{center}
\begin{tabular}{l}
|\input{childdoc.def}|\\
|\childdocforward{|\textit{main}|}|\\
\end{tabular}
\end{center}
%
or alternatively with:
%
\begin{center}
\begin{tabular}{l}
|\input{childdoc.def}|\\
|\childdocby{|\textit{main}|}|\\
\end{tabular}
\end{center}
%
Both forms have slightly different effects as described above.
The main file is prepared as usual, see \secref{sec:include}.

%%%%%%%%%%%%%%%%%%%%%%%%%%%%%%%%%%%%%%%%%%%%%%%%%%%%%%%%%%%%%%%%%%%%%%%%%%%%%%%%
\subsection{Legacy Detection}
\label{sec:detection}

The directive |\childdocmain| in the main file can detect
whether the complete document or merely a child is to be compiled
even without using the directive |\childdocof|.
This method is deprecated because it is less robust
and there is no compelling reason to use it;
it is merely provided for backward compatibility
and it may be removed in future versions.

If the detection mechanism is to be used,
it is mandatory to correctly specify
the filename of the main file as the argument of |\childdocmain|:
%
\begin{center}
\begin{tabular}{l}
|\input{childdoc.def}|\\
|\childdocmain{|\textit{main}|}|\\
\end{tabular}
\end{center}
%
If |\jobname| does not match the argument \textit{main} of |\childdocmain|,
it is assumed that |\jobname| points to the child file to be compiled.
When using |\childdocmain| with the main file specified as argument,
it suffices to start a child file
with just |\input{|\textit{main}|}|
without loading of the package and using |\childdocof|.
If instead all processing is done
with the appropriate \textsf{childdoc} directives,
the argument of \textit{main} of |\childdocmain| can be empty.

An alternative version of the command line processing described
in \secref{sec:commandline} using the detection mechanism reads:
%
\begin{center}
|... -jobname "|\textit{target}|" "|[\textit{flags}]%
[|\def\jobname{|\textit{dest}|}|]|\input{|\textit{main}|}"|
\end{center}

%%%%%%%%%%%%%%%%%%%%%%%%%%%%%%%%%%%%%%%%%%%%%%%%%%%%%%%%%%%%%%%%%%%%%%%%%%%%%%%%
\subsection{Manual Code}
\label{sec:manual}

In case one cannot be certain whether the definitions file |childdoc.def|
is installed on the target \TeX{} distribution
and one prefers not to ship it,
it is conceivable to paste a few relevant commands into the sources.

To that end, drop all statements |\input{childdoc.def}|
and perform the replacements as outlined below.
Instead of |\childdocmain{|\textit{main}|}| add the following code
to the top of the main file:
%
\begin{center}
\begin{tabular}{l}
|\||ifdefined\childdocname\endinput\||fi\newif\ifchilddoc|\\
|\edef\childdocname{\scantokens\expandafter{\jobname\noexpand}}|\\
|\def\childdocmain{|\textit{main}|}\||ifx\childdocmain\childdocname\||else|\\
|\childdoctrue\includeonly{\childdocname}\let\jobname\childdocmain\||fi|\\
\end{tabular}
\end{center}
%
Instead of |\childdocof{|\textit{main}|}| just include the main file
at the top of each child file:
%
\begin{center}
|\input{|\textit{main}|}|
\end{center}
%
A simple redirection |\childdocforward{|\textit{dest}|}| is achieved by:
%
\begin{center}
|\def\jobname{|\textit{dest}|}\input{\jobname}|
\end{center}
%
The redirection with prefix
|\childdocforwardprefix[|\textit{prefix}|]{|\textit{dest}|}|
is accomplished by:
%
\begin{center}
\begin{tabular}{l}
|{\edef\jobname{\scantokens\expandafter{\jobname\noexpand}}|\\
|\def\redirectjob |\textit{prefix}|#1~~~{\gdef\jobname{|\textit{dest}|#1}}|\\
|\expandafter\redirectjob\jobname~~~}\input{\jobname}|
\end{tabular}
\end{center}

In an alternative approach,
child documents can be compiled by a specific command line
without additional code or specific definitions:
%
\begin{center}
|... -jobname "|\textit{target}|" "|[\textit{flags}]%
|\includeonly{|\textit{dest}|}\input{|\textit{main}|}"|
\end{center}
%

%%%%%%%%%%%%%%%%%%%%%%%%%%%%%%%%%%%%%%%%%%%%%%%%%%%%%%%%%%%%%%%%%%%%%%%%%%%%%%%%
%%%%%%%%%%%%%%%%%%%%%%%%%%%%%%%%%%%%%%%%%%%%%%%%%%%%%%%%%%%%%%%%%%%%%%%%%%%%%%%%
\section{Information}

%%%%%%%%%%%%%%%%%%%%%%%%%%%%%%%%%%%%%%%%%%%%%%%%%%%%%%%%%%%%%%%%%%%%%%%%%%%%%%%%
\subsection{Copyright}

Copyright \copyright{} 2017--2018 Niklas Beisert

This work may be distributed and/or modified under the
conditions of the \LaTeX{} Project Public License, either version 1.3
of this license or (at your option) any later version.
The latest version of this license is in
  \url{http://www.latex-project.org/lppl.txt}
and version 1.3 or later is part of all distributions of \LaTeX{}
version 2005/12/01 or later.

This work has the LPPL maintenance status `maintained'.

The Current Maintainer of this work is Niklas Beisert.

This work consists of the files |README.txt|, |childdoc.ins| and |childdoc.dtx|
as well as the derived files |childdoc.def|, |cdocsamp.tex|
with |cdocsch1.tex|, |cdocsch2.tex|, |cdocspt3.tex|, |cdocspt4.tex|,
|cdocsdrf.tex|, |cdocsfn1.tex|, |cdocsfn2.tex|
as well as |childdoc.pdf|.

%%%%%%%%%%%%%%%%%%%%%%%%%%%%%%%%%%%%%%%%%%%%%%%%%%%%%%%%%%%%%%%%%%%%%%%%%%%%%%%%
\subsection{Files and Installation}

The package consists of the files:
%
\begin{center}
\begin{tabular}{ll}
    |README.txt|   & readme file \\
    |childdoc.ins| & installation file \\
    |childdoc.dtx| & source file \\
    |childdoc.def| & definition file \\
    |cdocsamp.tex| & sample main file \\
    |cdocsch1.tex| & sample include file \\
    |cdocsch2.tex| & sample include file \\
    |cdocspt3.tex| & sample part file \\
    |cdocspt4.tex| & sample part file \\
    |cdocsdrf.tex| & sample redirection file \\
    |cdocsfn1.tex| & sample redirection file \\
    |cdocsfn2.tex| & sample redirection file \\
    |childdoc.pdf| & manual
\end{tabular}
\end{center}
%
The distribution consists of the files
|README.txt|, |childdoc.ins| and |childdoc.dtx|.
%
\begin{itemize}
\item
Run (pdf)\LaTeX{} on |childdoc.dtx|
to compile the manual |childdoc.pdf| (this file).
\item
Run \LaTeX{} on |childdoc.ins| to create the definitions file |childdoc.def|
and the sample |cdocsamp.tex| with include files
|cdocsch1.tex|, |cdocsch2.tex|, |cdocspt3.tex|, |cdocspt4.tex|,
|cdocsdrf.tex|, |cdocsfn1.tex|, |cdocsfn2.tex|.
Then copy the file |childdoc.def| to an appropriate directory of your \LaTeX{}
distribution, e.g.\ \textit{texmf-root}|/tex/latex/childdoc|.
\end{itemize}

%%%%%%%%%%%%%%%%%%%%%%%%%%%%%%%%%%%%%%%%%%%%%%%%%%%%%%%%%%%%%%%%%%%%%%%%%%%%%%%%
\subsection{Related CTAN Packages}

There are several other packages which offer a similar functionality:
%
\begin{itemize}
\item
The packages
\href{http://ctan.org/pkg/docmute}{\textsf{docmute}},
\href{http://ctan.org/pkg/includex}{\textsf{includex}} and
\href{http://ctan.org/pkg/standalone}{\textsf{standalone}}
provide commands to include only the document body of
a child file thus allowing both files to be compiled individually.
\item
The packages \href{http://ctan.org/pkg/subdocs}{\textsf{subdocs}}
and \href{http://ctan.org/pkg/subfiles}{\textsf{subfiles}}
provide structures in which the main and child documents can be
encapsulated and allowing them to be compiled individually.
The inclusion mechanism is different from the conventional |\include|.
\item
The package \href{http://ctan.org/pkg/combine}{\textsf{combine}}
is an elaborate solution to combine several documents into one.
\end{itemize}
%
See also the CTAN topic \href{http://ctan.org/topic/subdocs}{\textsf{subdocs}}
for further related packages.
The present package differs from the above solutions in that
a document structure constructed with the conventional |\include| mechanism
just needs two extra commands at the top of every file
such that all constituent files can be compiled individually.

%%%%%%%%%%%%%%%%%%%%%%%%%%%%%%%%%%%%%%%%%%%%%%%%%%%%%%%%%%%%%%%%%%%%%%%%%%%%%%%%
%\subsection{Feature Suggestions}
%
%The following is a list of features which may be useful for future
%versions of this package:
%%
%\begin{itemize}
%\item
%\ldots
%\end{itemize}

%%%%%%%%%%%%%%%%%%%%%%%%%%%%%%%%%%%%%%%%%%%%%%%%%%%%%%%%%%%%%%%%%%%%%%%%%%%%%%%%
\subsection{Revision History}

%%%%%%%%%%%%%%%%%%%%%%%%%%%%%%%%%%%%%%%%
\paragraph{v2.0:} 2018/12/30

\begin{itemize}
\item
immediate forward processing
\item
added |\childdocby| mechanism
\item
manual restructured
\end{itemize}

%%%%%%%%%%%%%%%%%%%%%%%%%%%%%%%%%%%%%%%%
\paragraph{v1.6:} 2018/01/17

\begin{itemize}
\item
application for development of include files
\item
corrections to manual
\end{itemize}

%%%%%%%%%%%%%%%%%%%%%%%%%%%%%%%%%%%%%%%%
\paragraph{v1.5:} 2017/05/21

\begin{itemize}
\item
more complete structuring introduced
\item
|\childdocof| introduced
\item
|\childdoc| renamed to |\childdocmain|
\item
|\childredirect| renamed to |\childdocforward| and |\childdocforwardprefix|
and functionality expanded
\end{itemize}

%%%%%%%%%%%%%%%%%%%%%%%%%%%%%%%%%%%%%%%%
\paragraph{v1.0:} 2017/04/27

\begin{itemize}
\item
manual and install package
\item
first version published on CTAN
\end{itemize}

%%%%%%%%%%%%%%%%%%%%%%%%%%%%%%%%%%%%%%%%
\paragraph{v0.6:} 2017/04/26

\begin{itemize}
\item
redirection mechanism added
\end{itemize}

%%%%%%%%%%%%%%%%%%%%%%%%%%%%%%%%%%%%%%%%
\paragraph{v0.5:} 2017/04/26

\begin{itemize}
\item
functionality in definition file
\end{itemize}


%%%%%%%%%%%%%%%%%%%%%%%%%%%%%%%%%%%%%%%%%%%%%%%%%%%%%%%%%%%%%%%%%%%%%%%%%%%%%%%%
%%%%%%%%%%%%%%%%%%%%%%%%%%%%%%%%%%%%%%%%%%%%%%%%%%%%%%%%%%%%%%%%%%%%%%%%%%%%%%%%
%%%%%%%%%%%%%%%%%%%%%%%%%%%%%%%%%%%%%%%%%%%%%%%%%%%%%%%%%%%%%%%%%%%%%%%%%%%%%%%%
\appendix

\settowidth\MacroIndent{\rmfamily\scriptsize 000\ }

 \DocInput{childdoc.dtx}

\end{document}
%</driver>
% \fi
%
% %%%%%%%%%%%%%%%%%%%%%%%%%%%%%%%%%%%%%%%%%%%%%%%%%%%%%%%%%%%%%%%%%%%%%%%%%%%%%%
% %%%%%%%%%%%%%%%%%%%%%%%%%%%%%%%%%%%%%%%%%%%%%%%%%%%%%%%%%%%%%%%%%%%%%%%%%%%%%%
% \section{Sample}
%\iffalse
%<*samplemain>
%\fi
%
% The following presents a sample document
% with two chapters, two parts, a title page,
% a compile flag as well as three forwarding files to set the flag.
% It consists of eight |.tex| files:
% \begin{center}
% \begin{tabular}{ll}
% |cdocsamp.tex|&main file\\
% |cdocsch1.tex|&include file for chapter 1\\
% |cdocsch2.tex|&include file for chapter 2\\
% |cdocspt3.tex|&include file for part 3\\
% |cdocspt4.tex|&include file for part 4\\
% |cdocsdrf.tex|&forwarding file for main file in draft mode\\
% |cdocsfi1.tex|&forwarding file for final version of chapter 1\\
% |cdocsfi2.tex|&forwarding file for final version of chapter 2\\
% \end{tabular}
% \end{center}
% Each of the eight files can be compiled directly by the \LaTeX{} compiler.
%
% %%%%%%%%%%%%%%%%%%%%%%%%%%%%%%%%%%%%%%
% \paragraph{Main File.}
%
% The main file is called |cdocsamp.tex|.
%
% Load the \textsf{childdoc} definitions and
% declare the filename for the main document:
%    \begin{macrocode}
\input{childdoc.def}
\childdocmain{}
%    \end{macrocode}

% Optional override for |\version| flag:
%    \begin{macrocode}
%%\ifchilddoc\else\providecommand{\version}{draft}\fi
%    \end{macrocode}

% Define the default values for the |\version| flag
% (|final| for the main file and |draft| for childs):
%    \begin{macrocode}
\ifchilddoc
\providecommand{\version}{draft}
\else
\providecommand{\version}{final}
\fi
%    \end{macrocode}

% Load the standard document class:
%    \begin{macrocode}
\documentclass[12pt]{article}
%    \end{macrocode}

% Start the document body:
%    \begin{macrocode}
\begin{document}
%    \end{macrocode}

% Declare a title page.
% Print title, part of document being processed and version flag:
%    \begin{macrocode}
\addtocounter{page}{-1}
\begin{center}
{\LARGE\bfseries{}childdoc example\par}
\vspace{1cm}
\ifchilddoc
\ifchilddocmanual part\else chapter\fi:
`\childdocname' of `\childdocjob'\par
\else
main document: `\childdocjob'\par
\fi
version: \version\par
\end{center}
\newpage
%    \end{macrocode}

% Manually include selected file,
% otherwise process as usual:
%    \begin{macrocode}
\ifchilddocmanual
\section*{part `\childdocname'}
\input{\childdocname}
\else
%    \end{macrocode}

% Include the two chapters:
%    \begin{macrocode}
\include{cdocsch1}
\include{cdocsch2}
%    \end{macrocode}

% Include the two parts unless only chapters should be displayed:
%    \begin{macrocode}
\ifchilddoc\else
\section{part three}
\input{cdocspt3}
\section{part four}
\input{cdocspt4}
\fi
%    \end{macrocode}

% Process as usual until here:
%    \begin{macrocode}
\fi
%    \end{macrocode}

% End of document body:
%    \begin{macrocode}
\end{document}
%    \end{macrocode}
%\iffalse
%</samplemain>
%\fi
%
% %%%%%%%%%%%%%%%%%%%%%%%%%%%%%%%%%%%%%%
% \paragraph{Chapter Include Files.}
%
% The include files are called |cdocsch1.tex| and |cdocsch2.tex|.
%
%\iffalse
%<*samplechap1|samplechap2>
%\fi

% Optional override for |\version| flag:
%    \begin{macrocode}
%%\providecommand{\version}{final}
%    \end{macrocode}

% Include the main document:
%    \begin{macrocode}
\input{childdoc.def}
\childdocof{cdocsamp}
%    \end{macrocode}

%\iffalse
%</samplechap1|samplechap2>
%\fi
%
%\iffalse
%<*samplechap1>
%\fi
% Some text for chapter 1:
%    \begin{macrocode}
\section{one}
some text in chapter one
%    \end{macrocode}

%\iffalse
%</samplechap1>
%\fi
% Some text for chapter 2:
%\iffalse
%<*samplechap2>
%\fi
%    \begin{macrocode}
\section{two}
more text in chapter two
%    \end{macrocode}

%\iffalse
%</samplechap2>
%\fi
%
% %%%%%%%%%%%%%%%%%%%%%%%%%%%%%%%%%%%%%%
% \paragraph{Part Include Files.}
%
% The include files are called |cdocspt3.tex| and |cdocspt4.tex|.
%
%\iffalse
%<*samplepart3|samplepart4>
%\fi

% Optional override for |\version| flag:
%    \begin{macrocode}
%%\providecommand{\version}{final}
%    \end{macrocode}

% Include the main document:
%    \begin{macrocode}
\input{childdoc.def}
\childdocby{cdocsamp}
%    \end{macrocode}

%\iffalse
%</samplepart3|samplepart4>
%\fi
%
%\iffalse
%<*samplepart3>
%\fi
% Some text for part 3:
%    \begin{macrocode}
some text in part three
%    \end{macrocode}

%\iffalse
%</samplepart3>
%\fi
% Some text for part 4:
%\iffalse
%<*samplepart4>
%\fi
%    \begin{macrocode}
more text in part four
%    \end{macrocode}

%\iffalse
%</samplepart4>
%\fi
%
% %%%%%%%%%%%%%%%%%%%%%%%%%%%%%%%%%%%%%%
% \paragraph{Forwarding for a Complete Draft.}
%
% The following forwarding file |cdocsdrf.tex|
% compiles the main document in draft mode:
%\iffalse
%<*sampledraft>
%\fi
%    \begin{macrocode}
\def\version{draft}
\input{childdoc.def}
\childdocforward{cdocsamp}
%    \end{macrocode}

%\iffalse
%</sampledraft>
%\fi
%
% %%%%%%%%%%%%%%%%%%%%%%%%%%%%%%%%%%%%%%
% \paragraph{Forwarding for Final Version of the Chapters.}
%
% The following forwarding files |cdocsfn1.tex| and |cdocsfn2.tex|
% (with identical content)
% compile the final versions of the child documents
% |cdocsch1.tex| and |cdocsch2.tex|, respectively:
%\iffalse
%<*samplefinal>
%\fi
%    \begin{macrocode}
\def\version{final}
\input{childdoc.def}
\childdocforwardprefix[cdocsamp]{cdocsfn}{cdocsch}
%    \end{macrocode}

%\iffalse
%</samplefinal>
%\fi
%
% %%%%%%%%%%%%%%%%%%%%%%%%%%%%%%%%%%%%%%
% \paragraph{Command Line Processing.}
%
% The following three command lines generate the output files
% |cdocscld|, |cdocscl1| and |cdocscl2|
% which should be identical to
% |cdocsdrf|, |cdocsch1| and |cdocsfn2|, respectively:
% \begin{center}
% \begin{tabular}{l}
% |latex -jobname cdocscld \|\\
% |  "\def\version{draft}\input{childdoc.def}\childdocforward{cdocsamp}"|\\
% |latex -jobname cdocscl1 \|\\
% |  "\input{childdoc.def}\childdocforward[cdocsamp]{cdocsch1}"|\\
% |latex -jobname cdocscl2 \|\\
% |  "\def\version{final}\input{childdoc.def}\childdocforward{cdocsch2}"|
% \end{tabular}
% \end{center}
% Note that the trailing backslash on each first line
% merely continues the input to the second line
% (for convenient cut ant paste).
% Furthermore, the command |latex| can be replaced by any
% of its alternative versions such as |pdflatex|.
%
% %%%%%%%%%%%%%%%%%%%%%%%%%%%%%%%%%%%%%%%%%%%%%%%%%%%%%%%%%%%%%%%%%%%%%%%%%%%%%%
% %%%%%%%%%%%%%%%%%%%%%%%%%%%%%%%%%%%%%%%%%%%%%%%%%%%%%%%%%%%%%%%%%%%%%%%%%%%%%%
% \section{Implementation}
%\iffalse
%<*package>
%\fi
%
% This section describes the definitions file |childdoc.def|.

% The definitions cannot be loaded using |\usepackage| or |\RequirePackage|
% which has a mechanism to prevent loading a style file more than once.
% When loading the definitions by means of |\input|
% multiple instances have to be prevented manually:
%\iffalse
%This code needs to be before the `\ProvidesFile' directive
%which is defined at the beginning of this file.
%Therefore it is also placed there and commented out here.
%</package>
%<*discard>
%\fi
%    \begin{macrocode}
\ifdefined\childdocmain\endinput\fi
%    \end{macrocode}
%\iffalse
%</discard>
%<*package>
%\fi
%
% \macro{\ifchilddoc}
% \macro{\ifchilddocmanual}
% The conditional |\ifchilddoc| tells whether a
% child (true) or main (false) document is being compiled.
% The conditional |\ifchilddocmanual| tells whether
% the |\includeonly| mechanism is used (false) or
% the selection of child files must be performed manually (true).
% The definitions initialise to false:
%    \begin{macrocode}
\newif\ifchilddoc
\newif\ifchilddocmanual
%    \end{macrocode}

% \macro{\childdocname}
% \macro{\childdocjob}
% The macro |\childdocname| stores the name of the main document
% to be compiled. The macro |\childdocjob| stores the name of
% the document on which the \LaTeX{} compiler was originally invoked.
% The content of |\jobname| cannot be compared
% to filenames specified in the source due to different catcodes.
% The following code rescans |\jobname|, stores the result
% in |\childdocname| and saves a copy in |\childdocjob|:
%    \begin{macrocode}
\edef\childdocname{\scantokens\expandafter{\jobname\noexpand}}
\let\childdocjob\childdocname
%    \end{macrocode}

% \macro{\childdocdisable}
% The macro |\childdocdisable| prevents the main file
% from being processed more than once.
% At this stage, the main document command |\childdocmain|
% is assumed to be called once again where it should do nothing.
% Any subsequent call to it should prevent
% a secondary processing of the main document
% It overwrites the forwarding commands
% |\childdocof| and |\childdocforward|
% with empty macros to prevent further inclusions of the main document:
%    \begin{macrocode}
\newcommand{\childdocdisable}
{
  \renewcommand{\childdocmain}[1]{\renewcommand{\childdocmain}[1]{\endinput}}
  \renewcommand{\childdocof}[1]{}
  \renewcommand{\childdocby}[2][]{}
  \renewcommand{\childdocforward}[2][]{}
  \renewcommand{\childdocdisable}{}
}
%    \end{macrocode}

% \macro{\childdocmain}
% The macro |\childdocmain| is to be called at the top of the main file
% with nothing or the main filename (without extension) as argument.
% First, it breaks loops.
% If the argument is not empty and does not match |\childdocname|
% (which is set by the first inclusion of |childdoc.def|),
% |\ifchilddoc| is set to true, |\includeonly| is applied to the child file
% and |\jobname| is set to the main file
% (for proper handling of |.aux| files):
%    \begin{macrocode}
\newcommand{\childdocmain}[1]
{
  \childdocdisable\childdocmain{}
  \if?#1?\else
    \begingroup
      \def\childdoctmp{#1}
      \ifx\childdoctmp\childdocname
        \def\childdoctmp{}
      \else
        \def\childdoctmp
        {
          \childdoctrue
          \includeonly{\childdocname}
          \def\childdocjob{#1}
          \def\jobname{#1}
        }
      \fi
      \expandafter
    \endgroup
    \childdoctmp
  \fi
}
%    \end{macrocode}

% \macro{\childdocof}
% The command |\childdocof| redirects
% compilation to the main file |#1|.
%    \begin{macrocode}
\newcommand{\childdocof}[1]
{
  \childdocdisable
  \childdoctrue
  \includeonly{\childdocname}
  \def\jobname{#1}
  \def\childdocjob{#1}
  \input{#1}
}
%    \end{macrocode}

% \macro{\childdocby}
% The command |\childdocby| ....
%    \begin{macrocode}
\newcommand{\childdocby}[2][]
{
  \childdocdisable
  \childdoctrue
  \childdocmanualtrue
  \if?#1?\else
    \def\jobname{#2}
  \fi
  \def\childdocjob{#2}
  \input{#2}
  \endinput
}
%    \end{macrocode}

% \macro{\childdocforward}
% The command |\childdocforward| redirects
% compilation to the main file or
% (if the optional argument is given) a child file.
% Parameters are set as if the main file
% or a child file starting with |\childdocof| was compiled.
% Then compilation is handed over to the main file:
%    \begin{macrocode}
\newcommand{\childdocforward}[2][]
{
  \begingroup
    \if?#1?
      \def\childdoctmp
      {
        \def\childdocname{#2}
        \def\childdocjob{#2}
        \def\jobname{#2}
        \input{#2}
        \endinput
      }
    \else
      \def\childdoctmp
      {
        \childdocdisable
        \def\childdocname{#2}
        \childdoctrue
        \includeonly{#2}
        \def\childdocjob{#1}
        \def\jobname{#1}
        \input{#1}
        \endinput
      }
    \fi
    \expandafter
  \endgroup
  \childdoctmp
}
%    \end{macrocode}

% \macro{\childdocforwardprefix}
% The command |\childdocforwardprefix| redirects
% compilation to the main or a child file by means of a pattern.
% The prefix |#1| in the current filename is replaced by |#2|
% and the suffix of the current filename is kept
% (it is assumed that the filename does not contain the substring `|~~~|'
% which is used as a delimiter).
% Compilation is handed over to the new file by |\childdocforward|:
%    \begin{macrocode}
\newcommand{\childdocforwardprefix}[3][]
{
  \begingroup
    \def\childdocextract #2##1~~~{\def\childdoctmp{\childdocforward[#1]{#3##1}}}
    \expandafter\childdocextract\childdocname~~~
    \expandafter
  \endgroup
  \childdoctmp
}
%    \end{macrocode}

% \macro{\childdoc}
% The deprecated macro |\childdoc| is a legacy version of |\childdocmain|:
%    \begin{macrocode}
\newcommand{\childdoc}{\childdocmain}
%    \end{macrocode}

% \macro{\childdocredirect}
% The deprecated macro |\childdocredirect| is a legacy version
% of |\childdocforward| and |\childdocforwardprefix|:
%    \begin{macrocode}
\newcommand{\childdocredirect}[2][]
{
  \begingroup
    \if?#1?
      \def\childdoctmp{\childdocforward{#2}}
    \else
      \def\childdoctmp{\childdocforwardprefix{#1}{#2}}
    \fi
    \expandafter
  \endgroup
  \childdoctmp
}
%    \end{macrocode}

%\iffalse
%</package>
%\fi
%
\endinput
|\\
|\childdocforward[|\textit{main}|]{|\textit{dest}|}|\\
\end{tabular}
\end{center}
%
The argument \textit{dest} is the destination file
(without extension).
It should be the main file or one of the child files.
Note that further \textsf{childdoc} directives
such as |\childdocof| and |\childdocforward|
in the indicated file will be processed in this form.
The optional argument \textit{main}
passes on directly to the main file \textit{main}
while pretending to compile the child \textit{dest}.
This form behaves as if \textit{dest}
issues |\childdocof{|\textit{main}|}| right away,
and no further \textsf{childdoc} directives will be processed.

%%%%%%%%%%%%%%%%%%%%%%%%%%%%%%%%%%%%%%%%
\DescribeMacro{\...prefix}
In the alternative form |\childdocforwardprefix|,
%
\begin{center}
\begin{tabular}{l}
|% \iffalse
%
% childdoc.dtx Copyright (C) 2017-2018 Niklas Beisert
%
% This work may be distributed and/or modified under the
% conditions of the LaTeX Project Public License, either version 1.3
% of this license or (at your option) any later version.
% The latest version of this license is in
%   http://www.latex-project.org/lppl.txt
% and version 1.3 or later is part of all distributions of LaTeX
% version 2005/12/01 or later.
%
% This work has the LPPL maintenance status `maintained'.
%
% The Current Maintainer of this work is Niklas Beisert.
%
% This work consists of the files childdoc.dtx and childdoc.ins
% and the derived files childdoc.def and cdocsamp.tex with
% cdocsch1.tex, cdocsch2.tex, cdocsdrf.tex, cdocsfn1.tex, cdocsfn2.tex.
%
%<package>\ifdefined\childdocmain\endinput\fi
%<package>\ProvidesFile{childdoc.def}[2018/12/30 v2.0 child document driver]
%<samplemain>\ProvidesFile{cdocsamp.tex}[2018/12/30 v2.0 sample for childdoc]
%<*driver>
%\ProvidesFile{childdoc.drv}[2018/12/30 v2.0 childdoc reference manual file]
\PassOptionsToClass{10pt,a4paper}{article}
\documentclass{ltxdoc}

\usepackage[margin=35mm]{geometry}
\usepackage{hyperref}
\usepackage{hyperxmp}
\usepackage[usenames]{color}

\hypersetup{colorlinks=true}
\hypersetup{pdfstartview=FitH}
\hypersetup{pdfpagemode=UseNone}
\hypersetup{pdfsource={}}
\hypersetup{pdflang={en-UK}}
\hypersetup{pdfcopyright={Copyright 2017-2018 Niklas Beisert.
  This work may be distributed and/or modified under the
  conditions of the LaTeX Project Public License, either version 1.3
  of this license or (at your option) any later version.}}
\hypersetup{pdflicenseurl={http://www.latex-project.org/lppl.txt}}
\hypersetup{pdfcontactaddress={ETH Zurich, ITP, HIT K,
  Wolfgang-Pauli-Strasse 27}}
\hypersetup{pdfcontactpostcode={8093}}
\hypersetup{pdfcontactcity={Zurich}}
\hypersetup{pdfcontactcountry={Switzerland}}
\hypersetup{pdfcontactemail={nbeisert@itp.phys.ethz.ch}}
\hypersetup{pdfcontacturl={http://people.phys.ethz.ch/\xmptilde nbeisert/}}

\newcommand{\secref}[1]{\hyperref[#1]{section \ref*{#1}}}

\parskip1ex
\parindent0pt
\let\olditemize\itemize
\def\itemize{\olditemize\parskip0pt}

\begin{document}

\title{The \textsf{childdoc} Package}
\hypersetup{pdftitle={The childdoc Package}}
\author{Niklas Beisert\\[2ex]
  Institut f\"ur Theoretische Physik\\
  Eidgen\"ossische Technische Hochschule Z\"urich\\
  Wolfgang-Pauli-Strasse 27, 8093 Z\"urich, Switzerland\\[1ex]
  \href{mailto:nbeisert@itp.phys.ethz.ch}
  {\texttt{nbeisert@itp.phys.ethz.ch}}}
\hypersetup{pdfauthor={Niklas Beisert}}
\hypersetup{pdfsubject={Manual for the LaTeX2e Package childdoc}}
\date{30 December 2018, \textsf{v2.0}}
\maketitle

\begin{abstract}\noindent
\textsf{childdoc} is a \LaTeXe{} package
that enables the direct compilation
of document sections included by |\include|
to individual files.
\end{abstract}

\begingroup
\parskip0ex
\tableofcontents
\endgroup

%%%%%%%%%%%%%%%%%%%%%%%%%%%%%%%%%%%%%%%%%%%%%%%%%%%%%%%%%%%%%%%%%%%%%%%%%%%%%%%%
%%%%%%%%%%%%%%%%%%%%%%%%%%%%%%%%%%%%%%%%%%%%%%%%%%%%%%%%%%%%%%%%%%%%%%%%%%%%%%%%
\section{Introduction}

\LaTeX{} provides a mechanism to structure a large document (such as a book)
into a main file and several child files (containing the chapters)
using the |\include| command.
This mechanism is beneficial for documents
which span hundreds of pages in order to
make the source file(s) more manageable.
Moreover, compilation can be restricted to
selected child files by means of the |\includeonly| command.
The latter feature can be used to reduce the compilation time while editing
(this was significantly more useful in the earlier days of \LaTeX{})
or to generate a smaller document which is easier to navigate.
Another application of |\includeonly| is to generate
documents consisting of selected parts of the complete document.

However, there are a few drawbacks of the plain |\include| mechanism:
\begin{itemize}
\item
The child files cannot be compiled on their own,
they can only be compiled via the main file.
A naive editing environment
(such as a text editor with an option
to have the current file processed by \LaTeX)
may require one to switch to the main file before compiling;
attempting to compile the child file produces errors.
\item
The main file must be modified (each time)
to adjust the |\includeonly| command
to the present needs. This easily leaves the main file in a messy state.
\item
The generated document will always carry the filename
of the main document. This is inconvenient if
several child files are to be compiled and
to be kept for distribution.
\end{itemize}

The present package provides a simple interface
to make child files individually compilable by \LaTeX{}.
Compiling a child file then has the same effect as compiling
the main file with an |\includeonly| command
to select the appropriate child.
Moreover the generated document will carry the name of the child
rather than the main file.
This resolves all three above issues.

This feature is meant to make the editing of books,
thesis documents and lecture notes somewhat more convenient.
However, the package can also be used efficiently for
composing a series of documents (such as exercise sheets)
which are typically distributed individually.
It then assists the author in generating the individual documents
(potentially in different versions)
as well as a document containing the collected series.
Another application is in developing style files
or other kinds of included material
where compilation of the style file could redirect
to a sample or test file.

%%%%%%%%%%%%%%%%%%%%%%%%%%%%%%%%%%%%%%%%%%%%%%%%%%%%%%%%%%%%%%%%%%%%%%%%%%%%%%%%
%%%%%%%%%%%%%%%%%%%%%%%%%%%%%%%%%%%%%%%%%%%%%%%%%%%%%%%%%%%%%%%%%%%%%%%%%%%%%%%%
\section{Usage}

First of all, the package \textsf{childdoc} is \emph{not} a standard
\LaTeXe{} |.sty| style file! Therefore it needs to be invoked in
a non-standard way.

%%%%%%%%%%%%%%%%%%%%%%%%%%%%%%%%%%%%%%%%%%%%%%%%%%%%%%%%%%%%%%%%%%%%%%%%%%%%%%%%
\subsection{Included Files}
\label{sec:include}

%%%%%%%%%%%%%%%%%%%%%%%%%%%%%%%%%%%%%%%%
\DescribeMacro{\childdocmain}
To use the package, add the commands
\begin{center}
\begin{tabular}{l}
|\input{childdoc.def}|\\
|\childdocmain{}|\\
\end{tabular}
\end{center}
at the very top of the main \LaTeX{} file,
in particular \emph{before} the |\documentclass| statement!
The argument of |\childdocmain| should be left empty
(but it must be present).

%%%%%%%%%%%%%%%%%%%%%%%%%%%%%%%%%%%%%%%%
\DescribeMacro{\childdocof}
Furthermore, add the commands
\begin{center}
\begin{tabular}{l}
|\input{childdoc.def}|\\
|\childdocof{|\textit{main}|}|\\
\end{tabular}
\end{center}
at the top of every child file \textit{child}
which is included by |\include{|\textit{child}|}|
from within the main file
(or at least for those files to be compiled individually).
The argument \textit{main} must be the filename of the main file.

There are a couple of
considerations in setting up the main and child documents:

%%%%%%%%%%%%%%%%%%%%%%%%%%%%%%%%%%%%%%%%
\paragraph{Restrictions.}

Please note the following restrictions:
\begin{itemize}
\item
|\childdocmain| must be called with one argument \textit{main}
to ensure compatibility with earlier version of the package.
It must either be empty (|\childdocmain{}|)
or precisely match the filename of the main file in which it is specified.
See \secref{sec:detection} for further information.
\item
The filename \textit{main} must be specified without the |.tex| extension.
\item
The filename \textit{main} is case sensitive
(even in case-insensitive file systems)
due to internal string comparison.
\item
The argument \textit{main} should be fully expanded, it cannot be a macro.
\item
Subdirectories and special characters should be avoided in filenames.
\item
The command |\childdocmain{|\textit{main}|}| must be followed by a whitespace.
It should not be followed immediately by another command
or by a comment mark `|%|'.
This is because the \TeX{} parser reads the token immediately following
the argument of |\childdocmain| and puts it
at the beginning of every child section;
however, a white\-space is ignored.
\end{itemize}

%%%%%%%%%%%%%%%%%%%%%%%%%%%%%%%%%%%%%%%%
\paragraph{Content of Main File.}

It is advisable to place all content in the child files included by |\include|.
Any output contained in the main file will appear in all child documents
unless suppressed manually;
it cannot be suppressed automatically by the |\includeonly| directive
and thus should normally be avoided.
A method to include some content in the main file
by means of conditional processing is described in \secref{sec:conditional}.

%%%%%%%%%%%%%%%%%%%%%%%%%%%%%%%%%%%%%%%%
\paragraph{Page Numbering.}

When only a part of the document is compiled,
the appropriate numbering of pages
(as well as other status parameters)
is determined from the |.aux| files.
The latter contain information from previous passes.
However this information needs to propagate through
all intermediate child documents.
Therefore the page numbering in child documents may well
be inconsistent until the complete document is compiled at least once.

A useful (if unconventional) way to always ensure a consistent
page numbering is to restart the numbering in each child document
and denote the pages by `\textit{child}|.|\textit{page}'
where \textit{child} represents the chapter/section number of the child file.
This can be achieved by the command
|\numberwithin{page}{|\textit{child}|}|
of the \textsf{amsmath} package
where \textit{child} can be |chapter| or |section|
depending on the chosen structuring.
Alternatively, one can modify the macro |\thepage| appropriately
and reset the counter |page| at the start of each child file.

%%%%%%%%%%%%%%%%%%%%%%%%%%%%%%%%%%%%%%%%%%%%%%%%%%%%%%%%%%%%%%%%%%%%%%%%%%%%%%%%
\subsection{Conditional Processing}
\label{sec:conditional}

The package provides a mechanism to compile different versions
of a document. To customise the versions further some conditional processing
can come in handy to distinguish which version is being compiled.
The package provides two macros to describe the compilation context:

%%%%%%%%%%%%%%%%%%%%%%%%%%%%%%%%%%%%%%%%
\DescribeMacro{\ifchilddoc}
The conditional |\ifchilddoc| distinguishes between the compilation of
child documents and the main document:
%
\begin{center}
|\ifchilddoc |\textit{child-code}| |[|\||else |\textit{main-code}]| \||fi|
\end{center}

%%%%%%%%%%%%%%%%%%%%%%%%%%%%%%%%%%%%%%%%
\DescribeMacro{\childdocname}
\DescribeMacro{\childdocjob}
The macro |\childdocname| contains the filename (without extension)
of the main or child file being processed.
Note that |\childdocjob| will always contain the name of the main file.

%%%%%%%%%%%%%%%%%%%%%%%%%%%%%%%%%%%%%%%%
\paragraph{Title Page.}

Conditional processing can be used to include a title or banner page
in the main document when proper precautions are taken.
Importantly, the code in the main file should ensure that the page counter
(as well as other status parameters which are stored in the |.aux| files)
takes the same value after the conditional processing.
Otherwise the page numbers may take divergent values
depending on which part is compiled.

For example, a title page could be declared by:
%
\begin{center}
\begin{tabular}{l}
|\ifchilddoc\||else|\\
|\addtocounter{page}{-1}|\\
\textit{code for title page}\\
|\newpage|\\
|\||fi|
\end{tabular}
\end{center}
%
A banner page for the child documents can be generated by:
%
\begin{center}
\begin{tabular}{l}
|\ifchilddoc|\\
|\addtocounter{page}{-1}|\\
\textit{code for banner page}\\
|\newpage|\\
|\||fi|
\end{tabular}
\end{center}
%
Here one could write a message such as:
\begin{center}
|This is the part \childdocname{} of \childdocjob{}.|
\end{center}

%%%%%%%%%%%%%%%%%%%%%%%%%%%%%%%%%%%%%%%%%%%%%%%%%%%%%%%%%%%%%%%%%%%%%%%%%%%%%%%%
\subsection{Flags}
\label{sec:flags}

The package makes it easy to generate different versions
of the main or child documents.
To this end compilation flags can be defined
and assigned different default values.
They will be particularly useful in conjunction
with the forwarding mechanism described in \secref{sec:forward}.

For example, it may be useful to have a flag |\version|
which can be set to |draft| or |final|.
The document source will contain some conditional code
depending on the value of |\version|.
Suppose further, the flag should default to |final| for the main file
and to |draft| for child files
which is a natural assignment for editing the document.
This is achieved by placing the following code
in the preamble of the main document
(below the |\childdocmain| directive):
%
\begin{center}
\begin{tabular}{l}
|\ifchilddoc|\\
|\providecommand{\version}{draft}|\\
|\||else|\\
|\providecommand{\version}{final}|\\
|\||fi|
\end{tabular}
\end{center}
%
The definition by |\providecommand| makes sure
that previous definitions are not overwritten.
Further statements |\providecommand{\version}{...}|
can thus be added before the above code to override it.

For the main file, one might add a line
(between |\childdocmain| and the above block)
%
\begin{center}
|%\ifchilddoc\||else\providecommand{\version}{draft}\||fi|
\end{center}
%
which can be uncommented to produce a draft version.
Likewise one can add a line to the very top of a child file
(above the |\childdocof{|\textit{main}|}| directive)
%
\begin{center}
|%\providecommand{\version}{final}|
\end{center}
%
which can be uncommented to produce the final version of this child document.

%%%%%%%%%%%%%%%%%%%%%%%%%%%%%%%%%%%%%%%%%%%%%%%%%%%%%%%%%%%%%%%%%%%%%%%%%%%%%%%%
\subsection{Forwarding}
\label{sec:forward}

Different versions of the main or child documents
using compilation flags as described in \secref{sec:flags}
can be (permanently) stored in different files
for convenient compilation, viewing and distribution.
To this end, the package defines a command
to pass on compilation to a different file:

%%%%%%%%%%%%%%%%%%%%%%%%%%%%%%%%%%%%%%%%
\DescribeMacro{\childdocforward}
The command |\childdocforward| redirects processing to
another source file:
%
\begin{center}
\begin{tabular}{l}
|\input{childdoc.def}|\\
|\childdocforward[|\textit{main}|]{|\textit{dest}|}|\\
\end{tabular}
\end{center}
%
The argument \textit{dest} is the destination file
(without extension).
It should be the main file or one of the child files.
Note that further \textsf{childdoc} directives
such as |\childdocof| and |\childdocforward|
in the indicated file will be processed in this form.
The optional argument \textit{main}
passes on directly to the main file \textit{main}
while pretending to compile the child \textit{dest}.
This form behaves as if \textit{dest}
issues |\childdocof{|\textit{main}|}| right away,
and no further \textsf{childdoc} directives will be processed.

%%%%%%%%%%%%%%%%%%%%%%%%%%%%%%%%%%%%%%%%
\DescribeMacro{\...prefix}
In the alternative form |\childdocforwardprefix|,
%
\begin{center}
\begin{tabular}{l}
|\input{childdoc.def}|\\
|\childdocforwardprefix[|\textit{main}|]{|\textit{prefix}|}{|\textit{dest}|}|
\end{tabular}
\end{center}
%
the destination file is determined by a pattern
depending on the current file:
To make this work, the current file must be called
`{\textit{prefix}\hspace{0.2em}\textit{suffix}}'
with \textit{prefix} matching precisely the argument.
Processing is then passed on to the file
`{\textit{dest}\hspace{0.2em}\textit{suffix}}'.
Surely, the same effect is achieved by
directly specifying the
argument `{\textit{dest}\hspace{0.2em}\textit{suffix}}'
in the first form.
However, that requires to set up a different file
for each child. With the alternative form of the command
all these files can have exactly the same content
which simplifies setting them up and maintaining them.

For example, the following file |draft.tex|
with a compilation flag |\version| as described in \secref{sec:flags}
compiles the main document as a draft:
%
\begin{center}
\begin{tabular}{l}
|\def\version{draft}|\\
|\input{childdoc.def}|\\
|\childdocforward{|\textit{main}|}|
\end{tabular}
\end{center}
%
Likewise, the following files |final|\textit{nn}|.tex|
compile the final version of the child document
|child|\textit{nn}|.tex|:
%
\begin{center}
\begin{tabular}{l}
|\def\version{final}|\\
|\input{childdoc.def}|\\
|\childdocforwardprefix{final}{child}|
\end{tabular}
\end{center}
%

Note that when several versions of a main file and/or of each child file
are to be generated, it may be convenient to set up a |Makefile| or
shell script to automatise the process.

%%%%%%%%%%%%%%%%%%%%%%%%%%%%%%%%%%%%%%%%%%%%%%%%%%%%%%%%%%%%%%%%%%%%%%%%%%%%%%%%
\subsection{Command Line Processing}
\label{sec:commandline}

The effect of redirection files can also be achieved by invoking
the \LaTeX{} compiler with a more elaborate command line.
Most conveniently this should be done as part
of a shell script or a |Makefile|.

When using \textsf{childdoc} in the main file, the following
command lines effectively perform a redirection
(note that depending on the shell being used,
backslashes may have to be doubled: `|\|' $\to$ `|\\|'):
%
\begin{center}
|... -jobname "|\textit{target}|" |\\|"|[\textit{flags}]%
|\input{childdoc.def}\childdocforward[|\textit{main}|]{|\textit{dest}|}"|
\end{center}
%
Here \textit{target} is the name of the output file,
\textit{main} is the name of the main file
and \textit{dest} is the name of the main or child file to be processed
(all filenames without extensions).
The optional argument \textit{main} can be omitted
if \textit{main} matches \textit{dest}.
Optionally, compilation \textit{flags} can be defined via |\def| commands.
This command line makes the \TeX{} engine believe
it is compiling the file \textit{target}
whose content is specified as the latter parameter.
The provided code then forwards the processing to
\textit{main} or \textit{dest} as described in \secref{sec:forward}.

%%%%%%%%%%%%%%%%%%%%%%%%%%%%%%%%%%%%%%%%%%%%%%%%%%%%%%%%%%%%%%%%%%%%%%%%%%%%%%%%
\subsection{Include by Input}
\label{sec:input}

Including child documents by |\include| has some restrictions by design.
Most notably, the content of a child document always occupies
its own set of pages; pages cannot be shared between child documents.
Usually, this behaviour makes perfect sense
because each child document contain an essential part of the document.
However, in some situations it may be desirable to compose
a document from a collection of parts
without having mandatory page breaks between then.
For this case, the package
provides a mechanism to include parts
by |\input| which can also be processed individually.
However, by construction this mechanism
requires manual handling of the content to be output.

%%%%%%%%%%%%%%%%%%%%%%%%%%%%%%%%%%%%%%%%
\DescribeMacro{\ifchilddocmanual}
The main file should be prepared as usual, see \secref{sec:include}.
However, the document body must make a distinction
between processing of an individual part and of the main document, e.g.:
%
\begin{center}
\begin{tabular}{l}
|\ifchilddocmanual|\\
|\input{\childdocname}|\\
|\||else|\\
\textit{document body with }|\input{|\textit{part}|}|\\
|\||fi|
\end{tabular}
\end{center}
%
The conditional |\ifchilddocmanual| is true whenever
a part to be included by |\input| is being compiled,
and the name of the part is stored in |\childdocname|.

%%%%%%%%%%%%%%%%%%%%%%%%%%%%%%%%%%%%%%%%
\DescribeMacro{\childdocby}
Each part to be included by |\input| should start with:
%
\begin{center}
\begin{tabular}{l}
|\input{childdoc.def}|\\
|\childdocby{|\textit{main}|}|\\
\end{tabular}
\end{center}
%
The directive |\childdocby| is similar to |\childdocof|
described in \secref{sec:include},
but the subsequent selection of content must be done manually.
To that end, both |\ifchilddoc| and |\ifchilddocmanual|
will be true upon processing of a part,
and the name of the part is stored in |\childdocname|.
Note that |\jobname| will be set to the filename of the current part
so that each part receives an individual |.aux| file
that does not interfere with the |.aux| file(s) of the main document.
This behaviour can be altered by the alternative form
|\childdocby[*]{|\textit{main}|}| (with a non-empty optional argument)
which uses the |.aux| file of the main document
by setting |\jobname| to \textit{main}.

%%%%%%%%%%%%%%%%%%%%%%%%%%%%%%%%%%%%%%%%%%%%%%%%%%%%%%%%%%%%%%%%%%%%%%%%%%%%%%%%
\subsection{Driver Development}
\label{sec:driver}

The \textsf{childdoc} mechanism can also be use for the development
of definition files such as \LaTeX{} styles or classes.
This case differs from the above setup with multiple parts
included by |\include| in that no |\includeonly| should be invoked.
This can be achieved by starting the include file
(before |\ProvidesPackage|) with:
%
\begin{center}
\begin{tabular}{l}
|\input{childdoc.def}|\\
|\childdocforward{|\textit{main}|}|\\
\end{tabular}
\end{center}
%
or alternatively with:
%
\begin{center}
\begin{tabular}{l}
|\input{childdoc.def}|\\
|\childdocby{|\textit{main}|}|\\
\end{tabular}
\end{center}
%
Both forms have slightly different effects as described above.
The main file is prepared as usual, see \secref{sec:include}.

%%%%%%%%%%%%%%%%%%%%%%%%%%%%%%%%%%%%%%%%%%%%%%%%%%%%%%%%%%%%%%%%%%%%%%%%%%%%%%%%
\subsection{Legacy Detection}
\label{sec:detection}

The directive |\childdocmain| in the main file can detect
whether the complete document or merely a child is to be compiled
even without using the directive |\childdocof|.
This method is deprecated because it is less robust
and there is no compelling reason to use it;
it is merely provided for backward compatibility
and it may be removed in future versions.

If the detection mechanism is to be used,
it is mandatory to correctly specify
the filename of the main file as the argument of |\childdocmain|:
%
\begin{center}
\begin{tabular}{l}
|\input{childdoc.def}|\\
|\childdocmain{|\textit{main}|}|\\
\end{tabular}
\end{center}
%
If |\jobname| does not match the argument \textit{main} of |\childdocmain|,
it is assumed that |\jobname| points to the child file to be compiled.
When using |\childdocmain| with the main file specified as argument,
it suffices to start a child file
with just |\input{|\textit{main}|}|
without loading of the package and using |\childdocof|.
If instead all processing is done
with the appropriate \textsf{childdoc} directives,
the argument of \textit{main} of |\childdocmain| can be empty.

An alternative version of the command line processing described
in \secref{sec:commandline} using the detection mechanism reads:
%
\begin{center}
|... -jobname "|\textit{target}|" "|[\textit{flags}]%
[|\def\jobname{|\textit{dest}|}|]|\input{|\textit{main}|}"|
\end{center}

%%%%%%%%%%%%%%%%%%%%%%%%%%%%%%%%%%%%%%%%%%%%%%%%%%%%%%%%%%%%%%%%%%%%%%%%%%%%%%%%
\subsection{Manual Code}
\label{sec:manual}

In case one cannot be certain whether the definitions file |childdoc.def|
is installed on the target \TeX{} distribution
and one prefers not to ship it,
it is conceivable to paste a few relevant commands into the sources.

To that end, drop all statements |\input{childdoc.def}|
and perform the replacements as outlined below.
Instead of |\childdocmain{|\textit{main}|}| add the following code
to the top of the main file:
%
\begin{center}
\begin{tabular}{l}
|\||ifdefined\childdocname\endinput\||fi\newif\ifchilddoc|\\
|\edef\childdocname{\scantokens\expandafter{\jobname\noexpand}}|\\
|\def\childdocmain{|\textit{main}|}\||ifx\childdocmain\childdocname\||else|\\
|\childdoctrue\includeonly{\childdocname}\let\jobname\childdocmain\||fi|\\
\end{tabular}
\end{center}
%
Instead of |\childdocof{|\textit{main}|}| just include the main file
at the top of each child file:
%
\begin{center}
|\input{|\textit{main}|}|
\end{center}
%
A simple redirection |\childdocforward{|\textit{dest}|}| is achieved by:
%
\begin{center}
|\def\jobname{|\textit{dest}|}\input{\jobname}|
\end{center}
%
The redirection with prefix
|\childdocforwardprefix[|\textit{prefix}|]{|\textit{dest}|}|
is accomplished by:
%
\begin{center}
\begin{tabular}{l}
|{\edef\jobname{\scantokens\expandafter{\jobname\noexpand}}|\\
|\def\redirectjob |\textit{prefix}|#1~~~{\gdef\jobname{|\textit{dest}|#1}}|\\
|\expandafter\redirectjob\jobname~~~}\input{\jobname}|
\end{tabular}
\end{center}

In an alternative approach,
child documents can be compiled by a specific command line
without additional code or specific definitions:
%
\begin{center}
|... -jobname "|\textit{target}|" "|[\textit{flags}]%
|\includeonly{|\textit{dest}|}\input{|\textit{main}|}"|
\end{center}
%

%%%%%%%%%%%%%%%%%%%%%%%%%%%%%%%%%%%%%%%%%%%%%%%%%%%%%%%%%%%%%%%%%%%%%%%%%%%%%%%%
%%%%%%%%%%%%%%%%%%%%%%%%%%%%%%%%%%%%%%%%%%%%%%%%%%%%%%%%%%%%%%%%%%%%%%%%%%%%%%%%
\section{Information}

%%%%%%%%%%%%%%%%%%%%%%%%%%%%%%%%%%%%%%%%%%%%%%%%%%%%%%%%%%%%%%%%%%%%%%%%%%%%%%%%
\subsection{Copyright}

Copyright \copyright{} 2017--2018 Niklas Beisert

This work may be distributed and/or modified under the
conditions of the \LaTeX{} Project Public License, either version 1.3
of this license or (at your option) any later version.
The latest version of this license is in
  \url{http://www.latex-project.org/lppl.txt}
and version 1.3 or later is part of all distributions of \LaTeX{}
version 2005/12/01 or later.

This work has the LPPL maintenance status `maintained'.

The Current Maintainer of this work is Niklas Beisert.

This work consists of the files |README.txt|, |childdoc.ins| and |childdoc.dtx|
as well as the derived files |childdoc.def|, |cdocsamp.tex|
with |cdocsch1.tex|, |cdocsch2.tex|, |cdocspt3.tex|, |cdocspt4.tex|,
|cdocsdrf.tex|, |cdocsfn1.tex|, |cdocsfn2.tex|
as well as |childdoc.pdf|.

%%%%%%%%%%%%%%%%%%%%%%%%%%%%%%%%%%%%%%%%%%%%%%%%%%%%%%%%%%%%%%%%%%%%%%%%%%%%%%%%
\subsection{Files and Installation}

The package consists of the files:
%
\begin{center}
\begin{tabular}{ll}
    |README.txt|   & readme file \\
    |childdoc.ins| & installation file \\
    |childdoc.dtx| & source file \\
    |childdoc.def| & definition file \\
    |cdocsamp.tex| & sample main file \\
    |cdocsch1.tex| & sample include file \\
    |cdocsch2.tex| & sample include file \\
    |cdocspt3.tex| & sample part file \\
    |cdocspt4.tex| & sample part file \\
    |cdocsdrf.tex| & sample redirection file \\
    |cdocsfn1.tex| & sample redirection file \\
    |cdocsfn2.tex| & sample redirection file \\
    |childdoc.pdf| & manual
\end{tabular}
\end{center}
%
The distribution consists of the files
|README.txt|, |childdoc.ins| and |childdoc.dtx|.
%
\begin{itemize}
\item
Run (pdf)\LaTeX{} on |childdoc.dtx|
to compile the manual |childdoc.pdf| (this file).
\item
Run \LaTeX{} on |childdoc.ins| to create the definitions file |childdoc.def|
and the sample |cdocsamp.tex| with include files
|cdocsch1.tex|, |cdocsch2.tex|, |cdocspt3.tex|, |cdocspt4.tex|,
|cdocsdrf.tex|, |cdocsfn1.tex|, |cdocsfn2.tex|.
Then copy the file |childdoc.def| to an appropriate directory of your \LaTeX{}
distribution, e.g.\ \textit{texmf-root}|/tex/latex/childdoc|.
\end{itemize}

%%%%%%%%%%%%%%%%%%%%%%%%%%%%%%%%%%%%%%%%%%%%%%%%%%%%%%%%%%%%%%%%%%%%%%%%%%%%%%%%
\subsection{Related CTAN Packages}

There are several other packages which offer a similar functionality:
%
\begin{itemize}
\item
The packages
\href{http://ctan.org/pkg/docmute}{\textsf{docmute}},
\href{http://ctan.org/pkg/includex}{\textsf{includex}} and
\href{http://ctan.org/pkg/standalone}{\textsf{standalone}}
provide commands to include only the document body of
a child file thus allowing both files to be compiled individually.
\item
The packages \href{http://ctan.org/pkg/subdocs}{\textsf{subdocs}}
and \href{http://ctan.org/pkg/subfiles}{\textsf{subfiles}}
provide structures in which the main and child documents can be
encapsulated and allowing them to be compiled individually.
The inclusion mechanism is different from the conventional |\include|.
\item
The package \href{http://ctan.org/pkg/combine}{\textsf{combine}}
is an elaborate solution to combine several documents into one.
\end{itemize}
%
See also the CTAN topic \href{http://ctan.org/topic/subdocs}{\textsf{subdocs}}
for further related packages.
The present package differs from the above solutions in that
a document structure constructed with the conventional |\include| mechanism
just needs two extra commands at the top of every file
such that all constituent files can be compiled individually.

%%%%%%%%%%%%%%%%%%%%%%%%%%%%%%%%%%%%%%%%%%%%%%%%%%%%%%%%%%%%%%%%%%%%%%%%%%%%%%%%
%\subsection{Feature Suggestions}
%
%The following is a list of features which may be useful for future
%versions of this package:
%%
%\begin{itemize}
%\item
%\ldots
%\end{itemize}

%%%%%%%%%%%%%%%%%%%%%%%%%%%%%%%%%%%%%%%%%%%%%%%%%%%%%%%%%%%%%%%%%%%%%%%%%%%%%%%%
\subsection{Revision History}

%%%%%%%%%%%%%%%%%%%%%%%%%%%%%%%%%%%%%%%%
\paragraph{v2.0:} 2018/12/30

\begin{itemize}
\item
immediate forward processing
\item
added |\childdocby| mechanism
\item
manual restructured
\end{itemize}

%%%%%%%%%%%%%%%%%%%%%%%%%%%%%%%%%%%%%%%%
\paragraph{v1.6:} 2018/01/17

\begin{itemize}
\item
application for development of include files
\item
corrections to manual
\end{itemize}

%%%%%%%%%%%%%%%%%%%%%%%%%%%%%%%%%%%%%%%%
\paragraph{v1.5:} 2017/05/21

\begin{itemize}
\item
more complete structuring introduced
\item
|\childdocof| introduced
\item
|\childdoc| renamed to |\childdocmain|
\item
|\childredirect| renamed to |\childdocforward| and |\childdocforwardprefix|
and functionality expanded
\end{itemize}

%%%%%%%%%%%%%%%%%%%%%%%%%%%%%%%%%%%%%%%%
\paragraph{v1.0:} 2017/04/27

\begin{itemize}
\item
manual and install package
\item
first version published on CTAN
\end{itemize}

%%%%%%%%%%%%%%%%%%%%%%%%%%%%%%%%%%%%%%%%
\paragraph{v0.6:} 2017/04/26

\begin{itemize}
\item
redirection mechanism added
\end{itemize}

%%%%%%%%%%%%%%%%%%%%%%%%%%%%%%%%%%%%%%%%
\paragraph{v0.5:} 2017/04/26

\begin{itemize}
\item
functionality in definition file
\end{itemize}


%%%%%%%%%%%%%%%%%%%%%%%%%%%%%%%%%%%%%%%%%%%%%%%%%%%%%%%%%%%%%%%%%%%%%%%%%%%%%%%%
%%%%%%%%%%%%%%%%%%%%%%%%%%%%%%%%%%%%%%%%%%%%%%%%%%%%%%%%%%%%%%%%%%%%%%%%%%%%%%%%
%%%%%%%%%%%%%%%%%%%%%%%%%%%%%%%%%%%%%%%%%%%%%%%%%%%%%%%%%%%%%%%%%%%%%%%%%%%%%%%%
\appendix

\settowidth\MacroIndent{\rmfamily\scriptsize 000\ }

 \DocInput{childdoc.dtx}

\end{document}
%</driver>
% \fi
%
% %%%%%%%%%%%%%%%%%%%%%%%%%%%%%%%%%%%%%%%%%%%%%%%%%%%%%%%%%%%%%%%%%%%%%%%%%%%%%%
% %%%%%%%%%%%%%%%%%%%%%%%%%%%%%%%%%%%%%%%%%%%%%%%%%%%%%%%%%%%%%%%%%%%%%%%%%%%%%%
% \section{Sample}
%\iffalse
%<*samplemain>
%\fi
%
% The following presents a sample document
% with two chapters, two parts, a title page,
% a compile flag as well as three forwarding files to set the flag.
% It consists of eight |.tex| files:
% \begin{center}
% \begin{tabular}{ll}
% |cdocsamp.tex|&main file\\
% |cdocsch1.tex|&include file for chapter 1\\
% |cdocsch2.tex|&include file for chapter 2\\
% |cdocspt3.tex|&include file for part 3\\
% |cdocspt4.tex|&include file for part 4\\
% |cdocsdrf.tex|&forwarding file for main file in draft mode\\
% |cdocsfi1.tex|&forwarding file for final version of chapter 1\\
% |cdocsfi2.tex|&forwarding file for final version of chapter 2\\
% \end{tabular}
% \end{center}
% Each of the eight files can be compiled directly by the \LaTeX{} compiler.
%
% %%%%%%%%%%%%%%%%%%%%%%%%%%%%%%%%%%%%%%
% \paragraph{Main File.}
%
% The main file is called |cdocsamp.tex|.
%
% Load the \textsf{childdoc} definitions and
% declare the filename for the main document:
%    \begin{macrocode}
\input{childdoc.def}
\childdocmain{}
%    \end{macrocode}

% Optional override for |\version| flag:
%    \begin{macrocode}
%%\ifchilddoc\else\providecommand{\version}{draft}\fi
%    \end{macrocode}

% Define the default values for the |\version| flag
% (|final| for the main file and |draft| for childs):
%    \begin{macrocode}
\ifchilddoc
\providecommand{\version}{draft}
\else
\providecommand{\version}{final}
\fi
%    \end{macrocode}

% Load the standard document class:
%    \begin{macrocode}
\documentclass[12pt]{article}
%    \end{macrocode}

% Start the document body:
%    \begin{macrocode}
\begin{document}
%    \end{macrocode}

% Declare a title page.
% Print title, part of document being processed and version flag:
%    \begin{macrocode}
\addtocounter{page}{-1}
\begin{center}
{\LARGE\bfseries{}childdoc example\par}
\vspace{1cm}
\ifchilddoc
\ifchilddocmanual part\else chapter\fi:
`\childdocname' of `\childdocjob'\par
\else
main document: `\childdocjob'\par
\fi
version: \version\par
\end{center}
\newpage
%    \end{macrocode}

% Manually include selected file,
% otherwise process as usual:
%    \begin{macrocode}
\ifchilddocmanual
\section*{part `\childdocname'}
\input{\childdocname}
\else
%    \end{macrocode}

% Include the two chapters:
%    \begin{macrocode}
\include{cdocsch1}
\include{cdocsch2}
%    \end{macrocode}

% Include the two parts unless only chapters should be displayed:
%    \begin{macrocode}
\ifchilddoc\else
\section{part three}
\input{cdocspt3}
\section{part four}
\input{cdocspt4}
\fi
%    \end{macrocode}

% Process as usual until here:
%    \begin{macrocode}
\fi
%    \end{macrocode}

% End of document body:
%    \begin{macrocode}
\end{document}
%    \end{macrocode}
%\iffalse
%</samplemain>
%\fi
%
% %%%%%%%%%%%%%%%%%%%%%%%%%%%%%%%%%%%%%%
% \paragraph{Chapter Include Files.}
%
% The include files are called |cdocsch1.tex| and |cdocsch2.tex|.
%
%\iffalse
%<*samplechap1|samplechap2>
%\fi

% Optional override for |\version| flag:
%    \begin{macrocode}
%%\providecommand{\version}{final}
%    \end{macrocode}

% Include the main document:
%    \begin{macrocode}
\input{childdoc.def}
\childdocof{cdocsamp}
%    \end{macrocode}

%\iffalse
%</samplechap1|samplechap2>
%\fi
%
%\iffalse
%<*samplechap1>
%\fi
% Some text for chapter 1:
%    \begin{macrocode}
\section{one}
some text in chapter one
%    \end{macrocode}

%\iffalse
%</samplechap1>
%\fi
% Some text for chapter 2:
%\iffalse
%<*samplechap2>
%\fi
%    \begin{macrocode}
\section{two}
more text in chapter two
%    \end{macrocode}

%\iffalse
%</samplechap2>
%\fi
%
% %%%%%%%%%%%%%%%%%%%%%%%%%%%%%%%%%%%%%%
% \paragraph{Part Include Files.}
%
% The include files are called |cdocspt3.tex| and |cdocspt4.tex|.
%
%\iffalse
%<*samplepart3|samplepart4>
%\fi

% Optional override for |\version| flag:
%    \begin{macrocode}
%%\providecommand{\version}{final}
%    \end{macrocode}

% Include the main document:
%    \begin{macrocode}
\input{childdoc.def}
\childdocby{cdocsamp}
%    \end{macrocode}

%\iffalse
%</samplepart3|samplepart4>
%\fi
%
%\iffalse
%<*samplepart3>
%\fi
% Some text for part 3:
%    \begin{macrocode}
some text in part three
%    \end{macrocode}

%\iffalse
%</samplepart3>
%\fi
% Some text for part 4:
%\iffalse
%<*samplepart4>
%\fi
%    \begin{macrocode}
more text in part four
%    \end{macrocode}

%\iffalse
%</samplepart4>
%\fi
%
% %%%%%%%%%%%%%%%%%%%%%%%%%%%%%%%%%%%%%%
% \paragraph{Forwarding for a Complete Draft.}
%
% The following forwarding file |cdocsdrf.tex|
% compiles the main document in draft mode:
%\iffalse
%<*sampledraft>
%\fi
%    \begin{macrocode}
\def\version{draft}
\input{childdoc.def}
\childdocforward{cdocsamp}
%    \end{macrocode}

%\iffalse
%</sampledraft>
%\fi
%
% %%%%%%%%%%%%%%%%%%%%%%%%%%%%%%%%%%%%%%
% \paragraph{Forwarding for Final Version of the Chapters.}
%
% The following forwarding files |cdocsfn1.tex| and |cdocsfn2.tex|
% (with identical content)
% compile the final versions of the child documents
% |cdocsch1.tex| and |cdocsch2.tex|, respectively:
%\iffalse
%<*samplefinal>
%\fi
%    \begin{macrocode}
\def\version{final}
\input{childdoc.def}
\childdocforwardprefix[cdocsamp]{cdocsfn}{cdocsch}
%    \end{macrocode}

%\iffalse
%</samplefinal>
%\fi
%
% %%%%%%%%%%%%%%%%%%%%%%%%%%%%%%%%%%%%%%
% \paragraph{Command Line Processing.}
%
% The following three command lines generate the output files
% |cdocscld|, |cdocscl1| and |cdocscl2|
% which should be identical to
% |cdocsdrf|, |cdocsch1| and |cdocsfn2|, respectively:
% \begin{center}
% \begin{tabular}{l}
% |latex -jobname cdocscld \|\\
% |  "\def\version{draft}\input{childdoc.def}\childdocforward{cdocsamp}"|\\
% |latex -jobname cdocscl1 \|\\
% |  "\input{childdoc.def}\childdocforward[cdocsamp]{cdocsch1}"|\\
% |latex -jobname cdocscl2 \|\\
% |  "\def\version{final}\input{childdoc.def}\childdocforward{cdocsch2}"|
% \end{tabular}
% \end{center}
% Note that the trailing backslash on each first line
% merely continues the input to the second line
% (for convenient cut ant paste).
% Furthermore, the command |latex| can be replaced by any
% of its alternative versions such as |pdflatex|.
%
% %%%%%%%%%%%%%%%%%%%%%%%%%%%%%%%%%%%%%%%%%%%%%%%%%%%%%%%%%%%%%%%%%%%%%%%%%%%%%%
% %%%%%%%%%%%%%%%%%%%%%%%%%%%%%%%%%%%%%%%%%%%%%%%%%%%%%%%%%%%%%%%%%%%%%%%%%%%%%%
% \section{Implementation}
%\iffalse
%<*package>
%\fi
%
% This section describes the definitions file |childdoc.def|.

% The definitions cannot be loaded using |\usepackage| or |\RequirePackage|
% which has a mechanism to prevent loading a style file more than once.
% When loading the definitions by means of |\input|
% multiple instances have to be prevented manually:
%\iffalse
%This code needs to be before the `\ProvidesFile' directive
%which is defined at the beginning of this file.
%Therefore it is also placed there and commented out here.
%</package>
%<*discard>
%\fi
%    \begin{macrocode}
\ifdefined\childdocmain\endinput\fi
%    \end{macrocode}
%\iffalse
%</discard>
%<*package>
%\fi
%
% \macro{\ifchilddoc}
% \macro{\ifchilddocmanual}
% The conditional |\ifchilddoc| tells whether a
% child (true) or main (false) document is being compiled.
% The conditional |\ifchilddocmanual| tells whether
% the |\includeonly| mechanism is used (false) or
% the selection of child files must be performed manually (true).
% The definitions initialise to false:
%    \begin{macrocode}
\newif\ifchilddoc
\newif\ifchilddocmanual
%    \end{macrocode}

% \macro{\childdocname}
% \macro{\childdocjob}
% The macro |\childdocname| stores the name of the main document
% to be compiled. The macro |\childdocjob| stores the name of
% the document on which the \LaTeX{} compiler was originally invoked.
% The content of |\jobname| cannot be compared
% to filenames specified in the source due to different catcodes.
% The following code rescans |\jobname|, stores the result
% in |\childdocname| and saves a copy in |\childdocjob|:
%    \begin{macrocode}
\edef\childdocname{\scantokens\expandafter{\jobname\noexpand}}
\let\childdocjob\childdocname
%    \end{macrocode}

% \macro{\childdocdisable}
% The macro |\childdocdisable| prevents the main file
% from being processed more than once.
% At this stage, the main document command |\childdocmain|
% is assumed to be called once again where it should do nothing.
% Any subsequent call to it should prevent
% a secondary processing of the main document
% It overwrites the forwarding commands
% |\childdocof| and |\childdocforward|
% with empty macros to prevent further inclusions of the main document:
%    \begin{macrocode}
\newcommand{\childdocdisable}
{
  \renewcommand{\childdocmain}[1]{\renewcommand{\childdocmain}[1]{\endinput}}
  \renewcommand{\childdocof}[1]{}
  \renewcommand{\childdocby}[2][]{}
  \renewcommand{\childdocforward}[2][]{}
  \renewcommand{\childdocdisable}{}
}
%    \end{macrocode}

% \macro{\childdocmain}
% The macro |\childdocmain| is to be called at the top of the main file
% with nothing or the main filename (without extension) as argument.
% First, it breaks loops.
% If the argument is not empty and does not match |\childdocname|
% (which is set by the first inclusion of |childdoc.def|),
% |\ifchilddoc| is set to true, |\includeonly| is applied to the child file
% and |\jobname| is set to the main file
% (for proper handling of |.aux| files):
%    \begin{macrocode}
\newcommand{\childdocmain}[1]
{
  \childdocdisable\childdocmain{}
  \if?#1?\else
    \begingroup
      \def\childdoctmp{#1}
      \ifx\childdoctmp\childdocname
        \def\childdoctmp{}
      \else
        \def\childdoctmp
        {
          \childdoctrue
          \includeonly{\childdocname}
          \def\childdocjob{#1}
          \def\jobname{#1}
        }
      \fi
      \expandafter
    \endgroup
    \childdoctmp
  \fi
}
%    \end{macrocode}

% \macro{\childdocof}
% The command |\childdocof| redirects
% compilation to the main file |#1|.
%    \begin{macrocode}
\newcommand{\childdocof}[1]
{
  \childdocdisable
  \childdoctrue
  \includeonly{\childdocname}
  \def\jobname{#1}
  \def\childdocjob{#1}
  \input{#1}
}
%    \end{macrocode}

% \macro{\childdocby}
% The command |\childdocby| ....
%    \begin{macrocode}
\newcommand{\childdocby}[2][]
{
  \childdocdisable
  \childdoctrue
  \childdocmanualtrue
  \if?#1?\else
    \def\jobname{#2}
  \fi
  \def\childdocjob{#2}
  \input{#2}
  \endinput
}
%    \end{macrocode}

% \macro{\childdocforward}
% The command |\childdocforward| redirects
% compilation to the main file or
% (if the optional argument is given) a child file.
% Parameters are set as if the main file
% or a child file starting with |\childdocof| was compiled.
% Then compilation is handed over to the main file:
%    \begin{macrocode}
\newcommand{\childdocforward}[2][]
{
  \begingroup
    \if?#1?
      \def\childdoctmp
      {
        \def\childdocname{#2}
        \def\childdocjob{#2}
        \def\jobname{#2}
        \input{#2}
        \endinput
      }
    \else
      \def\childdoctmp
      {
        \childdocdisable
        \def\childdocname{#2}
        \childdoctrue
        \includeonly{#2}
        \def\childdocjob{#1}
        \def\jobname{#1}
        \input{#1}
        \endinput
      }
    \fi
    \expandafter
  \endgroup
  \childdoctmp
}
%    \end{macrocode}

% \macro{\childdocforwardprefix}
% The command |\childdocforwardprefix| redirects
% compilation to the main or a child file by means of a pattern.
% The prefix |#1| in the current filename is replaced by |#2|
% and the suffix of the current filename is kept
% (it is assumed that the filename does not contain the substring `|~~~|'
% which is used as a delimiter).
% Compilation is handed over to the new file by |\childdocforward|:
%    \begin{macrocode}
\newcommand{\childdocforwardprefix}[3][]
{
  \begingroup
    \def\childdocextract #2##1~~~{\def\childdoctmp{\childdocforward[#1]{#3##1}}}
    \expandafter\childdocextract\childdocname~~~
    \expandafter
  \endgroup
  \childdoctmp
}
%    \end{macrocode}

% \macro{\childdoc}
% The deprecated macro |\childdoc| is a legacy version of |\childdocmain|:
%    \begin{macrocode}
\newcommand{\childdoc}{\childdocmain}
%    \end{macrocode}

% \macro{\childdocredirect}
% The deprecated macro |\childdocredirect| is a legacy version
% of |\childdocforward| and |\childdocforwardprefix|:
%    \begin{macrocode}
\newcommand{\childdocredirect}[2][]
{
  \begingroup
    \if?#1?
      \def\childdoctmp{\childdocforward{#2}}
    \else
      \def\childdoctmp{\childdocforwardprefix{#1}{#2}}
    \fi
    \expandafter
  \endgroup
  \childdoctmp
}
%    \end{macrocode}

%\iffalse
%</package>
%\fi
%
\endinput
|\\
|\childdocforwardprefix[|\textit{main}|]{|\textit{prefix}|}{|\textit{dest}|}|
\end{tabular}
\end{center}
%
the destination file is determined by a pattern
depending on the current file:
To make this work, the current file must be called
`{\textit{prefix}\hspace{0.2em}\textit{suffix}}'
with \textit{prefix} matching precisely the argument.
Processing is then passed on to the file
`{\textit{dest}\hspace{0.2em}\textit{suffix}}'.
Surely, the same effect is achieved by
directly specifying the
argument `{\textit{dest}\hspace{0.2em}\textit{suffix}}'
in the first form.
However, that requires to set up a different file
for each child. With the alternative form of the command
all these files can have exactly the same content
which simplifies setting them up and maintaining them.

For example, the following file |draft.tex|
with a compilation flag |\version| as described in \secref{sec:flags}
compiles the main document as a draft:
%
\begin{center}
\begin{tabular}{l}
|\def\version{draft}|\\
|% \iffalse
%
% childdoc.dtx Copyright (C) 2017-2018 Niklas Beisert
%
% This work may be distributed and/or modified under the
% conditions of the LaTeX Project Public License, either version 1.3
% of this license or (at your option) any later version.
% The latest version of this license is in
%   http://www.latex-project.org/lppl.txt
% and version 1.3 or later is part of all distributions of LaTeX
% version 2005/12/01 or later.
%
% This work has the LPPL maintenance status `maintained'.
%
% The Current Maintainer of this work is Niklas Beisert.
%
% This work consists of the files childdoc.dtx and childdoc.ins
% and the derived files childdoc.def and cdocsamp.tex with
% cdocsch1.tex, cdocsch2.tex, cdocsdrf.tex, cdocsfn1.tex, cdocsfn2.tex.
%
%<package>\ifdefined\childdocmain\endinput\fi
%<package>\ProvidesFile{childdoc.def}[2018/12/30 v2.0 child document driver]
%<samplemain>\ProvidesFile{cdocsamp.tex}[2018/12/30 v2.0 sample for childdoc]
%<*driver>
%\ProvidesFile{childdoc.drv}[2018/12/30 v2.0 childdoc reference manual file]
\PassOptionsToClass{10pt,a4paper}{article}
\documentclass{ltxdoc}

\usepackage[margin=35mm]{geometry}
\usepackage{hyperref}
\usepackage{hyperxmp}
\usepackage[usenames]{color}

\hypersetup{colorlinks=true}
\hypersetup{pdfstartview=FitH}
\hypersetup{pdfpagemode=UseNone}
\hypersetup{pdfsource={}}
\hypersetup{pdflang={en-UK}}
\hypersetup{pdfcopyright={Copyright 2017-2018 Niklas Beisert.
  This work may be distributed and/or modified under the
  conditions of the LaTeX Project Public License, either version 1.3
  of this license or (at your option) any later version.}}
\hypersetup{pdflicenseurl={http://www.latex-project.org/lppl.txt}}
\hypersetup{pdfcontactaddress={ETH Zurich, ITP, HIT K,
  Wolfgang-Pauli-Strasse 27}}
\hypersetup{pdfcontactpostcode={8093}}
\hypersetup{pdfcontactcity={Zurich}}
\hypersetup{pdfcontactcountry={Switzerland}}
\hypersetup{pdfcontactemail={nbeisert@itp.phys.ethz.ch}}
\hypersetup{pdfcontacturl={http://people.phys.ethz.ch/\xmptilde nbeisert/}}

\newcommand{\secref}[1]{\hyperref[#1]{section \ref*{#1}}}

\parskip1ex
\parindent0pt
\let\olditemize\itemize
\def\itemize{\olditemize\parskip0pt}

\begin{document}

\title{The \textsf{childdoc} Package}
\hypersetup{pdftitle={The childdoc Package}}
\author{Niklas Beisert\\[2ex]
  Institut f\"ur Theoretische Physik\\
  Eidgen\"ossische Technische Hochschule Z\"urich\\
  Wolfgang-Pauli-Strasse 27, 8093 Z\"urich, Switzerland\\[1ex]
  \href{mailto:nbeisert@itp.phys.ethz.ch}
  {\texttt{nbeisert@itp.phys.ethz.ch}}}
\hypersetup{pdfauthor={Niklas Beisert}}
\hypersetup{pdfsubject={Manual for the LaTeX2e Package childdoc}}
\date{30 December 2018, \textsf{v2.0}}
\maketitle

\begin{abstract}\noindent
\textsf{childdoc} is a \LaTeXe{} package
that enables the direct compilation
of document sections included by |\include|
to individual files.
\end{abstract}

\begingroup
\parskip0ex
\tableofcontents
\endgroup

%%%%%%%%%%%%%%%%%%%%%%%%%%%%%%%%%%%%%%%%%%%%%%%%%%%%%%%%%%%%%%%%%%%%%%%%%%%%%%%%
%%%%%%%%%%%%%%%%%%%%%%%%%%%%%%%%%%%%%%%%%%%%%%%%%%%%%%%%%%%%%%%%%%%%%%%%%%%%%%%%
\section{Introduction}

\LaTeX{} provides a mechanism to structure a large document (such as a book)
into a main file and several child files (containing the chapters)
using the |\include| command.
This mechanism is beneficial for documents
which span hundreds of pages in order to
make the source file(s) more manageable.
Moreover, compilation can be restricted to
selected child files by means of the |\includeonly| command.
The latter feature can be used to reduce the compilation time while editing
(this was significantly more useful in the earlier days of \LaTeX{})
or to generate a smaller document which is easier to navigate.
Another application of |\includeonly| is to generate
documents consisting of selected parts of the complete document.

However, there are a few drawbacks of the plain |\include| mechanism:
\begin{itemize}
\item
The child files cannot be compiled on their own,
they can only be compiled via the main file.
A naive editing environment
(such as a text editor with an option
to have the current file processed by \LaTeX)
may require one to switch to the main file before compiling;
attempting to compile the child file produces errors.
\item
The main file must be modified (each time)
to adjust the |\includeonly| command
to the present needs. This easily leaves the main file in a messy state.
\item
The generated document will always carry the filename
of the main document. This is inconvenient if
several child files are to be compiled and
to be kept for distribution.
\end{itemize}

The present package provides a simple interface
to make child files individually compilable by \LaTeX{}.
Compiling a child file then has the same effect as compiling
the main file with an |\includeonly| command
to select the appropriate child.
Moreover the generated document will carry the name of the child
rather than the main file.
This resolves all three above issues.

This feature is meant to make the editing of books,
thesis documents and lecture notes somewhat more convenient.
However, the package can also be used efficiently for
composing a series of documents (such as exercise sheets)
which are typically distributed individually.
It then assists the author in generating the individual documents
(potentially in different versions)
as well as a document containing the collected series.
Another application is in developing style files
or other kinds of included material
where compilation of the style file could redirect
to a sample or test file.

%%%%%%%%%%%%%%%%%%%%%%%%%%%%%%%%%%%%%%%%%%%%%%%%%%%%%%%%%%%%%%%%%%%%%%%%%%%%%%%%
%%%%%%%%%%%%%%%%%%%%%%%%%%%%%%%%%%%%%%%%%%%%%%%%%%%%%%%%%%%%%%%%%%%%%%%%%%%%%%%%
\section{Usage}

First of all, the package \textsf{childdoc} is \emph{not} a standard
\LaTeXe{} |.sty| style file! Therefore it needs to be invoked in
a non-standard way.

%%%%%%%%%%%%%%%%%%%%%%%%%%%%%%%%%%%%%%%%%%%%%%%%%%%%%%%%%%%%%%%%%%%%%%%%%%%%%%%%
\subsection{Included Files}
\label{sec:include}

%%%%%%%%%%%%%%%%%%%%%%%%%%%%%%%%%%%%%%%%
\DescribeMacro{\childdocmain}
To use the package, add the commands
\begin{center}
\begin{tabular}{l}
|\input{childdoc.def}|\\
|\childdocmain{}|\\
\end{tabular}
\end{center}
at the very top of the main \LaTeX{} file,
in particular \emph{before} the |\documentclass| statement!
The argument of |\childdocmain| should be left empty
(but it must be present).

%%%%%%%%%%%%%%%%%%%%%%%%%%%%%%%%%%%%%%%%
\DescribeMacro{\childdocof}
Furthermore, add the commands
\begin{center}
\begin{tabular}{l}
|\input{childdoc.def}|\\
|\childdocof{|\textit{main}|}|\\
\end{tabular}
\end{center}
at the top of every child file \textit{child}
which is included by |\include{|\textit{child}|}|
from within the main file
(or at least for those files to be compiled individually).
The argument \textit{main} must be the filename of the main file.

There are a couple of
considerations in setting up the main and child documents:

%%%%%%%%%%%%%%%%%%%%%%%%%%%%%%%%%%%%%%%%
\paragraph{Restrictions.}

Please note the following restrictions:
\begin{itemize}
\item
|\childdocmain| must be called with one argument \textit{main}
to ensure compatibility with earlier version of the package.
It must either be empty (|\childdocmain{}|)
or precisely match the filename of the main file in which it is specified.
See \secref{sec:detection} for further information.
\item
The filename \textit{main} must be specified without the |.tex| extension.
\item
The filename \textit{main} is case sensitive
(even in case-insensitive file systems)
due to internal string comparison.
\item
The argument \textit{main} should be fully expanded, it cannot be a macro.
\item
Subdirectories and special characters should be avoided in filenames.
\item
The command |\childdocmain{|\textit{main}|}| must be followed by a whitespace.
It should not be followed immediately by another command
or by a comment mark `|%|'.
This is because the \TeX{} parser reads the token immediately following
the argument of |\childdocmain| and puts it
at the beginning of every child section;
however, a white\-space is ignored.
\end{itemize}

%%%%%%%%%%%%%%%%%%%%%%%%%%%%%%%%%%%%%%%%
\paragraph{Content of Main File.}

It is advisable to place all content in the child files included by |\include|.
Any output contained in the main file will appear in all child documents
unless suppressed manually;
it cannot be suppressed automatically by the |\includeonly| directive
and thus should normally be avoided.
A method to include some content in the main file
by means of conditional processing is described in \secref{sec:conditional}.

%%%%%%%%%%%%%%%%%%%%%%%%%%%%%%%%%%%%%%%%
\paragraph{Page Numbering.}

When only a part of the document is compiled,
the appropriate numbering of pages
(as well as other status parameters)
is determined from the |.aux| files.
The latter contain information from previous passes.
However this information needs to propagate through
all intermediate child documents.
Therefore the page numbering in child documents may well
be inconsistent until the complete document is compiled at least once.

A useful (if unconventional) way to always ensure a consistent
page numbering is to restart the numbering in each child document
and denote the pages by `\textit{child}|.|\textit{page}'
where \textit{child} represents the chapter/section number of the child file.
This can be achieved by the command
|\numberwithin{page}{|\textit{child}|}|
of the \textsf{amsmath} package
where \textit{child} can be |chapter| or |section|
depending on the chosen structuring.
Alternatively, one can modify the macro |\thepage| appropriately
and reset the counter |page| at the start of each child file.

%%%%%%%%%%%%%%%%%%%%%%%%%%%%%%%%%%%%%%%%%%%%%%%%%%%%%%%%%%%%%%%%%%%%%%%%%%%%%%%%
\subsection{Conditional Processing}
\label{sec:conditional}

The package provides a mechanism to compile different versions
of a document. To customise the versions further some conditional processing
can come in handy to distinguish which version is being compiled.
The package provides two macros to describe the compilation context:

%%%%%%%%%%%%%%%%%%%%%%%%%%%%%%%%%%%%%%%%
\DescribeMacro{\ifchilddoc}
The conditional |\ifchilddoc| distinguishes between the compilation of
child documents and the main document:
%
\begin{center}
|\ifchilddoc |\textit{child-code}| |[|\||else |\textit{main-code}]| \||fi|
\end{center}

%%%%%%%%%%%%%%%%%%%%%%%%%%%%%%%%%%%%%%%%
\DescribeMacro{\childdocname}
\DescribeMacro{\childdocjob}
The macro |\childdocname| contains the filename (without extension)
of the main or child file being processed.
Note that |\childdocjob| will always contain the name of the main file.

%%%%%%%%%%%%%%%%%%%%%%%%%%%%%%%%%%%%%%%%
\paragraph{Title Page.}

Conditional processing can be used to include a title or banner page
in the main document when proper precautions are taken.
Importantly, the code in the main file should ensure that the page counter
(as well as other status parameters which are stored in the |.aux| files)
takes the same value after the conditional processing.
Otherwise the page numbers may take divergent values
depending on which part is compiled.

For example, a title page could be declared by:
%
\begin{center}
\begin{tabular}{l}
|\ifchilddoc\||else|\\
|\addtocounter{page}{-1}|\\
\textit{code for title page}\\
|\newpage|\\
|\||fi|
\end{tabular}
\end{center}
%
A banner page for the child documents can be generated by:
%
\begin{center}
\begin{tabular}{l}
|\ifchilddoc|\\
|\addtocounter{page}{-1}|\\
\textit{code for banner page}\\
|\newpage|\\
|\||fi|
\end{tabular}
\end{center}
%
Here one could write a message such as:
\begin{center}
|This is the part \childdocname{} of \childdocjob{}.|
\end{center}

%%%%%%%%%%%%%%%%%%%%%%%%%%%%%%%%%%%%%%%%%%%%%%%%%%%%%%%%%%%%%%%%%%%%%%%%%%%%%%%%
\subsection{Flags}
\label{sec:flags}

The package makes it easy to generate different versions
of the main or child documents.
To this end compilation flags can be defined
and assigned different default values.
They will be particularly useful in conjunction
with the forwarding mechanism described in \secref{sec:forward}.

For example, it may be useful to have a flag |\version|
which can be set to |draft| or |final|.
The document source will contain some conditional code
depending on the value of |\version|.
Suppose further, the flag should default to |final| for the main file
and to |draft| for child files
which is a natural assignment for editing the document.
This is achieved by placing the following code
in the preamble of the main document
(below the |\childdocmain| directive):
%
\begin{center}
\begin{tabular}{l}
|\ifchilddoc|\\
|\providecommand{\version}{draft}|\\
|\||else|\\
|\providecommand{\version}{final}|\\
|\||fi|
\end{tabular}
\end{center}
%
The definition by |\providecommand| makes sure
that previous definitions are not overwritten.
Further statements |\providecommand{\version}{...}|
can thus be added before the above code to override it.

For the main file, one might add a line
(between |\childdocmain| and the above block)
%
\begin{center}
|%\ifchilddoc\||else\providecommand{\version}{draft}\||fi|
\end{center}
%
which can be uncommented to produce a draft version.
Likewise one can add a line to the very top of a child file
(above the |\childdocof{|\textit{main}|}| directive)
%
\begin{center}
|%\providecommand{\version}{final}|
\end{center}
%
which can be uncommented to produce the final version of this child document.

%%%%%%%%%%%%%%%%%%%%%%%%%%%%%%%%%%%%%%%%%%%%%%%%%%%%%%%%%%%%%%%%%%%%%%%%%%%%%%%%
\subsection{Forwarding}
\label{sec:forward}

Different versions of the main or child documents
using compilation flags as described in \secref{sec:flags}
can be (permanently) stored in different files
for convenient compilation, viewing and distribution.
To this end, the package defines a command
to pass on compilation to a different file:

%%%%%%%%%%%%%%%%%%%%%%%%%%%%%%%%%%%%%%%%
\DescribeMacro{\childdocforward}
The command |\childdocforward| redirects processing to
another source file:
%
\begin{center}
\begin{tabular}{l}
|\input{childdoc.def}|\\
|\childdocforward[|\textit{main}|]{|\textit{dest}|}|\\
\end{tabular}
\end{center}
%
The argument \textit{dest} is the destination file
(without extension).
It should be the main file or one of the child files.
Note that further \textsf{childdoc} directives
such as |\childdocof| and |\childdocforward|
in the indicated file will be processed in this form.
The optional argument \textit{main}
passes on directly to the main file \textit{main}
while pretending to compile the child \textit{dest}.
This form behaves as if \textit{dest}
issues |\childdocof{|\textit{main}|}| right away,
and no further \textsf{childdoc} directives will be processed.

%%%%%%%%%%%%%%%%%%%%%%%%%%%%%%%%%%%%%%%%
\DescribeMacro{\...prefix}
In the alternative form |\childdocforwardprefix|,
%
\begin{center}
\begin{tabular}{l}
|\input{childdoc.def}|\\
|\childdocforwardprefix[|\textit{main}|]{|\textit{prefix}|}{|\textit{dest}|}|
\end{tabular}
\end{center}
%
the destination file is determined by a pattern
depending on the current file:
To make this work, the current file must be called
`{\textit{prefix}\hspace{0.2em}\textit{suffix}}'
with \textit{prefix} matching precisely the argument.
Processing is then passed on to the file
`{\textit{dest}\hspace{0.2em}\textit{suffix}}'.
Surely, the same effect is achieved by
directly specifying the
argument `{\textit{dest}\hspace{0.2em}\textit{suffix}}'
in the first form.
However, that requires to set up a different file
for each child. With the alternative form of the command
all these files can have exactly the same content
which simplifies setting them up and maintaining them.

For example, the following file |draft.tex|
with a compilation flag |\version| as described in \secref{sec:flags}
compiles the main document as a draft:
%
\begin{center}
\begin{tabular}{l}
|\def\version{draft}|\\
|\input{childdoc.def}|\\
|\childdocforward{|\textit{main}|}|
\end{tabular}
\end{center}
%
Likewise, the following files |final|\textit{nn}|.tex|
compile the final version of the child document
|child|\textit{nn}|.tex|:
%
\begin{center}
\begin{tabular}{l}
|\def\version{final}|\\
|\input{childdoc.def}|\\
|\childdocforwardprefix{final}{child}|
\end{tabular}
\end{center}
%

Note that when several versions of a main file and/or of each child file
are to be generated, it may be convenient to set up a |Makefile| or
shell script to automatise the process.

%%%%%%%%%%%%%%%%%%%%%%%%%%%%%%%%%%%%%%%%%%%%%%%%%%%%%%%%%%%%%%%%%%%%%%%%%%%%%%%%
\subsection{Command Line Processing}
\label{sec:commandline}

The effect of redirection files can also be achieved by invoking
the \LaTeX{} compiler with a more elaborate command line.
Most conveniently this should be done as part
of a shell script or a |Makefile|.

When using \textsf{childdoc} in the main file, the following
command lines effectively perform a redirection
(note that depending on the shell being used,
backslashes may have to be doubled: `|\|' $\to$ `|\\|'):
%
\begin{center}
|... -jobname "|\textit{target}|" |\\|"|[\textit{flags}]%
|\input{childdoc.def}\childdocforward[|\textit{main}|]{|\textit{dest}|}"|
\end{center}
%
Here \textit{target} is the name of the output file,
\textit{main} is the name of the main file
and \textit{dest} is the name of the main or child file to be processed
(all filenames without extensions).
The optional argument \textit{main} can be omitted
if \textit{main} matches \textit{dest}.
Optionally, compilation \textit{flags} can be defined via |\def| commands.
This command line makes the \TeX{} engine believe
it is compiling the file \textit{target}
whose content is specified as the latter parameter.
The provided code then forwards the processing to
\textit{main} or \textit{dest} as described in \secref{sec:forward}.

%%%%%%%%%%%%%%%%%%%%%%%%%%%%%%%%%%%%%%%%%%%%%%%%%%%%%%%%%%%%%%%%%%%%%%%%%%%%%%%%
\subsection{Include by Input}
\label{sec:input}

Including child documents by |\include| has some restrictions by design.
Most notably, the content of a child document always occupies
its own set of pages; pages cannot be shared between child documents.
Usually, this behaviour makes perfect sense
because each child document contain an essential part of the document.
However, in some situations it may be desirable to compose
a document from a collection of parts
without having mandatory page breaks between then.
For this case, the package
provides a mechanism to include parts
by |\input| which can also be processed individually.
However, by construction this mechanism
requires manual handling of the content to be output.

%%%%%%%%%%%%%%%%%%%%%%%%%%%%%%%%%%%%%%%%
\DescribeMacro{\ifchilddocmanual}
The main file should be prepared as usual, see \secref{sec:include}.
However, the document body must make a distinction
between processing of an individual part and of the main document, e.g.:
%
\begin{center}
\begin{tabular}{l}
|\ifchilddocmanual|\\
|\input{\childdocname}|\\
|\||else|\\
\textit{document body with }|\input{|\textit{part}|}|\\
|\||fi|
\end{tabular}
\end{center}
%
The conditional |\ifchilddocmanual| is true whenever
a part to be included by |\input| is being compiled,
and the name of the part is stored in |\childdocname|.

%%%%%%%%%%%%%%%%%%%%%%%%%%%%%%%%%%%%%%%%
\DescribeMacro{\childdocby}
Each part to be included by |\input| should start with:
%
\begin{center}
\begin{tabular}{l}
|\input{childdoc.def}|\\
|\childdocby{|\textit{main}|}|\\
\end{tabular}
\end{center}
%
The directive |\childdocby| is similar to |\childdocof|
described in \secref{sec:include},
but the subsequent selection of content must be done manually.
To that end, both |\ifchilddoc| and |\ifchilddocmanual|
will be true upon processing of a part,
and the name of the part is stored in |\childdocname|.
Note that |\jobname| will be set to the filename of the current part
so that each part receives an individual |.aux| file
that does not interfere with the |.aux| file(s) of the main document.
This behaviour can be altered by the alternative form
|\childdocby[*]{|\textit{main}|}| (with a non-empty optional argument)
which uses the |.aux| file of the main document
by setting |\jobname| to \textit{main}.

%%%%%%%%%%%%%%%%%%%%%%%%%%%%%%%%%%%%%%%%%%%%%%%%%%%%%%%%%%%%%%%%%%%%%%%%%%%%%%%%
\subsection{Driver Development}
\label{sec:driver}

The \textsf{childdoc} mechanism can also be use for the development
of definition files such as \LaTeX{} styles or classes.
This case differs from the above setup with multiple parts
included by |\include| in that no |\includeonly| should be invoked.
This can be achieved by starting the include file
(before |\ProvidesPackage|) with:
%
\begin{center}
\begin{tabular}{l}
|\input{childdoc.def}|\\
|\childdocforward{|\textit{main}|}|\\
\end{tabular}
\end{center}
%
or alternatively with:
%
\begin{center}
\begin{tabular}{l}
|\input{childdoc.def}|\\
|\childdocby{|\textit{main}|}|\\
\end{tabular}
\end{center}
%
Both forms have slightly different effects as described above.
The main file is prepared as usual, see \secref{sec:include}.

%%%%%%%%%%%%%%%%%%%%%%%%%%%%%%%%%%%%%%%%%%%%%%%%%%%%%%%%%%%%%%%%%%%%%%%%%%%%%%%%
\subsection{Legacy Detection}
\label{sec:detection}

The directive |\childdocmain| in the main file can detect
whether the complete document or merely a child is to be compiled
even without using the directive |\childdocof|.
This method is deprecated because it is less robust
and there is no compelling reason to use it;
it is merely provided for backward compatibility
and it may be removed in future versions.

If the detection mechanism is to be used,
it is mandatory to correctly specify
the filename of the main file as the argument of |\childdocmain|:
%
\begin{center}
\begin{tabular}{l}
|\input{childdoc.def}|\\
|\childdocmain{|\textit{main}|}|\\
\end{tabular}
\end{center}
%
If |\jobname| does not match the argument \textit{main} of |\childdocmain|,
it is assumed that |\jobname| points to the child file to be compiled.
When using |\childdocmain| with the main file specified as argument,
it suffices to start a child file
with just |\input{|\textit{main}|}|
without loading of the package and using |\childdocof|.
If instead all processing is done
with the appropriate \textsf{childdoc} directives,
the argument of \textit{main} of |\childdocmain| can be empty.

An alternative version of the command line processing described
in \secref{sec:commandline} using the detection mechanism reads:
%
\begin{center}
|... -jobname "|\textit{target}|" "|[\textit{flags}]%
[|\def\jobname{|\textit{dest}|}|]|\input{|\textit{main}|}"|
\end{center}

%%%%%%%%%%%%%%%%%%%%%%%%%%%%%%%%%%%%%%%%%%%%%%%%%%%%%%%%%%%%%%%%%%%%%%%%%%%%%%%%
\subsection{Manual Code}
\label{sec:manual}

In case one cannot be certain whether the definitions file |childdoc.def|
is installed on the target \TeX{} distribution
and one prefers not to ship it,
it is conceivable to paste a few relevant commands into the sources.

To that end, drop all statements |\input{childdoc.def}|
and perform the replacements as outlined below.
Instead of |\childdocmain{|\textit{main}|}| add the following code
to the top of the main file:
%
\begin{center}
\begin{tabular}{l}
|\||ifdefined\childdocname\endinput\||fi\newif\ifchilddoc|\\
|\edef\childdocname{\scantokens\expandafter{\jobname\noexpand}}|\\
|\def\childdocmain{|\textit{main}|}\||ifx\childdocmain\childdocname\||else|\\
|\childdoctrue\includeonly{\childdocname}\let\jobname\childdocmain\||fi|\\
\end{tabular}
\end{center}
%
Instead of |\childdocof{|\textit{main}|}| just include the main file
at the top of each child file:
%
\begin{center}
|\input{|\textit{main}|}|
\end{center}
%
A simple redirection |\childdocforward{|\textit{dest}|}| is achieved by:
%
\begin{center}
|\def\jobname{|\textit{dest}|}\input{\jobname}|
\end{center}
%
The redirection with prefix
|\childdocforwardprefix[|\textit{prefix}|]{|\textit{dest}|}|
is accomplished by:
%
\begin{center}
\begin{tabular}{l}
|{\edef\jobname{\scantokens\expandafter{\jobname\noexpand}}|\\
|\def\redirectjob |\textit{prefix}|#1~~~{\gdef\jobname{|\textit{dest}|#1}}|\\
|\expandafter\redirectjob\jobname~~~}\input{\jobname}|
\end{tabular}
\end{center}

In an alternative approach,
child documents can be compiled by a specific command line
without additional code or specific definitions:
%
\begin{center}
|... -jobname "|\textit{target}|" "|[\textit{flags}]%
|\includeonly{|\textit{dest}|}\input{|\textit{main}|}"|
\end{center}
%

%%%%%%%%%%%%%%%%%%%%%%%%%%%%%%%%%%%%%%%%%%%%%%%%%%%%%%%%%%%%%%%%%%%%%%%%%%%%%%%%
%%%%%%%%%%%%%%%%%%%%%%%%%%%%%%%%%%%%%%%%%%%%%%%%%%%%%%%%%%%%%%%%%%%%%%%%%%%%%%%%
\section{Information}

%%%%%%%%%%%%%%%%%%%%%%%%%%%%%%%%%%%%%%%%%%%%%%%%%%%%%%%%%%%%%%%%%%%%%%%%%%%%%%%%
\subsection{Copyright}

Copyright \copyright{} 2017--2018 Niklas Beisert

This work may be distributed and/or modified under the
conditions of the \LaTeX{} Project Public License, either version 1.3
of this license or (at your option) any later version.
The latest version of this license is in
  \url{http://www.latex-project.org/lppl.txt}
and version 1.3 or later is part of all distributions of \LaTeX{}
version 2005/12/01 or later.

This work has the LPPL maintenance status `maintained'.

The Current Maintainer of this work is Niklas Beisert.

This work consists of the files |README.txt|, |childdoc.ins| and |childdoc.dtx|
as well as the derived files |childdoc.def|, |cdocsamp.tex|
with |cdocsch1.tex|, |cdocsch2.tex|, |cdocspt3.tex|, |cdocspt4.tex|,
|cdocsdrf.tex|, |cdocsfn1.tex|, |cdocsfn2.tex|
as well as |childdoc.pdf|.

%%%%%%%%%%%%%%%%%%%%%%%%%%%%%%%%%%%%%%%%%%%%%%%%%%%%%%%%%%%%%%%%%%%%%%%%%%%%%%%%
\subsection{Files and Installation}

The package consists of the files:
%
\begin{center}
\begin{tabular}{ll}
    |README.txt|   & readme file \\
    |childdoc.ins| & installation file \\
    |childdoc.dtx| & source file \\
    |childdoc.def| & definition file \\
    |cdocsamp.tex| & sample main file \\
    |cdocsch1.tex| & sample include file \\
    |cdocsch2.tex| & sample include file \\
    |cdocspt3.tex| & sample part file \\
    |cdocspt4.tex| & sample part file \\
    |cdocsdrf.tex| & sample redirection file \\
    |cdocsfn1.tex| & sample redirection file \\
    |cdocsfn2.tex| & sample redirection file \\
    |childdoc.pdf| & manual
\end{tabular}
\end{center}
%
The distribution consists of the files
|README.txt|, |childdoc.ins| and |childdoc.dtx|.
%
\begin{itemize}
\item
Run (pdf)\LaTeX{} on |childdoc.dtx|
to compile the manual |childdoc.pdf| (this file).
\item
Run \LaTeX{} on |childdoc.ins| to create the definitions file |childdoc.def|
and the sample |cdocsamp.tex| with include files
|cdocsch1.tex|, |cdocsch2.tex|, |cdocspt3.tex|, |cdocspt4.tex|,
|cdocsdrf.tex|, |cdocsfn1.tex|, |cdocsfn2.tex|.
Then copy the file |childdoc.def| to an appropriate directory of your \LaTeX{}
distribution, e.g.\ \textit{texmf-root}|/tex/latex/childdoc|.
\end{itemize}

%%%%%%%%%%%%%%%%%%%%%%%%%%%%%%%%%%%%%%%%%%%%%%%%%%%%%%%%%%%%%%%%%%%%%%%%%%%%%%%%
\subsection{Related CTAN Packages}

There are several other packages which offer a similar functionality:
%
\begin{itemize}
\item
The packages
\href{http://ctan.org/pkg/docmute}{\textsf{docmute}},
\href{http://ctan.org/pkg/includex}{\textsf{includex}} and
\href{http://ctan.org/pkg/standalone}{\textsf{standalone}}
provide commands to include only the document body of
a child file thus allowing both files to be compiled individually.
\item
The packages \href{http://ctan.org/pkg/subdocs}{\textsf{subdocs}}
and \href{http://ctan.org/pkg/subfiles}{\textsf{subfiles}}
provide structures in which the main and child documents can be
encapsulated and allowing them to be compiled individually.
The inclusion mechanism is different from the conventional |\include|.
\item
The package \href{http://ctan.org/pkg/combine}{\textsf{combine}}
is an elaborate solution to combine several documents into one.
\end{itemize}
%
See also the CTAN topic \href{http://ctan.org/topic/subdocs}{\textsf{subdocs}}
for further related packages.
The present package differs from the above solutions in that
a document structure constructed with the conventional |\include| mechanism
just needs two extra commands at the top of every file
such that all constituent files can be compiled individually.

%%%%%%%%%%%%%%%%%%%%%%%%%%%%%%%%%%%%%%%%%%%%%%%%%%%%%%%%%%%%%%%%%%%%%%%%%%%%%%%%
%\subsection{Feature Suggestions}
%
%The following is a list of features which may be useful for future
%versions of this package:
%%
%\begin{itemize}
%\item
%\ldots
%\end{itemize}

%%%%%%%%%%%%%%%%%%%%%%%%%%%%%%%%%%%%%%%%%%%%%%%%%%%%%%%%%%%%%%%%%%%%%%%%%%%%%%%%
\subsection{Revision History}

%%%%%%%%%%%%%%%%%%%%%%%%%%%%%%%%%%%%%%%%
\paragraph{v2.0:} 2018/12/30

\begin{itemize}
\item
immediate forward processing
\item
added |\childdocby| mechanism
\item
manual restructured
\end{itemize}

%%%%%%%%%%%%%%%%%%%%%%%%%%%%%%%%%%%%%%%%
\paragraph{v1.6:} 2018/01/17

\begin{itemize}
\item
application for development of include files
\item
corrections to manual
\end{itemize}

%%%%%%%%%%%%%%%%%%%%%%%%%%%%%%%%%%%%%%%%
\paragraph{v1.5:} 2017/05/21

\begin{itemize}
\item
more complete structuring introduced
\item
|\childdocof| introduced
\item
|\childdoc| renamed to |\childdocmain|
\item
|\childredirect| renamed to |\childdocforward| and |\childdocforwardprefix|
and functionality expanded
\end{itemize}

%%%%%%%%%%%%%%%%%%%%%%%%%%%%%%%%%%%%%%%%
\paragraph{v1.0:} 2017/04/27

\begin{itemize}
\item
manual and install package
\item
first version published on CTAN
\end{itemize}

%%%%%%%%%%%%%%%%%%%%%%%%%%%%%%%%%%%%%%%%
\paragraph{v0.6:} 2017/04/26

\begin{itemize}
\item
redirection mechanism added
\end{itemize}

%%%%%%%%%%%%%%%%%%%%%%%%%%%%%%%%%%%%%%%%
\paragraph{v0.5:} 2017/04/26

\begin{itemize}
\item
functionality in definition file
\end{itemize}


%%%%%%%%%%%%%%%%%%%%%%%%%%%%%%%%%%%%%%%%%%%%%%%%%%%%%%%%%%%%%%%%%%%%%%%%%%%%%%%%
%%%%%%%%%%%%%%%%%%%%%%%%%%%%%%%%%%%%%%%%%%%%%%%%%%%%%%%%%%%%%%%%%%%%%%%%%%%%%%%%
%%%%%%%%%%%%%%%%%%%%%%%%%%%%%%%%%%%%%%%%%%%%%%%%%%%%%%%%%%%%%%%%%%%%%%%%%%%%%%%%
\appendix

\settowidth\MacroIndent{\rmfamily\scriptsize 000\ }

 \DocInput{childdoc.dtx}

\end{document}
%</driver>
% \fi
%
% %%%%%%%%%%%%%%%%%%%%%%%%%%%%%%%%%%%%%%%%%%%%%%%%%%%%%%%%%%%%%%%%%%%%%%%%%%%%%%
% %%%%%%%%%%%%%%%%%%%%%%%%%%%%%%%%%%%%%%%%%%%%%%%%%%%%%%%%%%%%%%%%%%%%%%%%%%%%%%
% \section{Sample}
%\iffalse
%<*samplemain>
%\fi
%
% The following presents a sample document
% with two chapters, two parts, a title page,
% a compile flag as well as three forwarding files to set the flag.
% It consists of eight |.tex| files:
% \begin{center}
% \begin{tabular}{ll}
% |cdocsamp.tex|&main file\\
% |cdocsch1.tex|&include file for chapter 1\\
% |cdocsch2.tex|&include file for chapter 2\\
% |cdocspt3.tex|&include file for part 3\\
% |cdocspt4.tex|&include file for part 4\\
% |cdocsdrf.tex|&forwarding file for main file in draft mode\\
% |cdocsfi1.tex|&forwarding file for final version of chapter 1\\
% |cdocsfi2.tex|&forwarding file for final version of chapter 2\\
% \end{tabular}
% \end{center}
% Each of the eight files can be compiled directly by the \LaTeX{} compiler.
%
% %%%%%%%%%%%%%%%%%%%%%%%%%%%%%%%%%%%%%%
% \paragraph{Main File.}
%
% The main file is called |cdocsamp.tex|.
%
% Load the \textsf{childdoc} definitions and
% declare the filename for the main document:
%    \begin{macrocode}
\input{childdoc.def}
\childdocmain{}
%    \end{macrocode}

% Optional override for |\version| flag:
%    \begin{macrocode}
%%\ifchilddoc\else\providecommand{\version}{draft}\fi
%    \end{macrocode}

% Define the default values for the |\version| flag
% (|final| for the main file and |draft| for childs):
%    \begin{macrocode}
\ifchilddoc
\providecommand{\version}{draft}
\else
\providecommand{\version}{final}
\fi
%    \end{macrocode}

% Load the standard document class:
%    \begin{macrocode}
\documentclass[12pt]{article}
%    \end{macrocode}

% Start the document body:
%    \begin{macrocode}
\begin{document}
%    \end{macrocode}

% Declare a title page.
% Print title, part of document being processed and version flag:
%    \begin{macrocode}
\addtocounter{page}{-1}
\begin{center}
{\LARGE\bfseries{}childdoc example\par}
\vspace{1cm}
\ifchilddoc
\ifchilddocmanual part\else chapter\fi:
`\childdocname' of `\childdocjob'\par
\else
main document: `\childdocjob'\par
\fi
version: \version\par
\end{center}
\newpage
%    \end{macrocode}

% Manually include selected file,
% otherwise process as usual:
%    \begin{macrocode}
\ifchilddocmanual
\section*{part `\childdocname'}
\input{\childdocname}
\else
%    \end{macrocode}

% Include the two chapters:
%    \begin{macrocode}
\include{cdocsch1}
\include{cdocsch2}
%    \end{macrocode}

% Include the two parts unless only chapters should be displayed:
%    \begin{macrocode}
\ifchilddoc\else
\section{part three}
\input{cdocspt3}
\section{part four}
\input{cdocspt4}
\fi
%    \end{macrocode}

% Process as usual until here:
%    \begin{macrocode}
\fi
%    \end{macrocode}

% End of document body:
%    \begin{macrocode}
\end{document}
%    \end{macrocode}
%\iffalse
%</samplemain>
%\fi
%
% %%%%%%%%%%%%%%%%%%%%%%%%%%%%%%%%%%%%%%
% \paragraph{Chapter Include Files.}
%
% The include files are called |cdocsch1.tex| and |cdocsch2.tex|.
%
%\iffalse
%<*samplechap1|samplechap2>
%\fi

% Optional override for |\version| flag:
%    \begin{macrocode}
%%\providecommand{\version}{final}
%    \end{macrocode}

% Include the main document:
%    \begin{macrocode}
\input{childdoc.def}
\childdocof{cdocsamp}
%    \end{macrocode}

%\iffalse
%</samplechap1|samplechap2>
%\fi
%
%\iffalse
%<*samplechap1>
%\fi
% Some text for chapter 1:
%    \begin{macrocode}
\section{one}
some text in chapter one
%    \end{macrocode}

%\iffalse
%</samplechap1>
%\fi
% Some text for chapter 2:
%\iffalse
%<*samplechap2>
%\fi
%    \begin{macrocode}
\section{two}
more text in chapter two
%    \end{macrocode}

%\iffalse
%</samplechap2>
%\fi
%
% %%%%%%%%%%%%%%%%%%%%%%%%%%%%%%%%%%%%%%
% \paragraph{Part Include Files.}
%
% The include files are called |cdocspt3.tex| and |cdocspt4.tex|.
%
%\iffalse
%<*samplepart3|samplepart4>
%\fi

% Optional override for |\version| flag:
%    \begin{macrocode}
%%\providecommand{\version}{final}
%    \end{macrocode}

% Include the main document:
%    \begin{macrocode}
\input{childdoc.def}
\childdocby{cdocsamp}
%    \end{macrocode}

%\iffalse
%</samplepart3|samplepart4>
%\fi
%
%\iffalse
%<*samplepart3>
%\fi
% Some text for part 3:
%    \begin{macrocode}
some text in part three
%    \end{macrocode}

%\iffalse
%</samplepart3>
%\fi
% Some text for part 4:
%\iffalse
%<*samplepart4>
%\fi
%    \begin{macrocode}
more text in part four
%    \end{macrocode}

%\iffalse
%</samplepart4>
%\fi
%
% %%%%%%%%%%%%%%%%%%%%%%%%%%%%%%%%%%%%%%
% \paragraph{Forwarding for a Complete Draft.}
%
% The following forwarding file |cdocsdrf.tex|
% compiles the main document in draft mode:
%\iffalse
%<*sampledraft>
%\fi
%    \begin{macrocode}
\def\version{draft}
\input{childdoc.def}
\childdocforward{cdocsamp}
%    \end{macrocode}

%\iffalse
%</sampledraft>
%\fi
%
% %%%%%%%%%%%%%%%%%%%%%%%%%%%%%%%%%%%%%%
% \paragraph{Forwarding for Final Version of the Chapters.}
%
% The following forwarding files |cdocsfn1.tex| and |cdocsfn2.tex|
% (with identical content)
% compile the final versions of the child documents
% |cdocsch1.tex| and |cdocsch2.tex|, respectively:
%\iffalse
%<*samplefinal>
%\fi
%    \begin{macrocode}
\def\version{final}
\input{childdoc.def}
\childdocforwardprefix[cdocsamp]{cdocsfn}{cdocsch}
%    \end{macrocode}

%\iffalse
%</samplefinal>
%\fi
%
% %%%%%%%%%%%%%%%%%%%%%%%%%%%%%%%%%%%%%%
% \paragraph{Command Line Processing.}
%
% The following three command lines generate the output files
% |cdocscld|, |cdocscl1| and |cdocscl2|
% which should be identical to
% |cdocsdrf|, |cdocsch1| and |cdocsfn2|, respectively:
% \begin{center}
% \begin{tabular}{l}
% |latex -jobname cdocscld \|\\
% |  "\def\version{draft}\input{childdoc.def}\childdocforward{cdocsamp}"|\\
% |latex -jobname cdocscl1 \|\\
% |  "\input{childdoc.def}\childdocforward[cdocsamp]{cdocsch1}"|\\
% |latex -jobname cdocscl2 \|\\
% |  "\def\version{final}\input{childdoc.def}\childdocforward{cdocsch2}"|
% \end{tabular}
% \end{center}
% Note that the trailing backslash on each first line
% merely continues the input to the second line
% (for convenient cut ant paste).
% Furthermore, the command |latex| can be replaced by any
% of its alternative versions such as |pdflatex|.
%
% %%%%%%%%%%%%%%%%%%%%%%%%%%%%%%%%%%%%%%%%%%%%%%%%%%%%%%%%%%%%%%%%%%%%%%%%%%%%%%
% %%%%%%%%%%%%%%%%%%%%%%%%%%%%%%%%%%%%%%%%%%%%%%%%%%%%%%%%%%%%%%%%%%%%%%%%%%%%%%
% \section{Implementation}
%\iffalse
%<*package>
%\fi
%
% This section describes the definitions file |childdoc.def|.

% The definitions cannot be loaded using |\usepackage| or |\RequirePackage|
% which has a mechanism to prevent loading a style file more than once.
% When loading the definitions by means of |\input|
% multiple instances have to be prevented manually:
%\iffalse
%This code needs to be before the `\ProvidesFile' directive
%which is defined at the beginning of this file.
%Therefore it is also placed there and commented out here.
%</package>
%<*discard>
%\fi
%    \begin{macrocode}
\ifdefined\childdocmain\endinput\fi
%    \end{macrocode}
%\iffalse
%</discard>
%<*package>
%\fi
%
% \macro{\ifchilddoc}
% \macro{\ifchilddocmanual}
% The conditional |\ifchilddoc| tells whether a
% child (true) or main (false) document is being compiled.
% The conditional |\ifchilddocmanual| tells whether
% the |\includeonly| mechanism is used (false) or
% the selection of child files must be performed manually (true).
% The definitions initialise to false:
%    \begin{macrocode}
\newif\ifchilddoc
\newif\ifchilddocmanual
%    \end{macrocode}

% \macro{\childdocname}
% \macro{\childdocjob}
% The macro |\childdocname| stores the name of the main document
% to be compiled. The macro |\childdocjob| stores the name of
% the document on which the \LaTeX{} compiler was originally invoked.
% The content of |\jobname| cannot be compared
% to filenames specified in the source due to different catcodes.
% The following code rescans |\jobname|, stores the result
% in |\childdocname| and saves a copy in |\childdocjob|:
%    \begin{macrocode}
\edef\childdocname{\scantokens\expandafter{\jobname\noexpand}}
\let\childdocjob\childdocname
%    \end{macrocode}

% \macro{\childdocdisable}
% The macro |\childdocdisable| prevents the main file
% from being processed more than once.
% At this stage, the main document command |\childdocmain|
% is assumed to be called once again where it should do nothing.
% Any subsequent call to it should prevent
% a secondary processing of the main document
% It overwrites the forwarding commands
% |\childdocof| and |\childdocforward|
% with empty macros to prevent further inclusions of the main document:
%    \begin{macrocode}
\newcommand{\childdocdisable}
{
  \renewcommand{\childdocmain}[1]{\renewcommand{\childdocmain}[1]{\endinput}}
  \renewcommand{\childdocof}[1]{}
  \renewcommand{\childdocby}[2][]{}
  \renewcommand{\childdocforward}[2][]{}
  \renewcommand{\childdocdisable}{}
}
%    \end{macrocode}

% \macro{\childdocmain}
% The macro |\childdocmain| is to be called at the top of the main file
% with nothing or the main filename (without extension) as argument.
% First, it breaks loops.
% If the argument is not empty and does not match |\childdocname|
% (which is set by the first inclusion of |childdoc.def|),
% |\ifchilddoc| is set to true, |\includeonly| is applied to the child file
% and |\jobname| is set to the main file
% (for proper handling of |.aux| files):
%    \begin{macrocode}
\newcommand{\childdocmain}[1]
{
  \childdocdisable\childdocmain{}
  \if?#1?\else
    \begingroup
      \def\childdoctmp{#1}
      \ifx\childdoctmp\childdocname
        \def\childdoctmp{}
      \else
        \def\childdoctmp
        {
          \childdoctrue
          \includeonly{\childdocname}
          \def\childdocjob{#1}
          \def\jobname{#1}
        }
      \fi
      \expandafter
    \endgroup
    \childdoctmp
  \fi
}
%    \end{macrocode}

% \macro{\childdocof}
% The command |\childdocof| redirects
% compilation to the main file |#1|.
%    \begin{macrocode}
\newcommand{\childdocof}[1]
{
  \childdocdisable
  \childdoctrue
  \includeonly{\childdocname}
  \def\jobname{#1}
  \def\childdocjob{#1}
  \input{#1}
}
%    \end{macrocode}

% \macro{\childdocby}
% The command |\childdocby| ....
%    \begin{macrocode}
\newcommand{\childdocby}[2][]
{
  \childdocdisable
  \childdoctrue
  \childdocmanualtrue
  \if?#1?\else
    \def\jobname{#2}
  \fi
  \def\childdocjob{#2}
  \input{#2}
  \endinput
}
%    \end{macrocode}

% \macro{\childdocforward}
% The command |\childdocforward| redirects
% compilation to the main file or
% (if the optional argument is given) a child file.
% Parameters are set as if the main file
% or a child file starting with |\childdocof| was compiled.
% Then compilation is handed over to the main file:
%    \begin{macrocode}
\newcommand{\childdocforward}[2][]
{
  \begingroup
    \if?#1?
      \def\childdoctmp
      {
        \def\childdocname{#2}
        \def\childdocjob{#2}
        \def\jobname{#2}
        \input{#2}
        \endinput
      }
    \else
      \def\childdoctmp
      {
        \childdocdisable
        \def\childdocname{#2}
        \childdoctrue
        \includeonly{#2}
        \def\childdocjob{#1}
        \def\jobname{#1}
        \input{#1}
        \endinput
      }
    \fi
    \expandafter
  \endgroup
  \childdoctmp
}
%    \end{macrocode}

% \macro{\childdocforwardprefix}
% The command |\childdocforwardprefix| redirects
% compilation to the main or a child file by means of a pattern.
% The prefix |#1| in the current filename is replaced by |#2|
% and the suffix of the current filename is kept
% (it is assumed that the filename does not contain the substring `|~~~|'
% which is used as a delimiter).
% Compilation is handed over to the new file by |\childdocforward|:
%    \begin{macrocode}
\newcommand{\childdocforwardprefix}[3][]
{
  \begingroup
    \def\childdocextract #2##1~~~{\def\childdoctmp{\childdocforward[#1]{#3##1}}}
    \expandafter\childdocextract\childdocname~~~
    \expandafter
  \endgroup
  \childdoctmp
}
%    \end{macrocode}

% \macro{\childdoc}
% The deprecated macro |\childdoc| is a legacy version of |\childdocmain|:
%    \begin{macrocode}
\newcommand{\childdoc}{\childdocmain}
%    \end{macrocode}

% \macro{\childdocredirect}
% The deprecated macro |\childdocredirect| is a legacy version
% of |\childdocforward| and |\childdocforwardprefix|:
%    \begin{macrocode}
\newcommand{\childdocredirect}[2][]
{
  \begingroup
    \if?#1?
      \def\childdoctmp{\childdocforward{#2}}
    \else
      \def\childdoctmp{\childdocforwardprefix{#1}{#2}}
    \fi
    \expandafter
  \endgroup
  \childdoctmp
}
%    \end{macrocode}

%\iffalse
%</package>
%\fi
%
\endinput
|\\
|\childdocforward{|\textit{main}|}|
\end{tabular}
\end{center}
%
Likewise, the following files |final|\textit{nn}|.tex|
compile the final version of the child document
|child|\textit{nn}|.tex|:
%
\begin{center}
\begin{tabular}{l}
|\def\version{final}|\\
|% \iffalse
%
% childdoc.dtx Copyright (C) 2017-2018 Niklas Beisert
%
% This work may be distributed and/or modified under the
% conditions of the LaTeX Project Public License, either version 1.3
% of this license or (at your option) any later version.
% The latest version of this license is in
%   http://www.latex-project.org/lppl.txt
% and version 1.3 or later is part of all distributions of LaTeX
% version 2005/12/01 or later.
%
% This work has the LPPL maintenance status `maintained'.
%
% The Current Maintainer of this work is Niklas Beisert.
%
% This work consists of the files childdoc.dtx and childdoc.ins
% and the derived files childdoc.def and cdocsamp.tex with
% cdocsch1.tex, cdocsch2.tex, cdocsdrf.tex, cdocsfn1.tex, cdocsfn2.tex.
%
%<package>\ifdefined\childdocmain\endinput\fi
%<package>\ProvidesFile{childdoc.def}[2018/12/30 v2.0 child document driver]
%<samplemain>\ProvidesFile{cdocsamp.tex}[2018/12/30 v2.0 sample for childdoc]
%<*driver>
%\ProvidesFile{childdoc.drv}[2018/12/30 v2.0 childdoc reference manual file]
\PassOptionsToClass{10pt,a4paper}{article}
\documentclass{ltxdoc}

\usepackage[margin=35mm]{geometry}
\usepackage{hyperref}
\usepackage{hyperxmp}
\usepackage[usenames]{color}

\hypersetup{colorlinks=true}
\hypersetup{pdfstartview=FitH}
\hypersetup{pdfpagemode=UseNone}
\hypersetup{pdfsource={}}
\hypersetup{pdflang={en-UK}}
\hypersetup{pdfcopyright={Copyright 2017-2018 Niklas Beisert.
  This work may be distributed and/or modified under the
  conditions of the LaTeX Project Public License, either version 1.3
  of this license or (at your option) any later version.}}
\hypersetup{pdflicenseurl={http://www.latex-project.org/lppl.txt}}
\hypersetup{pdfcontactaddress={ETH Zurich, ITP, HIT K,
  Wolfgang-Pauli-Strasse 27}}
\hypersetup{pdfcontactpostcode={8093}}
\hypersetup{pdfcontactcity={Zurich}}
\hypersetup{pdfcontactcountry={Switzerland}}
\hypersetup{pdfcontactemail={nbeisert@itp.phys.ethz.ch}}
\hypersetup{pdfcontacturl={http://people.phys.ethz.ch/\xmptilde nbeisert/}}

\newcommand{\secref}[1]{\hyperref[#1]{section \ref*{#1}}}

\parskip1ex
\parindent0pt
\let\olditemize\itemize
\def\itemize{\olditemize\parskip0pt}

\begin{document}

\title{The \textsf{childdoc} Package}
\hypersetup{pdftitle={The childdoc Package}}
\author{Niklas Beisert\\[2ex]
  Institut f\"ur Theoretische Physik\\
  Eidgen\"ossische Technische Hochschule Z\"urich\\
  Wolfgang-Pauli-Strasse 27, 8093 Z\"urich, Switzerland\\[1ex]
  \href{mailto:nbeisert@itp.phys.ethz.ch}
  {\texttt{nbeisert@itp.phys.ethz.ch}}}
\hypersetup{pdfauthor={Niklas Beisert}}
\hypersetup{pdfsubject={Manual for the LaTeX2e Package childdoc}}
\date{30 December 2018, \textsf{v2.0}}
\maketitle

\begin{abstract}\noindent
\textsf{childdoc} is a \LaTeXe{} package
that enables the direct compilation
of document sections included by |\include|
to individual files.
\end{abstract}

\begingroup
\parskip0ex
\tableofcontents
\endgroup

%%%%%%%%%%%%%%%%%%%%%%%%%%%%%%%%%%%%%%%%%%%%%%%%%%%%%%%%%%%%%%%%%%%%%%%%%%%%%%%%
%%%%%%%%%%%%%%%%%%%%%%%%%%%%%%%%%%%%%%%%%%%%%%%%%%%%%%%%%%%%%%%%%%%%%%%%%%%%%%%%
\section{Introduction}

\LaTeX{} provides a mechanism to structure a large document (such as a book)
into a main file and several child files (containing the chapters)
using the |\include| command.
This mechanism is beneficial for documents
which span hundreds of pages in order to
make the source file(s) more manageable.
Moreover, compilation can be restricted to
selected child files by means of the |\includeonly| command.
The latter feature can be used to reduce the compilation time while editing
(this was significantly more useful in the earlier days of \LaTeX{})
or to generate a smaller document which is easier to navigate.
Another application of |\includeonly| is to generate
documents consisting of selected parts of the complete document.

However, there are a few drawbacks of the plain |\include| mechanism:
\begin{itemize}
\item
The child files cannot be compiled on their own,
they can only be compiled via the main file.
A naive editing environment
(such as a text editor with an option
to have the current file processed by \LaTeX)
may require one to switch to the main file before compiling;
attempting to compile the child file produces errors.
\item
The main file must be modified (each time)
to adjust the |\includeonly| command
to the present needs. This easily leaves the main file in a messy state.
\item
The generated document will always carry the filename
of the main document. This is inconvenient if
several child files are to be compiled and
to be kept for distribution.
\end{itemize}

The present package provides a simple interface
to make child files individually compilable by \LaTeX{}.
Compiling a child file then has the same effect as compiling
the main file with an |\includeonly| command
to select the appropriate child.
Moreover the generated document will carry the name of the child
rather than the main file.
This resolves all three above issues.

This feature is meant to make the editing of books,
thesis documents and lecture notes somewhat more convenient.
However, the package can also be used efficiently for
composing a series of documents (such as exercise sheets)
which are typically distributed individually.
It then assists the author in generating the individual documents
(potentially in different versions)
as well as a document containing the collected series.
Another application is in developing style files
or other kinds of included material
where compilation of the style file could redirect
to a sample or test file.

%%%%%%%%%%%%%%%%%%%%%%%%%%%%%%%%%%%%%%%%%%%%%%%%%%%%%%%%%%%%%%%%%%%%%%%%%%%%%%%%
%%%%%%%%%%%%%%%%%%%%%%%%%%%%%%%%%%%%%%%%%%%%%%%%%%%%%%%%%%%%%%%%%%%%%%%%%%%%%%%%
\section{Usage}

First of all, the package \textsf{childdoc} is \emph{not} a standard
\LaTeXe{} |.sty| style file! Therefore it needs to be invoked in
a non-standard way.

%%%%%%%%%%%%%%%%%%%%%%%%%%%%%%%%%%%%%%%%%%%%%%%%%%%%%%%%%%%%%%%%%%%%%%%%%%%%%%%%
\subsection{Included Files}
\label{sec:include}

%%%%%%%%%%%%%%%%%%%%%%%%%%%%%%%%%%%%%%%%
\DescribeMacro{\childdocmain}
To use the package, add the commands
\begin{center}
\begin{tabular}{l}
|\input{childdoc.def}|\\
|\childdocmain{}|\\
\end{tabular}
\end{center}
at the very top of the main \LaTeX{} file,
in particular \emph{before} the |\documentclass| statement!
The argument of |\childdocmain| should be left empty
(but it must be present).

%%%%%%%%%%%%%%%%%%%%%%%%%%%%%%%%%%%%%%%%
\DescribeMacro{\childdocof}
Furthermore, add the commands
\begin{center}
\begin{tabular}{l}
|\input{childdoc.def}|\\
|\childdocof{|\textit{main}|}|\\
\end{tabular}
\end{center}
at the top of every child file \textit{child}
which is included by |\include{|\textit{child}|}|
from within the main file
(or at least for those files to be compiled individually).
The argument \textit{main} must be the filename of the main file.

There are a couple of
considerations in setting up the main and child documents:

%%%%%%%%%%%%%%%%%%%%%%%%%%%%%%%%%%%%%%%%
\paragraph{Restrictions.}

Please note the following restrictions:
\begin{itemize}
\item
|\childdocmain| must be called with one argument \textit{main}
to ensure compatibility with earlier version of the package.
It must either be empty (|\childdocmain{}|)
or precisely match the filename of the main file in which it is specified.
See \secref{sec:detection} for further information.
\item
The filename \textit{main} must be specified without the |.tex| extension.
\item
The filename \textit{main} is case sensitive
(even in case-insensitive file systems)
due to internal string comparison.
\item
The argument \textit{main} should be fully expanded, it cannot be a macro.
\item
Subdirectories and special characters should be avoided in filenames.
\item
The command |\childdocmain{|\textit{main}|}| must be followed by a whitespace.
It should not be followed immediately by another command
or by a comment mark `|%|'.
This is because the \TeX{} parser reads the token immediately following
the argument of |\childdocmain| and puts it
at the beginning of every child section;
however, a white\-space is ignored.
\end{itemize}

%%%%%%%%%%%%%%%%%%%%%%%%%%%%%%%%%%%%%%%%
\paragraph{Content of Main File.}

It is advisable to place all content in the child files included by |\include|.
Any output contained in the main file will appear in all child documents
unless suppressed manually;
it cannot be suppressed automatically by the |\includeonly| directive
and thus should normally be avoided.
A method to include some content in the main file
by means of conditional processing is described in \secref{sec:conditional}.

%%%%%%%%%%%%%%%%%%%%%%%%%%%%%%%%%%%%%%%%
\paragraph{Page Numbering.}

When only a part of the document is compiled,
the appropriate numbering of pages
(as well as other status parameters)
is determined from the |.aux| files.
The latter contain information from previous passes.
However this information needs to propagate through
all intermediate child documents.
Therefore the page numbering in child documents may well
be inconsistent until the complete document is compiled at least once.

A useful (if unconventional) way to always ensure a consistent
page numbering is to restart the numbering in each child document
and denote the pages by `\textit{child}|.|\textit{page}'
where \textit{child} represents the chapter/section number of the child file.
This can be achieved by the command
|\numberwithin{page}{|\textit{child}|}|
of the \textsf{amsmath} package
where \textit{child} can be |chapter| or |section|
depending on the chosen structuring.
Alternatively, one can modify the macro |\thepage| appropriately
and reset the counter |page| at the start of each child file.

%%%%%%%%%%%%%%%%%%%%%%%%%%%%%%%%%%%%%%%%%%%%%%%%%%%%%%%%%%%%%%%%%%%%%%%%%%%%%%%%
\subsection{Conditional Processing}
\label{sec:conditional}

The package provides a mechanism to compile different versions
of a document. To customise the versions further some conditional processing
can come in handy to distinguish which version is being compiled.
The package provides two macros to describe the compilation context:

%%%%%%%%%%%%%%%%%%%%%%%%%%%%%%%%%%%%%%%%
\DescribeMacro{\ifchilddoc}
The conditional |\ifchilddoc| distinguishes between the compilation of
child documents and the main document:
%
\begin{center}
|\ifchilddoc |\textit{child-code}| |[|\||else |\textit{main-code}]| \||fi|
\end{center}

%%%%%%%%%%%%%%%%%%%%%%%%%%%%%%%%%%%%%%%%
\DescribeMacro{\childdocname}
\DescribeMacro{\childdocjob}
The macro |\childdocname| contains the filename (without extension)
of the main or child file being processed.
Note that |\childdocjob| will always contain the name of the main file.

%%%%%%%%%%%%%%%%%%%%%%%%%%%%%%%%%%%%%%%%
\paragraph{Title Page.}

Conditional processing can be used to include a title or banner page
in the main document when proper precautions are taken.
Importantly, the code in the main file should ensure that the page counter
(as well as other status parameters which are stored in the |.aux| files)
takes the same value after the conditional processing.
Otherwise the page numbers may take divergent values
depending on which part is compiled.

For example, a title page could be declared by:
%
\begin{center}
\begin{tabular}{l}
|\ifchilddoc\||else|\\
|\addtocounter{page}{-1}|\\
\textit{code for title page}\\
|\newpage|\\
|\||fi|
\end{tabular}
\end{center}
%
A banner page for the child documents can be generated by:
%
\begin{center}
\begin{tabular}{l}
|\ifchilddoc|\\
|\addtocounter{page}{-1}|\\
\textit{code for banner page}\\
|\newpage|\\
|\||fi|
\end{tabular}
\end{center}
%
Here one could write a message such as:
\begin{center}
|This is the part \childdocname{} of \childdocjob{}.|
\end{center}

%%%%%%%%%%%%%%%%%%%%%%%%%%%%%%%%%%%%%%%%%%%%%%%%%%%%%%%%%%%%%%%%%%%%%%%%%%%%%%%%
\subsection{Flags}
\label{sec:flags}

The package makes it easy to generate different versions
of the main or child documents.
To this end compilation flags can be defined
and assigned different default values.
They will be particularly useful in conjunction
with the forwarding mechanism described in \secref{sec:forward}.

For example, it may be useful to have a flag |\version|
which can be set to |draft| or |final|.
The document source will contain some conditional code
depending on the value of |\version|.
Suppose further, the flag should default to |final| for the main file
and to |draft| for child files
which is a natural assignment for editing the document.
This is achieved by placing the following code
in the preamble of the main document
(below the |\childdocmain| directive):
%
\begin{center}
\begin{tabular}{l}
|\ifchilddoc|\\
|\providecommand{\version}{draft}|\\
|\||else|\\
|\providecommand{\version}{final}|\\
|\||fi|
\end{tabular}
\end{center}
%
The definition by |\providecommand| makes sure
that previous definitions are not overwritten.
Further statements |\providecommand{\version}{...}|
can thus be added before the above code to override it.

For the main file, one might add a line
(between |\childdocmain| and the above block)
%
\begin{center}
|%\ifchilddoc\||else\providecommand{\version}{draft}\||fi|
\end{center}
%
which can be uncommented to produce a draft version.
Likewise one can add a line to the very top of a child file
(above the |\childdocof{|\textit{main}|}| directive)
%
\begin{center}
|%\providecommand{\version}{final}|
\end{center}
%
which can be uncommented to produce the final version of this child document.

%%%%%%%%%%%%%%%%%%%%%%%%%%%%%%%%%%%%%%%%%%%%%%%%%%%%%%%%%%%%%%%%%%%%%%%%%%%%%%%%
\subsection{Forwarding}
\label{sec:forward}

Different versions of the main or child documents
using compilation flags as described in \secref{sec:flags}
can be (permanently) stored in different files
for convenient compilation, viewing and distribution.
To this end, the package defines a command
to pass on compilation to a different file:

%%%%%%%%%%%%%%%%%%%%%%%%%%%%%%%%%%%%%%%%
\DescribeMacro{\childdocforward}
The command |\childdocforward| redirects processing to
another source file:
%
\begin{center}
\begin{tabular}{l}
|\input{childdoc.def}|\\
|\childdocforward[|\textit{main}|]{|\textit{dest}|}|\\
\end{tabular}
\end{center}
%
The argument \textit{dest} is the destination file
(without extension).
It should be the main file or one of the child files.
Note that further \textsf{childdoc} directives
such as |\childdocof| and |\childdocforward|
in the indicated file will be processed in this form.
The optional argument \textit{main}
passes on directly to the main file \textit{main}
while pretending to compile the child \textit{dest}.
This form behaves as if \textit{dest}
issues |\childdocof{|\textit{main}|}| right away,
and no further \textsf{childdoc} directives will be processed.

%%%%%%%%%%%%%%%%%%%%%%%%%%%%%%%%%%%%%%%%
\DescribeMacro{\...prefix}
In the alternative form |\childdocforwardprefix|,
%
\begin{center}
\begin{tabular}{l}
|\input{childdoc.def}|\\
|\childdocforwardprefix[|\textit{main}|]{|\textit{prefix}|}{|\textit{dest}|}|
\end{tabular}
\end{center}
%
the destination file is determined by a pattern
depending on the current file:
To make this work, the current file must be called
`{\textit{prefix}\hspace{0.2em}\textit{suffix}}'
with \textit{prefix} matching precisely the argument.
Processing is then passed on to the file
`{\textit{dest}\hspace{0.2em}\textit{suffix}}'.
Surely, the same effect is achieved by
directly specifying the
argument `{\textit{dest}\hspace{0.2em}\textit{suffix}}'
in the first form.
However, that requires to set up a different file
for each child. With the alternative form of the command
all these files can have exactly the same content
which simplifies setting them up and maintaining them.

For example, the following file |draft.tex|
with a compilation flag |\version| as described in \secref{sec:flags}
compiles the main document as a draft:
%
\begin{center}
\begin{tabular}{l}
|\def\version{draft}|\\
|\input{childdoc.def}|\\
|\childdocforward{|\textit{main}|}|
\end{tabular}
\end{center}
%
Likewise, the following files |final|\textit{nn}|.tex|
compile the final version of the child document
|child|\textit{nn}|.tex|:
%
\begin{center}
\begin{tabular}{l}
|\def\version{final}|\\
|\input{childdoc.def}|\\
|\childdocforwardprefix{final}{child}|
\end{tabular}
\end{center}
%

Note that when several versions of a main file and/or of each child file
are to be generated, it may be convenient to set up a |Makefile| or
shell script to automatise the process.

%%%%%%%%%%%%%%%%%%%%%%%%%%%%%%%%%%%%%%%%%%%%%%%%%%%%%%%%%%%%%%%%%%%%%%%%%%%%%%%%
\subsection{Command Line Processing}
\label{sec:commandline}

The effect of redirection files can also be achieved by invoking
the \LaTeX{} compiler with a more elaborate command line.
Most conveniently this should be done as part
of a shell script or a |Makefile|.

When using \textsf{childdoc} in the main file, the following
command lines effectively perform a redirection
(note that depending on the shell being used,
backslashes may have to be doubled: `|\|' $\to$ `|\\|'):
%
\begin{center}
|... -jobname "|\textit{target}|" |\\|"|[\textit{flags}]%
|\input{childdoc.def}\childdocforward[|\textit{main}|]{|\textit{dest}|}"|
\end{center}
%
Here \textit{target} is the name of the output file,
\textit{main} is the name of the main file
and \textit{dest} is the name of the main or child file to be processed
(all filenames without extensions).
The optional argument \textit{main} can be omitted
if \textit{main} matches \textit{dest}.
Optionally, compilation \textit{flags} can be defined via |\def| commands.
This command line makes the \TeX{} engine believe
it is compiling the file \textit{target}
whose content is specified as the latter parameter.
The provided code then forwards the processing to
\textit{main} or \textit{dest} as described in \secref{sec:forward}.

%%%%%%%%%%%%%%%%%%%%%%%%%%%%%%%%%%%%%%%%%%%%%%%%%%%%%%%%%%%%%%%%%%%%%%%%%%%%%%%%
\subsection{Include by Input}
\label{sec:input}

Including child documents by |\include| has some restrictions by design.
Most notably, the content of a child document always occupies
its own set of pages; pages cannot be shared between child documents.
Usually, this behaviour makes perfect sense
because each child document contain an essential part of the document.
However, in some situations it may be desirable to compose
a document from a collection of parts
without having mandatory page breaks between then.
For this case, the package
provides a mechanism to include parts
by |\input| which can also be processed individually.
However, by construction this mechanism
requires manual handling of the content to be output.

%%%%%%%%%%%%%%%%%%%%%%%%%%%%%%%%%%%%%%%%
\DescribeMacro{\ifchilddocmanual}
The main file should be prepared as usual, see \secref{sec:include}.
However, the document body must make a distinction
between processing of an individual part and of the main document, e.g.:
%
\begin{center}
\begin{tabular}{l}
|\ifchilddocmanual|\\
|\input{\childdocname}|\\
|\||else|\\
\textit{document body with }|\input{|\textit{part}|}|\\
|\||fi|
\end{tabular}
\end{center}
%
The conditional |\ifchilddocmanual| is true whenever
a part to be included by |\input| is being compiled,
and the name of the part is stored in |\childdocname|.

%%%%%%%%%%%%%%%%%%%%%%%%%%%%%%%%%%%%%%%%
\DescribeMacro{\childdocby}
Each part to be included by |\input| should start with:
%
\begin{center}
\begin{tabular}{l}
|\input{childdoc.def}|\\
|\childdocby{|\textit{main}|}|\\
\end{tabular}
\end{center}
%
The directive |\childdocby| is similar to |\childdocof|
described in \secref{sec:include},
but the subsequent selection of content must be done manually.
To that end, both |\ifchilddoc| and |\ifchilddocmanual|
will be true upon processing of a part,
and the name of the part is stored in |\childdocname|.
Note that |\jobname| will be set to the filename of the current part
so that each part receives an individual |.aux| file
that does not interfere with the |.aux| file(s) of the main document.
This behaviour can be altered by the alternative form
|\childdocby[*]{|\textit{main}|}| (with a non-empty optional argument)
which uses the |.aux| file of the main document
by setting |\jobname| to \textit{main}.

%%%%%%%%%%%%%%%%%%%%%%%%%%%%%%%%%%%%%%%%%%%%%%%%%%%%%%%%%%%%%%%%%%%%%%%%%%%%%%%%
\subsection{Driver Development}
\label{sec:driver}

The \textsf{childdoc} mechanism can also be use for the development
of definition files such as \LaTeX{} styles or classes.
This case differs from the above setup with multiple parts
included by |\include| in that no |\includeonly| should be invoked.
This can be achieved by starting the include file
(before |\ProvidesPackage|) with:
%
\begin{center}
\begin{tabular}{l}
|\input{childdoc.def}|\\
|\childdocforward{|\textit{main}|}|\\
\end{tabular}
\end{center}
%
or alternatively with:
%
\begin{center}
\begin{tabular}{l}
|\input{childdoc.def}|\\
|\childdocby{|\textit{main}|}|\\
\end{tabular}
\end{center}
%
Both forms have slightly different effects as described above.
The main file is prepared as usual, see \secref{sec:include}.

%%%%%%%%%%%%%%%%%%%%%%%%%%%%%%%%%%%%%%%%%%%%%%%%%%%%%%%%%%%%%%%%%%%%%%%%%%%%%%%%
\subsection{Legacy Detection}
\label{sec:detection}

The directive |\childdocmain| in the main file can detect
whether the complete document or merely a child is to be compiled
even without using the directive |\childdocof|.
This method is deprecated because it is less robust
and there is no compelling reason to use it;
it is merely provided for backward compatibility
and it may be removed in future versions.

If the detection mechanism is to be used,
it is mandatory to correctly specify
the filename of the main file as the argument of |\childdocmain|:
%
\begin{center}
\begin{tabular}{l}
|\input{childdoc.def}|\\
|\childdocmain{|\textit{main}|}|\\
\end{tabular}
\end{center}
%
If |\jobname| does not match the argument \textit{main} of |\childdocmain|,
it is assumed that |\jobname| points to the child file to be compiled.
When using |\childdocmain| with the main file specified as argument,
it suffices to start a child file
with just |\input{|\textit{main}|}|
without loading of the package and using |\childdocof|.
If instead all processing is done
with the appropriate \textsf{childdoc} directives,
the argument of \textit{main} of |\childdocmain| can be empty.

An alternative version of the command line processing described
in \secref{sec:commandline} using the detection mechanism reads:
%
\begin{center}
|... -jobname "|\textit{target}|" "|[\textit{flags}]%
[|\def\jobname{|\textit{dest}|}|]|\input{|\textit{main}|}"|
\end{center}

%%%%%%%%%%%%%%%%%%%%%%%%%%%%%%%%%%%%%%%%%%%%%%%%%%%%%%%%%%%%%%%%%%%%%%%%%%%%%%%%
\subsection{Manual Code}
\label{sec:manual}

In case one cannot be certain whether the definitions file |childdoc.def|
is installed on the target \TeX{} distribution
and one prefers not to ship it,
it is conceivable to paste a few relevant commands into the sources.

To that end, drop all statements |\input{childdoc.def}|
and perform the replacements as outlined below.
Instead of |\childdocmain{|\textit{main}|}| add the following code
to the top of the main file:
%
\begin{center}
\begin{tabular}{l}
|\||ifdefined\childdocname\endinput\||fi\newif\ifchilddoc|\\
|\edef\childdocname{\scantokens\expandafter{\jobname\noexpand}}|\\
|\def\childdocmain{|\textit{main}|}\||ifx\childdocmain\childdocname\||else|\\
|\childdoctrue\includeonly{\childdocname}\let\jobname\childdocmain\||fi|\\
\end{tabular}
\end{center}
%
Instead of |\childdocof{|\textit{main}|}| just include the main file
at the top of each child file:
%
\begin{center}
|\input{|\textit{main}|}|
\end{center}
%
A simple redirection |\childdocforward{|\textit{dest}|}| is achieved by:
%
\begin{center}
|\def\jobname{|\textit{dest}|}\input{\jobname}|
\end{center}
%
The redirection with prefix
|\childdocforwardprefix[|\textit{prefix}|]{|\textit{dest}|}|
is accomplished by:
%
\begin{center}
\begin{tabular}{l}
|{\edef\jobname{\scantokens\expandafter{\jobname\noexpand}}|\\
|\def\redirectjob |\textit{prefix}|#1~~~{\gdef\jobname{|\textit{dest}|#1}}|\\
|\expandafter\redirectjob\jobname~~~}\input{\jobname}|
\end{tabular}
\end{center}

In an alternative approach,
child documents can be compiled by a specific command line
without additional code or specific definitions:
%
\begin{center}
|... -jobname "|\textit{target}|" "|[\textit{flags}]%
|\includeonly{|\textit{dest}|}\input{|\textit{main}|}"|
\end{center}
%

%%%%%%%%%%%%%%%%%%%%%%%%%%%%%%%%%%%%%%%%%%%%%%%%%%%%%%%%%%%%%%%%%%%%%%%%%%%%%%%%
%%%%%%%%%%%%%%%%%%%%%%%%%%%%%%%%%%%%%%%%%%%%%%%%%%%%%%%%%%%%%%%%%%%%%%%%%%%%%%%%
\section{Information}

%%%%%%%%%%%%%%%%%%%%%%%%%%%%%%%%%%%%%%%%%%%%%%%%%%%%%%%%%%%%%%%%%%%%%%%%%%%%%%%%
\subsection{Copyright}

Copyright \copyright{} 2017--2018 Niklas Beisert

This work may be distributed and/or modified under the
conditions of the \LaTeX{} Project Public License, either version 1.3
of this license or (at your option) any later version.
The latest version of this license is in
  \url{http://www.latex-project.org/lppl.txt}
and version 1.3 or later is part of all distributions of \LaTeX{}
version 2005/12/01 or later.

This work has the LPPL maintenance status `maintained'.

The Current Maintainer of this work is Niklas Beisert.

This work consists of the files |README.txt|, |childdoc.ins| and |childdoc.dtx|
as well as the derived files |childdoc.def|, |cdocsamp.tex|
with |cdocsch1.tex|, |cdocsch2.tex|, |cdocspt3.tex|, |cdocspt4.tex|,
|cdocsdrf.tex|, |cdocsfn1.tex|, |cdocsfn2.tex|
as well as |childdoc.pdf|.

%%%%%%%%%%%%%%%%%%%%%%%%%%%%%%%%%%%%%%%%%%%%%%%%%%%%%%%%%%%%%%%%%%%%%%%%%%%%%%%%
\subsection{Files and Installation}

The package consists of the files:
%
\begin{center}
\begin{tabular}{ll}
    |README.txt|   & readme file \\
    |childdoc.ins| & installation file \\
    |childdoc.dtx| & source file \\
    |childdoc.def| & definition file \\
    |cdocsamp.tex| & sample main file \\
    |cdocsch1.tex| & sample include file \\
    |cdocsch2.tex| & sample include file \\
    |cdocspt3.tex| & sample part file \\
    |cdocspt4.tex| & sample part file \\
    |cdocsdrf.tex| & sample redirection file \\
    |cdocsfn1.tex| & sample redirection file \\
    |cdocsfn2.tex| & sample redirection file \\
    |childdoc.pdf| & manual
\end{tabular}
\end{center}
%
The distribution consists of the files
|README.txt|, |childdoc.ins| and |childdoc.dtx|.
%
\begin{itemize}
\item
Run (pdf)\LaTeX{} on |childdoc.dtx|
to compile the manual |childdoc.pdf| (this file).
\item
Run \LaTeX{} on |childdoc.ins| to create the definitions file |childdoc.def|
and the sample |cdocsamp.tex| with include files
|cdocsch1.tex|, |cdocsch2.tex|, |cdocspt3.tex|, |cdocspt4.tex|,
|cdocsdrf.tex|, |cdocsfn1.tex|, |cdocsfn2.tex|.
Then copy the file |childdoc.def| to an appropriate directory of your \LaTeX{}
distribution, e.g.\ \textit{texmf-root}|/tex/latex/childdoc|.
\end{itemize}

%%%%%%%%%%%%%%%%%%%%%%%%%%%%%%%%%%%%%%%%%%%%%%%%%%%%%%%%%%%%%%%%%%%%%%%%%%%%%%%%
\subsection{Related CTAN Packages}

There are several other packages which offer a similar functionality:
%
\begin{itemize}
\item
The packages
\href{http://ctan.org/pkg/docmute}{\textsf{docmute}},
\href{http://ctan.org/pkg/includex}{\textsf{includex}} and
\href{http://ctan.org/pkg/standalone}{\textsf{standalone}}
provide commands to include only the document body of
a child file thus allowing both files to be compiled individually.
\item
The packages \href{http://ctan.org/pkg/subdocs}{\textsf{subdocs}}
and \href{http://ctan.org/pkg/subfiles}{\textsf{subfiles}}
provide structures in which the main and child documents can be
encapsulated and allowing them to be compiled individually.
The inclusion mechanism is different from the conventional |\include|.
\item
The package \href{http://ctan.org/pkg/combine}{\textsf{combine}}
is an elaborate solution to combine several documents into one.
\end{itemize}
%
See also the CTAN topic \href{http://ctan.org/topic/subdocs}{\textsf{subdocs}}
for further related packages.
The present package differs from the above solutions in that
a document structure constructed with the conventional |\include| mechanism
just needs two extra commands at the top of every file
such that all constituent files can be compiled individually.

%%%%%%%%%%%%%%%%%%%%%%%%%%%%%%%%%%%%%%%%%%%%%%%%%%%%%%%%%%%%%%%%%%%%%%%%%%%%%%%%
%\subsection{Feature Suggestions}
%
%The following is a list of features which may be useful for future
%versions of this package:
%%
%\begin{itemize}
%\item
%\ldots
%\end{itemize}

%%%%%%%%%%%%%%%%%%%%%%%%%%%%%%%%%%%%%%%%%%%%%%%%%%%%%%%%%%%%%%%%%%%%%%%%%%%%%%%%
\subsection{Revision History}

%%%%%%%%%%%%%%%%%%%%%%%%%%%%%%%%%%%%%%%%
\paragraph{v2.0:} 2018/12/30

\begin{itemize}
\item
immediate forward processing
\item
added |\childdocby| mechanism
\item
manual restructured
\end{itemize}

%%%%%%%%%%%%%%%%%%%%%%%%%%%%%%%%%%%%%%%%
\paragraph{v1.6:} 2018/01/17

\begin{itemize}
\item
application for development of include files
\item
corrections to manual
\end{itemize}

%%%%%%%%%%%%%%%%%%%%%%%%%%%%%%%%%%%%%%%%
\paragraph{v1.5:} 2017/05/21

\begin{itemize}
\item
more complete structuring introduced
\item
|\childdocof| introduced
\item
|\childdoc| renamed to |\childdocmain|
\item
|\childredirect| renamed to |\childdocforward| and |\childdocforwardprefix|
and functionality expanded
\end{itemize}

%%%%%%%%%%%%%%%%%%%%%%%%%%%%%%%%%%%%%%%%
\paragraph{v1.0:} 2017/04/27

\begin{itemize}
\item
manual and install package
\item
first version published on CTAN
\end{itemize}

%%%%%%%%%%%%%%%%%%%%%%%%%%%%%%%%%%%%%%%%
\paragraph{v0.6:} 2017/04/26

\begin{itemize}
\item
redirection mechanism added
\end{itemize}

%%%%%%%%%%%%%%%%%%%%%%%%%%%%%%%%%%%%%%%%
\paragraph{v0.5:} 2017/04/26

\begin{itemize}
\item
functionality in definition file
\end{itemize}


%%%%%%%%%%%%%%%%%%%%%%%%%%%%%%%%%%%%%%%%%%%%%%%%%%%%%%%%%%%%%%%%%%%%%%%%%%%%%%%%
%%%%%%%%%%%%%%%%%%%%%%%%%%%%%%%%%%%%%%%%%%%%%%%%%%%%%%%%%%%%%%%%%%%%%%%%%%%%%%%%
%%%%%%%%%%%%%%%%%%%%%%%%%%%%%%%%%%%%%%%%%%%%%%%%%%%%%%%%%%%%%%%%%%%%%%%%%%%%%%%%
\appendix

\settowidth\MacroIndent{\rmfamily\scriptsize 000\ }

 \DocInput{childdoc.dtx}

\end{document}
%</driver>
% \fi
%
% %%%%%%%%%%%%%%%%%%%%%%%%%%%%%%%%%%%%%%%%%%%%%%%%%%%%%%%%%%%%%%%%%%%%%%%%%%%%%%
% %%%%%%%%%%%%%%%%%%%%%%%%%%%%%%%%%%%%%%%%%%%%%%%%%%%%%%%%%%%%%%%%%%%%%%%%%%%%%%
% \section{Sample}
%\iffalse
%<*samplemain>
%\fi
%
% The following presents a sample document
% with two chapters, two parts, a title page,
% a compile flag as well as three forwarding files to set the flag.
% It consists of eight |.tex| files:
% \begin{center}
% \begin{tabular}{ll}
% |cdocsamp.tex|&main file\\
% |cdocsch1.tex|&include file for chapter 1\\
% |cdocsch2.tex|&include file for chapter 2\\
% |cdocspt3.tex|&include file for part 3\\
% |cdocspt4.tex|&include file for part 4\\
% |cdocsdrf.tex|&forwarding file for main file in draft mode\\
% |cdocsfi1.tex|&forwarding file for final version of chapter 1\\
% |cdocsfi2.tex|&forwarding file for final version of chapter 2\\
% \end{tabular}
% \end{center}
% Each of the eight files can be compiled directly by the \LaTeX{} compiler.
%
% %%%%%%%%%%%%%%%%%%%%%%%%%%%%%%%%%%%%%%
% \paragraph{Main File.}
%
% The main file is called |cdocsamp.tex|.
%
% Load the \textsf{childdoc} definitions and
% declare the filename for the main document:
%    \begin{macrocode}
\input{childdoc.def}
\childdocmain{}
%    \end{macrocode}

% Optional override for |\version| flag:
%    \begin{macrocode}
%%\ifchilddoc\else\providecommand{\version}{draft}\fi
%    \end{macrocode}

% Define the default values for the |\version| flag
% (|final| for the main file and |draft| for childs):
%    \begin{macrocode}
\ifchilddoc
\providecommand{\version}{draft}
\else
\providecommand{\version}{final}
\fi
%    \end{macrocode}

% Load the standard document class:
%    \begin{macrocode}
\documentclass[12pt]{article}
%    \end{macrocode}

% Start the document body:
%    \begin{macrocode}
\begin{document}
%    \end{macrocode}

% Declare a title page.
% Print title, part of document being processed and version flag:
%    \begin{macrocode}
\addtocounter{page}{-1}
\begin{center}
{\LARGE\bfseries{}childdoc example\par}
\vspace{1cm}
\ifchilddoc
\ifchilddocmanual part\else chapter\fi:
`\childdocname' of `\childdocjob'\par
\else
main document: `\childdocjob'\par
\fi
version: \version\par
\end{center}
\newpage
%    \end{macrocode}

% Manually include selected file,
% otherwise process as usual:
%    \begin{macrocode}
\ifchilddocmanual
\section*{part `\childdocname'}
\input{\childdocname}
\else
%    \end{macrocode}

% Include the two chapters:
%    \begin{macrocode}
\include{cdocsch1}
\include{cdocsch2}
%    \end{macrocode}

% Include the two parts unless only chapters should be displayed:
%    \begin{macrocode}
\ifchilddoc\else
\section{part three}
\input{cdocspt3}
\section{part four}
\input{cdocspt4}
\fi
%    \end{macrocode}

% Process as usual until here:
%    \begin{macrocode}
\fi
%    \end{macrocode}

% End of document body:
%    \begin{macrocode}
\end{document}
%    \end{macrocode}
%\iffalse
%</samplemain>
%\fi
%
% %%%%%%%%%%%%%%%%%%%%%%%%%%%%%%%%%%%%%%
% \paragraph{Chapter Include Files.}
%
% The include files are called |cdocsch1.tex| and |cdocsch2.tex|.
%
%\iffalse
%<*samplechap1|samplechap2>
%\fi

% Optional override for |\version| flag:
%    \begin{macrocode}
%%\providecommand{\version}{final}
%    \end{macrocode}

% Include the main document:
%    \begin{macrocode}
\input{childdoc.def}
\childdocof{cdocsamp}
%    \end{macrocode}

%\iffalse
%</samplechap1|samplechap2>
%\fi
%
%\iffalse
%<*samplechap1>
%\fi
% Some text for chapter 1:
%    \begin{macrocode}
\section{one}
some text in chapter one
%    \end{macrocode}

%\iffalse
%</samplechap1>
%\fi
% Some text for chapter 2:
%\iffalse
%<*samplechap2>
%\fi
%    \begin{macrocode}
\section{two}
more text in chapter two
%    \end{macrocode}

%\iffalse
%</samplechap2>
%\fi
%
% %%%%%%%%%%%%%%%%%%%%%%%%%%%%%%%%%%%%%%
% \paragraph{Part Include Files.}
%
% The include files are called |cdocspt3.tex| and |cdocspt4.tex|.
%
%\iffalse
%<*samplepart3|samplepart4>
%\fi

% Optional override for |\version| flag:
%    \begin{macrocode}
%%\providecommand{\version}{final}
%    \end{macrocode}

% Include the main document:
%    \begin{macrocode}
\input{childdoc.def}
\childdocby{cdocsamp}
%    \end{macrocode}

%\iffalse
%</samplepart3|samplepart4>
%\fi
%
%\iffalse
%<*samplepart3>
%\fi
% Some text for part 3:
%    \begin{macrocode}
some text in part three
%    \end{macrocode}

%\iffalse
%</samplepart3>
%\fi
% Some text for part 4:
%\iffalse
%<*samplepart4>
%\fi
%    \begin{macrocode}
more text in part four
%    \end{macrocode}

%\iffalse
%</samplepart4>
%\fi
%
% %%%%%%%%%%%%%%%%%%%%%%%%%%%%%%%%%%%%%%
% \paragraph{Forwarding for a Complete Draft.}
%
% The following forwarding file |cdocsdrf.tex|
% compiles the main document in draft mode:
%\iffalse
%<*sampledraft>
%\fi
%    \begin{macrocode}
\def\version{draft}
\input{childdoc.def}
\childdocforward{cdocsamp}
%    \end{macrocode}

%\iffalse
%</sampledraft>
%\fi
%
% %%%%%%%%%%%%%%%%%%%%%%%%%%%%%%%%%%%%%%
% \paragraph{Forwarding for Final Version of the Chapters.}
%
% The following forwarding files |cdocsfn1.tex| and |cdocsfn2.tex|
% (with identical content)
% compile the final versions of the child documents
% |cdocsch1.tex| and |cdocsch2.tex|, respectively:
%\iffalse
%<*samplefinal>
%\fi
%    \begin{macrocode}
\def\version{final}
\input{childdoc.def}
\childdocforwardprefix[cdocsamp]{cdocsfn}{cdocsch}
%    \end{macrocode}

%\iffalse
%</samplefinal>
%\fi
%
% %%%%%%%%%%%%%%%%%%%%%%%%%%%%%%%%%%%%%%
% \paragraph{Command Line Processing.}
%
% The following three command lines generate the output files
% |cdocscld|, |cdocscl1| and |cdocscl2|
% which should be identical to
% |cdocsdrf|, |cdocsch1| and |cdocsfn2|, respectively:
% \begin{center}
% \begin{tabular}{l}
% |latex -jobname cdocscld \|\\
% |  "\def\version{draft}\input{childdoc.def}\childdocforward{cdocsamp}"|\\
% |latex -jobname cdocscl1 \|\\
% |  "\input{childdoc.def}\childdocforward[cdocsamp]{cdocsch1}"|\\
% |latex -jobname cdocscl2 \|\\
% |  "\def\version{final}\input{childdoc.def}\childdocforward{cdocsch2}"|
% \end{tabular}
% \end{center}
% Note that the trailing backslash on each first line
% merely continues the input to the second line
% (for convenient cut ant paste).
% Furthermore, the command |latex| can be replaced by any
% of its alternative versions such as |pdflatex|.
%
% %%%%%%%%%%%%%%%%%%%%%%%%%%%%%%%%%%%%%%%%%%%%%%%%%%%%%%%%%%%%%%%%%%%%%%%%%%%%%%
% %%%%%%%%%%%%%%%%%%%%%%%%%%%%%%%%%%%%%%%%%%%%%%%%%%%%%%%%%%%%%%%%%%%%%%%%%%%%%%
% \section{Implementation}
%\iffalse
%<*package>
%\fi
%
% This section describes the definitions file |childdoc.def|.

% The definitions cannot be loaded using |\usepackage| or |\RequirePackage|
% which has a mechanism to prevent loading a style file more than once.
% When loading the definitions by means of |\input|
% multiple instances have to be prevented manually:
%\iffalse
%This code needs to be before the `\ProvidesFile' directive
%which is defined at the beginning of this file.
%Therefore it is also placed there and commented out here.
%</package>
%<*discard>
%\fi
%    \begin{macrocode}
\ifdefined\childdocmain\endinput\fi
%    \end{macrocode}
%\iffalse
%</discard>
%<*package>
%\fi
%
% \macro{\ifchilddoc}
% \macro{\ifchilddocmanual}
% The conditional |\ifchilddoc| tells whether a
% child (true) or main (false) document is being compiled.
% The conditional |\ifchilddocmanual| tells whether
% the |\includeonly| mechanism is used (false) or
% the selection of child files must be performed manually (true).
% The definitions initialise to false:
%    \begin{macrocode}
\newif\ifchilddoc
\newif\ifchilddocmanual
%    \end{macrocode}

% \macro{\childdocname}
% \macro{\childdocjob}
% The macro |\childdocname| stores the name of the main document
% to be compiled. The macro |\childdocjob| stores the name of
% the document on which the \LaTeX{} compiler was originally invoked.
% The content of |\jobname| cannot be compared
% to filenames specified in the source due to different catcodes.
% The following code rescans |\jobname|, stores the result
% in |\childdocname| and saves a copy in |\childdocjob|:
%    \begin{macrocode}
\edef\childdocname{\scantokens\expandafter{\jobname\noexpand}}
\let\childdocjob\childdocname
%    \end{macrocode}

% \macro{\childdocdisable}
% The macro |\childdocdisable| prevents the main file
% from being processed more than once.
% At this stage, the main document command |\childdocmain|
% is assumed to be called once again where it should do nothing.
% Any subsequent call to it should prevent
% a secondary processing of the main document
% It overwrites the forwarding commands
% |\childdocof| and |\childdocforward|
% with empty macros to prevent further inclusions of the main document:
%    \begin{macrocode}
\newcommand{\childdocdisable}
{
  \renewcommand{\childdocmain}[1]{\renewcommand{\childdocmain}[1]{\endinput}}
  \renewcommand{\childdocof}[1]{}
  \renewcommand{\childdocby}[2][]{}
  \renewcommand{\childdocforward}[2][]{}
  \renewcommand{\childdocdisable}{}
}
%    \end{macrocode}

% \macro{\childdocmain}
% The macro |\childdocmain| is to be called at the top of the main file
% with nothing or the main filename (without extension) as argument.
% First, it breaks loops.
% If the argument is not empty and does not match |\childdocname|
% (which is set by the first inclusion of |childdoc.def|),
% |\ifchilddoc| is set to true, |\includeonly| is applied to the child file
% and |\jobname| is set to the main file
% (for proper handling of |.aux| files):
%    \begin{macrocode}
\newcommand{\childdocmain}[1]
{
  \childdocdisable\childdocmain{}
  \if?#1?\else
    \begingroup
      \def\childdoctmp{#1}
      \ifx\childdoctmp\childdocname
        \def\childdoctmp{}
      \else
        \def\childdoctmp
        {
          \childdoctrue
          \includeonly{\childdocname}
          \def\childdocjob{#1}
          \def\jobname{#1}
        }
      \fi
      \expandafter
    \endgroup
    \childdoctmp
  \fi
}
%    \end{macrocode}

% \macro{\childdocof}
% The command |\childdocof| redirects
% compilation to the main file |#1|.
%    \begin{macrocode}
\newcommand{\childdocof}[1]
{
  \childdocdisable
  \childdoctrue
  \includeonly{\childdocname}
  \def\jobname{#1}
  \def\childdocjob{#1}
  \input{#1}
}
%    \end{macrocode}

% \macro{\childdocby}
% The command |\childdocby| ....
%    \begin{macrocode}
\newcommand{\childdocby}[2][]
{
  \childdocdisable
  \childdoctrue
  \childdocmanualtrue
  \if?#1?\else
    \def\jobname{#2}
  \fi
  \def\childdocjob{#2}
  \input{#2}
  \endinput
}
%    \end{macrocode}

% \macro{\childdocforward}
% The command |\childdocforward| redirects
% compilation to the main file or
% (if the optional argument is given) a child file.
% Parameters are set as if the main file
% or a child file starting with |\childdocof| was compiled.
% Then compilation is handed over to the main file:
%    \begin{macrocode}
\newcommand{\childdocforward}[2][]
{
  \begingroup
    \if?#1?
      \def\childdoctmp
      {
        \def\childdocname{#2}
        \def\childdocjob{#2}
        \def\jobname{#2}
        \input{#2}
        \endinput
      }
    \else
      \def\childdoctmp
      {
        \childdocdisable
        \def\childdocname{#2}
        \childdoctrue
        \includeonly{#2}
        \def\childdocjob{#1}
        \def\jobname{#1}
        \input{#1}
        \endinput
      }
    \fi
    \expandafter
  \endgroup
  \childdoctmp
}
%    \end{macrocode}

% \macro{\childdocforwardprefix}
% The command |\childdocforwardprefix| redirects
% compilation to the main or a child file by means of a pattern.
% The prefix |#1| in the current filename is replaced by |#2|
% and the suffix of the current filename is kept
% (it is assumed that the filename does not contain the substring `|~~~|'
% which is used as a delimiter).
% Compilation is handed over to the new file by |\childdocforward|:
%    \begin{macrocode}
\newcommand{\childdocforwardprefix}[3][]
{
  \begingroup
    \def\childdocextract #2##1~~~{\def\childdoctmp{\childdocforward[#1]{#3##1}}}
    \expandafter\childdocextract\childdocname~~~
    \expandafter
  \endgroup
  \childdoctmp
}
%    \end{macrocode}

% \macro{\childdoc}
% The deprecated macro |\childdoc| is a legacy version of |\childdocmain|:
%    \begin{macrocode}
\newcommand{\childdoc}{\childdocmain}
%    \end{macrocode}

% \macro{\childdocredirect}
% The deprecated macro |\childdocredirect| is a legacy version
% of |\childdocforward| and |\childdocforwardprefix|:
%    \begin{macrocode}
\newcommand{\childdocredirect}[2][]
{
  \begingroup
    \if?#1?
      \def\childdoctmp{\childdocforward{#2}}
    \else
      \def\childdoctmp{\childdocforwardprefix{#1}{#2}}
    \fi
    \expandafter
  \endgroup
  \childdoctmp
}
%    \end{macrocode}

%\iffalse
%</package>
%\fi
%
\endinput
|\\
|\childdocforwardprefix{final}{child}|
\end{tabular}
\end{center}
%

Note that when several versions of a main file and/or of each child file
are to be generated, it may be convenient to set up a |Makefile| or
shell script to automatise the process.

%%%%%%%%%%%%%%%%%%%%%%%%%%%%%%%%%%%%%%%%%%%%%%%%%%%%%%%%%%%%%%%%%%%%%%%%%%%%%%%%
\subsection{Command Line Processing}
\label{sec:commandline}

The effect of redirection files can also be achieved by invoking
the \LaTeX{} compiler with a more elaborate command line.
Most conveniently this should be done as part
of a shell script or a |Makefile|.

When using \textsf{childdoc} in the main file, the following
command lines effectively perform a redirection
(note that depending on the shell being used,
backslashes may have to be doubled: `|\|' $\to$ `|\\|'):
%
\begin{center}
|... -jobname "|\textit{target}|" |\\|"|[\textit{flags}]%
|% \iffalse
%
% childdoc.dtx Copyright (C) 2017-2018 Niklas Beisert
%
% This work may be distributed and/or modified under the
% conditions of the LaTeX Project Public License, either version 1.3
% of this license or (at your option) any later version.
% The latest version of this license is in
%   http://www.latex-project.org/lppl.txt
% and version 1.3 or later is part of all distributions of LaTeX
% version 2005/12/01 or later.
%
% This work has the LPPL maintenance status `maintained'.
%
% The Current Maintainer of this work is Niklas Beisert.
%
% This work consists of the files childdoc.dtx and childdoc.ins
% and the derived files childdoc.def and cdocsamp.tex with
% cdocsch1.tex, cdocsch2.tex, cdocsdrf.tex, cdocsfn1.tex, cdocsfn2.tex.
%
%<package>\ifdefined\childdocmain\endinput\fi
%<package>\ProvidesFile{childdoc.def}[2018/12/30 v2.0 child document driver]
%<samplemain>\ProvidesFile{cdocsamp.tex}[2018/12/30 v2.0 sample for childdoc]
%<*driver>
%\ProvidesFile{childdoc.drv}[2018/12/30 v2.0 childdoc reference manual file]
\PassOptionsToClass{10pt,a4paper}{article}
\documentclass{ltxdoc}

\usepackage[margin=35mm]{geometry}
\usepackage{hyperref}
\usepackage{hyperxmp}
\usepackage[usenames]{color}

\hypersetup{colorlinks=true}
\hypersetup{pdfstartview=FitH}
\hypersetup{pdfpagemode=UseNone}
\hypersetup{pdfsource={}}
\hypersetup{pdflang={en-UK}}
\hypersetup{pdfcopyright={Copyright 2017-2018 Niklas Beisert.
  This work may be distributed and/or modified under the
  conditions of the LaTeX Project Public License, either version 1.3
  of this license or (at your option) any later version.}}
\hypersetup{pdflicenseurl={http://www.latex-project.org/lppl.txt}}
\hypersetup{pdfcontactaddress={ETH Zurich, ITP, HIT K,
  Wolfgang-Pauli-Strasse 27}}
\hypersetup{pdfcontactpostcode={8093}}
\hypersetup{pdfcontactcity={Zurich}}
\hypersetup{pdfcontactcountry={Switzerland}}
\hypersetup{pdfcontactemail={nbeisert@itp.phys.ethz.ch}}
\hypersetup{pdfcontacturl={http://people.phys.ethz.ch/\xmptilde nbeisert/}}

\newcommand{\secref}[1]{\hyperref[#1]{section \ref*{#1}}}

\parskip1ex
\parindent0pt
\let\olditemize\itemize
\def\itemize{\olditemize\parskip0pt}

\begin{document}

\title{The \textsf{childdoc} Package}
\hypersetup{pdftitle={The childdoc Package}}
\author{Niklas Beisert\\[2ex]
  Institut f\"ur Theoretische Physik\\
  Eidgen\"ossische Technische Hochschule Z\"urich\\
  Wolfgang-Pauli-Strasse 27, 8093 Z\"urich, Switzerland\\[1ex]
  \href{mailto:nbeisert@itp.phys.ethz.ch}
  {\texttt{nbeisert@itp.phys.ethz.ch}}}
\hypersetup{pdfauthor={Niklas Beisert}}
\hypersetup{pdfsubject={Manual for the LaTeX2e Package childdoc}}
\date{30 December 2018, \textsf{v2.0}}
\maketitle

\begin{abstract}\noindent
\textsf{childdoc} is a \LaTeXe{} package
that enables the direct compilation
of document sections included by |\include|
to individual files.
\end{abstract}

\begingroup
\parskip0ex
\tableofcontents
\endgroup

%%%%%%%%%%%%%%%%%%%%%%%%%%%%%%%%%%%%%%%%%%%%%%%%%%%%%%%%%%%%%%%%%%%%%%%%%%%%%%%%
%%%%%%%%%%%%%%%%%%%%%%%%%%%%%%%%%%%%%%%%%%%%%%%%%%%%%%%%%%%%%%%%%%%%%%%%%%%%%%%%
\section{Introduction}

\LaTeX{} provides a mechanism to structure a large document (such as a book)
into a main file and several child files (containing the chapters)
using the |\include| command.
This mechanism is beneficial for documents
which span hundreds of pages in order to
make the source file(s) more manageable.
Moreover, compilation can be restricted to
selected child files by means of the |\includeonly| command.
The latter feature can be used to reduce the compilation time while editing
(this was significantly more useful in the earlier days of \LaTeX{})
or to generate a smaller document which is easier to navigate.
Another application of |\includeonly| is to generate
documents consisting of selected parts of the complete document.

However, there are a few drawbacks of the plain |\include| mechanism:
\begin{itemize}
\item
The child files cannot be compiled on their own,
they can only be compiled via the main file.
A naive editing environment
(such as a text editor with an option
to have the current file processed by \LaTeX)
may require one to switch to the main file before compiling;
attempting to compile the child file produces errors.
\item
The main file must be modified (each time)
to adjust the |\includeonly| command
to the present needs. This easily leaves the main file in a messy state.
\item
The generated document will always carry the filename
of the main document. This is inconvenient if
several child files are to be compiled and
to be kept for distribution.
\end{itemize}

The present package provides a simple interface
to make child files individually compilable by \LaTeX{}.
Compiling a child file then has the same effect as compiling
the main file with an |\includeonly| command
to select the appropriate child.
Moreover the generated document will carry the name of the child
rather than the main file.
This resolves all three above issues.

This feature is meant to make the editing of books,
thesis documents and lecture notes somewhat more convenient.
However, the package can also be used efficiently for
composing a series of documents (such as exercise sheets)
which are typically distributed individually.
It then assists the author in generating the individual documents
(potentially in different versions)
as well as a document containing the collected series.
Another application is in developing style files
or other kinds of included material
where compilation of the style file could redirect
to a sample or test file.

%%%%%%%%%%%%%%%%%%%%%%%%%%%%%%%%%%%%%%%%%%%%%%%%%%%%%%%%%%%%%%%%%%%%%%%%%%%%%%%%
%%%%%%%%%%%%%%%%%%%%%%%%%%%%%%%%%%%%%%%%%%%%%%%%%%%%%%%%%%%%%%%%%%%%%%%%%%%%%%%%
\section{Usage}

First of all, the package \textsf{childdoc} is \emph{not} a standard
\LaTeXe{} |.sty| style file! Therefore it needs to be invoked in
a non-standard way.

%%%%%%%%%%%%%%%%%%%%%%%%%%%%%%%%%%%%%%%%%%%%%%%%%%%%%%%%%%%%%%%%%%%%%%%%%%%%%%%%
\subsection{Included Files}
\label{sec:include}

%%%%%%%%%%%%%%%%%%%%%%%%%%%%%%%%%%%%%%%%
\DescribeMacro{\childdocmain}
To use the package, add the commands
\begin{center}
\begin{tabular}{l}
|\input{childdoc.def}|\\
|\childdocmain{}|\\
\end{tabular}
\end{center}
at the very top of the main \LaTeX{} file,
in particular \emph{before} the |\documentclass| statement!
The argument of |\childdocmain| should be left empty
(but it must be present).

%%%%%%%%%%%%%%%%%%%%%%%%%%%%%%%%%%%%%%%%
\DescribeMacro{\childdocof}
Furthermore, add the commands
\begin{center}
\begin{tabular}{l}
|\input{childdoc.def}|\\
|\childdocof{|\textit{main}|}|\\
\end{tabular}
\end{center}
at the top of every child file \textit{child}
which is included by |\include{|\textit{child}|}|
from within the main file
(or at least for those files to be compiled individually).
The argument \textit{main} must be the filename of the main file.

There are a couple of
considerations in setting up the main and child documents:

%%%%%%%%%%%%%%%%%%%%%%%%%%%%%%%%%%%%%%%%
\paragraph{Restrictions.}

Please note the following restrictions:
\begin{itemize}
\item
|\childdocmain| must be called with one argument \textit{main}
to ensure compatibility with earlier version of the package.
It must either be empty (|\childdocmain{}|)
or precisely match the filename of the main file in which it is specified.
See \secref{sec:detection} for further information.
\item
The filename \textit{main} must be specified without the |.tex| extension.
\item
The filename \textit{main} is case sensitive
(even in case-insensitive file systems)
due to internal string comparison.
\item
The argument \textit{main} should be fully expanded, it cannot be a macro.
\item
Subdirectories and special characters should be avoided in filenames.
\item
The command |\childdocmain{|\textit{main}|}| must be followed by a whitespace.
It should not be followed immediately by another command
or by a comment mark `|%|'.
This is because the \TeX{} parser reads the token immediately following
the argument of |\childdocmain| and puts it
at the beginning of every child section;
however, a white\-space is ignored.
\end{itemize}

%%%%%%%%%%%%%%%%%%%%%%%%%%%%%%%%%%%%%%%%
\paragraph{Content of Main File.}

It is advisable to place all content in the child files included by |\include|.
Any output contained in the main file will appear in all child documents
unless suppressed manually;
it cannot be suppressed automatically by the |\includeonly| directive
and thus should normally be avoided.
A method to include some content in the main file
by means of conditional processing is described in \secref{sec:conditional}.

%%%%%%%%%%%%%%%%%%%%%%%%%%%%%%%%%%%%%%%%
\paragraph{Page Numbering.}

When only a part of the document is compiled,
the appropriate numbering of pages
(as well as other status parameters)
is determined from the |.aux| files.
The latter contain information from previous passes.
However this information needs to propagate through
all intermediate child documents.
Therefore the page numbering in child documents may well
be inconsistent until the complete document is compiled at least once.

A useful (if unconventional) way to always ensure a consistent
page numbering is to restart the numbering in each child document
and denote the pages by `\textit{child}|.|\textit{page}'
where \textit{child} represents the chapter/section number of the child file.
This can be achieved by the command
|\numberwithin{page}{|\textit{child}|}|
of the \textsf{amsmath} package
where \textit{child} can be |chapter| or |section|
depending on the chosen structuring.
Alternatively, one can modify the macro |\thepage| appropriately
and reset the counter |page| at the start of each child file.

%%%%%%%%%%%%%%%%%%%%%%%%%%%%%%%%%%%%%%%%%%%%%%%%%%%%%%%%%%%%%%%%%%%%%%%%%%%%%%%%
\subsection{Conditional Processing}
\label{sec:conditional}

The package provides a mechanism to compile different versions
of a document. To customise the versions further some conditional processing
can come in handy to distinguish which version is being compiled.
The package provides two macros to describe the compilation context:

%%%%%%%%%%%%%%%%%%%%%%%%%%%%%%%%%%%%%%%%
\DescribeMacro{\ifchilddoc}
The conditional |\ifchilddoc| distinguishes between the compilation of
child documents and the main document:
%
\begin{center}
|\ifchilddoc |\textit{child-code}| |[|\||else |\textit{main-code}]| \||fi|
\end{center}

%%%%%%%%%%%%%%%%%%%%%%%%%%%%%%%%%%%%%%%%
\DescribeMacro{\childdocname}
\DescribeMacro{\childdocjob}
The macro |\childdocname| contains the filename (without extension)
of the main or child file being processed.
Note that |\childdocjob| will always contain the name of the main file.

%%%%%%%%%%%%%%%%%%%%%%%%%%%%%%%%%%%%%%%%
\paragraph{Title Page.}

Conditional processing can be used to include a title or banner page
in the main document when proper precautions are taken.
Importantly, the code in the main file should ensure that the page counter
(as well as other status parameters which are stored in the |.aux| files)
takes the same value after the conditional processing.
Otherwise the page numbers may take divergent values
depending on which part is compiled.

For example, a title page could be declared by:
%
\begin{center}
\begin{tabular}{l}
|\ifchilddoc\||else|\\
|\addtocounter{page}{-1}|\\
\textit{code for title page}\\
|\newpage|\\
|\||fi|
\end{tabular}
\end{center}
%
A banner page for the child documents can be generated by:
%
\begin{center}
\begin{tabular}{l}
|\ifchilddoc|\\
|\addtocounter{page}{-1}|\\
\textit{code for banner page}\\
|\newpage|\\
|\||fi|
\end{tabular}
\end{center}
%
Here one could write a message such as:
\begin{center}
|This is the part \childdocname{} of \childdocjob{}.|
\end{center}

%%%%%%%%%%%%%%%%%%%%%%%%%%%%%%%%%%%%%%%%%%%%%%%%%%%%%%%%%%%%%%%%%%%%%%%%%%%%%%%%
\subsection{Flags}
\label{sec:flags}

The package makes it easy to generate different versions
of the main or child documents.
To this end compilation flags can be defined
and assigned different default values.
They will be particularly useful in conjunction
with the forwarding mechanism described in \secref{sec:forward}.

For example, it may be useful to have a flag |\version|
which can be set to |draft| or |final|.
The document source will contain some conditional code
depending on the value of |\version|.
Suppose further, the flag should default to |final| for the main file
and to |draft| for child files
which is a natural assignment for editing the document.
This is achieved by placing the following code
in the preamble of the main document
(below the |\childdocmain| directive):
%
\begin{center}
\begin{tabular}{l}
|\ifchilddoc|\\
|\providecommand{\version}{draft}|\\
|\||else|\\
|\providecommand{\version}{final}|\\
|\||fi|
\end{tabular}
\end{center}
%
The definition by |\providecommand| makes sure
that previous definitions are not overwritten.
Further statements |\providecommand{\version}{...}|
can thus be added before the above code to override it.

For the main file, one might add a line
(between |\childdocmain| and the above block)
%
\begin{center}
|%\ifchilddoc\||else\providecommand{\version}{draft}\||fi|
\end{center}
%
which can be uncommented to produce a draft version.
Likewise one can add a line to the very top of a child file
(above the |\childdocof{|\textit{main}|}| directive)
%
\begin{center}
|%\providecommand{\version}{final}|
\end{center}
%
which can be uncommented to produce the final version of this child document.

%%%%%%%%%%%%%%%%%%%%%%%%%%%%%%%%%%%%%%%%%%%%%%%%%%%%%%%%%%%%%%%%%%%%%%%%%%%%%%%%
\subsection{Forwarding}
\label{sec:forward}

Different versions of the main or child documents
using compilation flags as described in \secref{sec:flags}
can be (permanently) stored in different files
for convenient compilation, viewing and distribution.
To this end, the package defines a command
to pass on compilation to a different file:

%%%%%%%%%%%%%%%%%%%%%%%%%%%%%%%%%%%%%%%%
\DescribeMacro{\childdocforward}
The command |\childdocforward| redirects processing to
another source file:
%
\begin{center}
\begin{tabular}{l}
|\input{childdoc.def}|\\
|\childdocforward[|\textit{main}|]{|\textit{dest}|}|\\
\end{tabular}
\end{center}
%
The argument \textit{dest} is the destination file
(without extension).
It should be the main file or one of the child files.
Note that further \textsf{childdoc} directives
such as |\childdocof| and |\childdocforward|
in the indicated file will be processed in this form.
The optional argument \textit{main}
passes on directly to the main file \textit{main}
while pretending to compile the child \textit{dest}.
This form behaves as if \textit{dest}
issues |\childdocof{|\textit{main}|}| right away,
and no further \textsf{childdoc} directives will be processed.

%%%%%%%%%%%%%%%%%%%%%%%%%%%%%%%%%%%%%%%%
\DescribeMacro{\...prefix}
In the alternative form |\childdocforwardprefix|,
%
\begin{center}
\begin{tabular}{l}
|\input{childdoc.def}|\\
|\childdocforwardprefix[|\textit{main}|]{|\textit{prefix}|}{|\textit{dest}|}|
\end{tabular}
\end{center}
%
the destination file is determined by a pattern
depending on the current file:
To make this work, the current file must be called
`{\textit{prefix}\hspace{0.2em}\textit{suffix}}'
with \textit{prefix} matching precisely the argument.
Processing is then passed on to the file
`{\textit{dest}\hspace{0.2em}\textit{suffix}}'.
Surely, the same effect is achieved by
directly specifying the
argument `{\textit{dest}\hspace{0.2em}\textit{suffix}}'
in the first form.
However, that requires to set up a different file
for each child. With the alternative form of the command
all these files can have exactly the same content
which simplifies setting them up and maintaining them.

For example, the following file |draft.tex|
with a compilation flag |\version| as described in \secref{sec:flags}
compiles the main document as a draft:
%
\begin{center}
\begin{tabular}{l}
|\def\version{draft}|\\
|\input{childdoc.def}|\\
|\childdocforward{|\textit{main}|}|
\end{tabular}
\end{center}
%
Likewise, the following files |final|\textit{nn}|.tex|
compile the final version of the child document
|child|\textit{nn}|.tex|:
%
\begin{center}
\begin{tabular}{l}
|\def\version{final}|\\
|\input{childdoc.def}|\\
|\childdocforwardprefix{final}{child}|
\end{tabular}
\end{center}
%

Note that when several versions of a main file and/or of each child file
are to be generated, it may be convenient to set up a |Makefile| or
shell script to automatise the process.

%%%%%%%%%%%%%%%%%%%%%%%%%%%%%%%%%%%%%%%%%%%%%%%%%%%%%%%%%%%%%%%%%%%%%%%%%%%%%%%%
\subsection{Command Line Processing}
\label{sec:commandline}

The effect of redirection files can also be achieved by invoking
the \LaTeX{} compiler with a more elaborate command line.
Most conveniently this should be done as part
of a shell script or a |Makefile|.

When using \textsf{childdoc} in the main file, the following
command lines effectively perform a redirection
(note that depending on the shell being used,
backslashes may have to be doubled: `|\|' $\to$ `|\\|'):
%
\begin{center}
|... -jobname "|\textit{target}|" |\\|"|[\textit{flags}]%
|\input{childdoc.def}\childdocforward[|\textit{main}|]{|\textit{dest}|}"|
\end{center}
%
Here \textit{target} is the name of the output file,
\textit{main} is the name of the main file
and \textit{dest} is the name of the main or child file to be processed
(all filenames without extensions).
The optional argument \textit{main} can be omitted
if \textit{main} matches \textit{dest}.
Optionally, compilation \textit{flags} can be defined via |\def| commands.
This command line makes the \TeX{} engine believe
it is compiling the file \textit{target}
whose content is specified as the latter parameter.
The provided code then forwards the processing to
\textit{main} or \textit{dest} as described in \secref{sec:forward}.

%%%%%%%%%%%%%%%%%%%%%%%%%%%%%%%%%%%%%%%%%%%%%%%%%%%%%%%%%%%%%%%%%%%%%%%%%%%%%%%%
\subsection{Include by Input}
\label{sec:input}

Including child documents by |\include| has some restrictions by design.
Most notably, the content of a child document always occupies
its own set of pages; pages cannot be shared between child documents.
Usually, this behaviour makes perfect sense
because each child document contain an essential part of the document.
However, in some situations it may be desirable to compose
a document from a collection of parts
without having mandatory page breaks between then.
For this case, the package
provides a mechanism to include parts
by |\input| which can also be processed individually.
However, by construction this mechanism
requires manual handling of the content to be output.

%%%%%%%%%%%%%%%%%%%%%%%%%%%%%%%%%%%%%%%%
\DescribeMacro{\ifchilddocmanual}
The main file should be prepared as usual, see \secref{sec:include}.
However, the document body must make a distinction
between processing of an individual part and of the main document, e.g.:
%
\begin{center}
\begin{tabular}{l}
|\ifchilddocmanual|\\
|\input{\childdocname}|\\
|\||else|\\
\textit{document body with }|\input{|\textit{part}|}|\\
|\||fi|
\end{tabular}
\end{center}
%
The conditional |\ifchilddocmanual| is true whenever
a part to be included by |\input| is being compiled,
and the name of the part is stored in |\childdocname|.

%%%%%%%%%%%%%%%%%%%%%%%%%%%%%%%%%%%%%%%%
\DescribeMacro{\childdocby}
Each part to be included by |\input| should start with:
%
\begin{center}
\begin{tabular}{l}
|\input{childdoc.def}|\\
|\childdocby{|\textit{main}|}|\\
\end{tabular}
\end{center}
%
The directive |\childdocby| is similar to |\childdocof|
described in \secref{sec:include},
but the subsequent selection of content must be done manually.
To that end, both |\ifchilddoc| and |\ifchilddocmanual|
will be true upon processing of a part,
and the name of the part is stored in |\childdocname|.
Note that |\jobname| will be set to the filename of the current part
so that each part receives an individual |.aux| file
that does not interfere with the |.aux| file(s) of the main document.
This behaviour can be altered by the alternative form
|\childdocby[*]{|\textit{main}|}| (with a non-empty optional argument)
which uses the |.aux| file of the main document
by setting |\jobname| to \textit{main}.

%%%%%%%%%%%%%%%%%%%%%%%%%%%%%%%%%%%%%%%%%%%%%%%%%%%%%%%%%%%%%%%%%%%%%%%%%%%%%%%%
\subsection{Driver Development}
\label{sec:driver}

The \textsf{childdoc} mechanism can also be use for the development
of definition files such as \LaTeX{} styles or classes.
This case differs from the above setup with multiple parts
included by |\include| in that no |\includeonly| should be invoked.
This can be achieved by starting the include file
(before |\ProvidesPackage|) with:
%
\begin{center}
\begin{tabular}{l}
|\input{childdoc.def}|\\
|\childdocforward{|\textit{main}|}|\\
\end{tabular}
\end{center}
%
or alternatively with:
%
\begin{center}
\begin{tabular}{l}
|\input{childdoc.def}|\\
|\childdocby{|\textit{main}|}|\\
\end{tabular}
\end{center}
%
Both forms have slightly different effects as described above.
The main file is prepared as usual, see \secref{sec:include}.

%%%%%%%%%%%%%%%%%%%%%%%%%%%%%%%%%%%%%%%%%%%%%%%%%%%%%%%%%%%%%%%%%%%%%%%%%%%%%%%%
\subsection{Legacy Detection}
\label{sec:detection}

The directive |\childdocmain| in the main file can detect
whether the complete document or merely a child is to be compiled
even without using the directive |\childdocof|.
This method is deprecated because it is less robust
and there is no compelling reason to use it;
it is merely provided for backward compatibility
and it may be removed in future versions.

If the detection mechanism is to be used,
it is mandatory to correctly specify
the filename of the main file as the argument of |\childdocmain|:
%
\begin{center}
\begin{tabular}{l}
|\input{childdoc.def}|\\
|\childdocmain{|\textit{main}|}|\\
\end{tabular}
\end{center}
%
If |\jobname| does not match the argument \textit{main} of |\childdocmain|,
it is assumed that |\jobname| points to the child file to be compiled.
When using |\childdocmain| with the main file specified as argument,
it suffices to start a child file
with just |\input{|\textit{main}|}|
without loading of the package and using |\childdocof|.
If instead all processing is done
with the appropriate \textsf{childdoc} directives,
the argument of \textit{main} of |\childdocmain| can be empty.

An alternative version of the command line processing described
in \secref{sec:commandline} using the detection mechanism reads:
%
\begin{center}
|... -jobname "|\textit{target}|" "|[\textit{flags}]%
[|\def\jobname{|\textit{dest}|}|]|\input{|\textit{main}|}"|
\end{center}

%%%%%%%%%%%%%%%%%%%%%%%%%%%%%%%%%%%%%%%%%%%%%%%%%%%%%%%%%%%%%%%%%%%%%%%%%%%%%%%%
\subsection{Manual Code}
\label{sec:manual}

In case one cannot be certain whether the definitions file |childdoc.def|
is installed on the target \TeX{} distribution
and one prefers not to ship it,
it is conceivable to paste a few relevant commands into the sources.

To that end, drop all statements |\input{childdoc.def}|
and perform the replacements as outlined below.
Instead of |\childdocmain{|\textit{main}|}| add the following code
to the top of the main file:
%
\begin{center}
\begin{tabular}{l}
|\||ifdefined\childdocname\endinput\||fi\newif\ifchilddoc|\\
|\edef\childdocname{\scantokens\expandafter{\jobname\noexpand}}|\\
|\def\childdocmain{|\textit{main}|}\||ifx\childdocmain\childdocname\||else|\\
|\childdoctrue\includeonly{\childdocname}\let\jobname\childdocmain\||fi|\\
\end{tabular}
\end{center}
%
Instead of |\childdocof{|\textit{main}|}| just include the main file
at the top of each child file:
%
\begin{center}
|\input{|\textit{main}|}|
\end{center}
%
A simple redirection |\childdocforward{|\textit{dest}|}| is achieved by:
%
\begin{center}
|\def\jobname{|\textit{dest}|}\input{\jobname}|
\end{center}
%
The redirection with prefix
|\childdocforwardprefix[|\textit{prefix}|]{|\textit{dest}|}|
is accomplished by:
%
\begin{center}
\begin{tabular}{l}
|{\edef\jobname{\scantokens\expandafter{\jobname\noexpand}}|\\
|\def\redirectjob |\textit{prefix}|#1~~~{\gdef\jobname{|\textit{dest}|#1}}|\\
|\expandafter\redirectjob\jobname~~~}\input{\jobname}|
\end{tabular}
\end{center}

In an alternative approach,
child documents can be compiled by a specific command line
without additional code or specific definitions:
%
\begin{center}
|... -jobname "|\textit{target}|" "|[\textit{flags}]%
|\includeonly{|\textit{dest}|}\input{|\textit{main}|}"|
\end{center}
%

%%%%%%%%%%%%%%%%%%%%%%%%%%%%%%%%%%%%%%%%%%%%%%%%%%%%%%%%%%%%%%%%%%%%%%%%%%%%%%%%
%%%%%%%%%%%%%%%%%%%%%%%%%%%%%%%%%%%%%%%%%%%%%%%%%%%%%%%%%%%%%%%%%%%%%%%%%%%%%%%%
\section{Information}

%%%%%%%%%%%%%%%%%%%%%%%%%%%%%%%%%%%%%%%%%%%%%%%%%%%%%%%%%%%%%%%%%%%%%%%%%%%%%%%%
\subsection{Copyright}

Copyright \copyright{} 2017--2018 Niklas Beisert

This work may be distributed and/or modified under the
conditions of the \LaTeX{} Project Public License, either version 1.3
of this license or (at your option) any later version.
The latest version of this license is in
  \url{http://www.latex-project.org/lppl.txt}
and version 1.3 or later is part of all distributions of \LaTeX{}
version 2005/12/01 or later.

This work has the LPPL maintenance status `maintained'.

The Current Maintainer of this work is Niklas Beisert.

This work consists of the files |README.txt|, |childdoc.ins| and |childdoc.dtx|
as well as the derived files |childdoc.def|, |cdocsamp.tex|
with |cdocsch1.tex|, |cdocsch2.tex|, |cdocspt3.tex|, |cdocspt4.tex|,
|cdocsdrf.tex|, |cdocsfn1.tex|, |cdocsfn2.tex|
as well as |childdoc.pdf|.

%%%%%%%%%%%%%%%%%%%%%%%%%%%%%%%%%%%%%%%%%%%%%%%%%%%%%%%%%%%%%%%%%%%%%%%%%%%%%%%%
\subsection{Files and Installation}

The package consists of the files:
%
\begin{center}
\begin{tabular}{ll}
    |README.txt|   & readme file \\
    |childdoc.ins| & installation file \\
    |childdoc.dtx| & source file \\
    |childdoc.def| & definition file \\
    |cdocsamp.tex| & sample main file \\
    |cdocsch1.tex| & sample include file \\
    |cdocsch2.tex| & sample include file \\
    |cdocspt3.tex| & sample part file \\
    |cdocspt4.tex| & sample part file \\
    |cdocsdrf.tex| & sample redirection file \\
    |cdocsfn1.tex| & sample redirection file \\
    |cdocsfn2.tex| & sample redirection file \\
    |childdoc.pdf| & manual
\end{tabular}
\end{center}
%
The distribution consists of the files
|README.txt|, |childdoc.ins| and |childdoc.dtx|.
%
\begin{itemize}
\item
Run (pdf)\LaTeX{} on |childdoc.dtx|
to compile the manual |childdoc.pdf| (this file).
\item
Run \LaTeX{} on |childdoc.ins| to create the definitions file |childdoc.def|
and the sample |cdocsamp.tex| with include files
|cdocsch1.tex|, |cdocsch2.tex|, |cdocspt3.tex|, |cdocspt4.tex|,
|cdocsdrf.tex|, |cdocsfn1.tex|, |cdocsfn2.tex|.
Then copy the file |childdoc.def| to an appropriate directory of your \LaTeX{}
distribution, e.g.\ \textit{texmf-root}|/tex/latex/childdoc|.
\end{itemize}

%%%%%%%%%%%%%%%%%%%%%%%%%%%%%%%%%%%%%%%%%%%%%%%%%%%%%%%%%%%%%%%%%%%%%%%%%%%%%%%%
\subsection{Related CTAN Packages}

There are several other packages which offer a similar functionality:
%
\begin{itemize}
\item
The packages
\href{http://ctan.org/pkg/docmute}{\textsf{docmute}},
\href{http://ctan.org/pkg/includex}{\textsf{includex}} and
\href{http://ctan.org/pkg/standalone}{\textsf{standalone}}
provide commands to include only the document body of
a child file thus allowing both files to be compiled individually.
\item
The packages \href{http://ctan.org/pkg/subdocs}{\textsf{subdocs}}
and \href{http://ctan.org/pkg/subfiles}{\textsf{subfiles}}
provide structures in which the main and child documents can be
encapsulated and allowing them to be compiled individually.
The inclusion mechanism is different from the conventional |\include|.
\item
The package \href{http://ctan.org/pkg/combine}{\textsf{combine}}
is an elaborate solution to combine several documents into one.
\end{itemize}
%
See also the CTAN topic \href{http://ctan.org/topic/subdocs}{\textsf{subdocs}}
for further related packages.
The present package differs from the above solutions in that
a document structure constructed with the conventional |\include| mechanism
just needs two extra commands at the top of every file
such that all constituent files can be compiled individually.

%%%%%%%%%%%%%%%%%%%%%%%%%%%%%%%%%%%%%%%%%%%%%%%%%%%%%%%%%%%%%%%%%%%%%%%%%%%%%%%%
%\subsection{Feature Suggestions}
%
%The following is a list of features which may be useful for future
%versions of this package:
%%
%\begin{itemize}
%\item
%\ldots
%\end{itemize}

%%%%%%%%%%%%%%%%%%%%%%%%%%%%%%%%%%%%%%%%%%%%%%%%%%%%%%%%%%%%%%%%%%%%%%%%%%%%%%%%
\subsection{Revision History}

%%%%%%%%%%%%%%%%%%%%%%%%%%%%%%%%%%%%%%%%
\paragraph{v2.0:} 2018/12/30

\begin{itemize}
\item
immediate forward processing
\item
added |\childdocby| mechanism
\item
manual restructured
\end{itemize}

%%%%%%%%%%%%%%%%%%%%%%%%%%%%%%%%%%%%%%%%
\paragraph{v1.6:} 2018/01/17

\begin{itemize}
\item
application for development of include files
\item
corrections to manual
\end{itemize}

%%%%%%%%%%%%%%%%%%%%%%%%%%%%%%%%%%%%%%%%
\paragraph{v1.5:} 2017/05/21

\begin{itemize}
\item
more complete structuring introduced
\item
|\childdocof| introduced
\item
|\childdoc| renamed to |\childdocmain|
\item
|\childredirect| renamed to |\childdocforward| and |\childdocforwardprefix|
and functionality expanded
\end{itemize}

%%%%%%%%%%%%%%%%%%%%%%%%%%%%%%%%%%%%%%%%
\paragraph{v1.0:} 2017/04/27

\begin{itemize}
\item
manual and install package
\item
first version published on CTAN
\end{itemize}

%%%%%%%%%%%%%%%%%%%%%%%%%%%%%%%%%%%%%%%%
\paragraph{v0.6:} 2017/04/26

\begin{itemize}
\item
redirection mechanism added
\end{itemize}

%%%%%%%%%%%%%%%%%%%%%%%%%%%%%%%%%%%%%%%%
\paragraph{v0.5:} 2017/04/26

\begin{itemize}
\item
functionality in definition file
\end{itemize}


%%%%%%%%%%%%%%%%%%%%%%%%%%%%%%%%%%%%%%%%%%%%%%%%%%%%%%%%%%%%%%%%%%%%%%%%%%%%%%%%
%%%%%%%%%%%%%%%%%%%%%%%%%%%%%%%%%%%%%%%%%%%%%%%%%%%%%%%%%%%%%%%%%%%%%%%%%%%%%%%%
%%%%%%%%%%%%%%%%%%%%%%%%%%%%%%%%%%%%%%%%%%%%%%%%%%%%%%%%%%%%%%%%%%%%%%%%%%%%%%%%
\appendix

\settowidth\MacroIndent{\rmfamily\scriptsize 000\ }

 \DocInput{childdoc.dtx}

\end{document}
%</driver>
% \fi
%
% %%%%%%%%%%%%%%%%%%%%%%%%%%%%%%%%%%%%%%%%%%%%%%%%%%%%%%%%%%%%%%%%%%%%%%%%%%%%%%
% %%%%%%%%%%%%%%%%%%%%%%%%%%%%%%%%%%%%%%%%%%%%%%%%%%%%%%%%%%%%%%%%%%%%%%%%%%%%%%
% \section{Sample}
%\iffalse
%<*samplemain>
%\fi
%
% The following presents a sample document
% with two chapters, two parts, a title page,
% a compile flag as well as three forwarding files to set the flag.
% It consists of eight |.tex| files:
% \begin{center}
% \begin{tabular}{ll}
% |cdocsamp.tex|&main file\\
% |cdocsch1.tex|&include file for chapter 1\\
% |cdocsch2.tex|&include file for chapter 2\\
% |cdocspt3.tex|&include file for part 3\\
% |cdocspt4.tex|&include file for part 4\\
% |cdocsdrf.tex|&forwarding file for main file in draft mode\\
% |cdocsfi1.tex|&forwarding file for final version of chapter 1\\
% |cdocsfi2.tex|&forwarding file for final version of chapter 2\\
% \end{tabular}
% \end{center}
% Each of the eight files can be compiled directly by the \LaTeX{} compiler.
%
% %%%%%%%%%%%%%%%%%%%%%%%%%%%%%%%%%%%%%%
% \paragraph{Main File.}
%
% The main file is called |cdocsamp.tex|.
%
% Load the \textsf{childdoc} definitions and
% declare the filename for the main document:
%    \begin{macrocode}
\input{childdoc.def}
\childdocmain{}
%    \end{macrocode}

% Optional override for |\version| flag:
%    \begin{macrocode}
%%\ifchilddoc\else\providecommand{\version}{draft}\fi
%    \end{macrocode}

% Define the default values for the |\version| flag
% (|final| for the main file and |draft| for childs):
%    \begin{macrocode}
\ifchilddoc
\providecommand{\version}{draft}
\else
\providecommand{\version}{final}
\fi
%    \end{macrocode}

% Load the standard document class:
%    \begin{macrocode}
\documentclass[12pt]{article}
%    \end{macrocode}

% Start the document body:
%    \begin{macrocode}
\begin{document}
%    \end{macrocode}

% Declare a title page.
% Print title, part of document being processed and version flag:
%    \begin{macrocode}
\addtocounter{page}{-1}
\begin{center}
{\LARGE\bfseries{}childdoc example\par}
\vspace{1cm}
\ifchilddoc
\ifchilddocmanual part\else chapter\fi:
`\childdocname' of `\childdocjob'\par
\else
main document: `\childdocjob'\par
\fi
version: \version\par
\end{center}
\newpage
%    \end{macrocode}

% Manually include selected file,
% otherwise process as usual:
%    \begin{macrocode}
\ifchilddocmanual
\section*{part `\childdocname'}
\input{\childdocname}
\else
%    \end{macrocode}

% Include the two chapters:
%    \begin{macrocode}
\include{cdocsch1}
\include{cdocsch2}
%    \end{macrocode}

% Include the two parts unless only chapters should be displayed:
%    \begin{macrocode}
\ifchilddoc\else
\section{part three}
\input{cdocspt3}
\section{part four}
\input{cdocspt4}
\fi
%    \end{macrocode}

% Process as usual until here:
%    \begin{macrocode}
\fi
%    \end{macrocode}

% End of document body:
%    \begin{macrocode}
\end{document}
%    \end{macrocode}
%\iffalse
%</samplemain>
%\fi
%
% %%%%%%%%%%%%%%%%%%%%%%%%%%%%%%%%%%%%%%
% \paragraph{Chapter Include Files.}
%
% The include files are called |cdocsch1.tex| and |cdocsch2.tex|.
%
%\iffalse
%<*samplechap1|samplechap2>
%\fi

% Optional override for |\version| flag:
%    \begin{macrocode}
%%\providecommand{\version}{final}
%    \end{macrocode}

% Include the main document:
%    \begin{macrocode}
\input{childdoc.def}
\childdocof{cdocsamp}
%    \end{macrocode}

%\iffalse
%</samplechap1|samplechap2>
%\fi
%
%\iffalse
%<*samplechap1>
%\fi
% Some text for chapter 1:
%    \begin{macrocode}
\section{one}
some text in chapter one
%    \end{macrocode}

%\iffalse
%</samplechap1>
%\fi
% Some text for chapter 2:
%\iffalse
%<*samplechap2>
%\fi
%    \begin{macrocode}
\section{two}
more text in chapter two
%    \end{macrocode}

%\iffalse
%</samplechap2>
%\fi
%
% %%%%%%%%%%%%%%%%%%%%%%%%%%%%%%%%%%%%%%
% \paragraph{Part Include Files.}
%
% The include files are called |cdocspt3.tex| and |cdocspt4.tex|.
%
%\iffalse
%<*samplepart3|samplepart4>
%\fi

% Optional override for |\version| flag:
%    \begin{macrocode}
%%\providecommand{\version}{final}
%    \end{macrocode}

% Include the main document:
%    \begin{macrocode}
\input{childdoc.def}
\childdocby{cdocsamp}
%    \end{macrocode}

%\iffalse
%</samplepart3|samplepart4>
%\fi
%
%\iffalse
%<*samplepart3>
%\fi
% Some text for part 3:
%    \begin{macrocode}
some text in part three
%    \end{macrocode}

%\iffalse
%</samplepart3>
%\fi
% Some text for part 4:
%\iffalse
%<*samplepart4>
%\fi
%    \begin{macrocode}
more text in part four
%    \end{macrocode}

%\iffalse
%</samplepart4>
%\fi
%
% %%%%%%%%%%%%%%%%%%%%%%%%%%%%%%%%%%%%%%
% \paragraph{Forwarding for a Complete Draft.}
%
% The following forwarding file |cdocsdrf.tex|
% compiles the main document in draft mode:
%\iffalse
%<*sampledraft>
%\fi
%    \begin{macrocode}
\def\version{draft}
\input{childdoc.def}
\childdocforward{cdocsamp}
%    \end{macrocode}

%\iffalse
%</sampledraft>
%\fi
%
% %%%%%%%%%%%%%%%%%%%%%%%%%%%%%%%%%%%%%%
% \paragraph{Forwarding for Final Version of the Chapters.}
%
% The following forwarding files |cdocsfn1.tex| and |cdocsfn2.tex|
% (with identical content)
% compile the final versions of the child documents
% |cdocsch1.tex| and |cdocsch2.tex|, respectively:
%\iffalse
%<*samplefinal>
%\fi
%    \begin{macrocode}
\def\version{final}
\input{childdoc.def}
\childdocforwardprefix[cdocsamp]{cdocsfn}{cdocsch}
%    \end{macrocode}

%\iffalse
%</samplefinal>
%\fi
%
% %%%%%%%%%%%%%%%%%%%%%%%%%%%%%%%%%%%%%%
% \paragraph{Command Line Processing.}
%
% The following three command lines generate the output files
% |cdocscld|, |cdocscl1| and |cdocscl2|
% which should be identical to
% |cdocsdrf|, |cdocsch1| and |cdocsfn2|, respectively:
% \begin{center}
% \begin{tabular}{l}
% |latex -jobname cdocscld \|\\
% |  "\def\version{draft}\input{childdoc.def}\childdocforward{cdocsamp}"|\\
% |latex -jobname cdocscl1 \|\\
% |  "\input{childdoc.def}\childdocforward[cdocsamp]{cdocsch1}"|\\
% |latex -jobname cdocscl2 \|\\
% |  "\def\version{final}\input{childdoc.def}\childdocforward{cdocsch2}"|
% \end{tabular}
% \end{center}
% Note that the trailing backslash on each first line
% merely continues the input to the second line
% (for convenient cut ant paste).
% Furthermore, the command |latex| can be replaced by any
% of its alternative versions such as |pdflatex|.
%
% %%%%%%%%%%%%%%%%%%%%%%%%%%%%%%%%%%%%%%%%%%%%%%%%%%%%%%%%%%%%%%%%%%%%%%%%%%%%%%
% %%%%%%%%%%%%%%%%%%%%%%%%%%%%%%%%%%%%%%%%%%%%%%%%%%%%%%%%%%%%%%%%%%%%%%%%%%%%%%
% \section{Implementation}
%\iffalse
%<*package>
%\fi
%
% This section describes the definitions file |childdoc.def|.

% The definitions cannot be loaded using |\usepackage| or |\RequirePackage|
% which has a mechanism to prevent loading a style file more than once.
% When loading the definitions by means of |\input|
% multiple instances have to be prevented manually:
%\iffalse
%This code needs to be before the `\ProvidesFile' directive
%which is defined at the beginning of this file.
%Therefore it is also placed there and commented out here.
%</package>
%<*discard>
%\fi
%    \begin{macrocode}
\ifdefined\childdocmain\endinput\fi
%    \end{macrocode}
%\iffalse
%</discard>
%<*package>
%\fi
%
% \macro{\ifchilddoc}
% \macro{\ifchilddocmanual}
% The conditional |\ifchilddoc| tells whether a
% child (true) or main (false) document is being compiled.
% The conditional |\ifchilddocmanual| tells whether
% the |\includeonly| mechanism is used (false) or
% the selection of child files must be performed manually (true).
% The definitions initialise to false:
%    \begin{macrocode}
\newif\ifchilddoc
\newif\ifchilddocmanual
%    \end{macrocode}

% \macro{\childdocname}
% \macro{\childdocjob}
% The macro |\childdocname| stores the name of the main document
% to be compiled. The macro |\childdocjob| stores the name of
% the document on which the \LaTeX{} compiler was originally invoked.
% The content of |\jobname| cannot be compared
% to filenames specified in the source due to different catcodes.
% The following code rescans |\jobname|, stores the result
% in |\childdocname| and saves a copy in |\childdocjob|:
%    \begin{macrocode}
\edef\childdocname{\scantokens\expandafter{\jobname\noexpand}}
\let\childdocjob\childdocname
%    \end{macrocode}

% \macro{\childdocdisable}
% The macro |\childdocdisable| prevents the main file
% from being processed more than once.
% At this stage, the main document command |\childdocmain|
% is assumed to be called once again where it should do nothing.
% Any subsequent call to it should prevent
% a secondary processing of the main document
% It overwrites the forwarding commands
% |\childdocof| and |\childdocforward|
% with empty macros to prevent further inclusions of the main document:
%    \begin{macrocode}
\newcommand{\childdocdisable}
{
  \renewcommand{\childdocmain}[1]{\renewcommand{\childdocmain}[1]{\endinput}}
  \renewcommand{\childdocof}[1]{}
  \renewcommand{\childdocby}[2][]{}
  \renewcommand{\childdocforward}[2][]{}
  \renewcommand{\childdocdisable}{}
}
%    \end{macrocode}

% \macro{\childdocmain}
% The macro |\childdocmain| is to be called at the top of the main file
% with nothing or the main filename (without extension) as argument.
% First, it breaks loops.
% If the argument is not empty and does not match |\childdocname|
% (which is set by the first inclusion of |childdoc.def|),
% |\ifchilddoc| is set to true, |\includeonly| is applied to the child file
% and |\jobname| is set to the main file
% (for proper handling of |.aux| files):
%    \begin{macrocode}
\newcommand{\childdocmain}[1]
{
  \childdocdisable\childdocmain{}
  \if?#1?\else
    \begingroup
      \def\childdoctmp{#1}
      \ifx\childdoctmp\childdocname
        \def\childdoctmp{}
      \else
        \def\childdoctmp
        {
          \childdoctrue
          \includeonly{\childdocname}
          \def\childdocjob{#1}
          \def\jobname{#1}
        }
      \fi
      \expandafter
    \endgroup
    \childdoctmp
  \fi
}
%    \end{macrocode}

% \macro{\childdocof}
% The command |\childdocof| redirects
% compilation to the main file |#1|.
%    \begin{macrocode}
\newcommand{\childdocof}[1]
{
  \childdocdisable
  \childdoctrue
  \includeonly{\childdocname}
  \def\jobname{#1}
  \def\childdocjob{#1}
  \input{#1}
}
%    \end{macrocode}

% \macro{\childdocby}
% The command |\childdocby| ....
%    \begin{macrocode}
\newcommand{\childdocby}[2][]
{
  \childdocdisable
  \childdoctrue
  \childdocmanualtrue
  \if?#1?\else
    \def\jobname{#2}
  \fi
  \def\childdocjob{#2}
  \input{#2}
  \endinput
}
%    \end{macrocode}

% \macro{\childdocforward}
% The command |\childdocforward| redirects
% compilation to the main file or
% (if the optional argument is given) a child file.
% Parameters are set as if the main file
% or a child file starting with |\childdocof| was compiled.
% Then compilation is handed over to the main file:
%    \begin{macrocode}
\newcommand{\childdocforward}[2][]
{
  \begingroup
    \if?#1?
      \def\childdoctmp
      {
        \def\childdocname{#2}
        \def\childdocjob{#2}
        \def\jobname{#2}
        \input{#2}
        \endinput
      }
    \else
      \def\childdoctmp
      {
        \childdocdisable
        \def\childdocname{#2}
        \childdoctrue
        \includeonly{#2}
        \def\childdocjob{#1}
        \def\jobname{#1}
        \input{#1}
        \endinput
      }
    \fi
    \expandafter
  \endgroup
  \childdoctmp
}
%    \end{macrocode}

% \macro{\childdocforwardprefix}
% The command |\childdocforwardprefix| redirects
% compilation to the main or a child file by means of a pattern.
% The prefix |#1| in the current filename is replaced by |#2|
% and the suffix of the current filename is kept
% (it is assumed that the filename does not contain the substring `|~~~|'
% which is used as a delimiter).
% Compilation is handed over to the new file by |\childdocforward|:
%    \begin{macrocode}
\newcommand{\childdocforwardprefix}[3][]
{
  \begingroup
    \def\childdocextract #2##1~~~{\def\childdoctmp{\childdocforward[#1]{#3##1}}}
    \expandafter\childdocextract\childdocname~~~
    \expandafter
  \endgroup
  \childdoctmp
}
%    \end{macrocode}

% \macro{\childdoc}
% The deprecated macro |\childdoc| is a legacy version of |\childdocmain|:
%    \begin{macrocode}
\newcommand{\childdoc}{\childdocmain}
%    \end{macrocode}

% \macro{\childdocredirect}
% The deprecated macro |\childdocredirect| is a legacy version
% of |\childdocforward| and |\childdocforwardprefix|:
%    \begin{macrocode}
\newcommand{\childdocredirect}[2][]
{
  \begingroup
    \if?#1?
      \def\childdoctmp{\childdocforward{#2}}
    \else
      \def\childdoctmp{\childdocforwardprefix{#1}{#2}}
    \fi
    \expandafter
  \endgroup
  \childdoctmp
}
%    \end{macrocode}

%\iffalse
%</package>
%\fi
%
\endinput
\childdocforward[|\textit{main}|]{|\textit{dest}|}"|
\end{center}
%
Here \textit{target} is the name of the output file,
\textit{main} is the name of the main file
and \textit{dest} is the name of the main or child file to be processed
(all filenames without extensions).
The optional argument \textit{main} can be omitted
if \textit{main} matches \textit{dest}.
Optionally, compilation \textit{flags} can be defined via |\def| commands.
This command line makes the \TeX{} engine believe
it is compiling the file \textit{target}
whose content is specified as the latter parameter.
The provided code then forwards the processing to
\textit{main} or \textit{dest} as described in \secref{sec:forward}.

%%%%%%%%%%%%%%%%%%%%%%%%%%%%%%%%%%%%%%%%%%%%%%%%%%%%%%%%%%%%%%%%%%%%%%%%%%%%%%%%
\subsection{Include by Input}
\label{sec:input}

Including child documents by |\include| has some restrictions by design.
Most notably, the content of a child document always occupies
its own set of pages; pages cannot be shared between child documents.
Usually, this behaviour makes perfect sense
because each child document contain an essential part of the document.
However, in some situations it may be desirable to compose
a document from a collection of parts
without having mandatory page breaks between then.
For this case, the package
provides a mechanism to include parts
by |\input| which can also be processed individually.
However, by construction this mechanism
requires manual handling of the content to be output.

%%%%%%%%%%%%%%%%%%%%%%%%%%%%%%%%%%%%%%%%
\DescribeMacro{\ifchilddocmanual}
The main file should be prepared as usual, see \secref{sec:include}.
However, the document body must make a distinction
between processing of an individual part and of the main document, e.g.:
%
\begin{center}
\begin{tabular}{l}
|\ifchilddocmanual|\\
|\input{\childdocname}|\\
|\||else|\\
\textit{document body with }|\input{|\textit{part}|}|\\
|\||fi|
\end{tabular}
\end{center}
%
The conditional |\ifchilddocmanual| is true whenever
a part to be included by |\input| is being compiled,
and the name of the part is stored in |\childdocname|.

%%%%%%%%%%%%%%%%%%%%%%%%%%%%%%%%%%%%%%%%
\DescribeMacro{\childdocby}
Each part to be included by |\input| should start with:
%
\begin{center}
\begin{tabular}{l}
|% \iffalse
%
% childdoc.dtx Copyright (C) 2017-2018 Niklas Beisert
%
% This work may be distributed and/or modified under the
% conditions of the LaTeX Project Public License, either version 1.3
% of this license or (at your option) any later version.
% The latest version of this license is in
%   http://www.latex-project.org/lppl.txt
% and version 1.3 or later is part of all distributions of LaTeX
% version 2005/12/01 or later.
%
% This work has the LPPL maintenance status `maintained'.
%
% The Current Maintainer of this work is Niklas Beisert.
%
% This work consists of the files childdoc.dtx and childdoc.ins
% and the derived files childdoc.def and cdocsamp.tex with
% cdocsch1.tex, cdocsch2.tex, cdocsdrf.tex, cdocsfn1.tex, cdocsfn2.tex.
%
%<package>\ifdefined\childdocmain\endinput\fi
%<package>\ProvidesFile{childdoc.def}[2018/12/30 v2.0 child document driver]
%<samplemain>\ProvidesFile{cdocsamp.tex}[2018/12/30 v2.0 sample for childdoc]
%<*driver>
%\ProvidesFile{childdoc.drv}[2018/12/30 v2.0 childdoc reference manual file]
\PassOptionsToClass{10pt,a4paper}{article}
\documentclass{ltxdoc}

\usepackage[margin=35mm]{geometry}
\usepackage{hyperref}
\usepackage{hyperxmp}
\usepackage[usenames]{color}

\hypersetup{colorlinks=true}
\hypersetup{pdfstartview=FitH}
\hypersetup{pdfpagemode=UseNone}
\hypersetup{pdfsource={}}
\hypersetup{pdflang={en-UK}}
\hypersetup{pdfcopyright={Copyright 2017-2018 Niklas Beisert.
  This work may be distributed and/or modified under the
  conditions of the LaTeX Project Public License, either version 1.3
  of this license or (at your option) any later version.}}
\hypersetup{pdflicenseurl={http://www.latex-project.org/lppl.txt}}
\hypersetup{pdfcontactaddress={ETH Zurich, ITP, HIT K,
  Wolfgang-Pauli-Strasse 27}}
\hypersetup{pdfcontactpostcode={8093}}
\hypersetup{pdfcontactcity={Zurich}}
\hypersetup{pdfcontactcountry={Switzerland}}
\hypersetup{pdfcontactemail={nbeisert@itp.phys.ethz.ch}}
\hypersetup{pdfcontacturl={http://people.phys.ethz.ch/\xmptilde nbeisert/}}

\newcommand{\secref}[1]{\hyperref[#1]{section \ref*{#1}}}

\parskip1ex
\parindent0pt
\let\olditemize\itemize
\def\itemize{\olditemize\parskip0pt}

\begin{document}

\title{The \textsf{childdoc} Package}
\hypersetup{pdftitle={The childdoc Package}}
\author{Niklas Beisert\\[2ex]
  Institut f\"ur Theoretische Physik\\
  Eidgen\"ossische Technische Hochschule Z\"urich\\
  Wolfgang-Pauli-Strasse 27, 8093 Z\"urich, Switzerland\\[1ex]
  \href{mailto:nbeisert@itp.phys.ethz.ch}
  {\texttt{nbeisert@itp.phys.ethz.ch}}}
\hypersetup{pdfauthor={Niklas Beisert}}
\hypersetup{pdfsubject={Manual for the LaTeX2e Package childdoc}}
\date{30 December 2018, \textsf{v2.0}}
\maketitle

\begin{abstract}\noindent
\textsf{childdoc} is a \LaTeXe{} package
that enables the direct compilation
of document sections included by |\include|
to individual files.
\end{abstract}

\begingroup
\parskip0ex
\tableofcontents
\endgroup

%%%%%%%%%%%%%%%%%%%%%%%%%%%%%%%%%%%%%%%%%%%%%%%%%%%%%%%%%%%%%%%%%%%%%%%%%%%%%%%%
%%%%%%%%%%%%%%%%%%%%%%%%%%%%%%%%%%%%%%%%%%%%%%%%%%%%%%%%%%%%%%%%%%%%%%%%%%%%%%%%
\section{Introduction}

\LaTeX{} provides a mechanism to structure a large document (such as a book)
into a main file and several child files (containing the chapters)
using the |\include| command.
This mechanism is beneficial for documents
which span hundreds of pages in order to
make the source file(s) more manageable.
Moreover, compilation can be restricted to
selected child files by means of the |\includeonly| command.
The latter feature can be used to reduce the compilation time while editing
(this was significantly more useful in the earlier days of \LaTeX{})
or to generate a smaller document which is easier to navigate.
Another application of |\includeonly| is to generate
documents consisting of selected parts of the complete document.

However, there are a few drawbacks of the plain |\include| mechanism:
\begin{itemize}
\item
The child files cannot be compiled on their own,
they can only be compiled via the main file.
A naive editing environment
(such as a text editor with an option
to have the current file processed by \LaTeX)
may require one to switch to the main file before compiling;
attempting to compile the child file produces errors.
\item
The main file must be modified (each time)
to adjust the |\includeonly| command
to the present needs. This easily leaves the main file in a messy state.
\item
The generated document will always carry the filename
of the main document. This is inconvenient if
several child files are to be compiled and
to be kept for distribution.
\end{itemize}

The present package provides a simple interface
to make child files individually compilable by \LaTeX{}.
Compiling a child file then has the same effect as compiling
the main file with an |\includeonly| command
to select the appropriate child.
Moreover the generated document will carry the name of the child
rather than the main file.
This resolves all three above issues.

This feature is meant to make the editing of books,
thesis documents and lecture notes somewhat more convenient.
However, the package can also be used efficiently for
composing a series of documents (such as exercise sheets)
which are typically distributed individually.
It then assists the author in generating the individual documents
(potentially in different versions)
as well as a document containing the collected series.
Another application is in developing style files
or other kinds of included material
where compilation of the style file could redirect
to a sample or test file.

%%%%%%%%%%%%%%%%%%%%%%%%%%%%%%%%%%%%%%%%%%%%%%%%%%%%%%%%%%%%%%%%%%%%%%%%%%%%%%%%
%%%%%%%%%%%%%%%%%%%%%%%%%%%%%%%%%%%%%%%%%%%%%%%%%%%%%%%%%%%%%%%%%%%%%%%%%%%%%%%%
\section{Usage}

First of all, the package \textsf{childdoc} is \emph{not} a standard
\LaTeXe{} |.sty| style file! Therefore it needs to be invoked in
a non-standard way.

%%%%%%%%%%%%%%%%%%%%%%%%%%%%%%%%%%%%%%%%%%%%%%%%%%%%%%%%%%%%%%%%%%%%%%%%%%%%%%%%
\subsection{Included Files}
\label{sec:include}

%%%%%%%%%%%%%%%%%%%%%%%%%%%%%%%%%%%%%%%%
\DescribeMacro{\childdocmain}
To use the package, add the commands
\begin{center}
\begin{tabular}{l}
|\input{childdoc.def}|\\
|\childdocmain{}|\\
\end{tabular}
\end{center}
at the very top of the main \LaTeX{} file,
in particular \emph{before} the |\documentclass| statement!
The argument of |\childdocmain| should be left empty
(but it must be present).

%%%%%%%%%%%%%%%%%%%%%%%%%%%%%%%%%%%%%%%%
\DescribeMacro{\childdocof}
Furthermore, add the commands
\begin{center}
\begin{tabular}{l}
|\input{childdoc.def}|\\
|\childdocof{|\textit{main}|}|\\
\end{tabular}
\end{center}
at the top of every child file \textit{child}
which is included by |\include{|\textit{child}|}|
from within the main file
(or at least for those files to be compiled individually).
The argument \textit{main} must be the filename of the main file.

There are a couple of
considerations in setting up the main and child documents:

%%%%%%%%%%%%%%%%%%%%%%%%%%%%%%%%%%%%%%%%
\paragraph{Restrictions.}

Please note the following restrictions:
\begin{itemize}
\item
|\childdocmain| must be called with one argument \textit{main}
to ensure compatibility with earlier version of the package.
It must either be empty (|\childdocmain{}|)
or precisely match the filename of the main file in which it is specified.
See \secref{sec:detection} for further information.
\item
The filename \textit{main} must be specified without the |.tex| extension.
\item
The filename \textit{main} is case sensitive
(even in case-insensitive file systems)
due to internal string comparison.
\item
The argument \textit{main} should be fully expanded, it cannot be a macro.
\item
Subdirectories and special characters should be avoided in filenames.
\item
The command |\childdocmain{|\textit{main}|}| must be followed by a whitespace.
It should not be followed immediately by another command
or by a comment mark `|%|'.
This is because the \TeX{} parser reads the token immediately following
the argument of |\childdocmain| and puts it
at the beginning of every child section;
however, a white\-space is ignored.
\end{itemize}

%%%%%%%%%%%%%%%%%%%%%%%%%%%%%%%%%%%%%%%%
\paragraph{Content of Main File.}

It is advisable to place all content in the child files included by |\include|.
Any output contained in the main file will appear in all child documents
unless suppressed manually;
it cannot be suppressed automatically by the |\includeonly| directive
and thus should normally be avoided.
A method to include some content in the main file
by means of conditional processing is described in \secref{sec:conditional}.

%%%%%%%%%%%%%%%%%%%%%%%%%%%%%%%%%%%%%%%%
\paragraph{Page Numbering.}

When only a part of the document is compiled,
the appropriate numbering of pages
(as well as other status parameters)
is determined from the |.aux| files.
The latter contain information from previous passes.
However this information needs to propagate through
all intermediate child documents.
Therefore the page numbering in child documents may well
be inconsistent until the complete document is compiled at least once.

A useful (if unconventional) way to always ensure a consistent
page numbering is to restart the numbering in each child document
and denote the pages by `\textit{child}|.|\textit{page}'
where \textit{child} represents the chapter/section number of the child file.
This can be achieved by the command
|\numberwithin{page}{|\textit{child}|}|
of the \textsf{amsmath} package
where \textit{child} can be |chapter| or |section|
depending on the chosen structuring.
Alternatively, one can modify the macro |\thepage| appropriately
and reset the counter |page| at the start of each child file.

%%%%%%%%%%%%%%%%%%%%%%%%%%%%%%%%%%%%%%%%%%%%%%%%%%%%%%%%%%%%%%%%%%%%%%%%%%%%%%%%
\subsection{Conditional Processing}
\label{sec:conditional}

The package provides a mechanism to compile different versions
of a document. To customise the versions further some conditional processing
can come in handy to distinguish which version is being compiled.
The package provides two macros to describe the compilation context:

%%%%%%%%%%%%%%%%%%%%%%%%%%%%%%%%%%%%%%%%
\DescribeMacro{\ifchilddoc}
The conditional |\ifchilddoc| distinguishes between the compilation of
child documents and the main document:
%
\begin{center}
|\ifchilddoc |\textit{child-code}| |[|\||else |\textit{main-code}]| \||fi|
\end{center}

%%%%%%%%%%%%%%%%%%%%%%%%%%%%%%%%%%%%%%%%
\DescribeMacro{\childdocname}
\DescribeMacro{\childdocjob}
The macro |\childdocname| contains the filename (without extension)
of the main or child file being processed.
Note that |\childdocjob| will always contain the name of the main file.

%%%%%%%%%%%%%%%%%%%%%%%%%%%%%%%%%%%%%%%%
\paragraph{Title Page.}

Conditional processing can be used to include a title or banner page
in the main document when proper precautions are taken.
Importantly, the code in the main file should ensure that the page counter
(as well as other status parameters which are stored in the |.aux| files)
takes the same value after the conditional processing.
Otherwise the page numbers may take divergent values
depending on which part is compiled.

For example, a title page could be declared by:
%
\begin{center}
\begin{tabular}{l}
|\ifchilddoc\||else|\\
|\addtocounter{page}{-1}|\\
\textit{code for title page}\\
|\newpage|\\
|\||fi|
\end{tabular}
\end{center}
%
A banner page for the child documents can be generated by:
%
\begin{center}
\begin{tabular}{l}
|\ifchilddoc|\\
|\addtocounter{page}{-1}|\\
\textit{code for banner page}\\
|\newpage|\\
|\||fi|
\end{tabular}
\end{center}
%
Here one could write a message such as:
\begin{center}
|This is the part \childdocname{} of \childdocjob{}.|
\end{center}

%%%%%%%%%%%%%%%%%%%%%%%%%%%%%%%%%%%%%%%%%%%%%%%%%%%%%%%%%%%%%%%%%%%%%%%%%%%%%%%%
\subsection{Flags}
\label{sec:flags}

The package makes it easy to generate different versions
of the main or child documents.
To this end compilation flags can be defined
and assigned different default values.
They will be particularly useful in conjunction
with the forwarding mechanism described in \secref{sec:forward}.

For example, it may be useful to have a flag |\version|
which can be set to |draft| or |final|.
The document source will contain some conditional code
depending on the value of |\version|.
Suppose further, the flag should default to |final| for the main file
and to |draft| for child files
which is a natural assignment for editing the document.
This is achieved by placing the following code
in the preamble of the main document
(below the |\childdocmain| directive):
%
\begin{center}
\begin{tabular}{l}
|\ifchilddoc|\\
|\providecommand{\version}{draft}|\\
|\||else|\\
|\providecommand{\version}{final}|\\
|\||fi|
\end{tabular}
\end{center}
%
The definition by |\providecommand| makes sure
that previous definitions are not overwritten.
Further statements |\providecommand{\version}{...}|
can thus be added before the above code to override it.

For the main file, one might add a line
(between |\childdocmain| and the above block)
%
\begin{center}
|%\ifchilddoc\||else\providecommand{\version}{draft}\||fi|
\end{center}
%
which can be uncommented to produce a draft version.
Likewise one can add a line to the very top of a child file
(above the |\childdocof{|\textit{main}|}| directive)
%
\begin{center}
|%\providecommand{\version}{final}|
\end{center}
%
which can be uncommented to produce the final version of this child document.

%%%%%%%%%%%%%%%%%%%%%%%%%%%%%%%%%%%%%%%%%%%%%%%%%%%%%%%%%%%%%%%%%%%%%%%%%%%%%%%%
\subsection{Forwarding}
\label{sec:forward}

Different versions of the main or child documents
using compilation flags as described in \secref{sec:flags}
can be (permanently) stored in different files
for convenient compilation, viewing and distribution.
To this end, the package defines a command
to pass on compilation to a different file:

%%%%%%%%%%%%%%%%%%%%%%%%%%%%%%%%%%%%%%%%
\DescribeMacro{\childdocforward}
The command |\childdocforward| redirects processing to
another source file:
%
\begin{center}
\begin{tabular}{l}
|\input{childdoc.def}|\\
|\childdocforward[|\textit{main}|]{|\textit{dest}|}|\\
\end{tabular}
\end{center}
%
The argument \textit{dest} is the destination file
(without extension).
It should be the main file or one of the child files.
Note that further \textsf{childdoc} directives
such as |\childdocof| and |\childdocforward|
in the indicated file will be processed in this form.
The optional argument \textit{main}
passes on directly to the main file \textit{main}
while pretending to compile the child \textit{dest}.
This form behaves as if \textit{dest}
issues |\childdocof{|\textit{main}|}| right away,
and no further \textsf{childdoc} directives will be processed.

%%%%%%%%%%%%%%%%%%%%%%%%%%%%%%%%%%%%%%%%
\DescribeMacro{\...prefix}
In the alternative form |\childdocforwardprefix|,
%
\begin{center}
\begin{tabular}{l}
|\input{childdoc.def}|\\
|\childdocforwardprefix[|\textit{main}|]{|\textit{prefix}|}{|\textit{dest}|}|
\end{tabular}
\end{center}
%
the destination file is determined by a pattern
depending on the current file:
To make this work, the current file must be called
`{\textit{prefix}\hspace{0.2em}\textit{suffix}}'
with \textit{prefix} matching precisely the argument.
Processing is then passed on to the file
`{\textit{dest}\hspace{0.2em}\textit{suffix}}'.
Surely, the same effect is achieved by
directly specifying the
argument `{\textit{dest}\hspace{0.2em}\textit{suffix}}'
in the first form.
However, that requires to set up a different file
for each child. With the alternative form of the command
all these files can have exactly the same content
which simplifies setting them up and maintaining them.

For example, the following file |draft.tex|
with a compilation flag |\version| as described in \secref{sec:flags}
compiles the main document as a draft:
%
\begin{center}
\begin{tabular}{l}
|\def\version{draft}|\\
|\input{childdoc.def}|\\
|\childdocforward{|\textit{main}|}|
\end{tabular}
\end{center}
%
Likewise, the following files |final|\textit{nn}|.tex|
compile the final version of the child document
|child|\textit{nn}|.tex|:
%
\begin{center}
\begin{tabular}{l}
|\def\version{final}|\\
|\input{childdoc.def}|\\
|\childdocforwardprefix{final}{child}|
\end{tabular}
\end{center}
%

Note that when several versions of a main file and/or of each child file
are to be generated, it may be convenient to set up a |Makefile| or
shell script to automatise the process.

%%%%%%%%%%%%%%%%%%%%%%%%%%%%%%%%%%%%%%%%%%%%%%%%%%%%%%%%%%%%%%%%%%%%%%%%%%%%%%%%
\subsection{Command Line Processing}
\label{sec:commandline}

The effect of redirection files can also be achieved by invoking
the \LaTeX{} compiler with a more elaborate command line.
Most conveniently this should be done as part
of a shell script or a |Makefile|.

When using \textsf{childdoc} in the main file, the following
command lines effectively perform a redirection
(note that depending on the shell being used,
backslashes may have to be doubled: `|\|' $\to$ `|\\|'):
%
\begin{center}
|... -jobname "|\textit{target}|" |\\|"|[\textit{flags}]%
|\input{childdoc.def}\childdocforward[|\textit{main}|]{|\textit{dest}|}"|
\end{center}
%
Here \textit{target} is the name of the output file,
\textit{main} is the name of the main file
and \textit{dest} is the name of the main or child file to be processed
(all filenames without extensions).
The optional argument \textit{main} can be omitted
if \textit{main} matches \textit{dest}.
Optionally, compilation \textit{flags} can be defined via |\def| commands.
This command line makes the \TeX{} engine believe
it is compiling the file \textit{target}
whose content is specified as the latter parameter.
The provided code then forwards the processing to
\textit{main} or \textit{dest} as described in \secref{sec:forward}.

%%%%%%%%%%%%%%%%%%%%%%%%%%%%%%%%%%%%%%%%%%%%%%%%%%%%%%%%%%%%%%%%%%%%%%%%%%%%%%%%
\subsection{Include by Input}
\label{sec:input}

Including child documents by |\include| has some restrictions by design.
Most notably, the content of a child document always occupies
its own set of pages; pages cannot be shared between child documents.
Usually, this behaviour makes perfect sense
because each child document contain an essential part of the document.
However, in some situations it may be desirable to compose
a document from a collection of parts
without having mandatory page breaks between then.
For this case, the package
provides a mechanism to include parts
by |\input| which can also be processed individually.
However, by construction this mechanism
requires manual handling of the content to be output.

%%%%%%%%%%%%%%%%%%%%%%%%%%%%%%%%%%%%%%%%
\DescribeMacro{\ifchilddocmanual}
The main file should be prepared as usual, see \secref{sec:include}.
However, the document body must make a distinction
between processing of an individual part and of the main document, e.g.:
%
\begin{center}
\begin{tabular}{l}
|\ifchilddocmanual|\\
|\input{\childdocname}|\\
|\||else|\\
\textit{document body with }|\input{|\textit{part}|}|\\
|\||fi|
\end{tabular}
\end{center}
%
The conditional |\ifchilddocmanual| is true whenever
a part to be included by |\input| is being compiled,
and the name of the part is stored in |\childdocname|.

%%%%%%%%%%%%%%%%%%%%%%%%%%%%%%%%%%%%%%%%
\DescribeMacro{\childdocby}
Each part to be included by |\input| should start with:
%
\begin{center}
\begin{tabular}{l}
|\input{childdoc.def}|\\
|\childdocby{|\textit{main}|}|\\
\end{tabular}
\end{center}
%
The directive |\childdocby| is similar to |\childdocof|
described in \secref{sec:include},
but the subsequent selection of content must be done manually.
To that end, both |\ifchilddoc| and |\ifchilddocmanual|
will be true upon processing of a part,
and the name of the part is stored in |\childdocname|.
Note that |\jobname| will be set to the filename of the current part
so that each part receives an individual |.aux| file
that does not interfere with the |.aux| file(s) of the main document.
This behaviour can be altered by the alternative form
|\childdocby[*]{|\textit{main}|}| (with a non-empty optional argument)
which uses the |.aux| file of the main document
by setting |\jobname| to \textit{main}.

%%%%%%%%%%%%%%%%%%%%%%%%%%%%%%%%%%%%%%%%%%%%%%%%%%%%%%%%%%%%%%%%%%%%%%%%%%%%%%%%
\subsection{Driver Development}
\label{sec:driver}

The \textsf{childdoc} mechanism can also be use for the development
of definition files such as \LaTeX{} styles or classes.
This case differs from the above setup with multiple parts
included by |\include| in that no |\includeonly| should be invoked.
This can be achieved by starting the include file
(before |\ProvidesPackage|) with:
%
\begin{center}
\begin{tabular}{l}
|\input{childdoc.def}|\\
|\childdocforward{|\textit{main}|}|\\
\end{tabular}
\end{center}
%
or alternatively with:
%
\begin{center}
\begin{tabular}{l}
|\input{childdoc.def}|\\
|\childdocby{|\textit{main}|}|\\
\end{tabular}
\end{center}
%
Both forms have slightly different effects as described above.
The main file is prepared as usual, see \secref{sec:include}.

%%%%%%%%%%%%%%%%%%%%%%%%%%%%%%%%%%%%%%%%%%%%%%%%%%%%%%%%%%%%%%%%%%%%%%%%%%%%%%%%
\subsection{Legacy Detection}
\label{sec:detection}

The directive |\childdocmain| in the main file can detect
whether the complete document or merely a child is to be compiled
even without using the directive |\childdocof|.
This method is deprecated because it is less robust
and there is no compelling reason to use it;
it is merely provided for backward compatibility
and it may be removed in future versions.

If the detection mechanism is to be used,
it is mandatory to correctly specify
the filename of the main file as the argument of |\childdocmain|:
%
\begin{center}
\begin{tabular}{l}
|\input{childdoc.def}|\\
|\childdocmain{|\textit{main}|}|\\
\end{tabular}
\end{center}
%
If |\jobname| does not match the argument \textit{main} of |\childdocmain|,
it is assumed that |\jobname| points to the child file to be compiled.
When using |\childdocmain| with the main file specified as argument,
it suffices to start a child file
with just |\input{|\textit{main}|}|
without loading of the package and using |\childdocof|.
If instead all processing is done
with the appropriate \textsf{childdoc} directives,
the argument of \textit{main} of |\childdocmain| can be empty.

An alternative version of the command line processing described
in \secref{sec:commandline} using the detection mechanism reads:
%
\begin{center}
|... -jobname "|\textit{target}|" "|[\textit{flags}]%
[|\def\jobname{|\textit{dest}|}|]|\input{|\textit{main}|}"|
\end{center}

%%%%%%%%%%%%%%%%%%%%%%%%%%%%%%%%%%%%%%%%%%%%%%%%%%%%%%%%%%%%%%%%%%%%%%%%%%%%%%%%
\subsection{Manual Code}
\label{sec:manual}

In case one cannot be certain whether the definitions file |childdoc.def|
is installed on the target \TeX{} distribution
and one prefers not to ship it,
it is conceivable to paste a few relevant commands into the sources.

To that end, drop all statements |\input{childdoc.def}|
and perform the replacements as outlined below.
Instead of |\childdocmain{|\textit{main}|}| add the following code
to the top of the main file:
%
\begin{center}
\begin{tabular}{l}
|\||ifdefined\childdocname\endinput\||fi\newif\ifchilddoc|\\
|\edef\childdocname{\scantokens\expandafter{\jobname\noexpand}}|\\
|\def\childdocmain{|\textit{main}|}\||ifx\childdocmain\childdocname\||else|\\
|\childdoctrue\includeonly{\childdocname}\let\jobname\childdocmain\||fi|\\
\end{tabular}
\end{center}
%
Instead of |\childdocof{|\textit{main}|}| just include the main file
at the top of each child file:
%
\begin{center}
|\input{|\textit{main}|}|
\end{center}
%
A simple redirection |\childdocforward{|\textit{dest}|}| is achieved by:
%
\begin{center}
|\def\jobname{|\textit{dest}|}\input{\jobname}|
\end{center}
%
The redirection with prefix
|\childdocforwardprefix[|\textit{prefix}|]{|\textit{dest}|}|
is accomplished by:
%
\begin{center}
\begin{tabular}{l}
|{\edef\jobname{\scantokens\expandafter{\jobname\noexpand}}|\\
|\def\redirectjob |\textit{prefix}|#1~~~{\gdef\jobname{|\textit{dest}|#1}}|\\
|\expandafter\redirectjob\jobname~~~}\input{\jobname}|
\end{tabular}
\end{center}

In an alternative approach,
child documents can be compiled by a specific command line
without additional code or specific definitions:
%
\begin{center}
|... -jobname "|\textit{target}|" "|[\textit{flags}]%
|\includeonly{|\textit{dest}|}\input{|\textit{main}|}"|
\end{center}
%

%%%%%%%%%%%%%%%%%%%%%%%%%%%%%%%%%%%%%%%%%%%%%%%%%%%%%%%%%%%%%%%%%%%%%%%%%%%%%%%%
%%%%%%%%%%%%%%%%%%%%%%%%%%%%%%%%%%%%%%%%%%%%%%%%%%%%%%%%%%%%%%%%%%%%%%%%%%%%%%%%
\section{Information}

%%%%%%%%%%%%%%%%%%%%%%%%%%%%%%%%%%%%%%%%%%%%%%%%%%%%%%%%%%%%%%%%%%%%%%%%%%%%%%%%
\subsection{Copyright}

Copyright \copyright{} 2017--2018 Niklas Beisert

This work may be distributed and/or modified under the
conditions of the \LaTeX{} Project Public License, either version 1.3
of this license or (at your option) any later version.
The latest version of this license is in
  \url{http://www.latex-project.org/lppl.txt}
and version 1.3 or later is part of all distributions of \LaTeX{}
version 2005/12/01 or later.

This work has the LPPL maintenance status `maintained'.

The Current Maintainer of this work is Niklas Beisert.

This work consists of the files |README.txt|, |childdoc.ins| and |childdoc.dtx|
as well as the derived files |childdoc.def|, |cdocsamp.tex|
with |cdocsch1.tex|, |cdocsch2.tex|, |cdocspt3.tex|, |cdocspt4.tex|,
|cdocsdrf.tex|, |cdocsfn1.tex|, |cdocsfn2.tex|
as well as |childdoc.pdf|.

%%%%%%%%%%%%%%%%%%%%%%%%%%%%%%%%%%%%%%%%%%%%%%%%%%%%%%%%%%%%%%%%%%%%%%%%%%%%%%%%
\subsection{Files and Installation}

The package consists of the files:
%
\begin{center}
\begin{tabular}{ll}
    |README.txt|   & readme file \\
    |childdoc.ins| & installation file \\
    |childdoc.dtx| & source file \\
    |childdoc.def| & definition file \\
    |cdocsamp.tex| & sample main file \\
    |cdocsch1.tex| & sample include file \\
    |cdocsch2.tex| & sample include file \\
    |cdocspt3.tex| & sample part file \\
    |cdocspt4.tex| & sample part file \\
    |cdocsdrf.tex| & sample redirection file \\
    |cdocsfn1.tex| & sample redirection file \\
    |cdocsfn2.tex| & sample redirection file \\
    |childdoc.pdf| & manual
\end{tabular}
\end{center}
%
The distribution consists of the files
|README.txt|, |childdoc.ins| and |childdoc.dtx|.
%
\begin{itemize}
\item
Run (pdf)\LaTeX{} on |childdoc.dtx|
to compile the manual |childdoc.pdf| (this file).
\item
Run \LaTeX{} on |childdoc.ins| to create the definitions file |childdoc.def|
and the sample |cdocsamp.tex| with include files
|cdocsch1.tex|, |cdocsch2.tex|, |cdocspt3.tex|, |cdocspt4.tex|,
|cdocsdrf.tex|, |cdocsfn1.tex|, |cdocsfn2.tex|.
Then copy the file |childdoc.def| to an appropriate directory of your \LaTeX{}
distribution, e.g.\ \textit{texmf-root}|/tex/latex/childdoc|.
\end{itemize}

%%%%%%%%%%%%%%%%%%%%%%%%%%%%%%%%%%%%%%%%%%%%%%%%%%%%%%%%%%%%%%%%%%%%%%%%%%%%%%%%
\subsection{Related CTAN Packages}

There are several other packages which offer a similar functionality:
%
\begin{itemize}
\item
The packages
\href{http://ctan.org/pkg/docmute}{\textsf{docmute}},
\href{http://ctan.org/pkg/includex}{\textsf{includex}} and
\href{http://ctan.org/pkg/standalone}{\textsf{standalone}}
provide commands to include only the document body of
a child file thus allowing both files to be compiled individually.
\item
The packages \href{http://ctan.org/pkg/subdocs}{\textsf{subdocs}}
and \href{http://ctan.org/pkg/subfiles}{\textsf{subfiles}}
provide structures in which the main and child documents can be
encapsulated and allowing them to be compiled individually.
The inclusion mechanism is different from the conventional |\include|.
\item
The package \href{http://ctan.org/pkg/combine}{\textsf{combine}}
is an elaborate solution to combine several documents into one.
\end{itemize}
%
See also the CTAN topic \href{http://ctan.org/topic/subdocs}{\textsf{subdocs}}
for further related packages.
The present package differs from the above solutions in that
a document structure constructed with the conventional |\include| mechanism
just needs two extra commands at the top of every file
such that all constituent files can be compiled individually.

%%%%%%%%%%%%%%%%%%%%%%%%%%%%%%%%%%%%%%%%%%%%%%%%%%%%%%%%%%%%%%%%%%%%%%%%%%%%%%%%
%\subsection{Feature Suggestions}
%
%The following is a list of features which may be useful for future
%versions of this package:
%%
%\begin{itemize}
%\item
%\ldots
%\end{itemize}

%%%%%%%%%%%%%%%%%%%%%%%%%%%%%%%%%%%%%%%%%%%%%%%%%%%%%%%%%%%%%%%%%%%%%%%%%%%%%%%%
\subsection{Revision History}

%%%%%%%%%%%%%%%%%%%%%%%%%%%%%%%%%%%%%%%%
\paragraph{v2.0:} 2018/12/30

\begin{itemize}
\item
immediate forward processing
\item
added |\childdocby| mechanism
\item
manual restructured
\end{itemize}

%%%%%%%%%%%%%%%%%%%%%%%%%%%%%%%%%%%%%%%%
\paragraph{v1.6:} 2018/01/17

\begin{itemize}
\item
application for development of include files
\item
corrections to manual
\end{itemize}

%%%%%%%%%%%%%%%%%%%%%%%%%%%%%%%%%%%%%%%%
\paragraph{v1.5:} 2017/05/21

\begin{itemize}
\item
more complete structuring introduced
\item
|\childdocof| introduced
\item
|\childdoc| renamed to |\childdocmain|
\item
|\childredirect| renamed to |\childdocforward| and |\childdocforwardprefix|
and functionality expanded
\end{itemize}

%%%%%%%%%%%%%%%%%%%%%%%%%%%%%%%%%%%%%%%%
\paragraph{v1.0:} 2017/04/27

\begin{itemize}
\item
manual and install package
\item
first version published on CTAN
\end{itemize}

%%%%%%%%%%%%%%%%%%%%%%%%%%%%%%%%%%%%%%%%
\paragraph{v0.6:} 2017/04/26

\begin{itemize}
\item
redirection mechanism added
\end{itemize}

%%%%%%%%%%%%%%%%%%%%%%%%%%%%%%%%%%%%%%%%
\paragraph{v0.5:} 2017/04/26

\begin{itemize}
\item
functionality in definition file
\end{itemize}


%%%%%%%%%%%%%%%%%%%%%%%%%%%%%%%%%%%%%%%%%%%%%%%%%%%%%%%%%%%%%%%%%%%%%%%%%%%%%%%%
%%%%%%%%%%%%%%%%%%%%%%%%%%%%%%%%%%%%%%%%%%%%%%%%%%%%%%%%%%%%%%%%%%%%%%%%%%%%%%%%
%%%%%%%%%%%%%%%%%%%%%%%%%%%%%%%%%%%%%%%%%%%%%%%%%%%%%%%%%%%%%%%%%%%%%%%%%%%%%%%%
\appendix

\settowidth\MacroIndent{\rmfamily\scriptsize 000\ }

 \DocInput{childdoc.dtx}

\end{document}
%</driver>
% \fi
%
% %%%%%%%%%%%%%%%%%%%%%%%%%%%%%%%%%%%%%%%%%%%%%%%%%%%%%%%%%%%%%%%%%%%%%%%%%%%%%%
% %%%%%%%%%%%%%%%%%%%%%%%%%%%%%%%%%%%%%%%%%%%%%%%%%%%%%%%%%%%%%%%%%%%%%%%%%%%%%%
% \section{Sample}
%\iffalse
%<*samplemain>
%\fi
%
% The following presents a sample document
% with two chapters, two parts, a title page,
% a compile flag as well as three forwarding files to set the flag.
% It consists of eight |.tex| files:
% \begin{center}
% \begin{tabular}{ll}
% |cdocsamp.tex|&main file\\
% |cdocsch1.tex|&include file for chapter 1\\
% |cdocsch2.tex|&include file for chapter 2\\
% |cdocspt3.tex|&include file for part 3\\
% |cdocspt4.tex|&include file for part 4\\
% |cdocsdrf.tex|&forwarding file for main file in draft mode\\
% |cdocsfi1.tex|&forwarding file for final version of chapter 1\\
% |cdocsfi2.tex|&forwarding file for final version of chapter 2\\
% \end{tabular}
% \end{center}
% Each of the eight files can be compiled directly by the \LaTeX{} compiler.
%
% %%%%%%%%%%%%%%%%%%%%%%%%%%%%%%%%%%%%%%
% \paragraph{Main File.}
%
% The main file is called |cdocsamp.tex|.
%
% Load the \textsf{childdoc} definitions and
% declare the filename for the main document:
%    \begin{macrocode}
\input{childdoc.def}
\childdocmain{}
%    \end{macrocode}

% Optional override for |\version| flag:
%    \begin{macrocode}
%%\ifchilddoc\else\providecommand{\version}{draft}\fi
%    \end{macrocode}

% Define the default values for the |\version| flag
% (|final| for the main file and |draft| for childs):
%    \begin{macrocode}
\ifchilddoc
\providecommand{\version}{draft}
\else
\providecommand{\version}{final}
\fi
%    \end{macrocode}

% Load the standard document class:
%    \begin{macrocode}
\documentclass[12pt]{article}
%    \end{macrocode}

% Start the document body:
%    \begin{macrocode}
\begin{document}
%    \end{macrocode}

% Declare a title page.
% Print title, part of document being processed and version flag:
%    \begin{macrocode}
\addtocounter{page}{-1}
\begin{center}
{\LARGE\bfseries{}childdoc example\par}
\vspace{1cm}
\ifchilddoc
\ifchilddocmanual part\else chapter\fi:
`\childdocname' of `\childdocjob'\par
\else
main document: `\childdocjob'\par
\fi
version: \version\par
\end{center}
\newpage
%    \end{macrocode}

% Manually include selected file,
% otherwise process as usual:
%    \begin{macrocode}
\ifchilddocmanual
\section*{part `\childdocname'}
\input{\childdocname}
\else
%    \end{macrocode}

% Include the two chapters:
%    \begin{macrocode}
\include{cdocsch1}
\include{cdocsch2}
%    \end{macrocode}

% Include the two parts unless only chapters should be displayed:
%    \begin{macrocode}
\ifchilddoc\else
\section{part three}
\input{cdocspt3}
\section{part four}
\input{cdocspt4}
\fi
%    \end{macrocode}

% Process as usual until here:
%    \begin{macrocode}
\fi
%    \end{macrocode}

% End of document body:
%    \begin{macrocode}
\end{document}
%    \end{macrocode}
%\iffalse
%</samplemain>
%\fi
%
% %%%%%%%%%%%%%%%%%%%%%%%%%%%%%%%%%%%%%%
% \paragraph{Chapter Include Files.}
%
% The include files are called |cdocsch1.tex| and |cdocsch2.tex|.
%
%\iffalse
%<*samplechap1|samplechap2>
%\fi

% Optional override for |\version| flag:
%    \begin{macrocode}
%%\providecommand{\version}{final}
%    \end{macrocode}

% Include the main document:
%    \begin{macrocode}
\input{childdoc.def}
\childdocof{cdocsamp}
%    \end{macrocode}

%\iffalse
%</samplechap1|samplechap2>
%\fi
%
%\iffalse
%<*samplechap1>
%\fi
% Some text for chapter 1:
%    \begin{macrocode}
\section{one}
some text in chapter one
%    \end{macrocode}

%\iffalse
%</samplechap1>
%\fi
% Some text for chapter 2:
%\iffalse
%<*samplechap2>
%\fi
%    \begin{macrocode}
\section{two}
more text in chapter two
%    \end{macrocode}

%\iffalse
%</samplechap2>
%\fi
%
% %%%%%%%%%%%%%%%%%%%%%%%%%%%%%%%%%%%%%%
% \paragraph{Part Include Files.}
%
% The include files are called |cdocspt3.tex| and |cdocspt4.tex|.
%
%\iffalse
%<*samplepart3|samplepart4>
%\fi

% Optional override for |\version| flag:
%    \begin{macrocode}
%%\providecommand{\version}{final}
%    \end{macrocode}

% Include the main document:
%    \begin{macrocode}
\input{childdoc.def}
\childdocby{cdocsamp}
%    \end{macrocode}

%\iffalse
%</samplepart3|samplepart4>
%\fi
%
%\iffalse
%<*samplepart3>
%\fi
% Some text for part 3:
%    \begin{macrocode}
some text in part three
%    \end{macrocode}

%\iffalse
%</samplepart3>
%\fi
% Some text for part 4:
%\iffalse
%<*samplepart4>
%\fi
%    \begin{macrocode}
more text in part four
%    \end{macrocode}

%\iffalse
%</samplepart4>
%\fi
%
% %%%%%%%%%%%%%%%%%%%%%%%%%%%%%%%%%%%%%%
% \paragraph{Forwarding for a Complete Draft.}
%
% The following forwarding file |cdocsdrf.tex|
% compiles the main document in draft mode:
%\iffalse
%<*sampledraft>
%\fi
%    \begin{macrocode}
\def\version{draft}
\input{childdoc.def}
\childdocforward{cdocsamp}
%    \end{macrocode}

%\iffalse
%</sampledraft>
%\fi
%
% %%%%%%%%%%%%%%%%%%%%%%%%%%%%%%%%%%%%%%
% \paragraph{Forwarding for Final Version of the Chapters.}
%
% The following forwarding files |cdocsfn1.tex| and |cdocsfn2.tex|
% (with identical content)
% compile the final versions of the child documents
% |cdocsch1.tex| and |cdocsch2.tex|, respectively:
%\iffalse
%<*samplefinal>
%\fi
%    \begin{macrocode}
\def\version{final}
\input{childdoc.def}
\childdocforwardprefix[cdocsamp]{cdocsfn}{cdocsch}
%    \end{macrocode}

%\iffalse
%</samplefinal>
%\fi
%
% %%%%%%%%%%%%%%%%%%%%%%%%%%%%%%%%%%%%%%
% \paragraph{Command Line Processing.}
%
% The following three command lines generate the output files
% |cdocscld|, |cdocscl1| and |cdocscl2|
% which should be identical to
% |cdocsdrf|, |cdocsch1| and |cdocsfn2|, respectively:
% \begin{center}
% \begin{tabular}{l}
% |latex -jobname cdocscld \|\\
% |  "\def\version{draft}\input{childdoc.def}\childdocforward{cdocsamp}"|\\
% |latex -jobname cdocscl1 \|\\
% |  "\input{childdoc.def}\childdocforward[cdocsamp]{cdocsch1}"|\\
% |latex -jobname cdocscl2 \|\\
% |  "\def\version{final}\input{childdoc.def}\childdocforward{cdocsch2}"|
% \end{tabular}
% \end{center}
% Note that the trailing backslash on each first line
% merely continues the input to the second line
% (for convenient cut ant paste).
% Furthermore, the command |latex| can be replaced by any
% of its alternative versions such as |pdflatex|.
%
% %%%%%%%%%%%%%%%%%%%%%%%%%%%%%%%%%%%%%%%%%%%%%%%%%%%%%%%%%%%%%%%%%%%%%%%%%%%%%%
% %%%%%%%%%%%%%%%%%%%%%%%%%%%%%%%%%%%%%%%%%%%%%%%%%%%%%%%%%%%%%%%%%%%%%%%%%%%%%%
% \section{Implementation}
%\iffalse
%<*package>
%\fi
%
% This section describes the definitions file |childdoc.def|.

% The definitions cannot be loaded using |\usepackage| or |\RequirePackage|
% which has a mechanism to prevent loading a style file more than once.
% When loading the definitions by means of |\input|
% multiple instances have to be prevented manually:
%\iffalse
%This code needs to be before the `\ProvidesFile' directive
%which is defined at the beginning of this file.
%Therefore it is also placed there and commented out here.
%</package>
%<*discard>
%\fi
%    \begin{macrocode}
\ifdefined\childdocmain\endinput\fi
%    \end{macrocode}
%\iffalse
%</discard>
%<*package>
%\fi
%
% \macro{\ifchilddoc}
% \macro{\ifchilddocmanual}
% The conditional |\ifchilddoc| tells whether a
% child (true) or main (false) document is being compiled.
% The conditional |\ifchilddocmanual| tells whether
% the |\includeonly| mechanism is used (false) or
% the selection of child files must be performed manually (true).
% The definitions initialise to false:
%    \begin{macrocode}
\newif\ifchilddoc
\newif\ifchilddocmanual
%    \end{macrocode}

% \macro{\childdocname}
% \macro{\childdocjob}
% The macro |\childdocname| stores the name of the main document
% to be compiled. The macro |\childdocjob| stores the name of
% the document on which the \LaTeX{} compiler was originally invoked.
% The content of |\jobname| cannot be compared
% to filenames specified in the source due to different catcodes.
% The following code rescans |\jobname|, stores the result
% in |\childdocname| and saves a copy in |\childdocjob|:
%    \begin{macrocode}
\edef\childdocname{\scantokens\expandafter{\jobname\noexpand}}
\let\childdocjob\childdocname
%    \end{macrocode}

% \macro{\childdocdisable}
% The macro |\childdocdisable| prevents the main file
% from being processed more than once.
% At this stage, the main document command |\childdocmain|
% is assumed to be called once again where it should do nothing.
% Any subsequent call to it should prevent
% a secondary processing of the main document
% It overwrites the forwarding commands
% |\childdocof| and |\childdocforward|
% with empty macros to prevent further inclusions of the main document:
%    \begin{macrocode}
\newcommand{\childdocdisable}
{
  \renewcommand{\childdocmain}[1]{\renewcommand{\childdocmain}[1]{\endinput}}
  \renewcommand{\childdocof}[1]{}
  \renewcommand{\childdocby}[2][]{}
  \renewcommand{\childdocforward}[2][]{}
  \renewcommand{\childdocdisable}{}
}
%    \end{macrocode}

% \macro{\childdocmain}
% The macro |\childdocmain| is to be called at the top of the main file
% with nothing or the main filename (without extension) as argument.
% First, it breaks loops.
% If the argument is not empty and does not match |\childdocname|
% (which is set by the first inclusion of |childdoc.def|),
% |\ifchilddoc| is set to true, |\includeonly| is applied to the child file
% and |\jobname| is set to the main file
% (for proper handling of |.aux| files):
%    \begin{macrocode}
\newcommand{\childdocmain}[1]
{
  \childdocdisable\childdocmain{}
  \if?#1?\else
    \begingroup
      \def\childdoctmp{#1}
      \ifx\childdoctmp\childdocname
        \def\childdoctmp{}
      \else
        \def\childdoctmp
        {
          \childdoctrue
          \includeonly{\childdocname}
          \def\childdocjob{#1}
          \def\jobname{#1}
        }
      \fi
      \expandafter
    \endgroup
    \childdoctmp
  \fi
}
%    \end{macrocode}

% \macro{\childdocof}
% The command |\childdocof| redirects
% compilation to the main file |#1|.
%    \begin{macrocode}
\newcommand{\childdocof}[1]
{
  \childdocdisable
  \childdoctrue
  \includeonly{\childdocname}
  \def\jobname{#1}
  \def\childdocjob{#1}
  \input{#1}
}
%    \end{macrocode}

% \macro{\childdocby}
% The command |\childdocby| ....
%    \begin{macrocode}
\newcommand{\childdocby}[2][]
{
  \childdocdisable
  \childdoctrue
  \childdocmanualtrue
  \if?#1?\else
    \def\jobname{#2}
  \fi
  \def\childdocjob{#2}
  \input{#2}
  \endinput
}
%    \end{macrocode}

% \macro{\childdocforward}
% The command |\childdocforward| redirects
% compilation to the main file or
% (if the optional argument is given) a child file.
% Parameters are set as if the main file
% or a child file starting with |\childdocof| was compiled.
% Then compilation is handed over to the main file:
%    \begin{macrocode}
\newcommand{\childdocforward}[2][]
{
  \begingroup
    \if?#1?
      \def\childdoctmp
      {
        \def\childdocname{#2}
        \def\childdocjob{#2}
        \def\jobname{#2}
        \input{#2}
        \endinput
      }
    \else
      \def\childdoctmp
      {
        \childdocdisable
        \def\childdocname{#2}
        \childdoctrue
        \includeonly{#2}
        \def\childdocjob{#1}
        \def\jobname{#1}
        \input{#1}
        \endinput
      }
    \fi
    \expandafter
  \endgroup
  \childdoctmp
}
%    \end{macrocode}

% \macro{\childdocforwardprefix}
% The command |\childdocforwardprefix| redirects
% compilation to the main or a child file by means of a pattern.
% The prefix |#1| in the current filename is replaced by |#2|
% and the suffix of the current filename is kept
% (it is assumed that the filename does not contain the substring `|~~~|'
% which is used as a delimiter).
% Compilation is handed over to the new file by |\childdocforward|:
%    \begin{macrocode}
\newcommand{\childdocforwardprefix}[3][]
{
  \begingroup
    \def\childdocextract #2##1~~~{\def\childdoctmp{\childdocforward[#1]{#3##1}}}
    \expandafter\childdocextract\childdocname~~~
    \expandafter
  \endgroup
  \childdoctmp
}
%    \end{macrocode}

% \macro{\childdoc}
% The deprecated macro |\childdoc| is a legacy version of |\childdocmain|:
%    \begin{macrocode}
\newcommand{\childdoc}{\childdocmain}
%    \end{macrocode}

% \macro{\childdocredirect}
% The deprecated macro |\childdocredirect| is a legacy version
% of |\childdocforward| and |\childdocforwardprefix|:
%    \begin{macrocode}
\newcommand{\childdocredirect}[2][]
{
  \begingroup
    \if?#1?
      \def\childdoctmp{\childdocforward{#2}}
    \else
      \def\childdoctmp{\childdocforwardprefix{#1}{#2}}
    \fi
    \expandafter
  \endgroup
  \childdoctmp
}
%    \end{macrocode}

%\iffalse
%</package>
%\fi
%
\endinput
|\\
|\childdocby{|\textit{main}|}|\\
\end{tabular}
\end{center}
%
The directive |\childdocby| is similar to |\childdocof|
described in \secref{sec:include},
but the subsequent selection of content must be done manually.
To that end, both |\ifchilddoc| and |\ifchilddocmanual|
will be true upon processing of a part,
and the name of the part is stored in |\childdocname|.
Note that |\jobname| will be set to the filename of the current part
so that each part receives an individual |.aux| file
that does not interfere with the |.aux| file(s) of the main document.
This behaviour can be altered by the alternative form
|\childdocby[*]{|\textit{main}|}| (with a non-empty optional argument)
which uses the |.aux| file of the main document
by setting |\jobname| to \textit{main}.

%%%%%%%%%%%%%%%%%%%%%%%%%%%%%%%%%%%%%%%%%%%%%%%%%%%%%%%%%%%%%%%%%%%%%%%%%%%%%%%%
\subsection{Driver Development}
\label{sec:driver}

The \textsf{childdoc} mechanism can also be use for the development
of definition files such as \LaTeX{} styles or classes.
This case differs from the above setup with multiple parts
included by |\include| in that no |\includeonly| should be invoked.
This can be achieved by starting the include file
(before |\ProvidesPackage|) with:
%
\begin{center}
\begin{tabular}{l}
|% \iffalse
%
% childdoc.dtx Copyright (C) 2017-2018 Niklas Beisert
%
% This work may be distributed and/or modified under the
% conditions of the LaTeX Project Public License, either version 1.3
% of this license or (at your option) any later version.
% The latest version of this license is in
%   http://www.latex-project.org/lppl.txt
% and version 1.3 or later is part of all distributions of LaTeX
% version 2005/12/01 or later.
%
% This work has the LPPL maintenance status `maintained'.
%
% The Current Maintainer of this work is Niklas Beisert.
%
% This work consists of the files childdoc.dtx and childdoc.ins
% and the derived files childdoc.def and cdocsamp.tex with
% cdocsch1.tex, cdocsch2.tex, cdocsdrf.tex, cdocsfn1.tex, cdocsfn2.tex.
%
%<package>\ifdefined\childdocmain\endinput\fi
%<package>\ProvidesFile{childdoc.def}[2018/12/30 v2.0 child document driver]
%<samplemain>\ProvidesFile{cdocsamp.tex}[2018/12/30 v2.0 sample for childdoc]
%<*driver>
%\ProvidesFile{childdoc.drv}[2018/12/30 v2.0 childdoc reference manual file]
\PassOptionsToClass{10pt,a4paper}{article}
\documentclass{ltxdoc}

\usepackage[margin=35mm]{geometry}
\usepackage{hyperref}
\usepackage{hyperxmp}
\usepackage[usenames]{color}

\hypersetup{colorlinks=true}
\hypersetup{pdfstartview=FitH}
\hypersetup{pdfpagemode=UseNone}
\hypersetup{pdfsource={}}
\hypersetup{pdflang={en-UK}}
\hypersetup{pdfcopyright={Copyright 2017-2018 Niklas Beisert.
  This work may be distributed and/or modified under the
  conditions of the LaTeX Project Public License, either version 1.3
  of this license or (at your option) any later version.}}
\hypersetup{pdflicenseurl={http://www.latex-project.org/lppl.txt}}
\hypersetup{pdfcontactaddress={ETH Zurich, ITP, HIT K,
  Wolfgang-Pauli-Strasse 27}}
\hypersetup{pdfcontactpostcode={8093}}
\hypersetup{pdfcontactcity={Zurich}}
\hypersetup{pdfcontactcountry={Switzerland}}
\hypersetup{pdfcontactemail={nbeisert@itp.phys.ethz.ch}}
\hypersetup{pdfcontacturl={http://people.phys.ethz.ch/\xmptilde nbeisert/}}

\newcommand{\secref}[1]{\hyperref[#1]{section \ref*{#1}}}

\parskip1ex
\parindent0pt
\let\olditemize\itemize
\def\itemize{\olditemize\parskip0pt}

\begin{document}

\title{The \textsf{childdoc} Package}
\hypersetup{pdftitle={The childdoc Package}}
\author{Niklas Beisert\\[2ex]
  Institut f\"ur Theoretische Physik\\
  Eidgen\"ossische Technische Hochschule Z\"urich\\
  Wolfgang-Pauli-Strasse 27, 8093 Z\"urich, Switzerland\\[1ex]
  \href{mailto:nbeisert@itp.phys.ethz.ch}
  {\texttt{nbeisert@itp.phys.ethz.ch}}}
\hypersetup{pdfauthor={Niklas Beisert}}
\hypersetup{pdfsubject={Manual for the LaTeX2e Package childdoc}}
\date{30 December 2018, \textsf{v2.0}}
\maketitle

\begin{abstract}\noindent
\textsf{childdoc} is a \LaTeXe{} package
that enables the direct compilation
of document sections included by |\include|
to individual files.
\end{abstract}

\begingroup
\parskip0ex
\tableofcontents
\endgroup

%%%%%%%%%%%%%%%%%%%%%%%%%%%%%%%%%%%%%%%%%%%%%%%%%%%%%%%%%%%%%%%%%%%%%%%%%%%%%%%%
%%%%%%%%%%%%%%%%%%%%%%%%%%%%%%%%%%%%%%%%%%%%%%%%%%%%%%%%%%%%%%%%%%%%%%%%%%%%%%%%
\section{Introduction}

\LaTeX{} provides a mechanism to structure a large document (such as a book)
into a main file and several child files (containing the chapters)
using the |\include| command.
This mechanism is beneficial for documents
which span hundreds of pages in order to
make the source file(s) more manageable.
Moreover, compilation can be restricted to
selected child files by means of the |\includeonly| command.
The latter feature can be used to reduce the compilation time while editing
(this was significantly more useful in the earlier days of \LaTeX{})
or to generate a smaller document which is easier to navigate.
Another application of |\includeonly| is to generate
documents consisting of selected parts of the complete document.

However, there are a few drawbacks of the plain |\include| mechanism:
\begin{itemize}
\item
The child files cannot be compiled on their own,
they can only be compiled via the main file.
A naive editing environment
(such as a text editor with an option
to have the current file processed by \LaTeX)
may require one to switch to the main file before compiling;
attempting to compile the child file produces errors.
\item
The main file must be modified (each time)
to adjust the |\includeonly| command
to the present needs. This easily leaves the main file in a messy state.
\item
The generated document will always carry the filename
of the main document. This is inconvenient if
several child files are to be compiled and
to be kept for distribution.
\end{itemize}

The present package provides a simple interface
to make child files individually compilable by \LaTeX{}.
Compiling a child file then has the same effect as compiling
the main file with an |\includeonly| command
to select the appropriate child.
Moreover the generated document will carry the name of the child
rather than the main file.
This resolves all three above issues.

This feature is meant to make the editing of books,
thesis documents and lecture notes somewhat more convenient.
However, the package can also be used efficiently for
composing a series of documents (such as exercise sheets)
which are typically distributed individually.
It then assists the author in generating the individual documents
(potentially in different versions)
as well as a document containing the collected series.
Another application is in developing style files
or other kinds of included material
where compilation of the style file could redirect
to a sample or test file.

%%%%%%%%%%%%%%%%%%%%%%%%%%%%%%%%%%%%%%%%%%%%%%%%%%%%%%%%%%%%%%%%%%%%%%%%%%%%%%%%
%%%%%%%%%%%%%%%%%%%%%%%%%%%%%%%%%%%%%%%%%%%%%%%%%%%%%%%%%%%%%%%%%%%%%%%%%%%%%%%%
\section{Usage}

First of all, the package \textsf{childdoc} is \emph{not} a standard
\LaTeXe{} |.sty| style file! Therefore it needs to be invoked in
a non-standard way.

%%%%%%%%%%%%%%%%%%%%%%%%%%%%%%%%%%%%%%%%%%%%%%%%%%%%%%%%%%%%%%%%%%%%%%%%%%%%%%%%
\subsection{Included Files}
\label{sec:include}

%%%%%%%%%%%%%%%%%%%%%%%%%%%%%%%%%%%%%%%%
\DescribeMacro{\childdocmain}
To use the package, add the commands
\begin{center}
\begin{tabular}{l}
|\input{childdoc.def}|\\
|\childdocmain{}|\\
\end{tabular}
\end{center}
at the very top of the main \LaTeX{} file,
in particular \emph{before} the |\documentclass| statement!
The argument of |\childdocmain| should be left empty
(but it must be present).

%%%%%%%%%%%%%%%%%%%%%%%%%%%%%%%%%%%%%%%%
\DescribeMacro{\childdocof}
Furthermore, add the commands
\begin{center}
\begin{tabular}{l}
|\input{childdoc.def}|\\
|\childdocof{|\textit{main}|}|\\
\end{tabular}
\end{center}
at the top of every child file \textit{child}
which is included by |\include{|\textit{child}|}|
from within the main file
(or at least for those files to be compiled individually).
The argument \textit{main} must be the filename of the main file.

There are a couple of
considerations in setting up the main and child documents:

%%%%%%%%%%%%%%%%%%%%%%%%%%%%%%%%%%%%%%%%
\paragraph{Restrictions.}

Please note the following restrictions:
\begin{itemize}
\item
|\childdocmain| must be called with one argument \textit{main}
to ensure compatibility with earlier version of the package.
It must either be empty (|\childdocmain{}|)
or precisely match the filename of the main file in which it is specified.
See \secref{sec:detection} for further information.
\item
The filename \textit{main} must be specified without the |.tex| extension.
\item
The filename \textit{main} is case sensitive
(even in case-insensitive file systems)
due to internal string comparison.
\item
The argument \textit{main} should be fully expanded, it cannot be a macro.
\item
Subdirectories and special characters should be avoided in filenames.
\item
The command |\childdocmain{|\textit{main}|}| must be followed by a whitespace.
It should not be followed immediately by another command
or by a comment mark `|%|'.
This is because the \TeX{} parser reads the token immediately following
the argument of |\childdocmain| and puts it
at the beginning of every child section;
however, a white\-space is ignored.
\end{itemize}

%%%%%%%%%%%%%%%%%%%%%%%%%%%%%%%%%%%%%%%%
\paragraph{Content of Main File.}

It is advisable to place all content in the child files included by |\include|.
Any output contained in the main file will appear in all child documents
unless suppressed manually;
it cannot be suppressed automatically by the |\includeonly| directive
and thus should normally be avoided.
A method to include some content in the main file
by means of conditional processing is described in \secref{sec:conditional}.

%%%%%%%%%%%%%%%%%%%%%%%%%%%%%%%%%%%%%%%%
\paragraph{Page Numbering.}

When only a part of the document is compiled,
the appropriate numbering of pages
(as well as other status parameters)
is determined from the |.aux| files.
The latter contain information from previous passes.
However this information needs to propagate through
all intermediate child documents.
Therefore the page numbering in child documents may well
be inconsistent until the complete document is compiled at least once.

A useful (if unconventional) way to always ensure a consistent
page numbering is to restart the numbering in each child document
and denote the pages by `\textit{child}|.|\textit{page}'
where \textit{child} represents the chapter/section number of the child file.
This can be achieved by the command
|\numberwithin{page}{|\textit{child}|}|
of the \textsf{amsmath} package
where \textit{child} can be |chapter| or |section|
depending on the chosen structuring.
Alternatively, one can modify the macro |\thepage| appropriately
and reset the counter |page| at the start of each child file.

%%%%%%%%%%%%%%%%%%%%%%%%%%%%%%%%%%%%%%%%%%%%%%%%%%%%%%%%%%%%%%%%%%%%%%%%%%%%%%%%
\subsection{Conditional Processing}
\label{sec:conditional}

The package provides a mechanism to compile different versions
of a document. To customise the versions further some conditional processing
can come in handy to distinguish which version is being compiled.
The package provides two macros to describe the compilation context:

%%%%%%%%%%%%%%%%%%%%%%%%%%%%%%%%%%%%%%%%
\DescribeMacro{\ifchilddoc}
The conditional |\ifchilddoc| distinguishes between the compilation of
child documents and the main document:
%
\begin{center}
|\ifchilddoc |\textit{child-code}| |[|\||else |\textit{main-code}]| \||fi|
\end{center}

%%%%%%%%%%%%%%%%%%%%%%%%%%%%%%%%%%%%%%%%
\DescribeMacro{\childdocname}
\DescribeMacro{\childdocjob}
The macro |\childdocname| contains the filename (without extension)
of the main or child file being processed.
Note that |\childdocjob| will always contain the name of the main file.

%%%%%%%%%%%%%%%%%%%%%%%%%%%%%%%%%%%%%%%%
\paragraph{Title Page.}

Conditional processing can be used to include a title or banner page
in the main document when proper precautions are taken.
Importantly, the code in the main file should ensure that the page counter
(as well as other status parameters which are stored in the |.aux| files)
takes the same value after the conditional processing.
Otherwise the page numbers may take divergent values
depending on which part is compiled.

For example, a title page could be declared by:
%
\begin{center}
\begin{tabular}{l}
|\ifchilddoc\||else|\\
|\addtocounter{page}{-1}|\\
\textit{code for title page}\\
|\newpage|\\
|\||fi|
\end{tabular}
\end{center}
%
A banner page for the child documents can be generated by:
%
\begin{center}
\begin{tabular}{l}
|\ifchilddoc|\\
|\addtocounter{page}{-1}|\\
\textit{code for banner page}\\
|\newpage|\\
|\||fi|
\end{tabular}
\end{center}
%
Here one could write a message such as:
\begin{center}
|This is the part \childdocname{} of \childdocjob{}.|
\end{center}

%%%%%%%%%%%%%%%%%%%%%%%%%%%%%%%%%%%%%%%%%%%%%%%%%%%%%%%%%%%%%%%%%%%%%%%%%%%%%%%%
\subsection{Flags}
\label{sec:flags}

The package makes it easy to generate different versions
of the main or child documents.
To this end compilation flags can be defined
and assigned different default values.
They will be particularly useful in conjunction
with the forwarding mechanism described in \secref{sec:forward}.

For example, it may be useful to have a flag |\version|
which can be set to |draft| or |final|.
The document source will contain some conditional code
depending on the value of |\version|.
Suppose further, the flag should default to |final| for the main file
and to |draft| for child files
which is a natural assignment for editing the document.
This is achieved by placing the following code
in the preamble of the main document
(below the |\childdocmain| directive):
%
\begin{center}
\begin{tabular}{l}
|\ifchilddoc|\\
|\providecommand{\version}{draft}|\\
|\||else|\\
|\providecommand{\version}{final}|\\
|\||fi|
\end{tabular}
\end{center}
%
The definition by |\providecommand| makes sure
that previous definitions are not overwritten.
Further statements |\providecommand{\version}{...}|
can thus be added before the above code to override it.

For the main file, one might add a line
(between |\childdocmain| and the above block)
%
\begin{center}
|%\ifchilddoc\||else\providecommand{\version}{draft}\||fi|
\end{center}
%
which can be uncommented to produce a draft version.
Likewise one can add a line to the very top of a child file
(above the |\childdocof{|\textit{main}|}| directive)
%
\begin{center}
|%\providecommand{\version}{final}|
\end{center}
%
which can be uncommented to produce the final version of this child document.

%%%%%%%%%%%%%%%%%%%%%%%%%%%%%%%%%%%%%%%%%%%%%%%%%%%%%%%%%%%%%%%%%%%%%%%%%%%%%%%%
\subsection{Forwarding}
\label{sec:forward}

Different versions of the main or child documents
using compilation flags as described in \secref{sec:flags}
can be (permanently) stored in different files
for convenient compilation, viewing and distribution.
To this end, the package defines a command
to pass on compilation to a different file:

%%%%%%%%%%%%%%%%%%%%%%%%%%%%%%%%%%%%%%%%
\DescribeMacro{\childdocforward}
The command |\childdocforward| redirects processing to
another source file:
%
\begin{center}
\begin{tabular}{l}
|\input{childdoc.def}|\\
|\childdocforward[|\textit{main}|]{|\textit{dest}|}|\\
\end{tabular}
\end{center}
%
The argument \textit{dest} is the destination file
(without extension).
It should be the main file or one of the child files.
Note that further \textsf{childdoc} directives
such as |\childdocof| and |\childdocforward|
in the indicated file will be processed in this form.
The optional argument \textit{main}
passes on directly to the main file \textit{main}
while pretending to compile the child \textit{dest}.
This form behaves as if \textit{dest}
issues |\childdocof{|\textit{main}|}| right away,
and no further \textsf{childdoc} directives will be processed.

%%%%%%%%%%%%%%%%%%%%%%%%%%%%%%%%%%%%%%%%
\DescribeMacro{\...prefix}
In the alternative form |\childdocforwardprefix|,
%
\begin{center}
\begin{tabular}{l}
|\input{childdoc.def}|\\
|\childdocforwardprefix[|\textit{main}|]{|\textit{prefix}|}{|\textit{dest}|}|
\end{tabular}
\end{center}
%
the destination file is determined by a pattern
depending on the current file:
To make this work, the current file must be called
`{\textit{prefix}\hspace{0.2em}\textit{suffix}}'
with \textit{prefix} matching precisely the argument.
Processing is then passed on to the file
`{\textit{dest}\hspace{0.2em}\textit{suffix}}'.
Surely, the same effect is achieved by
directly specifying the
argument `{\textit{dest}\hspace{0.2em}\textit{suffix}}'
in the first form.
However, that requires to set up a different file
for each child. With the alternative form of the command
all these files can have exactly the same content
which simplifies setting them up and maintaining them.

For example, the following file |draft.tex|
with a compilation flag |\version| as described in \secref{sec:flags}
compiles the main document as a draft:
%
\begin{center}
\begin{tabular}{l}
|\def\version{draft}|\\
|\input{childdoc.def}|\\
|\childdocforward{|\textit{main}|}|
\end{tabular}
\end{center}
%
Likewise, the following files |final|\textit{nn}|.tex|
compile the final version of the child document
|child|\textit{nn}|.tex|:
%
\begin{center}
\begin{tabular}{l}
|\def\version{final}|\\
|\input{childdoc.def}|\\
|\childdocforwardprefix{final}{child}|
\end{tabular}
\end{center}
%

Note that when several versions of a main file and/or of each child file
are to be generated, it may be convenient to set up a |Makefile| or
shell script to automatise the process.

%%%%%%%%%%%%%%%%%%%%%%%%%%%%%%%%%%%%%%%%%%%%%%%%%%%%%%%%%%%%%%%%%%%%%%%%%%%%%%%%
\subsection{Command Line Processing}
\label{sec:commandline}

The effect of redirection files can also be achieved by invoking
the \LaTeX{} compiler with a more elaborate command line.
Most conveniently this should be done as part
of a shell script or a |Makefile|.

When using \textsf{childdoc} in the main file, the following
command lines effectively perform a redirection
(note that depending on the shell being used,
backslashes may have to be doubled: `|\|' $\to$ `|\\|'):
%
\begin{center}
|... -jobname "|\textit{target}|" |\\|"|[\textit{flags}]%
|\input{childdoc.def}\childdocforward[|\textit{main}|]{|\textit{dest}|}"|
\end{center}
%
Here \textit{target} is the name of the output file,
\textit{main} is the name of the main file
and \textit{dest} is the name of the main or child file to be processed
(all filenames without extensions).
The optional argument \textit{main} can be omitted
if \textit{main} matches \textit{dest}.
Optionally, compilation \textit{flags} can be defined via |\def| commands.
This command line makes the \TeX{} engine believe
it is compiling the file \textit{target}
whose content is specified as the latter parameter.
The provided code then forwards the processing to
\textit{main} or \textit{dest} as described in \secref{sec:forward}.

%%%%%%%%%%%%%%%%%%%%%%%%%%%%%%%%%%%%%%%%%%%%%%%%%%%%%%%%%%%%%%%%%%%%%%%%%%%%%%%%
\subsection{Include by Input}
\label{sec:input}

Including child documents by |\include| has some restrictions by design.
Most notably, the content of a child document always occupies
its own set of pages; pages cannot be shared between child documents.
Usually, this behaviour makes perfect sense
because each child document contain an essential part of the document.
However, in some situations it may be desirable to compose
a document from a collection of parts
without having mandatory page breaks between then.
For this case, the package
provides a mechanism to include parts
by |\input| which can also be processed individually.
However, by construction this mechanism
requires manual handling of the content to be output.

%%%%%%%%%%%%%%%%%%%%%%%%%%%%%%%%%%%%%%%%
\DescribeMacro{\ifchilddocmanual}
The main file should be prepared as usual, see \secref{sec:include}.
However, the document body must make a distinction
between processing of an individual part and of the main document, e.g.:
%
\begin{center}
\begin{tabular}{l}
|\ifchilddocmanual|\\
|\input{\childdocname}|\\
|\||else|\\
\textit{document body with }|\input{|\textit{part}|}|\\
|\||fi|
\end{tabular}
\end{center}
%
The conditional |\ifchilddocmanual| is true whenever
a part to be included by |\input| is being compiled,
and the name of the part is stored in |\childdocname|.

%%%%%%%%%%%%%%%%%%%%%%%%%%%%%%%%%%%%%%%%
\DescribeMacro{\childdocby}
Each part to be included by |\input| should start with:
%
\begin{center}
\begin{tabular}{l}
|\input{childdoc.def}|\\
|\childdocby{|\textit{main}|}|\\
\end{tabular}
\end{center}
%
The directive |\childdocby| is similar to |\childdocof|
described in \secref{sec:include},
but the subsequent selection of content must be done manually.
To that end, both |\ifchilddoc| and |\ifchilddocmanual|
will be true upon processing of a part,
and the name of the part is stored in |\childdocname|.
Note that |\jobname| will be set to the filename of the current part
so that each part receives an individual |.aux| file
that does not interfere with the |.aux| file(s) of the main document.
This behaviour can be altered by the alternative form
|\childdocby[*]{|\textit{main}|}| (with a non-empty optional argument)
which uses the |.aux| file of the main document
by setting |\jobname| to \textit{main}.

%%%%%%%%%%%%%%%%%%%%%%%%%%%%%%%%%%%%%%%%%%%%%%%%%%%%%%%%%%%%%%%%%%%%%%%%%%%%%%%%
\subsection{Driver Development}
\label{sec:driver}

The \textsf{childdoc} mechanism can also be use for the development
of definition files such as \LaTeX{} styles or classes.
This case differs from the above setup with multiple parts
included by |\include| in that no |\includeonly| should be invoked.
This can be achieved by starting the include file
(before |\ProvidesPackage|) with:
%
\begin{center}
\begin{tabular}{l}
|\input{childdoc.def}|\\
|\childdocforward{|\textit{main}|}|\\
\end{tabular}
\end{center}
%
or alternatively with:
%
\begin{center}
\begin{tabular}{l}
|\input{childdoc.def}|\\
|\childdocby{|\textit{main}|}|\\
\end{tabular}
\end{center}
%
Both forms have slightly different effects as described above.
The main file is prepared as usual, see \secref{sec:include}.

%%%%%%%%%%%%%%%%%%%%%%%%%%%%%%%%%%%%%%%%%%%%%%%%%%%%%%%%%%%%%%%%%%%%%%%%%%%%%%%%
\subsection{Legacy Detection}
\label{sec:detection}

The directive |\childdocmain| in the main file can detect
whether the complete document or merely a child is to be compiled
even without using the directive |\childdocof|.
This method is deprecated because it is less robust
and there is no compelling reason to use it;
it is merely provided for backward compatibility
and it may be removed in future versions.

If the detection mechanism is to be used,
it is mandatory to correctly specify
the filename of the main file as the argument of |\childdocmain|:
%
\begin{center}
\begin{tabular}{l}
|\input{childdoc.def}|\\
|\childdocmain{|\textit{main}|}|\\
\end{tabular}
\end{center}
%
If |\jobname| does not match the argument \textit{main} of |\childdocmain|,
it is assumed that |\jobname| points to the child file to be compiled.
When using |\childdocmain| with the main file specified as argument,
it suffices to start a child file
with just |\input{|\textit{main}|}|
without loading of the package and using |\childdocof|.
If instead all processing is done
with the appropriate \textsf{childdoc} directives,
the argument of \textit{main} of |\childdocmain| can be empty.

An alternative version of the command line processing described
in \secref{sec:commandline} using the detection mechanism reads:
%
\begin{center}
|... -jobname "|\textit{target}|" "|[\textit{flags}]%
[|\def\jobname{|\textit{dest}|}|]|\input{|\textit{main}|}"|
\end{center}

%%%%%%%%%%%%%%%%%%%%%%%%%%%%%%%%%%%%%%%%%%%%%%%%%%%%%%%%%%%%%%%%%%%%%%%%%%%%%%%%
\subsection{Manual Code}
\label{sec:manual}

In case one cannot be certain whether the definitions file |childdoc.def|
is installed on the target \TeX{} distribution
and one prefers not to ship it,
it is conceivable to paste a few relevant commands into the sources.

To that end, drop all statements |\input{childdoc.def}|
and perform the replacements as outlined below.
Instead of |\childdocmain{|\textit{main}|}| add the following code
to the top of the main file:
%
\begin{center}
\begin{tabular}{l}
|\||ifdefined\childdocname\endinput\||fi\newif\ifchilddoc|\\
|\edef\childdocname{\scantokens\expandafter{\jobname\noexpand}}|\\
|\def\childdocmain{|\textit{main}|}\||ifx\childdocmain\childdocname\||else|\\
|\childdoctrue\includeonly{\childdocname}\let\jobname\childdocmain\||fi|\\
\end{tabular}
\end{center}
%
Instead of |\childdocof{|\textit{main}|}| just include the main file
at the top of each child file:
%
\begin{center}
|\input{|\textit{main}|}|
\end{center}
%
A simple redirection |\childdocforward{|\textit{dest}|}| is achieved by:
%
\begin{center}
|\def\jobname{|\textit{dest}|}\input{\jobname}|
\end{center}
%
The redirection with prefix
|\childdocforwardprefix[|\textit{prefix}|]{|\textit{dest}|}|
is accomplished by:
%
\begin{center}
\begin{tabular}{l}
|{\edef\jobname{\scantokens\expandafter{\jobname\noexpand}}|\\
|\def\redirectjob |\textit{prefix}|#1~~~{\gdef\jobname{|\textit{dest}|#1}}|\\
|\expandafter\redirectjob\jobname~~~}\input{\jobname}|
\end{tabular}
\end{center}

In an alternative approach,
child documents can be compiled by a specific command line
without additional code or specific definitions:
%
\begin{center}
|... -jobname "|\textit{target}|" "|[\textit{flags}]%
|\includeonly{|\textit{dest}|}\input{|\textit{main}|}"|
\end{center}
%

%%%%%%%%%%%%%%%%%%%%%%%%%%%%%%%%%%%%%%%%%%%%%%%%%%%%%%%%%%%%%%%%%%%%%%%%%%%%%%%%
%%%%%%%%%%%%%%%%%%%%%%%%%%%%%%%%%%%%%%%%%%%%%%%%%%%%%%%%%%%%%%%%%%%%%%%%%%%%%%%%
\section{Information}

%%%%%%%%%%%%%%%%%%%%%%%%%%%%%%%%%%%%%%%%%%%%%%%%%%%%%%%%%%%%%%%%%%%%%%%%%%%%%%%%
\subsection{Copyright}

Copyright \copyright{} 2017--2018 Niklas Beisert

This work may be distributed and/or modified under the
conditions of the \LaTeX{} Project Public License, either version 1.3
of this license or (at your option) any later version.
The latest version of this license is in
  \url{http://www.latex-project.org/lppl.txt}
and version 1.3 or later is part of all distributions of \LaTeX{}
version 2005/12/01 or later.

This work has the LPPL maintenance status `maintained'.

The Current Maintainer of this work is Niklas Beisert.

This work consists of the files |README.txt|, |childdoc.ins| and |childdoc.dtx|
as well as the derived files |childdoc.def|, |cdocsamp.tex|
with |cdocsch1.tex|, |cdocsch2.tex|, |cdocspt3.tex|, |cdocspt4.tex|,
|cdocsdrf.tex|, |cdocsfn1.tex|, |cdocsfn2.tex|
as well as |childdoc.pdf|.

%%%%%%%%%%%%%%%%%%%%%%%%%%%%%%%%%%%%%%%%%%%%%%%%%%%%%%%%%%%%%%%%%%%%%%%%%%%%%%%%
\subsection{Files and Installation}

The package consists of the files:
%
\begin{center}
\begin{tabular}{ll}
    |README.txt|   & readme file \\
    |childdoc.ins| & installation file \\
    |childdoc.dtx| & source file \\
    |childdoc.def| & definition file \\
    |cdocsamp.tex| & sample main file \\
    |cdocsch1.tex| & sample include file \\
    |cdocsch2.tex| & sample include file \\
    |cdocspt3.tex| & sample part file \\
    |cdocspt4.tex| & sample part file \\
    |cdocsdrf.tex| & sample redirection file \\
    |cdocsfn1.tex| & sample redirection file \\
    |cdocsfn2.tex| & sample redirection file \\
    |childdoc.pdf| & manual
\end{tabular}
\end{center}
%
The distribution consists of the files
|README.txt|, |childdoc.ins| and |childdoc.dtx|.
%
\begin{itemize}
\item
Run (pdf)\LaTeX{} on |childdoc.dtx|
to compile the manual |childdoc.pdf| (this file).
\item
Run \LaTeX{} on |childdoc.ins| to create the definitions file |childdoc.def|
and the sample |cdocsamp.tex| with include files
|cdocsch1.tex|, |cdocsch2.tex|, |cdocspt3.tex|, |cdocspt4.tex|,
|cdocsdrf.tex|, |cdocsfn1.tex|, |cdocsfn2.tex|.
Then copy the file |childdoc.def| to an appropriate directory of your \LaTeX{}
distribution, e.g.\ \textit{texmf-root}|/tex/latex/childdoc|.
\end{itemize}

%%%%%%%%%%%%%%%%%%%%%%%%%%%%%%%%%%%%%%%%%%%%%%%%%%%%%%%%%%%%%%%%%%%%%%%%%%%%%%%%
\subsection{Related CTAN Packages}

There are several other packages which offer a similar functionality:
%
\begin{itemize}
\item
The packages
\href{http://ctan.org/pkg/docmute}{\textsf{docmute}},
\href{http://ctan.org/pkg/includex}{\textsf{includex}} and
\href{http://ctan.org/pkg/standalone}{\textsf{standalone}}
provide commands to include only the document body of
a child file thus allowing both files to be compiled individually.
\item
The packages \href{http://ctan.org/pkg/subdocs}{\textsf{subdocs}}
and \href{http://ctan.org/pkg/subfiles}{\textsf{subfiles}}
provide structures in which the main and child documents can be
encapsulated and allowing them to be compiled individually.
The inclusion mechanism is different from the conventional |\include|.
\item
The package \href{http://ctan.org/pkg/combine}{\textsf{combine}}
is an elaborate solution to combine several documents into one.
\end{itemize}
%
See also the CTAN topic \href{http://ctan.org/topic/subdocs}{\textsf{subdocs}}
for further related packages.
The present package differs from the above solutions in that
a document structure constructed with the conventional |\include| mechanism
just needs two extra commands at the top of every file
such that all constituent files can be compiled individually.

%%%%%%%%%%%%%%%%%%%%%%%%%%%%%%%%%%%%%%%%%%%%%%%%%%%%%%%%%%%%%%%%%%%%%%%%%%%%%%%%
%\subsection{Feature Suggestions}
%
%The following is a list of features which may be useful for future
%versions of this package:
%%
%\begin{itemize}
%\item
%\ldots
%\end{itemize}

%%%%%%%%%%%%%%%%%%%%%%%%%%%%%%%%%%%%%%%%%%%%%%%%%%%%%%%%%%%%%%%%%%%%%%%%%%%%%%%%
\subsection{Revision History}

%%%%%%%%%%%%%%%%%%%%%%%%%%%%%%%%%%%%%%%%
\paragraph{v2.0:} 2018/12/30

\begin{itemize}
\item
immediate forward processing
\item
added |\childdocby| mechanism
\item
manual restructured
\end{itemize}

%%%%%%%%%%%%%%%%%%%%%%%%%%%%%%%%%%%%%%%%
\paragraph{v1.6:} 2018/01/17

\begin{itemize}
\item
application for development of include files
\item
corrections to manual
\end{itemize}

%%%%%%%%%%%%%%%%%%%%%%%%%%%%%%%%%%%%%%%%
\paragraph{v1.5:} 2017/05/21

\begin{itemize}
\item
more complete structuring introduced
\item
|\childdocof| introduced
\item
|\childdoc| renamed to |\childdocmain|
\item
|\childredirect| renamed to |\childdocforward| and |\childdocforwardprefix|
and functionality expanded
\end{itemize}

%%%%%%%%%%%%%%%%%%%%%%%%%%%%%%%%%%%%%%%%
\paragraph{v1.0:} 2017/04/27

\begin{itemize}
\item
manual and install package
\item
first version published on CTAN
\end{itemize}

%%%%%%%%%%%%%%%%%%%%%%%%%%%%%%%%%%%%%%%%
\paragraph{v0.6:} 2017/04/26

\begin{itemize}
\item
redirection mechanism added
\end{itemize}

%%%%%%%%%%%%%%%%%%%%%%%%%%%%%%%%%%%%%%%%
\paragraph{v0.5:} 2017/04/26

\begin{itemize}
\item
functionality in definition file
\end{itemize}


%%%%%%%%%%%%%%%%%%%%%%%%%%%%%%%%%%%%%%%%%%%%%%%%%%%%%%%%%%%%%%%%%%%%%%%%%%%%%%%%
%%%%%%%%%%%%%%%%%%%%%%%%%%%%%%%%%%%%%%%%%%%%%%%%%%%%%%%%%%%%%%%%%%%%%%%%%%%%%%%%
%%%%%%%%%%%%%%%%%%%%%%%%%%%%%%%%%%%%%%%%%%%%%%%%%%%%%%%%%%%%%%%%%%%%%%%%%%%%%%%%
\appendix

\settowidth\MacroIndent{\rmfamily\scriptsize 000\ }

 \DocInput{childdoc.dtx}

\end{document}
%</driver>
% \fi
%
% %%%%%%%%%%%%%%%%%%%%%%%%%%%%%%%%%%%%%%%%%%%%%%%%%%%%%%%%%%%%%%%%%%%%%%%%%%%%%%
% %%%%%%%%%%%%%%%%%%%%%%%%%%%%%%%%%%%%%%%%%%%%%%%%%%%%%%%%%%%%%%%%%%%%%%%%%%%%%%
% \section{Sample}
%\iffalse
%<*samplemain>
%\fi
%
% The following presents a sample document
% with two chapters, two parts, a title page,
% a compile flag as well as three forwarding files to set the flag.
% It consists of eight |.tex| files:
% \begin{center}
% \begin{tabular}{ll}
% |cdocsamp.tex|&main file\\
% |cdocsch1.tex|&include file for chapter 1\\
% |cdocsch2.tex|&include file for chapter 2\\
% |cdocspt3.tex|&include file for part 3\\
% |cdocspt4.tex|&include file for part 4\\
% |cdocsdrf.tex|&forwarding file for main file in draft mode\\
% |cdocsfi1.tex|&forwarding file for final version of chapter 1\\
% |cdocsfi2.tex|&forwarding file for final version of chapter 2\\
% \end{tabular}
% \end{center}
% Each of the eight files can be compiled directly by the \LaTeX{} compiler.
%
% %%%%%%%%%%%%%%%%%%%%%%%%%%%%%%%%%%%%%%
% \paragraph{Main File.}
%
% The main file is called |cdocsamp.tex|.
%
% Load the \textsf{childdoc} definitions and
% declare the filename for the main document:
%    \begin{macrocode}
\input{childdoc.def}
\childdocmain{}
%    \end{macrocode}

% Optional override for |\version| flag:
%    \begin{macrocode}
%%\ifchilddoc\else\providecommand{\version}{draft}\fi
%    \end{macrocode}

% Define the default values for the |\version| flag
% (|final| for the main file and |draft| for childs):
%    \begin{macrocode}
\ifchilddoc
\providecommand{\version}{draft}
\else
\providecommand{\version}{final}
\fi
%    \end{macrocode}

% Load the standard document class:
%    \begin{macrocode}
\documentclass[12pt]{article}
%    \end{macrocode}

% Start the document body:
%    \begin{macrocode}
\begin{document}
%    \end{macrocode}

% Declare a title page.
% Print title, part of document being processed and version flag:
%    \begin{macrocode}
\addtocounter{page}{-1}
\begin{center}
{\LARGE\bfseries{}childdoc example\par}
\vspace{1cm}
\ifchilddoc
\ifchilddocmanual part\else chapter\fi:
`\childdocname' of `\childdocjob'\par
\else
main document: `\childdocjob'\par
\fi
version: \version\par
\end{center}
\newpage
%    \end{macrocode}

% Manually include selected file,
% otherwise process as usual:
%    \begin{macrocode}
\ifchilddocmanual
\section*{part `\childdocname'}
\input{\childdocname}
\else
%    \end{macrocode}

% Include the two chapters:
%    \begin{macrocode}
\include{cdocsch1}
\include{cdocsch2}
%    \end{macrocode}

% Include the two parts unless only chapters should be displayed:
%    \begin{macrocode}
\ifchilddoc\else
\section{part three}
\input{cdocspt3}
\section{part four}
\input{cdocspt4}
\fi
%    \end{macrocode}

% Process as usual until here:
%    \begin{macrocode}
\fi
%    \end{macrocode}

% End of document body:
%    \begin{macrocode}
\end{document}
%    \end{macrocode}
%\iffalse
%</samplemain>
%\fi
%
% %%%%%%%%%%%%%%%%%%%%%%%%%%%%%%%%%%%%%%
% \paragraph{Chapter Include Files.}
%
% The include files are called |cdocsch1.tex| and |cdocsch2.tex|.
%
%\iffalse
%<*samplechap1|samplechap2>
%\fi

% Optional override for |\version| flag:
%    \begin{macrocode}
%%\providecommand{\version}{final}
%    \end{macrocode}

% Include the main document:
%    \begin{macrocode}
\input{childdoc.def}
\childdocof{cdocsamp}
%    \end{macrocode}

%\iffalse
%</samplechap1|samplechap2>
%\fi
%
%\iffalse
%<*samplechap1>
%\fi
% Some text for chapter 1:
%    \begin{macrocode}
\section{one}
some text in chapter one
%    \end{macrocode}

%\iffalse
%</samplechap1>
%\fi
% Some text for chapter 2:
%\iffalse
%<*samplechap2>
%\fi
%    \begin{macrocode}
\section{two}
more text in chapter two
%    \end{macrocode}

%\iffalse
%</samplechap2>
%\fi
%
% %%%%%%%%%%%%%%%%%%%%%%%%%%%%%%%%%%%%%%
% \paragraph{Part Include Files.}
%
% The include files are called |cdocspt3.tex| and |cdocspt4.tex|.
%
%\iffalse
%<*samplepart3|samplepart4>
%\fi

% Optional override for |\version| flag:
%    \begin{macrocode}
%%\providecommand{\version}{final}
%    \end{macrocode}

% Include the main document:
%    \begin{macrocode}
\input{childdoc.def}
\childdocby{cdocsamp}
%    \end{macrocode}

%\iffalse
%</samplepart3|samplepart4>
%\fi
%
%\iffalse
%<*samplepart3>
%\fi
% Some text for part 3:
%    \begin{macrocode}
some text in part three
%    \end{macrocode}

%\iffalse
%</samplepart3>
%\fi
% Some text for part 4:
%\iffalse
%<*samplepart4>
%\fi
%    \begin{macrocode}
more text in part four
%    \end{macrocode}

%\iffalse
%</samplepart4>
%\fi
%
% %%%%%%%%%%%%%%%%%%%%%%%%%%%%%%%%%%%%%%
% \paragraph{Forwarding for a Complete Draft.}
%
% The following forwarding file |cdocsdrf.tex|
% compiles the main document in draft mode:
%\iffalse
%<*sampledraft>
%\fi
%    \begin{macrocode}
\def\version{draft}
\input{childdoc.def}
\childdocforward{cdocsamp}
%    \end{macrocode}

%\iffalse
%</sampledraft>
%\fi
%
% %%%%%%%%%%%%%%%%%%%%%%%%%%%%%%%%%%%%%%
% \paragraph{Forwarding for Final Version of the Chapters.}
%
% The following forwarding files |cdocsfn1.tex| and |cdocsfn2.tex|
% (with identical content)
% compile the final versions of the child documents
% |cdocsch1.tex| and |cdocsch2.tex|, respectively:
%\iffalse
%<*samplefinal>
%\fi
%    \begin{macrocode}
\def\version{final}
\input{childdoc.def}
\childdocforwardprefix[cdocsamp]{cdocsfn}{cdocsch}
%    \end{macrocode}

%\iffalse
%</samplefinal>
%\fi
%
% %%%%%%%%%%%%%%%%%%%%%%%%%%%%%%%%%%%%%%
% \paragraph{Command Line Processing.}
%
% The following three command lines generate the output files
% |cdocscld|, |cdocscl1| and |cdocscl2|
% which should be identical to
% |cdocsdrf|, |cdocsch1| and |cdocsfn2|, respectively:
% \begin{center}
% \begin{tabular}{l}
% |latex -jobname cdocscld \|\\
% |  "\def\version{draft}\input{childdoc.def}\childdocforward{cdocsamp}"|\\
% |latex -jobname cdocscl1 \|\\
% |  "\input{childdoc.def}\childdocforward[cdocsamp]{cdocsch1}"|\\
% |latex -jobname cdocscl2 \|\\
% |  "\def\version{final}\input{childdoc.def}\childdocforward{cdocsch2}"|
% \end{tabular}
% \end{center}
% Note that the trailing backslash on each first line
% merely continues the input to the second line
% (for convenient cut ant paste).
% Furthermore, the command |latex| can be replaced by any
% of its alternative versions such as |pdflatex|.
%
% %%%%%%%%%%%%%%%%%%%%%%%%%%%%%%%%%%%%%%%%%%%%%%%%%%%%%%%%%%%%%%%%%%%%%%%%%%%%%%
% %%%%%%%%%%%%%%%%%%%%%%%%%%%%%%%%%%%%%%%%%%%%%%%%%%%%%%%%%%%%%%%%%%%%%%%%%%%%%%
% \section{Implementation}
%\iffalse
%<*package>
%\fi
%
% This section describes the definitions file |childdoc.def|.

% The definitions cannot be loaded using |\usepackage| or |\RequirePackage|
% which has a mechanism to prevent loading a style file more than once.
% When loading the definitions by means of |\input|
% multiple instances have to be prevented manually:
%\iffalse
%This code needs to be before the `\ProvidesFile' directive
%which is defined at the beginning of this file.
%Therefore it is also placed there and commented out here.
%</package>
%<*discard>
%\fi
%    \begin{macrocode}
\ifdefined\childdocmain\endinput\fi
%    \end{macrocode}
%\iffalse
%</discard>
%<*package>
%\fi
%
% \macro{\ifchilddoc}
% \macro{\ifchilddocmanual}
% The conditional |\ifchilddoc| tells whether a
% child (true) or main (false) document is being compiled.
% The conditional |\ifchilddocmanual| tells whether
% the |\includeonly| mechanism is used (false) or
% the selection of child files must be performed manually (true).
% The definitions initialise to false:
%    \begin{macrocode}
\newif\ifchilddoc
\newif\ifchilddocmanual
%    \end{macrocode}

% \macro{\childdocname}
% \macro{\childdocjob}
% The macro |\childdocname| stores the name of the main document
% to be compiled. The macro |\childdocjob| stores the name of
% the document on which the \LaTeX{} compiler was originally invoked.
% The content of |\jobname| cannot be compared
% to filenames specified in the source due to different catcodes.
% The following code rescans |\jobname|, stores the result
% in |\childdocname| and saves a copy in |\childdocjob|:
%    \begin{macrocode}
\edef\childdocname{\scantokens\expandafter{\jobname\noexpand}}
\let\childdocjob\childdocname
%    \end{macrocode}

% \macro{\childdocdisable}
% The macro |\childdocdisable| prevents the main file
% from being processed more than once.
% At this stage, the main document command |\childdocmain|
% is assumed to be called once again where it should do nothing.
% Any subsequent call to it should prevent
% a secondary processing of the main document
% It overwrites the forwarding commands
% |\childdocof| and |\childdocforward|
% with empty macros to prevent further inclusions of the main document:
%    \begin{macrocode}
\newcommand{\childdocdisable}
{
  \renewcommand{\childdocmain}[1]{\renewcommand{\childdocmain}[1]{\endinput}}
  \renewcommand{\childdocof}[1]{}
  \renewcommand{\childdocby}[2][]{}
  \renewcommand{\childdocforward}[2][]{}
  \renewcommand{\childdocdisable}{}
}
%    \end{macrocode}

% \macro{\childdocmain}
% The macro |\childdocmain| is to be called at the top of the main file
% with nothing or the main filename (without extension) as argument.
% First, it breaks loops.
% If the argument is not empty and does not match |\childdocname|
% (which is set by the first inclusion of |childdoc.def|),
% |\ifchilddoc| is set to true, |\includeonly| is applied to the child file
% and |\jobname| is set to the main file
% (for proper handling of |.aux| files):
%    \begin{macrocode}
\newcommand{\childdocmain}[1]
{
  \childdocdisable\childdocmain{}
  \if?#1?\else
    \begingroup
      \def\childdoctmp{#1}
      \ifx\childdoctmp\childdocname
        \def\childdoctmp{}
      \else
        \def\childdoctmp
        {
          \childdoctrue
          \includeonly{\childdocname}
          \def\childdocjob{#1}
          \def\jobname{#1}
        }
      \fi
      \expandafter
    \endgroup
    \childdoctmp
  \fi
}
%    \end{macrocode}

% \macro{\childdocof}
% The command |\childdocof| redirects
% compilation to the main file |#1|.
%    \begin{macrocode}
\newcommand{\childdocof}[1]
{
  \childdocdisable
  \childdoctrue
  \includeonly{\childdocname}
  \def\jobname{#1}
  \def\childdocjob{#1}
  \input{#1}
}
%    \end{macrocode}

% \macro{\childdocby}
% The command |\childdocby| ....
%    \begin{macrocode}
\newcommand{\childdocby}[2][]
{
  \childdocdisable
  \childdoctrue
  \childdocmanualtrue
  \if?#1?\else
    \def\jobname{#2}
  \fi
  \def\childdocjob{#2}
  \input{#2}
  \endinput
}
%    \end{macrocode}

% \macro{\childdocforward}
% The command |\childdocforward| redirects
% compilation to the main file or
% (if the optional argument is given) a child file.
% Parameters are set as if the main file
% or a child file starting with |\childdocof| was compiled.
% Then compilation is handed over to the main file:
%    \begin{macrocode}
\newcommand{\childdocforward}[2][]
{
  \begingroup
    \if?#1?
      \def\childdoctmp
      {
        \def\childdocname{#2}
        \def\childdocjob{#2}
        \def\jobname{#2}
        \input{#2}
        \endinput
      }
    \else
      \def\childdoctmp
      {
        \childdocdisable
        \def\childdocname{#2}
        \childdoctrue
        \includeonly{#2}
        \def\childdocjob{#1}
        \def\jobname{#1}
        \input{#1}
        \endinput
      }
    \fi
    \expandafter
  \endgroup
  \childdoctmp
}
%    \end{macrocode}

% \macro{\childdocforwardprefix}
% The command |\childdocforwardprefix| redirects
% compilation to the main or a child file by means of a pattern.
% The prefix |#1| in the current filename is replaced by |#2|
% and the suffix of the current filename is kept
% (it is assumed that the filename does not contain the substring `|~~~|'
% which is used as a delimiter).
% Compilation is handed over to the new file by |\childdocforward|:
%    \begin{macrocode}
\newcommand{\childdocforwardprefix}[3][]
{
  \begingroup
    \def\childdocextract #2##1~~~{\def\childdoctmp{\childdocforward[#1]{#3##1}}}
    \expandafter\childdocextract\childdocname~~~
    \expandafter
  \endgroup
  \childdoctmp
}
%    \end{macrocode}

% \macro{\childdoc}
% The deprecated macro |\childdoc| is a legacy version of |\childdocmain|:
%    \begin{macrocode}
\newcommand{\childdoc}{\childdocmain}
%    \end{macrocode}

% \macro{\childdocredirect}
% The deprecated macro |\childdocredirect| is a legacy version
% of |\childdocforward| and |\childdocforwardprefix|:
%    \begin{macrocode}
\newcommand{\childdocredirect}[2][]
{
  \begingroup
    \if?#1?
      \def\childdoctmp{\childdocforward{#2}}
    \else
      \def\childdoctmp{\childdocforwardprefix{#1}{#2}}
    \fi
    \expandafter
  \endgroup
  \childdoctmp
}
%    \end{macrocode}

%\iffalse
%</package>
%\fi
%
\endinput
|\\
|\childdocforward{|\textit{main}|}|\\
\end{tabular}
\end{center}
%
or alternatively with:
%
\begin{center}
\begin{tabular}{l}
|% \iffalse
%
% childdoc.dtx Copyright (C) 2017-2018 Niklas Beisert
%
% This work may be distributed and/or modified under the
% conditions of the LaTeX Project Public License, either version 1.3
% of this license or (at your option) any later version.
% The latest version of this license is in
%   http://www.latex-project.org/lppl.txt
% and version 1.3 or later is part of all distributions of LaTeX
% version 2005/12/01 or later.
%
% This work has the LPPL maintenance status `maintained'.
%
% The Current Maintainer of this work is Niklas Beisert.
%
% This work consists of the files childdoc.dtx and childdoc.ins
% and the derived files childdoc.def and cdocsamp.tex with
% cdocsch1.tex, cdocsch2.tex, cdocsdrf.tex, cdocsfn1.tex, cdocsfn2.tex.
%
%<package>\ifdefined\childdocmain\endinput\fi
%<package>\ProvidesFile{childdoc.def}[2018/12/30 v2.0 child document driver]
%<samplemain>\ProvidesFile{cdocsamp.tex}[2018/12/30 v2.0 sample for childdoc]
%<*driver>
%\ProvidesFile{childdoc.drv}[2018/12/30 v2.0 childdoc reference manual file]
\PassOptionsToClass{10pt,a4paper}{article}
\documentclass{ltxdoc}

\usepackage[margin=35mm]{geometry}
\usepackage{hyperref}
\usepackage{hyperxmp}
\usepackage[usenames]{color}

\hypersetup{colorlinks=true}
\hypersetup{pdfstartview=FitH}
\hypersetup{pdfpagemode=UseNone}
\hypersetup{pdfsource={}}
\hypersetup{pdflang={en-UK}}
\hypersetup{pdfcopyright={Copyright 2017-2018 Niklas Beisert.
  This work may be distributed and/or modified under the
  conditions of the LaTeX Project Public License, either version 1.3
  of this license or (at your option) any later version.}}
\hypersetup{pdflicenseurl={http://www.latex-project.org/lppl.txt}}
\hypersetup{pdfcontactaddress={ETH Zurich, ITP, HIT K,
  Wolfgang-Pauli-Strasse 27}}
\hypersetup{pdfcontactpostcode={8093}}
\hypersetup{pdfcontactcity={Zurich}}
\hypersetup{pdfcontactcountry={Switzerland}}
\hypersetup{pdfcontactemail={nbeisert@itp.phys.ethz.ch}}
\hypersetup{pdfcontacturl={http://people.phys.ethz.ch/\xmptilde nbeisert/}}

\newcommand{\secref}[1]{\hyperref[#1]{section \ref*{#1}}}

\parskip1ex
\parindent0pt
\let\olditemize\itemize
\def\itemize{\olditemize\parskip0pt}

\begin{document}

\title{The \textsf{childdoc} Package}
\hypersetup{pdftitle={The childdoc Package}}
\author{Niklas Beisert\\[2ex]
  Institut f\"ur Theoretische Physik\\
  Eidgen\"ossische Technische Hochschule Z\"urich\\
  Wolfgang-Pauli-Strasse 27, 8093 Z\"urich, Switzerland\\[1ex]
  \href{mailto:nbeisert@itp.phys.ethz.ch}
  {\texttt{nbeisert@itp.phys.ethz.ch}}}
\hypersetup{pdfauthor={Niklas Beisert}}
\hypersetup{pdfsubject={Manual for the LaTeX2e Package childdoc}}
\date{30 December 2018, \textsf{v2.0}}
\maketitle

\begin{abstract}\noindent
\textsf{childdoc} is a \LaTeXe{} package
that enables the direct compilation
of document sections included by |\include|
to individual files.
\end{abstract}

\begingroup
\parskip0ex
\tableofcontents
\endgroup

%%%%%%%%%%%%%%%%%%%%%%%%%%%%%%%%%%%%%%%%%%%%%%%%%%%%%%%%%%%%%%%%%%%%%%%%%%%%%%%%
%%%%%%%%%%%%%%%%%%%%%%%%%%%%%%%%%%%%%%%%%%%%%%%%%%%%%%%%%%%%%%%%%%%%%%%%%%%%%%%%
\section{Introduction}

\LaTeX{} provides a mechanism to structure a large document (such as a book)
into a main file and several child files (containing the chapters)
using the |\include| command.
This mechanism is beneficial for documents
which span hundreds of pages in order to
make the source file(s) more manageable.
Moreover, compilation can be restricted to
selected child files by means of the |\includeonly| command.
The latter feature can be used to reduce the compilation time while editing
(this was significantly more useful in the earlier days of \LaTeX{})
or to generate a smaller document which is easier to navigate.
Another application of |\includeonly| is to generate
documents consisting of selected parts of the complete document.

However, there are a few drawbacks of the plain |\include| mechanism:
\begin{itemize}
\item
The child files cannot be compiled on their own,
they can only be compiled via the main file.
A naive editing environment
(such as a text editor with an option
to have the current file processed by \LaTeX)
may require one to switch to the main file before compiling;
attempting to compile the child file produces errors.
\item
The main file must be modified (each time)
to adjust the |\includeonly| command
to the present needs. This easily leaves the main file in a messy state.
\item
The generated document will always carry the filename
of the main document. This is inconvenient if
several child files are to be compiled and
to be kept for distribution.
\end{itemize}

The present package provides a simple interface
to make child files individually compilable by \LaTeX{}.
Compiling a child file then has the same effect as compiling
the main file with an |\includeonly| command
to select the appropriate child.
Moreover the generated document will carry the name of the child
rather than the main file.
This resolves all three above issues.

This feature is meant to make the editing of books,
thesis documents and lecture notes somewhat more convenient.
However, the package can also be used efficiently for
composing a series of documents (such as exercise sheets)
which are typically distributed individually.
It then assists the author in generating the individual documents
(potentially in different versions)
as well as a document containing the collected series.
Another application is in developing style files
or other kinds of included material
where compilation of the style file could redirect
to a sample or test file.

%%%%%%%%%%%%%%%%%%%%%%%%%%%%%%%%%%%%%%%%%%%%%%%%%%%%%%%%%%%%%%%%%%%%%%%%%%%%%%%%
%%%%%%%%%%%%%%%%%%%%%%%%%%%%%%%%%%%%%%%%%%%%%%%%%%%%%%%%%%%%%%%%%%%%%%%%%%%%%%%%
\section{Usage}

First of all, the package \textsf{childdoc} is \emph{not} a standard
\LaTeXe{} |.sty| style file! Therefore it needs to be invoked in
a non-standard way.

%%%%%%%%%%%%%%%%%%%%%%%%%%%%%%%%%%%%%%%%%%%%%%%%%%%%%%%%%%%%%%%%%%%%%%%%%%%%%%%%
\subsection{Included Files}
\label{sec:include}

%%%%%%%%%%%%%%%%%%%%%%%%%%%%%%%%%%%%%%%%
\DescribeMacro{\childdocmain}
To use the package, add the commands
\begin{center}
\begin{tabular}{l}
|\input{childdoc.def}|\\
|\childdocmain{}|\\
\end{tabular}
\end{center}
at the very top of the main \LaTeX{} file,
in particular \emph{before} the |\documentclass| statement!
The argument of |\childdocmain| should be left empty
(but it must be present).

%%%%%%%%%%%%%%%%%%%%%%%%%%%%%%%%%%%%%%%%
\DescribeMacro{\childdocof}
Furthermore, add the commands
\begin{center}
\begin{tabular}{l}
|\input{childdoc.def}|\\
|\childdocof{|\textit{main}|}|\\
\end{tabular}
\end{center}
at the top of every child file \textit{child}
which is included by |\include{|\textit{child}|}|
from within the main file
(or at least for those files to be compiled individually).
The argument \textit{main} must be the filename of the main file.

There are a couple of
considerations in setting up the main and child documents:

%%%%%%%%%%%%%%%%%%%%%%%%%%%%%%%%%%%%%%%%
\paragraph{Restrictions.}

Please note the following restrictions:
\begin{itemize}
\item
|\childdocmain| must be called with one argument \textit{main}
to ensure compatibility with earlier version of the package.
It must either be empty (|\childdocmain{}|)
or precisely match the filename of the main file in which it is specified.
See \secref{sec:detection} for further information.
\item
The filename \textit{main} must be specified without the |.tex| extension.
\item
The filename \textit{main} is case sensitive
(even in case-insensitive file systems)
due to internal string comparison.
\item
The argument \textit{main} should be fully expanded, it cannot be a macro.
\item
Subdirectories and special characters should be avoided in filenames.
\item
The command |\childdocmain{|\textit{main}|}| must be followed by a whitespace.
It should not be followed immediately by another command
or by a comment mark `|%|'.
This is because the \TeX{} parser reads the token immediately following
the argument of |\childdocmain| and puts it
at the beginning of every child section;
however, a white\-space is ignored.
\end{itemize}

%%%%%%%%%%%%%%%%%%%%%%%%%%%%%%%%%%%%%%%%
\paragraph{Content of Main File.}

It is advisable to place all content in the child files included by |\include|.
Any output contained in the main file will appear in all child documents
unless suppressed manually;
it cannot be suppressed automatically by the |\includeonly| directive
and thus should normally be avoided.
A method to include some content in the main file
by means of conditional processing is described in \secref{sec:conditional}.

%%%%%%%%%%%%%%%%%%%%%%%%%%%%%%%%%%%%%%%%
\paragraph{Page Numbering.}

When only a part of the document is compiled,
the appropriate numbering of pages
(as well as other status parameters)
is determined from the |.aux| files.
The latter contain information from previous passes.
However this information needs to propagate through
all intermediate child documents.
Therefore the page numbering in child documents may well
be inconsistent until the complete document is compiled at least once.

A useful (if unconventional) way to always ensure a consistent
page numbering is to restart the numbering in each child document
and denote the pages by `\textit{child}|.|\textit{page}'
where \textit{child} represents the chapter/section number of the child file.
This can be achieved by the command
|\numberwithin{page}{|\textit{child}|}|
of the \textsf{amsmath} package
where \textit{child} can be |chapter| or |section|
depending on the chosen structuring.
Alternatively, one can modify the macro |\thepage| appropriately
and reset the counter |page| at the start of each child file.

%%%%%%%%%%%%%%%%%%%%%%%%%%%%%%%%%%%%%%%%%%%%%%%%%%%%%%%%%%%%%%%%%%%%%%%%%%%%%%%%
\subsection{Conditional Processing}
\label{sec:conditional}

The package provides a mechanism to compile different versions
of a document. To customise the versions further some conditional processing
can come in handy to distinguish which version is being compiled.
The package provides two macros to describe the compilation context:

%%%%%%%%%%%%%%%%%%%%%%%%%%%%%%%%%%%%%%%%
\DescribeMacro{\ifchilddoc}
The conditional |\ifchilddoc| distinguishes between the compilation of
child documents and the main document:
%
\begin{center}
|\ifchilddoc |\textit{child-code}| |[|\||else |\textit{main-code}]| \||fi|
\end{center}

%%%%%%%%%%%%%%%%%%%%%%%%%%%%%%%%%%%%%%%%
\DescribeMacro{\childdocname}
\DescribeMacro{\childdocjob}
The macro |\childdocname| contains the filename (without extension)
of the main or child file being processed.
Note that |\childdocjob| will always contain the name of the main file.

%%%%%%%%%%%%%%%%%%%%%%%%%%%%%%%%%%%%%%%%
\paragraph{Title Page.}

Conditional processing can be used to include a title or banner page
in the main document when proper precautions are taken.
Importantly, the code in the main file should ensure that the page counter
(as well as other status parameters which are stored in the |.aux| files)
takes the same value after the conditional processing.
Otherwise the page numbers may take divergent values
depending on which part is compiled.

For example, a title page could be declared by:
%
\begin{center}
\begin{tabular}{l}
|\ifchilddoc\||else|\\
|\addtocounter{page}{-1}|\\
\textit{code for title page}\\
|\newpage|\\
|\||fi|
\end{tabular}
\end{center}
%
A banner page for the child documents can be generated by:
%
\begin{center}
\begin{tabular}{l}
|\ifchilddoc|\\
|\addtocounter{page}{-1}|\\
\textit{code for banner page}\\
|\newpage|\\
|\||fi|
\end{tabular}
\end{center}
%
Here one could write a message such as:
\begin{center}
|This is the part \childdocname{} of \childdocjob{}.|
\end{center}

%%%%%%%%%%%%%%%%%%%%%%%%%%%%%%%%%%%%%%%%%%%%%%%%%%%%%%%%%%%%%%%%%%%%%%%%%%%%%%%%
\subsection{Flags}
\label{sec:flags}

The package makes it easy to generate different versions
of the main or child documents.
To this end compilation flags can be defined
and assigned different default values.
They will be particularly useful in conjunction
with the forwarding mechanism described in \secref{sec:forward}.

For example, it may be useful to have a flag |\version|
which can be set to |draft| or |final|.
The document source will contain some conditional code
depending on the value of |\version|.
Suppose further, the flag should default to |final| for the main file
and to |draft| for child files
which is a natural assignment for editing the document.
This is achieved by placing the following code
in the preamble of the main document
(below the |\childdocmain| directive):
%
\begin{center}
\begin{tabular}{l}
|\ifchilddoc|\\
|\providecommand{\version}{draft}|\\
|\||else|\\
|\providecommand{\version}{final}|\\
|\||fi|
\end{tabular}
\end{center}
%
The definition by |\providecommand| makes sure
that previous definitions are not overwritten.
Further statements |\providecommand{\version}{...}|
can thus be added before the above code to override it.

For the main file, one might add a line
(between |\childdocmain| and the above block)
%
\begin{center}
|%\ifchilddoc\||else\providecommand{\version}{draft}\||fi|
\end{center}
%
which can be uncommented to produce a draft version.
Likewise one can add a line to the very top of a child file
(above the |\childdocof{|\textit{main}|}| directive)
%
\begin{center}
|%\providecommand{\version}{final}|
\end{center}
%
which can be uncommented to produce the final version of this child document.

%%%%%%%%%%%%%%%%%%%%%%%%%%%%%%%%%%%%%%%%%%%%%%%%%%%%%%%%%%%%%%%%%%%%%%%%%%%%%%%%
\subsection{Forwarding}
\label{sec:forward}

Different versions of the main or child documents
using compilation flags as described in \secref{sec:flags}
can be (permanently) stored in different files
for convenient compilation, viewing and distribution.
To this end, the package defines a command
to pass on compilation to a different file:

%%%%%%%%%%%%%%%%%%%%%%%%%%%%%%%%%%%%%%%%
\DescribeMacro{\childdocforward}
The command |\childdocforward| redirects processing to
another source file:
%
\begin{center}
\begin{tabular}{l}
|\input{childdoc.def}|\\
|\childdocforward[|\textit{main}|]{|\textit{dest}|}|\\
\end{tabular}
\end{center}
%
The argument \textit{dest} is the destination file
(without extension).
It should be the main file or one of the child files.
Note that further \textsf{childdoc} directives
such as |\childdocof| and |\childdocforward|
in the indicated file will be processed in this form.
The optional argument \textit{main}
passes on directly to the main file \textit{main}
while pretending to compile the child \textit{dest}.
This form behaves as if \textit{dest}
issues |\childdocof{|\textit{main}|}| right away,
and no further \textsf{childdoc} directives will be processed.

%%%%%%%%%%%%%%%%%%%%%%%%%%%%%%%%%%%%%%%%
\DescribeMacro{\...prefix}
In the alternative form |\childdocforwardprefix|,
%
\begin{center}
\begin{tabular}{l}
|\input{childdoc.def}|\\
|\childdocforwardprefix[|\textit{main}|]{|\textit{prefix}|}{|\textit{dest}|}|
\end{tabular}
\end{center}
%
the destination file is determined by a pattern
depending on the current file:
To make this work, the current file must be called
`{\textit{prefix}\hspace{0.2em}\textit{suffix}}'
with \textit{prefix} matching precisely the argument.
Processing is then passed on to the file
`{\textit{dest}\hspace{0.2em}\textit{suffix}}'.
Surely, the same effect is achieved by
directly specifying the
argument `{\textit{dest}\hspace{0.2em}\textit{suffix}}'
in the first form.
However, that requires to set up a different file
for each child. With the alternative form of the command
all these files can have exactly the same content
which simplifies setting them up and maintaining them.

For example, the following file |draft.tex|
with a compilation flag |\version| as described in \secref{sec:flags}
compiles the main document as a draft:
%
\begin{center}
\begin{tabular}{l}
|\def\version{draft}|\\
|\input{childdoc.def}|\\
|\childdocforward{|\textit{main}|}|
\end{tabular}
\end{center}
%
Likewise, the following files |final|\textit{nn}|.tex|
compile the final version of the child document
|child|\textit{nn}|.tex|:
%
\begin{center}
\begin{tabular}{l}
|\def\version{final}|\\
|\input{childdoc.def}|\\
|\childdocforwardprefix{final}{child}|
\end{tabular}
\end{center}
%

Note that when several versions of a main file and/or of each child file
are to be generated, it may be convenient to set up a |Makefile| or
shell script to automatise the process.

%%%%%%%%%%%%%%%%%%%%%%%%%%%%%%%%%%%%%%%%%%%%%%%%%%%%%%%%%%%%%%%%%%%%%%%%%%%%%%%%
\subsection{Command Line Processing}
\label{sec:commandline}

The effect of redirection files can also be achieved by invoking
the \LaTeX{} compiler with a more elaborate command line.
Most conveniently this should be done as part
of a shell script or a |Makefile|.

When using \textsf{childdoc} in the main file, the following
command lines effectively perform a redirection
(note that depending on the shell being used,
backslashes may have to be doubled: `|\|' $\to$ `|\\|'):
%
\begin{center}
|... -jobname "|\textit{target}|" |\\|"|[\textit{flags}]%
|\input{childdoc.def}\childdocforward[|\textit{main}|]{|\textit{dest}|}"|
\end{center}
%
Here \textit{target} is the name of the output file,
\textit{main} is the name of the main file
and \textit{dest} is the name of the main or child file to be processed
(all filenames without extensions).
The optional argument \textit{main} can be omitted
if \textit{main} matches \textit{dest}.
Optionally, compilation \textit{flags} can be defined via |\def| commands.
This command line makes the \TeX{} engine believe
it is compiling the file \textit{target}
whose content is specified as the latter parameter.
The provided code then forwards the processing to
\textit{main} or \textit{dest} as described in \secref{sec:forward}.

%%%%%%%%%%%%%%%%%%%%%%%%%%%%%%%%%%%%%%%%%%%%%%%%%%%%%%%%%%%%%%%%%%%%%%%%%%%%%%%%
\subsection{Include by Input}
\label{sec:input}

Including child documents by |\include| has some restrictions by design.
Most notably, the content of a child document always occupies
its own set of pages; pages cannot be shared between child documents.
Usually, this behaviour makes perfect sense
because each child document contain an essential part of the document.
However, in some situations it may be desirable to compose
a document from a collection of parts
without having mandatory page breaks between then.
For this case, the package
provides a mechanism to include parts
by |\input| which can also be processed individually.
However, by construction this mechanism
requires manual handling of the content to be output.

%%%%%%%%%%%%%%%%%%%%%%%%%%%%%%%%%%%%%%%%
\DescribeMacro{\ifchilddocmanual}
The main file should be prepared as usual, see \secref{sec:include}.
However, the document body must make a distinction
between processing of an individual part and of the main document, e.g.:
%
\begin{center}
\begin{tabular}{l}
|\ifchilddocmanual|\\
|\input{\childdocname}|\\
|\||else|\\
\textit{document body with }|\input{|\textit{part}|}|\\
|\||fi|
\end{tabular}
\end{center}
%
The conditional |\ifchilddocmanual| is true whenever
a part to be included by |\input| is being compiled,
and the name of the part is stored in |\childdocname|.

%%%%%%%%%%%%%%%%%%%%%%%%%%%%%%%%%%%%%%%%
\DescribeMacro{\childdocby}
Each part to be included by |\input| should start with:
%
\begin{center}
\begin{tabular}{l}
|\input{childdoc.def}|\\
|\childdocby{|\textit{main}|}|\\
\end{tabular}
\end{center}
%
The directive |\childdocby| is similar to |\childdocof|
described in \secref{sec:include},
but the subsequent selection of content must be done manually.
To that end, both |\ifchilddoc| and |\ifchilddocmanual|
will be true upon processing of a part,
and the name of the part is stored in |\childdocname|.
Note that |\jobname| will be set to the filename of the current part
so that each part receives an individual |.aux| file
that does not interfere with the |.aux| file(s) of the main document.
This behaviour can be altered by the alternative form
|\childdocby[*]{|\textit{main}|}| (with a non-empty optional argument)
which uses the |.aux| file of the main document
by setting |\jobname| to \textit{main}.

%%%%%%%%%%%%%%%%%%%%%%%%%%%%%%%%%%%%%%%%%%%%%%%%%%%%%%%%%%%%%%%%%%%%%%%%%%%%%%%%
\subsection{Driver Development}
\label{sec:driver}

The \textsf{childdoc} mechanism can also be use for the development
of definition files such as \LaTeX{} styles or classes.
This case differs from the above setup with multiple parts
included by |\include| in that no |\includeonly| should be invoked.
This can be achieved by starting the include file
(before |\ProvidesPackage|) with:
%
\begin{center}
\begin{tabular}{l}
|\input{childdoc.def}|\\
|\childdocforward{|\textit{main}|}|\\
\end{tabular}
\end{center}
%
or alternatively with:
%
\begin{center}
\begin{tabular}{l}
|\input{childdoc.def}|\\
|\childdocby{|\textit{main}|}|\\
\end{tabular}
\end{center}
%
Both forms have slightly different effects as described above.
The main file is prepared as usual, see \secref{sec:include}.

%%%%%%%%%%%%%%%%%%%%%%%%%%%%%%%%%%%%%%%%%%%%%%%%%%%%%%%%%%%%%%%%%%%%%%%%%%%%%%%%
\subsection{Legacy Detection}
\label{sec:detection}

The directive |\childdocmain| in the main file can detect
whether the complete document or merely a child is to be compiled
even without using the directive |\childdocof|.
This method is deprecated because it is less robust
and there is no compelling reason to use it;
it is merely provided for backward compatibility
and it may be removed in future versions.

If the detection mechanism is to be used,
it is mandatory to correctly specify
the filename of the main file as the argument of |\childdocmain|:
%
\begin{center}
\begin{tabular}{l}
|\input{childdoc.def}|\\
|\childdocmain{|\textit{main}|}|\\
\end{tabular}
\end{center}
%
If |\jobname| does not match the argument \textit{main} of |\childdocmain|,
it is assumed that |\jobname| points to the child file to be compiled.
When using |\childdocmain| with the main file specified as argument,
it suffices to start a child file
with just |\input{|\textit{main}|}|
without loading of the package and using |\childdocof|.
If instead all processing is done
with the appropriate \textsf{childdoc} directives,
the argument of \textit{main} of |\childdocmain| can be empty.

An alternative version of the command line processing described
in \secref{sec:commandline} using the detection mechanism reads:
%
\begin{center}
|... -jobname "|\textit{target}|" "|[\textit{flags}]%
[|\def\jobname{|\textit{dest}|}|]|\input{|\textit{main}|}"|
\end{center}

%%%%%%%%%%%%%%%%%%%%%%%%%%%%%%%%%%%%%%%%%%%%%%%%%%%%%%%%%%%%%%%%%%%%%%%%%%%%%%%%
\subsection{Manual Code}
\label{sec:manual}

In case one cannot be certain whether the definitions file |childdoc.def|
is installed on the target \TeX{} distribution
and one prefers not to ship it,
it is conceivable to paste a few relevant commands into the sources.

To that end, drop all statements |\input{childdoc.def}|
and perform the replacements as outlined below.
Instead of |\childdocmain{|\textit{main}|}| add the following code
to the top of the main file:
%
\begin{center}
\begin{tabular}{l}
|\||ifdefined\childdocname\endinput\||fi\newif\ifchilddoc|\\
|\edef\childdocname{\scantokens\expandafter{\jobname\noexpand}}|\\
|\def\childdocmain{|\textit{main}|}\||ifx\childdocmain\childdocname\||else|\\
|\childdoctrue\includeonly{\childdocname}\let\jobname\childdocmain\||fi|\\
\end{tabular}
\end{center}
%
Instead of |\childdocof{|\textit{main}|}| just include the main file
at the top of each child file:
%
\begin{center}
|\input{|\textit{main}|}|
\end{center}
%
A simple redirection |\childdocforward{|\textit{dest}|}| is achieved by:
%
\begin{center}
|\def\jobname{|\textit{dest}|}\input{\jobname}|
\end{center}
%
The redirection with prefix
|\childdocforwardprefix[|\textit{prefix}|]{|\textit{dest}|}|
is accomplished by:
%
\begin{center}
\begin{tabular}{l}
|{\edef\jobname{\scantokens\expandafter{\jobname\noexpand}}|\\
|\def\redirectjob |\textit{prefix}|#1~~~{\gdef\jobname{|\textit{dest}|#1}}|\\
|\expandafter\redirectjob\jobname~~~}\input{\jobname}|
\end{tabular}
\end{center}

In an alternative approach,
child documents can be compiled by a specific command line
without additional code or specific definitions:
%
\begin{center}
|... -jobname "|\textit{target}|" "|[\textit{flags}]%
|\includeonly{|\textit{dest}|}\input{|\textit{main}|}"|
\end{center}
%

%%%%%%%%%%%%%%%%%%%%%%%%%%%%%%%%%%%%%%%%%%%%%%%%%%%%%%%%%%%%%%%%%%%%%%%%%%%%%%%%
%%%%%%%%%%%%%%%%%%%%%%%%%%%%%%%%%%%%%%%%%%%%%%%%%%%%%%%%%%%%%%%%%%%%%%%%%%%%%%%%
\section{Information}

%%%%%%%%%%%%%%%%%%%%%%%%%%%%%%%%%%%%%%%%%%%%%%%%%%%%%%%%%%%%%%%%%%%%%%%%%%%%%%%%
\subsection{Copyright}

Copyright \copyright{} 2017--2018 Niklas Beisert

This work may be distributed and/or modified under the
conditions of the \LaTeX{} Project Public License, either version 1.3
of this license or (at your option) any later version.
The latest version of this license is in
  \url{http://www.latex-project.org/lppl.txt}
and version 1.3 or later is part of all distributions of \LaTeX{}
version 2005/12/01 or later.

This work has the LPPL maintenance status `maintained'.

The Current Maintainer of this work is Niklas Beisert.

This work consists of the files |README.txt|, |childdoc.ins| and |childdoc.dtx|
as well as the derived files |childdoc.def|, |cdocsamp.tex|
with |cdocsch1.tex|, |cdocsch2.tex|, |cdocspt3.tex|, |cdocspt4.tex|,
|cdocsdrf.tex|, |cdocsfn1.tex|, |cdocsfn2.tex|
as well as |childdoc.pdf|.

%%%%%%%%%%%%%%%%%%%%%%%%%%%%%%%%%%%%%%%%%%%%%%%%%%%%%%%%%%%%%%%%%%%%%%%%%%%%%%%%
\subsection{Files and Installation}

The package consists of the files:
%
\begin{center}
\begin{tabular}{ll}
    |README.txt|   & readme file \\
    |childdoc.ins| & installation file \\
    |childdoc.dtx| & source file \\
    |childdoc.def| & definition file \\
    |cdocsamp.tex| & sample main file \\
    |cdocsch1.tex| & sample include file \\
    |cdocsch2.tex| & sample include file \\
    |cdocspt3.tex| & sample part file \\
    |cdocspt4.tex| & sample part file \\
    |cdocsdrf.tex| & sample redirection file \\
    |cdocsfn1.tex| & sample redirection file \\
    |cdocsfn2.tex| & sample redirection file \\
    |childdoc.pdf| & manual
\end{tabular}
\end{center}
%
The distribution consists of the files
|README.txt|, |childdoc.ins| and |childdoc.dtx|.
%
\begin{itemize}
\item
Run (pdf)\LaTeX{} on |childdoc.dtx|
to compile the manual |childdoc.pdf| (this file).
\item
Run \LaTeX{} on |childdoc.ins| to create the definitions file |childdoc.def|
and the sample |cdocsamp.tex| with include files
|cdocsch1.tex|, |cdocsch2.tex|, |cdocspt3.tex|, |cdocspt4.tex|,
|cdocsdrf.tex|, |cdocsfn1.tex|, |cdocsfn2.tex|.
Then copy the file |childdoc.def| to an appropriate directory of your \LaTeX{}
distribution, e.g.\ \textit{texmf-root}|/tex/latex/childdoc|.
\end{itemize}

%%%%%%%%%%%%%%%%%%%%%%%%%%%%%%%%%%%%%%%%%%%%%%%%%%%%%%%%%%%%%%%%%%%%%%%%%%%%%%%%
\subsection{Related CTAN Packages}

There are several other packages which offer a similar functionality:
%
\begin{itemize}
\item
The packages
\href{http://ctan.org/pkg/docmute}{\textsf{docmute}},
\href{http://ctan.org/pkg/includex}{\textsf{includex}} and
\href{http://ctan.org/pkg/standalone}{\textsf{standalone}}
provide commands to include only the document body of
a child file thus allowing both files to be compiled individually.
\item
The packages \href{http://ctan.org/pkg/subdocs}{\textsf{subdocs}}
and \href{http://ctan.org/pkg/subfiles}{\textsf{subfiles}}
provide structures in which the main and child documents can be
encapsulated and allowing them to be compiled individually.
The inclusion mechanism is different from the conventional |\include|.
\item
The package \href{http://ctan.org/pkg/combine}{\textsf{combine}}
is an elaborate solution to combine several documents into one.
\end{itemize}
%
See also the CTAN topic \href{http://ctan.org/topic/subdocs}{\textsf{subdocs}}
for further related packages.
The present package differs from the above solutions in that
a document structure constructed with the conventional |\include| mechanism
just needs two extra commands at the top of every file
such that all constituent files can be compiled individually.

%%%%%%%%%%%%%%%%%%%%%%%%%%%%%%%%%%%%%%%%%%%%%%%%%%%%%%%%%%%%%%%%%%%%%%%%%%%%%%%%
%\subsection{Feature Suggestions}
%
%The following is a list of features which may be useful for future
%versions of this package:
%%
%\begin{itemize}
%\item
%\ldots
%\end{itemize}

%%%%%%%%%%%%%%%%%%%%%%%%%%%%%%%%%%%%%%%%%%%%%%%%%%%%%%%%%%%%%%%%%%%%%%%%%%%%%%%%
\subsection{Revision History}

%%%%%%%%%%%%%%%%%%%%%%%%%%%%%%%%%%%%%%%%
\paragraph{v2.0:} 2018/12/30

\begin{itemize}
\item
immediate forward processing
\item
added |\childdocby| mechanism
\item
manual restructured
\end{itemize}

%%%%%%%%%%%%%%%%%%%%%%%%%%%%%%%%%%%%%%%%
\paragraph{v1.6:} 2018/01/17

\begin{itemize}
\item
application for development of include files
\item
corrections to manual
\end{itemize}

%%%%%%%%%%%%%%%%%%%%%%%%%%%%%%%%%%%%%%%%
\paragraph{v1.5:} 2017/05/21

\begin{itemize}
\item
more complete structuring introduced
\item
|\childdocof| introduced
\item
|\childdoc| renamed to |\childdocmain|
\item
|\childredirect| renamed to |\childdocforward| and |\childdocforwardprefix|
and functionality expanded
\end{itemize}

%%%%%%%%%%%%%%%%%%%%%%%%%%%%%%%%%%%%%%%%
\paragraph{v1.0:} 2017/04/27

\begin{itemize}
\item
manual and install package
\item
first version published on CTAN
\end{itemize}

%%%%%%%%%%%%%%%%%%%%%%%%%%%%%%%%%%%%%%%%
\paragraph{v0.6:} 2017/04/26

\begin{itemize}
\item
redirection mechanism added
\end{itemize}

%%%%%%%%%%%%%%%%%%%%%%%%%%%%%%%%%%%%%%%%
\paragraph{v0.5:} 2017/04/26

\begin{itemize}
\item
functionality in definition file
\end{itemize}


%%%%%%%%%%%%%%%%%%%%%%%%%%%%%%%%%%%%%%%%%%%%%%%%%%%%%%%%%%%%%%%%%%%%%%%%%%%%%%%%
%%%%%%%%%%%%%%%%%%%%%%%%%%%%%%%%%%%%%%%%%%%%%%%%%%%%%%%%%%%%%%%%%%%%%%%%%%%%%%%%
%%%%%%%%%%%%%%%%%%%%%%%%%%%%%%%%%%%%%%%%%%%%%%%%%%%%%%%%%%%%%%%%%%%%%%%%%%%%%%%%
\appendix

\settowidth\MacroIndent{\rmfamily\scriptsize 000\ }

 \DocInput{childdoc.dtx}

\end{document}
%</driver>
% \fi
%
% %%%%%%%%%%%%%%%%%%%%%%%%%%%%%%%%%%%%%%%%%%%%%%%%%%%%%%%%%%%%%%%%%%%%%%%%%%%%%%
% %%%%%%%%%%%%%%%%%%%%%%%%%%%%%%%%%%%%%%%%%%%%%%%%%%%%%%%%%%%%%%%%%%%%%%%%%%%%%%
% \section{Sample}
%\iffalse
%<*samplemain>
%\fi
%
% The following presents a sample document
% with two chapters, two parts, a title page,
% a compile flag as well as three forwarding files to set the flag.
% It consists of eight |.tex| files:
% \begin{center}
% \begin{tabular}{ll}
% |cdocsamp.tex|&main file\\
% |cdocsch1.tex|&include file for chapter 1\\
% |cdocsch2.tex|&include file for chapter 2\\
% |cdocspt3.tex|&include file for part 3\\
% |cdocspt4.tex|&include file for part 4\\
% |cdocsdrf.tex|&forwarding file for main file in draft mode\\
% |cdocsfi1.tex|&forwarding file for final version of chapter 1\\
% |cdocsfi2.tex|&forwarding file for final version of chapter 2\\
% \end{tabular}
% \end{center}
% Each of the eight files can be compiled directly by the \LaTeX{} compiler.
%
% %%%%%%%%%%%%%%%%%%%%%%%%%%%%%%%%%%%%%%
% \paragraph{Main File.}
%
% The main file is called |cdocsamp.tex|.
%
% Load the \textsf{childdoc} definitions and
% declare the filename for the main document:
%    \begin{macrocode}
\input{childdoc.def}
\childdocmain{}
%    \end{macrocode}

% Optional override for |\version| flag:
%    \begin{macrocode}
%%\ifchilddoc\else\providecommand{\version}{draft}\fi
%    \end{macrocode}

% Define the default values for the |\version| flag
% (|final| for the main file and |draft| for childs):
%    \begin{macrocode}
\ifchilddoc
\providecommand{\version}{draft}
\else
\providecommand{\version}{final}
\fi
%    \end{macrocode}

% Load the standard document class:
%    \begin{macrocode}
\documentclass[12pt]{article}
%    \end{macrocode}

% Start the document body:
%    \begin{macrocode}
\begin{document}
%    \end{macrocode}

% Declare a title page.
% Print title, part of document being processed and version flag:
%    \begin{macrocode}
\addtocounter{page}{-1}
\begin{center}
{\LARGE\bfseries{}childdoc example\par}
\vspace{1cm}
\ifchilddoc
\ifchilddocmanual part\else chapter\fi:
`\childdocname' of `\childdocjob'\par
\else
main document: `\childdocjob'\par
\fi
version: \version\par
\end{center}
\newpage
%    \end{macrocode}

% Manually include selected file,
% otherwise process as usual:
%    \begin{macrocode}
\ifchilddocmanual
\section*{part `\childdocname'}
\input{\childdocname}
\else
%    \end{macrocode}

% Include the two chapters:
%    \begin{macrocode}
\include{cdocsch1}
\include{cdocsch2}
%    \end{macrocode}

% Include the two parts unless only chapters should be displayed:
%    \begin{macrocode}
\ifchilddoc\else
\section{part three}
\input{cdocspt3}
\section{part four}
\input{cdocspt4}
\fi
%    \end{macrocode}

% Process as usual until here:
%    \begin{macrocode}
\fi
%    \end{macrocode}

% End of document body:
%    \begin{macrocode}
\end{document}
%    \end{macrocode}
%\iffalse
%</samplemain>
%\fi
%
% %%%%%%%%%%%%%%%%%%%%%%%%%%%%%%%%%%%%%%
% \paragraph{Chapter Include Files.}
%
% The include files are called |cdocsch1.tex| and |cdocsch2.tex|.
%
%\iffalse
%<*samplechap1|samplechap2>
%\fi

% Optional override for |\version| flag:
%    \begin{macrocode}
%%\providecommand{\version}{final}
%    \end{macrocode}

% Include the main document:
%    \begin{macrocode}
\input{childdoc.def}
\childdocof{cdocsamp}
%    \end{macrocode}

%\iffalse
%</samplechap1|samplechap2>
%\fi
%
%\iffalse
%<*samplechap1>
%\fi
% Some text for chapter 1:
%    \begin{macrocode}
\section{one}
some text in chapter one
%    \end{macrocode}

%\iffalse
%</samplechap1>
%\fi
% Some text for chapter 2:
%\iffalse
%<*samplechap2>
%\fi
%    \begin{macrocode}
\section{two}
more text in chapter two
%    \end{macrocode}

%\iffalse
%</samplechap2>
%\fi
%
% %%%%%%%%%%%%%%%%%%%%%%%%%%%%%%%%%%%%%%
% \paragraph{Part Include Files.}
%
% The include files are called |cdocspt3.tex| and |cdocspt4.tex|.
%
%\iffalse
%<*samplepart3|samplepart4>
%\fi

% Optional override for |\version| flag:
%    \begin{macrocode}
%%\providecommand{\version}{final}
%    \end{macrocode}

% Include the main document:
%    \begin{macrocode}
\input{childdoc.def}
\childdocby{cdocsamp}
%    \end{macrocode}

%\iffalse
%</samplepart3|samplepart4>
%\fi
%
%\iffalse
%<*samplepart3>
%\fi
% Some text for part 3:
%    \begin{macrocode}
some text in part three
%    \end{macrocode}

%\iffalse
%</samplepart3>
%\fi
% Some text for part 4:
%\iffalse
%<*samplepart4>
%\fi
%    \begin{macrocode}
more text in part four
%    \end{macrocode}

%\iffalse
%</samplepart4>
%\fi
%
% %%%%%%%%%%%%%%%%%%%%%%%%%%%%%%%%%%%%%%
% \paragraph{Forwarding for a Complete Draft.}
%
% The following forwarding file |cdocsdrf.tex|
% compiles the main document in draft mode:
%\iffalse
%<*sampledraft>
%\fi
%    \begin{macrocode}
\def\version{draft}
\input{childdoc.def}
\childdocforward{cdocsamp}
%    \end{macrocode}

%\iffalse
%</sampledraft>
%\fi
%
% %%%%%%%%%%%%%%%%%%%%%%%%%%%%%%%%%%%%%%
% \paragraph{Forwarding for Final Version of the Chapters.}
%
% The following forwarding files |cdocsfn1.tex| and |cdocsfn2.tex|
% (with identical content)
% compile the final versions of the child documents
% |cdocsch1.tex| and |cdocsch2.tex|, respectively:
%\iffalse
%<*samplefinal>
%\fi
%    \begin{macrocode}
\def\version{final}
\input{childdoc.def}
\childdocforwardprefix[cdocsamp]{cdocsfn}{cdocsch}
%    \end{macrocode}

%\iffalse
%</samplefinal>
%\fi
%
% %%%%%%%%%%%%%%%%%%%%%%%%%%%%%%%%%%%%%%
% \paragraph{Command Line Processing.}
%
% The following three command lines generate the output files
% |cdocscld|, |cdocscl1| and |cdocscl2|
% which should be identical to
% |cdocsdrf|, |cdocsch1| and |cdocsfn2|, respectively:
% \begin{center}
% \begin{tabular}{l}
% |latex -jobname cdocscld \|\\
% |  "\def\version{draft}\input{childdoc.def}\childdocforward{cdocsamp}"|\\
% |latex -jobname cdocscl1 \|\\
% |  "\input{childdoc.def}\childdocforward[cdocsamp]{cdocsch1}"|\\
% |latex -jobname cdocscl2 \|\\
% |  "\def\version{final}\input{childdoc.def}\childdocforward{cdocsch2}"|
% \end{tabular}
% \end{center}
% Note that the trailing backslash on each first line
% merely continues the input to the second line
% (for convenient cut ant paste).
% Furthermore, the command |latex| can be replaced by any
% of its alternative versions such as |pdflatex|.
%
% %%%%%%%%%%%%%%%%%%%%%%%%%%%%%%%%%%%%%%%%%%%%%%%%%%%%%%%%%%%%%%%%%%%%%%%%%%%%%%
% %%%%%%%%%%%%%%%%%%%%%%%%%%%%%%%%%%%%%%%%%%%%%%%%%%%%%%%%%%%%%%%%%%%%%%%%%%%%%%
% \section{Implementation}
%\iffalse
%<*package>
%\fi
%
% This section describes the definitions file |childdoc.def|.

% The definitions cannot be loaded using |\usepackage| or |\RequirePackage|
% which has a mechanism to prevent loading a style file more than once.
% When loading the definitions by means of |\input|
% multiple instances have to be prevented manually:
%\iffalse
%This code needs to be before the `\ProvidesFile' directive
%which is defined at the beginning of this file.
%Therefore it is also placed there and commented out here.
%</package>
%<*discard>
%\fi
%    \begin{macrocode}
\ifdefined\childdocmain\endinput\fi
%    \end{macrocode}
%\iffalse
%</discard>
%<*package>
%\fi
%
% \macro{\ifchilddoc}
% \macro{\ifchilddocmanual}
% The conditional |\ifchilddoc| tells whether a
% child (true) or main (false) document is being compiled.
% The conditional |\ifchilddocmanual| tells whether
% the |\includeonly| mechanism is used (false) or
% the selection of child files must be performed manually (true).
% The definitions initialise to false:
%    \begin{macrocode}
\newif\ifchilddoc
\newif\ifchilddocmanual
%    \end{macrocode}

% \macro{\childdocname}
% \macro{\childdocjob}
% The macro |\childdocname| stores the name of the main document
% to be compiled. The macro |\childdocjob| stores the name of
% the document on which the \LaTeX{} compiler was originally invoked.
% The content of |\jobname| cannot be compared
% to filenames specified in the source due to different catcodes.
% The following code rescans |\jobname|, stores the result
% in |\childdocname| and saves a copy in |\childdocjob|:
%    \begin{macrocode}
\edef\childdocname{\scantokens\expandafter{\jobname\noexpand}}
\let\childdocjob\childdocname
%    \end{macrocode}

% \macro{\childdocdisable}
% The macro |\childdocdisable| prevents the main file
% from being processed more than once.
% At this stage, the main document command |\childdocmain|
% is assumed to be called once again where it should do nothing.
% Any subsequent call to it should prevent
% a secondary processing of the main document
% It overwrites the forwarding commands
% |\childdocof| and |\childdocforward|
% with empty macros to prevent further inclusions of the main document:
%    \begin{macrocode}
\newcommand{\childdocdisable}
{
  \renewcommand{\childdocmain}[1]{\renewcommand{\childdocmain}[1]{\endinput}}
  \renewcommand{\childdocof}[1]{}
  \renewcommand{\childdocby}[2][]{}
  \renewcommand{\childdocforward}[2][]{}
  \renewcommand{\childdocdisable}{}
}
%    \end{macrocode}

% \macro{\childdocmain}
% The macro |\childdocmain| is to be called at the top of the main file
% with nothing or the main filename (without extension) as argument.
% First, it breaks loops.
% If the argument is not empty and does not match |\childdocname|
% (which is set by the first inclusion of |childdoc.def|),
% |\ifchilddoc| is set to true, |\includeonly| is applied to the child file
% and |\jobname| is set to the main file
% (for proper handling of |.aux| files):
%    \begin{macrocode}
\newcommand{\childdocmain}[1]
{
  \childdocdisable\childdocmain{}
  \if?#1?\else
    \begingroup
      \def\childdoctmp{#1}
      \ifx\childdoctmp\childdocname
        \def\childdoctmp{}
      \else
        \def\childdoctmp
        {
          \childdoctrue
          \includeonly{\childdocname}
          \def\childdocjob{#1}
          \def\jobname{#1}
        }
      \fi
      \expandafter
    \endgroup
    \childdoctmp
  \fi
}
%    \end{macrocode}

% \macro{\childdocof}
% The command |\childdocof| redirects
% compilation to the main file |#1|.
%    \begin{macrocode}
\newcommand{\childdocof}[1]
{
  \childdocdisable
  \childdoctrue
  \includeonly{\childdocname}
  \def\jobname{#1}
  \def\childdocjob{#1}
  \input{#1}
}
%    \end{macrocode}

% \macro{\childdocby}
% The command |\childdocby| ....
%    \begin{macrocode}
\newcommand{\childdocby}[2][]
{
  \childdocdisable
  \childdoctrue
  \childdocmanualtrue
  \if?#1?\else
    \def\jobname{#2}
  \fi
  \def\childdocjob{#2}
  \input{#2}
  \endinput
}
%    \end{macrocode}

% \macro{\childdocforward}
% The command |\childdocforward| redirects
% compilation to the main file or
% (if the optional argument is given) a child file.
% Parameters are set as if the main file
% or a child file starting with |\childdocof| was compiled.
% Then compilation is handed over to the main file:
%    \begin{macrocode}
\newcommand{\childdocforward}[2][]
{
  \begingroup
    \if?#1?
      \def\childdoctmp
      {
        \def\childdocname{#2}
        \def\childdocjob{#2}
        \def\jobname{#2}
        \input{#2}
        \endinput
      }
    \else
      \def\childdoctmp
      {
        \childdocdisable
        \def\childdocname{#2}
        \childdoctrue
        \includeonly{#2}
        \def\childdocjob{#1}
        \def\jobname{#1}
        \input{#1}
        \endinput
      }
    \fi
    \expandafter
  \endgroup
  \childdoctmp
}
%    \end{macrocode}

% \macro{\childdocforwardprefix}
% The command |\childdocforwardprefix| redirects
% compilation to the main or a child file by means of a pattern.
% The prefix |#1| in the current filename is replaced by |#2|
% and the suffix of the current filename is kept
% (it is assumed that the filename does not contain the substring `|~~~|'
% which is used as a delimiter).
% Compilation is handed over to the new file by |\childdocforward|:
%    \begin{macrocode}
\newcommand{\childdocforwardprefix}[3][]
{
  \begingroup
    \def\childdocextract #2##1~~~{\def\childdoctmp{\childdocforward[#1]{#3##1}}}
    \expandafter\childdocextract\childdocname~~~
    \expandafter
  \endgroup
  \childdoctmp
}
%    \end{macrocode}

% \macro{\childdoc}
% The deprecated macro |\childdoc| is a legacy version of |\childdocmain|:
%    \begin{macrocode}
\newcommand{\childdoc}{\childdocmain}
%    \end{macrocode}

% \macro{\childdocredirect}
% The deprecated macro |\childdocredirect| is a legacy version
% of |\childdocforward| and |\childdocforwardprefix|:
%    \begin{macrocode}
\newcommand{\childdocredirect}[2][]
{
  \begingroup
    \if?#1?
      \def\childdoctmp{\childdocforward{#2}}
    \else
      \def\childdoctmp{\childdocforwardprefix{#1}{#2}}
    \fi
    \expandafter
  \endgroup
  \childdoctmp
}
%    \end{macrocode}

%\iffalse
%</package>
%\fi
%
\endinput
|\\
|\childdocby{|\textit{main}|}|\\
\end{tabular}
\end{center}
%
Both forms have slightly different effects as described above.
The main file is prepared as usual, see \secref{sec:include}.

%%%%%%%%%%%%%%%%%%%%%%%%%%%%%%%%%%%%%%%%%%%%%%%%%%%%%%%%%%%%%%%%%%%%%%%%%%%%%%%%
\subsection{Legacy Detection}
\label{sec:detection}

The directive |\childdocmain| in the main file can detect
whether the complete document or merely a child is to be compiled
even without using the directive |\childdocof|.
This method is deprecated because it is less robust
and there is no compelling reason to use it;
it is merely provided for backward compatibility
and it may be removed in future versions.

If the detection mechanism is to be used,
it is mandatory to correctly specify
the filename of the main file as the argument of |\childdocmain|:
%
\begin{center}
\begin{tabular}{l}
|% \iffalse
%
% childdoc.dtx Copyright (C) 2017-2018 Niklas Beisert
%
% This work may be distributed and/or modified under the
% conditions of the LaTeX Project Public License, either version 1.3
% of this license or (at your option) any later version.
% The latest version of this license is in
%   http://www.latex-project.org/lppl.txt
% and version 1.3 or later is part of all distributions of LaTeX
% version 2005/12/01 or later.
%
% This work has the LPPL maintenance status `maintained'.
%
% The Current Maintainer of this work is Niklas Beisert.
%
% This work consists of the files childdoc.dtx and childdoc.ins
% and the derived files childdoc.def and cdocsamp.tex with
% cdocsch1.tex, cdocsch2.tex, cdocsdrf.tex, cdocsfn1.tex, cdocsfn2.tex.
%
%<package>\ifdefined\childdocmain\endinput\fi
%<package>\ProvidesFile{childdoc.def}[2018/12/30 v2.0 child document driver]
%<samplemain>\ProvidesFile{cdocsamp.tex}[2018/12/30 v2.0 sample for childdoc]
%<*driver>
%\ProvidesFile{childdoc.drv}[2018/12/30 v2.0 childdoc reference manual file]
\PassOptionsToClass{10pt,a4paper}{article}
\documentclass{ltxdoc}

\usepackage[margin=35mm]{geometry}
\usepackage{hyperref}
\usepackage{hyperxmp}
\usepackage[usenames]{color}

\hypersetup{colorlinks=true}
\hypersetup{pdfstartview=FitH}
\hypersetup{pdfpagemode=UseNone}
\hypersetup{pdfsource={}}
\hypersetup{pdflang={en-UK}}
\hypersetup{pdfcopyright={Copyright 2017-2018 Niklas Beisert.
  This work may be distributed and/or modified under the
  conditions of the LaTeX Project Public License, either version 1.3
  of this license or (at your option) any later version.}}
\hypersetup{pdflicenseurl={http://www.latex-project.org/lppl.txt}}
\hypersetup{pdfcontactaddress={ETH Zurich, ITP, HIT K,
  Wolfgang-Pauli-Strasse 27}}
\hypersetup{pdfcontactpostcode={8093}}
\hypersetup{pdfcontactcity={Zurich}}
\hypersetup{pdfcontactcountry={Switzerland}}
\hypersetup{pdfcontactemail={nbeisert@itp.phys.ethz.ch}}
\hypersetup{pdfcontacturl={http://people.phys.ethz.ch/\xmptilde nbeisert/}}

\newcommand{\secref}[1]{\hyperref[#1]{section \ref*{#1}}}

\parskip1ex
\parindent0pt
\let\olditemize\itemize
\def\itemize{\olditemize\parskip0pt}

\begin{document}

\title{The \textsf{childdoc} Package}
\hypersetup{pdftitle={The childdoc Package}}
\author{Niklas Beisert\\[2ex]
  Institut f\"ur Theoretische Physik\\
  Eidgen\"ossische Technische Hochschule Z\"urich\\
  Wolfgang-Pauli-Strasse 27, 8093 Z\"urich, Switzerland\\[1ex]
  \href{mailto:nbeisert@itp.phys.ethz.ch}
  {\texttt{nbeisert@itp.phys.ethz.ch}}}
\hypersetup{pdfauthor={Niklas Beisert}}
\hypersetup{pdfsubject={Manual for the LaTeX2e Package childdoc}}
\date{30 December 2018, \textsf{v2.0}}
\maketitle

\begin{abstract}\noindent
\textsf{childdoc} is a \LaTeXe{} package
that enables the direct compilation
of document sections included by |\include|
to individual files.
\end{abstract}

\begingroup
\parskip0ex
\tableofcontents
\endgroup

%%%%%%%%%%%%%%%%%%%%%%%%%%%%%%%%%%%%%%%%%%%%%%%%%%%%%%%%%%%%%%%%%%%%%%%%%%%%%%%%
%%%%%%%%%%%%%%%%%%%%%%%%%%%%%%%%%%%%%%%%%%%%%%%%%%%%%%%%%%%%%%%%%%%%%%%%%%%%%%%%
\section{Introduction}

\LaTeX{} provides a mechanism to structure a large document (such as a book)
into a main file and several child files (containing the chapters)
using the |\include| command.
This mechanism is beneficial for documents
which span hundreds of pages in order to
make the source file(s) more manageable.
Moreover, compilation can be restricted to
selected child files by means of the |\includeonly| command.
The latter feature can be used to reduce the compilation time while editing
(this was significantly more useful in the earlier days of \LaTeX{})
or to generate a smaller document which is easier to navigate.
Another application of |\includeonly| is to generate
documents consisting of selected parts of the complete document.

However, there are a few drawbacks of the plain |\include| mechanism:
\begin{itemize}
\item
The child files cannot be compiled on their own,
they can only be compiled via the main file.
A naive editing environment
(such as a text editor with an option
to have the current file processed by \LaTeX)
may require one to switch to the main file before compiling;
attempting to compile the child file produces errors.
\item
The main file must be modified (each time)
to adjust the |\includeonly| command
to the present needs. This easily leaves the main file in a messy state.
\item
The generated document will always carry the filename
of the main document. This is inconvenient if
several child files are to be compiled and
to be kept for distribution.
\end{itemize}

The present package provides a simple interface
to make child files individually compilable by \LaTeX{}.
Compiling a child file then has the same effect as compiling
the main file with an |\includeonly| command
to select the appropriate child.
Moreover the generated document will carry the name of the child
rather than the main file.
This resolves all three above issues.

This feature is meant to make the editing of books,
thesis documents and lecture notes somewhat more convenient.
However, the package can also be used efficiently for
composing a series of documents (such as exercise sheets)
which are typically distributed individually.
It then assists the author in generating the individual documents
(potentially in different versions)
as well as a document containing the collected series.
Another application is in developing style files
or other kinds of included material
where compilation of the style file could redirect
to a sample or test file.

%%%%%%%%%%%%%%%%%%%%%%%%%%%%%%%%%%%%%%%%%%%%%%%%%%%%%%%%%%%%%%%%%%%%%%%%%%%%%%%%
%%%%%%%%%%%%%%%%%%%%%%%%%%%%%%%%%%%%%%%%%%%%%%%%%%%%%%%%%%%%%%%%%%%%%%%%%%%%%%%%
\section{Usage}

First of all, the package \textsf{childdoc} is \emph{not} a standard
\LaTeXe{} |.sty| style file! Therefore it needs to be invoked in
a non-standard way.

%%%%%%%%%%%%%%%%%%%%%%%%%%%%%%%%%%%%%%%%%%%%%%%%%%%%%%%%%%%%%%%%%%%%%%%%%%%%%%%%
\subsection{Included Files}
\label{sec:include}

%%%%%%%%%%%%%%%%%%%%%%%%%%%%%%%%%%%%%%%%
\DescribeMacro{\childdocmain}
To use the package, add the commands
\begin{center}
\begin{tabular}{l}
|\input{childdoc.def}|\\
|\childdocmain{}|\\
\end{tabular}
\end{center}
at the very top of the main \LaTeX{} file,
in particular \emph{before} the |\documentclass| statement!
The argument of |\childdocmain| should be left empty
(but it must be present).

%%%%%%%%%%%%%%%%%%%%%%%%%%%%%%%%%%%%%%%%
\DescribeMacro{\childdocof}
Furthermore, add the commands
\begin{center}
\begin{tabular}{l}
|\input{childdoc.def}|\\
|\childdocof{|\textit{main}|}|\\
\end{tabular}
\end{center}
at the top of every child file \textit{child}
which is included by |\include{|\textit{child}|}|
from within the main file
(or at least for those files to be compiled individually).
The argument \textit{main} must be the filename of the main file.

There are a couple of
considerations in setting up the main and child documents:

%%%%%%%%%%%%%%%%%%%%%%%%%%%%%%%%%%%%%%%%
\paragraph{Restrictions.}

Please note the following restrictions:
\begin{itemize}
\item
|\childdocmain| must be called with one argument \textit{main}
to ensure compatibility with earlier version of the package.
It must either be empty (|\childdocmain{}|)
or precisely match the filename of the main file in which it is specified.
See \secref{sec:detection} for further information.
\item
The filename \textit{main} must be specified without the |.tex| extension.
\item
The filename \textit{main} is case sensitive
(even in case-insensitive file systems)
due to internal string comparison.
\item
The argument \textit{main} should be fully expanded, it cannot be a macro.
\item
Subdirectories and special characters should be avoided in filenames.
\item
The command |\childdocmain{|\textit{main}|}| must be followed by a whitespace.
It should not be followed immediately by another command
or by a comment mark `|%|'.
This is because the \TeX{} parser reads the token immediately following
the argument of |\childdocmain| and puts it
at the beginning of every child section;
however, a white\-space is ignored.
\end{itemize}

%%%%%%%%%%%%%%%%%%%%%%%%%%%%%%%%%%%%%%%%
\paragraph{Content of Main File.}

It is advisable to place all content in the child files included by |\include|.
Any output contained in the main file will appear in all child documents
unless suppressed manually;
it cannot be suppressed automatically by the |\includeonly| directive
and thus should normally be avoided.
A method to include some content in the main file
by means of conditional processing is described in \secref{sec:conditional}.

%%%%%%%%%%%%%%%%%%%%%%%%%%%%%%%%%%%%%%%%
\paragraph{Page Numbering.}

When only a part of the document is compiled,
the appropriate numbering of pages
(as well as other status parameters)
is determined from the |.aux| files.
The latter contain information from previous passes.
However this information needs to propagate through
all intermediate child documents.
Therefore the page numbering in child documents may well
be inconsistent until the complete document is compiled at least once.

A useful (if unconventional) way to always ensure a consistent
page numbering is to restart the numbering in each child document
and denote the pages by `\textit{child}|.|\textit{page}'
where \textit{child} represents the chapter/section number of the child file.
This can be achieved by the command
|\numberwithin{page}{|\textit{child}|}|
of the \textsf{amsmath} package
where \textit{child} can be |chapter| or |section|
depending on the chosen structuring.
Alternatively, one can modify the macro |\thepage| appropriately
and reset the counter |page| at the start of each child file.

%%%%%%%%%%%%%%%%%%%%%%%%%%%%%%%%%%%%%%%%%%%%%%%%%%%%%%%%%%%%%%%%%%%%%%%%%%%%%%%%
\subsection{Conditional Processing}
\label{sec:conditional}

The package provides a mechanism to compile different versions
of a document. To customise the versions further some conditional processing
can come in handy to distinguish which version is being compiled.
The package provides two macros to describe the compilation context:

%%%%%%%%%%%%%%%%%%%%%%%%%%%%%%%%%%%%%%%%
\DescribeMacro{\ifchilddoc}
The conditional |\ifchilddoc| distinguishes between the compilation of
child documents and the main document:
%
\begin{center}
|\ifchilddoc |\textit{child-code}| |[|\||else |\textit{main-code}]| \||fi|
\end{center}

%%%%%%%%%%%%%%%%%%%%%%%%%%%%%%%%%%%%%%%%
\DescribeMacro{\childdocname}
\DescribeMacro{\childdocjob}
The macro |\childdocname| contains the filename (without extension)
of the main or child file being processed.
Note that |\childdocjob| will always contain the name of the main file.

%%%%%%%%%%%%%%%%%%%%%%%%%%%%%%%%%%%%%%%%
\paragraph{Title Page.}

Conditional processing can be used to include a title or banner page
in the main document when proper precautions are taken.
Importantly, the code in the main file should ensure that the page counter
(as well as other status parameters which are stored in the |.aux| files)
takes the same value after the conditional processing.
Otherwise the page numbers may take divergent values
depending on which part is compiled.

For example, a title page could be declared by:
%
\begin{center}
\begin{tabular}{l}
|\ifchilddoc\||else|\\
|\addtocounter{page}{-1}|\\
\textit{code for title page}\\
|\newpage|\\
|\||fi|
\end{tabular}
\end{center}
%
A banner page for the child documents can be generated by:
%
\begin{center}
\begin{tabular}{l}
|\ifchilddoc|\\
|\addtocounter{page}{-1}|\\
\textit{code for banner page}\\
|\newpage|\\
|\||fi|
\end{tabular}
\end{center}
%
Here one could write a message such as:
\begin{center}
|This is the part \childdocname{} of \childdocjob{}.|
\end{center}

%%%%%%%%%%%%%%%%%%%%%%%%%%%%%%%%%%%%%%%%%%%%%%%%%%%%%%%%%%%%%%%%%%%%%%%%%%%%%%%%
\subsection{Flags}
\label{sec:flags}

The package makes it easy to generate different versions
of the main or child documents.
To this end compilation flags can be defined
and assigned different default values.
They will be particularly useful in conjunction
with the forwarding mechanism described in \secref{sec:forward}.

For example, it may be useful to have a flag |\version|
which can be set to |draft| or |final|.
The document source will contain some conditional code
depending on the value of |\version|.
Suppose further, the flag should default to |final| for the main file
and to |draft| for child files
which is a natural assignment for editing the document.
This is achieved by placing the following code
in the preamble of the main document
(below the |\childdocmain| directive):
%
\begin{center}
\begin{tabular}{l}
|\ifchilddoc|\\
|\providecommand{\version}{draft}|\\
|\||else|\\
|\providecommand{\version}{final}|\\
|\||fi|
\end{tabular}
\end{center}
%
The definition by |\providecommand| makes sure
that previous definitions are not overwritten.
Further statements |\providecommand{\version}{...}|
can thus be added before the above code to override it.

For the main file, one might add a line
(between |\childdocmain| and the above block)
%
\begin{center}
|%\ifchilddoc\||else\providecommand{\version}{draft}\||fi|
\end{center}
%
which can be uncommented to produce a draft version.
Likewise one can add a line to the very top of a child file
(above the |\childdocof{|\textit{main}|}| directive)
%
\begin{center}
|%\providecommand{\version}{final}|
\end{center}
%
which can be uncommented to produce the final version of this child document.

%%%%%%%%%%%%%%%%%%%%%%%%%%%%%%%%%%%%%%%%%%%%%%%%%%%%%%%%%%%%%%%%%%%%%%%%%%%%%%%%
\subsection{Forwarding}
\label{sec:forward}

Different versions of the main or child documents
using compilation flags as described in \secref{sec:flags}
can be (permanently) stored in different files
for convenient compilation, viewing and distribution.
To this end, the package defines a command
to pass on compilation to a different file:

%%%%%%%%%%%%%%%%%%%%%%%%%%%%%%%%%%%%%%%%
\DescribeMacro{\childdocforward}
The command |\childdocforward| redirects processing to
another source file:
%
\begin{center}
\begin{tabular}{l}
|\input{childdoc.def}|\\
|\childdocforward[|\textit{main}|]{|\textit{dest}|}|\\
\end{tabular}
\end{center}
%
The argument \textit{dest} is the destination file
(without extension).
It should be the main file or one of the child files.
Note that further \textsf{childdoc} directives
such as |\childdocof| and |\childdocforward|
in the indicated file will be processed in this form.
The optional argument \textit{main}
passes on directly to the main file \textit{main}
while pretending to compile the child \textit{dest}.
This form behaves as if \textit{dest}
issues |\childdocof{|\textit{main}|}| right away,
and no further \textsf{childdoc} directives will be processed.

%%%%%%%%%%%%%%%%%%%%%%%%%%%%%%%%%%%%%%%%
\DescribeMacro{\...prefix}
In the alternative form |\childdocforwardprefix|,
%
\begin{center}
\begin{tabular}{l}
|\input{childdoc.def}|\\
|\childdocforwardprefix[|\textit{main}|]{|\textit{prefix}|}{|\textit{dest}|}|
\end{tabular}
\end{center}
%
the destination file is determined by a pattern
depending on the current file:
To make this work, the current file must be called
`{\textit{prefix}\hspace{0.2em}\textit{suffix}}'
with \textit{prefix} matching precisely the argument.
Processing is then passed on to the file
`{\textit{dest}\hspace{0.2em}\textit{suffix}}'.
Surely, the same effect is achieved by
directly specifying the
argument `{\textit{dest}\hspace{0.2em}\textit{suffix}}'
in the first form.
However, that requires to set up a different file
for each child. With the alternative form of the command
all these files can have exactly the same content
which simplifies setting them up and maintaining them.

For example, the following file |draft.tex|
with a compilation flag |\version| as described in \secref{sec:flags}
compiles the main document as a draft:
%
\begin{center}
\begin{tabular}{l}
|\def\version{draft}|\\
|\input{childdoc.def}|\\
|\childdocforward{|\textit{main}|}|
\end{tabular}
\end{center}
%
Likewise, the following files |final|\textit{nn}|.tex|
compile the final version of the child document
|child|\textit{nn}|.tex|:
%
\begin{center}
\begin{tabular}{l}
|\def\version{final}|\\
|\input{childdoc.def}|\\
|\childdocforwardprefix{final}{child}|
\end{tabular}
\end{center}
%

Note that when several versions of a main file and/or of each child file
are to be generated, it may be convenient to set up a |Makefile| or
shell script to automatise the process.

%%%%%%%%%%%%%%%%%%%%%%%%%%%%%%%%%%%%%%%%%%%%%%%%%%%%%%%%%%%%%%%%%%%%%%%%%%%%%%%%
\subsection{Command Line Processing}
\label{sec:commandline}

The effect of redirection files can also be achieved by invoking
the \LaTeX{} compiler with a more elaborate command line.
Most conveniently this should be done as part
of a shell script or a |Makefile|.

When using \textsf{childdoc} in the main file, the following
command lines effectively perform a redirection
(note that depending on the shell being used,
backslashes may have to be doubled: `|\|' $\to$ `|\\|'):
%
\begin{center}
|... -jobname "|\textit{target}|" |\\|"|[\textit{flags}]%
|\input{childdoc.def}\childdocforward[|\textit{main}|]{|\textit{dest}|}"|
\end{center}
%
Here \textit{target} is the name of the output file,
\textit{main} is the name of the main file
and \textit{dest} is the name of the main or child file to be processed
(all filenames without extensions).
The optional argument \textit{main} can be omitted
if \textit{main} matches \textit{dest}.
Optionally, compilation \textit{flags} can be defined via |\def| commands.
This command line makes the \TeX{} engine believe
it is compiling the file \textit{target}
whose content is specified as the latter parameter.
The provided code then forwards the processing to
\textit{main} or \textit{dest} as described in \secref{sec:forward}.

%%%%%%%%%%%%%%%%%%%%%%%%%%%%%%%%%%%%%%%%%%%%%%%%%%%%%%%%%%%%%%%%%%%%%%%%%%%%%%%%
\subsection{Include by Input}
\label{sec:input}

Including child documents by |\include| has some restrictions by design.
Most notably, the content of a child document always occupies
its own set of pages; pages cannot be shared between child documents.
Usually, this behaviour makes perfect sense
because each child document contain an essential part of the document.
However, in some situations it may be desirable to compose
a document from a collection of parts
without having mandatory page breaks between then.
For this case, the package
provides a mechanism to include parts
by |\input| which can also be processed individually.
However, by construction this mechanism
requires manual handling of the content to be output.

%%%%%%%%%%%%%%%%%%%%%%%%%%%%%%%%%%%%%%%%
\DescribeMacro{\ifchilddocmanual}
The main file should be prepared as usual, see \secref{sec:include}.
However, the document body must make a distinction
between processing of an individual part and of the main document, e.g.:
%
\begin{center}
\begin{tabular}{l}
|\ifchilddocmanual|\\
|\input{\childdocname}|\\
|\||else|\\
\textit{document body with }|\input{|\textit{part}|}|\\
|\||fi|
\end{tabular}
\end{center}
%
The conditional |\ifchilddocmanual| is true whenever
a part to be included by |\input| is being compiled,
and the name of the part is stored in |\childdocname|.

%%%%%%%%%%%%%%%%%%%%%%%%%%%%%%%%%%%%%%%%
\DescribeMacro{\childdocby}
Each part to be included by |\input| should start with:
%
\begin{center}
\begin{tabular}{l}
|\input{childdoc.def}|\\
|\childdocby{|\textit{main}|}|\\
\end{tabular}
\end{center}
%
The directive |\childdocby| is similar to |\childdocof|
described in \secref{sec:include},
but the subsequent selection of content must be done manually.
To that end, both |\ifchilddoc| and |\ifchilddocmanual|
will be true upon processing of a part,
and the name of the part is stored in |\childdocname|.
Note that |\jobname| will be set to the filename of the current part
so that each part receives an individual |.aux| file
that does not interfere with the |.aux| file(s) of the main document.
This behaviour can be altered by the alternative form
|\childdocby[*]{|\textit{main}|}| (with a non-empty optional argument)
which uses the |.aux| file of the main document
by setting |\jobname| to \textit{main}.

%%%%%%%%%%%%%%%%%%%%%%%%%%%%%%%%%%%%%%%%%%%%%%%%%%%%%%%%%%%%%%%%%%%%%%%%%%%%%%%%
\subsection{Driver Development}
\label{sec:driver}

The \textsf{childdoc} mechanism can also be use for the development
of definition files such as \LaTeX{} styles or classes.
This case differs from the above setup with multiple parts
included by |\include| in that no |\includeonly| should be invoked.
This can be achieved by starting the include file
(before |\ProvidesPackage|) with:
%
\begin{center}
\begin{tabular}{l}
|\input{childdoc.def}|\\
|\childdocforward{|\textit{main}|}|\\
\end{tabular}
\end{center}
%
or alternatively with:
%
\begin{center}
\begin{tabular}{l}
|\input{childdoc.def}|\\
|\childdocby{|\textit{main}|}|\\
\end{tabular}
\end{center}
%
Both forms have slightly different effects as described above.
The main file is prepared as usual, see \secref{sec:include}.

%%%%%%%%%%%%%%%%%%%%%%%%%%%%%%%%%%%%%%%%%%%%%%%%%%%%%%%%%%%%%%%%%%%%%%%%%%%%%%%%
\subsection{Legacy Detection}
\label{sec:detection}

The directive |\childdocmain| in the main file can detect
whether the complete document or merely a child is to be compiled
even without using the directive |\childdocof|.
This method is deprecated because it is less robust
and there is no compelling reason to use it;
it is merely provided for backward compatibility
and it may be removed in future versions.

If the detection mechanism is to be used,
it is mandatory to correctly specify
the filename of the main file as the argument of |\childdocmain|:
%
\begin{center}
\begin{tabular}{l}
|\input{childdoc.def}|\\
|\childdocmain{|\textit{main}|}|\\
\end{tabular}
\end{center}
%
If |\jobname| does not match the argument \textit{main} of |\childdocmain|,
it is assumed that |\jobname| points to the child file to be compiled.
When using |\childdocmain| with the main file specified as argument,
it suffices to start a child file
with just |\input{|\textit{main}|}|
without loading of the package and using |\childdocof|.
If instead all processing is done
with the appropriate \textsf{childdoc} directives,
the argument of \textit{main} of |\childdocmain| can be empty.

An alternative version of the command line processing described
in \secref{sec:commandline} using the detection mechanism reads:
%
\begin{center}
|... -jobname "|\textit{target}|" "|[\textit{flags}]%
[|\def\jobname{|\textit{dest}|}|]|\input{|\textit{main}|}"|
\end{center}

%%%%%%%%%%%%%%%%%%%%%%%%%%%%%%%%%%%%%%%%%%%%%%%%%%%%%%%%%%%%%%%%%%%%%%%%%%%%%%%%
\subsection{Manual Code}
\label{sec:manual}

In case one cannot be certain whether the definitions file |childdoc.def|
is installed on the target \TeX{} distribution
and one prefers not to ship it,
it is conceivable to paste a few relevant commands into the sources.

To that end, drop all statements |\input{childdoc.def}|
and perform the replacements as outlined below.
Instead of |\childdocmain{|\textit{main}|}| add the following code
to the top of the main file:
%
\begin{center}
\begin{tabular}{l}
|\||ifdefined\childdocname\endinput\||fi\newif\ifchilddoc|\\
|\edef\childdocname{\scantokens\expandafter{\jobname\noexpand}}|\\
|\def\childdocmain{|\textit{main}|}\||ifx\childdocmain\childdocname\||else|\\
|\childdoctrue\includeonly{\childdocname}\let\jobname\childdocmain\||fi|\\
\end{tabular}
\end{center}
%
Instead of |\childdocof{|\textit{main}|}| just include the main file
at the top of each child file:
%
\begin{center}
|\input{|\textit{main}|}|
\end{center}
%
A simple redirection |\childdocforward{|\textit{dest}|}| is achieved by:
%
\begin{center}
|\def\jobname{|\textit{dest}|}\input{\jobname}|
\end{center}
%
The redirection with prefix
|\childdocforwardprefix[|\textit{prefix}|]{|\textit{dest}|}|
is accomplished by:
%
\begin{center}
\begin{tabular}{l}
|{\edef\jobname{\scantokens\expandafter{\jobname\noexpand}}|\\
|\def\redirectjob |\textit{prefix}|#1~~~{\gdef\jobname{|\textit{dest}|#1}}|\\
|\expandafter\redirectjob\jobname~~~}\input{\jobname}|
\end{tabular}
\end{center}

In an alternative approach,
child documents can be compiled by a specific command line
without additional code or specific definitions:
%
\begin{center}
|... -jobname "|\textit{target}|" "|[\textit{flags}]%
|\includeonly{|\textit{dest}|}\input{|\textit{main}|}"|
\end{center}
%

%%%%%%%%%%%%%%%%%%%%%%%%%%%%%%%%%%%%%%%%%%%%%%%%%%%%%%%%%%%%%%%%%%%%%%%%%%%%%%%%
%%%%%%%%%%%%%%%%%%%%%%%%%%%%%%%%%%%%%%%%%%%%%%%%%%%%%%%%%%%%%%%%%%%%%%%%%%%%%%%%
\section{Information}

%%%%%%%%%%%%%%%%%%%%%%%%%%%%%%%%%%%%%%%%%%%%%%%%%%%%%%%%%%%%%%%%%%%%%%%%%%%%%%%%
\subsection{Copyright}

Copyright \copyright{} 2017--2018 Niklas Beisert

This work may be distributed and/or modified under the
conditions of the \LaTeX{} Project Public License, either version 1.3
of this license or (at your option) any later version.
The latest version of this license is in
  \url{http://www.latex-project.org/lppl.txt}
and version 1.3 or later is part of all distributions of \LaTeX{}
version 2005/12/01 or later.

This work has the LPPL maintenance status `maintained'.

The Current Maintainer of this work is Niklas Beisert.

This work consists of the files |README.txt|, |childdoc.ins| and |childdoc.dtx|
as well as the derived files |childdoc.def|, |cdocsamp.tex|
with |cdocsch1.tex|, |cdocsch2.tex|, |cdocspt3.tex|, |cdocspt4.tex|,
|cdocsdrf.tex|, |cdocsfn1.tex|, |cdocsfn2.tex|
as well as |childdoc.pdf|.

%%%%%%%%%%%%%%%%%%%%%%%%%%%%%%%%%%%%%%%%%%%%%%%%%%%%%%%%%%%%%%%%%%%%%%%%%%%%%%%%
\subsection{Files and Installation}

The package consists of the files:
%
\begin{center}
\begin{tabular}{ll}
    |README.txt|   & readme file \\
    |childdoc.ins| & installation file \\
    |childdoc.dtx| & source file \\
    |childdoc.def| & definition file \\
    |cdocsamp.tex| & sample main file \\
    |cdocsch1.tex| & sample include file \\
    |cdocsch2.tex| & sample include file \\
    |cdocspt3.tex| & sample part file \\
    |cdocspt4.tex| & sample part file \\
    |cdocsdrf.tex| & sample redirection file \\
    |cdocsfn1.tex| & sample redirection file \\
    |cdocsfn2.tex| & sample redirection file \\
    |childdoc.pdf| & manual
\end{tabular}
\end{center}
%
The distribution consists of the files
|README.txt|, |childdoc.ins| and |childdoc.dtx|.
%
\begin{itemize}
\item
Run (pdf)\LaTeX{} on |childdoc.dtx|
to compile the manual |childdoc.pdf| (this file).
\item
Run \LaTeX{} on |childdoc.ins| to create the definitions file |childdoc.def|
and the sample |cdocsamp.tex| with include files
|cdocsch1.tex|, |cdocsch2.tex|, |cdocspt3.tex|, |cdocspt4.tex|,
|cdocsdrf.tex|, |cdocsfn1.tex|, |cdocsfn2.tex|.
Then copy the file |childdoc.def| to an appropriate directory of your \LaTeX{}
distribution, e.g.\ \textit{texmf-root}|/tex/latex/childdoc|.
\end{itemize}

%%%%%%%%%%%%%%%%%%%%%%%%%%%%%%%%%%%%%%%%%%%%%%%%%%%%%%%%%%%%%%%%%%%%%%%%%%%%%%%%
\subsection{Related CTAN Packages}

There are several other packages which offer a similar functionality:
%
\begin{itemize}
\item
The packages
\href{http://ctan.org/pkg/docmute}{\textsf{docmute}},
\href{http://ctan.org/pkg/includex}{\textsf{includex}} and
\href{http://ctan.org/pkg/standalone}{\textsf{standalone}}
provide commands to include only the document body of
a child file thus allowing both files to be compiled individually.
\item
The packages \href{http://ctan.org/pkg/subdocs}{\textsf{subdocs}}
and \href{http://ctan.org/pkg/subfiles}{\textsf{subfiles}}
provide structures in which the main and child documents can be
encapsulated and allowing them to be compiled individually.
The inclusion mechanism is different from the conventional |\include|.
\item
The package \href{http://ctan.org/pkg/combine}{\textsf{combine}}
is an elaborate solution to combine several documents into one.
\end{itemize}
%
See also the CTAN topic \href{http://ctan.org/topic/subdocs}{\textsf{subdocs}}
for further related packages.
The present package differs from the above solutions in that
a document structure constructed with the conventional |\include| mechanism
just needs two extra commands at the top of every file
such that all constituent files can be compiled individually.

%%%%%%%%%%%%%%%%%%%%%%%%%%%%%%%%%%%%%%%%%%%%%%%%%%%%%%%%%%%%%%%%%%%%%%%%%%%%%%%%
%\subsection{Feature Suggestions}
%
%The following is a list of features which may be useful for future
%versions of this package:
%%
%\begin{itemize}
%\item
%\ldots
%\end{itemize}

%%%%%%%%%%%%%%%%%%%%%%%%%%%%%%%%%%%%%%%%%%%%%%%%%%%%%%%%%%%%%%%%%%%%%%%%%%%%%%%%
\subsection{Revision History}

%%%%%%%%%%%%%%%%%%%%%%%%%%%%%%%%%%%%%%%%
\paragraph{v2.0:} 2018/12/30

\begin{itemize}
\item
immediate forward processing
\item
added |\childdocby| mechanism
\item
manual restructured
\end{itemize}

%%%%%%%%%%%%%%%%%%%%%%%%%%%%%%%%%%%%%%%%
\paragraph{v1.6:} 2018/01/17

\begin{itemize}
\item
application for development of include files
\item
corrections to manual
\end{itemize}

%%%%%%%%%%%%%%%%%%%%%%%%%%%%%%%%%%%%%%%%
\paragraph{v1.5:} 2017/05/21

\begin{itemize}
\item
more complete structuring introduced
\item
|\childdocof| introduced
\item
|\childdoc| renamed to |\childdocmain|
\item
|\childredirect| renamed to |\childdocforward| and |\childdocforwardprefix|
and functionality expanded
\end{itemize}

%%%%%%%%%%%%%%%%%%%%%%%%%%%%%%%%%%%%%%%%
\paragraph{v1.0:} 2017/04/27

\begin{itemize}
\item
manual and install package
\item
first version published on CTAN
\end{itemize}

%%%%%%%%%%%%%%%%%%%%%%%%%%%%%%%%%%%%%%%%
\paragraph{v0.6:} 2017/04/26

\begin{itemize}
\item
redirection mechanism added
\end{itemize}

%%%%%%%%%%%%%%%%%%%%%%%%%%%%%%%%%%%%%%%%
\paragraph{v0.5:} 2017/04/26

\begin{itemize}
\item
functionality in definition file
\end{itemize}


%%%%%%%%%%%%%%%%%%%%%%%%%%%%%%%%%%%%%%%%%%%%%%%%%%%%%%%%%%%%%%%%%%%%%%%%%%%%%%%%
%%%%%%%%%%%%%%%%%%%%%%%%%%%%%%%%%%%%%%%%%%%%%%%%%%%%%%%%%%%%%%%%%%%%%%%%%%%%%%%%
%%%%%%%%%%%%%%%%%%%%%%%%%%%%%%%%%%%%%%%%%%%%%%%%%%%%%%%%%%%%%%%%%%%%%%%%%%%%%%%%
\appendix

\settowidth\MacroIndent{\rmfamily\scriptsize 000\ }

 \DocInput{childdoc.dtx}

\end{document}
%</driver>
% \fi
%
% %%%%%%%%%%%%%%%%%%%%%%%%%%%%%%%%%%%%%%%%%%%%%%%%%%%%%%%%%%%%%%%%%%%%%%%%%%%%%%
% %%%%%%%%%%%%%%%%%%%%%%%%%%%%%%%%%%%%%%%%%%%%%%%%%%%%%%%%%%%%%%%%%%%%%%%%%%%%%%
% \section{Sample}
%\iffalse
%<*samplemain>
%\fi
%
% The following presents a sample document
% with two chapters, two parts, a title page,
% a compile flag as well as three forwarding files to set the flag.
% It consists of eight |.tex| files:
% \begin{center}
% \begin{tabular}{ll}
% |cdocsamp.tex|&main file\\
% |cdocsch1.tex|&include file for chapter 1\\
% |cdocsch2.tex|&include file for chapter 2\\
% |cdocspt3.tex|&include file for part 3\\
% |cdocspt4.tex|&include file for part 4\\
% |cdocsdrf.tex|&forwarding file for main file in draft mode\\
% |cdocsfi1.tex|&forwarding file for final version of chapter 1\\
% |cdocsfi2.tex|&forwarding file for final version of chapter 2\\
% \end{tabular}
% \end{center}
% Each of the eight files can be compiled directly by the \LaTeX{} compiler.
%
% %%%%%%%%%%%%%%%%%%%%%%%%%%%%%%%%%%%%%%
% \paragraph{Main File.}
%
% The main file is called |cdocsamp.tex|.
%
% Load the \textsf{childdoc} definitions and
% declare the filename for the main document:
%    \begin{macrocode}
\input{childdoc.def}
\childdocmain{}
%    \end{macrocode}

% Optional override for |\version| flag:
%    \begin{macrocode}
%%\ifchilddoc\else\providecommand{\version}{draft}\fi
%    \end{macrocode}

% Define the default values for the |\version| flag
% (|final| for the main file and |draft| for childs):
%    \begin{macrocode}
\ifchilddoc
\providecommand{\version}{draft}
\else
\providecommand{\version}{final}
\fi
%    \end{macrocode}

% Load the standard document class:
%    \begin{macrocode}
\documentclass[12pt]{article}
%    \end{macrocode}

% Start the document body:
%    \begin{macrocode}
\begin{document}
%    \end{macrocode}

% Declare a title page.
% Print title, part of document being processed and version flag:
%    \begin{macrocode}
\addtocounter{page}{-1}
\begin{center}
{\LARGE\bfseries{}childdoc example\par}
\vspace{1cm}
\ifchilddoc
\ifchilddocmanual part\else chapter\fi:
`\childdocname' of `\childdocjob'\par
\else
main document: `\childdocjob'\par
\fi
version: \version\par
\end{center}
\newpage
%    \end{macrocode}

% Manually include selected file,
% otherwise process as usual:
%    \begin{macrocode}
\ifchilddocmanual
\section*{part `\childdocname'}
\input{\childdocname}
\else
%    \end{macrocode}

% Include the two chapters:
%    \begin{macrocode}
\include{cdocsch1}
\include{cdocsch2}
%    \end{macrocode}

% Include the two parts unless only chapters should be displayed:
%    \begin{macrocode}
\ifchilddoc\else
\section{part three}
\input{cdocspt3}
\section{part four}
\input{cdocspt4}
\fi
%    \end{macrocode}

% Process as usual until here:
%    \begin{macrocode}
\fi
%    \end{macrocode}

% End of document body:
%    \begin{macrocode}
\end{document}
%    \end{macrocode}
%\iffalse
%</samplemain>
%\fi
%
% %%%%%%%%%%%%%%%%%%%%%%%%%%%%%%%%%%%%%%
% \paragraph{Chapter Include Files.}
%
% The include files are called |cdocsch1.tex| and |cdocsch2.tex|.
%
%\iffalse
%<*samplechap1|samplechap2>
%\fi

% Optional override for |\version| flag:
%    \begin{macrocode}
%%\providecommand{\version}{final}
%    \end{macrocode}

% Include the main document:
%    \begin{macrocode}
\input{childdoc.def}
\childdocof{cdocsamp}
%    \end{macrocode}

%\iffalse
%</samplechap1|samplechap2>
%\fi
%
%\iffalse
%<*samplechap1>
%\fi
% Some text for chapter 1:
%    \begin{macrocode}
\section{one}
some text in chapter one
%    \end{macrocode}

%\iffalse
%</samplechap1>
%\fi
% Some text for chapter 2:
%\iffalse
%<*samplechap2>
%\fi
%    \begin{macrocode}
\section{two}
more text in chapter two
%    \end{macrocode}

%\iffalse
%</samplechap2>
%\fi
%
% %%%%%%%%%%%%%%%%%%%%%%%%%%%%%%%%%%%%%%
% \paragraph{Part Include Files.}
%
% The include files are called |cdocspt3.tex| and |cdocspt4.tex|.
%
%\iffalse
%<*samplepart3|samplepart4>
%\fi

% Optional override for |\version| flag:
%    \begin{macrocode}
%%\providecommand{\version}{final}
%    \end{macrocode}

% Include the main document:
%    \begin{macrocode}
\input{childdoc.def}
\childdocby{cdocsamp}
%    \end{macrocode}

%\iffalse
%</samplepart3|samplepart4>
%\fi
%
%\iffalse
%<*samplepart3>
%\fi
% Some text for part 3:
%    \begin{macrocode}
some text in part three
%    \end{macrocode}

%\iffalse
%</samplepart3>
%\fi
% Some text for part 4:
%\iffalse
%<*samplepart4>
%\fi
%    \begin{macrocode}
more text in part four
%    \end{macrocode}

%\iffalse
%</samplepart4>
%\fi
%
% %%%%%%%%%%%%%%%%%%%%%%%%%%%%%%%%%%%%%%
% \paragraph{Forwarding for a Complete Draft.}
%
% The following forwarding file |cdocsdrf.tex|
% compiles the main document in draft mode:
%\iffalse
%<*sampledraft>
%\fi
%    \begin{macrocode}
\def\version{draft}
\input{childdoc.def}
\childdocforward{cdocsamp}
%    \end{macrocode}

%\iffalse
%</sampledraft>
%\fi
%
% %%%%%%%%%%%%%%%%%%%%%%%%%%%%%%%%%%%%%%
% \paragraph{Forwarding for Final Version of the Chapters.}
%
% The following forwarding files |cdocsfn1.tex| and |cdocsfn2.tex|
% (with identical content)
% compile the final versions of the child documents
% |cdocsch1.tex| and |cdocsch2.tex|, respectively:
%\iffalse
%<*samplefinal>
%\fi
%    \begin{macrocode}
\def\version{final}
\input{childdoc.def}
\childdocforwardprefix[cdocsamp]{cdocsfn}{cdocsch}
%    \end{macrocode}

%\iffalse
%</samplefinal>
%\fi
%
% %%%%%%%%%%%%%%%%%%%%%%%%%%%%%%%%%%%%%%
% \paragraph{Command Line Processing.}
%
% The following three command lines generate the output files
% |cdocscld|, |cdocscl1| and |cdocscl2|
% which should be identical to
% |cdocsdrf|, |cdocsch1| and |cdocsfn2|, respectively:
% \begin{center}
% \begin{tabular}{l}
% |latex -jobname cdocscld \|\\
% |  "\def\version{draft}\input{childdoc.def}\childdocforward{cdocsamp}"|\\
% |latex -jobname cdocscl1 \|\\
% |  "\input{childdoc.def}\childdocforward[cdocsamp]{cdocsch1}"|\\
% |latex -jobname cdocscl2 \|\\
% |  "\def\version{final}\input{childdoc.def}\childdocforward{cdocsch2}"|
% \end{tabular}
% \end{center}
% Note that the trailing backslash on each first line
% merely continues the input to the second line
% (for convenient cut ant paste).
% Furthermore, the command |latex| can be replaced by any
% of its alternative versions such as |pdflatex|.
%
% %%%%%%%%%%%%%%%%%%%%%%%%%%%%%%%%%%%%%%%%%%%%%%%%%%%%%%%%%%%%%%%%%%%%%%%%%%%%%%
% %%%%%%%%%%%%%%%%%%%%%%%%%%%%%%%%%%%%%%%%%%%%%%%%%%%%%%%%%%%%%%%%%%%%%%%%%%%%%%
% \section{Implementation}
%\iffalse
%<*package>
%\fi
%
% This section describes the definitions file |childdoc.def|.

% The definitions cannot be loaded using |\usepackage| or |\RequirePackage|
% which has a mechanism to prevent loading a style file more than once.
% When loading the definitions by means of |\input|
% multiple instances have to be prevented manually:
%\iffalse
%This code needs to be before the `\ProvidesFile' directive
%which is defined at the beginning of this file.
%Therefore it is also placed there and commented out here.
%</package>
%<*discard>
%\fi
%    \begin{macrocode}
\ifdefined\childdocmain\endinput\fi
%    \end{macrocode}
%\iffalse
%</discard>
%<*package>
%\fi
%
% \macro{\ifchilddoc}
% \macro{\ifchilddocmanual}
% The conditional |\ifchilddoc| tells whether a
% child (true) or main (false) document is being compiled.
% The conditional |\ifchilddocmanual| tells whether
% the |\includeonly| mechanism is used (false) or
% the selection of child files must be performed manually (true).
% The definitions initialise to false:
%    \begin{macrocode}
\newif\ifchilddoc
\newif\ifchilddocmanual
%    \end{macrocode}

% \macro{\childdocname}
% \macro{\childdocjob}
% The macro |\childdocname| stores the name of the main document
% to be compiled. The macro |\childdocjob| stores the name of
% the document on which the \LaTeX{} compiler was originally invoked.
% The content of |\jobname| cannot be compared
% to filenames specified in the source due to different catcodes.
% The following code rescans |\jobname|, stores the result
% in |\childdocname| and saves a copy in |\childdocjob|:
%    \begin{macrocode}
\edef\childdocname{\scantokens\expandafter{\jobname\noexpand}}
\let\childdocjob\childdocname
%    \end{macrocode}

% \macro{\childdocdisable}
% The macro |\childdocdisable| prevents the main file
% from being processed more than once.
% At this stage, the main document command |\childdocmain|
% is assumed to be called once again where it should do nothing.
% Any subsequent call to it should prevent
% a secondary processing of the main document
% It overwrites the forwarding commands
% |\childdocof| and |\childdocforward|
% with empty macros to prevent further inclusions of the main document:
%    \begin{macrocode}
\newcommand{\childdocdisable}
{
  \renewcommand{\childdocmain}[1]{\renewcommand{\childdocmain}[1]{\endinput}}
  \renewcommand{\childdocof}[1]{}
  \renewcommand{\childdocby}[2][]{}
  \renewcommand{\childdocforward}[2][]{}
  \renewcommand{\childdocdisable}{}
}
%    \end{macrocode}

% \macro{\childdocmain}
% The macro |\childdocmain| is to be called at the top of the main file
% with nothing or the main filename (without extension) as argument.
% First, it breaks loops.
% If the argument is not empty and does not match |\childdocname|
% (which is set by the first inclusion of |childdoc.def|),
% |\ifchilddoc| is set to true, |\includeonly| is applied to the child file
% and |\jobname| is set to the main file
% (for proper handling of |.aux| files):
%    \begin{macrocode}
\newcommand{\childdocmain}[1]
{
  \childdocdisable\childdocmain{}
  \if?#1?\else
    \begingroup
      \def\childdoctmp{#1}
      \ifx\childdoctmp\childdocname
        \def\childdoctmp{}
      \else
        \def\childdoctmp
        {
          \childdoctrue
          \includeonly{\childdocname}
          \def\childdocjob{#1}
          \def\jobname{#1}
        }
      \fi
      \expandafter
    \endgroup
    \childdoctmp
  \fi
}
%    \end{macrocode}

% \macro{\childdocof}
% The command |\childdocof| redirects
% compilation to the main file |#1|.
%    \begin{macrocode}
\newcommand{\childdocof}[1]
{
  \childdocdisable
  \childdoctrue
  \includeonly{\childdocname}
  \def\jobname{#1}
  \def\childdocjob{#1}
  \input{#1}
}
%    \end{macrocode}

% \macro{\childdocby}
% The command |\childdocby| ....
%    \begin{macrocode}
\newcommand{\childdocby}[2][]
{
  \childdocdisable
  \childdoctrue
  \childdocmanualtrue
  \if?#1?\else
    \def\jobname{#2}
  \fi
  \def\childdocjob{#2}
  \input{#2}
  \endinput
}
%    \end{macrocode}

% \macro{\childdocforward}
% The command |\childdocforward| redirects
% compilation to the main file or
% (if the optional argument is given) a child file.
% Parameters are set as if the main file
% or a child file starting with |\childdocof| was compiled.
% Then compilation is handed over to the main file:
%    \begin{macrocode}
\newcommand{\childdocforward}[2][]
{
  \begingroup
    \if?#1?
      \def\childdoctmp
      {
        \def\childdocname{#2}
        \def\childdocjob{#2}
        \def\jobname{#2}
        \input{#2}
        \endinput
      }
    \else
      \def\childdoctmp
      {
        \childdocdisable
        \def\childdocname{#2}
        \childdoctrue
        \includeonly{#2}
        \def\childdocjob{#1}
        \def\jobname{#1}
        \input{#1}
        \endinput
      }
    \fi
    \expandafter
  \endgroup
  \childdoctmp
}
%    \end{macrocode}

% \macro{\childdocforwardprefix}
% The command |\childdocforwardprefix| redirects
% compilation to the main or a child file by means of a pattern.
% The prefix |#1| in the current filename is replaced by |#2|
% and the suffix of the current filename is kept
% (it is assumed that the filename does not contain the substring `|~~~|'
% which is used as a delimiter).
% Compilation is handed over to the new file by |\childdocforward|:
%    \begin{macrocode}
\newcommand{\childdocforwardprefix}[3][]
{
  \begingroup
    \def\childdocextract #2##1~~~{\def\childdoctmp{\childdocforward[#1]{#3##1}}}
    \expandafter\childdocextract\childdocname~~~
    \expandafter
  \endgroup
  \childdoctmp
}
%    \end{macrocode}

% \macro{\childdoc}
% The deprecated macro |\childdoc| is a legacy version of |\childdocmain|:
%    \begin{macrocode}
\newcommand{\childdoc}{\childdocmain}
%    \end{macrocode}

% \macro{\childdocredirect}
% The deprecated macro |\childdocredirect| is a legacy version
% of |\childdocforward| and |\childdocforwardprefix|:
%    \begin{macrocode}
\newcommand{\childdocredirect}[2][]
{
  \begingroup
    \if?#1?
      \def\childdoctmp{\childdocforward{#2}}
    \else
      \def\childdoctmp{\childdocforwardprefix{#1}{#2}}
    \fi
    \expandafter
  \endgroup
  \childdoctmp
}
%    \end{macrocode}

%\iffalse
%</package>
%\fi
%
\endinput
|\\
|\childdocmain{|\textit{main}|}|\\
\end{tabular}
\end{center}
%
If |\jobname| does not match the argument \textit{main} of |\childdocmain|,
it is assumed that |\jobname| points to the child file to be compiled.
When using |\childdocmain| with the main file specified as argument,
it suffices to start a child file
with just |\input{|\textit{main}|}|
without loading of the package and using |\childdocof|.
If instead all processing is done
with the appropriate \textsf{childdoc} directives,
the argument of \textit{main} of |\childdocmain| can be empty.

An alternative version of the command line processing described
in \secref{sec:commandline} using the detection mechanism reads:
%
\begin{center}
|... -jobname "|\textit{target}|" "|[\textit{flags}]%
[|\def\jobname{|\textit{dest}|}|]|\input{|\textit{main}|}"|
\end{center}

%%%%%%%%%%%%%%%%%%%%%%%%%%%%%%%%%%%%%%%%%%%%%%%%%%%%%%%%%%%%%%%%%%%%%%%%%%%%%%%%
\subsection{Manual Code}
\label{sec:manual}

In case one cannot be certain whether the definitions file |childdoc.def|
is installed on the target \TeX{} distribution
and one prefers not to ship it,
it is conceivable to paste a few relevant commands into the sources.

To that end, drop all statements |% \iffalse
%
% childdoc.dtx Copyright (C) 2017-2018 Niklas Beisert
%
% This work may be distributed and/or modified under the
% conditions of the LaTeX Project Public License, either version 1.3
% of this license or (at your option) any later version.
% The latest version of this license is in
%   http://www.latex-project.org/lppl.txt
% and version 1.3 or later is part of all distributions of LaTeX
% version 2005/12/01 or later.
%
% This work has the LPPL maintenance status `maintained'.
%
% The Current Maintainer of this work is Niklas Beisert.
%
% This work consists of the files childdoc.dtx and childdoc.ins
% and the derived files childdoc.def and cdocsamp.tex with
% cdocsch1.tex, cdocsch2.tex, cdocsdrf.tex, cdocsfn1.tex, cdocsfn2.tex.
%
%<package>\ifdefined\childdocmain\endinput\fi
%<package>\ProvidesFile{childdoc.def}[2018/12/30 v2.0 child document driver]
%<samplemain>\ProvidesFile{cdocsamp.tex}[2018/12/30 v2.0 sample for childdoc]
%<*driver>
%\ProvidesFile{childdoc.drv}[2018/12/30 v2.0 childdoc reference manual file]
\PassOptionsToClass{10pt,a4paper}{article}
\documentclass{ltxdoc}

\usepackage[margin=35mm]{geometry}
\usepackage{hyperref}
\usepackage{hyperxmp}
\usepackage[usenames]{color}

\hypersetup{colorlinks=true}
\hypersetup{pdfstartview=FitH}
\hypersetup{pdfpagemode=UseNone}
\hypersetup{pdfsource={}}
\hypersetup{pdflang={en-UK}}
\hypersetup{pdfcopyright={Copyright 2017-2018 Niklas Beisert.
  This work may be distributed and/or modified under the
  conditions of the LaTeX Project Public License, either version 1.3
  of this license or (at your option) any later version.}}
\hypersetup{pdflicenseurl={http://www.latex-project.org/lppl.txt}}
\hypersetup{pdfcontactaddress={ETH Zurich, ITP, HIT K,
  Wolfgang-Pauli-Strasse 27}}
\hypersetup{pdfcontactpostcode={8093}}
\hypersetup{pdfcontactcity={Zurich}}
\hypersetup{pdfcontactcountry={Switzerland}}
\hypersetup{pdfcontactemail={nbeisert@itp.phys.ethz.ch}}
\hypersetup{pdfcontacturl={http://people.phys.ethz.ch/\xmptilde nbeisert/}}

\newcommand{\secref}[1]{\hyperref[#1]{section \ref*{#1}}}

\parskip1ex
\parindent0pt
\let\olditemize\itemize
\def\itemize{\olditemize\parskip0pt}

\begin{document}

\title{The \textsf{childdoc} Package}
\hypersetup{pdftitle={The childdoc Package}}
\author{Niklas Beisert\\[2ex]
  Institut f\"ur Theoretische Physik\\
  Eidgen\"ossische Technische Hochschule Z\"urich\\
  Wolfgang-Pauli-Strasse 27, 8093 Z\"urich, Switzerland\\[1ex]
  \href{mailto:nbeisert@itp.phys.ethz.ch}
  {\texttt{nbeisert@itp.phys.ethz.ch}}}
\hypersetup{pdfauthor={Niklas Beisert}}
\hypersetup{pdfsubject={Manual for the LaTeX2e Package childdoc}}
\date{30 December 2018, \textsf{v2.0}}
\maketitle

\begin{abstract}\noindent
\textsf{childdoc} is a \LaTeXe{} package
that enables the direct compilation
of document sections included by |\include|
to individual files.
\end{abstract}

\begingroup
\parskip0ex
\tableofcontents
\endgroup

%%%%%%%%%%%%%%%%%%%%%%%%%%%%%%%%%%%%%%%%%%%%%%%%%%%%%%%%%%%%%%%%%%%%%%%%%%%%%%%%
%%%%%%%%%%%%%%%%%%%%%%%%%%%%%%%%%%%%%%%%%%%%%%%%%%%%%%%%%%%%%%%%%%%%%%%%%%%%%%%%
\section{Introduction}

\LaTeX{} provides a mechanism to structure a large document (such as a book)
into a main file and several child files (containing the chapters)
using the |\include| command.
This mechanism is beneficial for documents
which span hundreds of pages in order to
make the source file(s) more manageable.
Moreover, compilation can be restricted to
selected child files by means of the |\includeonly| command.
The latter feature can be used to reduce the compilation time while editing
(this was significantly more useful in the earlier days of \LaTeX{})
or to generate a smaller document which is easier to navigate.
Another application of |\includeonly| is to generate
documents consisting of selected parts of the complete document.

However, there are a few drawbacks of the plain |\include| mechanism:
\begin{itemize}
\item
The child files cannot be compiled on their own,
they can only be compiled via the main file.
A naive editing environment
(such as a text editor with an option
to have the current file processed by \LaTeX)
may require one to switch to the main file before compiling;
attempting to compile the child file produces errors.
\item
The main file must be modified (each time)
to adjust the |\includeonly| command
to the present needs. This easily leaves the main file in a messy state.
\item
The generated document will always carry the filename
of the main document. This is inconvenient if
several child files are to be compiled and
to be kept for distribution.
\end{itemize}

The present package provides a simple interface
to make child files individually compilable by \LaTeX{}.
Compiling a child file then has the same effect as compiling
the main file with an |\includeonly| command
to select the appropriate child.
Moreover the generated document will carry the name of the child
rather than the main file.
This resolves all three above issues.

This feature is meant to make the editing of books,
thesis documents and lecture notes somewhat more convenient.
However, the package can also be used efficiently for
composing a series of documents (such as exercise sheets)
which are typically distributed individually.
It then assists the author in generating the individual documents
(potentially in different versions)
as well as a document containing the collected series.
Another application is in developing style files
or other kinds of included material
where compilation of the style file could redirect
to a sample or test file.

%%%%%%%%%%%%%%%%%%%%%%%%%%%%%%%%%%%%%%%%%%%%%%%%%%%%%%%%%%%%%%%%%%%%%%%%%%%%%%%%
%%%%%%%%%%%%%%%%%%%%%%%%%%%%%%%%%%%%%%%%%%%%%%%%%%%%%%%%%%%%%%%%%%%%%%%%%%%%%%%%
\section{Usage}

First of all, the package \textsf{childdoc} is \emph{not} a standard
\LaTeXe{} |.sty| style file! Therefore it needs to be invoked in
a non-standard way.

%%%%%%%%%%%%%%%%%%%%%%%%%%%%%%%%%%%%%%%%%%%%%%%%%%%%%%%%%%%%%%%%%%%%%%%%%%%%%%%%
\subsection{Included Files}
\label{sec:include}

%%%%%%%%%%%%%%%%%%%%%%%%%%%%%%%%%%%%%%%%
\DescribeMacro{\childdocmain}
To use the package, add the commands
\begin{center}
\begin{tabular}{l}
|\input{childdoc.def}|\\
|\childdocmain{}|\\
\end{tabular}
\end{center}
at the very top of the main \LaTeX{} file,
in particular \emph{before} the |\documentclass| statement!
The argument of |\childdocmain| should be left empty
(but it must be present).

%%%%%%%%%%%%%%%%%%%%%%%%%%%%%%%%%%%%%%%%
\DescribeMacro{\childdocof}
Furthermore, add the commands
\begin{center}
\begin{tabular}{l}
|\input{childdoc.def}|\\
|\childdocof{|\textit{main}|}|\\
\end{tabular}
\end{center}
at the top of every child file \textit{child}
which is included by |\include{|\textit{child}|}|
from within the main file
(or at least for those files to be compiled individually).
The argument \textit{main} must be the filename of the main file.

There are a couple of
considerations in setting up the main and child documents:

%%%%%%%%%%%%%%%%%%%%%%%%%%%%%%%%%%%%%%%%
\paragraph{Restrictions.}

Please note the following restrictions:
\begin{itemize}
\item
|\childdocmain| must be called with one argument \textit{main}
to ensure compatibility with earlier version of the package.
It must either be empty (|\childdocmain{}|)
or precisely match the filename of the main file in which it is specified.
See \secref{sec:detection} for further information.
\item
The filename \textit{main} must be specified without the |.tex| extension.
\item
The filename \textit{main} is case sensitive
(even in case-insensitive file systems)
due to internal string comparison.
\item
The argument \textit{main} should be fully expanded, it cannot be a macro.
\item
Subdirectories and special characters should be avoided in filenames.
\item
The command |\childdocmain{|\textit{main}|}| must be followed by a whitespace.
It should not be followed immediately by another command
or by a comment mark `|%|'.
This is because the \TeX{} parser reads the token immediately following
the argument of |\childdocmain| and puts it
at the beginning of every child section;
however, a white\-space is ignored.
\end{itemize}

%%%%%%%%%%%%%%%%%%%%%%%%%%%%%%%%%%%%%%%%
\paragraph{Content of Main File.}

It is advisable to place all content in the child files included by |\include|.
Any output contained in the main file will appear in all child documents
unless suppressed manually;
it cannot be suppressed automatically by the |\includeonly| directive
and thus should normally be avoided.
A method to include some content in the main file
by means of conditional processing is described in \secref{sec:conditional}.

%%%%%%%%%%%%%%%%%%%%%%%%%%%%%%%%%%%%%%%%
\paragraph{Page Numbering.}

When only a part of the document is compiled,
the appropriate numbering of pages
(as well as other status parameters)
is determined from the |.aux| files.
The latter contain information from previous passes.
However this information needs to propagate through
all intermediate child documents.
Therefore the page numbering in child documents may well
be inconsistent until the complete document is compiled at least once.

A useful (if unconventional) way to always ensure a consistent
page numbering is to restart the numbering in each child document
and denote the pages by `\textit{child}|.|\textit{page}'
where \textit{child} represents the chapter/section number of the child file.
This can be achieved by the command
|\numberwithin{page}{|\textit{child}|}|
of the \textsf{amsmath} package
where \textit{child} can be |chapter| or |section|
depending on the chosen structuring.
Alternatively, one can modify the macro |\thepage| appropriately
and reset the counter |page| at the start of each child file.

%%%%%%%%%%%%%%%%%%%%%%%%%%%%%%%%%%%%%%%%%%%%%%%%%%%%%%%%%%%%%%%%%%%%%%%%%%%%%%%%
\subsection{Conditional Processing}
\label{sec:conditional}

The package provides a mechanism to compile different versions
of a document. To customise the versions further some conditional processing
can come in handy to distinguish which version is being compiled.
The package provides two macros to describe the compilation context:

%%%%%%%%%%%%%%%%%%%%%%%%%%%%%%%%%%%%%%%%
\DescribeMacro{\ifchilddoc}
The conditional |\ifchilddoc| distinguishes between the compilation of
child documents and the main document:
%
\begin{center}
|\ifchilddoc |\textit{child-code}| |[|\||else |\textit{main-code}]| \||fi|
\end{center}

%%%%%%%%%%%%%%%%%%%%%%%%%%%%%%%%%%%%%%%%
\DescribeMacro{\childdocname}
\DescribeMacro{\childdocjob}
The macro |\childdocname| contains the filename (without extension)
of the main or child file being processed.
Note that |\childdocjob| will always contain the name of the main file.

%%%%%%%%%%%%%%%%%%%%%%%%%%%%%%%%%%%%%%%%
\paragraph{Title Page.}

Conditional processing can be used to include a title or banner page
in the main document when proper precautions are taken.
Importantly, the code in the main file should ensure that the page counter
(as well as other status parameters which are stored in the |.aux| files)
takes the same value after the conditional processing.
Otherwise the page numbers may take divergent values
depending on which part is compiled.

For example, a title page could be declared by:
%
\begin{center}
\begin{tabular}{l}
|\ifchilddoc\||else|\\
|\addtocounter{page}{-1}|\\
\textit{code for title page}\\
|\newpage|\\
|\||fi|
\end{tabular}
\end{center}
%
A banner page for the child documents can be generated by:
%
\begin{center}
\begin{tabular}{l}
|\ifchilddoc|\\
|\addtocounter{page}{-1}|\\
\textit{code for banner page}\\
|\newpage|\\
|\||fi|
\end{tabular}
\end{center}
%
Here one could write a message such as:
\begin{center}
|This is the part \childdocname{} of \childdocjob{}.|
\end{center}

%%%%%%%%%%%%%%%%%%%%%%%%%%%%%%%%%%%%%%%%%%%%%%%%%%%%%%%%%%%%%%%%%%%%%%%%%%%%%%%%
\subsection{Flags}
\label{sec:flags}

The package makes it easy to generate different versions
of the main or child documents.
To this end compilation flags can be defined
and assigned different default values.
They will be particularly useful in conjunction
with the forwarding mechanism described in \secref{sec:forward}.

For example, it may be useful to have a flag |\version|
which can be set to |draft| or |final|.
The document source will contain some conditional code
depending on the value of |\version|.
Suppose further, the flag should default to |final| for the main file
and to |draft| for child files
which is a natural assignment for editing the document.
This is achieved by placing the following code
in the preamble of the main document
(below the |\childdocmain| directive):
%
\begin{center}
\begin{tabular}{l}
|\ifchilddoc|\\
|\providecommand{\version}{draft}|\\
|\||else|\\
|\providecommand{\version}{final}|\\
|\||fi|
\end{tabular}
\end{center}
%
The definition by |\providecommand| makes sure
that previous definitions are not overwritten.
Further statements |\providecommand{\version}{...}|
can thus be added before the above code to override it.

For the main file, one might add a line
(between |\childdocmain| and the above block)
%
\begin{center}
|%\ifchilddoc\||else\providecommand{\version}{draft}\||fi|
\end{center}
%
which can be uncommented to produce a draft version.
Likewise one can add a line to the very top of a child file
(above the |\childdocof{|\textit{main}|}| directive)
%
\begin{center}
|%\providecommand{\version}{final}|
\end{center}
%
which can be uncommented to produce the final version of this child document.

%%%%%%%%%%%%%%%%%%%%%%%%%%%%%%%%%%%%%%%%%%%%%%%%%%%%%%%%%%%%%%%%%%%%%%%%%%%%%%%%
\subsection{Forwarding}
\label{sec:forward}

Different versions of the main or child documents
using compilation flags as described in \secref{sec:flags}
can be (permanently) stored in different files
for convenient compilation, viewing and distribution.
To this end, the package defines a command
to pass on compilation to a different file:

%%%%%%%%%%%%%%%%%%%%%%%%%%%%%%%%%%%%%%%%
\DescribeMacro{\childdocforward}
The command |\childdocforward| redirects processing to
another source file:
%
\begin{center}
\begin{tabular}{l}
|\input{childdoc.def}|\\
|\childdocforward[|\textit{main}|]{|\textit{dest}|}|\\
\end{tabular}
\end{center}
%
The argument \textit{dest} is the destination file
(without extension).
It should be the main file or one of the child files.
Note that further \textsf{childdoc} directives
such as |\childdocof| and |\childdocforward|
in the indicated file will be processed in this form.
The optional argument \textit{main}
passes on directly to the main file \textit{main}
while pretending to compile the child \textit{dest}.
This form behaves as if \textit{dest}
issues |\childdocof{|\textit{main}|}| right away,
and no further \textsf{childdoc} directives will be processed.

%%%%%%%%%%%%%%%%%%%%%%%%%%%%%%%%%%%%%%%%
\DescribeMacro{\...prefix}
In the alternative form |\childdocforwardprefix|,
%
\begin{center}
\begin{tabular}{l}
|\input{childdoc.def}|\\
|\childdocforwardprefix[|\textit{main}|]{|\textit{prefix}|}{|\textit{dest}|}|
\end{tabular}
\end{center}
%
the destination file is determined by a pattern
depending on the current file:
To make this work, the current file must be called
`{\textit{prefix}\hspace{0.2em}\textit{suffix}}'
with \textit{prefix} matching precisely the argument.
Processing is then passed on to the file
`{\textit{dest}\hspace{0.2em}\textit{suffix}}'.
Surely, the same effect is achieved by
directly specifying the
argument `{\textit{dest}\hspace{0.2em}\textit{suffix}}'
in the first form.
However, that requires to set up a different file
for each child. With the alternative form of the command
all these files can have exactly the same content
which simplifies setting them up and maintaining them.

For example, the following file |draft.tex|
with a compilation flag |\version| as described in \secref{sec:flags}
compiles the main document as a draft:
%
\begin{center}
\begin{tabular}{l}
|\def\version{draft}|\\
|\input{childdoc.def}|\\
|\childdocforward{|\textit{main}|}|
\end{tabular}
\end{center}
%
Likewise, the following files |final|\textit{nn}|.tex|
compile the final version of the child document
|child|\textit{nn}|.tex|:
%
\begin{center}
\begin{tabular}{l}
|\def\version{final}|\\
|\input{childdoc.def}|\\
|\childdocforwardprefix{final}{child}|
\end{tabular}
\end{center}
%

Note that when several versions of a main file and/or of each child file
are to be generated, it may be convenient to set up a |Makefile| or
shell script to automatise the process.

%%%%%%%%%%%%%%%%%%%%%%%%%%%%%%%%%%%%%%%%%%%%%%%%%%%%%%%%%%%%%%%%%%%%%%%%%%%%%%%%
\subsection{Command Line Processing}
\label{sec:commandline}

The effect of redirection files can also be achieved by invoking
the \LaTeX{} compiler with a more elaborate command line.
Most conveniently this should be done as part
of a shell script or a |Makefile|.

When using \textsf{childdoc} in the main file, the following
command lines effectively perform a redirection
(note that depending on the shell being used,
backslashes may have to be doubled: `|\|' $\to$ `|\\|'):
%
\begin{center}
|... -jobname "|\textit{target}|" |\\|"|[\textit{flags}]%
|\input{childdoc.def}\childdocforward[|\textit{main}|]{|\textit{dest}|}"|
\end{center}
%
Here \textit{target} is the name of the output file,
\textit{main} is the name of the main file
and \textit{dest} is the name of the main or child file to be processed
(all filenames without extensions).
The optional argument \textit{main} can be omitted
if \textit{main} matches \textit{dest}.
Optionally, compilation \textit{flags} can be defined via |\def| commands.
This command line makes the \TeX{} engine believe
it is compiling the file \textit{target}
whose content is specified as the latter parameter.
The provided code then forwards the processing to
\textit{main} or \textit{dest} as described in \secref{sec:forward}.

%%%%%%%%%%%%%%%%%%%%%%%%%%%%%%%%%%%%%%%%%%%%%%%%%%%%%%%%%%%%%%%%%%%%%%%%%%%%%%%%
\subsection{Include by Input}
\label{sec:input}

Including child documents by |\include| has some restrictions by design.
Most notably, the content of a child document always occupies
its own set of pages; pages cannot be shared between child documents.
Usually, this behaviour makes perfect sense
because each child document contain an essential part of the document.
However, in some situations it may be desirable to compose
a document from a collection of parts
without having mandatory page breaks between then.
For this case, the package
provides a mechanism to include parts
by |\input| which can also be processed individually.
However, by construction this mechanism
requires manual handling of the content to be output.

%%%%%%%%%%%%%%%%%%%%%%%%%%%%%%%%%%%%%%%%
\DescribeMacro{\ifchilddocmanual}
The main file should be prepared as usual, see \secref{sec:include}.
However, the document body must make a distinction
between processing of an individual part and of the main document, e.g.:
%
\begin{center}
\begin{tabular}{l}
|\ifchilddocmanual|\\
|\input{\childdocname}|\\
|\||else|\\
\textit{document body with }|\input{|\textit{part}|}|\\
|\||fi|
\end{tabular}
\end{center}
%
The conditional |\ifchilddocmanual| is true whenever
a part to be included by |\input| is being compiled,
and the name of the part is stored in |\childdocname|.

%%%%%%%%%%%%%%%%%%%%%%%%%%%%%%%%%%%%%%%%
\DescribeMacro{\childdocby}
Each part to be included by |\input| should start with:
%
\begin{center}
\begin{tabular}{l}
|\input{childdoc.def}|\\
|\childdocby{|\textit{main}|}|\\
\end{tabular}
\end{center}
%
The directive |\childdocby| is similar to |\childdocof|
described in \secref{sec:include},
but the subsequent selection of content must be done manually.
To that end, both |\ifchilddoc| and |\ifchilddocmanual|
will be true upon processing of a part,
and the name of the part is stored in |\childdocname|.
Note that |\jobname| will be set to the filename of the current part
so that each part receives an individual |.aux| file
that does not interfere with the |.aux| file(s) of the main document.
This behaviour can be altered by the alternative form
|\childdocby[*]{|\textit{main}|}| (with a non-empty optional argument)
which uses the |.aux| file of the main document
by setting |\jobname| to \textit{main}.

%%%%%%%%%%%%%%%%%%%%%%%%%%%%%%%%%%%%%%%%%%%%%%%%%%%%%%%%%%%%%%%%%%%%%%%%%%%%%%%%
\subsection{Driver Development}
\label{sec:driver}

The \textsf{childdoc} mechanism can also be use for the development
of definition files such as \LaTeX{} styles or classes.
This case differs from the above setup with multiple parts
included by |\include| in that no |\includeonly| should be invoked.
This can be achieved by starting the include file
(before |\ProvidesPackage|) with:
%
\begin{center}
\begin{tabular}{l}
|\input{childdoc.def}|\\
|\childdocforward{|\textit{main}|}|\\
\end{tabular}
\end{center}
%
or alternatively with:
%
\begin{center}
\begin{tabular}{l}
|\input{childdoc.def}|\\
|\childdocby{|\textit{main}|}|\\
\end{tabular}
\end{center}
%
Both forms have slightly different effects as described above.
The main file is prepared as usual, see \secref{sec:include}.

%%%%%%%%%%%%%%%%%%%%%%%%%%%%%%%%%%%%%%%%%%%%%%%%%%%%%%%%%%%%%%%%%%%%%%%%%%%%%%%%
\subsection{Legacy Detection}
\label{sec:detection}

The directive |\childdocmain| in the main file can detect
whether the complete document or merely a child is to be compiled
even without using the directive |\childdocof|.
This method is deprecated because it is less robust
and there is no compelling reason to use it;
it is merely provided for backward compatibility
and it may be removed in future versions.

If the detection mechanism is to be used,
it is mandatory to correctly specify
the filename of the main file as the argument of |\childdocmain|:
%
\begin{center}
\begin{tabular}{l}
|\input{childdoc.def}|\\
|\childdocmain{|\textit{main}|}|\\
\end{tabular}
\end{center}
%
If |\jobname| does not match the argument \textit{main} of |\childdocmain|,
it is assumed that |\jobname| points to the child file to be compiled.
When using |\childdocmain| with the main file specified as argument,
it suffices to start a child file
with just |\input{|\textit{main}|}|
without loading of the package and using |\childdocof|.
If instead all processing is done
with the appropriate \textsf{childdoc} directives,
the argument of \textit{main} of |\childdocmain| can be empty.

An alternative version of the command line processing described
in \secref{sec:commandline} using the detection mechanism reads:
%
\begin{center}
|... -jobname "|\textit{target}|" "|[\textit{flags}]%
[|\def\jobname{|\textit{dest}|}|]|\input{|\textit{main}|}"|
\end{center}

%%%%%%%%%%%%%%%%%%%%%%%%%%%%%%%%%%%%%%%%%%%%%%%%%%%%%%%%%%%%%%%%%%%%%%%%%%%%%%%%
\subsection{Manual Code}
\label{sec:manual}

In case one cannot be certain whether the definitions file |childdoc.def|
is installed on the target \TeX{} distribution
and one prefers not to ship it,
it is conceivable to paste a few relevant commands into the sources.

To that end, drop all statements |\input{childdoc.def}|
and perform the replacements as outlined below.
Instead of |\childdocmain{|\textit{main}|}| add the following code
to the top of the main file:
%
\begin{center}
\begin{tabular}{l}
|\||ifdefined\childdocname\endinput\||fi\newif\ifchilddoc|\\
|\edef\childdocname{\scantokens\expandafter{\jobname\noexpand}}|\\
|\def\childdocmain{|\textit{main}|}\||ifx\childdocmain\childdocname\||else|\\
|\childdoctrue\includeonly{\childdocname}\let\jobname\childdocmain\||fi|\\
\end{tabular}
\end{center}
%
Instead of |\childdocof{|\textit{main}|}| just include the main file
at the top of each child file:
%
\begin{center}
|\input{|\textit{main}|}|
\end{center}
%
A simple redirection |\childdocforward{|\textit{dest}|}| is achieved by:
%
\begin{center}
|\def\jobname{|\textit{dest}|}\input{\jobname}|
\end{center}
%
The redirection with prefix
|\childdocforwardprefix[|\textit{prefix}|]{|\textit{dest}|}|
is accomplished by:
%
\begin{center}
\begin{tabular}{l}
|{\edef\jobname{\scantokens\expandafter{\jobname\noexpand}}|\\
|\def\redirectjob |\textit{prefix}|#1~~~{\gdef\jobname{|\textit{dest}|#1}}|\\
|\expandafter\redirectjob\jobname~~~}\input{\jobname}|
\end{tabular}
\end{center}

In an alternative approach,
child documents can be compiled by a specific command line
without additional code or specific definitions:
%
\begin{center}
|... -jobname "|\textit{target}|" "|[\textit{flags}]%
|\includeonly{|\textit{dest}|}\input{|\textit{main}|}"|
\end{center}
%

%%%%%%%%%%%%%%%%%%%%%%%%%%%%%%%%%%%%%%%%%%%%%%%%%%%%%%%%%%%%%%%%%%%%%%%%%%%%%%%%
%%%%%%%%%%%%%%%%%%%%%%%%%%%%%%%%%%%%%%%%%%%%%%%%%%%%%%%%%%%%%%%%%%%%%%%%%%%%%%%%
\section{Information}

%%%%%%%%%%%%%%%%%%%%%%%%%%%%%%%%%%%%%%%%%%%%%%%%%%%%%%%%%%%%%%%%%%%%%%%%%%%%%%%%
\subsection{Copyright}

Copyright \copyright{} 2017--2018 Niklas Beisert

This work may be distributed and/or modified under the
conditions of the \LaTeX{} Project Public License, either version 1.3
of this license or (at your option) any later version.
The latest version of this license is in
  \url{http://www.latex-project.org/lppl.txt}
and version 1.3 or later is part of all distributions of \LaTeX{}
version 2005/12/01 or later.

This work has the LPPL maintenance status `maintained'.

The Current Maintainer of this work is Niklas Beisert.

This work consists of the files |README.txt|, |childdoc.ins| and |childdoc.dtx|
as well as the derived files |childdoc.def|, |cdocsamp.tex|
with |cdocsch1.tex|, |cdocsch2.tex|, |cdocspt3.tex|, |cdocspt4.tex|,
|cdocsdrf.tex|, |cdocsfn1.tex|, |cdocsfn2.tex|
as well as |childdoc.pdf|.

%%%%%%%%%%%%%%%%%%%%%%%%%%%%%%%%%%%%%%%%%%%%%%%%%%%%%%%%%%%%%%%%%%%%%%%%%%%%%%%%
\subsection{Files and Installation}

The package consists of the files:
%
\begin{center}
\begin{tabular}{ll}
    |README.txt|   & readme file \\
    |childdoc.ins| & installation file \\
    |childdoc.dtx| & source file \\
    |childdoc.def| & definition file \\
    |cdocsamp.tex| & sample main file \\
    |cdocsch1.tex| & sample include file \\
    |cdocsch2.tex| & sample include file \\
    |cdocspt3.tex| & sample part file \\
    |cdocspt4.tex| & sample part file \\
    |cdocsdrf.tex| & sample redirection file \\
    |cdocsfn1.tex| & sample redirection file \\
    |cdocsfn2.tex| & sample redirection file \\
    |childdoc.pdf| & manual
\end{tabular}
\end{center}
%
The distribution consists of the files
|README.txt|, |childdoc.ins| and |childdoc.dtx|.
%
\begin{itemize}
\item
Run (pdf)\LaTeX{} on |childdoc.dtx|
to compile the manual |childdoc.pdf| (this file).
\item
Run \LaTeX{} on |childdoc.ins| to create the definitions file |childdoc.def|
and the sample |cdocsamp.tex| with include files
|cdocsch1.tex|, |cdocsch2.tex|, |cdocspt3.tex|, |cdocspt4.tex|,
|cdocsdrf.tex|, |cdocsfn1.tex|, |cdocsfn2.tex|.
Then copy the file |childdoc.def| to an appropriate directory of your \LaTeX{}
distribution, e.g.\ \textit{texmf-root}|/tex/latex/childdoc|.
\end{itemize}

%%%%%%%%%%%%%%%%%%%%%%%%%%%%%%%%%%%%%%%%%%%%%%%%%%%%%%%%%%%%%%%%%%%%%%%%%%%%%%%%
\subsection{Related CTAN Packages}

There are several other packages which offer a similar functionality:
%
\begin{itemize}
\item
The packages
\href{http://ctan.org/pkg/docmute}{\textsf{docmute}},
\href{http://ctan.org/pkg/includex}{\textsf{includex}} and
\href{http://ctan.org/pkg/standalone}{\textsf{standalone}}
provide commands to include only the document body of
a child file thus allowing both files to be compiled individually.
\item
The packages \href{http://ctan.org/pkg/subdocs}{\textsf{subdocs}}
and \href{http://ctan.org/pkg/subfiles}{\textsf{subfiles}}
provide structures in which the main and child documents can be
encapsulated and allowing them to be compiled individually.
The inclusion mechanism is different from the conventional |\include|.
\item
The package \href{http://ctan.org/pkg/combine}{\textsf{combine}}
is an elaborate solution to combine several documents into one.
\end{itemize}
%
See also the CTAN topic \href{http://ctan.org/topic/subdocs}{\textsf{subdocs}}
for further related packages.
The present package differs from the above solutions in that
a document structure constructed with the conventional |\include| mechanism
just needs two extra commands at the top of every file
such that all constituent files can be compiled individually.

%%%%%%%%%%%%%%%%%%%%%%%%%%%%%%%%%%%%%%%%%%%%%%%%%%%%%%%%%%%%%%%%%%%%%%%%%%%%%%%%
%\subsection{Feature Suggestions}
%
%The following is a list of features which may be useful for future
%versions of this package:
%%
%\begin{itemize}
%\item
%\ldots
%\end{itemize}

%%%%%%%%%%%%%%%%%%%%%%%%%%%%%%%%%%%%%%%%%%%%%%%%%%%%%%%%%%%%%%%%%%%%%%%%%%%%%%%%
\subsection{Revision History}

%%%%%%%%%%%%%%%%%%%%%%%%%%%%%%%%%%%%%%%%
\paragraph{v2.0:} 2018/12/30

\begin{itemize}
\item
immediate forward processing
\item
added |\childdocby| mechanism
\item
manual restructured
\end{itemize}

%%%%%%%%%%%%%%%%%%%%%%%%%%%%%%%%%%%%%%%%
\paragraph{v1.6:} 2018/01/17

\begin{itemize}
\item
application for development of include files
\item
corrections to manual
\end{itemize}

%%%%%%%%%%%%%%%%%%%%%%%%%%%%%%%%%%%%%%%%
\paragraph{v1.5:} 2017/05/21

\begin{itemize}
\item
more complete structuring introduced
\item
|\childdocof| introduced
\item
|\childdoc| renamed to |\childdocmain|
\item
|\childredirect| renamed to |\childdocforward| and |\childdocforwardprefix|
and functionality expanded
\end{itemize}

%%%%%%%%%%%%%%%%%%%%%%%%%%%%%%%%%%%%%%%%
\paragraph{v1.0:} 2017/04/27

\begin{itemize}
\item
manual and install package
\item
first version published on CTAN
\end{itemize}

%%%%%%%%%%%%%%%%%%%%%%%%%%%%%%%%%%%%%%%%
\paragraph{v0.6:} 2017/04/26

\begin{itemize}
\item
redirection mechanism added
\end{itemize}

%%%%%%%%%%%%%%%%%%%%%%%%%%%%%%%%%%%%%%%%
\paragraph{v0.5:} 2017/04/26

\begin{itemize}
\item
functionality in definition file
\end{itemize}


%%%%%%%%%%%%%%%%%%%%%%%%%%%%%%%%%%%%%%%%%%%%%%%%%%%%%%%%%%%%%%%%%%%%%%%%%%%%%%%%
%%%%%%%%%%%%%%%%%%%%%%%%%%%%%%%%%%%%%%%%%%%%%%%%%%%%%%%%%%%%%%%%%%%%%%%%%%%%%%%%
%%%%%%%%%%%%%%%%%%%%%%%%%%%%%%%%%%%%%%%%%%%%%%%%%%%%%%%%%%%%%%%%%%%%%%%%%%%%%%%%
\appendix

\settowidth\MacroIndent{\rmfamily\scriptsize 000\ }

 \DocInput{childdoc.dtx}

\end{document}
%</driver>
% \fi
%
% %%%%%%%%%%%%%%%%%%%%%%%%%%%%%%%%%%%%%%%%%%%%%%%%%%%%%%%%%%%%%%%%%%%%%%%%%%%%%%
% %%%%%%%%%%%%%%%%%%%%%%%%%%%%%%%%%%%%%%%%%%%%%%%%%%%%%%%%%%%%%%%%%%%%%%%%%%%%%%
% \section{Sample}
%\iffalse
%<*samplemain>
%\fi
%
% The following presents a sample document
% with two chapters, two parts, a title page,
% a compile flag as well as three forwarding files to set the flag.
% It consists of eight |.tex| files:
% \begin{center}
% \begin{tabular}{ll}
% |cdocsamp.tex|&main file\\
% |cdocsch1.tex|&include file for chapter 1\\
% |cdocsch2.tex|&include file for chapter 2\\
% |cdocspt3.tex|&include file for part 3\\
% |cdocspt4.tex|&include file for part 4\\
% |cdocsdrf.tex|&forwarding file for main file in draft mode\\
% |cdocsfi1.tex|&forwarding file for final version of chapter 1\\
% |cdocsfi2.tex|&forwarding file for final version of chapter 2\\
% \end{tabular}
% \end{center}
% Each of the eight files can be compiled directly by the \LaTeX{} compiler.
%
% %%%%%%%%%%%%%%%%%%%%%%%%%%%%%%%%%%%%%%
% \paragraph{Main File.}
%
% The main file is called |cdocsamp.tex|.
%
% Load the \textsf{childdoc} definitions and
% declare the filename for the main document:
%    \begin{macrocode}
\input{childdoc.def}
\childdocmain{}
%    \end{macrocode}

% Optional override for |\version| flag:
%    \begin{macrocode}
%%\ifchilddoc\else\providecommand{\version}{draft}\fi
%    \end{macrocode}

% Define the default values for the |\version| flag
% (|final| for the main file and |draft| for childs):
%    \begin{macrocode}
\ifchilddoc
\providecommand{\version}{draft}
\else
\providecommand{\version}{final}
\fi
%    \end{macrocode}

% Load the standard document class:
%    \begin{macrocode}
\documentclass[12pt]{article}
%    \end{macrocode}

% Start the document body:
%    \begin{macrocode}
\begin{document}
%    \end{macrocode}

% Declare a title page.
% Print title, part of document being processed and version flag:
%    \begin{macrocode}
\addtocounter{page}{-1}
\begin{center}
{\LARGE\bfseries{}childdoc example\par}
\vspace{1cm}
\ifchilddoc
\ifchilddocmanual part\else chapter\fi:
`\childdocname' of `\childdocjob'\par
\else
main document: `\childdocjob'\par
\fi
version: \version\par
\end{center}
\newpage
%    \end{macrocode}

% Manually include selected file,
% otherwise process as usual:
%    \begin{macrocode}
\ifchilddocmanual
\section*{part `\childdocname'}
\input{\childdocname}
\else
%    \end{macrocode}

% Include the two chapters:
%    \begin{macrocode}
\include{cdocsch1}
\include{cdocsch2}
%    \end{macrocode}

% Include the two parts unless only chapters should be displayed:
%    \begin{macrocode}
\ifchilddoc\else
\section{part three}
\input{cdocspt3}
\section{part four}
\input{cdocspt4}
\fi
%    \end{macrocode}

% Process as usual until here:
%    \begin{macrocode}
\fi
%    \end{macrocode}

% End of document body:
%    \begin{macrocode}
\end{document}
%    \end{macrocode}
%\iffalse
%</samplemain>
%\fi
%
% %%%%%%%%%%%%%%%%%%%%%%%%%%%%%%%%%%%%%%
% \paragraph{Chapter Include Files.}
%
% The include files are called |cdocsch1.tex| and |cdocsch2.tex|.
%
%\iffalse
%<*samplechap1|samplechap2>
%\fi

% Optional override for |\version| flag:
%    \begin{macrocode}
%%\providecommand{\version}{final}
%    \end{macrocode}

% Include the main document:
%    \begin{macrocode}
\input{childdoc.def}
\childdocof{cdocsamp}
%    \end{macrocode}

%\iffalse
%</samplechap1|samplechap2>
%\fi
%
%\iffalse
%<*samplechap1>
%\fi
% Some text for chapter 1:
%    \begin{macrocode}
\section{one}
some text in chapter one
%    \end{macrocode}

%\iffalse
%</samplechap1>
%\fi
% Some text for chapter 2:
%\iffalse
%<*samplechap2>
%\fi
%    \begin{macrocode}
\section{two}
more text in chapter two
%    \end{macrocode}

%\iffalse
%</samplechap2>
%\fi
%
% %%%%%%%%%%%%%%%%%%%%%%%%%%%%%%%%%%%%%%
% \paragraph{Part Include Files.}
%
% The include files are called |cdocspt3.tex| and |cdocspt4.tex|.
%
%\iffalse
%<*samplepart3|samplepart4>
%\fi

% Optional override for |\version| flag:
%    \begin{macrocode}
%%\providecommand{\version}{final}
%    \end{macrocode}

% Include the main document:
%    \begin{macrocode}
\input{childdoc.def}
\childdocby{cdocsamp}
%    \end{macrocode}

%\iffalse
%</samplepart3|samplepart4>
%\fi
%
%\iffalse
%<*samplepart3>
%\fi
% Some text for part 3:
%    \begin{macrocode}
some text in part three
%    \end{macrocode}

%\iffalse
%</samplepart3>
%\fi
% Some text for part 4:
%\iffalse
%<*samplepart4>
%\fi
%    \begin{macrocode}
more text in part four
%    \end{macrocode}

%\iffalse
%</samplepart4>
%\fi
%
% %%%%%%%%%%%%%%%%%%%%%%%%%%%%%%%%%%%%%%
% \paragraph{Forwarding for a Complete Draft.}
%
% The following forwarding file |cdocsdrf.tex|
% compiles the main document in draft mode:
%\iffalse
%<*sampledraft>
%\fi
%    \begin{macrocode}
\def\version{draft}
\input{childdoc.def}
\childdocforward{cdocsamp}
%    \end{macrocode}

%\iffalse
%</sampledraft>
%\fi
%
% %%%%%%%%%%%%%%%%%%%%%%%%%%%%%%%%%%%%%%
% \paragraph{Forwarding for Final Version of the Chapters.}
%
% The following forwarding files |cdocsfn1.tex| and |cdocsfn2.tex|
% (with identical content)
% compile the final versions of the child documents
% |cdocsch1.tex| and |cdocsch2.tex|, respectively:
%\iffalse
%<*samplefinal>
%\fi
%    \begin{macrocode}
\def\version{final}
\input{childdoc.def}
\childdocforwardprefix[cdocsamp]{cdocsfn}{cdocsch}
%    \end{macrocode}

%\iffalse
%</samplefinal>
%\fi
%
% %%%%%%%%%%%%%%%%%%%%%%%%%%%%%%%%%%%%%%
% \paragraph{Command Line Processing.}
%
% The following three command lines generate the output files
% |cdocscld|, |cdocscl1| and |cdocscl2|
% which should be identical to
% |cdocsdrf|, |cdocsch1| and |cdocsfn2|, respectively:
% \begin{center}
% \begin{tabular}{l}
% |latex -jobname cdocscld \|\\
% |  "\def\version{draft}\input{childdoc.def}\childdocforward{cdocsamp}"|\\
% |latex -jobname cdocscl1 \|\\
% |  "\input{childdoc.def}\childdocforward[cdocsamp]{cdocsch1}"|\\
% |latex -jobname cdocscl2 \|\\
% |  "\def\version{final}\input{childdoc.def}\childdocforward{cdocsch2}"|
% \end{tabular}
% \end{center}
% Note that the trailing backslash on each first line
% merely continues the input to the second line
% (for convenient cut ant paste).
% Furthermore, the command |latex| can be replaced by any
% of its alternative versions such as |pdflatex|.
%
% %%%%%%%%%%%%%%%%%%%%%%%%%%%%%%%%%%%%%%%%%%%%%%%%%%%%%%%%%%%%%%%%%%%%%%%%%%%%%%
% %%%%%%%%%%%%%%%%%%%%%%%%%%%%%%%%%%%%%%%%%%%%%%%%%%%%%%%%%%%%%%%%%%%%%%%%%%%%%%
% \section{Implementation}
%\iffalse
%<*package>
%\fi
%
% This section describes the definitions file |childdoc.def|.

% The definitions cannot be loaded using |\usepackage| or |\RequirePackage|
% which has a mechanism to prevent loading a style file more than once.
% When loading the definitions by means of |\input|
% multiple instances have to be prevented manually:
%\iffalse
%This code needs to be before the `\ProvidesFile' directive
%which is defined at the beginning of this file.
%Therefore it is also placed there and commented out here.
%</package>
%<*discard>
%\fi
%    \begin{macrocode}
\ifdefined\childdocmain\endinput\fi
%    \end{macrocode}
%\iffalse
%</discard>
%<*package>
%\fi
%
% \macro{\ifchilddoc}
% \macro{\ifchilddocmanual}
% The conditional |\ifchilddoc| tells whether a
% child (true) or main (false) document is being compiled.
% The conditional |\ifchilddocmanual| tells whether
% the |\includeonly| mechanism is used (false) or
% the selection of child files must be performed manually (true).
% The definitions initialise to false:
%    \begin{macrocode}
\newif\ifchilddoc
\newif\ifchilddocmanual
%    \end{macrocode}

% \macro{\childdocname}
% \macro{\childdocjob}
% The macro |\childdocname| stores the name of the main document
% to be compiled. The macro |\childdocjob| stores the name of
% the document on which the \LaTeX{} compiler was originally invoked.
% The content of |\jobname| cannot be compared
% to filenames specified in the source due to different catcodes.
% The following code rescans |\jobname|, stores the result
% in |\childdocname| and saves a copy in |\childdocjob|:
%    \begin{macrocode}
\edef\childdocname{\scantokens\expandafter{\jobname\noexpand}}
\let\childdocjob\childdocname
%    \end{macrocode}

% \macro{\childdocdisable}
% The macro |\childdocdisable| prevents the main file
% from being processed more than once.
% At this stage, the main document command |\childdocmain|
% is assumed to be called once again where it should do nothing.
% Any subsequent call to it should prevent
% a secondary processing of the main document
% It overwrites the forwarding commands
% |\childdocof| and |\childdocforward|
% with empty macros to prevent further inclusions of the main document:
%    \begin{macrocode}
\newcommand{\childdocdisable}
{
  \renewcommand{\childdocmain}[1]{\renewcommand{\childdocmain}[1]{\endinput}}
  \renewcommand{\childdocof}[1]{}
  \renewcommand{\childdocby}[2][]{}
  \renewcommand{\childdocforward}[2][]{}
  \renewcommand{\childdocdisable}{}
}
%    \end{macrocode}

% \macro{\childdocmain}
% The macro |\childdocmain| is to be called at the top of the main file
% with nothing or the main filename (without extension) as argument.
% First, it breaks loops.
% If the argument is not empty and does not match |\childdocname|
% (which is set by the first inclusion of |childdoc.def|),
% |\ifchilddoc| is set to true, |\includeonly| is applied to the child file
% and |\jobname| is set to the main file
% (for proper handling of |.aux| files):
%    \begin{macrocode}
\newcommand{\childdocmain}[1]
{
  \childdocdisable\childdocmain{}
  \if?#1?\else
    \begingroup
      \def\childdoctmp{#1}
      \ifx\childdoctmp\childdocname
        \def\childdoctmp{}
      \else
        \def\childdoctmp
        {
          \childdoctrue
          \includeonly{\childdocname}
          \def\childdocjob{#1}
          \def\jobname{#1}
        }
      \fi
      \expandafter
    \endgroup
    \childdoctmp
  \fi
}
%    \end{macrocode}

% \macro{\childdocof}
% The command |\childdocof| redirects
% compilation to the main file |#1|.
%    \begin{macrocode}
\newcommand{\childdocof}[1]
{
  \childdocdisable
  \childdoctrue
  \includeonly{\childdocname}
  \def\jobname{#1}
  \def\childdocjob{#1}
  \input{#1}
}
%    \end{macrocode}

% \macro{\childdocby}
% The command |\childdocby| ....
%    \begin{macrocode}
\newcommand{\childdocby}[2][]
{
  \childdocdisable
  \childdoctrue
  \childdocmanualtrue
  \if?#1?\else
    \def\jobname{#2}
  \fi
  \def\childdocjob{#2}
  \input{#2}
  \endinput
}
%    \end{macrocode}

% \macro{\childdocforward}
% The command |\childdocforward| redirects
% compilation to the main file or
% (if the optional argument is given) a child file.
% Parameters are set as if the main file
% or a child file starting with |\childdocof| was compiled.
% Then compilation is handed over to the main file:
%    \begin{macrocode}
\newcommand{\childdocforward}[2][]
{
  \begingroup
    \if?#1?
      \def\childdoctmp
      {
        \def\childdocname{#2}
        \def\childdocjob{#2}
        \def\jobname{#2}
        \input{#2}
        \endinput
      }
    \else
      \def\childdoctmp
      {
        \childdocdisable
        \def\childdocname{#2}
        \childdoctrue
        \includeonly{#2}
        \def\childdocjob{#1}
        \def\jobname{#1}
        \input{#1}
        \endinput
      }
    \fi
    \expandafter
  \endgroup
  \childdoctmp
}
%    \end{macrocode}

% \macro{\childdocforwardprefix}
% The command |\childdocforwardprefix| redirects
% compilation to the main or a child file by means of a pattern.
% The prefix |#1| in the current filename is replaced by |#2|
% and the suffix of the current filename is kept
% (it is assumed that the filename does not contain the substring `|~~~|'
% which is used as a delimiter).
% Compilation is handed over to the new file by |\childdocforward|:
%    \begin{macrocode}
\newcommand{\childdocforwardprefix}[3][]
{
  \begingroup
    \def\childdocextract #2##1~~~{\def\childdoctmp{\childdocforward[#1]{#3##1}}}
    \expandafter\childdocextract\childdocname~~~
    \expandafter
  \endgroup
  \childdoctmp
}
%    \end{macrocode}

% \macro{\childdoc}
% The deprecated macro |\childdoc| is a legacy version of |\childdocmain|:
%    \begin{macrocode}
\newcommand{\childdoc}{\childdocmain}
%    \end{macrocode}

% \macro{\childdocredirect}
% The deprecated macro |\childdocredirect| is a legacy version
% of |\childdocforward| and |\childdocforwardprefix|:
%    \begin{macrocode}
\newcommand{\childdocredirect}[2][]
{
  \begingroup
    \if?#1?
      \def\childdoctmp{\childdocforward{#2}}
    \else
      \def\childdoctmp{\childdocforwardprefix{#1}{#2}}
    \fi
    \expandafter
  \endgroup
  \childdoctmp
}
%    \end{macrocode}

%\iffalse
%</package>
%\fi
%
\endinput
|
and perform the replacements as outlined below.
Instead of |\childdocmain{|\textit{main}|}| add the following code
to the top of the main file:
%
\begin{center}
\begin{tabular}{l}
|\||ifdefined\childdocname\endinput\||fi\newif\ifchilddoc|\\
|\edef\childdocname{\scantokens\expandafter{\jobname\noexpand}}|\\
|\def\childdocmain{|\textit{main}|}\||ifx\childdocmain\childdocname\||else|\\
|\childdoctrue\includeonly{\childdocname}\let\jobname\childdocmain\||fi|\\
\end{tabular}
\end{center}
%
Instead of |\childdocof{|\textit{main}|}| just include the main file
at the top of each child file:
%
\begin{center}
|\input{|\textit{main}|}|
\end{center}
%
A simple redirection |\childdocforward{|\textit{dest}|}| is achieved by:
%
\begin{center}
|\def\jobname{|\textit{dest}|}\input{\jobname}|
\end{center}
%
The redirection with prefix
|\childdocforwardprefix[|\textit{prefix}|]{|\textit{dest}|}|
is accomplished by:
%
\begin{center}
\begin{tabular}{l}
|{\edef\jobname{\scantokens\expandafter{\jobname\noexpand}}|\\
|\def\redirectjob |\textit{prefix}|#1~~~{\gdef\jobname{|\textit{dest}|#1}}|\\
|\expandafter\redirectjob\jobname~~~}\input{\jobname}|
\end{tabular}
\end{center}

In an alternative approach,
child documents can be compiled by a specific command line
without additional code or specific definitions:
%
\begin{center}
|... -jobname "|\textit{target}|" "|[\textit{flags}]%
|\includeonly{|\textit{dest}|}\input{|\textit{main}|}"|
\end{center}
%

%%%%%%%%%%%%%%%%%%%%%%%%%%%%%%%%%%%%%%%%%%%%%%%%%%%%%%%%%%%%%%%%%%%%%%%%%%%%%%%%
%%%%%%%%%%%%%%%%%%%%%%%%%%%%%%%%%%%%%%%%%%%%%%%%%%%%%%%%%%%%%%%%%%%%%%%%%%%%%%%%
\section{Information}

%%%%%%%%%%%%%%%%%%%%%%%%%%%%%%%%%%%%%%%%%%%%%%%%%%%%%%%%%%%%%%%%%%%%%%%%%%%%%%%%
\subsection{Copyright}

Copyright \copyright{} 2017--2018 Niklas Beisert

This work may be distributed and/or modified under the
conditions of the \LaTeX{} Project Public License, either version 1.3
of this license or (at your option) any later version.
The latest version of this license is in
  \url{http://www.latex-project.org/lppl.txt}
and version 1.3 or later is part of all distributions of \LaTeX{}
version 2005/12/01 or later.

This work has the LPPL maintenance status `maintained'.

The Current Maintainer of this work is Niklas Beisert.

This work consists of the files |README.txt|, |childdoc.ins| and |childdoc.dtx|
as well as the derived files |childdoc.def|, |cdocsamp.tex|
with |cdocsch1.tex|, |cdocsch2.tex|, |cdocspt3.tex|, |cdocspt4.tex|,
|cdocsdrf.tex|, |cdocsfn1.tex|, |cdocsfn2.tex|
as well as |childdoc.pdf|.

%%%%%%%%%%%%%%%%%%%%%%%%%%%%%%%%%%%%%%%%%%%%%%%%%%%%%%%%%%%%%%%%%%%%%%%%%%%%%%%%
\subsection{Files and Installation}

The package consists of the files:
%
\begin{center}
\begin{tabular}{ll}
    |README.txt|   & readme file \\
    |childdoc.ins| & installation file \\
    |childdoc.dtx| & source file \\
    |childdoc.def| & definition file \\
    |cdocsamp.tex| & sample main file \\
    |cdocsch1.tex| & sample include file \\
    |cdocsch2.tex| & sample include file \\
    |cdocspt3.tex| & sample part file \\
    |cdocspt4.tex| & sample part file \\
    |cdocsdrf.tex| & sample redirection file \\
    |cdocsfn1.tex| & sample redirection file \\
    |cdocsfn2.tex| & sample redirection file \\
    |childdoc.pdf| & manual
\end{tabular}
\end{center}
%
The distribution consists of the files
|README.txt|, |childdoc.ins| and |childdoc.dtx|.
%
\begin{itemize}
\item
Run (pdf)\LaTeX{} on |childdoc.dtx|
to compile the manual |childdoc.pdf| (this file).
\item
Run \LaTeX{} on |childdoc.ins| to create the definitions file |childdoc.def|
and the sample |cdocsamp.tex| with include files
|cdocsch1.tex|, |cdocsch2.tex|, |cdocspt3.tex|, |cdocspt4.tex|,
|cdocsdrf.tex|, |cdocsfn1.tex|, |cdocsfn2.tex|.
Then copy the file |childdoc.def| to an appropriate directory of your \LaTeX{}
distribution, e.g.\ \textit{texmf-root}|/tex/latex/childdoc|.
\end{itemize}

%%%%%%%%%%%%%%%%%%%%%%%%%%%%%%%%%%%%%%%%%%%%%%%%%%%%%%%%%%%%%%%%%%%%%%%%%%%%%%%%
\subsection{Related CTAN Packages}

There are several other packages which offer a similar functionality:
%
\begin{itemize}
\item
The packages
\href{http://ctan.org/pkg/docmute}{\textsf{docmute}},
\href{http://ctan.org/pkg/includex}{\textsf{includex}} and
\href{http://ctan.org/pkg/standalone}{\textsf{standalone}}
provide commands to include only the document body of
a child file thus allowing both files to be compiled individually.
\item
The packages \href{http://ctan.org/pkg/subdocs}{\textsf{subdocs}}
and \href{http://ctan.org/pkg/subfiles}{\textsf{subfiles}}
provide structures in which the main and child documents can be
encapsulated and allowing them to be compiled individually.
The inclusion mechanism is different from the conventional |\include|.
\item
The package \href{http://ctan.org/pkg/combine}{\textsf{combine}}
is an elaborate solution to combine several documents into one.
\end{itemize}
%
See also the CTAN topic \href{http://ctan.org/topic/subdocs}{\textsf{subdocs}}
for further related packages.
The present package differs from the above solutions in that
a document structure constructed with the conventional |\include| mechanism
just needs two extra commands at the top of every file
such that all constituent files can be compiled individually.

%%%%%%%%%%%%%%%%%%%%%%%%%%%%%%%%%%%%%%%%%%%%%%%%%%%%%%%%%%%%%%%%%%%%%%%%%%%%%%%%
%\subsection{Feature Suggestions}
%
%The following is a list of features which may be useful for future
%versions of this package:
%%
%\begin{itemize}
%\item
%\ldots
%\end{itemize}

%%%%%%%%%%%%%%%%%%%%%%%%%%%%%%%%%%%%%%%%%%%%%%%%%%%%%%%%%%%%%%%%%%%%%%%%%%%%%%%%
\subsection{Revision History}

%%%%%%%%%%%%%%%%%%%%%%%%%%%%%%%%%%%%%%%%
\paragraph{v2.0:} 2018/12/30

\begin{itemize}
\item
immediate forward processing
\item
added |\childdocby| mechanism
\item
manual restructured
\end{itemize}

%%%%%%%%%%%%%%%%%%%%%%%%%%%%%%%%%%%%%%%%
\paragraph{v1.6:} 2018/01/17

\begin{itemize}
\item
application for development of include files
\item
corrections to manual
\end{itemize}

%%%%%%%%%%%%%%%%%%%%%%%%%%%%%%%%%%%%%%%%
\paragraph{v1.5:} 2017/05/21

\begin{itemize}
\item
more complete structuring introduced
\item
|\childdocof| introduced
\item
|\childdoc| renamed to |\childdocmain|
\item
|\childredirect| renamed to |\childdocforward| and |\childdocforwardprefix|
and functionality expanded
\end{itemize}

%%%%%%%%%%%%%%%%%%%%%%%%%%%%%%%%%%%%%%%%
\paragraph{v1.0:} 2017/04/27

\begin{itemize}
\item
manual and install package
\item
first version published on CTAN
\end{itemize}

%%%%%%%%%%%%%%%%%%%%%%%%%%%%%%%%%%%%%%%%
\paragraph{v0.6:} 2017/04/26

\begin{itemize}
\item
redirection mechanism added
\end{itemize}

%%%%%%%%%%%%%%%%%%%%%%%%%%%%%%%%%%%%%%%%
\paragraph{v0.5:} 2017/04/26

\begin{itemize}
\item
functionality in definition file
\end{itemize}


%%%%%%%%%%%%%%%%%%%%%%%%%%%%%%%%%%%%%%%%%%%%%%%%%%%%%%%%%%%%%%%%%%%%%%%%%%%%%%%%
%%%%%%%%%%%%%%%%%%%%%%%%%%%%%%%%%%%%%%%%%%%%%%%%%%%%%%%%%%%%%%%%%%%%%%%%%%%%%%%%
%%%%%%%%%%%%%%%%%%%%%%%%%%%%%%%%%%%%%%%%%%%%%%%%%%%%%%%%%%%%%%%%%%%%%%%%%%%%%%%%
\appendix

\settowidth\MacroIndent{\rmfamily\scriptsize 000\ }

 \DocInput{childdoc.dtx}

\end{document}
%</driver>
% \fi
%
% %%%%%%%%%%%%%%%%%%%%%%%%%%%%%%%%%%%%%%%%%%%%%%%%%%%%%%%%%%%%%%%%%%%%%%%%%%%%%%
% %%%%%%%%%%%%%%%%%%%%%%%%%%%%%%%%%%%%%%%%%%%%%%%%%%%%%%%%%%%%%%%%%%%%%%%%%%%%%%
% \section{Sample}
%\iffalse
%<*samplemain>
%\fi
%
% The following presents a sample document
% with two chapters, two parts, a title page,
% a compile flag as well as three forwarding files to set the flag.
% It consists of eight |.tex| files:
% \begin{center}
% \begin{tabular}{ll}
% |cdocsamp.tex|&main file\\
% |cdocsch1.tex|&include file for chapter 1\\
% |cdocsch2.tex|&include file for chapter 2\\
% |cdocspt3.tex|&include file for part 3\\
% |cdocspt4.tex|&include file for part 4\\
% |cdocsdrf.tex|&forwarding file for main file in draft mode\\
% |cdocsfi1.tex|&forwarding file for final version of chapter 1\\
% |cdocsfi2.tex|&forwarding file for final version of chapter 2\\
% \end{tabular}
% \end{center}
% Each of the eight files can be compiled directly by the \LaTeX{} compiler.
%
% %%%%%%%%%%%%%%%%%%%%%%%%%%%%%%%%%%%%%%
% \paragraph{Main File.}
%
% The main file is called |cdocsamp.tex|.
%
% Load the \textsf{childdoc} definitions and
% declare the filename for the main document:
%    \begin{macrocode}
% \iffalse
%
% childdoc.dtx Copyright (C) 2017-2018 Niklas Beisert
%
% This work may be distributed and/or modified under the
% conditions of the LaTeX Project Public License, either version 1.3
% of this license or (at your option) any later version.
% The latest version of this license is in
%   http://www.latex-project.org/lppl.txt
% and version 1.3 or later is part of all distributions of LaTeX
% version 2005/12/01 or later.
%
% This work has the LPPL maintenance status `maintained'.
%
% The Current Maintainer of this work is Niklas Beisert.
%
% This work consists of the files childdoc.dtx and childdoc.ins
% and the derived files childdoc.def and cdocsamp.tex with
% cdocsch1.tex, cdocsch2.tex, cdocsdrf.tex, cdocsfn1.tex, cdocsfn2.tex.
%
%<package>\ifdefined\childdocmain\endinput\fi
%<package>\ProvidesFile{childdoc.def}[2018/12/30 v2.0 child document driver]
%<samplemain>\ProvidesFile{cdocsamp.tex}[2018/12/30 v2.0 sample for childdoc]
%<*driver>
%\ProvidesFile{childdoc.drv}[2018/12/30 v2.0 childdoc reference manual file]
\PassOptionsToClass{10pt,a4paper}{article}
\documentclass{ltxdoc}

\usepackage[margin=35mm]{geometry}
\usepackage{hyperref}
\usepackage{hyperxmp}
\usepackage[usenames]{color}

\hypersetup{colorlinks=true}
\hypersetup{pdfstartview=FitH}
\hypersetup{pdfpagemode=UseNone}
\hypersetup{pdfsource={}}
\hypersetup{pdflang={en-UK}}
\hypersetup{pdfcopyright={Copyright 2017-2018 Niklas Beisert.
  This work may be distributed and/or modified under the
  conditions of the LaTeX Project Public License, either version 1.3
  of this license or (at your option) any later version.}}
\hypersetup{pdflicenseurl={http://www.latex-project.org/lppl.txt}}
\hypersetup{pdfcontactaddress={ETH Zurich, ITP, HIT K,
  Wolfgang-Pauli-Strasse 27}}
\hypersetup{pdfcontactpostcode={8093}}
\hypersetup{pdfcontactcity={Zurich}}
\hypersetup{pdfcontactcountry={Switzerland}}
\hypersetup{pdfcontactemail={nbeisert@itp.phys.ethz.ch}}
\hypersetup{pdfcontacturl={http://people.phys.ethz.ch/\xmptilde nbeisert/}}

\newcommand{\secref}[1]{\hyperref[#1]{section \ref*{#1}}}

\parskip1ex
\parindent0pt
\let\olditemize\itemize
\def\itemize{\olditemize\parskip0pt}

\begin{document}

\title{The \textsf{childdoc} Package}
\hypersetup{pdftitle={The childdoc Package}}
\author{Niklas Beisert\\[2ex]
  Institut f\"ur Theoretische Physik\\
  Eidgen\"ossische Technische Hochschule Z\"urich\\
  Wolfgang-Pauli-Strasse 27, 8093 Z\"urich, Switzerland\\[1ex]
  \href{mailto:nbeisert@itp.phys.ethz.ch}
  {\texttt{nbeisert@itp.phys.ethz.ch}}}
\hypersetup{pdfauthor={Niklas Beisert}}
\hypersetup{pdfsubject={Manual for the LaTeX2e Package childdoc}}
\date{30 December 2018, \textsf{v2.0}}
\maketitle

\begin{abstract}\noindent
\textsf{childdoc} is a \LaTeXe{} package
that enables the direct compilation
of document sections included by |\include|
to individual files.
\end{abstract}

\begingroup
\parskip0ex
\tableofcontents
\endgroup

%%%%%%%%%%%%%%%%%%%%%%%%%%%%%%%%%%%%%%%%%%%%%%%%%%%%%%%%%%%%%%%%%%%%%%%%%%%%%%%%
%%%%%%%%%%%%%%%%%%%%%%%%%%%%%%%%%%%%%%%%%%%%%%%%%%%%%%%%%%%%%%%%%%%%%%%%%%%%%%%%
\section{Introduction}

\LaTeX{} provides a mechanism to structure a large document (such as a book)
into a main file and several child files (containing the chapters)
using the |\include| command.
This mechanism is beneficial for documents
which span hundreds of pages in order to
make the source file(s) more manageable.
Moreover, compilation can be restricted to
selected child files by means of the |\includeonly| command.
The latter feature can be used to reduce the compilation time while editing
(this was significantly more useful in the earlier days of \LaTeX{})
or to generate a smaller document which is easier to navigate.
Another application of |\includeonly| is to generate
documents consisting of selected parts of the complete document.

However, there are a few drawbacks of the plain |\include| mechanism:
\begin{itemize}
\item
The child files cannot be compiled on their own,
they can only be compiled via the main file.
A naive editing environment
(such as a text editor with an option
to have the current file processed by \LaTeX)
may require one to switch to the main file before compiling;
attempting to compile the child file produces errors.
\item
The main file must be modified (each time)
to adjust the |\includeonly| command
to the present needs. This easily leaves the main file in a messy state.
\item
The generated document will always carry the filename
of the main document. This is inconvenient if
several child files are to be compiled and
to be kept for distribution.
\end{itemize}

The present package provides a simple interface
to make child files individually compilable by \LaTeX{}.
Compiling a child file then has the same effect as compiling
the main file with an |\includeonly| command
to select the appropriate child.
Moreover the generated document will carry the name of the child
rather than the main file.
This resolves all three above issues.

This feature is meant to make the editing of books,
thesis documents and lecture notes somewhat more convenient.
However, the package can also be used efficiently for
composing a series of documents (such as exercise sheets)
which are typically distributed individually.
It then assists the author in generating the individual documents
(potentially in different versions)
as well as a document containing the collected series.
Another application is in developing style files
or other kinds of included material
where compilation of the style file could redirect
to a sample or test file.

%%%%%%%%%%%%%%%%%%%%%%%%%%%%%%%%%%%%%%%%%%%%%%%%%%%%%%%%%%%%%%%%%%%%%%%%%%%%%%%%
%%%%%%%%%%%%%%%%%%%%%%%%%%%%%%%%%%%%%%%%%%%%%%%%%%%%%%%%%%%%%%%%%%%%%%%%%%%%%%%%
\section{Usage}

First of all, the package \textsf{childdoc} is \emph{not} a standard
\LaTeXe{} |.sty| style file! Therefore it needs to be invoked in
a non-standard way.

%%%%%%%%%%%%%%%%%%%%%%%%%%%%%%%%%%%%%%%%%%%%%%%%%%%%%%%%%%%%%%%%%%%%%%%%%%%%%%%%
\subsection{Included Files}
\label{sec:include}

%%%%%%%%%%%%%%%%%%%%%%%%%%%%%%%%%%%%%%%%
\DescribeMacro{\childdocmain}
To use the package, add the commands
\begin{center}
\begin{tabular}{l}
|\input{childdoc.def}|\\
|\childdocmain{}|\\
\end{tabular}
\end{center}
at the very top of the main \LaTeX{} file,
in particular \emph{before} the |\documentclass| statement!
The argument of |\childdocmain| should be left empty
(but it must be present).

%%%%%%%%%%%%%%%%%%%%%%%%%%%%%%%%%%%%%%%%
\DescribeMacro{\childdocof}
Furthermore, add the commands
\begin{center}
\begin{tabular}{l}
|\input{childdoc.def}|\\
|\childdocof{|\textit{main}|}|\\
\end{tabular}
\end{center}
at the top of every child file \textit{child}
which is included by |\include{|\textit{child}|}|
from within the main file
(or at least for those files to be compiled individually).
The argument \textit{main} must be the filename of the main file.

There are a couple of
considerations in setting up the main and child documents:

%%%%%%%%%%%%%%%%%%%%%%%%%%%%%%%%%%%%%%%%
\paragraph{Restrictions.}

Please note the following restrictions:
\begin{itemize}
\item
|\childdocmain| must be called with one argument \textit{main}
to ensure compatibility with earlier version of the package.
It must either be empty (|\childdocmain{}|)
or precisely match the filename of the main file in which it is specified.
See \secref{sec:detection} for further information.
\item
The filename \textit{main} must be specified without the |.tex| extension.
\item
The filename \textit{main} is case sensitive
(even in case-insensitive file systems)
due to internal string comparison.
\item
The argument \textit{main} should be fully expanded, it cannot be a macro.
\item
Subdirectories and special characters should be avoided in filenames.
\item
The command |\childdocmain{|\textit{main}|}| must be followed by a whitespace.
It should not be followed immediately by another command
or by a comment mark `|%|'.
This is because the \TeX{} parser reads the token immediately following
the argument of |\childdocmain| and puts it
at the beginning of every child section;
however, a white\-space is ignored.
\end{itemize}

%%%%%%%%%%%%%%%%%%%%%%%%%%%%%%%%%%%%%%%%
\paragraph{Content of Main File.}

It is advisable to place all content in the child files included by |\include|.
Any output contained in the main file will appear in all child documents
unless suppressed manually;
it cannot be suppressed automatically by the |\includeonly| directive
and thus should normally be avoided.
A method to include some content in the main file
by means of conditional processing is described in \secref{sec:conditional}.

%%%%%%%%%%%%%%%%%%%%%%%%%%%%%%%%%%%%%%%%
\paragraph{Page Numbering.}

When only a part of the document is compiled,
the appropriate numbering of pages
(as well as other status parameters)
is determined from the |.aux| files.
The latter contain information from previous passes.
However this information needs to propagate through
all intermediate child documents.
Therefore the page numbering in child documents may well
be inconsistent until the complete document is compiled at least once.

A useful (if unconventional) way to always ensure a consistent
page numbering is to restart the numbering in each child document
and denote the pages by `\textit{child}|.|\textit{page}'
where \textit{child} represents the chapter/section number of the child file.
This can be achieved by the command
|\numberwithin{page}{|\textit{child}|}|
of the \textsf{amsmath} package
where \textit{child} can be |chapter| or |section|
depending on the chosen structuring.
Alternatively, one can modify the macro |\thepage| appropriately
and reset the counter |page| at the start of each child file.

%%%%%%%%%%%%%%%%%%%%%%%%%%%%%%%%%%%%%%%%%%%%%%%%%%%%%%%%%%%%%%%%%%%%%%%%%%%%%%%%
\subsection{Conditional Processing}
\label{sec:conditional}

The package provides a mechanism to compile different versions
of a document. To customise the versions further some conditional processing
can come in handy to distinguish which version is being compiled.
The package provides two macros to describe the compilation context:

%%%%%%%%%%%%%%%%%%%%%%%%%%%%%%%%%%%%%%%%
\DescribeMacro{\ifchilddoc}
The conditional |\ifchilddoc| distinguishes between the compilation of
child documents and the main document:
%
\begin{center}
|\ifchilddoc |\textit{child-code}| |[|\||else |\textit{main-code}]| \||fi|
\end{center}

%%%%%%%%%%%%%%%%%%%%%%%%%%%%%%%%%%%%%%%%
\DescribeMacro{\childdocname}
\DescribeMacro{\childdocjob}
The macro |\childdocname| contains the filename (without extension)
of the main or child file being processed.
Note that |\childdocjob| will always contain the name of the main file.

%%%%%%%%%%%%%%%%%%%%%%%%%%%%%%%%%%%%%%%%
\paragraph{Title Page.}

Conditional processing can be used to include a title or banner page
in the main document when proper precautions are taken.
Importantly, the code in the main file should ensure that the page counter
(as well as other status parameters which are stored in the |.aux| files)
takes the same value after the conditional processing.
Otherwise the page numbers may take divergent values
depending on which part is compiled.

For example, a title page could be declared by:
%
\begin{center}
\begin{tabular}{l}
|\ifchilddoc\||else|\\
|\addtocounter{page}{-1}|\\
\textit{code for title page}\\
|\newpage|\\
|\||fi|
\end{tabular}
\end{center}
%
A banner page for the child documents can be generated by:
%
\begin{center}
\begin{tabular}{l}
|\ifchilddoc|\\
|\addtocounter{page}{-1}|\\
\textit{code for banner page}\\
|\newpage|\\
|\||fi|
\end{tabular}
\end{center}
%
Here one could write a message such as:
\begin{center}
|This is the part \childdocname{} of \childdocjob{}.|
\end{center}

%%%%%%%%%%%%%%%%%%%%%%%%%%%%%%%%%%%%%%%%%%%%%%%%%%%%%%%%%%%%%%%%%%%%%%%%%%%%%%%%
\subsection{Flags}
\label{sec:flags}

The package makes it easy to generate different versions
of the main or child documents.
To this end compilation flags can be defined
and assigned different default values.
They will be particularly useful in conjunction
with the forwarding mechanism described in \secref{sec:forward}.

For example, it may be useful to have a flag |\version|
which can be set to |draft| or |final|.
The document source will contain some conditional code
depending on the value of |\version|.
Suppose further, the flag should default to |final| for the main file
and to |draft| for child files
which is a natural assignment for editing the document.
This is achieved by placing the following code
in the preamble of the main document
(below the |\childdocmain| directive):
%
\begin{center}
\begin{tabular}{l}
|\ifchilddoc|\\
|\providecommand{\version}{draft}|\\
|\||else|\\
|\providecommand{\version}{final}|\\
|\||fi|
\end{tabular}
\end{center}
%
The definition by |\providecommand| makes sure
that previous definitions are not overwritten.
Further statements |\providecommand{\version}{...}|
can thus be added before the above code to override it.

For the main file, one might add a line
(between |\childdocmain| and the above block)
%
\begin{center}
|%\ifchilddoc\||else\providecommand{\version}{draft}\||fi|
\end{center}
%
which can be uncommented to produce a draft version.
Likewise one can add a line to the very top of a child file
(above the |\childdocof{|\textit{main}|}| directive)
%
\begin{center}
|%\providecommand{\version}{final}|
\end{center}
%
which can be uncommented to produce the final version of this child document.

%%%%%%%%%%%%%%%%%%%%%%%%%%%%%%%%%%%%%%%%%%%%%%%%%%%%%%%%%%%%%%%%%%%%%%%%%%%%%%%%
\subsection{Forwarding}
\label{sec:forward}

Different versions of the main or child documents
using compilation flags as described in \secref{sec:flags}
can be (permanently) stored in different files
for convenient compilation, viewing and distribution.
To this end, the package defines a command
to pass on compilation to a different file:

%%%%%%%%%%%%%%%%%%%%%%%%%%%%%%%%%%%%%%%%
\DescribeMacro{\childdocforward}
The command |\childdocforward| redirects processing to
another source file:
%
\begin{center}
\begin{tabular}{l}
|\input{childdoc.def}|\\
|\childdocforward[|\textit{main}|]{|\textit{dest}|}|\\
\end{tabular}
\end{center}
%
The argument \textit{dest} is the destination file
(without extension).
It should be the main file or one of the child files.
Note that further \textsf{childdoc} directives
such as |\childdocof| and |\childdocforward|
in the indicated file will be processed in this form.
The optional argument \textit{main}
passes on directly to the main file \textit{main}
while pretending to compile the child \textit{dest}.
This form behaves as if \textit{dest}
issues |\childdocof{|\textit{main}|}| right away,
and no further \textsf{childdoc} directives will be processed.

%%%%%%%%%%%%%%%%%%%%%%%%%%%%%%%%%%%%%%%%
\DescribeMacro{\...prefix}
In the alternative form |\childdocforwardprefix|,
%
\begin{center}
\begin{tabular}{l}
|\input{childdoc.def}|\\
|\childdocforwardprefix[|\textit{main}|]{|\textit{prefix}|}{|\textit{dest}|}|
\end{tabular}
\end{center}
%
the destination file is determined by a pattern
depending on the current file:
To make this work, the current file must be called
`{\textit{prefix}\hspace{0.2em}\textit{suffix}}'
with \textit{prefix} matching precisely the argument.
Processing is then passed on to the file
`{\textit{dest}\hspace{0.2em}\textit{suffix}}'.
Surely, the same effect is achieved by
directly specifying the
argument `{\textit{dest}\hspace{0.2em}\textit{suffix}}'
in the first form.
However, that requires to set up a different file
for each child. With the alternative form of the command
all these files can have exactly the same content
which simplifies setting them up and maintaining them.

For example, the following file |draft.tex|
with a compilation flag |\version| as described in \secref{sec:flags}
compiles the main document as a draft:
%
\begin{center}
\begin{tabular}{l}
|\def\version{draft}|\\
|\input{childdoc.def}|\\
|\childdocforward{|\textit{main}|}|
\end{tabular}
\end{center}
%
Likewise, the following files |final|\textit{nn}|.tex|
compile the final version of the child document
|child|\textit{nn}|.tex|:
%
\begin{center}
\begin{tabular}{l}
|\def\version{final}|\\
|\input{childdoc.def}|\\
|\childdocforwardprefix{final}{child}|
\end{tabular}
\end{center}
%

Note that when several versions of a main file and/or of each child file
are to be generated, it may be convenient to set up a |Makefile| or
shell script to automatise the process.

%%%%%%%%%%%%%%%%%%%%%%%%%%%%%%%%%%%%%%%%%%%%%%%%%%%%%%%%%%%%%%%%%%%%%%%%%%%%%%%%
\subsection{Command Line Processing}
\label{sec:commandline}

The effect of redirection files can also be achieved by invoking
the \LaTeX{} compiler with a more elaborate command line.
Most conveniently this should be done as part
of a shell script or a |Makefile|.

When using \textsf{childdoc} in the main file, the following
command lines effectively perform a redirection
(note that depending on the shell being used,
backslashes may have to be doubled: `|\|' $\to$ `|\\|'):
%
\begin{center}
|... -jobname "|\textit{target}|" |\\|"|[\textit{flags}]%
|\input{childdoc.def}\childdocforward[|\textit{main}|]{|\textit{dest}|}"|
\end{center}
%
Here \textit{target} is the name of the output file,
\textit{main} is the name of the main file
and \textit{dest} is the name of the main or child file to be processed
(all filenames without extensions).
The optional argument \textit{main} can be omitted
if \textit{main} matches \textit{dest}.
Optionally, compilation \textit{flags} can be defined via |\def| commands.
This command line makes the \TeX{} engine believe
it is compiling the file \textit{target}
whose content is specified as the latter parameter.
The provided code then forwards the processing to
\textit{main} or \textit{dest} as described in \secref{sec:forward}.

%%%%%%%%%%%%%%%%%%%%%%%%%%%%%%%%%%%%%%%%%%%%%%%%%%%%%%%%%%%%%%%%%%%%%%%%%%%%%%%%
\subsection{Include by Input}
\label{sec:input}

Including child documents by |\include| has some restrictions by design.
Most notably, the content of a child document always occupies
its own set of pages; pages cannot be shared between child documents.
Usually, this behaviour makes perfect sense
because each child document contain an essential part of the document.
However, in some situations it may be desirable to compose
a document from a collection of parts
without having mandatory page breaks between then.
For this case, the package
provides a mechanism to include parts
by |\input| which can also be processed individually.
However, by construction this mechanism
requires manual handling of the content to be output.

%%%%%%%%%%%%%%%%%%%%%%%%%%%%%%%%%%%%%%%%
\DescribeMacro{\ifchilddocmanual}
The main file should be prepared as usual, see \secref{sec:include}.
However, the document body must make a distinction
between processing of an individual part and of the main document, e.g.:
%
\begin{center}
\begin{tabular}{l}
|\ifchilddocmanual|\\
|\input{\childdocname}|\\
|\||else|\\
\textit{document body with }|\input{|\textit{part}|}|\\
|\||fi|
\end{tabular}
\end{center}
%
The conditional |\ifchilddocmanual| is true whenever
a part to be included by |\input| is being compiled,
and the name of the part is stored in |\childdocname|.

%%%%%%%%%%%%%%%%%%%%%%%%%%%%%%%%%%%%%%%%
\DescribeMacro{\childdocby}
Each part to be included by |\input| should start with:
%
\begin{center}
\begin{tabular}{l}
|\input{childdoc.def}|\\
|\childdocby{|\textit{main}|}|\\
\end{tabular}
\end{center}
%
The directive |\childdocby| is similar to |\childdocof|
described in \secref{sec:include},
but the subsequent selection of content must be done manually.
To that end, both |\ifchilddoc| and |\ifchilddocmanual|
will be true upon processing of a part,
and the name of the part is stored in |\childdocname|.
Note that |\jobname| will be set to the filename of the current part
so that each part receives an individual |.aux| file
that does not interfere with the |.aux| file(s) of the main document.
This behaviour can be altered by the alternative form
|\childdocby[*]{|\textit{main}|}| (with a non-empty optional argument)
which uses the |.aux| file of the main document
by setting |\jobname| to \textit{main}.

%%%%%%%%%%%%%%%%%%%%%%%%%%%%%%%%%%%%%%%%%%%%%%%%%%%%%%%%%%%%%%%%%%%%%%%%%%%%%%%%
\subsection{Driver Development}
\label{sec:driver}

The \textsf{childdoc} mechanism can also be use for the development
of definition files such as \LaTeX{} styles or classes.
This case differs from the above setup with multiple parts
included by |\include| in that no |\includeonly| should be invoked.
This can be achieved by starting the include file
(before |\ProvidesPackage|) with:
%
\begin{center}
\begin{tabular}{l}
|\input{childdoc.def}|\\
|\childdocforward{|\textit{main}|}|\\
\end{tabular}
\end{center}
%
or alternatively with:
%
\begin{center}
\begin{tabular}{l}
|\input{childdoc.def}|\\
|\childdocby{|\textit{main}|}|\\
\end{tabular}
\end{center}
%
Both forms have slightly different effects as described above.
The main file is prepared as usual, see \secref{sec:include}.

%%%%%%%%%%%%%%%%%%%%%%%%%%%%%%%%%%%%%%%%%%%%%%%%%%%%%%%%%%%%%%%%%%%%%%%%%%%%%%%%
\subsection{Legacy Detection}
\label{sec:detection}

The directive |\childdocmain| in the main file can detect
whether the complete document or merely a child is to be compiled
even without using the directive |\childdocof|.
This method is deprecated because it is less robust
and there is no compelling reason to use it;
it is merely provided for backward compatibility
and it may be removed in future versions.

If the detection mechanism is to be used,
it is mandatory to correctly specify
the filename of the main file as the argument of |\childdocmain|:
%
\begin{center}
\begin{tabular}{l}
|\input{childdoc.def}|\\
|\childdocmain{|\textit{main}|}|\\
\end{tabular}
\end{center}
%
If |\jobname| does not match the argument \textit{main} of |\childdocmain|,
it is assumed that |\jobname| points to the child file to be compiled.
When using |\childdocmain| with the main file specified as argument,
it suffices to start a child file
with just |\input{|\textit{main}|}|
without loading of the package and using |\childdocof|.
If instead all processing is done
with the appropriate \textsf{childdoc} directives,
the argument of \textit{main} of |\childdocmain| can be empty.

An alternative version of the command line processing described
in \secref{sec:commandline} using the detection mechanism reads:
%
\begin{center}
|... -jobname "|\textit{target}|" "|[\textit{flags}]%
[|\def\jobname{|\textit{dest}|}|]|\input{|\textit{main}|}"|
\end{center}

%%%%%%%%%%%%%%%%%%%%%%%%%%%%%%%%%%%%%%%%%%%%%%%%%%%%%%%%%%%%%%%%%%%%%%%%%%%%%%%%
\subsection{Manual Code}
\label{sec:manual}

In case one cannot be certain whether the definitions file |childdoc.def|
is installed on the target \TeX{} distribution
and one prefers not to ship it,
it is conceivable to paste a few relevant commands into the sources.

To that end, drop all statements |\input{childdoc.def}|
and perform the replacements as outlined below.
Instead of |\childdocmain{|\textit{main}|}| add the following code
to the top of the main file:
%
\begin{center}
\begin{tabular}{l}
|\||ifdefined\childdocname\endinput\||fi\newif\ifchilddoc|\\
|\edef\childdocname{\scantokens\expandafter{\jobname\noexpand}}|\\
|\def\childdocmain{|\textit{main}|}\||ifx\childdocmain\childdocname\||else|\\
|\childdoctrue\includeonly{\childdocname}\let\jobname\childdocmain\||fi|\\
\end{tabular}
\end{center}
%
Instead of |\childdocof{|\textit{main}|}| just include the main file
at the top of each child file:
%
\begin{center}
|\input{|\textit{main}|}|
\end{center}
%
A simple redirection |\childdocforward{|\textit{dest}|}| is achieved by:
%
\begin{center}
|\def\jobname{|\textit{dest}|}\input{\jobname}|
\end{center}
%
The redirection with prefix
|\childdocforwardprefix[|\textit{prefix}|]{|\textit{dest}|}|
is accomplished by:
%
\begin{center}
\begin{tabular}{l}
|{\edef\jobname{\scantokens\expandafter{\jobname\noexpand}}|\\
|\def\redirectjob |\textit{prefix}|#1~~~{\gdef\jobname{|\textit{dest}|#1}}|\\
|\expandafter\redirectjob\jobname~~~}\input{\jobname}|
\end{tabular}
\end{center}

In an alternative approach,
child documents can be compiled by a specific command line
without additional code or specific definitions:
%
\begin{center}
|... -jobname "|\textit{target}|" "|[\textit{flags}]%
|\includeonly{|\textit{dest}|}\input{|\textit{main}|}"|
\end{center}
%

%%%%%%%%%%%%%%%%%%%%%%%%%%%%%%%%%%%%%%%%%%%%%%%%%%%%%%%%%%%%%%%%%%%%%%%%%%%%%%%%
%%%%%%%%%%%%%%%%%%%%%%%%%%%%%%%%%%%%%%%%%%%%%%%%%%%%%%%%%%%%%%%%%%%%%%%%%%%%%%%%
\section{Information}

%%%%%%%%%%%%%%%%%%%%%%%%%%%%%%%%%%%%%%%%%%%%%%%%%%%%%%%%%%%%%%%%%%%%%%%%%%%%%%%%
\subsection{Copyright}

Copyright \copyright{} 2017--2018 Niklas Beisert

This work may be distributed and/or modified under the
conditions of the \LaTeX{} Project Public License, either version 1.3
of this license or (at your option) any later version.
The latest version of this license is in
  \url{http://www.latex-project.org/lppl.txt}
and version 1.3 or later is part of all distributions of \LaTeX{}
version 2005/12/01 or later.

This work has the LPPL maintenance status `maintained'.

The Current Maintainer of this work is Niklas Beisert.

This work consists of the files |README.txt|, |childdoc.ins| and |childdoc.dtx|
as well as the derived files |childdoc.def|, |cdocsamp.tex|
with |cdocsch1.tex|, |cdocsch2.tex|, |cdocspt3.tex|, |cdocspt4.tex|,
|cdocsdrf.tex|, |cdocsfn1.tex|, |cdocsfn2.tex|
as well as |childdoc.pdf|.

%%%%%%%%%%%%%%%%%%%%%%%%%%%%%%%%%%%%%%%%%%%%%%%%%%%%%%%%%%%%%%%%%%%%%%%%%%%%%%%%
\subsection{Files and Installation}

The package consists of the files:
%
\begin{center}
\begin{tabular}{ll}
    |README.txt|   & readme file \\
    |childdoc.ins| & installation file \\
    |childdoc.dtx| & source file \\
    |childdoc.def| & definition file \\
    |cdocsamp.tex| & sample main file \\
    |cdocsch1.tex| & sample include file \\
    |cdocsch2.tex| & sample include file \\
    |cdocspt3.tex| & sample part file \\
    |cdocspt4.tex| & sample part file \\
    |cdocsdrf.tex| & sample redirection file \\
    |cdocsfn1.tex| & sample redirection file \\
    |cdocsfn2.tex| & sample redirection file \\
    |childdoc.pdf| & manual
\end{tabular}
\end{center}
%
The distribution consists of the files
|README.txt|, |childdoc.ins| and |childdoc.dtx|.
%
\begin{itemize}
\item
Run (pdf)\LaTeX{} on |childdoc.dtx|
to compile the manual |childdoc.pdf| (this file).
\item
Run \LaTeX{} on |childdoc.ins| to create the definitions file |childdoc.def|
and the sample |cdocsamp.tex| with include files
|cdocsch1.tex|, |cdocsch2.tex|, |cdocspt3.tex|, |cdocspt4.tex|,
|cdocsdrf.tex|, |cdocsfn1.tex|, |cdocsfn2.tex|.
Then copy the file |childdoc.def| to an appropriate directory of your \LaTeX{}
distribution, e.g.\ \textit{texmf-root}|/tex/latex/childdoc|.
\end{itemize}

%%%%%%%%%%%%%%%%%%%%%%%%%%%%%%%%%%%%%%%%%%%%%%%%%%%%%%%%%%%%%%%%%%%%%%%%%%%%%%%%
\subsection{Related CTAN Packages}

There are several other packages which offer a similar functionality:
%
\begin{itemize}
\item
The packages
\href{http://ctan.org/pkg/docmute}{\textsf{docmute}},
\href{http://ctan.org/pkg/includex}{\textsf{includex}} and
\href{http://ctan.org/pkg/standalone}{\textsf{standalone}}
provide commands to include only the document body of
a child file thus allowing both files to be compiled individually.
\item
The packages \href{http://ctan.org/pkg/subdocs}{\textsf{subdocs}}
and \href{http://ctan.org/pkg/subfiles}{\textsf{subfiles}}
provide structures in which the main and child documents can be
encapsulated and allowing them to be compiled individually.
The inclusion mechanism is different from the conventional |\include|.
\item
The package \href{http://ctan.org/pkg/combine}{\textsf{combine}}
is an elaborate solution to combine several documents into one.
\end{itemize}
%
See also the CTAN topic \href{http://ctan.org/topic/subdocs}{\textsf{subdocs}}
for further related packages.
The present package differs from the above solutions in that
a document structure constructed with the conventional |\include| mechanism
just needs two extra commands at the top of every file
such that all constituent files can be compiled individually.

%%%%%%%%%%%%%%%%%%%%%%%%%%%%%%%%%%%%%%%%%%%%%%%%%%%%%%%%%%%%%%%%%%%%%%%%%%%%%%%%
%\subsection{Feature Suggestions}
%
%The following is a list of features which may be useful for future
%versions of this package:
%%
%\begin{itemize}
%\item
%\ldots
%\end{itemize}

%%%%%%%%%%%%%%%%%%%%%%%%%%%%%%%%%%%%%%%%%%%%%%%%%%%%%%%%%%%%%%%%%%%%%%%%%%%%%%%%
\subsection{Revision History}

%%%%%%%%%%%%%%%%%%%%%%%%%%%%%%%%%%%%%%%%
\paragraph{v2.0:} 2018/12/30

\begin{itemize}
\item
immediate forward processing
\item
added |\childdocby| mechanism
\item
manual restructured
\end{itemize}

%%%%%%%%%%%%%%%%%%%%%%%%%%%%%%%%%%%%%%%%
\paragraph{v1.6:} 2018/01/17

\begin{itemize}
\item
application for development of include files
\item
corrections to manual
\end{itemize}

%%%%%%%%%%%%%%%%%%%%%%%%%%%%%%%%%%%%%%%%
\paragraph{v1.5:} 2017/05/21

\begin{itemize}
\item
more complete structuring introduced
\item
|\childdocof| introduced
\item
|\childdoc| renamed to |\childdocmain|
\item
|\childredirect| renamed to |\childdocforward| and |\childdocforwardprefix|
and functionality expanded
\end{itemize}

%%%%%%%%%%%%%%%%%%%%%%%%%%%%%%%%%%%%%%%%
\paragraph{v1.0:} 2017/04/27

\begin{itemize}
\item
manual and install package
\item
first version published on CTAN
\end{itemize}

%%%%%%%%%%%%%%%%%%%%%%%%%%%%%%%%%%%%%%%%
\paragraph{v0.6:} 2017/04/26

\begin{itemize}
\item
redirection mechanism added
\end{itemize}

%%%%%%%%%%%%%%%%%%%%%%%%%%%%%%%%%%%%%%%%
\paragraph{v0.5:} 2017/04/26

\begin{itemize}
\item
functionality in definition file
\end{itemize}


%%%%%%%%%%%%%%%%%%%%%%%%%%%%%%%%%%%%%%%%%%%%%%%%%%%%%%%%%%%%%%%%%%%%%%%%%%%%%%%%
%%%%%%%%%%%%%%%%%%%%%%%%%%%%%%%%%%%%%%%%%%%%%%%%%%%%%%%%%%%%%%%%%%%%%%%%%%%%%%%%
%%%%%%%%%%%%%%%%%%%%%%%%%%%%%%%%%%%%%%%%%%%%%%%%%%%%%%%%%%%%%%%%%%%%%%%%%%%%%%%%
\appendix

\settowidth\MacroIndent{\rmfamily\scriptsize 000\ }

 \DocInput{childdoc.dtx}

\end{document}
%</driver>
% \fi
%
% %%%%%%%%%%%%%%%%%%%%%%%%%%%%%%%%%%%%%%%%%%%%%%%%%%%%%%%%%%%%%%%%%%%%%%%%%%%%%%
% %%%%%%%%%%%%%%%%%%%%%%%%%%%%%%%%%%%%%%%%%%%%%%%%%%%%%%%%%%%%%%%%%%%%%%%%%%%%%%
% \section{Sample}
%\iffalse
%<*samplemain>
%\fi
%
% The following presents a sample document
% with two chapters, two parts, a title page,
% a compile flag as well as three forwarding files to set the flag.
% It consists of eight |.tex| files:
% \begin{center}
% \begin{tabular}{ll}
% |cdocsamp.tex|&main file\\
% |cdocsch1.tex|&include file for chapter 1\\
% |cdocsch2.tex|&include file for chapter 2\\
% |cdocspt3.tex|&include file for part 3\\
% |cdocspt4.tex|&include file for part 4\\
% |cdocsdrf.tex|&forwarding file for main file in draft mode\\
% |cdocsfi1.tex|&forwarding file for final version of chapter 1\\
% |cdocsfi2.tex|&forwarding file for final version of chapter 2\\
% \end{tabular}
% \end{center}
% Each of the eight files can be compiled directly by the \LaTeX{} compiler.
%
% %%%%%%%%%%%%%%%%%%%%%%%%%%%%%%%%%%%%%%
% \paragraph{Main File.}
%
% The main file is called |cdocsamp.tex|.
%
% Load the \textsf{childdoc} definitions and
% declare the filename for the main document:
%    \begin{macrocode}
\input{childdoc.def}
\childdocmain{}
%    \end{macrocode}

% Optional override for |\version| flag:
%    \begin{macrocode}
%%\ifchilddoc\else\providecommand{\version}{draft}\fi
%    \end{macrocode}

% Define the default values for the |\version| flag
% (|final| for the main file and |draft| for childs):
%    \begin{macrocode}
\ifchilddoc
\providecommand{\version}{draft}
\else
\providecommand{\version}{final}
\fi
%    \end{macrocode}

% Load the standard document class:
%    \begin{macrocode}
\documentclass[12pt]{article}
%    \end{macrocode}

% Start the document body:
%    \begin{macrocode}
\begin{document}
%    \end{macrocode}

% Declare a title page.
% Print title, part of document being processed and version flag:
%    \begin{macrocode}
\addtocounter{page}{-1}
\begin{center}
{\LARGE\bfseries{}childdoc example\par}
\vspace{1cm}
\ifchilddoc
\ifchilddocmanual part\else chapter\fi:
`\childdocname' of `\childdocjob'\par
\else
main document: `\childdocjob'\par
\fi
version: \version\par
\end{center}
\newpage
%    \end{macrocode}

% Manually include selected file,
% otherwise process as usual:
%    \begin{macrocode}
\ifchilddocmanual
\section*{part `\childdocname'}
\input{\childdocname}
\else
%    \end{macrocode}

% Include the two chapters:
%    \begin{macrocode}
\include{cdocsch1}
\include{cdocsch2}
%    \end{macrocode}

% Include the two parts unless only chapters should be displayed:
%    \begin{macrocode}
\ifchilddoc\else
\section{part three}
\input{cdocspt3}
\section{part four}
\input{cdocspt4}
\fi
%    \end{macrocode}

% Process as usual until here:
%    \begin{macrocode}
\fi
%    \end{macrocode}

% End of document body:
%    \begin{macrocode}
\end{document}
%    \end{macrocode}
%\iffalse
%</samplemain>
%\fi
%
% %%%%%%%%%%%%%%%%%%%%%%%%%%%%%%%%%%%%%%
% \paragraph{Chapter Include Files.}
%
% The include files are called |cdocsch1.tex| and |cdocsch2.tex|.
%
%\iffalse
%<*samplechap1|samplechap2>
%\fi

% Optional override for |\version| flag:
%    \begin{macrocode}
%%\providecommand{\version}{final}
%    \end{macrocode}

% Include the main document:
%    \begin{macrocode}
\input{childdoc.def}
\childdocof{cdocsamp}
%    \end{macrocode}

%\iffalse
%</samplechap1|samplechap2>
%\fi
%
%\iffalse
%<*samplechap1>
%\fi
% Some text for chapter 1:
%    \begin{macrocode}
\section{one}
some text in chapter one
%    \end{macrocode}

%\iffalse
%</samplechap1>
%\fi
% Some text for chapter 2:
%\iffalse
%<*samplechap2>
%\fi
%    \begin{macrocode}
\section{two}
more text in chapter two
%    \end{macrocode}

%\iffalse
%</samplechap2>
%\fi
%
% %%%%%%%%%%%%%%%%%%%%%%%%%%%%%%%%%%%%%%
% \paragraph{Part Include Files.}
%
% The include files are called |cdocspt3.tex| and |cdocspt4.tex|.
%
%\iffalse
%<*samplepart3|samplepart4>
%\fi

% Optional override for |\version| flag:
%    \begin{macrocode}
%%\providecommand{\version}{final}
%    \end{macrocode}

% Include the main document:
%    \begin{macrocode}
\input{childdoc.def}
\childdocby{cdocsamp}
%    \end{macrocode}

%\iffalse
%</samplepart3|samplepart4>
%\fi
%
%\iffalse
%<*samplepart3>
%\fi
% Some text for part 3:
%    \begin{macrocode}
some text in part three
%    \end{macrocode}

%\iffalse
%</samplepart3>
%\fi
% Some text for part 4:
%\iffalse
%<*samplepart4>
%\fi
%    \begin{macrocode}
more text in part four
%    \end{macrocode}

%\iffalse
%</samplepart4>
%\fi
%
% %%%%%%%%%%%%%%%%%%%%%%%%%%%%%%%%%%%%%%
% \paragraph{Forwarding for a Complete Draft.}
%
% The following forwarding file |cdocsdrf.tex|
% compiles the main document in draft mode:
%\iffalse
%<*sampledraft>
%\fi
%    \begin{macrocode}
\def\version{draft}
\input{childdoc.def}
\childdocforward{cdocsamp}
%    \end{macrocode}

%\iffalse
%</sampledraft>
%\fi
%
% %%%%%%%%%%%%%%%%%%%%%%%%%%%%%%%%%%%%%%
% \paragraph{Forwarding for Final Version of the Chapters.}
%
% The following forwarding files |cdocsfn1.tex| and |cdocsfn2.tex|
% (with identical content)
% compile the final versions of the child documents
% |cdocsch1.tex| and |cdocsch2.tex|, respectively:
%\iffalse
%<*samplefinal>
%\fi
%    \begin{macrocode}
\def\version{final}
\input{childdoc.def}
\childdocforwardprefix[cdocsamp]{cdocsfn}{cdocsch}
%    \end{macrocode}

%\iffalse
%</samplefinal>
%\fi
%
% %%%%%%%%%%%%%%%%%%%%%%%%%%%%%%%%%%%%%%
% \paragraph{Command Line Processing.}
%
% The following three command lines generate the output files
% |cdocscld|, |cdocscl1| and |cdocscl2|
% which should be identical to
% |cdocsdrf|, |cdocsch1| and |cdocsfn2|, respectively:
% \begin{center}
% \begin{tabular}{l}
% |latex -jobname cdocscld \|\\
% |  "\def\version{draft}\input{childdoc.def}\childdocforward{cdocsamp}"|\\
% |latex -jobname cdocscl1 \|\\
% |  "\input{childdoc.def}\childdocforward[cdocsamp]{cdocsch1}"|\\
% |latex -jobname cdocscl2 \|\\
% |  "\def\version{final}\input{childdoc.def}\childdocforward{cdocsch2}"|
% \end{tabular}
% \end{center}
% Note that the trailing backslash on each first line
% merely continues the input to the second line
% (for convenient cut ant paste).
% Furthermore, the command |latex| can be replaced by any
% of its alternative versions such as |pdflatex|.
%
% %%%%%%%%%%%%%%%%%%%%%%%%%%%%%%%%%%%%%%%%%%%%%%%%%%%%%%%%%%%%%%%%%%%%%%%%%%%%%%
% %%%%%%%%%%%%%%%%%%%%%%%%%%%%%%%%%%%%%%%%%%%%%%%%%%%%%%%%%%%%%%%%%%%%%%%%%%%%%%
% \section{Implementation}
%\iffalse
%<*package>
%\fi
%
% This section describes the definitions file |childdoc.def|.

% The definitions cannot be loaded using |\usepackage| or |\RequirePackage|
% which has a mechanism to prevent loading a style file more than once.
% When loading the definitions by means of |\input|
% multiple instances have to be prevented manually:
%\iffalse
%This code needs to be before the `\ProvidesFile' directive
%which is defined at the beginning of this file.
%Therefore it is also placed there and commented out here.
%</package>
%<*discard>
%\fi
%    \begin{macrocode}
\ifdefined\childdocmain\endinput\fi
%    \end{macrocode}
%\iffalse
%</discard>
%<*package>
%\fi
%
% \macro{\ifchilddoc}
% \macro{\ifchilddocmanual}
% The conditional |\ifchilddoc| tells whether a
% child (true) or main (false) document is being compiled.
% The conditional |\ifchilddocmanual| tells whether
% the |\includeonly| mechanism is used (false) or
% the selection of child files must be performed manually (true).
% The definitions initialise to false:
%    \begin{macrocode}
\newif\ifchilddoc
\newif\ifchilddocmanual
%    \end{macrocode}

% \macro{\childdocname}
% \macro{\childdocjob}
% The macro |\childdocname| stores the name of the main document
% to be compiled. The macro |\childdocjob| stores the name of
% the document on which the \LaTeX{} compiler was originally invoked.
% The content of |\jobname| cannot be compared
% to filenames specified in the source due to different catcodes.
% The following code rescans |\jobname|, stores the result
% in |\childdocname| and saves a copy in |\childdocjob|:
%    \begin{macrocode}
\edef\childdocname{\scantokens\expandafter{\jobname\noexpand}}
\let\childdocjob\childdocname
%    \end{macrocode}

% \macro{\childdocdisable}
% The macro |\childdocdisable| prevents the main file
% from being processed more than once.
% At this stage, the main document command |\childdocmain|
% is assumed to be called once again where it should do nothing.
% Any subsequent call to it should prevent
% a secondary processing of the main document
% It overwrites the forwarding commands
% |\childdocof| and |\childdocforward|
% with empty macros to prevent further inclusions of the main document:
%    \begin{macrocode}
\newcommand{\childdocdisable}
{
  \renewcommand{\childdocmain}[1]{\renewcommand{\childdocmain}[1]{\endinput}}
  \renewcommand{\childdocof}[1]{}
  \renewcommand{\childdocby}[2][]{}
  \renewcommand{\childdocforward}[2][]{}
  \renewcommand{\childdocdisable}{}
}
%    \end{macrocode}

% \macro{\childdocmain}
% The macro |\childdocmain| is to be called at the top of the main file
% with nothing or the main filename (without extension) as argument.
% First, it breaks loops.
% If the argument is not empty and does not match |\childdocname|
% (which is set by the first inclusion of |childdoc.def|),
% |\ifchilddoc| is set to true, |\includeonly| is applied to the child file
% and |\jobname| is set to the main file
% (for proper handling of |.aux| files):
%    \begin{macrocode}
\newcommand{\childdocmain}[1]
{
  \childdocdisable\childdocmain{}
  \if?#1?\else
    \begingroup
      \def\childdoctmp{#1}
      \ifx\childdoctmp\childdocname
        \def\childdoctmp{}
      \else
        \def\childdoctmp
        {
          \childdoctrue
          \includeonly{\childdocname}
          \def\childdocjob{#1}
          \def\jobname{#1}
        }
      \fi
      \expandafter
    \endgroup
    \childdoctmp
  \fi
}
%    \end{macrocode}

% \macro{\childdocof}
% The command |\childdocof| redirects
% compilation to the main file |#1|.
%    \begin{macrocode}
\newcommand{\childdocof}[1]
{
  \childdocdisable
  \childdoctrue
  \includeonly{\childdocname}
  \def\jobname{#1}
  \def\childdocjob{#1}
  \input{#1}
}
%    \end{macrocode}

% \macro{\childdocby}
% The command |\childdocby| ....
%    \begin{macrocode}
\newcommand{\childdocby}[2][]
{
  \childdocdisable
  \childdoctrue
  \childdocmanualtrue
  \if?#1?\else
    \def\jobname{#2}
  \fi
  \def\childdocjob{#2}
  \input{#2}
  \endinput
}
%    \end{macrocode}

% \macro{\childdocforward}
% The command |\childdocforward| redirects
% compilation to the main file or
% (if the optional argument is given) a child file.
% Parameters are set as if the main file
% or a child file starting with |\childdocof| was compiled.
% Then compilation is handed over to the main file:
%    \begin{macrocode}
\newcommand{\childdocforward}[2][]
{
  \begingroup
    \if?#1?
      \def\childdoctmp
      {
        \def\childdocname{#2}
        \def\childdocjob{#2}
        \def\jobname{#2}
        \input{#2}
        \endinput
      }
    \else
      \def\childdoctmp
      {
        \childdocdisable
        \def\childdocname{#2}
        \childdoctrue
        \includeonly{#2}
        \def\childdocjob{#1}
        \def\jobname{#1}
        \input{#1}
        \endinput
      }
    \fi
    \expandafter
  \endgroup
  \childdoctmp
}
%    \end{macrocode}

% \macro{\childdocforwardprefix}
% The command |\childdocforwardprefix| redirects
% compilation to the main or a child file by means of a pattern.
% The prefix |#1| in the current filename is replaced by |#2|
% and the suffix of the current filename is kept
% (it is assumed that the filename does not contain the substring `|~~~|'
% which is used as a delimiter).
% Compilation is handed over to the new file by |\childdocforward|:
%    \begin{macrocode}
\newcommand{\childdocforwardprefix}[3][]
{
  \begingroup
    \def\childdocextract #2##1~~~{\def\childdoctmp{\childdocforward[#1]{#3##1}}}
    \expandafter\childdocextract\childdocname~~~
    \expandafter
  \endgroup
  \childdoctmp
}
%    \end{macrocode}

% \macro{\childdoc}
% The deprecated macro |\childdoc| is a legacy version of |\childdocmain|:
%    \begin{macrocode}
\newcommand{\childdoc}{\childdocmain}
%    \end{macrocode}

% \macro{\childdocredirect}
% The deprecated macro |\childdocredirect| is a legacy version
% of |\childdocforward| and |\childdocforwardprefix|:
%    \begin{macrocode}
\newcommand{\childdocredirect}[2][]
{
  \begingroup
    \if?#1?
      \def\childdoctmp{\childdocforward{#2}}
    \else
      \def\childdoctmp{\childdocforwardprefix{#1}{#2}}
    \fi
    \expandafter
  \endgroup
  \childdoctmp
}
%    \end{macrocode}

%\iffalse
%</package>
%\fi
%
\endinput

\childdocmain{}
%    \end{macrocode}

% Optional override for |\version| flag:
%    \begin{macrocode}
%%\ifchilddoc\else\providecommand{\version}{draft}\fi
%    \end{macrocode}

% Define the default values for the |\version| flag
% (|final| for the main file and |draft| for childs):
%    \begin{macrocode}
\ifchilddoc
\providecommand{\version}{draft}
\else
\providecommand{\version}{final}
\fi
%    \end{macrocode}

% Load the standard document class:
%    \begin{macrocode}
\documentclass[12pt]{article}
%    \end{macrocode}

% Start the document body:
%    \begin{macrocode}
\begin{document}
%    \end{macrocode}

% Declare a title page.
% Print title, part of document being processed and version flag:
%    \begin{macrocode}
\addtocounter{page}{-1}
\begin{center}
{\LARGE\bfseries{}childdoc example\par}
\vspace{1cm}
\ifchilddoc
\ifchilddocmanual part\else chapter\fi:
`\childdocname' of `\childdocjob'\par
\else
main document: `\childdocjob'\par
\fi
version: \version\par
\end{center}
\newpage
%    \end{macrocode}

% Manually include selected file,
% otherwise process as usual:
%    \begin{macrocode}
\ifchilddocmanual
\section*{part `\childdocname'}
\input{\childdocname}
\else
%    \end{macrocode}

% Include the two chapters:
%    \begin{macrocode}
\include{cdocsch1}
\include{cdocsch2}
%    \end{macrocode}

% Include the two parts unless only chapters should be displayed:
%    \begin{macrocode}
\ifchilddoc\else
\section{part three}
\input{cdocspt3}
\section{part four}
\input{cdocspt4}
\fi
%    \end{macrocode}

% Process as usual until here:
%    \begin{macrocode}
\fi
%    \end{macrocode}

% End of document body:
%    \begin{macrocode}
\end{document}
%    \end{macrocode}
%\iffalse
%</samplemain>
%\fi
%
% %%%%%%%%%%%%%%%%%%%%%%%%%%%%%%%%%%%%%%
% \paragraph{Chapter Include Files.}
%
% The include files are called |cdocsch1.tex| and |cdocsch2.tex|.
%
%\iffalse
%<*samplechap1|samplechap2>
%\fi

% Optional override for |\version| flag:
%    \begin{macrocode}
%%\providecommand{\version}{final}
%    \end{macrocode}

% Include the main document:
%    \begin{macrocode}
% \iffalse
%
% childdoc.dtx Copyright (C) 2017-2018 Niklas Beisert
%
% This work may be distributed and/or modified under the
% conditions of the LaTeX Project Public License, either version 1.3
% of this license or (at your option) any later version.
% The latest version of this license is in
%   http://www.latex-project.org/lppl.txt
% and version 1.3 or later is part of all distributions of LaTeX
% version 2005/12/01 or later.
%
% This work has the LPPL maintenance status `maintained'.
%
% The Current Maintainer of this work is Niklas Beisert.
%
% This work consists of the files childdoc.dtx and childdoc.ins
% and the derived files childdoc.def and cdocsamp.tex with
% cdocsch1.tex, cdocsch2.tex, cdocsdrf.tex, cdocsfn1.tex, cdocsfn2.tex.
%
%<package>\ifdefined\childdocmain\endinput\fi
%<package>\ProvidesFile{childdoc.def}[2018/12/30 v2.0 child document driver]
%<samplemain>\ProvidesFile{cdocsamp.tex}[2018/12/30 v2.0 sample for childdoc]
%<*driver>
%\ProvidesFile{childdoc.drv}[2018/12/30 v2.0 childdoc reference manual file]
\PassOptionsToClass{10pt,a4paper}{article}
\documentclass{ltxdoc}

\usepackage[margin=35mm]{geometry}
\usepackage{hyperref}
\usepackage{hyperxmp}
\usepackage[usenames]{color}

\hypersetup{colorlinks=true}
\hypersetup{pdfstartview=FitH}
\hypersetup{pdfpagemode=UseNone}
\hypersetup{pdfsource={}}
\hypersetup{pdflang={en-UK}}
\hypersetup{pdfcopyright={Copyright 2017-2018 Niklas Beisert.
  This work may be distributed and/or modified under the
  conditions of the LaTeX Project Public License, either version 1.3
  of this license or (at your option) any later version.}}
\hypersetup{pdflicenseurl={http://www.latex-project.org/lppl.txt}}
\hypersetup{pdfcontactaddress={ETH Zurich, ITP, HIT K,
  Wolfgang-Pauli-Strasse 27}}
\hypersetup{pdfcontactpostcode={8093}}
\hypersetup{pdfcontactcity={Zurich}}
\hypersetup{pdfcontactcountry={Switzerland}}
\hypersetup{pdfcontactemail={nbeisert@itp.phys.ethz.ch}}
\hypersetup{pdfcontacturl={http://people.phys.ethz.ch/\xmptilde nbeisert/}}

\newcommand{\secref}[1]{\hyperref[#1]{section \ref*{#1}}}

\parskip1ex
\parindent0pt
\let\olditemize\itemize
\def\itemize{\olditemize\parskip0pt}

\begin{document}

\title{The \textsf{childdoc} Package}
\hypersetup{pdftitle={The childdoc Package}}
\author{Niklas Beisert\\[2ex]
  Institut f\"ur Theoretische Physik\\
  Eidgen\"ossische Technische Hochschule Z\"urich\\
  Wolfgang-Pauli-Strasse 27, 8093 Z\"urich, Switzerland\\[1ex]
  \href{mailto:nbeisert@itp.phys.ethz.ch}
  {\texttt{nbeisert@itp.phys.ethz.ch}}}
\hypersetup{pdfauthor={Niklas Beisert}}
\hypersetup{pdfsubject={Manual for the LaTeX2e Package childdoc}}
\date{30 December 2018, \textsf{v2.0}}
\maketitle

\begin{abstract}\noindent
\textsf{childdoc} is a \LaTeXe{} package
that enables the direct compilation
of document sections included by |\include|
to individual files.
\end{abstract}

\begingroup
\parskip0ex
\tableofcontents
\endgroup

%%%%%%%%%%%%%%%%%%%%%%%%%%%%%%%%%%%%%%%%%%%%%%%%%%%%%%%%%%%%%%%%%%%%%%%%%%%%%%%%
%%%%%%%%%%%%%%%%%%%%%%%%%%%%%%%%%%%%%%%%%%%%%%%%%%%%%%%%%%%%%%%%%%%%%%%%%%%%%%%%
\section{Introduction}

\LaTeX{} provides a mechanism to structure a large document (such as a book)
into a main file and several child files (containing the chapters)
using the |\include| command.
This mechanism is beneficial for documents
which span hundreds of pages in order to
make the source file(s) more manageable.
Moreover, compilation can be restricted to
selected child files by means of the |\includeonly| command.
The latter feature can be used to reduce the compilation time while editing
(this was significantly more useful in the earlier days of \LaTeX{})
or to generate a smaller document which is easier to navigate.
Another application of |\includeonly| is to generate
documents consisting of selected parts of the complete document.

However, there are a few drawbacks of the plain |\include| mechanism:
\begin{itemize}
\item
The child files cannot be compiled on their own,
they can only be compiled via the main file.
A naive editing environment
(such as a text editor with an option
to have the current file processed by \LaTeX)
may require one to switch to the main file before compiling;
attempting to compile the child file produces errors.
\item
The main file must be modified (each time)
to adjust the |\includeonly| command
to the present needs. This easily leaves the main file in a messy state.
\item
The generated document will always carry the filename
of the main document. This is inconvenient if
several child files are to be compiled and
to be kept for distribution.
\end{itemize}

The present package provides a simple interface
to make child files individually compilable by \LaTeX{}.
Compiling a child file then has the same effect as compiling
the main file with an |\includeonly| command
to select the appropriate child.
Moreover the generated document will carry the name of the child
rather than the main file.
This resolves all three above issues.

This feature is meant to make the editing of books,
thesis documents and lecture notes somewhat more convenient.
However, the package can also be used efficiently for
composing a series of documents (such as exercise sheets)
which are typically distributed individually.
It then assists the author in generating the individual documents
(potentially in different versions)
as well as a document containing the collected series.
Another application is in developing style files
or other kinds of included material
where compilation of the style file could redirect
to a sample or test file.

%%%%%%%%%%%%%%%%%%%%%%%%%%%%%%%%%%%%%%%%%%%%%%%%%%%%%%%%%%%%%%%%%%%%%%%%%%%%%%%%
%%%%%%%%%%%%%%%%%%%%%%%%%%%%%%%%%%%%%%%%%%%%%%%%%%%%%%%%%%%%%%%%%%%%%%%%%%%%%%%%
\section{Usage}

First of all, the package \textsf{childdoc} is \emph{not} a standard
\LaTeXe{} |.sty| style file! Therefore it needs to be invoked in
a non-standard way.

%%%%%%%%%%%%%%%%%%%%%%%%%%%%%%%%%%%%%%%%%%%%%%%%%%%%%%%%%%%%%%%%%%%%%%%%%%%%%%%%
\subsection{Included Files}
\label{sec:include}

%%%%%%%%%%%%%%%%%%%%%%%%%%%%%%%%%%%%%%%%
\DescribeMacro{\childdocmain}
To use the package, add the commands
\begin{center}
\begin{tabular}{l}
|\input{childdoc.def}|\\
|\childdocmain{}|\\
\end{tabular}
\end{center}
at the very top of the main \LaTeX{} file,
in particular \emph{before} the |\documentclass| statement!
The argument of |\childdocmain| should be left empty
(but it must be present).

%%%%%%%%%%%%%%%%%%%%%%%%%%%%%%%%%%%%%%%%
\DescribeMacro{\childdocof}
Furthermore, add the commands
\begin{center}
\begin{tabular}{l}
|\input{childdoc.def}|\\
|\childdocof{|\textit{main}|}|\\
\end{tabular}
\end{center}
at the top of every child file \textit{child}
which is included by |\include{|\textit{child}|}|
from within the main file
(or at least for those files to be compiled individually).
The argument \textit{main} must be the filename of the main file.

There are a couple of
considerations in setting up the main and child documents:

%%%%%%%%%%%%%%%%%%%%%%%%%%%%%%%%%%%%%%%%
\paragraph{Restrictions.}

Please note the following restrictions:
\begin{itemize}
\item
|\childdocmain| must be called with one argument \textit{main}
to ensure compatibility with earlier version of the package.
It must either be empty (|\childdocmain{}|)
or precisely match the filename of the main file in which it is specified.
See \secref{sec:detection} for further information.
\item
The filename \textit{main} must be specified without the |.tex| extension.
\item
The filename \textit{main} is case sensitive
(even in case-insensitive file systems)
due to internal string comparison.
\item
The argument \textit{main} should be fully expanded, it cannot be a macro.
\item
Subdirectories and special characters should be avoided in filenames.
\item
The command |\childdocmain{|\textit{main}|}| must be followed by a whitespace.
It should not be followed immediately by another command
or by a comment mark `|%|'.
This is because the \TeX{} parser reads the token immediately following
the argument of |\childdocmain| and puts it
at the beginning of every child section;
however, a white\-space is ignored.
\end{itemize}

%%%%%%%%%%%%%%%%%%%%%%%%%%%%%%%%%%%%%%%%
\paragraph{Content of Main File.}

It is advisable to place all content in the child files included by |\include|.
Any output contained in the main file will appear in all child documents
unless suppressed manually;
it cannot be suppressed automatically by the |\includeonly| directive
and thus should normally be avoided.
A method to include some content in the main file
by means of conditional processing is described in \secref{sec:conditional}.

%%%%%%%%%%%%%%%%%%%%%%%%%%%%%%%%%%%%%%%%
\paragraph{Page Numbering.}

When only a part of the document is compiled,
the appropriate numbering of pages
(as well as other status parameters)
is determined from the |.aux| files.
The latter contain information from previous passes.
However this information needs to propagate through
all intermediate child documents.
Therefore the page numbering in child documents may well
be inconsistent until the complete document is compiled at least once.

A useful (if unconventional) way to always ensure a consistent
page numbering is to restart the numbering in each child document
and denote the pages by `\textit{child}|.|\textit{page}'
where \textit{child} represents the chapter/section number of the child file.
This can be achieved by the command
|\numberwithin{page}{|\textit{child}|}|
of the \textsf{amsmath} package
where \textit{child} can be |chapter| or |section|
depending on the chosen structuring.
Alternatively, one can modify the macro |\thepage| appropriately
and reset the counter |page| at the start of each child file.

%%%%%%%%%%%%%%%%%%%%%%%%%%%%%%%%%%%%%%%%%%%%%%%%%%%%%%%%%%%%%%%%%%%%%%%%%%%%%%%%
\subsection{Conditional Processing}
\label{sec:conditional}

The package provides a mechanism to compile different versions
of a document. To customise the versions further some conditional processing
can come in handy to distinguish which version is being compiled.
The package provides two macros to describe the compilation context:

%%%%%%%%%%%%%%%%%%%%%%%%%%%%%%%%%%%%%%%%
\DescribeMacro{\ifchilddoc}
The conditional |\ifchilddoc| distinguishes between the compilation of
child documents and the main document:
%
\begin{center}
|\ifchilddoc |\textit{child-code}| |[|\||else |\textit{main-code}]| \||fi|
\end{center}

%%%%%%%%%%%%%%%%%%%%%%%%%%%%%%%%%%%%%%%%
\DescribeMacro{\childdocname}
\DescribeMacro{\childdocjob}
The macro |\childdocname| contains the filename (without extension)
of the main or child file being processed.
Note that |\childdocjob| will always contain the name of the main file.

%%%%%%%%%%%%%%%%%%%%%%%%%%%%%%%%%%%%%%%%
\paragraph{Title Page.}

Conditional processing can be used to include a title or banner page
in the main document when proper precautions are taken.
Importantly, the code in the main file should ensure that the page counter
(as well as other status parameters which are stored in the |.aux| files)
takes the same value after the conditional processing.
Otherwise the page numbers may take divergent values
depending on which part is compiled.

For example, a title page could be declared by:
%
\begin{center}
\begin{tabular}{l}
|\ifchilddoc\||else|\\
|\addtocounter{page}{-1}|\\
\textit{code for title page}\\
|\newpage|\\
|\||fi|
\end{tabular}
\end{center}
%
A banner page for the child documents can be generated by:
%
\begin{center}
\begin{tabular}{l}
|\ifchilddoc|\\
|\addtocounter{page}{-1}|\\
\textit{code for banner page}\\
|\newpage|\\
|\||fi|
\end{tabular}
\end{center}
%
Here one could write a message such as:
\begin{center}
|This is the part \childdocname{} of \childdocjob{}.|
\end{center}

%%%%%%%%%%%%%%%%%%%%%%%%%%%%%%%%%%%%%%%%%%%%%%%%%%%%%%%%%%%%%%%%%%%%%%%%%%%%%%%%
\subsection{Flags}
\label{sec:flags}

The package makes it easy to generate different versions
of the main or child documents.
To this end compilation flags can be defined
and assigned different default values.
They will be particularly useful in conjunction
with the forwarding mechanism described in \secref{sec:forward}.

For example, it may be useful to have a flag |\version|
which can be set to |draft| or |final|.
The document source will contain some conditional code
depending on the value of |\version|.
Suppose further, the flag should default to |final| for the main file
and to |draft| for child files
which is a natural assignment for editing the document.
This is achieved by placing the following code
in the preamble of the main document
(below the |\childdocmain| directive):
%
\begin{center}
\begin{tabular}{l}
|\ifchilddoc|\\
|\providecommand{\version}{draft}|\\
|\||else|\\
|\providecommand{\version}{final}|\\
|\||fi|
\end{tabular}
\end{center}
%
The definition by |\providecommand| makes sure
that previous definitions are not overwritten.
Further statements |\providecommand{\version}{...}|
can thus be added before the above code to override it.

For the main file, one might add a line
(between |\childdocmain| and the above block)
%
\begin{center}
|%\ifchilddoc\||else\providecommand{\version}{draft}\||fi|
\end{center}
%
which can be uncommented to produce a draft version.
Likewise one can add a line to the very top of a child file
(above the |\childdocof{|\textit{main}|}| directive)
%
\begin{center}
|%\providecommand{\version}{final}|
\end{center}
%
which can be uncommented to produce the final version of this child document.

%%%%%%%%%%%%%%%%%%%%%%%%%%%%%%%%%%%%%%%%%%%%%%%%%%%%%%%%%%%%%%%%%%%%%%%%%%%%%%%%
\subsection{Forwarding}
\label{sec:forward}

Different versions of the main or child documents
using compilation flags as described in \secref{sec:flags}
can be (permanently) stored in different files
for convenient compilation, viewing and distribution.
To this end, the package defines a command
to pass on compilation to a different file:

%%%%%%%%%%%%%%%%%%%%%%%%%%%%%%%%%%%%%%%%
\DescribeMacro{\childdocforward}
The command |\childdocforward| redirects processing to
another source file:
%
\begin{center}
\begin{tabular}{l}
|\input{childdoc.def}|\\
|\childdocforward[|\textit{main}|]{|\textit{dest}|}|\\
\end{tabular}
\end{center}
%
The argument \textit{dest} is the destination file
(without extension).
It should be the main file or one of the child files.
Note that further \textsf{childdoc} directives
such as |\childdocof| and |\childdocforward|
in the indicated file will be processed in this form.
The optional argument \textit{main}
passes on directly to the main file \textit{main}
while pretending to compile the child \textit{dest}.
This form behaves as if \textit{dest}
issues |\childdocof{|\textit{main}|}| right away,
and no further \textsf{childdoc} directives will be processed.

%%%%%%%%%%%%%%%%%%%%%%%%%%%%%%%%%%%%%%%%
\DescribeMacro{\...prefix}
In the alternative form |\childdocforwardprefix|,
%
\begin{center}
\begin{tabular}{l}
|\input{childdoc.def}|\\
|\childdocforwardprefix[|\textit{main}|]{|\textit{prefix}|}{|\textit{dest}|}|
\end{tabular}
\end{center}
%
the destination file is determined by a pattern
depending on the current file:
To make this work, the current file must be called
`{\textit{prefix}\hspace{0.2em}\textit{suffix}}'
with \textit{prefix} matching precisely the argument.
Processing is then passed on to the file
`{\textit{dest}\hspace{0.2em}\textit{suffix}}'.
Surely, the same effect is achieved by
directly specifying the
argument `{\textit{dest}\hspace{0.2em}\textit{suffix}}'
in the first form.
However, that requires to set up a different file
for each child. With the alternative form of the command
all these files can have exactly the same content
which simplifies setting them up and maintaining them.

For example, the following file |draft.tex|
with a compilation flag |\version| as described in \secref{sec:flags}
compiles the main document as a draft:
%
\begin{center}
\begin{tabular}{l}
|\def\version{draft}|\\
|\input{childdoc.def}|\\
|\childdocforward{|\textit{main}|}|
\end{tabular}
\end{center}
%
Likewise, the following files |final|\textit{nn}|.tex|
compile the final version of the child document
|child|\textit{nn}|.tex|:
%
\begin{center}
\begin{tabular}{l}
|\def\version{final}|\\
|\input{childdoc.def}|\\
|\childdocforwardprefix{final}{child}|
\end{tabular}
\end{center}
%

Note that when several versions of a main file and/or of each child file
are to be generated, it may be convenient to set up a |Makefile| or
shell script to automatise the process.

%%%%%%%%%%%%%%%%%%%%%%%%%%%%%%%%%%%%%%%%%%%%%%%%%%%%%%%%%%%%%%%%%%%%%%%%%%%%%%%%
\subsection{Command Line Processing}
\label{sec:commandline}

The effect of redirection files can also be achieved by invoking
the \LaTeX{} compiler with a more elaborate command line.
Most conveniently this should be done as part
of a shell script or a |Makefile|.

When using \textsf{childdoc} in the main file, the following
command lines effectively perform a redirection
(note that depending on the shell being used,
backslashes may have to be doubled: `|\|' $\to$ `|\\|'):
%
\begin{center}
|... -jobname "|\textit{target}|" |\\|"|[\textit{flags}]%
|\input{childdoc.def}\childdocforward[|\textit{main}|]{|\textit{dest}|}"|
\end{center}
%
Here \textit{target} is the name of the output file,
\textit{main} is the name of the main file
and \textit{dest} is the name of the main or child file to be processed
(all filenames without extensions).
The optional argument \textit{main} can be omitted
if \textit{main} matches \textit{dest}.
Optionally, compilation \textit{flags} can be defined via |\def| commands.
This command line makes the \TeX{} engine believe
it is compiling the file \textit{target}
whose content is specified as the latter parameter.
The provided code then forwards the processing to
\textit{main} or \textit{dest} as described in \secref{sec:forward}.

%%%%%%%%%%%%%%%%%%%%%%%%%%%%%%%%%%%%%%%%%%%%%%%%%%%%%%%%%%%%%%%%%%%%%%%%%%%%%%%%
\subsection{Include by Input}
\label{sec:input}

Including child documents by |\include| has some restrictions by design.
Most notably, the content of a child document always occupies
its own set of pages; pages cannot be shared between child documents.
Usually, this behaviour makes perfect sense
because each child document contain an essential part of the document.
However, in some situations it may be desirable to compose
a document from a collection of parts
without having mandatory page breaks between then.
For this case, the package
provides a mechanism to include parts
by |\input| which can also be processed individually.
However, by construction this mechanism
requires manual handling of the content to be output.

%%%%%%%%%%%%%%%%%%%%%%%%%%%%%%%%%%%%%%%%
\DescribeMacro{\ifchilddocmanual}
The main file should be prepared as usual, see \secref{sec:include}.
However, the document body must make a distinction
between processing of an individual part and of the main document, e.g.:
%
\begin{center}
\begin{tabular}{l}
|\ifchilddocmanual|\\
|\input{\childdocname}|\\
|\||else|\\
\textit{document body with }|\input{|\textit{part}|}|\\
|\||fi|
\end{tabular}
\end{center}
%
The conditional |\ifchilddocmanual| is true whenever
a part to be included by |\input| is being compiled,
and the name of the part is stored in |\childdocname|.

%%%%%%%%%%%%%%%%%%%%%%%%%%%%%%%%%%%%%%%%
\DescribeMacro{\childdocby}
Each part to be included by |\input| should start with:
%
\begin{center}
\begin{tabular}{l}
|\input{childdoc.def}|\\
|\childdocby{|\textit{main}|}|\\
\end{tabular}
\end{center}
%
The directive |\childdocby| is similar to |\childdocof|
described in \secref{sec:include},
but the subsequent selection of content must be done manually.
To that end, both |\ifchilddoc| and |\ifchilddocmanual|
will be true upon processing of a part,
and the name of the part is stored in |\childdocname|.
Note that |\jobname| will be set to the filename of the current part
so that each part receives an individual |.aux| file
that does not interfere with the |.aux| file(s) of the main document.
This behaviour can be altered by the alternative form
|\childdocby[*]{|\textit{main}|}| (with a non-empty optional argument)
which uses the |.aux| file of the main document
by setting |\jobname| to \textit{main}.

%%%%%%%%%%%%%%%%%%%%%%%%%%%%%%%%%%%%%%%%%%%%%%%%%%%%%%%%%%%%%%%%%%%%%%%%%%%%%%%%
\subsection{Driver Development}
\label{sec:driver}

The \textsf{childdoc} mechanism can also be use for the development
of definition files such as \LaTeX{} styles or classes.
This case differs from the above setup with multiple parts
included by |\include| in that no |\includeonly| should be invoked.
This can be achieved by starting the include file
(before |\ProvidesPackage|) with:
%
\begin{center}
\begin{tabular}{l}
|\input{childdoc.def}|\\
|\childdocforward{|\textit{main}|}|\\
\end{tabular}
\end{center}
%
or alternatively with:
%
\begin{center}
\begin{tabular}{l}
|\input{childdoc.def}|\\
|\childdocby{|\textit{main}|}|\\
\end{tabular}
\end{center}
%
Both forms have slightly different effects as described above.
The main file is prepared as usual, see \secref{sec:include}.

%%%%%%%%%%%%%%%%%%%%%%%%%%%%%%%%%%%%%%%%%%%%%%%%%%%%%%%%%%%%%%%%%%%%%%%%%%%%%%%%
\subsection{Legacy Detection}
\label{sec:detection}

The directive |\childdocmain| in the main file can detect
whether the complete document or merely a child is to be compiled
even without using the directive |\childdocof|.
This method is deprecated because it is less robust
and there is no compelling reason to use it;
it is merely provided for backward compatibility
and it may be removed in future versions.

If the detection mechanism is to be used,
it is mandatory to correctly specify
the filename of the main file as the argument of |\childdocmain|:
%
\begin{center}
\begin{tabular}{l}
|\input{childdoc.def}|\\
|\childdocmain{|\textit{main}|}|\\
\end{tabular}
\end{center}
%
If |\jobname| does not match the argument \textit{main} of |\childdocmain|,
it is assumed that |\jobname| points to the child file to be compiled.
When using |\childdocmain| with the main file specified as argument,
it suffices to start a child file
with just |\input{|\textit{main}|}|
without loading of the package and using |\childdocof|.
If instead all processing is done
with the appropriate \textsf{childdoc} directives,
the argument of \textit{main} of |\childdocmain| can be empty.

An alternative version of the command line processing described
in \secref{sec:commandline} using the detection mechanism reads:
%
\begin{center}
|... -jobname "|\textit{target}|" "|[\textit{flags}]%
[|\def\jobname{|\textit{dest}|}|]|\input{|\textit{main}|}"|
\end{center}

%%%%%%%%%%%%%%%%%%%%%%%%%%%%%%%%%%%%%%%%%%%%%%%%%%%%%%%%%%%%%%%%%%%%%%%%%%%%%%%%
\subsection{Manual Code}
\label{sec:manual}

In case one cannot be certain whether the definitions file |childdoc.def|
is installed on the target \TeX{} distribution
and one prefers not to ship it,
it is conceivable to paste a few relevant commands into the sources.

To that end, drop all statements |\input{childdoc.def}|
and perform the replacements as outlined below.
Instead of |\childdocmain{|\textit{main}|}| add the following code
to the top of the main file:
%
\begin{center}
\begin{tabular}{l}
|\||ifdefined\childdocname\endinput\||fi\newif\ifchilddoc|\\
|\edef\childdocname{\scantokens\expandafter{\jobname\noexpand}}|\\
|\def\childdocmain{|\textit{main}|}\||ifx\childdocmain\childdocname\||else|\\
|\childdoctrue\includeonly{\childdocname}\let\jobname\childdocmain\||fi|\\
\end{tabular}
\end{center}
%
Instead of |\childdocof{|\textit{main}|}| just include the main file
at the top of each child file:
%
\begin{center}
|\input{|\textit{main}|}|
\end{center}
%
A simple redirection |\childdocforward{|\textit{dest}|}| is achieved by:
%
\begin{center}
|\def\jobname{|\textit{dest}|}\input{\jobname}|
\end{center}
%
The redirection with prefix
|\childdocforwardprefix[|\textit{prefix}|]{|\textit{dest}|}|
is accomplished by:
%
\begin{center}
\begin{tabular}{l}
|{\edef\jobname{\scantokens\expandafter{\jobname\noexpand}}|\\
|\def\redirectjob |\textit{prefix}|#1~~~{\gdef\jobname{|\textit{dest}|#1}}|\\
|\expandafter\redirectjob\jobname~~~}\input{\jobname}|
\end{tabular}
\end{center}

In an alternative approach,
child documents can be compiled by a specific command line
without additional code or specific definitions:
%
\begin{center}
|... -jobname "|\textit{target}|" "|[\textit{flags}]%
|\includeonly{|\textit{dest}|}\input{|\textit{main}|}"|
\end{center}
%

%%%%%%%%%%%%%%%%%%%%%%%%%%%%%%%%%%%%%%%%%%%%%%%%%%%%%%%%%%%%%%%%%%%%%%%%%%%%%%%%
%%%%%%%%%%%%%%%%%%%%%%%%%%%%%%%%%%%%%%%%%%%%%%%%%%%%%%%%%%%%%%%%%%%%%%%%%%%%%%%%
\section{Information}

%%%%%%%%%%%%%%%%%%%%%%%%%%%%%%%%%%%%%%%%%%%%%%%%%%%%%%%%%%%%%%%%%%%%%%%%%%%%%%%%
\subsection{Copyright}

Copyright \copyright{} 2017--2018 Niklas Beisert

This work may be distributed and/or modified under the
conditions of the \LaTeX{} Project Public License, either version 1.3
of this license or (at your option) any later version.
The latest version of this license is in
  \url{http://www.latex-project.org/lppl.txt}
and version 1.3 or later is part of all distributions of \LaTeX{}
version 2005/12/01 or later.

This work has the LPPL maintenance status `maintained'.

The Current Maintainer of this work is Niklas Beisert.

This work consists of the files |README.txt|, |childdoc.ins| and |childdoc.dtx|
as well as the derived files |childdoc.def|, |cdocsamp.tex|
with |cdocsch1.tex|, |cdocsch2.tex|, |cdocspt3.tex|, |cdocspt4.tex|,
|cdocsdrf.tex|, |cdocsfn1.tex|, |cdocsfn2.tex|
as well as |childdoc.pdf|.

%%%%%%%%%%%%%%%%%%%%%%%%%%%%%%%%%%%%%%%%%%%%%%%%%%%%%%%%%%%%%%%%%%%%%%%%%%%%%%%%
\subsection{Files and Installation}

The package consists of the files:
%
\begin{center}
\begin{tabular}{ll}
    |README.txt|   & readme file \\
    |childdoc.ins| & installation file \\
    |childdoc.dtx| & source file \\
    |childdoc.def| & definition file \\
    |cdocsamp.tex| & sample main file \\
    |cdocsch1.tex| & sample include file \\
    |cdocsch2.tex| & sample include file \\
    |cdocspt3.tex| & sample part file \\
    |cdocspt4.tex| & sample part file \\
    |cdocsdrf.tex| & sample redirection file \\
    |cdocsfn1.tex| & sample redirection file \\
    |cdocsfn2.tex| & sample redirection file \\
    |childdoc.pdf| & manual
\end{tabular}
\end{center}
%
The distribution consists of the files
|README.txt|, |childdoc.ins| and |childdoc.dtx|.
%
\begin{itemize}
\item
Run (pdf)\LaTeX{} on |childdoc.dtx|
to compile the manual |childdoc.pdf| (this file).
\item
Run \LaTeX{} on |childdoc.ins| to create the definitions file |childdoc.def|
and the sample |cdocsamp.tex| with include files
|cdocsch1.tex|, |cdocsch2.tex|, |cdocspt3.tex|, |cdocspt4.tex|,
|cdocsdrf.tex|, |cdocsfn1.tex|, |cdocsfn2.tex|.
Then copy the file |childdoc.def| to an appropriate directory of your \LaTeX{}
distribution, e.g.\ \textit{texmf-root}|/tex/latex/childdoc|.
\end{itemize}

%%%%%%%%%%%%%%%%%%%%%%%%%%%%%%%%%%%%%%%%%%%%%%%%%%%%%%%%%%%%%%%%%%%%%%%%%%%%%%%%
\subsection{Related CTAN Packages}

There are several other packages which offer a similar functionality:
%
\begin{itemize}
\item
The packages
\href{http://ctan.org/pkg/docmute}{\textsf{docmute}},
\href{http://ctan.org/pkg/includex}{\textsf{includex}} and
\href{http://ctan.org/pkg/standalone}{\textsf{standalone}}
provide commands to include only the document body of
a child file thus allowing both files to be compiled individually.
\item
The packages \href{http://ctan.org/pkg/subdocs}{\textsf{subdocs}}
and \href{http://ctan.org/pkg/subfiles}{\textsf{subfiles}}
provide structures in which the main and child documents can be
encapsulated and allowing them to be compiled individually.
The inclusion mechanism is different from the conventional |\include|.
\item
The package \href{http://ctan.org/pkg/combine}{\textsf{combine}}
is an elaborate solution to combine several documents into one.
\end{itemize}
%
See also the CTAN topic \href{http://ctan.org/topic/subdocs}{\textsf{subdocs}}
for further related packages.
The present package differs from the above solutions in that
a document structure constructed with the conventional |\include| mechanism
just needs two extra commands at the top of every file
such that all constituent files can be compiled individually.

%%%%%%%%%%%%%%%%%%%%%%%%%%%%%%%%%%%%%%%%%%%%%%%%%%%%%%%%%%%%%%%%%%%%%%%%%%%%%%%%
%\subsection{Feature Suggestions}
%
%The following is a list of features which may be useful for future
%versions of this package:
%%
%\begin{itemize}
%\item
%\ldots
%\end{itemize}

%%%%%%%%%%%%%%%%%%%%%%%%%%%%%%%%%%%%%%%%%%%%%%%%%%%%%%%%%%%%%%%%%%%%%%%%%%%%%%%%
\subsection{Revision History}

%%%%%%%%%%%%%%%%%%%%%%%%%%%%%%%%%%%%%%%%
\paragraph{v2.0:} 2018/12/30

\begin{itemize}
\item
immediate forward processing
\item
added |\childdocby| mechanism
\item
manual restructured
\end{itemize}

%%%%%%%%%%%%%%%%%%%%%%%%%%%%%%%%%%%%%%%%
\paragraph{v1.6:} 2018/01/17

\begin{itemize}
\item
application for development of include files
\item
corrections to manual
\end{itemize}

%%%%%%%%%%%%%%%%%%%%%%%%%%%%%%%%%%%%%%%%
\paragraph{v1.5:} 2017/05/21

\begin{itemize}
\item
more complete structuring introduced
\item
|\childdocof| introduced
\item
|\childdoc| renamed to |\childdocmain|
\item
|\childredirect| renamed to |\childdocforward| and |\childdocforwardprefix|
and functionality expanded
\end{itemize}

%%%%%%%%%%%%%%%%%%%%%%%%%%%%%%%%%%%%%%%%
\paragraph{v1.0:} 2017/04/27

\begin{itemize}
\item
manual and install package
\item
first version published on CTAN
\end{itemize}

%%%%%%%%%%%%%%%%%%%%%%%%%%%%%%%%%%%%%%%%
\paragraph{v0.6:} 2017/04/26

\begin{itemize}
\item
redirection mechanism added
\end{itemize}

%%%%%%%%%%%%%%%%%%%%%%%%%%%%%%%%%%%%%%%%
\paragraph{v0.5:} 2017/04/26

\begin{itemize}
\item
functionality in definition file
\end{itemize}


%%%%%%%%%%%%%%%%%%%%%%%%%%%%%%%%%%%%%%%%%%%%%%%%%%%%%%%%%%%%%%%%%%%%%%%%%%%%%%%%
%%%%%%%%%%%%%%%%%%%%%%%%%%%%%%%%%%%%%%%%%%%%%%%%%%%%%%%%%%%%%%%%%%%%%%%%%%%%%%%%
%%%%%%%%%%%%%%%%%%%%%%%%%%%%%%%%%%%%%%%%%%%%%%%%%%%%%%%%%%%%%%%%%%%%%%%%%%%%%%%%
\appendix

\settowidth\MacroIndent{\rmfamily\scriptsize 000\ }

 \DocInput{childdoc.dtx}

\end{document}
%</driver>
% \fi
%
% %%%%%%%%%%%%%%%%%%%%%%%%%%%%%%%%%%%%%%%%%%%%%%%%%%%%%%%%%%%%%%%%%%%%%%%%%%%%%%
% %%%%%%%%%%%%%%%%%%%%%%%%%%%%%%%%%%%%%%%%%%%%%%%%%%%%%%%%%%%%%%%%%%%%%%%%%%%%%%
% \section{Sample}
%\iffalse
%<*samplemain>
%\fi
%
% The following presents a sample document
% with two chapters, two parts, a title page,
% a compile flag as well as three forwarding files to set the flag.
% It consists of eight |.tex| files:
% \begin{center}
% \begin{tabular}{ll}
% |cdocsamp.tex|&main file\\
% |cdocsch1.tex|&include file for chapter 1\\
% |cdocsch2.tex|&include file for chapter 2\\
% |cdocspt3.tex|&include file for part 3\\
% |cdocspt4.tex|&include file for part 4\\
% |cdocsdrf.tex|&forwarding file for main file in draft mode\\
% |cdocsfi1.tex|&forwarding file for final version of chapter 1\\
% |cdocsfi2.tex|&forwarding file for final version of chapter 2\\
% \end{tabular}
% \end{center}
% Each of the eight files can be compiled directly by the \LaTeX{} compiler.
%
% %%%%%%%%%%%%%%%%%%%%%%%%%%%%%%%%%%%%%%
% \paragraph{Main File.}
%
% The main file is called |cdocsamp.tex|.
%
% Load the \textsf{childdoc} definitions and
% declare the filename for the main document:
%    \begin{macrocode}
\input{childdoc.def}
\childdocmain{}
%    \end{macrocode}

% Optional override for |\version| flag:
%    \begin{macrocode}
%%\ifchilddoc\else\providecommand{\version}{draft}\fi
%    \end{macrocode}

% Define the default values for the |\version| flag
% (|final| for the main file and |draft| for childs):
%    \begin{macrocode}
\ifchilddoc
\providecommand{\version}{draft}
\else
\providecommand{\version}{final}
\fi
%    \end{macrocode}

% Load the standard document class:
%    \begin{macrocode}
\documentclass[12pt]{article}
%    \end{macrocode}

% Start the document body:
%    \begin{macrocode}
\begin{document}
%    \end{macrocode}

% Declare a title page.
% Print title, part of document being processed and version flag:
%    \begin{macrocode}
\addtocounter{page}{-1}
\begin{center}
{\LARGE\bfseries{}childdoc example\par}
\vspace{1cm}
\ifchilddoc
\ifchilddocmanual part\else chapter\fi:
`\childdocname' of `\childdocjob'\par
\else
main document: `\childdocjob'\par
\fi
version: \version\par
\end{center}
\newpage
%    \end{macrocode}

% Manually include selected file,
% otherwise process as usual:
%    \begin{macrocode}
\ifchilddocmanual
\section*{part `\childdocname'}
\input{\childdocname}
\else
%    \end{macrocode}

% Include the two chapters:
%    \begin{macrocode}
\include{cdocsch1}
\include{cdocsch2}
%    \end{macrocode}

% Include the two parts unless only chapters should be displayed:
%    \begin{macrocode}
\ifchilddoc\else
\section{part three}
\input{cdocspt3}
\section{part four}
\input{cdocspt4}
\fi
%    \end{macrocode}

% Process as usual until here:
%    \begin{macrocode}
\fi
%    \end{macrocode}

% End of document body:
%    \begin{macrocode}
\end{document}
%    \end{macrocode}
%\iffalse
%</samplemain>
%\fi
%
% %%%%%%%%%%%%%%%%%%%%%%%%%%%%%%%%%%%%%%
% \paragraph{Chapter Include Files.}
%
% The include files are called |cdocsch1.tex| and |cdocsch2.tex|.
%
%\iffalse
%<*samplechap1|samplechap2>
%\fi

% Optional override for |\version| flag:
%    \begin{macrocode}
%%\providecommand{\version}{final}
%    \end{macrocode}

% Include the main document:
%    \begin{macrocode}
\input{childdoc.def}
\childdocof{cdocsamp}
%    \end{macrocode}

%\iffalse
%</samplechap1|samplechap2>
%\fi
%
%\iffalse
%<*samplechap1>
%\fi
% Some text for chapter 1:
%    \begin{macrocode}
\section{one}
some text in chapter one
%    \end{macrocode}

%\iffalse
%</samplechap1>
%\fi
% Some text for chapter 2:
%\iffalse
%<*samplechap2>
%\fi
%    \begin{macrocode}
\section{two}
more text in chapter two
%    \end{macrocode}

%\iffalse
%</samplechap2>
%\fi
%
% %%%%%%%%%%%%%%%%%%%%%%%%%%%%%%%%%%%%%%
% \paragraph{Part Include Files.}
%
% The include files are called |cdocspt3.tex| and |cdocspt4.tex|.
%
%\iffalse
%<*samplepart3|samplepart4>
%\fi

% Optional override for |\version| flag:
%    \begin{macrocode}
%%\providecommand{\version}{final}
%    \end{macrocode}

% Include the main document:
%    \begin{macrocode}
\input{childdoc.def}
\childdocby{cdocsamp}
%    \end{macrocode}

%\iffalse
%</samplepart3|samplepart4>
%\fi
%
%\iffalse
%<*samplepart3>
%\fi
% Some text for part 3:
%    \begin{macrocode}
some text in part three
%    \end{macrocode}

%\iffalse
%</samplepart3>
%\fi
% Some text for part 4:
%\iffalse
%<*samplepart4>
%\fi
%    \begin{macrocode}
more text in part four
%    \end{macrocode}

%\iffalse
%</samplepart4>
%\fi
%
% %%%%%%%%%%%%%%%%%%%%%%%%%%%%%%%%%%%%%%
% \paragraph{Forwarding for a Complete Draft.}
%
% The following forwarding file |cdocsdrf.tex|
% compiles the main document in draft mode:
%\iffalse
%<*sampledraft>
%\fi
%    \begin{macrocode}
\def\version{draft}
\input{childdoc.def}
\childdocforward{cdocsamp}
%    \end{macrocode}

%\iffalse
%</sampledraft>
%\fi
%
% %%%%%%%%%%%%%%%%%%%%%%%%%%%%%%%%%%%%%%
% \paragraph{Forwarding for Final Version of the Chapters.}
%
% The following forwarding files |cdocsfn1.tex| and |cdocsfn2.tex|
% (with identical content)
% compile the final versions of the child documents
% |cdocsch1.tex| and |cdocsch2.tex|, respectively:
%\iffalse
%<*samplefinal>
%\fi
%    \begin{macrocode}
\def\version{final}
\input{childdoc.def}
\childdocforwardprefix[cdocsamp]{cdocsfn}{cdocsch}
%    \end{macrocode}

%\iffalse
%</samplefinal>
%\fi
%
% %%%%%%%%%%%%%%%%%%%%%%%%%%%%%%%%%%%%%%
% \paragraph{Command Line Processing.}
%
% The following three command lines generate the output files
% |cdocscld|, |cdocscl1| and |cdocscl2|
% which should be identical to
% |cdocsdrf|, |cdocsch1| and |cdocsfn2|, respectively:
% \begin{center}
% \begin{tabular}{l}
% |latex -jobname cdocscld \|\\
% |  "\def\version{draft}\input{childdoc.def}\childdocforward{cdocsamp}"|\\
% |latex -jobname cdocscl1 \|\\
% |  "\input{childdoc.def}\childdocforward[cdocsamp]{cdocsch1}"|\\
% |latex -jobname cdocscl2 \|\\
% |  "\def\version{final}\input{childdoc.def}\childdocforward{cdocsch2}"|
% \end{tabular}
% \end{center}
% Note that the trailing backslash on each first line
% merely continues the input to the second line
% (for convenient cut ant paste).
% Furthermore, the command |latex| can be replaced by any
% of its alternative versions such as |pdflatex|.
%
% %%%%%%%%%%%%%%%%%%%%%%%%%%%%%%%%%%%%%%%%%%%%%%%%%%%%%%%%%%%%%%%%%%%%%%%%%%%%%%
% %%%%%%%%%%%%%%%%%%%%%%%%%%%%%%%%%%%%%%%%%%%%%%%%%%%%%%%%%%%%%%%%%%%%%%%%%%%%%%
% \section{Implementation}
%\iffalse
%<*package>
%\fi
%
% This section describes the definitions file |childdoc.def|.

% The definitions cannot be loaded using |\usepackage| or |\RequirePackage|
% which has a mechanism to prevent loading a style file more than once.
% When loading the definitions by means of |\input|
% multiple instances have to be prevented manually:
%\iffalse
%This code needs to be before the `\ProvidesFile' directive
%which is defined at the beginning of this file.
%Therefore it is also placed there and commented out here.
%</package>
%<*discard>
%\fi
%    \begin{macrocode}
\ifdefined\childdocmain\endinput\fi
%    \end{macrocode}
%\iffalse
%</discard>
%<*package>
%\fi
%
% \macro{\ifchilddoc}
% \macro{\ifchilddocmanual}
% The conditional |\ifchilddoc| tells whether a
% child (true) or main (false) document is being compiled.
% The conditional |\ifchilddocmanual| tells whether
% the |\includeonly| mechanism is used (false) or
% the selection of child files must be performed manually (true).
% The definitions initialise to false:
%    \begin{macrocode}
\newif\ifchilddoc
\newif\ifchilddocmanual
%    \end{macrocode}

% \macro{\childdocname}
% \macro{\childdocjob}
% The macro |\childdocname| stores the name of the main document
% to be compiled. The macro |\childdocjob| stores the name of
% the document on which the \LaTeX{} compiler was originally invoked.
% The content of |\jobname| cannot be compared
% to filenames specified in the source due to different catcodes.
% The following code rescans |\jobname|, stores the result
% in |\childdocname| and saves a copy in |\childdocjob|:
%    \begin{macrocode}
\edef\childdocname{\scantokens\expandafter{\jobname\noexpand}}
\let\childdocjob\childdocname
%    \end{macrocode}

% \macro{\childdocdisable}
% The macro |\childdocdisable| prevents the main file
% from being processed more than once.
% At this stage, the main document command |\childdocmain|
% is assumed to be called once again where it should do nothing.
% Any subsequent call to it should prevent
% a secondary processing of the main document
% It overwrites the forwarding commands
% |\childdocof| and |\childdocforward|
% with empty macros to prevent further inclusions of the main document:
%    \begin{macrocode}
\newcommand{\childdocdisable}
{
  \renewcommand{\childdocmain}[1]{\renewcommand{\childdocmain}[1]{\endinput}}
  \renewcommand{\childdocof}[1]{}
  \renewcommand{\childdocby}[2][]{}
  \renewcommand{\childdocforward}[2][]{}
  \renewcommand{\childdocdisable}{}
}
%    \end{macrocode}

% \macro{\childdocmain}
% The macro |\childdocmain| is to be called at the top of the main file
% with nothing or the main filename (without extension) as argument.
% First, it breaks loops.
% If the argument is not empty and does not match |\childdocname|
% (which is set by the first inclusion of |childdoc.def|),
% |\ifchilddoc| is set to true, |\includeonly| is applied to the child file
% and |\jobname| is set to the main file
% (for proper handling of |.aux| files):
%    \begin{macrocode}
\newcommand{\childdocmain}[1]
{
  \childdocdisable\childdocmain{}
  \if?#1?\else
    \begingroup
      \def\childdoctmp{#1}
      \ifx\childdoctmp\childdocname
        \def\childdoctmp{}
      \else
        \def\childdoctmp
        {
          \childdoctrue
          \includeonly{\childdocname}
          \def\childdocjob{#1}
          \def\jobname{#1}
        }
      \fi
      \expandafter
    \endgroup
    \childdoctmp
  \fi
}
%    \end{macrocode}

% \macro{\childdocof}
% The command |\childdocof| redirects
% compilation to the main file |#1|.
%    \begin{macrocode}
\newcommand{\childdocof}[1]
{
  \childdocdisable
  \childdoctrue
  \includeonly{\childdocname}
  \def\jobname{#1}
  \def\childdocjob{#1}
  \input{#1}
}
%    \end{macrocode}

% \macro{\childdocby}
% The command |\childdocby| ....
%    \begin{macrocode}
\newcommand{\childdocby}[2][]
{
  \childdocdisable
  \childdoctrue
  \childdocmanualtrue
  \if?#1?\else
    \def\jobname{#2}
  \fi
  \def\childdocjob{#2}
  \input{#2}
  \endinput
}
%    \end{macrocode}

% \macro{\childdocforward}
% The command |\childdocforward| redirects
% compilation to the main file or
% (if the optional argument is given) a child file.
% Parameters are set as if the main file
% or a child file starting with |\childdocof| was compiled.
% Then compilation is handed over to the main file:
%    \begin{macrocode}
\newcommand{\childdocforward}[2][]
{
  \begingroup
    \if?#1?
      \def\childdoctmp
      {
        \def\childdocname{#2}
        \def\childdocjob{#2}
        \def\jobname{#2}
        \input{#2}
        \endinput
      }
    \else
      \def\childdoctmp
      {
        \childdocdisable
        \def\childdocname{#2}
        \childdoctrue
        \includeonly{#2}
        \def\childdocjob{#1}
        \def\jobname{#1}
        \input{#1}
        \endinput
      }
    \fi
    \expandafter
  \endgroup
  \childdoctmp
}
%    \end{macrocode}

% \macro{\childdocforwardprefix}
% The command |\childdocforwardprefix| redirects
% compilation to the main or a child file by means of a pattern.
% The prefix |#1| in the current filename is replaced by |#2|
% and the suffix of the current filename is kept
% (it is assumed that the filename does not contain the substring `|~~~|'
% which is used as a delimiter).
% Compilation is handed over to the new file by |\childdocforward|:
%    \begin{macrocode}
\newcommand{\childdocforwardprefix}[3][]
{
  \begingroup
    \def\childdocextract #2##1~~~{\def\childdoctmp{\childdocforward[#1]{#3##1}}}
    \expandafter\childdocextract\childdocname~~~
    \expandafter
  \endgroup
  \childdoctmp
}
%    \end{macrocode}

% \macro{\childdoc}
% The deprecated macro |\childdoc| is a legacy version of |\childdocmain|:
%    \begin{macrocode}
\newcommand{\childdoc}{\childdocmain}
%    \end{macrocode}

% \macro{\childdocredirect}
% The deprecated macro |\childdocredirect| is a legacy version
% of |\childdocforward| and |\childdocforwardprefix|:
%    \begin{macrocode}
\newcommand{\childdocredirect}[2][]
{
  \begingroup
    \if?#1?
      \def\childdoctmp{\childdocforward{#2}}
    \else
      \def\childdoctmp{\childdocforwardprefix{#1}{#2}}
    \fi
    \expandafter
  \endgroup
  \childdoctmp
}
%    \end{macrocode}

%\iffalse
%</package>
%\fi
%
\endinput

\childdocof{cdocsamp}
%    \end{macrocode}

%\iffalse
%</samplechap1|samplechap2>
%\fi
%
%\iffalse
%<*samplechap1>
%\fi
% Some text for chapter 1:
%    \begin{macrocode}
\section{one}
some text in chapter one
%    \end{macrocode}

%\iffalse
%</samplechap1>
%\fi
% Some text for chapter 2:
%\iffalse
%<*samplechap2>
%\fi
%    \begin{macrocode}
\section{two}
more text in chapter two
%    \end{macrocode}

%\iffalse
%</samplechap2>
%\fi
%
% %%%%%%%%%%%%%%%%%%%%%%%%%%%%%%%%%%%%%%
% \paragraph{Part Include Files.}
%
% The include files are called |cdocspt3.tex| and |cdocspt4.tex|.
%
%\iffalse
%<*samplepart3|samplepart4>
%\fi

% Optional override for |\version| flag:
%    \begin{macrocode}
%%\providecommand{\version}{final}
%    \end{macrocode}

% Include the main document:
%    \begin{macrocode}
% \iffalse
%
% childdoc.dtx Copyright (C) 2017-2018 Niklas Beisert
%
% This work may be distributed and/or modified under the
% conditions of the LaTeX Project Public License, either version 1.3
% of this license or (at your option) any later version.
% The latest version of this license is in
%   http://www.latex-project.org/lppl.txt
% and version 1.3 or later is part of all distributions of LaTeX
% version 2005/12/01 or later.
%
% This work has the LPPL maintenance status `maintained'.
%
% The Current Maintainer of this work is Niklas Beisert.
%
% This work consists of the files childdoc.dtx and childdoc.ins
% and the derived files childdoc.def and cdocsamp.tex with
% cdocsch1.tex, cdocsch2.tex, cdocsdrf.tex, cdocsfn1.tex, cdocsfn2.tex.
%
%<package>\ifdefined\childdocmain\endinput\fi
%<package>\ProvidesFile{childdoc.def}[2018/12/30 v2.0 child document driver]
%<samplemain>\ProvidesFile{cdocsamp.tex}[2018/12/30 v2.0 sample for childdoc]
%<*driver>
%\ProvidesFile{childdoc.drv}[2018/12/30 v2.0 childdoc reference manual file]
\PassOptionsToClass{10pt,a4paper}{article}
\documentclass{ltxdoc}

\usepackage[margin=35mm]{geometry}
\usepackage{hyperref}
\usepackage{hyperxmp}
\usepackage[usenames]{color}

\hypersetup{colorlinks=true}
\hypersetup{pdfstartview=FitH}
\hypersetup{pdfpagemode=UseNone}
\hypersetup{pdfsource={}}
\hypersetup{pdflang={en-UK}}
\hypersetup{pdfcopyright={Copyright 2017-2018 Niklas Beisert.
  This work may be distributed and/or modified under the
  conditions of the LaTeX Project Public License, either version 1.3
  of this license or (at your option) any later version.}}
\hypersetup{pdflicenseurl={http://www.latex-project.org/lppl.txt}}
\hypersetup{pdfcontactaddress={ETH Zurich, ITP, HIT K,
  Wolfgang-Pauli-Strasse 27}}
\hypersetup{pdfcontactpostcode={8093}}
\hypersetup{pdfcontactcity={Zurich}}
\hypersetup{pdfcontactcountry={Switzerland}}
\hypersetup{pdfcontactemail={nbeisert@itp.phys.ethz.ch}}
\hypersetup{pdfcontacturl={http://people.phys.ethz.ch/\xmptilde nbeisert/}}

\newcommand{\secref}[1]{\hyperref[#1]{section \ref*{#1}}}

\parskip1ex
\parindent0pt
\let\olditemize\itemize
\def\itemize{\olditemize\parskip0pt}

\begin{document}

\title{The \textsf{childdoc} Package}
\hypersetup{pdftitle={The childdoc Package}}
\author{Niklas Beisert\\[2ex]
  Institut f\"ur Theoretische Physik\\
  Eidgen\"ossische Technische Hochschule Z\"urich\\
  Wolfgang-Pauli-Strasse 27, 8093 Z\"urich, Switzerland\\[1ex]
  \href{mailto:nbeisert@itp.phys.ethz.ch}
  {\texttt{nbeisert@itp.phys.ethz.ch}}}
\hypersetup{pdfauthor={Niklas Beisert}}
\hypersetup{pdfsubject={Manual for the LaTeX2e Package childdoc}}
\date{30 December 2018, \textsf{v2.0}}
\maketitle

\begin{abstract}\noindent
\textsf{childdoc} is a \LaTeXe{} package
that enables the direct compilation
of document sections included by |\include|
to individual files.
\end{abstract}

\begingroup
\parskip0ex
\tableofcontents
\endgroup

%%%%%%%%%%%%%%%%%%%%%%%%%%%%%%%%%%%%%%%%%%%%%%%%%%%%%%%%%%%%%%%%%%%%%%%%%%%%%%%%
%%%%%%%%%%%%%%%%%%%%%%%%%%%%%%%%%%%%%%%%%%%%%%%%%%%%%%%%%%%%%%%%%%%%%%%%%%%%%%%%
\section{Introduction}

\LaTeX{} provides a mechanism to structure a large document (such as a book)
into a main file and several child files (containing the chapters)
using the |\include| command.
This mechanism is beneficial for documents
which span hundreds of pages in order to
make the source file(s) more manageable.
Moreover, compilation can be restricted to
selected child files by means of the |\includeonly| command.
The latter feature can be used to reduce the compilation time while editing
(this was significantly more useful in the earlier days of \LaTeX{})
or to generate a smaller document which is easier to navigate.
Another application of |\includeonly| is to generate
documents consisting of selected parts of the complete document.

However, there are a few drawbacks of the plain |\include| mechanism:
\begin{itemize}
\item
The child files cannot be compiled on their own,
they can only be compiled via the main file.
A naive editing environment
(such as a text editor with an option
to have the current file processed by \LaTeX)
may require one to switch to the main file before compiling;
attempting to compile the child file produces errors.
\item
The main file must be modified (each time)
to adjust the |\includeonly| command
to the present needs. This easily leaves the main file in a messy state.
\item
The generated document will always carry the filename
of the main document. This is inconvenient if
several child files are to be compiled and
to be kept for distribution.
\end{itemize}

The present package provides a simple interface
to make child files individually compilable by \LaTeX{}.
Compiling a child file then has the same effect as compiling
the main file with an |\includeonly| command
to select the appropriate child.
Moreover the generated document will carry the name of the child
rather than the main file.
This resolves all three above issues.

This feature is meant to make the editing of books,
thesis documents and lecture notes somewhat more convenient.
However, the package can also be used efficiently for
composing a series of documents (such as exercise sheets)
which are typically distributed individually.
It then assists the author in generating the individual documents
(potentially in different versions)
as well as a document containing the collected series.
Another application is in developing style files
or other kinds of included material
where compilation of the style file could redirect
to a sample or test file.

%%%%%%%%%%%%%%%%%%%%%%%%%%%%%%%%%%%%%%%%%%%%%%%%%%%%%%%%%%%%%%%%%%%%%%%%%%%%%%%%
%%%%%%%%%%%%%%%%%%%%%%%%%%%%%%%%%%%%%%%%%%%%%%%%%%%%%%%%%%%%%%%%%%%%%%%%%%%%%%%%
\section{Usage}

First of all, the package \textsf{childdoc} is \emph{not} a standard
\LaTeXe{} |.sty| style file! Therefore it needs to be invoked in
a non-standard way.

%%%%%%%%%%%%%%%%%%%%%%%%%%%%%%%%%%%%%%%%%%%%%%%%%%%%%%%%%%%%%%%%%%%%%%%%%%%%%%%%
\subsection{Included Files}
\label{sec:include}

%%%%%%%%%%%%%%%%%%%%%%%%%%%%%%%%%%%%%%%%
\DescribeMacro{\childdocmain}
To use the package, add the commands
\begin{center}
\begin{tabular}{l}
|\input{childdoc.def}|\\
|\childdocmain{}|\\
\end{tabular}
\end{center}
at the very top of the main \LaTeX{} file,
in particular \emph{before} the |\documentclass| statement!
The argument of |\childdocmain| should be left empty
(but it must be present).

%%%%%%%%%%%%%%%%%%%%%%%%%%%%%%%%%%%%%%%%
\DescribeMacro{\childdocof}
Furthermore, add the commands
\begin{center}
\begin{tabular}{l}
|\input{childdoc.def}|\\
|\childdocof{|\textit{main}|}|\\
\end{tabular}
\end{center}
at the top of every child file \textit{child}
which is included by |\include{|\textit{child}|}|
from within the main file
(or at least for those files to be compiled individually).
The argument \textit{main} must be the filename of the main file.

There are a couple of
considerations in setting up the main and child documents:

%%%%%%%%%%%%%%%%%%%%%%%%%%%%%%%%%%%%%%%%
\paragraph{Restrictions.}

Please note the following restrictions:
\begin{itemize}
\item
|\childdocmain| must be called with one argument \textit{main}
to ensure compatibility with earlier version of the package.
It must either be empty (|\childdocmain{}|)
or precisely match the filename of the main file in which it is specified.
See \secref{sec:detection} for further information.
\item
The filename \textit{main} must be specified without the |.tex| extension.
\item
The filename \textit{main} is case sensitive
(even in case-insensitive file systems)
due to internal string comparison.
\item
The argument \textit{main} should be fully expanded, it cannot be a macro.
\item
Subdirectories and special characters should be avoided in filenames.
\item
The command |\childdocmain{|\textit{main}|}| must be followed by a whitespace.
It should not be followed immediately by another command
or by a comment mark `|%|'.
This is because the \TeX{} parser reads the token immediately following
the argument of |\childdocmain| and puts it
at the beginning of every child section;
however, a white\-space is ignored.
\end{itemize}

%%%%%%%%%%%%%%%%%%%%%%%%%%%%%%%%%%%%%%%%
\paragraph{Content of Main File.}

It is advisable to place all content in the child files included by |\include|.
Any output contained in the main file will appear in all child documents
unless suppressed manually;
it cannot be suppressed automatically by the |\includeonly| directive
and thus should normally be avoided.
A method to include some content in the main file
by means of conditional processing is described in \secref{sec:conditional}.

%%%%%%%%%%%%%%%%%%%%%%%%%%%%%%%%%%%%%%%%
\paragraph{Page Numbering.}

When only a part of the document is compiled,
the appropriate numbering of pages
(as well as other status parameters)
is determined from the |.aux| files.
The latter contain information from previous passes.
However this information needs to propagate through
all intermediate child documents.
Therefore the page numbering in child documents may well
be inconsistent until the complete document is compiled at least once.

A useful (if unconventional) way to always ensure a consistent
page numbering is to restart the numbering in each child document
and denote the pages by `\textit{child}|.|\textit{page}'
where \textit{child} represents the chapter/section number of the child file.
This can be achieved by the command
|\numberwithin{page}{|\textit{child}|}|
of the \textsf{amsmath} package
where \textit{child} can be |chapter| or |section|
depending on the chosen structuring.
Alternatively, one can modify the macro |\thepage| appropriately
and reset the counter |page| at the start of each child file.

%%%%%%%%%%%%%%%%%%%%%%%%%%%%%%%%%%%%%%%%%%%%%%%%%%%%%%%%%%%%%%%%%%%%%%%%%%%%%%%%
\subsection{Conditional Processing}
\label{sec:conditional}

The package provides a mechanism to compile different versions
of a document. To customise the versions further some conditional processing
can come in handy to distinguish which version is being compiled.
The package provides two macros to describe the compilation context:

%%%%%%%%%%%%%%%%%%%%%%%%%%%%%%%%%%%%%%%%
\DescribeMacro{\ifchilddoc}
The conditional |\ifchilddoc| distinguishes between the compilation of
child documents and the main document:
%
\begin{center}
|\ifchilddoc |\textit{child-code}| |[|\||else |\textit{main-code}]| \||fi|
\end{center}

%%%%%%%%%%%%%%%%%%%%%%%%%%%%%%%%%%%%%%%%
\DescribeMacro{\childdocname}
\DescribeMacro{\childdocjob}
The macro |\childdocname| contains the filename (without extension)
of the main or child file being processed.
Note that |\childdocjob| will always contain the name of the main file.

%%%%%%%%%%%%%%%%%%%%%%%%%%%%%%%%%%%%%%%%
\paragraph{Title Page.}

Conditional processing can be used to include a title or banner page
in the main document when proper precautions are taken.
Importantly, the code in the main file should ensure that the page counter
(as well as other status parameters which are stored in the |.aux| files)
takes the same value after the conditional processing.
Otherwise the page numbers may take divergent values
depending on which part is compiled.

For example, a title page could be declared by:
%
\begin{center}
\begin{tabular}{l}
|\ifchilddoc\||else|\\
|\addtocounter{page}{-1}|\\
\textit{code for title page}\\
|\newpage|\\
|\||fi|
\end{tabular}
\end{center}
%
A banner page for the child documents can be generated by:
%
\begin{center}
\begin{tabular}{l}
|\ifchilddoc|\\
|\addtocounter{page}{-1}|\\
\textit{code for banner page}\\
|\newpage|\\
|\||fi|
\end{tabular}
\end{center}
%
Here one could write a message such as:
\begin{center}
|This is the part \childdocname{} of \childdocjob{}.|
\end{center}

%%%%%%%%%%%%%%%%%%%%%%%%%%%%%%%%%%%%%%%%%%%%%%%%%%%%%%%%%%%%%%%%%%%%%%%%%%%%%%%%
\subsection{Flags}
\label{sec:flags}

The package makes it easy to generate different versions
of the main or child documents.
To this end compilation flags can be defined
and assigned different default values.
They will be particularly useful in conjunction
with the forwarding mechanism described in \secref{sec:forward}.

For example, it may be useful to have a flag |\version|
which can be set to |draft| or |final|.
The document source will contain some conditional code
depending on the value of |\version|.
Suppose further, the flag should default to |final| for the main file
and to |draft| for child files
which is a natural assignment for editing the document.
This is achieved by placing the following code
in the preamble of the main document
(below the |\childdocmain| directive):
%
\begin{center}
\begin{tabular}{l}
|\ifchilddoc|\\
|\providecommand{\version}{draft}|\\
|\||else|\\
|\providecommand{\version}{final}|\\
|\||fi|
\end{tabular}
\end{center}
%
The definition by |\providecommand| makes sure
that previous definitions are not overwritten.
Further statements |\providecommand{\version}{...}|
can thus be added before the above code to override it.

For the main file, one might add a line
(between |\childdocmain| and the above block)
%
\begin{center}
|%\ifchilddoc\||else\providecommand{\version}{draft}\||fi|
\end{center}
%
which can be uncommented to produce a draft version.
Likewise one can add a line to the very top of a child file
(above the |\childdocof{|\textit{main}|}| directive)
%
\begin{center}
|%\providecommand{\version}{final}|
\end{center}
%
which can be uncommented to produce the final version of this child document.

%%%%%%%%%%%%%%%%%%%%%%%%%%%%%%%%%%%%%%%%%%%%%%%%%%%%%%%%%%%%%%%%%%%%%%%%%%%%%%%%
\subsection{Forwarding}
\label{sec:forward}

Different versions of the main or child documents
using compilation flags as described in \secref{sec:flags}
can be (permanently) stored in different files
for convenient compilation, viewing and distribution.
To this end, the package defines a command
to pass on compilation to a different file:

%%%%%%%%%%%%%%%%%%%%%%%%%%%%%%%%%%%%%%%%
\DescribeMacro{\childdocforward}
The command |\childdocforward| redirects processing to
another source file:
%
\begin{center}
\begin{tabular}{l}
|\input{childdoc.def}|\\
|\childdocforward[|\textit{main}|]{|\textit{dest}|}|\\
\end{tabular}
\end{center}
%
The argument \textit{dest} is the destination file
(without extension).
It should be the main file or one of the child files.
Note that further \textsf{childdoc} directives
such as |\childdocof| and |\childdocforward|
in the indicated file will be processed in this form.
The optional argument \textit{main}
passes on directly to the main file \textit{main}
while pretending to compile the child \textit{dest}.
This form behaves as if \textit{dest}
issues |\childdocof{|\textit{main}|}| right away,
and no further \textsf{childdoc} directives will be processed.

%%%%%%%%%%%%%%%%%%%%%%%%%%%%%%%%%%%%%%%%
\DescribeMacro{\...prefix}
In the alternative form |\childdocforwardprefix|,
%
\begin{center}
\begin{tabular}{l}
|\input{childdoc.def}|\\
|\childdocforwardprefix[|\textit{main}|]{|\textit{prefix}|}{|\textit{dest}|}|
\end{tabular}
\end{center}
%
the destination file is determined by a pattern
depending on the current file:
To make this work, the current file must be called
`{\textit{prefix}\hspace{0.2em}\textit{suffix}}'
with \textit{prefix} matching precisely the argument.
Processing is then passed on to the file
`{\textit{dest}\hspace{0.2em}\textit{suffix}}'.
Surely, the same effect is achieved by
directly specifying the
argument `{\textit{dest}\hspace{0.2em}\textit{suffix}}'
in the first form.
However, that requires to set up a different file
for each child. With the alternative form of the command
all these files can have exactly the same content
which simplifies setting them up and maintaining them.

For example, the following file |draft.tex|
with a compilation flag |\version| as described in \secref{sec:flags}
compiles the main document as a draft:
%
\begin{center}
\begin{tabular}{l}
|\def\version{draft}|\\
|\input{childdoc.def}|\\
|\childdocforward{|\textit{main}|}|
\end{tabular}
\end{center}
%
Likewise, the following files |final|\textit{nn}|.tex|
compile the final version of the child document
|child|\textit{nn}|.tex|:
%
\begin{center}
\begin{tabular}{l}
|\def\version{final}|\\
|\input{childdoc.def}|\\
|\childdocforwardprefix{final}{child}|
\end{tabular}
\end{center}
%

Note that when several versions of a main file and/or of each child file
are to be generated, it may be convenient to set up a |Makefile| or
shell script to automatise the process.

%%%%%%%%%%%%%%%%%%%%%%%%%%%%%%%%%%%%%%%%%%%%%%%%%%%%%%%%%%%%%%%%%%%%%%%%%%%%%%%%
\subsection{Command Line Processing}
\label{sec:commandline}

The effect of redirection files can also be achieved by invoking
the \LaTeX{} compiler with a more elaborate command line.
Most conveniently this should be done as part
of a shell script or a |Makefile|.

When using \textsf{childdoc} in the main file, the following
command lines effectively perform a redirection
(note that depending on the shell being used,
backslashes may have to be doubled: `|\|' $\to$ `|\\|'):
%
\begin{center}
|... -jobname "|\textit{target}|" |\\|"|[\textit{flags}]%
|\input{childdoc.def}\childdocforward[|\textit{main}|]{|\textit{dest}|}"|
\end{center}
%
Here \textit{target} is the name of the output file,
\textit{main} is the name of the main file
and \textit{dest} is the name of the main or child file to be processed
(all filenames without extensions).
The optional argument \textit{main} can be omitted
if \textit{main} matches \textit{dest}.
Optionally, compilation \textit{flags} can be defined via |\def| commands.
This command line makes the \TeX{} engine believe
it is compiling the file \textit{target}
whose content is specified as the latter parameter.
The provided code then forwards the processing to
\textit{main} or \textit{dest} as described in \secref{sec:forward}.

%%%%%%%%%%%%%%%%%%%%%%%%%%%%%%%%%%%%%%%%%%%%%%%%%%%%%%%%%%%%%%%%%%%%%%%%%%%%%%%%
\subsection{Include by Input}
\label{sec:input}

Including child documents by |\include| has some restrictions by design.
Most notably, the content of a child document always occupies
its own set of pages; pages cannot be shared between child documents.
Usually, this behaviour makes perfect sense
because each child document contain an essential part of the document.
However, in some situations it may be desirable to compose
a document from a collection of parts
without having mandatory page breaks between then.
For this case, the package
provides a mechanism to include parts
by |\input| which can also be processed individually.
However, by construction this mechanism
requires manual handling of the content to be output.

%%%%%%%%%%%%%%%%%%%%%%%%%%%%%%%%%%%%%%%%
\DescribeMacro{\ifchilddocmanual}
The main file should be prepared as usual, see \secref{sec:include}.
However, the document body must make a distinction
between processing of an individual part and of the main document, e.g.:
%
\begin{center}
\begin{tabular}{l}
|\ifchilddocmanual|\\
|\input{\childdocname}|\\
|\||else|\\
\textit{document body with }|\input{|\textit{part}|}|\\
|\||fi|
\end{tabular}
\end{center}
%
The conditional |\ifchilddocmanual| is true whenever
a part to be included by |\input| is being compiled,
and the name of the part is stored in |\childdocname|.

%%%%%%%%%%%%%%%%%%%%%%%%%%%%%%%%%%%%%%%%
\DescribeMacro{\childdocby}
Each part to be included by |\input| should start with:
%
\begin{center}
\begin{tabular}{l}
|\input{childdoc.def}|\\
|\childdocby{|\textit{main}|}|\\
\end{tabular}
\end{center}
%
The directive |\childdocby| is similar to |\childdocof|
described in \secref{sec:include},
but the subsequent selection of content must be done manually.
To that end, both |\ifchilddoc| and |\ifchilddocmanual|
will be true upon processing of a part,
and the name of the part is stored in |\childdocname|.
Note that |\jobname| will be set to the filename of the current part
so that each part receives an individual |.aux| file
that does not interfere with the |.aux| file(s) of the main document.
This behaviour can be altered by the alternative form
|\childdocby[*]{|\textit{main}|}| (with a non-empty optional argument)
which uses the |.aux| file of the main document
by setting |\jobname| to \textit{main}.

%%%%%%%%%%%%%%%%%%%%%%%%%%%%%%%%%%%%%%%%%%%%%%%%%%%%%%%%%%%%%%%%%%%%%%%%%%%%%%%%
\subsection{Driver Development}
\label{sec:driver}

The \textsf{childdoc} mechanism can also be use for the development
of definition files such as \LaTeX{} styles or classes.
This case differs from the above setup with multiple parts
included by |\include| in that no |\includeonly| should be invoked.
This can be achieved by starting the include file
(before |\ProvidesPackage|) with:
%
\begin{center}
\begin{tabular}{l}
|\input{childdoc.def}|\\
|\childdocforward{|\textit{main}|}|\\
\end{tabular}
\end{center}
%
or alternatively with:
%
\begin{center}
\begin{tabular}{l}
|\input{childdoc.def}|\\
|\childdocby{|\textit{main}|}|\\
\end{tabular}
\end{center}
%
Both forms have slightly different effects as described above.
The main file is prepared as usual, see \secref{sec:include}.

%%%%%%%%%%%%%%%%%%%%%%%%%%%%%%%%%%%%%%%%%%%%%%%%%%%%%%%%%%%%%%%%%%%%%%%%%%%%%%%%
\subsection{Legacy Detection}
\label{sec:detection}

The directive |\childdocmain| in the main file can detect
whether the complete document or merely a child is to be compiled
even without using the directive |\childdocof|.
This method is deprecated because it is less robust
and there is no compelling reason to use it;
it is merely provided for backward compatibility
and it may be removed in future versions.

If the detection mechanism is to be used,
it is mandatory to correctly specify
the filename of the main file as the argument of |\childdocmain|:
%
\begin{center}
\begin{tabular}{l}
|\input{childdoc.def}|\\
|\childdocmain{|\textit{main}|}|\\
\end{tabular}
\end{center}
%
If |\jobname| does not match the argument \textit{main} of |\childdocmain|,
it is assumed that |\jobname| points to the child file to be compiled.
When using |\childdocmain| with the main file specified as argument,
it suffices to start a child file
with just |\input{|\textit{main}|}|
without loading of the package and using |\childdocof|.
If instead all processing is done
with the appropriate \textsf{childdoc} directives,
the argument of \textit{main} of |\childdocmain| can be empty.

An alternative version of the command line processing described
in \secref{sec:commandline} using the detection mechanism reads:
%
\begin{center}
|... -jobname "|\textit{target}|" "|[\textit{flags}]%
[|\def\jobname{|\textit{dest}|}|]|\input{|\textit{main}|}"|
\end{center}

%%%%%%%%%%%%%%%%%%%%%%%%%%%%%%%%%%%%%%%%%%%%%%%%%%%%%%%%%%%%%%%%%%%%%%%%%%%%%%%%
\subsection{Manual Code}
\label{sec:manual}

In case one cannot be certain whether the definitions file |childdoc.def|
is installed on the target \TeX{} distribution
and one prefers not to ship it,
it is conceivable to paste a few relevant commands into the sources.

To that end, drop all statements |\input{childdoc.def}|
and perform the replacements as outlined below.
Instead of |\childdocmain{|\textit{main}|}| add the following code
to the top of the main file:
%
\begin{center}
\begin{tabular}{l}
|\||ifdefined\childdocname\endinput\||fi\newif\ifchilddoc|\\
|\edef\childdocname{\scantokens\expandafter{\jobname\noexpand}}|\\
|\def\childdocmain{|\textit{main}|}\||ifx\childdocmain\childdocname\||else|\\
|\childdoctrue\includeonly{\childdocname}\let\jobname\childdocmain\||fi|\\
\end{tabular}
\end{center}
%
Instead of |\childdocof{|\textit{main}|}| just include the main file
at the top of each child file:
%
\begin{center}
|\input{|\textit{main}|}|
\end{center}
%
A simple redirection |\childdocforward{|\textit{dest}|}| is achieved by:
%
\begin{center}
|\def\jobname{|\textit{dest}|}\input{\jobname}|
\end{center}
%
The redirection with prefix
|\childdocforwardprefix[|\textit{prefix}|]{|\textit{dest}|}|
is accomplished by:
%
\begin{center}
\begin{tabular}{l}
|{\edef\jobname{\scantokens\expandafter{\jobname\noexpand}}|\\
|\def\redirectjob |\textit{prefix}|#1~~~{\gdef\jobname{|\textit{dest}|#1}}|\\
|\expandafter\redirectjob\jobname~~~}\input{\jobname}|
\end{tabular}
\end{center}

In an alternative approach,
child documents can be compiled by a specific command line
without additional code or specific definitions:
%
\begin{center}
|... -jobname "|\textit{target}|" "|[\textit{flags}]%
|\includeonly{|\textit{dest}|}\input{|\textit{main}|}"|
\end{center}
%

%%%%%%%%%%%%%%%%%%%%%%%%%%%%%%%%%%%%%%%%%%%%%%%%%%%%%%%%%%%%%%%%%%%%%%%%%%%%%%%%
%%%%%%%%%%%%%%%%%%%%%%%%%%%%%%%%%%%%%%%%%%%%%%%%%%%%%%%%%%%%%%%%%%%%%%%%%%%%%%%%
\section{Information}

%%%%%%%%%%%%%%%%%%%%%%%%%%%%%%%%%%%%%%%%%%%%%%%%%%%%%%%%%%%%%%%%%%%%%%%%%%%%%%%%
\subsection{Copyright}

Copyright \copyright{} 2017--2018 Niklas Beisert

This work may be distributed and/or modified under the
conditions of the \LaTeX{} Project Public License, either version 1.3
of this license or (at your option) any later version.
The latest version of this license is in
  \url{http://www.latex-project.org/lppl.txt}
and version 1.3 or later is part of all distributions of \LaTeX{}
version 2005/12/01 or later.

This work has the LPPL maintenance status `maintained'.

The Current Maintainer of this work is Niklas Beisert.

This work consists of the files |README.txt|, |childdoc.ins| and |childdoc.dtx|
as well as the derived files |childdoc.def|, |cdocsamp.tex|
with |cdocsch1.tex|, |cdocsch2.tex|, |cdocspt3.tex|, |cdocspt4.tex|,
|cdocsdrf.tex|, |cdocsfn1.tex|, |cdocsfn2.tex|
as well as |childdoc.pdf|.

%%%%%%%%%%%%%%%%%%%%%%%%%%%%%%%%%%%%%%%%%%%%%%%%%%%%%%%%%%%%%%%%%%%%%%%%%%%%%%%%
\subsection{Files and Installation}

The package consists of the files:
%
\begin{center}
\begin{tabular}{ll}
    |README.txt|   & readme file \\
    |childdoc.ins| & installation file \\
    |childdoc.dtx| & source file \\
    |childdoc.def| & definition file \\
    |cdocsamp.tex| & sample main file \\
    |cdocsch1.tex| & sample include file \\
    |cdocsch2.tex| & sample include file \\
    |cdocspt3.tex| & sample part file \\
    |cdocspt4.tex| & sample part file \\
    |cdocsdrf.tex| & sample redirection file \\
    |cdocsfn1.tex| & sample redirection file \\
    |cdocsfn2.tex| & sample redirection file \\
    |childdoc.pdf| & manual
\end{tabular}
\end{center}
%
The distribution consists of the files
|README.txt|, |childdoc.ins| and |childdoc.dtx|.
%
\begin{itemize}
\item
Run (pdf)\LaTeX{} on |childdoc.dtx|
to compile the manual |childdoc.pdf| (this file).
\item
Run \LaTeX{} on |childdoc.ins| to create the definitions file |childdoc.def|
and the sample |cdocsamp.tex| with include files
|cdocsch1.tex|, |cdocsch2.tex|, |cdocspt3.tex|, |cdocspt4.tex|,
|cdocsdrf.tex|, |cdocsfn1.tex|, |cdocsfn2.tex|.
Then copy the file |childdoc.def| to an appropriate directory of your \LaTeX{}
distribution, e.g.\ \textit{texmf-root}|/tex/latex/childdoc|.
\end{itemize}

%%%%%%%%%%%%%%%%%%%%%%%%%%%%%%%%%%%%%%%%%%%%%%%%%%%%%%%%%%%%%%%%%%%%%%%%%%%%%%%%
\subsection{Related CTAN Packages}

There are several other packages which offer a similar functionality:
%
\begin{itemize}
\item
The packages
\href{http://ctan.org/pkg/docmute}{\textsf{docmute}},
\href{http://ctan.org/pkg/includex}{\textsf{includex}} and
\href{http://ctan.org/pkg/standalone}{\textsf{standalone}}
provide commands to include only the document body of
a child file thus allowing both files to be compiled individually.
\item
The packages \href{http://ctan.org/pkg/subdocs}{\textsf{subdocs}}
and \href{http://ctan.org/pkg/subfiles}{\textsf{subfiles}}
provide structures in which the main and child documents can be
encapsulated and allowing them to be compiled individually.
The inclusion mechanism is different from the conventional |\include|.
\item
The package \href{http://ctan.org/pkg/combine}{\textsf{combine}}
is an elaborate solution to combine several documents into one.
\end{itemize}
%
See also the CTAN topic \href{http://ctan.org/topic/subdocs}{\textsf{subdocs}}
for further related packages.
The present package differs from the above solutions in that
a document structure constructed with the conventional |\include| mechanism
just needs two extra commands at the top of every file
such that all constituent files can be compiled individually.

%%%%%%%%%%%%%%%%%%%%%%%%%%%%%%%%%%%%%%%%%%%%%%%%%%%%%%%%%%%%%%%%%%%%%%%%%%%%%%%%
%\subsection{Feature Suggestions}
%
%The following is a list of features which may be useful for future
%versions of this package:
%%
%\begin{itemize}
%\item
%\ldots
%\end{itemize}

%%%%%%%%%%%%%%%%%%%%%%%%%%%%%%%%%%%%%%%%%%%%%%%%%%%%%%%%%%%%%%%%%%%%%%%%%%%%%%%%
\subsection{Revision History}

%%%%%%%%%%%%%%%%%%%%%%%%%%%%%%%%%%%%%%%%
\paragraph{v2.0:} 2018/12/30

\begin{itemize}
\item
immediate forward processing
\item
added |\childdocby| mechanism
\item
manual restructured
\end{itemize}

%%%%%%%%%%%%%%%%%%%%%%%%%%%%%%%%%%%%%%%%
\paragraph{v1.6:} 2018/01/17

\begin{itemize}
\item
application for development of include files
\item
corrections to manual
\end{itemize}

%%%%%%%%%%%%%%%%%%%%%%%%%%%%%%%%%%%%%%%%
\paragraph{v1.5:} 2017/05/21

\begin{itemize}
\item
more complete structuring introduced
\item
|\childdocof| introduced
\item
|\childdoc| renamed to |\childdocmain|
\item
|\childredirect| renamed to |\childdocforward| and |\childdocforwardprefix|
and functionality expanded
\end{itemize}

%%%%%%%%%%%%%%%%%%%%%%%%%%%%%%%%%%%%%%%%
\paragraph{v1.0:} 2017/04/27

\begin{itemize}
\item
manual and install package
\item
first version published on CTAN
\end{itemize}

%%%%%%%%%%%%%%%%%%%%%%%%%%%%%%%%%%%%%%%%
\paragraph{v0.6:} 2017/04/26

\begin{itemize}
\item
redirection mechanism added
\end{itemize}

%%%%%%%%%%%%%%%%%%%%%%%%%%%%%%%%%%%%%%%%
\paragraph{v0.5:} 2017/04/26

\begin{itemize}
\item
functionality in definition file
\end{itemize}


%%%%%%%%%%%%%%%%%%%%%%%%%%%%%%%%%%%%%%%%%%%%%%%%%%%%%%%%%%%%%%%%%%%%%%%%%%%%%%%%
%%%%%%%%%%%%%%%%%%%%%%%%%%%%%%%%%%%%%%%%%%%%%%%%%%%%%%%%%%%%%%%%%%%%%%%%%%%%%%%%
%%%%%%%%%%%%%%%%%%%%%%%%%%%%%%%%%%%%%%%%%%%%%%%%%%%%%%%%%%%%%%%%%%%%%%%%%%%%%%%%
\appendix

\settowidth\MacroIndent{\rmfamily\scriptsize 000\ }

 \DocInput{childdoc.dtx}

\end{document}
%</driver>
% \fi
%
% %%%%%%%%%%%%%%%%%%%%%%%%%%%%%%%%%%%%%%%%%%%%%%%%%%%%%%%%%%%%%%%%%%%%%%%%%%%%%%
% %%%%%%%%%%%%%%%%%%%%%%%%%%%%%%%%%%%%%%%%%%%%%%%%%%%%%%%%%%%%%%%%%%%%%%%%%%%%%%
% \section{Sample}
%\iffalse
%<*samplemain>
%\fi
%
% The following presents a sample document
% with two chapters, two parts, a title page,
% a compile flag as well as three forwarding files to set the flag.
% It consists of eight |.tex| files:
% \begin{center}
% \begin{tabular}{ll}
% |cdocsamp.tex|&main file\\
% |cdocsch1.tex|&include file for chapter 1\\
% |cdocsch2.tex|&include file for chapter 2\\
% |cdocspt3.tex|&include file for part 3\\
% |cdocspt4.tex|&include file for part 4\\
% |cdocsdrf.tex|&forwarding file for main file in draft mode\\
% |cdocsfi1.tex|&forwarding file for final version of chapter 1\\
% |cdocsfi2.tex|&forwarding file for final version of chapter 2\\
% \end{tabular}
% \end{center}
% Each of the eight files can be compiled directly by the \LaTeX{} compiler.
%
% %%%%%%%%%%%%%%%%%%%%%%%%%%%%%%%%%%%%%%
% \paragraph{Main File.}
%
% The main file is called |cdocsamp.tex|.
%
% Load the \textsf{childdoc} definitions and
% declare the filename for the main document:
%    \begin{macrocode}
\input{childdoc.def}
\childdocmain{}
%    \end{macrocode}

% Optional override for |\version| flag:
%    \begin{macrocode}
%%\ifchilddoc\else\providecommand{\version}{draft}\fi
%    \end{macrocode}

% Define the default values for the |\version| flag
% (|final| for the main file and |draft| for childs):
%    \begin{macrocode}
\ifchilddoc
\providecommand{\version}{draft}
\else
\providecommand{\version}{final}
\fi
%    \end{macrocode}

% Load the standard document class:
%    \begin{macrocode}
\documentclass[12pt]{article}
%    \end{macrocode}

% Start the document body:
%    \begin{macrocode}
\begin{document}
%    \end{macrocode}

% Declare a title page.
% Print title, part of document being processed and version flag:
%    \begin{macrocode}
\addtocounter{page}{-1}
\begin{center}
{\LARGE\bfseries{}childdoc example\par}
\vspace{1cm}
\ifchilddoc
\ifchilddocmanual part\else chapter\fi:
`\childdocname' of `\childdocjob'\par
\else
main document: `\childdocjob'\par
\fi
version: \version\par
\end{center}
\newpage
%    \end{macrocode}

% Manually include selected file,
% otherwise process as usual:
%    \begin{macrocode}
\ifchilddocmanual
\section*{part `\childdocname'}
\input{\childdocname}
\else
%    \end{macrocode}

% Include the two chapters:
%    \begin{macrocode}
\include{cdocsch1}
\include{cdocsch2}
%    \end{macrocode}

% Include the two parts unless only chapters should be displayed:
%    \begin{macrocode}
\ifchilddoc\else
\section{part three}
\input{cdocspt3}
\section{part four}
\input{cdocspt4}
\fi
%    \end{macrocode}

% Process as usual until here:
%    \begin{macrocode}
\fi
%    \end{macrocode}

% End of document body:
%    \begin{macrocode}
\end{document}
%    \end{macrocode}
%\iffalse
%</samplemain>
%\fi
%
% %%%%%%%%%%%%%%%%%%%%%%%%%%%%%%%%%%%%%%
% \paragraph{Chapter Include Files.}
%
% The include files are called |cdocsch1.tex| and |cdocsch2.tex|.
%
%\iffalse
%<*samplechap1|samplechap2>
%\fi

% Optional override for |\version| flag:
%    \begin{macrocode}
%%\providecommand{\version}{final}
%    \end{macrocode}

% Include the main document:
%    \begin{macrocode}
\input{childdoc.def}
\childdocof{cdocsamp}
%    \end{macrocode}

%\iffalse
%</samplechap1|samplechap2>
%\fi
%
%\iffalse
%<*samplechap1>
%\fi
% Some text for chapter 1:
%    \begin{macrocode}
\section{one}
some text in chapter one
%    \end{macrocode}

%\iffalse
%</samplechap1>
%\fi
% Some text for chapter 2:
%\iffalse
%<*samplechap2>
%\fi
%    \begin{macrocode}
\section{two}
more text in chapter two
%    \end{macrocode}

%\iffalse
%</samplechap2>
%\fi
%
% %%%%%%%%%%%%%%%%%%%%%%%%%%%%%%%%%%%%%%
% \paragraph{Part Include Files.}
%
% The include files are called |cdocspt3.tex| and |cdocspt4.tex|.
%
%\iffalse
%<*samplepart3|samplepart4>
%\fi

% Optional override for |\version| flag:
%    \begin{macrocode}
%%\providecommand{\version}{final}
%    \end{macrocode}

% Include the main document:
%    \begin{macrocode}
\input{childdoc.def}
\childdocby{cdocsamp}
%    \end{macrocode}

%\iffalse
%</samplepart3|samplepart4>
%\fi
%
%\iffalse
%<*samplepart3>
%\fi
% Some text for part 3:
%    \begin{macrocode}
some text in part three
%    \end{macrocode}

%\iffalse
%</samplepart3>
%\fi
% Some text for part 4:
%\iffalse
%<*samplepart4>
%\fi
%    \begin{macrocode}
more text in part four
%    \end{macrocode}

%\iffalse
%</samplepart4>
%\fi
%
% %%%%%%%%%%%%%%%%%%%%%%%%%%%%%%%%%%%%%%
% \paragraph{Forwarding for a Complete Draft.}
%
% The following forwarding file |cdocsdrf.tex|
% compiles the main document in draft mode:
%\iffalse
%<*sampledraft>
%\fi
%    \begin{macrocode}
\def\version{draft}
\input{childdoc.def}
\childdocforward{cdocsamp}
%    \end{macrocode}

%\iffalse
%</sampledraft>
%\fi
%
% %%%%%%%%%%%%%%%%%%%%%%%%%%%%%%%%%%%%%%
% \paragraph{Forwarding for Final Version of the Chapters.}
%
% The following forwarding files |cdocsfn1.tex| and |cdocsfn2.tex|
% (with identical content)
% compile the final versions of the child documents
% |cdocsch1.tex| and |cdocsch2.tex|, respectively:
%\iffalse
%<*samplefinal>
%\fi
%    \begin{macrocode}
\def\version{final}
\input{childdoc.def}
\childdocforwardprefix[cdocsamp]{cdocsfn}{cdocsch}
%    \end{macrocode}

%\iffalse
%</samplefinal>
%\fi
%
% %%%%%%%%%%%%%%%%%%%%%%%%%%%%%%%%%%%%%%
% \paragraph{Command Line Processing.}
%
% The following three command lines generate the output files
% |cdocscld|, |cdocscl1| and |cdocscl2|
% which should be identical to
% |cdocsdrf|, |cdocsch1| and |cdocsfn2|, respectively:
% \begin{center}
% \begin{tabular}{l}
% |latex -jobname cdocscld \|\\
% |  "\def\version{draft}\input{childdoc.def}\childdocforward{cdocsamp}"|\\
% |latex -jobname cdocscl1 \|\\
% |  "\input{childdoc.def}\childdocforward[cdocsamp]{cdocsch1}"|\\
% |latex -jobname cdocscl2 \|\\
% |  "\def\version{final}\input{childdoc.def}\childdocforward{cdocsch2}"|
% \end{tabular}
% \end{center}
% Note that the trailing backslash on each first line
% merely continues the input to the second line
% (for convenient cut ant paste).
% Furthermore, the command |latex| can be replaced by any
% of its alternative versions such as |pdflatex|.
%
% %%%%%%%%%%%%%%%%%%%%%%%%%%%%%%%%%%%%%%%%%%%%%%%%%%%%%%%%%%%%%%%%%%%%%%%%%%%%%%
% %%%%%%%%%%%%%%%%%%%%%%%%%%%%%%%%%%%%%%%%%%%%%%%%%%%%%%%%%%%%%%%%%%%%%%%%%%%%%%
% \section{Implementation}
%\iffalse
%<*package>
%\fi
%
% This section describes the definitions file |childdoc.def|.

% The definitions cannot be loaded using |\usepackage| or |\RequirePackage|
% which has a mechanism to prevent loading a style file more than once.
% When loading the definitions by means of |\input|
% multiple instances have to be prevented manually:
%\iffalse
%This code needs to be before the `\ProvidesFile' directive
%which is defined at the beginning of this file.
%Therefore it is also placed there and commented out here.
%</package>
%<*discard>
%\fi
%    \begin{macrocode}
\ifdefined\childdocmain\endinput\fi
%    \end{macrocode}
%\iffalse
%</discard>
%<*package>
%\fi
%
% \macro{\ifchilddoc}
% \macro{\ifchilddocmanual}
% The conditional |\ifchilddoc| tells whether a
% child (true) or main (false) document is being compiled.
% The conditional |\ifchilddocmanual| tells whether
% the |\includeonly| mechanism is used (false) or
% the selection of child files must be performed manually (true).
% The definitions initialise to false:
%    \begin{macrocode}
\newif\ifchilddoc
\newif\ifchilddocmanual
%    \end{macrocode}

% \macro{\childdocname}
% \macro{\childdocjob}
% The macro |\childdocname| stores the name of the main document
% to be compiled. The macro |\childdocjob| stores the name of
% the document on which the \LaTeX{} compiler was originally invoked.
% The content of |\jobname| cannot be compared
% to filenames specified in the source due to different catcodes.
% The following code rescans |\jobname|, stores the result
% in |\childdocname| and saves a copy in |\childdocjob|:
%    \begin{macrocode}
\edef\childdocname{\scantokens\expandafter{\jobname\noexpand}}
\let\childdocjob\childdocname
%    \end{macrocode}

% \macro{\childdocdisable}
% The macro |\childdocdisable| prevents the main file
% from being processed more than once.
% At this stage, the main document command |\childdocmain|
% is assumed to be called once again where it should do nothing.
% Any subsequent call to it should prevent
% a secondary processing of the main document
% It overwrites the forwarding commands
% |\childdocof| and |\childdocforward|
% with empty macros to prevent further inclusions of the main document:
%    \begin{macrocode}
\newcommand{\childdocdisable}
{
  \renewcommand{\childdocmain}[1]{\renewcommand{\childdocmain}[1]{\endinput}}
  \renewcommand{\childdocof}[1]{}
  \renewcommand{\childdocby}[2][]{}
  \renewcommand{\childdocforward}[2][]{}
  \renewcommand{\childdocdisable}{}
}
%    \end{macrocode}

% \macro{\childdocmain}
% The macro |\childdocmain| is to be called at the top of the main file
% with nothing or the main filename (without extension) as argument.
% First, it breaks loops.
% If the argument is not empty and does not match |\childdocname|
% (which is set by the first inclusion of |childdoc.def|),
% |\ifchilddoc| is set to true, |\includeonly| is applied to the child file
% and |\jobname| is set to the main file
% (for proper handling of |.aux| files):
%    \begin{macrocode}
\newcommand{\childdocmain}[1]
{
  \childdocdisable\childdocmain{}
  \if?#1?\else
    \begingroup
      \def\childdoctmp{#1}
      \ifx\childdoctmp\childdocname
        \def\childdoctmp{}
      \else
        \def\childdoctmp
        {
          \childdoctrue
          \includeonly{\childdocname}
          \def\childdocjob{#1}
          \def\jobname{#1}
        }
      \fi
      \expandafter
    \endgroup
    \childdoctmp
  \fi
}
%    \end{macrocode}

% \macro{\childdocof}
% The command |\childdocof| redirects
% compilation to the main file |#1|.
%    \begin{macrocode}
\newcommand{\childdocof}[1]
{
  \childdocdisable
  \childdoctrue
  \includeonly{\childdocname}
  \def\jobname{#1}
  \def\childdocjob{#1}
  \input{#1}
}
%    \end{macrocode}

% \macro{\childdocby}
% The command |\childdocby| ....
%    \begin{macrocode}
\newcommand{\childdocby}[2][]
{
  \childdocdisable
  \childdoctrue
  \childdocmanualtrue
  \if?#1?\else
    \def\jobname{#2}
  \fi
  \def\childdocjob{#2}
  \input{#2}
  \endinput
}
%    \end{macrocode}

% \macro{\childdocforward}
% The command |\childdocforward| redirects
% compilation to the main file or
% (if the optional argument is given) a child file.
% Parameters are set as if the main file
% or a child file starting with |\childdocof| was compiled.
% Then compilation is handed over to the main file:
%    \begin{macrocode}
\newcommand{\childdocforward}[2][]
{
  \begingroup
    \if?#1?
      \def\childdoctmp
      {
        \def\childdocname{#2}
        \def\childdocjob{#2}
        \def\jobname{#2}
        \input{#2}
        \endinput
      }
    \else
      \def\childdoctmp
      {
        \childdocdisable
        \def\childdocname{#2}
        \childdoctrue
        \includeonly{#2}
        \def\childdocjob{#1}
        \def\jobname{#1}
        \input{#1}
        \endinput
      }
    \fi
    \expandafter
  \endgroup
  \childdoctmp
}
%    \end{macrocode}

% \macro{\childdocforwardprefix}
% The command |\childdocforwardprefix| redirects
% compilation to the main or a child file by means of a pattern.
% The prefix |#1| in the current filename is replaced by |#2|
% and the suffix of the current filename is kept
% (it is assumed that the filename does not contain the substring `|~~~|'
% which is used as a delimiter).
% Compilation is handed over to the new file by |\childdocforward|:
%    \begin{macrocode}
\newcommand{\childdocforwardprefix}[3][]
{
  \begingroup
    \def\childdocextract #2##1~~~{\def\childdoctmp{\childdocforward[#1]{#3##1}}}
    \expandafter\childdocextract\childdocname~~~
    \expandafter
  \endgroup
  \childdoctmp
}
%    \end{macrocode}

% \macro{\childdoc}
% The deprecated macro |\childdoc| is a legacy version of |\childdocmain|:
%    \begin{macrocode}
\newcommand{\childdoc}{\childdocmain}
%    \end{macrocode}

% \macro{\childdocredirect}
% The deprecated macro |\childdocredirect| is a legacy version
% of |\childdocforward| and |\childdocforwardprefix|:
%    \begin{macrocode}
\newcommand{\childdocredirect}[2][]
{
  \begingroup
    \if?#1?
      \def\childdoctmp{\childdocforward{#2}}
    \else
      \def\childdoctmp{\childdocforwardprefix{#1}{#2}}
    \fi
    \expandafter
  \endgroup
  \childdoctmp
}
%    \end{macrocode}

%\iffalse
%</package>
%\fi
%
\endinput

\childdocby{cdocsamp}
%    \end{macrocode}

%\iffalse
%</samplepart3|samplepart4>
%\fi
%
%\iffalse
%<*samplepart3>
%\fi
% Some text for part 3:
%    \begin{macrocode}
some text in part three
%    \end{macrocode}

%\iffalse
%</samplepart3>
%\fi
% Some text for part 4:
%\iffalse
%<*samplepart4>
%\fi
%    \begin{macrocode}
more text in part four
%    \end{macrocode}

%\iffalse
%</samplepart4>
%\fi
%
% %%%%%%%%%%%%%%%%%%%%%%%%%%%%%%%%%%%%%%
% \paragraph{Forwarding for a Complete Draft.}
%
% The following forwarding file |cdocsdrf.tex|
% compiles the main document in draft mode:
%\iffalse
%<*sampledraft>
%\fi
%    \begin{macrocode}
\def\version{draft}
% \iffalse
%
% childdoc.dtx Copyright (C) 2017-2018 Niklas Beisert
%
% This work may be distributed and/or modified under the
% conditions of the LaTeX Project Public License, either version 1.3
% of this license or (at your option) any later version.
% The latest version of this license is in
%   http://www.latex-project.org/lppl.txt
% and version 1.3 or later is part of all distributions of LaTeX
% version 2005/12/01 or later.
%
% This work has the LPPL maintenance status `maintained'.
%
% The Current Maintainer of this work is Niklas Beisert.
%
% This work consists of the files childdoc.dtx and childdoc.ins
% and the derived files childdoc.def and cdocsamp.tex with
% cdocsch1.tex, cdocsch2.tex, cdocsdrf.tex, cdocsfn1.tex, cdocsfn2.tex.
%
%<package>\ifdefined\childdocmain\endinput\fi
%<package>\ProvidesFile{childdoc.def}[2018/12/30 v2.0 child document driver]
%<samplemain>\ProvidesFile{cdocsamp.tex}[2018/12/30 v2.0 sample for childdoc]
%<*driver>
%\ProvidesFile{childdoc.drv}[2018/12/30 v2.0 childdoc reference manual file]
\PassOptionsToClass{10pt,a4paper}{article}
\documentclass{ltxdoc}

\usepackage[margin=35mm]{geometry}
\usepackage{hyperref}
\usepackage{hyperxmp}
\usepackage[usenames]{color}

\hypersetup{colorlinks=true}
\hypersetup{pdfstartview=FitH}
\hypersetup{pdfpagemode=UseNone}
\hypersetup{pdfsource={}}
\hypersetup{pdflang={en-UK}}
\hypersetup{pdfcopyright={Copyright 2017-2018 Niklas Beisert.
  This work may be distributed and/or modified under the
  conditions of the LaTeX Project Public License, either version 1.3
  of this license or (at your option) any later version.}}
\hypersetup{pdflicenseurl={http://www.latex-project.org/lppl.txt}}
\hypersetup{pdfcontactaddress={ETH Zurich, ITP, HIT K,
  Wolfgang-Pauli-Strasse 27}}
\hypersetup{pdfcontactpostcode={8093}}
\hypersetup{pdfcontactcity={Zurich}}
\hypersetup{pdfcontactcountry={Switzerland}}
\hypersetup{pdfcontactemail={nbeisert@itp.phys.ethz.ch}}
\hypersetup{pdfcontacturl={http://people.phys.ethz.ch/\xmptilde nbeisert/}}

\newcommand{\secref}[1]{\hyperref[#1]{section \ref*{#1}}}

\parskip1ex
\parindent0pt
\let\olditemize\itemize
\def\itemize{\olditemize\parskip0pt}

\begin{document}

\title{The \textsf{childdoc} Package}
\hypersetup{pdftitle={The childdoc Package}}
\author{Niklas Beisert\\[2ex]
  Institut f\"ur Theoretische Physik\\
  Eidgen\"ossische Technische Hochschule Z\"urich\\
  Wolfgang-Pauli-Strasse 27, 8093 Z\"urich, Switzerland\\[1ex]
  \href{mailto:nbeisert@itp.phys.ethz.ch}
  {\texttt{nbeisert@itp.phys.ethz.ch}}}
\hypersetup{pdfauthor={Niklas Beisert}}
\hypersetup{pdfsubject={Manual for the LaTeX2e Package childdoc}}
\date{30 December 2018, \textsf{v2.0}}
\maketitle

\begin{abstract}\noindent
\textsf{childdoc} is a \LaTeXe{} package
that enables the direct compilation
of document sections included by |\include|
to individual files.
\end{abstract}

\begingroup
\parskip0ex
\tableofcontents
\endgroup

%%%%%%%%%%%%%%%%%%%%%%%%%%%%%%%%%%%%%%%%%%%%%%%%%%%%%%%%%%%%%%%%%%%%%%%%%%%%%%%%
%%%%%%%%%%%%%%%%%%%%%%%%%%%%%%%%%%%%%%%%%%%%%%%%%%%%%%%%%%%%%%%%%%%%%%%%%%%%%%%%
\section{Introduction}

\LaTeX{} provides a mechanism to structure a large document (such as a book)
into a main file and several child files (containing the chapters)
using the |\include| command.
This mechanism is beneficial for documents
which span hundreds of pages in order to
make the source file(s) more manageable.
Moreover, compilation can be restricted to
selected child files by means of the |\includeonly| command.
The latter feature can be used to reduce the compilation time while editing
(this was significantly more useful in the earlier days of \LaTeX{})
or to generate a smaller document which is easier to navigate.
Another application of |\includeonly| is to generate
documents consisting of selected parts of the complete document.

However, there are a few drawbacks of the plain |\include| mechanism:
\begin{itemize}
\item
The child files cannot be compiled on their own,
they can only be compiled via the main file.
A naive editing environment
(such as a text editor with an option
to have the current file processed by \LaTeX)
may require one to switch to the main file before compiling;
attempting to compile the child file produces errors.
\item
The main file must be modified (each time)
to adjust the |\includeonly| command
to the present needs. This easily leaves the main file in a messy state.
\item
The generated document will always carry the filename
of the main document. This is inconvenient if
several child files are to be compiled and
to be kept for distribution.
\end{itemize}

The present package provides a simple interface
to make child files individually compilable by \LaTeX{}.
Compiling a child file then has the same effect as compiling
the main file with an |\includeonly| command
to select the appropriate child.
Moreover the generated document will carry the name of the child
rather than the main file.
This resolves all three above issues.

This feature is meant to make the editing of books,
thesis documents and lecture notes somewhat more convenient.
However, the package can also be used efficiently for
composing a series of documents (such as exercise sheets)
which are typically distributed individually.
It then assists the author in generating the individual documents
(potentially in different versions)
as well as a document containing the collected series.
Another application is in developing style files
or other kinds of included material
where compilation of the style file could redirect
to a sample or test file.

%%%%%%%%%%%%%%%%%%%%%%%%%%%%%%%%%%%%%%%%%%%%%%%%%%%%%%%%%%%%%%%%%%%%%%%%%%%%%%%%
%%%%%%%%%%%%%%%%%%%%%%%%%%%%%%%%%%%%%%%%%%%%%%%%%%%%%%%%%%%%%%%%%%%%%%%%%%%%%%%%
\section{Usage}

First of all, the package \textsf{childdoc} is \emph{not} a standard
\LaTeXe{} |.sty| style file! Therefore it needs to be invoked in
a non-standard way.

%%%%%%%%%%%%%%%%%%%%%%%%%%%%%%%%%%%%%%%%%%%%%%%%%%%%%%%%%%%%%%%%%%%%%%%%%%%%%%%%
\subsection{Included Files}
\label{sec:include}

%%%%%%%%%%%%%%%%%%%%%%%%%%%%%%%%%%%%%%%%
\DescribeMacro{\childdocmain}
To use the package, add the commands
\begin{center}
\begin{tabular}{l}
|\input{childdoc.def}|\\
|\childdocmain{}|\\
\end{tabular}
\end{center}
at the very top of the main \LaTeX{} file,
in particular \emph{before} the |\documentclass| statement!
The argument of |\childdocmain| should be left empty
(but it must be present).

%%%%%%%%%%%%%%%%%%%%%%%%%%%%%%%%%%%%%%%%
\DescribeMacro{\childdocof}
Furthermore, add the commands
\begin{center}
\begin{tabular}{l}
|\input{childdoc.def}|\\
|\childdocof{|\textit{main}|}|\\
\end{tabular}
\end{center}
at the top of every child file \textit{child}
which is included by |\include{|\textit{child}|}|
from within the main file
(or at least for those files to be compiled individually).
The argument \textit{main} must be the filename of the main file.

There are a couple of
considerations in setting up the main and child documents:

%%%%%%%%%%%%%%%%%%%%%%%%%%%%%%%%%%%%%%%%
\paragraph{Restrictions.}

Please note the following restrictions:
\begin{itemize}
\item
|\childdocmain| must be called with one argument \textit{main}
to ensure compatibility with earlier version of the package.
It must either be empty (|\childdocmain{}|)
or precisely match the filename of the main file in which it is specified.
See \secref{sec:detection} for further information.
\item
The filename \textit{main} must be specified without the |.tex| extension.
\item
The filename \textit{main} is case sensitive
(even in case-insensitive file systems)
due to internal string comparison.
\item
The argument \textit{main} should be fully expanded, it cannot be a macro.
\item
Subdirectories and special characters should be avoided in filenames.
\item
The command |\childdocmain{|\textit{main}|}| must be followed by a whitespace.
It should not be followed immediately by another command
or by a comment mark `|%|'.
This is because the \TeX{} parser reads the token immediately following
the argument of |\childdocmain| and puts it
at the beginning of every child section;
however, a white\-space is ignored.
\end{itemize}

%%%%%%%%%%%%%%%%%%%%%%%%%%%%%%%%%%%%%%%%
\paragraph{Content of Main File.}

It is advisable to place all content in the child files included by |\include|.
Any output contained in the main file will appear in all child documents
unless suppressed manually;
it cannot be suppressed automatically by the |\includeonly| directive
and thus should normally be avoided.
A method to include some content in the main file
by means of conditional processing is described in \secref{sec:conditional}.

%%%%%%%%%%%%%%%%%%%%%%%%%%%%%%%%%%%%%%%%
\paragraph{Page Numbering.}

When only a part of the document is compiled,
the appropriate numbering of pages
(as well as other status parameters)
is determined from the |.aux| files.
The latter contain information from previous passes.
However this information needs to propagate through
all intermediate child documents.
Therefore the page numbering in child documents may well
be inconsistent until the complete document is compiled at least once.

A useful (if unconventional) way to always ensure a consistent
page numbering is to restart the numbering in each child document
and denote the pages by `\textit{child}|.|\textit{page}'
where \textit{child} represents the chapter/section number of the child file.
This can be achieved by the command
|\numberwithin{page}{|\textit{child}|}|
of the \textsf{amsmath} package
where \textit{child} can be |chapter| or |section|
depending on the chosen structuring.
Alternatively, one can modify the macro |\thepage| appropriately
and reset the counter |page| at the start of each child file.

%%%%%%%%%%%%%%%%%%%%%%%%%%%%%%%%%%%%%%%%%%%%%%%%%%%%%%%%%%%%%%%%%%%%%%%%%%%%%%%%
\subsection{Conditional Processing}
\label{sec:conditional}

The package provides a mechanism to compile different versions
of a document. To customise the versions further some conditional processing
can come in handy to distinguish which version is being compiled.
The package provides two macros to describe the compilation context:

%%%%%%%%%%%%%%%%%%%%%%%%%%%%%%%%%%%%%%%%
\DescribeMacro{\ifchilddoc}
The conditional |\ifchilddoc| distinguishes between the compilation of
child documents and the main document:
%
\begin{center}
|\ifchilddoc |\textit{child-code}| |[|\||else |\textit{main-code}]| \||fi|
\end{center}

%%%%%%%%%%%%%%%%%%%%%%%%%%%%%%%%%%%%%%%%
\DescribeMacro{\childdocname}
\DescribeMacro{\childdocjob}
The macro |\childdocname| contains the filename (without extension)
of the main or child file being processed.
Note that |\childdocjob| will always contain the name of the main file.

%%%%%%%%%%%%%%%%%%%%%%%%%%%%%%%%%%%%%%%%
\paragraph{Title Page.}

Conditional processing can be used to include a title or banner page
in the main document when proper precautions are taken.
Importantly, the code in the main file should ensure that the page counter
(as well as other status parameters which are stored in the |.aux| files)
takes the same value after the conditional processing.
Otherwise the page numbers may take divergent values
depending on which part is compiled.

For example, a title page could be declared by:
%
\begin{center}
\begin{tabular}{l}
|\ifchilddoc\||else|\\
|\addtocounter{page}{-1}|\\
\textit{code for title page}\\
|\newpage|\\
|\||fi|
\end{tabular}
\end{center}
%
A banner page for the child documents can be generated by:
%
\begin{center}
\begin{tabular}{l}
|\ifchilddoc|\\
|\addtocounter{page}{-1}|\\
\textit{code for banner page}\\
|\newpage|\\
|\||fi|
\end{tabular}
\end{center}
%
Here one could write a message such as:
\begin{center}
|This is the part \childdocname{} of \childdocjob{}.|
\end{center}

%%%%%%%%%%%%%%%%%%%%%%%%%%%%%%%%%%%%%%%%%%%%%%%%%%%%%%%%%%%%%%%%%%%%%%%%%%%%%%%%
\subsection{Flags}
\label{sec:flags}

The package makes it easy to generate different versions
of the main or child documents.
To this end compilation flags can be defined
and assigned different default values.
They will be particularly useful in conjunction
with the forwarding mechanism described in \secref{sec:forward}.

For example, it may be useful to have a flag |\version|
which can be set to |draft| or |final|.
The document source will contain some conditional code
depending on the value of |\version|.
Suppose further, the flag should default to |final| for the main file
and to |draft| for child files
which is a natural assignment for editing the document.
This is achieved by placing the following code
in the preamble of the main document
(below the |\childdocmain| directive):
%
\begin{center}
\begin{tabular}{l}
|\ifchilddoc|\\
|\providecommand{\version}{draft}|\\
|\||else|\\
|\providecommand{\version}{final}|\\
|\||fi|
\end{tabular}
\end{center}
%
The definition by |\providecommand| makes sure
that previous definitions are not overwritten.
Further statements |\providecommand{\version}{...}|
can thus be added before the above code to override it.

For the main file, one might add a line
(between |\childdocmain| and the above block)
%
\begin{center}
|%\ifchilddoc\||else\providecommand{\version}{draft}\||fi|
\end{center}
%
which can be uncommented to produce a draft version.
Likewise one can add a line to the very top of a child file
(above the |\childdocof{|\textit{main}|}| directive)
%
\begin{center}
|%\providecommand{\version}{final}|
\end{center}
%
which can be uncommented to produce the final version of this child document.

%%%%%%%%%%%%%%%%%%%%%%%%%%%%%%%%%%%%%%%%%%%%%%%%%%%%%%%%%%%%%%%%%%%%%%%%%%%%%%%%
\subsection{Forwarding}
\label{sec:forward}

Different versions of the main or child documents
using compilation flags as described in \secref{sec:flags}
can be (permanently) stored in different files
for convenient compilation, viewing and distribution.
To this end, the package defines a command
to pass on compilation to a different file:

%%%%%%%%%%%%%%%%%%%%%%%%%%%%%%%%%%%%%%%%
\DescribeMacro{\childdocforward}
The command |\childdocforward| redirects processing to
another source file:
%
\begin{center}
\begin{tabular}{l}
|\input{childdoc.def}|\\
|\childdocforward[|\textit{main}|]{|\textit{dest}|}|\\
\end{tabular}
\end{center}
%
The argument \textit{dest} is the destination file
(without extension).
It should be the main file or one of the child files.
Note that further \textsf{childdoc} directives
such as |\childdocof| and |\childdocforward|
in the indicated file will be processed in this form.
The optional argument \textit{main}
passes on directly to the main file \textit{main}
while pretending to compile the child \textit{dest}.
This form behaves as if \textit{dest}
issues |\childdocof{|\textit{main}|}| right away,
and no further \textsf{childdoc} directives will be processed.

%%%%%%%%%%%%%%%%%%%%%%%%%%%%%%%%%%%%%%%%
\DescribeMacro{\...prefix}
In the alternative form |\childdocforwardprefix|,
%
\begin{center}
\begin{tabular}{l}
|\input{childdoc.def}|\\
|\childdocforwardprefix[|\textit{main}|]{|\textit{prefix}|}{|\textit{dest}|}|
\end{tabular}
\end{center}
%
the destination file is determined by a pattern
depending on the current file:
To make this work, the current file must be called
`{\textit{prefix}\hspace{0.2em}\textit{suffix}}'
with \textit{prefix} matching precisely the argument.
Processing is then passed on to the file
`{\textit{dest}\hspace{0.2em}\textit{suffix}}'.
Surely, the same effect is achieved by
directly specifying the
argument `{\textit{dest}\hspace{0.2em}\textit{suffix}}'
in the first form.
However, that requires to set up a different file
for each child. With the alternative form of the command
all these files can have exactly the same content
which simplifies setting them up and maintaining them.

For example, the following file |draft.tex|
with a compilation flag |\version| as described in \secref{sec:flags}
compiles the main document as a draft:
%
\begin{center}
\begin{tabular}{l}
|\def\version{draft}|\\
|\input{childdoc.def}|\\
|\childdocforward{|\textit{main}|}|
\end{tabular}
\end{center}
%
Likewise, the following files |final|\textit{nn}|.tex|
compile the final version of the child document
|child|\textit{nn}|.tex|:
%
\begin{center}
\begin{tabular}{l}
|\def\version{final}|\\
|\input{childdoc.def}|\\
|\childdocforwardprefix{final}{child}|
\end{tabular}
\end{center}
%

Note that when several versions of a main file and/or of each child file
are to be generated, it may be convenient to set up a |Makefile| or
shell script to automatise the process.

%%%%%%%%%%%%%%%%%%%%%%%%%%%%%%%%%%%%%%%%%%%%%%%%%%%%%%%%%%%%%%%%%%%%%%%%%%%%%%%%
\subsection{Command Line Processing}
\label{sec:commandline}

The effect of redirection files can also be achieved by invoking
the \LaTeX{} compiler with a more elaborate command line.
Most conveniently this should be done as part
of a shell script or a |Makefile|.

When using \textsf{childdoc} in the main file, the following
command lines effectively perform a redirection
(note that depending on the shell being used,
backslashes may have to be doubled: `|\|' $\to$ `|\\|'):
%
\begin{center}
|... -jobname "|\textit{target}|" |\\|"|[\textit{flags}]%
|\input{childdoc.def}\childdocforward[|\textit{main}|]{|\textit{dest}|}"|
\end{center}
%
Here \textit{target} is the name of the output file,
\textit{main} is the name of the main file
and \textit{dest} is the name of the main or child file to be processed
(all filenames without extensions).
The optional argument \textit{main} can be omitted
if \textit{main} matches \textit{dest}.
Optionally, compilation \textit{flags} can be defined via |\def| commands.
This command line makes the \TeX{} engine believe
it is compiling the file \textit{target}
whose content is specified as the latter parameter.
The provided code then forwards the processing to
\textit{main} or \textit{dest} as described in \secref{sec:forward}.

%%%%%%%%%%%%%%%%%%%%%%%%%%%%%%%%%%%%%%%%%%%%%%%%%%%%%%%%%%%%%%%%%%%%%%%%%%%%%%%%
\subsection{Include by Input}
\label{sec:input}

Including child documents by |\include| has some restrictions by design.
Most notably, the content of a child document always occupies
its own set of pages; pages cannot be shared between child documents.
Usually, this behaviour makes perfect sense
because each child document contain an essential part of the document.
However, in some situations it may be desirable to compose
a document from a collection of parts
without having mandatory page breaks between then.
For this case, the package
provides a mechanism to include parts
by |\input| which can also be processed individually.
However, by construction this mechanism
requires manual handling of the content to be output.

%%%%%%%%%%%%%%%%%%%%%%%%%%%%%%%%%%%%%%%%
\DescribeMacro{\ifchilddocmanual}
The main file should be prepared as usual, see \secref{sec:include}.
However, the document body must make a distinction
between processing of an individual part and of the main document, e.g.:
%
\begin{center}
\begin{tabular}{l}
|\ifchilddocmanual|\\
|\input{\childdocname}|\\
|\||else|\\
\textit{document body with }|\input{|\textit{part}|}|\\
|\||fi|
\end{tabular}
\end{center}
%
The conditional |\ifchilddocmanual| is true whenever
a part to be included by |\input| is being compiled,
and the name of the part is stored in |\childdocname|.

%%%%%%%%%%%%%%%%%%%%%%%%%%%%%%%%%%%%%%%%
\DescribeMacro{\childdocby}
Each part to be included by |\input| should start with:
%
\begin{center}
\begin{tabular}{l}
|\input{childdoc.def}|\\
|\childdocby{|\textit{main}|}|\\
\end{tabular}
\end{center}
%
The directive |\childdocby| is similar to |\childdocof|
described in \secref{sec:include},
but the subsequent selection of content must be done manually.
To that end, both |\ifchilddoc| and |\ifchilddocmanual|
will be true upon processing of a part,
and the name of the part is stored in |\childdocname|.
Note that |\jobname| will be set to the filename of the current part
so that each part receives an individual |.aux| file
that does not interfere with the |.aux| file(s) of the main document.
This behaviour can be altered by the alternative form
|\childdocby[*]{|\textit{main}|}| (with a non-empty optional argument)
which uses the |.aux| file of the main document
by setting |\jobname| to \textit{main}.

%%%%%%%%%%%%%%%%%%%%%%%%%%%%%%%%%%%%%%%%%%%%%%%%%%%%%%%%%%%%%%%%%%%%%%%%%%%%%%%%
\subsection{Driver Development}
\label{sec:driver}

The \textsf{childdoc} mechanism can also be use for the development
of definition files such as \LaTeX{} styles or classes.
This case differs from the above setup with multiple parts
included by |\include| in that no |\includeonly| should be invoked.
This can be achieved by starting the include file
(before |\ProvidesPackage|) with:
%
\begin{center}
\begin{tabular}{l}
|\input{childdoc.def}|\\
|\childdocforward{|\textit{main}|}|\\
\end{tabular}
\end{center}
%
or alternatively with:
%
\begin{center}
\begin{tabular}{l}
|\input{childdoc.def}|\\
|\childdocby{|\textit{main}|}|\\
\end{tabular}
\end{center}
%
Both forms have slightly different effects as described above.
The main file is prepared as usual, see \secref{sec:include}.

%%%%%%%%%%%%%%%%%%%%%%%%%%%%%%%%%%%%%%%%%%%%%%%%%%%%%%%%%%%%%%%%%%%%%%%%%%%%%%%%
\subsection{Legacy Detection}
\label{sec:detection}

The directive |\childdocmain| in the main file can detect
whether the complete document or merely a child is to be compiled
even without using the directive |\childdocof|.
This method is deprecated because it is less robust
and there is no compelling reason to use it;
it is merely provided for backward compatibility
and it may be removed in future versions.

If the detection mechanism is to be used,
it is mandatory to correctly specify
the filename of the main file as the argument of |\childdocmain|:
%
\begin{center}
\begin{tabular}{l}
|\input{childdoc.def}|\\
|\childdocmain{|\textit{main}|}|\\
\end{tabular}
\end{center}
%
If |\jobname| does not match the argument \textit{main} of |\childdocmain|,
it is assumed that |\jobname| points to the child file to be compiled.
When using |\childdocmain| with the main file specified as argument,
it suffices to start a child file
with just |\input{|\textit{main}|}|
without loading of the package and using |\childdocof|.
If instead all processing is done
with the appropriate \textsf{childdoc} directives,
the argument of \textit{main} of |\childdocmain| can be empty.

An alternative version of the command line processing described
in \secref{sec:commandline} using the detection mechanism reads:
%
\begin{center}
|... -jobname "|\textit{target}|" "|[\textit{flags}]%
[|\def\jobname{|\textit{dest}|}|]|\input{|\textit{main}|}"|
\end{center}

%%%%%%%%%%%%%%%%%%%%%%%%%%%%%%%%%%%%%%%%%%%%%%%%%%%%%%%%%%%%%%%%%%%%%%%%%%%%%%%%
\subsection{Manual Code}
\label{sec:manual}

In case one cannot be certain whether the definitions file |childdoc.def|
is installed on the target \TeX{} distribution
and one prefers not to ship it,
it is conceivable to paste a few relevant commands into the sources.

To that end, drop all statements |\input{childdoc.def}|
and perform the replacements as outlined below.
Instead of |\childdocmain{|\textit{main}|}| add the following code
to the top of the main file:
%
\begin{center}
\begin{tabular}{l}
|\||ifdefined\childdocname\endinput\||fi\newif\ifchilddoc|\\
|\edef\childdocname{\scantokens\expandafter{\jobname\noexpand}}|\\
|\def\childdocmain{|\textit{main}|}\||ifx\childdocmain\childdocname\||else|\\
|\childdoctrue\includeonly{\childdocname}\let\jobname\childdocmain\||fi|\\
\end{tabular}
\end{center}
%
Instead of |\childdocof{|\textit{main}|}| just include the main file
at the top of each child file:
%
\begin{center}
|\input{|\textit{main}|}|
\end{center}
%
A simple redirection |\childdocforward{|\textit{dest}|}| is achieved by:
%
\begin{center}
|\def\jobname{|\textit{dest}|}\input{\jobname}|
\end{center}
%
The redirection with prefix
|\childdocforwardprefix[|\textit{prefix}|]{|\textit{dest}|}|
is accomplished by:
%
\begin{center}
\begin{tabular}{l}
|{\edef\jobname{\scantokens\expandafter{\jobname\noexpand}}|\\
|\def\redirectjob |\textit{prefix}|#1~~~{\gdef\jobname{|\textit{dest}|#1}}|\\
|\expandafter\redirectjob\jobname~~~}\input{\jobname}|
\end{tabular}
\end{center}

In an alternative approach,
child documents can be compiled by a specific command line
without additional code or specific definitions:
%
\begin{center}
|... -jobname "|\textit{target}|" "|[\textit{flags}]%
|\includeonly{|\textit{dest}|}\input{|\textit{main}|}"|
\end{center}
%

%%%%%%%%%%%%%%%%%%%%%%%%%%%%%%%%%%%%%%%%%%%%%%%%%%%%%%%%%%%%%%%%%%%%%%%%%%%%%%%%
%%%%%%%%%%%%%%%%%%%%%%%%%%%%%%%%%%%%%%%%%%%%%%%%%%%%%%%%%%%%%%%%%%%%%%%%%%%%%%%%
\section{Information}

%%%%%%%%%%%%%%%%%%%%%%%%%%%%%%%%%%%%%%%%%%%%%%%%%%%%%%%%%%%%%%%%%%%%%%%%%%%%%%%%
\subsection{Copyright}

Copyright \copyright{} 2017--2018 Niklas Beisert

This work may be distributed and/or modified under the
conditions of the \LaTeX{} Project Public License, either version 1.3
of this license or (at your option) any later version.
The latest version of this license is in
  \url{http://www.latex-project.org/lppl.txt}
and version 1.3 or later is part of all distributions of \LaTeX{}
version 2005/12/01 or later.

This work has the LPPL maintenance status `maintained'.

The Current Maintainer of this work is Niklas Beisert.

This work consists of the files |README.txt|, |childdoc.ins| and |childdoc.dtx|
as well as the derived files |childdoc.def|, |cdocsamp.tex|
with |cdocsch1.tex|, |cdocsch2.tex|, |cdocspt3.tex|, |cdocspt4.tex|,
|cdocsdrf.tex|, |cdocsfn1.tex|, |cdocsfn2.tex|
as well as |childdoc.pdf|.

%%%%%%%%%%%%%%%%%%%%%%%%%%%%%%%%%%%%%%%%%%%%%%%%%%%%%%%%%%%%%%%%%%%%%%%%%%%%%%%%
\subsection{Files and Installation}

The package consists of the files:
%
\begin{center}
\begin{tabular}{ll}
    |README.txt|   & readme file \\
    |childdoc.ins| & installation file \\
    |childdoc.dtx| & source file \\
    |childdoc.def| & definition file \\
    |cdocsamp.tex| & sample main file \\
    |cdocsch1.tex| & sample include file \\
    |cdocsch2.tex| & sample include file \\
    |cdocspt3.tex| & sample part file \\
    |cdocspt4.tex| & sample part file \\
    |cdocsdrf.tex| & sample redirection file \\
    |cdocsfn1.tex| & sample redirection file \\
    |cdocsfn2.tex| & sample redirection file \\
    |childdoc.pdf| & manual
\end{tabular}
\end{center}
%
The distribution consists of the files
|README.txt|, |childdoc.ins| and |childdoc.dtx|.
%
\begin{itemize}
\item
Run (pdf)\LaTeX{} on |childdoc.dtx|
to compile the manual |childdoc.pdf| (this file).
\item
Run \LaTeX{} on |childdoc.ins| to create the definitions file |childdoc.def|
and the sample |cdocsamp.tex| with include files
|cdocsch1.tex|, |cdocsch2.tex|, |cdocspt3.tex|, |cdocspt4.tex|,
|cdocsdrf.tex|, |cdocsfn1.tex|, |cdocsfn2.tex|.
Then copy the file |childdoc.def| to an appropriate directory of your \LaTeX{}
distribution, e.g.\ \textit{texmf-root}|/tex/latex/childdoc|.
\end{itemize}

%%%%%%%%%%%%%%%%%%%%%%%%%%%%%%%%%%%%%%%%%%%%%%%%%%%%%%%%%%%%%%%%%%%%%%%%%%%%%%%%
\subsection{Related CTAN Packages}

There are several other packages which offer a similar functionality:
%
\begin{itemize}
\item
The packages
\href{http://ctan.org/pkg/docmute}{\textsf{docmute}},
\href{http://ctan.org/pkg/includex}{\textsf{includex}} and
\href{http://ctan.org/pkg/standalone}{\textsf{standalone}}
provide commands to include only the document body of
a child file thus allowing both files to be compiled individually.
\item
The packages \href{http://ctan.org/pkg/subdocs}{\textsf{subdocs}}
and \href{http://ctan.org/pkg/subfiles}{\textsf{subfiles}}
provide structures in which the main and child documents can be
encapsulated and allowing them to be compiled individually.
The inclusion mechanism is different from the conventional |\include|.
\item
The package \href{http://ctan.org/pkg/combine}{\textsf{combine}}
is an elaborate solution to combine several documents into one.
\end{itemize}
%
See also the CTAN topic \href{http://ctan.org/topic/subdocs}{\textsf{subdocs}}
for further related packages.
The present package differs from the above solutions in that
a document structure constructed with the conventional |\include| mechanism
just needs two extra commands at the top of every file
such that all constituent files can be compiled individually.

%%%%%%%%%%%%%%%%%%%%%%%%%%%%%%%%%%%%%%%%%%%%%%%%%%%%%%%%%%%%%%%%%%%%%%%%%%%%%%%%
%\subsection{Feature Suggestions}
%
%The following is a list of features which may be useful for future
%versions of this package:
%%
%\begin{itemize}
%\item
%\ldots
%\end{itemize}

%%%%%%%%%%%%%%%%%%%%%%%%%%%%%%%%%%%%%%%%%%%%%%%%%%%%%%%%%%%%%%%%%%%%%%%%%%%%%%%%
\subsection{Revision History}

%%%%%%%%%%%%%%%%%%%%%%%%%%%%%%%%%%%%%%%%
\paragraph{v2.0:} 2018/12/30

\begin{itemize}
\item
immediate forward processing
\item
added |\childdocby| mechanism
\item
manual restructured
\end{itemize}

%%%%%%%%%%%%%%%%%%%%%%%%%%%%%%%%%%%%%%%%
\paragraph{v1.6:} 2018/01/17

\begin{itemize}
\item
application for development of include files
\item
corrections to manual
\end{itemize}

%%%%%%%%%%%%%%%%%%%%%%%%%%%%%%%%%%%%%%%%
\paragraph{v1.5:} 2017/05/21

\begin{itemize}
\item
more complete structuring introduced
\item
|\childdocof| introduced
\item
|\childdoc| renamed to |\childdocmain|
\item
|\childredirect| renamed to |\childdocforward| and |\childdocforwardprefix|
and functionality expanded
\end{itemize}

%%%%%%%%%%%%%%%%%%%%%%%%%%%%%%%%%%%%%%%%
\paragraph{v1.0:} 2017/04/27

\begin{itemize}
\item
manual and install package
\item
first version published on CTAN
\end{itemize}

%%%%%%%%%%%%%%%%%%%%%%%%%%%%%%%%%%%%%%%%
\paragraph{v0.6:} 2017/04/26

\begin{itemize}
\item
redirection mechanism added
\end{itemize}

%%%%%%%%%%%%%%%%%%%%%%%%%%%%%%%%%%%%%%%%
\paragraph{v0.5:} 2017/04/26

\begin{itemize}
\item
functionality in definition file
\end{itemize}


%%%%%%%%%%%%%%%%%%%%%%%%%%%%%%%%%%%%%%%%%%%%%%%%%%%%%%%%%%%%%%%%%%%%%%%%%%%%%%%%
%%%%%%%%%%%%%%%%%%%%%%%%%%%%%%%%%%%%%%%%%%%%%%%%%%%%%%%%%%%%%%%%%%%%%%%%%%%%%%%%
%%%%%%%%%%%%%%%%%%%%%%%%%%%%%%%%%%%%%%%%%%%%%%%%%%%%%%%%%%%%%%%%%%%%%%%%%%%%%%%%
\appendix

\settowidth\MacroIndent{\rmfamily\scriptsize 000\ }

 \DocInput{childdoc.dtx}

\end{document}
%</driver>
% \fi
%
% %%%%%%%%%%%%%%%%%%%%%%%%%%%%%%%%%%%%%%%%%%%%%%%%%%%%%%%%%%%%%%%%%%%%%%%%%%%%%%
% %%%%%%%%%%%%%%%%%%%%%%%%%%%%%%%%%%%%%%%%%%%%%%%%%%%%%%%%%%%%%%%%%%%%%%%%%%%%%%
% \section{Sample}
%\iffalse
%<*samplemain>
%\fi
%
% The following presents a sample document
% with two chapters, two parts, a title page,
% a compile flag as well as three forwarding files to set the flag.
% It consists of eight |.tex| files:
% \begin{center}
% \begin{tabular}{ll}
% |cdocsamp.tex|&main file\\
% |cdocsch1.tex|&include file for chapter 1\\
% |cdocsch2.tex|&include file for chapter 2\\
% |cdocspt3.tex|&include file for part 3\\
% |cdocspt4.tex|&include file for part 4\\
% |cdocsdrf.tex|&forwarding file for main file in draft mode\\
% |cdocsfi1.tex|&forwarding file for final version of chapter 1\\
% |cdocsfi2.tex|&forwarding file for final version of chapter 2\\
% \end{tabular}
% \end{center}
% Each of the eight files can be compiled directly by the \LaTeX{} compiler.
%
% %%%%%%%%%%%%%%%%%%%%%%%%%%%%%%%%%%%%%%
% \paragraph{Main File.}
%
% The main file is called |cdocsamp.tex|.
%
% Load the \textsf{childdoc} definitions and
% declare the filename for the main document:
%    \begin{macrocode}
\input{childdoc.def}
\childdocmain{}
%    \end{macrocode}

% Optional override for |\version| flag:
%    \begin{macrocode}
%%\ifchilddoc\else\providecommand{\version}{draft}\fi
%    \end{macrocode}

% Define the default values for the |\version| flag
% (|final| for the main file and |draft| for childs):
%    \begin{macrocode}
\ifchilddoc
\providecommand{\version}{draft}
\else
\providecommand{\version}{final}
\fi
%    \end{macrocode}

% Load the standard document class:
%    \begin{macrocode}
\documentclass[12pt]{article}
%    \end{macrocode}

% Start the document body:
%    \begin{macrocode}
\begin{document}
%    \end{macrocode}

% Declare a title page.
% Print title, part of document being processed and version flag:
%    \begin{macrocode}
\addtocounter{page}{-1}
\begin{center}
{\LARGE\bfseries{}childdoc example\par}
\vspace{1cm}
\ifchilddoc
\ifchilddocmanual part\else chapter\fi:
`\childdocname' of `\childdocjob'\par
\else
main document: `\childdocjob'\par
\fi
version: \version\par
\end{center}
\newpage
%    \end{macrocode}

% Manually include selected file,
% otherwise process as usual:
%    \begin{macrocode}
\ifchilddocmanual
\section*{part `\childdocname'}
\input{\childdocname}
\else
%    \end{macrocode}

% Include the two chapters:
%    \begin{macrocode}
\include{cdocsch1}
\include{cdocsch2}
%    \end{macrocode}

% Include the two parts unless only chapters should be displayed:
%    \begin{macrocode}
\ifchilddoc\else
\section{part three}
\input{cdocspt3}
\section{part four}
\input{cdocspt4}
\fi
%    \end{macrocode}

% Process as usual until here:
%    \begin{macrocode}
\fi
%    \end{macrocode}

% End of document body:
%    \begin{macrocode}
\end{document}
%    \end{macrocode}
%\iffalse
%</samplemain>
%\fi
%
% %%%%%%%%%%%%%%%%%%%%%%%%%%%%%%%%%%%%%%
% \paragraph{Chapter Include Files.}
%
% The include files are called |cdocsch1.tex| and |cdocsch2.tex|.
%
%\iffalse
%<*samplechap1|samplechap2>
%\fi

% Optional override for |\version| flag:
%    \begin{macrocode}
%%\providecommand{\version}{final}
%    \end{macrocode}

% Include the main document:
%    \begin{macrocode}
\input{childdoc.def}
\childdocof{cdocsamp}
%    \end{macrocode}

%\iffalse
%</samplechap1|samplechap2>
%\fi
%
%\iffalse
%<*samplechap1>
%\fi
% Some text for chapter 1:
%    \begin{macrocode}
\section{one}
some text in chapter one
%    \end{macrocode}

%\iffalse
%</samplechap1>
%\fi
% Some text for chapter 2:
%\iffalse
%<*samplechap2>
%\fi
%    \begin{macrocode}
\section{two}
more text in chapter two
%    \end{macrocode}

%\iffalse
%</samplechap2>
%\fi
%
% %%%%%%%%%%%%%%%%%%%%%%%%%%%%%%%%%%%%%%
% \paragraph{Part Include Files.}
%
% The include files are called |cdocspt3.tex| and |cdocspt4.tex|.
%
%\iffalse
%<*samplepart3|samplepart4>
%\fi

% Optional override for |\version| flag:
%    \begin{macrocode}
%%\providecommand{\version}{final}
%    \end{macrocode}

% Include the main document:
%    \begin{macrocode}
\input{childdoc.def}
\childdocby{cdocsamp}
%    \end{macrocode}

%\iffalse
%</samplepart3|samplepart4>
%\fi
%
%\iffalse
%<*samplepart3>
%\fi
% Some text for part 3:
%    \begin{macrocode}
some text in part three
%    \end{macrocode}

%\iffalse
%</samplepart3>
%\fi
% Some text for part 4:
%\iffalse
%<*samplepart4>
%\fi
%    \begin{macrocode}
more text in part four
%    \end{macrocode}

%\iffalse
%</samplepart4>
%\fi
%
% %%%%%%%%%%%%%%%%%%%%%%%%%%%%%%%%%%%%%%
% \paragraph{Forwarding for a Complete Draft.}
%
% The following forwarding file |cdocsdrf.tex|
% compiles the main document in draft mode:
%\iffalse
%<*sampledraft>
%\fi
%    \begin{macrocode}
\def\version{draft}
\input{childdoc.def}
\childdocforward{cdocsamp}
%    \end{macrocode}

%\iffalse
%</sampledraft>
%\fi
%
% %%%%%%%%%%%%%%%%%%%%%%%%%%%%%%%%%%%%%%
% \paragraph{Forwarding for Final Version of the Chapters.}
%
% The following forwarding files |cdocsfn1.tex| and |cdocsfn2.tex|
% (with identical content)
% compile the final versions of the child documents
% |cdocsch1.tex| and |cdocsch2.tex|, respectively:
%\iffalse
%<*samplefinal>
%\fi
%    \begin{macrocode}
\def\version{final}
\input{childdoc.def}
\childdocforwardprefix[cdocsamp]{cdocsfn}{cdocsch}
%    \end{macrocode}

%\iffalse
%</samplefinal>
%\fi
%
% %%%%%%%%%%%%%%%%%%%%%%%%%%%%%%%%%%%%%%
% \paragraph{Command Line Processing.}
%
% The following three command lines generate the output files
% |cdocscld|, |cdocscl1| and |cdocscl2|
% which should be identical to
% |cdocsdrf|, |cdocsch1| and |cdocsfn2|, respectively:
% \begin{center}
% \begin{tabular}{l}
% |latex -jobname cdocscld \|\\
% |  "\def\version{draft}\input{childdoc.def}\childdocforward{cdocsamp}"|\\
% |latex -jobname cdocscl1 \|\\
% |  "\input{childdoc.def}\childdocforward[cdocsamp]{cdocsch1}"|\\
% |latex -jobname cdocscl2 \|\\
% |  "\def\version{final}\input{childdoc.def}\childdocforward{cdocsch2}"|
% \end{tabular}
% \end{center}
% Note that the trailing backslash on each first line
% merely continues the input to the second line
% (for convenient cut ant paste).
% Furthermore, the command |latex| can be replaced by any
% of its alternative versions such as |pdflatex|.
%
% %%%%%%%%%%%%%%%%%%%%%%%%%%%%%%%%%%%%%%%%%%%%%%%%%%%%%%%%%%%%%%%%%%%%%%%%%%%%%%
% %%%%%%%%%%%%%%%%%%%%%%%%%%%%%%%%%%%%%%%%%%%%%%%%%%%%%%%%%%%%%%%%%%%%%%%%%%%%%%
% \section{Implementation}
%\iffalse
%<*package>
%\fi
%
% This section describes the definitions file |childdoc.def|.

% The definitions cannot be loaded using |\usepackage| or |\RequirePackage|
% which has a mechanism to prevent loading a style file more than once.
% When loading the definitions by means of |\input|
% multiple instances have to be prevented manually:
%\iffalse
%This code needs to be before the `\ProvidesFile' directive
%which is defined at the beginning of this file.
%Therefore it is also placed there and commented out here.
%</package>
%<*discard>
%\fi
%    \begin{macrocode}
\ifdefined\childdocmain\endinput\fi
%    \end{macrocode}
%\iffalse
%</discard>
%<*package>
%\fi
%
% \macro{\ifchilddoc}
% \macro{\ifchilddocmanual}
% The conditional |\ifchilddoc| tells whether a
% child (true) or main (false) document is being compiled.
% The conditional |\ifchilddocmanual| tells whether
% the |\includeonly| mechanism is used (false) or
% the selection of child files must be performed manually (true).
% The definitions initialise to false:
%    \begin{macrocode}
\newif\ifchilddoc
\newif\ifchilddocmanual
%    \end{macrocode}

% \macro{\childdocname}
% \macro{\childdocjob}
% The macro |\childdocname| stores the name of the main document
% to be compiled. The macro |\childdocjob| stores the name of
% the document on which the \LaTeX{} compiler was originally invoked.
% The content of |\jobname| cannot be compared
% to filenames specified in the source due to different catcodes.
% The following code rescans |\jobname|, stores the result
% in |\childdocname| and saves a copy in |\childdocjob|:
%    \begin{macrocode}
\edef\childdocname{\scantokens\expandafter{\jobname\noexpand}}
\let\childdocjob\childdocname
%    \end{macrocode}

% \macro{\childdocdisable}
% The macro |\childdocdisable| prevents the main file
% from being processed more than once.
% At this stage, the main document command |\childdocmain|
% is assumed to be called once again where it should do nothing.
% Any subsequent call to it should prevent
% a secondary processing of the main document
% It overwrites the forwarding commands
% |\childdocof| and |\childdocforward|
% with empty macros to prevent further inclusions of the main document:
%    \begin{macrocode}
\newcommand{\childdocdisable}
{
  \renewcommand{\childdocmain}[1]{\renewcommand{\childdocmain}[1]{\endinput}}
  \renewcommand{\childdocof}[1]{}
  \renewcommand{\childdocby}[2][]{}
  \renewcommand{\childdocforward}[2][]{}
  \renewcommand{\childdocdisable}{}
}
%    \end{macrocode}

% \macro{\childdocmain}
% The macro |\childdocmain| is to be called at the top of the main file
% with nothing or the main filename (without extension) as argument.
% First, it breaks loops.
% If the argument is not empty and does not match |\childdocname|
% (which is set by the first inclusion of |childdoc.def|),
% |\ifchilddoc| is set to true, |\includeonly| is applied to the child file
% and |\jobname| is set to the main file
% (for proper handling of |.aux| files):
%    \begin{macrocode}
\newcommand{\childdocmain}[1]
{
  \childdocdisable\childdocmain{}
  \if?#1?\else
    \begingroup
      \def\childdoctmp{#1}
      \ifx\childdoctmp\childdocname
        \def\childdoctmp{}
      \else
        \def\childdoctmp
        {
          \childdoctrue
          \includeonly{\childdocname}
          \def\childdocjob{#1}
          \def\jobname{#1}
        }
      \fi
      \expandafter
    \endgroup
    \childdoctmp
  \fi
}
%    \end{macrocode}

% \macro{\childdocof}
% The command |\childdocof| redirects
% compilation to the main file |#1|.
%    \begin{macrocode}
\newcommand{\childdocof}[1]
{
  \childdocdisable
  \childdoctrue
  \includeonly{\childdocname}
  \def\jobname{#1}
  \def\childdocjob{#1}
  \input{#1}
}
%    \end{macrocode}

% \macro{\childdocby}
% The command |\childdocby| ....
%    \begin{macrocode}
\newcommand{\childdocby}[2][]
{
  \childdocdisable
  \childdoctrue
  \childdocmanualtrue
  \if?#1?\else
    \def\jobname{#2}
  \fi
  \def\childdocjob{#2}
  \input{#2}
  \endinput
}
%    \end{macrocode}

% \macro{\childdocforward}
% The command |\childdocforward| redirects
% compilation to the main file or
% (if the optional argument is given) a child file.
% Parameters are set as if the main file
% or a child file starting with |\childdocof| was compiled.
% Then compilation is handed over to the main file:
%    \begin{macrocode}
\newcommand{\childdocforward}[2][]
{
  \begingroup
    \if?#1?
      \def\childdoctmp
      {
        \def\childdocname{#2}
        \def\childdocjob{#2}
        \def\jobname{#2}
        \input{#2}
        \endinput
      }
    \else
      \def\childdoctmp
      {
        \childdocdisable
        \def\childdocname{#2}
        \childdoctrue
        \includeonly{#2}
        \def\childdocjob{#1}
        \def\jobname{#1}
        \input{#1}
        \endinput
      }
    \fi
    \expandafter
  \endgroup
  \childdoctmp
}
%    \end{macrocode}

% \macro{\childdocforwardprefix}
% The command |\childdocforwardprefix| redirects
% compilation to the main or a child file by means of a pattern.
% The prefix |#1| in the current filename is replaced by |#2|
% and the suffix of the current filename is kept
% (it is assumed that the filename does not contain the substring `|~~~|'
% which is used as a delimiter).
% Compilation is handed over to the new file by |\childdocforward|:
%    \begin{macrocode}
\newcommand{\childdocforwardprefix}[3][]
{
  \begingroup
    \def\childdocextract #2##1~~~{\def\childdoctmp{\childdocforward[#1]{#3##1}}}
    \expandafter\childdocextract\childdocname~~~
    \expandafter
  \endgroup
  \childdoctmp
}
%    \end{macrocode}

% \macro{\childdoc}
% The deprecated macro |\childdoc| is a legacy version of |\childdocmain|:
%    \begin{macrocode}
\newcommand{\childdoc}{\childdocmain}
%    \end{macrocode}

% \macro{\childdocredirect}
% The deprecated macro |\childdocredirect| is a legacy version
% of |\childdocforward| and |\childdocforwardprefix|:
%    \begin{macrocode}
\newcommand{\childdocredirect}[2][]
{
  \begingroup
    \if?#1?
      \def\childdoctmp{\childdocforward{#2}}
    \else
      \def\childdoctmp{\childdocforwardprefix{#1}{#2}}
    \fi
    \expandafter
  \endgroup
  \childdoctmp
}
%    \end{macrocode}

%\iffalse
%</package>
%\fi
%
\endinput

\childdocforward{cdocsamp}
%    \end{macrocode}

%\iffalse
%</sampledraft>
%\fi
%
% %%%%%%%%%%%%%%%%%%%%%%%%%%%%%%%%%%%%%%
% \paragraph{Forwarding for Final Version of the Chapters.}
%
% The following forwarding files |cdocsfn1.tex| and |cdocsfn2.tex|
% (with identical content)
% compile the final versions of the child documents
% |cdocsch1.tex| and |cdocsch2.tex|, respectively:
%\iffalse
%<*samplefinal>
%\fi
%    \begin{macrocode}
\def\version{final}
% \iffalse
%
% childdoc.dtx Copyright (C) 2017-2018 Niklas Beisert
%
% This work may be distributed and/or modified under the
% conditions of the LaTeX Project Public License, either version 1.3
% of this license or (at your option) any later version.
% The latest version of this license is in
%   http://www.latex-project.org/lppl.txt
% and version 1.3 or later is part of all distributions of LaTeX
% version 2005/12/01 or later.
%
% This work has the LPPL maintenance status `maintained'.
%
% The Current Maintainer of this work is Niklas Beisert.
%
% This work consists of the files childdoc.dtx and childdoc.ins
% and the derived files childdoc.def and cdocsamp.tex with
% cdocsch1.tex, cdocsch2.tex, cdocsdrf.tex, cdocsfn1.tex, cdocsfn2.tex.
%
%<package>\ifdefined\childdocmain\endinput\fi
%<package>\ProvidesFile{childdoc.def}[2018/12/30 v2.0 child document driver]
%<samplemain>\ProvidesFile{cdocsamp.tex}[2018/12/30 v2.0 sample for childdoc]
%<*driver>
%\ProvidesFile{childdoc.drv}[2018/12/30 v2.0 childdoc reference manual file]
\PassOptionsToClass{10pt,a4paper}{article}
\documentclass{ltxdoc}

\usepackage[margin=35mm]{geometry}
\usepackage{hyperref}
\usepackage{hyperxmp}
\usepackage[usenames]{color}

\hypersetup{colorlinks=true}
\hypersetup{pdfstartview=FitH}
\hypersetup{pdfpagemode=UseNone}
\hypersetup{pdfsource={}}
\hypersetup{pdflang={en-UK}}
\hypersetup{pdfcopyright={Copyright 2017-2018 Niklas Beisert.
  This work may be distributed and/or modified under the
  conditions of the LaTeX Project Public License, either version 1.3
  of this license or (at your option) any later version.}}
\hypersetup{pdflicenseurl={http://www.latex-project.org/lppl.txt}}
\hypersetup{pdfcontactaddress={ETH Zurich, ITP, HIT K,
  Wolfgang-Pauli-Strasse 27}}
\hypersetup{pdfcontactpostcode={8093}}
\hypersetup{pdfcontactcity={Zurich}}
\hypersetup{pdfcontactcountry={Switzerland}}
\hypersetup{pdfcontactemail={nbeisert@itp.phys.ethz.ch}}
\hypersetup{pdfcontacturl={http://people.phys.ethz.ch/\xmptilde nbeisert/}}

\newcommand{\secref}[1]{\hyperref[#1]{section \ref*{#1}}}

\parskip1ex
\parindent0pt
\let\olditemize\itemize
\def\itemize{\olditemize\parskip0pt}

\begin{document}

\title{The \textsf{childdoc} Package}
\hypersetup{pdftitle={The childdoc Package}}
\author{Niklas Beisert\\[2ex]
  Institut f\"ur Theoretische Physik\\
  Eidgen\"ossische Technische Hochschule Z\"urich\\
  Wolfgang-Pauli-Strasse 27, 8093 Z\"urich, Switzerland\\[1ex]
  \href{mailto:nbeisert@itp.phys.ethz.ch}
  {\texttt{nbeisert@itp.phys.ethz.ch}}}
\hypersetup{pdfauthor={Niklas Beisert}}
\hypersetup{pdfsubject={Manual for the LaTeX2e Package childdoc}}
\date{30 December 2018, \textsf{v2.0}}
\maketitle

\begin{abstract}\noindent
\textsf{childdoc} is a \LaTeXe{} package
that enables the direct compilation
of document sections included by |\include|
to individual files.
\end{abstract}

\begingroup
\parskip0ex
\tableofcontents
\endgroup

%%%%%%%%%%%%%%%%%%%%%%%%%%%%%%%%%%%%%%%%%%%%%%%%%%%%%%%%%%%%%%%%%%%%%%%%%%%%%%%%
%%%%%%%%%%%%%%%%%%%%%%%%%%%%%%%%%%%%%%%%%%%%%%%%%%%%%%%%%%%%%%%%%%%%%%%%%%%%%%%%
\section{Introduction}

\LaTeX{} provides a mechanism to structure a large document (such as a book)
into a main file and several child files (containing the chapters)
using the |\include| command.
This mechanism is beneficial for documents
which span hundreds of pages in order to
make the source file(s) more manageable.
Moreover, compilation can be restricted to
selected child files by means of the |\includeonly| command.
The latter feature can be used to reduce the compilation time while editing
(this was significantly more useful in the earlier days of \LaTeX{})
or to generate a smaller document which is easier to navigate.
Another application of |\includeonly| is to generate
documents consisting of selected parts of the complete document.

However, there are a few drawbacks of the plain |\include| mechanism:
\begin{itemize}
\item
The child files cannot be compiled on their own,
they can only be compiled via the main file.
A naive editing environment
(such as a text editor with an option
to have the current file processed by \LaTeX)
may require one to switch to the main file before compiling;
attempting to compile the child file produces errors.
\item
The main file must be modified (each time)
to adjust the |\includeonly| command
to the present needs. This easily leaves the main file in a messy state.
\item
The generated document will always carry the filename
of the main document. This is inconvenient if
several child files are to be compiled and
to be kept for distribution.
\end{itemize}

The present package provides a simple interface
to make child files individually compilable by \LaTeX{}.
Compiling a child file then has the same effect as compiling
the main file with an |\includeonly| command
to select the appropriate child.
Moreover the generated document will carry the name of the child
rather than the main file.
This resolves all three above issues.

This feature is meant to make the editing of books,
thesis documents and lecture notes somewhat more convenient.
However, the package can also be used efficiently for
composing a series of documents (such as exercise sheets)
which are typically distributed individually.
It then assists the author in generating the individual documents
(potentially in different versions)
as well as a document containing the collected series.
Another application is in developing style files
or other kinds of included material
where compilation of the style file could redirect
to a sample or test file.

%%%%%%%%%%%%%%%%%%%%%%%%%%%%%%%%%%%%%%%%%%%%%%%%%%%%%%%%%%%%%%%%%%%%%%%%%%%%%%%%
%%%%%%%%%%%%%%%%%%%%%%%%%%%%%%%%%%%%%%%%%%%%%%%%%%%%%%%%%%%%%%%%%%%%%%%%%%%%%%%%
\section{Usage}

First of all, the package \textsf{childdoc} is \emph{not} a standard
\LaTeXe{} |.sty| style file! Therefore it needs to be invoked in
a non-standard way.

%%%%%%%%%%%%%%%%%%%%%%%%%%%%%%%%%%%%%%%%%%%%%%%%%%%%%%%%%%%%%%%%%%%%%%%%%%%%%%%%
\subsection{Included Files}
\label{sec:include}

%%%%%%%%%%%%%%%%%%%%%%%%%%%%%%%%%%%%%%%%
\DescribeMacro{\childdocmain}
To use the package, add the commands
\begin{center}
\begin{tabular}{l}
|\input{childdoc.def}|\\
|\childdocmain{}|\\
\end{tabular}
\end{center}
at the very top of the main \LaTeX{} file,
in particular \emph{before} the |\documentclass| statement!
The argument of |\childdocmain| should be left empty
(but it must be present).

%%%%%%%%%%%%%%%%%%%%%%%%%%%%%%%%%%%%%%%%
\DescribeMacro{\childdocof}
Furthermore, add the commands
\begin{center}
\begin{tabular}{l}
|\input{childdoc.def}|\\
|\childdocof{|\textit{main}|}|\\
\end{tabular}
\end{center}
at the top of every child file \textit{child}
which is included by |\include{|\textit{child}|}|
from within the main file
(or at least for those files to be compiled individually).
The argument \textit{main} must be the filename of the main file.

There are a couple of
considerations in setting up the main and child documents:

%%%%%%%%%%%%%%%%%%%%%%%%%%%%%%%%%%%%%%%%
\paragraph{Restrictions.}

Please note the following restrictions:
\begin{itemize}
\item
|\childdocmain| must be called with one argument \textit{main}
to ensure compatibility with earlier version of the package.
It must either be empty (|\childdocmain{}|)
or precisely match the filename of the main file in which it is specified.
See \secref{sec:detection} for further information.
\item
The filename \textit{main} must be specified without the |.tex| extension.
\item
The filename \textit{main} is case sensitive
(even in case-insensitive file systems)
due to internal string comparison.
\item
The argument \textit{main} should be fully expanded, it cannot be a macro.
\item
Subdirectories and special characters should be avoided in filenames.
\item
The command |\childdocmain{|\textit{main}|}| must be followed by a whitespace.
It should not be followed immediately by another command
or by a comment mark `|%|'.
This is because the \TeX{} parser reads the token immediately following
the argument of |\childdocmain| and puts it
at the beginning of every child section;
however, a white\-space is ignored.
\end{itemize}

%%%%%%%%%%%%%%%%%%%%%%%%%%%%%%%%%%%%%%%%
\paragraph{Content of Main File.}

It is advisable to place all content in the child files included by |\include|.
Any output contained in the main file will appear in all child documents
unless suppressed manually;
it cannot be suppressed automatically by the |\includeonly| directive
and thus should normally be avoided.
A method to include some content in the main file
by means of conditional processing is described in \secref{sec:conditional}.

%%%%%%%%%%%%%%%%%%%%%%%%%%%%%%%%%%%%%%%%
\paragraph{Page Numbering.}

When only a part of the document is compiled,
the appropriate numbering of pages
(as well as other status parameters)
is determined from the |.aux| files.
The latter contain information from previous passes.
However this information needs to propagate through
all intermediate child documents.
Therefore the page numbering in child documents may well
be inconsistent until the complete document is compiled at least once.

A useful (if unconventional) way to always ensure a consistent
page numbering is to restart the numbering in each child document
and denote the pages by `\textit{child}|.|\textit{page}'
where \textit{child} represents the chapter/section number of the child file.
This can be achieved by the command
|\numberwithin{page}{|\textit{child}|}|
of the \textsf{amsmath} package
where \textit{child} can be |chapter| or |section|
depending on the chosen structuring.
Alternatively, one can modify the macro |\thepage| appropriately
and reset the counter |page| at the start of each child file.

%%%%%%%%%%%%%%%%%%%%%%%%%%%%%%%%%%%%%%%%%%%%%%%%%%%%%%%%%%%%%%%%%%%%%%%%%%%%%%%%
\subsection{Conditional Processing}
\label{sec:conditional}

The package provides a mechanism to compile different versions
of a document. To customise the versions further some conditional processing
can come in handy to distinguish which version is being compiled.
The package provides two macros to describe the compilation context:

%%%%%%%%%%%%%%%%%%%%%%%%%%%%%%%%%%%%%%%%
\DescribeMacro{\ifchilddoc}
The conditional |\ifchilddoc| distinguishes between the compilation of
child documents and the main document:
%
\begin{center}
|\ifchilddoc |\textit{child-code}| |[|\||else |\textit{main-code}]| \||fi|
\end{center}

%%%%%%%%%%%%%%%%%%%%%%%%%%%%%%%%%%%%%%%%
\DescribeMacro{\childdocname}
\DescribeMacro{\childdocjob}
The macro |\childdocname| contains the filename (without extension)
of the main or child file being processed.
Note that |\childdocjob| will always contain the name of the main file.

%%%%%%%%%%%%%%%%%%%%%%%%%%%%%%%%%%%%%%%%
\paragraph{Title Page.}

Conditional processing can be used to include a title or banner page
in the main document when proper precautions are taken.
Importantly, the code in the main file should ensure that the page counter
(as well as other status parameters which are stored in the |.aux| files)
takes the same value after the conditional processing.
Otherwise the page numbers may take divergent values
depending on which part is compiled.

For example, a title page could be declared by:
%
\begin{center}
\begin{tabular}{l}
|\ifchilddoc\||else|\\
|\addtocounter{page}{-1}|\\
\textit{code for title page}\\
|\newpage|\\
|\||fi|
\end{tabular}
\end{center}
%
A banner page for the child documents can be generated by:
%
\begin{center}
\begin{tabular}{l}
|\ifchilddoc|\\
|\addtocounter{page}{-1}|\\
\textit{code for banner page}\\
|\newpage|\\
|\||fi|
\end{tabular}
\end{center}
%
Here one could write a message such as:
\begin{center}
|This is the part \childdocname{} of \childdocjob{}.|
\end{center}

%%%%%%%%%%%%%%%%%%%%%%%%%%%%%%%%%%%%%%%%%%%%%%%%%%%%%%%%%%%%%%%%%%%%%%%%%%%%%%%%
\subsection{Flags}
\label{sec:flags}

The package makes it easy to generate different versions
of the main or child documents.
To this end compilation flags can be defined
and assigned different default values.
They will be particularly useful in conjunction
with the forwarding mechanism described in \secref{sec:forward}.

For example, it may be useful to have a flag |\version|
which can be set to |draft| or |final|.
The document source will contain some conditional code
depending on the value of |\version|.
Suppose further, the flag should default to |final| for the main file
and to |draft| for child files
which is a natural assignment for editing the document.
This is achieved by placing the following code
in the preamble of the main document
(below the |\childdocmain| directive):
%
\begin{center}
\begin{tabular}{l}
|\ifchilddoc|\\
|\providecommand{\version}{draft}|\\
|\||else|\\
|\providecommand{\version}{final}|\\
|\||fi|
\end{tabular}
\end{center}
%
The definition by |\providecommand| makes sure
that previous definitions are not overwritten.
Further statements |\providecommand{\version}{...}|
can thus be added before the above code to override it.

For the main file, one might add a line
(between |\childdocmain| and the above block)
%
\begin{center}
|%\ifchilddoc\||else\providecommand{\version}{draft}\||fi|
\end{center}
%
which can be uncommented to produce a draft version.
Likewise one can add a line to the very top of a child file
(above the |\childdocof{|\textit{main}|}| directive)
%
\begin{center}
|%\providecommand{\version}{final}|
\end{center}
%
which can be uncommented to produce the final version of this child document.

%%%%%%%%%%%%%%%%%%%%%%%%%%%%%%%%%%%%%%%%%%%%%%%%%%%%%%%%%%%%%%%%%%%%%%%%%%%%%%%%
\subsection{Forwarding}
\label{sec:forward}

Different versions of the main or child documents
using compilation flags as described in \secref{sec:flags}
can be (permanently) stored in different files
for convenient compilation, viewing and distribution.
To this end, the package defines a command
to pass on compilation to a different file:

%%%%%%%%%%%%%%%%%%%%%%%%%%%%%%%%%%%%%%%%
\DescribeMacro{\childdocforward}
The command |\childdocforward| redirects processing to
another source file:
%
\begin{center}
\begin{tabular}{l}
|\input{childdoc.def}|\\
|\childdocforward[|\textit{main}|]{|\textit{dest}|}|\\
\end{tabular}
\end{center}
%
The argument \textit{dest} is the destination file
(without extension).
It should be the main file or one of the child files.
Note that further \textsf{childdoc} directives
such as |\childdocof| and |\childdocforward|
in the indicated file will be processed in this form.
The optional argument \textit{main}
passes on directly to the main file \textit{main}
while pretending to compile the child \textit{dest}.
This form behaves as if \textit{dest}
issues |\childdocof{|\textit{main}|}| right away,
and no further \textsf{childdoc} directives will be processed.

%%%%%%%%%%%%%%%%%%%%%%%%%%%%%%%%%%%%%%%%
\DescribeMacro{\...prefix}
In the alternative form |\childdocforwardprefix|,
%
\begin{center}
\begin{tabular}{l}
|\input{childdoc.def}|\\
|\childdocforwardprefix[|\textit{main}|]{|\textit{prefix}|}{|\textit{dest}|}|
\end{tabular}
\end{center}
%
the destination file is determined by a pattern
depending on the current file:
To make this work, the current file must be called
`{\textit{prefix}\hspace{0.2em}\textit{suffix}}'
with \textit{prefix} matching precisely the argument.
Processing is then passed on to the file
`{\textit{dest}\hspace{0.2em}\textit{suffix}}'.
Surely, the same effect is achieved by
directly specifying the
argument `{\textit{dest}\hspace{0.2em}\textit{suffix}}'
in the first form.
However, that requires to set up a different file
for each child. With the alternative form of the command
all these files can have exactly the same content
which simplifies setting them up and maintaining them.

For example, the following file |draft.tex|
with a compilation flag |\version| as described in \secref{sec:flags}
compiles the main document as a draft:
%
\begin{center}
\begin{tabular}{l}
|\def\version{draft}|\\
|\input{childdoc.def}|\\
|\childdocforward{|\textit{main}|}|
\end{tabular}
\end{center}
%
Likewise, the following files |final|\textit{nn}|.tex|
compile the final version of the child document
|child|\textit{nn}|.tex|:
%
\begin{center}
\begin{tabular}{l}
|\def\version{final}|\\
|\input{childdoc.def}|\\
|\childdocforwardprefix{final}{child}|
\end{tabular}
\end{center}
%

Note that when several versions of a main file and/or of each child file
are to be generated, it may be convenient to set up a |Makefile| or
shell script to automatise the process.

%%%%%%%%%%%%%%%%%%%%%%%%%%%%%%%%%%%%%%%%%%%%%%%%%%%%%%%%%%%%%%%%%%%%%%%%%%%%%%%%
\subsection{Command Line Processing}
\label{sec:commandline}

The effect of redirection files can also be achieved by invoking
the \LaTeX{} compiler with a more elaborate command line.
Most conveniently this should be done as part
of a shell script or a |Makefile|.

When using \textsf{childdoc} in the main file, the following
command lines effectively perform a redirection
(note that depending on the shell being used,
backslashes may have to be doubled: `|\|' $\to$ `|\\|'):
%
\begin{center}
|... -jobname "|\textit{target}|" |\\|"|[\textit{flags}]%
|\input{childdoc.def}\childdocforward[|\textit{main}|]{|\textit{dest}|}"|
\end{center}
%
Here \textit{target} is the name of the output file,
\textit{main} is the name of the main file
and \textit{dest} is the name of the main or child file to be processed
(all filenames without extensions).
The optional argument \textit{main} can be omitted
if \textit{main} matches \textit{dest}.
Optionally, compilation \textit{flags} can be defined via |\def| commands.
This command line makes the \TeX{} engine believe
it is compiling the file \textit{target}
whose content is specified as the latter parameter.
The provided code then forwards the processing to
\textit{main} or \textit{dest} as described in \secref{sec:forward}.

%%%%%%%%%%%%%%%%%%%%%%%%%%%%%%%%%%%%%%%%%%%%%%%%%%%%%%%%%%%%%%%%%%%%%%%%%%%%%%%%
\subsection{Include by Input}
\label{sec:input}

Including child documents by |\include| has some restrictions by design.
Most notably, the content of a child document always occupies
its own set of pages; pages cannot be shared between child documents.
Usually, this behaviour makes perfect sense
because each child document contain an essential part of the document.
However, in some situations it may be desirable to compose
a document from a collection of parts
without having mandatory page breaks between then.
For this case, the package
provides a mechanism to include parts
by |\input| which can also be processed individually.
However, by construction this mechanism
requires manual handling of the content to be output.

%%%%%%%%%%%%%%%%%%%%%%%%%%%%%%%%%%%%%%%%
\DescribeMacro{\ifchilddocmanual}
The main file should be prepared as usual, see \secref{sec:include}.
However, the document body must make a distinction
between processing of an individual part and of the main document, e.g.:
%
\begin{center}
\begin{tabular}{l}
|\ifchilddocmanual|\\
|\input{\childdocname}|\\
|\||else|\\
\textit{document body with }|\input{|\textit{part}|}|\\
|\||fi|
\end{tabular}
\end{center}
%
The conditional |\ifchilddocmanual| is true whenever
a part to be included by |\input| is being compiled,
and the name of the part is stored in |\childdocname|.

%%%%%%%%%%%%%%%%%%%%%%%%%%%%%%%%%%%%%%%%
\DescribeMacro{\childdocby}
Each part to be included by |\input| should start with:
%
\begin{center}
\begin{tabular}{l}
|\input{childdoc.def}|\\
|\childdocby{|\textit{main}|}|\\
\end{tabular}
\end{center}
%
The directive |\childdocby| is similar to |\childdocof|
described in \secref{sec:include},
but the subsequent selection of content must be done manually.
To that end, both |\ifchilddoc| and |\ifchilddocmanual|
will be true upon processing of a part,
and the name of the part is stored in |\childdocname|.
Note that |\jobname| will be set to the filename of the current part
so that each part receives an individual |.aux| file
that does not interfere with the |.aux| file(s) of the main document.
This behaviour can be altered by the alternative form
|\childdocby[*]{|\textit{main}|}| (with a non-empty optional argument)
which uses the |.aux| file of the main document
by setting |\jobname| to \textit{main}.

%%%%%%%%%%%%%%%%%%%%%%%%%%%%%%%%%%%%%%%%%%%%%%%%%%%%%%%%%%%%%%%%%%%%%%%%%%%%%%%%
\subsection{Driver Development}
\label{sec:driver}

The \textsf{childdoc} mechanism can also be use for the development
of definition files such as \LaTeX{} styles or classes.
This case differs from the above setup with multiple parts
included by |\include| in that no |\includeonly| should be invoked.
This can be achieved by starting the include file
(before |\ProvidesPackage|) with:
%
\begin{center}
\begin{tabular}{l}
|\input{childdoc.def}|\\
|\childdocforward{|\textit{main}|}|\\
\end{tabular}
\end{center}
%
or alternatively with:
%
\begin{center}
\begin{tabular}{l}
|\input{childdoc.def}|\\
|\childdocby{|\textit{main}|}|\\
\end{tabular}
\end{center}
%
Both forms have slightly different effects as described above.
The main file is prepared as usual, see \secref{sec:include}.

%%%%%%%%%%%%%%%%%%%%%%%%%%%%%%%%%%%%%%%%%%%%%%%%%%%%%%%%%%%%%%%%%%%%%%%%%%%%%%%%
\subsection{Legacy Detection}
\label{sec:detection}

The directive |\childdocmain| in the main file can detect
whether the complete document or merely a child is to be compiled
even without using the directive |\childdocof|.
This method is deprecated because it is less robust
and there is no compelling reason to use it;
it is merely provided for backward compatibility
and it may be removed in future versions.

If the detection mechanism is to be used,
it is mandatory to correctly specify
the filename of the main file as the argument of |\childdocmain|:
%
\begin{center}
\begin{tabular}{l}
|\input{childdoc.def}|\\
|\childdocmain{|\textit{main}|}|\\
\end{tabular}
\end{center}
%
If |\jobname| does not match the argument \textit{main} of |\childdocmain|,
it is assumed that |\jobname| points to the child file to be compiled.
When using |\childdocmain| with the main file specified as argument,
it suffices to start a child file
with just |\input{|\textit{main}|}|
without loading of the package and using |\childdocof|.
If instead all processing is done
with the appropriate \textsf{childdoc} directives,
the argument of \textit{main} of |\childdocmain| can be empty.

An alternative version of the command line processing described
in \secref{sec:commandline} using the detection mechanism reads:
%
\begin{center}
|... -jobname "|\textit{target}|" "|[\textit{flags}]%
[|\def\jobname{|\textit{dest}|}|]|\input{|\textit{main}|}"|
\end{center}

%%%%%%%%%%%%%%%%%%%%%%%%%%%%%%%%%%%%%%%%%%%%%%%%%%%%%%%%%%%%%%%%%%%%%%%%%%%%%%%%
\subsection{Manual Code}
\label{sec:manual}

In case one cannot be certain whether the definitions file |childdoc.def|
is installed on the target \TeX{} distribution
and one prefers not to ship it,
it is conceivable to paste a few relevant commands into the sources.

To that end, drop all statements |\input{childdoc.def}|
and perform the replacements as outlined below.
Instead of |\childdocmain{|\textit{main}|}| add the following code
to the top of the main file:
%
\begin{center}
\begin{tabular}{l}
|\||ifdefined\childdocname\endinput\||fi\newif\ifchilddoc|\\
|\edef\childdocname{\scantokens\expandafter{\jobname\noexpand}}|\\
|\def\childdocmain{|\textit{main}|}\||ifx\childdocmain\childdocname\||else|\\
|\childdoctrue\includeonly{\childdocname}\let\jobname\childdocmain\||fi|\\
\end{tabular}
\end{center}
%
Instead of |\childdocof{|\textit{main}|}| just include the main file
at the top of each child file:
%
\begin{center}
|\input{|\textit{main}|}|
\end{center}
%
A simple redirection |\childdocforward{|\textit{dest}|}| is achieved by:
%
\begin{center}
|\def\jobname{|\textit{dest}|}\input{\jobname}|
\end{center}
%
The redirection with prefix
|\childdocforwardprefix[|\textit{prefix}|]{|\textit{dest}|}|
is accomplished by:
%
\begin{center}
\begin{tabular}{l}
|{\edef\jobname{\scantokens\expandafter{\jobname\noexpand}}|\\
|\def\redirectjob |\textit{prefix}|#1~~~{\gdef\jobname{|\textit{dest}|#1}}|\\
|\expandafter\redirectjob\jobname~~~}\input{\jobname}|
\end{tabular}
\end{center}

In an alternative approach,
child documents can be compiled by a specific command line
without additional code or specific definitions:
%
\begin{center}
|... -jobname "|\textit{target}|" "|[\textit{flags}]%
|\includeonly{|\textit{dest}|}\input{|\textit{main}|}"|
\end{center}
%

%%%%%%%%%%%%%%%%%%%%%%%%%%%%%%%%%%%%%%%%%%%%%%%%%%%%%%%%%%%%%%%%%%%%%%%%%%%%%%%%
%%%%%%%%%%%%%%%%%%%%%%%%%%%%%%%%%%%%%%%%%%%%%%%%%%%%%%%%%%%%%%%%%%%%%%%%%%%%%%%%
\section{Information}

%%%%%%%%%%%%%%%%%%%%%%%%%%%%%%%%%%%%%%%%%%%%%%%%%%%%%%%%%%%%%%%%%%%%%%%%%%%%%%%%
\subsection{Copyright}

Copyright \copyright{} 2017--2018 Niklas Beisert

This work may be distributed and/or modified under the
conditions of the \LaTeX{} Project Public License, either version 1.3
of this license or (at your option) any later version.
The latest version of this license is in
  \url{http://www.latex-project.org/lppl.txt}
and version 1.3 or later is part of all distributions of \LaTeX{}
version 2005/12/01 or later.

This work has the LPPL maintenance status `maintained'.

The Current Maintainer of this work is Niklas Beisert.

This work consists of the files |README.txt|, |childdoc.ins| and |childdoc.dtx|
as well as the derived files |childdoc.def|, |cdocsamp.tex|
with |cdocsch1.tex|, |cdocsch2.tex|, |cdocspt3.tex|, |cdocspt4.tex|,
|cdocsdrf.tex|, |cdocsfn1.tex|, |cdocsfn2.tex|
as well as |childdoc.pdf|.

%%%%%%%%%%%%%%%%%%%%%%%%%%%%%%%%%%%%%%%%%%%%%%%%%%%%%%%%%%%%%%%%%%%%%%%%%%%%%%%%
\subsection{Files and Installation}

The package consists of the files:
%
\begin{center}
\begin{tabular}{ll}
    |README.txt|   & readme file \\
    |childdoc.ins| & installation file \\
    |childdoc.dtx| & source file \\
    |childdoc.def| & definition file \\
    |cdocsamp.tex| & sample main file \\
    |cdocsch1.tex| & sample include file \\
    |cdocsch2.tex| & sample include file \\
    |cdocspt3.tex| & sample part file \\
    |cdocspt4.tex| & sample part file \\
    |cdocsdrf.tex| & sample redirection file \\
    |cdocsfn1.tex| & sample redirection file \\
    |cdocsfn2.tex| & sample redirection file \\
    |childdoc.pdf| & manual
\end{tabular}
\end{center}
%
The distribution consists of the files
|README.txt|, |childdoc.ins| and |childdoc.dtx|.
%
\begin{itemize}
\item
Run (pdf)\LaTeX{} on |childdoc.dtx|
to compile the manual |childdoc.pdf| (this file).
\item
Run \LaTeX{} on |childdoc.ins| to create the definitions file |childdoc.def|
and the sample |cdocsamp.tex| with include files
|cdocsch1.tex|, |cdocsch2.tex|, |cdocspt3.tex|, |cdocspt4.tex|,
|cdocsdrf.tex|, |cdocsfn1.tex|, |cdocsfn2.tex|.
Then copy the file |childdoc.def| to an appropriate directory of your \LaTeX{}
distribution, e.g.\ \textit{texmf-root}|/tex/latex/childdoc|.
\end{itemize}

%%%%%%%%%%%%%%%%%%%%%%%%%%%%%%%%%%%%%%%%%%%%%%%%%%%%%%%%%%%%%%%%%%%%%%%%%%%%%%%%
\subsection{Related CTAN Packages}

There are several other packages which offer a similar functionality:
%
\begin{itemize}
\item
The packages
\href{http://ctan.org/pkg/docmute}{\textsf{docmute}},
\href{http://ctan.org/pkg/includex}{\textsf{includex}} and
\href{http://ctan.org/pkg/standalone}{\textsf{standalone}}
provide commands to include only the document body of
a child file thus allowing both files to be compiled individually.
\item
The packages \href{http://ctan.org/pkg/subdocs}{\textsf{subdocs}}
and \href{http://ctan.org/pkg/subfiles}{\textsf{subfiles}}
provide structures in which the main and child documents can be
encapsulated and allowing them to be compiled individually.
The inclusion mechanism is different from the conventional |\include|.
\item
The package \href{http://ctan.org/pkg/combine}{\textsf{combine}}
is an elaborate solution to combine several documents into one.
\end{itemize}
%
See also the CTAN topic \href{http://ctan.org/topic/subdocs}{\textsf{subdocs}}
for further related packages.
The present package differs from the above solutions in that
a document structure constructed with the conventional |\include| mechanism
just needs two extra commands at the top of every file
such that all constituent files can be compiled individually.

%%%%%%%%%%%%%%%%%%%%%%%%%%%%%%%%%%%%%%%%%%%%%%%%%%%%%%%%%%%%%%%%%%%%%%%%%%%%%%%%
%\subsection{Feature Suggestions}
%
%The following is a list of features which may be useful for future
%versions of this package:
%%
%\begin{itemize}
%\item
%\ldots
%\end{itemize}

%%%%%%%%%%%%%%%%%%%%%%%%%%%%%%%%%%%%%%%%%%%%%%%%%%%%%%%%%%%%%%%%%%%%%%%%%%%%%%%%
\subsection{Revision History}

%%%%%%%%%%%%%%%%%%%%%%%%%%%%%%%%%%%%%%%%
\paragraph{v2.0:} 2018/12/30

\begin{itemize}
\item
immediate forward processing
\item
added |\childdocby| mechanism
\item
manual restructured
\end{itemize}

%%%%%%%%%%%%%%%%%%%%%%%%%%%%%%%%%%%%%%%%
\paragraph{v1.6:} 2018/01/17

\begin{itemize}
\item
application for development of include files
\item
corrections to manual
\end{itemize}

%%%%%%%%%%%%%%%%%%%%%%%%%%%%%%%%%%%%%%%%
\paragraph{v1.5:} 2017/05/21

\begin{itemize}
\item
more complete structuring introduced
\item
|\childdocof| introduced
\item
|\childdoc| renamed to |\childdocmain|
\item
|\childredirect| renamed to |\childdocforward| and |\childdocforwardprefix|
and functionality expanded
\end{itemize}

%%%%%%%%%%%%%%%%%%%%%%%%%%%%%%%%%%%%%%%%
\paragraph{v1.0:} 2017/04/27

\begin{itemize}
\item
manual and install package
\item
first version published on CTAN
\end{itemize}

%%%%%%%%%%%%%%%%%%%%%%%%%%%%%%%%%%%%%%%%
\paragraph{v0.6:} 2017/04/26

\begin{itemize}
\item
redirection mechanism added
\end{itemize}

%%%%%%%%%%%%%%%%%%%%%%%%%%%%%%%%%%%%%%%%
\paragraph{v0.5:} 2017/04/26

\begin{itemize}
\item
functionality in definition file
\end{itemize}


%%%%%%%%%%%%%%%%%%%%%%%%%%%%%%%%%%%%%%%%%%%%%%%%%%%%%%%%%%%%%%%%%%%%%%%%%%%%%%%%
%%%%%%%%%%%%%%%%%%%%%%%%%%%%%%%%%%%%%%%%%%%%%%%%%%%%%%%%%%%%%%%%%%%%%%%%%%%%%%%%
%%%%%%%%%%%%%%%%%%%%%%%%%%%%%%%%%%%%%%%%%%%%%%%%%%%%%%%%%%%%%%%%%%%%%%%%%%%%%%%%
\appendix

\settowidth\MacroIndent{\rmfamily\scriptsize 000\ }

 \DocInput{childdoc.dtx}

\end{document}
%</driver>
% \fi
%
% %%%%%%%%%%%%%%%%%%%%%%%%%%%%%%%%%%%%%%%%%%%%%%%%%%%%%%%%%%%%%%%%%%%%%%%%%%%%%%
% %%%%%%%%%%%%%%%%%%%%%%%%%%%%%%%%%%%%%%%%%%%%%%%%%%%%%%%%%%%%%%%%%%%%%%%%%%%%%%
% \section{Sample}
%\iffalse
%<*samplemain>
%\fi
%
% The following presents a sample document
% with two chapters, two parts, a title page,
% a compile flag as well as three forwarding files to set the flag.
% It consists of eight |.tex| files:
% \begin{center}
% \begin{tabular}{ll}
% |cdocsamp.tex|&main file\\
% |cdocsch1.tex|&include file for chapter 1\\
% |cdocsch2.tex|&include file for chapter 2\\
% |cdocspt3.tex|&include file for part 3\\
% |cdocspt4.tex|&include file for part 4\\
% |cdocsdrf.tex|&forwarding file for main file in draft mode\\
% |cdocsfi1.tex|&forwarding file for final version of chapter 1\\
% |cdocsfi2.tex|&forwarding file for final version of chapter 2\\
% \end{tabular}
% \end{center}
% Each of the eight files can be compiled directly by the \LaTeX{} compiler.
%
% %%%%%%%%%%%%%%%%%%%%%%%%%%%%%%%%%%%%%%
% \paragraph{Main File.}
%
% The main file is called |cdocsamp.tex|.
%
% Load the \textsf{childdoc} definitions and
% declare the filename for the main document:
%    \begin{macrocode}
\input{childdoc.def}
\childdocmain{}
%    \end{macrocode}

% Optional override for |\version| flag:
%    \begin{macrocode}
%%\ifchilddoc\else\providecommand{\version}{draft}\fi
%    \end{macrocode}

% Define the default values for the |\version| flag
% (|final| for the main file and |draft| for childs):
%    \begin{macrocode}
\ifchilddoc
\providecommand{\version}{draft}
\else
\providecommand{\version}{final}
\fi
%    \end{macrocode}

% Load the standard document class:
%    \begin{macrocode}
\documentclass[12pt]{article}
%    \end{macrocode}

% Start the document body:
%    \begin{macrocode}
\begin{document}
%    \end{macrocode}

% Declare a title page.
% Print title, part of document being processed and version flag:
%    \begin{macrocode}
\addtocounter{page}{-1}
\begin{center}
{\LARGE\bfseries{}childdoc example\par}
\vspace{1cm}
\ifchilddoc
\ifchilddocmanual part\else chapter\fi:
`\childdocname' of `\childdocjob'\par
\else
main document: `\childdocjob'\par
\fi
version: \version\par
\end{center}
\newpage
%    \end{macrocode}

% Manually include selected file,
% otherwise process as usual:
%    \begin{macrocode}
\ifchilddocmanual
\section*{part `\childdocname'}
\input{\childdocname}
\else
%    \end{macrocode}

% Include the two chapters:
%    \begin{macrocode}
\include{cdocsch1}
\include{cdocsch2}
%    \end{macrocode}

% Include the two parts unless only chapters should be displayed:
%    \begin{macrocode}
\ifchilddoc\else
\section{part three}
\input{cdocspt3}
\section{part four}
\input{cdocspt4}
\fi
%    \end{macrocode}

% Process as usual until here:
%    \begin{macrocode}
\fi
%    \end{macrocode}

% End of document body:
%    \begin{macrocode}
\end{document}
%    \end{macrocode}
%\iffalse
%</samplemain>
%\fi
%
% %%%%%%%%%%%%%%%%%%%%%%%%%%%%%%%%%%%%%%
% \paragraph{Chapter Include Files.}
%
% The include files are called |cdocsch1.tex| and |cdocsch2.tex|.
%
%\iffalse
%<*samplechap1|samplechap2>
%\fi

% Optional override for |\version| flag:
%    \begin{macrocode}
%%\providecommand{\version}{final}
%    \end{macrocode}

% Include the main document:
%    \begin{macrocode}
\input{childdoc.def}
\childdocof{cdocsamp}
%    \end{macrocode}

%\iffalse
%</samplechap1|samplechap2>
%\fi
%
%\iffalse
%<*samplechap1>
%\fi
% Some text for chapter 1:
%    \begin{macrocode}
\section{one}
some text in chapter one
%    \end{macrocode}

%\iffalse
%</samplechap1>
%\fi
% Some text for chapter 2:
%\iffalse
%<*samplechap2>
%\fi
%    \begin{macrocode}
\section{two}
more text in chapter two
%    \end{macrocode}

%\iffalse
%</samplechap2>
%\fi
%
% %%%%%%%%%%%%%%%%%%%%%%%%%%%%%%%%%%%%%%
% \paragraph{Part Include Files.}
%
% The include files are called |cdocspt3.tex| and |cdocspt4.tex|.
%
%\iffalse
%<*samplepart3|samplepart4>
%\fi

% Optional override for |\version| flag:
%    \begin{macrocode}
%%\providecommand{\version}{final}
%    \end{macrocode}

% Include the main document:
%    \begin{macrocode}
\input{childdoc.def}
\childdocby{cdocsamp}
%    \end{macrocode}

%\iffalse
%</samplepart3|samplepart4>
%\fi
%
%\iffalse
%<*samplepart3>
%\fi
% Some text for part 3:
%    \begin{macrocode}
some text in part three
%    \end{macrocode}

%\iffalse
%</samplepart3>
%\fi
% Some text for part 4:
%\iffalse
%<*samplepart4>
%\fi
%    \begin{macrocode}
more text in part four
%    \end{macrocode}

%\iffalse
%</samplepart4>
%\fi
%
% %%%%%%%%%%%%%%%%%%%%%%%%%%%%%%%%%%%%%%
% \paragraph{Forwarding for a Complete Draft.}
%
% The following forwarding file |cdocsdrf.tex|
% compiles the main document in draft mode:
%\iffalse
%<*sampledraft>
%\fi
%    \begin{macrocode}
\def\version{draft}
\input{childdoc.def}
\childdocforward{cdocsamp}
%    \end{macrocode}

%\iffalse
%</sampledraft>
%\fi
%
% %%%%%%%%%%%%%%%%%%%%%%%%%%%%%%%%%%%%%%
% \paragraph{Forwarding for Final Version of the Chapters.}
%
% The following forwarding files |cdocsfn1.tex| and |cdocsfn2.tex|
% (with identical content)
% compile the final versions of the child documents
% |cdocsch1.tex| and |cdocsch2.tex|, respectively:
%\iffalse
%<*samplefinal>
%\fi
%    \begin{macrocode}
\def\version{final}
\input{childdoc.def}
\childdocforwardprefix[cdocsamp]{cdocsfn}{cdocsch}
%    \end{macrocode}

%\iffalse
%</samplefinal>
%\fi
%
% %%%%%%%%%%%%%%%%%%%%%%%%%%%%%%%%%%%%%%
% \paragraph{Command Line Processing.}
%
% The following three command lines generate the output files
% |cdocscld|, |cdocscl1| and |cdocscl2|
% which should be identical to
% |cdocsdrf|, |cdocsch1| and |cdocsfn2|, respectively:
% \begin{center}
% \begin{tabular}{l}
% |latex -jobname cdocscld \|\\
% |  "\def\version{draft}\input{childdoc.def}\childdocforward{cdocsamp}"|\\
% |latex -jobname cdocscl1 \|\\
% |  "\input{childdoc.def}\childdocforward[cdocsamp]{cdocsch1}"|\\
% |latex -jobname cdocscl2 \|\\
% |  "\def\version{final}\input{childdoc.def}\childdocforward{cdocsch2}"|
% \end{tabular}
% \end{center}
% Note that the trailing backslash on each first line
% merely continues the input to the second line
% (for convenient cut ant paste).
% Furthermore, the command |latex| can be replaced by any
% of its alternative versions such as |pdflatex|.
%
% %%%%%%%%%%%%%%%%%%%%%%%%%%%%%%%%%%%%%%%%%%%%%%%%%%%%%%%%%%%%%%%%%%%%%%%%%%%%%%
% %%%%%%%%%%%%%%%%%%%%%%%%%%%%%%%%%%%%%%%%%%%%%%%%%%%%%%%%%%%%%%%%%%%%%%%%%%%%%%
% \section{Implementation}
%\iffalse
%<*package>
%\fi
%
% This section describes the definitions file |childdoc.def|.

% The definitions cannot be loaded using |\usepackage| or |\RequirePackage|
% which has a mechanism to prevent loading a style file more than once.
% When loading the definitions by means of |\input|
% multiple instances have to be prevented manually:
%\iffalse
%This code needs to be before the `\ProvidesFile' directive
%which is defined at the beginning of this file.
%Therefore it is also placed there and commented out here.
%</package>
%<*discard>
%\fi
%    \begin{macrocode}
\ifdefined\childdocmain\endinput\fi
%    \end{macrocode}
%\iffalse
%</discard>
%<*package>
%\fi
%
% \macro{\ifchilddoc}
% \macro{\ifchilddocmanual}
% The conditional |\ifchilddoc| tells whether a
% child (true) or main (false) document is being compiled.
% The conditional |\ifchilddocmanual| tells whether
% the |\includeonly| mechanism is used (false) or
% the selection of child files must be performed manually (true).
% The definitions initialise to false:
%    \begin{macrocode}
\newif\ifchilddoc
\newif\ifchilddocmanual
%    \end{macrocode}

% \macro{\childdocname}
% \macro{\childdocjob}
% The macro |\childdocname| stores the name of the main document
% to be compiled. The macro |\childdocjob| stores the name of
% the document on which the \LaTeX{} compiler was originally invoked.
% The content of |\jobname| cannot be compared
% to filenames specified in the source due to different catcodes.
% The following code rescans |\jobname|, stores the result
% in |\childdocname| and saves a copy in |\childdocjob|:
%    \begin{macrocode}
\edef\childdocname{\scantokens\expandafter{\jobname\noexpand}}
\let\childdocjob\childdocname
%    \end{macrocode}

% \macro{\childdocdisable}
% The macro |\childdocdisable| prevents the main file
% from being processed more than once.
% At this stage, the main document command |\childdocmain|
% is assumed to be called once again where it should do nothing.
% Any subsequent call to it should prevent
% a secondary processing of the main document
% It overwrites the forwarding commands
% |\childdocof| and |\childdocforward|
% with empty macros to prevent further inclusions of the main document:
%    \begin{macrocode}
\newcommand{\childdocdisable}
{
  \renewcommand{\childdocmain}[1]{\renewcommand{\childdocmain}[1]{\endinput}}
  \renewcommand{\childdocof}[1]{}
  \renewcommand{\childdocby}[2][]{}
  \renewcommand{\childdocforward}[2][]{}
  \renewcommand{\childdocdisable}{}
}
%    \end{macrocode}

% \macro{\childdocmain}
% The macro |\childdocmain| is to be called at the top of the main file
% with nothing or the main filename (without extension) as argument.
% First, it breaks loops.
% If the argument is not empty and does not match |\childdocname|
% (which is set by the first inclusion of |childdoc.def|),
% |\ifchilddoc| is set to true, |\includeonly| is applied to the child file
% and |\jobname| is set to the main file
% (for proper handling of |.aux| files):
%    \begin{macrocode}
\newcommand{\childdocmain}[1]
{
  \childdocdisable\childdocmain{}
  \if?#1?\else
    \begingroup
      \def\childdoctmp{#1}
      \ifx\childdoctmp\childdocname
        \def\childdoctmp{}
      \else
        \def\childdoctmp
        {
          \childdoctrue
          \includeonly{\childdocname}
          \def\childdocjob{#1}
          \def\jobname{#1}
        }
      \fi
      \expandafter
    \endgroup
    \childdoctmp
  \fi
}
%    \end{macrocode}

% \macro{\childdocof}
% The command |\childdocof| redirects
% compilation to the main file |#1|.
%    \begin{macrocode}
\newcommand{\childdocof}[1]
{
  \childdocdisable
  \childdoctrue
  \includeonly{\childdocname}
  \def\jobname{#1}
  \def\childdocjob{#1}
  \input{#1}
}
%    \end{macrocode}

% \macro{\childdocby}
% The command |\childdocby| ....
%    \begin{macrocode}
\newcommand{\childdocby}[2][]
{
  \childdocdisable
  \childdoctrue
  \childdocmanualtrue
  \if?#1?\else
    \def\jobname{#2}
  \fi
  \def\childdocjob{#2}
  \input{#2}
  \endinput
}
%    \end{macrocode}

% \macro{\childdocforward}
% The command |\childdocforward| redirects
% compilation to the main file or
% (if the optional argument is given) a child file.
% Parameters are set as if the main file
% or a child file starting with |\childdocof| was compiled.
% Then compilation is handed over to the main file:
%    \begin{macrocode}
\newcommand{\childdocforward}[2][]
{
  \begingroup
    \if?#1?
      \def\childdoctmp
      {
        \def\childdocname{#2}
        \def\childdocjob{#2}
        \def\jobname{#2}
        \input{#2}
        \endinput
      }
    \else
      \def\childdoctmp
      {
        \childdocdisable
        \def\childdocname{#2}
        \childdoctrue
        \includeonly{#2}
        \def\childdocjob{#1}
        \def\jobname{#1}
        \input{#1}
        \endinput
      }
    \fi
    \expandafter
  \endgroup
  \childdoctmp
}
%    \end{macrocode}

% \macro{\childdocforwardprefix}
% The command |\childdocforwardprefix| redirects
% compilation to the main or a child file by means of a pattern.
% The prefix |#1| in the current filename is replaced by |#2|
% and the suffix of the current filename is kept
% (it is assumed that the filename does not contain the substring `|~~~|'
% which is used as a delimiter).
% Compilation is handed over to the new file by |\childdocforward|:
%    \begin{macrocode}
\newcommand{\childdocforwardprefix}[3][]
{
  \begingroup
    \def\childdocextract #2##1~~~{\def\childdoctmp{\childdocforward[#1]{#3##1}}}
    \expandafter\childdocextract\childdocname~~~
    \expandafter
  \endgroup
  \childdoctmp
}
%    \end{macrocode}

% \macro{\childdoc}
% The deprecated macro |\childdoc| is a legacy version of |\childdocmain|:
%    \begin{macrocode}
\newcommand{\childdoc}{\childdocmain}
%    \end{macrocode}

% \macro{\childdocredirect}
% The deprecated macro |\childdocredirect| is a legacy version
% of |\childdocforward| and |\childdocforwardprefix|:
%    \begin{macrocode}
\newcommand{\childdocredirect}[2][]
{
  \begingroup
    \if?#1?
      \def\childdoctmp{\childdocforward{#2}}
    \else
      \def\childdoctmp{\childdocforwardprefix{#1}{#2}}
    \fi
    \expandafter
  \endgroup
  \childdoctmp
}
%    \end{macrocode}

%\iffalse
%</package>
%\fi
%
\endinput

\childdocforwardprefix[cdocsamp]{cdocsfn}{cdocsch}
%    \end{macrocode}

%\iffalse
%</samplefinal>
%\fi
%
% %%%%%%%%%%%%%%%%%%%%%%%%%%%%%%%%%%%%%%
% \paragraph{Command Line Processing.}
%
% The following three command lines generate the output files
% |cdocscld|, |cdocscl1| and |cdocscl2|
% which should be identical to
% |cdocsdrf|, |cdocsch1| and |cdocsfn2|, respectively:
% \begin{center}
% \begin{tabular}{l}
% |latex -jobname cdocscld \|\\
% |  "\def\version{draft}% \iffalse
%
% childdoc.dtx Copyright (C) 2017-2018 Niklas Beisert
%
% This work may be distributed and/or modified under the
% conditions of the LaTeX Project Public License, either version 1.3
% of this license or (at your option) any later version.
% The latest version of this license is in
%   http://www.latex-project.org/lppl.txt
% and version 1.3 or later is part of all distributions of LaTeX
% version 2005/12/01 or later.
%
% This work has the LPPL maintenance status `maintained'.
%
% The Current Maintainer of this work is Niklas Beisert.
%
% This work consists of the files childdoc.dtx and childdoc.ins
% and the derived files childdoc.def and cdocsamp.tex with
% cdocsch1.tex, cdocsch2.tex, cdocsdrf.tex, cdocsfn1.tex, cdocsfn2.tex.
%
%<package>\ifdefined\childdocmain\endinput\fi
%<package>\ProvidesFile{childdoc.def}[2018/12/30 v2.0 child document driver]
%<samplemain>\ProvidesFile{cdocsamp.tex}[2018/12/30 v2.0 sample for childdoc]
%<*driver>
%\ProvidesFile{childdoc.drv}[2018/12/30 v2.0 childdoc reference manual file]
\PassOptionsToClass{10pt,a4paper}{article}
\documentclass{ltxdoc}

\usepackage[margin=35mm]{geometry}
\usepackage{hyperref}
\usepackage{hyperxmp}
\usepackage[usenames]{color}

\hypersetup{colorlinks=true}
\hypersetup{pdfstartview=FitH}
\hypersetup{pdfpagemode=UseNone}
\hypersetup{pdfsource={}}
\hypersetup{pdflang={en-UK}}
\hypersetup{pdfcopyright={Copyright 2017-2018 Niklas Beisert.
  This work may be distributed and/or modified under the
  conditions of the LaTeX Project Public License, either version 1.3
  of this license or (at your option) any later version.}}
\hypersetup{pdflicenseurl={http://www.latex-project.org/lppl.txt}}
\hypersetup{pdfcontactaddress={ETH Zurich, ITP, HIT K,
  Wolfgang-Pauli-Strasse 27}}
\hypersetup{pdfcontactpostcode={8093}}
\hypersetup{pdfcontactcity={Zurich}}
\hypersetup{pdfcontactcountry={Switzerland}}
\hypersetup{pdfcontactemail={nbeisert@itp.phys.ethz.ch}}
\hypersetup{pdfcontacturl={http://people.phys.ethz.ch/\xmptilde nbeisert/}}

\newcommand{\secref}[1]{\hyperref[#1]{section \ref*{#1}}}

\parskip1ex
\parindent0pt
\let\olditemize\itemize
\def\itemize{\olditemize\parskip0pt}

\begin{document}

\title{The \textsf{childdoc} Package}
\hypersetup{pdftitle={The childdoc Package}}
\author{Niklas Beisert\\[2ex]
  Institut f\"ur Theoretische Physik\\
  Eidgen\"ossische Technische Hochschule Z\"urich\\
  Wolfgang-Pauli-Strasse 27, 8093 Z\"urich, Switzerland\\[1ex]
  \href{mailto:nbeisert@itp.phys.ethz.ch}
  {\texttt{nbeisert@itp.phys.ethz.ch}}}
\hypersetup{pdfauthor={Niklas Beisert}}
\hypersetup{pdfsubject={Manual for the LaTeX2e Package childdoc}}
\date{30 December 2018, \textsf{v2.0}}
\maketitle

\begin{abstract}\noindent
\textsf{childdoc} is a \LaTeXe{} package
that enables the direct compilation
of document sections included by |\include|
to individual files.
\end{abstract}

\begingroup
\parskip0ex
\tableofcontents
\endgroup

%%%%%%%%%%%%%%%%%%%%%%%%%%%%%%%%%%%%%%%%%%%%%%%%%%%%%%%%%%%%%%%%%%%%%%%%%%%%%%%%
%%%%%%%%%%%%%%%%%%%%%%%%%%%%%%%%%%%%%%%%%%%%%%%%%%%%%%%%%%%%%%%%%%%%%%%%%%%%%%%%
\section{Introduction}

\LaTeX{} provides a mechanism to structure a large document (such as a book)
into a main file and several child files (containing the chapters)
using the |\include| command.
This mechanism is beneficial for documents
which span hundreds of pages in order to
make the source file(s) more manageable.
Moreover, compilation can be restricted to
selected child files by means of the |\includeonly| command.
The latter feature can be used to reduce the compilation time while editing
(this was significantly more useful in the earlier days of \LaTeX{})
or to generate a smaller document which is easier to navigate.
Another application of |\includeonly| is to generate
documents consisting of selected parts of the complete document.

However, there are a few drawbacks of the plain |\include| mechanism:
\begin{itemize}
\item
The child files cannot be compiled on their own,
they can only be compiled via the main file.
A naive editing environment
(such as a text editor with an option
to have the current file processed by \LaTeX)
may require one to switch to the main file before compiling;
attempting to compile the child file produces errors.
\item
The main file must be modified (each time)
to adjust the |\includeonly| command
to the present needs. This easily leaves the main file in a messy state.
\item
The generated document will always carry the filename
of the main document. This is inconvenient if
several child files are to be compiled and
to be kept for distribution.
\end{itemize}

The present package provides a simple interface
to make child files individually compilable by \LaTeX{}.
Compiling a child file then has the same effect as compiling
the main file with an |\includeonly| command
to select the appropriate child.
Moreover the generated document will carry the name of the child
rather than the main file.
This resolves all three above issues.

This feature is meant to make the editing of books,
thesis documents and lecture notes somewhat more convenient.
However, the package can also be used efficiently for
composing a series of documents (such as exercise sheets)
which are typically distributed individually.
It then assists the author in generating the individual documents
(potentially in different versions)
as well as a document containing the collected series.
Another application is in developing style files
or other kinds of included material
where compilation of the style file could redirect
to a sample or test file.

%%%%%%%%%%%%%%%%%%%%%%%%%%%%%%%%%%%%%%%%%%%%%%%%%%%%%%%%%%%%%%%%%%%%%%%%%%%%%%%%
%%%%%%%%%%%%%%%%%%%%%%%%%%%%%%%%%%%%%%%%%%%%%%%%%%%%%%%%%%%%%%%%%%%%%%%%%%%%%%%%
\section{Usage}

First of all, the package \textsf{childdoc} is \emph{not} a standard
\LaTeXe{} |.sty| style file! Therefore it needs to be invoked in
a non-standard way.

%%%%%%%%%%%%%%%%%%%%%%%%%%%%%%%%%%%%%%%%%%%%%%%%%%%%%%%%%%%%%%%%%%%%%%%%%%%%%%%%
\subsection{Included Files}
\label{sec:include}

%%%%%%%%%%%%%%%%%%%%%%%%%%%%%%%%%%%%%%%%
\DescribeMacro{\childdocmain}
To use the package, add the commands
\begin{center}
\begin{tabular}{l}
|\input{childdoc.def}|\\
|\childdocmain{}|\\
\end{tabular}
\end{center}
at the very top of the main \LaTeX{} file,
in particular \emph{before} the |\documentclass| statement!
The argument of |\childdocmain| should be left empty
(but it must be present).

%%%%%%%%%%%%%%%%%%%%%%%%%%%%%%%%%%%%%%%%
\DescribeMacro{\childdocof}
Furthermore, add the commands
\begin{center}
\begin{tabular}{l}
|\input{childdoc.def}|\\
|\childdocof{|\textit{main}|}|\\
\end{tabular}
\end{center}
at the top of every child file \textit{child}
which is included by |\include{|\textit{child}|}|
from within the main file
(or at least for those files to be compiled individually).
The argument \textit{main} must be the filename of the main file.

There are a couple of
considerations in setting up the main and child documents:

%%%%%%%%%%%%%%%%%%%%%%%%%%%%%%%%%%%%%%%%
\paragraph{Restrictions.}

Please note the following restrictions:
\begin{itemize}
\item
|\childdocmain| must be called with one argument \textit{main}
to ensure compatibility with earlier version of the package.
It must either be empty (|\childdocmain{}|)
or precisely match the filename of the main file in which it is specified.
See \secref{sec:detection} for further information.
\item
The filename \textit{main} must be specified without the |.tex| extension.
\item
The filename \textit{main} is case sensitive
(even in case-insensitive file systems)
due to internal string comparison.
\item
The argument \textit{main} should be fully expanded, it cannot be a macro.
\item
Subdirectories and special characters should be avoided in filenames.
\item
The command |\childdocmain{|\textit{main}|}| must be followed by a whitespace.
It should not be followed immediately by another command
or by a comment mark `|%|'.
This is because the \TeX{} parser reads the token immediately following
the argument of |\childdocmain| and puts it
at the beginning of every child section;
however, a white\-space is ignored.
\end{itemize}

%%%%%%%%%%%%%%%%%%%%%%%%%%%%%%%%%%%%%%%%
\paragraph{Content of Main File.}

It is advisable to place all content in the child files included by |\include|.
Any output contained in the main file will appear in all child documents
unless suppressed manually;
it cannot be suppressed automatically by the |\includeonly| directive
and thus should normally be avoided.
A method to include some content in the main file
by means of conditional processing is described in \secref{sec:conditional}.

%%%%%%%%%%%%%%%%%%%%%%%%%%%%%%%%%%%%%%%%
\paragraph{Page Numbering.}

When only a part of the document is compiled,
the appropriate numbering of pages
(as well as other status parameters)
is determined from the |.aux| files.
The latter contain information from previous passes.
However this information needs to propagate through
all intermediate child documents.
Therefore the page numbering in child documents may well
be inconsistent until the complete document is compiled at least once.

A useful (if unconventional) way to always ensure a consistent
page numbering is to restart the numbering in each child document
and denote the pages by `\textit{child}|.|\textit{page}'
where \textit{child} represents the chapter/section number of the child file.
This can be achieved by the command
|\numberwithin{page}{|\textit{child}|}|
of the \textsf{amsmath} package
where \textit{child} can be |chapter| or |section|
depending on the chosen structuring.
Alternatively, one can modify the macro |\thepage| appropriately
and reset the counter |page| at the start of each child file.

%%%%%%%%%%%%%%%%%%%%%%%%%%%%%%%%%%%%%%%%%%%%%%%%%%%%%%%%%%%%%%%%%%%%%%%%%%%%%%%%
\subsection{Conditional Processing}
\label{sec:conditional}

The package provides a mechanism to compile different versions
of a document. To customise the versions further some conditional processing
can come in handy to distinguish which version is being compiled.
The package provides two macros to describe the compilation context:

%%%%%%%%%%%%%%%%%%%%%%%%%%%%%%%%%%%%%%%%
\DescribeMacro{\ifchilddoc}
The conditional |\ifchilddoc| distinguishes between the compilation of
child documents and the main document:
%
\begin{center}
|\ifchilddoc |\textit{child-code}| |[|\||else |\textit{main-code}]| \||fi|
\end{center}

%%%%%%%%%%%%%%%%%%%%%%%%%%%%%%%%%%%%%%%%
\DescribeMacro{\childdocname}
\DescribeMacro{\childdocjob}
The macro |\childdocname| contains the filename (without extension)
of the main or child file being processed.
Note that |\childdocjob| will always contain the name of the main file.

%%%%%%%%%%%%%%%%%%%%%%%%%%%%%%%%%%%%%%%%
\paragraph{Title Page.}

Conditional processing can be used to include a title or banner page
in the main document when proper precautions are taken.
Importantly, the code in the main file should ensure that the page counter
(as well as other status parameters which are stored in the |.aux| files)
takes the same value after the conditional processing.
Otherwise the page numbers may take divergent values
depending on which part is compiled.

For example, a title page could be declared by:
%
\begin{center}
\begin{tabular}{l}
|\ifchilddoc\||else|\\
|\addtocounter{page}{-1}|\\
\textit{code for title page}\\
|\newpage|\\
|\||fi|
\end{tabular}
\end{center}
%
A banner page for the child documents can be generated by:
%
\begin{center}
\begin{tabular}{l}
|\ifchilddoc|\\
|\addtocounter{page}{-1}|\\
\textit{code for banner page}\\
|\newpage|\\
|\||fi|
\end{tabular}
\end{center}
%
Here one could write a message such as:
\begin{center}
|This is the part \childdocname{} of \childdocjob{}.|
\end{center}

%%%%%%%%%%%%%%%%%%%%%%%%%%%%%%%%%%%%%%%%%%%%%%%%%%%%%%%%%%%%%%%%%%%%%%%%%%%%%%%%
\subsection{Flags}
\label{sec:flags}

The package makes it easy to generate different versions
of the main or child documents.
To this end compilation flags can be defined
and assigned different default values.
They will be particularly useful in conjunction
with the forwarding mechanism described in \secref{sec:forward}.

For example, it may be useful to have a flag |\version|
which can be set to |draft| or |final|.
The document source will contain some conditional code
depending on the value of |\version|.
Suppose further, the flag should default to |final| for the main file
and to |draft| for child files
which is a natural assignment for editing the document.
This is achieved by placing the following code
in the preamble of the main document
(below the |\childdocmain| directive):
%
\begin{center}
\begin{tabular}{l}
|\ifchilddoc|\\
|\providecommand{\version}{draft}|\\
|\||else|\\
|\providecommand{\version}{final}|\\
|\||fi|
\end{tabular}
\end{center}
%
The definition by |\providecommand| makes sure
that previous definitions are not overwritten.
Further statements |\providecommand{\version}{...}|
can thus be added before the above code to override it.

For the main file, one might add a line
(between |\childdocmain| and the above block)
%
\begin{center}
|%\ifchilddoc\||else\providecommand{\version}{draft}\||fi|
\end{center}
%
which can be uncommented to produce a draft version.
Likewise one can add a line to the very top of a child file
(above the |\childdocof{|\textit{main}|}| directive)
%
\begin{center}
|%\providecommand{\version}{final}|
\end{center}
%
which can be uncommented to produce the final version of this child document.

%%%%%%%%%%%%%%%%%%%%%%%%%%%%%%%%%%%%%%%%%%%%%%%%%%%%%%%%%%%%%%%%%%%%%%%%%%%%%%%%
\subsection{Forwarding}
\label{sec:forward}

Different versions of the main or child documents
using compilation flags as described in \secref{sec:flags}
can be (permanently) stored in different files
for convenient compilation, viewing and distribution.
To this end, the package defines a command
to pass on compilation to a different file:

%%%%%%%%%%%%%%%%%%%%%%%%%%%%%%%%%%%%%%%%
\DescribeMacro{\childdocforward}
The command |\childdocforward| redirects processing to
another source file:
%
\begin{center}
\begin{tabular}{l}
|\input{childdoc.def}|\\
|\childdocforward[|\textit{main}|]{|\textit{dest}|}|\\
\end{tabular}
\end{center}
%
The argument \textit{dest} is the destination file
(without extension).
It should be the main file or one of the child files.
Note that further \textsf{childdoc} directives
such as |\childdocof| and |\childdocforward|
in the indicated file will be processed in this form.
The optional argument \textit{main}
passes on directly to the main file \textit{main}
while pretending to compile the child \textit{dest}.
This form behaves as if \textit{dest}
issues |\childdocof{|\textit{main}|}| right away,
and no further \textsf{childdoc} directives will be processed.

%%%%%%%%%%%%%%%%%%%%%%%%%%%%%%%%%%%%%%%%
\DescribeMacro{\...prefix}
In the alternative form |\childdocforwardprefix|,
%
\begin{center}
\begin{tabular}{l}
|\input{childdoc.def}|\\
|\childdocforwardprefix[|\textit{main}|]{|\textit{prefix}|}{|\textit{dest}|}|
\end{tabular}
\end{center}
%
the destination file is determined by a pattern
depending on the current file:
To make this work, the current file must be called
`{\textit{prefix}\hspace{0.2em}\textit{suffix}}'
with \textit{prefix} matching precisely the argument.
Processing is then passed on to the file
`{\textit{dest}\hspace{0.2em}\textit{suffix}}'.
Surely, the same effect is achieved by
directly specifying the
argument `{\textit{dest}\hspace{0.2em}\textit{suffix}}'
in the first form.
However, that requires to set up a different file
for each child. With the alternative form of the command
all these files can have exactly the same content
which simplifies setting them up and maintaining them.

For example, the following file |draft.tex|
with a compilation flag |\version| as described in \secref{sec:flags}
compiles the main document as a draft:
%
\begin{center}
\begin{tabular}{l}
|\def\version{draft}|\\
|\input{childdoc.def}|\\
|\childdocforward{|\textit{main}|}|
\end{tabular}
\end{center}
%
Likewise, the following files |final|\textit{nn}|.tex|
compile the final version of the child document
|child|\textit{nn}|.tex|:
%
\begin{center}
\begin{tabular}{l}
|\def\version{final}|\\
|\input{childdoc.def}|\\
|\childdocforwardprefix{final}{child}|
\end{tabular}
\end{center}
%

Note that when several versions of a main file and/or of each child file
are to be generated, it may be convenient to set up a |Makefile| or
shell script to automatise the process.

%%%%%%%%%%%%%%%%%%%%%%%%%%%%%%%%%%%%%%%%%%%%%%%%%%%%%%%%%%%%%%%%%%%%%%%%%%%%%%%%
\subsection{Command Line Processing}
\label{sec:commandline}

The effect of redirection files can also be achieved by invoking
the \LaTeX{} compiler with a more elaborate command line.
Most conveniently this should be done as part
of a shell script or a |Makefile|.

When using \textsf{childdoc} in the main file, the following
command lines effectively perform a redirection
(note that depending on the shell being used,
backslashes may have to be doubled: `|\|' $\to$ `|\\|'):
%
\begin{center}
|... -jobname "|\textit{target}|" |\\|"|[\textit{flags}]%
|\input{childdoc.def}\childdocforward[|\textit{main}|]{|\textit{dest}|}"|
\end{center}
%
Here \textit{target} is the name of the output file,
\textit{main} is the name of the main file
and \textit{dest} is the name of the main or child file to be processed
(all filenames without extensions).
The optional argument \textit{main} can be omitted
if \textit{main} matches \textit{dest}.
Optionally, compilation \textit{flags} can be defined via |\def| commands.
This command line makes the \TeX{} engine believe
it is compiling the file \textit{target}
whose content is specified as the latter parameter.
The provided code then forwards the processing to
\textit{main} or \textit{dest} as described in \secref{sec:forward}.

%%%%%%%%%%%%%%%%%%%%%%%%%%%%%%%%%%%%%%%%%%%%%%%%%%%%%%%%%%%%%%%%%%%%%%%%%%%%%%%%
\subsection{Include by Input}
\label{sec:input}

Including child documents by |\include| has some restrictions by design.
Most notably, the content of a child document always occupies
its own set of pages; pages cannot be shared between child documents.
Usually, this behaviour makes perfect sense
because each child document contain an essential part of the document.
However, in some situations it may be desirable to compose
a document from a collection of parts
without having mandatory page breaks between then.
For this case, the package
provides a mechanism to include parts
by |\input| which can also be processed individually.
However, by construction this mechanism
requires manual handling of the content to be output.

%%%%%%%%%%%%%%%%%%%%%%%%%%%%%%%%%%%%%%%%
\DescribeMacro{\ifchilddocmanual}
The main file should be prepared as usual, see \secref{sec:include}.
However, the document body must make a distinction
between processing of an individual part and of the main document, e.g.:
%
\begin{center}
\begin{tabular}{l}
|\ifchilddocmanual|\\
|\input{\childdocname}|\\
|\||else|\\
\textit{document body with }|\input{|\textit{part}|}|\\
|\||fi|
\end{tabular}
\end{center}
%
The conditional |\ifchilddocmanual| is true whenever
a part to be included by |\input| is being compiled,
and the name of the part is stored in |\childdocname|.

%%%%%%%%%%%%%%%%%%%%%%%%%%%%%%%%%%%%%%%%
\DescribeMacro{\childdocby}
Each part to be included by |\input| should start with:
%
\begin{center}
\begin{tabular}{l}
|\input{childdoc.def}|\\
|\childdocby{|\textit{main}|}|\\
\end{tabular}
\end{center}
%
The directive |\childdocby| is similar to |\childdocof|
described in \secref{sec:include},
but the subsequent selection of content must be done manually.
To that end, both |\ifchilddoc| and |\ifchilddocmanual|
will be true upon processing of a part,
and the name of the part is stored in |\childdocname|.
Note that |\jobname| will be set to the filename of the current part
so that each part receives an individual |.aux| file
that does not interfere with the |.aux| file(s) of the main document.
This behaviour can be altered by the alternative form
|\childdocby[*]{|\textit{main}|}| (with a non-empty optional argument)
which uses the |.aux| file of the main document
by setting |\jobname| to \textit{main}.

%%%%%%%%%%%%%%%%%%%%%%%%%%%%%%%%%%%%%%%%%%%%%%%%%%%%%%%%%%%%%%%%%%%%%%%%%%%%%%%%
\subsection{Driver Development}
\label{sec:driver}

The \textsf{childdoc} mechanism can also be use for the development
of definition files such as \LaTeX{} styles or classes.
This case differs from the above setup with multiple parts
included by |\include| in that no |\includeonly| should be invoked.
This can be achieved by starting the include file
(before |\ProvidesPackage|) with:
%
\begin{center}
\begin{tabular}{l}
|\input{childdoc.def}|\\
|\childdocforward{|\textit{main}|}|\\
\end{tabular}
\end{center}
%
or alternatively with:
%
\begin{center}
\begin{tabular}{l}
|\input{childdoc.def}|\\
|\childdocby{|\textit{main}|}|\\
\end{tabular}
\end{center}
%
Both forms have slightly different effects as described above.
The main file is prepared as usual, see \secref{sec:include}.

%%%%%%%%%%%%%%%%%%%%%%%%%%%%%%%%%%%%%%%%%%%%%%%%%%%%%%%%%%%%%%%%%%%%%%%%%%%%%%%%
\subsection{Legacy Detection}
\label{sec:detection}

The directive |\childdocmain| in the main file can detect
whether the complete document or merely a child is to be compiled
even without using the directive |\childdocof|.
This method is deprecated because it is less robust
and there is no compelling reason to use it;
it is merely provided for backward compatibility
and it may be removed in future versions.

If the detection mechanism is to be used,
it is mandatory to correctly specify
the filename of the main file as the argument of |\childdocmain|:
%
\begin{center}
\begin{tabular}{l}
|\input{childdoc.def}|\\
|\childdocmain{|\textit{main}|}|\\
\end{tabular}
\end{center}
%
If |\jobname| does not match the argument \textit{main} of |\childdocmain|,
it is assumed that |\jobname| points to the child file to be compiled.
When using |\childdocmain| with the main file specified as argument,
it suffices to start a child file
with just |\input{|\textit{main}|}|
without loading of the package and using |\childdocof|.
If instead all processing is done
with the appropriate \textsf{childdoc} directives,
the argument of \textit{main} of |\childdocmain| can be empty.

An alternative version of the command line processing described
in \secref{sec:commandline} using the detection mechanism reads:
%
\begin{center}
|... -jobname "|\textit{target}|" "|[\textit{flags}]%
[|\def\jobname{|\textit{dest}|}|]|\input{|\textit{main}|}"|
\end{center}

%%%%%%%%%%%%%%%%%%%%%%%%%%%%%%%%%%%%%%%%%%%%%%%%%%%%%%%%%%%%%%%%%%%%%%%%%%%%%%%%
\subsection{Manual Code}
\label{sec:manual}

In case one cannot be certain whether the definitions file |childdoc.def|
is installed on the target \TeX{} distribution
and one prefers not to ship it,
it is conceivable to paste a few relevant commands into the sources.

To that end, drop all statements |\input{childdoc.def}|
and perform the replacements as outlined below.
Instead of |\childdocmain{|\textit{main}|}| add the following code
to the top of the main file:
%
\begin{center}
\begin{tabular}{l}
|\||ifdefined\childdocname\endinput\||fi\newif\ifchilddoc|\\
|\edef\childdocname{\scantokens\expandafter{\jobname\noexpand}}|\\
|\def\childdocmain{|\textit{main}|}\||ifx\childdocmain\childdocname\||else|\\
|\childdoctrue\includeonly{\childdocname}\let\jobname\childdocmain\||fi|\\
\end{tabular}
\end{center}
%
Instead of |\childdocof{|\textit{main}|}| just include the main file
at the top of each child file:
%
\begin{center}
|\input{|\textit{main}|}|
\end{center}
%
A simple redirection |\childdocforward{|\textit{dest}|}| is achieved by:
%
\begin{center}
|\def\jobname{|\textit{dest}|}\input{\jobname}|
\end{center}
%
The redirection with prefix
|\childdocforwardprefix[|\textit{prefix}|]{|\textit{dest}|}|
is accomplished by:
%
\begin{center}
\begin{tabular}{l}
|{\edef\jobname{\scantokens\expandafter{\jobname\noexpand}}|\\
|\def\redirectjob |\textit{prefix}|#1~~~{\gdef\jobname{|\textit{dest}|#1}}|\\
|\expandafter\redirectjob\jobname~~~}\input{\jobname}|
\end{tabular}
\end{center}

In an alternative approach,
child documents can be compiled by a specific command line
without additional code or specific definitions:
%
\begin{center}
|... -jobname "|\textit{target}|" "|[\textit{flags}]%
|\includeonly{|\textit{dest}|}\input{|\textit{main}|}"|
\end{center}
%

%%%%%%%%%%%%%%%%%%%%%%%%%%%%%%%%%%%%%%%%%%%%%%%%%%%%%%%%%%%%%%%%%%%%%%%%%%%%%%%%
%%%%%%%%%%%%%%%%%%%%%%%%%%%%%%%%%%%%%%%%%%%%%%%%%%%%%%%%%%%%%%%%%%%%%%%%%%%%%%%%
\section{Information}

%%%%%%%%%%%%%%%%%%%%%%%%%%%%%%%%%%%%%%%%%%%%%%%%%%%%%%%%%%%%%%%%%%%%%%%%%%%%%%%%
\subsection{Copyright}

Copyright \copyright{} 2017--2018 Niklas Beisert

This work may be distributed and/or modified under the
conditions of the \LaTeX{} Project Public License, either version 1.3
of this license or (at your option) any later version.
The latest version of this license is in
  \url{http://www.latex-project.org/lppl.txt}
and version 1.3 or later is part of all distributions of \LaTeX{}
version 2005/12/01 or later.

This work has the LPPL maintenance status `maintained'.

The Current Maintainer of this work is Niklas Beisert.

This work consists of the files |README.txt|, |childdoc.ins| and |childdoc.dtx|
as well as the derived files |childdoc.def|, |cdocsamp.tex|
with |cdocsch1.tex|, |cdocsch2.tex|, |cdocspt3.tex|, |cdocspt4.tex|,
|cdocsdrf.tex|, |cdocsfn1.tex|, |cdocsfn2.tex|
as well as |childdoc.pdf|.

%%%%%%%%%%%%%%%%%%%%%%%%%%%%%%%%%%%%%%%%%%%%%%%%%%%%%%%%%%%%%%%%%%%%%%%%%%%%%%%%
\subsection{Files and Installation}

The package consists of the files:
%
\begin{center}
\begin{tabular}{ll}
    |README.txt|   & readme file \\
    |childdoc.ins| & installation file \\
    |childdoc.dtx| & source file \\
    |childdoc.def| & definition file \\
    |cdocsamp.tex| & sample main file \\
    |cdocsch1.tex| & sample include file \\
    |cdocsch2.tex| & sample include file \\
    |cdocspt3.tex| & sample part file \\
    |cdocspt4.tex| & sample part file \\
    |cdocsdrf.tex| & sample redirection file \\
    |cdocsfn1.tex| & sample redirection file \\
    |cdocsfn2.tex| & sample redirection file \\
    |childdoc.pdf| & manual
\end{tabular}
\end{center}
%
The distribution consists of the files
|README.txt|, |childdoc.ins| and |childdoc.dtx|.
%
\begin{itemize}
\item
Run (pdf)\LaTeX{} on |childdoc.dtx|
to compile the manual |childdoc.pdf| (this file).
\item
Run \LaTeX{} on |childdoc.ins| to create the definitions file |childdoc.def|
and the sample |cdocsamp.tex| with include files
|cdocsch1.tex|, |cdocsch2.tex|, |cdocspt3.tex|, |cdocspt4.tex|,
|cdocsdrf.tex|, |cdocsfn1.tex|, |cdocsfn2.tex|.
Then copy the file |childdoc.def| to an appropriate directory of your \LaTeX{}
distribution, e.g.\ \textit{texmf-root}|/tex/latex/childdoc|.
\end{itemize}

%%%%%%%%%%%%%%%%%%%%%%%%%%%%%%%%%%%%%%%%%%%%%%%%%%%%%%%%%%%%%%%%%%%%%%%%%%%%%%%%
\subsection{Related CTAN Packages}

There are several other packages which offer a similar functionality:
%
\begin{itemize}
\item
The packages
\href{http://ctan.org/pkg/docmute}{\textsf{docmute}},
\href{http://ctan.org/pkg/includex}{\textsf{includex}} and
\href{http://ctan.org/pkg/standalone}{\textsf{standalone}}
provide commands to include only the document body of
a child file thus allowing both files to be compiled individually.
\item
The packages \href{http://ctan.org/pkg/subdocs}{\textsf{subdocs}}
and \href{http://ctan.org/pkg/subfiles}{\textsf{subfiles}}
provide structures in which the main and child documents can be
encapsulated and allowing them to be compiled individually.
The inclusion mechanism is different from the conventional |\include|.
\item
The package \href{http://ctan.org/pkg/combine}{\textsf{combine}}
is an elaborate solution to combine several documents into one.
\end{itemize}
%
See also the CTAN topic \href{http://ctan.org/topic/subdocs}{\textsf{subdocs}}
for further related packages.
The present package differs from the above solutions in that
a document structure constructed with the conventional |\include| mechanism
just needs two extra commands at the top of every file
such that all constituent files can be compiled individually.

%%%%%%%%%%%%%%%%%%%%%%%%%%%%%%%%%%%%%%%%%%%%%%%%%%%%%%%%%%%%%%%%%%%%%%%%%%%%%%%%
%\subsection{Feature Suggestions}
%
%The following is a list of features which may be useful for future
%versions of this package:
%%
%\begin{itemize}
%\item
%\ldots
%\end{itemize}

%%%%%%%%%%%%%%%%%%%%%%%%%%%%%%%%%%%%%%%%%%%%%%%%%%%%%%%%%%%%%%%%%%%%%%%%%%%%%%%%
\subsection{Revision History}

%%%%%%%%%%%%%%%%%%%%%%%%%%%%%%%%%%%%%%%%
\paragraph{v2.0:} 2018/12/30

\begin{itemize}
\item
immediate forward processing
\item
added |\childdocby| mechanism
\item
manual restructured
\end{itemize}

%%%%%%%%%%%%%%%%%%%%%%%%%%%%%%%%%%%%%%%%
\paragraph{v1.6:} 2018/01/17

\begin{itemize}
\item
application for development of include files
\item
corrections to manual
\end{itemize}

%%%%%%%%%%%%%%%%%%%%%%%%%%%%%%%%%%%%%%%%
\paragraph{v1.5:} 2017/05/21

\begin{itemize}
\item
more complete structuring introduced
\item
|\childdocof| introduced
\item
|\childdoc| renamed to |\childdocmain|
\item
|\childredirect| renamed to |\childdocforward| and |\childdocforwardprefix|
and functionality expanded
\end{itemize}

%%%%%%%%%%%%%%%%%%%%%%%%%%%%%%%%%%%%%%%%
\paragraph{v1.0:} 2017/04/27

\begin{itemize}
\item
manual and install package
\item
first version published on CTAN
\end{itemize}

%%%%%%%%%%%%%%%%%%%%%%%%%%%%%%%%%%%%%%%%
\paragraph{v0.6:} 2017/04/26

\begin{itemize}
\item
redirection mechanism added
\end{itemize}

%%%%%%%%%%%%%%%%%%%%%%%%%%%%%%%%%%%%%%%%
\paragraph{v0.5:} 2017/04/26

\begin{itemize}
\item
functionality in definition file
\end{itemize}


%%%%%%%%%%%%%%%%%%%%%%%%%%%%%%%%%%%%%%%%%%%%%%%%%%%%%%%%%%%%%%%%%%%%%%%%%%%%%%%%
%%%%%%%%%%%%%%%%%%%%%%%%%%%%%%%%%%%%%%%%%%%%%%%%%%%%%%%%%%%%%%%%%%%%%%%%%%%%%%%%
%%%%%%%%%%%%%%%%%%%%%%%%%%%%%%%%%%%%%%%%%%%%%%%%%%%%%%%%%%%%%%%%%%%%%%%%%%%%%%%%
\appendix

\settowidth\MacroIndent{\rmfamily\scriptsize 000\ }

 \DocInput{childdoc.dtx}

\end{document}
%</driver>
% \fi
%
% %%%%%%%%%%%%%%%%%%%%%%%%%%%%%%%%%%%%%%%%%%%%%%%%%%%%%%%%%%%%%%%%%%%%%%%%%%%%%%
% %%%%%%%%%%%%%%%%%%%%%%%%%%%%%%%%%%%%%%%%%%%%%%%%%%%%%%%%%%%%%%%%%%%%%%%%%%%%%%
% \section{Sample}
%\iffalse
%<*samplemain>
%\fi
%
% The following presents a sample document
% with two chapters, two parts, a title page,
% a compile flag as well as three forwarding files to set the flag.
% It consists of eight |.tex| files:
% \begin{center}
% \begin{tabular}{ll}
% |cdocsamp.tex|&main file\\
% |cdocsch1.tex|&include file for chapter 1\\
% |cdocsch2.tex|&include file for chapter 2\\
% |cdocspt3.tex|&include file for part 3\\
% |cdocspt4.tex|&include file for part 4\\
% |cdocsdrf.tex|&forwarding file for main file in draft mode\\
% |cdocsfi1.tex|&forwarding file for final version of chapter 1\\
% |cdocsfi2.tex|&forwarding file for final version of chapter 2\\
% \end{tabular}
% \end{center}
% Each of the eight files can be compiled directly by the \LaTeX{} compiler.
%
% %%%%%%%%%%%%%%%%%%%%%%%%%%%%%%%%%%%%%%
% \paragraph{Main File.}
%
% The main file is called |cdocsamp.tex|.
%
% Load the \textsf{childdoc} definitions and
% declare the filename for the main document:
%    \begin{macrocode}
\input{childdoc.def}
\childdocmain{}
%    \end{macrocode}

% Optional override for |\version| flag:
%    \begin{macrocode}
%%\ifchilddoc\else\providecommand{\version}{draft}\fi
%    \end{macrocode}

% Define the default values for the |\version| flag
% (|final| for the main file and |draft| for childs):
%    \begin{macrocode}
\ifchilddoc
\providecommand{\version}{draft}
\else
\providecommand{\version}{final}
\fi
%    \end{macrocode}

% Load the standard document class:
%    \begin{macrocode}
\documentclass[12pt]{article}
%    \end{macrocode}

% Start the document body:
%    \begin{macrocode}
\begin{document}
%    \end{macrocode}

% Declare a title page.
% Print title, part of document being processed and version flag:
%    \begin{macrocode}
\addtocounter{page}{-1}
\begin{center}
{\LARGE\bfseries{}childdoc example\par}
\vspace{1cm}
\ifchilddoc
\ifchilddocmanual part\else chapter\fi:
`\childdocname' of `\childdocjob'\par
\else
main document: `\childdocjob'\par
\fi
version: \version\par
\end{center}
\newpage
%    \end{macrocode}

% Manually include selected file,
% otherwise process as usual:
%    \begin{macrocode}
\ifchilddocmanual
\section*{part `\childdocname'}
\input{\childdocname}
\else
%    \end{macrocode}

% Include the two chapters:
%    \begin{macrocode}
\include{cdocsch1}
\include{cdocsch2}
%    \end{macrocode}

% Include the two parts unless only chapters should be displayed:
%    \begin{macrocode}
\ifchilddoc\else
\section{part three}
\input{cdocspt3}
\section{part four}
\input{cdocspt4}
\fi
%    \end{macrocode}

% Process as usual until here:
%    \begin{macrocode}
\fi
%    \end{macrocode}

% End of document body:
%    \begin{macrocode}
\end{document}
%    \end{macrocode}
%\iffalse
%</samplemain>
%\fi
%
% %%%%%%%%%%%%%%%%%%%%%%%%%%%%%%%%%%%%%%
% \paragraph{Chapter Include Files.}
%
% The include files are called |cdocsch1.tex| and |cdocsch2.tex|.
%
%\iffalse
%<*samplechap1|samplechap2>
%\fi

% Optional override for |\version| flag:
%    \begin{macrocode}
%%\providecommand{\version}{final}
%    \end{macrocode}

% Include the main document:
%    \begin{macrocode}
\input{childdoc.def}
\childdocof{cdocsamp}
%    \end{macrocode}

%\iffalse
%</samplechap1|samplechap2>
%\fi
%
%\iffalse
%<*samplechap1>
%\fi
% Some text for chapter 1:
%    \begin{macrocode}
\section{one}
some text in chapter one
%    \end{macrocode}

%\iffalse
%</samplechap1>
%\fi
% Some text for chapter 2:
%\iffalse
%<*samplechap2>
%\fi
%    \begin{macrocode}
\section{two}
more text in chapter two
%    \end{macrocode}

%\iffalse
%</samplechap2>
%\fi
%
% %%%%%%%%%%%%%%%%%%%%%%%%%%%%%%%%%%%%%%
% \paragraph{Part Include Files.}
%
% The include files are called |cdocspt3.tex| and |cdocspt4.tex|.
%
%\iffalse
%<*samplepart3|samplepart4>
%\fi

% Optional override for |\version| flag:
%    \begin{macrocode}
%%\providecommand{\version}{final}
%    \end{macrocode}

% Include the main document:
%    \begin{macrocode}
\input{childdoc.def}
\childdocby{cdocsamp}
%    \end{macrocode}

%\iffalse
%</samplepart3|samplepart4>
%\fi
%
%\iffalse
%<*samplepart3>
%\fi
% Some text for part 3:
%    \begin{macrocode}
some text in part three
%    \end{macrocode}

%\iffalse
%</samplepart3>
%\fi
% Some text for part 4:
%\iffalse
%<*samplepart4>
%\fi
%    \begin{macrocode}
more text in part four
%    \end{macrocode}

%\iffalse
%</samplepart4>
%\fi
%
% %%%%%%%%%%%%%%%%%%%%%%%%%%%%%%%%%%%%%%
% \paragraph{Forwarding for a Complete Draft.}
%
% The following forwarding file |cdocsdrf.tex|
% compiles the main document in draft mode:
%\iffalse
%<*sampledraft>
%\fi
%    \begin{macrocode}
\def\version{draft}
\input{childdoc.def}
\childdocforward{cdocsamp}
%    \end{macrocode}

%\iffalse
%</sampledraft>
%\fi
%
% %%%%%%%%%%%%%%%%%%%%%%%%%%%%%%%%%%%%%%
% \paragraph{Forwarding for Final Version of the Chapters.}
%
% The following forwarding files |cdocsfn1.tex| and |cdocsfn2.tex|
% (with identical content)
% compile the final versions of the child documents
% |cdocsch1.tex| and |cdocsch2.tex|, respectively:
%\iffalse
%<*samplefinal>
%\fi
%    \begin{macrocode}
\def\version{final}
\input{childdoc.def}
\childdocforwardprefix[cdocsamp]{cdocsfn}{cdocsch}
%    \end{macrocode}

%\iffalse
%</samplefinal>
%\fi
%
% %%%%%%%%%%%%%%%%%%%%%%%%%%%%%%%%%%%%%%
% \paragraph{Command Line Processing.}
%
% The following three command lines generate the output files
% |cdocscld|, |cdocscl1| and |cdocscl2|
% which should be identical to
% |cdocsdrf|, |cdocsch1| and |cdocsfn2|, respectively:
% \begin{center}
% \begin{tabular}{l}
% |latex -jobname cdocscld \|\\
% |  "\def\version{draft}\input{childdoc.def}\childdocforward{cdocsamp}"|\\
% |latex -jobname cdocscl1 \|\\
% |  "\input{childdoc.def}\childdocforward[cdocsamp]{cdocsch1}"|\\
% |latex -jobname cdocscl2 \|\\
% |  "\def\version{final}\input{childdoc.def}\childdocforward{cdocsch2}"|
% \end{tabular}
% \end{center}
% Note that the trailing backslash on each first line
% merely continues the input to the second line
% (for convenient cut ant paste).
% Furthermore, the command |latex| can be replaced by any
% of its alternative versions such as |pdflatex|.
%
% %%%%%%%%%%%%%%%%%%%%%%%%%%%%%%%%%%%%%%%%%%%%%%%%%%%%%%%%%%%%%%%%%%%%%%%%%%%%%%
% %%%%%%%%%%%%%%%%%%%%%%%%%%%%%%%%%%%%%%%%%%%%%%%%%%%%%%%%%%%%%%%%%%%%%%%%%%%%%%
% \section{Implementation}
%\iffalse
%<*package>
%\fi
%
% This section describes the definitions file |childdoc.def|.

% The definitions cannot be loaded using |\usepackage| or |\RequirePackage|
% which has a mechanism to prevent loading a style file more than once.
% When loading the definitions by means of |\input|
% multiple instances have to be prevented manually:
%\iffalse
%This code needs to be before the `\ProvidesFile' directive
%which is defined at the beginning of this file.
%Therefore it is also placed there and commented out here.
%</package>
%<*discard>
%\fi
%    \begin{macrocode}
\ifdefined\childdocmain\endinput\fi
%    \end{macrocode}
%\iffalse
%</discard>
%<*package>
%\fi
%
% \macro{\ifchilddoc}
% \macro{\ifchilddocmanual}
% The conditional |\ifchilddoc| tells whether a
% child (true) or main (false) document is being compiled.
% The conditional |\ifchilddocmanual| tells whether
% the |\includeonly| mechanism is used (false) or
% the selection of child files must be performed manually (true).
% The definitions initialise to false:
%    \begin{macrocode}
\newif\ifchilddoc
\newif\ifchilddocmanual
%    \end{macrocode}

% \macro{\childdocname}
% \macro{\childdocjob}
% The macro |\childdocname| stores the name of the main document
% to be compiled. The macro |\childdocjob| stores the name of
% the document on which the \LaTeX{} compiler was originally invoked.
% The content of |\jobname| cannot be compared
% to filenames specified in the source due to different catcodes.
% The following code rescans |\jobname|, stores the result
% in |\childdocname| and saves a copy in |\childdocjob|:
%    \begin{macrocode}
\edef\childdocname{\scantokens\expandafter{\jobname\noexpand}}
\let\childdocjob\childdocname
%    \end{macrocode}

% \macro{\childdocdisable}
% The macro |\childdocdisable| prevents the main file
% from being processed more than once.
% At this stage, the main document command |\childdocmain|
% is assumed to be called once again where it should do nothing.
% Any subsequent call to it should prevent
% a secondary processing of the main document
% It overwrites the forwarding commands
% |\childdocof| and |\childdocforward|
% with empty macros to prevent further inclusions of the main document:
%    \begin{macrocode}
\newcommand{\childdocdisable}
{
  \renewcommand{\childdocmain}[1]{\renewcommand{\childdocmain}[1]{\endinput}}
  \renewcommand{\childdocof}[1]{}
  \renewcommand{\childdocby}[2][]{}
  \renewcommand{\childdocforward}[2][]{}
  \renewcommand{\childdocdisable}{}
}
%    \end{macrocode}

% \macro{\childdocmain}
% The macro |\childdocmain| is to be called at the top of the main file
% with nothing or the main filename (without extension) as argument.
% First, it breaks loops.
% If the argument is not empty and does not match |\childdocname|
% (which is set by the first inclusion of |childdoc.def|),
% |\ifchilddoc| is set to true, |\includeonly| is applied to the child file
% and |\jobname| is set to the main file
% (for proper handling of |.aux| files):
%    \begin{macrocode}
\newcommand{\childdocmain}[1]
{
  \childdocdisable\childdocmain{}
  \if?#1?\else
    \begingroup
      \def\childdoctmp{#1}
      \ifx\childdoctmp\childdocname
        \def\childdoctmp{}
      \else
        \def\childdoctmp
        {
          \childdoctrue
          \includeonly{\childdocname}
          \def\childdocjob{#1}
          \def\jobname{#1}
        }
      \fi
      \expandafter
    \endgroup
    \childdoctmp
  \fi
}
%    \end{macrocode}

% \macro{\childdocof}
% The command |\childdocof| redirects
% compilation to the main file |#1|.
%    \begin{macrocode}
\newcommand{\childdocof}[1]
{
  \childdocdisable
  \childdoctrue
  \includeonly{\childdocname}
  \def\jobname{#1}
  \def\childdocjob{#1}
  \input{#1}
}
%    \end{macrocode}

% \macro{\childdocby}
% The command |\childdocby| ....
%    \begin{macrocode}
\newcommand{\childdocby}[2][]
{
  \childdocdisable
  \childdoctrue
  \childdocmanualtrue
  \if?#1?\else
    \def\jobname{#2}
  \fi
  \def\childdocjob{#2}
  \input{#2}
  \endinput
}
%    \end{macrocode}

% \macro{\childdocforward}
% The command |\childdocforward| redirects
% compilation to the main file or
% (if the optional argument is given) a child file.
% Parameters are set as if the main file
% or a child file starting with |\childdocof| was compiled.
% Then compilation is handed over to the main file:
%    \begin{macrocode}
\newcommand{\childdocforward}[2][]
{
  \begingroup
    \if?#1?
      \def\childdoctmp
      {
        \def\childdocname{#2}
        \def\childdocjob{#2}
        \def\jobname{#2}
        \input{#2}
        \endinput
      }
    \else
      \def\childdoctmp
      {
        \childdocdisable
        \def\childdocname{#2}
        \childdoctrue
        \includeonly{#2}
        \def\childdocjob{#1}
        \def\jobname{#1}
        \input{#1}
        \endinput
      }
    \fi
    \expandafter
  \endgroup
  \childdoctmp
}
%    \end{macrocode}

% \macro{\childdocforwardprefix}
% The command |\childdocforwardprefix| redirects
% compilation to the main or a child file by means of a pattern.
% The prefix |#1| in the current filename is replaced by |#2|
% and the suffix of the current filename is kept
% (it is assumed that the filename does not contain the substring `|~~~|'
% which is used as a delimiter).
% Compilation is handed over to the new file by |\childdocforward|:
%    \begin{macrocode}
\newcommand{\childdocforwardprefix}[3][]
{
  \begingroup
    \def\childdocextract #2##1~~~{\def\childdoctmp{\childdocforward[#1]{#3##1}}}
    \expandafter\childdocextract\childdocname~~~
    \expandafter
  \endgroup
  \childdoctmp
}
%    \end{macrocode}

% \macro{\childdoc}
% The deprecated macro |\childdoc| is a legacy version of |\childdocmain|:
%    \begin{macrocode}
\newcommand{\childdoc}{\childdocmain}
%    \end{macrocode}

% \macro{\childdocredirect}
% The deprecated macro |\childdocredirect| is a legacy version
% of |\childdocforward| and |\childdocforwardprefix|:
%    \begin{macrocode}
\newcommand{\childdocredirect}[2][]
{
  \begingroup
    \if?#1?
      \def\childdoctmp{\childdocforward{#2}}
    \else
      \def\childdoctmp{\childdocforwardprefix{#1}{#2}}
    \fi
    \expandafter
  \endgroup
  \childdoctmp
}
%    \end{macrocode}

%\iffalse
%</package>
%\fi
%
\endinput
\childdocforward{cdocsamp}"|\\
% |latex -jobname cdocscl1 \|\\
% |  "% \iffalse
%
% childdoc.dtx Copyright (C) 2017-2018 Niklas Beisert
%
% This work may be distributed and/or modified under the
% conditions of the LaTeX Project Public License, either version 1.3
% of this license or (at your option) any later version.
% The latest version of this license is in
%   http://www.latex-project.org/lppl.txt
% and version 1.3 or later is part of all distributions of LaTeX
% version 2005/12/01 or later.
%
% This work has the LPPL maintenance status `maintained'.
%
% The Current Maintainer of this work is Niklas Beisert.
%
% This work consists of the files childdoc.dtx and childdoc.ins
% and the derived files childdoc.def and cdocsamp.tex with
% cdocsch1.tex, cdocsch2.tex, cdocsdrf.tex, cdocsfn1.tex, cdocsfn2.tex.
%
%<package>\ifdefined\childdocmain\endinput\fi
%<package>\ProvidesFile{childdoc.def}[2018/12/30 v2.0 child document driver]
%<samplemain>\ProvidesFile{cdocsamp.tex}[2018/12/30 v2.0 sample for childdoc]
%<*driver>
%\ProvidesFile{childdoc.drv}[2018/12/30 v2.0 childdoc reference manual file]
\PassOptionsToClass{10pt,a4paper}{article}
\documentclass{ltxdoc}

\usepackage[margin=35mm]{geometry}
\usepackage{hyperref}
\usepackage{hyperxmp}
\usepackage[usenames]{color}

\hypersetup{colorlinks=true}
\hypersetup{pdfstartview=FitH}
\hypersetup{pdfpagemode=UseNone}
\hypersetup{pdfsource={}}
\hypersetup{pdflang={en-UK}}
\hypersetup{pdfcopyright={Copyright 2017-2018 Niklas Beisert.
  This work may be distributed and/or modified under the
  conditions of the LaTeX Project Public License, either version 1.3
  of this license or (at your option) any later version.}}
\hypersetup{pdflicenseurl={http://www.latex-project.org/lppl.txt}}
\hypersetup{pdfcontactaddress={ETH Zurich, ITP, HIT K,
  Wolfgang-Pauli-Strasse 27}}
\hypersetup{pdfcontactpostcode={8093}}
\hypersetup{pdfcontactcity={Zurich}}
\hypersetup{pdfcontactcountry={Switzerland}}
\hypersetup{pdfcontactemail={nbeisert@itp.phys.ethz.ch}}
\hypersetup{pdfcontacturl={http://people.phys.ethz.ch/\xmptilde nbeisert/}}

\newcommand{\secref}[1]{\hyperref[#1]{section \ref*{#1}}}

\parskip1ex
\parindent0pt
\let\olditemize\itemize
\def\itemize{\olditemize\parskip0pt}

\begin{document}

\title{The \textsf{childdoc} Package}
\hypersetup{pdftitle={The childdoc Package}}
\author{Niklas Beisert\\[2ex]
  Institut f\"ur Theoretische Physik\\
  Eidgen\"ossische Technische Hochschule Z\"urich\\
  Wolfgang-Pauli-Strasse 27, 8093 Z\"urich, Switzerland\\[1ex]
  \href{mailto:nbeisert@itp.phys.ethz.ch}
  {\texttt{nbeisert@itp.phys.ethz.ch}}}
\hypersetup{pdfauthor={Niklas Beisert}}
\hypersetup{pdfsubject={Manual for the LaTeX2e Package childdoc}}
\date{30 December 2018, \textsf{v2.0}}
\maketitle

\begin{abstract}\noindent
\textsf{childdoc} is a \LaTeXe{} package
that enables the direct compilation
of document sections included by |\include|
to individual files.
\end{abstract}

\begingroup
\parskip0ex
\tableofcontents
\endgroup

%%%%%%%%%%%%%%%%%%%%%%%%%%%%%%%%%%%%%%%%%%%%%%%%%%%%%%%%%%%%%%%%%%%%%%%%%%%%%%%%
%%%%%%%%%%%%%%%%%%%%%%%%%%%%%%%%%%%%%%%%%%%%%%%%%%%%%%%%%%%%%%%%%%%%%%%%%%%%%%%%
\section{Introduction}

\LaTeX{} provides a mechanism to structure a large document (such as a book)
into a main file and several child files (containing the chapters)
using the |\include| command.
This mechanism is beneficial for documents
which span hundreds of pages in order to
make the source file(s) more manageable.
Moreover, compilation can be restricted to
selected child files by means of the |\includeonly| command.
The latter feature can be used to reduce the compilation time while editing
(this was significantly more useful in the earlier days of \LaTeX{})
or to generate a smaller document which is easier to navigate.
Another application of |\includeonly| is to generate
documents consisting of selected parts of the complete document.

However, there are a few drawbacks of the plain |\include| mechanism:
\begin{itemize}
\item
The child files cannot be compiled on their own,
they can only be compiled via the main file.
A naive editing environment
(such as a text editor with an option
to have the current file processed by \LaTeX)
may require one to switch to the main file before compiling;
attempting to compile the child file produces errors.
\item
The main file must be modified (each time)
to adjust the |\includeonly| command
to the present needs. This easily leaves the main file in a messy state.
\item
The generated document will always carry the filename
of the main document. This is inconvenient if
several child files are to be compiled and
to be kept for distribution.
\end{itemize}

The present package provides a simple interface
to make child files individually compilable by \LaTeX{}.
Compiling a child file then has the same effect as compiling
the main file with an |\includeonly| command
to select the appropriate child.
Moreover the generated document will carry the name of the child
rather than the main file.
This resolves all three above issues.

This feature is meant to make the editing of books,
thesis documents and lecture notes somewhat more convenient.
However, the package can also be used efficiently for
composing a series of documents (such as exercise sheets)
which are typically distributed individually.
It then assists the author in generating the individual documents
(potentially in different versions)
as well as a document containing the collected series.
Another application is in developing style files
or other kinds of included material
where compilation of the style file could redirect
to a sample or test file.

%%%%%%%%%%%%%%%%%%%%%%%%%%%%%%%%%%%%%%%%%%%%%%%%%%%%%%%%%%%%%%%%%%%%%%%%%%%%%%%%
%%%%%%%%%%%%%%%%%%%%%%%%%%%%%%%%%%%%%%%%%%%%%%%%%%%%%%%%%%%%%%%%%%%%%%%%%%%%%%%%
\section{Usage}

First of all, the package \textsf{childdoc} is \emph{not} a standard
\LaTeXe{} |.sty| style file! Therefore it needs to be invoked in
a non-standard way.

%%%%%%%%%%%%%%%%%%%%%%%%%%%%%%%%%%%%%%%%%%%%%%%%%%%%%%%%%%%%%%%%%%%%%%%%%%%%%%%%
\subsection{Included Files}
\label{sec:include}

%%%%%%%%%%%%%%%%%%%%%%%%%%%%%%%%%%%%%%%%
\DescribeMacro{\childdocmain}
To use the package, add the commands
\begin{center}
\begin{tabular}{l}
|\input{childdoc.def}|\\
|\childdocmain{}|\\
\end{tabular}
\end{center}
at the very top of the main \LaTeX{} file,
in particular \emph{before} the |\documentclass| statement!
The argument of |\childdocmain| should be left empty
(but it must be present).

%%%%%%%%%%%%%%%%%%%%%%%%%%%%%%%%%%%%%%%%
\DescribeMacro{\childdocof}
Furthermore, add the commands
\begin{center}
\begin{tabular}{l}
|\input{childdoc.def}|\\
|\childdocof{|\textit{main}|}|\\
\end{tabular}
\end{center}
at the top of every child file \textit{child}
which is included by |\include{|\textit{child}|}|
from within the main file
(or at least for those files to be compiled individually).
The argument \textit{main} must be the filename of the main file.

There are a couple of
considerations in setting up the main and child documents:

%%%%%%%%%%%%%%%%%%%%%%%%%%%%%%%%%%%%%%%%
\paragraph{Restrictions.}

Please note the following restrictions:
\begin{itemize}
\item
|\childdocmain| must be called with one argument \textit{main}
to ensure compatibility with earlier version of the package.
It must either be empty (|\childdocmain{}|)
or precisely match the filename of the main file in which it is specified.
See \secref{sec:detection} for further information.
\item
The filename \textit{main} must be specified without the |.tex| extension.
\item
The filename \textit{main} is case sensitive
(even in case-insensitive file systems)
due to internal string comparison.
\item
The argument \textit{main} should be fully expanded, it cannot be a macro.
\item
Subdirectories and special characters should be avoided in filenames.
\item
The command |\childdocmain{|\textit{main}|}| must be followed by a whitespace.
It should not be followed immediately by another command
or by a comment mark `|%|'.
This is because the \TeX{} parser reads the token immediately following
the argument of |\childdocmain| and puts it
at the beginning of every child section;
however, a white\-space is ignored.
\end{itemize}

%%%%%%%%%%%%%%%%%%%%%%%%%%%%%%%%%%%%%%%%
\paragraph{Content of Main File.}

It is advisable to place all content in the child files included by |\include|.
Any output contained in the main file will appear in all child documents
unless suppressed manually;
it cannot be suppressed automatically by the |\includeonly| directive
and thus should normally be avoided.
A method to include some content in the main file
by means of conditional processing is described in \secref{sec:conditional}.

%%%%%%%%%%%%%%%%%%%%%%%%%%%%%%%%%%%%%%%%
\paragraph{Page Numbering.}

When only a part of the document is compiled,
the appropriate numbering of pages
(as well as other status parameters)
is determined from the |.aux| files.
The latter contain information from previous passes.
However this information needs to propagate through
all intermediate child documents.
Therefore the page numbering in child documents may well
be inconsistent until the complete document is compiled at least once.

A useful (if unconventional) way to always ensure a consistent
page numbering is to restart the numbering in each child document
and denote the pages by `\textit{child}|.|\textit{page}'
where \textit{child} represents the chapter/section number of the child file.
This can be achieved by the command
|\numberwithin{page}{|\textit{child}|}|
of the \textsf{amsmath} package
where \textit{child} can be |chapter| or |section|
depending on the chosen structuring.
Alternatively, one can modify the macro |\thepage| appropriately
and reset the counter |page| at the start of each child file.

%%%%%%%%%%%%%%%%%%%%%%%%%%%%%%%%%%%%%%%%%%%%%%%%%%%%%%%%%%%%%%%%%%%%%%%%%%%%%%%%
\subsection{Conditional Processing}
\label{sec:conditional}

The package provides a mechanism to compile different versions
of a document. To customise the versions further some conditional processing
can come in handy to distinguish which version is being compiled.
The package provides two macros to describe the compilation context:

%%%%%%%%%%%%%%%%%%%%%%%%%%%%%%%%%%%%%%%%
\DescribeMacro{\ifchilddoc}
The conditional |\ifchilddoc| distinguishes between the compilation of
child documents and the main document:
%
\begin{center}
|\ifchilddoc |\textit{child-code}| |[|\||else |\textit{main-code}]| \||fi|
\end{center}

%%%%%%%%%%%%%%%%%%%%%%%%%%%%%%%%%%%%%%%%
\DescribeMacro{\childdocname}
\DescribeMacro{\childdocjob}
The macro |\childdocname| contains the filename (without extension)
of the main or child file being processed.
Note that |\childdocjob| will always contain the name of the main file.

%%%%%%%%%%%%%%%%%%%%%%%%%%%%%%%%%%%%%%%%
\paragraph{Title Page.}

Conditional processing can be used to include a title or banner page
in the main document when proper precautions are taken.
Importantly, the code in the main file should ensure that the page counter
(as well as other status parameters which are stored in the |.aux| files)
takes the same value after the conditional processing.
Otherwise the page numbers may take divergent values
depending on which part is compiled.

For example, a title page could be declared by:
%
\begin{center}
\begin{tabular}{l}
|\ifchilddoc\||else|\\
|\addtocounter{page}{-1}|\\
\textit{code for title page}\\
|\newpage|\\
|\||fi|
\end{tabular}
\end{center}
%
A banner page for the child documents can be generated by:
%
\begin{center}
\begin{tabular}{l}
|\ifchilddoc|\\
|\addtocounter{page}{-1}|\\
\textit{code for banner page}\\
|\newpage|\\
|\||fi|
\end{tabular}
\end{center}
%
Here one could write a message such as:
\begin{center}
|This is the part \childdocname{} of \childdocjob{}.|
\end{center}

%%%%%%%%%%%%%%%%%%%%%%%%%%%%%%%%%%%%%%%%%%%%%%%%%%%%%%%%%%%%%%%%%%%%%%%%%%%%%%%%
\subsection{Flags}
\label{sec:flags}

The package makes it easy to generate different versions
of the main or child documents.
To this end compilation flags can be defined
and assigned different default values.
They will be particularly useful in conjunction
with the forwarding mechanism described in \secref{sec:forward}.

For example, it may be useful to have a flag |\version|
which can be set to |draft| or |final|.
The document source will contain some conditional code
depending on the value of |\version|.
Suppose further, the flag should default to |final| for the main file
and to |draft| for child files
which is a natural assignment for editing the document.
This is achieved by placing the following code
in the preamble of the main document
(below the |\childdocmain| directive):
%
\begin{center}
\begin{tabular}{l}
|\ifchilddoc|\\
|\providecommand{\version}{draft}|\\
|\||else|\\
|\providecommand{\version}{final}|\\
|\||fi|
\end{tabular}
\end{center}
%
The definition by |\providecommand| makes sure
that previous definitions are not overwritten.
Further statements |\providecommand{\version}{...}|
can thus be added before the above code to override it.

For the main file, one might add a line
(between |\childdocmain| and the above block)
%
\begin{center}
|%\ifchilddoc\||else\providecommand{\version}{draft}\||fi|
\end{center}
%
which can be uncommented to produce a draft version.
Likewise one can add a line to the very top of a child file
(above the |\childdocof{|\textit{main}|}| directive)
%
\begin{center}
|%\providecommand{\version}{final}|
\end{center}
%
which can be uncommented to produce the final version of this child document.

%%%%%%%%%%%%%%%%%%%%%%%%%%%%%%%%%%%%%%%%%%%%%%%%%%%%%%%%%%%%%%%%%%%%%%%%%%%%%%%%
\subsection{Forwarding}
\label{sec:forward}

Different versions of the main or child documents
using compilation flags as described in \secref{sec:flags}
can be (permanently) stored in different files
for convenient compilation, viewing and distribution.
To this end, the package defines a command
to pass on compilation to a different file:

%%%%%%%%%%%%%%%%%%%%%%%%%%%%%%%%%%%%%%%%
\DescribeMacro{\childdocforward}
The command |\childdocforward| redirects processing to
another source file:
%
\begin{center}
\begin{tabular}{l}
|\input{childdoc.def}|\\
|\childdocforward[|\textit{main}|]{|\textit{dest}|}|\\
\end{tabular}
\end{center}
%
The argument \textit{dest} is the destination file
(without extension).
It should be the main file or one of the child files.
Note that further \textsf{childdoc} directives
such as |\childdocof| and |\childdocforward|
in the indicated file will be processed in this form.
The optional argument \textit{main}
passes on directly to the main file \textit{main}
while pretending to compile the child \textit{dest}.
This form behaves as if \textit{dest}
issues |\childdocof{|\textit{main}|}| right away,
and no further \textsf{childdoc} directives will be processed.

%%%%%%%%%%%%%%%%%%%%%%%%%%%%%%%%%%%%%%%%
\DescribeMacro{\...prefix}
In the alternative form |\childdocforwardprefix|,
%
\begin{center}
\begin{tabular}{l}
|\input{childdoc.def}|\\
|\childdocforwardprefix[|\textit{main}|]{|\textit{prefix}|}{|\textit{dest}|}|
\end{tabular}
\end{center}
%
the destination file is determined by a pattern
depending on the current file:
To make this work, the current file must be called
`{\textit{prefix}\hspace{0.2em}\textit{suffix}}'
with \textit{prefix} matching precisely the argument.
Processing is then passed on to the file
`{\textit{dest}\hspace{0.2em}\textit{suffix}}'.
Surely, the same effect is achieved by
directly specifying the
argument `{\textit{dest}\hspace{0.2em}\textit{suffix}}'
in the first form.
However, that requires to set up a different file
for each child. With the alternative form of the command
all these files can have exactly the same content
which simplifies setting them up and maintaining them.

For example, the following file |draft.tex|
with a compilation flag |\version| as described in \secref{sec:flags}
compiles the main document as a draft:
%
\begin{center}
\begin{tabular}{l}
|\def\version{draft}|\\
|\input{childdoc.def}|\\
|\childdocforward{|\textit{main}|}|
\end{tabular}
\end{center}
%
Likewise, the following files |final|\textit{nn}|.tex|
compile the final version of the child document
|child|\textit{nn}|.tex|:
%
\begin{center}
\begin{tabular}{l}
|\def\version{final}|\\
|\input{childdoc.def}|\\
|\childdocforwardprefix{final}{child}|
\end{tabular}
\end{center}
%

Note that when several versions of a main file and/or of each child file
are to be generated, it may be convenient to set up a |Makefile| or
shell script to automatise the process.

%%%%%%%%%%%%%%%%%%%%%%%%%%%%%%%%%%%%%%%%%%%%%%%%%%%%%%%%%%%%%%%%%%%%%%%%%%%%%%%%
\subsection{Command Line Processing}
\label{sec:commandline}

The effect of redirection files can also be achieved by invoking
the \LaTeX{} compiler with a more elaborate command line.
Most conveniently this should be done as part
of a shell script or a |Makefile|.

When using \textsf{childdoc} in the main file, the following
command lines effectively perform a redirection
(note that depending on the shell being used,
backslashes may have to be doubled: `|\|' $\to$ `|\\|'):
%
\begin{center}
|... -jobname "|\textit{target}|" |\\|"|[\textit{flags}]%
|\input{childdoc.def}\childdocforward[|\textit{main}|]{|\textit{dest}|}"|
\end{center}
%
Here \textit{target} is the name of the output file,
\textit{main} is the name of the main file
and \textit{dest} is the name of the main or child file to be processed
(all filenames without extensions).
The optional argument \textit{main} can be omitted
if \textit{main} matches \textit{dest}.
Optionally, compilation \textit{flags} can be defined via |\def| commands.
This command line makes the \TeX{} engine believe
it is compiling the file \textit{target}
whose content is specified as the latter parameter.
The provided code then forwards the processing to
\textit{main} or \textit{dest} as described in \secref{sec:forward}.

%%%%%%%%%%%%%%%%%%%%%%%%%%%%%%%%%%%%%%%%%%%%%%%%%%%%%%%%%%%%%%%%%%%%%%%%%%%%%%%%
\subsection{Include by Input}
\label{sec:input}

Including child documents by |\include| has some restrictions by design.
Most notably, the content of a child document always occupies
its own set of pages; pages cannot be shared between child documents.
Usually, this behaviour makes perfect sense
because each child document contain an essential part of the document.
However, in some situations it may be desirable to compose
a document from a collection of parts
without having mandatory page breaks between then.
For this case, the package
provides a mechanism to include parts
by |\input| which can also be processed individually.
However, by construction this mechanism
requires manual handling of the content to be output.

%%%%%%%%%%%%%%%%%%%%%%%%%%%%%%%%%%%%%%%%
\DescribeMacro{\ifchilddocmanual}
The main file should be prepared as usual, see \secref{sec:include}.
However, the document body must make a distinction
between processing of an individual part and of the main document, e.g.:
%
\begin{center}
\begin{tabular}{l}
|\ifchilddocmanual|\\
|\input{\childdocname}|\\
|\||else|\\
\textit{document body with }|\input{|\textit{part}|}|\\
|\||fi|
\end{tabular}
\end{center}
%
The conditional |\ifchilddocmanual| is true whenever
a part to be included by |\input| is being compiled,
and the name of the part is stored in |\childdocname|.

%%%%%%%%%%%%%%%%%%%%%%%%%%%%%%%%%%%%%%%%
\DescribeMacro{\childdocby}
Each part to be included by |\input| should start with:
%
\begin{center}
\begin{tabular}{l}
|\input{childdoc.def}|\\
|\childdocby{|\textit{main}|}|\\
\end{tabular}
\end{center}
%
The directive |\childdocby| is similar to |\childdocof|
described in \secref{sec:include},
but the subsequent selection of content must be done manually.
To that end, both |\ifchilddoc| and |\ifchilddocmanual|
will be true upon processing of a part,
and the name of the part is stored in |\childdocname|.
Note that |\jobname| will be set to the filename of the current part
so that each part receives an individual |.aux| file
that does not interfere with the |.aux| file(s) of the main document.
This behaviour can be altered by the alternative form
|\childdocby[*]{|\textit{main}|}| (with a non-empty optional argument)
which uses the |.aux| file of the main document
by setting |\jobname| to \textit{main}.

%%%%%%%%%%%%%%%%%%%%%%%%%%%%%%%%%%%%%%%%%%%%%%%%%%%%%%%%%%%%%%%%%%%%%%%%%%%%%%%%
\subsection{Driver Development}
\label{sec:driver}

The \textsf{childdoc} mechanism can also be use for the development
of definition files such as \LaTeX{} styles or classes.
This case differs from the above setup with multiple parts
included by |\include| in that no |\includeonly| should be invoked.
This can be achieved by starting the include file
(before |\ProvidesPackage|) with:
%
\begin{center}
\begin{tabular}{l}
|\input{childdoc.def}|\\
|\childdocforward{|\textit{main}|}|\\
\end{tabular}
\end{center}
%
or alternatively with:
%
\begin{center}
\begin{tabular}{l}
|\input{childdoc.def}|\\
|\childdocby{|\textit{main}|}|\\
\end{tabular}
\end{center}
%
Both forms have slightly different effects as described above.
The main file is prepared as usual, see \secref{sec:include}.

%%%%%%%%%%%%%%%%%%%%%%%%%%%%%%%%%%%%%%%%%%%%%%%%%%%%%%%%%%%%%%%%%%%%%%%%%%%%%%%%
\subsection{Legacy Detection}
\label{sec:detection}

The directive |\childdocmain| in the main file can detect
whether the complete document or merely a child is to be compiled
even without using the directive |\childdocof|.
This method is deprecated because it is less robust
and there is no compelling reason to use it;
it is merely provided for backward compatibility
and it may be removed in future versions.

If the detection mechanism is to be used,
it is mandatory to correctly specify
the filename of the main file as the argument of |\childdocmain|:
%
\begin{center}
\begin{tabular}{l}
|\input{childdoc.def}|\\
|\childdocmain{|\textit{main}|}|\\
\end{tabular}
\end{center}
%
If |\jobname| does not match the argument \textit{main} of |\childdocmain|,
it is assumed that |\jobname| points to the child file to be compiled.
When using |\childdocmain| with the main file specified as argument,
it suffices to start a child file
with just |\input{|\textit{main}|}|
without loading of the package and using |\childdocof|.
If instead all processing is done
with the appropriate \textsf{childdoc} directives,
the argument of \textit{main} of |\childdocmain| can be empty.

An alternative version of the command line processing described
in \secref{sec:commandline} using the detection mechanism reads:
%
\begin{center}
|... -jobname "|\textit{target}|" "|[\textit{flags}]%
[|\def\jobname{|\textit{dest}|}|]|\input{|\textit{main}|}"|
\end{center}

%%%%%%%%%%%%%%%%%%%%%%%%%%%%%%%%%%%%%%%%%%%%%%%%%%%%%%%%%%%%%%%%%%%%%%%%%%%%%%%%
\subsection{Manual Code}
\label{sec:manual}

In case one cannot be certain whether the definitions file |childdoc.def|
is installed on the target \TeX{} distribution
and one prefers not to ship it,
it is conceivable to paste a few relevant commands into the sources.

To that end, drop all statements |\input{childdoc.def}|
and perform the replacements as outlined below.
Instead of |\childdocmain{|\textit{main}|}| add the following code
to the top of the main file:
%
\begin{center}
\begin{tabular}{l}
|\||ifdefined\childdocname\endinput\||fi\newif\ifchilddoc|\\
|\edef\childdocname{\scantokens\expandafter{\jobname\noexpand}}|\\
|\def\childdocmain{|\textit{main}|}\||ifx\childdocmain\childdocname\||else|\\
|\childdoctrue\includeonly{\childdocname}\let\jobname\childdocmain\||fi|\\
\end{tabular}
\end{center}
%
Instead of |\childdocof{|\textit{main}|}| just include the main file
at the top of each child file:
%
\begin{center}
|\input{|\textit{main}|}|
\end{center}
%
A simple redirection |\childdocforward{|\textit{dest}|}| is achieved by:
%
\begin{center}
|\def\jobname{|\textit{dest}|}\input{\jobname}|
\end{center}
%
The redirection with prefix
|\childdocforwardprefix[|\textit{prefix}|]{|\textit{dest}|}|
is accomplished by:
%
\begin{center}
\begin{tabular}{l}
|{\edef\jobname{\scantokens\expandafter{\jobname\noexpand}}|\\
|\def\redirectjob |\textit{prefix}|#1~~~{\gdef\jobname{|\textit{dest}|#1}}|\\
|\expandafter\redirectjob\jobname~~~}\input{\jobname}|
\end{tabular}
\end{center}

In an alternative approach,
child documents can be compiled by a specific command line
without additional code or specific definitions:
%
\begin{center}
|... -jobname "|\textit{target}|" "|[\textit{flags}]%
|\includeonly{|\textit{dest}|}\input{|\textit{main}|}"|
\end{center}
%

%%%%%%%%%%%%%%%%%%%%%%%%%%%%%%%%%%%%%%%%%%%%%%%%%%%%%%%%%%%%%%%%%%%%%%%%%%%%%%%%
%%%%%%%%%%%%%%%%%%%%%%%%%%%%%%%%%%%%%%%%%%%%%%%%%%%%%%%%%%%%%%%%%%%%%%%%%%%%%%%%
\section{Information}

%%%%%%%%%%%%%%%%%%%%%%%%%%%%%%%%%%%%%%%%%%%%%%%%%%%%%%%%%%%%%%%%%%%%%%%%%%%%%%%%
\subsection{Copyright}

Copyright \copyright{} 2017--2018 Niklas Beisert

This work may be distributed and/or modified under the
conditions of the \LaTeX{} Project Public License, either version 1.3
of this license or (at your option) any later version.
The latest version of this license is in
  \url{http://www.latex-project.org/lppl.txt}
and version 1.3 or later is part of all distributions of \LaTeX{}
version 2005/12/01 or later.

This work has the LPPL maintenance status `maintained'.

The Current Maintainer of this work is Niklas Beisert.

This work consists of the files |README.txt|, |childdoc.ins| and |childdoc.dtx|
as well as the derived files |childdoc.def|, |cdocsamp.tex|
with |cdocsch1.tex|, |cdocsch2.tex|, |cdocspt3.tex|, |cdocspt4.tex|,
|cdocsdrf.tex|, |cdocsfn1.tex|, |cdocsfn2.tex|
as well as |childdoc.pdf|.

%%%%%%%%%%%%%%%%%%%%%%%%%%%%%%%%%%%%%%%%%%%%%%%%%%%%%%%%%%%%%%%%%%%%%%%%%%%%%%%%
\subsection{Files and Installation}

The package consists of the files:
%
\begin{center}
\begin{tabular}{ll}
    |README.txt|   & readme file \\
    |childdoc.ins| & installation file \\
    |childdoc.dtx| & source file \\
    |childdoc.def| & definition file \\
    |cdocsamp.tex| & sample main file \\
    |cdocsch1.tex| & sample include file \\
    |cdocsch2.tex| & sample include file \\
    |cdocspt3.tex| & sample part file \\
    |cdocspt4.tex| & sample part file \\
    |cdocsdrf.tex| & sample redirection file \\
    |cdocsfn1.tex| & sample redirection file \\
    |cdocsfn2.tex| & sample redirection file \\
    |childdoc.pdf| & manual
\end{tabular}
\end{center}
%
The distribution consists of the files
|README.txt|, |childdoc.ins| and |childdoc.dtx|.
%
\begin{itemize}
\item
Run (pdf)\LaTeX{} on |childdoc.dtx|
to compile the manual |childdoc.pdf| (this file).
\item
Run \LaTeX{} on |childdoc.ins| to create the definitions file |childdoc.def|
and the sample |cdocsamp.tex| with include files
|cdocsch1.tex|, |cdocsch2.tex|, |cdocspt3.tex|, |cdocspt4.tex|,
|cdocsdrf.tex|, |cdocsfn1.tex|, |cdocsfn2.tex|.
Then copy the file |childdoc.def| to an appropriate directory of your \LaTeX{}
distribution, e.g.\ \textit{texmf-root}|/tex/latex/childdoc|.
\end{itemize}

%%%%%%%%%%%%%%%%%%%%%%%%%%%%%%%%%%%%%%%%%%%%%%%%%%%%%%%%%%%%%%%%%%%%%%%%%%%%%%%%
\subsection{Related CTAN Packages}

There are several other packages which offer a similar functionality:
%
\begin{itemize}
\item
The packages
\href{http://ctan.org/pkg/docmute}{\textsf{docmute}},
\href{http://ctan.org/pkg/includex}{\textsf{includex}} and
\href{http://ctan.org/pkg/standalone}{\textsf{standalone}}
provide commands to include only the document body of
a child file thus allowing both files to be compiled individually.
\item
The packages \href{http://ctan.org/pkg/subdocs}{\textsf{subdocs}}
and \href{http://ctan.org/pkg/subfiles}{\textsf{subfiles}}
provide structures in which the main and child documents can be
encapsulated and allowing them to be compiled individually.
The inclusion mechanism is different from the conventional |\include|.
\item
The package \href{http://ctan.org/pkg/combine}{\textsf{combine}}
is an elaborate solution to combine several documents into one.
\end{itemize}
%
See also the CTAN topic \href{http://ctan.org/topic/subdocs}{\textsf{subdocs}}
for further related packages.
The present package differs from the above solutions in that
a document structure constructed with the conventional |\include| mechanism
just needs two extra commands at the top of every file
such that all constituent files can be compiled individually.

%%%%%%%%%%%%%%%%%%%%%%%%%%%%%%%%%%%%%%%%%%%%%%%%%%%%%%%%%%%%%%%%%%%%%%%%%%%%%%%%
%\subsection{Feature Suggestions}
%
%The following is a list of features which may be useful for future
%versions of this package:
%%
%\begin{itemize}
%\item
%\ldots
%\end{itemize}

%%%%%%%%%%%%%%%%%%%%%%%%%%%%%%%%%%%%%%%%%%%%%%%%%%%%%%%%%%%%%%%%%%%%%%%%%%%%%%%%
\subsection{Revision History}

%%%%%%%%%%%%%%%%%%%%%%%%%%%%%%%%%%%%%%%%
\paragraph{v2.0:} 2018/12/30

\begin{itemize}
\item
immediate forward processing
\item
added |\childdocby| mechanism
\item
manual restructured
\end{itemize}

%%%%%%%%%%%%%%%%%%%%%%%%%%%%%%%%%%%%%%%%
\paragraph{v1.6:} 2018/01/17

\begin{itemize}
\item
application for development of include files
\item
corrections to manual
\end{itemize}

%%%%%%%%%%%%%%%%%%%%%%%%%%%%%%%%%%%%%%%%
\paragraph{v1.5:} 2017/05/21

\begin{itemize}
\item
more complete structuring introduced
\item
|\childdocof| introduced
\item
|\childdoc| renamed to |\childdocmain|
\item
|\childredirect| renamed to |\childdocforward| and |\childdocforwardprefix|
and functionality expanded
\end{itemize}

%%%%%%%%%%%%%%%%%%%%%%%%%%%%%%%%%%%%%%%%
\paragraph{v1.0:} 2017/04/27

\begin{itemize}
\item
manual and install package
\item
first version published on CTAN
\end{itemize}

%%%%%%%%%%%%%%%%%%%%%%%%%%%%%%%%%%%%%%%%
\paragraph{v0.6:} 2017/04/26

\begin{itemize}
\item
redirection mechanism added
\end{itemize}

%%%%%%%%%%%%%%%%%%%%%%%%%%%%%%%%%%%%%%%%
\paragraph{v0.5:} 2017/04/26

\begin{itemize}
\item
functionality in definition file
\end{itemize}


%%%%%%%%%%%%%%%%%%%%%%%%%%%%%%%%%%%%%%%%%%%%%%%%%%%%%%%%%%%%%%%%%%%%%%%%%%%%%%%%
%%%%%%%%%%%%%%%%%%%%%%%%%%%%%%%%%%%%%%%%%%%%%%%%%%%%%%%%%%%%%%%%%%%%%%%%%%%%%%%%
%%%%%%%%%%%%%%%%%%%%%%%%%%%%%%%%%%%%%%%%%%%%%%%%%%%%%%%%%%%%%%%%%%%%%%%%%%%%%%%%
\appendix

\settowidth\MacroIndent{\rmfamily\scriptsize 000\ }

 \DocInput{childdoc.dtx}

\end{document}
%</driver>
% \fi
%
% %%%%%%%%%%%%%%%%%%%%%%%%%%%%%%%%%%%%%%%%%%%%%%%%%%%%%%%%%%%%%%%%%%%%%%%%%%%%%%
% %%%%%%%%%%%%%%%%%%%%%%%%%%%%%%%%%%%%%%%%%%%%%%%%%%%%%%%%%%%%%%%%%%%%%%%%%%%%%%
% \section{Sample}
%\iffalse
%<*samplemain>
%\fi
%
% The following presents a sample document
% with two chapters, two parts, a title page,
% a compile flag as well as three forwarding files to set the flag.
% It consists of eight |.tex| files:
% \begin{center}
% \begin{tabular}{ll}
% |cdocsamp.tex|&main file\\
% |cdocsch1.tex|&include file for chapter 1\\
% |cdocsch2.tex|&include file for chapter 2\\
% |cdocspt3.tex|&include file for part 3\\
% |cdocspt4.tex|&include file for part 4\\
% |cdocsdrf.tex|&forwarding file for main file in draft mode\\
% |cdocsfi1.tex|&forwarding file for final version of chapter 1\\
% |cdocsfi2.tex|&forwarding file for final version of chapter 2\\
% \end{tabular}
% \end{center}
% Each of the eight files can be compiled directly by the \LaTeX{} compiler.
%
% %%%%%%%%%%%%%%%%%%%%%%%%%%%%%%%%%%%%%%
% \paragraph{Main File.}
%
% The main file is called |cdocsamp.tex|.
%
% Load the \textsf{childdoc} definitions and
% declare the filename for the main document:
%    \begin{macrocode}
\input{childdoc.def}
\childdocmain{}
%    \end{macrocode}

% Optional override for |\version| flag:
%    \begin{macrocode}
%%\ifchilddoc\else\providecommand{\version}{draft}\fi
%    \end{macrocode}

% Define the default values for the |\version| flag
% (|final| for the main file and |draft| for childs):
%    \begin{macrocode}
\ifchilddoc
\providecommand{\version}{draft}
\else
\providecommand{\version}{final}
\fi
%    \end{macrocode}

% Load the standard document class:
%    \begin{macrocode}
\documentclass[12pt]{article}
%    \end{macrocode}

% Start the document body:
%    \begin{macrocode}
\begin{document}
%    \end{macrocode}

% Declare a title page.
% Print title, part of document being processed and version flag:
%    \begin{macrocode}
\addtocounter{page}{-1}
\begin{center}
{\LARGE\bfseries{}childdoc example\par}
\vspace{1cm}
\ifchilddoc
\ifchilddocmanual part\else chapter\fi:
`\childdocname' of `\childdocjob'\par
\else
main document: `\childdocjob'\par
\fi
version: \version\par
\end{center}
\newpage
%    \end{macrocode}

% Manually include selected file,
% otherwise process as usual:
%    \begin{macrocode}
\ifchilddocmanual
\section*{part `\childdocname'}
\input{\childdocname}
\else
%    \end{macrocode}

% Include the two chapters:
%    \begin{macrocode}
\include{cdocsch1}
\include{cdocsch2}
%    \end{macrocode}

% Include the two parts unless only chapters should be displayed:
%    \begin{macrocode}
\ifchilddoc\else
\section{part three}
\input{cdocspt3}
\section{part four}
\input{cdocspt4}
\fi
%    \end{macrocode}

% Process as usual until here:
%    \begin{macrocode}
\fi
%    \end{macrocode}

% End of document body:
%    \begin{macrocode}
\end{document}
%    \end{macrocode}
%\iffalse
%</samplemain>
%\fi
%
% %%%%%%%%%%%%%%%%%%%%%%%%%%%%%%%%%%%%%%
% \paragraph{Chapter Include Files.}
%
% The include files are called |cdocsch1.tex| and |cdocsch2.tex|.
%
%\iffalse
%<*samplechap1|samplechap2>
%\fi

% Optional override for |\version| flag:
%    \begin{macrocode}
%%\providecommand{\version}{final}
%    \end{macrocode}

% Include the main document:
%    \begin{macrocode}
\input{childdoc.def}
\childdocof{cdocsamp}
%    \end{macrocode}

%\iffalse
%</samplechap1|samplechap2>
%\fi
%
%\iffalse
%<*samplechap1>
%\fi
% Some text for chapter 1:
%    \begin{macrocode}
\section{one}
some text in chapter one
%    \end{macrocode}

%\iffalse
%</samplechap1>
%\fi
% Some text for chapter 2:
%\iffalse
%<*samplechap2>
%\fi
%    \begin{macrocode}
\section{two}
more text in chapter two
%    \end{macrocode}

%\iffalse
%</samplechap2>
%\fi
%
% %%%%%%%%%%%%%%%%%%%%%%%%%%%%%%%%%%%%%%
% \paragraph{Part Include Files.}
%
% The include files are called |cdocspt3.tex| and |cdocspt4.tex|.
%
%\iffalse
%<*samplepart3|samplepart4>
%\fi

% Optional override for |\version| flag:
%    \begin{macrocode}
%%\providecommand{\version}{final}
%    \end{macrocode}

% Include the main document:
%    \begin{macrocode}
\input{childdoc.def}
\childdocby{cdocsamp}
%    \end{macrocode}

%\iffalse
%</samplepart3|samplepart4>
%\fi
%
%\iffalse
%<*samplepart3>
%\fi
% Some text for part 3:
%    \begin{macrocode}
some text in part three
%    \end{macrocode}

%\iffalse
%</samplepart3>
%\fi
% Some text for part 4:
%\iffalse
%<*samplepart4>
%\fi
%    \begin{macrocode}
more text in part four
%    \end{macrocode}

%\iffalse
%</samplepart4>
%\fi
%
% %%%%%%%%%%%%%%%%%%%%%%%%%%%%%%%%%%%%%%
% \paragraph{Forwarding for a Complete Draft.}
%
% The following forwarding file |cdocsdrf.tex|
% compiles the main document in draft mode:
%\iffalse
%<*sampledraft>
%\fi
%    \begin{macrocode}
\def\version{draft}
\input{childdoc.def}
\childdocforward{cdocsamp}
%    \end{macrocode}

%\iffalse
%</sampledraft>
%\fi
%
% %%%%%%%%%%%%%%%%%%%%%%%%%%%%%%%%%%%%%%
% \paragraph{Forwarding for Final Version of the Chapters.}
%
% The following forwarding files |cdocsfn1.tex| and |cdocsfn2.tex|
% (with identical content)
% compile the final versions of the child documents
% |cdocsch1.tex| and |cdocsch2.tex|, respectively:
%\iffalse
%<*samplefinal>
%\fi
%    \begin{macrocode}
\def\version{final}
\input{childdoc.def}
\childdocforwardprefix[cdocsamp]{cdocsfn}{cdocsch}
%    \end{macrocode}

%\iffalse
%</samplefinal>
%\fi
%
% %%%%%%%%%%%%%%%%%%%%%%%%%%%%%%%%%%%%%%
% \paragraph{Command Line Processing.}
%
% The following three command lines generate the output files
% |cdocscld|, |cdocscl1| and |cdocscl2|
% which should be identical to
% |cdocsdrf|, |cdocsch1| and |cdocsfn2|, respectively:
% \begin{center}
% \begin{tabular}{l}
% |latex -jobname cdocscld \|\\
% |  "\def\version{draft}\input{childdoc.def}\childdocforward{cdocsamp}"|\\
% |latex -jobname cdocscl1 \|\\
% |  "\input{childdoc.def}\childdocforward[cdocsamp]{cdocsch1}"|\\
% |latex -jobname cdocscl2 \|\\
% |  "\def\version{final}\input{childdoc.def}\childdocforward{cdocsch2}"|
% \end{tabular}
% \end{center}
% Note that the trailing backslash on each first line
% merely continues the input to the second line
% (for convenient cut ant paste).
% Furthermore, the command |latex| can be replaced by any
% of its alternative versions such as |pdflatex|.
%
% %%%%%%%%%%%%%%%%%%%%%%%%%%%%%%%%%%%%%%%%%%%%%%%%%%%%%%%%%%%%%%%%%%%%%%%%%%%%%%
% %%%%%%%%%%%%%%%%%%%%%%%%%%%%%%%%%%%%%%%%%%%%%%%%%%%%%%%%%%%%%%%%%%%%%%%%%%%%%%
% \section{Implementation}
%\iffalse
%<*package>
%\fi
%
% This section describes the definitions file |childdoc.def|.

% The definitions cannot be loaded using |\usepackage| or |\RequirePackage|
% which has a mechanism to prevent loading a style file more than once.
% When loading the definitions by means of |\input|
% multiple instances have to be prevented manually:
%\iffalse
%This code needs to be before the `\ProvidesFile' directive
%which is defined at the beginning of this file.
%Therefore it is also placed there and commented out here.
%</package>
%<*discard>
%\fi
%    \begin{macrocode}
\ifdefined\childdocmain\endinput\fi
%    \end{macrocode}
%\iffalse
%</discard>
%<*package>
%\fi
%
% \macro{\ifchilddoc}
% \macro{\ifchilddocmanual}
% The conditional |\ifchilddoc| tells whether a
% child (true) or main (false) document is being compiled.
% The conditional |\ifchilddocmanual| tells whether
% the |\includeonly| mechanism is used (false) or
% the selection of child files must be performed manually (true).
% The definitions initialise to false:
%    \begin{macrocode}
\newif\ifchilddoc
\newif\ifchilddocmanual
%    \end{macrocode}

% \macro{\childdocname}
% \macro{\childdocjob}
% The macro |\childdocname| stores the name of the main document
% to be compiled. The macro |\childdocjob| stores the name of
% the document on which the \LaTeX{} compiler was originally invoked.
% The content of |\jobname| cannot be compared
% to filenames specified in the source due to different catcodes.
% The following code rescans |\jobname|, stores the result
% in |\childdocname| and saves a copy in |\childdocjob|:
%    \begin{macrocode}
\edef\childdocname{\scantokens\expandafter{\jobname\noexpand}}
\let\childdocjob\childdocname
%    \end{macrocode}

% \macro{\childdocdisable}
% The macro |\childdocdisable| prevents the main file
% from being processed more than once.
% At this stage, the main document command |\childdocmain|
% is assumed to be called once again where it should do nothing.
% Any subsequent call to it should prevent
% a secondary processing of the main document
% It overwrites the forwarding commands
% |\childdocof| and |\childdocforward|
% with empty macros to prevent further inclusions of the main document:
%    \begin{macrocode}
\newcommand{\childdocdisable}
{
  \renewcommand{\childdocmain}[1]{\renewcommand{\childdocmain}[1]{\endinput}}
  \renewcommand{\childdocof}[1]{}
  \renewcommand{\childdocby}[2][]{}
  \renewcommand{\childdocforward}[2][]{}
  \renewcommand{\childdocdisable}{}
}
%    \end{macrocode}

% \macro{\childdocmain}
% The macro |\childdocmain| is to be called at the top of the main file
% with nothing or the main filename (without extension) as argument.
% First, it breaks loops.
% If the argument is not empty and does not match |\childdocname|
% (which is set by the first inclusion of |childdoc.def|),
% |\ifchilddoc| is set to true, |\includeonly| is applied to the child file
% and |\jobname| is set to the main file
% (for proper handling of |.aux| files):
%    \begin{macrocode}
\newcommand{\childdocmain}[1]
{
  \childdocdisable\childdocmain{}
  \if?#1?\else
    \begingroup
      \def\childdoctmp{#1}
      \ifx\childdoctmp\childdocname
        \def\childdoctmp{}
      \else
        \def\childdoctmp
        {
          \childdoctrue
          \includeonly{\childdocname}
          \def\childdocjob{#1}
          \def\jobname{#1}
        }
      \fi
      \expandafter
    \endgroup
    \childdoctmp
  \fi
}
%    \end{macrocode}

% \macro{\childdocof}
% The command |\childdocof| redirects
% compilation to the main file |#1|.
%    \begin{macrocode}
\newcommand{\childdocof}[1]
{
  \childdocdisable
  \childdoctrue
  \includeonly{\childdocname}
  \def\jobname{#1}
  \def\childdocjob{#1}
  \input{#1}
}
%    \end{macrocode}

% \macro{\childdocby}
% The command |\childdocby| ....
%    \begin{macrocode}
\newcommand{\childdocby}[2][]
{
  \childdocdisable
  \childdoctrue
  \childdocmanualtrue
  \if?#1?\else
    \def\jobname{#2}
  \fi
  \def\childdocjob{#2}
  \input{#2}
  \endinput
}
%    \end{macrocode}

% \macro{\childdocforward}
% The command |\childdocforward| redirects
% compilation to the main file or
% (if the optional argument is given) a child file.
% Parameters are set as if the main file
% or a child file starting with |\childdocof| was compiled.
% Then compilation is handed over to the main file:
%    \begin{macrocode}
\newcommand{\childdocforward}[2][]
{
  \begingroup
    \if?#1?
      \def\childdoctmp
      {
        \def\childdocname{#2}
        \def\childdocjob{#2}
        \def\jobname{#2}
        \input{#2}
        \endinput
      }
    \else
      \def\childdoctmp
      {
        \childdocdisable
        \def\childdocname{#2}
        \childdoctrue
        \includeonly{#2}
        \def\childdocjob{#1}
        \def\jobname{#1}
        \input{#1}
        \endinput
      }
    \fi
    \expandafter
  \endgroup
  \childdoctmp
}
%    \end{macrocode}

% \macro{\childdocforwardprefix}
% The command |\childdocforwardprefix| redirects
% compilation to the main or a child file by means of a pattern.
% The prefix |#1| in the current filename is replaced by |#2|
% and the suffix of the current filename is kept
% (it is assumed that the filename does not contain the substring `|~~~|'
% which is used as a delimiter).
% Compilation is handed over to the new file by |\childdocforward|:
%    \begin{macrocode}
\newcommand{\childdocforwardprefix}[3][]
{
  \begingroup
    \def\childdocextract #2##1~~~{\def\childdoctmp{\childdocforward[#1]{#3##1}}}
    \expandafter\childdocextract\childdocname~~~
    \expandafter
  \endgroup
  \childdoctmp
}
%    \end{macrocode}

% \macro{\childdoc}
% The deprecated macro |\childdoc| is a legacy version of |\childdocmain|:
%    \begin{macrocode}
\newcommand{\childdoc}{\childdocmain}
%    \end{macrocode}

% \macro{\childdocredirect}
% The deprecated macro |\childdocredirect| is a legacy version
% of |\childdocforward| and |\childdocforwardprefix|:
%    \begin{macrocode}
\newcommand{\childdocredirect}[2][]
{
  \begingroup
    \if?#1?
      \def\childdoctmp{\childdocforward{#2}}
    \else
      \def\childdoctmp{\childdocforwardprefix{#1}{#2}}
    \fi
    \expandafter
  \endgroup
  \childdoctmp
}
%    \end{macrocode}

%\iffalse
%</package>
%\fi
%
\endinput
\childdocforward[cdocsamp]{cdocsch1}"|\\
% |latex -jobname cdocscl2 \|\\
% |  "\def\version{final}% \iffalse
%
% childdoc.dtx Copyright (C) 2017-2018 Niklas Beisert
%
% This work may be distributed and/or modified under the
% conditions of the LaTeX Project Public License, either version 1.3
% of this license or (at your option) any later version.
% The latest version of this license is in
%   http://www.latex-project.org/lppl.txt
% and version 1.3 or later is part of all distributions of LaTeX
% version 2005/12/01 or later.
%
% This work has the LPPL maintenance status `maintained'.
%
% The Current Maintainer of this work is Niklas Beisert.
%
% This work consists of the files childdoc.dtx and childdoc.ins
% and the derived files childdoc.def and cdocsamp.tex with
% cdocsch1.tex, cdocsch2.tex, cdocsdrf.tex, cdocsfn1.tex, cdocsfn2.tex.
%
%<package>\ifdefined\childdocmain\endinput\fi
%<package>\ProvidesFile{childdoc.def}[2018/12/30 v2.0 child document driver]
%<samplemain>\ProvidesFile{cdocsamp.tex}[2018/12/30 v2.0 sample for childdoc]
%<*driver>
%\ProvidesFile{childdoc.drv}[2018/12/30 v2.0 childdoc reference manual file]
\PassOptionsToClass{10pt,a4paper}{article}
\documentclass{ltxdoc}

\usepackage[margin=35mm]{geometry}
\usepackage{hyperref}
\usepackage{hyperxmp}
\usepackage[usenames]{color}

\hypersetup{colorlinks=true}
\hypersetup{pdfstartview=FitH}
\hypersetup{pdfpagemode=UseNone}
\hypersetup{pdfsource={}}
\hypersetup{pdflang={en-UK}}
\hypersetup{pdfcopyright={Copyright 2017-2018 Niklas Beisert.
  This work may be distributed and/or modified under the
  conditions of the LaTeX Project Public License, either version 1.3
  of this license or (at your option) any later version.}}
\hypersetup{pdflicenseurl={http://www.latex-project.org/lppl.txt}}
\hypersetup{pdfcontactaddress={ETH Zurich, ITP, HIT K,
  Wolfgang-Pauli-Strasse 27}}
\hypersetup{pdfcontactpostcode={8093}}
\hypersetup{pdfcontactcity={Zurich}}
\hypersetup{pdfcontactcountry={Switzerland}}
\hypersetup{pdfcontactemail={nbeisert@itp.phys.ethz.ch}}
\hypersetup{pdfcontacturl={http://people.phys.ethz.ch/\xmptilde nbeisert/}}

\newcommand{\secref}[1]{\hyperref[#1]{section \ref*{#1}}}

\parskip1ex
\parindent0pt
\let\olditemize\itemize
\def\itemize{\olditemize\parskip0pt}

\begin{document}

\title{The \textsf{childdoc} Package}
\hypersetup{pdftitle={The childdoc Package}}
\author{Niklas Beisert\\[2ex]
  Institut f\"ur Theoretische Physik\\
  Eidgen\"ossische Technische Hochschule Z\"urich\\
  Wolfgang-Pauli-Strasse 27, 8093 Z\"urich, Switzerland\\[1ex]
  \href{mailto:nbeisert@itp.phys.ethz.ch}
  {\texttt{nbeisert@itp.phys.ethz.ch}}}
\hypersetup{pdfauthor={Niklas Beisert}}
\hypersetup{pdfsubject={Manual for the LaTeX2e Package childdoc}}
\date{30 December 2018, \textsf{v2.0}}
\maketitle

\begin{abstract}\noindent
\textsf{childdoc} is a \LaTeXe{} package
that enables the direct compilation
of document sections included by |\include|
to individual files.
\end{abstract}

\begingroup
\parskip0ex
\tableofcontents
\endgroup

%%%%%%%%%%%%%%%%%%%%%%%%%%%%%%%%%%%%%%%%%%%%%%%%%%%%%%%%%%%%%%%%%%%%%%%%%%%%%%%%
%%%%%%%%%%%%%%%%%%%%%%%%%%%%%%%%%%%%%%%%%%%%%%%%%%%%%%%%%%%%%%%%%%%%%%%%%%%%%%%%
\section{Introduction}

\LaTeX{} provides a mechanism to structure a large document (such as a book)
into a main file and several child files (containing the chapters)
using the |\include| command.
This mechanism is beneficial for documents
which span hundreds of pages in order to
make the source file(s) more manageable.
Moreover, compilation can be restricted to
selected child files by means of the |\includeonly| command.
The latter feature can be used to reduce the compilation time while editing
(this was significantly more useful in the earlier days of \LaTeX{})
or to generate a smaller document which is easier to navigate.
Another application of |\includeonly| is to generate
documents consisting of selected parts of the complete document.

However, there are a few drawbacks of the plain |\include| mechanism:
\begin{itemize}
\item
The child files cannot be compiled on their own,
they can only be compiled via the main file.
A naive editing environment
(such as a text editor with an option
to have the current file processed by \LaTeX)
may require one to switch to the main file before compiling;
attempting to compile the child file produces errors.
\item
The main file must be modified (each time)
to adjust the |\includeonly| command
to the present needs. This easily leaves the main file in a messy state.
\item
The generated document will always carry the filename
of the main document. This is inconvenient if
several child files are to be compiled and
to be kept for distribution.
\end{itemize}

The present package provides a simple interface
to make child files individually compilable by \LaTeX{}.
Compiling a child file then has the same effect as compiling
the main file with an |\includeonly| command
to select the appropriate child.
Moreover the generated document will carry the name of the child
rather than the main file.
This resolves all three above issues.

This feature is meant to make the editing of books,
thesis documents and lecture notes somewhat more convenient.
However, the package can also be used efficiently for
composing a series of documents (such as exercise sheets)
which are typically distributed individually.
It then assists the author in generating the individual documents
(potentially in different versions)
as well as a document containing the collected series.
Another application is in developing style files
or other kinds of included material
where compilation of the style file could redirect
to a sample or test file.

%%%%%%%%%%%%%%%%%%%%%%%%%%%%%%%%%%%%%%%%%%%%%%%%%%%%%%%%%%%%%%%%%%%%%%%%%%%%%%%%
%%%%%%%%%%%%%%%%%%%%%%%%%%%%%%%%%%%%%%%%%%%%%%%%%%%%%%%%%%%%%%%%%%%%%%%%%%%%%%%%
\section{Usage}

First of all, the package \textsf{childdoc} is \emph{not} a standard
\LaTeXe{} |.sty| style file! Therefore it needs to be invoked in
a non-standard way.

%%%%%%%%%%%%%%%%%%%%%%%%%%%%%%%%%%%%%%%%%%%%%%%%%%%%%%%%%%%%%%%%%%%%%%%%%%%%%%%%
\subsection{Included Files}
\label{sec:include}

%%%%%%%%%%%%%%%%%%%%%%%%%%%%%%%%%%%%%%%%
\DescribeMacro{\childdocmain}
To use the package, add the commands
\begin{center}
\begin{tabular}{l}
|\input{childdoc.def}|\\
|\childdocmain{}|\\
\end{tabular}
\end{center}
at the very top of the main \LaTeX{} file,
in particular \emph{before} the |\documentclass| statement!
The argument of |\childdocmain| should be left empty
(but it must be present).

%%%%%%%%%%%%%%%%%%%%%%%%%%%%%%%%%%%%%%%%
\DescribeMacro{\childdocof}
Furthermore, add the commands
\begin{center}
\begin{tabular}{l}
|\input{childdoc.def}|\\
|\childdocof{|\textit{main}|}|\\
\end{tabular}
\end{center}
at the top of every child file \textit{child}
which is included by |\include{|\textit{child}|}|
from within the main file
(or at least for those files to be compiled individually).
The argument \textit{main} must be the filename of the main file.

There are a couple of
considerations in setting up the main and child documents:

%%%%%%%%%%%%%%%%%%%%%%%%%%%%%%%%%%%%%%%%
\paragraph{Restrictions.}

Please note the following restrictions:
\begin{itemize}
\item
|\childdocmain| must be called with one argument \textit{main}
to ensure compatibility with earlier version of the package.
It must either be empty (|\childdocmain{}|)
or precisely match the filename of the main file in which it is specified.
See \secref{sec:detection} for further information.
\item
The filename \textit{main} must be specified without the |.tex| extension.
\item
The filename \textit{main} is case sensitive
(even in case-insensitive file systems)
due to internal string comparison.
\item
The argument \textit{main} should be fully expanded, it cannot be a macro.
\item
Subdirectories and special characters should be avoided in filenames.
\item
The command |\childdocmain{|\textit{main}|}| must be followed by a whitespace.
It should not be followed immediately by another command
or by a comment mark `|%|'.
This is because the \TeX{} parser reads the token immediately following
the argument of |\childdocmain| and puts it
at the beginning of every child section;
however, a white\-space is ignored.
\end{itemize}

%%%%%%%%%%%%%%%%%%%%%%%%%%%%%%%%%%%%%%%%
\paragraph{Content of Main File.}

It is advisable to place all content in the child files included by |\include|.
Any output contained in the main file will appear in all child documents
unless suppressed manually;
it cannot be suppressed automatically by the |\includeonly| directive
and thus should normally be avoided.
A method to include some content in the main file
by means of conditional processing is described in \secref{sec:conditional}.

%%%%%%%%%%%%%%%%%%%%%%%%%%%%%%%%%%%%%%%%
\paragraph{Page Numbering.}

When only a part of the document is compiled,
the appropriate numbering of pages
(as well as other status parameters)
is determined from the |.aux| files.
The latter contain information from previous passes.
However this information needs to propagate through
all intermediate child documents.
Therefore the page numbering in child documents may well
be inconsistent until the complete document is compiled at least once.

A useful (if unconventional) way to always ensure a consistent
page numbering is to restart the numbering in each child document
and denote the pages by `\textit{child}|.|\textit{page}'
where \textit{child} represents the chapter/section number of the child file.
This can be achieved by the command
|\numberwithin{page}{|\textit{child}|}|
of the \textsf{amsmath} package
where \textit{child} can be |chapter| or |section|
depending on the chosen structuring.
Alternatively, one can modify the macro |\thepage| appropriately
and reset the counter |page| at the start of each child file.

%%%%%%%%%%%%%%%%%%%%%%%%%%%%%%%%%%%%%%%%%%%%%%%%%%%%%%%%%%%%%%%%%%%%%%%%%%%%%%%%
\subsection{Conditional Processing}
\label{sec:conditional}

The package provides a mechanism to compile different versions
of a document. To customise the versions further some conditional processing
can come in handy to distinguish which version is being compiled.
The package provides two macros to describe the compilation context:

%%%%%%%%%%%%%%%%%%%%%%%%%%%%%%%%%%%%%%%%
\DescribeMacro{\ifchilddoc}
The conditional |\ifchilddoc| distinguishes between the compilation of
child documents and the main document:
%
\begin{center}
|\ifchilddoc |\textit{child-code}| |[|\||else |\textit{main-code}]| \||fi|
\end{center}

%%%%%%%%%%%%%%%%%%%%%%%%%%%%%%%%%%%%%%%%
\DescribeMacro{\childdocname}
\DescribeMacro{\childdocjob}
The macro |\childdocname| contains the filename (without extension)
of the main or child file being processed.
Note that |\childdocjob| will always contain the name of the main file.

%%%%%%%%%%%%%%%%%%%%%%%%%%%%%%%%%%%%%%%%
\paragraph{Title Page.}

Conditional processing can be used to include a title or banner page
in the main document when proper precautions are taken.
Importantly, the code in the main file should ensure that the page counter
(as well as other status parameters which are stored in the |.aux| files)
takes the same value after the conditional processing.
Otherwise the page numbers may take divergent values
depending on which part is compiled.

For example, a title page could be declared by:
%
\begin{center}
\begin{tabular}{l}
|\ifchilddoc\||else|\\
|\addtocounter{page}{-1}|\\
\textit{code for title page}\\
|\newpage|\\
|\||fi|
\end{tabular}
\end{center}
%
A banner page for the child documents can be generated by:
%
\begin{center}
\begin{tabular}{l}
|\ifchilddoc|\\
|\addtocounter{page}{-1}|\\
\textit{code for banner page}\\
|\newpage|\\
|\||fi|
\end{tabular}
\end{center}
%
Here one could write a message such as:
\begin{center}
|This is the part \childdocname{} of \childdocjob{}.|
\end{center}

%%%%%%%%%%%%%%%%%%%%%%%%%%%%%%%%%%%%%%%%%%%%%%%%%%%%%%%%%%%%%%%%%%%%%%%%%%%%%%%%
\subsection{Flags}
\label{sec:flags}

The package makes it easy to generate different versions
of the main or child documents.
To this end compilation flags can be defined
and assigned different default values.
They will be particularly useful in conjunction
with the forwarding mechanism described in \secref{sec:forward}.

For example, it may be useful to have a flag |\version|
which can be set to |draft| or |final|.
The document source will contain some conditional code
depending on the value of |\version|.
Suppose further, the flag should default to |final| for the main file
and to |draft| for child files
which is a natural assignment for editing the document.
This is achieved by placing the following code
in the preamble of the main document
(below the |\childdocmain| directive):
%
\begin{center}
\begin{tabular}{l}
|\ifchilddoc|\\
|\providecommand{\version}{draft}|\\
|\||else|\\
|\providecommand{\version}{final}|\\
|\||fi|
\end{tabular}
\end{center}
%
The definition by |\providecommand| makes sure
that previous definitions are not overwritten.
Further statements |\providecommand{\version}{...}|
can thus be added before the above code to override it.

For the main file, one might add a line
(between |\childdocmain| and the above block)
%
\begin{center}
|%\ifchilddoc\||else\providecommand{\version}{draft}\||fi|
\end{center}
%
which can be uncommented to produce a draft version.
Likewise one can add a line to the very top of a child file
(above the |\childdocof{|\textit{main}|}| directive)
%
\begin{center}
|%\providecommand{\version}{final}|
\end{center}
%
which can be uncommented to produce the final version of this child document.

%%%%%%%%%%%%%%%%%%%%%%%%%%%%%%%%%%%%%%%%%%%%%%%%%%%%%%%%%%%%%%%%%%%%%%%%%%%%%%%%
\subsection{Forwarding}
\label{sec:forward}

Different versions of the main or child documents
using compilation flags as described in \secref{sec:flags}
can be (permanently) stored in different files
for convenient compilation, viewing and distribution.
To this end, the package defines a command
to pass on compilation to a different file:

%%%%%%%%%%%%%%%%%%%%%%%%%%%%%%%%%%%%%%%%
\DescribeMacro{\childdocforward}
The command |\childdocforward| redirects processing to
another source file:
%
\begin{center}
\begin{tabular}{l}
|\input{childdoc.def}|\\
|\childdocforward[|\textit{main}|]{|\textit{dest}|}|\\
\end{tabular}
\end{center}
%
The argument \textit{dest} is the destination file
(without extension).
It should be the main file or one of the child files.
Note that further \textsf{childdoc} directives
such as |\childdocof| and |\childdocforward|
in the indicated file will be processed in this form.
The optional argument \textit{main}
passes on directly to the main file \textit{main}
while pretending to compile the child \textit{dest}.
This form behaves as if \textit{dest}
issues |\childdocof{|\textit{main}|}| right away,
and no further \textsf{childdoc} directives will be processed.

%%%%%%%%%%%%%%%%%%%%%%%%%%%%%%%%%%%%%%%%
\DescribeMacro{\...prefix}
In the alternative form |\childdocforwardprefix|,
%
\begin{center}
\begin{tabular}{l}
|\input{childdoc.def}|\\
|\childdocforwardprefix[|\textit{main}|]{|\textit{prefix}|}{|\textit{dest}|}|
\end{tabular}
\end{center}
%
the destination file is determined by a pattern
depending on the current file:
To make this work, the current file must be called
`{\textit{prefix}\hspace{0.2em}\textit{suffix}}'
with \textit{prefix} matching precisely the argument.
Processing is then passed on to the file
`{\textit{dest}\hspace{0.2em}\textit{suffix}}'.
Surely, the same effect is achieved by
directly specifying the
argument `{\textit{dest}\hspace{0.2em}\textit{suffix}}'
in the first form.
However, that requires to set up a different file
for each child. With the alternative form of the command
all these files can have exactly the same content
which simplifies setting them up and maintaining them.

For example, the following file |draft.tex|
with a compilation flag |\version| as described in \secref{sec:flags}
compiles the main document as a draft:
%
\begin{center}
\begin{tabular}{l}
|\def\version{draft}|\\
|\input{childdoc.def}|\\
|\childdocforward{|\textit{main}|}|
\end{tabular}
\end{center}
%
Likewise, the following files |final|\textit{nn}|.tex|
compile the final version of the child document
|child|\textit{nn}|.tex|:
%
\begin{center}
\begin{tabular}{l}
|\def\version{final}|\\
|\input{childdoc.def}|\\
|\childdocforwardprefix{final}{child}|
\end{tabular}
\end{center}
%

Note that when several versions of a main file and/or of each child file
are to be generated, it may be convenient to set up a |Makefile| or
shell script to automatise the process.

%%%%%%%%%%%%%%%%%%%%%%%%%%%%%%%%%%%%%%%%%%%%%%%%%%%%%%%%%%%%%%%%%%%%%%%%%%%%%%%%
\subsection{Command Line Processing}
\label{sec:commandline}

The effect of redirection files can also be achieved by invoking
the \LaTeX{} compiler with a more elaborate command line.
Most conveniently this should be done as part
of a shell script or a |Makefile|.

When using \textsf{childdoc} in the main file, the following
command lines effectively perform a redirection
(note that depending on the shell being used,
backslashes may have to be doubled: `|\|' $\to$ `|\\|'):
%
\begin{center}
|... -jobname "|\textit{target}|" |\\|"|[\textit{flags}]%
|\input{childdoc.def}\childdocforward[|\textit{main}|]{|\textit{dest}|}"|
\end{center}
%
Here \textit{target} is the name of the output file,
\textit{main} is the name of the main file
and \textit{dest} is the name of the main or child file to be processed
(all filenames without extensions).
The optional argument \textit{main} can be omitted
if \textit{main} matches \textit{dest}.
Optionally, compilation \textit{flags} can be defined via |\def| commands.
This command line makes the \TeX{} engine believe
it is compiling the file \textit{target}
whose content is specified as the latter parameter.
The provided code then forwards the processing to
\textit{main} or \textit{dest} as described in \secref{sec:forward}.

%%%%%%%%%%%%%%%%%%%%%%%%%%%%%%%%%%%%%%%%%%%%%%%%%%%%%%%%%%%%%%%%%%%%%%%%%%%%%%%%
\subsection{Include by Input}
\label{sec:input}

Including child documents by |\include| has some restrictions by design.
Most notably, the content of a child document always occupies
its own set of pages; pages cannot be shared between child documents.
Usually, this behaviour makes perfect sense
because each child document contain an essential part of the document.
However, in some situations it may be desirable to compose
a document from a collection of parts
without having mandatory page breaks between then.
For this case, the package
provides a mechanism to include parts
by |\input| which can also be processed individually.
However, by construction this mechanism
requires manual handling of the content to be output.

%%%%%%%%%%%%%%%%%%%%%%%%%%%%%%%%%%%%%%%%
\DescribeMacro{\ifchilddocmanual}
The main file should be prepared as usual, see \secref{sec:include}.
However, the document body must make a distinction
between processing of an individual part and of the main document, e.g.:
%
\begin{center}
\begin{tabular}{l}
|\ifchilddocmanual|\\
|\input{\childdocname}|\\
|\||else|\\
\textit{document body with }|\input{|\textit{part}|}|\\
|\||fi|
\end{tabular}
\end{center}
%
The conditional |\ifchilddocmanual| is true whenever
a part to be included by |\input| is being compiled,
and the name of the part is stored in |\childdocname|.

%%%%%%%%%%%%%%%%%%%%%%%%%%%%%%%%%%%%%%%%
\DescribeMacro{\childdocby}
Each part to be included by |\input| should start with:
%
\begin{center}
\begin{tabular}{l}
|\input{childdoc.def}|\\
|\childdocby{|\textit{main}|}|\\
\end{tabular}
\end{center}
%
The directive |\childdocby| is similar to |\childdocof|
described in \secref{sec:include},
but the subsequent selection of content must be done manually.
To that end, both |\ifchilddoc| and |\ifchilddocmanual|
will be true upon processing of a part,
and the name of the part is stored in |\childdocname|.
Note that |\jobname| will be set to the filename of the current part
so that each part receives an individual |.aux| file
that does not interfere with the |.aux| file(s) of the main document.
This behaviour can be altered by the alternative form
|\childdocby[*]{|\textit{main}|}| (with a non-empty optional argument)
which uses the |.aux| file of the main document
by setting |\jobname| to \textit{main}.

%%%%%%%%%%%%%%%%%%%%%%%%%%%%%%%%%%%%%%%%%%%%%%%%%%%%%%%%%%%%%%%%%%%%%%%%%%%%%%%%
\subsection{Driver Development}
\label{sec:driver}

The \textsf{childdoc} mechanism can also be use for the development
of definition files such as \LaTeX{} styles or classes.
This case differs from the above setup with multiple parts
included by |\include| in that no |\includeonly| should be invoked.
This can be achieved by starting the include file
(before |\ProvidesPackage|) with:
%
\begin{center}
\begin{tabular}{l}
|\input{childdoc.def}|\\
|\childdocforward{|\textit{main}|}|\\
\end{tabular}
\end{center}
%
or alternatively with:
%
\begin{center}
\begin{tabular}{l}
|\input{childdoc.def}|\\
|\childdocby{|\textit{main}|}|\\
\end{tabular}
\end{center}
%
Both forms have slightly different effects as described above.
The main file is prepared as usual, see \secref{sec:include}.

%%%%%%%%%%%%%%%%%%%%%%%%%%%%%%%%%%%%%%%%%%%%%%%%%%%%%%%%%%%%%%%%%%%%%%%%%%%%%%%%
\subsection{Legacy Detection}
\label{sec:detection}

The directive |\childdocmain| in the main file can detect
whether the complete document or merely a child is to be compiled
even without using the directive |\childdocof|.
This method is deprecated because it is less robust
and there is no compelling reason to use it;
it is merely provided for backward compatibility
and it may be removed in future versions.

If the detection mechanism is to be used,
it is mandatory to correctly specify
the filename of the main file as the argument of |\childdocmain|:
%
\begin{center}
\begin{tabular}{l}
|\input{childdoc.def}|\\
|\childdocmain{|\textit{main}|}|\\
\end{tabular}
\end{center}
%
If |\jobname| does not match the argument \textit{main} of |\childdocmain|,
it is assumed that |\jobname| points to the child file to be compiled.
When using |\childdocmain| with the main file specified as argument,
it suffices to start a child file
with just |\input{|\textit{main}|}|
without loading of the package and using |\childdocof|.
If instead all processing is done
with the appropriate \textsf{childdoc} directives,
the argument of \textit{main} of |\childdocmain| can be empty.

An alternative version of the command line processing described
in \secref{sec:commandline} using the detection mechanism reads:
%
\begin{center}
|... -jobname "|\textit{target}|" "|[\textit{flags}]%
[|\def\jobname{|\textit{dest}|}|]|\input{|\textit{main}|}"|
\end{center}

%%%%%%%%%%%%%%%%%%%%%%%%%%%%%%%%%%%%%%%%%%%%%%%%%%%%%%%%%%%%%%%%%%%%%%%%%%%%%%%%
\subsection{Manual Code}
\label{sec:manual}

In case one cannot be certain whether the definitions file |childdoc.def|
is installed on the target \TeX{} distribution
and one prefers not to ship it,
it is conceivable to paste a few relevant commands into the sources.

To that end, drop all statements |\input{childdoc.def}|
and perform the replacements as outlined below.
Instead of |\childdocmain{|\textit{main}|}| add the following code
to the top of the main file:
%
\begin{center}
\begin{tabular}{l}
|\||ifdefined\childdocname\endinput\||fi\newif\ifchilddoc|\\
|\edef\childdocname{\scantokens\expandafter{\jobname\noexpand}}|\\
|\def\childdocmain{|\textit{main}|}\||ifx\childdocmain\childdocname\||else|\\
|\childdoctrue\includeonly{\childdocname}\let\jobname\childdocmain\||fi|\\
\end{tabular}
\end{center}
%
Instead of |\childdocof{|\textit{main}|}| just include the main file
at the top of each child file:
%
\begin{center}
|\input{|\textit{main}|}|
\end{center}
%
A simple redirection |\childdocforward{|\textit{dest}|}| is achieved by:
%
\begin{center}
|\def\jobname{|\textit{dest}|}\input{\jobname}|
\end{center}
%
The redirection with prefix
|\childdocforwardprefix[|\textit{prefix}|]{|\textit{dest}|}|
is accomplished by:
%
\begin{center}
\begin{tabular}{l}
|{\edef\jobname{\scantokens\expandafter{\jobname\noexpand}}|\\
|\def\redirectjob |\textit{prefix}|#1~~~{\gdef\jobname{|\textit{dest}|#1}}|\\
|\expandafter\redirectjob\jobname~~~}\input{\jobname}|
\end{tabular}
\end{center}

In an alternative approach,
child documents can be compiled by a specific command line
without additional code or specific definitions:
%
\begin{center}
|... -jobname "|\textit{target}|" "|[\textit{flags}]%
|\includeonly{|\textit{dest}|}\input{|\textit{main}|}"|
\end{center}
%

%%%%%%%%%%%%%%%%%%%%%%%%%%%%%%%%%%%%%%%%%%%%%%%%%%%%%%%%%%%%%%%%%%%%%%%%%%%%%%%%
%%%%%%%%%%%%%%%%%%%%%%%%%%%%%%%%%%%%%%%%%%%%%%%%%%%%%%%%%%%%%%%%%%%%%%%%%%%%%%%%
\section{Information}

%%%%%%%%%%%%%%%%%%%%%%%%%%%%%%%%%%%%%%%%%%%%%%%%%%%%%%%%%%%%%%%%%%%%%%%%%%%%%%%%
\subsection{Copyright}

Copyright \copyright{} 2017--2018 Niklas Beisert

This work may be distributed and/or modified under the
conditions of the \LaTeX{} Project Public License, either version 1.3
of this license or (at your option) any later version.
The latest version of this license is in
  \url{http://www.latex-project.org/lppl.txt}
and version 1.3 or later is part of all distributions of \LaTeX{}
version 2005/12/01 or later.

This work has the LPPL maintenance status `maintained'.

The Current Maintainer of this work is Niklas Beisert.

This work consists of the files |README.txt|, |childdoc.ins| and |childdoc.dtx|
as well as the derived files |childdoc.def|, |cdocsamp.tex|
with |cdocsch1.tex|, |cdocsch2.tex|, |cdocspt3.tex|, |cdocspt4.tex|,
|cdocsdrf.tex|, |cdocsfn1.tex|, |cdocsfn2.tex|
as well as |childdoc.pdf|.

%%%%%%%%%%%%%%%%%%%%%%%%%%%%%%%%%%%%%%%%%%%%%%%%%%%%%%%%%%%%%%%%%%%%%%%%%%%%%%%%
\subsection{Files and Installation}

The package consists of the files:
%
\begin{center}
\begin{tabular}{ll}
    |README.txt|   & readme file \\
    |childdoc.ins| & installation file \\
    |childdoc.dtx| & source file \\
    |childdoc.def| & definition file \\
    |cdocsamp.tex| & sample main file \\
    |cdocsch1.tex| & sample include file \\
    |cdocsch2.tex| & sample include file \\
    |cdocspt3.tex| & sample part file \\
    |cdocspt4.tex| & sample part file \\
    |cdocsdrf.tex| & sample redirection file \\
    |cdocsfn1.tex| & sample redirection file \\
    |cdocsfn2.tex| & sample redirection file \\
    |childdoc.pdf| & manual
\end{tabular}
\end{center}
%
The distribution consists of the files
|README.txt|, |childdoc.ins| and |childdoc.dtx|.
%
\begin{itemize}
\item
Run (pdf)\LaTeX{} on |childdoc.dtx|
to compile the manual |childdoc.pdf| (this file).
\item
Run \LaTeX{} on |childdoc.ins| to create the definitions file |childdoc.def|
and the sample |cdocsamp.tex| with include files
|cdocsch1.tex|, |cdocsch2.tex|, |cdocspt3.tex|, |cdocspt4.tex|,
|cdocsdrf.tex|, |cdocsfn1.tex|, |cdocsfn2.tex|.
Then copy the file |childdoc.def| to an appropriate directory of your \LaTeX{}
distribution, e.g.\ \textit{texmf-root}|/tex/latex/childdoc|.
\end{itemize}

%%%%%%%%%%%%%%%%%%%%%%%%%%%%%%%%%%%%%%%%%%%%%%%%%%%%%%%%%%%%%%%%%%%%%%%%%%%%%%%%
\subsection{Related CTAN Packages}

There are several other packages which offer a similar functionality:
%
\begin{itemize}
\item
The packages
\href{http://ctan.org/pkg/docmute}{\textsf{docmute}},
\href{http://ctan.org/pkg/includex}{\textsf{includex}} and
\href{http://ctan.org/pkg/standalone}{\textsf{standalone}}
provide commands to include only the document body of
a child file thus allowing both files to be compiled individually.
\item
The packages \href{http://ctan.org/pkg/subdocs}{\textsf{subdocs}}
and \href{http://ctan.org/pkg/subfiles}{\textsf{subfiles}}
provide structures in which the main and child documents can be
encapsulated and allowing them to be compiled individually.
The inclusion mechanism is different from the conventional |\include|.
\item
The package \href{http://ctan.org/pkg/combine}{\textsf{combine}}
is an elaborate solution to combine several documents into one.
\end{itemize}
%
See also the CTAN topic \href{http://ctan.org/topic/subdocs}{\textsf{subdocs}}
for further related packages.
The present package differs from the above solutions in that
a document structure constructed with the conventional |\include| mechanism
just needs two extra commands at the top of every file
such that all constituent files can be compiled individually.

%%%%%%%%%%%%%%%%%%%%%%%%%%%%%%%%%%%%%%%%%%%%%%%%%%%%%%%%%%%%%%%%%%%%%%%%%%%%%%%%
%\subsection{Feature Suggestions}
%
%The following is a list of features which may be useful for future
%versions of this package:
%%
%\begin{itemize}
%\item
%\ldots
%\end{itemize}

%%%%%%%%%%%%%%%%%%%%%%%%%%%%%%%%%%%%%%%%%%%%%%%%%%%%%%%%%%%%%%%%%%%%%%%%%%%%%%%%
\subsection{Revision History}

%%%%%%%%%%%%%%%%%%%%%%%%%%%%%%%%%%%%%%%%
\paragraph{v2.0:} 2018/12/30

\begin{itemize}
\item
immediate forward processing
\item
added |\childdocby| mechanism
\item
manual restructured
\end{itemize}

%%%%%%%%%%%%%%%%%%%%%%%%%%%%%%%%%%%%%%%%
\paragraph{v1.6:} 2018/01/17

\begin{itemize}
\item
application for development of include files
\item
corrections to manual
\end{itemize}

%%%%%%%%%%%%%%%%%%%%%%%%%%%%%%%%%%%%%%%%
\paragraph{v1.5:} 2017/05/21

\begin{itemize}
\item
more complete structuring introduced
\item
|\childdocof| introduced
\item
|\childdoc| renamed to |\childdocmain|
\item
|\childredirect| renamed to |\childdocforward| and |\childdocforwardprefix|
and functionality expanded
\end{itemize}

%%%%%%%%%%%%%%%%%%%%%%%%%%%%%%%%%%%%%%%%
\paragraph{v1.0:} 2017/04/27

\begin{itemize}
\item
manual and install package
\item
first version published on CTAN
\end{itemize}

%%%%%%%%%%%%%%%%%%%%%%%%%%%%%%%%%%%%%%%%
\paragraph{v0.6:} 2017/04/26

\begin{itemize}
\item
redirection mechanism added
\end{itemize}

%%%%%%%%%%%%%%%%%%%%%%%%%%%%%%%%%%%%%%%%
\paragraph{v0.5:} 2017/04/26

\begin{itemize}
\item
functionality in definition file
\end{itemize}


%%%%%%%%%%%%%%%%%%%%%%%%%%%%%%%%%%%%%%%%%%%%%%%%%%%%%%%%%%%%%%%%%%%%%%%%%%%%%%%%
%%%%%%%%%%%%%%%%%%%%%%%%%%%%%%%%%%%%%%%%%%%%%%%%%%%%%%%%%%%%%%%%%%%%%%%%%%%%%%%%
%%%%%%%%%%%%%%%%%%%%%%%%%%%%%%%%%%%%%%%%%%%%%%%%%%%%%%%%%%%%%%%%%%%%%%%%%%%%%%%%
\appendix

\settowidth\MacroIndent{\rmfamily\scriptsize 000\ }

 \DocInput{childdoc.dtx}

\end{document}
%</driver>
% \fi
%
% %%%%%%%%%%%%%%%%%%%%%%%%%%%%%%%%%%%%%%%%%%%%%%%%%%%%%%%%%%%%%%%%%%%%%%%%%%%%%%
% %%%%%%%%%%%%%%%%%%%%%%%%%%%%%%%%%%%%%%%%%%%%%%%%%%%%%%%%%%%%%%%%%%%%%%%%%%%%%%
% \section{Sample}
%\iffalse
%<*samplemain>
%\fi
%
% The following presents a sample document
% with two chapters, two parts, a title page,
% a compile flag as well as three forwarding files to set the flag.
% It consists of eight |.tex| files:
% \begin{center}
% \begin{tabular}{ll}
% |cdocsamp.tex|&main file\\
% |cdocsch1.tex|&include file for chapter 1\\
% |cdocsch2.tex|&include file for chapter 2\\
% |cdocspt3.tex|&include file for part 3\\
% |cdocspt4.tex|&include file for part 4\\
% |cdocsdrf.tex|&forwarding file for main file in draft mode\\
% |cdocsfi1.tex|&forwarding file for final version of chapter 1\\
% |cdocsfi2.tex|&forwarding file for final version of chapter 2\\
% \end{tabular}
% \end{center}
% Each of the eight files can be compiled directly by the \LaTeX{} compiler.
%
% %%%%%%%%%%%%%%%%%%%%%%%%%%%%%%%%%%%%%%
% \paragraph{Main File.}
%
% The main file is called |cdocsamp.tex|.
%
% Load the \textsf{childdoc} definitions and
% declare the filename for the main document:
%    \begin{macrocode}
\input{childdoc.def}
\childdocmain{}
%    \end{macrocode}

% Optional override for |\version| flag:
%    \begin{macrocode}
%%\ifchilddoc\else\providecommand{\version}{draft}\fi
%    \end{macrocode}

% Define the default values for the |\version| flag
% (|final| for the main file and |draft| for childs):
%    \begin{macrocode}
\ifchilddoc
\providecommand{\version}{draft}
\else
\providecommand{\version}{final}
\fi
%    \end{macrocode}

% Load the standard document class:
%    \begin{macrocode}
\documentclass[12pt]{article}
%    \end{macrocode}

% Start the document body:
%    \begin{macrocode}
\begin{document}
%    \end{macrocode}

% Declare a title page.
% Print title, part of document being processed and version flag:
%    \begin{macrocode}
\addtocounter{page}{-1}
\begin{center}
{\LARGE\bfseries{}childdoc example\par}
\vspace{1cm}
\ifchilddoc
\ifchilddocmanual part\else chapter\fi:
`\childdocname' of `\childdocjob'\par
\else
main document: `\childdocjob'\par
\fi
version: \version\par
\end{center}
\newpage
%    \end{macrocode}

% Manually include selected file,
% otherwise process as usual:
%    \begin{macrocode}
\ifchilddocmanual
\section*{part `\childdocname'}
\input{\childdocname}
\else
%    \end{macrocode}

% Include the two chapters:
%    \begin{macrocode}
\include{cdocsch1}
\include{cdocsch2}
%    \end{macrocode}

% Include the two parts unless only chapters should be displayed:
%    \begin{macrocode}
\ifchilddoc\else
\section{part three}
\input{cdocspt3}
\section{part four}
\input{cdocspt4}
\fi
%    \end{macrocode}

% Process as usual until here:
%    \begin{macrocode}
\fi
%    \end{macrocode}

% End of document body:
%    \begin{macrocode}
\end{document}
%    \end{macrocode}
%\iffalse
%</samplemain>
%\fi
%
% %%%%%%%%%%%%%%%%%%%%%%%%%%%%%%%%%%%%%%
% \paragraph{Chapter Include Files.}
%
% The include files are called |cdocsch1.tex| and |cdocsch2.tex|.
%
%\iffalse
%<*samplechap1|samplechap2>
%\fi

% Optional override for |\version| flag:
%    \begin{macrocode}
%%\providecommand{\version}{final}
%    \end{macrocode}

% Include the main document:
%    \begin{macrocode}
\input{childdoc.def}
\childdocof{cdocsamp}
%    \end{macrocode}

%\iffalse
%</samplechap1|samplechap2>
%\fi
%
%\iffalse
%<*samplechap1>
%\fi
% Some text for chapter 1:
%    \begin{macrocode}
\section{one}
some text in chapter one
%    \end{macrocode}

%\iffalse
%</samplechap1>
%\fi
% Some text for chapter 2:
%\iffalse
%<*samplechap2>
%\fi
%    \begin{macrocode}
\section{two}
more text in chapter two
%    \end{macrocode}

%\iffalse
%</samplechap2>
%\fi
%
% %%%%%%%%%%%%%%%%%%%%%%%%%%%%%%%%%%%%%%
% \paragraph{Part Include Files.}
%
% The include files are called |cdocspt3.tex| and |cdocspt4.tex|.
%
%\iffalse
%<*samplepart3|samplepart4>
%\fi

% Optional override for |\version| flag:
%    \begin{macrocode}
%%\providecommand{\version}{final}
%    \end{macrocode}

% Include the main document:
%    \begin{macrocode}
\input{childdoc.def}
\childdocby{cdocsamp}
%    \end{macrocode}

%\iffalse
%</samplepart3|samplepart4>
%\fi
%
%\iffalse
%<*samplepart3>
%\fi
% Some text for part 3:
%    \begin{macrocode}
some text in part three
%    \end{macrocode}

%\iffalse
%</samplepart3>
%\fi
% Some text for part 4:
%\iffalse
%<*samplepart4>
%\fi
%    \begin{macrocode}
more text in part four
%    \end{macrocode}

%\iffalse
%</samplepart4>
%\fi
%
% %%%%%%%%%%%%%%%%%%%%%%%%%%%%%%%%%%%%%%
% \paragraph{Forwarding for a Complete Draft.}
%
% The following forwarding file |cdocsdrf.tex|
% compiles the main document in draft mode:
%\iffalse
%<*sampledraft>
%\fi
%    \begin{macrocode}
\def\version{draft}
\input{childdoc.def}
\childdocforward{cdocsamp}
%    \end{macrocode}

%\iffalse
%</sampledraft>
%\fi
%
% %%%%%%%%%%%%%%%%%%%%%%%%%%%%%%%%%%%%%%
% \paragraph{Forwarding for Final Version of the Chapters.}
%
% The following forwarding files |cdocsfn1.tex| and |cdocsfn2.tex|
% (with identical content)
% compile the final versions of the child documents
% |cdocsch1.tex| and |cdocsch2.tex|, respectively:
%\iffalse
%<*samplefinal>
%\fi
%    \begin{macrocode}
\def\version{final}
\input{childdoc.def}
\childdocforwardprefix[cdocsamp]{cdocsfn}{cdocsch}
%    \end{macrocode}

%\iffalse
%</samplefinal>
%\fi
%
% %%%%%%%%%%%%%%%%%%%%%%%%%%%%%%%%%%%%%%
% \paragraph{Command Line Processing.}
%
% The following three command lines generate the output files
% |cdocscld|, |cdocscl1| and |cdocscl2|
% which should be identical to
% |cdocsdrf|, |cdocsch1| and |cdocsfn2|, respectively:
% \begin{center}
% \begin{tabular}{l}
% |latex -jobname cdocscld \|\\
% |  "\def\version{draft}\input{childdoc.def}\childdocforward{cdocsamp}"|\\
% |latex -jobname cdocscl1 \|\\
% |  "\input{childdoc.def}\childdocforward[cdocsamp]{cdocsch1}"|\\
% |latex -jobname cdocscl2 \|\\
% |  "\def\version{final}\input{childdoc.def}\childdocforward{cdocsch2}"|
% \end{tabular}
% \end{center}
% Note that the trailing backslash on each first line
% merely continues the input to the second line
% (for convenient cut ant paste).
% Furthermore, the command |latex| can be replaced by any
% of its alternative versions such as |pdflatex|.
%
% %%%%%%%%%%%%%%%%%%%%%%%%%%%%%%%%%%%%%%%%%%%%%%%%%%%%%%%%%%%%%%%%%%%%%%%%%%%%%%
% %%%%%%%%%%%%%%%%%%%%%%%%%%%%%%%%%%%%%%%%%%%%%%%%%%%%%%%%%%%%%%%%%%%%%%%%%%%%%%
% \section{Implementation}
%\iffalse
%<*package>
%\fi
%
% This section describes the definitions file |childdoc.def|.

% The definitions cannot be loaded using |\usepackage| or |\RequirePackage|
% which has a mechanism to prevent loading a style file more than once.
% When loading the definitions by means of |\input|
% multiple instances have to be prevented manually:
%\iffalse
%This code needs to be before the `\ProvidesFile' directive
%which is defined at the beginning of this file.
%Therefore it is also placed there and commented out here.
%</package>
%<*discard>
%\fi
%    \begin{macrocode}
\ifdefined\childdocmain\endinput\fi
%    \end{macrocode}
%\iffalse
%</discard>
%<*package>
%\fi
%
% \macro{\ifchilddoc}
% \macro{\ifchilddocmanual}
% The conditional |\ifchilddoc| tells whether a
% child (true) or main (false) document is being compiled.
% The conditional |\ifchilddocmanual| tells whether
% the |\includeonly| mechanism is used (false) or
% the selection of child files must be performed manually (true).
% The definitions initialise to false:
%    \begin{macrocode}
\newif\ifchilddoc
\newif\ifchilddocmanual
%    \end{macrocode}

% \macro{\childdocname}
% \macro{\childdocjob}
% The macro |\childdocname| stores the name of the main document
% to be compiled. The macro |\childdocjob| stores the name of
% the document on which the \LaTeX{} compiler was originally invoked.
% The content of |\jobname| cannot be compared
% to filenames specified in the source due to different catcodes.
% The following code rescans |\jobname|, stores the result
% in |\childdocname| and saves a copy in |\childdocjob|:
%    \begin{macrocode}
\edef\childdocname{\scantokens\expandafter{\jobname\noexpand}}
\let\childdocjob\childdocname
%    \end{macrocode}

% \macro{\childdocdisable}
% The macro |\childdocdisable| prevents the main file
% from being processed more than once.
% At this stage, the main document command |\childdocmain|
% is assumed to be called once again where it should do nothing.
% Any subsequent call to it should prevent
% a secondary processing of the main document
% It overwrites the forwarding commands
% |\childdocof| and |\childdocforward|
% with empty macros to prevent further inclusions of the main document:
%    \begin{macrocode}
\newcommand{\childdocdisable}
{
  \renewcommand{\childdocmain}[1]{\renewcommand{\childdocmain}[1]{\endinput}}
  \renewcommand{\childdocof}[1]{}
  \renewcommand{\childdocby}[2][]{}
  \renewcommand{\childdocforward}[2][]{}
  \renewcommand{\childdocdisable}{}
}
%    \end{macrocode}

% \macro{\childdocmain}
% The macro |\childdocmain| is to be called at the top of the main file
% with nothing or the main filename (without extension) as argument.
% First, it breaks loops.
% If the argument is not empty and does not match |\childdocname|
% (which is set by the first inclusion of |childdoc.def|),
% |\ifchilddoc| is set to true, |\includeonly| is applied to the child file
% and |\jobname| is set to the main file
% (for proper handling of |.aux| files):
%    \begin{macrocode}
\newcommand{\childdocmain}[1]
{
  \childdocdisable\childdocmain{}
  \if?#1?\else
    \begingroup
      \def\childdoctmp{#1}
      \ifx\childdoctmp\childdocname
        \def\childdoctmp{}
      \else
        \def\childdoctmp
        {
          \childdoctrue
          \includeonly{\childdocname}
          \def\childdocjob{#1}
          \def\jobname{#1}
        }
      \fi
      \expandafter
    \endgroup
    \childdoctmp
  \fi
}
%    \end{macrocode}

% \macro{\childdocof}
% The command |\childdocof| redirects
% compilation to the main file |#1|.
%    \begin{macrocode}
\newcommand{\childdocof}[1]
{
  \childdocdisable
  \childdoctrue
  \includeonly{\childdocname}
  \def\jobname{#1}
  \def\childdocjob{#1}
  \input{#1}
}
%    \end{macrocode}

% \macro{\childdocby}
% The command |\childdocby| ....
%    \begin{macrocode}
\newcommand{\childdocby}[2][]
{
  \childdocdisable
  \childdoctrue
  \childdocmanualtrue
  \if?#1?\else
    \def\jobname{#2}
  \fi
  \def\childdocjob{#2}
  \input{#2}
  \endinput
}
%    \end{macrocode}

% \macro{\childdocforward}
% The command |\childdocforward| redirects
% compilation to the main file or
% (if the optional argument is given) a child file.
% Parameters are set as if the main file
% or a child file starting with |\childdocof| was compiled.
% Then compilation is handed over to the main file:
%    \begin{macrocode}
\newcommand{\childdocforward}[2][]
{
  \begingroup
    \if?#1?
      \def\childdoctmp
      {
        \def\childdocname{#2}
        \def\childdocjob{#2}
        \def\jobname{#2}
        \input{#2}
        \endinput
      }
    \else
      \def\childdoctmp
      {
        \childdocdisable
        \def\childdocname{#2}
        \childdoctrue
        \includeonly{#2}
        \def\childdocjob{#1}
        \def\jobname{#1}
        \input{#1}
        \endinput
      }
    \fi
    \expandafter
  \endgroup
  \childdoctmp
}
%    \end{macrocode}

% \macro{\childdocforwardprefix}
% The command |\childdocforwardprefix| redirects
% compilation to the main or a child file by means of a pattern.
% The prefix |#1| in the current filename is replaced by |#2|
% and the suffix of the current filename is kept
% (it is assumed that the filename does not contain the substring `|~~~|'
% which is used as a delimiter).
% Compilation is handed over to the new file by |\childdocforward|:
%    \begin{macrocode}
\newcommand{\childdocforwardprefix}[3][]
{
  \begingroup
    \def\childdocextract #2##1~~~{\def\childdoctmp{\childdocforward[#1]{#3##1}}}
    \expandafter\childdocextract\childdocname~~~
    \expandafter
  \endgroup
  \childdoctmp
}
%    \end{macrocode}

% \macro{\childdoc}
% The deprecated macro |\childdoc| is a legacy version of |\childdocmain|:
%    \begin{macrocode}
\newcommand{\childdoc}{\childdocmain}
%    \end{macrocode}

% \macro{\childdocredirect}
% The deprecated macro |\childdocredirect| is a legacy version
% of |\childdocforward| and |\childdocforwardprefix|:
%    \begin{macrocode}
\newcommand{\childdocredirect}[2][]
{
  \begingroup
    \if?#1?
      \def\childdoctmp{\childdocforward{#2}}
    \else
      \def\childdoctmp{\childdocforwardprefix{#1}{#2}}
    \fi
    \expandafter
  \endgroup
  \childdoctmp
}
%    \end{macrocode}

%\iffalse
%</package>
%\fi
%
\endinput
\childdocforward{cdocsch2}"|
% \end{tabular}
% \end{center}
% Note that the trailing backslash on each first line
% merely continues the input to the second line
% (for convenient cut ant paste).
% Furthermore, the command |latex| can be replaced by any
% of its alternative versions such as |pdflatex|.
%
% %%%%%%%%%%%%%%%%%%%%%%%%%%%%%%%%%%%%%%%%%%%%%%%%%%%%%%%%%%%%%%%%%%%%%%%%%%%%%%
% %%%%%%%%%%%%%%%%%%%%%%%%%%%%%%%%%%%%%%%%%%%%%%%%%%%%%%%%%%%%%%%%%%%%%%%%%%%%%%
% \section{Implementation}
%\iffalse
%<*package>
%\fi
%
% This section describes the definitions file |childdoc.def|.

% The definitions cannot be loaded using |\usepackage| or |\RequirePackage|
% which has a mechanism to prevent loading a style file more than once.
% When loading the definitions by means of |\input|
% multiple instances have to be prevented manually:
%\iffalse
%This code needs to be before the `\ProvidesFile' directive
%which is defined at the beginning of this file.
%Therefore it is also placed there and commented out here.
%</package>
%<*discard>
%\fi
%    \begin{macrocode}
\ifdefined\childdocmain\endinput\fi
%    \end{macrocode}
%\iffalse
%</discard>
%<*package>
%\fi
%
% \macro{\ifchilddoc}
% \macro{\ifchilddocmanual}
% The conditional |\ifchilddoc| tells whether a
% child (true) or main (false) document is being compiled.
% The conditional |\ifchilddocmanual| tells whether
% the |\includeonly| mechanism is used (false) or
% the selection of child files must be performed manually (true).
% The definitions initialise to false:
%    \begin{macrocode}
\newif\ifchilddoc
\newif\ifchilddocmanual
%    \end{macrocode}

% \macro{\childdocname}
% \macro{\childdocjob}
% The macro |\childdocname| stores the name of the main document
% to be compiled. The macro |\childdocjob| stores the name of
% the document on which the \LaTeX{} compiler was originally invoked.
% The content of |\jobname| cannot be compared
% to filenames specified in the source due to different catcodes.
% The following code rescans |\jobname|, stores the result
% in |\childdocname| and saves a copy in |\childdocjob|:
%    \begin{macrocode}
\edef\childdocname{\scantokens\expandafter{\jobname\noexpand}}
\let\childdocjob\childdocname
%    \end{macrocode}

% \macro{\childdocdisable}
% The macro |\childdocdisable| prevents the main file
% from being processed more than once.
% At this stage, the main document command |\childdocmain|
% is assumed to be called once again where it should do nothing.
% Any subsequent call to it should prevent
% a secondary processing of the main document
% It overwrites the forwarding commands
% |\childdocof| and |\childdocforward|
% with empty macros to prevent further inclusions of the main document:
%    \begin{macrocode}
\newcommand{\childdocdisable}
{
  \renewcommand{\childdocmain}[1]{\renewcommand{\childdocmain}[1]{\endinput}}
  \renewcommand{\childdocof}[1]{}
  \renewcommand{\childdocby}[2][]{}
  \renewcommand{\childdocforward}[2][]{}
  \renewcommand{\childdocdisable}{}
}
%    \end{macrocode}

% \macro{\childdocmain}
% The macro |\childdocmain| is to be called at the top of the main file
% with nothing or the main filename (without extension) as argument.
% First, it breaks loops.
% If the argument is not empty and does not match |\childdocname|
% (which is set by the first inclusion of |childdoc.def|),
% |\ifchilddoc| is set to true, |\includeonly| is applied to the child file
% and |\jobname| is set to the main file
% (for proper handling of |.aux| files):
%    \begin{macrocode}
\newcommand{\childdocmain}[1]
{
  \childdocdisable\childdocmain{}
  \if?#1?\else
    \begingroup
      \def\childdoctmp{#1}
      \ifx\childdoctmp\childdocname
        \def\childdoctmp{}
      \else
        \def\childdoctmp
        {
          \childdoctrue
          \includeonly{\childdocname}
          \def\childdocjob{#1}
          \def\jobname{#1}
        }
      \fi
      \expandafter
    \endgroup
    \childdoctmp
  \fi
}
%    \end{macrocode}

% \macro{\childdocof}
% The command |\childdocof| redirects
% compilation to the main file |#1|.
%    \begin{macrocode}
\newcommand{\childdocof}[1]
{
  \childdocdisable
  \childdoctrue
  \includeonly{\childdocname}
  \def\jobname{#1}
  \def\childdocjob{#1}
  \input{#1}
}
%    \end{macrocode}

% \macro{\childdocby}
% The command |\childdocby| ....
%    \begin{macrocode}
\newcommand{\childdocby}[2][]
{
  \childdocdisable
  \childdoctrue
  \childdocmanualtrue
  \if?#1?\else
    \def\jobname{#2}
  \fi
  \def\childdocjob{#2}
  \input{#2}
  \endinput
}
%    \end{macrocode}

% \macro{\childdocforward}
% The command |\childdocforward| redirects
% compilation to the main file or
% (if the optional argument is given) a child file.
% Parameters are set as if the main file
% or a child file starting with |\childdocof| was compiled.
% Then compilation is handed over to the main file:
%    \begin{macrocode}
\newcommand{\childdocforward}[2][]
{
  \begingroup
    \if?#1?
      \def\childdoctmp
      {
        \def\childdocname{#2}
        \def\childdocjob{#2}
        \def\jobname{#2}
        \input{#2}
        \endinput
      }
    \else
      \def\childdoctmp
      {
        \childdocdisable
        \def\childdocname{#2}
        \childdoctrue
        \includeonly{#2}
        \def\childdocjob{#1}
        \def\jobname{#1}
        \input{#1}
        \endinput
      }
    \fi
    \expandafter
  \endgroup
  \childdoctmp
}
%    \end{macrocode}

% \macro{\childdocforwardprefix}
% The command |\childdocforwardprefix| redirects
% compilation to the main or a child file by means of a pattern.
% The prefix |#1| in the current filename is replaced by |#2|
% and the suffix of the current filename is kept
% (it is assumed that the filename does not contain the substring `|~~~|'
% which is used as a delimiter).
% Compilation is handed over to the new file by |\childdocforward|:
%    \begin{macrocode}
\newcommand{\childdocforwardprefix}[3][]
{
  \begingroup
    \def\childdocextract #2##1~~~{\def\childdoctmp{\childdocforward[#1]{#3##1}}}
    \expandafter\childdocextract\childdocname~~~
    \expandafter
  \endgroup
  \childdoctmp
}
%    \end{macrocode}

% \macro{\childdoc}
% The deprecated macro |\childdoc| is a legacy version of |\childdocmain|:
%    \begin{macrocode}
\newcommand{\childdoc}{\childdocmain}
%    \end{macrocode}

% \macro{\childdocredirect}
% The deprecated macro |\childdocredirect| is a legacy version
% of |\childdocforward| and |\childdocforwardprefix|:
%    \begin{macrocode}
\newcommand{\childdocredirect}[2][]
{
  \begingroup
    \if?#1?
      \def\childdoctmp{\childdocforward{#2}}
    \else
      \def\childdoctmp{\childdocforwardprefix{#1}{#2}}
    \fi
    \expandafter
  \endgroup
  \childdoctmp
}
%    \end{macrocode}

%\iffalse
%</package>
%\fi
%
\endinput

\childdocby{cdocsamp}
%    \end{macrocode}

%\iffalse
%</samplepart3|samplepart4>
%\fi
%
%\iffalse
%<*samplepart3>
%\fi
% Some text for part 3:
%    \begin{macrocode}
some text in part three
%    \end{macrocode}

%\iffalse
%</samplepart3>
%\fi
% Some text for part 4:
%\iffalse
%<*samplepart4>
%\fi
%    \begin{macrocode}
more text in part four
%    \end{macrocode}

%\iffalse
%</samplepart4>
%\fi
%
% %%%%%%%%%%%%%%%%%%%%%%%%%%%%%%%%%%%%%%
% \paragraph{Forwarding for a Complete Draft.}
%
% The following forwarding file |cdocsdrf.tex|
% compiles the main document in draft mode:
%\iffalse
%<*sampledraft>
%\fi
%    \begin{macrocode}
\def\version{draft}
% \iffalse
%
% childdoc.dtx Copyright (C) 2017-2018 Niklas Beisert
%
% This work may be distributed and/or modified under the
% conditions of the LaTeX Project Public License, either version 1.3
% of this license or (at your option) any later version.
% The latest version of this license is in
%   http://www.latex-project.org/lppl.txt
% and version 1.3 or later is part of all distributions of LaTeX
% version 2005/12/01 or later.
%
% This work has the LPPL maintenance status `maintained'.
%
% The Current Maintainer of this work is Niklas Beisert.
%
% This work consists of the files childdoc.dtx and childdoc.ins
% and the derived files childdoc.def and cdocsamp.tex with
% cdocsch1.tex, cdocsch2.tex, cdocsdrf.tex, cdocsfn1.tex, cdocsfn2.tex.
%
%<package>\ifdefined\childdocmain\endinput\fi
%<package>\ProvidesFile{childdoc.def}[2018/12/30 v2.0 child document driver]
%<samplemain>\ProvidesFile{cdocsamp.tex}[2018/12/30 v2.0 sample for childdoc]
%<*driver>
%\ProvidesFile{childdoc.drv}[2018/12/30 v2.0 childdoc reference manual file]
\PassOptionsToClass{10pt,a4paper}{article}
\documentclass{ltxdoc}

\usepackage[margin=35mm]{geometry}
\usepackage{hyperref}
\usepackage{hyperxmp}
\usepackage[usenames]{color}

\hypersetup{colorlinks=true}
\hypersetup{pdfstartview=FitH}
\hypersetup{pdfpagemode=UseNone}
\hypersetup{pdfsource={}}
\hypersetup{pdflang={en-UK}}
\hypersetup{pdfcopyright={Copyright 2017-2018 Niklas Beisert.
  This work may be distributed and/or modified under the
  conditions of the LaTeX Project Public License, either version 1.3
  of this license or (at your option) any later version.}}
\hypersetup{pdflicenseurl={http://www.latex-project.org/lppl.txt}}
\hypersetup{pdfcontactaddress={ETH Zurich, ITP, HIT K,
  Wolfgang-Pauli-Strasse 27}}
\hypersetup{pdfcontactpostcode={8093}}
\hypersetup{pdfcontactcity={Zurich}}
\hypersetup{pdfcontactcountry={Switzerland}}
\hypersetup{pdfcontactemail={nbeisert@itp.phys.ethz.ch}}
\hypersetup{pdfcontacturl={http://people.phys.ethz.ch/\xmptilde nbeisert/}}

\newcommand{\secref}[1]{\hyperref[#1]{section \ref*{#1}}}

\parskip1ex
\parindent0pt
\let\olditemize\itemize
\def\itemize{\olditemize\parskip0pt}

\begin{document}

\title{The \textsf{childdoc} Package}
\hypersetup{pdftitle={The childdoc Package}}
\author{Niklas Beisert\\[2ex]
  Institut f\"ur Theoretische Physik\\
  Eidgen\"ossische Technische Hochschule Z\"urich\\
  Wolfgang-Pauli-Strasse 27, 8093 Z\"urich, Switzerland\\[1ex]
  \href{mailto:nbeisert@itp.phys.ethz.ch}
  {\texttt{nbeisert@itp.phys.ethz.ch}}}
\hypersetup{pdfauthor={Niklas Beisert}}
\hypersetup{pdfsubject={Manual for the LaTeX2e Package childdoc}}
\date{30 December 2018, \textsf{v2.0}}
\maketitle

\begin{abstract}\noindent
\textsf{childdoc} is a \LaTeXe{} package
that enables the direct compilation
of document sections included by |\include|
to individual files.
\end{abstract}

\begingroup
\parskip0ex
\tableofcontents
\endgroup

%%%%%%%%%%%%%%%%%%%%%%%%%%%%%%%%%%%%%%%%%%%%%%%%%%%%%%%%%%%%%%%%%%%%%%%%%%%%%%%%
%%%%%%%%%%%%%%%%%%%%%%%%%%%%%%%%%%%%%%%%%%%%%%%%%%%%%%%%%%%%%%%%%%%%%%%%%%%%%%%%
\section{Introduction}

\LaTeX{} provides a mechanism to structure a large document (such as a book)
into a main file and several child files (containing the chapters)
using the |\include| command.
This mechanism is beneficial for documents
which span hundreds of pages in order to
make the source file(s) more manageable.
Moreover, compilation can be restricted to
selected child files by means of the |\includeonly| command.
The latter feature can be used to reduce the compilation time while editing
(this was significantly more useful in the earlier days of \LaTeX{})
or to generate a smaller document which is easier to navigate.
Another application of |\includeonly| is to generate
documents consisting of selected parts of the complete document.

However, there are a few drawbacks of the plain |\include| mechanism:
\begin{itemize}
\item
The child files cannot be compiled on their own,
they can only be compiled via the main file.
A naive editing environment
(such as a text editor with an option
to have the current file processed by \LaTeX)
may require one to switch to the main file before compiling;
attempting to compile the child file produces errors.
\item
The main file must be modified (each time)
to adjust the |\includeonly| command
to the present needs. This easily leaves the main file in a messy state.
\item
The generated document will always carry the filename
of the main document. This is inconvenient if
several child files are to be compiled and
to be kept for distribution.
\end{itemize}

The present package provides a simple interface
to make child files individually compilable by \LaTeX{}.
Compiling a child file then has the same effect as compiling
the main file with an |\includeonly| command
to select the appropriate child.
Moreover the generated document will carry the name of the child
rather than the main file.
This resolves all three above issues.

This feature is meant to make the editing of books,
thesis documents and lecture notes somewhat more convenient.
However, the package can also be used efficiently for
composing a series of documents (such as exercise sheets)
which are typically distributed individually.
It then assists the author in generating the individual documents
(potentially in different versions)
as well as a document containing the collected series.
Another application is in developing style files
or other kinds of included material
where compilation of the style file could redirect
to a sample or test file.

%%%%%%%%%%%%%%%%%%%%%%%%%%%%%%%%%%%%%%%%%%%%%%%%%%%%%%%%%%%%%%%%%%%%%%%%%%%%%%%%
%%%%%%%%%%%%%%%%%%%%%%%%%%%%%%%%%%%%%%%%%%%%%%%%%%%%%%%%%%%%%%%%%%%%%%%%%%%%%%%%
\section{Usage}

First of all, the package \textsf{childdoc} is \emph{not} a standard
\LaTeXe{} |.sty| style file! Therefore it needs to be invoked in
a non-standard way.

%%%%%%%%%%%%%%%%%%%%%%%%%%%%%%%%%%%%%%%%%%%%%%%%%%%%%%%%%%%%%%%%%%%%%%%%%%%%%%%%
\subsection{Included Files}
\label{sec:include}

%%%%%%%%%%%%%%%%%%%%%%%%%%%%%%%%%%%%%%%%
\DescribeMacro{\childdocmain}
To use the package, add the commands
\begin{center}
\begin{tabular}{l}
|% \iffalse
%
% childdoc.dtx Copyright (C) 2017-2018 Niklas Beisert
%
% This work may be distributed and/or modified under the
% conditions of the LaTeX Project Public License, either version 1.3
% of this license or (at your option) any later version.
% The latest version of this license is in
%   http://www.latex-project.org/lppl.txt
% and version 1.3 or later is part of all distributions of LaTeX
% version 2005/12/01 or later.
%
% This work has the LPPL maintenance status `maintained'.
%
% The Current Maintainer of this work is Niklas Beisert.
%
% This work consists of the files childdoc.dtx and childdoc.ins
% and the derived files childdoc.def and cdocsamp.tex with
% cdocsch1.tex, cdocsch2.tex, cdocsdrf.tex, cdocsfn1.tex, cdocsfn2.tex.
%
%<package>\ifdefined\childdocmain\endinput\fi
%<package>\ProvidesFile{childdoc.def}[2018/12/30 v2.0 child document driver]
%<samplemain>\ProvidesFile{cdocsamp.tex}[2018/12/30 v2.0 sample for childdoc]
%<*driver>
%\ProvidesFile{childdoc.drv}[2018/12/30 v2.0 childdoc reference manual file]
\PassOptionsToClass{10pt,a4paper}{article}
\documentclass{ltxdoc}

\usepackage[margin=35mm]{geometry}
\usepackage{hyperref}
\usepackage{hyperxmp}
\usepackage[usenames]{color}

\hypersetup{colorlinks=true}
\hypersetup{pdfstartview=FitH}
\hypersetup{pdfpagemode=UseNone}
\hypersetup{pdfsource={}}
\hypersetup{pdflang={en-UK}}
\hypersetup{pdfcopyright={Copyright 2017-2018 Niklas Beisert.
  This work may be distributed and/or modified under the
  conditions of the LaTeX Project Public License, either version 1.3
  of this license or (at your option) any later version.}}
\hypersetup{pdflicenseurl={http://www.latex-project.org/lppl.txt}}
\hypersetup{pdfcontactaddress={ETH Zurich, ITP, HIT K,
  Wolfgang-Pauli-Strasse 27}}
\hypersetup{pdfcontactpostcode={8093}}
\hypersetup{pdfcontactcity={Zurich}}
\hypersetup{pdfcontactcountry={Switzerland}}
\hypersetup{pdfcontactemail={nbeisert@itp.phys.ethz.ch}}
\hypersetup{pdfcontacturl={http://people.phys.ethz.ch/\xmptilde nbeisert/}}

\newcommand{\secref}[1]{\hyperref[#1]{section \ref*{#1}}}

\parskip1ex
\parindent0pt
\let\olditemize\itemize
\def\itemize{\olditemize\parskip0pt}

\begin{document}

\title{The \textsf{childdoc} Package}
\hypersetup{pdftitle={The childdoc Package}}
\author{Niklas Beisert\\[2ex]
  Institut f\"ur Theoretische Physik\\
  Eidgen\"ossische Technische Hochschule Z\"urich\\
  Wolfgang-Pauli-Strasse 27, 8093 Z\"urich, Switzerland\\[1ex]
  \href{mailto:nbeisert@itp.phys.ethz.ch}
  {\texttt{nbeisert@itp.phys.ethz.ch}}}
\hypersetup{pdfauthor={Niklas Beisert}}
\hypersetup{pdfsubject={Manual for the LaTeX2e Package childdoc}}
\date{30 December 2018, \textsf{v2.0}}
\maketitle

\begin{abstract}\noindent
\textsf{childdoc} is a \LaTeXe{} package
that enables the direct compilation
of document sections included by |\include|
to individual files.
\end{abstract}

\begingroup
\parskip0ex
\tableofcontents
\endgroup

%%%%%%%%%%%%%%%%%%%%%%%%%%%%%%%%%%%%%%%%%%%%%%%%%%%%%%%%%%%%%%%%%%%%%%%%%%%%%%%%
%%%%%%%%%%%%%%%%%%%%%%%%%%%%%%%%%%%%%%%%%%%%%%%%%%%%%%%%%%%%%%%%%%%%%%%%%%%%%%%%
\section{Introduction}

\LaTeX{} provides a mechanism to structure a large document (such as a book)
into a main file and several child files (containing the chapters)
using the |\include| command.
This mechanism is beneficial for documents
which span hundreds of pages in order to
make the source file(s) more manageable.
Moreover, compilation can be restricted to
selected child files by means of the |\includeonly| command.
The latter feature can be used to reduce the compilation time while editing
(this was significantly more useful in the earlier days of \LaTeX{})
or to generate a smaller document which is easier to navigate.
Another application of |\includeonly| is to generate
documents consisting of selected parts of the complete document.

However, there are a few drawbacks of the plain |\include| mechanism:
\begin{itemize}
\item
The child files cannot be compiled on their own,
they can only be compiled via the main file.
A naive editing environment
(such as a text editor with an option
to have the current file processed by \LaTeX)
may require one to switch to the main file before compiling;
attempting to compile the child file produces errors.
\item
The main file must be modified (each time)
to adjust the |\includeonly| command
to the present needs. This easily leaves the main file in a messy state.
\item
The generated document will always carry the filename
of the main document. This is inconvenient if
several child files are to be compiled and
to be kept for distribution.
\end{itemize}

The present package provides a simple interface
to make child files individually compilable by \LaTeX{}.
Compiling a child file then has the same effect as compiling
the main file with an |\includeonly| command
to select the appropriate child.
Moreover the generated document will carry the name of the child
rather than the main file.
This resolves all three above issues.

This feature is meant to make the editing of books,
thesis documents and lecture notes somewhat more convenient.
However, the package can also be used efficiently for
composing a series of documents (such as exercise sheets)
which are typically distributed individually.
It then assists the author in generating the individual documents
(potentially in different versions)
as well as a document containing the collected series.
Another application is in developing style files
or other kinds of included material
where compilation of the style file could redirect
to a sample or test file.

%%%%%%%%%%%%%%%%%%%%%%%%%%%%%%%%%%%%%%%%%%%%%%%%%%%%%%%%%%%%%%%%%%%%%%%%%%%%%%%%
%%%%%%%%%%%%%%%%%%%%%%%%%%%%%%%%%%%%%%%%%%%%%%%%%%%%%%%%%%%%%%%%%%%%%%%%%%%%%%%%
\section{Usage}

First of all, the package \textsf{childdoc} is \emph{not} a standard
\LaTeXe{} |.sty| style file! Therefore it needs to be invoked in
a non-standard way.

%%%%%%%%%%%%%%%%%%%%%%%%%%%%%%%%%%%%%%%%%%%%%%%%%%%%%%%%%%%%%%%%%%%%%%%%%%%%%%%%
\subsection{Included Files}
\label{sec:include}

%%%%%%%%%%%%%%%%%%%%%%%%%%%%%%%%%%%%%%%%
\DescribeMacro{\childdocmain}
To use the package, add the commands
\begin{center}
\begin{tabular}{l}
|\input{childdoc.def}|\\
|\childdocmain{}|\\
\end{tabular}
\end{center}
at the very top of the main \LaTeX{} file,
in particular \emph{before} the |\documentclass| statement!
The argument of |\childdocmain| should be left empty
(but it must be present).

%%%%%%%%%%%%%%%%%%%%%%%%%%%%%%%%%%%%%%%%
\DescribeMacro{\childdocof}
Furthermore, add the commands
\begin{center}
\begin{tabular}{l}
|\input{childdoc.def}|\\
|\childdocof{|\textit{main}|}|\\
\end{tabular}
\end{center}
at the top of every child file \textit{child}
which is included by |\include{|\textit{child}|}|
from within the main file
(or at least for those files to be compiled individually).
The argument \textit{main} must be the filename of the main file.

There are a couple of
considerations in setting up the main and child documents:

%%%%%%%%%%%%%%%%%%%%%%%%%%%%%%%%%%%%%%%%
\paragraph{Restrictions.}

Please note the following restrictions:
\begin{itemize}
\item
|\childdocmain| must be called with one argument \textit{main}
to ensure compatibility with earlier version of the package.
It must either be empty (|\childdocmain{}|)
or precisely match the filename of the main file in which it is specified.
See \secref{sec:detection} for further information.
\item
The filename \textit{main} must be specified without the |.tex| extension.
\item
The filename \textit{main} is case sensitive
(even in case-insensitive file systems)
due to internal string comparison.
\item
The argument \textit{main} should be fully expanded, it cannot be a macro.
\item
Subdirectories and special characters should be avoided in filenames.
\item
The command |\childdocmain{|\textit{main}|}| must be followed by a whitespace.
It should not be followed immediately by another command
or by a comment mark `|%|'.
This is because the \TeX{} parser reads the token immediately following
the argument of |\childdocmain| and puts it
at the beginning of every child section;
however, a white\-space is ignored.
\end{itemize}

%%%%%%%%%%%%%%%%%%%%%%%%%%%%%%%%%%%%%%%%
\paragraph{Content of Main File.}

It is advisable to place all content in the child files included by |\include|.
Any output contained in the main file will appear in all child documents
unless suppressed manually;
it cannot be suppressed automatically by the |\includeonly| directive
and thus should normally be avoided.
A method to include some content in the main file
by means of conditional processing is described in \secref{sec:conditional}.

%%%%%%%%%%%%%%%%%%%%%%%%%%%%%%%%%%%%%%%%
\paragraph{Page Numbering.}

When only a part of the document is compiled,
the appropriate numbering of pages
(as well as other status parameters)
is determined from the |.aux| files.
The latter contain information from previous passes.
However this information needs to propagate through
all intermediate child documents.
Therefore the page numbering in child documents may well
be inconsistent until the complete document is compiled at least once.

A useful (if unconventional) way to always ensure a consistent
page numbering is to restart the numbering in each child document
and denote the pages by `\textit{child}|.|\textit{page}'
where \textit{child} represents the chapter/section number of the child file.
This can be achieved by the command
|\numberwithin{page}{|\textit{child}|}|
of the \textsf{amsmath} package
where \textit{child} can be |chapter| or |section|
depending on the chosen structuring.
Alternatively, one can modify the macro |\thepage| appropriately
and reset the counter |page| at the start of each child file.

%%%%%%%%%%%%%%%%%%%%%%%%%%%%%%%%%%%%%%%%%%%%%%%%%%%%%%%%%%%%%%%%%%%%%%%%%%%%%%%%
\subsection{Conditional Processing}
\label{sec:conditional}

The package provides a mechanism to compile different versions
of a document. To customise the versions further some conditional processing
can come in handy to distinguish which version is being compiled.
The package provides two macros to describe the compilation context:

%%%%%%%%%%%%%%%%%%%%%%%%%%%%%%%%%%%%%%%%
\DescribeMacro{\ifchilddoc}
The conditional |\ifchilddoc| distinguishes between the compilation of
child documents and the main document:
%
\begin{center}
|\ifchilddoc |\textit{child-code}| |[|\||else |\textit{main-code}]| \||fi|
\end{center}

%%%%%%%%%%%%%%%%%%%%%%%%%%%%%%%%%%%%%%%%
\DescribeMacro{\childdocname}
\DescribeMacro{\childdocjob}
The macro |\childdocname| contains the filename (without extension)
of the main or child file being processed.
Note that |\childdocjob| will always contain the name of the main file.

%%%%%%%%%%%%%%%%%%%%%%%%%%%%%%%%%%%%%%%%
\paragraph{Title Page.}

Conditional processing can be used to include a title or banner page
in the main document when proper precautions are taken.
Importantly, the code in the main file should ensure that the page counter
(as well as other status parameters which are stored in the |.aux| files)
takes the same value after the conditional processing.
Otherwise the page numbers may take divergent values
depending on which part is compiled.

For example, a title page could be declared by:
%
\begin{center}
\begin{tabular}{l}
|\ifchilddoc\||else|\\
|\addtocounter{page}{-1}|\\
\textit{code for title page}\\
|\newpage|\\
|\||fi|
\end{tabular}
\end{center}
%
A banner page for the child documents can be generated by:
%
\begin{center}
\begin{tabular}{l}
|\ifchilddoc|\\
|\addtocounter{page}{-1}|\\
\textit{code for banner page}\\
|\newpage|\\
|\||fi|
\end{tabular}
\end{center}
%
Here one could write a message such as:
\begin{center}
|This is the part \childdocname{} of \childdocjob{}.|
\end{center}

%%%%%%%%%%%%%%%%%%%%%%%%%%%%%%%%%%%%%%%%%%%%%%%%%%%%%%%%%%%%%%%%%%%%%%%%%%%%%%%%
\subsection{Flags}
\label{sec:flags}

The package makes it easy to generate different versions
of the main or child documents.
To this end compilation flags can be defined
and assigned different default values.
They will be particularly useful in conjunction
with the forwarding mechanism described in \secref{sec:forward}.

For example, it may be useful to have a flag |\version|
which can be set to |draft| or |final|.
The document source will contain some conditional code
depending on the value of |\version|.
Suppose further, the flag should default to |final| for the main file
and to |draft| for child files
which is a natural assignment for editing the document.
This is achieved by placing the following code
in the preamble of the main document
(below the |\childdocmain| directive):
%
\begin{center}
\begin{tabular}{l}
|\ifchilddoc|\\
|\providecommand{\version}{draft}|\\
|\||else|\\
|\providecommand{\version}{final}|\\
|\||fi|
\end{tabular}
\end{center}
%
The definition by |\providecommand| makes sure
that previous definitions are not overwritten.
Further statements |\providecommand{\version}{...}|
can thus be added before the above code to override it.

For the main file, one might add a line
(between |\childdocmain| and the above block)
%
\begin{center}
|%\ifchilddoc\||else\providecommand{\version}{draft}\||fi|
\end{center}
%
which can be uncommented to produce a draft version.
Likewise one can add a line to the very top of a child file
(above the |\childdocof{|\textit{main}|}| directive)
%
\begin{center}
|%\providecommand{\version}{final}|
\end{center}
%
which can be uncommented to produce the final version of this child document.

%%%%%%%%%%%%%%%%%%%%%%%%%%%%%%%%%%%%%%%%%%%%%%%%%%%%%%%%%%%%%%%%%%%%%%%%%%%%%%%%
\subsection{Forwarding}
\label{sec:forward}

Different versions of the main or child documents
using compilation flags as described in \secref{sec:flags}
can be (permanently) stored in different files
for convenient compilation, viewing and distribution.
To this end, the package defines a command
to pass on compilation to a different file:

%%%%%%%%%%%%%%%%%%%%%%%%%%%%%%%%%%%%%%%%
\DescribeMacro{\childdocforward}
The command |\childdocforward| redirects processing to
another source file:
%
\begin{center}
\begin{tabular}{l}
|\input{childdoc.def}|\\
|\childdocforward[|\textit{main}|]{|\textit{dest}|}|\\
\end{tabular}
\end{center}
%
The argument \textit{dest} is the destination file
(without extension).
It should be the main file or one of the child files.
Note that further \textsf{childdoc} directives
such as |\childdocof| and |\childdocforward|
in the indicated file will be processed in this form.
The optional argument \textit{main}
passes on directly to the main file \textit{main}
while pretending to compile the child \textit{dest}.
This form behaves as if \textit{dest}
issues |\childdocof{|\textit{main}|}| right away,
and no further \textsf{childdoc} directives will be processed.

%%%%%%%%%%%%%%%%%%%%%%%%%%%%%%%%%%%%%%%%
\DescribeMacro{\...prefix}
In the alternative form |\childdocforwardprefix|,
%
\begin{center}
\begin{tabular}{l}
|\input{childdoc.def}|\\
|\childdocforwardprefix[|\textit{main}|]{|\textit{prefix}|}{|\textit{dest}|}|
\end{tabular}
\end{center}
%
the destination file is determined by a pattern
depending on the current file:
To make this work, the current file must be called
`{\textit{prefix}\hspace{0.2em}\textit{suffix}}'
with \textit{prefix} matching precisely the argument.
Processing is then passed on to the file
`{\textit{dest}\hspace{0.2em}\textit{suffix}}'.
Surely, the same effect is achieved by
directly specifying the
argument `{\textit{dest}\hspace{0.2em}\textit{suffix}}'
in the first form.
However, that requires to set up a different file
for each child. With the alternative form of the command
all these files can have exactly the same content
which simplifies setting them up and maintaining them.

For example, the following file |draft.tex|
with a compilation flag |\version| as described in \secref{sec:flags}
compiles the main document as a draft:
%
\begin{center}
\begin{tabular}{l}
|\def\version{draft}|\\
|\input{childdoc.def}|\\
|\childdocforward{|\textit{main}|}|
\end{tabular}
\end{center}
%
Likewise, the following files |final|\textit{nn}|.tex|
compile the final version of the child document
|child|\textit{nn}|.tex|:
%
\begin{center}
\begin{tabular}{l}
|\def\version{final}|\\
|\input{childdoc.def}|\\
|\childdocforwardprefix{final}{child}|
\end{tabular}
\end{center}
%

Note that when several versions of a main file and/or of each child file
are to be generated, it may be convenient to set up a |Makefile| or
shell script to automatise the process.

%%%%%%%%%%%%%%%%%%%%%%%%%%%%%%%%%%%%%%%%%%%%%%%%%%%%%%%%%%%%%%%%%%%%%%%%%%%%%%%%
\subsection{Command Line Processing}
\label{sec:commandline}

The effect of redirection files can also be achieved by invoking
the \LaTeX{} compiler with a more elaborate command line.
Most conveniently this should be done as part
of a shell script or a |Makefile|.

When using \textsf{childdoc} in the main file, the following
command lines effectively perform a redirection
(note that depending on the shell being used,
backslashes may have to be doubled: `|\|' $\to$ `|\\|'):
%
\begin{center}
|... -jobname "|\textit{target}|" |\\|"|[\textit{flags}]%
|\input{childdoc.def}\childdocforward[|\textit{main}|]{|\textit{dest}|}"|
\end{center}
%
Here \textit{target} is the name of the output file,
\textit{main} is the name of the main file
and \textit{dest} is the name of the main or child file to be processed
(all filenames without extensions).
The optional argument \textit{main} can be omitted
if \textit{main} matches \textit{dest}.
Optionally, compilation \textit{flags} can be defined via |\def| commands.
This command line makes the \TeX{} engine believe
it is compiling the file \textit{target}
whose content is specified as the latter parameter.
The provided code then forwards the processing to
\textit{main} or \textit{dest} as described in \secref{sec:forward}.

%%%%%%%%%%%%%%%%%%%%%%%%%%%%%%%%%%%%%%%%%%%%%%%%%%%%%%%%%%%%%%%%%%%%%%%%%%%%%%%%
\subsection{Include by Input}
\label{sec:input}

Including child documents by |\include| has some restrictions by design.
Most notably, the content of a child document always occupies
its own set of pages; pages cannot be shared between child documents.
Usually, this behaviour makes perfect sense
because each child document contain an essential part of the document.
However, in some situations it may be desirable to compose
a document from a collection of parts
without having mandatory page breaks between then.
For this case, the package
provides a mechanism to include parts
by |\input| which can also be processed individually.
However, by construction this mechanism
requires manual handling of the content to be output.

%%%%%%%%%%%%%%%%%%%%%%%%%%%%%%%%%%%%%%%%
\DescribeMacro{\ifchilddocmanual}
The main file should be prepared as usual, see \secref{sec:include}.
However, the document body must make a distinction
between processing of an individual part and of the main document, e.g.:
%
\begin{center}
\begin{tabular}{l}
|\ifchilddocmanual|\\
|\input{\childdocname}|\\
|\||else|\\
\textit{document body with }|\input{|\textit{part}|}|\\
|\||fi|
\end{tabular}
\end{center}
%
The conditional |\ifchilddocmanual| is true whenever
a part to be included by |\input| is being compiled,
and the name of the part is stored in |\childdocname|.

%%%%%%%%%%%%%%%%%%%%%%%%%%%%%%%%%%%%%%%%
\DescribeMacro{\childdocby}
Each part to be included by |\input| should start with:
%
\begin{center}
\begin{tabular}{l}
|\input{childdoc.def}|\\
|\childdocby{|\textit{main}|}|\\
\end{tabular}
\end{center}
%
The directive |\childdocby| is similar to |\childdocof|
described in \secref{sec:include},
but the subsequent selection of content must be done manually.
To that end, both |\ifchilddoc| and |\ifchilddocmanual|
will be true upon processing of a part,
and the name of the part is stored in |\childdocname|.
Note that |\jobname| will be set to the filename of the current part
so that each part receives an individual |.aux| file
that does not interfere with the |.aux| file(s) of the main document.
This behaviour can be altered by the alternative form
|\childdocby[*]{|\textit{main}|}| (with a non-empty optional argument)
which uses the |.aux| file of the main document
by setting |\jobname| to \textit{main}.

%%%%%%%%%%%%%%%%%%%%%%%%%%%%%%%%%%%%%%%%%%%%%%%%%%%%%%%%%%%%%%%%%%%%%%%%%%%%%%%%
\subsection{Driver Development}
\label{sec:driver}

The \textsf{childdoc} mechanism can also be use for the development
of definition files such as \LaTeX{} styles or classes.
This case differs from the above setup with multiple parts
included by |\include| in that no |\includeonly| should be invoked.
This can be achieved by starting the include file
(before |\ProvidesPackage|) with:
%
\begin{center}
\begin{tabular}{l}
|\input{childdoc.def}|\\
|\childdocforward{|\textit{main}|}|\\
\end{tabular}
\end{center}
%
or alternatively with:
%
\begin{center}
\begin{tabular}{l}
|\input{childdoc.def}|\\
|\childdocby{|\textit{main}|}|\\
\end{tabular}
\end{center}
%
Both forms have slightly different effects as described above.
The main file is prepared as usual, see \secref{sec:include}.

%%%%%%%%%%%%%%%%%%%%%%%%%%%%%%%%%%%%%%%%%%%%%%%%%%%%%%%%%%%%%%%%%%%%%%%%%%%%%%%%
\subsection{Legacy Detection}
\label{sec:detection}

The directive |\childdocmain| in the main file can detect
whether the complete document or merely a child is to be compiled
even without using the directive |\childdocof|.
This method is deprecated because it is less robust
and there is no compelling reason to use it;
it is merely provided for backward compatibility
and it may be removed in future versions.

If the detection mechanism is to be used,
it is mandatory to correctly specify
the filename of the main file as the argument of |\childdocmain|:
%
\begin{center}
\begin{tabular}{l}
|\input{childdoc.def}|\\
|\childdocmain{|\textit{main}|}|\\
\end{tabular}
\end{center}
%
If |\jobname| does not match the argument \textit{main} of |\childdocmain|,
it is assumed that |\jobname| points to the child file to be compiled.
When using |\childdocmain| with the main file specified as argument,
it suffices to start a child file
with just |\input{|\textit{main}|}|
without loading of the package and using |\childdocof|.
If instead all processing is done
with the appropriate \textsf{childdoc} directives,
the argument of \textit{main} of |\childdocmain| can be empty.

An alternative version of the command line processing described
in \secref{sec:commandline} using the detection mechanism reads:
%
\begin{center}
|... -jobname "|\textit{target}|" "|[\textit{flags}]%
[|\def\jobname{|\textit{dest}|}|]|\input{|\textit{main}|}"|
\end{center}

%%%%%%%%%%%%%%%%%%%%%%%%%%%%%%%%%%%%%%%%%%%%%%%%%%%%%%%%%%%%%%%%%%%%%%%%%%%%%%%%
\subsection{Manual Code}
\label{sec:manual}

In case one cannot be certain whether the definitions file |childdoc.def|
is installed on the target \TeX{} distribution
and one prefers not to ship it,
it is conceivable to paste a few relevant commands into the sources.

To that end, drop all statements |\input{childdoc.def}|
and perform the replacements as outlined below.
Instead of |\childdocmain{|\textit{main}|}| add the following code
to the top of the main file:
%
\begin{center}
\begin{tabular}{l}
|\||ifdefined\childdocname\endinput\||fi\newif\ifchilddoc|\\
|\edef\childdocname{\scantokens\expandafter{\jobname\noexpand}}|\\
|\def\childdocmain{|\textit{main}|}\||ifx\childdocmain\childdocname\||else|\\
|\childdoctrue\includeonly{\childdocname}\let\jobname\childdocmain\||fi|\\
\end{tabular}
\end{center}
%
Instead of |\childdocof{|\textit{main}|}| just include the main file
at the top of each child file:
%
\begin{center}
|\input{|\textit{main}|}|
\end{center}
%
A simple redirection |\childdocforward{|\textit{dest}|}| is achieved by:
%
\begin{center}
|\def\jobname{|\textit{dest}|}\input{\jobname}|
\end{center}
%
The redirection with prefix
|\childdocforwardprefix[|\textit{prefix}|]{|\textit{dest}|}|
is accomplished by:
%
\begin{center}
\begin{tabular}{l}
|{\edef\jobname{\scantokens\expandafter{\jobname\noexpand}}|\\
|\def\redirectjob |\textit{prefix}|#1~~~{\gdef\jobname{|\textit{dest}|#1}}|\\
|\expandafter\redirectjob\jobname~~~}\input{\jobname}|
\end{tabular}
\end{center}

In an alternative approach,
child documents can be compiled by a specific command line
without additional code or specific definitions:
%
\begin{center}
|... -jobname "|\textit{target}|" "|[\textit{flags}]%
|\includeonly{|\textit{dest}|}\input{|\textit{main}|}"|
\end{center}
%

%%%%%%%%%%%%%%%%%%%%%%%%%%%%%%%%%%%%%%%%%%%%%%%%%%%%%%%%%%%%%%%%%%%%%%%%%%%%%%%%
%%%%%%%%%%%%%%%%%%%%%%%%%%%%%%%%%%%%%%%%%%%%%%%%%%%%%%%%%%%%%%%%%%%%%%%%%%%%%%%%
\section{Information}

%%%%%%%%%%%%%%%%%%%%%%%%%%%%%%%%%%%%%%%%%%%%%%%%%%%%%%%%%%%%%%%%%%%%%%%%%%%%%%%%
\subsection{Copyright}

Copyright \copyright{} 2017--2018 Niklas Beisert

This work may be distributed and/or modified under the
conditions of the \LaTeX{} Project Public License, either version 1.3
of this license or (at your option) any later version.
The latest version of this license is in
  \url{http://www.latex-project.org/lppl.txt}
and version 1.3 or later is part of all distributions of \LaTeX{}
version 2005/12/01 or later.

This work has the LPPL maintenance status `maintained'.

The Current Maintainer of this work is Niklas Beisert.

This work consists of the files |README.txt|, |childdoc.ins| and |childdoc.dtx|
as well as the derived files |childdoc.def|, |cdocsamp.tex|
with |cdocsch1.tex|, |cdocsch2.tex|, |cdocspt3.tex|, |cdocspt4.tex|,
|cdocsdrf.tex|, |cdocsfn1.tex|, |cdocsfn2.tex|
as well as |childdoc.pdf|.

%%%%%%%%%%%%%%%%%%%%%%%%%%%%%%%%%%%%%%%%%%%%%%%%%%%%%%%%%%%%%%%%%%%%%%%%%%%%%%%%
\subsection{Files and Installation}

The package consists of the files:
%
\begin{center}
\begin{tabular}{ll}
    |README.txt|   & readme file \\
    |childdoc.ins| & installation file \\
    |childdoc.dtx| & source file \\
    |childdoc.def| & definition file \\
    |cdocsamp.tex| & sample main file \\
    |cdocsch1.tex| & sample include file \\
    |cdocsch2.tex| & sample include file \\
    |cdocspt3.tex| & sample part file \\
    |cdocspt4.tex| & sample part file \\
    |cdocsdrf.tex| & sample redirection file \\
    |cdocsfn1.tex| & sample redirection file \\
    |cdocsfn2.tex| & sample redirection file \\
    |childdoc.pdf| & manual
\end{tabular}
\end{center}
%
The distribution consists of the files
|README.txt|, |childdoc.ins| and |childdoc.dtx|.
%
\begin{itemize}
\item
Run (pdf)\LaTeX{} on |childdoc.dtx|
to compile the manual |childdoc.pdf| (this file).
\item
Run \LaTeX{} on |childdoc.ins| to create the definitions file |childdoc.def|
and the sample |cdocsamp.tex| with include files
|cdocsch1.tex|, |cdocsch2.tex|, |cdocspt3.tex|, |cdocspt4.tex|,
|cdocsdrf.tex|, |cdocsfn1.tex|, |cdocsfn2.tex|.
Then copy the file |childdoc.def| to an appropriate directory of your \LaTeX{}
distribution, e.g.\ \textit{texmf-root}|/tex/latex/childdoc|.
\end{itemize}

%%%%%%%%%%%%%%%%%%%%%%%%%%%%%%%%%%%%%%%%%%%%%%%%%%%%%%%%%%%%%%%%%%%%%%%%%%%%%%%%
\subsection{Related CTAN Packages}

There are several other packages which offer a similar functionality:
%
\begin{itemize}
\item
The packages
\href{http://ctan.org/pkg/docmute}{\textsf{docmute}},
\href{http://ctan.org/pkg/includex}{\textsf{includex}} and
\href{http://ctan.org/pkg/standalone}{\textsf{standalone}}
provide commands to include only the document body of
a child file thus allowing both files to be compiled individually.
\item
The packages \href{http://ctan.org/pkg/subdocs}{\textsf{subdocs}}
and \href{http://ctan.org/pkg/subfiles}{\textsf{subfiles}}
provide structures in which the main and child documents can be
encapsulated and allowing them to be compiled individually.
The inclusion mechanism is different from the conventional |\include|.
\item
The package \href{http://ctan.org/pkg/combine}{\textsf{combine}}
is an elaborate solution to combine several documents into one.
\end{itemize}
%
See also the CTAN topic \href{http://ctan.org/topic/subdocs}{\textsf{subdocs}}
for further related packages.
The present package differs from the above solutions in that
a document structure constructed with the conventional |\include| mechanism
just needs two extra commands at the top of every file
such that all constituent files can be compiled individually.

%%%%%%%%%%%%%%%%%%%%%%%%%%%%%%%%%%%%%%%%%%%%%%%%%%%%%%%%%%%%%%%%%%%%%%%%%%%%%%%%
%\subsection{Feature Suggestions}
%
%The following is a list of features which may be useful for future
%versions of this package:
%%
%\begin{itemize}
%\item
%\ldots
%\end{itemize}

%%%%%%%%%%%%%%%%%%%%%%%%%%%%%%%%%%%%%%%%%%%%%%%%%%%%%%%%%%%%%%%%%%%%%%%%%%%%%%%%
\subsection{Revision History}

%%%%%%%%%%%%%%%%%%%%%%%%%%%%%%%%%%%%%%%%
\paragraph{v2.0:} 2018/12/30

\begin{itemize}
\item
immediate forward processing
\item
added |\childdocby| mechanism
\item
manual restructured
\end{itemize}

%%%%%%%%%%%%%%%%%%%%%%%%%%%%%%%%%%%%%%%%
\paragraph{v1.6:} 2018/01/17

\begin{itemize}
\item
application for development of include files
\item
corrections to manual
\end{itemize}

%%%%%%%%%%%%%%%%%%%%%%%%%%%%%%%%%%%%%%%%
\paragraph{v1.5:} 2017/05/21

\begin{itemize}
\item
more complete structuring introduced
\item
|\childdocof| introduced
\item
|\childdoc| renamed to |\childdocmain|
\item
|\childredirect| renamed to |\childdocforward| and |\childdocforwardprefix|
and functionality expanded
\end{itemize}

%%%%%%%%%%%%%%%%%%%%%%%%%%%%%%%%%%%%%%%%
\paragraph{v1.0:} 2017/04/27

\begin{itemize}
\item
manual and install package
\item
first version published on CTAN
\end{itemize}

%%%%%%%%%%%%%%%%%%%%%%%%%%%%%%%%%%%%%%%%
\paragraph{v0.6:} 2017/04/26

\begin{itemize}
\item
redirection mechanism added
\end{itemize}

%%%%%%%%%%%%%%%%%%%%%%%%%%%%%%%%%%%%%%%%
\paragraph{v0.5:} 2017/04/26

\begin{itemize}
\item
functionality in definition file
\end{itemize}


%%%%%%%%%%%%%%%%%%%%%%%%%%%%%%%%%%%%%%%%%%%%%%%%%%%%%%%%%%%%%%%%%%%%%%%%%%%%%%%%
%%%%%%%%%%%%%%%%%%%%%%%%%%%%%%%%%%%%%%%%%%%%%%%%%%%%%%%%%%%%%%%%%%%%%%%%%%%%%%%%
%%%%%%%%%%%%%%%%%%%%%%%%%%%%%%%%%%%%%%%%%%%%%%%%%%%%%%%%%%%%%%%%%%%%%%%%%%%%%%%%
\appendix

\settowidth\MacroIndent{\rmfamily\scriptsize 000\ }

 \DocInput{childdoc.dtx}

\end{document}
%</driver>
% \fi
%
% %%%%%%%%%%%%%%%%%%%%%%%%%%%%%%%%%%%%%%%%%%%%%%%%%%%%%%%%%%%%%%%%%%%%%%%%%%%%%%
% %%%%%%%%%%%%%%%%%%%%%%%%%%%%%%%%%%%%%%%%%%%%%%%%%%%%%%%%%%%%%%%%%%%%%%%%%%%%%%
% \section{Sample}
%\iffalse
%<*samplemain>
%\fi
%
% The following presents a sample document
% with two chapters, two parts, a title page,
% a compile flag as well as three forwarding files to set the flag.
% It consists of eight |.tex| files:
% \begin{center}
% \begin{tabular}{ll}
% |cdocsamp.tex|&main file\\
% |cdocsch1.tex|&include file for chapter 1\\
% |cdocsch2.tex|&include file for chapter 2\\
% |cdocspt3.tex|&include file for part 3\\
% |cdocspt4.tex|&include file for part 4\\
% |cdocsdrf.tex|&forwarding file for main file in draft mode\\
% |cdocsfi1.tex|&forwarding file for final version of chapter 1\\
% |cdocsfi2.tex|&forwarding file for final version of chapter 2\\
% \end{tabular}
% \end{center}
% Each of the eight files can be compiled directly by the \LaTeX{} compiler.
%
% %%%%%%%%%%%%%%%%%%%%%%%%%%%%%%%%%%%%%%
% \paragraph{Main File.}
%
% The main file is called |cdocsamp.tex|.
%
% Load the \textsf{childdoc} definitions and
% declare the filename for the main document:
%    \begin{macrocode}
\input{childdoc.def}
\childdocmain{}
%    \end{macrocode}

% Optional override for |\version| flag:
%    \begin{macrocode}
%%\ifchilddoc\else\providecommand{\version}{draft}\fi
%    \end{macrocode}

% Define the default values for the |\version| flag
% (|final| for the main file and |draft| for childs):
%    \begin{macrocode}
\ifchilddoc
\providecommand{\version}{draft}
\else
\providecommand{\version}{final}
\fi
%    \end{macrocode}

% Load the standard document class:
%    \begin{macrocode}
\documentclass[12pt]{article}
%    \end{macrocode}

% Start the document body:
%    \begin{macrocode}
\begin{document}
%    \end{macrocode}

% Declare a title page.
% Print title, part of document being processed and version flag:
%    \begin{macrocode}
\addtocounter{page}{-1}
\begin{center}
{\LARGE\bfseries{}childdoc example\par}
\vspace{1cm}
\ifchilddoc
\ifchilddocmanual part\else chapter\fi:
`\childdocname' of `\childdocjob'\par
\else
main document: `\childdocjob'\par
\fi
version: \version\par
\end{center}
\newpage
%    \end{macrocode}

% Manually include selected file,
% otherwise process as usual:
%    \begin{macrocode}
\ifchilddocmanual
\section*{part `\childdocname'}
\input{\childdocname}
\else
%    \end{macrocode}

% Include the two chapters:
%    \begin{macrocode}
\include{cdocsch1}
\include{cdocsch2}
%    \end{macrocode}

% Include the two parts unless only chapters should be displayed:
%    \begin{macrocode}
\ifchilddoc\else
\section{part three}
\input{cdocspt3}
\section{part four}
\input{cdocspt4}
\fi
%    \end{macrocode}

% Process as usual until here:
%    \begin{macrocode}
\fi
%    \end{macrocode}

% End of document body:
%    \begin{macrocode}
\end{document}
%    \end{macrocode}
%\iffalse
%</samplemain>
%\fi
%
% %%%%%%%%%%%%%%%%%%%%%%%%%%%%%%%%%%%%%%
% \paragraph{Chapter Include Files.}
%
% The include files are called |cdocsch1.tex| and |cdocsch2.tex|.
%
%\iffalse
%<*samplechap1|samplechap2>
%\fi

% Optional override for |\version| flag:
%    \begin{macrocode}
%%\providecommand{\version}{final}
%    \end{macrocode}

% Include the main document:
%    \begin{macrocode}
\input{childdoc.def}
\childdocof{cdocsamp}
%    \end{macrocode}

%\iffalse
%</samplechap1|samplechap2>
%\fi
%
%\iffalse
%<*samplechap1>
%\fi
% Some text for chapter 1:
%    \begin{macrocode}
\section{one}
some text in chapter one
%    \end{macrocode}

%\iffalse
%</samplechap1>
%\fi
% Some text for chapter 2:
%\iffalse
%<*samplechap2>
%\fi
%    \begin{macrocode}
\section{two}
more text in chapter two
%    \end{macrocode}

%\iffalse
%</samplechap2>
%\fi
%
% %%%%%%%%%%%%%%%%%%%%%%%%%%%%%%%%%%%%%%
% \paragraph{Part Include Files.}
%
% The include files are called |cdocspt3.tex| and |cdocspt4.tex|.
%
%\iffalse
%<*samplepart3|samplepart4>
%\fi

% Optional override for |\version| flag:
%    \begin{macrocode}
%%\providecommand{\version}{final}
%    \end{macrocode}

% Include the main document:
%    \begin{macrocode}
\input{childdoc.def}
\childdocby{cdocsamp}
%    \end{macrocode}

%\iffalse
%</samplepart3|samplepart4>
%\fi
%
%\iffalse
%<*samplepart3>
%\fi
% Some text for part 3:
%    \begin{macrocode}
some text in part three
%    \end{macrocode}

%\iffalse
%</samplepart3>
%\fi
% Some text for part 4:
%\iffalse
%<*samplepart4>
%\fi
%    \begin{macrocode}
more text in part four
%    \end{macrocode}

%\iffalse
%</samplepart4>
%\fi
%
% %%%%%%%%%%%%%%%%%%%%%%%%%%%%%%%%%%%%%%
% \paragraph{Forwarding for a Complete Draft.}
%
% The following forwarding file |cdocsdrf.tex|
% compiles the main document in draft mode:
%\iffalse
%<*sampledraft>
%\fi
%    \begin{macrocode}
\def\version{draft}
\input{childdoc.def}
\childdocforward{cdocsamp}
%    \end{macrocode}

%\iffalse
%</sampledraft>
%\fi
%
% %%%%%%%%%%%%%%%%%%%%%%%%%%%%%%%%%%%%%%
% \paragraph{Forwarding for Final Version of the Chapters.}
%
% The following forwarding files |cdocsfn1.tex| and |cdocsfn2.tex|
% (with identical content)
% compile the final versions of the child documents
% |cdocsch1.tex| and |cdocsch2.tex|, respectively:
%\iffalse
%<*samplefinal>
%\fi
%    \begin{macrocode}
\def\version{final}
\input{childdoc.def}
\childdocforwardprefix[cdocsamp]{cdocsfn}{cdocsch}
%    \end{macrocode}

%\iffalse
%</samplefinal>
%\fi
%
% %%%%%%%%%%%%%%%%%%%%%%%%%%%%%%%%%%%%%%
% \paragraph{Command Line Processing.}
%
% The following three command lines generate the output files
% |cdocscld|, |cdocscl1| and |cdocscl2|
% which should be identical to
% |cdocsdrf|, |cdocsch1| and |cdocsfn2|, respectively:
% \begin{center}
% \begin{tabular}{l}
% |latex -jobname cdocscld \|\\
% |  "\def\version{draft}\input{childdoc.def}\childdocforward{cdocsamp}"|\\
% |latex -jobname cdocscl1 \|\\
% |  "\input{childdoc.def}\childdocforward[cdocsamp]{cdocsch1}"|\\
% |latex -jobname cdocscl2 \|\\
% |  "\def\version{final}\input{childdoc.def}\childdocforward{cdocsch2}"|
% \end{tabular}
% \end{center}
% Note that the trailing backslash on each first line
% merely continues the input to the second line
% (for convenient cut ant paste).
% Furthermore, the command |latex| can be replaced by any
% of its alternative versions such as |pdflatex|.
%
% %%%%%%%%%%%%%%%%%%%%%%%%%%%%%%%%%%%%%%%%%%%%%%%%%%%%%%%%%%%%%%%%%%%%%%%%%%%%%%
% %%%%%%%%%%%%%%%%%%%%%%%%%%%%%%%%%%%%%%%%%%%%%%%%%%%%%%%%%%%%%%%%%%%%%%%%%%%%%%
% \section{Implementation}
%\iffalse
%<*package>
%\fi
%
% This section describes the definitions file |childdoc.def|.

% The definitions cannot be loaded using |\usepackage| or |\RequirePackage|
% which has a mechanism to prevent loading a style file more than once.
% When loading the definitions by means of |\input|
% multiple instances have to be prevented manually:
%\iffalse
%This code needs to be before the `\ProvidesFile' directive
%which is defined at the beginning of this file.
%Therefore it is also placed there and commented out here.
%</package>
%<*discard>
%\fi
%    \begin{macrocode}
\ifdefined\childdocmain\endinput\fi
%    \end{macrocode}
%\iffalse
%</discard>
%<*package>
%\fi
%
% \macro{\ifchilddoc}
% \macro{\ifchilddocmanual}
% The conditional |\ifchilddoc| tells whether a
% child (true) or main (false) document is being compiled.
% The conditional |\ifchilddocmanual| tells whether
% the |\includeonly| mechanism is used (false) or
% the selection of child files must be performed manually (true).
% The definitions initialise to false:
%    \begin{macrocode}
\newif\ifchilddoc
\newif\ifchilddocmanual
%    \end{macrocode}

% \macro{\childdocname}
% \macro{\childdocjob}
% The macro |\childdocname| stores the name of the main document
% to be compiled. The macro |\childdocjob| stores the name of
% the document on which the \LaTeX{} compiler was originally invoked.
% The content of |\jobname| cannot be compared
% to filenames specified in the source due to different catcodes.
% The following code rescans |\jobname|, stores the result
% in |\childdocname| and saves a copy in |\childdocjob|:
%    \begin{macrocode}
\edef\childdocname{\scantokens\expandafter{\jobname\noexpand}}
\let\childdocjob\childdocname
%    \end{macrocode}

% \macro{\childdocdisable}
% The macro |\childdocdisable| prevents the main file
% from being processed more than once.
% At this stage, the main document command |\childdocmain|
% is assumed to be called once again where it should do nothing.
% Any subsequent call to it should prevent
% a secondary processing of the main document
% It overwrites the forwarding commands
% |\childdocof| and |\childdocforward|
% with empty macros to prevent further inclusions of the main document:
%    \begin{macrocode}
\newcommand{\childdocdisable}
{
  \renewcommand{\childdocmain}[1]{\renewcommand{\childdocmain}[1]{\endinput}}
  \renewcommand{\childdocof}[1]{}
  \renewcommand{\childdocby}[2][]{}
  \renewcommand{\childdocforward}[2][]{}
  \renewcommand{\childdocdisable}{}
}
%    \end{macrocode}

% \macro{\childdocmain}
% The macro |\childdocmain| is to be called at the top of the main file
% with nothing or the main filename (without extension) as argument.
% First, it breaks loops.
% If the argument is not empty and does not match |\childdocname|
% (which is set by the first inclusion of |childdoc.def|),
% |\ifchilddoc| is set to true, |\includeonly| is applied to the child file
% and |\jobname| is set to the main file
% (for proper handling of |.aux| files):
%    \begin{macrocode}
\newcommand{\childdocmain}[1]
{
  \childdocdisable\childdocmain{}
  \if?#1?\else
    \begingroup
      \def\childdoctmp{#1}
      \ifx\childdoctmp\childdocname
        \def\childdoctmp{}
      \else
        \def\childdoctmp
        {
          \childdoctrue
          \includeonly{\childdocname}
          \def\childdocjob{#1}
          \def\jobname{#1}
        }
      \fi
      \expandafter
    \endgroup
    \childdoctmp
  \fi
}
%    \end{macrocode}

% \macro{\childdocof}
% The command |\childdocof| redirects
% compilation to the main file |#1|.
%    \begin{macrocode}
\newcommand{\childdocof}[1]
{
  \childdocdisable
  \childdoctrue
  \includeonly{\childdocname}
  \def\jobname{#1}
  \def\childdocjob{#1}
  \input{#1}
}
%    \end{macrocode}

% \macro{\childdocby}
% The command |\childdocby| ....
%    \begin{macrocode}
\newcommand{\childdocby}[2][]
{
  \childdocdisable
  \childdoctrue
  \childdocmanualtrue
  \if?#1?\else
    \def\jobname{#2}
  \fi
  \def\childdocjob{#2}
  \input{#2}
  \endinput
}
%    \end{macrocode}

% \macro{\childdocforward}
% The command |\childdocforward| redirects
% compilation to the main file or
% (if the optional argument is given) a child file.
% Parameters are set as if the main file
% or a child file starting with |\childdocof| was compiled.
% Then compilation is handed over to the main file:
%    \begin{macrocode}
\newcommand{\childdocforward}[2][]
{
  \begingroup
    \if?#1?
      \def\childdoctmp
      {
        \def\childdocname{#2}
        \def\childdocjob{#2}
        \def\jobname{#2}
        \input{#2}
        \endinput
      }
    \else
      \def\childdoctmp
      {
        \childdocdisable
        \def\childdocname{#2}
        \childdoctrue
        \includeonly{#2}
        \def\childdocjob{#1}
        \def\jobname{#1}
        \input{#1}
        \endinput
      }
    \fi
    \expandafter
  \endgroup
  \childdoctmp
}
%    \end{macrocode}

% \macro{\childdocforwardprefix}
% The command |\childdocforwardprefix| redirects
% compilation to the main or a child file by means of a pattern.
% The prefix |#1| in the current filename is replaced by |#2|
% and the suffix of the current filename is kept
% (it is assumed that the filename does not contain the substring `|~~~|'
% which is used as a delimiter).
% Compilation is handed over to the new file by |\childdocforward|:
%    \begin{macrocode}
\newcommand{\childdocforwardprefix}[3][]
{
  \begingroup
    \def\childdocextract #2##1~~~{\def\childdoctmp{\childdocforward[#1]{#3##1}}}
    \expandafter\childdocextract\childdocname~~~
    \expandafter
  \endgroup
  \childdoctmp
}
%    \end{macrocode}

% \macro{\childdoc}
% The deprecated macro |\childdoc| is a legacy version of |\childdocmain|:
%    \begin{macrocode}
\newcommand{\childdoc}{\childdocmain}
%    \end{macrocode}

% \macro{\childdocredirect}
% The deprecated macro |\childdocredirect| is a legacy version
% of |\childdocforward| and |\childdocforwardprefix|:
%    \begin{macrocode}
\newcommand{\childdocredirect}[2][]
{
  \begingroup
    \if?#1?
      \def\childdoctmp{\childdocforward{#2}}
    \else
      \def\childdoctmp{\childdocforwardprefix{#1}{#2}}
    \fi
    \expandafter
  \endgroup
  \childdoctmp
}
%    \end{macrocode}

%\iffalse
%</package>
%\fi
%
\endinput
|\\
|\childdocmain{}|\\
\end{tabular}
\end{center}
at the very top of the main \LaTeX{} file,
in particular \emph{before} the |\documentclass| statement!
The argument of |\childdocmain| should be left empty
(but it must be present).

%%%%%%%%%%%%%%%%%%%%%%%%%%%%%%%%%%%%%%%%
\DescribeMacro{\childdocof}
Furthermore, add the commands
\begin{center}
\begin{tabular}{l}
|% \iffalse
%
% childdoc.dtx Copyright (C) 2017-2018 Niklas Beisert
%
% This work may be distributed and/or modified under the
% conditions of the LaTeX Project Public License, either version 1.3
% of this license or (at your option) any later version.
% The latest version of this license is in
%   http://www.latex-project.org/lppl.txt
% and version 1.3 or later is part of all distributions of LaTeX
% version 2005/12/01 or later.
%
% This work has the LPPL maintenance status `maintained'.
%
% The Current Maintainer of this work is Niklas Beisert.
%
% This work consists of the files childdoc.dtx and childdoc.ins
% and the derived files childdoc.def and cdocsamp.tex with
% cdocsch1.tex, cdocsch2.tex, cdocsdrf.tex, cdocsfn1.tex, cdocsfn2.tex.
%
%<package>\ifdefined\childdocmain\endinput\fi
%<package>\ProvidesFile{childdoc.def}[2018/12/30 v2.0 child document driver]
%<samplemain>\ProvidesFile{cdocsamp.tex}[2018/12/30 v2.0 sample for childdoc]
%<*driver>
%\ProvidesFile{childdoc.drv}[2018/12/30 v2.0 childdoc reference manual file]
\PassOptionsToClass{10pt,a4paper}{article}
\documentclass{ltxdoc}

\usepackage[margin=35mm]{geometry}
\usepackage{hyperref}
\usepackage{hyperxmp}
\usepackage[usenames]{color}

\hypersetup{colorlinks=true}
\hypersetup{pdfstartview=FitH}
\hypersetup{pdfpagemode=UseNone}
\hypersetup{pdfsource={}}
\hypersetup{pdflang={en-UK}}
\hypersetup{pdfcopyright={Copyright 2017-2018 Niklas Beisert.
  This work may be distributed and/or modified under the
  conditions of the LaTeX Project Public License, either version 1.3
  of this license or (at your option) any later version.}}
\hypersetup{pdflicenseurl={http://www.latex-project.org/lppl.txt}}
\hypersetup{pdfcontactaddress={ETH Zurich, ITP, HIT K,
  Wolfgang-Pauli-Strasse 27}}
\hypersetup{pdfcontactpostcode={8093}}
\hypersetup{pdfcontactcity={Zurich}}
\hypersetup{pdfcontactcountry={Switzerland}}
\hypersetup{pdfcontactemail={nbeisert@itp.phys.ethz.ch}}
\hypersetup{pdfcontacturl={http://people.phys.ethz.ch/\xmptilde nbeisert/}}

\newcommand{\secref}[1]{\hyperref[#1]{section \ref*{#1}}}

\parskip1ex
\parindent0pt
\let\olditemize\itemize
\def\itemize{\olditemize\parskip0pt}

\begin{document}

\title{The \textsf{childdoc} Package}
\hypersetup{pdftitle={The childdoc Package}}
\author{Niklas Beisert\\[2ex]
  Institut f\"ur Theoretische Physik\\
  Eidgen\"ossische Technische Hochschule Z\"urich\\
  Wolfgang-Pauli-Strasse 27, 8093 Z\"urich, Switzerland\\[1ex]
  \href{mailto:nbeisert@itp.phys.ethz.ch}
  {\texttt{nbeisert@itp.phys.ethz.ch}}}
\hypersetup{pdfauthor={Niklas Beisert}}
\hypersetup{pdfsubject={Manual for the LaTeX2e Package childdoc}}
\date{30 December 2018, \textsf{v2.0}}
\maketitle

\begin{abstract}\noindent
\textsf{childdoc} is a \LaTeXe{} package
that enables the direct compilation
of document sections included by |\include|
to individual files.
\end{abstract}

\begingroup
\parskip0ex
\tableofcontents
\endgroup

%%%%%%%%%%%%%%%%%%%%%%%%%%%%%%%%%%%%%%%%%%%%%%%%%%%%%%%%%%%%%%%%%%%%%%%%%%%%%%%%
%%%%%%%%%%%%%%%%%%%%%%%%%%%%%%%%%%%%%%%%%%%%%%%%%%%%%%%%%%%%%%%%%%%%%%%%%%%%%%%%
\section{Introduction}

\LaTeX{} provides a mechanism to structure a large document (such as a book)
into a main file and several child files (containing the chapters)
using the |\include| command.
This mechanism is beneficial for documents
which span hundreds of pages in order to
make the source file(s) more manageable.
Moreover, compilation can be restricted to
selected child files by means of the |\includeonly| command.
The latter feature can be used to reduce the compilation time while editing
(this was significantly more useful in the earlier days of \LaTeX{})
or to generate a smaller document which is easier to navigate.
Another application of |\includeonly| is to generate
documents consisting of selected parts of the complete document.

However, there are a few drawbacks of the plain |\include| mechanism:
\begin{itemize}
\item
The child files cannot be compiled on their own,
they can only be compiled via the main file.
A naive editing environment
(such as a text editor with an option
to have the current file processed by \LaTeX)
may require one to switch to the main file before compiling;
attempting to compile the child file produces errors.
\item
The main file must be modified (each time)
to adjust the |\includeonly| command
to the present needs. This easily leaves the main file in a messy state.
\item
The generated document will always carry the filename
of the main document. This is inconvenient if
several child files are to be compiled and
to be kept for distribution.
\end{itemize}

The present package provides a simple interface
to make child files individually compilable by \LaTeX{}.
Compiling a child file then has the same effect as compiling
the main file with an |\includeonly| command
to select the appropriate child.
Moreover the generated document will carry the name of the child
rather than the main file.
This resolves all three above issues.

This feature is meant to make the editing of books,
thesis documents and lecture notes somewhat more convenient.
However, the package can also be used efficiently for
composing a series of documents (such as exercise sheets)
which are typically distributed individually.
It then assists the author in generating the individual documents
(potentially in different versions)
as well as a document containing the collected series.
Another application is in developing style files
or other kinds of included material
where compilation of the style file could redirect
to a sample or test file.

%%%%%%%%%%%%%%%%%%%%%%%%%%%%%%%%%%%%%%%%%%%%%%%%%%%%%%%%%%%%%%%%%%%%%%%%%%%%%%%%
%%%%%%%%%%%%%%%%%%%%%%%%%%%%%%%%%%%%%%%%%%%%%%%%%%%%%%%%%%%%%%%%%%%%%%%%%%%%%%%%
\section{Usage}

First of all, the package \textsf{childdoc} is \emph{not} a standard
\LaTeXe{} |.sty| style file! Therefore it needs to be invoked in
a non-standard way.

%%%%%%%%%%%%%%%%%%%%%%%%%%%%%%%%%%%%%%%%%%%%%%%%%%%%%%%%%%%%%%%%%%%%%%%%%%%%%%%%
\subsection{Included Files}
\label{sec:include}

%%%%%%%%%%%%%%%%%%%%%%%%%%%%%%%%%%%%%%%%
\DescribeMacro{\childdocmain}
To use the package, add the commands
\begin{center}
\begin{tabular}{l}
|\input{childdoc.def}|\\
|\childdocmain{}|\\
\end{tabular}
\end{center}
at the very top of the main \LaTeX{} file,
in particular \emph{before} the |\documentclass| statement!
The argument of |\childdocmain| should be left empty
(but it must be present).

%%%%%%%%%%%%%%%%%%%%%%%%%%%%%%%%%%%%%%%%
\DescribeMacro{\childdocof}
Furthermore, add the commands
\begin{center}
\begin{tabular}{l}
|\input{childdoc.def}|\\
|\childdocof{|\textit{main}|}|\\
\end{tabular}
\end{center}
at the top of every child file \textit{child}
which is included by |\include{|\textit{child}|}|
from within the main file
(or at least for those files to be compiled individually).
The argument \textit{main} must be the filename of the main file.

There are a couple of
considerations in setting up the main and child documents:

%%%%%%%%%%%%%%%%%%%%%%%%%%%%%%%%%%%%%%%%
\paragraph{Restrictions.}

Please note the following restrictions:
\begin{itemize}
\item
|\childdocmain| must be called with one argument \textit{main}
to ensure compatibility with earlier version of the package.
It must either be empty (|\childdocmain{}|)
or precisely match the filename of the main file in which it is specified.
See \secref{sec:detection} for further information.
\item
The filename \textit{main} must be specified without the |.tex| extension.
\item
The filename \textit{main} is case sensitive
(even in case-insensitive file systems)
due to internal string comparison.
\item
The argument \textit{main} should be fully expanded, it cannot be a macro.
\item
Subdirectories and special characters should be avoided in filenames.
\item
The command |\childdocmain{|\textit{main}|}| must be followed by a whitespace.
It should not be followed immediately by another command
or by a comment mark `|%|'.
This is because the \TeX{} parser reads the token immediately following
the argument of |\childdocmain| and puts it
at the beginning of every child section;
however, a white\-space is ignored.
\end{itemize}

%%%%%%%%%%%%%%%%%%%%%%%%%%%%%%%%%%%%%%%%
\paragraph{Content of Main File.}

It is advisable to place all content in the child files included by |\include|.
Any output contained in the main file will appear in all child documents
unless suppressed manually;
it cannot be suppressed automatically by the |\includeonly| directive
and thus should normally be avoided.
A method to include some content in the main file
by means of conditional processing is described in \secref{sec:conditional}.

%%%%%%%%%%%%%%%%%%%%%%%%%%%%%%%%%%%%%%%%
\paragraph{Page Numbering.}

When only a part of the document is compiled,
the appropriate numbering of pages
(as well as other status parameters)
is determined from the |.aux| files.
The latter contain information from previous passes.
However this information needs to propagate through
all intermediate child documents.
Therefore the page numbering in child documents may well
be inconsistent until the complete document is compiled at least once.

A useful (if unconventional) way to always ensure a consistent
page numbering is to restart the numbering in each child document
and denote the pages by `\textit{child}|.|\textit{page}'
where \textit{child} represents the chapter/section number of the child file.
This can be achieved by the command
|\numberwithin{page}{|\textit{child}|}|
of the \textsf{amsmath} package
where \textit{child} can be |chapter| or |section|
depending on the chosen structuring.
Alternatively, one can modify the macro |\thepage| appropriately
and reset the counter |page| at the start of each child file.

%%%%%%%%%%%%%%%%%%%%%%%%%%%%%%%%%%%%%%%%%%%%%%%%%%%%%%%%%%%%%%%%%%%%%%%%%%%%%%%%
\subsection{Conditional Processing}
\label{sec:conditional}

The package provides a mechanism to compile different versions
of a document. To customise the versions further some conditional processing
can come in handy to distinguish which version is being compiled.
The package provides two macros to describe the compilation context:

%%%%%%%%%%%%%%%%%%%%%%%%%%%%%%%%%%%%%%%%
\DescribeMacro{\ifchilddoc}
The conditional |\ifchilddoc| distinguishes between the compilation of
child documents and the main document:
%
\begin{center}
|\ifchilddoc |\textit{child-code}| |[|\||else |\textit{main-code}]| \||fi|
\end{center}

%%%%%%%%%%%%%%%%%%%%%%%%%%%%%%%%%%%%%%%%
\DescribeMacro{\childdocname}
\DescribeMacro{\childdocjob}
The macro |\childdocname| contains the filename (without extension)
of the main or child file being processed.
Note that |\childdocjob| will always contain the name of the main file.

%%%%%%%%%%%%%%%%%%%%%%%%%%%%%%%%%%%%%%%%
\paragraph{Title Page.}

Conditional processing can be used to include a title or banner page
in the main document when proper precautions are taken.
Importantly, the code in the main file should ensure that the page counter
(as well as other status parameters which are stored in the |.aux| files)
takes the same value after the conditional processing.
Otherwise the page numbers may take divergent values
depending on which part is compiled.

For example, a title page could be declared by:
%
\begin{center}
\begin{tabular}{l}
|\ifchilddoc\||else|\\
|\addtocounter{page}{-1}|\\
\textit{code for title page}\\
|\newpage|\\
|\||fi|
\end{tabular}
\end{center}
%
A banner page for the child documents can be generated by:
%
\begin{center}
\begin{tabular}{l}
|\ifchilddoc|\\
|\addtocounter{page}{-1}|\\
\textit{code for banner page}\\
|\newpage|\\
|\||fi|
\end{tabular}
\end{center}
%
Here one could write a message such as:
\begin{center}
|This is the part \childdocname{} of \childdocjob{}.|
\end{center}

%%%%%%%%%%%%%%%%%%%%%%%%%%%%%%%%%%%%%%%%%%%%%%%%%%%%%%%%%%%%%%%%%%%%%%%%%%%%%%%%
\subsection{Flags}
\label{sec:flags}

The package makes it easy to generate different versions
of the main or child documents.
To this end compilation flags can be defined
and assigned different default values.
They will be particularly useful in conjunction
with the forwarding mechanism described in \secref{sec:forward}.

For example, it may be useful to have a flag |\version|
which can be set to |draft| or |final|.
The document source will contain some conditional code
depending on the value of |\version|.
Suppose further, the flag should default to |final| for the main file
and to |draft| for child files
which is a natural assignment for editing the document.
This is achieved by placing the following code
in the preamble of the main document
(below the |\childdocmain| directive):
%
\begin{center}
\begin{tabular}{l}
|\ifchilddoc|\\
|\providecommand{\version}{draft}|\\
|\||else|\\
|\providecommand{\version}{final}|\\
|\||fi|
\end{tabular}
\end{center}
%
The definition by |\providecommand| makes sure
that previous definitions are not overwritten.
Further statements |\providecommand{\version}{...}|
can thus be added before the above code to override it.

For the main file, one might add a line
(between |\childdocmain| and the above block)
%
\begin{center}
|%\ifchilddoc\||else\providecommand{\version}{draft}\||fi|
\end{center}
%
which can be uncommented to produce a draft version.
Likewise one can add a line to the very top of a child file
(above the |\childdocof{|\textit{main}|}| directive)
%
\begin{center}
|%\providecommand{\version}{final}|
\end{center}
%
which can be uncommented to produce the final version of this child document.

%%%%%%%%%%%%%%%%%%%%%%%%%%%%%%%%%%%%%%%%%%%%%%%%%%%%%%%%%%%%%%%%%%%%%%%%%%%%%%%%
\subsection{Forwarding}
\label{sec:forward}

Different versions of the main or child documents
using compilation flags as described in \secref{sec:flags}
can be (permanently) stored in different files
for convenient compilation, viewing and distribution.
To this end, the package defines a command
to pass on compilation to a different file:

%%%%%%%%%%%%%%%%%%%%%%%%%%%%%%%%%%%%%%%%
\DescribeMacro{\childdocforward}
The command |\childdocforward| redirects processing to
another source file:
%
\begin{center}
\begin{tabular}{l}
|\input{childdoc.def}|\\
|\childdocforward[|\textit{main}|]{|\textit{dest}|}|\\
\end{tabular}
\end{center}
%
The argument \textit{dest} is the destination file
(without extension).
It should be the main file or one of the child files.
Note that further \textsf{childdoc} directives
such as |\childdocof| and |\childdocforward|
in the indicated file will be processed in this form.
The optional argument \textit{main}
passes on directly to the main file \textit{main}
while pretending to compile the child \textit{dest}.
This form behaves as if \textit{dest}
issues |\childdocof{|\textit{main}|}| right away,
and no further \textsf{childdoc} directives will be processed.

%%%%%%%%%%%%%%%%%%%%%%%%%%%%%%%%%%%%%%%%
\DescribeMacro{\...prefix}
In the alternative form |\childdocforwardprefix|,
%
\begin{center}
\begin{tabular}{l}
|\input{childdoc.def}|\\
|\childdocforwardprefix[|\textit{main}|]{|\textit{prefix}|}{|\textit{dest}|}|
\end{tabular}
\end{center}
%
the destination file is determined by a pattern
depending on the current file:
To make this work, the current file must be called
`{\textit{prefix}\hspace{0.2em}\textit{suffix}}'
with \textit{prefix} matching precisely the argument.
Processing is then passed on to the file
`{\textit{dest}\hspace{0.2em}\textit{suffix}}'.
Surely, the same effect is achieved by
directly specifying the
argument `{\textit{dest}\hspace{0.2em}\textit{suffix}}'
in the first form.
However, that requires to set up a different file
for each child. With the alternative form of the command
all these files can have exactly the same content
which simplifies setting them up and maintaining them.

For example, the following file |draft.tex|
with a compilation flag |\version| as described in \secref{sec:flags}
compiles the main document as a draft:
%
\begin{center}
\begin{tabular}{l}
|\def\version{draft}|\\
|\input{childdoc.def}|\\
|\childdocforward{|\textit{main}|}|
\end{tabular}
\end{center}
%
Likewise, the following files |final|\textit{nn}|.tex|
compile the final version of the child document
|child|\textit{nn}|.tex|:
%
\begin{center}
\begin{tabular}{l}
|\def\version{final}|\\
|\input{childdoc.def}|\\
|\childdocforwardprefix{final}{child}|
\end{tabular}
\end{center}
%

Note that when several versions of a main file and/or of each child file
are to be generated, it may be convenient to set up a |Makefile| or
shell script to automatise the process.

%%%%%%%%%%%%%%%%%%%%%%%%%%%%%%%%%%%%%%%%%%%%%%%%%%%%%%%%%%%%%%%%%%%%%%%%%%%%%%%%
\subsection{Command Line Processing}
\label{sec:commandline}

The effect of redirection files can also be achieved by invoking
the \LaTeX{} compiler with a more elaborate command line.
Most conveniently this should be done as part
of a shell script or a |Makefile|.

When using \textsf{childdoc} in the main file, the following
command lines effectively perform a redirection
(note that depending on the shell being used,
backslashes may have to be doubled: `|\|' $\to$ `|\\|'):
%
\begin{center}
|... -jobname "|\textit{target}|" |\\|"|[\textit{flags}]%
|\input{childdoc.def}\childdocforward[|\textit{main}|]{|\textit{dest}|}"|
\end{center}
%
Here \textit{target} is the name of the output file,
\textit{main} is the name of the main file
and \textit{dest} is the name of the main or child file to be processed
(all filenames without extensions).
The optional argument \textit{main} can be omitted
if \textit{main} matches \textit{dest}.
Optionally, compilation \textit{flags} can be defined via |\def| commands.
This command line makes the \TeX{} engine believe
it is compiling the file \textit{target}
whose content is specified as the latter parameter.
The provided code then forwards the processing to
\textit{main} or \textit{dest} as described in \secref{sec:forward}.

%%%%%%%%%%%%%%%%%%%%%%%%%%%%%%%%%%%%%%%%%%%%%%%%%%%%%%%%%%%%%%%%%%%%%%%%%%%%%%%%
\subsection{Include by Input}
\label{sec:input}

Including child documents by |\include| has some restrictions by design.
Most notably, the content of a child document always occupies
its own set of pages; pages cannot be shared between child documents.
Usually, this behaviour makes perfect sense
because each child document contain an essential part of the document.
However, in some situations it may be desirable to compose
a document from a collection of parts
without having mandatory page breaks between then.
For this case, the package
provides a mechanism to include parts
by |\input| which can also be processed individually.
However, by construction this mechanism
requires manual handling of the content to be output.

%%%%%%%%%%%%%%%%%%%%%%%%%%%%%%%%%%%%%%%%
\DescribeMacro{\ifchilddocmanual}
The main file should be prepared as usual, see \secref{sec:include}.
However, the document body must make a distinction
between processing of an individual part and of the main document, e.g.:
%
\begin{center}
\begin{tabular}{l}
|\ifchilddocmanual|\\
|\input{\childdocname}|\\
|\||else|\\
\textit{document body with }|\input{|\textit{part}|}|\\
|\||fi|
\end{tabular}
\end{center}
%
The conditional |\ifchilddocmanual| is true whenever
a part to be included by |\input| is being compiled,
and the name of the part is stored in |\childdocname|.

%%%%%%%%%%%%%%%%%%%%%%%%%%%%%%%%%%%%%%%%
\DescribeMacro{\childdocby}
Each part to be included by |\input| should start with:
%
\begin{center}
\begin{tabular}{l}
|\input{childdoc.def}|\\
|\childdocby{|\textit{main}|}|\\
\end{tabular}
\end{center}
%
The directive |\childdocby| is similar to |\childdocof|
described in \secref{sec:include},
but the subsequent selection of content must be done manually.
To that end, both |\ifchilddoc| and |\ifchilddocmanual|
will be true upon processing of a part,
and the name of the part is stored in |\childdocname|.
Note that |\jobname| will be set to the filename of the current part
so that each part receives an individual |.aux| file
that does not interfere with the |.aux| file(s) of the main document.
This behaviour can be altered by the alternative form
|\childdocby[*]{|\textit{main}|}| (with a non-empty optional argument)
which uses the |.aux| file of the main document
by setting |\jobname| to \textit{main}.

%%%%%%%%%%%%%%%%%%%%%%%%%%%%%%%%%%%%%%%%%%%%%%%%%%%%%%%%%%%%%%%%%%%%%%%%%%%%%%%%
\subsection{Driver Development}
\label{sec:driver}

The \textsf{childdoc} mechanism can also be use for the development
of definition files such as \LaTeX{} styles or classes.
This case differs from the above setup with multiple parts
included by |\include| in that no |\includeonly| should be invoked.
This can be achieved by starting the include file
(before |\ProvidesPackage|) with:
%
\begin{center}
\begin{tabular}{l}
|\input{childdoc.def}|\\
|\childdocforward{|\textit{main}|}|\\
\end{tabular}
\end{center}
%
or alternatively with:
%
\begin{center}
\begin{tabular}{l}
|\input{childdoc.def}|\\
|\childdocby{|\textit{main}|}|\\
\end{tabular}
\end{center}
%
Both forms have slightly different effects as described above.
The main file is prepared as usual, see \secref{sec:include}.

%%%%%%%%%%%%%%%%%%%%%%%%%%%%%%%%%%%%%%%%%%%%%%%%%%%%%%%%%%%%%%%%%%%%%%%%%%%%%%%%
\subsection{Legacy Detection}
\label{sec:detection}

The directive |\childdocmain| in the main file can detect
whether the complete document or merely a child is to be compiled
even without using the directive |\childdocof|.
This method is deprecated because it is less robust
and there is no compelling reason to use it;
it is merely provided for backward compatibility
and it may be removed in future versions.

If the detection mechanism is to be used,
it is mandatory to correctly specify
the filename of the main file as the argument of |\childdocmain|:
%
\begin{center}
\begin{tabular}{l}
|\input{childdoc.def}|\\
|\childdocmain{|\textit{main}|}|\\
\end{tabular}
\end{center}
%
If |\jobname| does not match the argument \textit{main} of |\childdocmain|,
it is assumed that |\jobname| points to the child file to be compiled.
When using |\childdocmain| with the main file specified as argument,
it suffices to start a child file
with just |\input{|\textit{main}|}|
without loading of the package and using |\childdocof|.
If instead all processing is done
with the appropriate \textsf{childdoc} directives,
the argument of \textit{main} of |\childdocmain| can be empty.

An alternative version of the command line processing described
in \secref{sec:commandline} using the detection mechanism reads:
%
\begin{center}
|... -jobname "|\textit{target}|" "|[\textit{flags}]%
[|\def\jobname{|\textit{dest}|}|]|\input{|\textit{main}|}"|
\end{center}

%%%%%%%%%%%%%%%%%%%%%%%%%%%%%%%%%%%%%%%%%%%%%%%%%%%%%%%%%%%%%%%%%%%%%%%%%%%%%%%%
\subsection{Manual Code}
\label{sec:manual}

In case one cannot be certain whether the definitions file |childdoc.def|
is installed on the target \TeX{} distribution
and one prefers not to ship it,
it is conceivable to paste a few relevant commands into the sources.

To that end, drop all statements |\input{childdoc.def}|
and perform the replacements as outlined below.
Instead of |\childdocmain{|\textit{main}|}| add the following code
to the top of the main file:
%
\begin{center}
\begin{tabular}{l}
|\||ifdefined\childdocname\endinput\||fi\newif\ifchilddoc|\\
|\edef\childdocname{\scantokens\expandafter{\jobname\noexpand}}|\\
|\def\childdocmain{|\textit{main}|}\||ifx\childdocmain\childdocname\||else|\\
|\childdoctrue\includeonly{\childdocname}\let\jobname\childdocmain\||fi|\\
\end{tabular}
\end{center}
%
Instead of |\childdocof{|\textit{main}|}| just include the main file
at the top of each child file:
%
\begin{center}
|\input{|\textit{main}|}|
\end{center}
%
A simple redirection |\childdocforward{|\textit{dest}|}| is achieved by:
%
\begin{center}
|\def\jobname{|\textit{dest}|}\input{\jobname}|
\end{center}
%
The redirection with prefix
|\childdocforwardprefix[|\textit{prefix}|]{|\textit{dest}|}|
is accomplished by:
%
\begin{center}
\begin{tabular}{l}
|{\edef\jobname{\scantokens\expandafter{\jobname\noexpand}}|\\
|\def\redirectjob |\textit{prefix}|#1~~~{\gdef\jobname{|\textit{dest}|#1}}|\\
|\expandafter\redirectjob\jobname~~~}\input{\jobname}|
\end{tabular}
\end{center}

In an alternative approach,
child documents can be compiled by a specific command line
without additional code or specific definitions:
%
\begin{center}
|... -jobname "|\textit{target}|" "|[\textit{flags}]%
|\includeonly{|\textit{dest}|}\input{|\textit{main}|}"|
\end{center}
%

%%%%%%%%%%%%%%%%%%%%%%%%%%%%%%%%%%%%%%%%%%%%%%%%%%%%%%%%%%%%%%%%%%%%%%%%%%%%%%%%
%%%%%%%%%%%%%%%%%%%%%%%%%%%%%%%%%%%%%%%%%%%%%%%%%%%%%%%%%%%%%%%%%%%%%%%%%%%%%%%%
\section{Information}

%%%%%%%%%%%%%%%%%%%%%%%%%%%%%%%%%%%%%%%%%%%%%%%%%%%%%%%%%%%%%%%%%%%%%%%%%%%%%%%%
\subsection{Copyright}

Copyright \copyright{} 2017--2018 Niklas Beisert

This work may be distributed and/or modified under the
conditions of the \LaTeX{} Project Public License, either version 1.3
of this license or (at your option) any later version.
The latest version of this license is in
  \url{http://www.latex-project.org/lppl.txt}
and version 1.3 or later is part of all distributions of \LaTeX{}
version 2005/12/01 or later.

This work has the LPPL maintenance status `maintained'.

The Current Maintainer of this work is Niklas Beisert.

This work consists of the files |README.txt|, |childdoc.ins| and |childdoc.dtx|
as well as the derived files |childdoc.def|, |cdocsamp.tex|
with |cdocsch1.tex|, |cdocsch2.tex|, |cdocspt3.tex|, |cdocspt4.tex|,
|cdocsdrf.tex|, |cdocsfn1.tex|, |cdocsfn2.tex|
as well as |childdoc.pdf|.

%%%%%%%%%%%%%%%%%%%%%%%%%%%%%%%%%%%%%%%%%%%%%%%%%%%%%%%%%%%%%%%%%%%%%%%%%%%%%%%%
\subsection{Files and Installation}

The package consists of the files:
%
\begin{center}
\begin{tabular}{ll}
    |README.txt|   & readme file \\
    |childdoc.ins| & installation file \\
    |childdoc.dtx| & source file \\
    |childdoc.def| & definition file \\
    |cdocsamp.tex| & sample main file \\
    |cdocsch1.tex| & sample include file \\
    |cdocsch2.tex| & sample include file \\
    |cdocspt3.tex| & sample part file \\
    |cdocspt4.tex| & sample part file \\
    |cdocsdrf.tex| & sample redirection file \\
    |cdocsfn1.tex| & sample redirection file \\
    |cdocsfn2.tex| & sample redirection file \\
    |childdoc.pdf| & manual
\end{tabular}
\end{center}
%
The distribution consists of the files
|README.txt|, |childdoc.ins| and |childdoc.dtx|.
%
\begin{itemize}
\item
Run (pdf)\LaTeX{} on |childdoc.dtx|
to compile the manual |childdoc.pdf| (this file).
\item
Run \LaTeX{} on |childdoc.ins| to create the definitions file |childdoc.def|
and the sample |cdocsamp.tex| with include files
|cdocsch1.tex|, |cdocsch2.tex|, |cdocspt3.tex|, |cdocspt4.tex|,
|cdocsdrf.tex|, |cdocsfn1.tex|, |cdocsfn2.tex|.
Then copy the file |childdoc.def| to an appropriate directory of your \LaTeX{}
distribution, e.g.\ \textit{texmf-root}|/tex/latex/childdoc|.
\end{itemize}

%%%%%%%%%%%%%%%%%%%%%%%%%%%%%%%%%%%%%%%%%%%%%%%%%%%%%%%%%%%%%%%%%%%%%%%%%%%%%%%%
\subsection{Related CTAN Packages}

There are several other packages which offer a similar functionality:
%
\begin{itemize}
\item
The packages
\href{http://ctan.org/pkg/docmute}{\textsf{docmute}},
\href{http://ctan.org/pkg/includex}{\textsf{includex}} and
\href{http://ctan.org/pkg/standalone}{\textsf{standalone}}
provide commands to include only the document body of
a child file thus allowing both files to be compiled individually.
\item
The packages \href{http://ctan.org/pkg/subdocs}{\textsf{subdocs}}
and \href{http://ctan.org/pkg/subfiles}{\textsf{subfiles}}
provide structures in which the main and child documents can be
encapsulated and allowing them to be compiled individually.
The inclusion mechanism is different from the conventional |\include|.
\item
The package \href{http://ctan.org/pkg/combine}{\textsf{combine}}
is an elaborate solution to combine several documents into one.
\end{itemize}
%
See also the CTAN topic \href{http://ctan.org/topic/subdocs}{\textsf{subdocs}}
for further related packages.
The present package differs from the above solutions in that
a document structure constructed with the conventional |\include| mechanism
just needs two extra commands at the top of every file
such that all constituent files can be compiled individually.

%%%%%%%%%%%%%%%%%%%%%%%%%%%%%%%%%%%%%%%%%%%%%%%%%%%%%%%%%%%%%%%%%%%%%%%%%%%%%%%%
%\subsection{Feature Suggestions}
%
%The following is a list of features which may be useful for future
%versions of this package:
%%
%\begin{itemize}
%\item
%\ldots
%\end{itemize}

%%%%%%%%%%%%%%%%%%%%%%%%%%%%%%%%%%%%%%%%%%%%%%%%%%%%%%%%%%%%%%%%%%%%%%%%%%%%%%%%
\subsection{Revision History}

%%%%%%%%%%%%%%%%%%%%%%%%%%%%%%%%%%%%%%%%
\paragraph{v2.0:} 2018/12/30

\begin{itemize}
\item
immediate forward processing
\item
added |\childdocby| mechanism
\item
manual restructured
\end{itemize}

%%%%%%%%%%%%%%%%%%%%%%%%%%%%%%%%%%%%%%%%
\paragraph{v1.6:} 2018/01/17

\begin{itemize}
\item
application for development of include files
\item
corrections to manual
\end{itemize}

%%%%%%%%%%%%%%%%%%%%%%%%%%%%%%%%%%%%%%%%
\paragraph{v1.5:} 2017/05/21

\begin{itemize}
\item
more complete structuring introduced
\item
|\childdocof| introduced
\item
|\childdoc| renamed to |\childdocmain|
\item
|\childredirect| renamed to |\childdocforward| and |\childdocforwardprefix|
and functionality expanded
\end{itemize}

%%%%%%%%%%%%%%%%%%%%%%%%%%%%%%%%%%%%%%%%
\paragraph{v1.0:} 2017/04/27

\begin{itemize}
\item
manual and install package
\item
first version published on CTAN
\end{itemize}

%%%%%%%%%%%%%%%%%%%%%%%%%%%%%%%%%%%%%%%%
\paragraph{v0.6:} 2017/04/26

\begin{itemize}
\item
redirection mechanism added
\end{itemize}

%%%%%%%%%%%%%%%%%%%%%%%%%%%%%%%%%%%%%%%%
\paragraph{v0.5:} 2017/04/26

\begin{itemize}
\item
functionality in definition file
\end{itemize}


%%%%%%%%%%%%%%%%%%%%%%%%%%%%%%%%%%%%%%%%%%%%%%%%%%%%%%%%%%%%%%%%%%%%%%%%%%%%%%%%
%%%%%%%%%%%%%%%%%%%%%%%%%%%%%%%%%%%%%%%%%%%%%%%%%%%%%%%%%%%%%%%%%%%%%%%%%%%%%%%%
%%%%%%%%%%%%%%%%%%%%%%%%%%%%%%%%%%%%%%%%%%%%%%%%%%%%%%%%%%%%%%%%%%%%%%%%%%%%%%%%
\appendix

\settowidth\MacroIndent{\rmfamily\scriptsize 000\ }

 \DocInput{childdoc.dtx}

\end{document}
%</driver>
% \fi
%
% %%%%%%%%%%%%%%%%%%%%%%%%%%%%%%%%%%%%%%%%%%%%%%%%%%%%%%%%%%%%%%%%%%%%%%%%%%%%%%
% %%%%%%%%%%%%%%%%%%%%%%%%%%%%%%%%%%%%%%%%%%%%%%%%%%%%%%%%%%%%%%%%%%%%%%%%%%%%%%
% \section{Sample}
%\iffalse
%<*samplemain>
%\fi
%
% The following presents a sample document
% with two chapters, two parts, a title page,
% a compile flag as well as three forwarding files to set the flag.
% It consists of eight |.tex| files:
% \begin{center}
% \begin{tabular}{ll}
% |cdocsamp.tex|&main file\\
% |cdocsch1.tex|&include file for chapter 1\\
% |cdocsch2.tex|&include file for chapter 2\\
% |cdocspt3.tex|&include file for part 3\\
% |cdocspt4.tex|&include file for part 4\\
% |cdocsdrf.tex|&forwarding file for main file in draft mode\\
% |cdocsfi1.tex|&forwarding file for final version of chapter 1\\
% |cdocsfi2.tex|&forwarding file for final version of chapter 2\\
% \end{tabular}
% \end{center}
% Each of the eight files can be compiled directly by the \LaTeX{} compiler.
%
% %%%%%%%%%%%%%%%%%%%%%%%%%%%%%%%%%%%%%%
% \paragraph{Main File.}
%
% The main file is called |cdocsamp.tex|.
%
% Load the \textsf{childdoc} definitions and
% declare the filename for the main document:
%    \begin{macrocode}
\input{childdoc.def}
\childdocmain{}
%    \end{macrocode}

% Optional override for |\version| flag:
%    \begin{macrocode}
%%\ifchilddoc\else\providecommand{\version}{draft}\fi
%    \end{macrocode}

% Define the default values for the |\version| flag
% (|final| for the main file and |draft| for childs):
%    \begin{macrocode}
\ifchilddoc
\providecommand{\version}{draft}
\else
\providecommand{\version}{final}
\fi
%    \end{macrocode}

% Load the standard document class:
%    \begin{macrocode}
\documentclass[12pt]{article}
%    \end{macrocode}

% Start the document body:
%    \begin{macrocode}
\begin{document}
%    \end{macrocode}

% Declare a title page.
% Print title, part of document being processed and version flag:
%    \begin{macrocode}
\addtocounter{page}{-1}
\begin{center}
{\LARGE\bfseries{}childdoc example\par}
\vspace{1cm}
\ifchilddoc
\ifchilddocmanual part\else chapter\fi:
`\childdocname' of `\childdocjob'\par
\else
main document: `\childdocjob'\par
\fi
version: \version\par
\end{center}
\newpage
%    \end{macrocode}

% Manually include selected file,
% otherwise process as usual:
%    \begin{macrocode}
\ifchilddocmanual
\section*{part `\childdocname'}
\input{\childdocname}
\else
%    \end{macrocode}

% Include the two chapters:
%    \begin{macrocode}
\include{cdocsch1}
\include{cdocsch2}
%    \end{macrocode}

% Include the two parts unless only chapters should be displayed:
%    \begin{macrocode}
\ifchilddoc\else
\section{part three}
\input{cdocspt3}
\section{part four}
\input{cdocspt4}
\fi
%    \end{macrocode}

% Process as usual until here:
%    \begin{macrocode}
\fi
%    \end{macrocode}

% End of document body:
%    \begin{macrocode}
\end{document}
%    \end{macrocode}
%\iffalse
%</samplemain>
%\fi
%
% %%%%%%%%%%%%%%%%%%%%%%%%%%%%%%%%%%%%%%
% \paragraph{Chapter Include Files.}
%
% The include files are called |cdocsch1.tex| and |cdocsch2.tex|.
%
%\iffalse
%<*samplechap1|samplechap2>
%\fi

% Optional override for |\version| flag:
%    \begin{macrocode}
%%\providecommand{\version}{final}
%    \end{macrocode}

% Include the main document:
%    \begin{macrocode}
\input{childdoc.def}
\childdocof{cdocsamp}
%    \end{macrocode}

%\iffalse
%</samplechap1|samplechap2>
%\fi
%
%\iffalse
%<*samplechap1>
%\fi
% Some text for chapter 1:
%    \begin{macrocode}
\section{one}
some text in chapter one
%    \end{macrocode}

%\iffalse
%</samplechap1>
%\fi
% Some text for chapter 2:
%\iffalse
%<*samplechap2>
%\fi
%    \begin{macrocode}
\section{two}
more text in chapter two
%    \end{macrocode}

%\iffalse
%</samplechap2>
%\fi
%
% %%%%%%%%%%%%%%%%%%%%%%%%%%%%%%%%%%%%%%
% \paragraph{Part Include Files.}
%
% The include files are called |cdocspt3.tex| and |cdocspt4.tex|.
%
%\iffalse
%<*samplepart3|samplepart4>
%\fi

% Optional override for |\version| flag:
%    \begin{macrocode}
%%\providecommand{\version}{final}
%    \end{macrocode}

% Include the main document:
%    \begin{macrocode}
\input{childdoc.def}
\childdocby{cdocsamp}
%    \end{macrocode}

%\iffalse
%</samplepart3|samplepart4>
%\fi
%
%\iffalse
%<*samplepart3>
%\fi
% Some text for part 3:
%    \begin{macrocode}
some text in part three
%    \end{macrocode}

%\iffalse
%</samplepart3>
%\fi
% Some text for part 4:
%\iffalse
%<*samplepart4>
%\fi
%    \begin{macrocode}
more text in part four
%    \end{macrocode}

%\iffalse
%</samplepart4>
%\fi
%
% %%%%%%%%%%%%%%%%%%%%%%%%%%%%%%%%%%%%%%
% \paragraph{Forwarding for a Complete Draft.}
%
% The following forwarding file |cdocsdrf.tex|
% compiles the main document in draft mode:
%\iffalse
%<*sampledraft>
%\fi
%    \begin{macrocode}
\def\version{draft}
\input{childdoc.def}
\childdocforward{cdocsamp}
%    \end{macrocode}

%\iffalse
%</sampledraft>
%\fi
%
% %%%%%%%%%%%%%%%%%%%%%%%%%%%%%%%%%%%%%%
% \paragraph{Forwarding for Final Version of the Chapters.}
%
% The following forwarding files |cdocsfn1.tex| and |cdocsfn2.tex|
% (with identical content)
% compile the final versions of the child documents
% |cdocsch1.tex| and |cdocsch2.tex|, respectively:
%\iffalse
%<*samplefinal>
%\fi
%    \begin{macrocode}
\def\version{final}
\input{childdoc.def}
\childdocforwardprefix[cdocsamp]{cdocsfn}{cdocsch}
%    \end{macrocode}

%\iffalse
%</samplefinal>
%\fi
%
% %%%%%%%%%%%%%%%%%%%%%%%%%%%%%%%%%%%%%%
% \paragraph{Command Line Processing.}
%
% The following three command lines generate the output files
% |cdocscld|, |cdocscl1| and |cdocscl2|
% which should be identical to
% |cdocsdrf|, |cdocsch1| and |cdocsfn2|, respectively:
% \begin{center}
% \begin{tabular}{l}
% |latex -jobname cdocscld \|\\
% |  "\def\version{draft}\input{childdoc.def}\childdocforward{cdocsamp}"|\\
% |latex -jobname cdocscl1 \|\\
% |  "\input{childdoc.def}\childdocforward[cdocsamp]{cdocsch1}"|\\
% |latex -jobname cdocscl2 \|\\
% |  "\def\version{final}\input{childdoc.def}\childdocforward{cdocsch2}"|
% \end{tabular}
% \end{center}
% Note that the trailing backslash on each first line
% merely continues the input to the second line
% (for convenient cut ant paste).
% Furthermore, the command |latex| can be replaced by any
% of its alternative versions such as |pdflatex|.
%
% %%%%%%%%%%%%%%%%%%%%%%%%%%%%%%%%%%%%%%%%%%%%%%%%%%%%%%%%%%%%%%%%%%%%%%%%%%%%%%
% %%%%%%%%%%%%%%%%%%%%%%%%%%%%%%%%%%%%%%%%%%%%%%%%%%%%%%%%%%%%%%%%%%%%%%%%%%%%%%
% \section{Implementation}
%\iffalse
%<*package>
%\fi
%
% This section describes the definitions file |childdoc.def|.

% The definitions cannot be loaded using |\usepackage| or |\RequirePackage|
% which has a mechanism to prevent loading a style file more than once.
% When loading the definitions by means of |\input|
% multiple instances have to be prevented manually:
%\iffalse
%This code needs to be before the `\ProvidesFile' directive
%which is defined at the beginning of this file.
%Therefore it is also placed there and commented out here.
%</package>
%<*discard>
%\fi
%    \begin{macrocode}
\ifdefined\childdocmain\endinput\fi
%    \end{macrocode}
%\iffalse
%</discard>
%<*package>
%\fi
%
% \macro{\ifchilddoc}
% \macro{\ifchilddocmanual}
% The conditional |\ifchilddoc| tells whether a
% child (true) or main (false) document is being compiled.
% The conditional |\ifchilddocmanual| tells whether
% the |\includeonly| mechanism is used (false) or
% the selection of child files must be performed manually (true).
% The definitions initialise to false:
%    \begin{macrocode}
\newif\ifchilddoc
\newif\ifchilddocmanual
%    \end{macrocode}

% \macro{\childdocname}
% \macro{\childdocjob}
% The macro |\childdocname| stores the name of the main document
% to be compiled. The macro |\childdocjob| stores the name of
% the document on which the \LaTeX{} compiler was originally invoked.
% The content of |\jobname| cannot be compared
% to filenames specified in the source due to different catcodes.
% The following code rescans |\jobname|, stores the result
% in |\childdocname| and saves a copy in |\childdocjob|:
%    \begin{macrocode}
\edef\childdocname{\scantokens\expandafter{\jobname\noexpand}}
\let\childdocjob\childdocname
%    \end{macrocode}

% \macro{\childdocdisable}
% The macro |\childdocdisable| prevents the main file
% from being processed more than once.
% At this stage, the main document command |\childdocmain|
% is assumed to be called once again where it should do nothing.
% Any subsequent call to it should prevent
% a secondary processing of the main document
% It overwrites the forwarding commands
% |\childdocof| and |\childdocforward|
% with empty macros to prevent further inclusions of the main document:
%    \begin{macrocode}
\newcommand{\childdocdisable}
{
  \renewcommand{\childdocmain}[1]{\renewcommand{\childdocmain}[1]{\endinput}}
  \renewcommand{\childdocof}[1]{}
  \renewcommand{\childdocby}[2][]{}
  \renewcommand{\childdocforward}[2][]{}
  \renewcommand{\childdocdisable}{}
}
%    \end{macrocode}

% \macro{\childdocmain}
% The macro |\childdocmain| is to be called at the top of the main file
% with nothing or the main filename (without extension) as argument.
% First, it breaks loops.
% If the argument is not empty and does not match |\childdocname|
% (which is set by the first inclusion of |childdoc.def|),
% |\ifchilddoc| is set to true, |\includeonly| is applied to the child file
% and |\jobname| is set to the main file
% (for proper handling of |.aux| files):
%    \begin{macrocode}
\newcommand{\childdocmain}[1]
{
  \childdocdisable\childdocmain{}
  \if?#1?\else
    \begingroup
      \def\childdoctmp{#1}
      \ifx\childdoctmp\childdocname
        \def\childdoctmp{}
      \else
        \def\childdoctmp
        {
          \childdoctrue
          \includeonly{\childdocname}
          \def\childdocjob{#1}
          \def\jobname{#1}
        }
      \fi
      \expandafter
    \endgroup
    \childdoctmp
  \fi
}
%    \end{macrocode}

% \macro{\childdocof}
% The command |\childdocof| redirects
% compilation to the main file |#1|.
%    \begin{macrocode}
\newcommand{\childdocof}[1]
{
  \childdocdisable
  \childdoctrue
  \includeonly{\childdocname}
  \def\jobname{#1}
  \def\childdocjob{#1}
  \input{#1}
}
%    \end{macrocode}

% \macro{\childdocby}
% The command |\childdocby| ....
%    \begin{macrocode}
\newcommand{\childdocby}[2][]
{
  \childdocdisable
  \childdoctrue
  \childdocmanualtrue
  \if?#1?\else
    \def\jobname{#2}
  \fi
  \def\childdocjob{#2}
  \input{#2}
  \endinput
}
%    \end{macrocode}

% \macro{\childdocforward}
% The command |\childdocforward| redirects
% compilation to the main file or
% (if the optional argument is given) a child file.
% Parameters are set as if the main file
% or a child file starting with |\childdocof| was compiled.
% Then compilation is handed over to the main file:
%    \begin{macrocode}
\newcommand{\childdocforward}[2][]
{
  \begingroup
    \if?#1?
      \def\childdoctmp
      {
        \def\childdocname{#2}
        \def\childdocjob{#2}
        \def\jobname{#2}
        \input{#2}
        \endinput
      }
    \else
      \def\childdoctmp
      {
        \childdocdisable
        \def\childdocname{#2}
        \childdoctrue
        \includeonly{#2}
        \def\childdocjob{#1}
        \def\jobname{#1}
        \input{#1}
        \endinput
      }
    \fi
    \expandafter
  \endgroup
  \childdoctmp
}
%    \end{macrocode}

% \macro{\childdocforwardprefix}
% The command |\childdocforwardprefix| redirects
% compilation to the main or a child file by means of a pattern.
% The prefix |#1| in the current filename is replaced by |#2|
% and the suffix of the current filename is kept
% (it is assumed that the filename does not contain the substring `|~~~|'
% which is used as a delimiter).
% Compilation is handed over to the new file by |\childdocforward|:
%    \begin{macrocode}
\newcommand{\childdocforwardprefix}[3][]
{
  \begingroup
    \def\childdocextract #2##1~~~{\def\childdoctmp{\childdocforward[#1]{#3##1}}}
    \expandafter\childdocextract\childdocname~~~
    \expandafter
  \endgroup
  \childdoctmp
}
%    \end{macrocode}

% \macro{\childdoc}
% The deprecated macro |\childdoc| is a legacy version of |\childdocmain|:
%    \begin{macrocode}
\newcommand{\childdoc}{\childdocmain}
%    \end{macrocode}

% \macro{\childdocredirect}
% The deprecated macro |\childdocredirect| is a legacy version
% of |\childdocforward| and |\childdocforwardprefix|:
%    \begin{macrocode}
\newcommand{\childdocredirect}[2][]
{
  \begingroup
    \if?#1?
      \def\childdoctmp{\childdocforward{#2}}
    \else
      \def\childdoctmp{\childdocforwardprefix{#1}{#2}}
    \fi
    \expandafter
  \endgroup
  \childdoctmp
}
%    \end{macrocode}

%\iffalse
%</package>
%\fi
%
\endinput
|\\
|\childdocof{|\textit{main}|}|\\
\end{tabular}
\end{center}
at the top of every child file \textit{child}
which is included by |\include{|\textit{child}|}|
from within the main file
(or at least for those files to be compiled individually).
The argument \textit{main} must be the filename of the main file.

There are a couple of
considerations in setting up the main and child documents:

%%%%%%%%%%%%%%%%%%%%%%%%%%%%%%%%%%%%%%%%
\paragraph{Restrictions.}

Please note the following restrictions:
\begin{itemize}
\item
|\childdocmain| must be called with one argument \textit{main}
to ensure compatibility with earlier version of the package.
It must either be empty (|\childdocmain{}|)
or precisely match the filename of the main file in which it is specified.
See \secref{sec:detection} for further information.
\item
The filename \textit{main} must be specified without the |.tex| extension.
\item
The filename \textit{main} is case sensitive
(even in case-insensitive file systems)
due to internal string comparison.
\item
The argument \textit{main} should be fully expanded, it cannot be a macro.
\item
Subdirectories and special characters should be avoided in filenames.
\item
The command |\childdocmain{|\textit{main}|}| must be followed by a whitespace.
It should not be followed immediately by another command
or by a comment mark `|%|'.
This is because the \TeX{} parser reads the token immediately following
the argument of |\childdocmain| and puts it
at the beginning of every child section;
however, a white\-space is ignored.
\end{itemize}

%%%%%%%%%%%%%%%%%%%%%%%%%%%%%%%%%%%%%%%%
\paragraph{Content of Main File.}

It is advisable to place all content in the child files included by |\include|.
Any output contained in the main file will appear in all child documents
unless suppressed manually;
it cannot be suppressed automatically by the |\includeonly| directive
and thus should normally be avoided.
A method to include some content in the main file
by means of conditional processing is described in \secref{sec:conditional}.

%%%%%%%%%%%%%%%%%%%%%%%%%%%%%%%%%%%%%%%%
\paragraph{Page Numbering.}

When only a part of the document is compiled,
the appropriate numbering of pages
(as well as other status parameters)
is determined from the |.aux| files.
The latter contain information from previous passes.
However this information needs to propagate through
all intermediate child documents.
Therefore the page numbering in child documents may well
be inconsistent until the complete document is compiled at least once.

A useful (if unconventional) way to always ensure a consistent
page numbering is to restart the numbering in each child document
and denote the pages by `\textit{child}|.|\textit{page}'
where \textit{child} represents the chapter/section number of the child file.
This can be achieved by the command
|\numberwithin{page}{|\textit{child}|}|
of the \textsf{amsmath} package
where \textit{child} can be |chapter| or |section|
depending on the chosen structuring.
Alternatively, one can modify the macro |\thepage| appropriately
and reset the counter |page| at the start of each child file.

%%%%%%%%%%%%%%%%%%%%%%%%%%%%%%%%%%%%%%%%%%%%%%%%%%%%%%%%%%%%%%%%%%%%%%%%%%%%%%%%
\subsection{Conditional Processing}
\label{sec:conditional}

The package provides a mechanism to compile different versions
of a document. To customise the versions further some conditional processing
can come in handy to distinguish which version is being compiled.
The package provides two macros to describe the compilation context:

%%%%%%%%%%%%%%%%%%%%%%%%%%%%%%%%%%%%%%%%
\DescribeMacro{\ifchilddoc}
The conditional |\ifchilddoc| distinguishes between the compilation of
child documents and the main document:
%
\begin{center}
|\ifchilddoc |\textit{child-code}| |[|\||else |\textit{main-code}]| \||fi|
\end{center}

%%%%%%%%%%%%%%%%%%%%%%%%%%%%%%%%%%%%%%%%
\DescribeMacro{\childdocname}
\DescribeMacro{\childdocjob}
The macro |\childdocname| contains the filename (without extension)
of the main or child file being processed.
Note that |\childdocjob| will always contain the name of the main file.

%%%%%%%%%%%%%%%%%%%%%%%%%%%%%%%%%%%%%%%%
\paragraph{Title Page.}

Conditional processing can be used to include a title or banner page
in the main document when proper precautions are taken.
Importantly, the code in the main file should ensure that the page counter
(as well as other status parameters which are stored in the |.aux| files)
takes the same value after the conditional processing.
Otherwise the page numbers may take divergent values
depending on which part is compiled.

For example, a title page could be declared by:
%
\begin{center}
\begin{tabular}{l}
|\ifchilddoc\||else|\\
|\addtocounter{page}{-1}|\\
\textit{code for title page}\\
|\newpage|\\
|\||fi|
\end{tabular}
\end{center}
%
A banner page for the child documents can be generated by:
%
\begin{center}
\begin{tabular}{l}
|\ifchilddoc|\\
|\addtocounter{page}{-1}|\\
\textit{code for banner page}\\
|\newpage|\\
|\||fi|
\end{tabular}
\end{center}
%
Here one could write a message such as:
\begin{center}
|This is the part \childdocname{} of \childdocjob{}.|
\end{center}

%%%%%%%%%%%%%%%%%%%%%%%%%%%%%%%%%%%%%%%%%%%%%%%%%%%%%%%%%%%%%%%%%%%%%%%%%%%%%%%%
\subsection{Flags}
\label{sec:flags}

The package makes it easy to generate different versions
of the main or child documents.
To this end compilation flags can be defined
and assigned different default values.
They will be particularly useful in conjunction
with the forwarding mechanism described in \secref{sec:forward}.

For example, it may be useful to have a flag |\version|
which can be set to |draft| or |final|.
The document source will contain some conditional code
depending on the value of |\version|.
Suppose further, the flag should default to |final| for the main file
and to |draft| for child files
which is a natural assignment for editing the document.
This is achieved by placing the following code
in the preamble of the main document
(below the |\childdocmain| directive):
%
\begin{center}
\begin{tabular}{l}
|\ifchilddoc|\\
|\providecommand{\version}{draft}|\\
|\||else|\\
|\providecommand{\version}{final}|\\
|\||fi|
\end{tabular}
\end{center}
%
The definition by |\providecommand| makes sure
that previous definitions are not overwritten.
Further statements |\providecommand{\version}{...}|
can thus be added before the above code to override it.

For the main file, one might add a line
(between |\childdocmain| and the above block)
%
\begin{center}
|%\ifchilddoc\||else\providecommand{\version}{draft}\||fi|
\end{center}
%
which can be uncommented to produce a draft version.
Likewise one can add a line to the very top of a child file
(above the |\childdocof{|\textit{main}|}| directive)
%
\begin{center}
|%\providecommand{\version}{final}|
\end{center}
%
which can be uncommented to produce the final version of this child document.

%%%%%%%%%%%%%%%%%%%%%%%%%%%%%%%%%%%%%%%%%%%%%%%%%%%%%%%%%%%%%%%%%%%%%%%%%%%%%%%%
\subsection{Forwarding}
\label{sec:forward}

Different versions of the main or child documents
using compilation flags as described in \secref{sec:flags}
can be (permanently) stored in different files
for convenient compilation, viewing and distribution.
To this end, the package defines a command
to pass on compilation to a different file:

%%%%%%%%%%%%%%%%%%%%%%%%%%%%%%%%%%%%%%%%
\DescribeMacro{\childdocforward}
The command |\childdocforward| redirects processing to
another source file:
%
\begin{center}
\begin{tabular}{l}
|% \iffalse
%
% childdoc.dtx Copyright (C) 2017-2018 Niklas Beisert
%
% This work may be distributed and/or modified under the
% conditions of the LaTeX Project Public License, either version 1.3
% of this license or (at your option) any later version.
% The latest version of this license is in
%   http://www.latex-project.org/lppl.txt
% and version 1.3 or later is part of all distributions of LaTeX
% version 2005/12/01 or later.
%
% This work has the LPPL maintenance status `maintained'.
%
% The Current Maintainer of this work is Niklas Beisert.
%
% This work consists of the files childdoc.dtx and childdoc.ins
% and the derived files childdoc.def and cdocsamp.tex with
% cdocsch1.tex, cdocsch2.tex, cdocsdrf.tex, cdocsfn1.tex, cdocsfn2.tex.
%
%<package>\ifdefined\childdocmain\endinput\fi
%<package>\ProvidesFile{childdoc.def}[2018/12/30 v2.0 child document driver]
%<samplemain>\ProvidesFile{cdocsamp.tex}[2018/12/30 v2.0 sample for childdoc]
%<*driver>
%\ProvidesFile{childdoc.drv}[2018/12/30 v2.0 childdoc reference manual file]
\PassOptionsToClass{10pt,a4paper}{article}
\documentclass{ltxdoc}

\usepackage[margin=35mm]{geometry}
\usepackage{hyperref}
\usepackage{hyperxmp}
\usepackage[usenames]{color}

\hypersetup{colorlinks=true}
\hypersetup{pdfstartview=FitH}
\hypersetup{pdfpagemode=UseNone}
\hypersetup{pdfsource={}}
\hypersetup{pdflang={en-UK}}
\hypersetup{pdfcopyright={Copyright 2017-2018 Niklas Beisert.
  This work may be distributed and/or modified under the
  conditions of the LaTeX Project Public License, either version 1.3
  of this license or (at your option) any later version.}}
\hypersetup{pdflicenseurl={http://www.latex-project.org/lppl.txt}}
\hypersetup{pdfcontactaddress={ETH Zurich, ITP, HIT K,
  Wolfgang-Pauli-Strasse 27}}
\hypersetup{pdfcontactpostcode={8093}}
\hypersetup{pdfcontactcity={Zurich}}
\hypersetup{pdfcontactcountry={Switzerland}}
\hypersetup{pdfcontactemail={nbeisert@itp.phys.ethz.ch}}
\hypersetup{pdfcontacturl={http://people.phys.ethz.ch/\xmptilde nbeisert/}}

\newcommand{\secref}[1]{\hyperref[#1]{section \ref*{#1}}}

\parskip1ex
\parindent0pt
\let\olditemize\itemize
\def\itemize{\olditemize\parskip0pt}

\begin{document}

\title{The \textsf{childdoc} Package}
\hypersetup{pdftitle={The childdoc Package}}
\author{Niklas Beisert\\[2ex]
  Institut f\"ur Theoretische Physik\\
  Eidgen\"ossische Technische Hochschule Z\"urich\\
  Wolfgang-Pauli-Strasse 27, 8093 Z\"urich, Switzerland\\[1ex]
  \href{mailto:nbeisert@itp.phys.ethz.ch}
  {\texttt{nbeisert@itp.phys.ethz.ch}}}
\hypersetup{pdfauthor={Niklas Beisert}}
\hypersetup{pdfsubject={Manual for the LaTeX2e Package childdoc}}
\date{30 December 2018, \textsf{v2.0}}
\maketitle

\begin{abstract}\noindent
\textsf{childdoc} is a \LaTeXe{} package
that enables the direct compilation
of document sections included by |\include|
to individual files.
\end{abstract}

\begingroup
\parskip0ex
\tableofcontents
\endgroup

%%%%%%%%%%%%%%%%%%%%%%%%%%%%%%%%%%%%%%%%%%%%%%%%%%%%%%%%%%%%%%%%%%%%%%%%%%%%%%%%
%%%%%%%%%%%%%%%%%%%%%%%%%%%%%%%%%%%%%%%%%%%%%%%%%%%%%%%%%%%%%%%%%%%%%%%%%%%%%%%%
\section{Introduction}

\LaTeX{} provides a mechanism to structure a large document (such as a book)
into a main file and several child files (containing the chapters)
using the |\include| command.
This mechanism is beneficial for documents
which span hundreds of pages in order to
make the source file(s) more manageable.
Moreover, compilation can be restricted to
selected child files by means of the |\includeonly| command.
The latter feature can be used to reduce the compilation time while editing
(this was significantly more useful in the earlier days of \LaTeX{})
or to generate a smaller document which is easier to navigate.
Another application of |\includeonly| is to generate
documents consisting of selected parts of the complete document.

However, there are a few drawbacks of the plain |\include| mechanism:
\begin{itemize}
\item
The child files cannot be compiled on their own,
they can only be compiled via the main file.
A naive editing environment
(such as a text editor with an option
to have the current file processed by \LaTeX)
may require one to switch to the main file before compiling;
attempting to compile the child file produces errors.
\item
The main file must be modified (each time)
to adjust the |\includeonly| command
to the present needs. This easily leaves the main file in a messy state.
\item
The generated document will always carry the filename
of the main document. This is inconvenient if
several child files are to be compiled and
to be kept for distribution.
\end{itemize}

The present package provides a simple interface
to make child files individually compilable by \LaTeX{}.
Compiling a child file then has the same effect as compiling
the main file with an |\includeonly| command
to select the appropriate child.
Moreover the generated document will carry the name of the child
rather than the main file.
This resolves all three above issues.

This feature is meant to make the editing of books,
thesis documents and lecture notes somewhat more convenient.
However, the package can also be used efficiently for
composing a series of documents (such as exercise sheets)
which are typically distributed individually.
It then assists the author in generating the individual documents
(potentially in different versions)
as well as a document containing the collected series.
Another application is in developing style files
or other kinds of included material
where compilation of the style file could redirect
to a sample or test file.

%%%%%%%%%%%%%%%%%%%%%%%%%%%%%%%%%%%%%%%%%%%%%%%%%%%%%%%%%%%%%%%%%%%%%%%%%%%%%%%%
%%%%%%%%%%%%%%%%%%%%%%%%%%%%%%%%%%%%%%%%%%%%%%%%%%%%%%%%%%%%%%%%%%%%%%%%%%%%%%%%
\section{Usage}

First of all, the package \textsf{childdoc} is \emph{not} a standard
\LaTeXe{} |.sty| style file! Therefore it needs to be invoked in
a non-standard way.

%%%%%%%%%%%%%%%%%%%%%%%%%%%%%%%%%%%%%%%%%%%%%%%%%%%%%%%%%%%%%%%%%%%%%%%%%%%%%%%%
\subsection{Included Files}
\label{sec:include}

%%%%%%%%%%%%%%%%%%%%%%%%%%%%%%%%%%%%%%%%
\DescribeMacro{\childdocmain}
To use the package, add the commands
\begin{center}
\begin{tabular}{l}
|\input{childdoc.def}|\\
|\childdocmain{}|\\
\end{tabular}
\end{center}
at the very top of the main \LaTeX{} file,
in particular \emph{before} the |\documentclass| statement!
The argument of |\childdocmain| should be left empty
(but it must be present).

%%%%%%%%%%%%%%%%%%%%%%%%%%%%%%%%%%%%%%%%
\DescribeMacro{\childdocof}
Furthermore, add the commands
\begin{center}
\begin{tabular}{l}
|\input{childdoc.def}|\\
|\childdocof{|\textit{main}|}|\\
\end{tabular}
\end{center}
at the top of every child file \textit{child}
which is included by |\include{|\textit{child}|}|
from within the main file
(or at least for those files to be compiled individually).
The argument \textit{main} must be the filename of the main file.

There are a couple of
considerations in setting up the main and child documents:

%%%%%%%%%%%%%%%%%%%%%%%%%%%%%%%%%%%%%%%%
\paragraph{Restrictions.}

Please note the following restrictions:
\begin{itemize}
\item
|\childdocmain| must be called with one argument \textit{main}
to ensure compatibility with earlier version of the package.
It must either be empty (|\childdocmain{}|)
or precisely match the filename of the main file in which it is specified.
See \secref{sec:detection} for further information.
\item
The filename \textit{main} must be specified without the |.tex| extension.
\item
The filename \textit{main} is case sensitive
(even in case-insensitive file systems)
due to internal string comparison.
\item
The argument \textit{main} should be fully expanded, it cannot be a macro.
\item
Subdirectories and special characters should be avoided in filenames.
\item
The command |\childdocmain{|\textit{main}|}| must be followed by a whitespace.
It should not be followed immediately by another command
or by a comment mark `|%|'.
This is because the \TeX{} parser reads the token immediately following
the argument of |\childdocmain| and puts it
at the beginning of every child section;
however, a white\-space is ignored.
\end{itemize}

%%%%%%%%%%%%%%%%%%%%%%%%%%%%%%%%%%%%%%%%
\paragraph{Content of Main File.}

It is advisable to place all content in the child files included by |\include|.
Any output contained in the main file will appear in all child documents
unless suppressed manually;
it cannot be suppressed automatically by the |\includeonly| directive
and thus should normally be avoided.
A method to include some content in the main file
by means of conditional processing is described in \secref{sec:conditional}.

%%%%%%%%%%%%%%%%%%%%%%%%%%%%%%%%%%%%%%%%
\paragraph{Page Numbering.}

When only a part of the document is compiled,
the appropriate numbering of pages
(as well as other status parameters)
is determined from the |.aux| files.
The latter contain information from previous passes.
However this information needs to propagate through
all intermediate child documents.
Therefore the page numbering in child documents may well
be inconsistent until the complete document is compiled at least once.

A useful (if unconventional) way to always ensure a consistent
page numbering is to restart the numbering in each child document
and denote the pages by `\textit{child}|.|\textit{page}'
where \textit{child} represents the chapter/section number of the child file.
This can be achieved by the command
|\numberwithin{page}{|\textit{child}|}|
of the \textsf{amsmath} package
where \textit{child} can be |chapter| or |section|
depending on the chosen structuring.
Alternatively, one can modify the macro |\thepage| appropriately
and reset the counter |page| at the start of each child file.

%%%%%%%%%%%%%%%%%%%%%%%%%%%%%%%%%%%%%%%%%%%%%%%%%%%%%%%%%%%%%%%%%%%%%%%%%%%%%%%%
\subsection{Conditional Processing}
\label{sec:conditional}

The package provides a mechanism to compile different versions
of a document. To customise the versions further some conditional processing
can come in handy to distinguish which version is being compiled.
The package provides two macros to describe the compilation context:

%%%%%%%%%%%%%%%%%%%%%%%%%%%%%%%%%%%%%%%%
\DescribeMacro{\ifchilddoc}
The conditional |\ifchilddoc| distinguishes between the compilation of
child documents and the main document:
%
\begin{center}
|\ifchilddoc |\textit{child-code}| |[|\||else |\textit{main-code}]| \||fi|
\end{center}

%%%%%%%%%%%%%%%%%%%%%%%%%%%%%%%%%%%%%%%%
\DescribeMacro{\childdocname}
\DescribeMacro{\childdocjob}
The macro |\childdocname| contains the filename (without extension)
of the main or child file being processed.
Note that |\childdocjob| will always contain the name of the main file.

%%%%%%%%%%%%%%%%%%%%%%%%%%%%%%%%%%%%%%%%
\paragraph{Title Page.}

Conditional processing can be used to include a title or banner page
in the main document when proper precautions are taken.
Importantly, the code in the main file should ensure that the page counter
(as well as other status parameters which are stored in the |.aux| files)
takes the same value after the conditional processing.
Otherwise the page numbers may take divergent values
depending on which part is compiled.

For example, a title page could be declared by:
%
\begin{center}
\begin{tabular}{l}
|\ifchilddoc\||else|\\
|\addtocounter{page}{-1}|\\
\textit{code for title page}\\
|\newpage|\\
|\||fi|
\end{tabular}
\end{center}
%
A banner page for the child documents can be generated by:
%
\begin{center}
\begin{tabular}{l}
|\ifchilddoc|\\
|\addtocounter{page}{-1}|\\
\textit{code for banner page}\\
|\newpage|\\
|\||fi|
\end{tabular}
\end{center}
%
Here one could write a message such as:
\begin{center}
|This is the part \childdocname{} of \childdocjob{}.|
\end{center}

%%%%%%%%%%%%%%%%%%%%%%%%%%%%%%%%%%%%%%%%%%%%%%%%%%%%%%%%%%%%%%%%%%%%%%%%%%%%%%%%
\subsection{Flags}
\label{sec:flags}

The package makes it easy to generate different versions
of the main or child documents.
To this end compilation flags can be defined
and assigned different default values.
They will be particularly useful in conjunction
with the forwarding mechanism described in \secref{sec:forward}.

For example, it may be useful to have a flag |\version|
which can be set to |draft| or |final|.
The document source will contain some conditional code
depending on the value of |\version|.
Suppose further, the flag should default to |final| for the main file
and to |draft| for child files
which is a natural assignment for editing the document.
This is achieved by placing the following code
in the preamble of the main document
(below the |\childdocmain| directive):
%
\begin{center}
\begin{tabular}{l}
|\ifchilddoc|\\
|\providecommand{\version}{draft}|\\
|\||else|\\
|\providecommand{\version}{final}|\\
|\||fi|
\end{tabular}
\end{center}
%
The definition by |\providecommand| makes sure
that previous definitions are not overwritten.
Further statements |\providecommand{\version}{...}|
can thus be added before the above code to override it.

For the main file, one might add a line
(between |\childdocmain| and the above block)
%
\begin{center}
|%\ifchilddoc\||else\providecommand{\version}{draft}\||fi|
\end{center}
%
which can be uncommented to produce a draft version.
Likewise one can add a line to the very top of a child file
(above the |\childdocof{|\textit{main}|}| directive)
%
\begin{center}
|%\providecommand{\version}{final}|
\end{center}
%
which can be uncommented to produce the final version of this child document.

%%%%%%%%%%%%%%%%%%%%%%%%%%%%%%%%%%%%%%%%%%%%%%%%%%%%%%%%%%%%%%%%%%%%%%%%%%%%%%%%
\subsection{Forwarding}
\label{sec:forward}

Different versions of the main or child documents
using compilation flags as described in \secref{sec:flags}
can be (permanently) stored in different files
for convenient compilation, viewing and distribution.
To this end, the package defines a command
to pass on compilation to a different file:

%%%%%%%%%%%%%%%%%%%%%%%%%%%%%%%%%%%%%%%%
\DescribeMacro{\childdocforward}
The command |\childdocforward| redirects processing to
another source file:
%
\begin{center}
\begin{tabular}{l}
|\input{childdoc.def}|\\
|\childdocforward[|\textit{main}|]{|\textit{dest}|}|\\
\end{tabular}
\end{center}
%
The argument \textit{dest} is the destination file
(without extension).
It should be the main file or one of the child files.
Note that further \textsf{childdoc} directives
such as |\childdocof| and |\childdocforward|
in the indicated file will be processed in this form.
The optional argument \textit{main}
passes on directly to the main file \textit{main}
while pretending to compile the child \textit{dest}.
This form behaves as if \textit{dest}
issues |\childdocof{|\textit{main}|}| right away,
and no further \textsf{childdoc} directives will be processed.

%%%%%%%%%%%%%%%%%%%%%%%%%%%%%%%%%%%%%%%%
\DescribeMacro{\...prefix}
In the alternative form |\childdocforwardprefix|,
%
\begin{center}
\begin{tabular}{l}
|\input{childdoc.def}|\\
|\childdocforwardprefix[|\textit{main}|]{|\textit{prefix}|}{|\textit{dest}|}|
\end{tabular}
\end{center}
%
the destination file is determined by a pattern
depending on the current file:
To make this work, the current file must be called
`{\textit{prefix}\hspace{0.2em}\textit{suffix}}'
with \textit{prefix} matching precisely the argument.
Processing is then passed on to the file
`{\textit{dest}\hspace{0.2em}\textit{suffix}}'.
Surely, the same effect is achieved by
directly specifying the
argument `{\textit{dest}\hspace{0.2em}\textit{suffix}}'
in the first form.
However, that requires to set up a different file
for each child. With the alternative form of the command
all these files can have exactly the same content
which simplifies setting them up and maintaining them.

For example, the following file |draft.tex|
with a compilation flag |\version| as described in \secref{sec:flags}
compiles the main document as a draft:
%
\begin{center}
\begin{tabular}{l}
|\def\version{draft}|\\
|\input{childdoc.def}|\\
|\childdocforward{|\textit{main}|}|
\end{tabular}
\end{center}
%
Likewise, the following files |final|\textit{nn}|.tex|
compile the final version of the child document
|child|\textit{nn}|.tex|:
%
\begin{center}
\begin{tabular}{l}
|\def\version{final}|\\
|\input{childdoc.def}|\\
|\childdocforwardprefix{final}{child}|
\end{tabular}
\end{center}
%

Note that when several versions of a main file and/or of each child file
are to be generated, it may be convenient to set up a |Makefile| or
shell script to automatise the process.

%%%%%%%%%%%%%%%%%%%%%%%%%%%%%%%%%%%%%%%%%%%%%%%%%%%%%%%%%%%%%%%%%%%%%%%%%%%%%%%%
\subsection{Command Line Processing}
\label{sec:commandline}

The effect of redirection files can also be achieved by invoking
the \LaTeX{} compiler with a more elaborate command line.
Most conveniently this should be done as part
of a shell script or a |Makefile|.

When using \textsf{childdoc} in the main file, the following
command lines effectively perform a redirection
(note that depending on the shell being used,
backslashes may have to be doubled: `|\|' $\to$ `|\\|'):
%
\begin{center}
|... -jobname "|\textit{target}|" |\\|"|[\textit{flags}]%
|\input{childdoc.def}\childdocforward[|\textit{main}|]{|\textit{dest}|}"|
\end{center}
%
Here \textit{target} is the name of the output file,
\textit{main} is the name of the main file
and \textit{dest} is the name of the main or child file to be processed
(all filenames without extensions).
The optional argument \textit{main} can be omitted
if \textit{main} matches \textit{dest}.
Optionally, compilation \textit{flags} can be defined via |\def| commands.
This command line makes the \TeX{} engine believe
it is compiling the file \textit{target}
whose content is specified as the latter parameter.
The provided code then forwards the processing to
\textit{main} or \textit{dest} as described in \secref{sec:forward}.

%%%%%%%%%%%%%%%%%%%%%%%%%%%%%%%%%%%%%%%%%%%%%%%%%%%%%%%%%%%%%%%%%%%%%%%%%%%%%%%%
\subsection{Include by Input}
\label{sec:input}

Including child documents by |\include| has some restrictions by design.
Most notably, the content of a child document always occupies
its own set of pages; pages cannot be shared between child documents.
Usually, this behaviour makes perfect sense
because each child document contain an essential part of the document.
However, in some situations it may be desirable to compose
a document from a collection of parts
without having mandatory page breaks between then.
For this case, the package
provides a mechanism to include parts
by |\input| which can also be processed individually.
However, by construction this mechanism
requires manual handling of the content to be output.

%%%%%%%%%%%%%%%%%%%%%%%%%%%%%%%%%%%%%%%%
\DescribeMacro{\ifchilddocmanual}
The main file should be prepared as usual, see \secref{sec:include}.
However, the document body must make a distinction
between processing of an individual part and of the main document, e.g.:
%
\begin{center}
\begin{tabular}{l}
|\ifchilddocmanual|\\
|\input{\childdocname}|\\
|\||else|\\
\textit{document body with }|\input{|\textit{part}|}|\\
|\||fi|
\end{tabular}
\end{center}
%
The conditional |\ifchilddocmanual| is true whenever
a part to be included by |\input| is being compiled,
and the name of the part is stored in |\childdocname|.

%%%%%%%%%%%%%%%%%%%%%%%%%%%%%%%%%%%%%%%%
\DescribeMacro{\childdocby}
Each part to be included by |\input| should start with:
%
\begin{center}
\begin{tabular}{l}
|\input{childdoc.def}|\\
|\childdocby{|\textit{main}|}|\\
\end{tabular}
\end{center}
%
The directive |\childdocby| is similar to |\childdocof|
described in \secref{sec:include},
but the subsequent selection of content must be done manually.
To that end, both |\ifchilddoc| and |\ifchilddocmanual|
will be true upon processing of a part,
and the name of the part is stored in |\childdocname|.
Note that |\jobname| will be set to the filename of the current part
so that each part receives an individual |.aux| file
that does not interfere with the |.aux| file(s) of the main document.
This behaviour can be altered by the alternative form
|\childdocby[*]{|\textit{main}|}| (with a non-empty optional argument)
which uses the |.aux| file of the main document
by setting |\jobname| to \textit{main}.

%%%%%%%%%%%%%%%%%%%%%%%%%%%%%%%%%%%%%%%%%%%%%%%%%%%%%%%%%%%%%%%%%%%%%%%%%%%%%%%%
\subsection{Driver Development}
\label{sec:driver}

The \textsf{childdoc} mechanism can also be use for the development
of definition files such as \LaTeX{} styles or classes.
This case differs from the above setup with multiple parts
included by |\include| in that no |\includeonly| should be invoked.
This can be achieved by starting the include file
(before |\ProvidesPackage|) with:
%
\begin{center}
\begin{tabular}{l}
|\input{childdoc.def}|\\
|\childdocforward{|\textit{main}|}|\\
\end{tabular}
\end{center}
%
or alternatively with:
%
\begin{center}
\begin{tabular}{l}
|\input{childdoc.def}|\\
|\childdocby{|\textit{main}|}|\\
\end{tabular}
\end{center}
%
Both forms have slightly different effects as described above.
The main file is prepared as usual, see \secref{sec:include}.

%%%%%%%%%%%%%%%%%%%%%%%%%%%%%%%%%%%%%%%%%%%%%%%%%%%%%%%%%%%%%%%%%%%%%%%%%%%%%%%%
\subsection{Legacy Detection}
\label{sec:detection}

The directive |\childdocmain| in the main file can detect
whether the complete document or merely a child is to be compiled
even without using the directive |\childdocof|.
This method is deprecated because it is less robust
and there is no compelling reason to use it;
it is merely provided for backward compatibility
and it may be removed in future versions.

If the detection mechanism is to be used,
it is mandatory to correctly specify
the filename of the main file as the argument of |\childdocmain|:
%
\begin{center}
\begin{tabular}{l}
|\input{childdoc.def}|\\
|\childdocmain{|\textit{main}|}|\\
\end{tabular}
\end{center}
%
If |\jobname| does not match the argument \textit{main} of |\childdocmain|,
it is assumed that |\jobname| points to the child file to be compiled.
When using |\childdocmain| with the main file specified as argument,
it suffices to start a child file
with just |\input{|\textit{main}|}|
without loading of the package and using |\childdocof|.
If instead all processing is done
with the appropriate \textsf{childdoc} directives,
the argument of \textit{main} of |\childdocmain| can be empty.

An alternative version of the command line processing described
in \secref{sec:commandline} using the detection mechanism reads:
%
\begin{center}
|... -jobname "|\textit{target}|" "|[\textit{flags}]%
[|\def\jobname{|\textit{dest}|}|]|\input{|\textit{main}|}"|
\end{center}

%%%%%%%%%%%%%%%%%%%%%%%%%%%%%%%%%%%%%%%%%%%%%%%%%%%%%%%%%%%%%%%%%%%%%%%%%%%%%%%%
\subsection{Manual Code}
\label{sec:manual}

In case one cannot be certain whether the definitions file |childdoc.def|
is installed on the target \TeX{} distribution
and one prefers not to ship it,
it is conceivable to paste a few relevant commands into the sources.

To that end, drop all statements |\input{childdoc.def}|
and perform the replacements as outlined below.
Instead of |\childdocmain{|\textit{main}|}| add the following code
to the top of the main file:
%
\begin{center}
\begin{tabular}{l}
|\||ifdefined\childdocname\endinput\||fi\newif\ifchilddoc|\\
|\edef\childdocname{\scantokens\expandafter{\jobname\noexpand}}|\\
|\def\childdocmain{|\textit{main}|}\||ifx\childdocmain\childdocname\||else|\\
|\childdoctrue\includeonly{\childdocname}\let\jobname\childdocmain\||fi|\\
\end{tabular}
\end{center}
%
Instead of |\childdocof{|\textit{main}|}| just include the main file
at the top of each child file:
%
\begin{center}
|\input{|\textit{main}|}|
\end{center}
%
A simple redirection |\childdocforward{|\textit{dest}|}| is achieved by:
%
\begin{center}
|\def\jobname{|\textit{dest}|}\input{\jobname}|
\end{center}
%
The redirection with prefix
|\childdocforwardprefix[|\textit{prefix}|]{|\textit{dest}|}|
is accomplished by:
%
\begin{center}
\begin{tabular}{l}
|{\edef\jobname{\scantokens\expandafter{\jobname\noexpand}}|\\
|\def\redirectjob |\textit{prefix}|#1~~~{\gdef\jobname{|\textit{dest}|#1}}|\\
|\expandafter\redirectjob\jobname~~~}\input{\jobname}|
\end{tabular}
\end{center}

In an alternative approach,
child documents can be compiled by a specific command line
without additional code or specific definitions:
%
\begin{center}
|... -jobname "|\textit{target}|" "|[\textit{flags}]%
|\includeonly{|\textit{dest}|}\input{|\textit{main}|}"|
\end{center}
%

%%%%%%%%%%%%%%%%%%%%%%%%%%%%%%%%%%%%%%%%%%%%%%%%%%%%%%%%%%%%%%%%%%%%%%%%%%%%%%%%
%%%%%%%%%%%%%%%%%%%%%%%%%%%%%%%%%%%%%%%%%%%%%%%%%%%%%%%%%%%%%%%%%%%%%%%%%%%%%%%%
\section{Information}

%%%%%%%%%%%%%%%%%%%%%%%%%%%%%%%%%%%%%%%%%%%%%%%%%%%%%%%%%%%%%%%%%%%%%%%%%%%%%%%%
\subsection{Copyright}

Copyright \copyright{} 2017--2018 Niklas Beisert

This work may be distributed and/or modified under the
conditions of the \LaTeX{} Project Public License, either version 1.3
of this license or (at your option) any later version.
The latest version of this license is in
  \url{http://www.latex-project.org/lppl.txt}
and version 1.3 or later is part of all distributions of \LaTeX{}
version 2005/12/01 or later.

This work has the LPPL maintenance status `maintained'.

The Current Maintainer of this work is Niklas Beisert.

This work consists of the files |README.txt|, |childdoc.ins| and |childdoc.dtx|
as well as the derived files |childdoc.def|, |cdocsamp.tex|
with |cdocsch1.tex|, |cdocsch2.tex|, |cdocspt3.tex|, |cdocspt4.tex|,
|cdocsdrf.tex|, |cdocsfn1.tex|, |cdocsfn2.tex|
as well as |childdoc.pdf|.

%%%%%%%%%%%%%%%%%%%%%%%%%%%%%%%%%%%%%%%%%%%%%%%%%%%%%%%%%%%%%%%%%%%%%%%%%%%%%%%%
\subsection{Files and Installation}

The package consists of the files:
%
\begin{center}
\begin{tabular}{ll}
    |README.txt|   & readme file \\
    |childdoc.ins| & installation file \\
    |childdoc.dtx| & source file \\
    |childdoc.def| & definition file \\
    |cdocsamp.tex| & sample main file \\
    |cdocsch1.tex| & sample include file \\
    |cdocsch2.tex| & sample include file \\
    |cdocspt3.tex| & sample part file \\
    |cdocspt4.tex| & sample part file \\
    |cdocsdrf.tex| & sample redirection file \\
    |cdocsfn1.tex| & sample redirection file \\
    |cdocsfn2.tex| & sample redirection file \\
    |childdoc.pdf| & manual
\end{tabular}
\end{center}
%
The distribution consists of the files
|README.txt|, |childdoc.ins| and |childdoc.dtx|.
%
\begin{itemize}
\item
Run (pdf)\LaTeX{} on |childdoc.dtx|
to compile the manual |childdoc.pdf| (this file).
\item
Run \LaTeX{} on |childdoc.ins| to create the definitions file |childdoc.def|
and the sample |cdocsamp.tex| with include files
|cdocsch1.tex|, |cdocsch2.tex|, |cdocspt3.tex|, |cdocspt4.tex|,
|cdocsdrf.tex|, |cdocsfn1.tex|, |cdocsfn2.tex|.
Then copy the file |childdoc.def| to an appropriate directory of your \LaTeX{}
distribution, e.g.\ \textit{texmf-root}|/tex/latex/childdoc|.
\end{itemize}

%%%%%%%%%%%%%%%%%%%%%%%%%%%%%%%%%%%%%%%%%%%%%%%%%%%%%%%%%%%%%%%%%%%%%%%%%%%%%%%%
\subsection{Related CTAN Packages}

There are several other packages which offer a similar functionality:
%
\begin{itemize}
\item
The packages
\href{http://ctan.org/pkg/docmute}{\textsf{docmute}},
\href{http://ctan.org/pkg/includex}{\textsf{includex}} and
\href{http://ctan.org/pkg/standalone}{\textsf{standalone}}
provide commands to include only the document body of
a child file thus allowing both files to be compiled individually.
\item
The packages \href{http://ctan.org/pkg/subdocs}{\textsf{subdocs}}
and \href{http://ctan.org/pkg/subfiles}{\textsf{subfiles}}
provide structures in which the main and child documents can be
encapsulated and allowing them to be compiled individually.
The inclusion mechanism is different from the conventional |\include|.
\item
The package \href{http://ctan.org/pkg/combine}{\textsf{combine}}
is an elaborate solution to combine several documents into one.
\end{itemize}
%
See also the CTAN topic \href{http://ctan.org/topic/subdocs}{\textsf{subdocs}}
for further related packages.
The present package differs from the above solutions in that
a document structure constructed with the conventional |\include| mechanism
just needs two extra commands at the top of every file
such that all constituent files can be compiled individually.

%%%%%%%%%%%%%%%%%%%%%%%%%%%%%%%%%%%%%%%%%%%%%%%%%%%%%%%%%%%%%%%%%%%%%%%%%%%%%%%%
%\subsection{Feature Suggestions}
%
%The following is a list of features which may be useful for future
%versions of this package:
%%
%\begin{itemize}
%\item
%\ldots
%\end{itemize}

%%%%%%%%%%%%%%%%%%%%%%%%%%%%%%%%%%%%%%%%%%%%%%%%%%%%%%%%%%%%%%%%%%%%%%%%%%%%%%%%
\subsection{Revision History}

%%%%%%%%%%%%%%%%%%%%%%%%%%%%%%%%%%%%%%%%
\paragraph{v2.0:} 2018/12/30

\begin{itemize}
\item
immediate forward processing
\item
added |\childdocby| mechanism
\item
manual restructured
\end{itemize}

%%%%%%%%%%%%%%%%%%%%%%%%%%%%%%%%%%%%%%%%
\paragraph{v1.6:} 2018/01/17

\begin{itemize}
\item
application for development of include files
\item
corrections to manual
\end{itemize}

%%%%%%%%%%%%%%%%%%%%%%%%%%%%%%%%%%%%%%%%
\paragraph{v1.5:} 2017/05/21

\begin{itemize}
\item
more complete structuring introduced
\item
|\childdocof| introduced
\item
|\childdoc| renamed to |\childdocmain|
\item
|\childredirect| renamed to |\childdocforward| and |\childdocforwardprefix|
and functionality expanded
\end{itemize}

%%%%%%%%%%%%%%%%%%%%%%%%%%%%%%%%%%%%%%%%
\paragraph{v1.0:} 2017/04/27

\begin{itemize}
\item
manual and install package
\item
first version published on CTAN
\end{itemize}

%%%%%%%%%%%%%%%%%%%%%%%%%%%%%%%%%%%%%%%%
\paragraph{v0.6:} 2017/04/26

\begin{itemize}
\item
redirection mechanism added
\end{itemize}

%%%%%%%%%%%%%%%%%%%%%%%%%%%%%%%%%%%%%%%%
\paragraph{v0.5:} 2017/04/26

\begin{itemize}
\item
functionality in definition file
\end{itemize}


%%%%%%%%%%%%%%%%%%%%%%%%%%%%%%%%%%%%%%%%%%%%%%%%%%%%%%%%%%%%%%%%%%%%%%%%%%%%%%%%
%%%%%%%%%%%%%%%%%%%%%%%%%%%%%%%%%%%%%%%%%%%%%%%%%%%%%%%%%%%%%%%%%%%%%%%%%%%%%%%%
%%%%%%%%%%%%%%%%%%%%%%%%%%%%%%%%%%%%%%%%%%%%%%%%%%%%%%%%%%%%%%%%%%%%%%%%%%%%%%%%
\appendix

\settowidth\MacroIndent{\rmfamily\scriptsize 000\ }

 \DocInput{childdoc.dtx}

\end{document}
%</driver>
% \fi
%
% %%%%%%%%%%%%%%%%%%%%%%%%%%%%%%%%%%%%%%%%%%%%%%%%%%%%%%%%%%%%%%%%%%%%%%%%%%%%%%
% %%%%%%%%%%%%%%%%%%%%%%%%%%%%%%%%%%%%%%%%%%%%%%%%%%%%%%%%%%%%%%%%%%%%%%%%%%%%%%
% \section{Sample}
%\iffalse
%<*samplemain>
%\fi
%
% The following presents a sample document
% with two chapters, two parts, a title page,
% a compile flag as well as three forwarding files to set the flag.
% It consists of eight |.tex| files:
% \begin{center}
% \begin{tabular}{ll}
% |cdocsamp.tex|&main file\\
% |cdocsch1.tex|&include file for chapter 1\\
% |cdocsch2.tex|&include file for chapter 2\\
% |cdocspt3.tex|&include file for part 3\\
% |cdocspt4.tex|&include file for part 4\\
% |cdocsdrf.tex|&forwarding file for main file in draft mode\\
% |cdocsfi1.tex|&forwarding file for final version of chapter 1\\
% |cdocsfi2.tex|&forwarding file for final version of chapter 2\\
% \end{tabular}
% \end{center}
% Each of the eight files can be compiled directly by the \LaTeX{} compiler.
%
% %%%%%%%%%%%%%%%%%%%%%%%%%%%%%%%%%%%%%%
% \paragraph{Main File.}
%
% The main file is called |cdocsamp.tex|.
%
% Load the \textsf{childdoc} definitions and
% declare the filename for the main document:
%    \begin{macrocode}
\input{childdoc.def}
\childdocmain{}
%    \end{macrocode}

% Optional override for |\version| flag:
%    \begin{macrocode}
%%\ifchilddoc\else\providecommand{\version}{draft}\fi
%    \end{macrocode}

% Define the default values for the |\version| flag
% (|final| for the main file and |draft| for childs):
%    \begin{macrocode}
\ifchilddoc
\providecommand{\version}{draft}
\else
\providecommand{\version}{final}
\fi
%    \end{macrocode}

% Load the standard document class:
%    \begin{macrocode}
\documentclass[12pt]{article}
%    \end{macrocode}

% Start the document body:
%    \begin{macrocode}
\begin{document}
%    \end{macrocode}

% Declare a title page.
% Print title, part of document being processed and version flag:
%    \begin{macrocode}
\addtocounter{page}{-1}
\begin{center}
{\LARGE\bfseries{}childdoc example\par}
\vspace{1cm}
\ifchilddoc
\ifchilddocmanual part\else chapter\fi:
`\childdocname' of `\childdocjob'\par
\else
main document: `\childdocjob'\par
\fi
version: \version\par
\end{center}
\newpage
%    \end{macrocode}

% Manually include selected file,
% otherwise process as usual:
%    \begin{macrocode}
\ifchilddocmanual
\section*{part `\childdocname'}
\input{\childdocname}
\else
%    \end{macrocode}

% Include the two chapters:
%    \begin{macrocode}
\include{cdocsch1}
\include{cdocsch2}
%    \end{macrocode}

% Include the two parts unless only chapters should be displayed:
%    \begin{macrocode}
\ifchilddoc\else
\section{part three}
\input{cdocspt3}
\section{part four}
\input{cdocspt4}
\fi
%    \end{macrocode}

% Process as usual until here:
%    \begin{macrocode}
\fi
%    \end{macrocode}

% End of document body:
%    \begin{macrocode}
\end{document}
%    \end{macrocode}
%\iffalse
%</samplemain>
%\fi
%
% %%%%%%%%%%%%%%%%%%%%%%%%%%%%%%%%%%%%%%
% \paragraph{Chapter Include Files.}
%
% The include files are called |cdocsch1.tex| and |cdocsch2.tex|.
%
%\iffalse
%<*samplechap1|samplechap2>
%\fi

% Optional override for |\version| flag:
%    \begin{macrocode}
%%\providecommand{\version}{final}
%    \end{macrocode}

% Include the main document:
%    \begin{macrocode}
\input{childdoc.def}
\childdocof{cdocsamp}
%    \end{macrocode}

%\iffalse
%</samplechap1|samplechap2>
%\fi
%
%\iffalse
%<*samplechap1>
%\fi
% Some text for chapter 1:
%    \begin{macrocode}
\section{one}
some text in chapter one
%    \end{macrocode}

%\iffalse
%</samplechap1>
%\fi
% Some text for chapter 2:
%\iffalse
%<*samplechap2>
%\fi
%    \begin{macrocode}
\section{two}
more text in chapter two
%    \end{macrocode}

%\iffalse
%</samplechap2>
%\fi
%
% %%%%%%%%%%%%%%%%%%%%%%%%%%%%%%%%%%%%%%
% \paragraph{Part Include Files.}
%
% The include files are called |cdocspt3.tex| and |cdocspt4.tex|.
%
%\iffalse
%<*samplepart3|samplepart4>
%\fi

% Optional override for |\version| flag:
%    \begin{macrocode}
%%\providecommand{\version}{final}
%    \end{macrocode}

% Include the main document:
%    \begin{macrocode}
\input{childdoc.def}
\childdocby{cdocsamp}
%    \end{macrocode}

%\iffalse
%</samplepart3|samplepart4>
%\fi
%
%\iffalse
%<*samplepart3>
%\fi
% Some text for part 3:
%    \begin{macrocode}
some text in part three
%    \end{macrocode}

%\iffalse
%</samplepart3>
%\fi
% Some text for part 4:
%\iffalse
%<*samplepart4>
%\fi
%    \begin{macrocode}
more text in part four
%    \end{macrocode}

%\iffalse
%</samplepart4>
%\fi
%
% %%%%%%%%%%%%%%%%%%%%%%%%%%%%%%%%%%%%%%
% \paragraph{Forwarding for a Complete Draft.}
%
% The following forwarding file |cdocsdrf.tex|
% compiles the main document in draft mode:
%\iffalse
%<*sampledraft>
%\fi
%    \begin{macrocode}
\def\version{draft}
\input{childdoc.def}
\childdocforward{cdocsamp}
%    \end{macrocode}

%\iffalse
%</sampledraft>
%\fi
%
% %%%%%%%%%%%%%%%%%%%%%%%%%%%%%%%%%%%%%%
% \paragraph{Forwarding for Final Version of the Chapters.}
%
% The following forwarding files |cdocsfn1.tex| and |cdocsfn2.tex|
% (with identical content)
% compile the final versions of the child documents
% |cdocsch1.tex| and |cdocsch2.tex|, respectively:
%\iffalse
%<*samplefinal>
%\fi
%    \begin{macrocode}
\def\version{final}
\input{childdoc.def}
\childdocforwardprefix[cdocsamp]{cdocsfn}{cdocsch}
%    \end{macrocode}

%\iffalse
%</samplefinal>
%\fi
%
% %%%%%%%%%%%%%%%%%%%%%%%%%%%%%%%%%%%%%%
% \paragraph{Command Line Processing.}
%
% The following three command lines generate the output files
% |cdocscld|, |cdocscl1| and |cdocscl2|
% which should be identical to
% |cdocsdrf|, |cdocsch1| and |cdocsfn2|, respectively:
% \begin{center}
% \begin{tabular}{l}
% |latex -jobname cdocscld \|\\
% |  "\def\version{draft}\input{childdoc.def}\childdocforward{cdocsamp}"|\\
% |latex -jobname cdocscl1 \|\\
% |  "\input{childdoc.def}\childdocforward[cdocsamp]{cdocsch1}"|\\
% |latex -jobname cdocscl2 \|\\
% |  "\def\version{final}\input{childdoc.def}\childdocforward{cdocsch2}"|
% \end{tabular}
% \end{center}
% Note that the trailing backslash on each first line
% merely continues the input to the second line
% (for convenient cut ant paste).
% Furthermore, the command |latex| can be replaced by any
% of its alternative versions such as |pdflatex|.
%
% %%%%%%%%%%%%%%%%%%%%%%%%%%%%%%%%%%%%%%%%%%%%%%%%%%%%%%%%%%%%%%%%%%%%%%%%%%%%%%
% %%%%%%%%%%%%%%%%%%%%%%%%%%%%%%%%%%%%%%%%%%%%%%%%%%%%%%%%%%%%%%%%%%%%%%%%%%%%%%
% \section{Implementation}
%\iffalse
%<*package>
%\fi
%
% This section describes the definitions file |childdoc.def|.

% The definitions cannot be loaded using |\usepackage| or |\RequirePackage|
% which has a mechanism to prevent loading a style file more than once.
% When loading the definitions by means of |\input|
% multiple instances have to be prevented manually:
%\iffalse
%This code needs to be before the `\ProvidesFile' directive
%which is defined at the beginning of this file.
%Therefore it is also placed there and commented out here.
%</package>
%<*discard>
%\fi
%    \begin{macrocode}
\ifdefined\childdocmain\endinput\fi
%    \end{macrocode}
%\iffalse
%</discard>
%<*package>
%\fi
%
% \macro{\ifchilddoc}
% \macro{\ifchilddocmanual}
% The conditional |\ifchilddoc| tells whether a
% child (true) or main (false) document is being compiled.
% The conditional |\ifchilddocmanual| tells whether
% the |\includeonly| mechanism is used (false) or
% the selection of child files must be performed manually (true).
% The definitions initialise to false:
%    \begin{macrocode}
\newif\ifchilddoc
\newif\ifchilddocmanual
%    \end{macrocode}

% \macro{\childdocname}
% \macro{\childdocjob}
% The macro |\childdocname| stores the name of the main document
% to be compiled. The macro |\childdocjob| stores the name of
% the document on which the \LaTeX{} compiler was originally invoked.
% The content of |\jobname| cannot be compared
% to filenames specified in the source due to different catcodes.
% The following code rescans |\jobname|, stores the result
% in |\childdocname| and saves a copy in |\childdocjob|:
%    \begin{macrocode}
\edef\childdocname{\scantokens\expandafter{\jobname\noexpand}}
\let\childdocjob\childdocname
%    \end{macrocode}

% \macro{\childdocdisable}
% The macro |\childdocdisable| prevents the main file
% from being processed more than once.
% At this stage, the main document command |\childdocmain|
% is assumed to be called once again where it should do nothing.
% Any subsequent call to it should prevent
% a secondary processing of the main document
% It overwrites the forwarding commands
% |\childdocof| and |\childdocforward|
% with empty macros to prevent further inclusions of the main document:
%    \begin{macrocode}
\newcommand{\childdocdisable}
{
  \renewcommand{\childdocmain}[1]{\renewcommand{\childdocmain}[1]{\endinput}}
  \renewcommand{\childdocof}[1]{}
  \renewcommand{\childdocby}[2][]{}
  \renewcommand{\childdocforward}[2][]{}
  \renewcommand{\childdocdisable}{}
}
%    \end{macrocode}

% \macro{\childdocmain}
% The macro |\childdocmain| is to be called at the top of the main file
% with nothing or the main filename (without extension) as argument.
% First, it breaks loops.
% If the argument is not empty and does not match |\childdocname|
% (which is set by the first inclusion of |childdoc.def|),
% |\ifchilddoc| is set to true, |\includeonly| is applied to the child file
% and |\jobname| is set to the main file
% (for proper handling of |.aux| files):
%    \begin{macrocode}
\newcommand{\childdocmain}[1]
{
  \childdocdisable\childdocmain{}
  \if?#1?\else
    \begingroup
      \def\childdoctmp{#1}
      \ifx\childdoctmp\childdocname
        \def\childdoctmp{}
      \else
        \def\childdoctmp
        {
          \childdoctrue
          \includeonly{\childdocname}
          \def\childdocjob{#1}
          \def\jobname{#1}
        }
      \fi
      \expandafter
    \endgroup
    \childdoctmp
  \fi
}
%    \end{macrocode}

% \macro{\childdocof}
% The command |\childdocof| redirects
% compilation to the main file |#1|.
%    \begin{macrocode}
\newcommand{\childdocof}[1]
{
  \childdocdisable
  \childdoctrue
  \includeonly{\childdocname}
  \def\jobname{#1}
  \def\childdocjob{#1}
  \input{#1}
}
%    \end{macrocode}

% \macro{\childdocby}
% The command |\childdocby| ....
%    \begin{macrocode}
\newcommand{\childdocby}[2][]
{
  \childdocdisable
  \childdoctrue
  \childdocmanualtrue
  \if?#1?\else
    \def\jobname{#2}
  \fi
  \def\childdocjob{#2}
  \input{#2}
  \endinput
}
%    \end{macrocode}

% \macro{\childdocforward}
% The command |\childdocforward| redirects
% compilation to the main file or
% (if the optional argument is given) a child file.
% Parameters are set as if the main file
% or a child file starting with |\childdocof| was compiled.
% Then compilation is handed over to the main file:
%    \begin{macrocode}
\newcommand{\childdocforward}[2][]
{
  \begingroup
    \if?#1?
      \def\childdoctmp
      {
        \def\childdocname{#2}
        \def\childdocjob{#2}
        \def\jobname{#2}
        \input{#2}
        \endinput
      }
    \else
      \def\childdoctmp
      {
        \childdocdisable
        \def\childdocname{#2}
        \childdoctrue
        \includeonly{#2}
        \def\childdocjob{#1}
        \def\jobname{#1}
        \input{#1}
        \endinput
      }
    \fi
    \expandafter
  \endgroup
  \childdoctmp
}
%    \end{macrocode}

% \macro{\childdocforwardprefix}
% The command |\childdocforwardprefix| redirects
% compilation to the main or a child file by means of a pattern.
% The prefix |#1| in the current filename is replaced by |#2|
% and the suffix of the current filename is kept
% (it is assumed that the filename does not contain the substring `|~~~|'
% which is used as a delimiter).
% Compilation is handed over to the new file by |\childdocforward|:
%    \begin{macrocode}
\newcommand{\childdocforwardprefix}[3][]
{
  \begingroup
    \def\childdocextract #2##1~~~{\def\childdoctmp{\childdocforward[#1]{#3##1}}}
    \expandafter\childdocextract\childdocname~~~
    \expandafter
  \endgroup
  \childdoctmp
}
%    \end{macrocode}

% \macro{\childdoc}
% The deprecated macro |\childdoc| is a legacy version of |\childdocmain|:
%    \begin{macrocode}
\newcommand{\childdoc}{\childdocmain}
%    \end{macrocode}

% \macro{\childdocredirect}
% The deprecated macro |\childdocredirect| is a legacy version
% of |\childdocforward| and |\childdocforwardprefix|:
%    \begin{macrocode}
\newcommand{\childdocredirect}[2][]
{
  \begingroup
    \if?#1?
      \def\childdoctmp{\childdocforward{#2}}
    \else
      \def\childdoctmp{\childdocforwardprefix{#1}{#2}}
    \fi
    \expandafter
  \endgroup
  \childdoctmp
}
%    \end{macrocode}

%\iffalse
%</package>
%\fi
%
\endinput
|\\
|\childdocforward[|\textit{main}|]{|\textit{dest}|}|\\
\end{tabular}
\end{center}
%
The argument \textit{dest} is the destination file
(without extension).
It should be the main file or one of the child files.
Note that further \textsf{childdoc} directives
such as |\childdocof| and |\childdocforward|
in the indicated file will be processed in this form.
The optional argument \textit{main}
passes on directly to the main file \textit{main}
while pretending to compile the child \textit{dest}.
This form behaves as if \textit{dest}
issues |\childdocof{|\textit{main}|}| right away,
and no further \textsf{childdoc} directives will be processed.

%%%%%%%%%%%%%%%%%%%%%%%%%%%%%%%%%%%%%%%%
\DescribeMacro{\...prefix}
In the alternative form |\childdocforwardprefix|,
%
\begin{center}
\begin{tabular}{l}
|% \iffalse
%
% childdoc.dtx Copyright (C) 2017-2018 Niklas Beisert
%
% This work may be distributed and/or modified under the
% conditions of the LaTeX Project Public License, either version 1.3
% of this license or (at your option) any later version.
% The latest version of this license is in
%   http://www.latex-project.org/lppl.txt
% and version 1.3 or later is part of all distributions of LaTeX
% version 2005/12/01 or later.
%
% This work has the LPPL maintenance status `maintained'.
%
% The Current Maintainer of this work is Niklas Beisert.
%
% This work consists of the files childdoc.dtx and childdoc.ins
% and the derived files childdoc.def and cdocsamp.tex with
% cdocsch1.tex, cdocsch2.tex, cdocsdrf.tex, cdocsfn1.tex, cdocsfn2.tex.
%
%<package>\ifdefined\childdocmain\endinput\fi
%<package>\ProvidesFile{childdoc.def}[2018/12/30 v2.0 child document driver]
%<samplemain>\ProvidesFile{cdocsamp.tex}[2018/12/30 v2.0 sample for childdoc]
%<*driver>
%\ProvidesFile{childdoc.drv}[2018/12/30 v2.0 childdoc reference manual file]
\PassOptionsToClass{10pt,a4paper}{article}
\documentclass{ltxdoc}

\usepackage[margin=35mm]{geometry}
\usepackage{hyperref}
\usepackage{hyperxmp}
\usepackage[usenames]{color}

\hypersetup{colorlinks=true}
\hypersetup{pdfstartview=FitH}
\hypersetup{pdfpagemode=UseNone}
\hypersetup{pdfsource={}}
\hypersetup{pdflang={en-UK}}
\hypersetup{pdfcopyright={Copyright 2017-2018 Niklas Beisert.
  This work may be distributed and/or modified under the
  conditions of the LaTeX Project Public License, either version 1.3
  of this license or (at your option) any later version.}}
\hypersetup{pdflicenseurl={http://www.latex-project.org/lppl.txt}}
\hypersetup{pdfcontactaddress={ETH Zurich, ITP, HIT K,
  Wolfgang-Pauli-Strasse 27}}
\hypersetup{pdfcontactpostcode={8093}}
\hypersetup{pdfcontactcity={Zurich}}
\hypersetup{pdfcontactcountry={Switzerland}}
\hypersetup{pdfcontactemail={nbeisert@itp.phys.ethz.ch}}
\hypersetup{pdfcontacturl={http://people.phys.ethz.ch/\xmptilde nbeisert/}}

\newcommand{\secref}[1]{\hyperref[#1]{section \ref*{#1}}}

\parskip1ex
\parindent0pt
\let\olditemize\itemize
\def\itemize{\olditemize\parskip0pt}

\begin{document}

\title{The \textsf{childdoc} Package}
\hypersetup{pdftitle={The childdoc Package}}
\author{Niklas Beisert\\[2ex]
  Institut f\"ur Theoretische Physik\\
  Eidgen\"ossische Technische Hochschule Z\"urich\\
  Wolfgang-Pauli-Strasse 27, 8093 Z\"urich, Switzerland\\[1ex]
  \href{mailto:nbeisert@itp.phys.ethz.ch}
  {\texttt{nbeisert@itp.phys.ethz.ch}}}
\hypersetup{pdfauthor={Niklas Beisert}}
\hypersetup{pdfsubject={Manual for the LaTeX2e Package childdoc}}
\date{30 December 2018, \textsf{v2.0}}
\maketitle

\begin{abstract}\noindent
\textsf{childdoc} is a \LaTeXe{} package
that enables the direct compilation
of document sections included by |\include|
to individual files.
\end{abstract}

\begingroup
\parskip0ex
\tableofcontents
\endgroup

%%%%%%%%%%%%%%%%%%%%%%%%%%%%%%%%%%%%%%%%%%%%%%%%%%%%%%%%%%%%%%%%%%%%%%%%%%%%%%%%
%%%%%%%%%%%%%%%%%%%%%%%%%%%%%%%%%%%%%%%%%%%%%%%%%%%%%%%%%%%%%%%%%%%%%%%%%%%%%%%%
\section{Introduction}

\LaTeX{} provides a mechanism to structure a large document (such as a book)
into a main file and several child files (containing the chapters)
using the |\include| command.
This mechanism is beneficial for documents
which span hundreds of pages in order to
make the source file(s) more manageable.
Moreover, compilation can be restricted to
selected child files by means of the |\includeonly| command.
The latter feature can be used to reduce the compilation time while editing
(this was significantly more useful in the earlier days of \LaTeX{})
or to generate a smaller document which is easier to navigate.
Another application of |\includeonly| is to generate
documents consisting of selected parts of the complete document.

However, there are a few drawbacks of the plain |\include| mechanism:
\begin{itemize}
\item
The child files cannot be compiled on their own,
they can only be compiled via the main file.
A naive editing environment
(such as a text editor with an option
to have the current file processed by \LaTeX)
may require one to switch to the main file before compiling;
attempting to compile the child file produces errors.
\item
The main file must be modified (each time)
to adjust the |\includeonly| command
to the present needs. This easily leaves the main file in a messy state.
\item
The generated document will always carry the filename
of the main document. This is inconvenient if
several child files are to be compiled and
to be kept for distribution.
\end{itemize}

The present package provides a simple interface
to make child files individually compilable by \LaTeX{}.
Compiling a child file then has the same effect as compiling
the main file with an |\includeonly| command
to select the appropriate child.
Moreover the generated document will carry the name of the child
rather than the main file.
This resolves all three above issues.

This feature is meant to make the editing of books,
thesis documents and lecture notes somewhat more convenient.
However, the package can also be used efficiently for
composing a series of documents (such as exercise sheets)
which are typically distributed individually.
It then assists the author in generating the individual documents
(potentially in different versions)
as well as a document containing the collected series.
Another application is in developing style files
or other kinds of included material
where compilation of the style file could redirect
to a sample or test file.

%%%%%%%%%%%%%%%%%%%%%%%%%%%%%%%%%%%%%%%%%%%%%%%%%%%%%%%%%%%%%%%%%%%%%%%%%%%%%%%%
%%%%%%%%%%%%%%%%%%%%%%%%%%%%%%%%%%%%%%%%%%%%%%%%%%%%%%%%%%%%%%%%%%%%%%%%%%%%%%%%
\section{Usage}

First of all, the package \textsf{childdoc} is \emph{not} a standard
\LaTeXe{} |.sty| style file! Therefore it needs to be invoked in
a non-standard way.

%%%%%%%%%%%%%%%%%%%%%%%%%%%%%%%%%%%%%%%%%%%%%%%%%%%%%%%%%%%%%%%%%%%%%%%%%%%%%%%%
\subsection{Included Files}
\label{sec:include}

%%%%%%%%%%%%%%%%%%%%%%%%%%%%%%%%%%%%%%%%
\DescribeMacro{\childdocmain}
To use the package, add the commands
\begin{center}
\begin{tabular}{l}
|\input{childdoc.def}|\\
|\childdocmain{}|\\
\end{tabular}
\end{center}
at the very top of the main \LaTeX{} file,
in particular \emph{before} the |\documentclass| statement!
The argument of |\childdocmain| should be left empty
(but it must be present).

%%%%%%%%%%%%%%%%%%%%%%%%%%%%%%%%%%%%%%%%
\DescribeMacro{\childdocof}
Furthermore, add the commands
\begin{center}
\begin{tabular}{l}
|\input{childdoc.def}|\\
|\childdocof{|\textit{main}|}|\\
\end{tabular}
\end{center}
at the top of every child file \textit{child}
which is included by |\include{|\textit{child}|}|
from within the main file
(or at least for those files to be compiled individually).
The argument \textit{main} must be the filename of the main file.

There are a couple of
considerations in setting up the main and child documents:

%%%%%%%%%%%%%%%%%%%%%%%%%%%%%%%%%%%%%%%%
\paragraph{Restrictions.}

Please note the following restrictions:
\begin{itemize}
\item
|\childdocmain| must be called with one argument \textit{main}
to ensure compatibility with earlier version of the package.
It must either be empty (|\childdocmain{}|)
or precisely match the filename of the main file in which it is specified.
See \secref{sec:detection} for further information.
\item
The filename \textit{main} must be specified without the |.tex| extension.
\item
The filename \textit{main} is case sensitive
(even in case-insensitive file systems)
due to internal string comparison.
\item
The argument \textit{main} should be fully expanded, it cannot be a macro.
\item
Subdirectories and special characters should be avoided in filenames.
\item
The command |\childdocmain{|\textit{main}|}| must be followed by a whitespace.
It should not be followed immediately by another command
or by a comment mark `|%|'.
This is because the \TeX{} parser reads the token immediately following
the argument of |\childdocmain| and puts it
at the beginning of every child section;
however, a white\-space is ignored.
\end{itemize}

%%%%%%%%%%%%%%%%%%%%%%%%%%%%%%%%%%%%%%%%
\paragraph{Content of Main File.}

It is advisable to place all content in the child files included by |\include|.
Any output contained in the main file will appear in all child documents
unless suppressed manually;
it cannot be suppressed automatically by the |\includeonly| directive
and thus should normally be avoided.
A method to include some content in the main file
by means of conditional processing is described in \secref{sec:conditional}.

%%%%%%%%%%%%%%%%%%%%%%%%%%%%%%%%%%%%%%%%
\paragraph{Page Numbering.}

When only a part of the document is compiled,
the appropriate numbering of pages
(as well as other status parameters)
is determined from the |.aux| files.
The latter contain information from previous passes.
However this information needs to propagate through
all intermediate child documents.
Therefore the page numbering in child documents may well
be inconsistent until the complete document is compiled at least once.

A useful (if unconventional) way to always ensure a consistent
page numbering is to restart the numbering in each child document
and denote the pages by `\textit{child}|.|\textit{page}'
where \textit{child} represents the chapter/section number of the child file.
This can be achieved by the command
|\numberwithin{page}{|\textit{child}|}|
of the \textsf{amsmath} package
where \textit{child} can be |chapter| or |section|
depending on the chosen structuring.
Alternatively, one can modify the macro |\thepage| appropriately
and reset the counter |page| at the start of each child file.

%%%%%%%%%%%%%%%%%%%%%%%%%%%%%%%%%%%%%%%%%%%%%%%%%%%%%%%%%%%%%%%%%%%%%%%%%%%%%%%%
\subsection{Conditional Processing}
\label{sec:conditional}

The package provides a mechanism to compile different versions
of a document. To customise the versions further some conditional processing
can come in handy to distinguish which version is being compiled.
The package provides two macros to describe the compilation context:

%%%%%%%%%%%%%%%%%%%%%%%%%%%%%%%%%%%%%%%%
\DescribeMacro{\ifchilddoc}
The conditional |\ifchilddoc| distinguishes between the compilation of
child documents and the main document:
%
\begin{center}
|\ifchilddoc |\textit{child-code}| |[|\||else |\textit{main-code}]| \||fi|
\end{center}

%%%%%%%%%%%%%%%%%%%%%%%%%%%%%%%%%%%%%%%%
\DescribeMacro{\childdocname}
\DescribeMacro{\childdocjob}
The macro |\childdocname| contains the filename (without extension)
of the main or child file being processed.
Note that |\childdocjob| will always contain the name of the main file.

%%%%%%%%%%%%%%%%%%%%%%%%%%%%%%%%%%%%%%%%
\paragraph{Title Page.}

Conditional processing can be used to include a title or banner page
in the main document when proper precautions are taken.
Importantly, the code in the main file should ensure that the page counter
(as well as other status parameters which are stored in the |.aux| files)
takes the same value after the conditional processing.
Otherwise the page numbers may take divergent values
depending on which part is compiled.

For example, a title page could be declared by:
%
\begin{center}
\begin{tabular}{l}
|\ifchilddoc\||else|\\
|\addtocounter{page}{-1}|\\
\textit{code for title page}\\
|\newpage|\\
|\||fi|
\end{tabular}
\end{center}
%
A banner page for the child documents can be generated by:
%
\begin{center}
\begin{tabular}{l}
|\ifchilddoc|\\
|\addtocounter{page}{-1}|\\
\textit{code for banner page}\\
|\newpage|\\
|\||fi|
\end{tabular}
\end{center}
%
Here one could write a message such as:
\begin{center}
|This is the part \childdocname{} of \childdocjob{}.|
\end{center}

%%%%%%%%%%%%%%%%%%%%%%%%%%%%%%%%%%%%%%%%%%%%%%%%%%%%%%%%%%%%%%%%%%%%%%%%%%%%%%%%
\subsection{Flags}
\label{sec:flags}

The package makes it easy to generate different versions
of the main or child documents.
To this end compilation flags can be defined
and assigned different default values.
They will be particularly useful in conjunction
with the forwarding mechanism described in \secref{sec:forward}.

For example, it may be useful to have a flag |\version|
which can be set to |draft| or |final|.
The document source will contain some conditional code
depending on the value of |\version|.
Suppose further, the flag should default to |final| for the main file
and to |draft| for child files
which is a natural assignment for editing the document.
This is achieved by placing the following code
in the preamble of the main document
(below the |\childdocmain| directive):
%
\begin{center}
\begin{tabular}{l}
|\ifchilddoc|\\
|\providecommand{\version}{draft}|\\
|\||else|\\
|\providecommand{\version}{final}|\\
|\||fi|
\end{tabular}
\end{center}
%
The definition by |\providecommand| makes sure
that previous definitions are not overwritten.
Further statements |\providecommand{\version}{...}|
can thus be added before the above code to override it.

For the main file, one might add a line
(between |\childdocmain| and the above block)
%
\begin{center}
|%\ifchilddoc\||else\providecommand{\version}{draft}\||fi|
\end{center}
%
which can be uncommented to produce a draft version.
Likewise one can add a line to the very top of a child file
(above the |\childdocof{|\textit{main}|}| directive)
%
\begin{center}
|%\providecommand{\version}{final}|
\end{center}
%
which can be uncommented to produce the final version of this child document.

%%%%%%%%%%%%%%%%%%%%%%%%%%%%%%%%%%%%%%%%%%%%%%%%%%%%%%%%%%%%%%%%%%%%%%%%%%%%%%%%
\subsection{Forwarding}
\label{sec:forward}

Different versions of the main or child documents
using compilation flags as described in \secref{sec:flags}
can be (permanently) stored in different files
for convenient compilation, viewing and distribution.
To this end, the package defines a command
to pass on compilation to a different file:

%%%%%%%%%%%%%%%%%%%%%%%%%%%%%%%%%%%%%%%%
\DescribeMacro{\childdocforward}
The command |\childdocforward| redirects processing to
another source file:
%
\begin{center}
\begin{tabular}{l}
|\input{childdoc.def}|\\
|\childdocforward[|\textit{main}|]{|\textit{dest}|}|\\
\end{tabular}
\end{center}
%
The argument \textit{dest} is the destination file
(without extension).
It should be the main file or one of the child files.
Note that further \textsf{childdoc} directives
such as |\childdocof| and |\childdocforward|
in the indicated file will be processed in this form.
The optional argument \textit{main}
passes on directly to the main file \textit{main}
while pretending to compile the child \textit{dest}.
This form behaves as if \textit{dest}
issues |\childdocof{|\textit{main}|}| right away,
and no further \textsf{childdoc} directives will be processed.

%%%%%%%%%%%%%%%%%%%%%%%%%%%%%%%%%%%%%%%%
\DescribeMacro{\...prefix}
In the alternative form |\childdocforwardprefix|,
%
\begin{center}
\begin{tabular}{l}
|\input{childdoc.def}|\\
|\childdocforwardprefix[|\textit{main}|]{|\textit{prefix}|}{|\textit{dest}|}|
\end{tabular}
\end{center}
%
the destination file is determined by a pattern
depending on the current file:
To make this work, the current file must be called
`{\textit{prefix}\hspace{0.2em}\textit{suffix}}'
with \textit{prefix} matching precisely the argument.
Processing is then passed on to the file
`{\textit{dest}\hspace{0.2em}\textit{suffix}}'.
Surely, the same effect is achieved by
directly specifying the
argument `{\textit{dest}\hspace{0.2em}\textit{suffix}}'
in the first form.
However, that requires to set up a different file
for each child. With the alternative form of the command
all these files can have exactly the same content
which simplifies setting them up and maintaining them.

For example, the following file |draft.tex|
with a compilation flag |\version| as described in \secref{sec:flags}
compiles the main document as a draft:
%
\begin{center}
\begin{tabular}{l}
|\def\version{draft}|\\
|\input{childdoc.def}|\\
|\childdocforward{|\textit{main}|}|
\end{tabular}
\end{center}
%
Likewise, the following files |final|\textit{nn}|.tex|
compile the final version of the child document
|child|\textit{nn}|.tex|:
%
\begin{center}
\begin{tabular}{l}
|\def\version{final}|\\
|\input{childdoc.def}|\\
|\childdocforwardprefix{final}{child}|
\end{tabular}
\end{center}
%

Note that when several versions of a main file and/or of each child file
are to be generated, it may be convenient to set up a |Makefile| or
shell script to automatise the process.

%%%%%%%%%%%%%%%%%%%%%%%%%%%%%%%%%%%%%%%%%%%%%%%%%%%%%%%%%%%%%%%%%%%%%%%%%%%%%%%%
\subsection{Command Line Processing}
\label{sec:commandline}

The effect of redirection files can also be achieved by invoking
the \LaTeX{} compiler with a more elaborate command line.
Most conveniently this should be done as part
of a shell script or a |Makefile|.

When using \textsf{childdoc} in the main file, the following
command lines effectively perform a redirection
(note that depending on the shell being used,
backslashes may have to be doubled: `|\|' $\to$ `|\\|'):
%
\begin{center}
|... -jobname "|\textit{target}|" |\\|"|[\textit{flags}]%
|\input{childdoc.def}\childdocforward[|\textit{main}|]{|\textit{dest}|}"|
\end{center}
%
Here \textit{target} is the name of the output file,
\textit{main} is the name of the main file
and \textit{dest} is the name of the main or child file to be processed
(all filenames without extensions).
The optional argument \textit{main} can be omitted
if \textit{main} matches \textit{dest}.
Optionally, compilation \textit{flags} can be defined via |\def| commands.
This command line makes the \TeX{} engine believe
it is compiling the file \textit{target}
whose content is specified as the latter parameter.
The provided code then forwards the processing to
\textit{main} or \textit{dest} as described in \secref{sec:forward}.

%%%%%%%%%%%%%%%%%%%%%%%%%%%%%%%%%%%%%%%%%%%%%%%%%%%%%%%%%%%%%%%%%%%%%%%%%%%%%%%%
\subsection{Include by Input}
\label{sec:input}

Including child documents by |\include| has some restrictions by design.
Most notably, the content of a child document always occupies
its own set of pages; pages cannot be shared between child documents.
Usually, this behaviour makes perfect sense
because each child document contain an essential part of the document.
However, in some situations it may be desirable to compose
a document from a collection of parts
without having mandatory page breaks between then.
For this case, the package
provides a mechanism to include parts
by |\input| which can also be processed individually.
However, by construction this mechanism
requires manual handling of the content to be output.

%%%%%%%%%%%%%%%%%%%%%%%%%%%%%%%%%%%%%%%%
\DescribeMacro{\ifchilddocmanual}
The main file should be prepared as usual, see \secref{sec:include}.
However, the document body must make a distinction
between processing of an individual part and of the main document, e.g.:
%
\begin{center}
\begin{tabular}{l}
|\ifchilddocmanual|\\
|\input{\childdocname}|\\
|\||else|\\
\textit{document body with }|\input{|\textit{part}|}|\\
|\||fi|
\end{tabular}
\end{center}
%
The conditional |\ifchilddocmanual| is true whenever
a part to be included by |\input| is being compiled,
and the name of the part is stored in |\childdocname|.

%%%%%%%%%%%%%%%%%%%%%%%%%%%%%%%%%%%%%%%%
\DescribeMacro{\childdocby}
Each part to be included by |\input| should start with:
%
\begin{center}
\begin{tabular}{l}
|\input{childdoc.def}|\\
|\childdocby{|\textit{main}|}|\\
\end{tabular}
\end{center}
%
The directive |\childdocby| is similar to |\childdocof|
described in \secref{sec:include},
but the subsequent selection of content must be done manually.
To that end, both |\ifchilddoc| and |\ifchilddocmanual|
will be true upon processing of a part,
and the name of the part is stored in |\childdocname|.
Note that |\jobname| will be set to the filename of the current part
so that each part receives an individual |.aux| file
that does not interfere with the |.aux| file(s) of the main document.
This behaviour can be altered by the alternative form
|\childdocby[*]{|\textit{main}|}| (with a non-empty optional argument)
which uses the |.aux| file of the main document
by setting |\jobname| to \textit{main}.

%%%%%%%%%%%%%%%%%%%%%%%%%%%%%%%%%%%%%%%%%%%%%%%%%%%%%%%%%%%%%%%%%%%%%%%%%%%%%%%%
\subsection{Driver Development}
\label{sec:driver}

The \textsf{childdoc} mechanism can also be use for the development
of definition files such as \LaTeX{} styles or classes.
This case differs from the above setup with multiple parts
included by |\include| in that no |\includeonly| should be invoked.
This can be achieved by starting the include file
(before |\ProvidesPackage|) with:
%
\begin{center}
\begin{tabular}{l}
|\input{childdoc.def}|\\
|\childdocforward{|\textit{main}|}|\\
\end{tabular}
\end{center}
%
or alternatively with:
%
\begin{center}
\begin{tabular}{l}
|\input{childdoc.def}|\\
|\childdocby{|\textit{main}|}|\\
\end{tabular}
\end{center}
%
Both forms have slightly different effects as described above.
The main file is prepared as usual, see \secref{sec:include}.

%%%%%%%%%%%%%%%%%%%%%%%%%%%%%%%%%%%%%%%%%%%%%%%%%%%%%%%%%%%%%%%%%%%%%%%%%%%%%%%%
\subsection{Legacy Detection}
\label{sec:detection}

The directive |\childdocmain| in the main file can detect
whether the complete document or merely a child is to be compiled
even without using the directive |\childdocof|.
This method is deprecated because it is less robust
and there is no compelling reason to use it;
it is merely provided for backward compatibility
and it may be removed in future versions.

If the detection mechanism is to be used,
it is mandatory to correctly specify
the filename of the main file as the argument of |\childdocmain|:
%
\begin{center}
\begin{tabular}{l}
|\input{childdoc.def}|\\
|\childdocmain{|\textit{main}|}|\\
\end{tabular}
\end{center}
%
If |\jobname| does not match the argument \textit{main} of |\childdocmain|,
it is assumed that |\jobname| points to the child file to be compiled.
When using |\childdocmain| with the main file specified as argument,
it suffices to start a child file
with just |\input{|\textit{main}|}|
without loading of the package and using |\childdocof|.
If instead all processing is done
with the appropriate \textsf{childdoc} directives,
the argument of \textit{main} of |\childdocmain| can be empty.

An alternative version of the command line processing described
in \secref{sec:commandline} using the detection mechanism reads:
%
\begin{center}
|... -jobname "|\textit{target}|" "|[\textit{flags}]%
[|\def\jobname{|\textit{dest}|}|]|\input{|\textit{main}|}"|
\end{center}

%%%%%%%%%%%%%%%%%%%%%%%%%%%%%%%%%%%%%%%%%%%%%%%%%%%%%%%%%%%%%%%%%%%%%%%%%%%%%%%%
\subsection{Manual Code}
\label{sec:manual}

In case one cannot be certain whether the definitions file |childdoc.def|
is installed on the target \TeX{} distribution
and one prefers not to ship it,
it is conceivable to paste a few relevant commands into the sources.

To that end, drop all statements |\input{childdoc.def}|
and perform the replacements as outlined below.
Instead of |\childdocmain{|\textit{main}|}| add the following code
to the top of the main file:
%
\begin{center}
\begin{tabular}{l}
|\||ifdefined\childdocname\endinput\||fi\newif\ifchilddoc|\\
|\edef\childdocname{\scantokens\expandafter{\jobname\noexpand}}|\\
|\def\childdocmain{|\textit{main}|}\||ifx\childdocmain\childdocname\||else|\\
|\childdoctrue\includeonly{\childdocname}\let\jobname\childdocmain\||fi|\\
\end{tabular}
\end{center}
%
Instead of |\childdocof{|\textit{main}|}| just include the main file
at the top of each child file:
%
\begin{center}
|\input{|\textit{main}|}|
\end{center}
%
A simple redirection |\childdocforward{|\textit{dest}|}| is achieved by:
%
\begin{center}
|\def\jobname{|\textit{dest}|}\input{\jobname}|
\end{center}
%
The redirection with prefix
|\childdocforwardprefix[|\textit{prefix}|]{|\textit{dest}|}|
is accomplished by:
%
\begin{center}
\begin{tabular}{l}
|{\edef\jobname{\scantokens\expandafter{\jobname\noexpand}}|\\
|\def\redirectjob |\textit{prefix}|#1~~~{\gdef\jobname{|\textit{dest}|#1}}|\\
|\expandafter\redirectjob\jobname~~~}\input{\jobname}|
\end{tabular}
\end{center}

In an alternative approach,
child documents can be compiled by a specific command line
without additional code or specific definitions:
%
\begin{center}
|... -jobname "|\textit{target}|" "|[\textit{flags}]%
|\includeonly{|\textit{dest}|}\input{|\textit{main}|}"|
\end{center}
%

%%%%%%%%%%%%%%%%%%%%%%%%%%%%%%%%%%%%%%%%%%%%%%%%%%%%%%%%%%%%%%%%%%%%%%%%%%%%%%%%
%%%%%%%%%%%%%%%%%%%%%%%%%%%%%%%%%%%%%%%%%%%%%%%%%%%%%%%%%%%%%%%%%%%%%%%%%%%%%%%%
\section{Information}

%%%%%%%%%%%%%%%%%%%%%%%%%%%%%%%%%%%%%%%%%%%%%%%%%%%%%%%%%%%%%%%%%%%%%%%%%%%%%%%%
\subsection{Copyright}

Copyright \copyright{} 2017--2018 Niklas Beisert

This work may be distributed and/or modified under the
conditions of the \LaTeX{} Project Public License, either version 1.3
of this license or (at your option) any later version.
The latest version of this license is in
  \url{http://www.latex-project.org/lppl.txt}
and version 1.3 or later is part of all distributions of \LaTeX{}
version 2005/12/01 or later.

This work has the LPPL maintenance status `maintained'.

The Current Maintainer of this work is Niklas Beisert.

This work consists of the files |README.txt|, |childdoc.ins| and |childdoc.dtx|
as well as the derived files |childdoc.def|, |cdocsamp.tex|
with |cdocsch1.tex|, |cdocsch2.tex|, |cdocspt3.tex|, |cdocspt4.tex|,
|cdocsdrf.tex|, |cdocsfn1.tex|, |cdocsfn2.tex|
as well as |childdoc.pdf|.

%%%%%%%%%%%%%%%%%%%%%%%%%%%%%%%%%%%%%%%%%%%%%%%%%%%%%%%%%%%%%%%%%%%%%%%%%%%%%%%%
\subsection{Files and Installation}

The package consists of the files:
%
\begin{center}
\begin{tabular}{ll}
    |README.txt|   & readme file \\
    |childdoc.ins| & installation file \\
    |childdoc.dtx| & source file \\
    |childdoc.def| & definition file \\
    |cdocsamp.tex| & sample main file \\
    |cdocsch1.tex| & sample include file \\
    |cdocsch2.tex| & sample include file \\
    |cdocspt3.tex| & sample part file \\
    |cdocspt4.tex| & sample part file \\
    |cdocsdrf.tex| & sample redirection file \\
    |cdocsfn1.tex| & sample redirection file \\
    |cdocsfn2.tex| & sample redirection file \\
    |childdoc.pdf| & manual
\end{tabular}
\end{center}
%
The distribution consists of the files
|README.txt|, |childdoc.ins| and |childdoc.dtx|.
%
\begin{itemize}
\item
Run (pdf)\LaTeX{} on |childdoc.dtx|
to compile the manual |childdoc.pdf| (this file).
\item
Run \LaTeX{} on |childdoc.ins| to create the definitions file |childdoc.def|
and the sample |cdocsamp.tex| with include files
|cdocsch1.tex|, |cdocsch2.tex|, |cdocspt3.tex|, |cdocspt4.tex|,
|cdocsdrf.tex|, |cdocsfn1.tex|, |cdocsfn2.tex|.
Then copy the file |childdoc.def| to an appropriate directory of your \LaTeX{}
distribution, e.g.\ \textit{texmf-root}|/tex/latex/childdoc|.
\end{itemize}

%%%%%%%%%%%%%%%%%%%%%%%%%%%%%%%%%%%%%%%%%%%%%%%%%%%%%%%%%%%%%%%%%%%%%%%%%%%%%%%%
\subsection{Related CTAN Packages}

There are several other packages which offer a similar functionality:
%
\begin{itemize}
\item
The packages
\href{http://ctan.org/pkg/docmute}{\textsf{docmute}},
\href{http://ctan.org/pkg/includex}{\textsf{includex}} and
\href{http://ctan.org/pkg/standalone}{\textsf{standalone}}
provide commands to include only the document body of
a child file thus allowing both files to be compiled individually.
\item
The packages \href{http://ctan.org/pkg/subdocs}{\textsf{subdocs}}
and \href{http://ctan.org/pkg/subfiles}{\textsf{subfiles}}
provide structures in which the main and child documents can be
encapsulated and allowing them to be compiled individually.
The inclusion mechanism is different from the conventional |\include|.
\item
The package \href{http://ctan.org/pkg/combine}{\textsf{combine}}
is an elaborate solution to combine several documents into one.
\end{itemize}
%
See also the CTAN topic \href{http://ctan.org/topic/subdocs}{\textsf{subdocs}}
for further related packages.
The present package differs from the above solutions in that
a document structure constructed with the conventional |\include| mechanism
just needs two extra commands at the top of every file
such that all constituent files can be compiled individually.

%%%%%%%%%%%%%%%%%%%%%%%%%%%%%%%%%%%%%%%%%%%%%%%%%%%%%%%%%%%%%%%%%%%%%%%%%%%%%%%%
%\subsection{Feature Suggestions}
%
%The following is a list of features which may be useful for future
%versions of this package:
%%
%\begin{itemize}
%\item
%\ldots
%\end{itemize}

%%%%%%%%%%%%%%%%%%%%%%%%%%%%%%%%%%%%%%%%%%%%%%%%%%%%%%%%%%%%%%%%%%%%%%%%%%%%%%%%
\subsection{Revision History}

%%%%%%%%%%%%%%%%%%%%%%%%%%%%%%%%%%%%%%%%
\paragraph{v2.0:} 2018/12/30

\begin{itemize}
\item
immediate forward processing
\item
added |\childdocby| mechanism
\item
manual restructured
\end{itemize}

%%%%%%%%%%%%%%%%%%%%%%%%%%%%%%%%%%%%%%%%
\paragraph{v1.6:} 2018/01/17

\begin{itemize}
\item
application for development of include files
\item
corrections to manual
\end{itemize}

%%%%%%%%%%%%%%%%%%%%%%%%%%%%%%%%%%%%%%%%
\paragraph{v1.5:} 2017/05/21

\begin{itemize}
\item
more complete structuring introduced
\item
|\childdocof| introduced
\item
|\childdoc| renamed to |\childdocmain|
\item
|\childredirect| renamed to |\childdocforward| and |\childdocforwardprefix|
and functionality expanded
\end{itemize}

%%%%%%%%%%%%%%%%%%%%%%%%%%%%%%%%%%%%%%%%
\paragraph{v1.0:} 2017/04/27

\begin{itemize}
\item
manual and install package
\item
first version published on CTAN
\end{itemize}

%%%%%%%%%%%%%%%%%%%%%%%%%%%%%%%%%%%%%%%%
\paragraph{v0.6:} 2017/04/26

\begin{itemize}
\item
redirection mechanism added
\end{itemize}

%%%%%%%%%%%%%%%%%%%%%%%%%%%%%%%%%%%%%%%%
\paragraph{v0.5:} 2017/04/26

\begin{itemize}
\item
functionality in definition file
\end{itemize}


%%%%%%%%%%%%%%%%%%%%%%%%%%%%%%%%%%%%%%%%%%%%%%%%%%%%%%%%%%%%%%%%%%%%%%%%%%%%%%%%
%%%%%%%%%%%%%%%%%%%%%%%%%%%%%%%%%%%%%%%%%%%%%%%%%%%%%%%%%%%%%%%%%%%%%%%%%%%%%%%%
%%%%%%%%%%%%%%%%%%%%%%%%%%%%%%%%%%%%%%%%%%%%%%%%%%%%%%%%%%%%%%%%%%%%%%%%%%%%%%%%
\appendix

\settowidth\MacroIndent{\rmfamily\scriptsize 000\ }

 \DocInput{childdoc.dtx}

\end{document}
%</driver>
% \fi
%
% %%%%%%%%%%%%%%%%%%%%%%%%%%%%%%%%%%%%%%%%%%%%%%%%%%%%%%%%%%%%%%%%%%%%%%%%%%%%%%
% %%%%%%%%%%%%%%%%%%%%%%%%%%%%%%%%%%%%%%%%%%%%%%%%%%%%%%%%%%%%%%%%%%%%%%%%%%%%%%
% \section{Sample}
%\iffalse
%<*samplemain>
%\fi
%
% The following presents a sample document
% with two chapters, two parts, a title page,
% a compile flag as well as three forwarding files to set the flag.
% It consists of eight |.tex| files:
% \begin{center}
% \begin{tabular}{ll}
% |cdocsamp.tex|&main file\\
% |cdocsch1.tex|&include file for chapter 1\\
% |cdocsch2.tex|&include file for chapter 2\\
% |cdocspt3.tex|&include file for part 3\\
% |cdocspt4.tex|&include file for part 4\\
% |cdocsdrf.tex|&forwarding file for main file in draft mode\\
% |cdocsfi1.tex|&forwarding file for final version of chapter 1\\
% |cdocsfi2.tex|&forwarding file for final version of chapter 2\\
% \end{tabular}
% \end{center}
% Each of the eight files can be compiled directly by the \LaTeX{} compiler.
%
% %%%%%%%%%%%%%%%%%%%%%%%%%%%%%%%%%%%%%%
% \paragraph{Main File.}
%
% The main file is called |cdocsamp.tex|.
%
% Load the \textsf{childdoc} definitions and
% declare the filename for the main document:
%    \begin{macrocode}
\input{childdoc.def}
\childdocmain{}
%    \end{macrocode}

% Optional override for |\version| flag:
%    \begin{macrocode}
%%\ifchilddoc\else\providecommand{\version}{draft}\fi
%    \end{macrocode}

% Define the default values for the |\version| flag
% (|final| for the main file and |draft| for childs):
%    \begin{macrocode}
\ifchilddoc
\providecommand{\version}{draft}
\else
\providecommand{\version}{final}
\fi
%    \end{macrocode}

% Load the standard document class:
%    \begin{macrocode}
\documentclass[12pt]{article}
%    \end{macrocode}

% Start the document body:
%    \begin{macrocode}
\begin{document}
%    \end{macrocode}

% Declare a title page.
% Print title, part of document being processed and version flag:
%    \begin{macrocode}
\addtocounter{page}{-1}
\begin{center}
{\LARGE\bfseries{}childdoc example\par}
\vspace{1cm}
\ifchilddoc
\ifchilddocmanual part\else chapter\fi:
`\childdocname' of `\childdocjob'\par
\else
main document: `\childdocjob'\par
\fi
version: \version\par
\end{center}
\newpage
%    \end{macrocode}

% Manually include selected file,
% otherwise process as usual:
%    \begin{macrocode}
\ifchilddocmanual
\section*{part `\childdocname'}
\input{\childdocname}
\else
%    \end{macrocode}

% Include the two chapters:
%    \begin{macrocode}
\include{cdocsch1}
\include{cdocsch2}
%    \end{macrocode}

% Include the two parts unless only chapters should be displayed:
%    \begin{macrocode}
\ifchilddoc\else
\section{part three}
\input{cdocspt3}
\section{part four}
\input{cdocspt4}
\fi
%    \end{macrocode}

% Process as usual until here:
%    \begin{macrocode}
\fi
%    \end{macrocode}

% End of document body:
%    \begin{macrocode}
\end{document}
%    \end{macrocode}
%\iffalse
%</samplemain>
%\fi
%
% %%%%%%%%%%%%%%%%%%%%%%%%%%%%%%%%%%%%%%
% \paragraph{Chapter Include Files.}
%
% The include files are called |cdocsch1.tex| and |cdocsch2.tex|.
%
%\iffalse
%<*samplechap1|samplechap2>
%\fi

% Optional override for |\version| flag:
%    \begin{macrocode}
%%\providecommand{\version}{final}
%    \end{macrocode}

% Include the main document:
%    \begin{macrocode}
\input{childdoc.def}
\childdocof{cdocsamp}
%    \end{macrocode}

%\iffalse
%</samplechap1|samplechap2>
%\fi
%
%\iffalse
%<*samplechap1>
%\fi
% Some text for chapter 1:
%    \begin{macrocode}
\section{one}
some text in chapter one
%    \end{macrocode}

%\iffalse
%</samplechap1>
%\fi
% Some text for chapter 2:
%\iffalse
%<*samplechap2>
%\fi
%    \begin{macrocode}
\section{two}
more text in chapter two
%    \end{macrocode}

%\iffalse
%</samplechap2>
%\fi
%
% %%%%%%%%%%%%%%%%%%%%%%%%%%%%%%%%%%%%%%
% \paragraph{Part Include Files.}
%
% The include files are called |cdocspt3.tex| and |cdocspt4.tex|.
%
%\iffalse
%<*samplepart3|samplepart4>
%\fi

% Optional override for |\version| flag:
%    \begin{macrocode}
%%\providecommand{\version}{final}
%    \end{macrocode}

% Include the main document:
%    \begin{macrocode}
\input{childdoc.def}
\childdocby{cdocsamp}
%    \end{macrocode}

%\iffalse
%</samplepart3|samplepart4>
%\fi
%
%\iffalse
%<*samplepart3>
%\fi
% Some text for part 3:
%    \begin{macrocode}
some text in part three
%    \end{macrocode}

%\iffalse
%</samplepart3>
%\fi
% Some text for part 4:
%\iffalse
%<*samplepart4>
%\fi
%    \begin{macrocode}
more text in part four
%    \end{macrocode}

%\iffalse
%</samplepart4>
%\fi
%
% %%%%%%%%%%%%%%%%%%%%%%%%%%%%%%%%%%%%%%
% \paragraph{Forwarding for a Complete Draft.}
%
% The following forwarding file |cdocsdrf.tex|
% compiles the main document in draft mode:
%\iffalse
%<*sampledraft>
%\fi
%    \begin{macrocode}
\def\version{draft}
\input{childdoc.def}
\childdocforward{cdocsamp}
%    \end{macrocode}

%\iffalse
%</sampledraft>
%\fi
%
% %%%%%%%%%%%%%%%%%%%%%%%%%%%%%%%%%%%%%%
% \paragraph{Forwarding for Final Version of the Chapters.}
%
% The following forwarding files |cdocsfn1.tex| and |cdocsfn2.tex|
% (with identical content)
% compile the final versions of the child documents
% |cdocsch1.tex| and |cdocsch2.tex|, respectively:
%\iffalse
%<*samplefinal>
%\fi
%    \begin{macrocode}
\def\version{final}
\input{childdoc.def}
\childdocforwardprefix[cdocsamp]{cdocsfn}{cdocsch}
%    \end{macrocode}

%\iffalse
%</samplefinal>
%\fi
%
% %%%%%%%%%%%%%%%%%%%%%%%%%%%%%%%%%%%%%%
% \paragraph{Command Line Processing.}
%
% The following three command lines generate the output files
% |cdocscld|, |cdocscl1| and |cdocscl2|
% which should be identical to
% |cdocsdrf|, |cdocsch1| and |cdocsfn2|, respectively:
% \begin{center}
% \begin{tabular}{l}
% |latex -jobname cdocscld \|\\
% |  "\def\version{draft}\input{childdoc.def}\childdocforward{cdocsamp}"|\\
% |latex -jobname cdocscl1 \|\\
% |  "\input{childdoc.def}\childdocforward[cdocsamp]{cdocsch1}"|\\
% |latex -jobname cdocscl2 \|\\
% |  "\def\version{final}\input{childdoc.def}\childdocforward{cdocsch2}"|
% \end{tabular}
% \end{center}
% Note that the trailing backslash on each first line
% merely continues the input to the second line
% (for convenient cut ant paste).
% Furthermore, the command |latex| can be replaced by any
% of its alternative versions such as |pdflatex|.
%
% %%%%%%%%%%%%%%%%%%%%%%%%%%%%%%%%%%%%%%%%%%%%%%%%%%%%%%%%%%%%%%%%%%%%%%%%%%%%%%
% %%%%%%%%%%%%%%%%%%%%%%%%%%%%%%%%%%%%%%%%%%%%%%%%%%%%%%%%%%%%%%%%%%%%%%%%%%%%%%
% \section{Implementation}
%\iffalse
%<*package>
%\fi
%
% This section describes the definitions file |childdoc.def|.

% The definitions cannot be loaded using |\usepackage| or |\RequirePackage|
% which has a mechanism to prevent loading a style file more than once.
% When loading the definitions by means of |\input|
% multiple instances have to be prevented manually:
%\iffalse
%This code needs to be before the `\ProvidesFile' directive
%which is defined at the beginning of this file.
%Therefore it is also placed there and commented out here.
%</package>
%<*discard>
%\fi
%    \begin{macrocode}
\ifdefined\childdocmain\endinput\fi
%    \end{macrocode}
%\iffalse
%</discard>
%<*package>
%\fi
%
% \macro{\ifchilddoc}
% \macro{\ifchilddocmanual}
% The conditional |\ifchilddoc| tells whether a
% child (true) or main (false) document is being compiled.
% The conditional |\ifchilddocmanual| tells whether
% the |\includeonly| mechanism is used (false) or
% the selection of child files must be performed manually (true).
% The definitions initialise to false:
%    \begin{macrocode}
\newif\ifchilddoc
\newif\ifchilddocmanual
%    \end{macrocode}

% \macro{\childdocname}
% \macro{\childdocjob}
% The macro |\childdocname| stores the name of the main document
% to be compiled. The macro |\childdocjob| stores the name of
% the document on which the \LaTeX{} compiler was originally invoked.
% The content of |\jobname| cannot be compared
% to filenames specified in the source due to different catcodes.
% The following code rescans |\jobname|, stores the result
% in |\childdocname| and saves a copy in |\childdocjob|:
%    \begin{macrocode}
\edef\childdocname{\scantokens\expandafter{\jobname\noexpand}}
\let\childdocjob\childdocname
%    \end{macrocode}

% \macro{\childdocdisable}
% The macro |\childdocdisable| prevents the main file
% from being processed more than once.
% At this stage, the main document command |\childdocmain|
% is assumed to be called once again where it should do nothing.
% Any subsequent call to it should prevent
% a secondary processing of the main document
% It overwrites the forwarding commands
% |\childdocof| and |\childdocforward|
% with empty macros to prevent further inclusions of the main document:
%    \begin{macrocode}
\newcommand{\childdocdisable}
{
  \renewcommand{\childdocmain}[1]{\renewcommand{\childdocmain}[1]{\endinput}}
  \renewcommand{\childdocof}[1]{}
  \renewcommand{\childdocby}[2][]{}
  \renewcommand{\childdocforward}[2][]{}
  \renewcommand{\childdocdisable}{}
}
%    \end{macrocode}

% \macro{\childdocmain}
% The macro |\childdocmain| is to be called at the top of the main file
% with nothing or the main filename (without extension) as argument.
% First, it breaks loops.
% If the argument is not empty and does not match |\childdocname|
% (which is set by the first inclusion of |childdoc.def|),
% |\ifchilddoc| is set to true, |\includeonly| is applied to the child file
% and |\jobname| is set to the main file
% (for proper handling of |.aux| files):
%    \begin{macrocode}
\newcommand{\childdocmain}[1]
{
  \childdocdisable\childdocmain{}
  \if?#1?\else
    \begingroup
      \def\childdoctmp{#1}
      \ifx\childdoctmp\childdocname
        \def\childdoctmp{}
      \else
        \def\childdoctmp
        {
          \childdoctrue
          \includeonly{\childdocname}
          \def\childdocjob{#1}
          \def\jobname{#1}
        }
      \fi
      \expandafter
    \endgroup
    \childdoctmp
  \fi
}
%    \end{macrocode}

% \macro{\childdocof}
% The command |\childdocof| redirects
% compilation to the main file |#1|.
%    \begin{macrocode}
\newcommand{\childdocof}[1]
{
  \childdocdisable
  \childdoctrue
  \includeonly{\childdocname}
  \def\jobname{#1}
  \def\childdocjob{#1}
  \input{#1}
}
%    \end{macrocode}

% \macro{\childdocby}
% The command |\childdocby| ....
%    \begin{macrocode}
\newcommand{\childdocby}[2][]
{
  \childdocdisable
  \childdoctrue
  \childdocmanualtrue
  \if?#1?\else
    \def\jobname{#2}
  \fi
  \def\childdocjob{#2}
  \input{#2}
  \endinput
}
%    \end{macrocode}

% \macro{\childdocforward}
% The command |\childdocforward| redirects
% compilation to the main file or
% (if the optional argument is given) a child file.
% Parameters are set as if the main file
% or a child file starting with |\childdocof| was compiled.
% Then compilation is handed over to the main file:
%    \begin{macrocode}
\newcommand{\childdocforward}[2][]
{
  \begingroup
    \if?#1?
      \def\childdoctmp
      {
        \def\childdocname{#2}
        \def\childdocjob{#2}
        \def\jobname{#2}
        \input{#2}
        \endinput
      }
    \else
      \def\childdoctmp
      {
        \childdocdisable
        \def\childdocname{#2}
        \childdoctrue
        \includeonly{#2}
        \def\childdocjob{#1}
        \def\jobname{#1}
        \input{#1}
        \endinput
      }
    \fi
    \expandafter
  \endgroup
  \childdoctmp
}
%    \end{macrocode}

% \macro{\childdocforwardprefix}
% The command |\childdocforwardprefix| redirects
% compilation to the main or a child file by means of a pattern.
% The prefix |#1| in the current filename is replaced by |#2|
% and the suffix of the current filename is kept
% (it is assumed that the filename does not contain the substring `|~~~|'
% which is used as a delimiter).
% Compilation is handed over to the new file by |\childdocforward|:
%    \begin{macrocode}
\newcommand{\childdocforwardprefix}[3][]
{
  \begingroup
    \def\childdocextract #2##1~~~{\def\childdoctmp{\childdocforward[#1]{#3##1}}}
    \expandafter\childdocextract\childdocname~~~
    \expandafter
  \endgroup
  \childdoctmp
}
%    \end{macrocode}

% \macro{\childdoc}
% The deprecated macro |\childdoc| is a legacy version of |\childdocmain|:
%    \begin{macrocode}
\newcommand{\childdoc}{\childdocmain}
%    \end{macrocode}

% \macro{\childdocredirect}
% The deprecated macro |\childdocredirect| is a legacy version
% of |\childdocforward| and |\childdocforwardprefix|:
%    \begin{macrocode}
\newcommand{\childdocredirect}[2][]
{
  \begingroup
    \if?#1?
      \def\childdoctmp{\childdocforward{#2}}
    \else
      \def\childdoctmp{\childdocforwardprefix{#1}{#2}}
    \fi
    \expandafter
  \endgroup
  \childdoctmp
}
%    \end{macrocode}

%\iffalse
%</package>
%\fi
%
\endinput
|\\
|\childdocforwardprefix[|\textit{main}|]{|\textit{prefix}|}{|\textit{dest}|}|
\end{tabular}
\end{center}
%
the destination file is determined by a pattern
depending on the current file:
To make this work, the current file must be called
`{\textit{prefix}\hspace{0.2em}\textit{suffix}}'
with \textit{prefix} matching precisely the argument.
Processing is then passed on to the file
`{\textit{dest}\hspace{0.2em}\textit{suffix}}'.
Surely, the same effect is achieved by
directly specifying the
argument `{\textit{dest}\hspace{0.2em}\textit{suffix}}'
in the first form.
However, that requires to set up a different file
for each child. With the alternative form of the command
all these files can have exactly the same content
which simplifies setting them up and maintaining them.

For example, the following file |draft.tex|
with a compilation flag |\version| as described in \secref{sec:flags}
compiles the main document as a draft:
%
\begin{center}
\begin{tabular}{l}
|\def\version{draft}|\\
|% \iffalse
%
% childdoc.dtx Copyright (C) 2017-2018 Niklas Beisert
%
% This work may be distributed and/or modified under the
% conditions of the LaTeX Project Public License, either version 1.3
% of this license or (at your option) any later version.
% The latest version of this license is in
%   http://www.latex-project.org/lppl.txt
% and version 1.3 or later is part of all distributions of LaTeX
% version 2005/12/01 or later.
%
% This work has the LPPL maintenance status `maintained'.
%
% The Current Maintainer of this work is Niklas Beisert.
%
% This work consists of the files childdoc.dtx and childdoc.ins
% and the derived files childdoc.def and cdocsamp.tex with
% cdocsch1.tex, cdocsch2.tex, cdocsdrf.tex, cdocsfn1.tex, cdocsfn2.tex.
%
%<package>\ifdefined\childdocmain\endinput\fi
%<package>\ProvidesFile{childdoc.def}[2018/12/30 v2.0 child document driver]
%<samplemain>\ProvidesFile{cdocsamp.tex}[2018/12/30 v2.0 sample for childdoc]
%<*driver>
%\ProvidesFile{childdoc.drv}[2018/12/30 v2.0 childdoc reference manual file]
\PassOptionsToClass{10pt,a4paper}{article}
\documentclass{ltxdoc}

\usepackage[margin=35mm]{geometry}
\usepackage{hyperref}
\usepackage{hyperxmp}
\usepackage[usenames]{color}

\hypersetup{colorlinks=true}
\hypersetup{pdfstartview=FitH}
\hypersetup{pdfpagemode=UseNone}
\hypersetup{pdfsource={}}
\hypersetup{pdflang={en-UK}}
\hypersetup{pdfcopyright={Copyright 2017-2018 Niklas Beisert.
  This work may be distributed and/or modified under the
  conditions of the LaTeX Project Public License, either version 1.3
  of this license or (at your option) any later version.}}
\hypersetup{pdflicenseurl={http://www.latex-project.org/lppl.txt}}
\hypersetup{pdfcontactaddress={ETH Zurich, ITP, HIT K,
  Wolfgang-Pauli-Strasse 27}}
\hypersetup{pdfcontactpostcode={8093}}
\hypersetup{pdfcontactcity={Zurich}}
\hypersetup{pdfcontactcountry={Switzerland}}
\hypersetup{pdfcontactemail={nbeisert@itp.phys.ethz.ch}}
\hypersetup{pdfcontacturl={http://people.phys.ethz.ch/\xmptilde nbeisert/}}

\newcommand{\secref}[1]{\hyperref[#1]{section \ref*{#1}}}

\parskip1ex
\parindent0pt
\let\olditemize\itemize
\def\itemize{\olditemize\parskip0pt}

\begin{document}

\title{The \textsf{childdoc} Package}
\hypersetup{pdftitle={The childdoc Package}}
\author{Niklas Beisert\\[2ex]
  Institut f\"ur Theoretische Physik\\
  Eidgen\"ossische Technische Hochschule Z\"urich\\
  Wolfgang-Pauli-Strasse 27, 8093 Z\"urich, Switzerland\\[1ex]
  \href{mailto:nbeisert@itp.phys.ethz.ch}
  {\texttt{nbeisert@itp.phys.ethz.ch}}}
\hypersetup{pdfauthor={Niklas Beisert}}
\hypersetup{pdfsubject={Manual for the LaTeX2e Package childdoc}}
\date{30 December 2018, \textsf{v2.0}}
\maketitle

\begin{abstract}\noindent
\textsf{childdoc} is a \LaTeXe{} package
that enables the direct compilation
of document sections included by |\include|
to individual files.
\end{abstract}

\begingroup
\parskip0ex
\tableofcontents
\endgroup

%%%%%%%%%%%%%%%%%%%%%%%%%%%%%%%%%%%%%%%%%%%%%%%%%%%%%%%%%%%%%%%%%%%%%%%%%%%%%%%%
%%%%%%%%%%%%%%%%%%%%%%%%%%%%%%%%%%%%%%%%%%%%%%%%%%%%%%%%%%%%%%%%%%%%%%%%%%%%%%%%
\section{Introduction}

\LaTeX{} provides a mechanism to structure a large document (such as a book)
into a main file and several child files (containing the chapters)
using the |\include| command.
This mechanism is beneficial for documents
which span hundreds of pages in order to
make the source file(s) more manageable.
Moreover, compilation can be restricted to
selected child files by means of the |\includeonly| command.
The latter feature can be used to reduce the compilation time while editing
(this was significantly more useful in the earlier days of \LaTeX{})
or to generate a smaller document which is easier to navigate.
Another application of |\includeonly| is to generate
documents consisting of selected parts of the complete document.

However, there are a few drawbacks of the plain |\include| mechanism:
\begin{itemize}
\item
The child files cannot be compiled on their own,
they can only be compiled via the main file.
A naive editing environment
(such as a text editor with an option
to have the current file processed by \LaTeX)
may require one to switch to the main file before compiling;
attempting to compile the child file produces errors.
\item
The main file must be modified (each time)
to adjust the |\includeonly| command
to the present needs. This easily leaves the main file in a messy state.
\item
The generated document will always carry the filename
of the main document. This is inconvenient if
several child files are to be compiled and
to be kept for distribution.
\end{itemize}

The present package provides a simple interface
to make child files individually compilable by \LaTeX{}.
Compiling a child file then has the same effect as compiling
the main file with an |\includeonly| command
to select the appropriate child.
Moreover the generated document will carry the name of the child
rather than the main file.
This resolves all three above issues.

This feature is meant to make the editing of books,
thesis documents and lecture notes somewhat more convenient.
However, the package can also be used efficiently for
composing a series of documents (such as exercise sheets)
which are typically distributed individually.
It then assists the author in generating the individual documents
(potentially in different versions)
as well as a document containing the collected series.
Another application is in developing style files
or other kinds of included material
where compilation of the style file could redirect
to a sample or test file.

%%%%%%%%%%%%%%%%%%%%%%%%%%%%%%%%%%%%%%%%%%%%%%%%%%%%%%%%%%%%%%%%%%%%%%%%%%%%%%%%
%%%%%%%%%%%%%%%%%%%%%%%%%%%%%%%%%%%%%%%%%%%%%%%%%%%%%%%%%%%%%%%%%%%%%%%%%%%%%%%%
\section{Usage}

First of all, the package \textsf{childdoc} is \emph{not} a standard
\LaTeXe{} |.sty| style file! Therefore it needs to be invoked in
a non-standard way.

%%%%%%%%%%%%%%%%%%%%%%%%%%%%%%%%%%%%%%%%%%%%%%%%%%%%%%%%%%%%%%%%%%%%%%%%%%%%%%%%
\subsection{Included Files}
\label{sec:include}

%%%%%%%%%%%%%%%%%%%%%%%%%%%%%%%%%%%%%%%%
\DescribeMacro{\childdocmain}
To use the package, add the commands
\begin{center}
\begin{tabular}{l}
|\input{childdoc.def}|\\
|\childdocmain{}|\\
\end{tabular}
\end{center}
at the very top of the main \LaTeX{} file,
in particular \emph{before} the |\documentclass| statement!
The argument of |\childdocmain| should be left empty
(but it must be present).

%%%%%%%%%%%%%%%%%%%%%%%%%%%%%%%%%%%%%%%%
\DescribeMacro{\childdocof}
Furthermore, add the commands
\begin{center}
\begin{tabular}{l}
|\input{childdoc.def}|\\
|\childdocof{|\textit{main}|}|\\
\end{tabular}
\end{center}
at the top of every child file \textit{child}
which is included by |\include{|\textit{child}|}|
from within the main file
(or at least for those files to be compiled individually).
The argument \textit{main} must be the filename of the main file.

There are a couple of
considerations in setting up the main and child documents:

%%%%%%%%%%%%%%%%%%%%%%%%%%%%%%%%%%%%%%%%
\paragraph{Restrictions.}

Please note the following restrictions:
\begin{itemize}
\item
|\childdocmain| must be called with one argument \textit{main}
to ensure compatibility with earlier version of the package.
It must either be empty (|\childdocmain{}|)
or precisely match the filename of the main file in which it is specified.
See \secref{sec:detection} for further information.
\item
The filename \textit{main} must be specified without the |.tex| extension.
\item
The filename \textit{main} is case sensitive
(even in case-insensitive file systems)
due to internal string comparison.
\item
The argument \textit{main} should be fully expanded, it cannot be a macro.
\item
Subdirectories and special characters should be avoided in filenames.
\item
The command |\childdocmain{|\textit{main}|}| must be followed by a whitespace.
It should not be followed immediately by another command
or by a comment mark `|%|'.
This is because the \TeX{} parser reads the token immediately following
the argument of |\childdocmain| and puts it
at the beginning of every child section;
however, a white\-space is ignored.
\end{itemize}

%%%%%%%%%%%%%%%%%%%%%%%%%%%%%%%%%%%%%%%%
\paragraph{Content of Main File.}

It is advisable to place all content in the child files included by |\include|.
Any output contained in the main file will appear in all child documents
unless suppressed manually;
it cannot be suppressed automatically by the |\includeonly| directive
and thus should normally be avoided.
A method to include some content in the main file
by means of conditional processing is described in \secref{sec:conditional}.

%%%%%%%%%%%%%%%%%%%%%%%%%%%%%%%%%%%%%%%%
\paragraph{Page Numbering.}

When only a part of the document is compiled,
the appropriate numbering of pages
(as well as other status parameters)
is determined from the |.aux| files.
The latter contain information from previous passes.
However this information needs to propagate through
all intermediate child documents.
Therefore the page numbering in child documents may well
be inconsistent until the complete document is compiled at least once.

A useful (if unconventional) way to always ensure a consistent
page numbering is to restart the numbering in each child document
and denote the pages by `\textit{child}|.|\textit{page}'
where \textit{child} represents the chapter/section number of the child file.
This can be achieved by the command
|\numberwithin{page}{|\textit{child}|}|
of the \textsf{amsmath} package
where \textit{child} can be |chapter| or |section|
depending on the chosen structuring.
Alternatively, one can modify the macro |\thepage| appropriately
and reset the counter |page| at the start of each child file.

%%%%%%%%%%%%%%%%%%%%%%%%%%%%%%%%%%%%%%%%%%%%%%%%%%%%%%%%%%%%%%%%%%%%%%%%%%%%%%%%
\subsection{Conditional Processing}
\label{sec:conditional}

The package provides a mechanism to compile different versions
of a document. To customise the versions further some conditional processing
can come in handy to distinguish which version is being compiled.
The package provides two macros to describe the compilation context:

%%%%%%%%%%%%%%%%%%%%%%%%%%%%%%%%%%%%%%%%
\DescribeMacro{\ifchilddoc}
The conditional |\ifchilddoc| distinguishes between the compilation of
child documents and the main document:
%
\begin{center}
|\ifchilddoc |\textit{child-code}| |[|\||else |\textit{main-code}]| \||fi|
\end{center}

%%%%%%%%%%%%%%%%%%%%%%%%%%%%%%%%%%%%%%%%
\DescribeMacro{\childdocname}
\DescribeMacro{\childdocjob}
The macro |\childdocname| contains the filename (without extension)
of the main or child file being processed.
Note that |\childdocjob| will always contain the name of the main file.

%%%%%%%%%%%%%%%%%%%%%%%%%%%%%%%%%%%%%%%%
\paragraph{Title Page.}

Conditional processing can be used to include a title or banner page
in the main document when proper precautions are taken.
Importantly, the code in the main file should ensure that the page counter
(as well as other status parameters which are stored in the |.aux| files)
takes the same value after the conditional processing.
Otherwise the page numbers may take divergent values
depending on which part is compiled.

For example, a title page could be declared by:
%
\begin{center}
\begin{tabular}{l}
|\ifchilddoc\||else|\\
|\addtocounter{page}{-1}|\\
\textit{code for title page}\\
|\newpage|\\
|\||fi|
\end{tabular}
\end{center}
%
A banner page for the child documents can be generated by:
%
\begin{center}
\begin{tabular}{l}
|\ifchilddoc|\\
|\addtocounter{page}{-1}|\\
\textit{code for banner page}\\
|\newpage|\\
|\||fi|
\end{tabular}
\end{center}
%
Here one could write a message such as:
\begin{center}
|This is the part \childdocname{} of \childdocjob{}.|
\end{center}

%%%%%%%%%%%%%%%%%%%%%%%%%%%%%%%%%%%%%%%%%%%%%%%%%%%%%%%%%%%%%%%%%%%%%%%%%%%%%%%%
\subsection{Flags}
\label{sec:flags}

The package makes it easy to generate different versions
of the main or child documents.
To this end compilation flags can be defined
and assigned different default values.
They will be particularly useful in conjunction
with the forwarding mechanism described in \secref{sec:forward}.

For example, it may be useful to have a flag |\version|
which can be set to |draft| or |final|.
The document source will contain some conditional code
depending on the value of |\version|.
Suppose further, the flag should default to |final| for the main file
and to |draft| for child files
which is a natural assignment for editing the document.
This is achieved by placing the following code
in the preamble of the main document
(below the |\childdocmain| directive):
%
\begin{center}
\begin{tabular}{l}
|\ifchilddoc|\\
|\providecommand{\version}{draft}|\\
|\||else|\\
|\providecommand{\version}{final}|\\
|\||fi|
\end{tabular}
\end{center}
%
The definition by |\providecommand| makes sure
that previous definitions are not overwritten.
Further statements |\providecommand{\version}{...}|
can thus be added before the above code to override it.

For the main file, one might add a line
(between |\childdocmain| and the above block)
%
\begin{center}
|%\ifchilddoc\||else\providecommand{\version}{draft}\||fi|
\end{center}
%
which can be uncommented to produce a draft version.
Likewise one can add a line to the very top of a child file
(above the |\childdocof{|\textit{main}|}| directive)
%
\begin{center}
|%\providecommand{\version}{final}|
\end{center}
%
which can be uncommented to produce the final version of this child document.

%%%%%%%%%%%%%%%%%%%%%%%%%%%%%%%%%%%%%%%%%%%%%%%%%%%%%%%%%%%%%%%%%%%%%%%%%%%%%%%%
\subsection{Forwarding}
\label{sec:forward}

Different versions of the main or child documents
using compilation flags as described in \secref{sec:flags}
can be (permanently) stored in different files
for convenient compilation, viewing and distribution.
To this end, the package defines a command
to pass on compilation to a different file:

%%%%%%%%%%%%%%%%%%%%%%%%%%%%%%%%%%%%%%%%
\DescribeMacro{\childdocforward}
The command |\childdocforward| redirects processing to
another source file:
%
\begin{center}
\begin{tabular}{l}
|\input{childdoc.def}|\\
|\childdocforward[|\textit{main}|]{|\textit{dest}|}|\\
\end{tabular}
\end{center}
%
The argument \textit{dest} is the destination file
(without extension).
It should be the main file or one of the child files.
Note that further \textsf{childdoc} directives
such as |\childdocof| and |\childdocforward|
in the indicated file will be processed in this form.
The optional argument \textit{main}
passes on directly to the main file \textit{main}
while pretending to compile the child \textit{dest}.
This form behaves as if \textit{dest}
issues |\childdocof{|\textit{main}|}| right away,
and no further \textsf{childdoc} directives will be processed.

%%%%%%%%%%%%%%%%%%%%%%%%%%%%%%%%%%%%%%%%
\DescribeMacro{\...prefix}
In the alternative form |\childdocforwardprefix|,
%
\begin{center}
\begin{tabular}{l}
|\input{childdoc.def}|\\
|\childdocforwardprefix[|\textit{main}|]{|\textit{prefix}|}{|\textit{dest}|}|
\end{tabular}
\end{center}
%
the destination file is determined by a pattern
depending on the current file:
To make this work, the current file must be called
`{\textit{prefix}\hspace{0.2em}\textit{suffix}}'
with \textit{prefix} matching precisely the argument.
Processing is then passed on to the file
`{\textit{dest}\hspace{0.2em}\textit{suffix}}'.
Surely, the same effect is achieved by
directly specifying the
argument `{\textit{dest}\hspace{0.2em}\textit{suffix}}'
in the first form.
However, that requires to set up a different file
for each child. With the alternative form of the command
all these files can have exactly the same content
which simplifies setting them up and maintaining them.

For example, the following file |draft.tex|
with a compilation flag |\version| as described in \secref{sec:flags}
compiles the main document as a draft:
%
\begin{center}
\begin{tabular}{l}
|\def\version{draft}|\\
|\input{childdoc.def}|\\
|\childdocforward{|\textit{main}|}|
\end{tabular}
\end{center}
%
Likewise, the following files |final|\textit{nn}|.tex|
compile the final version of the child document
|child|\textit{nn}|.tex|:
%
\begin{center}
\begin{tabular}{l}
|\def\version{final}|\\
|\input{childdoc.def}|\\
|\childdocforwardprefix{final}{child}|
\end{tabular}
\end{center}
%

Note that when several versions of a main file and/or of each child file
are to be generated, it may be convenient to set up a |Makefile| or
shell script to automatise the process.

%%%%%%%%%%%%%%%%%%%%%%%%%%%%%%%%%%%%%%%%%%%%%%%%%%%%%%%%%%%%%%%%%%%%%%%%%%%%%%%%
\subsection{Command Line Processing}
\label{sec:commandline}

The effect of redirection files can also be achieved by invoking
the \LaTeX{} compiler with a more elaborate command line.
Most conveniently this should be done as part
of a shell script or a |Makefile|.

When using \textsf{childdoc} in the main file, the following
command lines effectively perform a redirection
(note that depending on the shell being used,
backslashes may have to be doubled: `|\|' $\to$ `|\\|'):
%
\begin{center}
|... -jobname "|\textit{target}|" |\\|"|[\textit{flags}]%
|\input{childdoc.def}\childdocforward[|\textit{main}|]{|\textit{dest}|}"|
\end{center}
%
Here \textit{target} is the name of the output file,
\textit{main} is the name of the main file
and \textit{dest} is the name of the main or child file to be processed
(all filenames without extensions).
The optional argument \textit{main} can be omitted
if \textit{main} matches \textit{dest}.
Optionally, compilation \textit{flags} can be defined via |\def| commands.
This command line makes the \TeX{} engine believe
it is compiling the file \textit{target}
whose content is specified as the latter parameter.
The provided code then forwards the processing to
\textit{main} or \textit{dest} as described in \secref{sec:forward}.

%%%%%%%%%%%%%%%%%%%%%%%%%%%%%%%%%%%%%%%%%%%%%%%%%%%%%%%%%%%%%%%%%%%%%%%%%%%%%%%%
\subsection{Include by Input}
\label{sec:input}

Including child documents by |\include| has some restrictions by design.
Most notably, the content of a child document always occupies
its own set of pages; pages cannot be shared between child documents.
Usually, this behaviour makes perfect sense
because each child document contain an essential part of the document.
However, in some situations it may be desirable to compose
a document from a collection of parts
without having mandatory page breaks between then.
For this case, the package
provides a mechanism to include parts
by |\input| which can also be processed individually.
However, by construction this mechanism
requires manual handling of the content to be output.

%%%%%%%%%%%%%%%%%%%%%%%%%%%%%%%%%%%%%%%%
\DescribeMacro{\ifchilddocmanual}
The main file should be prepared as usual, see \secref{sec:include}.
However, the document body must make a distinction
between processing of an individual part and of the main document, e.g.:
%
\begin{center}
\begin{tabular}{l}
|\ifchilddocmanual|\\
|\input{\childdocname}|\\
|\||else|\\
\textit{document body with }|\input{|\textit{part}|}|\\
|\||fi|
\end{tabular}
\end{center}
%
The conditional |\ifchilddocmanual| is true whenever
a part to be included by |\input| is being compiled,
and the name of the part is stored in |\childdocname|.

%%%%%%%%%%%%%%%%%%%%%%%%%%%%%%%%%%%%%%%%
\DescribeMacro{\childdocby}
Each part to be included by |\input| should start with:
%
\begin{center}
\begin{tabular}{l}
|\input{childdoc.def}|\\
|\childdocby{|\textit{main}|}|\\
\end{tabular}
\end{center}
%
The directive |\childdocby| is similar to |\childdocof|
described in \secref{sec:include},
but the subsequent selection of content must be done manually.
To that end, both |\ifchilddoc| and |\ifchilddocmanual|
will be true upon processing of a part,
and the name of the part is stored in |\childdocname|.
Note that |\jobname| will be set to the filename of the current part
so that each part receives an individual |.aux| file
that does not interfere with the |.aux| file(s) of the main document.
This behaviour can be altered by the alternative form
|\childdocby[*]{|\textit{main}|}| (with a non-empty optional argument)
which uses the |.aux| file of the main document
by setting |\jobname| to \textit{main}.

%%%%%%%%%%%%%%%%%%%%%%%%%%%%%%%%%%%%%%%%%%%%%%%%%%%%%%%%%%%%%%%%%%%%%%%%%%%%%%%%
\subsection{Driver Development}
\label{sec:driver}

The \textsf{childdoc} mechanism can also be use for the development
of definition files such as \LaTeX{} styles or classes.
This case differs from the above setup with multiple parts
included by |\include| in that no |\includeonly| should be invoked.
This can be achieved by starting the include file
(before |\ProvidesPackage|) with:
%
\begin{center}
\begin{tabular}{l}
|\input{childdoc.def}|\\
|\childdocforward{|\textit{main}|}|\\
\end{tabular}
\end{center}
%
or alternatively with:
%
\begin{center}
\begin{tabular}{l}
|\input{childdoc.def}|\\
|\childdocby{|\textit{main}|}|\\
\end{tabular}
\end{center}
%
Both forms have slightly different effects as described above.
The main file is prepared as usual, see \secref{sec:include}.

%%%%%%%%%%%%%%%%%%%%%%%%%%%%%%%%%%%%%%%%%%%%%%%%%%%%%%%%%%%%%%%%%%%%%%%%%%%%%%%%
\subsection{Legacy Detection}
\label{sec:detection}

The directive |\childdocmain| in the main file can detect
whether the complete document or merely a child is to be compiled
even without using the directive |\childdocof|.
This method is deprecated because it is less robust
and there is no compelling reason to use it;
it is merely provided for backward compatibility
and it may be removed in future versions.

If the detection mechanism is to be used,
it is mandatory to correctly specify
the filename of the main file as the argument of |\childdocmain|:
%
\begin{center}
\begin{tabular}{l}
|\input{childdoc.def}|\\
|\childdocmain{|\textit{main}|}|\\
\end{tabular}
\end{center}
%
If |\jobname| does not match the argument \textit{main} of |\childdocmain|,
it is assumed that |\jobname| points to the child file to be compiled.
When using |\childdocmain| with the main file specified as argument,
it suffices to start a child file
with just |\input{|\textit{main}|}|
without loading of the package and using |\childdocof|.
If instead all processing is done
with the appropriate \textsf{childdoc} directives,
the argument of \textit{main} of |\childdocmain| can be empty.

An alternative version of the command line processing described
in \secref{sec:commandline} using the detection mechanism reads:
%
\begin{center}
|... -jobname "|\textit{target}|" "|[\textit{flags}]%
[|\def\jobname{|\textit{dest}|}|]|\input{|\textit{main}|}"|
\end{center}

%%%%%%%%%%%%%%%%%%%%%%%%%%%%%%%%%%%%%%%%%%%%%%%%%%%%%%%%%%%%%%%%%%%%%%%%%%%%%%%%
\subsection{Manual Code}
\label{sec:manual}

In case one cannot be certain whether the definitions file |childdoc.def|
is installed on the target \TeX{} distribution
and one prefers not to ship it,
it is conceivable to paste a few relevant commands into the sources.

To that end, drop all statements |\input{childdoc.def}|
and perform the replacements as outlined below.
Instead of |\childdocmain{|\textit{main}|}| add the following code
to the top of the main file:
%
\begin{center}
\begin{tabular}{l}
|\||ifdefined\childdocname\endinput\||fi\newif\ifchilddoc|\\
|\edef\childdocname{\scantokens\expandafter{\jobname\noexpand}}|\\
|\def\childdocmain{|\textit{main}|}\||ifx\childdocmain\childdocname\||else|\\
|\childdoctrue\includeonly{\childdocname}\let\jobname\childdocmain\||fi|\\
\end{tabular}
\end{center}
%
Instead of |\childdocof{|\textit{main}|}| just include the main file
at the top of each child file:
%
\begin{center}
|\input{|\textit{main}|}|
\end{center}
%
A simple redirection |\childdocforward{|\textit{dest}|}| is achieved by:
%
\begin{center}
|\def\jobname{|\textit{dest}|}\input{\jobname}|
\end{center}
%
The redirection with prefix
|\childdocforwardprefix[|\textit{prefix}|]{|\textit{dest}|}|
is accomplished by:
%
\begin{center}
\begin{tabular}{l}
|{\edef\jobname{\scantokens\expandafter{\jobname\noexpand}}|\\
|\def\redirectjob |\textit{prefix}|#1~~~{\gdef\jobname{|\textit{dest}|#1}}|\\
|\expandafter\redirectjob\jobname~~~}\input{\jobname}|
\end{tabular}
\end{center}

In an alternative approach,
child documents can be compiled by a specific command line
without additional code or specific definitions:
%
\begin{center}
|... -jobname "|\textit{target}|" "|[\textit{flags}]%
|\includeonly{|\textit{dest}|}\input{|\textit{main}|}"|
\end{center}
%

%%%%%%%%%%%%%%%%%%%%%%%%%%%%%%%%%%%%%%%%%%%%%%%%%%%%%%%%%%%%%%%%%%%%%%%%%%%%%%%%
%%%%%%%%%%%%%%%%%%%%%%%%%%%%%%%%%%%%%%%%%%%%%%%%%%%%%%%%%%%%%%%%%%%%%%%%%%%%%%%%
\section{Information}

%%%%%%%%%%%%%%%%%%%%%%%%%%%%%%%%%%%%%%%%%%%%%%%%%%%%%%%%%%%%%%%%%%%%%%%%%%%%%%%%
\subsection{Copyright}

Copyright \copyright{} 2017--2018 Niklas Beisert

This work may be distributed and/or modified under the
conditions of the \LaTeX{} Project Public License, either version 1.3
of this license or (at your option) any later version.
The latest version of this license is in
  \url{http://www.latex-project.org/lppl.txt}
and version 1.3 or later is part of all distributions of \LaTeX{}
version 2005/12/01 or later.

This work has the LPPL maintenance status `maintained'.

The Current Maintainer of this work is Niklas Beisert.

This work consists of the files |README.txt|, |childdoc.ins| and |childdoc.dtx|
as well as the derived files |childdoc.def|, |cdocsamp.tex|
with |cdocsch1.tex|, |cdocsch2.tex|, |cdocspt3.tex|, |cdocspt4.tex|,
|cdocsdrf.tex|, |cdocsfn1.tex|, |cdocsfn2.tex|
as well as |childdoc.pdf|.

%%%%%%%%%%%%%%%%%%%%%%%%%%%%%%%%%%%%%%%%%%%%%%%%%%%%%%%%%%%%%%%%%%%%%%%%%%%%%%%%
\subsection{Files and Installation}

The package consists of the files:
%
\begin{center}
\begin{tabular}{ll}
    |README.txt|   & readme file \\
    |childdoc.ins| & installation file \\
    |childdoc.dtx| & source file \\
    |childdoc.def| & definition file \\
    |cdocsamp.tex| & sample main file \\
    |cdocsch1.tex| & sample include file \\
    |cdocsch2.tex| & sample include file \\
    |cdocspt3.tex| & sample part file \\
    |cdocspt4.tex| & sample part file \\
    |cdocsdrf.tex| & sample redirection file \\
    |cdocsfn1.tex| & sample redirection file \\
    |cdocsfn2.tex| & sample redirection file \\
    |childdoc.pdf| & manual
\end{tabular}
\end{center}
%
The distribution consists of the files
|README.txt|, |childdoc.ins| and |childdoc.dtx|.
%
\begin{itemize}
\item
Run (pdf)\LaTeX{} on |childdoc.dtx|
to compile the manual |childdoc.pdf| (this file).
\item
Run \LaTeX{} on |childdoc.ins| to create the definitions file |childdoc.def|
and the sample |cdocsamp.tex| with include files
|cdocsch1.tex|, |cdocsch2.tex|, |cdocspt3.tex|, |cdocspt4.tex|,
|cdocsdrf.tex|, |cdocsfn1.tex|, |cdocsfn2.tex|.
Then copy the file |childdoc.def| to an appropriate directory of your \LaTeX{}
distribution, e.g.\ \textit{texmf-root}|/tex/latex/childdoc|.
\end{itemize}

%%%%%%%%%%%%%%%%%%%%%%%%%%%%%%%%%%%%%%%%%%%%%%%%%%%%%%%%%%%%%%%%%%%%%%%%%%%%%%%%
\subsection{Related CTAN Packages}

There are several other packages which offer a similar functionality:
%
\begin{itemize}
\item
The packages
\href{http://ctan.org/pkg/docmute}{\textsf{docmute}},
\href{http://ctan.org/pkg/includex}{\textsf{includex}} and
\href{http://ctan.org/pkg/standalone}{\textsf{standalone}}
provide commands to include only the document body of
a child file thus allowing both files to be compiled individually.
\item
The packages \href{http://ctan.org/pkg/subdocs}{\textsf{subdocs}}
and \href{http://ctan.org/pkg/subfiles}{\textsf{subfiles}}
provide structures in which the main and child documents can be
encapsulated and allowing them to be compiled individually.
The inclusion mechanism is different from the conventional |\include|.
\item
The package \href{http://ctan.org/pkg/combine}{\textsf{combine}}
is an elaborate solution to combine several documents into one.
\end{itemize}
%
See also the CTAN topic \href{http://ctan.org/topic/subdocs}{\textsf{subdocs}}
for further related packages.
The present package differs from the above solutions in that
a document structure constructed with the conventional |\include| mechanism
just needs two extra commands at the top of every file
such that all constituent files can be compiled individually.

%%%%%%%%%%%%%%%%%%%%%%%%%%%%%%%%%%%%%%%%%%%%%%%%%%%%%%%%%%%%%%%%%%%%%%%%%%%%%%%%
%\subsection{Feature Suggestions}
%
%The following is a list of features which may be useful for future
%versions of this package:
%%
%\begin{itemize}
%\item
%\ldots
%\end{itemize}

%%%%%%%%%%%%%%%%%%%%%%%%%%%%%%%%%%%%%%%%%%%%%%%%%%%%%%%%%%%%%%%%%%%%%%%%%%%%%%%%
\subsection{Revision History}

%%%%%%%%%%%%%%%%%%%%%%%%%%%%%%%%%%%%%%%%
\paragraph{v2.0:} 2018/12/30

\begin{itemize}
\item
immediate forward processing
\item
added |\childdocby| mechanism
\item
manual restructured
\end{itemize}

%%%%%%%%%%%%%%%%%%%%%%%%%%%%%%%%%%%%%%%%
\paragraph{v1.6:} 2018/01/17

\begin{itemize}
\item
application for development of include files
\item
corrections to manual
\end{itemize}

%%%%%%%%%%%%%%%%%%%%%%%%%%%%%%%%%%%%%%%%
\paragraph{v1.5:} 2017/05/21

\begin{itemize}
\item
more complete structuring introduced
\item
|\childdocof| introduced
\item
|\childdoc| renamed to |\childdocmain|
\item
|\childredirect| renamed to |\childdocforward| and |\childdocforwardprefix|
and functionality expanded
\end{itemize}

%%%%%%%%%%%%%%%%%%%%%%%%%%%%%%%%%%%%%%%%
\paragraph{v1.0:} 2017/04/27

\begin{itemize}
\item
manual and install package
\item
first version published on CTAN
\end{itemize}

%%%%%%%%%%%%%%%%%%%%%%%%%%%%%%%%%%%%%%%%
\paragraph{v0.6:} 2017/04/26

\begin{itemize}
\item
redirection mechanism added
\end{itemize}

%%%%%%%%%%%%%%%%%%%%%%%%%%%%%%%%%%%%%%%%
\paragraph{v0.5:} 2017/04/26

\begin{itemize}
\item
functionality in definition file
\end{itemize}


%%%%%%%%%%%%%%%%%%%%%%%%%%%%%%%%%%%%%%%%%%%%%%%%%%%%%%%%%%%%%%%%%%%%%%%%%%%%%%%%
%%%%%%%%%%%%%%%%%%%%%%%%%%%%%%%%%%%%%%%%%%%%%%%%%%%%%%%%%%%%%%%%%%%%%%%%%%%%%%%%
%%%%%%%%%%%%%%%%%%%%%%%%%%%%%%%%%%%%%%%%%%%%%%%%%%%%%%%%%%%%%%%%%%%%%%%%%%%%%%%%
\appendix

\settowidth\MacroIndent{\rmfamily\scriptsize 000\ }

 \DocInput{childdoc.dtx}

\end{document}
%</driver>
% \fi
%
% %%%%%%%%%%%%%%%%%%%%%%%%%%%%%%%%%%%%%%%%%%%%%%%%%%%%%%%%%%%%%%%%%%%%%%%%%%%%%%
% %%%%%%%%%%%%%%%%%%%%%%%%%%%%%%%%%%%%%%%%%%%%%%%%%%%%%%%%%%%%%%%%%%%%%%%%%%%%%%
% \section{Sample}
%\iffalse
%<*samplemain>
%\fi
%
% The following presents a sample document
% with two chapters, two parts, a title page,
% a compile flag as well as three forwarding files to set the flag.
% It consists of eight |.tex| files:
% \begin{center}
% \begin{tabular}{ll}
% |cdocsamp.tex|&main file\\
% |cdocsch1.tex|&include file for chapter 1\\
% |cdocsch2.tex|&include file for chapter 2\\
% |cdocspt3.tex|&include file for part 3\\
% |cdocspt4.tex|&include file for part 4\\
% |cdocsdrf.tex|&forwarding file for main file in draft mode\\
% |cdocsfi1.tex|&forwarding file for final version of chapter 1\\
% |cdocsfi2.tex|&forwarding file for final version of chapter 2\\
% \end{tabular}
% \end{center}
% Each of the eight files can be compiled directly by the \LaTeX{} compiler.
%
% %%%%%%%%%%%%%%%%%%%%%%%%%%%%%%%%%%%%%%
% \paragraph{Main File.}
%
% The main file is called |cdocsamp.tex|.
%
% Load the \textsf{childdoc} definitions and
% declare the filename for the main document:
%    \begin{macrocode}
\input{childdoc.def}
\childdocmain{}
%    \end{macrocode}

% Optional override for |\version| flag:
%    \begin{macrocode}
%%\ifchilddoc\else\providecommand{\version}{draft}\fi
%    \end{macrocode}

% Define the default values for the |\version| flag
% (|final| for the main file and |draft| for childs):
%    \begin{macrocode}
\ifchilddoc
\providecommand{\version}{draft}
\else
\providecommand{\version}{final}
\fi
%    \end{macrocode}

% Load the standard document class:
%    \begin{macrocode}
\documentclass[12pt]{article}
%    \end{macrocode}

% Start the document body:
%    \begin{macrocode}
\begin{document}
%    \end{macrocode}

% Declare a title page.
% Print title, part of document being processed and version flag:
%    \begin{macrocode}
\addtocounter{page}{-1}
\begin{center}
{\LARGE\bfseries{}childdoc example\par}
\vspace{1cm}
\ifchilddoc
\ifchilddocmanual part\else chapter\fi:
`\childdocname' of `\childdocjob'\par
\else
main document: `\childdocjob'\par
\fi
version: \version\par
\end{center}
\newpage
%    \end{macrocode}

% Manually include selected file,
% otherwise process as usual:
%    \begin{macrocode}
\ifchilddocmanual
\section*{part `\childdocname'}
\input{\childdocname}
\else
%    \end{macrocode}

% Include the two chapters:
%    \begin{macrocode}
\include{cdocsch1}
\include{cdocsch2}
%    \end{macrocode}

% Include the two parts unless only chapters should be displayed:
%    \begin{macrocode}
\ifchilddoc\else
\section{part three}
\input{cdocspt3}
\section{part four}
\input{cdocspt4}
\fi
%    \end{macrocode}

% Process as usual until here:
%    \begin{macrocode}
\fi
%    \end{macrocode}

% End of document body:
%    \begin{macrocode}
\end{document}
%    \end{macrocode}
%\iffalse
%</samplemain>
%\fi
%
% %%%%%%%%%%%%%%%%%%%%%%%%%%%%%%%%%%%%%%
% \paragraph{Chapter Include Files.}
%
% The include files are called |cdocsch1.tex| and |cdocsch2.tex|.
%
%\iffalse
%<*samplechap1|samplechap2>
%\fi

% Optional override for |\version| flag:
%    \begin{macrocode}
%%\providecommand{\version}{final}
%    \end{macrocode}

% Include the main document:
%    \begin{macrocode}
\input{childdoc.def}
\childdocof{cdocsamp}
%    \end{macrocode}

%\iffalse
%</samplechap1|samplechap2>
%\fi
%
%\iffalse
%<*samplechap1>
%\fi
% Some text for chapter 1:
%    \begin{macrocode}
\section{one}
some text in chapter one
%    \end{macrocode}

%\iffalse
%</samplechap1>
%\fi
% Some text for chapter 2:
%\iffalse
%<*samplechap2>
%\fi
%    \begin{macrocode}
\section{two}
more text in chapter two
%    \end{macrocode}

%\iffalse
%</samplechap2>
%\fi
%
% %%%%%%%%%%%%%%%%%%%%%%%%%%%%%%%%%%%%%%
% \paragraph{Part Include Files.}
%
% The include files are called |cdocspt3.tex| and |cdocspt4.tex|.
%
%\iffalse
%<*samplepart3|samplepart4>
%\fi

% Optional override for |\version| flag:
%    \begin{macrocode}
%%\providecommand{\version}{final}
%    \end{macrocode}

% Include the main document:
%    \begin{macrocode}
\input{childdoc.def}
\childdocby{cdocsamp}
%    \end{macrocode}

%\iffalse
%</samplepart3|samplepart4>
%\fi
%
%\iffalse
%<*samplepart3>
%\fi
% Some text for part 3:
%    \begin{macrocode}
some text in part three
%    \end{macrocode}

%\iffalse
%</samplepart3>
%\fi
% Some text for part 4:
%\iffalse
%<*samplepart4>
%\fi
%    \begin{macrocode}
more text in part four
%    \end{macrocode}

%\iffalse
%</samplepart4>
%\fi
%
% %%%%%%%%%%%%%%%%%%%%%%%%%%%%%%%%%%%%%%
% \paragraph{Forwarding for a Complete Draft.}
%
% The following forwarding file |cdocsdrf.tex|
% compiles the main document in draft mode:
%\iffalse
%<*sampledraft>
%\fi
%    \begin{macrocode}
\def\version{draft}
\input{childdoc.def}
\childdocforward{cdocsamp}
%    \end{macrocode}

%\iffalse
%</sampledraft>
%\fi
%
% %%%%%%%%%%%%%%%%%%%%%%%%%%%%%%%%%%%%%%
% \paragraph{Forwarding for Final Version of the Chapters.}
%
% The following forwarding files |cdocsfn1.tex| and |cdocsfn2.tex|
% (with identical content)
% compile the final versions of the child documents
% |cdocsch1.tex| and |cdocsch2.tex|, respectively:
%\iffalse
%<*samplefinal>
%\fi
%    \begin{macrocode}
\def\version{final}
\input{childdoc.def}
\childdocforwardprefix[cdocsamp]{cdocsfn}{cdocsch}
%    \end{macrocode}

%\iffalse
%</samplefinal>
%\fi
%
% %%%%%%%%%%%%%%%%%%%%%%%%%%%%%%%%%%%%%%
% \paragraph{Command Line Processing.}
%
% The following three command lines generate the output files
% |cdocscld|, |cdocscl1| and |cdocscl2|
% which should be identical to
% |cdocsdrf|, |cdocsch1| and |cdocsfn2|, respectively:
% \begin{center}
% \begin{tabular}{l}
% |latex -jobname cdocscld \|\\
% |  "\def\version{draft}\input{childdoc.def}\childdocforward{cdocsamp}"|\\
% |latex -jobname cdocscl1 \|\\
% |  "\input{childdoc.def}\childdocforward[cdocsamp]{cdocsch1}"|\\
% |latex -jobname cdocscl2 \|\\
% |  "\def\version{final}\input{childdoc.def}\childdocforward{cdocsch2}"|
% \end{tabular}
% \end{center}
% Note that the trailing backslash on each first line
% merely continues the input to the second line
% (for convenient cut ant paste).
% Furthermore, the command |latex| can be replaced by any
% of its alternative versions such as |pdflatex|.
%
% %%%%%%%%%%%%%%%%%%%%%%%%%%%%%%%%%%%%%%%%%%%%%%%%%%%%%%%%%%%%%%%%%%%%%%%%%%%%%%
% %%%%%%%%%%%%%%%%%%%%%%%%%%%%%%%%%%%%%%%%%%%%%%%%%%%%%%%%%%%%%%%%%%%%%%%%%%%%%%
% \section{Implementation}
%\iffalse
%<*package>
%\fi
%
% This section describes the definitions file |childdoc.def|.

% The definitions cannot be loaded using |\usepackage| or |\RequirePackage|
% which has a mechanism to prevent loading a style file more than once.
% When loading the definitions by means of |\input|
% multiple instances have to be prevented manually:
%\iffalse
%This code needs to be before the `\ProvidesFile' directive
%which is defined at the beginning of this file.
%Therefore it is also placed there and commented out here.
%</package>
%<*discard>
%\fi
%    \begin{macrocode}
\ifdefined\childdocmain\endinput\fi
%    \end{macrocode}
%\iffalse
%</discard>
%<*package>
%\fi
%
% \macro{\ifchilddoc}
% \macro{\ifchilddocmanual}
% The conditional |\ifchilddoc| tells whether a
% child (true) or main (false) document is being compiled.
% The conditional |\ifchilddocmanual| tells whether
% the |\includeonly| mechanism is used (false) or
% the selection of child files must be performed manually (true).
% The definitions initialise to false:
%    \begin{macrocode}
\newif\ifchilddoc
\newif\ifchilddocmanual
%    \end{macrocode}

% \macro{\childdocname}
% \macro{\childdocjob}
% The macro |\childdocname| stores the name of the main document
% to be compiled. The macro |\childdocjob| stores the name of
% the document on which the \LaTeX{} compiler was originally invoked.
% The content of |\jobname| cannot be compared
% to filenames specified in the source due to different catcodes.
% The following code rescans |\jobname|, stores the result
% in |\childdocname| and saves a copy in |\childdocjob|:
%    \begin{macrocode}
\edef\childdocname{\scantokens\expandafter{\jobname\noexpand}}
\let\childdocjob\childdocname
%    \end{macrocode}

% \macro{\childdocdisable}
% The macro |\childdocdisable| prevents the main file
% from being processed more than once.
% At this stage, the main document command |\childdocmain|
% is assumed to be called once again where it should do nothing.
% Any subsequent call to it should prevent
% a secondary processing of the main document
% It overwrites the forwarding commands
% |\childdocof| and |\childdocforward|
% with empty macros to prevent further inclusions of the main document:
%    \begin{macrocode}
\newcommand{\childdocdisable}
{
  \renewcommand{\childdocmain}[1]{\renewcommand{\childdocmain}[1]{\endinput}}
  \renewcommand{\childdocof}[1]{}
  \renewcommand{\childdocby}[2][]{}
  \renewcommand{\childdocforward}[2][]{}
  \renewcommand{\childdocdisable}{}
}
%    \end{macrocode}

% \macro{\childdocmain}
% The macro |\childdocmain| is to be called at the top of the main file
% with nothing or the main filename (without extension) as argument.
% First, it breaks loops.
% If the argument is not empty and does not match |\childdocname|
% (which is set by the first inclusion of |childdoc.def|),
% |\ifchilddoc| is set to true, |\includeonly| is applied to the child file
% and |\jobname| is set to the main file
% (for proper handling of |.aux| files):
%    \begin{macrocode}
\newcommand{\childdocmain}[1]
{
  \childdocdisable\childdocmain{}
  \if?#1?\else
    \begingroup
      \def\childdoctmp{#1}
      \ifx\childdoctmp\childdocname
        \def\childdoctmp{}
      \else
        \def\childdoctmp
        {
          \childdoctrue
          \includeonly{\childdocname}
          \def\childdocjob{#1}
          \def\jobname{#1}
        }
      \fi
      \expandafter
    \endgroup
    \childdoctmp
  \fi
}
%    \end{macrocode}

% \macro{\childdocof}
% The command |\childdocof| redirects
% compilation to the main file |#1|.
%    \begin{macrocode}
\newcommand{\childdocof}[1]
{
  \childdocdisable
  \childdoctrue
  \includeonly{\childdocname}
  \def\jobname{#1}
  \def\childdocjob{#1}
  \input{#1}
}
%    \end{macrocode}

% \macro{\childdocby}
% The command |\childdocby| ....
%    \begin{macrocode}
\newcommand{\childdocby}[2][]
{
  \childdocdisable
  \childdoctrue
  \childdocmanualtrue
  \if?#1?\else
    \def\jobname{#2}
  \fi
  \def\childdocjob{#2}
  \input{#2}
  \endinput
}
%    \end{macrocode}

% \macro{\childdocforward}
% The command |\childdocforward| redirects
% compilation to the main file or
% (if the optional argument is given) a child file.
% Parameters are set as if the main file
% or a child file starting with |\childdocof| was compiled.
% Then compilation is handed over to the main file:
%    \begin{macrocode}
\newcommand{\childdocforward}[2][]
{
  \begingroup
    \if?#1?
      \def\childdoctmp
      {
        \def\childdocname{#2}
        \def\childdocjob{#2}
        \def\jobname{#2}
        \input{#2}
        \endinput
      }
    \else
      \def\childdoctmp
      {
        \childdocdisable
        \def\childdocname{#2}
        \childdoctrue
        \includeonly{#2}
        \def\childdocjob{#1}
        \def\jobname{#1}
        \input{#1}
        \endinput
      }
    \fi
    \expandafter
  \endgroup
  \childdoctmp
}
%    \end{macrocode}

% \macro{\childdocforwardprefix}
% The command |\childdocforwardprefix| redirects
% compilation to the main or a child file by means of a pattern.
% The prefix |#1| in the current filename is replaced by |#2|
% and the suffix of the current filename is kept
% (it is assumed that the filename does not contain the substring `|~~~|'
% which is used as a delimiter).
% Compilation is handed over to the new file by |\childdocforward|:
%    \begin{macrocode}
\newcommand{\childdocforwardprefix}[3][]
{
  \begingroup
    \def\childdocextract #2##1~~~{\def\childdoctmp{\childdocforward[#1]{#3##1}}}
    \expandafter\childdocextract\childdocname~~~
    \expandafter
  \endgroup
  \childdoctmp
}
%    \end{macrocode}

% \macro{\childdoc}
% The deprecated macro |\childdoc| is a legacy version of |\childdocmain|:
%    \begin{macrocode}
\newcommand{\childdoc}{\childdocmain}
%    \end{macrocode}

% \macro{\childdocredirect}
% The deprecated macro |\childdocredirect| is a legacy version
% of |\childdocforward| and |\childdocforwardprefix|:
%    \begin{macrocode}
\newcommand{\childdocredirect}[2][]
{
  \begingroup
    \if?#1?
      \def\childdoctmp{\childdocforward{#2}}
    \else
      \def\childdoctmp{\childdocforwardprefix{#1}{#2}}
    \fi
    \expandafter
  \endgroup
  \childdoctmp
}
%    \end{macrocode}

%\iffalse
%</package>
%\fi
%
\endinput
|\\
|\childdocforward{|\textit{main}|}|
\end{tabular}
\end{center}
%
Likewise, the following files |final|\textit{nn}|.tex|
compile the final version of the child document
|child|\textit{nn}|.tex|:
%
\begin{center}
\begin{tabular}{l}
|\def\version{final}|\\
|% \iffalse
%
% childdoc.dtx Copyright (C) 2017-2018 Niklas Beisert
%
% This work may be distributed and/or modified under the
% conditions of the LaTeX Project Public License, either version 1.3
% of this license or (at your option) any later version.
% The latest version of this license is in
%   http://www.latex-project.org/lppl.txt
% and version 1.3 or later is part of all distributions of LaTeX
% version 2005/12/01 or later.
%
% This work has the LPPL maintenance status `maintained'.
%
% The Current Maintainer of this work is Niklas Beisert.
%
% This work consists of the files childdoc.dtx and childdoc.ins
% and the derived files childdoc.def and cdocsamp.tex with
% cdocsch1.tex, cdocsch2.tex, cdocsdrf.tex, cdocsfn1.tex, cdocsfn2.tex.
%
%<package>\ifdefined\childdocmain\endinput\fi
%<package>\ProvidesFile{childdoc.def}[2018/12/30 v2.0 child document driver]
%<samplemain>\ProvidesFile{cdocsamp.tex}[2018/12/30 v2.0 sample for childdoc]
%<*driver>
%\ProvidesFile{childdoc.drv}[2018/12/30 v2.0 childdoc reference manual file]
\PassOptionsToClass{10pt,a4paper}{article}
\documentclass{ltxdoc}

\usepackage[margin=35mm]{geometry}
\usepackage{hyperref}
\usepackage{hyperxmp}
\usepackage[usenames]{color}

\hypersetup{colorlinks=true}
\hypersetup{pdfstartview=FitH}
\hypersetup{pdfpagemode=UseNone}
\hypersetup{pdfsource={}}
\hypersetup{pdflang={en-UK}}
\hypersetup{pdfcopyright={Copyright 2017-2018 Niklas Beisert.
  This work may be distributed and/or modified under the
  conditions of the LaTeX Project Public License, either version 1.3
  of this license or (at your option) any later version.}}
\hypersetup{pdflicenseurl={http://www.latex-project.org/lppl.txt}}
\hypersetup{pdfcontactaddress={ETH Zurich, ITP, HIT K,
  Wolfgang-Pauli-Strasse 27}}
\hypersetup{pdfcontactpostcode={8093}}
\hypersetup{pdfcontactcity={Zurich}}
\hypersetup{pdfcontactcountry={Switzerland}}
\hypersetup{pdfcontactemail={nbeisert@itp.phys.ethz.ch}}
\hypersetup{pdfcontacturl={http://people.phys.ethz.ch/\xmptilde nbeisert/}}

\newcommand{\secref}[1]{\hyperref[#1]{section \ref*{#1}}}

\parskip1ex
\parindent0pt
\let\olditemize\itemize
\def\itemize{\olditemize\parskip0pt}

\begin{document}

\title{The \textsf{childdoc} Package}
\hypersetup{pdftitle={The childdoc Package}}
\author{Niklas Beisert\\[2ex]
  Institut f\"ur Theoretische Physik\\
  Eidgen\"ossische Technische Hochschule Z\"urich\\
  Wolfgang-Pauli-Strasse 27, 8093 Z\"urich, Switzerland\\[1ex]
  \href{mailto:nbeisert@itp.phys.ethz.ch}
  {\texttt{nbeisert@itp.phys.ethz.ch}}}
\hypersetup{pdfauthor={Niklas Beisert}}
\hypersetup{pdfsubject={Manual for the LaTeX2e Package childdoc}}
\date{30 December 2018, \textsf{v2.0}}
\maketitle

\begin{abstract}\noindent
\textsf{childdoc} is a \LaTeXe{} package
that enables the direct compilation
of document sections included by |\include|
to individual files.
\end{abstract}

\begingroup
\parskip0ex
\tableofcontents
\endgroup

%%%%%%%%%%%%%%%%%%%%%%%%%%%%%%%%%%%%%%%%%%%%%%%%%%%%%%%%%%%%%%%%%%%%%%%%%%%%%%%%
%%%%%%%%%%%%%%%%%%%%%%%%%%%%%%%%%%%%%%%%%%%%%%%%%%%%%%%%%%%%%%%%%%%%%%%%%%%%%%%%
\section{Introduction}

\LaTeX{} provides a mechanism to structure a large document (such as a book)
into a main file and several child files (containing the chapters)
using the |\include| command.
This mechanism is beneficial for documents
which span hundreds of pages in order to
make the source file(s) more manageable.
Moreover, compilation can be restricted to
selected child files by means of the |\includeonly| command.
The latter feature can be used to reduce the compilation time while editing
(this was significantly more useful in the earlier days of \LaTeX{})
or to generate a smaller document which is easier to navigate.
Another application of |\includeonly| is to generate
documents consisting of selected parts of the complete document.

However, there are a few drawbacks of the plain |\include| mechanism:
\begin{itemize}
\item
The child files cannot be compiled on their own,
they can only be compiled via the main file.
A naive editing environment
(such as a text editor with an option
to have the current file processed by \LaTeX)
may require one to switch to the main file before compiling;
attempting to compile the child file produces errors.
\item
The main file must be modified (each time)
to adjust the |\includeonly| command
to the present needs. This easily leaves the main file in a messy state.
\item
The generated document will always carry the filename
of the main document. This is inconvenient if
several child files are to be compiled and
to be kept for distribution.
\end{itemize}

The present package provides a simple interface
to make child files individually compilable by \LaTeX{}.
Compiling a child file then has the same effect as compiling
the main file with an |\includeonly| command
to select the appropriate child.
Moreover the generated document will carry the name of the child
rather than the main file.
This resolves all three above issues.

This feature is meant to make the editing of books,
thesis documents and lecture notes somewhat more convenient.
However, the package can also be used efficiently for
composing a series of documents (such as exercise sheets)
which are typically distributed individually.
It then assists the author in generating the individual documents
(potentially in different versions)
as well as a document containing the collected series.
Another application is in developing style files
or other kinds of included material
where compilation of the style file could redirect
to a sample or test file.

%%%%%%%%%%%%%%%%%%%%%%%%%%%%%%%%%%%%%%%%%%%%%%%%%%%%%%%%%%%%%%%%%%%%%%%%%%%%%%%%
%%%%%%%%%%%%%%%%%%%%%%%%%%%%%%%%%%%%%%%%%%%%%%%%%%%%%%%%%%%%%%%%%%%%%%%%%%%%%%%%
\section{Usage}

First of all, the package \textsf{childdoc} is \emph{not} a standard
\LaTeXe{} |.sty| style file! Therefore it needs to be invoked in
a non-standard way.

%%%%%%%%%%%%%%%%%%%%%%%%%%%%%%%%%%%%%%%%%%%%%%%%%%%%%%%%%%%%%%%%%%%%%%%%%%%%%%%%
\subsection{Included Files}
\label{sec:include}

%%%%%%%%%%%%%%%%%%%%%%%%%%%%%%%%%%%%%%%%
\DescribeMacro{\childdocmain}
To use the package, add the commands
\begin{center}
\begin{tabular}{l}
|\input{childdoc.def}|\\
|\childdocmain{}|\\
\end{tabular}
\end{center}
at the very top of the main \LaTeX{} file,
in particular \emph{before} the |\documentclass| statement!
The argument of |\childdocmain| should be left empty
(but it must be present).

%%%%%%%%%%%%%%%%%%%%%%%%%%%%%%%%%%%%%%%%
\DescribeMacro{\childdocof}
Furthermore, add the commands
\begin{center}
\begin{tabular}{l}
|\input{childdoc.def}|\\
|\childdocof{|\textit{main}|}|\\
\end{tabular}
\end{center}
at the top of every child file \textit{child}
which is included by |\include{|\textit{child}|}|
from within the main file
(or at least for those files to be compiled individually).
The argument \textit{main} must be the filename of the main file.

There are a couple of
considerations in setting up the main and child documents:

%%%%%%%%%%%%%%%%%%%%%%%%%%%%%%%%%%%%%%%%
\paragraph{Restrictions.}

Please note the following restrictions:
\begin{itemize}
\item
|\childdocmain| must be called with one argument \textit{main}
to ensure compatibility with earlier version of the package.
It must either be empty (|\childdocmain{}|)
or precisely match the filename of the main file in which it is specified.
See \secref{sec:detection} for further information.
\item
The filename \textit{main} must be specified without the |.tex| extension.
\item
The filename \textit{main} is case sensitive
(even in case-insensitive file systems)
due to internal string comparison.
\item
The argument \textit{main} should be fully expanded, it cannot be a macro.
\item
Subdirectories and special characters should be avoided in filenames.
\item
The command |\childdocmain{|\textit{main}|}| must be followed by a whitespace.
It should not be followed immediately by another command
or by a comment mark `|%|'.
This is because the \TeX{} parser reads the token immediately following
the argument of |\childdocmain| and puts it
at the beginning of every child section;
however, a white\-space is ignored.
\end{itemize}

%%%%%%%%%%%%%%%%%%%%%%%%%%%%%%%%%%%%%%%%
\paragraph{Content of Main File.}

It is advisable to place all content in the child files included by |\include|.
Any output contained in the main file will appear in all child documents
unless suppressed manually;
it cannot be suppressed automatically by the |\includeonly| directive
and thus should normally be avoided.
A method to include some content in the main file
by means of conditional processing is described in \secref{sec:conditional}.

%%%%%%%%%%%%%%%%%%%%%%%%%%%%%%%%%%%%%%%%
\paragraph{Page Numbering.}

When only a part of the document is compiled,
the appropriate numbering of pages
(as well as other status parameters)
is determined from the |.aux| files.
The latter contain information from previous passes.
However this information needs to propagate through
all intermediate child documents.
Therefore the page numbering in child documents may well
be inconsistent until the complete document is compiled at least once.

A useful (if unconventional) way to always ensure a consistent
page numbering is to restart the numbering in each child document
and denote the pages by `\textit{child}|.|\textit{page}'
where \textit{child} represents the chapter/section number of the child file.
This can be achieved by the command
|\numberwithin{page}{|\textit{child}|}|
of the \textsf{amsmath} package
where \textit{child} can be |chapter| or |section|
depending on the chosen structuring.
Alternatively, one can modify the macro |\thepage| appropriately
and reset the counter |page| at the start of each child file.

%%%%%%%%%%%%%%%%%%%%%%%%%%%%%%%%%%%%%%%%%%%%%%%%%%%%%%%%%%%%%%%%%%%%%%%%%%%%%%%%
\subsection{Conditional Processing}
\label{sec:conditional}

The package provides a mechanism to compile different versions
of a document. To customise the versions further some conditional processing
can come in handy to distinguish which version is being compiled.
The package provides two macros to describe the compilation context:

%%%%%%%%%%%%%%%%%%%%%%%%%%%%%%%%%%%%%%%%
\DescribeMacro{\ifchilddoc}
The conditional |\ifchilddoc| distinguishes between the compilation of
child documents and the main document:
%
\begin{center}
|\ifchilddoc |\textit{child-code}| |[|\||else |\textit{main-code}]| \||fi|
\end{center}

%%%%%%%%%%%%%%%%%%%%%%%%%%%%%%%%%%%%%%%%
\DescribeMacro{\childdocname}
\DescribeMacro{\childdocjob}
The macro |\childdocname| contains the filename (without extension)
of the main or child file being processed.
Note that |\childdocjob| will always contain the name of the main file.

%%%%%%%%%%%%%%%%%%%%%%%%%%%%%%%%%%%%%%%%
\paragraph{Title Page.}

Conditional processing can be used to include a title or banner page
in the main document when proper precautions are taken.
Importantly, the code in the main file should ensure that the page counter
(as well as other status parameters which are stored in the |.aux| files)
takes the same value after the conditional processing.
Otherwise the page numbers may take divergent values
depending on which part is compiled.

For example, a title page could be declared by:
%
\begin{center}
\begin{tabular}{l}
|\ifchilddoc\||else|\\
|\addtocounter{page}{-1}|\\
\textit{code for title page}\\
|\newpage|\\
|\||fi|
\end{tabular}
\end{center}
%
A banner page for the child documents can be generated by:
%
\begin{center}
\begin{tabular}{l}
|\ifchilddoc|\\
|\addtocounter{page}{-1}|\\
\textit{code for banner page}\\
|\newpage|\\
|\||fi|
\end{tabular}
\end{center}
%
Here one could write a message such as:
\begin{center}
|This is the part \childdocname{} of \childdocjob{}.|
\end{center}

%%%%%%%%%%%%%%%%%%%%%%%%%%%%%%%%%%%%%%%%%%%%%%%%%%%%%%%%%%%%%%%%%%%%%%%%%%%%%%%%
\subsection{Flags}
\label{sec:flags}

The package makes it easy to generate different versions
of the main or child documents.
To this end compilation flags can be defined
and assigned different default values.
They will be particularly useful in conjunction
with the forwarding mechanism described in \secref{sec:forward}.

For example, it may be useful to have a flag |\version|
which can be set to |draft| or |final|.
The document source will contain some conditional code
depending on the value of |\version|.
Suppose further, the flag should default to |final| for the main file
and to |draft| for child files
which is a natural assignment for editing the document.
This is achieved by placing the following code
in the preamble of the main document
(below the |\childdocmain| directive):
%
\begin{center}
\begin{tabular}{l}
|\ifchilddoc|\\
|\providecommand{\version}{draft}|\\
|\||else|\\
|\providecommand{\version}{final}|\\
|\||fi|
\end{tabular}
\end{center}
%
The definition by |\providecommand| makes sure
that previous definitions are not overwritten.
Further statements |\providecommand{\version}{...}|
can thus be added before the above code to override it.

For the main file, one might add a line
(between |\childdocmain| and the above block)
%
\begin{center}
|%\ifchilddoc\||else\providecommand{\version}{draft}\||fi|
\end{center}
%
which can be uncommented to produce a draft version.
Likewise one can add a line to the very top of a child file
(above the |\childdocof{|\textit{main}|}| directive)
%
\begin{center}
|%\providecommand{\version}{final}|
\end{center}
%
which can be uncommented to produce the final version of this child document.

%%%%%%%%%%%%%%%%%%%%%%%%%%%%%%%%%%%%%%%%%%%%%%%%%%%%%%%%%%%%%%%%%%%%%%%%%%%%%%%%
\subsection{Forwarding}
\label{sec:forward}

Different versions of the main or child documents
using compilation flags as described in \secref{sec:flags}
can be (permanently) stored in different files
for convenient compilation, viewing and distribution.
To this end, the package defines a command
to pass on compilation to a different file:

%%%%%%%%%%%%%%%%%%%%%%%%%%%%%%%%%%%%%%%%
\DescribeMacro{\childdocforward}
The command |\childdocforward| redirects processing to
another source file:
%
\begin{center}
\begin{tabular}{l}
|\input{childdoc.def}|\\
|\childdocforward[|\textit{main}|]{|\textit{dest}|}|\\
\end{tabular}
\end{center}
%
The argument \textit{dest} is the destination file
(without extension).
It should be the main file or one of the child files.
Note that further \textsf{childdoc} directives
such as |\childdocof| and |\childdocforward|
in the indicated file will be processed in this form.
The optional argument \textit{main}
passes on directly to the main file \textit{main}
while pretending to compile the child \textit{dest}.
This form behaves as if \textit{dest}
issues |\childdocof{|\textit{main}|}| right away,
and no further \textsf{childdoc} directives will be processed.

%%%%%%%%%%%%%%%%%%%%%%%%%%%%%%%%%%%%%%%%
\DescribeMacro{\...prefix}
In the alternative form |\childdocforwardprefix|,
%
\begin{center}
\begin{tabular}{l}
|\input{childdoc.def}|\\
|\childdocforwardprefix[|\textit{main}|]{|\textit{prefix}|}{|\textit{dest}|}|
\end{tabular}
\end{center}
%
the destination file is determined by a pattern
depending on the current file:
To make this work, the current file must be called
`{\textit{prefix}\hspace{0.2em}\textit{suffix}}'
with \textit{prefix} matching precisely the argument.
Processing is then passed on to the file
`{\textit{dest}\hspace{0.2em}\textit{suffix}}'.
Surely, the same effect is achieved by
directly specifying the
argument `{\textit{dest}\hspace{0.2em}\textit{suffix}}'
in the first form.
However, that requires to set up a different file
for each child. With the alternative form of the command
all these files can have exactly the same content
which simplifies setting them up and maintaining them.

For example, the following file |draft.tex|
with a compilation flag |\version| as described in \secref{sec:flags}
compiles the main document as a draft:
%
\begin{center}
\begin{tabular}{l}
|\def\version{draft}|\\
|\input{childdoc.def}|\\
|\childdocforward{|\textit{main}|}|
\end{tabular}
\end{center}
%
Likewise, the following files |final|\textit{nn}|.tex|
compile the final version of the child document
|child|\textit{nn}|.tex|:
%
\begin{center}
\begin{tabular}{l}
|\def\version{final}|\\
|\input{childdoc.def}|\\
|\childdocforwardprefix{final}{child}|
\end{tabular}
\end{center}
%

Note that when several versions of a main file and/or of each child file
are to be generated, it may be convenient to set up a |Makefile| or
shell script to automatise the process.

%%%%%%%%%%%%%%%%%%%%%%%%%%%%%%%%%%%%%%%%%%%%%%%%%%%%%%%%%%%%%%%%%%%%%%%%%%%%%%%%
\subsection{Command Line Processing}
\label{sec:commandline}

The effect of redirection files can also be achieved by invoking
the \LaTeX{} compiler with a more elaborate command line.
Most conveniently this should be done as part
of a shell script or a |Makefile|.

When using \textsf{childdoc} in the main file, the following
command lines effectively perform a redirection
(note that depending on the shell being used,
backslashes may have to be doubled: `|\|' $\to$ `|\\|'):
%
\begin{center}
|... -jobname "|\textit{target}|" |\\|"|[\textit{flags}]%
|\input{childdoc.def}\childdocforward[|\textit{main}|]{|\textit{dest}|}"|
\end{center}
%
Here \textit{target} is the name of the output file,
\textit{main} is the name of the main file
and \textit{dest} is the name of the main or child file to be processed
(all filenames without extensions).
The optional argument \textit{main} can be omitted
if \textit{main} matches \textit{dest}.
Optionally, compilation \textit{flags} can be defined via |\def| commands.
This command line makes the \TeX{} engine believe
it is compiling the file \textit{target}
whose content is specified as the latter parameter.
The provided code then forwards the processing to
\textit{main} or \textit{dest} as described in \secref{sec:forward}.

%%%%%%%%%%%%%%%%%%%%%%%%%%%%%%%%%%%%%%%%%%%%%%%%%%%%%%%%%%%%%%%%%%%%%%%%%%%%%%%%
\subsection{Include by Input}
\label{sec:input}

Including child documents by |\include| has some restrictions by design.
Most notably, the content of a child document always occupies
its own set of pages; pages cannot be shared between child documents.
Usually, this behaviour makes perfect sense
because each child document contain an essential part of the document.
However, in some situations it may be desirable to compose
a document from a collection of parts
without having mandatory page breaks between then.
For this case, the package
provides a mechanism to include parts
by |\input| which can also be processed individually.
However, by construction this mechanism
requires manual handling of the content to be output.

%%%%%%%%%%%%%%%%%%%%%%%%%%%%%%%%%%%%%%%%
\DescribeMacro{\ifchilddocmanual}
The main file should be prepared as usual, see \secref{sec:include}.
However, the document body must make a distinction
between processing of an individual part and of the main document, e.g.:
%
\begin{center}
\begin{tabular}{l}
|\ifchilddocmanual|\\
|\input{\childdocname}|\\
|\||else|\\
\textit{document body with }|\input{|\textit{part}|}|\\
|\||fi|
\end{tabular}
\end{center}
%
The conditional |\ifchilddocmanual| is true whenever
a part to be included by |\input| is being compiled,
and the name of the part is stored in |\childdocname|.

%%%%%%%%%%%%%%%%%%%%%%%%%%%%%%%%%%%%%%%%
\DescribeMacro{\childdocby}
Each part to be included by |\input| should start with:
%
\begin{center}
\begin{tabular}{l}
|\input{childdoc.def}|\\
|\childdocby{|\textit{main}|}|\\
\end{tabular}
\end{center}
%
The directive |\childdocby| is similar to |\childdocof|
described in \secref{sec:include},
but the subsequent selection of content must be done manually.
To that end, both |\ifchilddoc| and |\ifchilddocmanual|
will be true upon processing of a part,
and the name of the part is stored in |\childdocname|.
Note that |\jobname| will be set to the filename of the current part
so that each part receives an individual |.aux| file
that does not interfere with the |.aux| file(s) of the main document.
This behaviour can be altered by the alternative form
|\childdocby[*]{|\textit{main}|}| (with a non-empty optional argument)
which uses the |.aux| file of the main document
by setting |\jobname| to \textit{main}.

%%%%%%%%%%%%%%%%%%%%%%%%%%%%%%%%%%%%%%%%%%%%%%%%%%%%%%%%%%%%%%%%%%%%%%%%%%%%%%%%
\subsection{Driver Development}
\label{sec:driver}

The \textsf{childdoc} mechanism can also be use for the development
of definition files such as \LaTeX{} styles or classes.
This case differs from the above setup with multiple parts
included by |\include| in that no |\includeonly| should be invoked.
This can be achieved by starting the include file
(before |\ProvidesPackage|) with:
%
\begin{center}
\begin{tabular}{l}
|\input{childdoc.def}|\\
|\childdocforward{|\textit{main}|}|\\
\end{tabular}
\end{center}
%
or alternatively with:
%
\begin{center}
\begin{tabular}{l}
|\input{childdoc.def}|\\
|\childdocby{|\textit{main}|}|\\
\end{tabular}
\end{center}
%
Both forms have slightly different effects as described above.
The main file is prepared as usual, see \secref{sec:include}.

%%%%%%%%%%%%%%%%%%%%%%%%%%%%%%%%%%%%%%%%%%%%%%%%%%%%%%%%%%%%%%%%%%%%%%%%%%%%%%%%
\subsection{Legacy Detection}
\label{sec:detection}

The directive |\childdocmain| in the main file can detect
whether the complete document or merely a child is to be compiled
even without using the directive |\childdocof|.
This method is deprecated because it is less robust
and there is no compelling reason to use it;
it is merely provided for backward compatibility
and it may be removed in future versions.

If the detection mechanism is to be used,
it is mandatory to correctly specify
the filename of the main file as the argument of |\childdocmain|:
%
\begin{center}
\begin{tabular}{l}
|\input{childdoc.def}|\\
|\childdocmain{|\textit{main}|}|\\
\end{tabular}
\end{center}
%
If |\jobname| does not match the argument \textit{main} of |\childdocmain|,
it is assumed that |\jobname| points to the child file to be compiled.
When using |\childdocmain| with the main file specified as argument,
it suffices to start a child file
with just |\input{|\textit{main}|}|
without loading of the package and using |\childdocof|.
If instead all processing is done
with the appropriate \textsf{childdoc} directives,
the argument of \textit{main} of |\childdocmain| can be empty.

An alternative version of the command line processing described
in \secref{sec:commandline} using the detection mechanism reads:
%
\begin{center}
|... -jobname "|\textit{target}|" "|[\textit{flags}]%
[|\def\jobname{|\textit{dest}|}|]|\input{|\textit{main}|}"|
\end{center}

%%%%%%%%%%%%%%%%%%%%%%%%%%%%%%%%%%%%%%%%%%%%%%%%%%%%%%%%%%%%%%%%%%%%%%%%%%%%%%%%
\subsection{Manual Code}
\label{sec:manual}

In case one cannot be certain whether the definitions file |childdoc.def|
is installed on the target \TeX{} distribution
and one prefers not to ship it,
it is conceivable to paste a few relevant commands into the sources.

To that end, drop all statements |\input{childdoc.def}|
and perform the replacements as outlined below.
Instead of |\childdocmain{|\textit{main}|}| add the following code
to the top of the main file:
%
\begin{center}
\begin{tabular}{l}
|\||ifdefined\childdocname\endinput\||fi\newif\ifchilddoc|\\
|\edef\childdocname{\scantokens\expandafter{\jobname\noexpand}}|\\
|\def\childdocmain{|\textit{main}|}\||ifx\childdocmain\childdocname\||else|\\
|\childdoctrue\includeonly{\childdocname}\let\jobname\childdocmain\||fi|\\
\end{tabular}
\end{center}
%
Instead of |\childdocof{|\textit{main}|}| just include the main file
at the top of each child file:
%
\begin{center}
|\input{|\textit{main}|}|
\end{center}
%
A simple redirection |\childdocforward{|\textit{dest}|}| is achieved by:
%
\begin{center}
|\def\jobname{|\textit{dest}|}\input{\jobname}|
\end{center}
%
The redirection with prefix
|\childdocforwardprefix[|\textit{prefix}|]{|\textit{dest}|}|
is accomplished by:
%
\begin{center}
\begin{tabular}{l}
|{\edef\jobname{\scantokens\expandafter{\jobname\noexpand}}|\\
|\def\redirectjob |\textit{prefix}|#1~~~{\gdef\jobname{|\textit{dest}|#1}}|\\
|\expandafter\redirectjob\jobname~~~}\input{\jobname}|
\end{tabular}
\end{center}

In an alternative approach,
child documents can be compiled by a specific command line
without additional code or specific definitions:
%
\begin{center}
|... -jobname "|\textit{target}|" "|[\textit{flags}]%
|\includeonly{|\textit{dest}|}\input{|\textit{main}|}"|
\end{center}
%

%%%%%%%%%%%%%%%%%%%%%%%%%%%%%%%%%%%%%%%%%%%%%%%%%%%%%%%%%%%%%%%%%%%%%%%%%%%%%%%%
%%%%%%%%%%%%%%%%%%%%%%%%%%%%%%%%%%%%%%%%%%%%%%%%%%%%%%%%%%%%%%%%%%%%%%%%%%%%%%%%
\section{Information}

%%%%%%%%%%%%%%%%%%%%%%%%%%%%%%%%%%%%%%%%%%%%%%%%%%%%%%%%%%%%%%%%%%%%%%%%%%%%%%%%
\subsection{Copyright}

Copyright \copyright{} 2017--2018 Niklas Beisert

This work may be distributed and/or modified under the
conditions of the \LaTeX{} Project Public License, either version 1.3
of this license or (at your option) any later version.
The latest version of this license is in
  \url{http://www.latex-project.org/lppl.txt}
and version 1.3 or later is part of all distributions of \LaTeX{}
version 2005/12/01 or later.

This work has the LPPL maintenance status `maintained'.

The Current Maintainer of this work is Niklas Beisert.

This work consists of the files |README.txt|, |childdoc.ins| and |childdoc.dtx|
as well as the derived files |childdoc.def|, |cdocsamp.tex|
with |cdocsch1.tex|, |cdocsch2.tex|, |cdocspt3.tex|, |cdocspt4.tex|,
|cdocsdrf.tex|, |cdocsfn1.tex|, |cdocsfn2.tex|
as well as |childdoc.pdf|.

%%%%%%%%%%%%%%%%%%%%%%%%%%%%%%%%%%%%%%%%%%%%%%%%%%%%%%%%%%%%%%%%%%%%%%%%%%%%%%%%
\subsection{Files and Installation}

The package consists of the files:
%
\begin{center}
\begin{tabular}{ll}
    |README.txt|   & readme file \\
    |childdoc.ins| & installation file \\
    |childdoc.dtx| & source file \\
    |childdoc.def| & definition file \\
    |cdocsamp.tex| & sample main file \\
    |cdocsch1.tex| & sample include file \\
    |cdocsch2.tex| & sample include file \\
    |cdocspt3.tex| & sample part file \\
    |cdocspt4.tex| & sample part file \\
    |cdocsdrf.tex| & sample redirection file \\
    |cdocsfn1.tex| & sample redirection file \\
    |cdocsfn2.tex| & sample redirection file \\
    |childdoc.pdf| & manual
\end{tabular}
\end{center}
%
The distribution consists of the files
|README.txt|, |childdoc.ins| and |childdoc.dtx|.
%
\begin{itemize}
\item
Run (pdf)\LaTeX{} on |childdoc.dtx|
to compile the manual |childdoc.pdf| (this file).
\item
Run \LaTeX{} on |childdoc.ins| to create the definitions file |childdoc.def|
and the sample |cdocsamp.tex| with include files
|cdocsch1.tex|, |cdocsch2.tex|, |cdocspt3.tex|, |cdocspt4.tex|,
|cdocsdrf.tex|, |cdocsfn1.tex|, |cdocsfn2.tex|.
Then copy the file |childdoc.def| to an appropriate directory of your \LaTeX{}
distribution, e.g.\ \textit{texmf-root}|/tex/latex/childdoc|.
\end{itemize}

%%%%%%%%%%%%%%%%%%%%%%%%%%%%%%%%%%%%%%%%%%%%%%%%%%%%%%%%%%%%%%%%%%%%%%%%%%%%%%%%
\subsection{Related CTAN Packages}

There are several other packages which offer a similar functionality:
%
\begin{itemize}
\item
The packages
\href{http://ctan.org/pkg/docmute}{\textsf{docmute}},
\href{http://ctan.org/pkg/includex}{\textsf{includex}} and
\href{http://ctan.org/pkg/standalone}{\textsf{standalone}}
provide commands to include only the document body of
a child file thus allowing both files to be compiled individually.
\item
The packages \href{http://ctan.org/pkg/subdocs}{\textsf{subdocs}}
and \href{http://ctan.org/pkg/subfiles}{\textsf{subfiles}}
provide structures in which the main and child documents can be
encapsulated and allowing them to be compiled individually.
The inclusion mechanism is different from the conventional |\include|.
\item
The package \href{http://ctan.org/pkg/combine}{\textsf{combine}}
is an elaborate solution to combine several documents into one.
\end{itemize}
%
See also the CTAN topic \href{http://ctan.org/topic/subdocs}{\textsf{subdocs}}
for further related packages.
The present package differs from the above solutions in that
a document structure constructed with the conventional |\include| mechanism
just needs two extra commands at the top of every file
such that all constituent files can be compiled individually.

%%%%%%%%%%%%%%%%%%%%%%%%%%%%%%%%%%%%%%%%%%%%%%%%%%%%%%%%%%%%%%%%%%%%%%%%%%%%%%%%
%\subsection{Feature Suggestions}
%
%The following is a list of features which may be useful for future
%versions of this package:
%%
%\begin{itemize}
%\item
%\ldots
%\end{itemize}

%%%%%%%%%%%%%%%%%%%%%%%%%%%%%%%%%%%%%%%%%%%%%%%%%%%%%%%%%%%%%%%%%%%%%%%%%%%%%%%%
\subsection{Revision History}

%%%%%%%%%%%%%%%%%%%%%%%%%%%%%%%%%%%%%%%%
\paragraph{v2.0:} 2018/12/30

\begin{itemize}
\item
immediate forward processing
\item
added |\childdocby| mechanism
\item
manual restructured
\end{itemize}

%%%%%%%%%%%%%%%%%%%%%%%%%%%%%%%%%%%%%%%%
\paragraph{v1.6:} 2018/01/17

\begin{itemize}
\item
application for development of include files
\item
corrections to manual
\end{itemize}

%%%%%%%%%%%%%%%%%%%%%%%%%%%%%%%%%%%%%%%%
\paragraph{v1.5:} 2017/05/21

\begin{itemize}
\item
more complete structuring introduced
\item
|\childdocof| introduced
\item
|\childdoc| renamed to |\childdocmain|
\item
|\childredirect| renamed to |\childdocforward| and |\childdocforwardprefix|
and functionality expanded
\end{itemize}

%%%%%%%%%%%%%%%%%%%%%%%%%%%%%%%%%%%%%%%%
\paragraph{v1.0:} 2017/04/27

\begin{itemize}
\item
manual and install package
\item
first version published on CTAN
\end{itemize}

%%%%%%%%%%%%%%%%%%%%%%%%%%%%%%%%%%%%%%%%
\paragraph{v0.6:} 2017/04/26

\begin{itemize}
\item
redirection mechanism added
\end{itemize}

%%%%%%%%%%%%%%%%%%%%%%%%%%%%%%%%%%%%%%%%
\paragraph{v0.5:} 2017/04/26

\begin{itemize}
\item
functionality in definition file
\end{itemize}


%%%%%%%%%%%%%%%%%%%%%%%%%%%%%%%%%%%%%%%%%%%%%%%%%%%%%%%%%%%%%%%%%%%%%%%%%%%%%%%%
%%%%%%%%%%%%%%%%%%%%%%%%%%%%%%%%%%%%%%%%%%%%%%%%%%%%%%%%%%%%%%%%%%%%%%%%%%%%%%%%
%%%%%%%%%%%%%%%%%%%%%%%%%%%%%%%%%%%%%%%%%%%%%%%%%%%%%%%%%%%%%%%%%%%%%%%%%%%%%%%%
\appendix

\settowidth\MacroIndent{\rmfamily\scriptsize 000\ }

 \DocInput{childdoc.dtx}

\end{document}
%</driver>
% \fi
%
% %%%%%%%%%%%%%%%%%%%%%%%%%%%%%%%%%%%%%%%%%%%%%%%%%%%%%%%%%%%%%%%%%%%%%%%%%%%%%%
% %%%%%%%%%%%%%%%%%%%%%%%%%%%%%%%%%%%%%%%%%%%%%%%%%%%%%%%%%%%%%%%%%%%%%%%%%%%%%%
% \section{Sample}
%\iffalse
%<*samplemain>
%\fi
%
% The following presents a sample document
% with two chapters, two parts, a title page,
% a compile flag as well as three forwarding files to set the flag.
% It consists of eight |.tex| files:
% \begin{center}
% \begin{tabular}{ll}
% |cdocsamp.tex|&main file\\
% |cdocsch1.tex|&include file for chapter 1\\
% |cdocsch2.tex|&include file for chapter 2\\
% |cdocspt3.tex|&include file for part 3\\
% |cdocspt4.tex|&include file for part 4\\
% |cdocsdrf.tex|&forwarding file for main file in draft mode\\
% |cdocsfi1.tex|&forwarding file for final version of chapter 1\\
% |cdocsfi2.tex|&forwarding file for final version of chapter 2\\
% \end{tabular}
% \end{center}
% Each of the eight files can be compiled directly by the \LaTeX{} compiler.
%
% %%%%%%%%%%%%%%%%%%%%%%%%%%%%%%%%%%%%%%
% \paragraph{Main File.}
%
% The main file is called |cdocsamp.tex|.
%
% Load the \textsf{childdoc} definitions and
% declare the filename for the main document:
%    \begin{macrocode}
\input{childdoc.def}
\childdocmain{}
%    \end{macrocode}

% Optional override for |\version| flag:
%    \begin{macrocode}
%%\ifchilddoc\else\providecommand{\version}{draft}\fi
%    \end{macrocode}

% Define the default values for the |\version| flag
% (|final| for the main file and |draft| for childs):
%    \begin{macrocode}
\ifchilddoc
\providecommand{\version}{draft}
\else
\providecommand{\version}{final}
\fi
%    \end{macrocode}

% Load the standard document class:
%    \begin{macrocode}
\documentclass[12pt]{article}
%    \end{macrocode}

% Start the document body:
%    \begin{macrocode}
\begin{document}
%    \end{macrocode}

% Declare a title page.
% Print title, part of document being processed and version flag:
%    \begin{macrocode}
\addtocounter{page}{-1}
\begin{center}
{\LARGE\bfseries{}childdoc example\par}
\vspace{1cm}
\ifchilddoc
\ifchilddocmanual part\else chapter\fi:
`\childdocname' of `\childdocjob'\par
\else
main document: `\childdocjob'\par
\fi
version: \version\par
\end{center}
\newpage
%    \end{macrocode}

% Manually include selected file,
% otherwise process as usual:
%    \begin{macrocode}
\ifchilddocmanual
\section*{part `\childdocname'}
\input{\childdocname}
\else
%    \end{macrocode}

% Include the two chapters:
%    \begin{macrocode}
\include{cdocsch1}
\include{cdocsch2}
%    \end{macrocode}

% Include the two parts unless only chapters should be displayed:
%    \begin{macrocode}
\ifchilddoc\else
\section{part three}
\input{cdocspt3}
\section{part four}
\input{cdocspt4}
\fi
%    \end{macrocode}

% Process as usual until here:
%    \begin{macrocode}
\fi
%    \end{macrocode}

% End of document body:
%    \begin{macrocode}
\end{document}
%    \end{macrocode}
%\iffalse
%</samplemain>
%\fi
%
% %%%%%%%%%%%%%%%%%%%%%%%%%%%%%%%%%%%%%%
% \paragraph{Chapter Include Files.}
%
% The include files are called |cdocsch1.tex| and |cdocsch2.tex|.
%
%\iffalse
%<*samplechap1|samplechap2>
%\fi

% Optional override for |\version| flag:
%    \begin{macrocode}
%%\providecommand{\version}{final}
%    \end{macrocode}

% Include the main document:
%    \begin{macrocode}
\input{childdoc.def}
\childdocof{cdocsamp}
%    \end{macrocode}

%\iffalse
%</samplechap1|samplechap2>
%\fi
%
%\iffalse
%<*samplechap1>
%\fi
% Some text for chapter 1:
%    \begin{macrocode}
\section{one}
some text in chapter one
%    \end{macrocode}

%\iffalse
%</samplechap1>
%\fi
% Some text for chapter 2:
%\iffalse
%<*samplechap2>
%\fi
%    \begin{macrocode}
\section{two}
more text in chapter two
%    \end{macrocode}

%\iffalse
%</samplechap2>
%\fi
%
% %%%%%%%%%%%%%%%%%%%%%%%%%%%%%%%%%%%%%%
% \paragraph{Part Include Files.}
%
% The include files are called |cdocspt3.tex| and |cdocspt4.tex|.
%
%\iffalse
%<*samplepart3|samplepart4>
%\fi

% Optional override for |\version| flag:
%    \begin{macrocode}
%%\providecommand{\version}{final}
%    \end{macrocode}

% Include the main document:
%    \begin{macrocode}
\input{childdoc.def}
\childdocby{cdocsamp}
%    \end{macrocode}

%\iffalse
%</samplepart3|samplepart4>
%\fi
%
%\iffalse
%<*samplepart3>
%\fi
% Some text for part 3:
%    \begin{macrocode}
some text in part three
%    \end{macrocode}

%\iffalse
%</samplepart3>
%\fi
% Some text for part 4:
%\iffalse
%<*samplepart4>
%\fi
%    \begin{macrocode}
more text in part four
%    \end{macrocode}

%\iffalse
%</samplepart4>
%\fi
%
% %%%%%%%%%%%%%%%%%%%%%%%%%%%%%%%%%%%%%%
% \paragraph{Forwarding for a Complete Draft.}
%
% The following forwarding file |cdocsdrf.tex|
% compiles the main document in draft mode:
%\iffalse
%<*sampledraft>
%\fi
%    \begin{macrocode}
\def\version{draft}
\input{childdoc.def}
\childdocforward{cdocsamp}
%    \end{macrocode}

%\iffalse
%</sampledraft>
%\fi
%
% %%%%%%%%%%%%%%%%%%%%%%%%%%%%%%%%%%%%%%
% \paragraph{Forwarding for Final Version of the Chapters.}
%
% The following forwarding files |cdocsfn1.tex| and |cdocsfn2.tex|
% (with identical content)
% compile the final versions of the child documents
% |cdocsch1.tex| and |cdocsch2.tex|, respectively:
%\iffalse
%<*samplefinal>
%\fi
%    \begin{macrocode}
\def\version{final}
\input{childdoc.def}
\childdocforwardprefix[cdocsamp]{cdocsfn}{cdocsch}
%    \end{macrocode}

%\iffalse
%</samplefinal>
%\fi
%
% %%%%%%%%%%%%%%%%%%%%%%%%%%%%%%%%%%%%%%
% \paragraph{Command Line Processing.}
%
% The following three command lines generate the output files
% |cdocscld|, |cdocscl1| and |cdocscl2|
% which should be identical to
% |cdocsdrf|, |cdocsch1| and |cdocsfn2|, respectively:
% \begin{center}
% \begin{tabular}{l}
% |latex -jobname cdocscld \|\\
% |  "\def\version{draft}\input{childdoc.def}\childdocforward{cdocsamp}"|\\
% |latex -jobname cdocscl1 \|\\
% |  "\input{childdoc.def}\childdocforward[cdocsamp]{cdocsch1}"|\\
% |latex -jobname cdocscl2 \|\\
% |  "\def\version{final}\input{childdoc.def}\childdocforward{cdocsch2}"|
% \end{tabular}
% \end{center}
% Note that the trailing backslash on each first line
% merely continues the input to the second line
% (for convenient cut ant paste).
% Furthermore, the command |latex| can be replaced by any
% of its alternative versions such as |pdflatex|.
%
% %%%%%%%%%%%%%%%%%%%%%%%%%%%%%%%%%%%%%%%%%%%%%%%%%%%%%%%%%%%%%%%%%%%%%%%%%%%%%%
% %%%%%%%%%%%%%%%%%%%%%%%%%%%%%%%%%%%%%%%%%%%%%%%%%%%%%%%%%%%%%%%%%%%%%%%%%%%%%%
% \section{Implementation}
%\iffalse
%<*package>
%\fi
%
% This section describes the definitions file |childdoc.def|.

% The definitions cannot be loaded using |\usepackage| or |\RequirePackage|
% which has a mechanism to prevent loading a style file more than once.
% When loading the definitions by means of |\input|
% multiple instances have to be prevented manually:
%\iffalse
%This code needs to be before the `\ProvidesFile' directive
%which is defined at the beginning of this file.
%Therefore it is also placed there and commented out here.
%</package>
%<*discard>
%\fi
%    \begin{macrocode}
\ifdefined\childdocmain\endinput\fi
%    \end{macrocode}
%\iffalse
%</discard>
%<*package>
%\fi
%
% \macro{\ifchilddoc}
% \macro{\ifchilddocmanual}
% The conditional |\ifchilddoc| tells whether a
% child (true) or main (false) document is being compiled.
% The conditional |\ifchilddocmanual| tells whether
% the |\includeonly| mechanism is used (false) or
% the selection of child files must be performed manually (true).
% The definitions initialise to false:
%    \begin{macrocode}
\newif\ifchilddoc
\newif\ifchilddocmanual
%    \end{macrocode}

% \macro{\childdocname}
% \macro{\childdocjob}
% The macro |\childdocname| stores the name of the main document
% to be compiled. The macro |\childdocjob| stores the name of
% the document on which the \LaTeX{} compiler was originally invoked.
% The content of |\jobname| cannot be compared
% to filenames specified in the source due to different catcodes.
% The following code rescans |\jobname|, stores the result
% in |\childdocname| and saves a copy in |\childdocjob|:
%    \begin{macrocode}
\edef\childdocname{\scantokens\expandafter{\jobname\noexpand}}
\let\childdocjob\childdocname
%    \end{macrocode}

% \macro{\childdocdisable}
% The macro |\childdocdisable| prevents the main file
% from being processed more than once.
% At this stage, the main document command |\childdocmain|
% is assumed to be called once again where it should do nothing.
% Any subsequent call to it should prevent
% a secondary processing of the main document
% It overwrites the forwarding commands
% |\childdocof| and |\childdocforward|
% with empty macros to prevent further inclusions of the main document:
%    \begin{macrocode}
\newcommand{\childdocdisable}
{
  \renewcommand{\childdocmain}[1]{\renewcommand{\childdocmain}[1]{\endinput}}
  \renewcommand{\childdocof}[1]{}
  \renewcommand{\childdocby}[2][]{}
  \renewcommand{\childdocforward}[2][]{}
  \renewcommand{\childdocdisable}{}
}
%    \end{macrocode}

% \macro{\childdocmain}
% The macro |\childdocmain| is to be called at the top of the main file
% with nothing or the main filename (without extension) as argument.
% First, it breaks loops.
% If the argument is not empty and does not match |\childdocname|
% (which is set by the first inclusion of |childdoc.def|),
% |\ifchilddoc| is set to true, |\includeonly| is applied to the child file
% and |\jobname| is set to the main file
% (for proper handling of |.aux| files):
%    \begin{macrocode}
\newcommand{\childdocmain}[1]
{
  \childdocdisable\childdocmain{}
  \if?#1?\else
    \begingroup
      \def\childdoctmp{#1}
      \ifx\childdoctmp\childdocname
        \def\childdoctmp{}
      \else
        \def\childdoctmp
        {
          \childdoctrue
          \includeonly{\childdocname}
          \def\childdocjob{#1}
          \def\jobname{#1}
        }
      \fi
      \expandafter
    \endgroup
    \childdoctmp
  \fi
}
%    \end{macrocode}

% \macro{\childdocof}
% The command |\childdocof| redirects
% compilation to the main file |#1|.
%    \begin{macrocode}
\newcommand{\childdocof}[1]
{
  \childdocdisable
  \childdoctrue
  \includeonly{\childdocname}
  \def\jobname{#1}
  \def\childdocjob{#1}
  \input{#1}
}
%    \end{macrocode}

% \macro{\childdocby}
% The command |\childdocby| ....
%    \begin{macrocode}
\newcommand{\childdocby}[2][]
{
  \childdocdisable
  \childdoctrue
  \childdocmanualtrue
  \if?#1?\else
    \def\jobname{#2}
  \fi
  \def\childdocjob{#2}
  \input{#2}
  \endinput
}
%    \end{macrocode}

% \macro{\childdocforward}
% The command |\childdocforward| redirects
% compilation to the main file or
% (if the optional argument is given) a child file.
% Parameters are set as if the main file
% or a child file starting with |\childdocof| was compiled.
% Then compilation is handed over to the main file:
%    \begin{macrocode}
\newcommand{\childdocforward}[2][]
{
  \begingroup
    \if?#1?
      \def\childdoctmp
      {
        \def\childdocname{#2}
        \def\childdocjob{#2}
        \def\jobname{#2}
        \input{#2}
        \endinput
      }
    \else
      \def\childdoctmp
      {
        \childdocdisable
        \def\childdocname{#2}
        \childdoctrue
        \includeonly{#2}
        \def\childdocjob{#1}
        \def\jobname{#1}
        \input{#1}
        \endinput
      }
    \fi
    \expandafter
  \endgroup
  \childdoctmp
}
%    \end{macrocode}

% \macro{\childdocforwardprefix}
% The command |\childdocforwardprefix| redirects
% compilation to the main or a child file by means of a pattern.
% The prefix |#1| in the current filename is replaced by |#2|
% and the suffix of the current filename is kept
% (it is assumed that the filename does not contain the substring `|~~~|'
% which is used as a delimiter).
% Compilation is handed over to the new file by |\childdocforward|:
%    \begin{macrocode}
\newcommand{\childdocforwardprefix}[3][]
{
  \begingroup
    \def\childdocextract #2##1~~~{\def\childdoctmp{\childdocforward[#1]{#3##1}}}
    \expandafter\childdocextract\childdocname~~~
    \expandafter
  \endgroup
  \childdoctmp
}
%    \end{macrocode}

% \macro{\childdoc}
% The deprecated macro |\childdoc| is a legacy version of |\childdocmain|:
%    \begin{macrocode}
\newcommand{\childdoc}{\childdocmain}
%    \end{macrocode}

% \macro{\childdocredirect}
% The deprecated macro |\childdocredirect| is a legacy version
% of |\childdocforward| and |\childdocforwardprefix|:
%    \begin{macrocode}
\newcommand{\childdocredirect}[2][]
{
  \begingroup
    \if?#1?
      \def\childdoctmp{\childdocforward{#2}}
    \else
      \def\childdoctmp{\childdocforwardprefix{#1}{#2}}
    \fi
    \expandafter
  \endgroup
  \childdoctmp
}
%    \end{macrocode}

%\iffalse
%</package>
%\fi
%
\endinput
|\\
|\childdocforwardprefix{final}{child}|
\end{tabular}
\end{center}
%

Note that when several versions of a main file and/or of each child file
are to be generated, it may be convenient to set up a |Makefile| or
shell script to automatise the process.

%%%%%%%%%%%%%%%%%%%%%%%%%%%%%%%%%%%%%%%%%%%%%%%%%%%%%%%%%%%%%%%%%%%%%%%%%%%%%%%%
\subsection{Command Line Processing}
\label{sec:commandline}

The effect of redirection files can also be achieved by invoking
the \LaTeX{} compiler with a more elaborate command line.
Most conveniently this should be done as part
of a shell script or a |Makefile|.

When using \textsf{childdoc} in the main file, the following
command lines effectively perform a redirection
(note that depending on the shell being used,
backslashes may have to be doubled: `|\|' $\to$ `|\\|'):
%
\begin{center}
|... -jobname "|\textit{target}|" |\\|"|[\textit{flags}]%
|% \iffalse
%
% childdoc.dtx Copyright (C) 2017-2018 Niklas Beisert
%
% This work may be distributed and/or modified under the
% conditions of the LaTeX Project Public License, either version 1.3
% of this license or (at your option) any later version.
% The latest version of this license is in
%   http://www.latex-project.org/lppl.txt
% and version 1.3 or later is part of all distributions of LaTeX
% version 2005/12/01 or later.
%
% This work has the LPPL maintenance status `maintained'.
%
% The Current Maintainer of this work is Niklas Beisert.
%
% This work consists of the files childdoc.dtx and childdoc.ins
% and the derived files childdoc.def and cdocsamp.tex with
% cdocsch1.tex, cdocsch2.tex, cdocsdrf.tex, cdocsfn1.tex, cdocsfn2.tex.
%
%<package>\ifdefined\childdocmain\endinput\fi
%<package>\ProvidesFile{childdoc.def}[2018/12/30 v2.0 child document driver]
%<samplemain>\ProvidesFile{cdocsamp.tex}[2018/12/30 v2.0 sample for childdoc]
%<*driver>
%\ProvidesFile{childdoc.drv}[2018/12/30 v2.0 childdoc reference manual file]
\PassOptionsToClass{10pt,a4paper}{article}
\documentclass{ltxdoc}

\usepackage[margin=35mm]{geometry}
\usepackage{hyperref}
\usepackage{hyperxmp}
\usepackage[usenames]{color}

\hypersetup{colorlinks=true}
\hypersetup{pdfstartview=FitH}
\hypersetup{pdfpagemode=UseNone}
\hypersetup{pdfsource={}}
\hypersetup{pdflang={en-UK}}
\hypersetup{pdfcopyright={Copyright 2017-2018 Niklas Beisert.
  This work may be distributed and/or modified under the
  conditions of the LaTeX Project Public License, either version 1.3
  of this license or (at your option) any later version.}}
\hypersetup{pdflicenseurl={http://www.latex-project.org/lppl.txt}}
\hypersetup{pdfcontactaddress={ETH Zurich, ITP, HIT K,
  Wolfgang-Pauli-Strasse 27}}
\hypersetup{pdfcontactpostcode={8093}}
\hypersetup{pdfcontactcity={Zurich}}
\hypersetup{pdfcontactcountry={Switzerland}}
\hypersetup{pdfcontactemail={nbeisert@itp.phys.ethz.ch}}
\hypersetup{pdfcontacturl={http://people.phys.ethz.ch/\xmptilde nbeisert/}}

\newcommand{\secref}[1]{\hyperref[#1]{section \ref*{#1}}}

\parskip1ex
\parindent0pt
\let\olditemize\itemize
\def\itemize{\olditemize\parskip0pt}

\begin{document}

\title{The \textsf{childdoc} Package}
\hypersetup{pdftitle={The childdoc Package}}
\author{Niklas Beisert\\[2ex]
  Institut f\"ur Theoretische Physik\\
  Eidgen\"ossische Technische Hochschule Z\"urich\\
  Wolfgang-Pauli-Strasse 27, 8093 Z\"urich, Switzerland\\[1ex]
  \href{mailto:nbeisert@itp.phys.ethz.ch}
  {\texttt{nbeisert@itp.phys.ethz.ch}}}
\hypersetup{pdfauthor={Niklas Beisert}}
\hypersetup{pdfsubject={Manual for the LaTeX2e Package childdoc}}
\date{30 December 2018, \textsf{v2.0}}
\maketitle

\begin{abstract}\noindent
\textsf{childdoc} is a \LaTeXe{} package
that enables the direct compilation
of document sections included by |\include|
to individual files.
\end{abstract}

\begingroup
\parskip0ex
\tableofcontents
\endgroup

%%%%%%%%%%%%%%%%%%%%%%%%%%%%%%%%%%%%%%%%%%%%%%%%%%%%%%%%%%%%%%%%%%%%%%%%%%%%%%%%
%%%%%%%%%%%%%%%%%%%%%%%%%%%%%%%%%%%%%%%%%%%%%%%%%%%%%%%%%%%%%%%%%%%%%%%%%%%%%%%%
\section{Introduction}

\LaTeX{} provides a mechanism to structure a large document (such as a book)
into a main file and several child files (containing the chapters)
using the |\include| command.
This mechanism is beneficial for documents
which span hundreds of pages in order to
make the source file(s) more manageable.
Moreover, compilation can be restricted to
selected child files by means of the |\includeonly| command.
The latter feature can be used to reduce the compilation time while editing
(this was significantly more useful in the earlier days of \LaTeX{})
or to generate a smaller document which is easier to navigate.
Another application of |\includeonly| is to generate
documents consisting of selected parts of the complete document.

However, there are a few drawbacks of the plain |\include| mechanism:
\begin{itemize}
\item
The child files cannot be compiled on their own,
they can only be compiled via the main file.
A naive editing environment
(such as a text editor with an option
to have the current file processed by \LaTeX)
may require one to switch to the main file before compiling;
attempting to compile the child file produces errors.
\item
The main file must be modified (each time)
to adjust the |\includeonly| command
to the present needs. This easily leaves the main file in a messy state.
\item
The generated document will always carry the filename
of the main document. This is inconvenient if
several child files are to be compiled and
to be kept for distribution.
\end{itemize}

The present package provides a simple interface
to make child files individually compilable by \LaTeX{}.
Compiling a child file then has the same effect as compiling
the main file with an |\includeonly| command
to select the appropriate child.
Moreover the generated document will carry the name of the child
rather than the main file.
This resolves all three above issues.

This feature is meant to make the editing of books,
thesis documents and lecture notes somewhat more convenient.
However, the package can also be used efficiently for
composing a series of documents (such as exercise sheets)
which are typically distributed individually.
It then assists the author in generating the individual documents
(potentially in different versions)
as well as a document containing the collected series.
Another application is in developing style files
or other kinds of included material
where compilation of the style file could redirect
to a sample or test file.

%%%%%%%%%%%%%%%%%%%%%%%%%%%%%%%%%%%%%%%%%%%%%%%%%%%%%%%%%%%%%%%%%%%%%%%%%%%%%%%%
%%%%%%%%%%%%%%%%%%%%%%%%%%%%%%%%%%%%%%%%%%%%%%%%%%%%%%%%%%%%%%%%%%%%%%%%%%%%%%%%
\section{Usage}

First of all, the package \textsf{childdoc} is \emph{not} a standard
\LaTeXe{} |.sty| style file! Therefore it needs to be invoked in
a non-standard way.

%%%%%%%%%%%%%%%%%%%%%%%%%%%%%%%%%%%%%%%%%%%%%%%%%%%%%%%%%%%%%%%%%%%%%%%%%%%%%%%%
\subsection{Included Files}
\label{sec:include}

%%%%%%%%%%%%%%%%%%%%%%%%%%%%%%%%%%%%%%%%
\DescribeMacro{\childdocmain}
To use the package, add the commands
\begin{center}
\begin{tabular}{l}
|\input{childdoc.def}|\\
|\childdocmain{}|\\
\end{tabular}
\end{center}
at the very top of the main \LaTeX{} file,
in particular \emph{before} the |\documentclass| statement!
The argument of |\childdocmain| should be left empty
(but it must be present).

%%%%%%%%%%%%%%%%%%%%%%%%%%%%%%%%%%%%%%%%
\DescribeMacro{\childdocof}
Furthermore, add the commands
\begin{center}
\begin{tabular}{l}
|\input{childdoc.def}|\\
|\childdocof{|\textit{main}|}|\\
\end{tabular}
\end{center}
at the top of every child file \textit{child}
which is included by |\include{|\textit{child}|}|
from within the main file
(or at least for those files to be compiled individually).
The argument \textit{main} must be the filename of the main file.

There are a couple of
considerations in setting up the main and child documents:

%%%%%%%%%%%%%%%%%%%%%%%%%%%%%%%%%%%%%%%%
\paragraph{Restrictions.}

Please note the following restrictions:
\begin{itemize}
\item
|\childdocmain| must be called with one argument \textit{main}
to ensure compatibility with earlier version of the package.
It must either be empty (|\childdocmain{}|)
or precisely match the filename of the main file in which it is specified.
See \secref{sec:detection} for further information.
\item
The filename \textit{main} must be specified without the |.tex| extension.
\item
The filename \textit{main} is case sensitive
(even in case-insensitive file systems)
due to internal string comparison.
\item
The argument \textit{main} should be fully expanded, it cannot be a macro.
\item
Subdirectories and special characters should be avoided in filenames.
\item
The command |\childdocmain{|\textit{main}|}| must be followed by a whitespace.
It should not be followed immediately by another command
or by a comment mark `|%|'.
This is because the \TeX{} parser reads the token immediately following
the argument of |\childdocmain| and puts it
at the beginning of every child section;
however, a white\-space is ignored.
\end{itemize}

%%%%%%%%%%%%%%%%%%%%%%%%%%%%%%%%%%%%%%%%
\paragraph{Content of Main File.}

It is advisable to place all content in the child files included by |\include|.
Any output contained in the main file will appear in all child documents
unless suppressed manually;
it cannot be suppressed automatically by the |\includeonly| directive
and thus should normally be avoided.
A method to include some content in the main file
by means of conditional processing is described in \secref{sec:conditional}.

%%%%%%%%%%%%%%%%%%%%%%%%%%%%%%%%%%%%%%%%
\paragraph{Page Numbering.}

When only a part of the document is compiled,
the appropriate numbering of pages
(as well as other status parameters)
is determined from the |.aux| files.
The latter contain information from previous passes.
However this information needs to propagate through
all intermediate child documents.
Therefore the page numbering in child documents may well
be inconsistent until the complete document is compiled at least once.

A useful (if unconventional) way to always ensure a consistent
page numbering is to restart the numbering in each child document
and denote the pages by `\textit{child}|.|\textit{page}'
where \textit{child} represents the chapter/section number of the child file.
This can be achieved by the command
|\numberwithin{page}{|\textit{child}|}|
of the \textsf{amsmath} package
where \textit{child} can be |chapter| or |section|
depending on the chosen structuring.
Alternatively, one can modify the macro |\thepage| appropriately
and reset the counter |page| at the start of each child file.

%%%%%%%%%%%%%%%%%%%%%%%%%%%%%%%%%%%%%%%%%%%%%%%%%%%%%%%%%%%%%%%%%%%%%%%%%%%%%%%%
\subsection{Conditional Processing}
\label{sec:conditional}

The package provides a mechanism to compile different versions
of a document. To customise the versions further some conditional processing
can come in handy to distinguish which version is being compiled.
The package provides two macros to describe the compilation context:

%%%%%%%%%%%%%%%%%%%%%%%%%%%%%%%%%%%%%%%%
\DescribeMacro{\ifchilddoc}
The conditional |\ifchilddoc| distinguishes between the compilation of
child documents and the main document:
%
\begin{center}
|\ifchilddoc |\textit{child-code}| |[|\||else |\textit{main-code}]| \||fi|
\end{center}

%%%%%%%%%%%%%%%%%%%%%%%%%%%%%%%%%%%%%%%%
\DescribeMacro{\childdocname}
\DescribeMacro{\childdocjob}
The macro |\childdocname| contains the filename (without extension)
of the main or child file being processed.
Note that |\childdocjob| will always contain the name of the main file.

%%%%%%%%%%%%%%%%%%%%%%%%%%%%%%%%%%%%%%%%
\paragraph{Title Page.}

Conditional processing can be used to include a title or banner page
in the main document when proper precautions are taken.
Importantly, the code in the main file should ensure that the page counter
(as well as other status parameters which are stored in the |.aux| files)
takes the same value after the conditional processing.
Otherwise the page numbers may take divergent values
depending on which part is compiled.

For example, a title page could be declared by:
%
\begin{center}
\begin{tabular}{l}
|\ifchilddoc\||else|\\
|\addtocounter{page}{-1}|\\
\textit{code for title page}\\
|\newpage|\\
|\||fi|
\end{tabular}
\end{center}
%
A banner page for the child documents can be generated by:
%
\begin{center}
\begin{tabular}{l}
|\ifchilddoc|\\
|\addtocounter{page}{-1}|\\
\textit{code for banner page}\\
|\newpage|\\
|\||fi|
\end{tabular}
\end{center}
%
Here one could write a message such as:
\begin{center}
|This is the part \childdocname{} of \childdocjob{}.|
\end{center}

%%%%%%%%%%%%%%%%%%%%%%%%%%%%%%%%%%%%%%%%%%%%%%%%%%%%%%%%%%%%%%%%%%%%%%%%%%%%%%%%
\subsection{Flags}
\label{sec:flags}

The package makes it easy to generate different versions
of the main or child documents.
To this end compilation flags can be defined
and assigned different default values.
They will be particularly useful in conjunction
with the forwarding mechanism described in \secref{sec:forward}.

For example, it may be useful to have a flag |\version|
which can be set to |draft| or |final|.
The document source will contain some conditional code
depending on the value of |\version|.
Suppose further, the flag should default to |final| for the main file
and to |draft| for child files
which is a natural assignment for editing the document.
This is achieved by placing the following code
in the preamble of the main document
(below the |\childdocmain| directive):
%
\begin{center}
\begin{tabular}{l}
|\ifchilddoc|\\
|\providecommand{\version}{draft}|\\
|\||else|\\
|\providecommand{\version}{final}|\\
|\||fi|
\end{tabular}
\end{center}
%
The definition by |\providecommand| makes sure
that previous definitions are not overwritten.
Further statements |\providecommand{\version}{...}|
can thus be added before the above code to override it.

For the main file, one might add a line
(between |\childdocmain| and the above block)
%
\begin{center}
|%\ifchilddoc\||else\providecommand{\version}{draft}\||fi|
\end{center}
%
which can be uncommented to produce a draft version.
Likewise one can add a line to the very top of a child file
(above the |\childdocof{|\textit{main}|}| directive)
%
\begin{center}
|%\providecommand{\version}{final}|
\end{center}
%
which can be uncommented to produce the final version of this child document.

%%%%%%%%%%%%%%%%%%%%%%%%%%%%%%%%%%%%%%%%%%%%%%%%%%%%%%%%%%%%%%%%%%%%%%%%%%%%%%%%
\subsection{Forwarding}
\label{sec:forward}

Different versions of the main or child documents
using compilation flags as described in \secref{sec:flags}
can be (permanently) stored in different files
for convenient compilation, viewing and distribution.
To this end, the package defines a command
to pass on compilation to a different file:

%%%%%%%%%%%%%%%%%%%%%%%%%%%%%%%%%%%%%%%%
\DescribeMacro{\childdocforward}
The command |\childdocforward| redirects processing to
another source file:
%
\begin{center}
\begin{tabular}{l}
|\input{childdoc.def}|\\
|\childdocforward[|\textit{main}|]{|\textit{dest}|}|\\
\end{tabular}
\end{center}
%
The argument \textit{dest} is the destination file
(without extension).
It should be the main file or one of the child files.
Note that further \textsf{childdoc} directives
such as |\childdocof| and |\childdocforward|
in the indicated file will be processed in this form.
The optional argument \textit{main}
passes on directly to the main file \textit{main}
while pretending to compile the child \textit{dest}.
This form behaves as if \textit{dest}
issues |\childdocof{|\textit{main}|}| right away,
and no further \textsf{childdoc} directives will be processed.

%%%%%%%%%%%%%%%%%%%%%%%%%%%%%%%%%%%%%%%%
\DescribeMacro{\...prefix}
In the alternative form |\childdocforwardprefix|,
%
\begin{center}
\begin{tabular}{l}
|\input{childdoc.def}|\\
|\childdocforwardprefix[|\textit{main}|]{|\textit{prefix}|}{|\textit{dest}|}|
\end{tabular}
\end{center}
%
the destination file is determined by a pattern
depending on the current file:
To make this work, the current file must be called
`{\textit{prefix}\hspace{0.2em}\textit{suffix}}'
with \textit{prefix} matching precisely the argument.
Processing is then passed on to the file
`{\textit{dest}\hspace{0.2em}\textit{suffix}}'.
Surely, the same effect is achieved by
directly specifying the
argument `{\textit{dest}\hspace{0.2em}\textit{suffix}}'
in the first form.
However, that requires to set up a different file
for each child. With the alternative form of the command
all these files can have exactly the same content
which simplifies setting them up and maintaining them.

For example, the following file |draft.tex|
with a compilation flag |\version| as described in \secref{sec:flags}
compiles the main document as a draft:
%
\begin{center}
\begin{tabular}{l}
|\def\version{draft}|\\
|\input{childdoc.def}|\\
|\childdocforward{|\textit{main}|}|
\end{tabular}
\end{center}
%
Likewise, the following files |final|\textit{nn}|.tex|
compile the final version of the child document
|child|\textit{nn}|.tex|:
%
\begin{center}
\begin{tabular}{l}
|\def\version{final}|\\
|\input{childdoc.def}|\\
|\childdocforwardprefix{final}{child}|
\end{tabular}
\end{center}
%

Note that when several versions of a main file and/or of each child file
are to be generated, it may be convenient to set up a |Makefile| or
shell script to automatise the process.

%%%%%%%%%%%%%%%%%%%%%%%%%%%%%%%%%%%%%%%%%%%%%%%%%%%%%%%%%%%%%%%%%%%%%%%%%%%%%%%%
\subsection{Command Line Processing}
\label{sec:commandline}

The effect of redirection files can also be achieved by invoking
the \LaTeX{} compiler with a more elaborate command line.
Most conveniently this should be done as part
of a shell script or a |Makefile|.

When using \textsf{childdoc} in the main file, the following
command lines effectively perform a redirection
(note that depending on the shell being used,
backslashes may have to be doubled: `|\|' $\to$ `|\\|'):
%
\begin{center}
|... -jobname "|\textit{target}|" |\\|"|[\textit{flags}]%
|\input{childdoc.def}\childdocforward[|\textit{main}|]{|\textit{dest}|}"|
\end{center}
%
Here \textit{target} is the name of the output file,
\textit{main} is the name of the main file
and \textit{dest} is the name of the main or child file to be processed
(all filenames without extensions).
The optional argument \textit{main} can be omitted
if \textit{main} matches \textit{dest}.
Optionally, compilation \textit{flags} can be defined via |\def| commands.
This command line makes the \TeX{} engine believe
it is compiling the file \textit{target}
whose content is specified as the latter parameter.
The provided code then forwards the processing to
\textit{main} or \textit{dest} as described in \secref{sec:forward}.

%%%%%%%%%%%%%%%%%%%%%%%%%%%%%%%%%%%%%%%%%%%%%%%%%%%%%%%%%%%%%%%%%%%%%%%%%%%%%%%%
\subsection{Include by Input}
\label{sec:input}

Including child documents by |\include| has some restrictions by design.
Most notably, the content of a child document always occupies
its own set of pages; pages cannot be shared between child documents.
Usually, this behaviour makes perfect sense
because each child document contain an essential part of the document.
However, in some situations it may be desirable to compose
a document from a collection of parts
without having mandatory page breaks between then.
For this case, the package
provides a mechanism to include parts
by |\input| which can also be processed individually.
However, by construction this mechanism
requires manual handling of the content to be output.

%%%%%%%%%%%%%%%%%%%%%%%%%%%%%%%%%%%%%%%%
\DescribeMacro{\ifchilddocmanual}
The main file should be prepared as usual, see \secref{sec:include}.
However, the document body must make a distinction
between processing of an individual part and of the main document, e.g.:
%
\begin{center}
\begin{tabular}{l}
|\ifchilddocmanual|\\
|\input{\childdocname}|\\
|\||else|\\
\textit{document body with }|\input{|\textit{part}|}|\\
|\||fi|
\end{tabular}
\end{center}
%
The conditional |\ifchilddocmanual| is true whenever
a part to be included by |\input| is being compiled,
and the name of the part is stored in |\childdocname|.

%%%%%%%%%%%%%%%%%%%%%%%%%%%%%%%%%%%%%%%%
\DescribeMacro{\childdocby}
Each part to be included by |\input| should start with:
%
\begin{center}
\begin{tabular}{l}
|\input{childdoc.def}|\\
|\childdocby{|\textit{main}|}|\\
\end{tabular}
\end{center}
%
The directive |\childdocby| is similar to |\childdocof|
described in \secref{sec:include},
but the subsequent selection of content must be done manually.
To that end, both |\ifchilddoc| and |\ifchilddocmanual|
will be true upon processing of a part,
and the name of the part is stored in |\childdocname|.
Note that |\jobname| will be set to the filename of the current part
so that each part receives an individual |.aux| file
that does not interfere with the |.aux| file(s) of the main document.
This behaviour can be altered by the alternative form
|\childdocby[*]{|\textit{main}|}| (with a non-empty optional argument)
which uses the |.aux| file of the main document
by setting |\jobname| to \textit{main}.

%%%%%%%%%%%%%%%%%%%%%%%%%%%%%%%%%%%%%%%%%%%%%%%%%%%%%%%%%%%%%%%%%%%%%%%%%%%%%%%%
\subsection{Driver Development}
\label{sec:driver}

The \textsf{childdoc} mechanism can also be use for the development
of definition files such as \LaTeX{} styles or classes.
This case differs from the above setup with multiple parts
included by |\include| in that no |\includeonly| should be invoked.
This can be achieved by starting the include file
(before |\ProvidesPackage|) with:
%
\begin{center}
\begin{tabular}{l}
|\input{childdoc.def}|\\
|\childdocforward{|\textit{main}|}|\\
\end{tabular}
\end{center}
%
or alternatively with:
%
\begin{center}
\begin{tabular}{l}
|\input{childdoc.def}|\\
|\childdocby{|\textit{main}|}|\\
\end{tabular}
\end{center}
%
Both forms have slightly different effects as described above.
The main file is prepared as usual, see \secref{sec:include}.

%%%%%%%%%%%%%%%%%%%%%%%%%%%%%%%%%%%%%%%%%%%%%%%%%%%%%%%%%%%%%%%%%%%%%%%%%%%%%%%%
\subsection{Legacy Detection}
\label{sec:detection}

The directive |\childdocmain| in the main file can detect
whether the complete document or merely a child is to be compiled
even without using the directive |\childdocof|.
This method is deprecated because it is less robust
and there is no compelling reason to use it;
it is merely provided for backward compatibility
and it may be removed in future versions.

If the detection mechanism is to be used,
it is mandatory to correctly specify
the filename of the main file as the argument of |\childdocmain|:
%
\begin{center}
\begin{tabular}{l}
|\input{childdoc.def}|\\
|\childdocmain{|\textit{main}|}|\\
\end{tabular}
\end{center}
%
If |\jobname| does not match the argument \textit{main} of |\childdocmain|,
it is assumed that |\jobname| points to the child file to be compiled.
When using |\childdocmain| with the main file specified as argument,
it suffices to start a child file
with just |\input{|\textit{main}|}|
without loading of the package and using |\childdocof|.
If instead all processing is done
with the appropriate \textsf{childdoc} directives,
the argument of \textit{main} of |\childdocmain| can be empty.

An alternative version of the command line processing described
in \secref{sec:commandline} using the detection mechanism reads:
%
\begin{center}
|... -jobname "|\textit{target}|" "|[\textit{flags}]%
[|\def\jobname{|\textit{dest}|}|]|\input{|\textit{main}|}"|
\end{center}

%%%%%%%%%%%%%%%%%%%%%%%%%%%%%%%%%%%%%%%%%%%%%%%%%%%%%%%%%%%%%%%%%%%%%%%%%%%%%%%%
\subsection{Manual Code}
\label{sec:manual}

In case one cannot be certain whether the definitions file |childdoc.def|
is installed on the target \TeX{} distribution
and one prefers not to ship it,
it is conceivable to paste a few relevant commands into the sources.

To that end, drop all statements |\input{childdoc.def}|
and perform the replacements as outlined below.
Instead of |\childdocmain{|\textit{main}|}| add the following code
to the top of the main file:
%
\begin{center}
\begin{tabular}{l}
|\||ifdefined\childdocname\endinput\||fi\newif\ifchilddoc|\\
|\edef\childdocname{\scantokens\expandafter{\jobname\noexpand}}|\\
|\def\childdocmain{|\textit{main}|}\||ifx\childdocmain\childdocname\||else|\\
|\childdoctrue\includeonly{\childdocname}\let\jobname\childdocmain\||fi|\\
\end{tabular}
\end{center}
%
Instead of |\childdocof{|\textit{main}|}| just include the main file
at the top of each child file:
%
\begin{center}
|\input{|\textit{main}|}|
\end{center}
%
A simple redirection |\childdocforward{|\textit{dest}|}| is achieved by:
%
\begin{center}
|\def\jobname{|\textit{dest}|}\input{\jobname}|
\end{center}
%
The redirection with prefix
|\childdocforwardprefix[|\textit{prefix}|]{|\textit{dest}|}|
is accomplished by:
%
\begin{center}
\begin{tabular}{l}
|{\edef\jobname{\scantokens\expandafter{\jobname\noexpand}}|\\
|\def\redirectjob |\textit{prefix}|#1~~~{\gdef\jobname{|\textit{dest}|#1}}|\\
|\expandafter\redirectjob\jobname~~~}\input{\jobname}|
\end{tabular}
\end{center}

In an alternative approach,
child documents can be compiled by a specific command line
without additional code or specific definitions:
%
\begin{center}
|... -jobname "|\textit{target}|" "|[\textit{flags}]%
|\includeonly{|\textit{dest}|}\input{|\textit{main}|}"|
\end{center}
%

%%%%%%%%%%%%%%%%%%%%%%%%%%%%%%%%%%%%%%%%%%%%%%%%%%%%%%%%%%%%%%%%%%%%%%%%%%%%%%%%
%%%%%%%%%%%%%%%%%%%%%%%%%%%%%%%%%%%%%%%%%%%%%%%%%%%%%%%%%%%%%%%%%%%%%%%%%%%%%%%%
\section{Information}

%%%%%%%%%%%%%%%%%%%%%%%%%%%%%%%%%%%%%%%%%%%%%%%%%%%%%%%%%%%%%%%%%%%%%%%%%%%%%%%%
\subsection{Copyright}

Copyright \copyright{} 2017--2018 Niklas Beisert

This work may be distributed and/or modified under the
conditions of the \LaTeX{} Project Public License, either version 1.3
of this license or (at your option) any later version.
The latest version of this license is in
  \url{http://www.latex-project.org/lppl.txt}
and version 1.3 or later is part of all distributions of \LaTeX{}
version 2005/12/01 or later.

This work has the LPPL maintenance status `maintained'.

The Current Maintainer of this work is Niklas Beisert.

This work consists of the files |README.txt|, |childdoc.ins| and |childdoc.dtx|
as well as the derived files |childdoc.def|, |cdocsamp.tex|
with |cdocsch1.tex|, |cdocsch2.tex|, |cdocspt3.tex|, |cdocspt4.tex|,
|cdocsdrf.tex|, |cdocsfn1.tex|, |cdocsfn2.tex|
as well as |childdoc.pdf|.

%%%%%%%%%%%%%%%%%%%%%%%%%%%%%%%%%%%%%%%%%%%%%%%%%%%%%%%%%%%%%%%%%%%%%%%%%%%%%%%%
\subsection{Files and Installation}

The package consists of the files:
%
\begin{center}
\begin{tabular}{ll}
    |README.txt|   & readme file \\
    |childdoc.ins| & installation file \\
    |childdoc.dtx| & source file \\
    |childdoc.def| & definition file \\
    |cdocsamp.tex| & sample main file \\
    |cdocsch1.tex| & sample include file \\
    |cdocsch2.tex| & sample include file \\
    |cdocspt3.tex| & sample part file \\
    |cdocspt4.tex| & sample part file \\
    |cdocsdrf.tex| & sample redirection file \\
    |cdocsfn1.tex| & sample redirection file \\
    |cdocsfn2.tex| & sample redirection file \\
    |childdoc.pdf| & manual
\end{tabular}
\end{center}
%
The distribution consists of the files
|README.txt|, |childdoc.ins| and |childdoc.dtx|.
%
\begin{itemize}
\item
Run (pdf)\LaTeX{} on |childdoc.dtx|
to compile the manual |childdoc.pdf| (this file).
\item
Run \LaTeX{} on |childdoc.ins| to create the definitions file |childdoc.def|
and the sample |cdocsamp.tex| with include files
|cdocsch1.tex|, |cdocsch2.tex|, |cdocspt3.tex|, |cdocspt4.tex|,
|cdocsdrf.tex|, |cdocsfn1.tex|, |cdocsfn2.tex|.
Then copy the file |childdoc.def| to an appropriate directory of your \LaTeX{}
distribution, e.g.\ \textit{texmf-root}|/tex/latex/childdoc|.
\end{itemize}

%%%%%%%%%%%%%%%%%%%%%%%%%%%%%%%%%%%%%%%%%%%%%%%%%%%%%%%%%%%%%%%%%%%%%%%%%%%%%%%%
\subsection{Related CTAN Packages}

There are several other packages which offer a similar functionality:
%
\begin{itemize}
\item
The packages
\href{http://ctan.org/pkg/docmute}{\textsf{docmute}},
\href{http://ctan.org/pkg/includex}{\textsf{includex}} and
\href{http://ctan.org/pkg/standalone}{\textsf{standalone}}
provide commands to include only the document body of
a child file thus allowing both files to be compiled individually.
\item
The packages \href{http://ctan.org/pkg/subdocs}{\textsf{subdocs}}
and \href{http://ctan.org/pkg/subfiles}{\textsf{subfiles}}
provide structures in which the main and child documents can be
encapsulated and allowing them to be compiled individually.
The inclusion mechanism is different from the conventional |\include|.
\item
The package \href{http://ctan.org/pkg/combine}{\textsf{combine}}
is an elaborate solution to combine several documents into one.
\end{itemize}
%
See also the CTAN topic \href{http://ctan.org/topic/subdocs}{\textsf{subdocs}}
for further related packages.
The present package differs from the above solutions in that
a document structure constructed with the conventional |\include| mechanism
just needs two extra commands at the top of every file
such that all constituent files can be compiled individually.

%%%%%%%%%%%%%%%%%%%%%%%%%%%%%%%%%%%%%%%%%%%%%%%%%%%%%%%%%%%%%%%%%%%%%%%%%%%%%%%%
%\subsection{Feature Suggestions}
%
%The following is a list of features which may be useful for future
%versions of this package:
%%
%\begin{itemize}
%\item
%\ldots
%\end{itemize}

%%%%%%%%%%%%%%%%%%%%%%%%%%%%%%%%%%%%%%%%%%%%%%%%%%%%%%%%%%%%%%%%%%%%%%%%%%%%%%%%
\subsection{Revision History}

%%%%%%%%%%%%%%%%%%%%%%%%%%%%%%%%%%%%%%%%
\paragraph{v2.0:} 2018/12/30

\begin{itemize}
\item
immediate forward processing
\item
added |\childdocby| mechanism
\item
manual restructured
\end{itemize}

%%%%%%%%%%%%%%%%%%%%%%%%%%%%%%%%%%%%%%%%
\paragraph{v1.6:} 2018/01/17

\begin{itemize}
\item
application for development of include files
\item
corrections to manual
\end{itemize}

%%%%%%%%%%%%%%%%%%%%%%%%%%%%%%%%%%%%%%%%
\paragraph{v1.5:} 2017/05/21

\begin{itemize}
\item
more complete structuring introduced
\item
|\childdocof| introduced
\item
|\childdoc| renamed to |\childdocmain|
\item
|\childredirect| renamed to |\childdocforward| and |\childdocforwardprefix|
and functionality expanded
\end{itemize}

%%%%%%%%%%%%%%%%%%%%%%%%%%%%%%%%%%%%%%%%
\paragraph{v1.0:} 2017/04/27

\begin{itemize}
\item
manual and install package
\item
first version published on CTAN
\end{itemize}

%%%%%%%%%%%%%%%%%%%%%%%%%%%%%%%%%%%%%%%%
\paragraph{v0.6:} 2017/04/26

\begin{itemize}
\item
redirection mechanism added
\end{itemize}

%%%%%%%%%%%%%%%%%%%%%%%%%%%%%%%%%%%%%%%%
\paragraph{v0.5:} 2017/04/26

\begin{itemize}
\item
functionality in definition file
\end{itemize}


%%%%%%%%%%%%%%%%%%%%%%%%%%%%%%%%%%%%%%%%%%%%%%%%%%%%%%%%%%%%%%%%%%%%%%%%%%%%%%%%
%%%%%%%%%%%%%%%%%%%%%%%%%%%%%%%%%%%%%%%%%%%%%%%%%%%%%%%%%%%%%%%%%%%%%%%%%%%%%%%%
%%%%%%%%%%%%%%%%%%%%%%%%%%%%%%%%%%%%%%%%%%%%%%%%%%%%%%%%%%%%%%%%%%%%%%%%%%%%%%%%
\appendix

\settowidth\MacroIndent{\rmfamily\scriptsize 000\ }

 \DocInput{childdoc.dtx}

\end{document}
%</driver>
% \fi
%
% %%%%%%%%%%%%%%%%%%%%%%%%%%%%%%%%%%%%%%%%%%%%%%%%%%%%%%%%%%%%%%%%%%%%%%%%%%%%%%
% %%%%%%%%%%%%%%%%%%%%%%%%%%%%%%%%%%%%%%%%%%%%%%%%%%%%%%%%%%%%%%%%%%%%%%%%%%%%%%
% \section{Sample}
%\iffalse
%<*samplemain>
%\fi
%
% The following presents a sample document
% with two chapters, two parts, a title page,
% a compile flag as well as three forwarding files to set the flag.
% It consists of eight |.tex| files:
% \begin{center}
% \begin{tabular}{ll}
% |cdocsamp.tex|&main file\\
% |cdocsch1.tex|&include file for chapter 1\\
% |cdocsch2.tex|&include file for chapter 2\\
% |cdocspt3.tex|&include file for part 3\\
% |cdocspt4.tex|&include file for part 4\\
% |cdocsdrf.tex|&forwarding file for main file in draft mode\\
% |cdocsfi1.tex|&forwarding file for final version of chapter 1\\
% |cdocsfi2.tex|&forwarding file for final version of chapter 2\\
% \end{tabular}
% \end{center}
% Each of the eight files can be compiled directly by the \LaTeX{} compiler.
%
% %%%%%%%%%%%%%%%%%%%%%%%%%%%%%%%%%%%%%%
% \paragraph{Main File.}
%
% The main file is called |cdocsamp.tex|.
%
% Load the \textsf{childdoc} definitions and
% declare the filename for the main document:
%    \begin{macrocode}
\input{childdoc.def}
\childdocmain{}
%    \end{macrocode}

% Optional override for |\version| flag:
%    \begin{macrocode}
%%\ifchilddoc\else\providecommand{\version}{draft}\fi
%    \end{macrocode}

% Define the default values for the |\version| flag
% (|final| for the main file and |draft| for childs):
%    \begin{macrocode}
\ifchilddoc
\providecommand{\version}{draft}
\else
\providecommand{\version}{final}
\fi
%    \end{macrocode}

% Load the standard document class:
%    \begin{macrocode}
\documentclass[12pt]{article}
%    \end{macrocode}

% Start the document body:
%    \begin{macrocode}
\begin{document}
%    \end{macrocode}

% Declare a title page.
% Print title, part of document being processed and version flag:
%    \begin{macrocode}
\addtocounter{page}{-1}
\begin{center}
{\LARGE\bfseries{}childdoc example\par}
\vspace{1cm}
\ifchilddoc
\ifchilddocmanual part\else chapter\fi:
`\childdocname' of `\childdocjob'\par
\else
main document: `\childdocjob'\par
\fi
version: \version\par
\end{center}
\newpage
%    \end{macrocode}

% Manually include selected file,
% otherwise process as usual:
%    \begin{macrocode}
\ifchilddocmanual
\section*{part `\childdocname'}
\input{\childdocname}
\else
%    \end{macrocode}

% Include the two chapters:
%    \begin{macrocode}
\include{cdocsch1}
\include{cdocsch2}
%    \end{macrocode}

% Include the two parts unless only chapters should be displayed:
%    \begin{macrocode}
\ifchilddoc\else
\section{part three}
\input{cdocspt3}
\section{part four}
\input{cdocspt4}
\fi
%    \end{macrocode}

% Process as usual until here:
%    \begin{macrocode}
\fi
%    \end{macrocode}

% End of document body:
%    \begin{macrocode}
\end{document}
%    \end{macrocode}
%\iffalse
%</samplemain>
%\fi
%
% %%%%%%%%%%%%%%%%%%%%%%%%%%%%%%%%%%%%%%
% \paragraph{Chapter Include Files.}
%
% The include files are called |cdocsch1.tex| and |cdocsch2.tex|.
%
%\iffalse
%<*samplechap1|samplechap2>
%\fi

% Optional override for |\version| flag:
%    \begin{macrocode}
%%\providecommand{\version}{final}
%    \end{macrocode}

% Include the main document:
%    \begin{macrocode}
\input{childdoc.def}
\childdocof{cdocsamp}
%    \end{macrocode}

%\iffalse
%</samplechap1|samplechap2>
%\fi
%
%\iffalse
%<*samplechap1>
%\fi
% Some text for chapter 1:
%    \begin{macrocode}
\section{one}
some text in chapter one
%    \end{macrocode}

%\iffalse
%</samplechap1>
%\fi
% Some text for chapter 2:
%\iffalse
%<*samplechap2>
%\fi
%    \begin{macrocode}
\section{two}
more text in chapter two
%    \end{macrocode}

%\iffalse
%</samplechap2>
%\fi
%
% %%%%%%%%%%%%%%%%%%%%%%%%%%%%%%%%%%%%%%
% \paragraph{Part Include Files.}
%
% The include files are called |cdocspt3.tex| and |cdocspt4.tex|.
%
%\iffalse
%<*samplepart3|samplepart4>
%\fi

% Optional override for |\version| flag:
%    \begin{macrocode}
%%\providecommand{\version}{final}
%    \end{macrocode}

% Include the main document:
%    \begin{macrocode}
\input{childdoc.def}
\childdocby{cdocsamp}
%    \end{macrocode}

%\iffalse
%</samplepart3|samplepart4>
%\fi
%
%\iffalse
%<*samplepart3>
%\fi
% Some text for part 3:
%    \begin{macrocode}
some text in part three
%    \end{macrocode}

%\iffalse
%</samplepart3>
%\fi
% Some text for part 4:
%\iffalse
%<*samplepart4>
%\fi
%    \begin{macrocode}
more text in part four
%    \end{macrocode}

%\iffalse
%</samplepart4>
%\fi
%
% %%%%%%%%%%%%%%%%%%%%%%%%%%%%%%%%%%%%%%
% \paragraph{Forwarding for a Complete Draft.}
%
% The following forwarding file |cdocsdrf.tex|
% compiles the main document in draft mode:
%\iffalse
%<*sampledraft>
%\fi
%    \begin{macrocode}
\def\version{draft}
\input{childdoc.def}
\childdocforward{cdocsamp}
%    \end{macrocode}

%\iffalse
%</sampledraft>
%\fi
%
% %%%%%%%%%%%%%%%%%%%%%%%%%%%%%%%%%%%%%%
% \paragraph{Forwarding for Final Version of the Chapters.}
%
% The following forwarding files |cdocsfn1.tex| and |cdocsfn2.tex|
% (with identical content)
% compile the final versions of the child documents
% |cdocsch1.tex| and |cdocsch2.tex|, respectively:
%\iffalse
%<*samplefinal>
%\fi
%    \begin{macrocode}
\def\version{final}
\input{childdoc.def}
\childdocforwardprefix[cdocsamp]{cdocsfn}{cdocsch}
%    \end{macrocode}

%\iffalse
%</samplefinal>
%\fi
%
% %%%%%%%%%%%%%%%%%%%%%%%%%%%%%%%%%%%%%%
% \paragraph{Command Line Processing.}
%
% The following three command lines generate the output files
% |cdocscld|, |cdocscl1| and |cdocscl2|
% which should be identical to
% |cdocsdrf|, |cdocsch1| and |cdocsfn2|, respectively:
% \begin{center}
% \begin{tabular}{l}
% |latex -jobname cdocscld \|\\
% |  "\def\version{draft}\input{childdoc.def}\childdocforward{cdocsamp}"|\\
% |latex -jobname cdocscl1 \|\\
% |  "\input{childdoc.def}\childdocforward[cdocsamp]{cdocsch1}"|\\
% |latex -jobname cdocscl2 \|\\
% |  "\def\version{final}\input{childdoc.def}\childdocforward{cdocsch2}"|
% \end{tabular}
% \end{center}
% Note that the trailing backslash on each first line
% merely continues the input to the second line
% (for convenient cut ant paste).
% Furthermore, the command |latex| can be replaced by any
% of its alternative versions such as |pdflatex|.
%
% %%%%%%%%%%%%%%%%%%%%%%%%%%%%%%%%%%%%%%%%%%%%%%%%%%%%%%%%%%%%%%%%%%%%%%%%%%%%%%
% %%%%%%%%%%%%%%%%%%%%%%%%%%%%%%%%%%%%%%%%%%%%%%%%%%%%%%%%%%%%%%%%%%%%%%%%%%%%%%
% \section{Implementation}
%\iffalse
%<*package>
%\fi
%
% This section describes the definitions file |childdoc.def|.

% The definitions cannot be loaded using |\usepackage| or |\RequirePackage|
% which has a mechanism to prevent loading a style file more than once.
% When loading the definitions by means of |\input|
% multiple instances have to be prevented manually:
%\iffalse
%This code needs to be before the `\ProvidesFile' directive
%which is defined at the beginning of this file.
%Therefore it is also placed there and commented out here.
%</package>
%<*discard>
%\fi
%    \begin{macrocode}
\ifdefined\childdocmain\endinput\fi
%    \end{macrocode}
%\iffalse
%</discard>
%<*package>
%\fi
%
% \macro{\ifchilddoc}
% \macro{\ifchilddocmanual}
% The conditional |\ifchilddoc| tells whether a
% child (true) or main (false) document is being compiled.
% The conditional |\ifchilddocmanual| tells whether
% the |\includeonly| mechanism is used (false) or
% the selection of child files must be performed manually (true).
% The definitions initialise to false:
%    \begin{macrocode}
\newif\ifchilddoc
\newif\ifchilddocmanual
%    \end{macrocode}

% \macro{\childdocname}
% \macro{\childdocjob}
% The macro |\childdocname| stores the name of the main document
% to be compiled. The macro |\childdocjob| stores the name of
% the document on which the \LaTeX{} compiler was originally invoked.
% The content of |\jobname| cannot be compared
% to filenames specified in the source due to different catcodes.
% The following code rescans |\jobname|, stores the result
% in |\childdocname| and saves a copy in |\childdocjob|:
%    \begin{macrocode}
\edef\childdocname{\scantokens\expandafter{\jobname\noexpand}}
\let\childdocjob\childdocname
%    \end{macrocode}

% \macro{\childdocdisable}
% The macro |\childdocdisable| prevents the main file
% from being processed more than once.
% At this stage, the main document command |\childdocmain|
% is assumed to be called once again where it should do nothing.
% Any subsequent call to it should prevent
% a secondary processing of the main document
% It overwrites the forwarding commands
% |\childdocof| and |\childdocforward|
% with empty macros to prevent further inclusions of the main document:
%    \begin{macrocode}
\newcommand{\childdocdisable}
{
  \renewcommand{\childdocmain}[1]{\renewcommand{\childdocmain}[1]{\endinput}}
  \renewcommand{\childdocof}[1]{}
  \renewcommand{\childdocby}[2][]{}
  \renewcommand{\childdocforward}[2][]{}
  \renewcommand{\childdocdisable}{}
}
%    \end{macrocode}

% \macro{\childdocmain}
% The macro |\childdocmain| is to be called at the top of the main file
% with nothing or the main filename (without extension) as argument.
% First, it breaks loops.
% If the argument is not empty and does not match |\childdocname|
% (which is set by the first inclusion of |childdoc.def|),
% |\ifchilddoc| is set to true, |\includeonly| is applied to the child file
% and |\jobname| is set to the main file
% (for proper handling of |.aux| files):
%    \begin{macrocode}
\newcommand{\childdocmain}[1]
{
  \childdocdisable\childdocmain{}
  \if?#1?\else
    \begingroup
      \def\childdoctmp{#1}
      \ifx\childdoctmp\childdocname
        \def\childdoctmp{}
      \else
        \def\childdoctmp
        {
          \childdoctrue
          \includeonly{\childdocname}
          \def\childdocjob{#1}
          \def\jobname{#1}
        }
      \fi
      \expandafter
    \endgroup
    \childdoctmp
  \fi
}
%    \end{macrocode}

% \macro{\childdocof}
% The command |\childdocof| redirects
% compilation to the main file |#1|.
%    \begin{macrocode}
\newcommand{\childdocof}[1]
{
  \childdocdisable
  \childdoctrue
  \includeonly{\childdocname}
  \def\jobname{#1}
  \def\childdocjob{#1}
  \input{#1}
}
%    \end{macrocode}

% \macro{\childdocby}
% The command |\childdocby| ....
%    \begin{macrocode}
\newcommand{\childdocby}[2][]
{
  \childdocdisable
  \childdoctrue
  \childdocmanualtrue
  \if?#1?\else
    \def\jobname{#2}
  \fi
  \def\childdocjob{#2}
  \input{#2}
  \endinput
}
%    \end{macrocode}

% \macro{\childdocforward}
% The command |\childdocforward| redirects
% compilation to the main file or
% (if the optional argument is given) a child file.
% Parameters are set as if the main file
% or a child file starting with |\childdocof| was compiled.
% Then compilation is handed over to the main file:
%    \begin{macrocode}
\newcommand{\childdocforward}[2][]
{
  \begingroup
    \if?#1?
      \def\childdoctmp
      {
        \def\childdocname{#2}
        \def\childdocjob{#2}
        \def\jobname{#2}
        \input{#2}
        \endinput
      }
    \else
      \def\childdoctmp
      {
        \childdocdisable
        \def\childdocname{#2}
        \childdoctrue
        \includeonly{#2}
        \def\childdocjob{#1}
        \def\jobname{#1}
        \input{#1}
        \endinput
      }
    \fi
    \expandafter
  \endgroup
  \childdoctmp
}
%    \end{macrocode}

% \macro{\childdocforwardprefix}
% The command |\childdocforwardprefix| redirects
% compilation to the main or a child file by means of a pattern.
% The prefix |#1| in the current filename is replaced by |#2|
% and the suffix of the current filename is kept
% (it is assumed that the filename does not contain the substring `|~~~|'
% which is used as a delimiter).
% Compilation is handed over to the new file by |\childdocforward|:
%    \begin{macrocode}
\newcommand{\childdocforwardprefix}[3][]
{
  \begingroup
    \def\childdocextract #2##1~~~{\def\childdoctmp{\childdocforward[#1]{#3##1}}}
    \expandafter\childdocextract\childdocname~~~
    \expandafter
  \endgroup
  \childdoctmp
}
%    \end{macrocode}

% \macro{\childdoc}
% The deprecated macro |\childdoc| is a legacy version of |\childdocmain|:
%    \begin{macrocode}
\newcommand{\childdoc}{\childdocmain}
%    \end{macrocode}

% \macro{\childdocredirect}
% The deprecated macro |\childdocredirect| is a legacy version
% of |\childdocforward| and |\childdocforwardprefix|:
%    \begin{macrocode}
\newcommand{\childdocredirect}[2][]
{
  \begingroup
    \if?#1?
      \def\childdoctmp{\childdocforward{#2}}
    \else
      \def\childdoctmp{\childdocforwardprefix{#1}{#2}}
    \fi
    \expandafter
  \endgroup
  \childdoctmp
}
%    \end{macrocode}

%\iffalse
%</package>
%\fi
%
\endinput
\childdocforward[|\textit{main}|]{|\textit{dest}|}"|
\end{center}
%
Here \textit{target} is the name of the output file,
\textit{main} is the name of the main file
and \textit{dest} is the name of the main or child file to be processed
(all filenames without extensions).
The optional argument \textit{main} can be omitted
if \textit{main} matches \textit{dest}.
Optionally, compilation \textit{flags} can be defined via |\def| commands.
This command line makes the \TeX{} engine believe
it is compiling the file \textit{target}
whose content is specified as the latter parameter.
The provided code then forwards the processing to
\textit{main} or \textit{dest} as described in \secref{sec:forward}.

%%%%%%%%%%%%%%%%%%%%%%%%%%%%%%%%%%%%%%%%%%%%%%%%%%%%%%%%%%%%%%%%%%%%%%%%%%%%%%%%
\subsection{Include by Input}
\label{sec:input}

Including child documents by |\include| has some restrictions by design.
Most notably, the content of a child document always occupies
its own set of pages; pages cannot be shared between child documents.
Usually, this behaviour makes perfect sense
because each child document contain an essential part of the document.
However, in some situations it may be desirable to compose
a document from a collection of parts
without having mandatory page breaks between then.
For this case, the package
provides a mechanism to include parts
by |\input| which can also be processed individually.
However, by construction this mechanism
requires manual handling of the content to be output.

%%%%%%%%%%%%%%%%%%%%%%%%%%%%%%%%%%%%%%%%
\DescribeMacro{\ifchilddocmanual}
The main file should be prepared as usual, see \secref{sec:include}.
However, the document body must make a distinction
between processing of an individual part and of the main document, e.g.:
%
\begin{center}
\begin{tabular}{l}
|\ifchilddocmanual|\\
|\input{\childdocname}|\\
|\||else|\\
\textit{document body with }|\input{|\textit{part}|}|\\
|\||fi|
\end{tabular}
\end{center}
%
The conditional |\ifchilddocmanual| is true whenever
a part to be included by |\input| is being compiled,
and the name of the part is stored in |\childdocname|.

%%%%%%%%%%%%%%%%%%%%%%%%%%%%%%%%%%%%%%%%
\DescribeMacro{\childdocby}
Each part to be included by |\input| should start with:
%
\begin{center}
\begin{tabular}{l}
|% \iffalse
%
% childdoc.dtx Copyright (C) 2017-2018 Niklas Beisert
%
% This work may be distributed and/or modified under the
% conditions of the LaTeX Project Public License, either version 1.3
% of this license or (at your option) any later version.
% The latest version of this license is in
%   http://www.latex-project.org/lppl.txt
% and version 1.3 or later is part of all distributions of LaTeX
% version 2005/12/01 or later.
%
% This work has the LPPL maintenance status `maintained'.
%
% The Current Maintainer of this work is Niklas Beisert.
%
% This work consists of the files childdoc.dtx and childdoc.ins
% and the derived files childdoc.def and cdocsamp.tex with
% cdocsch1.tex, cdocsch2.tex, cdocsdrf.tex, cdocsfn1.tex, cdocsfn2.tex.
%
%<package>\ifdefined\childdocmain\endinput\fi
%<package>\ProvidesFile{childdoc.def}[2018/12/30 v2.0 child document driver]
%<samplemain>\ProvidesFile{cdocsamp.tex}[2018/12/30 v2.0 sample for childdoc]
%<*driver>
%\ProvidesFile{childdoc.drv}[2018/12/30 v2.0 childdoc reference manual file]
\PassOptionsToClass{10pt,a4paper}{article}
\documentclass{ltxdoc}

\usepackage[margin=35mm]{geometry}
\usepackage{hyperref}
\usepackage{hyperxmp}
\usepackage[usenames]{color}

\hypersetup{colorlinks=true}
\hypersetup{pdfstartview=FitH}
\hypersetup{pdfpagemode=UseNone}
\hypersetup{pdfsource={}}
\hypersetup{pdflang={en-UK}}
\hypersetup{pdfcopyright={Copyright 2017-2018 Niklas Beisert.
  This work may be distributed and/or modified under the
  conditions of the LaTeX Project Public License, either version 1.3
  of this license or (at your option) any later version.}}
\hypersetup{pdflicenseurl={http://www.latex-project.org/lppl.txt}}
\hypersetup{pdfcontactaddress={ETH Zurich, ITP, HIT K,
  Wolfgang-Pauli-Strasse 27}}
\hypersetup{pdfcontactpostcode={8093}}
\hypersetup{pdfcontactcity={Zurich}}
\hypersetup{pdfcontactcountry={Switzerland}}
\hypersetup{pdfcontactemail={nbeisert@itp.phys.ethz.ch}}
\hypersetup{pdfcontacturl={http://people.phys.ethz.ch/\xmptilde nbeisert/}}

\newcommand{\secref}[1]{\hyperref[#1]{section \ref*{#1}}}

\parskip1ex
\parindent0pt
\let\olditemize\itemize
\def\itemize{\olditemize\parskip0pt}

\begin{document}

\title{The \textsf{childdoc} Package}
\hypersetup{pdftitle={The childdoc Package}}
\author{Niklas Beisert\\[2ex]
  Institut f\"ur Theoretische Physik\\
  Eidgen\"ossische Technische Hochschule Z\"urich\\
  Wolfgang-Pauli-Strasse 27, 8093 Z\"urich, Switzerland\\[1ex]
  \href{mailto:nbeisert@itp.phys.ethz.ch}
  {\texttt{nbeisert@itp.phys.ethz.ch}}}
\hypersetup{pdfauthor={Niklas Beisert}}
\hypersetup{pdfsubject={Manual for the LaTeX2e Package childdoc}}
\date{30 December 2018, \textsf{v2.0}}
\maketitle

\begin{abstract}\noindent
\textsf{childdoc} is a \LaTeXe{} package
that enables the direct compilation
of document sections included by |\include|
to individual files.
\end{abstract}

\begingroup
\parskip0ex
\tableofcontents
\endgroup

%%%%%%%%%%%%%%%%%%%%%%%%%%%%%%%%%%%%%%%%%%%%%%%%%%%%%%%%%%%%%%%%%%%%%%%%%%%%%%%%
%%%%%%%%%%%%%%%%%%%%%%%%%%%%%%%%%%%%%%%%%%%%%%%%%%%%%%%%%%%%%%%%%%%%%%%%%%%%%%%%
\section{Introduction}

\LaTeX{} provides a mechanism to structure a large document (such as a book)
into a main file and several child files (containing the chapters)
using the |\include| command.
This mechanism is beneficial for documents
which span hundreds of pages in order to
make the source file(s) more manageable.
Moreover, compilation can be restricted to
selected child files by means of the |\includeonly| command.
The latter feature can be used to reduce the compilation time while editing
(this was significantly more useful in the earlier days of \LaTeX{})
or to generate a smaller document which is easier to navigate.
Another application of |\includeonly| is to generate
documents consisting of selected parts of the complete document.

However, there are a few drawbacks of the plain |\include| mechanism:
\begin{itemize}
\item
The child files cannot be compiled on their own,
they can only be compiled via the main file.
A naive editing environment
(such as a text editor with an option
to have the current file processed by \LaTeX)
may require one to switch to the main file before compiling;
attempting to compile the child file produces errors.
\item
The main file must be modified (each time)
to adjust the |\includeonly| command
to the present needs. This easily leaves the main file in a messy state.
\item
The generated document will always carry the filename
of the main document. This is inconvenient if
several child files are to be compiled and
to be kept for distribution.
\end{itemize}

The present package provides a simple interface
to make child files individually compilable by \LaTeX{}.
Compiling a child file then has the same effect as compiling
the main file with an |\includeonly| command
to select the appropriate child.
Moreover the generated document will carry the name of the child
rather than the main file.
This resolves all three above issues.

This feature is meant to make the editing of books,
thesis documents and lecture notes somewhat more convenient.
However, the package can also be used efficiently for
composing a series of documents (such as exercise sheets)
which are typically distributed individually.
It then assists the author in generating the individual documents
(potentially in different versions)
as well as a document containing the collected series.
Another application is in developing style files
or other kinds of included material
where compilation of the style file could redirect
to a sample or test file.

%%%%%%%%%%%%%%%%%%%%%%%%%%%%%%%%%%%%%%%%%%%%%%%%%%%%%%%%%%%%%%%%%%%%%%%%%%%%%%%%
%%%%%%%%%%%%%%%%%%%%%%%%%%%%%%%%%%%%%%%%%%%%%%%%%%%%%%%%%%%%%%%%%%%%%%%%%%%%%%%%
\section{Usage}

First of all, the package \textsf{childdoc} is \emph{not} a standard
\LaTeXe{} |.sty| style file! Therefore it needs to be invoked in
a non-standard way.

%%%%%%%%%%%%%%%%%%%%%%%%%%%%%%%%%%%%%%%%%%%%%%%%%%%%%%%%%%%%%%%%%%%%%%%%%%%%%%%%
\subsection{Included Files}
\label{sec:include}

%%%%%%%%%%%%%%%%%%%%%%%%%%%%%%%%%%%%%%%%
\DescribeMacro{\childdocmain}
To use the package, add the commands
\begin{center}
\begin{tabular}{l}
|\input{childdoc.def}|\\
|\childdocmain{}|\\
\end{tabular}
\end{center}
at the very top of the main \LaTeX{} file,
in particular \emph{before} the |\documentclass| statement!
The argument of |\childdocmain| should be left empty
(but it must be present).

%%%%%%%%%%%%%%%%%%%%%%%%%%%%%%%%%%%%%%%%
\DescribeMacro{\childdocof}
Furthermore, add the commands
\begin{center}
\begin{tabular}{l}
|\input{childdoc.def}|\\
|\childdocof{|\textit{main}|}|\\
\end{tabular}
\end{center}
at the top of every child file \textit{child}
which is included by |\include{|\textit{child}|}|
from within the main file
(or at least for those files to be compiled individually).
The argument \textit{main} must be the filename of the main file.

There are a couple of
considerations in setting up the main and child documents:

%%%%%%%%%%%%%%%%%%%%%%%%%%%%%%%%%%%%%%%%
\paragraph{Restrictions.}

Please note the following restrictions:
\begin{itemize}
\item
|\childdocmain| must be called with one argument \textit{main}
to ensure compatibility with earlier version of the package.
It must either be empty (|\childdocmain{}|)
or precisely match the filename of the main file in which it is specified.
See \secref{sec:detection} for further information.
\item
The filename \textit{main} must be specified without the |.tex| extension.
\item
The filename \textit{main} is case sensitive
(even in case-insensitive file systems)
due to internal string comparison.
\item
The argument \textit{main} should be fully expanded, it cannot be a macro.
\item
Subdirectories and special characters should be avoided in filenames.
\item
The command |\childdocmain{|\textit{main}|}| must be followed by a whitespace.
It should not be followed immediately by another command
or by a comment mark `|%|'.
This is because the \TeX{} parser reads the token immediately following
the argument of |\childdocmain| and puts it
at the beginning of every child section;
however, a white\-space is ignored.
\end{itemize}

%%%%%%%%%%%%%%%%%%%%%%%%%%%%%%%%%%%%%%%%
\paragraph{Content of Main File.}

It is advisable to place all content in the child files included by |\include|.
Any output contained in the main file will appear in all child documents
unless suppressed manually;
it cannot be suppressed automatically by the |\includeonly| directive
and thus should normally be avoided.
A method to include some content in the main file
by means of conditional processing is described in \secref{sec:conditional}.

%%%%%%%%%%%%%%%%%%%%%%%%%%%%%%%%%%%%%%%%
\paragraph{Page Numbering.}

When only a part of the document is compiled,
the appropriate numbering of pages
(as well as other status parameters)
is determined from the |.aux| files.
The latter contain information from previous passes.
However this information needs to propagate through
all intermediate child documents.
Therefore the page numbering in child documents may well
be inconsistent until the complete document is compiled at least once.

A useful (if unconventional) way to always ensure a consistent
page numbering is to restart the numbering in each child document
and denote the pages by `\textit{child}|.|\textit{page}'
where \textit{child} represents the chapter/section number of the child file.
This can be achieved by the command
|\numberwithin{page}{|\textit{child}|}|
of the \textsf{amsmath} package
where \textit{child} can be |chapter| or |section|
depending on the chosen structuring.
Alternatively, one can modify the macro |\thepage| appropriately
and reset the counter |page| at the start of each child file.

%%%%%%%%%%%%%%%%%%%%%%%%%%%%%%%%%%%%%%%%%%%%%%%%%%%%%%%%%%%%%%%%%%%%%%%%%%%%%%%%
\subsection{Conditional Processing}
\label{sec:conditional}

The package provides a mechanism to compile different versions
of a document. To customise the versions further some conditional processing
can come in handy to distinguish which version is being compiled.
The package provides two macros to describe the compilation context:

%%%%%%%%%%%%%%%%%%%%%%%%%%%%%%%%%%%%%%%%
\DescribeMacro{\ifchilddoc}
The conditional |\ifchilddoc| distinguishes between the compilation of
child documents and the main document:
%
\begin{center}
|\ifchilddoc |\textit{child-code}| |[|\||else |\textit{main-code}]| \||fi|
\end{center}

%%%%%%%%%%%%%%%%%%%%%%%%%%%%%%%%%%%%%%%%
\DescribeMacro{\childdocname}
\DescribeMacro{\childdocjob}
The macro |\childdocname| contains the filename (without extension)
of the main or child file being processed.
Note that |\childdocjob| will always contain the name of the main file.

%%%%%%%%%%%%%%%%%%%%%%%%%%%%%%%%%%%%%%%%
\paragraph{Title Page.}

Conditional processing can be used to include a title or banner page
in the main document when proper precautions are taken.
Importantly, the code in the main file should ensure that the page counter
(as well as other status parameters which are stored in the |.aux| files)
takes the same value after the conditional processing.
Otherwise the page numbers may take divergent values
depending on which part is compiled.

For example, a title page could be declared by:
%
\begin{center}
\begin{tabular}{l}
|\ifchilddoc\||else|\\
|\addtocounter{page}{-1}|\\
\textit{code for title page}\\
|\newpage|\\
|\||fi|
\end{tabular}
\end{center}
%
A banner page for the child documents can be generated by:
%
\begin{center}
\begin{tabular}{l}
|\ifchilddoc|\\
|\addtocounter{page}{-1}|\\
\textit{code for banner page}\\
|\newpage|\\
|\||fi|
\end{tabular}
\end{center}
%
Here one could write a message such as:
\begin{center}
|This is the part \childdocname{} of \childdocjob{}.|
\end{center}

%%%%%%%%%%%%%%%%%%%%%%%%%%%%%%%%%%%%%%%%%%%%%%%%%%%%%%%%%%%%%%%%%%%%%%%%%%%%%%%%
\subsection{Flags}
\label{sec:flags}

The package makes it easy to generate different versions
of the main or child documents.
To this end compilation flags can be defined
and assigned different default values.
They will be particularly useful in conjunction
with the forwarding mechanism described in \secref{sec:forward}.

For example, it may be useful to have a flag |\version|
which can be set to |draft| or |final|.
The document source will contain some conditional code
depending on the value of |\version|.
Suppose further, the flag should default to |final| for the main file
and to |draft| for child files
which is a natural assignment for editing the document.
This is achieved by placing the following code
in the preamble of the main document
(below the |\childdocmain| directive):
%
\begin{center}
\begin{tabular}{l}
|\ifchilddoc|\\
|\providecommand{\version}{draft}|\\
|\||else|\\
|\providecommand{\version}{final}|\\
|\||fi|
\end{tabular}
\end{center}
%
The definition by |\providecommand| makes sure
that previous definitions are not overwritten.
Further statements |\providecommand{\version}{...}|
can thus be added before the above code to override it.

For the main file, one might add a line
(between |\childdocmain| and the above block)
%
\begin{center}
|%\ifchilddoc\||else\providecommand{\version}{draft}\||fi|
\end{center}
%
which can be uncommented to produce a draft version.
Likewise one can add a line to the very top of a child file
(above the |\childdocof{|\textit{main}|}| directive)
%
\begin{center}
|%\providecommand{\version}{final}|
\end{center}
%
which can be uncommented to produce the final version of this child document.

%%%%%%%%%%%%%%%%%%%%%%%%%%%%%%%%%%%%%%%%%%%%%%%%%%%%%%%%%%%%%%%%%%%%%%%%%%%%%%%%
\subsection{Forwarding}
\label{sec:forward}

Different versions of the main or child documents
using compilation flags as described in \secref{sec:flags}
can be (permanently) stored in different files
for convenient compilation, viewing and distribution.
To this end, the package defines a command
to pass on compilation to a different file:

%%%%%%%%%%%%%%%%%%%%%%%%%%%%%%%%%%%%%%%%
\DescribeMacro{\childdocforward}
The command |\childdocforward| redirects processing to
another source file:
%
\begin{center}
\begin{tabular}{l}
|\input{childdoc.def}|\\
|\childdocforward[|\textit{main}|]{|\textit{dest}|}|\\
\end{tabular}
\end{center}
%
The argument \textit{dest} is the destination file
(without extension).
It should be the main file or one of the child files.
Note that further \textsf{childdoc} directives
such as |\childdocof| and |\childdocforward|
in the indicated file will be processed in this form.
The optional argument \textit{main}
passes on directly to the main file \textit{main}
while pretending to compile the child \textit{dest}.
This form behaves as if \textit{dest}
issues |\childdocof{|\textit{main}|}| right away,
and no further \textsf{childdoc} directives will be processed.

%%%%%%%%%%%%%%%%%%%%%%%%%%%%%%%%%%%%%%%%
\DescribeMacro{\...prefix}
In the alternative form |\childdocforwardprefix|,
%
\begin{center}
\begin{tabular}{l}
|\input{childdoc.def}|\\
|\childdocforwardprefix[|\textit{main}|]{|\textit{prefix}|}{|\textit{dest}|}|
\end{tabular}
\end{center}
%
the destination file is determined by a pattern
depending on the current file:
To make this work, the current file must be called
`{\textit{prefix}\hspace{0.2em}\textit{suffix}}'
with \textit{prefix} matching precisely the argument.
Processing is then passed on to the file
`{\textit{dest}\hspace{0.2em}\textit{suffix}}'.
Surely, the same effect is achieved by
directly specifying the
argument `{\textit{dest}\hspace{0.2em}\textit{suffix}}'
in the first form.
However, that requires to set up a different file
for each child. With the alternative form of the command
all these files can have exactly the same content
which simplifies setting them up and maintaining them.

For example, the following file |draft.tex|
with a compilation flag |\version| as described in \secref{sec:flags}
compiles the main document as a draft:
%
\begin{center}
\begin{tabular}{l}
|\def\version{draft}|\\
|\input{childdoc.def}|\\
|\childdocforward{|\textit{main}|}|
\end{tabular}
\end{center}
%
Likewise, the following files |final|\textit{nn}|.tex|
compile the final version of the child document
|child|\textit{nn}|.tex|:
%
\begin{center}
\begin{tabular}{l}
|\def\version{final}|\\
|\input{childdoc.def}|\\
|\childdocforwardprefix{final}{child}|
\end{tabular}
\end{center}
%

Note that when several versions of a main file and/or of each child file
are to be generated, it may be convenient to set up a |Makefile| or
shell script to automatise the process.

%%%%%%%%%%%%%%%%%%%%%%%%%%%%%%%%%%%%%%%%%%%%%%%%%%%%%%%%%%%%%%%%%%%%%%%%%%%%%%%%
\subsection{Command Line Processing}
\label{sec:commandline}

The effect of redirection files can also be achieved by invoking
the \LaTeX{} compiler with a more elaborate command line.
Most conveniently this should be done as part
of a shell script or a |Makefile|.

When using \textsf{childdoc} in the main file, the following
command lines effectively perform a redirection
(note that depending on the shell being used,
backslashes may have to be doubled: `|\|' $\to$ `|\\|'):
%
\begin{center}
|... -jobname "|\textit{target}|" |\\|"|[\textit{flags}]%
|\input{childdoc.def}\childdocforward[|\textit{main}|]{|\textit{dest}|}"|
\end{center}
%
Here \textit{target} is the name of the output file,
\textit{main} is the name of the main file
and \textit{dest} is the name of the main or child file to be processed
(all filenames without extensions).
The optional argument \textit{main} can be omitted
if \textit{main} matches \textit{dest}.
Optionally, compilation \textit{flags} can be defined via |\def| commands.
This command line makes the \TeX{} engine believe
it is compiling the file \textit{target}
whose content is specified as the latter parameter.
The provided code then forwards the processing to
\textit{main} or \textit{dest} as described in \secref{sec:forward}.

%%%%%%%%%%%%%%%%%%%%%%%%%%%%%%%%%%%%%%%%%%%%%%%%%%%%%%%%%%%%%%%%%%%%%%%%%%%%%%%%
\subsection{Include by Input}
\label{sec:input}

Including child documents by |\include| has some restrictions by design.
Most notably, the content of a child document always occupies
its own set of pages; pages cannot be shared between child documents.
Usually, this behaviour makes perfect sense
because each child document contain an essential part of the document.
However, in some situations it may be desirable to compose
a document from a collection of parts
without having mandatory page breaks between then.
For this case, the package
provides a mechanism to include parts
by |\input| which can also be processed individually.
However, by construction this mechanism
requires manual handling of the content to be output.

%%%%%%%%%%%%%%%%%%%%%%%%%%%%%%%%%%%%%%%%
\DescribeMacro{\ifchilddocmanual}
The main file should be prepared as usual, see \secref{sec:include}.
However, the document body must make a distinction
between processing of an individual part and of the main document, e.g.:
%
\begin{center}
\begin{tabular}{l}
|\ifchilddocmanual|\\
|\input{\childdocname}|\\
|\||else|\\
\textit{document body with }|\input{|\textit{part}|}|\\
|\||fi|
\end{tabular}
\end{center}
%
The conditional |\ifchilddocmanual| is true whenever
a part to be included by |\input| is being compiled,
and the name of the part is stored in |\childdocname|.

%%%%%%%%%%%%%%%%%%%%%%%%%%%%%%%%%%%%%%%%
\DescribeMacro{\childdocby}
Each part to be included by |\input| should start with:
%
\begin{center}
\begin{tabular}{l}
|\input{childdoc.def}|\\
|\childdocby{|\textit{main}|}|\\
\end{tabular}
\end{center}
%
The directive |\childdocby| is similar to |\childdocof|
described in \secref{sec:include},
but the subsequent selection of content must be done manually.
To that end, both |\ifchilddoc| and |\ifchilddocmanual|
will be true upon processing of a part,
and the name of the part is stored in |\childdocname|.
Note that |\jobname| will be set to the filename of the current part
so that each part receives an individual |.aux| file
that does not interfere with the |.aux| file(s) of the main document.
This behaviour can be altered by the alternative form
|\childdocby[*]{|\textit{main}|}| (with a non-empty optional argument)
which uses the |.aux| file of the main document
by setting |\jobname| to \textit{main}.

%%%%%%%%%%%%%%%%%%%%%%%%%%%%%%%%%%%%%%%%%%%%%%%%%%%%%%%%%%%%%%%%%%%%%%%%%%%%%%%%
\subsection{Driver Development}
\label{sec:driver}

The \textsf{childdoc} mechanism can also be use for the development
of definition files such as \LaTeX{} styles or classes.
This case differs from the above setup with multiple parts
included by |\include| in that no |\includeonly| should be invoked.
This can be achieved by starting the include file
(before |\ProvidesPackage|) with:
%
\begin{center}
\begin{tabular}{l}
|\input{childdoc.def}|\\
|\childdocforward{|\textit{main}|}|\\
\end{tabular}
\end{center}
%
or alternatively with:
%
\begin{center}
\begin{tabular}{l}
|\input{childdoc.def}|\\
|\childdocby{|\textit{main}|}|\\
\end{tabular}
\end{center}
%
Both forms have slightly different effects as described above.
The main file is prepared as usual, see \secref{sec:include}.

%%%%%%%%%%%%%%%%%%%%%%%%%%%%%%%%%%%%%%%%%%%%%%%%%%%%%%%%%%%%%%%%%%%%%%%%%%%%%%%%
\subsection{Legacy Detection}
\label{sec:detection}

The directive |\childdocmain| in the main file can detect
whether the complete document or merely a child is to be compiled
even without using the directive |\childdocof|.
This method is deprecated because it is less robust
and there is no compelling reason to use it;
it is merely provided for backward compatibility
and it may be removed in future versions.

If the detection mechanism is to be used,
it is mandatory to correctly specify
the filename of the main file as the argument of |\childdocmain|:
%
\begin{center}
\begin{tabular}{l}
|\input{childdoc.def}|\\
|\childdocmain{|\textit{main}|}|\\
\end{tabular}
\end{center}
%
If |\jobname| does not match the argument \textit{main} of |\childdocmain|,
it is assumed that |\jobname| points to the child file to be compiled.
When using |\childdocmain| with the main file specified as argument,
it suffices to start a child file
with just |\input{|\textit{main}|}|
without loading of the package and using |\childdocof|.
If instead all processing is done
with the appropriate \textsf{childdoc} directives,
the argument of \textit{main} of |\childdocmain| can be empty.

An alternative version of the command line processing described
in \secref{sec:commandline} using the detection mechanism reads:
%
\begin{center}
|... -jobname "|\textit{target}|" "|[\textit{flags}]%
[|\def\jobname{|\textit{dest}|}|]|\input{|\textit{main}|}"|
\end{center}

%%%%%%%%%%%%%%%%%%%%%%%%%%%%%%%%%%%%%%%%%%%%%%%%%%%%%%%%%%%%%%%%%%%%%%%%%%%%%%%%
\subsection{Manual Code}
\label{sec:manual}

In case one cannot be certain whether the definitions file |childdoc.def|
is installed on the target \TeX{} distribution
and one prefers not to ship it,
it is conceivable to paste a few relevant commands into the sources.

To that end, drop all statements |\input{childdoc.def}|
and perform the replacements as outlined below.
Instead of |\childdocmain{|\textit{main}|}| add the following code
to the top of the main file:
%
\begin{center}
\begin{tabular}{l}
|\||ifdefined\childdocname\endinput\||fi\newif\ifchilddoc|\\
|\edef\childdocname{\scantokens\expandafter{\jobname\noexpand}}|\\
|\def\childdocmain{|\textit{main}|}\||ifx\childdocmain\childdocname\||else|\\
|\childdoctrue\includeonly{\childdocname}\let\jobname\childdocmain\||fi|\\
\end{tabular}
\end{center}
%
Instead of |\childdocof{|\textit{main}|}| just include the main file
at the top of each child file:
%
\begin{center}
|\input{|\textit{main}|}|
\end{center}
%
A simple redirection |\childdocforward{|\textit{dest}|}| is achieved by:
%
\begin{center}
|\def\jobname{|\textit{dest}|}\input{\jobname}|
\end{center}
%
The redirection with prefix
|\childdocforwardprefix[|\textit{prefix}|]{|\textit{dest}|}|
is accomplished by:
%
\begin{center}
\begin{tabular}{l}
|{\edef\jobname{\scantokens\expandafter{\jobname\noexpand}}|\\
|\def\redirectjob |\textit{prefix}|#1~~~{\gdef\jobname{|\textit{dest}|#1}}|\\
|\expandafter\redirectjob\jobname~~~}\input{\jobname}|
\end{tabular}
\end{center}

In an alternative approach,
child documents can be compiled by a specific command line
without additional code or specific definitions:
%
\begin{center}
|... -jobname "|\textit{target}|" "|[\textit{flags}]%
|\includeonly{|\textit{dest}|}\input{|\textit{main}|}"|
\end{center}
%

%%%%%%%%%%%%%%%%%%%%%%%%%%%%%%%%%%%%%%%%%%%%%%%%%%%%%%%%%%%%%%%%%%%%%%%%%%%%%%%%
%%%%%%%%%%%%%%%%%%%%%%%%%%%%%%%%%%%%%%%%%%%%%%%%%%%%%%%%%%%%%%%%%%%%%%%%%%%%%%%%
\section{Information}

%%%%%%%%%%%%%%%%%%%%%%%%%%%%%%%%%%%%%%%%%%%%%%%%%%%%%%%%%%%%%%%%%%%%%%%%%%%%%%%%
\subsection{Copyright}

Copyright \copyright{} 2017--2018 Niklas Beisert

This work may be distributed and/or modified under the
conditions of the \LaTeX{} Project Public License, either version 1.3
of this license or (at your option) any later version.
The latest version of this license is in
  \url{http://www.latex-project.org/lppl.txt}
and version 1.3 or later is part of all distributions of \LaTeX{}
version 2005/12/01 or later.

This work has the LPPL maintenance status `maintained'.

The Current Maintainer of this work is Niklas Beisert.

This work consists of the files |README.txt|, |childdoc.ins| and |childdoc.dtx|
as well as the derived files |childdoc.def|, |cdocsamp.tex|
with |cdocsch1.tex|, |cdocsch2.tex|, |cdocspt3.tex|, |cdocspt4.tex|,
|cdocsdrf.tex|, |cdocsfn1.tex|, |cdocsfn2.tex|
as well as |childdoc.pdf|.

%%%%%%%%%%%%%%%%%%%%%%%%%%%%%%%%%%%%%%%%%%%%%%%%%%%%%%%%%%%%%%%%%%%%%%%%%%%%%%%%
\subsection{Files and Installation}

The package consists of the files:
%
\begin{center}
\begin{tabular}{ll}
    |README.txt|   & readme file \\
    |childdoc.ins| & installation file \\
    |childdoc.dtx| & source file \\
    |childdoc.def| & definition file \\
    |cdocsamp.tex| & sample main file \\
    |cdocsch1.tex| & sample include file \\
    |cdocsch2.tex| & sample include file \\
    |cdocspt3.tex| & sample part file \\
    |cdocspt4.tex| & sample part file \\
    |cdocsdrf.tex| & sample redirection file \\
    |cdocsfn1.tex| & sample redirection file \\
    |cdocsfn2.tex| & sample redirection file \\
    |childdoc.pdf| & manual
\end{tabular}
\end{center}
%
The distribution consists of the files
|README.txt|, |childdoc.ins| and |childdoc.dtx|.
%
\begin{itemize}
\item
Run (pdf)\LaTeX{} on |childdoc.dtx|
to compile the manual |childdoc.pdf| (this file).
\item
Run \LaTeX{} on |childdoc.ins| to create the definitions file |childdoc.def|
and the sample |cdocsamp.tex| with include files
|cdocsch1.tex|, |cdocsch2.tex|, |cdocspt3.tex|, |cdocspt4.tex|,
|cdocsdrf.tex|, |cdocsfn1.tex|, |cdocsfn2.tex|.
Then copy the file |childdoc.def| to an appropriate directory of your \LaTeX{}
distribution, e.g.\ \textit{texmf-root}|/tex/latex/childdoc|.
\end{itemize}

%%%%%%%%%%%%%%%%%%%%%%%%%%%%%%%%%%%%%%%%%%%%%%%%%%%%%%%%%%%%%%%%%%%%%%%%%%%%%%%%
\subsection{Related CTAN Packages}

There are several other packages which offer a similar functionality:
%
\begin{itemize}
\item
The packages
\href{http://ctan.org/pkg/docmute}{\textsf{docmute}},
\href{http://ctan.org/pkg/includex}{\textsf{includex}} and
\href{http://ctan.org/pkg/standalone}{\textsf{standalone}}
provide commands to include only the document body of
a child file thus allowing both files to be compiled individually.
\item
The packages \href{http://ctan.org/pkg/subdocs}{\textsf{subdocs}}
and \href{http://ctan.org/pkg/subfiles}{\textsf{subfiles}}
provide structures in which the main and child documents can be
encapsulated and allowing them to be compiled individually.
The inclusion mechanism is different from the conventional |\include|.
\item
The package \href{http://ctan.org/pkg/combine}{\textsf{combine}}
is an elaborate solution to combine several documents into one.
\end{itemize}
%
See also the CTAN topic \href{http://ctan.org/topic/subdocs}{\textsf{subdocs}}
for further related packages.
The present package differs from the above solutions in that
a document structure constructed with the conventional |\include| mechanism
just needs two extra commands at the top of every file
such that all constituent files can be compiled individually.

%%%%%%%%%%%%%%%%%%%%%%%%%%%%%%%%%%%%%%%%%%%%%%%%%%%%%%%%%%%%%%%%%%%%%%%%%%%%%%%%
%\subsection{Feature Suggestions}
%
%The following is a list of features which may be useful for future
%versions of this package:
%%
%\begin{itemize}
%\item
%\ldots
%\end{itemize}

%%%%%%%%%%%%%%%%%%%%%%%%%%%%%%%%%%%%%%%%%%%%%%%%%%%%%%%%%%%%%%%%%%%%%%%%%%%%%%%%
\subsection{Revision History}

%%%%%%%%%%%%%%%%%%%%%%%%%%%%%%%%%%%%%%%%
\paragraph{v2.0:} 2018/12/30

\begin{itemize}
\item
immediate forward processing
\item
added |\childdocby| mechanism
\item
manual restructured
\end{itemize}

%%%%%%%%%%%%%%%%%%%%%%%%%%%%%%%%%%%%%%%%
\paragraph{v1.6:} 2018/01/17

\begin{itemize}
\item
application for development of include files
\item
corrections to manual
\end{itemize}

%%%%%%%%%%%%%%%%%%%%%%%%%%%%%%%%%%%%%%%%
\paragraph{v1.5:} 2017/05/21

\begin{itemize}
\item
more complete structuring introduced
\item
|\childdocof| introduced
\item
|\childdoc| renamed to |\childdocmain|
\item
|\childredirect| renamed to |\childdocforward| and |\childdocforwardprefix|
and functionality expanded
\end{itemize}

%%%%%%%%%%%%%%%%%%%%%%%%%%%%%%%%%%%%%%%%
\paragraph{v1.0:} 2017/04/27

\begin{itemize}
\item
manual and install package
\item
first version published on CTAN
\end{itemize}

%%%%%%%%%%%%%%%%%%%%%%%%%%%%%%%%%%%%%%%%
\paragraph{v0.6:} 2017/04/26

\begin{itemize}
\item
redirection mechanism added
\end{itemize}

%%%%%%%%%%%%%%%%%%%%%%%%%%%%%%%%%%%%%%%%
\paragraph{v0.5:} 2017/04/26

\begin{itemize}
\item
functionality in definition file
\end{itemize}


%%%%%%%%%%%%%%%%%%%%%%%%%%%%%%%%%%%%%%%%%%%%%%%%%%%%%%%%%%%%%%%%%%%%%%%%%%%%%%%%
%%%%%%%%%%%%%%%%%%%%%%%%%%%%%%%%%%%%%%%%%%%%%%%%%%%%%%%%%%%%%%%%%%%%%%%%%%%%%%%%
%%%%%%%%%%%%%%%%%%%%%%%%%%%%%%%%%%%%%%%%%%%%%%%%%%%%%%%%%%%%%%%%%%%%%%%%%%%%%%%%
\appendix

\settowidth\MacroIndent{\rmfamily\scriptsize 000\ }

 \DocInput{childdoc.dtx}

\end{document}
%</driver>
% \fi
%
% %%%%%%%%%%%%%%%%%%%%%%%%%%%%%%%%%%%%%%%%%%%%%%%%%%%%%%%%%%%%%%%%%%%%%%%%%%%%%%
% %%%%%%%%%%%%%%%%%%%%%%%%%%%%%%%%%%%%%%%%%%%%%%%%%%%%%%%%%%%%%%%%%%%%%%%%%%%%%%
% \section{Sample}
%\iffalse
%<*samplemain>
%\fi
%
% The following presents a sample document
% with two chapters, two parts, a title page,
% a compile flag as well as three forwarding files to set the flag.
% It consists of eight |.tex| files:
% \begin{center}
% \begin{tabular}{ll}
% |cdocsamp.tex|&main file\\
% |cdocsch1.tex|&include file for chapter 1\\
% |cdocsch2.tex|&include file for chapter 2\\
% |cdocspt3.tex|&include file for part 3\\
% |cdocspt4.tex|&include file for part 4\\
% |cdocsdrf.tex|&forwarding file for main file in draft mode\\
% |cdocsfi1.tex|&forwarding file for final version of chapter 1\\
% |cdocsfi2.tex|&forwarding file for final version of chapter 2\\
% \end{tabular}
% \end{center}
% Each of the eight files can be compiled directly by the \LaTeX{} compiler.
%
% %%%%%%%%%%%%%%%%%%%%%%%%%%%%%%%%%%%%%%
% \paragraph{Main File.}
%
% The main file is called |cdocsamp.tex|.
%
% Load the \textsf{childdoc} definitions and
% declare the filename for the main document:
%    \begin{macrocode}
\input{childdoc.def}
\childdocmain{}
%    \end{macrocode}

% Optional override for |\version| flag:
%    \begin{macrocode}
%%\ifchilddoc\else\providecommand{\version}{draft}\fi
%    \end{macrocode}

% Define the default values for the |\version| flag
% (|final| for the main file and |draft| for childs):
%    \begin{macrocode}
\ifchilddoc
\providecommand{\version}{draft}
\else
\providecommand{\version}{final}
\fi
%    \end{macrocode}

% Load the standard document class:
%    \begin{macrocode}
\documentclass[12pt]{article}
%    \end{macrocode}

% Start the document body:
%    \begin{macrocode}
\begin{document}
%    \end{macrocode}

% Declare a title page.
% Print title, part of document being processed and version flag:
%    \begin{macrocode}
\addtocounter{page}{-1}
\begin{center}
{\LARGE\bfseries{}childdoc example\par}
\vspace{1cm}
\ifchilddoc
\ifchilddocmanual part\else chapter\fi:
`\childdocname' of `\childdocjob'\par
\else
main document: `\childdocjob'\par
\fi
version: \version\par
\end{center}
\newpage
%    \end{macrocode}

% Manually include selected file,
% otherwise process as usual:
%    \begin{macrocode}
\ifchilddocmanual
\section*{part `\childdocname'}
\input{\childdocname}
\else
%    \end{macrocode}

% Include the two chapters:
%    \begin{macrocode}
\include{cdocsch1}
\include{cdocsch2}
%    \end{macrocode}

% Include the two parts unless only chapters should be displayed:
%    \begin{macrocode}
\ifchilddoc\else
\section{part three}
\input{cdocspt3}
\section{part four}
\input{cdocspt4}
\fi
%    \end{macrocode}

% Process as usual until here:
%    \begin{macrocode}
\fi
%    \end{macrocode}

% End of document body:
%    \begin{macrocode}
\end{document}
%    \end{macrocode}
%\iffalse
%</samplemain>
%\fi
%
% %%%%%%%%%%%%%%%%%%%%%%%%%%%%%%%%%%%%%%
% \paragraph{Chapter Include Files.}
%
% The include files are called |cdocsch1.tex| and |cdocsch2.tex|.
%
%\iffalse
%<*samplechap1|samplechap2>
%\fi

% Optional override for |\version| flag:
%    \begin{macrocode}
%%\providecommand{\version}{final}
%    \end{macrocode}

% Include the main document:
%    \begin{macrocode}
\input{childdoc.def}
\childdocof{cdocsamp}
%    \end{macrocode}

%\iffalse
%</samplechap1|samplechap2>
%\fi
%
%\iffalse
%<*samplechap1>
%\fi
% Some text for chapter 1:
%    \begin{macrocode}
\section{one}
some text in chapter one
%    \end{macrocode}

%\iffalse
%</samplechap1>
%\fi
% Some text for chapter 2:
%\iffalse
%<*samplechap2>
%\fi
%    \begin{macrocode}
\section{two}
more text in chapter two
%    \end{macrocode}

%\iffalse
%</samplechap2>
%\fi
%
% %%%%%%%%%%%%%%%%%%%%%%%%%%%%%%%%%%%%%%
% \paragraph{Part Include Files.}
%
% The include files are called |cdocspt3.tex| and |cdocspt4.tex|.
%
%\iffalse
%<*samplepart3|samplepart4>
%\fi

% Optional override for |\version| flag:
%    \begin{macrocode}
%%\providecommand{\version}{final}
%    \end{macrocode}

% Include the main document:
%    \begin{macrocode}
\input{childdoc.def}
\childdocby{cdocsamp}
%    \end{macrocode}

%\iffalse
%</samplepart3|samplepart4>
%\fi
%
%\iffalse
%<*samplepart3>
%\fi
% Some text for part 3:
%    \begin{macrocode}
some text in part three
%    \end{macrocode}

%\iffalse
%</samplepart3>
%\fi
% Some text for part 4:
%\iffalse
%<*samplepart4>
%\fi
%    \begin{macrocode}
more text in part four
%    \end{macrocode}

%\iffalse
%</samplepart4>
%\fi
%
% %%%%%%%%%%%%%%%%%%%%%%%%%%%%%%%%%%%%%%
% \paragraph{Forwarding for a Complete Draft.}
%
% The following forwarding file |cdocsdrf.tex|
% compiles the main document in draft mode:
%\iffalse
%<*sampledraft>
%\fi
%    \begin{macrocode}
\def\version{draft}
\input{childdoc.def}
\childdocforward{cdocsamp}
%    \end{macrocode}

%\iffalse
%</sampledraft>
%\fi
%
% %%%%%%%%%%%%%%%%%%%%%%%%%%%%%%%%%%%%%%
% \paragraph{Forwarding for Final Version of the Chapters.}
%
% The following forwarding files |cdocsfn1.tex| and |cdocsfn2.tex|
% (with identical content)
% compile the final versions of the child documents
% |cdocsch1.tex| and |cdocsch2.tex|, respectively:
%\iffalse
%<*samplefinal>
%\fi
%    \begin{macrocode}
\def\version{final}
\input{childdoc.def}
\childdocforwardprefix[cdocsamp]{cdocsfn}{cdocsch}
%    \end{macrocode}

%\iffalse
%</samplefinal>
%\fi
%
% %%%%%%%%%%%%%%%%%%%%%%%%%%%%%%%%%%%%%%
% \paragraph{Command Line Processing.}
%
% The following three command lines generate the output files
% |cdocscld|, |cdocscl1| and |cdocscl2|
% which should be identical to
% |cdocsdrf|, |cdocsch1| and |cdocsfn2|, respectively:
% \begin{center}
% \begin{tabular}{l}
% |latex -jobname cdocscld \|\\
% |  "\def\version{draft}\input{childdoc.def}\childdocforward{cdocsamp}"|\\
% |latex -jobname cdocscl1 \|\\
% |  "\input{childdoc.def}\childdocforward[cdocsamp]{cdocsch1}"|\\
% |latex -jobname cdocscl2 \|\\
% |  "\def\version{final}\input{childdoc.def}\childdocforward{cdocsch2}"|
% \end{tabular}
% \end{center}
% Note that the trailing backslash on each first line
% merely continues the input to the second line
% (for convenient cut ant paste).
% Furthermore, the command |latex| can be replaced by any
% of its alternative versions such as |pdflatex|.
%
% %%%%%%%%%%%%%%%%%%%%%%%%%%%%%%%%%%%%%%%%%%%%%%%%%%%%%%%%%%%%%%%%%%%%%%%%%%%%%%
% %%%%%%%%%%%%%%%%%%%%%%%%%%%%%%%%%%%%%%%%%%%%%%%%%%%%%%%%%%%%%%%%%%%%%%%%%%%%%%
% \section{Implementation}
%\iffalse
%<*package>
%\fi
%
% This section describes the definitions file |childdoc.def|.

% The definitions cannot be loaded using |\usepackage| or |\RequirePackage|
% which has a mechanism to prevent loading a style file more than once.
% When loading the definitions by means of |\input|
% multiple instances have to be prevented manually:
%\iffalse
%This code needs to be before the `\ProvidesFile' directive
%which is defined at the beginning of this file.
%Therefore it is also placed there and commented out here.
%</package>
%<*discard>
%\fi
%    \begin{macrocode}
\ifdefined\childdocmain\endinput\fi
%    \end{macrocode}
%\iffalse
%</discard>
%<*package>
%\fi
%
% \macro{\ifchilddoc}
% \macro{\ifchilddocmanual}
% The conditional |\ifchilddoc| tells whether a
% child (true) or main (false) document is being compiled.
% The conditional |\ifchilddocmanual| tells whether
% the |\includeonly| mechanism is used (false) or
% the selection of child files must be performed manually (true).
% The definitions initialise to false:
%    \begin{macrocode}
\newif\ifchilddoc
\newif\ifchilddocmanual
%    \end{macrocode}

% \macro{\childdocname}
% \macro{\childdocjob}
% The macro |\childdocname| stores the name of the main document
% to be compiled. The macro |\childdocjob| stores the name of
% the document on which the \LaTeX{} compiler was originally invoked.
% The content of |\jobname| cannot be compared
% to filenames specified in the source due to different catcodes.
% The following code rescans |\jobname|, stores the result
% in |\childdocname| and saves a copy in |\childdocjob|:
%    \begin{macrocode}
\edef\childdocname{\scantokens\expandafter{\jobname\noexpand}}
\let\childdocjob\childdocname
%    \end{macrocode}

% \macro{\childdocdisable}
% The macro |\childdocdisable| prevents the main file
% from being processed more than once.
% At this stage, the main document command |\childdocmain|
% is assumed to be called once again where it should do nothing.
% Any subsequent call to it should prevent
% a secondary processing of the main document
% It overwrites the forwarding commands
% |\childdocof| and |\childdocforward|
% with empty macros to prevent further inclusions of the main document:
%    \begin{macrocode}
\newcommand{\childdocdisable}
{
  \renewcommand{\childdocmain}[1]{\renewcommand{\childdocmain}[1]{\endinput}}
  \renewcommand{\childdocof}[1]{}
  \renewcommand{\childdocby}[2][]{}
  \renewcommand{\childdocforward}[2][]{}
  \renewcommand{\childdocdisable}{}
}
%    \end{macrocode}

% \macro{\childdocmain}
% The macro |\childdocmain| is to be called at the top of the main file
% with nothing or the main filename (without extension) as argument.
% First, it breaks loops.
% If the argument is not empty and does not match |\childdocname|
% (which is set by the first inclusion of |childdoc.def|),
% |\ifchilddoc| is set to true, |\includeonly| is applied to the child file
% and |\jobname| is set to the main file
% (for proper handling of |.aux| files):
%    \begin{macrocode}
\newcommand{\childdocmain}[1]
{
  \childdocdisable\childdocmain{}
  \if?#1?\else
    \begingroup
      \def\childdoctmp{#1}
      \ifx\childdoctmp\childdocname
        \def\childdoctmp{}
      \else
        \def\childdoctmp
        {
          \childdoctrue
          \includeonly{\childdocname}
          \def\childdocjob{#1}
          \def\jobname{#1}
        }
      \fi
      \expandafter
    \endgroup
    \childdoctmp
  \fi
}
%    \end{macrocode}

% \macro{\childdocof}
% The command |\childdocof| redirects
% compilation to the main file |#1|.
%    \begin{macrocode}
\newcommand{\childdocof}[1]
{
  \childdocdisable
  \childdoctrue
  \includeonly{\childdocname}
  \def\jobname{#1}
  \def\childdocjob{#1}
  \input{#1}
}
%    \end{macrocode}

% \macro{\childdocby}
% The command |\childdocby| ....
%    \begin{macrocode}
\newcommand{\childdocby}[2][]
{
  \childdocdisable
  \childdoctrue
  \childdocmanualtrue
  \if?#1?\else
    \def\jobname{#2}
  \fi
  \def\childdocjob{#2}
  \input{#2}
  \endinput
}
%    \end{macrocode}

% \macro{\childdocforward}
% The command |\childdocforward| redirects
% compilation to the main file or
% (if the optional argument is given) a child file.
% Parameters are set as if the main file
% or a child file starting with |\childdocof| was compiled.
% Then compilation is handed over to the main file:
%    \begin{macrocode}
\newcommand{\childdocforward}[2][]
{
  \begingroup
    \if?#1?
      \def\childdoctmp
      {
        \def\childdocname{#2}
        \def\childdocjob{#2}
        \def\jobname{#2}
        \input{#2}
        \endinput
      }
    \else
      \def\childdoctmp
      {
        \childdocdisable
        \def\childdocname{#2}
        \childdoctrue
        \includeonly{#2}
        \def\childdocjob{#1}
        \def\jobname{#1}
        \input{#1}
        \endinput
      }
    \fi
    \expandafter
  \endgroup
  \childdoctmp
}
%    \end{macrocode}

% \macro{\childdocforwardprefix}
% The command |\childdocforwardprefix| redirects
% compilation to the main or a child file by means of a pattern.
% The prefix |#1| in the current filename is replaced by |#2|
% and the suffix of the current filename is kept
% (it is assumed that the filename does not contain the substring `|~~~|'
% which is used as a delimiter).
% Compilation is handed over to the new file by |\childdocforward|:
%    \begin{macrocode}
\newcommand{\childdocforwardprefix}[3][]
{
  \begingroup
    \def\childdocextract #2##1~~~{\def\childdoctmp{\childdocforward[#1]{#3##1}}}
    \expandafter\childdocextract\childdocname~~~
    \expandafter
  \endgroup
  \childdoctmp
}
%    \end{macrocode}

% \macro{\childdoc}
% The deprecated macro |\childdoc| is a legacy version of |\childdocmain|:
%    \begin{macrocode}
\newcommand{\childdoc}{\childdocmain}
%    \end{macrocode}

% \macro{\childdocredirect}
% The deprecated macro |\childdocredirect| is a legacy version
% of |\childdocforward| and |\childdocforwardprefix|:
%    \begin{macrocode}
\newcommand{\childdocredirect}[2][]
{
  \begingroup
    \if?#1?
      \def\childdoctmp{\childdocforward{#2}}
    \else
      \def\childdoctmp{\childdocforwardprefix{#1}{#2}}
    \fi
    \expandafter
  \endgroup
  \childdoctmp
}
%    \end{macrocode}

%\iffalse
%</package>
%\fi
%
\endinput
|\\
|\childdocby{|\textit{main}|}|\\
\end{tabular}
\end{center}
%
The directive |\childdocby| is similar to |\childdocof|
described in \secref{sec:include},
but the subsequent selection of content must be done manually.
To that end, both |\ifchilddoc| and |\ifchilddocmanual|
will be true upon processing of a part,
and the name of the part is stored in |\childdocname|.
Note that |\jobname| will be set to the filename of the current part
so that each part receives an individual |.aux| file
that does not interfere with the |.aux| file(s) of the main document.
This behaviour can be altered by the alternative form
|\childdocby[*]{|\textit{main}|}| (with a non-empty optional argument)
which uses the |.aux| file of the main document
by setting |\jobname| to \textit{main}.

%%%%%%%%%%%%%%%%%%%%%%%%%%%%%%%%%%%%%%%%%%%%%%%%%%%%%%%%%%%%%%%%%%%%%%%%%%%%%%%%
\subsection{Driver Development}
\label{sec:driver}

The \textsf{childdoc} mechanism can also be use for the development
of definition files such as \LaTeX{} styles or classes.
This case differs from the above setup with multiple parts
included by |\include| in that no |\includeonly| should be invoked.
This can be achieved by starting the include file
(before |\ProvidesPackage|) with:
%
\begin{center}
\begin{tabular}{l}
|% \iffalse
%
% childdoc.dtx Copyright (C) 2017-2018 Niklas Beisert
%
% This work may be distributed and/or modified under the
% conditions of the LaTeX Project Public License, either version 1.3
% of this license or (at your option) any later version.
% The latest version of this license is in
%   http://www.latex-project.org/lppl.txt
% and version 1.3 or later is part of all distributions of LaTeX
% version 2005/12/01 or later.
%
% This work has the LPPL maintenance status `maintained'.
%
% The Current Maintainer of this work is Niklas Beisert.
%
% This work consists of the files childdoc.dtx and childdoc.ins
% and the derived files childdoc.def and cdocsamp.tex with
% cdocsch1.tex, cdocsch2.tex, cdocsdrf.tex, cdocsfn1.tex, cdocsfn2.tex.
%
%<package>\ifdefined\childdocmain\endinput\fi
%<package>\ProvidesFile{childdoc.def}[2018/12/30 v2.0 child document driver]
%<samplemain>\ProvidesFile{cdocsamp.tex}[2018/12/30 v2.0 sample for childdoc]
%<*driver>
%\ProvidesFile{childdoc.drv}[2018/12/30 v2.0 childdoc reference manual file]
\PassOptionsToClass{10pt,a4paper}{article}
\documentclass{ltxdoc}

\usepackage[margin=35mm]{geometry}
\usepackage{hyperref}
\usepackage{hyperxmp}
\usepackage[usenames]{color}

\hypersetup{colorlinks=true}
\hypersetup{pdfstartview=FitH}
\hypersetup{pdfpagemode=UseNone}
\hypersetup{pdfsource={}}
\hypersetup{pdflang={en-UK}}
\hypersetup{pdfcopyright={Copyright 2017-2018 Niklas Beisert.
  This work may be distributed and/or modified under the
  conditions of the LaTeX Project Public License, either version 1.3
  of this license or (at your option) any later version.}}
\hypersetup{pdflicenseurl={http://www.latex-project.org/lppl.txt}}
\hypersetup{pdfcontactaddress={ETH Zurich, ITP, HIT K,
  Wolfgang-Pauli-Strasse 27}}
\hypersetup{pdfcontactpostcode={8093}}
\hypersetup{pdfcontactcity={Zurich}}
\hypersetup{pdfcontactcountry={Switzerland}}
\hypersetup{pdfcontactemail={nbeisert@itp.phys.ethz.ch}}
\hypersetup{pdfcontacturl={http://people.phys.ethz.ch/\xmptilde nbeisert/}}

\newcommand{\secref}[1]{\hyperref[#1]{section \ref*{#1}}}

\parskip1ex
\parindent0pt
\let\olditemize\itemize
\def\itemize{\olditemize\parskip0pt}

\begin{document}

\title{The \textsf{childdoc} Package}
\hypersetup{pdftitle={The childdoc Package}}
\author{Niklas Beisert\\[2ex]
  Institut f\"ur Theoretische Physik\\
  Eidgen\"ossische Technische Hochschule Z\"urich\\
  Wolfgang-Pauli-Strasse 27, 8093 Z\"urich, Switzerland\\[1ex]
  \href{mailto:nbeisert@itp.phys.ethz.ch}
  {\texttt{nbeisert@itp.phys.ethz.ch}}}
\hypersetup{pdfauthor={Niklas Beisert}}
\hypersetup{pdfsubject={Manual for the LaTeX2e Package childdoc}}
\date{30 December 2018, \textsf{v2.0}}
\maketitle

\begin{abstract}\noindent
\textsf{childdoc} is a \LaTeXe{} package
that enables the direct compilation
of document sections included by |\include|
to individual files.
\end{abstract}

\begingroup
\parskip0ex
\tableofcontents
\endgroup

%%%%%%%%%%%%%%%%%%%%%%%%%%%%%%%%%%%%%%%%%%%%%%%%%%%%%%%%%%%%%%%%%%%%%%%%%%%%%%%%
%%%%%%%%%%%%%%%%%%%%%%%%%%%%%%%%%%%%%%%%%%%%%%%%%%%%%%%%%%%%%%%%%%%%%%%%%%%%%%%%
\section{Introduction}

\LaTeX{} provides a mechanism to structure a large document (such as a book)
into a main file and several child files (containing the chapters)
using the |\include| command.
This mechanism is beneficial for documents
which span hundreds of pages in order to
make the source file(s) more manageable.
Moreover, compilation can be restricted to
selected child files by means of the |\includeonly| command.
The latter feature can be used to reduce the compilation time while editing
(this was significantly more useful in the earlier days of \LaTeX{})
or to generate a smaller document which is easier to navigate.
Another application of |\includeonly| is to generate
documents consisting of selected parts of the complete document.

However, there are a few drawbacks of the plain |\include| mechanism:
\begin{itemize}
\item
The child files cannot be compiled on their own,
they can only be compiled via the main file.
A naive editing environment
(such as a text editor with an option
to have the current file processed by \LaTeX)
may require one to switch to the main file before compiling;
attempting to compile the child file produces errors.
\item
The main file must be modified (each time)
to adjust the |\includeonly| command
to the present needs. This easily leaves the main file in a messy state.
\item
The generated document will always carry the filename
of the main document. This is inconvenient if
several child files are to be compiled and
to be kept for distribution.
\end{itemize}

The present package provides a simple interface
to make child files individually compilable by \LaTeX{}.
Compiling a child file then has the same effect as compiling
the main file with an |\includeonly| command
to select the appropriate child.
Moreover the generated document will carry the name of the child
rather than the main file.
This resolves all three above issues.

This feature is meant to make the editing of books,
thesis documents and lecture notes somewhat more convenient.
However, the package can also be used efficiently for
composing a series of documents (such as exercise sheets)
which are typically distributed individually.
It then assists the author in generating the individual documents
(potentially in different versions)
as well as a document containing the collected series.
Another application is in developing style files
or other kinds of included material
where compilation of the style file could redirect
to a sample or test file.

%%%%%%%%%%%%%%%%%%%%%%%%%%%%%%%%%%%%%%%%%%%%%%%%%%%%%%%%%%%%%%%%%%%%%%%%%%%%%%%%
%%%%%%%%%%%%%%%%%%%%%%%%%%%%%%%%%%%%%%%%%%%%%%%%%%%%%%%%%%%%%%%%%%%%%%%%%%%%%%%%
\section{Usage}

First of all, the package \textsf{childdoc} is \emph{not} a standard
\LaTeXe{} |.sty| style file! Therefore it needs to be invoked in
a non-standard way.

%%%%%%%%%%%%%%%%%%%%%%%%%%%%%%%%%%%%%%%%%%%%%%%%%%%%%%%%%%%%%%%%%%%%%%%%%%%%%%%%
\subsection{Included Files}
\label{sec:include}

%%%%%%%%%%%%%%%%%%%%%%%%%%%%%%%%%%%%%%%%
\DescribeMacro{\childdocmain}
To use the package, add the commands
\begin{center}
\begin{tabular}{l}
|\input{childdoc.def}|\\
|\childdocmain{}|\\
\end{tabular}
\end{center}
at the very top of the main \LaTeX{} file,
in particular \emph{before} the |\documentclass| statement!
The argument of |\childdocmain| should be left empty
(but it must be present).

%%%%%%%%%%%%%%%%%%%%%%%%%%%%%%%%%%%%%%%%
\DescribeMacro{\childdocof}
Furthermore, add the commands
\begin{center}
\begin{tabular}{l}
|\input{childdoc.def}|\\
|\childdocof{|\textit{main}|}|\\
\end{tabular}
\end{center}
at the top of every child file \textit{child}
which is included by |\include{|\textit{child}|}|
from within the main file
(or at least for those files to be compiled individually).
The argument \textit{main} must be the filename of the main file.

There are a couple of
considerations in setting up the main and child documents:

%%%%%%%%%%%%%%%%%%%%%%%%%%%%%%%%%%%%%%%%
\paragraph{Restrictions.}

Please note the following restrictions:
\begin{itemize}
\item
|\childdocmain| must be called with one argument \textit{main}
to ensure compatibility with earlier version of the package.
It must either be empty (|\childdocmain{}|)
or precisely match the filename of the main file in which it is specified.
See \secref{sec:detection} for further information.
\item
The filename \textit{main} must be specified without the |.tex| extension.
\item
The filename \textit{main} is case sensitive
(even in case-insensitive file systems)
due to internal string comparison.
\item
The argument \textit{main} should be fully expanded, it cannot be a macro.
\item
Subdirectories and special characters should be avoided in filenames.
\item
The command |\childdocmain{|\textit{main}|}| must be followed by a whitespace.
It should not be followed immediately by another command
or by a comment mark `|%|'.
This is because the \TeX{} parser reads the token immediately following
the argument of |\childdocmain| and puts it
at the beginning of every child section;
however, a white\-space is ignored.
\end{itemize}

%%%%%%%%%%%%%%%%%%%%%%%%%%%%%%%%%%%%%%%%
\paragraph{Content of Main File.}

It is advisable to place all content in the child files included by |\include|.
Any output contained in the main file will appear in all child documents
unless suppressed manually;
it cannot be suppressed automatically by the |\includeonly| directive
and thus should normally be avoided.
A method to include some content in the main file
by means of conditional processing is described in \secref{sec:conditional}.

%%%%%%%%%%%%%%%%%%%%%%%%%%%%%%%%%%%%%%%%
\paragraph{Page Numbering.}

When only a part of the document is compiled,
the appropriate numbering of pages
(as well as other status parameters)
is determined from the |.aux| files.
The latter contain information from previous passes.
However this information needs to propagate through
all intermediate child documents.
Therefore the page numbering in child documents may well
be inconsistent until the complete document is compiled at least once.

A useful (if unconventional) way to always ensure a consistent
page numbering is to restart the numbering in each child document
and denote the pages by `\textit{child}|.|\textit{page}'
where \textit{child} represents the chapter/section number of the child file.
This can be achieved by the command
|\numberwithin{page}{|\textit{child}|}|
of the \textsf{amsmath} package
where \textit{child} can be |chapter| or |section|
depending on the chosen structuring.
Alternatively, one can modify the macro |\thepage| appropriately
and reset the counter |page| at the start of each child file.

%%%%%%%%%%%%%%%%%%%%%%%%%%%%%%%%%%%%%%%%%%%%%%%%%%%%%%%%%%%%%%%%%%%%%%%%%%%%%%%%
\subsection{Conditional Processing}
\label{sec:conditional}

The package provides a mechanism to compile different versions
of a document. To customise the versions further some conditional processing
can come in handy to distinguish which version is being compiled.
The package provides two macros to describe the compilation context:

%%%%%%%%%%%%%%%%%%%%%%%%%%%%%%%%%%%%%%%%
\DescribeMacro{\ifchilddoc}
The conditional |\ifchilddoc| distinguishes between the compilation of
child documents and the main document:
%
\begin{center}
|\ifchilddoc |\textit{child-code}| |[|\||else |\textit{main-code}]| \||fi|
\end{center}

%%%%%%%%%%%%%%%%%%%%%%%%%%%%%%%%%%%%%%%%
\DescribeMacro{\childdocname}
\DescribeMacro{\childdocjob}
The macro |\childdocname| contains the filename (without extension)
of the main or child file being processed.
Note that |\childdocjob| will always contain the name of the main file.

%%%%%%%%%%%%%%%%%%%%%%%%%%%%%%%%%%%%%%%%
\paragraph{Title Page.}

Conditional processing can be used to include a title or banner page
in the main document when proper precautions are taken.
Importantly, the code in the main file should ensure that the page counter
(as well as other status parameters which are stored in the |.aux| files)
takes the same value after the conditional processing.
Otherwise the page numbers may take divergent values
depending on which part is compiled.

For example, a title page could be declared by:
%
\begin{center}
\begin{tabular}{l}
|\ifchilddoc\||else|\\
|\addtocounter{page}{-1}|\\
\textit{code for title page}\\
|\newpage|\\
|\||fi|
\end{tabular}
\end{center}
%
A banner page for the child documents can be generated by:
%
\begin{center}
\begin{tabular}{l}
|\ifchilddoc|\\
|\addtocounter{page}{-1}|\\
\textit{code for banner page}\\
|\newpage|\\
|\||fi|
\end{tabular}
\end{center}
%
Here one could write a message such as:
\begin{center}
|This is the part \childdocname{} of \childdocjob{}.|
\end{center}

%%%%%%%%%%%%%%%%%%%%%%%%%%%%%%%%%%%%%%%%%%%%%%%%%%%%%%%%%%%%%%%%%%%%%%%%%%%%%%%%
\subsection{Flags}
\label{sec:flags}

The package makes it easy to generate different versions
of the main or child documents.
To this end compilation flags can be defined
and assigned different default values.
They will be particularly useful in conjunction
with the forwarding mechanism described in \secref{sec:forward}.

For example, it may be useful to have a flag |\version|
which can be set to |draft| or |final|.
The document source will contain some conditional code
depending on the value of |\version|.
Suppose further, the flag should default to |final| for the main file
and to |draft| for child files
which is a natural assignment for editing the document.
This is achieved by placing the following code
in the preamble of the main document
(below the |\childdocmain| directive):
%
\begin{center}
\begin{tabular}{l}
|\ifchilddoc|\\
|\providecommand{\version}{draft}|\\
|\||else|\\
|\providecommand{\version}{final}|\\
|\||fi|
\end{tabular}
\end{center}
%
The definition by |\providecommand| makes sure
that previous definitions are not overwritten.
Further statements |\providecommand{\version}{...}|
can thus be added before the above code to override it.

For the main file, one might add a line
(between |\childdocmain| and the above block)
%
\begin{center}
|%\ifchilddoc\||else\providecommand{\version}{draft}\||fi|
\end{center}
%
which can be uncommented to produce a draft version.
Likewise one can add a line to the very top of a child file
(above the |\childdocof{|\textit{main}|}| directive)
%
\begin{center}
|%\providecommand{\version}{final}|
\end{center}
%
which can be uncommented to produce the final version of this child document.

%%%%%%%%%%%%%%%%%%%%%%%%%%%%%%%%%%%%%%%%%%%%%%%%%%%%%%%%%%%%%%%%%%%%%%%%%%%%%%%%
\subsection{Forwarding}
\label{sec:forward}

Different versions of the main or child documents
using compilation flags as described in \secref{sec:flags}
can be (permanently) stored in different files
for convenient compilation, viewing and distribution.
To this end, the package defines a command
to pass on compilation to a different file:

%%%%%%%%%%%%%%%%%%%%%%%%%%%%%%%%%%%%%%%%
\DescribeMacro{\childdocforward}
The command |\childdocforward| redirects processing to
another source file:
%
\begin{center}
\begin{tabular}{l}
|\input{childdoc.def}|\\
|\childdocforward[|\textit{main}|]{|\textit{dest}|}|\\
\end{tabular}
\end{center}
%
The argument \textit{dest} is the destination file
(without extension).
It should be the main file or one of the child files.
Note that further \textsf{childdoc} directives
such as |\childdocof| and |\childdocforward|
in the indicated file will be processed in this form.
The optional argument \textit{main}
passes on directly to the main file \textit{main}
while pretending to compile the child \textit{dest}.
This form behaves as if \textit{dest}
issues |\childdocof{|\textit{main}|}| right away,
and no further \textsf{childdoc} directives will be processed.

%%%%%%%%%%%%%%%%%%%%%%%%%%%%%%%%%%%%%%%%
\DescribeMacro{\...prefix}
In the alternative form |\childdocforwardprefix|,
%
\begin{center}
\begin{tabular}{l}
|\input{childdoc.def}|\\
|\childdocforwardprefix[|\textit{main}|]{|\textit{prefix}|}{|\textit{dest}|}|
\end{tabular}
\end{center}
%
the destination file is determined by a pattern
depending on the current file:
To make this work, the current file must be called
`{\textit{prefix}\hspace{0.2em}\textit{suffix}}'
with \textit{prefix} matching precisely the argument.
Processing is then passed on to the file
`{\textit{dest}\hspace{0.2em}\textit{suffix}}'.
Surely, the same effect is achieved by
directly specifying the
argument `{\textit{dest}\hspace{0.2em}\textit{suffix}}'
in the first form.
However, that requires to set up a different file
for each child. With the alternative form of the command
all these files can have exactly the same content
which simplifies setting them up and maintaining them.

For example, the following file |draft.tex|
with a compilation flag |\version| as described in \secref{sec:flags}
compiles the main document as a draft:
%
\begin{center}
\begin{tabular}{l}
|\def\version{draft}|\\
|\input{childdoc.def}|\\
|\childdocforward{|\textit{main}|}|
\end{tabular}
\end{center}
%
Likewise, the following files |final|\textit{nn}|.tex|
compile the final version of the child document
|child|\textit{nn}|.tex|:
%
\begin{center}
\begin{tabular}{l}
|\def\version{final}|\\
|\input{childdoc.def}|\\
|\childdocforwardprefix{final}{child}|
\end{tabular}
\end{center}
%

Note that when several versions of a main file and/or of each child file
are to be generated, it may be convenient to set up a |Makefile| or
shell script to automatise the process.

%%%%%%%%%%%%%%%%%%%%%%%%%%%%%%%%%%%%%%%%%%%%%%%%%%%%%%%%%%%%%%%%%%%%%%%%%%%%%%%%
\subsection{Command Line Processing}
\label{sec:commandline}

The effect of redirection files can also be achieved by invoking
the \LaTeX{} compiler with a more elaborate command line.
Most conveniently this should be done as part
of a shell script or a |Makefile|.

When using \textsf{childdoc} in the main file, the following
command lines effectively perform a redirection
(note that depending on the shell being used,
backslashes may have to be doubled: `|\|' $\to$ `|\\|'):
%
\begin{center}
|... -jobname "|\textit{target}|" |\\|"|[\textit{flags}]%
|\input{childdoc.def}\childdocforward[|\textit{main}|]{|\textit{dest}|}"|
\end{center}
%
Here \textit{target} is the name of the output file,
\textit{main} is the name of the main file
and \textit{dest} is the name of the main or child file to be processed
(all filenames without extensions).
The optional argument \textit{main} can be omitted
if \textit{main} matches \textit{dest}.
Optionally, compilation \textit{flags} can be defined via |\def| commands.
This command line makes the \TeX{} engine believe
it is compiling the file \textit{target}
whose content is specified as the latter parameter.
The provided code then forwards the processing to
\textit{main} or \textit{dest} as described in \secref{sec:forward}.

%%%%%%%%%%%%%%%%%%%%%%%%%%%%%%%%%%%%%%%%%%%%%%%%%%%%%%%%%%%%%%%%%%%%%%%%%%%%%%%%
\subsection{Include by Input}
\label{sec:input}

Including child documents by |\include| has some restrictions by design.
Most notably, the content of a child document always occupies
its own set of pages; pages cannot be shared between child documents.
Usually, this behaviour makes perfect sense
because each child document contain an essential part of the document.
However, in some situations it may be desirable to compose
a document from a collection of parts
without having mandatory page breaks between then.
For this case, the package
provides a mechanism to include parts
by |\input| which can also be processed individually.
However, by construction this mechanism
requires manual handling of the content to be output.

%%%%%%%%%%%%%%%%%%%%%%%%%%%%%%%%%%%%%%%%
\DescribeMacro{\ifchilddocmanual}
The main file should be prepared as usual, see \secref{sec:include}.
However, the document body must make a distinction
between processing of an individual part and of the main document, e.g.:
%
\begin{center}
\begin{tabular}{l}
|\ifchilddocmanual|\\
|\input{\childdocname}|\\
|\||else|\\
\textit{document body with }|\input{|\textit{part}|}|\\
|\||fi|
\end{tabular}
\end{center}
%
The conditional |\ifchilddocmanual| is true whenever
a part to be included by |\input| is being compiled,
and the name of the part is stored in |\childdocname|.

%%%%%%%%%%%%%%%%%%%%%%%%%%%%%%%%%%%%%%%%
\DescribeMacro{\childdocby}
Each part to be included by |\input| should start with:
%
\begin{center}
\begin{tabular}{l}
|\input{childdoc.def}|\\
|\childdocby{|\textit{main}|}|\\
\end{tabular}
\end{center}
%
The directive |\childdocby| is similar to |\childdocof|
described in \secref{sec:include},
but the subsequent selection of content must be done manually.
To that end, both |\ifchilddoc| and |\ifchilddocmanual|
will be true upon processing of a part,
and the name of the part is stored in |\childdocname|.
Note that |\jobname| will be set to the filename of the current part
so that each part receives an individual |.aux| file
that does not interfere with the |.aux| file(s) of the main document.
This behaviour can be altered by the alternative form
|\childdocby[*]{|\textit{main}|}| (with a non-empty optional argument)
which uses the |.aux| file of the main document
by setting |\jobname| to \textit{main}.

%%%%%%%%%%%%%%%%%%%%%%%%%%%%%%%%%%%%%%%%%%%%%%%%%%%%%%%%%%%%%%%%%%%%%%%%%%%%%%%%
\subsection{Driver Development}
\label{sec:driver}

The \textsf{childdoc} mechanism can also be use for the development
of definition files such as \LaTeX{} styles or classes.
This case differs from the above setup with multiple parts
included by |\include| in that no |\includeonly| should be invoked.
This can be achieved by starting the include file
(before |\ProvidesPackage|) with:
%
\begin{center}
\begin{tabular}{l}
|\input{childdoc.def}|\\
|\childdocforward{|\textit{main}|}|\\
\end{tabular}
\end{center}
%
or alternatively with:
%
\begin{center}
\begin{tabular}{l}
|\input{childdoc.def}|\\
|\childdocby{|\textit{main}|}|\\
\end{tabular}
\end{center}
%
Both forms have slightly different effects as described above.
The main file is prepared as usual, see \secref{sec:include}.

%%%%%%%%%%%%%%%%%%%%%%%%%%%%%%%%%%%%%%%%%%%%%%%%%%%%%%%%%%%%%%%%%%%%%%%%%%%%%%%%
\subsection{Legacy Detection}
\label{sec:detection}

The directive |\childdocmain| in the main file can detect
whether the complete document or merely a child is to be compiled
even without using the directive |\childdocof|.
This method is deprecated because it is less robust
and there is no compelling reason to use it;
it is merely provided for backward compatibility
and it may be removed in future versions.

If the detection mechanism is to be used,
it is mandatory to correctly specify
the filename of the main file as the argument of |\childdocmain|:
%
\begin{center}
\begin{tabular}{l}
|\input{childdoc.def}|\\
|\childdocmain{|\textit{main}|}|\\
\end{tabular}
\end{center}
%
If |\jobname| does not match the argument \textit{main} of |\childdocmain|,
it is assumed that |\jobname| points to the child file to be compiled.
When using |\childdocmain| with the main file specified as argument,
it suffices to start a child file
with just |\input{|\textit{main}|}|
without loading of the package and using |\childdocof|.
If instead all processing is done
with the appropriate \textsf{childdoc} directives,
the argument of \textit{main} of |\childdocmain| can be empty.

An alternative version of the command line processing described
in \secref{sec:commandline} using the detection mechanism reads:
%
\begin{center}
|... -jobname "|\textit{target}|" "|[\textit{flags}]%
[|\def\jobname{|\textit{dest}|}|]|\input{|\textit{main}|}"|
\end{center}

%%%%%%%%%%%%%%%%%%%%%%%%%%%%%%%%%%%%%%%%%%%%%%%%%%%%%%%%%%%%%%%%%%%%%%%%%%%%%%%%
\subsection{Manual Code}
\label{sec:manual}

In case one cannot be certain whether the definitions file |childdoc.def|
is installed on the target \TeX{} distribution
and one prefers not to ship it,
it is conceivable to paste a few relevant commands into the sources.

To that end, drop all statements |\input{childdoc.def}|
and perform the replacements as outlined below.
Instead of |\childdocmain{|\textit{main}|}| add the following code
to the top of the main file:
%
\begin{center}
\begin{tabular}{l}
|\||ifdefined\childdocname\endinput\||fi\newif\ifchilddoc|\\
|\edef\childdocname{\scantokens\expandafter{\jobname\noexpand}}|\\
|\def\childdocmain{|\textit{main}|}\||ifx\childdocmain\childdocname\||else|\\
|\childdoctrue\includeonly{\childdocname}\let\jobname\childdocmain\||fi|\\
\end{tabular}
\end{center}
%
Instead of |\childdocof{|\textit{main}|}| just include the main file
at the top of each child file:
%
\begin{center}
|\input{|\textit{main}|}|
\end{center}
%
A simple redirection |\childdocforward{|\textit{dest}|}| is achieved by:
%
\begin{center}
|\def\jobname{|\textit{dest}|}\input{\jobname}|
\end{center}
%
The redirection with prefix
|\childdocforwardprefix[|\textit{prefix}|]{|\textit{dest}|}|
is accomplished by:
%
\begin{center}
\begin{tabular}{l}
|{\edef\jobname{\scantokens\expandafter{\jobname\noexpand}}|\\
|\def\redirectjob |\textit{prefix}|#1~~~{\gdef\jobname{|\textit{dest}|#1}}|\\
|\expandafter\redirectjob\jobname~~~}\input{\jobname}|
\end{tabular}
\end{center}

In an alternative approach,
child documents can be compiled by a specific command line
without additional code or specific definitions:
%
\begin{center}
|... -jobname "|\textit{target}|" "|[\textit{flags}]%
|\includeonly{|\textit{dest}|}\input{|\textit{main}|}"|
\end{center}
%

%%%%%%%%%%%%%%%%%%%%%%%%%%%%%%%%%%%%%%%%%%%%%%%%%%%%%%%%%%%%%%%%%%%%%%%%%%%%%%%%
%%%%%%%%%%%%%%%%%%%%%%%%%%%%%%%%%%%%%%%%%%%%%%%%%%%%%%%%%%%%%%%%%%%%%%%%%%%%%%%%
\section{Information}

%%%%%%%%%%%%%%%%%%%%%%%%%%%%%%%%%%%%%%%%%%%%%%%%%%%%%%%%%%%%%%%%%%%%%%%%%%%%%%%%
\subsection{Copyright}

Copyright \copyright{} 2017--2018 Niklas Beisert

This work may be distributed and/or modified under the
conditions of the \LaTeX{} Project Public License, either version 1.3
of this license or (at your option) any later version.
The latest version of this license is in
  \url{http://www.latex-project.org/lppl.txt}
and version 1.3 or later is part of all distributions of \LaTeX{}
version 2005/12/01 or later.

This work has the LPPL maintenance status `maintained'.

The Current Maintainer of this work is Niklas Beisert.

This work consists of the files |README.txt|, |childdoc.ins| and |childdoc.dtx|
as well as the derived files |childdoc.def|, |cdocsamp.tex|
with |cdocsch1.tex|, |cdocsch2.tex|, |cdocspt3.tex|, |cdocspt4.tex|,
|cdocsdrf.tex|, |cdocsfn1.tex|, |cdocsfn2.tex|
as well as |childdoc.pdf|.

%%%%%%%%%%%%%%%%%%%%%%%%%%%%%%%%%%%%%%%%%%%%%%%%%%%%%%%%%%%%%%%%%%%%%%%%%%%%%%%%
\subsection{Files and Installation}

The package consists of the files:
%
\begin{center}
\begin{tabular}{ll}
    |README.txt|   & readme file \\
    |childdoc.ins| & installation file \\
    |childdoc.dtx| & source file \\
    |childdoc.def| & definition file \\
    |cdocsamp.tex| & sample main file \\
    |cdocsch1.tex| & sample include file \\
    |cdocsch2.tex| & sample include file \\
    |cdocspt3.tex| & sample part file \\
    |cdocspt4.tex| & sample part file \\
    |cdocsdrf.tex| & sample redirection file \\
    |cdocsfn1.tex| & sample redirection file \\
    |cdocsfn2.tex| & sample redirection file \\
    |childdoc.pdf| & manual
\end{tabular}
\end{center}
%
The distribution consists of the files
|README.txt|, |childdoc.ins| and |childdoc.dtx|.
%
\begin{itemize}
\item
Run (pdf)\LaTeX{} on |childdoc.dtx|
to compile the manual |childdoc.pdf| (this file).
\item
Run \LaTeX{} on |childdoc.ins| to create the definitions file |childdoc.def|
and the sample |cdocsamp.tex| with include files
|cdocsch1.tex|, |cdocsch2.tex|, |cdocspt3.tex|, |cdocspt4.tex|,
|cdocsdrf.tex|, |cdocsfn1.tex|, |cdocsfn2.tex|.
Then copy the file |childdoc.def| to an appropriate directory of your \LaTeX{}
distribution, e.g.\ \textit{texmf-root}|/tex/latex/childdoc|.
\end{itemize}

%%%%%%%%%%%%%%%%%%%%%%%%%%%%%%%%%%%%%%%%%%%%%%%%%%%%%%%%%%%%%%%%%%%%%%%%%%%%%%%%
\subsection{Related CTAN Packages}

There are several other packages which offer a similar functionality:
%
\begin{itemize}
\item
The packages
\href{http://ctan.org/pkg/docmute}{\textsf{docmute}},
\href{http://ctan.org/pkg/includex}{\textsf{includex}} and
\href{http://ctan.org/pkg/standalone}{\textsf{standalone}}
provide commands to include only the document body of
a child file thus allowing both files to be compiled individually.
\item
The packages \href{http://ctan.org/pkg/subdocs}{\textsf{subdocs}}
and \href{http://ctan.org/pkg/subfiles}{\textsf{subfiles}}
provide structures in which the main and child documents can be
encapsulated and allowing them to be compiled individually.
The inclusion mechanism is different from the conventional |\include|.
\item
The package \href{http://ctan.org/pkg/combine}{\textsf{combine}}
is an elaborate solution to combine several documents into one.
\end{itemize}
%
See also the CTAN topic \href{http://ctan.org/topic/subdocs}{\textsf{subdocs}}
for further related packages.
The present package differs from the above solutions in that
a document structure constructed with the conventional |\include| mechanism
just needs two extra commands at the top of every file
such that all constituent files can be compiled individually.

%%%%%%%%%%%%%%%%%%%%%%%%%%%%%%%%%%%%%%%%%%%%%%%%%%%%%%%%%%%%%%%%%%%%%%%%%%%%%%%%
%\subsection{Feature Suggestions}
%
%The following is a list of features which may be useful for future
%versions of this package:
%%
%\begin{itemize}
%\item
%\ldots
%\end{itemize}

%%%%%%%%%%%%%%%%%%%%%%%%%%%%%%%%%%%%%%%%%%%%%%%%%%%%%%%%%%%%%%%%%%%%%%%%%%%%%%%%
\subsection{Revision History}

%%%%%%%%%%%%%%%%%%%%%%%%%%%%%%%%%%%%%%%%
\paragraph{v2.0:} 2018/12/30

\begin{itemize}
\item
immediate forward processing
\item
added |\childdocby| mechanism
\item
manual restructured
\end{itemize}

%%%%%%%%%%%%%%%%%%%%%%%%%%%%%%%%%%%%%%%%
\paragraph{v1.6:} 2018/01/17

\begin{itemize}
\item
application for development of include files
\item
corrections to manual
\end{itemize}

%%%%%%%%%%%%%%%%%%%%%%%%%%%%%%%%%%%%%%%%
\paragraph{v1.5:} 2017/05/21

\begin{itemize}
\item
more complete structuring introduced
\item
|\childdocof| introduced
\item
|\childdoc| renamed to |\childdocmain|
\item
|\childredirect| renamed to |\childdocforward| and |\childdocforwardprefix|
and functionality expanded
\end{itemize}

%%%%%%%%%%%%%%%%%%%%%%%%%%%%%%%%%%%%%%%%
\paragraph{v1.0:} 2017/04/27

\begin{itemize}
\item
manual and install package
\item
first version published on CTAN
\end{itemize}

%%%%%%%%%%%%%%%%%%%%%%%%%%%%%%%%%%%%%%%%
\paragraph{v0.6:} 2017/04/26

\begin{itemize}
\item
redirection mechanism added
\end{itemize}

%%%%%%%%%%%%%%%%%%%%%%%%%%%%%%%%%%%%%%%%
\paragraph{v0.5:} 2017/04/26

\begin{itemize}
\item
functionality in definition file
\end{itemize}


%%%%%%%%%%%%%%%%%%%%%%%%%%%%%%%%%%%%%%%%%%%%%%%%%%%%%%%%%%%%%%%%%%%%%%%%%%%%%%%%
%%%%%%%%%%%%%%%%%%%%%%%%%%%%%%%%%%%%%%%%%%%%%%%%%%%%%%%%%%%%%%%%%%%%%%%%%%%%%%%%
%%%%%%%%%%%%%%%%%%%%%%%%%%%%%%%%%%%%%%%%%%%%%%%%%%%%%%%%%%%%%%%%%%%%%%%%%%%%%%%%
\appendix

\settowidth\MacroIndent{\rmfamily\scriptsize 000\ }

 \DocInput{childdoc.dtx}

\end{document}
%</driver>
% \fi
%
% %%%%%%%%%%%%%%%%%%%%%%%%%%%%%%%%%%%%%%%%%%%%%%%%%%%%%%%%%%%%%%%%%%%%%%%%%%%%%%
% %%%%%%%%%%%%%%%%%%%%%%%%%%%%%%%%%%%%%%%%%%%%%%%%%%%%%%%%%%%%%%%%%%%%%%%%%%%%%%
% \section{Sample}
%\iffalse
%<*samplemain>
%\fi
%
% The following presents a sample document
% with two chapters, two parts, a title page,
% a compile flag as well as three forwarding files to set the flag.
% It consists of eight |.tex| files:
% \begin{center}
% \begin{tabular}{ll}
% |cdocsamp.tex|&main file\\
% |cdocsch1.tex|&include file for chapter 1\\
% |cdocsch2.tex|&include file for chapter 2\\
% |cdocspt3.tex|&include file for part 3\\
% |cdocspt4.tex|&include file for part 4\\
% |cdocsdrf.tex|&forwarding file for main file in draft mode\\
% |cdocsfi1.tex|&forwarding file for final version of chapter 1\\
% |cdocsfi2.tex|&forwarding file for final version of chapter 2\\
% \end{tabular}
% \end{center}
% Each of the eight files can be compiled directly by the \LaTeX{} compiler.
%
% %%%%%%%%%%%%%%%%%%%%%%%%%%%%%%%%%%%%%%
% \paragraph{Main File.}
%
% The main file is called |cdocsamp.tex|.
%
% Load the \textsf{childdoc} definitions and
% declare the filename for the main document:
%    \begin{macrocode}
\input{childdoc.def}
\childdocmain{}
%    \end{macrocode}

% Optional override for |\version| flag:
%    \begin{macrocode}
%%\ifchilddoc\else\providecommand{\version}{draft}\fi
%    \end{macrocode}

% Define the default values for the |\version| flag
% (|final| for the main file and |draft| for childs):
%    \begin{macrocode}
\ifchilddoc
\providecommand{\version}{draft}
\else
\providecommand{\version}{final}
\fi
%    \end{macrocode}

% Load the standard document class:
%    \begin{macrocode}
\documentclass[12pt]{article}
%    \end{macrocode}

% Start the document body:
%    \begin{macrocode}
\begin{document}
%    \end{macrocode}

% Declare a title page.
% Print title, part of document being processed and version flag:
%    \begin{macrocode}
\addtocounter{page}{-1}
\begin{center}
{\LARGE\bfseries{}childdoc example\par}
\vspace{1cm}
\ifchilddoc
\ifchilddocmanual part\else chapter\fi:
`\childdocname' of `\childdocjob'\par
\else
main document: `\childdocjob'\par
\fi
version: \version\par
\end{center}
\newpage
%    \end{macrocode}

% Manually include selected file,
% otherwise process as usual:
%    \begin{macrocode}
\ifchilddocmanual
\section*{part `\childdocname'}
\input{\childdocname}
\else
%    \end{macrocode}

% Include the two chapters:
%    \begin{macrocode}
\include{cdocsch1}
\include{cdocsch2}
%    \end{macrocode}

% Include the two parts unless only chapters should be displayed:
%    \begin{macrocode}
\ifchilddoc\else
\section{part three}
\input{cdocspt3}
\section{part four}
\input{cdocspt4}
\fi
%    \end{macrocode}

% Process as usual until here:
%    \begin{macrocode}
\fi
%    \end{macrocode}

% End of document body:
%    \begin{macrocode}
\end{document}
%    \end{macrocode}
%\iffalse
%</samplemain>
%\fi
%
% %%%%%%%%%%%%%%%%%%%%%%%%%%%%%%%%%%%%%%
% \paragraph{Chapter Include Files.}
%
% The include files are called |cdocsch1.tex| and |cdocsch2.tex|.
%
%\iffalse
%<*samplechap1|samplechap2>
%\fi

% Optional override for |\version| flag:
%    \begin{macrocode}
%%\providecommand{\version}{final}
%    \end{macrocode}

% Include the main document:
%    \begin{macrocode}
\input{childdoc.def}
\childdocof{cdocsamp}
%    \end{macrocode}

%\iffalse
%</samplechap1|samplechap2>
%\fi
%
%\iffalse
%<*samplechap1>
%\fi
% Some text for chapter 1:
%    \begin{macrocode}
\section{one}
some text in chapter one
%    \end{macrocode}

%\iffalse
%</samplechap1>
%\fi
% Some text for chapter 2:
%\iffalse
%<*samplechap2>
%\fi
%    \begin{macrocode}
\section{two}
more text in chapter two
%    \end{macrocode}

%\iffalse
%</samplechap2>
%\fi
%
% %%%%%%%%%%%%%%%%%%%%%%%%%%%%%%%%%%%%%%
% \paragraph{Part Include Files.}
%
% The include files are called |cdocspt3.tex| and |cdocspt4.tex|.
%
%\iffalse
%<*samplepart3|samplepart4>
%\fi

% Optional override for |\version| flag:
%    \begin{macrocode}
%%\providecommand{\version}{final}
%    \end{macrocode}

% Include the main document:
%    \begin{macrocode}
\input{childdoc.def}
\childdocby{cdocsamp}
%    \end{macrocode}

%\iffalse
%</samplepart3|samplepart4>
%\fi
%
%\iffalse
%<*samplepart3>
%\fi
% Some text for part 3:
%    \begin{macrocode}
some text in part three
%    \end{macrocode}

%\iffalse
%</samplepart3>
%\fi
% Some text for part 4:
%\iffalse
%<*samplepart4>
%\fi
%    \begin{macrocode}
more text in part four
%    \end{macrocode}

%\iffalse
%</samplepart4>
%\fi
%
% %%%%%%%%%%%%%%%%%%%%%%%%%%%%%%%%%%%%%%
% \paragraph{Forwarding for a Complete Draft.}
%
% The following forwarding file |cdocsdrf.tex|
% compiles the main document in draft mode:
%\iffalse
%<*sampledraft>
%\fi
%    \begin{macrocode}
\def\version{draft}
\input{childdoc.def}
\childdocforward{cdocsamp}
%    \end{macrocode}

%\iffalse
%</sampledraft>
%\fi
%
% %%%%%%%%%%%%%%%%%%%%%%%%%%%%%%%%%%%%%%
% \paragraph{Forwarding for Final Version of the Chapters.}
%
% The following forwarding files |cdocsfn1.tex| and |cdocsfn2.tex|
% (with identical content)
% compile the final versions of the child documents
% |cdocsch1.tex| and |cdocsch2.tex|, respectively:
%\iffalse
%<*samplefinal>
%\fi
%    \begin{macrocode}
\def\version{final}
\input{childdoc.def}
\childdocforwardprefix[cdocsamp]{cdocsfn}{cdocsch}
%    \end{macrocode}

%\iffalse
%</samplefinal>
%\fi
%
% %%%%%%%%%%%%%%%%%%%%%%%%%%%%%%%%%%%%%%
% \paragraph{Command Line Processing.}
%
% The following three command lines generate the output files
% |cdocscld|, |cdocscl1| and |cdocscl2|
% which should be identical to
% |cdocsdrf|, |cdocsch1| and |cdocsfn2|, respectively:
% \begin{center}
% \begin{tabular}{l}
% |latex -jobname cdocscld \|\\
% |  "\def\version{draft}\input{childdoc.def}\childdocforward{cdocsamp}"|\\
% |latex -jobname cdocscl1 \|\\
% |  "\input{childdoc.def}\childdocforward[cdocsamp]{cdocsch1}"|\\
% |latex -jobname cdocscl2 \|\\
% |  "\def\version{final}\input{childdoc.def}\childdocforward{cdocsch2}"|
% \end{tabular}
% \end{center}
% Note that the trailing backslash on each first line
% merely continues the input to the second line
% (for convenient cut ant paste).
% Furthermore, the command |latex| can be replaced by any
% of its alternative versions such as |pdflatex|.
%
% %%%%%%%%%%%%%%%%%%%%%%%%%%%%%%%%%%%%%%%%%%%%%%%%%%%%%%%%%%%%%%%%%%%%%%%%%%%%%%
% %%%%%%%%%%%%%%%%%%%%%%%%%%%%%%%%%%%%%%%%%%%%%%%%%%%%%%%%%%%%%%%%%%%%%%%%%%%%%%
% \section{Implementation}
%\iffalse
%<*package>
%\fi
%
% This section describes the definitions file |childdoc.def|.

% The definitions cannot be loaded using |\usepackage| or |\RequirePackage|
% which has a mechanism to prevent loading a style file more than once.
% When loading the definitions by means of |\input|
% multiple instances have to be prevented manually:
%\iffalse
%This code needs to be before the `\ProvidesFile' directive
%which is defined at the beginning of this file.
%Therefore it is also placed there and commented out here.
%</package>
%<*discard>
%\fi
%    \begin{macrocode}
\ifdefined\childdocmain\endinput\fi
%    \end{macrocode}
%\iffalse
%</discard>
%<*package>
%\fi
%
% \macro{\ifchilddoc}
% \macro{\ifchilddocmanual}
% The conditional |\ifchilddoc| tells whether a
% child (true) or main (false) document is being compiled.
% The conditional |\ifchilddocmanual| tells whether
% the |\includeonly| mechanism is used (false) or
% the selection of child files must be performed manually (true).
% The definitions initialise to false:
%    \begin{macrocode}
\newif\ifchilddoc
\newif\ifchilddocmanual
%    \end{macrocode}

% \macro{\childdocname}
% \macro{\childdocjob}
% The macro |\childdocname| stores the name of the main document
% to be compiled. The macro |\childdocjob| stores the name of
% the document on which the \LaTeX{} compiler was originally invoked.
% The content of |\jobname| cannot be compared
% to filenames specified in the source due to different catcodes.
% The following code rescans |\jobname|, stores the result
% in |\childdocname| and saves a copy in |\childdocjob|:
%    \begin{macrocode}
\edef\childdocname{\scantokens\expandafter{\jobname\noexpand}}
\let\childdocjob\childdocname
%    \end{macrocode}

% \macro{\childdocdisable}
% The macro |\childdocdisable| prevents the main file
% from being processed more than once.
% At this stage, the main document command |\childdocmain|
% is assumed to be called once again where it should do nothing.
% Any subsequent call to it should prevent
% a secondary processing of the main document
% It overwrites the forwarding commands
% |\childdocof| and |\childdocforward|
% with empty macros to prevent further inclusions of the main document:
%    \begin{macrocode}
\newcommand{\childdocdisable}
{
  \renewcommand{\childdocmain}[1]{\renewcommand{\childdocmain}[1]{\endinput}}
  \renewcommand{\childdocof}[1]{}
  \renewcommand{\childdocby}[2][]{}
  \renewcommand{\childdocforward}[2][]{}
  \renewcommand{\childdocdisable}{}
}
%    \end{macrocode}

% \macro{\childdocmain}
% The macro |\childdocmain| is to be called at the top of the main file
% with nothing or the main filename (without extension) as argument.
% First, it breaks loops.
% If the argument is not empty and does not match |\childdocname|
% (which is set by the first inclusion of |childdoc.def|),
% |\ifchilddoc| is set to true, |\includeonly| is applied to the child file
% and |\jobname| is set to the main file
% (for proper handling of |.aux| files):
%    \begin{macrocode}
\newcommand{\childdocmain}[1]
{
  \childdocdisable\childdocmain{}
  \if?#1?\else
    \begingroup
      \def\childdoctmp{#1}
      \ifx\childdoctmp\childdocname
        \def\childdoctmp{}
      \else
        \def\childdoctmp
        {
          \childdoctrue
          \includeonly{\childdocname}
          \def\childdocjob{#1}
          \def\jobname{#1}
        }
      \fi
      \expandafter
    \endgroup
    \childdoctmp
  \fi
}
%    \end{macrocode}

% \macro{\childdocof}
% The command |\childdocof| redirects
% compilation to the main file |#1|.
%    \begin{macrocode}
\newcommand{\childdocof}[1]
{
  \childdocdisable
  \childdoctrue
  \includeonly{\childdocname}
  \def\jobname{#1}
  \def\childdocjob{#1}
  \input{#1}
}
%    \end{macrocode}

% \macro{\childdocby}
% The command |\childdocby| ....
%    \begin{macrocode}
\newcommand{\childdocby}[2][]
{
  \childdocdisable
  \childdoctrue
  \childdocmanualtrue
  \if?#1?\else
    \def\jobname{#2}
  \fi
  \def\childdocjob{#2}
  \input{#2}
  \endinput
}
%    \end{macrocode}

% \macro{\childdocforward}
% The command |\childdocforward| redirects
% compilation to the main file or
% (if the optional argument is given) a child file.
% Parameters are set as if the main file
% or a child file starting with |\childdocof| was compiled.
% Then compilation is handed over to the main file:
%    \begin{macrocode}
\newcommand{\childdocforward}[2][]
{
  \begingroup
    \if?#1?
      \def\childdoctmp
      {
        \def\childdocname{#2}
        \def\childdocjob{#2}
        \def\jobname{#2}
        \input{#2}
        \endinput
      }
    \else
      \def\childdoctmp
      {
        \childdocdisable
        \def\childdocname{#2}
        \childdoctrue
        \includeonly{#2}
        \def\childdocjob{#1}
        \def\jobname{#1}
        \input{#1}
        \endinput
      }
    \fi
    \expandafter
  \endgroup
  \childdoctmp
}
%    \end{macrocode}

% \macro{\childdocforwardprefix}
% The command |\childdocforwardprefix| redirects
% compilation to the main or a child file by means of a pattern.
% The prefix |#1| in the current filename is replaced by |#2|
% and the suffix of the current filename is kept
% (it is assumed that the filename does not contain the substring `|~~~|'
% which is used as a delimiter).
% Compilation is handed over to the new file by |\childdocforward|:
%    \begin{macrocode}
\newcommand{\childdocforwardprefix}[3][]
{
  \begingroup
    \def\childdocextract #2##1~~~{\def\childdoctmp{\childdocforward[#1]{#3##1}}}
    \expandafter\childdocextract\childdocname~~~
    \expandafter
  \endgroup
  \childdoctmp
}
%    \end{macrocode}

% \macro{\childdoc}
% The deprecated macro |\childdoc| is a legacy version of |\childdocmain|:
%    \begin{macrocode}
\newcommand{\childdoc}{\childdocmain}
%    \end{macrocode}

% \macro{\childdocredirect}
% The deprecated macro |\childdocredirect| is a legacy version
% of |\childdocforward| and |\childdocforwardprefix|:
%    \begin{macrocode}
\newcommand{\childdocredirect}[2][]
{
  \begingroup
    \if?#1?
      \def\childdoctmp{\childdocforward{#2}}
    \else
      \def\childdoctmp{\childdocforwardprefix{#1}{#2}}
    \fi
    \expandafter
  \endgroup
  \childdoctmp
}
%    \end{macrocode}

%\iffalse
%</package>
%\fi
%
\endinput
|\\
|\childdocforward{|\textit{main}|}|\\
\end{tabular}
\end{center}
%
or alternatively with:
%
\begin{center}
\begin{tabular}{l}
|% \iffalse
%
% childdoc.dtx Copyright (C) 2017-2018 Niklas Beisert
%
% This work may be distributed and/or modified under the
% conditions of the LaTeX Project Public License, either version 1.3
% of this license or (at your option) any later version.
% The latest version of this license is in
%   http://www.latex-project.org/lppl.txt
% and version 1.3 or later is part of all distributions of LaTeX
% version 2005/12/01 or later.
%
% This work has the LPPL maintenance status `maintained'.
%
% The Current Maintainer of this work is Niklas Beisert.
%
% This work consists of the files childdoc.dtx and childdoc.ins
% and the derived files childdoc.def and cdocsamp.tex with
% cdocsch1.tex, cdocsch2.tex, cdocsdrf.tex, cdocsfn1.tex, cdocsfn2.tex.
%
%<package>\ifdefined\childdocmain\endinput\fi
%<package>\ProvidesFile{childdoc.def}[2018/12/30 v2.0 child document driver]
%<samplemain>\ProvidesFile{cdocsamp.tex}[2018/12/30 v2.0 sample for childdoc]
%<*driver>
%\ProvidesFile{childdoc.drv}[2018/12/30 v2.0 childdoc reference manual file]
\PassOptionsToClass{10pt,a4paper}{article}
\documentclass{ltxdoc}

\usepackage[margin=35mm]{geometry}
\usepackage{hyperref}
\usepackage{hyperxmp}
\usepackage[usenames]{color}

\hypersetup{colorlinks=true}
\hypersetup{pdfstartview=FitH}
\hypersetup{pdfpagemode=UseNone}
\hypersetup{pdfsource={}}
\hypersetup{pdflang={en-UK}}
\hypersetup{pdfcopyright={Copyright 2017-2018 Niklas Beisert.
  This work may be distributed and/or modified under the
  conditions of the LaTeX Project Public License, either version 1.3
  of this license or (at your option) any later version.}}
\hypersetup{pdflicenseurl={http://www.latex-project.org/lppl.txt}}
\hypersetup{pdfcontactaddress={ETH Zurich, ITP, HIT K,
  Wolfgang-Pauli-Strasse 27}}
\hypersetup{pdfcontactpostcode={8093}}
\hypersetup{pdfcontactcity={Zurich}}
\hypersetup{pdfcontactcountry={Switzerland}}
\hypersetup{pdfcontactemail={nbeisert@itp.phys.ethz.ch}}
\hypersetup{pdfcontacturl={http://people.phys.ethz.ch/\xmptilde nbeisert/}}

\newcommand{\secref}[1]{\hyperref[#1]{section \ref*{#1}}}

\parskip1ex
\parindent0pt
\let\olditemize\itemize
\def\itemize{\olditemize\parskip0pt}

\begin{document}

\title{The \textsf{childdoc} Package}
\hypersetup{pdftitle={The childdoc Package}}
\author{Niklas Beisert\\[2ex]
  Institut f\"ur Theoretische Physik\\
  Eidgen\"ossische Technische Hochschule Z\"urich\\
  Wolfgang-Pauli-Strasse 27, 8093 Z\"urich, Switzerland\\[1ex]
  \href{mailto:nbeisert@itp.phys.ethz.ch}
  {\texttt{nbeisert@itp.phys.ethz.ch}}}
\hypersetup{pdfauthor={Niklas Beisert}}
\hypersetup{pdfsubject={Manual for the LaTeX2e Package childdoc}}
\date{30 December 2018, \textsf{v2.0}}
\maketitle

\begin{abstract}\noindent
\textsf{childdoc} is a \LaTeXe{} package
that enables the direct compilation
of document sections included by |\include|
to individual files.
\end{abstract}

\begingroup
\parskip0ex
\tableofcontents
\endgroup

%%%%%%%%%%%%%%%%%%%%%%%%%%%%%%%%%%%%%%%%%%%%%%%%%%%%%%%%%%%%%%%%%%%%%%%%%%%%%%%%
%%%%%%%%%%%%%%%%%%%%%%%%%%%%%%%%%%%%%%%%%%%%%%%%%%%%%%%%%%%%%%%%%%%%%%%%%%%%%%%%
\section{Introduction}

\LaTeX{} provides a mechanism to structure a large document (such as a book)
into a main file and several child files (containing the chapters)
using the |\include| command.
This mechanism is beneficial for documents
which span hundreds of pages in order to
make the source file(s) more manageable.
Moreover, compilation can be restricted to
selected child files by means of the |\includeonly| command.
The latter feature can be used to reduce the compilation time while editing
(this was significantly more useful in the earlier days of \LaTeX{})
or to generate a smaller document which is easier to navigate.
Another application of |\includeonly| is to generate
documents consisting of selected parts of the complete document.

However, there are a few drawbacks of the plain |\include| mechanism:
\begin{itemize}
\item
The child files cannot be compiled on their own,
they can only be compiled via the main file.
A naive editing environment
(such as a text editor with an option
to have the current file processed by \LaTeX)
may require one to switch to the main file before compiling;
attempting to compile the child file produces errors.
\item
The main file must be modified (each time)
to adjust the |\includeonly| command
to the present needs. This easily leaves the main file in a messy state.
\item
The generated document will always carry the filename
of the main document. This is inconvenient if
several child files are to be compiled and
to be kept for distribution.
\end{itemize}

The present package provides a simple interface
to make child files individually compilable by \LaTeX{}.
Compiling a child file then has the same effect as compiling
the main file with an |\includeonly| command
to select the appropriate child.
Moreover the generated document will carry the name of the child
rather than the main file.
This resolves all three above issues.

This feature is meant to make the editing of books,
thesis documents and lecture notes somewhat more convenient.
However, the package can also be used efficiently for
composing a series of documents (such as exercise sheets)
which are typically distributed individually.
It then assists the author in generating the individual documents
(potentially in different versions)
as well as a document containing the collected series.
Another application is in developing style files
or other kinds of included material
where compilation of the style file could redirect
to a sample or test file.

%%%%%%%%%%%%%%%%%%%%%%%%%%%%%%%%%%%%%%%%%%%%%%%%%%%%%%%%%%%%%%%%%%%%%%%%%%%%%%%%
%%%%%%%%%%%%%%%%%%%%%%%%%%%%%%%%%%%%%%%%%%%%%%%%%%%%%%%%%%%%%%%%%%%%%%%%%%%%%%%%
\section{Usage}

First of all, the package \textsf{childdoc} is \emph{not} a standard
\LaTeXe{} |.sty| style file! Therefore it needs to be invoked in
a non-standard way.

%%%%%%%%%%%%%%%%%%%%%%%%%%%%%%%%%%%%%%%%%%%%%%%%%%%%%%%%%%%%%%%%%%%%%%%%%%%%%%%%
\subsection{Included Files}
\label{sec:include}

%%%%%%%%%%%%%%%%%%%%%%%%%%%%%%%%%%%%%%%%
\DescribeMacro{\childdocmain}
To use the package, add the commands
\begin{center}
\begin{tabular}{l}
|\input{childdoc.def}|\\
|\childdocmain{}|\\
\end{tabular}
\end{center}
at the very top of the main \LaTeX{} file,
in particular \emph{before} the |\documentclass| statement!
The argument of |\childdocmain| should be left empty
(but it must be present).

%%%%%%%%%%%%%%%%%%%%%%%%%%%%%%%%%%%%%%%%
\DescribeMacro{\childdocof}
Furthermore, add the commands
\begin{center}
\begin{tabular}{l}
|\input{childdoc.def}|\\
|\childdocof{|\textit{main}|}|\\
\end{tabular}
\end{center}
at the top of every child file \textit{child}
which is included by |\include{|\textit{child}|}|
from within the main file
(or at least for those files to be compiled individually).
The argument \textit{main} must be the filename of the main file.

There are a couple of
considerations in setting up the main and child documents:

%%%%%%%%%%%%%%%%%%%%%%%%%%%%%%%%%%%%%%%%
\paragraph{Restrictions.}

Please note the following restrictions:
\begin{itemize}
\item
|\childdocmain| must be called with one argument \textit{main}
to ensure compatibility with earlier version of the package.
It must either be empty (|\childdocmain{}|)
or precisely match the filename of the main file in which it is specified.
See \secref{sec:detection} for further information.
\item
The filename \textit{main} must be specified without the |.tex| extension.
\item
The filename \textit{main} is case sensitive
(even in case-insensitive file systems)
due to internal string comparison.
\item
The argument \textit{main} should be fully expanded, it cannot be a macro.
\item
Subdirectories and special characters should be avoided in filenames.
\item
The command |\childdocmain{|\textit{main}|}| must be followed by a whitespace.
It should not be followed immediately by another command
or by a comment mark `|%|'.
This is because the \TeX{} parser reads the token immediately following
the argument of |\childdocmain| and puts it
at the beginning of every child section;
however, a white\-space is ignored.
\end{itemize}

%%%%%%%%%%%%%%%%%%%%%%%%%%%%%%%%%%%%%%%%
\paragraph{Content of Main File.}

It is advisable to place all content in the child files included by |\include|.
Any output contained in the main file will appear in all child documents
unless suppressed manually;
it cannot be suppressed automatically by the |\includeonly| directive
and thus should normally be avoided.
A method to include some content in the main file
by means of conditional processing is described in \secref{sec:conditional}.

%%%%%%%%%%%%%%%%%%%%%%%%%%%%%%%%%%%%%%%%
\paragraph{Page Numbering.}

When only a part of the document is compiled,
the appropriate numbering of pages
(as well as other status parameters)
is determined from the |.aux| files.
The latter contain information from previous passes.
However this information needs to propagate through
all intermediate child documents.
Therefore the page numbering in child documents may well
be inconsistent until the complete document is compiled at least once.

A useful (if unconventional) way to always ensure a consistent
page numbering is to restart the numbering in each child document
and denote the pages by `\textit{child}|.|\textit{page}'
where \textit{child} represents the chapter/section number of the child file.
This can be achieved by the command
|\numberwithin{page}{|\textit{child}|}|
of the \textsf{amsmath} package
where \textit{child} can be |chapter| or |section|
depending on the chosen structuring.
Alternatively, one can modify the macro |\thepage| appropriately
and reset the counter |page| at the start of each child file.

%%%%%%%%%%%%%%%%%%%%%%%%%%%%%%%%%%%%%%%%%%%%%%%%%%%%%%%%%%%%%%%%%%%%%%%%%%%%%%%%
\subsection{Conditional Processing}
\label{sec:conditional}

The package provides a mechanism to compile different versions
of a document. To customise the versions further some conditional processing
can come in handy to distinguish which version is being compiled.
The package provides two macros to describe the compilation context:

%%%%%%%%%%%%%%%%%%%%%%%%%%%%%%%%%%%%%%%%
\DescribeMacro{\ifchilddoc}
The conditional |\ifchilddoc| distinguishes between the compilation of
child documents and the main document:
%
\begin{center}
|\ifchilddoc |\textit{child-code}| |[|\||else |\textit{main-code}]| \||fi|
\end{center}

%%%%%%%%%%%%%%%%%%%%%%%%%%%%%%%%%%%%%%%%
\DescribeMacro{\childdocname}
\DescribeMacro{\childdocjob}
The macro |\childdocname| contains the filename (without extension)
of the main or child file being processed.
Note that |\childdocjob| will always contain the name of the main file.

%%%%%%%%%%%%%%%%%%%%%%%%%%%%%%%%%%%%%%%%
\paragraph{Title Page.}

Conditional processing can be used to include a title or banner page
in the main document when proper precautions are taken.
Importantly, the code in the main file should ensure that the page counter
(as well as other status parameters which are stored in the |.aux| files)
takes the same value after the conditional processing.
Otherwise the page numbers may take divergent values
depending on which part is compiled.

For example, a title page could be declared by:
%
\begin{center}
\begin{tabular}{l}
|\ifchilddoc\||else|\\
|\addtocounter{page}{-1}|\\
\textit{code for title page}\\
|\newpage|\\
|\||fi|
\end{tabular}
\end{center}
%
A banner page for the child documents can be generated by:
%
\begin{center}
\begin{tabular}{l}
|\ifchilddoc|\\
|\addtocounter{page}{-1}|\\
\textit{code for banner page}\\
|\newpage|\\
|\||fi|
\end{tabular}
\end{center}
%
Here one could write a message such as:
\begin{center}
|This is the part \childdocname{} of \childdocjob{}.|
\end{center}

%%%%%%%%%%%%%%%%%%%%%%%%%%%%%%%%%%%%%%%%%%%%%%%%%%%%%%%%%%%%%%%%%%%%%%%%%%%%%%%%
\subsection{Flags}
\label{sec:flags}

The package makes it easy to generate different versions
of the main or child documents.
To this end compilation flags can be defined
and assigned different default values.
They will be particularly useful in conjunction
with the forwarding mechanism described in \secref{sec:forward}.

For example, it may be useful to have a flag |\version|
which can be set to |draft| or |final|.
The document source will contain some conditional code
depending on the value of |\version|.
Suppose further, the flag should default to |final| for the main file
and to |draft| for child files
which is a natural assignment for editing the document.
This is achieved by placing the following code
in the preamble of the main document
(below the |\childdocmain| directive):
%
\begin{center}
\begin{tabular}{l}
|\ifchilddoc|\\
|\providecommand{\version}{draft}|\\
|\||else|\\
|\providecommand{\version}{final}|\\
|\||fi|
\end{tabular}
\end{center}
%
The definition by |\providecommand| makes sure
that previous definitions are not overwritten.
Further statements |\providecommand{\version}{...}|
can thus be added before the above code to override it.

For the main file, one might add a line
(between |\childdocmain| and the above block)
%
\begin{center}
|%\ifchilddoc\||else\providecommand{\version}{draft}\||fi|
\end{center}
%
which can be uncommented to produce a draft version.
Likewise one can add a line to the very top of a child file
(above the |\childdocof{|\textit{main}|}| directive)
%
\begin{center}
|%\providecommand{\version}{final}|
\end{center}
%
which can be uncommented to produce the final version of this child document.

%%%%%%%%%%%%%%%%%%%%%%%%%%%%%%%%%%%%%%%%%%%%%%%%%%%%%%%%%%%%%%%%%%%%%%%%%%%%%%%%
\subsection{Forwarding}
\label{sec:forward}

Different versions of the main or child documents
using compilation flags as described in \secref{sec:flags}
can be (permanently) stored in different files
for convenient compilation, viewing and distribution.
To this end, the package defines a command
to pass on compilation to a different file:

%%%%%%%%%%%%%%%%%%%%%%%%%%%%%%%%%%%%%%%%
\DescribeMacro{\childdocforward}
The command |\childdocforward| redirects processing to
another source file:
%
\begin{center}
\begin{tabular}{l}
|\input{childdoc.def}|\\
|\childdocforward[|\textit{main}|]{|\textit{dest}|}|\\
\end{tabular}
\end{center}
%
The argument \textit{dest} is the destination file
(without extension).
It should be the main file or one of the child files.
Note that further \textsf{childdoc} directives
such as |\childdocof| and |\childdocforward|
in the indicated file will be processed in this form.
The optional argument \textit{main}
passes on directly to the main file \textit{main}
while pretending to compile the child \textit{dest}.
This form behaves as if \textit{dest}
issues |\childdocof{|\textit{main}|}| right away,
and no further \textsf{childdoc} directives will be processed.

%%%%%%%%%%%%%%%%%%%%%%%%%%%%%%%%%%%%%%%%
\DescribeMacro{\...prefix}
In the alternative form |\childdocforwardprefix|,
%
\begin{center}
\begin{tabular}{l}
|\input{childdoc.def}|\\
|\childdocforwardprefix[|\textit{main}|]{|\textit{prefix}|}{|\textit{dest}|}|
\end{tabular}
\end{center}
%
the destination file is determined by a pattern
depending on the current file:
To make this work, the current file must be called
`{\textit{prefix}\hspace{0.2em}\textit{suffix}}'
with \textit{prefix} matching precisely the argument.
Processing is then passed on to the file
`{\textit{dest}\hspace{0.2em}\textit{suffix}}'.
Surely, the same effect is achieved by
directly specifying the
argument `{\textit{dest}\hspace{0.2em}\textit{suffix}}'
in the first form.
However, that requires to set up a different file
for each child. With the alternative form of the command
all these files can have exactly the same content
which simplifies setting them up and maintaining them.

For example, the following file |draft.tex|
with a compilation flag |\version| as described in \secref{sec:flags}
compiles the main document as a draft:
%
\begin{center}
\begin{tabular}{l}
|\def\version{draft}|\\
|\input{childdoc.def}|\\
|\childdocforward{|\textit{main}|}|
\end{tabular}
\end{center}
%
Likewise, the following files |final|\textit{nn}|.tex|
compile the final version of the child document
|child|\textit{nn}|.tex|:
%
\begin{center}
\begin{tabular}{l}
|\def\version{final}|\\
|\input{childdoc.def}|\\
|\childdocforwardprefix{final}{child}|
\end{tabular}
\end{center}
%

Note that when several versions of a main file and/or of each child file
are to be generated, it may be convenient to set up a |Makefile| or
shell script to automatise the process.

%%%%%%%%%%%%%%%%%%%%%%%%%%%%%%%%%%%%%%%%%%%%%%%%%%%%%%%%%%%%%%%%%%%%%%%%%%%%%%%%
\subsection{Command Line Processing}
\label{sec:commandline}

The effect of redirection files can also be achieved by invoking
the \LaTeX{} compiler with a more elaborate command line.
Most conveniently this should be done as part
of a shell script or a |Makefile|.

When using \textsf{childdoc} in the main file, the following
command lines effectively perform a redirection
(note that depending on the shell being used,
backslashes may have to be doubled: `|\|' $\to$ `|\\|'):
%
\begin{center}
|... -jobname "|\textit{target}|" |\\|"|[\textit{flags}]%
|\input{childdoc.def}\childdocforward[|\textit{main}|]{|\textit{dest}|}"|
\end{center}
%
Here \textit{target} is the name of the output file,
\textit{main} is the name of the main file
and \textit{dest} is the name of the main or child file to be processed
(all filenames without extensions).
The optional argument \textit{main} can be omitted
if \textit{main} matches \textit{dest}.
Optionally, compilation \textit{flags} can be defined via |\def| commands.
This command line makes the \TeX{} engine believe
it is compiling the file \textit{target}
whose content is specified as the latter parameter.
The provided code then forwards the processing to
\textit{main} or \textit{dest} as described in \secref{sec:forward}.

%%%%%%%%%%%%%%%%%%%%%%%%%%%%%%%%%%%%%%%%%%%%%%%%%%%%%%%%%%%%%%%%%%%%%%%%%%%%%%%%
\subsection{Include by Input}
\label{sec:input}

Including child documents by |\include| has some restrictions by design.
Most notably, the content of a child document always occupies
its own set of pages; pages cannot be shared between child documents.
Usually, this behaviour makes perfect sense
because each child document contain an essential part of the document.
However, in some situations it may be desirable to compose
a document from a collection of parts
without having mandatory page breaks between then.
For this case, the package
provides a mechanism to include parts
by |\input| which can also be processed individually.
However, by construction this mechanism
requires manual handling of the content to be output.

%%%%%%%%%%%%%%%%%%%%%%%%%%%%%%%%%%%%%%%%
\DescribeMacro{\ifchilddocmanual}
The main file should be prepared as usual, see \secref{sec:include}.
However, the document body must make a distinction
between processing of an individual part and of the main document, e.g.:
%
\begin{center}
\begin{tabular}{l}
|\ifchilddocmanual|\\
|\input{\childdocname}|\\
|\||else|\\
\textit{document body with }|\input{|\textit{part}|}|\\
|\||fi|
\end{tabular}
\end{center}
%
The conditional |\ifchilddocmanual| is true whenever
a part to be included by |\input| is being compiled,
and the name of the part is stored in |\childdocname|.

%%%%%%%%%%%%%%%%%%%%%%%%%%%%%%%%%%%%%%%%
\DescribeMacro{\childdocby}
Each part to be included by |\input| should start with:
%
\begin{center}
\begin{tabular}{l}
|\input{childdoc.def}|\\
|\childdocby{|\textit{main}|}|\\
\end{tabular}
\end{center}
%
The directive |\childdocby| is similar to |\childdocof|
described in \secref{sec:include},
but the subsequent selection of content must be done manually.
To that end, both |\ifchilddoc| and |\ifchilddocmanual|
will be true upon processing of a part,
and the name of the part is stored in |\childdocname|.
Note that |\jobname| will be set to the filename of the current part
so that each part receives an individual |.aux| file
that does not interfere with the |.aux| file(s) of the main document.
This behaviour can be altered by the alternative form
|\childdocby[*]{|\textit{main}|}| (with a non-empty optional argument)
which uses the |.aux| file of the main document
by setting |\jobname| to \textit{main}.

%%%%%%%%%%%%%%%%%%%%%%%%%%%%%%%%%%%%%%%%%%%%%%%%%%%%%%%%%%%%%%%%%%%%%%%%%%%%%%%%
\subsection{Driver Development}
\label{sec:driver}

The \textsf{childdoc} mechanism can also be use for the development
of definition files such as \LaTeX{} styles or classes.
This case differs from the above setup with multiple parts
included by |\include| in that no |\includeonly| should be invoked.
This can be achieved by starting the include file
(before |\ProvidesPackage|) with:
%
\begin{center}
\begin{tabular}{l}
|\input{childdoc.def}|\\
|\childdocforward{|\textit{main}|}|\\
\end{tabular}
\end{center}
%
or alternatively with:
%
\begin{center}
\begin{tabular}{l}
|\input{childdoc.def}|\\
|\childdocby{|\textit{main}|}|\\
\end{tabular}
\end{center}
%
Both forms have slightly different effects as described above.
The main file is prepared as usual, see \secref{sec:include}.

%%%%%%%%%%%%%%%%%%%%%%%%%%%%%%%%%%%%%%%%%%%%%%%%%%%%%%%%%%%%%%%%%%%%%%%%%%%%%%%%
\subsection{Legacy Detection}
\label{sec:detection}

The directive |\childdocmain| in the main file can detect
whether the complete document or merely a child is to be compiled
even without using the directive |\childdocof|.
This method is deprecated because it is less robust
and there is no compelling reason to use it;
it is merely provided for backward compatibility
and it may be removed in future versions.

If the detection mechanism is to be used,
it is mandatory to correctly specify
the filename of the main file as the argument of |\childdocmain|:
%
\begin{center}
\begin{tabular}{l}
|\input{childdoc.def}|\\
|\childdocmain{|\textit{main}|}|\\
\end{tabular}
\end{center}
%
If |\jobname| does not match the argument \textit{main} of |\childdocmain|,
it is assumed that |\jobname| points to the child file to be compiled.
When using |\childdocmain| with the main file specified as argument,
it suffices to start a child file
with just |\input{|\textit{main}|}|
without loading of the package and using |\childdocof|.
If instead all processing is done
with the appropriate \textsf{childdoc} directives,
the argument of \textit{main} of |\childdocmain| can be empty.

An alternative version of the command line processing described
in \secref{sec:commandline} using the detection mechanism reads:
%
\begin{center}
|... -jobname "|\textit{target}|" "|[\textit{flags}]%
[|\def\jobname{|\textit{dest}|}|]|\input{|\textit{main}|}"|
\end{center}

%%%%%%%%%%%%%%%%%%%%%%%%%%%%%%%%%%%%%%%%%%%%%%%%%%%%%%%%%%%%%%%%%%%%%%%%%%%%%%%%
\subsection{Manual Code}
\label{sec:manual}

In case one cannot be certain whether the definitions file |childdoc.def|
is installed on the target \TeX{} distribution
and one prefers not to ship it,
it is conceivable to paste a few relevant commands into the sources.

To that end, drop all statements |\input{childdoc.def}|
and perform the replacements as outlined below.
Instead of |\childdocmain{|\textit{main}|}| add the following code
to the top of the main file:
%
\begin{center}
\begin{tabular}{l}
|\||ifdefined\childdocname\endinput\||fi\newif\ifchilddoc|\\
|\edef\childdocname{\scantokens\expandafter{\jobname\noexpand}}|\\
|\def\childdocmain{|\textit{main}|}\||ifx\childdocmain\childdocname\||else|\\
|\childdoctrue\includeonly{\childdocname}\let\jobname\childdocmain\||fi|\\
\end{tabular}
\end{center}
%
Instead of |\childdocof{|\textit{main}|}| just include the main file
at the top of each child file:
%
\begin{center}
|\input{|\textit{main}|}|
\end{center}
%
A simple redirection |\childdocforward{|\textit{dest}|}| is achieved by:
%
\begin{center}
|\def\jobname{|\textit{dest}|}\input{\jobname}|
\end{center}
%
The redirection with prefix
|\childdocforwardprefix[|\textit{prefix}|]{|\textit{dest}|}|
is accomplished by:
%
\begin{center}
\begin{tabular}{l}
|{\edef\jobname{\scantokens\expandafter{\jobname\noexpand}}|\\
|\def\redirectjob |\textit{prefix}|#1~~~{\gdef\jobname{|\textit{dest}|#1}}|\\
|\expandafter\redirectjob\jobname~~~}\input{\jobname}|
\end{tabular}
\end{center}

In an alternative approach,
child documents can be compiled by a specific command line
without additional code or specific definitions:
%
\begin{center}
|... -jobname "|\textit{target}|" "|[\textit{flags}]%
|\includeonly{|\textit{dest}|}\input{|\textit{main}|}"|
\end{center}
%

%%%%%%%%%%%%%%%%%%%%%%%%%%%%%%%%%%%%%%%%%%%%%%%%%%%%%%%%%%%%%%%%%%%%%%%%%%%%%%%%
%%%%%%%%%%%%%%%%%%%%%%%%%%%%%%%%%%%%%%%%%%%%%%%%%%%%%%%%%%%%%%%%%%%%%%%%%%%%%%%%
\section{Information}

%%%%%%%%%%%%%%%%%%%%%%%%%%%%%%%%%%%%%%%%%%%%%%%%%%%%%%%%%%%%%%%%%%%%%%%%%%%%%%%%
\subsection{Copyright}

Copyright \copyright{} 2017--2018 Niklas Beisert

This work may be distributed and/or modified under the
conditions of the \LaTeX{} Project Public License, either version 1.3
of this license or (at your option) any later version.
The latest version of this license is in
  \url{http://www.latex-project.org/lppl.txt}
and version 1.3 or later is part of all distributions of \LaTeX{}
version 2005/12/01 or later.

This work has the LPPL maintenance status `maintained'.

The Current Maintainer of this work is Niklas Beisert.

This work consists of the files |README.txt|, |childdoc.ins| and |childdoc.dtx|
as well as the derived files |childdoc.def|, |cdocsamp.tex|
with |cdocsch1.tex|, |cdocsch2.tex|, |cdocspt3.tex|, |cdocspt4.tex|,
|cdocsdrf.tex|, |cdocsfn1.tex|, |cdocsfn2.tex|
as well as |childdoc.pdf|.

%%%%%%%%%%%%%%%%%%%%%%%%%%%%%%%%%%%%%%%%%%%%%%%%%%%%%%%%%%%%%%%%%%%%%%%%%%%%%%%%
\subsection{Files and Installation}

The package consists of the files:
%
\begin{center}
\begin{tabular}{ll}
    |README.txt|   & readme file \\
    |childdoc.ins| & installation file \\
    |childdoc.dtx| & source file \\
    |childdoc.def| & definition file \\
    |cdocsamp.tex| & sample main file \\
    |cdocsch1.tex| & sample include file \\
    |cdocsch2.tex| & sample include file \\
    |cdocspt3.tex| & sample part file \\
    |cdocspt4.tex| & sample part file \\
    |cdocsdrf.tex| & sample redirection file \\
    |cdocsfn1.tex| & sample redirection file \\
    |cdocsfn2.tex| & sample redirection file \\
    |childdoc.pdf| & manual
\end{tabular}
\end{center}
%
The distribution consists of the files
|README.txt|, |childdoc.ins| and |childdoc.dtx|.
%
\begin{itemize}
\item
Run (pdf)\LaTeX{} on |childdoc.dtx|
to compile the manual |childdoc.pdf| (this file).
\item
Run \LaTeX{} on |childdoc.ins| to create the definitions file |childdoc.def|
and the sample |cdocsamp.tex| with include files
|cdocsch1.tex|, |cdocsch2.tex|, |cdocspt3.tex|, |cdocspt4.tex|,
|cdocsdrf.tex|, |cdocsfn1.tex|, |cdocsfn2.tex|.
Then copy the file |childdoc.def| to an appropriate directory of your \LaTeX{}
distribution, e.g.\ \textit{texmf-root}|/tex/latex/childdoc|.
\end{itemize}

%%%%%%%%%%%%%%%%%%%%%%%%%%%%%%%%%%%%%%%%%%%%%%%%%%%%%%%%%%%%%%%%%%%%%%%%%%%%%%%%
\subsection{Related CTAN Packages}

There are several other packages which offer a similar functionality:
%
\begin{itemize}
\item
The packages
\href{http://ctan.org/pkg/docmute}{\textsf{docmute}},
\href{http://ctan.org/pkg/includex}{\textsf{includex}} and
\href{http://ctan.org/pkg/standalone}{\textsf{standalone}}
provide commands to include only the document body of
a child file thus allowing both files to be compiled individually.
\item
The packages \href{http://ctan.org/pkg/subdocs}{\textsf{subdocs}}
and \href{http://ctan.org/pkg/subfiles}{\textsf{subfiles}}
provide structures in which the main and child documents can be
encapsulated and allowing them to be compiled individually.
The inclusion mechanism is different from the conventional |\include|.
\item
The package \href{http://ctan.org/pkg/combine}{\textsf{combine}}
is an elaborate solution to combine several documents into one.
\end{itemize}
%
See also the CTAN topic \href{http://ctan.org/topic/subdocs}{\textsf{subdocs}}
for further related packages.
The present package differs from the above solutions in that
a document structure constructed with the conventional |\include| mechanism
just needs two extra commands at the top of every file
such that all constituent files can be compiled individually.

%%%%%%%%%%%%%%%%%%%%%%%%%%%%%%%%%%%%%%%%%%%%%%%%%%%%%%%%%%%%%%%%%%%%%%%%%%%%%%%%
%\subsection{Feature Suggestions}
%
%The following is a list of features which may be useful for future
%versions of this package:
%%
%\begin{itemize}
%\item
%\ldots
%\end{itemize}

%%%%%%%%%%%%%%%%%%%%%%%%%%%%%%%%%%%%%%%%%%%%%%%%%%%%%%%%%%%%%%%%%%%%%%%%%%%%%%%%
\subsection{Revision History}

%%%%%%%%%%%%%%%%%%%%%%%%%%%%%%%%%%%%%%%%
\paragraph{v2.0:} 2018/12/30

\begin{itemize}
\item
immediate forward processing
\item
added |\childdocby| mechanism
\item
manual restructured
\end{itemize}

%%%%%%%%%%%%%%%%%%%%%%%%%%%%%%%%%%%%%%%%
\paragraph{v1.6:} 2018/01/17

\begin{itemize}
\item
application for development of include files
\item
corrections to manual
\end{itemize}

%%%%%%%%%%%%%%%%%%%%%%%%%%%%%%%%%%%%%%%%
\paragraph{v1.5:} 2017/05/21

\begin{itemize}
\item
more complete structuring introduced
\item
|\childdocof| introduced
\item
|\childdoc| renamed to |\childdocmain|
\item
|\childredirect| renamed to |\childdocforward| and |\childdocforwardprefix|
and functionality expanded
\end{itemize}

%%%%%%%%%%%%%%%%%%%%%%%%%%%%%%%%%%%%%%%%
\paragraph{v1.0:} 2017/04/27

\begin{itemize}
\item
manual and install package
\item
first version published on CTAN
\end{itemize}

%%%%%%%%%%%%%%%%%%%%%%%%%%%%%%%%%%%%%%%%
\paragraph{v0.6:} 2017/04/26

\begin{itemize}
\item
redirection mechanism added
\end{itemize}

%%%%%%%%%%%%%%%%%%%%%%%%%%%%%%%%%%%%%%%%
\paragraph{v0.5:} 2017/04/26

\begin{itemize}
\item
functionality in definition file
\end{itemize}


%%%%%%%%%%%%%%%%%%%%%%%%%%%%%%%%%%%%%%%%%%%%%%%%%%%%%%%%%%%%%%%%%%%%%%%%%%%%%%%%
%%%%%%%%%%%%%%%%%%%%%%%%%%%%%%%%%%%%%%%%%%%%%%%%%%%%%%%%%%%%%%%%%%%%%%%%%%%%%%%%
%%%%%%%%%%%%%%%%%%%%%%%%%%%%%%%%%%%%%%%%%%%%%%%%%%%%%%%%%%%%%%%%%%%%%%%%%%%%%%%%
\appendix

\settowidth\MacroIndent{\rmfamily\scriptsize 000\ }

 \DocInput{childdoc.dtx}

\end{document}
%</driver>
% \fi
%
% %%%%%%%%%%%%%%%%%%%%%%%%%%%%%%%%%%%%%%%%%%%%%%%%%%%%%%%%%%%%%%%%%%%%%%%%%%%%%%
% %%%%%%%%%%%%%%%%%%%%%%%%%%%%%%%%%%%%%%%%%%%%%%%%%%%%%%%%%%%%%%%%%%%%%%%%%%%%%%
% \section{Sample}
%\iffalse
%<*samplemain>
%\fi
%
% The following presents a sample document
% with two chapters, two parts, a title page,
% a compile flag as well as three forwarding files to set the flag.
% It consists of eight |.tex| files:
% \begin{center}
% \begin{tabular}{ll}
% |cdocsamp.tex|&main file\\
% |cdocsch1.tex|&include file for chapter 1\\
% |cdocsch2.tex|&include file for chapter 2\\
% |cdocspt3.tex|&include file for part 3\\
% |cdocspt4.tex|&include file for part 4\\
% |cdocsdrf.tex|&forwarding file for main file in draft mode\\
% |cdocsfi1.tex|&forwarding file for final version of chapter 1\\
% |cdocsfi2.tex|&forwarding file for final version of chapter 2\\
% \end{tabular}
% \end{center}
% Each of the eight files can be compiled directly by the \LaTeX{} compiler.
%
% %%%%%%%%%%%%%%%%%%%%%%%%%%%%%%%%%%%%%%
% \paragraph{Main File.}
%
% The main file is called |cdocsamp.tex|.
%
% Load the \textsf{childdoc} definitions and
% declare the filename for the main document:
%    \begin{macrocode}
\input{childdoc.def}
\childdocmain{}
%    \end{macrocode}

% Optional override for |\version| flag:
%    \begin{macrocode}
%%\ifchilddoc\else\providecommand{\version}{draft}\fi
%    \end{macrocode}

% Define the default values for the |\version| flag
% (|final| for the main file and |draft| for childs):
%    \begin{macrocode}
\ifchilddoc
\providecommand{\version}{draft}
\else
\providecommand{\version}{final}
\fi
%    \end{macrocode}

% Load the standard document class:
%    \begin{macrocode}
\documentclass[12pt]{article}
%    \end{macrocode}

% Start the document body:
%    \begin{macrocode}
\begin{document}
%    \end{macrocode}

% Declare a title page.
% Print title, part of document being processed and version flag:
%    \begin{macrocode}
\addtocounter{page}{-1}
\begin{center}
{\LARGE\bfseries{}childdoc example\par}
\vspace{1cm}
\ifchilddoc
\ifchilddocmanual part\else chapter\fi:
`\childdocname' of `\childdocjob'\par
\else
main document: `\childdocjob'\par
\fi
version: \version\par
\end{center}
\newpage
%    \end{macrocode}

% Manually include selected file,
% otherwise process as usual:
%    \begin{macrocode}
\ifchilddocmanual
\section*{part `\childdocname'}
\input{\childdocname}
\else
%    \end{macrocode}

% Include the two chapters:
%    \begin{macrocode}
\include{cdocsch1}
\include{cdocsch2}
%    \end{macrocode}

% Include the two parts unless only chapters should be displayed:
%    \begin{macrocode}
\ifchilddoc\else
\section{part three}
\input{cdocspt3}
\section{part four}
\input{cdocspt4}
\fi
%    \end{macrocode}

% Process as usual until here:
%    \begin{macrocode}
\fi
%    \end{macrocode}

% End of document body:
%    \begin{macrocode}
\end{document}
%    \end{macrocode}
%\iffalse
%</samplemain>
%\fi
%
% %%%%%%%%%%%%%%%%%%%%%%%%%%%%%%%%%%%%%%
% \paragraph{Chapter Include Files.}
%
% The include files are called |cdocsch1.tex| and |cdocsch2.tex|.
%
%\iffalse
%<*samplechap1|samplechap2>
%\fi

% Optional override for |\version| flag:
%    \begin{macrocode}
%%\providecommand{\version}{final}
%    \end{macrocode}

% Include the main document:
%    \begin{macrocode}
\input{childdoc.def}
\childdocof{cdocsamp}
%    \end{macrocode}

%\iffalse
%</samplechap1|samplechap2>
%\fi
%
%\iffalse
%<*samplechap1>
%\fi
% Some text for chapter 1:
%    \begin{macrocode}
\section{one}
some text in chapter one
%    \end{macrocode}

%\iffalse
%</samplechap1>
%\fi
% Some text for chapter 2:
%\iffalse
%<*samplechap2>
%\fi
%    \begin{macrocode}
\section{two}
more text in chapter two
%    \end{macrocode}

%\iffalse
%</samplechap2>
%\fi
%
% %%%%%%%%%%%%%%%%%%%%%%%%%%%%%%%%%%%%%%
% \paragraph{Part Include Files.}
%
% The include files are called |cdocspt3.tex| and |cdocspt4.tex|.
%
%\iffalse
%<*samplepart3|samplepart4>
%\fi

% Optional override for |\version| flag:
%    \begin{macrocode}
%%\providecommand{\version}{final}
%    \end{macrocode}

% Include the main document:
%    \begin{macrocode}
\input{childdoc.def}
\childdocby{cdocsamp}
%    \end{macrocode}

%\iffalse
%</samplepart3|samplepart4>
%\fi
%
%\iffalse
%<*samplepart3>
%\fi
% Some text for part 3:
%    \begin{macrocode}
some text in part three
%    \end{macrocode}

%\iffalse
%</samplepart3>
%\fi
% Some text for part 4:
%\iffalse
%<*samplepart4>
%\fi
%    \begin{macrocode}
more text in part four
%    \end{macrocode}

%\iffalse
%</samplepart4>
%\fi
%
% %%%%%%%%%%%%%%%%%%%%%%%%%%%%%%%%%%%%%%
% \paragraph{Forwarding for a Complete Draft.}
%
% The following forwarding file |cdocsdrf.tex|
% compiles the main document in draft mode:
%\iffalse
%<*sampledraft>
%\fi
%    \begin{macrocode}
\def\version{draft}
\input{childdoc.def}
\childdocforward{cdocsamp}
%    \end{macrocode}

%\iffalse
%</sampledraft>
%\fi
%
% %%%%%%%%%%%%%%%%%%%%%%%%%%%%%%%%%%%%%%
% \paragraph{Forwarding for Final Version of the Chapters.}
%
% The following forwarding files |cdocsfn1.tex| and |cdocsfn2.tex|
% (with identical content)
% compile the final versions of the child documents
% |cdocsch1.tex| and |cdocsch2.tex|, respectively:
%\iffalse
%<*samplefinal>
%\fi
%    \begin{macrocode}
\def\version{final}
\input{childdoc.def}
\childdocforwardprefix[cdocsamp]{cdocsfn}{cdocsch}
%    \end{macrocode}

%\iffalse
%</samplefinal>
%\fi
%
% %%%%%%%%%%%%%%%%%%%%%%%%%%%%%%%%%%%%%%
% \paragraph{Command Line Processing.}
%
% The following three command lines generate the output files
% |cdocscld|, |cdocscl1| and |cdocscl2|
% which should be identical to
% |cdocsdrf|, |cdocsch1| and |cdocsfn2|, respectively:
% \begin{center}
% \begin{tabular}{l}
% |latex -jobname cdocscld \|\\
% |  "\def\version{draft}\input{childdoc.def}\childdocforward{cdocsamp}"|\\
% |latex -jobname cdocscl1 \|\\
% |  "\input{childdoc.def}\childdocforward[cdocsamp]{cdocsch1}"|\\
% |latex -jobname cdocscl2 \|\\
% |  "\def\version{final}\input{childdoc.def}\childdocforward{cdocsch2}"|
% \end{tabular}
% \end{center}
% Note that the trailing backslash on each first line
% merely continues the input to the second line
% (for convenient cut ant paste).
% Furthermore, the command |latex| can be replaced by any
% of its alternative versions such as |pdflatex|.
%
% %%%%%%%%%%%%%%%%%%%%%%%%%%%%%%%%%%%%%%%%%%%%%%%%%%%%%%%%%%%%%%%%%%%%%%%%%%%%%%
% %%%%%%%%%%%%%%%%%%%%%%%%%%%%%%%%%%%%%%%%%%%%%%%%%%%%%%%%%%%%%%%%%%%%%%%%%%%%%%
% \section{Implementation}
%\iffalse
%<*package>
%\fi
%
% This section describes the definitions file |childdoc.def|.

% The definitions cannot be loaded using |\usepackage| or |\RequirePackage|
% which has a mechanism to prevent loading a style file more than once.
% When loading the definitions by means of |\input|
% multiple instances have to be prevented manually:
%\iffalse
%This code needs to be before the `\ProvidesFile' directive
%which is defined at the beginning of this file.
%Therefore it is also placed there and commented out here.
%</package>
%<*discard>
%\fi
%    \begin{macrocode}
\ifdefined\childdocmain\endinput\fi
%    \end{macrocode}
%\iffalse
%</discard>
%<*package>
%\fi
%
% \macro{\ifchilddoc}
% \macro{\ifchilddocmanual}
% The conditional |\ifchilddoc| tells whether a
% child (true) or main (false) document is being compiled.
% The conditional |\ifchilddocmanual| tells whether
% the |\includeonly| mechanism is used (false) or
% the selection of child files must be performed manually (true).
% The definitions initialise to false:
%    \begin{macrocode}
\newif\ifchilddoc
\newif\ifchilddocmanual
%    \end{macrocode}

% \macro{\childdocname}
% \macro{\childdocjob}
% The macro |\childdocname| stores the name of the main document
% to be compiled. The macro |\childdocjob| stores the name of
% the document on which the \LaTeX{} compiler was originally invoked.
% The content of |\jobname| cannot be compared
% to filenames specified in the source due to different catcodes.
% The following code rescans |\jobname|, stores the result
% in |\childdocname| and saves a copy in |\childdocjob|:
%    \begin{macrocode}
\edef\childdocname{\scantokens\expandafter{\jobname\noexpand}}
\let\childdocjob\childdocname
%    \end{macrocode}

% \macro{\childdocdisable}
% The macro |\childdocdisable| prevents the main file
% from being processed more than once.
% At this stage, the main document command |\childdocmain|
% is assumed to be called once again where it should do nothing.
% Any subsequent call to it should prevent
% a secondary processing of the main document
% It overwrites the forwarding commands
% |\childdocof| and |\childdocforward|
% with empty macros to prevent further inclusions of the main document:
%    \begin{macrocode}
\newcommand{\childdocdisable}
{
  \renewcommand{\childdocmain}[1]{\renewcommand{\childdocmain}[1]{\endinput}}
  \renewcommand{\childdocof}[1]{}
  \renewcommand{\childdocby}[2][]{}
  \renewcommand{\childdocforward}[2][]{}
  \renewcommand{\childdocdisable}{}
}
%    \end{macrocode}

% \macro{\childdocmain}
% The macro |\childdocmain| is to be called at the top of the main file
% with nothing or the main filename (without extension) as argument.
% First, it breaks loops.
% If the argument is not empty and does not match |\childdocname|
% (which is set by the first inclusion of |childdoc.def|),
% |\ifchilddoc| is set to true, |\includeonly| is applied to the child file
% and |\jobname| is set to the main file
% (for proper handling of |.aux| files):
%    \begin{macrocode}
\newcommand{\childdocmain}[1]
{
  \childdocdisable\childdocmain{}
  \if?#1?\else
    \begingroup
      \def\childdoctmp{#1}
      \ifx\childdoctmp\childdocname
        \def\childdoctmp{}
      \else
        \def\childdoctmp
        {
          \childdoctrue
          \includeonly{\childdocname}
          \def\childdocjob{#1}
          \def\jobname{#1}
        }
      \fi
      \expandafter
    \endgroup
    \childdoctmp
  \fi
}
%    \end{macrocode}

% \macro{\childdocof}
% The command |\childdocof| redirects
% compilation to the main file |#1|.
%    \begin{macrocode}
\newcommand{\childdocof}[1]
{
  \childdocdisable
  \childdoctrue
  \includeonly{\childdocname}
  \def\jobname{#1}
  \def\childdocjob{#1}
  \input{#1}
}
%    \end{macrocode}

% \macro{\childdocby}
% The command |\childdocby| ....
%    \begin{macrocode}
\newcommand{\childdocby}[2][]
{
  \childdocdisable
  \childdoctrue
  \childdocmanualtrue
  \if?#1?\else
    \def\jobname{#2}
  \fi
  \def\childdocjob{#2}
  \input{#2}
  \endinput
}
%    \end{macrocode}

% \macro{\childdocforward}
% The command |\childdocforward| redirects
% compilation to the main file or
% (if the optional argument is given) a child file.
% Parameters are set as if the main file
% or a child file starting with |\childdocof| was compiled.
% Then compilation is handed over to the main file:
%    \begin{macrocode}
\newcommand{\childdocforward}[2][]
{
  \begingroup
    \if?#1?
      \def\childdoctmp
      {
        \def\childdocname{#2}
        \def\childdocjob{#2}
        \def\jobname{#2}
        \input{#2}
        \endinput
      }
    \else
      \def\childdoctmp
      {
        \childdocdisable
        \def\childdocname{#2}
        \childdoctrue
        \includeonly{#2}
        \def\childdocjob{#1}
        \def\jobname{#1}
        \input{#1}
        \endinput
      }
    \fi
    \expandafter
  \endgroup
  \childdoctmp
}
%    \end{macrocode}

% \macro{\childdocforwardprefix}
% The command |\childdocforwardprefix| redirects
% compilation to the main or a child file by means of a pattern.
% The prefix |#1| in the current filename is replaced by |#2|
% and the suffix of the current filename is kept
% (it is assumed that the filename does not contain the substring `|~~~|'
% which is used as a delimiter).
% Compilation is handed over to the new file by |\childdocforward|:
%    \begin{macrocode}
\newcommand{\childdocforwardprefix}[3][]
{
  \begingroup
    \def\childdocextract #2##1~~~{\def\childdoctmp{\childdocforward[#1]{#3##1}}}
    \expandafter\childdocextract\childdocname~~~
    \expandafter
  \endgroup
  \childdoctmp
}
%    \end{macrocode}

% \macro{\childdoc}
% The deprecated macro |\childdoc| is a legacy version of |\childdocmain|:
%    \begin{macrocode}
\newcommand{\childdoc}{\childdocmain}
%    \end{macrocode}

% \macro{\childdocredirect}
% The deprecated macro |\childdocredirect| is a legacy version
% of |\childdocforward| and |\childdocforwardprefix|:
%    \begin{macrocode}
\newcommand{\childdocredirect}[2][]
{
  \begingroup
    \if?#1?
      \def\childdoctmp{\childdocforward{#2}}
    \else
      \def\childdoctmp{\childdocforwardprefix{#1}{#2}}
    \fi
    \expandafter
  \endgroup
  \childdoctmp
}
%    \end{macrocode}

%\iffalse
%</package>
%\fi
%
\endinput
|\\
|\childdocby{|\textit{main}|}|\\
\end{tabular}
\end{center}
%
Both forms have slightly different effects as described above.
The main file is prepared as usual, see \secref{sec:include}.

%%%%%%%%%%%%%%%%%%%%%%%%%%%%%%%%%%%%%%%%%%%%%%%%%%%%%%%%%%%%%%%%%%%%%%%%%%%%%%%%
\subsection{Legacy Detection}
\label{sec:detection}

The directive |\childdocmain| in the main file can detect
whether the complete document or merely a child is to be compiled
even without using the directive |\childdocof|.
This method is deprecated because it is less robust
and there is no compelling reason to use it;
it is merely provided for backward compatibility
and it may be removed in future versions.

If the detection mechanism is to be used,
it is mandatory to correctly specify
the filename of the main file as the argument of |\childdocmain|:
%
\begin{center}
\begin{tabular}{l}
|% \iffalse
%
% childdoc.dtx Copyright (C) 2017-2018 Niklas Beisert
%
% This work may be distributed and/or modified under the
% conditions of the LaTeX Project Public License, either version 1.3
% of this license or (at your option) any later version.
% The latest version of this license is in
%   http://www.latex-project.org/lppl.txt
% and version 1.3 or later is part of all distributions of LaTeX
% version 2005/12/01 or later.
%
% This work has the LPPL maintenance status `maintained'.
%
% The Current Maintainer of this work is Niklas Beisert.
%
% This work consists of the files childdoc.dtx and childdoc.ins
% and the derived files childdoc.def and cdocsamp.tex with
% cdocsch1.tex, cdocsch2.tex, cdocsdrf.tex, cdocsfn1.tex, cdocsfn2.tex.
%
%<package>\ifdefined\childdocmain\endinput\fi
%<package>\ProvidesFile{childdoc.def}[2018/12/30 v2.0 child document driver]
%<samplemain>\ProvidesFile{cdocsamp.tex}[2018/12/30 v2.0 sample for childdoc]
%<*driver>
%\ProvidesFile{childdoc.drv}[2018/12/30 v2.0 childdoc reference manual file]
\PassOptionsToClass{10pt,a4paper}{article}
\documentclass{ltxdoc}

\usepackage[margin=35mm]{geometry}
\usepackage{hyperref}
\usepackage{hyperxmp}
\usepackage[usenames]{color}

\hypersetup{colorlinks=true}
\hypersetup{pdfstartview=FitH}
\hypersetup{pdfpagemode=UseNone}
\hypersetup{pdfsource={}}
\hypersetup{pdflang={en-UK}}
\hypersetup{pdfcopyright={Copyright 2017-2018 Niklas Beisert.
  This work may be distributed and/or modified under the
  conditions of the LaTeX Project Public License, either version 1.3
  of this license or (at your option) any later version.}}
\hypersetup{pdflicenseurl={http://www.latex-project.org/lppl.txt}}
\hypersetup{pdfcontactaddress={ETH Zurich, ITP, HIT K,
  Wolfgang-Pauli-Strasse 27}}
\hypersetup{pdfcontactpostcode={8093}}
\hypersetup{pdfcontactcity={Zurich}}
\hypersetup{pdfcontactcountry={Switzerland}}
\hypersetup{pdfcontactemail={nbeisert@itp.phys.ethz.ch}}
\hypersetup{pdfcontacturl={http://people.phys.ethz.ch/\xmptilde nbeisert/}}

\newcommand{\secref}[1]{\hyperref[#1]{section \ref*{#1}}}

\parskip1ex
\parindent0pt
\let\olditemize\itemize
\def\itemize{\olditemize\parskip0pt}

\begin{document}

\title{The \textsf{childdoc} Package}
\hypersetup{pdftitle={The childdoc Package}}
\author{Niklas Beisert\\[2ex]
  Institut f\"ur Theoretische Physik\\
  Eidgen\"ossische Technische Hochschule Z\"urich\\
  Wolfgang-Pauli-Strasse 27, 8093 Z\"urich, Switzerland\\[1ex]
  \href{mailto:nbeisert@itp.phys.ethz.ch}
  {\texttt{nbeisert@itp.phys.ethz.ch}}}
\hypersetup{pdfauthor={Niklas Beisert}}
\hypersetup{pdfsubject={Manual for the LaTeX2e Package childdoc}}
\date{30 December 2018, \textsf{v2.0}}
\maketitle

\begin{abstract}\noindent
\textsf{childdoc} is a \LaTeXe{} package
that enables the direct compilation
of document sections included by |\include|
to individual files.
\end{abstract}

\begingroup
\parskip0ex
\tableofcontents
\endgroup

%%%%%%%%%%%%%%%%%%%%%%%%%%%%%%%%%%%%%%%%%%%%%%%%%%%%%%%%%%%%%%%%%%%%%%%%%%%%%%%%
%%%%%%%%%%%%%%%%%%%%%%%%%%%%%%%%%%%%%%%%%%%%%%%%%%%%%%%%%%%%%%%%%%%%%%%%%%%%%%%%
\section{Introduction}

\LaTeX{} provides a mechanism to structure a large document (such as a book)
into a main file and several child files (containing the chapters)
using the |\include| command.
This mechanism is beneficial for documents
which span hundreds of pages in order to
make the source file(s) more manageable.
Moreover, compilation can be restricted to
selected child files by means of the |\includeonly| command.
The latter feature can be used to reduce the compilation time while editing
(this was significantly more useful in the earlier days of \LaTeX{})
or to generate a smaller document which is easier to navigate.
Another application of |\includeonly| is to generate
documents consisting of selected parts of the complete document.

However, there are a few drawbacks of the plain |\include| mechanism:
\begin{itemize}
\item
The child files cannot be compiled on their own,
they can only be compiled via the main file.
A naive editing environment
(such as a text editor with an option
to have the current file processed by \LaTeX)
may require one to switch to the main file before compiling;
attempting to compile the child file produces errors.
\item
The main file must be modified (each time)
to adjust the |\includeonly| command
to the present needs. This easily leaves the main file in a messy state.
\item
The generated document will always carry the filename
of the main document. This is inconvenient if
several child files are to be compiled and
to be kept for distribution.
\end{itemize}

The present package provides a simple interface
to make child files individually compilable by \LaTeX{}.
Compiling a child file then has the same effect as compiling
the main file with an |\includeonly| command
to select the appropriate child.
Moreover the generated document will carry the name of the child
rather than the main file.
This resolves all three above issues.

This feature is meant to make the editing of books,
thesis documents and lecture notes somewhat more convenient.
However, the package can also be used efficiently for
composing a series of documents (such as exercise sheets)
which are typically distributed individually.
It then assists the author in generating the individual documents
(potentially in different versions)
as well as a document containing the collected series.
Another application is in developing style files
or other kinds of included material
where compilation of the style file could redirect
to a sample or test file.

%%%%%%%%%%%%%%%%%%%%%%%%%%%%%%%%%%%%%%%%%%%%%%%%%%%%%%%%%%%%%%%%%%%%%%%%%%%%%%%%
%%%%%%%%%%%%%%%%%%%%%%%%%%%%%%%%%%%%%%%%%%%%%%%%%%%%%%%%%%%%%%%%%%%%%%%%%%%%%%%%
\section{Usage}

First of all, the package \textsf{childdoc} is \emph{not} a standard
\LaTeXe{} |.sty| style file! Therefore it needs to be invoked in
a non-standard way.

%%%%%%%%%%%%%%%%%%%%%%%%%%%%%%%%%%%%%%%%%%%%%%%%%%%%%%%%%%%%%%%%%%%%%%%%%%%%%%%%
\subsection{Included Files}
\label{sec:include}

%%%%%%%%%%%%%%%%%%%%%%%%%%%%%%%%%%%%%%%%
\DescribeMacro{\childdocmain}
To use the package, add the commands
\begin{center}
\begin{tabular}{l}
|\input{childdoc.def}|\\
|\childdocmain{}|\\
\end{tabular}
\end{center}
at the very top of the main \LaTeX{} file,
in particular \emph{before} the |\documentclass| statement!
The argument of |\childdocmain| should be left empty
(but it must be present).

%%%%%%%%%%%%%%%%%%%%%%%%%%%%%%%%%%%%%%%%
\DescribeMacro{\childdocof}
Furthermore, add the commands
\begin{center}
\begin{tabular}{l}
|\input{childdoc.def}|\\
|\childdocof{|\textit{main}|}|\\
\end{tabular}
\end{center}
at the top of every child file \textit{child}
which is included by |\include{|\textit{child}|}|
from within the main file
(or at least for those files to be compiled individually).
The argument \textit{main} must be the filename of the main file.

There are a couple of
considerations in setting up the main and child documents:

%%%%%%%%%%%%%%%%%%%%%%%%%%%%%%%%%%%%%%%%
\paragraph{Restrictions.}

Please note the following restrictions:
\begin{itemize}
\item
|\childdocmain| must be called with one argument \textit{main}
to ensure compatibility with earlier version of the package.
It must either be empty (|\childdocmain{}|)
or precisely match the filename of the main file in which it is specified.
See \secref{sec:detection} for further information.
\item
The filename \textit{main} must be specified without the |.tex| extension.
\item
The filename \textit{main} is case sensitive
(even in case-insensitive file systems)
due to internal string comparison.
\item
The argument \textit{main} should be fully expanded, it cannot be a macro.
\item
Subdirectories and special characters should be avoided in filenames.
\item
The command |\childdocmain{|\textit{main}|}| must be followed by a whitespace.
It should not be followed immediately by another command
or by a comment mark `|%|'.
This is because the \TeX{} parser reads the token immediately following
the argument of |\childdocmain| and puts it
at the beginning of every child section;
however, a white\-space is ignored.
\end{itemize}

%%%%%%%%%%%%%%%%%%%%%%%%%%%%%%%%%%%%%%%%
\paragraph{Content of Main File.}

It is advisable to place all content in the child files included by |\include|.
Any output contained in the main file will appear in all child documents
unless suppressed manually;
it cannot be suppressed automatically by the |\includeonly| directive
and thus should normally be avoided.
A method to include some content in the main file
by means of conditional processing is described in \secref{sec:conditional}.

%%%%%%%%%%%%%%%%%%%%%%%%%%%%%%%%%%%%%%%%
\paragraph{Page Numbering.}

When only a part of the document is compiled,
the appropriate numbering of pages
(as well as other status parameters)
is determined from the |.aux| files.
The latter contain information from previous passes.
However this information needs to propagate through
all intermediate child documents.
Therefore the page numbering in child documents may well
be inconsistent until the complete document is compiled at least once.

A useful (if unconventional) way to always ensure a consistent
page numbering is to restart the numbering in each child document
and denote the pages by `\textit{child}|.|\textit{page}'
where \textit{child} represents the chapter/section number of the child file.
This can be achieved by the command
|\numberwithin{page}{|\textit{child}|}|
of the \textsf{amsmath} package
where \textit{child} can be |chapter| or |section|
depending on the chosen structuring.
Alternatively, one can modify the macro |\thepage| appropriately
and reset the counter |page| at the start of each child file.

%%%%%%%%%%%%%%%%%%%%%%%%%%%%%%%%%%%%%%%%%%%%%%%%%%%%%%%%%%%%%%%%%%%%%%%%%%%%%%%%
\subsection{Conditional Processing}
\label{sec:conditional}

The package provides a mechanism to compile different versions
of a document. To customise the versions further some conditional processing
can come in handy to distinguish which version is being compiled.
The package provides two macros to describe the compilation context:

%%%%%%%%%%%%%%%%%%%%%%%%%%%%%%%%%%%%%%%%
\DescribeMacro{\ifchilddoc}
The conditional |\ifchilddoc| distinguishes between the compilation of
child documents and the main document:
%
\begin{center}
|\ifchilddoc |\textit{child-code}| |[|\||else |\textit{main-code}]| \||fi|
\end{center}

%%%%%%%%%%%%%%%%%%%%%%%%%%%%%%%%%%%%%%%%
\DescribeMacro{\childdocname}
\DescribeMacro{\childdocjob}
The macro |\childdocname| contains the filename (without extension)
of the main or child file being processed.
Note that |\childdocjob| will always contain the name of the main file.

%%%%%%%%%%%%%%%%%%%%%%%%%%%%%%%%%%%%%%%%
\paragraph{Title Page.}

Conditional processing can be used to include a title or banner page
in the main document when proper precautions are taken.
Importantly, the code in the main file should ensure that the page counter
(as well as other status parameters which are stored in the |.aux| files)
takes the same value after the conditional processing.
Otherwise the page numbers may take divergent values
depending on which part is compiled.

For example, a title page could be declared by:
%
\begin{center}
\begin{tabular}{l}
|\ifchilddoc\||else|\\
|\addtocounter{page}{-1}|\\
\textit{code for title page}\\
|\newpage|\\
|\||fi|
\end{tabular}
\end{center}
%
A banner page for the child documents can be generated by:
%
\begin{center}
\begin{tabular}{l}
|\ifchilddoc|\\
|\addtocounter{page}{-1}|\\
\textit{code for banner page}\\
|\newpage|\\
|\||fi|
\end{tabular}
\end{center}
%
Here one could write a message such as:
\begin{center}
|This is the part \childdocname{} of \childdocjob{}.|
\end{center}

%%%%%%%%%%%%%%%%%%%%%%%%%%%%%%%%%%%%%%%%%%%%%%%%%%%%%%%%%%%%%%%%%%%%%%%%%%%%%%%%
\subsection{Flags}
\label{sec:flags}

The package makes it easy to generate different versions
of the main or child documents.
To this end compilation flags can be defined
and assigned different default values.
They will be particularly useful in conjunction
with the forwarding mechanism described in \secref{sec:forward}.

For example, it may be useful to have a flag |\version|
which can be set to |draft| or |final|.
The document source will contain some conditional code
depending on the value of |\version|.
Suppose further, the flag should default to |final| for the main file
and to |draft| for child files
which is a natural assignment for editing the document.
This is achieved by placing the following code
in the preamble of the main document
(below the |\childdocmain| directive):
%
\begin{center}
\begin{tabular}{l}
|\ifchilddoc|\\
|\providecommand{\version}{draft}|\\
|\||else|\\
|\providecommand{\version}{final}|\\
|\||fi|
\end{tabular}
\end{center}
%
The definition by |\providecommand| makes sure
that previous definitions are not overwritten.
Further statements |\providecommand{\version}{...}|
can thus be added before the above code to override it.

For the main file, one might add a line
(between |\childdocmain| and the above block)
%
\begin{center}
|%\ifchilddoc\||else\providecommand{\version}{draft}\||fi|
\end{center}
%
which can be uncommented to produce a draft version.
Likewise one can add a line to the very top of a child file
(above the |\childdocof{|\textit{main}|}| directive)
%
\begin{center}
|%\providecommand{\version}{final}|
\end{center}
%
which can be uncommented to produce the final version of this child document.

%%%%%%%%%%%%%%%%%%%%%%%%%%%%%%%%%%%%%%%%%%%%%%%%%%%%%%%%%%%%%%%%%%%%%%%%%%%%%%%%
\subsection{Forwarding}
\label{sec:forward}

Different versions of the main or child documents
using compilation flags as described in \secref{sec:flags}
can be (permanently) stored in different files
for convenient compilation, viewing and distribution.
To this end, the package defines a command
to pass on compilation to a different file:

%%%%%%%%%%%%%%%%%%%%%%%%%%%%%%%%%%%%%%%%
\DescribeMacro{\childdocforward}
The command |\childdocforward| redirects processing to
another source file:
%
\begin{center}
\begin{tabular}{l}
|\input{childdoc.def}|\\
|\childdocforward[|\textit{main}|]{|\textit{dest}|}|\\
\end{tabular}
\end{center}
%
The argument \textit{dest} is the destination file
(without extension).
It should be the main file or one of the child files.
Note that further \textsf{childdoc} directives
such as |\childdocof| and |\childdocforward|
in the indicated file will be processed in this form.
The optional argument \textit{main}
passes on directly to the main file \textit{main}
while pretending to compile the child \textit{dest}.
This form behaves as if \textit{dest}
issues |\childdocof{|\textit{main}|}| right away,
and no further \textsf{childdoc} directives will be processed.

%%%%%%%%%%%%%%%%%%%%%%%%%%%%%%%%%%%%%%%%
\DescribeMacro{\...prefix}
In the alternative form |\childdocforwardprefix|,
%
\begin{center}
\begin{tabular}{l}
|\input{childdoc.def}|\\
|\childdocforwardprefix[|\textit{main}|]{|\textit{prefix}|}{|\textit{dest}|}|
\end{tabular}
\end{center}
%
the destination file is determined by a pattern
depending on the current file:
To make this work, the current file must be called
`{\textit{prefix}\hspace{0.2em}\textit{suffix}}'
with \textit{prefix} matching precisely the argument.
Processing is then passed on to the file
`{\textit{dest}\hspace{0.2em}\textit{suffix}}'.
Surely, the same effect is achieved by
directly specifying the
argument `{\textit{dest}\hspace{0.2em}\textit{suffix}}'
in the first form.
However, that requires to set up a different file
for each child. With the alternative form of the command
all these files can have exactly the same content
which simplifies setting them up and maintaining them.

For example, the following file |draft.tex|
with a compilation flag |\version| as described in \secref{sec:flags}
compiles the main document as a draft:
%
\begin{center}
\begin{tabular}{l}
|\def\version{draft}|\\
|\input{childdoc.def}|\\
|\childdocforward{|\textit{main}|}|
\end{tabular}
\end{center}
%
Likewise, the following files |final|\textit{nn}|.tex|
compile the final version of the child document
|child|\textit{nn}|.tex|:
%
\begin{center}
\begin{tabular}{l}
|\def\version{final}|\\
|\input{childdoc.def}|\\
|\childdocforwardprefix{final}{child}|
\end{tabular}
\end{center}
%

Note that when several versions of a main file and/or of each child file
are to be generated, it may be convenient to set up a |Makefile| or
shell script to automatise the process.

%%%%%%%%%%%%%%%%%%%%%%%%%%%%%%%%%%%%%%%%%%%%%%%%%%%%%%%%%%%%%%%%%%%%%%%%%%%%%%%%
\subsection{Command Line Processing}
\label{sec:commandline}

The effect of redirection files can also be achieved by invoking
the \LaTeX{} compiler with a more elaborate command line.
Most conveniently this should be done as part
of a shell script or a |Makefile|.

When using \textsf{childdoc} in the main file, the following
command lines effectively perform a redirection
(note that depending on the shell being used,
backslashes may have to be doubled: `|\|' $\to$ `|\\|'):
%
\begin{center}
|... -jobname "|\textit{target}|" |\\|"|[\textit{flags}]%
|\input{childdoc.def}\childdocforward[|\textit{main}|]{|\textit{dest}|}"|
\end{center}
%
Here \textit{target} is the name of the output file,
\textit{main} is the name of the main file
and \textit{dest} is the name of the main or child file to be processed
(all filenames without extensions).
The optional argument \textit{main} can be omitted
if \textit{main} matches \textit{dest}.
Optionally, compilation \textit{flags} can be defined via |\def| commands.
This command line makes the \TeX{} engine believe
it is compiling the file \textit{target}
whose content is specified as the latter parameter.
The provided code then forwards the processing to
\textit{main} or \textit{dest} as described in \secref{sec:forward}.

%%%%%%%%%%%%%%%%%%%%%%%%%%%%%%%%%%%%%%%%%%%%%%%%%%%%%%%%%%%%%%%%%%%%%%%%%%%%%%%%
\subsection{Include by Input}
\label{sec:input}

Including child documents by |\include| has some restrictions by design.
Most notably, the content of a child document always occupies
its own set of pages; pages cannot be shared between child documents.
Usually, this behaviour makes perfect sense
because each child document contain an essential part of the document.
However, in some situations it may be desirable to compose
a document from a collection of parts
without having mandatory page breaks between then.
For this case, the package
provides a mechanism to include parts
by |\input| which can also be processed individually.
However, by construction this mechanism
requires manual handling of the content to be output.

%%%%%%%%%%%%%%%%%%%%%%%%%%%%%%%%%%%%%%%%
\DescribeMacro{\ifchilddocmanual}
The main file should be prepared as usual, see \secref{sec:include}.
However, the document body must make a distinction
between processing of an individual part and of the main document, e.g.:
%
\begin{center}
\begin{tabular}{l}
|\ifchilddocmanual|\\
|\input{\childdocname}|\\
|\||else|\\
\textit{document body with }|\input{|\textit{part}|}|\\
|\||fi|
\end{tabular}
\end{center}
%
The conditional |\ifchilddocmanual| is true whenever
a part to be included by |\input| is being compiled,
and the name of the part is stored in |\childdocname|.

%%%%%%%%%%%%%%%%%%%%%%%%%%%%%%%%%%%%%%%%
\DescribeMacro{\childdocby}
Each part to be included by |\input| should start with:
%
\begin{center}
\begin{tabular}{l}
|\input{childdoc.def}|\\
|\childdocby{|\textit{main}|}|\\
\end{tabular}
\end{center}
%
The directive |\childdocby| is similar to |\childdocof|
described in \secref{sec:include},
but the subsequent selection of content must be done manually.
To that end, both |\ifchilddoc| and |\ifchilddocmanual|
will be true upon processing of a part,
and the name of the part is stored in |\childdocname|.
Note that |\jobname| will be set to the filename of the current part
so that each part receives an individual |.aux| file
that does not interfere with the |.aux| file(s) of the main document.
This behaviour can be altered by the alternative form
|\childdocby[*]{|\textit{main}|}| (with a non-empty optional argument)
which uses the |.aux| file of the main document
by setting |\jobname| to \textit{main}.

%%%%%%%%%%%%%%%%%%%%%%%%%%%%%%%%%%%%%%%%%%%%%%%%%%%%%%%%%%%%%%%%%%%%%%%%%%%%%%%%
\subsection{Driver Development}
\label{sec:driver}

The \textsf{childdoc} mechanism can also be use for the development
of definition files such as \LaTeX{} styles or classes.
This case differs from the above setup with multiple parts
included by |\include| in that no |\includeonly| should be invoked.
This can be achieved by starting the include file
(before |\ProvidesPackage|) with:
%
\begin{center}
\begin{tabular}{l}
|\input{childdoc.def}|\\
|\childdocforward{|\textit{main}|}|\\
\end{tabular}
\end{center}
%
or alternatively with:
%
\begin{center}
\begin{tabular}{l}
|\input{childdoc.def}|\\
|\childdocby{|\textit{main}|}|\\
\end{tabular}
\end{center}
%
Both forms have slightly different effects as described above.
The main file is prepared as usual, see \secref{sec:include}.

%%%%%%%%%%%%%%%%%%%%%%%%%%%%%%%%%%%%%%%%%%%%%%%%%%%%%%%%%%%%%%%%%%%%%%%%%%%%%%%%
\subsection{Legacy Detection}
\label{sec:detection}

The directive |\childdocmain| in the main file can detect
whether the complete document or merely a child is to be compiled
even without using the directive |\childdocof|.
This method is deprecated because it is less robust
and there is no compelling reason to use it;
it is merely provided for backward compatibility
and it may be removed in future versions.

If the detection mechanism is to be used,
it is mandatory to correctly specify
the filename of the main file as the argument of |\childdocmain|:
%
\begin{center}
\begin{tabular}{l}
|\input{childdoc.def}|\\
|\childdocmain{|\textit{main}|}|\\
\end{tabular}
\end{center}
%
If |\jobname| does not match the argument \textit{main} of |\childdocmain|,
it is assumed that |\jobname| points to the child file to be compiled.
When using |\childdocmain| with the main file specified as argument,
it suffices to start a child file
with just |\input{|\textit{main}|}|
without loading of the package and using |\childdocof|.
If instead all processing is done
with the appropriate \textsf{childdoc} directives,
the argument of \textit{main} of |\childdocmain| can be empty.

An alternative version of the command line processing described
in \secref{sec:commandline} using the detection mechanism reads:
%
\begin{center}
|... -jobname "|\textit{target}|" "|[\textit{flags}]%
[|\def\jobname{|\textit{dest}|}|]|\input{|\textit{main}|}"|
\end{center}

%%%%%%%%%%%%%%%%%%%%%%%%%%%%%%%%%%%%%%%%%%%%%%%%%%%%%%%%%%%%%%%%%%%%%%%%%%%%%%%%
\subsection{Manual Code}
\label{sec:manual}

In case one cannot be certain whether the definitions file |childdoc.def|
is installed on the target \TeX{} distribution
and one prefers not to ship it,
it is conceivable to paste a few relevant commands into the sources.

To that end, drop all statements |\input{childdoc.def}|
and perform the replacements as outlined below.
Instead of |\childdocmain{|\textit{main}|}| add the following code
to the top of the main file:
%
\begin{center}
\begin{tabular}{l}
|\||ifdefined\childdocname\endinput\||fi\newif\ifchilddoc|\\
|\edef\childdocname{\scantokens\expandafter{\jobname\noexpand}}|\\
|\def\childdocmain{|\textit{main}|}\||ifx\childdocmain\childdocname\||else|\\
|\childdoctrue\includeonly{\childdocname}\let\jobname\childdocmain\||fi|\\
\end{tabular}
\end{center}
%
Instead of |\childdocof{|\textit{main}|}| just include the main file
at the top of each child file:
%
\begin{center}
|\input{|\textit{main}|}|
\end{center}
%
A simple redirection |\childdocforward{|\textit{dest}|}| is achieved by:
%
\begin{center}
|\def\jobname{|\textit{dest}|}\input{\jobname}|
\end{center}
%
The redirection with prefix
|\childdocforwardprefix[|\textit{prefix}|]{|\textit{dest}|}|
is accomplished by:
%
\begin{center}
\begin{tabular}{l}
|{\edef\jobname{\scantokens\expandafter{\jobname\noexpand}}|\\
|\def\redirectjob |\textit{prefix}|#1~~~{\gdef\jobname{|\textit{dest}|#1}}|\\
|\expandafter\redirectjob\jobname~~~}\input{\jobname}|
\end{tabular}
\end{center}

In an alternative approach,
child documents can be compiled by a specific command line
without additional code or specific definitions:
%
\begin{center}
|... -jobname "|\textit{target}|" "|[\textit{flags}]%
|\includeonly{|\textit{dest}|}\input{|\textit{main}|}"|
\end{center}
%

%%%%%%%%%%%%%%%%%%%%%%%%%%%%%%%%%%%%%%%%%%%%%%%%%%%%%%%%%%%%%%%%%%%%%%%%%%%%%%%%
%%%%%%%%%%%%%%%%%%%%%%%%%%%%%%%%%%%%%%%%%%%%%%%%%%%%%%%%%%%%%%%%%%%%%%%%%%%%%%%%
\section{Information}

%%%%%%%%%%%%%%%%%%%%%%%%%%%%%%%%%%%%%%%%%%%%%%%%%%%%%%%%%%%%%%%%%%%%%%%%%%%%%%%%
\subsection{Copyright}

Copyright \copyright{} 2017--2018 Niklas Beisert

This work may be distributed and/or modified under the
conditions of the \LaTeX{} Project Public License, either version 1.3
of this license or (at your option) any later version.
The latest version of this license is in
  \url{http://www.latex-project.org/lppl.txt}
and version 1.3 or later is part of all distributions of \LaTeX{}
version 2005/12/01 or later.

This work has the LPPL maintenance status `maintained'.

The Current Maintainer of this work is Niklas Beisert.

This work consists of the files |README.txt|, |childdoc.ins| and |childdoc.dtx|
as well as the derived files |childdoc.def|, |cdocsamp.tex|
with |cdocsch1.tex|, |cdocsch2.tex|, |cdocspt3.tex|, |cdocspt4.tex|,
|cdocsdrf.tex|, |cdocsfn1.tex|, |cdocsfn2.tex|
as well as |childdoc.pdf|.

%%%%%%%%%%%%%%%%%%%%%%%%%%%%%%%%%%%%%%%%%%%%%%%%%%%%%%%%%%%%%%%%%%%%%%%%%%%%%%%%
\subsection{Files and Installation}

The package consists of the files:
%
\begin{center}
\begin{tabular}{ll}
    |README.txt|   & readme file \\
    |childdoc.ins| & installation file \\
    |childdoc.dtx| & source file \\
    |childdoc.def| & definition file \\
    |cdocsamp.tex| & sample main file \\
    |cdocsch1.tex| & sample include file \\
    |cdocsch2.tex| & sample include file \\
    |cdocspt3.tex| & sample part file \\
    |cdocspt4.tex| & sample part file \\
    |cdocsdrf.tex| & sample redirection file \\
    |cdocsfn1.tex| & sample redirection file \\
    |cdocsfn2.tex| & sample redirection file \\
    |childdoc.pdf| & manual
\end{tabular}
\end{center}
%
The distribution consists of the files
|README.txt|, |childdoc.ins| and |childdoc.dtx|.
%
\begin{itemize}
\item
Run (pdf)\LaTeX{} on |childdoc.dtx|
to compile the manual |childdoc.pdf| (this file).
\item
Run \LaTeX{} on |childdoc.ins| to create the definitions file |childdoc.def|
and the sample |cdocsamp.tex| with include files
|cdocsch1.tex|, |cdocsch2.tex|, |cdocspt3.tex|, |cdocspt4.tex|,
|cdocsdrf.tex|, |cdocsfn1.tex|, |cdocsfn2.tex|.
Then copy the file |childdoc.def| to an appropriate directory of your \LaTeX{}
distribution, e.g.\ \textit{texmf-root}|/tex/latex/childdoc|.
\end{itemize}

%%%%%%%%%%%%%%%%%%%%%%%%%%%%%%%%%%%%%%%%%%%%%%%%%%%%%%%%%%%%%%%%%%%%%%%%%%%%%%%%
\subsection{Related CTAN Packages}

There are several other packages which offer a similar functionality:
%
\begin{itemize}
\item
The packages
\href{http://ctan.org/pkg/docmute}{\textsf{docmute}},
\href{http://ctan.org/pkg/includex}{\textsf{includex}} and
\href{http://ctan.org/pkg/standalone}{\textsf{standalone}}
provide commands to include only the document body of
a child file thus allowing both files to be compiled individually.
\item
The packages \href{http://ctan.org/pkg/subdocs}{\textsf{subdocs}}
and \href{http://ctan.org/pkg/subfiles}{\textsf{subfiles}}
provide structures in which the main and child documents can be
encapsulated and allowing them to be compiled individually.
The inclusion mechanism is different from the conventional |\include|.
\item
The package \href{http://ctan.org/pkg/combine}{\textsf{combine}}
is an elaborate solution to combine several documents into one.
\end{itemize}
%
See also the CTAN topic \href{http://ctan.org/topic/subdocs}{\textsf{subdocs}}
for further related packages.
The present package differs from the above solutions in that
a document structure constructed with the conventional |\include| mechanism
just needs two extra commands at the top of every file
such that all constituent files can be compiled individually.

%%%%%%%%%%%%%%%%%%%%%%%%%%%%%%%%%%%%%%%%%%%%%%%%%%%%%%%%%%%%%%%%%%%%%%%%%%%%%%%%
%\subsection{Feature Suggestions}
%
%The following is a list of features which may be useful for future
%versions of this package:
%%
%\begin{itemize}
%\item
%\ldots
%\end{itemize}

%%%%%%%%%%%%%%%%%%%%%%%%%%%%%%%%%%%%%%%%%%%%%%%%%%%%%%%%%%%%%%%%%%%%%%%%%%%%%%%%
\subsection{Revision History}

%%%%%%%%%%%%%%%%%%%%%%%%%%%%%%%%%%%%%%%%
\paragraph{v2.0:} 2018/12/30

\begin{itemize}
\item
immediate forward processing
\item
added |\childdocby| mechanism
\item
manual restructured
\end{itemize}

%%%%%%%%%%%%%%%%%%%%%%%%%%%%%%%%%%%%%%%%
\paragraph{v1.6:} 2018/01/17

\begin{itemize}
\item
application for development of include files
\item
corrections to manual
\end{itemize}

%%%%%%%%%%%%%%%%%%%%%%%%%%%%%%%%%%%%%%%%
\paragraph{v1.5:} 2017/05/21

\begin{itemize}
\item
more complete structuring introduced
\item
|\childdocof| introduced
\item
|\childdoc| renamed to |\childdocmain|
\item
|\childredirect| renamed to |\childdocforward| and |\childdocforwardprefix|
and functionality expanded
\end{itemize}

%%%%%%%%%%%%%%%%%%%%%%%%%%%%%%%%%%%%%%%%
\paragraph{v1.0:} 2017/04/27

\begin{itemize}
\item
manual and install package
\item
first version published on CTAN
\end{itemize}

%%%%%%%%%%%%%%%%%%%%%%%%%%%%%%%%%%%%%%%%
\paragraph{v0.6:} 2017/04/26

\begin{itemize}
\item
redirection mechanism added
\end{itemize}

%%%%%%%%%%%%%%%%%%%%%%%%%%%%%%%%%%%%%%%%
\paragraph{v0.5:} 2017/04/26

\begin{itemize}
\item
functionality in definition file
\end{itemize}


%%%%%%%%%%%%%%%%%%%%%%%%%%%%%%%%%%%%%%%%%%%%%%%%%%%%%%%%%%%%%%%%%%%%%%%%%%%%%%%%
%%%%%%%%%%%%%%%%%%%%%%%%%%%%%%%%%%%%%%%%%%%%%%%%%%%%%%%%%%%%%%%%%%%%%%%%%%%%%%%%
%%%%%%%%%%%%%%%%%%%%%%%%%%%%%%%%%%%%%%%%%%%%%%%%%%%%%%%%%%%%%%%%%%%%%%%%%%%%%%%%
\appendix

\settowidth\MacroIndent{\rmfamily\scriptsize 000\ }

 \DocInput{childdoc.dtx}

\end{document}
%</driver>
% \fi
%
% %%%%%%%%%%%%%%%%%%%%%%%%%%%%%%%%%%%%%%%%%%%%%%%%%%%%%%%%%%%%%%%%%%%%%%%%%%%%%%
% %%%%%%%%%%%%%%%%%%%%%%%%%%%%%%%%%%%%%%%%%%%%%%%%%%%%%%%%%%%%%%%%%%%%%%%%%%%%%%
% \section{Sample}
%\iffalse
%<*samplemain>
%\fi
%
% The following presents a sample document
% with two chapters, two parts, a title page,
% a compile flag as well as three forwarding files to set the flag.
% It consists of eight |.tex| files:
% \begin{center}
% \begin{tabular}{ll}
% |cdocsamp.tex|&main file\\
% |cdocsch1.tex|&include file for chapter 1\\
% |cdocsch2.tex|&include file for chapter 2\\
% |cdocspt3.tex|&include file for part 3\\
% |cdocspt4.tex|&include file for part 4\\
% |cdocsdrf.tex|&forwarding file for main file in draft mode\\
% |cdocsfi1.tex|&forwarding file for final version of chapter 1\\
% |cdocsfi2.tex|&forwarding file for final version of chapter 2\\
% \end{tabular}
% \end{center}
% Each of the eight files can be compiled directly by the \LaTeX{} compiler.
%
% %%%%%%%%%%%%%%%%%%%%%%%%%%%%%%%%%%%%%%
% \paragraph{Main File.}
%
% The main file is called |cdocsamp.tex|.
%
% Load the \textsf{childdoc} definitions and
% declare the filename for the main document:
%    \begin{macrocode}
\input{childdoc.def}
\childdocmain{}
%    \end{macrocode}

% Optional override for |\version| flag:
%    \begin{macrocode}
%%\ifchilddoc\else\providecommand{\version}{draft}\fi
%    \end{macrocode}

% Define the default values for the |\version| flag
% (|final| for the main file and |draft| for childs):
%    \begin{macrocode}
\ifchilddoc
\providecommand{\version}{draft}
\else
\providecommand{\version}{final}
\fi
%    \end{macrocode}

% Load the standard document class:
%    \begin{macrocode}
\documentclass[12pt]{article}
%    \end{macrocode}

% Start the document body:
%    \begin{macrocode}
\begin{document}
%    \end{macrocode}

% Declare a title page.
% Print title, part of document being processed and version flag:
%    \begin{macrocode}
\addtocounter{page}{-1}
\begin{center}
{\LARGE\bfseries{}childdoc example\par}
\vspace{1cm}
\ifchilddoc
\ifchilddocmanual part\else chapter\fi:
`\childdocname' of `\childdocjob'\par
\else
main document: `\childdocjob'\par
\fi
version: \version\par
\end{center}
\newpage
%    \end{macrocode}

% Manually include selected file,
% otherwise process as usual:
%    \begin{macrocode}
\ifchilddocmanual
\section*{part `\childdocname'}
\input{\childdocname}
\else
%    \end{macrocode}

% Include the two chapters:
%    \begin{macrocode}
\include{cdocsch1}
\include{cdocsch2}
%    \end{macrocode}

% Include the two parts unless only chapters should be displayed:
%    \begin{macrocode}
\ifchilddoc\else
\section{part three}
\input{cdocspt3}
\section{part four}
\input{cdocspt4}
\fi
%    \end{macrocode}

% Process as usual until here:
%    \begin{macrocode}
\fi
%    \end{macrocode}

% End of document body:
%    \begin{macrocode}
\end{document}
%    \end{macrocode}
%\iffalse
%</samplemain>
%\fi
%
% %%%%%%%%%%%%%%%%%%%%%%%%%%%%%%%%%%%%%%
% \paragraph{Chapter Include Files.}
%
% The include files are called |cdocsch1.tex| and |cdocsch2.tex|.
%
%\iffalse
%<*samplechap1|samplechap2>
%\fi

% Optional override for |\version| flag:
%    \begin{macrocode}
%%\providecommand{\version}{final}
%    \end{macrocode}

% Include the main document:
%    \begin{macrocode}
\input{childdoc.def}
\childdocof{cdocsamp}
%    \end{macrocode}

%\iffalse
%</samplechap1|samplechap2>
%\fi
%
%\iffalse
%<*samplechap1>
%\fi
% Some text for chapter 1:
%    \begin{macrocode}
\section{one}
some text in chapter one
%    \end{macrocode}

%\iffalse
%</samplechap1>
%\fi
% Some text for chapter 2:
%\iffalse
%<*samplechap2>
%\fi
%    \begin{macrocode}
\section{two}
more text in chapter two
%    \end{macrocode}

%\iffalse
%</samplechap2>
%\fi
%
% %%%%%%%%%%%%%%%%%%%%%%%%%%%%%%%%%%%%%%
% \paragraph{Part Include Files.}
%
% The include files are called |cdocspt3.tex| and |cdocspt4.tex|.
%
%\iffalse
%<*samplepart3|samplepart4>
%\fi

% Optional override for |\version| flag:
%    \begin{macrocode}
%%\providecommand{\version}{final}
%    \end{macrocode}

% Include the main document:
%    \begin{macrocode}
\input{childdoc.def}
\childdocby{cdocsamp}
%    \end{macrocode}

%\iffalse
%</samplepart3|samplepart4>
%\fi
%
%\iffalse
%<*samplepart3>
%\fi
% Some text for part 3:
%    \begin{macrocode}
some text in part three
%    \end{macrocode}

%\iffalse
%</samplepart3>
%\fi
% Some text for part 4:
%\iffalse
%<*samplepart4>
%\fi
%    \begin{macrocode}
more text in part four
%    \end{macrocode}

%\iffalse
%</samplepart4>
%\fi
%
% %%%%%%%%%%%%%%%%%%%%%%%%%%%%%%%%%%%%%%
% \paragraph{Forwarding for a Complete Draft.}
%
% The following forwarding file |cdocsdrf.tex|
% compiles the main document in draft mode:
%\iffalse
%<*sampledraft>
%\fi
%    \begin{macrocode}
\def\version{draft}
\input{childdoc.def}
\childdocforward{cdocsamp}
%    \end{macrocode}

%\iffalse
%</sampledraft>
%\fi
%
% %%%%%%%%%%%%%%%%%%%%%%%%%%%%%%%%%%%%%%
% \paragraph{Forwarding for Final Version of the Chapters.}
%
% The following forwarding files |cdocsfn1.tex| and |cdocsfn2.tex|
% (with identical content)
% compile the final versions of the child documents
% |cdocsch1.tex| and |cdocsch2.tex|, respectively:
%\iffalse
%<*samplefinal>
%\fi
%    \begin{macrocode}
\def\version{final}
\input{childdoc.def}
\childdocforwardprefix[cdocsamp]{cdocsfn}{cdocsch}
%    \end{macrocode}

%\iffalse
%</samplefinal>
%\fi
%
% %%%%%%%%%%%%%%%%%%%%%%%%%%%%%%%%%%%%%%
% \paragraph{Command Line Processing.}
%
% The following three command lines generate the output files
% |cdocscld|, |cdocscl1| and |cdocscl2|
% which should be identical to
% |cdocsdrf|, |cdocsch1| and |cdocsfn2|, respectively:
% \begin{center}
% \begin{tabular}{l}
% |latex -jobname cdocscld \|\\
% |  "\def\version{draft}\input{childdoc.def}\childdocforward{cdocsamp}"|\\
% |latex -jobname cdocscl1 \|\\
% |  "\input{childdoc.def}\childdocforward[cdocsamp]{cdocsch1}"|\\
% |latex -jobname cdocscl2 \|\\
% |  "\def\version{final}\input{childdoc.def}\childdocforward{cdocsch2}"|
% \end{tabular}
% \end{center}
% Note that the trailing backslash on each first line
% merely continues the input to the second line
% (for convenient cut ant paste).
% Furthermore, the command |latex| can be replaced by any
% of its alternative versions such as |pdflatex|.
%
% %%%%%%%%%%%%%%%%%%%%%%%%%%%%%%%%%%%%%%%%%%%%%%%%%%%%%%%%%%%%%%%%%%%%%%%%%%%%%%
% %%%%%%%%%%%%%%%%%%%%%%%%%%%%%%%%%%%%%%%%%%%%%%%%%%%%%%%%%%%%%%%%%%%%%%%%%%%%%%
% \section{Implementation}
%\iffalse
%<*package>
%\fi
%
% This section describes the definitions file |childdoc.def|.

% The definitions cannot be loaded using |\usepackage| or |\RequirePackage|
% which has a mechanism to prevent loading a style file more than once.
% When loading the definitions by means of |\input|
% multiple instances have to be prevented manually:
%\iffalse
%This code needs to be before the `\ProvidesFile' directive
%which is defined at the beginning of this file.
%Therefore it is also placed there and commented out here.
%</package>
%<*discard>
%\fi
%    \begin{macrocode}
\ifdefined\childdocmain\endinput\fi
%    \end{macrocode}
%\iffalse
%</discard>
%<*package>
%\fi
%
% \macro{\ifchilddoc}
% \macro{\ifchilddocmanual}
% The conditional |\ifchilddoc| tells whether a
% child (true) or main (false) document is being compiled.
% The conditional |\ifchilddocmanual| tells whether
% the |\includeonly| mechanism is used (false) or
% the selection of child files must be performed manually (true).
% The definitions initialise to false:
%    \begin{macrocode}
\newif\ifchilddoc
\newif\ifchilddocmanual
%    \end{macrocode}

% \macro{\childdocname}
% \macro{\childdocjob}
% The macro |\childdocname| stores the name of the main document
% to be compiled. The macro |\childdocjob| stores the name of
% the document on which the \LaTeX{} compiler was originally invoked.
% The content of |\jobname| cannot be compared
% to filenames specified in the source due to different catcodes.
% The following code rescans |\jobname|, stores the result
% in |\childdocname| and saves a copy in |\childdocjob|:
%    \begin{macrocode}
\edef\childdocname{\scantokens\expandafter{\jobname\noexpand}}
\let\childdocjob\childdocname
%    \end{macrocode}

% \macro{\childdocdisable}
% The macro |\childdocdisable| prevents the main file
% from being processed more than once.
% At this stage, the main document command |\childdocmain|
% is assumed to be called once again where it should do nothing.
% Any subsequent call to it should prevent
% a secondary processing of the main document
% It overwrites the forwarding commands
% |\childdocof| and |\childdocforward|
% with empty macros to prevent further inclusions of the main document:
%    \begin{macrocode}
\newcommand{\childdocdisable}
{
  \renewcommand{\childdocmain}[1]{\renewcommand{\childdocmain}[1]{\endinput}}
  \renewcommand{\childdocof}[1]{}
  \renewcommand{\childdocby}[2][]{}
  \renewcommand{\childdocforward}[2][]{}
  \renewcommand{\childdocdisable}{}
}
%    \end{macrocode}

% \macro{\childdocmain}
% The macro |\childdocmain| is to be called at the top of the main file
% with nothing or the main filename (without extension) as argument.
% First, it breaks loops.
% If the argument is not empty and does not match |\childdocname|
% (which is set by the first inclusion of |childdoc.def|),
% |\ifchilddoc| is set to true, |\includeonly| is applied to the child file
% and |\jobname| is set to the main file
% (for proper handling of |.aux| files):
%    \begin{macrocode}
\newcommand{\childdocmain}[1]
{
  \childdocdisable\childdocmain{}
  \if?#1?\else
    \begingroup
      \def\childdoctmp{#1}
      \ifx\childdoctmp\childdocname
        \def\childdoctmp{}
      \else
        \def\childdoctmp
        {
          \childdoctrue
          \includeonly{\childdocname}
          \def\childdocjob{#1}
          \def\jobname{#1}
        }
      \fi
      \expandafter
    \endgroup
    \childdoctmp
  \fi
}
%    \end{macrocode}

% \macro{\childdocof}
% The command |\childdocof| redirects
% compilation to the main file |#1|.
%    \begin{macrocode}
\newcommand{\childdocof}[1]
{
  \childdocdisable
  \childdoctrue
  \includeonly{\childdocname}
  \def\jobname{#1}
  \def\childdocjob{#1}
  \input{#1}
}
%    \end{macrocode}

% \macro{\childdocby}
% The command |\childdocby| ....
%    \begin{macrocode}
\newcommand{\childdocby}[2][]
{
  \childdocdisable
  \childdoctrue
  \childdocmanualtrue
  \if?#1?\else
    \def\jobname{#2}
  \fi
  \def\childdocjob{#2}
  \input{#2}
  \endinput
}
%    \end{macrocode}

% \macro{\childdocforward}
% The command |\childdocforward| redirects
% compilation to the main file or
% (if the optional argument is given) a child file.
% Parameters are set as if the main file
% or a child file starting with |\childdocof| was compiled.
% Then compilation is handed over to the main file:
%    \begin{macrocode}
\newcommand{\childdocforward}[2][]
{
  \begingroup
    \if?#1?
      \def\childdoctmp
      {
        \def\childdocname{#2}
        \def\childdocjob{#2}
        \def\jobname{#2}
        \input{#2}
        \endinput
      }
    \else
      \def\childdoctmp
      {
        \childdocdisable
        \def\childdocname{#2}
        \childdoctrue
        \includeonly{#2}
        \def\childdocjob{#1}
        \def\jobname{#1}
        \input{#1}
        \endinput
      }
    \fi
    \expandafter
  \endgroup
  \childdoctmp
}
%    \end{macrocode}

% \macro{\childdocforwardprefix}
% The command |\childdocforwardprefix| redirects
% compilation to the main or a child file by means of a pattern.
% The prefix |#1| in the current filename is replaced by |#2|
% and the suffix of the current filename is kept
% (it is assumed that the filename does not contain the substring `|~~~|'
% which is used as a delimiter).
% Compilation is handed over to the new file by |\childdocforward|:
%    \begin{macrocode}
\newcommand{\childdocforwardprefix}[3][]
{
  \begingroup
    \def\childdocextract #2##1~~~{\def\childdoctmp{\childdocforward[#1]{#3##1}}}
    \expandafter\childdocextract\childdocname~~~
    \expandafter
  \endgroup
  \childdoctmp
}
%    \end{macrocode}

% \macro{\childdoc}
% The deprecated macro |\childdoc| is a legacy version of |\childdocmain|:
%    \begin{macrocode}
\newcommand{\childdoc}{\childdocmain}
%    \end{macrocode}

% \macro{\childdocredirect}
% The deprecated macro |\childdocredirect| is a legacy version
% of |\childdocforward| and |\childdocforwardprefix|:
%    \begin{macrocode}
\newcommand{\childdocredirect}[2][]
{
  \begingroup
    \if?#1?
      \def\childdoctmp{\childdocforward{#2}}
    \else
      \def\childdoctmp{\childdocforwardprefix{#1}{#2}}
    \fi
    \expandafter
  \endgroup
  \childdoctmp
}
%    \end{macrocode}

%\iffalse
%</package>
%\fi
%
\endinput
|\\
|\childdocmain{|\textit{main}|}|\\
\end{tabular}
\end{center}
%
If |\jobname| does not match the argument \textit{main} of |\childdocmain|,
it is assumed that |\jobname| points to the child file to be compiled.
When using |\childdocmain| with the main file specified as argument,
it suffices to start a child file
with just |\input{|\textit{main}|}|
without loading of the package and using |\childdocof|.
If instead all processing is done
with the appropriate \textsf{childdoc} directives,
the argument of \textit{main} of |\childdocmain| can be empty.

An alternative version of the command line processing described
in \secref{sec:commandline} using the detection mechanism reads:
%
\begin{center}
|... -jobname "|\textit{target}|" "|[\textit{flags}]%
[|\def\jobname{|\textit{dest}|}|]|\input{|\textit{main}|}"|
\end{center}

%%%%%%%%%%%%%%%%%%%%%%%%%%%%%%%%%%%%%%%%%%%%%%%%%%%%%%%%%%%%%%%%%%%%%%%%%%%%%%%%
\subsection{Manual Code}
\label{sec:manual}

In case one cannot be certain whether the definitions file |childdoc.def|
is installed on the target \TeX{} distribution
and one prefers not to ship it,
it is conceivable to paste a few relevant commands into the sources.

To that end, drop all statements |% \iffalse
%
% childdoc.dtx Copyright (C) 2017-2018 Niklas Beisert
%
% This work may be distributed and/or modified under the
% conditions of the LaTeX Project Public License, either version 1.3
% of this license or (at your option) any later version.
% The latest version of this license is in
%   http://www.latex-project.org/lppl.txt
% and version 1.3 or later is part of all distributions of LaTeX
% version 2005/12/01 or later.
%
% This work has the LPPL maintenance status `maintained'.
%
% The Current Maintainer of this work is Niklas Beisert.
%
% This work consists of the files childdoc.dtx and childdoc.ins
% and the derived files childdoc.def and cdocsamp.tex with
% cdocsch1.tex, cdocsch2.tex, cdocsdrf.tex, cdocsfn1.tex, cdocsfn2.tex.
%
%<package>\ifdefined\childdocmain\endinput\fi
%<package>\ProvidesFile{childdoc.def}[2018/12/30 v2.0 child document driver]
%<samplemain>\ProvidesFile{cdocsamp.tex}[2018/12/30 v2.0 sample for childdoc]
%<*driver>
%\ProvidesFile{childdoc.drv}[2018/12/30 v2.0 childdoc reference manual file]
\PassOptionsToClass{10pt,a4paper}{article}
\documentclass{ltxdoc}

\usepackage[margin=35mm]{geometry}
\usepackage{hyperref}
\usepackage{hyperxmp}
\usepackage[usenames]{color}

\hypersetup{colorlinks=true}
\hypersetup{pdfstartview=FitH}
\hypersetup{pdfpagemode=UseNone}
\hypersetup{pdfsource={}}
\hypersetup{pdflang={en-UK}}
\hypersetup{pdfcopyright={Copyright 2017-2018 Niklas Beisert.
  This work may be distributed and/or modified under the
  conditions of the LaTeX Project Public License, either version 1.3
  of this license or (at your option) any later version.}}
\hypersetup{pdflicenseurl={http://www.latex-project.org/lppl.txt}}
\hypersetup{pdfcontactaddress={ETH Zurich, ITP, HIT K,
  Wolfgang-Pauli-Strasse 27}}
\hypersetup{pdfcontactpostcode={8093}}
\hypersetup{pdfcontactcity={Zurich}}
\hypersetup{pdfcontactcountry={Switzerland}}
\hypersetup{pdfcontactemail={nbeisert@itp.phys.ethz.ch}}
\hypersetup{pdfcontacturl={http://people.phys.ethz.ch/\xmptilde nbeisert/}}

\newcommand{\secref}[1]{\hyperref[#1]{section \ref*{#1}}}

\parskip1ex
\parindent0pt
\let\olditemize\itemize
\def\itemize{\olditemize\parskip0pt}

\begin{document}

\title{The \textsf{childdoc} Package}
\hypersetup{pdftitle={The childdoc Package}}
\author{Niklas Beisert\\[2ex]
  Institut f\"ur Theoretische Physik\\
  Eidgen\"ossische Technische Hochschule Z\"urich\\
  Wolfgang-Pauli-Strasse 27, 8093 Z\"urich, Switzerland\\[1ex]
  \href{mailto:nbeisert@itp.phys.ethz.ch}
  {\texttt{nbeisert@itp.phys.ethz.ch}}}
\hypersetup{pdfauthor={Niklas Beisert}}
\hypersetup{pdfsubject={Manual for the LaTeX2e Package childdoc}}
\date{30 December 2018, \textsf{v2.0}}
\maketitle

\begin{abstract}\noindent
\textsf{childdoc} is a \LaTeXe{} package
that enables the direct compilation
of document sections included by |\include|
to individual files.
\end{abstract}

\begingroup
\parskip0ex
\tableofcontents
\endgroup

%%%%%%%%%%%%%%%%%%%%%%%%%%%%%%%%%%%%%%%%%%%%%%%%%%%%%%%%%%%%%%%%%%%%%%%%%%%%%%%%
%%%%%%%%%%%%%%%%%%%%%%%%%%%%%%%%%%%%%%%%%%%%%%%%%%%%%%%%%%%%%%%%%%%%%%%%%%%%%%%%
\section{Introduction}

\LaTeX{} provides a mechanism to structure a large document (such as a book)
into a main file and several child files (containing the chapters)
using the |\include| command.
This mechanism is beneficial for documents
which span hundreds of pages in order to
make the source file(s) more manageable.
Moreover, compilation can be restricted to
selected child files by means of the |\includeonly| command.
The latter feature can be used to reduce the compilation time while editing
(this was significantly more useful in the earlier days of \LaTeX{})
or to generate a smaller document which is easier to navigate.
Another application of |\includeonly| is to generate
documents consisting of selected parts of the complete document.

However, there are a few drawbacks of the plain |\include| mechanism:
\begin{itemize}
\item
The child files cannot be compiled on their own,
they can only be compiled via the main file.
A naive editing environment
(such as a text editor with an option
to have the current file processed by \LaTeX)
may require one to switch to the main file before compiling;
attempting to compile the child file produces errors.
\item
The main file must be modified (each time)
to adjust the |\includeonly| command
to the present needs. This easily leaves the main file in a messy state.
\item
The generated document will always carry the filename
of the main document. This is inconvenient if
several child files are to be compiled and
to be kept for distribution.
\end{itemize}

The present package provides a simple interface
to make child files individually compilable by \LaTeX{}.
Compiling a child file then has the same effect as compiling
the main file with an |\includeonly| command
to select the appropriate child.
Moreover the generated document will carry the name of the child
rather than the main file.
This resolves all three above issues.

This feature is meant to make the editing of books,
thesis documents and lecture notes somewhat more convenient.
However, the package can also be used efficiently for
composing a series of documents (such as exercise sheets)
which are typically distributed individually.
It then assists the author in generating the individual documents
(potentially in different versions)
as well as a document containing the collected series.
Another application is in developing style files
or other kinds of included material
where compilation of the style file could redirect
to a sample or test file.

%%%%%%%%%%%%%%%%%%%%%%%%%%%%%%%%%%%%%%%%%%%%%%%%%%%%%%%%%%%%%%%%%%%%%%%%%%%%%%%%
%%%%%%%%%%%%%%%%%%%%%%%%%%%%%%%%%%%%%%%%%%%%%%%%%%%%%%%%%%%%%%%%%%%%%%%%%%%%%%%%
\section{Usage}

First of all, the package \textsf{childdoc} is \emph{not} a standard
\LaTeXe{} |.sty| style file! Therefore it needs to be invoked in
a non-standard way.

%%%%%%%%%%%%%%%%%%%%%%%%%%%%%%%%%%%%%%%%%%%%%%%%%%%%%%%%%%%%%%%%%%%%%%%%%%%%%%%%
\subsection{Included Files}
\label{sec:include}

%%%%%%%%%%%%%%%%%%%%%%%%%%%%%%%%%%%%%%%%
\DescribeMacro{\childdocmain}
To use the package, add the commands
\begin{center}
\begin{tabular}{l}
|\input{childdoc.def}|\\
|\childdocmain{}|\\
\end{tabular}
\end{center}
at the very top of the main \LaTeX{} file,
in particular \emph{before} the |\documentclass| statement!
The argument of |\childdocmain| should be left empty
(but it must be present).

%%%%%%%%%%%%%%%%%%%%%%%%%%%%%%%%%%%%%%%%
\DescribeMacro{\childdocof}
Furthermore, add the commands
\begin{center}
\begin{tabular}{l}
|\input{childdoc.def}|\\
|\childdocof{|\textit{main}|}|\\
\end{tabular}
\end{center}
at the top of every child file \textit{child}
which is included by |\include{|\textit{child}|}|
from within the main file
(or at least for those files to be compiled individually).
The argument \textit{main} must be the filename of the main file.

There are a couple of
considerations in setting up the main and child documents:

%%%%%%%%%%%%%%%%%%%%%%%%%%%%%%%%%%%%%%%%
\paragraph{Restrictions.}

Please note the following restrictions:
\begin{itemize}
\item
|\childdocmain| must be called with one argument \textit{main}
to ensure compatibility with earlier version of the package.
It must either be empty (|\childdocmain{}|)
or precisely match the filename of the main file in which it is specified.
See \secref{sec:detection} for further information.
\item
The filename \textit{main} must be specified without the |.tex| extension.
\item
The filename \textit{main} is case sensitive
(even in case-insensitive file systems)
due to internal string comparison.
\item
The argument \textit{main} should be fully expanded, it cannot be a macro.
\item
Subdirectories and special characters should be avoided in filenames.
\item
The command |\childdocmain{|\textit{main}|}| must be followed by a whitespace.
It should not be followed immediately by another command
or by a comment mark `|%|'.
This is because the \TeX{} parser reads the token immediately following
the argument of |\childdocmain| and puts it
at the beginning of every child section;
however, a white\-space is ignored.
\end{itemize}

%%%%%%%%%%%%%%%%%%%%%%%%%%%%%%%%%%%%%%%%
\paragraph{Content of Main File.}

It is advisable to place all content in the child files included by |\include|.
Any output contained in the main file will appear in all child documents
unless suppressed manually;
it cannot be suppressed automatically by the |\includeonly| directive
and thus should normally be avoided.
A method to include some content in the main file
by means of conditional processing is described in \secref{sec:conditional}.

%%%%%%%%%%%%%%%%%%%%%%%%%%%%%%%%%%%%%%%%
\paragraph{Page Numbering.}

When only a part of the document is compiled,
the appropriate numbering of pages
(as well as other status parameters)
is determined from the |.aux| files.
The latter contain information from previous passes.
However this information needs to propagate through
all intermediate child documents.
Therefore the page numbering in child documents may well
be inconsistent until the complete document is compiled at least once.

A useful (if unconventional) way to always ensure a consistent
page numbering is to restart the numbering in each child document
and denote the pages by `\textit{child}|.|\textit{page}'
where \textit{child} represents the chapter/section number of the child file.
This can be achieved by the command
|\numberwithin{page}{|\textit{child}|}|
of the \textsf{amsmath} package
where \textit{child} can be |chapter| or |section|
depending on the chosen structuring.
Alternatively, one can modify the macro |\thepage| appropriately
and reset the counter |page| at the start of each child file.

%%%%%%%%%%%%%%%%%%%%%%%%%%%%%%%%%%%%%%%%%%%%%%%%%%%%%%%%%%%%%%%%%%%%%%%%%%%%%%%%
\subsection{Conditional Processing}
\label{sec:conditional}

The package provides a mechanism to compile different versions
of a document. To customise the versions further some conditional processing
can come in handy to distinguish which version is being compiled.
The package provides two macros to describe the compilation context:

%%%%%%%%%%%%%%%%%%%%%%%%%%%%%%%%%%%%%%%%
\DescribeMacro{\ifchilddoc}
The conditional |\ifchilddoc| distinguishes between the compilation of
child documents and the main document:
%
\begin{center}
|\ifchilddoc |\textit{child-code}| |[|\||else |\textit{main-code}]| \||fi|
\end{center}

%%%%%%%%%%%%%%%%%%%%%%%%%%%%%%%%%%%%%%%%
\DescribeMacro{\childdocname}
\DescribeMacro{\childdocjob}
The macro |\childdocname| contains the filename (without extension)
of the main or child file being processed.
Note that |\childdocjob| will always contain the name of the main file.

%%%%%%%%%%%%%%%%%%%%%%%%%%%%%%%%%%%%%%%%
\paragraph{Title Page.}

Conditional processing can be used to include a title or banner page
in the main document when proper precautions are taken.
Importantly, the code in the main file should ensure that the page counter
(as well as other status parameters which are stored in the |.aux| files)
takes the same value after the conditional processing.
Otherwise the page numbers may take divergent values
depending on which part is compiled.

For example, a title page could be declared by:
%
\begin{center}
\begin{tabular}{l}
|\ifchilddoc\||else|\\
|\addtocounter{page}{-1}|\\
\textit{code for title page}\\
|\newpage|\\
|\||fi|
\end{tabular}
\end{center}
%
A banner page for the child documents can be generated by:
%
\begin{center}
\begin{tabular}{l}
|\ifchilddoc|\\
|\addtocounter{page}{-1}|\\
\textit{code for banner page}\\
|\newpage|\\
|\||fi|
\end{tabular}
\end{center}
%
Here one could write a message such as:
\begin{center}
|This is the part \childdocname{} of \childdocjob{}.|
\end{center}

%%%%%%%%%%%%%%%%%%%%%%%%%%%%%%%%%%%%%%%%%%%%%%%%%%%%%%%%%%%%%%%%%%%%%%%%%%%%%%%%
\subsection{Flags}
\label{sec:flags}

The package makes it easy to generate different versions
of the main or child documents.
To this end compilation flags can be defined
and assigned different default values.
They will be particularly useful in conjunction
with the forwarding mechanism described in \secref{sec:forward}.

For example, it may be useful to have a flag |\version|
which can be set to |draft| or |final|.
The document source will contain some conditional code
depending on the value of |\version|.
Suppose further, the flag should default to |final| for the main file
and to |draft| for child files
which is a natural assignment for editing the document.
This is achieved by placing the following code
in the preamble of the main document
(below the |\childdocmain| directive):
%
\begin{center}
\begin{tabular}{l}
|\ifchilddoc|\\
|\providecommand{\version}{draft}|\\
|\||else|\\
|\providecommand{\version}{final}|\\
|\||fi|
\end{tabular}
\end{center}
%
The definition by |\providecommand| makes sure
that previous definitions are not overwritten.
Further statements |\providecommand{\version}{...}|
can thus be added before the above code to override it.

For the main file, one might add a line
(between |\childdocmain| and the above block)
%
\begin{center}
|%\ifchilddoc\||else\providecommand{\version}{draft}\||fi|
\end{center}
%
which can be uncommented to produce a draft version.
Likewise one can add a line to the very top of a child file
(above the |\childdocof{|\textit{main}|}| directive)
%
\begin{center}
|%\providecommand{\version}{final}|
\end{center}
%
which can be uncommented to produce the final version of this child document.

%%%%%%%%%%%%%%%%%%%%%%%%%%%%%%%%%%%%%%%%%%%%%%%%%%%%%%%%%%%%%%%%%%%%%%%%%%%%%%%%
\subsection{Forwarding}
\label{sec:forward}

Different versions of the main or child documents
using compilation flags as described in \secref{sec:flags}
can be (permanently) stored in different files
for convenient compilation, viewing and distribution.
To this end, the package defines a command
to pass on compilation to a different file:

%%%%%%%%%%%%%%%%%%%%%%%%%%%%%%%%%%%%%%%%
\DescribeMacro{\childdocforward}
The command |\childdocforward| redirects processing to
another source file:
%
\begin{center}
\begin{tabular}{l}
|\input{childdoc.def}|\\
|\childdocforward[|\textit{main}|]{|\textit{dest}|}|\\
\end{tabular}
\end{center}
%
The argument \textit{dest} is the destination file
(without extension).
It should be the main file or one of the child files.
Note that further \textsf{childdoc} directives
such as |\childdocof| and |\childdocforward|
in the indicated file will be processed in this form.
The optional argument \textit{main}
passes on directly to the main file \textit{main}
while pretending to compile the child \textit{dest}.
This form behaves as if \textit{dest}
issues |\childdocof{|\textit{main}|}| right away,
and no further \textsf{childdoc} directives will be processed.

%%%%%%%%%%%%%%%%%%%%%%%%%%%%%%%%%%%%%%%%
\DescribeMacro{\...prefix}
In the alternative form |\childdocforwardprefix|,
%
\begin{center}
\begin{tabular}{l}
|\input{childdoc.def}|\\
|\childdocforwardprefix[|\textit{main}|]{|\textit{prefix}|}{|\textit{dest}|}|
\end{tabular}
\end{center}
%
the destination file is determined by a pattern
depending on the current file:
To make this work, the current file must be called
`{\textit{prefix}\hspace{0.2em}\textit{suffix}}'
with \textit{prefix} matching precisely the argument.
Processing is then passed on to the file
`{\textit{dest}\hspace{0.2em}\textit{suffix}}'.
Surely, the same effect is achieved by
directly specifying the
argument `{\textit{dest}\hspace{0.2em}\textit{suffix}}'
in the first form.
However, that requires to set up a different file
for each child. With the alternative form of the command
all these files can have exactly the same content
which simplifies setting them up and maintaining them.

For example, the following file |draft.tex|
with a compilation flag |\version| as described in \secref{sec:flags}
compiles the main document as a draft:
%
\begin{center}
\begin{tabular}{l}
|\def\version{draft}|\\
|\input{childdoc.def}|\\
|\childdocforward{|\textit{main}|}|
\end{tabular}
\end{center}
%
Likewise, the following files |final|\textit{nn}|.tex|
compile the final version of the child document
|child|\textit{nn}|.tex|:
%
\begin{center}
\begin{tabular}{l}
|\def\version{final}|\\
|\input{childdoc.def}|\\
|\childdocforwardprefix{final}{child}|
\end{tabular}
\end{center}
%

Note that when several versions of a main file and/or of each child file
are to be generated, it may be convenient to set up a |Makefile| or
shell script to automatise the process.

%%%%%%%%%%%%%%%%%%%%%%%%%%%%%%%%%%%%%%%%%%%%%%%%%%%%%%%%%%%%%%%%%%%%%%%%%%%%%%%%
\subsection{Command Line Processing}
\label{sec:commandline}

The effect of redirection files can also be achieved by invoking
the \LaTeX{} compiler with a more elaborate command line.
Most conveniently this should be done as part
of a shell script or a |Makefile|.

When using \textsf{childdoc} in the main file, the following
command lines effectively perform a redirection
(note that depending on the shell being used,
backslashes may have to be doubled: `|\|' $\to$ `|\\|'):
%
\begin{center}
|... -jobname "|\textit{target}|" |\\|"|[\textit{flags}]%
|\input{childdoc.def}\childdocforward[|\textit{main}|]{|\textit{dest}|}"|
\end{center}
%
Here \textit{target} is the name of the output file,
\textit{main} is the name of the main file
and \textit{dest} is the name of the main or child file to be processed
(all filenames without extensions).
The optional argument \textit{main} can be omitted
if \textit{main} matches \textit{dest}.
Optionally, compilation \textit{flags} can be defined via |\def| commands.
This command line makes the \TeX{} engine believe
it is compiling the file \textit{target}
whose content is specified as the latter parameter.
The provided code then forwards the processing to
\textit{main} or \textit{dest} as described in \secref{sec:forward}.

%%%%%%%%%%%%%%%%%%%%%%%%%%%%%%%%%%%%%%%%%%%%%%%%%%%%%%%%%%%%%%%%%%%%%%%%%%%%%%%%
\subsection{Include by Input}
\label{sec:input}

Including child documents by |\include| has some restrictions by design.
Most notably, the content of a child document always occupies
its own set of pages; pages cannot be shared between child documents.
Usually, this behaviour makes perfect sense
because each child document contain an essential part of the document.
However, in some situations it may be desirable to compose
a document from a collection of parts
without having mandatory page breaks between then.
For this case, the package
provides a mechanism to include parts
by |\input| which can also be processed individually.
However, by construction this mechanism
requires manual handling of the content to be output.

%%%%%%%%%%%%%%%%%%%%%%%%%%%%%%%%%%%%%%%%
\DescribeMacro{\ifchilddocmanual}
The main file should be prepared as usual, see \secref{sec:include}.
However, the document body must make a distinction
between processing of an individual part and of the main document, e.g.:
%
\begin{center}
\begin{tabular}{l}
|\ifchilddocmanual|\\
|\input{\childdocname}|\\
|\||else|\\
\textit{document body with }|\input{|\textit{part}|}|\\
|\||fi|
\end{tabular}
\end{center}
%
The conditional |\ifchilddocmanual| is true whenever
a part to be included by |\input| is being compiled,
and the name of the part is stored in |\childdocname|.

%%%%%%%%%%%%%%%%%%%%%%%%%%%%%%%%%%%%%%%%
\DescribeMacro{\childdocby}
Each part to be included by |\input| should start with:
%
\begin{center}
\begin{tabular}{l}
|\input{childdoc.def}|\\
|\childdocby{|\textit{main}|}|\\
\end{tabular}
\end{center}
%
The directive |\childdocby| is similar to |\childdocof|
described in \secref{sec:include},
but the subsequent selection of content must be done manually.
To that end, both |\ifchilddoc| and |\ifchilddocmanual|
will be true upon processing of a part,
and the name of the part is stored in |\childdocname|.
Note that |\jobname| will be set to the filename of the current part
so that each part receives an individual |.aux| file
that does not interfere with the |.aux| file(s) of the main document.
This behaviour can be altered by the alternative form
|\childdocby[*]{|\textit{main}|}| (with a non-empty optional argument)
which uses the |.aux| file of the main document
by setting |\jobname| to \textit{main}.

%%%%%%%%%%%%%%%%%%%%%%%%%%%%%%%%%%%%%%%%%%%%%%%%%%%%%%%%%%%%%%%%%%%%%%%%%%%%%%%%
\subsection{Driver Development}
\label{sec:driver}

The \textsf{childdoc} mechanism can also be use for the development
of definition files such as \LaTeX{} styles or classes.
This case differs from the above setup with multiple parts
included by |\include| in that no |\includeonly| should be invoked.
This can be achieved by starting the include file
(before |\ProvidesPackage|) with:
%
\begin{center}
\begin{tabular}{l}
|\input{childdoc.def}|\\
|\childdocforward{|\textit{main}|}|\\
\end{tabular}
\end{center}
%
or alternatively with:
%
\begin{center}
\begin{tabular}{l}
|\input{childdoc.def}|\\
|\childdocby{|\textit{main}|}|\\
\end{tabular}
\end{center}
%
Both forms have slightly different effects as described above.
The main file is prepared as usual, see \secref{sec:include}.

%%%%%%%%%%%%%%%%%%%%%%%%%%%%%%%%%%%%%%%%%%%%%%%%%%%%%%%%%%%%%%%%%%%%%%%%%%%%%%%%
\subsection{Legacy Detection}
\label{sec:detection}

The directive |\childdocmain| in the main file can detect
whether the complete document or merely a child is to be compiled
even without using the directive |\childdocof|.
This method is deprecated because it is less robust
and there is no compelling reason to use it;
it is merely provided for backward compatibility
and it may be removed in future versions.

If the detection mechanism is to be used,
it is mandatory to correctly specify
the filename of the main file as the argument of |\childdocmain|:
%
\begin{center}
\begin{tabular}{l}
|\input{childdoc.def}|\\
|\childdocmain{|\textit{main}|}|\\
\end{tabular}
\end{center}
%
If |\jobname| does not match the argument \textit{main} of |\childdocmain|,
it is assumed that |\jobname| points to the child file to be compiled.
When using |\childdocmain| with the main file specified as argument,
it suffices to start a child file
with just |\input{|\textit{main}|}|
without loading of the package and using |\childdocof|.
If instead all processing is done
with the appropriate \textsf{childdoc} directives,
the argument of \textit{main} of |\childdocmain| can be empty.

An alternative version of the command line processing described
in \secref{sec:commandline} using the detection mechanism reads:
%
\begin{center}
|... -jobname "|\textit{target}|" "|[\textit{flags}]%
[|\def\jobname{|\textit{dest}|}|]|\input{|\textit{main}|}"|
\end{center}

%%%%%%%%%%%%%%%%%%%%%%%%%%%%%%%%%%%%%%%%%%%%%%%%%%%%%%%%%%%%%%%%%%%%%%%%%%%%%%%%
\subsection{Manual Code}
\label{sec:manual}

In case one cannot be certain whether the definitions file |childdoc.def|
is installed on the target \TeX{} distribution
and one prefers not to ship it,
it is conceivable to paste a few relevant commands into the sources.

To that end, drop all statements |\input{childdoc.def}|
and perform the replacements as outlined below.
Instead of |\childdocmain{|\textit{main}|}| add the following code
to the top of the main file:
%
\begin{center}
\begin{tabular}{l}
|\||ifdefined\childdocname\endinput\||fi\newif\ifchilddoc|\\
|\edef\childdocname{\scantokens\expandafter{\jobname\noexpand}}|\\
|\def\childdocmain{|\textit{main}|}\||ifx\childdocmain\childdocname\||else|\\
|\childdoctrue\includeonly{\childdocname}\let\jobname\childdocmain\||fi|\\
\end{tabular}
\end{center}
%
Instead of |\childdocof{|\textit{main}|}| just include the main file
at the top of each child file:
%
\begin{center}
|\input{|\textit{main}|}|
\end{center}
%
A simple redirection |\childdocforward{|\textit{dest}|}| is achieved by:
%
\begin{center}
|\def\jobname{|\textit{dest}|}\input{\jobname}|
\end{center}
%
The redirection with prefix
|\childdocforwardprefix[|\textit{prefix}|]{|\textit{dest}|}|
is accomplished by:
%
\begin{center}
\begin{tabular}{l}
|{\edef\jobname{\scantokens\expandafter{\jobname\noexpand}}|\\
|\def\redirectjob |\textit{prefix}|#1~~~{\gdef\jobname{|\textit{dest}|#1}}|\\
|\expandafter\redirectjob\jobname~~~}\input{\jobname}|
\end{tabular}
\end{center}

In an alternative approach,
child documents can be compiled by a specific command line
without additional code or specific definitions:
%
\begin{center}
|... -jobname "|\textit{target}|" "|[\textit{flags}]%
|\includeonly{|\textit{dest}|}\input{|\textit{main}|}"|
\end{center}
%

%%%%%%%%%%%%%%%%%%%%%%%%%%%%%%%%%%%%%%%%%%%%%%%%%%%%%%%%%%%%%%%%%%%%%%%%%%%%%%%%
%%%%%%%%%%%%%%%%%%%%%%%%%%%%%%%%%%%%%%%%%%%%%%%%%%%%%%%%%%%%%%%%%%%%%%%%%%%%%%%%
\section{Information}

%%%%%%%%%%%%%%%%%%%%%%%%%%%%%%%%%%%%%%%%%%%%%%%%%%%%%%%%%%%%%%%%%%%%%%%%%%%%%%%%
\subsection{Copyright}

Copyright \copyright{} 2017--2018 Niklas Beisert

This work may be distributed and/or modified under the
conditions of the \LaTeX{} Project Public License, either version 1.3
of this license or (at your option) any later version.
The latest version of this license is in
  \url{http://www.latex-project.org/lppl.txt}
and version 1.3 or later is part of all distributions of \LaTeX{}
version 2005/12/01 or later.

This work has the LPPL maintenance status `maintained'.

The Current Maintainer of this work is Niklas Beisert.

This work consists of the files |README.txt|, |childdoc.ins| and |childdoc.dtx|
as well as the derived files |childdoc.def|, |cdocsamp.tex|
with |cdocsch1.tex|, |cdocsch2.tex|, |cdocspt3.tex|, |cdocspt4.tex|,
|cdocsdrf.tex|, |cdocsfn1.tex|, |cdocsfn2.tex|
as well as |childdoc.pdf|.

%%%%%%%%%%%%%%%%%%%%%%%%%%%%%%%%%%%%%%%%%%%%%%%%%%%%%%%%%%%%%%%%%%%%%%%%%%%%%%%%
\subsection{Files and Installation}

The package consists of the files:
%
\begin{center}
\begin{tabular}{ll}
    |README.txt|   & readme file \\
    |childdoc.ins| & installation file \\
    |childdoc.dtx| & source file \\
    |childdoc.def| & definition file \\
    |cdocsamp.tex| & sample main file \\
    |cdocsch1.tex| & sample include file \\
    |cdocsch2.tex| & sample include file \\
    |cdocspt3.tex| & sample part file \\
    |cdocspt4.tex| & sample part file \\
    |cdocsdrf.tex| & sample redirection file \\
    |cdocsfn1.tex| & sample redirection file \\
    |cdocsfn2.tex| & sample redirection file \\
    |childdoc.pdf| & manual
\end{tabular}
\end{center}
%
The distribution consists of the files
|README.txt|, |childdoc.ins| and |childdoc.dtx|.
%
\begin{itemize}
\item
Run (pdf)\LaTeX{} on |childdoc.dtx|
to compile the manual |childdoc.pdf| (this file).
\item
Run \LaTeX{} on |childdoc.ins| to create the definitions file |childdoc.def|
and the sample |cdocsamp.tex| with include files
|cdocsch1.tex|, |cdocsch2.tex|, |cdocspt3.tex|, |cdocspt4.tex|,
|cdocsdrf.tex|, |cdocsfn1.tex|, |cdocsfn2.tex|.
Then copy the file |childdoc.def| to an appropriate directory of your \LaTeX{}
distribution, e.g.\ \textit{texmf-root}|/tex/latex/childdoc|.
\end{itemize}

%%%%%%%%%%%%%%%%%%%%%%%%%%%%%%%%%%%%%%%%%%%%%%%%%%%%%%%%%%%%%%%%%%%%%%%%%%%%%%%%
\subsection{Related CTAN Packages}

There are several other packages which offer a similar functionality:
%
\begin{itemize}
\item
The packages
\href{http://ctan.org/pkg/docmute}{\textsf{docmute}},
\href{http://ctan.org/pkg/includex}{\textsf{includex}} and
\href{http://ctan.org/pkg/standalone}{\textsf{standalone}}
provide commands to include only the document body of
a child file thus allowing both files to be compiled individually.
\item
The packages \href{http://ctan.org/pkg/subdocs}{\textsf{subdocs}}
and \href{http://ctan.org/pkg/subfiles}{\textsf{subfiles}}
provide structures in which the main and child documents can be
encapsulated and allowing them to be compiled individually.
The inclusion mechanism is different from the conventional |\include|.
\item
The package \href{http://ctan.org/pkg/combine}{\textsf{combine}}
is an elaborate solution to combine several documents into one.
\end{itemize}
%
See also the CTAN topic \href{http://ctan.org/topic/subdocs}{\textsf{subdocs}}
for further related packages.
The present package differs from the above solutions in that
a document structure constructed with the conventional |\include| mechanism
just needs two extra commands at the top of every file
such that all constituent files can be compiled individually.

%%%%%%%%%%%%%%%%%%%%%%%%%%%%%%%%%%%%%%%%%%%%%%%%%%%%%%%%%%%%%%%%%%%%%%%%%%%%%%%%
%\subsection{Feature Suggestions}
%
%The following is a list of features which may be useful for future
%versions of this package:
%%
%\begin{itemize}
%\item
%\ldots
%\end{itemize}

%%%%%%%%%%%%%%%%%%%%%%%%%%%%%%%%%%%%%%%%%%%%%%%%%%%%%%%%%%%%%%%%%%%%%%%%%%%%%%%%
\subsection{Revision History}

%%%%%%%%%%%%%%%%%%%%%%%%%%%%%%%%%%%%%%%%
\paragraph{v2.0:} 2018/12/30

\begin{itemize}
\item
immediate forward processing
\item
added |\childdocby| mechanism
\item
manual restructured
\end{itemize}

%%%%%%%%%%%%%%%%%%%%%%%%%%%%%%%%%%%%%%%%
\paragraph{v1.6:} 2018/01/17

\begin{itemize}
\item
application for development of include files
\item
corrections to manual
\end{itemize}

%%%%%%%%%%%%%%%%%%%%%%%%%%%%%%%%%%%%%%%%
\paragraph{v1.5:} 2017/05/21

\begin{itemize}
\item
more complete structuring introduced
\item
|\childdocof| introduced
\item
|\childdoc| renamed to |\childdocmain|
\item
|\childredirect| renamed to |\childdocforward| and |\childdocforwardprefix|
and functionality expanded
\end{itemize}

%%%%%%%%%%%%%%%%%%%%%%%%%%%%%%%%%%%%%%%%
\paragraph{v1.0:} 2017/04/27

\begin{itemize}
\item
manual and install package
\item
first version published on CTAN
\end{itemize}

%%%%%%%%%%%%%%%%%%%%%%%%%%%%%%%%%%%%%%%%
\paragraph{v0.6:} 2017/04/26

\begin{itemize}
\item
redirection mechanism added
\end{itemize}

%%%%%%%%%%%%%%%%%%%%%%%%%%%%%%%%%%%%%%%%
\paragraph{v0.5:} 2017/04/26

\begin{itemize}
\item
functionality in definition file
\end{itemize}


%%%%%%%%%%%%%%%%%%%%%%%%%%%%%%%%%%%%%%%%%%%%%%%%%%%%%%%%%%%%%%%%%%%%%%%%%%%%%%%%
%%%%%%%%%%%%%%%%%%%%%%%%%%%%%%%%%%%%%%%%%%%%%%%%%%%%%%%%%%%%%%%%%%%%%%%%%%%%%%%%
%%%%%%%%%%%%%%%%%%%%%%%%%%%%%%%%%%%%%%%%%%%%%%%%%%%%%%%%%%%%%%%%%%%%%%%%%%%%%%%%
\appendix

\settowidth\MacroIndent{\rmfamily\scriptsize 000\ }

 \DocInput{childdoc.dtx}

\end{document}
%</driver>
% \fi
%
% %%%%%%%%%%%%%%%%%%%%%%%%%%%%%%%%%%%%%%%%%%%%%%%%%%%%%%%%%%%%%%%%%%%%%%%%%%%%%%
% %%%%%%%%%%%%%%%%%%%%%%%%%%%%%%%%%%%%%%%%%%%%%%%%%%%%%%%%%%%%%%%%%%%%%%%%%%%%%%
% \section{Sample}
%\iffalse
%<*samplemain>
%\fi
%
% The following presents a sample document
% with two chapters, two parts, a title page,
% a compile flag as well as three forwarding files to set the flag.
% It consists of eight |.tex| files:
% \begin{center}
% \begin{tabular}{ll}
% |cdocsamp.tex|&main file\\
% |cdocsch1.tex|&include file for chapter 1\\
% |cdocsch2.tex|&include file for chapter 2\\
% |cdocspt3.tex|&include file for part 3\\
% |cdocspt4.tex|&include file for part 4\\
% |cdocsdrf.tex|&forwarding file for main file in draft mode\\
% |cdocsfi1.tex|&forwarding file for final version of chapter 1\\
% |cdocsfi2.tex|&forwarding file for final version of chapter 2\\
% \end{tabular}
% \end{center}
% Each of the eight files can be compiled directly by the \LaTeX{} compiler.
%
% %%%%%%%%%%%%%%%%%%%%%%%%%%%%%%%%%%%%%%
% \paragraph{Main File.}
%
% The main file is called |cdocsamp.tex|.
%
% Load the \textsf{childdoc} definitions and
% declare the filename for the main document:
%    \begin{macrocode}
\input{childdoc.def}
\childdocmain{}
%    \end{macrocode}

% Optional override for |\version| flag:
%    \begin{macrocode}
%%\ifchilddoc\else\providecommand{\version}{draft}\fi
%    \end{macrocode}

% Define the default values for the |\version| flag
% (|final| for the main file and |draft| for childs):
%    \begin{macrocode}
\ifchilddoc
\providecommand{\version}{draft}
\else
\providecommand{\version}{final}
\fi
%    \end{macrocode}

% Load the standard document class:
%    \begin{macrocode}
\documentclass[12pt]{article}
%    \end{macrocode}

% Start the document body:
%    \begin{macrocode}
\begin{document}
%    \end{macrocode}

% Declare a title page.
% Print title, part of document being processed and version flag:
%    \begin{macrocode}
\addtocounter{page}{-1}
\begin{center}
{\LARGE\bfseries{}childdoc example\par}
\vspace{1cm}
\ifchilddoc
\ifchilddocmanual part\else chapter\fi:
`\childdocname' of `\childdocjob'\par
\else
main document: `\childdocjob'\par
\fi
version: \version\par
\end{center}
\newpage
%    \end{macrocode}

% Manually include selected file,
% otherwise process as usual:
%    \begin{macrocode}
\ifchilddocmanual
\section*{part `\childdocname'}
\input{\childdocname}
\else
%    \end{macrocode}

% Include the two chapters:
%    \begin{macrocode}
\include{cdocsch1}
\include{cdocsch2}
%    \end{macrocode}

% Include the two parts unless only chapters should be displayed:
%    \begin{macrocode}
\ifchilddoc\else
\section{part three}
\input{cdocspt3}
\section{part four}
\input{cdocspt4}
\fi
%    \end{macrocode}

% Process as usual until here:
%    \begin{macrocode}
\fi
%    \end{macrocode}

% End of document body:
%    \begin{macrocode}
\end{document}
%    \end{macrocode}
%\iffalse
%</samplemain>
%\fi
%
% %%%%%%%%%%%%%%%%%%%%%%%%%%%%%%%%%%%%%%
% \paragraph{Chapter Include Files.}
%
% The include files are called |cdocsch1.tex| and |cdocsch2.tex|.
%
%\iffalse
%<*samplechap1|samplechap2>
%\fi

% Optional override for |\version| flag:
%    \begin{macrocode}
%%\providecommand{\version}{final}
%    \end{macrocode}

% Include the main document:
%    \begin{macrocode}
\input{childdoc.def}
\childdocof{cdocsamp}
%    \end{macrocode}

%\iffalse
%</samplechap1|samplechap2>
%\fi
%
%\iffalse
%<*samplechap1>
%\fi
% Some text for chapter 1:
%    \begin{macrocode}
\section{one}
some text in chapter one
%    \end{macrocode}

%\iffalse
%</samplechap1>
%\fi
% Some text for chapter 2:
%\iffalse
%<*samplechap2>
%\fi
%    \begin{macrocode}
\section{two}
more text in chapter two
%    \end{macrocode}

%\iffalse
%</samplechap2>
%\fi
%
% %%%%%%%%%%%%%%%%%%%%%%%%%%%%%%%%%%%%%%
% \paragraph{Part Include Files.}
%
% The include files are called |cdocspt3.tex| and |cdocspt4.tex|.
%
%\iffalse
%<*samplepart3|samplepart4>
%\fi

% Optional override for |\version| flag:
%    \begin{macrocode}
%%\providecommand{\version}{final}
%    \end{macrocode}

% Include the main document:
%    \begin{macrocode}
\input{childdoc.def}
\childdocby{cdocsamp}
%    \end{macrocode}

%\iffalse
%</samplepart3|samplepart4>
%\fi
%
%\iffalse
%<*samplepart3>
%\fi
% Some text for part 3:
%    \begin{macrocode}
some text in part three
%    \end{macrocode}

%\iffalse
%</samplepart3>
%\fi
% Some text for part 4:
%\iffalse
%<*samplepart4>
%\fi
%    \begin{macrocode}
more text in part four
%    \end{macrocode}

%\iffalse
%</samplepart4>
%\fi
%
% %%%%%%%%%%%%%%%%%%%%%%%%%%%%%%%%%%%%%%
% \paragraph{Forwarding for a Complete Draft.}
%
% The following forwarding file |cdocsdrf.tex|
% compiles the main document in draft mode:
%\iffalse
%<*sampledraft>
%\fi
%    \begin{macrocode}
\def\version{draft}
\input{childdoc.def}
\childdocforward{cdocsamp}
%    \end{macrocode}

%\iffalse
%</sampledraft>
%\fi
%
% %%%%%%%%%%%%%%%%%%%%%%%%%%%%%%%%%%%%%%
% \paragraph{Forwarding for Final Version of the Chapters.}
%
% The following forwarding files |cdocsfn1.tex| and |cdocsfn2.tex|
% (with identical content)
% compile the final versions of the child documents
% |cdocsch1.tex| and |cdocsch2.tex|, respectively:
%\iffalse
%<*samplefinal>
%\fi
%    \begin{macrocode}
\def\version{final}
\input{childdoc.def}
\childdocforwardprefix[cdocsamp]{cdocsfn}{cdocsch}
%    \end{macrocode}

%\iffalse
%</samplefinal>
%\fi
%
% %%%%%%%%%%%%%%%%%%%%%%%%%%%%%%%%%%%%%%
% \paragraph{Command Line Processing.}
%
% The following three command lines generate the output files
% |cdocscld|, |cdocscl1| and |cdocscl2|
% which should be identical to
% |cdocsdrf|, |cdocsch1| and |cdocsfn2|, respectively:
% \begin{center}
% \begin{tabular}{l}
% |latex -jobname cdocscld \|\\
% |  "\def\version{draft}\input{childdoc.def}\childdocforward{cdocsamp}"|\\
% |latex -jobname cdocscl1 \|\\
% |  "\input{childdoc.def}\childdocforward[cdocsamp]{cdocsch1}"|\\
% |latex -jobname cdocscl2 \|\\
% |  "\def\version{final}\input{childdoc.def}\childdocforward{cdocsch2}"|
% \end{tabular}
% \end{center}
% Note that the trailing backslash on each first line
% merely continues the input to the second line
% (for convenient cut ant paste).
% Furthermore, the command |latex| can be replaced by any
% of its alternative versions such as |pdflatex|.
%
% %%%%%%%%%%%%%%%%%%%%%%%%%%%%%%%%%%%%%%%%%%%%%%%%%%%%%%%%%%%%%%%%%%%%%%%%%%%%%%
% %%%%%%%%%%%%%%%%%%%%%%%%%%%%%%%%%%%%%%%%%%%%%%%%%%%%%%%%%%%%%%%%%%%%%%%%%%%%%%
% \section{Implementation}
%\iffalse
%<*package>
%\fi
%
% This section describes the definitions file |childdoc.def|.

% The definitions cannot be loaded using |\usepackage| or |\RequirePackage|
% which has a mechanism to prevent loading a style file more than once.
% When loading the definitions by means of |\input|
% multiple instances have to be prevented manually:
%\iffalse
%This code needs to be before the `\ProvidesFile' directive
%which is defined at the beginning of this file.
%Therefore it is also placed there and commented out here.
%</package>
%<*discard>
%\fi
%    \begin{macrocode}
\ifdefined\childdocmain\endinput\fi
%    \end{macrocode}
%\iffalse
%</discard>
%<*package>
%\fi
%
% \macro{\ifchilddoc}
% \macro{\ifchilddocmanual}
% The conditional |\ifchilddoc| tells whether a
% child (true) or main (false) document is being compiled.
% The conditional |\ifchilddocmanual| tells whether
% the |\includeonly| mechanism is used (false) or
% the selection of child files must be performed manually (true).
% The definitions initialise to false:
%    \begin{macrocode}
\newif\ifchilddoc
\newif\ifchilddocmanual
%    \end{macrocode}

% \macro{\childdocname}
% \macro{\childdocjob}
% The macro |\childdocname| stores the name of the main document
% to be compiled. The macro |\childdocjob| stores the name of
% the document on which the \LaTeX{} compiler was originally invoked.
% The content of |\jobname| cannot be compared
% to filenames specified in the source due to different catcodes.
% The following code rescans |\jobname|, stores the result
% in |\childdocname| and saves a copy in |\childdocjob|:
%    \begin{macrocode}
\edef\childdocname{\scantokens\expandafter{\jobname\noexpand}}
\let\childdocjob\childdocname
%    \end{macrocode}

% \macro{\childdocdisable}
% The macro |\childdocdisable| prevents the main file
% from being processed more than once.
% At this stage, the main document command |\childdocmain|
% is assumed to be called once again where it should do nothing.
% Any subsequent call to it should prevent
% a secondary processing of the main document
% It overwrites the forwarding commands
% |\childdocof| and |\childdocforward|
% with empty macros to prevent further inclusions of the main document:
%    \begin{macrocode}
\newcommand{\childdocdisable}
{
  \renewcommand{\childdocmain}[1]{\renewcommand{\childdocmain}[1]{\endinput}}
  \renewcommand{\childdocof}[1]{}
  \renewcommand{\childdocby}[2][]{}
  \renewcommand{\childdocforward}[2][]{}
  \renewcommand{\childdocdisable}{}
}
%    \end{macrocode}

% \macro{\childdocmain}
% The macro |\childdocmain| is to be called at the top of the main file
% with nothing or the main filename (without extension) as argument.
% First, it breaks loops.
% If the argument is not empty and does not match |\childdocname|
% (which is set by the first inclusion of |childdoc.def|),
% |\ifchilddoc| is set to true, |\includeonly| is applied to the child file
% and |\jobname| is set to the main file
% (for proper handling of |.aux| files):
%    \begin{macrocode}
\newcommand{\childdocmain}[1]
{
  \childdocdisable\childdocmain{}
  \if?#1?\else
    \begingroup
      \def\childdoctmp{#1}
      \ifx\childdoctmp\childdocname
        \def\childdoctmp{}
      \else
        \def\childdoctmp
        {
          \childdoctrue
          \includeonly{\childdocname}
          \def\childdocjob{#1}
          \def\jobname{#1}
        }
      \fi
      \expandafter
    \endgroup
    \childdoctmp
  \fi
}
%    \end{macrocode}

% \macro{\childdocof}
% The command |\childdocof| redirects
% compilation to the main file |#1|.
%    \begin{macrocode}
\newcommand{\childdocof}[1]
{
  \childdocdisable
  \childdoctrue
  \includeonly{\childdocname}
  \def\jobname{#1}
  \def\childdocjob{#1}
  \input{#1}
}
%    \end{macrocode}

% \macro{\childdocby}
% The command |\childdocby| ....
%    \begin{macrocode}
\newcommand{\childdocby}[2][]
{
  \childdocdisable
  \childdoctrue
  \childdocmanualtrue
  \if?#1?\else
    \def\jobname{#2}
  \fi
  \def\childdocjob{#2}
  \input{#2}
  \endinput
}
%    \end{macrocode}

% \macro{\childdocforward}
% The command |\childdocforward| redirects
% compilation to the main file or
% (if the optional argument is given) a child file.
% Parameters are set as if the main file
% or a child file starting with |\childdocof| was compiled.
% Then compilation is handed over to the main file:
%    \begin{macrocode}
\newcommand{\childdocforward}[2][]
{
  \begingroup
    \if?#1?
      \def\childdoctmp
      {
        \def\childdocname{#2}
        \def\childdocjob{#2}
        \def\jobname{#2}
        \input{#2}
        \endinput
      }
    \else
      \def\childdoctmp
      {
        \childdocdisable
        \def\childdocname{#2}
        \childdoctrue
        \includeonly{#2}
        \def\childdocjob{#1}
        \def\jobname{#1}
        \input{#1}
        \endinput
      }
    \fi
    \expandafter
  \endgroup
  \childdoctmp
}
%    \end{macrocode}

% \macro{\childdocforwardprefix}
% The command |\childdocforwardprefix| redirects
% compilation to the main or a child file by means of a pattern.
% The prefix |#1| in the current filename is replaced by |#2|
% and the suffix of the current filename is kept
% (it is assumed that the filename does not contain the substring `|~~~|'
% which is used as a delimiter).
% Compilation is handed over to the new file by |\childdocforward|:
%    \begin{macrocode}
\newcommand{\childdocforwardprefix}[3][]
{
  \begingroup
    \def\childdocextract #2##1~~~{\def\childdoctmp{\childdocforward[#1]{#3##1}}}
    \expandafter\childdocextract\childdocname~~~
    \expandafter
  \endgroup
  \childdoctmp
}
%    \end{macrocode}

% \macro{\childdoc}
% The deprecated macro |\childdoc| is a legacy version of |\childdocmain|:
%    \begin{macrocode}
\newcommand{\childdoc}{\childdocmain}
%    \end{macrocode}

% \macro{\childdocredirect}
% The deprecated macro |\childdocredirect| is a legacy version
% of |\childdocforward| and |\childdocforwardprefix|:
%    \begin{macrocode}
\newcommand{\childdocredirect}[2][]
{
  \begingroup
    \if?#1?
      \def\childdoctmp{\childdocforward{#2}}
    \else
      \def\childdoctmp{\childdocforwardprefix{#1}{#2}}
    \fi
    \expandafter
  \endgroup
  \childdoctmp
}
%    \end{macrocode}

%\iffalse
%</package>
%\fi
%
\endinput
|
and perform the replacements as outlined below.
Instead of |\childdocmain{|\textit{main}|}| add the following code
to the top of the main file:
%
\begin{center}
\begin{tabular}{l}
|\||ifdefined\childdocname\endinput\||fi\newif\ifchilddoc|\\
|\edef\childdocname{\scantokens\expandafter{\jobname\noexpand}}|\\
|\def\childdocmain{|\textit{main}|}\||ifx\childdocmain\childdocname\||else|\\
|\childdoctrue\includeonly{\childdocname}\let\jobname\childdocmain\||fi|\\
\end{tabular}
\end{center}
%
Instead of |\childdocof{|\textit{main}|}| just include the main file
at the top of each child file:
%
\begin{center}
|\input{|\textit{main}|}|
\end{center}
%
A simple redirection |\childdocforward{|\textit{dest}|}| is achieved by:
%
\begin{center}
|\def\jobname{|\textit{dest}|}\input{\jobname}|
\end{center}
%
The redirection with prefix
|\childdocforwardprefix[|\textit{prefix}|]{|\textit{dest}|}|
is accomplished by:
%
\begin{center}
\begin{tabular}{l}
|{\edef\jobname{\scantokens\expandafter{\jobname\noexpand}}|\\
|\def\redirectjob |\textit{prefix}|#1~~~{\gdef\jobname{|\textit{dest}|#1}}|\\
|\expandafter\redirectjob\jobname~~~}\input{\jobname}|
\end{tabular}
\end{center}

In an alternative approach,
child documents can be compiled by a specific command line
without additional code or specific definitions:
%
\begin{center}
|... -jobname "|\textit{target}|" "|[\textit{flags}]%
|\includeonly{|\textit{dest}|}\input{|\textit{main}|}"|
\end{center}
%

%%%%%%%%%%%%%%%%%%%%%%%%%%%%%%%%%%%%%%%%%%%%%%%%%%%%%%%%%%%%%%%%%%%%%%%%%%%%%%%%
%%%%%%%%%%%%%%%%%%%%%%%%%%%%%%%%%%%%%%%%%%%%%%%%%%%%%%%%%%%%%%%%%%%%%%%%%%%%%%%%
\section{Information}

%%%%%%%%%%%%%%%%%%%%%%%%%%%%%%%%%%%%%%%%%%%%%%%%%%%%%%%%%%%%%%%%%%%%%%%%%%%%%%%%
\subsection{Copyright}

Copyright \copyright{} 2017--2018 Niklas Beisert

This work may be distributed and/or modified under the
conditions of the \LaTeX{} Project Public License, either version 1.3
of this license or (at your option) any later version.
The latest version of this license is in
  \url{http://www.latex-project.org/lppl.txt}
and version 1.3 or later is part of all distributions of \LaTeX{}
version 2005/12/01 or later.

This work has the LPPL maintenance status `maintained'.

The Current Maintainer of this work is Niklas Beisert.

This work consists of the files |README.txt|, |childdoc.ins| and |childdoc.dtx|
as well as the derived files |childdoc.def|, |cdocsamp.tex|
with |cdocsch1.tex|, |cdocsch2.tex|, |cdocspt3.tex|, |cdocspt4.tex|,
|cdocsdrf.tex|, |cdocsfn1.tex|, |cdocsfn2.tex|
as well as |childdoc.pdf|.

%%%%%%%%%%%%%%%%%%%%%%%%%%%%%%%%%%%%%%%%%%%%%%%%%%%%%%%%%%%%%%%%%%%%%%%%%%%%%%%%
\subsection{Files and Installation}

The package consists of the files:
%
\begin{center}
\begin{tabular}{ll}
    |README.txt|   & readme file \\
    |childdoc.ins| & installation file \\
    |childdoc.dtx| & source file \\
    |childdoc.def| & definition file \\
    |cdocsamp.tex| & sample main file \\
    |cdocsch1.tex| & sample include file \\
    |cdocsch2.tex| & sample include file \\
    |cdocspt3.tex| & sample part file \\
    |cdocspt4.tex| & sample part file \\
    |cdocsdrf.tex| & sample redirection file \\
    |cdocsfn1.tex| & sample redirection file \\
    |cdocsfn2.tex| & sample redirection file \\
    |childdoc.pdf| & manual
\end{tabular}
\end{center}
%
The distribution consists of the files
|README.txt|, |childdoc.ins| and |childdoc.dtx|.
%
\begin{itemize}
\item
Run (pdf)\LaTeX{} on |childdoc.dtx|
to compile the manual |childdoc.pdf| (this file).
\item
Run \LaTeX{} on |childdoc.ins| to create the definitions file |childdoc.def|
and the sample |cdocsamp.tex| with include files
|cdocsch1.tex|, |cdocsch2.tex|, |cdocspt3.tex|, |cdocspt4.tex|,
|cdocsdrf.tex|, |cdocsfn1.tex|, |cdocsfn2.tex|.
Then copy the file |childdoc.def| to an appropriate directory of your \LaTeX{}
distribution, e.g.\ \textit{texmf-root}|/tex/latex/childdoc|.
\end{itemize}

%%%%%%%%%%%%%%%%%%%%%%%%%%%%%%%%%%%%%%%%%%%%%%%%%%%%%%%%%%%%%%%%%%%%%%%%%%%%%%%%
\subsection{Related CTAN Packages}

There are several other packages which offer a similar functionality:
%
\begin{itemize}
\item
The packages
\href{http://ctan.org/pkg/docmute}{\textsf{docmute}},
\href{http://ctan.org/pkg/includex}{\textsf{includex}} and
\href{http://ctan.org/pkg/standalone}{\textsf{standalone}}
provide commands to include only the document body of
a child file thus allowing both files to be compiled individually.
\item
The packages \href{http://ctan.org/pkg/subdocs}{\textsf{subdocs}}
and \href{http://ctan.org/pkg/subfiles}{\textsf{subfiles}}
provide structures in which the main and child documents can be
encapsulated and allowing them to be compiled individually.
The inclusion mechanism is different from the conventional |\include|.
\item
The package \href{http://ctan.org/pkg/combine}{\textsf{combine}}
is an elaborate solution to combine several documents into one.
\end{itemize}
%
See also the CTAN topic \href{http://ctan.org/topic/subdocs}{\textsf{subdocs}}
for further related packages.
The present package differs from the above solutions in that
a document structure constructed with the conventional |\include| mechanism
just needs two extra commands at the top of every file
such that all constituent files can be compiled individually.

%%%%%%%%%%%%%%%%%%%%%%%%%%%%%%%%%%%%%%%%%%%%%%%%%%%%%%%%%%%%%%%%%%%%%%%%%%%%%%%%
%\subsection{Feature Suggestions}
%
%The following is a list of features which may be useful for future
%versions of this package:
%%
%\begin{itemize}
%\item
%\ldots
%\end{itemize}

%%%%%%%%%%%%%%%%%%%%%%%%%%%%%%%%%%%%%%%%%%%%%%%%%%%%%%%%%%%%%%%%%%%%%%%%%%%%%%%%
\subsection{Revision History}

%%%%%%%%%%%%%%%%%%%%%%%%%%%%%%%%%%%%%%%%
\paragraph{v2.0:} 2018/12/30

\begin{itemize}
\item
immediate forward processing
\item
added |\childdocby| mechanism
\item
manual restructured
\end{itemize}

%%%%%%%%%%%%%%%%%%%%%%%%%%%%%%%%%%%%%%%%
\paragraph{v1.6:} 2018/01/17

\begin{itemize}
\item
application for development of include files
\item
corrections to manual
\end{itemize}

%%%%%%%%%%%%%%%%%%%%%%%%%%%%%%%%%%%%%%%%
\paragraph{v1.5:} 2017/05/21

\begin{itemize}
\item
more complete structuring introduced
\item
|\childdocof| introduced
\item
|\childdoc| renamed to |\childdocmain|
\item
|\childredirect| renamed to |\childdocforward| and |\childdocforwardprefix|
and functionality expanded
\end{itemize}

%%%%%%%%%%%%%%%%%%%%%%%%%%%%%%%%%%%%%%%%
\paragraph{v1.0:} 2017/04/27

\begin{itemize}
\item
manual and install package
\item
first version published on CTAN
\end{itemize}

%%%%%%%%%%%%%%%%%%%%%%%%%%%%%%%%%%%%%%%%
\paragraph{v0.6:} 2017/04/26

\begin{itemize}
\item
redirection mechanism added
\end{itemize}

%%%%%%%%%%%%%%%%%%%%%%%%%%%%%%%%%%%%%%%%
\paragraph{v0.5:} 2017/04/26

\begin{itemize}
\item
functionality in definition file
\end{itemize}


%%%%%%%%%%%%%%%%%%%%%%%%%%%%%%%%%%%%%%%%%%%%%%%%%%%%%%%%%%%%%%%%%%%%%%%%%%%%%%%%
%%%%%%%%%%%%%%%%%%%%%%%%%%%%%%%%%%%%%%%%%%%%%%%%%%%%%%%%%%%%%%%%%%%%%%%%%%%%%%%%
%%%%%%%%%%%%%%%%%%%%%%%%%%%%%%%%%%%%%%%%%%%%%%%%%%%%%%%%%%%%%%%%%%%%%%%%%%%%%%%%
\appendix

\settowidth\MacroIndent{\rmfamily\scriptsize 000\ }

 \DocInput{childdoc.dtx}

\end{document}
%</driver>
% \fi
%
% %%%%%%%%%%%%%%%%%%%%%%%%%%%%%%%%%%%%%%%%%%%%%%%%%%%%%%%%%%%%%%%%%%%%%%%%%%%%%%
% %%%%%%%%%%%%%%%%%%%%%%%%%%%%%%%%%%%%%%%%%%%%%%%%%%%%%%%%%%%%%%%%%%%%%%%%%%%%%%
% \section{Sample}
%\iffalse
%<*samplemain>
%\fi
%
% The following presents a sample document
% with two chapters, two parts, a title page,
% a compile flag as well as three forwarding files to set the flag.
% It consists of eight |.tex| files:
% \begin{center}
% \begin{tabular}{ll}
% |cdocsamp.tex|&main file\\
% |cdocsch1.tex|&include file for chapter 1\\
% |cdocsch2.tex|&include file for chapter 2\\
% |cdocspt3.tex|&include file for part 3\\
% |cdocspt4.tex|&include file for part 4\\
% |cdocsdrf.tex|&forwarding file for main file in draft mode\\
% |cdocsfi1.tex|&forwarding file for final version of chapter 1\\
% |cdocsfi2.tex|&forwarding file for final version of chapter 2\\
% \end{tabular}
% \end{center}
% Each of the eight files can be compiled directly by the \LaTeX{} compiler.
%
% %%%%%%%%%%%%%%%%%%%%%%%%%%%%%%%%%%%%%%
% \paragraph{Main File.}
%
% The main file is called |cdocsamp.tex|.
%
% Load the \textsf{childdoc} definitions and
% declare the filename for the main document:
%    \begin{macrocode}
% \iffalse
%
% childdoc.dtx Copyright (C) 2017-2018 Niklas Beisert
%
% This work may be distributed and/or modified under the
% conditions of the LaTeX Project Public License, either version 1.3
% of this license or (at your option) any later version.
% The latest version of this license is in
%   http://www.latex-project.org/lppl.txt
% and version 1.3 or later is part of all distributions of LaTeX
% version 2005/12/01 or later.
%
% This work has the LPPL maintenance status `maintained'.
%
% The Current Maintainer of this work is Niklas Beisert.
%
% This work consists of the files childdoc.dtx and childdoc.ins
% and the derived files childdoc.def and cdocsamp.tex with
% cdocsch1.tex, cdocsch2.tex, cdocsdrf.tex, cdocsfn1.tex, cdocsfn2.tex.
%
%<package>\ifdefined\childdocmain\endinput\fi
%<package>\ProvidesFile{childdoc.def}[2018/12/30 v2.0 child document driver]
%<samplemain>\ProvidesFile{cdocsamp.tex}[2018/12/30 v2.0 sample for childdoc]
%<*driver>
%\ProvidesFile{childdoc.drv}[2018/12/30 v2.0 childdoc reference manual file]
\PassOptionsToClass{10pt,a4paper}{article}
\documentclass{ltxdoc}

\usepackage[margin=35mm]{geometry}
\usepackage{hyperref}
\usepackage{hyperxmp}
\usepackage[usenames]{color}

\hypersetup{colorlinks=true}
\hypersetup{pdfstartview=FitH}
\hypersetup{pdfpagemode=UseNone}
\hypersetup{pdfsource={}}
\hypersetup{pdflang={en-UK}}
\hypersetup{pdfcopyright={Copyright 2017-2018 Niklas Beisert.
  This work may be distributed and/or modified under the
  conditions of the LaTeX Project Public License, either version 1.3
  of this license or (at your option) any later version.}}
\hypersetup{pdflicenseurl={http://www.latex-project.org/lppl.txt}}
\hypersetup{pdfcontactaddress={ETH Zurich, ITP, HIT K,
  Wolfgang-Pauli-Strasse 27}}
\hypersetup{pdfcontactpostcode={8093}}
\hypersetup{pdfcontactcity={Zurich}}
\hypersetup{pdfcontactcountry={Switzerland}}
\hypersetup{pdfcontactemail={nbeisert@itp.phys.ethz.ch}}
\hypersetup{pdfcontacturl={http://people.phys.ethz.ch/\xmptilde nbeisert/}}

\newcommand{\secref}[1]{\hyperref[#1]{section \ref*{#1}}}

\parskip1ex
\parindent0pt
\let\olditemize\itemize
\def\itemize{\olditemize\parskip0pt}

\begin{document}

\title{The \textsf{childdoc} Package}
\hypersetup{pdftitle={The childdoc Package}}
\author{Niklas Beisert\\[2ex]
  Institut f\"ur Theoretische Physik\\
  Eidgen\"ossische Technische Hochschule Z\"urich\\
  Wolfgang-Pauli-Strasse 27, 8093 Z\"urich, Switzerland\\[1ex]
  \href{mailto:nbeisert@itp.phys.ethz.ch}
  {\texttt{nbeisert@itp.phys.ethz.ch}}}
\hypersetup{pdfauthor={Niklas Beisert}}
\hypersetup{pdfsubject={Manual for the LaTeX2e Package childdoc}}
\date{30 December 2018, \textsf{v2.0}}
\maketitle

\begin{abstract}\noindent
\textsf{childdoc} is a \LaTeXe{} package
that enables the direct compilation
of document sections included by |\include|
to individual files.
\end{abstract}

\begingroup
\parskip0ex
\tableofcontents
\endgroup

%%%%%%%%%%%%%%%%%%%%%%%%%%%%%%%%%%%%%%%%%%%%%%%%%%%%%%%%%%%%%%%%%%%%%%%%%%%%%%%%
%%%%%%%%%%%%%%%%%%%%%%%%%%%%%%%%%%%%%%%%%%%%%%%%%%%%%%%%%%%%%%%%%%%%%%%%%%%%%%%%
\section{Introduction}

\LaTeX{} provides a mechanism to structure a large document (such as a book)
into a main file and several child files (containing the chapters)
using the |\include| command.
This mechanism is beneficial for documents
which span hundreds of pages in order to
make the source file(s) more manageable.
Moreover, compilation can be restricted to
selected child files by means of the |\includeonly| command.
The latter feature can be used to reduce the compilation time while editing
(this was significantly more useful in the earlier days of \LaTeX{})
or to generate a smaller document which is easier to navigate.
Another application of |\includeonly| is to generate
documents consisting of selected parts of the complete document.

However, there are a few drawbacks of the plain |\include| mechanism:
\begin{itemize}
\item
The child files cannot be compiled on their own,
they can only be compiled via the main file.
A naive editing environment
(such as a text editor with an option
to have the current file processed by \LaTeX)
may require one to switch to the main file before compiling;
attempting to compile the child file produces errors.
\item
The main file must be modified (each time)
to adjust the |\includeonly| command
to the present needs. This easily leaves the main file in a messy state.
\item
The generated document will always carry the filename
of the main document. This is inconvenient if
several child files are to be compiled and
to be kept for distribution.
\end{itemize}

The present package provides a simple interface
to make child files individually compilable by \LaTeX{}.
Compiling a child file then has the same effect as compiling
the main file with an |\includeonly| command
to select the appropriate child.
Moreover the generated document will carry the name of the child
rather than the main file.
This resolves all three above issues.

This feature is meant to make the editing of books,
thesis documents and lecture notes somewhat more convenient.
However, the package can also be used efficiently for
composing a series of documents (such as exercise sheets)
which are typically distributed individually.
It then assists the author in generating the individual documents
(potentially in different versions)
as well as a document containing the collected series.
Another application is in developing style files
or other kinds of included material
where compilation of the style file could redirect
to a sample or test file.

%%%%%%%%%%%%%%%%%%%%%%%%%%%%%%%%%%%%%%%%%%%%%%%%%%%%%%%%%%%%%%%%%%%%%%%%%%%%%%%%
%%%%%%%%%%%%%%%%%%%%%%%%%%%%%%%%%%%%%%%%%%%%%%%%%%%%%%%%%%%%%%%%%%%%%%%%%%%%%%%%
\section{Usage}

First of all, the package \textsf{childdoc} is \emph{not} a standard
\LaTeXe{} |.sty| style file! Therefore it needs to be invoked in
a non-standard way.

%%%%%%%%%%%%%%%%%%%%%%%%%%%%%%%%%%%%%%%%%%%%%%%%%%%%%%%%%%%%%%%%%%%%%%%%%%%%%%%%
\subsection{Included Files}
\label{sec:include}

%%%%%%%%%%%%%%%%%%%%%%%%%%%%%%%%%%%%%%%%
\DescribeMacro{\childdocmain}
To use the package, add the commands
\begin{center}
\begin{tabular}{l}
|\input{childdoc.def}|\\
|\childdocmain{}|\\
\end{tabular}
\end{center}
at the very top of the main \LaTeX{} file,
in particular \emph{before} the |\documentclass| statement!
The argument of |\childdocmain| should be left empty
(but it must be present).

%%%%%%%%%%%%%%%%%%%%%%%%%%%%%%%%%%%%%%%%
\DescribeMacro{\childdocof}
Furthermore, add the commands
\begin{center}
\begin{tabular}{l}
|\input{childdoc.def}|\\
|\childdocof{|\textit{main}|}|\\
\end{tabular}
\end{center}
at the top of every child file \textit{child}
which is included by |\include{|\textit{child}|}|
from within the main file
(or at least for those files to be compiled individually).
The argument \textit{main} must be the filename of the main file.

There are a couple of
considerations in setting up the main and child documents:

%%%%%%%%%%%%%%%%%%%%%%%%%%%%%%%%%%%%%%%%
\paragraph{Restrictions.}

Please note the following restrictions:
\begin{itemize}
\item
|\childdocmain| must be called with one argument \textit{main}
to ensure compatibility with earlier version of the package.
It must either be empty (|\childdocmain{}|)
or precisely match the filename of the main file in which it is specified.
See \secref{sec:detection} for further information.
\item
The filename \textit{main} must be specified without the |.tex| extension.
\item
The filename \textit{main} is case sensitive
(even in case-insensitive file systems)
due to internal string comparison.
\item
The argument \textit{main} should be fully expanded, it cannot be a macro.
\item
Subdirectories and special characters should be avoided in filenames.
\item
The command |\childdocmain{|\textit{main}|}| must be followed by a whitespace.
It should not be followed immediately by another command
or by a comment mark `|%|'.
This is because the \TeX{} parser reads the token immediately following
the argument of |\childdocmain| and puts it
at the beginning of every child section;
however, a white\-space is ignored.
\end{itemize}

%%%%%%%%%%%%%%%%%%%%%%%%%%%%%%%%%%%%%%%%
\paragraph{Content of Main File.}

It is advisable to place all content in the child files included by |\include|.
Any output contained in the main file will appear in all child documents
unless suppressed manually;
it cannot be suppressed automatically by the |\includeonly| directive
and thus should normally be avoided.
A method to include some content in the main file
by means of conditional processing is described in \secref{sec:conditional}.

%%%%%%%%%%%%%%%%%%%%%%%%%%%%%%%%%%%%%%%%
\paragraph{Page Numbering.}

When only a part of the document is compiled,
the appropriate numbering of pages
(as well as other status parameters)
is determined from the |.aux| files.
The latter contain information from previous passes.
However this information needs to propagate through
all intermediate child documents.
Therefore the page numbering in child documents may well
be inconsistent until the complete document is compiled at least once.

A useful (if unconventional) way to always ensure a consistent
page numbering is to restart the numbering in each child document
and denote the pages by `\textit{child}|.|\textit{page}'
where \textit{child} represents the chapter/section number of the child file.
This can be achieved by the command
|\numberwithin{page}{|\textit{child}|}|
of the \textsf{amsmath} package
where \textit{child} can be |chapter| or |section|
depending on the chosen structuring.
Alternatively, one can modify the macro |\thepage| appropriately
and reset the counter |page| at the start of each child file.

%%%%%%%%%%%%%%%%%%%%%%%%%%%%%%%%%%%%%%%%%%%%%%%%%%%%%%%%%%%%%%%%%%%%%%%%%%%%%%%%
\subsection{Conditional Processing}
\label{sec:conditional}

The package provides a mechanism to compile different versions
of a document. To customise the versions further some conditional processing
can come in handy to distinguish which version is being compiled.
The package provides two macros to describe the compilation context:

%%%%%%%%%%%%%%%%%%%%%%%%%%%%%%%%%%%%%%%%
\DescribeMacro{\ifchilddoc}
The conditional |\ifchilddoc| distinguishes between the compilation of
child documents and the main document:
%
\begin{center}
|\ifchilddoc |\textit{child-code}| |[|\||else |\textit{main-code}]| \||fi|
\end{center}

%%%%%%%%%%%%%%%%%%%%%%%%%%%%%%%%%%%%%%%%
\DescribeMacro{\childdocname}
\DescribeMacro{\childdocjob}
The macro |\childdocname| contains the filename (without extension)
of the main or child file being processed.
Note that |\childdocjob| will always contain the name of the main file.

%%%%%%%%%%%%%%%%%%%%%%%%%%%%%%%%%%%%%%%%
\paragraph{Title Page.}

Conditional processing can be used to include a title or banner page
in the main document when proper precautions are taken.
Importantly, the code in the main file should ensure that the page counter
(as well as other status parameters which are stored in the |.aux| files)
takes the same value after the conditional processing.
Otherwise the page numbers may take divergent values
depending on which part is compiled.

For example, a title page could be declared by:
%
\begin{center}
\begin{tabular}{l}
|\ifchilddoc\||else|\\
|\addtocounter{page}{-1}|\\
\textit{code for title page}\\
|\newpage|\\
|\||fi|
\end{tabular}
\end{center}
%
A banner page for the child documents can be generated by:
%
\begin{center}
\begin{tabular}{l}
|\ifchilddoc|\\
|\addtocounter{page}{-1}|\\
\textit{code for banner page}\\
|\newpage|\\
|\||fi|
\end{tabular}
\end{center}
%
Here one could write a message such as:
\begin{center}
|This is the part \childdocname{} of \childdocjob{}.|
\end{center}

%%%%%%%%%%%%%%%%%%%%%%%%%%%%%%%%%%%%%%%%%%%%%%%%%%%%%%%%%%%%%%%%%%%%%%%%%%%%%%%%
\subsection{Flags}
\label{sec:flags}

The package makes it easy to generate different versions
of the main or child documents.
To this end compilation flags can be defined
and assigned different default values.
They will be particularly useful in conjunction
with the forwarding mechanism described in \secref{sec:forward}.

For example, it may be useful to have a flag |\version|
which can be set to |draft| or |final|.
The document source will contain some conditional code
depending on the value of |\version|.
Suppose further, the flag should default to |final| for the main file
and to |draft| for child files
which is a natural assignment for editing the document.
This is achieved by placing the following code
in the preamble of the main document
(below the |\childdocmain| directive):
%
\begin{center}
\begin{tabular}{l}
|\ifchilddoc|\\
|\providecommand{\version}{draft}|\\
|\||else|\\
|\providecommand{\version}{final}|\\
|\||fi|
\end{tabular}
\end{center}
%
The definition by |\providecommand| makes sure
that previous definitions are not overwritten.
Further statements |\providecommand{\version}{...}|
can thus be added before the above code to override it.

For the main file, one might add a line
(between |\childdocmain| and the above block)
%
\begin{center}
|%\ifchilddoc\||else\providecommand{\version}{draft}\||fi|
\end{center}
%
which can be uncommented to produce a draft version.
Likewise one can add a line to the very top of a child file
(above the |\childdocof{|\textit{main}|}| directive)
%
\begin{center}
|%\providecommand{\version}{final}|
\end{center}
%
which can be uncommented to produce the final version of this child document.

%%%%%%%%%%%%%%%%%%%%%%%%%%%%%%%%%%%%%%%%%%%%%%%%%%%%%%%%%%%%%%%%%%%%%%%%%%%%%%%%
\subsection{Forwarding}
\label{sec:forward}

Different versions of the main or child documents
using compilation flags as described in \secref{sec:flags}
can be (permanently) stored in different files
for convenient compilation, viewing and distribution.
To this end, the package defines a command
to pass on compilation to a different file:

%%%%%%%%%%%%%%%%%%%%%%%%%%%%%%%%%%%%%%%%
\DescribeMacro{\childdocforward}
The command |\childdocforward| redirects processing to
another source file:
%
\begin{center}
\begin{tabular}{l}
|\input{childdoc.def}|\\
|\childdocforward[|\textit{main}|]{|\textit{dest}|}|\\
\end{tabular}
\end{center}
%
The argument \textit{dest} is the destination file
(without extension).
It should be the main file or one of the child files.
Note that further \textsf{childdoc} directives
such as |\childdocof| and |\childdocforward|
in the indicated file will be processed in this form.
The optional argument \textit{main}
passes on directly to the main file \textit{main}
while pretending to compile the child \textit{dest}.
This form behaves as if \textit{dest}
issues |\childdocof{|\textit{main}|}| right away,
and no further \textsf{childdoc} directives will be processed.

%%%%%%%%%%%%%%%%%%%%%%%%%%%%%%%%%%%%%%%%
\DescribeMacro{\...prefix}
In the alternative form |\childdocforwardprefix|,
%
\begin{center}
\begin{tabular}{l}
|\input{childdoc.def}|\\
|\childdocforwardprefix[|\textit{main}|]{|\textit{prefix}|}{|\textit{dest}|}|
\end{tabular}
\end{center}
%
the destination file is determined by a pattern
depending on the current file:
To make this work, the current file must be called
`{\textit{prefix}\hspace{0.2em}\textit{suffix}}'
with \textit{prefix} matching precisely the argument.
Processing is then passed on to the file
`{\textit{dest}\hspace{0.2em}\textit{suffix}}'.
Surely, the same effect is achieved by
directly specifying the
argument `{\textit{dest}\hspace{0.2em}\textit{suffix}}'
in the first form.
However, that requires to set up a different file
for each child. With the alternative form of the command
all these files can have exactly the same content
which simplifies setting them up and maintaining them.

For example, the following file |draft.tex|
with a compilation flag |\version| as described in \secref{sec:flags}
compiles the main document as a draft:
%
\begin{center}
\begin{tabular}{l}
|\def\version{draft}|\\
|\input{childdoc.def}|\\
|\childdocforward{|\textit{main}|}|
\end{tabular}
\end{center}
%
Likewise, the following files |final|\textit{nn}|.tex|
compile the final version of the child document
|child|\textit{nn}|.tex|:
%
\begin{center}
\begin{tabular}{l}
|\def\version{final}|\\
|\input{childdoc.def}|\\
|\childdocforwardprefix{final}{child}|
\end{tabular}
\end{center}
%

Note that when several versions of a main file and/or of each child file
are to be generated, it may be convenient to set up a |Makefile| or
shell script to automatise the process.

%%%%%%%%%%%%%%%%%%%%%%%%%%%%%%%%%%%%%%%%%%%%%%%%%%%%%%%%%%%%%%%%%%%%%%%%%%%%%%%%
\subsection{Command Line Processing}
\label{sec:commandline}

The effect of redirection files can also be achieved by invoking
the \LaTeX{} compiler with a more elaborate command line.
Most conveniently this should be done as part
of a shell script or a |Makefile|.

When using \textsf{childdoc} in the main file, the following
command lines effectively perform a redirection
(note that depending on the shell being used,
backslashes may have to be doubled: `|\|' $\to$ `|\\|'):
%
\begin{center}
|... -jobname "|\textit{target}|" |\\|"|[\textit{flags}]%
|\input{childdoc.def}\childdocforward[|\textit{main}|]{|\textit{dest}|}"|
\end{center}
%
Here \textit{target} is the name of the output file,
\textit{main} is the name of the main file
and \textit{dest} is the name of the main or child file to be processed
(all filenames without extensions).
The optional argument \textit{main} can be omitted
if \textit{main} matches \textit{dest}.
Optionally, compilation \textit{flags} can be defined via |\def| commands.
This command line makes the \TeX{} engine believe
it is compiling the file \textit{target}
whose content is specified as the latter parameter.
The provided code then forwards the processing to
\textit{main} or \textit{dest} as described in \secref{sec:forward}.

%%%%%%%%%%%%%%%%%%%%%%%%%%%%%%%%%%%%%%%%%%%%%%%%%%%%%%%%%%%%%%%%%%%%%%%%%%%%%%%%
\subsection{Include by Input}
\label{sec:input}

Including child documents by |\include| has some restrictions by design.
Most notably, the content of a child document always occupies
its own set of pages; pages cannot be shared between child documents.
Usually, this behaviour makes perfect sense
because each child document contain an essential part of the document.
However, in some situations it may be desirable to compose
a document from a collection of parts
without having mandatory page breaks between then.
For this case, the package
provides a mechanism to include parts
by |\input| which can also be processed individually.
However, by construction this mechanism
requires manual handling of the content to be output.

%%%%%%%%%%%%%%%%%%%%%%%%%%%%%%%%%%%%%%%%
\DescribeMacro{\ifchilddocmanual}
The main file should be prepared as usual, see \secref{sec:include}.
However, the document body must make a distinction
between processing of an individual part and of the main document, e.g.:
%
\begin{center}
\begin{tabular}{l}
|\ifchilddocmanual|\\
|\input{\childdocname}|\\
|\||else|\\
\textit{document body with }|\input{|\textit{part}|}|\\
|\||fi|
\end{tabular}
\end{center}
%
The conditional |\ifchilddocmanual| is true whenever
a part to be included by |\input| is being compiled,
and the name of the part is stored in |\childdocname|.

%%%%%%%%%%%%%%%%%%%%%%%%%%%%%%%%%%%%%%%%
\DescribeMacro{\childdocby}
Each part to be included by |\input| should start with:
%
\begin{center}
\begin{tabular}{l}
|\input{childdoc.def}|\\
|\childdocby{|\textit{main}|}|\\
\end{tabular}
\end{center}
%
The directive |\childdocby| is similar to |\childdocof|
described in \secref{sec:include},
but the subsequent selection of content must be done manually.
To that end, both |\ifchilddoc| and |\ifchilddocmanual|
will be true upon processing of a part,
and the name of the part is stored in |\childdocname|.
Note that |\jobname| will be set to the filename of the current part
so that each part receives an individual |.aux| file
that does not interfere with the |.aux| file(s) of the main document.
This behaviour can be altered by the alternative form
|\childdocby[*]{|\textit{main}|}| (with a non-empty optional argument)
which uses the |.aux| file of the main document
by setting |\jobname| to \textit{main}.

%%%%%%%%%%%%%%%%%%%%%%%%%%%%%%%%%%%%%%%%%%%%%%%%%%%%%%%%%%%%%%%%%%%%%%%%%%%%%%%%
\subsection{Driver Development}
\label{sec:driver}

The \textsf{childdoc} mechanism can also be use for the development
of definition files such as \LaTeX{} styles or classes.
This case differs from the above setup with multiple parts
included by |\include| in that no |\includeonly| should be invoked.
This can be achieved by starting the include file
(before |\ProvidesPackage|) with:
%
\begin{center}
\begin{tabular}{l}
|\input{childdoc.def}|\\
|\childdocforward{|\textit{main}|}|\\
\end{tabular}
\end{center}
%
or alternatively with:
%
\begin{center}
\begin{tabular}{l}
|\input{childdoc.def}|\\
|\childdocby{|\textit{main}|}|\\
\end{tabular}
\end{center}
%
Both forms have slightly different effects as described above.
The main file is prepared as usual, see \secref{sec:include}.

%%%%%%%%%%%%%%%%%%%%%%%%%%%%%%%%%%%%%%%%%%%%%%%%%%%%%%%%%%%%%%%%%%%%%%%%%%%%%%%%
\subsection{Legacy Detection}
\label{sec:detection}

The directive |\childdocmain| in the main file can detect
whether the complete document or merely a child is to be compiled
even without using the directive |\childdocof|.
This method is deprecated because it is less robust
and there is no compelling reason to use it;
it is merely provided for backward compatibility
and it may be removed in future versions.

If the detection mechanism is to be used,
it is mandatory to correctly specify
the filename of the main file as the argument of |\childdocmain|:
%
\begin{center}
\begin{tabular}{l}
|\input{childdoc.def}|\\
|\childdocmain{|\textit{main}|}|\\
\end{tabular}
\end{center}
%
If |\jobname| does not match the argument \textit{main} of |\childdocmain|,
it is assumed that |\jobname| points to the child file to be compiled.
When using |\childdocmain| with the main file specified as argument,
it suffices to start a child file
with just |\input{|\textit{main}|}|
without loading of the package and using |\childdocof|.
If instead all processing is done
with the appropriate \textsf{childdoc} directives,
the argument of \textit{main} of |\childdocmain| can be empty.

An alternative version of the command line processing described
in \secref{sec:commandline} using the detection mechanism reads:
%
\begin{center}
|... -jobname "|\textit{target}|" "|[\textit{flags}]%
[|\def\jobname{|\textit{dest}|}|]|\input{|\textit{main}|}"|
\end{center}

%%%%%%%%%%%%%%%%%%%%%%%%%%%%%%%%%%%%%%%%%%%%%%%%%%%%%%%%%%%%%%%%%%%%%%%%%%%%%%%%
\subsection{Manual Code}
\label{sec:manual}

In case one cannot be certain whether the definitions file |childdoc.def|
is installed on the target \TeX{} distribution
and one prefers not to ship it,
it is conceivable to paste a few relevant commands into the sources.

To that end, drop all statements |\input{childdoc.def}|
and perform the replacements as outlined below.
Instead of |\childdocmain{|\textit{main}|}| add the following code
to the top of the main file:
%
\begin{center}
\begin{tabular}{l}
|\||ifdefined\childdocname\endinput\||fi\newif\ifchilddoc|\\
|\edef\childdocname{\scantokens\expandafter{\jobname\noexpand}}|\\
|\def\childdocmain{|\textit{main}|}\||ifx\childdocmain\childdocname\||else|\\
|\childdoctrue\includeonly{\childdocname}\let\jobname\childdocmain\||fi|\\
\end{tabular}
\end{center}
%
Instead of |\childdocof{|\textit{main}|}| just include the main file
at the top of each child file:
%
\begin{center}
|\input{|\textit{main}|}|
\end{center}
%
A simple redirection |\childdocforward{|\textit{dest}|}| is achieved by:
%
\begin{center}
|\def\jobname{|\textit{dest}|}\input{\jobname}|
\end{center}
%
The redirection with prefix
|\childdocforwardprefix[|\textit{prefix}|]{|\textit{dest}|}|
is accomplished by:
%
\begin{center}
\begin{tabular}{l}
|{\edef\jobname{\scantokens\expandafter{\jobname\noexpand}}|\\
|\def\redirectjob |\textit{prefix}|#1~~~{\gdef\jobname{|\textit{dest}|#1}}|\\
|\expandafter\redirectjob\jobname~~~}\input{\jobname}|
\end{tabular}
\end{center}

In an alternative approach,
child documents can be compiled by a specific command line
without additional code or specific definitions:
%
\begin{center}
|... -jobname "|\textit{target}|" "|[\textit{flags}]%
|\includeonly{|\textit{dest}|}\input{|\textit{main}|}"|
\end{center}
%

%%%%%%%%%%%%%%%%%%%%%%%%%%%%%%%%%%%%%%%%%%%%%%%%%%%%%%%%%%%%%%%%%%%%%%%%%%%%%%%%
%%%%%%%%%%%%%%%%%%%%%%%%%%%%%%%%%%%%%%%%%%%%%%%%%%%%%%%%%%%%%%%%%%%%%%%%%%%%%%%%
\section{Information}

%%%%%%%%%%%%%%%%%%%%%%%%%%%%%%%%%%%%%%%%%%%%%%%%%%%%%%%%%%%%%%%%%%%%%%%%%%%%%%%%
\subsection{Copyright}

Copyright \copyright{} 2017--2018 Niklas Beisert

This work may be distributed and/or modified under the
conditions of the \LaTeX{} Project Public License, either version 1.3
of this license or (at your option) any later version.
The latest version of this license is in
  \url{http://www.latex-project.org/lppl.txt}
and version 1.3 or later is part of all distributions of \LaTeX{}
version 2005/12/01 or later.

This work has the LPPL maintenance status `maintained'.

The Current Maintainer of this work is Niklas Beisert.

This work consists of the files |README.txt|, |childdoc.ins| and |childdoc.dtx|
as well as the derived files |childdoc.def|, |cdocsamp.tex|
with |cdocsch1.tex|, |cdocsch2.tex|, |cdocspt3.tex|, |cdocspt4.tex|,
|cdocsdrf.tex|, |cdocsfn1.tex|, |cdocsfn2.tex|
as well as |childdoc.pdf|.

%%%%%%%%%%%%%%%%%%%%%%%%%%%%%%%%%%%%%%%%%%%%%%%%%%%%%%%%%%%%%%%%%%%%%%%%%%%%%%%%
\subsection{Files and Installation}

The package consists of the files:
%
\begin{center}
\begin{tabular}{ll}
    |README.txt|   & readme file \\
    |childdoc.ins| & installation file \\
    |childdoc.dtx| & source file \\
    |childdoc.def| & definition file \\
    |cdocsamp.tex| & sample main file \\
    |cdocsch1.tex| & sample include file \\
    |cdocsch2.tex| & sample include file \\
    |cdocspt3.tex| & sample part file \\
    |cdocspt4.tex| & sample part file \\
    |cdocsdrf.tex| & sample redirection file \\
    |cdocsfn1.tex| & sample redirection file \\
    |cdocsfn2.tex| & sample redirection file \\
    |childdoc.pdf| & manual
\end{tabular}
\end{center}
%
The distribution consists of the files
|README.txt|, |childdoc.ins| and |childdoc.dtx|.
%
\begin{itemize}
\item
Run (pdf)\LaTeX{} on |childdoc.dtx|
to compile the manual |childdoc.pdf| (this file).
\item
Run \LaTeX{} on |childdoc.ins| to create the definitions file |childdoc.def|
and the sample |cdocsamp.tex| with include files
|cdocsch1.tex|, |cdocsch2.tex|, |cdocspt3.tex|, |cdocspt4.tex|,
|cdocsdrf.tex|, |cdocsfn1.tex|, |cdocsfn2.tex|.
Then copy the file |childdoc.def| to an appropriate directory of your \LaTeX{}
distribution, e.g.\ \textit{texmf-root}|/tex/latex/childdoc|.
\end{itemize}

%%%%%%%%%%%%%%%%%%%%%%%%%%%%%%%%%%%%%%%%%%%%%%%%%%%%%%%%%%%%%%%%%%%%%%%%%%%%%%%%
\subsection{Related CTAN Packages}

There are several other packages which offer a similar functionality:
%
\begin{itemize}
\item
The packages
\href{http://ctan.org/pkg/docmute}{\textsf{docmute}},
\href{http://ctan.org/pkg/includex}{\textsf{includex}} and
\href{http://ctan.org/pkg/standalone}{\textsf{standalone}}
provide commands to include only the document body of
a child file thus allowing both files to be compiled individually.
\item
The packages \href{http://ctan.org/pkg/subdocs}{\textsf{subdocs}}
and \href{http://ctan.org/pkg/subfiles}{\textsf{subfiles}}
provide structures in which the main and child documents can be
encapsulated and allowing them to be compiled individually.
The inclusion mechanism is different from the conventional |\include|.
\item
The package \href{http://ctan.org/pkg/combine}{\textsf{combine}}
is an elaborate solution to combine several documents into one.
\end{itemize}
%
See also the CTAN topic \href{http://ctan.org/topic/subdocs}{\textsf{subdocs}}
for further related packages.
The present package differs from the above solutions in that
a document structure constructed with the conventional |\include| mechanism
just needs two extra commands at the top of every file
such that all constituent files can be compiled individually.

%%%%%%%%%%%%%%%%%%%%%%%%%%%%%%%%%%%%%%%%%%%%%%%%%%%%%%%%%%%%%%%%%%%%%%%%%%%%%%%%
%\subsection{Feature Suggestions}
%
%The following is a list of features which may be useful for future
%versions of this package:
%%
%\begin{itemize}
%\item
%\ldots
%\end{itemize}

%%%%%%%%%%%%%%%%%%%%%%%%%%%%%%%%%%%%%%%%%%%%%%%%%%%%%%%%%%%%%%%%%%%%%%%%%%%%%%%%
\subsection{Revision History}

%%%%%%%%%%%%%%%%%%%%%%%%%%%%%%%%%%%%%%%%
\paragraph{v2.0:} 2018/12/30

\begin{itemize}
\item
immediate forward processing
\item
added |\childdocby| mechanism
\item
manual restructured
\end{itemize}

%%%%%%%%%%%%%%%%%%%%%%%%%%%%%%%%%%%%%%%%
\paragraph{v1.6:} 2018/01/17

\begin{itemize}
\item
application for development of include files
\item
corrections to manual
\end{itemize}

%%%%%%%%%%%%%%%%%%%%%%%%%%%%%%%%%%%%%%%%
\paragraph{v1.5:} 2017/05/21

\begin{itemize}
\item
more complete structuring introduced
\item
|\childdocof| introduced
\item
|\childdoc| renamed to |\childdocmain|
\item
|\childredirect| renamed to |\childdocforward| and |\childdocforwardprefix|
and functionality expanded
\end{itemize}

%%%%%%%%%%%%%%%%%%%%%%%%%%%%%%%%%%%%%%%%
\paragraph{v1.0:} 2017/04/27

\begin{itemize}
\item
manual and install package
\item
first version published on CTAN
\end{itemize}

%%%%%%%%%%%%%%%%%%%%%%%%%%%%%%%%%%%%%%%%
\paragraph{v0.6:} 2017/04/26

\begin{itemize}
\item
redirection mechanism added
\end{itemize}

%%%%%%%%%%%%%%%%%%%%%%%%%%%%%%%%%%%%%%%%
\paragraph{v0.5:} 2017/04/26

\begin{itemize}
\item
functionality in definition file
\end{itemize}


%%%%%%%%%%%%%%%%%%%%%%%%%%%%%%%%%%%%%%%%%%%%%%%%%%%%%%%%%%%%%%%%%%%%%%%%%%%%%%%%
%%%%%%%%%%%%%%%%%%%%%%%%%%%%%%%%%%%%%%%%%%%%%%%%%%%%%%%%%%%%%%%%%%%%%%%%%%%%%%%%
%%%%%%%%%%%%%%%%%%%%%%%%%%%%%%%%%%%%%%%%%%%%%%%%%%%%%%%%%%%%%%%%%%%%%%%%%%%%%%%%
\appendix

\settowidth\MacroIndent{\rmfamily\scriptsize 000\ }

 \DocInput{childdoc.dtx}

\end{document}
%</driver>
% \fi
%
% %%%%%%%%%%%%%%%%%%%%%%%%%%%%%%%%%%%%%%%%%%%%%%%%%%%%%%%%%%%%%%%%%%%%%%%%%%%%%%
% %%%%%%%%%%%%%%%%%%%%%%%%%%%%%%%%%%%%%%%%%%%%%%%%%%%%%%%%%%%%%%%%%%%%%%%%%%%%%%
% \section{Sample}
%\iffalse
%<*samplemain>
%\fi
%
% The following presents a sample document
% with two chapters, two parts, a title page,
% a compile flag as well as three forwarding files to set the flag.
% It consists of eight |.tex| files:
% \begin{center}
% \begin{tabular}{ll}
% |cdocsamp.tex|&main file\\
% |cdocsch1.tex|&include file for chapter 1\\
% |cdocsch2.tex|&include file for chapter 2\\
% |cdocspt3.tex|&include file for part 3\\
% |cdocspt4.tex|&include file for part 4\\
% |cdocsdrf.tex|&forwarding file for main file in draft mode\\
% |cdocsfi1.tex|&forwarding file for final version of chapter 1\\
% |cdocsfi2.tex|&forwarding file for final version of chapter 2\\
% \end{tabular}
% \end{center}
% Each of the eight files can be compiled directly by the \LaTeX{} compiler.
%
% %%%%%%%%%%%%%%%%%%%%%%%%%%%%%%%%%%%%%%
% \paragraph{Main File.}
%
% The main file is called |cdocsamp.tex|.
%
% Load the \textsf{childdoc} definitions and
% declare the filename for the main document:
%    \begin{macrocode}
\input{childdoc.def}
\childdocmain{}
%    \end{macrocode}

% Optional override for |\version| flag:
%    \begin{macrocode}
%%\ifchilddoc\else\providecommand{\version}{draft}\fi
%    \end{macrocode}

% Define the default values for the |\version| flag
% (|final| for the main file and |draft| for childs):
%    \begin{macrocode}
\ifchilddoc
\providecommand{\version}{draft}
\else
\providecommand{\version}{final}
\fi
%    \end{macrocode}

% Load the standard document class:
%    \begin{macrocode}
\documentclass[12pt]{article}
%    \end{macrocode}

% Start the document body:
%    \begin{macrocode}
\begin{document}
%    \end{macrocode}

% Declare a title page.
% Print title, part of document being processed and version flag:
%    \begin{macrocode}
\addtocounter{page}{-1}
\begin{center}
{\LARGE\bfseries{}childdoc example\par}
\vspace{1cm}
\ifchilddoc
\ifchilddocmanual part\else chapter\fi:
`\childdocname' of `\childdocjob'\par
\else
main document: `\childdocjob'\par
\fi
version: \version\par
\end{center}
\newpage
%    \end{macrocode}

% Manually include selected file,
% otherwise process as usual:
%    \begin{macrocode}
\ifchilddocmanual
\section*{part `\childdocname'}
\input{\childdocname}
\else
%    \end{macrocode}

% Include the two chapters:
%    \begin{macrocode}
\include{cdocsch1}
\include{cdocsch2}
%    \end{macrocode}

% Include the two parts unless only chapters should be displayed:
%    \begin{macrocode}
\ifchilddoc\else
\section{part three}
\input{cdocspt3}
\section{part four}
\input{cdocspt4}
\fi
%    \end{macrocode}

% Process as usual until here:
%    \begin{macrocode}
\fi
%    \end{macrocode}

% End of document body:
%    \begin{macrocode}
\end{document}
%    \end{macrocode}
%\iffalse
%</samplemain>
%\fi
%
% %%%%%%%%%%%%%%%%%%%%%%%%%%%%%%%%%%%%%%
% \paragraph{Chapter Include Files.}
%
% The include files are called |cdocsch1.tex| and |cdocsch2.tex|.
%
%\iffalse
%<*samplechap1|samplechap2>
%\fi

% Optional override for |\version| flag:
%    \begin{macrocode}
%%\providecommand{\version}{final}
%    \end{macrocode}

% Include the main document:
%    \begin{macrocode}
\input{childdoc.def}
\childdocof{cdocsamp}
%    \end{macrocode}

%\iffalse
%</samplechap1|samplechap2>
%\fi
%
%\iffalse
%<*samplechap1>
%\fi
% Some text for chapter 1:
%    \begin{macrocode}
\section{one}
some text in chapter one
%    \end{macrocode}

%\iffalse
%</samplechap1>
%\fi
% Some text for chapter 2:
%\iffalse
%<*samplechap2>
%\fi
%    \begin{macrocode}
\section{two}
more text in chapter two
%    \end{macrocode}

%\iffalse
%</samplechap2>
%\fi
%
% %%%%%%%%%%%%%%%%%%%%%%%%%%%%%%%%%%%%%%
% \paragraph{Part Include Files.}
%
% The include files are called |cdocspt3.tex| and |cdocspt4.tex|.
%
%\iffalse
%<*samplepart3|samplepart4>
%\fi

% Optional override for |\version| flag:
%    \begin{macrocode}
%%\providecommand{\version}{final}
%    \end{macrocode}

% Include the main document:
%    \begin{macrocode}
\input{childdoc.def}
\childdocby{cdocsamp}
%    \end{macrocode}

%\iffalse
%</samplepart3|samplepart4>
%\fi
%
%\iffalse
%<*samplepart3>
%\fi
% Some text for part 3:
%    \begin{macrocode}
some text in part three
%    \end{macrocode}

%\iffalse
%</samplepart3>
%\fi
% Some text for part 4:
%\iffalse
%<*samplepart4>
%\fi
%    \begin{macrocode}
more text in part four
%    \end{macrocode}

%\iffalse
%</samplepart4>
%\fi
%
% %%%%%%%%%%%%%%%%%%%%%%%%%%%%%%%%%%%%%%
% \paragraph{Forwarding for a Complete Draft.}
%
% The following forwarding file |cdocsdrf.tex|
% compiles the main document in draft mode:
%\iffalse
%<*sampledraft>
%\fi
%    \begin{macrocode}
\def\version{draft}
\input{childdoc.def}
\childdocforward{cdocsamp}
%    \end{macrocode}

%\iffalse
%</sampledraft>
%\fi
%
% %%%%%%%%%%%%%%%%%%%%%%%%%%%%%%%%%%%%%%
% \paragraph{Forwarding for Final Version of the Chapters.}
%
% The following forwarding files |cdocsfn1.tex| and |cdocsfn2.tex|
% (with identical content)
% compile the final versions of the child documents
% |cdocsch1.tex| and |cdocsch2.tex|, respectively:
%\iffalse
%<*samplefinal>
%\fi
%    \begin{macrocode}
\def\version{final}
\input{childdoc.def}
\childdocforwardprefix[cdocsamp]{cdocsfn}{cdocsch}
%    \end{macrocode}

%\iffalse
%</samplefinal>
%\fi
%
% %%%%%%%%%%%%%%%%%%%%%%%%%%%%%%%%%%%%%%
% \paragraph{Command Line Processing.}
%
% The following three command lines generate the output files
% |cdocscld|, |cdocscl1| and |cdocscl2|
% which should be identical to
% |cdocsdrf|, |cdocsch1| and |cdocsfn2|, respectively:
% \begin{center}
% \begin{tabular}{l}
% |latex -jobname cdocscld \|\\
% |  "\def\version{draft}\input{childdoc.def}\childdocforward{cdocsamp}"|\\
% |latex -jobname cdocscl1 \|\\
% |  "\input{childdoc.def}\childdocforward[cdocsamp]{cdocsch1}"|\\
% |latex -jobname cdocscl2 \|\\
% |  "\def\version{final}\input{childdoc.def}\childdocforward{cdocsch2}"|
% \end{tabular}
% \end{center}
% Note that the trailing backslash on each first line
% merely continues the input to the second line
% (for convenient cut ant paste).
% Furthermore, the command |latex| can be replaced by any
% of its alternative versions such as |pdflatex|.
%
% %%%%%%%%%%%%%%%%%%%%%%%%%%%%%%%%%%%%%%%%%%%%%%%%%%%%%%%%%%%%%%%%%%%%%%%%%%%%%%
% %%%%%%%%%%%%%%%%%%%%%%%%%%%%%%%%%%%%%%%%%%%%%%%%%%%%%%%%%%%%%%%%%%%%%%%%%%%%%%
% \section{Implementation}
%\iffalse
%<*package>
%\fi
%
% This section describes the definitions file |childdoc.def|.

% The definitions cannot be loaded using |\usepackage| or |\RequirePackage|
% which has a mechanism to prevent loading a style file more than once.
% When loading the definitions by means of |\input|
% multiple instances have to be prevented manually:
%\iffalse
%This code needs to be before the `\ProvidesFile' directive
%which is defined at the beginning of this file.
%Therefore it is also placed there and commented out here.
%</package>
%<*discard>
%\fi
%    \begin{macrocode}
\ifdefined\childdocmain\endinput\fi
%    \end{macrocode}
%\iffalse
%</discard>
%<*package>
%\fi
%
% \macro{\ifchilddoc}
% \macro{\ifchilddocmanual}
% The conditional |\ifchilddoc| tells whether a
% child (true) or main (false) document is being compiled.
% The conditional |\ifchilddocmanual| tells whether
% the |\includeonly| mechanism is used (false) or
% the selection of child files must be performed manually (true).
% The definitions initialise to false:
%    \begin{macrocode}
\newif\ifchilddoc
\newif\ifchilddocmanual
%    \end{macrocode}

% \macro{\childdocname}
% \macro{\childdocjob}
% The macro |\childdocname| stores the name of the main document
% to be compiled. The macro |\childdocjob| stores the name of
% the document on which the \LaTeX{} compiler was originally invoked.
% The content of |\jobname| cannot be compared
% to filenames specified in the source due to different catcodes.
% The following code rescans |\jobname|, stores the result
% in |\childdocname| and saves a copy in |\childdocjob|:
%    \begin{macrocode}
\edef\childdocname{\scantokens\expandafter{\jobname\noexpand}}
\let\childdocjob\childdocname
%    \end{macrocode}

% \macro{\childdocdisable}
% The macro |\childdocdisable| prevents the main file
% from being processed more than once.
% At this stage, the main document command |\childdocmain|
% is assumed to be called once again where it should do nothing.
% Any subsequent call to it should prevent
% a secondary processing of the main document
% It overwrites the forwarding commands
% |\childdocof| and |\childdocforward|
% with empty macros to prevent further inclusions of the main document:
%    \begin{macrocode}
\newcommand{\childdocdisable}
{
  \renewcommand{\childdocmain}[1]{\renewcommand{\childdocmain}[1]{\endinput}}
  \renewcommand{\childdocof}[1]{}
  \renewcommand{\childdocby}[2][]{}
  \renewcommand{\childdocforward}[2][]{}
  \renewcommand{\childdocdisable}{}
}
%    \end{macrocode}

% \macro{\childdocmain}
% The macro |\childdocmain| is to be called at the top of the main file
% with nothing or the main filename (without extension) as argument.
% First, it breaks loops.
% If the argument is not empty and does not match |\childdocname|
% (which is set by the first inclusion of |childdoc.def|),
% |\ifchilddoc| is set to true, |\includeonly| is applied to the child file
% and |\jobname| is set to the main file
% (for proper handling of |.aux| files):
%    \begin{macrocode}
\newcommand{\childdocmain}[1]
{
  \childdocdisable\childdocmain{}
  \if?#1?\else
    \begingroup
      \def\childdoctmp{#1}
      \ifx\childdoctmp\childdocname
        \def\childdoctmp{}
      \else
        \def\childdoctmp
        {
          \childdoctrue
          \includeonly{\childdocname}
          \def\childdocjob{#1}
          \def\jobname{#1}
        }
      \fi
      \expandafter
    \endgroup
    \childdoctmp
  \fi
}
%    \end{macrocode}

% \macro{\childdocof}
% The command |\childdocof| redirects
% compilation to the main file |#1|.
%    \begin{macrocode}
\newcommand{\childdocof}[1]
{
  \childdocdisable
  \childdoctrue
  \includeonly{\childdocname}
  \def\jobname{#1}
  \def\childdocjob{#1}
  \input{#1}
}
%    \end{macrocode}

% \macro{\childdocby}
% The command |\childdocby| ....
%    \begin{macrocode}
\newcommand{\childdocby}[2][]
{
  \childdocdisable
  \childdoctrue
  \childdocmanualtrue
  \if?#1?\else
    \def\jobname{#2}
  \fi
  \def\childdocjob{#2}
  \input{#2}
  \endinput
}
%    \end{macrocode}

% \macro{\childdocforward}
% The command |\childdocforward| redirects
% compilation to the main file or
% (if the optional argument is given) a child file.
% Parameters are set as if the main file
% or a child file starting with |\childdocof| was compiled.
% Then compilation is handed over to the main file:
%    \begin{macrocode}
\newcommand{\childdocforward}[2][]
{
  \begingroup
    \if?#1?
      \def\childdoctmp
      {
        \def\childdocname{#2}
        \def\childdocjob{#2}
        \def\jobname{#2}
        \input{#2}
        \endinput
      }
    \else
      \def\childdoctmp
      {
        \childdocdisable
        \def\childdocname{#2}
        \childdoctrue
        \includeonly{#2}
        \def\childdocjob{#1}
        \def\jobname{#1}
        \input{#1}
        \endinput
      }
    \fi
    \expandafter
  \endgroup
  \childdoctmp
}
%    \end{macrocode}

% \macro{\childdocforwardprefix}
% The command |\childdocforwardprefix| redirects
% compilation to the main or a child file by means of a pattern.
% The prefix |#1| in the current filename is replaced by |#2|
% and the suffix of the current filename is kept
% (it is assumed that the filename does not contain the substring `|~~~|'
% which is used as a delimiter).
% Compilation is handed over to the new file by |\childdocforward|:
%    \begin{macrocode}
\newcommand{\childdocforwardprefix}[3][]
{
  \begingroup
    \def\childdocextract #2##1~~~{\def\childdoctmp{\childdocforward[#1]{#3##1}}}
    \expandafter\childdocextract\childdocname~~~
    \expandafter
  \endgroup
  \childdoctmp
}
%    \end{macrocode}

% \macro{\childdoc}
% The deprecated macro |\childdoc| is a legacy version of |\childdocmain|:
%    \begin{macrocode}
\newcommand{\childdoc}{\childdocmain}
%    \end{macrocode}

% \macro{\childdocredirect}
% The deprecated macro |\childdocredirect| is a legacy version
% of |\childdocforward| and |\childdocforwardprefix|:
%    \begin{macrocode}
\newcommand{\childdocredirect}[2][]
{
  \begingroup
    \if?#1?
      \def\childdoctmp{\childdocforward{#2}}
    \else
      \def\childdoctmp{\childdocforwardprefix{#1}{#2}}
    \fi
    \expandafter
  \endgroup
  \childdoctmp
}
%    \end{macrocode}

%\iffalse
%</package>
%\fi
%
\endinput

\childdocmain{}
%    \end{macrocode}

% Optional override for |\version| flag:
%    \begin{macrocode}
%%\ifchilddoc\else\providecommand{\version}{draft}\fi
%    \end{macrocode}

% Define the default values for the |\version| flag
% (|final| for the main file and |draft| for childs):
%    \begin{macrocode}
\ifchilddoc
\providecommand{\version}{draft}
\else
\providecommand{\version}{final}
\fi
%    \end{macrocode}

% Load the standard document class:
%    \begin{macrocode}
\documentclass[12pt]{article}
%    \end{macrocode}

% Start the document body:
%    \begin{macrocode}
\begin{document}
%    \end{macrocode}

% Declare a title page.
% Print title, part of document being processed and version flag:
%    \begin{macrocode}
\addtocounter{page}{-1}
\begin{center}
{\LARGE\bfseries{}childdoc example\par}
\vspace{1cm}
\ifchilddoc
\ifchilddocmanual part\else chapter\fi:
`\childdocname' of `\childdocjob'\par
\else
main document: `\childdocjob'\par
\fi
version: \version\par
\end{center}
\newpage
%    \end{macrocode}

% Manually include selected file,
% otherwise process as usual:
%    \begin{macrocode}
\ifchilddocmanual
\section*{part `\childdocname'}
\input{\childdocname}
\else
%    \end{macrocode}

% Include the two chapters:
%    \begin{macrocode}
\include{cdocsch1}
\include{cdocsch2}
%    \end{macrocode}

% Include the two parts unless only chapters should be displayed:
%    \begin{macrocode}
\ifchilddoc\else
\section{part three}
\input{cdocspt3}
\section{part four}
\input{cdocspt4}
\fi
%    \end{macrocode}

% Process as usual until here:
%    \begin{macrocode}
\fi
%    \end{macrocode}

% End of document body:
%    \begin{macrocode}
\end{document}
%    \end{macrocode}
%\iffalse
%</samplemain>
%\fi
%
% %%%%%%%%%%%%%%%%%%%%%%%%%%%%%%%%%%%%%%
% \paragraph{Chapter Include Files.}
%
% The include files are called |cdocsch1.tex| and |cdocsch2.tex|.
%
%\iffalse
%<*samplechap1|samplechap2>
%\fi

% Optional override for |\version| flag:
%    \begin{macrocode}
%%\providecommand{\version}{final}
%    \end{macrocode}

% Include the main document:
%    \begin{macrocode}
% \iffalse
%
% childdoc.dtx Copyright (C) 2017-2018 Niklas Beisert
%
% This work may be distributed and/or modified under the
% conditions of the LaTeX Project Public License, either version 1.3
% of this license or (at your option) any later version.
% The latest version of this license is in
%   http://www.latex-project.org/lppl.txt
% and version 1.3 or later is part of all distributions of LaTeX
% version 2005/12/01 or later.
%
% This work has the LPPL maintenance status `maintained'.
%
% The Current Maintainer of this work is Niklas Beisert.
%
% This work consists of the files childdoc.dtx and childdoc.ins
% and the derived files childdoc.def and cdocsamp.tex with
% cdocsch1.tex, cdocsch2.tex, cdocsdrf.tex, cdocsfn1.tex, cdocsfn2.tex.
%
%<package>\ifdefined\childdocmain\endinput\fi
%<package>\ProvidesFile{childdoc.def}[2018/12/30 v2.0 child document driver]
%<samplemain>\ProvidesFile{cdocsamp.tex}[2018/12/30 v2.0 sample for childdoc]
%<*driver>
%\ProvidesFile{childdoc.drv}[2018/12/30 v2.0 childdoc reference manual file]
\PassOptionsToClass{10pt,a4paper}{article}
\documentclass{ltxdoc}

\usepackage[margin=35mm]{geometry}
\usepackage{hyperref}
\usepackage{hyperxmp}
\usepackage[usenames]{color}

\hypersetup{colorlinks=true}
\hypersetup{pdfstartview=FitH}
\hypersetup{pdfpagemode=UseNone}
\hypersetup{pdfsource={}}
\hypersetup{pdflang={en-UK}}
\hypersetup{pdfcopyright={Copyright 2017-2018 Niklas Beisert.
  This work may be distributed and/or modified under the
  conditions of the LaTeX Project Public License, either version 1.3
  of this license or (at your option) any later version.}}
\hypersetup{pdflicenseurl={http://www.latex-project.org/lppl.txt}}
\hypersetup{pdfcontactaddress={ETH Zurich, ITP, HIT K,
  Wolfgang-Pauli-Strasse 27}}
\hypersetup{pdfcontactpostcode={8093}}
\hypersetup{pdfcontactcity={Zurich}}
\hypersetup{pdfcontactcountry={Switzerland}}
\hypersetup{pdfcontactemail={nbeisert@itp.phys.ethz.ch}}
\hypersetup{pdfcontacturl={http://people.phys.ethz.ch/\xmptilde nbeisert/}}

\newcommand{\secref}[1]{\hyperref[#1]{section \ref*{#1}}}

\parskip1ex
\parindent0pt
\let\olditemize\itemize
\def\itemize{\olditemize\parskip0pt}

\begin{document}

\title{The \textsf{childdoc} Package}
\hypersetup{pdftitle={The childdoc Package}}
\author{Niklas Beisert\\[2ex]
  Institut f\"ur Theoretische Physik\\
  Eidgen\"ossische Technische Hochschule Z\"urich\\
  Wolfgang-Pauli-Strasse 27, 8093 Z\"urich, Switzerland\\[1ex]
  \href{mailto:nbeisert@itp.phys.ethz.ch}
  {\texttt{nbeisert@itp.phys.ethz.ch}}}
\hypersetup{pdfauthor={Niklas Beisert}}
\hypersetup{pdfsubject={Manual for the LaTeX2e Package childdoc}}
\date{30 December 2018, \textsf{v2.0}}
\maketitle

\begin{abstract}\noindent
\textsf{childdoc} is a \LaTeXe{} package
that enables the direct compilation
of document sections included by |\include|
to individual files.
\end{abstract}

\begingroup
\parskip0ex
\tableofcontents
\endgroup

%%%%%%%%%%%%%%%%%%%%%%%%%%%%%%%%%%%%%%%%%%%%%%%%%%%%%%%%%%%%%%%%%%%%%%%%%%%%%%%%
%%%%%%%%%%%%%%%%%%%%%%%%%%%%%%%%%%%%%%%%%%%%%%%%%%%%%%%%%%%%%%%%%%%%%%%%%%%%%%%%
\section{Introduction}

\LaTeX{} provides a mechanism to structure a large document (such as a book)
into a main file and several child files (containing the chapters)
using the |\include| command.
This mechanism is beneficial for documents
which span hundreds of pages in order to
make the source file(s) more manageable.
Moreover, compilation can be restricted to
selected child files by means of the |\includeonly| command.
The latter feature can be used to reduce the compilation time while editing
(this was significantly more useful in the earlier days of \LaTeX{})
or to generate a smaller document which is easier to navigate.
Another application of |\includeonly| is to generate
documents consisting of selected parts of the complete document.

However, there are a few drawbacks of the plain |\include| mechanism:
\begin{itemize}
\item
The child files cannot be compiled on their own,
they can only be compiled via the main file.
A naive editing environment
(such as a text editor with an option
to have the current file processed by \LaTeX)
may require one to switch to the main file before compiling;
attempting to compile the child file produces errors.
\item
The main file must be modified (each time)
to adjust the |\includeonly| command
to the present needs. This easily leaves the main file in a messy state.
\item
The generated document will always carry the filename
of the main document. This is inconvenient if
several child files are to be compiled and
to be kept for distribution.
\end{itemize}

The present package provides a simple interface
to make child files individually compilable by \LaTeX{}.
Compiling a child file then has the same effect as compiling
the main file with an |\includeonly| command
to select the appropriate child.
Moreover the generated document will carry the name of the child
rather than the main file.
This resolves all three above issues.

This feature is meant to make the editing of books,
thesis documents and lecture notes somewhat more convenient.
However, the package can also be used efficiently for
composing a series of documents (such as exercise sheets)
which are typically distributed individually.
It then assists the author in generating the individual documents
(potentially in different versions)
as well as a document containing the collected series.
Another application is in developing style files
or other kinds of included material
where compilation of the style file could redirect
to a sample or test file.

%%%%%%%%%%%%%%%%%%%%%%%%%%%%%%%%%%%%%%%%%%%%%%%%%%%%%%%%%%%%%%%%%%%%%%%%%%%%%%%%
%%%%%%%%%%%%%%%%%%%%%%%%%%%%%%%%%%%%%%%%%%%%%%%%%%%%%%%%%%%%%%%%%%%%%%%%%%%%%%%%
\section{Usage}

First of all, the package \textsf{childdoc} is \emph{not} a standard
\LaTeXe{} |.sty| style file! Therefore it needs to be invoked in
a non-standard way.

%%%%%%%%%%%%%%%%%%%%%%%%%%%%%%%%%%%%%%%%%%%%%%%%%%%%%%%%%%%%%%%%%%%%%%%%%%%%%%%%
\subsection{Included Files}
\label{sec:include}

%%%%%%%%%%%%%%%%%%%%%%%%%%%%%%%%%%%%%%%%
\DescribeMacro{\childdocmain}
To use the package, add the commands
\begin{center}
\begin{tabular}{l}
|\input{childdoc.def}|\\
|\childdocmain{}|\\
\end{tabular}
\end{center}
at the very top of the main \LaTeX{} file,
in particular \emph{before} the |\documentclass| statement!
The argument of |\childdocmain| should be left empty
(but it must be present).

%%%%%%%%%%%%%%%%%%%%%%%%%%%%%%%%%%%%%%%%
\DescribeMacro{\childdocof}
Furthermore, add the commands
\begin{center}
\begin{tabular}{l}
|\input{childdoc.def}|\\
|\childdocof{|\textit{main}|}|\\
\end{tabular}
\end{center}
at the top of every child file \textit{child}
which is included by |\include{|\textit{child}|}|
from within the main file
(or at least for those files to be compiled individually).
The argument \textit{main} must be the filename of the main file.

There are a couple of
considerations in setting up the main and child documents:

%%%%%%%%%%%%%%%%%%%%%%%%%%%%%%%%%%%%%%%%
\paragraph{Restrictions.}

Please note the following restrictions:
\begin{itemize}
\item
|\childdocmain| must be called with one argument \textit{main}
to ensure compatibility with earlier version of the package.
It must either be empty (|\childdocmain{}|)
or precisely match the filename of the main file in which it is specified.
See \secref{sec:detection} for further information.
\item
The filename \textit{main} must be specified without the |.tex| extension.
\item
The filename \textit{main} is case sensitive
(even in case-insensitive file systems)
due to internal string comparison.
\item
The argument \textit{main} should be fully expanded, it cannot be a macro.
\item
Subdirectories and special characters should be avoided in filenames.
\item
The command |\childdocmain{|\textit{main}|}| must be followed by a whitespace.
It should not be followed immediately by another command
or by a comment mark `|%|'.
This is because the \TeX{} parser reads the token immediately following
the argument of |\childdocmain| and puts it
at the beginning of every child section;
however, a white\-space is ignored.
\end{itemize}

%%%%%%%%%%%%%%%%%%%%%%%%%%%%%%%%%%%%%%%%
\paragraph{Content of Main File.}

It is advisable to place all content in the child files included by |\include|.
Any output contained in the main file will appear in all child documents
unless suppressed manually;
it cannot be suppressed automatically by the |\includeonly| directive
and thus should normally be avoided.
A method to include some content in the main file
by means of conditional processing is described in \secref{sec:conditional}.

%%%%%%%%%%%%%%%%%%%%%%%%%%%%%%%%%%%%%%%%
\paragraph{Page Numbering.}

When only a part of the document is compiled,
the appropriate numbering of pages
(as well as other status parameters)
is determined from the |.aux| files.
The latter contain information from previous passes.
However this information needs to propagate through
all intermediate child documents.
Therefore the page numbering in child documents may well
be inconsistent until the complete document is compiled at least once.

A useful (if unconventional) way to always ensure a consistent
page numbering is to restart the numbering in each child document
and denote the pages by `\textit{child}|.|\textit{page}'
where \textit{child} represents the chapter/section number of the child file.
This can be achieved by the command
|\numberwithin{page}{|\textit{child}|}|
of the \textsf{amsmath} package
where \textit{child} can be |chapter| or |section|
depending on the chosen structuring.
Alternatively, one can modify the macro |\thepage| appropriately
and reset the counter |page| at the start of each child file.

%%%%%%%%%%%%%%%%%%%%%%%%%%%%%%%%%%%%%%%%%%%%%%%%%%%%%%%%%%%%%%%%%%%%%%%%%%%%%%%%
\subsection{Conditional Processing}
\label{sec:conditional}

The package provides a mechanism to compile different versions
of a document. To customise the versions further some conditional processing
can come in handy to distinguish which version is being compiled.
The package provides two macros to describe the compilation context:

%%%%%%%%%%%%%%%%%%%%%%%%%%%%%%%%%%%%%%%%
\DescribeMacro{\ifchilddoc}
The conditional |\ifchilddoc| distinguishes between the compilation of
child documents and the main document:
%
\begin{center}
|\ifchilddoc |\textit{child-code}| |[|\||else |\textit{main-code}]| \||fi|
\end{center}

%%%%%%%%%%%%%%%%%%%%%%%%%%%%%%%%%%%%%%%%
\DescribeMacro{\childdocname}
\DescribeMacro{\childdocjob}
The macro |\childdocname| contains the filename (without extension)
of the main or child file being processed.
Note that |\childdocjob| will always contain the name of the main file.

%%%%%%%%%%%%%%%%%%%%%%%%%%%%%%%%%%%%%%%%
\paragraph{Title Page.}

Conditional processing can be used to include a title or banner page
in the main document when proper precautions are taken.
Importantly, the code in the main file should ensure that the page counter
(as well as other status parameters which are stored in the |.aux| files)
takes the same value after the conditional processing.
Otherwise the page numbers may take divergent values
depending on which part is compiled.

For example, a title page could be declared by:
%
\begin{center}
\begin{tabular}{l}
|\ifchilddoc\||else|\\
|\addtocounter{page}{-1}|\\
\textit{code for title page}\\
|\newpage|\\
|\||fi|
\end{tabular}
\end{center}
%
A banner page for the child documents can be generated by:
%
\begin{center}
\begin{tabular}{l}
|\ifchilddoc|\\
|\addtocounter{page}{-1}|\\
\textit{code for banner page}\\
|\newpage|\\
|\||fi|
\end{tabular}
\end{center}
%
Here one could write a message such as:
\begin{center}
|This is the part \childdocname{} of \childdocjob{}.|
\end{center}

%%%%%%%%%%%%%%%%%%%%%%%%%%%%%%%%%%%%%%%%%%%%%%%%%%%%%%%%%%%%%%%%%%%%%%%%%%%%%%%%
\subsection{Flags}
\label{sec:flags}

The package makes it easy to generate different versions
of the main or child documents.
To this end compilation flags can be defined
and assigned different default values.
They will be particularly useful in conjunction
with the forwarding mechanism described in \secref{sec:forward}.

For example, it may be useful to have a flag |\version|
which can be set to |draft| or |final|.
The document source will contain some conditional code
depending on the value of |\version|.
Suppose further, the flag should default to |final| for the main file
and to |draft| for child files
which is a natural assignment for editing the document.
This is achieved by placing the following code
in the preamble of the main document
(below the |\childdocmain| directive):
%
\begin{center}
\begin{tabular}{l}
|\ifchilddoc|\\
|\providecommand{\version}{draft}|\\
|\||else|\\
|\providecommand{\version}{final}|\\
|\||fi|
\end{tabular}
\end{center}
%
The definition by |\providecommand| makes sure
that previous definitions are not overwritten.
Further statements |\providecommand{\version}{...}|
can thus be added before the above code to override it.

For the main file, one might add a line
(between |\childdocmain| and the above block)
%
\begin{center}
|%\ifchilddoc\||else\providecommand{\version}{draft}\||fi|
\end{center}
%
which can be uncommented to produce a draft version.
Likewise one can add a line to the very top of a child file
(above the |\childdocof{|\textit{main}|}| directive)
%
\begin{center}
|%\providecommand{\version}{final}|
\end{center}
%
which can be uncommented to produce the final version of this child document.

%%%%%%%%%%%%%%%%%%%%%%%%%%%%%%%%%%%%%%%%%%%%%%%%%%%%%%%%%%%%%%%%%%%%%%%%%%%%%%%%
\subsection{Forwarding}
\label{sec:forward}

Different versions of the main or child documents
using compilation flags as described in \secref{sec:flags}
can be (permanently) stored in different files
for convenient compilation, viewing and distribution.
To this end, the package defines a command
to pass on compilation to a different file:

%%%%%%%%%%%%%%%%%%%%%%%%%%%%%%%%%%%%%%%%
\DescribeMacro{\childdocforward}
The command |\childdocforward| redirects processing to
another source file:
%
\begin{center}
\begin{tabular}{l}
|\input{childdoc.def}|\\
|\childdocforward[|\textit{main}|]{|\textit{dest}|}|\\
\end{tabular}
\end{center}
%
The argument \textit{dest} is the destination file
(without extension).
It should be the main file or one of the child files.
Note that further \textsf{childdoc} directives
such as |\childdocof| and |\childdocforward|
in the indicated file will be processed in this form.
The optional argument \textit{main}
passes on directly to the main file \textit{main}
while pretending to compile the child \textit{dest}.
This form behaves as if \textit{dest}
issues |\childdocof{|\textit{main}|}| right away,
and no further \textsf{childdoc} directives will be processed.

%%%%%%%%%%%%%%%%%%%%%%%%%%%%%%%%%%%%%%%%
\DescribeMacro{\...prefix}
In the alternative form |\childdocforwardprefix|,
%
\begin{center}
\begin{tabular}{l}
|\input{childdoc.def}|\\
|\childdocforwardprefix[|\textit{main}|]{|\textit{prefix}|}{|\textit{dest}|}|
\end{tabular}
\end{center}
%
the destination file is determined by a pattern
depending on the current file:
To make this work, the current file must be called
`{\textit{prefix}\hspace{0.2em}\textit{suffix}}'
with \textit{prefix} matching precisely the argument.
Processing is then passed on to the file
`{\textit{dest}\hspace{0.2em}\textit{suffix}}'.
Surely, the same effect is achieved by
directly specifying the
argument `{\textit{dest}\hspace{0.2em}\textit{suffix}}'
in the first form.
However, that requires to set up a different file
for each child. With the alternative form of the command
all these files can have exactly the same content
which simplifies setting them up and maintaining them.

For example, the following file |draft.tex|
with a compilation flag |\version| as described in \secref{sec:flags}
compiles the main document as a draft:
%
\begin{center}
\begin{tabular}{l}
|\def\version{draft}|\\
|\input{childdoc.def}|\\
|\childdocforward{|\textit{main}|}|
\end{tabular}
\end{center}
%
Likewise, the following files |final|\textit{nn}|.tex|
compile the final version of the child document
|child|\textit{nn}|.tex|:
%
\begin{center}
\begin{tabular}{l}
|\def\version{final}|\\
|\input{childdoc.def}|\\
|\childdocforwardprefix{final}{child}|
\end{tabular}
\end{center}
%

Note that when several versions of a main file and/or of each child file
are to be generated, it may be convenient to set up a |Makefile| or
shell script to automatise the process.

%%%%%%%%%%%%%%%%%%%%%%%%%%%%%%%%%%%%%%%%%%%%%%%%%%%%%%%%%%%%%%%%%%%%%%%%%%%%%%%%
\subsection{Command Line Processing}
\label{sec:commandline}

The effect of redirection files can also be achieved by invoking
the \LaTeX{} compiler with a more elaborate command line.
Most conveniently this should be done as part
of a shell script or a |Makefile|.

When using \textsf{childdoc} in the main file, the following
command lines effectively perform a redirection
(note that depending on the shell being used,
backslashes may have to be doubled: `|\|' $\to$ `|\\|'):
%
\begin{center}
|... -jobname "|\textit{target}|" |\\|"|[\textit{flags}]%
|\input{childdoc.def}\childdocforward[|\textit{main}|]{|\textit{dest}|}"|
\end{center}
%
Here \textit{target} is the name of the output file,
\textit{main} is the name of the main file
and \textit{dest} is the name of the main or child file to be processed
(all filenames without extensions).
The optional argument \textit{main} can be omitted
if \textit{main} matches \textit{dest}.
Optionally, compilation \textit{flags} can be defined via |\def| commands.
This command line makes the \TeX{} engine believe
it is compiling the file \textit{target}
whose content is specified as the latter parameter.
The provided code then forwards the processing to
\textit{main} or \textit{dest} as described in \secref{sec:forward}.

%%%%%%%%%%%%%%%%%%%%%%%%%%%%%%%%%%%%%%%%%%%%%%%%%%%%%%%%%%%%%%%%%%%%%%%%%%%%%%%%
\subsection{Include by Input}
\label{sec:input}

Including child documents by |\include| has some restrictions by design.
Most notably, the content of a child document always occupies
its own set of pages; pages cannot be shared between child documents.
Usually, this behaviour makes perfect sense
because each child document contain an essential part of the document.
However, in some situations it may be desirable to compose
a document from a collection of parts
without having mandatory page breaks between then.
For this case, the package
provides a mechanism to include parts
by |\input| which can also be processed individually.
However, by construction this mechanism
requires manual handling of the content to be output.

%%%%%%%%%%%%%%%%%%%%%%%%%%%%%%%%%%%%%%%%
\DescribeMacro{\ifchilddocmanual}
The main file should be prepared as usual, see \secref{sec:include}.
However, the document body must make a distinction
between processing of an individual part and of the main document, e.g.:
%
\begin{center}
\begin{tabular}{l}
|\ifchilddocmanual|\\
|\input{\childdocname}|\\
|\||else|\\
\textit{document body with }|\input{|\textit{part}|}|\\
|\||fi|
\end{tabular}
\end{center}
%
The conditional |\ifchilddocmanual| is true whenever
a part to be included by |\input| is being compiled,
and the name of the part is stored in |\childdocname|.

%%%%%%%%%%%%%%%%%%%%%%%%%%%%%%%%%%%%%%%%
\DescribeMacro{\childdocby}
Each part to be included by |\input| should start with:
%
\begin{center}
\begin{tabular}{l}
|\input{childdoc.def}|\\
|\childdocby{|\textit{main}|}|\\
\end{tabular}
\end{center}
%
The directive |\childdocby| is similar to |\childdocof|
described in \secref{sec:include},
but the subsequent selection of content must be done manually.
To that end, both |\ifchilddoc| and |\ifchilddocmanual|
will be true upon processing of a part,
and the name of the part is stored in |\childdocname|.
Note that |\jobname| will be set to the filename of the current part
so that each part receives an individual |.aux| file
that does not interfere with the |.aux| file(s) of the main document.
This behaviour can be altered by the alternative form
|\childdocby[*]{|\textit{main}|}| (with a non-empty optional argument)
which uses the |.aux| file of the main document
by setting |\jobname| to \textit{main}.

%%%%%%%%%%%%%%%%%%%%%%%%%%%%%%%%%%%%%%%%%%%%%%%%%%%%%%%%%%%%%%%%%%%%%%%%%%%%%%%%
\subsection{Driver Development}
\label{sec:driver}

The \textsf{childdoc} mechanism can also be use for the development
of definition files such as \LaTeX{} styles or classes.
This case differs from the above setup with multiple parts
included by |\include| in that no |\includeonly| should be invoked.
This can be achieved by starting the include file
(before |\ProvidesPackage|) with:
%
\begin{center}
\begin{tabular}{l}
|\input{childdoc.def}|\\
|\childdocforward{|\textit{main}|}|\\
\end{tabular}
\end{center}
%
or alternatively with:
%
\begin{center}
\begin{tabular}{l}
|\input{childdoc.def}|\\
|\childdocby{|\textit{main}|}|\\
\end{tabular}
\end{center}
%
Both forms have slightly different effects as described above.
The main file is prepared as usual, see \secref{sec:include}.

%%%%%%%%%%%%%%%%%%%%%%%%%%%%%%%%%%%%%%%%%%%%%%%%%%%%%%%%%%%%%%%%%%%%%%%%%%%%%%%%
\subsection{Legacy Detection}
\label{sec:detection}

The directive |\childdocmain| in the main file can detect
whether the complete document or merely a child is to be compiled
even without using the directive |\childdocof|.
This method is deprecated because it is less robust
and there is no compelling reason to use it;
it is merely provided for backward compatibility
and it may be removed in future versions.

If the detection mechanism is to be used,
it is mandatory to correctly specify
the filename of the main file as the argument of |\childdocmain|:
%
\begin{center}
\begin{tabular}{l}
|\input{childdoc.def}|\\
|\childdocmain{|\textit{main}|}|\\
\end{tabular}
\end{center}
%
If |\jobname| does not match the argument \textit{main} of |\childdocmain|,
it is assumed that |\jobname| points to the child file to be compiled.
When using |\childdocmain| with the main file specified as argument,
it suffices to start a child file
with just |\input{|\textit{main}|}|
without loading of the package and using |\childdocof|.
If instead all processing is done
with the appropriate \textsf{childdoc} directives,
the argument of \textit{main} of |\childdocmain| can be empty.

An alternative version of the command line processing described
in \secref{sec:commandline} using the detection mechanism reads:
%
\begin{center}
|... -jobname "|\textit{target}|" "|[\textit{flags}]%
[|\def\jobname{|\textit{dest}|}|]|\input{|\textit{main}|}"|
\end{center}

%%%%%%%%%%%%%%%%%%%%%%%%%%%%%%%%%%%%%%%%%%%%%%%%%%%%%%%%%%%%%%%%%%%%%%%%%%%%%%%%
\subsection{Manual Code}
\label{sec:manual}

In case one cannot be certain whether the definitions file |childdoc.def|
is installed on the target \TeX{} distribution
and one prefers not to ship it,
it is conceivable to paste a few relevant commands into the sources.

To that end, drop all statements |\input{childdoc.def}|
and perform the replacements as outlined below.
Instead of |\childdocmain{|\textit{main}|}| add the following code
to the top of the main file:
%
\begin{center}
\begin{tabular}{l}
|\||ifdefined\childdocname\endinput\||fi\newif\ifchilddoc|\\
|\edef\childdocname{\scantokens\expandafter{\jobname\noexpand}}|\\
|\def\childdocmain{|\textit{main}|}\||ifx\childdocmain\childdocname\||else|\\
|\childdoctrue\includeonly{\childdocname}\let\jobname\childdocmain\||fi|\\
\end{tabular}
\end{center}
%
Instead of |\childdocof{|\textit{main}|}| just include the main file
at the top of each child file:
%
\begin{center}
|\input{|\textit{main}|}|
\end{center}
%
A simple redirection |\childdocforward{|\textit{dest}|}| is achieved by:
%
\begin{center}
|\def\jobname{|\textit{dest}|}\input{\jobname}|
\end{center}
%
The redirection with prefix
|\childdocforwardprefix[|\textit{prefix}|]{|\textit{dest}|}|
is accomplished by:
%
\begin{center}
\begin{tabular}{l}
|{\edef\jobname{\scantokens\expandafter{\jobname\noexpand}}|\\
|\def\redirectjob |\textit{prefix}|#1~~~{\gdef\jobname{|\textit{dest}|#1}}|\\
|\expandafter\redirectjob\jobname~~~}\input{\jobname}|
\end{tabular}
\end{center}

In an alternative approach,
child documents can be compiled by a specific command line
without additional code or specific definitions:
%
\begin{center}
|... -jobname "|\textit{target}|" "|[\textit{flags}]%
|\includeonly{|\textit{dest}|}\input{|\textit{main}|}"|
\end{center}
%

%%%%%%%%%%%%%%%%%%%%%%%%%%%%%%%%%%%%%%%%%%%%%%%%%%%%%%%%%%%%%%%%%%%%%%%%%%%%%%%%
%%%%%%%%%%%%%%%%%%%%%%%%%%%%%%%%%%%%%%%%%%%%%%%%%%%%%%%%%%%%%%%%%%%%%%%%%%%%%%%%
\section{Information}

%%%%%%%%%%%%%%%%%%%%%%%%%%%%%%%%%%%%%%%%%%%%%%%%%%%%%%%%%%%%%%%%%%%%%%%%%%%%%%%%
\subsection{Copyright}

Copyright \copyright{} 2017--2018 Niklas Beisert

This work may be distributed and/or modified under the
conditions of the \LaTeX{} Project Public License, either version 1.3
of this license or (at your option) any later version.
The latest version of this license is in
  \url{http://www.latex-project.org/lppl.txt}
and version 1.3 or later is part of all distributions of \LaTeX{}
version 2005/12/01 or later.

This work has the LPPL maintenance status `maintained'.

The Current Maintainer of this work is Niklas Beisert.

This work consists of the files |README.txt|, |childdoc.ins| and |childdoc.dtx|
as well as the derived files |childdoc.def|, |cdocsamp.tex|
with |cdocsch1.tex|, |cdocsch2.tex|, |cdocspt3.tex|, |cdocspt4.tex|,
|cdocsdrf.tex|, |cdocsfn1.tex|, |cdocsfn2.tex|
as well as |childdoc.pdf|.

%%%%%%%%%%%%%%%%%%%%%%%%%%%%%%%%%%%%%%%%%%%%%%%%%%%%%%%%%%%%%%%%%%%%%%%%%%%%%%%%
\subsection{Files and Installation}

The package consists of the files:
%
\begin{center}
\begin{tabular}{ll}
    |README.txt|   & readme file \\
    |childdoc.ins| & installation file \\
    |childdoc.dtx| & source file \\
    |childdoc.def| & definition file \\
    |cdocsamp.tex| & sample main file \\
    |cdocsch1.tex| & sample include file \\
    |cdocsch2.tex| & sample include file \\
    |cdocspt3.tex| & sample part file \\
    |cdocspt4.tex| & sample part file \\
    |cdocsdrf.tex| & sample redirection file \\
    |cdocsfn1.tex| & sample redirection file \\
    |cdocsfn2.tex| & sample redirection file \\
    |childdoc.pdf| & manual
\end{tabular}
\end{center}
%
The distribution consists of the files
|README.txt|, |childdoc.ins| and |childdoc.dtx|.
%
\begin{itemize}
\item
Run (pdf)\LaTeX{} on |childdoc.dtx|
to compile the manual |childdoc.pdf| (this file).
\item
Run \LaTeX{} on |childdoc.ins| to create the definitions file |childdoc.def|
and the sample |cdocsamp.tex| with include files
|cdocsch1.tex|, |cdocsch2.tex|, |cdocspt3.tex|, |cdocspt4.tex|,
|cdocsdrf.tex|, |cdocsfn1.tex|, |cdocsfn2.tex|.
Then copy the file |childdoc.def| to an appropriate directory of your \LaTeX{}
distribution, e.g.\ \textit{texmf-root}|/tex/latex/childdoc|.
\end{itemize}

%%%%%%%%%%%%%%%%%%%%%%%%%%%%%%%%%%%%%%%%%%%%%%%%%%%%%%%%%%%%%%%%%%%%%%%%%%%%%%%%
\subsection{Related CTAN Packages}

There are several other packages which offer a similar functionality:
%
\begin{itemize}
\item
The packages
\href{http://ctan.org/pkg/docmute}{\textsf{docmute}},
\href{http://ctan.org/pkg/includex}{\textsf{includex}} and
\href{http://ctan.org/pkg/standalone}{\textsf{standalone}}
provide commands to include only the document body of
a child file thus allowing both files to be compiled individually.
\item
The packages \href{http://ctan.org/pkg/subdocs}{\textsf{subdocs}}
and \href{http://ctan.org/pkg/subfiles}{\textsf{subfiles}}
provide structures in which the main and child documents can be
encapsulated and allowing them to be compiled individually.
The inclusion mechanism is different from the conventional |\include|.
\item
The package \href{http://ctan.org/pkg/combine}{\textsf{combine}}
is an elaborate solution to combine several documents into one.
\end{itemize}
%
See also the CTAN topic \href{http://ctan.org/topic/subdocs}{\textsf{subdocs}}
for further related packages.
The present package differs from the above solutions in that
a document structure constructed with the conventional |\include| mechanism
just needs two extra commands at the top of every file
such that all constituent files can be compiled individually.

%%%%%%%%%%%%%%%%%%%%%%%%%%%%%%%%%%%%%%%%%%%%%%%%%%%%%%%%%%%%%%%%%%%%%%%%%%%%%%%%
%\subsection{Feature Suggestions}
%
%The following is a list of features which may be useful for future
%versions of this package:
%%
%\begin{itemize}
%\item
%\ldots
%\end{itemize}

%%%%%%%%%%%%%%%%%%%%%%%%%%%%%%%%%%%%%%%%%%%%%%%%%%%%%%%%%%%%%%%%%%%%%%%%%%%%%%%%
\subsection{Revision History}

%%%%%%%%%%%%%%%%%%%%%%%%%%%%%%%%%%%%%%%%
\paragraph{v2.0:} 2018/12/30

\begin{itemize}
\item
immediate forward processing
\item
added |\childdocby| mechanism
\item
manual restructured
\end{itemize}

%%%%%%%%%%%%%%%%%%%%%%%%%%%%%%%%%%%%%%%%
\paragraph{v1.6:} 2018/01/17

\begin{itemize}
\item
application for development of include files
\item
corrections to manual
\end{itemize}

%%%%%%%%%%%%%%%%%%%%%%%%%%%%%%%%%%%%%%%%
\paragraph{v1.5:} 2017/05/21

\begin{itemize}
\item
more complete structuring introduced
\item
|\childdocof| introduced
\item
|\childdoc| renamed to |\childdocmain|
\item
|\childredirect| renamed to |\childdocforward| and |\childdocforwardprefix|
and functionality expanded
\end{itemize}

%%%%%%%%%%%%%%%%%%%%%%%%%%%%%%%%%%%%%%%%
\paragraph{v1.0:} 2017/04/27

\begin{itemize}
\item
manual and install package
\item
first version published on CTAN
\end{itemize}

%%%%%%%%%%%%%%%%%%%%%%%%%%%%%%%%%%%%%%%%
\paragraph{v0.6:} 2017/04/26

\begin{itemize}
\item
redirection mechanism added
\end{itemize}

%%%%%%%%%%%%%%%%%%%%%%%%%%%%%%%%%%%%%%%%
\paragraph{v0.5:} 2017/04/26

\begin{itemize}
\item
functionality in definition file
\end{itemize}


%%%%%%%%%%%%%%%%%%%%%%%%%%%%%%%%%%%%%%%%%%%%%%%%%%%%%%%%%%%%%%%%%%%%%%%%%%%%%%%%
%%%%%%%%%%%%%%%%%%%%%%%%%%%%%%%%%%%%%%%%%%%%%%%%%%%%%%%%%%%%%%%%%%%%%%%%%%%%%%%%
%%%%%%%%%%%%%%%%%%%%%%%%%%%%%%%%%%%%%%%%%%%%%%%%%%%%%%%%%%%%%%%%%%%%%%%%%%%%%%%%
\appendix

\settowidth\MacroIndent{\rmfamily\scriptsize 000\ }

 \DocInput{childdoc.dtx}

\end{document}
%</driver>
% \fi
%
% %%%%%%%%%%%%%%%%%%%%%%%%%%%%%%%%%%%%%%%%%%%%%%%%%%%%%%%%%%%%%%%%%%%%%%%%%%%%%%
% %%%%%%%%%%%%%%%%%%%%%%%%%%%%%%%%%%%%%%%%%%%%%%%%%%%%%%%%%%%%%%%%%%%%%%%%%%%%%%
% \section{Sample}
%\iffalse
%<*samplemain>
%\fi
%
% The following presents a sample document
% with two chapters, two parts, a title page,
% a compile flag as well as three forwarding files to set the flag.
% It consists of eight |.tex| files:
% \begin{center}
% \begin{tabular}{ll}
% |cdocsamp.tex|&main file\\
% |cdocsch1.tex|&include file for chapter 1\\
% |cdocsch2.tex|&include file for chapter 2\\
% |cdocspt3.tex|&include file for part 3\\
% |cdocspt4.tex|&include file for part 4\\
% |cdocsdrf.tex|&forwarding file for main file in draft mode\\
% |cdocsfi1.tex|&forwarding file for final version of chapter 1\\
% |cdocsfi2.tex|&forwarding file for final version of chapter 2\\
% \end{tabular}
% \end{center}
% Each of the eight files can be compiled directly by the \LaTeX{} compiler.
%
% %%%%%%%%%%%%%%%%%%%%%%%%%%%%%%%%%%%%%%
% \paragraph{Main File.}
%
% The main file is called |cdocsamp.tex|.
%
% Load the \textsf{childdoc} definitions and
% declare the filename for the main document:
%    \begin{macrocode}
\input{childdoc.def}
\childdocmain{}
%    \end{macrocode}

% Optional override for |\version| flag:
%    \begin{macrocode}
%%\ifchilddoc\else\providecommand{\version}{draft}\fi
%    \end{macrocode}

% Define the default values for the |\version| flag
% (|final| for the main file and |draft| for childs):
%    \begin{macrocode}
\ifchilddoc
\providecommand{\version}{draft}
\else
\providecommand{\version}{final}
\fi
%    \end{macrocode}

% Load the standard document class:
%    \begin{macrocode}
\documentclass[12pt]{article}
%    \end{macrocode}

% Start the document body:
%    \begin{macrocode}
\begin{document}
%    \end{macrocode}

% Declare a title page.
% Print title, part of document being processed and version flag:
%    \begin{macrocode}
\addtocounter{page}{-1}
\begin{center}
{\LARGE\bfseries{}childdoc example\par}
\vspace{1cm}
\ifchilddoc
\ifchilddocmanual part\else chapter\fi:
`\childdocname' of `\childdocjob'\par
\else
main document: `\childdocjob'\par
\fi
version: \version\par
\end{center}
\newpage
%    \end{macrocode}

% Manually include selected file,
% otherwise process as usual:
%    \begin{macrocode}
\ifchilddocmanual
\section*{part `\childdocname'}
\input{\childdocname}
\else
%    \end{macrocode}

% Include the two chapters:
%    \begin{macrocode}
\include{cdocsch1}
\include{cdocsch2}
%    \end{macrocode}

% Include the two parts unless only chapters should be displayed:
%    \begin{macrocode}
\ifchilddoc\else
\section{part three}
\input{cdocspt3}
\section{part four}
\input{cdocspt4}
\fi
%    \end{macrocode}

% Process as usual until here:
%    \begin{macrocode}
\fi
%    \end{macrocode}

% End of document body:
%    \begin{macrocode}
\end{document}
%    \end{macrocode}
%\iffalse
%</samplemain>
%\fi
%
% %%%%%%%%%%%%%%%%%%%%%%%%%%%%%%%%%%%%%%
% \paragraph{Chapter Include Files.}
%
% The include files are called |cdocsch1.tex| and |cdocsch2.tex|.
%
%\iffalse
%<*samplechap1|samplechap2>
%\fi

% Optional override for |\version| flag:
%    \begin{macrocode}
%%\providecommand{\version}{final}
%    \end{macrocode}

% Include the main document:
%    \begin{macrocode}
\input{childdoc.def}
\childdocof{cdocsamp}
%    \end{macrocode}

%\iffalse
%</samplechap1|samplechap2>
%\fi
%
%\iffalse
%<*samplechap1>
%\fi
% Some text for chapter 1:
%    \begin{macrocode}
\section{one}
some text in chapter one
%    \end{macrocode}

%\iffalse
%</samplechap1>
%\fi
% Some text for chapter 2:
%\iffalse
%<*samplechap2>
%\fi
%    \begin{macrocode}
\section{two}
more text in chapter two
%    \end{macrocode}

%\iffalse
%</samplechap2>
%\fi
%
% %%%%%%%%%%%%%%%%%%%%%%%%%%%%%%%%%%%%%%
% \paragraph{Part Include Files.}
%
% The include files are called |cdocspt3.tex| and |cdocspt4.tex|.
%
%\iffalse
%<*samplepart3|samplepart4>
%\fi

% Optional override for |\version| flag:
%    \begin{macrocode}
%%\providecommand{\version}{final}
%    \end{macrocode}

% Include the main document:
%    \begin{macrocode}
\input{childdoc.def}
\childdocby{cdocsamp}
%    \end{macrocode}

%\iffalse
%</samplepart3|samplepart4>
%\fi
%
%\iffalse
%<*samplepart3>
%\fi
% Some text for part 3:
%    \begin{macrocode}
some text in part three
%    \end{macrocode}

%\iffalse
%</samplepart3>
%\fi
% Some text for part 4:
%\iffalse
%<*samplepart4>
%\fi
%    \begin{macrocode}
more text in part four
%    \end{macrocode}

%\iffalse
%</samplepart4>
%\fi
%
% %%%%%%%%%%%%%%%%%%%%%%%%%%%%%%%%%%%%%%
% \paragraph{Forwarding for a Complete Draft.}
%
% The following forwarding file |cdocsdrf.tex|
% compiles the main document in draft mode:
%\iffalse
%<*sampledraft>
%\fi
%    \begin{macrocode}
\def\version{draft}
\input{childdoc.def}
\childdocforward{cdocsamp}
%    \end{macrocode}

%\iffalse
%</sampledraft>
%\fi
%
% %%%%%%%%%%%%%%%%%%%%%%%%%%%%%%%%%%%%%%
% \paragraph{Forwarding for Final Version of the Chapters.}
%
% The following forwarding files |cdocsfn1.tex| and |cdocsfn2.tex|
% (with identical content)
% compile the final versions of the child documents
% |cdocsch1.tex| and |cdocsch2.tex|, respectively:
%\iffalse
%<*samplefinal>
%\fi
%    \begin{macrocode}
\def\version{final}
\input{childdoc.def}
\childdocforwardprefix[cdocsamp]{cdocsfn}{cdocsch}
%    \end{macrocode}

%\iffalse
%</samplefinal>
%\fi
%
% %%%%%%%%%%%%%%%%%%%%%%%%%%%%%%%%%%%%%%
% \paragraph{Command Line Processing.}
%
% The following three command lines generate the output files
% |cdocscld|, |cdocscl1| and |cdocscl2|
% which should be identical to
% |cdocsdrf|, |cdocsch1| and |cdocsfn2|, respectively:
% \begin{center}
% \begin{tabular}{l}
% |latex -jobname cdocscld \|\\
% |  "\def\version{draft}\input{childdoc.def}\childdocforward{cdocsamp}"|\\
% |latex -jobname cdocscl1 \|\\
% |  "\input{childdoc.def}\childdocforward[cdocsamp]{cdocsch1}"|\\
% |latex -jobname cdocscl2 \|\\
% |  "\def\version{final}\input{childdoc.def}\childdocforward{cdocsch2}"|
% \end{tabular}
% \end{center}
% Note that the trailing backslash on each first line
% merely continues the input to the second line
% (for convenient cut ant paste).
% Furthermore, the command |latex| can be replaced by any
% of its alternative versions such as |pdflatex|.
%
% %%%%%%%%%%%%%%%%%%%%%%%%%%%%%%%%%%%%%%%%%%%%%%%%%%%%%%%%%%%%%%%%%%%%%%%%%%%%%%
% %%%%%%%%%%%%%%%%%%%%%%%%%%%%%%%%%%%%%%%%%%%%%%%%%%%%%%%%%%%%%%%%%%%%%%%%%%%%%%
% \section{Implementation}
%\iffalse
%<*package>
%\fi
%
% This section describes the definitions file |childdoc.def|.

% The definitions cannot be loaded using |\usepackage| or |\RequirePackage|
% which has a mechanism to prevent loading a style file more than once.
% When loading the definitions by means of |\input|
% multiple instances have to be prevented manually:
%\iffalse
%This code needs to be before the `\ProvidesFile' directive
%which is defined at the beginning of this file.
%Therefore it is also placed there and commented out here.
%</package>
%<*discard>
%\fi
%    \begin{macrocode}
\ifdefined\childdocmain\endinput\fi
%    \end{macrocode}
%\iffalse
%</discard>
%<*package>
%\fi
%
% \macro{\ifchilddoc}
% \macro{\ifchilddocmanual}
% The conditional |\ifchilddoc| tells whether a
% child (true) or main (false) document is being compiled.
% The conditional |\ifchilddocmanual| tells whether
% the |\includeonly| mechanism is used (false) or
% the selection of child files must be performed manually (true).
% The definitions initialise to false:
%    \begin{macrocode}
\newif\ifchilddoc
\newif\ifchilddocmanual
%    \end{macrocode}

% \macro{\childdocname}
% \macro{\childdocjob}
% The macro |\childdocname| stores the name of the main document
% to be compiled. The macro |\childdocjob| stores the name of
% the document on which the \LaTeX{} compiler was originally invoked.
% The content of |\jobname| cannot be compared
% to filenames specified in the source due to different catcodes.
% The following code rescans |\jobname|, stores the result
% in |\childdocname| and saves a copy in |\childdocjob|:
%    \begin{macrocode}
\edef\childdocname{\scantokens\expandafter{\jobname\noexpand}}
\let\childdocjob\childdocname
%    \end{macrocode}

% \macro{\childdocdisable}
% The macro |\childdocdisable| prevents the main file
% from being processed more than once.
% At this stage, the main document command |\childdocmain|
% is assumed to be called once again where it should do nothing.
% Any subsequent call to it should prevent
% a secondary processing of the main document
% It overwrites the forwarding commands
% |\childdocof| and |\childdocforward|
% with empty macros to prevent further inclusions of the main document:
%    \begin{macrocode}
\newcommand{\childdocdisable}
{
  \renewcommand{\childdocmain}[1]{\renewcommand{\childdocmain}[1]{\endinput}}
  \renewcommand{\childdocof}[1]{}
  \renewcommand{\childdocby}[2][]{}
  \renewcommand{\childdocforward}[2][]{}
  \renewcommand{\childdocdisable}{}
}
%    \end{macrocode}

% \macro{\childdocmain}
% The macro |\childdocmain| is to be called at the top of the main file
% with nothing or the main filename (without extension) as argument.
% First, it breaks loops.
% If the argument is not empty and does not match |\childdocname|
% (which is set by the first inclusion of |childdoc.def|),
% |\ifchilddoc| is set to true, |\includeonly| is applied to the child file
% and |\jobname| is set to the main file
% (for proper handling of |.aux| files):
%    \begin{macrocode}
\newcommand{\childdocmain}[1]
{
  \childdocdisable\childdocmain{}
  \if?#1?\else
    \begingroup
      \def\childdoctmp{#1}
      \ifx\childdoctmp\childdocname
        \def\childdoctmp{}
      \else
        \def\childdoctmp
        {
          \childdoctrue
          \includeonly{\childdocname}
          \def\childdocjob{#1}
          \def\jobname{#1}
        }
      \fi
      \expandafter
    \endgroup
    \childdoctmp
  \fi
}
%    \end{macrocode}

% \macro{\childdocof}
% The command |\childdocof| redirects
% compilation to the main file |#1|.
%    \begin{macrocode}
\newcommand{\childdocof}[1]
{
  \childdocdisable
  \childdoctrue
  \includeonly{\childdocname}
  \def\jobname{#1}
  \def\childdocjob{#1}
  \input{#1}
}
%    \end{macrocode}

% \macro{\childdocby}
% The command |\childdocby| ....
%    \begin{macrocode}
\newcommand{\childdocby}[2][]
{
  \childdocdisable
  \childdoctrue
  \childdocmanualtrue
  \if?#1?\else
    \def\jobname{#2}
  \fi
  \def\childdocjob{#2}
  \input{#2}
  \endinput
}
%    \end{macrocode}

% \macro{\childdocforward}
% The command |\childdocforward| redirects
% compilation to the main file or
% (if the optional argument is given) a child file.
% Parameters are set as if the main file
% or a child file starting with |\childdocof| was compiled.
% Then compilation is handed over to the main file:
%    \begin{macrocode}
\newcommand{\childdocforward}[2][]
{
  \begingroup
    \if?#1?
      \def\childdoctmp
      {
        \def\childdocname{#2}
        \def\childdocjob{#2}
        \def\jobname{#2}
        \input{#2}
        \endinput
      }
    \else
      \def\childdoctmp
      {
        \childdocdisable
        \def\childdocname{#2}
        \childdoctrue
        \includeonly{#2}
        \def\childdocjob{#1}
        \def\jobname{#1}
        \input{#1}
        \endinput
      }
    \fi
    \expandafter
  \endgroup
  \childdoctmp
}
%    \end{macrocode}

% \macro{\childdocforwardprefix}
% The command |\childdocforwardprefix| redirects
% compilation to the main or a child file by means of a pattern.
% The prefix |#1| in the current filename is replaced by |#2|
% and the suffix of the current filename is kept
% (it is assumed that the filename does not contain the substring `|~~~|'
% which is used as a delimiter).
% Compilation is handed over to the new file by |\childdocforward|:
%    \begin{macrocode}
\newcommand{\childdocforwardprefix}[3][]
{
  \begingroup
    \def\childdocextract #2##1~~~{\def\childdoctmp{\childdocforward[#1]{#3##1}}}
    \expandafter\childdocextract\childdocname~~~
    \expandafter
  \endgroup
  \childdoctmp
}
%    \end{macrocode}

% \macro{\childdoc}
% The deprecated macro |\childdoc| is a legacy version of |\childdocmain|:
%    \begin{macrocode}
\newcommand{\childdoc}{\childdocmain}
%    \end{macrocode}

% \macro{\childdocredirect}
% The deprecated macro |\childdocredirect| is a legacy version
% of |\childdocforward| and |\childdocforwardprefix|:
%    \begin{macrocode}
\newcommand{\childdocredirect}[2][]
{
  \begingroup
    \if?#1?
      \def\childdoctmp{\childdocforward{#2}}
    \else
      \def\childdoctmp{\childdocforwardprefix{#1}{#2}}
    \fi
    \expandafter
  \endgroup
  \childdoctmp
}
%    \end{macrocode}

%\iffalse
%</package>
%\fi
%
\endinput

\childdocof{cdocsamp}
%    \end{macrocode}

%\iffalse
%</samplechap1|samplechap2>
%\fi
%
%\iffalse
%<*samplechap1>
%\fi
% Some text for chapter 1:
%    \begin{macrocode}
\section{one}
some text in chapter one
%    \end{macrocode}

%\iffalse
%</samplechap1>
%\fi
% Some text for chapter 2:
%\iffalse
%<*samplechap2>
%\fi
%    \begin{macrocode}
\section{two}
more text in chapter two
%    \end{macrocode}

%\iffalse
%</samplechap2>
%\fi
%
% %%%%%%%%%%%%%%%%%%%%%%%%%%%%%%%%%%%%%%
% \paragraph{Part Include Files.}
%
% The include files are called |cdocspt3.tex| and |cdocspt4.tex|.
%
%\iffalse
%<*samplepart3|samplepart4>
%\fi

% Optional override for |\version| flag:
%    \begin{macrocode}
%%\providecommand{\version}{final}
%    \end{macrocode}

% Include the main document:
%    \begin{macrocode}
% \iffalse
%
% childdoc.dtx Copyright (C) 2017-2018 Niklas Beisert
%
% This work may be distributed and/or modified under the
% conditions of the LaTeX Project Public License, either version 1.3
% of this license or (at your option) any later version.
% The latest version of this license is in
%   http://www.latex-project.org/lppl.txt
% and version 1.3 or later is part of all distributions of LaTeX
% version 2005/12/01 or later.
%
% This work has the LPPL maintenance status `maintained'.
%
% The Current Maintainer of this work is Niklas Beisert.
%
% This work consists of the files childdoc.dtx and childdoc.ins
% and the derived files childdoc.def and cdocsamp.tex with
% cdocsch1.tex, cdocsch2.tex, cdocsdrf.tex, cdocsfn1.tex, cdocsfn2.tex.
%
%<package>\ifdefined\childdocmain\endinput\fi
%<package>\ProvidesFile{childdoc.def}[2018/12/30 v2.0 child document driver]
%<samplemain>\ProvidesFile{cdocsamp.tex}[2018/12/30 v2.0 sample for childdoc]
%<*driver>
%\ProvidesFile{childdoc.drv}[2018/12/30 v2.0 childdoc reference manual file]
\PassOptionsToClass{10pt,a4paper}{article}
\documentclass{ltxdoc}

\usepackage[margin=35mm]{geometry}
\usepackage{hyperref}
\usepackage{hyperxmp}
\usepackage[usenames]{color}

\hypersetup{colorlinks=true}
\hypersetup{pdfstartview=FitH}
\hypersetup{pdfpagemode=UseNone}
\hypersetup{pdfsource={}}
\hypersetup{pdflang={en-UK}}
\hypersetup{pdfcopyright={Copyright 2017-2018 Niklas Beisert.
  This work may be distributed and/or modified under the
  conditions of the LaTeX Project Public License, either version 1.3
  of this license or (at your option) any later version.}}
\hypersetup{pdflicenseurl={http://www.latex-project.org/lppl.txt}}
\hypersetup{pdfcontactaddress={ETH Zurich, ITP, HIT K,
  Wolfgang-Pauli-Strasse 27}}
\hypersetup{pdfcontactpostcode={8093}}
\hypersetup{pdfcontactcity={Zurich}}
\hypersetup{pdfcontactcountry={Switzerland}}
\hypersetup{pdfcontactemail={nbeisert@itp.phys.ethz.ch}}
\hypersetup{pdfcontacturl={http://people.phys.ethz.ch/\xmptilde nbeisert/}}

\newcommand{\secref}[1]{\hyperref[#1]{section \ref*{#1}}}

\parskip1ex
\parindent0pt
\let\olditemize\itemize
\def\itemize{\olditemize\parskip0pt}

\begin{document}

\title{The \textsf{childdoc} Package}
\hypersetup{pdftitle={The childdoc Package}}
\author{Niklas Beisert\\[2ex]
  Institut f\"ur Theoretische Physik\\
  Eidgen\"ossische Technische Hochschule Z\"urich\\
  Wolfgang-Pauli-Strasse 27, 8093 Z\"urich, Switzerland\\[1ex]
  \href{mailto:nbeisert@itp.phys.ethz.ch}
  {\texttt{nbeisert@itp.phys.ethz.ch}}}
\hypersetup{pdfauthor={Niklas Beisert}}
\hypersetup{pdfsubject={Manual for the LaTeX2e Package childdoc}}
\date{30 December 2018, \textsf{v2.0}}
\maketitle

\begin{abstract}\noindent
\textsf{childdoc} is a \LaTeXe{} package
that enables the direct compilation
of document sections included by |\include|
to individual files.
\end{abstract}

\begingroup
\parskip0ex
\tableofcontents
\endgroup

%%%%%%%%%%%%%%%%%%%%%%%%%%%%%%%%%%%%%%%%%%%%%%%%%%%%%%%%%%%%%%%%%%%%%%%%%%%%%%%%
%%%%%%%%%%%%%%%%%%%%%%%%%%%%%%%%%%%%%%%%%%%%%%%%%%%%%%%%%%%%%%%%%%%%%%%%%%%%%%%%
\section{Introduction}

\LaTeX{} provides a mechanism to structure a large document (such as a book)
into a main file and several child files (containing the chapters)
using the |\include| command.
This mechanism is beneficial for documents
which span hundreds of pages in order to
make the source file(s) more manageable.
Moreover, compilation can be restricted to
selected child files by means of the |\includeonly| command.
The latter feature can be used to reduce the compilation time while editing
(this was significantly more useful in the earlier days of \LaTeX{})
or to generate a smaller document which is easier to navigate.
Another application of |\includeonly| is to generate
documents consisting of selected parts of the complete document.

However, there are a few drawbacks of the plain |\include| mechanism:
\begin{itemize}
\item
The child files cannot be compiled on their own,
they can only be compiled via the main file.
A naive editing environment
(such as a text editor with an option
to have the current file processed by \LaTeX)
may require one to switch to the main file before compiling;
attempting to compile the child file produces errors.
\item
The main file must be modified (each time)
to adjust the |\includeonly| command
to the present needs. This easily leaves the main file in a messy state.
\item
The generated document will always carry the filename
of the main document. This is inconvenient if
several child files are to be compiled and
to be kept for distribution.
\end{itemize}

The present package provides a simple interface
to make child files individually compilable by \LaTeX{}.
Compiling a child file then has the same effect as compiling
the main file with an |\includeonly| command
to select the appropriate child.
Moreover the generated document will carry the name of the child
rather than the main file.
This resolves all three above issues.

This feature is meant to make the editing of books,
thesis documents and lecture notes somewhat more convenient.
However, the package can also be used efficiently for
composing a series of documents (such as exercise sheets)
which are typically distributed individually.
It then assists the author in generating the individual documents
(potentially in different versions)
as well as a document containing the collected series.
Another application is in developing style files
or other kinds of included material
where compilation of the style file could redirect
to a sample or test file.

%%%%%%%%%%%%%%%%%%%%%%%%%%%%%%%%%%%%%%%%%%%%%%%%%%%%%%%%%%%%%%%%%%%%%%%%%%%%%%%%
%%%%%%%%%%%%%%%%%%%%%%%%%%%%%%%%%%%%%%%%%%%%%%%%%%%%%%%%%%%%%%%%%%%%%%%%%%%%%%%%
\section{Usage}

First of all, the package \textsf{childdoc} is \emph{not} a standard
\LaTeXe{} |.sty| style file! Therefore it needs to be invoked in
a non-standard way.

%%%%%%%%%%%%%%%%%%%%%%%%%%%%%%%%%%%%%%%%%%%%%%%%%%%%%%%%%%%%%%%%%%%%%%%%%%%%%%%%
\subsection{Included Files}
\label{sec:include}

%%%%%%%%%%%%%%%%%%%%%%%%%%%%%%%%%%%%%%%%
\DescribeMacro{\childdocmain}
To use the package, add the commands
\begin{center}
\begin{tabular}{l}
|\input{childdoc.def}|\\
|\childdocmain{}|\\
\end{tabular}
\end{center}
at the very top of the main \LaTeX{} file,
in particular \emph{before} the |\documentclass| statement!
The argument of |\childdocmain| should be left empty
(but it must be present).

%%%%%%%%%%%%%%%%%%%%%%%%%%%%%%%%%%%%%%%%
\DescribeMacro{\childdocof}
Furthermore, add the commands
\begin{center}
\begin{tabular}{l}
|\input{childdoc.def}|\\
|\childdocof{|\textit{main}|}|\\
\end{tabular}
\end{center}
at the top of every child file \textit{child}
which is included by |\include{|\textit{child}|}|
from within the main file
(or at least for those files to be compiled individually).
The argument \textit{main} must be the filename of the main file.

There are a couple of
considerations in setting up the main and child documents:

%%%%%%%%%%%%%%%%%%%%%%%%%%%%%%%%%%%%%%%%
\paragraph{Restrictions.}

Please note the following restrictions:
\begin{itemize}
\item
|\childdocmain| must be called with one argument \textit{main}
to ensure compatibility with earlier version of the package.
It must either be empty (|\childdocmain{}|)
or precisely match the filename of the main file in which it is specified.
See \secref{sec:detection} for further information.
\item
The filename \textit{main} must be specified without the |.tex| extension.
\item
The filename \textit{main} is case sensitive
(even in case-insensitive file systems)
due to internal string comparison.
\item
The argument \textit{main} should be fully expanded, it cannot be a macro.
\item
Subdirectories and special characters should be avoided in filenames.
\item
The command |\childdocmain{|\textit{main}|}| must be followed by a whitespace.
It should not be followed immediately by another command
or by a comment mark `|%|'.
This is because the \TeX{} parser reads the token immediately following
the argument of |\childdocmain| and puts it
at the beginning of every child section;
however, a white\-space is ignored.
\end{itemize}

%%%%%%%%%%%%%%%%%%%%%%%%%%%%%%%%%%%%%%%%
\paragraph{Content of Main File.}

It is advisable to place all content in the child files included by |\include|.
Any output contained in the main file will appear in all child documents
unless suppressed manually;
it cannot be suppressed automatically by the |\includeonly| directive
and thus should normally be avoided.
A method to include some content in the main file
by means of conditional processing is described in \secref{sec:conditional}.

%%%%%%%%%%%%%%%%%%%%%%%%%%%%%%%%%%%%%%%%
\paragraph{Page Numbering.}

When only a part of the document is compiled,
the appropriate numbering of pages
(as well as other status parameters)
is determined from the |.aux| files.
The latter contain information from previous passes.
However this information needs to propagate through
all intermediate child documents.
Therefore the page numbering in child documents may well
be inconsistent until the complete document is compiled at least once.

A useful (if unconventional) way to always ensure a consistent
page numbering is to restart the numbering in each child document
and denote the pages by `\textit{child}|.|\textit{page}'
where \textit{child} represents the chapter/section number of the child file.
This can be achieved by the command
|\numberwithin{page}{|\textit{child}|}|
of the \textsf{amsmath} package
where \textit{child} can be |chapter| or |section|
depending on the chosen structuring.
Alternatively, one can modify the macro |\thepage| appropriately
and reset the counter |page| at the start of each child file.

%%%%%%%%%%%%%%%%%%%%%%%%%%%%%%%%%%%%%%%%%%%%%%%%%%%%%%%%%%%%%%%%%%%%%%%%%%%%%%%%
\subsection{Conditional Processing}
\label{sec:conditional}

The package provides a mechanism to compile different versions
of a document. To customise the versions further some conditional processing
can come in handy to distinguish which version is being compiled.
The package provides two macros to describe the compilation context:

%%%%%%%%%%%%%%%%%%%%%%%%%%%%%%%%%%%%%%%%
\DescribeMacro{\ifchilddoc}
The conditional |\ifchilddoc| distinguishes between the compilation of
child documents and the main document:
%
\begin{center}
|\ifchilddoc |\textit{child-code}| |[|\||else |\textit{main-code}]| \||fi|
\end{center}

%%%%%%%%%%%%%%%%%%%%%%%%%%%%%%%%%%%%%%%%
\DescribeMacro{\childdocname}
\DescribeMacro{\childdocjob}
The macro |\childdocname| contains the filename (without extension)
of the main or child file being processed.
Note that |\childdocjob| will always contain the name of the main file.

%%%%%%%%%%%%%%%%%%%%%%%%%%%%%%%%%%%%%%%%
\paragraph{Title Page.}

Conditional processing can be used to include a title or banner page
in the main document when proper precautions are taken.
Importantly, the code in the main file should ensure that the page counter
(as well as other status parameters which are stored in the |.aux| files)
takes the same value after the conditional processing.
Otherwise the page numbers may take divergent values
depending on which part is compiled.

For example, a title page could be declared by:
%
\begin{center}
\begin{tabular}{l}
|\ifchilddoc\||else|\\
|\addtocounter{page}{-1}|\\
\textit{code for title page}\\
|\newpage|\\
|\||fi|
\end{tabular}
\end{center}
%
A banner page for the child documents can be generated by:
%
\begin{center}
\begin{tabular}{l}
|\ifchilddoc|\\
|\addtocounter{page}{-1}|\\
\textit{code for banner page}\\
|\newpage|\\
|\||fi|
\end{tabular}
\end{center}
%
Here one could write a message such as:
\begin{center}
|This is the part \childdocname{} of \childdocjob{}.|
\end{center}

%%%%%%%%%%%%%%%%%%%%%%%%%%%%%%%%%%%%%%%%%%%%%%%%%%%%%%%%%%%%%%%%%%%%%%%%%%%%%%%%
\subsection{Flags}
\label{sec:flags}

The package makes it easy to generate different versions
of the main or child documents.
To this end compilation flags can be defined
and assigned different default values.
They will be particularly useful in conjunction
with the forwarding mechanism described in \secref{sec:forward}.

For example, it may be useful to have a flag |\version|
which can be set to |draft| or |final|.
The document source will contain some conditional code
depending on the value of |\version|.
Suppose further, the flag should default to |final| for the main file
and to |draft| for child files
which is a natural assignment for editing the document.
This is achieved by placing the following code
in the preamble of the main document
(below the |\childdocmain| directive):
%
\begin{center}
\begin{tabular}{l}
|\ifchilddoc|\\
|\providecommand{\version}{draft}|\\
|\||else|\\
|\providecommand{\version}{final}|\\
|\||fi|
\end{tabular}
\end{center}
%
The definition by |\providecommand| makes sure
that previous definitions are not overwritten.
Further statements |\providecommand{\version}{...}|
can thus be added before the above code to override it.

For the main file, one might add a line
(between |\childdocmain| and the above block)
%
\begin{center}
|%\ifchilddoc\||else\providecommand{\version}{draft}\||fi|
\end{center}
%
which can be uncommented to produce a draft version.
Likewise one can add a line to the very top of a child file
(above the |\childdocof{|\textit{main}|}| directive)
%
\begin{center}
|%\providecommand{\version}{final}|
\end{center}
%
which can be uncommented to produce the final version of this child document.

%%%%%%%%%%%%%%%%%%%%%%%%%%%%%%%%%%%%%%%%%%%%%%%%%%%%%%%%%%%%%%%%%%%%%%%%%%%%%%%%
\subsection{Forwarding}
\label{sec:forward}

Different versions of the main or child documents
using compilation flags as described in \secref{sec:flags}
can be (permanently) stored in different files
for convenient compilation, viewing and distribution.
To this end, the package defines a command
to pass on compilation to a different file:

%%%%%%%%%%%%%%%%%%%%%%%%%%%%%%%%%%%%%%%%
\DescribeMacro{\childdocforward}
The command |\childdocforward| redirects processing to
another source file:
%
\begin{center}
\begin{tabular}{l}
|\input{childdoc.def}|\\
|\childdocforward[|\textit{main}|]{|\textit{dest}|}|\\
\end{tabular}
\end{center}
%
The argument \textit{dest} is the destination file
(without extension).
It should be the main file or one of the child files.
Note that further \textsf{childdoc} directives
such as |\childdocof| and |\childdocforward|
in the indicated file will be processed in this form.
The optional argument \textit{main}
passes on directly to the main file \textit{main}
while pretending to compile the child \textit{dest}.
This form behaves as if \textit{dest}
issues |\childdocof{|\textit{main}|}| right away,
and no further \textsf{childdoc} directives will be processed.

%%%%%%%%%%%%%%%%%%%%%%%%%%%%%%%%%%%%%%%%
\DescribeMacro{\...prefix}
In the alternative form |\childdocforwardprefix|,
%
\begin{center}
\begin{tabular}{l}
|\input{childdoc.def}|\\
|\childdocforwardprefix[|\textit{main}|]{|\textit{prefix}|}{|\textit{dest}|}|
\end{tabular}
\end{center}
%
the destination file is determined by a pattern
depending on the current file:
To make this work, the current file must be called
`{\textit{prefix}\hspace{0.2em}\textit{suffix}}'
with \textit{prefix} matching precisely the argument.
Processing is then passed on to the file
`{\textit{dest}\hspace{0.2em}\textit{suffix}}'.
Surely, the same effect is achieved by
directly specifying the
argument `{\textit{dest}\hspace{0.2em}\textit{suffix}}'
in the first form.
However, that requires to set up a different file
for each child. With the alternative form of the command
all these files can have exactly the same content
which simplifies setting them up and maintaining them.

For example, the following file |draft.tex|
with a compilation flag |\version| as described in \secref{sec:flags}
compiles the main document as a draft:
%
\begin{center}
\begin{tabular}{l}
|\def\version{draft}|\\
|\input{childdoc.def}|\\
|\childdocforward{|\textit{main}|}|
\end{tabular}
\end{center}
%
Likewise, the following files |final|\textit{nn}|.tex|
compile the final version of the child document
|child|\textit{nn}|.tex|:
%
\begin{center}
\begin{tabular}{l}
|\def\version{final}|\\
|\input{childdoc.def}|\\
|\childdocforwardprefix{final}{child}|
\end{tabular}
\end{center}
%

Note that when several versions of a main file and/or of each child file
are to be generated, it may be convenient to set up a |Makefile| or
shell script to automatise the process.

%%%%%%%%%%%%%%%%%%%%%%%%%%%%%%%%%%%%%%%%%%%%%%%%%%%%%%%%%%%%%%%%%%%%%%%%%%%%%%%%
\subsection{Command Line Processing}
\label{sec:commandline}

The effect of redirection files can also be achieved by invoking
the \LaTeX{} compiler with a more elaborate command line.
Most conveniently this should be done as part
of a shell script or a |Makefile|.

When using \textsf{childdoc} in the main file, the following
command lines effectively perform a redirection
(note that depending on the shell being used,
backslashes may have to be doubled: `|\|' $\to$ `|\\|'):
%
\begin{center}
|... -jobname "|\textit{target}|" |\\|"|[\textit{flags}]%
|\input{childdoc.def}\childdocforward[|\textit{main}|]{|\textit{dest}|}"|
\end{center}
%
Here \textit{target} is the name of the output file,
\textit{main} is the name of the main file
and \textit{dest} is the name of the main or child file to be processed
(all filenames without extensions).
The optional argument \textit{main} can be omitted
if \textit{main} matches \textit{dest}.
Optionally, compilation \textit{flags} can be defined via |\def| commands.
This command line makes the \TeX{} engine believe
it is compiling the file \textit{target}
whose content is specified as the latter parameter.
The provided code then forwards the processing to
\textit{main} or \textit{dest} as described in \secref{sec:forward}.

%%%%%%%%%%%%%%%%%%%%%%%%%%%%%%%%%%%%%%%%%%%%%%%%%%%%%%%%%%%%%%%%%%%%%%%%%%%%%%%%
\subsection{Include by Input}
\label{sec:input}

Including child documents by |\include| has some restrictions by design.
Most notably, the content of a child document always occupies
its own set of pages; pages cannot be shared between child documents.
Usually, this behaviour makes perfect sense
because each child document contain an essential part of the document.
However, in some situations it may be desirable to compose
a document from a collection of parts
without having mandatory page breaks between then.
For this case, the package
provides a mechanism to include parts
by |\input| which can also be processed individually.
However, by construction this mechanism
requires manual handling of the content to be output.

%%%%%%%%%%%%%%%%%%%%%%%%%%%%%%%%%%%%%%%%
\DescribeMacro{\ifchilddocmanual}
The main file should be prepared as usual, see \secref{sec:include}.
However, the document body must make a distinction
between processing of an individual part and of the main document, e.g.:
%
\begin{center}
\begin{tabular}{l}
|\ifchilddocmanual|\\
|\input{\childdocname}|\\
|\||else|\\
\textit{document body with }|\input{|\textit{part}|}|\\
|\||fi|
\end{tabular}
\end{center}
%
The conditional |\ifchilddocmanual| is true whenever
a part to be included by |\input| is being compiled,
and the name of the part is stored in |\childdocname|.

%%%%%%%%%%%%%%%%%%%%%%%%%%%%%%%%%%%%%%%%
\DescribeMacro{\childdocby}
Each part to be included by |\input| should start with:
%
\begin{center}
\begin{tabular}{l}
|\input{childdoc.def}|\\
|\childdocby{|\textit{main}|}|\\
\end{tabular}
\end{center}
%
The directive |\childdocby| is similar to |\childdocof|
described in \secref{sec:include},
but the subsequent selection of content must be done manually.
To that end, both |\ifchilddoc| and |\ifchilddocmanual|
will be true upon processing of a part,
and the name of the part is stored in |\childdocname|.
Note that |\jobname| will be set to the filename of the current part
so that each part receives an individual |.aux| file
that does not interfere with the |.aux| file(s) of the main document.
This behaviour can be altered by the alternative form
|\childdocby[*]{|\textit{main}|}| (with a non-empty optional argument)
which uses the |.aux| file of the main document
by setting |\jobname| to \textit{main}.

%%%%%%%%%%%%%%%%%%%%%%%%%%%%%%%%%%%%%%%%%%%%%%%%%%%%%%%%%%%%%%%%%%%%%%%%%%%%%%%%
\subsection{Driver Development}
\label{sec:driver}

The \textsf{childdoc} mechanism can also be use for the development
of definition files such as \LaTeX{} styles or classes.
This case differs from the above setup with multiple parts
included by |\include| in that no |\includeonly| should be invoked.
This can be achieved by starting the include file
(before |\ProvidesPackage|) with:
%
\begin{center}
\begin{tabular}{l}
|\input{childdoc.def}|\\
|\childdocforward{|\textit{main}|}|\\
\end{tabular}
\end{center}
%
or alternatively with:
%
\begin{center}
\begin{tabular}{l}
|\input{childdoc.def}|\\
|\childdocby{|\textit{main}|}|\\
\end{tabular}
\end{center}
%
Both forms have slightly different effects as described above.
The main file is prepared as usual, see \secref{sec:include}.

%%%%%%%%%%%%%%%%%%%%%%%%%%%%%%%%%%%%%%%%%%%%%%%%%%%%%%%%%%%%%%%%%%%%%%%%%%%%%%%%
\subsection{Legacy Detection}
\label{sec:detection}

The directive |\childdocmain| in the main file can detect
whether the complete document or merely a child is to be compiled
even without using the directive |\childdocof|.
This method is deprecated because it is less robust
and there is no compelling reason to use it;
it is merely provided for backward compatibility
and it may be removed in future versions.

If the detection mechanism is to be used,
it is mandatory to correctly specify
the filename of the main file as the argument of |\childdocmain|:
%
\begin{center}
\begin{tabular}{l}
|\input{childdoc.def}|\\
|\childdocmain{|\textit{main}|}|\\
\end{tabular}
\end{center}
%
If |\jobname| does not match the argument \textit{main} of |\childdocmain|,
it is assumed that |\jobname| points to the child file to be compiled.
When using |\childdocmain| with the main file specified as argument,
it suffices to start a child file
with just |\input{|\textit{main}|}|
without loading of the package and using |\childdocof|.
If instead all processing is done
with the appropriate \textsf{childdoc} directives,
the argument of \textit{main} of |\childdocmain| can be empty.

An alternative version of the command line processing described
in \secref{sec:commandline} using the detection mechanism reads:
%
\begin{center}
|... -jobname "|\textit{target}|" "|[\textit{flags}]%
[|\def\jobname{|\textit{dest}|}|]|\input{|\textit{main}|}"|
\end{center}

%%%%%%%%%%%%%%%%%%%%%%%%%%%%%%%%%%%%%%%%%%%%%%%%%%%%%%%%%%%%%%%%%%%%%%%%%%%%%%%%
\subsection{Manual Code}
\label{sec:manual}

In case one cannot be certain whether the definitions file |childdoc.def|
is installed on the target \TeX{} distribution
and one prefers not to ship it,
it is conceivable to paste a few relevant commands into the sources.

To that end, drop all statements |\input{childdoc.def}|
and perform the replacements as outlined below.
Instead of |\childdocmain{|\textit{main}|}| add the following code
to the top of the main file:
%
\begin{center}
\begin{tabular}{l}
|\||ifdefined\childdocname\endinput\||fi\newif\ifchilddoc|\\
|\edef\childdocname{\scantokens\expandafter{\jobname\noexpand}}|\\
|\def\childdocmain{|\textit{main}|}\||ifx\childdocmain\childdocname\||else|\\
|\childdoctrue\includeonly{\childdocname}\let\jobname\childdocmain\||fi|\\
\end{tabular}
\end{center}
%
Instead of |\childdocof{|\textit{main}|}| just include the main file
at the top of each child file:
%
\begin{center}
|\input{|\textit{main}|}|
\end{center}
%
A simple redirection |\childdocforward{|\textit{dest}|}| is achieved by:
%
\begin{center}
|\def\jobname{|\textit{dest}|}\input{\jobname}|
\end{center}
%
The redirection with prefix
|\childdocforwardprefix[|\textit{prefix}|]{|\textit{dest}|}|
is accomplished by:
%
\begin{center}
\begin{tabular}{l}
|{\edef\jobname{\scantokens\expandafter{\jobname\noexpand}}|\\
|\def\redirectjob |\textit{prefix}|#1~~~{\gdef\jobname{|\textit{dest}|#1}}|\\
|\expandafter\redirectjob\jobname~~~}\input{\jobname}|
\end{tabular}
\end{center}

In an alternative approach,
child documents can be compiled by a specific command line
without additional code or specific definitions:
%
\begin{center}
|... -jobname "|\textit{target}|" "|[\textit{flags}]%
|\includeonly{|\textit{dest}|}\input{|\textit{main}|}"|
\end{center}
%

%%%%%%%%%%%%%%%%%%%%%%%%%%%%%%%%%%%%%%%%%%%%%%%%%%%%%%%%%%%%%%%%%%%%%%%%%%%%%%%%
%%%%%%%%%%%%%%%%%%%%%%%%%%%%%%%%%%%%%%%%%%%%%%%%%%%%%%%%%%%%%%%%%%%%%%%%%%%%%%%%
\section{Information}

%%%%%%%%%%%%%%%%%%%%%%%%%%%%%%%%%%%%%%%%%%%%%%%%%%%%%%%%%%%%%%%%%%%%%%%%%%%%%%%%
\subsection{Copyright}

Copyright \copyright{} 2017--2018 Niklas Beisert

This work may be distributed and/or modified under the
conditions of the \LaTeX{} Project Public License, either version 1.3
of this license or (at your option) any later version.
The latest version of this license is in
  \url{http://www.latex-project.org/lppl.txt}
and version 1.3 or later is part of all distributions of \LaTeX{}
version 2005/12/01 or later.

This work has the LPPL maintenance status `maintained'.

The Current Maintainer of this work is Niklas Beisert.

This work consists of the files |README.txt|, |childdoc.ins| and |childdoc.dtx|
as well as the derived files |childdoc.def|, |cdocsamp.tex|
with |cdocsch1.tex|, |cdocsch2.tex|, |cdocspt3.tex|, |cdocspt4.tex|,
|cdocsdrf.tex|, |cdocsfn1.tex|, |cdocsfn2.tex|
as well as |childdoc.pdf|.

%%%%%%%%%%%%%%%%%%%%%%%%%%%%%%%%%%%%%%%%%%%%%%%%%%%%%%%%%%%%%%%%%%%%%%%%%%%%%%%%
\subsection{Files and Installation}

The package consists of the files:
%
\begin{center}
\begin{tabular}{ll}
    |README.txt|   & readme file \\
    |childdoc.ins| & installation file \\
    |childdoc.dtx| & source file \\
    |childdoc.def| & definition file \\
    |cdocsamp.tex| & sample main file \\
    |cdocsch1.tex| & sample include file \\
    |cdocsch2.tex| & sample include file \\
    |cdocspt3.tex| & sample part file \\
    |cdocspt4.tex| & sample part file \\
    |cdocsdrf.tex| & sample redirection file \\
    |cdocsfn1.tex| & sample redirection file \\
    |cdocsfn2.tex| & sample redirection file \\
    |childdoc.pdf| & manual
\end{tabular}
\end{center}
%
The distribution consists of the files
|README.txt|, |childdoc.ins| and |childdoc.dtx|.
%
\begin{itemize}
\item
Run (pdf)\LaTeX{} on |childdoc.dtx|
to compile the manual |childdoc.pdf| (this file).
\item
Run \LaTeX{} on |childdoc.ins| to create the definitions file |childdoc.def|
and the sample |cdocsamp.tex| with include files
|cdocsch1.tex|, |cdocsch2.tex|, |cdocspt3.tex|, |cdocspt4.tex|,
|cdocsdrf.tex|, |cdocsfn1.tex|, |cdocsfn2.tex|.
Then copy the file |childdoc.def| to an appropriate directory of your \LaTeX{}
distribution, e.g.\ \textit{texmf-root}|/tex/latex/childdoc|.
\end{itemize}

%%%%%%%%%%%%%%%%%%%%%%%%%%%%%%%%%%%%%%%%%%%%%%%%%%%%%%%%%%%%%%%%%%%%%%%%%%%%%%%%
\subsection{Related CTAN Packages}

There are several other packages which offer a similar functionality:
%
\begin{itemize}
\item
The packages
\href{http://ctan.org/pkg/docmute}{\textsf{docmute}},
\href{http://ctan.org/pkg/includex}{\textsf{includex}} and
\href{http://ctan.org/pkg/standalone}{\textsf{standalone}}
provide commands to include only the document body of
a child file thus allowing both files to be compiled individually.
\item
The packages \href{http://ctan.org/pkg/subdocs}{\textsf{subdocs}}
and \href{http://ctan.org/pkg/subfiles}{\textsf{subfiles}}
provide structures in which the main and child documents can be
encapsulated and allowing them to be compiled individually.
The inclusion mechanism is different from the conventional |\include|.
\item
The package \href{http://ctan.org/pkg/combine}{\textsf{combine}}
is an elaborate solution to combine several documents into one.
\end{itemize}
%
See also the CTAN topic \href{http://ctan.org/topic/subdocs}{\textsf{subdocs}}
for further related packages.
The present package differs from the above solutions in that
a document structure constructed with the conventional |\include| mechanism
just needs two extra commands at the top of every file
such that all constituent files can be compiled individually.

%%%%%%%%%%%%%%%%%%%%%%%%%%%%%%%%%%%%%%%%%%%%%%%%%%%%%%%%%%%%%%%%%%%%%%%%%%%%%%%%
%\subsection{Feature Suggestions}
%
%The following is a list of features which may be useful for future
%versions of this package:
%%
%\begin{itemize}
%\item
%\ldots
%\end{itemize}

%%%%%%%%%%%%%%%%%%%%%%%%%%%%%%%%%%%%%%%%%%%%%%%%%%%%%%%%%%%%%%%%%%%%%%%%%%%%%%%%
\subsection{Revision History}

%%%%%%%%%%%%%%%%%%%%%%%%%%%%%%%%%%%%%%%%
\paragraph{v2.0:} 2018/12/30

\begin{itemize}
\item
immediate forward processing
\item
added |\childdocby| mechanism
\item
manual restructured
\end{itemize}

%%%%%%%%%%%%%%%%%%%%%%%%%%%%%%%%%%%%%%%%
\paragraph{v1.6:} 2018/01/17

\begin{itemize}
\item
application for development of include files
\item
corrections to manual
\end{itemize}

%%%%%%%%%%%%%%%%%%%%%%%%%%%%%%%%%%%%%%%%
\paragraph{v1.5:} 2017/05/21

\begin{itemize}
\item
more complete structuring introduced
\item
|\childdocof| introduced
\item
|\childdoc| renamed to |\childdocmain|
\item
|\childredirect| renamed to |\childdocforward| and |\childdocforwardprefix|
and functionality expanded
\end{itemize}

%%%%%%%%%%%%%%%%%%%%%%%%%%%%%%%%%%%%%%%%
\paragraph{v1.0:} 2017/04/27

\begin{itemize}
\item
manual and install package
\item
first version published on CTAN
\end{itemize}

%%%%%%%%%%%%%%%%%%%%%%%%%%%%%%%%%%%%%%%%
\paragraph{v0.6:} 2017/04/26

\begin{itemize}
\item
redirection mechanism added
\end{itemize}

%%%%%%%%%%%%%%%%%%%%%%%%%%%%%%%%%%%%%%%%
\paragraph{v0.5:} 2017/04/26

\begin{itemize}
\item
functionality in definition file
\end{itemize}


%%%%%%%%%%%%%%%%%%%%%%%%%%%%%%%%%%%%%%%%%%%%%%%%%%%%%%%%%%%%%%%%%%%%%%%%%%%%%%%%
%%%%%%%%%%%%%%%%%%%%%%%%%%%%%%%%%%%%%%%%%%%%%%%%%%%%%%%%%%%%%%%%%%%%%%%%%%%%%%%%
%%%%%%%%%%%%%%%%%%%%%%%%%%%%%%%%%%%%%%%%%%%%%%%%%%%%%%%%%%%%%%%%%%%%%%%%%%%%%%%%
\appendix

\settowidth\MacroIndent{\rmfamily\scriptsize 000\ }

 \DocInput{childdoc.dtx}

\end{document}
%</driver>
% \fi
%
% %%%%%%%%%%%%%%%%%%%%%%%%%%%%%%%%%%%%%%%%%%%%%%%%%%%%%%%%%%%%%%%%%%%%%%%%%%%%%%
% %%%%%%%%%%%%%%%%%%%%%%%%%%%%%%%%%%%%%%%%%%%%%%%%%%%%%%%%%%%%%%%%%%%%%%%%%%%%%%
% \section{Sample}
%\iffalse
%<*samplemain>
%\fi
%
% The following presents a sample document
% with two chapters, two parts, a title page,
% a compile flag as well as three forwarding files to set the flag.
% It consists of eight |.tex| files:
% \begin{center}
% \begin{tabular}{ll}
% |cdocsamp.tex|&main file\\
% |cdocsch1.tex|&include file for chapter 1\\
% |cdocsch2.tex|&include file for chapter 2\\
% |cdocspt3.tex|&include file for part 3\\
% |cdocspt4.tex|&include file for part 4\\
% |cdocsdrf.tex|&forwarding file for main file in draft mode\\
% |cdocsfi1.tex|&forwarding file for final version of chapter 1\\
% |cdocsfi2.tex|&forwarding file for final version of chapter 2\\
% \end{tabular}
% \end{center}
% Each of the eight files can be compiled directly by the \LaTeX{} compiler.
%
% %%%%%%%%%%%%%%%%%%%%%%%%%%%%%%%%%%%%%%
% \paragraph{Main File.}
%
% The main file is called |cdocsamp.tex|.
%
% Load the \textsf{childdoc} definitions and
% declare the filename for the main document:
%    \begin{macrocode}
\input{childdoc.def}
\childdocmain{}
%    \end{macrocode}

% Optional override for |\version| flag:
%    \begin{macrocode}
%%\ifchilddoc\else\providecommand{\version}{draft}\fi
%    \end{macrocode}

% Define the default values for the |\version| flag
% (|final| for the main file and |draft| for childs):
%    \begin{macrocode}
\ifchilddoc
\providecommand{\version}{draft}
\else
\providecommand{\version}{final}
\fi
%    \end{macrocode}

% Load the standard document class:
%    \begin{macrocode}
\documentclass[12pt]{article}
%    \end{macrocode}

% Start the document body:
%    \begin{macrocode}
\begin{document}
%    \end{macrocode}

% Declare a title page.
% Print title, part of document being processed and version flag:
%    \begin{macrocode}
\addtocounter{page}{-1}
\begin{center}
{\LARGE\bfseries{}childdoc example\par}
\vspace{1cm}
\ifchilddoc
\ifchilddocmanual part\else chapter\fi:
`\childdocname' of `\childdocjob'\par
\else
main document: `\childdocjob'\par
\fi
version: \version\par
\end{center}
\newpage
%    \end{macrocode}

% Manually include selected file,
% otherwise process as usual:
%    \begin{macrocode}
\ifchilddocmanual
\section*{part `\childdocname'}
\input{\childdocname}
\else
%    \end{macrocode}

% Include the two chapters:
%    \begin{macrocode}
\include{cdocsch1}
\include{cdocsch2}
%    \end{macrocode}

% Include the two parts unless only chapters should be displayed:
%    \begin{macrocode}
\ifchilddoc\else
\section{part three}
\input{cdocspt3}
\section{part four}
\input{cdocspt4}
\fi
%    \end{macrocode}

% Process as usual until here:
%    \begin{macrocode}
\fi
%    \end{macrocode}

% End of document body:
%    \begin{macrocode}
\end{document}
%    \end{macrocode}
%\iffalse
%</samplemain>
%\fi
%
% %%%%%%%%%%%%%%%%%%%%%%%%%%%%%%%%%%%%%%
% \paragraph{Chapter Include Files.}
%
% The include files are called |cdocsch1.tex| and |cdocsch2.tex|.
%
%\iffalse
%<*samplechap1|samplechap2>
%\fi

% Optional override for |\version| flag:
%    \begin{macrocode}
%%\providecommand{\version}{final}
%    \end{macrocode}

% Include the main document:
%    \begin{macrocode}
\input{childdoc.def}
\childdocof{cdocsamp}
%    \end{macrocode}

%\iffalse
%</samplechap1|samplechap2>
%\fi
%
%\iffalse
%<*samplechap1>
%\fi
% Some text for chapter 1:
%    \begin{macrocode}
\section{one}
some text in chapter one
%    \end{macrocode}

%\iffalse
%</samplechap1>
%\fi
% Some text for chapter 2:
%\iffalse
%<*samplechap2>
%\fi
%    \begin{macrocode}
\section{two}
more text in chapter two
%    \end{macrocode}

%\iffalse
%</samplechap2>
%\fi
%
% %%%%%%%%%%%%%%%%%%%%%%%%%%%%%%%%%%%%%%
% \paragraph{Part Include Files.}
%
% The include files are called |cdocspt3.tex| and |cdocspt4.tex|.
%
%\iffalse
%<*samplepart3|samplepart4>
%\fi

% Optional override for |\version| flag:
%    \begin{macrocode}
%%\providecommand{\version}{final}
%    \end{macrocode}

% Include the main document:
%    \begin{macrocode}
\input{childdoc.def}
\childdocby{cdocsamp}
%    \end{macrocode}

%\iffalse
%</samplepart3|samplepart4>
%\fi
%
%\iffalse
%<*samplepart3>
%\fi
% Some text for part 3:
%    \begin{macrocode}
some text in part three
%    \end{macrocode}

%\iffalse
%</samplepart3>
%\fi
% Some text for part 4:
%\iffalse
%<*samplepart4>
%\fi
%    \begin{macrocode}
more text in part four
%    \end{macrocode}

%\iffalse
%</samplepart4>
%\fi
%
% %%%%%%%%%%%%%%%%%%%%%%%%%%%%%%%%%%%%%%
% \paragraph{Forwarding for a Complete Draft.}
%
% The following forwarding file |cdocsdrf.tex|
% compiles the main document in draft mode:
%\iffalse
%<*sampledraft>
%\fi
%    \begin{macrocode}
\def\version{draft}
\input{childdoc.def}
\childdocforward{cdocsamp}
%    \end{macrocode}

%\iffalse
%</sampledraft>
%\fi
%
% %%%%%%%%%%%%%%%%%%%%%%%%%%%%%%%%%%%%%%
% \paragraph{Forwarding for Final Version of the Chapters.}
%
% The following forwarding files |cdocsfn1.tex| and |cdocsfn2.tex|
% (with identical content)
% compile the final versions of the child documents
% |cdocsch1.tex| and |cdocsch2.tex|, respectively:
%\iffalse
%<*samplefinal>
%\fi
%    \begin{macrocode}
\def\version{final}
\input{childdoc.def}
\childdocforwardprefix[cdocsamp]{cdocsfn}{cdocsch}
%    \end{macrocode}

%\iffalse
%</samplefinal>
%\fi
%
% %%%%%%%%%%%%%%%%%%%%%%%%%%%%%%%%%%%%%%
% \paragraph{Command Line Processing.}
%
% The following three command lines generate the output files
% |cdocscld|, |cdocscl1| and |cdocscl2|
% which should be identical to
% |cdocsdrf|, |cdocsch1| and |cdocsfn2|, respectively:
% \begin{center}
% \begin{tabular}{l}
% |latex -jobname cdocscld \|\\
% |  "\def\version{draft}\input{childdoc.def}\childdocforward{cdocsamp}"|\\
% |latex -jobname cdocscl1 \|\\
% |  "\input{childdoc.def}\childdocforward[cdocsamp]{cdocsch1}"|\\
% |latex -jobname cdocscl2 \|\\
% |  "\def\version{final}\input{childdoc.def}\childdocforward{cdocsch2}"|
% \end{tabular}
% \end{center}
% Note that the trailing backslash on each first line
% merely continues the input to the second line
% (for convenient cut ant paste).
% Furthermore, the command |latex| can be replaced by any
% of its alternative versions such as |pdflatex|.
%
% %%%%%%%%%%%%%%%%%%%%%%%%%%%%%%%%%%%%%%%%%%%%%%%%%%%%%%%%%%%%%%%%%%%%%%%%%%%%%%
% %%%%%%%%%%%%%%%%%%%%%%%%%%%%%%%%%%%%%%%%%%%%%%%%%%%%%%%%%%%%%%%%%%%%%%%%%%%%%%
% \section{Implementation}
%\iffalse
%<*package>
%\fi
%
% This section describes the definitions file |childdoc.def|.

% The definitions cannot be loaded using |\usepackage| or |\RequirePackage|
% which has a mechanism to prevent loading a style file more than once.
% When loading the definitions by means of |\input|
% multiple instances have to be prevented manually:
%\iffalse
%This code needs to be before the `\ProvidesFile' directive
%which is defined at the beginning of this file.
%Therefore it is also placed there and commented out here.
%</package>
%<*discard>
%\fi
%    \begin{macrocode}
\ifdefined\childdocmain\endinput\fi
%    \end{macrocode}
%\iffalse
%</discard>
%<*package>
%\fi
%
% \macro{\ifchilddoc}
% \macro{\ifchilddocmanual}
% The conditional |\ifchilddoc| tells whether a
% child (true) or main (false) document is being compiled.
% The conditional |\ifchilddocmanual| tells whether
% the |\includeonly| mechanism is used (false) or
% the selection of child files must be performed manually (true).
% The definitions initialise to false:
%    \begin{macrocode}
\newif\ifchilddoc
\newif\ifchilddocmanual
%    \end{macrocode}

% \macro{\childdocname}
% \macro{\childdocjob}
% The macro |\childdocname| stores the name of the main document
% to be compiled. The macro |\childdocjob| stores the name of
% the document on which the \LaTeX{} compiler was originally invoked.
% The content of |\jobname| cannot be compared
% to filenames specified in the source due to different catcodes.
% The following code rescans |\jobname|, stores the result
% in |\childdocname| and saves a copy in |\childdocjob|:
%    \begin{macrocode}
\edef\childdocname{\scantokens\expandafter{\jobname\noexpand}}
\let\childdocjob\childdocname
%    \end{macrocode}

% \macro{\childdocdisable}
% The macro |\childdocdisable| prevents the main file
% from being processed more than once.
% At this stage, the main document command |\childdocmain|
% is assumed to be called once again where it should do nothing.
% Any subsequent call to it should prevent
% a secondary processing of the main document
% It overwrites the forwarding commands
% |\childdocof| and |\childdocforward|
% with empty macros to prevent further inclusions of the main document:
%    \begin{macrocode}
\newcommand{\childdocdisable}
{
  \renewcommand{\childdocmain}[1]{\renewcommand{\childdocmain}[1]{\endinput}}
  \renewcommand{\childdocof}[1]{}
  \renewcommand{\childdocby}[2][]{}
  \renewcommand{\childdocforward}[2][]{}
  \renewcommand{\childdocdisable}{}
}
%    \end{macrocode}

% \macro{\childdocmain}
% The macro |\childdocmain| is to be called at the top of the main file
% with nothing or the main filename (without extension) as argument.
% First, it breaks loops.
% If the argument is not empty and does not match |\childdocname|
% (which is set by the first inclusion of |childdoc.def|),
% |\ifchilddoc| is set to true, |\includeonly| is applied to the child file
% and |\jobname| is set to the main file
% (for proper handling of |.aux| files):
%    \begin{macrocode}
\newcommand{\childdocmain}[1]
{
  \childdocdisable\childdocmain{}
  \if?#1?\else
    \begingroup
      \def\childdoctmp{#1}
      \ifx\childdoctmp\childdocname
        \def\childdoctmp{}
      \else
        \def\childdoctmp
        {
          \childdoctrue
          \includeonly{\childdocname}
          \def\childdocjob{#1}
          \def\jobname{#1}
        }
      \fi
      \expandafter
    \endgroup
    \childdoctmp
  \fi
}
%    \end{macrocode}

% \macro{\childdocof}
% The command |\childdocof| redirects
% compilation to the main file |#1|.
%    \begin{macrocode}
\newcommand{\childdocof}[1]
{
  \childdocdisable
  \childdoctrue
  \includeonly{\childdocname}
  \def\jobname{#1}
  \def\childdocjob{#1}
  \input{#1}
}
%    \end{macrocode}

% \macro{\childdocby}
% The command |\childdocby| ....
%    \begin{macrocode}
\newcommand{\childdocby}[2][]
{
  \childdocdisable
  \childdoctrue
  \childdocmanualtrue
  \if?#1?\else
    \def\jobname{#2}
  \fi
  \def\childdocjob{#2}
  \input{#2}
  \endinput
}
%    \end{macrocode}

% \macro{\childdocforward}
% The command |\childdocforward| redirects
% compilation to the main file or
% (if the optional argument is given) a child file.
% Parameters are set as if the main file
% or a child file starting with |\childdocof| was compiled.
% Then compilation is handed over to the main file:
%    \begin{macrocode}
\newcommand{\childdocforward}[2][]
{
  \begingroup
    \if?#1?
      \def\childdoctmp
      {
        \def\childdocname{#2}
        \def\childdocjob{#2}
        \def\jobname{#2}
        \input{#2}
        \endinput
      }
    \else
      \def\childdoctmp
      {
        \childdocdisable
        \def\childdocname{#2}
        \childdoctrue
        \includeonly{#2}
        \def\childdocjob{#1}
        \def\jobname{#1}
        \input{#1}
        \endinput
      }
    \fi
    \expandafter
  \endgroup
  \childdoctmp
}
%    \end{macrocode}

% \macro{\childdocforwardprefix}
% The command |\childdocforwardprefix| redirects
% compilation to the main or a child file by means of a pattern.
% The prefix |#1| in the current filename is replaced by |#2|
% and the suffix of the current filename is kept
% (it is assumed that the filename does not contain the substring `|~~~|'
% which is used as a delimiter).
% Compilation is handed over to the new file by |\childdocforward|:
%    \begin{macrocode}
\newcommand{\childdocforwardprefix}[3][]
{
  \begingroup
    \def\childdocextract #2##1~~~{\def\childdoctmp{\childdocforward[#1]{#3##1}}}
    \expandafter\childdocextract\childdocname~~~
    \expandafter
  \endgroup
  \childdoctmp
}
%    \end{macrocode}

% \macro{\childdoc}
% The deprecated macro |\childdoc| is a legacy version of |\childdocmain|:
%    \begin{macrocode}
\newcommand{\childdoc}{\childdocmain}
%    \end{macrocode}

% \macro{\childdocredirect}
% The deprecated macro |\childdocredirect| is a legacy version
% of |\childdocforward| and |\childdocforwardprefix|:
%    \begin{macrocode}
\newcommand{\childdocredirect}[2][]
{
  \begingroup
    \if?#1?
      \def\childdoctmp{\childdocforward{#2}}
    \else
      \def\childdoctmp{\childdocforwardprefix{#1}{#2}}
    \fi
    \expandafter
  \endgroup
  \childdoctmp
}
%    \end{macrocode}

%\iffalse
%</package>
%\fi
%
\endinput

\childdocby{cdocsamp}
%    \end{macrocode}

%\iffalse
%</samplepart3|samplepart4>
%\fi
%
%\iffalse
%<*samplepart3>
%\fi
% Some text for part 3:
%    \begin{macrocode}
some text in part three
%    \end{macrocode}

%\iffalse
%</samplepart3>
%\fi
% Some text for part 4:
%\iffalse
%<*samplepart4>
%\fi
%    \begin{macrocode}
more text in part four
%    \end{macrocode}

%\iffalse
%</samplepart4>
%\fi
%
% %%%%%%%%%%%%%%%%%%%%%%%%%%%%%%%%%%%%%%
% \paragraph{Forwarding for a Complete Draft.}
%
% The following forwarding file |cdocsdrf.tex|
% compiles the main document in draft mode:
%\iffalse
%<*sampledraft>
%\fi
%    \begin{macrocode}
\def\version{draft}
% \iffalse
%
% childdoc.dtx Copyright (C) 2017-2018 Niklas Beisert
%
% This work may be distributed and/or modified under the
% conditions of the LaTeX Project Public License, either version 1.3
% of this license or (at your option) any later version.
% The latest version of this license is in
%   http://www.latex-project.org/lppl.txt
% and version 1.3 or later is part of all distributions of LaTeX
% version 2005/12/01 or later.
%
% This work has the LPPL maintenance status `maintained'.
%
% The Current Maintainer of this work is Niklas Beisert.
%
% This work consists of the files childdoc.dtx and childdoc.ins
% and the derived files childdoc.def and cdocsamp.tex with
% cdocsch1.tex, cdocsch2.tex, cdocsdrf.tex, cdocsfn1.tex, cdocsfn2.tex.
%
%<package>\ifdefined\childdocmain\endinput\fi
%<package>\ProvidesFile{childdoc.def}[2018/12/30 v2.0 child document driver]
%<samplemain>\ProvidesFile{cdocsamp.tex}[2018/12/30 v2.0 sample for childdoc]
%<*driver>
%\ProvidesFile{childdoc.drv}[2018/12/30 v2.0 childdoc reference manual file]
\PassOptionsToClass{10pt,a4paper}{article}
\documentclass{ltxdoc}

\usepackage[margin=35mm]{geometry}
\usepackage{hyperref}
\usepackage{hyperxmp}
\usepackage[usenames]{color}

\hypersetup{colorlinks=true}
\hypersetup{pdfstartview=FitH}
\hypersetup{pdfpagemode=UseNone}
\hypersetup{pdfsource={}}
\hypersetup{pdflang={en-UK}}
\hypersetup{pdfcopyright={Copyright 2017-2018 Niklas Beisert.
  This work may be distributed and/or modified under the
  conditions of the LaTeX Project Public License, either version 1.3
  of this license or (at your option) any later version.}}
\hypersetup{pdflicenseurl={http://www.latex-project.org/lppl.txt}}
\hypersetup{pdfcontactaddress={ETH Zurich, ITP, HIT K,
  Wolfgang-Pauli-Strasse 27}}
\hypersetup{pdfcontactpostcode={8093}}
\hypersetup{pdfcontactcity={Zurich}}
\hypersetup{pdfcontactcountry={Switzerland}}
\hypersetup{pdfcontactemail={nbeisert@itp.phys.ethz.ch}}
\hypersetup{pdfcontacturl={http://people.phys.ethz.ch/\xmptilde nbeisert/}}

\newcommand{\secref}[1]{\hyperref[#1]{section \ref*{#1}}}

\parskip1ex
\parindent0pt
\let\olditemize\itemize
\def\itemize{\olditemize\parskip0pt}

\begin{document}

\title{The \textsf{childdoc} Package}
\hypersetup{pdftitle={The childdoc Package}}
\author{Niklas Beisert\\[2ex]
  Institut f\"ur Theoretische Physik\\
  Eidgen\"ossische Technische Hochschule Z\"urich\\
  Wolfgang-Pauli-Strasse 27, 8093 Z\"urich, Switzerland\\[1ex]
  \href{mailto:nbeisert@itp.phys.ethz.ch}
  {\texttt{nbeisert@itp.phys.ethz.ch}}}
\hypersetup{pdfauthor={Niklas Beisert}}
\hypersetup{pdfsubject={Manual for the LaTeX2e Package childdoc}}
\date{30 December 2018, \textsf{v2.0}}
\maketitle

\begin{abstract}\noindent
\textsf{childdoc} is a \LaTeXe{} package
that enables the direct compilation
of document sections included by |\include|
to individual files.
\end{abstract}

\begingroup
\parskip0ex
\tableofcontents
\endgroup

%%%%%%%%%%%%%%%%%%%%%%%%%%%%%%%%%%%%%%%%%%%%%%%%%%%%%%%%%%%%%%%%%%%%%%%%%%%%%%%%
%%%%%%%%%%%%%%%%%%%%%%%%%%%%%%%%%%%%%%%%%%%%%%%%%%%%%%%%%%%%%%%%%%%%%%%%%%%%%%%%
\section{Introduction}

\LaTeX{} provides a mechanism to structure a large document (such as a book)
into a main file and several child files (containing the chapters)
using the |\include| command.
This mechanism is beneficial for documents
which span hundreds of pages in order to
make the source file(s) more manageable.
Moreover, compilation can be restricted to
selected child files by means of the |\includeonly| command.
The latter feature can be used to reduce the compilation time while editing
(this was significantly more useful in the earlier days of \LaTeX{})
or to generate a smaller document which is easier to navigate.
Another application of |\includeonly| is to generate
documents consisting of selected parts of the complete document.

However, there are a few drawbacks of the plain |\include| mechanism:
\begin{itemize}
\item
The child files cannot be compiled on their own,
they can only be compiled via the main file.
A naive editing environment
(such as a text editor with an option
to have the current file processed by \LaTeX)
may require one to switch to the main file before compiling;
attempting to compile the child file produces errors.
\item
The main file must be modified (each time)
to adjust the |\includeonly| command
to the present needs. This easily leaves the main file in a messy state.
\item
The generated document will always carry the filename
of the main document. This is inconvenient if
several child files are to be compiled and
to be kept for distribution.
\end{itemize}

The present package provides a simple interface
to make child files individually compilable by \LaTeX{}.
Compiling a child file then has the same effect as compiling
the main file with an |\includeonly| command
to select the appropriate child.
Moreover the generated document will carry the name of the child
rather than the main file.
This resolves all three above issues.

This feature is meant to make the editing of books,
thesis documents and lecture notes somewhat more convenient.
However, the package can also be used efficiently for
composing a series of documents (such as exercise sheets)
which are typically distributed individually.
It then assists the author in generating the individual documents
(potentially in different versions)
as well as a document containing the collected series.
Another application is in developing style files
or other kinds of included material
where compilation of the style file could redirect
to a sample or test file.

%%%%%%%%%%%%%%%%%%%%%%%%%%%%%%%%%%%%%%%%%%%%%%%%%%%%%%%%%%%%%%%%%%%%%%%%%%%%%%%%
%%%%%%%%%%%%%%%%%%%%%%%%%%%%%%%%%%%%%%%%%%%%%%%%%%%%%%%%%%%%%%%%%%%%%%%%%%%%%%%%
\section{Usage}

First of all, the package \textsf{childdoc} is \emph{not} a standard
\LaTeXe{} |.sty| style file! Therefore it needs to be invoked in
a non-standard way.

%%%%%%%%%%%%%%%%%%%%%%%%%%%%%%%%%%%%%%%%%%%%%%%%%%%%%%%%%%%%%%%%%%%%%%%%%%%%%%%%
\subsection{Included Files}
\label{sec:include}

%%%%%%%%%%%%%%%%%%%%%%%%%%%%%%%%%%%%%%%%
\DescribeMacro{\childdocmain}
To use the package, add the commands
\begin{center}
\begin{tabular}{l}
|\input{childdoc.def}|\\
|\childdocmain{}|\\
\end{tabular}
\end{center}
at the very top of the main \LaTeX{} file,
in particular \emph{before} the |\documentclass| statement!
The argument of |\childdocmain| should be left empty
(but it must be present).

%%%%%%%%%%%%%%%%%%%%%%%%%%%%%%%%%%%%%%%%
\DescribeMacro{\childdocof}
Furthermore, add the commands
\begin{center}
\begin{tabular}{l}
|\input{childdoc.def}|\\
|\childdocof{|\textit{main}|}|\\
\end{tabular}
\end{center}
at the top of every child file \textit{child}
which is included by |\include{|\textit{child}|}|
from within the main file
(or at least for those files to be compiled individually).
The argument \textit{main} must be the filename of the main file.

There are a couple of
considerations in setting up the main and child documents:

%%%%%%%%%%%%%%%%%%%%%%%%%%%%%%%%%%%%%%%%
\paragraph{Restrictions.}

Please note the following restrictions:
\begin{itemize}
\item
|\childdocmain| must be called with one argument \textit{main}
to ensure compatibility with earlier version of the package.
It must either be empty (|\childdocmain{}|)
or precisely match the filename of the main file in which it is specified.
See \secref{sec:detection} for further information.
\item
The filename \textit{main} must be specified without the |.tex| extension.
\item
The filename \textit{main} is case sensitive
(even in case-insensitive file systems)
due to internal string comparison.
\item
The argument \textit{main} should be fully expanded, it cannot be a macro.
\item
Subdirectories and special characters should be avoided in filenames.
\item
The command |\childdocmain{|\textit{main}|}| must be followed by a whitespace.
It should not be followed immediately by another command
or by a comment mark `|%|'.
This is because the \TeX{} parser reads the token immediately following
the argument of |\childdocmain| and puts it
at the beginning of every child section;
however, a white\-space is ignored.
\end{itemize}

%%%%%%%%%%%%%%%%%%%%%%%%%%%%%%%%%%%%%%%%
\paragraph{Content of Main File.}

It is advisable to place all content in the child files included by |\include|.
Any output contained in the main file will appear in all child documents
unless suppressed manually;
it cannot be suppressed automatically by the |\includeonly| directive
and thus should normally be avoided.
A method to include some content in the main file
by means of conditional processing is described in \secref{sec:conditional}.

%%%%%%%%%%%%%%%%%%%%%%%%%%%%%%%%%%%%%%%%
\paragraph{Page Numbering.}

When only a part of the document is compiled,
the appropriate numbering of pages
(as well as other status parameters)
is determined from the |.aux| files.
The latter contain information from previous passes.
However this information needs to propagate through
all intermediate child documents.
Therefore the page numbering in child documents may well
be inconsistent until the complete document is compiled at least once.

A useful (if unconventional) way to always ensure a consistent
page numbering is to restart the numbering in each child document
and denote the pages by `\textit{child}|.|\textit{page}'
where \textit{child} represents the chapter/section number of the child file.
This can be achieved by the command
|\numberwithin{page}{|\textit{child}|}|
of the \textsf{amsmath} package
where \textit{child} can be |chapter| or |section|
depending on the chosen structuring.
Alternatively, one can modify the macro |\thepage| appropriately
and reset the counter |page| at the start of each child file.

%%%%%%%%%%%%%%%%%%%%%%%%%%%%%%%%%%%%%%%%%%%%%%%%%%%%%%%%%%%%%%%%%%%%%%%%%%%%%%%%
\subsection{Conditional Processing}
\label{sec:conditional}

The package provides a mechanism to compile different versions
of a document. To customise the versions further some conditional processing
can come in handy to distinguish which version is being compiled.
The package provides two macros to describe the compilation context:

%%%%%%%%%%%%%%%%%%%%%%%%%%%%%%%%%%%%%%%%
\DescribeMacro{\ifchilddoc}
The conditional |\ifchilddoc| distinguishes between the compilation of
child documents and the main document:
%
\begin{center}
|\ifchilddoc |\textit{child-code}| |[|\||else |\textit{main-code}]| \||fi|
\end{center}

%%%%%%%%%%%%%%%%%%%%%%%%%%%%%%%%%%%%%%%%
\DescribeMacro{\childdocname}
\DescribeMacro{\childdocjob}
The macro |\childdocname| contains the filename (without extension)
of the main or child file being processed.
Note that |\childdocjob| will always contain the name of the main file.

%%%%%%%%%%%%%%%%%%%%%%%%%%%%%%%%%%%%%%%%
\paragraph{Title Page.}

Conditional processing can be used to include a title or banner page
in the main document when proper precautions are taken.
Importantly, the code in the main file should ensure that the page counter
(as well as other status parameters which are stored in the |.aux| files)
takes the same value after the conditional processing.
Otherwise the page numbers may take divergent values
depending on which part is compiled.

For example, a title page could be declared by:
%
\begin{center}
\begin{tabular}{l}
|\ifchilddoc\||else|\\
|\addtocounter{page}{-1}|\\
\textit{code for title page}\\
|\newpage|\\
|\||fi|
\end{tabular}
\end{center}
%
A banner page for the child documents can be generated by:
%
\begin{center}
\begin{tabular}{l}
|\ifchilddoc|\\
|\addtocounter{page}{-1}|\\
\textit{code for banner page}\\
|\newpage|\\
|\||fi|
\end{tabular}
\end{center}
%
Here one could write a message such as:
\begin{center}
|This is the part \childdocname{} of \childdocjob{}.|
\end{center}

%%%%%%%%%%%%%%%%%%%%%%%%%%%%%%%%%%%%%%%%%%%%%%%%%%%%%%%%%%%%%%%%%%%%%%%%%%%%%%%%
\subsection{Flags}
\label{sec:flags}

The package makes it easy to generate different versions
of the main or child documents.
To this end compilation flags can be defined
and assigned different default values.
They will be particularly useful in conjunction
with the forwarding mechanism described in \secref{sec:forward}.

For example, it may be useful to have a flag |\version|
which can be set to |draft| or |final|.
The document source will contain some conditional code
depending on the value of |\version|.
Suppose further, the flag should default to |final| for the main file
and to |draft| for child files
which is a natural assignment for editing the document.
This is achieved by placing the following code
in the preamble of the main document
(below the |\childdocmain| directive):
%
\begin{center}
\begin{tabular}{l}
|\ifchilddoc|\\
|\providecommand{\version}{draft}|\\
|\||else|\\
|\providecommand{\version}{final}|\\
|\||fi|
\end{tabular}
\end{center}
%
The definition by |\providecommand| makes sure
that previous definitions are not overwritten.
Further statements |\providecommand{\version}{...}|
can thus be added before the above code to override it.

For the main file, one might add a line
(between |\childdocmain| and the above block)
%
\begin{center}
|%\ifchilddoc\||else\providecommand{\version}{draft}\||fi|
\end{center}
%
which can be uncommented to produce a draft version.
Likewise one can add a line to the very top of a child file
(above the |\childdocof{|\textit{main}|}| directive)
%
\begin{center}
|%\providecommand{\version}{final}|
\end{center}
%
which can be uncommented to produce the final version of this child document.

%%%%%%%%%%%%%%%%%%%%%%%%%%%%%%%%%%%%%%%%%%%%%%%%%%%%%%%%%%%%%%%%%%%%%%%%%%%%%%%%
\subsection{Forwarding}
\label{sec:forward}

Different versions of the main or child documents
using compilation flags as described in \secref{sec:flags}
can be (permanently) stored in different files
for convenient compilation, viewing and distribution.
To this end, the package defines a command
to pass on compilation to a different file:

%%%%%%%%%%%%%%%%%%%%%%%%%%%%%%%%%%%%%%%%
\DescribeMacro{\childdocforward}
The command |\childdocforward| redirects processing to
another source file:
%
\begin{center}
\begin{tabular}{l}
|\input{childdoc.def}|\\
|\childdocforward[|\textit{main}|]{|\textit{dest}|}|\\
\end{tabular}
\end{center}
%
The argument \textit{dest} is the destination file
(without extension).
It should be the main file or one of the child files.
Note that further \textsf{childdoc} directives
such as |\childdocof| and |\childdocforward|
in the indicated file will be processed in this form.
The optional argument \textit{main}
passes on directly to the main file \textit{main}
while pretending to compile the child \textit{dest}.
This form behaves as if \textit{dest}
issues |\childdocof{|\textit{main}|}| right away,
and no further \textsf{childdoc} directives will be processed.

%%%%%%%%%%%%%%%%%%%%%%%%%%%%%%%%%%%%%%%%
\DescribeMacro{\...prefix}
In the alternative form |\childdocforwardprefix|,
%
\begin{center}
\begin{tabular}{l}
|\input{childdoc.def}|\\
|\childdocforwardprefix[|\textit{main}|]{|\textit{prefix}|}{|\textit{dest}|}|
\end{tabular}
\end{center}
%
the destination file is determined by a pattern
depending on the current file:
To make this work, the current file must be called
`{\textit{prefix}\hspace{0.2em}\textit{suffix}}'
with \textit{prefix} matching precisely the argument.
Processing is then passed on to the file
`{\textit{dest}\hspace{0.2em}\textit{suffix}}'.
Surely, the same effect is achieved by
directly specifying the
argument `{\textit{dest}\hspace{0.2em}\textit{suffix}}'
in the first form.
However, that requires to set up a different file
for each child. With the alternative form of the command
all these files can have exactly the same content
which simplifies setting them up and maintaining them.

For example, the following file |draft.tex|
with a compilation flag |\version| as described in \secref{sec:flags}
compiles the main document as a draft:
%
\begin{center}
\begin{tabular}{l}
|\def\version{draft}|\\
|\input{childdoc.def}|\\
|\childdocforward{|\textit{main}|}|
\end{tabular}
\end{center}
%
Likewise, the following files |final|\textit{nn}|.tex|
compile the final version of the child document
|child|\textit{nn}|.tex|:
%
\begin{center}
\begin{tabular}{l}
|\def\version{final}|\\
|\input{childdoc.def}|\\
|\childdocforwardprefix{final}{child}|
\end{tabular}
\end{center}
%

Note that when several versions of a main file and/or of each child file
are to be generated, it may be convenient to set up a |Makefile| or
shell script to automatise the process.

%%%%%%%%%%%%%%%%%%%%%%%%%%%%%%%%%%%%%%%%%%%%%%%%%%%%%%%%%%%%%%%%%%%%%%%%%%%%%%%%
\subsection{Command Line Processing}
\label{sec:commandline}

The effect of redirection files can also be achieved by invoking
the \LaTeX{} compiler with a more elaborate command line.
Most conveniently this should be done as part
of a shell script or a |Makefile|.

When using \textsf{childdoc} in the main file, the following
command lines effectively perform a redirection
(note that depending on the shell being used,
backslashes may have to be doubled: `|\|' $\to$ `|\\|'):
%
\begin{center}
|... -jobname "|\textit{target}|" |\\|"|[\textit{flags}]%
|\input{childdoc.def}\childdocforward[|\textit{main}|]{|\textit{dest}|}"|
\end{center}
%
Here \textit{target} is the name of the output file,
\textit{main} is the name of the main file
and \textit{dest} is the name of the main or child file to be processed
(all filenames without extensions).
The optional argument \textit{main} can be omitted
if \textit{main} matches \textit{dest}.
Optionally, compilation \textit{flags} can be defined via |\def| commands.
This command line makes the \TeX{} engine believe
it is compiling the file \textit{target}
whose content is specified as the latter parameter.
The provided code then forwards the processing to
\textit{main} or \textit{dest} as described in \secref{sec:forward}.

%%%%%%%%%%%%%%%%%%%%%%%%%%%%%%%%%%%%%%%%%%%%%%%%%%%%%%%%%%%%%%%%%%%%%%%%%%%%%%%%
\subsection{Include by Input}
\label{sec:input}

Including child documents by |\include| has some restrictions by design.
Most notably, the content of a child document always occupies
its own set of pages; pages cannot be shared between child documents.
Usually, this behaviour makes perfect sense
because each child document contain an essential part of the document.
However, in some situations it may be desirable to compose
a document from a collection of parts
without having mandatory page breaks between then.
For this case, the package
provides a mechanism to include parts
by |\input| which can also be processed individually.
However, by construction this mechanism
requires manual handling of the content to be output.

%%%%%%%%%%%%%%%%%%%%%%%%%%%%%%%%%%%%%%%%
\DescribeMacro{\ifchilddocmanual}
The main file should be prepared as usual, see \secref{sec:include}.
However, the document body must make a distinction
between processing of an individual part and of the main document, e.g.:
%
\begin{center}
\begin{tabular}{l}
|\ifchilddocmanual|\\
|\input{\childdocname}|\\
|\||else|\\
\textit{document body with }|\input{|\textit{part}|}|\\
|\||fi|
\end{tabular}
\end{center}
%
The conditional |\ifchilddocmanual| is true whenever
a part to be included by |\input| is being compiled,
and the name of the part is stored in |\childdocname|.

%%%%%%%%%%%%%%%%%%%%%%%%%%%%%%%%%%%%%%%%
\DescribeMacro{\childdocby}
Each part to be included by |\input| should start with:
%
\begin{center}
\begin{tabular}{l}
|\input{childdoc.def}|\\
|\childdocby{|\textit{main}|}|\\
\end{tabular}
\end{center}
%
The directive |\childdocby| is similar to |\childdocof|
described in \secref{sec:include},
but the subsequent selection of content must be done manually.
To that end, both |\ifchilddoc| and |\ifchilddocmanual|
will be true upon processing of a part,
and the name of the part is stored in |\childdocname|.
Note that |\jobname| will be set to the filename of the current part
so that each part receives an individual |.aux| file
that does not interfere with the |.aux| file(s) of the main document.
This behaviour can be altered by the alternative form
|\childdocby[*]{|\textit{main}|}| (with a non-empty optional argument)
which uses the |.aux| file of the main document
by setting |\jobname| to \textit{main}.

%%%%%%%%%%%%%%%%%%%%%%%%%%%%%%%%%%%%%%%%%%%%%%%%%%%%%%%%%%%%%%%%%%%%%%%%%%%%%%%%
\subsection{Driver Development}
\label{sec:driver}

The \textsf{childdoc} mechanism can also be use for the development
of definition files such as \LaTeX{} styles or classes.
This case differs from the above setup with multiple parts
included by |\include| in that no |\includeonly| should be invoked.
This can be achieved by starting the include file
(before |\ProvidesPackage|) with:
%
\begin{center}
\begin{tabular}{l}
|\input{childdoc.def}|\\
|\childdocforward{|\textit{main}|}|\\
\end{tabular}
\end{center}
%
or alternatively with:
%
\begin{center}
\begin{tabular}{l}
|\input{childdoc.def}|\\
|\childdocby{|\textit{main}|}|\\
\end{tabular}
\end{center}
%
Both forms have slightly different effects as described above.
The main file is prepared as usual, see \secref{sec:include}.

%%%%%%%%%%%%%%%%%%%%%%%%%%%%%%%%%%%%%%%%%%%%%%%%%%%%%%%%%%%%%%%%%%%%%%%%%%%%%%%%
\subsection{Legacy Detection}
\label{sec:detection}

The directive |\childdocmain| in the main file can detect
whether the complete document or merely a child is to be compiled
even without using the directive |\childdocof|.
This method is deprecated because it is less robust
and there is no compelling reason to use it;
it is merely provided for backward compatibility
and it may be removed in future versions.

If the detection mechanism is to be used,
it is mandatory to correctly specify
the filename of the main file as the argument of |\childdocmain|:
%
\begin{center}
\begin{tabular}{l}
|\input{childdoc.def}|\\
|\childdocmain{|\textit{main}|}|\\
\end{tabular}
\end{center}
%
If |\jobname| does not match the argument \textit{main} of |\childdocmain|,
it is assumed that |\jobname| points to the child file to be compiled.
When using |\childdocmain| with the main file specified as argument,
it suffices to start a child file
with just |\input{|\textit{main}|}|
without loading of the package and using |\childdocof|.
If instead all processing is done
with the appropriate \textsf{childdoc} directives,
the argument of \textit{main} of |\childdocmain| can be empty.

An alternative version of the command line processing described
in \secref{sec:commandline} using the detection mechanism reads:
%
\begin{center}
|... -jobname "|\textit{target}|" "|[\textit{flags}]%
[|\def\jobname{|\textit{dest}|}|]|\input{|\textit{main}|}"|
\end{center}

%%%%%%%%%%%%%%%%%%%%%%%%%%%%%%%%%%%%%%%%%%%%%%%%%%%%%%%%%%%%%%%%%%%%%%%%%%%%%%%%
\subsection{Manual Code}
\label{sec:manual}

In case one cannot be certain whether the definitions file |childdoc.def|
is installed on the target \TeX{} distribution
and one prefers not to ship it,
it is conceivable to paste a few relevant commands into the sources.

To that end, drop all statements |\input{childdoc.def}|
and perform the replacements as outlined below.
Instead of |\childdocmain{|\textit{main}|}| add the following code
to the top of the main file:
%
\begin{center}
\begin{tabular}{l}
|\||ifdefined\childdocname\endinput\||fi\newif\ifchilddoc|\\
|\edef\childdocname{\scantokens\expandafter{\jobname\noexpand}}|\\
|\def\childdocmain{|\textit{main}|}\||ifx\childdocmain\childdocname\||else|\\
|\childdoctrue\includeonly{\childdocname}\let\jobname\childdocmain\||fi|\\
\end{tabular}
\end{center}
%
Instead of |\childdocof{|\textit{main}|}| just include the main file
at the top of each child file:
%
\begin{center}
|\input{|\textit{main}|}|
\end{center}
%
A simple redirection |\childdocforward{|\textit{dest}|}| is achieved by:
%
\begin{center}
|\def\jobname{|\textit{dest}|}\input{\jobname}|
\end{center}
%
The redirection with prefix
|\childdocforwardprefix[|\textit{prefix}|]{|\textit{dest}|}|
is accomplished by:
%
\begin{center}
\begin{tabular}{l}
|{\edef\jobname{\scantokens\expandafter{\jobname\noexpand}}|\\
|\def\redirectjob |\textit{prefix}|#1~~~{\gdef\jobname{|\textit{dest}|#1}}|\\
|\expandafter\redirectjob\jobname~~~}\input{\jobname}|
\end{tabular}
\end{center}

In an alternative approach,
child documents can be compiled by a specific command line
without additional code or specific definitions:
%
\begin{center}
|... -jobname "|\textit{target}|" "|[\textit{flags}]%
|\includeonly{|\textit{dest}|}\input{|\textit{main}|}"|
\end{center}
%

%%%%%%%%%%%%%%%%%%%%%%%%%%%%%%%%%%%%%%%%%%%%%%%%%%%%%%%%%%%%%%%%%%%%%%%%%%%%%%%%
%%%%%%%%%%%%%%%%%%%%%%%%%%%%%%%%%%%%%%%%%%%%%%%%%%%%%%%%%%%%%%%%%%%%%%%%%%%%%%%%
\section{Information}

%%%%%%%%%%%%%%%%%%%%%%%%%%%%%%%%%%%%%%%%%%%%%%%%%%%%%%%%%%%%%%%%%%%%%%%%%%%%%%%%
\subsection{Copyright}

Copyright \copyright{} 2017--2018 Niklas Beisert

This work may be distributed and/or modified under the
conditions of the \LaTeX{} Project Public License, either version 1.3
of this license or (at your option) any later version.
The latest version of this license is in
  \url{http://www.latex-project.org/lppl.txt}
and version 1.3 or later is part of all distributions of \LaTeX{}
version 2005/12/01 or later.

This work has the LPPL maintenance status `maintained'.

The Current Maintainer of this work is Niklas Beisert.

This work consists of the files |README.txt|, |childdoc.ins| and |childdoc.dtx|
as well as the derived files |childdoc.def|, |cdocsamp.tex|
with |cdocsch1.tex|, |cdocsch2.tex|, |cdocspt3.tex|, |cdocspt4.tex|,
|cdocsdrf.tex|, |cdocsfn1.tex|, |cdocsfn2.tex|
as well as |childdoc.pdf|.

%%%%%%%%%%%%%%%%%%%%%%%%%%%%%%%%%%%%%%%%%%%%%%%%%%%%%%%%%%%%%%%%%%%%%%%%%%%%%%%%
\subsection{Files and Installation}

The package consists of the files:
%
\begin{center}
\begin{tabular}{ll}
    |README.txt|   & readme file \\
    |childdoc.ins| & installation file \\
    |childdoc.dtx| & source file \\
    |childdoc.def| & definition file \\
    |cdocsamp.tex| & sample main file \\
    |cdocsch1.tex| & sample include file \\
    |cdocsch2.tex| & sample include file \\
    |cdocspt3.tex| & sample part file \\
    |cdocspt4.tex| & sample part file \\
    |cdocsdrf.tex| & sample redirection file \\
    |cdocsfn1.tex| & sample redirection file \\
    |cdocsfn2.tex| & sample redirection file \\
    |childdoc.pdf| & manual
\end{tabular}
\end{center}
%
The distribution consists of the files
|README.txt|, |childdoc.ins| and |childdoc.dtx|.
%
\begin{itemize}
\item
Run (pdf)\LaTeX{} on |childdoc.dtx|
to compile the manual |childdoc.pdf| (this file).
\item
Run \LaTeX{} on |childdoc.ins| to create the definitions file |childdoc.def|
and the sample |cdocsamp.tex| with include files
|cdocsch1.tex|, |cdocsch2.tex|, |cdocspt3.tex|, |cdocspt4.tex|,
|cdocsdrf.tex|, |cdocsfn1.tex|, |cdocsfn2.tex|.
Then copy the file |childdoc.def| to an appropriate directory of your \LaTeX{}
distribution, e.g.\ \textit{texmf-root}|/tex/latex/childdoc|.
\end{itemize}

%%%%%%%%%%%%%%%%%%%%%%%%%%%%%%%%%%%%%%%%%%%%%%%%%%%%%%%%%%%%%%%%%%%%%%%%%%%%%%%%
\subsection{Related CTAN Packages}

There are several other packages which offer a similar functionality:
%
\begin{itemize}
\item
The packages
\href{http://ctan.org/pkg/docmute}{\textsf{docmute}},
\href{http://ctan.org/pkg/includex}{\textsf{includex}} and
\href{http://ctan.org/pkg/standalone}{\textsf{standalone}}
provide commands to include only the document body of
a child file thus allowing both files to be compiled individually.
\item
The packages \href{http://ctan.org/pkg/subdocs}{\textsf{subdocs}}
and \href{http://ctan.org/pkg/subfiles}{\textsf{subfiles}}
provide structures in which the main and child documents can be
encapsulated and allowing them to be compiled individually.
The inclusion mechanism is different from the conventional |\include|.
\item
The package \href{http://ctan.org/pkg/combine}{\textsf{combine}}
is an elaborate solution to combine several documents into one.
\end{itemize}
%
See also the CTAN topic \href{http://ctan.org/topic/subdocs}{\textsf{subdocs}}
for further related packages.
The present package differs from the above solutions in that
a document structure constructed with the conventional |\include| mechanism
just needs two extra commands at the top of every file
such that all constituent files can be compiled individually.

%%%%%%%%%%%%%%%%%%%%%%%%%%%%%%%%%%%%%%%%%%%%%%%%%%%%%%%%%%%%%%%%%%%%%%%%%%%%%%%%
%\subsection{Feature Suggestions}
%
%The following is a list of features which may be useful for future
%versions of this package:
%%
%\begin{itemize}
%\item
%\ldots
%\end{itemize}

%%%%%%%%%%%%%%%%%%%%%%%%%%%%%%%%%%%%%%%%%%%%%%%%%%%%%%%%%%%%%%%%%%%%%%%%%%%%%%%%
\subsection{Revision History}

%%%%%%%%%%%%%%%%%%%%%%%%%%%%%%%%%%%%%%%%
\paragraph{v2.0:} 2018/12/30

\begin{itemize}
\item
immediate forward processing
\item
added |\childdocby| mechanism
\item
manual restructured
\end{itemize}

%%%%%%%%%%%%%%%%%%%%%%%%%%%%%%%%%%%%%%%%
\paragraph{v1.6:} 2018/01/17

\begin{itemize}
\item
application for development of include files
\item
corrections to manual
\end{itemize}

%%%%%%%%%%%%%%%%%%%%%%%%%%%%%%%%%%%%%%%%
\paragraph{v1.5:} 2017/05/21

\begin{itemize}
\item
more complete structuring introduced
\item
|\childdocof| introduced
\item
|\childdoc| renamed to |\childdocmain|
\item
|\childredirect| renamed to |\childdocforward| and |\childdocforwardprefix|
and functionality expanded
\end{itemize}

%%%%%%%%%%%%%%%%%%%%%%%%%%%%%%%%%%%%%%%%
\paragraph{v1.0:} 2017/04/27

\begin{itemize}
\item
manual and install package
\item
first version published on CTAN
\end{itemize}

%%%%%%%%%%%%%%%%%%%%%%%%%%%%%%%%%%%%%%%%
\paragraph{v0.6:} 2017/04/26

\begin{itemize}
\item
redirection mechanism added
\end{itemize}

%%%%%%%%%%%%%%%%%%%%%%%%%%%%%%%%%%%%%%%%
\paragraph{v0.5:} 2017/04/26

\begin{itemize}
\item
functionality in definition file
\end{itemize}


%%%%%%%%%%%%%%%%%%%%%%%%%%%%%%%%%%%%%%%%%%%%%%%%%%%%%%%%%%%%%%%%%%%%%%%%%%%%%%%%
%%%%%%%%%%%%%%%%%%%%%%%%%%%%%%%%%%%%%%%%%%%%%%%%%%%%%%%%%%%%%%%%%%%%%%%%%%%%%%%%
%%%%%%%%%%%%%%%%%%%%%%%%%%%%%%%%%%%%%%%%%%%%%%%%%%%%%%%%%%%%%%%%%%%%%%%%%%%%%%%%
\appendix

\settowidth\MacroIndent{\rmfamily\scriptsize 000\ }

 \DocInput{childdoc.dtx}

\end{document}
%</driver>
% \fi
%
% %%%%%%%%%%%%%%%%%%%%%%%%%%%%%%%%%%%%%%%%%%%%%%%%%%%%%%%%%%%%%%%%%%%%%%%%%%%%%%
% %%%%%%%%%%%%%%%%%%%%%%%%%%%%%%%%%%%%%%%%%%%%%%%%%%%%%%%%%%%%%%%%%%%%%%%%%%%%%%
% \section{Sample}
%\iffalse
%<*samplemain>
%\fi
%
% The following presents a sample document
% with two chapters, two parts, a title page,
% a compile flag as well as three forwarding files to set the flag.
% It consists of eight |.tex| files:
% \begin{center}
% \begin{tabular}{ll}
% |cdocsamp.tex|&main file\\
% |cdocsch1.tex|&include file for chapter 1\\
% |cdocsch2.tex|&include file for chapter 2\\
% |cdocspt3.tex|&include file for part 3\\
% |cdocspt4.tex|&include file for part 4\\
% |cdocsdrf.tex|&forwarding file for main file in draft mode\\
% |cdocsfi1.tex|&forwarding file for final version of chapter 1\\
% |cdocsfi2.tex|&forwarding file for final version of chapter 2\\
% \end{tabular}
% \end{center}
% Each of the eight files can be compiled directly by the \LaTeX{} compiler.
%
% %%%%%%%%%%%%%%%%%%%%%%%%%%%%%%%%%%%%%%
% \paragraph{Main File.}
%
% The main file is called |cdocsamp.tex|.
%
% Load the \textsf{childdoc} definitions and
% declare the filename for the main document:
%    \begin{macrocode}
\input{childdoc.def}
\childdocmain{}
%    \end{macrocode}

% Optional override for |\version| flag:
%    \begin{macrocode}
%%\ifchilddoc\else\providecommand{\version}{draft}\fi
%    \end{macrocode}

% Define the default values for the |\version| flag
% (|final| for the main file and |draft| for childs):
%    \begin{macrocode}
\ifchilddoc
\providecommand{\version}{draft}
\else
\providecommand{\version}{final}
\fi
%    \end{macrocode}

% Load the standard document class:
%    \begin{macrocode}
\documentclass[12pt]{article}
%    \end{macrocode}

% Start the document body:
%    \begin{macrocode}
\begin{document}
%    \end{macrocode}

% Declare a title page.
% Print title, part of document being processed and version flag:
%    \begin{macrocode}
\addtocounter{page}{-1}
\begin{center}
{\LARGE\bfseries{}childdoc example\par}
\vspace{1cm}
\ifchilddoc
\ifchilddocmanual part\else chapter\fi:
`\childdocname' of `\childdocjob'\par
\else
main document: `\childdocjob'\par
\fi
version: \version\par
\end{center}
\newpage
%    \end{macrocode}

% Manually include selected file,
% otherwise process as usual:
%    \begin{macrocode}
\ifchilddocmanual
\section*{part `\childdocname'}
\input{\childdocname}
\else
%    \end{macrocode}

% Include the two chapters:
%    \begin{macrocode}
\include{cdocsch1}
\include{cdocsch2}
%    \end{macrocode}

% Include the two parts unless only chapters should be displayed:
%    \begin{macrocode}
\ifchilddoc\else
\section{part three}
\input{cdocspt3}
\section{part four}
\input{cdocspt4}
\fi
%    \end{macrocode}

% Process as usual until here:
%    \begin{macrocode}
\fi
%    \end{macrocode}

% End of document body:
%    \begin{macrocode}
\end{document}
%    \end{macrocode}
%\iffalse
%</samplemain>
%\fi
%
% %%%%%%%%%%%%%%%%%%%%%%%%%%%%%%%%%%%%%%
% \paragraph{Chapter Include Files.}
%
% The include files are called |cdocsch1.tex| and |cdocsch2.tex|.
%
%\iffalse
%<*samplechap1|samplechap2>
%\fi

% Optional override for |\version| flag:
%    \begin{macrocode}
%%\providecommand{\version}{final}
%    \end{macrocode}

% Include the main document:
%    \begin{macrocode}
\input{childdoc.def}
\childdocof{cdocsamp}
%    \end{macrocode}

%\iffalse
%</samplechap1|samplechap2>
%\fi
%
%\iffalse
%<*samplechap1>
%\fi
% Some text for chapter 1:
%    \begin{macrocode}
\section{one}
some text in chapter one
%    \end{macrocode}

%\iffalse
%</samplechap1>
%\fi
% Some text for chapter 2:
%\iffalse
%<*samplechap2>
%\fi
%    \begin{macrocode}
\section{two}
more text in chapter two
%    \end{macrocode}

%\iffalse
%</samplechap2>
%\fi
%
% %%%%%%%%%%%%%%%%%%%%%%%%%%%%%%%%%%%%%%
% \paragraph{Part Include Files.}
%
% The include files are called |cdocspt3.tex| and |cdocspt4.tex|.
%
%\iffalse
%<*samplepart3|samplepart4>
%\fi

% Optional override for |\version| flag:
%    \begin{macrocode}
%%\providecommand{\version}{final}
%    \end{macrocode}

% Include the main document:
%    \begin{macrocode}
\input{childdoc.def}
\childdocby{cdocsamp}
%    \end{macrocode}

%\iffalse
%</samplepart3|samplepart4>
%\fi
%
%\iffalse
%<*samplepart3>
%\fi
% Some text for part 3:
%    \begin{macrocode}
some text in part three
%    \end{macrocode}

%\iffalse
%</samplepart3>
%\fi
% Some text for part 4:
%\iffalse
%<*samplepart4>
%\fi
%    \begin{macrocode}
more text in part four
%    \end{macrocode}

%\iffalse
%</samplepart4>
%\fi
%
% %%%%%%%%%%%%%%%%%%%%%%%%%%%%%%%%%%%%%%
% \paragraph{Forwarding for a Complete Draft.}
%
% The following forwarding file |cdocsdrf.tex|
% compiles the main document in draft mode:
%\iffalse
%<*sampledraft>
%\fi
%    \begin{macrocode}
\def\version{draft}
\input{childdoc.def}
\childdocforward{cdocsamp}
%    \end{macrocode}

%\iffalse
%</sampledraft>
%\fi
%
% %%%%%%%%%%%%%%%%%%%%%%%%%%%%%%%%%%%%%%
% \paragraph{Forwarding for Final Version of the Chapters.}
%
% The following forwarding files |cdocsfn1.tex| and |cdocsfn2.tex|
% (with identical content)
% compile the final versions of the child documents
% |cdocsch1.tex| and |cdocsch2.tex|, respectively:
%\iffalse
%<*samplefinal>
%\fi
%    \begin{macrocode}
\def\version{final}
\input{childdoc.def}
\childdocforwardprefix[cdocsamp]{cdocsfn}{cdocsch}
%    \end{macrocode}

%\iffalse
%</samplefinal>
%\fi
%
% %%%%%%%%%%%%%%%%%%%%%%%%%%%%%%%%%%%%%%
% \paragraph{Command Line Processing.}
%
% The following three command lines generate the output files
% |cdocscld|, |cdocscl1| and |cdocscl2|
% which should be identical to
% |cdocsdrf|, |cdocsch1| and |cdocsfn2|, respectively:
% \begin{center}
% \begin{tabular}{l}
% |latex -jobname cdocscld \|\\
% |  "\def\version{draft}\input{childdoc.def}\childdocforward{cdocsamp}"|\\
% |latex -jobname cdocscl1 \|\\
% |  "\input{childdoc.def}\childdocforward[cdocsamp]{cdocsch1}"|\\
% |latex -jobname cdocscl2 \|\\
% |  "\def\version{final}\input{childdoc.def}\childdocforward{cdocsch2}"|
% \end{tabular}
% \end{center}
% Note that the trailing backslash on each first line
% merely continues the input to the second line
% (for convenient cut ant paste).
% Furthermore, the command |latex| can be replaced by any
% of its alternative versions such as |pdflatex|.
%
% %%%%%%%%%%%%%%%%%%%%%%%%%%%%%%%%%%%%%%%%%%%%%%%%%%%%%%%%%%%%%%%%%%%%%%%%%%%%%%
% %%%%%%%%%%%%%%%%%%%%%%%%%%%%%%%%%%%%%%%%%%%%%%%%%%%%%%%%%%%%%%%%%%%%%%%%%%%%%%
% \section{Implementation}
%\iffalse
%<*package>
%\fi
%
% This section describes the definitions file |childdoc.def|.

% The definitions cannot be loaded using |\usepackage| or |\RequirePackage|
% which has a mechanism to prevent loading a style file more than once.
% When loading the definitions by means of |\input|
% multiple instances have to be prevented manually:
%\iffalse
%This code needs to be before the `\ProvidesFile' directive
%which is defined at the beginning of this file.
%Therefore it is also placed there and commented out here.
%</package>
%<*discard>
%\fi
%    \begin{macrocode}
\ifdefined\childdocmain\endinput\fi
%    \end{macrocode}
%\iffalse
%</discard>
%<*package>
%\fi
%
% \macro{\ifchilddoc}
% \macro{\ifchilddocmanual}
% The conditional |\ifchilddoc| tells whether a
% child (true) or main (false) document is being compiled.
% The conditional |\ifchilddocmanual| tells whether
% the |\includeonly| mechanism is used (false) or
% the selection of child files must be performed manually (true).
% The definitions initialise to false:
%    \begin{macrocode}
\newif\ifchilddoc
\newif\ifchilddocmanual
%    \end{macrocode}

% \macro{\childdocname}
% \macro{\childdocjob}
% The macro |\childdocname| stores the name of the main document
% to be compiled. The macro |\childdocjob| stores the name of
% the document on which the \LaTeX{} compiler was originally invoked.
% The content of |\jobname| cannot be compared
% to filenames specified in the source due to different catcodes.
% The following code rescans |\jobname|, stores the result
% in |\childdocname| and saves a copy in |\childdocjob|:
%    \begin{macrocode}
\edef\childdocname{\scantokens\expandafter{\jobname\noexpand}}
\let\childdocjob\childdocname
%    \end{macrocode}

% \macro{\childdocdisable}
% The macro |\childdocdisable| prevents the main file
% from being processed more than once.
% At this stage, the main document command |\childdocmain|
% is assumed to be called once again where it should do nothing.
% Any subsequent call to it should prevent
% a secondary processing of the main document
% It overwrites the forwarding commands
% |\childdocof| and |\childdocforward|
% with empty macros to prevent further inclusions of the main document:
%    \begin{macrocode}
\newcommand{\childdocdisable}
{
  \renewcommand{\childdocmain}[1]{\renewcommand{\childdocmain}[1]{\endinput}}
  \renewcommand{\childdocof}[1]{}
  \renewcommand{\childdocby}[2][]{}
  \renewcommand{\childdocforward}[2][]{}
  \renewcommand{\childdocdisable}{}
}
%    \end{macrocode}

% \macro{\childdocmain}
% The macro |\childdocmain| is to be called at the top of the main file
% with nothing or the main filename (without extension) as argument.
% First, it breaks loops.
% If the argument is not empty and does not match |\childdocname|
% (which is set by the first inclusion of |childdoc.def|),
% |\ifchilddoc| is set to true, |\includeonly| is applied to the child file
% and |\jobname| is set to the main file
% (for proper handling of |.aux| files):
%    \begin{macrocode}
\newcommand{\childdocmain}[1]
{
  \childdocdisable\childdocmain{}
  \if?#1?\else
    \begingroup
      \def\childdoctmp{#1}
      \ifx\childdoctmp\childdocname
        \def\childdoctmp{}
      \else
        \def\childdoctmp
        {
          \childdoctrue
          \includeonly{\childdocname}
          \def\childdocjob{#1}
          \def\jobname{#1}
        }
      \fi
      \expandafter
    \endgroup
    \childdoctmp
  \fi
}
%    \end{macrocode}

% \macro{\childdocof}
% The command |\childdocof| redirects
% compilation to the main file |#1|.
%    \begin{macrocode}
\newcommand{\childdocof}[1]
{
  \childdocdisable
  \childdoctrue
  \includeonly{\childdocname}
  \def\jobname{#1}
  \def\childdocjob{#1}
  \input{#1}
}
%    \end{macrocode}

% \macro{\childdocby}
% The command |\childdocby| ....
%    \begin{macrocode}
\newcommand{\childdocby}[2][]
{
  \childdocdisable
  \childdoctrue
  \childdocmanualtrue
  \if?#1?\else
    \def\jobname{#2}
  \fi
  \def\childdocjob{#2}
  \input{#2}
  \endinput
}
%    \end{macrocode}

% \macro{\childdocforward}
% The command |\childdocforward| redirects
% compilation to the main file or
% (if the optional argument is given) a child file.
% Parameters are set as if the main file
% or a child file starting with |\childdocof| was compiled.
% Then compilation is handed over to the main file:
%    \begin{macrocode}
\newcommand{\childdocforward}[2][]
{
  \begingroup
    \if?#1?
      \def\childdoctmp
      {
        \def\childdocname{#2}
        \def\childdocjob{#2}
        \def\jobname{#2}
        \input{#2}
        \endinput
      }
    \else
      \def\childdoctmp
      {
        \childdocdisable
        \def\childdocname{#2}
        \childdoctrue
        \includeonly{#2}
        \def\childdocjob{#1}
        \def\jobname{#1}
        \input{#1}
        \endinput
      }
    \fi
    \expandafter
  \endgroup
  \childdoctmp
}
%    \end{macrocode}

% \macro{\childdocforwardprefix}
% The command |\childdocforwardprefix| redirects
% compilation to the main or a child file by means of a pattern.
% The prefix |#1| in the current filename is replaced by |#2|
% and the suffix of the current filename is kept
% (it is assumed that the filename does not contain the substring `|~~~|'
% which is used as a delimiter).
% Compilation is handed over to the new file by |\childdocforward|:
%    \begin{macrocode}
\newcommand{\childdocforwardprefix}[3][]
{
  \begingroup
    \def\childdocextract #2##1~~~{\def\childdoctmp{\childdocforward[#1]{#3##1}}}
    \expandafter\childdocextract\childdocname~~~
    \expandafter
  \endgroup
  \childdoctmp
}
%    \end{macrocode}

% \macro{\childdoc}
% The deprecated macro |\childdoc| is a legacy version of |\childdocmain|:
%    \begin{macrocode}
\newcommand{\childdoc}{\childdocmain}
%    \end{macrocode}

% \macro{\childdocredirect}
% The deprecated macro |\childdocredirect| is a legacy version
% of |\childdocforward| and |\childdocforwardprefix|:
%    \begin{macrocode}
\newcommand{\childdocredirect}[2][]
{
  \begingroup
    \if?#1?
      \def\childdoctmp{\childdocforward{#2}}
    \else
      \def\childdoctmp{\childdocforwardprefix{#1}{#2}}
    \fi
    \expandafter
  \endgroup
  \childdoctmp
}
%    \end{macrocode}

%\iffalse
%</package>
%\fi
%
\endinput

\childdocforward{cdocsamp}
%    \end{macrocode}

%\iffalse
%</sampledraft>
%\fi
%
% %%%%%%%%%%%%%%%%%%%%%%%%%%%%%%%%%%%%%%
% \paragraph{Forwarding for Final Version of the Chapters.}
%
% The following forwarding files |cdocsfn1.tex| and |cdocsfn2.tex|
% (with identical content)
% compile the final versions of the child documents
% |cdocsch1.tex| and |cdocsch2.tex|, respectively:
%\iffalse
%<*samplefinal>
%\fi
%    \begin{macrocode}
\def\version{final}
% \iffalse
%
% childdoc.dtx Copyright (C) 2017-2018 Niklas Beisert
%
% This work may be distributed and/or modified under the
% conditions of the LaTeX Project Public License, either version 1.3
% of this license or (at your option) any later version.
% The latest version of this license is in
%   http://www.latex-project.org/lppl.txt
% and version 1.3 or later is part of all distributions of LaTeX
% version 2005/12/01 or later.
%
% This work has the LPPL maintenance status `maintained'.
%
% The Current Maintainer of this work is Niklas Beisert.
%
% This work consists of the files childdoc.dtx and childdoc.ins
% and the derived files childdoc.def and cdocsamp.tex with
% cdocsch1.tex, cdocsch2.tex, cdocsdrf.tex, cdocsfn1.tex, cdocsfn2.tex.
%
%<package>\ifdefined\childdocmain\endinput\fi
%<package>\ProvidesFile{childdoc.def}[2018/12/30 v2.0 child document driver]
%<samplemain>\ProvidesFile{cdocsamp.tex}[2018/12/30 v2.0 sample for childdoc]
%<*driver>
%\ProvidesFile{childdoc.drv}[2018/12/30 v2.0 childdoc reference manual file]
\PassOptionsToClass{10pt,a4paper}{article}
\documentclass{ltxdoc}

\usepackage[margin=35mm]{geometry}
\usepackage{hyperref}
\usepackage{hyperxmp}
\usepackage[usenames]{color}

\hypersetup{colorlinks=true}
\hypersetup{pdfstartview=FitH}
\hypersetup{pdfpagemode=UseNone}
\hypersetup{pdfsource={}}
\hypersetup{pdflang={en-UK}}
\hypersetup{pdfcopyright={Copyright 2017-2018 Niklas Beisert.
  This work may be distributed and/or modified under the
  conditions of the LaTeX Project Public License, either version 1.3
  of this license or (at your option) any later version.}}
\hypersetup{pdflicenseurl={http://www.latex-project.org/lppl.txt}}
\hypersetup{pdfcontactaddress={ETH Zurich, ITP, HIT K,
  Wolfgang-Pauli-Strasse 27}}
\hypersetup{pdfcontactpostcode={8093}}
\hypersetup{pdfcontactcity={Zurich}}
\hypersetup{pdfcontactcountry={Switzerland}}
\hypersetup{pdfcontactemail={nbeisert@itp.phys.ethz.ch}}
\hypersetup{pdfcontacturl={http://people.phys.ethz.ch/\xmptilde nbeisert/}}

\newcommand{\secref}[1]{\hyperref[#1]{section \ref*{#1}}}

\parskip1ex
\parindent0pt
\let\olditemize\itemize
\def\itemize{\olditemize\parskip0pt}

\begin{document}

\title{The \textsf{childdoc} Package}
\hypersetup{pdftitle={The childdoc Package}}
\author{Niklas Beisert\\[2ex]
  Institut f\"ur Theoretische Physik\\
  Eidgen\"ossische Technische Hochschule Z\"urich\\
  Wolfgang-Pauli-Strasse 27, 8093 Z\"urich, Switzerland\\[1ex]
  \href{mailto:nbeisert@itp.phys.ethz.ch}
  {\texttt{nbeisert@itp.phys.ethz.ch}}}
\hypersetup{pdfauthor={Niklas Beisert}}
\hypersetup{pdfsubject={Manual for the LaTeX2e Package childdoc}}
\date{30 December 2018, \textsf{v2.0}}
\maketitle

\begin{abstract}\noindent
\textsf{childdoc} is a \LaTeXe{} package
that enables the direct compilation
of document sections included by |\include|
to individual files.
\end{abstract}

\begingroup
\parskip0ex
\tableofcontents
\endgroup

%%%%%%%%%%%%%%%%%%%%%%%%%%%%%%%%%%%%%%%%%%%%%%%%%%%%%%%%%%%%%%%%%%%%%%%%%%%%%%%%
%%%%%%%%%%%%%%%%%%%%%%%%%%%%%%%%%%%%%%%%%%%%%%%%%%%%%%%%%%%%%%%%%%%%%%%%%%%%%%%%
\section{Introduction}

\LaTeX{} provides a mechanism to structure a large document (such as a book)
into a main file and several child files (containing the chapters)
using the |\include| command.
This mechanism is beneficial for documents
which span hundreds of pages in order to
make the source file(s) more manageable.
Moreover, compilation can be restricted to
selected child files by means of the |\includeonly| command.
The latter feature can be used to reduce the compilation time while editing
(this was significantly more useful in the earlier days of \LaTeX{})
or to generate a smaller document which is easier to navigate.
Another application of |\includeonly| is to generate
documents consisting of selected parts of the complete document.

However, there are a few drawbacks of the plain |\include| mechanism:
\begin{itemize}
\item
The child files cannot be compiled on their own,
they can only be compiled via the main file.
A naive editing environment
(such as a text editor with an option
to have the current file processed by \LaTeX)
may require one to switch to the main file before compiling;
attempting to compile the child file produces errors.
\item
The main file must be modified (each time)
to adjust the |\includeonly| command
to the present needs. This easily leaves the main file in a messy state.
\item
The generated document will always carry the filename
of the main document. This is inconvenient if
several child files are to be compiled and
to be kept for distribution.
\end{itemize}

The present package provides a simple interface
to make child files individually compilable by \LaTeX{}.
Compiling a child file then has the same effect as compiling
the main file with an |\includeonly| command
to select the appropriate child.
Moreover the generated document will carry the name of the child
rather than the main file.
This resolves all three above issues.

This feature is meant to make the editing of books,
thesis documents and lecture notes somewhat more convenient.
However, the package can also be used efficiently for
composing a series of documents (such as exercise sheets)
which are typically distributed individually.
It then assists the author in generating the individual documents
(potentially in different versions)
as well as a document containing the collected series.
Another application is in developing style files
or other kinds of included material
where compilation of the style file could redirect
to a sample or test file.

%%%%%%%%%%%%%%%%%%%%%%%%%%%%%%%%%%%%%%%%%%%%%%%%%%%%%%%%%%%%%%%%%%%%%%%%%%%%%%%%
%%%%%%%%%%%%%%%%%%%%%%%%%%%%%%%%%%%%%%%%%%%%%%%%%%%%%%%%%%%%%%%%%%%%%%%%%%%%%%%%
\section{Usage}

First of all, the package \textsf{childdoc} is \emph{not} a standard
\LaTeXe{} |.sty| style file! Therefore it needs to be invoked in
a non-standard way.

%%%%%%%%%%%%%%%%%%%%%%%%%%%%%%%%%%%%%%%%%%%%%%%%%%%%%%%%%%%%%%%%%%%%%%%%%%%%%%%%
\subsection{Included Files}
\label{sec:include}

%%%%%%%%%%%%%%%%%%%%%%%%%%%%%%%%%%%%%%%%
\DescribeMacro{\childdocmain}
To use the package, add the commands
\begin{center}
\begin{tabular}{l}
|\input{childdoc.def}|\\
|\childdocmain{}|\\
\end{tabular}
\end{center}
at the very top of the main \LaTeX{} file,
in particular \emph{before} the |\documentclass| statement!
The argument of |\childdocmain| should be left empty
(but it must be present).

%%%%%%%%%%%%%%%%%%%%%%%%%%%%%%%%%%%%%%%%
\DescribeMacro{\childdocof}
Furthermore, add the commands
\begin{center}
\begin{tabular}{l}
|\input{childdoc.def}|\\
|\childdocof{|\textit{main}|}|\\
\end{tabular}
\end{center}
at the top of every child file \textit{child}
which is included by |\include{|\textit{child}|}|
from within the main file
(or at least for those files to be compiled individually).
The argument \textit{main} must be the filename of the main file.

There are a couple of
considerations in setting up the main and child documents:

%%%%%%%%%%%%%%%%%%%%%%%%%%%%%%%%%%%%%%%%
\paragraph{Restrictions.}

Please note the following restrictions:
\begin{itemize}
\item
|\childdocmain| must be called with one argument \textit{main}
to ensure compatibility with earlier version of the package.
It must either be empty (|\childdocmain{}|)
or precisely match the filename of the main file in which it is specified.
See \secref{sec:detection} for further information.
\item
The filename \textit{main} must be specified without the |.tex| extension.
\item
The filename \textit{main} is case sensitive
(even in case-insensitive file systems)
due to internal string comparison.
\item
The argument \textit{main} should be fully expanded, it cannot be a macro.
\item
Subdirectories and special characters should be avoided in filenames.
\item
The command |\childdocmain{|\textit{main}|}| must be followed by a whitespace.
It should not be followed immediately by another command
or by a comment mark `|%|'.
This is because the \TeX{} parser reads the token immediately following
the argument of |\childdocmain| and puts it
at the beginning of every child section;
however, a white\-space is ignored.
\end{itemize}

%%%%%%%%%%%%%%%%%%%%%%%%%%%%%%%%%%%%%%%%
\paragraph{Content of Main File.}

It is advisable to place all content in the child files included by |\include|.
Any output contained in the main file will appear in all child documents
unless suppressed manually;
it cannot be suppressed automatically by the |\includeonly| directive
and thus should normally be avoided.
A method to include some content in the main file
by means of conditional processing is described in \secref{sec:conditional}.

%%%%%%%%%%%%%%%%%%%%%%%%%%%%%%%%%%%%%%%%
\paragraph{Page Numbering.}

When only a part of the document is compiled,
the appropriate numbering of pages
(as well as other status parameters)
is determined from the |.aux| files.
The latter contain information from previous passes.
However this information needs to propagate through
all intermediate child documents.
Therefore the page numbering in child documents may well
be inconsistent until the complete document is compiled at least once.

A useful (if unconventional) way to always ensure a consistent
page numbering is to restart the numbering in each child document
and denote the pages by `\textit{child}|.|\textit{page}'
where \textit{child} represents the chapter/section number of the child file.
This can be achieved by the command
|\numberwithin{page}{|\textit{child}|}|
of the \textsf{amsmath} package
where \textit{child} can be |chapter| or |section|
depending on the chosen structuring.
Alternatively, one can modify the macro |\thepage| appropriately
and reset the counter |page| at the start of each child file.

%%%%%%%%%%%%%%%%%%%%%%%%%%%%%%%%%%%%%%%%%%%%%%%%%%%%%%%%%%%%%%%%%%%%%%%%%%%%%%%%
\subsection{Conditional Processing}
\label{sec:conditional}

The package provides a mechanism to compile different versions
of a document. To customise the versions further some conditional processing
can come in handy to distinguish which version is being compiled.
The package provides two macros to describe the compilation context:

%%%%%%%%%%%%%%%%%%%%%%%%%%%%%%%%%%%%%%%%
\DescribeMacro{\ifchilddoc}
The conditional |\ifchilddoc| distinguishes between the compilation of
child documents and the main document:
%
\begin{center}
|\ifchilddoc |\textit{child-code}| |[|\||else |\textit{main-code}]| \||fi|
\end{center}

%%%%%%%%%%%%%%%%%%%%%%%%%%%%%%%%%%%%%%%%
\DescribeMacro{\childdocname}
\DescribeMacro{\childdocjob}
The macro |\childdocname| contains the filename (without extension)
of the main or child file being processed.
Note that |\childdocjob| will always contain the name of the main file.

%%%%%%%%%%%%%%%%%%%%%%%%%%%%%%%%%%%%%%%%
\paragraph{Title Page.}

Conditional processing can be used to include a title or banner page
in the main document when proper precautions are taken.
Importantly, the code in the main file should ensure that the page counter
(as well as other status parameters which are stored in the |.aux| files)
takes the same value after the conditional processing.
Otherwise the page numbers may take divergent values
depending on which part is compiled.

For example, a title page could be declared by:
%
\begin{center}
\begin{tabular}{l}
|\ifchilddoc\||else|\\
|\addtocounter{page}{-1}|\\
\textit{code for title page}\\
|\newpage|\\
|\||fi|
\end{tabular}
\end{center}
%
A banner page for the child documents can be generated by:
%
\begin{center}
\begin{tabular}{l}
|\ifchilddoc|\\
|\addtocounter{page}{-1}|\\
\textit{code for banner page}\\
|\newpage|\\
|\||fi|
\end{tabular}
\end{center}
%
Here one could write a message such as:
\begin{center}
|This is the part \childdocname{} of \childdocjob{}.|
\end{center}

%%%%%%%%%%%%%%%%%%%%%%%%%%%%%%%%%%%%%%%%%%%%%%%%%%%%%%%%%%%%%%%%%%%%%%%%%%%%%%%%
\subsection{Flags}
\label{sec:flags}

The package makes it easy to generate different versions
of the main or child documents.
To this end compilation flags can be defined
and assigned different default values.
They will be particularly useful in conjunction
with the forwarding mechanism described in \secref{sec:forward}.

For example, it may be useful to have a flag |\version|
which can be set to |draft| or |final|.
The document source will contain some conditional code
depending on the value of |\version|.
Suppose further, the flag should default to |final| for the main file
and to |draft| for child files
which is a natural assignment for editing the document.
This is achieved by placing the following code
in the preamble of the main document
(below the |\childdocmain| directive):
%
\begin{center}
\begin{tabular}{l}
|\ifchilddoc|\\
|\providecommand{\version}{draft}|\\
|\||else|\\
|\providecommand{\version}{final}|\\
|\||fi|
\end{tabular}
\end{center}
%
The definition by |\providecommand| makes sure
that previous definitions are not overwritten.
Further statements |\providecommand{\version}{...}|
can thus be added before the above code to override it.

For the main file, one might add a line
(between |\childdocmain| and the above block)
%
\begin{center}
|%\ifchilddoc\||else\providecommand{\version}{draft}\||fi|
\end{center}
%
which can be uncommented to produce a draft version.
Likewise one can add a line to the very top of a child file
(above the |\childdocof{|\textit{main}|}| directive)
%
\begin{center}
|%\providecommand{\version}{final}|
\end{center}
%
which can be uncommented to produce the final version of this child document.

%%%%%%%%%%%%%%%%%%%%%%%%%%%%%%%%%%%%%%%%%%%%%%%%%%%%%%%%%%%%%%%%%%%%%%%%%%%%%%%%
\subsection{Forwarding}
\label{sec:forward}

Different versions of the main or child documents
using compilation flags as described in \secref{sec:flags}
can be (permanently) stored in different files
for convenient compilation, viewing and distribution.
To this end, the package defines a command
to pass on compilation to a different file:

%%%%%%%%%%%%%%%%%%%%%%%%%%%%%%%%%%%%%%%%
\DescribeMacro{\childdocforward}
The command |\childdocforward| redirects processing to
another source file:
%
\begin{center}
\begin{tabular}{l}
|\input{childdoc.def}|\\
|\childdocforward[|\textit{main}|]{|\textit{dest}|}|\\
\end{tabular}
\end{center}
%
The argument \textit{dest} is the destination file
(without extension).
It should be the main file or one of the child files.
Note that further \textsf{childdoc} directives
such as |\childdocof| and |\childdocforward|
in the indicated file will be processed in this form.
The optional argument \textit{main}
passes on directly to the main file \textit{main}
while pretending to compile the child \textit{dest}.
This form behaves as if \textit{dest}
issues |\childdocof{|\textit{main}|}| right away,
and no further \textsf{childdoc} directives will be processed.

%%%%%%%%%%%%%%%%%%%%%%%%%%%%%%%%%%%%%%%%
\DescribeMacro{\...prefix}
In the alternative form |\childdocforwardprefix|,
%
\begin{center}
\begin{tabular}{l}
|\input{childdoc.def}|\\
|\childdocforwardprefix[|\textit{main}|]{|\textit{prefix}|}{|\textit{dest}|}|
\end{tabular}
\end{center}
%
the destination file is determined by a pattern
depending on the current file:
To make this work, the current file must be called
`{\textit{prefix}\hspace{0.2em}\textit{suffix}}'
with \textit{prefix} matching precisely the argument.
Processing is then passed on to the file
`{\textit{dest}\hspace{0.2em}\textit{suffix}}'.
Surely, the same effect is achieved by
directly specifying the
argument `{\textit{dest}\hspace{0.2em}\textit{suffix}}'
in the first form.
However, that requires to set up a different file
for each child. With the alternative form of the command
all these files can have exactly the same content
which simplifies setting them up and maintaining them.

For example, the following file |draft.tex|
with a compilation flag |\version| as described in \secref{sec:flags}
compiles the main document as a draft:
%
\begin{center}
\begin{tabular}{l}
|\def\version{draft}|\\
|\input{childdoc.def}|\\
|\childdocforward{|\textit{main}|}|
\end{tabular}
\end{center}
%
Likewise, the following files |final|\textit{nn}|.tex|
compile the final version of the child document
|child|\textit{nn}|.tex|:
%
\begin{center}
\begin{tabular}{l}
|\def\version{final}|\\
|\input{childdoc.def}|\\
|\childdocforwardprefix{final}{child}|
\end{tabular}
\end{center}
%

Note that when several versions of a main file and/or of each child file
are to be generated, it may be convenient to set up a |Makefile| or
shell script to automatise the process.

%%%%%%%%%%%%%%%%%%%%%%%%%%%%%%%%%%%%%%%%%%%%%%%%%%%%%%%%%%%%%%%%%%%%%%%%%%%%%%%%
\subsection{Command Line Processing}
\label{sec:commandline}

The effect of redirection files can also be achieved by invoking
the \LaTeX{} compiler with a more elaborate command line.
Most conveniently this should be done as part
of a shell script or a |Makefile|.

When using \textsf{childdoc} in the main file, the following
command lines effectively perform a redirection
(note that depending on the shell being used,
backslashes may have to be doubled: `|\|' $\to$ `|\\|'):
%
\begin{center}
|... -jobname "|\textit{target}|" |\\|"|[\textit{flags}]%
|\input{childdoc.def}\childdocforward[|\textit{main}|]{|\textit{dest}|}"|
\end{center}
%
Here \textit{target} is the name of the output file,
\textit{main} is the name of the main file
and \textit{dest} is the name of the main or child file to be processed
(all filenames without extensions).
The optional argument \textit{main} can be omitted
if \textit{main} matches \textit{dest}.
Optionally, compilation \textit{flags} can be defined via |\def| commands.
This command line makes the \TeX{} engine believe
it is compiling the file \textit{target}
whose content is specified as the latter parameter.
The provided code then forwards the processing to
\textit{main} or \textit{dest} as described in \secref{sec:forward}.

%%%%%%%%%%%%%%%%%%%%%%%%%%%%%%%%%%%%%%%%%%%%%%%%%%%%%%%%%%%%%%%%%%%%%%%%%%%%%%%%
\subsection{Include by Input}
\label{sec:input}

Including child documents by |\include| has some restrictions by design.
Most notably, the content of a child document always occupies
its own set of pages; pages cannot be shared between child documents.
Usually, this behaviour makes perfect sense
because each child document contain an essential part of the document.
However, in some situations it may be desirable to compose
a document from a collection of parts
without having mandatory page breaks between then.
For this case, the package
provides a mechanism to include parts
by |\input| which can also be processed individually.
However, by construction this mechanism
requires manual handling of the content to be output.

%%%%%%%%%%%%%%%%%%%%%%%%%%%%%%%%%%%%%%%%
\DescribeMacro{\ifchilddocmanual}
The main file should be prepared as usual, see \secref{sec:include}.
However, the document body must make a distinction
between processing of an individual part and of the main document, e.g.:
%
\begin{center}
\begin{tabular}{l}
|\ifchilddocmanual|\\
|\input{\childdocname}|\\
|\||else|\\
\textit{document body with }|\input{|\textit{part}|}|\\
|\||fi|
\end{tabular}
\end{center}
%
The conditional |\ifchilddocmanual| is true whenever
a part to be included by |\input| is being compiled,
and the name of the part is stored in |\childdocname|.

%%%%%%%%%%%%%%%%%%%%%%%%%%%%%%%%%%%%%%%%
\DescribeMacro{\childdocby}
Each part to be included by |\input| should start with:
%
\begin{center}
\begin{tabular}{l}
|\input{childdoc.def}|\\
|\childdocby{|\textit{main}|}|\\
\end{tabular}
\end{center}
%
The directive |\childdocby| is similar to |\childdocof|
described in \secref{sec:include},
but the subsequent selection of content must be done manually.
To that end, both |\ifchilddoc| and |\ifchilddocmanual|
will be true upon processing of a part,
and the name of the part is stored in |\childdocname|.
Note that |\jobname| will be set to the filename of the current part
so that each part receives an individual |.aux| file
that does not interfere with the |.aux| file(s) of the main document.
This behaviour can be altered by the alternative form
|\childdocby[*]{|\textit{main}|}| (with a non-empty optional argument)
which uses the |.aux| file of the main document
by setting |\jobname| to \textit{main}.

%%%%%%%%%%%%%%%%%%%%%%%%%%%%%%%%%%%%%%%%%%%%%%%%%%%%%%%%%%%%%%%%%%%%%%%%%%%%%%%%
\subsection{Driver Development}
\label{sec:driver}

The \textsf{childdoc} mechanism can also be use for the development
of definition files such as \LaTeX{} styles or classes.
This case differs from the above setup with multiple parts
included by |\include| in that no |\includeonly| should be invoked.
This can be achieved by starting the include file
(before |\ProvidesPackage|) with:
%
\begin{center}
\begin{tabular}{l}
|\input{childdoc.def}|\\
|\childdocforward{|\textit{main}|}|\\
\end{tabular}
\end{center}
%
or alternatively with:
%
\begin{center}
\begin{tabular}{l}
|\input{childdoc.def}|\\
|\childdocby{|\textit{main}|}|\\
\end{tabular}
\end{center}
%
Both forms have slightly different effects as described above.
The main file is prepared as usual, see \secref{sec:include}.

%%%%%%%%%%%%%%%%%%%%%%%%%%%%%%%%%%%%%%%%%%%%%%%%%%%%%%%%%%%%%%%%%%%%%%%%%%%%%%%%
\subsection{Legacy Detection}
\label{sec:detection}

The directive |\childdocmain| in the main file can detect
whether the complete document or merely a child is to be compiled
even without using the directive |\childdocof|.
This method is deprecated because it is less robust
and there is no compelling reason to use it;
it is merely provided for backward compatibility
and it may be removed in future versions.

If the detection mechanism is to be used,
it is mandatory to correctly specify
the filename of the main file as the argument of |\childdocmain|:
%
\begin{center}
\begin{tabular}{l}
|\input{childdoc.def}|\\
|\childdocmain{|\textit{main}|}|\\
\end{tabular}
\end{center}
%
If |\jobname| does not match the argument \textit{main} of |\childdocmain|,
it is assumed that |\jobname| points to the child file to be compiled.
When using |\childdocmain| with the main file specified as argument,
it suffices to start a child file
with just |\input{|\textit{main}|}|
without loading of the package and using |\childdocof|.
If instead all processing is done
with the appropriate \textsf{childdoc} directives,
the argument of \textit{main} of |\childdocmain| can be empty.

An alternative version of the command line processing described
in \secref{sec:commandline} using the detection mechanism reads:
%
\begin{center}
|... -jobname "|\textit{target}|" "|[\textit{flags}]%
[|\def\jobname{|\textit{dest}|}|]|\input{|\textit{main}|}"|
\end{center}

%%%%%%%%%%%%%%%%%%%%%%%%%%%%%%%%%%%%%%%%%%%%%%%%%%%%%%%%%%%%%%%%%%%%%%%%%%%%%%%%
\subsection{Manual Code}
\label{sec:manual}

In case one cannot be certain whether the definitions file |childdoc.def|
is installed on the target \TeX{} distribution
and one prefers not to ship it,
it is conceivable to paste a few relevant commands into the sources.

To that end, drop all statements |\input{childdoc.def}|
and perform the replacements as outlined below.
Instead of |\childdocmain{|\textit{main}|}| add the following code
to the top of the main file:
%
\begin{center}
\begin{tabular}{l}
|\||ifdefined\childdocname\endinput\||fi\newif\ifchilddoc|\\
|\edef\childdocname{\scantokens\expandafter{\jobname\noexpand}}|\\
|\def\childdocmain{|\textit{main}|}\||ifx\childdocmain\childdocname\||else|\\
|\childdoctrue\includeonly{\childdocname}\let\jobname\childdocmain\||fi|\\
\end{tabular}
\end{center}
%
Instead of |\childdocof{|\textit{main}|}| just include the main file
at the top of each child file:
%
\begin{center}
|\input{|\textit{main}|}|
\end{center}
%
A simple redirection |\childdocforward{|\textit{dest}|}| is achieved by:
%
\begin{center}
|\def\jobname{|\textit{dest}|}\input{\jobname}|
\end{center}
%
The redirection with prefix
|\childdocforwardprefix[|\textit{prefix}|]{|\textit{dest}|}|
is accomplished by:
%
\begin{center}
\begin{tabular}{l}
|{\edef\jobname{\scantokens\expandafter{\jobname\noexpand}}|\\
|\def\redirectjob |\textit{prefix}|#1~~~{\gdef\jobname{|\textit{dest}|#1}}|\\
|\expandafter\redirectjob\jobname~~~}\input{\jobname}|
\end{tabular}
\end{center}

In an alternative approach,
child documents can be compiled by a specific command line
without additional code or specific definitions:
%
\begin{center}
|... -jobname "|\textit{target}|" "|[\textit{flags}]%
|\includeonly{|\textit{dest}|}\input{|\textit{main}|}"|
\end{center}
%

%%%%%%%%%%%%%%%%%%%%%%%%%%%%%%%%%%%%%%%%%%%%%%%%%%%%%%%%%%%%%%%%%%%%%%%%%%%%%%%%
%%%%%%%%%%%%%%%%%%%%%%%%%%%%%%%%%%%%%%%%%%%%%%%%%%%%%%%%%%%%%%%%%%%%%%%%%%%%%%%%
\section{Information}

%%%%%%%%%%%%%%%%%%%%%%%%%%%%%%%%%%%%%%%%%%%%%%%%%%%%%%%%%%%%%%%%%%%%%%%%%%%%%%%%
\subsection{Copyright}

Copyright \copyright{} 2017--2018 Niklas Beisert

This work may be distributed and/or modified under the
conditions of the \LaTeX{} Project Public License, either version 1.3
of this license or (at your option) any later version.
The latest version of this license is in
  \url{http://www.latex-project.org/lppl.txt}
and version 1.3 or later is part of all distributions of \LaTeX{}
version 2005/12/01 or later.

This work has the LPPL maintenance status `maintained'.

The Current Maintainer of this work is Niklas Beisert.

This work consists of the files |README.txt|, |childdoc.ins| and |childdoc.dtx|
as well as the derived files |childdoc.def|, |cdocsamp.tex|
with |cdocsch1.tex|, |cdocsch2.tex|, |cdocspt3.tex|, |cdocspt4.tex|,
|cdocsdrf.tex|, |cdocsfn1.tex|, |cdocsfn2.tex|
as well as |childdoc.pdf|.

%%%%%%%%%%%%%%%%%%%%%%%%%%%%%%%%%%%%%%%%%%%%%%%%%%%%%%%%%%%%%%%%%%%%%%%%%%%%%%%%
\subsection{Files and Installation}

The package consists of the files:
%
\begin{center}
\begin{tabular}{ll}
    |README.txt|   & readme file \\
    |childdoc.ins| & installation file \\
    |childdoc.dtx| & source file \\
    |childdoc.def| & definition file \\
    |cdocsamp.tex| & sample main file \\
    |cdocsch1.tex| & sample include file \\
    |cdocsch2.tex| & sample include file \\
    |cdocspt3.tex| & sample part file \\
    |cdocspt4.tex| & sample part file \\
    |cdocsdrf.tex| & sample redirection file \\
    |cdocsfn1.tex| & sample redirection file \\
    |cdocsfn2.tex| & sample redirection file \\
    |childdoc.pdf| & manual
\end{tabular}
\end{center}
%
The distribution consists of the files
|README.txt|, |childdoc.ins| and |childdoc.dtx|.
%
\begin{itemize}
\item
Run (pdf)\LaTeX{} on |childdoc.dtx|
to compile the manual |childdoc.pdf| (this file).
\item
Run \LaTeX{} on |childdoc.ins| to create the definitions file |childdoc.def|
and the sample |cdocsamp.tex| with include files
|cdocsch1.tex|, |cdocsch2.tex|, |cdocspt3.tex|, |cdocspt4.tex|,
|cdocsdrf.tex|, |cdocsfn1.tex|, |cdocsfn2.tex|.
Then copy the file |childdoc.def| to an appropriate directory of your \LaTeX{}
distribution, e.g.\ \textit{texmf-root}|/tex/latex/childdoc|.
\end{itemize}

%%%%%%%%%%%%%%%%%%%%%%%%%%%%%%%%%%%%%%%%%%%%%%%%%%%%%%%%%%%%%%%%%%%%%%%%%%%%%%%%
\subsection{Related CTAN Packages}

There are several other packages which offer a similar functionality:
%
\begin{itemize}
\item
The packages
\href{http://ctan.org/pkg/docmute}{\textsf{docmute}},
\href{http://ctan.org/pkg/includex}{\textsf{includex}} and
\href{http://ctan.org/pkg/standalone}{\textsf{standalone}}
provide commands to include only the document body of
a child file thus allowing both files to be compiled individually.
\item
The packages \href{http://ctan.org/pkg/subdocs}{\textsf{subdocs}}
and \href{http://ctan.org/pkg/subfiles}{\textsf{subfiles}}
provide structures in which the main and child documents can be
encapsulated and allowing them to be compiled individually.
The inclusion mechanism is different from the conventional |\include|.
\item
The package \href{http://ctan.org/pkg/combine}{\textsf{combine}}
is an elaborate solution to combine several documents into one.
\end{itemize}
%
See also the CTAN topic \href{http://ctan.org/topic/subdocs}{\textsf{subdocs}}
for further related packages.
The present package differs from the above solutions in that
a document structure constructed with the conventional |\include| mechanism
just needs two extra commands at the top of every file
such that all constituent files can be compiled individually.

%%%%%%%%%%%%%%%%%%%%%%%%%%%%%%%%%%%%%%%%%%%%%%%%%%%%%%%%%%%%%%%%%%%%%%%%%%%%%%%%
%\subsection{Feature Suggestions}
%
%The following is a list of features which may be useful for future
%versions of this package:
%%
%\begin{itemize}
%\item
%\ldots
%\end{itemize}

%%%%%%%%%%%%%%%%%%%%%%%%%%%%%%%%%%%%%%%%%%%%%%%%%%%%%%%%%%%%%%%%%%%%%%%%%%%%%%%%
\subsection{Revision History}

%%%%%%%%%%%%%%%%%%%%%%%%%%%%%%%%%%%%%%%%
\paragraph{v2.0:} 2018/12/30

\begin{itemize}
\item
immediate forward processing
\item
added |\childdocby| mechanism
\item
manual restructured
\end{itemize}

%%%%%%%%%%%%%%%%%%%%%%%%%%%%%%%%%%%%%%%%
\paragraph{v1.6:} 2018/01/17

\begin{itemize}
\item
application for development of include files
\item
corrections to manual
\end{itemize}

%%%%%%%%%%%%%%%%%%%%%%%%%%%%%%%%%%%%%%%%
\paragraph{v1.5:} 2017/05/21

\begin{itemize}
\item
more complete structuring introduced
\item
|\childdocof| introduced
\item
|\childdoc| renamed to |\childdocmain|
\item
|\childredirect| renamed to |\childdocforward| and |\childdocforwardprefix|
and functionality expanded
\end{itemize}

%%%%%%%%%%%%%%%%%%%%%%%%%%%%%%%%%%%%%%%%
\paragraph{v1.0:} 2017/04/27

\begin{itemize}
\item
manual and install package
\item
first version published on CTAN
\end{itemize}

%%%%%%%%%%%%%%%%%%%%%%%%%%%%%%%%%%%%%%%%
\paragraph{v0.6:} 2017/04/26

\begin{itemize}
\item
redirection mechanism added
\end{itemize}

%%%%%%%%%%%%%%%%%%%%%%%%%%%%%%%%%%%%%%%%
\paragraph{v0.5:} 2017/04/26

\begin{itemize}
\item
functionality in definition file
\end{itemize}


%%%%%%%%%%%%%%%%%%%%%%%%%%%%%%%%%%%%%%%%%%%%%%%%%%%%%%%%%%%%%%%%%%%%%%%%%%%%%%%%
%%%%%%%%%%%%%%%%%%%%%%%%%%%%%%%%%%%%%%%%%%%%%%%%%%%%%%%%%%%%%%%%%%%%%%%%%%%%%%%%
%%%%%%%%%%%%%%%%%%%%%%%%%%%%%%%%%%%%%%%%%%%%%%%%%%%%%%%%%%%%%%%%%%%%%%%%%%%%%%%%
\appendix

\settowidth\MacroIndent{\rmfamily\scriptsize 000\ }

 \DocInput{childdoc.dtx}

\end{document}
%</driver>
% \fi
%
% %%%%%%%%%%%%%%%%%%%%%%%%%%%%%%%%%%%%%%%%%%%%%%%%%%%%%%%%%%%%%%%%%%%%%%%%%%%%%%
% %%%%%%%%%%%%%%%%%%%%%%%%%%%%%%%%%%%%%%%%%%%%%%%%%%%%%%%%%%%%%%%%%%%%%%%%%%%%%%
% \section{Sample}
%\iffalse
%<*samplemain>
%\fi
%
% The following presents a sample document
% with two chapters, two parts, a title page,
% a compile flag as well as three forwarding files to set the flag.
% It consists of eight |.tex| files:
% \begin{center}
% \begin{tabular}{ll}
% |cdocsamp.tex|&main file\\
% |cdocsch1.tex|&include file for chapter 1\\
% |cdocsch2.tex|&include file for chapter 2\\
% |cdocspt3.tex|&include file for part 3\\
% |cdocspt4.tex|&include file for part 4\\
% |cdocsdrf.tex|&forwarding file for main file in draft mode\\
% |cdocsfi1.tex|&forwarding file for final version of chapter 1\\
% |cdocsfi2.tex|&forwarding file for final version of chapter 2\\
% \end{tabular}
% \end{center}
% Each of the eight files can be compiled directly by the \LaTeX{} compiler.
%
% %%%%%%%%%%%%%%%%%%%%%%%%%%%%%%%%%%%%%%
% \paragraph{Main File.}
%
% The main file is called |cdocsamp.tex|.
%
% Load the \textsf{childdoc} definitions and
% declare the filename for the main document:
%    \begin{macrocode}
\input{childdoc.def}
\childdocmain{}
%    \end{macrocode}

% Optional override for |\version| flag:
%    \begin{macrocode}
%%\ifchilddoc\else\providecommand{\version}{draft}\fi
%    \end{macrocode}

% Define the default values for the |\version| flag
% (|final| for the main file and |draft| for childs):
%    \begin{macrocode}
\ifchilddoc
\providecommand{\version}{draft}
\else
\providecommand{\version}{final}
\fi
%    \end{macrocode}

% Load the standard document class:
%    \begin{macrocode}
\documentclass[12pt]{article}
%    \end{macrocode}

% Start the document body:
%    \begin{macrocode}
\begin{document}
%    \end{macrocode}

% Declare a title page.
% Print title, part of document being processed and version flag:
%    \begin{macrocode}
\addtocounter{page}{-1}
\begin{center}
{\LARGE\bfseries{}childdoc example\par}
\vspace{1cm}
\ifchilddoc
\ifchilddocmanual part\else chapter\fi:
`\childdocname' of `\childdocjob'\par
\else
main document: `\childdocjob'\par
\fi
version: \version\par
\end{center}
\newpage
%    \end{macrocode}

% Manually include selected file,
% otherwise process as usual:
%    \begin{macrocode}
\ifchilddocmanual
\section*{part `\childdocname'}
\input{\childdocname}
\else
%    \end{macrocode}

% Include the two chapters:
%    \begin{macrocode}
\include{cdocsch1}
\include{cdocsch2}
%    \end{macrocode}

% Include the two parts unless only chapters should be displayed:
%    \begin{macrocode}
\ifchilddoc\else
\section{part three}
\input{cdocspt3}
\section{part four}
\input{cdocspt4}
\fi
%    \end{macrocode}

% Process as usual until here:
%    \begin{macrocode}
\fi
%    \end{macrocode}

% End of document body:
%    \begin{macrocode}
\end{document}
%    \end{macrocode}
%\iffalse
%</samplemain>
%\fi
%
% %%%%%%%%%%%%%%%%%%%%%%%%%%%%%%%%%%%%%%
% \paragraph{Chapter Include Files.}
%
% The include files are called |cdocsch1.tex| and |cdocsch2.tex|.
%
%\iffalse
%<*samplechap1|samplechap2>
%\fi

% Optional override for |\version| flag:
%    \begin{macrocode}
%%\providecommand{\version}{final}
%    \end{macrocode}

% Include the main document:
%    \begin{macrocode}
\input{childdoc.def}
\childdocof{cdocsamp}
%    \end{macrocode}

%\iffalse
%</samplechap1|samplechap2>
%\fi
%
%\iffalse
%<*samplechap1>
%\fi
% Some text for chapter 1:
%    \begin{macrocode}
\section{one}
some text in chapter one
%    \end{macrocode}

%\iffalse
%</samplechap1>
%\fi
% Some text for chapter 2:
%\iffalse
%<*samplechap2>
%\fi
%    \begin{macrocode}
\section{two}
more text in chapter two
%    \end{macrocode}

%\iffalse
%</samplechap2>
%\fi
%
% %%%%%%%%%%%%%%%%%%%%%%%%%%%%%%%%%%%%%%
% \paragraph{Part Include Files.}
%
% The include files are called |cdocspt3.tex| and |cdocspt4.tex|.
%
%\iffalse
%<*samplepart3|samplepart4>
%\fi

% Optional override for |\version| flag:
%    \begin{macrocode}
%%\providecommand{\version}{final}
%    \end{macrocode}

% Include the main document:
%    \begin{macrocode}
\input{childdoc.def}
\childdocby{cdocsamp}
%    \end{macrocode}

%\iffalse
%</samplepart3|samplepart4>
%\fi
%
%\iffalse
%<*samplepart3>
%\fi
% Some text for part 3:
%    \begin{macrocode}
some text in part three
%    \end{macrocode}

%\iffalse
%</samplepart3>
%\fi
% Some text for part 4:
%\iffalse
%<*samplepart4>
%\fi
%    \begin{macrocode}
more text in part four
%    \end{macrocode}

%\iffalse
%</samplepart4>
%\fi
%
% %%%%%%%%%%%%%%%%%%%%%%%%%%%%%%%%%%%%%%
% \paragraph{Forwarding for a Complete Draft.}
%
% The following forwarding file |cdocsdrf.tex|
% compiles the main document in draft mode:
%\iffalse
%<*sampledraft>
%\fi
%    \begin{macrocode}
\def\version{draft}
\input{childdoc.def}
\childdocforward{cdocsamp}
%    \end{macrocode}

%\iffalse
%</sampledraft>
%\fi
%
% %%%%%%%%%%%%%%%%%%%%%%%%%%%%%%%%%%%%%%
% \paragraph{Forwarding for Final Version of the Chapters.}
%
% The following forwarding files |cdocsfn1.tex| and |cdocsfn2.tex|
% (with identical content)
% compile the final versions of the child documents
% |cdocsch1.tex| and |cdocsch2.tex|, respectively:
%\iffalse
%<*samplefinal>
%\fi
%    \begin{macrocode}
\def\version{final}
\input{childdoc.def}
\childdocforwardprefix[cdocsamp]{cdocsfn}{cdocsch}
%    \end{macrocode}

%\iffalse
%</samplefinal>
%\fi
%
% %%%%%%%%%%%%%%%%%%%%%%%%%%%%%%%%%%%%%%
% \paragraph{Command Line Processing.}
%
% The following three command lines generate the output files
% |cdocscld|, |cdocscl1| and |cdocscl2|
% which should be identical to
% |cdocsdrf|, |cdocsch1| and |cdocsfn2|, respectively:
% \begin{center}
% \begin{tabular}{l}
% |latex -jobname cdocscld \|\\
% |  "\def\version{draft}\input{childdoc.def}\childdocforward{cdocsamp}"|\\
% |latex -jobname cdocscl1 \|\\
% |  "\input{childdoc.def}\childdocforward[cdocsamp]{cdocsch1}"|\\
% |latex -jobname cdocscl2 \|\\
% |  "\def\version{final}\input{childdoc.def}\childdocforward{cdocsch2}"|
% \end{tabular}
% \end{center}
% Note that the trailing backslash on each first line
% merely continues the input to the second line
% (for convenient cut ant paste).
% Furthermore, the command |latex| can be replaced by any
% of its alternative versions such as |pdflatex|.
%
% %%%%%%%%%%%%%%%%%%%%%%%%%%%%%%%%%%%%%%%%%%%%%%%%%%%%%%%%%%%%%%%%%%%%%%%%%%%%%%
% %%%%%%%%%%%%%%%%%%%%%%%%%%%%%%%%%%%%%%%%%%%%%%%%%%%%%%%%%%%%%%%%%%%%%%%%%%%%%%
% \section{Implementation}
%\iffalse
%<*package>
%\fi
%
% This section describes the definitions file |childdoc.def|.

% The definitions cannot be loaded using |\usepackage| or |\RequirePackage|
% which has a mechanism to prevent loading a style file more than once.
% When loading the definitions by means of |\input|
% multiple instances have to be prevented manually:
%\iffalse
%This code needs to be before the `\ProvidesFile' directive
%which is defined at the beginning of this file.
%Therefore it is also placed there and commented out here.
%</package>
%<*discard>
%\fi
%    \begin{macrocode}
\ifdefined\childdocmain\endinput\fi
%    \end{macrocode}
%\iffalse
%</discard>
%<*package>
%\fi
%
% \macro{\ifchilddoc}
% \macro{\ifchilddocmanual}
% The conditional |\ifchilddoc| tells whether a
% child (true) or main (false) document is being compiled.
% The conditional |\ifchilddocmanual| tells whether
% the |\includeonly| mechanism is used (false) or
% the selection of child files must be performed manually (true).
% The definitions initialise to false:
%    \begin{macrocode}
\newif\ifchilddoc
\newif\ifchilddocmanual
%    \end{macrocode}

% \macro{\childdocname}
% \macro{\childdocjob}
% The macro |\childdocname| stores the name of the main document
% to be compiled. The macro |\childdocjob| stores the name of
% the document on which the \LaTeX{} compiler was originally invoked.
% The content of |\jobname| cannot be compared
% to filenames specified in the source due to different catcodes.
% The following code rescans |\jobname|, stores the result
% in |\childdocname| and saves a copy in |\childdocjob|:
%    \begin{macrocode}
\edef\childdocname{\scantokens\expandafter{\jobname\noexpand}}
\let\childdocjob\childdocname
%    \end{macrocode}

% \macro{\childdocdisable}
% The macro |\childdocdisable| prevents the main file
% from being processed more than once.
% At this stage, the main document command |\childdocmain|
% is assumed to be called once again where it should do nothing.
% Any subsequent call to it should prevent
% a secondary processing of the main document
% It overwrites the forwarding commands
% |\childdocof| and |\childdocforward|
% with empty macros to prevent further inclusions of the main document:
%    \begin{macrocode}
\newcommand{\childdocdisable}
{
  \renewcommand{\childdocmain}[1]{\renewcommand{\childdocmain}[1]{\endinput}}
  \renewcommand{\childdocof}[1]{}
  \renewcommand{\childdocby}[2][]{}
  \renewcommand{\childdocforward}[2][]{}
  \renewcommand{\childdocdisable}{}
}
%    \end{macrocode}

% \macro{\childdocmain}
% The macro |\childdocmain| is to be called at the top of the main file
% with nothing or the main filename (without extension) as argument.
% First, it breaks loops.
% If the argument is not empty and does not match |\childdocname|
% (which is set by the first inclusion of |childdoc.def|),
% |\ifchilddoc| is set to true, |\includeonly| is applied to the child file
% and |\jobname| is set to the main file
% (for proper handling of |.aux| files):
%    \begin{macrocode}
\newcommand{\childdocmain}[1]
{
  \childdocdisable\childdocmain{}
  \if?#1?\else
    \begingroup
      \def\childdoctmp{#1}
      \ifx\childdoctmp\childdocname
        \def\childdoctmp{}
      \else
        \def\childdoctmp
        {
          \childdoctrue
          \includeonly{\childdocname}
          \def\childdocjob{#1}
          \def\jobname{#1}
        }
      \fi
      \expandafter
    \endgroup
    \childdoctmp
  \fi
}
%    \end{macrocode}

% \macro{\childdocof}
% The command |\childdocof| redirects
% compilation to the main file |#1|.
%    \begin{macrocode}
\newcommand{\childdocof}[1]
{
  \childdocdisable
  \childdoctrue
  \includeonly{\childdocname}
  \def\jobname{#1}
  \def\childdocjob{#1}
  \input{#1}
}
%    \end{macrocode}

% \macro{\childdocby}
% The command |\childdocby| ....
%    \begin{macrocode}
\newcommand{\childdocby}[2][]
{
  \childdocdisable
  \childdoctrue
  \childdocmanualtrue
  \if?#1?\else
    \def\jobname{#2}
  \fi
  \def\childdocjob{#2}
  \input{#2}
  \endinput
}
%    \end{macrocode}

% \macro{\childdocforward}
% The command |\childdocforward| redirects
% compilation to the main file or
% (if the optional argument is given) a child file.
% Parameters are set as if the main file
% or a child file starting with |\childdocof| was compiled.
% Then compilation is handed over to the main file:
%    \begin{macrocode}
\newcommand{\childdocforward}[2][]
{
  \begingroup
    \if?#1?
      \def\childdoctmp
      {
        \def\childdocname{#2}
        \def\childdocjob{#2}
        \def\jobname{#2}
        \input{#2}
        \endinput
      }
    \else
      \def\childdoctmp
      {
        \childdocdisable
        \def\childdocname{#2}
        \childdoctrue
        \includeonly{#2}
        \def\childdocjob{#1}
        \def\jobname{#1}
        \input{#1}
        \endinput
      }
    \fi
    \expandafter
  \endgroup
  \childdoctmp
}
%    \end{macrocode}

% \macro{\childdocforwardprefix}
% The command |\childdocforwardprefix| redirects
% compilation to the main or a child file by means of a pattern.
% The prefix |#1| in the current filename is replaced by |#2|
% and the suffix of the current filename is kept
% (it is assumed that the filename does not contain the substring `|~~~|'
% which is used as a delimiter).
% Compilation is handed over to the new file by |\childdocforward|:
%    \begin{macrocode}
\newcommand{\childdocforwardprefix}[3][]
{
  \begingroup
    \def\childdocextract #2##1~~~{\def\childdoctmp{\childdocforward[#1]{#3##1}}}
    \expandafter\childdocextract\childdocname~~~
    \expandafter
  \endgroup
  \childdoctmp
}
%    \end{macrocode}

% \macro{\childdoc}
% The deprecated macro |\childdoc| is a legacy version of |\childdocmain|:
%    \begin{macrocode}
\newcommand{\childdoc}{\childdocmain}
%    \end{macrocode}

% \macro{\childdocredirect}
% The deprecated macro |\childdocredirect| is a legacy version
% of |\childdocforward| and |\childdocforwardprefix|:
%    \begin{macrocode}
\newcommand{\childdocredirect}[2][]
{
  \begingroup
    \if?#1?
      \def\childdoctmp{\childdocforward{#2}}
    \else
      \def\childdoctmp{\childdocforwardprefix{#1}{#2}}
    \fi
    \expandafter
  \endgroup
  \childdoctmp
}
%    \end{macrocode}

%\iffalse
%</package>
%\fi
%
\endinput

\childdocforwardprefix[cdocsamp]{cdocsfn}{cdocsch}
%    \end{macrocode}

%\iffalse
%</samplefinal>
%\fi
%
% %%%%%%%%%%%%%%%%%%%%%%%%%%%%%%%%%%%%%%
% \paragraph{Command Line Processing.}
%
% The following three command lines generate the output files
% |cdocscld|, |cdocscl1| and |cdocscl2|
% which should be identical to
% |cdocsdrf|, |cdocsch1| and |cdocsfn2|, respectively:
% \begin{center}
% \begin{tabular}{l}
% |latex -jobname cdocscld \|\\
% |  "\def\version{draft}% \iffalse
%
% childdoc.dtx Copyright (C) 2017-2018 Niklas Beisert
%
% This work may be distributed and/or modified under the
% conditions of the LaTeX Project Public License, either version 1.3
% of this license or (at your option) any later version.
% The latest version of this license is in
%   http://www.latex-project.org/lppl.txt
% and version 1.3 or later is part of all distributions of LaTeX
% version 2005/12/01 or later.
%
% This work has the LPPL maintenance status `maintained'.
%
% The Current Maintainer of this work is Niklas Beisert.
%
% This work consists of the files childdoc.dtx and childdoc.ins
% and the derived files childdoc.def and cdocsamp.tex with
% cdocsch1.tex, cdocsch2.tex, cdocsdrf.tex, cdocsfn1.tex, cdocsfn2.tex.
%
%<package>\ifdefined\childdocmain\endinput\fi
%<package>\ProvidesFile{childdoc.def}[2018/12/30 v2.0 child document driver]
%<samplemain>\ProvidesFile{cdocsamp.tex}[2018/12/30 v2.0 sample for childdoc]
%<*driver>
%\ProvidesFile{childdoc.drv}[2018/12/30 v2.0 childdoc reference manual file]
\PassOptionsToClass{10pt,a4paper}{article}
\documentclass{ltxdoc}

\usepackage[margin=35mm]{geometry}
\usepackage{hyperref}
\usepackage{hyperxmp}
\usepackage[usenames]{color}

\hypersetup{colorlinks=true}
\hypersetup{pdfstartview=FitH}
\hypersetup{pdfpagemode=UseNone}
\hypersetup{pdfsource={}}
\hypersetup{pdflang={en-UK}}
\hypersetup{pdfcopyright={Copyright 2017-2018 Niklas Beisert.
  This work may be distributed and/or modified under the
  conditions of the LaTeX Project Public License, either version 1.3
  of this license or (at your option) any later version.}}
\hypersetup{pdflicenseurl={http://www.latex-project.org/lppl.txt}}
\hypersetup{pdfcontactaddress={ETH Zurich, ITP, HIT K,
  Wolfgang-Pauli-Strasse 27}}
\hypersetup{pdfcontactpostcode={8093}}
\hypersetup{pdfcontactcity={Zurich}}
\hypersetup{pdfcontactcountry={Switzerland}}
\hypersetup{pdfcontactemail={nbeisert@itp.phys.ethz.ch}}
\hypersetup{pdfcontacturl={http://people.phys.ethz.ch/\xmptilde nbeisert/}}

\newcommand{\secref}[1]{\hyperref[#1]{section \ref*{#1}}}

\parskip1ex
\parindent0pt
\let\olditemize\itemize
\def\itemize{\olditemize\parskip0pt}

\begin{document}

\title{The \textsf{childdoc} Package}
\hypersetup{pdftitle={The childdoc Package}}
\author{Niklas Beisert\\[2ex]
  Institut f\"ur Theoretische Physik\\
  Eidgen\"ossische Technische Hochschule Z\"urich\\
  Wolfgang-Pauli-Strasse 27, 8093 Z\"urich, Switzerland\\[1ex]
  \href{mailto:nbeisert@itp.phys.ethz.ch}
  {\texttt{nbeisert@itp.phys.ethz.ch}}}
\hypersetup{pdfauthor={Niklas Beisert}}
\hypersetup{pdfsubject={Manual for the LaTeX2e Package childdoc}}
\date{30 December 2018, \textsf{v2.0}}
\maketitle

\begin{abstract}\noindent
\textsf{childdoc} is a \LaTeXe{} package
that enables the direct compilation
of document sections included by |\include|
to individual files.
\end{abstract}

\begingroup
\parskip0ex
\tableofcontents
\endgroup

%%%%%%%%%%%%%%%%%%%%%%%%%%%%%%%%%%%%%%%%%%%%%%%%%%%%%%%%%%%%%%%%%%%%%%%%%%%%%%%%
%%%%%%%%%%%%%%%%%%%%%%%%%%%%%%%%%%%%%%%%%%%%%%%%%%%%%%%%%%%%%%%%%%%%%%%%%%%%%%%%
\section{Introduction}

\LaTeX{} provides a mechanism to structure a large document (such as a book)
into a main file and several child files (containing the chapters)
using the |\include| command.
This mechanism is beneficial for documents
which span hundreds of pages in order to
make the source file(s) more manageable.
Moreover, compilation can be restricted to
selected child files by means of the |\includeonly| command.
The latter feature can be used to reduce the compilation time while editing
(this was significantly more useful in the earlier days of \LaTeX{})
or to generate a smaller document which is easier to navigate.
Another application of |\includeonly| is to generate
documents consisting of selected parts of the complete document.

However, there are a few drawbacks of the plain |\include| mechanism:
\begin{itemize}
\item
The child files cannot be compiled on their own,
they can only be compiled via the main file.
A naive editing environment
(such as a text editor with an option
to have the current file processed by \LaTeX)
may require one to switch to the main file before compiling;
attempting to compile the child file produces errors.
\item
The main file must be modified (each time)
to adjust the |\includeonly| command
to the present needs. This easily leaves the main file in a messy state.
\item
The generated document will always carry the filename
of the main document. This is inconvenient if
several child files are to be compiled and
to be kept for distribution.
\end{itemize}

The present package provides a simple interface
to make child files individually compilable by \LaTeX{}.
Compiling a child file then has the same effect as compiling
the main file with an |\includeonly| command
to select the appropriate child.
Moreover the generated document will carry the name of the child
rather than the main file.
This resolves all three above issues.

This feature is meant to make the editing of books,
thesis documents and lecture notes somewhat more convenient.
However, the package can also be used efficiently for
composing a series of documents (such as exercise sheets)
which are typically distributed individually.
It then assists the author in generating the individual documents
(potentially in different versions)
as well as a document containing the collected series.
Another application is in developing style files
or other kinds of included material
where compilation of the style file could redirect
to a sample or test file.

%%%%%%%%%%%%%%%%%%%%%%%%%%%%%%%%%%%%%%%%%%%%%%%%%%%%%%%%%%%%%%%%%%%%%%%%%%%%%%%%
%%%%%%%%%%%%%%%%%%%%%%%%%%%%%%%%%%%%%%%%%%%%%%%%%%%%%%%%%%%%%%%%%%%%%%%%%%%%%%%%
\section{Usage}

First of all, the package \textsf{childdoc} is \emph{not} a standard
\LaTeXe{} |.sty| style file! Therefore it needs to be invoked in
a non-standard way.

%%%%%%%%%%%%%%%%%%%%%%%%%%%%%%%%%%%%%%%%%%%%%%%%%%%%%%%%%%%%%%%%%%%%%%%%%%%%%%%%
\subsection{Included Files}
\label{sec:include}

%%%%%%%%%%%%%%%%%%%%%%%%%%%%%%%%%%%%%%%%
\DescribeMacro{\childdocmain}
To use the package, add the commands
\begin{center}
\begin{tabular}{l}
|\input{childdoc.def}|\\
|\childdocmain{}|\\
\end{tabular}
\end{center}
at the very top of the main \LaTeX{} file,
in particular \emph{before} the |\documentclass| statement!
The argument of |\childdocmain| should be left empty
(but it must be present).

%%%%%%%%%%%%%%%%%%%%%%%%%%%%%%%%%%%%%%%%
\DescribeMacro{\childdocof}
Furthermore, add the commands
\begin{center}
\begin{tabular}{l}
|\input{childdoc.def}|\\
|\childdocof{|\textit{main}|}|\\
\end{tabular}
\end{center}
at the top of every child file \textit{child}
which is included by |\include{|\textit{child}|}|
from within the main file
(or at least for those files to be compiled individually).
The argument \textit{main} must be the filename of the main file.

There are a couple of
considerations in setting up the main and child documents:

%%%%%%%%%%%%%%%%%%%%%%%%%%%%%%%%%%%%%%%%
\paragraph{Restrictions.}

Please note the following restrictions:
\begin{itemize}
\item
|\childdocmain| must be called with one argument \textit{main}
to ensure compatibility with earlier version of the package.
It must either be empty (|\childdocmain{}|)
or precisely match the filename of the main file in which it is specified.
See \secref{sec:detection} for further information.
\item
The filename \textit{main} must be specified without the |.tex| extension.
\item
The filename \textit{main} is case sensitive
(even in case-insensitive file systems)
due to internal string comparison.
\item
The argument \textit{main} should be fully expanded, it cannot be a macro.
\item
Subdirectories and special characters should be avoided in filenames.
\item
The command |\childdocmain{|\textit{main}|}| must be followed by a whitespace.
It should not be followed immediately by another command
or by a comment mark `|%|'.
This is because the \TeX{} parser reads the token immediately following
the argument of |\childdocmain| and puts it
at the beginning of every child section;
however, a white\-space is ignored.
\end{itemize}

%%%%%%%%%%%%%%%%%%%%%%%%%%%%%%%%%%%%%%%%
\paragraph{Content of Main File.}

It is advisable to place all content in the child files included by |\include|.
Any output contained in the main file will appear in all child documents
unless suppressed manually;
it cannot be suppressed automatically by the |\includeonly| directive
and thus should normally be avoided.
A method to include some content in the main file
by means of conditional processing is described in \secref{sec:conditional}.

%%%%%%%%%%%%%%%%%%%%%%%%%%%%%%%%%%%%%%%%
\paragraph{Page Numbering.}

When only a part of the document is compiled,
the appropriate numbering of pages
(as well as other status parameters)
is determined from the |.aux| files.
The latter contain information from previous passes.
However this information needs to propagate through
all intermediate child documents.
Therefore the page numbering in child documents may well
be inconsistent until the complete document is compiled at least once.

A useful (if unconventional) way to always ensure a consistent
page numbering is to restart the numbering in each child document
and denote the pages by `\textit{child}|.|\textit{page}'
where \textit{child} represents the chapter/section number of the child file.
This can be achieved by the command
|\numberwithin{page}{|\textit{child}|}|
of the \textsf{amsmath} package
where \textit{child} can be |chapter| or |section|
depending on the chosen structuring.
Alternatively, one can modify the macro |\thepage| appropriately
and reset the counter |page| at the start of each child file.

%%%%%%%%%%%%%%%%%%%%%%%%%%%%%%%%%%%%%%%%%%%%%%%%%%%%%%%%%%%%%%%%%%%%%%%%%%%%%%%%
\subsection{Conditional Processing}
\label{sec:conditional}

The package provides a mechanism to compile different versions
of a document. To customise the versions further some conditional processing
can come in handy to distinguish which version is being compiled.
The package provides two macros to describe the compilation context:

%%%%%%%%%%%%%%%%%%%%%%%%%%%%%%%%%%%%%%%%
\DescribeMacro{\ifchilddoc}
The conditional |\ifchilddoc| distinguishes between the compilation of
child documents and the main document:
%
\begin{center}
|\ifchilddoc |\textit{child-code}| |[|\||else |\textit{main-code}]| \||fi|
\end{center}

%%%%%%%%%%%%%%%%%%%%%%%%%%%%%%%%%%%%%%%%
\DescribeMacro{\childdocname}
\DescribeMacro{\childdocjob}
The macro |\childdocname| contains the filename (without extension)
of the main or child file being processed.
Note that |\childdocjob| will always contain the name of the main file.

%%%%%%%%%%%%%%%%%%%%%%%%%%%%%%%%%%%%%%%%
\paragraph{Title Page.}

Conditional processing can be used to include a title or banner page
in the main document when proper precautions are taken.
Importantly, the code in the main file should ensure that the page counter
(as well as other status parameters which are stored in the |.aux| files)
takes the same value after the conditional processing.
Otherwise the page numbers may take divergent values
depending on which part is compiled.

For example, a title page could be declared by:
%
\begin{center}
\begin{tabular}{l}
|\ifchilddoc\||else|\\
|\addtocounter{page}{-1}|\\
\textit{code for title page}\\
|\newpage|\\
|\||fi|
\end{tabular}
\end{center}
%
A banner page for the child documents can be generated by:
%
\begin{center}
\begin{tabular}{l}
|\ifchilddoc|\\
|\addtocounter{page}{-1}|\\
\textit{code for banner page}\\
|\newpage|\\
|\||fi|
\end{tabular}
\end{center}
%
Here one could write a message such as:
\begin{center}
|This is the part \childdocname{} of \childdocjob{}.|
\end{center}

%%%%%%%%%%%%%%%%%%%%%%%%%%%%%%%%%%%%%%%%%%%%%%%%%%%%%%%%%%%%%%%%%%%%%%%%%%%%%%%%
\subsection{Flags}
\label{sec:flags}

The package makes it easy to generate different versions
of the main or child documents.
To this end compilation flags can be defined
and assigned different default values.
They will be particularly useful in conjunction
with the forwarding mechanism described in \secref{sec:forward}.

For example, it may be useful to have a flag |\version|
which can be set to |draft| or |final|.
The document source will contain some conditional code
depending on the value of |\version|.
Suppose further, the flag should default to |final| for the main file
and to |draft| for child files
which is a natural assignment for editing the document.
This is achieved by placing the following code
in the preamble of the main document
(below the |\childdocmain| directive):
%
\begin{center}
\begin{tabular}{l}
|\ifchilddoc|\\
|\providecommand{\version}{draft}|\\
|\||else|\\
|\providecommand{\version}{final}|\\
|\||fi|
\end{tabular}
\end{center}
%
The definition by |\providecommand| makes sure
that previous definitions are not overwritten.
Further statements |\providecommand{\version}{...}|
can thus be added before the above code to override it.

For the main file, one might add a line
(between |\childdocmain| and the above block)
%
\begin{center}
|%\ifchilddoc\||else\providecommand{\version}{draft}\||fi|
\end{center}
%
which can be uncommented to produce a draft version.
Likewise one can add a line to the very top of a child file
(above the |\childdocof{|\textit{main}|}| directive)
%
\begin{center}
|%\providecommand{\version}{final}|
\end{center}
%
which can be uncommented to produce the final version of this child document.

%%%%%%%%%%%%%%%%%%%%%%%%%%%%%%%%%%%%%%%%%%%%%%%%%%%%%%%%%%%%%%%%%%%%%%%%%%%%%%%%
\subsection{Forwarding}
\label{sec:forward}

Different versions of the main or child documents
using compilation flags as described in \secref{sec:flags}
can be (permanently) stored in different files
for convenient compilation, viewing and distribution.
To this end, the package defines a command
to pass on compilation to a different file:

%%%%%%%%%%%%%%%%%%%%%%%%%%%%%%%%%%%%%%%%
\DescribeMacro{\childdocforward}
The command |\childdocforward| redirects processing to
another source file:
%
\begin{center}
\begin{tabular}{l}
|\input{childdoc.def}|\\
|\childdocforward[|\textit{main}|]{|\textit{dest}|}|\\
\end{tabular}
\end{center}
%
The argument \textit{dest} is the destination file
(without extension).
It should be the main file or one of the child files.
Note that further \textsf{childdoc} directives
such as |\childdocof| and |\childdocforward|
in the indicated file will be processed in this form.
The optional argument \textit{main}
passes on directly to the main file \textit{main}
while pretending to compile the child \textit{dest}.
This form behaves as if \textit{dest}
issues |\childdocof{|\textit{main}|}| right away,
and no further \textsf{childdoc} directives will be processed.

%%%%%%%%%%%%%%%%%%%%%%%%%%%%%%%%%%%%%%%%
\DescribeMacro{\...prefix}
In the alternative form |\childdocforwardprefix|,
%
\begin{center}
\begin{tabular}{l}
|\input{childdoc.def}|\\
|\childdocforwardprefix[|\textit{main}|]{|\textit{prefix}|}{|\textit{dest}|}|
\end{tabular}
\end{center}
%
the destination file is determined by a pattern
depending on the current file:
To make this work, the current file must be called
`{\textit{prefix}\hspace{0.2em}\textit{suffix}}'
with \textit{prefix} matching precisely the argument.
Processing is then passed on to the file
`{\textit{dest}\hspace{0.2em}\textit{suffix}}'.
Surely, the same effect is achieved by
directly specifying the
argument `{\textit{dest}\hspace{0.2em}\textit{suffix}}'
in the first form.
However, that requires to set up a different file
for each child. With the alternative form of the command
all these files can have exactly the same content
which simplifies setting them up and maintaining them.

For example, the following file |draft.tex|
with a compilation flag |\version| as described in \secref{sec:flags}
compiles the main document as a draft:
%
\begin{center}
\begin{tabular}{l}
|\def\version{draft}|\\
|\input{childdoc.def}|\\
|\childdocforward{|\textit{main}|}|
\end{tabular}
\end{center}
%
Likewise, the following files |final|\textit{nn}|.tex|
compile the final version of the child document
|child|\textit{nn}|.tex|:
%
\begin{center}
\begin{tabular}{l}
|\def\version{final}|\\
|\input{childdoc.def}|\\
|\childdocforwardprefix{final}{child}|
\end{tabular}
\end{center}
%

Note that when several versions of a main file and/or of each child file
are to be generated, it may be convenient to set up a |Makefile| or
shell script to automatise the process.

%%%%%%%%%%%%%%%%%%%%%%%%%%%%%%%%%%%%%%%%%%%%%%%%%%%%%%%%%%%%%%%%%%%%%%%%%%%%%%%%
\subsection{Command Line Processing}
\label{sec:commandline}

The effect of redirection files can also be achieved by invoking
the \LaTeX{} compiler with a more elaborate command line.
Most conveniently this should be done as part
of a shell script or a |Makefile|.

When using \textsf{childdoc} in the main file, the following
command lines effectively perform a redirection
(note that depending on the shell being used,
backslashes may have to be doubled: `|\|' $\to$ `|\\|'):
%
\begin{center}
|... -jobname "|\textit{target}|" |\\|"|[\textit{flags}]%
|\input{childdoc.def}\childdocforward[|\textit{main}|]{|\textit{dest}|}"|
\end{center}
%
Here \textit{target} is the name of the output file,
\textit{main} is the name of the main file
and \textit{dest} is the name of the main or child file to be processed
(all filenames without extensions).
The optional argument \textit{main} can be omitted
if \textit{main} matches \textit{dest}.
Optionally, compilation \textit{flags} can be defined via |\def| commands.
This command line makes the \TeX{} engine believe
it is compiling the file \textit{target}
whose content is specified as the latter parameter.
The provided code then forwards the processing to
\textit{main} or \textit{dest} as described in \secref{sec:forward}.

%%%%%%%%%%%%%%%%%%%%%%%%%%%%%%%%%%%%%%%%%%%%%%%%%%%%%%%%%%%%%%%%%%%%%%%%%%%%%%%%
\subsection{Include by Input}
\label{sec:input}

Including child documents by |\include| has some restrictions by design.
Most notably, the content of a child document always occupies
its own set of pages; pages cannot be shared between child documents.
Usually, this behaviour makes perfect sense
because each child document contain an essential part of the document.
However, in some situations it may be desirable to compose
a document from a collection of parts
without having mandatory page breaks between then.
For this case, the package
provides a mechanism to include parts
by |\input| which can also be processed individually.
However, by construction this mechanism
requires manual handling of the content to be output.

%%%%%%%%%%%%%%%%%%%%%%%%%%%%%%%%%%%%%%%%
\DescribeMacro{\ifchilddocmanual}
The main file should be prepared as usual, see \secref{sec:include}.
However, the document body must make a distinction
between processing of an individual part and of the main document, e.g.:
%
\begin{center}
\begin{tabular}{l}
|\ifchilddocmanual|\\
|\input{\childdocname}|\\
|\||else|\\
\textit{document body with }|\input{|\textit{part}|}|\\
|\||fi|
\end{tabular}
\end{center}
%
The conditional |\ifchilddocmanual| is true whenever
a part to be included by |\input| is being compiled,
and the name of the part is stored in |\childdocname|.

%%%%%%%%%%%%%%%%%%%%%%%%%%%%%%%%%%%%%%%%
\DescribeMacro{\childdocby}
Each part to be included by |\input| should start with:
%
\begin{center}
\begin{tabular}{l}
|\input{childdoc.def}|\\
|\childdocby{|\textit{main}|}|\\
\end{tabular}
\end{center}
%
The directive |\childdocby| is similar to |\childdocof|
described in \secref{sec:include},
but the subsequent selection of content must be done manually.
To that end, both |\ifchilddoc| and |\ifchilddocmanual|
will be true upon processing of a part,
and the name of the part is stored in |\childdocname|.
Note that |\jobname| will be set to the filename of the current part
so that each part receives an individual |.aux| file
that does not interfere with the |.aux| file(s) of the main document.
This behaviour can be altered by the alternative form
|\childdocby[*]{|\textit{main}|}| (with a non-empty optional argument)
which uses the |.aux| file of the main document
by setting |\jobname| to \textit{main}.

%%%%%%%%%%%%%%%%%%%%%%%%%%%%%%%%%%%%%%%%%%%%%%%%%%%%%%%%%%%%%%%%%%%%%%%%%%%%%%%%
\subsection{Driver Development}
\label{sec:driver}

The \textsf{childdoc} mechanism can also be use for the development
of definition files such as \LaTeX{} styles or classes.
This case differs from the above setup with multiple parts
included by |\include| in that no |\includeonly| should be invoked.
This can be achieved by starting the include file
(before |\ProvidesPackage|) with:
%
\begin{center}
\begin{tabular}{l}
|\input{childdoc.def}|\\
|\childdocforward{|\textit{main}|}|\\
\end{tabular}
\end{center}
%
or alternatively with:
%
\begin{center}
\begin{tabular}{l}
|\input{childdoc.def}|\\
|\childdocby{|\textit{main}|}|\\
\end{tabular}
\end{center}
%
Both forms have slightly different effects as described above.
The main file is prepared as usual, see \secref{sec:include}.

%%%%%%%%%%%%%%%%%%%%%%%%%%%%%%%%%%%%%%%%%%%%%%%%%%%%%%%%%%%%%%%%%%%%%%%%%%%%%%%%
\subsection{Legacy Detection}
\label{sec:detection}

The directive |\childdocmain| in the main file can detect
whether the complete document or merely a child is to be compiled
even without using the directive |\childdocof|.
This method is deprecated because it is less robust
and there is no compelling reason to use it;
it is merely provided for backward compatibility
and it may be removed in future versions.

If the detection mechanism is to be used,
it is mandatory to correctly specify
the filename of the main file as the argument of |\childdocmain|:
%
\begin{center}
\begin{tabular}{l}
|\input{childdoc.def}|\\
|\childdocmain{|\textit{main}|}|\\
\end{tabular}
\end{center}
%
If |\jobname| does not match the argument \textit{main} of |\childdocmain|,
it is assumed that |\jobname| points to the child file to be compiled.
When using |\childdocmain| with the main file specified as argument,
it suffices to start a child file
with just |\input{|\textit{main}|}|
without loading of the package and using |\childdocof|.
If instead all processing is done
with the appropriate \textsf{childdoc} directives,
the argument of \textit{main} of |\childdocmain| can be empty.

An alternative version of the command line processing described
in \secref{sec:commandline} using the detection mechanism reads:
%
\begin{center}
|... -jobname "|\textit{target}|" "|[\textit{flags}]%
[|\def\jobname{|\textit{dest}|}|]|\input{|\textit{main}|}"|
\end{center}

%%%%%%%%%%%%%%%%%%%%%%%%%%%%%%%%%%%%%%%%%%%%%%%%%%%%%%%%%%%%%%%%%%%%%%%%%%%%%%%%
\subsection{Manual Code}
\label{sec:manual}

In case one cannot be certain whether the definitions file |childdoc.def|
is installed on the target \TeX{} distribution
and one prefers not to ship it,
it is conceivable to paste a few relevant commands into the sources.

To that end, drop all statements |\input{childdoc.def}|
and perform the replacements as outlined below.
Instead of |\childdocmain{|\textit{main}|}| add the following code
to the top of the main file:
%
\begin{center}
\begin{tabular}{l}
|\||ifdefined\childdocname\endinput\||fi\newif\ifchilddoc|\\
|\edef\childdocname{\scantokens\expandafter{\jobname\noexpand}}|\\
|\def\childdocmain{|\textit{main}|}\||ifx\childdocmain\childdocname\||else|\\
|\childdoctrue\includeonly{\childdocname}\let\jobname\childdocmain\||fi|\\
\end{tabular}
\end{center}
%
Instead of |\childdocof{|\textit{main}|}| just include the main file
at the top of each child file:
%
\begin{center}
|\input{|\textit{main}|}|
\end{center}
%
A simple redirection |\childdocforward{|\textit{dest}|}| is achieved by:
%
\begin{center}
|\def\jobname{|\textit{dest}|}\input{\jobname}|
\end{center}
%
The redirection with prefix
|\childdocforwardprefix[|\textit{prefix}|]{|\textit{dest}|}|
is accomplished by:
%
\begin{center}
\begin{tabular}{l}
|{\edef\jobname{\scantokens\expandafter{\jobname\noexpand}}|\\
|\def\redirectjob |\textit{prefix}|#1~~~{\gdef\jobname{|\textit{dest}|#1}}|\\
|\expandafter\redirectjob\jobname~~~}\input{\jobname}|
\end{tabular}
\end{center}

In an alternative approach,
child documents can be compiled by a specific command line
without additional code or specific definitions:
%
\begin{center}
|... -jobname "|\textit{target}|" "|[\textit{flags}]%
|\includeonly{|\textit{dest}|}\input{|\textit{main}|}"|
\end{center}
%

%%%%%%%%%%%%%%%%%%%%%%%%%%%%%%%%%%%%%%%%%%%%%%%%%%%%%%%%%%%%%%%%%%%%%%%%%%%%%%%%
%%%%%%%%%%%%%%%%%%%%%%%%%%%%%%%%%%%%%%%%%%%%%%%%%%%%%%%%%%%%%%%%%%%%%%%%%%%%%%%%
\section{Information}

%%%%%%%%%%%%%%%%%%%%%%%%%%%%%%%%%%%%%%%%%%%%%%%%%%%%%%%%%%%%%%%%%%%%%%%%%%%%%%%%
\subsection{Copyright}

Copyright \copyright{} 2017--2018 Niklas Beisert

This work may be distributed and/or modified under the
conditions of the \LaTeX{} Project Public License, either version 1.3
of this license or (at your option) any later version.
The latest version of this license is in
  \url{http://www.latex-project.org/lppl.txt}
and version 1.3 or later is part of all distributions of \LaTeX{}
version 2005/12/01 or later.

This work has the LPPL maintenance status `maintained'.

The Current Maintainer of this work is Niklas Beisert.

This work consists of the files |README.txt|, |childdoc.ins| and |childdoc.dtx|
as well as the derived files |childdoc.def|, |cdocsamp.tex|
with |cdocsch1.tex|, |cdocsch2.tex|, |cdocspt3.tex|, |cdocspt4.tex|,
|cdocsdrf.tex|, |cdocsfn1.tex|, |cdocsfn2.tex|
as well as |childdoc.pdf|.

%%%%%%%%%%%%%%%%%%%%%%%%%%%%%%%%%%%%%%%%%%%%%%%%%%%%%%%%%%%%%%%%%%%%%%%%%%%%%%%%
\subsection{Files and Installation}

The package consists of the files:
%
\begin{center}
\begin{tabular}{ll}
    |README.txt|   & readme file \\
    |childdoc.ins| & installation file \\
    |childdoc.dtx| & source file \\
    |childdoc.def| & definition file \\
    |cdocsamp.tex| & sample main file \\
    |cdocsch1.tex| & sample include file \\
    |cdocsch2.tex| & sample include file \\
    |cdocspt3.tex| & sample part file \\
    |cdocspt4.tex| & sample part file \\
    |cdocsdrf.tex| & sample redirection file \\
    |cdocsfn1.tex| & sample redirection file \\
    |cdocsfn2.tex| & sample redirection file \\
    |childdoc.pdf| & manual
\end{tabular}
\end{center}
%
The distribution consists of the files
|README.txt|, |childdoc.ins| and |childdoc.dtx|.
%
\begin{itemize}
\item
Run (pdf)\LaTeX{} on |childdoc.dtx|
to compile the manual |childdoc.pdf| (this file).
\item
Run \LaTeX{} on |childdoc.ins| to create the definitions file |childdoc.def|
and the sample |cdocsamp.tex| with include files
|cdocsch1.tex|, |cdocsch2.tex|, |cdocspt3.tex|, |cdocspt4.tex|,
|cdocsdrf.tex|, |cdocsfn1.tex|, |cdocsfn2.tex|.
Then copy the file |childdoc.def| to an appropriate directory of your \LaTeX{}
distribution, e.g.\ \textit{texmf-root}|/tex/latex/childdoc|.
\end{itemize}

%%%%%%%%%%%%%%%%%%%%%%%%%%%%%%%%%%%%%%%%%%%%%%%%%%%%%%%%%%%%%%%%%%%%%%%%%%%%%%%%
\subsection{Related CTAN Packages}

There are several other packages which offer a similar functionality:
%
\begin{itemize}
\item
The packages
\href{http://ctan.org/pkg/docmute}{\textsf{docmute}},
\href{http://ctan.org/pkg/includex}{\textsf{includex}} and
\href{http://ctan.org/pkg/standalone}{\textsf{standalone}}
provide commands to include only the document body of
a child file thus allowing both files to be compiled individually.
\item
The packages \href{http://ctan.org/pkg/subdocs}{\textsf{subdocs}}
and \href{http://ctan.org/pkg/subfiles}{\textsf{subfiles}}
provide structures in which the main and child documents can be
encapsulated and allowing them to be compiled individually.
The inclusion mechanism is different from the conventional |\include|.
\item
The package \href{http://ctan.org/pkg/combine}{\textsf{combine}}
is an elaborate solution to combine several documents into one.
\end{itemize}
%
See also the CTAN topic \href{http://ctan.org/topic/subdocs}{\textsf{subdocs}}
for further related packages.
The present package differs from the above solutions in that
a document structure constructed with the conventional |\include| mechanism
just needs two extra commands at the top of every file
such that all constituent files can be compiled individually.

%%%%%%%%%%%%%%%%%%%%%%%%%%%%%%%%%%%%%%%%%%%%%%%%%%%%%%%%%%%%%%%%%%%%%%%%%%%%%%%%
%\subsection{Feature Suggestions}
%
%The following is a list of features which may be useful for future
%versions of this package:
%%
%\begin{itemize}
%\item
%\ldots
%\end{itemize}

%%%%%%%%%%%%%%%%%%%%%%%%%%%%%%%%%%%%%%%%%%%%%%%%%%%%%%%%%%%%%%%%%%%%%%%%%%%%%%%%
\subsection{Revision History}

%%%%%%%%%%%%%%%%%%%%%%%%%%%%%%%%%%%%%%%%
\paragraph{v2.0:} 2018/12/30

\begin{itemize}
\item
immediate forward processing
\item
added |\childdocby| mechanism
\item
manual restructured
\end{itemize}

%%%%%%%%%%%%%%%%%%%%%%%%%%%%%%%%%%%%%%%%
\paragraph{v1.6:} 2018/01/17

\begin{itemize}
\item
application for development of include files
\item
corrections to manual
\end{itemize}

%%%%%%%%%%%%%%%%%%%%%%%%%%%%%%%%%%%%%%%%
\paragraph{v1.5:} 2017/05/21

\begin{itemize}
\item
more complete structuring introduced
\item
|\childdocof| introduced
\item
|\childdoc| renamed to |\childdocmain|
\item
|\childredirect| renamed to |\childdocforward| and |\childdocforwardprefix|
and functionality expanded
\end{itemize}

%%%%%%%%%%%%%%%%%%%%%%%%%%%%%%%%%%%%%%%%
\paragraph{v1.0:} 2017/04/27

\begin{itemize}
\item
manual and install package
\item
first version published on CTAN
\end{itemize}

%%%%%%%%%%%%%%%%%%%%%%%%%%%%%%%%%%%%%%%%
\paragraph{v0.6:} 2017/04/26

\begin{itemize}
\item
redirection mechanism added
\end{itemize}

%%%%%%%%%%%%%%%%%%%%%%%%%%%%%%%%%%%%%%%%
\paragraph{v0.5:} 2017/04/26

\begin{itemize}
\item
functionality in definition file
\end{itemize}


%%%%%%%%%%%%%%%%%%%%%%%%%%%%%%%%%%%%%%%%%%%%%%%%%%%%%%%%%%%%%%%%%%%%%%%%%%%%%%%%
%%%%%%%%%%%%%%%%%%%%%%%%%%%%%%%%%%%%%%%%%%%%%%%%%%%%%%%%%%%%%%%%%%%%%%%%%%%%%%%%
%%%%%%%%%%%%%%%%%%%%%%%%%%%%%%%%%%%%%%%%%%%%%%%%%%%%%%%%%%%%%%%%%%%%%%%%%%%%%%%%
\appendix

\settowidth\MacroIndent{\rmfamily\scriptsize 000\ }

 \DocInput{childdoc.dtx}

\end{document}
%</driver>
% \fi
%
% %%%%%%%%%%%%%%%%%%%%%%%%%%%%%%%%%%%%%%%%%%%%%%%%%%%%%%%%%%%%%%%%%%%%%%%%%%%%%%
% %%%%%%%%%%%%%%%%%%%%%%%%%%%%%%%%%%%%%%%%%%%%%%%%%%%%%%%%%%%%%%%%%%%%%%%%%%%%%%
% \section{Sample}
%\iffalse
%<*samplemain>
%\fi
%
% The following presents a sample document
% with two chapters, two parts, a title page,
% a compile flag as well as three forwarding files to set the flag.
% It consists of eight |.tex| files:
% \begin{center}
% \begin{tabular}{ll}
% |cdocsamp.tex|&main file\\
% |cdocsch1.tex|&include file for chapter 1\\
% |cdocsch2.tex|&include file for chapter 2\\
% |cdocspt3.tex|&include file for part 3\\
% |cdocspt4.tex|&include file for part 4\\
% |cdocsdrf.tex|&forwarding file for main file in draft mode\\
% |cdocsfi1.tex|&forwarding file for final version of chapter 1\\
% |cdocsfi2.tex|&forwarding file for final version of chapter 2\\
% \end{tabular}
% \end{center}
% Each of the eight files can be compiled directly by the \LaTeX{} compiler.
%
% %%%%%%%%%%%%%%%%%%%%%%%%%%%%%%%%%%%%%%
% \paragraph{Main File.}
%
% The main file is called |cdocsamp.tex|.
%
% Load the \textsf{childdoc} definitions and
% declare the filename for the main document:
%    \begin{macrocode}
\input{childdoc.def}
\childdocmain{}
%    \end{macrocode}

% Optional override for |\version| flag:
%    \begin{macrocode}
%%\ifchilddoc\else\providecommand{\version}{draft}\fi
%    \end{macrocode}

% Define the default values for the |\version| flag
% (|final| for the main file and |draft| for childs):
%    \begin{macrocode}
\ifchilddoc
\providecommand{\version}{draft}
\else
\providecommand{\version}{final}
\fi
%    \end{macrocode}

% Load the standard document class:
%    \begin{macrocode}
\documentclass[12pt]{article}
%    \end{macrocode}

% Start the document body:
%    \begin{macrocode}
\begin{document}
%    \end{macrocode}

% Declare a title page.
% Print title, part of document being processed and version flag:
%    \begin{macrocode}
\addtocounter{page}{-1}
\begin{center}
{\LARGE\bfseries{}childdoc example\par}
\vspace{1cm}
\ifchilddoc
\ifchilddocmanual part\else chapter\fi:
`\childdocname' of `\childdocjob'\par
\else
main document: `\childdocjob'\par
\fi
version: \version\par
\end{center}
\newpage
%    \end{macrocode}

% Manually include selected file,
% otherwise process as usual:
%    \begin{macrocode}
\ifchilddocmanual
\section*{part `\childdocname'}
\input{\childdocname}
\else
%    \end{macrocode}

% Include the two chapters:
%    \begin{macrocode}
\include{cdocsch1}
\include{cdocsch2}
%    \end{macrocode}

% Include the two parts unless only chapters should be displayed:
%    \begin{macrocode}
\ifchilddoc\else
\section{part three}
\input{cdocspt3}
\section{part four}
\input{cdocspt4}
\fi
%    \end{macrocode}

% Process as usual until here:
%    \begin{macrocode}
\fi
%    \end{macrocode}

% End of document body:
%    \begin{macrocode}
\end{document}
%    \end{macrocode}
%\iffalse
%</samplemain>
%\fi
%
% %%%%%%%%%%%%%%%%%%%%%%%%%%%%%%%%%%%%%%
% \paragraph{Chapter Include Files.}
%
% The include files are called |cdocsch1.tex| and |cdocsch2.tex|.
%
%\iffalse
%<*samplechap1|samplechap2>
%\fi

% Optional override for |\version| flag:
%    \begin{macrocode}
%%\providecommand{\version}{final}
%    \end{macrocode}

% Include the main document:
%    \begin{macrocode}
\input{childdoc.def}
\childdocof{cdocsamp}
%    \end{macrocode}

%\iffalse
%</samplechap1|samplechap2>
%\fi
%
%\iffalse
%<*samplechap1>
%\fi
% Some text for chapter 1:
%    \begin{macrocode}
\section{one}
some text in chapter one
%    \end{macrocode}

%\iffalse
%</samplechap1>
%\fi
% Some text for chapter 2:
%\iffalse
%<*samplechap2>
%\fi
%    \begin{macrocode}
\section{two}
more text in chapter two
%    \end{macrocode}

%\iffalse
%</samplechap2>
%\fi
%
% %%%%%%%%%%%%%%%%%%%%%%%%%%%%%%%%%%%%%%
% \paragraph{Part Include Files.}
%
% The include files are called |cdocspt3.tex| and |cdocspt4.tex|.
%
%\iffalse
%<*samplepart3|samplepart4>
%\fi

% Optional override for |\version| flag:
%    \begin{macrocode}
%%\providecommand{\version}{final}
%    \end{macrocode}

% Include the main document:
%    \begin{macrocode}
\input{childdoc.def}
\childdocby{cdocsamp}
%    \end{macrocode}

%\iffalse
%</samplepart3|samplepart4>
%\fi
%
%\iffalse
%<*samplepart3>
%\fi
% Some text for part 3:
%    \begin{macrocode}
some text in part three
%    \end{macrocode}

%\iffalse
%</samplepart3>
%\fi
% Some text for part 4:
%\iffalse
%<*samplepart4>
%\fi
%    \begin{macrocode}
more text in part four
%    \end{macrocode}

%\iffalse
%</samplepart4>
%\fi
%
% %%%%%%%%%%%%%%%%%%%%%%%%%%%%%%%%%%%%%%
% \paragraph{Forwarding for a Complete Draft.}
%
% The following forwarding file |cdocsdrf.tex|
% compiles the main document in draft mode:
%\iffalse
%<*sampledraft>
%\fi
%    \begin{macrocode}
\def\version{draft}
\input{childdoc.def}
\childdocforward{cdocsamp}
%    \end{macrocode}

%\iffalse
%</sampledraft>
%\fi
%
% %%%%%%%%%%%%%%%%%%%%%%%%%%%%%%%%%%%%%%
% \paragraph{Forwarding for Final Version of the Chapters.}
%
% The following forwarding files |cdocsfn1.tex| and |cdocsfn2.tex|
% (with identical content)
% compile the final versions of the child documents
% |cdocsch1.tex| and |cdocsch2.tex|, respectively:
%\iffalse
%<*samplefinal>
%\fi
%    \begin{macrocode}
\def\version{final}
\input{childdoc.def}
\childdocforwardprefix[cdocsamp]{cdocsfn}{cdocsch}
%    \end{macrocode}

%\iffalse
%</samplefinal>
%\fi
%
% %%%%%%%%%%%%%%%%%%%%%%%%%%%%%%%%%%%%%%
% \paragraph{Command Line Processing.}
%
% The following three command lines generate the output files
% |cdocscld|, |cdocscl1| and |cdocscl2|
% which should be identical to
% |cdocsdrf|, |cdocsch1| and |cdocsfn2|, respectively:
% \begin{center}
% \begin{tabular}{l}
% |latex -jobname cdocscld \|\\
% |  "\def\version{draft}\input{childdoc.def}\childdocforward{cdocsamp}"|\\
% |latex -jobname cdocscl1 \|\\
% |  "\input{childdoc.def}\childdocforward[cdocsamp]{cdocsch1}"|\\
% |latex -jobname cdocscl2 \|\\
% |  "\def\version{final}\input{childdoc.def}\childdocforward{cdocsch2}"|
% \end{tabular}
% \end{center}
% Note that the trailing backslash on each first line
% merely continues the input to the second line
% (for convenient cut ant paste).
% Furthermore, the command |latex| can be replaced by any
% of its alternative versions such as |pdflatex|.
%
% %%%%%%%%%%%%%%%%%%%%%%%%%%%%%%%%%%%%%%%%%%%%%%%%%%%%%%%%%%%%%%%%%%%%%%%%%%%%%%
% %%%%%%%%%%%%%%%%%%%%%%%%%%%%%%%%%%%%%%%%%%%%%%%%%%%%%%%%%%%%%%%%%%%%%%%%%%%%%%
% \section{Implementation}
%\iffalse
%<*package>
%\fi
%
% This section describes the definitions file |childdoc.def|.

% The definitions cannot be loaded using |\usepackage| or |\RequirePackage|
% which has a mechanism to prevent loading a style file more than once.
% When loading the definitions by means of |\input|
% multiple instances have to be prevented manually:
%\iffalse
%This code needs to be before the `\ProvidesFile' directive
%which is defined at the beginning of this file.
%Therefore it is also placed there and commented out here.
%</package>
%<*discard>
%\fi
%    \begin{macrocode}
\ifdefined\childdocmain\endinput\fi
%    \end{macrocode}
%\iffalse
%</discard>
%<*package>
%\fi
%
% \macro{\ifchilddoc}
% \macro{\ifchilddocmanual}
% The conditional |\ifchilddoc| tells whether a
% child (true) or main (false) document is being compiled.
% The conditional |\ifchilddocmanual| tells whether
% the |\includeonly| mechanism is used (false) or
% the selection of child files must be performed manually (true).
% The definitions initialise to false:
%    \begin{macrocode}
\newif\ifchilddoc
\newif\ifchilddocmanual
%    \end{macrocode}

% \macro{\childdocname}
% \macro{\childdocjob}
% The macro |\childdocname| stores the name of the main document
% to be compiled. The macro |\childdocjob| stores the name of
% the document on which the \LaTeX{} compiler was originally invoked.
% The content of |\jobname| cannot be compared
% to filenames specified in the source due to different catcodes.
% The following code rescans |\jobname|, stores the result
% in |\childdocname| and saves a copy in |\childdocjob|:
%    \begin{macrocode}
\edef\childdocname{\scantokens\expandafter{\jobname\noexpand}}
\let\childdocjob\childdocname
%    \end{macrocode}

% \macro{\childdocdisable}
% The macro |\childdocdisable| prevents the main file
% from being processed more than once.
% At this stage, the main document command |\childdocmain|
% is assumed to be called once again where it should do nothing.
% Any subsequent call to it should prevent
% a secondary processing of the main document
% It overwrites the forwarding commands
% |\childdocof| and |\childdocforward|
% with empty macros to prevent further inclusions of the main document:
%    \begin{macrocode}
\newcommand{\childdocdisable}
{
  \renewcommand{\childdocmain}[1]{\renewcommand{\childdocmain}[1]{\endinput}}
  \renewcommand{\childdocof}[1]{}
  \renewcommand{\childdocby}[2][]{}
  \renewcommand{\childdocforward}[2][]{}
  \renewcommand{\childdocdisable}{}
}
%    \end{macrocode}

% \macro{\childdocmain}
% The macro |\childdocmain| is to be called at the top of the main file
% with nothing or the main filename (without extension) as argument.
% First, it breaks loops.
% If the argument is not empty and does not match |\childdocname|
% (which is set by the first inclusion of |childdoc.def|),
% |\ifchilddoc| is set to true, |\includeonly| is applied to the child file
% and |\jobname| is set to the main file
% (for proper handling of |.aux| files):
%    \begin{macrocode}
\newcommand{\childdocmain}[1]
{
  \childdocdisable\childdocmain{}
  \if?#1?\else
    \begingroup
      \def\childdoctmp{#1}
      \ifx\childdoctmp\childdocname
        \def\childdoctmp{}
      \else
        \def\childdoctmp
        {
          \childdoctrue
          \includeonly{\childdocname}
          \def\childdocjob{#1}
          \def\jobname{#1}
        }
      \fi
      \expandafter
    \endgroup
    \childdoctmp
  \fi
}
%    \end{macrocode}

% \macro{\childdocof}
% The command |\childdocof| redirects
% compilation to the main file |#1|.
%    \begin{macrocode}
\newcommand{\childdocof}[1]
{
  \childdocdisable
  \childdoctrue
  \includeonly{\childdocname}
  \def\jobname{#1}
  \def\childdocjob{#1}
  \input{#1}
}
%    \end{macrocode}

% \macro{\childdocby}
% The command |\childdocby| ....
%    \begin{macrocode}
\newcommand{\childdocby}[2][]
{
  \childdocdisable
  \childdoctrue
  \childdocmanualtrue
  \if?#1?\else
    \def\jobname{#2}
  \fi
  \def\childdocjob{#2}
  \input{#2}
  \endinput
}
%    \end{macrocode}

% \macro{\childdocforward}
% The command |\childdocforward| redirects
% compilation to the main file or
% (if the optional argument is given) a child file.
% Parameters are set as if the main file
% or a child file starting with |\childdocof| was compiled.
% Then compilation is handed over to the main file:
%    \begin{macrocode}
\newcommand{\childdocforward}[2][]
{
  \begingroup
    \if?#1?
      \def\childdoctmp
      {
        \def\childdocname{#2}
        \def\childdocjob{#2}
        \def\jobname{#2}
        \input{#2}
        \endinput
      }
    \else
      \def\childdoctmp
      {
        \childdocdisable
        \def\childdocname{#2}
        \childdoctrue
        \includeonly{#2}
        \def\childdocjob{#1}
        \def\jobname{#1}
        \input{#1}
        \endinput
      }
    \fi
    \expandafter
  \endgroup
  \childdoctmp
}
%    \end{macrocode}

% \macro{\childdocforwardprefix}
% The command |\childdocforwardprefix| redirects
% compilation to the main or a child file by means of a pattern.
% The prefix |#1| in the current filename is replaced by |#2|
% and the suffix of the current filename is kept
% (it is assumed that the filename does not contain the substring `|~~~|'
% which is used as a delimiter).
% Compilation is handed over to the new file by |\childdocforward|:
%    \begin{macrocode}
\newcommand{\childdocforwardprefix}[3][]
{
  \begingroup
    \def\childdocextract #2##1~~~{\def\childdoctmp{\childdocforward[#1]{#3##1}}}
    \expandafter\childdocextract\childdocname~~~
    \expandafter
  \endgroup
  \childdoctmp
}
%    \end{macrocode}

% \macro{\childdoc}
% The deprecated macro |\childdoc| is a legacy version of |\childdocmain|:
%    \begin{macrocode}
\newcommand{\childdoc}{\childdocmain}
%    \end{macrocode}

% \macro{\childdocredirect}
% The deprecated macro |\childdocredirect| is a legacy version
% of |\childdocforward| and |\childdocforwardprefix|:
%    \begin{macrocode}
\newcommand{\childdocredirect}[2][]
{
  \begingroup
    \if?#1?
      \def\childdoctmp{\childdocforward{#2}}
    \else
      \def\childdoctmp{\childdocforwardprefix{#1}{#2}}
    \fi
    \expandafter
  \endgroup
  \childdoctmp
}
%    \end{macrocode}

%\iffalse
%</package>
%\fi
%
\endinput
\childdocforward{cdocsamp}"|\\
% |latex -jobname cdocscl1 \|\\
% |  "% \iffalse
%
% childdoc.dtx Copyright (C) 2017-2018 Niklas Beisert
%
% This work may be distributed and/or modified under the
% conditions of the LaTeX Project Public License, either version 1.3
% of this license or (at your option) any later version.
% The latest version of this license is in
%   http://www.latex-project.org/lppl.txt
% and version 1.3 or later is part of all distributions of LaTeX
% version 2005/12/01 or later.
%
% This work has the LPPL maintenance status `maintained'.
%
% The Current Maintainer of this work is Niklas Beisert.
%
% This work consists of the files childdoc.dtx and childdoc.ins
% and the derived files childdoc.def and cdocsamp.tex with
% cdocsch1.tex, cdocsch2.tex, cdocsdrf.tex, cdocsfn1.tex, cdocsfn2.tex.
%
%<package>\ifdefined\childdocmain\endinput\fi
%<package>\ProvidesFile{childdoc.def}[2018/12/30 v2.0 child document driver]
%<samplemain>\ProvidesFile{cdocsamp.tex}[2018/12/30 v2.0 sample for childdoc]
%<*driver>
%\ProvidesFile{childdoc.drv}[2018/12/30 v2.0 childdoc reference manual file]
\PassOptionsToClass{10pt,a4paper}{article}
\documentclass{ltxdoc}

\usepackage[margin=35mm]{geometry}
\usepackage{hyperref}
\usepackage{hyperxmp}
\usepackage[usenames]{color}

\hypersetup{colorlinks=true}
\hypersetup{pdfstartview=FitH}
\hypersetup{pdfpagemode=UseNone}
\hypersetup{pdfsource={}}
\hypersetup{pdflang={en-UK}}
\hypersetup{pdfcopyright={Copyright 2017-2018 Niklas Beisert.
  This work may be distributed and/or modified under the
  conditions of the LaTeX Project Public License, either version 1.3
  of this license or (at your option) any later version.}}
\hypersetup{pdflicenseurl={http://www.latex-project.org/lppl.txt}}
\hypersetup{pdfcontactaddress={ETH Zurich, ITP, HIT K,
  Wolfgang-Pauli-Strasse 27}}
\hypersetup{pdfcontactpostcode={8093}}
\hypersetup{pdfcontactcity={Zurich}}
\hypersetup{pdfcontactcountry={Switzerland}}
\hypersetup{pdfcontactemail={nbeisert@itp.phys.ethz.ch}}
\hypersetup{pdfcontacturl={http://people.phys.ethz.ch/\xmptilde nbeisert/}}

\newcommand{\secref}[1]{\hyperref[#1]{section \ref*{#1}}}

\parskip1ex
\parindent0pt
\let\olditemize\itemize
\def\itemize{\olditemize\parskip0pt}

\begin{document}

\title{The \textsf{childdoc} Package}
\hypersetup{pdftitle={The childdoc Package}}
\author{Niklas Beisert\\[2ex]
  Institut f\"ur Theoretische Physik\\
  Eidgen\"ossische Technische Hochschule Z\"urich\\
  Wolfgang-Pauli-Strasse 27, 8093 Z\"urich, Switzerland\\[1ex]
  \href{mailto:nbeisert@itp.phys.ethz.ch}
  {\texttt{nbeisert@itp.phys.ethz.ch}}}
\hypersetup{pdfauthor={Niklas Beisert}}
\hypersetup{pdfsubject={Manual for the LaTeX2e Package childdoc}}
\date{30 December 2018, \textsf{v2.0}}
\maketitle

\begin{abstract}\noindent
\textsf{childdoc} is a \LaTeXe{} package
that enables the direct compilation
of document sections included by |\include|
to individual files.
\end{abstract}

\begingroup
\parskip0ex
\tableofcontents
\endgroup

%%%%%%%%%%%%%%%%%%%%%%%%%%%%%%%%%%%%%%%%%%%%%%%%%%%%%%%%%%%%%%%%%%%%%%%%%%%%%%%%
%%%%%%%%%%%%%%%%%%%%%%%%%%%%%%%%%%%%%%%%%%%%%%%%%%%%%%%%%%%%%%%%%%%%%%%%%%%%%%%%
\section{Introduction}

\LaTeX{} provides a mechanism to structure a large document (such as a book)
into a main file and several child files (containing the chapters)
using the |\include| command.
This mechanism is beneficial for documents
which span hundreds of pages in order to
make the source file(s) more manageable.
Moreover, compilation can be restricted to
selected child files by means of the |\includeonly| command.
The latter feature can be used to reduce the compilation time while editing
(this was significantly more useful in the earlier days of \LaTeX{})
or to generate a smaller document which is easier to navigate.
Another application of |\includeonly| is to generate
documents consisting of selected parts of the complete document.

However, there are a few drawbacks of the plain |\include| mechanism:
\begin{itemize}
\item
The child files cannot be compiled on their own,
they can only be compiled via the main file.
A naive editing environment
(such as a text editor with an option
to have the current file processed by \LaTeX)
may require one to switch to the main file before compiling;
attempting to compile the child file produces errors.
\item
The main file must be modified (each time)
to adjust the |\includeonly| command
to the present needs. This easily leaves the main file in a messy state.
\item
The generated document will always carry the filename
of the main document. This is inconvenient if
several child files are to be compiled and
to be kept for distribution.
\end{itemize}

The present package provides a simple interface
to make child files individually compilable by \LaTeX{}.
Compiling a child file then has the same effect as compiling
the main file with an |\includeonly| command
to select the appropriate child.
Moreover the generated document will carry the name of the child
rather than the main file.
This resolves all three above issues.

This feature is meant to make the editing of books,
thesis documents and lecture notes somewhat more convenient.
However, the package can also be used efficiently for
composing a series of documents (such as exercise sheets)
which are typically distributed individually.
It then assists the author in generating the individual documents
(potentially in different versions)
as well as a document containing the collected series.
Another application is in developing style files
or other kinds of included material
where compilation of the style file could redirect
to a sample or test file.

%%%%%%%%%%%%%%%%%%%%%%%%%%%%%%%%%%%%%%%%%%%%%%%%%%%%%%%%%%%%%%%%%%%%%%%%%%%%%%%%
%%%%%%%%%%%%%%%%%%%%%%%%%%%%%%%%%%%%%%%%%%%%%%%%%%%%%%%%%%%%%%%%%%%%%%%%%%%%%%%%
\section{Usage}

First of all, the package \textsf{childdoc} is \emph{not} a standard
\LaTeXe{} |.sty| style file! Therefore it needs to be invoked in
a non-standard way.

%%%%%%%%%%%%%%%%%%%%%%%%%%%%%%%%%%%%%%%%%%%%%%%%%%%%%%%%%%%%%%%%%%%%%%%%%%%%%%%%
\subsection{Included Files}
\label{sec:include}

%%%%%%%%%%%%%%%%%%%%%%%%%%%%%%%%%%%%%%%%
\DescribeMacro{\childdocmain}
To use the package, add the commands
\begin{center}
\begin{tabular}{l}
|\input{childdoc.def}|\\
|\childdocmain{}|\\
\end{tabular}
\end{center}
at the very top of the main \LaTeX{} file,
in particular \emph{before} the |\documentclass| statement!
The argument of |\childdocmain| should be left empty
(but it must be present).

%%%%%%%%%%%%%%%%%%%%%%%%%%%%%%%%%%%%%%%%
\DescribeMacro{\childdocof}
Furthermore, add the commands
\begin{center}
\begin{tabular}{l}
|\input{childdoc.def}|\\
|\childdocof{|\textit{main}|}|\\
\end{tabular}
\end{center}
at the top of every child file \textit{child}
which is included by |\include{|\textit{child}|}|
from within the main file
(or at least for those files to be compiled individually).
The argument \textit{main} must be the filename of the main file.

There are a couple of
considerations in setting up the main and child documents:

%%%%%%%%%%%%%%%%%%%%%%%%%%%%%%%%%%%%%%%%
\paragraph{Restrictions.}

Please note the following restrictions:
\begin{itemize}
\item
|\childdocmain| must be called with one argument \textit{main}
to ensure compatibility with earlier version of the package.
It must either be empty (|\childdocmain{}|)
or precisely match the filename of the main file in which it is specified.
See \secref{sec:detection} for further information.
\item
The filename \textit{main} must be specified without the |.tex| extension.
\item
The filename \textit{main} is case sensitive
(even in case-insensitive file systems)
due to internal string comparison.
\item
The argument \textit{main} should be fully expanded, it cannot be a macro.
\item
Subdirectories and special characters should be avoided in filenames.
\item
The command |\childdocmain{|\textit{main}|}| must be followed by a whitespace.
It should not be followed immediately by another command
or by a comment mark `|%|'.
This is because the \TeX{} parser reads the token immediately following
the argument of |\childdocmain| and puts it
at the beginning of every child section;
however, a white\-space is ignored.
\end{itemize}

%%%%%%%%%%%%%%%%%%%%%%%%%%%%%%%%%%%%%%%%
\paragraph{Content of Main File.}

It is advisable to place all content in the child files included by |\include|.
Any output contained in the main file will appear in all child documents
unless suppressed manually;
it cannot be suppressed automatically by the |\includeonly| directive
and thus should normally be avoided.
A method to include some content in the main file
by means of conditional processing is described in \secref{sec:conditional}.

%%%%%%%%%%%%%%%%%%%%%%%%%%%%%%%%%%%%%%%%
\paragraph{Page Numbering.}

When only a part of the document is compiled,
the appropriate numbering of pages
(as well as other status parameters)
is determined from the |.aux| files.
The latter contain information from previous passes.
However this information needs to propagate through
all intermediate child documents.
Therefore the page numbering in child documents may well
be inconsistent until the complete document is compiled at least once.

A useful (if unconventional) way to always ensure a consistent
page numbering is to restart the numbering in each child document
and denote the pages by `\textit{child}|.|\textit{page}'
where \textit{child} represents the chapter/section number of the child file.
This can be achieved by the command
|\numberwithin{page}{|\textit{child}|}|
of the \textsf{amsmath} package
where \textit{child} can be |chapter| or |section|
depending on the chosen structuring.
Alternatively, one can modify the macro |\thepage| appropriately
and reset the counter |page| at the start of each child file.

%%%%%%%%%%%%%%%%%%%%%%%%%%%%%%%%%%%%%%%%%%%%%%%%%%%%%%%%%%%%%%%%%%%%%%%%%%%%%%%%
\subsection{Conditional Processing}
\label{sec:conditional}

The package provides a mechanism to compile different versions
of a document. To customise the versions further some conditional processing
can come in handy to distinguish which version is being compiled.
The package provides two macros to describe the compilation context:

%%%%%%%%%%%%%%%%%%%%%%%%%%%%%%%%%%%%%%%%
\DescribeMacro{\ifchilddoc}
The conditional |\ifchilddoc| distinguishes between the compilation of
child documents and the main document:
%
\begin{center}
|\ifchilddoc |\textit{child-code}| |[|\||else |\textit{main-code}]| \||fi|
\end{center}

%%%%%%%%%%%%%%%%%%%%%%%%%%%%%%%%%%%%%%%%
\DescribeMacro{\childdocname}
\DescribeMacro{\childdocjob}
The macro |\childdocname| contains the filename (without extension)
of the main or child file being processed.
Note that |\childdocjob| will always contain the name of the main file.

%%%%%%%%%%%%%%%%%%%%%%%%%%%%%%%%%%%%%%%%
\paragraph{Title Page.}

Conditional processing can be used to include a title or banner page
in the main document when proper precautions are taken.
Importantly, the code in the main file should ensure that the page counter
(as well as other status parameters which are stored in the |.aux| files)
takes the same value after the conditional processing.
Otherwise the page numbers may take divergent values
depending on which part is compiled.

For example, a title page could be declared by:
%
\begin{center}
\begin{tabular}{l}
|\ifchilddoc\||else|\\
|\addtocounter{page}{-1}|\\
\textit{code for title page}\\
|\newpage|\\
|\||fi|
\end{tabular}
\end{center}
%
A banner page for the child documents can be generated by:
%
\begin{center}
\begin{tabular}{l}
|\ifchilddoc|\\
|\addtocounter{page}{-1}|\\
\textit{code for banner page}\\
|\newpage|\\
|\||fi|
\end{tabular}
\end{center}
%
Here one could write a message such as:
\begin{center}
|This is the part \childdocname{} of \childdocjob{}.|
\end{center}

%%%%%%%%%%%%%%%%%%%%%%%%%%%%%%%%%%%%%%%%%%%%%%%%%%%%%%%%%%%%%%%%%%%%%%%%%%%%%%%%
\subsection{Flags}
\label{sec:flags}

The package makes it easy to generate different versions
of the main or child documents.
To this end compilation flags can be defined
and assigned different default values.
They will be particularly useful in conjunction
with the forwarding mechanism described in \secref{sec:forward}.

For example, it may be useful to have a flag |\version|
which can be set to |draft| or |final|.
The document source will contain some conditional code
depending on the value of |\version|.
Suppose further, the flag should default to |final| for the main file
and to |draft| for child files
which is a natural assignment for editing the document.
This is achieved by placing the following code
in the preamble of the main document
(below the |\childdocmain| directive):
%
\begin{center}
\begin{tabular}{l}
|\ifchilddoc|\\
|\providecommand{\version}{draft}|\\
|\||else|\\
|\providecommand{\version}{final}|\\
|\||fi|
\end{tabular}
\end{center}
%
The definition by |\providecommand| makes sure
that previous definitions are not overwritten.
Further statements |\providecommand{\version}{...}|
can thus be added before the above code to override it.

For the main file, one might add a line
(between |\childdocmain| and the above block)
%
\begin{center}
|%\ifchilddoc\||else\providecommand{\version}{draft}\||fi|
\end{center}
%
which can be uncommented to produce a draft version.
Likewise one can add a line to the very top of a child file
(above the |\childdocof{|\textit{main}|}| directive)
%
\begin{center}
|%\providecommand{\version}{final}|
\end{center}
%
which can be uncommented to produce the final version of this child document.

%%%%%%%%%%%%%%%%%%%%%%%%%%%%%%%%%%%%%%%%%%%%%%%%%%%%%%%%%%%%%%%%%%%%%%%%%%%%%%%%
\subsection{Forwarding}
\label{sec:forward}

Different versions of the main or child documents
using compilation flags as described in \secref{sec:flags}
can be (permanently) stored in different files
for convenient compilation, viewing and distribution.
To this end, the package defines a command
to pass on compilation to a different file:

%%%%%%%%%%%%%%%%%%%%%%%%%%%%%%%%%%%%%%%%
\DescribeMacro{\childdocforward}
The command |\childdocforward| redirects processing to
another source file:
%
\begin{center}
\begin{tabular}{l}
|\input{childdoc.def}|\\
|\childdocforward[|\textit{main}|]{|\textit{dest}|}|\\
\end{tabular}
\end{center}
%
The argument \textit{dest} is the destination file
(without extension).
It should be the main file or one of the child files.
Note that further \textsf{childdoc} directives
such as |\childdocof| and |\childdocforward|
in the indicated file will be processed in this form.
The optional argument \textit{main}
passes on directly to the main file \textit{main}
while pretending to compile the child \textit{dest}.
This form behaves as if \textit{dest}
issues |\childdocof{|\textit{main}|}| right away,
and no further \textsf{childdoc} directives will be processed.

%%%%%%%%%%%%%%%%%%%%%%%%%%%%%%%%%%%%%%%%
\DescribeMacro{\...prefix}
In the alternative form |\childdocforwardprefix|,
%
\begin{center}
\begin{tabular}{l}
|\input{childdoc.def}|\\
|\childdocforwardprefix[|\textit{main}|]{|\textit{prefix}|}{|\textit{dest}|}|
\end{tabular}
\end{center}
%
the destination file is determined by a pattern
depending on the current file:
To make this work, the current file must be called
`{\textit{prefix}\hspace{0.2em}\textit{suffix}}'
with \textit{prefix} matching precisely the argument.
Processing is then passed on to the file
`{\textit{dest}\hspace{0.2em}\textit{suffix}}'.
Surely, the same effect is achieved by
directly specifying the
argument `{\textit{dest}\hspace{0.2em}\textit{suffix}}'
in the first form.
However, that requires to set up a different file
for each child. With the alternative form of the command
all these files can have exactly the same content
which simplifies setting them up and maintaining them.

For example, the following file |draft.tex|
with a compilation flag |\version| as described in \secref{sec:flags}
compiles the main document as a draft:
%
\begin{center}
\begin{tabular}{l}
|\def\version{draft}|\\
|\input{childdoc.def}|\\
|\childdocforward{|\textit{main}|}|
\end{tabular}
\end{center}
%
Likewise, the following files |final|\textit{nn}|.tex|
compile the final version of the child document
|child|\textit{nn}|.tex|:
%
\begin{center}
\begin{tabular}{l}
|\def\version{final}|\\
|\input{childdoc.def}|\\
|\childdocforwardprefix{final}{child}|
\end{tabular}
\end{center}
%

Note that when several versions of a main file and/or of each child file
are to be generated, it may be convenient to set up a |Makefile| or
shell script to automatise the process.

%%%%%%%%%%%%%%%%%%%%%%%%%%%%%%%%%%%%%%%%%%%%%%%%%%%%%%%%%%%%%%%%%%%%%%%%%%%%%%%%
\subsection{Command Line Processing}
\label{sec:commandline}

The effect of redirection files can also be achieved by invoking
the \LaTeX{} compiler with a more elaborate command line.
Most conveniently this should be done as part
of a shell script or a |Makefile|.

When using \textsf{childdoc} in the main file, the following
command lines effectively perform a redirection
(note that depending on the shell being used,
backslashes may have to be doubled: `|\|' $\to$ `|\\|'):
%
\begin{center}
|... -jobname "|\textit{target}|" |\\|"|[\textit{flags}]%
|\input{childdoc.def}\childdocforward[|\textit{main}|]{|\textit{dest}|}"|
\end{center}
%
Here \textit{target} is the name of the output file,
\textit{main} is the name of the main file
and \textit{dest} is the name of the main or child file to be processed
(all filenames without extensions).
The optional argument \textit{main} can be omitted
if \textit{main} matches \textit{dest}.
Optionally, compilation \textit{flags} can be defined via |\def| commands.
This command line makes the \TeX{} engine believe
it is compiling the file \textit{target}
whose content is specified as the latter parameter.
The provided code then forwards the processing to
\textit{main} or \textit{dest} as described in \secref{sec:forward}.

%%%%%%%%%%%%%%%%%%%%%%%%%%%%%%%%%%%%%%%%%%%%%%%%%%%%%%%%%%%%%%%%%%%%%%%%%%%%%%%%
\subsection{Include by Input}
\label{sec:input}

Including child documents by |\include| has some restrictions by design.
Most notably, the content of a child document always occupies
its own set of pages; pages cannot be shared between child documents.
Usually, this behaviour makes perfect sense
because each child document contain an essential part of the document.
However, in some situations it may be desirable to compose
a document from a collection of parts
without having mandatory page breaks between then.
For this case, the package
provides a mechanism to include parts
by |\input| which can also be processed individually.
However, by construction this mechanism
requires manual handling of the content to be output.

%%%%%%%%%%%%%%%%%%%%%%%%%%%%%%%%%%%%%%%%
\DescribeMacro{\ifchilddocmanual}
The main file should be prepared as usual, see \secref{sec:include}.
However, the document body must make a distinction
between processing of an individual part and of the main document, e.g.:
%
\begin{center}
\begin{tabular}{l}
|\ifchilddocmanual|\\
|\input{\childdocname}|\\
|\||else|\\
\textit{document body with }|\input{|\textit{part}|}|\\
|\||fi|
\end{tabular}
\end{center}
%
The conditional |\ifchilddocmanual| is true whenever
a part to be included by |\input| is being compiled,
and the name of the part is stored in |\childdocname|.

%%%%%%%%%%%%%%%%%%%%%%%%%%%%%%%%%%%%%%%%
\DescribeMacro{\childdocby}
Each part to be included by |\input| should start with:
%
\begin{center}
\begin{tabular}{l}
|\input{childdoc.def}|\\
|\childdocby{|\textit{main}|}|\\
\end{tabular}
\end{center}
%
The directive |\childdocby| is similar to |\childdocof|
described in \secref{sec:include},
but the subsequent selection of content must be done manually.
To that end, both |\ifchilddoc| and |\ifchilddocmanual|
will be true upon processing of a part,
and the name of the part is stored in |\childdocname|.
Note that |\jobname| will be set to the filename of the current part
so that each part receives an individual |.aux| file
that does not interfere with the |.aux| file(s) of the main document.
This behaviour can be altered by the alternative form
|\childdocby[*]{|\textit{main}|}| (with a non-empty optional argument)
which uses the |.aux| file of the main document
by setting |\jobname| to \textit{main}.

%%%%%%%%%%%%%%%%%%%%%%%%%%%%%%%%%%%%%%%%%%%%%%%%%%%%%%%%%%%%%%%%%%%%%%%%%%%%%%%%
\subsection{Driver Development}
\label{sec:driver}

The \textsf{childdoc} mechanism can also be use for the development
of definition files such as \LaTeX{} styles or classes.
This case differs from the above setup with multiple parts
included by |\include| in that no |\includeonly| should be invoked.
This can be achieved by starting the include file
(before |\ProvidesPackage|) with:
%
\begin{center}
\begin{tabular}{l}
|\input{childdoc.def}|\\
|\childdocforward{|\textit{main}|}|\\
\end{tabular}
\end{center}
%
or alternatively with:
%
\begin{center}
\begin{tabular}{l}
|\input{childdoc.def}|\\
|\childdocby{|\textit{main}|}|\\
\end{tabular}
\end{center}
%
Both forms have slightly different effects as described above.
The main file is prepared as usual, see \secref{sec:include}.

%%%%%%%%%%%%%%%%%%%%%%%%%%%%%%%%%%%%%%%%%%%%%%%%%%%%%%%%%%%%%%%%%%%%%%%%%%%%%%%%
\subsection{Legacy Detection}
\label{sec:detection}

The directive |\childdocmain| in the main file can detect
whether the complete document or merely a child is to be compiled
even without using the directive |\childdocof|.
This method is deprecated because it is less robust
and there is no compelling reason to use it;
it is merely provided for backward compatibility
and it may be removed in future versions.

If the detection mechanism is to be used,
it is mandatory to correctly specify
the filename of the main file as the argument of |\childdocmain|:
%
\begin{center}
\begin{tabular}{l}
|\input{childdoc.def}|\\
|\childdocmain{|\textit{main}|}|\\
\end{tabular}
\end{center}
%
If |\jobname| does not match the argument \textit{main} of |\childdocmain|,
it is assumed that |\jobname| points to the child file to be compiled.
When using |\childdocmain| with the main file specified as argument,
it suffices to start a child file
with just |\input{|\textit{main}|}|
without loading of the package and using |\childdocof|.
If instead all processing is done
with the appropriate \textsf{childdoc} directives,
the argument of \textit{main} of |\childdocmain| can be empty.

An alternative version of the command line processing described
in \secref{sec:commandline} using the detection mechanism reads:
%
\begin{center}
|... -jobname "|\textit{target}|" "|[\textit{flags}]%
[|\def\jobname{|\textit{dest}|}|]|\input{|\textit{main}|}"|
\end{center}

%%%%%%%%%%%%%%%%%%%%%%%%%%%%%%%%%%%%%%%%%%%%%%%%%%%%%%%%%%%%%%%%%%%%%%%%%%%%%%%%
\subsection{Manual Code}
\label{sec:manual}

In case one cannot be certain whether the definitions file |childdoc.def|
is installed on the target \TeX{} distribution
and one prefers not to ship it,
it is conceivable to paste a few relevant commands into the sources.

To that end, drop all statements |\input{childdoc.def}|
and perform the replacements as outlined below.
Instead of |\childdocmain{|\textit{main}|}| add the following code
to the top of the main file:
%
\begin{center}
\begin{tabular}{l}
|\||ifdefined\childdocname\endinput\||fi\newif\ifchilddoc|\\
|\edef\childdocname{\scantokens\expandafter{\jobname\noexpand}}|\\
|\def\childdocmain{|\textit{main}|}\||ifx\childdocmain\childdocname\||else|\\
|\childdoctrue\includeonly{\childdocname}\let\jobname\childdocmain\||fi|\\
\end{tabular}
\end{center}
%
Instead of |\childdocof{|\textit{main}|}| just include the main file
at the top of each child file:
%
\begin{center}
|\input{|\textit{main}|}|
\end{center}
%
A simple redirection |\childdocforward{|\textit{dest}|}| is achieved by:
%
\begin{center}
|\def\jobname{|\textit{dest}|}\input{\jobname}|
\end{center}
%
The redirection with prefix
|\childdocforwardprefix[|\textit{prefix}|]{|\textit{dest}|}|
is accomplished by:
%
\begin{center}
\begin{tabular}{l}
|{\edef\jobname{\scantokens\expandafter{\jobname\noexpand}}|\\
|\def\redirectjob |\textit{prefix}|#1~~~{\gdef\jobname{|\textit{dest}|#1}}|\\
|\expandafter\redirectjob\jobname~~~}\input{\jobname}|
\end{tabular}
\end{center}

In an alternative approach,
child documents can be compiled by a specific command line
without additional code or specific definitions:
%
\begin{center}
|... -jobname "|\textit{target}|" "|[\textit{flags}]%
|\includeonly{|\textit{dest}|}\input{|\textit{main}|}"|
\end{center}
%

%%%%%%%%%%%%%%%%%%%%%%%%%%%%%%%%%%%%%%%%%%%%%%%%%%%%%%%%%%%%%%%%%%%%%%%%%%%%%%%%
%%%%%%%%%%%%%%%%%%%%%%%%%%%%%%%%%%%%%%%%%%%%%%%%%%%%%%%%%%%%%%%%%%%%%%%%%%%%%%%%
\section{Information}

%%%%%%%%%%%%%%%%%%%%%%%%%%%%%%%%%%%%%%%%%%%%%%%%%%%%%%%%%%%%%%%%%%%%%%%%%%%%%%%%
\subsection{Copyright}

Copyright \copyright{} 2017--2018 Niklas Beisert

This work may be distributed and/or modified under the
conditions of the \LaTeX{} Project Public License, either version 1.3
of this license or (at your option) any later version.
The latest version of this license is in
  \url{http://www.latex-project.org/lppl.txt}
and version 1.3 or later is part of all distributions of \LaTeX{}
version 2005/12/01 or later.

This work has the LPPL maintenance status `maintained'.

The Current Maintainer of this work is Niklas Beisert.

This work consists of the files |README.txt|, |childdoc.ins| and |childdoc.dtx|
as well as the derived files |childdoc.def|, |cdocsamp.tex|
with |cdocsch1.tex|, |cdocsch2.tex|, |cdocspt3.tex|, |cdocspt4.tex|,
|cdocsdrf.tex|, |cdocsfn1.tex|, |cdocsfn2.tex|
as well as |childdoc.pdf|.

%%%%%%%%%%%%%%%%%%%%%%%%%%%%%%%%%%%%%%%%%%%%%%%%%%%%%%%%%%%%%%%%%%%%%%%%%%%%%%%%
\subsection{Files and Installation}

The package consists of the files:
%
\begin{center}
\begin{tabular}{ll}
    |README.txt|   & readme file \\
    |childdoc.ins| & installation file \\
    |childdoc.dtx| & source file \\
    |childdoc.def| & definition file \\
    |cdocsamp.tex| & sample main file \\
    |cdocsch1.tex| & sample include file \\
    |cdocsch2.tex| & sample include file \\
    |cdocspt3.tex| & sample part file \\
    |cdocspt4.tex| & sample part file \\
    |cdocsdrf.tex| & sample redirection file \\
    |cdocsfn1.tex| & sample redirection file \\
    |cdocsfn2.tex| & sample redirection file \\
    |childdoc.pdf| & manual
\end{tabular}
\end{center}
%
The distribution consists of the files
|README.txt|, |childdoc.ins| and |childdoc.dtx|.
%
\begin{itemize}
\item
Run (pdf)\LaTeX{} on |childdoc.dtx|
to compile the manual |childdoc.pdf| (this file).
\item
Run \LaTeX{} on |childdoc.ins| to create the definitions file |childdoc.def|
and the sample |cdocsamp.tex| with include files
|cdocsch1.tex|, |cdocsch2.tex|, |cdocspt3.tex|, |cdocspt4.tex|,
|cdocsdrf.tex|, |cdocsfn1.tex|, |cdocsfn2.tex|.
Then copy the file |childdoc.def| to an appropriate directory of your \LaTeX{}
distribution, e.g.\ \textit{texmf-root}|/tex/latex/childdoc|.
\end{itemize}

%%%%%%%%%%%%%%%%%%%%%%%%%%%%%%%%%%%%%%%%%%%%%%%%%%%%%%%%%%%%%%%%%%%%%%%%%%%%%%%%
\subsection{Related CTAN Packages}

There are several other packages which offer a similar functionality:
%
\begin{itemize}
\item
The packages
\href{http://ctan.org/pkg/docmute}{\textsf{docmute}},
\href{http://ctan.org/pkg/includex}{\textsf{includex}} and
\href{http://ctan.org/pkg/standalone}{\textsf{standalone}}
provide commands to include only the document body of
a child file thus allowing both files to be compiled individually.
\item
The packages \href{http://ctan.org/pkg/subdocs}{\textsf{subdocs}}
and \href{http://ctan.org/pkg/subfiles}{\textsf{subfiles}}
provide structures in which the main and child documents can be
encapsulated and allowing them to be compiled individually.
The inclusion mechanism is different from the conventional |\include|.
\item
The package \href{http://ctan.org/pkg/combine}{\textsf{combine}}
is an elaborate solution to combine several documents into one.
\end{itemize}
%
See also the CTAN topic \href{http://ctan.org/topic/subdocs}{\textsf{subdocs}}
for further related packages.
The present package differs from the above solutions in that
a document structure constructed with the conventional |\include| mechanism
just needs two extra commands at the top of every file
such that all constituent files can be compiled individually.

%%%%%%%%%%%%%%%%%%%%%%%%%%%%%%%%%%%%%%%%%%%%%%%%%%%%%%%%%%%%%%%%%%%%%%%%%%%%%%%%
%\subsection{Feature Suggestions}
%
%The following is a list of features which may be useful for future
%versions of this package:
%%
%\begin{itemize}
%\item
%\ldots
%\end{itemize}

%%%%%%%%%%%%%%%%%%%%%%%%%%%%%%%%%%%%%%%%%%%%%%%%%%%%%%%%%%%%%%%%%%%%%%%%%%%%%%%%
\subsection{Revision History}

%%%%%%%%%%%%%%%%%%%%%%%%%%%%%%%%%%%%%%%%
\paragraph{v2.0:} 2018/12/30

\begin{itemize}
\item
immediate forward processing
\item
added |\childdocby| mechanism
\item
manual restructured
\end{itemize}

%%%%%%%%%%%%%%%%%%%%%%%%%%%%%%%%%%%%%%%%
\paragraph{v1.6:} 2018/01/17

\begin{itemize}
\item
application for development of include files
\item
corrections to manual
\end{itemize}

%%%%%%%%%%%%%%%%%%%%%%%%%%%%%%%%%%%%%%%%
\paragraph{v1.5:} 2017/05/21

\begin{itemize}
\item
more complete structuring introduced
\item
|\childdocof| introduced
\item
|\childdoc| renamed to |\childdocmain|
\item
|\childredirect| renamed to |\childdocforward| and |\childdocforwardprefix|
and functionality expanded
\end{itemize}

%%%%%%%%%%%%%%%%%%%%%%%%%%%%%%%%%%%%%%%%
\paragraph{v1.0:} 2017/04/27

\begin{itemize}
\item
manual and install package
\item
first version published on CTAN
\end{itemize}

%%%%%%%%%%%%%%%%%%%%%%%%%%%%%%%%%%%%%%%%
\paragraph{v0.6:} 2017/04/26

\begin{itemize}
\item
redirection mechanism added
\end{itemize}

%%%%%%%%%%%%%%%%%%%%%%%%%%%%%%%%%%%%%%%%
\paragraph{v0.5:} 2017/04/26

\begin{itemize}
\item
functionality in definition file
\end{itemize}


%%%%%%%%%%%%%%%%%%%%%%%%%%%%%%%%%%%%%%%%%%%%%%%%%%%%%%%%%%%%%%%%%%%%%%%%%%%%%%%%
%%%%%%%%%%%%%%%%%%%%%%%%%%%%%%%%%%%%%%%%%%%%%%%%%%%%%%%%%%%%%%%%%%%%%%%%%%%%%%%%
%%%%%%%%%%%%%%%%%%%%%%%%%%%%%%%%%%%%%%%%%%%%%%%%%%%%%%%%%%%%%%%%%%%%%%%%%%%%%%%%
\appendix

\settowidth\MacroIndent{\rmfamily\scriptsize 000\ }

 \DocInput{childdoc.dtx}

\end{document}
%</driver>
% \fi
%
% %%%%%%%%%%%%%%%%%%%%%%%%%%%%%%%%%%%%%%%%%%%%%%%%%%%%%%%%%%%%%%%%%%%%%%%%%%%%%%
% %%%%%%%%%%%%%%%%%%%%%%%%%%%%%%%%%%%%%%%%%%%%%%%%%%%%%%%%%%%%%%%%%%%%%%%%%%%%%%
% \section{Sample}
%\iffalse
%<*samplemain>
%\fi
%
% The following presents a sample document
% with two chapters, two parts, a title page,
% a compile flag as well as three forwarding files to set the flag.
% It consists of eight |.tex| files:
% \begin{center}
% \begin{tabular}{ll}
% |cdocsamp.tex|&main file\\
% |cdocsch1.tex|&include file for chapter 1\\
% |cdocsch2.tex|&include file for chapter 2\\
% |cdocspt3.tex|&include file for part 3\\
% |cdocspt4.tex|&include file for part 4\\
% |cdocsdrf.tex|&forwarding file for main file in draft mode\\
% |cdocsfi1.tex|&forwarding file for final version of chapter 1\\
% |cdocsfi2.tex|&forwarding file for final version of chapter 2\\
% \end{tabular}
% \end{center}
% Each of the eight files can be compiled directly by the \LaTeX{} compiler.
%
% %%%%%%%%%%%%%%%%%%%%%%%%%%%%%%%%%%%%%%
% \paragraph{Main File.}
%
% The main file is called |cdocsamp.tex|.
%
% Load the \textsf{childdoc} definitions and
% declare the filename for the main document:
%    \begin{macrocode}
\input{childdoc.def}
\childdocmain{}
%    \end{macrocode}

% Optional override for |\version| flag:
%    \begin{macrocode}
%%\ifchilddoc\else\providecommand{\version}{draft}\fi
%    \end{macrocode}

% Define the default values for the |\version| flag
% (|final| for the main file and |draft| for childs):
%    \begin{macrocode}
\ifchilddoc
\providecommand{\version}{draft}
\else
\providecommand{\version}{final}
\fi
%    \end{macrocode}

% Load the standard document class:
%    \begin{macrocode}
\documentclass[12pt]{article}
%    \end{macrocode}

% Start the document body:
%    \begin{macrocode}
\begin{document}
%    \end{macrocode}

% Declare a title page.
% Print title, part of document being processed and version flag:
%    \begin{macrocode}
\addtocounter{page}{-1}
\begin{center}
{\LARGE\bfseries{}childdoc example\par}
\vspace{1cm}
\ifchilddoc
\ifchilddocmanual part\else chapter\fi:
`\childdocname' of `\childdocjob'\par
\else
main document: `\childdocjob'\par
\fi
version: \version\par
\end{center}
\newpage
%    \end{macrocode}

% Manually include selected file,
% otherwise process as usual:
%    \begin{macrocode}
\ifchilddocmanual
\section*{part `\childdocname'}
\input{\childdocname}
\else
%    \end{macrocode}

% Include the two chapters:
%    \begin{macrocode}
\include{cdocsch1}
\include{cdocsch2}
%    \end{macrocode}

% Include the two parts unless only chapters should be displayed:
%    \begin{macrocode}
\ifchilddoc\else
\section{part three}
\input{cdocspt3}
\section{part four}
\input{cdocspt4}
\fi
%    \end{macrocode}

% Process as usual until here:
%    \begin{macrocode}
\fi
%    \end{macrocode}

% End of document body:
%    \begin{macrocode}
\end{document}
%    \end{macrocode}
%\iffalse
%</samplemain>
%\fi
%
% %%%%%%%%%%%%%%%%%%%%%%%%%%%%%%%%%%%%%%
% \paragraph{Chapter Include Files.}
%
% The include files are called |cdocsch1.tex| and |cdocsch2.tex|.
%
%\iffalse
%<*samplechap1|samplechap2>
%\fi

% Optional override for |\version| flag:
%    \begin{macrocode}
%%\providecommand{\version}{final}
%    \end{macrocode}

% Include the main document:
%    \begin{macrocode}
\input{childdoc.def}
\childdocof{cdocsamp}
%    \end{macrocode}

%\iffalse
%</samplechap1|samplechap2>
%\fi
%
%\iffalse
%<*samplechap1>
%\fi
% Some text for chapter 1:
%    \begin{macrocode}
\section{one}
some text in chapter one
%    \end{macrocode}

%\iffalse
%</samplechap1>
%\fi
% Some text for chapter 2:
%\iffalse
%<*samplechap2>
%\fi
%    \begin{macrocode}
\section{two}
more text in chapter two
%    \end{macrocode}

%\iffalse
%</samplechap2>
%\fi
%
% %%%%%%%%%%%%%%%%%%%%%%%%%%%%%%%%%%%%%%
% \paragraph{Part Include Files.}
%
% The include files are called |cdocspt3.tex| and |cdocspt4.tex|.
%
%\iffalse
%<*samplepart3|samplepart4>
%\fi

% Optional override for |\version| flag:
%    \begin{macrocode}
%%\providecommand{\version}{final}
%    \end{macrocode}

% Include the main document:
%    \begin{macrocode}
\input{childdoc.def}
\childdocby{cdocsamp}
%    \end{macrocode}

%\iffalse
%</samplepart3|samplepart4>
%\fi
%
%\iffalse
%<*samplepart3>
%\fi
% Some text for part 3:
%    \begin{macrocode}
some text in part three
%    \end{macrocode}

%\iffalse
%</samplepart3>
%\fi
% Some text for part 4:
%\iffalse
%<*samplepart4>
%\fi
%    \begin{macrocode}
more text in part four
%    \end{macrocode}

%\iffalse
%</samplepart4>
%\fi
%
% %%%%%%%%%%%%%%%%%%%%%%%%%%%%%%%%%%%%%%
% \paragraph{Forwarding for a Complete Draft.}
%
% The following forwarding file |cdocsdrf.tex|
% compiles the main document in draft mode:
%\iffalse
%<*sampledraft>
%\fi
%    \begin{macrocode}
\def\version{draft}
\input{childdoc.def}
\childdocforward{cdocsamp}
%    \end{macrocode}

%\iffalse
%</sampledraft>
%\fi
%
% %%%%%%%%%%%%%%%%%%%%%%%%%%%%%%%%%%%%%%
% \paragraph{Forwarding for Final Version of the Chapters.}
%
% The following forwarding files |cdocsfn1.tex| and |cdocsfn2.tex|
% (with identical content)
% compile the final versions of the child documents
% |cdocsch1.tex| and |cdocsch2.tex|, respectively:
%\iffalse
%<*samplefinal>
%\fi
%    \begin{macrocode}
\def\version{final}
\input{childdoc.def}
\childdocforwardprefix[cdocsamp]{cdocsfn}{cdocsch}
%    \end{macrocode}

%\iffalse
%</samplefinal>
%\fi
%
% %%%%%%%%%%%%%%%%%%%%%%%%%%%%%%%%%%%%%%
% \paragraph{Command Line Processing.}
%
% The following three command lines generate the output files
% |cdocscld|, |cdocscl1| and |cdocscl2|
% which should be identical to
% |cdocsdrf|, |cdocsch1| and |cdocsfn2|, respectively:
% \begin{center}
% \begin{tabular}{l}
% |latex -jobname cdocscld \|\\
% |  "\def\version{draft}\input{childdoc.def}\childdocforward{cdocsamp}"|\\
% |latex -jobname cdocscl1 \|\\
% |  "\input{childdoc.def}\childdocforward[cdocsamp]{cdocsch1}"|\\
% |latex -jobname cdocscl2 \|\\
% |  "\def\version{final}\input{childdoc.def}\childdocforward{cdocsch2}"|
% \end{tabular}
% \end{center}
% Note that the trailing backslash on each first line
% merely continues the input to the second line
% (for convenient cut ant paste).
% Furthermore, the command |latex| can be replaced by any
% of its alternative versions such as |pdflatex|.
%
% %%%%%%%%%%%%%%%%%%%%%%%%%%%%%%%%%%%%%%%%%%%%%%%%%%%%%%%%%%%%%%%%%%%%%%%%%%%%%%
% %%%%%%%%%%%%%%%%%%%%%%%%%%%%%%%%%%%%%%%%%%%%%%%%%%%%%%%%%%%%%%%%%%%%%%%%%%%%%%
% \section{Implementation}
%\iffalse
%<*package>
%\fi
%
% This section describes the definitions file |childdoc.def|.

% The definitions cannot be loaded using |\usepackage| or |\RequirePackage|
% which has a mechanism to prevent loading a style file more than once.
% When loading the definitions by means of |\input|
% multiple instances have to be prevented manually:
%\iffalse
%This code needs to be before the `\ProvidesFile' directive
%which is defined at the beginning of this file.
%Therefore it is also placed there and commented out here.
%</package>
%<*discard>
%\fi
%    \begin{macrocode}
\ifdefined\childdocmain\endinput\fi
%    \end{macrocode}
%\iffalse
%</discard>
%<*package>
%\fi
%
% \macro{\ifchilddoc}
% \macro{\ifchilddocmanual}
% The conditional |\ifchilddoc| tells whether a
% child (true) or main (false) document is being compiled.
% The conditional |\ifchilddocmanual| tells whether
% the |\includeonly| mechanism is used (false) or
% the selection of child files must be performed manually (true).
% The definitions initialise to false:
%    \begin{macrocode}
\newif\ifchilddoc
\newif\ifchilddocmanual
%    \end{macrocode}

% \macro{\childdocname}
% \macro{\childdocjob}
% The macro |\childdocname| stores the name of the main document
% to be compiled. The macro |\childdocjob| stores the name of
% the document on which the \LaTeX{} compiler was originally invoked.
% The content of |\jobname| cannot be compared
% to filenames specified in the source due to different catcodes.
% The following code rescans |\jobname|, stores the result
% in |\childdocname| and saves a copy in |\childdocjob|:
%    \begin{macrocode}
\edef\childdocname{\scantokens\expandafter{\jobname\noexpand}}
\let\childdocjob\childdocname
%    \end{macrocode}

% \macro{\childdocdisable}
% The macro |\childdocdisable| prevents the main file
% from being processed more than once.
% At this stage, the main document command |\childdocmain|
% is assumed to be called once again where it should do nothing.
% Any subsequent call to it should prevent
% a secondary processing of the main document
% It overwrites the forwarding commands
% |\childdocof| and |\childdocforward|
% with empty macros to prevent further inclusions of the main document:
%    \begin{macrocode}
\newcommand{\childdocdisable}
{
  \renewcommand{\childdocmain}[1]{\renewcommand{\childdocmain}[1]{\endinput}}
  \renewcommand{\childdocof}[1]{}
  \renewcommand{\childdocby}[2][]{}
  \renewcommand{\childdocforward}[2][]{}
  \renewcommand{\childdocdisable}{}
}
%    \end{macrocode}

% \macro{\childdocmain}
% The macro |\childdocmain| is to be called at the top of the main file
% with nothing or the main filename (without extension) as argument.
% First, it breaks loops.
% If the argument is not empty and does not match |\childdocname|
% (which is set by the first inclusion of |childdoc.def|),
% |\ifchilddoc| is set to true, |\includeonly| is applied to the child file
% and |\jobname| is set to the main file
% (for proper handling of |.aux| files):
%    \begin{macrocode}
\newcommand{\childdocmain}[1]
{
  \childdocdisable\childdocmain{}
  \if?#1?\else
    \begingroup
      \def\childdoctmp{#1}
      \ifx\childdoctmp\childdocname
        \def\childdoctmp{}
      \else
        \def\childdoctmp
        {
          \childdoctrue
          \includeonly{\childdocname}
          \def\childdocjob{#1}
          \def\jobname{#1}
        }
      \fi
      \expandafter
    \endgroup
    \childdoctmp
  \fi
}
%    \end{macrocode}

% \macro{\childdocof}
% The command |\childdocof| redirects
% compilation to the main file |#1|.
%    \begin{macrocode}
\newcommand{\childdocof}[1]
{
  \childdocdisable
  \childdoctrue
  \includeonly{\childdocname}
  \def\jobname{#1}
  \def\childdocjob{#1}
  \input{#1}
}
%    \end{macrocode}

% \macro{\childdocby}
% The command |\childdocby| ....
%    \begin{macrocode}
\newcommand{\childdocby}[2][]
{
  \childdocdisable
  \childdoctrue
  \childdocmanualtrue
  \if?#1?\else
    \def\jobname{#2}
  \fi
  \def\childdocjob{#2}
  \input{#2}
  \endinput
}
%    \end{macrocode}

% \macro{\childdocforward}
% The command |\childdocforward| redirects
% compilation to the main file or
% (if the optional argument is given) a child file.
% Parameters are set as if the main file
% or a child file starting with |\childdocof| was compiled.
% Then compilation is handed over to the main file:
%    \begin{macrocode}
\newcommand{\childdocforward}[2][]
{
  \begingroup
    \if?#1?
      \def\childdoctmp
      {
        \def\childdocname{#2}
        \def\childdocjob{#2}
        \def\jobname{#2}
        \input{#2}
        \endinput
      }
    \else
      \def\childdoctmp
      {
        \childdocdisable
        \def\childdocname{#2}
        \childdoctrue
        \includeonly{#2}
        \def\childdocjob{#1}
        \def\jobname{#1}
        \input{#1}
        \endinput
      }
    \fi
    \expandafter
  \endgroup
  \childdoctmp
}
%    \end{macrocode}

% \macro{\childdocforwardprefix}
% The command |\childdocforwardprefix| redirects
% compilation to the main or a child file by means of a pattern.
% The prefix |#1| in the current filename is replaced by |#2|
% and the suffix of the current filename is kept
% (it is assumed that the filename does not contain the substring `|~~~|'
% which is used as a delimiter).
% Compilation is handed over to the new file by |\childdocforward|:
%    \begin{macrocode}
\newcommand{\childdocforwardprefix}[3][]
{
  \begingroup
    \def\childdocextract #2##1~~~{\def\childdoctmp{\childdocforward[#1]{#3##1}}}
    \expandafter\childdocextract\childdocname~~~
    \expandafter
  \endgroup
  \childdoctmp
}
%    \end{macrocode}

% \macro{\childdoc}
% The deprecated macro |\childdoc| is a legacy version of |\childdocmain|:
%    \begin{macrocode}
\newcommand{\childdoc}{\childdocmain}
%    \end{macrocode}

% \macro{\childdocredirect}
% The deprecated macro |\childdocredirect| is a legacy version
% of |\childdocforward| and |\childdocforwardprefix|:
%    \begin{macrocode}
\newcommand{\childdocredirect}[2][]
{
  \begingroup
    \if?#1?
      \def\childdoctmp{\childdocforward{#2}}
    \else
      \def\childdoctmp{\childdocforwardprefix{#1}{#2}}
    \fi
    \expandafter
  \endgroup
  \childdoctmp
}
%    \end{macrocode}

%\iffalse
%</package>
%\fi
%
\endinput
\childdocforward[cdocsamp]{cdocsch1}"|\\
% |latex -jobname cdocscl2 \|\\
% |  "\def\version{final}% \iffalse
%
% childdoc.dtx Copyright (C) 2017-2018 Niklas Beisert
%
% This work may be distributed and/or modified under the
% conditions of the LaTeX Project Public License, either version 1.3
% of this license or (at your option) any later version.
% The latest version of this license is in
%   http://www.latex-project.org/lppl.txt
% and version 1.3 or later is part of all distributions of LaTeX
% version 2005/12/01 or later.
%
% This work has the LPPL maintenance status `maintained'.
%
% The Current Maintainer of this work is Niklas Beisert.
%
% This work consists of the files childdoc.dtx and childdoc.ins
% and the derived files childdoc.def and cdocsamp.tex with
% cdocsch1.tex, cdocsch2.tex, cdocsdrf.tex, cdocsfn1.tex, cdocsfn2.tex.
%
%<package>\ifdefined\childdocmain\endinput\fi
%<package>\ProvidesFile{childdoc.def}[2018/12/30 v2.0 child document driver]
%<samplemain>\ProvidesFile{cdocsamp.tex}[2018/12/30 v2.0 sample for childdoc]
%<*driver>
%\ProvidesFile{childdoc.drv}[2018/12/30 v2.0 childdoc reference manual file]
\PassOptionsToClass{10pt,a4paper}{article}
\documentclass{ltxdoc}

\usepackage[margin=35mm]{geometry}
\usepackage{hyperref}
\usepackage{hyperxmp}
\usepackage[usenames]{color}

\hypersetup{colorlinks=true}
\hypersetup{pdfstartview=FitH}
\hypersetup{pdfpagemode=UseNone}
\hypersetup{pdfsource={}}
\hypersetup{pdflang={en-UK}}
\hypersetup{pdfcopyright={Copyright 2017-2018 Niklas Beisert.
  This work may be distributed and/or modified under the
  conditions of the LaTeX Project Public License, either version 1.3
  of this license or (at your option) any later version.}}
\hypersetup{pdflicenseurl={http://www.latex-project.org/lppl.txt}}
\hypersetup{pdfcontactaddress={ETH Zurich, ITP, HIT K,
  Wolfgang-Pauli-Strasse 27}}
\hypersetup{pdfcontactpostcode={8093}}
\hypersetup{pdfcontactcity={Zurich}}
\hypersetup{pdfcontactcountry={Switzerland}}
\hypersetup{pdfcontactemail={nbeisert@itp.phys.ethz.ch}}
\hypersetup{pdfcontacturl={http://people.phys.ethz.ch/\xmptilde nbeisert/}}

\newcommand{\secref}[1]{\hyperref[#1]{section \ref*{#1}}}

\parskip1ex
\parindent0pt
\let\olditemize\itemize
\def\itemize{\olditemize\parskip0pt}

\begin{document}

\title{The \textsf{childdoc} Package}
\hypersetup{pdftitle={The childdoc Package}}
\author{Niklas Beisert\\[2ex]
  Institut f\"ur Theoretische Physik\\
  Eidgen\"ossische Technische Hochschule Z\"urich\\
  Wolfgang-Pauli-Strasse 27, 8093 Z\"urich, Switzerland\\[1ex]
  \href{mailto:nbeisert@itp.phys.ethz.ch}
  {\texttt{nbeisert@itp.phys.ethz.ch}}}
\hypersetup{pdfauthor={Niklas Beisert}}
\hypersetup{pdfsubject={Manual for the LaTeX2e Package childdoc}}
\date{30 December 2018, \textsf{v2.0}}
\maketitle

\begin{abstract}\noindent
\textsf{childdoc} is a \LaTeXe{} package
that enables the direct compilation
of document sections included by |\include|
to individual files.
\end{abstract}

\begingroup
\parskip0ex
\tableofcontents
\endgroup

%%%%%%%%%%%%%%%%%%%%%%%%%%%%%%%%%%%%%%%%%%%%%%%%%%%%%%%%%%%%%%%%%%%%%%%%%%%%%%%%
%%%%%%%%%%%%%%%%%%%%%%%%%%%%%%%%%%%%%%%%%%%%%%%%%%%%%%%%%%%%%%%%%%%%%%%%%%%%%%%%
\section{Introduction}

\LaTeX{} provides a mechanism to structure a large document (such as a book)
into a main file and several child files (containing the chapters)
using the |\include| command.
This mechanism is beneficial for documents
which span hundreds of pages in order to
make the source file(s) more manageable.
Moreover, compilation can be restricted to
selected child files by means of the |\includeonly| command.
The latter feature can be used to reduce the compilation time while editing
(this was significantly more useful in the earlier days of \LaTeX{})
or to generate a smaller document which is easier to navigate.
Another application of |\includeonly| is to generate
documents consisting of selected parts of the complete document.

However, there are a few drawbacks of the plain |\include| mechanism:
\begin{itemize}
\item
The child files cannot be compiled on their own,
they can only be compiled via the main file.
A naive editing environment
(such as a text editor with an option
to have the current file processed by \LaTeX)
may require one to switch to the main file before compiling;
attempting to compile the child file produces errors.
\item
The main file must be modified (each time)
to adjust the |\includeonly| command
to the present needs. This easily leaves the main file in a messy state.
\item
The generated document will always carry the filename
of the main document. This is inconvenient if
several child files are to be compiled and
to be kept for distribution.
\end{itemize}

The present package provides a simple interface
to make child files individually compilable by \LaTeX{}.
Compiling a child file then has the same effect as compiling
the main file with an |\includeonly| command
to select the appropriate child.
Moreover the generated document will carry the name of the child
rather than the main file.
This resolves all three above issues.

This feature is meant to make the editing of books,
thesis documents and lecture notes somewhat more convenient.
However, the package can also be used efficiently for
composing a series of documents (such as exercise sheets)
which are typically distributed individually.
It then assists the author in generating the individual documents
(potentially in different versions)
as well as a document containing the collected series.
Another application is in developing style files
or other kinds of included material
where compilation of the style file could redirect
to a sample or test file.

%%%%%%%%%%%%%%%%%%%%%%%%%%%%%%%%%%%%%%%%%%%%%%%%%%%%%%%%%%%%%%%%%%%%%%%%%%%%%%%%
%%%%%%%%%%%%%%%%%%%%%%%%%%%%%%%%%%%%%%%%%%%%%%%%%%%%%%%%%%%%%%%%%%%%%%%%%%%%%%%%
\section{Usage}

First of all, the package \textsf{childdoc} is \emph{not} a standard
\LaTeXe{} |.sty| style file! Therefore it needs to be invoked in
a non-standard way.

%%%%%%%%%%%%%%%%%%%%%%%%%%%%%%%%%%%%%%%%%%%%%%%%%%%%%%%%%%%%%%%%%%%%%%%%%%%%%%%%
\subsection{Included Files}
\label{sec:include}

%%%%%%%%%%%%%%%%%%%%%%%%%%%%%%%%%%%%%%%%
\DescribeMacro{\childdocmain}
To use the package, add the commands
\begin{center}
\begin{tabular}{l}
|\input{childdoc.def}|\\
|\childdocmain{}|\\
\end{tabular}
\end{center}
at the very top of the main \LaTeX{} file,
in particular \emph{before} the |\documentclass| statement!
The argument of |\childdocmain| should be left empty
(but it must be present).

%%%%%%%%%%%%%%%%%%%%%%%%%%%%%%%%%%%%%%%%
\DescribeMacro{\childdocof}
Furthermore, add the commands
\begin{center}
\begin{tabular}{l}
|\input{childdoc.def}|\\
|\childdocof{|\textit{main}|}|\\
\end{tabular}
\end{center}
at the top of every child file \textit{child}
which is included by |\include{|\textit{child}|}|
from within the main file
(or at least for those files to be compiled individually).
The argument \textit{main} must be the filename of the main file.

There are a couple of
considerations in setting up the main and child documents:

%%%%%%%%%%%%%%%%%%%%%%%%%%%%%%%%%%%%%%%%
\paragraph{Restrictions.}

Please note the following restrictions:
\begin{itemize}
\item
|\childdocmain| must be called with one argument \textit{main}
to ensure compatibility with earlier version of the package.
It must either be empty (|\childdocmain{}|)
or precisely match the filename of the main file in which it is specified.
See \secref{sec:detection} for further information.
\item
The filename \textit{main} must be specified without the |.tex| extension.
\item
The filename \textit{main} is case sensitive
(even in case-insensitive file systems)
due to internal string comparison.
\item
The argument \textit{main} should be fully expanded, it cannot be a macro.
\item
Subdirectories and special characters should be avoided in filenames.
\item
The command |\childdocmain{|\textit{main}|}| must be followed by a whitespace.
It should not be followed immediately by another command
or by a comment mark `|%|'.
This is because the \TeX{} parser reads the token immediately following
the argument of |\childdocmain| and puts it
at the beginning of every child section;
however, a white\-space is ignored.
\end{itemize}

%%%%%%%%%%%%%%%%%%%%%%%%%%%%%%%%%%%%%%%%
\paragraph{Content of Main File.}

It is advisable to place all content in the child files included by |\include|.
Any output contained in the main file will appear in all child documents
unless suppressed manually;
it cannot be suppressed automatically by the |\includeonly| directive
and thus should normally be avoided.
A method to include some content in the main file
by means of conditional processing is described in \secref{sec:conditional}.

%%%%%%%%%%%%%%%%%%%%%%%%%%%%%%%%%%%%%%%%
\paragraph{Page Numbering.}

When only a part of the document is compiled,
the appropriate numbering of pages
(as well as other status parameters)
is determined from the |.aux| files.
The latter contain information from previous passes.
However this information needs to propagate through
all intermediate child documents.
Therefore the page numbering in child documents may well
be inconsistent until the complete document is compiled at least once.

A useful (if unconventional) way to always ensure a consistent
page numbering is to restart the numbering in each child document
and denote the pages by `\textit{child}|.|\textit{page}'
where \textit{child} represents the chapter/section number of the child file.
This can be achieved by the command
|\numberwithin{page}{|\textit{child}|}|
of the \textsf{amsmath} package
where \textit{child} can be |chapter| or |section|
depending on the chosen structuring.
Alternatively, one can modify the macro |\thepage| appropriately
and reset the counter |page| at the start of each child file.

%%%%%%%%%%%%%%%%%%%%%%%%%%%%%%%%%%%%%%%%%%%%%%%%%%%%%%%%%%%%%%%%%%%%%%%%%%%%%%%%
\subsection{Conditional Processing}
\label{sec:conditional}

The package provides a mechanism to compile different versions
of a document. To customise the versions further some conditional processing
can come in handy to distinguish which version is being compiled.
The package provides two macros to describe the compilation context:

%%%%%%%%%%%%%%%%%%%%%%%%%%%%%%%%%%%%%%%%
\DescribeMacro{\ifchilddoc}
The conditional |\ifchilddoc| distinguishes between the compilation of
child documents and the main document:
%
\begin{center}
|\ifchilddoc |\textit{child-code}| |[|\||else |\textit{main-code}]| \||fi|
\end{center}

%%%%%%%%%%%%%%%%%%%%%%%%%%%%%%%%%%%%%%%%
\DescribeMacro{\childdocname}
\DescribeMacro{\childdocjob}
The macro |\childdocname| contains the filename (without extension)
of the main or child file being processed.
Note that |\childdocjob| will always contain the name of the main file.

%%%%%%%%%%%%%%%%%%%%%%%%%%%%%%%%%%%%%%%%
\paragraph{Title Page.}

Conditional processing can be used to include a title or banner page
in the main document when proper precautions are taken.
Importantly, the code in the main file should ensure that the page counter
(as well as other status parameters which are stored in the |.aux| files)
takes the same value after the conditional processing.
Otherwise the page numbers may take divergent values
depending on which part is compiled.

For example, a title page could be declared by:
%
\begin{center}
\begin{tabular}{l}
|\ifchilddoc\||else|\\
|\addtocounter{page}{-1}|\\
\textit{code for title page}\\
|\newpage|\\
|\||fi|
\end{tabular}
\end{center}
%
A banner page for the child documents can be generated by:
%
\begin{center}
\begin{tabular}{l}
|\ifchilddoc|\\
|\addtocounter{page}{-1}|\\
\textit{code for banner page}\\
|\newpage|\\
|\||fi|
\end{tabular}
\end{center}
%
Here one could write a message such as:
\begin{center}
|This is the part \childdocname{} of \childdocjob{}.|
\end{center}

%%%%%%%%%%%%%%%%%%%%%%%%%%%%%%%%%%%%%%%%%%%%%%%%%%%%%%%%%%%%%%%%%%%%%%%%%%%%%%%%
\subsection{Flags}
\label{sec:flags}

The package makes it easy to generate different versions
of the main or child documents.
To this end compilation flags can be defined
and assigned different default values.
They will be particularly useful in conjunction
with the forwarding mechanism described in \secref{sec:forward}.

For example, it may be useful to have a flag |\version|
which can be set to |draft| or |final|.
The document source will contain some conditional code
depending on the value of |\version|.
Suppose further, the flag should default to |final| for the main file
and to |draft| for child files
which is a natural assignment for editing the document.
This is achieved by placing the following code
in the preamble of the main document
(below the |\childdocmain| directive):
%
\begin{center}
\begin{tabular}{l}
|\ifchilddoc|\\
|\providecommand{\version}{draft}|\\
|\||else|\\
|\providecommand{\version}{final}|\\
|\||fi|
\end{tabular}
\end{center}
%
The definition by |\providecommand| makes sure
that previous definitions are not overwritten.
Further statements |\providecommand{\version}{...}|
can thus be added before the above code to override it.

For the main file, one might add a line
(between |\childdocmain| and the above block)
%
\begin{center}
|%\ifchilddoc\||else\providecommand{\version}{draft}\||fi|
\end{center}
%
which can be uncommented to produce a draft version.
Likewise one can add a line to the very top of a child file
(above the |\childdocof{|\textit{main}|}| directive)
%
\begin{center}
|%\providecommand{\version}{final}|
\end{center}
%
which can be uncommented to produce the final version of this child document.

%%%%%%%%%%%%%%%%%%%%%%%%%%%%%%%%%%%%%%%%%%%%%%%%%%%%%%%%%%%%%%%%%%%%%%%%%%%%%%%%
\subsection{Forwarding}
\label{sec:forward}

Different versions of the main or child documents
using compilation flags as described in \secref{sec:flags}
can be (permanently) stored in different files
for convenient compilation, viewing and distribution.
To this end, the package defines a command
to pass on compilation to a different file:

%%%%%%%%%%%%%%%%%%%%%%%%%%%%%%%%%%%%%%%%
\DescribeMacro{\childdocforward}
The command |\childdocforward| redirects processing to
another source file:
%
\begin{center}
\begin{tabular}{l}
|\input{childdoc.def}|\\
|\childdocforward[|\textit{main}|]{|\textit{dest}|}|\\
\end{tabular}
\end{center}
%
The argument \textit{dest} is the destination file
(without extension).
It should be the main file or one of the child files.
Note that further \textsf{childdoc} directives
such as |\childdocof| and |\childdocforward|
in the indicated file will be processed in this form.
The optional argument \textit{main}
passes on directly to the main file \textit{main}
while pretending to compile the child \textit{dest}.
This form behaves as if \textit{dest}
issues |\childdocof{|\textit{main}|}| right away,
and no further \textsf{childdoc} directives will be processed.

%%%%%%%%%%%%%%%%%%%%%%%%%%%%%%%%%%%%%%%%
\DescribeMacro{\...prefix}
In the alternative form |\childdocforwardprefix|,
%
\begin{center}
\begin{tabular}{l}
|\input{childdoc.def}|\\
|\childdocforwardprefix[|\textit{main}|]{|\textit{prefix}|}{|\textit{dest}|}|
\end{tabular}
\end{center}
%
the destination file is determined by a pattern
depending on the current file:
To make this work, the current file must be called
`{\textit{prefix}\hspace{0.2em}\textit{suffix}}'
with \textit{prefix} matching precisely the argument.
Processing is then passed on to the file
`{\textit{dest}\hspace{0.2em}\textit{suffix}}'.
Surely, the same effect is achieved by
directly specifying the
argument `{\textit{dest}\hspace{0.2em}\textit{suffix}}'
in the first form.
However, that requires to set up a different file
for each child. With the alternative form of the command
all these files can have exactly the same content
which simplifies setting them up and maintaining them.

For example, the following file |draft.tex|
with a compilation flag |\version| as described in \secref{sec:flags}
compiles the main document as a draft:
%
\begin{center}
\begin{tabular}{l}
|\def\version{draft}|\\
|\input{childdoc.def}|\\
|\childdocforward{|\textit{main}|}|
\end{tabular}
\end{center}
%
Likewise, the following files |final|\textit{nn}|.tex|
compile the final version of the child document
|child|\textit{nn}|.tex|:
%
\begin{center}
\begin{tabular}{l}
|\def\version{final}|\\
|\input{childdoc.def}|\\
|\childdocforwardprefix{final}{child}|
\end{tabular}
\end{center}
%

Note that when several versions of a main file and/or of each child file
are to be generated, it may be convenient to set up a |Makefile| or
shell script to automatise the process.

%%%%%%%%%%%%%%%%%%%%%%%%%%%%%%%%%%%%%%%%%%%%%%%%%%%%%%%%%%%%%%%%%%%%%%%%%%%%%%%%
\subsection{Command Line Processing}
\label{sec:commandline}

The effect of redirection files can also be achieved by invoking
the \LaTeX{} compiler with a more elaborate command line.
Most conveniently this should be done as part
of a shell script or a |Makefile|.

When using \textsf{childdoc} in the main file, the following
command lines effectively perform a redirection
(note that depending on the shell being used,
backslashes may have to be doubled: `|\|' $\to$ `|\\|'):
%
\begin{center}
|... -jobname "|\textit{target}|" |\\|"|[\textit{flags}]%
|\input{childdoc.def}\childdocforward[|\textit{main}|]{|\textit{dest}|}"|
\end{center}
%
Here \textit{target} is the name of the output file,
\textit{main} is the name of the main file
and \textit{dest} is the name of the main or child file to be processed
(all filenames without extensions).
The optional argument \textit{main} can be omitted
if \textit{main} matches \textit{dest}.
Optionally, compilation \textit{flags} can be defined via |\def| commands.
This command line makes the \TeX{} engine believe
it is compiling the file \textit{target}
whose content is specified as the latter parameter.
The provided code then forwards the processing to
\textit{main} or \textit{dest} as described in \secref{sec:forward}.

%%%%%%%%%%%%%%%%%%%%%%%%%%%%%%%%%%%%%%%%%%%%%%%%%%%%%%%%%%%%%%%%%%%%%%%%%%%%%%%%
\subsection{Include by Input}
\label{sec:input}

Including child documents by |\include| has some restrictions by design.
Most notably, the content of a child document always occupies
its own set of pages; pages cannot be shared between child documents.
Usually, this behaviour makes perfect sense
because each child document contain an essential part of the document.
However, in some situations it may be desirable to compose
a document from a collection of parts
without having mandatory page breaks between then.
For this case, the package
provides a mechanism to include parts
by |\input| which can also be processed individually.
However, by construction this mechanism
requires manual handling of the content to be output.

%%%%%%%%%%%%%%%%%%%%%%%%%%%%%%%%%%%%%%%%
\DescribeMacro{\ifchilddocmanual}
The main file should be prepared as usual, see \secref{sec:include}.
However, the document body must make a distinction
between processing of an individual part and of the main document, e.g.:
%
\begin{center}
\begin{tabular}{l}
|\ifchilddocmanual|\\
|\input{\childdocname}|\\
|\||else|\\
\textit{document body with }|\input{|\textit{part}|}|\\
|\||fi|
\end{tabular}
\end{center}
%
The conditional |\ifchilddocmanual| is true whenever
a part to be included by |\input| is being compiled,
and the name of the part is stored in |\childdocname|.

%%%%%%%%%%%%%%%%%%%%%%%%%%%%%%%%%%%%%%%%
\DescribeMacro{\childdocby}
Each part to be included by |\input| should start with:
%
\begin{center}
\begin{tabular}{l}
|\input{childdoc.def}|\\
|\childdocby{|\textit{main}|}|\\
\end{tabular}
\end{center}
%
The directive |\childdocby| is similar to |\childdocof|
described in \secref{sec:include},
but the subsequent selection of content must be done manually.
To that end, both |\ifchilddoc| and |\ifchilddocmanual|
will be true upon processing of a part,
and the name of the part is stored in |\childdocname|.
Note that |\jobname| will be set to the filename of the current part
so that each part receives an individual |.aux| file
that does not interfere with the |.aux| file(s) of the main document.
This behaviour can be altered by the alternative form
|\childdocby[*]{|\textit{main}|}| (with a non-empty optional argument)
which uses the |.aux| file of the main document
by setting |\jobname| to \textit{main}.

%%%%%%%%%%%%%%%%%%%%%%%%%%%%%%%%%%%%%%%%%%%%%%%%%%%%%%%%%%%%%%%%%%%%%%%%%%%%%%%%
\subsection{Driver Development}
\label{sec:driver}

The \textsf{childdoc} mechanism can also be use for the development
of definition files such as \LaTeX{} styles or classes.
This case differs from the above setup with multiple parts
included by |\include| in that no |\includeonly| should be invoked.
This can be achieved by starting the include file
(before |\ProvidesPackage|) with:
%
\begin{center}
\begin{tabular}{l}
|\input{childdoc.def}|\\
|\childdocforward{|\textit{main}|}|\\
\end{tabular}
\end{center}
%
or alternatively with:
%
\begin{center}
\begin{tabular}{l}
|\input{childdoc.def}|\\
|\childdocby{|\textit{main}|}|\\
\end{tabular}
\end{center}
%
Both forms have slightly different effects as described above.
The main file is prepared as usual, see \secref{sec:include}.

%%%%%%%%%%%%%%%%%%%%%%%%%%%%%%%%%%%%%%%%%%%%%%%%%%%%%%%%%%%%%%%%%%%%%%%%%%%%%%%%
\subsection{Legacy Detection}
\label{sec:detection}

The directive |\childdocmain| in the main file can detect
whether the complete document or merely a child is to be compiled
even without using the directive |\childdocof|.
This method is deprecated because it is less robust
and there is no compelling reason to use it;
it is merely provided for backward compatibility
and it may be removed in future versions.

If the detection mechanism is to be used,
it is mandatory to correctly specify
the filename of the main file as the argument of |\childdocmain|:
%
\begin{center}
\begin{tabular}{l}
|\input{childdoc.def}|\\
|\childdocmain{|\textit{main}|}|\\
\end{tabular}
\end{center}
%
If |\jobname| does not match the argument \textit{main} of |\childdocmain|,
it is assumed that |\jobname| points to the child file to be compiled.
When using |\childdocmain| with the main file specified as argument,
it suffices to start a child file
with just |\input{|\textit{main}|}|
without loading of the package and using |\childdocof|.
If instead all processing is done
with the appropriate \textsf{childdoc} directives,
the argument of \textit{main} of |\childdocmain| can be empty.

An alternative version of the command line processing described
in \secref{sec:commandline} using the detection mechanism reads:
%
\begin{center}
|... -jobname "|\textit{target}|" "|[\textit{flags}]%
[|\def\jobname{|\textit{dest}|}|]|\input{|\textit{main}|}"|
\end{center}

%%%%%%%%%%%%%%%%%%%%%%%%%%%%%%%%%%%%%%%%%%%%%%%%%%%%%%%%%%%%%%%%%%%%%%%%%%%%%%%%
\subsection{Manual Code}
\label{sec:manual}

In case one cannot be certain whether the definitions file |childdoc.def|
is installed on the target \TeX{} distribution
and one prefers not to ship it,
it is conceivable to paste a few relevant commands into the sources.

To that end, drop all statements |\input{childdoc.def}|
and perform the replacements as outlined below.
Instead of |\childdocmain{|\textit{main}|}| add the following code
to the top of the main file:
%
\begin{center}
\begin{tabular}{l}
|\||ifdefined\childdocname\endinput\||fi\newif\ifchilddoc|\\
|\edef\childdocname{\scantokens\expandafter{\jobname\noexpand}}|\\
|\def\childdocmain{|\textit{main}|}\||ifx\childdocmain\childdocname\||else|\\
|\childdoctrue\includeonly{\childdocname}\let\jobname\childdocmain\||fi|\\
\end{tabular}
\end{center}
%
Instead of |\childdocof{|\textit{main}|}| just include the main file
at the top of each child file:
%
\begin{center}
|\input{|\textit{main}|}|
\end{center}
%
A simple redirection |\childdocforward{|\textit{dest}|}| is achieved by:
%
\begin{center}
|\def\jobname{|\textit{dest}|}\input{\jobname}|
\end{center}
%
The redirection with prefix
|\childdocforwardprefix[|\textit{prefix}|]{|\textit{dest}|}|
is accomplished by:
%
\begin{center}
\begin{tabular}{l}
|{\edef\jobname{\scantokens\expandafter{\jobname\noexpand}}|\\
|\def\redirectjob |\textit{prefix}|#1~~~{\gdef\jobname{|\textit{dest}|#1}}|\\
|\expandafter\redirectjob\jobname~~~}\input{\jobname}|
\end{tabular}
\end{center}

In an alternative approach,
child documents can be compiled by a specific command line
without additional code or specific definitions:
%
\begin{center}
|... -jobname "|\textit{target}|" "|[\textit{flags}]%
|\includeonly{|\textit{dest}|}\input{|\textit{main}|}"|
\end{center}
%

%%%%%%%%%%%%%%%%%%%%%%%%%%%%%%%%%%%%%%%%%%%%%%%%%%%%%%%%%%%%%%%%%%%%%%%%%%%%%%%%
%%%%%%%%%%%%%%%%%%%%%%%%%%%%%%%%%%%%%%%%%%%%%%%%%%%%%%%%%%%%%%%%%%%%%%%%%%%%%%%%
\section{Information}

%%%%%%%%%%%%%%%%%%%%%%%%%%%%%%%%%%%%%%%%%%%%%%%%%%%%%%%%%%%%%%%%%%%%%%%%%%%%%%%%
\subsection{Copyright}

Copyright \copyright{} 2017--2018 Niklas Beisert

This work may be distributed and/or modified under the
conditions of the \LaTeX{} Project Public License, either version 1.3
of this license or (at your option) any later version.
The latest version of this license is in
  \url{http://www.latex-project.org/lppl.txt}
and version 1.3 or later is part of all distributions of \LaTeX{}
version 2005/12/01 or later.

This work has the LPPL maintenance status `maintained'.

The Current Maintainer of this work is Niklas Beisert.

This work consists of the files |README.txt|, |childdoc.ins| and |childdoc.dtx|
as well as the derived files |childdoc.def|, |cdocsamp.tex|
with |cdocsch1.tex|, |cdocsch2.tex|, |cdocspt3.tex|, |cdocspt4.tex|,
|cdocsdrf.tex|, |cdocsfn1.tex|, |cdocsfn2.tex|
as well as |childdoc.pdf|.

%%%%%%%%%%%%%%%%%%%%%%%%%%%%%%%%%%%%%%%%%%%%%%%%%%%%%%%%%%%%%%%%%%%%%%%%%%%%%%%%
\subsection{Files and Installation}

The package consists of the files:
%
\begin{center}
\begin{tabular}{ll}
    |README.txt|   & readme file \\
    |childdoc.ins| & installation file \\
    |childdoc.dtx| & source file \\
    |childdoc.def| & definition file \\
    |cdocsamp.tex| & sample main file \\
    |cdocsch1.tex| & sample include file \\
    |cdocsch2.tex| & sample include file \\
    |cdocspt3.tex| & sample part file \\
    |cdocspt4.tex| & sample part file \\
    |cdocsdrf.tex| & sample redirection file \\
    |cdocsfn1.tex| & sample redirection file \\
    |cdocsfn2.tex| & sample redirection file \\
    |childdoc.pdf| & manual
\end{tabular}
\end{center}
%
The distribution consists of the files
|README.txt|, |childdoc.ins| and |childdoc.dtx|.
%
\begin{itemize}
\item
Run (pdf)\LaTeX{} on |childdoc.dtx|
to compile the manual |childdoc.pdf| (this file).
\item
Run \LaTeX{} on |childdoc.ins| to create the definitions file |childdoc.def|
and the sample |cdocsamp.tex| with include files
|cdocsch1.tex|, |cdocsch2.tex|, |cdocspt3.tex|, |cdocspt4.tex|,
|cdocsdrf.tex|, |cdocsfn1.tex|, |cdocsfn2.tex|.
Then copy the file |childdoc.def| to an appropriate directory of your \LaTeX{}
distribution, e.g.\ \textit{texmf-root}|/tex/latex/childdoc|.
\end{itemize}

%%%%%%%%%%%%%%%%%%%%%%%%%%%%%%%%%%%%%%%%%%%%%%%%%%%%%%%%%%%%%%%%%%%%%%%%%%%%%%%%
\subsection{Related CTAN Packages}

There are several other packages which offer a similar functionality:
%
\begin{itemize}
\item
The packages
\href{http://ctan.org/pkg/docmute}{\textsf{docmute}},
\href{http://ctan.org/pkg/includex}{\textsf{includex}} and
\href{http://ctan.org/pkg/standalone}{\textsf{standalone}}
provide commands to include only the document body of
a child file thus allowing both files to be compiled individually.
\item
The packages \href{http://ctan.org/pkg/subdocs}{\textsf{subdocs}}
and \href{http://ctan.org/pkg/subfiles}{\textsf{subfiles}}
provide structures in which the main and child documents can be
encapsulated and allowing them to be compiled individually.
The inclusion mechanism is different from the conventional |\include|.
\item
The package \href{http://ctan.org/pkg/combine}{\textsf{combine}}
is an elaborate solution to combine several documents into one.
\end{itemize}
%
See also the CTAN topic \href{http://ctan.org/topic/subdocs}{\textsf{subdocs}}
for further related packages.
The present package differs from the above solutions in that
a document structure constructed with the conventional |\include| mechanism
just needs two extra commands at the top of every file
such that all constituent files can be compiled individually.

%%%%%%%%%%%%%%%%%%%%%%%%%%%%%%%%%%%%%%%%%%%%%%%%%%%%%%%%%%%%%%%%%%%%%%%%%%%%%%%%
%\subsection{Feature Suggestions}
%
%The following is a list of features which may be useful for future
%versions of this package:
%%
%\begin{itemize}
%\item
%\ldots
%\end{itemize}

%%%%%%%%%%%%%%%%%%%%%%%%%%%%%%%%%%%%%%%%%%%%%%%%%%%%%%%%%%%%%%%%%%%%%%%%%%%%%%%%
\subsection{Revision History}

%%%%%%%%%%%%%%%%%%%%%%%%%%%%%%%%%%%%%%%%
\paragraph{v2.0:} 2018/12/30

\begin{itemize}
\item
immediate forward processing
\item
added |\childdocby| mechanism
\item
manual restructured
\end{itemize}

%%%%%%%%%%%%%%%%%%%%%%%%%%%%%%%%%%%%%%%%
\paragraph{v1.6:} 2018/01/17

\begin{itemize}
\item
application for development of include files
\item
corrections to manual
\end{itemize}

%%%%%%%%%%%%%%%%%%%%%%%%%%%%%%%%%%%%%%%%
\paragraph{v1.5:} 2017/05/21

\begin{itemize}
\item
more complete structuring introduced
\item
|\childdocof| introduced
\item
|\childdoc| renamed to |\childdocmain|
\item
|\childredirect| renamed to |\childdocforward| and |\childdocforwardprefix|
and functionality expanded
\end{itemize}

%%%%%%%%%%%%%%%%%%%%%%%%%%%%%%%%%%%%%%%%
\paragraph{v1.0:} 2017/04/27

\begin{itemize}
\item
manual and install package
\item
first version published on CTAN
\end{itemize}

%%%%%%%%%%%%%%%%%%%%%%%%%%%%%%%%%%%%%%%%
\paragraph{v0.6:} 2017/04/26

\begin{itemize}
\item
redirection mechanism added
\end{itemize}

%%%%%%%%%%%%%%%%%%%%%%%%%%%%%%%%%%%%%%%%
\paragraph{v0.5:} 2017/04/26

\begin{itemize}
\item
functionality in definition file
\end{itemize}


%%%%%%%%%%%%%%%%%%%%%%%%%%%%%%%%%%%%%%%%%%%%%%%%%%%%%%%%%%%%%%%%%%%%%%%%%%%%%%%%
%%%%%%%%%%%%%%%%%%%%%%%%%%%%%%%%%%%%%%%%%%%%%%%%%%%%%%%%%%%%%%%%%%%%%%%%%%%%%%%%
%%%%%%%%%%%%%%%%%%%%%%%%%%%%%%%%%%%%%%%%%%%%%%%%%%%%%%%%%%%%%%%%%%%%%%%%%%%%%%%%
\appendix

\settowidth\MacroIndent{\rmfamily\scriptsize 000\ }

 \DocInput{childdoc.dtx}

\end{document}
%</driver>
% \fi
%
% %%%%%%%%%%%%%%%%%%%%%%%%%%%%%%%%%%%%%%%%%%%%%%%%%%%%%%%%%%%%%%%%%%%%%%%%%%%%%%
% %%%%%%%%%%%%%%%%%%%%%%%%%%%%%%%%%%%%%%%%%%%%%%%%%%%%%%%%%%%%%%%%%%%%%%%%%%%%%%
% \section{Sample}
%\iffalse
%<*samplemain>
%\fi
%
% The following presents a sample document
% with two chapters, two parts, a title page,
% a compile flag as well as three forwarding files to set the flag.
% It consists of eight |.tex| files:
% \begin{center}
% \begin{tabular}{ll}
% |cdocsamp.tex|&main file\\
% |cdocsch1.tex|&include file for chapter 1\\
% |cdocsch2.tex|&include file for chapter 2\\
% |cdocspt3.tex|&include file for part 3\\
% |cdocspt4.tex|&include file for part 4\\
% |cdocsdrf.tex|&forwarding file for main file in draft mode\\
% |cdocsfi1.tex|&forwarding file for final version of chapter 1\\
% |cdocsfi2.tex|&forwarding file for final version of chapter 2\\
% \end{tabular}
% \end{center}
% Each of the eight files can be compiled directly by the \LaTeX{} compiler.
%
% %%%%%%%%%%%%%%%%%%%%%%%%%%%%%%%%%%%%%%
% \paragraph{Main File.}
%
% The main file is called |cdocsamp.tex|.
%
% Load the \textsf{childdoc} definitions and
% declare the filename for the main document:
%    \begin{macrocode}
\input{childdoc.def}
\childdocmain{}
%    \end{macrocode}

% Optional override for |\version| flag:
%    \begin{macrocode}
%%\ifchilddoc\else\providecommand{\version}{draft}\fi
%    \end{macrocode}

% Define the default values for the |\version| flag
% (|final| for the main file and |draft| for childs):
%    \begin{macrocode}
\ifchilddoc
\providecommand{\version}{draft}
\else
\providecommand{\version}{final}
\fi
%    \end{macrocode}

% Load the standard document class:
%    \begin{macrocode}
\documentclass[12pt]{article}
%    \end{macrocode}

% Start the document body:
%    \begin{macrocode}
\begin{document}
%    \end{macrocode}

% Declare a title page.
% Print title, part of document being processed and version flag:
%    \begin{macrocode}
\addtocounter{page}{-1}
\begin{center}
{\LARGE\bfseries{}childdoc example\par}
\vspace{1cm}
\ifchilddoc
\ifchilddocmanual part\else chapter\fi:
`\childdocname' of `\childdocjob'\par
\else
main document: `\childdocjob'\par
\fi
version: \version\par
\end{center}
\newpage
%    \end{macrocode}

% Manually include selected file,
% otherwise process as usual:
%    \begin{macrocode}
\ifchilddocmanual
\section*{part `\childdocname'}
\input{\childdocname}
\else
%    \end{macrocode}

% Include the two chapters:
%    \begin{macrocode}
\include{cdocsch1}
\include{cdocsch2}
%    \end{macrocode}

% Include the two parts unless only chapters should be displayed:
%    \begin{macrocode}
\ifchilddoc\else
\section{part three}
\input{cdocspt3}
\section{part four}
\input{cdocspt4}
\fi
%    \end{macrocode}

% Process as usual until here:
%    \begin{macrocode}
\fi
%    \end{macrocode}

% End of document body:
%    \begin{macrocode}
\end{document}
%    \end{macrocode}
%\iffalse
%</samplemain>
%\fi
%
% %%%%%%%%%%%%%%%%%%%%%%%%%%%%%%%%%%%%%%
% \paragraph{Chapter Include Files.}
%
% The include files are called |cdocsch1.tex| and |cdocsch2.tex|.
%
%\iffalse
%<*samplechap1|samplechap2>
%\fi

% Optional override for |\version| flag:
%    \begin{macrocode}
%%\providecommand{\version}{final}
%    \end{macrocode}

% Include the main document:
%    \begin{macrocode}
\input{childdoc.def}
\childdocof{cdocsamp}
%    \end{macrocode}

%\iffalse
%</samplechap1|samplechap2>
%\fi
%
%\iffalse
%<*samplechap1>
%\fi
% Some text for chapter 1:
%    \begin{macrocode}
\section{one}
some text in chapter one
%    \end{macrocode}

%\iffalse
%</samplechap1>
%\fi
% Some text for chapter 2:
%\iffalse
%<*samplechap2>
%\fi
%    \begin{macrocode}
\section{two}
more text in chapter two
%    \end{macrocode}

%\iffalse
%</samplechap2>
%\fi
%
% %%%%%%%%%%%%%%%%%%%%%%%%%%%%%%%%%%%%%%
% \paragraph{Part Include Files.}
%
% The include files are called |cdocspt3.tex| and |cdocspt4.tex|.
%
%\iffalse
%<*samplepart3|samplepart4>
%\fi

% Optional override for |\version| flag:
%    \begin{macrocode}
%%\providecommand{\version}{final}
%    \end{macrocode}

% Include the main document:
%    \begin{macrocode}
\input{childdoc.def}
\childdocby{cdocsamp}
%    \end{macrocode}

%\iffalse
%</samplepart3|samplepart4>
%\fi
%
%\iffalse
%<*samplepart3>
%\fi
% Some text for part 3:
%    \begin{macrocode}
some text in part three
%    \end{macrocode}

%\iffalse
%</samplepart3>
%\fi
% Some text for part 4:
%\iffalse
%<*samplepart4>
%\fi
%    \begin{macrocode}
more text in part four
%    \end{macrocode}

%\iffalse
%</samplepart4>
%\fi
%
% %%%%%%%%%%%%%%%%%%%%%%%%%%%%%%%%%%%%%%
% \paragraph{Forwarding for a Complete Draft.}
%
% The following forwarding file |cdocsdrf.tex|
% compiles the main document in draft mode:
%\iffalse
%<*sampledraft>
%\fi
%    \begin{macrocode}
\def\version{draft}
\input{childdoc.def}
\childdocforward{cdocsamp}
%    \end{macrocode}

%\iffalse
%</sampledraft>
%\fi
%
% %%%%%%%%%%%%%%%%%%%%%%%%%%%%%%%%%%%%%%
% \paragraph{Forwarding for Final Version of the Chapters.}
%
% The following forwarding files |cdocsfn1.tex| and |cdocsfn2.tex|
% (with identical content)
% compile the final versions of the child documents
% |cdocsch1.tex| and |cdocsch2.tex|, respectively:
%\iffalse
%<*samplefinal>
%\fi
%    \begin{macrocode}
\def\version{final}
\input{childdoc.def}
\childdocforwardprefix[cdocsamp]{cdocsfn}{cdocsch}
%    \end{macrocode}

%\iffalse
%</samplefinal>
%\fi
%
% %%%%%%%%%%%%%%%%%%%%%%%%%%%%%%%%%%%%%%
% \paragraph{Command Line Processing.}
%
% The following three command lines generate the output files
% |cdocscld|, |cdocscl1| and |cdocscl2|
% which should be identical to
% |cdocsdrf|, |cdocsch1| and |cdocsfn2|, respectively:
% \begin{center}
% \begin{tabular}{l}
% |latex -jobname cdocscld \|\\
% |  "\def\version{draft}\input{childdoc.def}\childdocforward{cdocsamp}"|\\
% |latex -jobname cdocscl1 \|\\
% |  "\input{childdoc.def}\childdocforward[cdocsamp]{cdocsch1}"|\\
% |latex -jobname cdocscl2 \|\\
% |  "\def\version{final}\input{childdoc.def}\childdocforward{cdocsch2}"|
% \end{tabular}
% \end{center}
% Note that the trailing backslash on each first line
% merely continues the input to the second line
% (for convenient cut ant paste).
% Furthermore, the command |latex| can be replaced by any
% of its alternative versions such as |pdflatex|.
%
% %%%%%%%%%%%%%%%%%%%%%%%%%%%%%%%%%%%%%%%%%%%%%%%%%%%%%%%%%%%%%%%%%%%%%%%%%%%%%%
% %%%%%%%%%%%%%%%%%%%%%%%%%%%%%%%%%%%%%%%%%%%%%%%%%%%%%%%%%%%%%%%%%%%%%%%%%%%%%%
% \section{Implementation}
%\iffalse
%<*package>
%\fi
%
% This section describes the definitions file |childdoc.def|.

% The definitions cannot be loaded using |\usepackage| or |\RequirePackage|
% which has a mechanism to prevent loading a style file more than once.
% When loading the definitions by means of |\input|
% multiple instances have to be prevented manually:
%\iffalse
%This code needs to be before the `\ProvidesFile' directive
%which is defined at the beginning of this file.
%Therefore it is also placed there and commented out here.
%</package>
%<*discard>
%\fi
%    \begin{macrocode}
\ifdefined\childdocmain\endinput\fi
%    \end{macrocode}
%\iffalse
%</discard>
%<*package>
%\fi
%
% \macro{\ifchilddoc}
% \macro{\ifchilddocmanual}
% The conditional |\ifchilddoc| tells whether a
% child (true) or main (false) document is being compiled.
% The conditional |\ifchilddocmanual| tells whether
% the |\includeonly| mechanism is used (false) or
% the selection of child files must be performed manually (true).
% The definitions initialise to false:
%    \begin{macrocode}
\newif\ifchilddoc
\newif\ifchilddocmanual
%    \end{macrocode}

% \macro{\childdocname}
% \macro{\childdocjob}
% The macro |\childdocname| stores the name of the main document
% to be compiled. The macro |\childdocjob| stores the name of
% the document on which the \LaTeX{} compiler was originally invoked.
% The content of |\jobname| cannot be compared
% to filenames specified in the source due to different catcodes.
% The following code rescans |\jobname|, stores the result
% in |\childdocname| and saves a copy in |\childdocjob|:
%    \begin{macrocode}
\edef\childdocname{\scantokens\expandafter{\jobname\noexpand}}
\let\childdocjob\childdocname
%    \end{macrocode}

% \macro{\childdocdisable}
% The macro |\childdocdisable| prevents the main file
% from being processed more than once.
% At this stage, the main document command |\childdocmain|
% is assumed to be called once again where it should do nothing.
% Any subsequent call to it should prevent
% a secondary processing of the main document
% It overwrites the forwarding commands
% |\childdocof| and |\childdocforward|
% with empty macros to prevent further inclusions of the main document:
%    \begin{macrocode}
\newcommand{\childdocdisable}
{
  \renewcommand{\childdocmain}[1]{\renewcommand{\childdocmain}[1]{\endinput}}
  \renewcommand{\childdocof}[1]{}
  \renewcommand{\childdocby}[2][]{}
  \renewcommand{\childdocforward}[2][]{}
  \renewcommand{\childdocdisable}{}
}
%    \end{macrocode}

% \macro{\childdocmain}
% The macro |\childdocmain| is to be called at the top of the main file
% with nothing or the main filename (without extension) as argument.
% First, it breaks loops.
% If the argument is not empty and does not match |\childdocname|
% (which is set by the first inclusion of |childdoc.def|),
% |\ifchilddoc| is set to true, |\includeonly| is applied to the child file
% and |\jobname| is set to the main file
% (for proper handling of |.aux| files):
%    \begin{macrocode}
\newcommand{\childdocmain}[1]
{
  \childdocdisable\childdocmain{}
  \if?#1?\else
    \begingroup
      \def\childdoctmp{#1}
      \ifx\childdoctmp\childdocname
        \def\childdoctmp{}
      \else
        \def\childdoctmp
        {
          \childdoctrue
          \includeonly{\childdocname}
          \def\childdocjob{#1}
          \def\jobname{#1}
        }
      \fi
      \expandafter
    \endgroup
    \childdoctmp
  \fi
}
%    \end{macrocode}

% \macro{\childdocof}
% The command |\childdocof| redirects
% compilation to the main file |#1|.
%    \begin{macrocode}
\newcommand{\childdocof}[1]
{
  \childdocdisable
  \childdoctrue
  \includeonly{\childdocname}
  \def\jobname{#1}
  \def\childdocjob{#1}
  \input{#1}
}
%    \end{macrocode}

% \macro{\childdocby}
% The command |\childdocby| ....
%    \begin{macrocode}
\newcommand{\childdocby}[2][]
{
  \childdocdisable
  \childdoctrue
  \childdocmanualtrue
  \if?#1?\else
    \def\jobname{#2}
  \fi
  \def\childdocjob{#2}
  \input{#2}
  \endinput
}
%    \end{macrocode}

% \macro{\childdocforward}
% The command |\childdocforward| redirects
% compilation to the main file or
% (if the optional argument is given) a child file.
% Parameters are set as if the main file
% or a child file starting with |\childdocof| was compiled.
% Then compilation is handed over to the main file:
%    \begin{macrocode}
\newcommand{\childdocforward}[2][]
{
  \begingroup
    \if?#1?
      \def\childdoctmp
      {
        \def\childdocname{#2}
        \def\childdocjob{#2}
        \def\jobname{#2}
        \input{#2}
        \endinput
      }
    \else
      \def\childdoctmp
      {
        \childdocdisable
        \def\childdocname{#2}
        \childdoctrue
        \includeonly{#2}
        \def\childdocjob{#1}
        \def\jobname{#1}
        \input{#1}
        \endinput
      }
    \fi
    \expandafter
  \endgroup
  \childdoctmp
}
%    \end{macrocode}

% \macro{\childdocforwardprefix}
% The command |\childdocforwardprefix| redirects
% compilation to the main or a child file by means of a pattern.
% The prefix |#1| in the current filename is replaced by |#2|
% and the suffix of the current filename is kept
% (it is assumed that the filename does not contain the substring `|~~~|'
% which is used as a delimiter).
% Compilation is handed over to the new file by |\childdocforward|:
%    \begin{macrocode}
\newcommand{\childdocforwardprefix}[3][]
{
  \begingroup
    \def\childdocextract #2##1~~~{\def\childdoctmp{\childdocforward[#1]{#3##1}}}
    \expandafter\childdocextract\childdocname~~~
    \expandafter
  \endgroup
  \childdoctmp
}
%    \end{macrocode}

% \macro{\childdoc}
% The deprecated macro |\childdoc| is a legacy version of |\childdocmain|:
%    \begin{macrocode}
\newcommand{\childdoc}{\childdocmain}
%    \end{macrocode}

% \macro{\childdocredirect}
% The deprecated macro |\childdocredirect| is a legacy version
% of |\childdocforward| and |\childdocforwardprefix|:
%    \begin{macrocode}
\newcommand{\childdocredirect}[2][]
{
  \begingroup
    \if?#1?
      \def\childdoctmp{\childdocforward{#2}}
    \else
      \def\childdoctmp{\childdocforwardprefix{#1}{#2}}
    \fi
    \expandafter
  \endgroup
  \childdoctmp
}
%    \end{macrocode}

%\iffalse
%</package>
%\fi
%
\endinput
\childdocforward{cdocsch2}"|
% \end{tabular}
% \end{center}
% Note that the trailing backslash on each first line
% merely continues the input to the second line
% (for convenient cut ant paste).
% Furthermore, the command |latex| can be replaced by any
% of its alternative versions such as |pdflatex|.
%
% %%%%%%%%%%%%%%%%%%%%%%%%%%%%%%%%%%%%%%%%%%%%%%%%%%%%%%%%%%%%%%%%%%%%%%%%%%%%%%
% %%%%%%%%%%%%%%%%%%%%%%%%%%%%%%%%%%%%%%%%%%%%%%%%%%%%%%%%%%%%%%%%%%%%%%%%%%%%%%
% \section{Implementation}
%\iffalse
%<*package>
%\fi
%
% This section describes the definitions file |childdoc.def|.

% The definitions cannot be loaded using |\usepackage| or |\RequirePackage|
% which has a mechanism to prevent loading a style file more than once.
% When loading the definitions by means of |\input|
% multiple instances have to be prevented manually:
%\iffalse
%This code needs to be before the `\ProvidesFile' directive
%which is defined at the beginning of this file.
%Therefore it is also placed there and commented out here.
%</package>
%<*discard>
%\fi
%    \begin{macrocode}
\ifdefined\childdocmain\endinput\fi
%    \end{macrocode}
%\iffalse
%</discard>
%<*package>
%\fi
%
% \macro{\ifchilddoc}
% \macro{\ifchilddocmanual}
% The conditional |\ifchilddoc| tells whether a
% child (true) or main (false) document is being compiled.
% The conditional |\ifchilddocmanual| tells whether
% the |\includeonly| mechanism is used (false) or
% the selection of child files must be performed manually (true).
% The definitions initialise to false:
%    \begin{macrocode}
\newif\ifchilddoc
\newif\ifchilddocmanual
%    \end{macrocode}

% \macro{\childdocname}
% \macro{\childdocjob}
% The macro |\childdocname| stores the name of the main document
% to be compiled. The macro |\childdocjob| stores the name of
% the document on which the \LaTeX{} compiler was originally invoked.
% The content of |\jobname| cannot be compared
% to filenames specified in the source due to different catcodes.
% The following code rescans |\jobname|, stores the result
% in |\childdocname| and saves a copy in |\childdocjob|:
%    \begin{macrocode}
\edef\childdocname{\scantokens\expandafter{\jobname\noexpand}}
\let\childdocjob\childdocname
%    \end{macrocode}

% \macro{\childdocdisable}
% The macro |\childdocdisable| prevents the main file
% from being processed more than once.
% At this stage, the main document command |\childdocmain|
% is assumed to be called once again where it should do nothing.
% Any subsequent call to it should prevent
% a secondary processing of the main document
% It overwrites the forwarding commands
% |\childdocof| and |\childdocforward|
% with empty macros to prevent further inclusions of the main document:
%    \begin{macrocode}
\newcommand{\childdocdisable}
{
  \renewcommand{\childdocmain}[1]{\renewcommand{\childdocmain}[1]{\endinput}}
  \renewcommand{\childdocof}[1]{}
  \renewcommand{\childdocby}[2][]{}
  \renewcommand{\childdocforward}[2][]{}
  \renewcommand{\childdocdisable}{}
}
%    \end{macrocode}

% \macro{\childdocmain}
% The macro |\childdocmain| is to be called at the top of the main file
% with nothing or the main filename (without extension) as argument.
% First, it breaks loops.
% If the argument is not empty and does not match |\childdocname|
% (which is set by the first inclusion of |childdoc.def|),
% |\ifchilddoc| is set to true, |\includeonly| is applied to the child file
% and |\jobname| is set to the main file
% (for proper handling of |.aux| files):
%    \begin{macrocode}
\newcommand{\childdocmain}[1]
{
  \childdocdisable\childdocmain{}
  \if?#1?\else
    \begingroup
      \def\childdoctmp{#1}
      \ifx\childdoctmp\childdocname
        \def\childdoctmp{}
      \else
        \def\childdoctmp
        {
          \childdoctrue
          \includeonly{\childdocname}
          \def\childdocjob{#1}
          \def\jobname{#1}
        }
      \fi
      \expandafter
    \endgroup
    \childdoctmp
  \fi
}
%    \end{macrocode}

% \macro{\childdocof}
% The command |\childdocof| redirects
% compilation to the main file |#1|.
%    \begin{macrocode}
\newcommand{\childdocof}[1]
{
  \childdocdisable
  \childdoctrue
  \includeonly{\childdocname}
  \def\jobname{#1}
  \def\childdocjob{#1}
  \input{#1}
}
%    \end{macrocode}

% \macro{\childdocby}
% The command |\childdocby| ....
%    \begin{macrocode}
\newcommand{\childdocby}[2][]
{
  \childdocdisable
  \childdoctrue
  \childdocmanualtrue
  \if?#1?\else
    \def\jobname{#2}
  \fi
  \def\childdocjob{#2}
  \input{#2}
  \endinput
}
%    \end{macrocode}

% \macro{\childdocforward}
% The command |\childdocforward| redirects
% compilation to the main file or
% (if the optional argument is given) a child file.
% Parameters are set as if the main file
% or a child file starting with |\childdocof| was compiled.
% Then compilation is handed over to the main file:
%    \begin{macrocode}
\newcommand{\childdocforward}[2][]
{
  \begingroup
    \if?#1?
      \def\childdoctmp
      {
        \def\childdocname{#2}
        \def\childdocjob{#2}
        \def\jobname{#2}
        \input{#2}
        \endinput
      }
    \else
      \def\childdoctmp
      {
        \childdocdisable
        \def\childdocname{#2}
        \childdoctrue
        \includeonly{#2}
        \def\childdocjob{#1}
        \def\jobname{#1}
        \input{#1}
        \endinput
      }
    \fi
    \expandafter
  \endgroup
  \childdoctmp
}
%    \end{macrocode}

% \macro{\childdocforwardprefix}
% The command |\childdocforwardprefix| redirects
% compilation to the main or a child file by means of a pattern.
% The prefix |#1| in the current filename is replaced by |#2|
% and the suffix of the current filename is kept
% (it is assumed that the filename does not contain the substring `|~~~|'
% which is used as a delimiter).
% Compilation is handed over to the new file by |\childdocforward|:
%    \begin{macrocode}
\newcommand{\childdocforwardprefix}[3][]
{
  \begingroup
    \def\childdocextract #2##1~~~{\def\childdoctmp{\childdocforward[#1]{#3##1}}}
    \expandafter\childdocextract\childdocname~~~
    \expandafter
  \endgroup
  \childdoctmp
}
%    \end{macrocode}

% \macro{\childdoc}
% The deprecated macro |\childdoc| is a legacy version of |\childdocmain|:
%    \begin{macrocode}
\newcommand{\childdoc}{\childdocmain}
%    \end{macrocode}

% \macro{\childdocredirect}
% The deprecated macro |\childdocredirect| is a legacy version
% of |\childdocforward| and |\childdocforwardprefix|:
%    \begin{macrocode}
\newcommand{\childdocredirect}[2][]
{
  \begingroup
    \if?#1?
      \def\childdoctmp{\childdocforward{#2}}
    \else
      \def\childdoctmp{\childdocforwardprefix{#1}{#2}}
    \fi
    \expandafter
  \endgroup
  \childdoctmp
}
%    \end{macrocode}

%\iffalse
%</package>
%\fi
%
\endinput

\childdocforward{cdocsamp}
%    \end{macrocode}

%\iffalse
%</sampledraft>
%\fi
%
% %%%%%%%%%%%%%%%%%%%%%%%%%%%%%%%%%%%%%%
% \paragraph{Forwarding for Final Version of the Chapters.}
%
% The following forwarding files |cdocsfn1.tex| and |cdocsfn2.tex|
% (with identical content)
% compile the final versions of the child documents
% |cdocsch1.tex| and |cdocsch2.tex|, respectively:
%\iffalse
%<*samplefinal>
%\fi
%    \begin{macrocode}
\def\version{final}
% \iffalse
%
% childdoc.dtx Copyright (C) 2017-2018 Niklas Beisert
%
% This work may be distributed and/or modified under the
% conditions of the LaTeX Project Public License, either version 1.3
% of this license or (at your option) any later version.
% The latest version of this license is in
%   http://www.latex-project.org/lppl.txt
% and version 1.3 or later is part of all distributions of LaTeX
% version 2005/12/01 or later.
%
% This work has the LPPL maintenance status `maintained'.
%
% The Current Maintainer of this work is Niklas Beisert.
%
% This work consists of the files childdoc.dtx and childdoc.ins
% and the derived files childdoc.def and cdocsamp.tex with
% cdocsch1.tex, cdocsch2.tex, cdocsdrf.tex, cdocsfn1.tex, cdocsfn2.tex.
%
%<package>\ifdefined\childdocmain\endinput\fi
%<package>\ProvidesFile{childdoc.def}[2018/12/30 v2.0 child document driver]
%<samplemain>\ProvidesFile{cdocsamp.tex}[2018/12/30 v2.0 sample for childdoc]
%<*driver>
%\ProvidesFile{childdoc.drv}[2018/12/30 v2.0 childdoc reference manual file]
\PassOptionsToClass{10pt,a4paper}{article}
\documentclass{ltxdoc}

\usepackage[margin=35mm]{geometry}
\usepackage{hyperref}
\usepackage{hyperxmp}
\usepackage[usenames]{color}

\hypersetup{colorlinks=true}
\hypersetup{pdfstartview=FitH}
\hypersetup{pdfpagemode=UseNone}
\hypersetup{pdfsource={}}
\hypersetup{pdflang={en-UK}}
\hypersetup{pdfcopyright={Copyright 2017-2018 Niklas Beisert.
  This work may be distributed and/or modified under the
  conditions of the LaTeX Project Public License, either version 1.3
  of this license or (at your option) any later version.}}
\hypersetup{pdflicenseurl={http://www.latex-project.org/lppl.txt}}
\hypersetup{pdfcontactaddress={ETH Zurich, ITP, HIT K,
  Wolfgang-Pauli-Strasse 27}}
\hypersetup{pdfcontactpostcode={8093}}
\hypersetup{pdfcontactcity={Zurich}}
\hypersetup{pdfcontactcountry={Switzerland}}
\hypersetup{pdfcontactemail={nbeisert@itp.phys.ethz.ch}}
\hypersetup{pdfcontacturl={http://people.phys.ethz.ch/\xmptilde nbeisert/}}

\newcommand{\secref}[1]{\hyperref[#1]{section \ref*{#1}}}

\parskip1ex
\parindent0pt
\let\olditemize\itemize
\def\itemize{\olditemize\parskip0pt}

\begin{document}

\title{The \textsf{childdoc} Package}
\hypersetup{pdftitle={The childdoc Package}}
\author{Niklas Beisert\\[2ex]
  Institut f\"ur Theoretische Physik\\
  Eidgen\"ossische Technische Hochschule Z\"urich\\
  Wolfgang-Pauli-Strasse 27, 8093 Z\"urich, Switzerland\\[1ex]
  \href{mailto:nbeisert@itp.phys.ethz.ch}
  {\texttt{nbeisert@itp.phys.ethz.ch}}}
\hypersetup{pdfauthor={Niklas Beisert}}
\hypersetup{pdfsubject={Manual for the LaTeX2e Package childdoc}}
\date{30 December 2018, \textsf{v2.0}}
\maketitle

\begin{abstract}\noindent
\textsf{childdoc} is a \LaTeXe{} package
that enables the direct compilation
of document sections included by |\include|
to individual files.
\end{abstract}

\begingroup
\parskip0ex
\tableofcontents
\endgroup

%%%%%%%%%%%%%%%%%%%%%%%%%%%%%%%%%%%%%%%%%%%%%%%%%%%%%%%%%%%%%%%%%%%%%%%%%%%%%%%%
%%%%%%%%%%%%%%%%%%%%%%%%%%%%%%%%%%%%%%%%%%%%%%%%%%%%%%%%%%%%%%%%%%%%%%%%%%%%%%%%
\section{Introduction}

\LaTeX{} provides a mechanism to structure a large document (such as a book)
into a main file and several child files (containing the chapters)
using the |\include| command.
This mechanism is beneficial for documents
which span hundreds of pages in order to
make the source file(s) more manageable.
Moreover, compilation can be restricted to
selected child files by means of the |\includeonly| command.
The latter feature can be used to reduce the compilation time while editing
(this was significantly more useful in the earlier days of \LaTeX{})
or to generate a smaller document which is easier to navigate.
Another application of |\includeonly| is to generate
documents consisting of selected parts of the complete document.

However, there are a few drawbacks of the plain |\include| mechanism:
\begin{itemize}
\item
The child files cannot be compiled on their own,
they can only be compiled via the main file.
A naive editing environment
(such as a text editor with an option
to have the current file processed by \LaTeX)
may require one to switch to the main file before compiling;
attempting to compile the child file produces errors.
\item
The main file must be modified (each time)
to adjust the |\includeonly| command
to the present needs. This easily leaves the main file in a messy state.
\item
The generated document will always carry the filename
of the main document. This is inconvenient if
several child files are to be compiled and
to be kept for distribution.
\end{itemize}

The present package provides a simple interface
to make child files individually compilable by \LaTeX{}.
Compiling a child file then has the same effect as compiling
the main file with an |\includeonly| command
to select the appropriate child.
Moreover the generated document will carry the name of the child
rather than the main file.
This resolves all three above issues.

This feature is meant to make the editing of books,
thesis documents and lecture notes somewhat more convenient.
However, the package can also be used efficiently for
composing a series of documents (such as exercise sheets)
which are typically distributed individually.
It then assists the author in generating the individual documents
(potentially in different versions)
as well as a document containing the collected series.
Another application is in developing style files
or other kinds of included material
where compilation of the style file could redirect
to a sample or test file.

%%%%%%%%%%%%%%%%%%%%%%%%%%%%%%%%%%%%%%%%%%%%%%%%%%%%%%%%%%%%%%%%%%%%%%%%%%%%%%%%
%%%%%%%%%%%%%%%%%%%%%%%%%%%%%%%%%%%%%%%%%%%%%%%%%%%%%%%%%%%%%%%%%%%%%%%%%%%%%%%%
\section{Usage}

First of all, the package \textsf{childdoc} is \emph{not} a standard
\LaTeXe{} |.sty| style file! Therefore it needs to be invoked in
a non-standard way.

%%%%%%%%%%%%%%%%%%%%%%%%%%%%%%%%%%%%%%%%%%%%%%%%%%%%%%%%%%%%%%%%%%%%%%%%%%%%%%%%
\subsection{Included Files}
\label{sec:include}

%%%%%%%%%%%%%%%%%%%%%%%%%%%%%%%%%%%%%%%%
\DescribeMacro{\childdocmain}
To use the package, add the commands
\begin{center}
\begin{tabular}{l}
|% \iffalse
%
% childdoc.dtx Copyright (C) 2017-2018 Niklas Beisert
%
% This work may be distributed and/or modified under the
% conditions of the LaTeX Project Public License, either version 1.3
% of this license or (at your option) any later version.
% The latest version of this license is in
%   http://www.latex-project.org/lppl.txt
% and version 1.3 or later is part of all distributions of LaTeX
% version 2005/12/01 or later.
%
% This work has the LPPL maintenance status `maintained'.
%
% The Current Maintainer of this work is Niklas Beisert.
%
% This work consists of the files childdoc.dtx and childdoc.ins
% and the derived files childdoc.def and cdocsamp.tex with
% cdocsch1.tex, cdocsch2.tex, cdocsdrf.tex, cdocsfn1.tex, cdocsfn2.tex.
%
%<package>\ifdefined\childdocmain\endinput\fi
%<package>\ProvidesFile{childdoc.def}[2018/12/30 v2.0 child document driver]
%<samplemain>\ProvidesFile{cdocsamp.tex}[2018/12/30 v2.0 sample for childdoc]
%<*driver>
%\ProvidesFile{childdoc.drv}[2018/12/30 v2.0 childdoc reference manual file]
\PassOptionsToClass{10pt,a4paper}{article}
\documentclass{ltxdoc}

\usepackage[margin=35mm]{geometry}
\usepackage{hyperref}
\usepackage{hyperxmp}
\usepackage[usenames]{color}

\hypersetup{colorlinks=true}
\hypersetup{pdfstartview=FitH}
\hypersetup{pdfpagemode=UseNone}
\hypersetup{pdfsource={}}
\hypersetup{pdflang={en-UK}}
\hypersetup{pdfcopyright={Copyright 2017-2018 Niklas Beisert.
  This work may be distributed and/or modified under the
  conditions of the LaTeX Project Public License, either version 1.3
  of this license or (at your option) any later version.}}
\hypersetup{pdflicenseurl={http://www.latex-project.org/lppl.txt}}
\hypersetup{pdfcontactaddress={ETH Zurich, ITP, HIT K,
  Wolfgang-Pauli-Strasse 27}}
\hypersetup{pdfcontactpostcode={8093}}
\hypersetup{pdfcontactcity={Zurich}}
\hypersetup{pdfcontactcountry={Switzerland}}
\hypersetup{pdfcontactemail={nbeisert@itp.phys.ethz.ch}}
\hypersetup{pdfcontacturl={http://people.phys.ethz.ch/\xmptilde nbeisert/}}

\newcommand{\secref}[1]{\hyperref[#1]{section \ref*{#1}}}

\parskip1ex
\parindent0pt
\let\olditemize\itemize
\def\itemize{\olditemize\parskip0pt}

\begin{document}

\title{The \textsf{childdoc} Package}
\hypersetup{pdftitle={The childdoc Package}}
\author{Niklas Beisert\\[2ex]
  Institut f\"ur Theoretische Physik\\
  Eidgen\"ossische Technische Hochschule Z\"urich\\
  Wolfgang-Pauli-Strasse 27, 8093 Z\"urich, Switzerland\\[1ex]
  \href{mailto:nbeisert@itp.phys.ethz.ch}
  {\texttt{nbeisert@itp.phys.ethz.ch}}}
\hypersetup{pdfauthor={Niklas Beisert}}
\hypersetup{pdfsubject={Manual for the LaTeX2e Package childdoc}}
\date{30 December 2018, \textsf{v2.0}}
\maketitle

\begin{abstract}\noindent
\textsf{childdoc} is a \LaTeXe{} package
that enables the direct compilation
of document sections included by |\include|
to individual files.
\end{abstract}

\begingroup
\parskip0ex
\tableofcontents
\endgroup

%%%%%%%%%%%%%%%%%%%%%%%%%%%%%%%%%%%%%%%%%%%%%%%%%%%%%%%%%%%%%%%%%%%%%%%%%%%%%%%%
%%%%%%%%%%%%%%%%%%%%%%%%%%%%%%%%%%%%%%%%%%%%%%%%%%%%%%%%%%%%%%%%%%%%%%%%%%%%%%%%
\section{Introduction}

\LaTeX{} provides a mechanism to structure a large document (such as a book)
into a main file and several child files (containing the chapters)
using the |\include| command.
This mechanism is beneficial for documents
which span hundreds of pages in order to
make the source file(s) more manageable.
Moreover, compilation can be restricted to
selected child files by means of the |\includeonly| command.
The latter feature can be used to reduce the compilation time while editing
(this was significantly more useful in the earlier days of \LaTeX{})
or to generate a smaller document which is easier to navigate.
Another application of |\includeonly| is to generate
documents consisting of selected parts of the complete document.

However, there are a few drawbacks of the plain |\include| mechanism:
\begin{itemize}
\item
The child files cannot be compiled on their own,
they can only be compiled via the main file.
A naive editing environment
(such as a text editor with an option
to have the current file processed by \LaTeX)
may require one to switch to the main file before compiling;
attempting to compile the child file produces errors.
\item
The main file must be modified (each time)
to adjust the |\includeonly| command
to the present needs. This easily leaves the main file in a messy state.
\item
The generated document will always carry the filename
of the main document. This is inconvenient if
several child files are to be compiled and
to be kept for distribution.
\end{itemize}

The present package provides a simple interface
to make child files individually compilable by \LaTeX{}.
Compiling a child file then has the same effect as compiling
the main file with an |\includeonly| command
to select the appropriate child.
Moreover the generated document will carry the name of the child
rather than the main file.
This resolves all three above issues.

This feature is meant to make the editing of books,
thesis documents and lecture notes somewhat more convenient.
However, the package can also be used efficiently for
composing a series of documents (such as exercise sheets)
which are typically distributed individually.
It then assists the author in generating the individual documents
(potentially in different versions)
as well as a document containing the collected series.
Another application is in developing style files
or other kinds of included material
where compilation of the style file could redirect
to a sample or test file.

%%%%%%%%%%%%%%%%%%%%%%%%%%%%%%%%%%%%%%%%%%%%%%%%%%%%%%%%%%%%%%%%%%%%%%%%%%%%%%%%
%%%%%%%%%%%%%%%%%%%%%%%%%%%%%%%%%%%%%%%%%%%%%%%%%%%%%%%%%%%%%%%%%%%%%%%%%%%%%%%%
\section{Usage}

First of all, the package \textsf{childdoc} is \emph{not} a standard
\LaTeXe{} |.sty| style file! Therefore it needs to be invoked in
a non-standard way.

%%%%%%%%%%%%%%%%%%%%%%%%%%%%%%%%%%%%%%%%%%%%%%%%%%%%%%%%%%%%%%%%%%%%%%%%%%%%%%%%
\subsection{Included Files}
\label{sec:include}

%%%%%%%%%%%%%%%%%%%%%%%%%%%%%%%%%%%%%%%%
\DescribeMacro{\childdocmain}
To use the package, add the commands
\begin{center}
\begin{tabular}{l}
|\input{childdoc.def}|\\
|\childdocmain{}|\\
\end{tabular}
\end{center}
at the very top of the main \LaTeX{} file,
in particular \emph{before} the |\documentclass| statement!
The argument of |\childdocmain| should be left empty
(but it must be present).

%%%%%%%%%%%%%%%%%%%%%%%%%%%%%%%%%%%%%%%%
\DescribeMacro{\childdocof}
Furthermore, add the commands
\begin{center}
\begin{tabular}{l}
|\input{childdoc.def}|\\
|\childdocof{|\textit{main}|}|\\
\end{tabular}
\end{center}
at the top of every child file \textit{child}
which is included by |\include{|\textit{child}|}|
from within the main file
(or at least for those files to be compiled individually).
The argument \textit{main} must be the filename of the main file.

There are a couple of
considerations in setting up the main and child documents:

%%%%%%%%%%%%%%%%%%%%%%%%%%%%%%%%%%%%%%%%
\paragraph{Restrictions.}

Please note the following restrictions:
\begin{itemize}
\item
|\childdocmain| must be called with one argument \textit{main}
to ensure compatibility with earlier version of the package.
It must either be empty (|\childdocmain{}|)
or precisely match the filename of the main file in which it is specified.
See \secref{sec:detection} for further information.
\item
The filename \textit{main} must be specified without the |.tex| extension.
\item
The filename \textit{main} is case sensitive
(even in case-insensitive file systems)
due to internal string comparison.
\item
The argument \textit{main} should be fully expanded, it cannot be a macro.
\item
Subdirectories and special characters should be avoided in filenames.
\item
The command |\childdocmain{|\textit{main}|}| must be followed by a whitespace.
It should not be followed immediately by another command
or by a comment mark `|%|'.
This is because the \TeX{} parser reads the token immediately following
the argument of |\childdocmain| and puts it
at the beginning of every child section;
however, a white\-space is ignored.
\end{itemize}

%%%%%%%%%%%%%%%%%%%%%%%%%%%%%%%%%%%%%%%%
\paragraph{Content of Main File.}

It is advisable to place all content in the child files included by |\include|.
Any output contained in the main file will appear in all child documents
unless suppressed manually;
it cannot be suppressed automatically by the |\includeonly| directive
and thus should normally be avoided.
A method to include some content in the main file
by means of conditional processing is described in \secref{sec:conditional}.

%%%%%%%%%%%%%%%%%%%%%%%%%%%%%%%%%%%%%%%%
\paragraph{Page Numbering.}

When only a part of the document is compiled,
the appropriate numbering of pages
(as well as other status parameters)
is determined from the |.aux| files.
The latter contain information from previous passes.
However this information needs to propagate through
all intermediate child documents.
Therefore the page numbering in child documents may well
be inconsistent until the complete document is compiled at least once.

A useful (if unconventional) way to always ensure a consistent
page numbering is to restart the numbering in each child document
and denote the pages by `\textit{child}|.|\textit{page}'
where \textit{child} represents the chapter/section number of the child file.
This can be achieved by the command
|\numberwithin{page}{|\textit{child}|}|
of the \textsf{amsmath} package
where \textit{child} can be |chapter| or |section|
depending on the chosen structuring.
Alternatively, one can modify the macro |\thepage| appropriately
and reset the counter |page| at the start of each child file.

%%%%%%%%%%%%%%%%%%%%%%%%%%%%%%%%%%%%%%%%%%%%%%%%%%%%%%%%%%%%%%%%%%%%%%%%%%%%%%%%
\subsection{Conditional Processing}
\label{sec:conditional}

The package provides a mechanism to compile different versions
of a document. To customise the versions further some conditional processing
can come in handy to distinguish which version is being compiled.
The package provides two macros to describe the compilation context:

%%%%%%%%%%%%%%%%%%%%%%%%%%%%%%%%%%%%%%%%
\DescribeMacro{\ifchilddoc}
The conditional |\ifchilddoc| distinguishes between the compilation of
child documents and the main document:
%
\begin{center}
|\ifchilddoc |\textit{child-code}| |[|\||else |\textit{main-code}]| \||fi|
\end{center}

%%%%%%%%%%%%%%%%%%%%%%%%%%%%%%%%%%%%%%%%
\DescribeMacro{\childdocname}
\DescribeMacro{\childdocjob}
The macro |\childdocname| contains the filename (without extension)
of the main or child file being processed.
Note that |\childdocjob| will always contain the name of the main file.

%%%%%%%%%%%%%%%%%%%%%%%%%%%%%%%%%%%%%%%%
\paragraph{Title Page.}

Conditional processing can be used to include a title or banner page
in the main document when proper precautions are taken.
Importantly, the code in the main file should ensure that the page counter
(as well as other status parameters which are stored in the |.aux| files)
takes the same value after the conditional processing.
Otherwise the page numbers may take divergent values
depending on which part is compiled.

For example, a title page could be declared by:
%
\begin{center}
\begin{tabular}{l}
|\ifchilddoc\||else|\\
|\addtocounter{page}{-1}|\\
\textit{code for title page}\\
|\newpage|\\
|\||fi|
\end{tabular}
\end{center}
%
A banner page for the child documents can be generated by:
%
\begin{center}
\begin{tabular}{l}
|\ifchilddoc|\\
|\addtocounter{page}{-1}|\\
\textit{code for banner page}\\
|\newpage|\\
|\||fi|
\end{tabular}
\end{center}
%
Here one could write a message such as:
\begin{center}
|This is the part \childdocname{} of \childdocjob{}.|
\end{center}

%%%%%%%%%%%%%%%%%%%%%%%%%%%%%%%%%%%%%%%%%%%%%%%%%%%%%%%%%%%%%%%%%%%%%%%%%%%%%%%%
\subsection{Flags}
\label{sec:flags}

The package makes it easy to generate different versions
of the main or child documents.
To this end compilation flags can be defined
and assigned different default values.
They will be particularly useful in conjunction
with the forwarding mechanism described in \secref{sec:forward}.

For example, it may be useful to have a flag |\version|
which can be set to |draft| or |final|.
The document source will contain some conditional code
depending on the value of |\version|.
Suppose further, the flag should default to |final| for the main file
and to |draft| for child files
which is a natural assignment for editing the document.
This is achieved by placing the following code
in the preamble of the main document
(below the |\childdocmain| directive):
%
\begin{center}
\begin{tabular}{l}
|\ifchilddoc|\\
|\providecommand{\version}{draft}|\\
|\||else|\\
|\providecommand{\version}{final}|\\
|\||fi|
\end{tabular}
\end{center}
%
The definition by |\providecommand| makes sure
that previous definitions are not overwritten.
Further statements |\providecommand{\version}{...}|
can thus be added before the above code to override it.

For the main file, one might add a line
(between |\childdocmain| and the above block)
%
\begin{center}
|%\ifchilddoc\||else\providecommand{\version}{draft}\||fi|
\end{center}
%
which can be uncommented to produce a draft version.
Likewise one can add a line to the very top of a child file
(above the |\childdocof{|\textit{main}|}| directive)
%
\begin{center}
|%\providecommand{\version}{final}|
\end{center}
%
which can be uncommented to produce the final version of this child document.

%%%%%%%%%%%%%%%%%%%%%%%%%%%%%%%%%%%%%%%%%%%%%%%%%%%%%%%%%%%%%%%%%%%%%%%%%%%%%%%%
\subsection{Forwarding}
\label{sec:forward}

Different versions of the main or child documents
using compilation flags as described in \secref{sec:flags}
can be (permanently) stored in different files
for convenient compilation, viewing and distribution.
To this end, the package defines a command
to pass on compilation to a different file:

%%%%%%%%%%%%%%%%%%%%%%%%%%%%%%%%%%%%%%%%
\DescribeMacro{\childdocforward}
The command |\childdocforward| redirects processing to
another source file:
%
\begin{center}
\begin{tabular}{l}
|\input{childdoc.def}|\\
|\childdocforward[|\textit{main}|]{|\textit{dest}|}|\\
\end{tabular}
\end{center}
%
The argument \textit{dest} is the destination file
(without extension).
It should be the main file or one of the child files.
Note that further \textsf{childdoc} directives
such as |\childdocof| and |\childdocforward|
in the indicated file will be processed in this form.
The optional argument \textit{main}
passes on directly to the main file \textit{main}
while pretending to compile the child \textit{dest}.
This form behaves as if \textit{dest}
issues |\childdocof{|\textit{main}|}| right away,
and no further \textsf{childdoc} directives will be processed.

%%%%%%%%%%%%%%%%%%%%%%%%%%%%%%%%%%%%%%%%
\DescribeMacro{\...prefix}
In the alternative form |\childdocforwardprefix|,
%
\begin{center}
\begin{tabular}{l}
|\input{childdoc.def}|\\
|\childdocforwardprefix[|\textit{main}|]{|\textit{prefix}|}{|\textit{dest}|}|
\end{tabular}
\end{center}
%
the destination file is determined by a pattern
depending on the current file:
To make this work, the current file must be called
`{\textit{prefix}\hspace{0.2em}\textit{suffix}}'
with \textit{prefix} matching precisely the argument.
Processing is then passed on to the file
`{\textit{dest}\hspace{0.2em}\textit{suffix}}'.
Surely, the same effect is achieved by
directly specifying the
argument `{\textit{dest}\hspace{0.2em}\textit{suffix}}'
in the first form.
However, that requires to set up a different file
for each child. With the alternative form of the command
all these files can have exactly the same content
which simplifies setting them up and maintaining them.

For example, the following file |draft.tex|
with a compilation flag |\version| as described in \secref{sec:flags}
compiles the main document as a draft:
%
\begin{center}
\begin{tabular}{l}
|\def\version{draft}|\\
|\input{childdoc.def}|\\
|\childdocforward{|\textit{main}|}|
\end{tabular}
\end{center}
%
Likewise, the following files |final|\textit{nn}|.tex|
compile the final version of the child document
|child|\textit{nn}|.tex|:
%
\begin{center}
\begin{tabular}{l}
|\def\version{final}|\\
|\input{childdoc.def}|\\
|\childdocforwardprefix{final}{child}|
\end{tabular}
\end{center}
%

Note that when several versions of a main file and/or of each child file
are to be generated, it may be convenient to set up a |Makefile| or
shell script to automatise the process.

%%%%%%%%%%%%%%%%%%%%%%%%%%%%%%%%%%%%%%%%%%%%%%%%%%%%%%%%%%%%%%%%%%%%%%%%%%%%%%%%
\subsection{Command Line Processing}
\label{sec:commandline}

The effect of redirection files can also be achieved by invoking
the \LaTeX{} compiler with a more elaborate command line.
Most conveniently this should be done as part
of a shell script or a |Makefile|.

When using \textsf{childdoc} in the main file, the following
command lines effectively perform a redirection
(note that depending on the shell being used,
backslashes may have to be doubled: `|\|' $\to$ `|\\|'):
%
\begin{center}
|... -jobname "|\textit{target}|" |\\|"|[\textit{flags}]%
|\input{childdoc.def}\childdocforward[|\textit{main}|]{|\textit{dest}|}"|
\end{center}
%
Here \textit{target} is the name of the output file,
\textit{main} is the name of the main file
and \textit{dest} is the name of the main or child file to be processed
(all filenames without extensions).
The optional argument \textit{main} can be omitted
if \textit{main} matches \textit{dest}.
Optionally, compilation \textit{flags} can be defined via |\def| commands.
This command line makes the \TeX{} engine believe
it is compiling the file \textit{target}
whose content is specified as the latter parameter.
The provided code then forwards the processing to
\textit{main} or \textit{dest} as described in \secref{sec:forward}.

%%%%%%%%%%%%%%%%%%%%%%%%%%%%%%%%%%%%%%%%%%%%%%%%%%%%%%%%%%%%%%%%%%%%%%%%%%%%%%%%
\subsection{Include by Input}
\label{sec:input}

Including child documents by |\include| has some restrictions by design.
Most notably, the content of a child document always occupies
its own set of pages; pages cannot be shared between child documents.
Usually, this behaviour makes perfect sense
because each child document contain an essential part of the document.
However, in some situations it may be desirable to compose
a document from a collection of parts
without having mandatory page breaks between then.
For this case, the package
provides a mechanism to include parts
by |\input| which can also be processed individually.
However, by construction this mechanism
requires manual handling of the content to be output.

%%%%%%%%%%%%%%%%%%%%%%%%%%%%%%%%%%%%%%%%
\DescribeMacro{\ifchilddocmanual}
The main file should be prepared as usual, see \secref{sec:include}.
However, the document body must make a distinction
between processing of an individual part and of the main document, e.g.:
%
\begin{center}
\begin{tabular}{l}
|\ifchilddocmanual|\\
|\input{\childdocname}|\\
|\||else|\\
\textit{document body with }|\input{|\textit{part}|}|\\
|\||fi|
\end{tabular}
\end{center}
%
The conditional |\ifchilddocmanual| is true whenever
a part to be included by |\input| is being compiled,
and the name of the part is stored in |\childdocname|.

%%%%%%%%%%%%%%%%%%%%%%%%%%%%%%%%%%%%%%%%
\DescribeMacro{\childdocby}
Each part to be included by |\input| should start with:
%
\begin{center}
\begin{tabular}{l}
|\input{childdoc.def}|\\
|\childdocby{|\textit{main}|}|\\
\end{tabular}
\end{center}
%
The directive |\childdocby| is similar to |\childdocof|
described in \secref{sec:include},
but the subsequent selection of content must be done manually.
To that end, both |\ifchilddoc| and |\ifchilddocmanual|
will be true upon processing of a part,
and the name of the part is stored in |\childdocname|.
Note that |\jobname| will be set to the filename of the current part
so that each part receives an individual |.aux| file
that does not interfere with the |.aux| file(s) of the main document.
This behaviour can be altered by the alternative form
|\childdocby[*]{|\textit{main}|}| (with a non-empty optional argument)
which uses the |.aux| file of the main document
by setting |\jobname| to \textit{main}.

%%%%%%%%%%%%%%%%%%%%%%%%%%%%%%%%%%%%%%%%%%%%%%%%%%%%%%%%%%%%%%%%%%%%%%%%%%%%%%%%
\subsection{Driver Development}
\label{sec:driver}

The \textsf{childdoc} mechanism can also be use for the development
of definition files such as \LaTeX{} styles or classes.
This case differs from the above setup with multiple parts
included by |\include| in that no |\includeonly| should be invoked.
This can be achieved by starting the include file
(before |\ProvidesPackage|) with:
%
\begin{center}
\begin{tabular}{l}
|\input{childdoc.def}|\\
|\childdocforward{|\textit{main}|}|\\
\end{tabular}
\end{center}
%
or alternatively with:
%
\begin{center}
\begin{tabular}{l}
|\input{childdoc.def}|\\
|\childdocby{|\textit{main}|}|\\
\end{tabular}
\end{center}
%
Both forms have slightly different effects as described above.
The main file is prepared as usual, see \secref{sec:include}.

%%%%%%%%%%%%%%%%%%%%%%%%%%%%%%%%%%%%%%%%%%%%%%%%%%%%%%%%%%%%%%%%%%%%%%%%%%%%%%%%
\subsection{Legacy Detection}
\label{sec:detection}

The directive |\childdocmain| in the main file can detect
whether the complete document or merely a child is to be compiled
even without using the directive |\childdocof|.
This method is deprecated because it is less robust
and there is no compelling reason to use it;
it is merely provided for backward compatibility
and it may be removed in future versions.

If the detection mechanism is to be used,
it is mandatory to correctly specify
the filename of the main file as the argument of |\childdocmain|:
%
\begin{center}
\begin{tabular}{l}
|\input{childdoc.def}|\\
|\childdocmain{|\textit{main}|}|\\
\end{tabular}
\end{center}
%
If |\jobname| does not match the argument \textit{main} of |\childdocmain|,
it is assumed that |\jobname| points to the child file to be compiled.
When using |\childdocmain| with the main file specified as argument,
it suffices to start a child file
with just |\input{|\textit{main}|}|
without loading of the package and using |\childdocof|.
If instead all processing is done
with the appropriate \textsf{childdoc} directives,
the argument of \textit{main} of |\childdocmain| can be empty.

An alternative version of the command line processing described
in \secref{sec:commandline} using the detection mechanism reads:
%
\begin{center}
|... -jobname "|\textit{target}|" "|[\textit{flags}]%
[|\def\jobname{|\textit{dest}|}|]|\input{|\textit{main}|}"|
\end{center}

%%%%%%%%%%%%%%%%%%%%%%%%%%%%%%%%%%%%%%%%%%%%%%%%%%%%%%%%%%%%%%%%%%%%%%%%%%%%%%%%
\subsection{Manual Code}
\label{sec:manual}

In case one cannot be certain whether the definitions file |childdoc.def|
is installed on the target \TeX{} distribution
and one prefers not to ship it,
it is conceivable to paste a few relevant commands into the sources.

To that end, drop all statements |\input{childdoc.def}|
and perform the replacements as outlined below.
Instead of |\childdocmain{|\textit{main}|}| add the following code
to the top of the main file:
%
\begin{center}
\begin{tabular}{l}
|\||ifdefined\childdocname\endinput\||fi\newif\ifchilddoc|\\
|\edef\childdocname{\scantokens\expandafter{\jobname\noexpand}}|\\
|\def\childdocmain{|\textit{main}|}\||ifx\childdocmain\childdocname\||else|\\
|\childdoctrue\includeonly{\childdocname}\let\jobname\childdocmain\||fi|\\
\end{tabular}
\end{center}
%
Instead of |\childdocof{|\textit{main}|}| just include the main file
at the top of each child file:
%
\begin{center}
|\input{|\textit{main}|}|
\end{center}
%
A simple redirection |\childdocforward{|\textit{dest}|}| is achieved by:
%
\begin{center}
|\def\jobname{|\textit{dest}|}\input{\jobname}|
\end{center}
%
The redirection with prefix
|\childdocforwardprefix[|\textit{prefix}|]{|\textit{dest}|}|
is accomplished by:
%
\begin{center}
\begin{tabular}{l}
|{\edef\jobname{\scantokens\expandafter{\jobname\noexpand}}|\\
|\def\redirectjob |\textit{prefix}|#1~~~{\gdef\jobname{|\textit{dest}|#1}}|\\
|\expandafter\redirectjob\jobname~~~}\input{\jobname}|
\end{tabular}
\end{center}

In an alternative approach,
child documents can be compiled by a specific command line
without additional code or specific definitions:
%
\begin{center}
|... -jobname "|\textit{target}|" "|[\textit{flags}]%
|\includeonly{|\textit{dest}|}\input{|\textit{main}|}"|
\end{center}
%

%%%%%%%%%%%%%%%%%%%%%%%%%%%%%%%%%%%%%%%%%%%%%%%%%%%%%%%%%%%%%%%%%%%%%%%%%%%%%%%%
%%%%%%%%%%%%%%%%%%%%%%%%%%%%%%%%%%%%%%%%%%%%%%%%%%%%%%%%%%%%%%%%%%%%%%%%%%%%%%%%
\section{Information}

%%%%%%%%%%%%%%%%%%%%%%%%%%%%%%%%%%%%%%%%%%%%%%%%%%%%%%%%%%%%%%%%%%%%%%%%%%%%%%%%
\subsection{Copyright}

Copyright \copyright{} 2017--2018 Niklas Beisert

This work may be distributed and/or modified under the
conditions of the \LaTeX{} Project Public License, either version 1.3
of this license or (at your option) any later version.
The latest version of this license is in
  \url{http://www.latex-project.org/lppl.txt}
and version 1.3 or later is part of all distributions of \LaTeX{}
version 2005/12/01 or later.

This work has the LPPL maintenance status `maintained'.

The Current Maintainer of this work is Niklas Beisert.

This work consists of the files |README.txt|, |childdoc.ins| and |childdoc.dtx|
as well as the derived files |childdoc.def|, |cdocsamp.tex|
with |cdocsch1.tex|, |cdocsch2.tex|, |cdocspt3.tex|, |cdocspt4.tex|,
|cdocsdrf.tex|, |cdocsfn1.tex|, |cdocsfn2.tex|
as well as |childdoc.pdf|.

%%%%%%%%%%%%%%%%%%%%%%%%%%%%%%%%%%%%%%%%%%%%%%%%%%%%%%%%%%%%%%%%%%%%%%%%%%%%%%%%
\subsection{Files and Installation}

The package consists of the files:
%
\begin{center}
\begin{tabular}{ll}
    |README.txt|   & readme file \\
    |childdoc.ins| & installation file \\
    |childdoc.dtx| & source file \\
    |childdoc.def| & definition file \\
    |cdocsamp.tex| & sample main file \\
    |cdocsch1.tex| & sample include file \\
    |cdocsch2.tex| & sample include file \\
    |cdocspt3.tex| & sample part file \\
    |cdocspt4.tex| & sample part file \\
    |cdocsdrf.tex| & sample redirection file \\
    |cdocsfn1.tex| & sample redirection file \\
    |cdocsfn2.tex| & sample redirection file \\
    |childdoc.pdf| & manual
\end{tabular}
\end{center}
%
The distribution consists of the files
|README.txt|, |childdoc.ins| and |childdoc.dtx|.
%
\begin{itemize}
\item
Run (pdf)\LaTeX{} on |childdoc.dtx|
to compile the manual |childdoc.pdf| (this file).
\item
Run \LaTeX{} on |childdoc.ins| to create the definitions file |childdoc.def|
and the sample |cdocsamp.tex| with include files
|cdocsch1.tex|, |cdocsch2.tex|, |cdocspt3.tex|, |cdocspt4.tex|,
|cdocsdrf.tex|, |cdocsfn1.tex|, |cdocsfn2.tex|.
Then copy the file |childdoc.def| to an appropriate directory of your \LaTeX{}
distribution, e.g.\ \textit{texmf-root}|/tex/latex/childdoc|.
\end{itemize}

%%%%%%%%%%%%%%%%%%%%%%%%%%%%%%%%%%%%%%%%%%%%%%%%%%%%%%%%%%%%%%%%%%%%%%%%%%%%%%%%
\subsection{Related CTAN Packages}

There are several other packages which offer a similar functionality:
%
\begin{itemize}
\item
The packages
\href{http://ctan.org/pkg/docmute}{\textsf{docmute}},
\href{http://ctan.org/pkg/includex}{\textsf{includex}} and
\href{http://ctan.org/pkg/standalone}{\textsf{standalone}}
provide commands to include only the document body of
a child file thus allowing both files to be compiled individually.
\item
The packages \href{http://ctan.org/pkg/subdocs}{\textsf{subdocs}}
and \href{http://ctan.org/pkg/subfiles}{\textsf{subfiles}}
provide structures in which the main and child documents can be
encapsulated and allowing them to be compiled individually.
The inclusion mechanism is different from the conventional |\include|.
\item
The package \href{http://ctan.org/pkg/combine}{\textsf{combine}}
is an elaborate solution to combine several documents into one.
\end{itemize}
%
See also the CTAN topic \href{http://ctan.org/topic/subdocs}{\textsf{subdocs}}
for further related packages.
The present package differs from the above solutions in that
a document structure constructed with the conventional |\include| mechanism
just needs two extra commands at the top of every file
such that all constituent files can be compiled individually.

%%%%%%%%%%%%%%%%%%%%%%%%%%%%%%%%%%%%%%%%%%%%%%%%%%%%%%%%%%%%%%%%%%%%%%%%%%%%%%%%
%\subsection{Feature Suggestions}
%
%The following is a list of features which may be useful for future
%versions of this package:
%%
%\begin{itemize}
%\item
%\ldots
%\end{itemize}

%%%%%%%%%%%%%%%%%%%%%%%%%%%%%%%%%%%%%%%%%%%%%%%%%%%%%%%%%%%%%%%%%%%%%%%%%%%%%%%%
\subsection{Revision History}

%%%%%%%%%%%%%%%%%%%%%%%%%%%%%%%%%%%%%%%%
\paragraph{v2.0:} 2018/12/30

\begin{itemize}
\item
immediate forward processing
\item
added |\childdocby| mechanism
\item
manual restructured
\end{itemize}

%%%%%%%%%%%%%%%%%%%%%%%%%%%%%%%%%%%%%%%%
\paragraph{v1.6:} 2018/01/17

\begin{itemize}
\item
application for development of include files
\item
corrections to manual
\end{itemize}

%%%%%%%%%%%%%%%%%%%%%%%%%%%%%%%%%%%%%%%%
\paragraph{v1.5:} 2017/05/21

\begin{itemize}
\item
more complete structuring introduced
\item
|\childdocof| introduced
\item
|\childdoc| renamed to |\childdocmain|
\item
|\childredirect| renamed to |\childdocforward| and |\childdocforwardprefix|
and functionality expanded
\end{itemize}

%%%%%%%%%%%%%%%%%%%%%%%%%%%%%%%%%%%%%%%%
\paragraph{v1.0:} 2017/04/27

\begin{itemize}
\item
manual and install package
\item
first version published on CTAN
\end{itemize}

%%%%%%%%%%%%%%%%%%%%%%%%%%%%%%%%%%%%%%%%
\paragraph{v0.6:} 2017/04/26

\begin{itemize}
\item
redirection mechanism added
\end{itemize}

%%%%%%%%%%%%%%%%%%%%%%%%%%%%%%%%%%%%%%%%
\paragraph{v0.5:} 2017/04/26

\begin{itemize}
\item
functionality in definition file
\end{itemize}


%%%%%%%%%%%%%%%%%%%%%%%%%%%%%%%%%%%%%%%%%%%%%%%%%%%%%%%%%%%%%%%%%%%%%%%%%%%%%%%%
%%%%%%%%%%%%%%%%%%%%%%%%%%%%%%%%%%%%%%%%%%%%%%%%%%%%%%%%%%%%%%%%%%%%%%%%%%%%%%%%
%%%%%%%%%%%%%%%%%%%%%%%%%%%%%%%%%%%%%%%%%%%%%%%%%%%%%%%%%%%%%%%%%%%%%%%%%%%%%%%%
\appendix

\settowidth\MacroIndent{\rmfamily\scriptsize 000\ }

 \DocInput{childdoc.dtx}

\end{document}
%</driver>
% \fi
%
% %%%%%%%%%%%%%%%%%%%%%%%%%%%%%%%%%%%%%%%%%%%%%%%%%%%%%%%%%%%%%%%%%%%%%%%%%%%%%%
% %%%%%%%%%%%%%%%%%%%%%%%%%%%%%%%%%%%%%%%%%%%%%%%%%%%%%%%%%%%%%%%%%%%%%%%%%%%%%%
% \section{Sample}
%\iffalse
%<*samplemain>
%\fi
%
% The following presents a sample document
% with two chapters, two parts, a title page,
% a compile flag as well as three forwarding files to set the flag.
% It consists of eight |.tex| files:
% \begin{center}
% \begin{tabular}{ll}
% |cdocsamp.tex|&main file\\
% |cdocsch1.tex|&include file for chapter 1\\
% |cdocsch2.tex|&include file for chapter 2\\
% |cdocspt3.tex|&include file for part 3\\
% |cdocspt4.tex|&include file for part 4\\
% |cdocsdrf.tex|&forwarding file for main file in draft mode\\
% |cdocsfi1.tex|&forwarding file for final version of chapter 1\\
% |cdocsfi2.tex|&forwarding file for final version of chapter 2\\
% \end{tabular}
% \end{center}
% Each of the eight files can be compiled directly by the \LaTeX{} compiler.
%
% %%%%%%%%%%%%%%%%%%%%%%%%%%%%%%%%%%%%%%
% \paragraph{Main File.}
%
% The main file is called |cdocsamp.tex|.
%
% Load the \textsf{childdoc} definitions and
% declare the filename for the main document:
%    \begin{macrocode}
\input{childdoc.def}
\childdocmain{}
%    \end{macrocode}

% Optional override for |\version| flag:
%    \begin{macrocode}
%%\ifchilddoc\else\providecommand{\version}{draft}\fi
%    \end{macrocode}

% Define the default values for the |\version| flag
% (|final| for the main file and |draft| for childs):
%    \begin{macrocode}
\ifchilddoc
\providecommand{\version}{draft}
\else
\providecommand{\version}{final}
\fi
%    \end{macrocode}

% Load the standard document class:
%    \begin{macrocode}
\documentclass[12pt]{article}
%    \end{macrocode}

% Start the document body:
%    \begin{macrocode}
\begin{document}
%    \end{macrocode}

% Declare a title page.
% Print title, part of document being processed and version flag:
%    \begin{macrocode}
\addtocounter{page}{-1}
\begin{center}
{\LARGE\bfseries{}childdoc example\par}
\vspace{1cm}
\ifchilddoc
\ifchilddocmanual part\else chapter\fi:
`\childdocname' of `\childdocjob'\par
\else
main document: `\childdocjob'\par
\fi
version: \version\par
\end{center}
\newpage
%    \end{macrocode}

% Manually include selected file,
% otherwise process as usual:
%    \begin{macrocode}
\ifchilddocmanual
\section*{part `\childdocname'}
\input{\childdocname}
\else
%    \end{macrocode}

% Include the two chapters:
%    \begin{macrocode}
\include{cdocsch1}
\include{cdocsch2}
%    \end{macrocode}

% Include the two parts unless only chapters should be displayed:
%    \begin{macrocode}
\ifchilddoc\else
\section{part three}
\input{cdocspt3}
\section{part four}
\input{cdocspt4}
\fi
%    \end{macrocode}

% Process as usual until here:
%    \begin{macrocode}
\fi
%    \end{macrocode}

% End of document body:
%    \begin{macrocode}
\end{document}
%    \end{macrocode}
%\iffalse
%</samplemain>
%\fi
%
% %%%%%%%%%%%%%%%%%%%%%%%%%%%%%%%%%%%%%%
% \paragraph{Chapter Include Files.}
%
% The include files are called |cdocsch1.tex| and |cdocsch2.tex|.
%
%\iffalse
%<*samplechap1|samplechap2>
%\fi

% Optional override for |\version| flag:
%    \begin{macrocode}
%%\providecommand{\version}{final}
%    \end{macrocode}

% Include the main document:
%    \begin{macrocode}
\input{childdoc.def}
\childdocof{cdocsamp}
%    \end{macrocode}

%\iffalse
%</samplechap1|samplechap2>
%\fi
%
%\iffalse
%<*samplechap1>
%\fi
% Some text for chapter 1:
%    \begin{macrocode}
\section{one}
some text in chapter one
%    \end{macrocode}

%\iffalse
%</samplechap1>
%\fi
% Some text for chapter 2:
%\iffalse
%<*samplechap2>
%\fi
%    \begin{macrocode}
\section{two}
more text in chapter two
%    \end{macrocode}

%\iffalse
%</samplechap2>
%\fi
%
% %%%%%%%%%%%%%%%%%%%%%%%%%%%%%%%%%%%%%%
% \paragraph{Part Include Files.}
%
% The include files are called |cdocspt3.tex| and |cdocspt4.tex|.
%
%\iffalse
%<*samplepart3|samplepart4>
%\fi

% Optional override for |\version| flag:
%    \begin{macrocode}
%%\providecommand{\version}{final}
%    \end{macrocode}

% Include the main document:
%    \begin{macrocode}
\input{childdoc.def}
\childdocby{cdocsamp}
%    \end{macrocode}

%\iffalse
%</samplepart3|samplepart4>
%\fi
%
%\iffalse
%<*samplepart3>
%\fi
% Some text for part 3:
%    \begin{macrocode}
some text in part three
%    \end{macrocode}

%\iffalse
%</samplepart3>
%\fi
% Some text for part 4:
%\iffalse
%<*samplepart4>
%\fi
%    \begin{macrocode}
more text in part four
%    \end{macrocode}

%\iffalse
%</samplepart4>
%\fi
%
% %%%%%%%%%%%%%%%%%%%%%%%%%%%%%%%%%%%%%%
% \paragraph{Forwarding for a Complete Draft.}
%
% The following forwarding file |cdocsdrf.tex|
% compiles the main document in draft mode:
%\iffalse
%<*sampledraft>
%\fi
%    \begin{macrocode}
\def\version{draft}
\input{childdoc.def}
\childdocforward{cdocsamp}
%    \end{macrocode}

%\iffalse
%</sampledraft>
%\fi
%
% %%%%%%%%%%%%%%%%%%%%%%%%%%%%%%%%%%%%%%
% \paragraph{Forwarding for Final Version of the Chapters.}
%
% The following forwarding files |cdocsfn1.tex| and |cdocsfn2.tex|
% (with identical content)
% compile the final versions of the child documents
% |cdocsch1.tex| and |cdocsch2.tex|, respectively:
%\iffalse
%<*samplefinal>
%\fi
%    \begin{macrocode}
\def\version{final}
\input{childdoc.def}
\childdocforwardprefix[cdocsamp]{cdocsfn}{cdocsch}
%    \end{macrocode}

%\iffalse
%</samplefinal>
%\fi
%
% %%%%%%%%%%%%%%%%%%%%%%%%%%%%%%%%%%%%%%
% \paragraph{Command Line Processing.}
%
% The following three command lines generate the output files
% |cdocscld|, |cdocscl1| and |cdocscl2|
% which should be identical to
% |cdocsdrf|, |cdocsch1| and |cdocsfn2|, respectively:
% \begin{center}
% \begin{tabular}{l}
% |latex -jobname cdocscld \|\\
% |  "\def\version{draft}\input{childdoc.def}\childdocforward{cdocsamp}"|\\
% |latex -jobname cdocscl1 \|\\
% |  "\input{childdoc.def}\childdocforward[cdocsamp]{cdocsch1}"|\\
% |latex -jobname cdocscl2 \|\\
% |  "\def\version{final}\input{childdoc.def}\childdocforward{cdocsch2}"|
% \end{tabular}
% \end{center}
% Note that the trailing backslash on each first line
% merely continues the input to the second line
% (for convenient cut ant paste).
% Furthermore, the command |latex| can be replaced by any
% of its alternative versions such as |pdflatex|.
%
% %%%%%%%%%%%%%%%%%%%%%%%%%%%%%%%%%%%%%%%%%%%%%%%%%%%%%%%%%%%%%%%%%%%%%%%%%%%%%%
% %%%%%%%%%%%%%%%%%%%%%%%%%%%%%%%%%%%%%%%%%%%%%%%%%%%%%%%%%%%%%%%%%%%%%%%%%%%%%%
% \section{Implementation}
%\iffalse
%<*package>
%\fi
%
% This section describes the definitions file |childdoc.def|.

% The definitions cannot be loaded using |\usepackage| or |\RequirePackage|
% which has a mechanism to prevent loading a style file more than once.
% When loading the definitions by means of |\input|
% multiple instances have to be prevented manually:
%\iffalse
%This code needs to be before the `\ProvidesFile' directive
%which is defined at the beginning of this file.
%Therefore it is also placed there and commented out here.
%</package>
%<*discard>
%\fi
%    \begin{macrocode}
\ifdefined\childdocmain\endinput\fi
%    \end{macrocode}
%\iffalse
%</discard>
%<*package>
%\fi
%
% \macro{\ifchilddoc}
% \macro{\ifchilddocmanual}
% The conditional |\ifchilddoc| tells whether a
% child (true) or main (false) document is being compiled.
% The conditional |\ifchilddocmanual| tells whether
% the |\includeonly| mechanism is used (false) or
% the selection of child files must be performed manually (true).
% The definitions initialise to false:
%    \begin{macrocode}
\newif\ifchilddoc
\newif\ifchilddocmanual
%    \end{macrocode}

% \macro{\childdocname}
% \macro{\childdocjob}
% The macro |\childdocname| stores the name of the main document
% to be compiled. The macro |\childdocjob| stores the name of
% the document on which the \LaTeX{} compiler was originally invoked.
% The content of |\jobname| cannot be compared
% to filenames specified in the source due to different catcodes.
% The following code rescans |\jobname|, stores the result
% in |\childdocname| and saves a copy in |\childdocjob|:
%    \begin{macrocode}
\edef\childdocname{\scantokens\expandafter{\jobname\noexpand}}
\let\childdocjob\childdocname
%    \end{macrocode}

% \macro{\childdocdisable}
% The macro |\childdocdisable| prevents the main file
% from being processed more than once.
% At this stage, the main document command |\childdocmain|
% is assumed to be called once again where it should do nothing.
% Any subsequent call to it should prevent
% a secondary processing of the main document
% It overwrites the forwarding commands
% |\childdocof| and |\childdocforward|
% with empty macros to prevent further inclusions of the main document:
%    \begin{macrocode}
\newcommand{\childdocdisable}
{
  \renewcommand{\childdocmain}[1]{\renewcommand{\childdocmain}[1]{\endinput}}
  \renewcommand{\childdocof}[1]{}
  \renewcommand{\childdocby}[2][]{}
  \renewcommand{\childdocforward}[2][]{}
  \renewcommand{\childdocdisable}{}
}
%    \end{macrocode}

% \macro{\childdocmain}
% The macro |\childdocmain| is to be called at the top of the main file
% with nothing or the main filename (without extension) as argument.
% First, it breaks loops.
% If the argument is not empty and does not match |\childdocname|
% (which is set by the first inclusion of |childdoc.def|),
% |\ifchilddoc| is set to true, |\includeonly| is applied to the child file
% and |\jobname| is set to the main file
% (for proper handling of |.aux| files):
%    \begin{macrocode}
\newcommand{\childdocmain}[1]
{
  \childdocdisable\childdocmain{}
  \if?#1?\else
    \begingroup
      \def\childdoctmp{#1}
      \ifx\childdoctmp\childdocname
        \def\childdoctmp{}
      \else
        \def\childdoctmp
        {
          \childdoctrue
          \includeonly{\childdocname}
          \def\childdocjob{#1}
          \def\jobname{#1}
        }
      \fi
      \expandafter
    \endgroup
    \childdoctmp
  \fi
}
%    \end{macrocode}

% \macro{\childdocof}
% The command |\childdocof| redirects
% compilation to the main file |#1|.
%    \begin{macrocode}
\newcommand{\childdocof}[1]
{
  \childdocdisable
  \childdoctrue
  \includeonly{\childdocname}
  \def\jobname{#1}
  \def\childdocjob{#1}
  \input{#1}
}
%    \end{macrocode}

% \macro{\childdocby}
% The command |\childdocby| ....
%    \begin{macrocode}
\newcommand{\childdocby}[2][]
{
  \childdocdisable
  \childdoctrue
  \childdocmanualtrue
  \if?#1?\else
    \def\jobname{#2}
  \fi
  \def\childdocjob{#2}
  \input{#2}
  \endinput
}
%    \end{macrocode}

% \macro{\childdocforward}
% The command |\childdocforward| redirects
% compilation to the main file or
% (if the optional argument is given) a child file.
% Parameters are set as if the main file
% or a child file starting with |\childdocof| was compiled.
% Then compilation is handed over to the main file:
%    \begin{macrocode}
\newcommand{\childdocforward}[2][]
{
  \begingroup
    \if?#1?
      \def\childdoctmp
      {
        \def\childdocname{#2}
        \def\childdocjob{#2}
        \def\jobname{#2}
        \input{#2}
        \endinput
      }
    \else
      \def\childdoctmp
      {
        \childdocdisable
        \def\childdocname{#2}
        \childdoctrue
        \includeonly{#2}
        \def\childdocjob{#1}
        \def\jobname{#1}
        \input{#1}
        \endinput
      }
    \fi
    \expandafter
  \endgroup
  \childdoctmp
}
%    \end{macrocode}

% \macro{\childdocforwardprefix}
% The command |\childdocforwardprefix| redirects
% compilation to the main or a child file by means of a pattern.
% The prefix |#1| in the current filename is replaced by |#2|
% and the suffix of the current filename is kept
% (it is assumed that the filename does not contain the substring `|~~~|'
% which is used as a delimiter).
% Compilation is handed over to the new file by |\childdocforward|:
%    \begin{macrocode}
\newcommand{\childdocforwardprefix}[3][]
{
  \begingroup
    \def\childdocextract #2##1~~~{\def\childdoctmp{\childdocforward[#1]{#3##1}}}
    \expandafter\childdocextract\childdocname~~~
    \expandafter
  \endgroup
  \childdoctmp
}
%    \end{macrocode}

% \macro{\childdoc}
% The deprecated macro |\childdoc| is a legacy version of |\childdocmain|:
%    \begin{macrocode}
\newcommand{\childdoc}{\childdocmain}
%    \end{macrocode}

% \macro{\childdocredirect}
% The deprecated macro |\childdocredirect| is a legacy version
% of |\childdocforward| and |\childdocforwardprefix|:
%    \begin{macrocode}
\newcommand{\childdocredirect}[2][]
{
  \begingroup
    \if?#1?
      \def\childdoctmp{\childdocforward{#2}}
    \else
      \def\childdoctmp{\childdocforwardprefix{#1}{#2}}
    \fi
    \expandafter
  \endgroup
  \childdoctmp
}
%    \end{macrocode}

%\iffalse
%</package>
%\fi
%
\endinput
|\\
|\childdocmain{}|\\
\end{tabular}
\end{center}
at the very top of the main \LaTeX{} file,
in particular \emph{before} the |\documentclass| statement!
The argument of |\childdocmain| should be left empty
(but it must be present).

%%%%%%%%%%%%%%%%%%%%%%%%%%%%%%%%%%%%%%%%
\DescribeMacro{\childdocof}
Furthermore, add the commands
\begin{center}
\begin{tabular}{l}
|% \iffalse
%
% childdoc.dtx Copyright (C) 2017-2018 Niklas Beisert
%
% This work may be distributed and/or modified under the
% conditions of the LaTeX Project Public License, either version 1.3
% of this license or (at your option) any later version.
% The latest version of this license is in
%   http://www.latex-project.org/lppl.txt
% and version 1.3 or later is part of all distributions of LaTeX
% version 2005/12/01 or later.
%
% This work has the LPPL maintenance status `maintained'.
%
% The Current Maintainer of this work is Niklas Beisert.
%
% This work consists of the files childdoc.dtx and childdoc.ins
% and the derived files childdoc.def and cdocsamp.tex with
% cdocsch1.tex, cdocsch2.tex, cdocsdrf.tex, cdocsfn1.tex, cdocsfn2.tex.
%
%<package>\ifdefined\childdocmain\endinput\fi
%<package>\ProvidesFile{childdoc.def}[2018/12/30 v2.0 child document driver]
%<samplemain>\ProvidesFile{cdocsamp.tex}[2018/12/30 v2.0 sample for childdoc]
%<*driver>
%\ProvidesFile{childdoc.drv}[2018/12/30 v2.0 childdoc reference manual file]
\PassOptionsToClass{10pt,a4paper}{article}
\documentclass{ltxdoc}

\usepackage[margin=35mm]{geometry}
\usepackage{hyperref}
\usepackage{hyperxmp}
\usepackage[usenames]{color}

\hypersetup{colorlinks=true}
\hypersetup{pdfstartview=FitH}
\hypersetup{pdfpagemode=UseNone}
\hypersetup{pdfsource={}}
\hypersetup{pdflang={en-UK}}
\hypersetup{pdfcopyright={Copyright 2017-2018 Niklas Beisert.
  This work may be distributed and/or modified under the
  conditions of the LaTeX Project Public License, either version 1.3
  of this license or (at your option) any later version.}}
\hypersetup{pdflicenseurl={http://www.latex-project.org/lppl.txt}}
\hypersetup{pdfcontactaddress={ETH Zurich, ITP, HIT K,
  Wolfgang-Pauli-Strasse 27}}
\hypersetup{pdfcontactpostcode={8093}}
\hypersetup{pdfcontactcity={Zurich}}
\hypersetup{pdfcontactcountry={Switzerland}}
\hypersetup{pdfcontactemail={nbeisert@itp.phys.ethz.ch}}
\hypersetup{pdfcontacturl={http://people.phys.ethz.ch/\xmptilde nbeisert/}}

\newcommand{\secref}[1]{\hyperref[#1]{section \ref*{#1}}}

\parskip1ex
\parindent0pt
\let\olditemize\itemize
\def\itemize{\olditemize\parskip0pt}

\begin{document}

\title{The \textsf{childdoc} Package}
\hypersetup{pdftitle={The childdoc Package}}
\author{Niklas Beisert\\[2ex]
  Institut f\"ur Theoretische Physik\\
  Eidgen\"ossische Technische Hochschule Z\"urich\\
  Wolfgang-Pauli-Strasse 27, 8093 Z\"urich, Switzerland\\[1ex]
  \href{mailto:nbeisert@itp.phys.ethz.ch}
  {\texttt{nbeisert@itp.phys.ethz.ch}}}
\hypersetup{pdfauthor={Niklas Beisert}}
\hypersetup{pdfsubject={Manual for the LaTeX2e Package childdoc}}
\date{30 December 2018, \textsf{v2.0}}
\maketitle

\begin{abstract}\noindent
\textsf{childdoc} is a \LaTeXe{} package
that enables the direct compilation
of document sections included by |\include|
to individual files.
\end{abstract}

\begingroup
\parskip0ex
\tableofcontents
\endgroup

%%%%%%%%%%%%%%%%%%%%%%%%%%%%%%%%%%%%%%%%%%%%%%%%%%%%%%%%%%%%%%%%%%%%%%%%%%%%%%%%
%%%%%%%%%%%%%%%%%%%%%%%%%%%%%%%%%%%%%%%%%%%%%%%%%%%%%%%%%%%%%%%%%%%%%%%%%%%%%%%%
\section{Introduction}

\LaTeX{} provides a mechanism to structure a large document (such as a book)
into a main file and several child files (containing the chapters)
using the |\include| command.
This mechanism is beneficial for documents
which span hundreds of pages in order to
make the source file(s) more manageable.
Moreover, compilation can be restricted to
selected child files by means of the |\includeonly| command.
The latter feature can be used to reduce the compilation time while editing
(this was significantly more useful in the earlier days of \LaTeX{})
or to generate a smaller document which is easier to navigate.
Another application of |\includeonly| is to generate
documents consisting of selected parts of the complete document.

However, there are a few drawbacks of the plain |\include| mechanism:
\begin{itemize}
\item
The child files cannot be compiled on their own,
they can only be compiled via the main file.
A naive editing environment
(such as a text editor with an option
to have the current file processed by \LaTeX)
may require one to switch to the main file before compiling;
attempting to compile the child file produces errors.
\item
The main file must be modified (each time)
to adjust the |\includeonly| command
to the present needs. This easily leaves the main file in a messy state.
\item
The generated document will always carry the filename
of the main document. This is inconvenient if
several child files are to be compiled and
to be kept for distribution.
\end{itemize}

The present package provides a simple interface
to make child files individually compilable by \LaTeX{}.
Compiling a child file then has the same effect as compiling
the main file with an |\includeonly| command
to select the appropriate child.
Moreover the generated document will carry the name of the child
rather than the main file.
This resolves all three above issues.

This feature is meant to make the editing of books,
thesis documents and lecture notes somewhat more convenient.
However, the package can also be used efficiently for
composing a series of documents (such as exercise sheets)
which are typically distributed individually.
It then assists the author in generating the individual documents
(potentially in different versions)
as well as a document containing the collected series.
Another application is in developing style files
or other kinds of included material
where compilation of the style file could redirect
to a sample or test file.

%%%%%%%%%%%%%%%%%%%%%%%%%%%%%%%%%%%%%%%%%%%%%%%%%%%%%%%%%%%%%%%%%%%%%%%%%%%%%%%%
%%%%%%%%%%%%%%%%%%%%%%%%%%%%%%%%%%%%%%%%%%%%%%%%%%%%%%%%%%%%%%%%%%%%%%%%%%%%%%%%
\section{Usage}

First of all, the package \textsf{childdoc} is \emph{not} a standard
\LaTeXe{} |.sty| style file! Therefore it needs to be invoked in
a non-standard way.

%%%%%%%%%%%%%%%%%%%%%%%%%%%%%%%%%%%%%%%%%%%%%%%%%%%%%%%%%%%%%%%%%%%%%%%%%%%%%%%%
\subsection{Included Files}
\label{sec:include}

%%%%%%%%%%%%%%%%%%%%%%%%%%%%%%%%%%%%%%%%
\DescribeMacro{\childdocmain}
To use the package, add the commands
\begin{center}
\begin{tabular}{l}
|\input{childdoc.def}|\\
|\childdocmain{}|\\
\end{tabular}
\end{center}
at the very top of the main \LaTeX{} file,
in particular \emph{before} the |\documentclass| statement!
The argument of |\childdocmain| should be left empty
(but it must be present).

%%%%%%%%%%%%%%%%%%%%%%%%%%%%%%%%%%%%%%%%
\DescribeMacro{\childdocof}
Furthermore, add the commands
\begin{center}
\begin{tabular}{l}
|\input{childdoc.def}|\\
|\childdocof{|\textit{main}|}|\\
\end{tabular}
\end{center}
at the top of every child file \textit{child}
which is included by |\include{|\textit{child}|}|
from within the main file
(or at least for those files to be compiled individually).
The argument \textit{main} must be the filename of the main file.

There are a couple of
considerations in setting up the main and child documents:

%%%%%%%%%%%%%%%%%%%%%%%%%%%%%%%%%%%%%%%%
\paragraph{Restrictions.}

Please note the following restrictions:
\begin{itemize}
\item
|\childdocmain| must be called with one argument \textit{main}
to ensure compatibility with earlier version of the package.
It must either be empty (|\childdocmain{}|)
or precisely match the filename of the main file in which it is specified.
See \secref{sec:detection} for further information.
\item
The filename \textit{main} must be specified without the |.tex| extension.
\item
The filename \textit{main} is case sensitive
(even in case-insensitive file systems)
due to internal string comparison.
\item
The argument \textit{main} should be fully expanded, it cannot be a macro.
\item
Subdirectories and special characters should be avoided in filenames.
\item
The command |\childdocmain{|\textit{main}|}| must be followed by a whitespace.
It should not be followed immediately by another command
or by a comment mark `|%|'.
This is because the \TeX{} parser reads the token immediately following
the argument of |\childdocmain| and puts it
at the beginning of every child section;
however, a white\-space is ignored.
\end{itemize}

%%%%%%%%%%%%%%%%%%%%%%%%%%%%%%%%%%%%%%%%
\paragraph{Content of Main File.}

It is advisable to place all content in the child files included by |\include|.
Any output contained in the main file will appear in all child documents
unless suppressed manually;
it cannot be suppressed automatically by the |\includeonly| directive
and thus should normally be avoided.
A method to include some content in the main file
by means of conditional processing is described in \secref{sec:conditional}.

%%%%%%%%%%%%%%%%%%%%%%%%%%%%%%%%%%%%%%%%
\paragraph{Page Numbering.}

When only a part of the document is compiled,
the appropriate numbering of pages
(as well as other status parameters)
is determined from the |.aux| files.
The latter contain information from previous passes.
However this information needs to propagate through
all intermediate child documents.
Therefore the page numbering in child documents may well
be inconsistent until the complete document is compiled at least once.

A useful (if unconventional) way to always ensure a consistent
page numbering is to restart the numbering in each child document
and denote the pages by `\textit{child}|.|\textit{page}'
where \textit{child} represents the chapter/section number of the child file.
This can be achieved by the command
|\numberwithin{page}{|\textit{child}|}|
of the \textsf{amsmath} package
where \textit{child} can be |chapter| or |section|
depending on the chosen structuring.
Alternatively, one can modify the macro |\thepage| appropriately
and reset the counter |page| at the start of each child file.

%%%%%%%%%%%%%%%%%%%%%%%%%%%%%%%%%%%%%%%%%%%%%%%%%%%%%%%%%%%%%%%%%%%%%%%%%%%%%%%%
\subsection{Conditional Processing}
\label{sec:conditional}

The package provides a mechanism to compile different versions
of a document. To customise the versions further some conditional processing
can come in handy to distinguish which version is being compiled.
The package provides two macros to describe the compilation context:

%%%%%%%%%%%%%%%%%%%%%%%%%%%%%%%%%%%%%%%%
\DescribeMacro{\ifchilddoc}
The conditional |\ifchilddoc| distinguishes between the compilation of
child documents and the main document:
%
\begin{center}
|\ifchilddoc |\textit{child-code}| |[|\||else |\textit{main-code}]| \||fi|
\end{center}

%%%%%%%%%%%%%%%%%%%%%%%%%%%%%%%%%%%%%%%%
\DescribeMacro{\childdocname}
\DescribeMacro{\childdocjob}
The macro |\childdocname| contains the filename (without extension)
of the main or child file being processed.
Note that |\childdocjob| will always contain the name of the main file.

%%%%%%%%%%%%%%%%%%%%%%%%%%%%%%%%%%%%%%%%
\paragraph{Title Page.}

Conditional processing can be used to include a title or banner page
in the main document when proper precautions are taken.
Importantly, the code in the main file should ensure that the page counter
(as well as other status parameters which are stored in the |.aux| files)
takes the same value after the conditional processing.
Otherwise the page numbers may take divergent values
depending on which part is compiled.

For example, a title page could be declared by:
%
\begin{center}
\begin{tabular}{l}
|\ifchilddoc\||else|\\
|\addtocounter{page}{-1}|\\
\textit{code for title page}\\
|\newpage|\\
|\||fi|
\end{tabular}
\end{center}
%
A banner page for the child documents can be generated by:
%
\begin{center}
\begin{tabular}{l}
|\ifchilddoc|\\
|\addtocounter{page}{-1}|\\
\textit{code for banner page}\\
|\newpage|\\
|\||fi|
\end{tabular}
\end{center}
%
Here one could write a message such as:
\begin{center}
|This is the part \childdocname{} of \childdocjob{}.|
\end{center}

%%%%%%%%%%%%%%%%%%%%%%%%%%%%%%%%%%%%%%%%%%%%%%%%%%%%%%%%%%%%%%%%%%%%%%%%%%%%%%%%
\subsection{Flags}
\label{sec:flags}

The package makes it easy to generate different versions
of the main or child documents.
To this end compilation flags can be defined
and assigned different default values.
They will be particularly useful in conjunction
with the forwarding mechanism described in \secref{sec:forward}.

For example, it may be useful to have a flag |\version|
which can be set to |draft| or |final|.
The document source will contain some conditional code
depending on the value of |\version|.
Suppose further, the flag should default to |final| for the main file
and to |draft| for child files
which is a natural assignment for editing the document.
This is achieved by placing the following code
in the preamble of the main document
(below the |\childdocmain| directive):
%
\begin{center}
\begin{tabular}{l}
|\ifchilddoc|\\
|\providecommand{\version}{draft}|\\
|\||else|\\
|\providecommand{\version}{final}|\\
|\||fi|
\end{tabular}
\end{center}
%
The definition by |\providecommand| makes sure
that previous definitions are not overwritten.
Further statements |\providecommand{\version}{...}|
can thus be added before the above code to override it.

For the main file, one might add a line
(between |\childdocmain| and the above block)
%
\begin{center}
|%\ifchilddoc\||else\providecommand{\version}{draft}\||fi|
\end{center}
%
which can be uncommented to produce a draft version.
Likewise one can add a line to the very top of a child file
(above the |\childdocof{|\textit{main}|}| directive)
%
\begin{center}
|%\providecommand{\version}{final}|
\end{center}
%
which can be uncommented to produce the final version of this child document.

%%%%%%%%%%%%%%%%%%%%%%%%%%%%%%%%%%%%%%%%%%%%%%%%%%%%%%%%%%%%%%%%%%%%%%%%%%%%%%%%
\subsection{Forwarding}
\label{sec:forward}

Different versions of the main or child documents
using compilation flags as described in \secref{sec:flags}
can be (permanently) stored in different files
for convenient compilation, viewing and distribution.
To this end, the package defines a command
to pass on compilation to a different file:

%%%%%%%%%%%%%%%%%%%%%%%%%%%%%%%%%%%%%%%%
\DescribeMacro{\childdocforward}
The command |\childdocforward| redirects processing to
another source file:
%
\begin{center}
\begin{tabular}{l}
|\input{childdoc.def}|\\
|\childdocforward[|\textit{main}|]{|\textit{dest}|}|\\
\end{tabular}
\end{center}
%
The argument \textit{dest} is the destination file
(without extension).
It should be the main file or one of the child files.
Note that further \textsf{childdoc} directives
such as |\childdocof| and |\childdocforward|
in the indicated file will be processed in this form.
The optional argument \textit{main}
passes on directly to the main file \textit{main}
while pretending to compile the child \textit{dest}.
This form behaves as if \textit{dest}
issues |\childdocof{|\textit{main}|}| right away,
and no further \textsf{childdoc} directives will be processed.

%%%%%%%%%%%%%%%%%%%%%%%%%%%%%%%%%%%%%%%%
\DescribeMacro{\...prefix}
In the alternative form |\childdocforwardprefix|,
%
\begin{center}
\begin{tabular}{l}
|\input{childdoc.def}|\\
|\childdocforwardprefix[|\textit{main}|]{|\textit{prefix}|}{|\textit{dest}|}|
\end{tabular}
\end{center}
%
the destination file is determined by a pattern
depending on the current file:
To make this work, the current file must be called
`{\textit{prefix}\hspace{0.2em}\textit{suffix}}'
with \textit{prefix} matching precisely the argument.
Processing is then passed on to the file
`{\textit{dest}\hspace{0.2em}\textit{suffix}}'.
Surely, the same effect is achieved by
directly specifying the
argument `{\textit{dest}\hspace{0.2em}\textit{suffix}}'
in the first form.
However, that requires to set up a different file
for each child. With the alternative form of the command
all these files can have exactly the same content
which simplifies setting them up and maintaining them.

For example, the following file |draft.tex|
with a compilation flag |\version| as described in \secref{sec:flags}
compiles the main document as a draft:
%
\begin{center}
\begin{tabular}{l}
|\def\version{draft}|\\
|\input{childdoc.def}|\\
|\childdocforward{|\textit{main}|}|
\end{tabular}
\end{center}
%
Likewise, the following files |final|\textit{nn}|.tex|
compile the final version of the child document
|child|\textit{nn}|.tex|:
%
\begin{center}
\begin{tabular}{l}
|\def\version{final}|\\
|\input{childdoc.def}|\\
|\childdocforwardprefix{final}{child}|
\end{tabular}
\end{center}
%

Note that when several versions of a main file and/or of each child file
are to be generated, it may be convenient to set up a |Makefile| or
shell script to automatise the process.

%%%%%%%%%%%%%%%%%%%%%%%%%%%%%%%%%%%%%%%%%%%%%%%%%%%%%%%%%%%%%%%%%%%%%%%%%%%%%%%%
\subsection{Command Line Processing}
\label{sec:commandline}

The effect of redirection files can also be achieved by invoking
the \LaTeX{} compiler with a more elaborate command line.
Most conveniently this should be done as part
of a shell script or a |Makefile|.

When using \textsf{childdoc} in the main file, the following
command lines effectively perform a redirection
(note that depending on the shell being used,
backslashes may have to be doubled: `|\|' $\to$ `|\\|'):
%
\begin{center}
|... -jobname "|\textit{target}|" |\\|"|[\textit{flags}]%
|\input{childdoc.def}\childdocforward[|\textit{main}|]{|\textit{dest}|}"|
\end{center}
%
Here \textit{target} is the name of the output file,
\textit{main} is the name of the main file
and \textit{dest} is the name of the main or child file to be processed
(all filenames without extensions).
The optional argument \textit{main} can be omitted
if \textit{main} matches \textit{dest}.
Optionally, compilation \textit{flags} can be defined via |\def| commands.
This command line makes the \TeX{} engine believe
it is compiling the file \textit{target}
whose content is specified as the latter parameter.
The provided code then forwards the processing to
\textit{main} or \textit{dest} as described in \secref{sec:forward}.

%%%%%%%%%%%%%%%%%%%%%%%%%%%%%%%%%%%%%%%%%%%%%%%%%%%%%%%%%%%%%%%%%%%%%%%%%%%%%%%%
\subsection{Include by Input}
\label{sec:input}

Including child documents by |\include| has some restrictions by design.
Most notably, the content of a child document always occupies
its own set of pages; pages cannot be shared between child documents.
Usually, this behaviour makes perfect sense
because each child document contain an essential part of the document.
However, in some situations it may be desirable to compose
a document from a collection of parts
without having mandatory page breaks between then.
For this case, the package
provides a mechanism to include parts
by |\input| which can also be processed individually.
However, by construction this mechanism
requires manual handling of the content to be output.

%%%%%%%%%%%%%%%%%%%%%%%%%%%%%%%%%%%%%%%%
\DescribeMacro{\ifchilddocmanual}
The main file should be prepared as usual, see \secref{sec:include}.
However, the document body must make a distinction
between processing of an individual part and of the main document, e.g.:
%
\begin{center}
\begin{tabular}{l}
|\ifchilddocmanual|\\
|\input{\childdocname}|\\
|\||else|\\
\textit{document body with }|\input{|\textit{part}|}|\\
|\||fi|
\end{tabular}
\end{center}
%
The conditional |\ifchilddocmanual| is true whenever
a part to be included by |\input| is being compiled,
and the name of the part is stored in |\childdocname|.

%%%%%%%%%%%%%%%%%%%%%%%%%%%%%%%%%%%%%%%%
\DescribeMacro{\childdocby}
Each part to be included by |\input| should start with:
%
\begin{center}
\begin{tabular}{l}
|\input{childdoc.def}|\\
|\childdocby{|\textit{main}|}|\\
\end{tabular}
\end{center}
%
The directive |\childdocby| is similar to |\childdocof|
described in \secref{sec:include},
but the subsequent selection of content must be done manually.
To that end, both |\ifchilddoc| and |\ifchilddocmanual|
will be true upon processing of a part,
and the name of the part is stored in |\childdocname|.
Note that |\jobname| will be set to the filename of the current part
so that each part receives an individual |.aux| file
that does not interfere with the |.aux| file(s) of the main document.
This behaviour can be altered by the alternative form
|\childdocby[*]{|\textit{main}|}| (with a non-empty optional argument)
which uses the |.aux| file of the main document
by setting |\jobname| to \textit{main}.

%%%%%%%%%%%%%%%%%%%%%%%%%%%%%%%%%%%%%%%%%%%%%%%%%%%%%%%%%%%%%%%%%%%%%%%%%%%%%%%%
\subsection{Driver Development}
\label{sec:driver}

The \textsf{childdoc} mechanism can also be use for the development
of definition files such as \LaTeX{} styles or classes.
This case differs from the above setup with multiple parts
included by |\include| in that no |\includeonly| should be invoked.
This can be achieved by starting the include file
(before |\ProvidesPackage|) with:
%
\begin{center}
\begin{tabular}{l}
|\input{childdoc.def}|\\
|\childdocforward{|\textit{main}|}|\\
\end{tabular}
\end{center}
%
or alternatively with:
%
\begin{center}
\begin{tabular}{l}
|\input{childdoc.def}|\\
|\childdocby{|\textit{main}|}|\\
\end{tabular}
\end{center}
%
Both forms have slightly different effects as described above.
The main file is prepared as usual, see \secref{sec:include}.

%%%%%%%%%%%%%%%%%%%%%%%%%%%%%%%%%%%%%%%%%%%%%%%%%%%%%%%%%%%%%%%%%%%%%%%%%%%%%%%%
\subsection{Legacy Detection}
\label{sec:detection}

The directive |\childdocmain| in the main file can detect
whether the complete document or merely a child is to be compiled
even without using the directive |\childdocof|.
This method is deprecated because it is less robust
and there is no compelling reason to use it;
it is merely provided for backward compatibility
and it may be removed in future versions.

If the detection mechanism is to be used,
it is mandatory to correctly specify
the filename of the main file as the argument of |\childdocmain|:
%
\begin{center}
\begin{tabular}{l}
|\input{childdoc.def}|\\
|\childdocmain{|\textit{main}|}|\\
\end{tabular}
\end{center}
%
If |\jobname| does not match the argument \textit{main} of |\childdocmain|,
it is assumed that |\jobname| points to the child file to be compiled.
When using |\childdocmain| with the main file specified as argument,
it suffices to start a child file
with just |\input{|\textit{main}|}|
without loading of the package and using |\childdocof|.
If instead all processing is done
with the appropriate \textsf{childdoc} directives,
the argument of \textit{main} of |\childdocmain| can be empty.

An alternative version of the command line processing described
in \secref{sec:commandline} using the detection mechanism reads:
%
\begin{center}
|... -jobname "|\textit{target}|" "|[\textit{flags}]%
[|\def\jobname{|\textit{dest}|}|]|\input{|\textit{main}|}"|
\end{center}

%%%%%%%%%%%%%%%%%%%%%%%%%%%%%%%%%%%%%%%%%%%%%%%%%%%%%%%%%%%%%%%%%%%%%%%%%%%%%%%%
\subsection{Manual Code}
\label{sec:manual}

In case one cannot be certain whether the definitions file |childdoc.def|
is installed on the target \TeX{} distribution
and one prefers not to ship it,
it is conceivable to paste a few relevant commands into the sources.

To that end, drop all statements |\input{childdoc.def}|
and perform the replacements as outlined below.
Instead of |\childdocmain{|\textit{main}|}| add the following code
to the top of the main file:
%
\begin{center}
\begin{tabular}{l}
|\||ifdefined\childdocname\endinput\||fi\newif\ifchilddoc|\\
|\edef\childdocname{\scantokens\expandafter{\jobname\noexpand}}|\\
|\def\childdocmain{|\textit{main}|}\||ifx\childdocmain\childdocname\||else|\\
|\childdoctrue\includeonly{\childdocname}\let\jobname\childdocmain\||fi|\\
\end{tabular}
\end{center}
%
Instead of |\childdocof{|\textit{main}|}| just include the main file
at the top of each child file:
%
\begin{center}
|\input{|\textit{main}|}|
\end{center}
%
A simple redirection |\childdocforward{|\textit{dest}|}| is achieved by:
%
\begin{center}
|\def\jobname{|\textit{dest}|}\input{\jobname}|
\end{center}
%
The redirection with prefix
|\childdocforwardprefix[|\textit{prefix}|]{|\textit{dest}|}|
is accomplished by:
%
\begin{center}
\begin{tabular}{l}
|{\edef\jobname{\scantokens\expandafter{\jobname\noexpand}}|\\
|\def\redirectjob |\textit{prefix}|#1~~~{\gdef\jobname{|\textit{dest}|#1}}|\\
|\expandafter\redirectjob\jobname~~~}\input{\jobname}|
\end{tabular}
\end{center}

In an alternative approach,
child documents can be compiled by a specific command line
without additional code or specific definitions:
%
\begin{center}
|... -jobname "|\textit{target}|" "|[\textit{flags}]%
|\includeonly{|\textit{dest}|}\input{|\textit{main}|}"|
\end{center}
%

%%%%%%%%%%%%%%%%%%%%%%%%%%%%%%%%%%%%%%%%%%%%%%%%%%%%%%%%%%%%%%%%%%%%%%%%%%%%%%%%
%%%%%%%%%%%%%%%%%%%%%%%%%%%%%%%%%%%%%%%%%%%%%%%%%%%%%%%%%%%%%%%%%%%%%%%%%%%%%%%%
\section{Information}

%%%%%%%%%%%%%%%%%%%%%%%%%%%%%%%%%%%%%%%%%%%%%%%%%%%%%%%%%%%%%%%%%%%%%%%%%%%%%%%%
\subsection{Copyright}

Copyright \copyright{} 2017--2018 Niklas Beisert

This work may be distributed and/or modified under the
conditions of the \LaTeX{} Project Public License, either version 1.3
of this license or (at your option) any later version.
The latest version of this license is in
  \url{http://www.latex-project.org/lppl.txt}
and version 1.3 or later is part of all distributions of \LaTeX{}
version 2005/12/01 or later.

This work has the LPPL maintenance status `maintained'.

The Current Maintainer of this work is Niklas Beisert.

This work consists of the files |README.txt|, |childdoc.ins| and |childdoc.dtx|
as well as the derived files |childdoc.def|, |cdocsamp.tex|
with |cdocsch1.tex|, |cdocsch2.tex|, |cdocspt3.tex|, |cdocspt4.tex|,
|cdocsdrf.tex|, |cdocsfn1.tex|, |cdocsfn2.tex|
as well as |childdoc.pdf|.

%%%%%%%%%%%%%%%%%%%%%%%%%%%%%%%%%%%%%%%%%%%%%%%%%%%%%%%%%%%%%%%%%%%%%%%%%%%%%%%%
\subsection{Files and Installation}

The package consists of the files:
%
\begin{center}
\begin{tabular}{ll}
    |README.txt|   & readme file \\
    |childdoc.ins| & installation file \\
    |childdoc.dtx| & source file \\
    |childdoc.def| & definition file \\
    |cdocsamp.tex| & sample main file \\
    |cdocsch1.tex| & sample include file \\
    |cdocsch2.tex| & sample include file \\
    |cdocspt3.tex| & sample part file \\
    |cdocspt4.tex| & sample part file \\
    |cdocsdrf.tex| & sample redirection file \\
    |cdocsfn1.tex| & sample redirection file \\
    |cdocsfn2.tex| & sample redirection file \\
    |childdoc.pdf| & manual
\end{tabular}
\end{center}
%
The distribution consists of the files
|README.txt|, |childdoc.ins| and |childdoc.dtx|.
%
\begin{itemize}
\item
Run (pdf)\LaTeX{} on |childdoc.dtx|
to compile the manual |childdoc.pdf| (this file).
\item
Run \LaTeX{} on |childdoc.ins| to create the definitions file |childdoc.def|
and the sample |cdocsamp.tex| with include files
|cdocsch1.tex|, |cdocsch2.tex|, |cdocspt3.tex|, |cdocspt4.tex|,
|cdocsdrf.tex|, |cdocsfn1.tex|, |cdocsfn2.tex|.
Then copy the file |childdoc.def| to an appropriate directory of your \LaTeX{}
distribution, e.g.\ \textit{texmf-root}|/tex/latex/childdoc|.
\end{itemize}

%%%%%%%%%%%%%%%%%%%%%%%%%%%%%%%%%%%%%%%%%%%%%%%%%%%%%%%%%%%%%%%%%%%%%%%%%%%%%%%%
\subsection{Related CTAN Packages}

There are several other packages which offer a similar functionality:
%
\begin{itemize}
\item
The packages
\href{http://ctan.org/pkg/docmute}{\textsf{docmute}},
\href{http://ctan.org/pkg/includex}{\textsf{includex}} and
\href{http://ctan.org/pkg/standalone}{\textsf{standalone}}
provide commands to include only the document body of
a child file thus allowing both files to be compiled individually.
\item
The packages \href{http://ctan.org/pkg/subdocs}{\textsf{subdocs}}
and \href{http://ctan.org/pkg/subfiles}{\textsf{subfiles}}
provide structures in which the main and child documents can be
encapsulated and allowing them to be compiled individually.
The inclusion mechanism is different from the conventional |\include|.
\item
The package \href{http://ctan.org/pkg/combine}{\textsf{combine}}
is an elaborate solution to combine several documents into one.
\end{itemize}
%
See also the CTAN topic \href{http://ctan.org/topic/subdocs}{\textsf{subdocs}}
for further related packages.
The present package differs from the above solutions in that
a document structure constructed with the conventional |\include| mechanism
just needs two extra commands at the top of every file
such that all constituent files can be compiled individually.

%%%%%%%%%%%%%%%%%%%%%%%%%%%%%%%%%%%%%%%%%%%%%%%%%%%%%%%%%%%%%%%%%%%%%%%%%%%%%%%%
%\subsection{Feature Suggestions}
%
%The following is a list of features which may be useful for future
%versions of this package:
%%
%\begin{itemize}
%\item
%\ldots
%\end{itemize}

%%%%%%%%%%%%%%%%%%%%%%%%%%%%%%%%%%%%%%%%%%%%%%%%%%%%%%%%%%%%%%%%%%%%%%%%%%%%%%%%
\subsection{Revision History}

%%%%%%%%%%%%%%%%%%%%%%%%%%%%%%%%%%%%%%%%
\paragraph{v2.0:} 2018/12/30

\begin{itemize}
\item
immediate forward processing
\item
added |\childdocby| mechanism
\item
manual restructured
\end{itemize}

%%%%%%%%%%%%%%%%%%%%%%%%%%%%%%%%%%%%%%%%
\paragraph{v1.6:} 2018/01/17

\begin{itemize}
\item
application for development of include files
\item
corrections to manual
\end{itemize}

%%%%%%%%%%%%%%%%%%%%%%%%%%%%%%%%%%%%%%%%
\paragraph{v1.5:} 2017/05/21

\begin{itemize}
\item
more complete structuring introduced
\item
|\childdocof| introduced
\item
|\childdoc| renamed to |\childdocmain|
\item
|\childredirect| renamed to |\childdocforward| and |\childdocforwardprefix|
and functionality expanded
\end{itemize}

%%%%%%%%%%%%%%%%%%%%%%%%%%%%%%%%%%%%%%%%
\paragraph{v1.0:} 2017/04/27

\begin{itemize}
\item
manual and install package
\item
first version published on CTAN
\end{itemize}

%%%%%%%%%%%%%%%%%%%%%%%%%%%%%%%%%%%%%%%%
\paragraph{v0.6:} 2017/04/26

\begin{itemize}
\item
redirection mechanism added
\end{itemize}

%%%%%%%%%%%%%%%%%%%%%%%%%%%%%%%%%%%%%%%%
\paragraph{v0.5:} 2017/04/26

\begin{itemize}
\item
functionality in definition file
\end{itemize}


%%%%%%%%%%%%%%%%%%%%%%%%%%%%%%%%%%%%%%%%%%%%%%%%%%%%%%%%%%%%%%%%%%%%%%%%%%%%%%%%
%%%%%%%%%%%%%%%%%%%%%%%%%%%%%%%%%%%%%%%%%%%%%%%%%%%%%%%%%%%%%%%%%%%%%%%%%%%%%%%%
%%%%%%%%%%%%%%%%%%%%%%%%%%%%%%%%%%%%%%%%%%%%%%%%%%%%%%%%%%%%%%%%%%%%%%%%%%%%%%%%
\appendix

\settowidth\MacroIndent{\rmfamily\scriptsize 000\ }

 \DocInput{childdoc.dtx}

\end{document}
%</driver>
% \fi
%
% %%%%%%%%%%%%%%%%%%%%%%%%%%%%%%%%%%%%%%%%%%%%%%%%%%%%%%%%%%%%%%%%%%%%%%%%%%%%%%
% %%%%%%%%%%%%%%%%%%%%%%%%%%%%%%%%%%%%%%%%%%%%%%%%%%%%%%%%%%%%%%%%%%%%%%%%%%%%%%
% \section{Sample}
%\iffalse
%<*samplemain>
%\fi
%
% The following presents a sample document
% with two chapters, two parts, a title page,
% a compile flag as well as three forwarding files to set the flag.
% It consists of eight |.tex| files:
% \begin{center}
% \begin{tabular}{ll}
% |cdocsamp.tex|&main file\\
% |cdocsch1.tex|&include file for chapter 1\\
% |cdocsch2.tex|&include file for chapter 2\\
% |cdocspt3.tex|&include file for part 3\\
% |cdocspt4.tex|&include file for part 4\\
% |cdocsdrf.tex|&forwarding file for main file in draft mode\\
% |cdocsfi1.tex|&forwarding file for final version of chapter 1\\
% |cdocsfi2.tex|&forwarding file for final version of chapter 2\\
% \end{tabular}
% \end{center}
% Each of the eight files can be compiled directly by the \LaTeX{} compiler.
%
% %%%%%%%%%%%%%%%%%%%%%%%%%%%%%%%%%%%%%%
% \paragraph{Main File.}
%
% The main file is called |cdocsamp.tex|.
%
% Load the \textsf{childdoc} definitions and
% declare the filename for the main document:
%    \begin{macrocode}
\input{childdoc.def}
\childdocmain{}
%    \end{macrocode}

% Optional override for |\version| flag:
%    \begin{macrocode}
%%\ifchilddoc\else\providecommand{\version}{draft}\fi
%    \end{macrocode}

% Define the default values for the |\version| flag
% (|final| for the main file and |draft| for childs):
%    \begin{macrocode}
\ifchilddoc
\providecommand{\version}{draft}
\else
\providecommand{\version}{final}
\fi
%    \end{macrocode}

% Load the standard document class:
%    \begin{macrocode}
\documentclass[12pt]{article}
%    \end{macrocode}

% Start the document body:
%    \begin{macrocode}
\begin{document}
%    \end{macrocode}

% Declare a title page.
% Print title, part of document being processed and version flag:
%    \begin{macrocode}
\addtocounter{page}{-1}
\begin{center}
{\LARGE\bfseries{}childdoc example\par}
\vspace{1cm}
\ifchilddoc
\ifchilddocmanual part\else chapter\fi:
`\childdocname' of `\childdocjob'\par
\else
main document: `\childdocjob'\par
\fi
version: \version\par
\end{center}
\newpage
%    \end{macrocode}

% Manually include selected file,
% otherwise process as usual:
%    \begin{macrocode}
\ifchilddocmanual
\section*{part `\childdocname'}
\input{\childdocname}
\else
%    \end{macrocode}

% Include the two chapters:
%    \begin{macrocode}
\include{cdocsch1}
\include{cdocsch2}
%    \end{macrocode}

% Include the two parts unless only chapters should be displayed:
%    \begin{macrocode}
\ifchilddoc\else
\section{part three}
\input{cdocspt3}
\section{part four}
\input{cdocspt4}
\fi
%    \end{macrocode}

% Process as usual until here:
%    \begin{macrocode}
\fi
%    \end{macrocode}

% End of document body:
%    \begin{macrocode}
\end{document}
%    \end{macrocode}
%\iffalse
%</samplemain>
%\fi
%
% %%%%%%%%%%%%%%%%%%%%%%%%%%%%%%%%%%%%%%
% \paragraph{Chapter Include Files.}
%
% The include files are called |cdocsch1.tex| and |cdocsch2.tex|.
%
%\iffalse
%<*samplechap1|samplechap2>
%\fi

% Optional override for |\version| flag:
%    \begin{macrocode}
%%\providecommand{\version}{final}
%    \end{macrocode}

% Include the main document:
%    \begin{macrocode}
\input{childdoc.def}
\childdocof{cdocsamp}
%    \end{macrocode}

%\iffalse
%</samplechap1|samplechap2>
%\fi
%
%\iffalse
%<*samplechap1>
%\fi
% Some text for chapter 1:
%    \begin{macrocode}
\section{one}
some text in chapter one
%    \end{macrocode}

%\iffalse
%</samplechap1>
%\fi
% Some text for chapter 2:
%\iffalse
%<*samplechap2>
%\fi
%    \begin{macrocode}
\section{two}
more text in chapter two
%    \end{macrocode}

%\iffalse
%</samplechap2>
%\fi
%
% %%%%%%%%%%%%%%%%%%%%%%%%%%%%%%%%%%%%%%
% \paragraph{Part Include Files.}
%
% The include files are called |cdocspt3.tex| and |cdocspt4.tex|.
%
%\iffalse
%<*samplepart3|samplepart4>
%\fi

% Optional override for |\version| flag:
%    \begin{macrocode}
%%\providecommand{\version}{final}
%    \end{macrocode}

% Include the main document:
%    \begin{macrocode}
\input{childdoc.def}
\childdocby{cdocsamp}
%    \end{macrocode}

%\iffalse
%</samplepart3|samplepart4>
%\fi
%
%\iffalse
%<*samplepart3>
%\fi
% Some text for part 3:
%    \begin{macrocode}
some text in part three
%    \end{macrocode}

%\iffalse
%</samplepart3>
%\fi
% Some text for part 4:
%\iffalse
%<*samplepart4>
%\fi
%    \begin{macrocode}
more text in part four
%    \end{macrocode}

%\iffalse
%</samplepart4>
%\fi
%
% %%%%%%%%%%%%%%%%%%%%%%%%%%%%%%%%%%%%%%
% \paragraph{Forwarding for a Complete Draft.}
%
% The following forwarding file |cdocsdrf.tex|
% compiles the main document in draft mode:
%\iffalse
%<*sampledraft>
%\fi
%    \begin{macrocode}
\def\version{draft}
\input{childdoc.def}
\childdocforward{cdocsamp}
%    \end{macrocode}

%\iffalse
%</sampledraft>
%\fi
%
% %%%%%%%%%%%%%%%%%%%%%%%%%%%%%%%%%%%%%%
% \paragraph{Forwarding for Final Version of the Chapters.}
%
% The following forwarding files |cdocsfn1.tex| and |cdocsfn2.tex|
% (with identical content)
% compile the final versions of the child documents
% |cdocsch1.tex| and |cdocsch2.tex|, respectively:
%\iffalse
%<*samplefinal>
%\fi
%    \begin{macrocode}
\def\version{final}
\input{childdoc.def}
\childdocforwardprefix[cdocsamp]{cdocsfn}{cdocsch}
%    \end{macrocode}

%\iffalse
%</samplefinal>
%\fi
%
% %%%%%%%%%%%%%%%%%%%%%%%%%%%%%%%%%%%%%%
% \paragraph{Command Line Processing.}
%
% The following three command lines generate the output files
% |cdocscld|, |cdocscl1| and |cdocscl2|
% which should be identical to
% |cdocsdrf|, |cdocsch1| and |cdocsfn2|, respectively:
% \begin{center}
% \begin{tabular}{l}
% |latex -jobname cdocscld \|\\
% |  "\def\version{draft}\input{childdoc.def}\childdocforward{cdocsamp}"|\\
% |latex -jobname cdocscl1 \|\\
% |  "\input{childdoc.def}\childdocforward[cdocsamp]{cdocsch1}"|\\
% |latex -jobname cdocscl2 \|\\
% |  "\def\version{final}\input{childdoc.def}\childdocforward{cdocsch2}"|
% \end{tabular}
% \end{center}
% Note that the trailing backslash on each first line
% merely continues the input to the second line
% (for convenient cut ant paste).
% Furthermore, the command |latex| can be replaced by any
% of its alternative versions such as |pdflatex|.
%
% %%%%%%%%%%%%%%%%%%%%%%%%%%%%%%%%%%%%%%%%%%%%%%%%%%%%%%%%%%%%%%%%%%%%%%%%%%%%%%
% %%%%%%%%%%%%%%%%%%%%%%%%%%%%%%%%%%%%%%%%%%%%%%%%%%%%%%%%%%%%%%%%%%%%%%%%%%%%%%
% \section{Implementation}
%\iffalse
%<*package>
%\fi
%
% This section describes the definitions file |childdoc.def|.

% The definitions cannot be loaded using |\usepackage| or |\RequirePackage|
% which has a mechanism to prevent loading a style file more than once.
% When loading the definitions by means of |\input|
% multiple instances have to be prevented manually:
%\iffalse
%This code needs to be before the `\ProvidesFile' directive
%which is defined at the beginning of this file.
%Therefore it is also placed there and commented out here.
%</package>
%<*discard>
%\fi
%    \begin{macrocode}
\ifdefined\childdocmain\endinput\fi
%    \end{macrocode}
%\iffalse
%</discard>
%<*package>
%\fi
%
% \macro{\ifchilddoc}
% \macro{\ifchilddocmanual}
% The conditional |\ifchilddoc| tells whether a
% child (true) or main (false) document is being compiled.
% The conditional |\ifchilddocmanual| tells whether
% the |\includeonly| mechanism is used (false) or
% the selection of child files must be performed manually (true).
% The definitions initialise to false:
%    \begin{macrocode}
\newif\ifchilddoc
\newif\ifchilddocmanual
%    \end{macrocode}

% \macro{\childdocname}
% \macro{\childdocjob}
% The macro |\childdocname| stores the name of the main document
% to be compiled. The macro |\childdocjob| stores the name of
% the document on which the \LaTeX{} compiler was originally invoked.
% The content of |\jobname| cannot be compared
% to filenames specified in the source due to different catcodes.
% The following code rescans |\jobname|, stores the result
% in |\childdocname| and saves a copy in |\childdocjob|:
%    \begin{macrocode}
\edef\childdocname{\scantokens\expandafter{\jobname\noexpand}}
\let\childdocjob\childdocname
%    \end{macrocode}

% \macro{\childdocdisable}
% The macro |\childdocdisable| prevents the main file
% from being processed more than once.
% At this stage, the main document command |\childdocmain|
% is assumed to be called once again where it should do nothing.
% Any subsequent call to it should prevent
% a secondary processing of the main document
% It overwrites the forwarding commands
% |\childdocof| and |\childdocforward|
% with empty macros to prevent further inclusions of the main document:
%    \begin{macrocode}
\newcommand{\childdocdisable}
{
  \renewcommand{\childdocmain}[1]{\renewcommand{\childdocmain}[1]{\endinput}}
  \renewcommand{\childdocof}[1]{}
  \renewcommand{\childdocby}[2][]{}
  \renewcommand{\childdocforward}[2][]{}
  \renewcommand{\childdocdisable}{}
}
%    \end{macrocode}

% \macro{\childdocmain}
% The macro |\childdocmain| is to be called at the top of the main file
% with nothing or the main filename (without extension) as argument.
% First, it breaks loops.
% If the argument is not empty and does not match |\childdocname|
% (which is set by the first inclusion of |childdoc.def|),
% |\ifchilddoc| is set to true, |\includeonly| is applied to the child file
% and |\jobname| is set to the main file
% (for proper handling of |.aux| files):
%    \begin{macrocode}
\newcommand{\childdocmain}[1]
{
  \childdocdisable\childdocmain{}
  \if?#1?\else
    \begingroup
      \def\childdoctmp{#1}
      \ifx\childdoctmp\childdocname
        \def\childdoctmp{}
      \else
        \def\childdoctmp
        {
          \childdoctrue
          \includeonly{\childdocname}
          \def\childdocjob{#1}
          \def\jobname{#1}
        }
      \fi
      \expandafter
    \endgroup
    \childdoctmp
  \fi
}
%    \end{macrocode}

% \macro{\childdocof}
% The command |\childdocof| redirects
% compilation to the main file |#1|.
%    \begin{macrocode}
\newcommand{\childdocof}[1]
{
  \childdocdisable
  \childdoctrue
  \includeonly{\childdocname}
  \def\jobname{#1}
  \def\childdocjob{#1}
  \input{#1}
}
%    \end{macrocode}

% \macro{\childdocby}
% The command |\childdocby| ....
%    \begin{macrocode}
\newcommand{\childdocby}[2][]
{
  \childdocdisable
  \childdoctrue
  \childdocmanualtrue
  \if?#1?\else
    \def\jobname{#2}
  \fi
  \def\childdocjob{#2}
  \input{#2}
  \endinput
}
%    \end{macrocode}

% \macro{\childdocforward}
% The command |\childdocforward| redirects
% compilation to the main file or
% (if the optional argument is given) a child file.
% Parameters are set as if the main file
% or a child file starting with |\childdocof| was compiled.
% Then compilation is handed over to the main file:
%    \begin{macrocode}
\newcommand{\childdocforward}[2][]
{
  \begingroup
    \if?#1?
      \def\childdoctmp
      {
        \def\childdocname{#2}
        \def\childdocjob{#2}
        \def\jobname{#2}
        \input{#2}
        \endinput
      }
    \else
      \def\childdoctmp
      {
        \childdocdisable
        \def\childdocname{#2}
        \childdoctrue
        \includeonly{#2}
        \def\childdocjob{#1}
        \def\jobname{#1}
        \input{#1}
        \endinput
      }
    \fi
    \expandafter
  \endgroup
  \childdoctmp
}
%    \end{macrocode}

% \macro{\childdocforwardprefix}
% The command |\childdocforwardprefix| redirects
% compilation to the main or a child file by means of a pattern.
% The prefix |#1| in the current filename is replaced by |#2|
% and the suffix of the current filename is kept
% (it is assumed that the filename does not contain the substring `|~~~|'
% which is used as a delimiter).
% Compilation is handed over to the new file by |\childdocforward|:
%    \begin{macrocode}
\newcommand{\childdocforwardprefix}[3][]
{
  \begingroup
    \def\childdocextract #2##1~~~{\def\childdoctmp{\childdocforward[#1]{#3##1}}}
    \expandafter\childdocextract\childdocname~~~
    \expandafter
  \endgroup
  \childdoctmp
}
%    \end{macrocode}

% \macro{\childdoc}
% The deprecated macro |\childdoc| is a legacy version of |\childdocmain|:
%    \begin{macrocode}
\newcommand{\childdoc}{\childdocmain}
%    \end{macrocode}

% \macro{\childdocredirect}
% The deprecated macro |\childdocredirect| is a legacy version
% of |\childdocforward| and |\childdocforwardprefix|:
%    \begin{macrocode}
\newcommand{\childdocredirect}[2][]
{
  \begingroup
    \if?#1?
      \def\childdoctmp{\childdocforward{#2}}
    \else
      \def\childdoctmp{\childdocforwardprefix{#1}{#2}}
    \fi
    \expandafter
  \endgroup
  \childdoctmp
}
%    \end{macrocode}

%\iffalse
%</package>
%\fi
%
\endinput
|\\
|\childdocof{|\textit{main}|}|\\
\end{tabular}
\end{center}
at the top of every child file \textit{child}
which is included by |\include{|\textit{child}|}|
from within the main file
(or at least for those files to be compiled individually).
The argument \textit{main} must be the filename of the main file.

There are a couple of
considerations in setting up the main and child documents:

%%%%%%%%%%%%%%%%%%%%%%%%%%%%%%%%%%%%%%%%
\paragraph{Restrictions.}

Please note the following restrictions:
\begin{itemize}
\item
|\childdocmain| must be called with one argument \textit{main}
to ensure compatibility with earlier version of the package.
It must either be empty (|\childdocmain{}|)
or precisely match the filename of the main file in which it is specified.
See \secref{sec:detection} for further information.
\item
The filename \textit{main} must be specified without the |.tex| extension.
\item
The filename \textit{main} is case sensitive
(even in case-insensitive file systems)
due to internal string comparison.
\item
The argument \textit{main} should be fully expanded, it cannot be a macro.
\item
Subdirectories and special characters should be avoided in filenames.
\item
The command |\childdocmain{|\textit{main}|}| must be followed by a whitespace.
It should not be followed immediately by another command
or by a comment mark `|%|'.
This is because the \TeX{} parser reads the token immediately following
the argument of |\childdocmain| and puts it
at the beginning of every child section;
however, a white\-space is ignored.
\end{itemize}

%%%%%%%%%%%%%%%%%%%%%%%%%%%%%%%%%%%%%%%%
\paragraph{Content of Main File.}

It is advisable to place all content in the child files included by |\include|.
Any output contained in the main file will appear in all child documents
unless suppressed manually;
it cannot be suppressed automatically by the |\includeonly| directive
and thus should normally be avoided.
A method to include some content in the main file
by means of conditional processing is described in \secref{sec:conditional}.

%%%%%%%%%%%%%%%%%%%%%%%%%%%%%%%%%%%%%%%%
\paragraph{Page Numbering.}

When only a part of the document is compiled,
the appropriate numbering of pages
(as well as other status parameters)
is determined from the |.aux| files.
The latter contain information from previous passes.
However this information needs to propagate through
all intermediate child documents.
Therefore the page numbering in child documents may well
be inconsistent until the complete document is compiled at least once.

A useful (if unconventional) way to always ensure a consistent
page numbering is to restart the numbering in each child document
and denote the pages by `\textit{child}|.|\textit{page}'
where \textit{child} represents the chapter/section number of the child file.
This can be achieved by the command
|\numberwithin{page}{|\textit{child}|}|
of the \textsf{amsmath} package
where \textit{child} can be |chapter| or |section|
depending on the chosen structuring.
Alternatively, one can modify the macro |\thepage| appropriately
and reset the counter |page| at the start of each child file.

%%%%%%%%%%%%%%%%%%%%%%%%%%%%%%%%%%%%%%%%%%%%%%%%%%%%%%%%%%%%%%%%%%%%%%%%%%%%%%%%
\subsection{Conditional Processing}
\label{sec:conditional}

The package provides a mechanism to compile different versions
of a document. To customise the versions further some conditional processing
can come in handy to distinguish which version is being compiled.
The package provides two macros to describe the compilation context:

%%%%%%%%%%%%%%%%%%%%%%%%%%%%%%%%%%%%%%%%
\DescribeMacro{\ifchilddoc}
The conditional |\ifchilddoc| distinguishes between the compilation of
child documents and the main document:
%
\begin{center}
|\ifchilddoc |\textit{child-code}| |[|\||else |\textit{main-code}]| \||fi|
\end{center}

%%%%%%%%%%%%%%%%%%%%%%%%%%%%%%%%%%%%%%%%
\DescribeMacro{\childdocname}
\DescribeMacro{\childdocjob}
The macro |\childdocname| contains the filename (without extension)
of the main or child file being processed.
Note that |\childdocjob| will always contain the name of the main file.

%%%%%%%%%%%%%%%%%%%%%%%%%%%%%%%%%%%%%%%%
\paragraph{Title Page.}

Conditional processing can be used to include a title or banner page
in the main document when proper precautions are taken.
Importantly, the code in the main file should ensure that the page counter
(as well as other status parameters which are stored in the |.aux| files)
takes the same value after the conditional processing.
Otherwise the page numbers may take divergent values
depending on which part is compiled.

For example, a title page could be declared by:
%
\begin{center}
\begin{tabular}{l}
|\ifchilddoc\||else|\\
|\addtocounter{page}{-1}|\\
\textit{code for title page}\\
|\newpage|\\
|\||fi|
\end{tabular}
\end{center}
%
A banner page for the child documents can be generated by:
%
\begin{center}
\begin{tabular}{l}
|\ifchilddoc|\\
|\addtocounter{page}{-1}|\\
\textit{code for banner page}\\
|\newpage|\\
|\||fi|
\end{tabular}
\end{center}
%
Here one could write a message such as:
\begin{center}
|This is the part \childdocname{} of \childdocjob{}.|
\end{center}

%%%%%%%%%%%%%%%%%%%%%%%%%%%%%%%%%%%%%%%%%%%%%%%%%%%%%%%%%%%%%%%%%%%%%%%%%%%%%%%%
\subsection{Flags}
\label{sec:flags}

The package makes it easy to generate different versions
of the main or child documents.
To this end compilation flags can be defined
and assigned different default values.
They will be particularly useful in conjunction
with the forwarding mechanism described in \secref{sec:forward}.

For example, it may be useful to have a flag |\version|
which can be set to |draft| or |final|.
The document source will contain some conditional code
depending on the value of |\version|.
Suppose further, the flag should default to |final| for the main file
and to |draft| for child files
which is a natural assignment for editing the document.
This is achieved by placing the following code
in the preamble of the main document
(below the |\childdocmain| directive):
%
\begin{center}
\begin{tabular}{l}
|\ifchilddoc|\\
|\providecommand{\version}{draft}|\\
|\||else|\\
|\providecommand{\version}{final}|\\
|\||fi|
\end{tabular}
\end{center}
%
The definition by |\providecommand| makes sure
that previous definitions are not overwritten.
Further statements |\providecommand{\version}{...}|
can thus be added before the above code to override it.

For the main file, one might add a line
(between |\childdocmain| and the above block)
%
\begin{center}
|%\ifchilddoc\||else\providecommand{\version}{draft}\||fi|
\end{center}
%
which can be uncommented to produce a draft version.
Likewise one can add a line to the very top of a child file
(above the |\childdocof{|\textit{main}|}| directive)
%
\begin{center}
|%\providecommand{\version}{final}|
\end{center}
%
which can be uncommented to produce the final version of this child document.

%%%%%%%%%%%%%%%%%%%%%%%%%%%%%%%%%%%%%%%%%%%%%%%%%%%%%%%%%%%%%%%%%%%%%%%%%%%%%%%%
\subsection{Forwarding}
\label{sec:forward}

Different versions of the main or child documents
using compilation flags as described in \secref{sec:flags}
can be (permanently) stored in different files
for convenient compilation, viewing and distribution.
To this end, the package defines a command
to pass on compilation to a different file:

%%%%%%%%%%%%%%%%%%%%%%%%%%%%%%%%%%%%%%%%
\DescribeMacro{\childdocforward}
The command |\childdocforward| redirects processing to
another source file:
%
\begin{center}
\begin{tabular}{l}
|% \iffalse
%
% childdoc.dtx Copyright (C) 2017-2018 Niklas Beisert
%
% This work may be distributed and/or modified under the
% conditions of the LaTeX Project Public License, either version 1.3
% of this license or (at your option) any later version.
% The latest version of this license is in
%   http://www.latex-project.org/lppl.txt
% and version 1.3 or later is part of all distributions of LaTeX
% version 2005/12/01 or later.
%
% This work has the LPPL maintenance status `maintained'.
%
% The Current Maintainer of this work is Niklas Beisert.
%
% This work consists of the files childdoc.dtx and childdoc.ins
% and the derived files childdoc.def and cdocsamp.tex with
% cdocsch1.tex, cdocsch2.tex, cdocsdrf.tex, cdocsfn1.tex, cdocsfn2.tex.
%
%<package>\ifdefined\childdocmain\endinput\fi
%<package>\ProvidesFile{childdoc.def}[2018/12/30 v2.0 child document driver]
%<samplemain>\ProvidesFile{cdocsamp.tex}[2018/12/30 v2.0 sample for childdoc]
%<*driver>
%\ProvidesFile{childdoc.drv}[2018/12/30 v2.0 childdoc reference manual file]
\PassOptionsToClass{10pt,a4paper}{article}
\documentclass{ltxdoc}

\usepackage[margin=35mm]{geometry}
\usepackage{hyperref}
\usepackage{hyperxmp}
\usepackage[usenames]{color}

\hypersetup{colorlinks=true}
\hypersetup{pdfstartview=FitH}
\hypersetup{pdfpagemode=UseNone}
\hypersetup{pdfsource={}}
\hypersetup{pdflang={en-UK}}
\hypersetup{pdfcopyright={Copyright 2017-2018 Niklas Beisert.
  This work may be distributed and/or modified under the
  conditions of the LaTeX Project Public License, either version 1.3
  of this license or (at your option) any later version.}}
\hypersetup{pdflicenseurl={http://www.latex-project.org/lppl.txt}}
\hypersetup{pdfcontactaddress={ETH Zurich, ITP, HIT K,
  Wolfgang-Pauli-Strasse 27}}
\hypersetup{pdfcontactpostcode={8093}}
\hypersetup{pdfcontactcity={Zurich}}
\hypersetup{pdfcontactcountry={Switzerland}}
\hypersetup{pdfcontactemail={nbeisert@itp.phys.ethz.ch}}
\hypersetup{pdfcontacturl={http://people.phys.ethz.ch/\xmptilde nbeisert/}}

\newcommand{\secref}[1]{\hyperref[#1]{section \ref*{#1}}}

\parskip1ex
\parindent0pt
\let\olditemize\itemize
\def\itemize{\olditemize\parskip0pt}

\begin{document}

\title{The \textsf{childdoc} Package}
\hypersetup{pdftitle={The childdoc Package}}
\author{Niklas Beisert\\[2ex]
  Institut f\"ur Theoretische Physik\\
  Eidgen\"ossische Technische Hochschule Z\"urich\\
  Wolfgang-Pauli-Strasse 27, 8093 Z\"urich, Switzerland\\[1ex]
  \href{mailto:nbeisert@itp.phys.ethz.ch}
  {\texttt{nbeisert@itp.phys.ethz.ch}}}
\hypersetup{pdfauthor={Niklas Beisert}}
\hypersetup{pdfsubject={Manual for the LaTeX2e Package childdoc}}
\date{30 December 2018, \textsf{v2.0}}
\maketitle

\begin{abstract}\noindent
\textsf{childdoc} is a \LaTeXe{} package
that enables the direct compilation
of document sections included by |\include|
to individual files.
\end{abstract}

\begingroup
\parskip0ex
\tableofcontents
\endgroup

%%%%%%%%%%%%%%%%%%%%%%%%%%%%%%%%%%%%%%%%%%%%%%%%%%%%%%%%%%%%%%%%%%%%%%%%%%%%%%%%
%%%%%%%%%%%%%%%%%%%%%%%%%%%%%%%%%%%%%%%%%%%%%%%%%%%%%%%%%%%%%%%%%%%%%%%%%%%%%%%%
\section{Introduction}

\LaTeX{} provides a mechanism to structure a large document (such as a book)
into a main file and several child files (containing the chapters)
using the |\include| command.
This mechanism is beneficial for documents
which span hundreds of pages in order to
make the source file(s) more manageable.
Moreover, compilation can be restricted to
selected child files by means of the |\includeonly| command.
The latter feature can be used to reduce the compilation time while editing
(this was significantly more useful in the earlier days of \LaTeX{})
or to generate a smaller document which is easier to navigate.
Another application of |\includeonly| is to generate
documents consisting of selected parts of the complete document.

However, there are a few drawbacks of the plain |\include| mechanism:
\begin{itemize}
\item
The child files cannot be compiled on their own,
they can only be compiled via the main file.
A naive editing environment
(such as a text editor with an option
to have the current file processed by \LaTeX)
may require one to switch to the main file before compiling;
attempting to compile the child file produces errors.
\item
The main file must be modified (each time)
to adjust the |\includeonly| command
to the present needs. This easily leaves the main file in a messy state.
\item
The generated document will always carry the filename
of the main document. This is inconvenient if
several child files are to be compiled and
to be kept for distribution.
\end{itemize}

The present package provides a simple interface
to make child files individually compilable by \LaTeX{}.
Compiling a child file then has the same effect as compiling
the main file with an |\includeonly| command
to select the appropriate child.
Moreover the generated document will carry the name of the child
rather than the main file.
This resolves all three above issues.

This feature is meant to make the editing of books,
thesis documents and lecture notes somewhat more convenient.
However, the package can also be used efficiently for
composing a series of documents (such as exercise sheets)
which are typically distributed individually.
It then assists the author in generating the individual documents
(potentially in different versions)
as well as a document containing the collected series.
Another application is in developing style files
or other kinds of included material
where compilation of the style file could redirect
to a sample or test file.

%%%%%%%%%%%%%%%%%%%%%%%%%%%%%%%%%%%%%%%%%%%%%%%%%%%%%%%%%%%%%%%%%%%%%%%%%%%%%%%%
%%%%%%%%%%%%%%%%%%%%%%%%%%%%%%%%%%%%%%%%%%%%%%%%%%%%%%%%%%%%%%%%%%%%%%%%%%%%%%%%
\section{Usage}

First of all, the package \textsf{childdoc} is \emph{not} a standard
\LaTeXe{} |.sty| style file! Therefore it needs to be invoked in
a non-standard way.

%%%%%%%%%%%%%%%%%%%%%%%%%%%%%%%%%%%%%%%%%%%%%%%%%%%%%%%%%%%%%%%%%%%%%%%%%%%%%%%%
\subsection{Included Files}
\label{sec:include}

%%%%%%%%%%%%%%%%%%%%%%%%%%%%%%%%%%%%%%%%
\DescribeMacro{\childdocmain}
To use the package, add the commands
\begin{center}
\begin{tabular}{l}
|\input{childdoc.def}|\\
|\childdocmain{}|\\
\end{tabular}
\end{center}
at the very top of the main \LaTeX{} file,
in particular \emph{before} the |\documentclass| statement!
The argument of |\childdocmain| should be left empty
(but it must be present).

%%%%%%%%%%%%%%%%%%%%%%%%%%%%%%%%%%%%%%%%
\DescribeMacro{\childdocof}
Furthermore, add the commands
\begin{center}
\begin{tabular}{l}
|\input{childdoc.def}|\\
|\childdocof{|\textit{main}|}|\\
\end{tabular}
\end{center}
at the top of every child file \textit{child}
which is included by |\include{|\textit{child}|}|
from within the main file
(or at least for those files to be compiled individually).
The argument \textit{main} must be the filename of the main file.

There are a couple of
considerations in setting up the main and child documents:

%%%%%%%%%%%%%%%%%%%%%%%%%%%%%%%%%%%%%%%%
\paragraph{Restrictions.}

Please note the following restrictions:
\begin{itemize}
\item
|\childdocmain| must be called with one argument \textit{main}
to ensure compatibility with earlier version of the package.
It must either be empty (|\childdocmain{}|)
or precisely match the filename of the main file in which it is specified.
See \secref{sec:detection} for further information.
\item
The filename \textit{main} must be specified without the |.tex| extension.
\item
The filename \textit{main} is case sensitive
(even in case-insensitive file systems)
due to internal string comparison.
\item
The argument \textit{main} should be fully expanded, it cannot be a macro.
\item
Subdirectories and special characters should be avoided in filenames.
\item
The command |\childdocmain{|\textit{main}|}| must be followed by a whitespace.
It should not be followed immediately by another command
or by a comment mark `|%|'.
This is because the \TeX{} parser reads the token immediately following
the argument of |\childdocmain| and puts it
at the beginning of every child section;
however, a white\-space is ignored.
\end{itemize}

%%%%%%%%%%%%%%%%%%%%%%%%%%%%%%%%%%%%%%%%
\paragraph{Content of Main File.}

It is advisable to place all content in the child files included by |\include|.
Any output contained in the main file will appear in all child documents
unless suppressed manually;
it cannot be suppressed automatically by the |\includeonly| directive
and thus should normally be avoided.
A method to include some content in the main file
by means of conditional processing is described in \secref{sec:conditional}.

%%%%%%%%%%%%%%%%%%%%%%%%%%%%%%%%%%%%%%%%
\paragraph{Page Numbering.}

When only a part of the document is compiled,
the appropriate numbering of pages
(as well as other status parameters)
is determined from the |.aux| files.
The latter contain information from previous passes.
However this information needs to propagate through
all intermediate child documents.
Therefore the page numbering in child documents may well
be inconsistent until the complete document is compiled at least once.

A useful (if unconventional) way to always ensure a consistent
page numbering is to restart the numbering in each child document
and denote the pages by `\textit{child}|.|\textit{page}'
where \textit{child} represents the chapter/section number of the child file.
This can be achieved by the command
|\numberwithin{page}{|\textit{child}|}|
of the \textsf{amsmath} package
where \textit{child} can be |chapter| or |section|
depending on the chosen structuring.
Alternatively, one can modify the macro |\thepage| appropriately
and reset the counter |page| at the start of each child file.

%%%%%%%%%%%%%%%%%%%%%%%%%%%%%%%%%%%%%%%%%%%%%%%%%%%%%%%%%%%%%%%%%%%%%%%%%%%%%%%%
\subsection{Conditional Processing}
\label{sec:conditional}

The package provides a mechanism to compile different versions
of a document. To customise the versions further some conditional processing
can come in handy to distinguish which version is being compiled.
The package provides two macros to describe the compilation context:

%%%%%%%%%%%%%%%%%%%%%%%%%%%%%%%%%%%%%%%%
\DescribeMacro{\ifchilddoc}
The conditional |\ifchilddoc| distinguishes between the compilation of
child documents and the main document:
%
\begin{center}
|\ifchilddoc |\textit{child-code}| |[|\||else |\textit{main-code}]| \||fi|
\end{center}

%%%%%%%%%%%%%%%%%%%%%%%%%%%%%%%%%%%%%%%%
\DescribeMacro{\childdocname}
\DescribeMacro{\childdocjob}
The macro |\childdocname| contains the filename (without extension)
of the main or child file being processed.
Note that |\childdocjob| will always contain the name of the main file.

%%%%%%%%%%%%%%%%%%%%%%%%%%%%%%%%%%%%%%%%
\paragraph{Title Page.}

Conditional processing can be used to include a title or banner page
in the main document when proper precautions are taken.
Importantly, the code in the main file should ensure that the page counter
(as well as other status parameters which are stored in the |.aux| files)
takes the same value after the conditional processing.
Otherwise the page numbers may take divergent values
depending on which part is compiled.

For example, a title page could be declared by:
%
\begin{center}
\begin{tabular}{l}
|\ifchilddoc\||else|\\
|\addtocounter{page}{-1}|\\
\textit{code for title page}\\
|\newpage|\\
|\||fi|
\end{tabular}
\end{center}
%
A banner page for the child documents can be generated by:
%
\begin{center}
\begin{tabular}{l}
|\ifchilddoc|\\
|\addtocounter{page}{-1}|\\
\textit{code for banner page}\\
|\newpage|\\
|\||fi|
\end{tabular}
\end{center}
%
Here one could write a message such as:
\begin{center}
|This is the part \childdocname{} of \childdocjob{}.|
\end{center}

%%%%%%%%%%%%%%%%%%%%%%%%%%%%%%%%%%%%%%%%%%%%%%%%%%%%%%%%%%%%%%%%%%%%%%%%%%%%%%%%
\subsection{Flags}
\label{sec:flags}

The package makes it easy to generate different versions
of the main or child documents.
To this end compilation flags can be defined
and assigned different default values.
They will be particularly useful in conjunction
with the forwarding mechanism described in \secref{sec:forward}.

For example, it may be useful to have a flag |\version|
which can be set to |draft| or |final|.
The document source will contain some conditional code
depending on the value of |\version|.
Suppose further, the flag should default to |final| for the main file
and to |draft| for child files
which is a natural assignment for editing the document.
This is achieved by placing the following code
in the preamble of the main document
(below the |\childdocmain| directive):
%
\begin{center}
\begin{tabular}{l}
|\ifchilddoc|\\
|\providecommand{\version}{draft}|\\
|\||else|\\
|\providecommand{\version}{final}|\\
|\||fi|
\end{tabular}
\end{center}
%
The definition by |\providecommand| makes sure
that previous definitions are not overwritten.
Further statements |\providecommand{\version}{...}|
can thus be added before the above code to override it.

For the main file, one might add a line
(between |\childdocmain| and the above block)
%
\begin{center}
|%\ifchilddoc\||else\providecommand{\version}{draft}\||fi|
\end{center}
%
which can be uncommented to produce a draft version.
Likewise one can add a line to the very top of a child file
(above the |\childdocof{|\textit{main}|}| directive)
%
\begin{center}
|%\providecommand{\version}{final}|
\end{center}
%
which can be uncommented to produce the final version of this child document.

%%%%%%%%%%%%%%%%%%%%%%%%%%%%%%%%%%%%%%%%%%%%%%%%%%%%%%%%%%%%%%%%%%%%%%%%%%%%%%%%
\subsection{Forwarding}
\label{sec:forward}

Different versions of the main or child documents
using compilation flags as described in \secref{sec:flags}
can be (permanently) stored in different files
for convenient compilation, viewing and distribution.
To this end, the package defines a command
to pass on compilation to a different file:

%%%%%%%%%%%%%%%%%%%%%%%%%%%%%%%%%%%%%%%%
\DescribeMacro{\childdocforward}
The command |\childdocforward| redirects processing to
another source file:
%
\begin{center}
\begin{tabular}{l}
|\input{childdoc.def}|\\
|\childdocforward[|\textit{main}|]{|\textit{dest}|}|\\
\end{tabular}
\end{center}
%
The argument \textit{dest} is the destination file
(without extension).
It should be the main file or one of the child files.
Note that further \textsf{childdoc} directives
such as |\childdocof| and |\childdocforward|
in the indicated file will be processed in this form.
The optional argument \textit{main}
passes on directly to the main file \textit{main}
while pretending to compile the child \textit{dest}.
This form behaves as if \textit{dest}
issues |\childdocof{|\textit{main}|}| right away,
and no further \textsf{childdoc} directives will be processed.

%%%%%%%%%%%%%%%%%%%%%%%%%%%%%%%%%%%%%%%%
\DescribeMacro{\...prefix}
In the alternative form |\childdocforwardprefix|,
%
\begin{center}
\begin{tabular}{l}
|\input{childdoc.def}|\\
|\childdocforwardprefix[|\textit{main}|]{|\textit{prefix}|}{|\textit{dest}|}|
\end{tabular}
\end{center}
%
the destination file is determined by a pattern
depending on the current file:
To make this work, the current file must be called
`{\textit{prefix}\hspace{0.2em}\textit{suffix}}'
with \textit{prefix} matching precisely the argument.
Processing is then passed on to the file
`{\textit{dest}\hspace{0.2em}\textit{suffix}}'.
Surely, the same effect is achieved by
directly specifying the
argument `{\textit{dest}\hspace{0.2em}\textit{suffix}}'
in the first form.
However, that requires to set up a different file
for each child. With the alternative form of the command
all these files can have exactly the same content
which simplifies setting them up and maintaining them.

For example, the following file |draft.tex|
with a compilation flag |\version| as described in \secref{sec:flags}
compiles the main document as a draft:
%
\begin{center}
\begin{tabular}{l}
|\def\version{draft}|\\
|\input{childdoc.def}|\\
|\childdocforward{|\textit{main}|}|
\end{tabular}
\end{center}
%
Likewise, the following files |final|\textit{nn}|.tex|
compile the final version of the child document
|child|\textit{nn}|.tex|:
%
\begin{center}
\begin{tabular}{l}
|\def\version{final}|\\
|\input{childdoc.def}|\\
|\childdocforwardprefix{final}{child}|
\end{tabular}
\end{center}
%

Note that when several versions of a main file and/or of each child file
are to be generated, it may be convenient to set up a |Makefile| or
shell script to automatise the process.

%%%%%%%%%%%%%%%%%%%%%%%%%%%%%%%%%%%%%%%%%%%%%%%%%%%%%%%%%%%%%%%%%%%%%%%%%%%%%%%%
\subsection{Command Line Processing}
\label{sec:commandline}

The effect of redirection files can also be achieved by invoking
the \LaTeX{} compiler with a more elaborate command line.
Most conveniently this should be done as part
of a shell script or a |Makefile|.

When using \textsf{childdoc} in the main file, the following
command lines effectively perform a redirection
(note that depending on the shell being used,
backslashes may have to be doubled: `|\|' $\to$ `|\\|'):
%
\begin{center}
|... -jobname "|\textit{target}|" |\\|"|[\textit{flags}]%
|\input{childdoc.def}\childdocforward[|\textit{main}|]{|\textit{dest}|}"|
\end{center}
%
Here \textit{target} is the name of the output file,
\textit{main} is the name of the main file
and \textit{dest} is the name of the main or child file to be processed
(all filenames without extensions).
The optional argument \textit{main} can be omitted
if \textit{main} matches \textit{dest}.
Optionally, compilation \textit{flags} can be defined via |\def| commands.
This command line makes the \TeX{} engine believe
it is compiling the file \textit{target}
whose content is specified as the latter parameter.
The provided code then forwards the processing to
\textit{main} or \textit{dest} as described in \secref{sec:forward}.

%%%%%%%%%%%%%%%%%%%%%%%%%%%%%%%%%%%%%%%%%%%%%%%%%%%%%%%%%%%%%%%%%%%%%%%%%%%%%%%%
\subsection{Include by Input}
\label{sec:input}

Including child documents by |\include| has some restrictions by design.
Most notably, the content of a child document always occupies
its own set of pages; pages cannot be shared between child documents.
Usually, this behaviour makes perfect sense
because each child document contain an essential part of the document.
However, in some situations it may be desirable to compose
a document from a collection of parts
without having mandatory page breaks between then.
For this case, the package
provides a mechanism to include parts
by |\input| which can also be processed individually.
However, by construction this mechanism
requires manual handling of the content to be output.

%%%%%%%%%%%%%%%%%%%%%%%%%%%%%%%%%%%%%%%%
\DescribeMacro{\ifchilddocmanual}
The main file should be prepared as usual, see \secref{sec:include}.
However, the document body must make a distinction
between processing of an individual part and of the main document, e.g.:
%
\begin{center}
\begin{tabular}{l}
|\ifchilddocmanual|\\
|\input{\childdocname}|\\
|\||else|\\
\textit{document body with }|\input{|\textit{part}|}|\\
|\||fi|
\end{tabular}
\end{center}
%
The conditional |\ifchilddocmanual| is true whenever
a part to be included by |\input| is being compiled,
and the name of the part is stored in |\childdocname|.

%%%%%%%%%%%%%%%%%%%%%%%%%%%%%%%%%%%%%%%%
\DescribeMacro{\childdocby}
Each part to be included by |\input| should start with:
%
\begin{center}
\begin{tabular}{l}
|\input{childdoc.def}|\\
|\childdocby{|\textit{main}|}|\\
\end{tabular}
\end{center}
%
The directive |\childdocby| is similar to |\childdocof|
described in \secref{sec:include},
but the subsequent selection of content must be done manually.
To that end, both |\ifchilddoc| and |\ifchilddocmanual|
will be true upon processing of a part,
and the name of the part is stored in |\childdocname|.
Note that |\jobname| will be set to the filename of the current part
so that each part receives an individual |.aux| file
that does not interfere with the |.aux| file(s) of the main document.
This behaviour can be altered by the alternative form
|\childdocby[*]{|\textit{main}|}| (with a non-empty optional argument)
which uses the |.aux| file of the main document
by setting |\jobname| to \textit{main}.

%%%%%%%%%%%%%%%%%%%%%%%%%%%%%%%%%%%%%%%%%%%%%%%%%%%%%%%%%%%%%%%%%%%%%%%%%%%%%%%%
\subsection{Driver Development}
\label{sec:driver}

The \textsf{childdoc} mechanism can also be use for the development
of definition files such as \LaTeX{} styles or classes.
This case differs from the above setup with multiple parts
included by |\include| in that no |\includeonly| should be invoked.
This can be achieved by starting the include file
(before |\ProvidesPackage|) with:
%
\begin{center}
\begin{tabular}{l}
|\input{childdoc.def}|\\
|\childdocforward{|\textit{main}|}|\\
\end{tabular}
\end{center}
%
or alternatively with:
%
\begin{center}
\begin{tabular}{l}
|\input{childdoc.def}|\\
|\childdocby{|\textit{main}|}|\\
\end{tabular}
\end{center}
%
Both forms have slightly different effects as described above.
The main file is prepared as usual, see \secref{sec:include}.

%%%%%%%%%%%%%%%%%%%%%%%%%%%%%%%%%%%%%%%%%%%%%%%%%%%%%%%%%%%%%%%%%%%%%%%%%%%%%%%%
\subsection{Legacy Detection}
\label{sec:detection}

The directive |\childdocmain| in the main file can detect
whether the complete document or merely a child is to be compiled
even without using the directive |\childdocof|.
This method is deprecated because it is less robust
and there is no compelling reason to use it;
it is merely provided for backward compatibility
and it may be removed in future versions.

If the detection mechanism is to be used,
it is mandatory to correctly specify
the filename of the main file as the argument of |\childdocmain|:
%
\begin{center}
\begin{tabular}{l}
|\input{childdoc.def}|\\
|\childdocmain{|\textit{main}|}|\\
\end{tabular}
\end{center}
%
If |\jobname| does not match the argument \textit{main} of |\childdocmain|,
it is assumed that |\jobname| points to the child file to be compiled.
When using |\childdocmain| with the main file specified as argument,
it suffices to start a child file
with just |\input{|\textit{main}|}|
without loading of the package and using |\childdocof|.
If instead all processing is done
with the appropriate \textsf{childdoc} directives,
the argument of \textit{main} of |\childdocmain| can be empty.

An alternative version of the command line processing described
in \secref{sec:commandline} using the detection mechanism reads:
%
\begin{center}
|... -jobname "|\textit{target}|" "|[\textit{flags}]%
[|\def\jobname{|\textit{dest}|}|]|\input{|\textit{main}|}"|
\end{center}

%%%%%%%%%%%%%%%%%%%%%%%%%%%%%%%%%%%%%%%%%%%%%%%%%%%%%%%%%%%%%%%%%%%%%%%%%%%%%%%%
\subsection{Manual Code}
\label{sec:manual}

In case one cannot be certain whether the definitions file |childdoc.def|
is installed on the target \TeX{} distribution
and one prefers not to ship it,
it is conceivable to paste a few relevant commands into the sources.

To that end, drop all statements |\input{childdoc.def}|
and perform the replacements as outlined below.
Instead of |\childdocmain{|\textit{main}|}| add the following code
to the top of the main file:
%
\begin{center}
\begin{tabular}{l}
|\||ifdefined\childdocname\endinput\||fi\newif\ifchilddoc|\\
|\edef\childdocname{\scantokens\expandafter{\jobname\noexpand}}|\\
|\def\childdocmain{|\textit{main}|}\||ifx\childdocmain\childdocname\||else|\\
|\childdoctrue\includeonly{\childdocname}\let\jobname\childdocmain\||fi|\\
\end{tabular}
\end{center}
%
Instead of |\childdocof{|\textit{main}|}| just include the main file
at the top of each child file:
%
\begin{center}
|\input{|\textit{main}|}|
\end{center}
%
A simple redirection |\childdocforward{|\textit{dest}|}| is achieved by:
%
\begin{center}
|\def\jobname{|\textit{dest}|}\input{\jobname}|
\end{center}
%
The redirection with prefix
|\childdocforwardprefix[|\textit{prefix}|]{|\textit{dest}|}|
is accomplished by:
%
\begin{center}
\begin{tabular}{l}
|{\edef\jobname{\scantokens\expandafter{\jobname\noexpand}}|\\
|\def\redirectjob |\textit{prefix}|#1~~~{\gdef\jobname{|\textit{dest}|#1}}|\\
|\expandafter\redirectjob\jobname~~~}\input{\jobname}|
\end{tabular}
\end{center}

In an alternative approach,
child documents can be compiled by a specific command line
without additional code or specific definitions:
%
\begin{center}
|... -jobname "|\textit{target}|" "|[\textit{flags}]%
|\includeonly{|\textit{dest}|}\input{|\textit{main}|}"|
\end{center}
%

%%%%%%%%%%%%%%%%%%%%%%%%%%%%%%%%%%%%%%%%%%%%%%%%%%%%%%%%%%%%%%%%%%%%%%%%%%%%%%%%
%%%%%%%%%%%%%%%%%%%%%%%%%%%%%%%%%%%%%%%%%%%%%%%%%%%%%%%%%%%%%%%%%%%%%%%%%%%%%%%%
\section{Information}

%%%%%%%%%%%%%%%%%%%%%%%%%%%%%%%%%%%%%%%%%%%%%%%%%%%%%%%%%%%%%%%%%%%%%%%%%%%%%%%%
\subsection{Copyright}

Copyright \copyright{} 2017--2018 Niklas Beisert

This work may be distributed and/or modified under the
conditions of the \LaTeX{} Project Public License, either version 1.3
of this license or (at your option) any later version.
The latest version of this license is in
  \url{http://www.latex-project.org/lppl.txt}
and version 1.3 or later is part of all distributions of \LaTeX{}
version 2005/12/01 or later.

This work has the LPPL maintenance status `maintained'.

The Current Maintainer of this work is Niklas Beisert.

This work consists of the files |README.txt|, |childdoc.ins| and |childdoc.dtx|
as well as the derived files |childdoc.def|, |cdocsamp.tex|
with |cdocsch1.tex|, |cdocsch2.tex|, |cdocspt3.tex|, |cdocspt4.tex|,
|cdocsdrf.tex|, |cdocsfn1.tex|, |cdocsfn2.tex|
as well as |childdoc.pdf|.

%%%%%%%%%%%%%%%%%%%%%%%%%%%%%%%%%%%%%%%%%%%%%%%%%%%%%%%%%%%%%%%%%%%%%%%%%%%%%%%%
\subsection{Files and Installation}

The package consists of the files:
%
\begin{center}
\begin{tabular}{ll}
    |README.txt|   & readme file \\
    |childdoc.ins| & installation file \\
    |childdoc.dtx| & source file \\
    |childdoc.def| & definition file \\
    |cdocsamp.tex| & sample main file \\
    |cdocsch1.tex| & sample include file \\
    |cdocsch2.tex| & sample include file \\
    |cdocspt3.tex| & sample part file \\
    |cdocspt4.tex| & sample part file \\
    |cdocsdrf.tex| & sample redirection file \\
    |cdocsfn1.tex| & sample redirection file \\
    |cdocsfn2.tex| & sample redirection file \\
    |childdoc.pdf| & manual
\end{tabular}
\end{center}
%
The distribution consists of the files
|README.txt|, |childdoc.ins| and |childdoc.dtx|.
%
\begin{itemize}
\item
Run (pdf)\LaTeX{} on |childdoc.dtx|
to compile the manual |childdoc.pdf| (this file).
\item
Run \LaTeX{} on |childdoc.ins| to create the definitions file |childdoc.def|
and the sample |cdocsamp.tex| with include files
|cdocsch1.tex|, |cdocsch2.tex|, |cdocspt3.tex|, |cdocspt4.tex|,
|cdocsdrf.tex|, |cdocsfn1.tex|, |cdocsfn2.tex|.
Then copy the file |childdoc.def| to an appropriate directory of your \LaTeX{}
distribution, e.g.\ \textit{texmf-root}|/tex/latex/childdoc|.
\end{itemize}

%%%%%%%%%%%%%%%%%%%%%%%%%%%%%%%%%%%%%%%%%%%%%%%%%%%%%%%%%%%%%%%%%%%%%%%%%%%%%%%%
\subsection{Related CTAN Packages}

There are several other packages which offer a similar functionality:
%
\begin{itemize}
\item
The packages
\href{http://ctan.org/pkg/docmute}{\textsf{docmute}},
\href{http://ctan.org/pkg/includex}{\textsf{includex}} and
\href{http://ctan.org/pkg/standalone}{\textsf{standalone}}
provide commands to include only the document body of
a child file thus allowing both files to be compiled individually.
\item
The packages \href{http://ctan.org/pkg/subdocs}{\textsf{subdocs}}
and \href{http://ctan.org/pkg/subfiles}{\textsf{subfiles}}
provide structures in which the main and child documents can be
encapsulated and allowing them to be compiled individually.
The inclusion mechanism is different from the conventional |\include|.
\item
The package \href{http://ctan.org/pkg/combine}{\textsf{combine}}
is an elaborate solution to combine several documents into one.
\end{itemize}
%
See also the CTAN topic \href{http://ctan.org/topic/subdocs}{\textsf{subdocs}}
for further related packages.
The present package differs from the above solutions in that
a document structure constructed with the conventional |\include| mechanism
just needs two extra commands at the top of every file
such that all constituent files can be compiled individually.

%%%%%%%%%%%%%%%%%%%%%%%%%%%%%%%%%%%%%%%%%%%%%%%%%%%%%%%%%%%%%%%%%%%%%%%%%%%%%%%%
%\subsection{Feature Suggestions}
%
%The following is a list of features which may be useful for future
%versions of this package:
%%
%\begin{itemize}
%\item
%\ldots
%\end{itemize}

%%%%%%%%%%%%%%%%%%%%%%%%%%%%%%%%%%%%%%%%%%%%%%%%%%%%%%%%%%%%%%%%%%%%%%%%%%%%%%%%
\subsection{Revision History}

%%%%%%%%%%%%%%%%%%%%%%%%%%%%%%%%%%%%%%%%
\paragraph{v2.0:} 2018/12/30

\begin{itemize}
\item
immediate forward processing
\item
added |\childdocby| mechanism
\item
manual restructured
\end{itemize}

%%%%%%%%%%%%%%%%%%%%%%%%%%%%%%%%%%%%%%%%
\paragraph{v1.6:} 2018/01/17

\begin{itemize}
\item
application for development of include files
\item
corrections to manual
\end{itemize}

%%%%%%%%%%%%%%%%%%%%%%%%%%%%%%%%%%%%%%%%
\paragraph{v1.5:} 2017/05/21

\begin{itemize}
\item
more complete structuring introduced
\item
|\childdocof| introduced
\item
|\childdoc| renamed to |\childdocmain|
\item
|\childredirect| renamed to |\childdocforward| and |\childdocforwardprefix|
and functionality expanded
\end{itemize}

%%%%%%%%%%%%%%%%%%%%%%%%%%%%%%%%%%%%%%%%
\paragraph{v1.0:} 2017/04/27

\begin{itemize}
\item
manual and install package
\item
first version published on CTAN
\end{itemize}

%%%%%%%%%%%%%%%%%%%%%%%%%%%%%%%%%%%%%%%%
\paragraph{v0.6:} 2017/04/26

\begin{itemize}
\item
redirection mechanism added
\end{itemize}

%%%%%%%%%%%%%%%%%%%%%%%%%%%%%%%%%%%%%%%%
\paragraph{v0.5:} 2017/04/26

\begin{itemize}
\item
functionality in definition file
\end{itemize}


%%%%%%%%%%%%%%%%%%%%%%%%%%%%%%%%%%%%%%%%%%%%%%%%%%%%%%%%%%%%%%%%%%%%%%%%%%%%%%%%
%%%%%%%%%%%%%%%%%%%%%%%%%%%%%%%%%%%%%%%%%%%%%%%%%%%%%%%%%%%%%%%%%%%%%%%%%%%%%%%%
%%%%%%%%%%%%%%%%%%%%%%%%%%%%%%%%%%%%%%%%%%%%%%%%%%%%%%%%%%%%%%%%%%%%%%%%%%%%%%%%
\appendix

\settowidth\MacroIndent{\rmfamily\scriptsize 000\ }

 \DocInput{childdoc.dtx}

\end{document}
%</driver>
% \fi
%
% %%%%%%%%%%%%%%%%%%%%%%%%%%%%%%%%%%%%%%%%%%%%%%%%%%%%%%%%%%%%%%%%%%%%%%%%%%%%%%
% %%%%%%%%%%%%%%%%%%%%%%%%%%%%%%%%%%%%%%%%%%%%%%%%%%%%%%%%%%%%%%%%%%%%%%%%%%%%%%
% \section{Sample}
%\iffalse
%<*samplemain>
%\fi
%
% The following presents a sample document
% with two chapters, two parts, a title page,
% a compile flag as well as three forwarding files to set the flag.
% It consists of eight |.tex| files:
% \begin{center}
% \begin{tabular}{ll}
% |cdocsamp.tex|&main file\\
% |cdocsch1.tex|&include file for chapter 1\\
% |cdocsch2.tex|&include file for chapter 2\\
% |cdocspt3.tex|&include file for part 3\\
% |cdocspt4.tex|&include file for part 4\\
% |cdocsdrf.tex|&forwarding file for main file in draft mode\\
% |cdocsfi1.tex|&forwarding file for final version of chapter 1\\
% |cdocsfi2.tex|&forwarding file for final version of chapter 2\\
% \end{tabular}
% \end{center}
% Each of the eight files can be compiled directly by the \LaTeX{} compiler.
%
% %%%%%%%%%%%%%%%%%%%%%%%%%%%%%%%%%%%%%%
% \paragraph{Main File.}
%
% The main file is called |cdocsamp.tex|.
%
% Load the \textsf{childdoc} definitions and
% declare the filename for the main document:
%    \begin{macrocode}
\input{childdoc.def}
\childdocmain{}
%    \end{macrocode}

% Optional override for |\version| flag:
%    \begin{macrocode}
%%\ifchilddoc\else\providecommand{\version}{draft}\fi
%    \end{macrocode}

% Define the default values for the |\version| flag
% (|final| for the main file and |draft| for childs):
%    \begin{macrocode}
\ifchilddoc
\providecommand{\version}{draft}
\else
\providecommand{\version}{final}
\fi
%    \end{macrocode}

% Load the standard document class:
%    \begin{macrocode}
\documentclass[12pt]{article}
%    \end{macrocode}

% Start the document body:
%    \begin{macrocode}
\begin{document}
%    \end{macrocode}

% Declare a title page.
% Print title, part of document being processed and version flag:
%    \begin{macrocode}
\addtocounter{page}{-1}
\begin{center}
{\LARGE\bfseries{}childdoc example\par}
\vspace{1cm}
\ifchilddoc
\ifchilddocmanual part\else chapter\fi:
`\childdocname' of `\childdocjob'\par
\else
main document: `\childdocjob'\par
\fi
version: \version\par
\end{center}
\newpage
%    \end{macrocode}

% Manually include selected file,
% otherwise process as usual:
%    \begin{macrocode}
\ifchilddocmanual
\section*{part `\childdocname'}
\input{\childdocname}
\else
%    \end{macrocode}

% Include the two chapters:
%    \begin{macrocode}
\include{cdocsch1}
\include{cdocsch2}
%    \end{macrocode}

% Include the two parts unless only chapters should be displayed:
%    \begin{macrocode}
\ifchilddoc\else
\section{part three}
\input{cdocspt3}
\section{part four}
\input{cdocspt4}
\fi
%    \end{macrocode}

% Process as usual until here:
%    \begin{macrocode}
\fi
%    \end{macrocode}

% End of document body:
%    \begin{macrocode}
\end{document}
%    \end{macrocode}
%\iffalse
%</samplemain>
%\fi
%
% %%%%%%%%%%%%%%%%%%%%%%%%%%%%%%%%%%%%%%
% \paragraph{Chapter Include Files.}
%
% The include files are called |cdocsch1.tex| and |cdocsch2.tex|.
%
%\iffalse
%<*samplechap1|samplechap2>
%\fi

% Optional override for |\version| flag:
%    \begin{macrocode}
%%\providecommand{\version}{final}
%    \end{macrocode}

% Include the main document:
%    \begin{macrocode}
\input{childdoc.def}
\childdocof{cdocsamp}
%    \end{macrocode}

%\iffalse
%</samplechap1|samplechap2>
%\fi
%
%\iffalse
%<*samplechap1>
%\fi
% Some text for chapter 1:
%    \begin{macrocode}
\section{one}
some text in chapter one
%    \end{macrocode}

%\iffalse
%</samplechap1>
%\fi
% Some text for chapter 2:
%\iffalse
%<*samplechap2>
%\fi
%    \begin{macrocode}
\section{two}
more text in chapter two
%    \end{macrocode}

%\iffalse
%</samplechap2>
%\fi
%
% %%%%%%%%%%%%%%%%%%%%%%%%%%%%%%%%%%%%%%
% \paragraph{Part Include Files.}
%
% The include files are called |cdocspt3.tex| and |cdocspt4.tex|.
%
%\iffalse
%<*samplepart3|samplepart4>
%\fi

% Optional override for |\version| flag:
%    \begin{macrocode}
%%\providecommand{\version}{final}
%    \end{macrocode}

% Include the main document:
%    \begin{macrocode}
\input{childdoc.def}
\childdocby{cdocsamp}
%    \end{macrocode}

%\iffalse
%</samplepart3|samplepart4>
%\fi
%
%\iffalse
%<*samplepart3>
%\fi
% Some text for part 3:
%    \begin{macrocode}
some text in part three
%    \end{macrocode}

%\iffalse
%</samplepart3>
%\fi
% Some text for part 4:
%\iffalse
%<*samplepart4>
%\fi
%    \begin{macrocode}
more text in part four
%    \end{macrocode}

%\iffalse
%</samplepart4>
%\fi
%
% %%%%%%%%%%%%%%%%%%%%%%%%%%%%%%%%%%%%%%
% \paragraph{Forwarding for a Complete Draft.}
%
% The following forwarding file |cdocsdrf.tex|
% compiles the main document in draft mode:
%\iffalse
%<*sampledraft>
%\fi
%    \begin{macrocode}
\def\version{draft}
\input{childdoc.def}
\childdocforward{cdocsamp}
%    \end{macrocode}

%\iffalse
%</sampledraft>
%\fi
%
% %%%%%%%%%%%%%%%%%%%%%%%%%%%%%%%%%%%%%%
% \paragraph{Forwarding for Final Version of the Chapters.}
%
% The following forwarding files |cdocsfn1.tex| and |cdocsfn2.tex|
% (with identical content)
% compile the final versions of the child documents
% |cdocsch1.tex| and |cdocsch2.tex|, respectively:
%\iffalse
%<*samplefinal>
%\fi
%    \begin{macrocode}
\def\version{final}
\input{childdoc.def}
\childdocforwardprefix[cdocsamp]{cdocsfn}{cdocsch}
%    \end{macrocode}

%\iffalse
%</samplefinal>
%\fi
%
% %%%%%%%%%%%%%%%%%%%%%%%%%%%%%%%%%%%%%%
% \paragraph{Command Line Processing.}
%
% The following three command lines generate the output files
% |cdocscld|, |cdocscl1| and |cdocscl2|
% which should be identical to
% |cdocsdrf|, |cdocsch1| and |cdocsfn2|, respectively:
% \begin{center}
% \begin{tabular}{l}
% |latex -jobname cdocscld \|\\
% |  "\def\version{draft}\input{childdoc.def}\childdocforward{cdocsamp}"|\\
% |latex -jobname cdocscl1 \|\\
% |  "\input{childdoc.def}\childdocforward[cdocsamp]{cdocsch1}"|\\
% |latex -jobname cdocscl2 \|\\
% |  "\def\version{final}\input{childdoc.def}\childdocforward{cdocsch2}"|
% \end{tabular}
% \end{center}
% Note that the trailing backslash on each first line
% merely continues the input to the second line
% (for convenient cut ant paste).
% Furthermore, the command |latex| can be replaced by any
% of its alternative versions such as |pdflatex|.
%
% %%%%%%%%%%%%%%%%%%%%%%%%%%%%%%%%%%%%%%%%%%%%%%%%%%%%%%%%%%%%%%%%%%%%%%%%%%%%%%
% %%%%%%%%%%%%%%%%%%%%%%%%%%%%%%%%%%%%%%%%%%%%%%%%%%%%%%%%%%%%%%%%%%%%%%%%%%%%%%
% \section{Implementation}
%\iffalse
%<*package>
%\fi
%
% This section describes the definitions file |childdoc.def|.

% The definitions cannot be loaded using |\usepackage| or |\RequirePackage|
% which has a mechanism to prevent loading a style file more than once.
% When loading the definitions by means of |\input|
% multiple instances have to be prevented manually:
%\iffalse
%This code needs to be before the `\ProvidesFile' directive
%which is defined at the beginning of this file.
%Therefore it is also placed there and commented out here.
%</package>
%<*discard>
%\fi
%    \begin{macrocode}
\ifdefined\childdocmain\endinput\fi
%    \end{macrocode}
%\iffalse
%</discard>
%<*package>
%\fi
%
% \macro{\ifchilddoc}
% \macro{\ifchilddocmanual}
% The conditional |\ifchilddoc| tells whether a
% child (true) or main (false) document is being compiled.
% The conditional |\ifchilddocmanual| tells whether
% the |\includeonly| mechanism is used (false) or
% the selection of child files must be performed manually (true).
% The definitions initialise to false:
%    \begin{macrocode}
\newif\ifchilddoc
\newif\ifchilddocmanual
%    \end{macrocode}

% \macro{\childdocname}
% \macro{\childdocjob}
% The macro |\childdocname| stores the name of the main document
% to be compiled. The macro |\childdocjob| stores the name of
% the document on which the \LaTeX{} compiler was originally invoked.
% The content of |\jobname| cannot be compared
% to filenames specified in the source due to different catcodes.
% The following code rescans |\jobname|, stores the result
% in |\childdocname| and saves a copy in |\childdocjob|:
%    \begin{macrocode}
\edef\childdocname{\scantokens\expandafter{\jobname\noexpand}}
\let\childdocjob\childdocname
%    \end{macrocode}

% \macro{\childdocdisable}
% The macro |\childdocdisable| prevents the main file
% from being processed more than once.
% At this stage, the main document command |\childdocmain|
% is assumed to be called once again where it should do nothing.
% Any subsequent call to it should prevent
% a secondary processing of the main document
% It overwrites the forwarding commands
% |\childdocof| and |\childdocforward|
% with empty macros to prevent further inclusions of the main document:
%    \begin{macrocode}
\newcommand{\childdocdisable}
{
  \renewcommand{\childdocmain}[1]{\renewcommand{\childdocmain}[1]{\endinput}}
  \renewcommand{\childdocof}[1]{}
  \renewcommand{\childdocby}[2][]{}
  \renewcommand{\childdocforward}[2][]{}
  \renewcommand{\childdocdisable}{}
}
%    \end{macrocode}

% \macro{\childdocmain}
% The macro |\childdocmain| is to be called at the top of the main file
% with nothing or the main filename (without extension) as argument.
% First, it breaks loops.
% If the argument is not empty and does not match |\childdocname|
% (which is set by the first inclusion of |childdoc.def|),
% |\ifchilddoc| is set to true, |\includeonly| is applied to the child file
% and |\jobname| is set to the main file
% (for proper handling of |.aux| files):
%    \begin{macrocode}
\newcommand{\childdocmain}[1]
{
  \childdocdisable\childdocmain{}
  \if?#1?\else
    \begingroup
      \def\childdoctmp{#1}
      \ifx\childdoctmp\childdocname
        \def\childdoctmp{}
      \else
        \def\childdoctmp
        {
          \childdoctrue
          \includeonly{\childdocname}
          \def\childdocjob{#1}
          \def\jobname{#1}
        }
      \fi
      \expandafter
    \endgroup
    \childdoctmp
  \fi
}
%    \end{macrocode}

% \macro{\childdocof}
% The command |\childdocof| redirects
% compilation to the main file |#1|.
%    \begin{macrocode}
\newcommand{\childdocof}[1]
{
  \childdocdisable
  \childdoctrue
  \includeonly{\childdocname}
  \def\jobname{#1}
  \def\childdocjob{#1}
  \input{#1}
}
%    \end{macrocode}

% \macro{\childdocby}
% The command |\childdocby| ....
%    \begin{macrocode}
\newcommand{\childdocby}[2][]
{
  \childdocdisable
  \childdoctrue
  \childdocmanualtrue
  \if?#1?\else
    \def\jobname{#2}
  \fi
  \def\childdocjob{#2}
  \input{#2}
  \endinput
}
%    \end{macrocode}

% \macro{\childdocforward}
% The command |\childdocforward| redirects
% compilation to the main file or
% (if the optional argument is given) a child file.
% Parameters are set as if the main file
% or a child file starting with |\childdocof| was compiled.
% Then compilation is handed over to the main file:
%    \begin{macrocode}
\newcommand{\childdocforward}[2][]
{
  \begingroup
    \if?#1?
      \def\childdoctmp
      {
        \def\childdocname{#2}
        \def\childdocjob{#2}
        \def\jobname{#2}
        \input{#2}
        \endinput
      }
    \else
      \def\childdoctmp
      {
        \childdocdisable
        \def\childdocname{#2}
        \childdoctrue
        \includeonly{#2}
        \def\childdocjob{#1}
        \def\jobname{#1}
        \input{#1}
        \endinput
      }
    \fi
    \expandafter
  \endgroup
  \childdoctmp
}
%    \end{macrocode}

% \macro{\childdocforwardprefix}
% The command |\childdocforwardprefix| redirects
% compilation to the main or a child file by means of a pattern.
% The prefix |#1| in the current filename is replaced by |#2|
% and the suffix of the current filename is kept
% (it is assumed that the filename does not contain the substring `|~~~|'
% which is used as a delimiter).
% Compilation is handed over to the new file by |\childdocforward|:
%    \begin{macrocode}
\newcommand{\childdocforwardprefix}[3][]
{
  \begingroup
    \def\childdocextract #2##1~~~{\def\childdoctmp{\childdocforward[#1]{#3##1}}}
    \expandafter\childdocextract\childdocname~~~
    \expandafter
  \endgroup
  \childdoctmp
}
%    \end{macrocode}

% \macro{\childdoc}
% The deprecated macro |\childdoc| is a legacy version of |\childdocmain|:
%    \begin{macrocode}
\newcommand{\childdoc}{\childdocmain}
%    \end{macrocode}

% \macro{\childdocredirect}
% The deprecated macro |\childdocredirect| is a legacy version
% of |\childdocforward| and |\childdocforwardprefix|:
%    \begin{macrocode}
\newcommand{\childdocredirect}[2][]
{
  \begingroup
    \if?#1?
      \def\childdoctmp{\childdocforward{#2}}
    \else
      \def\childdoctmp{\childdocforwardprefix{#1}{#2}}
    \fi
    \expandafter
  \endgroup
  \childdoctmp
}
%    \end{macrocode}

%\iffalse
%</package>
%\fi
%
\endinput
|\\
|\childdocforward[|\textit{main}|]{|\textit{dest}|}|\\
\end{tabular}
\end{center}
%
The argument \textit{dest} is the destination file
(without extension).
It should be the main file or one of the child files.
Note that further \textsf{childdoc} directives
such as |\childdocof| and |\childdocforward|
in the indicated file will be processed in this form.
The optional argument \textit{main}
passes on directly to the main file \textit{main}
while pretending to compile the child \textit{dest}.
This form behaves as if \textit{dest}
issues |\childdocof{|\textit{main}|}| right away,
and no further \textsf{childdoc} directives will be processed.

%%%%%%%%%%%%%%%%%%%%%%%%%%%%%%%%%%%%%%%%
\DescribeMacro{\...prefix}
In the alternative form |\childdocforwardprefix|,
%
\begin{center}
\begin{tabular}{l}
|% \iffalse
%
% childdoc.dtx Copyright (C) 2017-2018 Niklas Beisert
%
% This work may be distributed and/or modified under the
% conditions of the LaTeX Project Public License, either version 1.3
% of this license or (at your option) any later version.
% The latest version of this license is in
%   http://www.latex-project.org/lppl.txt
% and version 1.3 or later is part of all distributions of LaTeX
% version 2005/12/01 or later.
%
% This work has the LPPL maintenance status `maintained'.
%
% The Current Maintainer of this work is Niklas Beisert.
%
% This work consists of the files childdoc.dtx and childdoc.ins
% and the derived files childdoc.def and cdocsamp.tex with
% cdocsch1.tex, cdocsch2.tex, cdocsdrf.tex, cdocsfn1.tex, cdocsfn2.tex.
%
%<package>\ifdefined\childdocmain\endinput\fi
%<package>\ProvidesFile{childdoc.def}[2018/12/30 v2.0 child document driver]
%<samplemain>\ProvidesFile{cdocsamp.tex}[2018/12/30 v2.0 sample for childdoc]
%<*driver>
%\ProvidesFile{childdoc.drv}[2018/12/30 v2.0 childdoc reference manual file]
\PassOptionsToClass{10pt,a4paper}{article}
\documentclass{ltxdoc}

\usepackage[margin=35mm]{geometry}
\usepackage{hyperref}
\usepackage{hyperxmp}
\usepackage[usenames]{color}

\hypersetup{colorlinks=true}
\hypersetup{pdfstartview=FitH}
\hypersetup{pdfpagemode=UseNone}
\hypersetup{pdfsource={}}
\hypersetup{pdflang={en-UK}}
\hypersetup{pdfcopyright={Copyright 2017-2018 Niklas Beisert.
  This work may be distributed and/or modified under the
  conditions of the LaTeX Project Public License, either version 1.3
  of this license or (at your option) any later version.}}
\hypersetup{pdflicenseurl={http://www.latex-project.org/lppl.txt}}
\hypersetup{pdfcontactaddress={ETH Zurich, ITP, HIT K,
  Wolfgang-Pauli-Strasse 27}}
\hypersetup{pdfcontactpostcode={8093}}
\hypersetup{pdfcontactcity={Zurich}}
\hypersetup{pdfcontactcountry={Switzerland}}
\hypersetup{pdfcontactemail={nbeisert@itp.phys.ethz.ch}}
\hypersetup{pdfcontacturl={http://people.phys.ethz.ch/\xmptilde nbeisert/}}

\newcommand{\secref}[1]{\hyperref[#1]{section \ref*{#1}}}

\parskip1ex
\parindent0pt
\let\olditemize\itemize
\def\itemize{\olditemize\parskip0pt}

\begin{document}

\title{The \textsf{childdoc} Package}
\hypersetup{pdftitle={The childdoc Package}}
\author{Niklas Beisert\\[2ex]
  Institut f\"ur Theoretische Physik\\
  Eidgen\"ossische Technische Hochschule Z\"urich\\
  Wolfgang-Pauli-Strasse 27, 8093 Z\"urich, Switzerland\\[1ex]
  \href{mailto:nbeisert@itp.phys.ethz.ch}
  {\texttt{nbeisert@itp.phys.ethz.ch}}}
\hypersetup{pdfauthor={Niklas Beisert}}
\hypersetup{pdfsubject={Manual for the LaTeX2e Package childdoc}}
\date{30 December 2018, \textsf{v2.0}}
\maketitle

\begin{abstract}\noindent
\textsf{childdoc} is a \LaTeXe{} package
that enables the direct compilation
of document sections included by |\include|
to individual files.
\end{abstract}

\begingroup
\parskip0ex
\tableofcontents
\endgroup

%%%%%%%%%%%%%%%%%%%%%%%%%%%%%%%%%%%%%%%%%%%%%%%%%%%%%%%%%%%%%%%%%%%%%%%%%%%%%%%%
%%%%%%%%%%%%%%%%%%%%%%%%%%%%%%%%%%%%%%%%%%%%%%%%%%%%%%%%%%%%%%%%%%%%%%%%%%%%%%%%
\section{Introduction}

\LaTeX{} provides a mechanism to structure a large document (such as a book)
into a main file and several child files (containing the chapters)
using the |\include| command.
This mechanism is beneficial for documents
which span hundreds of pages in order to
make the source file(s) more manageable.
Moreover, compilation can be restricted to
selected child files by means of the |\includeonly| command.
The latter feature can be used to reduce the compilation time while editing
(this was significantly more useful in the earlier days of \LaTeX{})
or to generate a smaller document which is easier to navigate.
Another application of |\includeonly| is to generate
documents consisting of selected parts of the complete document.

However, there are a few drawbacks of the plain |\include| mechanism:
\begin{itemize}
\item
The child files cannot be compiled on their own,
they can only be compiled via the main file.
A naive editing environment
(such as a text editor with an option
to have the current file processed by \LaTeX)
may require one to switch to the main file before compiling;
attempting to compile the child file produces errors.
\item
The main file must be modified (each time)
to adjust the |\includeonly| command
to the present needs. This easily leaves the main file in a messy state.
\item
The generated document will always carry the filename
of the main document. This is inconvenient if
several child files are to be compiled and
to be kept for distribution.
\end{itemize}

The present package provides a simple interface
to make child files individually compilable by \LaTeX{}.
Compiling a child file then has the same effect as compiling
the main file with an |\includeonly| command
to select the appropriate child.
Moreover the generated document will carry the name of the child
rather than the main file.
This resolves all three above issues.

This feature is meant to make the editing of books,
thesis documents and lecture notes somewhat more convenient.
However, the package can also be used efficiently for
composing a series of documents (such as exercise sheets)
which are typically distributed individually.
It then assists the author in generating the individual documents
(potentially in different versions)
as well as a document containing the collected series.
Another application is in developing style files
or other kinds of included material
where compilation of the style file could redirect
to a sample or test file.

%%%%%%%%%%%%%%%%%%%%%%%%%%%%%%%%%%%%%%%%%%%%%%%%%%%%%%%%%%%%%%%%%%%%%%%%%%%%%%%%
%%%%%%%%%%%%%%%%%%%%%%%%%%%%%%%%%%%%%%%%%%%%%%%%%%%%%%%%%%%%%%%%%%%%%%%%%%%%%%%%
\section{Usage}

First of all, the package \textsf{childdoc} is \emph{not} a standard
\LaTeXe{} |.sty| style file! Therefore it needs to be invoked in
a non-standard way.

%%%%%%%%%%%%%%%%%%%%%%%%%%%%%%%%%%%%%%%%%%%%%%%%%%%%%%%%%%%%%%%%%%%%%%%%%%%%%%%%
\subsection{Included Files}
\label{sec:include}

%%%%%%%%%%%%%%%%%%%%%%%%%%%%%%%%%%%%%%%%
\DescribeMacro{\childdocmain}
To use the package, add the commands
\begin{center}
\begin{tabular}{l}
|\input{childdoc.def}|\\
|\childdocmain{}|\\
\end{tabular}
\end{center}
at the very top of the main \LaTeX{} file,
in particular \emph{before} the |\documentclass| statement!
The argument of |\childdocmain| should be left empty
(but it must be present).

%%%%%%%%%%%%%%%%%%%%%%%%%%%%%%%%%%%%%%%%
\DescribeMacro{\childdocof}
Furthermore, add the commands
\begin{center}
\begin{tabular}{l}
|\input{childdoc.def}|\\
|\childdocof{|\textit{main}|}|\\
\end{tabular}
\end{center}
at the top of every child file \textit{child}
which is included by |\include{|\textit{child}|}|
from within the main file
(or at least for those files to be compiled individually).
The argument \textit{main} must be the filename of the main file.

There are a couple of
considerations in setting up the main and child documents:

%%%%%%%%%%%%%%%%%%%%%%%%%%%%%%%%%%%%%%%%
\paragraph{Restrictions.}

Please note the following restrictions:
\begin{itemize}
\item
|\childdocmain| must be called with one argument \textit{main}
to ensure compatibility with earlier version of the package.
It must either be empty (|\childdocmain{}|)
or precisely match the filename of the main file in which it is specified.
See \secref{sec:detection} for further information.
\item
The filename \textit{main} must be specified without the |.tex| extension.
\item
The filename \textit{main} is case sensitive
(even in case-insensitive file systems)
due to internal string comparison.
\item
The argument \textit{main} should be fully expanded, it cannot be a macro.
\item
Subdirectories and special characters should be avoided in filenames.
\item
The command |\childdocmain{|\textit{main}|}| must be followed by a whitespace.
It should not be followed immediately by another command
or by a comment mark `|%|'.
This is because the \TeX{} parser reads the token immediately following
the argument of |\childdocmain| and puts it
at the beginning of every child section;
however, a white\-space is ignored.
\end{itemize}

%%%%%%%%%%%%%%%%%%%%%%%%%%%%%%%%%%%%%%%%
\paragraph{Content of Main File.}

It is advisable to place all content in the child files included by |\include|.
Any output contained in the main file will appear in all child documents
unless suppressed manually;
it cannot be suppressed automatically by the |\includeonly| directive
and thus should normally be avoided.
A method to include some content in the main file
by means of conditional processing is described in \secref{sec:conditional}.

%%%%%%%%%%%%%%%%%%%%%%%%%%%%%%%%%%%%%%%%
\paragraph{Page Numbering.}

When only a part of the document is compiled,
the appropriate numbering of pages
(as well as other status parameters)
is determined from the |.aux| files.
The latter contain information from previous passes.
However this information needs to propagate through
all intermediate child documents.
Therefore the page numbering in child documents may well
be inconsistent until the complete document is compiled at least once.

A useful (if unconventional) way to always ensure a consistent
page numbering is to restart the numbering in each child document
and denote the pages by `\textit{child}|.|\textit{page}'
where \textit{child} represents the chapter/section number of the child file.
This can be achieved by the command
|\numberwithin{page}{|\textit{child}|}|
of the \textsf{amsmath} package
where \textit{child} can be |chapter| or |section|
depending on the chosen structuring.
Alternatively, one can modify the macro |\thepage| appropriately
and reset the counter |page| at the start of each child file.

%%%%%%%%%%%%%%%%%%%%%%%%%%%%%%%%%%%%%%%%%%%%%%%%%%%%%%%%%%%%%%%%%%%%%%%%%%%%%%%%
\subsection{Conditional Processing}
\label{sec:conditional}

The package provides a mechanism to compile different versions
of a document. To customise the versions further some conditional processing
can come in handy to distinguish which version is being compiled.
The package provides two macros to describe the compilation context:

%%%%%%%%%%%%%%%%%%%%%%%%%%%%%%%%%%%%%%%%
\DescribeMacro{\ifchilddoc}
The conditional |\ifchilddoc| distinguishes between the compilation of
child documents and the main document:
%
\begin{center}
|\ifchilddoc |\textit{child-code}| |[|\||else |\textit{main-code}]| \||fi|
\end{center}

%%%%%%%%%%%%%%%%%%%%%%%%%%%%%%%%%%%%%%%%
\DescribeMacro{\childdocname}
\DescribeMacro{\childdocjob}
The macro |\childdocname| contains the filename (without extension)
of the main or child file being processed.
Note that |\childdocjob| will always contain the name of the main file.

%%%%%%%%%%%%%%%%%%%%%%%%%%%%%%%%%%%%%%%%
\paragraph{Title Page.}

Conditional processing can be used to include a title or banner page
in the main document when proper precautions are taken.
Importantly, the code in the main file should ensure that the page counter
(as well as other status parameters which are stored in the |.aux| files)
takes the same value after the conditional processing.
Otherwise the page numbers may take divergent values
depending on which part is compiled.

For example, a title page could be declared by:
%
\begin{center}
\begin{tabular}{l}
|\ifchilddoc\||else|\\
|\addtocounter{page}{-1}|\\
\textit{code for title page}\\
|\newpage|\\
|\||fi|
\end{tabular}
\end{center}
%
A banner page for the child documents can be generated by:
%
\begin{center}
\begin{tabular}{l}
|\ifchilddoc|\\
|\addtocounter{page}{-1}|\\
\textit{code for banner page}\\
|\newpage|\\
|\||fi|
\end{tabular}
\end{center}
%
Here one could write a message such as:
\begin{center}
|This is the part \childdocname{} of \childdocjob{}.|
\end{center}

%%%%%%%%%%%%%%%%%%%%%%%%%%%%%%%%%%%%%%%%%%%%%%%%%%%%%%%%%%%%%%%%%%%%%%%%%%%%%%%%
\subsection{Flags}
\label{sec:flags}

The package makes it easy to generate different versions
of the main or child documents.
To this end compilation flags can be defined
and assigned different default values.
They will be particularly useful in conjunction
with the forwarding mechanism described in \secref{sec:forward}.

For example, it may be useful to have a flag |\version|
which can be set to |draft| or |final|.
The document source will contain some conditional code
depending on the value of |\version|.
Suppose further, the flag should default to |final| for the main file
and to |draft| for child files
which is a natural assignment for editing the document.
This is achieved by placing the following code
in the preamble of the main document
(below the |\childdocmain| directive):
%
\begin{center}
\begin{tabular}{l}
|\ifchilddoc|\\
|\providecommand{\version}{draft}|\\
|\||else|\\
|\providecommand{\version}{final}|\\
|\||fi|
\end{tabular}
\end{center}
%
The definition by |\providecommand| makes sure
that previous definitions are not overwritten.
Further statements |\providecommand{\version}{...}|
can thus be added before the above code to override it.

For the main file, one might add a line
(between |\childdocmain| and the above block)
%
\begin{center}
|%\ifchilddoc\||else\providecommand{\version}{draft}\||fi|
\end{center}
%
which can be uncommented to produce a draft version.
Likewise one can add a line to the very top of a child file
(above the |\childdocof{|\textit{main}|}| directive)
%
\begin{center}
|%\providecommand{\version}{final}|
\end{center}
%
which can be uncommented to produce the final version of this child document.

%%%%%%%%%%%%%%%%%%%%%%%%%%%%%%%%%%%%%%%%%%%%%%%%%%%%%%%%%%%%%%%%%%%%%%%%%%%%%%%%
\subsection{Forwarding}
\label{sec:forward}

Different versions of the main or child documents
using compilation flags as described in \secref{sec:flags}
can be (permanently) stored in different files
for convenient compilation, viewing and distribution.
To this end, the package defines a command
to pass on compilation to a different file:

%%%%%%%%%%%%%%%%%%%%%%%%%%%%%%%%%%%%%%%%
\DescribeMacro{\childdocforward}
The command |\childdocforward| redirects processing to
another source file:
%
\begin{center}
\begin{tabular}{l}
|\input{childdoc.def}|\\
|\childdocforward[|\textit{main}|]{|\textit{dest}|}|\\
\end{tabular}
\end{center}
%
The argument \textit{dest} is the destination file
(without extension).
It should be the main file or one of the child files.
Note that further \textsf{childdoc} directives
such as |\childdocof| and |\childdocforward|
in the indicated file will be processed in this form.
The optional argument \textit{main}
passes on directly to the main file \textit{main}
while pretending to compile the child \textit{dest}.
This form behaves as if \textit{dest}
issues |\childdocof{|\textit{main}|}| right away,
and no further \textsf{childdoc} directives will be processed.

%%%%%%%%%%%%%%%%%%%%%%%%%%%%%%%%%%%%%%%%
\DescribeMacro{\...prefix}
In the alternative form |\childdocforwardprefix|,
%
\begin{center}
\begin{tabular}{l}
|\input{childdoc.def}|\\
|\childdocforwardprefix[|\textit{main}|]{|\textit{prefix}|}{|\textit{dest}|}|
\end{tabular}
\end{center}
%
the destination file is determined by a pattern
depending on the current file:
To make this work, the current file must be called
`{\textit{prefix}\hspace{0.2em}\textit{suffix}}'
with \textit{prefix} matching precisely the argument.
Processing is then passed on to the file
`{\textit{dest}\hspace{0.2em}\textit{suffix}}'.
Surely, the same effect is achieved by
directly specifying the
argument `{\textit{dest}\hspace{0.2em}\textit{suffix}}'
in the first form.
However, that requires to set up a different file
for each child. With the alternative form of the command
all these files can have exactly the same content
which simplifies setting them up and maintaining them.

For example, the following file |draft.tex|
with a compilation flag |\version| as described in \secref{sec:flags}
compiles the main document as a draft:
%
\begin{center}
\begin{tabular}{l}
|\def\version{draft}|\\
|\input{childdoc.def}|\\
|\childdocforward{|\textit{main}|}|
\end{tabular}
\end{center}
%
Likewise, the following files |final|\textit{nn}|.tex|
compile the final version of the child document
|child|\textit{nn}|.tex|:
%
\begin{center}
\begin{tabular}{l}
|\def\version{final}|\\
|\input{childdoc.def}|\\
|\childdocforwardprefix{final}{child}|
\end{tabular}
\end{center}
%

Note that when several versions of a main file and/or of each child file
are to be generated, it may be convenient to set up a |Makefile| or
shell script to automatise the process.

%%%%%%%%%%%%%%%%%%%%%%%%%%%%%%%%%%%%%%%%%%%%%%%%%%%%%%%%%%%%%%%%%%%%%%%%%%%%%%%%
\subsection{Command Line Processing}
\label{sec:commandline}

The effect of redirection files can also be achieved by invoking
the \LaTeX{} compiler with a more elaborate command line.
Most conveniently this should be done as part
of a shell script or a |Makefile|.

When using \textsf{childdoc} in the main file, the following
command lines effectively perform a redirection
(note that depending on the shell being used,
backslashes may have to be doubled: `|\|' $\to$ `|\\|'):
%
\begin{center}
|... -jobname "|\textit{target}|" |\\|"|[\textit{flags}]%
|\input{childdoc.def}\childdocforward[|\textit{main}|]{|\textit{dest}|}"|
\end{center}
%
Here \textit{target} is the name of the output file,
\textit{main} is the name of the main file
and \textit{dest} is the name of the main or child file to be processed
(all filenames without extensions).
The optional argument \textit{main} can be omitted
if \textit{main} matches \textit{dest}.
Optionally, compilation \textit{flags} can be defined via |\def| commands.
This command line makes the \TeX{} engine believe
it is compiling the file \textit{target}
whose content is specified as the latter parameter.
The provided code then forwards the processing to
\textit{main} or \textit{dest} as described in \secref{sec:forward}.

%%%%%%%%%%%%%%%%%%%%%%%%%%%%%%%%%%%%%%%%%%%%%%%%%%%%%%%%%%%%%%%%%%%%%%%%%%%%%%%%
\subsection{Include by Input}
\label{sec:input}

Including child documents by |\include| has some restrictions by design.
Most notably, the content of a child document always occupies
its own set of pages; pages cannot be shared between child documents.
Usually, this behaviour makes perfect sense
because each child document contain an essential part of the document.
However, in some situations it may be desirable to compose
a document from a collection of parts
without having mandatory page breaks between then.
For this case, the package
provides a mechanism to include parts
by |\input| which can also be processed individually.
However, by construction this mechanism
requires manual handling of the content to be output.

%%%%%%%%%%%%%%%%%%%%%%%%%%%%%%%%%%%%%%%%
\DescribeMacro{\ifchilddocmanual}
The main file should be prepared as usual, see \secref{sec:include}.
However, the document body must make a distinction
between processing of an individual part and of the main document, e.g.:
%
\begin{center}
\begin{tabular}{l}
|\ifchilddocmanual|\\
|\input{\childdocname}|\\
|\||else|\\
\textit{document body with }|\input{|\textit{part}|}|\\
|\||fi|
\end{tabular}
\end{center}
%
The conditional |\ifchilddocmanual| is true whenever
a part to be included by |\input| is being compiled,
and the name of the part is stored in |\childdocname|.

%%%%%%%%%%%%%%%%%%%%%%%%%%%%%%%%%%%%%%%%
\DescribeMacro{\childdocby}
Each part to be included by |\input| should start with:
%
\begin{center}
\begin{tabular}{l}
|\input{childdoc.def}|\\
|\childdocby{|\textit{main}|}|\\
\end{tabular}
\end{center}
%
The directive |\childdocby| is similar to |\childdocof|
described in \secref{sec:include},
but the subsequent selection of content must be done manually.
To that end, both |\ifchilddoc| and |\ifchilddocmanual|
will be true upon processing of a part,
and the name of the part is stored in |\childdocname|.
Note that |\jobname| will be set to the filename of the current part
so that each part receives an individual |.aux| file
that does not interfere with the |.aux| file(s) of the main document.
This behaviour can be altered by the alternative form
|\childdocby[*]{|\textit{main}|}| (with a non-empty optional argument)
which uses the |.aux| file of the main document
by setting |\jobname| to \textit{main}.

%%%%%%%%%%%%%%%%%%%%%%%%%%%%%%%%%%%%%%%%%%%%%%%%%%%%%%%%%%%%%%%%%%%%%%%%%%%%%%%%
\subsection{Driver Development}
\label{sec:driver}

The \textsf{childdoc} mechanism can also be use for the development
of definition files such as \LaTeX{} styles or classes.
This case differs from the above setup with multiple parts
included by |\include| in that no |\includeonly| should be invoked.
This can be achieved by starting the include file
(before |\ProvidesPackage|) with:
%
\begin{center}
\begin{tabular}{l}
|\input{childdoc.def}|\\
|\childdocforward{|\textit{main}|}|\\
\end{tabular}
\end{center}
%
or alternatively with:
%
\begin{center}
\begin{tabular}{l}
|\input{childdoc.def}|\\
|\childdocby{|\textit{main}|}|\\
\end{tabular}
\end{center}
%
Both forms have slightly different effects as described above.
The main file is prepared as usual, see \secref{sec:include}.

%%%%%%%%%%%%%%%%%%%%%%%%%%%%%%%%%%%%%%%%%%%%%%%%%%%%%%%%%%%%%%%%%%%%%%%%%%%%%%%%
\subsection{Legacy Detection}
\label{sec:detection}

The directive |\childdocmain| in the main file can detect
whether the complete document or merely a child is to be compiled
even without using the directive |\childdocof|.
This method is deprecated because it is less robust
and there is no compelling reason to use it;
it is merely provided for backward compatibility
and it may be removed in future versions.

If the detection mechanism is to be used,
it is mandatory to correctly specify
the filename of the main file as the argument of |\childdocmain|:
%
\begin{center}
\begin{tabular}{l}
|\input{childdoc.def}|\\
|\childdocmain{|\textit{main}|}|\\
\end{tabular}
\end{center}
%
If |\jobname| does not match the argument \textit{main} of |\childdocmain|,
it is assumed that |\jobname| points to the child file to be compiled.
When using |\childdocmain| with the main file specified as argument,
it suffices to start a child file
with just |\input{|\textit{main}|}|
without loading of the package and using |\childdocof|.
If instead all processing is done
with the appropriate \textsf{childdoc} directives,
the argument of \textit{main} of |\childdocmain| can be empty.

An alternative version of the command line processing described
in \secref{sec:commandline} using the detection mechanism reads:
%
\begin{center}
|... -jobname "|\textit{target}|" "|[\textit{flags}]%
[|\def\jobname{|\textit{dest}|}|]|\input{|\textit{main}|}"|
\end{center}

%%%%%%%%%%%%%%%%%%%%%%%%%%%%%%%%%%%%%%%%%%%%%%%%%%%%%%%%%%%%%%%%%%%%%%%%%%%%%%%%
\subsection{Manual Code}
\label{sec:manual}

In case one cannot be certain whether the definitions file |childdoc.def|
is installed on the target \TeX{} distribution
and one prefers not to ship it,
it is conceivable to paste a few relevant commands into the sources.

To that end, drop all statements |\input{childdoc.def}|
and perform the replacements as outlined below.
Instead of |\childdocmain{|\textit{main}|}| add the following code
to the top of the main file:
%
\begin{center}
\begin{tabular}{l}
|\||ifdefined\childdocname\endinput\||fi\newif\ifchilddoc|\\
|\edef\childdocname{\scantokens\expandafter{\jobname\noexpand}}|\\
|\def\childdocmain{|\textit{main}|}\||ifx\childdocmain\childdocname\||else|\\
|\childdoctrue\includeonly{\childdocname}\let\jobname\childdocmain\||fi|\\
\end{tabular}
\end{center}
%
Instead of |\childdocof{|\textit{main}|}| just include the main file
at the top of each child file:
%
\begin{center}
|\input{|\textit{main}|}|
\end{center}
%
A simple redirection |\childdocforward{|\textit{dest}|}| is achieved by:
%
\begin{center}
|\def\jobname{|\textit{dest}|}\input{\jobname}|
\end{center}
%
The redirection with prefix
|\childdocforwardprefix[|\textit{prefix}|]{|\textit{dest}|}|
is accomplished by:
%
\begin{center}
\begin{tabular}{l}
|{\edef\jobname{\scantokens\expandafter{\jobname\noexpand}}|\\
|\def\redirectjob |\textit{prefix}|#1~~~{\gdef\jobname{|\textit{dest}|#1}}|\\
|\expandafter\redirectjob\jobname~~~}\input{\jobname}|
\end{tabular}
\end{center}

In an alternative approach,
child documents can be compiled by a specific command line
without additional code or specific definitions:
%
\begin{center}
|... -jobname "|\textit{target}|" "|[\textit{flags}]%
|\includeonly{|\textit{dest}|}\input{|\textit{main}|}"|
\end{center}
%

%%%%%%%%%%%%%%%%%%%%%%%%%%%%%%%%%%%%%%%%%%%%%%%%%%%%%%%%%%%%%%%%%%%%%%%%%%%%%%%%
%%%%%%%%%%%%%%%%%%%%%%%%%%%%%%%%%%%%%%%%%%%%%%%%%%%%%%%%%%%%%%%%%%%%%%%%%%%%%%%%
\section{Information}

%%%%%%%%%%%%%%%%%%%%%%%%%%%%%%%%%%%%%%%%%%%%%%%%%%%%%%%%%%%%%%%%%%%%%%%%%%%%%%%%
\subsection{Copyright}

Copyright \copyright{} 2017--2018 Niklas Beisert

This work may be distributed and/or modified under the
conditions of the \LaTeX{} Project Public License, either version 1.3
of this license or (at your option) any later version.
The latest version of this license is in
  \url{http://www.latex-project.org/lppl.txt}
and version 1.3 or later is part of all distributions of \LaTeX{}
version 2005/12/01 or later.

This work has the LPPL maintenance status `maintained'.

The Current Maintainer of this work is Niklas Beisert.

This work consists of the files |README.txt|, |childdoc.ins| and |childdoc.dtx|
as well as the derived files |childdoc.def|, |cdocsamp.tex|
with |cdocsch1.tex|, |cdocsch2.tex|, |cdocspt3.tex|, |cdocspt4.tex|,
|cdocsdrf.tex|, |cdocsfn1.tex|, |cdocsfn2.tex|
as well as |childdoc.pdf|.

%%%%%%%%%%%%%%%%%%%%%%%%%%%%%%%%%%%%%%%%%%%%%%%%%%%%%%%%%%%%%%%%%%%%%%%%%%%%%%%%
\subsection{Files and Installation}

The package consists of the files:
%
\begin{center}
\begin{tabular}{ll}
    |README.txt|   & readme file \\
    |childdoc.ins| & installation file \\
    |childdoc.dtx| & source file \\
    |childdoc.def| & definition file \\
    |cdocsamp.tex| & sample main file \\
    |cdocsch1.tex| & sample include file \\
    |cdocsch2.tex| & sample include file \\
    |cdocspt3.tex| & sample part file \\
    |cdocspt4.tex| & sample part file \\
    |cdocsdrf.tex| & sample redirection file \\
    |cdocsfn1.tex| & sample redirection file \\
    |cdocsfn2.tex| & sample redirection file \\
    |childdoc.pdf| & manual
\end{tabular}
\end{center}
%
The distribution consists of the files
|README.txt|, |childdoc.ins| and |childdoc.dtx|.
%
\begin{itemize}
\item
Run (pdf)\LaTeX{} on |childdoc.dtx|
to compile the manual |childdoc.pdf| (this file).
\item
Run \LaTeX{} on |childdoc.ins| to create the definitions file |childdoc.def|
and the sample |cdocsamp.tex| with include files
|cdocsch1.tex|, |cdocsch2.tex|, |cdocspt3.tex|, |cdocspt4.tex|,
|cdocsdrf.tex|, |cdocsfn1.tex|, |cdocsfn2.tex|.
Then copy the file |childdoc.def| to an appropriate directory of your \LaTeX{}
distribution, e.g.\ \textit{texmf-root}|/tex/latex/childdoc|.
\end{itemize}

%%%%%%%%%%%%%%%%%%%%%%%%%%%%%%%%%%%%%%%%%%%%%%%%%%%%%%%%%%%%%%%%%%%%%%%%%%%%%%%%
\subsection{Related CTAN Packages}

There are several other packages which offer a similar functionality:
%
\begin{itemize}
\item
The packages
\href{http://ctan.org/pkg/docmute}{\textsf{docmute}},
\href{http://ctan.org/pkg/includex}{\textsf{includex}} and
\href{http://ctan.org/pkg/standalone}{\textsf{standalone}}
provide commands to include only the document body of
a child file thus allowing both files to be compiled individually.
\item
The packages \href{http://ctan.org/pkg/subdocs}{\textsf{subdocs}}
and \href{http://ctan.org/pkg/subfiles}{\textsf{subfiles}}
provide structures in which the main and child documents can be
encapsulated and allowing them to be compiled individually.
The inclusion mechanism is different from the conventional |\include|.
\item
The package \href{http://ctan.org/pkg/combine}{\textsf{combine}}
is an elaborate solution to combine several documents into one.
\end{itemize}
%
See also the CTAN topic \href{http://ctan.org/topic/subdocs}{\textsf{subdocs}}
for further related packages.
The present package differs from the above solutions in that
a document structure constructed with the conventional |\include| mechanism
just needs two extra commands at the top of every file
such that all constituent files can be compiled individually.

%%%%%%%%%%%%%%%%%%%%%%%%%%%%%%%%%%%%%%%%%%%%%%%%%%%%%%%%%%%%%%%%%%%%%%%%%%%%%%%%
%\subsection{Feature Suggestions}
%
%The following is a list of features which may be useful for future
%versions of this package:
%%
%\begin{itemize}
%\item
%\ldots
%\end{itemize}

%%%%%%%%%%%%%%%%%%%%%%%%%%%%%%%%%%%%%%%%%%%%%%%%%%%%%%%%%%%%%%%%%%%%%%%%%%%%%%%%
\subsection{Revision History}

%%%%%%%%%%%%%%%%%%%%%%%%%%%%%%%%%%%%%%%%
\paragraph{v2.0:} 2018/12/30

\begin{itemize}
\item
immediate forward processing
\item
added |\childdocby| mechanism
\item
manual restructured
\end{itemize}

%%%%%%%%%%%%%%%%%%%%%%%%%%%%%%%%%%%%%%%%
\paragraph{v1.6:} 2018/01/17

\begin{itemize}
\item
application for development of include files
\item
corrections to manual
\end{itemize}

%%%%%%%%%%%%%%%%%%%%%%%%%%%%%%%%%%%%%%%%
\paragraph{v1.5:} 2017/05/21

\begin{itemize}
\item
more complete structuring introduced
\item
|\childdocof| introduced
\item
|\childdoc| renamed to |\childdocmain|
\item
|\childredirect| renamed to |\childdocforward| and |\childdocforwardprefix|
and functionality expanded
\end{itemize}

%%%%%%%%%%%%%%%%%%%%%%%%%%%%%%%%%%%%%%%%
\paragraph{v1.0:} 2017/04/27

\begin{itemize}
\item
manual and install package
\item
first version published on CTAN
\end{itemize}

%%%%%%%%%%%%%%%%%%%%%%%%%%%%%%%%%%%%%%%%
\paragraph{v0.6:} 2017/04/26

\begin{itemize}
\item
redirection mechanism added
\end{itemize}

%%%%%%%%%%%%%%%%%%%%%%%%%%%%%%%%%%%%%%%%
\paragraph{v0.5:} 2017/04/26

\begin{itemize}
\item
functionality in definition file
\end{itemize}


%%%%%%%%%%%%%%%%%%%%%%%%%%%%%%%%%%%%%%%%%%%%%%%%%%%%%%%%%%%%%%%%%%%%%%%%%%%%%%%%
%%%%%%%%%%%%%%%%%%%%%%%%%%%%%%%%%%%%%%%%%%%%%%%%%%%%%%%%%%%%%%%%%%%%%%%%%%%%%%%%
%%%%%%%%%%%%%%%%%%%%%%%%%%%%%%%%%%%%%%%%%%%%%%%%%%%%%%%%%%%%%%%%%%%%%%%%%%%%%%%%
\appendix

\settowidth\MacroIndent{\rmfamily\scriptsize 000\ }

 \DocInput{childdoc.dtx}

\end{document}
%</driver>
% \fi
%
% %%%%%%%%%%%%%%%%%%%%%%%%%%%%%%%%%%%%%%%%%%%%%%%%%%%%%%%%%%%%%%%%%%%%%%%%%%%%%%
% %%%%%%%%%%%%%%%%%%%%%%%%%%%%%%%%%%%%%%%%%%%%%%%%%%%%%%%%%%%%%%%%%%%%%%%%%%%%%%
% \section{Sample}
%\iffalse
%<*samplemain>
%\fi
%
% The following presents a sample document
% with two chapters, two parts, a title page,
% a compile flag as well as three forwarding files to set the flag.
% It consists of eight |.tex| files:
% \begin{center}
% \begin{tabular}{ll}
% |cdocsamp.tex|&main file\\
% |cdocsch1.tex|&include file for chapter 1\\
% |cdocsch2.tex|&include file for chapter 2\\
% |cdocspt3.tex|&include file for part 3\\
% |cdocspt4.tex|&include file for part 4\\
% |cdocsdrf.tex|&forwarding file for main file in draft mode\\
% |cdocsfi1.tex|&forwarding file for final version of chapter 1\\
% |cdocsfi2.tex|&forwarding file for final version of chapter 2\\
% \end{tabular}
% \end{center}
% Each of the eight files can be compiled directly by the \LaTeX{} compiler.
%
% %%%%%%%%%%%%%%%%%%%%%%%%%%%%%%%%%%%%%%
% \paragraph{Main File.}
%
% The main file is called |cdocsamp.tex|.
%
% Load the \textsf{childdoc} definitions and
% declare the filename for the main document:
%    \begin{macrocode}
\input{childdoc.def}
\childdocmain{}
%    \end{macrocode}

% Optional override for |\version| flag:
%    \begin{macrocode}
%%\ifchilddoc\else\providecommand{\version}{draft}\fi
%    \end{macrocode}

% Define the default values for the |\version| flag
% (|final| for the main file and |draft| for childs):
%    \begin{macrocode}
\ifchilddoc
\providecommand{\version}{draft}
\else
\providecommand{\version}{final}
\fi
%    \end{macrocode}

% Load the standard document class:
%    \begin{macrocode}
\documentclass[12pt]{article}
%    \end{macrocode}

% Start the document body:
%    \begin{macrocode}
\begin{document}
%    \end{macrocode}

% Declare a title page.
% Print title, part of document being processed and version flag:
%    \begin{macrocode}
\addtocounter{page}{-1}
\begin{center}
{\LARGE\bfseries{}childdoc example\par}
\vspace{1cm}
\ifchilddoc
\ifchilddocmanual part\else chapter\fi:
`\childdocname' of `\childdocjob'\par
\else
main document: `\childdocjob'\par
\fi
version: \version\par
\end{center}
\newpage
%    \end{macrocode}

% Manually include selected file,
% otherwise process as usual:
%    \begin{macrocode}
\ifchilddocmanual
\section*{part `\childdocname'}
\input{\childdocname}
\else
%    \end{macrocode}

% Include the two chapters:
%    \begin{macrocode}
\include{cdocsch1}
\include{cdocsch2}
%    \end{macrocode}

% Include the two parts unless only chapters should be displayed:
%    \begin{macrocode}
\ifchilddoc\else
\section{part three}
\input{cdocspt3}
\section{part four}
\input{cdocspt4}
\fi
%    \end{macrocode}

% Process as usual until here:
%    \begin{macrocode}
\fi
%    \end{macrocode}

% End of document body:
%    \begin{macrocode}
\end{document}
%    \end{macrocode}
%\iffalse
%</samplemain>
%\fi
%
% %%%%%%%%%%%%%%%%%%%%%%%%%%%%%%%%%%%%%%
% \paragraph{Chapter Include Files.}
%
% The include files are called |cdocsch1.tex| and |cdocsch2.tex|.
%
%\iffalse
%<*samplechap1|samplechap2>
%\fi

% Optional override for |\version| flag:
%    \begin{macrocode}
%%\providecommand{\version}{final}
%    \end{macrocode}

% Include the main document:
%    \begin{macrocode}
\input{childdoc.def}
\childdocof{cdocsamp}
%    \end{macrocode}

%\iffalse
%</samplechap1|samplechap2>
%\fi
%
%\iffalse
%<*samplechap1>
%\fi
% Some text for chapter 1:
%    \begin{macrocode}
\section{one}
some text in chapter one
%    \end{macrocode}

%\iffalse
%</samplechap1>
%\fi
% Some text for chapter 2:
%\iffalse
%<*samplechap2>
%\fi
%    \begin{macrocode}
\section{two}
more text in chapter two
%    \end{macrocode}

%\iffalse
%</samplechap2>
%\fi
%
% %%%%%%%%%%%%%%%%%%%%%%%%%%%%%%%%%%%%%%
% \paragraph{Part Include Files.}
%
% The include files are called |cdocspt3.tex| and |cdocspt4.tex|.
%
%\iffalse
%<*samplepart3|samplepart4>
%\fi

% Optional override for |\version| flag:
%    \begin{macrocode}
%%\providecommand{\version}{final}
%    \end{macrocode}

% Include the main document:
%    \begin{macrocode}
\input{childdoc.def}
\childdocby{cdocsamp}
%    \end{macrocode}

%\iffalse
%</samplepart3|samplepart4>
%\fi
%
%\iffalse
%<*samplepart3>
%\fi
% Some text for part 3:
%    \begin{macrocode}
some text in part three
%    \end{macrocode}

%\iffalse
%</samplepart3>
%\fi
% Some text for part 4:
%\iffalse
%<*samplepart4>
%\fi
%    \begin{macrocode}
more text in part four
%    \end{macrocode}

%\iffalse
%</samplepart4>
%\fi
%
% %%%%%%%%%%%%%%%%%%%%%%%%%%%%%%%%%%%%%%
% \paragraph{Forwarding for a Complete Draft.}
%
% The following forwarding file |cdocsdrf.tex|
% compiles the main document in draft mode:
%\iffalse
%<*sampledraft>
%\fi
%    \begin{macrocode}
\def\version{draft}
\input{childdoc.def}
\childdocforward{cdocsamp}
%    \end{macrocode}

%\iffalse
%</sampledraft>
%\fi
%
% %%%%%%%%%%%%%%%%%%%%%%%%%%%%%%%%%%%%%%
% \paragraph{Forwarding for Final Version of the Chapters.}
%
% The following forwarding files |cdocsfn1.tex| and |cdocsfn2.tex|
% (with identical content)
% compile the final versions of the child documents
% |cdocsch1.tex| and |cdocsch2.tex|, respectively:
%\iffalse
%<*samplefinal>
%\fi
%    \begin{macrocode}
\def\version{final}
\input{childdoc.def}
\childdocforwardprefix[cdocsamp]{cdocsfn}{cdocsch}
%    \end{macrocode}

%\iffalse
%</samplefinal>
%\fi
%
% %%%%%%%%%%%%%%%%%%%%%%%%%%%%%%%%%%%%%%
% \paragraph{Command Line Processing.}
%
% The following three command lines generate the output files
% |cdocscld|, |cdocscl1| and |cdocscl2|
% which should be identical to
% |cdocsdrf|, |cdocsch1| and |cdocsfn2|, respectively:
% \begin{center}
% \begin{tabular}{l}
% |latex -jobname cdocscld \|\\
% |  "\def\version{draft}\input{childdoc.def}\childdocforward{cdocsamp}"|\\
% |latex -jobname cdocscl1 \|\\
% |  "\input{childdoc.def}\childdocforward[cdocsamp]{cdocsch1}"|\\
% |latex -jobname cdocscl2 \|\\
% |  "\def\version{final}\input{childdoc.def}\childdocforward{cdocsch2}"|
% \end{tabular}
% \end{center}
% Note that the trailing backslash on each first line
% merely continues the input to the second line
% (for convenient cut ant paste).
% Furthermore, the command |latex| can be replaced by any
% of its alternative versions such as |pdflatex|.
%
% %%%%%%%%%%%%%%%%%%%%%%%%%%%%%%%%%%%%%%%%%%%%%%%%%%%%%%%%%%%%%%%%%%%%%%%%%%%%%%
% %%%%%%%%%%%%%%%%%%%%%%%%%%%%%%%%%%%%%%%%%%%%%%%%%%%%%%%%%%%%%%%%%%%%%%%%%%%%%%
% \section{Implementation}
%\iffalse
%<*package>
%\fi
%
% This section describes the definitions file |childdoc.def|.

% The definitions cannot be loaded using |\usepackage| or |\RequirePackage|
% which has a mechanism to prevent loading a style file more than once.
% When loading the definitions by means of |\input|
% multiple instances have to be prevented manually:
%\iffalse
%This code needs to be before the `\ProvidesFile' directive
%which is defined at the beginning of this file.
%Therefore it is also placed there and commented out here.
%</package>
%<*discard>
%\fi
%    \begin{macrocode}
\ifdefined\childdocmain\endinput\fi
%    \end{macrocode}
%\iffalse
%</discard>
%<*package>
%\fi
%
% \macro{\ifchilddoc}
% \macro{\ifchilddocmanual}
% The conditional |\ifchilddoc| tells whether a
% child (true) or main (false) document is being compiled.
% The conditional |\ifchilddocmanual| tells whether
% the |\includeonly| mechanism is used (false) or
% the selection of child files must be performed manually (true).
% The definitions initialise to false:
%    \begin{macrocode}
\newif\ifchilddoc
\newif\ifchilddocmanual
%    \end{macrocode}

% \macro{\childdocname}
% \macro{\childdocjob}
% The macro |\childdocname| stores the name of the main document
% to be compiled. The macro |\childdocjob| stores the name of
% the document on which the \LaTeX{} compiler was originally invoked.
% The content of |\jobname| cannot be compared
% to filenames specified in the source due to different catcodes.
% The following code rescans |\jobname|, stores the result
% in |\childdocname| and saves a copy in |\childdocjob|:
%    \begin{macrocode}
\edef\childdocname{\scantokens\expandafter{\jobname\noexpand}}
\let\childdocjob\childdocname
%    \end{macrocode}

% \macro{\childdocdisable}
% The macro |\childdocdisable| prevents the main file
% from being processed more than once.
% At this stage, the main document command |\childdocmain|
% is assumed to be called once again where it should do nothing.
% Any subsequent call to it should prevent
% a secondary processing of the main document
% It overwrites the forwarding commands
% |\childdocof| and |\childdocforward|
% with empty macros to prevent further inclusions of the main document:
%    \begin{macrocode}
\newcommand{\childdocdisable}
{
  \renewcommand{\childdocmain}[1]{\renewcommand{\childdocmain}[1]{\endinput}}
  \renewcommand{\childdocof}[1]{}
  \renewcommand{\childdocby}[2][]{}
  \renewcommand{\childdocforward}[2][]{}
  \renewcommand{\childdocdisable}{}
}
%    \end{macrocode}

% \macro{\childdocmain}
% The macro |\childdocmain| is to be called at the top of the main file
% with nothing or the main filename (without extension) as argument.
% First, it breaks loops.
% If the argument is not empty and does not match |\childdocname|
% (which is set by the first inclusion of |childdoc.def|),
% |\ifchilddoc| is set to true, |\includeonly| is applied to the child file
% and |\jobname| is set to the main file
% (for proper handling of |.aux| files):
%    \begin{macrocode}
\newcommand{\childdocmain}[1]
{
  \childdocdisable\childdocmain{}
  \if?#1?\else
    \begingroup
      \def\childdoctmp{#1}
      \ifx\childdoctmp\childdocname
        \def\childdoctmp{}
      \else
        \def\childdoctmp
        {
          \childdoctrue
          \includeonly{\childdocname}
          \def\childdocjob{#1}
          \def\jobname{#1}
        }
      \fi
      \expandafter
    \endgroup
    \childdoctmp
  \fi
}
%    \end{macrocode}

% \macro{\childdocof}
% The command |\childdocof| redirects
% compilation to the main file |#1|.
%    \begin{macrocode}
\newcommand{\childdocof}[1]
{
  \childdocdisable
  \childdoctrue
  \includeonly{\childdocname}
  \def\jobname{#1}
  \def\childdocjob{#1}
  \input{#1}
}
%    \end{macrocode}

% \macro{\childdocby}
% The command |\childdocby| ....
%    \begin{macrocode}
\newcommand{\childdocby}[2][]
{
  \childdocdisable
  \childdoctrue
  \childdocmanualtrue
  \if?#1?\else
    \def\jobname{#2}
  \fi
  \def\childdocjob{#2}
  \input{#2}
  \endinput
}
%    \end{macrocode}

% \macro{\childdocforward}
% The command |\childdocforward| redirects
% compilation to the main file or
% (if the optional argument is given) a child file.
% Parameters are set as if the main file
% or a child file starting with |\childdocof| was compiled.
% Then compilation is handed over to the main file:
%    \begin{macrocode}
\newcommand{\childdocforward}[2][]
{
  \begingroup
    \if?#1?
      \def\childdoctmp
      {
        \def\childdocname{#2}
        \def\childdocjob{#2}
        \def\jobname{#2}
        \input{#2}
        \endinput
      }
    \else
      \def\childdoctmp
      {
        \childdocdisable
        \def\childdocname{#2}
        \childdoctrue
        \includeonly{#2}
        \def\childdocjob{#1}
        \def\jobname{#1}
        \input{#1}
        \endinput
      }
    \fi
    \expandafter
  \endgroup
  \childdoctmp
}
%    \end{macrocode}

% \macro{\childdocforwardprefix}
% The command |\childdocforwardprefix| redirects
% compilation to the main or a child file by means of a pattern.
% The prefix |#1| in the current filename is replaced by |#2|
% and the suffix of the current filename is kept
% (it is assumed that the filename does not contain the substring `|~~~|'
% which is used as a delimiter).
% Compilation is handed over to the new file by |\childdocforward|:
%    \begin{macrocode}
\newcommand{\childdocforwardprefix}[3][]
{
  \begingroup
    \def\childdocextract #2##1~~~{\def\childdoctmp{\childdocforward[#1]{#3##1}}}
    \expandafter\childdocextract\childdocname~~~
    \expandafter
  \endgroup
  \childdoctmp
}
%    \end{macrocode}

% \macro{\childdoc}
% The deprecated macro |\childdoc| is a legacy version of |\childdocmain|:
%    \begin{macrocode}
\newcommand{\childdoc}{\childdocmain}
%    \end{macrocode}

% \macro{\childdocredirect}
% The deprecated macro |\childdocredirect| is a legacy version
% of |\childdocforward| and |\childdocforwardprefix|:
%    \begin{macrocode}
\newcommand{\childdocredirect}[2][]
{
  \begingroup
    \if?#1?
      \def\childdoctmp{\childdocforward{#2}}
    \else
      \def\childdoctmp{\childdocforwardprefix{#1}{#2}}
    \fi
    \expandafter
  \endgroup
  \childdoctmp
}
%    \end{macrocode}

%\iffalse
%</package>
%\fi
%
\endinput
|\\
|\childdocforwardprefix[|\textit{main}|]{|\textit{prefix}|}{|\textit{dest}|}|
\end{tabular}
\end{center}
%
the destination file is determined by a pattern
depending on the current file:
To make this work, the current file must be called
`{\textit{prefix}\hspace{0.2em}\textit{suffix}}'
with \textit{prefix} matching precisely the argument.
Processing is then passed on to the file
`{\textit{dest}\hspace{0.2em}\textit{suffix}}'.
Surely, the same effect is achieved by
directly specifying the
argument `{\textit{dest}\hspace{0.2em}\textit{suffix}}'
in the first form.
However, that requires to set up a different file
for each child. With the alternative form of the command
all these files can have exactly the same content
which simplifies setting them up and maintaining them.

For example, the following file |draft.tex|
with a compilation flag |\version| as described in \secref{sec:flags}
compiles the main document as a draft:
%
\begin{center}
\begin{tabular}{l}
|\def\version{draft}|\\
|% \iffalse
%
% childdoc.dtx Copyright (C) 2017-2018 Niklas Beisert
%
% This work may be distributed and/or modified under the
% conditions of the LaTeX Project Public License, either version 1.3
% of this license or (at your option) any later version.
% The latest version of this license is in
%   http://www.latex-project.org/lppl.txt
% and version 1.3 or later is part of all distributions of LaTeX
% version 2005/12/01 or later.
%
% This work has the LPPL maintenance status `maintained'.
%
% The Current Maintainer of this work is Niklas Beisert.
%
% This work consists of the files childdoc.dtx and childdoc.ins
% and the derived files childdoc.def and cdocsamp.tex with
% cdocsch1.tex, cdocsch2.tex, cdocsdrf.tex, cdocsfn1.tex, cdocsfn2.tex.
%
%<package>\ifdefined\childdocmain\endinput\fi
%<package>\ProvidesFile{childdoc.def}[2018/12/30 v2.0 child document driver]
%<samplemain>\ProvidesFile{cdocsamp.tex}[2018/12/30 v2.0 sample for childdoc]
%<*driver>
%\ProvidesFile{childdoc.drv}[2018/12/30 v2.0 childdoc reference manual file]
\PassOptionsToClass{10pt,a4paper}{article}
\documentclass{ltxdoc}

\usepackage[margin=35mm]{geometry}
\usepackage{hyperref}
\usepackage{hyperxmp}
\usepackage[usenames]{color}

\hypersetup{colorlinks=true}
\hypersetup{pdfstartview=FitH}
\hypersetup{pdfpagemode=UseNone}
\hypersetup{pdfsource={}}
\hypersetup{pdflang={en-UK}}
\hypersetup{pdfcopyright={Copyright 2017-2018 Niklas Beisert.
  This work may be distributed and/or modified under the
  conditions of the LaTeX Project Public License, either version 1.3
  of this license or (at your option) any later version.}}
\hypersetup{pdflicenseurl={http://www.latex-project.org/lppl.txt}}
\hypersetup{pdfcontactaddress={ETH Zurich, ITP, HIT K,
  Wolfgang-Pauli-Strasse 27}}
\hypersetup{pdfcontactpostcode={8093}}
\hypersetup{pdfcontactcity={Zurich}}
\hypersetup{pdfcontactcountry={Switzerland}}
\hypersetup{pdfcontactemail={nbeisert@itp.phys.ethz.ch}}
\hypersetup{pdfcontacturl={http://people.phys.ethz.ch/\xmptilde nbeisert/}}

\newcommand{\secref}[1]{\hyperref[#1]{section \ref*{#1}}}

\parskip1ex
\parindent0pt
\let\olditemize\itemize
\def\itemize{\olditemize\parskip0pt}

\begin{document}

\title{The \textsf{childdoc} Package}
\hypersetup{pdftitle={The childdoc Package}}
\author{Niklas Beisert\\[2ex]
  Institut f\"ur Theoretische Physik\\
  Eidgen\"ossische Technische Hochschule Z\"urich\\
  Wolfgang-Pauli-Strasse 27, 8093 Z\"urich, Switzerland\\[1ex]
  \href{mailto:nbeisert@itp.phys.ethz.ch}
  {\texttt{nbeisert@itp.phys.ethz.ch}}}
\hypersetup{pdfauthor={Niklas Beisert}}
\hypersetup{pdfsubject={Manual for the LaTeX2e Package childdoc}}
\date{30 December 2018, \textsf{v2.0}}
\maketitle

\begin{abstract}\noindent
\textsf{childdoc} is a \LaTeXe{} package
that enables the direct compilation
of document sections included by |\include|
to individual files.
\end{abstract}

\begingroup
\parskip0ex
\tableofcontents
\endgroup

%%%%%%%%%%%%%%%%%%%%%%%%%%%%%%%%%%%%%%%%%%%%%%%%%%%%%%%%%%%%%%%%%%%%%%%%%%%%%%%%
%%%%%%%%%%%%%%%%%%%%%%%%%%%%%%%%%%%%%%%%%%%%%%%%%%%%%%%%%%%%%%%%%%%%%%%%%%%%%%%%
\section{Introduction}

\LaTeX{} provides a mechanism to structure a large document (such as a book)
into a main file and several child files (containing the chapters)
using the |\include| command.
This mechanism is beneficial for documents
which span hundreds of pages in order to
make the source file(s) more manageable.
Moreover, compilation can be restricted to
selected child files by means of the |\includeonly| command.
The latter feature can be used to reduce the compilation time while editing
(this was significantly more useful in the earlier days of \LaTeX{})
or to generate a smaller document which is easier to navigate.
Another application of |\includeonly| is to generate
documents consisting of selected parts of the complete document.

However, there are a few drawbacks of the plain |\include| mechanism:
\begin{itemize}
\item
The child files cannot be compiled on their own,
they can only be compiled via the main file.
A naive editing environment
(such as a text editor with an option
to have the current file processed by \LaTeX)
may require one to switch to the main file before compiling;
attempting to compile the child file produces errors.
\item
The main file must be modified (each time)
to adjust the |\includeonly| command
to the present needs. This easily leaves the main file in a messy state.
\item
The generated document will always carry the filename
of the main document. This is inconvenient if
several child files are to be compiled and
to be kept for distribution.
\end{itemize}

The present package provides a simple interface
to make child files individually compilable by \LaTeX{}.
Compiling a child file then has the same effect as compiling
the main file with an |\includeonly| command
to select the appropriate child.
Moreover the generated document will carry the name of the child
rather than the main file.
This resolves all three above issues.

This feature is meant to make the editing of books,
thesis documents and lecture notes somewhat more convenient.
However, the package can also be used efficiently for
composing a series of documents (such as exercise sheets)
which are typically distributed individually.
It then assists the author in generating the individual documents
(potentially in different versions)
as well as a document containing the collected series.
Another application is in developing style files
or other kinds of included material
where compilation of the style file could redirect
to a sample or test file.

%%%%%%%%%%%%%%%%%%%%%%%%%%%%%%%%%%%%%%%%%%%%%%%%%%%%%%%%%%%%%%%%%%%%%%%%%%%%%%%%
%%%%%%%%%%%%%%%%%%%%%%%%%%%%%%%%%%%%%%%%%%%%%%%%%%%%%%%%%%%%%%%%%%%%%%%%%%%%%%%%
\section{Usage}

First of all, the package \textsf{childdoc} is \emph{not} a standard
\LaTeXe{} |.sty| style file! Therefore it needs to be invoked in
a non-standard way.

%%%%%%%%%%%%%%%%%%%%%%%%%%%%%%%%%%%%%%%%%%%%%%%%%%%%%%%%%%%%%%%%%%%%%%%%%%%%%%%%
\subsection{Included Files}
\label{sec:include}

%%%%%%%%%%%%%%%%%%%%%%%%%%%%%%%%%%%%%%%%
\DescribeMacro{\childdocmain}
To use the package, add the commands
\begin{center}
\begin{tabular}{l}
|\input{childdoc.def}|\\
|\childdocmain{}|\\
\end{tabular}
\end{center}
at the very top of the main \LaTeX{} file,
in particular \emph{before} the |\documentclass| statement!
The argument of |\childdocmain| should be left empty
(but it must be present).

%%%%%%%%%%%%%%%%%%%%%%%%%%%%%%%%%%%%%%%%
\DescribeMacro{\childdocof}
Furthermore, add the commands
\begin{center}
\begin{tabular}{l}
|\input{childdoc.def}|\\
|\childdocof{|\textit{main}|}|\\
\end{tabular}
\end{center}
at the top of every child file \textit{child}
which is included by |\include{|\textit{child}|}|
from within the main file
(or at least for those files to be compiled individually).
The argument \textit{main} must be the filename of the main file.

There are a couple of
considerations in setting up the main and child documents:

%%%%%%%%%%%%%%%%%%%%%%%%%%%%%%%%%%%%%%%%
\paragraph{Restrictions.}

Please note the following restrictions:
\begin{itemize}
\item
|\childdocmain| must be called with one argument \textit{main}
to ensure compatibility with earlier version of the package.
It must either be empty (|\childdocmain{}|)
or precisely match the filename of the main file in which it is specified.
See \secref{sec:detection} for further information.
\item
The filename \textit{main} must be specified without the |.tex| extension.
\item
The filename \textit{main} is case sensitive
(even in case-insensitive file systems)
due to internal string comparison.
\item
The argument \textit{main} should be fully expanded, it cannot be a macro.
\item
Subdirectories and special characters should be avoided in filenames.
\item
The command |\childdocmain{|\textit{main}|}| must be followed by a whitespace.
It should not be followed immediately by another command
or by a comment mark `|%|'.
This is because the \TeX{} parser reads the token immediately following
the argument of |\childdocmain| and puts it
at the beginning of every child section;
however, a white\-space is ignored.
\end{itemize}

%%%%%%%%%%%%%%%%%%%%%%%%%%%%%%%%%%%%%%%%
\paragraph{Content of Main File.}

It is advisable to place all content in the child files included by |\include|.
Any output contained in the main file will appear in all child documents
unless suppressed manually;
it cannot be suppressed automatically by the |\includeonly| directive
and thus should normally be avoided.
A method to include some content in the main file
by means of conditional processing is described in \secref{sec:conditional}.

%%%%%%%%%%%%%%%%%%%%%%%%%%%%%%%%%%%%%%%%
\paragraph{Page Numbering.}

When only a part of the document is compiled,
the appropriate numbering of pages
(as well as other status parameters)
is determined from the |.aux| files.
The latter contain information from previous passes.
However this information needs to propagate through
all intermediate child documents.
Therefore the page numbering in child documents may well
be inconsistent until the complete document is compiled at least once.

A useful (if unconventional) way to always ensure a consistent
page numbering is to restart the numbering in each child document
and denote the pages by `\textit{child}|.|\textit{page}'
where \textit{child} represents the chapter/section number of the child file.
This can be achieved by the command
|\numberwithin{page}{|\textit{child}|}|
of the \textsf{amsmath} package
where \textit{child} can be |chapter| or |section|
depending on the chosen structuring.
Alternatively, one can modify the macro |\thepage| appropriately
and reset the counter |page| at the start of each child file.

%%%%%%%%%%%%%%%%%%%%%%%%%%%%%%%%%%%%%%%%%%%%%%%%%%%%%%%%%%%%%%%%%%%%%%%%%%%%%%%%
\subsection{Conditional Processing}
\label{sec:conditional}

The package provides a mechanism to compile different versions
of a document. To customise the versions further some conditional processing
can come in handy to distinguish which version is being compiled.
The package provides two macros to describe the compilation context:

%%%%%%%%%%%%%%%%%%%%%%%%%%%%%%%%%%%%%%%%
\DescribeMacro{\ifchilddoc}
The conditional |\ifchilddoc| distinguishes between the compilation of
child documents and the main document:
%
\begin{center}
|\ifchilddoc |\textit{child-code}| |[|\||else |\textit{main-code}]| \||fi|
\end{center}

%%%%%%%%%%%%%%%%%%%%%%%%%%%%%%%%%%%%%%%%
\DescribeMacro{\childdocname}
\DescribeMacro{\childdocjob}
The macro |\childdocname| contains the filename (without extension)
of the main or child file being processed.
Note that |\childdocjob| will always contain the name of the main file.

%%%%%%%%%%%%%%%%%%%%%%%%%%%%%%%%%%%%%%%%
\paragraph{Title Page.}

Conditional processing can be used to include a title or banner page
in the main document when proper precautions are taken.
Importantly, the code in the main file should ensure that the page counter
(as well as other status parameters which are stored in the |.aux| files)
takes the same value after the conditional processing.
Otherwise the page numbers may take divergent values
depending on which part is compiled.

For example, a title page could be declared by:
%
\begin{center}
\begin{tabular}{l}
|\ifchilddoc\||else|\\
|\addtocounter{page}{-1}|\\
\textit{code for title page}\\
|\newpage|\\
|\||fi|
\end{tabular}
\end{center}
%
A banner page for the child documents can be generated by:
%
\begin{center}
\begin{tabular}{l}
|\ifchilddoc|\\
|\addtocounter{page}{-1}|\\
\textit{code for banner page}\\
|\newpage|\\
|\||fi|
\end{tabular}
\end{center}
%
Here one could write a message such as:
\begin{center}
|This is the part \childdocname{} of \childdocjob{}.|
\end{center}

%%%%%%%%%%%%%%%%%%%%%%%%%%%%%%%%%%%%%%%%%%%%%%%%%%%%%%%%%%%%%%%%%%%%%%%%%%%%%%%%
\subsection{Flags}
\label{sec:flags}

The package makes it easy to generate different versions
of the main or child documents.
To this end compilation flags can be defined
and assigned different default values.
They will be particularly useful in conjunction
with the forwarding mechanism described in \secref{sec:forward}.

For example, it may be useful to have a flag |\version|
which can be set to |draft| or |final|.
The document source will contain some conditional code
depending on the value of |\version|.
Suppose further, the flag should default to |final| for the main file
and to |draft| for child files
which is a natural assignment for editing the document.
This is achieved by placing the following code
in the preamble of the main document
(below the |\childdocmain| directive):
%
\begin{center}
\begin{tabular}{l}
|\ifchilddoc|\\
|\providecommand{\version}{draft}|\\
|\||else|\\
|\providecommand{\version}{final}|\\
|\||fi|
\end{tabular}
\end{center}
%
The definition by |\providecommand| makes sure
that previous definitions are not overwritten.
Further statements |\providecommand{\version}{...}|
can thus be added before the above code to override it.

For the main file, one might add a line
(between |\childdocmain| and the above block)
%
\begin{center}
|%\ifchilddoc\||else\providecommand{\version}{draft}\||fi|
\end{center}
%
which can be uncommented to produce a draft version.
Likewise one can add a line to the very top of a child file
(above the |\childdocof{|\textit{main}|}| directive)
%
\begin{center}
|%\providecommand{\version}{final}|
\end{center}
%
which can be uncommented to produce the final version of this child document.

%%%%%%%%%%%%%%%%%%%%%%%%%%%%%%%%%%%%%%%%%%%%%%%%%%%%%%%%%%%%%%%%%%%%%%%%%%%%%%%%
\subsection{Forwarding}
\label{sec:forward}

Different versions of the main or child documents
using compilation flags as described in \secref{sec:flags}
can be (permanently) stored in different files
for convenient compilation, viewing and distribution.
To this end, the package defines a command
to pass on compilation to a different file:

%%%%%%%%%%%%%%%%%%%%%%%%%%%%%%%%%%%%%%%%
\DescribeMacro{\childdocforward}
The command |\childdocforward| redirects processing to
another source file:
%
\begin{center}
\begin{tabular}{l}
|\input{childdoc.def}|\\
|\childdocforward[|\textit{main}|]{|\textit{dest}|}|\\
\end{tabular}
\end{center}
%
The argument \textit{dest} is the destination file
(without extension).
It should be the main file or one of the child files.
Note that further \textsf{childdoc} directives
such as |\childdocof| and |\childdocforward|
in the indicated file will be processed in this form.
The optional argument \textit{main}
passes on directly to the main file \textit{main}
while pretending to compile the child \textit{dest}.
This form behaves as if \textit{dest}
issues |\childdocof{|\textit{main}|}| right away,
and no further \textsf{childdoc} directives will be processed.

%%%%%%%%%%%%%%%%%%%%%%%%%%%%%%%%%%%%%%%%
\DescribeMacro{\...prefix}
In the alternative form |\childdocforwardprefix|,
%
\begin{center}
\begin{tabular}{l}
|\input{childdoc.def}|\\
|\childdocforwardprefix[|\textit{main}|]{|\textit{prefix}|}{|\textit{dest}|}|
\end{tabular}
\end{center}
%
the destination file is determined by a pattern
depending on the current file:
To make this work, the current file must be called
`{\textit{prefix}\hspace{0.2em}\textit{suffix}}'
with \textit{prefix} matching precisely the argument.
Processing is then passed on to the file
`{\textit{dest}\hspace{0.2em}\textit{suffix}}'.
Surely, the same effect is achieved by
directly specifying the
argument `{\textit{dest}\hspace{0.2em}\textit{suffix}}'
in the first form.
However, that requires to set up a different file
for each child. With the alternative form of the command
all these files can have exactly the same content
which simplifies setting them up and maintaining them.

For example, the following file |draft.tex|
with a compilation flag |\version| as described in \secref{sec:flags}
compiles the main document as a draft:
%
\begin{center}
\begin{tabular}{l}
|\def\version{draft}|\\
|\input{childdoc.def}|\\
|\childdocforward{|\textit{main}|}|
\end{tabular}
\end{center}
%
Likewise, the following files |final|\textit{nn}|.tex|
compile the final version of the child document
|child|\textit{nn}|.tex|:
%
\begin{center}
\begin{tabular}{l}
|\def\version{final}|\\
|\input{childdoc.def}|\\
|\childdocforwardprefix{final}{child}|
\end{tabular}
\end{center}
%

Note that when several versions of a main file and/or of each child file
are to be generated, it may be convenient to set up a |Makefile| or
shell script to automatise the process.

%%%%%%%%%%%%%%%%%%%%%%%%%%%%%%%%%%%%%%%%%%%%%%%%%%%%%%%%%%%%%%%%%%%%%%%%%%%%%%%%
\subsection{Command Line Processing}
\label{sec:commandline}

The effect of redirection files can also be achieved by invoking
the \LaTeX{} compiler with a more elaborate command line.
Most conveniently this should be done as part
of a shell script or a |Makefile|.

When using \textsf{childdoc} in the main file, the following
command lines effectively perform a redirection
(note that depending on the shell being used,
backslashes may have to be doubled: `|\|' $\to$ `|\\|'):
%
\begin{center}
|... -jobname "|\textit{target}|" |\\|"|[\textit{flags}]%
|\input{childdoc.def}\childdocforward[|\textit{main}|]{|\textit{dest}|}"|
\end{center}
%
Here \textit{target} is the name of the output file,
\textit{main} is the name of the main file
and \textit{dest} is the name of the main or child file to be processed
(all filenames without extensions).
The optional argument \textit{main} can be omitted
if \textit{main} matches \textit{dest}.
Optionally, compilation \textit{flags} can be defined via |\def| commands.
This command line makes the \TeX{} engine believe
it is compiling the file \textit{target}
whose content is specified as the latter parameter.
The provided code then forwards the processing to
\textit{main} or \textit{dest} as described in \secref{sec:forward}.

%%%%%%%%%%%%%%%%%%%%%%%%%%%%%%%%%%%%%%%%%%%%%%%%%%%%%%%%%%%%%%%%%%%%%%%%%%%%%%%%
\subsection{Include by Input}
\label{sec:input}

Including child documents by |\include| has some restrictions by design.
Most notably, the content of a child document always occupies
its own set of pages; pages cannot be shared between child documents.
Usually, this behaviour makes perfect sense
because each child document contain an essential part of the document.
However, in some situations it may be desirable to compose
a document from a collection of parts
without having mandatory page breaks between then.
For this case, the package
provides a mechanism to include parts
by |\input| which can also be processed individually.
However, by construction this mechanism
requires manual handling of the content to be output.

%%%%%%%%%%%%%%%%%%%%%%%%%%%%%%%%%%%%%%%%
\DescribeMacro{\ifchilddocmanual}
The main file should be prepared as usual, see \secref{sec:include}.
However, the document body must make a distinction
between processing of an individual part and of the main document, e.g.:
%
\begin{center}
\begin{tabular}{l}
|\ifchilddocmanual|\\
|\input{\childdocname}|\\
|\||else|\\
\textit{document body with }|\input{|\textit{part}|}|\\
|\||fi|
\end{tabular}
\end{center}
%
The conditional |\ifchilddocmanual| is true whenever
a part to be included by |\input| is being compiled,
and the name of the part is stored in |\childdocname|.

%%%%%%%%%%%%%%%%%%%%%%%%%%%%%%%%%%%%%%%%
\DescribeMacro{\childdocby}
Each part to be included by |\input| should start with:
%
\begin{center}
\begin{tabular}{l}
|\input{childdoc.def}|\\
|\childdocby{|\textit{main}|}|\\
\end{tabular}
\end{center}
%
The directive |\childdocby| is similar to |\childdocof|
described in \secref{sec:include},
but the subsequent selection of content must be done manually.
To that end, both |\ifchilddoc| and |\ifchilddocmanual|
will be true upon processing of a part,
and the name of the part is stored in |\childdocname|.
Note that |\jobname| will be set to the filename of the current part
so that each part receives an individual |.aux| file
that does not interfere with the |.aux| file(s) of the main document.
This behaviour can be altered by the alternative form
|\childdocby[*]{|\textit{main}|}| (with a non-empty optional argument)
which uses the |.aux| file of the main document
by setting |\jobname| to \textit{main}.

%%%%%%%%%%%%%%%%%%%%%%%%%%%%%%%%%%%%%%%%%%%%%%%%%%%%%%%%%%%%%%%%%%%%%%%%%%%%%%%%
\subsection{Driver Development}
\label{sec:driver}

The \textsf{childdoc} mechanism can also be use for the development
of definition files such as \LaTeX{} styles or classes.
This case differs from the above setup with multiple parts
included by |\include| in that no |\includeonly| should be invoked.
This can be achieved by starting the include file
(before |\ProvidesPackage|) with:
%
\begin{center}
\begin{tabular}{l}
|\input{childdoc.def}|\\
|\childdocforward{|\textit{main}|}|\\
\end{tabular}
\end{center}
%
or alternatively with:
%
\begin{center}
\begin{tabular}{l}
|\input{childdoc.def}|\\
|\childdocby{|\textit{main}|}|\\
\end{tabular}
\end{center}
%
Both forms have slightly different effects as described above.
The main file is prepared as usual, see \secref{sec:include}.

%%%%%%%%%%%%%%%%%%%%%%%%%%%%%%%%%%%%%%%%%%%%%%%%%%%%%%%%%%%%%%%%%%%%%%%%%%%%%%%%
\subsection{Legacy Detection}
\label{sec:detection}

The directive |\childdocmain| in the main file can detect
whether the complete document or merely a child is to be compiled
even without using the directive |\childdocof|.
This method is deprecated because it is less robust
and there is no compelling reason to use it;
it is merely provided for backward compatibility
and it may be removed in future versions.

If the detection mechanism is to be used,
it is mandatory to correctly specify
the filename of the main file as the argument of |\childdocmain|:
%
\begin{center}
\begin{tabular}{l}
|\input{childdoc.def}|\\
|\childdocmain{|\textit{main}|}|\\
\end{tabular}
\end{center}
%
If |\jobname| does not match the argument \textit{main} of |\childdocmain|,
it is assumed that |\jobname| points to the child file to be compiled.
When using |\childdocmain| with the main file specified as argument,
it suffices to start a child file
with just |\input{|\textit{main}|}|
without loading of the package and using |\childdocof|.
If instead all processing is done
with the appropriate \textsf{childdoc} directives,
the argument of \textit{main} of |\childdocmain| can be empty.

An alternative version of the command line processing described
in \secref{sec:commandline} using the detection mechanism reads:
%
\begin{center}
|... -jobname "|\textit{target}|" "|[\textit{flags}]%
[|\def\jobname{|\textit{dest}|}|]|\input{|\textit{main}|}"|
\end{center}

%%%%%%%%%%%%%%%%%%%%%%%%%%%%%%%%%%%%%%%%%%%%%%%%%%%%%%%%%%%%%%%%%%%%%%%%%%%%%%%%
\subsection{Manual Code}
\label{sec:manual}

In case one cannot be certain whether the definitions file |childdoc.def|
is installed on the target \TeX{} distribution
and one prefers not to ship it,
it is conceivable to paste a few relevant commands into the sources.

To that end, drop all statements |\input{childdoc.def}|
and perform the replacements as outlined below.
Instead of |\childdocmain{|\textit{main}|}| add the following code
to the top of the main file:
%
\begin{center}
\begin{tabular}{l}
|\||ifdefined\childdocname\endinput\||fi\newif\ifchilddoc|\\
|\edef\childdocname{\scantokens\expandafter{\jobname\noexpand}}|\\
|\def\childdocmain{|\textit{main}|}\||ifx\childdocmain\childdocname\||else|\\
|\childdoctrue\includeonly{\childdocname}\let\jobname\childdocmain\||fi|\\
\end{tabular}
\end{center}
%
Instead of |\childdocof{|\textit{main}|}| just include the main file
at the top of each child file:
%
\begin{center}
|\input{|\textit{main}|}|
\end{center}
%
A simple redirection |\childdocforward{|\textit{dest}|}| is achieved by:
%
\begin{center}
|\def\jobname{|\textit{dest}|}\input{\jobname}|
\end{center}
%
The redirection with prefix
|\childdocforwardprefix[|\textit{prefix}|]{|\textit{dest}|}|
is accomplished by:
%
\begin{center}
\begin{tabular}{l}
|{\edef\jobname{\scantokens\expandafter{\jobname\noexpand}}|\\
|\def\redirectjob |\textit{prefix}|#1~~~{\gdef\jobname{|\textit{dest}|#1}}|\\
|\expandafter\redirectjob\jobname~~~}\input{\jobname}|
\end{tabular}
\end{center}

In an alternative approach,
child documents can be compiled by a specific command line
without additional code or specific definitions:
%
\begin{center}
|... -jobname "|\textit{target}|" "|[\textit{flags}]%
|\includeonly{|\textit{dest}|}\input{|\textit{main}|}"|
\end{center}
%

%%%%%%%%%%%%%%%%%%%%%%%%%%%%%%%%%%%%%%%%%%%%%%%%%%%%%%%%%%%%%%%%%%%%%%%%%%%%%%%%
%%%%%%%%%%%%%%%%%%%%%%%%%%%%%%%%%%%%%%%%%%%%%%%%%%%%%%%%%%%%%%%%%%%%%%%%%%%%%%%%
\section{Information}

%%%%%%%%%%%%%%%%%%%%%%%%%%%%%%%%%%%%%%%%%%%%%%%%%%%%%%%%%%%%%%%%%%%%%%%%%%%%%%%%
\subsection{Copyright}

Copyright \copyright{} 2017--2018 Niklas Beisert

This work may be distributed and/or modified under the
conditions of the \LaTeX{} Project Public License, either version 1.3
of this license or (at your option) any later version.
The latest version of this license is in
  \url{http://www.latex-project.org/lppl.txt}
and version 1.3 or later is part of all distributions of \LaTeX{}
version 2005/12/01 or later.

This work has the LPPL maintenance status `maintained'.

The Current Maintainer of this work is Niklas Beisert.

This work consists of the files |README.txt|, |childdoc.ins| and |childdoc.dtx|
as well as the derived files |childdoc.def|, |cdocsamp.tex|
with |cdocsch1.tex|, |cdocsch2.tex|, |cdocspt3.tex|, |cdocspt4.tex|,
|cdocsdrf.tex|, |cdocsfn1.tex|, |cdocsfn2.tex|
as well as |childdoc.pdf|.

%%%%%%%%%%%%%%%%%%%%%%%%%%%%%%%%%%%%%%%%%%%%%%%%%%%%%%%%%%%%%%%%%%%%%%%%%%%%%%%%
\subsection{Files and Installation}

The package consists of the files:
%
\begin{center}
\begin{tabular}{ll}
    |README.txt|   & readme file \\
    |childdoc.ins| & installation file \\
    |childdoc.dtx| & source file \\
    |childdoc.def| & definition file \\
    |cdocsamp.tex| & sample main file \\
    |cdocsch1.tex| & sample include file \\
    |cdocsch2.tex| & sample include file \\
    |cdocspt3.tex| & sample part file \\
    |cdocspt4.tex| & sample part file \\
    |cdocsdrf.tex| & sample redirection file \\
    |cdocsfn1.tex| & sample redirection file \\
    |cdocsfn2.tex| & sample redirection file \\
    |childdoc.pdf| & manual
\end{tabular}
\end{center}
%
The distribution consists of the files
|README.txt|, |childdoc.ins| and |childdoc.dtx|.
%
\begin{itemize}
\item
Run (pdf)\LaTeX{} on |childdoc.dtx|
to compile the manual |childdoc.pdf| (this file).
\item
Run \LaTeX{} on |childdoc.ins| to create the definitions file |childdoc.def|
and the sample |cdocsamp.tex| with include files
|cdocsch1.tex|, |cdocsch2.tex|, |cdocspt3.tex|, |cdocspt4.tex|,
|cdocsdrf.tex|, |cdocsfn1.tex|, |cdocsfn2.tex|.
Then copy the file |childdoc.def| to an appropriate directory of your \LaTeX{}
distribution, e.g.\ \textit{texmf-root}|/tex/latex/childdoc|.
\end{itemize}

%%%%%%%%%%%%%%%%%%%%%%%%%%%%%%%%%%%%%%%%%%%%%%%%%%%%%%%%%%%%%%%%%%%%%%%%%%%%%%%%
\subsection{Related CTAN Packages}

There are several other packages which offer a similar functionality:
%
\begin{itemize}
\item
The packages
\href{http://ctan.org/pkg/docmute}{\textsf{docmute}},
\href{http://ctan.org/pkg/includex}{\textsf{includex}} and
\href{http://ctan.org/pkg/standalone}{\textsf{standalone}}
provide commands to include only the document body of
a child file thus allowing both files to be compiled individually.
\item
The packages \href{http://ctan.org/pkg/subdocs}{\textsf{subdocs}}
and \href{http://ctan.org/pkg/subfiles}{\textsf{subfiles}}
provide structures in which the main and child documents can be
encapsulated and allowing them to be compiled individually.
The inclusion mechanism is different from the conventional |\include|.
\item
The package \href{http://ctan.org/pkg/combine}{\textsf{combine}}
is an elaborate solution to combine several documents into one.
\end{itemize}
%
See also the CTAN topic \href{http://ctan.org/topic/subdocs}{\textsf{subdocs}}
for further related packages.
The present package differs from the above solutions in that
a document structure constructed with the conventional |\include| mechanism
just needs two extra commands at the top of every file
such that all constituent files can be compiled individually.

%%%%%%%%%%%%%%%%%%%%%%%%%%%%%%%%%%%%%%%%%%%%%%%%%%%%%%%%%%%%%%%%%%%%%%%%%%%%%%%%
%\subsection{Feature Suggestions}
%
%The following is a list of features which may be useful for future
%versions of this package:
%%
%\begin{itemize}
%\item
%\ldots
%\end{itemize}

%%%%%%%%%%%%%%%%%%%%%%%%%%%%%%%%%%%%%%%%%%%%%%%%%%%%%%%%%%%%%%%%%%%%%%%%%%%%%%%%
\subsection{Revision History}

%%%%%%%%%%%%%%%%%%%%%%%%%%%%%%%%%%%%%%%%
\paragraph{v2.0:} 2018/12/30

\begin{itemize}
\item
immediate forward processing
\item
added |\childdocby| mechanism
\item
manual restructured
\end{itemize}

%%%%%%%%%%%%%%%%%%%%%%%%%%%%%%%%%%%%%%%%
\paragraph{v1.6:} 2018/01/17

\begin{itemize}
\item
application for development of include files
\item
corrections to manual
\end{itemize}

%%%%%%%%%%%%%%%%%%%%%%%%%%%%%%%%%%%%%%%%
\paragraph{v1.5:} 2017/05/21

\begin{itemize}
\item
more complete structuring introduced
\item
|\childdocof| introduced
\item
|\childdoc| renamed to |\childdocmain|
\item
|\childredirect| renamed to |\childdocforward| and |\childdocforwardprefix|
and functionality expanded
\end{itemize}

%%%%%%%%%%%%%%%%%%%%%%%%%%%%%%%%%%%%%%%%
\paragraph{v1.0:} 2017/04/27

\begin{itemize}
\item
manual and install package
\item
first version published on CTAN
\end{itemize}

%%%%%%%%%%%%%%%%%%%%%%%%%%%%%%%%%%%%%%%%
\paragraph{v0.6:} 2017/04/26

\begin{itemize}
\item
redirection mechanism added
\end{itemize}

%%%%%%%%%%%%%%%%%%%%%%%%%%%%%%%%%%%%%%%%
\paragraph{v0.5:} 2017/04/26

\begin{itemize}
\item
functionality in definition file
\end{itemize}


%%%%%%%%%%%%%%%%%%%%%%%%%%%%%%%%%%%%%%%%%%%%%%%%%%%%%%%%%%%%%%%%%%%%%%%%%%%%%%%%
%%%%%%%%%%%%%%%%%%%%%%%%%%%%%%%%%%%%%%%%%%%%%%%%%%%%%%%%%%%%%%%%%%%%%%%%%%%%%%%%
%%%%%%%%%%%%%%%%%%%%%%%%%%%%%%%%%%%%%%%%%%%%%%%%%%%%%%%%%%%%%%%%%%%%%%%%%%%%%%%%
\appendix

\settowidth\MacroIndent{\rmfamily\scriptsize 000\ }

 \DocInput{childdoc.dtx}

\end{document}
%</driver>
% \fi
%
% %%%%%%%%%%%%%%%%%%%%%%%%%%%%%%%%%%%%%%%%%%%%%%%%%%%%%%%%%%%%%%%%%%%%%%%%%%%%%%
% %%%%%%%%%%%%%%%%%%%%%%%%%%%%%%%%%%%%%%%%%%%%%%%%%%%%%%%%%%%%%%%%%%%%%%%%%%%%%%
% \section{Sample}
%\iffalse
%<*samplemain>
%\fi
%
% The following presents a sample document
% with two chapters, two parts, a title page,
% a compile flag as well as three forwarding files to set the flag.
% It consists of eight |.tex| files:
% \begin{center}
% \begin{tabular}{ll}
% |cdocsamp.tex|&main file\\
% |cdocsch1.tex|&include file for chapter 1\\
% |cdocsch2.tex|&include file for chapter 2\\
% |cdocspt3.tex|&include file for part 3\\
% |cdocspt4.tex|&include file for part 4\\
% |cdocsdrf.tex|&forwarding file for main file in draft mode\\
% |cdocsfi1.tex|&forwarding file for final version of chapter 1\\
% |cdocsfi2.tex|&forwarding file for final version of chapter 2\\
% \end{tabular}
% \end{center}
% Each of the eight files can be compiled directly by the \LaTeX{} compiler.
%
% %%%%%%%%%%%%%%%%%%%%%%%%%%%%%%%%%%%%%%
% \paragraph{Main File.}
%
% The main file is called |cdocsamp.tex|.
%
% Load the \textsf{childdoc} definitions and
% declare the filename for the main document:
%    \begin{macrocode}
\input{childdoc.def}
\childdocmain{}
%    \end{macrocode}

% Optional override for |\version| flag:
%    \begin{macrocode}
%%\ifchilddoc\else\providecommand{\version}{draft}\fi
%    \end{macrocode}

% Define the default values for the |\version| flag
% (|final| for the main file and |draft| for childs):
%    \begin{macrocode}
\ifchilddoc
\providecommand{\version}{draft}
\else
\providecommand{\version}{final}
\fi
%    \end{macrocode}

% Load the standard document class:
%    \begin{macrocode}
\documentclass[12pt]{article}
%    \end{macrocode}

% Start the document body:
%    \begin{macrocode}
\begin{document}
%    \end{macrocode}

% Declare a title page.
% Print title, part of document being processed and version flag:
%    \begin{macrocode}
\addtocounter{page}{-1}
\begin{center}
{\LARGE\bfseries{}childdoc example\par}
\vspace{1cm}
\ifchilddoc
\ifchilddocmanual part\else chapter\fi:
`\childdocname' of `\childdocjob'\par
\else
main document: `\childdocjob'\par
\fi
version: \version\par
\end{center}
\newpage
%    \end{macrocode}

% Manually include selected file,
% otherwise process as usual:
%    \begin{macrocode}
\ifchilddocmanual
\section*{part `\childdocname'}
\input{\childdocname}
\else
%    \end{macrocode}

% Include the two chapters:
%    \begin{macrocode}
\include{cdocsch1}
\include{cdocsch2}
%    \end{macrocode}

% Include the two parts unless only chapters should be displayed:
%    \begin{macrocode}
\ifchilddoc\else
\section{part three}
\input{cdocspt3}
\section{part four}
\input{cdocspt4}
\fi
%    \end{macrocode}

% Process as usual until here:
%    \begin{macrocode}
\fi
%    \end{macrocode}

% End of document body:
%    \begin{macrocode}
\end{document}
%    \end{macrocode}
%\iffalse
%</samplemain>
%\fi
%
% %%%%%%%%%%%%%%%%%%%%%%%%%%%%%%%%%%%%%%
% \paragraph{Chapter Include Files.}
%
% The include files are called |cdocsch1.tex| and |cdocsch2.tex|.
%
%\iffalse
%<*samplechap1|samplechap2>
%\fi

% Optional override for |\version| flag:
%    \begin{macrocode}
%%\providecommand{\version}{final}
%    \end{macrocode}

% Include the main document:
%    \begin{macrocode}
\input{childdoc.def}
\childdocof{cdocsamp}
%    \end{macrocode}

%\iffalse
%</samplechap1|samplechap2>
%\fi
%
%\iffalse
%<*samplechap1>
%\fi
% Some text for chapter 1:
%    \begin{macrocode}
\section{one}
some text in chapter one
%    \end{macrocode}

%\iffalse
%</samplechap1>
%\fi
% Some text for chapter 2:
%\iffalse
%<*samplechap2>
%\fi
%    \begin{macrocode}
\section{two}
more text in chapter two
%    \end{macrocode}

%\iffalse
%</samplechap2>
%\fi
%
% %%%%%%%%%%%%%%%%%%%%%%%%%%%%%%%%%%%%%%
% \paragraph{Part Include Files.}
%
% The include files are called |cdocspt3.tex| and |cdocspt4.tex|.
%
%\iffalse
%<*samplepart3|samplepart4>
%\fi

% Optional override for |\version| flag:
%    \begin{macrocode}
%%\providecommand{\version}{final}
%    \end{macrocode}

% Include the main document:
%    \begin{macrocode}
\input{childdoc.def}
\childdocby{cdocsamp}
%    \end{macrocode}

%\iffalse
%</samplepart3|samplepart4>
%\fi
%
%\iffalse
%<*samplepart3>
%\fi
% Some text for part 3:
%    \begin{macrocode}
some text in part three
%    \end{macrocode}

%\iffalse
%</samplepart3>
%\fi
% Some text for part 4:
%\iffalse
%<*samplepart4>
%\fi
%    \begin{macrocode}
more text in part four
%    \end{macrocode}

%\iffalse
%</samplepart4>
%\fi
%
% %%%%%%%%%%%%%%%%%%%%%%%%%%%%%%%%%%%%%%
% \paragraph{Forwarding for a Complete Draft.}
%
% The following forwarding file |cdocsdrf.tex|
% compiles the main document in draft mode:
%\iffalse
%<*sampledraft>
%\fi
%    \begin{macrocode}
\def\version{draft}
\input{childdoc.def}
\childdocforward{cdocsamp}
%    \end{macrocode}

%\iffalse
%</sampledraft>
%\fi
%
% %%%%%%%%%%%%%%%%%%%%%%%%%%%%%%%%%%%%%%
% \paragraph{Forwarding for Final Version of the Chapters.}
%
% The following forwarding files |cdocsfn1.tex| and |cdocsfn2.tex|
% (with identical content)
% compile the final versions of the child documents
% |cdocsch1.tex| and |cdocsch2.tex|, respectively:
%\iffalse
%<*samplefinal>
%\fi
%    \begin{macrocode}
\def\version{final}
\input{childdoc.def}
\childdocforwardprefix[cdocsamp]{cdocsfn}{cdocsch}
%    \end{macrocode}

%\iffalse
%</samplefinal>
%\fi
%
% %%%%%%%%%%%%%%%%%%%%%%%%%%%%%%%%%%%%%%
% \paragraph{Command Line Processing.}
%
% The following three command lines generate the output files
% |cdocscld|, |cdocscl1| and |cdocscl2|
% which should be identical to
% |cdocsdrf|, |cdocsch1| and |cdocsfn2|, respectively:
% \begin{center}
% \begin{tabular}{l}
% |latex -jobname cdocscld \|\\
% |  "\def\version{draft}\input{childdoc.def}\childdocforward{cdocsamp}"|\\
% |latex -jobname cdocscl1 \|\\
% |  "\input{childdoc.def}\childdocforward[cdocsamp]{cdocsch1}"|\\
% |latex -jobname cdocscl2 \|\\
% |  "\def\version{final}\input{childdoc.def}\childdocforward{cdocsch2}"|
% \end{tabular}
% \end{center}
% Note that the trailing backslash on each first line
% merely continues the input to the second line
% (for convenient cut ant paste).
% Furthermore, the command |latex| can be replaced by any
% of its alternative versions such as |pdflatex|.
%
% %%%%%%%%%%%%%%%%%%%%%%%%%%%%%%%%%%%%%%%%%%%%%%%%%%%%%%%%%%%%%%%%%%%%%%%%%%%%%%
% %%%%%%%%%%%%%%%%%%%%%%%%%%%%%%%%%%%%%%%%%%%%%%%%%%%%%%%%%%%%%%%%%%%%%%%%%%%%%%
% \section{Implementation}
%\iffalse
%<*package>
%\fi
%
% This section describes the definitions file |childdoc.def|.

% The definitions cannot be loaded using |\usepackage| or |\RequirePackage|
% which has a mechanism to prevent loading a style file more than once.
% When loading the definitions by means of |\input|
% multiple instances have to be prevented manually:
%\iffalse
%This code needs to be before the `\ProvidesFile' directive
%which is defined at the beginning of this file.
%Therefore it is also placed there and commented out here.
%</package>
%<*discard>
%\fi
%    \begin{macrocode}
\ifdefined\childdocmain\endinput\fi
%    \end{macrocode}
%\iffalse
%</discard>
%<*package>
%\fi
%
% \macro{\ifchilddoc}
% \macro{\ifchilddocmanual}
% The conditional |\ifchilddoc| tells whether a
% child (true) or main (false) document is being compiled.
% The conditional |\ifchilddocmanual| tells whether
% the |\includeonly| mechanism is used (false) or
% the selection of child files must be performed manually (true).
% The definitions initialise to false:
%    \begin{macrocode}
\newif\ifchilddoc
\newif\ifchilddocmanual
%    \end{macrocode}

% \macro{\childdocname}
% \macro{\childdocjob}
% The macro |\childdocname| stores the name of the main document
% to be compiled. The macro |\childdocjob| stores the name of
% the document on which the \LaTeX{} compiler was originally invoked.
% The content of |\jobname| cannot be compared
% to filenames specified in the source due to different catcodes.
% The following code rescans |\jobname|, stores the result
% in |\childdocname| and saves a copy in |\childdocjob|:
%    \begin{macrocode}
\edef\childdocname{\scantokens\expandafter{\jobname\noexpand}}
\let\childdocjob\childdocname
%    \end{macrocode}

% \macro{\childdocdisable}
% The macro |\childdocdisable| prevents the main file
% from being processed more than once.
% At this stage, the main document command |\childdocmain|
% is assumed to be called once again where it should do nothing.
% Any subsequent call to it should prevent
% a secondary processing of the main document
% It overwrites the forwarding commands
% |\childdocof| and |\childdocforward|
% with empty macros to prevent further inclusions of the main document:
%    \begin{macrocode}
\newcommand{\childdocdisable}
{
  \renewcommand{\childdocmain}[1]{\renewcommand{\childdocmain}[1]{\endinput}}
  \renewcommand{\childdocof}[1]{}
  \renewcommand{\childdocby}[2][]{}
  \renewcommand{\childdocforward}[2][]{}
  \renewcommand{\childdocdisable}{}
}
%    \end{macrocode}

% \macro{\childdocmain}
% The macro |\childdocmain| is to be called at the top of the main file
% with nothing or the main filename (without extension) as argument.
% First, it breaks loops.
% If the argument is not empty and does not match |\childdocname|
% (which is set by the first inclusion of |childdoc.def|),
% |\ifchilddoc| is set to true, |\includeonly| is applied to the child file
% and |\jobname| is set to the main file
% (for proper handling of |.aux| files):
%    \begin{macrocode}
\newcommand{\childdocmain}[1]
{
  \childdocdisable\childdocmain{}
  \if?#1?\else
    \begingroup
      \def\childdoctmp{#1}
      \ifx\childdoctmp\childdocname
        \def\childdoctmp{}
      \else
        \def\childdoctmp
        {
          \childdoctrue
          \includeonly{\childdocname}
          \def\childdocjob{#1}
          \def\jobname{#1}
        }
      \fi
      \expandafter
    \endgroup
    \childdoctmp
  \fi
}
%    \end{macrocode}

% \macro{\childdocof}
% The command |\childdocof| redirects
% compilation to the main file |#1|.
%    \begin{macrocode}
\newcommand{\childdocof}[1]
{
  \childdocdisable
  \childdoctrue
  \includeonly{\childdocname}
  \def\jobname{#1}
  \def\childdocjob{#1}
  \input{#1}
}
%    \end{macrocode}

% \macro{\childdocby}
% The command |\childdocby| ....
%    \begin{macrocode}
\newcommand{\childdocby}[2][]
{
  \childdocdisable
  \childdoctrue
  \childdocmanualtrue
  \if?#1?\else
    \def\jobname{#2}
  \fi
  \def\childdocjob{#2}
  \input{#2}
  \endinput
}
%    \end{macrocode}

% \macro{\childdocforward}
% The command |\childdocforward| redirects
% compilation to the main file or
% (if the optional argument is given) a child file.
% Parameters are set as if the main file
% or a child file starting with |\childdocof| was compiled.
% Then compilation is handed over to the main file:
%    \begin{macrocode}
\newcommand{\childdocforward}[2][]
{
  \begingroup
    \if?#1?
      \def\childdoctmp
      {
        \def\childdocname{#2}
        \def\childdocjob{#2}
        \def\jobname{#2}
        \input{#2}
        \endinput
      }
    \else
      \def\childdoctmp
      {
        \childdocdisable
        \def\childdocname{#2}
        \childdoctrue
        \includeonly{#2}
        \def\childdocjob{#1}
        \def\jobname{#1}
        \input{#1}
        \endinput
      }
    \fi
    \expandafter
  \endgroup
  \childdoctmp
}
%    \end{macrocode}

% \macro{\childdocforwardprefix}
% The command |\childdocforwardprefix| redirects
% compilation to the main or a child file by means of a pattern.
% The prefix |#1| in the current filename is replaced by |#2|
% and the suffix of the current filename is kept
% (it is assumed that the filename does not contain the substring `|~~~|'
% which is used as a delimiter).
% Compilation is handed over to the new file by |\childdocforward|:
%    \begin{macrocode}
\newcommand{\childdocforwardprefix}[3][]
{
  \begingroup
    \def\childdocextract #2##1~~~{\def\childdoctmp{\childdocforward[#1]{#3##1}}}
    \expandafter\childdocextract\childdocname~~~
    \expandafter
  \endgroup
  \childdoctmp
}
%    \end{macrocode}

% \macro{\childdoc}
% The deprecated macro |\childdoc| is a legacy version of |\childdocmain|:
%    \begin{macrocode}
\newcommand{\childdoc}{\childdocmain}
%    \end{macrocode}

% \macro{\childdocredirect}
% The deprecated macro |\childdocredirect| is a legacy version
% of |\childdocforward| and |\childdocforwardprefix|:
%    \begin{macrocode}
\newcommand{\childdocredirect}[2][]
{
  \begingroup
    \if?#1?
      \def\childdoctmp{\childdocforward{#2}}
    \else
      \def\childdoctmp{\childdocforwardprefix{#1}{#2}}
    \fi
    \expandafter
  \endgroup
  \childdoctmp
}
%    \end{macrocode}

%\iffalse
%</package>
%\fi
%
\endinput
|\\
|\childdocforward{|\textit{main}|}|
\end{tabular}
\end{center}
%
Likewise, the following files |final|\textit{nn}|.tex|
compile the final version of the child document
|child|\textit{nn}|.tex|:
%
\begin{center}
\begin{tabular}{l}
|\def\version{final}|\\
|% \iffalse
%
% childdoc.dtx Copyright (C) 2017-2018 Niklas Beisert
%
% This work may be distributed and/or modified under the
% conditions of the LaTeX Project Public License, either version 1.3
% of this license or (at your option) any later version.
% The latest version of this license is in
%   http://www.latex-project.org/lppl.txt
% and version 1.3 or later is part of all distributions of LaTeX
% version 2005/12/01 or later.
%
% This work has the LPPL maintenance status `maintained'.
%
% The Current Maintainer of this work is Niklas Beisert.
%
% This work consists of the files childdoc.dtx and childdoc.ins
% and the derived files childdoc.def and cdocsamp.tex with
% cdocsch1.tex, cdocsch2.tex, cdocsdrf.tex, cdocsfn1.tex, cdocsfn2.tex.
%
%<package>\ifdefined\childdocmain\endinput\fi
%<package>\ProvidesFile{childdoc.def}[2018/12/30 v2.0 child document driver]
%<samplemain>\ProvidesFile{cdocsamp.tex}[2018/12/30 v2.0 sample for childdoc]
%<*driver>
%\ProvidesFile{childdoc.drv}[2018/12/30 v2.0 childdoc reference manual file]
\PassOptionsToClass{10pt,a4paper}{article}
\documentclass{ltxdoc}

\usepackage[margin=35mm]{geometry}
\usepackage{hyperref}
\usepackage{hyperxmp}
\usepackage[usenames]{color}

\hypersetup{colorlinks=true}
\hypersetup{pdfstartview=FitH}
\hypersetup{pdfpagemode=UseNone}
\hypersetup{pdfsource={}}
\hypersetup{pdflang={en-UK}}
\hypersetup{pdfcopyright={Copyright 2017-2018 Niklas Beisert.
  This work may be distributed and/or modified under the
  conditions of the LaTeX Project Public License, either version 1.3
  of this license or (at your option) any later version.}}
\hypersetup{pdflicenseurl={http://www.latex-project.org/lppl.txt}}
\hypersetup{pdfcontactaddress={ETH Zurich, ITP, HIT K,
  Wolfgang-Pauli-Strasse 27}}
\hypersetup{pdfcontactpostcode={8093}}
\hypersetup{pdfcontactcity={Zurich}}
\hypersetup{pdfcontactcountry={Switzerland}}
\hypersetup{pdfcontactemail={nbeisert@itp.phys.ethz.ch}}
\hypersetup{pdfcontacturl={http://people.phys.ethz.ch/\xmptilde nbeisert/}}

\newcommand{\secref}[1]{\hyperref[#1]{section \ref*{#1}}}

\parskip1ex
\parindent0pt
\let\olditemize\itemize
\def\itemize{\olditemize\parskip0pt}

\begin{document}

\title{The \textsf{childdoc} Package}
\hypersetup{pdftitle={The childdoc Package}}
\author{Niklas Beisert\\[2ex]
  Institut f\"ur Theoretische Physik\\
  Eidgen\"ossische Technische Hochschule Z\"urich\\
  Wolfgang-Pauli-Strasse 27, 8093 Z\"urich, Switzerland\\[1ex]
  \href{mailto:nbeisert@itp.phys.ethz.ch}
  {\texttt{nbeisert@itp.phys.ethz.ch}}}
\hypersetup{pdfauthor={Niklas Beisert}}
\hypersetup{pdfsubject={Manual for the LaTeX2e Package childdoc}}
\date{30 December 2018, \textsf{v2.0}}
\maketitle

\begin{abstract}\noindent
\textsf{childdoc} is a \LaTeXe{} package
that enables the direct compilation
of document sections included by |\include|
to individual files.
\end{abstract}

\begingroup
\parskip0ex
\tableofcontents
\endgroup

%%%%%%%%%%%%%%%%%%%%%%%%%%%%%%%%%%%%%%%%%%%%%%%%%%%%%%%%%%%%%%%%%%%%%%%%%%%%%%%%
%%%%%%%%%%%%%%%%%%%%%%%%%%%%%%%%%%%%%%%%%%%%%%%%%%%%%%%%%%%%%%%%%%%%%%%%%%%%%%%%
\section{Introduction}

\LaTeX{} provides a mechanism to structure a large document (such as a book)
into a main file and several child files (containing the chapters)
using the |\include| command.
This mechanism is beneficial for documents
which span hundreds of pages in order to
make the source file(s) more manageable.
Moreover, compilation can be restricted to
selected child files by means of the |\includeonly| command.
The latter feature can be used to reduce the compilation time while editing
(this was significantly more useful in the earlier days of \LaTeX{})
or to generate a smaller document which is easier to navigate.
Another application of |\includeonly| is to generate
documents consisting of selected parts of the complete document.

However, there are a few drawbacks of the plain |\include| mechanism:
\begin{itemize}
\item
The child files cannot be compiled on their own,
they can only be compiled via the main file.
A naive editing environment
(such as a text editor with an option
to have the current file processed by \LaTeX)
may require one to switch to the main file before compiling;
attempting to compile the child file produces errors.
\item
The main file must be modified (each time)
to adjust the |\includeonly| command
to the present needs. This easily leaves the main file in a messy state.
\item
The generated document will always carry the filename
of the main document. This is inconvenient if
several child files are to be compiled and
to be kept for distribution.
\end{itemize}

The present package provides a simple interface
to make child files individually compilable by \LaTeX{}.
Compiling a child file then has the same effect as compiling
the main file with an |\includeonly| command
to select the appropriate child.
Moreover the generated document will carry the name of the child
rather than the main file.
This resolves all three above issues.

This feature is meant to make the editing of books,
thesis documents and lecture notes somewhat more convenient.
However, the package can also be used efficiently for
composing a series of documents (such as exercise sheets)
which are typically distributed individually.
It then assists the author in generating the individual documents
(potentially in different versions)
as well as a document containing the collected series.
Another application is in developing style files
or other kinds of included material
where compilation of the style file could redirect
to a sample or test file.

%%%%%%%%%%%%%%%%%%%%%%%%%%%%%%%%%%%%%%%%%%%%%%%%%%%%%%%%%%%%%%%%%%%%%%%%%%%%%%%%
%%%%%%%%%%%%%%%%%%%%%%%%%%%%%%%%%%%%%%%%%%%%%%%%%%%%%%%%%%%%%%%%%%%%%%%%%%%%%%%%
\section{Usage}

First of all, the package \textsf{childdoc} is \emph{not} a standard
\LaTeXe{} |.sty| style file! Therefore it needs to be invoked in
a non-standard way.

%%%%%%%%%%%%%%%%%%%%%%%%%%%%%%%%%%%%%%%%%%%%%%%%%%%%%%%%%%%%%%%%%%%%%%%%%%%%%%%%
\subsection{Included Files}
\label{sec:include}

%%%%%%%%%%%%%%%%%%%%%%%%%%%%%%%%%%%%%%%%
\DescribeMacro{\childdocmain}
To use the package, add the commands
\begin{center}
\begin{tabular}{l}
|\input{childdoc.def}|\\
|\childdocmain{}|\\
\end{tabular}
\end{center}
at the very top of the main \LaTeX{} file,
in particular \emph{before} the |\documentclass| statement!
The argument of |\childdocmain| should be left empty
(but it must be present).

%%%%%%%%%%%%%%%%%%%%%%%%%%%%%%%%%%%%%%%%
\DescribeMacro{\childdocof}
Furthermore, add the commands
\begin{center}
\begin{tabular}{l}
|\input{childdoc.def}|\\
|\childdocof{|\textit{main}|}|\\
\end{tabular}
\end{center}
at the top of every child file \textit{child}
which is included by |\include{|\textit{child}|}|
from within the main file
(or at least for those files to be compiled individually).
The argument \textit{main} must be the filename of the main file.

There are a couple of
considerations in setting up the main and child documents:

%%%%%%%%%%%%%%%%%%%%%%%%%%%%%%%%%%%%%%%%
\paragraph{Restrictions.}

Please note the following restrictions:
\begin{itemize}
\item
|\childdocmain| must be called with one argument \textit{main}
to ensure compatibility with earlier version of the package.
It must either be empty (|\childdocmain{}|)
or precisely match the filename of the main file in which it is specified.
See \secref{sec:detection} for further information.
\item
The filename \textit{main} must be specified without the |.tex| extension.
\item
The filename \textit{main} is case sensitive
(even in case-insensitive file systems)
due to internal string comparison.
\item
The argument \textit{main} should be fully expanded, it cannot be a macro.
\item
Subdirectories and special characters should be avoided in filenames.
\item
The command |\childdocmain{|\textit{main}|}| must be followed by a whitespace.
It should not be followed immediately by another command
or by a comment mark `|%|'.
This is because the \TeX{} parser reads the token immediately following
the argument of |\childdocmain| and puts it
at the beginning of every child section;
however, a white\-space is ignored.
\end{itemize}

%%%%%%%%%%%%%%%%%%%%%%%%%%%%%%%%%%%%%%%%
\paragraph{Content of Main File.}

It is advisable to place all content in the child files included by |\include|.
Any output contained in the main file will appear in all child documents
unless suppressed manually;
it cannot be suppressed automatically by the |\includeonly| directive
and thus should normally be avoided.
A method to include some content in the main file
by means of conditional processing is described in \secref{sec:conditional}.

%%%%%%%%%%%%%%%%%%%%%%%%%%%%%%%%%%%%%%%%
\paragraph{Page Numbering.}

When only a part of the document is compiled,
the appropriate numbering of pages
(as well as other status parameters)
is determined from the |.aux| files.
The latter contain information from previous passes.
However this information needs to propagate through
all intermediate child documents.
Therefore the page numbering in child documents may well
be inconsistent until the complete document is compiled at least once.

A useful (if unconventional) way to always ensure a consistent
page numbering is to restart the numbering in each child document
and denote the pages by `\textit{child}|.|\textit{page}'
where \textit{child} represents the chapter/section number of the child file.
This can be achieved by the command
|\numberwithin{page}{|\textit{child}|}|
of the \textsf{amsmath} package
where \textit{child} can be |chapter| or |section|
depending on the chosen structuring.
Alternatively, one can modify the macro |\thepage| appropriately
and reset the counter |page| at the start of each child file.

%%%%%%%%%%%%%%%%%%%%%%%%%%%%%%%%%%%%%%%%%%%%%%%%%%%%%%%%%%%%%%%%%%%%%%%%%%%%%%%%
\subsection{Conditional Processing}
\label{sec:conditional}

The package provides a mechanism to compile different versions
of a document. To customise the versions further some conditional processing
can come in handy to distinguish which version is being compiled.
The package provides two macros to describe the compilation context:

%%%%%%%%%%%%%%%%%%%%%%%%%%%%%%%%%%%%%%%%
\DescribeMacro{\ifchilddoc}
The conditional |\ifchilddoc| distinguishes between the compilation of
child documents and the main document:
%
\begin{center}
|\ifchilddoc |\textit{child-code}| |[|\||else |\textit{main-code}]| \||fi|
\end{center}

%%%%%%%%%%%%%%%%%%%%%%%%%%%%%%%%%%%%%%%%
\DescribeMacro{\childdocname}
\DescribeMacro{\childdocjob}
The macro |\childdocname| contains the filename (without extension)
of the main or child file being processed.
Note that |\childdocjob| will always contain the name of the main file.

%%%%%%%%%%%%%%%%%%%%%%%%%%%%%%%%%%%%%%%%
\paragraph{Title Page.}

Conditional processing can be used to include a title or banner page
in the main document when proper precautions are taken.
Importantly, the code in the main file should ensure that the page counter
(as well as other status parameters which are stored in the |.aux| files)
takes the same value after the conditional processing.
Otherwise the page numbers may take divergent values
depending on which part is compiled.

For example, a title page could be declared by:
%
\begin{center}
\begin{tabular}{l}
|\ifchilddoc\||else|\\
|\addtocounter{page}{-1}|\\
\textit{code for title page}\\
|\newpage|\\
|\||fi|
\end{tabular}
\end{center}
%
A banner page for the child documents can be generated by:
%
\begin{center}
\begin{tabular}{l}
|\ifchilddoc|\\
|\addtocounter{page}{-1}|\\
\textit{code for banner page}\\
|\newpage|\\
|\||fi|
\end{tabular}
\end{center}
%
Here one could write a message such as:
\begin{center}
|This is the part \childdocname{} of \childdocjob{}.|
\end{center}

%%%%%%%%%%%%%%%%%%%%%%%%%%%%%%%%%%%%%%%%%%%%%%%%%%%%%%%%%%%%%%%%%%%%%%%%%%%%%%%%
\subsection{Flags}
\label{sec:flags}

The package makes it easy to generate different versions
of the main or child documents.
To this end compilation flags can be defined
and assigned different default values.
They will be particularly useful in conjunction
with the forwarding mechanism described in \secref{sec:forward}.

For example, it may be useful to have a flag |\version|
which can be set to |draft| or |final|.
The document source will contain some conditional code
depending on the value of |\version|.
Suppose further, the flag should default to |final| for the main file
and to |draft| for child files
which is a natural assignment for editing the document.
This is achieved by placing the following code
in the preamble of the main document
(below the |\childdocmain| directive):
%
\begin{center}
\begin{tabular}{l}
|\ifchilddoc|\\
|\providecommand{\version}{draft}|\\
|\||else|\\
|\providecommand{\version}{final}|\\
|\||fi|
\end{tabular}
\end{center}
%
The definition by |\providecommand| makes sure
that previous definitions are not overwritten.
Further statements |\providecommand{\version}{...}|
can thus be added before the above code to override it.

For the main file, one might add a line
(between |\childdocmain| and the above block)
%
\begin{center}
|%\ifchilddoc\||else\providecommand{\version}{draft}\||fi|
\end{center}
%
which can be uncommented to produce a draft version.
Likewise one can add a line to the very top of a child file
(above the |\childdocof{|\textit{main}|}| directive)
%
\begin{center}
|%\providecommand{\version}{final}|
\end{center}
%
which can be uncommented to produce the final version of this child document.

%%%%%%%%%%%%%%%%%%%%%%%%%%%%%%%%%%%%%%%%%%%%%%%%%%%%%%%%%%%%%%%%%%%%%%%%%%%%%%%%
\subsection{Forwarding}
\label{sec:forward}

Different versions of the main or child documents
using compilation flags as described in \secref{sec:flags}
can be (permanently) stored in different files
for convenient compilation, viewing and distribution.
To this end, the package defines a command
to pass on compilation to a different file:

%%%%%%%%%%%%%%%%%%%%%%%%%%%%%%%%%%%%%%%%
\DescribeMacro{\childdocforward}
The command |\childdocforward| redirects processing to
another source file:
%
\begin{center}
\begin{tabular}{l}
|\input{childdoc.def}|\\
|\childdocforward[|\textit{main}|]{|\textit{dest}|}|\\
\end{tabular}
\end{center}
%
The argument \textit{dest} is the destination file
(without extension).
It should be the main file or one of the child files.
Note that further \textsf{childdoc} directives
such as |\childdocof| and |\childdocforward|
in the indicated file will be processed in this form.
The optional argument \textit{main}
passes on directly to the main file \textit{main}
while pretending to compile the child \textit{dest}.
This form behaves as if \textit{dest}
issues |\childdocof{|\textit{main}|}| right away,
and no further \textsf{childdoc} directives will be processed.

%%%%%%%%%%%%%%%%%%%%%%%%%%%%%%%%%%%%%%%%
\DescribeMacro{\...prefix}
In the alternative form |\childdocforwardprefix|,
%
\begin{center}
\begin{tabular}{l}
|\input{childdoc.def}|\\
|\childdocforwardprefix[|\textit{main}|]{|\textit{prefix}|}{|\textit{dest}|}|
\end{tabular}
\end{center}
%
the destination file is determined by a pattern
depending on the current file:
To make this work, the current file must be called
`{\textit{prefix}\hspace{0.2em}\textit{suffix}}'
with \textit{prefix} matching precisely the argument.
Processing is then passed on to the file
`{\textit{dest}\hspace{0.2em}\textit{suffix}}'.
Surely, the same effect is achieved by
directly specifying the
argument `{\textit{dest}\hspace{0.2em}\textit{suffix}}'
in the first form.
However, that requires to set up a different file
for each child. With the alternative form of the command
all these files can have exactly the same content
which simplifies setting them up and maintaining them.

For example, the following file |draft.tex|
with a compilation flag |\version| as described in \secref{sec:flags}
compiles the main document as a draft:
%
\begin{center}
\begin{tabular}{l}
|\def\version{draft}|\\
|\input{childdoc.def}|\\
|\childdocforward{|\textit{main}|}|
\end{tabular}
\end{center}
%
Likewise, the following files |final|\textit{nn}|.tex|
compile the final version of the child document
|child|\textit{nn}|.tex|:
%
\begin{center}
\begin{tabular}{l}
|\def\version{final}|\\
|\input{childdoc.def}|\\
|\childdocforwardprefix{final}{child}|
\end{tabular}
\end{center}
%

Note that when several versions of a main file and/or of each child file
are to be generated, it may be convenient to set up a |Makefile| or
shell script to automatise the process.

%%%%%%%%%%%%%%%%%%%%%%%%%%%%%%%%%%%%%%%%%%%%%%%%%%%%%%%%%%%%%%%%%%%%%%%%%%%%%%%%
\subsection{Command Line Processing}
\label{sec:commandline}

The effect of redirection files can also be achieved by invoking
the \LaTeX{} compiler with a more elaborate command line.
Most conveniently this should be done as part
of a shell script or a |Makefile|.

When using \textsf{childdoc} in the main file, the following
command lines effectively perform a redirection
(note that depending on the shell being used,
backslashes may have to be doubled: `|\|' $\to$ `|\\|'):
%
\begin{center}
|... -jobname "|\textit{target}|" |\\|"|[\textit{flags}]%
|\input{childdoc.def}\childdocforward[|\textit{main}|]{|\textit{dest}|}"|
\end{center}
%
Here \textit{target} is the name of the output file,
\textit{main} is the name of the main file
and \textit{dest} is the name of the main or child file to be processed
(all filenames without extensions).
The optional argument \textit{main} can be omitted
if \textit{main} matches \textit{dest}.
Optionally, compilation \textit{flags} can be defined via |\def| commands.
This command line makes the \TeX{} engine believe
it is compiling the file \textit{target}
whose content is specified as the latter parameter.
The provided code then forwards the processing to
\textit{main} or \textit{dest} as described in \secref{sec:forward}.

%%%%%%%%%%%%%%%%%%%%%%%%%%%%%%%%%%%%%%%%%%%%%%%%%%%%%%%%%%%%%%%%%%%%%%%%%%%%%%%%
\subsection{Include by Input}
\label{sec:input}

Including child documents by |\include| has some restrictions by design.
Most notably, the content of a child document always occupies
its own set of pages; pages cannot be shared between child documents.
Usually, this behaviour makes perfect sense
because each child document contain an essential part of the document.
However, in some situations it may be desirable to compose
a document from a collection of parts
without having mandatory page breaks between then.
For this case, the package
provides a mechanism to include parts
by |\input| which can also be processed individually.
However, by construction this mechanism
requires manual handling of the content to be output.

%%%%%%%%%%%%%%%%%%%%%%%%%%%%%%%%%%%%%%%%
\DescribeMacro{\ifchilddocmanual}
The main file should be prepared as usual, see \secref{sec:include}.
However, the document body must make a distinction
between processing of an individual part and of the main document, e.g.:
%
\begin{center}
\begin{tabular}{l}
|\ifchilddocmanual|\\
|\input{\childdocname}|\\
|\||else|\\
\textit{document body with }|\input{|\textit{part}|}|\\
|\||fi|
\end{tabular}
\end{center}
%
The conditional |\ifchilddocmanual| is true whenever
a part to be included by |\input| is being compiled,
and the name of the part is stored in |\childdocname|.

%%%%%%%%%%%%%%%%%%%%%%%%%%%%%%%%%%%%%%%%
\DescribeMacro{\childdocby}
Each part to be included by |\input| should start with:
%
\begin{center}
\begin{tabular}{l}
|\input{childdoc.def}|\\
|\childdocby{|\textit{main}|}|\\
\end{tabular}
\end{center}
%
The directive |\childdocby| is similar to |\childdocof|
described in \secref{sec:include},
but the subsequent selection of content must be done manually.
To that end, both |\ifchilddoc| and |\ifchilddocmanual|
will be true upon processing of a part,
and the name of the part is stored in |\childdocname|.
Note that |\jobname| will be set to the filename of the current part
so that each part receives an individual |.aux| file
that does not interfere with the |.aux| file(s) of the main document.
This behaviour can be altered by the alternative form
|\childdocby[*]{|\textit{main}|}| (with a non-empty optional argument)
which uses the |.aux| file of the main document
by setting |\jobname| to \textit{main}.

%%%%%%%%%%%%%%%%%%%%%%%%%%%%%%%%%%%%%%%%%%%%%%%%%%%%%%%%%%%%%%%%%%%%%%%%%%%%%%%%
\subsection{Driver Development}
\label{sec:driver}

The \textsf{childdoc} mechanism can also be use for the development
of definition files such as \LaTeX{} styles or classes.
This case differs from the above setup with multiple parts
included by |\include| in that no |\includeonly| should be invoked.
This can be achieved by starting the include file
(before |\ProvidesPackage|) with:
%
\begin{center}
\begin{tabular}{l}
|\input{childdoc.def}|\\
|\childdocforward{|\textit{main}|}|\\
\end{tabular}
\end{center}
%
or alternatively with:
%
\begin{center}
\begin{tabular}{l}
|\input{childdoc.def}|\\
|\childdocby{|\textit{main}|}|\\
\end{tabular}
\end{center}
%
Both forms have slightly different effects as described above.
The main file is prepared as usual, see \secref{sec:include}.

%%%%%%%%%%%%%%%%%%%%%%%%%%%%%%%%%%%%%%%%%%%%%%%%%%%%%%%%%%%%%%%%%%%%%%%%%%%%%%%%
\subsection{Legacy Detection}
\label{sec:detection}

The directive |\childdocmain| in the main file can detect
whether the complete document or merely a child is to be compiled
even without using the directive |\childdocof|.
This method is deprecated because it is less robust
and there is no compelling reason to use it;
it is merely provided for backward compatibility
and it may be removed in future versions.

If the detection mechanism is to be used,
it is mandatory to correctly specify
the filename of the main file as the argument of |\childdocmain|:
%
\begin{center}
\begin{tabular}{l}
|\input{childdoc.def}|\\
|\childdocmain{|\textit{main}|}|\\
\end{tabular}
\end{center}
%
If |\jobname| does not match the argument \textit{main} of |\childdocmain|,
it is assumed that |\jobname| points to the child file to be compiled.
When using |\childdocmain| with the main file specified as argument,
it suffices to start a child file
with just |\input{|\textit{main}|}|
without loading of the package and using |\childdocof|.
If instead all processing is done
with the appropriate \textsf{childdoc} directives,
the argument of \textit{main} of |\childdocmain| can be empty.

An alternative version of the command line processing described
in \secref{sec:commandline} using the detection mechanism reads:
%
\begin{center}
|... -jobname "|\textit{target}|" "|[\textit{flags}]%
[|\def\jobname{|\textit{dest}|}|]|\input{|\textit{main}|}"|
\end{center}

%%%%%%%%%%%%%%%%%%%%%%%%%%%%%%%%%%%%%%%%%%%%%%%%%%%%%%%%%%%%%%%%%%%%%%%%%%%%%%%%
\subsection{Manual Code}
\label{sec:manual}

In case one cannot be certain whether the definitions file |childdoc.def|
is installed on the target \TeX{} distribution
and one prefers not to ship it,
it is conceivable to paste a few relevant commands into the sources.

To that end, drop all statements |\input{childdoc.def}|
and perform the replacements as outlined below.
Instead of |\childdocmain{|\textit{main}|}| add the following code
to the top of the main file:
%
\begin{center}
\begin{tabular}{l}
|\||ifdefined\childdocname\endinput\||fi\newif\ifchilddoc|\\
|\edef\childdocname{\scantokens\expandafter{\jobname\noexpand}}|\\
|\def\childdocmain{|\textit{main}|}\||ifx\childdocmain\childdocname\||else|\\
|\childdoctrue\includeonly{\childdocname}\let\jobname\childdocmain\||fi|\\
\end{tabular}
\end{center}
%
Instead of |\childdocof{|\textit{main}|}| just include the main file
at the top of each child file:
%
\begin{center}
|\input{|\textit{main}|}|
\end{center}
%
A simple redirection |\childdocforward{|\textit{dest}|}| is achieved by:
%
\begin{center}
|\def\jobname{|\textit{dest}|}\input{\jobname}|
\end{center}
%
The redirection with prefix
|\childdocforwardprefix[|\textit{prefix}|]{|\textit{dest}|}|
is accomplished by:
%
\begin{center}
\begin{tabular}{l}
|{\edef\jobname{\scantokens\expandafter{\jobname\noexpand}}|\\
|\def\redirectjob |\textit{prefix}|#1~~~{\gdef\jobname{|\textit{dest}|#1}}|\\
|\expandafter\redirectjob\jobname~~~}\input{\jobname}|
\end{tabular}
\end{center}

In an alternative approach,
child documents can be compiled by a specific command line
without additional code or specific definitions:
%
\begin{center}
|... -jobname "|\textit{target}|" "|[\textit{flags}]%
|\includeonly{|\textit{dest}|}\input{|\textit{main}|}"|
\end{center}
%

%%%%%%%%%%%%%%%%%%%%%%%%%%%%%%%%%%%%%%%%%%%%%%%%%%%%%%%%%%%%%%%%%%%%%%%%%%%%%%%%
%%%%%%%%%%%%%%%%%%%%%%%%%%%%%%%%%%%%%%%%%%%%%%%%%%%%%%%%%%%%%%%%%%%%%%%%%%%%%%%%
\section{Information}

%%%%%%%%%%%%%%%%%%%%%%%%%%%%%%%%%%%%%%%%%%%%%%%%%%%%%%%%%%%%%%%%%%%%%%%%%%%%%%%%
\subsection{Copyright}

Copyright \copyright{} 2017--2018 Niklas Beisert

This work may be distributed and/or modified under the
conditions of the \LaTeX{} Project Public License, either version 1.3
of this license or (at your option) any later version.
The latest version of this license is in
  \url{http://www.latex-project.org/lppl.txt}
and version 1.3 or later is part of all distributions of \LaTeX{}
version 2005/12/01 or later.

This work has the LPPL maintenance status `maintained'.

The Current Maintainer of this work is Niklas Beisert.

This work consists of the files |README.txt|, |childdoc.ins| and |childdoc.dtx|
as well as the derived files |childdoc.def|, |cdocsamp.tex|
with |cdocsch1.tex|, |cdocsch2.tex|, |cdocspt3.tex|, |cdocspt4.tex|,
|cdocsdrf.tex|, |cdocsfn1.tex|, |cdocsfn2.tex|
as well as |childdoc.pdf|.

%%%%%%%%%%%%%%%%%%%%%%%%%%%%%%%%%%%%%%%%%%%%%%%%%%%%%%%%%%%%%%%%%%%%%%%%%%%%%%%%
\subsection{Files and Installation}

The package consists of the files:
%
\begin{center}
\begin{tabular}{ll}
    |README.txt|   & readme file \\
    |childdoc.ins| & installation file \\
    |childdoc.dtx| & source file \\
    |childdoc.def| & definition file \\
    |cdocsamp.tex| & sample main file \\
    |cdocsch1.tex| & sample include file \\
    |cdocsch2.tex| & sample include file \\
    |cdocspt3.tex| & sample part file \\
    |cdocspt4.tex| & sample part file \\
    |cdocsdrf.tex| & sample redirection file \\
    |cdocsfn1.tex| & sample redirection file \\
    |cdocsfn2.tex| & sample redirection file \\
    |childdoc.pdf| & manual
\end{tabular}
\end{center}
%
The distribution consists of the files
|README.txt|, |childdoc.ins| and |childdoc.dtx|.
%
\begin{itemize}
\item
Run (pdf)\LaTeX{} on |childdoc.dtx|
to compile the manual |childdoc.pdf| (this file).
\item
Run \LaTeX{} on |childdoc.ins| to create the definitions file |childdoc.def|
and the sample |cdocsamp.tex| with include files
|cdocsch1.tex|, |cdocsch2.tex|, |cdocspt3.tex|, |cdocspt4.tex|,
|cdocsdrf.tex|, |cdocsfn1.tex|, |cdocsfn2.tex|.
Then copy the file |childdoc.def| to an appropriate directory of your \LaTeX{}
distribution, e.g.\ \textit{texmf-root}|/tex/latex/childdoc|.
\end{itemize}

%%%%%%%%%%%%%%%%%%%%%%%%%%%%%%%%%%%%%%%%%%%%%%%%%%%%%%%%%%%%%%%%%%%%%%%%%%%%%%%%
\subsection{Related CTAN Packages}

There are several other packages which offer a similar functionality:
%
\begin{itemize}
\item
The packages
\href{http://ctan.org/pkg/docmute}{\textsf{docmute}},
\href{http://ctan.org/pkg/includex}{\textsf{includex}} and
\href{http://ctan.org/pkg/standalone}{\textsf{standalone}}
provide commands to include only the document body of
a child file thus allowing both files to be compiled individually.
\item
The packages \href{http://ctan.org/pkg/subdocs}{\textsf{subdocs}}
and \href{http://ctan.org/pkg/subfiles}{\textsf{subfiles}}
provide structures in which the main and child documents can be
encapsulated and allowing them to be compiled individually.
The inclusion mechanism is different from the conventional |\include|.
\item
The package \href{http://ctan.org/pkg/combine}{\textsf{combine}}
is an elaborate solution to combine several documents into one.
\end{itemize}
%
See also the CTAN topic \href{http://ctan.org/topic/subdocs}{\textsf{subdocs}}
for further related packages.
The present package differs from the above solutions in that
a document structure constructed with the conventional |\include| mechanism
just needs two extra commands at the top of every file
such that all constituent files can be compiled individually.

%%%%%%%%%%%%%%%%%%%%%%%%%%%%%%%%%%%%%%%%%%%%%%%%%%%%%%%%%%%%%%%%%%%%%%%%%%%%%%%%
%\subsection{Feature Suggestions}
%
%The following is a list of features which may be useful for future
%versions of this package:
%%
%\begin{itemize}
%\item
%\ldots
%\end{itemize}

%%%%%%%%%%%%%%%%%%%%%%%%%%%%%%%%%%%%%%%%%%%%%%%%%%%%%%%%%%%%%%%%%%%%%%%%%%%%%%%%
\subsection{Revision History}

%%%%%%%%%%%%%%%%%%%%%%%%%%%%%%%%%%%%%%%%
\paragraph{v2.0:} 2018/12/30

\begin{itemize}
\item
immediate forward processing
\item
added |\childdocby| mechanism
\item
manual restructured
\end{itemize}

%%%%%%%%%%%%%%%%%%%%%%%%%%%%%%%%%%%%%%%%
\paragraph{v1.6:} 2018/01/17

\begin{itemize}
\item
application for development of include files
\item
corrections to manual
\end{itemize}

%%%%%%%%%%%%%%%%%%%%%%%%%%%%%%%%%%%%%%%%
\paragraph{v1.5:} 2017/05/21

\begin{itemize}
\item
more complete structuring introduced
\item
|\childdocof| introduced
\item
|\childdoc| renamed to |\childdocmain|
\item
|\childredirect| renamed to |\childdocforward| and |\childdocforwardprefix|
and functionality expanded
\end{itemize}

%%%%%%%%%%%%%%%%%%%%%%%%%%%%%%%%%%%%%%%%
\paragraph{v1.0:} 2017/04/27

\begin{itemize}
\item
manual and install package
\item
first version published on CTAN
\end{itemize}

%%%%%%%%%%%%%%%%%%%%%%%%%%%%%%%%%%%%%%%%
\paragraph{v0.6:} 2017/04/26

\begin{itemize}
\item
redirection mechanism added
\end{itemize}

%%%%%%%%%%%%%%%%%%%%%%%%%%%%%%%%%%%%%%%%
\paragraph{v0.5:} 2017/04/26

\begin{itemize}
\item
functionality in definition file
\end{itemize}


%%%%%%%%%%%%%%%%%%%%%%%%%%%%%%%%%%%%%%%%%%%%%%%%%%%%%%%%%%%%%%%%%%%%%%%%%%%%%%%%
%%%%%%%%%%%%%%%%%%%%%%%%%%%%%%%%%%%%%%%%%%%%%%%%%%%%%%%%%%%%%%%%%%%%%%%%%%%%%%%%
%%%%%%%%%%%%%%%%%%%%%%%%%%%%%%%%%%%%%%%%%%%%%%%%%%%%%%%%%%%%%%%%%%%%%%%%%%%%%%%%
\appendix

\settowidth\MacroIndent{\rmfamily\scriptsize 000\ }

 \DocInput{childdoc.dtx}

\end{document}
%</driver>
% \fi
%
% %%%%%%%%%%%%%%%%%%%%%%%%%%%%%%%%%%%%%%%%%%%%%%%%%%%%%%%%%%%%%%%%%%%%%%%%%%%%%%
% %%%%%%%%%%%%%%%%%%%%%%%%%%%%%%%%%%%%%%%%%%%%%%%%%%%%%%%%%%%%%%%%%%%%%%%%%%%%%%
% \section{Sample}
%\iffalse
%<*samplemain>
%\fi
%
% The following presents a sample document
% with two chapters, two parts, a title page,
% a compile flag as well as three forwarding files to set the flag.
% It consists of eight |.tex| files:
% \begin{center}
% \begin{tabular}{ll}
% |cdocsamp.tex|&main file\\
% |cdocsch1.tex|&include file for chapter 1\\
% |cdocsch2.tex|&include file for chapter 2\\
% |cdocspt3.tex|&include file for part 3\\
% |cdocspt4.tex|&include file for part 4\\
% |cdocsdrf.tex|&forwarding file for main file in draft mode\\
% |cdocsfi1.tex|&forwarding file for final version of chapter 1\\
% |cdocsfi2.tex|&forwarding file for final version of chapter 2\\
% \end{tabular}
% \end{center}
% Each of the eight files can be compiled directly by the \LaTeX{} compiler.
%
% %%%%%%%%%%%%%%%%%%%%%%%%%%%%%%%%%%%%%%
% \paragraph{Main File.}
%
% The main file is called |cdocsamp.tex|.
%
% Load the \textsf{childdoc} definitions and
% declare the filename for the main document:
%    \begin{macrocode}
\input{childdoc.def}
\childdocmain{}
%    \end{macrocode}

% Optional override for |\version| flag:
%    \begin{macrocode}
%%\ifchilddoc\else\providecommand{\version}{draft}\fi
%    \end{macrocode}

% Define the default values for the |\version| flag
% (|final| for the main file and |draft| for childs):
%    \begin{macrocode}
\ifchilddoc
\providecommand{\version}{draft}
\else
\providecommand{\version}{final}
\fi
%    \end{macrocode}

% Load the standard document class:
%    \begin{macrocode}
\documentclass[12pt]{article}
%    \end{macrocode}

% Start the document body:
%    \begin{macrocode}
\begin{document}
%    \end{macrocode}

% Declare a title page.
% Print title, part of document being processed and version flag:
%    \begin{macrocode}
\addtocounter{page}{-1}
\begin{center}
{\LARGE\bfseries{}childdoc example\par}
\vspace{1cm}
\ifchilddoc
\ifchilddocmanual part\else chapter\fi:
`\childdocname' of `\childdocjob'\par
\else
main document: `\childdocjob'\par
\fi
version: \version\par
\end{center}
\newpage
%    \end{macrocode}

% Manually include selected file,
% otherwise process as usual:
%    \begin{macrocode}
\ifchilddocmanual
\section*{part `\childdocname'}
\input{\childdocname}
\else
%    \end{macrocode}

% Include the two chapters:
%    \begin{macrocode}
\include{cdocsch1}
\include{cdocsch2}
%    \end{macrocode}

% Include the two parts unless only chapters should be displayed:
%    \begin{macrocode}
\ifchilddoc\else
\section{part three}
\input{cdocspt3}
\section{part four}
\input{cdocspt4}
\fi
%    \end{macrocode}

% Process as usual until here:
%    \begin{macrocode}
\fi
%    \end{macrocode}

% End of document body:
%    \begin{macrocode}
\end{document}
%    \end{macrocode}
%\iffalse
%</samplemain>
%\fi
%
% %%%%%%%%%%%%%%%%%%%%%%%%%%%%%%%%%%%%%%
% \paragraph{Chapter Include Files.}
%
% The include files are called |cdocsch1.tex| and |cdocsch2.tex|.
%
%\iffalse
%<*samplechap1|samplechap2>
%\fi

% Optional override for |\version| flag:
%    \begin{macrocode}
%%\providecommand{\version}{final}
%    \end{macrocode}

% Include the main document:
%    \begin{macrocode}
\input{childdoc.def}
\childdocof{cdocsamp}
%    \end{macrocode}

%\iffalse
%</samplechap1|samplechap2>
%\fi
%
%\iffalse
%<*samplechap1>
%\fi
% Some text for chapter 1:
%    \begin{macrocode}
\section{one}
some text in chapter one
%    \end{macrocode}

%\iffalse
%</samplechap1>
%\fi
% Some text for chapter 2:
%\iffalse
%<*samplechap2>
%\fi
%    \begin{macrocode}
\section{two}
more text in chapter two
%    \end{macrocode}

%\iffalse
%</samplechap2>
%\fi
%
% %%%%%%%%%%%%%%%%%%%%%%%%%%%%%%%%%%%%%%
% \paragraph{Part Include Files.}
%
% The include files are called |cdocspt3.tex| and |cdocspt4.tex|.
%
%\iffalse
%<*samplepart3|samplepart4>
%\fi

% Optional override for |\version| flag:
%    \begin{macrocode}
%%\providecommand{\version}{final}
%    \end{macrocode}

% Include the main document:
%    \begin{macrocode}
\input{childdoc.def}
\childdocby{cdocsamp}
%    \end{macrocode}

%\iffalse
%</samplepart3|samplepart4>
%\fi
%
%\iffalse
%<*samplepart3>
%\fi
% Some text for part 3:
%    \begin{macrocode}
some text in part three
%    \end{macrocode}

%\iffalse
%</samplepart3>
%\fi
% Some text for part 4:
%\iffalse
%<*samplepart4>
%\fi
%    \begin{macrocode}
more text in part four
%    \end{macrocode}

%\iffalse
%</samplepart4>
%\fi
%
% %%%%%%%%%%%%%%%%%%%%%%%%%%%%%%%%%%%%%%
% \paragraph{Forwarding for a Complete Draft.}
%
% The following forwarding file |cdocsdrf.tex|
% compiles the main document in draft mode:
%\iffalse
%<*sampledraft>
%\fi
%    \begin{macrocode}
\def\version{draft}
\input{childdoc.def}
\childdocforward{cdocsamp}
%    \end{macrocode}

%\iffalse
%</sampledraft>
%\fi
%
% %%%%%%%%%%%%%%%%%%%%%%%%%%%%%%%%%%%%%%
% \paragraph{Forwarding for Final Version of the Chapters.}
%
% The following forwarding files |cdocsfn1.tex| and |cdocsfn2.tex|
% (with identical content)
% compile the final versions of the child documents
% |cdocsch1.tex| and |cdocsch2.tex|, respectively:
%\iffalse
%<*samplefinal>
%\fi
%    \begin{macrocode}
\def\version{final}
\input{childdoc.def}
\childdocforwardprefix[cdocsamp]{cdocsfn}{cdocsch}
%    \end{macrocode}

%\iffalse
%</samplefinal>
%\fi
%
% %%%%%%%%%%%%%%%%%%%%%%%%%%%%%%%%%%%%%%
% \paragraph{Command Line Processing.}
%
% The following three command lines generate the output files
% |cdocscld|, |cdocscl1| and |cdocscl2|
% which should be identical to
% |cdocsdrf|, |cdocsch1| and |cdocsfn2|, respectively:
% \begin{center}
% \begin{tabular}{l}
% |latex -jobname cdocscld \|\\
% |  "\def\version{draft}\input{childdoc.def}\childdocforward{cdocsamp}"|\\
% |latex -jobname cdocscl1 \|\\
% |  "\input{childdoc.def}\childdocforward[cdocsamp]{cdocsch1}"|\\
% |latex -jobname cdocscl2 \|\\
% |  "\def\version{final}\input{childdoc.def}\childdocforward{cdocsch2}"|
% \end{tabular}
% \end{center}
% Note that the trailing backslash on each first line
% merely continues the input to the second line
% (for convenient cut ant paste).
% Furthermore, the command |latex| can be replaced by any
% of its alternative versions such as |pdflatex|.
%
% %%%%%%%%%%%%%%%%%%%%%%%%%%%%%%%%%%%%%%%%%%%%%%%%%%%%%%%%%%%%%%%%%%%%%%%%%%%%%%
% %%%%%%%%%%%%%%%%%%%%%%%%%%%%%%%%%%%%%%%%%%%%%%%%%%%%%%%%%%%%%%%%%%%%%%%%%%%%%%
% \section{Implementation}
%\iffalse
%<*package>
%\fi
%
% This section describes the definitions file |childdoc.def|.

% The definitions cannot be loaded using |\usepackage| or |\RequirePackage|
% which has a mechanism to prevent loading a style file more than once.
% When loading the definitions by means of |\input|
% multiple instances have to be prevented manually:
%\iffalse
%This code needs to be before the `\ProvidesFile' directive
%which is defined at the beginning of this file.
%Therefore it is also placed there and commented out here.
%</package>
%<*discard>
%\fi
%    \begin{macrocode}
\ifdefined\childdocmain\endinput\fi
%    \end{macrocode}
%\iffalse
%</discard>
%<*package>
%\fi
%
% \macro{\ifchilddoc}
% \macro{\ifchilddocmanual}
% The conditional |\ifchilddoc| tells whether a
% child (true) or main (false) document is being compiled.
% The conditional |\ifchilddocmanual| tells whether
% the |\includeonly| mechanism is used (false) or
% the selection of child files must be performed manually (true).
% The definitions initialise to false:
%    \begin{macrocode}
\newif\ifchilddoc
\newif\ifchilddocmanual
%    \end{macrocode}

% \macro{\childdocname}
% \macro{\childdocjob}
% The macro |\childdocname| stores the name of the main document
% to be compiled. The macro |\childdocjob| stores the name of
% the document on which the \LaTeX{} compiler was originally invoked.
% The content of |\jobname| cannot be compared
% to filenames specified in the source due to different catcodes.
% The following code rescans |\jobname|, stores the result
% in |\childdocname| and saves a copy in |\childdocjob|:
%    \begin{macrocode}
\edef\childdocname{\scantokens\expandafter{\jobname\noexpand}}
\let\childdocjob\childdocname
%    \end{macrocode}

% \macro{\childdocdisable}
% The macro |\childdocdisable| prevents the main file
% from being processed more than once.
% At this stage, the main document command |\childdocmain|
% is assumed to be called once again where it should do nothing.
% Any subsequent call to it should prevent
% a secondary processing of the main document
% It overwrites the forwarding commands
% |\childdocof| and |\childdocforward|
% with empty macros to prevent further inclusions of the main document:
%    \begin{macrocode}
\newcommand{\childdocdisable}
{
  \renewcommand{\childdocmain}[1]{\renewcommand{\childdocmain}[1]{\endinput}}
  \renewcommand{\childdocof}[1]{}
  \renewcommand{\childdocby}[2][]{}
  \renewcommand{\childdocforward}[2][]{}
  \renewcommand{\childdocdisable}{}
}
%    \end{macrocode}

% \macro{\childdocmain}
% The macro |\childdocmain| is to be called at the top of the main file
% with nothing or the main filename (without extension) as argument.
% First, it breaks loops.
% If the argument is not empty and does not match |\childdocname|
% (which is set by the first inclusion of |childdoc.def|),
% |\ifchilddoc| is set to true, |\includeonly| is applied to the child file
% and |\jobname| is set to the main file
% (for proper handling of |.aux| files):
%    \begin{macrocode}
\newcommand{\childdocmain}[1]
{
  \childdocdisable\childdocmain{}
  \if?#1?\else
    \begingroup
      \def\childdoctmp{#1}
      \ifx\childdoctmp\childdocname
        \def\childdoctmp{}
      \else
        \def\childdoctmp
        {
          \childdoctrue
          \includeonly{\childdocname}
          \def\childdocjob{#1}
          \def\jobname{#1}
        }
      \fi
      \expandafter
    \endgroup
    \childdoctmp
  \fi
}
%    \end{macrocode}

% \macro{\childdocof}
% The command |\childdocof| redirects
% compilation to the main file |#1|.
%    \begin{macrocode}
\newcommand{\childdocof}[1]
{
  \childdocdisable
  \childdoctrue
  \includeonly{\childdocname}
  \def\jobname{#1}
  \def\childdocjob{#1}
  \input{#1}
}
%    \end{macrocode}

% \macro{\childdocby}
% The command |\childdocby| ....
%    \begin{macrocode}
\newcommand{\childdocby}[2][]
{
  \childdocdisable
  \childdoctrue
  \childdocmanualtrue
  \if?#1?\else
    \def\jobname{#2}
  \fi
  \def\childdocjob{#2}
  \input{#2}
  \endinput
}
%    \end{macrocode}

% \macro{\childdocforward}
% The command |\childdocforward| redirects
% compilation to the main file or
% (if the optional argument is given) a child file.
% Parameters are set as if the main file
% or a child file starting with |\childdocof| was compiled.
% Then compilation is handed over to the main file:
%    \begin{macrocode}
\newcommand{\childdocforward}[2][]
{
  \begingroup
    \if?#1?
      \def\childdoctmp
      {
        \def\childdocname{#2}
        \def\childdocjob{#2}
        \def\jobname{#2}
        \input{#2}
        \endinput
      }
    \else
      \def\childdoctmp
      {
        \childdocdisable
        \def\childdocname{#2}
        \childdoctrue
        \includeonly{#2}
        \def\childdocjob{#1}
        \def\jobname{#1}
        \input{#1}
        \endinput
      }
    \fi
    \expandafter
  \endgroup
  \childdoctmp
}
%    \end{macrocode}

% \macro{\childdocforwardprefix}
% The command |\childdocforwardprefix| redirects
% compilation to the main or a child file by means of a pattern.
% The prefix |#1| in the current filename is replaced by |#2|
% and the suffix of the current filename is kept
% (it is assumed that the filename does not contain the substring `|~~~|'
% which is used as a delimiter).
% Compilation is handed over to the new file by |\childdocforward|:
%    \begin{macrocode}
\newcommand{\childdocforwardprefix}[3][]
{
  \begingroup
    \def\childdocextract #2##1~~~{\def\childdoctmp{\childdocforward[#1]{#3##1}}}
    \expandafter\childdocextract\childdocname~~~
    \expandafter
  \endgroup
  \childdoctmp
}
%    \end{macrocode}

% \macro{\childdoc}
% The deprecated macro |\childdoc| is a legacy version of |\childdocmain|:
%    \begin{macrocode}
\newcommand{\childdoc}{\childdocmain}
%    \end{macrocode}

% \macro{\childdocredirect}
% The deprecated macro |\childdocredirect| is a legacy version
% of |\childdocforward| and |\childdocforwardprefix|:
%    \begin{macrocode}
\newcommand{\childdocredirect}[2][]
{
  \begingroup
    \if?#1?
      \def\childdoctmp{\childdocforward{#2}}
    \else
      \def\childdoctmp{\childdocforwardprefix{#1}{#2}}
    \fi
    \expandafter
  \endgroup
  \childdoctmp
}
%    \end{macrocode}

%\iffalse
%</package>
%\fi
%
\endinput
|\\
|\childdocforwardprefix{final}{child}|
\end{tabular}
\end{center}
%

Note that when several versions of a main file and/or of each child file
are to be generated, it may be convenient to set up a |Makefile| or
shell script to automatise the process.

%%%%%%%%%%%%%%%%%%%%%%%%%%%%%%%%%%%%%%%%%%%%%%%%%%%%%%%%%%%%%%%%%%%%%%%%%%%%%%%%
\subsection{Command Line Processing}
\label{sec:commandline}

The effect of redirection files can also be achieved by invoking
the \LaTeX{} compiler with a more elaborate command line.
Most conveniently this should be done as part
of a shell script or a |Makefile|.

When using \textsf{childdoc} in the main file, the following
command lines effectively perform a redirection
(note that depending on the shell being used,
backslashes may have to be doubled: `|\|' $\to$ `|\\|'):
%
\begin{center}
|... -jobname "|\textit{target}|" |\\|"|[\textit{flags}]%
|% \iffalse
%
% childdoc.dtx Copyright (C) 2017-2018 Niklas Beisert
%
% This work may be distributed and/or modified under the
% conditions of the LaTeX Project Public License, either version 1.3
% of this license or (at your option) any later version.
% The latest version of this license is in
%   http://www.latex-project.org/lppl.txt
% and version 1.3 or later is part of all distributions of LaTeX
% version 2005/12/01 or later.
%
% This work has the LPPL maintenance status `maintained'.
%
% The Current Maintainer of this work is Niklas Beisert.
%
% This work consists of the files childdoc.dtx and childdoc.ins
% and the derived files childdoc.def and cdocsamp.tex with
% cdocsch1.tex, cdocsch2.tex, cdocsdrf.tex, cdocsfn1.tex, cdocsfn2.tex.
%
%<package>\ifdefined\childdocmain\endinput\fi
%<package>\ProvidesFile{childdoc.def}[2018/12/30 v2.0 child document driver]
%<samplemain>\ProvidesFile{cdocsamp.tex}[2018/12/30 v2.0 sample for childdoc]
%<*driver>
%\ProvidesFile{childdoc.drv}[2018/12/30 v2.0 childdoc reference manual file]
\PassOptionsToClass{10pt,a4paper}{article}
\documentclass{ltxdoc}

\usepackage[margin=35mm]{geometry}
\usepackage{hyperref}
\usepackage{hyperxmp}
\usepackage[usenames]{color}

\hypersetup{colorlinks=true}
\hypersetup{pdfstartview=FitH}
\hypersetup{pdfpagemode=UseNone}
\hypersetup{pdfsource={}}
\hypersetup{pdflang={en-UK}}
\hypersetup{pdfcopyright={Copyright 2017-2018 Niklas Beisert.
  This work may be distributed and/or modified under the
  conditions of the LaTeX Project Public License, either version 1.3
  of this license or (at your option) any later version.}}
\hypersetup{pdflicenseurl={http://www.latex-project.org/lppl.txt}}
\hypersetup{pdfcontactaddress={ETH Zurich, ITP, HIT K,
  Wolfgang-Pauli-Strasse 27}}
\hypersetup{pdfcontactpostcode={8093}}
\hypersetup{pdfcontactcity={Zurich}}
\hypersetup{pdfcontactcountry={Switzerland}}
\hypersetup{pdfcontactemail={nbeisert@itp.phys.ethz.ch}}
\hypersetup{pdfcontacturl={http://people.phys.ethz.ch/\xmptilde nbeisert/}}

\newcommand{\secref}[1]{\hyperref[#1]{section \ref*{#1}}}

\parskip1ex
\parindent0pt
\let\olditemize\itemize
\def\itemize{\olditemize\parskip0pt}

\begin{document}

\title{The \textsf{childdoc} Package}
\hypersetup{pdftitle={The childdoc Package}}
\author{Niklas Beisert\\[2ex]
  Institut f\"ur Theoretische Physik\\
  Eidgen\"ossische Technische Hochschule Z\"urich\\
  Wolfgang-Pauli-Strasse 27, 8093 Z\"urich, Switzerland\\[1ex]
  \href{mailto:nbeisert@itp.phys.ethz.ch}
  {\texttt{nbeisert@itp.phys.ethz.ch}}}
\hypersetup{pdfauthor={Niklas Beisert}}
\hypersetup{pdfsubject={Manual for the LaTeX2e Package childdoc}}
\date{30 December 2018, \textsf{v2.0}}
\maketitle

\begin{abstract}\noindent
\textsf{childdoc} is a \LaTeXe{} package
that enables the direct compilation
of document sections included by |\include|
to individual files.
\end{abstract}

\begingroup
\parskip0ex
\tableofcontents
\endgroup

%%%%%%%%%%%%%%%%%%%%%%%%%%%%%%%%%%%%%%%%%%%%%%%%%%%%%%%%%%%%%%%%%%%%%%%%%%%%%%%%
%%%%%%%%%%%%%%%%%%%%%%%%%%%%%%%%%%%%%%%%%%%%%%%%%%%%%%%%%%%%%%%%%%%%%%%%%%%%%%%%
\section{Introduction}

\LaTeX{} provides a mechanism to structure a large document (such as a book)
into a main file and several child files (containing the chapters)
using the |\include| command.
This mechanism is beneficial for documents
which span hundreds of pages in order to
make the source file(s) more manageable.
Moreover, compilation can be restricted to
selected child files by means of the |\includeonly| command.
The latter feature can be used to reduce the compilation time while editing
(this was significantly more useful in the earlier days of \LaTeX{})
or to generate a smaller document which is easier to navigate.
Another application of |\includeonly| is to generate
documents consisting of selected parts of the complete document.

However, there are a few drawbacks of the plain |\include| mechanism:
\begin{itemize}
\item
The child files cannot be compiled on their own,
they can only be compiled via the main file.
A naive editing environment
(such as a text editor with an option
to have the current file processed by \LaTeX)
may require one to switch to the main file before compiling;
attempting to compile the child file produces errors.
\item
The main file must be modified (each time)
to adjust the |\includeonly| command
to the present needs. This easily leaves the main file in a messy state.
\item
The generated document will always carry the filename
of the main document. This is inconvenient if
several child files are to be compiled and
to be kept for distribution.
\end{itemize}

The present package provides a simple interface
to make child files individually compilable by \LaTeX{}.
Compiling a child file then has the same effect as compiling
the main file with an |\includeonly| command
to select the appropriate child.
Moreover the generated document will carry the name of the child
rather than the main file.
This resolves all three above issues.

This feature is meant to make the editing of books,
thesis documents and lecture notes somewhat more convenient.
However, the package can also be used efficiently for
composing a series of documents (such as exercise sheets)
which are typically distributed individually.
It then assists the author in generating the individual documents
(potentially in different versions)
as well as a document containing the collected series.
Another application is in developing style files
or other kinds of included material
where compilation of the style file could redirect
to a sample or test file.

%%%%%%%%%%%%%%%%%%%%%%%%%%%%%%%%%%%%%%%%%%%%%%%%%%%%%%%%%%%%%%%%%%%%%%%%%%%%%%%%
%%%%%%%%%%%%%%%%%%%%%%%%%%%%%%%%%%%%%%%%%%%%%%%%%%%%%%%%%%%%%%%%%%%%%%%%%%%%%%%%
\section{Usage}

First of all, the package \textsf{childdoc} is \emph{not} a standard
\LaTeXe{} |.sty| style file! Therefore it needs to be invoked in
a non-standard way.

%%%%%%%%%%%%%%%%%%%%%%%%%%%%%%%%%%%%%%%%%%%%%%%%%%%%%%%%%%%%%%%%%%%%%%%%%%%%%%%%
\subsection{Included Files}
\label{sec:include}

%%%%%%%%%%%%%%%%%%%%%%%%%%%%%%%%%%%%%%%%
\DescribeMacro{\childdocmain}
To use the package, add the commands
\begin{center}
\begin{tabular}{l}
|\input{childdoc.def}|\\
|\childdocmain{}|\\
\end{tabular}
\end{center}
at the very top of the main \LaTeX{} file,
in particular \emph{before} the |\documentclass| statement!
The argument of |\childdocmain| should be left empty
(but it must be present).

%%%%%%%%%%%%%%%%%%%%%%%%%%%%%%%%%%%%%%%%
\DescribeMacro{\childdocof}
Furthermore, add the commands
\begin{center}
\begin{tabular}{l}
|\input{childdoc.def}|\\
|\childdocof{|\textit{main}|}|\\
\end{tabular}
\end{center}
at the top of every child file \textit{child}
which is included by |\include{|\textit{child}|}|
from within the main file
(or at least for those files to be compiled individually).
The argument \textit{main} must be the filename of the main file.

There are a couple of
considerations in setting up the main and child documents:

%%%%%%%%%%%%%%%%%%%%%%%%%%%%%%%%%%%%%%%%
\paragraph{Restrictions.}

Please note the following restrictions:
\begin{itemize}
\item
|\childdocmain| must be called with one argument \textit{main}
to ensure compatibility with earlier version of the package.
It must either be empty (|\childdocmain{}|)
or precisely match the filename of the main file in which it is specified.
See \secref{sec:detection} for further information.
\item
The filename \textit{main} must be specified without the |.tex| extension.
\item
The filename \textit{main} is case sensitive
(even in case-insensitive file systems)
due to internal string comparison.
\item
The argument \textit{main} should be fully expanded, it cannot be a macro.
\item
Subdirectories and special characters should be avoided in filenames.
\item
The command |\childdocmain{|\textit{main}|}| must be followed by a whitespace.
It should not be followed immediately by another command
or by a comment mark `|%|'.
This is because the \TeX{} parser reads the token immediately following
the argument of |\childdocmain| and puts it
at the beginning of every child section;
however, a white\-space is ignored.
\end{itemize}

%%%%%%%%%%%%%%%%%%%%%%%%%%%%%%%%%%%%%%%%
\paragraph{Content of Main File.}

It is advisable to place all content in the child files included by |\include|.
Any output contained in the main file will appear in all child documents
unless suppressed manually;
it cannot be suppressed automatically by the |\includeonly| directive
and thus should normally be avoided.
A method to include some content in the main file
by means of conditional processing is described in \secref{sec:conditional}.

%%%%%%%%%%%%%%%%%%%%%%%%%%%%%%%%%%%%%%%%
\paragraph{Page Numbering.}

When only a part of the document is compiled,
the appropriate numbering of pages
(as well as other status parameters)
is determined from the |.aux| files.
The latter contain information from previous passes.
However this information needs to propagate through
all intermediate child documents.
Therefore the page numbering in child documents may well
be inconsistent until the complete document is compiled at least once.

A useful (if unconventional) way to always ensure a consistent
page numbering is to restart the numbering in each child document
and denote the pages by `\textit{child}|.|\textit{page}'
where \textit{child} represents the chapter/section number of the child file.
This can be achieved by the command
|\numberwithin{page}{|\textit{child}|}|
of the \textsf{amsmath} package
where \textit{child} can be |chapter| or |section|
depending on the chosen structuring.
Alternatively, one can modify the macro |\thepage| appropriately
and reset the counter |page| at the start of each child file.

%%%%%%%%%%%%%%%%%%%%%%%%%%%%%%%%%%%%%%%%%%%%%%%%%%%%%%%%%%%%%%%%%%%%%%%%%%%%%%%%
\subsection{Conditional Processing}
\label{sec:conditional}

The package provides a mechanism to compile different versions
of a document. To customise the versions further some conditional processing
can come in handy to distinguish which version is being compiled.
The package provides two macros to describe the compilation context:

%%%%%%%%%%%%%%%%%%%%%%%%%%%%%%%%%%%%%%%%
\DescribeMacro{\ifchilddoc}
The conditional |\ifchilddoc| distinguishes between the compilation of
child documents and the main document:
%
\begin{center}
|\ifchilddoc |\textit{child-code}| |[|\||else |\textit{main-code}]| \||fi|
\end{center}

%%%%%%%%%%%%%%%%%%%%%%%%%%%%%%%%%%%%%%%%
\DescribeMacro{\childdocname}
\DescribeMacro{\childdocjob}
The macro |\childdocname| contains the filename (without extension)
of the main or child file being processed.
Note that |\childdocjob| will always contain the name of the main file.

%%%%%%%%%%%%%%%%%%%%%%%%%%%%%%%%%%%%%%%%
\paragraph{Title Page.}

Conditional processing can be used to include a title or banner page
in the main document when proper precautions are taken.
Importantly, the code in the main file should ensure that the page counter
(as well as other status parameters which are stored in the |.aux| files)
takes the same value after the conditional processing.
Otherwise the page numbers may take divergent values
depending on which part is compiled.

For example, a title page could be declared by:
%
\begin{center}
\begin{tabular}{l}
|\ifchilddoc\||else|\\
|\addtocounter{page}{-1}|\\
\textit{code for title page}\\
|\newpage|\\
|\||fi|
\end{tabular}
\end{center}
%
A banner page for the child documents can be generated by:
%
\begin{center}
\begin{tabular}{l}
|\ifchilddoc|\\
|\addtocounter{page}{-1}|\\
\textit{code for banner page}\\
|\newpage|\\
|\||fi|
\end{tabular}
\end{center}
%
Here one could write a message such as:
\begin{center}
|This is the part \childdocname{} of \childdocjob{}.|
\end{center}

%%%%%%%%%%%%%%%%%%%%%%%%%%%%%%%%%%%%%%%%%%%%%%%%%%%%%%%%%%%%%%%%%%%%%%%%%%%%%%%%
\subsection{Flags}
\label{sec:flags}

The package makes it easy to generate different versions
of the main or child documents.
To this end compilation flags can be defined
and assigned different default values.
They will be particularly useful in conjunction
with the forwarding mechanism described in \secref{sec:forward}.

For example, it may be useful to have a flag |\version|
which can be set to |draft| or |final|.
The document source will contain some conditional code
depending on the value of |\version|.
Suppose further, the flag should default to |final| for the main file
and to |draft| for child files
which is a natural assignment for editing the document.
This is achieved by placing the following code
in the preamble of the main document
(below the |\childdocmain| directive):
%
\begin{center}
\begin{tabular}{l}
|\ifchilddoc|\\
|\providecommand{\version}{draft}|\\
|\||else|\\
|\providecommand{\version}{final}|\\
|\||fi|
\end{tabular}
\end{center}
%
The definition by |\providecommand| makes sure
that previous definitions are not overwritten.
Further statements |\providecommand{\version}{...}|
can thus be added before the above code to override it.

For the main file, one might add a line
(between |\childdocmain| and the above block)
%
\begin{center}
|%\ifchilddoc\||else\providecommand{\version}{draft}\||fi|
\end{center}
%
which can be uncommented to produce a draft version.
Likewise one can add a line to the very top of a child file
(above the |\childdocof{|\textit{main}|}| directive)
%
\begin{center}
|%\providecommand{\version}{final}|
\end{center}
%
which can be uncommented to produce the final version of this child document.

%%%%%%%%%%%%%%%%%%%%%%%%%%%%%%%%%%%%%%%%%%%%%%%%%%%%%%%%%%%%%%%%%%%%%%%%%%%%%%%%
\subsection{Forwarding}
\label{sec:forward}

Different versions of the main or child documents
using compilation flags as described in \secref{sec:flags}
can be (permanently) stored in different files
for convenient compilation, viewing and distribution.
To this end, the package defines a command
to pass on compilation to a different file:

%%%%%%%%%%%%%%%%%%%%%%%%%%%%%%%%%%%%%%%%
\DescribeMacro{\childdocforward}
The command |\childdocforward| redirects processing to
another source file:
%
\begin{center}
\begin{tabular}{l}
|\input{childdoc.def}|\\
|\childdocforward[|\textit{main}|]{|\textit{dest}|}|\\
\end{tabular}
\end{center}
%
The argument \textit{dest} is the destination file
(without extension).
It should be the main file or one of the child files.
Note that further \textsf{childdoc} directives
such as |\childdocof| and |\childdocforward|
in the indicated file will be processed in this form.
The optional argument \textit{main}
passes on directly to the main file \textit{main}
while pretending to compile the child \textit{dest}.
This form behaves as if \textit{dest}
issues |\childdocof{|\textit{main}|}| right away,
and no further \textsf{childdoc} directives will be processed.

%%%%%%%%%%%%%%%%%%%%%%%%%%%%%%%%%%%%%%%%
\DescribeMacro{\...prefix}
In the alternative form |\childdocforwardprefix|,
%
\begin{center}
\begin{tabular}{l}
|\input{childdoc.def}|\\
|\childdocforwardprefix[|\textit{main}|]{|\textit{prefix}|}{|\textit{dest}|}|
\end{tabular}
\end{center}
%
the destination file is determined by a pattern
depending on the current file:
To make this work, the current file must be called
`{\textit{prefix}\hspace{0.2em}\textit{suffix}}'
with \textit{prefix} matching precisely the argument.
Processing is then passed on to the file
`{\textit{dest}\hspace{0.2em}\textit{suffix}}'.
Surely, the same effect is achieved by
directly specifying the
argument `{\textit{dest}\hspace{0.2em}\textit{suffix}}'
in the first form.
However, that requires to set up a different file
for each child. With the alternative form of the command
all these files can have exactly the same content
which simplifies setting them up and maintaining them.

For example, the following file |draft.tex|
with a compilation flag |\version| as described in \secref{sec:flags}
compiles the main document as a draft:
%
\begin{center}
\begin{tabular}{l}
|\def\version{draft}|\\
|\input{childdoc.def}|\\
|\childdocforward{|\textit{main}|}|
\end{tabular}
\end{center}
%
Likewise, the following files |final|\textit{nn}|.tex|
compile the final version of the child document
|child|\textit{nn}|.tex|:
%
\begin{center}
\begin{tabular}{l}
|\def\version{final}|\\
|\input{childdoc.def}|\\
|\childdocforwardprefix{final}{child}|
\end{tabular}
\end{center}
%

Note that when several versions of a main file and/or of each child file
are to be generated, it may be convenient to set up a |Makefile| or
shell script to automatise the process.

%%%%%%%%%%%%%%%%%%%%%%%%%%%%%%%%%%%%%%%%%%%%%%%%%%%%%%%%%%%%%%%%%%%%%%%%%%%%%%%%
\subsection{Command Line Processing}
\label{sec:commandline}

The effect of redirection files can also be achieved by invoking
the \LaTeX{} compiler with a more elaborate command line.
Most conveniently this should be done as part
of a shell script or a |Makefile|.

When using \textsf{childdoc} in the main file, the following
command lines effectively perform a redirection
(note that depending on the shell being used,
backslashes may have to be doubled: `|\|' $\to$ `|\\|'):
%
\begin{center}
|... -jobname "|\textit{target}|" |\\|"|[\textit{flags}]%
|\input{childdoc.def}\childdocforward[|\textit{main}|]{|\textit{dest}|}"|
\end{center}
%
Here \textit{target} is the name of the output file,
\textit{main} is the name of the main file
and \textit{dest} is the name of the main or child file to be processed
(all filenames without extensions).
The optional argument \textit{main} can be omitted
if \textit{main} matches \textit{dest}.
Optionally, compilation \textit{flags} can be defined via |\def| commands.
This command line makes the \TeX{} engine believe
it is compiling the file \textit{target}
whose content is specified as the latter parameter.
The provided code then forwards the processing to
\textit{main} or \textit{dest} as described in \secref{sec:forward}.

%%%%%%%%%%%%%%%%%%%%%%%%%%%%%%%%%%%%%%%%%%%%%%%%%%%%%%%%%%%%%%%%%%%%%%%%%%%%%%%%
\subsection{Include by Input}
\label{sec:input}

Including child documents by |\include| has some restrictions by design.
Most notably, the content of a child document always occupies
its own set of pages; pages cannot be shared between child documents.
Usually, this behaviour makes perfect sense
because each child document contain an essential part of the document.
However, in some situations it may be desirable to compose
a document from a collection of parts
without having mandatory page breaks between then.
For this case, the package
provides a mechanism to include parts
by |\input| which can also be processed individually.
However, by construction this mechanism
requires manual handling of the content to be output.

%%%%%%%%%%%%%%%%%%%%%%%%%%%%%%%%%%%%%%%%
\DescribeMacro{\ifchilddocmanual}
The main file should be prepared as usual, see \secref{sec:include}.
However, the document body must make a distinction
between processing of an individual part and of the main document, e.g.:
%
\begin{center}
\begin{tabular}{l}
|\ifchilddocmanual|\\
|\input{\childdocname}|\\
|\||else|\\
\textit{document body with }|\input{|\textit{part}|}|\\
|\||fi|
\end{tabular}
\end{center}
%
The conditional |\ifchilddocmanual| is true whenever
a part to be included by |\input| is being compiled,
and the name of the part is stored in |\childdocname|.

%%%%%%%%%%%%%%%%%%%%%%%%%%%%%%%%%%%%%%%%
\DescribeMacro{\childdocby}
Each part to be included by |\input| should start with:
%
\begin{center}
\begin{tabular}{l}
|\input{childdoc.def}|\\
|\childdocby{|\textit{main}|}|\\
\end{tabular}
\end{center}
%
The directive |\childdocby| is similar to |\childdocof|
described in \secref{sec:include},
but the subsequent selection of content must be done manually.
To that end, both |\ifchilddoc| and |\ifchilddocmanual|
will be true upon processing of a part,
and the name of the part is stored in |\childdocname|.
Note that |\jobname| will be set to the filename of the current part
so that each part receives an individual |.aux| file
that does not interfere with the |.aux| file(s) of the main document.
This behaviour can be altered by the alternative form
|\childdocby[*]{|\textit{main}|}| (with a non-empty optional argument)
which uses the |.aux| file of the main document
by setting |\jobname| to \textit{main}.

%%%%%%%%%%%%%%%%%%%%%%%%%%%%%%%%%%%%%%%%%%%%%%%%%%%%%%%%%%%%%%%%%%%%%%%%%%%%%%%%
\subsection{Driver Development}
\label{sec:driver}

The \textsf{childdoc} mechanism can also be use for the development
of definition files such as \LaTeX{} styles or classes.
This case differs from the above setup with multiple parts
included by |\include| in that no |\includeonly| should be invoked.
This can be achieved by starting the include file
(before |\ProvidesPackage|) with:
%
\begin{center}
\begin{tabular}{l}
|\input{childdoc.def}|\\
|\childdocforward{|\textit{main}|}|\\
\end{tabular}
\end{center}
%
or alternatively with:
%
\begin{center}
\begin{tabular}{l}
|\input{childdoc.def}|\\
|\childdocby{|\textit{main}|}|\\
\end{tabular}
\end{center}
%
Both forms have slightly different effects as described above.
The main file is prepared as usual, see \secref{sec:include}.

%%%%%%%%%%%%%%%%%%%%%%%%%%%%%%%%%%%%%%%%%%%%%%%%%%%%%%%%%%%%%%%%%%%%%%%%%%%%%%%%
\subsection{Legacy Detection}
\label{sec:detection}

The directive |\childdocmain| in the main file can detect
whether the complete document or merely a child is to be compiled
even without using the directive |\childdocof|.
This method is deprecated because it is less robust
and there is no compelling reason to use it;
it is merely provided for backward compatibility
and it may be removed in future versions.

If the detection mechanism is to be used,
it is mandatory to correctly specify
the filename of the main file as the argument of |\childdocmain|:
%
\begin{center}
\begin{tabular}{l}
|\input{childdoc.def}|\\
|\childdocmain{|\textit{main}|}|\\
\end{tabular}
\end{center}
%
If |\jobname| does not match the argument \textit{main} of |\childdocmain|,
it is assumed that |\jobname| points to the child file to be compiled.
When using |\childdocmain| with the main file specified as argument,
it suffices to start a child file
with just |\input{|\textit{main}|}|
without loading of the package and using |\childdocof|.
If instead all processing is done
with the appropriate \textsf{childdoc} directives,
the argument of \textit{main} of |\childdocmain| can be empty.

An alternative version of the command line processing described
in \secref{sec:commandline} using the detection mechanism reads:
%
\begin{center}
|... -jobname "|\textit{target}|" "|[\textit{flags}]%
[|\def\jobname{|\textit{dest}|}|]|\input{|\textit{main}|}"|
\end{center}

%%%%%%%%%%%%%%%%%%%%%%%%%%%%%%%%%%%%%%%%%%%%%%%%%%%%%%%%%%%%%%%%%%%%%%%%%%%%%%%%
\subsection{Manual Code}
\label{sec:manual}

In case one cannot be certain whether the definitions file |childdoc.def|
is installed on the target \TeX{} distribution
and one prefers not to ship it,
it is conceivable to paste a few relevant commands into the sources.

To that end, drop all statements |\input{childdoc.def}|
and perform the replacements as outlined below.
Instead of |\childdocmain{|\textit{main}|}| add the following code
to the top of the main file:
%
\begin{center}
\begin{tabular}{l}
|\||ifdefined\childdocname\endinput\||fi\newif\ifchilddoc|\\
|\edef\childdocname{\scantokens\expandafter{\jobname\noexpand}}|\\
|\def\childdocmain{|\textit{main}|}\||ifx\childdocmain\childdocname\||else|\\
|\childdoctrue\includeonly{\childdocname}\let\jobname\childdocmain\||fi|\\
\end{tabular}
\end{center}
%
Instead of |\childdocof{|\textit{main}|}| just include the main file
at the top of each child file:
%
\begin{center}
|\input{|\textit{main}|}|
\end{center}
%
A simple redirection |\childdocforward{|\textit{dest}|}| is achieved by:
%
\begin{center}
|\def\jobname{|\textit{dest}|}\input{\jobname}|
\end{center}
%
The redirection with prefix
|\childdocforwardprefix[|\textit{prefix}|]{|\textit{dest}|}|
is accomplished by:
%
\begin{center}
\begin{tabular}{l}
|{\edef\jobname{\scantokens\expandafter{\jobname\noexpand}}|\\
|\def\redirectjob |\textit{prefix}|#1~~~{\gdef\jobname{|\textit{dest}|#1}}|\\
|\expandafter\redirectjob\jobname~~~}\input{\jobname}|
\end{tabular}
\end{center}

In an alternative approach,
child documents can be compiled by a specific command line
without additional code or specific definitions:
%
\begin{center}
|... -jobname "|\textit{target}|" "|[\textit{flags}]%
|\includeonly{|\textit{dest}|}\input{|\textit{main}|}"|
\end{center}
%

%%%%%%%%%%%%%%%%%%%%%%%%%%%%%%%%%%%%%%%%%%%%%%%%%%%%%%%%%%%%%%%%%%%%%%%%%%%%%%%%
%%%%%%%%%%%%%%%%%%%%%%%%%%%%%%%%%%%%%%%%%%%%%%%%%%%%%%%%%%%%%%%%%%%%%%%%%%%%%%%%
\section{Information}

%%%%%%%%%%%%%%%%%%%%%%%%%%%%%%%%%%%%%%%%%%%%%%%%%%%%%%%%%%%%%%%%%%%%%%%%%%%%%%%%
\subsection{Copyright}

Copyright \copyright{} 2017--2018 Niklas Beisert

This work may be distributed and/or modified under the
conditions of the \LaTeX{} Project Public License, either version 1.3
of this license or (at your option) any later version.
The latest version of this license is in
  \url{http://www.latex-project.org/lppl.txt}
and version 1.3 or later is part of all distributions of \LaTeX{}
version 2005/12/01 or later.

This work has the LPPL maintenance status `maintained'.

The Current Maintainer of this work is Niklas Beisert.

This work consists of the files |README.txt|, |childdoc.ins| and |childdoc.dtx|
as well as the derived files |childdoc.def|, |cdocsamp.tex|
with |cdocsch1.tex|, |cdocsch2.tex|, |cdocspt3.tex|, |cdocspt4.tex|,
|cdocsdrf.tex|, |cdocsfn1.tex|, |cdocsfn2.tex|
as well as |childdoc.pdf|.

%%%%%%%%%%%%%%%%%%%%%%%%%%%%%%%%%%%%%%%%%%%%%%%%%%%%%%%%%%%%%%%%%%%%%%%%%%%%%%%%
\subsection{Files and Installation}

The package consists of the files:
%
\begin{center}
\begin{tabular}{ll}
    |README.txt|   & readme file \\
    |childdoc.ins| & installation file \\
    |childdoc.dtx| & source file \\
    |childdoc.def| & definition file \\
    |cdocsamp.tex| & sample main file \\
    |cdocsch1.tex| & sample include file \\
    |cdocsch2.tex| & sample include file \\
    |cdocspt3.tex| & sample part file \\
    |cdocspt4.tex| & sample part file \\
    |cdocsdrf.tex| & sample redirection file \\
    |cdocsfn1.tex| & sample redirection file \\
    |cdocsfn2.tex| & sample redirection file \\
    |childdoc.pdf| & manual
\end{tabular}
\end{center}
%
The distribution consists of the files
|README.txt|, |childdoc.ins| and |childdoc.dtx|.
%
\begin{itemize}
\item
Run (pdf)\LaTeX{} on |childdoc.dtx|
to compile the manual |childdoc.pdf| (this file).
\item
Run \LaTeX{} on |childdoc.ins| to create the definitions file |childdoc.def|
and the sample |cdocsamp.tex| with include files
|cdocsch1.tex|, |cdocsch2.tex|, |cdocspt3.tex|, |cdocspt4.tex|,
|cdocsdrf.tex|, |cdocsfn1.tex|, |cdocsfn2.tex|.
Then copy the file |childdoc.def| to an appropriate directory of your \LaTeX{}
distribution, e.g.\ \textit{texmf-root}|/tex/latex/childdoc|.
\end{itemize}

%%%%%%%%%%%%%%%%%%%%%%%%%%%%%%%%%%%%%%%%%%%%%%%%%%%%%%%%%%%%%%%%%%%%%%%%%%%%%%%%
\subsection{Related CTAN Packages}

There are several other packages which offer a similar functionality:
%
\begin{itemize}
\item
The packages
\href{http://ctan.org/pkg/docmute}{\textsf{docmute}},
\href{http://ctan.org/pkg/includex}{\textsf{includex}} and
\href{http://ctan.org/pkg/standalone}{\textsf{standalone}}
provide commands to include only the document body of
a child file thus allowing both files to be compiled individually.
\item
The packages \href{http://ctan.org/pkg/subdocs}{\textsf{subdocs}}
and \href{http://ctan.org/pkg/subfiles}{\textsf{subfiles}}
provide structures in which the main and child documents can be
encapsulated and allowing them to be compiled individually.
The inclusion mechanism is different from the conventional |\include|.
\item
The package \href{http://ctan.org/pkg/combine}{\textsf{combine}}
is an elaborate solution to combine several documents into one.
\end{itemize}
%
See also the CTAN topic \href{http://ctan.org/topic/subdocs}{\textsf{subdocs}}
for further related packages.
The present package differs from the above solutions in that
a document structure constructed with the conventional |\include| mechanism
just needs two extra commands at the top of every file
such that all constituent files can be compiled individually.

%%%%%%%%%%%%%%%%%%%%%%%%%%%%%%%%%%%%%%%%%%%%%%%%%%%%%%%%%%%%%%%%%%%%%%%%%%%%%%%%
%\subsection{Feature Suggestions}
%
%The following is a list of features which may be useful for future
%versions of this package:
%%
%\begin{itemize}
%\item
%\ldots
%\end{itemize}

%%%%%%%%%%%%%%%%%%%%%%%%%%%%%%%%%%%%%%%%%%%%%%%%%%%%%%%%%%%%%%%%%%%%%%%%%%%%%%%%
\subsection{Revision History}

%%%%%%%%%%%%%%%%%%%%%%%%%%%%%%%%%%%%%%%%
\paragraph{v2.0:} 2018/12/30

\begin{itemize}
\item
immediate forward processing
\item
added |\childdocby| mechanism
\item
manual restructured
\end{itemize}

%%%%%%%%%%%%%%%%%%%%%%%%%%%%%%%%%%%%%%%%
\paragraph{v1.6:} 2018/01/17

\begin{itemize}
\item
application for development of include files
\item
corrections to manual
\end{itemize}

%%%%%%%%%%%%%%%%%%%%%%%%%%%%%%%%%%%%%%%%
\paragraph{v1.5:} 2017/05/21

\begin{itemize}
\item
more complete structuring introduced
\item
|\childdocof| introduced
\item
|\childdoc| renamed to |\childdocmain|
\item
|\childredirect| renamed to |\childdocforward| and |\childdocforwardprefix|
and functionality expanded
\end{itemize}

%%%%%%%%%%%%%%%%%%%%%%%%%%%%%%%%%%%%%%%%
\paragraph{v1.0:} 2017/04/27

\begin{itemize}
\item
manual and install package
\item
first version published on CTAN
\end{itemize}

%%%%%%%%%%%%%%%%%%%%%%%%%%%%%%%%%%%%%%%%
\paragraph{v0.6:} 2017/04/26

\begin{itemize}
\item
redirection mechanism added
\end{itemize}

%%%%%%%%%%%%%%%%%%%%%%%%%%%%%%%%%%%%%%%%
\paragraph{v0.5:} 2017/04/26

\begin{itemize}
\item
functionality in definition file
\end{itemize}


%%%%%%%%%%%%%%%%%%%%%%%%%%%%%%%%%%%%%%%%%%%%%%%%%%%%%%%%%%%%%%%%%%%%%%%%%%%%%%%%
%%%%%%%%%%%%%%%%%%%%%%%%%%%%%%%%%%%%%%%%%%%%%%%%%%%%%%%%%%%%%%%%%%%%%%%%%%%%%%%%
%%%%%%%%%%%%%%%%%%%%%%%%%%%%%%%%%%%%%%%%%%%%%%%%%%%%%%%%%%%%%%%%%%%%%%%%%%%%%%%%
\appendix

\settowidth\MacroIndent{\rmfamily\scriptsize 000\ }

 \DocInput{childdoc.dtx}

\end{document}
%</driver>
% \fi
%
% %%%%%%%%%%%%%%%%%%%%%%%%%%%%%%%%%%%%%%%%%%%%%%%%%%%%%%%%%%%%%%%%%%%%%%%%%%%%%%
% %%%%%%%%%%%%%%%%%%%%%%%%%%%%%%%%%%%%%%%%%%%%%%%%%%%%%%%%%%%%%%%%%%%%%%%%%%%%%%
% \section{Sample}
%\iffalse
%<*samplemain>
%\fi
%
% The following presents a sample document
% with two chapters, two parts, a title page,
% a compile flag as well as three forwarding files to set the flag.
% It consists of eight |.tex| files:
% \begin{center}
% \begin{tabular}{ll}
% |cdocsamp.tex|&main file\\
% |cdocsch1.tex|&include file for chapter 1\\
% |cdocsch2.tex|&include file for chapter 2\\
% |cdocspt3.tex|&include file for part 3\\
% |cdocspt4.tex|&include file for part 4\\
% |cdocsdrf.tex|&forwarding file for main file in draft mode\\
% |cdocsfi1.tex|&forwarding file for final version of chapter 1\\
% |cdocsfi2.tex|&forwarding file for final version of chapter 2\\
% \end{tabular}
% \end{center}
% Each of the eight files can be compiled directly by the \LaTeX{} compiler.
%
% %%%%%%%%%%%%%%%%%%%%%%%%%%%%%%%%%%%%%%
% \paragraph{Main File.}
%
% The main file is called |cdocsamp.tex|.
%
% Load the \textsf{childdoc} definitions and
% declare the filename for the main document:
%    \begin{macrocode}
\input{childdoc.def}
\childdocmain{}
%    \end{macrocode}

% Optional override for |\version| flag:
%    \begin{macrocode}
%%\ifchilddoc\else\providecommand{\version}{draft}\fi
%    \end{macrocode}

% Define the default values for the |\version| flag
% (|final| for the main file and |draft| for childs):
%    \begin{macrocode}
\ifchilddoc
\providecommand{\version}{draft}
\else
\providecommand{\version}{final}
\fi
%    \end{macrocode}

% Load the standard document class:
%    \begin{macrocode}
\documentclass[12pt]{article}
%    \end{macrocode}

% Start the document body:
%    \begin{macrocode}
\begin{document}
%    \end{macrocode}

% Declare a title page.
% Print title, part of document being processed and version flag:
%    \begin{macrocode}
\addtocounter{page}{-1}
\begin{center}
{\LARGE\bfseries{}childdoc example\par}
\vspace{1cm}
\ifchilddoc
\ifchilddocmanual part\else chapter\fi:
`\childdocname' of `\childdocjob'\par
\else
main document: `\childdocjob'\par
\fi
version: \version\par
\end{center}
\newpage
%    \end{macrocode}

% Manually include selected file,
% otherwise process as usual:
%    \begin{macrocode}
\ifchilddocmanual
\section*{part `\childdocname'}
\input{\childdocname}
\else
%    \end{macrocode}

% Include the two chapters:
%    \begin{macrocode}
\include{cdocsch1}
\include{cdocsch2}
%    \end{macrocode}

% Include the two parts unless only chapters should be displayed:
%    \begin{macrocode}
\ifchilddoc\else
\section{part three}
\input{cdocspt3}
\section{part four}
\input{cdocspt4}
\fi
%    \end{macrocode}

% Process as usual until here:
%    \begin{macrocode}
\fi
%    \end{macrocode}

% End of document body:
%    \begin{macrocode}
\end{document}
%    \end{macrocode}
%\iffalse
%</samplemain>
%\fi
%
% %%%%%%%%%%%%%%%%%%%%%%%%%%%%%%%%%%%%%%
% \paragraph{Chapter Include Files.}
%
% The include files are called |cdocsch1.tex| and |cdocsch2.tex|.
%
%\iffalse
%<*samplechap1|samplechap2>
%\fi

% Optional override for |\version| flag:
%    \begin{macrocode}
%%\providecommand{\version}{final}
%    \end{macrocode}

% Include the main document:
%    \begin{macrocode}
\input{childdoc.def}
\childdocof{cdocsamp}
%    \end{macrocode}

%\iffalse
%</samplechap1|samplechap2>
%\fi
%
%\iffalse
%<*samplechap1>
%\fi
% Some text for chapter 1:
%    \begin{macrocode}
\section{one}
some text in chapter one
%    \end{macrocode}

%\iffalse
%</samplechap1>
%\fi
% Some text for chapter 2:
%\iffalse
%<*samplechap2>
%\fi
%    \begin{macrocode}
\section{two}
more text in chapter two
%    \end{macrocode}

%\iffalse
%</samplechap2>
%\fi
%
% %%%%%%%%%%%%%%%%%%%%%%%%%%%%%%%%%%%%%%
% \paragraph{Part Include Files.}
%
% The include files are called |cdocspt3.tex| and |cdocspt4.tex|.
%
%\iffalse
%<*samplepart3|samplepart4>
%\fi

% Optional override for |\version| flag:
%    \begin{macrocode}
%%\providecommand{\version}{final}
%    \end{macrocode}

% Include the main document:
%    \begin{macrocode}
\input{childdoc.def}
\childdocby{cdocsamp}
%    \end{macrocode}

%\iffalse
%</samplepart3|samplepart4>
%\fi
%
%\iffalse
%<*samplepart3>
%\fi
% Some text for part 3:
%    \begin{macrocode}
some text in part three
%    \end{macrocode}

%\iffalse
%</samplepart3>
%\fi
% Some text for part 4:
%\iffalse
%<*samplepart4>
%\fi
%    \begin{macrocode}
more text in part four
%    \end{macrocode}

%\iffalse
%</samplepart4>
%\fi
%
% %%%%%%%%%%%%%%%%%%%%%%%%%%%%%%%%%%%%%%
% \paragraph{Forwarding for a Complete Draft.}
%
% The following forwarding file |cdocsdrf.tex|
% compiles the main document in draft mode:
%\iffalse
%<*sampledraft>
%\fi
%    \begin{macrocode}
\def\version{draft}
\input{childdoc.def}
\childdocforward{cdocsamp}
%    \end{macrocode}

%\iffalse
%</sampledraft>
%\fi
%
% %%%%%%%%%%%%%%%%%%%%%%%%%%%%%%%%%%%%%%
% \paragraph{Forwarding for Final Version of the Chapters.}
%
% The following forwarding files |cdocsfn1.tex| and |cdocsfn2.tex|
% (with identical content)
% compile the final versions of the child documents
% |cdocsch1.tex| and |cdocsch2.tex|, respectively:
%\iffalse
%<*samplefinal>
%\fi
%    \begin{macrocode}
\def\version{final}
\input{childdoc.def}
\childdocforwardprefix[cdocsamp]{cdocsfn}{cdocsch}
%    \end{macrocode}

%\iffalse
%</samplefinal>
%\fi
%
% %%%%%%%%%%%%%%%%%%%%%%%%%%%%%%%%%%%%%%
% \paragraph{Command Line Processing.}
%
% The following three command lines generate the output files
% |cdocscld|, |cdocscl1| and |cdocscl2|
% which should be identical to
% |cdocsdrf|, |cdocsch1| and |cdocsfn2|, respectively:
% \begin{center}
% \begin{tabular}{l}
% |latex -jobname cdocscld \|\\
% |  "\def\version{draft}\input{childdoc.def}\childdocforward{cdocsamp}"|\\
% |latex -jobname cdocscl1 \|\\
% |  "\input{childdoc.def}\childdocforward[cdocsamp]{cdocsch1}"|\\
% |latex -jobname cdocscl2 \|\\
% |  "\def\version{final}\input{childdoc.def}\childdocforward{cdocsch2}"|
% \end{tabular}
% \end{center}
% Note that the trailing backslash on each first line
% merely continues the input to the second line
% (for convenient cut ant paste).
% Furthermore, the command |latex| can be replaced by any
% of its alternative versions such as |pdflatex|.
%
% %%%%%%%%%%%%%%%%%%%%%%%%%%%%%%%%%%%%%%%%%%%%%%%%%%%%%%%%%%%%%%%%%%%%%%%%%%%%%%
% %%%%%%%%%%%%%%%%%%%%%%%%%%%%%%%%%%%%%%%%%%%%%%%%%%%%%%%%%%%%%%%%%%%%%%%%%%%%%%
% \section{Implementation}
%\iffalse
%<*package>
%\fi
%
% This section describes the definitions file |childdoc.def|.

% The definitions cannot be loaded using |\usepackage| or |\RequirePackage|
% which has a mechanism to prevent loading a style file more than once.
% When loading the definitions by means of |\input|
% multiple instances have to be prevented manually:
%\iffalse
%This code needs to be before the `\ProvidesFile' directive
%which is defined at the beginning of this file.
%Therefore it is also placed there and commented out here.
%</package>
%<*discard>
%\fi
%    \begin{macrocode}
\ifdefined\childdocmain\endinput\fi
%    \end{macrocode}
%\iffalse
%</discard>
%<*package>
%\fi
%
% \macro{\ifchilddoc}
% \macro{\ifchilddocmanual}
% The conditional |\ifchilddoc| tells whether a
% child (true) or main (false) document is being compiled.
% The conditional |\ifchilddocmanual| tells whether
% the |\includeonly| mechanism is used (false) or
% the selection of child files must be performed manually (true).
% The definitions initialise to false:
%    \begin{macrocode}
\newif\ifchilddoc
\newif\ifchilddocmanual
%    \end{macrocode}

% \macro{\childdocname}
% \macro{\childdocjob}
% The macro |\childdocname| stores the name of the main document
% to be compiled. The macro |\childdocjob| stores the name of
% the document on which the \LaTeX{} compiler was originally invoked.
% The content of |\jobname| cannot be compared
% to filenames specified in the source due to different catcodes.
% The following code rescans |\jobname|, stores the result
% in |\childdocname| and saves a copy in |\childdocjob|:
%    \begin{macrocode}
\edef\childdocname{\scantokens\expandafter{\jobname\noexpand}}
\let\childdocjob\childdocname
%    \end{macrocode}

% \macro{\childdocdisable}
% The macro |\childdocdisable| prevents the main file
% from being processed more than once.
% At this stage, the main document command |\childdocmain|
% is assumed to be called once again where it should do nothing.
% Any subsequent call to it should prevent
% a secondary processing of the main document
% It overwrites the forwarding commands
% |\childdocof| and |\childdocforward|
% with empty macros to prevent further inclusions of the main document:
%    \begin{macrocode}
\newcommand{\childdocdisable}
{
  \renewcommand{\childdocmain}[1]{\renewcommand{\childdocmain}[1]{\endinput}}
  \renewcommand{\childdocof}[1]{}
  \renewcommand{\childdocby}[2][]{}
  \renewcommand{\childdocforward}[2][]{}
  \renewcommand{\childdocdisable}{}
}
%    \end{macrocode}

% \macro{\childdocmain}
% The macro |\childdocmain| is to be called at the top of the main file
% with nothing or the main filename (without extension) as argument.
% First, it breaks loops.
% If the argument is not empty and does not match |\childdocname|
% (which is set by the first inclusion of |childdoc.def|),
% |\ifchilddoc| is set to true, |\includeonly| is applied to the child file
% and |\jobname| is set to the main file
% (for proper handling of |.aux| files):
%    \begin{macrocode}
\newcommand{\childdocmain}[1]
{
  \childdocdisable\childdocmain{}
  \if?#1?\else
    \begingroup
      \def\childdoctmp{#1}
      \ifx\childdoctmp\childdocname
        \def\childdoctmp{}
      \else
        \def\childdoctmp
        {
          \childdoctrue
          \includeonly{\childdocname}
          \def\childdocjob{#1}
          \def\jobname{#1}
        }
      \fi
      \expandafter
    \endgroup
    \childdoctmp
  \fi
}
%    \end{macrocode}

% \macro{\childdocof}
% The command |\childdocof| redirects
% compilation to the main file |#1|.
%    \begin{macrocode}
\newcommand{\childdocof}[1]
{
  \childdocdisable
  \childdoctrue
  \includeonly{\childdocname}
  \def\jobname{#1}
  \def\childdocjob{#1}
  \input{#1}
}
%    \end{macrocode}

% \macro{\childdocby}
% The command |\childdocby| ....
%    \begin{macrocode}
\newcommand{\childdocby}[2][]
{
  \childdocdisable
  \childdoctrue
  \childdocmanualtrue
  \if?#1?\else
    \def\jobname{#2}
  \fi
  \def\childdocjob{#2}
  \input{#2}
  \endinput
}
%    \end{macrocode}

% \macro{\childdocforward}
% The command |\childdocforward| redirects
% compilation to the main file or
% (if the optional argument is given) a child file.
% Parameters are set as if the main file
% or a child file starting with |\childdocof| was compiled.
% Then compilation is handed over to the main file:
%    \begin{macrocode}
\newcommand{\childdocforward}[2][]
{
  \begingroup
    \if?#1?
      \def\childdoctmp
      {
        \def\childdocname{#2}
        \def\childdocjob{#2}
        \def\jobname{#2}
        \input{#2}
        \endinput
      }
    \else
      \def\childdoctmp
      {
        \childdocdisable
        \def\childdocname{#2}
        \childdoctrue
        \includeonly{#2}
        \def\childdocjob{#1}
        \def\jobname{#1}
        \input{#1}
        \endinput
      }
    \fi
    \expandafter
  \endgroup
  \childdoctmp
}
%    \end{macrocode}

% \macro{\childdocforwardprefix}
% The command |\childdocforwardprefix| redirects
% compilation to the main or a child file by means of a pattern.
% The prefix |#1| in the current filename is replaced by |#2|
% and the suffix of the current filename is kept
% (it is assumed that the filename does not contain the substring `|~~~|'
% which is used as a delimiter).
% Compilation is handed over to the new file by |\childdocforward|:
%    \begin{macrocode}
\newcommand{\childdocforwardprefix}[3][]
{
  \begingroup
    \def\childdocextract #2##1~~~{\def\childdoctmp{\childdocforward[#1]{#3##1}}}
    \expandafter\childdocextract\childdocname~~~
    \expandafter
  \endgroup
  \childdoctmp
}
%    \end{macrocode}

% \macro{\childdoc}
% The deprecated macro |\childdoc| is a legacy version of |\childdocmain|:
%    \begin{macrocode}
\newcommand{\childdoc}{\childdocmain}
%    \end{macrocode}

% \macro{\childdocredirect}
% The deprecated macro |\childdocredirect| is a legacy version
% of |\childdocforward| and |\childdocforwardprefix|:
%    \begin{macrocode}
\newcommand{\childdocredirect}[2][]
{
  \begingroup
    \if?#1?
      \def\childdoctmp{\childdocforward{#2}}
    \else
      \def\childdoctmp{\childdocforwardprefix{#1}{#2}}
    \fi
    \expandafter
  \endgroup
  \childdoctmp
}
%    \end{macrocode}

%\iffalse
%</package>
%\fi
%
\endinput
\childdocforward[|\textit{main}|]{|\textit{dest}|}"|
\end{center}
%
Here \textit{target} is the name of the output file,
\textit{main} is the name of the main file
and \textit{dest} is the name of the main or child file to be processed
(all filenames without extensions).
The optional argument \textit{main} can be omitted
if \textit{main} matches \textit{dest}.
Optionally, compilation \textit{flags} can be defined via |\def| commands.
This command line makes the \TeX{} engine believe
it is compiling the file \textit{target}
whose content is specified as the latter parameter.
The provided code then forwards the processing to
\textit{main} or \textit{dest} as described in \secref{sec:forward}.

%%%%%%%%%%%%%%%%%%%%%%%%%%%%%%%%%%%%%%%%%%%%%%%%%%%%%%%%%%%%%%%%%%%%%%%%%%%%%%%%
\subsection{Include by Input}
\label{sec:input}

Including child documents by |\include| has some restrictions by design.
Most notably, the content of a child document always occupies
its own set of pages; pages cannot be shared between child documents.
Usually, this behaviour makes perfect sense
because each child document contain an essential part of the document.
However, in some situations it may be desirable to compose
a document from a collection of parts
without having mandatory page breaks between then.
For this case, the package
provides a mechanism to include parts
by |\input| which can also be processed individually.
However, by construction this mechanism
requires manual handling of the content to be output.

%%%%%%%%%%%%%%%%%%%%%%%%%%%%%%%%%%%%%%%%
\DescribeMacro{\ifchilddocmanual}
The main file should be prepared as usual, see \secref{sec:include}.
However, the document body must make a distinction
between processing of an individual part and of the main document, e.g.:
%
\begin{center}
\begin{tabular}{l}
|\ifchilddocmanual|\\
|\input{\childdocname}|\\
|\||else|\\
\textit{document body with }|\input{|\textit{part}|}|\\
|\||fi|
\end{tabular}
\end{center}
%
The conditional |\ifchilddocmanual| is true whenever
a part to be included by |\input| is being compiled,
and the name of the part is stored in |\childdocname|.

%%%%%%%%%%%%%%%%%%%%%%%%%%%%%%%%%%%%%%%%
\DescribeMacro{\childdocby}
Each part to be included by |\input| should start with:
%
\begin{center}
\begin{tabular}{l}
|% \iffalse
%
% childdoc.dtx Copyright (C) 2017-2018 Niklas Beisert
%
% This work may be distributed and/or modified under the
% conditions of the LaTeX Project Public License, either version 1.3
% of this license or (at your option) any later version.
% The latest version of this license is in
%   http://www.latex-project.org/lppl.txt
% and version 1.3 or later is part of all distributions of LaTeX
% version 2005/12/01 or later.
%
% This work has the LPPL maintenance status `maintained'.
%
% The Current Maintainer of this work is Niklas Beisert.
%
% This work consists of the files childdoc.dtx and childdoc.ins
% and the derived files childdoc.def and cdocsamp.tex with
% cdocsch1.tex, cdocsch2.tex, cdocsdrf.tex, cdocsfn1.tex, cdocsfn2.tex.
%
%<package>\ifdefined\childdocmain\endinput\fi
%<package>\ProvidesFile{childdoc.def}[2018/12/30 v2.0 child document driver]
%<samplemain>\ProvidesFile{cdocsamp.tex}[2018/12/30 v2.0 sample for childdoc]
%<*driver>
%\ProvidesFile{childdoc.drv}[2018/12/30 v2.0 childdoc reference manual file]
\PassOptionsToClass{10pt,a4paper}{article}
\documentclass{ltxdoc}

\usepackage[margin=35mm]{geometry}
\usepackage{hyperref}
\usepackage{hyperxmp}
\usepackage[usenames]{color}

\hypersetup{colorlinks=true}
\hypersetup{pdfstartview=FitH}
\hypersetup{pdfpagemode=UseNone}
\hypersetup{pdfsource={}}
\hypersetup{pdflang={en-UK}}
\hypersetup{pdfcopyright={Copyright 2017-2018 Niklas Beisert.
  This work may be distributed and/or modified under the
  conditions of the LaTeX Project Public License, either version 1.3
  of this license or (at your option) any later version.}}
\hypersetup{pdflicenseurl={http://www.latex-project.org/lppl.txt}}
\hypersetup{pdfcontactaddress={ETH Zurich, ITP, HIT K,
  Wolfgang-Pauli-Strasse 27}}
\hypersetup{pdfcontactpostcode={8093}}
\hypersetup{pdfcontactcity={Zurich}}
\hypersetup{pdfcontactcountry={Switzerland}}
\hypersetup{pdfcontactemail={nbeisert@itp.phys.ethz.ch}}
\hypersetup{pdfcontacturl={http://people.phys.ethz.ch/\xmptilde nbeisert/}}

\newcommand{\secref}[1]{\hyperref[#1]{section \ref*{#1}}}

\parskip1ex
\parindent0pt
\let\olditemize\itemize
\def\itemize{\olditemize\parskip0pt}

\begin{document}

\title{The \textsf{childdoc} Package}
\hypersetup{pdftitle={The childdoc Package}}
\author{Niklas Beisert\\[2ex]
  Institut f\"ur Theoretische Physik\\
  Eidgen\"ossische Technische Hochschule Z\"urich\\
  Wolfgang-Pauli-Strasse 27, 8093 Z\"urich, Switzerland\\[1ex]
  \href{mailto:nbeisert@itp.phys.ethz.ch}
  {\texttt{nbeisert@itp.phys.ethz.ch}}}
\hypersetup{pdfauthor={Niklas Beisert}}
\hypersetup{pdfsubject={Manual for the LaTeX2e Package childdoc}}
\date{30 December 2018, \textsf{v2.0}}
\maketitle

\begin{abstract}\noindent
\textsf{childdoc} is a \LaTeXe{} package
that enables the direct compilation
of document sections included by |\include|
to individual files.
\end{abstract}

\begingroup
\parskip0ex
\tableofcontents
\endgroup

%%%%%%%%%%%%%%%%%%%%%%%%%%%%%%%%%%%%%%%%%%%%%%%%%%%%%%%%%%%%%%%%%%%%%%%%%%%%%%%%
%%%%%%%%%%%%%%%%%%%%%%%%%%%%%%%%%%%%%%%%%%%%%%%%%%%%%%%%%%%%%%%%%%%%%%%%%%%%%%%%
\section{Introduction}

\LaTeX{} provides a mechanism to structure a large document (such as a book)
into a main file and several child files (containing the chapters)
using the |\include| command.
This mechanism is beneficial for documents
which span hundreds of pages in order to
make the source file(s) more manageable.
Moreover, compilation can be restricted to
selected child files by means of the |\includeonly| command.
The latter feature can be used to reduce the compilation time while editing
(this was significantly more useful in the earlier days of \LaTeX{})
or to generate a smaller document which is easier to navigate.
Another application of |\includeonly| is to generate
documents consisting of selected parts of the complete document.

However, there are a few drawbacks of the plain |\include| mechanism:
\begin{itemize}
\item
The child files cannot be compiled on their own,
they can only be compiled via the main file.
A naive editing environment
(such as a text editor with an option
to have the current file processed by \LaTeX)
may require one to switch to the main file before compiling;
attempting to compile the child file produces errors.
\item
The main file must be modified (each time)
to adjust the |\includeonly| command
to the present needs. This easily leaves the main file in a messy state.
\item
The generated document will always carry the filename
of the main document. This is inconvenient if
several child files are to be compiled and
to be kept for distribution.
\end{itemize}

The present package provides a simple interface
to make child files individually compilable by \LaTeX{}.
Compiling a child file then has the same effect as compiling
the main file with an |\includeonly| command
to select the appropriate child.
Moreover the generated document will carry the name of the child
rather than the main file.
This resolves all three above issues.

This feature is meant to make the editing of books,
thesis documents and lecture notes somewhat more convenient.
However, the package can also be used efficiently for
composing a series of documents (such as exercise sheets)
which are typically distributed individually.
It then assists the author in generating the individual documents
(potentially in different versions)
as well as a document containing the collected series.
Another application is in developing style files
or other kinds of included material
where compilation of the style file could redirect
to a sample or test file.

%%%%%%%%%%%%%%%%%%%%%%%%%%%%%%%%%%%%%%%%%%%%%%%%%%%%%%%%%%%%%%%%%%%%%%%%%%%%%%%%
%%%%%%%%%%%%%%%%%%%%%%%%%%%%%%%%%%%%%%%%%%%%%%%%%%%%%%%%%%%%%%%%%%%%%%%%%%%%%%%%
\section{Usage}

First of all, the package \textsf{childdoc} is \emph{not} a standard
\LaTeXe{} |.sty| style file! Therefore it needs to be invoked in
a non-standard way.

%%%%%%%%%%%%%%%%%%%%%%%%%%%%%%%%%%%%%%%%%%%%%%%%%%%%%%%%%%%%%%%%%%%%%%%%%%%%%%%%
\subsection{Included Files}
\label{sec:include}

%%%%%%%%%%%%%%%%%%%%%%%%%%%%%%%%%%%%%%%%
\DescribeMacro{\childdocmain}
To use the package, add the commands
\begin{center}
\begin{tabular}{l}
|\input{childdoc.def}|\\
|\childdocmain{}|\\
\end{tabular}
\end{center}
at the very top of the main \LaTeX{} file,
in particular \emph{before} the |\documentclass| statement!
The argument of |\childdocmain| should be left empty
(but it must be present).

%%%%%%%%%%%%%%%%%%%%%%%%%%%%%%%%%%%%%%%%
\DescribeMacro{\childdocof}
Furthermore, add the commands
\begin{center}
\begin{tabular}{l}
|\input{childdoc.def}|\\
|\childdocof{|\textit{main}|}|\\
\end{tabular}
\end{center}
at the top of every child file \textit{child}
which is included by |\include{|\textit{child}|}|
from within the main file
(or at least for those files to be compiled individually).
The argument \textit{main} must be the filename of the main file.

There are a couple of
considerations in setting up the main and child documents:

%%%%%%%%%%%%%%%%%%%%%%%%%%%%%%%%%%%%%%%%
\paragraph{Restrictions.}

Please note the following restrictions:
\begin{itemize}
\item
|\childdocmain| must be called with one argument \textit{main}
to ensure compatibility with earlier version of the package.
It must either be empty (|\childdocmain{}|)
or precisely match the filename of the main file in which it is specified.
See \secref{sec:detection} for further information.
\item
The filename \textit{main} must be specified without the |.tex| extension.
\item
The filename \textit{main} is case sensitive
(even in case-insensitive file systems)
due to internal string comparison.
\item
The argument \textit{main} should be fully expanded, it cannot be a macro.
\item
Subdirectories and special characters should be avoided in filenames.
\item
The command |\childdocmain{|\textit{main}|}| must be followed by a whitespace.
It should not be followed immediately by another command
or by a comment mark `|%|'.
This is because the \TeX{} parser reads the token immediately following
the argument of |\childdocmain| and puts it
at the beginning of every child section;
however, a white\-space is ignored.
\end{itemize}

%%%%%%%%%%%%%%%%%%%%%%%%%%%%%%%%%%%%%%%%
\paragraph{Content of Main File.}

It is advisable to place all content in the child files included by |\include|.
Any output contained in the main file will appear in all child documents
unless suppressed manually;
it cannot be suppressed automatically by the |\includeonly| directive
and thus should normally be avoided.
A method to include some content in the main file
by means of conditional processing is described in \secref{sec:conditional}.

%%%%%%%%%%%%%%%%%%%%%%%%%%%%%%%%%%%%%%%%
\paragraph{Page Numbering.}

When only a part of the document is compiled,
the appropriate numbering of pages
(as well as other status parameters)
is determined from the |.aux| files.
The latter contain information from previous passes.
However this information needs to propagate through
all intermediate child documents.
Therefore the page numbering in child documents may well
be inconsistent until the complete document is compiled at least once.

A useful (if unconventional) way to always ensure a consistent
page numbering is to restart the numbering in each child document
and denote the pages by `\textit{child}|.|\textit{page}'
where \textit{child} represents the chapter/section number of the child file.
This can be achieved by the command
|\numberwithin{page}{|\textit{child}|}|
of the \textsf{amsmath} package
where \textit{child} can be |chapter| or |section|
depending on the chosen structuring.
Alternatively, one can modify the macro |\thepage| appropriately
and reset the counter |page| at the start of each child file.

%%%%%%%%%%%%%%%%%%%%%%%%%%%%%%%%%%%%%%%%%%%%%%%%%%%%%%%%%%%%%%%%%%%%%%%%%%%%%%%%
\subsection{Conditional Processing}
\label{sec:conditional}

The package provides a mechanism to compile different versions
of a document. To customise the versions further some conditional processing
can come in handy to distinguish which version is being compiled.
The package provides two macros to describe the compilation context:

%%%%%%%%%%%%%%%%%%%%%%%%%%%%%%%%%%%%%%%%
\DescribeMacro{\ifchilddoc}
The conditional |\ifchilddoc| distinguishes between the compilation of
child documents and the main document:
%
\begin{center}
|\ifchilddoc |\textit{child-code}| |[|\||else |\textit{main-code}]| \||fi|
\end{center}

%%%%%%%%%%%%%%%%%%%%%%%%%%%%%%%%%%%%%%%%
\DescribeMacro{\childdocname}
\DescribeMacro{\childdocjob}
The macro |\childdocname| contains the filename (without extension)
of the main or child file being processed.
Note that |\childdocjob| will always contain the name of the main file.

%%%%%%%%%%%%%%%%%%%%%%%%%%%%%%%%%%%%%%%%
\paragraph{Title Page.}

Conditional processing can be used to include a title or banner page
in the main document when proper precautions are taken.
Importantly, the code in the main file should ensure that the page counter
(as well as other status parameters which are stored in the |.aux| files)
takes the same value after the conditional processing.
Otherwise the page numbers may take divergent values
depending on which part is compiled.

For example, a title page could be declared by:
%
\begin{center}
\begin{tabular}{l}
|\ifchilddoc\||else|\\
|\addtocounter{page}{-1}|\\
\textit{code for title page}\\
|\newpage|\\
|\||fi|
\end{tabular}
\end{center}
%
A banner page for the child documents can be generated by:
%
\begin{center}
\begin{tabular}{l}
|\ifchilddoc|\\
|\addtocounter{page}{-1}|\\
\textit{code for banner page}\\
|\newpage|\\
|\||fi|
\end{tabular}
\end{center}
%
Here one could write a message such as:
\begin{center}
|This is the part \childdocname{} of \childdocjob{}.|
\end{center}

%%%%%%%%%%%%%%%%%%%%%%%%%%%%%%%%%%%%%%%%%%%%%%%%%%%%%%%%%%%%%%%%%%%%%%%%%%%%%%%%
\subsection{Flags}
\label{sec:flags}

The package makes it easy to generate different versions
of the main or child documents.
To this end compilation flags can be defined
and assigned different default values.
They will be particularly useful in conjunction
with the forwarding mechanism described in \secref{sec:forward}.

For example, it may be useful to have a flag |\version|
which can be set to |draft| or |final|.
The document source will contain some conditional code
depending on the value of |\version|.
Suppose further, the flag should default to |final| for the main file
and to |draft| for child files
which is a natural assignment for editing the document.
This is achieved by placing the following code
in the preamble of the main document
(below the |\childdocmain| directive):
%
\begin{center}
\begin{tabular}{l}
|\ifchilddoc|\\
|\providecommand{\version}{draft}|\\
|\||else|\\
|\providecommand{\version}{final}|\\
|\||fi|
\end{tabular}
\end{center}
%
The definition by |\providecommand| makes sure
that previous definitions are not overwritten.
Further statements |\providecommand{\version}{...}|
can thus be added before the above code to override it.

For the main file, one might add a line
(between |\childdocmain| and the above block)
%
\begin{center}
|%\ifchilddoc\||else\providecommand{\version}{draft}\||fi|
\end{center}
%
which can be uncommented to produce a draft version.
Likewise one can add a line to the very top of a child file
(above the |\childdocof{|\textit{main}|}| directive)
%
\begin{center}
|%\providecommand{\version}{final}|
\end{center}
%
which can be uncommented to produce the final version of this child document.

%%%%%%%%%%%%%%%%%%%%%%%%%%%%%%%%%%%%%%%%%%%%%%%%%%%%%%%%%%%%%%%%%%%%%%%%%%%%%%%%
\subsection{Forwarding}
\label{sec:forward}

Different versions of the main or child documents
using compilation flags as described in \secref{sec:flags}
can be (permanently) stored in different files
for convenient compilation, viewing and distribution.
To this end, the package defines a command
to pass on compilation to a different file:

%%%%%%%%%%%%%%%%%%%%%%%%%%%%%%%%%%%%%%%%
\DescribeMacro{\childdocforward}
The command |\childdocforward| redirects processing to
another source file:
%
\begin{center}
\begin{tabular}{l}
|\input{childdoc.def}|\\
|\childdocforward[|\textit{main}|]{|\textit{dest}|}|\\
\end{tabular}
\end{center}
%
The argument \textit{dest} is the destination file
(without extension).
It should be the main file or one of the child files.
Note that further \textsf{childdoc} directives
such as |\childdocof| and |\childdocforward|
in the indicated file will be processed in this form.
The optional argument \textit{main}
passes on directly to the main file \textit{main}
while pretending to compile the child \textit{dest}.
This form behaves as if \textit{dest}
issues |\childdocof{|\textit{main}|}| right away,
and no further \textsf{childdoc} directives will be processed.

%%%%%%%%%%%%%%%%%%%%%%%%%%%%%%%%%%%%%%%%
\DescribeMacro{\...prefix}
In the alternative form |\childdocforwardprefix|,
%
\begin{center}
\begin{tabular}{l}
|\input{childdoc.def}|\\
|\childdocforwardprefix[|\textit{main}|]{|\textit{prefix}|}{|\textit{dest}|}|
\end{tabular}
\end{center}
%
the destination file is determined by a pattern
depending on the current file:
To make this work, the current file must be called
`{\textit{prefix}\hspace{0.2em}\textit{suffix}}'
with \textit{prefix} matching precisely the argument.
Processing is then passed on to the file
`{\textit{dest}\hspace{0.2em}\textit{suffix}}'.
Surely, the same effect is achieved by
directly specifying the
argument `{\textit{dest}\hspace{0.2em}\textit{suffix}}'
in the first form.
However, that requires to set up a different file
for each child. With the alternative form of the command
all these files can have exactly the same content
which simplifies setting them up and maintaining them.

For example, the following file |draft.tex|
with a compilation flag |\version| as described in \secref{sec:flags}
compiles the main document as a draft:
%
\begin{center}
\begin{tabular}{l}
|\def\version{draft}|\\
|\input{childdoc.def}|\\
|\childdocforward{|\textit{main}|}|
\end{tabular}
\end{center}
%
Likewise, the following files |final|\textit{nn}|.tex|
compile the final version of the child document
|child|\textit{nn}|.tex|:
%
\begin{center}
\begin{tabular}{l}
|\def\version{final}|\\
|\input{childdoc.def}|\\
|\childdocforwardprefix{final}{child}|
\end{tabular}
\end{center}
%

Note that when several versions of a main file and/or of each child file
are to be generated, it may be convenient to set up a |Makefile| or
shell script to automatise the process.

%%%%%%%%%%%%%%%%%%%%%%%%%%%%%%%%%%%%%%%%%%%%%%%%%%%%%%%%%%%%%%%%%%%%%%%%%%%%%%%%
\subsection{Command Line Processing}
\label{sec:commandline}

The effect of redirection files can also be achieved by invoking
the \LaTeX{} compiler with a more elaborate command line.
Most conveniently this should be done as part
of a shell script or a |Makefile|.

When using \textsf{childdoc} in the main file, the following
command lines effectively perform a redirection
(note that depending on the shell being used,
backslashes may have to be doubled: `|\|' $\to$ `|\\|'):
%
\begin{center}
|... -jobname "|\textit{target}|" |\\|"|[\textit{flags}]%
|\input{childdoc.def}\childdocforward[|\textit{main}|]{|\textit{dest}|}"|
\end{center}
%
Here \textit{target} is the name of the output file,
\textit{main} is the name of the main file
and \textit{dest} is the name of the main or child file to be processed
(all filenames without extensions).
The optional argument \textit{main} can be omitted
if \textit{main} matches \textit{dest}.
Optionally, compilation \textit{flags} can be defined via |\def| commands.
This command line makes the \TeX{} engine believe
it is compiling the file \textit{target}
whose content is specified as the latter parameter.
The provided code then forwards the processing to
\textit{main} or \textit{dest} as described in \secref{sec:forward}.

%%%%%%%%%%%%%%%%%%%%%%%%%%%%%%%%%%%%%%%%%%%%%%%%%%%%%%%%%%%%%%%%%%%%%%%%%%%%%%%%
\subsection{Include by Input}
\label{sec:input}

Including child documents by |\include| has some restrictions by design.
Most notably, the content of a child document always occupies
its own set of pages; pages cannot be shared between child documents.
Usually, this behaviour makes perfect sense
because each child document contain an essential part of the document.
However, in some situations it may be desirable to compose
a document from a collection of parts
without having mandatory page breaks between then.
For this case, the package
provides a mechanism to include parts
by |\input| which can also be processed individually.
However, by construction this mechanism
requires manual handling of the content to be output.

%%%%%%%%%%%%%%%%%%%%%%%%%%%%%%%%%%%%%%%%
\DescribeMacro{\ifchilddocmanual}
The main file should be prepared as usual, see \secref{sec:include}.
However, the document body must make a distinction
between processing of an individual part and of the main document, e.g.:
%
\begin{center}
\begin{tabular}{l}
|\ifchilddocmanual|\\
|\input{\childdocname}|\\
|\||else|\\
\textit{document body with }|\input{|\textit{part}|}|\\
|\||fi|
\end{tabular}
\end{center}
%
The conditional |\ifchilddocmanual| is true whenever
a part to be included by |\input| is being compiled,
and the name of the part is stored in |\childdocname|.

%%%%%%%%%%%%%%%%%%%%%%%%%%%%%%%%%%%%%%%%
\DescribeMacro{\childdocby}
Each part to be included by |\input| should start with:
%
\begin{center}
\begin{tabular}{l}
|\input{childdoc.def}|\\
|\childdocby{|\textit{main}|}|\\
\end{tabular}
\end{center}
%
The directive |\childdocby| is similar to |\childdocof|
described in \secref{sec:include},
but the subsequent selection of content must be done manually.
To that end, both |\ifchilddoc| and |\ifchilddocmanual|
will be true upon processing of a part,
and the name of the part is stored in |\childdocname|.
Note that |\jobname| will be set to the filename of the current part
so that each part receives an individual |.aux| file
that does not interfere with the |.aux| file(s) of the main document.
This behaviour can be altered by the alternative form
|\childdocby[*]{|\textit{main}|}| (with a non-empty optional argument)
which uses the |.aux| file of the main document
by setting |\jobname| to \textit{main}.

%%%%%%%%%%%%%%%%%%%%%%%%%%%%%%%%%%%%%%%%%%%%%%%%%%%%%%%%%%%%%%%%%%%%%%%%%%%%%%%%
\subsection{Driver Development}
\label{sec:driver}

The \textsf{childdoc} mechanism can also be use for the development
of definition files such as \LaTeX{} styles or classes.
This case differs from the above setup with multiple parts
included by |\include| in that no |\includeonly| should be invoked.
This can be achieved by starting the include file
(before |\ProvidesPackage|) with:
%
\begin{center}
\begin{tabular}{l}
|\input{childdoc.def}|\\
|\childdocforward{|\textit{main}|}|\\
\end{tabular}
\end{center}
%
or alternatively with:
%
\begin{center}
\begin{tabular}{l}
|\input{childdoc.def}|\\
|\childdocby{|\textit{main}|}|\\
\end{tabular}
\end{center}
%
Both forms have slightly different effects as described above.
The main file is prepared as usual, see \secref{sec:include}.

%%%%%%%%%%%%%%%%%%%%%%%%%%%%%%%%%%%%%%%%%%%%%%%%%%%%%%%%%%%%%%%%%%%%%%%%%%%%%%%%
\subsection{Legacy Detection}
\label{sec:detection}

The directive |\childdocmain| in the main file can detect
whether the complete document or merely a child is to be compiled
even without using the directive |\childdocof|.
This method is deprecated because it is less robust
and there is no compelling reason to use it;
it is merely provided for backward compatibility
and it may be removed in future versions.

If the detection mechanism is to be used,
it is mandatory to correctly specify
the filename of the main file as the argument of |\childdocmain|:
%
\begin{center}
\begin{tabular}{l}
|\input{childdoc.def}|\\
|\childdocmain{|\textit{main}|}|\\
\end{tabular}
\end{center}
%
If |\jobname| does not match the argument \textit{main} of |\childdocmain|,
it is assumed that |\jobname| points to the child file to be compiled.
When using |\childdocmain| with the main file specified as argument,
it suffices to start a child file
with just |\input{|\textit{main}|}|
without loading of the package and using |\childdocof|.
If instead all processing is done
with the appropriate \textsf{childdoc} directives,
the argument of \textit{main} of |\childdocmain| can be empty.

An alternative version of the command line processing described
in \secref{sec:commandline} using the detection mechanism reads:
%
\begin{center}
|... -jobname "|\textit{target}|" "|[\textit{flags}]%
[|\def\jobname{|\textit{dest}|}|]|\input{|\textit{main}|}"|
\end{center}

%%%%%%%%%%%%%%%%%%%%%%%%%%%%%%%%%%%%%%%%%%%%%%%%%%%%%%%%%%%%%%%%%%%%%%%%%%%%%%%%
\subsection{Manual Code}
\label{sec:manual}

In case one cannot be certain whether the definitions file |childdoc.def|
is installed on the target \TeX{} distribution
and one prefers not to ship it,
it is conceivable to paste a few relevant commands into the sources.

To that end, drop all statements |\input{childdoc.def}|
and perform the replacements as outlined below.
Instead of |\childdocmain{|\textit{main}|}| add the following code
to the top of the main file:
%
\begin{center}
\begin{tabular}{l}
|\||ifdefined\childdocname\endinput\||fi\newif\ifchilddoc|\\
|\edef\childdocname{\scantokens\expandafter{\jobname\noexpand}}|\\
|\def\childdocmain{|\textit{main}|}\||ifx\childdocmain\childdocname\||else|\\
|\childdoctrue\includeonly{\childdocname}\let\jobname\childdocmain\||fi|\\
\end{tabular}
\end{center}
%
Instead of |\childdocof{|\textit{main}|}| just include the main file
at the top of each child file:
%
\begin{center}
|\input{|\textit{main}|}|
\end{center}
%
A simple redirection |\childdocforward{|\textit{dest}|}| is achieved by:
%
\begin{center}
|\def\jobname{|\textit{dest}|}\input{\jobname}|
\end{center}
%
The redirection with prefix
|\childdocforwardprefix[|\textit{prefix}|]{|\textit{dest}|}|
is accomplished by:
%
\begin{center}
\begin{tabular}{l}
|{\edef\jobname{\scantokens\expandafter{\jobname\noexpand}}|\\
|\def\redirectjob |\textit{prefix}|#1~~~{\gdef\jobname{|\textit{dest}|#1}}|\\
|\expandafter\redirectjob\jobname~~~}\input{\jobname}|
\end{tabular}
\end{center}

In an alternative approach,
child documents can be compiled by a specific command line
without additional code or specific definitions:
%
\begin{center}
|... -jobname "|\textit{target}|" "|[\textit{flags}]%
|\includeonly{|\textit{dest}|}\input{|\textit{main}|}"|
\end{center}
%

%%%%%%%%%%%%%%%%%%%%%%%%%%%%%%%%%%%%%%%%%%%%%%%%%%%%%%%%%%%%%%%%%%%%%%%%%%%%%%%%
%%%%%%%%%%%%%%%%%%%%%%%%%%%%%%%%%%%%%%%%%%%%%%%%%%%%%%%%%%%%%%%%%%%%%%%%%%%%%%%%
\section{Information}

%%%%%%%%%%%%%%%%%%%%%%%%%%%%%%%%%%%%%%%%%%%%%%%%%%%%%%%%%%%%%%%%%%%%%%%%%%%%%%%%
\subsection{Copyright}

Copyright \copyright{} 2017--2018 Niklas Beisert

This work may be distributed and/or modified under the
conditions of the \LaTeX{} Project Public License, either version 1.3
of this license or (at your option) any later version.
The latest version of this license is in
  \url{http://www.latex-project.org/lppl.txt}
and version 1.3 or later is part of all distributions of \LaTeX{}
version 2005/12/01 or later.

This work has the LPPL maintenance status `maintained'.

The Current Maintainer of this work is Niklas Beisert.

This work consists of the files |README.txt|, |childdoc.ins| and |childdoc.dtx|
as well as the derived files |childdoc.def|, |cdocsamp.tex|
with |cdocsch1.tex|, |cdocsch2.tex|, |cdocspt3.tex|, |cdocspt4.tex|,
|cdocsdrf.tex|, |cdocsfn1.tex|, |cdocsfn2.tex|
as well as |childdoc.pdf|.

%%%%%%%%%%%%%%%%%%%%%%%%%%%%%%%%%%%%%%%%%%%%%%%%%%%%%%%%%%%%%%%%%%%%%%%%%%%%%%%%
\subsection{Files and Installation}

The package consists of the files:
%
\begin{center}
\begin{tabular}{ll}
    |README.txt|   & readme file \\
    |childdoc.ins| & installation file \\
    |childdoc.dtx| & source file \\
    |childdoc.def| & definition file \\
    |cdocsamp.tex| & sample main file \\
    |cdocsch1.tex| & sample include file \\
    |cdocsch2.tex| & sample include file \\
    |cdocspt3.tex| & sample part file \\
    |cdocspt4.tex| & sample part file \\
    |cdocsdrf.tex| & sample redirection file \\
    |cdocsfn1.tex| & sample redirection file \\
    |cdocsfn2.tex| & sample redirection file \\
    |childdoc.pdf| & manual
\end{tabular}
\end{center}
%
The distribution consists of the files
|README.txt|, |childdoc.ins| and |childdoc.dtx|.
%
\begin{itemize}
\item
Run (pdf)\LaTeX{} on |childdoc.dtx|
to compile the manual |childdoc.pdf| (this file).
\item
Run \LaTeX{} on |childdoc.ins| to create the definitions file |childdoc.def|
and the sample |cdocsamp.tex| with include files
|cdocsch1.tex|, |cdocsch2.tex|, |cdocspt3.tex|, |cdocspt4.tex|,
|cdocsdrf.tex|, |cdocsfn1.tex|, |cdocsfn2.tex|.
Then copy the file |childdoc.def| to an appropriate directory of your \LaTeX{}
distribution, e.g.\ \textit{texmf-root}|/tex/latex/childdoc|.
\end{itemize}

%%%%%%%%%%%%%%%%%%%%%%%%%%%%%%%%%%%%%%%%%%%%%%%%%%%%%%%%%%%%%%%%%%%%%%%%%%%%%%%%
\subsection{Related CTAN Packages}

There are several other packages which offer a similar functionality:
%
\begin{itemize}
\item
The packages
\href{http://ctan.org/pkg/docmute}{\textsf{docmute}},
\href{http://ctan.org/pkg/includex}{\textsf{includex}} and
\href{http://ctan.org/pkg/standalone}{\textsf{standalone}}
provide commands to include only the document body of
a child file thus allowing both files to be compiled individually.
\item
The packages \href{http://ctan.org/pkg/subdocs}{\textsf{subdocs}}
and \href{http://ctan.org/pkg/subfiles}{\textsf{subfiles}}
provide structures in which the main and child documents can be
encapsulated and allowing them to be compiled individually.
The inclusion mechanism is different from the conventional |\include|.
\item
The package \href{http://ctan.org/pkg/combine}{\textsf{combine}}
is an elaborate solution to combine several documents into one.
\end{itemize}
%
See also the CTAN topic \href{http://ctan.org/topic/subdocs}{\textsf{subdocs}}
for further related packages.
The present package differs from the above solutions in that
a document structure constructed with the conventional |\include| mechanism
just needs two extra commands at the top of every file
such that all constituent files can be compiled individually.

%%%%%%%%%%%%%%%%%%%%%%%%%%%%%%%%%%%%%%%%%%%%%%%%%%%%%%%%%%%%%%%%%%%%%%%%%%%%%%%%
%\subsection{Feature Suggestions}
%
%The following is a list of features which may be useful for future
%versions of this package:
%%
%\begin{itemize}
%\item
%\ldots
%\end{itemize}

%%%%%%%%%%%%%%%%%%%%%%%%%%%%%%%%%%%%%%%%%%%%%%%%%%%%%%%%%%%%%%%%%%%%%%%%%%%%%%%%
\subsection{Revision History}

%%%%%%%%%%%%%%%%%%%%%%%%%%%%%%%%%%%%%%%%
\paragraph{v2.0:} 2018/12/30

\begin{itemize}
\item
immediate forward processing
\item
added |\childdocby| mechanism
\item
manual restructured
\end{itemize}

%%%%%%%%%%%%%%%%%%%%%%%%%%%%%%%%%%%%%%%%
\paragraph{v1.6:} 2018/01/17

\begin{itemize}
\item
application for development of include files
\item
corrections to manual
\end{itemize}

%%%%%%%%%%%%%%%%%%%%%%%%%%%%%%%%%%%%%%%%
\paragraph{v1.5:} 2017/05/21

\begin{itemize}
\item
more complete structuring introduced
\item
|\childdocof| introduced
\item
|\childdoc| renamed to |\childdocmain|
\item
|\childredirect| renamed to |\childdocforward| and |\childdocforwardprefix|
and functionality expanded
\end{itemize}

%%%%%%%%%%%%%%%%%%%%%%%%%%%%%%%%%%%%%%%%
\paragraph{v1.0:} 2017/04/27

\begin{itemize}
\item
manual and install package
\item
first version published on CTAN
\end{itemize}

%%%%%%%%%%%%%%%%%%%%%%%%%%%%%%%%%%%%%%%%
\paragraph{v0.6:} 2017/04/26

\begin{itemize}
\item
redirection mechanism added
\end{itemize}

%%%%%%%%%%%%%%%%%%%%%%%%%%%%%%%%%%%%%%%%
\paragraph{v0.5:} 2017/04/26

\begin{itemize}
\item
functionality in definition file
\end{itemize}


%%%%%%%%%%%%%%%%%%%%%%%%%%%%%%%%%%%%%%%%%%%%%%%%%%%%%%%%%%%%%%%%%%%%%%%%%%%%%%%%
%%%%%%%%%%%%%%%%%%%%%%%%%%%%%%%%%%%%%%%%%%%%%%%%%%%%%%%%%%%%%%%%%%%%%%%%%%%%%%%%
%%%%%%%%%%%%%%%%%%%%%%%%%%%%%%%%%%%%%%%%%%%%%%%%%%%%%%%%%%%%%%%%%%%%%%%%%%%%%%%%
\appendix

\settowidth\MacroIndent{\rmfamily\scriptsize 000\ }

 \DocInput{childdoc.dtx}

\end{document}
%</driver>
% \fi
%
% %%%%%%%%%%%%%%%%%%%%%%%%%%%%%%%%%%%%%%%%%%%%%%%%%%%%%%%%%%%%%%%%%%%%%%%%%%%%%%
% %%%%%%%%%%%%%%%%%%%%%%%%%%%%%%%%%%%%%%%%%%%%%%%%%%%%%%%%%%%%%%%%%%%%%%%%%%%%%%
% \section{Sample}
%\iffalse
%<*samplemain>
%\fi
%
% The following presents a sample document
% with two chapters, two parts, a title page,
% a compile flag as well as three forwarding files to set the flag.
% It consists of eight |.tex| files:
% \begin{center}
% \begin{tabular}{ll}
% |cdocsamp.tex|&main file\\
% |cdocsch1.tex|&include file for chapter 1\\
% |cdocsch2.tex|&include file for chapter 2\\
% |cdocspt3.tex|&include file for part 3\\
% |cdocspt4.tex|&include file for part 4\\
% |cdocsdrf.tex|&forwarding file for main file in draft mode\\
% |cdocsfi1.tex|&forwarding file for final version of chapter 1\\
% |cdocsfi2.tex|&forwarding file for final version of chapter 2\\
% \end{tabular}
% \end{center}
% Each of the eight files can be compiled directly by the \LaTeX{} compiler.
%
% %%%%%%%%%%%%%%%%%%%%%%%%%%%%%%%%%%%%%%
% \paragraph{Main File.}
%
% The main file is called |cdocsamp.tex|.
%
% Load the \textsf{childdoc} definitions and
% declare the filename for the main document:
%    \begin{macrocode}
\input{childdoc.def}
\childdocmain{}
%    \end{macrocode}

% Optional override for |\version| flag:
%    \begin{macrocode}
%%\ifchilddoc\else\providecommand{\version}{draft}\fi
%    \end{macrocode}

% Define the default values for the |\version| flag
% (|final| for the main file and |draft| for childs):
%    \begin{macrocode}
\ifchilddoc
\providecommand{\version}{draft}
\else
\providecommand{\version}{final}
\fi
%    \end{macrocode}

% Load the standard document class:
%    \begin{macrocode}
\documentclass[12pt]{article}
%    \end{macrocode}

% Start the document body:
%    \begin{macrocode}
\begin{document}
%    \end{macrocode}

% Declare a title page.
% Print title, part of document being processed and version flag:
%    \begin{macrocode}
\addtocounter{page}{-1}
\begin{center}
{\LARGE\bfseries{}childdoc example\par}
\vspace{1cm}
\ifchilddoc
\ifchilddocmanual part\else chapter\fi:
`\childdocname' of `\childdocjob'\par
\else
main document: `\childdocjob'\par
\fi
version: \version\par
\end{center}
\newpage
%    \end{macrocode}

% Manually include selected file,
% otherwise process as usual:
%    \begin{macrocode}
\ifchilddocmanual
\section*{part `\childdocname'}
\input{\childdocname}
\else
%    \end{macrocode}

% Include the two chapters:
%    \begin{macrocode}
\include{cdocsch1}
\include{cdocsch2}
%    \end{macrocode}

% Include the two parts unless only chapters should be displayed:
%    \begin{macrocode}
\ifchilddoc\else
\section{part three}
\input{cdocspt3}
\section{part four}
\input{cdocspt4}
\fi
%    \end{macrocode}

% Process as usual until here:
%    \begin{macrocode}
\fi
%    \end{macrocode}

% End of document body:
%    \begin{macrocode}
\end{document}
%    \end{macrocode}
%\iffalse
%</samplemain>
%\fi
%
% %%%%%%%%%%%%%%%%%%%%%%%%%%%%%%%%%%%%%%
% \paragraph{Chapter Include Files.}
%
% The include files are called |cdocsch1.tex| and |cdocsch2.tex|.
%
%\iffalse
%<*samplechap1|samplechap2>
%\fi

% Optional override for |\version| flag:
%    \begin{macrocode}
%%\providecommand{\version}{final}
%    \end{macrocode}

% Include the main document:
%    \begin{macrocode}
\input{childdoc.def}
\childdocof{cdocsamp}
%    \end{macrocode}

%\iffalse
%</samplechap1|samplechap2>
%\fi
%
%\iffalse
%<*samplechap1>
%\fi
% Some text for chapter 1:
%    \begin{macrocode}
\section{one}
some text in chapter one
%    \end{macrocode}

%\iffalse
%</samplechap1>
%\fi
% Some text for chapter 2:
%\iffalse
%<*samplechap2>
%\fi
%    \begin{macrocode}
\section{two}
more text in chapter two
%    \end{macrocode}

%\iffalse
%</samplechap2>
%\fi
%
% %%%%%%%%%%%%%%%%%%%%%%%%%%%%%%%%%%%%%%
% \paragraph{Part Include Files.}
%
% The include files are called |cdocspt3.tex| and |cdocspt4.tex|.
%
%\iffalse
%<*samplepart3|samplepart4>
%\fi

% Optional override for |\version| flag:
%    \begin{macrocode}
%%\providecommand{\version}{final}
%    \end{macrocode}

% Include the main document:
%    \begin{macrocode}
\input{childdoc.def}
\childdocby{cdocsamp}
%    \end{macrocode}

%\iffalse
%</samplepart3|samplepart4>
%\fi
%
%\iffalse
%<*samplepart3>
%\fi
% Some text for part 3:
%    \begin{macrocode}
some text in part three
%    \end{macrocode}

%\iffalse
%</samplepart3>
%\fi
% Some text for part 4:
%\iffalse
%<*samplepart4>
%\fi
%    \begin{macrocode}
more text in part four
%    \end{macrocode}

%\iffalse
%</samplepart4>
%\fi
%
% %%%%%%%%%%%%%%%%%%%%%%%%%%%%%%%%%%%%%%
% \paragraph{Forwarding for a Complete Draft.}
%
% The following forwarding file |cdocsdrf.tex|
% compiles the main document in draft mode:
%\iffalse
%<*sampledraft>
%\fi
%    \begin{macrocode}
\def\version{draft}
\input{childdoc.def}
\childdocforward{cdocsamp}
%    \end{macrocode}

%\iffalse
%</sampledraft>
%\fi
%
% %%%%%%%%%%%%%%%%%%%%%%%%%%%%%%%%%%%%%%
% \paragraph{Forwarding for Final Version of the Chapters.}
%
% The following forwarding files |cdocsfn1.tex| and |cdocsfn2.tex|
% (with identical content)
% compile the final versions of the child documents
% |cdocsch1.tex| and |cdocsch2.tex|, respectively:
%\iffalse
%<*samplefinal>
%\fi
%    \begin{macrocode}
\def\version{final}
\input{childdoc.def}
\childdocforwardprefix[cdocsamp]{cdocsfn}{cdocsch}
%    \end{macrocode}

%\iffalse
%</samplefinal>
%\fi
%
% %%%%%%%%%%%%%%%%%%%%%%%%%%%%%%%%%%%%%%
% \paragraph{Command Line Processing.}
%
% The following three command lines generate the output files
% |cdocscld|, |cdocscl1| and |cdocscl2|
% which should be identical to
% |cdocsdrf|, |cdocsch1| and |cdocsfn2|, respectively:
% \begin{center}
% \begin{tabular}{l}
% |latex -jobname cdocscld \|\\
% |  "\def\version{draft}\input{childdoc.def}\childdocforward{cdocsamp}"|\\
% |latex -jobname cdocscl1 \|\\
% |  "\input{childdoc.def}\childdocforward[cdocsamp]{cdocsch1}"|\\
% |latex -jobname cdocscl2 \|\\
% |  "\def\version{final}\input{childdoc.def}\childdocforward{cdocsch2}"|
% \end{tabular}
% \end{center}
% Note that the trailing backslash on each first line
% merely continues the input to the second line
% (for convenient cut ant paste).
% Furthermore, the command |latex| can be replaced by any
% of its alternative versions such as |pdflatex|.
%
% %%%%%%%%%%%%%%%%%%%%%%%%%%%%%%%%%%%%%%%%%%%%%%%%%%%%%%%%%%%%%%%%%%%%%%%%%%%%%%
% %%%%%%%%%%%%%%%%%%%%%%%%%%%%%%%%%%%%%%%%%%%%%%%%%%%%%%%%%%%%%%%%%%%%%%%%%%%%%%
% \section{Implementation}
%\iffalse
%<*package>
%\fi
%
% This section describes the definitions file |childdoc.def|.

% The definitions cannot be loaded using |\usepackage| or |\RequirePackage|
% which has a mechanism to prevent loading a style file more than once.
% When loading the definitions by means of |\input|
% multiple instances have to be prevented manually:
%\iffalse
%This code needs to be before the `\ProvidesFile' directive
%which is defined at the beginning of this file.
%Therefore it is also placed there and commented out here.
%</package>
%<*discard>
%\fi
%    \begin{macrocode}
\ifdefined\childdocmain\endinput\fi
%    \end{macrocode}
%\iffalse
%</discard>
%<*package>
%\fi
%
% \macro{\ifchilddoc}
% \macro{\ifchilddocmanual}
% The conditional |\ifchilddoc| tells whether a
% child (true) or main (false) document is being compiled.
% The conditional |\ifchilddocmanual| tells whether
% the |\includeonly| mechanism is used (false) or
% the selection of child files must be performed manually (true).
% The definitions initialise to false:
%    \begin{macrocode}
\newif\ifchilddoc
\newif\ifchilddocmanual
%    \end{macrocode}

% \macro{\childdocname}
% \macro{\childdocjob}
% The macro |\childdocname| stores the name of the main document
% to be compiled. The macro |\childdocjob| stores the name of
% the document on which the \LaTeX{} compiler was originally invoked.
% The content of |\jobname| cannot be compared
% to filenames specified in the source due to different catcodes.
% The following code rescans |\jobname|, stores the result
% in |\childdocname| and saves a copy in |\childdocjob|:
%    \begin{macrocode}
\edef\childdocname{\scantokens\expandafter{\jobname\noexpand}}
\let\childdocjob\childdocname
%    \end{macrocode}

% \macro{\childdocdisable}
% The macro |\childdocdisable| prevents the main file
% from being processed more than once.
% At this stage, the main document command |\childdocmain|
% is assumed to be called once again where it should do nothing.
% Any subsequent call to it should prevent
% a secondary processing of the main document
% It overwrites the forwarding commands
% |\childdocof| and |\childdocforward|
% with empty macros to prevent further inclusions of the main document:
%    \begin{macrocode}
\newcommand{\childdocdisable}
{
  \renewcommand{\childdocmain}[1]{\renewcommand{\childdocmain}[1]{\endinput}}
  \renewcommand{\childdocof}[1]{}
  \renewcommand{\childdocby}[2][]{}
  \renewcommand{\childdocforward}[2][]{}
  \renewcommand{\childdocdisable}{}
}
%    \end{macrocode}

% \macro{\childdocmain}
% The macro |\childdocmain| is to be called at the top of the main file
% with nothing or the main filename (without extension) as argument.
% First, it breaks loops.
% If the argument is not empty and does not match |\childdocname|
% (which is set by the first inclusion of |childdoc.def|),
% |\ifchilddoc| is set to true, |\includeonly| is applied to the child file
% and |\jobname| is set to the main file
% (for proper handling of |.aux| files):
%    \begin{macrocode}
\newcommand{\childdocmain}[1]
{
  \childdocdisable\childdocmain{}
  \if?#1?\else
    \begingroup
      \def\childdoctmp{#1}
      \ifx\childdoctmp\childdocname
        \def\childdoctmp{}
      \else
        \def\childdoctmp
        {
          \childdoctrue
          \includeonly{\childdocname}
          \def\childdocjob{#1}
          \def\jobname{#1}
        }
      \fi
      \expandafter
    \endgroup
    \childdoctmp
  \fi
}
%    \end{macrocode}

% \macro{\childdocof}
% The command |\childdocof| redirects
% compilation to the main file |#1|.
%    \begin{macrocode}
\newcommand{\childdocof}[1]
{
  \childdocdisable
  \childdoctrue
  \includeonly{\childdocname}
  \def\jobname{#1}
  \def\childdocjob{#1}
  \input{#1}
}
%    \end{macrocode}

% \macro{\childdocby}
% The command |\childdocby| ....
%    \begin{macrocode}
\newcommand{\childdocby}[2][]
{
  \childdocdisable
  \childdoctrue
  \childdocmanualtrue
  \if?#1?\else
    \def\jobname{#2}
  \fi
  \def\childdocjob{#2}
  \input{#2}
  \endinput
}
%    \end{macrocode}

% \macro{\childdocforward}
% The command |\childdocforward| redirects
% compilation to the main file or
% (if the optional argument is given) a child file.
% Parameters are set as if the main file
% or a child file starting with |\childdocof| was compiled.
% Then compilation is handed over to the main file:
%    \begin{macrocode}
\newcommand{\childdocforward}[2][]
{
  \begingroup
    \if?#1?
      \def\childdoctmp
      {
        \def\childdocname{#2}
        \def\childdocjob{#2}
        \def\jobname{#2}
        \input{#2}
        \endinput
      }
    \else
      \def\childdoctmp
      {
        \childdocdisable
        \def\childdocname{#2}
        \childdoctrue
        \includeonly{#2}
        \def\childdocjob{#1}
        \def\jobname{#1}
        \input{#1}
        \endinput
      }
    \fi
    \expandafter
  \endgroup
  \childdoctmp
}
%    \end{macrocode}

% \macro{\childdocforwardprefix}
% The command |\childdocforwardprefix| redirects
% compilation to the main or a child file by means of a pattern.
% The prefix |#1| in the current filename is replaced by |#2|
% and the suffix of the current filename is kept
% (it is assumed that the filename does not contain the substring `|~~~|'
% which is used as a delimiter).
% Compilation is handed over to the new file by |\childdocforward|:
%    \begin{macrocode}
\newcommand{\childdocforwardprefix}[3][]
{
  \begingroup
    \def\childdocextract #2##1~~~{\def\childdoctmp{\childdocforward[#1]{#3##1}}}
    \expandafter\childdocextract\childdocname~~~
    \expandafter
  \endgroup
  \childdoctmp
}
%    \end{macrocode}

% \macro{\childdoc}
% The deprecated macro |\childdoc| is a legacy version of |\childdocmain|:
%    \begin{macrocode}
\newcommand{\childdoc}{\childdocmain}
%    \end{macrocode}

% \macro{\childdocredirect}
% The deprecated macro |\childdocredirect| is a legacy version
% of |\childdocforward| and |\childdocforwardprefix|:
%    \begin{macrocode}
\newcommand{\childdocredirect}[2][]
{
  \begingroup
    \if?#1?
      \def\childdoctmp{\childdocforward{#2}}
    \else
      \def\childdoctmp{\childdocforwardprefix{#1}{#2}}
    \fi
    \expandafter
  \endgroup
  \childdoctmp
}
%    \end{macrocode}

%\iffalse
%</package>
%\fi
%
\endinput
|\\
|\childdocby{|\textit{main}|}|\\
\end{tabular}
\end{center}
%
The directive |\childdocby| is similar to |\childdocof|
described in \secref{sec:include},
but the subsequent selection of content must be done manually.
To that end, both |\ifchilddoc| and |\ifchilddocmanual|
will be true upon processing of a part,
and the name of the part is stored in |\childdocname|.
Note that |\jobname| will be set to the filename of the current part
so that each part receives an individual |.aux| file
that does not interfere with the |.aux| file(s) of the main document.
This behaviour can be altered by the alternative form
|\childdocby[*]{|\textit{main}|}| (with a non-empty optional argument)
which uses the |.aux| file of the main document
by setting |\jobname| to \textit{main}.

%%%%%%%%%%%%%%%%%%%%%%%%%%%%%%%%%%%%%%%%%%%%%%%%%%%%%%%%%%%%%%%%%%%%%%%%%%%%%%%%
\subsection{Driver Development}
\label{sec:driver}

The \textsf{childdoc} mechanism can also be use for the development
of definition files such as \LaTeX{} styles or classes.
This case differs from the above setup with multiple parts
included by |\include| in that no |\includeonly| should be invoked.
This can be achieved by starting the include file
(before |\ProvidesPackage|) with:
%
\begin{center}
\begin{tabular}{l}
|% \iffalse
%
% childdoc.dtx Copyright (C) 2017-2018 Niklas Beisert
%
% This work may be distributed and/or modified under the
% conditions of the LaTeX Project Public License, either version 1.3
% of this license or (at your option) any later version.
% The latest version of this license is in
%   http://www.latex-project.org/lppl.txt
% and version 1.3 or later is part of all distributions of LaTeX
% version 2005/12/01 or later.
%
% This work has the LPPL maintenance status `maintained'.
%
% The Current Maintainer of this work is Niklas Beisert.
%
% This work consists of the files childdoc.dtx and childdoc.ins
% and the derived files childdoc.def and cdocsamp.tex with
% cdocsch1.tex, cdocsch2.tex, cdocsdrf.tex, cdocsfn1.tex, cdocsfn2.tex.
%
%<package>\ifdefined\childdocmain\endinput\fi
%<package>\ProvidesFile{childdoc.def}[2018/12/30 v2.0 child document driver]
%<samplemain>\ProvidesFile{cdocsamp.tex}[2018/12/30 v2.0 sample for childdoc]
%<*driver>
%\ProvidesFile{childdoc.drv}[2018/12/30 v2.0 childdoc reference manual file]
\PassOptionsToClass{10pt,a4paper}{article}
\documentclass{ltxdoc}

\usepackage[margin=35mm]{geometry}
\usepackage{hyperref}
\usepackage{hyperxmp}
\usepackage[usenames]{color}

\hypersetup{colorlinks=true}
\hypersetup{pdfstartview=FitH}
\hypersetup{pdfpagemode=UseNone}
\hypersetup{pdfsource={}}
\hypersetup{pdflang={en-UK}}
\hypersetup{pdfcopyright={Copyright 2017-2018 Niklas Beisert.
  This work may be distributed and/or modified under the
  conditions of the LaTeX Project Public License, either version 1.3
  of this license or (at your option) any later version.}}
\hypersetup{pdflicenseurl={http://www.latex-project.org/lppl.txt}}
\hypersetup{pdfcontactaddress={ETH Zurich, ITP, HIT K,
  Wolfgang-Pauli-Strasse 27}}
\hypersetup{pdfcontactpostcode={8093}}
\hypersetup{pdfcontactcity={Zurich}}
\hypersetup{pdfcontactcountry={Switzerland}}
\hypersetup{pdfcontactemail={nbeisert@itp.phys.ethz.ch}}
\hypersetup{pdfcontacturl={http://people.phys.ethz.ch/\xmptilde nbeisert/}}

\newcommand{\secref}[1]{\hyperref[#1]{section \ref*{#1}}}

\parskip1ex
\parindent0pt
\let\olditemize\itemize
\def\itemize{\olditemize\parskip0pt}

\begin{document}

\title{The \textsf{childdoc} Package}
\hypersetup{pdftitle={The childdoc Package}}
\author{Niklas Beisert\\[2ex]
  Institut f\"ur Theoretische Physik\\
  Eidgen\"ossische Technische Hochschule Z\"urich\\
  Wolfgang-Pauli-Strasse 27, 8093 Z\"urich, Switzerland\\[1ex]
  \href{mailto:nbeisert@itp.phys.ethz.ch}
  {\texttt{nbeisert@itp.phys.ethz.ch}}}
\hypersetup{pdfauthor={Niklas Beisert}}
\hypersetup{pdfsubject={Manual for the LaTeX2e Package childdoc}}
\date{30 December 2018, \textsf{v2.0}}
\maketitle

\begin{abstract}\noindent
\textsf{childdoc} is a \LaTeXe{} package
that enables the direct compilation
of document sections included by |\include|
to individual files.
\end{abstract}

\begingroup
\parskip0ex
\tableofcontents
\endgroup

%%%%%%%%%%%%%%%%%%%%%%%%%%%%%%%%%%%%%%%%%%%%%%%%%%%%%%%%%%%%%%%%%%%%%%%%%%%%%%%%
%%%%%%%%%%%%%%%%%%%%%%%%%%%%%%%%%%%%%%%%%%%%%%%%%%%%%%%%%%%%%%%%%%%%%%%%%%%%%%%%
\section{Introduction}

\LaTeX{} provides a mechanism to structure a large document (such as a book)
into a main file and several child files (containing the chapters)
using the |\include| command.
This mechanism is beneficial for documents
which span hundreds of pages in order to
make the source file(s) more manageable.
Moreover, compilation can be restricted to
selected child files by means of the |\includeonly| command.
The latter feature can be used to reduce the compilation time while editing
(this was significantly more useful in the earlier days of \LaTeX{})
or to generate a smaller document which is easier to navigate.
Another application of |\includeonly| is to generate
documents consisting of selected parts of the complete document.

However, there are a few drawbacks of the plain |\include| mechanism:
\begin{itemize}
\item
The child files cannot be compiled on their own,
they can only be compiled via the main file.
A naive editing environment
(such as a text editor with an option
to have the current file processed by \LaTeX)
may require one to switch to the main file before compiling;
attempting to compile the child file produces errors.
\item
The main file must be modified (each time)
to adjust the |\includeonly| command
to the present needs. This easily leaves the main file in a messy state.
\item
The generated document will always carry the filename
of the main document. This is inconvenient if
several child files are to be compiled and
to be kept for distribution.
\end{itemize}

The present package provides a simple interface
to make child files individually compilable by \LaTeX{}.
Compiling a child file then has the same effect as compiling
the main file with an |\includeonly| command
to select the appropriate child.
Moreover the generated document will carry the name of the child
rather than the main file.
This resolves all three above issues.

This feature is meant to make the editing of books,
thesis documents and lecture notes somewhat more convenient.
However, the package can also be used efficiently for
composing a series of documents (such as exercise sheets)
which are typically distributed individually.
It then assists the author in generating the individual documents
(potentially in different versions)
as well as a document containing the collected series.
Another application is in developing style files
or other kinds of included material
where compilation of the style file could redirect
to a sample or test file.

%%%%%%%%%%%%%%%%%%%%%%%%%%%%%%%%%%%%%%%%%%%%%%%%%%%%%%%%%%%%%%%%%%%%%%%%%%%%%%%%
%%%%%%%%%%%%%%%%%%%%%%%%%%%%%%%%%%%%%%%%%%%%%%%%%%%%%%%%%%%%%%%%%%%%%%%%%%%%%%%%
\section{Usage}

First of all, the package \textsf{childdoc} is \emph{not} a standard
\LaTeXe{} |.sty| style file! Therefore it needs to be invoked in
a non-standard way.

%%%%%%%%%%%%%%%%%%%%%%%%%%%%%%%%%%%%%%%%%%%%%%%%%%%%%%%%%%%%%%%%%%%%%%%%%%%%%%%%
\subsection{Included Files}
\label{sec:include}

%%%%%%%%%%%%%%%%%%%%%%%%%%%%%%%%%%%%%%%%
\DescribeMacro{\childdocmain}
To use the package, add the commands
\begin{center}
\begin{tabular}{l}
|\input{childdoc.def}|\\
|\childdocmain{}|\\
\end{tabular}
\end{center}
at the very top of the main \LaTeX{} file,
in particular \emph{before} the |\documentclass| statement!
The argument of |\childdocmain| should be left empty
(but it must be present).

%%%%%%%%%%%%%%%%%%%%%%%%%%%%%%%%%%%%%%%%
\DescribeMacro{\childdocof}
Furthermore, add the commands
\begin{center}
\begin{tabular}{l}
|\input{childdoc.def}|\\
|\childdocof{|\textit{main}|}|\\
\end{tabular}
\end{center}
at the top of every child file \textit{child}
which is included by |\include{|\textit{child}|}|
from within the main file
(or at least for those files to be compiled individually).
The argument \textit{main} must be the filename of the main file.

There are a couple of
considerations in setting up the main and child documents:

%%%%%%%%%%%%%%%%%%%%%%%%%%%%%%%%%%%%%%%%
\paragraph{Restrictions.}

Please note the following restrictions:
\begin{itemize}
\item
|\childdocmain| must be called with one argument \textit{main}
to ensure compatibility with earlier version of the package.
It must either be empty (|\childdocmain{}|)
or precisely match the filename of the main file in which it is specified.
See \secref{sec:detection} for further information.
\item
The filename \textit{main} must be specified without the |.tex| extension.
\item
The filename \textit{main} is case sensitive
(even in case-insensitive file systems)
due to internal string comparison.
\item
The argument \textit{main} should be fully expanded, it cannot be a macro.
\item
Subdirectories and special characters should be avoided in filenames.
\item
The command |\childdocmain{|\textit{main}|}| must be followed by a whitespace.
It should not be followed immediately by another command
or by a comment mark `|%|'.
This is because the \TeX{} parser reads the token immediately following
the argument of |\childdocmain| and puts it
at the beginning of every child section;
however, a white\-space is ignored.
\end{itemize}

%%%%%%%%%%%%%%%%%%%%%%%%%%%%%%%%%%%%%%%%
\paragraph{Content of Main File.}

It is advisable to place all content in the child files included by |\include|.
Any output contained in the main file will appear in all child documents
unless suppressed manually;
it cannot be suppressed automatically by the |\includeonly| directive
and thus should normally be avoided.
A method to include some content in the main file
by means of conditional processing is described in \secref{sec:conditional}.

%%%%%%%%%%%%%%%%%%%%%%%%%%%%%%%%%%%%%%%%
\paragraph{Page Numbering.}

When only a part of the document is compiled,
the appropriate numbering of pages
(as well as other status parameters)
is determined from the |.aux| files.
The latter contain information from previous passes.
However this information needs to propagate through
all intermediate child documents.
Therefore the page numbering in child documents may well
be inconsistent until the complete document is compiled at least once.

A useful (if unconventional) way to always ensure a consistent
page numbering is to restart the numbering in each child document
and denote the pages by `\textit{child}|.|\textit{page}'
where \textit{child} represents the chapter/section number of the child file.
This can be achieved by the command
|\numberwithin{page}{|\textit{child}|}|
of the \textsf{amsmath} package
where \textit{child} can be |chapter| or |section|
depending on the chosen structuring.
Alternatively, one can modify the macro |\thepage| appropriately
and reset the counter |page| at the start of each child file.

%%%%%%%%%%%%%%%%%%%%%%%%%%%%%%%%%%%%%%%%%%%%%%%%%%%%%%%%%%%%%%%%%%%%%%%%%%%%%%%%
\subsection{Conditional Processing}
\label{sec:conditional}

The package provides a mechanism to compile different versions
of a document. To customise the versions further some conditional processing
can come in handy to distinguish which version is being compiled.
The package provides two macros to describe the compilation context:

%%%%%%%%%%%%%%%%%%%%%%%%%%%%%%%%%%%%%%%%
\DescribeMacro{\ifchilddoc}
The conditional |\ifchilddoc| distinguishes between the compilation of
child documents and the main document:
%
\begin{center}
|\ifchilddoc |\textit{child-code}| |[|\||else |\textit{main-code}]| \||fi|
\end{center}

%%%%%%%%%%%%%%%%%%%%%%%%%%%%%%%%%%%%%%%%
\DescribeMacro{\childdocname}
\DescribeMacro{\childdocjob}
The macro |\childdocname| contains the filename (without extension)
of the main or child file being processed.
Note that |\childdocjob| will always contain the name of the main file.

%%%%%%%%%%%%%%%%%%%%%%%%%%%%%%%%%%%%%%%%
\paragraph{Title Page.}

Conditional processing can be used to include a title or banner page
in the main document when proper precautions are taken.
Importantly, the code in the main file should ensure that the page counter
(as well as other status parameters which are stored in the |.aux| files)
takes the same value after the conditional processing.
Otherwise the page numbers may take divergent values
depending on which part is compiled.

For example, a title page could be declared by:
%
\begin{center}
\begin{tabular}{l}
|\ifchilddoc\||else|\\
|\addtocounter{page}{-1}|\\
\textit{code for title page}\\
|\newpage|\\
|\||fi|
\end{tabular}
\end{center}
%
A banner page for the child documents can be generated by:
%
\begin{center}
\begin{tabular}{l}
|\ifchilddoc|\\
|\addtocounter{page}{-1}|\\
\textit{code for banner page}\\
|\newpage|\\
|\||fi|
\end{tabular}
\end{center}
%
Here one could write a message such as:
\begin{center}
|This is the part \childdocname{} of \childdocjob{}.|
\end{center}

%%%%%%%%%%%%%%%%%%%%%%%%%%%%%%%%%%%%%%%%%%%%%%%%%%%%%%%%%%%%%%%%%%%%%%%%%%%%%%%%
\subsection{Flags}
\label{sec:flags}

The package makes it easy to generate different versions
of the main or child documents.
To this end compilation flags can be defined
and assigned different default values.
They will be particularly useful in conjunction
with the forwarding mechanism described in \secref{sec:forward}.

For example, it may be useful to have a flag |\version|
which can be set to |draft| or |final|.
The document source will contain some conditional code
depending on the value of |\version|.
Suppose further, the flag should default to |final| for the main file
and to |draft| for child files
which is a natural assignment for editing the document.
This is achieved by placing the following code
in the preamble of the main document
(below the |\childdocmain| directive):
%
\begin{center}
\begin{tabular}{l}
|\ifchilddoc|\\
|\providecommand{\version}{draft}|\\
|\||else|\\
|\providecommand{\version}{final}|\\
|\||fi|
\end{tabular}
\end{center}
%
The definition by |\providecommand| makes sure
that previous definitions are not overwritten.
Further statements |\providecommand{\version}{...}|
can thus be added before the above code to override it.

For the main file, one might add a line
(between |\childdocmain| and the above block)
%
\begin{center}
|%\ifchilddoc\||else\providecommand{\version}{draft}\||fi|
\end{center}
%
which can be uncommented to produce a draft version.
Likewise one can add a line to the very top of a child file
(above the |\childdocof{|\textit{main}|}| directive)
%
\begin{center}
|%\providecommand{\version}{final}|
\end{center}
%
which can be uncommented to produce the final version of this child document.

%%%%%%%%%%%%%%%%%%%%%%%%%%%%%%%%%%%%%%%%%%%%%%%%%%%%%%%%%%%%%%%%%%%%%%%%%%%%%%%%
\subsection{Forwarding}
\label{sec:forward}

Different versions of the main or child documents
using compilation flags as described in \secref{sec:flags}
can be (permanently) stored in different files
for convenient compilation, viewing and distribution.
To this end, the package defines a command
to pass on compilation to a different file:

%%%%%%%%%%%%%%%%%%%%%%%%%%%%%%%%%%%%%%%%
\DescribeMacro{\childdocforward}
The command |\childdocforward| redirects processing to
another source file:
%
\begin{center}
\begin{tabular}{l}
|\input{childdoc.def}|\\
|\childdocforward[|\textit{main}|]{|\textit{dest}|}|\\
\end{tabular}
\end{center}
%
The argument \textit{dest} is the destination file
(without extension).
It should be the main file or one of the child files.
Note that further \textsf{childdoc} directives
such as |\childdocof| and |\childdocforward|
in the indicated file will be processed in this form.
The optional argument \textit{main}
passes on directly to the main file \textit{main}
while pretending to compile the child \textit{dest}.
This form behaves as if \textit{dest}
issues |\childdocof{|\textit{main}|}| right away,
and no further \textsf{childdoc} directives will be processed.

%%%%%%%%%%%%%%%%%%%%%%%%%%%%%%%%%%%%%%%%
\DescribeMacro{\...prefix}
In the alternative form |\childdocforwardprefix|,
%
\begin{center}
\begin{tabular}{l}
|\input{childdoc.def}|\\
|\childdocforwardprefix[|\textit{main}|]{|\textit{prefix}|}{|\textit{dest}|}|
\end{tabular}
\end{center}
%
the destination file is determined by a pattern
depending on the current file:
To make this work, the current file must be called
`{\textit{prefix}\hspace{0.2em}\textit{suffix}}'
with \textit{prefix} matching precisely the argument.
Processing is then passed on to the file
`{\textit{dest}\hspace{0.2em}\textit{suffix}}'.
Surely, the same effect is achieved by
directly specifying the
argument `{\textit{dest}\hspace{0.2em}\textit{suffix}}'
in the first form.
However, that requires to set up a different file
for each child. With the alternative form of the command
all these files can have exactly the same content
which simplifies setting them up and maintaining them.

For example, the following file |draft.tex|
with a compilation flag |\version| as described in \secref{sec:flags}
compiles the main document as a draft:
%
\begin{center}
\begin{tabular}{l}
|\def\version{draft}|\\
|\input{childdoc.def}|\\
|\childdocforward{|\textit{main}|}|
\end{tabular}
\end{center}
%
Likewise, the following files |final|\textit{nn}|.tex|
compile the final version of the child document
|child|\textit{nn}|.tex|:
%
\begin{center}
\begin{tabular}{l}
|\def\version{final}|\\
|\input{childdoc.def}|\\
|\childdocforwardprefix{final}{child}|
\end{tabular}
\end{center}
%

Note that when several versions of a main file and/or of each child file
are to be generated, it may be convenient to set up a |Makefile| or
shell script to automatise the process.

%%%%%%%%%%%%%%%%%%%%%%%%%%%%%%%%%%%%%%%%%%%%%%%%%%%%%%%%%%%%%%%%%%%%%%%%%%%%%%%%
\subsection{Command Line Processing}
\label{sec:commandline}

The effect of redirection files can also be achieved by invoking
the \LaTeX{} compiler with a more elaborate command line.
Most conveniently this should be done as part
of a shell script or a |Makefile|.

When using \textsf{childdoc} in the main file, the following
command lines effectively perform a redirection
(note that depending on the shell being used,
backslashes may have to be doubled: `|\|' $\to$ `|\\|'):
%
\begin{center}
|... -jobname "|\textit{target}|" |\\|"|[\textit{flags}]%
|\input{childdoc.def}\childdocforward[|\textit{main}|]{|\textit{dest}|}"|
\end{center}
%
Here \textit{target} is the name of the output file,
\textit{main} is the name of the main file
and \textit{dest} is the name of the main or child file to be processed
(all filenames without extensions).
The optional argument \textit{main} can be omitted
if \textit{main} matches \textit{dest}.
Optionally, compilation \textit{flags} can be defined via |\def| commands.
This command line makes the \TeX{} engine believe
it is compiling the file \textit{target}
whose content is specified as the latter parameter.
The provided code then forwards the processing to
\textit{main} or \textit{dest} as described in \secref{sec:forward}.

%%%%%%%%%%%%%%%%%%%%%%%%%%%%%%%%%%%%%%%%%%%%%%%%%%%%%%%%%%%%%%%%%%%%%%%%%%%%%%%%
\subsection{Include by Input}
\label{sec:input}

Including child documents by |\include| has some restrictions by design.
Most notably, the content of a child document always occupies
its own set of pages; pages cannot be shared between child documents.
Usually, this behaviour makes perfect sense
because each child document contain an essential part of the document.
However, in some situations it may be desirable to compose
a document from a collection of parts
without having mandatory page breaks between then.
For this case, the package
provides a mechanism to include parts
by |\input| which can also be processed individually.
However, by construction this mechanism
requires manual handling of the content to be output.

%%%%%%%%%%%%%%%%%%%%%%%%%%%%%%%%%%%%%%%%
\DescribeMacro{\ifchilddocmanual}
The main file should be prepared as usual, see \secref{sec:include}.
However, the document body must make a distinction
between processing of an individual part and of the main document, e.g.:
%
\begin{center}
\begin{tabular}{l}
|\ifchilddocmanual|\\
|\input{\childdocname}|\\
|\||else|\\
\textit{document body with }|\input{|\textit{part}|}|\\
|\||fi|
\end{tabular}
\end{center}
%
The conditional |\ifchilddocmanual| is true whenever
a part to be included by |\input| is being compiled,
and the name of the part is stored in |\childdocname|.

%%%%%%%%%%%%%%%%%%%%%%%%%%%%%%%%%%%%%%%%
\DescribeMacro{\childdocby}
Each part to be included by |\input| should start with:
%
\begin{center}
\begin{tabular}{l}
|\input{childdoc.def}|\\
|\childdocby{|\textit{main}|}|\\
\end{tabular}
\end{center}
%
The directive |\childdocby| is similar to |\childdocof|
described in \secref{sec:include},
but the subsequent selection of content must be done manually.
To that end, both |\ifchilddoc| and |\ifchilddocmanual|
will be true upon processing of a part,
and the name of the part is stored in |\childdocname|.
Note that |\jobname| will be set to the filename of the current part
so that each part receives an individual |.aux| file
that does not interfere with the |.aux| file(s) of the main document.
This behaviour can be altered by the alternative form
|\childdocby[*]{|\textit{main}|}| (with a non-empty optional argument)
which uses the |.aux| file of the main document
by setting |\jobname| to \textit{main}.

%%%%%%%%%%%%%%%%%%%%%%%%%%%%%%%%%%%%%%%%%%%%%%%%%%%%%%%%%%%%%%%%%%%%%%%%%%%%%%%%
\subsection{Driver Development}
\label{sec:driver}

The \textsf{childdoc} mechanism can also be use for the development
of definition files such as \LaTeX{} styles or classes.
This case differs from the above setup with multiple parts
included by |\include| in that no |\includeonly| should be invoked.
This can be achieved by starting the include file
(before |\ProvidesPackage|) with:
%
\begin{center}
\begin{tabular}{l}
|\input{childdoc.def}|\\
|\childdocforward{|\textit{main}|}|\\
\end{tabular}
\end{center}
%
or alternatively with:
%
\begin{center}
\begin{tabular}{l}
|\input{childdoc.def}|\\
|\childdocby{|\textit{main}|}|\\
\end{tabular}
\end{center}
%
Both forms have slightly different effects as described above.
The main file is prepared as usual, see \secref{sec:include}.

%%%%%%%%%%%%%%%%%%%%%%%%%%%%%%%%%%%%%%%%%%%%%%%%%%%%%%%%%%%%%%%%%%%%%%%%%%%%%%%%
\subsection{Legacy Detection}
\label{sec:detection}

The directive |\childdocmain| in the main file can detect
whether the complete document or merely a child is to be compiled
even without using the directive |\childdocof|.
This method is deprecated because it is less robust
and there is no compelling reason to use it;
it is merely provided for backward compatibility
and it may be removed in future versions.

If the detection mechanism is to be used,
it is mandatory to correctly specify
the filename of the main file as the argument of |\childdocmain|:
%
\begin{center}
\begin{tabular}{l}
|\input{childdoc.def}|\\
|\childdocmain{|\textit{main}|}|\\
\end{tabular}
\end{center}
%
If |\jobname| does not match the argument \textit{main} of |\childdocmain|,
it is assumed that |\jobname| points to the child file to be compiled.
When using |\childdocmain| with the main file specified as argument,
it suffices to start a child file
with just |\input{|\textit{main}|}|
without loading of the package and using |\childdocof|.
If instead all processing is done
with the appropriate \textsf{childdoc} directives,
the argument of \textit{main} of |\childdocmain| can be empty.

An alternative version of the command line processing described
in \secref{sec:commandline} using the detection mechanism reads:
%
\begin{center}
|... -jobname "|\textit{target}|" "|[\textit{flags}]%
[|\def\jobname{|\textit{dest}|}|]|\input{|\textit{main}|}"|
\end{center}

%%%%%%%%%%%%%%%%%%%%%%%%%%%%%%%%%%%%%%%%%%%%%%%%%%%%%%%%%%%%%%%%%%%%%%%%%%%%%%%%
\subsection{Manual Code}
\label{sec:manual}

In case one cannot be certain whether the definitions file |childdoc.def|
is installed on the target \TeX{} distribution
and one prefers not to ship it,
it is conceivable to paste a few relevant commands into the sources.

To that end, drop all statements |\input{childdoc.def}|
and perform the replacements as outlined below.
Instead of |\childdocmain{|\textit{main}|}| add the following code
to the top of the main file:
%
\begin{center}
\begin{tabular}{l}
|\||ifdefined\childdocname\endinput\||fi\newif\ifchilddoc|\\
|\edef\childdocname{\scantokens\expandafter{\jobname\noexpand}}|\\
|\def\childdocmain{|\textit{main}|}\||ifx\childdocmain\childdocname\||else|\\
|\childdoctrue\includeonly{\childdocname}\let\jobname\childdocmain\||fi|\\
\end{tabular}
\end{center}
%
Instead of |\childdocof{|\textit{main}|}| just include the main file
at the top of each child file:
%
\begin{center}
|\input{|\textit{main}|}|
\end{center}
%
A simple redirection |\childdocforward{|\textit{dest}|}| is achieved by:
%
\begin{center}
|\def\jobname{|\textit{dest}|}\input{\jobname}|
\end{center}
%
The redirection with prefix
|\childdocforwardprefix[|\textit{prefix}|]{|\textit{dest}|}|
is accomplished by:
%
\begin{center}
\begin{tabular}{l}
|{\edef\jobname{\scantokens\expandafter{\jobname\noexpand}}|\\
|\def\redirectjob |\textit{prefix}|#1~~~{\gdef\jobname{|\textit{dest}|#1}}|\\
|\expandafter\redirectjob\jobname~~~}\input{\jobname}|
\end{tabular}
\end{center}

In an alternative approach,
child documents can be compiled by a specific command line
without additional code or specific definitions:
%
\begin{center}
|... -jobname "|\textit{target}|" "|[\textit{flags}]%
|\includeonly{|\textit{dest}|}\input{|\textit{main}|}"|
\end{center}
%

%%%%%%%%%%%%%%%%%%%%%%%%%%%%%%%%%%%%%%%%%%%%%%%%%%%%%%%%%%%%%%%%%%%%%%%%%%%%%%%%
%%%%%%%%%%%%%%%%%%%%%%%%%%%%%%%%%%%%%%%%%%%%%%%%%%%%%%%%%%%%%%%%%%%%%%%%%%%%%%%%
\section{Information}

%%%%%%%%%%%%%%%%%%%%%%%%%%%%%%%%%%%%%%%%%%%%%%%%%%%%%%%%%%%%%%%%%%%%%%%%%%%%%%%%
\subsection{Copyright}

Copyright \copyright{} 2017--2018 Niklas Beisert

This work may be distributed and/or modified under the
conditions of the \LaTeX{} Project Public License, either version 1.3
of this license or (at your option) any later version.
The latest version of this license is in
  \url{http://www.latex-project.org/lppl.txt}
and version 1.3 or later is part of all distributions of \LaTeX{}
version 2005/12/01 or later.

This work has the LPPL maintenance status `maintained'.

The Current Maintainer of this work is Niklas Beisert.

This work consists of the files |README.txt|, |childdoc.ins| and |childdoc.dtx|
as well as the derived files |childdoc.def|, |cdocsamp.tex|
with |cdocsch1.tex|, |cdocsch2.tex|, |cdocspt3.tex|, |cdocspt4.tex|,
|cdocsdrf.tex|, |cdocsfn1.tex|, |cdocsfn2.tex|
as well as |childdoc.pdf|.

%%%%%%%%%%%%%%%%%%%%%%%%%%%%%%%%%%%%%%%%%%%%%%%%%%%%%%%%%%%%%%%%%%%%%%%%%%%%%%%%
\subsection{Files and Installation}

The package consists of the files:
%
\begin{center}
\begin{tabular}{ll}
    |README.txt|   & readme file \\
    |childdoc.ins| & installation file \\
    |childdoc.dtx| & source file \\
    |childdoc.def| & definition file \\
    |cdocsamp.tex| & sample main file \\
    |cdocsch1.tex| & sample include file \\
    |cdocsch2.tex| & sample include file \\
    |cdocspt3.tex| & sample part file \\
    |cdocspt4.tex| & sample part file \\
    |cdocsdrf.tex| & sample redirection file \\
    |cdocsfn1.tex| & sample redirection file \\
    |cdocsfn2.tex| & sample redirection file \\
    |childdoc.pdf| & manual
\end{tabular}
\end{center}
%
The distribution consists of the files
|README.txt|, |childdoc.ins| and |childdoc.dtx|.
%
\begin{itemize}
\item
Run (pdf)\LaTeX{} on |childdoc.dtx|
to compile the manual |childdoc.pdf| (this file).
\item
Run \LaTeX{} on |childdoc.ins| to create the definitions file |childdoc.def|
and the sample |cdocsamp.tex| with include files
|cdocsch1.tex|, |cdocsch2.tex|, |cdocspt3.tex|, |cdocspt4.tex|,
|cdocsdrf.tex|, |cdocsfn1.tex|, |cdocsfn2.tex|.
Then copy the file |childdoc.def| to an appropriate directory of your \LaTeX{}
distribution, e.g.\ \textit{texmf-root}|/tex/latex/childdoc|.
\end{itemize}

%%%%%%%%%%%%%%%%%%%%%%%%%%%%%%%%%%%%%%%%%%%%%%%%%%%%%%%%%%%%%%%%%%%%%%%%%%%%%%%%
\subsection{Related CTAN Packages}

There are several other packages which offer a similar functionality:
%
\begin{itemize}
\item
The packages
\href{http://ctan.org/pkg/docmute}{\textsf{docmute}},
\href{http://ctan.org/pkg/includex}{\textsf{includex}} and
\href{http://ctan.org/pkg/standalone}{\textsf{standalone}}
provide commands to include only the document body of
a child file thus allowing both files to be compiled individually.
\item
The packages \href{http://ctan.org/pkg/subdocs}{\textsf{subdocs}}
and \href{http://ctan.org/pkg/subfiles}{\textsf{subfiles}}
provide structures in which the main and child documents can be
encapsulated and allowing them to be compiled individually.
The inclusion mechanism is different from the conventional |\include|.
\item
The package \href{http://ctan.org/pkg/combine}{\textsf{combine}}
is an elaborate solution to combine several documents into one.
\end{itemize}
%
See also the CTAN topic \href{http://ctan.org/topic/subdocs}{\textsf{subdocs}}
for further related packages.
The present package differs from the above solutions in that
a document structure constructed with the conventional |\include| mechanism
just needs two extra commands at the top of every file
such that all constituent files can be compiled individually.

%%%%%%%%%%%%%%%%%%%%%%%%%%%%%%%%%%%%%%%%%%%%%%%%%%%%%%%%%%%%%%%%%%%%%%%%%%%%%%%%
%\subsection{Feature Suggestions}
%
%The following is a list of features which may be useful for future
%versions of this package:
%%
%\begin{itemize}
%\item
%\ldots
%\end{itemize}

%%%%%%%%%%%%%%%%%%%%%%%%%%%%%%%%%%%%%%%%%%%%%%%%%%%%%%%%%%%%%%%%%%%%%%%%%%%%%%%%
\subsection{Revision History}

%%%%%%%%%%%%%%%%%%%%%%%%%%%%%%%%%%%%%%%%
\paragraph{v2.0:} 2018/12/30

\begin{itemize}
\item
immediate forward processing
\item
added |\childdocby| mechanism
\item
manual restructured
\end{itemize}

%%%%%%%%%%%%%%%%%%%%%%%%%%%%%%%%%%%%%%%%
\paragraph{v1.6:} 2018/01/17

\begin{itemize}
\item
application for development of include files
\item
corrections to manual
\end{itemize}

%%%%%%%%%%%%%%%%%%%%%%%%%%%%%%%%%%%%%%%%
\paragraph{v1.5:} 2017/05/21

\begin{itemize}
\item
more complete structuring introduced
\item
|\childdocof| introduced
\item
|\childdoc| renamed to |\childdocmain|
\item
|\childredirect| renamed to |\childdocforward| and |\childdocforwardprefix|
and functionality expanded
\end{itemize}

%%%%%%%%%%%%%%%%%%%%%%%%%%%%%%%%%%%%%%%%
\paragraph{v1.0:} 2017/04/27

\begin{itemize}
\item
manual and install package
\item
first version published on CTAN
\end{itemize}

%%%%%%%%%%%%%%%%%%%%%%%%%%%%%%%%%%%%%%%%
\paragraph{v0.6:} 2017/04/26

\begin{itemize}
\item
redirection mechanism added
\end{itemize}

%%%%%%%%%%%%%%%%%%%%%%%%%%%%%%%%%%%%%%%%
\paragraph{v0.5:} 2017/04/26

\begin{itemize}
\item
functionality in definition file
\end{itemize}


%%%%%%%%%%%%%%%%%%%%%%%%%%%%%%%%%%%%%%%%%%%%%%%%%%%%%%%%%%%%%%%%%%%%%%%%%%%%%%%%
%%%%%%%%%%%%%%%%%%%%%%%%%%%%%%%%%%%%%%%%%%%%%%%%%%%%%%%%%%%%%%%%%%%%%%%%%%%%%%%%
%%%%%%%%%%%%%%%%%%%%%%%%%%%%%%%%%%%%%%%%%%%%%%%%%%%%%%%%%%%%%%%%%%%%%%%%%%%%%%%%
\appendix

\settowidth\MacroIndent{\rmfamily\scriptsize 000\ }

 \DocInput{childdoc.dtx}

\end{document}
%</driver>
% \fi
%
% %%%%%%%%%%%%%%%%%%%%%%%%%%%%%%%%%%%%%%%%%%%%%%%%%%%%%%%%%%%%%%%%%%%%%%%%%%%%%%
% %%%%%%%%%%%%%%%%%%%%%%%%%%%%%%%%%%%%%%%%%%%%%%%%%%%%%%%%%%%%%%%%%%%%%%%%%%%%%%
% \section{Sample}
%\iffalse
%<*samplemain>
%\fi
%
% The following presents a sample document
% with two chapters, two parts, a title page,
% a compile flag as well as three forwarding files to set the flag.
% It consists of eight |.tex| files:
% \begin{center}
% \begin{tabular}{ll}
% |cdocsamp.tex|&main file\\
% |cdocsch1.tex|&include file for chapter 1\\
% |cdocsch2.tex|&include file for chapter 2\\
% |cdocspt3.tex|&include file for part 3\\
% |cdocspt4.tex|&include file for part 4\\
% |cdocsdrf.tex|&forwarding file for main file in draft mode\\
% |cdocsfi1.tex|&forwarding file for final version of chapter 1\\
% |cdocsfi2.tex|&forwarding file for final version of chapter 2\\
% \end{tabular}
% \end{center}
% Each of the eight files can be compiled directly by the \LaTeX{} compiler.
%
% %%%%%%%%%%%%%%%%%%%%%%%%%%%%%%%%%%%%%%
% \paragraph{Main File.}
%
% The main file is called |cdocsamp.tex|.
%
% Load the \textsf{childdoc} definitions and
% declare the filename for the main document:
%    \begin{macrocode}
\input{childdoc.def}
\childdocmain{}
%    \end{macrocode}

% Optional override for |\version| flag:
%    \begin{macrocode}
%%\ifchilddoc\else\providecommand{\version}{draft}\fi
%    \end{macrocode}

% Define the default values for the |\version| flag
% (|final| for the main file and |draft| for childs):
%    \begin{macrocode}
\ifchilddoc
\providecommand{\version}{draft}
\else
\providecommand{\version}{final}
\fi
%    \end{macrocode}

% Load the standard document class:
%    \begin{macrocode}
\documentclass[12pt]{article}
%    \end{macrocode}

% Start the document body:
%    \begin{macrocode}
\begin{document}
%    \end{macrocode}

% Declare a title page.
% Print title, part of document being processed and version flag:
%    \begin{macrocode}
\addtocounter{page}{-1}
\begin{center}
{\LARGE\bfseries{}childdoc example\par}
\vspace{1cm}
\ifchilddoc
\ifchilddocmanual part\else chapter\fi:
`\childdocname' of `\childdocjob'\par
\else
main document: `\childdocjob'\par
\fi
version: \version\par
\end{center}
\newpage
%    \end{macrocode}

% Manually include selected file,
% otherwise process as usual:
%    \begin{macrocode}
\ifchilddocmanual
\section*{part `\childdocname'}
\input{\childdocname}
\else
%    \end{macrocode}

% Include the two chapters:
%    \begin{macrocode}
\include{cdocsch1}
\include{cdocsch2}
%    \end{macrocode}

% Include the two parts unless only chapters should be displayed:
%    \begin{macrocode}
\ifchilddoc\else
\section{part three}
\input{cdocspt3}
\section{part four}
\input{cdocspt4}
\fi
%    \end{macrocode}

% Process as usual until here:
%    \begin{macrocode}
\fi
%    \end{macrocode}

% End of document body:
%    \begin{macrocode}
\end{document}
%    \end{macrocode}
%\iffalse
%</samplemain>
%\fi
%
% %%%%%%%%%%%%%%%%%%%%%%%%%%%%%%%%%%%%%%
% \paragraph{Chapter Include Files.}
%
% The include files are called |cdocsch1.tex| and |cdocsch2.tex|.
%
%\iffalse
%<*samplechap1|samplechap2>
%\fi

% Optional override for |\version| flag:
%    \begin{macrocode}
%%\providecommand{\version}{final}
%    \end{macrocode}

% Include the main document:
%    \begin{macrocode}
\input{childdoc.def}
\childdocof{cdocsamp}
%    \end{macrocode}

%\iffalse
%</samplechap1|samplechap2>
%\fi
%
%\iffalse
%<*samplechap1>
%\fi
% Some text for chapter 1:
%    \begin{macrocode}
\section{one}
some text in chapter one
%    \end{macrocode}

%\iffalse
%</samplechap1>
%\fi
% Some text for chapter 2:
%\iffalse
%<*samplechap2>
%\fi
%    \begin{macrocode}
\section{two}
more text in chapter two
%    \end{macrocode}

%\iffalse
%</samplechap2>
%\fi
%
% %%%%%%%%%%%%%%%%%%%%%%%%%%%%%%%%%%%%%%
% \paragraph{Part Include Files.}
%
% The include files are called |cdocspt3.tex| and |cdocspt4.tex|.
%
%\iffalse
%<*samplepart3|samplepart4>
%\fi

% Optional override for |\version| flag:
%    \begin{macrocode}
%%\providecommand{\version}{final}
%    \end{macrocode}

% Include the main document:
%    \begin{macrocode}
\input{childdoc.def}
\childdocby{cdocsamp}
%    \end{macrocode}

%\iffalse
%</samplepart3|samplepart4>
%\fi
%
%\iffalse
%<*samplepart3>
%\fi
% Some text for part 3:
%    \begin{macrocode}
some text in part three
%    \end{macrocode}

%\iffalse
%</samplepart3>
%\fi
% Some text for part 4:
%\iffalse
%<*samplepart4>
%\fi
%    \begin{macrocode}
more text in part four
%    \end{macrocode}

%\iffalse
%</samplepart4>
%\fi
%
% %%%%%%%%%%%%%%%%%%%%%%%%%%%%%%%%%%%%%%
% \paragraph{Forwarding for a Complete Draft.}
%
% The following forwarding file |cdocsdrf.tex|
% compiles the main document in draft mode:
%\iffalse
%<*sampledraft>
%\fi
%    \begin{macrocode}
\def\version{draft}
\input{childdoc.def}
\childdocforward{cdocsamp}
%    \end{macrocode}

%\iffalse
%</sampledraft>
%\fi
%
% %%%%%%%%%%%%%%%%%%%%%%%%%%%%%%%%%%%%%%
% \paragraph{Forwarding for Final Version of the Chapters.}
%
% The following forwarding files |cdocsfn1.tex| and |cdocsfn2.tex|
% (with identical content)
% compile the final versions of the child documents
% |cdocsch1.tex| and |cdocsch2.tex|, respectively:
%\iffalse
%<*samplefinal>
%\fi
%    \begin{macrocode}
\def\version{final}
\input{childdoc.def}
\childdocforwardprefix[cdocsamp]{cdocsfn}{cdocsch}
%    \end{macrocode}

%\iffalse
%</samplefinal>
%\fi
%
% %%%%%%%%%%%%%%%%%%%%%%%%%%%%%%%%%%%%%%
% \paragraph{Command Line Processing.}
%
% The following three command lines generate the output files
% |cdocscld|, |cdocscl1| and |cdocscl2|
% which should be identical to
% |cdocsdrf|, |cdocsch1| and |cdocsfn2|, respectively:
% \begin{center}
% \begin{tabular}{l}
% |latex -jobname cdocscld \|\\
% |  "\def\version{draft}\input{childdoc.def}\childdocforward{cdocsamp}"|\\
% |latex -jobname cdocscl1 \|\\
% |  "\input{childdoc.def}\childdocforward[cdocsamp]{cdocsch1}"|\\
% |latex -jobname cdocscl2 \|\\
% |  "\def\version{final}\input{childdoc.def}\childdocforward{cdocsch2}"|
% \end{tabular}
% \end{center}
% Note that the trailing backslash on each first line
% merely continues the input to the second line
% (for convenient cut ant paste).
% Furthermore, the command |latex| can be replaced by any
% of its alternative versions such as |pdflatex|.
%
% %%%%%%%%%%%%%%%%%%%%%%%%%%%%%%%%%%%%%%%%%%%%%%%%%%%%%%%%%%%%%%%%%%%%%%%%%%%%%%
% %%%%%%%%%%%%%%%%%%%%%%%%%%%%%%%%%%%%%%%%%%%%%%%%%%%%%%%%%%%%%%%%%%%%%%%%%%%%%%
% \section{Implementation}
%\iffalse
%<*package>
%\fi
%
% This section describes the definitions file |childdoc.def|.

% The definitions cannot be loaded using |\usepackage| or |\RequirePackage|
% which has a mechanism to prevent loading a style file more than once.
% When loading the definitions by means of |\input|
% multiple instances have to be prevented manually:
%\iffalse
%This code needs to be before the `\ProvidesFile' directive
%which is defined at the beginning of this file.
%Therefore it is also placed there and commented out here.
%</package>
%<*discard>
%\fi
%    \begin{macrocode}
\ifdefined\childdocmain\endinput\fi
%    \end{macrocode}
%\iffalse
%</discard>
%<*package>
%\fi
%
% \macro{\ifchilddoc}
% \macro{\ifchilddocmanual}
% The conditional |\ifchilddoc| tells whether a
% child (true) or main (false) document is being compiled.
% The conditional |\ifchilddocmanual| tells whether
% the |\includeonly| mechanism is used (false) or
% the selection of child files must be performed manually (true).
% The definitions initialise to false:
%    \begin{macrocode}
\newif\ifchilddoc
\newif\ifchilddocmanual
%    \end{macrocode}

% \macro{\childdocname}
% \macro{\childdocjob}
% The macro |\childdocname| stores the name of the main document
% to be compiled. The macro |\childdocjob| stores the name of
% the document on which the \LaTeX{} compiler was originally invoked.
% The content of |\jobname| cannot be compared
% to filenames specified in the source due to different catcodes.
% The following code rescans |\jobname|, stores the result
% in |\childdocname| and saves a copy in |\childdocjob|:
%    \begin{macrocode}
\edef\childdocname{\scantokens\expandafter{\jobname\noexpand}}
\let\childdocjob\childdocname
%    \end{macrocode}

% \macro{\childdocdisable}
% The macro |\childdocdisable| prevents the main file
% from being processed more than once.
% At this stage, the main document command |\childdocmain|
% is assumed to be called once again where it should do nothing.
% Any subsequent call to it should prevent
% a secondary processing of the main document
% It overwrites the forwarding commands
% |\childdocof| and |\childdocforward|
% with empty macros to prevent further inclusions of the main document:
%    \begin{macrocode}
\newcommand{\childdocdisable}
{
  \renewcommand{\childdocmain}[1]{\renewcommand{\childdocmain}[1]{\endinput}}
  \renewcommand{\childdocof}[1]{}
  \renewcommand{\childdocby}[2][]{}
  \renewcommand{\childdocforward}[2][]{}
  \renewcommand{\childdocdisable}{}
}
%    \end{macrocode}

% \macro{\childdocmain}
% The macro |\childdocmain| is to be called at the top of the main file
% with nothing or the main filename (without extension) as argument.
% First, it breaks loops.
% If the argument is not empty and does not match |\childdocname|
% (which is set by the first inclusion of |childdoc.def|),
% |\ifchilddoc| is set to true, |\includeonly| is applied to the child file
% and |\jobname| is set to the main file
% (for proper handling of |.aux| files):
%    \begin{macrocode}
\newcommand{\childdocmain}[1]
{
  \childdocdisable\childdocmain{}
  \if?#1?\else
    \begingroup
      \def\childdoctmp{#1}
      \ifx\childdoctmp\childdocname
        \def\childdoctmp{}
      \else
        \def\childdoctmp
        {
          \childdoctrue
          \includeonly{\childdocname}
          \def\childdocjob{#1}
          \def\jobname{#1}
        }
      \fi
      \expandafter
    \endgroup
    \childdoctmp
  \fi
}
%    \end{macrocode}

% \macro{\childdocof}
% The command |\childdocof| redirects
% compilation to the main file |#1|.
%    \begin{macrocode}
\newcommand{\childdocof}[1]
{
  \childdocdisable
  \childdoctrue
  \includeonly{\childdocname}
  \def\jobname{#1}
  \def\childdocjob{#1}
  \input{#1}
}
%    \end{macrocode}

% \macro{\childdocby}
% The command |\childdocby| ....
%    \begin{macrocode}
\newcommand{\childdocby}[2][]
{
  \childdocdisable
  \childdoctrue
  \childdocmanualtrue
  \if?#1?\else
    \def\jobname{#2}
  \fi
  \def\childdocjob{#2}
  \input{#2}
  \endinput
}
%    \end{macrocode}

% \macro{\childdocforward}
% The command |\childdocforward| redirects
% compilation to the main file or
% (if the optional argument is given) a child file.
% Parameters are set as if the main file
% or a child file starting with |\childdocof| was compiled.
% Then compilation is handed over to the main file:
%    \begin{macrocode}
\newcommand{\childdocforward}[2][]
{
  \begingroup
    \if?#1?
      \def\childdoctmp
      {
        \def\childdocname{#2}
        \def\childdocjob{#2}
        \def\jobname{#2}
        \input{#2}
        \endinput
      }
    \else
      \def\childdoctmp
      {
        \childdocdisable
        \def\childdocname{#2}
        \childdoctrue
        \includeonly{#2}
        \def\childdocjob{#1}
        \def\jobname{#1}
        \input{#1}
        \endinput
      }
    \fi
    \expandafter
  \endgroup
  \childdoctmp
}
%    \end{macrocode}

% \macro{\childdocforwardprefix}
% The command |\childdocforwardprefix| redirects
% compilation to the main or a child file by means of a pattern.
% The prefix |#1| in the current filename is replaced by |#2|
% and the suffix of the current filename is kept
% (it is assumed that the filename does not contain the substring `|~~~|'
% which is used as a delimiter).
% Compilation is handed over to the new file by |\childdocforward|:
%    \begin{macrocode}
\newcommand{\childdocforwardprefix}[3][]
{
  \begingroup
    \def\childdocextract #2##1~~~{\def\childdoctmp{\childdocforward[#1]{#3##1}}}
    \expandafter\childdocextract\childdocname~~~
    \expandafter
  \endgroup
  \childdoctmp
}
%    \end{macrocode}

% \macro{\childdoc}
% The deprecated macro |\childdoc| is a legacy version of |\childdocmain|:
%    \begin{macrocode}
\newcommand{\childdoc}{\childdocmain}
%    \end{macrocode}

% \macro{\childdocredirect}
% The deprecated macro |\childdocredirect| is a legacy version
% of |\childdocforward| and |\childdocforwardprefix|:
%    \begin{macrocode}
\newcommand{\childdocredirect}[2][]
{
  \begingroup
    \if?#1?
      \def\childdoctmp{\childdocforward{#2}}
    \else
      \def\childdoctmp{\childdocforwardprefix{#1}{#2}}
    \fi
    \expandafter
  \endgroup
  \childdoctmp
}
%    \end{macrocode}

%\iffalse
%</package>
%\fi
%
\endinput
|\\
|\childdocforward{|\textit{main}|}|\\
\end{tabular}
\end{center}
%
or alternatively with:
%
\begin{center}
\begin{tabular}{l}
|% \iffalse
%
% childdoc.dtx Copyright (C) 2017-2018 Niklas Beisert
%
% This work may be distributed and/or modified under the
% conditions of the LaTeX Project Public License, either version 1.3
% of this license or (at your option) any later version.
% The latest version of this license is in
%   http://www.latex-project.org/lppl.txt
% and version 1.3 or later is part of all distributions of LaTeX
% version 2005/12/01 or later.
%
% This work has the LPPL maintenance status `maintained'.
%
% The Current Maintainer of this work is Niklas Beisert.
%
% This work consists of the files childdoc.dtx and childdoc.ins
% and the derived files childdoc.def and cdocsamp.tex with
% cdocsch1.tex, cdocsch2.tex, cdocsdrf.tex, cdocsfn1.tex, cdocsfn2.tex.
%
%<package>\ifdefined\childdocmain\endinput\fi
%<package>\ProvidesFile{childdoc.def}[2018/12/30 v2.0 child document driver]
%<samplemain>\ProvidesFile{cdocsamp.tex}[2018/12/30 v2.0 sample for childdoc]
%<*driver>
%\ProvidesFile{childdoc.drv}[2018/12/30 v2.0 childdoc reference manual file]
\PassOptionsToClass{10pt,a4paper}{article}
\documentclass{ltxdoc}

\usepackage[margin=35mm]{geometry}
\usepackage{hyperref}
\usepackage{hyperxmp}
\usepackage[usenames]{color}

\hypersetup{colorlinks=true}
\hypersetup{pdfstartview=FitH}
\hypersetup{pdfpagemode=UseNone}
\hypersetup{pdfsource={}}
\hypersetup{pdflang={en-UK}}
\hypersetup{pdfcopyright={Copyright 2017-2018 Niklas Beisert.
  This work may be distributed and/or modified under the
  conditions of the LaTeX Project Public License, either version 1.3
  of this license or (at your option) any later version.}}
\hypersetup{pdflicenseurl={http://www.latex-project.org/lppl.txt}}
\hypersetup{pdfcontactaddress={ETH Zurich, ITP, HIT K,
  Wolfgang-Pauli-Strasse 27}}
\hypersetup{pdfcontactpostcode={8093}}
\hypersetup{pdfcontactcity={Zurich}}
\hypersetup{pdfcontactcountry={Switzerland}}
\hypersetup{pdfcontactemail={nbeisert@itp.phys.ethz.ch}}
\hypersetup{pdfcontacturl={http://people.phys.ethz.ch/\xmptilde nbeisert/}}

\newcommand{\secref}[1]{\hyperref[#1]{section \ref*{#1}}}

\parskip1ex
\parindent0pt
\let\olditemize\itemize
\def\itemize{\olditemize\parskip0pt}

\begin{document}

\title{The \textsf{childdoc} Package}
\hypersetup{pdftitle={The childdoc Package}}
\author{Niklas Beisert\\[2ex]
  Institut f\"ur Theoretische Physik\\
  Eidgen\"ossische Technische Hochschule Z\"urich\\
  Wolfgang-Pauli-Strasse 27, 8093 Z\"urich, Switzerland\\[1ex]
  \href{mailto:nbeisert@itp.phys.ethz.ch}
  {\texttt{nbeisert@itp.phys.ethz.ch}}}
\hypersetup{pdfauthor={Niklas Beisert}}
\hypersetup{pdfsubject={Manual for the LaTeX2e Package childdoc}}
\date{30 December 2018, \textsf{v2.0}}
\maketitle

\begin{abstract}\noindent
\textsf{childdoc} is a \LaTeXe{} package
that enables the direct compilation
of document sections included by |\include|
to individual files.
\end{abstract}

\begingroup
\parskip0ex
\tableofcontents
\endgroup

%%%%%%%%%%%%%%%%%%%%%%%%%%%%%%%%%%%%%%%%%%%%%%%%%%%%%%%%%%%%%%%%%%%%%%%%%%%%%%%%
%%%%%%%%%%%%%%%%%%%%%%%%%%%%%%%%%%%%%%%%%%%%%%%%%%%%%%%%%%%%%%%%%%%%%%%%%%%%%%%%
\section{Introduction}

\LaTeX{} provides a mechanism to structure a large document (such as a book)
into a main file and several child files (containing the chapters)
using the |\include| command.
This mechanism is beneficial for documents
which span hundreds of pages in order to
make the source file(s) more manageable.
Moreover, compilation can be restricted to
selected child files by means of the |\includeonly| command.
The latter feature can be used to reduce the compilation time while editing
(this was significantly more useful in the earlier days of \LaTeX{})
or to generate a smaller document which is easier to navigate.
Another application of |\includeonly| is to generate
documents consisting of selected parts of the complete document.

However, there are a few drawbacks of the plain |\include| mechanism:
\begin{itemize}
\item
The child files cannot be compiled on their own,
they can only be compiled via the main file.
A naive editing environment
(such as a text editor with an option
to have the current file processed by \LaTeX)
may require one to switch to the main file before compiling;
attempting to compile the child file produces errors.
\item
The main file must be modified (each time)
to adjust the |\includeonly| command
to the present needs. This easily leaves the main file in a messy state.
\item
The generated document will always carry the filename
of the main document. This is inconvenient if
several child files are to be compiled and
to be kept for distribution.
\end{itemize}

The present package provides a simple interface
to make child files individually compilable by \LaTeX{}.
Compiling a child file then has the same effect as compiling
the main file with an |\includeonly| command
to select the appropriate child.
Moreover the generated document will carry the name of the child
rather than the main file.
This resolves all three above issues.

This feature is meant to make the editing of books,
thesis documents and lecture notes somewhat more convenient.
However, the package can also be used efficiently for
composing a series of documents (such as exercise sheets)
which are typically distributed individually.
It then assists the author in generating the individual documents
(potentially in different versions)
as well as a document containing the collected series.
Another application is in developing style files
or other kinds of included material
where compilation of the style file could redirect
to a sample or test file.

%%%%%%%%%%%%%%%%%%%%%%%%%%%%%%%%%%%%%%%%%%%%%%%%%%%%%%%%%%%%%%%%%%%%%%%%%%%%%%%%
%%%%%%%%%%%%%%%%%%%%%%%%%%%%%%%%%%%%%%%%%%%%%%%%%%%%%%%%%%%%%%%%%%%%%%%%%%%%%%%%
\section{Usage}

First of all, the package \textsf{childdoc} is \emph{not} a standard
\LaTeXe{} |.sty| style file! Therefore it needs to be invoked in
a non-standard way.

%%%%%%%%%%%%%%%%%%%%%%%%%%%%%%%%%%%%%%%%%%%%%%%%%%%%%%%%%%%%%%%%%%%%%%%%%%%%%%%%
\subsection{Included Files}
\label{sec:include}

%%%%%%%%%%%%%%%%%%%%%%%%%%%%%%%%%%%%%%%%
\DescribeMacro{\childdocmain}
To use the package, add the commands
\begin{center}
\begin{tabular}{l}
|\input{childdoc.def}|\\
|\childdocmain{}|\\
\end{tabular}
\end{center}
at the very top of the main \LaTeX{} file,
in particular \emph{before} the |\documentclass| statement!
The argument of |\childdocmain| should be left empty
(but it must be present).

%%%%%%%%%%%%%%%%%%%%%%%%%%%%%%%%%%%%%%%%
\DescribeMacro{\childdocof}
Furthermore, add the commands
\begin{center}
\begin{tabular}{l}
|\input{childdoc.def}|\\
|\childdocof{|\textit{main}|}|\\
\end{tabular}
\end{center}
at the top of every child file \textit{child}
which is included by |\include{|\textit{child}|}|
from within the main file
(or at least for those files to be compiled individually).
The argument \textit{main} must be the filename of the main file.

There are a couple of
considerations in setting up the main and child documents:

%%%%%%%%%%%%%%%%%%%%%%%%%%%%%%%%%%%%%%%%
\paragraph{Restrictions.}

Please note the following restrictions:
\begin{itemize}
\item
|\childdocmain| must be called with one argument \textit{main}
to ensure compatibility with earlier version of the package.
It must either be empty (|\childdocmain{}|)
or precisely match the filename of the main file in which it is specified.
See \secref{sec:detection} for further information.
\item
The filename \textit{main} must be specified without the |.tex| extension.
\item
The filename \textit{main} is case sensitive
(even in case-insensitive file systems)
due to internal string comparison.
\item
The argument \textit{main} should be fully expanded, it cannot be a macro.
\item
Subdirectories and special characters should be avoided in filenames.
\item
The command |\childdocmain{|\textit{main}|}| must be followed by a whitespace.
It should not be followed immediately by another command
or by a comment mark `|%|'.
This is because the \TeX{} parser reads the token immediately following
the argument of |\childdocmain| and puts it
at the beginning of every child section;
however, a white\-space is ignored.
\end{itemize}

%%%%%%%%%%%%%%%%%%%%%%%%%%%%%%%%%%%%%%%%
\paragraph{Content of Main File.}

It is advisable to place all content in the child files included by |\include|.
Any output contained in the main file will appear in all child documents
unless suppressed manually;
it cannot be suppressed automatically by the |\includeonly| directive
and thus should normally be avoided.
A method to include some content in the main file
by means of conditional processing is described in \secref{sec:conditional}.

%%%%%%%%%%%%%%%%%%%%%%%%%%%%%%%%%%%%%%%%
\paragraph{Page Numbering.}

When only a part of the document is compiled,
the appropriate numbering of pages
(as well as other status parameters)
is determined from the |.aux| files.
The latter contain information from previous passes.
However this information needs to propagate through
all intermediate child documents.
Therefore the page numbering in child documents may well
be inconsistent until the complete document is compiled at least once.

A useful (if unconventional) way to always ensure a consistent
page numbering is to restart the numbering in each child document
and denote the pages by `\textit{child}|.|\textit{page}'
where \textit{child} represents the chapter/section number of the child file.
This can be achieved by the command
|\numberwithin{page}{|\textit{child}|}|
of the \textsf{amsmath} package
where \textit{child} can be |chapter| or |section|
depending on the chosen structuring.
Alternatively, one can modify the macro |\thepage| appropriately
and reset the counter |page| at the start of each child file.

%%%%%%%%%%%%%%%%%%%%%%%%%%%%%%%%%%%%%%%%%%%%%%%%%%%%%%%%%%%%%%%%%%%%%%%%%%%%%%%%
\subsection{Conditional Processing}
\label{sec:conditional}

The package provides a mechanism to compile different versions
of a document. To customise the versions further some conditional processing
can come in handy to distinguish which version is being compiled.
The package provides two macros to describe the compilation context:

%%%%%%%%%%%%%%%%%%%%%%%%%%%%%%%%%%%%%%%%
\DescribeMacro{\ifchilddoc}
The conditional |\ifchilddoc| distinguishes between the compilation of
child documents and the main document:
%
\begin{center}
|\ifchilddoc |\textit{child-code}| |[|\||else |\textit{main-code}]| \||fi|
\end{center}

%%%%%%%%%%%%%%%%%%%%%%%%%%%%%%%%%%%%%%%%
\DescribeMacro{\childdocname}
\DescribeMacro{\childdocjob}
The macro |\childdocname| contains the filename (without extension)
of the main or child file being processed.
Note that |\childdocjob| will always contain the name of the main file.

%%%%%%%%%%%%%%%%%%%%%%%%%%%%%%%%%%%%%%%%
\paragraph{Title Page.}

Conditional processing can be used to include a title or banner page
in the main document when proper precautions are taken.
Importantly, the code in the main file should ensure that the page counter
(as well as other status parameters which are stored in the |.aux| files)
takes the same value after the conditional processing.
Otherwise the page numbers may take divergent values
depending on which part is compiled.

For example, a title page could be declared by:
%
\begin{center}
\begin{tabular}{l}
|\ifchilddoc\||else|\\
|\addtocounter{page}{-1}|\\
\textit{code for title page}\\
|\newpage|\\
|\||fi|
\end{tabular}
\end{center}
%
A banner page for the child documents can be generated by:
%
\begin{center}
\begin{tabular}{l}
|\ifchilddoc|\\
|\addtocounter{page}{-1}|\\
\textit{code for banner page}\\
|\newpage|\\
|\||fi|
\end{tabular}
\end{center}
%
Here one could write a message such as:
\begin{center}
|This is the part \childdocname{} of \childdocjob{}.|
\end{center}

%%%%%%%%%%%%%%%%%%%%%%%%%%%%%%%%%%%%%%%%%%%%%%%%%%%%%%%%%%%%%%%%%%%%%%%%%%%%%%%%
\subsection{Flags}
\label{sec:flags}

The package makes it easy to generate different versions
of the main or child documents.
To this end compilation flags can be defined
and assigned different default values.
They will be particularly useful in conjunction
with the forwarding mechanism described in \secref{sec:forward}.

For example, it may be useful to have a flag |\version|
which can be set to |draft| or |final|.
The document source will contain some conditional code
depending on the value of |\version|.
Suppose further, the flag should default to |final| for the main file
and to |draft| for child files
which is a natural assignment for editing the document.
This is achieved by placing the following code
in the preamble of the main document
(below the |\childdocmain| directive):
%
\begin{center}
\begin{tabular}{l}
|\ifchilddoc|\\
|\providecommand{\version}{draft}|\\
|\||else|\\
|\providecommand{\version}{final}|\\
|\||fi|
\end{tabular}
\end{center}
%
The definition by |\providecommand| makes sure
that previous definitions are not overwritten.
Further statements |\providecommand{\version}{...}|
can thus be added before the above code to override it.

For the main file, one might add a line
(between |\childdocmain| and the above block)
%
\begin{center}
|%\ifchilddoc\||else\providecommand{\version}{draft}\||fi|
\end{center}
%
which can be uncommented to produce a draft version.
Likewise one can add a line to the very top of a child file
(above the |\childdocof{|\textit{main}|}| directive)
%
\begin{center}
|%\providecommand{\version}{final}|
\end{center}
%
which can be uncommented to produce the final version of this child document.

%%%%%%%%%%%%%%%%%%%%%%%%%%%%%%%%%%%%%%%%%%%%%%%%%%%%%%%%%%%%%%%%%%%%%%%%%%%%%%%%
\subsection{Forwarding}
\label{sec:forward}

Different versions of the main or child documents
using compilation flags as described in \secref{sec:flags}
can be (permanently) stored in different files
for convenient compilation, viewing and distribution.
To this end, the package defines a command
to pass on compilation to a different file:

%%%%%%%%%%%%%%%%%%%%%%%%%%%%%%%%%%%%%%%%
\DescribeMacro{\childdocforward}
The command |\childdocforward| redirects processing to
another source file:
%
\begin{center}
\begin{tabular}{l}
|\input{childdoc.def}|\\
|\childdocforward[|\textit{main}|]{|\textit{dest}|}|\\
\end{tabular}
\end{center}
%
The argument \textit{dest} is the destination file
(without extension).
It should be the main file or one of the child files.
Note that further \textsf{childdoc} directives
such as |\childdocof| and |\childdocforward|
in the indicated file will be processed in this form.
The optional argument \textit{main}
passes on directly to the main file \textit{main}
while pretending to compile the child \textit{dest}.
This form behaves as if \textit{dest}
issues |\childdocof{|\textit{main}|}| right away,
and no further \textsf{childdoc} directives will be processed.

%%%%%%%%%%%%%%%%%%%%%%%%%%%%%%%%%%%%%%%%
\DescribeMacro{\...prefix}
In the alternative form |\childdocforwardprefix|,
%
\begin{center}
\begin{tabular}{l}
|\input{childdoc.def}|\\
|\childdocforwardprefix[|\textit{main}|]{|\textit{prefix}|}{|\textit{dest}|}|
\end{tabular}
\end{center}
%
the destination file is determined by a pattern
depending on the current file:
To make this work, the current file must be called
`{\textit{prefix}\hspace{0.2em}\textit{suffix}}'
with \textit{prefix} matching precisely the argument.
Processing is then passed on to the file
`{\textit{dest}\hspace{0.2em}\textit{suffix}}'.
Surely, the same effect is achieved by
directly specifying the
argument `{\textit{dest}\hspace{0.2em}\textit{suffix}}'
in the first form.
However, that requires to set up a different file
for each child. With the alternative form of the command
all these files can have exactly the same content
which simplifies setting them up and maintaining them.

For example, the following file |draft.tex|
with a compilation flag |\version| as described in \secref{sec:flags}
compiles the main document as a draft:
%
\begin{center}
\begin{tabular}{l}
|\def\version{draft}|\\
|\input{childdoc.def}|\\
|\childdocforward{|\textit{main}|}|
\end{tabular}
\end{center}
%
Likewise, the following files |final|\textit{nn}|.tex|
compile the final version of the child document
|child|\textit{nn}|.tex|:
%
\begin{center}
\begin{tabular}{l}
|\def\version{final}|\\
|\input{childdoc.def}|\\
|\childdocforwardprefix{final}{child}|
\end{tabular}
\end{center}
%

Note that when several versions of a main file and/or of each child file
are to be generated, it may be convenient to set up a |Makefile| or
shell script to automatise the process.

%%%%%%%%%%%%%%%%%%%%%%%%%%%%%%%%%%%%%%%%%%%%%%%%%%%%%%%%%%%%%%%%%%%%%%%%%%%%%%%%
\subsection{Command Line Processing}
\label{sec:commandline}

The effect of redirection files can also be achieved by invoking
the \LaTeX{} compiler with a more elaborate command line.
Most conveniently this should be done as part
of a shell script or a |Makefile|.

When using \textsf{childdoc} in the main file, the following
command lines effectively perform a redirection
(note that depending on the shell being used,
backslashes may have to be doubled: `|\|' $\to$ `|\\|'):
%
\begin{center}
|... -jobname "|\textit{target}|" |\\|"|[\textit{flags}]%
|\input{childdoc.def}\childdocforward[|\textit{main}|]{|\textit{dest}|}"|
\end{center}
%
Here \textit{target} is the name of the output file,
\textit{main} is the name of the main file
and \textit{dest} is the name of the main or child file to be processed
(all filenames without extensions).
The optional argument \textit{main} can be omitted
if \textit{main} matches \textit{dest}.
Optionally, compilation \textit{flags} can be defined via |\def| commands.
This command line makes the \TeX{} engine believe
it is compiling the file \textit{target}
whose content is specified as the latter parameter.
The provided code then forwards the processing to
\textit{main} or \textit{dest} as described in \secref{sec:forward}.

%%%%%%%%%%%%%%%%%%%%%%%%%%%%%%%%%%%%%%%%%%%%%%%%%%%%%%%%%%%%%%%%%%%%%%%%%%%%%%%%
\subsection{Include by Input}
\label{sec:input}

Including child documents by |\include| has some restrictions by design.
Most notably, the content of a child document always occupies
its own set of pages; pages cannot be shared between child documents.
Usually, this behaviour makes perfect sense
because each child document contain an essential part of the document.
However, in some situations it may be desirable to compose
a document from a collection of parts
without having mandatory page breaks between then.
For this case, the package
provides a mechanism to include parts
by |\input| which can also be processed individually.
However, by construction this mechanism
requires manual handling of the content to be output.

%%%%%%%%%%%%%%%%%%%%%%%%%%%%%%%%%%%%%%%%
\DescribeMacro{\ifchilddocmanual}
The main file should be prepared as usual, see \secref{sec:include}.
However, the document body must make a distinction
between processing of an individual part and of the main document, e.g.:
%
\begin{center}
\begin{tabular}{l}
|\ifchilddocmanual|\\
|\input{\childdocname}|\\
|\||else|\\
\textit{document body with }|\input{|\textit{part}|}|\\
|\||fi|
\end{tabular}
\end{center}
%
The conditional |\ifchilddocmanual| is true whenever
a part to be included by |\input| is being compiled,
and the name of the part is stored in |\childdocname|.

%%%%%%%%%%%%%%%%%%%%%%%%%%%%%%%%%%%%%%%%
\DescribeMacro{\childdocby}
Each part to be included by |\input| should start with:
%
\begin{center}
\begin{tabular}{l}
|\input{childdoc.def}|\\
|\childdocby{|\textit{main}|}|\\
\end{tabular}
\end{center}
%
The directive |\childdocby| is similar to |\childdocof|
described in \secref{sec:include},
but the subsequent selection of content must be done manually.
To that end, both |\ifchilddoc| and |\ifchilddocmanual|
will be true upon processing of a part,
and the name of the part is stored in |\childdocname|.
Note that |\jobname| will be set to the filename of the current part
so that each part receives an individual |.aux| file
that does not interfere with the |.aux| file(s) of the main document.
This behaviour can be altered by the alternative form
|\childdocby[*]{|\textit{main}|}| (with a non-empty optional argument)
which uses the |.aux| file of the main document
by setting |\jobname| to \textit{main}.

%%%%%%%%%%%%%%%%%%%%%%%%%%%%%%%%%%%%%%%%%%%%%%%%%%%%%%%%%%%%%%%%%%%%%%%%%%%%%%%%
\subsection{Driver Development}
\label{sec:driver}

The \textsf{childdoc} mechanism can also be use for the development
of definition files such as \LaTeX{} styles or classes.
This case differs from the above setup with multiple parts
included by |\include| in that no |\includeonly| should be invoked.
This can be achieved by starting the include file
(before |\ProvidesPackage|) with:
%
\begin{center}
\begin{tabular}{l}
|\input{childdoc.def}|\\
|\childdocforward{|\textit{main}|}|\\
\end{tabular}
\end{center}
%
or alternatively with:
%
\begin{center}
\begin{tabular}{l}
|\input{childdoc.def}|\\
|\childdocby{|\textit{main}|}|\\
\end{tabular}
\end{center}
%
Both forms have slightly different effects as described above.
The main file is prepared as usual, see \secref{sec:include}.

%%%%%%%%%%%%%%%%%%%%%%%%%%%%%%%%%%%%%%%%%%%%%%%%%%%%%%%%%%%%%%%%%%%%%%%%%%%%%%%%
\subsection{Legacy Detection}
\label{sec:detection}

The directive |\childdocmain| in the main file can detect
whether the complete document or merely a child is to be compiled
even without using the directive |\childdocof|.
This method is deprecated because it is less robust
and there is no compelling reason to use it;
it is merely provided for backward compatibility
and it may be removed in future versions.

If the detection mechanism is to be used,
it is mandatory to correctly specify
the filename of the main file as the argument of |\childdocmain|:
%
\begin{center}
\begin{tabular}{l}
|\input{childdoc.def}|\\
|\childdocmain{|\textit{main}|}|\\
\end{tabular}
\end{center}
%
If |\jobname| does not match the argument \textit{main} of |\childdocmain|,
it is assumed that |\jobname| points to the child file to be compiled.
When using |\childdocmain| with the main file specified as argument,
it suffices to start a child file
with just |\input{|\textit{main}|}|
without loading of the package and using |\childdocof|.
If instead all processing is done
with the appropriate \textsf{childdoc} directives,
the argument of \textit{main} of |\childdocmain| can be empty.

An alternative version of the command line processing described
in \secref{sec:commandline} using the detection mechanism reads:
%
\begin{center}
|... -jobname "|\textit{target}|" "|[\textit{flags}]%
[|\def\jobname{|\textit{dest}|}|]|\input{|\textit{main}|}"|
\end{center}

%%%%%%%%%%%%%%%%%%%%%%%%%%%%%%%%%%%%%%%%%%%%%%%%%%%%%%%%%%%%%%%%%%%%%%%%%%%%%%%%
\subsection{Manual Code}
\label{sec:manual}

In case one cannot be certain whether the definitions file |childdoc.def|
is installed on the target \TeX{} distribution
and one prefers not to ship it,
it is conceivable to paste a few relevant commands into the sources.

To that end, drop all statements |\input{childdoc.def}|
and perform the replacements as outlined below.
Instead of |\childdocmain{|\textit{main}|}| add the following code
to the top of the main file:
%
\begin{center}
\begin{tabular}{l}
|\||ifdefined\childdocname\endinput\||fi\newif\ifchilddoc|\\
|\edef\childdocname{\scantokens\expandafter{\jobname\noexpand}}|\\
|\def\childdocmain{|\textit{main}|}\||ifx\childdocmain\childdocname\||else|\\
|\childdoctrue\includeonly{\childdocname}\let\jobname\childdocmain\||fi|\\
\end{tabular}
\end{center}
%
Instead of |\childdocof{|\textit{main}|}| just include the main file
at the top of each child file:
%
\begin{center}
|\input{|\textit{main}|}|
\end{center}
%
A simple redirection |\childdocforward{|\textit{dest}|}| is achieved by:
%
\begin{center}
|\def\jobname{|\textit{dest}|}\input{\jobname}|
\end{center}
%
The redirection with prefix
|\childdocforwardprefix[|\textit{prefix}|]{|\textit{dest}|}|
is accomplished by:
%
\begin{center}
\begin{tabular}{l}
|{\edef\jobname{\scantokens\expandafter{\jobname\noexpand}}|\\
|\def\redirectjob |\textit{prefix}|#1~~~{\gdef\jobname{|\textit{dest}|#1}}|\\
|\expandafter\redirectjob\jobname~~~}\input{\jobname}|
\end{tabular}
\end{center}

In an alternative approach,
child documents can be compiled by a specific command line
without additional code or specific definitions:
%
\begin{center}
|... -jobname "|\textit{target}|" "|[\textit{flags}]%
|\includeonly{|\textit{dest}|}\input{|\textit{main}|}"|
\end{center}
%

%%%%%%%%%%%%%%%%%%%%%%%%%%%%%%%%%%%%%%%%%%%%%%%%%%%%%%%%%%%%%%%%%%%%%%%%%%%%%%%%
%%%%%%%%%%%%%%%%%%%%%%%%%%%%%%%%%%%%%%%%%%%%%%%%%%%%%%%%%%%%%%%%%%%%%%%%%%%%%%%%
\section{Information}

%%%%%%%%%%%%%%%%%%%%%%%%%%%%%%%%%%%%%%%%%%%%%%%%%%%%%%%%%%%%%%%%%%%%%%%%%%%%%%%%
\subsection{Copyright}

Copyright \copyright{} 2017--2018 Niklas Beisert

This work may be distributed and/or modified under the
conditions of the \LaTeX{} Project Public License, either version 1.3
of this license or (at your option) any later version.
The latest version of this license is in
  \url{http://www.latex-project.org/lppl.txt}
and version 1.3 or later is part of all distributions of \LaTeX{}
version 2005/12/01 or later.

This work has the LPPL maintenance status `maintained'.

The Current Maintainer of this work is Niklas Beisert.

This work consists of the files |README.txt|, |childdoc.ins| and |childdoc.dtx|
as well as the derived files |childdoc.def|, |cdocsamp.tex|
with |cdocsch1.tex|, |cdocsch2.tex|, |cdocspt3.tex|, |cdocspt4.tex|,
|cdocsdrf.tex|, |cdocsfn1.tex|, |cdocsfn2.tex|
as well as |childdoc.pdf|.

%%%%%%%%%%%%%%%%%%%%%%%%%%%%%%%%%%%%%%%%%%%%%%%%%%%%%%%%%%%%%%%%%%%%%%%%%%%%%%%%
\subsection{Files and Installation}

The package consists of the files:
%
\begin{center}
\begin{tabular}{ll}
    |README.txt|   & readme file \\
    |childdoc.ins| & installation file \\
    |childdoc.dtx| & source file \\
    |childdoc.def| & definition file \\
    |cdocsamp.tex| & sample main file \\
    |cdocsch1.tex| & sample include file \\
    |cdocsch2.tex| & sample include file \\
    |cdocspt3.tex| & sample part file \\
    |cdocspt4.tex| & sample part file \\
    |cdocsdrf.tex| & sample redirection file \\
    |cdocsfn1.tex| & sample redirection file \\
    |cdocsfn2.tex| & sample redirection file \\
    |childdoc.pdf| & manual
\end{tabular}
\end{center}
%
The distribution consists of the files
|README.txt|, |childdoc.ins| and |childdoc.dtx|.
%
\begin{itemize}
\item
Run (pdf)\LaTeX{} on |childdoc.dtx|
to compile the manual |childdoc.pdf| (this file).
\item
Run \LaTeX{} on |childdoc.ins| to create the definitions file |childdoc.def|
and the sample |cdocsamp.tex| with include files
|cdocsch1.tex|, |cdocsch2.tex|, |cdocspt3.tex|, |cdocspt4.tex|,
|cdocsdrf.tex|, |cdocsfn1.tex|, |cdocsfn2.tex|.
Then copy the file |childdoc.def| to an appropriate directory of your \LaTeX{}
distribution, e.g.\ \textit{texmf-root}|/tex/latex/childdoc|.
\end{itemize}

%%%%%%%%%%%%%%%%%%%%%%%%%%%%%%%%%%%%%%%%%%%%%%%%%%%%%%%%%%%%%%%%%%%%%%%%%%%%%%%%
\subsection{Related CTAN Packages}

There are several other packages which offer a similar functionality:
%
\begin{itemize}
\item
The packages
\href{http://ctan.org/pkg/docmute}{\textsf{docmute}},
\href{http://ctan.org/pkg/includex}{\textsf{includex}} and
\href{http://ctan.org/pkg/standalone}{\textsf{standalone}}
provide commands to include only the document body of
a child file thus allowing both files to be compiled individually.
\item
The packages \href{http://ctan.org/pkg/subdocs}{\textsf{subdocs}}
and \href{http://ctan.org/pkg/subfiles}{\textsf{subfiles}}
provide structures in which the main and child documents can be
encapsulated and allowing them to be compiled individually.
The inclusion mechanism is different from the conventional |\include|.
\item
The package \href{http://ctan.org/pkg/combine}{\textsf{combine}}
is an elaborate solution to combine several documents into one.
\end{itemize}
%
See also the CTAN topic \href{http://ctan.org/topic/subdocs}{\textsf{subdocs}}
for further related packages.
The present package differs from the above solutions in that
a document structure constructed with the conventional |\include| mechanism
just needs two extra commands at the top of every file
such that all constituent files can be compiled individually.

%%%%%%%%%%%%%%%%%%%%%%%%%%%%%%%%%%%%%%%%%%%%%%%%%%%%%%%%%%%%%%%%%%%%%%%%%%%%%%%%
%\subsection{Feature Suggestions}
%
%The following is a list of features which may be useful for future
%versions of this package:
%%
%\begin{itemize}
%\item
%\ldots
%\end{itemize}

%%%%%%%%%%%%%%%%%%%%%%%%%%%%%%%%%%%%%%%%%%%%%%%%%%%%%%%%%%%%%%%%%%%%%%%%%%%%%%%%
\subsection{Revision History}

%%%%%%%%%%%%%%%%%%%%%%%%%%%%%%%%%%%%%%%%
\paragraph{v2.0:} 2018/12/30

\begin{itemize}
\item
immediate forward processing
\item
added |\childdocby| mechanism
\item
manual restructured
\end{itemize}

%%%%%%%%%%%%%%%%%%%%%%%%%%%%%%%%%%%%%%%%
\paragraph{v1.6:} 2018/01/17

\begin{itemize}
\item
application for development of include files
\item
corrections to manual
\end{itemize}

%%%%%%%%%%%%%%%%%%%%%%%%%%%%%%%%%%%%%%%%
\paragraph{v1.5:} 2017/05/21

\begin{itemize}
\item
more complete structuring introduced
\item
|\childdocof| introduced
\item
|\childdoc| renamed to |\childdocmain|
\item
|\childredirect| renamed to |\childdocforward| and |\childdocforwardprefix|
and functionality expanded
\end{itemize}

%%%%%%%%%%%%%%%%%%%%%%%%%%%%%%%%%%%%%%%%
\paragraph{v1.0:} 2017/04/27

\begin{itemize}
\item
manual and install package
\item
first version published on CTAN
\end{itemize}

%%%%%%%%%%%%%%%%%%%%%%%%%%%%%%%%%%%%%%%%
\paragraph{v0.6:} 2017/04/26

\begin{itemize}
\item
redirection mechanism added
\end{itemize}

%%%%%%%%%%%%%%%%%%%%%%%%%%%%%%%%%%%%%%%%
\paragraph{v0.5:} 2017/04/26

\begin{itemize}
\item
functionality in definition file
\end{itemize}


%%%%%%%%%%%%%%%%%%%%%%%%%%%%%%%%%%%%%%%%%%%%%%%%%%%%%%%%%%%%%%%%%%%%%%%%%%%%%%%%
%%%%%%%%%%%%%%%%%%%%%%%%%%%%%%%%%%%%%%%%%%%%%%%%%%%%%%%%%%%%%%%%%%%%%%%%%%%%%%%%
%%%%%%%%%%%%%%%%%%%%%%%%%%%%%%%%%%%%%%%%%%%%%%%%%%%%%%%%%%%%%%%%%%%%%%%%%%%%%%%%
\appendix

\settowidth\MacroIndent{\rmfamily\scriptsize 000\ }

 \DocInput{childdoc.dtx}

\end{document}
%</driver>
% \fi
%
% %%%%%%%%%%%%%%%%%%%%%%%%%%%%%%%%%%%%%%%%%%%%%%%%%%%%%%%%%%%%%%%%%%%%%%%%%%%%%%
% %%%%%%%%%%%%%%%%%%%%%%%%%%%%%%%%%%%%%%%%%%%%%%%%%%%%%%%%%%%%%%%%%%%%%%%%%%%%%%
% \section{Sample}
%\iffalse
%<*samplemain>
%\fi
%
% The following presents a sample document
% with two chapters, two parts, a title page,
% a compile flag as well as three forwarding files to set the flag.
% It consists of eight |.tex| files:
% \begin{center}
% \begin{tabular}{ll}
% |cdocsamp.tex|&main file\\
% |cdocsch1.tex|&include file for chapter 1\\
% |cdocsch2.tex|&include file for chapter 2\\
% |cdocspt3.tex|&include file for part 3\\
% |cdocspt4.tex|&include file for part 4\\
% |cdocsdrf.tex|&forwarding file for main file in draft mode\\
% |cdocsfi1.tex|&forwarding file for final version of chapter 1\\
% |cdocsfi2.tex|&forwarding file for final version of chapter 2\\
% \end{tabular}
% \end{center}
% Each of the eight files can be compiled directly by the \LaTeX{} compiler.
%
% %%%%%%%%%%%%%%%%%%%%%%%%%%%%%%%%%%%%%%
% \paragraph{Main File.}
%
% The main file is called |cdocsamp.tex|.
%
% Load the \textsf{childdoc} definitions and
% declare the filename for the main document:
%    \begin{macrocode}
\input{childdoc.def}
\childdocmain{}
%    \end{macrocode}

% Optional override for |\version| flag:
%    \begin{macrocode}
%%\ifchilddoc\else\providecommand{\version}{draft}\fi
%    \end{macrocode}

% Define the default values for the |\version| flag
% (|final| for the main file and |draft| for childs):
%    \begin{macrocode}
\ifchilddoc
\providecommand{\version}{draft}
\else
\providecommand{\version}{final}
\fi
%    \end{macrocode}

% Load the standard document class:
%    \begin{macrocode}
\documentclass[12pt]{article}
%    \end{macrocode}

% Start the document body:
%    \begin{macrocode}
\begin{document}
%    \end{macrocode}

% Declare a title page.
% Print title, part of document being processed and version flag:
%    \begin{macrocode}
\addtocounter{page}{-1}
\begin{center}
{\LARGE\bfseries{}childdoc example\par}
\vspace{1cm}
\ifchilddoc
\ifchilddocmanual part\else chapter\fi:
`\childdocname' of `\childdocjob'\par
\else
main document: `\childdocjob'\par
\fi
version: \version\par
\end{center}
\newpage
%    \end{macrocode}

% Manually include selected file,
% otherwise process as usual:
%    \begin{macrocode}
\ifchilddocmanual
\section*{part `\childdocname'}
\input{\childdocname}
\else
%    \end{macrocode}

% Include the two chapters:
%    \begin{macrocode}
\include{cdocsch1}
\include{cdocsch2}
%    \end{macrocode}

% Include the two parts unless only chapters should be displayed:
%    \begin{macrocode}
\ifchilddoc\else
\section{part three}
\input{cdocspt3}
\section{part four}
\input{cdocspt4}
\fi
%    \end{macrocode}

% Process as usual until here:
%    \begin{macrocode}
\fi
%    \end{macrocode}

% End of document body:
%    \begin{macrocode}
\end{document}
%    \end{macrocode}
%\iffalse
%</samplemain>
%\fi
%
% %%%%%%%%%%%%%%%%%%%%%%%%%%%%%%%%%%%%%%
% \paragraph{Chapter Include Files.}
%
% The include files are called |cdocsch1.tex| and |cdocsch2.tex|.
%
%\iffalse
%<*samplechap1|samplechap2>
%\fi

% Optional override for |\version| flag:
%    \begin{macrocode}
%%\providecommand{\version}{final}
%    \end{macrocode}

% Include the main document:
%    \begin{macrocode}
\input{childdoc.def}
\childdocof{cdocsamp}
%    \end{macrocode}

%\iffalse
%</samplechap1|samplechap2>
%\fi
%
%\iffalse
%<*samplechap1>
%\fi
% Some text for chapter 1:
%    \begin{macrocode}
\section{one}
some text in chapter one
%    \end{macrocode}

%\iffalse
%</samplechap1>
%\fi
% Some text for chapter 2:
%\iffalse
%<*samplechap2>
%\fi
%    \begin{macrocode}
\section{two}
more text in chapter two
%    \end{macrocode}

%\iffalse
%</samplechap2>
%\fi
%
% %%%%%%%%%%%%%%%%%%%%%%%%%%%%%%%%%%%%%%
% \paragraph{Part Include Files.}
%
% The include files are called |cdocspt3.tex| and |cdocspt4.tex|.
%
%\iffalse
%<*samplepart3|samplepart4>
%\fi

% Optional override for |\version| flag:
%    \begin{macrocode}
%%\providecommand{\version}{final}
%    \end{macrocode}

% Include the main document:
%    \begin{macrocode}
\input{childdoc.def}
\childdocby{cdocsamp}
%    \end{macrocode}

%\iffalse
%</samplepart3|samplepart4>
%\fi
%
%\iffalse
%<*samplepart3>
%\fi
% Some text for part 3:
%    \begin{macrocode}
some text in part three
%    \end{macrocode}

%\iffalse
%</samplepart3>
%\fi
% Some text for part 4:
%\iffalse
%<*samplepart4>
%\fi
%    \begin{macrocode}
more text in part four
%    \end{macrocode}

%\iffalse
%</samplepart4>
%\fi
%
% %%%%%%%%%%%%%%%%%%%%%%%%%%%%%%%%%%%%%%
% \paragraph{Forwarding for a Complete Draft.}
%
% The following forwarding file |cdocsdrf.tex|
% compiles the main document in draft mode:
%\iffalse
%<*sampledraft>
%\fi
%    \begin{macrocode}
\def\version{draft}
\input{childdoc.def}
\childdocforward{cdocsamp}
%    \end{macrocode}

%\iffalse
%</sampledraft>
%\fi
%
% %%%%%%%%%%%%%%%%%%%%%%%%%%%%%%%%%%%%%%
% \paragraph{Forwarding for Final Version of the Chapters.}
%
% The following forwarding files |cdocsfn1.tex| and |cdocsfn2.tex|
% (with identical content)
% compile the final versions of the child documents
% |cdocsch1.tex| and |cdocsch2.tex|, respectively:
%\iffalse
%<*samplefinal>
%\fi
%    \begin{macrocode}
\def\version{final}
\input{childdoc.def}
\childdocforwardprefix[cdocsamp]{cdocsfn}{cdocsch}
%    \end{macrocode}

%\iffalse
%</samplefinal>
%\fi
%
% %%%%%%%%%%%%%%%%%%%%%%%%%%%%%%%%%%%%%%
% \paragraph{Command Line Processing.}
%
% The following three command lines generate the output files
% |cdocscld|, |cdocscl1| and |cdocscl2|
% which should be identical to
% |cdocsdrf|, |cdocsch1| and |cdocsfn2|, respectively:
% \begin{center}
% \begin{tabular}{l}
% |latex -jobname cdocscld \|\\
% |  "\def\version{draft}\input{childdoc.def}\childdocforward{cdocsamp}"|\\
% |latex -jobname cdocscl1 \|\\
% |  "\input{childdoc.def}\childdocforward[cdocsamp]{cdocsch1}"|\\
% |latex -jobname cdocscl2 \|\\
% |  "\def\version{final}\input{childdoc.def}\childdocforward{cdocsch2}"|
% \end{tabular}
% \end{center}
% Note that the trailing backslash on each first line
% merely continues the input to the second line
% (for convenient cut ant paste).
% Furthermore, the command |latex| can be replaced by any
% of its alternative versions such as |pdflatex|.
%
% %%%%%%%%%%%%%%%%%%%%%%%%%%%%%%%%%%%%%%%%%%%%%%%%%%%%%%%%%%%%%%%%%%%%%%%%%%%%%%
% %%%%%%%%%%%%%%%%%%%%%%%%%%%%%%%%%%%%%%%%%%%%%%%%%%%%%%%%%%%%%%%%%%%%%%%%%%%%%%
% \section{Implementation}
%\iffalse
%<*package>
%\fi
%
% This section describes the definitions file |childdoc.def|.

% The definitions cannot be loaded using |\usepackage| or |\RequirePackage|
% which has a mechanism to prevent loading a style file more than once.
% When loading the definitions by means of |\input|
% multiple instances have to be prevented manually:
%\iffalse
%This code needs to be before the `\ProvidesFile' directive
%which is defined at the beginning of this file.
%Therefore it is also placed there and commented out here.
%</package>
%<*discard>
%\fi
%    \begin{macrocode}
\ifdefined\childdocmain\endinput\fi
%    \end{macrocode}
%\iffalse
%</discard>
%<*package>
%\fi
%
% \macro{\ifchilddoc}
% \macro{\ifchilddocmanual}
% The conditional |\ifchilddoc| tells whether a
% child (true) or main (false) document is being compiled.
% The conditional |\ifchilddocmanual| tells whether
% the |\includeonly| mechanism is used (false) or
% the selection of child files must be performed manually (true).
% The definitions initialise to false:
%    \begin{macrocode}
\newif\ifchilddoc
\newif\ifchilddocmanual
%    \end{macrocode}

% \macro{\childdocname}
% \macro{\childdocjob}
% The macro |\childdocname| stores the name of the main document
% to be compiled. The macro |\childdocjob| stores the name of
% the document on which the \LaTeX{} compiler was originally invoked.
% The content of |\jobname| cannot be compared
% to filenames specified in the source due to different catcodes.
% The following code rescans |\jobname|, stores the result
% in |\childdocname| and saves a copy in |\childdocjob|:
%    \begin{macrocode}
\edef\childdocname{\scantokens\expandafter{\jobname\noexpand}}
\let\childdocjob\childdocname
%    \end{macrocode}

% \macro{\childdocdisable}
% The macro |\childdocdisable| prevents the main file
% from being processed more than once.
% At this stage, the main document command |\childdocmain|
% is assumed to be called once again where it should do nothing.
% Any subsequent call to it should prevent
% a secondary processing of the main document
% It overwrites the forwarding commands
% |\childdocof| and |\childdocforward|
% with empty macros to prevent further inclusions of the main document:
%    \begin{macrocode}
\newcommand{\childdocdisable}
{
  \renewcommand{\childdocmain}[1]{\renewcommand{\childdocmain}[1]{\endinput}}
  \renewcommand{\childdocof}[1]{}
  \renewcommand{\childdocby}[2][]{}
  \renewcommand{\childdocforward}[2][]{}
  \renewcommand{\childdocdisable}{}
}
%    \end{macrocode}

% \macro{\childdocmain}
% The macro |\childdocmain| is to be called at the top of the main file
% with nothing or the main filename (without extension) as argument.
% First, it breaks loops.
% If the argument is not empty and does not match |\childdocname|
% (which is set by the first inclusion of |childdoc.def|),
% |\ifchilddoc| is set to true, |\includeonly| is applied to the child file
% and |\jobname| is set to the main file
% (for proper handling of |.aux| files):
%    \begin{macrocode}
\newcommand{\childdocmain}[1]
{
  \childdocdisable\childdocmain{}
  \if?#1?\else
    \begingroup
      \def\childdoctmp{#1}
      \ifx\childdoctmp\childdocname
        \def\childdoctmp{}
      \else
        \def\childdoctmp
        {
          \childdoctrue
          \includeonly{\childdocname}
          \def\childdocjob{#1}
          \def\jobname{#1}
        }
      \fi
      \expandafter
    \endgroup
    \childdoctmp
  \fi
}
%    \end{macrocode}

% \macro{\childdocof}
% The command |\childdocof| redirects
% compilation to the main file |#1|.
%    \begin{macrocode}
\newcommand{\childdocof}[1]
{
  \childdocdisable
  \childdoctrue
  \includeonly{\childdocname}
  \def\jobname{#1}
  \def\childdocjob{#1}
  \input{#1}
}
%    \end{macrocode}

% \macro{\childdocby}
% The command |\childdocby| ....
%    \begin{macrocode}
\newcommand{\childdocby}[2][]
{
  \childdocdisable
  \childdoctrue
  \childdocmanualtrue
  \if?#1?\else
    \def\jobname{#2}
  \fi
  \def\childdocjob{#2}
  \input{#2}
  \endinput
}
%    \end{macrocode}

% \macro{\childdocforward}
% The command |\childdocforward| redirects
% compilation to the main file or
% (if the optional argument is given) a child file.
% Parameters are set as if the main file
% or a child file starting with |\childdocof| was compiled.
% Then compilation is handed over to the main file:
%    \begin{macrocode}
\newcommand{\childdocforward}[2][]
{
  \begingroup
    \if?#1?
      \def\childdoctmp
      {
        \def\childdocname{#2}
        \def\childdocjob{#2}
        \def\jobname{#2}
        \input{#2}
        \endinput
      }
    \else
      \def\childdoctmp
      {
        \childdocdisable
        \def\childdocname{#2}
        \childdoctrue
        \includeonly{#2}
        \def\childdocjob{#1}
        \def\jobname{#1}
        \input{#1}
        \endinput
      }
    \fi
    \expandafter
  \endgroup
  \childdoctmp
}
%    \end{macrocode}

% \macro{\childdocforwardprefix}
% The command |\childdocforwardprefix| redirects
% compilation to the main or a child file by means of a pattern.
% The prefix |#1| in the current filename is replaced by |#2|
% and the suffix of the current filename is kept
% (it is assumed that the filename does not contain the substring `|~~~|'
% which is used as a delimiter).
% Compilation is handed over to the new file by |\childdocforward|:
%    \begin{macrocode}
\newcommand{\childdocforwardprefix}[3][]
{
  \begingroup
    \def\childdocextract #2##1~~~{\def\childdoctmp{\childdocforward[#1]{#3##1}}}
    \expandafter\childdocextract\childdocname~~~
    \expandafter
  \endgroup
  \childdoctmp
}
%    \end{macrocode}

% \macro{\childdoc}
% The deprecated macro |\childdoc| is a legacy version of |\childdocmain|:
%    \begin{macrocode}
\newcommand{\childdoc}{\childdocmain}
%    \end{macrocode}

% \macro{\childdocredirect}
% The deprecated macro |\childdocredirect| is a legacy version
% of |\childdocforward| and |\childdocforwardprefix|:
%    \begin{macrocode}
\newcommand{\childdocredirect}[2][]
{
  \begingroup
    \if?#1?
      \def\childdoctmp{\childdocforward{#2}}
    \else
      \def\childdoctmp{\childdocforwardprefix{#1}{#2}}
    \fi
    \expandafter
  \endgroup
  \childdoctmp
}
%    \end{macrocode}

%\iffalse
%</package>
%\fi
%
\endinput
|\\
|\childdocby{|\textit{main}|}|\\
\end{tabular}
\end{center}
%
Both forms have slightly different effects as described above.
The main file is prepared as usual, see \secref{sec:include}.

%%%%%%%%%%%%%%%%%%%%%%%%%%%%%%%%%%%%%%%%%%%%%%%%%%%%%%%%%%%%%%%%%%%%%%%%%%%%%%%%
\subsection{Legacy Detection}
\label{sec:detection}

The directive |\childdocmain| in the main file can detect
whether the complete document or merely a child is to be compiled
even without using the directive |\childdocof|.
This method is deprecated because it is less robust
and there is no compelling reason to use it;
it is merely provided for backward compatibility
and it may be removed in future versions.

If the detection mechanism is to be used,
it is mandatory to correctly specify
the filename of the main file as the argument of |\childdocmain|:
%
\begin{center}
\begin{tabular}{l}
|% \iffalse
%
% childdoc.dtx Copyright (C) 2017-2018 Niklas Beisert
%
% This work may be distributed and/or modified under the
% conditions of the LaTeX Project Public License, either version 1.3
% of this license or (at your option) any later version.
% The latest version of this license is in
%   http://www.latex-project.org/lppl.txt
% and version 1.3 or later is part of all distributions of LaTeX
% version 2005/12/01 or later.
%
% This work has the LPPL maintenance status `maintained'.
%
% The Current Maintainer of this work is Niklas Beisert.
%
% This work consists of the files childdoc.dtx and childdoc.ins
% and the derived files childdoc.def and cdocsamp.tex with
% cdocsch1.tex, cdocsch2.tex, cdocsdrf.tex, cdocsfn1.tex, cdocsfn2.tex.
%
%<package>\ifdefined\childdocmain\endinput\fi
%<package>\ProvidesFile{childdoc.def}[2018/12/30 v2.0 child document driver]
%<samplemain>\ProvidesFile{cdocsamp.tex}[2018/12/30 v2.0 sample for childdoc]
%<*driver>
%\ProvidesFile{childdoc.drv}[2018/12/30 v2.0 childdoc reference manual file]
\PassOptionsToClass{10pt,a4paper}{article}
\documentclass{ltxdoc}

\usepackage[margin=35mm]{geometry}
\usepackage{hyperref}
\usepackage{hyperxmp}
\usepackage[usenames]{color}

\hypersetup{colorlinks=true}
\hypersetup{pdfstartview=FitH}
\hypersetup{pdfpagemode=UseNone}
\hypersetup{pdfsource={}}
\hypersetup{pdflang={en-UK}}
\hypersetup{pdfcopyright={Copyright 2017-2018 Niklas Beisert.
  This work may be distributed and/or modified under the
  conditions of the LaTeX Project Public License, either version 1.3
  of this license or (at your option) any later version.}}
\hypersetup{pdflicenseurl={http://www.latex-project.org/lppl.txt}}
\hypersetup{pdfcontactaddress={ETH Zurich, ITP, HIT K,
  Wolfgang-Pauli-Strasse 27}}
\hypersetup{pdfcontactpostcode={8093}}
\hypersetup{pdfcontactcity={Zurich}}
\hypersetup{pdfcontactcountry={Switzerland}}
\hypersetup{pdfcontactemail={nbeisert@itp.phys.ethz.ch}}
\hypersetup{pdfcontacturl={http://people.phys.ethz.ch/\xmptilde nbeisert/}}

\newcommand{\secref}[1]{\hyperref[#1]{section \ref*{#1}}}

\parskip1ex
\parindent0pt
\let\olditemize\itemize
\def\itemize{\olditemize\parskip0pt}

\begin{document}

\title{The \textsf{childdoc} Package}
\hypersetup{pdftitle={The childdoc Package}}
\author{Niklas Beisert\\[2ex]
  Institut f\"ur Theoretische Physik\\
  Eidgen\"ossische Technische Hochschule Z\"urich\\
  Wolfgang-Pauli-Strasse 27, 8093 Z\"urich, Switzerland\\[1ex]
  \href{mailto:nbeisert@itp.phys.ethz.ch}
  {\texttt{nbeisert@itp.phys.ethz.ch}}}
\hypersetup{pdfauthor={Niklas Beisert}}
\hypersetup{pdfsubject={Manual for the LaTeX2e Package childdoc}}
\date{30 December 2018, \textsf{v2.0}}
\maketitle

\begin{abstract}\noindent
\textsf{childdoc} is a \LaTeXe{} package
that enables the direct compilation
of document sections included by |\include|
to individual files.
\end{abstract}

\begingroup
\parskip0ex
\tableofcontents
\endgroup

%%%%%%%%%%%%%%%%%%%%%%%%%%%%%%%%%%%%%%%%%%%%%%%%%%%%%%%%%%%%%%%%%%%%%%%%%%%%%%%%
%%%%%%%%%%%%%%%%%%%%%%%%%%%%%%%%%%%%%%%%%%%%%%%%%%%%%%%%%%%%%%%%%%%%%%%%%%%%%%%%
\section{Introduction}

\LaTeX{} provides a mechanism to structure a large document (such as a book)
into a main file and several child files (containing the chapters)
using the |\include| command.
This mechanism is beneficial for documents
which span hundreds of pages in order to
make the source file(s) more manageable.
Moreover, compilation can be restricted to
selected child files by means of the |\includeonly| command.
The latter feature can be used to reduce the compilation time while editing
(this was significantly more useful in the earlier days of \LaTeX{})
or to generate a smaller document which is easier to navigate.
Another application of |\includeonly| is to generate
documents consisting of selected parts of the complete document.

However, there are a few drawbacks of the plain |\include| mechanism:
\begin{itemize}
\item
The child files cannot be compiled on their own,
they can only be compiled via the main file.
A naive editing environment
(such as a text editor with an option
to have the current file processed by \LaTeX)
may require one to switch to the main file before compiling;
attempting to compile the child file produces errors.
\item
The main file must be modified (each time)
to adjust the |\includeonly| command
to the present needs. This easily leaves the main file in a messy state.
\item
The generated document will always carry the filename
of the main document. This is inconvenient if
several child files are to be compiled and
to be kept for distribution.
\end{itemize}

The present package provides a simple interface
to make child files individually compilable by \LaTeX{}.
Compiling a child file then has the same effect as compiling
the main file with an |\includeonly| command
to select the appropriate child.
Moreover the generated document will carry the name of the child
rather than the main file.
This resolves all three above issues.

This feature is meant to make the editing of books,
thesis documents and lecture notes somewhat more convenient.
However, the package can also be used efficiently for
composing a series of documents (such as exercise sheets)
which are typically distributed individually.
It then assists the author in generating the individual documents
(potentially in different versions)
as well as a document containing the collected series.
Another application is in developing style files
or other kinds of included material
where compilation of the style file could redirect
to a sample or test file.

%%%%%%%%%%%%%%%%%%%%%%%%%%%%%%%%%%%%%%%%%%%%%%%%%%%%%%%%%%%%%%%%%%%%%%%%%%%%%%%%
%%%%%%%%%%%%%%%%%%%%%%%%%%%%%%%%%%%%%%%%%%%%%%%%%%%%%%%%%%%%%%%%%%%%%%%%%%%%%%%%
\section{Usage}

First of all, the package \textsf{childdoc} is \emph{not} a standard
\LaTeXe{} |.sty| style file! Therefore it needs to be invoked in
a non-standard way.

%%%%%%%%%%%%%%%%%%%%%%%%%%%%%%%%%%%%%%%%%%%%%%%%%%%%%%%%%%%%%%%%%%%%%%%%%%%%%%%%
\subsection{Included Files}
\label{sec:include}

%%%%%%%%%%%%%%%%%%%%%%%%%%%%%%%%%%%%%%%%
\DescribeMacro{\childdocmain}
To use the package, add the commands
\begin{center}
\begin{tabular}{l}
|\input{childdoc.def}|\\
|\childdocmain{}|\\
\end{tabular}
\end{center}
at the very top of the main \LaTeX{} file,
in particular \emph{before} the |\documentclass| statement!
The argument of |\childdocmain| should be left empty
(but it must be present).

%%%%%%%%%%%%%%%%%%%%%%%%%%%%%%%%%%%%%%%%
\DescribeMacro{\childdocof}
Furthermore, add the commands
\begin{center}
\begin{tabular}{l}
|\input{childdoc.def}|\\
|\childdocof{|\textit{main}|}|\\
\end{tabular}
\end{center}
at the top of every child file \textit{child}
which is included by |\include{|\textit{child}|}|
from within the main file
(or at least for those files to be compiled individually).
The argument \textit{main} must be the filename of the main file.

There are a couple of
considerations in setting up the main and child documents:

%%%%%%%%%%%%%%%%%%%%%%%%%%%%%%%%%%%%%%%%
\paragraph{Restrictions.}

Please note the following restrictions:
\begin{itemize}
\item
|\childdocmain| must be called with one argument \textit{main}
to ensure compatibility with earlier version of the package.
It must either be empty (|\childdocmain{}|)
or precisely match the filename of the main file in which it is specified.
See \secref{sec:detection} for further information.
\item
The filename \textit{main} must be specified without the |.tex| extension.
\item
The filename \textit{main} is case sensitive
(even in case-insensitive file systems)
due to internal string comparison.
\item
The argument \textit{main} should be fully expanded, it cannot be a macro.
\item
Subdirectories and special characters should be avoided in filenames.
\item
The command |\childdocmain{|\textit{main}|}| must be followed by a whitespace.
It should not be followed immediately by another command
or by a comment mark `|%|'.
This is because the \TeX{} parser reads the token immediately following
the argument of |\childdocmain| and puts it
at the beginning of every child section;
however, a white\-space is ignored.
\end{itemize}

%%%%%%%%%%%%%%%%%%%%%%%%%%%%%%%%%%%%%%%%
\paragraph{Content of Main File.}

It is advisable to place all content in the child files included by |\include|.
Any output contained in the main file will appear in all child documents
unless suppressed manually;
it cannot be suppressed automatically by the |\includeonly| directive
and thus should normally be avoided.
A method to include some content in the main file
by means of conditional processing is described in \secref{sec:conditional}.

%%%%%%%%%%%%%%%%%%%%%%%%%%%%%%%%%%%%%%%%
\paragraph{Page Numbering.}

When only a part of the document is compiled,
the appropriate numbering of pages
(as well as other status parameters)
is determined from the |.aux| files.
The latter contain information from previous passes.
However this information needs to propagate through
all intermediate child documents.
Therefore the page numbering in child documents may well
be inconsistent until the complete document is compiled at least once.

A useful (if unconventional) way to always ensure a consistent
page numbering is to restart the numbering in each child document
and denote the pages by `\textit{child}|.|\textit{page}'
where \textit{child} represents the chapter/section number of the child file.
This can be achieved by the command
|\numberwithin{page}{|\textit{child}|}|
of the \textsf{amsmath} package
where \textit{child} can be |chapter| or |section|
depending on the chosen structuring.
Alternatively, one can modify the macro |\thepage| appropriately
and reset the counter |page| at the start of each child file.

%%%%%%%%%%%%%%%%%%%%%%%%%%%%%%%%%%%%%%%%%%%%%%%%%%%%%%%%%%%%%%%%%%%%%%%%%%%%%%%%
\subsection{Conditional Processing}
\label{sec:conditional}

The package provides a mechanism to compile different versions
of a document. To customise the versions further some conditional processing
can come in handy to distinguish which version is being compiled.
The package provides two macros to describe the compilation context:

%%%%%%%%%%%%%%%%%%%%%%%%%%%%%%%%%%%%%%%%
\DescribeMacro{\ifchilddoc}
The conditional |\ifchilddoc| distinguishes between the compilation of
child documents and the main document:
%
\begin{center}
|\ifchilddoc |\textit{child-code}| |[|\||else |\textit{main-code}]| \||fi|
\end{center}

%%%%%%%%%%%%%%%%%%%%%%%%%%%%%%%%%%%%%%%%
\DescribeMacro{\childdocname}
\DescribeMacro{\childdocjob}
The macro |\childdocname| contains the filename (without extension)
of the main or child file being processed.
Note that |\childdocjob| will always contain the name of the main file.

%%%%%%%%%%%%%%%%%%%%%%%%%%%%%%%%%%%%%%%%
\paragraph{Title Page.}

Conditional processing can be used to include a title or banner page
in the main document when proper precautions are taken.
Importantly, the code in the main file should ensure that the page counter
(as well as other status parameters which are stored in the |.aux| files)
takes the same value after the conditional processing.
Otherwise the page numbers may take divergent values
depending on which part is compiled.

For example, a title page could be declared by:
%
\begin{center}
\begin{tabular}{l}
|\ifchilddoc\||else|\\
|\addtocounter{page}{-1}|\\
\textit{code for title page}\\
|\newpage|\\
|\||fi|
\end{tabular}
\end{center}
%
A banner page for the child documents can be generated by:
%
\begin{center}
\begin{tabular}{l}
|\ifchilddoc|\\
|\addtocounter{page}{-1}|\\
\textit{code for banner page}\\
|\newpage|\\
|\||fi|
\end{tabular}
\end{center}
%
Here one could write a message such as:
\begin{center}
|This is the part \childdocname{} of \childdocjob{}.|
\end{center}

%%%%%%%%%%%%%%%%%%%%%%%%%%%%%%%%%%%%%%%%%%%%%%%%%%%%%%%%%%%%%%%%%%%%%%%%%%%%%%%%
\subsection{Flags}
\label{sec:flags}

The package makes it easy to generate different versions
of the main or child documents.
To this end compilation flags can be defined
and assigned different default values.
They will be particularly useful in conjunction
with the forwarding mechanism described in \secref{sec:forward}.

For example, it may be useful to have a flag |\version|
which can be set to |draft| or |final|.
The document source will contain some conditional code
depending on the value of |\version|.
Suppose further, the flag should default to |final| for the main file
and to |draft| for child files
which is a natural assignment for editing the document.
This is achieved by placing the following code
in the preamble of the main document
(below the |\childdocmain| directive):
%
\begin{center}
\begin{tabular}{l}
|\ifchilddoc|\\
|\providecommand{\version}{draft}|\\
|\||else|\\
|\providecommand{\version}{final}|\\
|\||fi|
\end{tabular}
\end{center}
%
The definition by |\providecommand| makes sure
that previous definitions are not overwritten.
Further statements |\providecommand{\version}{...}|
can thus be added before the above code to override it.

For the main file, one might add a line
(between |\childdocmain| and the above block)
%
\begin{center}
|%\ifchilddoc\||else\providecommand{\version}{draft}\||fi|
\end{center}
%
which can be uncommented to produce a draft version.
Likewise one can add a line to the very top of a child file
(above the |\childdocof{|\textit{main}|}| directive)
%
\begin{center}
|%\providecommand{\version}{final}|
\end{center}
%
which can be uncommented to produce the final version of this child document.

%%%%%%%%%%%%%%%%%%%%%%%%%%%%%%%%%%%%%%%%%%%%%%%%%%%%%%%%%%%%%%%%%%%%%%%%%%%%%%%%
\subsection{Forwarding}
\label{sec:forward}

Different versions of the main or child documents
using compilation flags as described in \secref{sec:flags}
can be (permanently) stored in different files
for convenient compilation, viewing and distribution.
To this end, the package defines a command
to pass on compilation to a different file:

%%%%%%%%%%%%%%%%%%%%%%%%%%%%%%%%%%%%%%%%
\DescribeMacro{\childdocforward}
The command |\childdocforward| redirects processing to
another source file:
%
\begin{center}
\begin{tabular}{l}
|\input{childdoc.def}|\\
|\childdocforward[|\textit{main}|]{|\textit{dest}|}|\\
\end{tabular}
\end{center}
%
The argument \textit{dest} is the destination file
(without extension).
It should be the main file or one of the child files.
Note that further \textsf{childdoc} directives
such as |\childdocof| and |\childdocforward|
in the indicated file will be processed in this form.
The optional argument \textit{main}
passes on directly to the main file \textit{main}
while pretending to compile the child \textit{dest}.
This form behaves as if \textit{dest}
issues |\childdocof{|\textit{main}|}| right away,
and no further \textsf{childdoc} directives will be processed.

%%%%%%%%%%%%%%%%%%%%%%%%%%%%%%%%%%%%%%%%
\DescribeMacro{\...prefix}
In the alternative form |\childdocforwardprefix|,
%
\begin{center}
\begin{tabular}{l}
|\input{childdoc.def}|\\
|\childdocforwardprefix[|\textit{main}|]{|\textit{prefix}|}{|\textit{dest}|}|
\end{tabular}
\end{center}
%
the destination file is determined by a pattern
depending on the current file:
To make this work, the current file must be called
`{\textit{prefix}\hspace{0.2em}\textit{suffix}}'
with \textit{prefix} matching precisely the argument.
Processing is then passed on to the file
`{\textit{dest}\hspace{0.2em}\textit{suffix}}'.
Surely, the same effect is achieved by
directly specifying the
argument `{\textit{dest}\hspace{0.2em}\textit{suffix}}'
in the first form.
However, that requires to set up a different file
for each child. With the alternative form of the command
all these files can have exactly the same content
which simplifies setting them up and maintaining them.

For example, the following file |draft.tex|
with a compilation flag |\version| as described in \secref{sec:flags}
compiles the main document as a draft:
%
\begin{center}
\begin{tabular}{l}
|\def\version{draft}|\\
|\input{childdoc.def}|\\
|\childdocforward{|\textit{main}|}|
\end{tabular}
\end{center}
%
Likewise, the following files |final|\textit{nn}|.tex|
compile the final version of the child document
|child|\textit{nn}|.tex|:
%
\begin{center}
\begin{tabular}{l}
|\def\version{final}|\\
|\input{childdoc.def}|\\
|\childdocforwardprefix{final}{child}|
\end{tabular}
\end{center}
%

Note that when several versions of a main file and/or of each child file
are to be generated, it may be convenient to set up a |Makefile| or
shell script to automatise the process.

%%%%%%%%%%%%%%%%%%%%%%%%%%%%%%%%%%%%%%%%%%%%%%%%%%%%%%%%%%%%%%%%%%%%%%%%%%%%%%%%
\subsection{Command Line Processing}
\label{sec:commandline}

The effect of redirection files can also be achieved by invoking
the \LaTeX{} compiler with a more elaborate command line.
Most conveniently this should be done as part
of a shell script or a |Makefile|.

When using \textsf{childdoc} in the main file, the following
command lines effectively perform a redirection
(note that depending on the shell being used,
backslashes may have to be doubled: `|\|' $\to$ `|\\|'):
%
\begin{center}
|... -jobname "|\textit{target}|" |\\|"|[\textit{flags}]%
|\input{childdoc.def}\childdocforward[|\textit{main}|]{|\textit{dest}|}"|
\end{center}
%
Here \textit{target} is the name of the output file,
\textit{main} is the name of the main file
and \textit{dest} is the name of the main or child file to be processed
(all filenames without extensions).
The optional argument \textit{main} can be omitted
if \textit{main} matches \textit{dest}.
Optionally, compilation \textit{flags} can be defined via |\def| commands.
This command line makes the \TeX{} engine believe
it is compiling the file \textit{target}
whose content is specified as the latter parameter.
The provided code then forwards the processing to
\textit{main} or \textit{dest} as described in \secref{sec:forward}.

%%%%%%%%%%%%%%%%%%%%%%%%%%%%%%%%%%%%%%%%%%%%%%%%%%%%%%%%%%%%%%%%%%%%%%%%%%%%%%%%
\subsection{Include by Input}
\label{sec:input}

Including child documents by |\include| has some restrictions by design.
Most notably, the content of a child document always occupies
its own set of pages; pages cannot be shared between child documents.
Usually, this behaviour makes perfect sense
because each child document contain an essential part of the document.
However, in some situations it may be desirable to compose
a document from a collection of parts
without having mandatory page breaks between then.
For this case, the package
provides a mechanism to include parts
by |\input| which can also be processed individually.
However, by construction this mechanism
requires manual handling of the content to be output.

%%%%%%%%%%%%%%%%%%%%%%%%%%%%%%%%%%%%%%%%
\DescribeMacro{\ifchilddocmanual}
The main file should be prepared as usual, see \secref{sec:include}.
However, the document body must make a distinction
between processing of an individual part and of the main document, e.g.:
%
\begin{center}
\begin{tabular}{l}
|\ifchilddocmanual|\\
|\input{\childdocname}|\\
|\||else|\\
\textit{document body with }|\input{|\textit{part}|}|\\
|\||fi|
\end{tabular}
\end{center}
%
The conditional |\ifchilddocmanual| is true whenever
a part to be included by |\input| is being compiled,
and the name of the part is stored in |\childdocname|.

%%%%%%%%%%%%%%%%%%%%%%%%%%%%%%%%%%%%%%%%
\DescribeMacro{\childdocby}
Each part to be included by |\input| should start with:
%
\begin{center}
\begin{tabular}{l}
|\input{childdoc.def}|\\
|\childdocby{|\textit{main}|}|\\
\end{tabular}
\end{center}
%
The directive |\childdocby| is similar to |\childdocof|
described in \secref{sec:include},
but the subsequent selection of content must be done manually.
To that end, both |\ifchilddoc| and |\ifchilddocmanual|
will be true upon processing of a part,
and the name of the part is stored in |\childdocname|.
Note that |\jobname| will be set to the filename of the current part
so that each part receives an individual |.aux| file
that does not interfere with the |.aux| file(s) of the main document.
This behaviour can be altered by the alternative form
|\childdocby[*]{|\textit{main}|}| (with a non-empty optional argument)
which uses the |.aux| file of the main document
by setting |\jobname| to \textit{main}.

%%%%%%%%%%%%%%%%%%%%%%%%%%%%%%%%%%%%%%%%%%%%%%%%%%%%%%%%%%%%%%%%%%%%%%%%%%%%%%%%
\subsection{Driver Development}
\label{sec:driver}

The \textsf{childdoc} mechanism can also be use for the development
of definition files such as \LaTeX{} styles or classes.
This case differs from the above setup with multiple parts
included by |\include| in that no |\includeonly| should be invoked.
This can be achieved by starting the include file
(before |\ProvidesPackage|) with:
%
\begin{center}
\begin{tabular}{l}
|\input{childdoc.def}|\\
|\childdocforward{|\textit{main}|}|\\
\end{tabular}
\end{center}
%
or alternatively with:
%
\begin{center}
\begin{tabular}{l}
|\input{childdoc.def}|\\
|\childdocby{|\textit{main}|}|\\
\end{tabular}
\end{center}
%
Both forms have slightly different effects as described above.
The main file is prepared as usual, see \secref{sec:include}.

%%%%%%%%%%%%%%%%%%%%%%%%%%%%%%%%%%%%%%%%%%%%%%%%%%%%%%%%%%%%%%%%%%%%%%%%%%%%%%%%
\subsection{Legacy Detection}
\label{sec:detection}

The directive |\childdocmain| in the main file can detect
whether the complete document or merely a child is to be compiled
even without using the directive |\childdocof|.
This method is deprecated because it is less robust
and there is no compelling reason to use it;
it is merely provided for backward compatibility
and it may be removed in future versions.

If the detection mechanism is to be used,
it is mandatory to correctly specify
the filename of the main file as the argument of |\childdocmain|:
%
\begin{center}
\begin{tabular}{l}
|\input{childdoc.def}|\\
|\childdocmain{|\textit{main}|}|\\
\end{tabular}
\end{center}
%
If |\jobname| does not match the argument \textit{main} of |\childdocmain|,
it is assumed that |\jobname| points to the child file to be compiled.
When using |\childdocmain| with the main file specified as argument,
it suffices to start a child file
with just |\input{|\textit{main}|}|
without loading of the package and using |\childdocof|.
If instead all processing is done
with the appropriate \textsf{childdoc} directives,
the argument of \textit{main} of |\childdocmain| can be empty.

An alternative version of the command line processing described
in \secref{sec:commandline} using the detection mechanism reads:
%
\begin{center}
|... -jobname "|\textit{target}|" "|[\textit{flags}]%
[|\def\jobname{|\textit{dest}|}|]|\input{|\textit{main}|}"|
\end{center}

%%%%%%%%%%%%%%%%%%%%%%%%%%%%%%%%%%%%%%%%%%%%%%%%%%%%%%%%%%%%%%%%%%%%%%%%%%%%%%%%
\subsection{Manual Code}
\label{sec:manual}

In case one cannot be certain whether the definitions file |childdoc.def|
is installed on the target \TeX{} distribution
and one prefers not to ship it,
it is conceivable to paste a few relevant commands into the sources.

To that end, drop all statements |\input{childdoc.def}|
and perform the replacements as outlined below.
Instead of |\childdocmain{|\textit{main}|}| add the following code
to the top of the main file:
%
\begin{center}
\begin{tabular}{l}
|\||ifdefined\childdocname\endinput\||fi\newif\ifchilddoc|\\
|\edef\childdocname{\scantokens\expandafter{\jobname\noexpand}}|\\
|\def\childdocmain{|\textit{main}|}\||ifx\childdocmain\childdocname\||else|\\
|\childdoctrue\includeonly{\childdocname}\let\jobname\childdocmain\||fi|\\
\end{tabular}
\end{center}
%
Instead of |\childdocof{|\textit{main}|}| just include the main file
at the top of each child file:
%
\begin{center}
|\input{|\textit{main}|}|
\end{center}
%
A simple redirection |\childdocforward{|\textit{dest}|}| is achieved by:
%
\begin{center}
|\def\jobname{|\textit{dest}|}\input{\jobname}|
\end{center}
%
The redirection with prefix
|\childdocforwardprefix[|\textit{prefix}|]{|\textit{dest}|}|
is accomplished by:
%
\begin{center}
\begin{tabular}{l}
|{\edef\jobname{\scantokens\expandafter{\jobname\noexpand}}|\\
|\def\redirectjob |\textit{prefix}|#1~~~{\gdef\jobname{|\textit{dest}|#1}}|\\
|\expandafter\redirectjob\jobname~~~}\input{\jobname}|
\end{tabular}
\end{center}

In an alternative approach,
child documents can be compiled by a specific command line
without additional code or specific definitions:
%
\begin{center}
|... -jobname "|\textit{target}|" "|[\textit{flags}]%
|\includeonly{|\textit{dest}|}\input{|\textit{main}|}"|
\end{center}
%

%%%%%%%%%%%%%%%%%%%%%%%%%%%%%%%%%%%%%%%%%%%%%%%%%%%%%%%%%%%%%%%%%%%%%%%%%%%%%%%%
%%%%%%%%%%%%%%%%%%%%%%%%%%%%%%%%%%%%%%%%%%%%%%%%%%%%%%%%%%%%%%%%%%%%%%%%%%%%%%%%
\section{Information}

%%%%%%%%%%%%%%%%%%%%%%%%%%%%%%%%%%%%%%%%%%%%%%%%%%%%%%%%%%%%%%%%%%%%%%%%%%%%%%%%
\subsection{Copyright}

Copyright \copyright{} 2017--2018 Niklas Beisert

This work may be distributed and/or modified under the
conditions of the \LaTeX{} Project Public License, either version 1.3
of this license or (at your option) any later version.
The latest version of this license is in
  \url{http://www.latex-project.org/lppl.txt}
and version 1.3 or later is part of all distributions of \LaTeX{}
version 2005/12/01 or later.

This work has the LPPL maintenance status `maintained'.

The Current Maintainer of this work is Niklas Beisert.

This work consists of the files |README.txt|, |childdoc.ins| and |childdoc.dtx|
as well as the derived files |childdoc.def|, |cdocsamp.tex|
with |cdocsch1.tex|, |cdocsch2.tex|, |cdocspt3.tex|, |cdocspt4.tex|,
|cdocsdrf.tex|, |cdocsfn1.tex|, |cdocsfn2.tex|
as well as |childdoc.pdf|.

%%%%%%%%%%%%%%%%%%%%%%%%%%%%%%%%%%%%%%%%%%%%%%%%%%%%%%%%%%%%%%%%%%%%%%%%%%%%%%%%
\subsection{Files and Installation}

The package consists of the files:
%
\begin{center}
\begin{tabular}{ll}
    |README.txt|   & readme file \\
    |childdoc.ins| & installation file \\
    |childdoc.dtx| & source file \\
    |childdoc.def| & definition file \\
    |cdocsamp.tex| & sample main file \\
    |cdocsch1.tex| & sample include file \\
    |cdocsch2.tex| & sample include file \\
    |cdocspt3.tex| & sample part file \\
    |cdocspt4.tex| & sample part file \\
    |cdocsdrf.tex| & sample redirection file \\
    |cdocsfn1.tex| & sample redirection file \\
    |cdocsfn2.tex| & sample redirection file \\
    |childdoc.pdf| & manual
\end{tabular}
\end{center}
%
The distribution consists of the files
|README.txt|, |childdoc.ins| and |childdoc.dtx|.
%
\begin{itemize}
\item
Run (pdf)\LaTeX{} on |childdoc.dtx|
to compile the manual |childdoc.pdf| (this file).
\item
Run \LaTeX{} on |childdoc.ins| to create the definitions file |childdoc.def|
and the sample |cdocsamp.tex| with include files
|cdocsch1.tex|, |cdocsch2.tex|, |cdocspt3.tex|, |cdocspt4.tex|,
|cdocsdrf.tex|, |cdocsfn1.tex|, |cdocsfn2.tex|.
Then copy the file |childdoc.def| to an appropriate directory of your \LaTeX{}
distribution, e.g.\ \textit{texmf-root}|/tex/latex/childdoc|.
\end{itemize}

%%%%%%%%%%%%%%%%%%%%%%%%%%%%%%%%%%%%%%%%%%%%%%%%%%%%%%%%%%%%%%%%%%%%%%%%%%%%%%%%
\subsection{Related CTAN Packages}

There are several other packages which offer a similar functionality:
%
\begin{itemize}
\item
The packages
\href{http://ctan.org/pkg/docmute}{\textsf{docmute}},
\href{http://ctan.org/pkg/includex}{\textsf{includex}} and
\href{http://ctan.org/pkg/standalone}{\textsf{standalone}}
provide commands to include only the document body of
a child file thus allowing both files to be compiled individually.
\item
The packages \href{http://ctan.org/pkg/subdocs}{\textsf{subdocs}}
and \href{http://ctan.org/pkg/subfiles}{\textsf{subfiles}}
provide structures in which the main and child documents can be
encapsulated and allowing them to be compiled individually.
The inclusion mechanism is different from the conventional |\include|.
\item
The package \href{http://ctan.org/pkg/combine}{\textsf{combine}}
is an elaborate solution to combine several documents into one.
\end{itemize}
%
See also the CTAN topic \href{http://ctan.org/topic/subdocs}{\textsf{subdocs}}
for further related packages.
The present package differs from the above solutions in that
a document structure constructed with the conventional |\include| mechanism
just needs two extra commands at the top of every file
such that all constituent files can be compiled individually.

%%%%%%%%%%%%%%%%%%%%%%%%%%%%%%%%%%%%%%%%%%%%%%%%%%%%%%%%%%%%%%%%%%%%%%%%%%%%%%%%
%\subsection{Feature Suggestions}
%
%The following is a list of features which may be useful for future
%versions of this package:
%%
%\begin{itemize}
%\item
%\ldots
%\end{itemize}

%%%%%%%%%%%%%%%%%%%%%%%%%%%%%%%%%%%%%%%%%%%%%%%%%%%%%%%%%%%%%%%%%%%%%%%%%%%%%%%%
\subsection{Revision History}

%%%%%%%%%%%%%%%%%%%%%%%%%%%%%%%%%%%%%%%%
\paragraph{v2.0:} 2018/12/30

\begin{itemize}
\item
immediate forward processing
\item
added |\childdocby| mechanism
\item
manual restructured
\end{itemize}

%%%%%%%%%%%%%%%%%%%%%%%%%%%%%%%%%%%%%%%%
\paragraph{v1.6:} 2018/01/17

\begin{itemize}
\item
application for development of include files
\item
corrections to manual
\end{itemize}

%%%%%%%%%%%%%%%%%%%%%%%%%%%%%%%%%%%%%%%%
\paragraph{v1.5:} 2017/05/21

\begin{itemize}
\item
more complete structuring introduced
\item
|\childdocof| introduced
\item
|\childdoc| renamed to |\childdocmain|
\item
|\childredirect| renamed to |\childdocforward| and |\childdocforwardprefix|
and functionality expanded
\end{itemize}

%%%%%%%%%%%%%%%%%%%%%%%%%%%%%%%%%%%%%%%%
\paragraph{v1.0:} 2017/04/27

\begin{itemize}
\item
manual and install package
\item
first version published on CTAN
\end{itemize}

%%%%%%%%%%%%%%%%%%%%%%%%%%%%%%%%%%%%%%%%
\paragraph{v0.6:} 2017/04/26

\begin{itemize}
\item
redirection mechanism added
\end{itemize}

%%%%%%%%%%%%%%%%%%%%%%%%%%%%%%%%%%%%%%%%
\paragraph{v0.5:} 2017/04/26

\begin{itemize}
\item
functionality in definition file
\end{itemize}


%%%%%%%%%%%%%%%%%%%%%%%%%%%%%%%%%%%%%%%%%%%%%%%%%%%%%%%%%%%%%%%%%%%%%%%%%%%%%%%%
%%%%%%%%%%%%%%%%%%%%%%%%%%%%%%%%%%%%%%%%%%%%%%%%%%%%%%%%%%%%%%%%%%%%%%%%%%%%%%%%
%%%%%%%%%%%%%%%%%%%%%%%%%%%%%%%%%%%%%%%%%%%%%%%%%%%%%%%%%%%%%%%%%%%%%%%%%%%%%%%%
\appendix

\settowidth\MacroIndent{\rmfamily\scriptsize 000\ }

 \DocInput{childdoc.dtx}

\end{document}
%</driver>
% \fi
%
% %%%%%%%%%%%%%%%%%%%%%%%%%%%%%%%%%%%%%%%%%%%%%%%%%%%%%%%%%%%%%%%%%%%%%%%%%%%%%%
% %%%%%%%%%%%%%%%%%%%%%%%%%%%%%%%%%%%%%%%%%%%%%%%%%%%%%%%%%%%%%%%%%%%%%%%%%%%%%%
% \section{Sample}
%\iffalse
%<*samplemain>
%\fi
%
% The following presents a sample document
% with two chapters, two parts, a title page,
% a compile flag as well as three forwarding files to set the flag.
% It consists of eight |.tex| files:
% \begin{center}
% \begin{tabular}{ll}
% |cdocsamp.tex|&main file\\
% |cdocsch1.tex|&include file for chapter 1\\
% |cdocsch2.tex|&include file for chapter 2\\
% |cdocspt3.tex|&include file for part 3\\
% |cdocspt4.tex|&include file for part 4\\
% |cdocsdrf.tex|&forwarding file for main file in draft mode\\
% |cdocsfi1.tex|&forwarding file for final version of chapter 1\\
% |cdocsfi2.tex|&forwarding file for final version of chapter 2\\
% \end{tabular}
% \end{center}
% Each of the eight files can be compiled directly by the \LaTeX{} compiler.
%
% %%%%%%%%%%%%%%%%%%%%%%%%%%%%%%%%%%%%%%
% \paragraph{Main File.}
%
% The main file is called |cdocsamp.tex|.
%
% Load the \textsf{childdoc} definitions and
% declare the filename for the main document:
%    \begin{macrocode}
\input{childdoc.def}
\childdocmain{}
%    \end{macrocode}

% Optional override for |\version| flag:
%    \begin{macrocode}
%%\ifchilddoc\else\providecommand{\version}{draft}\fi
%    \end{macrocode}

% Define the default values for the |\version| flag
% (|final| for the main file and |draft| for childs):
%    \begin{macrocode}
\ifchilddoc
\providecommand{\version}{draft}
\else
\providecommand{\version}{final}
\fi
%    \end{macrocode}

% Load the standard document class:
%    \begin{macrocode}
\documentclass[12pt]{article}
%    \end{macrocode}

% Start the document body:
%    \begin{macrocode}
\begin{document}
%    \end{macrocode}

% Declare a title page.
% Print title, part of document being processed and version flag:
%    \begin{macrocode}
\addtocounter{page}{-1}
\begin{center}
{\LARGE\bfseries{}childdoc example\par}
\vspace{1cm}
\ifchilddoc
\ifchilddocmanual part\else chapter\fi:
`\childdocname' of `\childdocjob'\par
\else
main document: `\childdocjob'\par
\fi
version: \version\par
\end{center}
\newpage
%    \end{macrocode}

% Manually include selected file,
% otherwise process as usual:
%    \begin{macrocode}
\ifchilddocmanual
\section*{part `\childdocname'}
\input{\childdocname}
\else
%    \end{macrocode}

% Include the two chapters:
%    \begin{macrocode}
\include{cdocsch1}
\include{cdocsch2}
%    \end{macrocode}

% Include the two parts unless only chapters should be displayed:
%    \begin{macrocode}
\ifchilddoc\else
\section{part three}
\input{cdocspt3}
\section{part four}
\input{cdocspt4}
\fi
%    \end{macrocode}

% Process as usual until here:
%    \begin{macrocode}
\fi
%    \end{macrocode}

% End of document body:
%    \begin{macrocode}
\end{document}
%    \end{macrocode}
%\iffalse
%</samplemain>
%\fi
%
% %%%%%%%%%%%%%%%%%%%%%%%%%%%%%%%%%%%%%%
% \paragraph{Chapter Include Files.}
%
% The include files are called |cdocsch1.tex| and |cdocsch2.tex|.
%
%\iffalse
%<*samplechap1|samplechap2>
%\fi

% Optional override for |\version| flag:
%    \begin{macrocode}
%%\providecommand{\version}{final}
%    \end{macrocode}

% Include the main document:
%    \begin{macrocode}
\input{childdoc.def}
\childdocof{cdocsamp}
%    \end{macrocode}

%\iffalse
%</samplechap1|samplechap2>
%\fi
%
%\iffalse
%<*samplechap1>
%\fi
% Some text for chapter 1:
%    \begin{macrocode}
\section{one}
some text in chapter one
%    \end{macrocode}

%\iffalse
%</samplechap1>
%\fi
% Some text for chapter 2:
%\iffalse
%<*samplechap2>
%\fi
%    \begin{macrocode}
\section{two}
more text in chapter two
%    \end{macrocode}

%\iffalse
%</samplechap2>
%\fi
%
% %%%%%%%%%%%%%%%%%%%%%%%%%%%%%%%%%%%%%%
% \paragraph{Part Include Files.}
%
% The include files are called |cdocspt3.tex| and |cdocspt4.tex|.
%
%\iffalse
%<*samplepart3|samplepart4>
%\fi

% Optional override for |\version| flag:
%    \begin{macrocode}
%%\providecommand{\version}{final}
%    \end{macrocode}

% Include the main document:
%    \begin{macrocode}
\input{childdoc.def}
\childdocby{cdocsamp}
%    \end{macrocode}

%\iffalse
%</samplepart3|samplepart4>
%\fi
%
%\iffalse
%<*samplepart3>
%\fi
% Some text for part 3:
%    \begin{macrocode}
some text in part three
%    \end{macrocode}

%\iffalse
%</samplepart3>
%\fi
% Some text for part 4:
%\iffalse
%<*samplepart4>
%\fi
%    \begin{macrocode}
more text in part four
%    \end{macrocode}

%\iffalse
%</samplepart4>
%\fi
%
% %%%%%%%%%%%%%%%%%%%%%%%%%%%%%%%%%%%%%%
% \paragraph{Forwarding for a Complete Draft.}
%
% The following forwarding file |cdocsdrf.tex|
% compiles the main document in draft mode:
%\iffalse
%<*sampledraft>
%\fi
%    \begin{macrocode}
\def\version{draft}
\input{childdoc.def}
\childdocforward{cdocsamp}
%    \end{macrocode}

%\iffalse
%</sampledraft>
%\fi
%
% %%%%%%%%%%%%%%%%%%%%%%%%%%%%%%%%%%%%%%
% \paragraph{Forwarding for Final Version of the Chapters.}
%
% The following forwarding files |cdocsfn1.tex| and |cdocsfn2.tex|
% (with identical content)
% compile the final versions of the child documents
% |cdocsch1.tex| and |cdocsch2.tex|, respectively:
%\iffalse
%<*samplefinal>
%\fi
%    \begin{macrocode}
\def\version{final}
\input{childdoc.def}
\childdocforwardprefix[cdocsamp]{cdocsfn}{cdocsch}
%    \end{macrocode}

%\iffalse
%</samplefinal>
%\fi
%
% %%%%%%%%%%%%%%%%%%%%%%%%%%%%%%%%%%%%%%
% \paragraph{Command Line Processing.}
%
% The following three command lines generate the output files
% |cdocscld|, |cdocscl1| and |cdocscl2|
% which should be identical to
% |cdocsdrf|, |cdocsch1| and |cdocsfn2|, respectively:
% \begin{center}
% \begin{tabular}{l}
% |latex -jobname cdocscld \|\\
% |  "\def\version{draft}\input{childdoc.def}\childdocforward{cdocsamp}"|\\
% |latex -jobname cdocscl1 \|\\
% |  "\input{childdoc.def}\childdocforward[cdocsamp]{cdocsch1}"|\\
% |latex -jobname cdocscl2 \|\\
% |  "\def\version{final}\input{childdoc.def}\childdocforward{cdocsch2}"|
% \end{tabular}
% \end{center}
% Note that the trailing backslash on each first line
% merely continues the input to the second line
% (for convenient cut ant paste).
% Furthermore, the command |latex| can be replaced by any
% of its alternative versions such as |pdflatex|.
%
% %%%%%%%%%%%%%%%%%%%%%%%%%%%%%%%%%%%%%%%%%%%%%%%%%%%%%%%%%%%%%%%%%%%%%%%%%%%%%%
% %%%%%%%%%%%%%%%%%%%%%%%%%%%%%%%%%%%%%%%%%%%%%%%%%%%%%%%%%%%%%%%%%%%%%%%%%%%%%%
% \section{Implementation}
%\iffalse
%<*package>
%\fi
%
% This section describes the definitions file |childdoc.def|.

% The definitions cannot be loaded using |\usepackage| or |\RequirePackage|
% which has a mechanism to prevent loading a style file more than once.
% When loading the definitions by means of |\input|
% multiple instances have to be prevented manually:
%\iffalse
%This code needs to be before the `\ProvidesFile' directive
%which is defined at the beginning of this file.
%Therefore it is also placed there and commented out here.
%</package>
%<*discard>
%\fi
%    \begin{macrocode}
\ifdefined\childdocmain\endinput\fi
%    \end{macrocode}
%\iffalse
%</discard>
%<*package>
%\fi
%
% \macro{\ifchilddoc}
% \macro{\ifchilddocmanual}
% The conditional |\ifchilddoc| tells whether a
% child (true) or main (false) document is being compiled.
% The conditional |\ifchilddocmanual| tells whether
% the |\includeonly| mechanism is used (false) or
% the selection of child files must be performed manually (true).
% The definitions initialise to false:
%    \begin{macrocode}
\newif\ifchilddoc
\newif\ifchilddocmanual
%    \end{macrocode}

% \macro{\childdocname}
% \macro{\childdocjob}
% The macro |\childdocname| stores the name of the main document
% to be compiled. The macro |\childdocjob| stores the name of
% the document on which the \LaTeX{} compiler was originally invoked.
% The content of |\jobname| cannot be compared
% to filenames specified in the source due to different catcodes.
% The following code rescans |\jobname|, stores the result
% in |\childdocname| and saves a copy in |\childdocjob|:
%    \begin{macrocode}
\edef\childdocname{\scantokens\expandafter{\jobname\noexpand}}
\let\childdocjob\childdocname
%    \end{macrocode}

% \macro{\childdocdisable}
% The macro |\childdocdisable| prevents the main file
% from being processed more than once.
% At this stage, the main document command |\childdocmain|
% is assumed to be called once again where it should do nothing.
% Any subsequent call to it should prevent
% a secondary processing of the main document
% It overwrites the forwarding commands
% |\childdocof| and |\childdocforward|
% with empty macros to prevent further inclusions of the main document:
%    \begin{macrocode}
\newcommand{\childdocdisable}
{
  \renewcommand{\childdocmain}[1]{\renewcommand{\childdocmain}[1]{\endinput}}
  \renewcommand{\childdocof}[1]{}
  \renewcommand{\childdocby}[2][]{}
  \renewcommand{\childdocforward}[2][]{}
  \renewcommand{\childdocdisable}{}
}
%    \end{macrocode}

% \macro{\childdocmain}
% The macro |\childdocmain| is to be called at the top of the main file
% with nothing or the main filename (without extension) as argument.
% First, it breaks loops.
% If the argument is not empty and does not match |\childdocname|
% (which is set by the first inclusion of |childdoc.def|),
% |\ifchilddoc| is set to true, |\includeonly| is applied to the child file
% and |\jobname| is set to the main file
% (for proper handling of |.aux| files):
%    \begin{macrocode}
\newcommand{\childdocmain}[1]
{
  \childdocdisable\childdocmain{}
  \if?#1?\else
    \begingroup
      \def\childdoctmp{#1}
      \ifx\childdoctmp\childdocname
        \def\childdoctmp{}
      \else
        \def\childdoctmp
        {
          \childdoctrue
          \includeonly{\childdocname}
          \def\childdocjob{#1}
          \def\jobname{#1}
        }
      \fi
      \expandafter
    \endgroup
    \childdoctmp
  \fi
}
%    \end{macrocode}

% \macro{\childdocof}
% The command |\childdocof| redirects
% compilation to the main file |#1|.
%    \begin{macrocode}
\newcommand{\childdocof}[1]
{
  \childdocdisable
  \childdoctrue
  \includeonly{\childdocname}
  \def\jobname{#1}
  \def\childdocjob{#1}
  \input{#1}
}
%    \end{macrocode}

% \macro{\childdocby}
% The command |\childdocby| ....
%    \begin{macrocode}
\newcommand{\childdocby}[2][]
{
  \childdocdisable
  \childdoctrue
  \childdocmanualtrue
  \if?#1?\else
    \def\jobname{#2}
  \fi
  \def\childdocjob{#2}
  \input{#2}
  \endinput
}
%    \end{macrocode}

% \macro{\childdocforward}
% The command |\childdocforward| redirects
% compilation to the main file or
% (if the optional argument is given) a child file.
% Parameters are set as if the main file
% or a child file starting with |\childdocof| was compiled.
% Then compilation is handed over to the main file:
%    \begin{macrocode}
\newcommand{\childdocforward}[2][]
{
  \begingroup
    \if?#1?
      \def\childdoctmp
      {
        \def\childdocname{#2}
        \def\childdocjob{#2}
        \def\jobname{#2}
        \input{#2}
        \endinput
      }
    \else
      \def\childdoctmp
      {
        \childdocdisable
        \def\childdocname{#2}
        \childdoctrue
        \includeonly{#2}
        \def\childdocjob{#1}
        \def\jobname{#1}
        \input{#1}
        \endinput
      }
    \fi
    \expandafter
  \endgroup
  \childdoctmp
}
%    \end{macrocode}

% \macro{\childdocforwardprefix}
% The command |\childdocforwardprefix| redirects
% compilation to the main or a child file by means of a pattern.
% The prefix |#1| in the current filename is replaced by |#2|
% and the suffix of the current filename is kept
% (it is assumed that the filename does not contain the substring `|~~~|'
% which is used as a delimiter).
% Compilation is handed over to the new file by |\childdocforward|:
%    \begin{macrocode}
\newcommand{\childdocforwardprefix}[3][]
{
  \begingroup
    \def\childdocextract #2##1~~~{\def\childdoctmp{\childdocforward[#1]{#3##1}}}
    \expandafter\childdocextract\childdocname~~~
    \expandafter
  \endgroup
  \childdoctmp
}
%    \end{macrocode}

% \macro{\childdoc}
% The deprecated macro |\childdoc| is a legacy version of |\childdocmain|:
%    \begin{macrocode}
\newcommand{\childdoc}{\childdocmain}
%    \end{macrocode}

% \macro{\childdocredirect}
% The deprecated macro |\childdocredirect| is a legacy version
% of |\childdocforward| and |\childdocforwardprefix|:
%    \begin{macrocode}
\newcommand{\childdocredirect}[2][]
{
  \begingroup
    \if?#1?
      \def\childdoctmp{\childdocforward{#2}}
    \else
      \def\childdoctmp{\childdocforwardprefix{#1}{#2}}
    \fi
    \expandafter
  \endgroup
  \childdoctmp
}
%    \end{macrocode}

%\iffalse
%</package>
%\fi
%
\endinput
|\\
|\childdocmain{|\textit{main}|}|\\
\end{tabular}
\end{center}
%
If |\jobname| does not match the argument \textit{main} of |\childdocmain|,
it is assumed that |\jobname| points to the child file to be compiled.
When using |\childdocmain| with the main file specified as argument,
it suffices to start a child file
with just |\input{|\textit{main}|}|
without loading of the package and using |\childdocof|.
If instead all processing is done
with the appropriate \textsf{childdoc} directives,
the argument of \textit{main} of |\childdocmain| can be empty.

An alternative version of the command line processing described
in \secref{sec:commandline} using the detection mechanism reads:
%
\begin{center}
|... -jobname "|\textit{target}|" "|[\textit{flags}]%
[|\def\jobname{|\textit{dest}|}|]|\input{|\textit{main}|}"|
\end{center}

%%%%%%%%%%%%%%%%%%%%%%%%%%%%%%%%%%%%%%%%%%%%%%%%%%%%%%%%%%%%%%%%%%%%%%%%%%%%%%%%
\subsection{Manual Code}
\label{sec:manual}

In case one cannot be certain whether the definitions file |childdoc.def|
is installed on the target \TeX{} distribution
and one prefers not to ship it,
it is conceivable to paste a few relevant commands into the sources.

To that end, drop all statements |% \iffalse
%
% childdoc.dtx Copyright (C) 2017-2018 Niklas Beisert
%
% This work may be distributed and/or modified under the
% conditions of the LaTeX Project Public License, either version 1.3
% of this license or (at your option) any later version.
% The latest version of this license is in
%   http://www.latex-project.org/lppl.txt
% and version 1.3 or later is part of all distributions of LaTeX
% version 2005/12/01 or later.
%
% This work has the LPPL maintenance status `maintained'.
%
% The Current Maintainer of this work is Niklas Beisert.
%
% This work consists of the files childdoc.dtx and childdoc.ins
% and the derived files childdoc.def and cdocsamp.tex with
% cdocsch1.tex, cdocsch2.tex, cdocsdrf.tex, cdocsfn1.tex, cdocsfn2.tex.
%
%<package>\ifdefined\childdocmain\endinput\fi
%<package>\ProvidesFile{childdoc.def}[2018/12/30 v2.0 child document driver]
%<samplemain>\ProvidesFile{cdocsamp.tex}[2018/12/30 v2.0 sample for childdoc]
%<*driver>
%\ProvidesFile{childdoc.drv}[2018/12/30 v2.0 childdoc reference manual file]
\PassOptionsToClass{10pt,a4paper}{article}
\documentclass{ltxdoc}

\usepackage[margin=35mm]{geometry}
\usepackage{hyperref}
\usepackage{hyperxmp}
\usepackage[usenames]{color}

\hypersetup{colorlinks=true}
\hypersetup{pdfstartview=FitH}
\hypersetup{pdfpagemode=UseNone}
\hypersetup{pdfsource={}}
\hypersetup{pdflang={en-UK}}
\hypersetup{pdfcopyright={Copyright 2017-2018 Niklas Beisert.
  This work may be distributed and/or modified under the
  conditions of the LaTeX Project Public License, either version 1.3
  of this license or (at your option) any later version.}}
\hypersetup{pdflicenseurl={http://www.latex-project.org/lppl.txt}}
\hypersetup{pdfcontactaddress={ETH Zurich, ITP, HIT K,
  Wolfgang-Pauli-Strasse 27}}
\hypersetup{pdfcontactpostcode={8093}}
\hypersetup{pdfcontactcity={Zurich}}
\hypersetup{pdfcontactcountry={Switzerland}}
\hypersetup{pdfcontactemail={nbeisert@itp.phys.ethz.ch}}
\hypersetup{pdfcontacturl={http://people.phys.ethz.ch/\xmptilde nbeisert/}}

\newcommand{\secref}[1]{\hyperref[#1]{section \ref*{#1}}}

\parskip1ex
\parindent0pt
\let\olditemize\itemize
\def\itemize{\olditemize\parskip0pt}

\begin{document}

\title{The \textsf{childdoc} Package}
\hypersetup{pdftitle={The childdoc Package}}
\author{Niklas Beisert\\[2ex]
  Institut f\"ur Theoretische Physik\\
  Eidgen\"ossische Technische Hochschule Z\"urich\\
  Wolfgang-Pauli-Strasse 27, 8093 Z\"urich, Switzerland\\[1ex]
  \href{mailto:nbeisert@itp.phys.ethz.ch}
  {\texttt{nbeisert@itp.phys.ethz.ch}}}
\hypersetup{pdfauthor={Niklas Beisert}}
\hypersetup{pdfsubject={Manual for the LaTeX2e Package childdoc}}
\date{30 December 2018, \textsf{v2.0}}
\maketitle

\begin{abstract}\noindent
\textsf{childdoc} is a \LaTeXe{} package
that enables the direct compilation
of document sections included by |\include|
to individual files.
\end{abstract}

\begingroup
\parskip0ex
\tableofcontents
\endgroup

%%%%%%%%%%%%%%%%%%%%%%%%%%%%%%%%%%%%%%%%%%%%%%%%%%%%%%%%%%%%%%%%%%%%%%%%%%%%%%%%
%%%%%%%%%%%%%%%%%%%%%%%%%%%%%%%%%%%%%%%%%%%%%%%%%%%%%%%%%%%%%%%%%%%%%%%%%%%%%%%%
\section{Introduction}

\LaTeX{} provides a mechanism to structure a large document (such as a book)
into a main file and several child files (containing the chapters)
using the |\include| command.
This mechanism is beneficial for documents
which span hundreds of pages in order to
make the source file(s) more manageable.
Moreover, compilation can be restricted to
selected child files by means of the |\includeonly| command.
The latter feature can be used to reduce the compilation time while editing
(this was significantly more useful in the earlier days of \LaTeX{})
or to generate a smaller document which is easier to navigate.
Another application of |\includeonly| is to generate
documents consisting of selected parts of the complete document.

However, there are a few drawbacks of the plain |\include| mechanism:
\begin{itemize}
\item
The child files cannot be compiled on their own,
they can only be compiled via the main file.
A naive editing environment
(such as a text editor with an option
to have the current file processed by \LaTeX)
may require one to switch to the main file before compiling;
attempting to compile the child file produces errors.
\item
The main file must be modified (each time)
to adjust the |\includeonly| command
to the present needs. This easily leaves the main file in a messy state.
\item
The generated document will always carry the filename
of the main document. This is inconvenient if
several child files are to be compiled and
to be kept for distribution.
\end{itemize}

The present package provides a simple interface
to make child files individually compilable by \LaTeX{}.
Compiling a child file then has the same effect as compiling
the main file with an |\includeonly| command
to select the appropriate child.
Moreover the generated document will carry the name of the child
rather than the main file.
This resolves all three above issues.

This feature is meant to make the editing of books,
thesis documents and lecture notes somewhat more convenient.
However, the package can also be used efficiently for
composing a series of documents (such as exercise sheets)
which are typically distributed individually.
It then assists the author in generating the individual documents
(potentially in different versions)
as well as a document containing the collected series.
Another application is in developing style files
or other kinds of included material
where compilation of the style file could redirect
to a sample or test file.

%%%%%%%%%%%%%%%%%%%%%%%%%%%%%%%%%%%%%%%%%%%%%%%%%%%%%%%%%%%%%%%%%%%%%%%%%%%%%%%%
%%%%%%%%%%%%%%%%%%%%%%%%%%%%%%%%%%%%%%%%%%%%%%%%%%%%%%%%%%%%%%%%%%%%%%%%%%%%%%%%
\section{Usage}

First of all, the package \textsf{childdoc} is \emph{not} a standard
\LaTeXe{} |.sty| style file! Therefore it needs to be invoked in
a non-standard way.

%%%%%%%%%%%%%%%%%%%%%%%%%%%%%%%%%%%%%%%%%%%%%%%%%%%%%%%%%%%%%%%%%%%%%%%%%%%%%%%%
\subsection{Included Files}
\label{sec:include}

%%%%%%%%%%%%%%%%%%%%%%%%%%%%%%%%%%%%%%%%
\DescribeMacro{\childdocmain}
To use the package, add the commands
\begin{center}
\begin{tabular}{l}
|\input{childdoc.def}|\\
|\childdocmain{}|\\
\end{tabular}
\end{center}
at the very top of the main \LaTeX{} file,
in particular \emph{before} the |\documentclass| statement!
The argument of |\childdocmain| should be left empty
(but it must be present).

%%%%%%%%%%%%%%%%%%%%%%%%%%%%%%%%%%%%%%%%
\DescribeMacro{\childdocof}
Furthermore, add the commands
\begin{center}
\begin{tabular}{l}
|\input{childdoc.def}|\\
|\childdocof{|\textit{main}|}|\\
\end{tabular}
\end{center}
at the top of every child file \textit{child}
which is included by |\include{|\textit{child}|}|
from within the main file
(or at least for those files to be compiled individually).
The argument \textit{main} must be the filename of the main file.

There are a couple of
considerations in setting up the main and child documents:

%%%%%%%%%%%%%%%%%%%%%%%%%%%%%%%%%%%%%%%%
\paragraph{Restrictions.}

Please note the following restrictions:
\begin{itemize}
\item
|\childdocmain| must be called with one argument \textit{main}
to ensure compatibility with earlier version of the package.
It must either be empty (|\childdocmain{}|)
or precisely match the filename of the main file in which it is specified.
See \secref{sec:detection} for further information.
\item
The filename \textit{main} must be specified without the |.tex| extension.
\item
The filename \textit{main} is case sensitive
(even in case-insensitive file systems)
due to internal string comparison.
\item
The argument \textit{main} should be fully expanded, it cannot be a macro.
\item
Subdirectories and special characters should be avoided in filenames.
\item
The command |\childdocmain{|\textit{main}|}| must be followed by a whitespace.
It should not be followed immediately by another command
or by a comment mark `|%|'.
This is because the \TeX{} parser reads the token immediately following
the argument of |\childdocmain| and puts it
at the beginning of every child section;
however, a white\-space is ignored.
\end{itemize}

%%%%%%%%%%%%%%%%%%%%%%%%%%%%%%%%%%%%%%%%
\paragraph{Content of Main File.}

It is advisable to place all content in the child files included by |\include|.
Any output contained in the main file will appear in all child documents
unless suppressed manually;
it cannot be suppressed automatically by the |\includeonly| directive
and thus should normally be avoided.
A method to include some content in the main file
by means of conditional processing is described in \secref{sec:conditional}.

%%%%%%%%%%%%%%%%%%%%%%%%%%%%%%%%%%%%%%%%
\paragraph{Page Numbering.}

When only a part of the document is compiled,
the appropriate numbering of pages
(as well as other status parameters)
is determined from the |.aux| files.
The latter contain information from previous passes.
However this information needs to propagate through
all intermediate child documents.
Therefore the page numbering in child documents may well
be inconsistent until the complete document is compiled at least once.

A useful (if unconventional) way to always ensure a consistent
page numbering is to restart the numbering in each child document
and denote the pages by `\textit{child}|.|\textit{page}'
where \textit{child} represents the chapter/section number of the child file.
This can be achieved by the command
|\numberwithin{page}{|\textit{child}|}|
of the \textsf{amsmath} package
where \textit{child} can be |chapter| or |section|
depending on the chosen structuring.
Alternatively, one can modify the macro |\thepage| appropriately
and reset the counter |page| at the start of each child file.

%%%%%%%%%%%%%%%%%%%%%%%%%%%%%%%%%%%%%%%%%%%%%%%%%%%%%%%%%%%%%%%%%%%%%%%%%%%%%%%%
\subsection{Conditional Processing}
\label{sec:conditional}

The package provides a mechanism to compile different versions
of a document. To customise the versions further some conditional processing
can come in handy to distinguish which version is being compiled.
The package provides two macros to describe the compilation context:

%%%%%%%%%%%%%%%%%%%%%%%%%%%%%%%%%%%%%%%%
\DescribeMacro{\ifchilddoc}
The conditional |\ifchilddoc| distinguishes between the compilation of
child documents and the main document:
%
\begin{center}
|\ifchilddoc |\textit{child-code}| |[|\||else |\textit{main-code}]| \||fi|
\end{center}

%%%%%%%%%%%%%%%%%%%%%%%%%%%%%%%%%%%%%%%%
\DescribeMacro{\childdocname}
\DescribeMacro{\childdocjob}
The macro |\childdocname| contains the filename (without extension)
of the main or child file being processed.
Note that |\childdocjob| will always contain the name of the main file.

%%%%%%%%%%%%%%%%%%%%%%%%%%%%%%%%%%%%%%%%
\paragraph{Title Page.}

Conditional processing can be used to include a title or banner page
in the main document when proper precautions are taken.
Importantly, the code in the main file should ensure that the page counter
(as well as other status parameters which are stored in the |.aux| files)
takes the same value after the conditional processing.
Otherwise the page numbers may take divergent values
depending on which part is compiled.

For example, a title page could be declared by:
%
\begin{center}
\begin{tabular}{l}
|\ifchilddoc\||else|\\
|\addtocounter{page}{-1}|\\
\textit{code for title page}\\
|\newpage|\\
|\||fi|
\end{tabular}
\end{center}
%
A banner page for the child documents can be generated by:
%
\begin{center}
\begin{tabular}{l}
|\ifchilddoc|\\
|\addtocounter{page}{-1}|\\
\textit{code for banner page}\\
|\newpage|\\
|\||fi|
\end{tabular}
\end{center}
%
Here one could write a message such as:
\begin{center}
|This is the part \childdocname{} of \childdocjob{}.|
\end{center}

%%%%%%%%%%%%%%%%%%%%%%%%%%%%%%%%%%%%%%%%%%%%%%%%%%%%%%%%%%%%%%%%%%%%%%%%%%%%%%%%
\subsection{Flags}
\label{sec:flags}

The package makes it easy to generate different versions
of the main or child documents.
To this end compilation flags can be defined
and assigned different default values.
They will be particularly useful in conjunction
with the forwarding mechanism described in \secref{sec:forward}.

For example, it may be useful to have a flag |\version|
which can be set to |draft| or |final|.
The document source will contain some conditional code
depending on the value of |\version|.
Suppose further, the flag should default to |final| for the main file
and to |draft| for child files
which is a natural assignment for editing the document.
This is achieved by placing the following code
in the preamble of the main document
(below the |\childdocmain| directive):
%
\begin{center}
\begin{tabular}{l}
|\ifchilddoc|\\
|\providecommand{\version}{draft}|\\
|\||else|\\
|\providecommand{\version}{final}|\\
|\||fi|
\end{tabular}
\end{center}
%
The definition by |\providecommand| makes sure
that previous definitions are not overwritten.
Further statements |\providecommand{\version}{...}|
can thus be added before the above code to override it.

For the main file, one might add a line
(between |\childdocmain| and the above block)
%
\begin{center}
|%\ifchilddoc\||else\providecommand{\version}{draft}\||fi|
\end{center}
%
which can be uncommented to produce a draft version.
Likewise one can add a line to the very top of a child file
(above the |\childdocof{|\textit{main}|}| directive)
%
\begin{center}
|%\providecommand{\version}{final}|
\end{center}
%
which can be uncommented to produce the final version of this child document.

%%%%%%%%%%%%%%%%%%%%%%%%%%%%%%%%%%%%%%%%%%%%%%%%%%%%%%%%%%%%%%%%%%%%%%%%%%%%%%%%
\subsection{Forwarding}
\label{sec:forward}

Different versions of the main or child documents
using compilation flags as described in \secref{sec:flags}
can be (permanently) stored in different files
for convenient compilation, viewing and distribution.
To this end, the package defines a command
to pass on compilation to a different file:

%%%%%%%%%%%%%%%%%%%%%%%%%%%%%%%%%%%%%%%%
\DescribeMacro{\childdocforward}
The command |\childdocforward| redirects processing to
another source file:
%
\begin{center}
\begin{tabular}{l}
|\input{childdoc.def}|\\
|\childdocforward[|\textit{main}|]{|\textit{dest}|}|\\
\end{tabular}
\end{center}
%
The argument \textit{dest} is the destination file
(without extension).
It should be the main file or one of the child files.
Note that further \textsf{childdoc} directives
such as |\childdocof| and |\childdocforward|
in the indicated file will be processed in this form.
The optional argument \textit{main}
passes on directly to the main file \textit{main}
while pretending to compile the child \textit{dest}.
This form behaves as if \textit{dest}
issues |\childdocof{|\textit{main}|}| right away,
and no further \textsf{childdoc} directives will be processed.

%%%%%%%%%%%%%%%%%%%%%%%%%%%%%%%%%%%%%%%%
\DescribeMacro{\...prefix}
In the alternative form |\childdocforwardprefix|,
%
\begin{center}
\begin{tabular}{l}
|\input{childdoc.def}|\\
|\childdocforwardprefix[|\textit{main}|]{|\textit{prefix}|}{|\textit{dest}|}|
\end{tabular}
\end{center}
%
the destination file is determined by a pattern
depending on the current file:
To make this work, the current file must be called
`{\textit{prefix}\hspace{0.2em}\textit{suffix}}'
with \textit{prefix} matching precisely the argument.
Processing is then passed on to the file
`{\textit{dest}\hspace{0.2em}\textit{suffix}}'.
Surely, the same effect is achieved by
directly specifying the
argument `{\textit{dest}\hspace{0.2em}\textit{suffix}}'
in the first form.
However, that requires to set up a different file
for each child. With the alternative form of the command
all these files can have exactly the same content
which simplifies setting them up and maintaining them.

For example, the following file |draft.tex|
with a compilation flag |\version| as described in \secref{sec:flags}
compiles the main document as a draft:
%
\begin{center}
\begin{tabular}{l}
|\def\version{draft}|\\
|\input{childdoc.def}|\\
|\childdocforward{|\textit{main}|}|
\end{tabular}
\end{center}
%
Likewise, the following files |final|\textit{nn}|.tex|
compile the final version of the child document
|child|\textit{nn}|.tex|:
%
\begin{center}
\begin{tabular}{l}
|\def\version{final}|\\
|\input{childdoc.def}|\\
|\childdocforwardprefix{final}{child}|
\end{tabular}
\end{center}
%

Note that when several versions of a main file and/or of each child file
are to be generated, it may be convenient to set up a |Makefile| or
shell script to automatise the process.

%%%%%%%%%%%%%%%%%%%%%%%%%%%%%%%%%%%%%%%%%%%%%%%%%%%%%%%%%%%%%%%%%%%%%%%%%%%%%%%%
\subsection{Command Line Processing}
\label{sec:commandline}

The effect of redirection files can also be achieved by invoking
the \LaTeX{} compiler with a more elaborate command line.
Most conveniently this should be done as part
of a shell script or a |Makefile|.

When using \textsf{childdoc} in the main file, the following
command lines effectively perform a redirection
(note that depending on the shell being used,
backslashes may have to be doubled: `|\|' $\to$ `|\\|'):
%
\begin{center}
|... -jobname "|\textit{target}|" |\\|"|[\textit{flags}]%
|\input{childdoc.def}\childdocforward[|\textit{main}|]{|\textit{dest}|}"|
\end{center}
%
Here \textit{target} is the name of the output file,
\textit{main} is the name of the main file
and \textit{dest} is the name of the main or child file to be processed
(all filenames without extensions).
The optional argument \textit{main} can be omitted
if \textit{main} matches \textit{dest}.
Optionally, compilation \textit{flags} can be defined via |\def| commands.
This command line makes the \TeX{} engine believe
it is compiling the file \textit{target}
whose content is specified as the latter parameter.
The provided code then forwards the processing to
\textit{main} or \textit{dest} as described in \secref{sec:forward}.

%%%%%%%%%%%%%%%%%%%%%%%%%%%%%%%%%%%%%%%%%%%%%%%%%%%%%%%%%%%%%%%%%%%%%%%%%%%%%%%%
\subsection{Include by Input}
\label{sec:input}

Including child documents by |\include| has some restrictions by design.
Most notably, the content of a child document always occupies
its own set of pages; pages cannot be shared between child documents.
Usually, this behaviour makes perfect sense
because each child document contain an essential part of the document.
However, in some situations it may be desirable to compose
a document from a collection of parts
without having mandatory page breaks between then.
For this case, the package
provides a mechanism to include parts
by |\input| which can also be processed individually.
However, by construction this mechanism
requires manual handling of the content to be output.

%%%%%%%%%%%%%%%%%%%%%%%%%%%%%%%%%%%%%%%%
\DescribeMacro{\ifchilddocmanual}
The main file should be prepared as usual, see \secref{sec:include}.
However, the document body must make a distinction
between processing of an individual part and of the main document, e.g.:
%
\begin{center}
\begin{tabular}{l}
|\ifchilddocmanual|\\
|\input{\childdocname}|\\
|\||else|\\
\textit{document body with }|\input{|\textit{part}|}|\\
|\||fi|
\end{tabular}
\end{center}
%
The conditional |\ifchilddocmanual| is true whenever
a part to be included by |\input| is being compiled,
and the name of the part is stored in |\childdocname|.

%%%%%%%%%%%%%%%%%%%%%%%%%%%%%%%%%%%%%%%%
\DescribeMacro{\childdocby}
Each part to be included by |\input| should start with:
%
\begin{center}
\begin{tabular}{l}
|\input{childdoc.def}|\\
|\childdocby{|\textit{main}|}|\\
\end{tabular}
\end{center}
%
The directive |\childdocby| is similar to |\childdocof|
described in \secref{sec:include},
but the subsequent selection of content must be done manually.
To that end, both |\ifchilddoc| and |\ifchilddocmanual|
will be true upon processing of a part,
and the name of the part is stored in |\childdocname|.
Note that |\jobname| will be set to the filename of the current part
so that each part receives an individual |.aux| file
that does not interfere with the |.aux| file(s) of the main document.
This behaviour can be altered by the alternative form
|\childdocby[*]{|\textit{main}|}| (with a non-empty optional argument)
which uses the |.aux| file of the main document
by setting |\jobname| to \textit{main}.

%%%%%%%%%%%%%%%%%%%%%%%%%%%%%%%%%%%%%%%%%%%%%%%%%%%%%%%%%%%%%%%%%%%%%%%%%%%%%%%%
\subsection{Driver Development}
\label{sec:driver}

The \textsf{childdoc} mechanism can also be use for the development
of definition files such as \LaTeX{} styles or classes.
This case differs from the above setup with multiple parts
included by |\include| in that no |\includeonly| should be invoked.
This can be achieved by starting the include file
(before |\ProvidesPackage|) with:
%
\begin{center}
\begin{tabular}{l}
|\input{childdoc.def}|\\
|\childdocforward{|\textit{main}|}|\\
\end{tabular}
\end{center}
%
or alternatively with:
%
\begin{center}
\begin{tabular}{l}
|\input{childdoc.def}|\\
|\childdocby{|\textit{main}|}|\\
\end{tabular}
\end{center}
%
Both forms have slightly different effects as described above.
The main file is prepared as usual, see \secref{sec:include}.

%%%%%%%%%%%%%%%%%%%%%%%%%%%%%%%%%%%%%%%%%%%%%%%%%%%%%%%%%%%%%%%%%%%%%%%%%%%%%%%%
\subsection{Legacy Detection}
\label{sec:detection}

The directive |\childdocmain| in the main file can detect
whether the complete document or merely a child is to be compiled
even without using the directive |\childdocof|.
This method is deprecated because it is less robust
and there is no compelling reason to use it;
it is merely provided for backward compatibility
and it may be removed in future versions.

If the detection mechanism is to be used,
it is mandatory to correctly specify
the filename of the main file as the argument of |\childdocmain|:
%
\begin{center}
\begin{tabular}{l}
|\input{childdoc.def}|\\
|\childdocmain{|\textit{main}|}|\\
\end{tabular}
\end{center}
%
If |\jobname| does not match the argument \textit{main} of |\childdocmain|,
it is assumed that |\jobname| points to the child file to be compiled.
When using |\childdocmain| with the main file specified as argument,
it suffices to start a child file
with just |\input{|\textit{main}|}|
without loading of the package and using |\childdocof|.
If instead all processing is done
with the appropriate \textsf{childdoc} directives,
the argument of \textit{main} of |\childdocmain| can be empty.

An alternative version of the command line processing described
in \secref{sec:commandline} using the detection mechanism reads:
%
\begin{center}
|... -jobname "|\textit{target}|" "|[\textit{flags}]%
[|\def\jobname{|\textit{dest}|}|]|\input{|\textit{main}|}"|
\end{center}

%%%%%%%%%%%%%%%%%%%%%%%%%%%%%%%%%%%%%%%%%%%%%%%%%%%%%%%%%%%%%%%%%%%%%%%%%%%%%%%%
\subsection{Manual Code}
\label{sec:manual}

In case one cannot be certain whether the definitions file |childdoc.def|
is installed on the target \TeX{} distribution
and one prefers not to ship it,
it is conceivable to paste a few relevant commands into the sources.

To that end, drop all statements |\input{childdoc.def}|
and perform the replacements as outlined below.
Instead of |\childdocmain{|\textit{main}|}| add the following code
to the top of the main file:
%
\begin{center}
\begin{tabular}{l}
|\||ifdefined\childdocname\endinput\||fi\newif\ifchilddoc|\\
|\edef\childdocname{\scantokens\expandafter{\jobname\noexpand}}|\\
|\def\childdocmain{|\textit{main}|}\||ifx\childdocmain\childdocname\||else|\\
|\childdoctrue\includeonly{\childdocname}\let\jobname\childdocmain\||fi|\\
\end{tabular}
\end{center}
%
Instead of |\childdocof{|\textit{main}|}| just include the main file
at the top of each child file:
%
\begin{center}
|\input{|\textit{main}|}|
\end{center}
%
A simple redirection |\childdocforward{|\textit{dest}|}| is achieved by:
%
\begin{center}
|\def\jobname{|\textit{dest}|}\input{\jobname}|
\end{center}
%
The redirection with prefix
|\childdocforwardprefix[|\textit{prefix}|]{|\textit{dest}|}|
is accomplished by:
%
\begin{center}
\begin{tabular}{l}
|{\edef\jobname{\scantokens\expandafter{\jobname\noexpand}}|\\
|\def\redirectjob |\textit{prefix}|#1~~~{\gdef\jobname{|\textit{dest}|#1}}|\\
|\expandafter\redirectjob\jobname~~~}\input{\jobname}|
\end{tabular}
\end{center}

In an alternative approach,
child documents can be compiled by a specific command line
without additional code or specific definitions:
%
\begin{center}
|... -jobname "|\textit{target}|" "|[\textit{flags}]%
|\includeonly{|\textit{dest}|}\input{|\textit{main}|}"|
\end{center}
%

%%%%%%%%%%%%%%%%%%%%%%%%%%%%%%%%%%%%%%%%%%%%%%%%%%%%%%%%%%%%%%%%%%%%%%%%%%%%%%%%
%%%%%%%%%%%%%%%%%%%%%%%%%%%%%%%%%%%%%%%%%%%%%%%%%%%%%%%%%%%%%%%%%%%%%%%%%%%%%%%%
\section{Information}

%%%%%%%%%%%%%%%%%%%%%%%%%%%%%%%%%%%%%%%%%%%%%%%%%%%%%%%%%%%%%%%%%%%%%%%%%%%%%%%%
\subsection{Copyright}

Copyright \copyright{} 2017--2018 Niklas Beisert

This work may be distributed and/or modified under the
conditions of the \LaTeX{} Project Public License, either version 1.3
of this license or (at your option) any later version.
The latest version of this license is in
  \url{http://www.latex-project.org/lppl.txt}
and version 1.3 or later is part of all distributions of \LaTeX{}
version 2005/12/01 or later.

This work has the LPPL maintenance status `maintained'.

The Current Maintainer of this work is Niklas Beisert.

This work consists of the files |README.txt|, |childdoc.ins| and |childdoc.dtx|
as well as the derived files |childdoc.def|, |cdocsamp.tex|
with |cdocsch1.tex|, |cdocsch2.tex|, |cdocspt3.tex|, |cdocspt4.tex|,
|cdocsdrf.tex|, |cdocsfn1.tex|, |cdocsfn2.tex|
as well as |childdoc.pdf|.

%%%%%%%%%%%%%%%%%%%%%%%%%%%%%%%%%%%%%%%%%%%%%%%%%%%%%%%%%%%%%%%%%%%%%%%%%%%%%%%%
\subsection{Files and Installation}

The package consists of the files:
%
\begin{center}
\begin{tabular}{ll}
    |README.txt|   & readme file \\
    |childdoc.ins| & installation file \\
    |childdoc.dtx| & source file \\
    |childdoc.def| & definition file \\
    |cdocsamp.tex| & sample main file \\
    |cdocsch1.tex| & sample include file \\
    |cdocsch2.tex| & sample include file \\
    |cdocspt3.tex| & sample part file \\
    |cdocspt4.tex| & sample part file \\
    |cdocsdrf.tex| & sample redirection file \\
    |cdocsfn1.tex| & sample redirection file \\
    |cdocsfn2.tex| & sample redirection file \\
    |childdoc.pdf| & manual
\end{tabular}
\end{center}
%
The distribution consists of the files
|README.txt|, |childdoc.ins| and |childdoc.dtx|.
%
\begin{itemize}
\item
Run (pdf)\LaTeX{} on |childdoc.dtx|
to compile the manual |childdoc.pdf| (this file).
\item
Run \LaTeX{} on |childdoc.ins| to create the definitions file |childdoc.def|
and the sample |cdocsamp.tex| with include files
|cdocsch1.tex|, |cdocsch2.tex|, |cdocspt3.tex|, |cdocspt4.tex|,
|cdocsdrf.tex|, |cdocsfn1.tex|, |cdocsfn2.tex|.
Then copy the file |childdoc.def| to an appropriate directory of your \LaTeX{}
distribution, e.g.\ \textit{texmf-root}|/tex/latex/childdoc|.
\end{itemize}

%%%%%%%%%%%%%%%%%%%%%%%%%%%%%%%%%%%%%%%%%%%%%%%%%%%%%%%%%%%%%%%%%%%%%%%%%%%%%%%%
\subsection{Related CTAN Packages}

There are several other packages which offer a similar functionality:
%
\begin{itemize}
\item
The packages
\href{http://ctan.org/pkg/docmute}{\textsf{docmute}},
\href{http://ctan.org/pkg/includex}{\textsf{includex}} and
\href{http://ctan.org/pkg/standalone}{\textsf{standalone}}
provide commands to include only the document body of
a child file thus allowing both files to be compiled individually.
\item
The packages \href{http://ctan.org/pkg/subdocs}{\textsf{subdocs}}
and \href{http://ctan.org/pkg/subfiles}{\textsf{subfiles}}
provide structures in which the main and child documents can be
encapsulated and allowing them to be compiled individually.
The inclusion mechanism is different from the conventional |\include|.
\item
The package \href{http://ctan.org/pkg/combine}{\textsf{combine}}
is an elaborate solution to combine several documents into one.
\end{itemize}
%
See also the CTAN topic \href{http://ctan.org/topic/subdocs}{\textsf{subdocs}}
for further related packages.
The present package differs from the above solutions in that
a document structure constructed with the conventional |\include| mechanism
just needs two extra commands at the top of every file
such that all constituent files can be compiled individually.

%%%%%%%%%%%%%%%%%%%%%%%%%%%%%%%%%%%%%%%%%%%%%%%%%%%%%%%%%%%%%%%%%%%%%%%%%%%%%%%%
%\subsection{Feature Suggestions}
%
%The following is a list of features which may be useful for future
%versions of this package:
%%
%\begin{itemize}
%\item
%\ldots
%\end{itemize}

%%%%%%%%%%%%%%%%%%%%%%%%%%%%%%%%%%%%%%%%%%%%%%%%%%%%%%%%%%%%%%%%%%%%%%%%%%%%%%%%
\subsection{Revision History}

%%%%%%%%%%%%%%%%%%%%%%%%%%%%%%%%%%%%%%%%
\paragraph{v2.0:} 2018/12/30

\begin{itemize}
\item
immediate forward processing
\item
added |\childdocby| mechanism
\item
manual restructured
\end{itemize}

%%%%%%%%%%%%%%%%%%%%%%%%%%%%%%%%%%%%%%%%
\paragraph{v1.6:} 2018/01/17

\begin{itemize}
\item
application for development of include files
\item
corrections to manual
\end{itemize}

%%%%%%%%%%%%%%%%%%%%%%%%%%%%%%%%%%%%%%%%
\paragraph{v1.5:} 2017/05/21

\begin{itemize}
\item
more complete structuring introduced
\item
|\childdocof| introduced
\item
|\childdoc| renamed to |\childdocmain|
\item
|\childredirect| renamed to |\childdocforward| and |\childdocforwardprefix|
and functionality expanded
\end{itemize}

%%%%%%%%%%%%%%%%%%%%%%%%%%%%%%%%%%%%%%%%
\paragraph{v1.0:} 2017/04/27

\begin{itemize}
\item
manual and install package
\item
first version published on CTAN
\end{itemize}

%%%%%%%%%%%%%%%%%%%%%%%%%%%%%%%%%%%%%%%%
\paragraph{v0.6:} 2017/04/26

\begin{itemize}
\item
redirection mechanism added
\end{itemize}

%%%%%%%%%%%%%%%%%%%%%%%%%%%%%%%%%%%%%%%%
\paragraph{v0.5:} 2017/04/26

\begin{itemize}
\item
functionality in definition file
\end{itemize}


%%%%%%%%%%%%%%%%%%%%%%%%%%%%%%%%%%%%%%%%%%%%%%%%%%%%%%%%%%%%%%%%%%%%%%%%%%%%%%%%
%%%%%%%%%%%%%%%%%%%%%%%%%%%%%%%%%%%%%%%%%%%%%%%%%%%%%%%%%%%%%%%%%%%%%%%%%%%%%%%%
%%%%%%%%%%%%%%%%%%%%%%%%%%%%%%%%%%%%%%%%%%%%%%%%%%%%%%%%%%%%%%%%%%%%%%%%%%%%%%%%
\appendix

\settowidth\MacroIndent{\rmfamily\scriptsize 000\ }

 \DocInput{childdoc.dtx}

\end{document}
%</driver>
% \fi
%
% %%%%%%%%%%%%%%%%%%%%%%%%%%%%%%%%%%%%%%%%%%%%%%%%%%%%%%%%%%%%%%%%%%%%%%%%%%%%%%
% %%%%%%%%%%%%%%%%%%%%%%%%%%%%%%%%%%%%%%%%%%%%%%%%%%%%%%%%%%%%%%%%%%%%%%%%%%%%%%
% \section{Sample}
%\iffalse
%<*samplemain>
%\fi
%
% The following presents a sample document
% with two chapters, two parts, a title page,
% a compile flag as well as three forwarding files to set the flag.
% It consists of eight |.tex| files:
% \begin{center}
% \begin{tabular}{ll}
% |cdocsamp.tex|&main file\\
% |cdocsch1.tex|&include file for chapter 1\\
% |cdocsch2.tex|&include file for chapter 2\\
% |cdocspt3.tex|&include file for part 3\\
% |cdocspt4.tex|&include file for part 4\\
% |cdocsdrf.tex|&forwarding file for main file in draft mode\\
% |cdocsfi1.tex|&forwarding file for final version of chapter 1\\
% |cdocsfi2.tex|&forwarding file for final version of chapter 2\\
% \end{tabular}
% \end{center}
% Each of the eight files can be compiled directly by the \LaTeX{} compiler.
%
% %%%%%%%%%%%%%%%%%%%%%%%%%%%%%%%%%%%%%%
% \paragraph{Main File.}
%
% The main file is called |cdocsamp.tex|.
%
% Load the \textsf{childdoc} definitions and
% declare the filename for the main document:
%    \begin{macrocode}
\input{childdoc.def}
\childdocmain{}
%    \end{macrocode}

% Optional override for |\version| flag:
%    \begin{macrocode}
%%\ifchilddoc\else\providecommand{\version}{draft}\fi
%    \end{macrocode}

% Define the default values for the |\version| flag
% (|final| for the main file and |draft| for childs):
%    \begin{macrocode}
\ifchilddoc
\providecommand{\version}{draft}
\else
\providecommand{\version}{final}
\fi
%    \end{macrocode}

% Load the standard document class:
%    \begin{macrocode}
\documentclass[12pt]{article}
%    \end{macrocode}

% Start the document body:
%    \begin{macrocode}
\begin{document}
%    \end{macrocode}

% Declare a title page.
% Print title, part of document being processed and version flag:
%    \begin{macrocode}
\addtocounter{page}{-1}
\begin{center}
{\LARGE\bfseries{}childdoc example\par}
\vspace{1cm}
\ifchilddoc
\ifchilddocmanual part\else chapter\fi:
`\childdocname' of `\childdocjob'\par
\else
main document: `\childdocjob'\par
\fi
version: \version\par
\end{center}
\newpage
%    \end{macrocode}

% Manually include selected file,
% otherwise process as usual:
%    \begin{macrocode}
\ifchilddocmanual
\section*{part `\childdocname'}
\input{\childdocname}
\else
%    \end{macrocode}

% Include the two chapters:
%    \begin{macrocode}
\include{cdocsch1}
\include{cdocsch2}
%    \end{macrocode}

% Include the two parts unless only chapters should be displayed:
%    \begin{macrocode}
\ifchilddoc\else
\section{part three}
\input{cdocspt3}
\section{part four}
\input{cdocspt4}
\fi
%    \end{macrocode}

% Process as usual until here:
%    \begin{macrocode}
\fi
%    \end{macrocode}

% End of document body:
%    \begin{macrocode}
\end{document}
%    \end{macrocode}
%\iffalse
%</samplemain>
%\fi
%
% %%%%%%%%%%%%%%%%%%%%%%%%%%%%%%%%%%%%%%
% \paragraph{Chapter Include Files.}
%
% The include files are called |cdocsch1.tex| and |cdocsch2.tex|.
%
%\iffalse
%<*samplechap1|samplechap2>
%\fi

% Optional override for |\version| flag:
%    \begin{macrocode}
%%\providecommand{\version}{final}
%    \end{macrocode}

% Include the main document:
%    \begin{macrocode}
\input{childdoc.def}
\childdocof{cdocsamp}
%    \end{macrocode}

%\iffalse
%</samplechap1|samplechap2>
%\fi
%
%\iffalse
%<*samplechap1>
%\fi
% Some text for chapter 1:
%    \begin{macrocode}
\section{one}
some text in chapter one
%    \end{macrocode}

%\iffalse
%</samplechap1>
%\fi
% Some text for chapter 2:
%\iffalse
%<*samplechap2>
%\fi
%    \begin{macrocode}
\section{two}
more text in chapter two
%    \end{macrocode}

%\iffalse
%</samplechap2>
%\fi
%
% %%%%%%%%%%%%%%%%%%%%%%%%%%%%%%%%%%%%%%
% \paragraph{Part Include Files.}
%
% The include files are called |cdocspt3.tex| and |cdocspt4.tex|.
%
%\iffalse
%<*samplepart3|samplepart4>
%\fi

% Optional override for |\version| flag:
%    \begin{macrocode}
%%\providecommand{\version}{final}
%    \end{macrocode}

% Include the main document:
%    \begin{macrocode}
\input{childdoc.def}
\childdocby{cdocsamp}
%    \end{macrocode}

%\iffalse
%</samplepart3|samplepart4>
%\fi
%
%\iffalse
%<*samplepart3>
%\fi
% Some text for part 3:
%    \begin{macrocode}
some text in part three
%    \end{macrocode}

%\iffalse
%</samplepart3>
%\fi
% Some text for part 4:
%\iffalse
%<*samplepart4>
%\fi
%    \begin{macrocode}
more text in part four
%    \end{macrocode}

%\iffalse
%</samplepart4>
%\fi
%
% %%%%%%%%%%%%%%%%%%%%%%%%%%%%%%%%%%%%%%
% \paragraph{Forwarding for a Complete Draft.}
%
% The following forwarding file |cdocsdrf.tex|
% compiles the main document in draft mode:
%\iffalse
%<*sampledraft>
%\fi
%    \begin{macrocode}
\def\version{draft}
\input{childdoc.def}
\childdocforward{cdocsamp}
%    \end{macrocode}

%\iffalse
%</sampledraft>
%\fi
%
% %%%%%%%%%%%%%%%%%%%%%%%%%%%%%%%%%%%%%%
% \paragraph{Forwarding for Final Version of the Chapters.}
%
% The following forwarding files |cdocsfn1.tex| and |cdocsfn2.tex|
% (with identical content)
% compile the final versions of the child documents
% |cdocsch1.tex| and |cdocsch2.tex|, respectively:
%\iffalse
%<*samplefinal>
%\fi
%    \begin{macrocode}
\def\version{final}
\input{childdoc.def}
\childdocforwardprefix[cdocsamp]{cdocsfn}{cdocsch}
%    \end{macrocode}

%\iffalse
%</samplefinal>
%\fi
%
% %%%%%%%%%%%%%%%%%%%%%%%%%%%%%%%%%%%%%%
% \paragraph{Command Line Processing.}
%
% The following three command lines generate the output files
% |cdocscld|, |cdocscl1| and |cdocscl2|
% which should be identical to
% |cdocsdrf|, |cdocsch1| and |cdocsfn2|, respectively:
% \begin{center}
% \begin{tabular}{l}
% |latex -jobname cdocscld \|\\
% |  "\def\version{draft}\input{childdoc.def}\childdocforward{cdocsamp}"|\\
% |latex -jobname cdocscl1 \|\\
% |  "\input{childdoc.def}\childdocforward[cdocsamp]{cdocsch1}"|\\
% |latex -jobname cdocscl2 \|\\
% |  "\def\version{final}\input{childdoc.def}\childdocforward{cdocsch2}"|
% \end{tabular}
% \end{center}
% Note that the trailing backslash on each first line
% merely continues the input to the second line
% (for convenient cut ant paste).
% Furthermore, the command |latex| can be replaced by any
% of its alternative versions such as |pdflatex|.
%
% %%%%%%%%%%%%%%%%%%%%%%%%%%%%%%%%%%%%%%%%%%%%%%%%%%%%%%%%%%%%%%%%%%%%%%%%%%%%%%
% %%%%%%%%%%%%%%%%%%%%%%%%%%%%%%%%%%%%%%%%%%%%%%%%%%%%%%%%%%%%%%%%%%%%%%%%%%%%%%
% \section{Implementation}
%\iffalse
%<*package>
%\fi
%
% This section describes the definitions file |childdoc.def|.

% The definitions cannot be loaded using |\usepackage| or |\RequirePackage|
% which has a mechanism to prevent loading a style file more than once.
% When loading the definitions by means of |\input|
% multiple instances have to be prevented manually:
%\iffalse
%This code needs to be before the `\ProvidesFile' directive
%which is defined at the beginning of this file.
%Therefore it is also placed there and commented out here.
%</package>
%<*discard>
%\fi
%    \begin{macrocode}
\ifdefined\childdocmain\endinput\fi
%    \end{macrocode}
%\iffalse
%</discard>
%<*package>
%\fi
%
% \macro{\ifchilddoc}
% \macro{\ifchilddocmanual}
% The conditional |\ifchilddoc| tells whether a
% child (true) or main (false) document is being compiled.
% The conditional |\ifchilddocmanual| tells whether
% the |\includeonly| mechanism is used (false) or
% the selection of child files must be performed manually (true).
% The definitions initialise to false:
%    \begin{macrocode}
\newif\ifchilddoc
\newif\ifchilddocmanual
%    \end{macrocode}

% \macro{\childdocname}
% \macro{\childdocjob}
% The macro |\childdocname| stores the name of the main document
% to be compiled. The macro |\childdocjob| stores the name of
% the document on which the \LaTeX{} compiler was originally invoked.
% The content of |\jobname| cannot be compared
% to filenames specified in the source due to different catcodes.
% The following code rescans |\jobname|, stores the result
% in |\childdocname| and saves a copy in |\childdocjob|:
%    \begin{macrocode}
\edef\childdocname{\scantokens\expandafter{\jobname\noexpand}}
\let\childdocjob\childdocname
%    \end{macrocode}

% \macro{\childdocdisable}
% The macro |\childdocdisable| prevents the main file
% from being processed more than once.
% At this stage, the main document command |\childdocmain|
% is assumed to be called once again where it should do nothing.
% Any subsequent call to it should prevent
% a secondary processing of the main document
% It overwrites the forwarding commands
% |\childdocof| and |\childdocforward|
% with empty macros to prevent further inclusions of the main document:
%    \begin{macrocode}
\newcommand{\childdocdisable}
{
  \renewcommand{\childdocmain}[1]{\renewcommand{\childdocmain}[1]{\endinput}}
  \renewcommand{\childdocof}[1]{}
  \renewcommand{\childdocby}[2][]{}
  \renewcommand{\childdocforward}[2][]{}
  \renewcommand{\childdocdisable}{}
}
%    \end{macrocode}

% \macro{\childdocmain}
% The macro |\childdocmain| is to be called at the top of the main file
% with nothing or the main filename (without extension) as argument.
% First, it breaks loops.
% If the argument is not empty and does not match |\childdocname|
% (which is set by the first inclusion of |childdoc.def|),
% |\ifchilddoc| is set to true, |\includeonly| is applied to the child file
% and |\jobname| is set to the main file
% (for proper handling of |.aux| files):
%    \begin{macrocode}
\newcommand{\childdocmain}[1]
{
  \childdocdisable\childdocmain{}
  \if?#1?\else
    \begingroup
      \def\childdoctmp{#1}
      \ifx\childdoctmp\childdocname
        \def\childdoctmp{}
      \else
        \def\childdoctmp
        {
          \childdoctrue
          \includeonly{\childdocname}
          \def\childdocjob{#1}
          \def\jobname{#1}
        }
      \fi
      \expandafter
    \endgroup
    \childdoctmp
  \fi
}
%    \end{macrocode}

% \macro{\childdocof}
% The command |\childdocof| redirects
% compilation to the main file |#1|.
%    \begin{macrocode}
\newcommand{\childdocof}[1]
{
  \childdocdisable
  \childdoctrue
  \includeonly{\childdocname}
  \def\jobname{#1}
  \def\childdocjob{#1}
  \input{#1}
}
%    \end{macrocode}

% \macro{\childdocby}
% The command |\childdocby| ....
%    \begin{macrocode}
\newcommand{\childdocby}[2][]
{
  \childdocdisable
  \childdoctrue
  \childdocmanualtrue
  \if?#1?\else
    \def\jobname{#2}
  \fi
  \def\childdocjob{#2}
  \input{#2}
  \endinput
}
%    \end{macrocode}

% \macro{\childdocforward}
% The command |\childdocforward| redirects
% compilation to the main file or
% (if the optional argument is given) a child file.
% Parameters are set as if the main file
% or a child file starting with |\childdocof| was compiled.
% Then compilation is handed over to the main file:
%    \begin{macrocode}
\newcommand{\childdocforward}[2][]
{
  \begingroup
    \if?#1?
      \def\childdoctmp
      {
        \def\childdocname{#2}
        \def\childdocjob{#2}
        \def\jobname{#2}
        \input{#2}
        \endinput
      }
    \else
      \def\childdoctmp
      {
        \childdocdisable
        \def\childdocname{#2}
        \childdoctrue
        \includeonly{#2}
        \def\childdocjob{#1}
        \def\jobname{#1}
        \input{#1}
        \endinput
      }
    \fi
    \expandafter
  \endgroup
  \childdoctmp
}
%    \end{macrocode}

% \macro{\childdocforwardprefix}
% The command |\childdocforwardprefix| redirects
% compilation to the main or a child file by means of a pattern.
% The prefix |#1| in the current filename is replaced by |#2|
% and the suffix of the current filename is kept
% (it is assumed that the filename does not contain the substring `|~~~|'
% which is used as a delimiter).
% Compilation is handed over to the new file by |\childdocforward|:
%    \begin{macrocode}
\newcommand{\childdocforwardprefix}[3][]
{
  \begingroup
    \def\childdocextract #2##1~~~{\def\childdoctmp{\childdocforward[#1]{#3##1}}}
    \expandafter\childdocextract\childdocname~~~
    \expandafter
  \endgroup
  \childdoctmp
}
%    \end{macrocode}

% \macro{\childdoc}
% The deprecated macro |\childdoc| is a legacy version of |\childdocmain|:
%    \begin{macrocode}
\newcommand{\childdoc}{\childdocmain}
%    \end{macrocode}

% \macro{\childdocredirect}
% The deprecated macro |\childdocredirect| is a legacy version
% of |\childdocforward| and |\childdocforwardprefix|:
%    \begin{macrocode}
\newcommand{\childdocredirect}[2][]
{
  \begingroup
    \if?#1?
      \def\childdoctmp{\childdocforward{#2}}
    \else
      \def\childdoctmp{\childdocforwardprefix{#1}{#2}}
    \fi
    \expandafter
  \endgroup
  \childdoctmp
}
%    \end{macrocode}

%\iffalse
%</package>
%\fi
%
\endinput
|
and perform the replacements as outlined below.
Instead of |\childdocmain{|\textit{main}|}| add the following code
to the top of the main file:
%
\begin{center}
\begin{tabular}{l}
|\||ifdefined\childdocname\endinput\||fi\newif\ifchilddoc|\\
|\edef\childdocname{\scantokens\expandafter{\jobname\noexpand}}|\\
|\def\childdocmain{|\textit{main}|}\||ifx\childdocmain\childdocname\||else|\\
|\childdoctrue\includeonly{\childdocname}\let\jobname\childdocmain\||fi|\\
\end{tabular}
\end{center}
%
Instead of |\childdocof{|\textit{main}|}| just include the main file
at the top of each child file:
%
\begin{center}
|\input{|\textit{main}|}|
\end{center}
%
A simple redirection |\childdocforward{|\textit{dest}|}| is achieved by:
%
\begin{center}
|\def\jobname{|\textit{dest}|}\input{\jobname}|
\end{center}
%
The redirection with prefix
|\childdocforwardprefix[|\textit{prefix}|]{|\textit{dest}|}|
is accomplished by:
%
\begin{center}
\begin{tabular}{l}
|{\edef\jobname{\scantokens\expandafter{\jobname\noexpand}}|\\
|\def\redirectjob |\textit{prefix}|#1~~~{\gdef\jobname{|\textit{dest}|#1}}|\\
|\expandafter\redirectjob\jobname~~~}\input{\jobname}|
\end{tabular}
\end{center}

In an alternative approach,
child documents can be compiled by a specific command line
without additional code or specific definitions:
%
\begin{center}
|... -jobname "|\textit{target}|" "|[\textit{flags}]%
|\includeonly{|\textit{dest}|}\input{|\textit{main}|}"|
\end{center}
%

%%%%%%%%%%%%%%%%%%%%%%%%%%%%%%%%%%%%%%%%%%%%%%%%%%%%%%%%%%%%%%%%%%%%%%%%%%%%%%%%
%%%%%%%%%%%%%%%%%%%%%%%%%%%%%%%%%%%%%%%%%%%%%%%%%%%%%%%%%%%%%%%%%%%%%%%%%%%%%%%%
\section{Information}

%%%%%%%%%%%%%%%%%%%%%%%%%%%%%%%%%%%%%%%%%%%%%%%%%%%%%%%%%%%%%%%%%%%%%%%%%%%%%%%%
\subsection{Copyright}

Copyright \copyright{} 2017--2018 Niklas Beisert

This work may be distributed and/or modified under the
conditions of the \LaTeX{} Project Public License, either version 1.3
of this license or (at your option) any later version.
The latest version of this license is in
  \url{http://www.latex-project.org/lppl.txt}
and version 1.3 or later is part of all distributions of \LaTeX{}
version 2005/12/01 or later.

This work has the LPPL maintenance status `maintained'.

The Current Maintainer of this work is Niklas Beisert.

This work consists of the files |README.txt|, |childdoc.ins| and |childdoc.dtx|
as well as the derived files |childdoc.def|, |cdocsamp.tex|
with |cdocsch1.tex|, |cdocsch2.tex|, |cdocspt3.tex|, |cdocspt4.tex|,
|cdocsdrf.tex|, |cdocsfn1.tex|, |cdocsfn2.tex|
as well as |childdoc.pdf|.

%%%%%%%%%%%%%%%%%%%%%%%%%%%%%%%%%%%%%%%%%%%%%%%%%%%%%%%%%%%%%%%%%%%%%%%%%%%%%%%%
\subsection{Files and Installation}

The package consists of the files:
%
\begin{center}
\begin{tabular}{ll}
    |README.txt|   & readme file \\
    |childdoc.ins| & installation file \\
    |childdoc.dtx| & source file \\
    |childdoc.def| & definition file \\
    |cdocsamp.tex| & sample main file \\
    |cdocsch1.tex| & sample include file \\
    |cdocsch2.tex| & sample include file \\
    |cdocspt3.tex| & sample part file \\
    |cdocspt4.tex| & sample part file \\
    |cdocsdrf.tex| & sample redirection file \\
    |cdocsfn1.tex| & sample redirection file \\
    |cdocsfn2.tex| & sample redirection file \\
    |childdoc.pdf| & manual
\end{tabular}
\end{center}
%
The distribution consists of the files
|README.txt|, |childdoc.ins| and |childdoc.dtx|.
%
\begin{itemize}
\item
Run (pdf)\LaTeX{} on |childdoc.dtx|
to compile the manual |childdoc.pdf| (this file).
\item
Run \LaTeX{} on |childdoc.ins| to create the definitions file |childdoc.def|
and the sample |cdocsamp.tex| with include files
|cdocsch1.tex|, |cdocsch2.tex|, |cdocspt3.tex|, |cdocspt4.tex|,
|cdocsdrf.tex|, |cdocsfn1.tex|, |cdocsfn2.tex|.
Then copy the file |childdoc.def| to an appropriate directory of your \LaTeX{}
distribution, e.g.\ \textit{texmf-root}|/tex/latex/childdoc|.
\end{itemize}

%%%%%%%%%%%%%%%%%%%%%%%%%%%%%%%%%%%%%%%%%%%%%%%%%%%%%%%%%%%%%%%%%%%%%%%%%%%%%%%%
\subsection{Related CTAN Packages}

There are several other packages which offer a similar functionality:
%
\begin{itemize}
\item
The packages
\href{http://ctan.org/pkg/docmute}{\textsf{docmute}},
\href{http://ctan.org/pkg/includex}{\textsf{includex}} and
\href{http://ctan.org/pkg/standalone}{\textsf{standalone}}
provide commands to include only the document body of
a child file thus allowing both files to be compiled individually.
\item
The packages \href{http://ctan.org/pkg/subdocs}{\textsf{subdocs}}
and \href{http://ctan.org/pkg/subfiles}{\textsf{subfiles}}
provide structures in which the main and child documents can be
encapsulated and allowing them to be compiled individually.
The inclusion mechanism is different from the conventional |\include|.
\item
The package \href{http://ctan.org/pkg/combine}{\textsf{combine}}
is an elaborate solution to combine several documents into one.
\end{itemize}
%
See also the CTAN topic \href{http://ctan.org/topic/subdocs}{\textsf{subdocs}}
for further related packages.
The present package differs from the above solutions in that
a document structure constructed with the conventional |\include| mechanism
just needs two extra commands at the top of every file
such that all constituent files can be compiled individually.

%%%%%%%%%%%%%%%%%%%%%%%%%%%%%%%%%%%%%%%%%%%%%%%%%%%%%%%%%%%%%%%%%%%%%%%%%%%%%%%%
%\subsection{Feature Suggestions}
%
%The following is a list of features which may be useful for future
%versions of this package:
%%
%\begin{itemize}
%\item
%\ldots
%\end{itemize}

%%%%%%%%%%%%%%%%%%%%%%%%%%%%%%%%%%%%%%%%%%%%%%%%%%%%%%%%%%%%%%%%%%%%%%%%%%%%%%%%
\subsection{Revision History}

%%%%%%%%%%%%%%%%%%%%%%%%%%%%%%%%%%%%%%%%
\paragraph{v2.0:} 2018/12/30

\begin{itemize}
\item
immediate forward processing
\item
added |\childdocby| mechanism
\item
manual restructured
\end{itemize}

%%%%%%%%%%%%%%%%%%%%%%%%%%%%%%%%%%%%%%%%
\paragraph{v1.6:} 2018/01/17

\begin{itemize}
\item
application for development of include files
\item
corrections to manual
\end{itemize}

%%%%%%%%%%%%%%%%%%%%%%%%%%%%%%%%%%%%%%%%
\paragraph{v1.5:} 2017/05/21

\begin{itemize}
\item
more complete structuring introduced
\item
|\childdocof| introduced
\item
|\childdoc| renamed to |\childdocmain|
\item
|\childredirect| renamed to |\childdocforward| and |\childdocforwardprefix|
and functionality expanded
\end{itemize}

%%%%%%%%%%%%%%%%%%%%%%%%%%%%%%%%%%%%%%%%
\paragraph{v1.0:} 2017/04/27

\begin{itemize}
\item
manual and install package
\item
first version published on CTAN
\end{itemize}

%%%%%%%%%%%%%%%%%%%%%%%%%%%%%%%%%%%%%%%%
\paragraph{v0.6:} 2017/04/26

\begin{itemize}
\item
redirection mechanism added
\end{itemize}

%%%%%%%%%%%%%%%%%%%%%%%%%%%%%%%%%%%%%%%%
\paragraph{v0.5:} 2017/04/26

\begin{itemize}
\item
functionality in definition file
\end{itemize}


%%%%%%%%%%%%%%%%%%%%%%%%%%%%%%%%%%%%%%%%%%%%%%%%%%%%%%%%%%%%%%%%%%%%%%%%%%%%%%%%
%%%%%%%%%%%%%%%%%%%%%%%%%%%%%%%%%%%%%%%%%%%%%%%%%%%%%%%%%%%%%%%%%%%%%%%%%%%%%%%%
%%%%%%%%%%%%%%%%%%%%%%%%%%%%%%%%%%%%%%%%%%%%%%%%%%%%%%%%%%%%%%%%%%%%%%%%%%%%%%%%
\appendix

\settowidth\MacroIndent{\rmfamily\scriptsize 000\ }

 \DocInput{childdoc.dtx}

\end{document}
%</driver>
% \fi
%
% %%%%%%%%%%%%%%%%%%%%%%%%%%%%%%%%%%%%%%%%%%%%%%%%%%%%%%%%%%%%%%%%%%%%%%%%%%%%%%
% %%%%%%%%%%%%%%%%%%%%%%%%%%%%%%%%%%%%%%%%%%%%%%%%%%%%%%%%%%%%%%%%%%%%%%%%%%%%%%
% \section{Sample}
%\iffalse
%<*samplemain>
%\fi
%
% The following presents a sample document
% with two chapters, two parts, a title page,
% a compile flag as well as three forwarding files to set the flag.
% It consists of eight |.tex| files:
% \begin{center}
% \begin{tabular}{ll}
% |cdocsamp.tex|&main file\\
% |cdocsch1.tex|&include file for chapter 1\\
% |cdocsch2.tex|&include file for chapter 2\\
% |cdocspt3.tex|&include file for part 3\\
% |cdocspt4.tex|&include file for part 4\\
% |cdocsdrf.tex|&forwarding file for main file in draft mode\\
% |cdocsfi1.tex|&forwarding file for final version of chapter 1\\
% |cdocsfi2.tex|&forwarding file for final version of chapter 2\\
% \end{tabular}
% \end{center}
% Each of the eight files can be compiled directly by the \LaTeX{} compiler.
%
% %%%%%%%%%%%%%%%%%%%%%%%%%%%%%%%%%%%%%%
% \paragraph{Main File.}
%
% The main file is called |cdocsamp.tex|.
%
% Load the \textsf{childdoc} definitions and
% declare the filename for the main document:
%    \begin{macrocode}
% \iffalse
%
% childdoc.dtx Copyright (C) 2017-2018 Niklas Beisert
%
% This work may be distributed and/or modified under the
% conditions of the LaTeX Project Public License, either version 1.3
% of this license or (at your option) any later version.
% The latest version of this license is in
%   http://www.latex-project.org/lppl.txt
% and version 1.3 or later is part of all distributions of LaTeX
% version 2005/12/01 or later.
%
% This work has the LPPL maintenance status `maintained'.
%
% The Current Maintainer of this work is Niklas Beisert.
%
% This work consists of the files childdoc.dtx and childdoc.ins
% and the derived files childdoc.def and cdocsamp.tex with
% cdocsch1.tex, cdocsch2.tex, cdocsdrf.tex, cdocsfn1.tex, cdocsfn2.tex.
%
%<package>\ifdefined\childdocmain\endinput\fi
%<package>\ProvidesFile{childdoc.def}[2018/12/30 v2.0 child document driver]
%<samplemain>\ProvidesFile{cdocsamp.tex}[2018/12/30 v2.0 sample for childdoc]
%<*driver>
%\ProvidesFile{childdoc.drv}[2018/12/30 v2.0 childdoc reference manual file]
\PassOptionsToClass{10pt,a4paper}{article}
\documentclass{ltxdoc}

\usepackage[margin=35mm]{geometry}
\usepackage{hyperref}
\usepackage{hyperxmp}
\usepackage[usenames]{color}

\hypersetup{colorlinks=true}
\hypersetup{pdfstartview=FitH}
\hypersetup{pdfpagemode=UseNone}
\hypersetup{pdfsource={}}
\hypersetup{pdflang={en-UK}}
\hypersetup{pdfcopyright={Copyright 2017-2018 Niklas Beisert.
  This work may be distributed and/or modified under the
  conditions of the LaTeX Project Public License, either version 1.3
  of this license or (at your option) any later version.}}
\hypersetup{pdflicenseurl={http://www.latex-project.org/lppl.txt}}
\hypersetup{pdfcontactaddress={ETH Zurich, ITP, HIT K,
  Wolfgang-Pauli-Strasse 27}}
\hypersetup{pdfcontactpostcode={8093}}
\hypersetup{pdfcontactcity={Zurich}}
\hypersetup{pdfcontactcountry={Switzerland}}
\hypersetup{pdfcontactemail={nbeisert@itp.phys.ethz.ch}}
\hypersetup{pdfcontacturl={http://people.phys.ethz.ch/\xmptilde nbeisert/}}

\newcommand{\secref}[1]{\hyperref[#1]{section \ref*{#1}}}

\parskip1ex
\parindent0pt
\let\olditemize\itemize
\def\itemize{\olditemize\parskip0pt}

\begin{document}

\title{The \textsf{childdoc} Package}
\hypersetup{pdftitle={The childdoc Package}}
\author{Niklas Beisert\\[2ex]
  Institut f\"ur Theoretische Physik\\
  Eidgen\"ossische Technische Hochschule Z\"urich\\
  Wolfgang-Pauli-Strasse 27, 8093 Z\"urich, Switzerland\\[1ex]
  \href{mailto:nbeisert@itp.phys.ethz.ch}
  {\texttt{nbeisert@itp.phys.ethz.ch}}}
\hypersetup{pdfauthor={Niklas Beisert}}
\hypersetup{pdfsubject={Manual for the LaTeX2e Package childdoc}}
\date{30 December 2018, \textsf{v2.0}}
\maketitle

\begin{abstract}\noindent
\textsf{childdoc} is a \LaTeXe{} package
that enables the direct compilation
of document sections included by |\include|
to individual files.
\end{abstract}

\begingroup
\parskip0ex
\tableofcontents
\endgroup

%%%%%%%%%%%%%%%%%%%%%%%%%%%%%%%%%%%%%%%%%%%%%%%%%%%%%%%%%%%%%%%%%%%%%%%%%%%%%%%%
%%%%%%%%%%%%%%%%%%%%%%%%%%%%%%%%%%%%%%%%%%%%%%%%%%%%%%%%%%%%%%%%%%%%%%%%%%%%%%%%
\section{Introduction}

\LaTeX{} provides a mechanism to structure a large document (such as a book)
into a main file and several child files (containing the chapters)
using the |\include| command.
This mechanism is beneficial for documents
which span hundreds of pages in order to
make the source file(s) more manageable.
Moreover, compilation can be restricted to
selected child files by means of the |\includeonly| command.
The latter feature can be used to reduce the compilation time while editing
(this was significantly more useful in the earlier days of \LaTeX{})
or to generate a smaller document which is easier to navigate.
Another application of |\includeonly| is to generate
documents consisting of selected parts of the complete document.

However, there are a few drawbacks of the plain |\include| mechanism:
\begin{itemize}
\item
The child files cannot be compiled on their own,
they can only be compiled via the main file.
A naive editing environment
(such as a text editor with an option
to have the current file processed by \LaTeX)
may require one to switch to the main file before compiling;
attempting to compile the child file produces errors.
\item
The main file must be modified (each time)
to adjust the |\includeonly| command
to the present needs. This easily leaves the main file in a messy state.
\item
The generated document will always carry the filename
of the main document. This is inconvenient if
several child files are to be compiled and
to be kept for distribution.
\end{itemize}

The present package provides a simple interface
to make child files individually compilable by \LaTeX{}.
Compiling a child file then has the same effect as compiling
the main file with an |\includeonly| command
to select the appropriate child.
Moreover the generated document will carry the name of the child
rather than the main file.
This resolves all three above issues.

This feature is meant to make the editing of books,
thesis documents and lecture notes somewhat more convenient.
However, the package can also be used efficiently for
composing a series of documents (such as exercise sheets)
which are typically distributed individually.
It then assists the author in generating the individual documents
(potentially in different versions)
as well as a document containing the collected series.
Another application is in developing style files
or other kinds of included material
where compilation of the style file could redirect
to a sample or test file.

%%%%%%%%%%%%%%%%%%%%%%%%%%%%%%%%%%%%%%%%%%%%%%%%%%%%%%%%%%%%%%%%%%%%%%%%%%%%%%%%
%%%%%%%%%%%%%%%%%%%%%%%%%%%%%%%%%%%%%%%%%%%%%%%%%%%%%%%%%%%%%%%%%%%%%%%%%%%%%%%%
\section{Usage}

First of all, the package \textsf{childdoc} is \emph{not} a standard
\LaTeXe{} |.sty| style file! Therefore it needs to be invoked in
a non-standard way.

%%%%%%%%%%%%%%%%%%%%%%%%%%%%%%%%%%%%%%%%%%%%%%%%%%%%%%%%%%%%%%%%%%%%%%%%%%%%%%%%
\subsection{Included Files}
\label{sec:include}

%%%%%%%%%%%%%%%%%%%%%%%%%%%%%%%%%%%%%%%%
\DescribeMacro{\childdocmain}
To use the package, add the commands
\begin{center}
\begin{tabular}{l}
|\input{childdoc.def}|\\
|\childdocmain{}|\\
\end{tabular}
\end{center}
at the very top of the main \LaTeX{} file,
in particular \emph{before} the |\documentclass| statement!
The argument of |\childdocmain| should be left empty
(but it must be present).

%%%%%%%%%%%%%%%%%%%%%%%%%%%%%%%%%%%%%%%%
\DescribeMacro{\childdocof}
Furthermore, add the commands
\begin{center}
\begin{tabular}{l}
|\input{childdoc.def}|\\
|\childdocof{|\textit{main}|}|\\
\end{tabular}
\end{center}
at the top of every child file \textit{child}
which is included by |\include{|\textit{child}|}|
from within the main file
(or at least for those files to be compiled individually).
The argument \textit{main} must be the filename of the main file.

There are a couple of
considerations in setting up the main and child documents:

%%%%%%%%%%%%%%%%%%%%%%%%%%%%%%%%%%%%%%%%
\paragraph{Restrictions.}

Please note the following restrictions:
\begin{itemize}
\item
|\childdocmain| must be called with one argument \textit{main}
to ensure compatibility with earlier version of the package.
It must either be empty (|\childdocmain{}|)
or precisely match the filename of the main file in which it is specified.
See \secref{sec:detection} for further information.
\item
The filename \textit{main} must be specified without the |.tex| extension.
\item
The filename \textit{main} is case sensitive
(even in case-insensitive file systems)
due to internal string comparison.
\item
The argument \textit{main} should be fully expanded, it cannot be a macro.
\item
Subdirectories and special characters should be avoided in filenames.
\item
The command |\childdocmain{|\textit{main}|}| must be followed by a whitespace.
It should not be followed immediately by another command
or by a comment mark `|%|'.
This is because the \TeX{} parser reads the token immediately following
the argument of |\childdocmain| and puts it
at the beginning of every child section;
however, a white\-space is ignored.
\end{itemize}

%%%%%%%%%%%%%%%%%%%%%%%%%%%%%%%%%%%%%%%%
\paragraph{Content of Main File.}

It is advisable to place all content in the child files included by |\include|.
Any output contained in the main file will appear in all child documents
unless suppressed manually;
it cannot be suppressed automatically by the |\includeonly| directive
and thus should normally be avoided.
A method to include some content in the main file
by means of conditional processing is described in \secref{sec:conditional}.

%%%%%%%%%%%%%%%%%%%%%%%%%%%%%%%%%%%%%%%%
\paragraph{Page Numbering.}

When only a part of the document is compiled,
the appropriate numbering of pages
(as well as other status parameters)
is determined from the |.aux| files.
The latter contain information from previous passes.
However this information needs to propagate through
all intermediate child documents.
Therefore the page numbering in child documents may well
be inconsistent until the complete document is compiled at least once.

A useful (if unconventional) way to always ensure a consistent
page numbering is to restart the numbering in each child document
and denote the pages by `\textit{child}|.|\textit{page}'
where \textit{child} represents the chapter/section number of the child file.
This can be achieved by the command
|\numberwithin{page}{|\textit{child}|}|
of the \textsf{amsmath} package
where \textit{child} can be |chapter| or |section|
depending on the chosen structuring.
Alternatively, one can modify the macro |\thepage| appropriately
and reset the counter |page| at the start of each child file.

%%%%%%%%%%%%%%%%%%%%%%%%%%%%%%%%%%%%%%%%%%%%%%%%%%%%%%%%%%%%%%%%%%%%%%%%%%%%%%%%
\subsection{Conditional Processing}
\label{sec:conditional}

The package provides a mechanism to compile different versions
of a document. To customise the versions further some conditional processing
can come in handy to distinguish which version is being compiled.
The package provides two macros to describe the compilation context:

%%%%%%%%%%%%%%%%%%%%%%%%%%%%%%%%%%%%%%%%
\DescribeMacro{\ifchilddoc}
The conditional |\ifchilddoc| distinguishes between the compilation of
child documents and the main document:
%
\begin{center}
|\ifchilddoc |\textit{child-code}| |[|\||else |\textit{main-code}]| \||fi|
\end{center}

%%%%%%%%%%%%%%%%%%%%%%%%%%%%%%%%%%%%%%%%
\DescribeMacro{\childdocname}
\DescribeMacro{\childdocjob}
The macro |\childdocname| contains the filename (without extension)
of the main or child file being processed.
Note that |\childdocjob| will always contain the name of the main file.

%%%%%%%%%%%%%%%%%%%%%%%%%%%%%%%%%%%%%%%%
\paragraph{Title Page.}

Conditional processing can be used to include a title or banner page
in the main document when proper precautions are taken.
Importantly, the code in the main file should ensure that the page counter
(as well as other status parameters which are stored in the |.aux| files)
takes the same value after the conditional processing.
Otherwise the page numbers may take divergent values
depending on which part is compiled.

For example, a title page could be declared by:
%
\begin{center}
\begin{tabular}{l}
|\ifchilddoc\||else|\\
|\addtocounter{page}{-1}|\\
\textit{code for title page}\\
|\newpage|\\
|\||fi|
\end{tabular}
\end{center}
%
A banner page for the child documents can be generated by:
%
\begin{center}
\begin{tabular}{l}
|\ifchilddoc|\\
|\addtocounter{page}{-1}|\\
\textit{code for banner page}\\
|\newpage|\\
|\||fi|
\end{tabular}
\end{center}
%
Here one could write a message such as:
\begin{center}
|This is the part \childdocname{} of \childdocjob{}.|
\end{center}

%%%%%%%%%%%%%%%%%%%%%%%%%%%%%%%%%%%%%%%%%%%%%%%%%%%%%%%%%%%%%%%%%%%%%%%%%%%%%%%%
\subsection{Flags}
\label{sec:flags}

The package makes it easy to generate different versions
of the main or child documents.
To this end compilation flags can be defined
and assigned different default values.
They will be particularly useful in conjunction
with the forwarding mechanism described in \secref{sec:forward}.

For example, it may be useful to have a flag |\version|
which can be set to |draft| or |final|.
The document source will contain some conditional code
depending on the value of |\version|.
Suppose further, the flag should default to |final| for the main file
and to |draft| for child files
which is a natural assignment for editing the document.
This is achieved by placing the following code
in the preamble of the main document
(below the |\childdocmain| directive):
%
\begin{center}
\begin{tabular}{l}
|\ifchilddoc|\\
|\providecommand{\version}{draft}|\\
|\||else|\\
|\providecommand{\version}{final}|\\
|\||fi|
\end{tabular}
\end{center}
%
The definition by |\providecommand| makes sure
that previous definitions are not overwritten.
Further statements |\providecommand{\version}{...}|
can thus be added before the above code to override it.

For the main file, one might add a line
(between |\childdocmain| and the above block)
%
\begin{center}
|%\ifchilddoc\||else\providecommand{\version}{draft}\||fi|
\end{center}
%
which can be uncommented to produce a draft version.
Likewise one can add a line to the very top of a child file
(above the |\childdocof{|\textit{main}|}| directive)
%
\begin{center}
|%\providecommand{\version}{final}|
\end{center}
%
which can be uncommented to produce the final version of this child document.

%%%%%%%%%%%%%%%%%%%%%%%%%%%%%%%%%%%%%%%%%%%%%%%%%%%%%%%%%%%%%%%%%%%%%%%%%%%%%%%%
\subsection{Forwarding}
\label{sec:forward}

Different versions of the main or child documents
using compilation flags as described in \secref{sec:flags}
can be (permanently) stored in different files
for convenient compilation, viewing and distribution.
To this end, the package defines a command
to pass on compilation to a different file:

%%%%%%%%%%%%%%%%%%%%%%%%%%%%%%%%%%%%%%%%
\DescribeMacro{\childdocforward}
The command |\childdocforward| redirects processing to
another source file:
%
\begin{center}
\begin{tabular}{l}
|\input{childdoc.def}|\\
|\childdocforward[|\textit{main}|]{|\textit{dest}|}|\\
\end{tabular}
\end{center}
%
The argument \textit{dest} is the destination file
(without extension).
It should be the main file or one of the child files.
Note that further \textsf{childdoc} directives
such as |\childdocof| and |\childdocforward|
in the indicated file will be processed in this form.
The optional argument \textit{main}
passes on directly to the main file \textit{main}
while pretending to compile the child \textit{dest}.
This form behaves as if \textit{dest}
issues |\childdocof{|\textit{main}|}| right away,
and no further \textsf{childdoc} directives will be processed.

%%%%%%%%%%%%%%%%%%%%%%%%%%%%%%%%%%%%%%%%
\DescribeMacro{\...prefix}
In the alternative form |\childdocforwardprefix|,
%
\begin{center}
\begin{tabular}{l}
|\input{childdoc.def}|\\
|\childdocforwardprefix[|\textit{main}|]{|\textit{prefix}|}{|\textit{dest}|}|
\end{tabular}
\end{center}
%
the destination file is determined by a pattern
depending on the current file:
To make this work, the current file must be called
`{\textit{prefix}\hspace{0.2em}\textit{suffix}}'
with \textit{prefix} matching precisely the argument.
Processing is then passed on to the file
`{\textit{dest}\hspace{0.2em}\textit{suffix}}'.
Surely, the same effect is achieved by
directly specifying the
argument `{\textit{dest}\hspace{0.2em}\textit{suffix}}'
in the first form.
However, that requires to set up a different file
for each child. With the alternative form of the command
all these files can have exactly the same content
which simplifies setting them up and maintaining them.

For example, the following file |draft.tex|
with a compilation flag |\version| as described in \secref{sec:flags}
compiles the main document as a draft:
%
\begin{center}
\begin{tabular}{l}
|\def\version{draft}|\\
|\input{childdoc.def}|\\
|\childdocforward{|\textit{main}|}|
\end{tabular}
\end{center}
%
Likewise, the following files |final|\textit{nn}|.tex|
compile the final version of the child document
|child|\textit{nn}|.tex|:
%
\begin{center}
\begin{tabular}{l}
|\def\version{final}|\\
|\input{childdoc.def}|\\
|\childdocforwardprefix{final}{child}|
\end{tabular}
\end{center}
%

Note that when several versions of a main file and/or of each child file
are to be generated, it may be convenient to set up a |Makefile| or
shell script to automatise the process.

%%%%%%%%%%%%%%%%%%%%%%%%%%%%%%%%%%%%%%%%%%%%%%%%%%%%%%%%%%%%%%%%%%%%%%%%%%%%%%%%
\subsection{Command Line Processing}
\label{sec:commandline}

The effect of redirection files can also be achieved by invoking
the \LaTeX{} compiler with a more elaborate command line.
Most conveniently this should be done as part
of a shell script or a |Makefile|.

When using \textsf{childdoc} in the main file, the following
command lines effectively perform a redirection
(note that depending on the shell being used,
backslashes may have to be doubled: `|\|' $\to$ `|\\|'):
%
\begin{center}
|... -jobname "|\textit{target}|" |\\|"|[\textit{flags}]%
|\input{childdoc.def}\childdocforward[|\textit{main}|]{|\textit{dest}|}"|
\end{center}
%
Here \textit{target} is the name of the output file,
\textit{main} is the name of the main file
and \textit{dest} is the name of the main or child file to be processed
(all filenames without extensions).
The optional argument \textit{main} can be omitted
if \textit{main} matches \textit{dest}.
Optionally, compilation \textit{flags} can be defined via |\def| commands.
This command line makes the \TeX{} engine believe
it is compiling the file \textit{target}
whose content is specified as the latter parameter.
The provided code then forwards the processing to
\textit{main} or \textit{dest} as described in \secref{sec:forward}.

%%%%%%%%%%%%%%%%%%%%%%%%%%%%%%%%%%%%%%%%%%%%%%%%%%%%%%%%%%%%%%%%%%%%%%%%%%%%%%%%
\subsection{Include by Input}
\label{sec:input}

Including child documents by |\include| has some restrictions by design.
Most notably, the content of a child document always occupies
its own set of pages; pages cannot be shared between child documents.
Usually, this behaviour makes perfect sense
because each child document contain an essential part of the document.
However, in some situations it may be desirable to compose
a document from a collection of parts
without having mandatory page breaks between then.
For this case, the package
provides a mechanism to include parts
by |\input| which can also be processed individually.
However, by construction this mechanism
requires manual handling of the content to be output.

%%%%%%%%%%%%%%%%%%%%%%%%%%%%%%%%%%%%%%%%
\DescribeMacro{\ifchilddocmanual}
The main file should be prepared as usual, see \secref{sec:include}.
However, the document body must make a distinction
between processing of an individual part and of the main document, e.g.:
%
\begin{center}
\begin{tabular}{l}
|\ifchilddocmanual|\\
|\input{\childdocname}|\\
|\||else|\\
\textit{document body with }|\input{|\textit{part}|}|\\
|\||fi|
\end{tabular}
\end{center}
%
The conditional |\ifchilddocmanual| is true whenever
a part to be included by |\input| is being compiled,
and the name of the part is stored in |\childdocname|.

%%%%%%%%%%%%%%%%%%%%%%%%%%%%%%%%%%%%%%%%
\DescribeMacro{\childdocby}
Each part to be included by |\input| should start with:
%
\begin{center}
\begin{tabular}{l}
|\input{childdoc.def}|\\
|\childdocby{|\textit{main}|}|\\
\end{tabular}
\end{center}
%
The directive |\childdocby| is similar to |\childdocof|
described in \secref{sec:include},
but the subsequent selection of content must be done manually.
To that end, both |\ifchilddoc| and |\ifchilddocmanual|
will be true upon processing of a part,
and the name of the part is stored in |\childdocname|.
Note that |\jobname| will be set to the filename of the current part
so that each part receives an individual |.aux| file
that does not interfere with the |.aux| file(s) of the main document.
This behaviour can be altered by the alternative form
|\childdocby[*]{|\textit{main}|}| (with a non-empty optional argument)
which uses the |.aux| file of the main document
by setting |\jobname| to \textit{main}.

%%%%%%%%%%%%%%%%%%%%%%%%%%%%%%%%%%%%%%%%%%%%%%%%%%%%%%%%%%%%%%%%%%%%%%%%%%%%%%%%
\subsection{Driver Development}
\label{sec:driver}

The \textsf{childdoc} mechanism can also be use for the development
of definition files such as \LaTeX{} styles or classes.
This case differs from the above setup with multiple parts
included by |\include| in that no |\includeonly| should be invoked.
This can be achieved by starting the include file
(before |\ProvidesPackage|) with:
%
\begin{center}
\begin{tabular}{l}
|\input{childdoc.def}|\\
|\childdocforward{|\textit{main}|}|\\
\end{tabular}
\end{center}
%
or alternatively with:
%
\begin{center}
\begin{tabular}{l}
|\input{childdoc.def}|\\
|\childdocby{|\textit{main}|}|\\
\end{tabular}
\end{center}
%
Both forms have slightly different effects as described above.
The main file is prepared as usual, see \secref{sec:include}.

%%%%%%%%%%%%%%%%%%%%%%%%%%%%%%%%%%%%%%%%%%%%%%%%%%%%%%%%%%%%%%%%%%%%%%%%%%%%%%%%
\subsection{Legacy Detection}
\label{sec:detection}

The directive |\childdocmain| in the main file can detect
whether the complete document or merely a child is to be compiled
even without using the directive |\childdocof|.
This method is deprecated because it is less robust
and there is no compelling reason to use it;
it is merely provided for backward compatibility
and it may be removed in future versions.

If the detection mechanism is to be used,
it is mandatory to correctly specify
the filename of the main file as the argument of |\childdocmain|:
%
\begin{center}
\begin{tabular}{l}
|\input{childdoc.def}|\\
|\childdocmain{|\textit{main}|}|\\
\end{tabular}
\end{center}
%
If |\jobname| does not match the argument \textit{main} of |\childdocmain|,
it is assumed that |\jobname| points to the child file to be compiled.
When using |\childdocmain| with the main file specified as argument,
it suffices to start a child file
with just |\input{|\textit{main}|}|
without loading of the package and using |\childdocof|.
If instead all processing is done
with the appropriate \textsf{childdoc} directives,
the argument of \textit{main} of |\childdocmain| can be empty.

An alternative version of the command line processing described
in \secref{sec:commandline} using the detection mechanism reads:
%
\begin{center}
|... -jobname "|\textit{target}|" "|[\textit{flags}]%
[|\def\jobname{|\textit{dest}|}|]|\input{|\textit{main}|}"|
\end{center}

%%%%%%%%%%%%%%%%%%%%%%%%%%%%%%%%%%%%%%%%%%%%%%%%%%%%%%%%%%%%%%%%%%%%%%%%%%%%%%%%
\subsection{Manual Code}
\label{sec:manual}

In case one cannot be certain whether the definitions file |childdoc.def|
is installed on the target \TeX{} distribution
and one prefers not to ship it,
it is conceivable to paste a few relevant commands into the sources.

To that end, drop all statements |\input{childdoc.def}|
and perform the replacements as outlined below.
Instead of |\childdocmain{|\textit{main}|}| add the following code
to the top of the main file:
%
\begin{center}
\begin{tabular}{l}
|\||ifdefined\childdocname\endinput\||fi\newif\ifchilddoc|\\
|\edef\childdocname{\scantokens\expandafter{\jobname\noexpand}}|\\
|\def\childdocmain{|\textit{main}|}\||ifx\childdocmain\childdocname\||else|\\
|\childdoctrue\includeonly{\childdocname}\let\jobname\childdocmain\||fi|\\
\end{tabular}
\end{center}
%
Instead of |\childdocof{|\textit{main}|}| just include the main file
at the top of each child file:
%
\begin{center}
|\input{|\textit{main}|}|
\end{center}
%
A simple redirection |\childdocforward{|\textit{dest}|}| is achieved by:
%
\begin{center}
|\def\jobname{|\textit{dest}|}\input{\jobname}|
\end{center}
%
The redirection with prefix
|\childdocforwardprefix[|\textit{prefix}|]{|\textit{dest}|}|
is accomplished by:
%
\begin{center}
\begin{tabular}{l}
|{\edef\jobname{\scantokens\expandafter{\jobname\noexpand}}|\\
|\def\redirectjob |\textit{prefix}|#1~~~{\gdef\jobname{|\textit{dest}|#1}}|\\
|\expandafter\redirectjob\jobname~~~}\input{\jobname}|
\end{tabular}
\end{center}

In an alternative approach,
child documents can be compiled by a specific command line
without additional code or specific definitions:
%
\begin{center}
|... -jobname "|\textit{target}|" "|[\textit{flags}]%
|\includeonly{|\textit{dest}|}\input{|\textit{main}|}"|
\end{center}
%

%%%%%%%%%%%%%%%%%%%%%%%%%%%%%%%%%%%%%%%%%%%%%%%%%%%%%%%%%%%%%%%%%%%%%%%%%%%%%%%%
%%%%%%%%%%%%%%%%%%%%%%%%%%%%%%%%%%%%%%%%%%%%%%%%%%%%%%%%%%%%%%%%%%%%%%%%%%%%%%%%
\section{Information}

%%%%%%%%%%%%%%%%%%%%%%%%%%%%%%%%%%%%%%%%%%%%%%%%%%%%%%%%%%%%%%%%%%%%%%%%%%%%%%%%
\subsection{Copyright}

Copyright \copyright{} 2017--2018 Niklas Beisert

This work may be distributed and/or modified under the
conditions of the \LaTeX{} Project Public License, either version 1.3
of this license or (at your option) any later version.
The latest version of this license is in
  \url{http://www.latex-project.org/lppl.txt}
and version 1.3 or later is part of all distributions of \LaTeX{}
version 2005/12/01 or later.

This work has the LPPL maintenance status `maintained'.

The Current Maintainer of this work is Niklas Beisert.

This work consists of the files |README.txt|, |childdoc.ins| and |childdoc.dtx|
as well as the derived files |childdoc.def|, |cdocsamp.tex|
with |cdocsch1.tex|, |cdocsch2.tex|, |cdocspt3.tex|, |cdocspt4.tex|,
|cdocsdrf.tex|, |cdocsfn1.tex|, |cdocsfn2.tex|
as well as |childdoc.pdf|.

%%%%%%%%%%%%%%%%%%%%%%%%%%%%%%%%%%%%%%%%%%%%%%%%%%%%%%%%%%%%%%%%%%%%%%%%%%%%%%%%
\subsection{Files and Installation}

The package consists of the files:
%
\begin{center}
\begin{tabular}{ll}
    |README.txt|   & readme file \\
    |childdoc.ins| & installation file \\
    |childdoc.dtx| & source file \\
    |childdoc.def| & definition file \\
    |cdocsamp.tex| & sample main file \\
    |cdocsch1.tex| & sample include file \\
    |cdocsch2.tex| & sample include file \\
    |cdocspt3.tex| & sample part file \\
    |cdocspt4.tex| & sample part file \\
    |cdocsdrf.tex| & sample redirection file \\
    |cdocsfn1.tex| & sample redirection file \\
    |cdocsfn2.tex| & sample redirection file \\
    |childdoc.pdf| & manual
\end{tabular}
\end{center}
%
The distribution consists of the files
|README.txt|, |childdoc.ins| and |childdoc.dtx|.
%
\begin{itemize}
\item
Run (pdf)\LaTeX{} on |childdoc.dtx|
to compile the manual |childdoc.pdf| (this file).
\item
Run \LaTeX{} on |childdoc.ins| to create the definitions file |childdoc.def|
and the sample |cdocsamp.tex| with include files
|cdocsch1.tex|, |cdocsch2.tex|, |cdocspt3.tex|, |cdocspt4.tex|,
|cdocsdrf.tex|, |cdocsfn1.tex|, |cdocsfn2.tex|.
Then copy the file |childdoc.def| to an appropriate directory of your \LaTeX{}
distribution, e.g.\ \textit{texmf-root}|/tex/latex/childdoc|.
\end{itemize}

%%%%%%%%%%%%%%%%%%%%%%%%%%%%%%%%%%%%%%%%%%%%%%%%%%%%%%%%%%%%%%%%%%%%%%%%%%%%%%%%
\subsection{Related CTAN Packages}

There are several other packages which offer a similar functionality:
%
\begin{itemize}
\item
The packages
\href{http://ctan.org/pkg/docmute}{\textsf{docmute}},
\href{http://ctan.org/pkg/includex}{\textsf{includex}} and
\href{http://ctan.org/pkg/standalone}{\textsf{standalone}}
provide commands to include only the document body of
a child file thus allowing both files to be compiled individually.
\item
The packages \href{http://ctan.org/pkg/subdocs}{\textsf{subdocs}}
and \href{http://ctan.org/pkg/subfiles}{\textsf{subfiles}}
provide structures in which the main and child documents can be
encapsulated and allowing them to be compiled individually.
The inclusion mechanism is different from the conventional |\include|.
\item
The package \href{http://ctan.org/pkg/combine}{\textsf{combine}}
is an elaborate solution to combine several documents into one.
\end{itemize}
%
See also the CTAN topic \href{http://ctan.org/topic/subdocs}{\textsf{subdocs}}
for further related packages.
The present package differs from the above solutions in that
a document structure constructed with the conventional |\include| mechanism
just needs two extra commands at the top of every file
such that all constituent files can be compiled individually.

%%%%%%%%%%%%%%%%%%%%%%%%%%%%%%%%%%%%%%%%%%%%%%%%%%%%%%%%%%%%%%%%%%%%%%%%%%%%%%%%
%\subsection{Feature Suggestions}
%
%The following is a list of features which may be useful for future
%versions of this package:
%%
%\begin{itemize}
%\item
%\ldots
%\end{itemize}

%%%%%%%%%%%%%%%%%%%%%%%%%%%%%%%%%%%%%%%%%%%%%%%%%%%%%%%%%%%%%%%%%%%%%%%%%%%%%%%%
\subsection{Revision History}

%%%%%%%%%%%%%%%%%%%%%%%%%%%%%%%%%%%%%%%%
\paragraph{v2.0:} 2018/12/30

\begin{itemize}
\item
immediate forward processing
\item
added |\childdocby| mechanism
\item
manual restructured
\end{itemize}

%%%%%%%%%%%%%%%%%%%%%%%%%%%%%%%%%%%%%%%%
\paragraph{v1.6:} 2018/01/17

\begin{itemize}
\item
application for development of include files
\item
corrections to manual
\end{itemize}

%%%%%%%%%%%%%%%%%%%%%%%%%%%%%%%%%%%%%%%%
\paragraph{v1.5:} 2017/05/21

\begin{itemize}
\item
more complete structuring introduced
\item
|\childdocof| introduced
\item
|\childdoc| renamed to |\childdocmain|
\item
|\childredirect| renamed to |\childdocforward| and |\childdocforwardprefix|
and functionality expanded
\end{itemize}

%%%%%%%%%%%%%%%%%%%%%%%%%%%%%%%%%%%%%%%%
\paragraph{v1.0:} 2017/04/27

\begin{itemize}
\item
manual and install package
\item
first version published on CTAN
\end{itemize}

%%%%%%%%%%%%%%%%%%%%%%%%%%%%%%%%%%%%%%%%
\paragraph{v0.6:} 2017/04/26

\begin{itemize}
\item
redirection mechanism added
\end{itemize}

%%%%%%%%%%%%%%%%%%%%%%%%%%%%%%%%%%%%%%%%
\paragraph{v0.5:} 2017/04/26

\begin{itemize}
\item
functionality in definition file
\end{itemize}


%%%%%%%%%%%%%%%%%%%%%%%%%%%%%%%%%%%%%%%%%%%%%%%%%%%%%%%%%%%%%%%%%%%%%%%%%%%%%%%%
%%%%%%%%%%%%%%%%%%%%%%%%%%%%%%%%%%%%%%%%%%%%%%%%%%%%%%%%%%%%%%%%%%%%%%%%%%%%%%%%
%%%%%%%%%%%%%%%%%%%%%%%%%%%%%%%%%%%%%%%%%%%%%%%%%%%%%%%%%%%%%%%%%%%%%%%%%%%%%%%%
\appendix

\settowidth\MacroIndent{\rmfamily\scriptsize 000\ }

 \DocInput{childdoc.dtx}

\end{document}
%</driver>
% \fi
%
% %%%%%%%%%%%%%%%%%%%%%%%%%%%%%%%%%%%%%%%%%%%%%%%%%%%%%%%%%%%%%%%%%%%%%%%%%%%%%%
% %%%%%%%%%%%%%%%%%%%%%%%%%%%%%%%%%%%%%%%%%%%%%%%%%%%%%%%%%%%%%%%%%%%%%%%%%%%%%%
% \section{Sample}
%\iffalse
%<*samplemain>
%\fi
%
% The following presents a sample document
% with two chapters, two parts, a title page,
% a compile flag as well as three forwarding files to set the flag.
% It consists of eight |.tex| files:
% \begin{center}
% \begin{tabular}{ll}
% |cdocsamp.tex|&main file\\
% |cdocsch1.tex|&include file for chapter 1\\
% |cdocsch2.tex|&include file for chapter 2\\
% |cdocspt3.tex|&include file for part 3\\
% |cdocspt4.tex|&include file for part 4\\
% |cdocsdrf.tex|&forwarding file for main file in draft mode\\
% |cdocsfi1.tex|&forwarding file for final version of chapter 1\\
% |cdocsfi2.tex|&forwarding file for final version of chapter 2\\
% \end{tabular}
% \end{center}
% Each of the eight files can be compiled directly by the \LaTeX{} compiler.
%
% %%%%%%%%%%%%%%%%%%%%%%%%%%%%%%%%%%%%%%
% \paragraph{Main File.}
%
% The main file is called |cdocsamp.tex|.
%
% Load the \textsf{childdoc} definitions and
% declare the filename for the main document:
%    \begin{macrocode}
\input{childdoc.def}
\childdocmain{}
%    \end{macrocode}

% Optional override for |\version| flag:
%    \begin{macrocode}
%%\ifchilddoc\else\providecommand{\version}{draft}\fi
%    \end{macrocode}

% Define the default values for the |\version| flag
% (|final| for the main file and |draft| for childs):
%    \begin{macrocode}
\ifchilddoc
\providecommand{\version}{draft}
\else
\providecommand{\version}{final}
\fi
%    \end{macrocode}

% Load the standard document class:
%    \begin{macrocode}
\documentclass[12pt]{article}
%    \end{macrocode}

% Start the document body:
%    \begin{macrocode}
\begin{document}
%    \end{macrocode}

% Declare a title page.
% Print title, part of document being processed and version flag:
%    \begin{macrocode}
\addtocounter{page}{-1}
\begin{center}
{\LARGE\bfseries{}childdoc example\par}
\vspace{1cm}
\ifchilddoc
\ifchilddocmanual part\else chapter\fi:
`\childdocname' of `\childdocjob'\par
\else
main document: `\childdocjob'\par
\fi
version: \version\par
\end{center}
\newpage
%    \end{macrocode}

% Manually include selected file,
% otherwise process as usual:
%    \begin{macrocode}
\ifchilddocmanual
\section*{part `\childdocname'}
\input{\childdocname}
\else
%    \end{macrocode}

% Include the two chapters:
%    \begin{macrocode}
\include{cdocsch1}
\include{cdocsch2}
%    \end{macrocode}

% Include the two parts unless only chapters should be displayed:
%    \begin{macrocode}
\ifchilddoc\else
\section{part three}
\input{cdocspt3}
\section{part four}
\input{cdocspt4}
\fi
%    \end{macrocode}

% Process as usual until here:
%    \begin{macrocode}
\fi
%    \end{macrocode}

% End of document body:
%    \begin{macrocode}
\end{document}
%    \end{macrocode}
%\iffalse
%</samplemain>
%\fi
%
% %%%%%%%%%%%%%%%%%%%%%%%%%%%%%%%%%%%%%%
% \paragraph{Chapter Include Files.}
%
% The include files are called |cdocsch1.tex| and |cdocsch2.tex|.
%
%\iffalse
%<*samplechap1|samplechap2>
%\fi

% Optional override for |\version| flag:
%    \begin{macrocode}
%%\providecommand{\version}{final}
%    \end{macrocode}

% Include the main document:
%    \begin{macrocode}
\input{childdoc.def}
\childdocof{cdocsamp}
%    \end{macrocode}

%\iffalse
%</samplechap1|samplechap2>
%\fi
%
%\iffalse
%<*samplechap1>
%\fi
% Some text for chapter 1:
%    \begin{macrocode}
\section{one}
some text in chapter one
%    \end{macrocode}

%\iffalse
%</samplechap1>
%\fi
% Some text for chapter 2:
%\iffalse
%<*samplechap2>
%\fi
%    \begin{macrocode}
\section{two}
more text in chapter two
%    \end{macrocode}

%\iffalse
%</samplechap2>
%\fi
%
% %%%%%%%%%%%%%%%%%%%%%%%%%%%%%%%%%%%%%%
% \paragraph{Part Include Files.}
%
% The include files are called |cdocspt3.tex| and |cdocspt4.tex|.
%
%\iffalse
%<*samplepart3|samplepart4>
%\fi

% Optional override for |\version| flag:
%    \begin{macrocode}
%%\providecommand{\version}{final}
%    \end{macrocode}

% Include the main document:
%    \begin{macrocode}
\input{childdoc.def}
\childdocby{cdocsamp}
%    \end{macrocode}

%\iffalse
%</samplepart3|samplepart4>
%\fi
%
%\iffalse
%<*samplepart3>
%\fi
% Some text for part 3:
%    \begin{macrocode}
some text in part three
%    \end{macrocode}

%\iffalse
%</samplepart3>
%\fi
% Some text for part 4:
%\iffalse
%<*samplepart4>
%\fi
%    \begin{macrocode}
more text in part four
%    \end{macrocode}

%\iffalse
%</samplepart4>
%\fi
%
% %%%%%%%%%%%%%%%%%%%%%%%%%%%%%%%%%%%%%%
% \paragraph{Forwarding for a Complete Draft.}
%
% The following forwarding file |cdocsdrf.tex|
% compiles the main document in draft mode:
%\iffalse
%<*sampledraft>
%\fi
%    \begin{macrocode}
\def\version{draft}
\input{childdoc.def}
\childdocforward{cdocsamp}
%    \end{macrocode}

%\iffalse
%</sampledraft>
%\fi
%
% %%%%%%%%%%%%%%%%%%%%%%%%%%%%%%%%%%%%%%
% \paragraph{Forwarding for Final Version of the Chapters.}
%
% The following forwarding files |cdocsfn1.tex| and |cdocsfn2.tex|
% (with identical content)
% compile the final versions of the child documents
% |cdocsch1.tex| and |cdocsch2.tex|, respectively:
%\iffalse
%<*samplefinal>
%\fi
%    \begin{macrocode}
\def\version{final}
\input{childdoc.def}
\childdocforwardprefix[cdocsamp]{cdocsfn}{cdocsch}
%    \end{macrocode}

%\iffalse
%</samplefinal>
%\fi
%
% %%%%%%%%%%%%%%%%%%%%%%%%%%%%%%%%%%%%%%
% \paragraph{Command Line Processing.}
%
% The following three command lines generate the output files
% |cdocscld|, |cdocscl1| and |cdocscl2|
% which should be identical to
% |cdocsdrf|, |cdocsch1| and |cdocsfn2|, respectively:
% \begin{center}
% \begin{tabular}{l}
% |latex -jobname cdocscld \|\\
% |  "\def\version{draft}\input{childdoc.def}\childdocforward{cdocsamp}"|\\
% |latex -jobname cdocscl1 \|\\
% |  "\input{childdoc.def}\childdocforward[cdocsamp]{cdocsch1}"|\\
% |latex -jobname cdocscl2 \|\\
% |  "\def\version{final}\input{childdoc.def}\childdocforward{cdocsch2}"|
% \end{tabular}
% \end{center}
% Note that the trailing backslash on each first line
% merely continues the input to the second line
% (for convenient cut ant paste).
% Furthermore, the command |latex| can be replaced by any
% of its alternative versions such as |pdflatex|.
%
% %%%%%%%%%%%%%%%%%%%%%%%%%%%%%%%%%%%%%%%%%%%%%%%%%%%%%%%%%%%%%%%%%%%%%%%%%%%%%%
% %%%%%%%%%%%%%%%%%%%%%%%%%%%%%%%%%%%%%%%%%%%%%%%%%%%%%%%%%%%%%%%%%%%%%%%%%%%%%%
% \section{Implementation}
%\iffalse
%<*package>
%\fi
%
% This section describes the definitions file |childdoc.def|.

% The definitions cannot be loaded using |\usepackage| or |\RequirePackage|
% which has a mechanism to prevent loading a style file more than once.
% When loading the definitions by means of |\input|
% multiple instances have to be prevented manually:
%\iffalse
%This code needs to be before the `\ProvidesFile' directive
%which is defined at the beginning of this file.
%Therefore it is also placed there and commented out here.
%</package>
%<*discard>
%\fi
%    \begin{macrocode}
\ifdefined\childdocmain\endinput\fi
%    \end{macrocode}
%\iffalse
%</discard>
%<*package>
%\fi
%
% \macro{\ifchilddoc}
% \macro{\ifchilddocmanual}
% The conditional |\ifchilddoc| tells whether a
% child (true) or main (false) document is being compiled.
% The conditional |\ifchilddocmanual| tells whether
% the |\includeonly| mechanism is used (false) or
% the selection of child files must be performed manually (true).
% The definitions initialise to false:
%    \begin{macrocode}
\newif\ifchilddoc
\newif\ifchilddocmanual
%    \end{macrocode}

% \macro{\childdocname}
% \macro{\childdocjob}
% The macro |\childdocname| stores the name of the main document
% to be compiled. The macro |\childdocjob| stores the name of
% the document on which the \LaTeX{} compiler was originally invoked.
% The content of |\jobname| cannot be compared
% to filenames specified in the source due to different catcodes.
% The following code rescans |\jobname|, stores the result
% in |\childdocname| and saves a copy in |\childdocjob|:
%    \begin{macrocode}
\edef\childdocname{\scantokens\expandafter{\jobname\noexpand}}
\let\childdocjob\childdocname
%    \end{macrocode}

% \macro{\childdocdisable}
% The macro |\childdocdisable| prevents the main file
% from being processed more than once.
% At this stage, the main document command |\childdocmain|
% is assumed to be called once again where it should do nothing.
% Any subsequent call to it should prevent
% a secondary processing of the main document
% It overwrites the forwarding commands
% |\childdocof| and |\childdocforward|
% with empty macros to prevent further inclusions of the main document:
%    \begin{macrocode}
\newcommand{\childdocdisable}
{
  \renewcommand{\childdocmain}[1]{\renewcommand{\childdocmain}[1]{\endinput}}
  \renewcommand{\childdocof}[1]{}
  \renewcommand{\childdocby}[2][]{}
  \renewcommand{\childdocforward}[2][]{}
  \renewcommand{\childdocdisable}{}
}
%    \end{macrocode}

% \macro{\childdocmain}
% The macro |\childdocmain| is to be called at the top of the main file
% with nothing or the main filename (without extension) as argument.
% First, it breaks loops.
% If the argument is not empty and does not match |\childdocname|
% (which is set by the first inclusion of |childdoc.def|),
% |\ifchilddoc| is set to true, |\includeonly| is applied to the child file
% and |\jobname| is set to the main file
% (for proper handling of |.aux| files):
%    \begin{macrocode}
\newcommand{\childdocmain}[1]
{
  \childdocdisable\childdocmain{}
  \if?#1?\else
    \begingroup
      \def\childdoctmp{#1}
      \ifx\childdoctmp\childdocname
        \def\childdoctmp{}
      \else
        \def\childdoctmp
        {
          \childdoctrue
          \includeonly{\childdocname}
          \def\childdocjob{#1}
          \def\jobname{#1}
        }
      \fi
      \expandafter
    \endgroup
    \childdoctmp
  \fi
}
%    \end{macrocode}

% \macro{\childdocof}
% The command |\childdocof| redirects
% compilation to the main file |#1|.
%    \begin{macrocode}
\newcommand{\childdocof}[1]
{
  \childdocdisable
  \childdoctrue
  \includeonly{\childdocname}
  \def\jobname{#1}
  \def\childdocjob{#1}
  \input{#1}
}
%    \end{macrocode}

% \macro{\childdocby}
% The command |\childdocby| ....
%    \begin{macrocode}
\newcommand{\childdocby}[2][]
{
  \childdocdisable
  \childdoctrue
  \childdocmanualtrue
  \if?#1?\else
    \def\jobname{#2}
  \fi
  \def\childdocjob{#2}
  \input{#2}
  \endinput
}
%    \end{macrocode}

% \macro{\childdocforward}
% The command |\childdocforward| redirects
% compilation to the main file or
% (if the optional argument is given) a child file.
% Parameters are set as if the main file
% or a child file starting with |\childdocof| was compiled.
% Then compilation is handed over to the main file:
%    \begin{macrocode}
\newcommand{\childdocforward}[2][]
{
  \begingroup
    \if?#1?
      \def\childdoctmp
      {
        \def\childdocname{#2}
        \def\childdocjob{#2}
        \def\jobname{#2}
        \input{#2}
        \endinput
      }
    \else
      \def\childdoctmp
      {
        \childdocdisable
        \def\childdocname{#2}
        \childdoctrue
        \includeonly{#2}
        \def\childdocjob{#1}
        \def\jobname{#1}
        \input{#1}
        \endinput
      }
    \fi
    \expandafter
  \endgroup
  \childdoctmp
}
%    \end{macrocode}

% \macro{\childdocforwardprefix}
% The command |\childdocforwardprefix| redirects
% compilation to the main or a child file by means of a pattern.
% The prefix |#1| in the current filename is replaced by |#2|
% and the suffix of the current filename is kept
% (it is assumed that the filename does not contain the substring `|~~~|'
% which is used as a delimiter).
% Compilation is handed over to the new file by |\childdocforward|:
%    \begin{macrocode}
\newcommand{\childdocforwardprefix}[3][]
{
  \begingroup
    \def\childdocextract #2##1~~~{\def\childdoctmp{\childdocforward[#1]{#3##1}}}
    \expandafter\childdocextract\childdocname~~~
    \expandafter
  \endgroup
  \childdoctmp
}
%    \end{macrocode}

% \macro{\childdoc}
% The deprecated macro |\childdoc| is a legacy version of |\childdocmain|:
%    \begin{macrocode}
\newcommand{\childdoc}{\childdocmain}
%    \end{macrocode}

% \macro{\childdocredirect}
% The deprecated macro |\childdocredirect| is a legacy version
% of |\childdocforward| and |\childdocforwardprefix|:
%    \begin{macrocode}
\newcommand{\childdocredirect}[2][]
{
  \begingroup
    \if?#1?
      \def\childdoctmp{\childdocforward{#2}}
    \else
      \def\childdoctmp{\childdocforwardprefix{#1}{#2}}
    \fi
    \expandafter
  \endgroup
  \childdoctmp
}
%    \end{macrocode}

%\iffalse
%</package>
%\fi
%
\endinput

\childdocmain{}
%    \end{macrocode}

% Optional override for |\version| flag:
%    \begin{macrocode}
%%\ifchilddoc\else\providecommand{\version}{draft}\fi
%    \end{macrocode}

% Define the default values for the |\version| flag
% (|final| for the main file and |draft| for childs):
%    \begin{macrocode}
\ifchilddoc
\providecommand{\version}{draft}
\else
\providecommand{\version}{final}
\fi
%    \end{macrocode}

% Load the standard document class:
%    \begin{macrocode}
\documentclass[12pt]{article}
%    \end{macrocode}

% Start the document body:
%    \begin{macrocode}
\begin{document}
%    \end{macrocode}

% Declare a title page.
% Print title, part of document being processed and version flag:
%    \begin{macrocode}
\addtocounter{page}{-1}
\begin{center}
{\LARGE\bfseries{}childdoc example\par}
\vspace{1cm}
\ifchilddoc
\ifchilddocmanual part\else chapter\fi:
`\childdocname' of `\childdocjob'\par
\else
main document: `\childdocjob'\par
\fi
version: \version\par
\end{center}
\newpage
%    \end{macrocode}

% Manually include selected file,
% otherwise process as usual:
%    \begin{macrocode}
\ifchilddocmanual
\section*{part `\childdocname'}
\input{\childdocname}
\else
%    \end{macrocode}

% Include the two chapters:
%    \begin{macrocode}
\include{cdocsch1}
\include{cdocsch2}
%    \end{macrocode}

% Include the two parts unless only chapters should be displayed:
%    \begin{macrocode}
\ifchilddoc\else
\section{part three}
\input{cdocspt3}
\section{part four}
\input{cdocspt4}
\fi
%    \end{macrocode}

% Process as usual until here:
%    \begin{macrocode}
\fi
%    \end{macrocode}

% End of document body:
%    \begin{macrocode}
\end{document}
%    \end{macrocode}
%\iffalse
%</samplemain>
%\fi
%
% %%%%%%%%%%%%%%%%%%%%%%%%%%%%%%%%%%%%%%
% \paragraph{Chapter Include Files.}
%
% The include files are called |cdocsch1.tex| and |cdocsch2.tex|.
%
%\iffalse
%<*samplechap1|samplechap2>
%\fi

% Optional override for |\version| flag:
%    \begin{macrocode}
%%\providecommand{\version}{final}
%    \end{macrocode}

% Include the main document:
%    \begin{macrocode}
% \iffalse
%
% childdoc.dtx Copyright (C) 2017-2018 Niklas Beisert
%
% This work may be distributed and/or modified under the
% conditions of the LaTeX Project Public License, either version 1.3
% of this license or (at your option) any later version.
% The latest version of this license is in
%   http://www.latex-project.org/lppl.txt
% and version 1.3 or later is part of all distributions of LaTeX
% version 2005/12/01 or later.
%
% This work has the LPPL maintenance status `maintained'.
%
% The Current Maintainer of this work is Niklas Beisert.
%
% This work consists of the files childdoc.dtx and childdoc.ins
% and the derived files childdoc.def and cdocsamp.tex with
% cdocsch1.tex, cdocsch2.tex, cdocsdrf.tex, cdocsfn1.tex, cdocsfn2.tex.
%
%<package>\ifdefined\childdocmain\endinput\fi
%<package>\ProvidesFile{childdoc.def}[2018/12/30 v2.0 child document driver]
%<samplemain>\ProvidesFile{cdocsamp.tex}[2018/12/30 v2.0 sample for childdoc]
%<*driver>
%\ProvidesFile{childdoc.drv}[2018/12/30 v2.0 childdoc reference manual file]
\PassOptionsToClass{10pt,a4paper}{article}
\documentclass{ltxdoc}

\usepackage[margin=35mm]{geometry}
\usepackage{hyperref}
\usepackage{hyperxmp}
\usepackage[usenames]{color}

\hypersetup{colorlinks=true}
\hypersetup{pdfstartview=FitH}
\hypersetup{pdfpagemode=UseNone}
\hypersetup{pdfsource={}}
\hypersetup{pdflang={en-UK}}
\hypersetup{pdfcopyright={Copyright 2017-2018 Niklas Beisert.
  This work may be distributed and/or modified under the
  conditions of the LaTeX Project Public License, either version 1.3
  of this license or (at your option) any later version.}}
\hypersetup{pdflicenseurl={http://www.latex-project.org/lppl.txt}}
\hypersetup{pdfcontactaddress={ETH Zurich, ITP, HIT K,
  Wolfgang-Pauli-Strasse 27}}
\hypersetup{pdfcontactpostcode={8093}}
\hypersetup{pdfcontactcity={Zurich}}
\hypersetup{pdfcontactcountry={Switzerland}}
\hypersetup{pdfcontactemail={nbeisert@itp.phys.ethz.ch}}
\hypersetup{pdfcontacturl={http://people.phys.ethz.ch/\xmptilde nbeisert/}}

\newcommand{\secref}[1]{\hyperref[#1]{section \ref*{#1}}}

\parskip1ex
\parindent0pt
\let\olditemize\itemize
\def\itemize{\olditemize\parskip0pt}

\begin{document}

\title{The \textsf{childdoc} Package}
\hypersetup{pdftitle={The childdoc Package}}
\author{Niklas Beisert\\[2ex]
  Institut f\"ur Theoretische Physik\\
  Eidgen\"ossische Technische Hochschule Z\"urich\\
  Wolfgang-Pauli-Strasse 27, 8093 Z\"urich, Switzerland\\[1ex]
  \href{mailto:nbeisert@itp.phys.ethz.ch}
  {\texttt{nbeisert@itp.phys.ethz.ch}}}
\hypersetup{pdfauthor={Niklas Beisert}}
\hypersetup{pdfsubject={Manual for the LaTeX2e Package childdoc}}
\date{30 December 2018, \textsf{v2.0}}
\maketitle

\begin{abstract}\noindent
\textsf{childdoc} is a \LaTeXe{} package
that enables the direct compilation
of document sections included by |\include|
to individual files.
\end{abstract}

\begingroup
\parskip0ex
\tableofcontents
\endgroup

%%%%%%%%%%%%%%%%%%%%%%%%%%%%%%%%%%%%%%%%%%%%%%%%%%%%%%%%%%%%%%%%%%%%%%%%%%%%%%%%
%%%%%%%%%%%%%%%%%%%%%%%%%%%%%%%%%%%%%%%%%%%%%%%%%%%%%%%%%%%%%%%%%%%%%%%%%%%%%%%%
\section{Introduction}

\LaTeX{} provides a mechanism to structure a large document (such as a book)
into a main file and several child files (containing the chapters)
using the |\include| command.
This mechanism is beneficial for documents
which span hundreds of pages in order to
make the source file(s) more manageable.
Moreover, compilation can be restricted to
selected child files by means of the |\includeonly| command.
The latter feature can be used to reduce the compilation time while editing
(this was significantly more useful in the earlier days of \LaTeX{})
or to generate a smaller document which is easier to navigate.
Another application of |\includeonly| is to generate
documents consisting of selected parts of the complete document.

However, there are a few drawbacks of the plain |\include| mechanism:
\begin{itemize}
\item
The child files cannot be compiled on their own,
they can only be compiled via the main file.
A naive editing environment
(such as a text editor with an option
to have the current file processed by \LaTeX)
may require one to switch to the main file before compiling;
attempting to compile the child file produces errors.
\item
The main file must be modified (each time)
to adjust the |\includeonly| command
to the present needs. This easily leaves the main file in a messy state.
\item
The generated document will always carry the filename
of the main document. This is inconvenient if
several child files are to be compiled and
to be kept for distribution.
\end{itemize}

The present package provides a simple interface
to make child files individually compilable by \LaTeX{}.
Compiling a child file then has the same effect as compiling
the main file with an |\includeonly| command
to select the appropriate child.
Moreover the generated document will carry the name of the child
rather than the main file.
This resolves all three above issues.

This feature is meant to make the editing of books,
thesis documents and lecture notes somewhat more convenient.
However, the package can also be used efficiently for
composing a series of documents (such as exercise sheets)
which are typically distributed individually.
It then assists the author in generating the individual documents
(potentially in different versions)
as well as a document containing the collected series.
Another application is in developing style files
or other kinds of included material
where compilation of the style file could redirect
to a sample or test file.

%%%%%%%%%%%%%%%%%%%%%%%%%%%%%%%%%%%%%%%%%%%%%%%%%%%%%%%%%%%%%%%%%%%%%%%%%%%%%%%%
%%%%%%%%%%%%%%%%%%%%%%%%%%%%%%%%%%%%%%%%%%%%%%%%%%%%%%%%%%%%%%%%%%%%%%%%%%%%%%%%
\section{Usage}

First of all, the package \textsf{childdoc} is \emph{not} a standard
\LaTeXe{} |.sty| style file! Therefore it needs to be invoked in
a non-standard way.

%%%%%%%%%%%%%%%%%%%%%%%%%%%%%%%%%%%%%%%%%%%%%%%%%%%%%%%%%%%%%%%%%%%%%%%%%%%%%%%%
\subsection{Included Files}
\label{sec:include}

%%%%%%%%%%%%%%%%%%%%%%%%%%%%%%%%%%%%%%%%
\DescribeMacro{\childdocmain}
To use the package, add the commands
\begin{center}
\begin{tabular}{l}
|\input{childdoc.def}|\\
|\childdocmain{}|\\
\end{tabular}
\end{center}
at the very top of the main \LaTeX{} file,
in particular \emph{before} the |\documentclass| statement!
The argument of |\childdocmain| should be left empty
(but it must be present).

%%%%%%%%%%%%%%%%%%%%%%%%%%%%%%%%%%%%%%%%
\DescribeMacro{\childdocof}
Furthermore, add the commands
\begin{center}
\begin{tabular}{l}
|\input{childdoc.def}|\\
|\childdocof{|\textit{main}|}|\\
\end{tabular}
\end{center}
at the top of every child file \textit{child}
which is included by |\include{|\textit{child}|}|
from within the main file
(or at least for those files to be compiled individually).
The argument \textit{main} must be the filename of the main file.

There are a couple of
considerations in setting up the main and child documents:

%%%%%%%%%%%%%%%%%%%%%%%%%%%%%%%%%%%%%%%%
\paragraph{Restrictions.}

Please note the following restrictions:
\begin{itemize}
\item
|\childdocmain| must be called with one argument \textit{main}
to ensure compatibility with earlier version of the package.
It must either be empty (|\childdocmain{}|)
or precisely match the filename of the main file in which it is specified.
See \secref{sec:detection} for further information.
\item
The filename \textit{main} must be specified without the |.tex| extension.
\item
The filename \textit{main} is case sensitive
(even in case-insensitive file systems)
due to internal string comparison.
\item
The argument \textit{main} should be fully expanded, it cannot be a macro.
\item
Subdirectories and special characters should be avoided in filenames.
\item
The command |\childdocmain{|\textit{main}|}| must be followed by a whitespace.
It should not be followed immediately by another command
or by a comment mark `|%|'.
This is because the \TeX{} parser reads the token immediately following
the argument of |\childdocmain| and puts it
at the beginning of every child section;
however, a white\-space is ignored.
\end{itemize}

%%%%%%%%%%%%%%%%%%%%%%%%%%%%%%%%%%%%%%%%
\paragraph{Content of Main File.}

It is advisable to place all content in the child files included by |\include|.
Any output contained in the main file will appear in all child documents
unless suppressed manually;
it cannot be suppressed automatically by the |\includeonly| directive
and thus should normally be avoided.
A method to include some content in the main file
by means of conditional processing is described in \secref{sec:conditional}.

%%%%%%%%%%%%%%%%%%%%%%%%%%%%%%%%%%%%%%%%
\paragraph{Page Numbering.}

When only a part of the document is compiled,
the appropriate numbering of pages
(as well as other status parameters)
is determined from the |.aux| files.
The latter contain information from previous passes.
However this information needs to propagate through
all intermediate child documents.
Therefore the page numbering in child documents may well
be inconsistent until the complete document is compiled at least once.

A useful (if unconventional) way to always ensure a consistent
page numbering is to restart the numbering in each child document
and denote the pages by `\textit{child}|.|\textit{page}'
where \textit{child} represents the chapter/section number of the child file.
This can be achieved by the command
|\numberwithin{page}{|\textit{child}|}|
of the \textsf{amsmath} package
where \textit{child} can be |chapter| or |section|
depending on the chosen structuring.
Alternatively, one can modify the macro |\thepage| appropriately
and reset the counter |page| at the start of each child file.

%%%%%%%%%%%%%%%%%%%%%%%%%%%%%%%%%%%%%%%%%%%%%%%%%%%%%%%%%%%%%%%%%%%%%%%%%%%%%%%%
\subsection{Conditional Processing}
\label{sec:conditional}

The package provides a mechanism to compile different versions
of a document. To customise the versions further some conditional processing
can come in handy to distinguish which version is being compiled.
The package provides two macros to describe the compilation context:

%%%%%%%%%%%%%%%%%%%%%%%%%%%%%%%%%%%%%%%%
\DescribeMacro{\ifchilddoc}
The conditional |\ifchilddoc| distinguishes between the compilation of
child documents and the main document:
%
\begin{center}
|\ifchilddoc |\textit{child-code}| |[|\||else |\textit{main-code}]| \||fi|
\end{center}

%%%%%%%%%%%%%%%%%%%%%%%%%%%%%%%%%%%%%%%%
\DescribeMacro{\childdocname}
\DescribeMacro{\childdocjob}
The macro |\childdocname| contains the filename (without extension)
of the main or child file being processed.
Note that |\childdocjob| will always contain the name of the main file.

%%%%%%%%%%%%%%%%%%%%%%%%%%%%%%%%%%%%%%%%
\paragraph{Title Page.}

Conditional processing can be used to include a title or banner page
in the main document when proper precautions are taken.
Importantly, the code in the main file should ensure that the page counter
(as well as other status parameters which are stored in the |.aux| files)
takes the same value after the conditional processing.
Otherwise the page numbers may take divergent values
depending on which part is compiled.

For example, a title page could be declared by:
%
\begin{center}
\begin{tabular}{l}
|\ifchilddoc\||else|\\
|\addtocounter{page}{-1}|\\
\textit{code for title page}\\
|\newpage|\\
|\||fi|
\end{tabular}
\end{center}
%
A banner page for the child documents can be generated by:
%
\begin{center}
\begin{tabular}{l}
|\ifchilddoc|\\
|\addtocounter{page}{-1}|\\
\textit{code for banner page}\\
|\newpage|\\
|\||fi|
\end{tabular}
\end{center}
%
Here one could write a message such as:
\begin{center}
|This is the part \childdocname{} of \childdocjob{}.|
\end{center}

%%%%%%%%%%%%%%%%%%%%%%%%%%%%%%%%%%%%%%%%%%%%%%%%%%%%%%%%%%%%%%%%%%%%%%%%%%%%%%%%
\subsection{Flags}
\label{sec:flags}

The package makes it easy to generate different versions
of the main or child documents.
To this end compilation flags can be defined
and assigned different default values.
They will be particularly useful in conjunction
with the forwarding mechanism described in \secref{sec:forward}.

For example, it may be useful to have a flag |\version|
which can be set to |draft| or |final|.
The document source will contain some conditional code
depending on the value of |\version|.
Suppose further, the flag should default to |final| for the main file
and to |draft| for child files
which is a natural assignment for editing the document.
This is achieved by placing the following code
in the preamble of the main document
(below the |\childdocmain| directive):
%
\begin{center}
\begin{tabular}{l}
|\ifchilddoc|\\
|\providecommand{\version}{draft}|\\
|\||else|\\
|\providecommand{\version}{final}|\\
|\||fi|
\end{tabular}
\end{center}
%
The definition by |\providecommand| makes sure
that previous definitions are not overwritten.
Further statements |\providecommand{\version}{...}|
can thus be added before the above code to override it.

For the main file, one might add a line
(between |\childdocmain| and the above block)
%
\begin{center}
|%\ifchilddoc\||else\providecommand{\version}{draft}\||fi|
\end{center}
%
which can be uncommented to produce a draft version.
Likewise one can add a line to the very top of a child file
(above the |\childdocof{|\textit{main}|}| directive)
%
\begin{center}
|%\providecommand{\version}{final}|
\end{center}
%
which can be uncommented to produce the final version of this child document.

%%%%%%%%%%%%%%%%%%%%%%%%%%%%%%%%%%%%%%%%%%%%%%%%%%%%%%%%%%%%%%%%%%%%%%%%%%%%%%%%
\subsection{Forwarding}
\label{sec:forward}

Different versions of the main or child documents
using compilation flags as described in \secref{sec:flags}
can be (permanently) stored in different files
for convenient compilation, viewing and distribution.
To this end, the package defines a command
to pass on compilation to a different file:

%%%%%%%%%%%%%%%%%%%%%%%%%%%%%%%%%%%%%%%%
\DescribeMacro{\childdocforward}
The command |\childdocforward| redirects processing to
another source file:
%
\begin{center}
\begin{tabular}{l}
|\input{childdoc.def}|\\
|\childdocforward[|\textit{main}|]{|\textit{dest}|}|\\
\end{tabular}
\end{center}
%
The argument \textit{dest} is the destination file
(without extension).
It should be the main file or one of the child files.
Note that further \textsf{childdoc} directives
such as |\childdocof| and |\childdocforward|
in the indicated file will be processed in this form.
The optional argument \textit{main}
passes on directly to the main file \textit{main}
while pretending to compile the child \textit{dest}.
This form behaves as if \textit{dest}
issues |\childdocof{|\textit{main}|}| right away,
and no further \textsf{childdoc} directives will be processed.

%%%%%%%%%%%%%%%%%%%%%%%%%%%%%%%%%%%%%%%%
\DescribeMacro{\...prefix}
In the alternative form |\childdocforwardprefix|,
%
\begin{center}
\begin{tabular}{l}
|\input{childdoc.def}|\\
|\childdocforwardprefix[|\textit{main}|]{|\textit{prefix}|}{|\textit{dest}|}|
\end{tabular}
\end{center}
%
the destination file is determined by a pattern
depending on the current file:
To make this work, the current file must be called
`{\textit{prefix}\hspace{0.2em}\textit{suffix}}'
with \textit{prefix} matching precisely the argument.
Processing is then passed on to the file
`{\textit{dest}\hspace{0.2em}\textit{suffix}}'.
Surely, the same effect is achieved by
directly specifying the
argument `{\textit{dest}\hspace{0.2em}\textit{suffix}}'
in the first form.
However, that requires to set up a different file
for each child. With the alternative form of the command
all these files can have exactly the same content
which simplifies setting them up and maintaining them.

For example, the following file |draft.tex|
with a compilation flag |\version| as described in \secref{sec:flags}
compiles the main document as a draft:
%
\begin{center}
\begin{tabular}{l}
|\def\version{draft}|\\
|\input{childdoc.def}|\\
|\childdocforward{|\textit{main}|}|
\end{tabular}
\end{center}
%
Likewise, the following files |final|\textit{nn}|.tex|
compile the final version of the child document
|child|\textit{nn}|.tex|:
%
\begin{center}
\begin{tabular}{l}
|\def\version{final}|\\
|\input{childdoc.def}|\\
|\childdocforwardprefix{final}{child}|
\end{tabular}
\end{center}
%

Note that when several versions of a main file and/or of each child file
are to be generated, it may be convenient to set up a |Makefile| or
shell script to automatise the process.

%%%%%%%%%%%%%%%%%%%%%%%%%%%%%%%%%%%%%%%%%%%%%%%%%%%%%%%%%%%%%%%%%%%%%%%%%%%%%%%%
\subsection{Command Line Processing}
\label{sec:commandline}

The effect of redirection files can also be achieved by invoking
the \LaTeX{} compiler with a more elaborate command line.
Most conveniently this should be done as part
of a shell script or a |Makefile|.

When using \textsf{childdoc} in the main file, the following
command lines effectively perform a redirection
(note that depending on the shell being used,
backslashes may have to be doubled: `|\|' $\to$ `|\\|'):
%
\begin{center}
|... -jobname "|\textit{target}|" |\\|"|[\textit{flags}]%
|\input{childdoc.def}\childdocforward[|\textit{main}|]{|\textit{dest}|}"|
\end{center}
%
Here \textit{target} is the name of the output file,
\textit{main} is the name of the main file
and \textit{dest} is the name of the main or child file to be processed
(all filenames without extensions).
The optional argument \textit{main} can be omitted
if \textit{main} matches \textit{dest}.
Optionally, compilation \textit{flags} can be defined via |\def| commands.
This command line makes the \TeX{} engine believe
it is compiling the file \textit{target}
whose content is specified as the latter parameter.
The provided code then forwards the processing to
\textit{main} or \textit{dest} as described in \secref{sec:forward}.

%%%%%%%%%%%%%%%%%%%%%%%%%%%%%%%%%%%%%%%%%%%%%%%%%%%%%%%%%%%%%%%%%%%%%%%%%%%%%%%%
\subsection{Include by Input}
\label{sec:input}

Including child documents by |\include| has some restrictions by design.
Most notably, the content of a child document always occupies
its own set of pages; pages cannot be shared between child documents.
Usually, this behaviour makes perfect sense
because each child document contain an essential part of the document.
However, in some situations it may be desirable to compose
a document from a collection of parts
without having mandatory page breaks between then.
For this case, the package
provides a mechanism to include parts
by |\input| which can also be processed individually.
However, by construction this mechanism
requires manual handling of the content to be output.

%%%%%%%%%%%%%%%%%%%%%%%%%%%%%%%%%%%%%%%%
\DescribeMacro{\ifchilddocmanual}
The main file should be prepared as usual, see \secref{sec:include}.
However, the document body must make a distinction
between processing of an individual part and of the main document, e.g.:
%
\begin{center}
\begin{tabular}{l}
|\ifchilddocmanual|\\
|\input{\childdocname}|\\
|\||else|\\
\textit{document body with }|\input{|\textit{part}|}|\\
|\||fi|
\end{tabular}
\end{center}
%
The conditional |\ifchilddocmanual| is true whenever
a part to be included by |\input| is being compiled,
and the name of the part is stored in |\childdocname|.

%%%%%%%%%%%%%%%%%%%%%%%%%%%%%%%%%%%%%%%%
\DescribeMacro{\childdocby}
Each part to be included by |\input| should start with:
%
\begin{center}
\begin{tabular}{l}
|\input{childdoc.def}|\\
|\childdocby{|\textit{main}|}|\\
\end{tabular}
\end{center}
%
The directive |\childdocby| is similar to |\childdocof|
described in \secref{sec:include},
but the subsequent selection of content must be done manually.
To that end, both |\ifchilddoc| and |\ifchilddocmanual|
will be true upon processing of a part,
and the name of the part is stored in |\childdocname|.
Note that |\jobname| will be set to the filename of the current part
so that each part receives an individual |.aux| file
that does not interfere with the |.aux| file(s) of the main document.
This behaviour can be altered by the alternative form
|\childdocby[*]{|\textit{main}|}| (with a non-empty optional argument)
which uses the |.aux| file of the main document
by setting |\jobname| to \textit{main}.

%%%%%%%%%%%%%%%%%%%%%%%%%%%%%%%%%%%%%%%%%%%%%%%%%%%%%%%%%%%%%%%%%%%%%%%%%%%%%%%%
\subsection{Driver Development}
\label{sec:driver}

The \textsf{childdoc} mechanism can also be use for the development
of definition files such as \LaTeX{} styles or classes.
This case differs from the above setup with multiple parts
included by |\include| in that no |\includeonly| should be invoked.
This can be achieved by starting the include file
(before |\ProvidesPackage|) with:
%
\begin{center}
\begin{tabular}{l}
|\input{childdoc.def}|\\
|\childdocforward{|\textit{main}|}|\\
\end{tabular}
\end{center}
%
or alternatively with:
%
\begin{center}
\begin{tabular}{l}
|\input{childdoc.def}|\\
|\childdocby{|\textit{main}|}|\\
\end{tabular}
\end{center}
%
Both forms have slightly different effects as described above.
The main file is prepared as usual, see \secref{sec:include}.

%%%%%%%%%%%%%%%%%%%%%%%%%%%%%%%%%%%%%%%%%%%%%%%%%%%%%%%%%%%%%%%%%%%%%%%%%%%%%%%%
\subsection{Legacy Detection}
\label{sec:detection}

The directive |\childdocmain| in the main file can detect
whether the complete document or merely a child is to be compiled
even without using the directive |\childdocof|.
This method is deprecated because it is less robust
and there is no compelling reason to use it;
it is merely provided for backward compatibility
and it may be removed in future versions.

If the detection mechanism is to be used,
it is mandatory to correctly specify
the filename of the main file as the argument of |\childdocmain|:
%
\begin{center}
\begin{tabular}{l}
|\input{childdoc.def}|\\
|\childdocmain{|\textit{main}|}|\\
\end{tabular}
\end{center}
%
If |\jobname| does not match the argument \textit{main} of |\childdocmain|,
it is assumed that |\jobname| points to the child file to be compiled.
When using |\childdocmain| with the main file specified as argument,
it suffices to start a child file
with just |\input{|\textit{main}|}|
without loading of the package and using |\childdocof|.
If instead all processing is done
with the appropriate \textsf{childdoc} directives,
the argument of \textit{main} of |\childdocmain| can be empty.

An alternative version of the command line processing described
in \secref{sec:commandline} using the detection mechanism reads:
%
\begin{center}
|... -jobname "|\textit{target}|" "|[\textit{flags}]%
[|\def\jobname{|\textit{dest}|}|]|\input{|\textit{main}|}"|
\end{center}

%%%%%%%%%%%%%%%%%%%%%%%%%%%%%%%%%%%%%%%%%%%%%%%%%%%%%%%%%%%%%%%%%%%%%%%%%%%%%%%%
\subsection{Manual Code}
\label{sec:manual}

In case one cannot be certain whether the definitions file |childdoc.def|
is installed on the target \TeX{} distribution
and one prefers not to ship it,
it is conceivable to paste a few relevant commands into the sources.

To that end, drop all statements |\input{childdoc.def}|
and perform the replacements as outlined below.
Instead of |\childdocmain{|\textit{main}|}| add the following code
to the top of the main file:
%
\begin{center}
\begin{tabular}{l}
|\||ifdefined\childdocname\endinput\||fi\newif\ifchilddoc|\\
|\edef\childdocname{\scantokens\expandafter{\jobname\noexpand}}|\\
|\def\childdocmain{|\textit{main}|}\||ifx\childdocmain\childdocname\||else|\\
|\childdoctrue\includeonly{\childdocname}\let\jobname\childdocmain\||fi|\\
\end{tabular}
\end{center}
%
Instead of |\childdocof{|\textit{main}|}| just include the main file
at the top of each child file:
%
\begin{center}
|\input{|\textit{main}|}|
\end{center}
%
A simple redirection |\childdocforward{|\textit{dest}|}| is achieved by:
%
\begin{center}
|\def\jobname{|\textit{dest}|}\input{\jobname}|
\end{center}
%
The redirection with prefix
|\childdocforwardprefix[|\textit{prefix}|]{|\textit{dest}|}|
is accomplished by:
%
\begin{center}
\begin{tabular}{l}
|{\edef\jobname{\scantokens\expandafter{\jobname\noexpand}}|\\
|\def\redirectjob |\textit{prefix}|#1~~~{\gdef\jobname{|\textit{dest}|#1}}|\\
|\expandafter\redirectjob\jobname~~~}\input{\jobname}|
\end{tabular}
\end{center}

In an alternative approach,
child documents can be compiled by a specific command line
without additional code or specific definitions:
%
\begin{center}
|... -jobname "|\textit{target}|" "|[\textit{flags}]%
|\includeonly{|\textit{dest}|}\input{|\textit{main}|}"|
\end{center}
%

%%%%%%%%%%%%%%%%%%%%%%%%%%%%%%%%%%%%%%%%%%%%%%%%%%%%%%%%%%%%%%%%%%%%%%%%%%%%%%%%
%%%%%%%%%%%%%%%%%%%%%%%%%%%%%%%%%%%%%%%%%%%%%%%%%%%%%%%%%%%%%%%%%%%%%%%%%%%%%%%%
\section{Information}

%%%%%%%%%%%%%%%%%%%%%%%%%%%%%%%%%%%%%%%%%%%%%%%%%%%%%%%%%%%%%%%%%%%%%%%%%%%%%%%%
\subsection{Copyright}

Copyright \copyright{} 2017--2018 Niklas Beisert

This work may be distributed and/or modified under the
conditions of the \LaTeX{} Project Public License, either version 1.3
of this license or (at your option) any later version.
The latest version of this license is in
  \url{http://www.latex-project.org/lppl.txt}
and version 1.3 or later is part of all distributions of \LaTeX{}
version 2005/12/01 or later.

This work has the LPPL maintenance status `maintained'.

The Current Maintainer of this work is Niklas Beisert.

This work consists of the files |README.txt|, |childdoc.ins| and |childdoc.dtx|
as well as the derived files |childdoc.def|, |cdocsamp.tex|
with |cdocsch1.tex|, |cdocsch2.tex|, |cdocspt3.tex|, |cdocspt4.tex|,
|cdocsdrf.tex|, |cdocsfn1.tex|, |cdocsfn2.tex|
as well as |childdoc.pdf|.

%%%%%%%%%%%%%%%%%%%%%%%%%%%%%%%%%%%%%%%%%%%%%%%%%%%%%%%%%%%%%%%%%%%%%%%%%%%%%%%%
\subsection{Files and Installation}

The package consists of the files:
%
\begin{center}
\begin{tabular}{ll}
    |README.txt|   & readme file \\
    |childdoc.ins| & installation file \\
    |childdoc.dtx| & source file \\
    |childdoc.def| & definition file \\
    |cdocsamp.tex| & sample main file \\
    |cdocsch1.tex| & sample include file \\
    |cdocsch2.tex| & sample include file \\
    |cdocspt3.tex| & sample part file \\
    |cdocspt4.tex| & sample part file \\
    |cdocsdrf.tex| & sample redirection file \\
    |cdocsfn1.tex| & sample redirection file \\
    |cdocsfn2.tex| & sample redirection file \\
    |childdoc.pdf| & manual
\end{tabular}
\end{center}
%
The distribution consists of the files
|README.txt|, |childdoc.ins| and |childdoc.dtx|.
%
\begin{itemize}
\item
Run (pdf)\LaTeX{} on |childdoc.dtx|
to compile the manual |childdoc.pdf| (this file).
\item
Run \LaTeX{} on |childdoc.ins| to create the definitions file |childdoc.def|
and the sample |cdocsamp.tex| with include files
|cdocsch1.tex|, |cdocsch2.tex|, |cdocspt3.tex|, |cdocspt4.tex|,
|cdocsdrf.tex|, |cdocsfn1.tex|, |cdocsfn2.tex|.
Then copy the file |childdoc.def| to an appropriate directory of your \LaTeX{}
distribution, e.g.\ \textit{texmf-root}|/tex/latex/childdoc|.
\end{itemize}

%%%%%%%%%%%%%%%%%%%%%%%%%%%%%%%%%%%%%%%%%%%%%%%%%%%%%%%%%%%%%%%%%%%%%%%%%%%%%%%%
\subsection{Related CTAN Packages}

There are several other packages which offer a similar functionality:
%
\begin{itemize}
\item
The packages
\href{http://ctan.org/pkg/docmute}{\textsf{docmute}},
\href{http://ctan.org/pkg/includex}{\textsf{includex}} and
\href{http://ctan.org/pkg/standalone}{\textsf{standalone}}
provide commands to include only the document body of
a child file thus allowing both files to be compiled individually.
\item
The packages \href{http://ctan.org/pkg/subdocs}{\textsf{subdocs}}
and \href{http://ctan.org/pkg/subfiles}{\textsf{subfiles}}
provide structures in which the main and child documents can be
encapsulated and allowing them to be compiled individually.
The inclusion mechanism is different from the conventional |\include|.
\item
The package \href{http://ctan.org/pkg/combine}{\textsf{combine}}
is an elaborate solution to combine several documents into one.
\end{itemize}
%
See also the CTAN topic \href{http://ctan.org/topic/subdocs}{\textsf{subdocs}}
for further related packages.
The present package differs from the above solutions in that
a document structure constructed with the conventional |\include| mechanism
just needs two extra commands at the top of every file
such that all constituent files can be compiled individually.

%%%%%%%%%%%%%%%%%%%%%%%%%%%%%%%%%%%%%%%%%%%%%%%%%%%%%%%%%%%%%%%%%%%%%%%%%%%%%%%%
%\subsection{Feature Suggestions}
%
%The following is a list of features which may be useful for future
%versions of this package:
%%
%\begin{itemize}
%\item
%\ldots
%\end{itemize}

%%%%%%%%%%%%%%%%%%%%%%%%%%%%%%%%%%%%%%%%%%%%%%%%%%%%%%%%%%%%%%%%%%%%%%%%%%%%%%%%
\subsection{Revision History}

%%%%%%%%%%%%%%%%%%%%%%%%%%%%%%%%%%%%%%%%
\paragraph{v2.0:} 2018/12/30

\begin{itemize}
\item
immediate forward processing
\item
added |\childdocby| mechanism
\item
manual restructured
\end{itemize}

%%%%%%%%%%%%%%%%%%%%%%%%%%%%%%%%%%%%%%%%
\paragraph{v1.6:} 2018/01/17

\begin{itemize}
\item
application for development of include files
\item
corrections to manual
\end{itemize}

%%%%%%%%%%%%%%%%%%%%%%%%%%%%%%%%%%%%%%%%
\paragraph{v1.5:} 2017/05/21

\begin{itemize}
\item
more complete structuring introduced
\item
|\childdocof| introduced
\item
|\childdoc| renamed to |\childdocmain|
\item
|\childredirect| renamed to |\childdocforward| and |\childdocforwardprefix|
and functionality expanded
\end{itemize}

%%%%%%%%%%%%%%%%%%%%%%%%%%%%%%%%%%%%%%%%
\paragraph{v1.0:} 2017/04/27

\begin{itemize}
\item
manual and install package
\item
first version published on CTAN
\end{itemize}

%%%%%%%%%%%%%%%%%%%%%%%%%%%%%%%%%%%%%%%%
\paragraph{v0.6:} 2017/04/26

\begin{itemize}
\item
redirection mechanism added
\end{itemize}

%%%%%%%%%%%%%%%%%%%%%%%%%%%%%%%%%%%%%%%%
\paragraph{v0.5:} 2017/04/26

\begin{itemize}
\item
functionality in definition file
\end{itemize}


%%%%%%%%%%%%%%%%%%%%%%%%%%%%%%%%%%%%%%%%%%%%%%%%%%%%%%%%%%%%%%%%%%%%%%%%%%%%%%%%
%%%%%%%%%%%%%%%%%%%%%%%%%%%%%%%%%%%%%%%%%%%%%%%%%%%%%%%%%%%%%%%%%%%%%%%%%%%%%%%%
%%%%%%%%%%%%%%%%%%%%%%%%%%%%%%%%%%%%%%%%%%%%%%%%%%%%%%%%%%%%%%%%%%%%%%%%%%%%%%%%
\appendix

\settowidth\MacroIndent{\rmfamily\scriptsize 000\ }

 \DocInput{childdoc.dtx}

\end{document}
%</driver>
% \fi
%
% %%%%%%%%%%%%%%%%%%%%%%%%%%%%%%%%%%%%%%%%%%%%%%%%%%%%%%%%%%%%%%%%%%%%%%%%%%%%%%
% %%%%%%%%%%%%%%%%%%%%%%%%%%%%%%%%%%%%%%%%%%%%%%%%%%%%%%%%%%%%%%%%%%%%%%%%%%%%%%
% \section{Sample}
%\iffalse
%<*samplemain>
%\fi
%
% The following presents a sample document
% with two chapters, two parts, a title page,
% a compile flag as well as three forwarding files to set the flag.
% It consists of eight |.tex| files:
% \begin{center}
% \begin{tabular}{ll}
% |cdocsamp.tex|&main file\\
% |cdocsch1.tex|&include file for chapter 1\\
% |cdocsch2.tex|&include file for chapter 2\\
% |cdocspt3.tex|&include file for part 3\\
% |cdocspt4.tex|&include file for part 4\\
% |cdocsdrf.tex|&forwarding file for main file in draft mode\\
% |cdocsfi1.tex|&forwarding file for final version of chapter 1\\
% |cdocsfi2.tex|&forwarding file for final version of chapter 2\\
% \end{tabular}
% \end{center}
% Each of the eight files can be compiled directly by the \LaTeX{} compiler.
%
% %%%%%%%%%%%%%%%%%%%%%%%%%%%%%%%%%%%%%%
% \paragraph{Main File.}
%
% The main file is called |cdocsamp.tex|.
%
% Load the \textsf{childdoc} definitions and
% declare the filename for the main document:
%    \begin{macrocode}
\input{childdoc.def}
\childdocmain{}
%    \end{macrocode}

% Optional override for |\version| flag:
%    \begin{macrocode}
%%\ifchilddoc\else\providecommand{\version}{draft}\fi
%    \end{macrocode}

% Define the default values for the |\version| flag
% (|final| for the main file and |draft| for childs):
%    \begin{macrocode}
\ifchilddoc
\providecommand{\version}{draft}
\else
\providecommand{\version}{final}
\fi
%    \end{macrocode}

% Load the standard document class:
%    \begin{macrocode}
\documentclass[12pt]{article}
%    \end{macrocode}

% Start the document body:
%    \begin{macrocode}
\begin{document}
%    \end{macrocode}

% Declare a title page.
% Print title, part of document being processed and version flag:
%    \begin{macrocode}
\addtocounter{page}{-1}
\begin{center}
{\LARGE\bfseries{}childdoc example\par}
\vspace{1cm}
\ifchilddoc
\ifchilddocmanual part\else chapter\fi:
`\childdocname' of `\childdocjob'\par
\else
main document: `\childdocjob'\par
\fi
version: \version\par
\end{center}
\newpage
%    \end{macrocode}

% Manually include selected file,
% otherwise process as usual:
%    \begin{macrocode}
\ifchilddocmanual
\section*{part `\childdocname'}
\input{\childdocname}
\else
%    \end{macrocode}

% Include the two chapters:
%    \begin{macrocode}
\include{cdocsch1}
\include{cdocsch2}
%    \end{macrocode}

% Include the two parts unless only chapters should be displayed:
%    \begin{macrocode}
\ifchilddoc\else
\section{part three}
\input{cdocspt3}
\section{part four}
\input{cdocspt4}
\fi
%    \end{macrocode}

% Process as usual until here:
%    \begin{macrocode}
\fi
%    \end{macrocode}

% End of document body:
%    \begin{macrocode}
\end{document}
%    \end{macrocode}
%\iffalse
%</samplemain>
%\fi
%
% %%%%%%%%%%%%%%%%%%%%%%%%%%%%%%%%%%%%%%
% \paragraph{Chapter Include Files.}
%
% The include files are called |cdocsch1.tex| and |cdocsch2.tex|.
%
%\iffalse
%<*samplechap1|samplechap2>
%\fi

% Optional override for |\version| flag:
%    \begin{macrocode}
%%\providecommand{\version}{final}
%    \end{macrocode}

% Include the main document:
%    \begin{macrocode}
\input{childdoc.def}
\childdocof{cdocsamp}
%    \end{macrocode}

%\iffalse
%</samplechap1|samplechap2>
%\fi
%
%\iffalse
%<*samplechap1>
%\fi
% Some text for chapter 1:
%    \begin{macrocode}
\section{one}
some text in chapter one
%    \end{macrocode}

%\iffalse
%</samplechap1>
%\fi
% Some text for chapter 2:
%\iffalse
%<*samplechap2>
%\fi
%    \begin{macrocode}
\section{two}
more text in chapter two
%    \end{macrocode}

%\iffalse
%</samplechap2>
%\fi
%
% %%%%%%%%%%%%%%%%%%%%%%%%%%%%%%%%%%%%%%
% \paragraph{Part Include Files.}
%
% The include files are called |cdocspt3.tex| and |cdocspt4.tex|.
%
%\iffalse
%<*samplepart3|samplepart4>
%\fi

% Optional override for |\version| flag:
%    \begin{macrocode}
%%\providecommand{\version}{final}
%    \end{macrocode}

% Include the main document:
%    \begin{macrocode}
\input{childdoc.def}
\childdocby{cdocsamp}
%    \end{macrocode}

%\iffalse
%</samplepart3|samplepart4>
%\fi
%
%\iffalse
%<*samplepart3>
%\fi
% Some text for part 3:
%    \begin{macrocode}
some text in part three
%    \end{macrocode}

%\iffalse
%</samplepart3>
%\fi
% Some text for part 4:
%\iffalse
%<*samplepart4>
%\fi
%    \begin{macrocode}
more text in part four
%    \end{macrocode}

%\iffalse
%</samplepart4>
%\fi
%
% %%%%%%%%%%%%%%%%%%%%%%%%%%%%%%%%%%%%%%
% \paragraph{Forwarding for a Complete Draft.}
%
% The following forwarding file |cdocsdrf.tex|
% compiles the main document in draft mode:
%\iffalse
%<*sampledraft>
%\fi
%    \begin{macrocode}
\def\version{draft}
\input{childdoc.def}
\childdocforward{cdocsamp}
%    \end{macrocode}

%\iffalse
%</sampledraft>
%\fi
%
% %%%%%%%%%%%%%%%%%%%%%%%%%%%%%%%%%%%%%%
% \paragraph{Forwarding for Final Version of the Chapters.}
%
% The following forwarding files |cdocsfn1.tex| and |cdocsfn2.tex|
% (with identical content)
% compile the final versions of the child documents
% |cdocsch1.tex| and |cdocsch2.tex|, respectively:
%\iffalse
%<*samplefinal>
%\fi
%    \begin{macrocode}
\def\version{final}
\input{childdoc.def}
\childdocforwardprefix[cdocsamp]{cdocsfn}{cdocsch}
%    \end{macrocode}

%\iffalse
%</samplefinal>
%\fi
%
% %%%%%%%%%%%%%%%%%%%%%%%%%%%%%%%%%%%%%%
% \paragraph{Command Line Processing.}
%
% The following three command lines generate the output files
% |cdocscld|, |cdocscl1| and |cdocscl2|
% which should be identical to
% |cdocsdrf|, |cdocsch1| and |cdocsfn2|, respectively:
% \begin{center}
% \begin{tabular}{l}
% |latex -jobname cdocscld \|\\
% |  "\def\version{draft}\input{childdoc.def}\childdocforward{cdocsamp}"|\\
% |latex -jobname cdocscl1 \|\\
% |  "\input{childdoc.def}\childdocforward[cdocsamp]{cdocsch1}"|\\
% |latex -jobname cdocscl2 \|\\
% |  "\def\version{final}\input{childdoc.def}\childdocforward{cdocsch2}"|
% \end{tabular}
% \end{center}
% Note that the trailing backslash on each first line
% merely continues the input to the second line
% (for convenient cut ant paste).
% Furthermore, the command |latex| can be replaced by any
% of its alternative versions such as |pdflatex|.
%
% %%%%%%%%%%%%%%%%%%%%%%%%%%%%%%%%%%%%%%%%%%%%%%%%%%%%%%%%%%%%%%%%%%%%%%%%%%%%%%
% %%%%%%%%%%%%%%%%%%%%%%%%%%%%%%%%%%%%%%%%%%%%%%%%%%%%%%%%%%%%%%%%%%%%%%%%%%%%%%
% \section{Implementation}
%\iffalse
%<*package>
%\fi
%
% This section describes the definitions file |childdoc.def|.

% The definitions cannot be loaded using |\usepackage| or |\RequirePackage|
% which has a mechanism to prevent loading a style file more than once.
% When loading the definitions by means of |\input|
% multiple instances have to be prevented manually:
%\iffalse
%This code needs to be before the `\ProvidesFile' directive
%which is defined at the beginning of this file.
%Therefore it is also placed there and commented out here.
%</package>
%<*discard>
%\fi
%    \begin{macrocode}
\ifdefined\childdocmain\endinput\fi
%    \end{macrocode}
%\iffalse
%</discard>
%<*package>
%\fi
%
% \macro{\ifchilddoc}
% \macro{\ifchilddocmanual}
% The conditional |\ifchilddoc| tells whether a
% child (true) or main (false) document is being compiled.
% The conditional |\ifchilddocmanual| tells whether
% the |\includeonly| mechanism is used (false) or
% the selection of child files must be performed manually (true).
% The definitions initialise to false:
%    \begin{macrocode}
\newif\ifchilddoc
\newif\ifchilddocmanual
%    \end{macrocode}

% \macro{\childdocname}
% \macro{\childdocjob}
% The macro |\childdocname| stores the name of the main document
% to be compiled. The macro |\childdocjob| stores the name of
% the document on which the \LaTeX{} compiler was originally invoked.
% The content of |\jobname| cannot be compared
% to filenames specified in the source due to different catcodes.
% The following code rescans |\jobname|, stores the result
% in |\childdocname| and saves a copy in |\childdocjob|:
%    \begin{macrocode}
\edef\childdocname{\scantokens\expandafter{\jobname\noexpand}}
\let\childdocjob\childdocname
%    \end{macrocode}

% \macro{\childdocdisable}
% The macro |\childdocdisable| prevents the main file
% from being processed more than once.
% At this stage, the main document command |\childdocmain|
% is assumed to be called once again where it should do nothing.
% Any subsequent call to it should prevent
% a secondary processing of the main document
% It overwrites the forwarding commands
% |\childdocof| and |\childdocforward|
% with empty macros to prevent further inclusions of the main document:
%    \begin{macrocode}
\newcommand{\childdocdisable}
{
  \renewcommand{\childdocmain}[1]{\renewcommand{\childdocmain}[1]{\endinput}}
  \renewcommand{\childdocof}[1]{}
  \renewcommand{\childdocby}[2][]{}
  \renewcommand{\childdocforward}[2][]{}
  \renewcommand{\childdocdisable}{}
}
%    \end{macrocode}

% \macro{\childdocmain}
% The macro |\childdocmain| is to be called at the top of the main file
% with nothing or the main filename (without extension) as argument.
% First, it breaks loops.
% If the argument is not empty and does not match |\childdocname|
% (which is set by the first inclusion of |childdoc.def|),
% |\ifchilddoc| is set to true, |\includeonly| is applied to the child file
% and |\jobname| is set to the main file
% (for proper handling of |.aux| files):
%    \begin{macrocode}
\newcommand{\childdocmain}[1]
{
  \childdocdisable\childdocmain{}
  \if?#1?\else
    \begingroup
      \def\childdoctmp{#1}
      \ifx\childdoctmp\childdocname
        \def\childdoctmp{}
      \else
        \def\childdoctmp
        {
          \childdoctrue
          \includeonly{\childdocname}
          \def\childdocjob{#1}
          \def\jobname{#1}
        }
      \fi
      \expandafter
    \endgroup
    \childdoctmp
  \fi
}
%    \end{macrocode}

% \macro{\childdocof}
% The command |\childdocof| redirects
% compilation to the main file |#1|.
%    \begin{macrocode}
\newcommand{\childdocof}[1]
{
  \childdocdisable
  \childdoctrue
  \includeonly{\childdocname}
  \def\jobname{#1}
  \def\childdocjob{#1}
  \input{#1}
}
%    \end{macrocode}

% \macro{\childdocby}
% The command |\childdocby| ....
%    \begin{macrocode}
\newcommand{\childdocby}[2][]
{
  \childdocdisable
  \childdoctrue
  \childdocmanualtrue
  \if?#1?\else
    \def\jobname{#2}
  \fi
  \def\childdocjob{#2}
  \input{#2}
  \endinput
}
%    \end{macrocode}

% \macro{\childdocforward}
% The command |\childdocforward| redirects
% compilation to the main file or
% (if the optional argument is given) a child file.
% Parameters are set as if the main file
% or a child file starting with |\childdocof| was compiled.
% Then compilation is handed over to the main file:
%    \begin{macrocode}
\newcommand{\childdocforward}[2][]
{
  \begingroup
    \if?#1?
      \def\childdoctmp
      {
        \def\childdocname{#2}
        \def\childdocjob{#2}
        \def\jobname{#2}
        \input{#2}
        \endinput
      }
    \else
      \def\childdoctmp
      {
        \childdocdisable
        \def\childdocname{#2}
        \childdoctrue
        \includeonly{#2}
        \def\childdocjob{#1}
        \def\jobname{#1}
        \input{#1}
        \endinput
      }
    \fi
    \expandafter
  \endgroup
  \childdoctmp
}
%    \end{macrocode}

% \macro{\childdocforwardprefix}
% The command |\childdocforwardprefix| redirects
% compilation to the main or a child file by means of a pattern.
% The prefix |#1| in the current filename is replaced by |#2|
% and the suffix of the current filename is kept
% (it is assumed that the filename does not contain the substring `|~~~|'
% which is used as a delimiter).
% Compilation is handed over to the new file by |\childdocforward|:
%    \begin{macrocode}
\newcommand{\childdocforwardprefix}[3][]
{
  \begingroup
    \def\childdocextract #2##1~~~{\def\childdoctmp{\childdocforward[#1]{#3##1}}}
    \expandafter\childdocextract\childdocname~~~
    \expandafter
  \endgroup
  \childdoctmp
}
%    \end{macrocode}

% \macro{\childdoc}
% The deprecated macro |\childdoc| is a legacy version of |\childdocmain|:
%    \begin{macrocode}
\newcommand{\childdoc}{\childdocmain}
%    \end{macrocode}

% \macro{\childdocredirect}
% The deprecated macro |\childdocredirect| is a legacy version
% of |\childdocforward| and |\childdocforwardprefix|:
%    \begin{macrocode}
\newcommand{\childdocredirect}[2][]
{
  \begingroup
    \if?#1?
      \def\childdoctmp{\childdocforward{#2}}
    \else
      \def\childdoctmp{\childdocforwardprefix{#1}{#2}}
    \fi
    \expandafter
  \endgroup
  \childdoctmp
}
%    \end{macrocode}

%\iffalse
%</package>
%\fi
%
\endinput

\childdocof{cdocsamp}
%    \end{macrocode}

%\iffalse
%</samplechap1|samplechap2>
%\fi
%
%\iffalse
%<*samplechap1>
%\fi
% Some text for chapter 1:
%    \begin{macrocode}
\section{one}
some text in chapter one
%    \end{macrocode}

%\iffalse
%</samplechap1>
%\fi
% Some text for chapter 2:
%\iffalse
%<*samplechap2>
%\fi
%    \begin{macrocode}
\section{two}
more text in chapter two
%    \end{macrocode}

%\iffalse
%</samplechap2>
%\fi
%
% %%%%%%%%%%%%%%%%%%%%%%%%%%%%%%%%%%%%%%
% \paragraph{Part Include Files.}
%
% The include files are called |cdocspt3.tex| and |cdocspt4.tex|.
%
%\iffalse
%<*samplepart3|samplepart4>
%\fi

% Optional override for |\version| flag:
%    \begin{macrocode}
%%\providecommand{\version}{final}
%    \end{macrocode}

% Include the main document:
%    \begin{macrocode}
% \iffalse
%
% childdoc.dtx Copyright (C) 2017-2018 Niklas Beisert
%
% This work may be distributed and/or modified under the
% conditions of the LaTeX Project Public License, either version 1.3
% of this license or (at your option) any later version.
% The latest version of this license is in
%   http://www.latex-project.org/lppl.txt
% and version 1.3 or later is part of all distributions of LaTeX
% version 2005/12/01 or later.
%
% This work has the LPPL maintenance status `maintained'.
%
% The Current Maintainer of this work is Niklas Beisert.
%
% This work consists of the files childdoc.dtx and childdoc.ins
% and the derived files childdoc.def and cdocsamp.tex with
% cdocsch1.tex, cdocsch2.tex, cdocsdrf.tex, cdocsfn1.tex, cdocsfn2.tex.
%
%<package>\ifdefined\childdocmain\endinput\fi
%<package>\ProvidesFile{childdoc.def}[2018/12/30 v2.0 child document driver]
%<samplemain>\ProvidesFile{cdocsamp.tex}[2018/12/30 v2.0 sample for childdoc]
%<*driver>
%\ProvidesFile{childdoc.drv}[2018/12/30 v2.0 childdoc reference manual file]
\PassOptionsToClass{10pt,a4paper}{article}
\documentclass{ltxdoc}

\usepackage[margin=35mm]{geometry}
\usepackage{hyperref}
\usepackage{hyperxmp}
\usepackage[usenames]{color}

\hypersetup{colorlinks=true}
\hypersetup{pdfstartview=FitH}
\hypersetup{pdfpagemode=UseNone}
\hypersetup{pdfsource={}}
\hypersetup{pdflang={en-UK}}
\hypersetup{pdfcopyright={Copyright 2017-2018 Niklas Beisert.
  This work may be distributed and/or modified under the
  conditions of the LaTeX Project Public License, either version 1.3
  of this license or (at your option) any later version.}}
\hypersetup{pdflicenseurl={http://www.latex-project.org/lppl.txt}}
\hypersetup{pdfcontactaddress={ETH Zurich, ITP, HIT K,
  Wolfgang-Pauli-Strasse 27}}
\hypersetup{pdfcontactpostcode={8093}}
\hypersetup{pdfcontactcity={Zurich}}
\hypersetup{pdfcontactcountry={Switzerland}}
\hypersetup{pdfcontactemail={nbeisert@itp.phys.ethz.ch}}
\hypersetup{pdfcontacturl={http://people.phys.ethz.ch/\xmptilde nbeisert/}}

\newcommand{\secref}[1]{\hyperref[#1]{section \ref*{#1}}}

\parskip1ex
\parindent0pt
\let\olditemize\itemize
\def\itemize{\olditemize\parskip0pt}

\begin{document}

\title{The \textsf{childdoc} Package}
\hypersetup{pdftitle={The childdoc Package}}
\author{Niklas Beisert\\[2ex]
  Institut f\"ur Theoretische Physik\\
  Eidgen\"ossische Technische Hochschule Z\"urich\\
  Wolfgang-Pauli-Strasse 27, 8093 Z\"urich, Switzerland\\[1ex]
  \href{mailto:nbeisert@itp.phys.ethz.ch}
  {\texttt{nbeisert@itp.phys.ethz.ch}}}
\hypersetup{pdfauthor={Niklas Beisert}}
\hypersetup{pdfsubject={Manual for the LaTeX2e Package childdoc}}
\date{30 December 2018, \textsf{v2.0}}
\maketitle

\begin{abstract}\noindent
\textsf{childdoc} is a \LaTeXe{} package
that enables the direct compilation
of document sections included by |\include|
to individual files.
\end{abstract}

\begingroup
\parskip0ex
\tableofcontents
\endgroup

%%%%%%%%%%%%%%%%%%%%%%%%%%%%%%%%%%%%%%%%%%%%%%%%%%%%%%%%%%%%%%%%%%%%%%%%%%%%%%%%
%%%%%%%%%%%%%%%%%%%%%%%%%%%%%%%%%%%%%%%%%%%%%%%%%%%%%%%%%%%%%%%%%%%%%%%%%%%%%%%%
\section{Introduction}

\LaTeX{} provides a mechanism to structure a large document (such as a book)
into a main file and several child files (containing the chapters)
using the |\include| command.
This mechanism is beneficial for documents
which span hundreds of pages in order to
make the source file(s) more manageable.
Moreover, compilation can be restricted to
selected child files by means of the |\includeonly| command.
The latter feature can be used to reduce the compilation time while editing
(this was significantly more useful in the earlier days of \LaTeX{})
or to generate a smaller document which is easier to navigate.
Another application of |\includeonly| is to generate
documents consisting of selected parts of the complete document.

However, there are a few drawbacks of the plain |\include| mechanism:
\begin{itemize}
\item
The child files cannot be compiled on their own,
they can only be compiled via the main file.
A naive editing environment
(such as a text editor with an option
to have the current file processed by \LaTeX)
may require one to switch to the main file before compiling;
attempting to compile the child file produces errors.
\item
The main file must be modified (each time)
to adjust the |\includeonly| command
to the present needs. This easily leaves the main file in a messy state.
\item
The generated document will always carry the filename
of the main document. This is inconvenient if
several child files are to be compiled and
to be kept for distribution.
\end{itemize}

The present package provides a simple interface
to make child files individually compilable by \LaTeX{}.
Compiling a child file then has the same effect as compiling
the main file with an |\includeonly| command
to select the appropriate child.
Moreover the generated document will carry the name of the child
rather than the main file.
This resolves all three above issues.

This feature is meant to make the editing of books,
thesis documents and lecture notes somewhat more convenient.
However, the package can also be used efficiently for
composing a series of documents (such as exercise sheets)
which are typically distributed individually.
It then assists the author in generating the individual documents
(potentially in different versions)
as well as a document containing the collected series.
Another application is in developing style files
or other kinds of included material
where compilation of the style file could redirect
to a sample or test file.

%%%%%%%%%%%%%%%%%%%%%%%%%%%%%%%%%%%%%%%%%%%%%%%%%%%%%%%%%%%%%%%%%%%%%%%%%%%%%%%%
%%%%%%%%%%%%%%%%%%%%%%%%%%%%%%%%%%%%%%%%%%%%%%%%%%%%%%%%%%%%%%%%%%%%%%%%%%%%%%%%
\section{Usage}

First of all, the package \textsf{childdoc} is \emph{not} a standard
\LaTeXe{} |.sty| style file! Therefore it needs to be invoked in
a non-standard way.

%%%%%%%%%%%%%%%%%%%%%%%%%%%%%%%%%%%%%%%%%%%%%%%%%%%%%%%%%%%%%%%%%%%%%%%%%%%%%%%%
\subsection{Included Files}
\label{sec:include}

%%%%%%%%%%%%%%%%%%%%%%%%%%%%%%%%%%%%%%%%
\DescribeMacro{\childdocmain}
To use the package, add the commands
\begin{center}
\begin{tabular}{l}
|\input{childdoc.def}|\\
|\childdocmain{}|\\
\end{tabular}
\end{center}
at the very top of the main \LaTeX{} file,
in particular \emph{before} the |\documentclass| statement!
The argument of |\childdocmain| should be left empty
(but it must be present).

%%%%%%%%%%%%%%%%%%%%%%%%%%%%%%%%%%%%%%%%
\DescribeMacro{\childdocof}
Furthermore, add the commands
\begin{center}
\begin{tabular}{l}
|\input{childdoc.def}|\\
|\childdocof{|\textit{main}|}|\\
\end{tabular}
\end{center}
at the top of every child file \textit{child}
which is included by |\include{|\textit{child}|}|
from within the main file
(or at least for those files to be compiled individually).
The argument \textit{main} must be the filename of the main file.

There are a couple of
considerations in setting up the main and child documents:

%%%%%%%%%%%%%%%%%%%%%%%%%%%%%%%%%%%%%%%%
\paragraph{Restrictions.}

Please note the following restrictions:
\begin{itemize}
\item
|\childdocmain| must be called with one argument \textit{main}
to ensure compatibility with earlier version of the package.
It must either be empty (|\childdocmain{}|)
or precisely match the filename of the main file in which it is specified.
See \secref{sec:detection} for further information.
\item
The filename \textit{main} must be specified without the |.tex| extension.
\item
The filename \textit{main} is case sensitive
(even in case-insensitive file systems)
due to internal string comparison.
\item
The argument \textit{main} should be fully expanded, it cannot be a macro.
\item
Subdirectories and special characters should be avoided in filenames.
\item
The command |\childdocmain{|\textit{main}|}| must be followed by a whitespace.
It should not be followed immediately by another command
or by a comment mark `|%|'.
This is because the \TeX{} parser reads the token immediately following
the argument of |\childdocmain| and puts it
at the beginning of every child section;
however, a white\-space is ignored.
\end{itemize}

%%%%%%%%%%%%%%%%%%%%%%%%%%%%%%%%%%%%%%%%
\paragraph{Content of Main File.}

It is advisable to place all content in the child files included by |\include|.
Any output contained in the main file will appear in all child documents
unless suppressed manually;
it cannot be suppressed automatically by the |\includeonly| directive
and thus should normally be avoided.
A method to include some content in the main file
by means of conditional processing is described in \secref{sec:conditional}.

%%%%%%%%%%%%%%%%%%%%%%%%%%%%%%%%%%%%%%%%
\paragraph{Page Numbering.}

When only a part of the document is compiled,
the appropriate numbering of pages
(as well as other status parameters)
is determined from the |.aux| files.
The latter contain information from previous passes.
However this information needs to propagate through
all intermediate child documents.
Therefore the page numbering in child documents may well
be inconsistent until the complete document is compiled at least once.

A useful (if unconventional) way to always ensure a consistent
page numbering is to restart the numbering in each child document
and denote the pages by `\textit{child}|.|\textit{page}'
where \textit{child} represents the chapter/section number of the child file.
This can be achieved by the command
|\numberwithin{page}{|\textit{child}|}|
of the \textsf{amsmath} package
where \textit{child} can be |chapter| or |section|
depending on the chosen structuring.
Alternatively, one can modify the macro |\thepage| appropriately
and reset the counter |page| at the start of each child file.

%%%%%%%%%%%%%%%%%%%%%%%%%%%%%%%%%%%%%%%%%%%%%%%%%%%%%%%%%%%%%%%%%%%%%%%%%%%%%%%%
\subsection{Conditional Processing}
\label{sec:conditional}

The package provides a mechanism to compile different versions
of a document. To customise the versions further some conditional processing
can come in handy to distinguish which version is being compiled.
The package provides two macros to describe the compilation context:

%%%%%%%%%%%%%%%%%%%%%%%%%%%%%%%%%%%%%%%%
\DescribeMacro{\ifchilddoc}
The conditional |\ifchilddoc| distinguishes between the compilation of
child documents and the main document:
%
\begin{center}
|\ifchilddoc |\textit{child-code}| |[|\||else |\textit{main-code}]| \||fi|
\end{center}

%%%%%%%%%%%%%%%%%%%%%%%%%%%%%%%%%%%%%%%%
\DescribeMacro{\childdocname}
\DescribeMacro{\childdocjob}
The macro |\childdocname| contains the filename (without extension)
of the main or child file being processed.
Note that |\childdocjob| will always contain the name of the main file.

%%%%%%%%%%%%%%%%%%%%%%%%%%%%%%%%%%%%%%%%
\paragraph{Title Page.}

Conditional processing can be used to include a title or banner page
in the main document when proper precautions are taken.
Importantly, the code in the main file should ensure that the page counter
(as well as other status parameters which are stored in the |.aux| files)
takes the same value after the conditional processing.
Otherwise the page numbers may take divergent values
depending on which part is compiled.

For example, a title page could be declared by:
%
\begin{center}
\begin{tabular}{l}
|\ifchilddoc\||else|\\
|\addtocounter{page}{-1}|\\
\textit{code for title page}\\
|\newpage|\\
|\||fi|
\end{tabular}
\end{center}
%
A banner page for the child documents can be generated by:
%
\begin{center}
\begin{tabular}{l}
|\ifchilddoc|\\
|\addtocounter{page}{-1}|\\
\textit{code for banner page}\\
|\newpage|\\
|\||fi|
\end{tabular}
\end{center}
%
Here one could write a message such as:
\begin{center}
|This is the part \childdocname{} of \childdocjob{}.|
\end{center}

%%%%%%%%%%%%%%%%%%%%%%%%%%%%%%%%%%%%%%%%%%%%%%%%%%%%%%%%%%%%%%%%%%%%%%%%%%%%%%%%
\subsection{Flags}
\label{sec:flags}

The package makes it easy to generate different versions
of the main or child documents.
To this end compilation flags can be defined
and assigned different default values.
They will be particularly useful in conjunction
with the forwarding mechanism described in \secref{sec:forward}.

For example, it may be useful to have a flag |\version|
which can be set to |draft| or |final|.
The document source will contain some conditional code
depending on the value of |\version|.
Suppose further, the flag should default to |final| for the main file
and to |draft| for child files
which is a natural assignment for editing the document.
This is achieved by placing the following code
in the preamble of the main document
(below the |\childdocmain| directive):
%
\begin{center}
\begin{tabular}{l}
|\ifchilddoc|\\
|\providecommand{\version}{draft}|\\
|\||else|\\
|\providecommand{\version}{final}|\\
|\||fi|
\end{tabular}
\end{center}
%
The definition by |\providecommand| makes sure
that previous definitions are not overwritten.
Further statements |\providecommand{\version}{...}|
can thus be added before the above code to override it.

For the main file, one might add a line
(between |\childdocmain| and the above block)
%
\begin{center}
|%\ifchilddoc\||else\providecommand{\version}{draft}\||fi|
\end{center}
%
which can be uncommented to produce a draft version.
Likewise one can add a line to the very top of a child file
(above the |\childdocof{|\textit{main}|}| directive)
%
\begin{center}
|%\providecommand{\version}{final}|
\end{center}
%
which can be uncommented to produce the final version of this child document.

%%%%%%%%%%%%%%%%%%%%%%%%%%%%%%%%%%%%%%%%%%%%%%%%%%%%%%%%%%%%%%%%%%%%%%%%%%%%%%%%
\subsection{Forwarding}
\label{sec:forward}

Different versions of the main or child documents
using compilation flags as described in \secref{sec:flags}
can be (permanently) stored in different files
for convenient compilation, viewing and distribution.
To this end, the package defines a command
to pass on compilation to a different file:

%%%%%%%%%%%%%%%%%%%%%%%%%%%%%%%%%%%%%%%%
\DescribeMacro{\childdocforward}
The command |\childdocforward| redirects processing to
another source file:
%
\begin{center}
\begin{tabular}{l}
|\input{childdoc.def}|\\
|\childdocforward[|\textit{main}|]{|\textit{dest}|}|\\
\end{tabular}
\end{center}
%
The argument \textit{dest} is the destination file
(without extension).
It should be the main file or one of the child files.
Note that further \textsf{childdoc} directives
such as |\childdocof| and |\childdocforward|
in the indicated file will be processed in this form.
The optional argument \textit{main}
passes on directly to the main file \textit{main}
while pretending to compile the child \textit{dest}.
This form behaves as if \textit{dest}
issues |\childdocof{|\textit{main}|}| right away,
and no further \textsf{childdoc} directives will be processed.

%%%%%%%%%%%%%%%%%%%%%%%%%%%%%%%%%%%%%%%%
\DescribeMacro{\...prefix}
In the alternative form |\childdocforwardprefix|,
%
\begin{center}
\begin{tabular}{l}
|\input{childdoc.def}|\\
|\childdocforwardprefix[|\textit{main}|]{|\textit{prefix}|}{|\textit{dest}|}|
\end{tabular}
\end{center}
%
the destination file is determined by a pattern
depending on the current file:
To make this work, the current file must be called
`{\textit{prefix}\hspace{0.2em}\textit{suffix}}'
with \textit{prefix} matching precisely the argument.
Processing is then passed on to the file
`{\textit{dest}\hspace{0.2em}\textit{suffix}}'.
Surely, the same effect is achieved by
directly specifying the
argument `{\textit{dest}\hspace{0.2em}\textit{suffix}}'
in the first form.
However, that requires to set up a different file
for each child. With the alternative form of the command
all these files can have exactly the same content
which simplifies setting them up and maintaining them.

For example, the following file |draft.tex|
with a compilation flag |\version| as described in \secref{sec:flags}
compiles the main document as a draft:
%
\begin{center}
\begin{tabular}{l}
|\def\version{draft}|\\
|\input{childdoc.def}|\\
|\childdocforward{|\textit{main}|}|
\end{tabular}
\end{center}
%
Likewise, the following files |final|\textit{nn}|.tex|
compile the final version of the child document
|child|\textit{nn}|.tex|:
%
\begin{center}
\begin{tabular}{l}
|\def\version{final}|\\
|\input{childdoc.def}|\\
|\childdocforwardprefix{final}{child}|
\end{tabular}
\end{center}
%

Note that when several versions of a main file and/or of each child file
are to be generated, it may be convenient to set up a |Makefile| or
shell script to automatise the process.

%%%%%%%%%%%%%%%%%%%%%%%%%%%%%%%%%%%%%%%%%%%%%%%%%%%%%%%%%%%%%%%%%%%%%%%%%%%%%%%%
\subsection{Command Line Processing}
\label{sec:commandline}

The effect of redirection files can also be achieved by invoking
the \LaTeX{} compiler with a more elaborate command line.
Most conveniently this should be done as part
of a shell script or a |Makefile|.

When using \textsf{childdoc} in the main file, the following
command lines effectively perform a redirection
(note that depending on the shell being used,
backslashes may have to be doubled: `|\|' $\to$ `|\\|'):
%
\begin{center}
|... -jobname "|\textit{target}|" |\\|"|[\textit{flags}]%
|\input{childdoc.def}\childdocforward[|\textit{main}|]{|\textit{dest}|}"|
\end{center}
%
Here \textit{target} is the name of the output file,
\textit{main} is the name of the main file
and \textit{dest} is the name of the main or child file to be processed
(all filenames without extensions).
The optional argument \textit{main} can be omitted
if \textit{main} matches \textit{dest}.
Optionally, compilation \textit{flags} can be defined via |\def| commands.
This command line makes the \TeX{} engine believe
it is compiling the file \textit{target}
whose content is specified as the latter parameter.
The provided code then forwards the processing to
\textit{main} or \textit{dest} as described in \secref{sec:forward}.

%%%%%%%%%%%%%%%%%%%%%%%%%%%%%%%%%%%%%%%%%%%%%%%%%%%%%%%%%%%%%%%%%%%%%%%%%%%%%%%%
\subsection{Include by Input}
\label{sec:input}

Including child documents by |\include| has some restrictions by design.
Most notably, the content of a child document always occupies
its own set of pages; pages cannot be shared between child documents.
Usually, this behaviour makes perfect sense
because each child document contain an essential part of the document.
However, in some situations it may be desirable to compose
a document from a collection of parts
without having mandatory page breaks between then.
For this case, the package
provides a mechanism to include parts
by |\input| which can also be processed individually.
However, by construction this mechanism
requires manual handling of the content to be output.

%%%%%%%%%%%%%%%%%%%%%%%%%%%%%%%%%%%%%%%%
\DescribeMacro{\ifchilddocmanual}
The main file should be prepared as usual, see \secref{sec:include}.
However, the document body must make a distinction
between processing of an individual part and of the main document, e.g.:
%
\begin{center}
\begin{tabular}{l}
|\ifchilddocmanual|\\
|\input{\childdocname}|\\
|\||else|\\
\textit{document body with }|\input{|\textit{part}|}|\\
|\||fi|
\end{tabular}
\end{center}
%
The conditional |\ifchilddocmanual| is true whenever
a part to be included by |\input| is being compiled,
and the name of the part is stored in |\childdocname|.

%%%%%%%%%%%%%%%%%%%%%%%%%%%%%%%%%%%%%%%%
\DescribeMacro{\childdocby}
Each part to be included by |\input| should start with:
%
\begin{center}
\begin{tabular}{l}
|\input{childdoc.def}|\\
|\childdocby{|\textit{main}|}|\\
\end{tabular}
\end{center}
%
The directive |\childdocby| is similar to |\childdocof|
described in \secref{sec:include},
but the subsequent selection of content must be done manually.
To that end, both |\ifchilddoc| and |\ifchilddocmanual|
will be true upon processing of a part,
and the name of the part is stored in |\childdocname|.
Note that |\jobname| will be set to the filename of the current part
so that each part receives an individual |.aux| file
that does not interfere with the |.aux| file(s) of the main document.
This behaviour can be altered by the alternative form
|\childdocby[*]{|\textit{main}|}| (with a non-empty optional argument)
which uses the |.aux| file of the main document
by setting |\jobname| to \textit{main}.

%%%%%%%%%%%%%%%%%%%%%%%%%%%%%%%%%%%%%%%%%%%%%%%%%%%%%%%%%%%%%%%%%%%%%%%%%%%%%%%%
\subsection{Driver Development}
\label{sec:driver}

The \textsf{childdoc} mechanism can also be use for the development
of definition files such as \LaTeX{} styles or classes.
This case differs from the above setup with multiple parts
included by |\include| in that no |\includeonly| should be invoked.
This can be achieved by starting the include file
(before |\ProvidesPackage|) with:
%
\begin{center}
\begin{tabular}{l}
|\input{childdoc.def}|\\
|\childdocforward{|\textit{main}|}|\\
\end{tabular}
\end{center}
%
or alternatively with:
%
\begin{center}
\begin{tabular}{l}
|\input{childdoc.def}|\\
|\childdocby{|\textit{main}|}|\\
\end{tabular}
\end{center}
%
Both forms have slightly different effects as described above.
The main file is prepared as usual, see \secref{sec:include}.

%%%%%%%%%%%%%%%%%%%%%%%%%%%%%%%%%%%%%%%%%%%%%%%%%%%%%%%%%%%%%%%%%%%%%%%%%%%%%%%%
\subsection{Legacy Detection}
\label{sec:detection}

The directive |\childdocmain| in the main file can detect
whether the complete document or merely a child is to be compiled
even without using the directive |\childdocof|.
This method is deprecated because it is less robust
and there is no compelling reason to use it;
it is merely provided for backward compatibility
and it may be removed in future versions.

If the detection mechanism is to be used,
it is mandatory to correctly specify
the filename of the main file as the argument of |\childdocmain|:
%
\begin{center}
\begin{tabular}{l}
|\input{childdoc.def}|\\
|\childdocmain{|\textit{main}|}|\\
\end{tabular}
\end{center}
%
If |\jobname| does not match the argument \textit{main} of |\childdocmain|,
it is assumed that |\jobname| points to the child file to be compiled.
When using |\childdocmain| with the main file specified as argument,
it suffices to start a child file
with just |\input{|\textit{main}|}|
without loading of the package and using |\childdocof|.
If instead all processing is done
with the appropriate \textsf{childdoc} directives,
the argument of \textit{main} of |\childdocmain| can be empty.

An alternative version of the command line processing described
in \secref{sec:commandline} using the detection mechanism reads:
%
\begin{center}
|... -jobname "|\textit{target}|" "|[\textit{flags}]%
[|\def\jobname{|\textit{dest}|}|]|\input{|\textit{main}|}"|
\end{center}

%%%%%%%%%%%%%%%%%%%%%%%%%%%%%%%%%%%%%%%%%%%%%%%%%%%%%%%%%%%%%%%%%%%%%%%%%%%%%%%%
\subsection{Manual Code}
\label{sec:manual}

In case one cannot be certain whether the definitions file |childdoc.def|
is installed on the target \TeX{} distribution
and one prefers not to ship it,
it is conceivable to paste a few relevant commands into the sources.

To that end, drop all statements |\input{childdoc.def}|
and perform the replacements as outlined below.
Instead of |\childdocmain{|\textit{main}|}| add the following code
to the top of the main file:
%
\begin{center}
\begin{tabular}{l}
|\||ifdefined\childdocname\endinput\||fi\newif\ifchilddoc|\\
|\edef\childdocname{\scantokens\expandafter{\jobname\noexpand}}|\\
|\def\childdocmain{|\textit{main}|}\||ifx\childdocmain\childdocname\||else|\\
|\childdoctrue\includeonly{\childdocname}\let\jobname\childdocmain\||fi|\\
\end{tabular}
\end{center}
%
Instead of |\childdocof{|\textit{main}|}| just include the main file
at the top of each child file:
%
\begin{center}
|\input{|\textit{main}|}|
\end{center}
%
A simple redirection |\childdocforward{|\textit{dest}|}| is achieved by:
%
\begin{center}
|\def\jobname{|\textit{dest}|}\input{\jobname}|
\end{center}
%
The redirection with prefix
|\childdocforwardprefix[|\textit{prefix}|]{|\textit{dest}|}|
is accomplished by:
%
\begin{center}
\begin{tabular}{l}
|{\edef\jobname{\scantokens\expandafter{\jobname\noexpand}}|\\
|\def\redirectjob |\textit{prefix}|#1~~~{\gdef\jobname{|\textit{dest}|#1}}|\\
|\expandafter\redirectjob\jobname~~~}\input{\jobname}|
\end{tabular}
\end{center}

In an alternative approach,
child documents can be compiled by a specific command line
without additional code or specific definitions:
%
\begin{center}
|... -jobname "|\textit{target}|" "|[\textit{flags}]%
|\includeonly{|\textit{dest}|}\input{|\textit{main}|}"|
\end{center}
%

%%%%%%%%%%%%%%%%%%%%%%%%%%%%%%%%%%%%%%%%%%%%%%%%%%%%%%%%%%%%%%%%%%%%%%%%%%%%%%%%
%%%%%%%%%%%%%%%%%%%%%%%%%%%%%%%%%%%%%%%%%%%%%%%%%%%%%%%%%%%%%%%%%%%%%%%%%%%%%%%%
\section{Information}

%%%%%%%%%%%%%%%%%%%%%%%%%%%%%%%%%%%%%%%%%%%%%%%%%%%%%%%%%%%%%%%%%%%%%%%%%%%%%%%%
\subsection{Copyright}

Copyright \copyright{} 2017--2018 Niklas Beisert

This work may be distributed and/or modified under the
conditions of the \LaTeX{} Project Public License, either version 1.3
of this license or (at your option) any later version.
The latest version of this license is in
  \url{http://www.latex-project.org/lppl.txt}
and version 1.3 or later is part of all distributions of \LaTeX{}
version 2005/12/01 or later.

This work has the LPPL maintenance status `maintained'.

The Current Maintainer of this work is Niklas Beisert.

This work consists of the files |README.txt|, |childdoc.ins| and |childdoc.dtx|
as well as the derived files |childdoc.def|, |cdocsamp.tex|
with |cdocsch1.tex|, |cdocsch2.tex|, |cdocspt3.tex|, |cdocspt4.tex|,
|cdocsdrf.tex|, |cdocsfn1.tex|, |cdocsfn2.tex|
as well as |childdoc.pdf|.

%%%%%%%%%%%%%%%%%%%%%%%%%%%%%%%%%%%%%%%%%%%%%%%%%%%%%%%%%%%%%%%%%%%%%%%%%%%%%%%%
\subsection{Files and Installation}

The package consists of the files:
%
\begin{center}
\begin{tabular}{ll}
    |README.txt|   & readme file \\
    |childdoc.ins| & installation file \\
    |childdoc.dtx| & source file \\
    |childdoc.def| & definition file \\
    |cdocsamp.tex| & sample main file \\
    |cdocsch1.tex| & sample include file \\
    |cdocsch2.tex| & sample include file \\
    |cdocspt3.tex| & sample part file \\
    |cdocspt4.tex| & sample part file \\
    |cdocsdrf.tex| & sample redirection file \\
    |cdocsfn1.tex| & sample redirection file \\
    |cdocsfn2.tex| & sample redirection file \\
    |childdoc.pdf| & manual
\end{tabular}
\end{center}
%
The distribution consists of the files
|README.txt|, |childdoc.ins| and |childdoc.dtx|.
%
\begin{itemize}
\item
Run (pdf)\LaTeX{} on |childdoc.dtx|
to compile the manual |childdoc.pdf| (this file).
\item
Run \LaTeX{} on |childdoc.ins| to create the definitions file |childdoc.def|
and the sample |cdocsamp.tex| with include files
|cdocsch1.tex|, |cdocsch2.tex|, |cdocspt3.tex|, |cdocspt4.tex|,
|cdocsdrf.tex|, |cdocsfn1.tex|, |cdocsfn2.tex|.
Then copy the file |childdoc.def| to an appropriate directory of your \LaTeX{}
distribution, e.g.\ \textit{texmf-root}|/tex/latex/childdoc|.
\end{itemize}

%%%%%%%%%%%%%%%%%%%%%%%%%%%%%%%%%%%%%%%%%%%%%%%%%%%%%%%%%%%%%%%%%%%%%%%%%%%%%%%%
\subsection{Related CTAN Packages}

There are several other packages which offer a similar functionality:
%
\begin{itemize}
\item
The packages
\href{http://ctan.org/pkg/docmute}{\textsf{docmute}},
\href{http://ctan.org/pkg/includex}{\textsf{includex}} and
\href{http://ctan.org/pkg/standalone}{\textsf{standalone}}
provide commands to include only the document body of
a child file thus allowing both files to be compiled individually.
\item
The packages \href{http://ctan.org/pkg/subdocs}{\textsf{subdocs}}
and \href{http://ctan.org/pkg/subfiles}{\textsf{subfiles}}
provide structures in which the main and child documents can be
encapsulated and allowing them to be compiled individually.
The inclusion mechanism is different from the conventional |\include|.
\item
The package \href{http://ctan.org/pkg/combine}{\textsf{combine}}
is an elaborate solution to combine several documents into one.
\end{itemize}
%
See also the CTAN topic \href{http://ctan.org/topic/subdocs}{\textsf{subdocs}}
for further related packages.
The present package differs from the above solutions in that
a document structure constructed with the conventional |\include| mechanism
just needs two extra commands at the top of every file
such that all constituent files can be compiled individually.

%%%%%%%%%%%%%%%%%%%%%%%%%%%%%%%%%%%%%%%%%%%%%%%%%%%%%%%%%%%%%%%%%%%%%%%%%%%%%%%%
%\subsection{Feature Suggestions}
%
%The following is a list of features which may be useful for future
%versions of this package:
%%
%\begin{itemize}
%\item
%\ldots
%\end{itemize}

%%%%%%%%%%%%%%%%%%%%%%%%%%%%%%%%%%%%%%%%%%%%%%%%%%%%%%%%%%%%%%%%%%%%%%%%%%%%%%%%
\subsection{Revision History}

%%%%%%%%%%%%%%%%%%%%%%%%%%%%%%%%%%%%%%%%
\paragraph{v2.0:} 2018/12/30

\begin{itemize}
\item
immediate forward processing
\item
added |\childdocby| mechanism
\item
manual restructured
\end{itemize}

%%%%%%%%%%%%%%%%%%%%%%%%%%%%%%%%%%%%%%%%
\paragraph{v1.6:} 2018/01/17

\begin{itemize}
\item
application for development of include files
\item
corrections to manual
\end{itemize}

%%%%%%%%%%%%%%%%%%%%%%%%%%%%%%%%%%%%%%%%
\paragraph{v1.5:} 2017/05/21

\begin{itemize}
\item
more complete structuring introduced
\item
|\childdocof| introduced
\item
|\childdoc| renamed to |\childdocmain|
\item
|\childredirect| renamed to |\childdocforward| and |\childdocforwardprefix|
and functionality expanded
\end{itemize}

%%%%%%%%%%%%%%%%%%%%%%%%%%%%%%%%%%%%%%%%
\paragraph{v1.0:} 2017/04/27

\begin{itemize}
\item
manual and install package
\item
first version published on CTAN
\end{itemize}

%%%%%%%%%%%%%%%%%%%%%%%%%%%%%%%%%%%%%%%%
\paragraph{v0.6:} 2017/04/26

\begin{itemize}
\item
redirection mechanism added
\end{itemize}

%%%%%%%%%%%%%%%%%%%%%%%%%%%%%%%%%%%%%%%%
\paragraph{v0.5:} 2017/04/26

\begin{itemize}
\item
functionality in definition file
\end{itemize}


%%%%%%%%%%%%%%%%%%%%%%%%%%%%%%%%%%%%%%%%%%%%%%%%%%%%%%%%%%%%%%%%%%%%%%%%%%%%%%%%
%%%%%%%%%%%%%%%%%%%%%%%%%%%%%%%%%%%%%%%%%%%%%%%%%%%%%%%%%%%%%%%%%%%%%%%%%%%%%%%%
%%%%%%%%%%%%%%%%%%%%%%%%%%%%%%%%%%%%%%%%%%%%%%%%%%%%%%%%%%%%%%%%%%%%%%%%%%%%%%%%
\appendix

\settowidth\MacroIndent{\rmfamily\scriptsize 000\ }

 \DocInput{childdoc.dtx}

\end{document}
%</driver>
% \fi
%
% %%%%%%%%%%%%%%%%%%%%%%%%%%%%%%%%%%%%%%%%%%%%%%%%%%%%%%%%%%%%%%%%%%%%%%%%%%%%%%
% %%%%%%%%%%%%%%%%%%%%%%%%%%%%%%%%%%%%%%%%%%%%%%%%%%%%%%%%%%%%%%%%%%%%%%%%%%%%%%
% \section{Sample}
%\iffalse
%<*samplemain>
%\fi
%
% The following presents a sample document
% with two chapters, two parts, a title page,
% a compile flag as well as three forwarding files to set the flag.
% It consists of eight |.tex| files:
% \begin{center}
% \begin{tabular}{ll}
% |cdocsamp.tex|&main file\\
% |cdocsch1.tex|&include file for chapter 1\\
% |cdocsch2.tex|&include file for chapter 2\\
% |cdocspt3.tex|&include file for part 3\\
% |cdocspt4.tex|&include file for part 4\\
% |cdocsdrf.tex|&forwarding file for main file in draft mode\\
% |cdocsfi1.tex|&forwarding file for final version of chapter 1\\
% |cdocsfi2.tex|&forwarding file for final version of chapter 2\\
% \end{tabular}
% \end{center}
% Each of the eight files can be compiled directly by the \LaTeX{} compiler.
%
% %%%%%%%%%%%%%%%%%%%%%%%%%%%%%%%%%%%%%%
% \paragraph{Main File.}
%
% The main file is called |cdocsamp.tex|.
%
% Load the \textsf{childdoc} definitions and
% declare the filename for the main document:
%    \begin{macrocode}
\input{childdoc.def}
\childdocmain{}
%    \end{macrocode}

% Optional override for |\version| flag:
%    \begin{macrocode}
%%\ifchilddoc\else\providecommand{\version}{draft}\fi
%    \end{macrocode}

% Define the default values for the |\version| flag
% (|final| for the main file and |draft| for childs):
%    \begin{macrocode}
\ifchilddoc
\providecommand{\version}{draft}
\else
\providecommand{\version}{final}
\fi
%    \end{macrocode}

% Load the standard document class:
%    \begin{macrocode}
\documentclass[12pt]{article}
%    \end{macrocode}

% Start the document body:
%    \begin{macrocode}
\begin{document}
%    \end{macrocode}

% Declare a title page.
% Print title, part of document being processed and version flag:
%    \begin{macrocode}
\addtocounter{page}{-1}
\begin{center}
{\LARGE\bfseries{}childdoc example\par}
\vspace{1cm}
\ifchilddoc
\ifchilddocmanual part\else chapter\fi:
`\childdocname' of `\childdocjob'\par
\else
main document: `\childdocjob'\par
\fi
version: \version\par
\end{center}
\newpage
%    \end{macrocode}

% Manually include selected file,
% otherwise process as usual:
%    \begin{macrocode}
\ifchilddocmanual
\section*{part `\childdocname'}
\input{\childdocname}
\else
%    \end{macrocode}

% Include the two chapters:
%    \begin{macrocode}
\include{cdocsch1}
\include{cdocsch2}
%    \end{macrocode}

% Include the two parts unless only chapters should be displayed:
%    \begin{macrocode}
\ifchilddoc\else
\section{part three}
\input{cdocspt3}
\section{part four}
\input{cdocspt4}
\fi
%    \end{macrocode}

% Process as usual until here:
%    \begin{macrocode}
\fi
%    \end{macrocode}

% End of document body:
%    \begin{macrocode}
\end{document}
%    \end{macrocode}
%\iffalse
%</samplemain>
%\fi
%
% %%%%%%%%%%%%%%%%%%%%%%%%%%%%%%%%%%%%%%
% \paragraph{Chapter Include Files.}
%
% The include files are called |cdocsch1.tex| and |cdocsch2.tex|.
%
%\iffalse
%<*samplechap1|samplechap2>
%\fi

% Optional override for |\version| flag:
%    \begin{macrocode}
%%\providecommand{\version}{final}
%    \end{macrocode}

% Include the main document:
%    \begin{macrocode}
\input{childdoc.def}
\childdocof{cdocsamp}
%    \end{macrocode}

%\iffalse
%</samplechap1|samplechap2>
%\fi
%
%\iffalse
%<*samplechap1>
%\fi
% Some text for chapter 1:
%    \begin{macrocode}
\section{one}
some text in chapter one
%    \end{macrocode}

%\iffalse
%</samplechap1>
%\fi
% Some text for chapter 2:
%\iffalse
%<*samplechap2>
%\fi
%    \begin{macrocode}
\section{two}
more text in chapter two
%    \end{macrocode}

%\iffalse
%</samplechap2>
%\fi
%
% %%%%%%%%%%%%%%%%%%%%%%%%%%%%%%%%%%%%%%
% \paragraph{Part Include Files.}
%
% The include files are called |cdocspt3.tex| and |cdocspt4.tex|.
%
%\iffalse
%<*samplepart3|samplepart4>
%\fi

% Optional override for |\version| flag:
%    \begin{macrocode}
%%\providecommand{\version}{final}
%    \end{macrocode}

% Include the main document:
%    \begin{macrocode}
\input{childdoc.def}
\childdocby{cdocsamp}
%    \end{macrocode}

%\iffalse
%</samplepart3|samplepart4>
%\fi
%
%\iffalse
%<*samplepart3>
%\fi
% Some text for part 3:
%    \begin{macrocode}
some text in part three
%    \end{macrocode}

%\iffalse
%</samplepart3>
%\fi
% Some text for part 4:
%\iffalse
%<*samplepart4>
%\fi
%    \begin{macrocode}
more text in part four
%    \end{macrocode}

%\iffalse
%</samplepart4>
%\fi
%
% %%%%%%%%%%%%%%%%%%%%%%%%%%%%%%%%%%%%%%
% \paragraph{Forwarding for a Complete Draft.}
%
% The following forwarding file |cdocsdrf.tex|
% compiles the main document in draft mode:
%\iffalse
%<*sampledraft>
%\fi
%    \begin{macrocode}
\def\version{draft}
\input{childdoc.def}
\childdocforward{cdocsamp}
%    \end{macrocode}

%\iffalse
%</sampledraft>
%\fi
%
% %%%%%%%%%%%%%%%%%%%%%%%%%%%%%%%%%%%%%%
% \paragraph{Forwarding for Final Version of the Chapters.}
%
% The following forwarding files |cdocsfn1.tex| and |cdocsfn2.tex|
% (with identical content)
% compile the final versions of the child documents
% |cdocsch1.tex| and |cdocsch2.tex|, respectively:
%\iffalse
%<*samplefinal>
%\fi
%    \begin{macrocode}
\def\version{final}
\input{childdoc.def}
\childdocforwardprefix[cdocsamp]{cdocsfn}{cdocsch}
%    \end{macrocode}

%\iffalse
%</samplefinal>
%\fi
%
% %%%%%%%%%%%%%%%%%%%%%%%%%%%%%%%%%%%%%%
% \paragraph{Command Line Processing.}
%
% The following three command lines generate the output files
% |cdocscld|, |cdocscl1| and |cdocscl2|
% which should be identical to
% |cdocsdrf|, |cdocsch1| and |cdocsfn2|, respectively:
% \begin{center}
% \begin{tabular}{l}
% |latex -jobname cdocscld \|\\
% |  "\def\version{draft}\input{childdoc.def}\childdocforward{cdocsamp}"|\\
% |latex -jobname cdocscl1 \|\\
% |  "\input{childdoc.def}\childdocforward[cdocsamp]{cdocsch1}"|\\
% |latex -jobname cdocscl2 \|\\
% |  "\def\version{final}\input{childdoc.def}\childdocforward{cdocsch2}"|
% \end{tabular}
% \end{center}
% Note that the trailing backslash on each first line
% merely continues the input to the second line
% (for convenient cut ant paste).
% Furthermore, the command |latex| can be replaced by any
% of its alternative versions such as |pdflatex|.
%
% %%%%%%%%%%%%%%%%%%%%%%%%%%%%%%%%%%%%%%%%%%%%%%%%%%%%%%%%%%%%%%%%%%%%%%%%%%%%%%
% %%%%%%%%%%%%%%%%%%%%%%%%%%%%%%%%%%%%%%%%%%%%%%%%%%%%%%%%%%%%%%%%%%%%%%%%%%%%%%
% \section{Implementation}
%\iffalse
%<*package>
%\fi
%
% This section describes the definitions file |childdoc.def|.

% The definitions cannot be loaded using |\usepackage| or |\RequirePackage|
% which has a mechanism to prevent loading a style file more than once.
% When loading the definitions by means of |\input|
% multiple instances have to be prevented manually:
%\iffalse
%This code needs to be before the `\ProvidesFile' directive
%which is defined at the beginning of this file.
%Therefore it is also placed there and commented out here.
%</package>
%<*discard>
%\fi
%    \begin{macrocode}
\ifdefined\childdocmain\endinput\fi
%    \end{macrocode}
%\iffalse
%</discard>
%<*package>
%\fi
%
% \macro{\ifchilddoc}
% \macro{\ifchilddocmanual}
% The conditional |\ifchilddoc| tells whether a
% child (true) or main (false) document is being compiled.
% The conditional |\ifchilddocmanual| tells whether
% the |\includeonly| mechanism is used (false) or
% the selection of child files must be performed manually (true).
% The definitions initialise to false:
%    \begin{macrocode}
\newif\ifchilddoc
\newif\ifchilddocmanual
%    \end{macrocode}

% \macro{\childdocname}
% \macro{\childdocjob}
% The macro |\childdocname| stores the name of the main document
% to be compiled. The macro |\childdocjob| stores the name of
% the document on which the \LaTeX{} compiler was originally invoked.
% The content of |\jobname| cannot be compared
% to filenames specified in the source due to different catcodes.
% The following code rescans |\jobname|, stores the result
% in |\childdocname| and saves a copy in |\childdocjob|:
%    \begin{macrocode}
\edef\childdocname{\scantokens\expandafter{\jobname\noexpand}}
\let\childdocjob\childdocname
%    \end{macrocode}

% \macro{\childdocdisable}
% The macro |\childdocdisable| prevents the main file
% from being processed more than once.
% At this stage, the main document command |\childdocmain|
% is assumed to be called once again where it should do nothing.
% Any subsequent call to it should prevent
% a secondary processing of the main document
% It overwrites the forwarding commands
% |\childdocof| and |\childdocforward|
% with empty macros to prevent further inclusions of the main document:
%    \begin{macrocode}
\newcommand{\childdocdisable}
{
  \renewcommand{\childdocmain}[1]{\renewcommand{\childdocmain}[1]{\endinput}}
  \renewcommand{\childdocof}[1]{}
  \renewcommand{\childdocby}[2][]{}
  \renewcommand{\childdocforward}[2][]{}
  \renewcommand{\childdocdisable}{}
}
%    \end{macrocode}

% \macro{\childdocmain}
% The macro |\childdocmain| is to be called at the top of the main file
% with nothing or the main filename (without extension) as argument.
% First, it breaks loops.
% If the argument is not empty and does not match |\childdocname|
% (which is set by the first inclusion of |childdoc.def|),
% |\ifchilddoc| is set to true, |\includeonly| is applied to the child file
% and |\jobname| is set to the main file
% (for proper handling of |.aux| files):
%    \begin{macrocode}
\newcommand{\childdocmain}[1]
{
  \childdocdisable\childdocmain{}
  \if?#1?\else
    \begingroup
      \def\childdoctmp{#1}
      \ifx\childdoctmp\childdocname
        \def\childdoctmp{}
      \else
        \def\childdoctmp
        {
          \childdoctrue
          \includeonly{\childdocname}
          \def\childdocjob{#1}
          \def\jobname{#1}
        }
      \fi
      \expandafter
    \endgroup
    \childdoctmp
  \fi
}
%    \end{macrocode}

% \macro{\childdocof}
% The command |\childdocof| redirects
% compilation to the main file |#1|.
%    \begin{macrocode}
\newcommand{\childdocof}[1]
{
  \childdocdisable
  \childdoctrue
  \includeonly{\childdocname}
  \def\jobname{#1}
  \def\childdocjob{#1}
  \input{#1}
}
%    \end{macrocode}

% \macro{\childdocby}
% The command |\childdocby| ....
%    \begin{macrocode}
\newcommand{\childdocby}[2][]
{
  \childdocdisable
  \childdoctrue
  \childdocmanualtrue
  \if?#1?\else
    \def\jobname{#2}
  \fi
  \def\childdocjob{#2}
  \input{#2}
  \endinput
}
%    \end{macrocode}

% \macro{\childdocforward}
% The command |\childdocforward| redirects
% compilation to the main file or
% (if the optional argument is given) a child file.
% Parameters are set as if the main file
% or a child file starting with |\childdocof| was compiled.
% Then compilation is handed over to the main file:
%    \begin{macrocode}
\newcommand{\childdocforward}[2][]
{
  \begingroup
    \if?#1?
      \def\childdoctmp
      {
        \def\childdocname{#2}
        \def\childdocjob{#2}
        \def\jobname{#2}
        \input{#2}
        \endinput
      }
    \else
      \def\childdoctmp
      {
        \childdocdisable
        \def\childdocname{#2}
        \childdoctrue
        \includeonly{#2}
        \def\childdocjob{#1}
        \def\jobname{#1}
        \input{#1}
        \endinput
      }
    \fi
    \expandafter
  \endgroup
  \childdoctmp
}
%    \end{macrocode}

% \macro{\childdocforwardprefix}
% The command |\childdocforwardprefix| redirects
% compilation to the main or a child file by means of a pattern.
% The prefix |#1| in the current filename is replaced by |#2|
% and the suffix of the current filename is kept
% (it is assumed that the filename does not contain the substring `|~~~|'
% which is used as a delimiter).
% Compilation is handed over to the new file by |\childdocforward|:
%    \begin{macrocode}
\newcommand{\childdocforwardprefix}[3][]
{
  \begingroup
    \def\childdocextract #2##1~~~{\def\childdoctmp{\childdocforward[#1]{#3##1}}}
    \expandafter\childdocextract\childdocname~~~
    \expandafter
  \endgroup
  \childdoctmp
}
%    \end{macrocode}

% \macro{\childdoc}
% The deprecated macro |\childdoc| is a legacy version of |\childdocmain|:
%    \begin{macrocode}
\newcommand{\childdoc}{\childdocmain}
%    \end{macrocode}

% \macro{\childdocredirect}
% The deprecated macro |\childdocredirect| is a legacy version
% of |\childdocforward| and |\childdocforwardprefix|:
%    \begin{macrocode}
\newcommand{\childdocredirect}[2][]
{
  \begingroup
    \if?#1?
      \def\childdoctmp{\childdocforward{#2}}
    \else
      \def\childdoctmp{\childdocforwardprefix{#1}{#2}}
    \fi
    \expandafter
  \endgroup
  \childdoctmp
}
%    \end{macrocode}

%\iffalse
%</package>
%\fi
%
\endinput

\childdocby{cdocsamp}
%    \end{macrocode}

%\iffalse
%</samplepart3|samplepart4>
%\fi
%
%\iffalse
%<*samplepart3>
%\fi
% Some text for part 3:
%    \begin{macrocode}
some text in part three
%    \end{macrocode}

%\iffalse
%</samplepart3>
%\fi
% Some text for part 4:
%\iffalse
%<*samplepart4>
%\fi
%    \begin{macrocode}
more text in part four
%    \end{macrocode}

%\iffalse
%</samplepart4>
%\fi
%
% %%%%%%%%%%%%%%%%%%%%%%%%%%%%%%%%%%%%%%
% \paragraph{Forwarding for a Complete Draft.}
%
% The following forwarding file |cdocsdrf.tex|
% compiles the main document in draft mode:
%\iffalse
%<*sampledraft>
%\fi
%    \begin{macrocode}
\def\version{draft}
% \iffalse
%
% childdoc.dtx Copyright (C) 2017-2018 Niklas Beisert
%
% This work may be distributed and/or modified under the
% conditions of the LaTeX Project Public License, either version 1.3
% of this license or (at your option) any later version.
% The latest version of this license is in
%   http://www.latex-project.org/lppl.txt
% and version 1.3 or later is part of all distributions of LaTeX
% version 2005/12/01 or later.
%
% This work has the LPPL maintenance status `maintained'.
%
% The Current Maintainer of this work is Niklas Beisert.
%
% This work consists of the files childdoc.dtx and childdoc.ins
% and the derived files childdoc.def and cdocsamp.tex with
% cdocsch1.tex, cdocsch2.tex, cdocsdrf.tex, cdocsfn1.tex, cdocsfn2.tex.
%
%<package>\ifdefined\childdocmain\endinput\fi
%<package>\ProvidesFile{childdoc.def}[2018/12/30 v2.0 child document driver]
%<samplemain>\ProvidesFile{cdocsamp.tex}[2018/12/30 v2.0 sample for childdoc]
%<*driver>
%\ProvidesFile{childdoc.drv}[2018/12/30 v2.0 childdoc reference manual file]
\PassOptionsToClass{10pt,a4paper}{article}
\documentclass{ltxdoc}

\usepackage[margin=35mm]{geometry}
\usepackage{hyperref}
\usepackage{hyperxmp}
\usepackage[usenames]{color}

\hypersetup{colorlinks=true}
\hypersetup{pdfstartview=FitH}
\hypersetup{pdfpagemode=UseNone}
\hypersetup{pdfsource={}}
\hypersetup{pdflang={en-UK}}
\hypersetup{pdfcopyright={Copyright 2017-2018 Niklas Beisert.
  This work may be distributed and/or modified under the
  conditions of the LaTeX Project Public License, either version 1.3
  of this license or (at your option) any later version.}}
\hypersetup{pdflicenseurl={http://www.latex-project.org/lppl.txt}}
\hypersetup{pdfcontactaddress={ETH Zurich, ITP, HIT K,
  Wolfgang-Pauli-Strasse 27}}
\hypersetup{pdfcontactpostcode={8093}}
\hypersetup{pdfcontactcity={Zurich}}
\hypersetup{pdfcontactcountry={Switzerland}}
\hypersetup{pdfcontactemail={nbeisert@itp.phys.ethz.ch}}
\hypersetup{pdfcontacturl={http://people.phys.ethz.ch/\xmptilde nbeisert/}}

\newcommand{\secref}[1]{\hyperref[#1]{section \ref*{#1}}}

\parskip1ex
\parindent0pt
\let\olditemize\itemize
\def\itemize{\olditemize\parskip0pt}

\begin{document}

\title{The \textsf{childdoc} Package}
\hypersetup{pdftitle={The childdoc Package}}
\author{Niklas Beisert\\[2ex]
  Institut f\"ur Theoretische Physik\\
  Eidgen\"ossische Technische Hochschule Z\"urich\\
  Wolfgang-Pauli-Strasse 27, 8093 Z\"urich, Switzerland\\[1ex]
  \href{mailto:nbeisert@itp.phys.ethz.ch}
  {\texttt{nbeisert@itp.phys.ethz.ch}}}
\hypersetup{pdfauthor={Niklas Beisert}}
\hypersetup{pdfsubject={Manual for the LaTeX2e Package childdoc}}
\date{30 December 2018, \textsf{v2.0}}
\maketitle

\begin{abstract}\noindent
\textsf{childdoc} is a \LaTeXe{} package
that enables the direct compilation
of document sections included by |\include|
to individual files.
\end{abstract}

\begingroup
\parskip0ex
\tableofcontents
\endgroup

%%%%%%%%%%%%%%%%%%%%%%%%%%%%%%%%%%%%%%%%%%%%%%%%%%%%%%%%%%%%%%%%%%%%%%%%%%%%%%%%
%%%%%%%%%%%%%%%%%%%%%%%%%%%%%%%%%%%%%%%%%%%%%%%%%%%%%%%%%%%%%%%%%%%%%%%%%%%%%%%%
\section{Introduction}

\LaTeX{} provides a mechanism to structure a large document (such as a book)
into a main file and several child files (containing the chapters)
using the |\include| command.
This mechanism is beneficial for documents
which span hundreds of pages in order to
make the source file(s) more manageable.
Moreover, compilation can be restricted to
selected child files by means of the |\includeonly| command.
The latter feature can be used to reduce the compilation time while editing
(this was significantly more useful in the earlier days of \LaTeX{})
or to generate a smaller document which is easier to navigate.
Another application of |\includeonly| is to generate
documents consisting of selected parts of the complete document.

However, there are a few drawbacks of the plain |\include| mechanism:
\begin{itemize}
\item
The child files cannot be compiled on their own,
they can only be compiled via the main file.
A naive editing environment
(such as a text editor with an option
to have the current file processed by \LaTeX)
may require one to switch to the main file before compiling;
attempting to compile the child file produces errors.
\item
The main file must be modified (each time)
to adjust the |\includeonly| command
to the present needs. This easily leaves the main file in a messy state.
\item
The generated document will always carry the filename
of the main document. This is inconvenient if
several child files are to be compiled and
to be kept for distribution.
\end{itemize}

The present package provides a simple interface
to make child files individually compilable by \LaTeX{}.
Compiling a child file then has the same effect as compiling
the main file with an |\includeonly| command
to select the appropriate child.
Moreover the generated document will carry the name of the child
rather than the main file.
This resolves all three above issues.

This feature is meant to make the editing of books,
thesis documents and lecture notes somewhat more convenient.
However, the package can also be used efficiently for
composing a series of documents (such as exercise sheets)
which are typically distributed individually.
It then assists the author in generating the individual documents
(potentially in different versions)
as well as a document containing the collected series.
Another application is in developing style files
or other kinds of included material
where compilation of the style file could redirect
to a sample or test file.

%%%%%%%%%%%%%%%%%%%%%%%%%%%%%%%%%%%%%%%%%%%%%%%%%%%%%%%%%%%%%%%%%%%%%%%%%%%%%%%%
%%%%%%%%%%%%%%%%%%%%%%%%%%%%%%%%%%%%%%%%%%%%%%%%%%%%%%%%%%%%%%%%%%%%%%%%%%%%%%%%
\section{Usage}

First of all, the package \textsf{childdoc} is \emph{not} a standard
\LaTeXe{} |.sty| style file! Therefore it needs to be invoked in
a non-standard way.

%%%%%%%%%%%%%%%%%%%%%%%%%%%%%%%%%%%%%%%%%%%%%%%%%%%%%%%%%%%%%%%%%%%%%%%%%%%%%%%%
\subsection{Included Files}
\label{sec:include}

%%%%%%%%%%%%%%%%%%%%%%%%%%%%%%%%%%%%%%%%
\DescribeMacro{\childdocmain}
To use the package, add the commands
\begin{center}
\begin{tabular}{l}
|\input{childdoc.def}|\\
|\childdocmain{}|\\
\end{tabular}
\end{center}
at the very top of the main \LaTeX{} file,
in particular \emph{before} the |\documentclass| statement!
The argument of |\childdocmain| should be left empty
(but it must be present).

%%%%%%%%%%%%%%%%%%%%%%%%%%%%%%%%%%%%%%%%
\DescribeMacro{\childdocof}
Furthermore, add the commands
\begin{center}
\begin{tabular}{l}
|\input{childdoc.def}|\\
|\childdocof{|\textit{main}|}|\\
\end{tabular}
\end{center}
at the top of every child file \textit{child}
which is included by |\include{|\textit{child}|}|
from within the main file
(or at least for those files to be compiled individually).
The argument \textit{main} must be the filename of the main file.

There are a couple of
considerations in setting up the main and child documents:

%%%%%%%%%%%%%%%%%%%%%%%%%%%%%%%%%%%%%%%%
\paragraph{Restrictions.}

Please note the following restrictions:
\begin{itemize}
\item
|\childdocmain| must be called with one argument \textit{main}
to ensure compatibility with earlier version of the package.
It must either be empty (|\childdocmain{}|)
or precisely match the filename of the main file in which it is specified.
See \secref{sec:detection} for further information.
\item
The filename \textit{main} must be specified without the |.tex| extension.
\item
The filename \textit{main} is case sensitive
(even in case-insensitive file systems)
due to internal string comparison.
\item
The argument \textit{main} should be fully expanded, it cannot be a macro.
\item
Subdirectories and special characters should be avoided in filenames.
\item
The command |\childdocmain{|\textit{main}|}| must be followed by a whitespace.
It should not be followed immediately by another command
or by a comment mark `|%|'.
This is because the \TeX{} parser reads the token immediately following
the argument of |\childdocmain| and puts it
at the beginning of every child section;
however, a white\-space is ignored.
\end{itemize}

%%%%%%%%%%%%%%%%%%%%%%%%%%%%%%%%%%%%%%%%
\paragraph{Content of Main File.}

It is advisable to place all content in the child files included by |\include|.
Any output contained in the main file will appear in all child documents
unless suppressed manually;
it cannot be suppressed automatically by the |\includeonly| directive
and thus should normally be avoided.
A method to include some content in the main file
by means of conditional processing is described in \secref{sec:conditional}.

%%%%%%%%%%%%%%%%%%%%%%%%%%%%%%%%%%%%%%%%
\paragraph{Page Numbering.}

When only a part of the document is compiled,
the appropriate numbering of pages
(as well as other status parameters)
is determined from the |.aux| files.
The latter contain information from previous passes.
However this information needs to propagate through
all intermediate child documents.
Therefore the page numbering in child documents may well
be inconsistent until the complete document is compiled at least once.

A useful (if unconventional) way to always ensure a consistent
page numbering is to restart the numbering in each child document
and denote the pages by `\textit{child}|.|\textit{page}'
where \textit{child} represents the chapter/section number of the child file.
This can be achieved by the command
|\numberwithin{page}{|\textit{child}|}|
of the \textsf{amsmath} package
where \textit{child} can be |chapter| or |section|
depending on the chosen structuring.
Alternatively, one can modify the macro |\thepage| appropriately
and reset the counter |page| at the start of each child file.

%%%%%%%%%%%%%%%%%%%%%%%%%%%%%%%%%%%%%%%%%%%%%%%%%%%%%%%%%%%%%%%%%%%%%%%%%%%%%%%%
\subsection{Conditional Processing}
\label{sec:conditional}

The package provides a mechanism to compile different versions
of a document. To customise the versions further some conditional processing
can come in handy to distinguish which version is being compiled.
The package provides two macros to describe the compilation context:

%%%%%%%%%%%%%%%%%%%%%%%%%%%%%%%%%%%%%%%%
\DescribeMacro{\ifchilddoc}
The conditional |\ifchilddoc| distinguishes between the compilation of
child documents and the main document:
%
\begin{center}
|\ifchilddoc |\textit{child-code}| |[|\||else |\textit{main-code}]| \||fi|
\end{center}

%%%%%%%%%%%%%%%%%%%%%%%%%%%%%%%%%%%%%%%%
\DescribeMacro{\childdocname}
\DescribeMacro{\childdocjob}
The macro |\childdocname| contains the filename (without extension)
of the main or child file being processed.
Note that |\childdocjob| will always contain the name of the main file.

%%%%%%%%%%%%%%%%%%%%%%%%%%%%%%%%%%%%%%%%
\paragraph{Title Page.}

Conditional processing can be used to include a title or banner page
in the main document when proper precautions are taken.
Importantly, the code in the main file should ensure that the page counter
(as well as other status parameters which are stored in the |.aux| files)
takes the same value after the conditional processing.
Otherwise the page numbers may take divergent values
depending on which part is compiled.

For example, a title page could be declared by:
%
\begin{center}
\begin{tabular}{l}
|\ifchilddoc\||else|\\
|\addtocounter{page}{-1}|\\
\textit{code for title page}\\
|\newpage|\\
|\||fi|
\end{tabular}
\end{center}
%
A banner page for the child documents can be generated by:
%
\begin{center}
\begin{tabular}{l}
|\ifchilddoc|\\
|\addtocounter{page}{-1}|\\
\textit{code for banner page}\\
|\newpage|\\
|\||fi|
\end{tabular}
\end{center}
%
Here one could write a message such as:
\begin{center}
|This is the part \childdocname{} of \childdocjob{}.|
\end{center}

%%%%%%%%%%%%%%%%%%%%%%%%%%%%%%%%%%%%%%%%%%%%%%%%%%%%%%%%%%%%%%%%%%%%%%%%%%%%%%%%
\subsection{Flags}
\label{sec:flags}

The package makes it easy to generate different versions
of the main or child documents.
To this end compilation flags can be defined
and assigned different default values.
They will be particularly useful in conjunction
with the forwarding mechanism described in \secref{sec:forward}.

For example, it may be useful to have a flag |\version|
which can be set to |draft| or |final|.
The document source will contain some conditional code
depending on the value of |\version|.
Suppose further, the flag should default to |final| for the main file
and to |draft| for child files
which is a natural assignment for editing the document.
This is achieved by placing the following code
in the preamble of the main document
(below the |\childdocmain| directive):
%
\begin{center}
\begin{tabular}{l}
|\ifchilddoc|\\
|\providecommand{\version}{draft}|\\
|\||else|\\
|\providecommand{\version}{final}|\\
|\||fi|
\end{tabular}
\end{center}
%
The definition by |\providecommand| makes sure
that previous definitions are not overwritten.
Further statements |\providecommand{\version}{...}|
can thus be added before the above code to override it.

For the main file, one might add a line
(between |\childdocmain| and the above block)
%
\begin{center}
|%\ifchilddoc\||else\providecommand{\version}{draft}\||fi|
\end{center}
%
which can be uncommented to produce a draft version.
Likewise one can add a line to the very top of a child file
(above the |\childdocof{|\textit{main}|}| directive)
%
\begin{center}
|%\providecommand{\version}{final}|
\end{center}
%
which can be uncommented to produce the final version of this child document.

%%%%%%%%%%%%%%%%%%%%%%%%%%%%%%%%%%%%%%%%%%%%%%%%%%%%%%%%%%%%%%%%%%%%%%%%%%%%%%%%
\subsection{Forwarding}
\label{sec:forward}

Different versions of the main or child documents
using compilation flags as described in \secref{sec:flags}
can be (permanently) stored in different files
for convenient compilation, viewing and distribution.
To this end, the package defines a command
to pass on compilation to a different file:

%%%%%%%%%%%%%%%%%%%%%%%%%%%%%%%%%%%%%%%%
\DescribeMacro{\childdocforward}
The command |\childdocforward| redirects processing to
another source file:
%
\begin{center}
\begin{tabular}{l}
|\input{childdoc.def}|\\
|\childdocforward[|\textit{main}|]{|\textit{dest}|}|\\
\end{tabular}
\end{center}
%
The argument \textit{dest} is the destination file
(without extension).
It should be the main file or one of the child files.
Note that further \textsf{childdoc} directives
such as |\childdocof| and |\childdocforward|
in the indicated file will be processed in this form.
The optional argument \textit{main}
passes on directly to the main file \textit{main}
while pretending to compile the child \textit{dest}.
This form behaves as if \textit{dest}
issues |\childdocof{|\textit{main}|}| right away,
and no further \textsf{childdoc} directives will be processed.

%%%%%%%%%%%%%%%%%%%%%%%%%%%%%%%%%%%%%%%%
\DescribeMacro{\...prefix}
In the alternative form |\childdocforwardprefix|,
%
\begin{center}
\begin{tabular}{l}
|\input{childdoc.def}|\\
|\childdocforwardprefix[|\textit{main}|]{|\textit{prefix}|}{|\textit{dest}|}|
\end{tabular}
\end{center}
%
the destination file is determined by a pattern
depending on the current file:
To make this work, the current file must be called
`{\textit{prefix}\hspace{0.2em}\textit{suffix}}'
with \textit{prefix} matching precisely the argument.
Processing is then passed on to the file
`{\textit{dest}\hspace{0.2em}\textit{suffix}}'.
Surely, the same effect is achieved by
directly specifying the
argument `{\textit{dest}\hspace{0.2em}\textit{suffix}}'
in the first form.
However, that requires to set up a different file
for each child. With the alternative form of the command
all these files can have exactly the same content
which simplifies setting them up and maintaining them.

For example, the following file |draft.tex|
with a compilation flag |\version| as described in \secref{sec:flags}
compiles the main document as a draft:
%
\begin{center}
\begin{tabular}{l}
|\def\version{draft}|\\
|\input{childdoc.def}|\\
|\childdocforward{|\textit{main}|}|
\end{tabular}
\end{center}
%
Likewise, the following files |final|\textit{nn}|.tex|
compile the final version of the child document
|child|\textit{nn}|.tex|:
%
\begin{center}
\begin{tabular}{l}
|\def\version{final}|\\
|\input{childdoc.def}|\\
|\childdocforwardprefix{final}{child}|
\end{tabular}
\end{center}
%

Note that when several versions of a main file and/or of each child file
are to be generated, it may be convenient to set up a |Makefile| or
shell script to automatise the process.

%%%%%%%%%%%%%%%%%%%%%%%%%%%%%%%%%%%%%%%%%%%%%%%%%%%%%%%%%%%%%%%%%%%%%%%%%%%%%%%%
\subsection{Command Line Processing}
\label{sec:commandline}

The effect of redirection files can also be achieved by invoking
the \LaTeX{} compiler with a more elaborate command line.
Most conveniently this should be done as part
of a shell script or a |Makefile|.

When using \textsf{childdoc} in the main file, the following
command lines effectively perform a redirection
(note that depending on the shell being used,
backslashes may have to be doubled: `|\|' $\to$ `|\\|'):
%
\begin{center}
|... -jobname "|\textit{target}|" |\\|"|[\textit{flags}]%
|\input{childdoc.def}\childdocforward[|\textit{main}|]{|\textit{dest}|}"|
\end{center}
%
Here \textit{target} is the name of the output file,
\textit{main} is the name of the main file
and \textit{dest} is the name of the main or child file to be processed
(all filenames without extensions).
The optional argument \textit{main} can be omitted
if \textit{main} matches \textit{dest}.
Optionally, compilation \textit{flags} can be defined via |\def| commands.
This command line makes the \TeX{} engine believe
it is compiling the file \textit{target}
whose content is specified as the latter parameter.
The provided code then forwards the processing to
\textit{main} or \textit{dest} as described in \secref{sec:forward}.

%%%%%%%%%%%%%%%%%%%%%%%%%%%%%%%%%%%%%%%%%%%%%%%%%%%%%%%%%%%%%%%%%%%%%%%%%%%%%%%%
\subsection{Include by Input}
\label{sec:input}

Including child documents by |\include| has some restrictions by design.
Most notably, the content of a child document always occupies
its own set of pages; pages cannot be shared between child documents.
Usually, this behaviour makes perfect sense
because each child document contain an essential part of the document.
However, in some situations it may be desirable to compose
a document from a collection of parts
without having mandatory page breaks between then.
For this case, the package
provides a mechanism to include parts
by |\input| which can also be processed individually.
However, by construction this mechanism
requires manual handling of the content to be output.

%%%%%%%%%%%%%%%%%%%%%%%%%%%%%%%%%%%%%%%%
\DescribeMacro{\ifchilddocmanual}
The main file should be prepared as usual, see \secref{sec:include}.
However, the document body must make a distinction
between processing of an individual part and of the main document, e.g.:
%
\begin{center}
\begin{tabular}{l}
|\ifchilddocmanual|\\
|\input{\childdocname}|\\
|\||else|\\
\textit{document body with }|\input{|\textit{part}|}|\\
|\||fi|
\end{tabular}
\end{center}
%
The conditional |\ifchilddocmanual| is true whenever
a part to be included by |\input| is being compiled,
and the name of the part is stored in |\childdocname|.

%%%%%%%%%%%%%%%%%%%%%%%%%%%%%%%%%%%%%%%%
\DescribeMacro{\childdocby}
Each part to be included by |\input| should start with:
%
\begin{center}
\begin{tabular}{l}
|\input{childdoc.def}|\\
|\childdocby{|\textit{main}|}|\\
\end{tabular}
\end{center}
%
The directive |\childdocby| is similar to |\childdocof|
described in \secref{sec:include},
but the subsequent selection of content must be done manually.
To that end, both |\ifchilddoc| and |\ifchilddocmanual|
will be true upon processing of a part,
and the name of the part is stored in |\childdocname|.
Note that |\jobname| will be set to the filename of the current part
so that each part receives an individual |.aux| file
that does not interfere with the |.aux| file(s) of the main document.
This behaviour can be altered by the alternative form
|\childdocby[*]{|\textit{main}|}| (with a non-empty optional argument)
which uses the |.aux| file of the main document
by setting |\jobname| to \textit{main}.

%%%%%%%%%%%%%%%%%%%%%%%%%%%%%%%%%%%%%%%%%%%%%%%%%%%%%%%%%%%%%%%%%%%%%%%%%%%%%%%%
\subsection{Driver Development}
\label{sec:driver}

The \textsf{childdoc} mechanism can also be use for the development
of definition files such as \LaTeX{} styles or classes.
This case differs from the above setup with multiple parts
included by |\include| in that no |\includeonly| should be invoked.
This can be achieved by starting the include file
(before |\ProvidesPackage|) with:
%
\begin{center}
\begin{tabular}{l}
|\input{childdoc.def}|\\
|\childdocforward{|\textit{main}|}|\\
\end{tabular}
\end{center}
%
or alternatively with:
%
\begin{center}
\begin{tabular}{l}
|\input{childdoc.def}|\\
|\childdocby{|\textit{main}|}|\\
\end{tabular}
\end{center}
%
Both forms have slightly different effects as described above.
The main file is prepared as usual, see \secref{sec:include}.

%%%%%%%%%%%%%%%%%%%%%%%%%%%%%%%%%%%%%%%%%%%%%%%%%%%%%%%%%%%%%%%%%%%%%%%%%%%%%%%%
\subsection{Legacy Detection}
\label{sec:detection}

The directive |\childdocmain| in the main file can detect
whether the complete document or merely a child is to be compiled
even without using the directive |\childdocof|.
This method is deprecated because it is less robust
and there is no compelling reason to use it;
it is merely provided for backward compatibility
and it may be removed in future versions.

If the detection mechanism is to be used,
it is mandatory to correctly specify
the filename of the main file as the argument of |\childdocmain|:
%
\begin{center}
\begin{tabular}{l}
|\input{childdoc.def}|\\
|\childdocmain{|\textit{main}|}|\\
\end{tabular}
\end{center}
%
If |\jobname| does not match the argument \textit{main} of |\childdocmain|,
it is assumed that |\jobname| points to the child file to be compiled.
When using |\childdocmain| with the main file specified as argument,
it suffices to start a child file
with just |\input{|\textit{main}|}|
without loading of the package and using |\childdocof|.
If instead all processing is done
with the appropriate \textsf{childdoc} directives,
the argument of \textit{main} of |\childdocmain| can be empty.

An alternative version of the command line processing described
in \secref{sec:commandline} using the detection mechanism reads:
%
\begin{center}
|... -jobname "|\textit{target}|" "|[\textit{flags}]%
[|\def\jobname{|\textit{dest}|}|]|\input{|\textit{main}|}"|
\end{center}

%%%%%%%%%%%%%%%%%%%%%%%%%%%%%%%%%%%%%%%%%%%%%%%%%%%%%%%%%%%%%%%%%%%%%%%%%%%%%%%%
\subsection{Manual Code}
\label{sec:manual}

In case one cannot be certain whether the definitions file |childdoc.def|
is installed on the target \TeX{} distribution
and one prefers not to ship it,
it is conceivable to paste a few relevant commands into the sources.

To that end, drop all statements |\input{childdoc.def}|
and perform the replacements as outlined below.
Instead of |\childdocmain{|\textit{main}|}| add the following code
to the top of the main file:
%
\begin{center}
\begin{tabular}{l}
|\||ifdefined\childdocname\endinput\||fi\newif\ifchilddoc|\\
|\edef\childdocname{\scantokens\expandafter{\jobname\noexpand}}|\\
|\def\childdocmain{|\textit{main}|}\||ifx\childdocmain\childdocname\||else|\\
|\childdoctrue\includeonly{\childdocname}\let\jobname\childdocmain\||fi|\\
\end{tabular}
\end{center}
%
Instead of |\childdocof{|\textit{main}|}| just include the main file
at the top of each child file:
%
\begin{center}
|\input{|\textit{main}|}|
\end{center}
%
A simple redirection |\childdocforward{|\textit{dest}|}| is achieved by:
%
\begin{center}
|\def\jobname{|\textit{dest}|}\input{\jobname}|
\end{center}
%
The redirection with prefix
|\childdocforwardprefix[|\textit{prefix}|]{|\textit{dest}|}|
is accomplished by:
%
\begin{center}
\begin{tabular}{l}
|{\edef\jobname{\scantokens\expandafter{\jobname\noexpand}}|\\
|\def\redirectjob |\textit{prefix}|#1~~~{\gdef\jobname{|\textit{dest}|#1}}|\\
|\expandafter\redirectjob\jobname~~~}\input{\jobname}|
\end{tabular}
\end{center}

In an alternative approach,
child documents can be compiled by a specific command line
without additional code or specific definitions:
%
\begin{center}
|... -jobname "|\textit{target}|" "|[\textit{flags}]%
|\includeonly{|\textit{dest}|}\input{|\textit{main}|}"|
\end{center}
%

%%%%%%%%%%%%%%%%%%%%%%%%%%%%%%%%%%%%%%%%%%%%%%%%%%%%%%%%%%%%%%%%%%%%%%%%%%%%%%%%
%%%%%%%%%%%%%%%%%%%%%%%%%%%%%%%%%%%%%%%%%%%%%%%%%%%%%%%%%%%%%%%%%%%%%%%%%%%%%%%%
\section{Information}

%%%%%%%%%%%%%%%%%%%%%%%%%%%%%%%%%%%%%%%%%%%%%%%%%%%%%%%%%%%%%%%%%%%%%%%%%%%%%%%%
\subsection{Copyright}

Copyright \copyright{} 2017--2018 Niklas Beisert

This work may be distributed and/or modified under the
conditions of the \LaTeX{} Project Public License, either version 1.3
of this license or (at your option) any later version.
The latest version of this license is in
  \url{http://www.latex-project.org/lppl.txt}
and version 1.3 or later is part of all distributions of \LaTeX{}
version 2005/12/01 or later.

This work has the LPPL maintenance status `maintained'.

The Current Maintainer of this work is Niklas Beisert.

This work consists of the files |README.txt|, |childdoc.ins| and |childdoc.dtx|
as well as the derived files |childdoc.def|, |cdocsamp.tex|
with |cdocsch1.tex|, |cdocsch2.tex|, |cdocspt3.tex|, |cdocspt4.tex|,
|cdocsdrf.tex|, |cdocsfn1.tex|, |cdocsfn2.tex|
as well as |childdoc.pdf|.

%%%%%%%%%%%%%%%%%%%%%%%%%%%%%%%%%%%%%%%%%%%%%%%%%%%%%%%%%%%%%%%%%%%%%%%%%%%%%%%%
\subsection{Files and Installation}

The package consists of the files:
%
\begin{center}
\begin{tabular}{ll}
    |README.txt|   & readme file \\
    |childdoc.ins| & installation file \\
    |childdoc.dtx| & source file \\
    |childdoc.def| & definition file \\
    |cdocsamp.tex| & sample main file \\
    |cdocsch1.tex| & sample include file \\
    |cdocsch2.tex| & sample include file \\
    |cdocspt3.tex| & sample part file \\
    |cdocspt4.tex| & sample part file \\
    |cdocsdrf.tex| & sample redirection file \\
    |cdocsfn1.tex| & sample redirection file \\
    |cdocsfn2.tex| & sample redirection file \\
    |childdoc.pdf| & manual
\end{tabular}
\end{center}
%
The distribution consists of the files
|README.txt|, |childdoc.ins| and |childdoc.dtx|.
%
\begin{itemize}
\item
Run (pdf)\LaTeX{} on |childdoc.dtx|
to compile the manual |childdoc.pdf| (this file).
\item
Run \LaTeX{} on |childdoc.ins| to create the definitions file |childdoc.def|
and the sample |cdocsamp.tex| with include files
|cdocsch1.tex|, |cdocsch2.tex|, |cdocspt3.tex|, |cdocspt4.tex|,
|cdocsdrf.tex|, |cdocsfn1.tex|, |cdocsfn2.tex|.
Then copy the file |childdoc.def| to an appropriate directory of your \LaTeX{}
distribution, e.g.\ \textit{texmf-root}|/tex/latex/childdoc|.
\end{itemize}

%%%%%%%%%%%%%%%%%%%%%%%%%%%%%%%%%%%%%%%%%%%%%%%%%%%%%%%%%%%%%%%%%%%%%%%%%%%%%%%%
\subsection{Related CTAN Packages}

There are several other packages which offer a similar functionality:
%
\begin{itemize}
\item
The packages
\href{http://ctan.org/pkg/docmute}{\textsf{docmute}},
\href{http://ctan.org/pkg/includex}{\textsf{includex}} and
\href{http://ctan.org/pkg/standalone}{\textsf{standalone}}
provide commands to include only the document body of
a child file thus allowing both files to be compiled individually.
\item
The packages \href{http://ctan.org/pkg/subdocs}{\textsf{subdocs}}
and \href{http://ctan.org/pkg/subfiles}{\textsf{subfiles}}
provide structures in which the main and child documents can be
encapsulated and allowing them to be compiled individually.
The inclusion mechanism is different from the conventional |\include|.
\item
The package \href{http://ctan.org/pkg/combine}{\textsf{combine}}
is an elaborate solution to combine several documents into one.
\end{itemize}
%
See also the CTAN topic \href{http://ctan.org/topic/subdocs}{\textsf{subdocs}}
for further related packages.
The present package differs from the above solutions in that
a document structure constructed with the conventional |\include| mechanism
just needs two extra commands at the top of every file
such that all constituent files can be compiled individually.

%%%%%%%%%%%%%%%%%%%%%%%%%%%%%%%%%%%%%%%%%%%%%%%%%%%%%%%%%%%%%%%%%%%%%%%%%%%%%%%%
%\subsection{Feature Suggestions}
%
%The following is a list of features which may be useful for future
%versions of this package:
%%
%\begin{itemize}
%\item
%\ldots
%\end{itemize}

%%%%%%%%%%%%%%%%%%%%%%%%%%%%%%%%%%%%%%%%%%%%%%%%%%%%%%%%%%%%%%%%%%%%%%%%%%%%%%%%
\subsection{Revision History}

%%%%%%%%%%%%%%%%%%%%%%%%%%%%%%%%%%%%%%%%
\paragraph{v2.0:} 2018/12/30

\begin{itemize}
\item
immediate forward processing
\item
added |\childdocby| mechanism
\item
manual restructured
\end{itemize}

%%%%%%%%%%%%%%%%%%%%%%%%%%%%%%%%%%%%%%%%
\paragraph{v1.6:} 2018/01/17

\begin{itemize}
\item
application for development of include files
\item
corrections to manual
\end{itemize}

%%%%%%%%%%%%%%%%%%%%%%%%%%%%%%%%%%%%%%%%
\paragraph{v1.5:} 2017/05/21

\begin{itemize}
\item
more complete structuring introduced
\item
|\childdocof| introduced
\item
|\childdoc| renamed to |\childdocmain|
\item
|\childredirect| renamed to |\childdocforward| and |\childdocforwardprefix|
and functionality expanded
\end{itemize}

%%%%%%%%%%%%%%%%%%%%%%%%%%%%%%%%%%%%%%%%
\paragraph{v1.0:} 2017/04/27

\begin{itemize}
\item
manual and install package
\item
first version published on CTAN
\end{itemize}

%%%%%%%%%%%%%%%%%%%%%%%%%%%%%%%%%%%%%%%%
\paragraph{v0.6:} 2017/04/26

\begin{itemize}
\item
redirection mechanism added
\end{itemize}

%%%%%%%%%%%%%%%%%%%%%%%%%%%%%%%%%%%%%%%%
\paragraph{v0.5:} 2017/04/26

\begin{itemize}
\item
functionality in definition file
\end{itemize}


%%%%%%%%%%%%%%%%%%%%%%%%%%%%%%%%%%%%%%%%%%%%%%%%%%%%%%%%%%%%%%%%%%%%%%%%%%%%%%%%
%%%%%%%%%%%%%%%%%%%%%%%%%%%%%%%%%%%%%%%%%%%%%%%%%%%%%%%%%%%%%%%%%%%%%%%%%%%%%%%%
%%%%%%%%%%%%%%%%%%%%%%%%%%%%%%%%%%%%%%%%%%%%%%%%%%%%%%%%%%%%%%%%%%%%%%%%%%%%%%%%
\appendix

\settowidth\MacroIndent{\rmfamily\scriptsize 000\ }

 \DocInput{childdoc.dtx}

\end{document}
%</driver>
% \fi
%
% %%%%%%%%%%%%%%%%%%%%%%%%%%%%%%%%%%%%%%%%%%%%%%%%%%%%%%%%%%%%%%%%%%%%%%%%%%%%%%
% %%%%%%%%%%%%%%%%%%%%%%%%%%%%%%%%%%%%%%%%%%%%%%%%%%%%%%%%%%%%%%%%%%%%%%%%%%%%%%
% \section{Sample}
%\iffalse
%<*samplemain>
%\fi
%
% The following presents a sample document
% with two chapters, two parts, a title page,
% a compile flag as well as three forwarding files to set the flag.
% It consists of eight |.tex| files:
% \begin{center}
% \begin{tabular}{ll}
% |cdocsamp.tex|&main file\\
% |cdocsch1.tex|&include file for chapter 1\\
% |cdocsch2.tex|&include file for chapter 2\\
% |cdocspt3.tex|&include file for part 3\\
% |cdocspt4.tex|&include file for part 4\\
% |cdocsdrf.tex|&forwarding file for main file in draft mode\\
% |cdocsfi1.tex|&forwarding file for final version of chapter 1\\
% |cdocsfi2.tex|&forwarding file for final version of chapter 2\\
% \end{tabular}
% \end{center}
% Each of the eight files can be compiled directly by the \LaTeX{} compiler.
%
% %%%%%%%%%%%%%%%%%%%%%%%%%%%%%%%%%%%%%%
% \paragraph{Main File.}
%
% The main file is called |cdocsamp.tex|.
%
% Load the \textsf{childdoc} definitions and
% declare the filename for the main document:
%    \begin{macrocode}
\input{childdoc.def}
\childdocmain{}
%    \end{macrocode}

% Optional override for |\version| flag:
%    \begin{macrocode}
%%\ifchilddoc\else\providecommand{\version}{draft}\fi
%    \end{macrocode}

% Define the default values for the |\version| flag
% (|final| for the main file and |draft| for childs):
%    \begin{macrocode}
\ifchilddoc
\providecommand{\version}{draft}
\else
\providecommand{\version}{final}
\fi
%    \end{macrocode}

% Load the standard document class:
%    \begin{macrocode}
\documentclass[12pt]{article}
%    \end{macrocode}

% Start the document body:
%    \begin{macrocode}
\begin{document}
%    \end{macrocode}

% Declare a title page.
% Print title, part of document being processed and version flag:
%    \begin{macrocode}
\addtocounter{page}{-1}
\begin{center}
{\LARGE\bfseries{}childdoc example\par}
\vspace{1cm}
\ifchilddoc
\ifchilddocmanual part\else chapter\fi:
`\childdocname' of `\childdocjob'\par
\else
main document: `\childdocjob'\par
\fi
version: \version\par
\end{center}
\newpage
%    \end{macrocode}

% Manually include selected file,
% otherwise process as usual:
%    \begin{macrocode}
\ifchilddocmanual
\section*{part `\childdocname'}
\input{\childdocname}
\else
%    \end{macrocode}

% Include the two chapters:
%    \begin{macrocode}
\include{cdocsch1}
\include{cdocsch2}
%    \end{macrocode}

% Include the two parts unless only chapters should be displayed:
%    \begin{macrocode}
\ifchilddoc\else
\section{part three}
\input{cdocspt3}
\section{part four}
\input{cdocspt4}
\fi
%    \end{macrocode}

% Process as usual until here:
%    \begin{macrocode}
\fi
%    \end{macrocode}

% End of document body:
%    \begin{macrocode}
\end{document}
%    \end{macrocode}
%\iffalse
%</samplemain>
%\fi
%
% %%%%%%%%%%%%%%%%%%%%%%%%%%%%%%%%%%%%%%
% \paragraph{Chapter Include Files.}
%
% The include files are called |cdocsch1.tex| and |cdocsch2.tex|.
%
%\iffalse
%<*samplechap1|samplechap2>
%\fi

% Optional override for |\version| flag:
%    \begin{macrocode}
%%\providecommand{\version}{final}
%    \end{macrocode}

% Include the main document:
%    \begin{macrocode}
\input{childdoc.def}
\childdocof{cdocsamp}
%    \end{macrocode}

%\iffalse
%</samplechap1|samplechap2>
%\fi
%
%\iffalse
%<*samplechap1>
%\fi
% Some text for chapter 1:
%    \begin{macrocode}
\section{one}
some text in chapter one
%    \end{macrocode}

%\iffalse
%</samplechap1>
%\fi
% Some text for chapter 2:
%\iffalse
%<*samplechap2>
%\fi
%    \begin{macrocode}
\section{two}
more text in chapter two
%    \end{macrocode}

%\iffalse
%</samplechap2>
%\fi
%
% %%%%%%%%%%%%%%%%%%%%%%%%%%%%%%%%%%%%%%
% \paragraph{Part Include Files.}
%
% The include files are called |cdocspt3.tex| and |cdocspt4.tex|.
%
%\iffalse
%<*samplepart3|samplepart4>
%\fi

% Optional override for |\version| flag:
%    \begin{macrocode}
%%\providecommand{\version}{final}
%    \end{macrocode}

% Include the main document:
%    \begin{macrocode}
\input{childdoc.def}
\childdocby{cdocsamp}
%    \end{macrocode}

%\iffalse
%</samplepart3|samplepart4>
%\fi
%
%\iffalse
%<*samplepart3>
%\fi
% Some text for part 3:
%    \begin{macrocode}
some text in part three
%    \end{macrocode}

%\iffalse
%</samplepart3>
%\fi
% Some text for part 4:
%\iffalse
%<*samplepart4>
%\fi
%    \begin{macrocode}
more text in part four
%    \end{macrocode}

%\iffalse
%</samplepart4>
%\fi
%
% %%%%%%%%%%%%%%%%%%%%%%%%%%%%%%%%%%%%%%
% \paragraph{Forwarding for a Complete Draft.}
%
% The following forwarding file |cdocsdrf.tex|
% compiles the main document in draft mode:
%\iffalse
%<*sampledraft>
%\fi
%    \begin{macrocode}
\def\version{draft}
\input{childdoc.def}
\childdocforward{cdocsamp}
%    \end{macrocode}

%\iffalse
%</sampledraft>
%\fi
%
% %%%%%%%%%%%%%%%%%%%%%%%%%%%%%%%%%%%%%%
% \paragraph{Forwarding for Final Version of the Chapters.}
%
% The following forwarding files |cdocsfn1.tex| and |cdocsfn2.tex|
% (with identical content)
% compile the final versions of the child documents
% |cdocsch1.tex| and |cdocsch2.tex|, respectively:
%\iffalse
%<*samplefinal>
%\fi
%    \begin{macrocode}
\def\version{final}
\input{childdoc.def}
\childdocforwardprefix[cdocsamp]{cdocsfn}{cdocsch}
%    \end{macrocode}

%\iffalse
%</samplefinal>
%\fi
%
% %%%%%%%%%%%%%%%%%%%%%%%%%%%%%%%%%%%%%%
% \paragraph{Command Line Processing.}
%
% The following three command lines generate the output files
% |cdocscld|, |cdocscl1| and |cdocscl2|
% which should be identical to
% |cdocsdrf|, |cdocsch1| and |cdocsfn2|, respectively:
% \begin{center}
% \begin{tabular}{l}
% |latex -jobname cdocscld \|\\
% |  "\def\version{draft}\input{childdoc.def}\childdocforward{cdocsamp}"|\\
% |latex -jobname cdocscl1 \|\\
% |  "\input{childdoc.def}\childdocforward[cdocsamp]{cdocsch1}"|\\
% |latex -jobname cdocscl2 \|\\
% |  "\def\version{final}\input{childdoc.def}\childdocforward{cdocsch2}"|
% \end{tabular}
% \end{center}
% Note that the trailing backslash on each first line
% merely continues the input to the second line
% (for convenient cut ant paste).
% Furthermore, the command |latex| can be replaced by any
% of its alternative versions such as |pdflatex|.
%
% %%%%%%%%%%%%%%%%%%%%%%%%%%%%%%%%%%%%%%%%%%%%%%%%%%%%%%%%%%%%%%%%%%%%%%%%%%%%%%
% %%%%%%%%%%%%%%%%%%%%%%%%%%%%%%%%%%%%%%%%%%%%%%%%%%%%%%%%%%%%%%%%%%%%%%%%%%%%%%
% \section{Implementation}
%\iffalse
%<*package>
%\fi
%
% This section describes the definitions file |childdoc.def|.

% The definitions cannot be loaded using |\usepackage| or |\RequirePackage|
% which has a mechanism to prevent loading a style file more than once.
% When loading the definitions by means of |\input|
% multiple instances have to be prevented manually:
%\iffalse
%This code needs to be before the `\ProvidesFile' directive
%which is defined at the beginning of this file.
%Therefore it is also placed there and commented out here.
%</package>
%<*discard>
%\fi
%    \begin{macrocode}
\ifdefined\childdocmain\endinput\fi
%    \end{macrocode}
%\iffalse
%</discard>
%<*package>
%\fi
%
% \macro{\ifchilddoc}
% \macro{\ifchilddocmanual}
% The conditional |\ifchilddoc| tells whether a
% child (true) or main (false) document is being compiled.
% The conditional |\ifchilddocmanual| tells whether
% the |\includeonly| mechanism is used (false) or
% the selection of child files must be performed manually (true).
% The definitions initialise to false:
%    \begin{macrocode}
\newif\ifchilddoc
\newif\ifchilddocmanual
%    \end{macrocode}

% \macro{\childdocname}
% \macro{\childdocjob}
% The macro |\childdocname| stores the name of the main document
% to be compiled. The macro |\childdocjob| stores the name of
% the document on which the \LaTeX{} compiler was originally invoked.
% The content of |\jobname| cannot be compared
% to filenames specified in the source due to different catcodes.
% The following code rescans |\jobname|, stores the result
% in |\childdocname| and saves a copy in |\childdocjob|:
%    \begin{macrocode}
\edef\childdocname{\scantokens\expandafter{\jobname\noexpand}}
\let\childdocjob\childdocname
%    \end{macrocode}

% \macro{\childdocdisable}
% The macro |\childdocdisable| prevents the main file
% from being processed more than once.
% At this stage, the main document command |\childdocmain|
% is assumed to be called once again where it should do nothing.
% Any subsequent call to it should prevent
% a secondary processing of the main document
% It overwrites the forwarding commands
% |\childdocof| and |\childdocforward|
% with empty macros to prevent further inclusions of the main document:
%    \begin{macrocode}
\newcommand{\childdocdisable}
{
  \renewcommand{\childdocmain}[1]{\renewcommand{\childdocmain}[1]{\endinput}}
  \renewcommand{\childdocof}[1]{}
  \renewcommand{\childdocby}[2][]{}
  \renewcommand{\childdocforward}[2][]{}
  \renewcommand{\childdocdisable}{}
}
%    \end{macrocode}

% \macro{\childdocmain}
% The macro |\childdocmain| is to be called at the top of the main file
% with nothing or the main filename (without extension) as argument.
% First, it breaks loops.
% If the argument is not empty and does not match |\childdocname|
% (which is set by the first inclusion of |childdoc.def|),
% |\ifchilddoc| is set to true, |\includeonly| is applied to the child file
% and |\jobname| is set to the main file
% (for proper handling of |.aux| files):
%    \begin{macrocode}
\newcommand{\childdocmain}[1]
{
  \childdocdisable\childdocmain{}
  \if?#1?\else
    \begingroup
      \def\childdoctmp{#1}
      \ifx\childdoctmp\childdocname
        \def\childdoctmp{}
      \else
        \def\childdoctmp
        {
          \childdoctrue
          \includeonly{\childdocname}
          \def\childdocjob{#1}
          \def\jobname{#1}
        }
      \fi
      \expandafter
    \endgroup
    \childdoctmp
  \fi
}
%    \end{macrocode}

% \macro{\childdocof}
% The command |\childdocof| redirects
% compilation to the main file |#1|.
%    \begin{macrocode}
\newcommand{\childdocof}[1]
{
  \childdocdisable
  \childdoctrue
  \includeonly{\childdocname}
  \def\jobname{#1}
  \def\childdocjob{#1}
  \input{#1}
}
%    \end{macrocode}

% \macro{\childdocby}
% The command |\childdocby| ....
%    \begin{macrocode}
\newcommand{\childdocby}[2][]
{
  \childdocdisable
  \childdoctrue
  \childdocmanualtrue
  \if?#1?\else
    \def\jobname{#2}
  \fi
  \def\childdocjob{#2}
  \input{#2}
  \endinput
}
%    \end{macrocode}

% \macro{\childdocforward}
% The command |\childdocforward| redirects
% compilation to the main file or
% (if the optional argument is given) a child file.
% Parameters are set as if the main file
% or a child file starting with |\childdocof| was compiled.
% Then compilation is handed over to the main file:
%    \begin{macrocode}
\newcommand{\childdocforward}[2][]
{
  \begingroup
    \if?#1?
      \def\childdoctmp
      {
        \def\childdocname{#2}
        \def\childdocjob{#2}
        \def\jobname{#2}
        \input{#2}
        \endinput
      }
    \else
      \def\childdoctmp
      {
        \childdocdisable
        \def\childdocname{#2}
        \childdoctrue
        \includeonly{#2}
        \def\childdocjob{#1}
        \def\jobname{#1}
        \input{#1}
        \endinput
      }
    \fi
    \expandafter
  \endgroup
  \childdoctmp
}
%    \end{macrocode}

% \macro{\childdocforwardprefix}
% The command |\childdocforwardprefix| redirects
% compilation to the main or a child file by means of a pattern.
% The prefix |#1| in the current filename is replaced by |#2|
% and the suffix of the current filename is kept
% (it is assumed that the filename does not contain the substring `|~~~|'
% which is used as a delimiter).
% Compilation is handed over to the new file by |\childdocforward|:
%    \begin{macrocode}
\newcommand{\childdocforwardprefix}[3][]
{
  \begingroup
    \def\childdocextract #2##1~~~{\def\childdoctmp{\childdocforward[#1]{#3##1}}}
    \expandafter\childdocextract\childdocname~~~
    \expandafter
  \endgroup
  \childdoctmp
}
%    \end{macrocode}

% \macro{\childdoc}
% The deprecated macro |\childdoc| is a legacy version of |\childdocmain|:
%    \begin{macrocode}
\newcommand{\childdoc}{\childdocmain}
%    \end{macrocode}

% \macro{\childdocredirect}
% The deprecated macro |\childdocredirect| is a legacy version
% of |\childdocforward| and |\childdocforwardprefix|:
%    \begin{macrocode}
\newcommand{\childdocredirect}[2][]
{
  \begingroup
    \if?#1?
      \def\childdoctmp{\childdocforward{#2}}
    \else
      \def\childdoctmp{\childdocforwardprefix{#1}{#2}}
    \fi
    \expandafter
  \endgroup
  \childdoctmp
}
%    \end{macrocode}

%\iffalse
%</package>
%\fi
%
\endinput

\childdocforward{cdocsamp}
%    \end{macrocode}

%\iffalse
%</sampledraft>
%\fi
%
% %%%%%%%%%%%%%%%%%%%%%%%%%%%%%%%%%%%%%%
% \paragraph{Forwarding for Final Version of the Chapters.}
%
% The following forwarding files |cdocsfn1.tex| and |cdocsfn2.tex|
% (with identical content)
% compile the final versions of the child documents
% |cdocsch1.tex| and |cdocsch2.tex|, respectively:
%\iffalse
%<*samplefinal>
%\fi
%    \begin{macrocode}
\def\version{final}
% \iffalse
%
% childdoc.dtx Copyright (C) 2017-2018 Niklas Beisert
%
% This work may be distributed and/or modified under the
% conditions of the LaTeX Project Public License, either version 1.3
% of this license or (at your option) any later version.
% The latest version of this license is in
%   http://www.latex-project.org/lppl.txt
% and version 1.3 or later is part of all distributions of LaTeX
% version 2005/12/01 or later.
%
% This work has the LPPL maintenance status `maintained'.
%
% The Current Maintainer of this work is Niklas Beisert.
%
% This work consists of the files childdoc.dtx and childdoc.ins
% and the derived files childdoc.def and cdocsamp.tex with
% cdocsch1.tex, cdocsch2.tex, cdocsdrf.tex, cdocsfn1.tex, cdocsfn2.tex.
%
%<package>\ifdefined\childdocmain\endinput\fi
%<package>\ProvidesFile{childdoc.def}[2018/12/30 v2.0 child document driver]
%<samplemain>\ProvidesFile{cdocsamp.tex}[2018/12/30 v2.0 sample for childdoc]
%<*driver>
%\ProvidesFile{childdoc.drv}[2018/12/30 v2.0 childdoc reference manual file]
\PassOptionsToClass{10pt,a4paper}{article}
\documentclass{ltxdoc}

\usepackage[margin=35mm]{geometry}
\usepackage{hyperref}
\usepackage{hyperxmp}
\usepackage[usenames]{color}

\hypersetup{colorlinks=true}
\hypersetup{pdfstartview=FitH}
\hypersetup{pdfpagemode=UseNone}
\hypersetup{pdfsource={}}
\hypersetup{pdflang={en-UK}}
\hypersetup{pdfcopyright={Copyright 2017-2018 Niklas Beisert.
  This work may be distributed and/or modified under the
  conditions of the LaTeX Project Public License, either version 1.3
  of this license or (at your option) any later version.}}
\hypersetup{pdflicenseurl={http://www.latex-project.org/lppl.txt}}
\hypersetup{pdfcontactaddress={ETH Zurich, ITP, HIT K,
  Wolfgang-Pauli-Strasse 27}}
\hypersetup{pdfcontactpostcode={8093}}
\hypersetup{pdfcontactcity={Zurich}}
\hypersetup{pdfcontactcountry={Switzerland}}
\hypersetup{pdfcontactemail={nbeisert@itp.phys.ethz.ch}}
\hypersetup{pdfcontacturl={http://people.phys.ethz.ch/\xmptilde nbeisert/}}

\newcommand{\secref}[1]{\hyperref[#1]{section \ref*{#1}}}

\parskip1ex
\parindent0pt
\let\olditemize\itemize
\def\itemize{\olditemize\parskip0pt}

\begin{document}

\title{The \textsf{childdoc} Package}
\hypersetup{pdftitle={The childdoc Package}}
\author{Niklas Beisert\\[2ex]
  Institut f\"ur Theoretische Physik\\
  Eidgen\"ossische Technische Hochschule Z\"urich\\
  Wolfgang-Pauli-Strasse 27, 8093 Z\"urich, Switzerland\\[1ex]
  \href{mailto:nbeisert@itp.phys.ethz.ch}
  {\texttt{nbeisert@itp.phys.ethz.ch}}}
\hypersetup{pdfauthor={Niklas Beisert}}
\hypersetup{pdfsubject={Manual for the LaTeX2e Package childdoc}}
\date{30 December 2018, \textsf{v2.0}}
\maketitle

\begin{abstract}\noindent
\textsf{childdoc} is a \LaTeXe{} package
that enables the direct compilation
of document sections included by |\include|
to individual files.
\end{abstract}

\begingroup
\parskip0ex
\tableofcontents
\endgroup

%%%%%%%%%%%%%%%%%%%%%%%%%%%%%%%%%%%%%%%%%%%%%%%%%%%%%%%%%%%%%%%%%%%%%%%%%%%%%%%%
%%%%%%%%%%%%%%%%%%%%%%%%%%%%%%%%%%%%%%%%%%%%%%%%%%%%%%%%%%%%%%%%%%%%%%%%%%%%%%%%
\section{Introduction}

\LaTeX{} provides a mechanism to structure a large document (such as a book)
into a main file and several child files (containing the chapters)
using the |\include| command.
This mechanism is beneficial for documents
which span hundreds of pages in order to
make the source file(s) more manageable.
Moreover, compilation can be restricted to
selected child files by means of the |\includeonly| command.
The latter feature can be used to reduce the compilation time while editing
(this was significantly more useful in the earlier days of \LaTeX{})
or to generate a smaller document which is easier to navigate.
Another application of |\includeonly| is to generate
documents consisting of selected parts of the complete document.

However, there are a few drawbacks of the plain |\include| mechanism:
\begin{itemize}
\item
The child files cannot be compiled on their own,
they can only be compiled via the main file.
A naive editing environment
(such as a text editor with an option
to have the current file processed by \LaTeX)
may require one to switch to the main file before compiling;
attempting to compile the child file produces errors.
\item
The main file must be modified (each time)
to adjust the |\includeonly| command
to the present needs. This easily leaves the main file in a messy state.
\item
The generated document will always carry the filename
of the main document. This is inconvenient if
several child files are to be compiled and
to be kept for distribution.
\end{itemize}

The present package provides a simple interface
to make child files individually compilable by \LaTeX{}.
Compiling a child file then has the same effect as compiling
the main file with an |\includeonly| command
to select the appropriate child.
Moreover the generated document will carry the name of the child
rather than the main file.
This resolves all three above issues.

This feature is meant to make the editing of books,
thesis documents and lecture notes somewhat more convenient.
However, the package can also be used efficiently for
composing a series of documents (such as exercise sheets)
which are typically distributed individually.
It then assists the author in generating the individual documents
(potentially in different versions)
as well as a document containing the collected series.
Another application is in developing style files
or other kinds of included material
where compilation of the style file could redirect
to a sample or test file.

%%%%%%%%%%%%%%%%%%%%%%%%%%%%%%%%%%%%%%%%%%%%%%%%%%%%%%%%%%%%%%%%%%%%%%%%%%%%%%%%
%%%%%%%%%%%%%%%%%%%%%%%%%%%%%%%%%%%%%%%%%%%%%%%%%%%%%%%%%%%%%%%%%%%%%%%%%%%%%%%%
\section{Usage}

First of all, the package \textsf{childdoc} is \emph{not} a standard
\LaTeXe{} |.sty| style file! Therefore it needs to be invoked in
a non-standard way.

%%%%%%%%%%%%%%%%%%%%%%%%%%%%%%%%%%%%%%%%%%%%%%%%%%%%%%%%%%%%%%%%%%%%%%%%%%%%%%%%
\subsection{Included Files}
\label{sec:include}

%%%%%%%%%%%%%%%%%%%%%%%%%%%%%%%%%%%%%%%%
\DescribeMacro{\childdocmain}
To use the package, add the commands
\begin{center}
\begin{tabular}{l}
|\input{childdoc.def}|\\
|\childdocmain{}|\\
\end{tabular}
\end{center}
at the very top of the main \LaTeX{} file,
in particular \emph{before} the |\documentclass| statement!
The argument of |\childdocmain| should be left empty
(but it must be present).

%%%%%%%%%%%%%%%%%%%%%%%%%%%%%%%%%%%%%%%%
\DescribeMacro{\childdocof}
Furthermore, add the commands
\begin{center}
\begin{tabular}{l}
|\input{childdoc.def}|\\
|\childdocof{|\textit{main}|}|\\
\end{tabular}
\end{center}
at the top of every child file \textit{child}
which is included by |\include{|\textit{child}|}|
from within the main file
(or at least for those files to be compiled individually).
The argument \textit{main} must be the filename of the main file.

There are a couple of
considerations in setting up the main and child documents:

%%%%%%%%%%%%%%%%%%%%%%%%%%%%%%%%%%%%%%%%
\paragraph{Restrictions.}

Please note the following restrictions:
\begin{itemize}
\item
|\childdocmain| must be called with one argument \textit{main}
to ensure compatibility with earlier version of the package.
It must either be empty (|\childdocmain{}|)
or precisely match the filename of the main file in which it is specified.
See \secref{sec:detection} for further information.
\item
The filename \textit{main} must be specified without the |.tex| extension.
\item
The filename \textit{main} is case sensitive
(even in case-insensitive file systems)
due to internal string comparison.
\item
The argument \textit{main} should be fully expanded, it cannot be a macro.
\item
Subdirectories and special characters should be avoided in filenames.
\item
The command |\childdocmain{|\textit{main}|}| must be followed by a whitespace.
It should not be followed immediately by another command
or by a comment mark `|%|'.
This is because the \TeX{} parser reads the token immediately following
the argument of |\childdocmain| and puts it
at the beginning of every child section;
however, a white\-space is ignored.
\end{itemize}

%%%%%%%%%%%%%%%%%%%%%%%%%%%%%%%%%%%%%%%%
\paragraph{Content of Main File.}

It is advisable to place all content in the child files included by |\include|.
Any output contained in the main file will appear in all child documents
unless suppressed manually;
it cannot be suppressed automatically by the |\includeonly| directive
and thus should normally be avoided.
A method to include some content in the main file
by means of conditional processing is described in \secref{sec:conditional}.

%%%%%%%%%%%%%%%%%%%%%%%%%%%%%%%%%%%%%%%%
\paragraph{Page Numbering.}

When only a part of the document is compiled,
the appropriate numbering of pages
(as well as other status parameters)
is determined from the |.aux| files.
The latter contain information from previous passes.
However this information needs to propagate through
all intermediate child documents.
Therefore the page numbering in child documents may well
be inconsistent until the complete document is compiled at least once.

A useful (if unconventional) way to always ensure a consistent
page numbering is to restart the numbering in each child document
and denote the pages by `\textit{child}|.|\textit{page}'
where \textit{child} represents the chapter/section number of the child file.
This can be achieved by the command
|\numberwithin{page}{|\textit{child}|}|
of the \textsf{amsmath} package
where \textit{child} can be |chapter| or |section|
depending on the chosen structuring.
Alternatively, one can modify the macro |\thepage| appropriately
and reset the counter |page| at the start of each child file.

%%%%%%%%%%%%%%%%%%%%%%%%%%%%%%%%%%%%%%%%%%%%%%%%%%%%%%%%%%%%%%%%%%%%%%%%%%%%%%%%
\subsection{Conditional Processing}
\label{sec:conditional}

The package provides a mechanism to compile different versions
of a document. To customise the versions further some conditional processing
can come in handy to distinguish which version is being compiled.
The package provides two macros to describe the compilation context:

%%%%%%%%%%%%%%%%%%%%%%%%%%%%%%%%%%%%%%%%
\DescribeMacro{\ifchilddoc}
The conditional |\ifchilddoc| distinguishes between the compilation of
child documents and the main document:
%
\begin{center}
|\ifchilddoc |\textit{child-code}| |[|\||else |\textit{main-code}]| \||fi|
\end{center}

%%%%%%%%%%%%%%%%%%%%%%%%%%%%%%%%%%%%%%%%
\DescribeMacro{\childdocname}
\DescribeMacro{\childdocjob}
The macro |\childdocname| contains the filename (without extension)
of the main or child file being processed.
Note that |\childdocjob| will always contain the name of the main file.

%%%%%%%%%%%%%%%%%%%%%%%%%%%%%%%%%%%%%%%%
\paragraph{Title Page.}

Conditional processing can be used to include a title or banner page
in the main document when proper precautions are taken.
Importantly, the code in the main file should ensure that the page counter
(as well as other status parameters which are stored in the |.aux| files)
takes the same value after the conditional processing.
Otherwise the page numbers may take divergent values
depending on which part is compiled.

For example, a title page could be declared by:
%
\begin{center}
\begin{tabular}{l}
|\ifchilddoc\||else|\\
|\addtocounter{page}{-1}|\\
\textit{code for title page}\\
|\newpage|\\
|\||fi|
\end{tabular}
\end{center}
%
A banner page for the child documents can be generated by:
%
\begin{center}
\begin{tabular}{l}
|\ifchilddoc|\\
|\addtocounter{page}{-1}|\\
\textit{code for banner page}\\
|\newpage|\\
|\||fi|
\end{tabular}
\end{center}
%
Here one could write a message such as:
\begin{center}
|This is the part \childdocname{} of \childdocjob{}.|
\end{center}

%%%%%%%%%%%%%%%%%%%%%%%%%%%%%%%%%%%%%%%%%%%%%%%%%%%%%%%%%%%%%%%%%%%%%%%%%%%%%%%%
\subsection{Flags}
\label{sec:flags}

The package makes it easy to generate different versions
of the main or child documents.
To this end compilation flags can be defined
and assigned different default values.
They will be particularly useful in conjunction
with the forwarding mechanism described in \secref{sec:forward}.

For example, it may be useful to have a flag |\version|
which can be set to |draft| or |final|.
The document source will contain some conditional code
depending on the value of |\version|.
Suppose further, the flag should default to |final| for the main file
and to |draft| for child files
which is a natural assignment for editing the document.
This is achieved by placing the following code
in the preamble of the main document
(below the |\childdocmain| directive):
%
\begin{center}
\begin{tabular}{l}
|\ifchilddoc|\\
|\providecommand{\version}{draft}|\\
|\||else|\\
|\providecommand{\version}{final}|\\
|\||fi|
\end{tabular}
\end{center}
%
The definition by |\providecommand| makes sure
that previous definitions are not overwritten.
Further statements |\providecommand{\version}{...}|
can thus be added before the above code to override it.

For the main file, one might add a line
(between |\childdocmain| and the above block)
%
\begin{center}
|%\ifchilddoc\||else\providecommand{\version}{draft}\||fi|
\end{center}
%
which can be uncommented to produce a draft version.
Likewise one can add a line to the very top of a child file
(above the |\childdocof{|\textit{main}|}| directive)
%
\begin{center}
|%\providecommand{\version}{final}|
\end{center}
%
which can be uncommented to produce the final version of this child document.

%%%%%%%%%%%%%%%%%%%%%%%%%%%%%%%%%%%%%%%%%%%%%%%%%%%%%%%%%%%%%%%%%%%%%%%%%%%%%%%%
\subsection{Forwarding}
\label{sec:forward}

Different versions of the main or child documents
using compilation flags as described in \secref{sec:flags}
can be (permanently) stored in different files
for convenient compilation, viewing and distribution.
To this end, the package defines a command
to pass on compilation to a different file:

%%%%%%%%%%%%%%%%%%%%%%%%%%%%%%%%%%%%%%%%
\DescribeMacro{\childdocforward}
The command |\childdocforward| redirects processing to
another source file:
%
\begin{center}
\begin{tabular}{l}
|\input{childdoc.def}|\\
|\childdocforward[|\textit{main}|]{|\textit{dest}|}|\\
\end{tabular}
\end{center}
%
The argument \textit{dest} is the destination file
(without extension).
It should be the main file or one of the child files.
Note that further \textsf{childdoc} directives
such as |\childdocof| and |\childdocforward|
in the indicated file will be processed in this form.
The optional argument \textit{main}
passes on directly to the main file \textit{main}
while pretending to compile the child \textit{dest}.
This form behaves as if \textit{dest}
issues |\childdocof{|\textit{main}|}| right away,
and no further \textsf{childdoc} directives will be processed.

%%%%%%%%%%%%%%%%%%%%%%%%%%%%%%%%%%%%%%%%
\DescribeMacro{\...prefix}
In the alternative form |\childdocforwardprefix|,
%
\begin{center}
\begin{tabular}{l}
|\input{childdoc.def}|\\
|\childdocforwardprefix[|\textit{main}|]{|\textit{prefix}|}{|\textit{dest}|}|
\end{tabular}
\end{center}
%
the destination file is determined by a pattern
depending on the current file:
To make this work, the current file must be called
`{\textit{prefix}\hspace{0.2em}\textit{suffix}}'
with \textit{prefix} matching precisely the argument.
Processing is then passed on to the file
`{\textit{dest}\hspace{0.2em}\textit{suffix}}'.
Surely, the same effect is achieved by
directly specifying the
argument `{\textit{dest}\hspace{0.2em}\textit{suffix}}'
in the first form.
However, that requires to set up a different file
for each child. With the alternative form of the command
all these files can have exactly the same content
which simplifies setting them up and maintaining them.

For example, the following file |draft.tex|
with a compilation flag |\version| as described in \secref{sec:flags}
compiles the main document as a draft:
%
\begin{center}
\begin{tabular}{l}
|\def\version{draft}|\\
|\input{childdoc.def}|\\
|\childdocforward{|\textit{main}|}|
\end{tabular}
\end{center}
%
Likewise, the following files |final|\textit{nn}|.tex|
compile the final version of the child document
|child|\textit{nn}|.tex|:
%
\begin{center}
\begin{tabular}{l}
|\def\version{final}|\\
|\input{childdoc.def}|\\
|\childdocforwardprefix{final}{child}|
\end{tabular}
\end{center}
%

Note that when several versions of a main file and/or of each child file
are to be generated, it may be convenient to set up a |Makefile| or
shell script to automatise the process.

%%%%%%%%%%%%%%%%%%%%%%%%%%%%%%%%%%%%%%%%%%%%%%%%%%%%%%%%%%%%%%%%%%%%%%%%%%%%%%%%
\subsection{Command Line Processing}
\label{sec:commandline}

The effect of redirection files can also be achieved by invoking
the \LaTeX{} compiler with a more elaborate command line.
Most conveniently this should be done as part
of a shell script or a |Makefile|.

When using \textsf{childdoc} in the main file, the following
command lines effectively perform a redirection
(note that depending on the shell being used,
backslashes may have to be doubled: `|\|' $\to$ `|\\|'):
%
\begin{center}
|... -jobname "|\textit{target}|" |\\|"|[\textit{flags}]%
|\input{childdoc.def}\childdocforward[|\textit{main}|]{|\textit{dest}|}"|
\end{center}
%
Here \textit{target} is the name of the output file,
\textit{main} is the name of the main file
and \textit{dest} is the name of the main or child file to be processed
(all filenames without extensions).
The optional argument \textit{main} can be omitted
if \textit{main} matches \textit{dest}.
Optionally, compilation \textit{flags} can be defined via |\def| commands.
This command line makes the \TeX{} engine believe
it is compiling the file \textit{target}
whose content is specified as the latter parameter.
The provided code then forwards the processing to
\textit{main} or \textit{dest} as described in \secref{sec:forward}.

%%%%%%%%%%%%%%%%%%%%%%%%%%%%%%%%%%%%%%%%%%%%%%%%%%%%%%%%%%%%%%%%%%%%%%%%%%%%%%%%
\subsection{Include by Input}
\label{sec:input}

Including child documents by |\include| has some restrictions by design.
Most notably, the content of a child document always occupies
its own set of pages; pages cannot be shared between child documents.
Usually, this behaviour makes perfect sense
because each child document contain an essential part of the document.
However, in some situations it may be desirable to compose
a document from a collection of parts
without having mandatory page breaks between then.
For this case, the package
provides a mechanism to include parts
by |\input| which can also be processed individually.
However, by construction this mechanism
requires manual handling of the content to be output.

%%%%%%%%%%%%%%%%%%%%%%%%%%%%%%%%%%%%%%%%
\DescribeMacro{\ifchilddocmanual}
The main file should be prepared as usual, see \secref{sec:include}.
However, the document body must make a distinction
between processing of an individual part and of the main document, e.g.:
%
\begin{center}
\begin{tabular}{l}
|\ifchilddocmanual|\\
|\input{\childdocname}|\\
|\||else|\\
\textit{document body with }|\input{|\textit{part}|}|\\
|\||fi|
\end{tabular}
\end{center}
%
The conditional |\ifchilddocmanual| is true whenever
a part to be included by |\input| is being compiled,
and the name of the part is stored in |\childdocname|.

%%%%%%%%%%%%%%%%%%%%%%%%%%%%%%%%%%%%%%%%
\DescribeMacro{\childdocby}
Each part to be included by |\input| should start with:
%
\begin{center}
\begin{tabular}{l}
|\input{childdoc.def}|\\
|\childdocby{|\textit{main}|}|\\
\end{tabular}
\end{center}
%
The directive |\childdocby| is similar to |\childdocof|
described in \secref{sec:include},
but the subsequent selection of content must be done manually.
To that end, both |\ifchilddoc| and |\ifchilddocmanual|
will be true upon processing of a part,
and the name of the part is stored in |\childdocname|.
Note that |\jobname| will be set to the filename of the current part
so that each part receives an individual |.aux| file
that does not interfere with the |.aux| file(s) of the main document.
This behaviour can be altered by the alternative form
|\childdocby[*]{|\textit{main}|}| (with a non-empty optional argument)
which uses the |.aux| file of the main document
by setting |\jobname| to \textit{main}.

%%%%%%%%%%%%%%%%%%%%%%%%%%%%%%%%%%%%%%%%%%%%%%%%%%%%%%%%%%%%%%%%%%%%%%%%%%%%%%%%
\subsection{Driver Development}
\label{sec:driver}

The \textsf{childdoc} mechanism can also be use for the development
of definition files such as \LaTeX{} styles or classes.
This case differs from the above setup with multiple parts
included by |\include| in that no |\includeonly| should be invoked.
This can be achieved by starting the include file
(before |\ProvidesPackage|) with:
%
\begin{center}
\begin{tabular}{l}
|\input{childdoc.def}|\\
|\childdocforward{|\textit{main}|}|\\
\end{tabular}
\end{center}
%
or alternatively with:
%
\begin{center}
\begin{tabular}{l}
|\input{childdoc.def}|\\
|\childdocby{|\textit{main}|}|\\
\end{tabular}
\end{center}
%
Both forms have slightly different effects as described above.
The main file is prepared as usual, see \secref{sec:include}.

%%%%%%%%%%%%%%%%%%%%%%%%%%%%%%%%%%%%%%%%%%%%%%%%%%%%%%%%%%%%%%%%%%%%%%%%%%%%%%%%
\subsection{Legacy Detection}
\label{sec:detection}

The directive |\childdocmain| in the main file can detect
whether the complete document or merely a child is to be compiled
even without using the directive |\childdocof|.
This method is deprecated because it is less robust
and there is no compelling reason to use it;
it is merely provided for backward compatibility
and it may be removed in future versions.

If the detection mechanism is to be used,
it is mandatory to correctly specify
the filename of the main file as the argument of |\childdocmain|:
%
\begin{center}
\begin{tabular}{l}
|\input{childdoc.def}|\\
|\childdocmain{|\textit{main}|}|\\
\end{tabular}
\end{center}
%
If |\jobname| does not match the argument \textit{main} of |\childdocmain|,
it is assumed that |\jobname| points to the child file to be compiled.
When using |\childdocmain| with the main file specified as argument,
it suffices to start a child file
with just |\input{|\textit{main}|}|
without loading of the package and using |\childdocof|.
If instead all processing is done
with the appropriate \textsf{childdoc} directives,
the argument of \textit{main} of |\childdocmain| can be empty.

An alternative version of the command line processing described
in \secref{sec:commandline} using the detection mechanism reads:
%
\begin{center}
|... -jobname "|\textit{target}|" "|[\textit{flags}]%
[|\def\jobname{|\textit{dest}|}|]|\input{|\textit{main}|}"|
\end{center}

%%%%%%%%%%%%%%%%%%%%%%%%%%%%%%%%%%%%%%%%%%%%%%%%%%%%%%%%%%%%%%%%%%%%%%%%%%%%%%%%
\subsection{Manual Code}
\label{sec:manual}

In case one cannot be certain whether the definitions file |childdoc.def|
is installed on the target \TeX{} distribution
and one prefers not to ship it,
it is conceivable to paste a few relevant commands into the sources.

To that end, drop all statements |\input{childdoc.def}|
and perform the replacements as outlined below.
Instead of |\childdocmain{|\textit{main}|}| add the following code
to the top of the main file:
%
\begin{center}
\begin{tabular}{l}
|\||ifdefined\childdocname\endinput\||fi\newif\ifchilddoc|\\
|\edef\childdocname{\scantokens\expandafter{\jobname\noexpand}}|\\
|\def\childdocmain{|\textit{main}|}\||ifx\childdocmain\childdocname\||else|\\
|\childdoctrue\includeonly{\childdocname}\let\jobname\childdocmain\||fi|\\
\end{tabular}
\end{center}
%
Instead of |\childdocof{|\textit{main}|}| just include the main file
at the top of each child file:
%
\begin{center}
|\input{|\textit{main}|}|
\end{center}
%
A simple redirection |\childdocforward{|\textit{dest}|}| is achieved by:
%
\begin{center}
|\def\jobname{|\textit{dest}|}\input{\jobname}|
\end{center}
%
The redirection with prefix
|\childdocforwardprefix[|\textit{prefix}|]{|\textit{dest}|}|
is accomplished by:
%
\begin{center}
\begin{tabular}{l}
|{\edef\jobname{\scantokens\expandafter{\jobname\noexpand}}|\\
|\def\redirectjob |\textit{prefix}|#1~~~{\gdef\jobname{|\textit{dest}|#1}}|\\
|\expandafter\redirectjob\jobname~~~}\input{\jobname}|
\end{tabular}
\end{center}

In an alternative approach,
child documents can be compiled by a specific command line
without additional code or specific definitions:
%
\begin{center}
|... -jobname "|\textit{target}|" "|[\textit{flags}]%
|\includeonly{|\textit{dest}|}\input{|\textit{main}|}"|
\end{center}
%

%%%%%%%%%%%%%%%%%%%%%%%%%%%%%%%%%%%%%%%%%%%%%%%%%%%%%%%%%%%%%%%%%%%%%%%%%%%%%%%%
%%%%%%%%%%%%%%%%%%%%%%%%%%%%%%%%%%%%%%%%%%%%%%%%%%%%%%%%%%%%%%%%%%%%%%%%%%%%%%%%
\section{Information}

%%%%%%%%%%%%%%%%%%%%%%%%%%%%%%%%%%%%%%%%%%%%%%%%%%%%%%%%%%%%%%%%%%%%%%%%%%%%%%%%
\subsection{Copyright}

Copyright \copyright{} 2017--2018 Niklas Beisert

This work may be distributed and/or modified under the
conditions of the \LaTeX{} Project Public License, either version 1.3
of this license or (at your option) any later version.
The latest version of this license is in
  \url{http://www.latex-project.org/lppl.txt}
and version 1.3 or later is part of all distributions of \LaTeX{}
version 2005/12/01 or later.

This work has the LPPL maintenance status `maintained'.

The Current Maintainer of this work is Niklas Beisert.

This work consists of the files |README.txt|, |childdoc.ins| and |childdoc.dtx|
as well as the derived files |childdoc.def|, |cdocsamp.tex|
with |cdocsch1.tex|, |cdocsch2.tex|, |cdocspt3.tex|, |cdocspt4.tex|,
|cdocsdrf.tex|, |cdocsfn1.tex|, |cdocsfn2.tex|
as well as |childdoc.pdf|.

%%%%%%%%%%%%%%%%%%%%%%%%%%%%%%%%%%%%%%%%%%%%%%%%%%%%%%%%%%%%%%%%%%%%%%%%%%%%%%%%
\subsection{Files and Installation}

The package consists of the files:
%
\begin{center}
\begin{tabular}{ll}
    |README.txt|   & readme file \\
    |childdoc.ins| & installation file \\
    |childdoc.dtx| & source file \\
    |childdoc.def| & definition file \\
    |cdocsamp.tex| & sample main file \\
    |cdocsch1.tex| & sample include file \\
    |cdocsch2.tex| & sample include file \\
    |cdocspt3.tex| & sample part file \\
    |cdocspt4.tex| & sample part file \\
    |cdocsdrf.tex| & sample redirection file \\
    |cdocsfn1.tex| & sample redirection file \\
    |cdocsfn2.tex| & sample redirection file \\
    |childdoc.pdf| & manual
\end{tabular}
\end{center}
%
The distribution consists of the files
|README.txt|, |childdoc.ins| and |childdoc.dtx|.
%
\begin{itemize}
\item
Run (pdf)\LaTeX{} on |childdoc.dtx|
to compile the manual |childdoc.pdf| (this file).
\item
Run \LaTeX{} on |childdoc.ins| to create the definitions file |childdoc.def|
and the sample |cdocsamp.tex| with include files
|cdocsch1.tex|, |cdocsch2.tex|, |cdocspt3.tex|, |cdocspt4.tex|,
|cdocsdrf.tex|, |cdocsfn1.tex|, |cdocsfn2.tex|.
Then copy the file |childdoc.def| to an appropriate directory of your \LaTeX{}
distribution, e.g.\ \textit{texmf-root}|/tex/latex/childdoc|.
\end{itemize}

%%%%%%%%%%%%%%%%%%%%%%%%%%%%%%%%%%%%%%%%%%%%%%%%%%%%%%%%%%%%%%%%%%%%%%%%%%%%%%%%
\subsection{Related CTAN Packages}

There are several other packages which offer a similar functionality:
%
\begin{itemize}
\item
The packages
\href{http://ctan.org/pkg/docmute}{\textsf{docmute}},
\href{http://ctan.org/pkg/includex}{\textsf{includex}} and
\href{http://ctan.org/pkg/standalone}{\textsf{standalone}}
provide commands to include only the document body of
a child file thus allowing both files to be compiled individually.
\item
The packages \href{http://ctan.org/pkg/subdocs}{\textsf{subdocs}}
and \href{http://ctan.org/pkg/subfiles}{\textsf{subfiles}}
provide structures in which the main and child documents can be
encapsulated and allowing them to be compiled individually.
The inclusion mechanism is different from the conventional |\include|.
\item
The package \href{http://ctan.org/pkg/combine}{\textsf{combine}}
is an elaborate solution to combine several documents into one.
\end{itemize}
%
See also the CTAN topic \href{http://ctan.org/topic/subdocs}{\textsf{subdocs}}
for further related packages.
The present package differs from the above solutions in that
a document structure constructed with the conventional |\include| mechanism
just needs two extra commands at the top of every file
such that all constituent files can be compiled individually.

%%%%%%%%%%%%%%%%%%%%%%%%%%%%%%%%%%%%%%%%%%%%%%%%%%%%%%%%%%%%%%%%%%%%%%%%%%%%%%%%
%\subsection{Feature Suggestions}
%
%The following is a list of features which may be useful for future
%versions of this package:
%%
%\begin{itemize}
%\item
%\ldots
%\end{itemize}

%%%%%%%%%%%%%%%%%%%%%%%%%%%%%%%%%%%%%%%%%%%%%%%%%%%%%%%%%%%%%%%%%%%%%%%%%%%%%%%%
\subsection{Revision History}

%%%%%%%%%%%%%%%%%%%%%%%%%%%%%%%%%%%%%%%%
\paragraph{v2.0:} 2018/12/30

\begin{itemize}
\item
immediate forward processing
\item
added |\childdocby| mechanism
\item
manual restructured
\end{itemize}

%%%%%%%%%%%%%%%%%%%%%%%%%%%%%%%%%%%%%%%%
\paragraph{v1.6:} 2018/01/17

\begin{itemize}
\item
application for development of include files
\item
corrections to manual
\end{itemize}

%%%%%%%%%%%%%%%%%%%%%%%%%%%%%%%%%%%%%%%%
\paragraph{v1.5:} 2017/05/21

\begin{itemize}
\item
more complete structuring introduced
\item
|\childdocof| introduced
\item
|\childdoc| renamed to |\childdocmain|
\item
|\childredirect| renamed to |\childdocforward| and |\childdocforwardprefix|
and functionality expanded
\end{itemize}

%%%%%%%%%%%%%%%%%%%%%%%%%%%%%%%%%%%%%%%%
\paragraph{v1.0:} 2017/04/27

\begin{itemize}
\item
manual and install package
\item
first version published on CTAN
\end{itemize}

%%%%%%%%%%%%%%%%%%%%%%%%%%%%%%%%%%%%%%%%
\paragraph{v0.6:} 2017/04/26

\begin{itemize}
\item
redirection mechanism added
\end{itemize}

%%%%%%%%%%%%%%%%%%%%%%%%%%%%%%%%%%%%%%%%
\paragraph{v0.5:} 2017/04/26

\begin{itemize}
\item
functionality in definition file
\end{itemize}


%%%%%%%%%%%%%%%%%%%%%%%%%%%%%%%%%%%%%%%%%%%%%%%%%%%%%%%%%%%%%%%%%%%%%%%%%%%%%%%%
%%%%%%%%%%%%%%%%%%%%%%%%%%%%%%%%%%%%%%%%%%%%%%%%%%%%%%%%%%%%%%%%%%%%%%%%%%%%%%%%
%%%%%%%%%%%%%%%%%%%%%%%%%%%%%%%%%%%%%%%%%%%%%%%%%%%%%%%%%%%%%%%%%%%%%%%%%%%%%%%%
\appendix

\settowidth\MacroIndent{\rmfamily\scriptsize 000\ }

 \DocInput{childdoc.dtx}

\end{document}
%</driver>
% \fi
%
% %%%%%%%%%%%%%%%%%%%%%%%%%%%%%%%%%%%%%%%%%%%%%%%%%%%%%%%%%%%%%%%%%%%%%%%%%%%%%%
% %%%%%%%%%%%%%%%%%%%%%%%%%%%%%%%%%%%%%%%%%%%%%%%%%%%%%%%%%%%%%%%%%%%%%%%%%%%%%%
% \section{Sample}
%\iffalse
%<*samplemain>
%\fi
%
% The following presents a sample document
% with two chapters, two parts, a title page,
% a compile flag as well as three forwarding files to set the flag.
% It consists of eight |.tex| files:
% \begin{center}
% \begin{tabular}{ll}
% |cdocsamp.tex|&main file\\
% |cdocsch1.tex|&include file for chapter 1\\
% |cdocsch2.tex|&include file for chapter 2\\
% |cdocspt3.tex|&include file for part 3\\
% |cdocspt4.tex|&include file for part 4\\
% |cdocsdrf.tex|&forwarding file for main file in draft mode\\
% |cdocsfi1.tex|&forwarding file for final version of chapter 1\\
% |cdocsfi2.tex|&forwarding file for final version of chapter 2\\
% \end{tabular}
% \end{center}
% Each of the eight files can be compiled directly by the \LaTeX{} compiler.
%
% %%%%%%%%%%%%%%%%%%%%%%%%%%%%%%%%%%%%%%
% \paragraph{Main File.}
%
% The main file is called |cdocsamp.tex|.
%
% Load the \textsf{childdoc} definitions and
% declare the filename for the main document:
%    \begin{macrocode}
\input{childdoc.def}
\childdocmain{}
%    \end{macrocode}

% Optional override for |\version| flag:
%    \begin{macrocode}
%%\ifchilddoc\else\providecommand{\version}{draft}\fi
%    \end{macrocode}

% Define the default values for the |\version| flag
% (|final| for the main file and |draft| for childs):
%    \begin{macrocode}
\ifchilddoc
\providecommand{\version}{draft}
\else
\providecommand{\version}{final}
\fi
%    \end{macrocode}

% Load the standard document class:
%    \begin{macrocode}
\documentclass[12pt]{article}
%    \end{macrocode}

% Start the document body:
%    \begin{macrocode}
\begin{document}
%    \end{macrocode}

% Declare a title page.
% Print title, part of document being processed and version flag:
%    \begin{macrocode}
\addtocounter{page}{-1}
\begin{center}
{\LARGE\bfseries{}childdoc example\par}
\vspace{1cm}
\ifchilddoc
\ifchilddocmanual part\else chapter\fi:
`\childdocname' of `\childdocjob'\par
\else
main document: `\childdocjob'\par
\fi
version: \version\par
\end{center}
\newpage
%    \end{macrocode}

% Manually include selected file,
% otherwise process as usual:
%    \begin{macrocode}
\ifchilddocmanual
\section*{part `\childdocname'}
\input{\childdocname}
\else
%    \end{macrocode}

% Include the two chapters:
%    \begin{macrocode}
\include{cdocsch1}
\include{cdocsch2}
%    \end{macrocode}

% Include the two parts unless only chapters should be displayed:
%    \begin{macrocode}
\ifchilddoc\else
\section{part three}
\input{cdocspt3}
\section{part four}
\input{cdocspt4}
\fi
%    \end{macrocode}

% Process as usual until here:
%    \begin{macrocode}
\fi
%    \end{macrocode}

% End of document body:
%    \begin{macrocode}
\end{document}
%    \end{macrocode}
%\iffalse
%</samplemain>
%\fi
%
% %%%%%%%%%%%%%%%%%%%%%%%%%%%%%%%%%%%%%%
% \paragraph{Chapter Include Files.}
%
% The include files are called |cdocsch1.tex| and |cdocsch2.tex|.
%
%\iffalse
%<*samplechap1|samplechap2>
%\fi

% Optional override for |\version| flag:
%    \begin{macrocode}
%%\providecommand{\version}{final}
%    \end{macrocode}

% Include the main document:
%    \begin{macrocode}
\input{childdoc.def}
\childdocof{cdocsamp}
%    \end{macrocode}

%\iffalse
%</samplechap1|samplechap2>
%\fi
%
%\iffalse
%<*samplechap1>
%\fi
% Some text for chapter 1:
%    \begin{macrocode}
\section{one}
some text in chapter one
%    \end{macrocode}

%\iffalse
%</samplechap1>
%\fi
% Some text for chapter 2:
%\iffalse
%<*samplechap2>
%\fi
%    \begin{macrocode}
\section{two}
more text in chapter two
%    \end{macrocode}

%\iffalse
%</samplechap2>
%\fi
%
% %%%%%%%%%%%%%%%%%%%%%%%%%%%%%%%%%%%%%%
% \paragraph{Part Include Files.}
%
% The include files are called |cdocspt3.tex| and |cdocspt4.tex|.
%
%\iffalse
%<*samplepart3|samplepart4>
%\fi

% Optional override for |\version| flag:
%    \begin{macrocode}
%%\providecommand{\version}{final}
%    \end{macrocode}

% Include the main document:
%    \begin{macrocode}
\input{childdoc.def}
\childdocby{cdocsamp}
%    \end{macrocode}

%\iffalse
%</samplepart3|samplepart4>
%\fi
%
%\iffalse
%<*samplepart3>
%\fi
% Some text for part 3:
%    \begin{macrocode}
some text in part three
%    \end{macrocode}

%\iffalse
%</samplepart3>
%\fi
% Some text for part 4:
%\iffalse
%<*samplepart4>
%\fi
%    \begin{macrocode}
more text in part four
%    \end{macrocode}

%\iffalse
%</samplepart4>
%\fi
%
% %%%%%%%%%%%%%%%%%%%%%%%%%%%%%%%%%%%%%%
% \paragraph{Forwarding for a Complete Draft.}
%
% The following forwarding file |cdocsdrf.tex|
% compiles the main document in draft mode:
%\iffalse
%<*sampledraft>
%\fi
%    \begin{macrocode}
\def\version{draft}
\input{childdoc.def}
\childdocforward{cdocsamp}
%    \end{macrocode}

%\iffalse
%</sampledraft>
%\fi
%
% %%%%%%%%%%%%%%%%%%%%%%%%%%%%%%%%%%%%%%
% \paragraph{Forwarding for Final Version of the Chapters.}
%
% The following forwarding files |cdocsfn1.tex| and |cdocsfn2.tex|
% (with identical content)
% compile the final versions of the child documents
% |cdocsch1.tex| and |cdocsch2.tex|, respectively:
%\iffalse
%<*samplefinal>
%\fi
%    \begin{macrocode}
\def\version{final}
\input{childdoc.def}
\childdocforwardprefix[cdocsamp]{cdocsfn}{cdocsch}
%    \end{macrocode}

%\iffalse
%</samplefinal>
%\fi
%
% %%%%%%%%%%%%%%%%%%%%%%%%%%%%%%%%%%%%%%
% \paragraph{Command Line Processing.}
%
% The following three command lines generate the output files
% |cdocscld|, |cdocscl1| and |cdocscl2|
% which should be identical to
% |cdocsdrf|, |cdocsch1| and |cdocsfn2|, respectively:
% \begin{center}
% \begin{tabular}{l}
% |latex -jobname cdocscld \|\\
% |  "\def\version{draft}\input{childdoc.def}\childdocforward{cdocsamp}"|\\
% |latex -jobname cdocscl1 \|\\
% |  "\input{childdoc.def}\childdocforward[cdocsamp]{cdocsch1}"|\\
% |latex -jobname cdocscl2 \|\\
% |  "\def\version{final}\input{childdoc.def}\childdocforward{cdocsch2}"|
% \end{tabular}
% \end{center}
% Note that the trailing backslash on each first line
% merely continues the input to the second line
% (for convenient cut ant paste).
% Furthermore, the command |latex| can be replaced by any
% of its alternative versions such as |pdflatex|.
%
% %%%%%%%%%%%%%%%%%%%%%%%%%%%%%%%%%%%%%%%%%%%%%%%%%%%%%%%%%%%%%%%%%%%%%%%%%%%%%%
% %%%%%%%%%%%%%%%%%%%%%%%%%%%%%%%%%%%%%%%%%%%%%%%%%%%%%%%%%%%%%%%%%%%%%%%%%%%%%%
% \section{Implementation}
%\iffalse
%<*package>
%\fi
%
% This section describes the definitions file |childdoc.def|.

% The definitions cannot be loaded using |\usepackage| or |\RequirePackage|
% which has a mechanism to prevent loading a style file more than once.
% When loading the definitions by means of |\input|
% multiple instances have to be prevented manually:
%\iffalse
%This code needs to be before the `\ProvidesFile' directive
%which is defined at the beginning of this file.
%Therefore it is also placed there and commented out here.
%</package>
%<*discard>
%\fi
%    \begin{macrocode}
\ifdefined\childdocmain\endinput\fi
%    \end{macrocode}
%\iffalse
%</discard>
%<*package>
%\fi
%
% \macro{\ifchilddoc}
% \macro{\ifchilddocmanual}
% The conditional |\ifchilddoc| tells whether a
% child (true) or main (false) document is being compiled.
% The conditional |\ifchilddocmanual| tells whether
% the |\includeonly| mechanism is used (false) or
% the selection of child files must be performed manually (true).
% The definitions initialise to false:
%    \begin{macrocode}
\newif\ifchilddoc
\newif\ifchilddocmanual
%    \end{macrocode}

% \macro{\childdocname}
% \macro{\childdocjob}
% The macro |\childdocname| stores the name of the main document
% to be compiled. The macro |\childdocjob| stores the name of
% the document on which the \LaTeX{} compiler was originally invoked.
% The content of |\jobname| cannot be compared
% to filenames specified in the source due to different catcodes.
% The following code rescans |\jobname|, stores the result
% in |\childdocname| and saves a copy in |\childdocjob|:
%    \begin{macrocode}
\edef\childdocname{\scantokens\expandafter{\jobname\noexpand}}
\let\childdocjob\childdocname
%    \end{macrocode}

% \macro{\childdocdisable}
% The macro |\childdocdisable| prevents the main file
% from being processed more than once.
% At this stage, the main document command |\childdocmain|
% is assumed to be called once again where it should do nothing.
% Any subsequent call to it should prevent
% a secondary processing of the main document
% It overwrites the forwarding commands
% |\childdocof| and |\childdocforward|
% with empty macros to prevent further inclusions of the main document:
%    \begin{macrocode}
\newcommand{\childdocdisable}
{
  \renewcommand{\childdocmain}[1]{\renewcommand{\childdocmain}[1]{\endinput}}
  \renewcommand{\childdocof}[1]{}
  \renewcommand{\childdocby}[2][]{}
  \renewcommand{\childdocforward}[2][]{}
  \renewcommand{\childdocdisable}{}
}
%    \end{macrocode}

% \macro{\childdocmain}
% The macro |\childdocmain| is to be called at the top of the main file
% with nothing or the main filename (without extension) as argument.
% First, it breaks loops.
% If the argument is not empty and does not match |\childdocname|
% (which is set by the first inclusion of |childdoc.def|),
% |\ifchilddoc| is set to true, |\includeonly| is applied to the child file
% and |\jobname| is set to the main file
% (for proper handling of |.aux| files):
%    \begin{macrocode}
\newcommand{\childdocmain}[1]
{
  \childdocdisable\childdocmain{}
  \if?#1?\else
    \begingroup
      \def\childdoctmp{#1}
      \ifx\childdoctmp\childdocname
        \def\childdoctmp{}
      \else
        \def\childdoctmp
        {
          \childdoctrue
          \includeonly{\childdocname}
          \def\childdocjob{#1}
          \def\jobname{#1}
        }
      \fi
      \expandafter
    \endgroup
    \childdoctmp
  \fi
}
%    \end{macrocode}

% \macro{\childdocof}
% The command |\childdocof| redirects
% compilation to the main file |#1|.
%    \begin{macrocode}
\newcommand{\childdocof}[1]
{
  \childdocdisable
  \childdoctrue
  \includeonly{\childdocname}
  \def\jobname{#1}
  \def\childdocjob{#1}
  \input{#1}
}
%    \end{macrocode}

% \macro{\childdocby}
% The command |\childdocby| ....
%    \begin{macrocode}
\newcommand{\childdocby}[2][]
{
  \childdocdisable
  \childdoctrue
  \childdocmanualtrue
  \if?#1?\else
    \def\jobname{#2}
  \fi
  \def\childdocjob{#2}
  \input{#2}
  \endinput
}
%    \end{macrocode}

% \macro{\childdocforward}
% The command |\childdocforward| redirects
% compilation to the main file or
% (if the optional argument is given) a child file.
% Parameters are set as if the main file
% or a child file starting with |\childdocof| was compiled.
% Then compilation is handed over to the main file:
%    \begin{macrocode}
\newcommand{\childdocforward}[2][]
{
  \begingroup
    \if?#1?
      \def\childdoctmp
      {
        \def\childdocname{#2}
        \def\childdocjob{#2}
        \def\jobname{#2}
        \input{#2}
        \endinput
      }
    \else
      \def\childdoctmp
      {
        \childdocdisable
        \def\childdocname{#2}
        \childdoctrue
        \includeonly{#2}
        \def\childdocjob{#1}
        \def\jobname{#1}
        \input{#1}
        \endinput
      }
    \fi
    \expandafter
  \endgroup
  \childdoctmp
}
%    \end{macrocode}

% \macro{\childdocforwardprefix}
% The command |\childdocforwardprefix| redirects
% compilation to the main or a child file by means of a pattern.
% The prefix |#1| in the current filename is replaced by |#2|
% and the suffix of the current filename is kept
% (it is assumed that the filename does not contain the substring `|~~~|'
% which is used as a delimiter).
% Compilation is handed over to the new file by |\childdocforward|:
%    \begin{macrocode}
\newcommand{\childdocforwardprefix}[3][]
{
  \begingroup
    \def\childdocextract #2##1~~~{\def\childdoctmp{\childdocforward[#1]{#3##1}}}
    \expandafter\childdocextract\childdocname~~~
    \expandafter
  \endgroup
  \childdoctmp
}
%    \end{macrocode}

% \macro{\childdoc}
% The deprecated macro |\childdoc| is a legacy version of |\childdocmain|:
%    \begin{macrocode}
\newcommand{\childdoc}{\childdocmain}
%    \end{macrocode}

% \macro{\childdocredirect}
% The deprecated macro |\childdocredirect| is a legacy version
% of |\childdocforward| and |\childdocforwardprefix|:
%    \begin{macrocode}
\newcommand{\childdocredirect}[2][]
{
  \begingroup
    \if?#1?
      \def\childdoctmp{\childdocforward{#2}}
    \else
      \def\childdoctmp{\childdocforwardprefix{#1}{#2}}
    \fi
    \expandafter
  \endgroup
  \childdoctmp
}
%    \end{macrocode}

%\iffalse
%</package>
%\fi
%
\endinput

\childdocforwardprefix[cdocsamp]{cdocsfn}{cdocsch}
%    \end{macrocode}

%\iffalse
%</samplefinal>
%\fi
%
% %%%%%%%%%%%%%%%%%%%%%%%%%%%%%%%%%%%%%%
% \paragraph{Command Line Processing.}
%
% The following three command lines generate the output files
% |cdocscld|, |cdocscl1| and |cdocscl2|
% which should be identical to
% |cdocsdrf|, |cdocsch1| and |cdocsfn2|, respectively:
% \begin{center}
% \begin{tabular}{l}
% |latex -jobname cdocscld \|\\
% |  "\def\version{draft}% \iffalse
%
% childdoc.dtx Copyright (C) 2017-2018 Niklas Beisert
%
% This work may be distributed and/or modified under the
% conditions of the LaTeX Project Public License, either version 1.3
% of this license or (at your option) any later version.
% The latest version of this license is in
%   http://www.latex-project.org/lppl.txt
% and version 1.3 or later is part of all distributions of LaTeX
% version 2005/12/01 or later.
%
% This work has the LPPL maintenance status `maintained'.
%
% The Current Maintainer of this work is Niklas Beisert.
%
% This work consists of the files childdoc.dtx and childdoc.ins
% and the derived files childdoc.def and cdocsamp.tex with
% cdocsch1.tex, cdocsch2.tex, cdocsdrf.tex, cdocsfn1.tex, cdocsfn2.tex.
%
%<package>\ifdefined\childdocmain\endinput\fi
%<package>\ProvidesFile{childdoc.def}[2018/12/30 v2.0 child document driver]
%<samplemain>\ProvidesFile{cdocsamp.tex}[2018/12/30 v2.0 sample for childdoc]
%<*driver>
%\ProvidesFile{childdoc.drv}[2018/12/30 v2.0 childdoc reference manual file]
\PassOptionsToClass{10pt,a4paper}{article}
\documentclass{ltxdoc}

\usepackage[margin=35mm]{geometry}
\usepackage{hyperref}
\usepackage{hyperxmp}
\usepackage[usenames]{color}

\hypersetup{colorlinks=true}
\hypersetup{pdfstartview=FitH}
\hypersetup{pdfpagemode=UseNone}
\hypersetup{pdfsource={}}
\hypersetup{pdflang={en-UK}}
\hypersetup{pdfcopyright={Copyright 2017-2018 Niklas Beisert.
  This work may be distributed and/or modified under the
  conditions of the LaTeX Project Public License, either version 1.3
  of this license or (at your option) any later version.}}
\hypersetup{pdflicenseurl={http://www.latex-project.org/lppl.txt}}
\hypersetup{pdfcontactaddress={ETH Zurich, ITP, HIT K,
  Wolfgang-Pauli-Strasse 27}}
\hypersetup{pdfcontactpostcode={8093}}
\hypersetup{pdfcontactcity={Zurich}}
\hypersetup{pdfcontactcountry={Switzerland}}
\hypersetup{pdfcontactemail={nbeisert@itp.phys.ethz.ch}}
\hypersetup{pdfcontacturl={http://people.phys.ethz.ch/\xmptilde nbeisert/}}

\newcommand{\secref}[1]{\hyperref[#1]{section \ref*{#1}}}

\parskip1ex
\parindent0pt
\let\olditemize\itemize
\def\itemize{\olditemize\parskip0pt}

\begin{document}

\title{The \textsf{childdoc} Package}
\hypersetup{pdftitle={The childdoc Package}}
\author{Niklas Beisert\\[2ex]
  Institut f\"ur Theoretische Physik\\
  Eidgen\"ossische Technische Hochschule Z\"urich\\
  Wolfgang-Pauli-Strasse 27, 8093 Z\"urich, Switzerland\\[1ex]
  \href{mailto:nbeisert@itp.phys.ethz.ch}
  {\texttt{nbeisert@itp.phys.ethz.ch}}}
\hypersetup{pdfauthor={Niklas Beisert}}
\hypersetup{pdfsubject={Manual for the LaTeX2e Package childdoc}}
\date{30 December 2018, \textsf{v2.0}}
\maketitle

\begin{abstract}\noindent
\textsf{childdoc} is a \LaTeXe{} package
that enables the direct compilation
of document sections included by |\include|
to individual files.
\end{abstract}

\begingroup
\parskip0ex
\tableofcontents
\endgroup

%%%%%%%%%%%%%%%%%%%%%%%%%%%%%%%%%%%%%%%%%%%%%%%%%%%%%%%%%%%%%%%%%%%%%%%%%%%%%%%%
%%%%%%%%%%%%%%%%%%%%%%%%%%%%%%%%%%%%%%%%%%%%%%%%%%%%%%%%%%%%%%%%%%%%%%%%%%%%%%%%
\section{Introduction}

\LaTeX{} provides a mechanism to structure a large document (such as a book)
into a main file and several child files (containing the chapters)
using the |\include| command.
This mechanism is beneficial for documents
which span hundreds of pages in order to
make the source file(s) more manageable.
Moreover, compilation can be restricted to
selected child files by means of the |\includeonly| command.
The latter feature can be used to reduce the compilation time while editing
(this was significantly more useful in the earlier days of \LaTeX{})
or to generate a smaller document which is easier to navigate.
Another application of |\includeonly| is to generate
documents consisting of selected parts of the complete document.

However, there are a few drawbacks of the plain |\include| mechanism:
\begin{itemize}
\item
The child files cannot be compiled on their own,
they can only be compiled via the main file.
A naive editing environment
(such as a text editor with an option
to have the current file processed by \LaTeX)
may require one to switch to the main file before compiling;
attempting to compile the child file produces errors.
\item
The main file must be modified (each time)
to adjust the |\includeonly| command
to the present needs. This easily leaves the main file in a messy state.
\item
The generated document will always carry the filename
of the main document. This is inconvenient if
several child files are to be compiled and
to be kept for distribution.
\end{itemize}

The present package provides a simple interface
to make child files individually compilable by \LaTeX{}.
Compiling a child file then has the same effect as compiling
the main file with an |\includeonly| command
to select the appropriate child.
Moreover the generated document will carry the name of the child
rather than the main file.
This resolves all three above issues.

This feature is meant to make the editing of books,
thesis documents and lecture notes somewhat more convenient.
However, the package can also be used efficiently for
composing a series of documents (such as exercise sheets)
which are typically distributed individually.
It then assists the author in generating the individual documents
(potentially in different versions)
as well as a document containing the collected series.
Another application is in developing style files
or other kinds of included material
where compilation of the style file could redirect
to a sample or test file.

%%%%%%%%%%%%%%%%%%%%%%%%%%%%%%%%%%%%%%%%%%%%%%%%%%%%%%%%%%%%%%%%%%%%%%%%%%%%%%%%
%%%%%%%%%%%%%%%%%%%%%%%%%%%%%%%%%%%%%%%%%%%%%%%%%%%%%%%%%%%%%%%%%%%%%%%%%%%%%%%%
\section{Usage}

First of all, the package \textsf{childdoc} is \emph{not} a standard
\LaTeXe{} |.sty| style file! Therefore it needs to be invoked in
a non-standard way.

%%%%%%%%%%%%%%%%%%%%%%%%%%%%%%%%%%%%%%%%%%%%%%%%%%%%%%%%%%%%%%%%%%%%%%%%%%%%%%%%
\subsection{Included Files}
\label{sec:include}

%%%%%%%%%%%%%%%%%%%%%%%%%%%%%%%%%%%%%%%%
\DescribeMacro{\childdocmain}
To use the package, add the commands
\begin{center}
\begin{tabular}{l}
|\input{childdoc.def}|\\
|\childdocmain{}|\\
\end{tabular}
\end{center}
at the very top of the main \LaTeX{} file,
in particular \emph{before} the |\documentclass| statement!
The argument of |\childdocmain| should be left empty
(but it must be present).

%%%%%%%%%%%%%%%%%%%%%%%%%%%%%%%%%%%%%%%%
\DescribeMacro{\childdocof}
Furthermore, add the commands
\begin{center}
\begin{tabular}{l}
|\input{childdoc.def}|\\
|\childdocof{|\textit{main}|}|\\
\end{tabular}
\end{center}
at the top of every child file \textit{child}
which is included by |\include{|\textit{child}|}|
from within the main file
(or at least for those files to be compiled individually).
The argument \textit{main} must be the filename of the main file.

There are a couple of
considerations in setting up the main and child documents:

%%%%%%%%%%%%%%%%%%%%%%%%%%%%%%%%%%%%%%%%
\paragraph{Restrictions.}

Please note the following restrictions:
\begin{itemize}
\item
|\childdocmain| must be called with one argument \textit{main}
to ensure compatibility with earlier version of the package.
It must either be empty (|\childdocmain{}|)
or precisely match the filename of the main file in which it is specified.
See \secref{sec:detection} for further information.
\item
The filename \textit{main} must be specified without the |.tex| extension.
\item
The filename \textit{main} is case sensitive
(even in case-insensitive file systems)
due to internal string comparison.
\item
The argument \textit{main} should be fully expanded, it cannot be a macro.
\item
Subdirectories and special characters should be avoided in filenames.
\item
The command |\childdocmain{|\textit{main}|}| must be followed by a whitespace.
It should not be followed immediately by another command
or by a comment mark `|%|'.
This is because the \TeX{} parser reads the token immediately following
the argument of |\childdocmain| and puts it
at the beginning of every child section;
however, a white\-space is ignored.
\end{itemize}

%%%%%%%%%%%%%%%%%%%%%%%%%%%%%%%%%%%%%%%%
\paragraph{Content of Main File.}

It is advisable to place all content in the child files included by |\include|.
Any output contained in the main file will appear in all child documents
unless suppressed manually;
it cannot be suppressed automatically by the |\includeonly| directive
and thus should normally be avoided.
A method to include some content in the main file
by means of conditional processing is described in \secref{sec:conditional}.

%%%%%%%%%%%%%%%%%%%%%%%%%%%%%%%%%%%%%%%%
\paragraph{Page Numbering.}

When only a part of the document is compiled,
the appropriate numbering of pages
(as well as other status parameters)
is determined from the |.aux| files.
The latter contain information from previous passes.
However this information needs to propagate through
all intermediate child documents.
Therefore the page numbering in child documents may well
be inconsistent until the complete document is compiled at least once.

A useful (if unconventional) way to always ensure a consistent
page numbering is to restart the numbering in each child document
and denote the pages by `\textit{child}|.|\textit{page}'
where \textit{child} represents the chapter/section number of the child file.
This can be achieved by the command
|\numberwithin{page}{|\textit{child}|}|
of the \textsf{amsmath} package
where \textit{child} can be |chapter| or |section|
depending on the chosen structuring.
Alternatively, one can modify the macro |\thepage| appropriately
and reset the counter |page| at the start of each child file.

%%%%%%%%%%%%%%%%%%%%%%%%%%%%%%%%%%%%%%%%%%%%%%%%%%%%%%%%%%%%%%%%%%%%%%%%%%%%%%%%
\subsection{Conditional Processing}
\label{sec:conditional}

The package provides a mechanism to compile different versions
of a document. To customise the versions further some conditional processing
can come in handy to distinguish which version is being compiled.
The package provides two macros to describe the compilation context:

%%%%%%%%%%%%%%%%%%%%%%%%%%%%%%%%%%%%%%%%
\DescribeMacro{\ifchilddoc}
The conditional |\ifchilddoc| distinguishes between the compilation of
child documents and the main document:
%
\begin{center}
|\ifchilddoc |\textit{child-code}| |[|\||else |\textit{main-code}]| \||fi|
\end{center}

%%%%%%%%%%%%%%%%%%%%%%%%%%%%%%%%%%%%%%%%
\DescribeMacro{\childdocname}
\DescribeMacro{\childdocjob}
The macro |\childdocname| contains the filename (without extension)
of the main or child file being processed.
Note that |\childdocjob| will always contain the name of the main file.

%%%%%%%%%%%%%%%%%%%%%%%%%%%%%%%%%%%%%%%%
\paragraph{Title Page.}

Conditional processing can be used to include a title or banner page
in the main document when proper precautions are taken.
Importantly, the code in the main file should ensure that the page counter
(as well as other status parameters which are stored in the |.aux| files)
takes the same value after the conditional processing.
Otherwise the page numbers may take divergent values
depending on which part is compiled.

For example, a title page could be declared by:
%
\begin{center}
\begin{tabular}{l}
|\ifchilddoc\||else|\\
|\addtocounter{page}{-1}|\\
\textit{code for title page}\\
|\newpage|\\
|\||fi|
\end{tabular}
\end{center}
%
A banner page for the child documents can be generated by:
%
\begin{center}
\begin{tabular}{l}
|\ifchilddoc|\\
|\addtocounter{page}{-1}|\\
\textit{code for banner page}\\
|\newpage|\\
|\||fi|
\end{tabular}
\end{center}
%
Here one could write a message such as:
\begin{center}
|This is the part \childdocname{} of \childdocjob{}.|
\end{center}

%%%%%%%%%%%%%%%%%%%%%%%%%%%%%%%%%%%%%%%%%%%%%%%%%%%%%%%%%%%%%%%%%%%%%%%%%%%%%%%%
\subsection{Flags}
\label{sec:flags}

The package makes it easy to generate different versions
of the main or child documents.
To this end compilation flags can be defined
and assigned different default values.
They will be particularly useful in conjunction
with the forwarding mechanism described in \secref{sec:forward}.

For example, it may be useful to have a flag |\version|
which can be set to |draft| or |final|.
The document source will contain some conditional code
depending on the value of |\version|.
Suppose further, the flag should default to |final| for the main file
and to |draft| for child files
which is a natural assignment for editing the document.
This is achieved by placing the following code
in the preamble of the main document
(below the |\childdocmain| directive):
%
\begin{center}
\begin{tabular}{l}
|\ifchilddoc|\\
|\providecommand{\version}{draft}|\\
|\||else|\\
|\providecommand{\version}{final}|\\
|\||fi|
\end{tabular}
\end{center}
%
The definition by |\providecommand| makes sure
that previous definitions are not overwritten.
Further statements |\providecommand{\version}{...}|
can thus be added before the above code to override it.

For the main file, one might add a line
(between |\childdocmain| and the above block)
%
\begin{center}
|%\ifchilddoc\||else\providecommand{\version}{draft}\||fi|
\end{center}
%
which can be uncommented to produce a draft version.
Likewise one can add a line to the very top of a child file
(above the |\childdocof{|\textit{main}|}| directive)
%
\begin{center}
|%\providecommand{\version}{final}|
\end{center}
%
which can be uncommented to produce the final version of this child document.

%%%%%%%%%%%%%%%%%%%%%%%%%%%%%%%%%%%%%%%%%%%%%%%%%%%%%%%%%%%%%%%%%%%%%%%%%%%%%%%%
\subsection{Forwarding}
\label{sec:forward}

Different versions of the main or child documents
using compilation flags as described in \secref{sec:flags}
can be (permanently) stored in different files
for convenient compilation, viewing and distribution.
To this end, the package defines a command
to pass on compilation to a different file:

%%%%%%%%%%%%%%%%%%%%%%%%%%%%%%%%%%%%%%%%
\DescribeMacro{\childdocforward}
The command |\childdocforward| redirects processing to
another source file:
%
\begin{center}
\begin{tabular}{l}
|\input{childdoc.def}|\\
|\childdocforward[|\textit{main}|]{|\textit{dest}|}|\\
\end{tabular}
\end{center}
%
The argument \textit{dest} is the destination file
(without extension).
It should be the main file or one of the child files.
Note that further \textsf{childdoc} directives
such as |\childdocof| and |\childdocforward|
in the indicated file will be processed in this form.
The optional argument \textit{main}
passes on directly to the main file \textit{main}
while pretending to compile the child \textit{dest}.
This form behaves as if \textit{dest}
issues |\childdocof{|\textit{main}|}| right away,
and no further \textsf{childdoc} directives will be processed.

%%%%%%%%%%%%%%%%%%%%%%%%%%%%%%%%%%%%%%%%
\DescribeMacro{\...prefix}
In the alternative form |\childdocforwardprefix|,
%
\begin{center}
\begin{tabular}{l}
|\input{childdoc.def}|\\
|\childdocforwardprefix[|\textit{main}|]{|\textit{prefix}|}{|\textit{dest}|}|
\end{tabular}
\end{center}
%
the destination file is determined by a pattern
depending on the current file:
To make this work, the current file must be called
`{\textit{prefix}\hspace{0.2em}\textit{suffix}}'
with \textit{prefix} matching precisely the argument.
Processing is then passed on to the file
`{\textit{dest}\hspace{0.2em}\textit{suffix}}'.
Surely, the same effect is achieved by
directly specifying the
argument `{\textit{dest}\hspace{0.2em}\textit{suffix}}'
in the first form.
However, that requires to set up a different file
for each child. With the alternative form of the command
all these files can have exactly the same content
which simplifies setting them up and maintaining them.

For example, the following file |draft.tex|
with a compilation flag |\version| as described in \secref{sec:flags}
compiles the main document as a draft:
%
\begin{center}
\begin{tabular}{l}
|\def\version{draft}|\\
|\input{childdoc.def}|\\
|\childdocforward{|\textit{main}|}|
\end{tabular}
\end{center}
%
Likewise, the following files |final|\textit{nn}|.tex|
compile the final version of the child document
|child|\textit{nn}|.tex|:
%
\begin{center}
\begin{tabular}{l}
|\def\version{final}|\\
|\input{childdoc.def}|\\
|\childdocforwardprefix{final}{child}|
\end{tabular}
\end{center}
%

Note that when several versions of a main file and/or of each child file
are to be generated, it may be convenient to set up a |Makefile| or
shell script to automatise the process.

%%%%%%%%%%%%%%%%%%%%%%%%%%%%%%%%%%%%%%%%%%%%%%%%%%%%%%%%%%%%%%%%%%%%%%%%%%%%%%%%
\subsection{Command Line Processing}
\label{sec:commandline}

The effect of redirection files can also be achieved by invoking
the \LaTeX{} compiler with a more elaborate command line.
Most conveniently this should be done as part
of a shell script or a |Makefile|.

When using \textsf{childdoc} in the main file, the following
command lines effectively perform a redirection
(note that depending on the shell being used,
backslashes may have to be doubled: `|\|' $\to$ `|\\|'):
%
\begin{center}
|... -jobname "|\textit{target}|" |\\|"|[\textit{flags}]%
|\input{childdoc.def}\childdocforward[|\textit{main}|]{|\textit{dest}|}"|
\end{center}
%
Here \textit{target} is the name of the output file,
\textit{main} is the name of the main file
and \textit{dest} is the name of the main or child file to be processed
(all filenames without extensions).
The optional argument \textit{main} can be omitted
if \textit{main} matches \textit{dest}.
Optionally, compilation \textit{flags} can be defined via |\def| commands.
This command line makes the \TeX{} engine believe
it is compiling the file \textit{target}
whose content is specified as the latter parameter.
The provided code then forwards the processing to
\textit{main} or \textit{dest} as described in \secref{sec:forward}.

%%%%%%%%%%%%%%%%%%%%%%%%%%%%%%%%%%%%%%%%%%%%%%%%%%%%%%%%%%%%%%%%%%%%%%%%%%%%%%%%
\subsection{Include by Input}
\label{sec:input}

Including child documents by |\include| has some restrictions by design.
Most notably, the content of a child document always occupies
its own set of pages; pages cannot be shared between child documents.
Usually, this behaviour makes perfect sense
because each child document contain an essential part of the document.
However, in some situations it may be desirable to compose
a document from a collection of parts
without having mandatory page breaks between then.
For this case, the package
provides a mechanism to include parts
by |\input| which can also be processed individually.
However, by construction this mechanism
requires manual handling of the content to be output.

%%%%%%%%%%%%%%%%%%%%%%%%%%%%%%%%%%%%%%%%
\DescribeMacro{\ifchilddocmanual}
The main file should be prepared as usual, see \secref{sec:include}.
However, the document body must make a distinction
between processing of an individual part and of the main document, e.g.:
%
\begin{center}
\begin{tabular}{l}
|\ifchilddocmanual|\\
|\input{\childdocname}|\\
|\||else|\\
\textit{document body with }|\input{|\textit{part}|}|\\
|\||fi|
\end{tabular}
\end{center}
%
The conditional |\ifchilddocmanual| is true whenever
a part to be included by |\input| is being compiled,
and the name of the part is stored in |\childdocname|.

%%%%%%%%%%%%%%%%%%%%%%%%%%%%%%%%%%%%%%%%
\DescribeMacro{\childdocby}
Each part to be included by |\input| should start with:
%
\begin{center}
\begin{tabular}{l}
|\input{childdoc.def}|\\
|\childdocby{|\textit{main}|}|\\
\end{tabular}
\end{center}
%
The directive |\childdocby| is similar to |\childdocof|
described in \secref{sec:include},
but the subsequent selection of content must be done manually.
To that end, both |\ifchilddoc| and |\ifchilddocmanual|
will be true upon processing of a part,
and the name of the part is stored in |\childdocname|.
Note that |\jobname| will be set to the filename of the current part
so that each part receives an individual |.aux| file
that does not interfere with the |.aux| file(s) of the main document.
This behaviour can be altered by the alternative form
|\childdocby[*]{|\textit{main}|}| (with a non-empty optional argument)
which uses the |.aux| file of the main document
by setting |\jobname| to \textit{main}.

%%%%%%%%%%%%%%%%%%%%%%%%%%%%%%%%%%%%%%%%%%%%%%%%%%%%%%%%%%%%%%%%%%%%%%%%%%%%%%%%
\subsection{Driver Development}
\label{sec:driver}

The \textsf{childdoc} mechanism can also be use for the development
of definition files such as \LaTeX{} styles or classes.
This case differs from the above setup with multiple parts
included by |\include| in that no |\includeonly| should be invoked.
This can be achieved by starting the include file
(before |\ProvidesPackage|) with:
%
\begin{center}
\begin{tabular}{l}
|\input{childdoc.def}|\\
|\childdocforward{|\textit{main}|}|\\
\end{tabular}
\end{center}
%
or alternatively with:
%
\begin{center}
\begin{tabular}{l}
|\input{childdoc.def}|\\
|\childdocby{|\textit{main}|}|\\
\end{tabular}
\end{center}
%
Both forms have slightly different effects as described above.
The main file is prepared as usual, see \secref{sec:include}.

%%%%%%%%%%%%%%%%%%%%%%%%%%%%%%%%%%%%%%%%%%%%%%%%%%%%%%%%%%%%%%%%%%%%%%%%%%%%%%%%
\subsection{Legacy Detection}
\label{sec:detection}

The directive |\childdocmain| in the main file can detect
whether the complete document or merely a child is to be compiled
even without using the directive |\childdocof|.
This method is deprecated because it is less robust
and there is no compelling reason to use it;
it is merely provided for backward compatibility
and it may be removed in future versions.

If the detection mechanism is to be used,
it is mandatory to correctly specify
the filename of the main file as the argument of |\childdocmain|:
%
\begin{center}
\begin{tabular}{l}
|\input{childdoc.def}|\\
|\childdocmain{|\textit{main}|}|\\
\end{tabular}
\end{center}
%
If |\jobname| does not match the argument \textit{main} of |\childdocmain|,
it is assumed that |\jobname| points to the child file to be compiled.
When using |\childdocmain| with the main file specified as argument,
it suffices to start a child file
with just |\input{|\textit{main}|}|
without loading of the package and using |\childdocof|.
If instead all processing is done
with the appropriate \textsf{childdoc} directives,
the argument of \textit{main} of |\childdocmain| can be empty.

An alternative version of the command line processing described
in \secref{sec:commandline} using the detection mechanism reads:
%
\begin{center}
|... -jobname "|\textit{target}|" "|[\textit{flags}]%
[|\def\jobname{|\textit{dest}|}|]|\input{|\textit{main}|}"|
\end{center}

%%%%%%%%%%%%%%%%%%%%%%%%%%%%%%%%%%%%%%%%%%%%%%%%%%%%%%%%%%%%%%%%%%%%%%%%%%%%%%%%
\subsection{Manual Code}
\label{sec:manual}

In case one cannot be certain whether the definitions file |childdoc.def|
is installed on the target \TeX{} distribution
and one prefers not to ship it,
it is conceivable to paste a few relevant commands into the sources.

To that end, drop all statements |\input{childdoc.def}|
and perform the replacements as outlined below.
Instead of |\childdocmain{|\textit{main}|}| add the following code
to the top of the main file:
%
\begin{center}
\begin{tabular}{l}
|\||ifdefined\childdocname\endinput\||fi\newif\ifchilddoc|\\
|\edef\childdocname{\scantokens\expandafter{\jobname\noexpand}}|\\
|\def\childdocmain{|\textit{main}|}\||ifx\childdocmain\childdocname\||else|\\
|\childdoctrue\includeonly{\childdocname}\let\jobname\childdocmain\||fi|\\
\end{tabular}
\end{center}
%
Instead of |\childdocof{|\textit{main}|}| just include the main file
at the top of each child file:
%
\begin{center}
|\input{|\textit{main}|}|
\end{center}
%
A simple redirection |\childdocforward{|\textit{dest}|}| is achieved by:
%
\begin{center}
|\def\jobname{|\textit{dest}|}\input{\jobname}|
\end{center}
%
The redirection with prefix
|\childdocforwardprefix[|\textit{prefix}|]{|\textit{dest}|}|
is accomplished by:
%
\begin{center}
\begin{tabular}{l}
|{\edef\jobname{\scantokens\expandafter{\jobname\noexpand}}|\\
|\def\redirectjob |\textit{prefix}|#1~~~{\gdef\jobname{|\textit{dest}|#1}}|\\
|\expandafter\redirectjob\jobname~~~}\input{\jobname}|
\end{tabular}
\end{center}

In an alternative approach,
child documents can be compiled by a specific command line
without additional code or specific definitions:
%
\begin{center}
|... -jobname "|\textit{target}|" "|[\textit{flags}]%
|\includeonly{|\textit{dest}|}\input{|\textit{main}|}"|
\end{center}
%

%%%%%%%%%%%%%%%%%%%%%%%%%%%%%%%%%%%%%%%%%%%%%%%%%%%%%%%%%%%%%%%%%%%%%%%%%%%%%%%%
%%%%%%%%%%%%%%%%%%%%%%%%%%%%%%%%%%%%%%%%%%%%%%%%%%%%%%%%%%%%%%%%%%%%%%%%%%%%%%%%
\section{Information}

%%%%%%%%%%%%%%%%%%%%%%%%%%%%%%%%%%%%%%%%%%%%%%%%%%%%%%%%%%%%%%%%%%%%%%%%%%%%%%%%
\subsection{Copyright}

Copyright \copyright{} 2017--2018 Niklas Beisert

This work may be distributed and/or modified under the
conditions of the \LaTeX{} Project Public License, either version 1.3
of this license or (at your option) any later version.
The latest version of this license is in
  \url{http://www.latex-project.org/lppl.txt}
and version 1.3 or later is part of all distributions of \LaTeX{}
version 2005/12/01 or later.

This work has the LPPL maintenance status `maintained'.

The Current Maintainer of this work is Niklas Beisert.

This work consists of the files |README.txt|, |childdoc.ins| and |childdoc.dtx|
as well as the derived files |childdoc.def|, |cdocsamp.tex|
with |cdocsch1.tex|, |cdocsch2.tex|, |cdocspt3.tex|, |cdocspt4.tex|,
|cdocsdrf.tex|, |cdocsfn1.tex|, |cdocsfn2.tex|
as well as |childdoc.pdf|.

%%%%%%%%%%%%%%%%%%%%%%%%%%%%%%%%%%%%%%%%%%%%%%%%%%%%%%%%%%%%%%%%%%%%%%%%%%%%%%%%
\subsection{Files and Installation}

The package consists of the files:
%
\begin{center}
\begin{tabular}{ll}
    |README.txt|   & readme file \\
    |childdoc.ins| & installation file \\
    |childdoc.dtx| & source file \\
    |childdoc.def| & definition file \\
    |cdocsamp.tex| & sample main file \\
    |cdocsch1.tex| & sample include file \\
    |cdocsch2.tex| & sample include file \\
    |cdocspt3.tex| & sample part file \\
    |cdocspt4.tex| & sample part file \\
    |cdocsdrf.tex| & sample redirection file \\
    |cdocsfn1.tex| & sample redirection file \\
    |cdocsfn2.tex| & sample redirection file \\
    |childdoc.pdf| & manual
\end{tabular}
\end{center}
%
The distribution consists of the files
|README.txt|, |childdoc.ins| and |childdoc.dtx|.
%
\begin{itemize}
\item
Run (pdf)\LaTeX{} on |childdoc.dtx|
to compile the manual |childdoc.pdf| (this file).
\item
Run \LaTeX{} on |childdoc.ins| to create the definitions file |childdoc.def|
and the sample |cdocsamp.tex| with include files
|cdocsch1.tex|, |cdocsch2.tex|, |cdocspt3.tex|, |cdocspt4.tex|,
|cdocsdrf.tex|, |cdocsfn1.tex|, |cdocsfn2.tex|.
Then copy the file |childdoc.def| to an appropriate directory of your \LaTeX{}
distribution, e.g.\ \textit{texmf-root}|/tex/latex/childdoc|.
\end{itemize}

%%%%%%%%%%%%%%%%%%%%%%%%%%%%%%%%%%%%%%%%%%%%%%%%%%%%%%%%%%%%%%%%%%%%%%%%%%%%%%%%
\subsection{Related CTAN Packages}

There are several other packages which offer a similar functionality:
%
\begin{itemize}
\item
The packages
\href{http://ctan.org/pkg/docmute}{\textsf{docmute}},
\href{http://ctan.org/pkg/includex}{\textsf{includex}} and
\href{http://ctan.org/pkg/standalone}{\textsf{standalone}}
provide commands to include only the document body of
a child file thus allowing both files to be compiled individually.
\item
The packages \href{http://ctan.org/pkg/subdocs}{\textsf{subdocs}}
and \href{http://ctan.org/pkg/subfiles}{\textsf{subfiles}}
provide structures in which the main and child documents can be
encapsulated and allowing them to be compiled individually.
The inclusion mechanism is different from the conventional |\include|.
\item
The package \href{http://ctan.org/pkg/combine}{\textsf{combine}}
is an elaborate solution to combine several documents into one.
\end{itemize}
%
See also the CTAN topic \href{http://ctan.org/topic/subdocs}{\textsf{subdocs}}
for further related packages.
The present package differs from the above solutions in that
a document structure constructed with the conventional |\include| mechanism
just needs two extra commands at the top of every file
such that all constituent files can be compiled individually.

%%%%%%%%%%%%%%%%%%%%%%%%%%%%%%%%%%%%%%%%%%%%%%%%%%%%%%%%%%%%%%%%%%%%%%%%%%%%%%%%
%\subsection{Feature Suggestions}
%
%The following is a list of features which may be useful for future
%versions of this package:
%%
%\begin{itemize}
%\item
%\ldots
%\end{itemize}

%%%%%%%%%%%%%%%%%%%%%%%%%%%%%%%%%%%%%%%%%%%%%%%%%%%%%%%%%%%%%%%%%%%%%%%%%%%%%%%%
\subsection{Revision History}

%%%%%%%%%%%%%%%%%%%%%%%%%%%%%%%%%%%%%%%%
\paragraph{v2.0:} 2018/12/30

\begin{itemize}
\item
immediate forward processing
\item
added |\childdocby| mechanism
\item
manual restructured
\end{itemize}

%%%%%%%%%%%%%%%%%%%%%%%%%%%%%%%%%%%%%%%%
\paragraph{v1.6:} 2018/01/17

\begin{itemize}
\item
application for development of include files
\item
corrections to manual
\end{itemize}

%%%%%%%%%%%%%%%%%%%%%%%%%%%%%%%%%%%%%%%%
\paragraph{v1.5:} 2017/05/21

\begin{itemize}
\item
more complete structuring introduced
\item
|\childdocof| introduced
\item
|\childdoc| renamed to |\childdocmain|
\item
|\childredirect| renamed to |\childdocforward| and |\childdocforwardprefix|
and functionality expanded
\end{itemize}

%%%%%%%%%%%%%%%%%%%%%%%%%%%%%%%%%%%%%%%%
\paragraph{v1.0:} 2017/04/27

\begin{itemize}
\item
manual and install package
\item
first version published on CTAN
\end{itemize}

%%%%%%%%%%%%%%%%%%%%%%%%%%%%%%%%%%%%%%%%
\paragraph{v0.6:} 2017/04/26

\begin{itemize}
\item
redirection mechanism added
\end{itemize}

%%%%%%%%%%%%%%%%%%%%%%%%%%%%%%%%%%%%%%%%
\paragraph{v0.5:} 2017/04/26

\begin{itemize}
\item
functionality in definition file
\end{itemize}


%%%%%%%%%%%%%%%%%%%%%%%%%%%%%%%%%%%%%%%%%%%%%%%%%%%%%%%%%%%%%%%%%%%%%%%%%%%%%%%%
%%%%%%%%%%%%%%%%%%%%%%%%%%%%%%%%%%%%%%%%%%%%%%%%%%%%%%%%%%%%%%%%%%%%%%%%%%%%%%%%
%%%%%%%%%%%%%%%%%%%%%%%%%%%%%%%%%%%%%%%%%%%%%%%%%%%%%%%%%%%%%%%%%%%%%%%%%%%%%%%%
\appendix

\settowidth\MacroIndent{\rmfamily\scriptsize 000\ }

 \DocInput{childdoc.dtx}

\end{document}
%</driver>
% \fi
%
% %%%%%%%%%%%%%%%%%%%%%%%%%%%%%%%%%%%%%%%%%%%%%%%%%%%%%%%%%%%%%%%%%%%%%%%%%%%%%%
% %%%%%%%%%%%%%%%%%%%%%%%%%%%%%%%%%%%%%%%%%%%%%%%%%%%%%%%%%%%%%%%%%%%%%%%%%%%%%%
% \section{Sample}
%\iffalse
%<*samplemain>
%\fi
%
% The following presents a sample document
% with two chapters, two parts, a title page,
% a compile flag as well as three forwarding files to set the flag.
% It consists of eight |.tex| files:
% \begin{center}
% \begin{tabular}{ll}
% |cdocsamp.tex|&main file\\
% |cdocsch1.tex|&include file for chapter 1\\
% |cdocsch2.tex|&include file for chapter 2\\
% |cdocspt3.tex|&include file for part 3\\
% |cdocspt4.tex|&include file for part 4\\
% |cdocsdrf.tex|&forwarding file for main file in draft mode\\
% |cdocsfi1.tex|&forwarding file for final version of chapter 1\\
% |cdocsfi2.tex|&forwarding file for final version of chapter 2\\
% \end{tabular}
% \end{center}
% Each of the eight files can be compiled directly by the \LaTeX{} compiler.
%
% %%%%%%%%%%%%%%%%%%%%%%%%%%%%%%%%%%%%%%
% \paragraph{Main File.}
%
% The main file is called |cdocsamp.tex|.
%
% Load the \textsf{childdoc} definitions and
% declare the filename for the main document:
%    \begin{macrocode}
\input{childdoc.def}
\childdocmain{}
%    \end{macrocode}

% Optional override for |\version| flag:
%    \begin{macrocode}
%%\ifchilddoc\else\providecommand{\version}{draft}\fi
%    \end{macrocode}

% Define the default values for the |\version| flag
% (|final| for the main file and |draft| for childs):
%    \begin{macrocode}
\ifchilddoc
\providecommand{\version}{draft}
\else
\providecommand{\version}{final}
\fi
%    \end{macrocode}

% Load the standard document class:
%    \begin{macrocode}
\documentclass[12pt]{article}
%    \end{macrocode}

% Start the document body:
%    \begin{macrocode}
\begin{document}
%    \end{macrocode}

% Declare a title page.
% Print title, part of document being processed and version flag:
%    \begin{macrocode}
\addtocounter{page}{-1}
\begin{center}
{\LARGE\bfseries{}childdoc example\par}
\vspace{1cm}
\ifchilddoc
\ifchilddocmanual part\else chapter\fi:
`\childdocname' of `\childdocjob'\par
\else
main document: `\childdocjob'\par
\fi
version: \version\par
\end{center}
\newpage
%    \end{macrocode}

% Manually include selected file,
% otherwise process as usual:
%    \begin{macrocode}
\ifchilddocmanual
\section*{part `\childdocname'}
\input{\childdocname}
\else
%    \end{macrocode}

% Include the two chapters:
%    \begin{macrocode}
\include{cdocsch1}
\include{cdocsch2}
%    \end{macrocode}

% Include the two parts unless only chapters should be displayed:
%    \begin{macrocode}
\ifchilddoc\else
\section{part three}
\input{cdocspt3}
\section{part four}
\input{cdocspt4}
\fi
%    \end{macrocode}

% Process as usual until here:
%    \begin{macrocode}
\fi
%    \end{macrocode}

% End of document body:
%    \begin{macrocode}
\end{document}
%    \end{macrocode}
%\iffalse
%</samplemain>
%\fi
%
% %%%%%%%%%%%%%%%%%%%%%%%%%%%%%%%%%%%%%%
% \paragraph{Chapter Include Files.}
%
% The include files are called |cdocsch1.tex| and |cdocsch2.tex|.
%
%\iffalse
%<*samplechap1|samplechap2>
%\fi

% Optional override for |\version| flag:
%    \begin{macrocode}
%%\providecommand{\version}{final}
%    \end{macrocode}

% Include the main document:
%    \begin{macrocode}
\input{childdoc.def}
\childdocof{cdocsamp}
%    \end{macrocode}

%\iffalse
%</samplechap1|samplechap2>
%\fi
%
%\iffalse
%<*samplechap1>
%\fi
% Some text for chapter 1:
%    \begin{macrocode}
\section{one}
some text in chapter one
%    \end{macrocode}

%\iffalse
%</samplechap1>
%\fi
% Some text for chapter 2:
%\iffalse
%<*samplechap2>
%\fi
%    \begin{macrocode}
\section{two}
more text in chapter two
%    \end{macrocode}

%\iffalse
%</samplechap2>
%\fi
%
% %%%%%%%%%%%%%%%%%%%%%%%%%%%%%%%%%%%%%%
% \paragraph{Part Include Files.}
%
% The include files are called |cdocspt3.tex| and |cdocspt4.tex|.
%
%\iffalse
%<*samplepart3|samplepart4>
%\fi

% Optional override for |\version| flag:
%    \begin{macrocode}
%%\providecommand{\version}{final}
%    \end{macrocode}

% Include the main document:
%    \begin{macrocode}
\input{childdoc.def}
\childdocby{cdocsamp}
%    \end{macrocode}

%\iffalse
%</samplepart3|samplepart4>
%\fi
%
%\iffalse
%<*samplepart3>
%\fi
% Some text for part 3:
%    \begin{macrocode}
some text in part three
%    \end{macrocode}

%\iffalse
%</samplepart3>
%\fi
% Some text for part 4:
%\iffalse
%<*samplepart4>
%\fi
%    \begin{macrocode}
more text in part four
%    \end{macrocode}

%\iffalse
%</samplepart4>
%\fi
%
% %%%%%%%%%%%%%%%%%%%%%%%%%%%%%%%%%%%%%%
% \paragraph{Forwarding for a Complete Draft.}
%
% The following forwarding file |cdocsdrf.tex|
% compiles the main document in draft mode:
%\iffalse
%<*sampledraft>
%\fi
%    \begin{macrocode}
\def\version{draft}
\input{childdoc.def}
\childdocforward{cdocsamp}
%    \end{macrocode}

%\iffalse
%</sampledraft>
%\fi
%
% %%%%%%%%%%%%%%%%%%%%%%%%%%%%%%%%%%%%%%
% \paragraph{Forwarding for Final Version of the Chapters.}
%
% The following forwarding files |cdocsfn1.tex| and |cdocsfn2.tex|
% (with identical content)
% compile the final versions of the child documents
% |cdocsch1.tex| and |cdocsch2.tex|, respectively:
%\iffalse
%<*samplefinal>
%\fi
%    \begin{macrocode}
\def\version{final}
\input{childdoc.def}
\childdocforwardprefix[cdocsamp]{cdocsfn}{cdocsch}
%    \end{macrocode}

%\iffalse
%</samplefinal>
%\fi
%
% %%%%%%%%%%%%%%%%%%%%%%%%%%%%%%%%%%%%%%
% \paragraph{Command Line Processing.}
%
% The following three command lines generate the output files
% |cdocscld|, |cdocscl1| and |cdocscl2|
% which should be identical to
% |cdocsdrf|, |cdocsch1| and |cdocsfn2|, respectively:
% \begin{center}
% \begin{tabular}{l}
% |latex -jobname cdocscld \|\\
% |  "\def\version{draft}\input{childdoc.def}\childdocforward{cdocsamp}"|\\
% |latex -jobname cdocscl1 \|\\
% |  "\input{childdoc.def}\childdocforward[cdocsamp]{cdocsch1}"|\\
% |latex -jobname cdocscl2 \|\\
% |  "\def\version{final}\input{childdoc.def}\childdocforward{cdocsch2}"|
% \end{tabular}
% \end{center}
% Note that the trailing backslash on each first line
% merely continues the input to the second line
% (for convenient cut ant paste).
% Furthermore, the command |latex| can be replaced by any
% of its alternative versions such as |pdflatex|.
%
% %%%%%%%%%%%%%%%%%%%%%%%%%%%%%%%%%%%%%%%%%%%%%%%%%%%%%%%%%%%%%%%%%%%%%%%%%%%%%%
% %%%%%%%%%%%%%%%%%%%%%%%%%%%%%%%%%%%%%%%%%%%%%%%%%%%%%%%%%%%%%%%%%%%%%%%%%%%%%%
% \section{Implementation}
%\iffalse
%<*package>
%\fi
%
% This section describes the definitions file |childdoc.def|.

% The definitions cannot be loaded using |\usepackage| or |\RequirePackage|
% which has a mechanism to prevent loading a style file more than once.
% When loading the definitions by means of |\input|
% multiple instances have to be prevented manually:
%\iffalse
%This code needs to be before the `\ProvidesFile' directive
%which is defined at the beginning of this file.
%Therefore it is also placed there and commented out here.
%</package>
%<*discard>
%\fi
%    \begin{macrocode}
\ifdefined\childdocmain\endinput\fi
%    \end{macrocode}
%\iffalse
%</discard>
%<*package>
%\fi
%
% \macro{\ifchilddoc}
% \macro{\ifchilddocmanual}
% The conditional |\ifchilddoc| tells whether a
% child (true) or main (false) document is being compiled.
% The conditional |\ifchilddocmanual| tells whether
% the |\includeonly| mechanism is used (false) or
% the selection of child files must be performed manually (true).
% The definitions initialise to false:
%    \begin{macrocode}
\newif\ifchilddoc
\newif\ifchilddocmanual
%    \end{macrocode}

% \macro{\childdocname}
% \macro{\childdocjob}
% The macro |\childdocname| stores the name of the main document
% to be compiled. The macro |\childdocjob| stores the name of
% the document on which the \LaTeX{} compiler was originally invoked.
% The content of |\jobname| cannot be compared
% to filenames specified in the source due to different catcodes.
% The following code rescans |\jobname|, stores the result
% in |\childdocname| and saves a copy in |\childdocjob|:
%    \begin{macrocode}
\edef\childdocname{\scantokens\expandafter{\jobname\noexpand}}
\let\childdocjob\childdocname
%    \end{macrocode}

% \macro{\childdocdisable}
% The macro |\childdocdisable| prevents the main file
% from being processed more than once.
% At this stage, the main document command |\childdocmain|
% is assumed to be called once again where it should do nothing.
% Any subsequent call to it should prevent
% a secondary processing of the main document
% It overwrites the forwarding commands
% |\childdocof| and |\childdocforward|
% with empty macros to prevent further inclusions of the main document:
%    \begin{macrocode}
\newcommand{\childdocdisable}
{
  \renewcommand{\childdocmain}[1]{\renewcommand{\childdocmain}[1]{\endinput}}
  \renewcommand{\childdocof}[1]{}
  \renewcommand{\childdocby}[2][]{}
  \renewcommand{\childdocforward}[2][]{}
  \renewcommand{\childdocdisable}{}
}
%    \end{macrocode}

% \macro{\childdocmain}
% The macro |\childdocmain| is to be called at the top of the main file
% with nothing or the main filename (without extension) as argument.
% First, it breaks loops.
% If the argument is not empty and does not match |\childdocname|
% (which is set by the first inclusion of |childdoc.def|),
% |\ifchilddoc| is set to true, |\includeonly| is applied to the child file
% and |\jobname| is set to the main file
% (for proper handling of |.aux| files):
%    \begin{macrocode}
\newcommand{\childdocmain}[1]
{
  \childdocdisable\childdocmain{}
  \if?#1?\else
    \begingroup
      \def\childdoctmp{#1}
      \ifx\childdoctmp\childdocname
        \def\childdoctmp{}
      \else
        \def\childdoctmp
        {
          \childdoctrue
          \includeonly{\childdocname}
          \def\childdocjob{#1}
          \def\jobname{#1}
        }
      \fi
      \expandafter
    \endgroup
    \childdoctmp
  \fi
}
%    \end{macrocode}

% \macro{\childdocof}
% The command |\childdocof| redirects
% compilation to the main file |#1|.
%    \begin{macrocode}
\newcommand{\childdocof}[1]
{
  \childdocdisable
  \childdoctrue
  \includeonly{\childdocname}
  \def\jobname{#1}
  \def\childdocjob{#1}
  \input{#1}
}
%    \end{macrocode}

% \macro{\childdocby}
% The command |\childdocby| ....
%    \begin{macrocode}
\newcommand{\childdocby}[2][]
{
  \childdocdisable
  \childdoctrue
  \childdocmanualtrue
  \if?#1?\else
    \def\jobname{#2}
  \fi
  \def\childdocjob{#2}
  \input{#2}
  \endinput
}
%    \end{macrocode}

% \macro{\childdocforward}
% The command |\childdocforward| redirects
% compilation to the main file or
% (if the optional argument is given) a child file.
% Parameters are set as if the main file
% or a child file starting with |\childdocof| was compiled.
% Then compilation is handed over to the main file:
%    \begin{macrocode}
\newcommand{\childdocforward}[2][]
{
  \begingroup
    \if?#1?
      \def\childdoctmp
      {
        \def\childdocname{#2}
        \def\childdocjob{#2}
        \def\jobname{#2}
        \input{#2}
        \endinput
      }
    \else
      \def\childdoctmp
      {
        \childdocdisable
        \def\childdocname{#2}
        \childdoctrue
        \includeonly{#2}
        \def\childdocjob{#1}
        \def\jobname{#1}
        \input{#1}
        \endinput
      }
    \fi
    \expandafter
  \endgroup
  \childdoctmp
}
%    \end{macrocode}

% \macro{\childdocforwardprefix}
% The command |\childdocforwardprefix| redirects
% compilation to the main or a child file by means of a pattern.
% The prefix |#1| in the current filename is replaced by |#2|
% and the suffix of the current filename is kept
% (it is assumed that the filename does not contain the substring `|~~~|'
% which is used as a delimiter).
% Compilation is handed over to the new file by |\childdocforward|:
%    \begin{macrocode}
\newcommand{\childdocforwardprefix}[3][]
{
  \begingroup
    \def\childdocextract #2##1~~~{\def\childdoctmp{\childdocforward[#1]{#3##1}}}
    \expandafter\childdocextract\childdocname~~~
    \expandafter
  \endgroup
  \childdoctmp
}
%    \end{macrocode}

% \macro{\childdoc}
% The deprecated macro |\childdoc| is a legacy version of |\childdocmain|:
%    \begin{macrocode}
\newcommand{\childdoc}{\childdocmain}
%    \end{macrocode}

% \macro{\childdocredirect}
% The deprecated macro |\childdocredirect| is a legacy version
% of |\childdocforward| and |\childdocforwardprefix|:
%    \begin{macrocode}
\newcommand{\childdocredirect}[2][]
{
  \begingroup
    \if?#1?
      \def\childdoctmp{\childdocforward{#2}}
    \else
      \def\childdoctmp{\childdocforwardprefix{#1}{#2}}
    \fi
    \expandafter
  \endgroup
  \childdoctmp
}
%    \end{macrocode}

%\iffalse
%</package>
%\fi
%
\endinput
\childdocforward{cdocsamp}"|\\
% |latex -jobname cdocscl1 \|\\
% |  "% \iffalse
%
% childdoc.dtx Copyright (C) 2017-2018 Niklas Beisert
%
% This work may be distributed and/or modified under the
% conditions of the LaTeX Project Public License, either version 1.3
% of this license or (at your option) any later version.
% The latest version of this license is in
%   http://www.latex-project.org/lppl.txt
% and version 1.3 or later is part of all distributions of LaTeX
% version 2005/12/01 or later.
%
% This work has the LPPL maintenance status `maintained'.
%
% The Current Maintainer of this work is Niklas Beisert.
%
% This work consists of the files childdoc.dtx and childdoc.ins
% and the derived files childdoc.def and cdocsamp.tex with
% cdocsch1.tex, cdocsch2.tex, cdocsdrf.tex, cdocsfn1.tex, cdocsfn2.tex.
%
%<package>\ifdefined\childdocmain\endinput\fi
%<package>\ProvidesFile{childdoc.def}[2018/12/30 v2.0 child document driver]
%<samplemain>\ProvidesFile{cdocsamp.tex}[2018/12/30 v2.0 sample for childdoc]
%<*driver>
%\ProvidesFile{childdoc.drv}[2018/12/30 v2.0 childdoc reference manual file]
\PassOptionsToClass{10pt,a4paper}{article}
\documentclass{ltxdoc}

\usepackage[margin=35mm]{geometry}
\usepackage{hyperref}
\usepackage{hyperxmp}
\usepackage[usenames]{color}

\hypersetup{colorlinks=true}
\hypersetup{pdfstartview=FitH}
\hypersetup{pdfpagemode=UseNone}
\hypersetup{pdfsource={}}
\hypersetup{pdflang={en-UK}}
\hypersetup{pdfcopyright={Copyright 2017-2018 Niklas Beisert.
  This work may be distributed and/or modified under the
  conditions of the LaTeX Project Public License, either version 1.3
  of this license or (at your option) any later version.}}
\hypersetup{pdflicenseurl={http://www.latex-project.org/lppl.txt}}
\hypersetup{pdfcontactaddress={ETH Zurich, ITP, HIT K,
  Wolfgang-Pauli-Strasse 27}}
\hypersetup{pdfcontactpostcode={8093}}
\hypersetup{pdfcontactcity={Zurich}}
\hypersetup{pdfcontactcountry={Switzerland}}
\hypersetup{pdfcontactemail={nbeisert@itp.phys.ethz.ch}}
\hypersetup{pdfcontacturl={http://people.phys.ethz.ch/\xmptilde nbeisert/}}

\newcommand{\secref}[1]{\hyperref[#1]{section \ref*{#1}}}

\parskip1ex
\parindent0pt
\let\olditemize\itemize
\def\itemize{\olditemize\parskip0pt}

\begin{document}

\title{The \textsf{childdoc} Package}
\hypersetup{pdftitle={The childdoc Package}}
\author{Niklas Beisert\\[2ex]
  Institut f\"ur Theoretische Physik\\
  Eidgen\"ossische Technische Hochschule Z\"urich\\
  Wolfgang-Pauli-Strasse 27, 8093 Z\"urich, Switzerland\\[1ex]
  \href{mailto:nbeisert@itp.phys.ethz.ch}
  {\texttt{nbeisert@itp.phys.ethz.ch}}}
\hypersetup{pdfauthor={Niklas Beisert}}
\hypersetup{pdfsubject={Manual for the LaTeX2e Package childdoc}}
\date{30 December 2018, \textsf{v2.0}}
\maketitle

\begin{abstract}\noindent
\textsf{childdoc} is a \LaTeXe{} package
that enables the direct compilation
of document sections included by |\include|
to individual files.
\end{abstract}

\begingroup
\parskip0ex
\tableofcontents
\endgroup

%%%%%%%%%%%%%%%%%%%%%%%%%%%%%%%%%%%%%%%%%%%%%%%%%%%%%%%%%%%%%%%%%%%%%%%%%%%%%%%%
%%%%%%%%%%%%%%%%%%%%%%%%%%%%%%%%%%%%%%%%%%%%%%%%%%%%%%%%%%%%%%%%%%%%%%%%%%%%%%%%
\section{Introduction}

\LaTeX{} provides a mechanism to structure a large document (such as a book)
into a main file and several child files (containing the chapters)
using the |\include| command.
This mechanism is beneficial for documents
which span hundreds of pages in order to
make the source file(s) more manageable.
Moreover, compilation can be restricted to
selected child files by means of the |\includeonly| command.
The latter feature can be used to reduce the compilation time while editing
(this was significantly more useful in the earlier days of \LaTeX{})
or to generate a smaller document which is easier to navigate.
Another application of |\includeonly| is to generate
documents consisting of selected parts of the complete document.

However, there are a few drawbacks of the plain |\include| mechanism:
\begin{itemize}
\item
The child files cannot be compiled on their own,
they can only be compiled via the main file.
A naive editing environment
(such as a text editor with an option
to have the current file processed by \LaTeX)
may require one to switch to the main file before compiling;
attempting to compile the child file produces errors.
\item
The main file must be modified (each time)
to adjust the |\includeonly| command
to the present needs. This easily leaves the main file in a messy state.
\item
The generated document will always carry the filename
of the main document. This is inconvenient if
several child files are to be compiled and
to be kept for distribution.
\end{itemize}

The present package provides a simple interface
to make child files individually compilable by \LaTeX{}.
Compiling a child file then has the same effect as compiling
the main file with an |\includeonly| command
to select the appropriate child.
Moreover the generated document will carry the name of the child
rather than the main file.
This resolves all three above issues.

This feature is meant to make the editing of books,
thesis documents and lecture notes somewhat more convenient.
However, the package can also be used efficiently for
composing a series of documents (such as exercise sheets)
which are typically distributed individually.
It then assists the author in generating the individual documents
(potentially in different versions)
as well as a document containing the collected series.
Another application is in developing style files
or other kinds of included material
where compilation of the style file could redirect
to a sample or test file.

%%%%%%%%%%%%%%%%%%%%%%%%%%%%%%%%%%%%%%%%%%%%%%%%%%%%%%%%%%%%%%%%%%%%%%%%%%%%%%%%
%%%%%%%%%%%%%%%%%%%%%%%%%%%%%%%%%%%%%%%%%%%%%%%%%%%%%%%%%%%%%%%%%%%%%%%%%%%%%%%%
\section{Usage}

First of all, the package \textsf{childdoc} is \emph{not} a standard
\LaTeXe{} |.sty| style file! Therefore it needs to be invoked in
a non-standard way.

%%%%%%%%%%%%%%%%%%%%%%%%%%%%%%%%%%%%%%%%%%%%%%%%%%%%%%%%%%%%%%%%%%%%%%%%%%%%%%%%
\subsection{Included Files}
\label{sec:include}

%%%%%%%%%%%%%%%%%%%%%%%%%%%%%%%%%%%%%%%%
\DescribeMacro{\childdocmain}
To use the package, add the commands
\begin{center}
\begin{tabular}{l}
|\input{childdoc.def}|\\
|\childdocmain{}|\\
\end{tabular}
\end{center}
at the very top of the main \LaTeX{} file,
in particular \emph{before} the |\documentclass| statement!
The argument of |\childdocmain| should be left empty
(but it must be present).

%%%%%%%%%%%%%%%%%%%%%%%%%%%%%%%%%%%%%%%%
\DescribeMacro{\childdocof}
Furthermore, add the commands
\begin{center}
\begin{tabular}{l}
|\input{childdoc.def}|\\
|\childdocof{|\textit{main}|}|\\
\end{tabular}
\end{center}
at the top of every child file \textit{child}
which is included by |\include{|\textit{child}|}|
from within the main file
(or at least for those files to be compiled individually).
The argument \textit{main} must be the filename of the main file.

There are a couple of
considerations in setting up the main and child documents:

%%%%%%%%%%%%%%%%%%%%%%%%%%%%%%%%%%%%%%%%
\paragraph{Restrictions.}

Please note the following restrictions:
\begin{itemize}
\item
|\childdocmain| must be called with one argument \textit{main}
to ensure compatibility with earlier version of the package.
It must either be empty (|\childdocmain{}|)
or precisely match the filename of the main file in which it is specified.
See \secref{sec:detection} for further information.
\item
The filename \textit{main} must be specified without the |.tex| extension.
\item
The filename \textit{main} is case sensitive
(even in case-insensitive file systems)
due to internal string comparison.
\item
The argument \textit{main} should be fully expanded, it cannot be a macro.
\item
Subdirectories and special characters should be avoided in filenames.
\item
The command |\childdocmain{|\textit{main}|}| must be followed by a whitespace.
It should not be followed immediately by another command
or by a comment mark `|%|'.
This is because the \TeX{} parser reads the token immediately following
the argument of |\childdocmain| and puts it
at the beginning of every child section;
however, a white\-space is ignored.
\end{itemize}

%%%%%%%%%%%%%%%%%%%%%%%%%%%%%%%%%%%%%%%%
\paragraph{Content of Main File.}

It is advisable to place all content in the child files included by |\include|.
Any output contained in the main file will appear in all child documents
unless suppressed manually;
it cannot be suppressed automatically by the |\includeonly| directive
and thus should normally be avoided.
A method to include some content in the main file
by means of conditional processing is described in \secref{sec:conditional}.

%%%%%%%%%%%%%%%%%%%%%%%%%%%%%%%%%%%%%%%%
\paragraph{Page Numbering.}

When only a part of the document is compiled,
the appropriate numbering of pages
(as well as other status parameters)
is determined from the |.aux| files.
The latter contain information from previous passes.
However this information needs to propagate through
all intermediate child documents.
Therefore the page numbering in child documents may well
be inconsistent until the complete document is compiled at least once.

A useful (if unconventional) way to always ensure a consistent
page numbering is to restart the numbering in each child document
and denote the pages by `\textit{child}|.|\textit{page}'
where \textit{child} represents the chapter/section number of the child file.
This can be achieved by the command
|\numberwithin{page}{|\textit{child}|}|
of the \textsf{amsmath} package
where \textit{child} can be |chapter| or |section|
depending on the chosen structuring.
Alternatively, one can modify the macro |\thepage| appropriately
and reset the counter |page| at the start of each child file.

%%%%%%%%%%%%%%%%%%%%%%%%%%%%%%%%%%%%%%%%%%%%%%%%%%%%%%%%%%%%%%%%%%%%%%%%%%%%%%%%
\subsection{Conditional Processing}
\label{sec:conditional}

The package provides a mechanism to compile different versions
of a document. To customise the versions further some conditional processing
can come in handy to distinguish which version is being compiled.
The package provides two macros to describe the compilation context:

%%%%%%%%%%%%%%%%%%%%%%%%%%%%%%%%%%%%%%%%
\DescribeMacro{\ifchilddoc}
The conditional |\ifchilddoc| distinguishes between the compilation of
child documents and the main document:
%
\begin{center}
|\ifchilddoc |\textit{child-code}| |[|\||else |\textit{main-code}]| \||fi|
\end{center}

%%%%%%%%%%%%%%%%%%%%%%%%%%%%%%%%%%%%%%%%
\DescribeMacro{\childdocname}
\DescribeMacro{\childdocjob}
The macro |\childdocname| contains the filename (without extension)
of the main or child file being processed.
Note that |\childdocjob| will always contain the name of the main file.

%%%%%%%%%%%%%%%%%%%%%%%%%%%%%%%%%%%%%%%%
\paragraph{Title Page.}

Conditional processing can be used to include a title or banner page
in the main document when proper precautions are taken.
Importantly, the code in the main file should ensure that the page counter
(as well as other status parameters which are stored in the |.aux| files)
takes the same value after the conditional processing.
Otherwise the page numbers may take divergent values
depending on which part is compiled.

For example, a title page could be declared by:
%
\begin{center}
\begin{tabular}{l}
|\ifchilddoc\||else|\\
|\addtocounter{page}{-1}|\\
\textit{code for title page}\\
|\newpage|\\
|\||fi|
\end{tabular}
\end{center}
%
A banner page for the child documents can be generated by:
%
\begin{center}
\begin{tabular}{l}
|\ifchilddoc|\\
|\addtocounter{page}{-1}|\\
\textit{code for banner page}\\
|\newpage|\\
|\||fi|
\end{tabular}
\end{center}
%
Here one could write a message such as:
\begin{center}
|This is the part \childdocname{} of \childdocjob{}.|
\end{center}

%%%%%%%%%%%%%%%%%%%%%%%%%%%%%%%%%%%%%%%%%%%%%%%%%%%%%%%%%%%%%%%%%%%%%%%%%%%%%%%%
\subsection{Flags}
\label{sec:flags}

The package makes it easy to generate different versions
of the main or child documents.
To this end compilation flags can be defined
and assigned different default values.
They will be particularly useful in conjunction
with the forwarding mechanism described in \secref{sec:forward}.

For example, it may be useful to have a flag |\version|
which can be set to |draft| or |final|.
The document source will contain some conditional code
depending on the value of |\version|.
Suppose further, the flag should default to |final| for the main file
and to |draft| for child files
which is a natural assignment for editing the document.
This is achieved by placing the following code
in the preamble of the main document
(below the |\childdocmain| directive):
%
\begin{center}
\begin{tabular}{l}
|\ifchilddoc|\\
|\providecommand{\version}{draft}|\\
|\||else|\\
|\providecommand{\version}{final}|\\
|\||fi|
\end{tabular}
\end{center}
%
The definition by |\providecommand| makes sure
that previous definitions are not overwritten.
Further statements |\providecommand{\version}{...}|
can thus be added before the above code to override it.

For the main file, one might add a line
(between |\childdocmain| and the above block)
%
\begin{center}
|%\ifchilddoc\||else\providecommand{\version}{draft}\||fi|
\end{center}
%
which can be uncommented to produce a draft version.
Likewise one can add a line to the very top of a child file
(above the |\childdocof{|\textit{main}|}| directive)
%
\begin{center}
|%\providecommand{\version}{final}|
\end{center}
%
which can be uncommented to produce the final version of this child document.

%%%%%%%%%%%%%%%%%%%%%%%%%%%%%%%%%%%%%%%%%%%%%%%%%%%%%%%%%%%%%%%%%%%%%%%%%%%%%%%%
\subsection{Forwarding}
\label{sec:forward}

Different versions of the main or child documents
using compilation flags as described in \secref{sec:flags}
can be (permanently) stored in different files
for convenient compilation, viewing and distribution.
To this end, the package defines a command
to pass on compilation to a different file:

%%%%%%%%%%%%%%%%%%%%%%%%%%%%%%%%%%%%%%%%
\DescribeMacro{\childdocforward}
The command |\childdocforward| redirects processing to
another source file:
%
\begin{center}
\begin{tabular}{l}
|\input{childdoc.def}|\\
|\childdocforward[|\textit{main}|]{|\textit{dest}|}|\\
\end{tabular}
\end{center}
%
The argument \textit{dest} is the destination file
(without extension).
It should be the main file or one of the child files.
Note that further \textsf{childdoc} directives
such as |\childdocof| and |\childdocforward|
in the indicated file will be processed in this form.
The optional argument \textit{main}
passes on directly to the main file \textit{main}
while pretending to compile the child \textit{dest}.
This form behaves as if \textit{dest}
issues |\childdocof{|\textit{main}|}| right away,
and no further \textsf{childdoc} directives will be processed.

%%%%%%%%%%%%%%%%%%%%%%%%%%%%%%%%%%%%%%%%
\DescribeMacro{\...prefix}
In the alternative form |\childdocforwardprefix|,
%
\begin{center}
\begin{tabular}{l}
|\input{childdoc.def}|\\
|\childdocforwardprefix[|\textit{main}|]{|\textit{prefix}|}{|\textit{dest}|}|
\end{tabular}
\end{center}
%
the destination file is determined by a pattern
depending on the current file:
To make this work, the current file must be called
`{\textit{prefix}\hspace{0.2em}\textit{suffix}}'
with \textit{prefix} matching precisely the argument.
Processing is then passed on to the file
`{\textit{dest}\hspace{0.2em}\textit{suffix}}'.
Surely, the same effect is achieved by
directly specifying the
argument `{\textit{dest}\hspace{0.2em}\textit{suffix}}'
in the first form.
However, that requires to set up a different file
for each child. With the alternative form of the command
all these files can have exactly the same content
which simplifies setting them up and maintaining them.

For example, the following file |draft.tex|
with a compilation flag |\version| as described in \secref{sec:flags}
compiles the main document as a draft:
%
\begin{center}
\begin{tabular}{l}
|\def\version{draft}|\\
|\input{childdoc.def}|\\
|\childdocforward{|\textit{main}|}|
\end{tabular}
\end{center}
%
Likewise, the following files |final|\textit{nn}|.tex|
compile the final version of the child document
|child|\textit{nn}|.tex|:
%
\begin{center}
\begin{tabular}{l}
|\def\version{final}|\\
|\input{childdoc.def}|\\
|\childdocforwardprefix{final}{child}|
\end{tabular}
\end{center}
%

Note that when several versions of a main file and/or of each child file
are to be generated, it may be convenient to set up a |Makefile| or
shell script to automatise the process.

%%%%%%%%%%%%%%%%%%%%%%%%%%%%%%%%%%%%%%%%%%%%%%%%%%%%%%%%%%%%%%%%%%%%%%%%%%%%%%%%
\subsection{Command Line Processing}
\label{sec:commandline}

The effect of redirection files can also be achieved by invoking
the \LaTeX{} compiler with a more elaborate command line.
Most conveniently this should be done as part
of a shell script or a |Makefile|.

When using \textsf{childdoc} in the main file, the following
command lines effectively perform a redirection
(note that depending on the shell being used,
backslashes may have to be doubled: `|\|' $\to$ `|\\|'):
%
\begin{center}
|... -jobname "|\textit{target}|" |\\|"|[\textit{flags}]%
|\input{childdoc.def}\childdocforward[|\textit{main}|]{|\textit{dest}|}"|
\end{center}
%
Here \textit{target} is the name of the output file,
\textit{main} is the name of the main file
and \textit{dest} is the name of the main or child file to be processed
(all filenames without extensions).
The optional argument \textit{main} can be omitted
if \textit{main} matches \textit{dest}.
Optionally, compilation \textit{flags} can be defined via |\def| commands.
This command line makes the \TeX{} engine believe
it is compiling the file \textit{target}
whose content is specified as the latter parameter.
The provided code then forwards the processing to
\textit{main} or \textit{dest} as described in \secref{sec:forward}.

%%%%%%%%%%%%%%%%%%%%%%%%%%%%%%%%%%%%%%%%%%%%%%%%%%%%%%%%%%%%%%%%%%%%%%%%%%%%%%%%
\subsection{Include by Input}
\label{sec:input}

Including child documents by |\include| has some restrictions by design.
Most notably, the content of a child document always occupies
its own set of pages; pages cannot be shared between child documents.
Usually, this behaviour makes perfect sense
because each child document contain an essential part of the document.
However, in some situations it may be desirable to compose
a document from a collection of parts
without having mandatory page breaks between then.
For this case, the package
provides a mechanism to include parts
by |\input| which can also be processed individually.
However, by construction this mechanism
requires manual handling of the content to be output.

%%%%%%%%%%%%%%%%%%%%%%%%%%%%%%%%%%%%%%%%
\DescribeMacro{\ifchilddocmanual}
The main file should be prepared as usual, see \secref{sec:include}.
However, the document body must make a distinction
between processing of an individual part and of the main document, e.g.:
%
\begin{center}
\begin{tabular}{l}
|\ifchilddocmanual|\\
|\input{\childdocname}|\\
|\||else|\\
\textit{document body with }|\input{|\textit{part}|}|\\
|\||fi|
\end{tabular}
\end{center}
%
The conditional |\ifchilddocmanual| is true whenever
a part to be included by |\input| is being compiled,
and the name of the part is stored in |\childdocname|.

%%%%%%%%%%%%%%%%%%%%%%%%%%%%%%%%%%%%%%%%
\DescribeMacro{\childdocby}
Each part to be included by |\input| should start with:
%
\begin{center}
\begin{tabular}{l}
|\input{childdoc.def}|\\
|\childdocby{|\textit{main}|}|\\
\end{tabular}
\end{center}
%
The directive |\childdocby| is similar to |\childdocof|
described in \secref{sec:include},
but the subsequent selection of content must be done manually.
To that end, both |\ifchilddoc| and |\ifchilddocmanual|
will be true upon processing of a part,
and the name of the part is stored in |\childdocname|.
Note that |\jobname| will be set to the filename of the current part
so that each part receives an individual |.aux| file
that does not interfere with the |.aux| file(s) of the main document.
This behaviour can be altered by the alternative form
|\childdocby[*]{|\textit{main}|}| (with a non-empty optional argument)
which uses the |.aux| file of the main document
by setting |\jobname| to \textit{main}.

%%%%%%%%%%%%%%%%%%%%%%%%%%%%%%%%%%%%%%%%%%%%%%%%%%%%%%%%%%%%%%%%%%%%%%%%%%%%%%%%
\subsection{Driver Development}
\label{sec:driver}

The \textsf{childdoc} mechanism can also be use for the development
of definition files such as \LaTeX{} styles or classes.
This case differs from the above setup with multiple parts
included by |\include| in that no |\includeonly| should be invoked.
This can be achieved by starting the include file
(before |\ProvidesPackage|) with:
%
\begin{center}
\begin{tabular}{l}
|\input{childdoc.def}|\\
|\childdocforward{|\textit{main}|}|\\
\end{tabular}
\end{center}
%
or alternatively with:
%
\begin{center}
\begin{tabular}{l}
|\input{childdoc.def}|\\
|\childdocby{|\textit{main}|}|\\
\end{tabular}
\end{center}
%
Both forms have slightly different effects as described above.
The main file is prepared as usual, see \secref{sec:include}.

%%%%%%%%%%%%%%%%%%%%%%%%%%%%%%%%%%%%%%%%%%%%%%%%%%%%%%%%%%%%%%%%%%%%%%%%%%%%%%%%
\subsection{Legacy Detection}
\label{sec:detection}

The directive |\childdocmain| in the main file can detect
whether the complete document or merely a child is to be compiled
even without using the directive |\childdocof|.
This method is deprecated because it is less robust
and there is no compelling reason to use it;
it is merely provided for backward compatibility
and it may be removed in future versions.

If the detection mechanism is to be used,
it is mandatory to correctly specify
the filename of the main file as the argument of |\childdocmain|:
%
\begin{center}
\begin{tabular}{l}
|\input{childdoc.def}|\\
|\childdocmain{|\textit{main}|}|\\
\end{tabular}
\end{center}
%
If |\jobname| does not match the argument \textit{main} of |\childdocmain|,
it is assumed that |\jobname| points to the child file to be compiled.
When using |\childdocmain| with the main file specified as argument,
it suffices to start a child file
with just |\input{|\textit{main}|}|
without loading of the package and using |\childdocof|.
If instead all processing is done
with the appropriate \textsf{childdoc} directives,
the argument of \textit{main} of |\childdocmain| can be empty.

An alternative version of the command line processing described
in \secref{sec:commandline} using the detection mechanism reads:
%
\begin{center}
|... -jobname "|\textit{target}|" "|[\textit{flags}]%
[|\def\jobname{|\textit{dest}|}|]|\input{|\textit{main}|}"|
\end{center}

%%%%%%%%%%%%%%%%%%%%%%%%%%%%%%%%%%%%%%%%%%%%%%%%%%%%%%%%%%%%%%%%%%%%%%%%%%%%%%%%
\subsection{Manual Code}
\label{sec:manual}

In case one cannot be certain whether the definitions file |childdoc.def|
is installed on the target \TeX{} distribution
and one prefers not to ship it,
it is conceivable to paste a few relevant commands into the sources.

To that end, drop all statements |\input{childdoc.def}|
and perform the replacements as outlined below.
Instead of |\childdocmain{|\textit{main}|}| add the following code
to the top of the main file:
%
\begin{center}
\begin{tabular}{l}
|\||ifdefined\childdocname\endinput\||fi\newif\ifchilddoc|\\
|\edef\childdocname{\scantokens\expandafter{\jobname\noexpand}}|\\
|\def\childdocmain{|\textit{main}|}\||ifx\childdocmain\childdocname\||else|\\
|\childdoctrue\includeonly{\childdocname}\let\jobname\childdocmain\||fi|\\
\end{tabular}
\end{center}
%
Instead of |\childdocof{|\textit{main}|}| just include the main file
at the top of each child file:
%
\begin{center}
|\input{|\textit{main}|}|
\end{center}
%
A simple redirection |\childdocforward{|\textit{dest}|}| is achieved by:
%
\begin{center}
|\def\jobname{|\textit{dest}|}\input{\jobname}|
\end{center}
%
The redirection with prefix
|\childdocforwardprefix[|\textit{prefix}|]{|\textit{dest}|}|
is accomplished by:
%
\begin{center}
\begin{tabular}{l}
|{\edef\jobname{\scantokens\expandafter{\jobname\noexpand}}|\\
|\def\redirectjob |\textit{prefix}|#1~~~{\gdef\jobname{|\textit{dest}|#1}}|\\
|\expandafter\redirectjob\jobname~~~}\input{\jobname}|
\end{tabular}
\end{center}

In an alternative approach,
child documents can be compiled by a specific command line
without additional code or specific definitions:
%
\begin{center}
|... -jobname "|\textit{target}|" "|[\textit{flags}]%
|\includeonly{|\textit{dest}|}\input{|\textit{main}|}"|
\end{center}
%

%%%%%%%%%%%%%%%%%%%%%%%%%%%%%%%%%%%%%%%%%%%%%%%%%%%%%%%%%%%%%%%%%%%%%%%%%%%%%%%%
%%%%%%%%%%%%%%%%%%%%%%%%%%%%%%%%%%%%%%%%%%%%%%%%%%%%%%%%%%%%%%%%%%%%%%%%%%%%%%%%
\section{Information}

%%%%%%%%%%%%%%%%%%%%%%%%%%%%%%%%%%%%%%%%%%%%%%%%%%%%%%%%%%%%%%%%%%%%%%%%%%%%%%%%
\subsection{Copyright}

Copyright \copyright{} 2017--2018 Niklas Beisert

This work may be distributed and/or modified under the
conditions of the \LaTeX{} Project Public License, either version 1.3
of this license or (at your option) any later version.
The latest version of this license is in
  \url{http://www.latex-project.org/lppl.txt}
and version 1.3 or later is part of all distributions of \LaTeX{}
version 2005/12/01 or later.

This work has the LPPL maintenance status `maintained'.

The Current Maintainer of this work is Niklas Beisert.

This work consists of the files |README.txt|, |childdoc.ins| and |childdoc.dtx|
as well as the derived files |childdoc.def|, |cdocsamp.tex|
with |cdocsch1.tex|, |cdocsch2.tex|, |cdocspt3.tex|, |cdocspt4.tex|,
|cdocsdrf.tex|, |cdocsfn1.tex|, |cdocsfn2.tex|
as well as |childdoc.pdf|.

%%%%%%%%%%%%%%%%%%%%%%%%%%%%%%%%%%%%%%%%%%%%%%%%%%%%%%%%%%%%%%%%%%%%%%%%%%%%%%%%
\subsection{Files and Installation}

The package consists of the files:
%
\begin{center}
\begin{tabular}{ll}
    |README.txt|   & readme file \\
    |childdoc.ins| & installation file \\
    |childdoc.dtx| & source file \\
    |childdoc.def| & definition file \\
    |cdocsamp.tex| & sample main file \\
    |cdocsch1.tex| & sample include file \\
    |cdocsch2.tex| & sample include file \\
    |cdocspt3.tex| & sample part file \\
    |cdocspt4.tex| & sample part file \\
    |cdocsdrf.tex| & sample redirection file \\
    |cdocsfn1.tex| & sample redirection file \\
    |cdocsfn2.tex| & sample redirection file \\
    |childdoc.pdf| & manual
\end{tabular}
\end{center}
%
The distribution consists of the files
|README.txt|, |childdoc.ins| and |childdoc.dtx|.
%
\begin{itemize}
\item
Run (pdf)\LaTeX{} on |childdoc.dtx|
to compile the manual |childdoc.pdf| (this file).
\item
Run \LaTeX{} on |childdoc.ins| to create the definitions file |childdoc.def|
and the sample |cdocsamp.tex| with include files
|cdocsch1.tex|, |cdocsch2.tex|, |cdocspt3.tex|, |cdocspt4.tex|,
|cdocsdrf.tex|, |cdocsfn1.tex|, |cdocsfn2.tex|.
Then copy the file |childdoc.def| to an appropriate directory of your \LaTeX{}
distribution, e.g.\ \textit{texmf-root}|/tex/latex/childdoc|.
\end{itemize}

%%%%%%%%%%%%%%%%%%%%%%%%%%%%%%%%%%%%%%%%%%%%%%%%%%%%%%%%%%%%%%%%%%%%%%%%%%%%%%%%
\subsection{Related CTAN Packages}

There are several other packages which offer a similar functionality:
%
\begin{itemize}
\item
The packages
\href{http://ctan.org/pkg/docmute}{\textsf{docmute}},
\href{http://ctan.org/pkg/includex}{\textsf{includex}} and
\href{http://ctan.org/pkg/standalone}{\textsf{standalone}}
provide commands to include only the document body of
a child file thus allowing both files to be compiled individually.
\item
The packages \href{http://ctan.org/pkg/subdocs}{\textsf{subdocs}}
and \href{http://ctan.org/pkg/subfiles}{\textsf{subfiles}}
provide structures in which the main and child documents can be
encapsulated and allowing them to be compiled individually.
The inclusion mechanism is different from the conventional |\include|.
\item
The package \href{http://ctan.org/pkg/combine}{\textsf{combine}}
is an elaborate solution to combine several documents into one.
\end{itemize}
%
See also the CTAN topic \href{http://ctan.org/topic/subdocs}{\textsf{subdocs}}
for further related packages.
The present package differs from the above solutions in that
a document structure constructed with the conventional |\include| mechanism
just needs two extra commands at the top of every file
such that all constituent files can be compiled individually.

%%%%%%%%%%%%%%%%%%%%%%%%%%%%%%%%%%%%%%%%%%%%%%%%%%%%%%%%%%%%%%%%%%%%%%%%%%%%%%%%
%\subsection{Feature Suggestions}
%
%The following is a list of features which may be useful for future
%versions of this package:
%%
%\begin{itemize}
%\item
%\ldots
%\end{itemize}

%%%%%%%%%%%%%%%%%%%%%%%%%%%%%%%%%%%%%%%%%%%%%%%%%%%%%%%%%%%%%%%%%%%%%%%%%%%%%%%%
\subsection{Revision History}

%%%%%%%%%%%%%%%%%%%%%%%%%%%%%%%%%%%%%%%%
\paragraph{v2.0:} 2018/12/30

\begin{itemize}
\item
immediate forward processing
\item
added |\childdocby| mechanism
\item
manual restructured
\end{itemize}

%%%%%%%%%%%%%%%%%%%%%%%%%%%%%%%%%%%%%%%%
\paragraph{v1.6:} 2018/01/17

\begin{itemize}
\item
application for development of include files
\item
corrections to manual
\end{itemize}

%%%%%%%%%%%%%%%%%%%%%%%%%%%%%%%%%%%%%%%%
\paragraph{v1.5:} 2017/05/21

\begin{itemize}
\item
more complete structuring introduced
\item
|\childdocof| introduced
\item
|\childdoc| renamed to |\childdocmain|
\item
|\childredirect| renamed to |\childdocforward| and |\childdocforwardprefix|
and functionality expanded
\end{itemize}

%%%%%%%%%%%%%%%%%%%%%%%%%%%%%%%%%%%%%%%%
\paragraph{v1.0:} 2017/04/27

\begin{itemize}
\item
manual and install package
\item
first version published on CTAN
\end{itemize}

%%%%%%%%%%%%%%%%%%%%%%%%%%%%%%%%%%%%%%%%
\paragraph{v0.6:} 2017/04/26

\begin{itemize}
\item
redirection mechanism added
\end{itemize}

%%%%%%%%%%%%%%%%%%%%%%%%%%%%%%%%%%%%%%%%
\paragraph{v0.5:} 2017/04/26

\begin{itemize}
\item
functionality in definition file
\end{itemize}


%%%%%%%%%%%%%%%%%%%%%%%%%%%%%%%%%%%%%%%%%%%%%%%%%%%%%%%%%%%%%%%%%%%%%%%%%%%%%%%%
%%%%%%%%%%%%%%%%%%%%%%%%%%%%%%%%%%%%%%%%%%%%%%%%%%%%%%%%%%%%%%%%%%%%%%%%%%%%%%%%
%%%%%%%%%%%%%%%%%%%%%%%%%%%%%%%%%%%%%%%%%%%%%%%%%%%%%%%%%%%%%%%%%%%%%%%%%%%%%%%%
\appendix

\settowidth\MacroIndent{\rmfamily\scriptsize 000\ }

 \DocInput{childdoc.dtx}

\end{document}
%</driver>
% \fi
%
% %%%%%%%%%%%%%%%%%%%%%%%%%%%%%%%%%%%%%%%%%%%%%%%%%%%%%%%%%%%%%%%%%%%%%%%%%%%%%%
% %%%%%%%%%%%%%%%%%%%%%%%%%%%%%%%%%%%%%%%%%%%%%%%%%%%%%%%%%%%%%%%%%%%%%%%%%%%%%%
% \section{Sample}
%\iffalse
%<*samplemain>
%\fi
%
% The following presents a sample document
% with two chapters, two parts, a title page,
% a compile flag as well as three forwarding files to set the flag.
% It consists of eight |.tex| files:
% \begin{center}
% \begin{tabular}{ll}
% |cdocsamp.tex|&main file\\
% |cdocsch1.tex|&include file for chapter 1\\
% |cdocsch2.tex|&include file for chapter 2\\
% |cdocspt3.tex|&include file for part 3\\
% |cdocspt4.tex|&include file for part 4\\
% |cdocsdrf.tex|&forwarding file for main file in draft mode\\
% |cdocsfi1.tex|&forwarding file for final version of chapter 1\\
% |cdocsfi2.tex|&forwarding file for final version of chapter 2\\
% \end{tabular}
% \end{center}
% Each of the eight files can be compiled directly by the \LaTeX{} compiler.
%
% %%%%%%%%%%%%%%%%%%%%%%%%%%%%%%%%%%%%%%
% \paragraph{Main File.}
%
% The main file is called |cdocsamp.tex|.
%
% Load the \textsf{childdoc} definitions and
% declare the filename for the main document:
%    \begin{macrocode}
\input{childdoc.def}
\childdocmain{}
%    \end{macrocode}

% Optional override for |\version| flag:
%    \begin{macrocode}
%%\ifchilddoc\else\providecommand{\version}{draft}\fi
%    \end{macrocode}

% Define the default values for the |\version| flag
% (|final| for the main file and |draft| for childs):
%    \begin{macrocode}
\ifchilddoc
\providecommand{\version}{draft}
\else
\providecommand{\version}{final}
\fi
%    \end{macrocode}

% Load the standard document class:
%    \begin{macrocode}
\documentclass[12pt]{article}
%    \end{macrocode}

% Start the document body:
%    \begin{macrocode}
\begin{document}
%    \end{macrocode}

% Declare a title page.
% Print title, part of document being processed and version flag:
%    \begin{macrocode}
\addtocounter{page}{-1}
\begin{center}
{\LARGE\bfseries{}childdoc example\par}
\vspace{1cm}
\ifchilddoc
\ifchilddocmanual part\else chapter\fi:
`\childdocname' of `\childdocjob'\par
\else
main document: `\childdocjob'\par
\fi
version: \version\par
\end{center}
\newpage
%    \end{macrocode}

% Manually include selected file,
% otherwise process as usual:
%    \begin{macrocode}
\ifchilddocmanual
\section*{part `\childdocname'}
\input{\childdocname}
\else
%    \end{macrocode}

% Include the two chapters:
%    \begin{macrocode}
\include{cdocsch1}
\include{cdocsch2}
%    \end{macrocode}

% Include the two parts unless only chapters should be displayed:
%    \begin{macrocode}
\ifchilddoc\else
\section{part three}
\input{cdocspt3}
\section{part four}
\input{cdocspt4}
\fi
%    \end{macrocode}

% Process as usual until here:
%    \begin{macrocode}
\fi
%    \end{macrocode}

% End of document body:
%    \begin{macrocode}
\end{document}
%    \end{macrocode}
%\iffalse
%</samplemain>
%\fi
%
% %%%%%%%%%%%%%%%%%%%%%%%%%%%%%%%%%%%%%%
% \paragraph{Chapter Include Files.}
%
% The include files are called |cdocsch1.tex| and |cdocsch2.tex|.
%
%\iffalse
%<*samplechap1|samplechap2>
%\fi

% Optional override for |\version| flag:
%    \begin{macrocode}
%%\providecommand{\version}{final}
%    \end{macrocode}

% Include the main document:
%    \begin{macrocode}
\input{childdoc.def}
\childdocof{cdocsamp}
%    \end{macrocode}

%\iffalse
%</samplechap1|samplechap2>
%\fi
%
%\iffalse
%<*samplechap1>
%\fi
% Some text for chapter 1:
%    \begin{macrocode}
\section{one}
some text in chapter one
%    \end{macrocode}

%\iffalse
%</samplechap1>
%\fi
% Some text for chapter 2:
%\iffalse
%<*samplechap2>
%\fi
%    \begin{macrocode}
\section{two}
more text in chapter two
%    \end{macrocode}

%\iffalse
%</samplechap2>
%\fi
%
% %%%%%%%%%%%%%%%%%%%%%%%%%%%%%%%%%%%%%%
% \paragraph{Part Include Files.}
%
% The include files are called |cdocspt3.tex| and |cdocspt4.tex|.
%
%\iffalse
%<*samplepart3|samplepart4>
%\fi

% Optional override for |\version| flag:
%    \begin{macrocode}
%%\providecommand{\version}{final}
%    \end{macrocode}

% Include the main document:
%    \begin{macrocode}
\input{childdoc.def}
\childdocby{cdocsamp}
%    \end{macrocode}

%\iffalse
%</samplepart3|samplepart4>
%\fi
%
%\iffalse
%<*samplepart3>
%\fi
% Some text for part 3:
%    \begin{macrocode}
some text in part three
%    \end{macrocode}

%\iffalse
%</samplepart3>
%\fi
% Some text for part 4:
%\iffalse
%<*samplepart4>
%\fi
%    \begin{macrocode}
more text in part four
%    \end{macrocode}

%\iffalse
%</samplepart4>
%\fi
%
% %%%%%%%%%%%%%%%%%%%%%%%%%%%%%%%%%%%%%%
% \paragraph{Forwarding for a Complete Draft.}
%
% The following forwarding file |cdocsdrf.tex|
% compiles the main document in draft mode:
%\iffalse
%<*sampledraft>
%\fi
%    \begin{macrocode}
\def\version{draft}
\input{childdoc.def}
\childdocforward{cdocsamp}
%    \end{macrocode}

%\iffalse
%</sampledraft>
%\fi
%
% %%%%%%%%%%%%%%%%%%%%%%%%%%%%%%%%%%%%%%
% \paragraph{Forwarding for Final Version of the Chapters.}
%
% The following forwarding files |cdocsfn1.tex| and |cdocsfn2.tex|
% (with identical content)
% compile the final versions of the child documents
% |cdocsch1.tex| and |cdocsch2.tex|, respectively:
%\iffalse
%<*samplefinal>
%\fi
%    \begin{macrocode}
\def\version{final}
\input{childdoc.def}
\childdocforwardprefix[cdocsamp]{cdocsfn}{cdocsch}
%    \end{macrocode}

%\iffalse
%</samplefinal>
%\fi
%
% %%%%%%%%%%%%%%%%%%%%%%%%%%%%%%%%%%%%%%
% \paragraph{Command Line Processing.}
%
% The following three command lines generate the output files
% |cdocscld|, |cdocscl1| and |cdocscl2|
% which should be identical to
% |cdocsdrf|, |cdocsch1| and |cdocsfn2|, respectively:
% \begin{center}
% \begin{tabular}{l}
% |latex -jobname cdocscld \|\\
% |  "\def\version{draft}\input{childdoc.def}\childdocforward{cdocsamp}"|\\
% |latex -jobname cdocscl1 \|\\
% |  "\input{childdoc.def}\childdocforward[cdocsamp]{cdocsch1}"|\\
% |latex -jobname cdocscl2 \|\\
% |  "\def\version{final}\input{childdoc.def}\childdocforward{cdocsch2}"|
% \end{tabular}
% \end{center}
% Note that the trailing backslash on each first line
% merely continues the input to the second line
% (for convenient cut ant paste).
% Furthermore, the command |latex| can be replaced by any
% of its alternative versions such as |pdflatex|.
%
% %%%%%%%%%%%%%%%%%%%%%%%%%%%%%%%%%%%%%%%%%%%%%%%%%%%%%%%%%%%%%%%%%%%%%%%%%%%%%%
% %%%%%%%%%%%%%%%%%%%%%%%%%%%%%%%%%%%%%%%%%%%%%%%%%%%%%%%%%%%%%%%%%%%%%%%%%%%%%%
% \section{Implementation}
%\iffalse
%<*package>
%\fi
%
% This section describes the definitions file |childdoc.def|.

% The definitions cannot be loaded using |\usepackage| or |\RequirePackage|
% which has a mechanism to prevent loading a style file more than once.
% When loading the definitions by means of |\input|
% multiple instances have to be prevented manually:
%\iffalse
%This code needs to be before the `\ProvidesFile' directive
%which is defined at the beginning of this file.
%Therefore it is also placed there and commented out here.
%</package>
%<*discard>
%\fi
%    \begin{macrocode}
\ifdefined\childdocmain\endinput\fi
%    \end{macrocode}
%\iffalse
%</discard>
%<*package>
%\fi
%
% \macro{\ifchilddoc}
% \macro{\ifchilddocmanual}
% The conditional |\ifchilddoc| tells whether a
% child (true) or main (false) document is being compiled.
% The conditional |\ifchilddocmanual| tells whether
% the |\includeonly| mechanism is used (false) or
% the selection of child files must be performed manually (true).
% The definitions initialise to false:
%    \begin{macrocode}
\newif\ifchilddoc
\newif\ifchilddocmanual
%    \end{macrocode}

% \macro{\childdocname}
% \macro{\childdocjob}
% The macro |\childdocname| stores the name of the main document
% to be compiled. The macro |\childdocjob| stores the name of
% the document on which the \LaTeX{} compiler was originally invoked.
% The content of |\jobname| cannot be compared
% to filenames specified in the source due to different catcodes.
% The following code rescans |\jobname|, stores the result
% in |\childdocname| and saves a copy in |\childdocjob|:
%    \begin{macrocode}
\edef\childdocname{\scantokens\expandafter{\jobname\noexpand}}
\let\childdocjob\childdocname
%    \end{macrocode}

% \macro{\childdocdisable}
% The macro |\childdocdisable| prevents the main file
% from being processed more than once.
% At this stage, the main document command |\childdocmain|
% is assumed to be called once again where it should do nothing.
% Any subsequent call to it should prevent
% a secondary processing of the main document
% It overwrites the forwarding commands
% |\childdocof| and |\childdocforward|
% with empty macros to prevent further inclusions of the main document:
%    \begin{macrocode}
\newcommand{\childdocdisable}
{
  \renewcommand{\childdocmain}[1]{\renewcommand{\childdocmain}[1]{\endinput}}
  \renewcommand{\childdocof}[1]{}
  \renewcommand{\childdocby}[2][]{}
  \renewcommand{\childdocforward}[2][]{}
  \renewcommand{\childdocdisable}{}
}
%    \end{macrocode}

% \macro{\childdocmain}
% The macro |\childdocmain| is to be called at the top of the main file
% with nothing or the main filename (without extension) as argument.
% First, it breaks loops.
% If the argument is not empty and does not match |\childdocname|
% (which is set by the first inclusion of |childdoc.def|),
% |\ifchilddoc| is set to true, |\includeonly| is applied to the child file
% and |\jobname| is set to the main file
% (for proper handling of |.aux| files):
%    \begin{macrocode}
\newcommand{\childdocmain}[1]
{
  \childdocdisable\childdocmain{}
  \if?#1?\else
    \begingroup
      \def\childdoctmp{#1}
      \ifx\childdoctmp\childdocname
        \def\childdoctmp{}
      \else
        \def\childdoctmp
        {
          \childdoctrue
          \includeonly{\childdocname}
          \def\childdocjob{#1}
          \def\jobname{#1}
        }
      \fi
      \expandafter
    \endgroup
    \childdoctmp
  \fi
}
%    \end{macrocode}

% \macro{\childdocof}
% The command |\childdocof| redirects
% compilation to the main file |#1|.
%    \begin{macrocode}
\newcommand{\childdocof}[1]
{
  \childdocdisable
  \childdoctrue
  \includeonly{\childdocname}
  \def\jobname{#1}
  \def\childdocjob{#1}
  \input{#1}
}
%    \end{macrocode}

% \macro{\childdocby}
% The command |\childdocby| ....
%    \begin{macrocode}
\newcommand{\childdocby}[2][]
{
  \childdocdisable
  \childdoctrue
  \childdocmanualtrue
  \if?#1?\else
    \def\jobname{#2}
  \fi
  \def\childdocjob{#2}
  \input{#2}
  \endinput
}
%    \end{macrocode}

% \macro{\childdocforward}
% The command |\childdocforward| redirects
% compilation to the main file or
% (if the optional argument is given) a child file.
% Parameters are set as if the main file
% or a child file starting with |\childdocof| was compiled.
% Then compilation is handed over to the main file:
%    \begin{macrocode}
\newcommand{\childdocforward}[2][]
{
  \begingroup
    \if?#1?
      \def\childdoctmp
      {
        \def\childdocname{#2}
        \def\childdocjob{#2}
        \def\jobname{#2}
        \input{#2}
        \endinput
      }
    \else
      \def\childdoctmp
      {
        \childdocdisable
        \def\childdocname{#2}
        \childdoctrue
        \includeonly{#2}
        \def\childdocjob{#1}
        \def\jobname{#1}
        \input{#1}
        \endinput
      }
    \fi
    \expandafter
  \endgroup
  \childdoctmp
}
%    \end{macrocode}

% \macro{\childdocforwardprefix}
% The command |\childdocforwardprefix| redirects
% compilation to the main or a child file by means of a pattern.
% The prefix |#1| in the current filename is replaced by |#2|
% and the suffix of the current filename is kept
% (it is assumed that the filename does not contain the substring `|~~~|'
% which is used as a delimiter).
% Compilation is handed over to the new file by |\childdocforward|:
%    \begin{macrocode}
\newcommand{\childdocforwardprefix}[3][]
{
  \begingroup
    \def\childdocextract #2##1~~~{\def\childdoctmp{\childdocforward[#1]{#3##1}}}
    \expandafter\childdocextract\childdocname~~~
    \expandafter
  \endgroup
  \childdoctmp
}
%    \end{macrocode}

% \macro{\childdoc}
% The deprecated macro |\childdoc| is a legacy version of |\childdocmain|:
%    \begin{macrocode}
\newcommand{\childdoc}{\childdocmain}
%    \end{macrocode}

% \macro{\childdocredirect}
% The deprecated macro |\childdocredirect| is a legacy version
% of |\childdocforward| and |\childdocforwardprefix|:
%    \begin{macrocode}
\newcommand{\childdocredirect}[2][]
{
  \begingroup
    \if?#1?
      \def\childdoctmp{\childdocforward{#2}}
    \else
      \def\childdoctmp{\childdocforwardprefix{#1}{#2}}
    \fi
    \expandafter
  \endgroup
  \childdoctmp
}
%    \end{macrocode}

%\iffalse
%</package>
%\fi
%
\endinput
\childdocforward[cdocsamp]{cdocsch1}"|\\
% |latex -jobname cdocscl2 \|\\
% |  "\def\version{final}% \iffalse
%
% childdoc.dtx Copyright (C) 2017-2018 Niklas Beisert
%
% This work may be distributed and/or modified under the
% conditions of the LaTeX Project Public License, either version 1.3
% of this license or (at your option) any later version.
% The latest version of this license is in
%   http://www.latex-project.org/lppl.txt
% and version 1.3 or later is part of all distributions of LaTeX
% version 2005/12/01 or later.
%
% This work has the LPPL maintenance status `maintained'.
%
% The Current Maintainer of this work is Niklas Beisert.
%
% This work consists of the files childdoc.dtx and childdoc.ins
% and the derived files childdoc.def and cdocsamp.tex with
% cdocsch1.tex, cdocsch2.tex, cdocsdrf.tex, cdocsfn1.tex, cdocsfn2.tex.
%
%<package>\ifdefined\childdocmain\endinput\fi
%<package>\ProvidesFile{childdoc.def}[2018/12/30 v2.0 child document driver]
%<samplemain>\ProvidesFile{cdocsamp.tex}[2018/12/30 v2.0 sample for childdoc]
%<*driver>
%\ProvidesFile{childdoc.drv}[2018/12/30 v2.0 childdoc reference manual file]
\PassOptionsToClass{10pt,a4paper}{article}
\documentclass{ltxdoc}

\usepackage[margin=35mm]{geometry}
\usepackage{hyperref}
\usepackage{hyperxmp}
\usepackage[usenames]{color}

\hypersetup{colorlinks=true}
\hypersetup{pdfstartview=FitH}
\hypersetup{pdfpagemode=UseNone}
\hypersetup{pdfsource={}}
\hypersetup{pdflang={en-UK}}
\hypersetup{pdfcopyright={Copyright 2017-2018 Niklas Beisert.
  This work may be distributed and/or modified under the
  conditions of the LaTeX Project Public License, either version 1.3
  of this license or (at your option) any later version.}}
\hypersetup{pdflicenseurl={http://www.latex-project.org/lppl.txt}}
\hypersetup{pdfcontactaddress={ETH Zurich, ITP, HIT K,
  Wolfgang-Pauli-Strasse 27}}
\hypersetup{pdfcontactpostcode={8093}}
\hypersetup{pdfcontactcity={Zurich}}
\hypersetup{pdfcontactcountry={Switzerland}}
\hypersetup{pdfcontactemail={nbeisert@itp.phys.ethz.ch}}
\hypersetup{pdfcontacturl={http://people.phys.ethz.ch/\xmptilde nbeisert/}}

\newcommand{\secref}[1]{\hyperref[#1]{section \ref*{#1}}}

\parskip1ex
\parindent0pt
\let\olditemize\itemize
\def\itemize{\olditemize\parskip0pt}

\begin{document}

\title{The \textsf{childdoc} Package}
\hypersetup{pdftitle={The childdoc Package}}
\author{Niklas Beisert\\[2ex]
  Institut f\"ur Theoretische Physik\\
  Eidgen\"ossische Technische Hochschule Z\"urich\\
  Wolfgang-Pauli-Strasse 27, 8093 Z\"urich, Switzerland\\[1ex]
  \href{mailto:nbeisert@itp.phys.ethz.ch}
  {\texttt{nbeisert@itp.phys.ethz.ch}}}
\hypersetup{pdfauthor={Niklas Beisert}}
\hypersetup{pdfsubject={Manual for the LaTeX2e Package childdoc}}
\date{30 December 2018, \textsf{v2.0}}
\maketitle

\begin{abstract}\noindent
\textsf{childdoc} is a \LaTeXe{} package
that enables the direct compilation
of document sections included by |\include|
to individual files.
\end{abstract}

\begingroup
\parskip0ex
\tableofcontents
\endgroup

%%%%%%%%%%%%%%%%%%%%%%%%%%%%%%%%%%%%%%%%%%%%%%%%%%%%%%%%%%%%%%%%%%%%%%%%%%%%%%%%
%%%%%%%%%%%%%%%%%%%%%%%%%%%%%%%%%%%%%%%%%%%%%%%%%%%%%%%%%%%%%%%%%%%%%%%%%%%%%%%%
\section{Introduction}

\LaTeX{} provides a mechanism to structure a large document (such as a book)
into a main file and several child files (containing the chapters)
using the |\include| command.
This mechanism is beneficial for documents
which span hundreds of pages in order to
make the source file(s) more manageable.
Moreover, compilation can be restricted to
selected child files by means of the |\includeonly| command.
The latter feature can be used to reduce the compilation time while editing
(this was significantly more useful in the earlier days of \LaTeX{})
or to generate a smaller document which is easier to navigate.
Another application of |\includeonly| is to generate
documents consisting of selected parts of the complete document.

However, there are a few drawbacks of the plain |\include| mechanism:
\begin{itemize}
\item
The child files cannot be compiled on their own,
they can only be compiled via the main file.
A naive editing environment
(such as a text editor with an option
to have the current file processed by \LaTeX)
may require one to switch to the main file before compiling;
attempting to compile the child file produces errors.
\item
The main file must be modified (each time)
to adjust the |\includeonly| command
to the present needs. This easily leaves the main file in a messy state.
\item
The generated document will always carry the filename
of the main document. This is inconvenient if
several child files are to be compiled and
to be kept for distribution.
\end{itemize}

The present package provides a simple interface
to make child files individually compilable by \LaTeX{}.
Compiling a child file then has the same effect as compiling
the main file with an |\includeonly| command
to select the appropriate child.
Moreover the generated document will carry the name of the child
rather than the main file.
This resolves all three above issues.

This feature is meant to make the editing of books,
thesis documents and lecture notes somewhat more convenient.
However, the package can also be used efficiently for
composing a series of documents (such as exercise sheets)
which are typically distributed individually.
It then assists the author in generating the individual documents
(potentially in different versions)
as well as a document containing the collected series.
Another application is in developing style files
or other kinds of included material
where compilation of the style file could redirect
to a sample or test file.

%%%%%%%%%%%%%%%%%%%%%%%%%%%%%%%%%%%%%%%%%%%%%%%%%%%%%%%%%%%%%%%%%%%%%%%%%%%%%%%%
%%%%%%%%%%%%%%%%%%%%%%%%%%%%%%%%%%%%%%%%%%%%%%%%%%%%%%%%%%%%%%%%%%%%%%%%%%%%%%%%
\section{Usage}

First of all, the package \textsf{childdoc} is \emph{not} a standard
\LaTeXe{} |.sty| style file! Therefore it needs to be invoked in
a non-standard way.

%%%%%%%%%%%%%%%%%%%%%%%%%%%%%%%%%%%%%%%%%%%%%%%%%%%%%%%%%%%%%%%%%%%%%%%%%%%%%%%%
\subsection{Included Files}
\label{sec:include}

%%%%%%%%%%%%%%%%%%%%%%%%%%%%%%%%%%%%%%%%
\DescribeMacro{\childdocmain}
To use the package, add the commands
\begin{center}
\begin{tabular}{l}
|\input{childdoc.def}|\\
|\childdocmain{}|\\
\end{tabular}
\end{center}
at the very top of the main \LaTeX{} file,
in particular \emph{before} the |\documentclass| statement!
The argument of |\childdocmain| should be left empty
(but it must be present).

%%%%%%%%%%%%%%%%%%%%%%%%%%%%%%%%%%%%%%%%
\DescribeMacro{\childdocof}
Furthermore, add the commands
\begin{center}
\begin{tabular}{l}
|\input{childdoc.def}|\\
|\childdocof{|\textit{main}|}|\\
\end{tabular}
\end{center}
at the top of every child file \textit{child}
which is included by |\include{|\textit{child}|}|
from within the main file
(or at least for those files to be compiled individually).
The argument \textit{main} must be the filename of the main file.

There are a couple of
considerations in setting up the main and child documents:

%%%%%%%%%%%%%%%%%%%%%%%%%%%%%%%%%%%%%%%%
\paragraph{Restrictions.}

Please note the following restrictions:
\begin{itemize}
\item
|\childdocmain| must be called with one argument \textit{main}
to ensure compatibility with earlier version of the package.
It must either be empty (|\childdocmain{}|)
or precisely match the filename of the main file in which it is specified.
See \secref{sec:detection} for further information.
\item
The filename \textit{main} must be specified without the |.tex| extension.
\item
The filename \textit{main} is case sensitive
(even in case-insensitive file systems)
due to internal string comparison.
\item
The argument \textit{main} should be fully expanded, it cannot be a macro.
\item
Subdirectories and special characters should be avoided in filenames.
\item
The command |\childdocmain{|\textit{main}|}| must be followed by a whitespace.
It should not be followed immediately by another command
or by a comment mark `|%|'.
This is because the \TeX{} parser reads the token immediately following
the argument of |\childdocmain| and puts it
at the beginning of every child section;
however, a white\-space is ignored.
\end{itemize}

%%%%%%%%%%%%%%%%%%%%%%%%%%%%%%%%%%%%%%%%
\paragraph{Content of Main File.}

It is advisable to place all content in the child files included by |\include|.
Any output contained in the main file will appear in all child documents
unless suppressed manually;
it cannot be suppressed automatically by the |\includeonly| directive
and thus should normally be avoided.
A method to include some content in the main file
by means of conditional processing is described in \secref{sec:conditional}.

%%%%%%%%%%%%%%%%%%%%%%%%%%%%%%%%%%%%%%%%
\paragraph{Page Numbering.}

When only a part of the document is compiled,
the appropriate numbering of pages
(as well as other status parameters)
is determined from the |.aux| files.
The latter contain information from previous passes.
However this information needs to propagate through
all intermediate child documents.
Therefore the page numbering in child documents may well
be inconsistent until the complete document is compiled at least once.

A useful (if unconventional) way to always ensure a consistent
page numbering is to restart the numbering in each child document
and denote the pages by `\textit{child}|.|\textit{page}'
where \textit{child} represents the chapter/section number of the child file.
This can be achieved by the command
|\numberwithin{page}{|\textit{child}|}|
of the \textsf{amsmath} package
where \textit{child} can be |chapter| or |section|
depending on the chosen structuring.
Alternatively, one can modify the macro |\thepage| appropriately
and reset the counter |page| at the start of each child file.

%%%%%%%%%%%%%%%%%%%%%%%%%%%%%%%%%%%%%%%%%%%%%%%%%%%%%%%%%%%%%%%%%%%%%%%%%%%%%%%%
\subsection{Conditional Processing}
\label{sec:conditional}

The package provides a mechanism to compile different versions
of a document. To customise the versions further some conditional processing
can come in handy to distinguish which version is being compiled.
The package provides two macros to describe the compilation context:

%%%%%%%%%%%%%%%%%%%%%%%%%%%%%%%%%%%%%%%%
\DescribeMacro{\ifchilddoc}
The conditional |\ifchilddoc| distinguishes between the compilation of
child documents and the main document:
%
\begin{center}
|\ifchilddoc |\textit{child-code}| |[|\||else |\textit{main-code}]| \||fi|
\end{center}

%%%%%%%%%%%%%%%%%%%%%%%%%%%%%%%%%%%%%%%%
\DescribeMacro{\childdocname}
\DescribeMacro{\childdocjob}
The macro |\childdocname| contains the filename (without extension)
of the main or child file being processed.
Note that |\childdocjob| will always contain the name of the main file.

%%%%%%%%%%%%%%%%%%%%%%%%%%%%%%%%%%%%%%%%
\paragraph{Title Page.}

Conditional processing can be used to include a title or banner page
in the main document when proper precautions are taken.
Importantly, the code in the main file should ensure that the page counter
(as well as other status parameters which are stored in the |.aux| files)
takes the same value after the conditional processing.
Otherwise the page numbers may take divergent values
depending on which part is compiled.

For example, a title page could be declared by:
%
\begin{center}
\begin{tabular}{l}
|\ifchilddoc\||else|\\
|\addtocounter{page}{-1}|\\
\textit{code for title page}\\
|\newpage|\\
|\||fi|
\end{tabular}
\end{center}
%
A banner page for the child documents can be generated by:
%
\begin{center}
\begin{tabular}{l}
|\ifchilddoc|\\
|\addtocounter{page}{-1}|\\
\textit{code for banner page}\\
|\newpage|\\
|\||fi|
\end{tabular}
\end{center}
%
Here one could write a message such as:
\begin{center}
|This is the part \childdocname{} of \childdocjob{}.|
\end{center}

%%%%%%%%%%%%%%%%%%%%%%%%%%%%%%%%%%%%%%%%%%%%%%%%%%%%%%%%%%%%%%%%%%%%%%%%%%%%%%%%
\subsection{Flags}
\label{sec:flags}

The package makes it easy to generate different versions
of the main or child documents.
To this end compilation flags can be defined
and assigned different default values.
They will be particularly useful in conjunction
with the forwarding mechanism described in \secref{sec:forward}.

For example, it may be useful to have a flag |\version|
which can be set to |draft| or |final|.
The document source will contain some conditional code
depending on the value of |\version|.
Suppose further, the flag should default to |final| for the main file
and to |draft| for child files
which is a natural assignment for editing the document.
This is achieved by placing the following code
in the preamble of the main document
(below the |\childdocmain| directive):
%
\begin{center}
\begin{tabular}{l}
|\ifchilddoc|\\
|\providecommand{\version}{draft}|\\
|\||else|\\
|\providecommand{\version}{final}|\\
|\||fi|
\end{tabular}
\end{center}
%
The definition by |\providecommand| makes sure
that previous definitions are not overwritten.
Further statements |\providecommand{\version}{...}|
can thus be added before the above code to override it.

For the main file, one might add a line
(between |\childdocmain| and the above block)
%
\begin{center}
|%\ifchilddoc\||else\providecommand{\version}{draft}\||fi|
\end{center}
%
which can be uncommented to produce a draft version.
Likewise one can add a line to the very top of a child file
(above the |\childdocof{|\textit{main}|}| directive)
%
\begin{center}
|%\providecommand{\version}{final}|
\end{center}
%
which can be uncommented to produce the final version of this child document.

%%%%%%%%%%%%%%%%%%%%%%%%%%%%%%%%%%%%%%%%%%%%%%%%%%%%%%%%%%%%%%%%%%%%%%%%%%%%%%%%
\subsection{Forwarding}
\label{sec:forward}

Different versions of the main or child documents
using compilation flags as described in \secref{sec:flags}
can be (permanently) stored in different files
for convenient compilation, viewing and distribution.
To this end, the package defines a command
to pass on compilation to a different file:

%%%%%%%%%%%%%%%%%%%%%%%%%%%%%%%%%%%%%%%%
\DescribeMacro{\childdocforward}
The command |\childdocforward| redirects processing to
another source file:
%
\begin{center}
\begin{tabular}{l}
|\input{childdoc.def}|\\
|\childdocforward[|\textit{main}|]{|\textit{dest}|}|\\
\end{tabular}
\end{center}
%
The argument \textit{dest} is the destination file
(without extension).
It should be the main file or one of the child files.
Note that further \textsf{childdoc} directives
such as |\childdocof| and |\childdocforward|
in the indicated file will be processed in this form.
The optional argument \textit{main}
passes on directly to the main file \textit{main}
while pretending to compile the child \textit{dest}.
This form behaves as if \textit{dest}
issues |\childdocof{|\textit{main}|}| right away,
and no further \textsf{childdoc} directives will be processed.

%%%%%%%%%%%%%%%%%%%%%%%%%%%%%%%%%%%%%%%%
\DescribeMacro{\...prefix}
In the alternative form |\childdocforwardprefix|,
%
\begin{center}
\begin{tabular}{l}
|\input{childdoc.def}|\\
|\childdocforwardprefix[|\textit{main}|]{|\textit{prefix}|}{|\textit{dest}|}|
\end{tabular}
\end{center}
%
the destination file is determined by a pattern
depending on the current file:
To make this work, the current file must be called
`{\textit{prefix}\hspace{0.2em}\textit{suffix}}'
with \textit{prefix} matching precisely the argument.
Processing is then passed on to the file
`{\textit{dest}\hspace{0.2em}\textit{suffix}}'.
Surely, the same effect is achieved by
directly specifying the
argument `{\textit{dest}\hspace{0.2em}\textit{suffix}}'
in the first form.
However, that requires to set up a different file
for each child. With the alternative form of the command
all these files can have exactly the same content
which simplifies setting them up and maintaining them.

For example, the following file |draft.tex|
with a compilation flag |\version| as described in \secref{sec:flags}
compiles the main document as a draft:
%
\begin{center}
\begin{tabular}{l}
|\def\version{draft}|\\
|\input{childdoc.def}|\\
|\childdocforward{|\textit{main}|}|
\end{tabular}
\end{center}
%
Likewise, the following files |final|\textit{nn}|.tex|
compile the final version of the child document
|child|\textit{nn}|.tex|:
%
\begin{center}
\begin{tabular}{l}
|\def\version{final}|\\
|\input{childdoc.def}|\\
|\childdocforwardprefix{final}{child}|
\end{tabular}
\end{center}
%

Note that when several versions of a main file and/or of each child file
are to be generated, it may be convenient to set up a |Makefile| or
shell script to automatise the process.

%%%%%%%%%%%%%%%%%%%%%%%%%%%%%%%%%%%%%%%%%%%%%%%%%%%%%%%%%%%%%%%%%%%%%%%%%%%%%%%%
\subsection{Command Line Processing}
\label{sec:commandline}

The effect of redirection files can also be achieved by invoking
the \LaTeX{} compiler with a more elaborate command line.
Most conveniently this should be done as part
of a shell script or a |Makefile|.

When using \textsf{childdoc} in the main file, the following
command lines effectively perform a redirection
(note that depending on the shell being used,
backslashes may have to be doubled: `|\|' $\to$ `|\\|'):
%
\begin{center}
|... -jobname "|\textit{target}|" |\\|"|[\textit{flags}]%
|\input{childdoc.def}\childdocforward[|\textit{main}|]{|\textit{dest}|}"|
\end{center}
%
Here \textit{target} is the name of the output file,
\textit{main} is the name of the main file
and \textit{dest} is the name of the main or child file to be processed
(all filenames without extensions).
The optional argument \textit{main} can be omitted
if \textit{main} matches \textit{dest}.
Optionally, compilation \textit{flags} can be defined via |\def| commands.
This command line makes the \TeX{} engine believe
it is compiling the file \textit{target}
whose content is specified as the latter parameter.
The provided code then forwards the processing to
\textit{main} or \textit{dest} as described in \secref{sec:forward}.

%%%%%%%%%%%%%%%%%%%%%%%%%%%%%%%%%%%%%%%%%%%%%%%%%%%%%%%%%%%%%%%%%%%%%%%%%%%%%%%%
\subsection{Include by Input}
\label{sec:input}

Including child documents by |\include| has some restrictions by design.
Most notably, the content of a child document always occupies
its own set of pages; pages cannot be shared between child documents.
Usually, this behaviour makes perfect sense
because each child document contain an essential part of the document.
However, in some situations it may be desirable to compose
a document from a collection of parts
without having mandatory page breaks between then.
For this case, the package
provides a mechanism to include parts
by |\input| which can also be processed individually.
However, by construction this mechanism
requires manual handling of the content to be output.

%%%%%%%%%%%%%%%%%%%%%%%%%%%%%%%%%%%%%%%%
\DescribeMacro{\ifchilddocmanual}
The main file should be prepared as usual, see \secref{sec:include}.
However, the document body must make a distinction
between processing of an individual part and of the main document, e.g.:
%
\begin{center}
\begin{tabular}{l}
|\ifchilddocmanual|\\
|\input{\childdocname}|\\
|\||else|\\
\textit{document body with }|\input{|\textit{part}|}|\\
|\||fi|
\end{tabular}
\end{center}
%
The conditional |\ifchilddocmanual| is true whenever
a part to be included by |\input| is being compiled,
and the name of the part is stored in |\childdocname|.

%%%%%%%%%%%%%%%%%%%%%%%%%%%%%%%%%%%%%%%%
\DescribeMacro{\childdocby}
Each part to be included by |\input| should start with:
%
\begin{center}
\begin{tabular}{l}
|\input{childdoc.def}|\\
|\childdocby{|\textit{main}|}|\\
\end{tabular}
\end{center}
%
The directive |\childdocby| is similar to |\childdocof|
described in \secref{sec:include},
but the subsequent selection of content must be done manually.
To that end, both |\ifchilddoc| and |\ifchilddocmanual|
will be true upon processing of a part,
and the name of the part is stored in |\childdocname|.
Note that |\jobname| will be set to the filename of the current part
so that each part receives an individual |.aux| file
that does not interfere with the |.aux| file(s) of the main document.
This behaviour can be altered by the alternative form
|\childdocby[*]{|\textit{main}|}| (with a non-empty optional argument)
which uses the |.aux| file of the main document
by setting |\jobname| to \textit{main}.

%%%%%%%%%%%%%%%%%%%%%%%%%%%%%%%%%%%%%%%%%%%%%%%%%%%%%%%%%%%%%%%%%%%%%%%%%%%%%%%%
\subsection{Driver Development}
\label{sec:driver}

The \textsf{childdoc} mechanism can also be use for the development
of definition files such as \LaTeX{} styles or classes.
This case differs from the above setup with multiple parts
included by |\include| in that no |\includeonly| should be invoked.
This can be achieved by starting the include file
(before |\ProvidesPackage|) with:
%
\begin{center}
\begin{tabular}{l}
|\input{childdoc.def}|\\
|\childdocforward{|\textit{main}|}|\\
\end{tabular}
\end{center}
%
or alternatively with:
%
\begin{center}
\begin{tabular}{l}
|\input{childdoc.def}|\\
|\childdocby{|\textit{main}|}|\\
\end{tabular}
\end{center}
%
Both forms have slightly different effects as described above.
The main file is prepared as usual, see \secref{sec:include}.

%%%%%%%%%%%%%%%%%%%%%%%%%%%%%%%%%%%%%%%%%%%%%%%%%%%%%%%%%%%%%%%%%%%%%%%%%%%%%%%%
\subsection{Legacy Detection}
\label{sec:detection}

The directive |\childdocmain| in the main file can detect
whether the complete document or merely a child is to be compiled
even without using the directive |\childdocof|.
This method is deprecated because it is less robust
and there is no compelling reason to use it;
it is merely provided for backward compatibility
and it may be removed in future versions.

If the detection mechanism is to be used,
it is mandatory to correctly specify
the filename of the main file as the argument of |\childdocmain|:
%
\begin{center}
\begin{tabular}{l}
|\input{childdoc.def}|\\
|\childdocmain{|\textit{main}|}|\\
\end{tabular}
\end{center}
%
If |\jobname| does not match the argument \textit{main} of |\childdocmain|,
it is assumed that |\jobname| points to the child file to be compiled.
When using |\childdocmain| with the main file specified as argument,
it suffices to start a child file
with just |\input{|\textit{main}|}|
without loading of the package and using |\childdocof|.
If instead all processing is done
with the appropriate \textsf{childdoc} directives,
the argument of \textit{main} of |\childdocmain| can be empty.

An alternative version of the command line processing described
in \secref{sec:commandline} using the detection mechanism reads:
%
\begin{center}
|... -jobname "|\textit{target}|" "|[\textit{flags}]%
[|\def\jobname{|\textit{dest}|}|]|\input{|\textit{main}|}"|
\end{center}

%%%%%%%%%%%%%%%%%%%%%%%%%%%%%%%%%%%%%%%%%%%%%%%%%%%%%%%%%%%%%%%%%%%%%%%%%%%%%%%%
\subsection{Manual Code}
\label{sec:manual}

In case one cannot be certain whether the definitions file |childdoc.def|
is installed on the target \TeX{} distribution
and one prefers not to ship it,
it is conceivable to paste a few relevant commands into the sources.

To that end, drop all statements |\input{childdoc.def}|
and perform the replacements as outlined below.
Instead of |\childdocmain{|\textit{main}|}| add the following code
to the top of the main file:
%
\begin{center}
\begin{tabular}{l}
|\||ifdefined\childdocname\endinput\||fi\newif\ifchilddoc|\\
|\edef\childdocname{\scantokens\expandafter{\jobname\noexpand}}|\\
|\def\childdocmain{|\textit{main}|}\||ifx\childdocmain\childdocname\||else|\\
|\childdoctrue\includeonly{\childdocname}\let\jobname\childdocmain\||fi|\\
\end{tabular}
\end{center}
%
Instead of |\childdocof{|\textit{main}|}| just include the main file
at the top of each child file:
%
\begin{center}
|\input{|\textit{main}|}|
\end{center}
%
A simple redirection |\childdocforward{|\textit{dest}|}| is achieved by:
%
\begin{center}
|\def\jobname{|\textit{dest}|}\input{\jobname}|
\end{center}
%
The redirection with prefix
|\childdocforwardprefix[|\textit{prefix}|]{|\textit{dest}|}|
is accomplished by:
%
\begin{center}
\begin{tabular}{l}
|{\edef\jobname{\scantokens\expandafter{\jobname\noexpand}}|\\
|\def\redirectjob |\textit{prefix}|#1~~~{\gdef\jobname{|\textit{dest}|#1}}|\\
|\expandafter\redirectjob\jobname~~~}\input{\jobname}|
\end{tabular}
\end{center}

In an alternative approach,
child documents can be compiled by a specific command line
without additional code or specific definitions:
%
\begin{center}
|... -jobname "|\textit{target}|" "|[\textit{flags}]%
|\includeonly{|\textit{dest}|}\input{|\textit{main}|}"|
\end{center}
%

%%%%%%%%%%%%%%%%%%%%%%%%%%%%%%%%%%%%%%%%%%%%%%%%%%%%%%%%%%%%%%%%%%%%%%%%%%%%%%%%
%%%%%%%%%%%%%%%%%%%%%%%%%%%%%%%%%%%%%%%%%%%%%%%%%%%%%%%%%%%%%%%%%%%%%%%%%%%%%%%%
\section{Information}

%%%%%%%%%%%%%%%%%%%%%%%%%%%%%%%%%%%%%%%%%%%%%%%%%%%%%%%%%%%%%%%%%%%%%%%%%%%%%%%%
\subsection{Copyright}

Copyright \copyright{} 2017--2018 Niklas Beisert

This work may be distributed and/or modified under the
conditions of the \LaTeX{} Project Public License, either version 1.3
of this license or (at your option) any later version.
The latest version of this license is in
  \url{http://www.latex-project.org/lppl.txt}
and version 1.3 or later is part of all distributions of \LaTeX{}
version 2005/12/01 or later.

This work has the LPPL maintenance status `maintained'.

The Current Maintainer of this work is Niklas Beisert.

This work consists of the files |README.txt|, |childdoc.ins| and |childdoc.dtx|
as well as the derived files |childdoc.def|, |cdocsamp.tex|
with |cdocsch1.tex|, |cdocsch2.tex|, |cdocspt3.tex|, |cdocspt4.tex|,
|cdocsdrf.tex|, |cdocsfn1.tex|, |cdocsfn2.tex|
as well as |childdoc.pdf|.

%%%%%%%%%%%%%%%%%%%%%%%%%%%%%%%%%%%%%%%%%%%%%%%%%%%%%%%%%%%%%%%%%%%%%%%%%%%%%%%%
\subsection{Files and Installation}

The package consists of the files:
%
\begin{center}
\begin{tabular}{ll}
    |README.txt|   & readme file \\
    |childdoc.ins| & installation file \\
    |childdoc.dtx| & source file \\
    |childdoc.def| & definition file \\
    |cdocsamp.tex| & sample main file \\
    |cdocsch1.tex| & sample include file \\
    |cdocsch2.tex| & sample include file \\
    |cdocspt3.tex| & sample part file \\
    |cdocspt4.tex| & sample part file \\
    |cdocsdrf.tex| & sample redirection file \\
    |cdocsfn1.tex| & sample redirection file \\
    |cdocsfn2.tex| & sample redirection file \\
    |childdoc.pdf| & manual
\end{tabular}
\end{center}
%
The distribution consists of the files
|README.txt|, |childdoc.ins| and |childdoc.dtx|.
%
\begin{itemize}
\item
Run (pdf)\LaTeX{} on |childdoc.dtx|
to compile the manual |childdoc.pdf| (this file).
\item
Run \LaTeX{} on |childdoc.ins| to create the definitions file |childdoc.def|
and the sample |cdocsamp.tex| with include files
|cdocsch1.tex|, |cdocsch2.tex|, |cdocspt3.tex|, |cdocspt4.tex|,
|cdocsdrf.tex|, |cdocsfn1.tex|, |cdocsfn2.tex|.
Then copy the file |childdoc.def| to an appropriate directory of your \LaTeX{}
distribution, e.g.\ \textit{texmf-root}|/tex/latex/childdoc|.
\end{itemize}

%%%%%%%%%%%%%%%%%%%%%%%%%%%%%%%%%%%%%%%%%%%%%%%%%%%%%%%%%%%%%%%%%%%%%%%%%%%%%%%%
\subsection{Related CTAN Packages}

There are several other packages which offer a similar functionality:
%
\begin{itemize}
\item
The packages
\href{http://ctan.org/pkg/docmute}{\textsf{docmute}},
\href{http://ctan.org/pkg/includex}{\textsf{includex}} and
\href{http://ctan.org/pkg/standalone}{\textsf{standalone}}
provide commands to include only the document body of
a child file thus allowing both files to be compiled individually.
\item
The packages \href{http://ctan.org/pkg/subdocs}{\textsf{subdocs}}
and \href{http://ctan.org/pkg/subfiles}{\textsf{subfiles}}
provide structures in which the main and child documents can be
encapsulated and allowing them to be compiled individually.
The inclusion mechanism is different from the conventional |\include|.
\item
The package \href{http://ctan.org/pkg/combine}{\textsf{combine}}
is an elaborate solution to combine several documents into one.
\end{itemize}
%
See also the CTAN topic \href{http://ctan.org/topic/subdocs}{\textsf{subdocs}}
for further related packages.
The present package differs from the above solutions in that
a document structure constructed with the conventional |\include| mechanism
just needs two extra commands at the top of every file
such that all constituent files can be compiled individually.

%%%%%%%%%%%%%%%%%%%%%%%%%%%%%%%%%%%%%%%%%%%%%%%%%%%%%%%%%%%%%%%%%%%%%%%%%%%%%%%%
%\subsection{Feature Suggestions}
%
%The following is a list of features which may be useful for future
%versions of this package:
%%
%\begin{itemize}
%\item
%\ldots
%\end{itemize}

%%%%%%%%%%%%%%%%%%%%%%%%%%%%%%%%%%%%%%%%%%%%%%%%%%%%%%%%%%%%%%%%%%%%%%%%%%%%%%%%
\subsection{Revision History}

%%%%%%%%%%%%%%%%%%%%%%%%%%%%%%%%%%%%%%%%
\paragraph{v2.0:} 2018/12/30

\begin{itemize}
\item
immediate forward processing
\item
added |\childdocby| mechanism
\item
manual restructured
\end{itemize}

%%%%%%%%%%%%%%%%%%%%%%%%%%%%%%%%%%%%%%%%
\paragraph{v1.6:} 2018/01/17

\begin{itemize}
\item
application for development of include files
\item
corrections to manual
\end{itemize}

%%%%%%%%%%%%%%%%%%%%%%%%%%%%%%%%%%%%%%%%
\paragraph{v1.5:} 2017/05/21

\begin{itemize}
\item
more complete structuring introduced
\item
|\childdocof| introduced
\item
|\childdoc| renamed to |\childdocmain|
\item
|\childredirect| renamed to |\childdocforward| and |\childdocforwardprefix|
and functionality expanded
\end{itemize}

%%%%%%%%%%%%%%%%%%%%%%%%%%%%%%%%%%%%%%%%
\paragraph{v1.0:} 2017/04/27

\begin{itemize}
\item
manual and install package
\item
first version published on CTAN
\end{itemize}

%%%%%%%%%%%%%%%%%%%%%%%%%%%%%%%%%%%%%%%%
\paragraph{v0.6:} 2017/04/26

\begin{itemize}
\item
redirection mechanism added
\end{itemize}

%%%%%%%%%%%%%%%%%%%%%%%%%%%%%%%%%%%%%%%%
\paragraph{v0.5:} 2017/04/26

\begin{itemize}
\item
functionality in definition file
\end{itemize}


%%%%%%%%%%%%%%%%%%%%%%%%%%%%%%%%%%%%%%%%%%%%%%%%%%%%%%%%%%%%%%%%%%%%%%%%%%%%%%%%
%%%%%%%%%%%%%%%%%%%%%%%%%%%%%%%%%%%%%%%%%%%%%%%%%%%%%%%%%%%%%%%%%%%%%%%%%%%%%%%%
%%%%%%%%%%%%%%%%%%%%%%%%%%%%%%%%%%%%%%%%%%%%%%%%%%%%%%%%%%%%%%%%%%%%%%%%%%%%%%%%
\appendix

\settowidth\MacroIndent{\rmfamily\scriptsize 000\ }

 \DocInput{childdoc.dtx}

\end{document}
%</driver>
% \fi
%
% %%%%%%%%%%%%%%%%%%%%%%%%%%%%%%%%%%%%%%%%%%%%%%%%%%%%%%%%%%%%%%%%%%%%%%%%%%%%%%
% %%%%%%%%%%%%%%%%%%%%%%%%%%%%%%%%%%%%%%%%%%%%%%%%%%%%%%%%%%%%%%%%%%%%%%%%%%%%%%
% \section{Sample}
%\iffalse
%<*samplemain>
%\fi
%
% The following presents a sample document
% with two chapters, two parts, a title page,
% a compile flag as well as three forwarding files to set the flag.
% It consists of eight |.tex| files:
% \begin{center}
% \begin{tabular}{ll}
% |cdocsamp.tex|&main file\\
% |cdocsch1.tex|&include file for chapter 1\\
% |cdocsch2.tex|&include file for chapter 2\\
% |cdocspt3.tex|&include file for part 3\\
% |cdocspt4.tex|&include file for part 4\\
% |cdocsdrf.tex|&forwarding file for main file in draft mode\\
% |cdocsfi1.tex|&forwarding file for final version of chapter 1\\
% |cdocsfi2.tex|&forwarding file for final version of chapter 2\\
% \end{tabular}
% \end{center}
% Each of the eight files can be compiled directly by the \LaTeX{} compiler.
%
% %%%%%%%%%%%%%%%%%%%%%%%%%%%%%%%%%%%%%%
% \paragraph{Main File.}
%
% The main file is called |cdocsamp.tex|.
%
% Load the \textsf{childdoc} definitions and
% declare the filename for the main document:
%    \begin{macrocode}
\input{childdoc.def}
\childdocmain{}
%    \end{macrocode}

% Optional override for |\version| flag:
%    \begin{macrocode}
%%\ifchilddoc\else\providecommand{\version}{draft}\fi
%    \end{macrocode}

% Define the default values for the |\version| flag
% (|final| for the main file and |draft| for childs):
%    \begin{macrocode}
\ifchilddoc
\providecommand{\version}{draft}
\else
\providecommand{\version}{final}
\fi
%    \end{macrocode}

% Load the standard document class:
%    \begin{macrocode}
\documentclass[12pt]{article}
%    \end{macrocode}

% Start the document body:
%    \begin{macrocode}
\begin{document}
%    \end{macrocode}

% Declare a title page.
% Print title, part of document being processed and version flag:
%    \begin{macrocode}
\addtocounter{page}{-1}
\begin{center}
{\LARGE\bfseries{}childdoc example\par}
\vspace{1cm}
\ifchilddoc
\ifchilddocmanual part\else chapter\fi:
`\childdocname' of `\childdocjob'\par
\else
main document: `\childdocjob'\par
\fi
version: \version\par
\end{center}
\newpage
%    \end{macrocode}

% Manually include selected file,
% otherwise process as usual:
%    \begin{macrocode}
\ifchilddocmanual
\section*{part `\childdocname'}
\input{\childdocname}
\else
%    \end{macrocode}

% Include the two chapters:
%    \begin{macrocode}
\include{cdocsch1}
\include{cdocsch2}
%    \end{macrocode}

% Include the two parts unless only chapters should be displayed:
%    \begin{macrocode}
\ifchilddoc\else
\section{part three}
\input{cdocspt3}
\section{part four}
\input{cdocspt4}
\fi
%    \end{macrocode}

% Process as usual until here:
%    \begin{macrocode}
\fi
%    \end{macrocode}

% End of document body:
%    \begin{macrocode}
\end{document}
%    \end{macrocode}
%\iffalse
%</samplemain>
%\fi
%
% %%%%%%%%%%%%%%%%%%%%%%%%%%%%%%%%%%%%%%
% \paragraph{Chapter Include Files.}
%
% The include files are called |cdocsch1.tex| and |cdocsch2.tex|.
%
%\iffalse
%<*samplechap1|samplechap2>
%\fi

% Optional override for |\version| flag:
%    \begin{macrocode}
%%\providecommand{\version}{final}
%    \end{macrocode}

% Include the main document:
%    \begin{macrocode}
\input{childdoc.def}
\childdocof{cdocsamp}
%    \end{macrocode}

%\iffalse
%</samplechap1|samplechap2>
%\fi
%
%\iffalse
%<*samplechap1>
%\fi
% Some text for chapter 1:
%    \begin{macrocode}
\section{one}
some text in chapter one
%    \end{macrocode}

%\iffalse
%</samplechap1>
%\fi
% Some text for chapter 2:
%\iffalse
%<*samplechap2>
%\fi
%    \begin{macrocode}
\section{two}
more text in chapter two
%    \end{macrocode}

%\iffalse
%</samplechap2>
%\fi
%
% %%%%%%%%%%%%%%%%%%%%%%%%%%%%%%%%%%%%%%
% \paragraph{Part Include Files.}
%
% The include files are called |cdocspt3.tex| and |cdocspt4.tex|.
%
%\iffalse
%<*samplepart3|samplepart4>
%\fi

% Optional override for |\version| flag:
%    \begin{macrocode}
%%\providecommand{\version}{final}
%    \end{macrocode}

% Include the main document:
%    \begin{macrocode}
\input{childdoc.def}
\childdocby{cdocsamp}
%    \end{macrocode}

%\iffalse
%</samplepart3|samplepart4>
%\fi
%
%\iffalse
%<*samplepart3>
%\fi
% Some text for part 3:
%    \begin{macrocode}
some text in part three
%    \end{macrocode}

%\iffalse
%</samplepart3>
%\fi
% Some text for part 4:
%\iffalse
%<*samplepart4>
%\fi
%    \begin{macrocode}
more text in part four
%    \end{macrocode}

%\iffalse
%</samplepart4>
%\fi
%
% %%%%%%%%%%%%%%%%%%%%%%%%%%%%%%%%%%%%%%
% \paragraph{Forwarding for a Complete Draft.}
%
% The following forwarding file |cdocsdrf.tex|
% compiles the main document in draft mode:
%\iffalse
%<*sampledraft>
%\fi
%    \begin{macrocode}
\def\version{draft}
\input{childdoc.def}
\childdocforward{cdocsamp}
%    \end{macrocode}

%\iffalse
%</sampledraft>
%\fi
%
% %%%%%%%%%%%%%%%%%%%%%%%%%%%%%%%%%%%%%%
% \paragraph{Forwarding for Final Version of the Chapters.}
%
% The following forwarding files |cdocsfn1.tex| and |cdocsfn2.tex|
% (with identical content)
% compile the final versions of the child documents
% |cdocsch1.tex| and |cdocsch2.tex|, respectively:
%\iffalse
%<*samplefinal>
%\fi
%    \begin{macrocode}
\def\version{final}
\input{childdoc.def}
\childdocforwardprefix[cdocsamp]{cdocsfn}{cdocsch}
%    \end{macrocode}

%\iffalse
%</samplefinal>
%\fi
%
% %%%%%%%%%%%%%%%%%%%%%%%%%%%%%%%%%%%%%%
% \paragraph{Command Line Processing.}
%
% The following three command lines generate the output files
% |cdocscld|, |cdocscl1| and |cdocscl2|
% which should be identical to
% |cdocsdrf|, |cdocsch1| and |cdocsfn2|, respectively:
% \begin{center}
% \begin{tabular}{l}
% |latex -jobname cdocscld \|\\
% |  "\def\version{draft}\input{childdoc.def}\childdocforward{cdocsamp}"|\\
% |latex -jobname cdocscl1 \|\\
% |  "\input{childdoc.def}\childdocforward[cdocsamp]{cdocsch1}"|\\
% |latex -jobname cdocscl2 \|\\
% |  "\def\version{final}\input{childdoc.def}\childdocforward{cdocsch2}"|
% \end{tabular}
% \end{center}
% Note that the trailing backslash on each first line
% merely continues the input to the second line
% (for convenient cut ant paste).
% Furthermore, the command |latex| can be replaced by any
% of its alternative versions such as |pdflatex|.
%
% %%%%%%%%%%%%%%%%%%%%%%%%%%%%%%%%%%%%%%%%%%%%%%%%%%%%%%%%%%%%%%%%%%%%%%%%%%%%%%
% %%%%%%%%%%%%%%%%%%%%%%%%%%%%%%%%%%%%%%%%%%%%%%%%%%%%%%%%%%%%%%%%%%%%%%%%%%%%%%
% \section{Implementation}
%\iffalse
%<*package>
%\fi
%
% This section describes the definitions file |childdoc.def|.

% The definitions cannot be loaded using |\usepackage| or |\RequirePackage|
% which has a mechanism to prevent loading a style file more than once.
% When loading the definitions by means of |\input|
% multiple instances have to be prevented manually:
%\iffalse
%This code needs to be before the `\ProvidesFile' directive
%which is defined at the beginning of this file.
%Therefore it is also placed there and commented out here.
%</package>
%<*discard>
%\fi
%    \begin{macrocode}
\ifdefined\childdocmain\endinput\fi
%    \end{macrocode}
%\iffalse
%</discard>
%<*package>
%\fi
%
% \macro{\ifchilddoc}
% \macro{\ifchilddocmanual}
% The conditional |\ifchilddoc| tells whether a
% child (true) or main (false) document is being compiled.
% The conditional |\ifchilddocmanual| tells whether
% the |\includeonly| mechanism is used (false) or
% the selection of child files must be performed manually (true).
% The definitions initialise to false:
%    \begin{macrocode}
\newif\ifchilddoc
\newif\ifchilddocmanual
%    \end{macrocode}

% \macro{\childdocname}
% \macro{\childdocjob}
% The macro |\childdocname| stores the name of the main document
% to be compiled. The macro |\childdocjob| stores the name of
% the document on which the \LaTeX{} compiler was originally invoked.
% The content of |\jobname| cannot be compared
% to filenames specified in the source due to different catcodes.
% The following code rescans |\jobname|, stores the result
% in |\childdocname| and saves a copy in |\childdocjob|:
%    \begin{macrocode}
\edef\childdocname{\scantokens\expandafter{\jobname\noexpand}}
\let\childdocjob\childdocname
%    \end{macrocode}

% \macro{\childdocdisable}
% The macro |\childdocdisable| prevents the main file
% from being processed more than once.
% At this stage, the main document command |\childdocmain|
% is assumed to be called once again where it should do nothing.
% Any subsequent call to it should prevent
% a secondary processing of the main document
% It overwrites the forwarding commands
% |\childdocof| and |\childdocforward|
% with empty macros to prevent further inclusions of the main document:
%    \begin{macrocode}
\newcommand{\childdocdisable}
{
  \renewcommand{\childdocmain}[1]{\renewcommand{\childdocmain}[1]{\endinput}}
  \renewcommand{\childdocof}[1]{}
  \renewcommand{\childdocby}[2][]{}
  \renewcommand{\childdocforward}[2][]{}
  \renewcommand{\childdocdisable}{}
}
%    \end{macrocode}

% \macro{\childdocmain}
% The macro |\childdocmain| is to be called at the top of the main file
% with nothing or the main filename (without extension) as argument.
% First, it breaks loops.
% If the argument is not empty and does not match |\childdocname|
% (which is set by the first inclusion of |childdoc.def|),
% |\ifchilddoc| is set to true, |\includeonly| is applied to the child file
% and |\jobname| is set to the main file
% (for proper handling of |.aux| files):
%    \begin{macrocode}
\newcommand{\childdocmain}[1]
{
  \childdocdisable\childdocmain{}
  \if?#1?\else
    \begingroup
      \def\childdoctmp{#1}
      \ifx\childdoctmp\childdocname
        \def\childdoctmp{}
      \else
        \def\childdoctmp
        {
          \childdoctrue
          \includeonly{\childdocname}
          \def\childdocjob{#1}
          \def\jobname{#1}
        }
      \fi
      \expandafter
    \endgroup
    \childdoctmp
  \fi
}
%    \end{macrocode}

% \macro{\childdocof}
% The command |\childdocof| redirects
% compilation to the main file |#1|.
%    \begin{macrocode}
\newcommand{\childdocof}[1]
{
  \childdocdisable
  \childdoctrue
  \includeonly{\childdocname}
  \def\jobname{#1}
  \def\childdocjob{#1}
  \input{#1}
}
%    \end{macrocode}

% \macro{\childdocby}
% The command |\childdocby| ....
%    \begin{macrocode}
\newcommand{\childdocby}[2][]
{
  \childdocdisable
  \childdoctrue
  \childdocmanualtrue
  \if?#1?\else
    \def\jobname{#2}
  \fi
  \def\childdocjob{#2}
  \input{#2}
  \endinput
}
%    \end{macrocode}

% \macro{\childdocforward}
% The command |\childdocforward| redirects
% compilation to the main file or
% (if the optional argument is given) a child file.
% Parameters are set as if the main file
% or a child file starting with |\childdocof| was compiled.
% Then compilation is handed over to the main file:
%    \begin{macrocode}
\newcommand{\childdocforward}[2][]
{
  \begingroup
    \if?#1?
      \def\childdoctmp
      {
        \def\childdocname{#2}
        \def\childdocjob{#2}
        \def\jobname{#2}
        \input{#2}
        \endinput
      }
    \else
      \def\childdoctmp
      {
        \childdocdisable
        \def\childdocname{#2}
        \childdoctrue
        \includeonly{#2}
        \def\childdocjob{#1}
        \def\jobname{#1}
        \input{#1}
        \endinput
      }
    \fi
    \expandafter
  \endgroup
  \childdoctmp
}
%    \end{macrocode}

% \macro{\childdocforwardprefix}
% The command |\childdocforwardprefix| redirects
% compilation to the main or a child file by means of a pattern.
% The prefix |#1| in the current filename is replaced by |#2|
% and the suffix of the current filename is kept
% (it is assumed that the filename does not contain the substring `|~~~|'
% which is used as a delimiter).
% Compilation is handed over to the new file by |\childdocforward|:
%    \begin{macrocode}
\newcommand{\childdocforwardprefix}[3][]
{
  \begingroup
    \def\childdocextract #2##1~~~{\def\childdoctmp{\childdocforward[#1]{#3##1}}}
    \expandafter\childdocextract\childdocname~~~
    \expandafter
  \endgroup
  \childdoctmp
}
%    \end{macrocode}

% \macro{\childdoc}
% The deprecated macro |\childdoc| is a legacy version of |\childdocmain|:
%    \begin{macrocode}
\newcommand{\childdoc}{\childdocmain}
%    \end{macrocode}

% \macro{\childdocredirect}
% The deprecated macro |\childdocredirect| is a legacy version
% of |\childdocforward| and |\childdocforwardprefix|:
%    \begin{macrocode}
\newcommand{\childdocredirect}[2][]
{
  \begingroup
    \if?#1?
      \def\childdoctmp{\childdocforward{#2}}
    \else
      \def\childdoctmp{\childdocforwardprefix{#1}{#2}}
    \fi
    \expandafter
  \endgroup
  \childdoctmp
}
%    \end{macrocode}

%\iffalse
%</package>
%\fi
%
\endinput
\childdocforward{cdocsch2}"|
% \end{tabular}
% \end{center}
% Note that the trailing backslash on each first line
% merely continues the input to the second line
% (for convenient cut ant paste).
% Furthermore, the command |latex| can be replaced by any
% of its alternative versions such as |pdflatex|.
%
% %%%%%%%%%%%%%%%%%%%%%%%%%%%%%%%%%%%%%%%%%%%%%%%%%%%%%%%%%%%%%%%%%%%%%%%%%%%%%%
% %%%%%%%%%%%%%%%%%%%%%%%%%%%%%%%%%%%%%%%%%%%%%%%%%%%%%%%%%%%%%%%%%%%%%%%%%%%%%%
% \section{Implementation}
%\iffalse
%<*package>
%\fi
%
% This section describes the definitions file |childdoc.def|.

% The definitions cannot be loaded using |\usepackage| or |\RequirePackage|
% which has a mechanism to prevent loading a style file more than once.
% When loading the definitions by means of |\input|
% multiple instances have to be prevented manually:
%\iffalse
%This code needs to be before the `\ProvidesFile' directive
%which is defined at the beginning of this file.
%Therefore it is also placed there and commented out here.
%</package>
%<*discard>
%\fi
%    \begin{macrocode}
\ifdefined\childdocmain\endinput\fi
%    \end{macrocode}
%\iffalse
%</discard>
%<*package>
%\fi
%
% \macro{\ifchilddoc}
% \macro{\ifchilddocmanual}
% The conditional |\ifchilddoc| tells whether a
% child (true) or main (false) document is being compiled.
% The conditional |\ifchilddocmanual| tells whether
% the |\includeonly| mechanism is used (false) or
% the selection of child files must be performed manually (true).
% The definitions initialise to false:
%    \begin{macrocode}
\newif\ifchilddoc
\newif\ifchilddocmanual
%    \end{macrocode}

% \macro{\childdocname}
% \macro{\childdocjob}
% The macro |\childdocname| stores the name of the main document
% to be compiled. The macro |\childdocjob| stores the name of
% the document on which the \LaTeX{} compiler was originally invoked.
% The content of |\jobname| cannot be compared
% to filenames specified in the source due to different catcodes.
% The following code rescans |\jobname|, stores the result
% in |\childdocname| and saves a copy in |\childdocjob|:
%    \begin{macrocode}
\edef\childdocname{\scantokens\expandafter{\jobname\noexpand}}
\let\childdocjob\childdocname
%    \end{macrocode}

% \macro{\childdocdisable}
% The macro |\childdocdisable| prevents the main file
% from being processed more than once.
% At this stage, the main document command |\childdocmain|
% is assumed to be called once again where it should do nothing.
% Any subsequent call to it should prevent
% a secondary processing of the main document
% It overwrites the forwarding commands
% |\childdocof| and |\childdocforward|
% with empty macros to prevent further inclusions of the main document:
%    \begin{macrocode}
\newcommand{\childdocdisable}
{
  \renewcommand{\childdocmain}[1]{\renewcommand{\childdocmain}[1]{\endinput}}
  \renewcommand{\childdocof}[1]{}
  \renewcommand{\childdocby}[2][]{}
  \renewcommand{\childdocforward}[2][]{}
  \renewcommand{\childdocdisable}{}
}
%    \end{macrocode}

% \macro{\childdocmain}
% The macro |\childdocmain| is to be called at the top of the main file
% with nothing or the main filename (without extension) as argument.
% First, it breaks loops.
% If the argument is not empty and does not match |\childdocname|
% (which is set by the first inclusion of |childdoc.def|),
% |\ifchilddoc| is set to true, |\includeonly| is applied to the child file
% and |\jobname| is set to the main file
% (for proper handling of |.aux| files):
%    \begin{macrocode}
\newcommand{\childdocmain}[1]
{
  \childdocdisable\childdocmain{}
  \if?#1?\else
    \begingroup
      \def\childdoctmp{#1}
      \ifx\childdoctmp\childdocname
        \def\childdoctmp{}
      \else
        \def\childdoctmp
        {
          \childdoctrue
          \includeonly{\childdocname}
          \def\childdocjob{#1}
          \def\jobname{#1}
        }
      \fi
      \expandafter
    \endgroup
    \childdoctmp
  \fi
}
%    \end{macrocode}

% \macro{\childdocof}
% The command |\childdocof| redirects
% compilation to the main file |#1|.
%    \begin{macrocode}
\newcommand{\childdocof}[1]
{
  \childdocdisable
  \childdoctrue
  \includeonly{\childdocname}
  \def\jobname{#1}
  \def\childdocjob{#1}
  \input{#1}
}
%    \end{macrocode}

% \macro{\childdocby}
% The command |\childdocby| ....
%    \begin{macrocode}
\newcommand{\childdocby}[2][]
{
  \childdocdisable
  \childdoctrue
  \childdocmanualtrue
  \if?#1?\else
    \def\jobname{#2}
  \fi
  \def\childdocjob{#2}
  \input{#2}
  \endinput
}
%    \end{macrocode}

% \macro{\childdocforward}
% The command |\childdocforward| redirects
% compilation to the main file or
% (if the optional argument is given) a child file.
% Parameters are set as if the main file
% or a child file starting with |\childdocof| was compiled.
% Then compilation is handed over to the main file:
%    \begin{macrocode}
\newcommand{\childdocforward}[2][]
{
  \begingroup
    \if?#1?
      \def\childdoctmp
      {
        \def\childdocname{#2}
        \def\childdocjob{#2}
        \def\jobname{#2}
        \input{#2}
        \endinput
      }
    \else
      \def\childdoctmp
      {
        \childdocdisable
        \def\childdocname{#2}
        \childdoctrue
        \includeonly{#2}
        \def\childdocjob{#1}
        \def\jobname{#1}
        \input{#1}
        \endinput
      }
    \fi
    \expandafter
  \endgroup
  \childdoctmp
}
%    \end{macrocode}

% \macro{\childdocforwardprefix}
% The command |\childdocforwardprefix| redirects
% compilation to the main or a child file by means of a pattern.
% The prefix |#1| in the current filename is replaced by |#2|
% and the suffix of the current filename is kept
% (it is assumed that the filename does not contain the substring `|~~~|'
% which is used as a delimiter).
% Compilation is handed over to the new file by |\childdocforward|:
%    \begin{macrocode}
\newcommand{\childdocforwardprefix}[3][]
{
  \begingroup
    \def\childdocextract #2##1~~~{\def\childdoctmp{\childdocforward[#1]{#3##1}}}
    \expandafter\childdocextract\childdocname~~~
    \expandafter
  \endgroup
  \childdoctmp
}
%    \end{macrocode}

% \macro{\childdoc}
% The deprecated macro |\childdoc| is a legacy version of |\childdocmain|:
%    \begin{macrocode}
\newcommand{\childdoc}{\childdocmain}
%    \end{macrocode}

% \macro{\childdocredirect}
% The deprecated macro |\childdocredirect| is a legacy version
% of |\childdocforward| and |\childdocforwardprefix|:
%    \begin{macrocode}
\newcommand{\childdocredirect}[2][]
{
  \begingroup
    \if?#1?
      \def\childdoctmp{\childdocforward{#2}}
    \else
      \def\childdoctmp{\childdocforwardprefix{#1}{#2}}
    \fi
    \expandafter
  \endgroup
  \childdoctmp
}
%    \end{macrocode}

%\iffalse
%</package>
%\fi
%
\endinput

\childdocforwardprefix[cdocsamp]{cdocsfn}{cdocsch}
%    \end{macrocode}

%\iffalse
%</samplefinal>
%\fi
%
% %%%%%%%%%%%%%%%%%%%%%%%%%%%%%%%%%%%%%%
% \paragraph{Command Line Processing.}
%
% The following three command lines generate the output files
% |cdocscld|, |cdocscl1| and |cdocscl2|
% which should be identical to
% |cdocsdrf|, |cdocsch1| and |cdocsfn2|, respectively:
% \begin{center}
% \begin{tabular}{l}
% |latex -jobname cdocscld \|\\
% |  "\def\version{draft}% \iffalse
%
% childdoc.dtx Copyright (C) 2017-2018 Niklas Beisert
%
% This work may be distributed and/or modified under the
% conditions of the LaTeX Project Public License, either version 1.3
% of this license or (at your option) any later version.
% The latest version of this license is in
%   http://www.latex-project.org/lppl.txt
% and version 1.3 or later is part of all distributions of LaTeX
% version 2005/12/01 or later.
%
% This work has the LPPL maintenance status `maintained'.
%
% The Current Maintainer of this work is Niklas Beisert.
%
% This work consists of the files childdoc.dtx and childdoc.ins
% and the derived files childdoc.def and cdocsamp.tex with
% cdocsch1.tex, cdocsch2.tex, cdocsdrf.tex, cdocsfn1.tex, cdocsfn2.tex.
%
%<package>\ifdefined\childdocmain\endinput\fi
%<package>\ProvidesFile{childdoc.def}[2018/12/30 v2.0 child document driver]
%<samplemain>\ProvidesFile{cdocsamp.tex}[2018/12/30 v2.0 sample for childdoc]
%<*driver>
%\ProvidesFile{childdoc.drv}[2018/12/30 v2.0 childdoc reference manual file]
\PassOptionsToClass{10pt,a4paper}{article}
\documentclass{ltxdoc}

\usepackage[margin=35mm]{geometry}
\usepackage{hyperref}
\usepackage{hyperxmp}
\usepackage[usenames]{color}

\hypersetup{colorlinks=true}
\hypersetup{pdfstartview=FitH}
\hypersetup{pdfpagemode=UseNone}
\hypersetup{pdfsource={}}
\hypersetup{pdflang={en-UK}}
\hypersetup{pdfcopyright={Copyright 2017-2018 Niklas Beisert.
  This work may be distributed and/or modified under the
  conditions of the LaTeX Project Public License, either version 1.3
  of this license or (at your option) any later version.}}
\hypersetup{pdflicenseurl={http://www.latex-project.org/lppl.txt}}
\hypersetup{pdfcontactaddress={ETH Zurich, ITP, HIT K,
  Wolfgang-Pauli-Strasse 27}}
\hypersetup{pdfcontactpostcode={8093}}
\hypersetup{pdfcontactcity={Zurich}}
\hypersetup{pdfcontactcountry={Switzerland}}
\hypersetup{pdfcontactemail={nbeisert@itp.phys.ethz.ch}}
\hypersetup{pdfcontacturl={http://people.phys.ethz.ch/\xmptilde nbeisert/}}

\newcommand{\secref}[1]{\hyperref[#1]{section \ref*{#1}}}

\parskip1ex
\parindent0pt
\let\olditemize\itemize
\def\itemize{\olditemize\parskip0pt}

\begin{document}

\title{The \textsf{childdoc} Package}
\hypersetup{pdftitle={The childdoc Package}}
\author{Niklas Beisert\\[2ex]
  Institut f\"ur Theoretische Physik\\
  Eidgen\"ossische Technische Hochschule Z\"urich\\
  Wolfgang-Pauli-Strasse 27, 8093 Z\"urich, Switzerland\\[1ex]
  \href{mailto:nbeisert@itp.phys.ethz.ch}
  {\texttt{nbeisert@itp.phys.ethz.ch}}}
\hypersetup{pdfauthor={Niklas Beisert}}
\hypersetup{pdfsubject={Manual for the LaTeX2e Package childdoc}}
\date{30 December 2018, \textsf{v2.0}}
\maketitle

\begin{abstract}\noindent
\textsf{childdoc} is a \LaTeXe{} package
that enables the direct compilation
of document sections included by |\include|
to individual files.
\end{abstract}

\begingroup
\parskip0ex
\tableofcontents
\endgroup

%%%%%%%%%%%%%%%%%%%%%%%%%%%%%%%%%%%%%%%%%%%%%%%%%%%%%%%%%%%%%%%%%%%%%%%%%%%%%%%%
%%%%%%%%%%%%%%%%%%%%%%%%%%%%%%%%%%%%%%%%%%%%%%%%%%%%%%%%%%%%%%%%%%%%%%%%%%%%%%%%
\section{Introduction}

\LaTeX{} provides a mechanism to structure a large document (such as a book)
into a main file and several child files (containing the chapters)
using the |\include| command.
This mechanism is beneficial for documents
which span hundreds of pages in order to
make the source file(s) more manageable.
Moreover, compilation can be restricted to
selected child files by means of the |\includeonly| command.
The latter feature can be used to reduce the compilation time while editing
(this was significantly more useful in the earlier days of \LaTeX{})
or to generate a smaller document which is easier to navigate.
Another application of |\includeonly| is to generate
documents consisting of selected parts of the complete document.

However, there are a few drawbacks of the plain |\include| mechanism:
\begin{itemize}
\item
The child files cannot be compiled on their own,
they can only be compiled via the main file.
A naive editing environment
(such as a text editor with an option
to have the current file processed by \LaTeX)
may require one to switch to the main file before compiling;
attempting to compile the child file produces errors.
\item
The main file must be modified (each time)
to adjust the |\includeonly| command
to the present needs. This easily leaves the main file in a messy state.
\item
The generated document will always carry the filename
of the main document. This is inconvenient if
several child files are to be compiled and
to be kept for distribution.
\end{itemize}

The present package provides a simple interface
to make child files individually compilable by \LaTeX{}.
Compiling a child file then has the same effect as compiling
the main file with an |\includeonly| command
to select the appropriate child.
Moreover the generated document will carry the name of the child
rather than the main file.
This resolves all three above issues.

This feature is meant to make the editing of books,
thesis documents and lecture notes somewhat more convenient.
However, the package can also be used efficiently for
composing a series of documents (such as exercise sheets)
which are typically distributed individually.
It then assists the author in generating the individual documents
(potentially in different versions)
as well as a document containing the collected series.
Another application is in developing style files
or other kinds of included material
where compilation of the style file could redirect
to a sample or test file.

%%%%%%%%%%%%%%%%%%%%%%%%%%%%%%%%%%%%%%%%%%%%%%%%%%%%%%%%%%%%%%%%%%%%%%%%%%%%%%%%
%%%%%%%%%%%%%%%%%%%%%%%%%%%%%%%%%%%%%%%%%%%%%%%%%%%%%%%%%%%%%%%%%%%%%%%%%%%%%%%%
\section{Usage}

First of all, the package \textsf{childdoc} is \emph{not} a standard
\LaTeXe{} |.sty| style file! Therefore it needs to be invoked in
a non-standard way.

%%%%%%%%%%%%%%%%%%%%%%%%%%%%%%%%%%%%%%%%%%%%%%%%%%%%%%%%%%%%%%%%%%%%%%%%%%%%%%%%
\subsection{Included Files}
\label{sec:include}

%%%%%%%%%%%%%%%%%%%%%%%%%%%%%%%%%%%%%%%%
\DescribeMacro{\childdocmain}
To use the package, add the commands
\begin{center}
\begin{tabular}{l}
|% \iffalse
%
% childdoc.dtx Copyright (C) 2017-2018 Niklas Beisert
%
% This work may be distributed and/or modified under the
% conditions of the LaTeX Project Public License, either version 1.3
% of this license or (at your option) any later version.
% The latest version of this license is in
%   http://www.latex-project.org/lppl.txt
% and version 1.3 or later is part of all distributions of LaTeX
% version 2005/12/01 or later.
%
% This work has the LPPL maintenance status `maintained'.
%
% The Current Maintainer of this work is Niklas Beisert.
%
% This work consists of the files childdoc.dtx and childdoc.ins
% and the derived files childdoc.def and cdocsamp.tex with
% cdocsch1.tex, cdocsch2.tex, cdocsdrf.tex, cdocsfn1.tex, cdocsfn2.tex.
%
%<package>\ifdefined\childdocmain\endinput\fi
%<package>\ProvidesFile{childdoc.def}[2018/12/30 v2.0 child document driver]
%<samplemain>\ProvidesFile{cdocsamp.tex}[2018/12/30 v2.0 sample for childdoc]
%<*driver>
%\ProvidesFile{childdoc.drv}[2018/12/30 v2.0 childdoc reference manual file]
\PassOptionsToClass{10pt,a4paper}{article}
\documentclass{ltxdoc}

\usepackage[margin=35mm]{geometry}
\usepackage{hyperref}
\usepackage{hyperxmp}
\usepackage[usenames]{color}

\hypersetup{colorlinks=true}
\hypersetup{pdfstartview=FitH}
\hypersetup{pdfpagemode=UseNone}
\hypersetup{pdfsource={}}
\hypersetup{pdflang={en-UK}}
\hypersetup{pdfcopyright={Copyright 2017-2018 Niklas Beisert.
  This work may be distributed and/or modified under the
  conditions of the LaTeX Project Public License, either version 1.3
  of this license or (at your option) any later version.}}
\hypersetup{pdflicenseurl={http://www.latex-project.org/lppl.txt}}
\hypersetup{pdfcontactaddress={ETH Zurich, ITP, HIT K,
  Wolfgang-Pauli-Strasse 27}}
\hypersetup{pdfcontactpostcode={8093}}
\hypersetup{pdfcontactcity={Zurich}}
\hypersetup{pdfcontactcountry={Switzerland}}
\hypersetup{pdfcontactemail={nbeisert@itp.phys.ethz.ch}}
\hypersetup{pdfcontacturl={http://people.phys.ethz.ch/\xmptilde nbeisert/}}

\newcommand{\secref}[1]{\hyperref[#1]{section \ref*{#1}}}

\parskip1ex
\parindent0pt
\let\olditemize\itemize
\def\itemize{\olditemize\parskip0pt}

\begin{document}

\title{The \textsf{childdoc} Package}
\hypersetup{pdftitle={The childdoc Package}}
\author{Niklas Beisert\\[2ex]
  Institut f\"ur Theoretische Physik\\
  Eidgen\"ossische Technische Hochschule Z\"urich\\
  Wolfgang-Pauli-Strasse 27, 8093 Z\"urich, Switzerland\\[1ex]
  \href{mailto:nbeisert@itp.phys.ethz.ch}
  {\texttt{nbeisert@itp.phys.ethz.ch}}}
\hypersetup{pdfauthor={Niklas Beisert}}
\hypersetup{pdfsubject={Manual for the LaTeX2e Package childdoc}}
\date{30 December 2018, \textsf{v2.0}}
\maketitle

\begin{abstract}\noindent
\textsf{childdoc} is a \LaTeXe{} package
that enables the direct compilation
of document sections included by |\include|
to individual files.
\end{abstract}

\begingroup
\parskip0ex
\tableofcontents
\endgroup

%%%%%%%%%%%%%%%%%%%%%%%%%%%%%%%%%%%%%%%%%%%%%%%%%%%%%%%%%%%%%%%%%%%%%%%%%%%%%%%%
%%%%%%%%%%%%%%%%%%%%%%%%%%%%%%%%%%%%%%%%%%%%%%%%%%%%%%%%%%%%%%%%%%%%%%%%%%%%%%%%
\section{Introduction}

\LaTeX{} provides a mechanism to structure a large document (such as a book)
into a main file and several child files (containing the chapters)
using the |\include| command.
This mechanism is beneficial for documents
which span hundreds of pages in order to
make the source file(s) more manageable.
Moreover, compilation can be restricted to
selected child files by means of the |\includeonly| command.
The latter feature can be used to reduce the compilation time while editing
(this was significantly more useful in the earlier days of \LaTeX{})
or to generate a smaller document which is easier to navigate.
Another application of |\includeonly| is to generate
documents consisting of selected parts of the complete document.

However, there are a few drawbacks of the plain |\include| mechanism:
\begin{itemize}
\item
The child files cannot be compiled on their own,
they can only be compiled via the main file.
A naive editing environment
(such as a text editor with an option
to have the current file processed by \LaTeX)
may require one to switch to the main file before compiling;
attempting to compile the child file produces errors.
\item
The main file must be modified (each time)
to adjust the |\includeonly| command
to the present needs. This easily leaves the main file in a messy state.
\item
The generated document will always carry the filename
of the main document. This is inconvenient if
several child files are to be compiled and
to be kept for distribution.
\end{itemize}

The present package provides a simple interface
to make child files individually compilable by \LaTeX{}.
Compiling a child file then has the same effect as compiling
the main file with an |\includeonly| command
to select the appropriate child.
Moreover the generated document will carry the name of the child
rather than the main file.
This resolves all three above issues.

This feature is meant to make the editing of books,
thesis documents and lecture notes somewhat more convenient.
However, the package can also be used efficiently for
composing a series of documents (such as exercise sheets)
which are typically distributed individually.
It then assists the author in generating the individual documents
(potentially in different versions)
as well as a document containing the collected series.
Another application is in developing style files
or other kinds of included material
where compilation of the style file could redirect
to a sample or test file.

%%%%%%%%%%%%%%%%%%%%%%%%%%%%%%%%%%%%%%%%%%%%%%%%%%%%%%%%%%%%%%%%%%%%%%%%%%%%%%%%
%%%%%%%%%%%%%%%%%%%%%%%%%%%%%%%%%%%%%%%%%%%%%%%%%%%%%%%%%%%%%%%%%%%%%%%%%%%%%%%%
\section{Usage}

First of all, the package \textsf{childdoc} is \emph{not} a standard
\LaTeXe{} |.sty| style file! Therefore it needs to be invoked in
a non-standard way.

%%%%%%%%%%%%%%%%%%%%%%%%%%%%%%%%%%%%%%%%%%%%%%%%%%%%%%%%%%%%%%%%%%%%%%%%%%%%%%%%
\subsection{Included Files}
\label{sec:include}

%%%%%%%%%%%%%%%%%%%%%%%%%%%%%%%%%%%%%%%%
\DescribeMacro{\childdocmain}
To use the package, add the commands
\begin{center}
\begin{tabular}{l}
|\input{childdoc.def}|\\
|\childdocmain{}|\\
\end{tabular}
\end{center}
at the very top of the main \LaTeX{} file,
in particular \emph{before} the |\documentclass| statement!
The argument of |\childdocmain| should be left empty
(but it must be present).

%%%%%%%%%%%%%%%%%%%%%%%%%%%%%%%%%%%%%%%%
\DescribeMacro{\childdocof}
Furthermore, add the commands
\begin{center}
\begin{tabular}{l}
|\input{childdoc.def}|\\
|\childdocof{|\textit{main}|}|\\
\end{tabular}
\end{center}
at the top of every child file \textit{child}
which is included by |\include{|\textit{child}|}|
from within the main file
(or at least for those files to be compiled individually).
The argument \textit{main} must be the filename of the main file.

There are a couple of
considerations in setting up the main and child documents:

%%%%%%%%%%%%%%%%%%%%%%%%%%%%%%%%%%%%%%%%
\paragraph{Restrictions.}

Please note the following restrictions:
\begin{itemize}
\item
|\childdocmain| must be called with one argument \textit{main}
to ensure compatibility with earlier version of the package.
It must either be empty (|\childdocmain{}|)
or precisely match the filename of the main file in which it is specified.
See \secref{sec:detection} for further information.
\item
The filename \textit{main} must be specified without the |.tex| extension.
\item
The filename \textit{main} is case sensitive
(even in case-insensitive file systems)
due to internal string comparison.
\item
The argument \textit{main} should be fully expanded, it cannot be a macro.
\item
Subdirectories and special characters should be avoided in filenames.
\item
The command |\childdocmain{|\textit{main}|}| must be followed by a whitespace.
It should not be followed immediately by another command
or by a comment mark `|%|'.
This is because the \TeX{} parser reads the token immediately following
the argument of |\childdocmain| and puts it
at the beginning of every child section;
however, a white\-space is ignored.
\end{itemize}

%%%%%%%%%%%%%%%%%%%%%%%%%%%%%%%%%%%%%%%%
\paragraph{Content of Main File.}

It is advisable to place all content in the child files included by |\include|.
Any output contained in the main file will appear in all child documents
unless suppressed manually;
it cannot be suppressed automatically by the |\includeonly| directive
and thus should normally be avoided.
A method to include some content in the main file
by means of conditional processing is described in \secref{sec:conditional}.

%%%%%%%%%%%%%%%%%%%%%%%%%%%%%%%%%%%%%%%%
\paragraph{Page Numbering.}

When only a part of the document is compiled,
the appropriate numbering of pages
(as well as other status parameters)
is determined from the |.aux| files.
The latter contain information from previous passes.
However this information needs to propagate through
all intermediate child documents.
Therefore the page numbering in child documents may well
be inconsistent until the complete document is compiled at least once.

A useful (if unconventional) way to always ensure a consistent
page numbering is to restart the numbering in each child document
and denote the pages by `\textit{child}|.|\textit{page}'
where \textit{child} represents the chapter/section number of the child file.
This can be achieved by the command
|\numberwithin{page}{|\textit{child}|}|
of the \textsf{amsmath} package
where \textit{child} can be |chapter| or |section|
depending on the chosen structuring.
Alternatively, one can modify the macro |\thepage| appropriately
and reset the counter |page| at the start of each child file.

%%%%%%%%%%%%%%%%%%%%%%%%%%%%%%%%%%%%%%%%%%%%%%%%%%%%%%%%%%%%%%%%%%%%%%%%%%%%%%%%
\subsection{Conditional Processing}
\label{sec:conditional}

The package provides a mechanism to compile different versions
of a document. To customise the versions further some conditional processing
can come in handy to distinguish which version is being compiled.
The package provides two macros to describe the compilation context:

%%%%%%%%%%%%%%%%%%%%%%%%%%%%%%%%%%%%%%%%
\DescribeMacro{\ifchilddoc}
The conditional |\ifchilddoc| distinguishes between the compilation of
child documents and the main document:
%
\begin{center}
|\ifchilddoc |\textit{child-code}| |[|\||else |\textit{main-code}]| \||fi|
\end{center}

%%%%%%%%%%%%%%%%%%%%%%%%%%%%%%%%%%%%%%%%
\DescribeMacro{\childdocname}
\DescribeMacro{\childdocjob}
The macro |\childdocname| contains the filename (without extension)
of the main or child file being processed.
Note that |\childdocjob| will always contain the name of the main file.

%%%%%%%%%%%%%%%%%%%%%%%%%%%%%%%%%%%%%%%%
\paragraph{Title Page.}

Conditional processing can be used to include a title or banner page
in the main document when proper precautions are taken.
Importantly, the code in the main file should ensure that the page counter
(as well as other status parameters which are stored in the |.aux| files)
takes the same value after the conditional processing.
Otherwise the page numbers may take divergent values
depending on which part is compiled.

For example, a title page could be declared by:
%
\begin{center}
\begin{tabular}{l}
|\ifchilddoc\||else|\\
|\addtocounter{page}{-1}|\\
\textit{code for title page}\\
|\newpage|\\
|\||fi|
\end{tabular}
\end{center}
%
A banner page for the child documents can be generated by:
%
\begin{center}
\begin{tabular}{l}
|\ifchilddoc|\\
|\addtocounter{page}{-1}|\\
\textit{code for banner page}\\
|\newpage|\\
|\||fi|
\end{tabular}
\end{center}
%
Here one could write a message such as:
\begin{center}
|This is the part \childdocname{} of \childdocjob{}.|
\end{center}

%%%%%%%%%%%%%%%%%%%%%%%%%%%%%%%%%%%%%%%%%%%%%%%%%%%%%%%%%%%%%%%%%%%%%%%%%%%%%%%%
\subsection{Flags}
\label{sec:flags}

The package makes it easy to generate different versions
of the main or child documents.
To this end compilation flags can be defined
and assigned different default values.
They will be particularly useful in conjunction
with the forwarding mechanism described in \secref{sec:forward}.

For example, it may be useful to have a flag |\version|
which can be set to |draft| or |final|.
The document source will contain some conditional code
depending on the value of |\version|.
Suppose further, the flag should default to |final| for the main file
and to |draft| for child files
which is a natural assignment for editing the document.
This is achieved by placing the following code
in the preamble of the main document
(below the |\childdocmain| directive):
%
\begin{center}
\begin{tabular}{l}
|\ifchilddoc|\\
|\providecommand{\version}{draft}|\\
|\||else|\\
|\providecommand{\version}{final}|\\
|\||fi|
\end{tabular}
\end{center}
%
The definition by |\providecommand| makes sure
that previous definitions are not overwritten.
Further statements |\providecommand{\version}{...}|
can thus be added before the above code to override it.

For the main file, one might add a line
(between |\childdocmain| and the above block)
%
\begin{center}
|%\ifchilddoc\||else\providecommand{\version}{draft}\||fi|
\end{center}
%
which can be uncommented to produce a draft version.
Likewise one can add a line to the very top of a child file
(above the |\childdocof{|\textit{main}|}| directive)
%
\begin{center}
|%\providecommand{\version}{final}|
\end{center}
%
which can be uncommented to produce the final version of this child document.

%%%%%%%%%%%%%%%%%%%%%%%%%%%%%%%%%%%%%%%%%%%%%%%%%%%%%%%%%%%%%%%%%%%%%%%%%%%%%%%%
\subsection{Forwarding}
\label{sec:forward}

Different versions of the main or child documents
using compilation flags as described in \secref{sec:flags}
can be (permanently) stored in different files
for convenient compilation, viewing and distribution.
To this end, the package defines a command
to pass on compilation to a different file:

%%%%%%%%%%%%%%%%%%%%%%%%%%%%%%%%%%%%%%%%
\DescribeMacro{\childdocforward}
The command |\childdocforward| redirects processing to
another source file:
%
\begin{center}
\begin{tabular}{l}
|\input{childdoc.def}|\\
|\childdocforward[|\textit{main}|]{|\textit{dest}|}|\\
\end{tabular}
\end{center}
%
The argument \textit{dest} is the destination file
(without extension).
It should be the main file or one of the child files.
Note that further \textsf{childdoc} directives
such as |\childdocof| and |\childdocforward|
in the indicated file will be processed in this form.
The optional argument \textit{main}
passes on directly to the main file \textit{main}
while pretending to compile the child \textit{dest}.
This form behaves as if \textit{dest}
issues |\childdocof{|\textit{main}|}| right away,
and no further \textsf{childdoc} directives will be processed.

%%%%%%%%%%%%%%%%%%%%%%%%%%%%%%%%%%%%%%%%
\DescribeMacro{\...prefix}
In the alternative form |\childdocforwardprefix|,
%
\begin{center}
\begin{tabular}{l}
|\input{childdoc.def}|\\
|\childdocforwardprefix[|\textit{main}|]{|\textit{prefix}|}{|\textit{dest}|}|
\end{tabular}
\end{center}
%
the destination file is determined by a pattern
depending on the current file:
To make this work, the current file must be called
`{\textit{prefix}\hspace{0.2em}\textit{suffix}}'
with \textit{prefix} matching precisely the argument.
Processing is then passed on to the file
`{\textit{dest}\hspace{0.2em}\textit{suffix}}'.
Surely, the same effect is achieved by
directly specifying the
argument `{\textit{dest}\hspace{0.2em}\textit{suffix}}'
in the first form.
However, that requires to set up a different file
for each child. With the alternative form of the command
all these files can have exactly the same content
which simplifies setting them up and maintaining them.

For example, the following file |draft.tex|
with a compilation flag |\version| as described in \secref{sec:flags}
compiles the main document as a draft:
%
\begin{center}
\begin{tabular}{l}
|\def\version{draft}|\\
|\input{childdoc.def}|\\
|\childdocforward{|\textit{main}|}|
\end{tabular}
\end{center}
%
Likewise, the following files |final|\textit{nn}|.tex|
compile the final version of the child document
|child|\textit{nn}|.tex|:
%
\begin{center}
\begin{tabular}{l}
|\def\version{final}|\\
|\input{childdoc.def}|\\
|\childdocforwardprefix{final}{child}|
\end{tabular}
\end{center}
%

Note that when several versions of a main file and/or of each child file
are to be generated, it may be convenient to set up a |Makefile| or
shell script to automatise the process.

%%%%%%%%%%%%%%%%%%%%%%%%%%%%%%%%%%%%%%%%%%%%%%%%%%%%%%%%%%%%%%%%%%%%%%%%%%%%%%%%
\subsection{Command Line Processing}
\label{sec:commandline}

The effect of redirection files can also be achieved by invoking
the \LaTeX{} compiler with a more elaborate command line.
Most conveniently this should be done as part
of a shell script or a |Makefile|.

When using \textsf{childdoc} in the main file, the following
command lines effectively perform a redirection
(note that depending on the shell being used,
backslashes may have to be doubled: `|\|' $\to$ `|\\|'):
%
\begin{center}
|... -jobname "|\textit{target}|" |\\|"|[\textit{flags}]%
|\input{childdoc.def}\childdocforward[|\textit{main}|]{|\textit{dest}|}"|
\end{center}
%
Here \textit{target} is the name of the output file,
\textit{main} is the name of the main file
and \textit{dest} is the name of the main or child file to be processed
(all filenames without extensions).
The optional argument \textit{main} can be omitted
if \textit{main} matches \textit{dest}.
Optionally, compilation \textit{flags} can be defined via |\def| commands.
This command line makes the \TeX{} engine believe
it is compiling the file \textit{target}
whose content is specified as the latter parameter.
The provided code then forwards the processing to
\textit{main} or \textit{dest} as described in \secref{sec:forward}.

%%%%%%%%%%%%%%%%%%%%%%%%%%%%%%%%%%%%%%%%%%%%%%%%%%%%%%%%%%%%%%%%%%%%%%%%%%%%%%%%
\subsection{Include by Input}
\label{sec:input}

Including child documents by |\include| has some restrictions by design.
Most notably, the content of a child document always occupies
its own set of pages; pages cannot be shared between child documents.
Usually, this behaviour makes perfect sense
because each child document contain an essential part of the document.
However, in some situations it may be desirable to compose
a document from a collection of parts
without having mandatory page breaks between then.
For this case, the package
provides a mechanism to include parts
by |\input| which can also be processed individually.
However, by construction this mechanism
requires manual handling of the content to be output.

%%%%%%%%%%%%%%%%%%%%%%%%%%%%%%%%%%%%%%%%
\DescribeMacro{\ifchilddocmanual}
The main file should be prepared as usual, see \secref{sec:include}.
However, the document body must make a distinction
between processing of an individual part and of the main document, e.g.:
%
\begin{center}
\begin{tabular}{l}
|\ifchilddocmanual|\\
|\input{\childdocname}|\\
|\||else|\\
\textit{document body with }|\input{|\textit{part}|}|\\
|\||fi|
\end{tabular}
\end{center}
%
The conditional |\ifchilddocmanual| is true whenever
a part to be included by |\input| is being compiled,
and the name of the part is stored in |\childdocname|.

%%%%%%%%%%%%%%%%%%%%%%%%%%%%%%%%%%%%%%%%
\DescribeMacro{\childdocby}
Each part to be included by |\input| should start with:
%
\begin{center}
\begin{tabular}{l}
|\input{childdoc.def}|\\
|\childdocby{|\textit{main}|}|\\
\end{tabular}
\end{center}
%
The directive |\childdocby| is similar to |\childdocof|
described in \secref{sec:include},
but the subsequent selection of content must be done manually.
To that end, both |\ifchilddoc| and |\ifchilddocmanual|
will be true upon processing of a part,
and the name of the part is stored in |\childdocname|.
Note that |\jobname| will be set to the filename of the current part
so that each part receives an individual |.aux| file
that does not interfere with the |.aux| file(s) of the main document.
This behaviour can be altered by the alternative form
|\childdocby[*]{|\textit{main}|}| (with a non-empty optional argument)
which uses the |.aux| file of the main document
by setting |\jobname| to \textit{main}.

%%%%%%%%%%%%%%%%%%%%%%%%%%%%%%%%%%%%%%%%%%%%%%%%%%%%%%%%%%%%%%%%%%%%%%%%%%%%%%%%
\subsection{Driver Development}
\label{sec:driver}

The \textsf{childdoc} mechanism can also be use for the development
of definition files such as \LaTeX{} styles or classes.
This case differs from the above setup with multiple parts
included by |\include| in that no |\includeonly| should be invoked.
This can be achieved by starting the include file
(before |\ProvidesPackage|) with:
%
\begin{center}
\begin{tabular}{l}
|\input{childdoc.def}|\\
|\childdocforward{|\textit{main}|}|\\
\end{tabular}
\end{center}
%
or alternatively with:
%
\begin{center}
\begin{tabular}{l}
|\input{childdoc.def}|\\
|\childdocby{|\textit{main}|}|\\
\end{tabular}
\end{center}
%
Both forms have slightly different effects as described above.
The main file is prepared as usual, see \secref{sec:include}.

%%%%%%%%%%%%%%%%%%%%%%%%%%%%%%%%%%%%%%%%%%%%%%%%%%%%%%%%%%%%%%%%%%%%%%%%%%%%%%%%
\subsection{Legacy Detection}
\label{sec:detection}

The directive |\childdocmain| in the main file can detect
whether the complete document or merely a child is to be compiled
even without using the directive |\childdocof|.
This method is deprecated because it is less robust
and there is no compelling reason to use it;
it is merely provided for backward compatibility
and it may be removed in future versions.

If the detection mechanism is to be used,
it is mandatory to correctly specify
the filename of the main file as the argument of |\childdocmain|:
%
\begin{center}
\begin{tabular}{l}
|\input{childdoc.def}|\\
|\childdocmain{|\textit{main}|}|\\
\end{tabular}
\end{center}
%
If |\jobname| does not match the argument \textit{main} of |\childdocmain|,
it is assumed that |\jobname| points to the child file to be compiled.
When using |\childdocmain| with the main file specified as argument,
it suffices to start a child file
with just |\input{|\textit{main}|}|
without loading of the package and using |\childdocof|.
If instead all processing is done
with the appropriate \textsf{childdoc} directives,
the argument of \textit{main} of |\childdocmain| can be empty.

An alternative version of the command line processing described
in \secref{sec:commandline} using the detection mechanism reads:
%
\begin{center}
|... -jobname "|\textit{target}|" "|[\textit{flags}]%
[|\def\jobname{|\textit{dest}|}|]|\input{|\textit{main}|}"|
\end{center}

%%%%%%%%%%%%%%%%%%%%%%%%%%%%%%%%%%%%%%%%%%%%%%%%%%%%%%%%%%%%%%%%%%%%%%%%%%%%%%%%
\subsection{Manual Code}
\label{sec:manual}

In case one cannot be certain whether the definitions file |childdoc.def|
is installed on the target \TeX{} distribution
and one prefers not to ship it,
it is conceivable to paste a few relevant commands into the sources.

To that end, drop all statements |\input{childdoc.def}|
and perform the replacements as outlined below.
Instead of |\childdocmain{|\textit{main}|}| add the following code
to the top of the main file:
%
\begin{center}
\begin{tabular}{l}
|\||ifdefined\childdocname\endinput\||fi\newif\ifchilddoc|\\
|\edef\childdocname{\scantokens\expandafter{\jobname\noexpand}}|\\
|\def\childdocmain{|\textit{main}|}\||ifx\childdocmain\childdocname\||else|\\
|\childdoctrue\includeonly{\childdocname}\let\jobname\childdocmain\||fi|\\
\end{tabular}
\end{center}
%
Instead of |\childdocof{|\textit{main}|}| just include the main file
at the top of each child file:
%
\begin{center}
|\input{|\textit{main}|}|
\end{center}
%
A simple redirection |\childdocforward{|\textit{dest}|}| is achieved by:
%
\begin{center}
|\def\jobname{|\textit{dest}|}\input{\jobname}|
\end{center}
%
The redirection with prefix
|\childdocforwardprefix[|\textit{prefix}|]{|\textit{dest}|}|
is accomplished by:
%
\begin{center}
\begin{tabular}{l}
|{\edef\jobname{\scantokens\expandafter{\jobname\noexpand}}|\\
|\def\redirectjob |\textit{prefix}|#1~~~{\gdef\jobname{|\textit{dest}|#1}}|\\
|\expandafter\redirectjob\jobname~~~}\input{\jobname}|
\end{tabular}
\end{center}

In an alternative approach,
child documents can be compiled by a specific command line
without additional code or specific definitions:
%
\begin{center}
|... -jobname "|\textit{target}|" "|[\textit{flags}]%
|\includeonly{|\textit{dest}|}\input{|\textit{main}|}"|
\end{center}
%

%%%%%%%%%%%%%%%%%%%%%%%%%%%%%%%%%%%%%%%%%%%%%%%%%%%%%%%%%%%%%%%%%%%%%%%%%%%%%%%%
%%%%%%%%%%%%%%%%%%%%%%%%%%%%%%%%%%%%%%%%%%%%%%%%%%%%%%%%%%%%%%%%%%%%%%%%%%%%%%%%
\section{Information}

%%%%%%%%%%%%%%%%%%%%%%%%%%%%%%%%%%%%%%%%%%%%%%%%%%%%%%%%%%%%%%%%%%%%%%%%%%%%%%%%
\subsection{Copyright}

Copyright \copyright{} 2017--2018 Niklas Beisert

This work may be distributed and/or modified under the
conditions of the \LaTeX{} Project Public License, either version 1.3
of this license or (at your option) any later version.
The latest version of this license is in
  \url{http://www.latex-project.org/lppl.txt}
and version 1.3 or later is part of all distributions of \LaTeX{}
version 2005/12/01 or later.

This work has the LPPL maintenance status `maintained'.

The Current Maintainer of this work is Niklas Beisert.

This work consists of the files |README.txt|, |childdoc.ins| and |childdoc.dtx|
as well as the derived files |childdoc.def|, |cdocsamp.tex|
with |cdocsch1.tex|, |cdocsch2.tex|, |cdocspt3.tex|, |cdocspt4.tex|,
|cdocsdrf.tex|, |cdocsfn1.tex|, |cdocsfn2.tex|
as well as |childdoc.pdf|.

%%%%%%%%%%%%%%%%%%%%%%%%%%%%%%%%%%%%%%%%%%%%%%%%%%%%%%%%%%%%%%%%%%%%%%%%%%%%%%%%
\subsection{Files and Installation}

The package consists of the files:
%
\begin{center}
\begin{tabular}{ll}
    |README.txt|   & readme file \\
    |childdoc.ins| & installation file \\
    |childdoc.dtx| & source file \\
    |childdoc.def| & definition file \\
    |cdocsamp.tex| & sample main file \\
    |cdocsch1.tex| & sample include file \\
    |cdocsch2.tex| & sample include file \\
    |cdocspt3.tex| & sample part file \\
    |cdocspt4.tex| & sample part file \\
    |cdocsdrf.tex| & sample redirection file \\
    |cdocsfn1.tex| & sample redirection file \\
    |cdocsfn2.tex| & sample redirection file \\
    |childdoc.pdf| & manual
\end{tabular}
\end{center}
%
The distribution consists of the files
|README.txt|, |childdoc.ins| and |childdoc.dtx|.
%
\begin{itemize}
\item
Run (pdf)\LaTeX{} on |childdoc.dtx|
to compile the manual |childdoc.pdf| (this file).
\item
Run \LaTeX{} on |childdoc.ins| to create the definitions file |childdoc.def|
and the sample |cdocsamp.tex| with include files
|cdocsch1.tex|, |cdocsch2.tex|, |cdocspt3.tex|, |cdocspt4.tex|,
|cdocsdrf.tex|, |cdocsfn1.tex|, |cdocsfn2.tex|.
Then copy the file |childdoc.def| to an appropriate directory of your \LaTeX{}
distribution, e.g.\ \textit{texmf-root}|/tex/latex/childdoc|.
\end{itemize}

%%%%%%%%%%%%%%%%%%%%%%%%%%%%%%%%%%%%%%%%%%%%%%%%%%%%%%%%%%%%%%%%%%%%%%%%%%%%%%%%
\subsection{Related CTAN Packages}

There are several other packages which offer a similar functionality:
%
\begin{itemize}
\item
The packages
\href{http://ctan.org/pkg/docmute}{\textsf{docmute}},
\href{http://ctan.org/pkg/includex}{\textsf{includex}} and
\href{http://ctan.org/pkg/standalone}{\textsf{standalone}}
provide commands to include only the document body of
a child file thus allowing both files to be compiled individually.
\item
The packages \href{http://ctan.org/pkg/subdocs}{\textsf{subdocs}}
and \href{http://ctan.org/pkg/subfiles}{\textsf{subfiles}}
provide structures in which the main and child documents can be
encapsulated and allowing them to be compiled individually.
The inclusion mechanism is different from the conventional |\include|.
\item
The package \href{http://ctan.org/pkg/combine}{\textsf{combine}}
is an elaborate solution to combine several documents into one.
\end{itemize}
%
See also the CTAN topic \href{http://ctan.org/topic/subdocs}{\textsf{subdocs}}
for further related packages.
The present package differs from the above solutions in that
a document structure constructed with the conventional |\include| mechanism
just needs two extra commands at the top of every file
such that all constituent files can be compiled individually.

%%%%%%%%%%%%%%%%%%%%%%%%%%%%%%%%%%%%%%%%%%%%%%%%%%%%%%%%%%%%%%%%%%%%%%%%%%%%%%%%
%\subsection{Feature Suggestions}
%
%The following is a list of features which may be useful for future
%versions of this package:
%%
%\begin{itemize}
%\item
%\ldots
%\end{itemize}

%%%%%%%%%%%%%%%%%%%%%%%%%%%%%%%%%%%%%%%%%%%%%%%%%%%%%%%%%%%%%%%%%%%%%%%%%%%%%%%%
\subsection{Revision History}

%%%%%%%%%%%%%%%%%%%%%%%%%%%%%%%%%%%%%%%%
\paragraph{v2.0:} 2018/12/30

\begin{itemize}
\item
immediate forward processing
\item
added |\childdocby| mechanism
\item
manual restructured
\end{itemize}

%%%%%%%%%%%%%%%%%%%%%%%%%%%%%%%%%%%%%%%%
\paragraph{v1.6:} 2018/01/17

\begin{itemize}
\item
application for development of include files
\item
corrections to manual
\end{itemize}

%%%%%%%%%%%%%%%%%%%%%%%%%%%%%%%%%%%%%%%%
\paragraph{v1.5:} 2017/05/21

\begin{itemize}
\item
more complete structuring introduced
\item
|\childdocof| introduced
\item
|\childdoc| renamed to |\childdocmain|
\item
|\childredirect| renamed to |\childdocforward| and |\childdocforwardprefix|
and functionality expanded
\end{itemize}

%%%%%%%%%%%%%%%%%%%%%%%%%%%%%%%%%%%%%%%%
\paragraph{v1.0:} 2017/04/27

\begin{itemize}
\item
manual and install package
\item
first version published on CTAN
\end{itemize}

%%%%%%%%%%%%%%%%%%%%%%%%%%%%%%%%%%%%%%%%
\paragraph{v0.6:} 2017/04/26

\begin{itemize}
\item
redirection mechanism added
\end{itemize}

%%%%%%%%%%%%%%%%%%%%%%%%%%%%%%%%%%%%%%%%
\paragraph{v0.5:} 2017/04/26

\begin{itemize}
\item
functionality in definition file
\end{itemize}


%%%%%%%%%%%%%%%%%%%%%%%%%%%%%%%%%%%%%%%%%%%%%%%%%%%%%%%%%%%%%%%%%%%%%%%%%%%%%%%%
%%%%%%%%%%%%%%%%%%%%%%%%%%%%%%%%%%%%%%%%%%%%%%%%%%%%%%%%%%%%%%%%%%%%%%%%%%%%%%%%
%%%%%%%%%%%%%%%%%%%%%%%%%%%%%%%%%%%%%%%%%%%%%%%%%%%%%%%%%%%%%%%%%%%%%%%%%%%%%%%%
\appendix

\settowidth\MacroIndent{\rmfamily\scriptsize 000\ }

 \DocInput{childdoc.dtx}

\end{document}
%</driver>
% \fi
%
% %%%%%%%%%%%%%%%%%%%%%%%%%%%%%%%%%%%%%%%%%%%%%%%%%%%%%%%%%%%%%%%%%%%%%%%%%%%%%%
% %%%%%%%%%%%%%%%%%%%%%%%%%%%%%%%%%%%%%%%%%%%%%%%%%%%%%%%%%%%%%%%%%%%%%%%%%%%%%%
% \section{Sample}
%\iffalse
%<*samplemain>
%\fi
%
% The following presents a sample document
% with two chapters, two parts, a title page,
% a compile flag as well as three forwarding files to set the flag.
% It consists of eight |.tex| files:
% \begin{center}
% \begin{tabular}{ll}
% |cdocsamp.tex|&main file\\
% |cdocsch1.tex|&include file for chapter 1\\
% |cdocsch2.tex|&include file for chapter 2\\
% |cdocspt3.tex|&include file for part 3\\
% |cdocspt4.tex|&include file for part 4\\
% |cdocsdrf.tex|&forwarding file for main file in draft mode\\
% |cdocsfi1.tex|&forwarding file for final version of chapter 1\\
% |cdocsfi2.tex|&forwarding file for final version of chapter 2\\
% \end{tabular}
% \end{center}
% Each of the eight files can be compiled directly by the \LaTeX{} compiler.
%
% %%%%%%%%%%%%%%%%%%%%%%%%%%%%%%%%%%%%%%
% \paragraph{Main File.}
%
% The main file is called |cdocsamp.tex|.
%
% Load the \textsf{childdoc} definitions and
% declare the filename for the main document:
%    \begin{macrocode}
\input{childdoc.def}
\childdocmain{}
%    \end{macrocode}

% Optional override for |\version| flag:
%    \begin{macrocode}
%%\ifchilddoc\else\providecommand{\version}{draft}\fi
%    \end{macrocode}

% Define the default values for the |\version| flag
% (|final| for the main file and |draft| for childs):
%    \begin{macrocode}
\ifchilddoc
\providecommand{\version}{draft}
\else
\providecommand{\version}{final}
\fi
%    \end{macrocode}

% Load the standard document class:
%    \begin{macrocode}
\documentclass[12pt]{article}
%    \end{macrocode}

% Start the document body:
%    \begin{macrocode}
\begin{document}
%    \end{macrocode}

% Declare a title page.
% Print title, part of document being processed and version flag:
%    \begin{macrocode}
\addtocounter{page}{-1}
\begin{center}
{\LARGE\bfseries{}childdoc example\par}
\vspace{1cm}
\ifchilddoc
\ifchilddocmanual part\else chapter\fi:
`\childdocname' of `\childdocjob'\par
\else
main document: `\childdocjob'\par
\fi
version: \version\par
\end{center}
\newpage
%    \end{macrocode}

% Manually include selected file,
% otherwise process as usual:
%    \begin{macrocode}
\ifchilddocmanual
\section*{part `\childdocname'}
\input{\childdocname}
\else
%    \end{macrocode}

% Include the two chapters:
%    \begin{macrocode}
\include{cdocsch1}
\include{cdocsch2}
%    \end{macrocode}

% Include the two parts unless only chapters should be displayed:
%    \begin{macrocode}
\ifchilddoc\else
\section{part three}
\input{cdocspt3}
\section{part four}
\input{cdocspt4}
\fi
%    \end{macrocode}

% Process as usual until here:
%    \begin{macrocode}
\fi
%    \end{macrocode}

% End of document body:
%    \begin{macrocode}
\end{document}
%    \end{macrocode}
%\iffalse
%</samplemain>
%\fi
%
% %%%%%%%%%%%%%%%%%%%%%%%%%%%%%%%%%%%%%%
% \paragraph{Chapter Include Files.}
%
% The include files are called |cdocsch1.tex| and |cdocsch2.tex|.
%
%\iffalse
%<*samplechap1|samplechap2>
%\fi

% Optional override for |\version| flag:
%    \begin{macrocode}
%%\providecommand{\version}{final}
%    \end{macrocode}

% Include the main document:
%    \begin{macrocode}
\input{childdoc.def}
\childdocof{cdocsamp}
%    \end{macrocode}

%\iffalse
%</samplechap1|samplechap2>
%\fi
%
%\iffalse
%<*samplechap1>
%\fi
% Some text for chapter 1:
%    \begin{macrocode}
\section{one}
some text in chapter one
%    \end{macrocode}

%\iffalse
%</samplechap1>
%\fi
% Some text for chapter 2:
%\iffalse
%<*samplechap2>
%\fi
%    \begin{macrocode}
\section{two}
more text in chapter two
%    \end{macrocode}

%\iffalse
%</samplechap2>
%\fi
%
% %%%%%%%%%%%%%%%%%%%%%%%%%%%%%%%%%%%%%%
% \paragraph{Part Include Files.}
%
% The include files are called |cdocspt3.tex| and |cdocspt4.tex|.
%
%\iffalse
%<*samplepart3|samplepart4>
%\fi

% Optional override for |\version| flag:
%    \begin{macrocode}
%%\providecommand{\version}{final}
%    \end{macrocode}

% Include the main document:
%    \begin{macrocode}
\input{childdoc.def}
\childdocby{cdocsamp}
%    \end{macrocode}

%\iffalse
%</samplepart3|samplepart4>
%\fi
%
%\iffalse
%<*samplepart3>
%\fi
% Some text for part 3:
%    \begin{macrocode}
some text in part three
%    \end{macrocode}

%\iffalse
%</samplepart3>
%\fi
% Some text for part 4:
%\iffalse
%<*samplepart4>
%\fi
%    \begin{macrocode}
more text in part four
%    \end{macrocode}

%\iffalse
%</samplepart4>
%\fi
%
% %%%%%%%%%%%%%%%%%%%%%%%%%%%%%%%%%%%%%%
% \paragraph{Forwarding for a Complete Draft.}
%
% The following forwarding file |cdocsdrf.tex|
% compiles the main document in draft mode:
%\iffalse
%<*sampledraft>
%\fi
%    \begin{macrocode}
\def\version{draft}
\input{childdoc.def}
\childdocforward{cdocsamp}
%    \end{macrocode}

%\iffalse
%</sampledraft>
%\fi
%
% %%%%%%%%%%%%%%%%%%%%%%%%%%%%%%%%%%%%%%
% \paragraph{Forwarding for Final Version of the Chapters.}
%
% The following forwarding files |cdocsfn1.tex| and |cdocsfn2.tex|
% (with identical content)
% compile the final versions of the child documents
% |cdocsch1.tex| and |cdocsch2.tex|, respectively:
%\iffalse
%<*samplefinal>
%\fi
%    \begin{macrocode}
\def\version{final}
\input{childdoc.def}
\childdocforwardprefix[cdocsamp]{cdocsfn}{cdocsch}
%    \end{macrocode}

%\iffalse
%</samplefinal>
%\fi
%
% %%%%%%%%%%%%%%%%%%%%%%%%%%%%%%%%%%%%%%
% \paragraph{Command Line Processing.}
%
% The following three command lines generate the output files
% |cdocscld|, |cdocscl1| and |cdocscl2|
% which should be identical to
% |cdocsdrf|, |cdocsch1| and |cdocsfn2|, respectively:
% \begin{center}
% \begin{tabular}{l}
% |latex -jobname cdocscld \|\\
% |  "\def\version{draft}\input{childdoc.def}\childdocforward{cdocsamp}"|\\
% |latex -jobname cdocscl1 \|\\
% |  "\input{childdoc.def}\childdocforward[cdocsamp]{cdocsch1}"|\\
% |latex -jobname cdocscl2 \|\\
% |  "\def\version{final}\input{childdoc.def}\childdocforward{cdocsch2}"|
% \end{tabular}
% \end{center}
% Note that the trailing backslash on each first line
% merely continues the input to the second line
% (for convenient cut ant paste).
% Furthermore, the command |latex| can be replaced by any
% of its alternative versions such as |pdflatex|.
%
% %%%%%%%%%%%%%%%%%%%%%%%%%%%%%%%%%%%%%%%%%%%%%%%%%%%%%%%%%%%%%%%%%%%%%%%%%%%%%%
% %%%%%%%%%%%%%%%%%%%%%%%%%%%%%%%%%%%%%%%%%%%%%%%%%%%%%%%%%%%%%%%%%%%%%%%%%%%%%%
% \section{Implementation}
%\iffalse
%<*package>
%\fi
%
% This section describes the definitions file |childdoc.def|.

% The definitions cannot be loaded using |\usepackage| or |\RequirePackage|
% which has a mechanism to prevent loading a style file more than once.
% When loading the definitions by means of |\input|
% multiple instances have to be prevented manually:
%\iffalse
%This code needs to be before the `\ProvidesFile' directive
%which is defined at the beginning of this file.
%Therefore it is also placed there and commented out here.
%</package>
%<*discard>
%\fi
%    \begin{macrocode}
\ifdefined\childdocmain\endinput\fi
%    \end{macrocode}
%\iffalse
%</discard>
%<*package>
%\fi
%
% \macro{\ifchilddoc}
% \macro{\ifchilddocmanual}
% The conditional |\ifchilddoc| tells whether a
% child (true) or main (false) document is being compiled.
% The conditional |\ifchilddocmanual| tells whether
% the |\includeonly| mechanism is used (false) or
% the selection of child files must be performed manually (true).
% The definitions initialise to false:
%    \begin{macrocode}
\newif\ifchilddoc
\newif\ifchilddocmanual
%    \end{macrocode}

% \macro{\childdocname}
% \macro{\childdocjob}
% The macro |\childdocname| stores the name of the main document
% to be compiled. The macro |\childdocjob| stores the name of
% the document on which the \LaTeX{} compiler was originally invoked.
% The content of |\jobname| cannot be compared
% to filenames specified in the source due to different catcodes.
% The following code rescans |\jobname|, stores the result
% in |\childdocname| and saves a copy in |\childdocjob|:
%    \begin{macrocode}
\edef\childdocname{\scantokens\expandafter{\jobname\noexpand}}
\let\childdocjob\childdocname
%    \end{macrocode}

% \macro{\childdocdisable}
% The macro |\childdocdisable| prevents the main file
% from being processed more than once.
% At this stage, the main document command |\childdocmain|
% is assumed to be called once again where it should do nothing.
% Any subsequent call to it should prevent
% a secondary processing of the main document
% It overwrites the forwarding commands
% |\childdocof| and |\childdocforward|
% with empty macros to prevent further inclusions of the main document:
%    \begin{macrocode}
\newcommand{\childdocdisable}
{
  \renewcommand{\childdocmain}[1]{\renewcommand{\childdocmain}[1]{\endinput}}
  \renewcommand{\childdocof}[1]{}
  \renewcommand{\childdocby}[2][]{}
  \renewcommand{\childdocforward}[2][]{}
  \renewcommand{\childdocdisable}{}
}
%    \end{macrocode}

% \macro{\childdocmain}
% The macro |\childdocmain| is to be called at the top of the main file
% with nothing or the main filename (without extension) as argument.
% First, it breaks loops.
% If the argument is not empty and does not match |\childdocname|
% (which is set by the first inclusion of |childdoc.def|),
% |\ifchilddoc| is set to true, |\includeonly| is applied to the child file
% and |\jobname| is set to the main file
% (for proper handling of |.aux| files):
%    \begin{macrocode}
\newcommand{\childdocmain}[1]
{
  \childdocdisable\childdocmain{}
  \if?#1?\else
    \begingroup
      \def\childdoctmp{#1}
      \ifx\childdoctmp\childdocname
        \def\childdoctmp{}
      \else
        \def\childdoctmp
        {
          \childdoctrue
          \includeonly{\childdocname}
          \def\childdocjob{#1}
          \def\jobname{#1}
        }
      \fi
      \expandafter
    \endgroup
    \childdoctmp
  \fi
}
%    \end{macrocode}

% \macro{\childdocof}
% The command |\childdocof| redirects
% compilation to the main file |#1|.
%    \begin{macrocode}
\newcommand{\childdocof}[1]
{
  \childdocdisable
  \childdoctrue
  \includeonly{\childdocname}
  \def\jobname{#1}
  \def\childdocjob{#1}
  \input{#1}
}
%    \end{macrocode}

% \macro{\childdocby}
% The command |\childdocby| ....
%    \begin{macrocode}
\newcommand{\childdocby}[2][]
{
  \childdocdisable
  \childdoctrue
  \childdocmanualtrue
  \if?#1?\else
    \def\jobname{#2}
  \fi
  \def\childdocjob{#2}
  \input{#2}
  \endinput
}
%    \end{macrocode}

% \macro{\childdocforward}
% The command |\childdocforward| redirects
% compilation to the main file or
% (if the optional argument is given) a child file.
% Parameters are set as if the main file
% or a child file starting with |\childdocof| was compiled.
% Then compilation is handed over to the main file:
%    \begin{macrocode}
\newcommand{\childdocforward}[2][]
{
  \begingroup
    \if?#1?
      \def\childdoctmp
      {
        \def\childdocname{#2}
        \def\childdocjob{#2}
        \def\jobname{#2}
        \input{#2}
        \endinput
      }
    \else
      \def\childdoctmp
      {
        \childdocdisable
        \def\childdocname{#2}
        \childdoctrue
        \includeonly{#2}
        \def\childdocjob{#1}
        \def\jobname{#1}
        \input{#1}
        \endinput
      }
    \fi
    \expandafter
  \endgroup
  \childdoctmp
}
%    \end{macrocode}

% \macro{\childdocforwardprefix}
% The command |\childdocforwardprefix| redirects
% compilation to the main or a child file by means of a pattern.
% The prefix |#1| in the current filename is replaced by |#2|
% and the suffix of the current filename is kept
% (it is assumed that the filename does not contain the substring `|~~~|'
% which is used as a delimiter).
% Compilation is handed over to the new file by |\childdocforward|:
%    \begin{macrocode}
\newcommand{\childdocforwardprefix}[3][]
{
  \begingroup
    \def\childdocextract #2##1~~~{\def\childdoctmp{\childdocforward[#1]{#3##1}}}
    \expandafter\childdocextract\childdocname~~~
    \expandafter
  \endgroup
  \childdoctmp
}
%    \end{macrocode}

% \macro{\childdoc}
% The deprecated macro |\childdoc| is a legacy version of |\childdocmain|:
%    \begin{macrocode}
\newcommand{\childdoc}{\childdocmain}
%    \end{macrocode}

% \macro{\childdocredirect}
% The deprecated macro |\childdocredirect| is a legacy version
% of |\childdocforward| and |\childdocforwardprefix|:
%    \begin{macrocode}
\newcommand{\childdocredirect}[2][]
{
  \begingroup
    \if?#1?
      \def\childdoctmp{\childdocforward{#2}}
    \else
      \def\childdoctmp{\childdocforwardprefix{#1}{#2}}
    \fi
    \expandafter
  \endgroup
  \childdoctmp
}
%    \end{macrocode}

%\iffalse
%</package>
%\fi
%
\endinput
|\\
|\childdocmain{}|\\
\end{tabular}
\end{center}
at the very top of the main \LaTeX{} file,
in particular \emph{before} the |\documentclass| statement!
The argument of |\childdocmain| should be left empty
(but it must be present).

%%%%%%%%%%%%%%%%%%%%%%%%%%%%%%%%%%%%%%%%
\DescribeMacro{\childdocof}
Furthermore, add the commands
\begin{center}
\begin{tabular}{l}
|% \iffalse
%
% childdoc.dtx Copyright (C) 2017-2018 Niklas Beisert
%
% This work may be distributed and/or modified under the
% conditions of the LaTeX Project Public License, either version 1.3
% of this license or (at your option) any later version.
% The latest version of this license is in
%   http://www.latex-project.org/lppl.txt
% and version 1.3 or later is part of all distributions of LaTeX
% version 2005/12/01 or later.
%
% This work has the LPPL maintenance status `maintained'.
%
% The Current Maintainer of this work is Niklas Beisert.
%
% This work consists of the files childdoc.dtx and childdoc.ins
% and the derived files childdoc.def and cdocsamp.tex with
% cdocsch1.tex, cdocsch2.tex, cdocsdrf.tex, cdocsfn1.tex, cdocsfn2.tex.
%
%<package>\ifdefined\childdocmain\endinput\fi
%<package>\ProvidesFile{childdoc.def}[2018/12/30 v2.0 child document driver]
%<samplemain>\ProvidesFile{cdocsamp.tex}[2018/12/30 v2.0 sample for childdoc]
%<*driver>
%\ProvidesFile{childdoc.drv}[2018/12/30 v2.0 childdoc reference manual file]
\PassOptionsToClass{10pt,a4paper}{article}
\documentclass{ltxdoc}

\usepackage[margin=35mm]{geometry}
\usepackage{hyperref}
\usepackage{hyperxmp}
\usepackage[usenames]{color}

\hypersetup{colorlinks=true}
\hypersetup{pdfstartview=FitH}
\hypersetup{pdfpagemode=UseNone}
\hypersetup{pdfsource={}}
\hypersetup{pdflang={en-UK}}
\hypersetup{pdfcopyright={Copyright 2017-2018 Niklas Beisert.
  This work may be distributed and/or modified under the
  conditions of the LaTeX Project Public License, either version 1.3
  of this license or (at your option) any later version.}}
\hypersetup{pdflicenseurl={http://www.latex-project.org/lppl.txt}}
\hypersetup{pdfcontactaddress={ETH Zurich, ITP, HIT K,
  Wolfgang-Pauli-Strasse 27}}
\hypersetup{pdfcontactpostcode={8093}}
\hypersetup{pdfcontactcity={Zurich}}
\hypersetup{pdfcontactcountry={Switzerland}}
\hypersetup{pdfcontactemail={nbeisert@itp.phys.ethz.ch}}
\hypersetup{pdfcontacturl={http://people.phys.ethz.ch/\xmptilde nbeisert/}}

\newcommand{\secref}[1]{\hyperref[#1]{section \ref*{#1}}}

\parskip1ex
\parindent0pt
\let\olditemize\itemize
\def\itemize{\olditemize\parskip0pt}

\begin{document}

\title{The \textsf{childdoc} Package}
\hypersetup{pdftitle={The childdoc Package}}
\author{Niklas Beisert\\[2ex]
  Institut f\"ur Theoretische Physik\\
  Eidgen\"ossische Technische Hochschule Z\"urich\\
  Wolfgang-Pauli-Strasse 27, 8093 Z\"urich, Switzerland\\[1ex]
  \href{mailto:nbeisert@itp.phys.ethz.ch}
  {\texttt{nbeisert@itp.phys.ethz.ch}}}
\hypersetup{pdfauthor={Niklas Beisert}}
\hypersetup{pdfsubject={Manual for the LaTeX2e Package childdoc}}
\date{30 December 2018, \textsf{v2.0}}
\maketitle

\begin{abstract}\noindent
\textsf{childdoc} is a \LaTeXe{} package
that enables the direct compilation
of document sections included by |\include|
to individual files.
\end{abstract}

\begingroup
\parskip0ex
\tableofcontents
\endgroup

%%%%%%%%%%%%%%%%%%%%%%%%%%%%%%%%%%%%%%%%%%%%%%%%%%%%%%%%%%%%%%%%%%%%%%%%%%%%%%%%
%%%%%%%%%%%%%%%%%%%%%%%%%%%%%%%%%%%%%%%%%%%%%%%%%%%%%%%%%%%%%%%%%%%%%%%%%%%%%%%%
\section{Introduction}

\LaTeX{} provides a mechanism to structure a large document (such as a book)
into a main file and several child files (containing the chapters)
using the |\include| command.
This mechanism is beneficial for documents
which span hundreds of pages in order to
make the source file(s) more manageable.
Moreover, compilation can be restricted to
selected child files by means of the |\includeonly| command.
The latter feature can be used to reduce the compilation time while editing
(this was significantly more useful in the earlier days of \LaTeX{})
or to generate a smaller document which is easier to navigate.
Another application of |\includeonly| is to generate
documents consisting of selected parts of the complete document.

However, there are a few drawbacks of the plain |\include| mechanism:
\begin{itemize}
\item
The child files cannot be compiled on their own,
they can only be compiled via the main file.
A naive editing environment
(such as a text editor with an option
to have the current file processed by \LaTeX)
may require one to switch to the main file before compiling;
attempting to compile the child file produces errors.
\item
The main file must be modified (each time)
to adjust the |\includeonly| command
to the present needs. This easily leaves the main file in a messy state.
\item
The generated document will always carry the filename
of the main document. This is inconvenient if
several child files are to be compiled and
to be kept for distribution.
\end{itemize}

The present package provides a simple interface
to make child files individually compilable by \LaTeX{}.
Compiling a child file then has the same effect as compiling
the main file with an |\includeonly| command
to select the appropriate child.
Moreover the generated document will carry the name of the child
rather than the main file.
This resolves all three above issues.

This feature is meant to make the editing of books,
thesis documents and lecture notes somewhat more convenient.
However, the package can also be used efficiently for
composing a series of documents (such as exercise sheets)
which are typically distributed individually.
It then assists the author in generating the individual documents
(potentially in different versions)
as well as a document containing the collected series.
Another application is in developing style files
or other kinds of included material
where compilation of the style file could redirect
to a sample or test file.

%%%%%%%%%%%%%%%%%%%%%%%%%%%%%%%%%%%%%%%%%%%%%%%%%%%%%%%%%%%%%%%%%%%%%%%%%%%%%%%%
%%%%%%%%%%%%%%%%%%%%%%%%%%%%%%%%%%%%%%%%%%%%%%%%%%%%%%%%%%%%%%%%%%%%%%%%%%%%%%%%
\section{Usage}

First of all, the package \textsf{childdoc} is \emph{not} a standard
\LaTeXe{} |.sty| style file! Therefore it needs to be invoked in
a non-standard way.

%%%%%%%%%%%%%%%%%%%%%%%%%%%%%%%%%%%%%%%%%%%%%%%%%%%%%%%%%%%%%%%%%%%%%%%%%%%%%%%%
\subsection{Included Files}
\label{sec:include}

%%%%%%%%%%%%%%%%%%%%%%%%%%%%%%%%%%%%%%%%
\DescribeMacro{\childdocmain}
To use the package, add the commands
\begin{center}
\begin{tabular}{l}
|\input{childdoc.def}|\\
|\childdocmain{}|\\
\end{tabular}
\end{center}
at the very top of the main \LaTeX{} file,
in particular \emph{before} the |\documentclass| statement!
The argument of |\childdocmain| should be left empty
(but it must be present).

%%%%%%%%%%%%%%%%%%%%%%%%%%%%%%%%%%%%%%%%
\DescribeMacro{\childdocof}
Furthermore, add the commands
\begin{center}
\begin{tabular}{l}
|\input{childdoc.def}|\\
|\childdocof{|\textit{main}|}|\\
\end{tabular}
\end{center}
at the top of every child file \textit{child}
which is included by |\include{|\textit{child}|}|
from within the main file
(or at least for those files to be compiled individually).
The argument \textit{main} must be the filename of the main file.

There are a couple of
considerations in setting up the main and child documents:

%%%%%%%%%%%%%%%%%%%%%%%%%%%%%%%%%%%%%%%%
\paragraph{Restrictions.}

Please note the following restrictions:
\begin{itemize}
\item
|\childdocmain| must be called with one argument \textit{main}
to ensure compatibility with earlier version of the package.
It must either be empty (|\childdocmain{}|)
or precisely match the filename of the main file in which it is specified.
See \secref{sec:detection} for further information.
\item
The filename \textit{main} must be specified without the |.tex| extension.
\item
The filename \textit{main} is case sensitive
(even in case-insensitive file systems)
due to internal string comparison.
\item
The argument \textit{main} should be fully expanded, it cannot be a macro.
\item
Subdirectories and special characters should be avoided in filenames.
\item
The command |\childdocmain{|\textit{main}|}| must be followed by a whitespace.
It should not be followed immediately by another command
or by a comment mark `|%|'.
This is because the \TeX{} parser reads the token immediately following
the argument of |\childdocmain| and puts it
at the beginning of every child section;
however, a white\-space is ignored.
\end{itemize}

%%%%%%%%%%%%%%%%%%%%%%%%%%%%%%%%%%%%%%%%
\paragraph{Content of Main File.}

It is advisable to place all content in the child files included by |\include|.
Any output contained in the main file will appear in all child documents
unless suppressed manually;
it cannot be suppressed automatically by the |\includeonly| directive
and thus should normally be avoided.
A method to include some content in the main file
by means of conditional processing is described in \secref{sec:conditional}.

%%%%%%%%%%%%%%%%%%%%%%%%%%%%%%%%%%%%%%%%
\paragraph{Page Numbering.}

When only a part of the document is compiled,
the appropriate numbering of pages
(as well as other status parameters)
is determined from the |.aux| files.
The latter contain information from previous passes.
However this information needs to propagate through
all intermediate child documents.
Therefore the page numbering in child documents may well
be inconsistent until the complete document is compiled at least once.

A useful (if unconventional) way to always ensure a consistent
page numbering is to restart the numbering in each child document
and denote the pages by `\textit{child}|.|\textit{page}'
where \textit{child} represents the chapter/section number of the child file.
This can be achieved by the command
|\numberwithin{page}{|\textit{child}|}|
of the \textsf{amsmath} package
where \textit{child} can be |chapter| or |section|
depending on the chosen structuring.
Alternatively, one can modify the macro |\thepage| appropriately
and reset the counter |page| at the start of each child file.

%%%%%%%%%%%%%%%%%%%%%%%%%%%%%%%%%%%%%%%%%%%%%%%%%%%%%%%%%%%%%%%%%%%%%%%%%%%%%%%%
\subsection{Conditional Processing}
\label{sec:conditional}

The package provides a mechanism to compile different versions
of a document. To customise the versions further some conditional processing
can come in handy to distinguish which version is being compiled.
The package provides two macros to describe the compilation context:

%%%%%%%%%%%%%%%%%%%%%%%%%%%%%%%%%%%%%%%%
\DescribeMacro{\ifchilddoc}
The conditional |\ifchilddoc| distinguishes between the compilation of
child documents and the main document:
%
\begin{center}
|\ifchilddoc |\textit{child-code}| |[|\||else |\textit{main-code}]| \||fi|
\end{center}

%%%%%%%%%%%%%%%%%%%%%%%%%%%%%%%%%%%%%%%%
\DescribeMacro{\childdocname}
\DescribeMacro{\childdocjob}
The macro |\childdocname| contains the filename (without extension)
of the main or child file being processed.
Note that |\childdocjob| will always contain the name of the main file.

%%%%%%%%%%%%%%%%%%%%%%%%%%%%%%%%%%%%%%%%
\paragraph{Title Page.}

Conditional processing can be used to include a title or banner page
in the main document when proper precautions are taken.
Importantly, the code in the main file should ensure that the page counter
(as well as other status parameters which are stored in the |.aux| files)
takes the same value after the conditional processing.
Otherwise the page numbers may take divergent values
depending on which part is compiled.

For example, a title page could be declared by:
%
\begin{center}
\begin{tabular}{l}
|\ifchilddoc\||else|\\
|\addtocounter{page}{-1}|\\
\textit{code for title page}\\
|\newpage|\\
|\||fi|
\end{tabular}
\end{center}
%
A banner page for the child documents can be generated by:
%
\begin{center}
\begin{tabular}{l}
|\ifchilddoc|\\
|\addtocounter{page}{-1}|\\
\textit{code for banner page}\\
|\newpage|\\
|\||fi|
\end{tabular}
\end{center}
%
Here one could write a message such as:
\begin{center}
|This is the part \childdocname{} of \childdocjob{}.|
\end{center}

%%%%%%%%%%%%%%%%%%%%%%%%%%%%%%%%%%%%%%%%%%%%%%%%%%%%%%%%%%%%%%%%%%%%%%%%%%%%%%%%
\subsection{Flags}
\label{sec:flags}

The package makes it easy to generate different versions
of the main or child documents.
To this end compilation flags can be defined
and assigned different default values.
They will be particularly useful in conjunction
with the forwarding mechanism described in \secref{sec:forward}.

For example, it may be useful to have a flag |\version|
which can be set to |draft| or |final|.
The document source will contain some conditional code
depending on the value of |\version|.
Suppose further, the flag should default to |final| for the main file
and to |draft| for child files
which is a natural assignment for editing the document.
This is achieved by placing the following code
in the preamble of the main document
(below the |\childdocmain| directive):
%
\begin{center}
\begin{tabular}{l}
|\ifchilddoc|\\
|\providecommand{\version}{draft}|\\
|\||else|\\
|\providecommand{\version}{final}|\\
|\||fi|
\end{tabular}
\end{center}
%
The definition by |\providecommand| makes sure
that previous definitions are not overwritten.
Further statements |\providecommand{\version}{...}|
can thus be added before the above code to override it.

For the main file, one might add a line
(between |\childdocmain| and the above block)
%
\begin{center}
|%\ifchilddoc\||else\providecommand{\version}{draft}\||fi|
\end{center}
%
which can be uncommented to produce a draft version.
Likewise one can add a line to the very top of a child file
(above the |\childdocof{|\textit{main}|}| directive)
%
\begin{center}
|%\providecommand{\version}{final}|
\end{center}
%
which can be uncommented to produce the final version of this child document.

%%%%%%%%%%%%%%%%%%%%%%%%%%%%%%%%%%%%%%%%%%%%%%%%%%%%%%%%%%%%%%%%%%%%%%%%%%%%%%%%
\subsection{Forwarding}
\label{sec:forward}

Different versions of the main or child documents
using compilation flags as described in \secref{sec:flags}
can be (permanently) stored in different files
for convenient compilation, viewing and distribution.
To this end, the package defines a command
to pass on compilation to a different file:

%%%%%%%%%%%%%%%%%%%%%%%%%%%%%%%%%%%%%%%%
\DescribeMacro{\childdocforward}
The command |\childdocforward| redirects processing to
another source file:
%
\begin{center}
\begin{tabular}{l}
|\input{childdoc.def}|\\
|\childdocforward[|\textit{main}|]{|\textit{dest}|}|\\
\end{tabular}
\end{center}
%
The argument \textit{dest} is the destination file
(without extension).
It should be the main file or one of the child files.
Note that further \textsf{childdoc} directives
such as |\childdocof| and |\childdocforward|
in the indicated file will be processed in this form.
The optional argument \textit{main}
passes on directly to the main file \textit{main}
while pretending to compile the child \textit{dest}.
This form behaves as if \textit{dest}
issues |\childdocof{|\textit{main}|}| right away,
and no further \textsf{childdoc} directives will be processed.

%%%%%%%%%%%%%%%%%%%%%%%%%%%%%%%%%%%%%%%%
\DescribeMacro{\...prefix}
In the alternative form |\childdocforwardprefix|,
%
\begin{center}
\begin{tabular}{l}
|\input{childdoc.def}|\\
|\childdocforwardprefix[|\textit{main}|]{|\textit{prefix}|}{|\textit{dest}|}|
\end{tabular}
\end{center}
%
the destination file is determined by a pattern
depending on the current file:
To make this work, the current file must be called
`{\textit{prefix}\hspace{0.2em}\textit{suffix}}'
with \textit{prefix} matching precisely the argument.
Processing is then passed on to the file
`{\textit{dest}\hspace{0.2em}\textit{suffix}}'.
Surely, the same effect is achieved by
directly specifying the
argument `{\textit{dest}\hspace{0.2em}\textit{suffix}}'
in the first form.
However, that requires to set up a different file
for each child. With the alternative form of the command
all these files can have exactly the same content
which simplifies setting them up and maintaining them.

For example, the following file |draft.tex|
with a compilation flag |\version| as described in \secref{sec:flags}
compiles the main document as a draft:
%
\begin{center}
\begin{tabular}{l}
|\def\version{draft}|\\
|\input{childdoc.def}|\\
|\childdocforward{|\textit{main}|}|
\end{tabular}
\end{center}
%
Likewise, the following files |final|\textit{nn}|.tex|
compile the final version of the child document
|child|\textit{nn}|.tex|:
%
\begin{center}
\begin{tabular}{l}
|\def\version{final}|\\
|\input{childdoc.def}|\\
|\childdocforwardprefix{final}{child}|
\end{tabular}
\end{center}
%

Note that when several versions of a main file and/or of each child file
are to be generated, it may be convenient to set up a |Makefile| or
shell script to automatise the process.

%%%%%%%%%%%%%%%%%%%%%%%%%%%%%%%%%%%%%%%%%%%%%%%%%%%%%%%%%%%%%%%%%%%%%%%%%%%%%%%%
\subsection{Command Line Processing}
\label{sec:commandline}

The effect of redirection files can also be achieved by invoking
the \LaTeX{} compiler with a more elaborate command line.
Most conveniently this should be done as part
of a shell script or a |Makefile|.

When using \textsf{childdoc} in the main file, the following
command lines effectively perform a redirection
(note that depending on the shell being used,
backslashes may have to be doubled: `|\|' $\to$ `|\\|'):
%
\begin{center}
|... -jobname "|\textit{target}|" |\\|"|[\textit{flags}]%
|\input{childdoc.def}\childdocforward[|\textit{main}|]{|\textit{dest}|}"|
\end{center}
%
Here \textit{target} is the name of the output file,
\textit{main} is the name of the main file
and \textit{dest} is the name of the main or child file to be processed
(all filenames without extensions).
The optional argument \textit{main} can be omitted
if \textit{main} matches \textit{dest}.
Optionally, compilation \textit{flags} can be defined via |\def| commands.
This command line makes the \TeX{} engine believe
it is compiling the file \textit{target}
whose content is specified as the latter parameter.
The provided code then forwards the processing to
\textit{main} or \textit{dest} as described in \secref{sec:forward}.

%%%%%%%%%%%%%%%%%%%%%%%%%%%%%%%%%%%%%%%%%%%%%%%%%%%%%%%%%%%%%%%%%%%%%%%%%%%%%%%%
\subsection{Include by Input}
\label{sec:input}

Including child documents by |\include| has some restrictions by design.
Most notably, the content of a child document always occupies
its own set of pages; pages cannot be shared between child documents.
Usually, this behaviour makes perfect sense
because each child document contain an essential part of the document.
However, in some situations it may be desirable to compose
a document from a collection of parts
without having mandatory page breaks between then.
For this case, the package
provides a mechanism to include parts
by |\input| which can also be processed individually.
However, by construction this mechanism
requires manual handling of the content to be output.

%%%%%%%%%%%%%%%%%%%%%%%%%%%%%%%%%%%%%%%%
\DescribeMacro{\ifchilddocmanual}
The main file should be prepared as usual, see \secref{sec:include}.
However, the document body must make a distinction
between processing of an individual part and of the main document, e.g.:
%
\begin{center}
\begin{tabular}{l}
|\ifchilddocmanual|\\
|\input{\childdocname}|\\
|\||else|\\
\textit{document body with }|\input{|\textit{part}|}|\\
|\||fi|
\end{tabular}
\end{center}
%
The conditional |\ifchilddocmanual| is true whenever
a part to be included by |\input| is being compiled,
and the name of the part is stored in |\childdocname|.

%%%%%%%%%%%%%%%%%%%%%%%%%%%%%%%%%%%%%%%%
\DescribeMacro{\childdocby}
Each part to be included by |\input| should start with:
%
\begin{center}
\begin{tabular}{l}
|\input{childdoc.def}|\\
|\childdocby{|\textit{main}|}|\\
\end{tabular}
\end{center}
%
The directive |\childdocby| is similar to |\childdocof|
described in \secref{sec:include},
but the subsequent selection of content must be done manually.
To that end, both |\ifchilddoc| and |\ifchilddocmanual|
will be true upon processing of a part,
and the name of the part is stored in |\childdocname|.
Note that |\jobname| will be set to the filename of the current part
so that each part receives an individual |.aux| file
that does not interfere with the |.aux| file(s) of the main document.
This behaviour can be altered by the alternative form
|\childdocby[*]{|\textit{main}|}| (with a non-empty optional argument)
which uses the |.aux| file of the main document
by setting |\jobname| to \textit{main}.

%%%%%%%%%%%%%%%%%%%%%%%%%%%%%%%%%%%%%%%%%%%%%%%%%%%%%%%%%%%%%%%%%%%%%%%%%%%%%%%%
\subsection{Driver Development}
\label{sec:driver}

The \textsf{childdoc} mechanism can also be use for the development
of definition files such as \LaTeX{} styles or classes.
This case differs from the above setup with multiple parts
included by |\include| in that no |\includeonly| should be invoked.
This can be achieved by starting the include file
(before |\ProvidesPackage|) with:
%
\begin{center}
\begin{tabular}{l}
|\input{childdoc.def}|\\
|\childdocforward{|\textit{main}|}|\\
\end{tabular}
\end{center}
%
or alternatively with:
%
\begin{center}
\begin{tabular}{l}
|\input{childdoc.def}|\\
|\childdocby{|\textit{main}|}|\\
\end{tabular}
\end{center}
%
Both forms have slightly different effects as described above.
The main file is prepared as usual, see \secref{sec:include}.

%%%%%%%%%%%%%%%%%%%%%%%%%%%%%%%%%%%%%%%%%%%%%%%%%%%%%%%%%%%%%%%%%%%%%%%%%%%%%%%%
\subsection{Legacy Detection}
\label{sec:detection}

The directive |\childdocmain| in the main file can detect
whether the complete document or merely a child is to be compiled
even without using the directive |\childdocof|.
This method is deprecated because it is less robust
and there is no compelling reason to use it;
it is merely provided for backward compatibility
and it may be removed in future versions.

If the detection mechanism is to be used,
it is mandatory to correctly specify
the filename of the main file as the argument of |\childdocmain|:
%
\begin{center}
\begin{tabular}{l}
|\input{childdoc.def}|\\
|\childdocmain{|\textit{main}|}|\\
\end{tabular}
\end{center}
%
If |\jobname| does not match the argument \textit{main} of |\childdocmain|,
it is assumed that |\jobname| points to the child file to be compiled.
When using |\childdocmain| with the main file specified as argument,
it suffices to start a child file
with just |\input{|\textit{main}|}|
without loading of the package and using |\childdocof|.
If instead all processing is done
with the appropriate \textsf{childdoc} directives,
the argument of \textit{main} of |\childdocmain| can be empty.

An alternative version of the command line processing described
in \secref{sec:commandline} using the detection mechanism reads:
%
\begin{center}
|... -jobname "|\textit{target}|" "|[\textit{flags}]%
[|\def\jobname{|\textit{dest}|}|]|\input{|\textit{main}|}"|
\end{center}

%%%%%%%%%%%%%%%%%%%%%%%%%%%%%%%%%%%%%%%%%%%%%%%%%%%%%%%%%%%%%%%%%%%%%%%%%%%%%%%%
\subsection{Manual Code}
\label{sec:manual}

In case one cannot be certain whether the definitions file |childdoc.def|
is installed on the target \TeX{} distribution
and one prefers not to ship it,
it is conceivable to paste a few relevant commands into the sources.

To that end, drop all statements |\input{childdoc.def}|
and perform the replacements as outlined below.
Instead of |\childdocmain{|\textit{main}|}| add the following code
to the top of the main file:
%
\begin{center}
\begin{tabular}{l}
|\||ifdefined\childdocname\endinput\||fi\newif\ifchilddoc|\\
|\edef\childdocname{\scantokens\expandafter{\jobname\noexpand}}|\\
|\def\childdocmain{|\textit{main}|}\||ifx\childdocmain\childdocname\||else|\\
|\childdoctrue\includeonly{\childdocname}\let\jobname\childdocmain\||fi|\\
\end{tabular}
\end{center}
%
Instead of |\childdocof{|\textit{main}|}| just include the main file
at the top of each child file:
%
\begin{center}
|\input{|\textit{main}|}|
\end{center}
%
A simple redirection |\childdocforward{|\textit{dest}|}| is achieved by:
%
\begin{center}
|\def\jobname{|\textit{dest}|}\input{\jobname}|
\end{center}
%
The redirection with prefix
|\childdocforwardprefix[|\textit{prefix}|]{|\textit{dest}|}|
is accomplished by:
%
\begin{center}
\begin{tabular}{l}
|{\edef\jobname{\scantokens\expandafter{\jobname\noexpand}}|\\
|\def\redirectjob |\textit{prefix}|#1~~~{\gdef\jobname{|\textit{dest}|#1}}|\\
|\expandafter\redirectjob\jobname~~~}\input{\jobname}|
\end{tabular}
\end{center}

In an alternative approach,
child documents can be compiled by a specific command line
without additional code or specific definitions:
%
\begin{center}
|... -jobname "|\textit{target}|" "|[\textit{flags}]%
|\includeonly{|\textit{dest}|}\input{|\textit{main}|}"|
\end{center}
%

%%%%%%%%%%%%%%%%%%%%%%%%%%%%%%%%%%%%%%%%%%%%%%%%%%%%%%%%%%%%%%%%%%%%%%%%%%%%%%%%
%%%%%%%%%%%%%%%%%%%%%%%%%%%%%%%%%%%%%%%%%%%%%%%%%%%%%%%%%%%%%%%%%%%%%%%%%%%%%%%%
\section{Information}

%%%%%%%%%%%%%%%%%%%%%%%%%%%%%%%%%%%%%%%%%%%%%%%%%%%%%%%%%%%%%%%%%%%%%%%%%%%%%%%%
\subsection{Copyright}

Copyright \copyright{} 2017--2018 Niklas Beisert

This work may be distributed and/or modified under the
conditions of the \LaTeX{} Project Public License, either version 1.3
of this license or (at your option) any later version.
The latest version of this license is in
  \url{http://www.latex-project.org/lppl.txt}
and version 1.3 or later is part of all distributions of \LaTeX{}
version 2005/12/01 or later.

This work has the LPPL maintenance status `maintained'.

The Current Maintainer of this work is Niklas Beisert.

This work consists of the files |README.txt|, |childdoc.ins| and |childdoc.dtx|
as well as the derived files |childdoc.def|, |cdocsamp.tex|
with |cdocsch1.tex|, |cdocsch2.tex|, |cdocspt3.tex|, |cdocspt4.tex|,
|cdocsdrf.tex|, |cdocsfn1.tex|, |cdocsfn2.tex|
as well as |childdoc.pdf|.

%%%%%%%%%%%%%%%%%%%%%%%%%%%%%%%%%%%%%%%%%%%%%%%%%%%%%%%%%%%%%%%%%%%%%%%%%%%%%%%%
\subsection{Files and Installation}

The package consists of the files:
%
\begin{center}
\begin{tabular}{ll}
    |README.txt|   & readme file \\
    |childdoc.ins| & installation file \\
    |childdoc.dtx| & source file \\
    |childdoc.def| & definition file \\
    |cdocsamp.tex| & sample main file \\
    |cdocsch1.tex| & sample include file \\
    |cdocsch2.tex| & sample include file \\
    |cdocspt3.tex| & sample part file \\
    |cdocspt4.tex| & sample part file \\
    |cdocsdrf.tex| & sample redirection file \\
    |cdocsfn1.tex| & sample redirection file \\
    |cdocsfn2.tex| & sample redirection file \\
    |childdoc.pdf| & manual
\end{tabular}
\end{center}
%
The distribution consists of the files
|README.txt|, |childdoc.ins| and |childdoc.dtx|.
%
\begin{itemize}
\item
Run (pdf)\LaTeX{} on |childdoc.dtx|
to compile the manual |childdoc.pdf| (this file).
\item
Run \LaTeX{} on |childdoc.ins| to create the definitions file |childdoc.def|
and the sample |cdocsamp.tex| with include files
|cdocsch1.tex|, |cdocsch2.tex|, |cdocspt3.tex|, |cdocspt4.tex|,
|cdocsdrf.tex|, |cdocsfn1.tex|, |cdocsfn2.tex|.
Then copy the file |childdoc.def| to an appropriate directory of your \LaTeX{}
distribution, e.g.\ \textit{texmf-root}|/tex/latex/childdoc|.
\end{itemize}

%%%%%%%%%%%%%%%%%%%%%%%%%%%%%%%%%%%%%%%%%%%%%%%%%%%%%%%%%%%%%%%%%%%%%%%%%%%%%%%%
\subsection{Related CTAN Packages}

There are several other packages which offer a similar functionality:
%
\begin{itemize}
\item
The packages
\href{http://ctan.org/pkg/docmute}{\textsf{docmute}},
\href{http://ctan.org/pkg/includex}{\textsf{includex}} and
\href{http://ctan.org/pkg/standalone}{\textsf{standalone}}
provide commands to include only the document body of
a child file thus allowing both files to be compiled individually.
\item
The packages \href{http://ctan.org/pkg/subdocs}{\textsf{subdocs}}
and \href{http://ctan.org/pkg/subfiles}{\textsf{subfiles}}
provide structures in which the main and child documents can be
encapsulated and allowing them to be compiled individually.
The inclusion mechanism is different from the conventional |\include|.
\item
The package \href{http://ctan.org/pkg/combine}{\textsf{combine}}
is an elaborate solution to combine several documents into one.
\end{itemize}
%
See also the CTAN topic \href{http://ctan.org/topic/subdocs}{\textsf{subdocs}}
for further related packages.
The present package differs from the above solutions in that
a document structure constructed with the conventional |\include| mechanism
just needs two extra commands at the top of every file
such that all constituent files can be compiled individually.

%%%%%%%%%%%%%%%%%%%%%%%%%%%%%%%%%%%%%%%%%%%%%%%%%%%%%%%%%%%%%%%%%%%%%%%%%%%%%%%%
%\subsection{Feature Suggestions}
%
%The following is a list of features which may be useful for future
%versions of this package:
%%
%\begin{itemize}
%\item
%\ldots
%\end{itemize}

%%%%%%%%%%%%%%%%%%%%%%%%%%%%%%%%%%%%%%%%%%%%%%%%%%%%%%%%%%%%%%%%%%%%%%%%%%%%%%%%
\subsection{Revision History}

%%%%%%%%%%%%%%%%%%%%%%%%%%%%%%%%%%%%%%%%
\paragraph{v2.0:} 2018/12/30

\begin{itemize}
\item
immediate forward processing
\item
added |\childdocby| mechanism
\item
manual restructured
\end{itemize}

%%%%%%%%%%%%%%%%%%%%%%%%%%%%%%%%%%%%%%%%
\paragraph{v1.6:} 2018/01/17

\begin{itemize}
\item
application for development of include files
\item
corrections to manual
\end{itemize}

%%%%%%%%%%%%%%%%%%%%%%%%%%%%%%%%%%%%%%%%
\paragraph{v1.5:} 2017/05/21

\begin{itemize}
\item
more complete structuring introduced
\item
|\childdocof| introduced
\item
|\childdoc| renamed to |\childdocmain|
\item
|\childredirect| renamed to |\childdocforward| and |\childdocforwardprefix|
and functionality expanded
\end{itemize}

%%%%%%%%%%%%%%%%%%%%%%%%%%%%%%%%%%%%%%%%
\paragraph{v1.0:} 2017/04/27

\begin{itemize}
\item
manual and install package
\item
first version published on CTAN
\end{itemize}

%%%%%%%%%%%%%%%%%%%%%%%%%%%%%%%%%%%%%%%%
\paragraph{v0.6:} 2017/04/26

\begin{itemize}
\item
redirection mechanism added
\end{itemize}

%%%%%%%%%%%%%%%%%%%%%%%%%%%%%%%%%%%%%%%%
\paragraph{v0.5:} 2017/04/26

\begin{itemize}
\item
functionality in definition file
\end{itemize}


%%%%%%%%%%%%%%%%%%%%%%%%%%%%%%%%%%%%%%%%%%%%%%%%%%%%%%%%%%%%%%%%%%%%%%%%%%%%%%%%
%%%%%%%%%%%%%%%%%%%%%%%%%%%%%%%%%%%%%%%%%%%%%%%%%%%%%%%%%%%%%%%%%%%%%%%%%%%%%%%%
%%%%%%%%%%%%%%%%%%%%%%%%%%%%%%%%%%%%%%%%%%%%%%%%%%%%%%%%%%%%%%%%%%%%%%%%%%%%%%%%
\appendix

\settowidth\MacroIndent{\rmfamily\scriptsize 000\ }

 \DocInput{childdoc.dtx}

\end{document}
%</driver>
% \fi
%
% %%%%%%%%%%%%%%%%%%%%%%%%%%%%%%%%%%%%%%%%%%%%%%%%%%%%%%%%%%%%%%%%%%%%%%%%%%%%%%
% %%%%%%%%%%%%%%%%%%%%%%%%%%%%%%%%%%%%%%%%%%%%%%%%%%%%%%%%%%%%%%%%%%%%%%%%%%%%%%
% \section{Sample}
%\iffalse
%<*samplemain>
%\fi
%
% The following presents a sample document
% with two chapters, two parts, a title page,
% a compile flag as well as three forwarding files to set the flag.
% It consists of eight |.tex| files:
% \begin{center}
% \begin{tabular}{ll}
% |cdocsamp.tex|&main file\\
% |cdocsch1.tex|&include file for chapter 1\\
% |cdocsch2.tex|&include file for chapter 2\\
% |cdocspt3.tex|&include file for part 3\\
% |cdocspt4.tex|&include file for part 4\\
% |cdocsdrf.tex|&forwarding file for main file in draft mode\\
% |cdocsfi1.tex|&forwarding file for final version of chapter 1\\
% |cdocsfi2.tex|&forwarding file for final version of chapter 2\\
% \end{tabular}
% \end{center}
% Each of the eight files can be compiled directly by the \LaTeX{} compiler.
%
% %%%%%%%%%%%%%%%%%%%%%%%%%%%%%%%%%%%%%%
% \paragraph{Main File.}
%
% The main file is called |cdocsamp.tex|.
%
% Load the \textsf{childdoc} definitions and
% declare the filename for the main document:
%    \begin{macrocode}
\input{childdoc.def}
\childdocmain{}
%    \end{macrocode}

% Optional override for |\version| flag:
%    \begin{macrocode}
%%\ifchilddoc\else\providecommand{\version}{draft}\fi
%    \end{macrocode}

% Define the default values for the |\version| flag
% (|final| for the main file and |draft| for childs):
%    \begin{macrocode}
\ifchilddoc
\providecommand{\version}{draft}
\else
\providecommand{\version}{final}
\fi
%    \end{macrocode}

% Load the standard document class:
%    \begin{macrocode}
\documentclass[12pt]{article}
%    \end{macrocode}

% Start the document body:
%    \begin{macrocode}
\begin{document}
%    \end{macrocode}

% Declare a title page.
% Print title, part of document being processed and version flag:
%    \begin{macrocode}
\addtocounter{page}{-1}
\begin{center}
{\LARGE\bfseries{}childdoc example\par}
\vspace{1cm}
\ifchilddoc
\ifchilddocmanual part\else chapter\fi:
`\childdocname' of `\childdocjob'\par
\else
main document: `\childdocjob'\par
\fi
version: \version\par
\end{center}
\newpage
%    \end{macrocode}

% Manually include selected file,
% otherwise process as usual:
%    \begin{macrocode}
\ifchilddocmanual
\section*{part `\childdocname'}
\input{\childdocname}
\else
%    \end{macrocode}

% Include the two chapters:
%    \begin{macrocode}
\include{cdocsch1}
\include{cdocsch2}
%    \end{macrocode}

% Include the two parts unless only chapters should be displayed:
%    \begin{macrocode}
\ifchilddoc\else
\section{part three}
\input{cdocspt3}
\section{part four}
\input{cdocspt4}
\fi
%    \end{macrocode}

% Process as usual until here:
%    \begin{macrocode}
\fi
%    \end{macrocode}

% End of document body:
%    \begin{macrocode}
\end{document}
%    \end{macrocode}
%\iffalse
%</samplemain>
%\fi
%
% %%%%%%%%%%%%%%%%%%%%%%%%%%%%%%%%%%%%%%
% \paragraph{Chapter Include Files.}
%
% The include files are called |cdocsch1.tex| and |cdocsch2.tex|.
%
%\iffalse
%<*samplechap1|samplechap2>
%\fi

% Optional override for |\version| flag:
%    \begin{macrocode}
%%\providecommand{\version}{final}
%    \end{macrocode}

% Include the main document:
%    \begin{macrocode}
\input{childdoc.def}
\childdocof{cdocsamp}
%    \end{macrocode}

%\iffalse
%</samplechap1|samplechap2>
%\fi
%
%\iffalse
%<*samplechap1>
%\fi
% Some text for chapter 1:
%    \begin{macrocode}
\section{one}
some text in chapter one
%    \end{macrocode}

%\iffalse
%</samplechap1>
%\fi
% Some text for chapter 2:
%\iffalse
%<*samplechap2>
%\fi
%    \begin{macrocode}
\section{two}
more text in chapter two
%    \end{macrocode}

%\iffalse
%</samplechap2>
%\fi
%
% %%%%%%%%%%%%%%%%%%%%%%%%%%%%%%%%%%%%%%
% \paragraph{Part Include Files.}
%
% The include files are called |cdocspt3.tex| and |cdocspt4.tex|.
%
%\iffalse
%<*samplepart3|samplepart4>
%\fi

% Optional override for |\version| flag:
%    \begin{macrocode}
%%\providecommand{\version}{final}
%    \end{macrocode}

% Include the main document:
%    \begin{macrocode}
\input{childdoc.def}
\childdocby{cdocsamp}
%    \end{macrocode}

%\iffalse
%</samplepart3|samplepart4>
%\fi
%
%\iffalse
%<*samplepart3>
%\fi
% Some text for part 3:
%    \begin{macrocode}
some text in part three
%    \end{macrocode}

%\iffalse
%</samplepart3>
%\fi
% Some text for part 4:
%\iffalse
%<*samplepart4>
%\fi
%    \begin{macrocode}
more text in part four
%    \end{macrocode}

%\iffalse
%</samplepart4>
%\fi
%
% %%%%%%%%%%%%%%%%%%%%%%%%%%%%%%%%%%%%%%
% \paragraph{Forwarding for a Complete Draft.}
%
% The following forwarding file |cdocsdrf.tex|
% compiles the main document in draft mode:
%\iffalse
%<*sampledraft>
%\fi
%    \begin{macrocode}
\def\version{draft}
\input{childdoc.def}
\childdocforward{cdocsamp}
%    \end{macrocode}

%\iffalse
%</sampledraft>
%\fi
%
% %%%%%%%%%%%%%%%%%%%%%%%%%%%%%%%%%%%%%%
% \paragraph{Forwarding for Final Version of the Chapters.}
%
% The following forwarding files |cdocsfn1.tex| and |cdocsfn2.tex|
% (with identical content)
% compile the final versions of the child documents
% |cdocsch1.tex| and |cdocsch2.tex|, respectively:
%\iffalse
%<*samplefinal>
%\fi
%    \begin{macrocode}
\def\version{final}
\input{childdoc.def}
\childdocforwardprefix[cdocsamp]{cdocsfn}{cdocsch}
%    \end{macrocode}

%\iffalse
%</samplefinal>
%\fi
%
% %%%%%%%%%%%%%%%%%%%%%%%%%%%%%%%%%%%%%%
% \paragraph{Command Line Processing.}
%
% The following three command lines generate the output files
% |cdocscld|, |cdocscl1| and |cdocscl2|
% which should be identical to
% |cdocsdrf|, |cdocsch1| and |cdocsfn2|, respectively:
% \begin{center}
% \begin{tabular}{l}
% |latex -jobname cdocscld \|\\
% |  "\def\version{draft}\input{childdoc.def}\childdocforward{cdocsamp}"|\\
% |latex -jobname cdocscl1 \|\\
% |  "\input{childdoc.def}\childdocforward[cdocsamp]{cdocsch1}"|\\
% |latex -jobname cdocscl2 \|\\
% |  "\def\version{final}\input{childdoc.def}\childdocforward{cdocsch2}"|
% \end{tabular}
% \end{center}
% Note that the trailing backslash on each first line
% merely continues the input to the second line
% (for convenient cut ant paste).
% Furthermore, the command |latex| can be replaced by any
% of its alternative versions such as |pdflatex|.
%
% %%%%%%%%%%%%%%%%%%%%%%%%%%%%%%%%%%%%%%%%%%%%%%%%%%%%%%%%%%%%%%%%%%%%%%%%%%%%%%
% %%%%%%%%%%%%%%%%%%%%%%%%%%%%%%%%%%%%%%%%%%%%%%%%%%%%%%%%%%%%%%%%%%%%%%%%%%%%%%
% \section{Implementation}
%\iffalse
%<*package>
%\fi
%
% This section describes the definitions file |childdoc.def|.

% The definitions cannot be loaded using |\usepackage| or |\RequirePackage|
% which has a mechanism to prevent loading a style file more than once.
% When loading the definitions by means of |\input|
% multiple instances have to be prevented manually:
%\iffalse
%This code needs to be before the `\ProvidesFile' directive
%which is defined at the beginning of this file.
%Therefore it is also placed there and commented out here.
%</package>
%<*discard>
%\fi
%    \begin{macrocode}
\ifdefined\childdocmain\endinput\fi
%    \end{macrocode}
%\iffalse
%</discard>
%<*package>
%\fi
%
% \macro{\ifchilddoc}
% \macro{\ifchilddocmanual}
% The conditional |\ifchilddoc| tells whether a
% child (true) or main (false) document is being compiled.
% The conditional |\ifchilddocmanual| tells whether
% the |\includeonly| mechanism is used (false) or
% the selection of child files must be performed manually (true).
% The definitions initialise to false:
%    \begin{macrocode}
\newif\ifchilddoc
\newif\ifchilddocmanual
%    \end{macrocode}

% \macro{\childdocname}
% \macro{\childdocjob}
% The macro |\childdocname| stores the name of the main document
% to be compiled. The macro |\childdocjob| stores the name of
% the document on which the \LaTeX{} compiler was originally invoked.
% The content of |\jobname| cannot be compared
% to filenames specified in the source due to different catcodes.
% The following code rescans |\jobname|, stores the result
% in |\childdocname| and saves a copy in |\childdocjob|:
%    \begin{macrocode}
\edef\childdocname{\scantokens\expandafter{\jobname\noexpand}}
\let\childdocjob\childdocname
%    \end{macrocode}

% \macro{\childdocdisable}
% The macro |\childdocdisable| prevents the main file
% from being processed more than once.
% At this stage, the main document command |\childdocmain|
% is assumed to be called once again where it should do nothing.
% Any subsequent call to it should prevent
% a secondary processing of the main document
% It overwrites the forwarding commands
% |\childdocof| and |\childdocforward|
% with empty macros to prevent further inclusions of the main document:
%    \begin{macrocode}
\newcommand{\childdocdisable}
{
  \renewcommand{\childdocmain}[1]{\renewcommand{\childdocmain}[1]{\endinput}}
  \renewcommand{\childdocof}[1]{}
  \renewcommand{\childdocby}[2][]{}
  \renewcommand{\childdocforward}[2][]{}
  \renewcommand{\childdocdisable}{}
}
%    \end{macrocode}

% \macro{\childdocmain}
% The macro |\childdocmain| is to be called at the top of the main file
% with nothing or the main filename (without extension) as argument.
% First, it breaks loops.
% If the argument is not empty and does not match |\childdocname|
% (which is set by the first inclusion of |childdoc.def|),
% |\ifchilddoc| is set to true, |\includeonly| is applied to the child file
% and |\jobname| is set to the main file
% (for proper handling of |.aux| files):
%    \begin{macrocode}
\newcommand{\childdocmain}[1]
{
  \childdocdisable\childdocmain{}
  \if?#1?\else
    \begingroup
      \def\childdoctmp{#1}
      \ifx\childdoctmp\childdocname
        \def\childdoctmp{}
      \else
        \def\childdoctmp
        {
          \childdoctrue
          \includeonly{\childdocname}
          \def\childdocjob{#1}
          \def\jobname{#1}
        }
      \fi
      \expandafter
    \endgroup
    \childdoctmp
  \fi
}
%    \end{macrocode}

% \macro{\childdocof}
% The command |\childdocof| redirects
% compilation to the main file |#1|.
%    \begin{macrocode}
\newcommand{\childdocof}[1]
{
  \childdocdisable
  \childdoctrue
  \includeonly{\childdocname}
  \def\jobname{#1}
  \def\childdocjob{#1}
  \input{#1}
}
%    \end{macrocode}

% \macro{\childdocby}
% The command |\childdocby| ....
%    \begin{macrocode}
\newcommand{\childdocby}[2][]
{
  \childdocdisable
  \childdoctrue
  \childdocmanualtrue
  \if?#1?\else
    \def\jobname{#2}
  \fi
  \def\childdocjob{#2}
  \input{#2}
  \endinput
}
%    \end{macrocode}

% \macro{\childdocforward}
% The command |\childdocforward| redirects
% compilation to the main file or
% (if the optional argument is given) a child file.
% Parameters are set as if the main file
% or a child file starting with |\childdocof| was compiled.
% Then compilation is handed over to the main file:
%    \begin{macrocode}
\newcommand{\childdocforward}[2][]
{
  \begingroup
    \if?#1?
      \def\childdoctmp
      {
        \def\childdocname{#2}
        \def\childdocjob{#2}
        \def\jobname{#2}
        \input{#2}
        \endinput
      }
    \else
      \def\childdoctmp
      {
        \childdocdisable
        \def\childdocname{#2}
        \childdoctrue
        \includeonly{#2}
        \def\childdocjob{#1}
        \def\jobname{#1}
        \input{#1}
        \endinput
      }
    \fi
    \expandafter
  \endgroup
  \childdoctmp
}
%    \end{macrocode}

% \macro{\childdocforwardprefix}
% The command |\childdocforwardprefix| redirects
% compilation to the main or a child file by means of a pattern.
% The prefix |#1| in the current filename is replaced by |#2|
% and the suffix of the current filename is kept
% (it is assumed that the filename does not contain the substring `|~~~|'
% which is used as a delimiter).
% Compilation is handed over to the new file by |\childdocforward|:
%    \begin{macrocode}
\newcommand{\childdocforwardprefix}[3][]
{
  \begingroup
    \def\childdocextract #2##1~~~{\def\childdoctmp{\childdocforward[#1]{#3##1}}}
    \expandafter\childdocextract\childdocname~~~
    \expandafter
  \endgroup
  \childdoctmp
}
%    \end{macrocode}

% \macro{\childdoc}
% The deprecated macro |\childdoc| is a legacy version of |\childdocmain|:
%    \begin{macrocode}
\newcommand{\childdoc}{\childdocmain}
%    \end{macrocode}

% \macro{\childdocredirect}
% The deprecated macro |\childdocredirect| is a legacy version
% of |\childdocforward| and |\childdocforwardprefix|:
%    \begin{macrocode}
\newcommand{\childdocredirect}[2][]
{
  \begingroup
    \if?#1?
      \def\childdoctmp{\childdocforward{#2}}
    \else
      \def\childdoctmp{\childdocforwardprefix{#1}{#2}}
    \fi
    \expandafter
  \endgroup
  \childdoctmp
}
%    \end{macrocode}

%\iffalse
%</package>
%\fi
%
\endinput
|\\
|\childdocof{|\textit{main}|}|\\
\end{tabular}
\end{center}
at the top of every child file \textit{child}
which is included by |\include{|\textit{child}|}|
from within the main file
(or at least for those files to be compiled individually).
The argument \textit{main} must be the filename of the main file.

There are a couple of
considerations in setting up the main and child documents:

%%%%%%%%%%%%%%%%%%%%%%%%%%%%%%%%%%%%%%%%
\paragraph{Restrictions.}

Please note the following restrictions:
\begin{itemize}
\item
|\childdocmain| must be called with one argument \textit{main}
to ensure compatibility with earlier version of the package.
It must either be empty (|\childdocmain{}|)
or precisely match the filename of the main file in which it is specified.
See \secref{sec:detection} for further information.
\item
The filename \textit{main} must be specified without the |.tex| extension.
\item
The filename \textit{main} is case sensitive
(even in case-insensitive file systems)
due to internal string comparison.
\item
The argument \textit{main} should be fully expanded, it cannot be a macro.
\item
Subdirectories and special characters should be avoided in filenames.
\item
The command |\childdocmain{|\textit{main}|}| must be followed by a whitespace.
It should not be followed immediately by another command
or by a comment mark `|%|'.
This is because the \TeX{} parser reads the token immediately following
the argument of |\childdocmain| and puts it
at the beginning of every child section;
however, a white\-space is ignored.
\end{itemize}

%%%%%%%%%%%%%%%%%%%%%%%%%%%%%%%%%%%%%%%%
\paragraph{Content of Main File.}

It is advisable to place all content in the child files included by |\include|.
Any output contained in the main file will appear in all child documents
unless suppressed manually;
it cannot be suppressed automatically by the |\includeonly| directive
and thus should normally be avoided.
A method to include some content in the main file
by means of conditional processing is described in \secref{sec:conditional}.

%%%%%%%%%%%%%%%%%%%%%%%%%%%%%%%%%%%%%%%%
\paragraph{Page Numbering.}

When only a part of the document is compiled,
the appropriate numbering of pages
(as well as other status parameters)
is determined from the |.aux| files.
The latter contain information from previous passes.
However this information needs to propagate through
all intermediate child documents.
Therefore the page numbering in child documents may well
be inconsistent until the complete document is compiled at least once.

A useful (if unconventional) way to always ensure a consistent
page numbering is to restart the numbering in each child document
and denote the pages by `\textit{child}|.|\textit{page}'
where \textit{child} represents the chapter/section number of the child file.
This can be achieved by the command
|\numberwithin{page}{|\textit{child}|}|
of the \textsf{amsmath} package
where \textit{child} can be |chapter| or |section|
depending on the chosen structuring.
Alternatively, one can modify the macro |\thepage| appropriately
and reset the counter |page| at the start of each child file.

%%%%%%%%%%%%%%%%%%%%%%%%%%%%%%%%%%%%%%%%%%%%%%%%%%%%%%%%%%%%%%%%%%%%%%%%%%%%%%%%
\subsection{Conditional Processing}
\label{sec:conditional}

The package provides a mechanism to compile different versions
of a document. To customise the versions further some conditional processing
can come in handy to distinguish which version is being compiled.
The package provides two macros to describe the compilation context:

%%%%%%%%%%%%%%%%%%%%%%%%%%%%%%%%%%%%%%%%
\DescribeMacro{\ifchilddoc}
The conditional |\ifchilddoc| distinguishes between the compilation of
child documents and the main document:
%
\begin{center}
|\ifchilddoc |\textit{child-code}| |[|\||else |\textit{main-code}]| \||fi|
\end{center}

%%%%%%%%%%%%%%%%%%%%%%%%%%%%%%%%%%%%%%%%
\DescribeMacro{\childdocname}
\DescribeMacro{\childdocjob}
The macro |\childdocname| contains the filename (without extension)
of the main or child file being processed.
Note that |\childdocjob| will always contain the name of the main file.

%%%%%%%%%%%%%%%%%%%%%%%%%%%%%%%%%%%%%%%%
\paragraph{Title Page.}

Conditional processing can be used to include a title or banner page
in the main document when proper precautions are taken.
Importantly, the code in the main file should ensure that the page counter
(as well as other status parameters which are stored in the |.aux| files)
takes the same value after the conditional processing.
Otherwise the page numbers may take divergent values
depending on which part is compiled.

For example, a title page could be declared by:
%
\begin{center}
\begin{tabular}{l}
|\ifchilddoc\||else|\\
|\addtocounter{page}{-1}|\\
\textit{code for title page}\\
|\newpage|\\
|\||fi|
\end{tabular}
\end{center}
%
A banner page for the child documents can be generated by:
%
\begin{center}
\begin{tabular}{l}
|\ifchilddoc|\\
|\addtocounter{page}{-1}|\\
\textit{code for banner page}\\
|\newpage|\\
|\||fi|
\end{tabular}
\end{center}
%
Here one could write a message such as:
\begin{center}
|This is the part \childdocname{} of \childdocjob{}.|
\end{center}

%%%%%%%%%%%%%%%%%%%%%%%%%%%%%%%%%%%%%%%%%%%%%%%%%%%%%%%%%%%%%%%%%%%%%%%%%%%%%%%%
\subsection{Flags}
\label{sec:flags}

The package makes it easy to generate different versions
of the main or child documents.
To this end compilation flags can be defined
and assigned different default values.
They will be particularly useful in conjunction
with the forwarding mechanism described in \secref{sec:forward}.

For example, it may be useful to have a flag |\version|
which can be set to |draft| or |final|.
The document source will contain some conditional code
depending on the value of |\version|.
Suppose further, the flag should default to |final| for the main file
and to |draft| for child files
which is a natural assignment for editing the document.
This is achieved by placing the following code
in the preamble of the main document
(below the |\childdocmain| directive):
%
\begin{center}
\begin{tabular}{l}
|\ifchilddoc|\\
|\providecommand{\version}{draft}|\\
|\||else|\\
|\providecommand{\version}{final}|\\
|\||fi|
\end{tabular}
\end{center}
%
The definition by |\providecommand| makes sure
that previous definitions are not overwritten.
Further statements |\providecommand{\version}{...}|
can thus be added before the above code to override it.

For the main file, one might add a line
(between |\childdocmain| and the above block)
%
\begin{center}
|%\ifchilddoc\||else\providecommand{\version}{draft}\||fi|
\end{center}
%
which can be uncommented to produce a draft version.
Likewise one can add a line to the very top of a child file
(above the |\childdocof{|\textit{main}|}| directive)
%
\begin{center}
|%\providecommand{\version}{final}|
\end{center}
%
which can be uncommented to produce the final version of this child document.

%%%%%%%%%%%%%%%%%%%%%%%%%%%%%%%%%%%%%%%%%%%%%%%%%%%%%%%%%%%%%%%%%%%%%%%%%%%%%%%%
\subsection{Forwarding}
\label{sec:forward}

Different versions of the main or child documents
using compilation flags as described in \secref{sec:flags}
can be (permanently) stored in different files
for convenient compilation, viewing and distribution.
To this end, the package defines a command
to pass on compilation to a different file:

%%%%%%%%%%%%%%%%%%%%%%%%%%%%%%%%%%%%%%%%
\DescribeMacro{\childdocforward}
The command |\childdocforward| redirects processing to
another source file:
%
\begin{center}
\begin{tabular}{l}
|% \iffalse
%
% childdoc.dtx Copyright (C) 2017-2018 Niklas Beisert
%
% This work may be distributed and/or modified under the
% conditions of the LaTeX Project Public License, either version 1.3
% of this license or (at your option) any later version.
% The latest version of this license is in
%   http://www.latex-project.org/lppl.txt
% and version 1.3 or later is part of all distributions of LaTeX
% version 2005/12/01 or later.
%
% This work has the LPPL maintenance status `maintained'.
%
% The Current Maintainer of this work is Niklas Beisert.
%
% This work consists of the files childdoc.dtx and childdoc.ins
% and the derived files childdoc.def and cdocsamp.tex with
% cdocsch1.tex, cdocsch2.tex, cdocsdrf.tex, cdocsfn1.tex, cdocsfn2.tex.
%
%<package>\ifdefined\childdocmain\endinput\fi
%<package>\ProvidesFile{childdoc.def}[2018/12/30 v2.0 child document driver]
%<samplemain>\ProvidesFile{cdocsamp.tex}[2018/12/30 v2.0 sample for childdoc]
%<*driver>
%\ProvidesFile{childdoc.drv}[2018/12/30 v2.0 childdoc reference manual file]
\PassOptionsToClass{10pt,a4paper}{article}
\documentclass{ltxdoc}

\usepackage[margin=35mm]{geometry}
\usepackage{hyperref}
\usepackage{hyperxmp}
\usepackage[usenames]{color}

\hypersetup{colorlinks=true}
\hypersetup{pdfstartview=FitH}
\hypersetup{pdfpagemode=UseNone}
\hypersetup{pdfsource={}}
\hypersetup{pdflang={en-UK}}
\hypersetup{pdfcopyright={Copyright 2017-2018 Niklas Beisert.
  This work may be distributed and/or modified under the
  conditions of the LaTeX Project Public License, either version 1.3
  of this license or (at your option) any later version.}}
\hypersetup{pdflicenseurl={http://www.latex-project.org/lppl.txt}}
\hypersetup{pdfcontactaddress={ETH Zurich, ITP, HIT K,
  Wolfgang-Pauli-Strasse 27}}
\hypersetup{pdfcontactpostcode={8093}}
\hypersetup{pdfcontactcity={Zurich}}
\hypersetup{pdfcontactcountry={Switzerland}}
\hypersetup{pdfcontactemail={nbeisert@itp.phys.ethz.ch}}
\hypersetup{pdfcontacturl={http://people.phys.ethz.ch/\xmptilde nbeisert/}}

\newcommand{\secref}[1]{\hyperref[#1]{section \ref*{#1}}}

\parskip1ex
\parindent0pt
\let\olditemize\itemize
\def\itemize{\olditemize\parskip0pt}

\begin{document}

\title{The \textsf{childdoc} Package}
\hypersetup{pdftitle={The childdoc Package}}
\author{Niklas Beisert\\[2ex]
  Institut f\"ur Theoretische Physik\\
  Eidgen\"ossische Technische Hochschule Z\"urich\\
  Wolfgang-Pauli-Strasse 27, 8093 Z\"urich, Switzerland\\[1ex]
  \href{mailto:nbeisert@itp.phys.ethz.ch}
  {\texttt{nbeisert@itp.phys.ethz.ch}}}
\hypersetup{pdfauthor={Niklas Beisert}}
\hypersetup{pdfsubject={Manual for the LaTeX2e Package childdoc}}
\date{30 December 2018, \textsf{v2.0}}
\maketitle

\begin{abstract}\noindent
\textsf{childdoc} is a \LaTeXe{} package
that enables the direct compilation
of document sections included by |\include|
to individual files.
\end{abstract}

\begingroup
\parskip0ex
\tableofcontents
\endgroup

%%%%%%%%%%%%%%%%%%%%%%%%%%%%%%%%%%%%%%%%%%%%%%%%%%%%%%%%%%%%%%%%%%%%%%%%%%%%%%%%
%%%%%%%%%%%%%%%%%%%%%%%%%%%%%%%%%%%%%%%%%%%%%%%%%%%%%%%%%%%%%%%%%%%%%%%%%%%%%%%%
\section{Introduction}

\LaTeX{} provides a mechanism to structure a large document (such as a book)
into a main file and several child files (containing the chapters)
using the |\include| command.
This mechanism is beneficial for documents
which span hundreds of pages in order to
make the source file(s) more manageable.
Moreover, compilation can be restricted to
selected child files by means of the |\includeonly| command.
The latter feature can be used to reduce the compilation time while editing
(this was significantly more useful in the earlier days of \LaTeX{})
or to generate a smaller document which is easier to navigate.
Another application of |\includeonly| is to generate
documents consisting of selected parts of the complete document.

However, there are a few drawbacks of the plain |\include| mechanism:
\begin{itemize}
\item
The child files cannot be compiled on their own,
they can only be compiled via the main file.
A naive editing environment
(such as a text editor with an option
to have the current file processed by \LaTeX)
may require one to switch to the main file before compiling;
attempting to compile the child file produces errors.
\item
The main file must be modified (each time)
to adjust the |\includeonly| command
to the present needs. This easily leaves the main file in a messy state.
\item
The generated document will always carry the filename
of the main document. This is inconvenient if
several child files are to be compiled and
to be kept for distribution.
\end{itemize}

The present package provides a simple interface
to make child files individually compilable by \LaTeX{}.
Compiling a child file then has the same effect as compiling
the main file with an |\includeonly| command
to select the appropriate child.
Moreover the generated document will carry the name of the child
rather than the main file.
This resolves all three above issues.

This feature is meant to make the editing of books,
thesis documents and lecture notes somewhat more convenient.
However, the package can also be used efficiently for
composing a series of documents (such as exercise sheets)
which are typically distributed individually.
It then assists the author in generating the individual documents
(potentially in different versions)
as well as a document containing the collected series.
Another application is in developing style files
or other kinds of included material
where compilation of the style file could redirect
to a sample or test file.

%%%%%%%%%%%%%%%%%%%%%%%%%%%%%%%%%%%%%%%%%%%%%%%%%%%%%%%%%%%%%%%%%%%%%%%%%%%%%%%%
%%%%%%%%%%%%%%%%%%%%%%%%%%%%%%%%%%%%%%%%%%%%%%%%%%%%%%%%%%%%%%%%%%%%%%%%%%%%%%%%
\section{Usage}

First of all, the package \textsf{childdoc} is \emph{not} a standard
\LaTeXe{} |.sty| style file! Therefore it needs to be invoked in
a non-standard way.

%%%%%%%%%%%%%%%%%%%%%%%%%%%%%%%%%%%%%%%%%%%%%%%%%%%%%%%%%%%%%%%%%%%%%%%%%%%%%%%%
\subsection{Included Files}
\label{sec:include}

%%%%%%%%%%%%%%%%%%%%%%%%%%%%%%%%%%%%%%%%
\DescribeMacro{\childdocmain}
To use the package, add the commands
\begin{center}
\begin{tabular}{l}
|\input{childdoc.def}|\\
|\childdocmain{}|\\
\end{tabular}
\end{center}
at the very top of the main \LaTeX{} file,
in particular \emph{before} the |\documentclass| statement!
The argument of |\childdocmain| should be left empty
(but it must be present).

%%%%%%%%%%%%%%%%%%%%%%%%%%%%%%%%%%%%%%%%
\DescribeMacro{\childdocof}
Furthermore, add the commands
\begin{center}
\begin{tabular}{l}
|\input{childdoc.def}|\\
|\childdocof{|\textit{main}|}|\\
\end{tabular}
\end{center}
at the top of every child file \textit{child}
which is included by |\include{|\textit{child}|}|
from within the main file
(or at least for those files to be compiled individually).
The argument \textit{main} must be the filename of the main file.

There are a couple of
considerations in setting up the main and child documents:

%%%%%%%%%%%%%%%%%%%%%%%%%%%%%%%%%%%%%%%%
\paragraph{Restrictions.}

Please note the following restrictions:
\begin{itemize}
\item
|\childdocmain| must be called with one argument \textit{main}
to ensure compatibility with earlier version of the package.
It must either be empty (|\childdocmain{}|)
or precisely match the filename of the main file in which it is specified.
See \secref{sec:detection} for further information.
\item
The filename \textit{main} must be specified without the |.tex| extension.
\item
The filename \textit{main} is case sensitive
(even in case-insensitive file systems)
due to internal string comparison.
\item
The argument \textit{main} should be fully expanded, it cannot be a macro.
\item
Subdirectories and special characters should be avoided in filenames.
\item
The command |\childdocmain{|\textit{main}|}| must be followed by a whitespace.
It should not be followed immediately by another command
or by a comment mark `|%|'.
This is because the \TeX{} parser reads the token immediately following
the argument of |\childdocmain| and puts it
at the beginning of every child section;
however, a white\-space is ignored.
\end{itemize}

%%%%%%%%%%%%%%%%%%%%%%%%%%%%%%%%%%%%%%%%
\paragraph{Content of Main File.}

It is advisable to place all content in the child files included by |\include|.
Any output contained in the main file will appear in all child documents
unless suppressed manually;
it cannot be suppressed automatically by the |\includeonly| directive
and thus should normally be avoided.
A method to include some content in the main file
by means of conditional processing is described in \secref{sec:conditional}.

%%%%%%%%%%%%%%%%%%%%%%%%%%%%%%%%%%%%%%%%
\paragraph{Page Numbering.}

When only a part of the document is compiled,
the appropriate numbering of pages
(as well as other status parameters)
is determined from the |.aux| files.
The latter contain information from previous passes.
However this information needs to propagate through
all intermediate child documents.
Therefore the page numbering in child documents may well
be inconsistent until the complete document is compiled at least once.

A useful (if unconventional) way to always ensure a consistent
page numbering is to restart the numbering in each child document
and denote the pages by `\textit{child}|.|\textit{page}'
where \textit{child} represents the chapter/section number of the child file.
This can be achieved by the command
|\numberwithin{page}{|\textit{child}|}|
of the \textsf{amsmath} package
where \textit{child} can be |chapter| or |section|
depending on the chosen structuring.
Alternatively, one can modify the macro |\thepage| appropriately
and reset the counter |page| at the start of each child file.

%%%%%%%%%%%%%%%%%%%%%%%%%%%%%%%%%%%%%%%%%%%%%%%%%%%%%%%%%%%%%%%%%%%%%%%%%%%%%%%%
\subsection{Conditional Processing}
\label{sec:conditional}

The package provides a mechanism to compile different versions
of a document. To customise the versions further some conditional processing
can come in handy to distinguish which version is being compiled.
The package provides two macros to describe the compilation context:

%%%%%%%%%%%%%%%%%%%%%%%%%%%%%%%%%%%%%%%%
\DescribeMacro{\ifchilddoc}
The conditional |\ifchilddoc| distinguishes between the compilation of
child documents and the main document:
%
\begin{center}
|\ifchilddoc |\textit{child-code}| |[|\||else |\textit{main-code}]| \||fi|
\end{center}

%%%%%%%%%%%%%%%%%%%%%%%%%%%%%%%%%%%%%%%%
\DescribeMacro{\childdocname}
\DescribeMacro{\childdocjob}
The macro |\childdocname| contains the filename (without extension)
of the main or child file being processed.
Note that |\childdocjob| will always contain the name of the main file.

%%%%%%%%%%%%%%%%%%%%%%%%%%%%%%%%%%%%%%%%
\paragraph{Title Page.}

Conditional processing can be used to include a title or banner page
in the main document when proper precautions are taken.
Importantly, the code in the main file should ensure that the page counter
(as well as other status parameters which are stored in the |.aux| files)
takes the same value after the conditional processing.
Otherwise the page numbers may take divergent values
depending on which part is compiled.

For example, a title page could be declared by:
%
\begin{center}
\begin{tabular}{l}
|\ifchilddoc\||else|\\
|\addtocounter{page}{-1}|\\
\textit{code for title page}\\
|\newpage|\\
|\||fi|
\end{tabular}
\end{center}
%
A banner page for the child documents can be generated by:
%
\begin{center}
\begin{tabular}{l}
|\ifchilddoc|\\
|\addtocounter{page}{-1}|\\
\textit{code for banner page}\\
|\newpage|\\
|\||fi|
\end{tabular}
\end{center}
%
Here one could write a message such as:
\begin{center}
|This is the part \childdocname{} of \childdocjob{}.|
\end{center}

%%%%%%%%%%%%%%%%%%%%%%%%%%%%%%%%%%%%%%%%%%%%%%%%%%%%%%%%%%%%%%%%%%%%%%%%%%%%%%%%
\subsection{Flags}
\label{sec:flags}

The package makes it easy to generate different versions
of the main or child documents.
To this end compilation flags can be defined
and assigned different default values.
They will be particularly useful in conjunction
with the forwarding mechanism described in \secref{sec:forward}.

For example, it may be useful to have a flag |\version|
which can be set to |draft| or |final|.
The document source will contain some conditional code
depending on the value of |\version|.
Suppose further, the flag should default to |final| for the main file
and to |draft| for child files
which is a natural assignment for editing the document.
This is achieved by placing the following code
in the preamble of the main document
(below the |\childdocmain| directive):
%
\begin{center}
\begin{tabular}{l}
|\ifchilddoc|\\
|\providecommand{\version}{draft}|\\
|\||else|\\
|\providecommand{\version}{final}|\\
|\||fi|
\end{tabular}
\end{center}
%
The definition by |\providecommand| makes sure
that previous definitions are not overwritten.
Further statements |\providecommand{\version}{...}|
can thus be added before the above code to override it.

For the main file, one might add a line
(between |\childdocmain| and the above block)
%
\begin{center}
|%\ifchilddoc\||else\providecommand{\version}{draft}\||fi|
\end{center}
%
which can be uncommented to produce a draft version.
Likewise one can add a line to the very top of a child file
(above the |\childdocof{|\textit{main}|}| directive)
%
\begin{center}
|%\providecommand{\version}{final}|
\end{center}
%
which can be uncommented to produce the final version of this child document.

%%%%%%%%%%%%%%%%%%%%%%%%%%%%%%%%%%%%%%%%%%%%%%%%%%%%%%%%%%%%%%%%%%%%%%%%%%%%%%%%
\subsection{Forwarding}
\label{sec:forward}

Different versions of the main or child documents
using compilation flags as described in \secref{sec:flags}
can be (permanently) stored in different files
for convenient compilation, viewing and distribution.
To this end, the package defines a command
to pass on compilation to a different file:

%%%%%%%%%%%%%%%%%%%%%%%%%%%%%%%%%%%%%%%%
\DescribeMacro{\childdocforward}
The command |\childdocforward| redirects processing to
another source file:
%
\begin{center}
\begin{tabular}{l}
|\input{childdoc.def}|\\
|\childdocforward[|\textit{main}|]{|\textit{dest}|}|\\
\end{tabular}
\end{center}
%
The argument \textit{dest} is the destination file
(without extension).
It should be the main file or one of the child files.
Note that further \textsf{childdoc} directives
such as |\childdocof| and |\childdocforward|
in the indicated file will be processed in this form.
The optional argument \textit{main}
passes on directly to the main file \textit{main}
while pretending to compile the child \textit{dest}.
This form behaves as if \textit{dest}
issues |\childdocof{|\textit{main}|}| right away,
and no further \textsf{childdoc} directives will be processed.

%%%%%%%%%%%%%%%%%%%%%%%%%%%%%%%%%%%%%%%%
\DescribeMacro{\...prefix}
In the alternative form |\childdocforwardprefix|,
%
\begin{center}
\begin{tabular}{l}
|\input{childdoc.def}|\\
|\childdocforwardprefix[|\textit{main}|]{|\textit{prefix}|}{|\textit{dest}|}|
\end{tabular}
\end{center}
%
the destination file is determined by a pattern
depending on the current file:
To make this work, the current file must be called
`{\textit{prefix}\hspace{0.2em}\textit{suffix}}'
with \textit{prefix} matching precisely the argument.
Processing is then passed on to the file
`{\textit{dest}\hspace{0.2em}\textit{suffix}}'.
Surely, the same effect is achieved by
directly specifying the
argument `{\textit{dest}\hspace{0.2em}\textit{suffix}}'
in the first form.
However, that requires to set up a different file
for each child. With the alternative form of the command
all these files can have exactly the same content
which simplifies setting them up and maintaining them.

For example, the following file |draft.tex|
with a compilation flag |\version| as described in \secref{sec:flags}
compiles the main document as a draft:
%
\begin{center}
\begin{tabular}{l}
|\def\version{draft}|\\
|\input{childdoc.def}|\\
|\childdocforward{|\textit{main}|}|
\end{tabular}
\end{center}
%
Likewise, the following files |final|\textit{nn}|.tex|
compile the final version of the child document
|child|\textit{nn}|.tex|:
%
\begin{center}
\begin{tabular}{l}
|\def\version{final}|\\
|\input{childdoc.def}|\\
|\childdocforwardprefix{final}{child}|
\end{tabular}
\end{center}
%

Note that when several versions of a main file and/or of each child file
are to be generated, it may be convenient to set up a |Makefile| or
shell script to automatise the process.

%%%%%%%%%%%%%%%%%%%%%%%%%%%%%%%%%%%%%%%%%%%%%%%%%%%%%%%%%%%%%%%%%%%%%%%%%%%%%%%%
\subsection{Command Line Processing}
\label{sec:commandline}

The effect of redirection files can also be achieved by invoking
the \LaTeX{} compiler with a more elaborate command line.
Most conveniently this should be done as part
of a shell script or a |Makefile|.

When using \textsf{childdoc} in the main file, the following
command lines effectively perform a redirection
(note that depending on the shell being used,
backslashes may have to be doubled: `|\|' $\to$ `|\\|'):
%
\begin{center}
|... -jobname "|\textit{target}|" |\\|"|[\textit{flags}]%
|\input{childdoc.def}\childdocforward[|\textit{main}|]{|\textit{dest}|}"|
\end{center}
%
Here \textit{target} is the name of the output file,
\textit{main} is the name of the main file
and \textit{dest} is the name of the main or child file to be processed
(all filenames without extensions).
The optional argument \textit{main} can be omitted
if \textit{main} matches \textit{dest}.
Optionally, compilation \textit{flags} can be defined via |\def| commands.
This command line makes the \TeX{} engine believe
it is compiling the file \textit{target}
whose content is specified as the latter parameter.
The provided code then forwards the processing to
\textit{main} or \textit{dest} as described in \secref{sec:forward}.

%%%%%%%%%%%%%%%%%%%%%%%%%%%%%%%%%%%%%%%%%%%%%%%%%%%%%%%%%%%%%%%%%%%%%%%%%%%%%%%%
\subsection{Include by Input}
\label{sec:input}

Including child documents by |\include| has some restrictions by design.
Most notably, the content of a child document always occupies
its own set of pages; pages cannot be shared between child documents.
Usually, this behaviour makes perfect sense
because each child document contain an essential part of the document.
However, in some situations it may be desirable to compose
a document from a collection of parts
without having mandatory page breaks between then.
For this case, the package
provides a mechanism to include parts
by |\input| which can also be processed individually.
However, by construction this mechanism
requires manual handling of the content to be output.

%%%%%%%%%%%%%%%%%%%%%%%%%%%%%%%%%%%%%%%%
\DescribeMacro{\ifchilddocmanual}
The main file should be prepared as usual, see \secref{sec:include}.
However, the document body must make a distinction
between processing of an individual part and of the main document, e.g.:
%
\begin{center}
\begin{tabular}{l}
|\ifchilddocmanual|\\
|\input{\childdocname}|\\
|\||else|\\
\textit{document body with }|\input{|\textit{part}|}|\\
|\||fi|
\end{tabular}
\end{center}
%
The conditional |\ifchilddocmanual| is true whenever
a part to be included by |\input| is being compiled,
and the name of the part is stored in |\childdocname|.

%%%%%%%%%%%%%%%%%%%%%%%%%%%%%%%%%%%%%%%%
\DescribeMacro{\childdocby}
Each part to be included by |\input| should start with:
%
\begin{center}
\begin{tabular}{l}
|\input{childdoc.def}|\\
|\childdocby{|\textit{main}|}|\\
\end{tabular}
\end{center}
%
The directive |\childdocby| is similar to |\childdocof|
described in \secref{sec:include},
but the subsequent selection of content must be done manually.
To that end, both |\ifchilddoc| and |\ifchilddocmanual|
will be true upon processing of a part,
and the name of the part is stored in |\childdocname|.
Note that |\jobname| will be set to the filename of the current part
so that each part receives an individual |.aux| file
that does not interfere with the |.aux| file(s) of the main document.
This behaviour can be altered by the alternative form
|\childdocby[*]{|\textit{main}|}| (with a non-empty optional argument)
which uses the |.aux| file of the main document
by setting |\jobname| to \textit{main}.

%%%%%%%%%%%%%%%%%%%%%%%%%%%%%%%%%%%%%%%%%%%%%%%%%%%%%%%%%%%%%%%%%%%%%%%%%%%%%%%%
\subsection{Driver Development}
\label{sec:driver}

The \textsf{childdoc} mechanism can also be use for the development
of definition files such as \LaTeX{} styles or classes.
This case differs from the above setup with multiple parts
included by |\include| in that no |\includeonly| should be invoked.
This can be achieved by starting the include file
(before |\ProvidesPackage|) with:
%
\begin{center}
\begin{tabular}{l}
|\input{childdoc.def}|\\
|\childdocforward{|\textit{main}|}|\\
\end{tabular}
\end{center}
%
or alternatively with:
%
\begin{center}
\begin{tabular}{l}
|\input{childdoc.def}|\\
|\childdocby{|\textit{main}|}|\\
\end{tabular}
\end{center}
%
Both forms have slightly different effects as described above.
The main file is prepared as usual, see \secref{sec:include}.

%%%%%%%%%%%%%%%%%%%%%%%%%%%%%%%%%%%%%%%%%%%%%%%%%%%%%%%%%%%%%%%%%%%%%%%%%%%%%%%%
\subsection{Legacy Detection}
\label{sec:detection}

The directive |\childdocmain| in the main file can detect
whether the complete document or merely a child is to be compiled
even without using the directive |\childdocof|.
This method is deprecated because it is less robust
and there is no compelling reason to use it;
it is merely provided for backward compatibility
and it may be removed in future versions.

If the detection mechanism is to be used,
it is mandatory to correctly specify
the filename of the main file as the argument of |\childdocmain|:
%
\begin{center}
\begin{tabular}{l}
|\input{childdoc.def}|\\
|\childdocmain{|\textit{main}|}|\\
\end{tabular}
\end{center}
%
If |\jobname| does not match the argument \textit{main} of |\childdocmain|,
it is assumed that |\jobname| points to the child file to be compiled.
When using |\childdocmain| with the main file specified as argument,
it suffices to start a child file
with just |\input{|\textit{main}|}|
without loading of the package and using |\childdocof|.
If instead all processing is done
with the appropriate \textsf{childdoc} directives,
the argument of \textit{main} of |\childdocmain| can be empty.

An alternative version of the command line processing described
in \secref{sec:commandline} using the detection mechanism reads:
%
\begin{center}
|... -jobname "|\textit{target}|" "|[\textit{flags}]%
[|\def\jobname{|\textit{dest}|}|]|\input{|\textit{main}|}"|
\end{center}

%%%%%%%%%%%%%%%%%%%%%%%%%%%%%%%%%%%%%%%%%%%%%%%%%%%%%%%%%%%%%%%%%%%%%%%%%%%%%%%%
\subsection{Manual Code}
\label{sec:manual}

In case one cannot be certain whether the definitions file |childdoc.def|
is installed on the target \TeX{} distribution
and one prefers not to ship it,
it is conceivable to paste a few relevant commands into the sources.

To that end, drop all statements |\input{childdoc.def}|
and perform the replacements as outlined below.
Instead of |\childdocmain{|\textit{main}|}| add the following code
to the top of the main file:
%
\begin{center}
\begin{tabular}{l}
|\||ifdefined\childdocname\endinput\||fi\newif\ifchilddoc|\\
|\edef\childdocname{\scantokens\expandafter{\jobname\noexpand}}|\\
|\def\childdocmain{|\textit{main}|}\||ifx\childdocmain\childdocname\||else|\\
|\childdoctrue\includeonly{\childdocname}\let\jobname\childdocmain\||fi|\\
\end{tabular}
\end{center}
%
Instead of |\childdocof{|\textit{main}|}| just include the main file
at the top of each child file:
%
\begin{center}
|\input{|\textit{main}|}|
\end{center}
%
A simple redirection |\childdocforward{|\textit{dest}|}| is achieved by:
%
\begin{center}
|\def\jobname{|\textit{dest}|}\input{\jobname}|
\end{center}
%
The redirection with prefix
|\childdocforwardprefix[|\textit{prefix}|]{|\textit{dest}|}|
is accomplished by:
%
\begin{center}
\begin{tabular}{l}
|{\edef\jobname{\scantokens\expandafter{\jobname\noexpand}}|\\
|\def\redirectjob |\textit{prefix}|#1~~~{\gdef\jobname{|\textit{dest}|#1}}|\\
|\expandafter\redirectjob\jobname~~~}\input{\jobname}|
\end{tabular}
\end{center}

In an alternative approach,
child documents can be compiled by a specific command line
without additional code or specific definitions:
%
\begin{center}
|... -jobname "|\textit{target}|" "|[\textit{flags}]%
|\includeonly{|\textit{dest}|}\input{|\textit{main}|}"|
\end{center}
%

%%%%%%%%%%%%%%%%%%%%%%%%%%%%%%%%%%%%%%%%%%%%%%%%%%%%%%%%%%%%%%%%%%%%%%%%%%%%%%%%
%%%%%%%%%%%%%%%%%%%%%%%%%%%%%%%%%%%%%%%%%%%%%%%%%%%%%%%%%%%%%%%%%%%%%%%%%%%%%%%%
\section{Information}

%%%%%%%%%%%%%%%%%%%%%%%%%%%%%%%%%%%%%%%%%%%%%%%%%%%%%%%%%%%%%%%%%%%%%%%%%%%%%%%%
\subsection{Copyright}

Copyright \copyright{} 2017--2018 Niklas Beisert

This work may be distributed and/or modified under the
conditions of the \LaTeX{} Project Public License, either version 1.3
of this license or (at your option) any later version.
The latest version of this license is in
  \url{http://www.latex-project.org/lppl.txt}
and version 1.3 or later is part of all distributions of \LaTeX{}
version 2005/12/01 or later.

This work has the LPPL maintenance status `maintained'.

The Current Maintainer of this work is Niklas Beisert.

This work consists of the files |README.txt|, |childdoc.ins| and |childdoc.dtx|
as well as the derived files |childdoc.def|, |cdocsamp.tex|
with |cdocsch1.tex|, |cdocsch2.tex|, |cdocspt3.tex|, |cdocspt4.tex|,
|cdocsdrf.tex|, |cdocsfn1.tex|, |cdocsfn2.tex|
as well as |childdoc.pdf|.

%%%%%%%%%%%%%%%%%%%%%%%%%%%%%%%%%%%%%%%%%%%%%%%%%%%%%%%%%%%%%%%%%%%%%%%%%%%%%%%%
\subsection{Files and Installation}

The package consists of the files:
%
\begin{center}
\begin{tabular}{ll}
    |README.txt|   & readme file \\
    |childdoc.ins| & installation file \\
    |childdoc.dtx| & source file \\
    |childdoc.def| & definition file \\
    |cdocsamp.tex| & sample main file \\
    |cdocsch1.tex| & sample include file \\
    |cdocsch2.tex| & sample include file \\
    |cdocspt3.tex| & sample part file \\
    |cdocspt4.tex| & sample part file \\
    |cdocsdrf.tex| & sample redirection file \\
    |cdocsfn1.tex| & sample redirection file \\
    |cdocsfn2.tex| & sample redirection file \\
    |childdoc.pdf| & manual
\end{tabular}
\end{center}
%
The distribution consists of the files
|README.txt|, |childdoc.ins| and |childdoc.dtx|.
%
\begin{itemize}
\item
Run (pdf)\LaTeX{} on |childdoc.dtx|
to compile the manual |childdoc.pdf| (this file).
\item
Run \LaTeX{} on |childdoc.ins| to create the definitions file |childdoc.def|
and the sample |cdocsamp.tex| with include files
|cdocsch1.tex|, |cdocsch2.tex|, |cdocspt3.tex|, |cdocspt4.tex|,
|cdocsdrf.tex|, |cdocsfn1.tex|, |cdocsfn2.tex|.
Then copy the file |childdoc.def| to an appropriate directory of your \LaTeX{}
distribution, e.g.\ \textit{texmf-root}|/tex/latex/childdoc|.
\end{itemize}

%%%%%%%%%%%%%%%%%%%%%%%%%%%%%%%%%%%%%%%%%%%%%%%%%%%%%%%%%%%%%%%%%%%%%%%%%%%%%%%%
\subsection{Related CTAN Packages}

There are several other packages which offer a similar functionality:
%
\begin{itemize}
\item
The packages
\href{http://ctan.org/pkg/docmute}{\textsf{docmute}},
\href{http://ctan.org/pkg/includex}{\textsf{includex}} and
\href{http://ctan.org/pkg/standalone}{\textsf{standalone}}
provide commands to include only the document body of
a child file thus allowing both files to be compiled individually.
\item
The packages \href{http://ctan.org/pkg/subdocs}{\textsf{subdocs}}
and \href{http://ctan.org/pkg/subfiles}{\textsf{subfiles}}
provide structures in which the main and child documents can be
encapsulated and allowing them to be compiled individually.
The inclusion mechanism is different from the conventional |\include|.
\item
The package \href{http://ctan.org/pkg/combine}{\textsf{combine}}
is an elaborate solution to combine several documents into one.
\end{itemize}
%
See also the CTAN topic \href{http://ctan.org/topic/subdocs}{\textsf{subdocs}}
for further related packages.
The present package differs from the above solutions in that
a document structure constructed with the conventional |\include| mechanism
just needs two extra commands at the top of every file
such that all constituent files can be compiled individually.

%%%%%%%%%%%%%%%%%%%%%%%%%%%%%%%%%%%%%%%%%%%%%%%%%%%%%%%%%%%%%%%%%%%%%%%%%%%%%%%%
%\subsection{Feature Suggestions}
%
%The following is a list of features which may be useful for future
%versions of this package:
%%
%\begin{itemize}
%\item
%\ldots
%\end{itemize}

%%%%%%%%%%%%%%%%%%%%%%%%%%%%%%%%%%%%%%%%%%%%%%%%%%%%%%%%%%%%%%%%%%%%%%%%%%%%%%%%
\subsection{Revision History}

%%%%%%%%%%%%%%%%%%%%%%%%%%%%%%%%%%%%%%%%
\paragraph{v2.0:} 2018/12/30

\begin{itemize}
\item
immediate forward processing
\item
added |\childdocby| mechanism
\item
manual restructured
\end{itemize}

%%%%%%%%%%%%%%%%%%%%%%%%%%%%%%%%%%%%%%%%
\paragraph{v1.6:} 2018/01/17

\begin{itemize}
\item
application for development of include files
\item
corrections to manual
\end{itemize}

%%%%%%%%%%%%%%%%%%%%%%%%%%%%%%%%%%%%%%%%
\paragraph{v1.5:} 2017/05/21

\begin{itemize}
\item
more complete structuring introduced
\item
|\childdocof| introduced
\item
|\childdoc| renamed to |\childdocmain|
\item
|\childredirect| renamed to |\childdocforward| and |\childdocforwardprefix|
and functionality expanded
\end{itemize}

%%%%%%%%%%%%%%%%%%%%%%%%%%%%%%%%%%%%%%%%
\paragraph{v1.0:} 2017/04/27

\begin{itemize}
\item
manual and install package
\item
first version published on CTAN
\end{itemize}

%%%%%%%%%%%%%%%%%%%%%%%%%%%%%%%%%%%%%%%%
\paragraph{v0.6:} 2017/04/26

\begin{itemize}
\item
redirection mechanism added
\end{itemize}

%%%%%%%%%%%%%%%%%%%%%%%%%%%%%%%%%%%%%%%%
\paragraph{v0.5:} 2017/04/26

\begin{itemize}
\item
functionality in definition file
\end{itemize}


%%%%%%%%%%%%%%%%%%%%%%%%%%%%%%%%%%%%%%%%%%%%%%%%%%%%%%%%%%%%%%%%%%%%%%%%%%%%%%%%
%%%%%%%%%%%%%%%%%%%%%%%%%%%%%%%%%%%%%%%%%%%%%%%%%%%%%%%%%%%%%%%%%%%%%%%%%%%%%%%%
%%%%%%%%%%%%%%%%%%%%%%%%%%%%%%%%%%%%%%%%%%%%%%%%%%%%%%%%%%%%%%%%%%%%%%%%%%%%%%%%
\appendix

\settowidth\MacroIndent{\rmfamily\scriptsize 000\ }

 \DocInput{childdoc.dtx}

\end{document}
%</driver>
% \fi
%
% %%%%%%%%%%%%%%%%%%%%%%%%%%%%%%%%%%%%%%%%%%%%%%%%%%%%%%%%%%%%%%%%%%%%%%%%%%%%%%
% %%%%%%%%%%%%%%%%%%%%%%%%%%%%%%%%%%%%%%%%%%%%%%%%%%%%%%%%%%%%%%%%%%%%%%%%%%%%%%
% \section{Sample}
%\iffalse
%<*samplemain>
%\fi
%
% The following presents a sample document
% with two chapters, two parts, a title page,
% a compile flag as well as three forwarding files to set the flag.
% It consists of eight |.tex| files:
% \begin{center}
% \begin{tabular}{ll}
% |cdocsamp.tex|&main file\\
% |cdocsch1.tex|&include file for chapter 1\\
% |cdocsch2.tex|&include file for chapter 2\\
% |cdocspt3.tex|&include file for part 3\\
% |cdocspt4.tex|&include file for part 4\\
% |cdocsdrf.tex|&forwarding file for main file in draft mode\\
% |cdocsfi1.tex|&forwarding file for final version of chapter 1\\
% |cdocsfi2.tex|&forwarding file for final version of chapter 2\\
% \end{tabular}
% \end{center}
% Each of the eight files can be compiled directly by the \LaTeX{} compiler.
%
% %%%%%%%%%%%%%%%%%%%%%%%%%%%%%%%%%%%%%%
% \paragraph{Main File.}
%
% The main file is called |cdocsamp.tex|.
%
% Load the \textsf{childdoc} definitions and
% declare the filename for the main document:
%    \begin{macrocode}
\input{childdoc.def}
\childdocmain{}
%    \end{macrocode}

% Optional override for |\version| flag:
%    \begin{macrocode}
%%\ifchilddoc\else\providecommand{\version}{draft}\fi
%    \end{macrocode}

% Define the default values for the |\version| flag
% (|final| for the main file and |draft| for childs):
%    \begin{macrocode}
\ifchilddoc
\providecommand{\version}{draft}
\else
\providecommand{\version}{final}
\fi
%    \end{macrocode}

% Load the standard document class:
%    \begin{macrocode}
\documentclass[12pt]{article}
%    \end{macrocode}

% Start the document body:
%    \begin{macrocode}
\begin{document}
%    \end{macrocode}

% Declare a title page.
% Print title, part of document being processed and version flag:
%    \begin{macrocode}
\addtocounter{page}{-1}
\begin{center}
{\LARGE\bfseries{}childdoc example\par}
\vspace{1cm}
\ifchilddoc
\ifchilddocmanual part\else chapter\fi:
`\childdocname' of `\childdocjob'\par
\else
main document: `\childdocjob'\par
\fi
version: \version\par
\end{center}
\newpage
%    \end{macrocode}

% Manually include selected file,
% otherwise process as usual:
%    \begin{macrocode}
\ifchilddocmanual
\section*{part `\childdocname'}
\input{\childdocname}
\else
%    \end{macrocode}

% Include the two chapters:
%    \begin{macrocode}
\include{cdocsch1}
\include{cdocsch2}
%    \end{macrocode}

% Include the two parts unless only chapters should be displayed:
%    \begin{macrocode}
\ifchilddoc\else
\section{part three}
\input{cdocspt3}
\section{part four}
\input{cdocspt4}
\fi
%    \end{macrocode}

% Process as usual until here:
%    \begin{macrocode}
\fi
%    \end{macrocode}

% End of document body:
%    \begin{macrocode}
\end{document}
%    \end{macrocode}
%\iffalse
%</samplemain>
%\fi
%
% %%%%%%%%%%%%%%%%%%%%%%%%%%%%%%%%%%%%%%
% \paragraph{Chapter Include Files.}
%
% The include files are called |cdocsch1.tex| and |cdocsch2.tex|.
%
%\iffalse
%<*samplechap1|samplechap2>
%\fi

% Optional override for |\version| flag:
%    \begin{macrocode}
%%\providecommand{\version}{final}
%    \end{macrocode}

% Include the main document:
%    \begin{macrocode}
\input{childdoc.def}
\childdocof{cdocsamp}
%    \end{macrocode}

%\iffalse
%</samplechap1|samplechap2>
%\fi
%
%\iffalse
%<*samplechap1>
%\fi
% Some text for chapter 1:
%    \begin{macrocode}
\section{one}
some text in chapter one
%    \end{macrocode}

%\iffalse
%</samplechap1>
%\fi
% Some text for chapter 2:
%\iffalse
%<*samplechap2>
%\fi
%    \begin{macrocode}
\section{two}
more text in chapter two
%    \end{macrocode}

%\iffalse
%</samplechap2>
%\fi
%
% %%%%%%%%%%%%%%%%%%%%%%%%%%%%%%%%%%%%%%
% \paragraph{Part Include Files.}
%
% The include files are called |cdocspt3.tex| and |cdocspt4.tex|.
%
%\iffalse
%<*samplepart3|samplepart4>
%\fi

% Optional override for |\version| flag:
%    \begin{macrocode}
%%\providecommand{\version}{final}
%    \end{macrocode}

% Include the main document:
%    \begin{macrocode}
\input{childdoc.def}
\childdocby{cdocsamp}
%    \end{macrocode}

%\iffalse
%</samplepart3|samplepart4>
%\fi
%
%\iffalse
%<*samplepart3>
%\fi
% Some text for part 3:
%    \begin{macrocode}
some text in part three
%    \end{macrocode}

%\iffalse
%</samplepart3>
%\fi
% Some text for part 4:
%\iffalse
%<*samplepart4>
%\fi
%    \begin{macrocode}
more text in part four
%    \end{macrocode}

%\iffalse
%</samplepart4>
%\fi
%
% %%%%%%%%%%%%%%%%%%%%%%%%%%%%%%%%%%%%%%
% \paragraph{Forwarding for a Complete Draft.}
%
% The following forwarding file |cdocsdrf.tex|
% compiles the main document in draft mode:
%\iffalse
%<*sampledraft>
%\fi
%    \begin{macrocode}
\def\version{draft}
\input{childdoc.def}
\childdocforward{cdocsamp}
%    \end{macrocode}

%\iffalse
%</sampledraft>
%\fi
%
% %%%%%%%%%%%%%%%%%%%%%%%%%%%%%%%%%%%%%%
% \paragraph{Forwarding for Final Version of the Chapters.}
%
% The following forwarding files |cdocsfn1.tex| and |cdocsfn2.tex|
% (with identical content)
% compile the final versions of the child documents
% |cdocsch1.tex| and |cdocsch2.tex|, respectively:
%\iffalse
%<*samplefinal>
%\fi
%    \begin{macrocode}
\def\version{final}
\input{childdoc.def}
\childdocforwardprefix[cdocsamp]{cdocsfn}{cdocsch}
%    \end{macrocode}

%\iffalse
%</samplefinal>
%\fi
%
% %%%%%%%%%%%%%%%%%%%%%%%%%%%%%%%%%%%%%%
% \paragraph{Command Line Processing.}
%
% The following three command lines generate the output files
% |cdocscld|, |cdocscl1| and |cdocscl2|
% which should be identical to
% |cdocsdrf|, |cdocsch1| and |cdocsfn2|, respectively:
% \begin{center}
% \begin{tabular}{l}
% |latex -jobname cdocscld \|\\
% |  "\def\version{draft}\input{childdoc.def}\childdocforward{cdocsamp}"|\\
% |latex -jobname cdocscl1 \|\\
% |  "\input{childdoc.def}\childdocforward[cdocsamp]{cdocsch1}"|\\
% |latex -jobname cdocscl2 \|\\
% |  "\def\version{final}\input{childdoc.def}\childdocforward{cdocsch2}"|
% \end{tabular}
% \end{center}
% Note that the trailing backslash on each first line
% merely continues the input to the second line
% (for convenient cut ant paste).
% Furthermore, the command |latex| can be replaced by any
% of its alternative versions such as |pdflatex|.
%
% %%%%%%%%%%%%%%%%%%%%%%%%%%%%%%%%%%%%%%%%%%%%%%%%%%%%%%%%%%%%%%%%%%%%%%%%%%%%%%
% %%%%%%%%%%%%%%%%%%%%%%%%%%%%%%%%%%%%%%%%%%%%%%%%%%%%%%%%%%%%%%%%%%%%%%%%%%%%%%
% \section{Implementation}
%\iffalse
%<*package>
%\fi
%
% This section describes the definitions file |childdoc.def|.

% The definitions cannot be loaded using |\usepackage| or |\RequirePackage|
% which has a mechanism to prevent loading a style file more than once.
% When loading the definitions by means of |\input|
% multiple instances have to be prevented manually:
%\iffalse
%This code needs to be before the `\ProvidesFile' directive
%which is defined at the beginning of this file.
%Therefore it is also placed there and commented out here.
%</package>
%<*discard>
%\fi
%    \begin{macrocode}
\ifdefined\childdocmain\endinput\fi
%    \end{macrocode}
%\iffalse
%</discard>
%<*package>
%\fi
%
% \macro{\ifchilddoc}
% \macro{\ifchilddocmanual}
% The conditional |\ifchilddoc| tells whether a
% child (true) or main (false) document is being compiled.
% The conditional |\ifchilddocmanual| tells whether
% the |\includeonly| mechanism is used (false) or
% the selection of child files must be performed manually (true).
% The definitions initialise to false:
%    \begin{macrocode}
\newif\ifchilddoc
\newif\ifchilddocmanual
%    \end{macrocode}

% \macro{\childdocname}
% \macro{\childdocjob}
% The macro |\childdocname| stores the name of the main document
% to be compiled. The macro |\childdocjob| stores the name of
% the document on which the \LaTeX{} compiler was originally invoked.
% The content of |\jobname| cannot be compared
% to filenames specified in the source due to different catcodes.
% The following code rescans |\jobname|, stores the result
% in |\childdocname| and saves a copy in |\childdocjob|:
%    \begin{macrocode}
\edef\childdocname{\scantokens\expandafter{\jobname\noexpand}}
\let\childdocjob\childdocname
%    \end{macrocode}

% \macro{\childdocdisable}
% The macro |\childdocdisable| prevents the main file
% from being processed more than once.
% At this stage, the main document command |\childdocmain|
% is assumed to be called once again where it should do nothing.
% Any subsequent call to it should prevent
% a secondary processing of the main document
% It overwrites the forwarding commands
% |\childdocof| and |\childdocforward|
% with empty macros to prevent further inclusions of the main document:
%    \begin{macrocode}
\newcommand{\childdocdisable}
{
  \renewcommand{\childdocmain}[1]{\renewcommand{\childdocmain}[1]{\endinput}}
  \renewcommand{\childdocof}[1]{}
  \renewcommand{\childdocby}[2][]{}
  \renewcommand{\childdocforward}[2][]{}
  \renewcommand{\childdocdisable}{}
}
%    \end{macrocode}

% \macro{\childdocmain}
% The macro |\childdocmain| is to be called at the top of the main file
% with nothing or the main filename (without extension) as argument.
% First, it breaks loops.
% If the argument is not empty and does not match |\childdocname|
% (which is set by the first inclusion of |childdoc.def|),
% |\ifchilddoc| is set to true, |\includeonly| is applied to the child file
% and |\jobname| is set to the main file
% (for proper handling of |.aux| files):
%    \begin{macrocode}
\newcommand{\childdocmain}[1]
{
  \childdocdisable\childdocmain{}
  \if?#1?\else
    \begingroup
      \def\childdoctmp{#1}
      \ifx\childdoctmp\childdocname
        \def\childdoctmp{}
      \else
        \def\childdoctmp
        {
          \childdoctrue
          \includeonly{\childdocname}
          \def\childdocjob{#1}
          \def\jobname{#1}
        }
      \fi
      \expandafter
    \endgroup
    \childdoctmp
  \fi
}
%    \end{macrocode}

% \macro{\childdocof}
% The command |\childdocof| redirects
% compilation to the main file |#1|.
%    \begin{macrocode}
\newcommand{\childdocof}[1]
{
  \childdocdisable
  \childdoctrue
  \includeonly{\childdocname}
  \def\jobname{#1}
  \def\childdocjob{#1}
  \input{#1}
}
%    \end{macrocode}

% \macro{\childdocby}
% The command |\childdocby| ....
%    \begin{macrocode}
\newcommand{\childdocby}[2][]
{
  \childdocdisable
  \childdoctrue
  \childdocmanualtrue
  \if?#1?\else
    \def\jobname{#2}
  \fi
  \def\childdocjob{#2}
  \input{#2}
  \endinput
}
%    \end{macrocode}

% \macro{\childdocforward}
% The command |\childdocforward| redirects
% compilation to the main file or
% (if the optional argument is given) a child file.
% Parameters are set as if the main file
% or a child file starting with |\childdocof| was compiled.
% Then compilation is handed over to the main file:
%    \begin{macrocode}
\newcommand{\childdocforward}[2][]
{
  \begingroup
    \if?#1?
      \def\childdoctmp
      {
        \def\childdocname{#2}
        \def\childdocjob{#2}
        \def\jobname{#2}
        \input{#2}
        \endinput
      }
    \else
      \def\childdoctmp
      {
        \childdocdisable
        \def\childdocname{#2}
        \childdoctrue
        \includeonly{#2}
        \def\childdocjob{#1}
        \def\jobname{#1}
        \input{#1}
        \endinput
      }
    \fi
    \expandafter
  \endgroup
  \childdoctmp
}
%    \end{macrocode}

% \macro{\childdocforwardprefix}
% The command |\childdocforwardprefix| redirects
% compilation to the main or a child file by means of a pattern.
% The prefix |#1| in the current filename is replaced by |#2|
% and the suffix of the current filename is kept
% (it is assumed that the filename does not contain the substring `|~~~|'
% which is used as a delimiter).
% Compilation is handed over to the new file by |\childdocforward|:
%    \begin{macrocode}
\newcommand{\childdocforwardprefix}[3][]
{
  \begingroup
    \def\childdocextract #2##1~~~{\def\childdoctmp{\childdocforward[#1]{#3##1}}}
    \expandafter\childdocextract\childdocname~~~
    \expandafter
  \endgroup
  \childdoctmp
}
%    \end{macrocode}

% \macro{\childdoc}
% The deprecated macro |\childdoc| is a legacy version of |\childdocmain|:
%    \begin{macrocode}
\newcommand{\childdoc}{\childdocmain}
%    \end{macrocode}

% \macro{\childdocredirect}
% The deprecated macro |\childdocredirect| is a legacy version
% of |\childdocforward| and |\childdocforwardprefix|:
%    \begin{macrocode}
\newcommand{\childdocredirect}[2][]
{
  \begingroup
    \if?#1?
      \def\childdoctmp{\childdocforward{#2}}
    \else
      \def\childdoctmp{\childdocforwardprefix{#1}{#2}}
    \fi
    \expandafter
  \endgroup
  \childdoctmp
}
%    \end{macrocode}

%\iffalse
%</package>
%\fi
%
\endinput
|\\
|\childdocforward[|\textit{main}|]{|\textit{dest}|}|\\
\end{tabular}
\end{center}
%
The argument \textit{dest} is the destination file
(without extension).
It should be the main file or one of the child files.
Note that further \textsf{childdoc} directives
such as |\childdocof| and |\childdocforward|
in the indicated file will be processed in this form.
The optional argument \textit{main}
passes on directly to the main file \textit{main}
while pretending to compile the child \textit{dest}.
This form behaves as if \textit{dest}
issues |\childdocof{|\textit{main}|}| right away,
and no further \textsf{childdoc} directives will be processed.

%%%%%%%%%%%%%%%%%%%%%%%%%%%%%%%%%%%%%%%%
\DescribeMacro{\...prefix}
In the alternative form |\childdocforwardprefix|,
%
\begin{center}
\begin{tabular}{l}
|% \iffalse
%
% childdoc.dtx Copyright (C) 2017-2018 Niklas Beisert
%
% This work may be distributed and/or modified under the
% conditions of the LaTeX Project Public License, either version 1.3
% of this license or (at your option) any later version.
% The latest version of this license is in
%   http://www.latex-project.org/lppl.txt
% and version 1.3 or later is part of all distributions of LaTeX
% version 2005/12/01 or later.
%
% This work has the LPPL maintenance status `maintained'.
%
% The Current Maintainer of this work is Niklas Beisert.
%
% This work consists of the files childdoc.dtx and childdoc.ins
% and the derived files childdoc.def and cdocsamp.tex with
% cdocsch1.tex, cdocsch2.tex, cdocsdrf.tex, cdocsfn1.tex, cdocsfn2.tex.
%
%<package>\ifdefined\childdocmain\endinput\fi
%<package>\ProvidesFile{childdoc.def}[2018/12/30 v2.0 child document driver]
%<samplemain>\ProvidesFile{cdocsamp.tex}[2018/12/30 v2.0 sample for childdoc]
%<*driver>
%\ProvidesFile{childdoc.drv}[2018/12/30 v2.0 childdoc reference manual file]
\PassOptionsToClass{10pt,a4paper}{article}
\documentclass{ltxdoc}

\usepackage[margin=35mm]{geometry}
\usepackage{hyperref}
\usepackage{hyperxmp}
\usepackage[usenames]{color}

\hypersetup{colorlinks=true}
\hypersetup{pdfstartview=FitH}
\hypersetup{pdfpagemode=UseNone}
\hypersetup{pdfsource={}}
\hypersetup{pdflang={en-UK}}
\hypersetup{pdfcopyright={Copyright 2017-2018 Niklas Beisert.
  This work may be distributed and/or modified under the
  conditions of the LaTeX Project Public License, either version 1.3
  of this license or (at your option) any later version.}}
\hypersetup{pdflicenseurl={http://www.latex-project.org/lppl.txt}}
\hypersetup{pdfcontactaddress={ETH Zurich, ITP, HIT K,
  Wolfgang-Pauli-Strasse 27}}
\hypersetup{pdfcontactpostcode={8093}}
\hypersetup{pdfcontactcity={Zurich}}
\hypersetup{pdfcontactcountry={Switzerland}}
\hypersetup{pdfcontactemail={nbeisert@itp.phys.ethz.ch}}
\hypersetup{pdfcontacturl={http://people.phys.ethz.ch/\xmptilde nbeisert/}}

\newcommand{\secref}[1]{\hyperref[#1]{section \ref*{#1}}}

\parskip1ex
\parindent0pt
\let\olditemize\itemize
\def\itemize{\olditemize\parskip0pt}

\begin{document}

\title{The \textsf{childdoc} Package}
\hypersetup{pdftitle={The childdoc Package}}
\author{Niklas Beisert\\[2ex]
  Institut f\"ur Theoretische Physik\\
  Eidgen\"ossische Technische Hochschule Z\"urich\\
  Wolfgang-Pauli-Strasse 27, 8093 Z\"urich, Switzerland\\[1ex]
  \href{mailto:nbeisert@itp.phys.ethz.ch}
  {\texttt{nbeisert@itp.phys.ethz.ch}}}
\hypersetup{pdfauthor={Niklas Beisert}}
\hypersetup{pdfsubject={Manual for the LaTeX2e Package childdoc}}
\date{30 December 2018, \textsf{v2.0}}
\maketitle

\begin{abstract}\noindent
\textsf{childdoc} is a \LaTeXe{} package
that enables the direct compilation
of document sections included by |\include|
to individual files.
\end{abstract}

\begingroup
\parskip0ex
\tableofcontents
\endgroup

%%%%%%%%%%%%%%%%%%%%%%%%%%%%%%%%%%%%%%%%%%%%%%%%%%%%%%%%%%%%%%%%%%%%%%%%%%%%%%%%
%%%%%%%%%%%%%%%%%%%%%%%%%%%%%%%%%%%%%%%%%%%%%%%%%%%%%%%%%%%%%%%%%%%%%%%%%%%%%%%%
\section{Introduction}

\LaTeX{} provides a mechanism to structure a large document (such as a book)
into a main file and several child files (containing the chapters)
using the |\include| command.
This mechanism is beneficial for documents
which span hundreds of pages in order to
make the source file(s) more manageable.
Moreover, compilation can be restricted to
selected child files by means of the |\includeonly| command.
The latter feature can be used to reduce the compilation time while editing
(this was significantly more useful in the earlier days of \LaTeX{})
or to generate a smaller document which is easier to navigate.
Another application of |\includeonly| is to generate
documents consisting of selected parts of the complete document.

However, there are a few drawbacks of the plain |\include| mechanism:
\begin{itemize}
\item
The child files cannot be compiled on their own,
they can only be compiled via the main file.
A naive editing environment
(such as a text editor with an option
to have the current file processed by \LaTeX)
may require one to switch to the main file before compiling;
attempting to compile the child file produces errors.
\item
The main file must be modified (each time)
to adjust the |\includeonly| command
to the present needs. This easily leaves the main file in a messy state.
\item
The generated document will always carry the filename
of the main document. This is inconvenient if
several child files are to be compiled and
to be kept for distribution.
\end{itemize}

The present package provides a simple interface
to make child files individually compilable by \LaTeX{}.
Compiling a child file then has the same effect as compiling
the main file with an |\includeonly| command
to select the appropriate child.
Moreover the generated document will carry the name of the child
rather than the main file.
This resolves all three above issues.

This feature is meant to make the editing of books,
thesis documents and lecture notes somewhat more convenient.
However, the package can also be used efficiently for
composing a series of documents (such as exercise sheets)
which are typically distributed individually.
It then assists the author in generating the individual documents
(potentially in different versions)
as well as a document containing the collected series.
Another application is in developing style files
or other kinds of included material
where compilation of the style file could redirect
to a sample or test file.

%%%%%%%%%%%%%%%%%%%%%%%%%%%%%%%%%%%%%%%%%%%%%%%%%%%%%%%%%%%%%%%%%%%%%%%%%%%%%%%%
%%%%%%%%%%%%%%%%%%%%%%%%%%%%%%%%%%%%%%%%%%%%%%%%%%%%%%%%%%%%%%%%%%%%%%%%%%%%%%%%
\section{Usage}

First of all, the package \textsf{childdoc} is \emph{not} a standard
\LaTeXe{} |.sty| style file! Therefore it needs to be invoked in
a non-standard way.

%%%%%%%%%%%%%%%%%%%%%%%%%%%%%%%%%%%%%%%%%%%%%%%%%%%%%%%%%%%%%%%%%%%%%%%%%%%%%%%%
\subsection{Included Files}
\label{sec:include}

%%%%%%%%%%%%%%%%%%%%%%%%%%%%%%%%%%%%%%%%
\DescribeMacro{\childdocmain}
To use the package, add the commands
\begin{center}
\begin{tabular}{l}
|\input{childdoc.def}|\\
|\childdocmain{}|\\
\end{tabular}
\end{center}
at the very top of the main \LaTeX{} file,
in particular \emph{before} the |\documentclass| statement!
The argument of |\childdocmain| should be left empty
(but it must be present).

%%%%%%%%%%%%%%%%%%%%%%%%%%%%%%%%%%%%%%%%
\DescribeMacro{\childdocof}
Furthermore, add the commands
\begin{center}
\begin{tabular}{l}
|\input{childdoc.def}|\\
|\childdocof{|\textit{main}|}|\\
\end{tabular}
\end{center}
at the top of every child file \textit{child}
which is included by |\include{|\textit{child}|}|
from within the main file
(or at least for those files to be compiled individually).
The argument \textit{main} must be the filename of the main file.

There are a couple of
considerations in setting up the main and child documents:

%%%%%%%%%%%%%%%%%%%%%%%%%%%%%%%%%%%%%%%%
\paragraph{Restrictions.}

Please note the following restrictions:
\begin{itemize}
\item
|\childdocmain| must be called with one argument \textit{main}
to ensure compatibility with earlier version of the package.
It must either be empty (|\childdocmain{}|)
or precisely match the filename of the main file in which it is specified.
See \secref{sec:detection} for further information.
\item
The filename \textit{main} must be specified without the |.tex| extension.
\item
The filename \textit{main} is case sensitive
(even in case-insensitive file systems)
due to internal string comparison.
\item
The argument \textit{main} should be fully expanded, it cannot be a macro.
\item
Subdirectories and special characters should be avoided in filenames.
\item
The command |\childdocmain{|\textit{main}|}| must be followed by a whitespace.
It should not be followed immediately by another command
or by a comment mark `|%|'.
This is because the \TeX{} parser reads the token immediately following
the argument of |\childdocmain| and puts it
at the beginning of every child section;
however, a white\-space is ignored.
\end{itemize}

%%%%%%%%%%%%%%%%%%%%%%%%%%%%%%%%%%%%%%%%
\paragraph{Content of Main File.}

It is advisable to place all content in the child files included by |\include|.
Any output contained in the main file will appear in all child documents
unless suppressed manually;
it cannot be suppressed automatically by the |\includeonly| directive
and thus should normally be avoided.
A method to include some content in the main file
by means of conditional processing is described in \secref{sec:conditional}.

%%%%%%%%%%%%%%%%%%%%%%%%%%%%%%%%%%%%%%%%
\paragraph{Page Numbering.}

When only a part of the document is compiled,
the appropriate numbering of pages
(as well as other status parameters)
is determined from the |.aux| files.
The latter contain information from previous passes.
However this information needs to propagate through
all intermediate child documents.
Therefore the page numbering in child documents may well
be inconsistent until the complete document is compiled at least once.

A useful (if unconventional) way to always ensure a consistent
page numbering is to restart the numbering in each child document
and denote the pages by `\textit{child}|.|\textit{page}'
where \textit{child} represents the chapter/section number of the child file.
This can be achieved by the command
|\numberwithin{page}{|\textit{child}|}|
of the \textsf{amsmath} package
where \textit{child} can be |chapter| or |section|
depending on the chosen structuring.
Alternatively, one can modify the macro |\thepage| appropriately
and reset the counter |page| at the start of each child file.

%%%%%%%%%%%%%%%%%%%%%%%%%%%%%%%%%%%%%%%%%%%%%%%%%%%%%%%%%%%%%%%%%%%%%%%%%%%%%%%%
\subsection{Conditional Processing}
\label{sec:conditional}

The package provides a mechanism to compile different versions
of a document. To customise the versions further some conditional processing
can come in handy to distinguish which version is being compiled.
The package provides two macros to describe the compilation context:

%%%%%%%%%%%%%%%%%%%%%%%%%%%%%%%%%%%%%%%%
\DescribeMacro{\ifchilddoc}
The conditional |\ifchilddoc| distinguishes between the compilation of
child documents and the main document:
%
\begin{center}
|\ifchilddoc |\textit{child-code}| |[|\||else |\textit{main-code}]| \||fi|
\end{center}

%%%%%%%%%%%%%%%%%%%%%%%%%%%%%%%%%%%%%%%%
\DescribeMacro{\childdocname}
\DescribeMacro{\childdocjob}
The macro |\childdocname| contains the filename (without extension)
of the main or child file being processed.
Note that |\childdocjob| will always contain the name of the main file.

%%%%%%%%%%%%%%%%%%%%%%%%%%%%%%%%%%%%%%%%
\paragraph{Title Page.}

Conditional processing can be used to include a title or banner page
in the main document when proper precautions are taken.
Importantly, the code in the main file should ensure that the page counter
(as well as other status parameters which are stored in the |.aux| files)
takes the same value after the conditional processing.
Otherwise the page numbers may take divergent values
depending on which part is compiled.

For example, a title page could be declared by:
%
\begin{center}
\begin{tabular}{l}
|\ifchilddoc\||else|\\
|\addtocounter{page}{-1}|\\
\textit{code for title page}\\
|\newpage|\\
|\||fi|
\end{tabular}
\end{center}
%
A banner page for the child documents can be generated by:
%
\begin{center}
\begin{tabular}{l}
|\ifchilddoc|\\
|\addtocounter{page}{-1}|\\
\textit{code for banner page}\\
|\newpage|\\
|\||fi|
\end{tabular}
\end{center}
%
Here one could write a message such as:
\begin{center}
|This is the part \childdocname{} of \childdocjob{}.|
\end{center}

%%%%%%%%%%%%%%%%%%%%%%%%%%%%%%%%%%%%%%%%%%%%%%%%%%%%%%%%%%%%%%%%%%%%%%%%%%%%%%%%
\subsection{Flags}
\label{sec:flags}

The package makes it easy to generate different versions
of the main or child documents.
To this end compilation flags can be defined
and assigned different default values.
They will be particularly useful in conjunction
with the forwarding mechanism described in \secref{sec:forward}.

For example, it may be useful to have a flag |\version|
which can be set to |draft| or |final|.
The document source will contain some conditional code
depending on the value of |\version|.
Suppose further, the flag should default to |final| for the main file
and to |draft| for child files
which is a natural assignment for editing the document.
This is achieved by placing the following code
in the preamble of the main document
(below the |\childdocmain| directive):
%
\begin{center}
\begin{tabular}{l}
|\ifchilddoc|\\
|\providecommand{\version}{draft}|\\
|\||else|\\
|\providecommand{\version}{final}|\\
|\||fi|
\end{tabular}
\end{center}
%
The definition by |\providecommand| makes sure
that previous definitions are not overwritten.
Further statements |\providecommand{\version}{...}|
can thus be added before the above code to override it.

For the main file, one might add a line
(between |\childdocmain| and the above block)
%
\begin{center}
|%\ifchilddoc\||else\providecommand{\version}{draft}\||fi|
\end{center}
%
which can be uncommented to produce a draft version.
Likewise one can add a line to the very top of a child file
(above the |\childdocof{|\textit{main}|}| directive)
%
\begin{center}
|%\providecommand{\version}{final}|
\end{center}
%
which can be uncommented to produce the final version of this child document.

%%%%%%%%%%%%%%%%%%%%%%%%%%%%%%%%%%%%%%%%%%%%%%%%%%%%%%%%%%%%%%%%%%%%%%%%%%%%%%%%
\subsection{Forwarding}
\label{sec:forward}

Different versions of the main or child documents
using compilation flags as described in \secref{sec:flags}
can be (permanently) stored in different files
for convenient compilation, viewing and distribution.
To this end, the package defines a command
to pass on compilation to a different file:

%%%%%%%%%%%%%%%%%%%%%%%%%%%%%%%%%%%%%%%%
\DescribeMacro{\childdocforward}
The command |\childdocforward| redirects processing to
another source file:
%
\begin{center}
\begin{tabular}{l}
|\input{childdoc.def}|\\
|\childdocforward[|\textit{main}|]{|\textit{dest}|}|\\
\end{tabular}
\end{center}
%
The argument \textit{dest} is the destination file
(without extension).
It should be the main file or one of the child files.
Note that further \textsf{childdoc} directives
such as |\childdocof| and |\childdocforward|
in the indicated file will be processed in this form.
The optional argument \textit{main}
passes on directly to the main file \textit{main}
while pretending to compile the child \textit{dest}.
This form behaves as if \textit{dest}
issues |\childdocof{|\textit{main}|}| right away,
and no further \textsf{childdoc} directives will be processed.

%%%%%%%%%%%%%%%%%%%%%%%%%%%%%%%%%%%%%%%%
\DescribeMacro{\...prefix}
In the alternative form |\childdocforwardprefix|,
%
\begin{center}
\begin{tabular}{l}
|\input{childdoc.def}|\\
|\childdocforwardprefix[|\textit{main}|]{|\textit{prefix}|}{|\textit{dest}|}|
\end{tabular}
\end{center}
%
the destination file is determined by a pattern
depending on the current file:
To make this work, the current file must be called
`{\textit{prefix}\hspace{0.2em}\textit{suffix}}'
with \textit{prefix} matching precisely the argument.
Processing is then passed on to the file
`{\textit{dest}\hspace{0.2em}\textit{suffix}}'.
Surely, the same effect is achieved by
directly specifying the
argument `{\textit{dest}\hspace{0.2em}\textit{suffix}}'
in the first form.
However, that requires to set up a different file
for each child. With the alternative form of the command
all these files can have exactly the same content
which simplifies setting them up and maintaining them.

For example, the following file |draft.tex|
with a compilation flag |\version| as described in \secref{sec:flags}
compiles the main document as a draft:
%
\begin{center}
\begin{tabular}{l}
|\def\version{draft}|\\
|\input{childdoc.def}|\\
|\childdocforward{|\textit{main}|}|
\end{tabular}
\end{center}
%
Likewise, the following files |final|\textit{nn}|.tex|
compile the final version of the child document
|child|\textit{nn}|.tex|:
%
\begin{center}
\begin{tabular}{l}
|\def\version{final}|\\
|\input{childdoc.def}|\\
|\childdocforwardprefix{final}{child}|
\end{tabular}
\end{center}
%

Note that when several versions of a main file and/or of each child file
are to be generated, it may be convenient to set up a |Makefile| or
shell script to automatise the process.

%%%%%%%%%%%%%%%%%%%%%%%%%%%%%%%%%%%%%%%%%%%%%%%%%%%%%%%%%%%%%%%%%%%%%%%%%%%%%%%%
\subsection{Command Line Processing}
\label{sec:commandline}

The effect of redirection files can also be achieved by invoking
the \LaTeX{} compiler with a more elaborate command line.
Most conveniently this should be done as part
of a shell script or a |Makefile|.

When using \textsf{childdoc} in the main file, the following
command lines effectively perform a redirection
(note that depending on the shell being used,
backslashes may have to be doubled: `|\|' $\to$ `|\\|'):
%
\begin{center}
|... -jobname "|\textit{target}|" |\\|"|[\textit{flags}]%
|\input{childdoc.def}\childdocforward[|\textit{main}|]{|\textit{dest}|}"|
\end{center}
%
Here \textit{target} is the name of the output file,
\textit{main} is the name of the main file
and \textit{dest} is the name of the main or child file to be processed
(all filenames without extensions).
The optional argument \textit{main} can be omitted
if \textit{main} matches \textit{dest}.
Optionally, compilation \textit{flags} can be defined via |\def| commands.
This command line makes the \TeX{} engine believe
it is compiling the file \textit{target}
whose content is specified as the latter parameter.
The provided code then forwards the processing to
\textit{main} or \textit{dest} as described in \secref{sec:forward}.

%%%%%%%%%%%%%%%%%%%%%%%%%%%%%%%%%%%%%%%%%%%%%%%%%%%%%%%%%%%%%%%%%%%%%%%%%%%%%%%%
\subsection{Include by Input}
\label{sec:input}

Including child documents by |\include| has some restrictions by design.
Most notably, the content of a child document always occupies
its own set of pages; pages cannot be shared between child documents.
Usually, this behaviour makes perfect sense
because each child document contain an essential part of the document.
However, in some situations it may be desirable to compose
a document from a collection of parts
without having mandatory page breaks between then.
For this case, the package
provides a mechanism to include parts
by |\input| which can also be processed individually.
However, by construction this mechanism
requires manual handling of the content to be output.

%%%%%%%%%%%%%%%%%%%%%%%%%%%%%%%%%%%%%%%%
\DescribeMacro{\ifchilddocmanual}
The main file should be prepared as usual, see \secref{sec:include}.
However, the document body must make a distinction
between processing of an individual part and of the main document, e.g.:
%
\begin{center}
\begin{tabular}{l}
|\ifchilddocmanual|\\
|\input{\childdocname}|\\
|\||else|\\
\textit{document body with }|\input{|\textit{part}|}|\\
|\||fi|
\end{tabular}
\end{center}
%
The conditional |\ifchilddocmanual| is true whenever
a part to be included by |\input| is being compiled,
and the name of the part is stored in |\childdocname|.

%%%%%%%%%%%%%%%%%%%%%%%%%%%%%%%%%%%%%%%%
\DescribeMacro{\childdocby}
Each part to be included by |\input| should start with:
%
\begin{center}
\begin{tabular}{l}
|\input{childdoc.def}|\\
|\childdocby{|\textit{main}|}|\\
\end{tabular}
\end{center}
%
The directive |\childdocby| is similar to |\childdocof|
described in \secref{sec:include},
but the subsequent selection of content must be done manually.
To that end, both |\ifchilddoc| and |\ifchilddocmanual|
will be true upon processing of a part,
and the name of the part is stored in |\childdocname|.
Note that |\jobname| will be set to the filename of the current part
so that each part receives an individual |.aux| file
that does not interfere with the |.aux| file(s) of the main document.
This behaviour can be altered by the alternative form
|\childdocby[*]{|\textit{main}|}| (with a non-empty optional argument)
which uses the |.aux| file of the main document
by setting |\jobname| to \textit{main}.

%%%%%%%%%%%%%%%%%%%%%%%%%%%%%%%%%%%%%%%%%%%%%%%%%%%%%%%%%%%%%%%%%%%%%%%%%%%%%%%%
\subsection{Driver Development}
\label{sec:driver}

The \textsf{childdoc} mechanism can also be use for the development
of definition files such as \LaTeX{} styles or classes.
This case differs from the above setup with multiple parts
included by |\include| in that no |\includeonly| should be invoked.
This can be achieved by starting the include file
(before |\ProvidesPackage|) with:
%
\begin{center}
\begin{tabular}{l}
|\input{childdoc.def}|\\
|\childdocforward{|\textit{main}|}|\\
\end{tabular}
\end{center}
%
or alternatively with:
%
\begin{center}
\begin{tabular}{l}
|\input{childdoc.def}|\\
|\childdocby{|\textit{main}|}|\\
\end{tabular}
\end{center}
%
Both forms have slightly different effects as described above.
The main file is prepared as usual, see \secref{sec:include}.

%%%%%%%%%%%%%%%%%%%%%%%%%%%%%%%%%%%%%%%%%%%%%%%%%%%%%%%%%%%%%%%%%%%%%%%%%%%%%%%%
\subsection{Legacy Detection}
\label{sec:detection}

The directive |\childdocmain| in the main file can detect
whether the complete document or merely a child is to be compiled
even without using the directive |\childdocof|.
This method is deprecated because it is less robust
and there is no compelling reason to use it;
it is merely provided for backward compatibility
and it may be removed in future versions.

If the detection mechanism is to be used,
it is mandatory to correctly specify
the filename of the main file as the argument of |\childdocmain|:
%
\begin{center}
\begin{tabular}{l}
|\input{childdoc.def}|\\
|\childdocmain{|\textit{main}|}|\\
\end{tabular}
\end{center}
%
If |\jobname| does not match the argument \textit{main} of |\childdocmain|,
it is assumed that |\jobname| points to the child file to be compiled.
When using |\childdocmain| with the main file specified as argument,
it suffices to start a child file
with just |\input{|\textit{main}|}|
without loading of the package and using |\childdocof|.
If instead all processing is done
with the appropriate \textsf{childdoc} directives,
the argument of \textit{main} of |\childdocmain| can be empty.

An alternative version of the command line processing described
in \secref{sec:commandline} using the detection mechanism reads:
%
\begin{center}
|... -jobname "|\textit{target}|" "|[\textit{flags}]%
[|\def\jobname{|\textit{dest}|}|]|\input{|\textit{main}|}"|
\end{center}

%%%%%%%%%%%%%%%%%%%%%%%%%%%%%%%%%%%%%%%%%%%%%%%%%%%%%%%%%%%%%%%%%%%%%%%%%%%%%%%%
\subsection{Manual Code}
\label{sec:manual}

In case one cannot be certain whether the definitions file |childdoc.def|
is installed on the target \TeX{} distribution
and one prefers not to ship it,
it is conceivable to paste a few relevant commands into the sources.

To that end, drop all statements |\input{childdoc.def}|
and perform the replacements as outlined below.
Instead of |\childdocmain{|\textit{main}|}| add the following code
to the top of the main file:
%
\begin{center}
\begin{tabular}{l}
|\||ifdefined\childdocname\endinput\||fi\newif\ifchilddoc|\\
|\edef\childdocname{\scantokens\expandafter{\jobname\noexpand}}|\\
|\def\childdocmain{|\textit{main}|}\||ifx\childdocmain\childdocname\||else|\\
|\childdoctrue\includeonly{\childdocname}\let\jobname\childdocmain\||fi|\\
\end{tabular}
\end{center}
%
Instead of |\childdocof{|\textit{main}|}| just include the main file
at the top of each child file:
%
\begin{center}
|\input{|\textit{main}|}|
\end{center}
%
A simple redirection |\childdocforward{|\textit{dest}|}| is achieved by:
%
\begin{center}
|\def\jobname{|\textit{dest}|}\input{\jobname}|
\end{center}
%
The redirection with prefix
|\childdocforwardprefix[|\textit{prefix}|]{|\textit{dest}|}|
is accomplished by:
%
\begin{center}
\begin{tabular}{l}
|{\edef\jobname{\scantokens\expandafter{\jobname\noexpand}}|\\
|\def\redirectjob |\textit{prefix}|#1~~~{\gdef\jobname{|\textit{dest}|#1}}|\\
|\expandafter\redirectjob\jobname~~~}\input{\jobname}|
\end{tabular}
\end{center}

In an alternative approach,
child documents can be compiled by a specific command line
without additional code or specific definitions:
%
\begin{center}
|... -jobname "|\textit{target}|" "|[\textit{flags}]%
|\includeonly{|\textit{dest}|}\input{|\textit{main}|}"|
\end{center}
%

%%%%%%%%%%%%%%%%%%%%%%%%%%%%%%%%%%%%%%%%%%%%%%%%%%%%%%%%%%%%%%%%%%%%%%%%%%%%%%%%
%%%%%%%%%%%%%%%%%%%%%%%%%%%%%%%%%%%%%%%%%%%%%%%%%%%%%%%%%%%%%%%%%%%%%%%%%%%%%%%%
\section{Information}

%%%%%%%%%%%%%%%%%%%%%%%%%%%%%%%%%%%%%%%%%%%%%%%%%%%%%%%%%%%%%%%%%%%%%%%%%%%%%%%%
\subsection{Copyright}

Copyright \copyright{} 2017--2018 Niklas Beisert

This work may be distributed and/or modified under the
conditions of the \LaTeX{} Project Public License, either version 1.3
of this license or (at your option) any later version.
The latest version of this license is in
  \url{http://www.latex-project.org/lppl.txt}
and version 1.3 or later is part of all distributions of \LaTeX{}
version 2005/12/01 or later.

This work has the LPPL maintenance status `maintained'.

The Current Maintainer of this work is Niklas Beisert.

This work consists of the files |README.txt|, |childdoc.ins| and |childdoc.dtx|
as well as the derived files |childdoc.def|, |cdocsamp.tex|
with |cdocsch1.tex|, |cdocsch2.tex|, |cdocspt3.tex|, |cdocspt4.tex|,
|cdocsdrf.tex|, |cdocsfn1.tex|, |cdocsfn2.tex|
as well as |childdoc.pdf|.

%%%%%%%%%%%%%%%%%%%%%%%%%%%%%%%%%%%%%%%%%%%%%%%%%%%%%%%%%%%%%%%%%%%%%%%%%%%%%%%%
\subsection{Files and Installation}

The package consists of the files:
%
\begin{center}
\begin{tabular}{ll}
    |README.txt|   & readme file \\
    |childdoc.ins| & installation file \\
    |childdoc.dtx| & source file \\
    |childdoc.def| & definition file \\
    |cdocsamp.tex| & sample main file \\
    |cdocsch1.tex| & sample include file \\
    |cdocsch2.tex| & sample include file \\
    |cdocspt3.tex| & sample part file \\
    |cdocspt4.tex| & sample part file \\
    |cdocsdrf.tex| & sample redirection file \\
    |cdocsfn1.tex| & sample redirection file \\
    |cdocsfn2.tex| & sample redirection file \\
    |childdoc.pdf| & manual
\end{tabular}
\end{center}
%
The distribution consists of the files
|README.txt|, |childdoc.ins| and |childdoc.dtx|.
%
\begin{itemize}
\item
Run (pdf)\LaTeX{} on |childdoc.dtx|
to compile the manual |childdoc.pdf| (this file).
\item
Run \LaTeX{} on |childdoc.ins| to create the definitions file |childdoc.def|
and the sample |cdocsamp.tex| with include files
|cdocsch1.tex|, |cdocsch2.tex|, |cdocspt3.tex|, |cdocspt4.tex|,
|cdocsdrf.tex|, |cdocsfn1.tex|, |cdocsfn2.tex|.
Then copy the file |childdoc.def| to an appropriate directory of your \LaTeX{}
distribution, e.g.\ \textit{texmf-root}|/tex/latex/childdoc|.
\end{itemize}

%%%%%%%%%%%%%%%%%%%%%%%%%%%%%%%%%%%%%%%%%%%%%%%%%%%%%%%%%%%%%%%%%%%%%%%%%%%%%%%%
\subsection{Related CTAN Packages}

There are several other packages which offer a similar functionality:
%
\begin{itemize}
\item
The packages
\href{http://ctan.org/pkg/docmute}{\textsf{docmute}},
\href{http://ctan.org/pkg/includex}{\textsf{includex}} and
\href{http://ctan.org/pkg/standalone}{\textsf{standalone}}
provide commands to include only the document body of
a child file thus allowing both files to be compiled individually.
\item
The packages \href{http://ctan.org/pkg/subdocs}{\textsf{subdocs}}
and \href{http://ctan.org/pkg/subfiles}{\textsf{subfiles}}
provide structures in which the main and child documents can be
encapsulated and allowing them to be compiled individually.
The inclusion mechanism is different from the conventional |\include|.
\item
The package \href{http://ctan.org/pkg/combine}{\textsf{combine}}
is an elaborate solution to combine several documents into one.
\end{itemize}
%
See also the CTAN topic \href{http://ctan.org/topic/subdocs}{\textsf{subdocs}}
for further related packages.
The present package differs from the above solutions in that
a document structure constructed with the conventional |\include| mechanism
just needs two extra commands at the top of every file
such that all constituent files can be compiled individually.

%%%%%%%%%%%%%%%%%%%%%%%%%%%%%%%%%%%%%%%%%%%%%%%%%%%%%%%%%%%%%%%%%%%%%%%%%%%%%%%%
%\subsection{Feature Suggestions}
%
%The following is a list of features which may be useful for future
%versions of this package:
%%
%\begin{itemize}
%\item
%\ldots
%\end{itemize}

%%%%%%%%%%%%%%%%%%%%%%%%%%%%%%%%%%%%%%%%%%%%%%%%%%%%%%%%%%%%%%%%%%%%%%%%%%%%%%%%
\subsection{Revision History}

%%%%%%%%%%%%%%%%%%%%%%%%%%%%%%%%%%%%%%%%
\paragraph{v2.0:} 2018/12/30

\begin{itemize}
\item
immediate forward processing
\item
added |\childdocby| mechanism
\item
manual restructured
\end{itemize}

%%%%%%%%%%%%%%%%%%%%%%%%%%%%%%%%%%%%%%%%
\paragraph{v1.6:} 2018/01/17

\begin{itemize}
\item
application for development of include files
\item
corrections to manual
\end{itemize}

%%%%%%%%%%%%%%%%%%%%%%%%%%%%%%%%%%%%%%%%
\paragraph{v1.5:} 2017/05/21

\begin{itemize}
\item
more complete structuring introduced
\item
|\childdocof| introduced
\item
|\childdoc| renamed to |\childdocmain|
\item
|\childredirect| renamed to |\childdocforward| and |\childdocforwardprefix|
and functionality expanded
\end{itemize}

%%%%%%%%%%%%%%%%%%%%%%%%%%%%%%%%%%%%%%%%
\paragraph{v1.0:} 2017/04/27

\begin{itemize}
\item
manual and install package
\item
first version published on CTAN
\end{itemize}

%%%%%%%%%%%%%%%%%%%%%%%%%%%%%%%%%%%%%%%%
\paragraph{v0.6:} 2017/04/26

\begin{itemize}
\item
redirection mechanism added
\end{itemize}

%%%%%%%%%%%%%%%%%%%%%%%%%%%%%%%%%%%%%%%%
\paragraph{v0.5:} 2017/04/26

\begin{itemize}
\item
functionality in definition file
\end{itemize}


%%%%%%%%%%%%%%%%%%%%%%%%%%%%%%%%%%%%%%%%%%%%%%%%%%%%%%%%%%%%%%%%%%%%%%%%%%%%%%%%
%%%%%%%%%%%%%%%%%%%%%%%%%%%%%%%%%%%%%%%%%%%%%%%%%%%%%%%%%%%%%%%%%%%%%%%%%%%%%%%%
%%%%%%%%%%%%%%%%%%%%%%%%%%%%%%%%%%%%%%%%%%%%%%%%%%%%%%%%%%%%%%%%%%%%%%%%%%%%%%%%
\appendix

\settowidth\MacroIndent{\rmfamily\scriptsize 000\ }

 \DocInput{childdoc.dtx}

\end{document}
%</driver>
% \fi
%
% %%%%%%%%%%%%%%%%%%%%%%%%%%%%%%%%%%%%%%%%%%%%%%%%%%%%%%%%%%%%%%%%%%%%%%%%%%%%%%
% %%%%%%%%%%%%%%%%%%%%%%%%%%%%%%%%%%%%%%%%%%%%%%%%%%%%%%%%%%%%%%%%%%%%%%%%%%%%%%
% \section{Sample}
%\iffalse
%<*samplemain>
%\fi
%
% The following presents a sample document
% with two chapters, two parts, a title page,
% a compile flag as well as three forwarding files to set the flag.
% It consists of eight |.tex| files:
% \begin{center}
% \begin{tabular}{ll}
% |cdocsamp.tex|&main file\\
% |cdocsch1.tex|&include file for chapter 1\\
% |cdocsch2.tex|&include file for chapter 2\\
% |cdocspt3.tex|&include file for part 3\\
% |cdocspt4.tex|&include file for part 4\\
% |cdocsdrf.tex|&forwarding file for main file in draft mode\\
% |cdocsfi1.tex|&forwarding file for final version of chapter 1\\
% |cdocsfi2.tex|&forwarding file for final version of chapter 2\\
% \end{tabular}
% \end{center}
% Each of the eight files can be compiled directly by the \LaTeX{} compiler.
%
% %%%%%%%%%%%%%%%%%%%%%%%%%%%%%%%%%%%%%%
% \paragraph{Main File.}
%
% The main file is called |cdocsamp.tex|.
%
% Load the \textsf{childdoc} definitions and
% declare the filename for the main document:
%    \begin{macrocode}
\input{childdoc.def}
\childdocmain{}
%    \end{macrocode}

% Optional override for |\version| flag:
%    \begin{macrocode}
%%\ifchilddoc\else\providecommand{\version}{draft}\fi
%    \end{macrocode}

% Define the default values for the |\version| flag
% (|final| for the main file and |draft| for childs):
%    \begin{macrocode}
\ifchilddoc
\providecommand{\version}{draft}
\else
\providecommand{\version}{final}
\fi
%    \end{macrocode}

% Load the standard document class:
%    \begin{macrocode}
\documentclass[12pt]{article}
%    \end{macrocode}

% Start the document body:
%    \begin{macrocode}
\begin{document}
%    \end{macrocode}

% Declare a title page.
% Print title, part of document being processed and version flag:
%    \begin{macrocode}
\addtocounter{page}{-1}
\begin{center}
{\LARGE\bfseries{}childdoc example\par}
\vspace{1cm}
\ifchilddoc
\ifchilddocmanual part\else chapter\fi:
`\childdocname' of `\childdocjob'\par
\else
main document: `\childdocjob'\par
\fi
version: \version\par
\end{center}
\newpage
%    \end{macrocode}

% Manually include selected file,
% otherwise process as usual:
%    \begin{macrocode}
\ifchilddocmanual
\section*{part `\childdocname'}
\input{\childdocname}
\else
%    \end{macrocode}

% Include the two chapters:
%    \begin{macrocode}
\include{cdocsch1}
\include{cdocsch2}
%    \end{macrocode}

% Include the two parts unless only chapters should be displayed:
%    \begin{macrocode}
\ifchilddoc\else
\section{part three}
\input{cdocspt3}
\section{part four}
\input{cdocspt4}
\fi
%    \end{macrocode}

% Process as usual until here:
%    \begin{macrocode}
\fi
%    \end{macrocode}

% End of document body:
%    \begin{macrocode}
\end{document}
%    \end{macrocode}
%\iffalse
%</samplemain>
%\fi
%
% %%%%%%%%%%%%%%%%%%%%%%%%%%%%%%%%%%%%%%
% \paragraph{Chapter Include Files.}
%
% The include files are called |cdocsch1.tex| and |cdocsch2.tex|.
%
%\iffalse
%<*samplechap1|samplechap2>
%\fi

% Optional override for |\version| flag:
%    \begin{macrocode}
%%\providecommand{\version}{final}
%    \end{macrocode}

% Include the main document:
%    \begin{macrocode}
\input{childdoc.def}
\childdocof{cdocsamp}
%    \end{macrocode}

%\iffalse
%</samplechap1|samplechap2>
%\fi
%
%\iffalse
%<*samplechap1>
%\fi
% Some text for chapter 1:
%    \begin{macrocode}
\section{one}
some text in chapter one
%    \end{macrocode}

%\iffalse
%</samplechap1>
%\fi
% Some text for chapter 2:
%\iffalse
%<*samplechap2>
%\fi
%    \begin{macrocode}
\section{two}
more text in chapter two
%    \end{macrocode}

%\iffalse
%</samplechap2>
%\fi
%
% %%%%%%%%%%%%%%%%%%%%%%%%%%%%%%%%%%%%%%
% \paragraph{Part Include Files.}
%
% The include files are called |cdocspt3.tex| and |cdocspt4.tex|.
%
%\iffalse
%<*samplepart3|samplepart4>
%\fi

% Optional override for |\version| flag:
%    \begin{macrocode}
%%\providecommand{\version}{final}
%    \end{macrocode}

% Include the main document:
%    \begin{macrocode}
\input{childdoc.def}
\childdocby{cdocsamp}
%    \end{macrocode}

%\iffalse
%</samplepart3|samplepart4>
%\fi
%
%\iffalse
%<*samplepart3>
%\fi
% Some text for part 3:
%    \begin{macrocode}
some text in part three
%    \end{macrocode}

%\iffalse
%</samplepart3>
%\fi
% Some text for part 4:
%\iffalse
%<*samplepart4>
%\fi
%    \begin{macrocode}
more text in part four
%    \end{macrocode}

%\iffalse
%</samplepart4>
%\fi
%
% %%%%%%%%%%%%%%%%%%%%%%%%%%%%%%%%%%%%%%
% \paragraph{Forwarding for a Complete Draft.}
%
% The following forwarding file |cdocsdrf.tex|
% compiles the main document in draft mode:
%\iffalse
%<*sampledraft>
%\fi
%    \begin{macrocode}
\def\version{draft}
\input{childdoc.def}
\childdocforward{cdocsamp}
%    \end{macrocode}

%\iffalse
%</sampledraft>
%\fi
%
% %%%%%%%%%%%%%%%%%%%%%%%%%%%%%%%%%%%%%%
% \paragraph{Forwarding for Final Version of the Chapters.}
%
% The following forwarding files |cdocsfn1.tex| and |cdocsfn2.tex|
% (with identical content)
% compile the final versions of the child documents
% |cdocsch1.tex| and |cdocsch2.tex|, respectively:
%\iffalse
%<*samplefinal>
%\fi
%    \begin{macrocode}
\def\version{final}
\input{childdoc.def}
\childdocforwardprefix[cdocsamp]{cdocsfn}{cdocsch}
%    \end{macrocode}

%\iffalse
%</samplefinal>
%\fi
%
% %%%%%%%%%%%%%%%%%%%%%%%%%%%%%%%%%%%%%%
% \paragraph{Command Line Processing.}
%
% The following three command lines generate the output files
% |cdocscld|, |cdocscl1| and |cdocscl2|
% which should be identical to
% |cdocsdrf|, |cdocsch1| and |cdocsfn2|, respectively:
% \begin{center}
% \begin{tabular}{l}
% |latex -jobname cdocscld \|\\
% |  "\def\version{draft}\input{childdoc.def}\childdocforward{cdocsamp}"|\\
% |latex -jobname cdocscl1 \|\\
% |  "\input{childdoc.def}\childdocforward[cdocsamp]{cdocsch1}"|\\
% |latex -jobname cdocscl2 \|\\
% |  "\def\version{final}\input{childdoc.def}\childdocforward{cdocsch2}"|
% \end{tabular}
% \end{center}
% Note that the trailing backslash on each first line
% merely continues the input to the second line
% (for convenient cut ant paste).
% Furthermore, the command |latex| can be replaced by any
% of its alternative versions such as |pdflatex|.
%
% %%%%%%%%%%%%%%%%%%%%%%%%%%%%%%%%%%%%%%%%%%%%%%%%%%%%%%%%%%%%%%%%%%%%%%%%%%%%%%
% %%%%%%%%%%%%%%%%%%%%%%%%%%%%%%%%%%%%%%%%%%%%%%%%%%%%%%%%%%%%%%%%%%%%%%%%%%%%%%
% \section{Implementation}
%\iffalse
%<*package>
%\fi
%
% This section describes the definitions file |childdoc.def|.

% The definitions cannot be loaded using |\usepackage| or |\RequirePackage|
% which has a mechanism to prevent loading a style file more than once.
% When loading the definitions by means of |\input|
% multiple instances have to be prevented manually:
%\iffalse
%This code needs to be before the `\ProvidesFile' directive
%which is defined at the beginning of this file.
%Therefore it is also placed there and commented out here.
%</package>
%<*discard>
%\fi
%    \begin{macrocode}
\ifdefined\childdocmain\endinput\fi
%    \end{macrocode}
%\iffalse
%</discard>
%<*package>
%\fi
%
% \macro{\ifchilddoc}
% \macro{\ifchilddocmanual}
% The conditional |\ifchilddoc| tells whether a
% child (true) or main (false) document is being compiled.
% The conditional |\ifchilddocmanual| tells whether
% the |\includeonly| mechanism is used (false) or
% the selection of child files must be performed manually (true).
% The definitions initialise to false:
%    \begin{macrocode}
\newif\ifchilddoc
\newif\ifchilddocmanual
%    \end{macrocode}

% \macro{\childdocname}
% \macro{\childdocjob}
% The macro |\childdocname| stores the name of the main document
% to be compiled. The macro |\childdocjob| stores the name of
% the document on which the \LaTeX{} compiler was originally invoked.
% The content of |\jobname| cannot be compared
% to filenames specified in the source due to different catcodes.
% The following code rescans |\jobname|, stores the result
% in |\childdocname| and saves a copy in |\childdocjob|:
%    \begin{macrocode}
\edef\childdocname{\scantokens\expandafter{\jobname\noexpand}}
\let\childdocjob\childdocname
%    \end{macrocode}

% \macro{\childdocdisable}
% The macro |\childdocdisable| prevents the main file
% from being processed more than once.
% At this stage, the main document command |\childdocmain|
% is assumed to be called once again where it should do nothing.
% Any subsequent call to it should prevent
% a secondary processing of the main document
% It overwrites the forwarding commands
% |\childdocof| and |\childdocforward|
% with empty macros to prevent further inclusions of the main document:
%    \begin{macrocode}
\newcommand{\childdocdisable}
{
  \renewcommand{\childdocmain}[1]{\renewcommand{\childdocmain}[1]{\endinput}}
  \renewcommand{\childdocof}[1]{}
  \renewcommand{\childdocby}[2][]{}
  \renewcommand{\childdocforward}[2][]{}
  \renewcommand{\childdocdisable}{}
}
%    \end{macrocode}

% \macro{\childdocmain}
% The macro |\childdocmain| is to be called at the top of the main file
% with nothing or the main filename (without extension) as argument.
% First, it breaks loops.
% If the argument is not empty and does not match |\childdocname|
% (which is set by the first inclusion of |childdoc.def|),
% |\ifchilddoc| is set to true, |\includeonly| is applied to the child file
% and |\jobname| is set to the main file
% (for proper handling of |.aux| files):
%    \begin{macrocode}
\newcommand{\childdocmain}[1]
{
  \childdocdisable\childdocmain{}
  \if?#1?\else
    \begingroup
      \def\childdoctmp{#1}
      \ifx\childdoctmp\childdocname
        \def\childdoctmp{}
      \else
        \def\childdoctmp
        {
          \childdoctrue
          \includeonly{\childdocname}
          \def\childdocjob{#1}
          \def\jobname{#1}
        }
      \fi
      \expandafter
    \endgroup
    \childdoctmp
  \fi
}
%    \end{macrocode}

% \macro{\childdocof}
% The command |\childdocof| redirects
% compilation to the main file |#1|.
%    \begin{macrocode}
\newcommand{\childdocof}[1]
{
  \childdocdisable
  \childdoctrue
  \includeonly{\childdocname}
  \def\jobname{#1}
  \def\childdocjob{#1}
  \input{#1}
}
%    \end{macrocode}

% \macro{\childdocby}
% The command |\childdocby| ....
%    \begin{macrocode}
\newcommand{\childdocby}[2][]
{
  \childdocdisable
  \childdoctrue
  \childdocmanualtrue
  \if?#1?\else
    \def\jobname{#2}
  \fi
  \def\childdocjob{#2}
  \input{#2}
  \endinput
}
%    \end{macrocode}

% \macro{\childdocforward}
% The command |\childdocforward| redirects
% compilation to the main file or
% (if the optional argument is given) a child file.
% Parameters are set as if the main file
% or a child file starting with |\childdocof| was compiled.
% Then compilation is handed over to the main file:
%    \begin{macrocode}
\newcommand{\childdocforward}[2][]
{
  \begingroup
    \if?#1?
      \def\childdoctmp
      {
        \def\childdocname{#2}
        \def\childdocjob{#2}
        \def\jobname{#2}
        \input{#2}
        \endinput
      }
    \else
      \def\childdoctmp
      {
        \childdocdisable
        \def\childdocname{#2}
        \childdoctrue
        \includeonly{#2}
        \def\childdocjob{#1}
        \def\jobname{#1}
        \input{#1}
        \endinput
      }
    \fi
    \expandafter
  \endgroup
  \childdoctmp
}
%    \end{macrocode}

% \macro{\childdocforwardprefix}
% The command |\childdocforwardprefix| redirects
% compilation to the main or a child file by means of a pattern.
% The prefix |#1| in the current filename is replaced by |#2|
% and the suffix of the current filename is kept
% (it is assumed that the filename does not contain the substring `|~~~|'
% which is used as a delimiter).
% Compilation is handed over to the new file by |\childdocforward|:
%    \begin{macrocode}
\newcommand{\childdocforwardprefix}[3][]
{
  \begingroup
    \def\childdocextract #2##1~~~{\def\childdoctmp{\childdocforward[#1]{#3##1}}}
    \expandafter\childdocextract\childdocname~~~
    \expandafter
  \endgroup
  \childdoctmp
}
%    \end{macrocode}

% \macro{\childdoc}
% The deprecated macro |\childdoc| is a legacy version of |\childdocmain|:
%    \begin{macrocode}
\newcommand{\childdoc}{\childdocmain}
%    \end{macrocode}

% \macro{\childdocredirect}
% The deprecated macro |\childdocredirect| is a legacy version
% of |\childdocforward| and |\childdocforwardprefix|:
%    \begin{macrocode}
\newcommand{\childdocredirect}[2][]
{
  \begingroup
    \if?#1?
      \def\childdoctmp{\childdocforward{#2}}
    \else
      \def\childdoctmp{\childdocforwardprefix{#1}{#2}}
    \fi
    \expandafter
  \endgroup
  \childdoctmp
}
%    \end{macrocode}

%\iffalse
%</package>
%\fi
%
\endinput
|\\
|\childdocforwardprefix[|\textit{main}|]{|\textit{prefix}|}{|\textit{dest}|}|
\end{tabular}
\end{center}
%
the destination file is determined by a pattern
depending on the current file:
To make this work, the current file must be called
`{\textit{prefix}\hspace{0.2em}\textit{suffix}}'
with \textit{prefix} matching precisely the argument.
Processing is then passed on to the file
`{\textit{dest}\hspace{0.2em}\textit{suffix}}'.
Surely, the same effect is achieved by
directly specifying the
argument `{\textit{dest}\hspace{0.2em}\textit{suffix}}'
in the first form.
However, that requires to set up a different file
for each child. With the alternative form of the command
all these files can have exactly the same content
which simplifies setting them up and maintaining them.

For example, the following file |draft.tex|
with a compilation flag |\version| as described in \secref{sec:flags}
compiles the main document as a draft:
%
\begin{center}
\begin{tabular}{l}
|\def\version{draft}|\\
|% \iffalse
%
% childdoc.dtx Copyright (C) 2017-2018 Niklas Beisert
%
% This work may be distributed and/or modified under the
% conditions of the LaTeX Project Public License, either version 1.3
% of this license or (at your option) any later version.
% The latest version of this license is in
%   http://www.latex-project.org/lppl.txt
% and version 1.3 or later is part of all distributions of LaTeX
% version 2005/12/01 or later.
%
% This work has the LPPL maintenance status `maintained'.
%
% The Current Maintainer of this work is Niklas Beisert.
%
% This work consists of the files childdoc.dtx and childdoc.ins
% and the derived files childdoc.def and cdocsamp.tex with
% cdocsch1.tex, cdocsch2.tex, cdocsdrf.tex, cdocsfn1.tex, cdocsfn2.tex.
%
%<package>\ifdefined\childdocmain\endinput\fi
%<package>\ProvidesFile{childdoc.def}[2018/12/30 v2.0 child document driver]
%<samplemain>\ProvidesFile{cdocsamp.tex}[2018/12/30 v2.0 sample for childdoc]
%<*driver>
%\ProvidesFile{childdoc.drv}[2018/12/30 v2.0 childdoc reference manual file]
\PassOptionsToClass{10pt,a4paper}{article}
\documentclass{ltxdoc}

\usepackage[margin=35mm]{geometry}
\usepackage{hyperref}
\usepackage{hyperxmp}
\usepackage[usenames]{color}

\hypersetup{colorlinks=true}
\hypersetup{pdfstartview=FitH}
\hypersetup{pdfpagemode=UseNone}
\hypersetup{pdfsource={}}
\hypersetup{pdflang={en-UK}}
\hypersetup{pdfcopyright={Copyright 2017-2018 Niklas Beisert.
  This work may be distributed and/or modified under the
  conditions of the LaTeX Project Public License, either version 1.3
  of this license or (at your option) any later version.}}
\hypersetup{pdflicenseurl={http://www.latex-project.org/lppl.txt}}
\hypersetup{pdfcontactaddress={ETH Zurich, ITP, HIT K,
  Wolfgang-Pauli-Strasse 27}}
\hypersetup{pdfcontactpostcode={8093}}
\hypersetup{pdfcontactcity={Zurich}}
\hypersetup{pdfcontactcountry={Switzerland}}
\hypersetup{pdfcontactemail={nbeisert@itp.phys.ethz.ch}}
\hypersetup{pdfcontacturl={http://people.phys.ethz.ch/\xmptilde nbeisert/}}

\newcommand{\secref}[1]{\hyperref[#1]{section \ref*{#1}}}

\parskip1ex
\parindent0pt
\let\olditemize\itemize
\def\itemize{\olditemize\parskip0pt}

\begin{document}

\title{The \textsf{childdoc} Package}
\hypersetup{pdftitle={The childdoc Package}}
\author{Niklas Beisert\\[2ex]
  Institut f\"ur Theoretische Physik\\
  Eidgen\"ossische Technische Hochschule Z\"urich\\
  Wolfgang-Pauli-Strasse 27, 8093 Z\"urich, Switzerland\\[1ex]
  \href{mailto:nbeisert@itp.phys.ethz.ch}
  {\texttt{nbeisert@itp.phys.ethz.ch}}}
\hypersetup{pdfauthor={Niklas Beisert}}
\hypersetup{pdfsubject={Manual for the LaTeX2e Package childdoc}}
\date{30 December 2018, \textsf{v2.0}}
\maketitle

\begin{abstract}\noindent
\textsf{childdoc} is a \LaTeXe{} package
that enables the direct compilation
of document sections included by |\include|
to individual files.
\end{abstract}

\begingroup
\parskip0ex
\tableofcontents
\endgroup

%%%%%%%%%%%%%%%%%%%%%%%%%%%%%%%%%%%%%%%%%%%%%%%%%%%%%%%%%%%%%%%%%%%%%%%%%%%%%%%%
%%%%%%%%%%%%%%%%%%%%%%%%%%%%%%%%%%%%%%%%%%%%%%%%%%%%%%%%%%%%%%%%%%%%%%%%%%%%%%%%
\section{Introduction}

\LaTeX{} provides a mechanism to structure a large document (such as a book)
into a main file and several child files (containing the chapters)
using the |\include| command.
This mechanism is beneficial for documents
which span hundreds of pages in order to
make the source file(s) more manageable.
Moreover, compilation can be restricted to
selected child files by means of the |\includeonly| command.
The latter feature can be used to reduce the compilation time while editing
(this was significantly more useful in the earlier days of \LaTeX{})
or to generate a smaller document which is easier to navigate.
Another application of |\includeonly| is to generate
documents consisting of selected parts of the complete document.

However, there are a few drawbacks of the plain |\include| mechanism:
\begin{itemize}
\item
The child files cannot be compiled on their own,
they can only be compiled via the main file.
A naive editing environment
(such as a text editor with an option
to have the current file processed by \LaTeX)
may require one to switch to the main file before compiling;
attempting to compile the child file produces errors.
\item
The main file must be modified (each time)
to adjust the |\includeonly| command
to the present needs. This easily leaves the main file in a messy state.
\item
The generated document will always carry the filename
of the main document. This is inconvenient if
several child files are to be compiled and
to be kept for distribution.
\end{itemize}

The present package provides a simple interface
to make child files individually compilable by \LaTeX{}.
Compiling a child file then has the same effect as compiling
the main file with an |\includeonly| command
to select the appropriate child.
Moreover the generated document will carry the name of the child
rather than the main file.
This resolves all three above issues.

This feature is meant to make the editing of books,
thesis documents and lecture notes somewhat more convenient.
However, the package can also be used efficiently for
composing a series of documents (such as exercise sheets)
which are typically distributed individually.
It then assists the author in generating the individual documents
(potentially in different versions)
as well as a document containing the collected series.
Another application is in developing style files
or other kinds of included material
where compilation of the style file could redirect
to a sample or test file.

%%%%%%%%%%%%%%%%%%%%%%%%%%%%%%%%%%%%%%%%%%%%%%%%%%%%%%%%%%%%%%%%%%%%%%%%%%%%%%%%
%%%%%%%%%%%%%%%%%%%%%%%%%%%%%%%%%%%%%%%%%%%%%%%%%%%%%%%%%%%%%%%%%%%%%%%%%%%%%%%%
\section{Usage}

First of all, the package \textsf{childdoc} is \emph{not} a standard
\LaTeXe{} |.sty| style file! Therefore it needs to be invoked in
a non-standard way.

%%%%%%%%%%%%%%%%%%%%%%%%%%%%%%%%%%%%%%%%%%%%%%%%%%%%%%%%%%%%%%%%%%%%%%%%%%%%%%%%
\subsection{Included Files}
\label{sec:include}

%%%%%%%%%%%%%%%%%%%%%%%%%%%%%%%%%%%%%%%%
\DescribeMacro{\childdocmain}
To use the package, add the commands
\begin{center}
\begin{tabular}{l}
|\input{childdoc.def}|\\
|\childdocmain{}|\\
\end{tabular}
\end{center}
at the very top of the main \LaTeX{} file,
in particular \emph{before} the |\documentclass| statement!
The argument of |\childdocmain| should be left empty
(but it must be present).

%%%%%%%%%%%%%%%%%%%%%%%%%%%%%%%%%%%%%%%%
\DescribeMacro{\childdocof}
Furthermore, add the commands
\begin{center}
\begin{tabular}{l}
|\input{childdoc.def}|\\
|\childdocof{|\textit{main}|}|\\
\end{tabular}
\end{center}
at the top of every child file \textit{child}
which is included by |\include{|\textit{child}|}|
from within the main file
(or at least for those files to be compiled individually).
The argument \textit{main} must be the filename of the main file.

There are a couple of
considerations in setting up the main and child documents:

%%%%%%%%%%%%%%%%%%%%%%%%%%%%%%%%%%%%%%%%
\paragraph{Restrictions.}

Please note the following restrictions:
\begin{itemize}
\item
|\childdocmain| must be called with one argument \textit{main}
to ensure compatibility with earlier version of the package.
It must either be empty (|\childdocmain{}|)
or precisely match the filename of the main file in which it is specified.
See \secref{sec:detection} for further information.
\item
The filename \textit{main} must be specified without the |.tex| extension.
\item
The filename \textit{main} is case sensitive
(even in case-insensitive file systems)
due to internal string comparison.
\item
The argument \textit{main} should be fully expanded, it cannot be a macro.
\item
Subdirectories and special characters should be avoided in filenames.
\item
The command |\childdocmain{|\textit{main}|}| must be followed by a whitespace.
It should not be followed immediately by another command
or by a comment mark `|%|'.
This is because the \TeX{} parser reads the token immediately following
the argument of |\childdocmain| and puts it
at the beginning of every child section;
however, a white\-space is ignored.
\end{itemize}

%%%%%%%%%%%%%%%%%%%%%%%%%%%%%%%%%%%%%%%%
\paragraph{Content of Main File.}

It is advisable to place all content in the child files included by |\include|.
Any output contained in the main file will appear in all child documents
unless suppressed manually;
it cannot be suppressed automatically by the |\includeonly| directive
and thus should normally be avoided.
A method to include some content in the main file
by means of conditional processing is described in \secref{sec:conditional}.

%%%%%%%%%%%%%%%%%%%%%%%%%%%%%%%%%%%%%%%%
\paragraph{Page Numbering.}

When only a part of the document is compiled,
the appropriate numbering of pages
(as well as other status parameters)
is determined from the |.aux| files.
The latter contain information from previous passes.
However this information needs to propagate through
all intermediate child documents.
Therefore the page numbering in child documents may well
be inconsistent until the complete document is compiled at least once.

A useful (if unconventional) way to always ensure a consistent
page numbering is to restart the numbering in each child document
and denote the pages by `\textit{child}|.|\textit{page}'
where \textit{child} represents the chapter/section number of the child file.
This can be achieved by the command
|\numberwithin{page}{|\textit{child}|}|
of the \textsf{amsmath} package
where \textit{child} can be |chapter| or |section|
depending on the chosen structuring.
Alternatively, one can modify the macro |\thepage| appropriately
and reset the counter |page| at the start of each child file.

%%%%%%%%%%%%%%%%%%%%%%%%%%%%%%%%%%%%%%%%%%%%%%%%%%%%%%%%%%%%%%%%%%%%%%%%%%%%%%%%
\subsection{Conditional Processing}
\label{sec:conditional}

The package provides a mechanism to compile different versions
of a document. To customise the versions further some conditional processing
can come in handy to distinguish which version is being compiled.
The package provides two macros to describe the compilation context:

%%%%%%%%%%%%%%%%%%%%%%%%%%%%%%%%%%%%%%%%
\DescribeMacro{\ifchilddoc}
The conditional |\ifchilddoc| distinguishes between the compilation of
child documents and the main document:
%
\begin{center}
|\ifchilddoc |\textit{child-code}| |[|\||else |\textit{main-code}]| \||fi|
\end{center}

%%%%%%%%%%%%%%%%%%%%%%%%%%%%%%%%%%%%%%%%
\DescribeMacro{\childdocname}
\DescribeMacro{\childdocjob}
The macro |\childdocname| contains the filename (without extension)
of the main or child file being processed.
Note that |\childdocjob| will always contain the name of the main file.

%%%%%%%%%%%%%%%%%%%%%%%%%%%%%%%%%%%%%%%%
\paragraph{Title Page.}

Conditional processing can be used to include a title or banner page
in the main document when proper precautions are taken.
Importantly, the code in the main file should ensure that the page counter
(as well as other status parameters which are stored in the |.aux| files)
takes the same value after the conditional processing.
Otherwise the page numbers may take divergent values
depending on which part is compiled.

For example, a title page could be declared by:
%
\begin{center}
\begin{tabular}{l}
|\ifchilddoc\||else|\\
|\addtocounter{page}{-1}|\\
\textit{code for title page}\\
|\newpage|\\
|\||fi|
\end{tabular}
\end{center}
%
A banner page for the child documents can be generated by:
%
\begin{center}
\begin{tabular}{l}
|\ifchilddoc|\\
|\addtocounter{page}{-1}|\\
\textit{code for banner page}\\
|\newpage|\\
|\||fi|
\end{tabular}
\end{center}
%
Here one could write a message such as:
\begin{center}
|This is the part \childdocname{} of \childdocjob{}.|
\end{center}

%%%%%%%%%%%%%%%%%%%%%%%%%%%%%%%%%%%%%%%%%%%%%%%%%%%%%%%%%%%%%%%%%%%%%%%%%%%%%%%%
\subsection{Flags}
\label{sec:flags}

The package makes it easy to generate different versions
of the main or child documents.
To this end compilation flags can be defined
and assigned different default values.
They will be particularly useful in conjunction
with the forwarding mechanism described in \secref{sec:forward}.

For example, it may be useful to have a flag |\version|
which can be set to |draft| or |final|.
The document source will contain some conditional code
depending on the value of |\version|.
Suppose further, the flag should default to |final| for the main file
and to |draft| for child files
which is a natural assignment for editing the document.
This is achieved by placing the following code
in the preamble of the main document
(below the |\childdocmain| directive):
%
\begin{center}
\begin{tabular}{l}
|\ifchilddoc|\\
|\providecommand{\version}{draft}|\\
|\||else|\\
|\providecommand{\version}{final}|\\
|\||fi|
\end{tabular}
\end{center}
%
The definition by |\providecommand| makes sure
that previous definitions are not overwritten.
Further statements |\providecommand{\version}{...}|
can thus be added before the above code to override it.

For the main file, one might add a line
(between |\childdocmain| and the above block)
%
\begin{center}
|%\ifchilddoc\||else\providecommand{\version}{draft}\||fi|
\end{center}
%
which can be uncommented to produce a draft version.
Likewise one can add a line to the very top of a child file
(above the |\childdocof{|\textit{main}|}| directive)
%
\begin{center}
|%\providecommand{\version}{final}|
\end{center}
%
which can be uncommented to produce the final version of this child document.

%%%%%%%%%%%%%%%%%%%%%%%%%%%%%%%%%%%%%%%%%%%%%%%%%%%%%%%%%%%%%%%%%%%%%%%%%%%%%%%%
\subsection{Forwarding}
\label{sec:forward}

Different versions of the main or child documents
using compilation flags as described in \secref{sec:flags}
can be (permanently) stored in different files
for convenient compilation, viewing and distribution.
To this end, the package defines a command
to pass on compilation to a different file:

%%%%%%%%%%%%%%%%%%%%%%%%%%%%%%%%%%%%%%%%
\DescribeMacro{\childdocforward}
The command |\childdocforward| redirects processing to
another source file:
%
\begin{center}
\begin{tabular}{l}
|\input{childdoc.def}|\\
|\childdocforward[|\textit{main}|]{|\textit{dest}|}|\\
\end{tabular}
\end{center}
%
The argument \textit{dest} is the destination file
(without extension).
It should be the main file or one of the child files.
Note that further \textsf{childdoc} directives
such as |\childdocof| and |\childdocforward|
in the indicated file will be processed in this form.
The optional argument \textit{main}
passes on directly to the main file \textit{main}
while pretending to compile the child \textit{dest}.
This form behaves as if \textit{dest}
issues |\childdocof{|\textit{main}|}| right away,
and no further \textsf{childdoc} directives will be processed.

%%%%%%%%%%%%%%%%%%%%%%%%%%%%%%%%%%%%%%%%
\DescribeMacro{\...prefix}
In the alternative form |\childdocforwardprefix|,
%
\begin{center}
\begin{tabular}{l}
|\input{childdoc.def}|\\
|\childdocforwardprefix[|\textit{main}|]{|\textit{prefix}|}{|\textit{dest}|}|
\end{tabular}
\end{center}
%
the destination file is determined by a pattern
depending on the current file:
To make this work, the current file must be called
`{\textit{prefix}\hspace{0.2em}\textit{suffix}}'
with \textit{prefix} matching precisely the argument.
Processing is then passed on to the file
`{\textit{dest}\hspace{0.2em}\textit{suffix}}'.
Surely, the same effect is achieved by
directly specifying the
argument `{\textit{dest}\hspace{0.2em}\textit{suffix}}'
in the first form.
However, that requires to set up a different file
for each child. With the alternative form of the command
all these files can have exactly the same content
which simplifies setting them up and maintaining them.

For example, the following file |draft.tex|
with a compilation flag |\version| as described in \secref{sec:flags}
compiles the main document as a draft:
%
\begin{center}
\begin{tabular}{l}
|\def\version{draft}|\\
|\input{childdoc.def}|\\
|\childdocforward{|\textit{main}|}|
\end{tabular}
\end{center}
%
Likewise, the following files |final|\textit{nn}|.tex|
compile the final version of the child document
|child|\textit{nn}|.tex|:
%
\begin{center}
\begin{tabular}{l}
|\def\version{final}|\\
|\input{childdoc.def}|\\
|\childdocforwardprefix{final}{child}|
\end{tabular}
\end{center}
%

Note that when several versions of a main file and/or of each child file
are to be generated, it may be convenient to set up a |Makefile| or
shell script to automatise the process.

%%%%%%%%%%%%%%%%%%%%%%%%%%%%%%%%%%%%%%%%%%%%%%%%%%%%%%%%%%%%%%%%%%%%%%%%%%%%%%%%
\subsection{Command Line Processing}
\label{sec:commandline}

The effect of redirection files can also be achieved by invoking
the \LaTeX{} compiler with a more elaborate command line.
Most conveniently this should be done as part
of a shell script or a |Makefile|.

When using \textsf{childdoc} in the main file, the following
command lines effectively perform a redirection
(note that depending on the shell being used,
backslashes may have to be doubled: `|\|' $\to$ `|\\|'):
%
\begin{center}
|... -jobname "|\textit{target}|" |\\|"|[\textit{flags}]%
|\input{childdoc.def}\childdocforward[|\textit{main}|]{|\textit{dest}|}"|
\end{center}
%
Here \textit{target} is the name of the output file,
\textit{main} is the name of the main file
and \textit{dest} is the name of the main or child file to be processed
(all filenames without extensions).
The optional argument \textit{main} can be omitted
if \textit{main} matches \textit{dest}.
Optionally, compilation \textit{flags} can be defined via |\def| commands.
This command line makes the \TeX{} engine believe
it is compiling the file \textit{target}
whose content is specified as the latter parameter.
The provided code then forwards the processing to
\textit{main} or \textit{dest} as described in \secref{sec:forward}.

%%%%%%%%%%%%%%%%%%%%%%%%%%%%%%%%%%%%%%%%%%%%%%%%%%%%%%%%%%%%%%%%%%%%%%%%%%%%%%%%
\subsection{Include by Input}
\label{sec:input}

Including child documents by |\include| has some restrictions by design.
Most notably, the content of a child document always occupies
its own set of pages; pages cannot be shared between child documents.
Usually, this behaviour makes perfect sense
because each child document contain an essential part of the document.
However, in some situations it may be desirable to compose
a document from a collection of parts
without having mandatory page breaks between then.
For this case, the package
provides a mechanism to include parts
by |\input| which can also be processed individually.
However, by construction this mechanism
requires manual handling of the content to be output.

%%%%%%%%%%%%%%%%%%%%%%%%%%%%%%%%%%%%%%%%
\DescribeMacro{\ifchilddocmanual}
The main file should be prepared as usual, see \secref{sec:include}.
However, the document body must make a distinction
between processing of an individual part and of the main document, e.g.:
%
\begin{center}
\begin{tabular}{l}
|\ifchilddocmanual|\\
|\input{\childdocname}|\\
|\||else|\\
\textit{document body with }|\input{|\textit{part}|}|\\
|\||fi|
\end{tabular}
\end{center}
%
The conditional |\ifchilddocmanual| is true whenever
a part to be included by |\input| is being compiled,
and the name of the part is stored in |\childdocname|.

%%%%%%%%%%%%%%%%%%%%%%%%%%%%%%%%%%%%%%%%
\DescribeMacro{\childdocby}
Each part to be included by |\input| should start with:
%
\begin{center}
\begin{tabular}{l}
|\input{childdoc.def}|\\
|\childdocby{|\textit{main}|}|\\
\end{tabular}
\end{center}
%
The directive |\childdocby| is similar to |\childdocof|
described in \secref{sec:include},
but the subsequent selection of content must be done manually.
To that end, both |\ifchilddoc| and |\ifchilddocmanual|
will be true upon processing of a part,
and the name of the part is stored in |\childdocname|.
Note that |\jobname| will be set to the filename of the current part
so that each part receives an individual |.aux| file
that does not interfere with the |.aux| file(s) of the main document.
This behaviour can be altered by the alternative form
|\childdocby[*]{|\textit{main}|}| (with a non-empty optional argument)
which uses the |.aux| file of the main document
by setting |\jobname| to \textit{main}.

%%%%%%%%%%%%%%%%%%%%%%%%%%%%%%%%%%%%%%%%%%%%%%%%%%%%%%%%%%%%%%%%%%%%%%%%%%%%%%%%
\subsection{Driver Development}
\label{sec:driver}

The \textsf{childdoc} mechanism can also be use for the development
of definition files such as \LaTeX{} styles or classes.
This case differs from the above setup with multiple parts
included by |\include| in that no |\includeonly| should be invoked.
This can be achieved by starting the include file
(before |\ProvidesPackage|) with:
%
\begin{center}
\begin{tabular}{l}
|\input{childdoc.def}|\\
|\childdocforward{|\textit{main}|}|\\
\end{tabular}
\end{center}
%
or alternatively with:
%
\begin{center}
\begin{tabular}{l}
|\input{childdoc.def}|\\
|\childdocby{|\textit{main}|}|\\
\end{tabular}
\end{center}
%
Both forms have slightly different effects as described above.
The main file is prepared as usual, see \secref{sec:include}.

%%%%%%%%%%%%%%%%%%%%%%%%%%%%%%%%%%%%%%%%%%%%%%%%%%%%%%%%%%%%%%%%%%%%%%%%%%%%%%%%
\subsection{Legacy Detection}
\label{sec:detection}

The directive |\childdocmain| in the main file can detect
whether the complete document or merely a child is to be compiled
even without using the directive |\childdocof|.
This method is deprecated because it is less robust
and there is no compelling reason to use it;
it is merely provided for backward compatibility
and it may be removed in future versions.

If the detection mechanism is to be used,
it is mandatory to correctly specify
the filename of the main file as the argument of |\childdocmain|:
%
\begin{center}
\begin{tabular}{l}
|\input{childdoc.def}|\\
|\childdocmain{|\textit{main}|}|\\
\end{tabular}
\end{center}
%
If |\jobname| does not match the argument \textit{main} of |\childdocmain|,
it is assumed that |\jobname| points to the child file to be compiled.
When using |\childdocmain| with the main file specified as argument,
it suffices to start a child file
with just |\input{|\textit{main}|}|
without loading of the package and using |\childdocof|.
If instead all processing is done
with the appropriate \textsf{childdoc} directives,
the argument of \textit{main} of |\childdocmain| can be empty.

An alternative version of the command line processing described
in \secref{sec:commandline} using the detection mechanism reads:
%
\begin{center}
|... -jobname "|\textit{target}|" "|[\textit{flags}]%
[|\def\jobname{|\textit{dest}|}|]|\input{|\textit{main}|}"|
\end{center}

%%%%%%%%%%%%%%%%%%%%%%%%%%%%%%%%%%%%%%%%%%%%%%%%%%%%%%%%%%%%%%%%%%%%%%%%%%%%%%%%
\subsection{Manual Code}
\label{sec:manual}

In case one cannot be certain whether the definitions file |childdoc.def|
is installed on the target \TeX{} distribution
and one prefers not to ship it,
it is conceivable to paste a few relevant commands into the sources.

To that end, drop all statements |\input{childdoc.def}|
and perform the replacements as outlined below.
Instead of |\childdocmain{|\textit{main}|}| add the following code
to the top of the main file:
%
\begin{center}
\begin{tabular}{l}
|\||ifdefined\childdocname\endinput\||fi\newif\ifchilddoc|\\
|\edef\childdocname{\scantokens\expandafter{\jobname\noexpand}}|\\
|\def\childdocmain{|\textit{main}|}\||ifx\childdocmain\childdocname\||else|\\
|\childdoctrue\includeonly{\childdocname}\let\jobname\childdocmain\||fi|\\
\end{tabular}
\end{center}
%
Instead of |\childdocof{|\textit{main}|}| just include the main file
at the top of each child file:
%
\begin{center}
|\input{|\textit{main}|}|
\end{center}
%
A simple redirection |\childdocforward{|\textit{dest}|}| is achieved by:
%
\begin{center}
|\def\jobname{|\textit{dest}|}\input{\jobname}|
\end{center}
%
The redirection with prefix
|\childdocforwardprefix[|\textit{prefix}|]{|\textit{dest}|}|
is accomplished by:
%
\begin{center}
\begin{tabular}{l}
|{\edef\jobname{\scantokens\expandafter{\jobname\noexpand}}|\\
|\def\redirectjob |\textit{prefix}|#1~~~{\gdef\jobname{|\textit{dest}|#1}}|\\
|\expandafter\redirectjob\jobname~~~}\input{\jobname}|
\end{tabular}
\end{center}

In an alternative approach,
child documents can be compiled by a specific command line
without additional code or specific definitions:
%
\begin{center}
|... -jobname "|\textit{target}|" "|[\textit{flags}]%
|\includeonly{|\textit{dest}|}\input{|\textit{main}|}"|
\end{center}
%

%%%%%%%%%%%%%%%%%%%%%%%%%%%%%%%%%%%%%%%%%%%%%%%%%%%%%%%%%%%%%%%%%%%%%%%%%%%%%%%%
%%%%%%%%%%%%%%%%%%%%%%%%%%%%%%%%%%%%%%%%%%%%%%%%%%%%%%%%%%%%%%%%%%%%%%%%%%%%%%%%
\section{Information}

%%%%%%%%%%%%%%%%%%%%%%%%%%%%%%%%%%%%%%%%%%%%%%%%%%%%%%%%%%%%%%%%%%%%%%%%%%%%%%%%
\subsection{Copyright}

Copyright \copyright{} 2017--2018 Niklas Beisert

This work may be distributed and/or modified under the
conditions of the \LaTeX{} Project Public License, either version 1.3
of this license or (at your option) any later version.
The latest version of this license is in
  \url{http://www.latex-project.org/lppl.txt}
and version 1.3 or later is part of all distributions of \LaTeX{}
version 2005/12/01 or later.

This work has the LPPL maintenance status `maintained'.

The Current Maintainer of this work is Niklas Beisert.

This work consists of the files |README.txt|, |childdoc.ins| and |childdoc.dtx|
as well as the derived files |childdoc.def|, |cdocsamp.tex|
with |cdocsch1.tex|, |cdocsch2.tex|, |cdocspt3.tex|, |cdocspt4.tex|,
|cdocsdrf.tex|, |cdocsfn1.tex|, |cdocsfn2.tex|
as well as |childdoc.pdf|.

%%%%%%%%%%%%%%%%%%%%%%%%%%%%%%%%%%%%%%%%%%%%%%%%%%%%%%%%%%%%%%%%%%%%%%%%%%%%%%%%
\subsection{Files and Installation}

The package consists of the files:
%
\begin{center}
\begin{tabular}{ll}
    |README.txt|   & readme file \\
    |childdoc.ins| & installation file \\
    |childdoc.dtx| & source file \\
    |childdoc.def| & definition file \\
    |cdocsamp.tex| & sample main file \\
    |cdocsch1.tex| & sample include file \\
    |cdocsch2.tex| & sample include file \\
    |cdocspt3.tex| & sample part file \\
    |cdocspt4.tex| & sample part file \\
    |cdocsdrf.tex| & sample redirection file \\
    |cdocsfn1.tex| & sample redirection file \\
    |cdocsfn2.tex| & sample redirection file \\
    |childdoc.pdf| & manual
\end{tabular}
\end{center}
%
The distribution consists of the files
|README.txt|, |childdoc.ins| and |childdoc.dtx|.
%
\begin{itemize}
\item
Run (pdf)\LaTeX{} on |childdoc.dtx|
to compile the manual |childdoc.pdf| (this file).
\item
Run \LaTeX{} on |childdoc.ins| to create the definitions file |childdoc.def|
and the sample |cdocsamp.tex| with include files
|cdocsch1.tex|, |cdocsch2.tex|, |cdocspt3.tex|, |cdocspt4.tex|,
|cdocsdrf.tex|, |cdocsfn1.tex|, |cdocsfn2.tex|.
Then copy the file |childdoc.def| to an appropriate directory of your \LaTeX{}
distribution, e.g.\ \textit{texmf-root}|/tex/latex/childdoc|.
\end{itemize}

%%%%%%%%%%%%%%%%%%%%%%%%%%%%%%%%%%%%%%%%%%%%%%%%%%%%%%%%%%%%%%%%%%%%%%%%%%%%%%%%
\subsection{Related CTAN Packages}

There are several other packages which offer a similar functionality:
%
\begin{itemize}
\item
The packages
\href{http://ctan.org/pkg/docmute}{\textsf{docmute}},
\href{http://ctan.org/pkg/includex}{\textsf{includex}} and
\href{http://ctan.org/pkg/standalone}{\textsf{standalone}}
provide commands to include only the document body of
a child file thus allowing both files to be compiled individually.
\item
The packages \href{http://ctan.org/pkg/subdocs}{\textsf{subdocs}}
and \href{http://ctan.org/pkg/subfiles}{\textsf{subfiles}}
provide structures in which the main and child documents can be
encapsulated and allowing them to be compiled individually.
The inclusion mechanism is different from the conventional |\include|.
\item
The package \href{http://ctan.org/pkg/combine}{\textsf{combine}}
is an elaborate solution to combine several documents into one.
\end{itemize}
%
See also the CTAN topic \href{http://ctan.org/topic/subdocs}{\textsf{subdocs}}
for further related packages.
The present package differs from the above solutions in that
a document structure constructed with the conventional |\include| mechanism
just needs two extra commands at the top of every file
such that all constituent files can be compiled individually.

%%%%%%%%%%%%%%%%%%%%%%%%%%%%%%%%%%%%%%%%%%%%%%%%%%%%%%%%%%%%%%%%%%%%%%%%%%%%%%%%
%\subsection{Feature Suggestions}
%
%The following is a list of features which may be useful for future
%versions of this package:
%%
%\begin{itemize}
%\item
%\ldots
%\end{itemize}

%%%%%%%%%%%%%%%%%%%%%%%%%%%%%%%%%%%%%%%%%%%%%%%%%%%%%%%%%%%%%%%%%%%%%%%%%%%%%%%%
\subsection{Revision History}

%%%%%%%%%%%%%%%%%%%%%%%%%%%%%%%%%%%%%%%%
\paragraph{v2.0:} 2018/12/30

\begin{itemize}
\item
immediate forward processing
\item
added |\childdocby| mechanism
\item
manual restructured
\end{itemize}

%%%%%%%%%%%%%%%%%%%%%%%%%%%%%%%%%%%%%%%%
\paragraph{v1.6:} 2018/01/17

\begin{itemize}
\item
application for development of include files
\item
corrections to manual
\end{itemize}

%%%%%%%%%%%%%%%%%%%%%%%%%%%%%%%%%%%%%%%%
\paragraph{v1.5:} 2017/05/21

\begin{itemize}
\item
more complete structuring introduced
\item
|\childdocof| introduced
\item
|\childdoc| renamed to |\childdocmain|
\item
|\childredirect| renamed to |\childdocforward| and |\childdocforwardprefix|
and functionality expanded
\end{itemize}

%%%%%%%%%%%%%%%%%%%%%%%%%%%%%%%%%%%%%%%%
\paragraph{v1.0:} 2017/04/27

\begin{itemize}
\item
manual and install package
\item
first version published on CTAN
\end{itemize}

%%%%%%%%%%%%%%%%%%%%%%%%%%%%%%%%%%%%%%%%
\paragraph{v0.6:} 2017/04/26

\begin{itemize}
\item
redirection mechanism added
\end{itemize}

%%%%%%%%%%%%%%%%%%%%%%%%%%%%%%%%%%%%%%%%
\paragraph{v0.5:} 2017/04/26

\begin{itemize}
\item
functionality in definition file
\end{itemize}


%%%%%%%%%%%%%%%%%%%%%%%%%%%%%%%%%%%%%%%%%%%%%%%%%%%%%%%%%%%%%%%%%%%%%%%%%%%%%%%%
%%%%%%%%%%%%%%%%%%%%%%%%%%%%%%%%%%%%%%%%%%%%%%%%%%%%%%%%%%%%%%%%%%%%%%%%%%%%%%%%
%%%%%%%%%%%%%%%%%%%%%%%%%%%%%%%%%%%%%%%%%%%%%%%%%%%%%%%%%%%%%%%%%%%%%%%%%%%%%%%%
\appendix

\settowidth\MacroIndent{\rmfamily\scriptsize 000\ }

 \DocInput{childdoc.dtx}

\end{document}
%</driver>
% \fi
%
% %%%%%%%%%%%%%%%%%%%%%%%%%%%%%%%%%%%%%%%%%%%%%%%%%%%%%%%%%%%%%%%%%%%%%%%%%%%%%%
% %%%%%%%%%%%%%%%%%%%%%%%%%%%%%%%%%%%%%%%%%%%%%%%%%%%%%%%%%%%%%%%%%%%%%%%%%%%%%%
% \section{Sample}
%\iffalse
%<*samplemain>
%\fi
%
% The following presents a sample document
% with two chapters, two parts, a title page,
% a compile flag as well as three forwarding files to set the flag.
% It consists of eight |.tex| files:
% \begin{center}
% \begin{tabular}{ll}
% |cdocsamp.tex|&main file\\
% |cdocsch1.tex|&include file for chapter 1\\
% |cdocsch2.tex|&include file for chapter 2\\
% |cdocspt3.tex|&include file for part 3\\
% |cdocspt4.tex|&include file for part 4\\
% |cdocsdrf.tex|&forwarding file for main file in draft mode\\
% |cdocsfi1.tex|&forwarding file for final version of chapter 1\\
% |cdocsfi2.tex|&forwarding file for final version of chapter 2\\
% \end{tabular}
% \end{center}
% Each of the eight files can be compiled directly by the \LaTeX{} compiler.
%
% %%%%%%%%%%%%%%%%%%%%%%%%%%%%%%%%%%%%%%
% \paragraph{Main File.}
%
% The main file is called |cdocsamp.tex|.
%
% Load the \textsf{childdoc} definitions and
% declare the filename for the main document:
%    \begin{macrocode}
\input{childdoc.def}
\childdocmain{}
%    \end{macrocode}

% Optional override for |\version| flag:
%    \begin{macrocode}
%%\ifchilddoc\else\providecommand{\version}{draft}\fi
%    \end{macrocode}

% Define the default values for the |\version| flag
% (|final| for the main file and |draft| for childs):
%    \begin{macrocode}
\ifchilddoc
\providecommand{\version}{draft}
\else
\providecommand{\version}{final}
\fi
%    \end{macrocode}

% Load the standard document class:
%    \begin{macrocode}
\documentclass[12pt]{article}
%    \end{macrocode}

% Start the document body:
%    \begin{macrocode}
\begin{document}
%    \end{macrocode}

% Declare a title page.
% Print title, part of document being processed and version flag:
%    \begin{macrocode}
\addtocounter{page}{-1}
\begin{center}
{\LARGE\bfseries{}childdoc example\par}
\vspace{1cm}
\ifchilddoc
\ifchilddocmanual part\else chapter\fi:
`\childdocname' of `\childdocjob'\par
\else
main document: `\childdocjob'\par
\fi
version: \version\par
\end{center}
\newpage
%    \end{macrocode}

% Manually include selected file,
% otherwise process as usual:
%    \begin{macrocode}
\ifchilddocmanual
\section*{part `\childdocname'}
\input{\childdocname}
\else
%    \end{macrocode}

% Include the two chapters:
%    \begin{macrocode}
\include{cdocsch1}
\include{cdocsch2}
%    \end{macrocode}

% Include the two parts unless only chapters should be displayed:
%    \begin{macrocode}
\ifchilddoc\else
\section{part three}
\input{cdocspt3}
\section{part four}
\input{cdocspt4}
\fi
%    \end{macrocode}

% Process as usual until here:
%    \begin{macrocode}
\fi
%    \end{macrocode}

% End of document body:
%    \begin{macrocode}
\end{document}
%    \end{macrocode}
%\iffalse
%</samplemain>
%\fi
%
% %%%%%%%%%%%%%%%%%%%%%%%%%%%%%%%%%%%%%%
% \paragraph{Chapter Include Files.}
%
% The include files are called |cdocsch1.tex| and |cdocsch2.tex|.
%
%\iffalse
%<*samplechap1|samplechap2>
%\fi

% Optional override for |\version| flag:
%    \begin{macrocode}
%%\providecommand{\version}{final}
%    \end{macrocode}

% Include the main document:
%    \begin{macrocode}
\input{childdoc.def}
\childdocof{cdocsamp}
%    \end{macrocode}

%\iffalse
%</samplechap1|samplechap2>
%\fi
%
%\iffalse
%<*samplechap1>
%\fi
% Some text for chapter 1:
%    \begin{macrocode}
\section{one}
some text in chapter one
%    \end{macrocode}

%\iffalse
%</samplechap1>
%\fi
% Some text for chapter 2:
%\iffalse
%<*samplechap2>
%\fi
%    \begin{macrocode}
\section{two}
more text in chapter two
%    \end{macrocode}

%\iffalse
%</samplechap2>
%\fi
%
% %%%%%%%%%%%%%%%%%%%%%%%%%%%%%%%%%%%%%%
% \paragraph{Part Include Files.}
%
% The include files are called |cdocspt3.tex| and |cdocspt4.tex|.
%
%\iffalse
%<*samplepart3|samplepart4>
%\fi

% Optional override for |\version| flag:
%    \begin{macrocode}
%%\providecommand{\version}{final}
%    \end{macrocode}

% Include the main document:
%    \begin{macrocode}
\input{childdoc.def}
\childdocby{cdocsamp}
%    \end{macrocode}

%\iffalse
%</samplepart3|samplepart4>
%\fi
%
%\iffalse
%<*samplepart3>
%\fi
% Some text for part 3:
%    \begin{macrocode}
some text in part three
%    \end{macrocode}

%\iffalse
%</samplepart3>
%\fi
% Some text for part 4:
%\iffalse
%<*samplepart4>
%\fi
%    \begin{macrocode}
more text in part four
%    \end{macrocode}

%\iffalse
%</samplepart4>
%\fi
%
% %%%%%%%%%%%%%%%%%%%%%%%%%%%%%%%%%%%%%%
% \paragraph{Forwarding for a Complete Draft.}
%
% The following forwarding file |cdocsdrf.tex|
% compiles the main document in draft mode:
%\iffalse
%<*sampledraft>
%\fi
%    \begin{macrocode}
\def\version{draft}
\input{childdoc.def}
\childdocforward{cdocsamp}
%    \end{macrocode}

%\iffalse
%</sampledraft>
%\fi
%
% %%%%%%%%%%%%%%%%%%%%%%%%%%%%%%%%%%%%%%
% \paragraph{Forwarding for Final Version of the Chapters.}
%
% The following forwarding files |cdocsfn1.tex| and |cdocsfn2.tex|
% (with identical content)
% compile the final versions of the child documents
% |cdocsch1.tex| and |cdocsch2.tex|, respectively:
%\iffalse
%<*samplefinal>
%\fi
%    \begin{macrocode}
\def\version{final}
\input{childdoc.def}
\childdocforwardprefix[cdocsamp]{cdocsfn}{cdocsch}
%    \end{macrocode}

%\iffalse
%</samplefinal>
%\fi
%
% %%%%%%%%%%%%%%%%%%%%%%%%%%%%%%%%%%%%%%
% \paragraph{Command Line Processing.}
%
% The following three command lines generate the output files
% |cdocscld|, |cdocscl1| and |cdocscl2|
% which should be identical to
% |cdocsdrf|, |cdocsch1| and |cdocsfn2|, respectively:
% \begin{center}
% \begin{tabular}{l}
% |latex -jobname cdocscld \|\\
% |  "\def\version{draft}\input{childdoc.def}\childdocforward{cdocsamp}"|\\
% |latex -jobname cdocscl1 \|\\
% |  "\input{childdoc.def}\childdocforward[cdocsamp]{cdocsch1}"|\\
% |latex -jobname cdocscl2 \|\\
% |  "\def\version{final}\input{childdoc.def}\childdocforward{cdocsch2}"|
% \end{tabular}
% \end{center}
% Note that the trailing backslash on each first line
% merely continues the input to the second line
% (for convenient cut ant paste).
% Furthermore, the command |latex| can be replaced by any
% of its alternative versions such as |pdflatex|.
%
% %%%%%%%%%%%%%%%%%%%%%%%%%%%%%%%%%%%%%%%%%%%%%%%%%%%%%%%%%%%%%%%%%%%%%%%%%%%%%%
% %%%%%%%%%%%%%%%%%%%%%%%%%%%%%%%%%%%%%%%%%%%%%%%%%%%%%%%%%%%%%%%%%%%%%%%%%%%%%%
% \section{Implementation}
%\iffalse
%<*package>
%\fi
%
% This section describes the definitions file |childdoc.def|.

% The definitions cannot be loaded using |\usepackage| or |\RequirePackage|
% which has a mechanism to prevent loading a style file more than once.
% When loading the definitions by means of |\input|
% multiple instances have to be prevented manually:
%\iffalse
%This code needs to be before the `\ProvidesFile' directive
%which is defined at the beginning of this file.
%Therefore it is also placed there and commented out here.
%</package>
%<*discard>
%\fi
%    \begin{macrocode}
\ifdefined\childdocmain\endinput\fi
%    \end{macrocode}
%\iffalse
%</discard>
%<*package>
%\fi
%
% \macro{\ifchilddoc}
% \macro{\ifchilddocmanual}
% The conditional |\ifchilddoc| tells whether a
% child (true) or main (false) document is being compiled.
% The conditional |\ifchilddocmanual| tells whether
% the |\includeonly| mechanism is used (false) or
% the selection of child files must be performed manually (true).
% The definitions initialise to false:
%    \begin{macrocode}
\newif\ifchilddoc
\newif\ifchilddocmanual
%    \end{macrocode}

% \macro{\childdocname}
% \macro{\childdocjob}
% The macro |\childdocname| stores the name of the main document
% to be compiled. The macro |\childdocjob| stores the name of
% the document on which the \LaTeX{} compiler was originally invoked.
% The content of |\jobname| cannot be compared
% to filenames specified in the source due to different catcodes.
% The following code rescans |\jobname|, stores the result
% in |\childdocname| and saves a copy in |\childdocjob|:
%    \begin{macrocode}
\edef\childdocname{\scantokens\expandafter{\jobname\noexpand}}
\let\childdocjob\childdocname
%    \end{macrocode}

% \macro{\childdocdisable}
% The macro |\childdocdisable| prevents the main file
% from being processed more than once.
% At this stage, the main document command |\childdocmain|
% is assumed to be called once again where it should do nothing.
% Any subsequent call to it should prevent
% a secondary processing of the main document
% It overwrites the forwarding commands
% |\childdocof| and |\childdocforward|
% with empty macros to prevent further inclusions of the main document:
%    \begin{macrocode}
\newcommand{\childdocdisable}
{
  \renewcommand{\childdocmain}[1]{\renewcommand{\childdocmain}[1]{\endinput}}
  \renewcommand{\childdocof}[1]{}
  \renewcommand{\childdocby}[2][]{}
  \renewcommand{\childdocforward}[2][]{}
  \renewcommand{\childdocdisable}{}
}
%    \end{macrocode}

% \macro{\childdocmain}
% The macro |\childdocmain| is to be called at the top of the main file
% with nothing or the main filename (without extension) as argument.
% First, it breaks loops.
% If the argument is not empty and does not match |\childdocname|
% (which is set by the first inclusion of |childdoc.def|),
% |\ifchilddoc| is set to true, |\includeonly| is applied to the child file
% and |\jobname| is set to the main file
% (for proper handling of |.aux| files):
%    \begin{macrocode}
\newcommand{\childdocmain}[1]
{
  \childdocdisable\childdocmain{}
  \if?#1?\else
    \begingroup
      \def\childdoctmp{#1}
      \ifx\childdoctmp\childdocname
        \def\childdoctmp{}
      \else
        \def\childdoctmp
        {
          \childdoctrue
          \includeonly{\childdocname}
          \def\childdocjob{#1}
          \def\jobname{#1}
        }
      \fi
      \expandafter
    \endgroup
    \childdoctmp
  \fi
}
%    \end{macrocode}

% \macro{\childdocof}
% The command |\childdocof| redirects
% compilation to the main file |#1|.
%    \begin{macrocode}
\newcommand{\childdocof}[1]
{
  \childdocdisable
  \childdoctrue
  \includeonly{\childdocname}
  \def\jobname{#1}
  \def\childdocjob{#1}
  \input{#1}
}
%    \end{macrocode}

% \macro{\childdocby}
% The command |\childdocby| ....
%    \begin{macrocode}
\newcommand{\childdocby}[2][]
{
  \childdocdisable
  \childdoctrue
  \childdocmanualtrue
  \if?#1?\else
    \def\jobname{#2}
  \fi
  \def\childdocjob{#2}
  \input{#2}
  \endinput
}
%    \end{macrocode}

% \macro{\childdocforward}
% The command |\childdocforward| redirects
% compilation to the main file or
% (if the optional argument is given) a child file.
% Parameters are set as if the main file
% or a child file starting with |\childdocof| was compiled.
% Then compilation is handed over to the main file:
%    \begin{macrocode}
\newcommand{\childdocforward}[2][]
{
  \begingroup
    \if?#1?
      \def\childdoctmp
      {
        \def\childdocname{#2}
        \def\childdocjob{#2}
        \def\jobname{#2}
        \input{#2}
        \endinput
      }
    \else
      \def\childdoctmp
      {
        \childdocdisable
        \def\childdocname{#2}
        \childdoctrue
        \includeonly{#2}
        \def\childdocjob{#1}
        \def\jobname{#1}
        \input{#1}
        \endinput
      }
    \fi
    \expandafter
  \endgroup
  \childdoctmp
}
%    \end{macrocode}

% \macro{\childdocforwardprefix}
% The command |\childdocforwardprefix| redirects
% compilation to the main or a child file by means of a pattern.
% The prefix |#1| in the current filename is replaced by |#2|
% and the suffix of the current filename is kept
% (it is assumed that the filename does not contain the substring `|~~~|'
% which is used as a delimiter).
% Compilation is handed over to the new file by |\childdocforward|:
%    \begin{macrocode}
\newcommand{\childdocforwardprefix}[3][]
{
  \begingroup
    \def\childdocextract #2##1~~~{\def\childdoctmp{\childdocforward[#1]{#3##1}}}
    \expandafter\childdocextract\childdocname~~~
    \expandafter
  \endgroup
  \childdoctmp
}
%    \end{macrocode}

% \macro{\childdoc}
% The deprecated macro |\childdoc| is a legacy version of |\childdocmain|:
%    \begin{macrocode}
\newcommand{\childdoc}{\childdocmain}
%    \end{macrocode}

% \macro{\childdocredirect}
% The deprecated macro |\childdocredirect| is a legacy version
% of |\childdocforward| and |\childdocforwardprefix|:
%    \begin{macrocode}
\newcommand{\childdocredirect}[2][]
{
  \begingroup
    \if?#1?
      \def\childdoctmp{\childdocforward{#2}}
    \else
      \def\childdoctmp{\childdocforwardprefix{#1}{#2}}
    \fi
    \expandafter
  \endgroup
  \childdoctmp
}
%    \end{macrocode}

%\iffalse
%</package>
%\fi
%
\endinput
|\\
|\childdocforward{|\textit{main}|}|
\end{tabular}
\end{center}
%
Likewise, the following files |final|\textit{nn}|.tex|
compile the final version of the child document
|child|\textit{nn}|.tex|:
%
\begin{center}
\begin{tabular}{l}
|\def\version{final}|\\
|% \iffalse
%
% childdoc.dtx Copyright (C) 2017-2018 Niklas Beisert
%
% This work may be distributed and/or modified under the
% conditions of the LaTeX Project Public License, either version 1.3
% of this license or (at your option) any later version.
% The latest version of this license is in
%   http://www.latex-project.org/lppl.txt
% and version 1.3 or later is part of all distributions of LaTeX
% version 2005/12/01 or later.
%
% This work has the LPPL maintenance status `maintained'.
%
% The Current Maintainer of this work is Niklas Beisert.
%
% This work consists of the files childdoc.dtx and childdoc.ins
% and the derived files childdoc.def and cdocsamp.tex with
% cdocsch1.tex, cdocsch2.tex, cdocsdrf.tex, cdocsfn1.tex, cdocsfn2.tex.
%
%<package>\ifdefined\childdocmain\endinput\fi
%<package>\ProvidesFile{childdoc.def}[2018/12/30 v2.0 child document driver]
%<samplemain>\ProvidesFile{cdocsamp.tex}[2018/12/30 v2.0 sample for childdoc]
%<*driver>
%\ProvidesFile{childdoc.drv}[2018/12/30 v2.0 childdoc reference manual file]
\PassOptionsToClass{10pt,a4paper}{article}
\documentclass{ltxdoc}

\usepackage[margin=35mm]{geometry}
\usepackage{hyperref}
\usepackage{hyperxmp}
\usepackage[usenames]{color}

\hypersetup{colorlinks=true}
\hypersetup{pdfstartview=FitH}
\hypersetup{pdfpagemode=UseNone}
\hypersetup{pdfsource={}}
\hypersetup{pdflang={en-UK}}
\hypersetup{pdfcopyright={Copyright 2017-2018 Niklas Beisert.
  This work may be distributed and/or modified under the
  conditions of the LaTeX Project Public License, either version 1.3
  of this license or (at your option) any later version.}}
\hypersetup{pdflicenseurl={http://www.latex-project.org/lppl.txt}}
\hypersetup{pdfcontactaddress={ETH Zurich, ITP, HIT K,
  Wolfgang-Pauli-Strasse 27}}
\hypersetup{pdfcontactpostcode={8093}}
\hypersetup{pdfcontactcity={Zurich}}
\hypersetup{pdfcontactcountry={Switzerland}}
\hypersetup{pdfcontactemail={nbeisert@itp.phys.ethz.ch}}
\hypersetup{pdfcontacturl={http://people.phys.ethz.ch/\xmptilde nbeisert/}}

\newcommand{\secref}[1]{\hyperref[#1]{section \ref*{#1}}}

\parskip1ex
\parindent0pt
\let\olditemize\itemize
\def\itemize{\olditemize\parskip0pt}

\begin{document}

\title{The \textsf{childdoc} Package}
\hypersetup{pdftitle={The childdoc Package}}
\author{Niklas Beisert\\[2ex]
  Institut f\"ur Theoretische Physik\\
  Eidgen\"ossische Technische Hochschule Z\"urich\\
  Wolfgang-Pauli-Strasse 27, 8093 Z\"urich, Switzerland\\[1ex]
  \href{mailto:nbeisert@itp.phys.ethz.ch}
  {\texttt{nbeisert@itp.phys.ethz.ch}}}
\hypersetup{pdfauthor={Niklas Beisert}}
\hypersetup{pdfsubject={Manual for the LaTeX2e Package childdoc}}
\date{30 December 2018, \textsf{v2.0}}
\maketitle

\begin{abstract}\noindent
\textsf{childdoc} is a \LaTeXe{} package
that enables the direct compilation
of document sections included by |\include|
to individual files.
\end{abstract}

\begingroup
\parskip0ex
\tableofcontents
\endgroup

%%%%%%%%%%%%%%%%%%%%%%%%%%%%%%%%%%%%%%%%%%%%%%%%%%%%%%%%%%%%%%%%%%%%%%%%%%%%%%%%
%%%%%%%%%%%%%%%%%%%%%%%%%%%%%%%%%%%%%%%%%%%%%%%%%%%%%%%%%%%%%%%%%%%%%%%%%%%%%%%%
\section{Introduction}

\LaTeX{} provides a mechanism to structure a large document (such as a book)
into a main file and several child files (containing the chapters)
using the |\include| command.
This mechanism is beneficial for documents
which span hundreds of pages in order to
make the source file(s) more manageable.
Moreover, compilation can be restricted to
selected child files by means of the |\includeonly| command.
The latter feature can be used to reduce the compilation time while editing
(this was significantly more useful in the earlier days of \LaTeX{})
or to generate a smaller document which is easier to navigate.
Another application of |\includeonly| is to generate
documents consisting of selected parts of the complete document.

However, there are a few drawbacks of the plain |\include| mechanism:
\begin{itemize}
\item
The child files cannot be compiled on their own,
they can only be compiled via the main file.
A naive editing environment
(such as a text editor with an option
to have the current file processed by \LaTeX)
may require one to switch to the main file before compiling;
attempting to compile the child file produces errors.
\item
The main file must be modified (each time)
to adjust the |\includeonly| command
to the present needs. This easily leaves the main file in a messy state.
\item
The generated document will always carry the filename
of the main document. This is inconvenient if
several child files are to be compiled and
to be kept for distribution.
\end{itemize}

The present package provides a simple interface
to make child files individually compilable by \LaTeX{}.
Compiling a child file then has the same effect as compiling
the main file with an |\includeonly| command
to select the appropriate child.
Moreover the generated document will carry the name of the child
rather than the main file.
This resolves all three above issues.

This feature is meant to make the editing of books,
thesis documents and lecture notes somewhat more convenient.
However, the package can also be used efficiently for
composing a series of documents (such as exercise sheets)
which are typically distributed individually.
It then assists the author in generating the individual documents
(potentially in different versions)
as well as a document containing the collected series.
Another application is in developing style files
or other kinds of included material
where compilation of the style file could redirect
to a sample or test file.

%%%%%%%%%%%%%%%%%%%%%%%%%%%%%%%%%%%%%%%%%%%%%%%%%%%%%%%%%%%%%%%%%%%%%%%%%%%%%%%%
%%%%%%%%%%%%%%%%%%%%%%%%%%%%%%%%%%%%%%%%%%%%%%%%%%%%%%%%%%%%%%%%%%%%%%%%%%%%%%%%
\section{Usage}

First of all, the package \textsf{childdoc} is \emph{not} a standard
\LaTeXe{} |.sty| style file! Therefore it needs to be invoked in
a non-standard way.

%%%%%%%%%%%%%%%%%%%%%%%%%%%%%%%%%%%%%%%%%%%%%%%%%%%%%%%%%%%%%%%%%%%%%%%%%%%%%%%%
\subsection{Included Files}
\label{sec:include}

%%%%%%%%%%%%%%%%%%%%%%%%%%%%%%%%%%%%%%%%
\DescribeMacro{\childdocmain}
To use the package, add the commands
\begin{center}
\begin{tabular}{l}
|\input{childdoc.def}|\\
|\childdocmain{}|\\
\end{tabular}
\end{center}
at the very top of the main \LaTeX{} file,
in particular \emph{before} the |\documentclass| statement!
The argument of |\childdocmain| should be left empty
(but it must be present).

%%%%%%%%%%%%%%%%%%%%%%%%%%%%%%%%%%%%%%%%
\DescribeMacro{\childdocof}
Furthermore, add the commands
\begin{center}
\begin{tabular}{l}
|\input{childdoc.def}|\\
|\childdocof{|\textit{main}|}|\\
\end{tabular}
\end{center}
at the top of every child file \textit{child}
which is included by |\include{|\textit{child}|}|
from within the main file
(or at least for those files to be compiled individually).
The argument \textit{main} must be the filename of the main file.

There are a couple of
considerations in setting up the main and child documents:

%%%%%%%%%%%%%%%%%%%%%%%%%%%%%%%%%%%%%%%%
\paragraph{Restrictions.}

Please note the following restrictions:
\begin{itemize}
\item
|\childdocmain| must be called with one argument \textit{main}
to ensure compatibility with earlier version of the package.
It must either be empty (|\childdocmain{}|)
or precisely match the filename of the main file in which it is specified.
See \secref{sec:detection} for further information.
\item
The filename \textit{main} must be specified without the |.tex| extension.
\item
The filename \textit{main} is case sensitive
(even in case-insensitive file systems)
due to internal string comparison.
\item
The argument \textit{main} should be fully expanded, it cannot be a macro.
\item
Subdirectories and special characters should be avoided in filenames.
\item
The command |\childdocmain{|\textit{main}|}| must be followed by a whitespace.
It should not be followed immediately by another command
or by a comment mark `|%|'.
This is because the \TeX{} parser reads the token immediately following
the argument of |\childdocmain| and puts it
at the beginning of every child section;
however, a white\-space is ignored.
\end{itemize}

%%%%%%%%%%%%%%%%%%%%%%%%%%%%%%%%%%%%%%%%
\paragraph{Content of Main File.}

It is advisable to place all content in the child files included by |\include|.
Any output contained in the main file will appear in all child documents
unless suppressed manually;
it cannot be suppressed automatically by the |\includeonly| directive
and thus should normally be avoided.
A method to include some content in the main file
by means of conditional processing is described in \secref{sec:conditional}.

%%%%%%%%%%%%%%%%%%%%%%%%%%%%%%%%%%%%%%%%
\paragraph{Page Numbering.}

When only a part of the document is compiled,
the appropriate numbering of pages
(as well as other status parameters)
is determined from the |.aux| files.
The latter contain information from previous passes.
However this information needs to propagate through
all intermediate child documents.
Therefore the page numbering in child documents may well
be inconsistent until the complete document is compiled at least once.

A useful (if unconventional) way to always ensure a consistent
page numbering is to restart the numbering in each child document
and denote the pages by `\textit{child}|.|\textit{page}'
where \textit{child} represents the chapter/section number of the child file.
This can be achieved by the command
|\numberwithin{page}{|\textit{child}|}|
of the \textsf{amsmath} package
where \textit{child} can be |chapter| or |section|
depending on the chosen structuring.
Alternatively, one can modify the macro |\thepage| appropriately
and reset the counter |page| at the start of each child file.

%%%%%%%%%%%%%%%%%%%%%%%%%%%%%%%%%%%%%%%%%%%%%%%%%%%%%%%%%%%%%%%%%%%%%%%%%%%%%%%%
\subsection{Conditional Processing}
\label{sec:conditional}

The package provides a mechanism to compile different versions
of a document. To customise the versions further some conditional processing
can come in handy to distinguish which version is being compiled.
The package provides two macros to describe the compilation context:

%%%%%%%%%%%%%%%%%%%%%%%%%%%%%%%%%%%%%%%%
\DescribeMacro{\ifchilddoc}
The conditional |\ifchilddoc| distinguishes between the compilation of
child documents and the main document:
%
\begin{center}
|\ifchilddoc |\textit{child-code}| |[|\||else |\textit{main-code}]| \||fi|
\end{center}

%%%%%%%%%%%%%%%%%%%%%%%%%%%%%%%%%%%%%%%%
\DescribeMacro{\childdocname}
\DescribeMacro{\childdocjob}
The macro |\childdocname| contains the filename (without extension)
of the main or child file being processed.
Note that |\childdocjob| will always contain the name of the main file.

%%%%%%%%%%%%%%%%%%%%%%%%%%%%%%%%%%%%%%%%
\paragraph{Title Page.}

Conditional processing can be used to include a title or banner page
in the main document when proper precautions are taken.
Importantly, the code in the main file should ensure that the page counter
(as well as other status parameters which are stored in the |.aux| files)
takes the same value after the conditional processing.
Otherwise the page numbers may take divergent values
depending on which part is compiled.

For example, a title page could be declared by:
%
\begin{center}
\begin{tabular}{l}
|\ifchilddoc\||else|\\
|\addtocounter{page}{-1}|\\
\textit{code for title page}\\
|\newpage|\\
|\||fi|
\end{tabular}
\end{center}
%
A banner page for the child documents can be generated by:
%
\begin{center}
\begin{tabular}{l}
|\ifchilddoc|\\
|\addtocounter{page}{-1}|\\
\textit{code for banner page}\\
|\newpage|\\
|\||fi|
\end{tabular}
\end{center}
%
Here one could write a message such as:
\begin{center}
|This is the part \childdocname{} of \childdocjob{}.|
\end{center}

%%%%%%%%%%%%%%%%%%%%%%%%%%%%%%%%%%%%%%%%%%%%%%%%%%%%%%%%%%%%%%%%%%%%%%%%%%%%%%%%
\subsection{Flags}
\label{sec:flags}

The package makes it easy to generate different versions
of the main or child documents.
To this end compilation flags can be defined
and assigned different default values.
They will be particularly useful in conjunction
with the forwarding mechanism described in \secref{sec:forward}.

For example, it may be useful to have a flag |\version|
which can be set to |draft| or |final|.
The document source will contain some conditional code
depending on the value of |\version|.
Suppose further, the flag should default to |final| for the main file
and to |draft| for child files
which is a natural assignment for editing the document.
This is achieved by placing the following code
in the preamble of the main document
(below the |\childdocmain| directive):
%
\begin{center}
\begin{tabular}{l}
|\ifchilddoc|\\
|\providecommand{\version}{draft}|\\
|\||else|\\
|\providecommand{\version}{final}|\\
|\||fi|
\end{tabular}
\end{center}
%
The definition by |\providecommand| makes sure
that previous definitions are not overwritten.
Further statements |\providecommand{\version}{...}|
can thus be added before the above code to override it.

For the main file, one might add a line
(between |\childdocmain| and the above block)
%
\begin{center}
|%\ifchilddoc\||else\providecommand{\version}{draft}\||fi|
\end{center}
%
which can be uncommented to produce a draft version.
Likewise one can add a line to the very top of a child file
(above the |\childdocof{|\textit{main}|}| directive)
%
\begin{center}
|%\providecommand{\version}{final}|
\end{center}
%
which can be uncommented to produce the final version of this child document.

%%%%%%%%%%%%%%%%%%%%%%%%%%%%%%%%%%%%%%%%%%%%%%%%%%%%%%%%%%%%%%%%%%%%%%%%%%%%%%%%
\subsection{Forwarding}
\label{sec:forward}

Different versions of the main or child documents
using compilation flags as described in \secref{sec:flags}
can be (permanently) stored in different files
for convenient compilation, viewing and distribution.
To this end, the package defines a command
to pass on compilation to a different file:

%%%%%%%%%%%%%%%%%%%%%%%%%%%%%%%%%%%%%%%%
\DescribeMacro{\childdocforward}
The command |\childdocforward| redirects processing to
another source file:
%
\begin{center}
\begin{tabular}{l}
|\input{childdoc.def}|\\
|\childdocforward[|\textit{main}|]{|\textit{dest}|}|\\
\end{tabular}
\end{center}
%
The argument \textit{dest} is the destination file
(without extension).
It should be the main file or one of the child files.
Note that further \textsf{childdoc} directives
such as |\childdocof| and |\childdocforward|
in the indicated file will be processed in this form.
The optional argument \textit{main}
passes on directly to the main file \textit{main}
while pretending to compile the child \textit{dest}.
This form behaves as if \textit{dest}
issues |\childdocof{|\textit{main}|}| right away,
and no further \textsf{childdoc} directives will be processed.

%%%%%%%%%%%%%%%%%%%%%%%%%%%%%%%%%%%%%%%%
\DescribeMacro{\...prefix}
In the alternative form |\childdocforwardprefix|,
%
\begin{center}
\begin{tabular}{l}
|\input{childdoc.def}|\\
|\childdocforwardprefix[|\textit{main}|]{|\textit{prefix}|}{|\textit{dest}|}|
\end{tabular}
\end{center}
%
the destination file is determined by a pattern
depending on the current file:
To make this work, the current file must be called
`{\textit{prefix}\hspace{0.2em}\textit{suffix}}'
with \textit{prefix} matching precisely the argument.
Processing is then passed on to the file
`{\textit{dest}\hspace{0.2em}\textit{suffix}}'.
Surely, the same effect is achieved by
directly specifying the
argument `{\textit{dest}\hspace{0.2em}\textit{suffix}}'
in the first form.
However, that requires to set up a different file
for each child. With the alternative form of the command
all these files can have exactly the same content
which simplifies setting them up and maintaining them.

For example, the following file |draft.tex|
with a compilation flag |\version| as described in \secref{sec:flags}
compiles the main document as a draft:
%
\begin{center}
\begin{tabular}{l}
|\def\version{draft}|\\
|\input{childdoc.def}|\\
|\childdocforward{|\textit{main}|}|
\end{tabular}
\end{center}
%
Likewise, the following files |final|\textit{nn}|.tex|
compile the final version of the child document
|child|\textit{nn}|.tex|:
%
\begin{center}
\begin{tabular}{l}
|\def\version{final}|\\
|\input{childdoc.def}|\\
|\childdocforwardprefix{final}{child}|
\end{tabular}
\end{center}
%

Note that when several versions of a main file and/or of each child file
are to be generated, it may be convenient to set up a |Makefile| or
shell script to automatise the process.

%%%%%%%%%%%%%%%%%%%%%%%%%%%%%%%%%%%%%%%%%%%%%%%%%%%%%%%%%%%%%%%%%%%%%%%%%%%%%%%%
\subsection{Command Line Processing}
\label{sec:commandline}

The effect of redirection files can also be achieved by invoking
the \LaTeX{} compiler with a more elaborate command line.
Most conveniently this should be done as part
of a shell script or a |Makefile|.

When using \textsf{childdoc} in the main file, the following
command lines effectively perform a redirection
(note that depending on the shell being used,
backslashes may have to be doubled: `|\|' $\to$ `|\\|'):
%
\begin{center}
|... -jobname "|\textit{target}|" |\\|"|[\textit{flags}]%
|\input{childdoc.def}\childdocforward[|\textit{main}|]{|\textit{dest}|}"|
\end{center}
%
Here \textit{target} is the name of the output file,
\textit{main} is the name of the main file
and \textit{dest} is the name of the main or child file to be processed
(all filenames without extensions).
The optional argument \textit{main} can be omitted
if \textit{main} matches \textit{dest}.
Optionally, compilation \textit{flags} can be defined via |\def| commands.
This command line makes the \TeX{} engine believe
it is compiling the file \textit{target}
whose content is specified as the latter parameter.
The provided code then forwards the processing to
\textit{main} or \textit{dest} as described in \secref{sec:forward}.

%%%%%%%%%%%%%%%%%%%%%%%%%%%%%%%%%%%%%%%%%%%%%%%%%%%%%%%%%%%%%%%%%%%%%%%%%%%%%%%%
\subsection{Include by Input}
\label{sec:input}

Including child documents by |\include| has some restrictions by design.
Most notably, the content of a child document always occupies
its own set of pages; pages cannot be shared between child documents.
Usually, this behaviour makes perfect sense
because each child document contain an essential part of the document.
However, in some situations it may be desirable to compose
a document from a collection of parts
without having mandatory page breaks between then.
For this case, the package
provides a mechanism to include parts
by |\input| which can also be processed individually.
However, by construction this mechanism
requires manual handling of the content to be output.

%%%%%%%%%%%%%%%%%%%%%%%%%%%%%%%%%%%%%%%%
\DescribeMacro{\ifchilddocmanual}
The main file should be prepared as usual, see \secref{sec:include}.
However, the document body must make a distinction
between processing of an individual part and of the main document, e.g.:
%
\begin{center}
\begin{tabular}{l}
|\ifchilddocmanual|\\
|\input{\childdocname}|\\
|\||else|\\
\textit{document body with }|\input{|\textit{part}|}|\\
|\||fi|
\end{tabular}
\end{center}
%
The conditional |\ifchilddocmanual| is true whenever
a part to be included by |\input| is being compiled,
and the name of the part is stored in |\childdocname|.

%%%%%%%%%%%%%%%%%%%%%%%%%%%%%%%%%%%%%%%%
\DescribeMacro{\childdocby}
Each part to be included by |\input| should start with:
%
\begin{center}
\begin{tabular}{l}
|\input{childdoc.def}|\\
|\childdocby{|\textit{main}|}|\\
\end{tabular}
\end{center}
%
The directive |\childdocby| is similar to |\childdocof|
described in \secref{sec:include},
but the subsequent selection of content must be done manually.
To that end, both |\ifchilddoc| and |\ifchilddocmanual|
will be true upon processing of a part,
and the name of the part is stored in |\childdocname|.
Note that |\jobname| will be set to the filename of the current part
so that each part receives an individual |.aux| file
that does not interfere with the |.aux| file(s) of the main document.
This behaviour can be altered by the alternative form
|\childdocby[*]{|\textit{main}|}| (with a non-empty optional argument)
which uses the |.aux| file of the main document
by setting |\jobname| to \textit{main}.

%%%%%%%%%%%%%%%%%%%%%%%%%%%%%%%%%%%%%%%%%%%%%%%%%%%%%%%%%%%%%%%%%%%%%%%%%%%%%%%%
\subsection{Driver Development}
\label{sec:driver}

The \textsf{childdoc} mechanism can also be use for the development
of definition files such as \LaTeX{} styles or classes.
This case differs from the above setup with multiple parts
included by |\include| in that no |\includeonly| should be invoked.
This can be achieved by starting the include file
(before |\ProvidesPackage|) with:
%
\begin{center}
\begin{tabular}{l}
|\input{childdoc.def}|\\
|\childdocforward{|\textit{main}|}|\\
\end{tabular}
\end{center}
%
or alternatively with:
%
\begin{center}
\begin{tabular}{l}
|\input{childdoc.def}|\\
|\childdocby{|\textit{main}|}|\\
\end{tabular}
\end{center}
%
Both forms have slightly different effects as described above.
The main file is prepared as usual, see \secref{sec:include}.

%%%%%%%%%%%%%%%%%%%%%%%%%%%%%%%%%%%%%%%%%%%%%%%%%%%%%%%%%%%%%%%%%%%%%%%%%%%%%%%%
\subsection{Legacy Detection}
\label{sec:detection}

The directive |\childdocmain| in the main file can detect
whether the complete document or merely a child is to be compiled
even without using the directive |\childdocof|.
This method is deprecated because it is less robust
and there is no compelling reason to use it;
it is merely provided for backward compatibility
and it may be removed in future versions.

If the detection mechanism is to be used,
it is mandatory to correctly specify
the filename of the main file as the argument of |\childdocmain|:
%
\begin{center}
\begin{tabular}{l}
|\input{childdoc.def}|\\
|\childdocmain{|\textit{main}|}|\\
\end{tabular}
\end{center}
%
If |\jobname| does not match the argument \textit{main} of |\childdocmain|,
it is assumed that |\jobname| points to the child file to be compiled.
When using |\childdocmain| with the main file specified as argument,
it suffices to start a child file
with just |\input{|\textit{main}|}|
without loading of the package and using |\childdocof|.
If instead all processing is done
with the appropriate \textsf{childdoc} directives,
the argument of \textit{main} of |\childdocmain| can be empty.

An alternative version of the command line processing described
in \secref{sec:commandline} using the detection mechanism reads:
%
\begin{center}
|... -jobname "|\textit{target}|" "|[\textit{flags}]%
[|\def\jobname{|\textit{dest}|}|]|\input{|\textit{main}|}"|
\end{center}

%%%%%%%%%%%%%%%%%%%%%%%%%%%%%%%%%%%%%%%%%%%%%%%%%%%%%%%%%%%%%%%%%%%%%%%%%%%%%%%%
\subsection{Manual Code}
\label{sec:manual}

In case one cannot be certain whether the definitions file |childdoc.def|
is installed on the target \TeX{} distribution
and one prefers not to ship it,
it is conceivable to paste a few relevant commands into the sources.

To that end, drop all statements |\input{childdoc.def}|
and perform the replacements as outlined below.
Instead of |\childdocmain{|\textit{main}|}| add the following code
to the top of the main file:
%
\begin{center}
\begin{tabular}{l}
|\||ifdefined\childdocname\endinput\||fi\newif\ifchilddoc|\\
|\edef\childdocname{\scantokens\expandafter{\jobname\noexpand}}|\\
|\def\childdocmain{|\textit{main}|}\||ifx\childdocmain\childdocname\||else|\\
|\childdoctrue\includeonly{\childdocname}\let\jobname\childdocmain\||fi|\\
\end{tabular}
\end{center}
%
Instead of |\childdocof{|\textit{main}|}| just include the main file
at the top of each child file:
%
\begin{center}
|\input{|\textit{main}|}|
\end{center}
%
A simple redirection |\childdocforward{|\textit{dest}|}| is achieved by:
%
\begin{center}
|\def\jobname{|\textit{dest}|}\input{\jobname}|
\end{center}
%
The redirection with prefix
|\childdocforwardprefix[|\textit{prefix}|]{|\textit{dest}|}|
is accomplished by:
%
\begin{center}
\begin{tabular}{l}
|{\edef\jobname{\scantokens\expandafter{\jobname\noexpand}}|\\
|\def\redirectjob |\textit{prefix}|#1~~~{\gdef\jobname{|\textit{dest}|#1}}|\\
|\expandafter\redirectjob\jobname~~~}\input{\jobname}|
\end{tabular}
\end{center}

In an alternative approach,
child documents can be compiled by a specific command line
without additional code or specific definitions:
%
\begin{center}
|... -jobname "|\textit{target}|" "|[\textit{flags}]%
|\includeonly{|\textit{dest}|}\input{|\textit{main}|}"|
\end{center}
%

%%%%%%%%%%%%%%%%%%%%%%%%%%%%%%%%%%%%%%%%%%%%%%%%%%%%%%%%%%%%%%%%%%%%%%%%%%%%%%%%
%%%%%%%%%%%%%%%%%%%%%%%%%%%%%%%%%%%%%%%%%%%%%%%%%%%%%%%%%%%%%%%%%%%%%%%%%%%%%%%%
\section{Information}

%%%%%%%%%%%%%%%%%%%%%%%%%%%%%%%%%%%%%%%%%%%%%%%%%%%%%%%%%%%%%%%%%%%%%%%%%%%%%%%%
\subsection{Copyright}

Copyright \copyright{} 2017--2018 Niklas Beisert

This work may be distributed and/or modified under the
conditions of the \LaTeX{} Project Public License, either version 1.3
of this license or (at your option) any later version.
The latest version of this license is in
  \url{http://www.latex-project.org/lppl.txt}
and version 1.3 or later is part of all distributions of \LaTeX{}
version 2005/12/01 or later.

This work has the LPPL maintenance status `maintained'.

The Current Maintainer of this work is Niklas Beisert.

This work consists of the files |README.txt|, |childdoc.ins| and |childdoc.dtx|
as well as the derived files |childdoc.def|, |cdocsamp.tex|
with |cdocsch1.tex|, |cdocsch2.tex|, |cdocspt3.tex|, |cdocspt4.tex|,
|cdocsdrf.tex|, |cdocsfn1.tex|, |cdocsfn2.tex|
as well as |childdoc.pdf|.

%%%%%%%%%%%%%%%%%%%%%%%%%%%%%%%%%%%%%%%%%%%%%%%%%%%%%%%%%%%%%%%%%%%%%%%%%%%%%%%%
\subsection{Files and Installation}

The package consists of the files:
%
\begin{center}
\begin{tabular}{ll}
    |README.txt|   & readme file \\
    |childdoc.ins| & installation file \\
    |childdoc.dtx| & source file \\
    |childdoc.def| & definition file \\
    |cdocsamp.tex| & sample main file \\
    |cdocsch1.tex| & sample include file \\
    |cdocsch2.tex| & sample include file \\
    |cdocspt3.tex| & sample part file \\
    |cdocspt4.tex| & sample part file \\
    |cdocsdrf.tex| & sample redirection file \\
    |cdocsfn1.tex| & sample redirection file \\
    |cdocsfn2.tex| & sample redirection file \\
    |childdoc.pdf| & manual
\end{tabular}
\end{center}
%
The distribution consists of the files
|README.txt|, |childdoc.ins| and |childdoc.dtx|.
%
\begin{itemize}
\item
Run (pdf)\LaTeX{} on |childdoc.dtx|
to compile the manual |childdoc.pdf| (this file).
\item
Run \LaTeX{} on |childdoc.ins| to create the definitions file |childdoc.def|
and the sample |cdocsamp.tex| with include files
|cdocsch1.tex|, |cdocsch2.tex|, |cdocspt3.tex|, |cdocspt4.tex|,
|cdocsdrf.tex|, |cdocsfn1.tex|, |cdocsfn2.tex|.
Then copy the file |childdoc.def| to an appropriate directory of your \LaTeX{}
distribution, e.g.\ \textit{texmf-root}|/tex/latex/childdoc|.
\end{itemize}

%%%%%%%%%%%%%%%%%%%%%%%%%%%%%%%%%%%%%%%%%%%%%%%%%%%%%%%%%%%%%%%%%%%%%%%%%%%%%%%%
\subsection{Related CTAN Packages}

There are several other packages which offer a similar functionality:
%
\begin{itemize}
\item
The packages
\href{http://ctan.org/pkg/docmute}{\textsf{docmute}},
\href{http://ctan.org/pkg/includex}{\textsf{includex}} and
\href{http://ctan.org/pkg/standalone}{\textsf{standalone}}
provide commands to include only the document body of
a child file thus allowing both files to be compiled individually.
\item
The packages \href{http://ctan.org/pkg/subdocs}{\textsf{subdocs}}
and \href{http://ctan.org/pkg/subfiles}{\textsf{subfiles}}
provide structures in which the main and child documents can be
encapsulated and allowing them to be compiled individually.
The inclusion mechanism is different from the conventional |\include|.
\item
The package \href{http://ctan.org/pkg/combine}{\textsf{combine}}
is an elaborate solution to combine several documents into one.
\end{itemize}
%
See also the CTAN topic \href{http://ctan.org/topic/subdocs}{\textsf{subdocs}}
for further related packages.
The present package differs from the above solutions in that
a document structure constructed with the conventional |\include| mechanism
just needs two extra commands at the top of every file
such that all constituent files can be compiled individually.

%%%%%%%%%%%%%%%%%%%%%%%%%%%%%%%%%%%%%%%%%%%%%%%%%%%%%%%%%%%%%%%%%%%%%%%%%%%%%%%%
%\subsection{Feature Suggestions}
%
%The following is a list of features which may be useful for future
%versions of this package:
%%
%\begin{itemize}
%\item
%\ldots
%\end{itemize}

%%%%%%%%%%%%%%%%%%%%%%%%%%%%%%%%%%%%%%%%%%%%%%%%%%%%%%%%%%%%%%%%%%%%%%%%%%%%%%%%
\subsection{Revision History}

%%%%%%%%%%%%%%%%%%%%%%%%%%%%%%%%%%%%%%%%
\paragraph{v2.0:} 2018/12/30

\begin{itemize}
\item
immediate forward processing
\item
added |\childdocby| mechanism
\item
manual restructured
\end{itemize}

%%%%%%%%%%%%%%%%%%%%%%%%%%%%%%%%%%%%%%%%
\paragraph{v1.6:} 2018/01/17

\begin{itemize}
\item
application for development of include files
\item
corrections to manual
\end{itemize}

%%%%%%%%%%%%%%%%%%%%%%%%%%%%%%%%%%%%%%%%
\paragraph{v1.5:} 2017/05/21

\begin{itemize}
\item
more complete structuring introduced
\item
|\childdocof| introduced
\item
|\childdoc| renamed to |\childdocmain|
\item
|\childredirect| renamed to |\childdocforward| and |\childdocforwardprefix|
and functionality expanded
\end{itemize}

%%%%%%%%%%%%%%%%%%%%%%%%%%%%%%%%%%%%%%%%
\paragraph{v1.0:} 2017/04/27

\begin{itemize}
\item
manual and install package
\item
first version published on CTAN
\end{itemize}

%%%%%%%%%%%%%%%%%%%%%%%%%%%%%%%%%%%%%%%%
\paragraph{v0.6:} 2017/04/26

\begin{itemize}
\item
redirection mechanism added
\end{itemize}

%%%%%%%%%%%%%%%%%%%%%%%%%%%%%%%%%%%%%%%%
\paragraph{v0.5:} 2017/04/26

\begin{itemize}
\item
functionality in definition file
\end{itemize}


%%%%%%%%%%%%%%%%%%%%%%%%%%%%%%%%%%%%%%%%%%%%%%%%%%%%%%%%%%%%%%%%%%%%%%%%%%%%%%%%
%%%%%%%%%%%%%%%%%%%%%%%%%%%%%%%%%%%%%%%%%%%%%%%%%%%%%%%%%%%%%%%%%%%%%%%%%%%%%%%%
%%%%%%%%%%%%%%%%%%%%%%%%%%%%%%%%%%%%%%%%%%%%%%%%%%%%%%%%%%%%%%%%%%%%%%%%%%%%%%%%
\appendix

\settowidth\MacroIndent{\rmfamily\scriptsize 000\ }

 \DocInput{childdoc.dtx}

\end{document}
%</driver>
% \fi
%
% %%%%%%%%%%%%%%%%%%%%%%%%%%%%%%%%%%%%%%%%%%%%%%%%%%%%%%%%%%%%%%%%%%%%%%%%%%%%%%
% %%%%%%%%%%%%%%%%%%%%%%%%%%%%%%%%%%%%%%%%%%%%%%%%%%%%%%%%%%%%%%%%%%%%%%%%%%%%%%
% \section{Sample}
%\iffalse
%<*samplemain>
%\fi
%
% The following presents a sample document
% with two chapters, two parts, a title page,
% a compile flag as well as three forwarding files to set the flag.
% It consists of eight |.tex| files:
% \begin{center}
% \begin{tabular}{ll}
% |cdocsamp.tex|&main file\\
% |cdocsch1.tex|&include file for chapter 1\\
% |cdocsch2.tex|&include file for chapter 2\\
% |cdocspt3.tex|&include file for part 3\\
% |cdocspt4.tex|&include file for part 4\\
% |cdocsdrf.tex|&forwarding file for main file in draft mode\\
% |cdocsfi1.tex|&forwarding file for final version of chapter 1\\
% |cdocsfi2.tex|&forwarding file for final version of chapter 2\\
% \end{tabular}
% \end{center}
% Each of the eight files can be compiled directly by the \LaTeX{} compiler.
%
% %%%%%%%%%%%%%%%%%%%%%%%%%%%%%%%%%%%%%%
% \paragraph{Main File.}
%
% The main file is called |cdocsamp.tex|.
%
% Load the \textsf{childdoc} definitions and
% declare the filename for the main document:
%    \begin{macrocode}
\input{childdoc.def}
\childdocmain{}
%    \end{macrocode}

% Optional override for |\version| flag:
%    \begin{macrocode}
%%\ifchilddoc\else\providecommand{\version}{draft}\fi
%    \end{macrocode}

% Define the default values for the |\version| flag
% (|final| for the main file and |draft| for childs):
%    \begin{macrocode}
\ifchilddoc
\providecommand{\version}{draft}
\else
\providecommand{\version}{final}
\fi
%    \end{macrocode}

% Load the standard document class:
%    \begin{macrocode}
\documentclass[12pt]{article}
%    \end{macrocode}

% Start the document body:
%    \begin{macrocode}
\begin{document}
%    \end{macrocode}

% Declare a title page.
% Print title, part of document being processed and version flag:
%    \begin{macrocode}
\addtocounter{page}{-1}
\begin{center}
{\LARGE\bfseries{}childdoc example\par}
\vspace{1cm}
\ifchilddoc
\ifchilddocmanual part\else chapter\fi:
`\childdocname' of `\childdocjob'\par
\else
main document: `\childdocjob'\par
\fi
version: \version\par
\end{center}
\newpage
%    \end{macrocode}

% Manually include selected file,
% otherwise process as usual:
%    \begin{macrocode}
\ifchilddocmanual
\section*{part `\childdocname'}
\input{\childdocname}
\else
%    \end{macrocode}

% Include the two chapters:
%    \begin{macrocode}
\include{cdocsch1}
\include{cdocsch2}
%    \end{macrocode}

% Include the two parts unless only chapters should be displayed:
%    \begin{macrocode}
\ifchilddoc\else
\section{part three}
\input{cdocspt3}
\section{part four}
\input{cdocspt4}
\fi
%    \end{macrocode}

% Process as usual until here:
%    \begin{macrocode}
\fi
%    \end{macrocode}

% End of document body:
%    \begin{macrocode}
\end{document}
%    \end{macrocode}
%\iffalse
%</samplemain>
%\fi
%
% %%%%%%%%%%%%%%%%%%%%%%%%%%%%%%%%%%%%%%
% \paragraph{Chapter Include Files.}
%
% The include files are called |cdocsch1.tex| and |cdocsch2.tex|.
%
%\iffalse
%<*samplechap1|samplechap2>
%\fi

% Optional override for |\version| flag:
%    \begin{macrocode}
%%\providecommand{\version}{final}
%    \end{macrocode}

% Include the main document:
%    \begin{macrocode}
\input{childdoc.def}
\childdocof{cdocsamp}
%    \end{macrocode}

%\iffalse
%</samplechap1|samplechap2>
%\fi
%
%\iffalse
%<*samplechap1>
%\fi
% Some text for chapter 1:
%    \begin{macrocode}
\section{one}
some text in chapter one
%    \end{macrocode}

%\iffalse
%</samplechap1>
%\fi
% Some text for chapter 2:
%\iffalse
%<*samplechap2>
%\fi
%    \begin{macrocode}
\section{two}
more text in chapter two
%    \end{macrocode}

%\iffalse
%</samplechap2>
%\fi
%
% %%%%%%%%%%%%%%%%%%%%%%%%%%%%%%%%%%%%%%
% \paragraph{Part Include Files.}
%
% The include files are called |cdocspt3.tex| and |cdocspt4.tex|.
%
%\iffalse
%<*samplepart3|samplepart4>
%\fi

% Optional override for |\version| flag:
%    \begin{macrocode}
%%\providecommand{\version}{final}
%    \end{macrocode}

% Include the main document:
%    \begin{macrocode}
\input{childdoc.def}
\childdocby{cdocsamp}
%    \end{macrocode}

%\iffalse
%</samplepart3|samplepart4>
%\fi
%
%\iffalse
%<*samplepart3>
%\fi
% Some text for part 3:
%    \begin{macrocode}
some text in part three
%    \end{macrocode}

%\iffalse
%</samplepart3>
%\fi
% Some text for part 4:
%\iffalse
%<*samplepart4>
%\fi
%    \begin{macrocode}
more text in part four
%    \end{macrocode}

%\iffalse
%</samplepart4>
%\fi
%
% %%%%%%%%%%%%%%%%%%%%%%%%%%%%%%%%%%%%%%
% \paragraph{Forwarding for a Complete Draft.}
%
% The following forwarding file |cdocsdrf.tex|
% compiles the main document in draft mode:
%\iffalse
%<*sampledraft>
%\fi
%    \begin{macrocode}
\def\version{draft}
\input{childdoc.def}
\childdocforward{cdocsamp}
%    \end{macrocode}

%\iffalse
%</sampledraft>
%\fi
%
% %%%%%%%%%%%%%%%%%%%%%%%%%%%%%%%%%%%%%%
% \paragraph{Forwarding for Final Version of the Chapters.}
%
% The following forwarding files |cdocsfn1.tex| and |cdocsfn2.tex|
% (with identical content)
% compile the final versions of the child documents
% |cdocsch1.tex| and |cdocsch2.tex|, respectively:
%\iffalse
%<*samplefinal>
%\fi
%    \begin{macrocode}
\def\version{final}
\input{childdoc.def}
\childdocforwardprefix[cdocsamp]{cdocsfn}{cdocsch}
%    \end{macrocode}

%\iffalse
%</samplefinal>
%\fi
%
% %%%%%%%%%%%%%%%%%%%%%%%%%%%%%%%%%%%%%%
% \paragraph{Command Line Processing.}
%
% The following three command lines generate the output files
% |cdocscld|, |cdocscl1| and |cdocscl2|
% which should be identical to
% |cdocsdrf|, |cdocsch1| and |cdocsfn2|, respectively:
% \begin{center}
% \begin{tabular}{l}
% |latex -jobname cdocscld \|\\
% |  "\def\version{draft}\input{childdoc.def}\childdocforward{cdocsamp}"|\\
% |latex -jobname cdocscl1 \|\\
% |  "\input{childdoc.def}\childdocforward[cdocsamp]{cdocsch1}"|\\
% |latex -jobname cdocscl2 \|\\
% |  "\def\version{final}\input{childdoc.def}\childdocforward{cdocsch2}"|
% \end{tabular}
% \end{center}
% Note that the trailing backslash on each first line
% merely continues the input to the second line
% (for convenient cut ant paste).
% Furthermore, the command |latex| can be replaced by any
% of its alternative versions such as |pdflatex|.
%
% %%%%%%%%%%%%%%%%%%%%%%%%%%%%%%%%%%%%%%%%%%%%%%%%%%%%%%%%%%%%%%%%%%%%%%%%%%%%%%
% %%%%%%%%%%%%%%%%%%%%%%%%%%%%%%%%%%%%%%%%%%%%%%%%%%%%%%%%%%%%%%%%%%%%%%%%%%%%%%
% \section{Implementation}
%\iffalse
%<*package>
%\fi
%
% This section describes the definitions file |childdoc.def|.

% The definitions cannot be loaded using |\usepackage| or |\RequirePackage|
% which has a mechanism to prevent loading a style file more than once.
% When loading the definitions by means of |\input|
% multiple instances have to be prevented manually:
%\iffalse
%This code needs to be before the `\ProvidesFile' directive
%which is defined at the beginning of this file.
%Therefore it is also placed there and commented out here.
%</package>
%<*discard>
%\fi
%    \begin{macrocode}
\ifdefined\childdocmain\endinput\fi
%    \end{macrocode}
%\iffalse
%</discard>
%<*package>
%\fi
%
% \macro{\ifchilddoc}
% \macro{\ifchilddocmanual}
% The conditional |\ifchilddoc| tells whether a
% child (true) or main (false) document is being compiled.
% The conditional |\ifchilddocmanual| tells whether
% the |\includeonly| mechanism is used (false) or
% the selection of child files must be performed manually (true).
% The definitions initialise to false:
%    \begin{macrocode}
\newif\ifchilddoc
\newif\ifchilddocmanual
%    \end{macrocode}

% \macro{\childdocname}
% \macro{\childdocjob}
% The macro |\childdocname| stores the name of the main document
% to be compiled. The macro |\childdocjob| stores the name of
% the document on which the \LaTeX{} compiler was originally invoked.
% The content of |\jobname| cannot be compared
% to filenames specified in the source due to different catcodes.
% The following code rescans |\jobname|, stores the result
% in |\childdocname| and saves a copy in |\childdocjob|:
%    \begin{macrocode}
\edef\childdocname{\scantokens\expandafter{\jobname\noexpand}}
\let\childdocjob\childdocname
%    \end{macrocode}

% \macro{\childdocdisable}
% The macro |\childdocdisable| prevents the main file
% from being processed more than once.
% At this stage, the main document command |\childdocmain|
% is assumed to be called once again where it should do nothing.
% Any subsequent call to it should prevent
% a secondary processing of the main document
% It overwrites the forwarding commands
% |\childdocof| and |\childdocforward|
% with empty macros to prevent further inclusions of the main document:
%    \begin{macrocode}
\newcommand{\childdocdisable}
{
  \renewcommand{\childdocmain}[1]{\renewcommand{\childdocmain}[1]{\endinput}}
  \renewcommand{\childdocof}[1]{}
  \renewcommand{\childdocby}[2][]{}
  \renewcommand{\childdocforward}[2][]{}
  \renewcommand{\childdocdisable}{}
}
%    \end{macrocode}

% \macro{\childdocmain}
% The macro |\childdocmain| is to be called at the top of the main file
% with nothing or the main filename (without extension) as argument.
% First, it breaks loops.
% If the argument is not empty and does not match |\childdocname|
% (which is set by the first inclusion of |childdoc.def|),
% |\ifchilddoc| is set to true, |\includeonly| is applied to the child file
% and |\jobname| is set to the main file
% (for proper handling of |.aux| files):
%    \begin{macrocode}
\newcommand{\childdocmain}[1]
{
  \childdocdisable\childdocmain{}
  \if?#1?\else
    \begingroup
      \def\childdoctmp{#1}
      \ifx\childdoctmp\childdocname
        \def\childdoctmp{}
      \else
        \def\childdoctmp
        {
          \childdoctrue
          \includeonly{\childdocname}
          \def\childdocjob{#1}
          \def\jobname{#1}
        }
      \fi
      \expandafter
    \endgroup
    \childdoctmp
  \fi
}
%    \end{macrocode}

% \macro{\childdocof}
% The command |\childdocof| redirects
% compilation to the main file |#1|.
%    \begin{macrocode}
\newcommand{\childdocof}[1]
{
  \childdocdisable
  \childdoctrue
  \includeonly{\childdocname}
  \def\jobname{#1}
  \def\childdocjob{#1}
  \input{#1}
}
%    \end{macrocode}

% \macro{\childdocby}
% The command |\childdocby| ....
%    \begin{macrocode}
\newcommand{\childdocby}[2][]
{
  \childdocdisable
  \childdoctrue
  \childdocmanualtrue
  \if?#1?\else
    \def\jobname{#2}
  \fi
  \def\childdocjob{#2}
  \input{#2}
  \endinput
}
%    \end{macrocode}

% \macro{\childdocforward}
% The command |\childdocforward| redirects
% compilation to the main file or
% (if the optional argument is given) a child file.
% Parameters are set as if the main file
% or a child file starting with |\childdocof| was compiled.
% Then compilation is handed over to the main file:
%    \begin{macrocode}
\newcommand{\childdocforward}[2][]
{
  \begingroup
    \if?#1?
      \def\childdoctmp
      {
        \def\childdocname{#2}
        \def\childdocjob{#2}
        \def\jobname{#2}
        \input{#2}
        \endinput
      }
    \else
      \def\childdoctmp
      {
        \childdocdisable
        \def\childdocname{#2}
        \childdoctrue
        \includeonly{#2}
        \def\childdocjob{#1}
        \def\jobname{#1}
        \input{#1}
        \endinput
      }
    \fi
    \expandafter
  \endgroup
  \childdoctmp
}
%    \end{macrocode}

% \macro{\childdocforwardprefix}
% The command |\childdocforwardprefix| redirects
% compilation to the main or a child file by means of a pattern.
% The prefix |#1| in the current filename is replaced by |#2|
% and the suffix of the current filename is kept
% (it is assumed that the filename does not contain the substring `|~~~|'
% which is used as a delimiter).
% Compilation is handed over to the new file by |\childdocforward|:
%    \begin{macrocode}
\newcommand{\childdocforwardprefix}[3][]
{
  \begingroup
    \def\childdocextract #2##1~~~{\def\childdoctmp{\childdocforward[#1]{#3##1}}}
    \expandafter\childdocextract\childdocname~~~
    \expandafter
  \endgroup
  \childdoctmp
}
%    \end{macrocode}

% \macro{\childdoc}
% The deprecated macro |\childdoc| is a legacy version of |\childdocmain|:
%    \begin{macrocode}
\newcommand{\childdoc}{\childdocmain}
%    \end{macrocode}

% \macro{\childdocredirect}
% The deprecated macro |\childdocredirect| is a legacy version
% of |\childdocforward| and |\childdocforwardprefix|:
%    \begin{macrocode}
\newcommand{\childdocredirect}[2][]
{
  \begingroup
    \if?#1?
      \def\childdoctmp{\childdocforward{#2}}
    \else
      \def\childdoctmp{\childdocforwardprefix{#1}{#2}}
    \fi
    \expandafter
  \endgroup
  \childdoctmp
}
%    \end{macrocode}

%\iffalse
%</package>
%\fi
%
\endinput
|\\
|\childdocforwardprefix{final}{child}|
\end{tabular}
\end{center}
%

Note that when several versions of a main file and/or of each child file
are to be generated, it may be convenient to set up a |Makefile| or
shell script to automatise the process.

%%%%%%%%%%%%%%%%%%%%%%%%%%%%%%%%%%%%%%%%%%%%%%%%%%%%%%%%%%%%%%%%%%%%%%%%%%%%%%%%
\subsection{Command Line Processing}
\label{sec:commandline}

The effect of redirection files can also be achieved by invoking
the \LaTeX{} compiler with a more elaborate command line.
Most conveniently this should be done as part
of a shell script or a |Makefile|.

When using \textsf{childdoc} in the main file, the following
command lines effectively perform a redirection
(note that depending on the shell being used,
backslashes may have to be doubled: `|\|' $\to$ `|\\|'):
%
\begin{center}
|... -jobname "|\textit{target}|" |\\|"|[\textit{flags}]%
|% \iffalse
%
% childdoc.dtx Copyright (C) 2017-2018 Niklas Beisert
%
% This work may be distributed and/or modified under the
% conditions of the LaTeX Project Public License, either version 1.3
% of this license or (at your option) any later version.
% The latest version of this license is in
%   http://www.latex-project.org/lppl.txt
% and version 1.3 or later is part of all distributions of LaTeX
% version 2005/12/01 or later.
%
% This work has the LPPL maintenance status `maintained'.
%
% The Current Maintainer of this work is Niklas Beisert.
%
% This work consists of the files childdoc.dtx and childdoc.ins
% and the derived files childdoc.def and cdocsamp.tex with
% cdocsch1.tex, cdocsch2.tex, cdocsdrf.tex, cdocsfn1.tex, cdocsfn2.tex.
%
%<package>\ifdefined\childdocmain\endinput\fi
%<package>\ProvidesFile{childdoc.def}[2018/12/30 v2.0 child document driver]
%<samplemain>\ProvidesFile{cdocsamp.tex}[2018/12/30 v2.0 sample for childdoc]
%<*driver>
%\ProvidesFile{childdoc.drv}[2018/12/30 v2.0 childdoc reference manual file]
\PassOptionsToClass{10pt,a4paper}{article}
\documentclass{ltxdoc}

\usepackage[margin=35mm]{geometry}
\usepackage{hyperref}
\usepackage{hyperxmp}
\usepackage[usenames]{color}

\hypersetup{colorlinks=true}
\hypersetup{pdfstartview=FitH}
\hypersetup{pdfpagemode=UseNone}
\hypersetup{pdfsource={}}
\hypersetup{pdflang={en-UK}}
\hypersetup{pdfcopyright={Copyright 2017-2018 Niklas Beisert.
  This work may be distributed and/or modified under the
  conditions of the LaTeX Project Public License, either version 1.3
  of this license or (at your option) any later version.}}
\hypersetup{pdflicenseurl={http://www.latex-project.org/lppl.txt}}
\hypersetup{pdfcontactaddress={ETH Zurich, ITP, HIT K,
  Wolfgang-Pauli-Strasse 27}}
\hypersetup{pdfcontactpostcode={8093}}
\hypersetup{pdfcontactcity={Zurich}}
\hypersetup{pdfcontactcountry={Switzerland}}
\hypersetup{pdfcontactemail={nbeisert@itp.phys.ethz.ch}}
\hypersetup{pdfcontacturl={http://people.phys.ethz.ch/\xmptilde nbeisert/}}

\newcommand{\secref}[1]{\hyperref[#1]{section \ref*{#1}}}

\parskip1ex
\parindent0pt
\let\olditemize\itemize
\def\itemize{\olditemize\parskip0pt}

\begin{document}

\title{The \textsf{childdoc} Package}
\hypersetup{pdftitle={The childdoc Package}}
\author{Niklas Beisert\\[2ex]
  Institut f\"ur Theoretische Physik\\
  Eidgen\"ossische Technische Hochschule Z\"urich\\
  Wolfgang-Pauli-Strasse 27, 8093 Z\"urich, Switzerland\\[1ex]
  \href{mailto:nbeisert@itp.phys.ethz.ch}
  {\texttt{nbeisert@itp.phys.ethz.ch}}}
\hypersetup{pdfauthor={Niklas Beisert}}
\hypersetup{pdfsubject={Manual for the LaTeX2e Package childdoc}}
\date{30 December 2018, \textsf{v2.0}}
\maketitle

\begin{abstract}\noindent
\textsf{childdoc} is a \LaTeXe{} package
that enables the direct compilation
of document sections included by |\include|
to individual files.
\end{abstract}

\begingroup
\parskip0ex
\tableofcontents
\endgroup

%%%%%%%%%%%%%%%%%%%%%%%%%%%%%%%%%%%%%%%%%%%%%%%%%%%%%%%%%%%%%%%%%%%%%%%%%%%%%%%%
%%%%%%%%%%%%%%%%%%%%%%%%%%%%%%%%%%%%%%%%%%%%%%%%%%%%%%%%%%%%%%%%%%%%%%%%%%%%%%%%
\section{Introduction}

\LaTeX{} provides a mechanism to structure a large document (such as a book)
into a main file and several child files (containing the chapters)
using the |\include| command.
This mechanism is beneficial for documents
which span hundreds of pages in order to
make the source file(s) more manageable.
Moreover, compilation can be restricted to
selected child files by means of the |\includeonly| command.
The latter feature can be used to reduce the compilation time while editing
(this was significantly more useful in the earlier days of \LaTeX{})
or to generate a smaller document which is easier to navigate.
Another application of |\includeonly| is to generate
documents consisting of selected parts of the complete document.

However, there are a few drawbacks of the plain |\include| mechanism:
\begin{itemize}
\item
The child files cannot be compiled on their own,
they can only be compiled via the main file.
A naive editing environment
(such as a text editor with an option
to have the current file processed by \LaTeX)
may require one to switch to the main file before compiling;
attempting to compile the child file produces errors.
\item
The main file must be modified (each time)
to adjust the |\includeonly| command
to the present needs. This easily leaves the main file in a messy state.
\item
The generated document will always carry the filename
of the main document. This is inconvenient if
several child files are to be compiled and
to be kept for distribution.
\end{itemize}

The present package provides a simple interface
to make child files individually compilable by \LaTeX{}.
Compiling a child file then has the same effect as compiling
the main file with an |\includeonly| command
to select the appropriate child.
Moreover the generated document will carry the name of the child
rather than the main file.
This resolves all three above issues.

This feature is meant to make the editing of books,
thesis documents and lecture notes somewhat more convenient.
However, the package can also be used efficiently for
composing a series of documents (such as exercise sheets)
which are typically distributed individually.
It then assists the author in generating the individual documents
(potentially in different versions)
as well as a document containing the collected series.
Another application is in developing style files
or other kinds of included material
where compilation of the style file could redirect
to a sample or test file.

%%%%%%%%%%%%%%%%%%%%%%%%%%%%%%%%%%%%%%%%%%%%%%%%%%%%%%%%%%%%%%%%%%%%%%%%%%%%%%%%
%%%%%%%%%%%%%%%%%%%%%%%%%%%%%%%%%%%%%%%%%%%%%%%%%%%%%%%%%%%%%%%%%%%%%%%%%%%%%%%%
\section{Usage}

First of all, the package \textsf{childdoc} is \emph{not} a standard
\LaTeXe{} |.sty| style file! Therefore it needs to be invoked in
a non-standard way.

%%%%%%%%%%%%%%%%%%%%%%%%%%%%%%%%%%%%%%%%%%%%%%%%%%%%%%%%%%%%%%%%%%%%%%%%%%%%%%%%
\subsection{Included Files}
\label{sec:include}

%%%%%%%%%%%%%%%%%%%%%%%%%%%%%%%%%%%%%%%%
\DescribeMacro{\childdocmain}
To use the package, add the commands
\begin{center}
\begin{tabular}{l}
|\input{childdoc.def}|\\
|\childdocmain{}|\\
\end{tabular}
\end{center}
at the very top of the main \LaTeX{} file,
in particular \emph{before} the |\documentclass| statement!
The argument of |\childdocmain| should be left empty
(but it must be present).

%%%%%%%%%%%%%%%%%%%%%%%%%%%%%%%%%%%%%%%%
\DescribeMacro{\childdocof}
Furthermore, add the commands
\begin{center}
\begin{tabular}{l}
|\input{childdoc.def}|\\
|\childdocof{|\textit{main}|}|\\
\end{tabular}
\end{center}
at the top of every child file \textit{child}
which is included by |\include{|\textit{child}|}|
from within the main file
(or at least for those files to be compiled individually).
The argument \textit{main} must be the filename of the main file.

There are a couple of
considerations in setting up the main and child documents:

%%%%%%%%%%%%%%%%%%%%%%%%%%%%%%%%%%%%%%%%
\paragraph{Restrictions.}

Please note the following restrictions:
\begin{itemize}
\item
|\childdocmain| must be called with one argument \textit{main}
to ensure compatibility with earlier version of the package.
It must either be empty (|\childdocmain{}|)
or precisely match the filename of the main file in which it is specified.
See \secref{sec:detection} for further information.
\item
The filename \textit{main} must be specified without the |.tex| extension.
\item
The filename \textit{main} is case sensitive
(even in case-insensitive file systems)
due to internal string comparison.
\item
The argument \textit{main} should be fully expanded, it cannot be a macro.
\item
Subdirectories and special characters should be avoided in filenames.
\item
The command |\childdocmain{|\textit{main}|}| must be followed by a whitespace.
It should not be followed immediately by another command
or by a comment mark `|%|'.
This is because the \TeX{} parser reads the token immediately following
the argument of |\childdocmain| and puts it
at the beginning of every child section;
however, a white\-space is ignored.
\end{itemize}

%%%%%%%%%%%%%%%%%%%%%%%%%%%%%%%%%%%%%%%%
\paragraph{Content of Main File.}

It is advisable to place all content in the child files included by |\include|.
Any output contained in the main file will appear in all child documents
unless suppressed manually;
it cannot be suppressed automatically by the |\includeonly| directive
and thus should normally be avoided.
A method to include some content in the main file
by means of conditional processing is described in \secref{sec:conditional}.

%%%%%%%%%%%%%%%%%%%%%%%%%%%%%%%%%%%%%%%%
\paragraph{Page Numbering.}

When only a part of the document is compiled,
the appropriate numbering of pages
(as well as other status parameters)
is determined from the |.aux| files.
The latter contain information from previous passes.
However this information needs to propagate through
all intermediate child documents.
Therefore the page numbering in child documents may well
be inconsistent until the complete document is compiled at least once.

A useful (if unconventional) way to always ensure a consistent
page numbering is to restart the numbering in each child document
and denote the pages by `\textit{child}|.|\textit{page}'
where \textit{child} represents the chapter/section number of the child file.
This can be achieved by the command
|\numberwithin{page}{|\textit{child}|}|
of the \textsf{amsmath} package
where \textit{child} can be |chapter| or |section|
depending on the chosen structuring.
Alternatively, one can modify the macro |\thepage| appropriately
and reset the counter |page| at the start of each child file.

%%%%%%%%%%%%%%%%%%%%%%%%%%%%%%%%%%%%%%%%%%%%%%%%%%%%%%%%%%%%%%%%%%%%%%%%%%%%%%%%
\subsection{Conditional Processing}
\label{sec:conditional}

The package provides a mechanism to compile different versions
of a document. To customise the versions further some conditional processing
can come in handy to distinguish which version is being compiled.
The package provides two macros to describe the compilation context:

%%%%%%%%%%%%%%%%%%%%%%%%%%%%%%%%%%%%%%%%
\DescribeMacro{\ifchilddoc}
The conditional |\ifchilddoc| distinguishes between the compilation of
child documents and the main document:
%
\begin{center}
|\ifchilddoc |\textit{child-code}| |[|\||else |\textit{main-code}]| \||fi|
\end{center}

%%%%%%%%%%%%%%%%%%%%%%%%%%%%%%%%%%%%%%%%
\DescribeMacro{\childdocname}
\DescribeMacro{\childdocjob}
The macro |\childdocname| contains the filename (without extension)
of the main or child file being processed.
Note that |\childdocjob| will always contain the name of the main file.

%%%%%%%%%%%%%%%%%%%%%%%%%%%%%%%%%%%%%%%%
\paragraph{Title Page.}

Conditional processing can be used to include a title or banner page
in the main document when proper precautions are taken.
Importantly, the code in the main file should ensure that the page counter
(as well as other status parameters which are stored in the |.aux| files)
takes the same value after the conditional processing.
Otherwise the page numbers may take divergent values
depending on which part is compiled.

For example, a title page could be declared by:
%
\begin{center}
\begin{tabular}{l}
|\ifchilddoc\||else|\\
|\addtocounter{page}{-1}|\\
\textit{code for title page}\\
|\newpage|\\
|\||fi|
\end{tabular}
\end{center}
%
A banner page for the child documents can be generated by:
%
\begin{center}
\begin{tabular}{l}
|\ifchilddoc|\\
|\addtocounter{page}{-1}|\\
\textit{code for banner page}\\
|\newpage|\\
|\||fi|
\end{tabular}
\end{center}
%
Here one could write a message such as:
\begin{center}
|This is the part \childdocname{} of \childdocjob{}.|
\end{center}

%%%%%%%%%%%%%%%%%%%%%%%%%%%%%%%%%%%%%%%%%%%%%%%%%%%%%%%%%%%%%%%%%%%%%%%%%%%%%%%%
\subsection{Flags}
\label{sec:flags}

The package makes it easy to generate different versions
of the main or child documents.
To this end compilation flags can be defined
and assigned different default values.
They will be particularly useful in conjunction
with the forwarding mechanism described in \secref{sec:forward}.

For example, it may be useful to have a flag |\version|
which can be set to |draft| or |final|.
The document source will contain some conditional code
depending on the value of |\version|.
Suppose further, the flag should default to |final| for the main file
and to |draft| for child files
which is a natural assignment for editing the document.
This is achieved by placing the following code
in the preamble of the main document
(below the |\childdocmain| directive):
%
\begin{center}
\begin{tabular}{l}
|\ifchilddoc|\\
|\providecommand{\version}{draft}|\\
|\||else|\\
|\providecommand{\version}{final}|\\
|\||fi|
\end{tabular}
\end{center}
%
The definition by |\providecommand| makes sure
that previous definitions are not overwritten.
Further statements |\providecommand{\version}{...}|
can thus be added before the above code to override it.

For the main file, one might add a line
(between |\childdocmain| and the above block)
%
\begin{center}
|%\ifchilddoc\||else\providecommand{\version}{draft}\||fi|
\end{center}
%
which can be uncommented to produce a draft version.
Likewise one can add a line to the very top of a child file
(above the |\childdocof{|\textit{main}|}| directive)
%
\begin{center}
|%\providecommand{\version}{final}|
\end{center}
%
which can be uncommented to produce the final version of this child document.

%%%%%%%%%%%%%%%%%%%%%%%%%%%%%%%%%%%%%%%%%%%%%%%%%%%%%%%%%%%%%%%%%%%%%%%%%%%%%%%%
\subsection{Forwarding}
\label{sec:forward}

Different versions of the main or child documents
using compilation flags as described in \secref{sec:flags}
can be (permanently) stored in different files
for convenient compilation, viewing and distribution.
To this end, the package defines a command
to pass on compilation to a different file:

%%%%%%%%%%%%%%%%%%%%%%%%%%%%%%%%%%%%%%%%
\DescribeMacro{\childdocforward}
The command |\childdocforward| redirects processing to
another source file:
%
\begin{center}
\begin{tabular}{l}
|\input{childdoc.def}|\\
|\childdocforward[|\textit{main}|]{|\textit{dest}|}|\\
\end{tabular}
\end{center}
%
The argument \textit{dest} is the destination file
(without extension).
It should be the main file or one of the child files.
Note that further \textsf{childdoc} directives
such as |\childdocof| and |\childdocforward|
in the indicated file will be processed in this form.
The optional argument \textit{main}
passes on directly to the main file \textit{main}
while pretending to compile the child \textit{dest}.
This form behaves as if \textit{dest}
issues |\childdocof{|\textit{main}|}| right away,
and no further \textsf{childdoc} directives will be processed.

%%%%%%%%%%%%%%%%%%%%%%%%%%%%%%%%%%%%%%%%
\DescribeMacro{\...prefix}
In the alternative form |\childdocforwardprefix|,
%
\begin{center}
\begin{tabular}{l}
|\input{childdoc.def}|\\
|\childdocforwardprefix[|\textit{main}|]{|\textit{prefix}|}{|\textit{dest}|}|
\end{tabular}
\end{center}
%
the destination file is determined by a pattern
depending on the current file:
To make this work, the current file must be called
`{\textit{prefix}\hspace{0.2em}\textit{suffix}}'
with \textit{prefix} matching precisely the argument.
Processing is then passed on to the file
`{\textit{dest}\hspace{0.2em}\textit{suffix}}'.
Surely, the same effect is achieved by
directly specifying the
argument `{\textit{dest}\hspace{0.2em}\textit{suffix}}'
in the first form.
However, that requires to set up a different file
for each child. With the alternative form of the command
all these files can have exactly the same content
which simplifies setting them up and maintaining them.

For example, the following file |draft.tex|
with a compilation flag |\version| as described in \secref{sec:flags}
compiles the main document as a draft:
%
\begin{center}
\begin{tabular}{l}
|\def\version{draft}|\\
|\input{childdoc.def}|\\
|\childdocforward{|\textit{main}|}|
\end{tabular}
\end{center}
%
Likewise, the following files |final|\textit{nn}|.tex|
compile the final version of the child document
|child|\textit{nn}|.tex|:
%
\begin{center}
\begin{tabular}{l}
|\def\version{final}|\\
|\input{childdoc.def}|\\
|\childdocforwardprefix{final}{child}|
\end{tabular}
\end{center}
%

Note that when several versions of a main file and/or of each child file
are to be generated, it may be convenient to set up a |Makefile| or
shell script to automatise the process.

%%%%%%%%%%%%%%%%%%%%%%%%%%%%%%%%%%%%%%%%%%%%%%%%%%%%%%%%%%%%%%%%%%%%%%%%%%%%%%%%
\subsection{Command Line Processing}
\label{sec:commandline}

The effect of redirection files can also be achieved by invoking
the \LaTeX{} compiler with a more elaborate command line.
Most conveniently this should be done as part
of a shell script or a |Makefile|.

When using \textsf{childdoc} in the main file, the following
command lines effectively perform a redirection
(note that depending on the shell being used,
backslashes may have to be doubled: `|\|' $\to$ `|\\|'):
%
\begin{center}
|... -jobname "|\textit{target}|" |\\|"|[\textit{flags}]%
|\input{childdoc.def}\childdocforward[|\textit{main}|]{|\textit{dest}|}"|
\end{center}
%
Here \textit{target} is the name of the output file,
\textit{main} is the name of the main file
and \textit{dest} is the name of the main or child file to be processed
(all filenames without extensions).
The optional argument \textit{main} can be omitted
if \textit{main} matches \textit{dest}.
Optionally, compilation \textit{flags} can be defined via |\def| commands.
This command line makes the \TeX{} engine believe
it is compiling the file \textit{target}
whose content is specified as the latter parameter.
The provided code then forwards the processing to
\textit{main} or \textit{dest} as described in \secref{sec:forward}.

%%%%%%%%%%%%%%%%%%%%%%%%%%%%%%%%%%%%%%%%%%%%%%%%%%%%%%%%%%%%%%%%%%%%%%%%%%%%%%%%
\subsection{Include by Input}
\label{sec:input}

Including child documents by |\include| has some restrictions by design.
Most notably, the content of a child document always occupies
its own set of pages; pages cannot be shared between child documents.
Usually, this behaviour makes perfect sense
because each child document contain an essential part of the document.
However, in some situations it may be desirable to compose
a document from a collection of parts
without having mandatory page breaks between then.
For this case, the package
provides a mechanism to include parts
by |\input| which can also be processed individually.
However, by construction this mechanism
requires manual handling of the content to be output.

%%%%%%%%%%%%%%%%%%%%%%%%%%%%%%%%%%%%%%%%
\DescribeMacro{\ifchilddocmanual}
The main file should be prepared as usual, see \secref{sec:include}.
However, the document body must make a distinction
between processing of an individual part and of the main document, e.g.:
%
\begin{center}
\begin{tabular}{l}
|\ifchilddocmanual|\\
|\input{\childdocname}|\\
|\||else|\\
\textit{document body with }|\input{|\textit{part}|}|\\
|\||fi|
\end{tabular}
\end{center}
%
The conditional |\ifchilddocmanual| is true whenever
a part to be included by |\input| is being compiled,
and the name of the part is stored in |\childdocname|.

%%%%%%%%%%%%%%%%%%%%%%%%%%%%%%%%%%%%%%%%
\DescribeMacro{\childdocby}
Each part to be included by |\input| should start with:
%
\begin{center}
\begin{tabular}{l}
|\input{childdoc.def}|\\
|\childdocby{|\textit{main}|}|\\
\end{tabular}
\end{center}
%
The directive |\childdocby| is similar to |\childdocof|
described in \secref{sec:include},
but the subsequent selection of content must be done manually.
To that end, both |\ifchilddoc| and |\ifchilddocmanual|
will be true upon processing of a part,
and the name of the part is stored in |\childdocname|.
Note that |\jobname| will be set to the filename of the current part
so that each part receives an individual |.aux| file
that does not interfere with the |.aux| file(s) of the main document.
This behaviour can be altered by the alternative form
|\childdocby[*]{|\textit{main}|}| (with a non-empty optional argument)
which uses the |.aux| file of the main document
by setting |\jobname| to \textit{main}.

%%%%%%%%%%%%%%%%%%%%%%%%%%%%%%%%%%%%%%%%%%%%%%%%%%%%%%%%%%%%%%%%%%%%%%%%%%%%%%%%
\subsection{Driver Development}
\label{sec:driver}

The \textsf{childdoc} mechanism can also be use for the development
of definition files such as \LaTeX{} styles or classes.
This case differs from the above setup with multiple parts
included by |\include| in that no |\includeonly| should be invoked.
This can be achieved by starting the include file
(before |\ProvidesPackage|) with:
%
\begin{center}
\begin{tabular}{l}
|\input{childdoc.def}|\\
|\childdocforward{|\textit{main}|}|\\
\end{tabular}
\end{center}
%
or alternatively with:
%
\begin{center}
\begin{tabular}{l}
|\input{childdoc.def}|\\
|\childdocby{|\textit{main}|}|\\
\end{tabular}
\end{center}
%
Both forms have slightly different effects as described above.
The main file is prepared as usual, see \secref{sec:include}.

%%%%%%%%%%%%%%%%%%%%%%%%%%%%%%%%%%%%%%%%%%%%%%%%%%%%%%%%%%%%%%%%%%%%%%%%%%%%%%%%
\subsection{Legacy Detection}
\label{sec:detection}

The directive |\childdocmain| in the main file can detect
whether the complete document or merely a child is to be compiled
even without using the directive |\childdocof|.
This method is deprecated because it is less robust
and there is no compelling reason to use it;
it is merely provided for backward compatibility
and it may be removed in future versions.

If the detection mechanism is to be used,
it is mandatory to correctly specify
the filename of the main file as the argument of |\childdocmain|:
%
\begin{center}
\begin{tabular}{l}
|\input{childdoc.def}|\\
|\childdocmain{|\textit{main}|}|\\
\end{tabular}
\end{center}
%
If |\jobname| does not match the argument \textit{main} of |\childdocmain|,
it is assumed that |\jobname| points to the child file to be compiled.
When using |\childdocmain| with the main file specified as argument,
it suffices to start a child file
with just |\input{|\textit{main}|}|
without loading of the package and using |\childdocof|.
If instead all processing is done
with the appropriate \textsf{childdoc} directives,
the argument of \textit{main} of |\childdocmain| can be empty.

An alternative version of the command line processing described
in \secref{sec:commandline} using the detection mechanism reads:
%
\begin{center}
|... -jobname "|\textit{target}|" "|[\textit{flags}]%
[|\def\jobname{|\textit{dest}|}|]|\input{|\textit{main}|}"|
\end{center}

%%%%%%%%%%%%%%%%%%%%%%%%%%%%%%%%%%%%%%%%%%%%%%%%%%%%%%%%%%%%%%%%%%%%%%%%%%%%%%%%
\subsection{Manual Code}
\label{sec:manual}

In case one cannot be certain whether the definitions file |childdoc.def|
is installed on the target \TeX{} distribution
and one prefers not to ship it,
it is conceivable to paste a few relevant commands into the sources.

To that end, drop all statements |\input{childdoc.def}|
and perform the replacements as outlined below.
Instead of |\childdocmain{|\textit{main}|}| add the following code
to the top of the main file:
%
\begin{center}
\begin{tabular}{l}
|\||ifdefined\childdocname\endinput\||fi\newif\ifchilddoc|\\
|\edef\childdocname{\scantokens\expandafter{\jobname\noexpand}}|\\
|\def\childdocmain{|\textit{main}|}\||ifx\childdocmain\childdocname\||else|\\
|\childdoctrue\includeonly{\childdocname}\let\jobname\childdocmain\||fi|\\
\end{tabular}
\end{center}
%
Instead of |\childdocof{|\textit{main}|}| just include the main file
at the top of each child file:
%
\begin{center}
|\input{|\textit{main}|}|
\end{center}
%
A simple redirection |\childdocforward{|\textit{dest}|}| is achieved by:
%
\begin{center}
|\def\jobname{|\textit{dest}|}\input{\jobname}|
\end{center}
%
The redirection with prefix
|\childdocforwardprefix[|\textit{prefix}|]{|\textit{dest}|}|
is accomplished by:
%
\begin{center}
\begin{tabular}{l}
|{\edef\jobname{\scantokens\expandafter{\jobname\noexpand}}|\\
|\def\redirectjob |\textit{prefix}|#1~~~{\gdef\jobname{|\textit{dest}|#1}}|\\
|\expandafter\redirectjob\jobname~~~}\input{\jobname}|
\end{tabular}
\end{center}

In an alternative approach,
child documents can be compiled by a specific command line
without additional code or specific definitions:
%
\begin{center}
|... -jobname "|\textit{target}|" "|[\textit{flags}]%
|\includeonly{|\textit{dest}|}\input{|\textit{main}|}"|
\end{center}
%

%%%%%%%%%%%%%%%%%%%%%%%%%%%%%%%%%%%%%%%%%%%%%%%%%%%%%%%%%%%%%%%%%%%%%%%%%%%%%%%%
%%%%%%%%%%%%%%%%%%%%%%%%%%%%%%%%%%%%%%%%%%%%%%%%%%%%%%%%%%%%%%%%%%%%%%%%%%%%%%%%
\section{Information}

%%%%%%%%%%%%%%%%%%%%%%%%%%%%%%%%%%%%%%%%%%%%%%%%%%%%%%%%%%%%%%%%%%%%%%%%%%%%%%%%
\subsection{Copyright}

Copyright \copyright{} 2017--2018 Niklas Beisert

This work may be distributed and/or modified under the
conditions of the \LaTeX{} Project Public License, either version 1.3
of this license or (at your option) any later version.
The latest version of this license is in
  \url{http://www.latex-project.org/lppl.txt}
and version 1.3 or later is part of all distributions of \LaTeX{}
version 2005/12/01 or later.

This work has the LPPL maintenance status `maintained'.

The Current Maintainer of this work is Niklas Beisert.

This work consists of the files |README.txt|, |childdoc.ins| and |childdoc.dtx|
as well as the derived files |childdoc.def|, |cdocsamp.tex|
with |cdocsch1.tex|, |cdocsch2.tex|, |cdocspt3.tex|, |cdocspt4.tex|,
|cdocsdrf.tex|, |cdocsfn1.tex|, |cdocsfn2.tex|
as well as |childdoc.pdf|.

%%%%%%%%%%%%%%%%%%%%%%%%%%%%%%%%%%%%%%%%%%%%%%%%%%%%%%%%%%%%%%%%%%%%%%%%%%%%%%%%
\subsection{Files and Installation}

The package consists of the files:
%
\begin{center}
\begin{tabular}{ll}
    |README.txt|   & readme file \\
    |childdoc.ins| & installation file \\
    |childdoc.dtx| & source file \\
    |childdoc.def| & definition file \\
    |cdocsamp.tex| & sample main file \\
    |cdocsch1.tex| & sample include file \\
    |cdocsch2.tex| & sample include file \\
    |cdocspt3.tex| & sample part file \\
    |cdocspt4.tex| & sample part file \\
    |cdocsdrf.tex| & sample redirection file \\
    |cdocsfn1.tex| & sample redirection file \\
    |cdocsfn2.tex| & sample redirection file \\
    |childdoc.pdf| & manual
\end{tabular}
\end{center}
%
The distribution consists of the files
|README.txt|, |childdoc.ins| and |childdoc.dtx|.
%
\begin{itemize}
\item
Run (pdf)\LaTeX{} on |childdoc.dtx|
to compile the manual |childdoc.pdf| (this file).
\item
Run \LaTeX{} on |childdoc.ins| to create the definitions file |childdoc.def|
and the sample |cdocsamp.tex| with include files
|cdocsch1.tex|, |cdocsch2.tex|, |cdocspt3.tex|, |cdocspt4.tex|,
|cdocsdrf.tex|, |cdocsfn1.tex|, |cdocsfn2.tex|.
Then copy the file |childdoc.def| to an appropriate directory of your \LaTeX{}
distribution, e.g.\ \textit{texmf-root}|/tex/latex/childdoc|.
\end{itemize}

%%%%%%%%%%%%%%%%%%%%%%%%%%%%%%%%%%%%%%%%%%%%%%%%%%%%%%%%%%%%%%%%%%%%%%%%%%%%%%%%
\subsection{Related CTAN Packages}

There are several other packages which offer a similar functionality:
%
\begin{itemize}
\item
The packages
\href{http://ctan.org/pkg/docmute}{\textsf{docmute}},
\href{http://ctan.org/pkg/includex}{\textsf{includex}} and
\href{http://ctan.org/pkg/standalone}{\textsf{standalone}}
provide commands to include only the document body of
a child file thus allowing both files to be compiled individually.
\item
The packages \href{http://ctan.org/pkg/subdocs}{\textsf{subdocs}}
and \href{http://ctan.org/pkg/subfiles}{\textsf{subfiles}}
provide structures in which the main and child documents can be
encapsulated and allowing them to be compiled individually.
The inclusion mechanism is different from the conventional |\include|.
\item
The package \href{http://ctan.org/pkg/combine}{\textsf{combine}}
is an elaborate solution to combine several documents into one.
\end{itemize}
%
See also the CTAN topic \href{http://ctan.org/topic/subdocs}{\textsf{subdocs}}
for further related packages.
The present package differs from the above solutions in that
a document structure constructed with the conventional |\include| mechanism
just needs two extra commands at the top of every file
such that all constituent files can be compiled individually.

%%%%%%%%%%%%%%%%%%%%%%%%%%%%%%%%%%%%%%%%%%%%%%%%%%%%%%%%%%%%%%%%%%%%%%%%%%%%%%%%
%\subsection{Feature Suggestions}
%
%The following is a list of features which may be useful for future
%versions of this package:
%%
%\begin{itemize}
%\item
%\ldots
%\end{itemize}

%%%%%%%%%%%%%%%%%%%%%%%%%%%%%%%%%%%%%%%%%%%%%%%%%%%%%%%%%%%%%%%%%%%%%%%%%%%%%%%%
\subsection{Revision History}

%%%%%%%%%%%%%%%%%%%%%%%%%%%%%%%%%%%%%%%%
\paragraph{v2.0:} 2018/12/30

\begin{itemize}
\item
immediate forward processing
\item
added |\childdocby| mechanism
\item
manual restructured
\end{itemize}

%%%%%%%%%%%%%%%%%%%%%%%%%%%%%%%%%%%%%%%%
\paragraph{v1.6:} 2018/01/17

\begin{itemize}
\item
application for development of include files
\item
corrections to manual
\end{itemize}

%%%%%%%%%%%%%%%%%%%%%%%%%%%%%%%%%%%%%%%%
\paragraph{v1.5:} 2017/05/21

\begin{itemize}
\item
more complete structuring introduced
\item
|\childdocof| introduced
\item
|\childdoc| renamed to |\childdocmain|
\item
|\childredirect| renamed to |\childdocforward| and |\childdocforwardprefix|
and functionality expanded
\end{itemize}

%%%%%%%%%%%%%%%%%%%%%%%%%%%%%%%%%%%%%%%%
\paragraph{v1.0:} 2017/04/27

\begin{itemize}
\item
manual and install package
\item
first version published on CTAN
\end{itemize}

%%%%%%%%%%%%%%%%%%%%%%%%%%%%%%%%%%%%%%%%
\paragraph{v0.6:} 2017/04/26

\begin{itemize}
\item
redirection mechanism added
\end{itemize}

%%%%%%%%%%%%%%%%%%%%%%%%%%%%%%%%%%%%%%%%
\paragraph{v0.5:} 2017/04/26

\begin{itemize}
\item
functionality in definition file
\end{itemize}


%%%%%%%%%%%%%%%%%%%%%%%%%%%%%%%%%%%%%%%%%%%%%%%%%%%%%%%%%%%%%%%%%%%%%%%%%%%%%%%%
%%%%%%%%%%%%%%%%%%%%%%%%%%%%%%%%%%%%%%%%%%%%%%%%%%%%%%%%%%%%%%%%%%%%%%%%%%%%%%%%
%%%%%%%%%%%%%%%%%%%%%%%%%%%%%%%%%%%%%%%%%%%%%%%%%%%%%%%%%%%%%%%%%%%%%%%%%%%%%%%%
\appendix

\settowidth\MacroIndent{\rmfamily\scriptsize 000\ }

 \DocInput{childdoc.dtx}

\end{document}
%</driver>
% \fi
%
% %%%%%%%%%%%%%%%%%%%%%%%%%%%%%%%%%%%%%%%%%%%%%%%%%%%%%%%%%%%%%%%%%%%%%%%%%%%%%%
% %%%%%%%%%%%%%%%%%%%%%%%%%%%%%%%%%%%%%%%%%%%%%%%%%%%%%%%%%%%%%%%%%%%%%%%%%%%%%%
% \section{Sample}
%\iffalse
%<*samplemain>
%\fi
%
% The following presents a sample document
% with two chapters, two parts, a title page,
% a compile flag as well as three forwarding files to set the flag.
% It consists of eight |.tex| files:
% \begin{center}
% \begin{tabular}{ll}
% |cdocsamp.tex|&main file\\
% |cdocsch1.tex|&include file for chapter 1\\
% |cdocsch2.tex|&include file for chapter 2\\
% |cdocspt3.tex|&include file for part 3\\
% |cdocspt4.tex|&include file for part 4\\
% |cdocsdrf.tex|&forwarding file for main file in draft mode\\
% |cdocsfi1.tex|&forwarding file for final version of chapter 1\\
% |cdocsfi2.tex|&forwarding file for final version of chapter 2\\
% \end{tabular}
% \end{center}
% Each of the eight files can be compiled directly by the \LaTeX{} compiler.
%
% %%%%%%%%%%%%%%%%%%%%%%%%%%%%%%%%%%%%%%
% \paragraph{Main File.}
%
% The main file is called |cdocsamp.tex|.
%
% Load the \textsf{childdoc} definitions and
% declare the filename for the main document:
%    \begin{macrocode}
\input{childdoc.def}
\childdocmain{}
%    \end{macrocode}

% Optional override for |\version| flag:
%    \begin{macrocode}
%%\ifchilddoc\else\providecommand{\version}{draft}\fi
%    \end{macrocode}

% Define the default values for the |\version| flag
% (|final| for the main file and |draft| for childs):
%    \begin{macrocode}
\ifchilddoc
\providecommand{\version}{draft}
\else
\providecommand{\version}{final}
\fi
%    \end{macrocode}

% Load the standard document class:
%    \begin{macrocode}
\documentclass[12pt]{article}
%    \end{macrocode}

% Start the document body:
%    \begin{macrocode}
\begin{document}
%    \end{macrocode}

% Declare a title page.
% Print title, part of document being processed and version flag:
%    \begin{macrocode}
\addtocounter{page}{-1}
\begin{center}
{\LARGE\bfseries{}childdoc example\par}
\vspace{1cm}
\ifchilddoc
\ifchilddocmanual part\else chapter\fi:
`\childdocname' of `\childdocjob'\par
\else
main document: `\childdocjob'\par
\fi
version: \version\par
\end{center}
\newpage
%    \end{macrocode}

% Manually include selected file,
% otherwise process as usual:
%    \begin{macrocode}
\ifchilddocmanual
\section*{part `\childdocname'}
\input{\childdocname}
\else
%    \end{macrocode}

% Include the two chapters:
%    \begin{macrocode}
\include{cdocsch1}
\include{cdocsch2}
%    \end{macrocode}

% Include the two parts unless only chapters should be displayed:
%    \begin{macrocode}
\ifchilddoc\else
\section{part three}
\input{cdocspt3}
\section{part four}
\input{cdocspt4}
\fi
%    \end{macrocode}

% Process as usual until here:
%    \begin{macrocode}
\fi
%    \end{macrocode}

% End of document body:
%    \begin{macrocode}
\end{document}
%    \end{macrocode}
%\iffalse
%</samplemain>
%\fi
%
% %%%%%%%%%%%%%%%%%%%%%%%%%%%%%%%%%%%%%%
% \paragraph{Chapter Include Files.}
%
% The include files are called |cdocsch1.tex| and |cdocsch2.tex|.
%
%\iffalse
%<*samplechap1|samplechap2>
%\fi

% Optional override for |\version| flag:
%    \begin{macrocode}
%%\providecommand{\version}{final}
%    \end{macrocode}

% Include the main document:
%    \begin{macrocode}
\input{childdoc.def}
\childdocof{cdocsamp}
%    \end{macrocode}

%\iffalse
%</samplechap1|samplechap2>
%\fi
%
%\iffalse
%<*samplechap1>
%\fi
% Some text for chapter 1:
%    \begin{macrocode}
\section{one}
some text in chapter one
%    \end{macrocode}

%\iffalse
%</samplechap1>
%\fi
% Some text for chapter 2:
%\iffalse
%<*samplechap2>
%\fi
%    \begin{macrocode}
\section{two}
more text in chapter two
%    \end{macrocode}

%\iffalse
%</samplechap2>
%\fi
%
% %%%%%%%%%%%%%%%%%%%%%%%%%%%%%%%%%%%%%%
% \paragraph{Part Include Files.}
%
% The include files are called |cdocspt3.tex| and |cdocspt4.tex|.
%
%\iffalse
%<*samplepart3|samplepart4>
%\fi

% Optional override for |\version| flag:
%    \begin{macrocode}
%%\providecommand{\version}{final}
%    \end{macrocode}

% Include the main document:
%    \begin{macrocode}
\input{childdoc.def}
\childdocby{cdocsamp}
%    \end{macrocode}

%\iffalse
%</samplepart3|samplepart4>
%\fi
%
%\iffalse
%<*samplepart3>
%\fi
% Some text for part 3:
%    \begin{macrocode}
some text in part three
%    \end{macrocode}

%\iffalse
%</samplepart3>
%\fi
% Some text for part 4:
%\iffalse
%<*samplepart4>
%\fi
%    \begin{macrocode}
more text in part four
%    \end{macrocode}

%\iffalse
%</samplepart4>
%\fi
%
% %%%%%%%%%%%%%%%%%%%%%%%%%%%%%%%%%%%%%%
% \paragraph{Forwarding for a Complete Draft.}
%
% The following forwarding file |cdocsdrf.tex|
% compiles the main document in draft mode:
%\iffalse
%<*sampledraft>
%\fi
%    \begin{macrocode}
\def\version{draft}
\input{childdoc.def}
\childdocforward{cdocsamp}
%    \end{macrocode}

%\iffalse
%</sampledraft>
%\fi
%
% %%%%%%%%%%%%%%%%%%%%%%%%%%%%%%%%%%%%%%
% \paragraph{Forwarding for Final Version of the Chapters.}
%
% The following forwarding files |cdocsfn1.tex| and |cdocsfn2.tex|
% (with identical content)
% compile the final versions of the child documents
% |cdocsch1.tex| and |cdocsch2.tex|, respectively:
%\iffalse
%<*samplefinal>
%\fi
%    \begin{macrocode}
\def\version{final}
\input{childdoc.def}
\childdocforwardprefix[cdocsamp]{cdocsfn}{cdocsch}
%    \end{macrocode}

%\iffalse
%</samplefinal>
%\fi
%
% %%%%%%%%%%%%%%%%%%%%%%%%%%%%%%%%%%%%%%
% \paragraph{Command Line Processing.}
%
% The following three command lines generate the output files
% |cdocscld|, |cdocscl1| and |cdocscl2|
% which should be identical to
% |cdocsdrf|, |cdocsch1| and |cdocsfn2|, respectively:
% \begin{center}
% \begin{tabular}{l}
% |latex -jobname cdocscld \|\\
% |  "\def\version{draft}\input{childdoc.def}\childdocforward{cdocsamp}"|\\
% |latex -jobname cdocscl1 \|\\
% |  "\input{childdoc.def}\childdocforward[cdocsamp]{cdocsch1}"|\\
% |latex -jobname cdocscl2 \|\\
% |  "\def\version{final}\input{childdoc.def}\childdocforward{cdocsch2}"|
% \end{tabular}
% \end{center}
% Note that the trailing backslash on each first line
% merely continues the input to the second line
% (for convenient cut ant paste).
% Furthermore, the command |latex| can be replaced by any
% of its alternative versions such as |pdflatex|.
%
% %%%%%%%%%%%%%%%%%%%%%%%%%%%%%%%%%%%%%%%%%%%%%%%%%%%%%%%%%%%%%%%%%%%%%%%%%%%%%%
% %%%%%%%%%%%%%%%%%%%%%%%%%%%%%%%%%%%%%%%%%%%%%%%%%%%%%%%%%%%%%%%%%%%%%%%%%%%%%%
% \section{Implementation}
%\iffalse
%<*package>
%\fi
%
% This section describes the definitions file |childdoc.def|.

% The definitions cannot be loaded using |\usepackage| or |\RequirePackage|
% which has a mechanism to prevent loading a style file more than once.
% When loading the definitions by means of |\input|
% multiple instances have to be prevented manually:
%\iffalse
%This code needs to be before the `\ProvidesFile' directive
%which is defined at the beginning of this file.
%Therefore it is also placed there and commented out here.
%</package>
%<*discard>
%\fi
%    \begin{macrocode}
\ifdefined\childdocmain\endinput\fi
%    \end{macrocode}
%\iffalse
%</discard>
%<*package>
%\fi
%
% \macro{\ifchilddoc}
% \macro{\ifchilddocmanual}
% The conditional |\ifchilddoc| tells whether a
% child (true) or main (false) document is being compiled.
% The conditional |\ifchilddocmanual| tells whether
% the |\includeonly| mechanism is used (false) or
% the selection of child files must be performed manually (true).
% The definitions initialise to false:
%    \begin{macrocode}
\newif\ifchilddoc
\newif\ifchilddocmanual
%    \end{macrocode}

% \macro{\childdocname}
% \macro{\childdocjob}
% The macro |\childdocname| stores the name of the main document
% to be compiled. The macro |\childdocjob| stores the name of
% the document on which the \LaTeX{} compiler was originally invoked.
% The content of |\jobname| cannot be compared
% to filenames specified in the source due to different catcodes.
% The following code rescans |\jobname|, stores the result
% in |\childdocname| and saves a copy in |\childdocjob|:
%    \begin{macrocode}
\edef\childdocname{\scantokens\expandafter{\jobname\noexpand}}
\let\childdocjob\childdocname
%    \end{macrocode}

% \macro{\childdocdisable}
% The macro |\childdocdisable| prevents the main file
% from being processed more than once.
% At this stage, the main document command |\childdocmain|
% is assumed to be called once again where it should do nothing.
% Any subsequent call to it should prevent
% a secondary processing of the main document
% It overwrites the forwarding commands
% |\childdocof| and |\childdocforward|
% with empty macros to prevent further inclusions of the main document:
%    \begin{macrocode}
\newcommand{\childdocdisable}
{
  \renewcommand{\childdocmain}[1]{\renewcommand{\childdocmain}[1]{\endinput}}
  \renewcommand{\childdocof}[1]{}
  \renewcommand{\childdocby}[2][]{}
  \renewcommand{\childdocforward}[2][]{}
  \renewcommand{\childdocdisable}{}
}
%    \end{macrocode}

% \macro{\childdocmain}
% The macro |\childdocmain| is to be called at the top of the main file
% with nothing or the main filename (without extension) as argument.
% First, it breaks loops.
% If the argument is not empty and does not match |\childdocname|
% (which is set by the first inclusion of |childdoc.def|),
% |\ifchilddoc| is set to true, |\includeonly| is applied to the child file
% and |\jobname| is set to the main file
% (for proper handling of |.aux| files):
%    \begin{macrocode}
\newcommand{\childdocmain}[1]
{
  \childdocdisable\childdocmain{}
  \if?#1?\else
    \begingroup
      \def\childdoctmp{#1}
      \ifx\childdoctmp\childdocname
        \def\childdoctmp{}
      \else
        \def\childdoctmp
        {
          \childdoctrue
          \includeonly{\childdocname}
          \def\childdocjob{#1}
          \def\jobname{#1}
        }
      \fi
      \expandafter
    \endgroup
    \childdoctmp
  \fi
}
%    \end{macrocode}

% \macro{\childdocof}
% The command |\childdocof| redirects
% compilation to the main file |#1|.
%    \begin{macrocode}
\newcommand{\childdocof}[1]
{
  \childdocdisable
  \childdoctrue
  \includeonly{\childdocname}
  \def\jobname{#1}
  \def\childdocjob{#1}
  \input{#1}
}
%    \end{macrocode}

% \macro{\childdocby}
% The command |\childdocby| ....
%    \begin{macrocode}
\newcommand{\childdocby}[2][]
{
  \childdocdisable
  \childdoctrue
  \childdocmanualtrue
  \if?#1?\else
    \def\jobname{#2}
  \fi
  \def\childdocjob{#2}
  \input{#2}
  \endinput
}
%    \end{macrocode}

% \macro{\childdocforward}
% The command |\childdocforward| redirects
% compilation to the main file or
% (if the optional argument is given) a child file.
% Parameters are set as if the main file
% or a child file starting with |\childdocof| was compiled.
% Then compilation is handed over to the main file:
%    \begin{macrocode}
\newcommand{\childdocforward}[2][]
{
  \begingroup
    \if?#1?
      \def\childdoctmp
      {
        \def\childdocname{#2}
        \def\childdocjob{#2}
        \def\jobname{#2}
        \input{#2}
        \endinput
      }
    \else
      \def\childdoctmp
      {
        \childdocdisable
        \def\childdocname{#2}
        \childdoctrue
        \includeonly{#2}
        \def\childdocjob{#1}
        \def\jobname{#1}
        \input{#1}
        \endinput
      }
    \fi
    \expandafter
  \endgroup
  \childdoctmp
}
%    \end{macrocode}

% \macro{\childdocforwardprefix}
% The command |\childdocforwardprefix| redirects
% compilation to the main or a child file by means of a pattern.
% The prefix |#1| in the current filename is replaced by |#2|
% and the suffix of the current filename is kept
% (it is assumed that the filename does not contain the substring `|~~~|'
% which is used as a delimiter).
% Compilation is handed over to the new file by |\childdocforward|:
%    \begin{macrocode}
\newcommand{\childdocforwardprefix}[3][]
{
  \begingroup
    \def\childdocextract #2##1~~~{\def\childdoctmp{\childdocforward[#1]{#3##1}}}
    \expandafter\childdocextract\childdocname~~~
    \expandafter
  \endgroup
  \childdoctmp
}
%    \end{macrocode}

% \macro{\childdoc}
% The deprecated macro |\childdoc| is a legacy version of |\childdocmain|:
%    \begin{macrocode}
\newcommand{\childdoc}{\childdocmain}
%    \end{macrocode}

% \macro{\childdocredirect}
% The deprecated macro |\childdocredirect| is a legacy version
% of |\childdocforward| and |\childdocforwardprefix|:
%    \begin{macrocode}
\newcommand{\childdocredirect}[2][]
{
  \begingroup
    \if?#1?
      \def\childdoctmp{\childdocforward{#2}}
    \else
      \def\childdoctmp{\childdocforwardprefix{#1}{#2}}
    \fi
    \expandafter
  \endgroup
  \childdoctmp
}
%    \end{macrocode}

%\iffalse
%</package>
%\fi
%
\endinput
\childdocforward[|\textit{main}|]{|\textit{dest}|}"|
\end{center}
%
Here \textit{target} is the name of the output file,
\textit{main} is the name of the main file
and \textit{dest} is the name of the main or child file to be processed
(all filenames without extensions).
The optional argument \textit{main} can be omitted
if \textit{main} matches \textit{dest}.
Optionally, compilation \textit{flags} can be defined via |\def| commands.
This command line makes the \TeX{} engine believe
it is compiling the file \textit{target}
whose content is specified as the latter parameter.
The provided code then forwards the processing to
\textit{main} or \textit{dest} as described in \secref{sec:forward}.

%%%%%%%%%%%%%%%%%%%%%%%%%%%%%%%%%%%%%%%%%%%%%%%%%%%%%%%%%%%%%%%%%%%%%%%%%%%%%%%%
\subsection{Include by Input}
\label{sec:input}

Including child documents by |\include| has some restrictions by design.
Most notably, the content of a child document always occupies
its own set of pages; pages cannot be shared between child documents.
Usually, this behaviour makes perfect sense
because each child document contain an essential part of the document.
However, in some situations it may be desirable to compose
a document from a collection of parts
without having mandatory page breaks between then.
For this case, the package
provides a mechanism to include parts
by |\input| which can also be processed individually.
However, by construction this mechanism
requires manual handling of the content to be output.

%%%%%%%%%%%%%%%%%%%%%%%%%%%%%%%%%%%%%%%%
\DescribeMacro{\ifchilddocmanual}
The main file should be prepared as usual, see \secref{sec:include}.
However, the document body must make a distinction
between processing of an individual part and of the main document, e.g.:
%
\begin{center}
\begin{tabular}{l}
|\ifchilddocmanual|\\
|\input{\childdocname}|\\
|\||else|\\
\textit{document body with }|\input{|\textit{part}|}|\\
|\||fi|
\end{tabular}
\end{center}
%
The conditional |\ifchilddocmanual| is true whenever
a part to be included by |\input| is being compiled,
and the name of the part is stored in |\childdocname|.

%%%%%%%%%%%%%%%%%%%%%%%%%%%%%%%%%%%%%%%%
\DescribeMacro{\childdocby}
Each part to be included by |\input| should start with:
%
\begin{center}
\begin{tabular}{l}
|% \iffalse
%
% childdoc.dtx Copyright (C) 2017-2018 Niklas Beisert
%
% This work may be distributed and/or modified under the
% conditions of the LaTeX Project Public License, either version 1.3
% of this license or (at your option) any later version.
% The latest version of this license is in
%   http://www.latex-project.org/lppl.txt
% and version 1.3 or later is part of all distributions of LaTeX
% version 2005/12/01 or later.
%
% This work has the LPPL maintenance status `maintained'.
%
% The Current Maintainer of this work is Niklas Beisert.
%
% This work consists of the files childdoc.dtx and childdoc.ins
% and the derived files childdoc.def and cdocsamp.tex with
% cdocsch1.tex, cdocsch2.tex, cdocsdrf.tex, cdocsfn1.tex, cdocsfn2.tex.
%
%<package>\ifdefined\childdocmain\endinput\fi
%<package>\ProvidesFile{childdoc.def}[2018/12/30 v2.0 child document driver]
%<samplemain>\ProvidesFile{cdocsamp.tex}[2018/12/30 v2.0 sample for childdoc]
%<*driver>
%\ProvidesFile{childdoc.drv}[2018/12/30 v2.0 childdoc reference manual file]
\PassOptionsToClass{10pt,a4paper}{article}
\documentclass{ltxdoc}

\usepackage[margin=35mm]{geometry}
\usepackage{hyperref}
\usepackage{hyperxmp}
\usepackage[usenames]{color}

\hypersetup{colorlinks=true}
\hypersetup{pdfstartview=FitH}
\hypersetup{pdfpagemode=UseNone}
\hypersetup{pdfsource={}}
\hypersetup{pdflang={en-UK}}
\hypersetup{pdfcopyright={Copyright 2017-2018 Niklas Beisert.
  This work may be distributed and/or modified under the
  conditions of the LaTeX Project Public License, either version 1.3
  of this license or (at your option) any later version.}}
\hypersetup{pdflicenseurl={http://www.latex-project.org/lppl.txt}}
\hypersetup{pdfcontactaddress={ETH Zurich, ITP, HIT K,
  Wolfgang-Pauli-Strasse 27}}
\hypersetup{pdfcontactpostcode={8093}}
\hypersetup{pdfcontactcity={Zurich}}
\hypersetup{pdfcontactcountry={Switzerland}}
\hypersetup{pdfcontactemail={nbeisert@itp.phys.ethz.ch}}
\hypersetup{pdfcontacturl={http://people.phys.ethz.ch/\xmptilde nbeisert/}}

\newcommand{\secref}[1]{\hyperref[#1]{section \ref*{#1}}}

\parskip1ex
\parindent0pt
\let\olditemize\itemize
\def\itemize{\olditemize\parskip0pt}

\begin{document}

\title{The \textsf{childdoc} Package}
\hypersetup{pdftitle={The childdoc Package}}
\author{Niklas Beisert\\[2ex]
  Institut f\"ur Theoretische Physik\\
  Eidgen\"ossische Technische Hochschule Z\"urich\\
  Wolfgang-Pauli-Strasse 27, 8093 Z\"urich, Switzerland\\[1ex]
  \href{mailto:nbeisert@itp.phys.ethz.ch}
  {\texttt{nbeisert@itp.phys.ethz.ch}}}
\hypersetup{pdfauthor={Niklas Beisert}}
\hypersetup{pdfsubject={Manual for the LaTeX2e Package childdoc}}
\date{30 December 2018, \textsf{v2.0}}
\maketitle

\begin{abstract}\noindent
\textsf{childdoc} is a \LaTeXe{} package
that enables the direct compilation
of document sections included by |\include|
to individual files.
\end{abstract}

\begingroup
\parskip0ex
\tableofcontents
\endgroup

%%%%%%%%%%%%%%%%%%%%%%%%%%%%%%%%%%%%%%%%%%%%%%%%%%%%%%%%%%%%%%%%%%%%%%%%%%%%%%%%
%%%%%%%%%%%%%%%%%%%%%%%%%%%%%%%%%%%%%%%%%%%%%%%%%%%%%%%%%%%%%%%%%%%%%%%%%%%%%%%%
\section{Introduction}

\LaTeX{} provides a mechanism to structure a large document (such as a book)
into a main file and several child files (containing the chapters)
using the |\include| command.
This mechanism is beneficial for documents
which span hundreds of pages in order to
make the source file(s) more manageable.
Moreover, compilation can be restricted to
selected child files by means of the |\includeonly| command.
The latter feature can be used to reduce the compilation time while editing
(this was significantly more useful in the earlier days of \LaTeX{})
or to generate a smaller document which is easier to navigate.
Another application of |\includeonly| is to generate
documents consisting of selected parts of the complete document.

However, there are a few drawbacks of the plain |\include| mechanism:
\begin{itemize}
\item
The child files cannot be compiled on their own,
they can only be compiled via the main file.
A naive editing environment
(such as a text editor with an option
to have the current file processed by \LaTeX)
may require one to switch to the main file before compiling;
attempting to compile the child file produces errors.
\item
The main file must be modified (each time)
to adjust the |\includeonly| command
to the present needs. This easily leaves the main file in a messy state.
\item
The generated document will always carry the filename
of the main document. This is inconvenient if
several child files are to be compiled and
to be kept for distribution.
\end{itemize}

The present package provides a simple interface
to make child files individually compilable by \LaTeX{}.
Compiling a child file then has the same effect as compiling
the main file with an |\includeonly| command
to select the appropriate child.
Moreover the generated document will carry the name of the child
rather than the main file.
This resolves all three above issues.

This feature is meant to make the editing of books,
thesis documents and lecture notes somewhat more convenient.
However, the package can also be used efficiently for
composing a series of documents (such as exercise sheets)
which are typically distributed individually.
It then assists the author in generating the individual documents
(potentially in different versions)
as well as a document containing the collected series.
Another application is in developing style files
or other kinds of included material
where compilation of the style file could redirect
to a sample or test file.

%%%%%%%%%%%%%%%%%%%%%%%%%%%%%%%%%%%%%%%%%%%%%%%%%%%%%%%%%%%%%%%%%%%%%%%%%%%%%%%%
%%%%%%%%%%%%%%%%%%%%%%%%%%%%%%%%%%%%%%%%%%%%%%%%%%%%%%%%%%%%%%%%%%%%%%%%%%%%%%%%
\section{Usage}

First of all, the package \textsf{childdoc} is \emph{not} a standard
\LaTeXe{} |.sty| style file! Therefore it needs to be invoked in
a non-standard way.

%%%%%%%%%%%%%%%%%%%%%%%%%%%%%%%%%%%%%%%%%%%%%%%%%%%%%%%%%%%%%%%%%%%%%%%%%%%%%%%%
\subsection{Included Files}
\label{sec:include}

%%%%%%%%%%%%%%%%%%%%%%%%%%%%%%%%%%%%%%%%
\DescribeMacro{\childdocmain}
To use the package, add the commands
\begin{center}
\begin{tabular}{l}
|\input{childdoc.def}|\\
|\childdocmain{}|\\
\end{tabular}
\end{center}
at the very top of the main \LaTeX{} file,
in particular \emph{before} the |\documentclass| statement!
The argument of |\childdocmain| should be left empty
(but it must be present).

%%%%%%%%%%%%%%%%%%%%%%%%%%%%%%%%%%%%%%%%
\DescribeMacro{\childdocof}
Furthermore, add the commands
\begin{center}
\begin{tabular}{l}
|\input{childdoc.def}|\\
|\childdocof{|\textit{main}|}|\\
\end{tabular}
\end{center}
at the top of every child file \textit{child}
which is included by |\include{|\textit{child}|}|
from within the main file
(or at least for those files to be compiled individually).
The argument \textit{main} must be the filename of the main file.

There are a couple of
considerations in setting up the main and child documents:

%%%%%%%%%%%%%%%%%%%%%%%%%%%%%%%%%%%%%%%%
\paragraph{Restrictions.}

Please note the following restrictions:
\begin{itemize}
\item
|\childdocmain| must be called with one argument \textit{main}
to ensure compatibility with earlier version of the package.
It must either be empty (|\childdocmain{}|)
or precisely match the filename of the main file in which it is specified.
See \secref{sec:detection} for further information.
\item
The filename \textit{main} must be specified without the |.tex| extension.
\item
The filename \textit{main} is case sensitive
(even in case-insensitive file systems)
due to internal string comparison.
\item
The argument \textit{main} should be fully expanded, it cannot be a macro.
\item
Subdirectories and special characters should be avoided in filenames.
\item
The command |\childdocmain{|\textit{main}|}| must be followed by a whitespace.
It should not be followed immediately by another command
or by a comment mark `|%|'.
This is because the \TeX{} parser reads the token immediately following
the argument of |\childdocmain| and puts it
at the beginning of every child section;
however, a white\-space is ignored.
\end{itemize}

%%%%%%%%%%%%%%%%%%%%%%%%%%%%%%%%%%%%%%%%
\paragraph{Content of Main File.}

It is advisable to place all content in the child files included by |\include|.
Any output contained in the main file will appear in all child documents
unless suppressed manually;
it cannot be suppressed automatically by the |\includeonly| directive
and thus should normally be avoided.
A method to include some content in the main file
by means of conditional processing is described in \secref{sec:conditional}.

%%%%%%%%%%%%%%%%%%%%%%%%%%%%%%%%%%%%%%%%
\paragraph{Page Numbering.}

When only a part of the document is compiled,
the appropriate numbering of pages
(as well as other status parameters)
is determined from the |.aux| files.
The latter contain information from previous passes.
However this information needs to propagate through
all intermediate child documents.
Therefore the page numbering in child documents may well
be inconsistent until the complete document is compiled at least once.

A useful (if unconventional) way to always ensure a consistent
page numbering is to restart the numbering in each child document
and denote the pages by `\textit{child}|.|\textit{page}'
where \textit{child} represents the chapter/section number of the child file.
This can be achieved by the command
|\numberwithin{page}{|\textit{child}|}|
of the \textsf{amsmath} package
where \textit{child} can be |chapter| or |section|
depending on the chosen structuring.
Alternatively, one can modify the macro |\thepage| appropriately
and reset the counter |page| at the start of each child file.

%%%%%%%%%%%%%%%%%%%%%%%%%%%%%%%%%%%%%%%%%%%%%%%%%%%%%%%%%%%%%%%%%%%%%%%%%%%%%%%%
\subsection{Conditional Processing}
\label{sec:conditional}

The package provides a mechanism to compile different versions
of a document. To customise the versions further some conditional processing
can come in handy to distinguish which version is being compiled.
The package provides two macros to describe the compilation context:

%%%%%%%%%%%%%%%%%%%%%%%%%%%%%%%%%%%%%%%%
\DescribeMacro{\ifchilddoc}
The conditional |\ifchilddoc| distinguishes between the compilation of
child documents and the main document:
%
\begin{center}
|\ifchilddoc |\textit{child-code}| |[|\||else |\textit{main-code}]| \||fi|
\end{center}

%%%%%%%%%%%%%%%%%%%%%%%%%%%%%%%%%%%%%%%%
\DescribeMacro{\childdocname}
\DescribeMacro{\childdocjob}
The macro |\childdocname| contains the filename (without extension)
of the main or child file being processed.
Note that |\childdocjob| will always contain the name of the main file.

%%%%%%%%%%%%%%%%%%%%%%%%%%%%%%%%%%%%%%%%
\paragraph{Title Page.}

Conditional processing can be used to include a title or banner page
in the main document when proper precautions are taken.
Importantly, the code in the main file should ensure that the page counter
(as well as other status parameters which are stored in the |.aux| files)
takes the same value after the conditional processing.
Otherwise the page numbers may take divergent values
depending on which part is compiled.

For example, a title page could be declared by:
%
\begin{center}
\begin{tabular}{l}
|\ifchilddoc\||else|\\
|\addtocounter{page}{-1}|\\
\textit{code for title page}\\
|\newpage|\\
|\||fi|
\end{tabular}
\end{center}
%
A banner page for the child documents can be generated by:
%
\begin{center}
\begin{tabular}{l}
|\ifchilddoc|\\
|\addtocounter{page}{-1}|\\
\textit{code for banner page}\\
|\newpage|\\
|\||fi|
\end{tabular}
\end{center}
%
Here one could write a message such as:
\begin{center}
|This is the part \childdocname{} of \childdocjob{}.|
\end{center}

%%%%%%%%%%%%%%%%%%%%%%%%%%%%%%%%%%%%%%%%%%%%%%%%%%%%%%%%%%%%%%%%%%%%%%%%%%%%%%%%
\subsection{Flags}
\label{sec:flags}

The package makes it easy to generate different versions
of the main or child documents.
To this end compilation flags can be defined
and assigned different default values.
They will be particularly useful in conjunction
with the forwarding mechanism described in \secref{sec:forward}.

For example, it may be useful to have a flag |\version|
which can be set to |draft| or |final|.
The document source will contain some conditional code
depending on the value of |\version|.
Suppose further, the flag should default to |final| for the main file
and to |draft| for child files
which is a natural assignment for editing the document.
This is achieved by placing the following code
in the preamble of the main document
(below the |\childdocmain| directive):
%
\begin{center}
\begin{tabular}{l}
|\ifchilddoc|\\
|\providecommand{\version}{draft}|\\
|\||else|\\
|\providecommand{\version}{final}|\\
|\||fi|
\end{tabular}
\end{center}
%
The definition by |\providecommand| makes sure
that previous definitions are not overwritten.
Further statements |\providecommand{\version}{...}|
can thus be added before the above code to override it.

For the main file, one might add a line
(between |\childdocmain| and the above block)
%
\begin{center}
|%\ifchilddoc\||else\providecommand{\version}{draft}\||fi|
\end{center}
%
which can be uncommented to produce a draft version.
Likewise one can add a line to the very top of a child file
(above the |\childdocof{|\textit{main}|}| directive)
%
\begin{center}
|%\providecommand{\version}{final}|
\end{center}
%
which can be uncommented to produce the final version of this child document.

%%%%%%%%%%%%%%%%%%%%%%%%%%%%%%%%%%%%%%%%%%%%%%%%%%%%%%%%%%%%%%%%%%%%%%%%%%%%%%%%
\subsection{Forwarding}
\label{sec:forward}

Different versions of the main or child documents
using compilation flags as described in \secref{sec:flags}
can be (permanently) stored in different files
for convenient compilation, viewing and distribution.
To this end, the package defines a command
to pass on compilation to a different file:

%%%%%%%%%%%%%%%%%%%%%%%%%%%%%%%%%%%%%%%%
\DescribeMacro{\childdocforward}
The command |\childdocforward| redirects processing to
another source file:
%
\begin{center}
\begin{tabular}{l}
|\input{childdoc.def}|\\
|\childdocforward[|\textit{main}|]{|\textit{dest}|}|\\
\end{tabular}
\end{center}
%
The argument \textit{dest} is the destination file
(without extension).
It should be the main file or one of the child files.
Note that further \textsf{childdoc} directives
such as |\childdocof| and |\childdocforward|
in the indicated file will be processed in this form.
The optional argument \textit{main}
passes on directly to the main file \textit{main}
while pretending to compile the child \textit{dest}.
This form behaves as if \textit{dest}
issues |\childdocof{|\textit{main}|}| right away,
and no further \textsf{childdoc} directives will be processed.

%%%%%%%%%%%%%%%%%%%%%%%%%%%%%%%%%%%%%%%%
\DescribeMacro{\...prefix}
In the alternative form |\childdocforwardprefix|,
%
\begin{center}
\begin{tabular}{l}
|\input{childdoc.def}|\\
|\childdocforwardprefix[|\textit{main}|]{|\textit{prefix}|}{|\textit{dest}|}|
\end{tabular}
\end{center}
%
the destination file is determined by a pattern
depending on the current file:
To make this work, the current file must be called
`{\textit{prefix}\hspace{0.2em}\textit{suffix}}'
with \textit{prefix} matching precisely the argument.
Processing is then passed on to the file
`{\textit{dest}\hspace{0.2em}\textit{suffix}}'.
Surely, the same effect is achieved by
directly specifying the
argument `{\textit{dest}\hspace{0.2em}\textit{suffix}}'
in the first form.
However, that requires to set up a different file
for each child. With the alternative form of the command
all these files can have exactly the same content
which simplifies setting them up and maintaining them.

For example, the following file |draft.tex|
with a compilation flag |\version| as described in \secref{sec:flags}
compiles the main document as a draft:
%
\begin{center}
\begin{tabular}{l}
|\def\version{draft}|\\
|\input{childdoc.def}|\\
|\childdocforward{|\textit{main}|}|
\end{tabular}
\end{center}
%
Likewise, the following files |final|\textit{nn}|.tex|
compile the final version of the child document
|child|\textit{nn}|.tex|:
%
\begin{center}
\begin{tabular}{l}
|\def\version{final}|\\
|\input{childdoc.def}|\\
|\childdocforwardprefix{final}{child}|
\end{tabular}
\end{center}
%

Note that when several versions of a main file and/or of each child file
are to be generated, it may be convenient to set up a |Makefile| or
shell script to automatise the process.

%%%%%%%%%%%%%%%%%%%%%%%%%%%%%%%%%%%%%%%%%%%%%%%%%%%%%%%%%%%%%%%%%%%%%%%%%%%%%%%%
\subsection{Command Line Processing}
\label{sec:commandline}

The effect of redirection files can also be achieved by invoking
the \LaTeX{} compiler with a more elaborate command line.
Most conveniently this should be done as part
of a shell script or a |Makefile|.

When using \textsf{childdoc} in the main file, the following
command lines effectively perform a redirection
(note that depending on the shell being used,
backslashes may have to be doubled: `|\|' $\to$ `|\\|'):
%
\begin{center}
|... -jobname "|\textit{target}|" |\\|"|[\textit{flags}]%
|\input{childdoc.def}\childdocforward[|\textit{main}|]{|\textit{dest}|}"|
\end{center}
%
Here \textit{target} is the name of the output file,
\textit{main} is the name of the main file
and \textit{dest} is the name of the main or child file to be processed
(all filenames without extensions).
The optional argument \textit{main} can be omitted
if \textit{main} matches \textit{dest}.
Optionally, compilation \textit{flags} can be defined via |\def| commands.
This command line makes the \TeX{} engine believe
it is compiling the file \textit{target}
whose content is specified as the latter parameter.
The provided code then forwards the processing to
\textit{main} or \textit{dest} as described in \secref{sec:forward}.

%%%%%%%%%%%%%%%%%%%%%%%%%%%%%%%%%%%%%%%%%%%%%%%%%%%%%%%%%%%%%%%%%%%%%%%%%%%%%%%%
\subsection{Include by Input}
\label{sec:input}

Including child documents by |\include| has some restrictions by design.
Most notably, the content of a child document always occupies
its own set of pages; pages cannot be shared between child documents.
Usually, this behaviour makes perfect sense
because each child document contain an essential part of the document.
However, in some situations it may be desirable to compose
a document from a collection of parts
without having mandatory page breaks between then.
For this case, the package
provides a mechanism to include parts
by |\input| which can also be processed individually.
However, by construction this mechanism
requires manual handling of the content to be output.

%%%%%%%%%%%%%%%%%%%%%%%%%%%%%%%%%%%%%%%%
\DescribeMacro{\ifchilddocmanual}
The main file should be prepared as usual, see \secref{sec:include}.
However, the document body must make a distinction
between processing of an individual part and of the main document, e.g.:
%
\begin{center}
\begin{tabular}{l}
|\ifchilddocmanual|\\
|\input{\childdocname}|\\
|\||else|\\
\textit{document body with }|\input{|\textit{part}|}|\\
|\||fi|
\end{tabular}
\end{center}
%
The conditional |\ifchilddocmanual| is true whenever
a part to be included by |\input| is being compiled,
and the name of the part is stored in |\childdocname|.

%%%%%%%%%%%%%%%%%%%%%%%%%%%%%%%%%%%%%%%%
\DescribeMacro{\childdocby}
Each part to be included by |\input| should start with:
%
\begin{center}
\begin{tabular}{l}
|\input{childdoc.def}|\\
|\childdocby{|\textit{main}|}|\\
\end{tabular}
\end{center}
%
The directive |\childdocby| is similar to |\childdocof|
described in \secref{sec:include},
but the subsequent selection of content must be done manually.
To that end, both |\ifchilddoc| and |\ifchilddocmanual|
will be true upon processing of a part,
and the name of the part is stored in |\childdocname|.
Note that |\jobname| will be set to the filename of the current part
so that each part receives an individual |.aux| file
that does not interfere with the |.aux| file(s) of the main document.
This behaviour can be altered by the alternative form
|\childdocby[*]{|\textit{main}|}| (with a non-empty optional argument)
which uses the |.aux| file of the main document
by setting |\jobname| to \textit{main}.

%%%%%%%%%%%%%%%%%%%%%%%%%%%%%%%%%%%%%%%%%%%%%%%%%%%%%%%%%%%%%%%%%%%%%%%%%%%%%%%%
\subsection{Driver Development}
\label{sec:driver}

The \textsf{childdoc} mechanism can also be use for the development
of definition files such as \LaTeX{} styles or classes.
This case differs from the above setup with multiple parts
included by |\include| in that no |\includeonly| should be invoked.
This can be achieved by starting the include file
(before |\ProvidesPackage|) with:
%
\begin{center}
\begin{tabular}{l}
|\input{childdoc.def}|\\
|\childdocforward{|\textit{main}|}|\\
\end{tabular}
\end{center}
%
or alternatively with:
%
\begin{center}
\begin{tabular}{l}
|\input{childdoc.def}|\\
|\childdocby{|\textit{main}|}|\\
\end{tabular}
\end{center}
%
Both forms have slightly different effects as described above.
The main file is prepared as usual, see \secref{sec:include}.

%%%%%%%%%%%%%%%%%%%%%%%%%%%%%%%%%%%%%%%%%%%%%%%%%%%%%%%%%%%%%%%%%%%%%%%%%%%%%%%%
\subsection{Legacy Detection}
\label{sec:detection}

The directive |\childdocmain| in the main file can detect
whether the complete document or merely a child is to be compiled
even without using the directive |\childdocof|.
This method is deprecated because it is less robust
and there is no compelling reason to use it;
it is merely provided for backward compatibility
and it may be removed in future versions.

If the detection mechanism is to be used,
it is mandatory to correctly specify
the filename of the main file as the argument of |\childdocmain|:
%
\begin{center}
\begin{tabular}{l}
|\input{childdoc.def}|\\
|\childdocmain{|\textit{main}|}|\\
\end{tabular}
\end{center}
%
If |\jobname| does not match the argument \textit{main} of |\childdocmain|,
it is assumed that |\jobname| points to the child file to be compiled.
When using |\childdocmain| with the main file specified as argument,
it suffices to start a child file
with just |\input{|\textit{main}|}|
without loading of the package and using |\childdocof|.
If instead all processing is done
with the appropriate \textsf{childdoc} directives,
the argument of \textit{main} of |\childdocmain| can be empty.

An alternative version of the command line processing described
in \secref{sec:commandline} using the detection mechanism reads:
%
\begin{center}
|... -jobname "|\textit{target}|" "|[\textit{flags}]%
[|\def\jobname{|\textit{dest}|}|]|\input{|\textit{main}|}"|
\end{center}

%%%%%%%%%%%%%%%%%%%%%%%%%%%%%%%%%%%%%%%%%%%%%%%%%%%%%%%%%%%%%%%%%%%%%%%%%%%%%%%%
\subsection{Manual Code}
\label{sec:manual}

In case one cannot be certain whether the definitions file |childdoc.def|
is installed on the target \TeX{} distribution
and one prefers not to ship it,
it is conceivable to paste a few relevant commands into the sources.

To that end, drop all statements |\input{childdoc.def}|
and perform the replacements as outlined below.
Instead of |\childdocmain{|\textit{main}|}| add the following code
to the top of the main file:
%
\begin{center}
\begin{tabular}{l}
|\||ifdefined\childdocname\endinput\||fi\newif\ifchilddoc|\\
|\edef\childdocname{\scantokens\expandafter{\jobname\noexpand}}|\\
|\def\childdocmain{|\textit{main}|}\||ifx\childdocmain\childdocname\||else|\\
|\childdoctrue\includeonly{\childdocname}\let\jobname\childdocmain\||fi|\\
\end{tabular}
\end{center}
%
Instead of |\childdocof{|\textit{main}|}| just include the main file
at the top of each child file:
%
\begin{center}
|\input{|\textit{main}|}|
\end{center}
%
A simple redirection |\childdocforward{|\textit{dest}|}| is achieved by:
%
\begin{center}
|\def\jobname{|\textit{dest}|}\input{\jobname}|
\end{center}
%
The redirection with prefix
|\childdocforwardprefix[|\textit{prefix}|]{|\textit{dest}|}|
is accomplished by:
%
\begin{center}
\begin{tabular}{l}
|{\edef\jobname{\scantokens\expandafter{\jobname\noexpand}}|\\
|\def\redirectjob |\textit{prefix}|#1~~~{\gdef\jobname{|\textit{dest}|#1}}|\\
|\expandafter\redirectjob\jobname~~~}\input{\jobname}|
\end{tabular}
\end{center}

In an alternative approach,
child documents can be compiled by a specific command line
without additional code or specific definitions:
%
\begin{center}
|... -jobname "|\textit{target}|" "|[\textit{flags}]%
|\includeonly{|\textit{dest}|}\input{|\textit{main}|}"|
\end{center}
%

%%%%%%%%%%%%%%%%%%%%%%%%%%%%%%%%%%%%%%%%%%%%%%%%%%%%%%%%%%%%%%%%%%%%%%%%%%%%%%%%
%%%%%%%%%%%%%%%%%%%%%%%%%%%%%%%%%%%%%%%%%%%%%%%%%%%%%%%%%%%%%%%%%%%%%%%%%%%%%%%%
\section{Information}

%%%%%%%%%%%%%%%%%%%%%%%%%%%%%%%%%%%%%%%%%%%%%%%%%%%%%%%%%%%%%%%%%%%%%%%%%%%%%%%%
\subsection{Copyright}

Copyright \copyright{} 2017--2018 Niklas Beisert

This work may be distributed and/or modified under the
conditions of the \LaTeX{} Project Public License, either version 1.3
of this license or (at your option) any later version.
The latest version of this license is in
  \url{http://www.latex-project.org/lppl.txt}
and version 1.3 or later is part of all distributions of \LaTeX{}
version 2005/12/01 or later.

This work has the LPPL maintenance status `maintained'.

The Current Maintainer of this work is Niklas Beisert.

This work consists of the files |README.txt|, |childdoc.ins| and |childdoc.dtx|
as well as the derived files |childdoc.def|, |cdocsamp.tex|
with |cdocsch1.tex|, |cdocsch2.tex|, |cdocspt3.tex|, |cdocspt4.tex|,
|cdocsdrf.tex|, |cdocsfn1.tex|, |cdocsfn2.tex|
as well as |childdoc.pdf|.

%%%%%%%%%%%%%%%%%%%%%%%%%%%%%%%%%%%%%%%%%%%%%%%%%%%%%%%%%%%%%%%%%%%%%%%%%%%%%%%%
\subsection{Files and Installation}

The package consists of the files:
%
\begin{center}
\begin{tabular}{ll}
    |README.txt|   & readme file \\
    |childdoc.ins| & installation file \\
    |childdoc.dtx| & source file \\
    |childdoc.def| & definition file \\
    |cdocsamp.tex| & sample main file \\
    |cdocsch1.tex| & sample include file \\
    |cdocsch2.tex| & sample include file \\
    |cdocspt3.tex| & sample part file \\
    |cdocspt4.tex| & sample part file \\
    |cdocsdrf.tex| & sample redirection file \\
    |cdocsfn1.tex| & sample redirection file \\
    |cdocsfn2.tex| & sample redirection file \\
    |childdoc.pdf| & manual
\end{tabular}
\end{center}
%
The distribution consists of the files
|README.txt|, |childdoc.ins| and |childdoc.dtx|.
%
\begin{itemize}
\item
Run (pdf)\LaTeX{} on |childdoc.dtx|
to compile the manual |childdoc.pdf| (this file).
\item
Run \LaTeX{} on |childdoc.ins| to create the definitions file |childdoc.def|
and the sample |cdocsamp.tex| with include files
|cdocsch1.tex|, |cdocsch2.tex|, |cdocspt3.tex|, |cdocspt4.tex|,
|cdocsdrf.tex|, |cdocsfn1.tex|, |cdocsfn2.tex|.
Then copy the file |childdoc.def| to an appropriate directory of your \LaTeX{}
distribution, e.g.\ \textit{texmf-root}|/tex/latex/childdoc|.
\end{itemize}

%%%%%%%%%%%%%%%%%%%%%%%%%%%%%%%%%%%%%%%%%%%%%%%%%%%%%%%%%%%%%%%%%%%%%%%%%%%%%%%%
\subsection{Related CTAN Packages}

There are several other packages which offer a similar functionality:
%
\begin{itemize}
\item
The packages
\href{http://ctan.org/pkg/docmute}{\textsf{docmute}},
\href{http://ctan.org/pkg/includex}{\textsf{includex}} and
\href{http://ctan.org/pkg/standalone}{\textsf{standalone}}
provide commands to include only the document body of
a child file thus allowing both files to be compiled individually.
\item
The packages \href{http://ctan.org/pkg/subdocs}{\textsf{subdocs}}
and \href{http://ctan.org/pkg/subfiles}{\textsf{subfiles}}
provide structures in which the main and child documents can be
encapsulated and allowing them to be compiled individually.
The inclusion mechanism is different from the conventional |\include|.
\item
The package \href{http://ctan.org/pkg/combine}{\textsf{combine}}
is an elaborate solution to combine several documents into one.
\end{itemize}
%
See also the CTAN topic \href{http://ctan.org/topic/subdocs}{\textsf{subdocs}}
for further related packages.
The present package differs from the above solutions in that
a document structure constructed with the conventional |\include| mechanism
just needs two extra commands at the top of every file
such that all constituent files can be compiled individually.

%%%%%%%%%%%%%%%%%%%%%%%%%%%%%%%%%%%%%%%%%%%%%%%%%%%%%%%%%%%%%%%%%%%%%%%%%%%%%%%%
%\subsection{Feature Suggestions}
%
%The following is a list of features which may be useful for future
%versions of this package:
%%
%\begin{itemize}
%\item
%\ldots
%\end{itemize}

%%%%%%%%%%%%%%%%%%%%%%%%%%%%%%%%%%%%%%%%%%%%%%%%%%%%%%%%%%%%%%%%%%%%%%%%%%%%%%%%
\subsection{Revision History}

%%%%%%%%%%%%%%%%%%%%%%%%%%%%%%%%%%%%%%%%
\paragraph{v2.0:} 2018/12/30

\begin{itemize}
\item
immediate forward processing
\item
added |\childdocby| mechanism
\item
manual restructured
\end{itemize}

%%%%%%%%%%%%%%%%%%%%%%%%%%%%%%%%%%%%%%%%
\paragraph{v1.6:} 2018/01/17

\begin{itemize}
\item
application for development of include files
\item
corrections to manual
\end{itemize}

%%%%%%%%%%%%%%%%%%%%%%%%%%%%%%%%%%%%%%%%
\paragraph{v1.5:} 2017/05/21

\begin{itemize}
\item
more complete structuring introduced
\item
|\childdocof| introduced
\item
|\childdoc| renamed to |\childdocmain|
\item
|\childredirect| renamed to |\childdocforward| and |\childdocforwardprefix|
and functionality expanded
\end{itemize}

%%%%%%%%%%%%%%%%%%%%%%%%%%%%%%%%%%%%%%%%
\paragraph{v1.0:} 2017/04/27

\begin{itemize}
\item
manual and install package
\item
first version published on CTAN
\end{itemize}

%%%%%%%%%%%%%%%%%%%%%%%%%%%%%%%%%%%%%%%%
\paragraph{v0.6:} 2017/04/26

\begin{itemize}
\item
redirection mechanism added
\end{itemize}

%%%%%%%%%%%%%%%%%%%%%%%%%%%%%%%%%%%%%%%%
\paragraph{v0.5:} 2017/04/26

\begin{itemize}
\item
functionality in definition file
\end{itemize}


%%%%%%%%%%%%%%%%%%%%%%%%%%%%%%%%%%%%%%%%%%%%%%%%%%%%%%%%%%%%%%%%%%%%%%%%%%%%%%%%
%%%%%%%%%%%%%%%%%%%%%%%%%%%%%%%%%%%%%%%%%%%%%%%%%%%%%%%%%%%%%%%%%%%%%%%%%%%%%%%%
%%%%%%%%%%%%%%%%%%%%%%%%%%%%%%%%%%%%%%%%%%%%%%%%%%%%%%%%%%%%%%%%%%%%%%%%%%%%%%%%
\appendix

\settowidth\MacroIndent{\rmfamily\scriptsize 000\ }

 \DocInput{childdoc.dtx}

\end{document}
%</driver>
% \fi
%
% %%%%%%%%%%%%%%%%%%%%%%%%%%%%%%%%%%%%%%%%%%%%%%%%%%%%%%%%%%%%%%%%%%%%%%%%%%%%%%
% %%%%%%%%%%%%%%%%%%%%%%%%%%%%%%%%%%%%%%%%%%%%%%%%%%%%%%%%%%%%%%%%%%%%%%%%%%%%%%
% \section{Sample}
%\iffalse
%<*samplemain>
%\fi
%
% The following presents a sample document
% with two chapters, two parts, a title page,
% a compile flag as well as three forwarding files to set the flag.
% It consists of eight |.tex| files:
% \begin{center}
% \begin{tabular}{ll}
% |cdocsamp.tex|&main file\\
% |cdocsch1.tex|&include file for chapter 1\\
% |cdocsch2.tex|&include file for chapter 2\\
% |cdocspt3.tex|&include file for part 3\\
% |cdocspt4.tex|&include file for part 4\\
% |cdocsdrf.tex|&forwarding file for main file in draft mode\\
% |cdocsfi1.tex|&forwarding file for final version of chapter 1\\
% |cdocsfi2.tex|&forwarding file for final version of chapter 2\\
% \end{tabular}
% \end{center}
% Each of the eight files can be compiled directly by the \LaTeX{} compiler.
%
% %%%%%%%%%%%%%%%%%%%%%%%%%%%%%%%%%%%%%%
% \paragraph{Main File.}
%
% The main file is called |cdocsamp.tex|.
%
% Load the \textsf{childdoc} definitions and
% declare the filename for the main document:
%    \begin{macrocode}
\input{childdoc.def}
\childdocmain{}
%    \end{macrocode}

% Optional override for |\version| flag:
%    \begin{macrocode}
%%\ifchilddoc\else\providecommand{\version}{draft}\fi
%    \end{macrocode}

% Define the default values for the |\version| flag
% (|final| for the main file and |draft| for childs):
%    \begin{macrocode}
\ifchilddoc
\providecommand{\version}{draft}
\else
\providecommand{\version}{final}
\fi
%    \end{macrocode}

% Load the standard document class:
%    \begin{macrocode}
\documentclass[12pt]{article}
%    \end{macrocode}

% Start the document body:
%    \begin{macrocode}
\begin{document}
%    \end{macrocode}

% Declare a title page.
% Print title, part of document being processed and version flag:
%    \begin{macrocode}
\addtocounter{page}{-1}
\begin{center}
{\LARGE\bfseries{}childdoc example\par}
\vspace{1cm}
\ifchilddoc
\ifchilddocmanual part\else chapter\fi:
`\childdocname' of `\childdocjob'\par
\else
main document: `\childdocjob'\par
\fi
version: \version\par
\end{center}
\newpage
%    \end{macrocode}

% Manually include selected file,
% otherwise process as usual:
%    \begin{macrocode}
\ifchilddocmanual
\section*{part `\childdocname'}
\input{\childdocname}
\else
%    \end{macrocode}

% Include the two chapters:
%    \begin{macrocode}
\include{cdocsch1}
\include{cdocsch2}
%    \end{macrocode}

% Include the two parts unless only chapters should be displayed:
%    \begin{macrocode}
\ifchilddoc\else
\section{part three}
\input{cdocspt3}
\section{part four}
\input{cdocspt4}
\fi
%    \end{macrocode}

% Process as usual until here:
%    \begin{macrocode}
\fi
%    \end{macrocode}

% End of document body:
%    \begin{macrocode}
\end{document}
%    \end{macrocode}
%\iffalse
%</samplemain>
%\fi
%
% %%%%%%%%%%%%%%%%%%%%%%%%%%%%%%%%%%%%%%
% \paragraph{Chapter Include Files.}
%
% The include files are called |cdocsch1.tex| and |cdocsch2.tex|.
%
%\iffalse
%<*samplechap1|samplechap2>
%\fi

% Optional override for |\version| flag:
%    \begin{macrocode}
%%\providecommand{\version}{final}
%    \end{macrocode}

% Include the main document:
%    \begin{macrocode}
\input{childdoc.def}
\childdocof{cdocsamp}
%    \end{macrocode}

%\iffalse
%</samplechap1|samplechap2>
%\fi
%
%\iffalse
%<*samplechap1>
%\fi
% Some text for chapter 1:
%    \begin{macrocode}
\section{one}
some text in chapter one
%    \end{macrocode}

%\iffalse
%</samplechap1>
%\fi
% Some text for chapter 2:
%\iffalse
%<*samplechap2>
%\fi
%    \begin{macrocode}
\section{two}
more text in chapter two
%    \end{macrocode}

%\iffalse
%</samplechap2>
%\fi
%
% %%%%%%%%%%%%%%%%%%%%%%%%%%%%%%%%%%%%%%
% \paragraph{Part Include Files.}
%
% The include files are called |cdocspt3.tex| and |cdocspt4.tex|.
%
%\iffalse
%<*samplepart3|samplepart4>
%\fi

% Optional override for |\version| flag:
%    \begin{macrocode}
%%\providecommand{\version}{final}
%    \end{macrocode}

% Include the main document:
%    \begin{macrocode}
\input{childdoc.def}
\childdocby{cdocsamp}
%    \end{macrocode}

%\iffalse
%</samplepart3|samplepart4>
%\fi
%
%\iffalse
%<*samplepart3>
%\fi
% Some text for part 3:
%    \begin{macrocode}
some text in part three
%    \end{macrocode}

%\iffalse
%</samplepart3>
%\fi
% Some text for part 4:
%\iffalse
%<*samplepart4>
%\fi
%    \begin{macrocode}
more text in part four
%    \end{macrocode}

%\iffalse
%</samplepart4>
%\fi
%
% %%%%%%%%%%%%%%%%%%%%%%%%%%%%%%%%%%%%%%
% \paragraph{Forwarding for a Complete Draft.}
%
% The following forwarding file |cdocsdrf.tex|
% compiles the main document in draft mode:
%\iffalse
%<*sampledraft>
%\fi
%    \begin{macrocode}
\def\version{draft}
\input{childdoc.def}
\childdocforward{cdocsamp}
%    \end{macrocode}

%\iffalse
%</sampledraft>
%\fi
%
% %%%%%%%%%%%%%%%%%%%%%%%%%%%%%%%%%%%%%%
% \paragraph{Forwarding for Final Version of the Chapters.}
%
% The following forwarding files |cdocsfn1.tex| and |cdocsfn2.tex|
% (with identical content)
% compile the final versions of the child documents
% |cdocsch1.tex| and |cdocsch2.tex|, respectively:
%\iffalse
%<*samplefinal>
%\fi
%    \begin{macrocode}
\def\version{final}
\input{childdoc.def}
\childdocforwardprefix[cdocsamp]{cdocsfn}{cdocsch}
%    \end{macrocode}

%\iffalse
%</samplefinal>
%\fi
%
% %%%%%%%%%%%%%%%%%%%%%%%%%%%%%%%%%%%%%%
% \paragraph{Command Line Processing.}
%
% The following three command lines generate the output files
% |cdocscld|, |cdocscl1| and |cdocscl2|
% which should be identical to
% |cdocsdrf|, |cdocsch1| and |cdocsfn2|, respectively:
% \begin{center}
% \begin{tabular}{l}
% |latex -jobname cdocscld \|\\
% |  "\def\version{draft}\input{childdoc.def}\childdocforward{cdocsamp}"|\\
% |latex -jobname cdocscl1 \|\\
% |  "\input{childdoc.def}\childdocforward[cdocsamp]{cdocsch1}"|\\
% |latex -jobname cdocscl2 \|\\
% |  "\def\version{final}\input{childdoc.def}\childdocforward{cdocsch2}"|
% \end{tabular}
% \end{center}
% Note that the trailing backslash on each first line
% merely continues the input to the second line
% (for convenient cut ant paste).
% Furthermore, the command |latex| can be replaced by any
% of its alternative versions such as |pdflatex|.
%
% %%%%%%%%%%%%%%%%%%%%%%%%%%%%%%%%%%%%%%%%%%%%%%%%%%%%%%%%%%%%%%%%%%%%%%%%%%%%%%
% %%%%%%%%%%%%%%%%%%%%%%%%%%%%%%%%%%%%%%%%%%%%%%%%%%%%%%%%%%%%%%%%%%%%%%%%%%%%%%
% \section{Implementation}
%\iffalse
%<*package>
%\fi
%
% This section describes the definitions file |childdoc.def|.

% The definitions cannot be loaded using |\usepackage| or |\RequirePackage|
% which has a mechanism to prevent loading a style file more than once.
% When loading the definitions by means of |\input|
% multiple instances have to be prevented manually:
%\iffalse
%This code needs to be before the `\ProvidesFile' directive
%which is defined at the beginning of this file.
%Therefore it is also placed there and commented out here.
%</package>
%<*discard>
%\fi
%    \begin{macrocode}
\ifdefined\childdocmain\endinput\fi
%    \end{macrocode}
%\iffalse
%</discard>
%<*package>
%\fi
%
% \macro{\ifchilddoc}
% \macro{\ifchilddocmanual}
% The conditional |\ifchilddoc| tells whether a
% child (true) or main (false) document is being compiled.
% The conditional |\ifchilddocmanual| tells whether
% the |\includeonly| mechanism is used (false) or
% the selection of child files must be performed manually (true).
% The definitions initialise to false:
%    \begin{macrocode}
\newif\ifchilddoc
\newif\ifchilddocmanual
%    \end{macrocode}

% \macro{\childdocname}
% \macro{\childdocjob}
% The macro |\childdocname| stores the name of the main document
% to be compiled. The macro |\childdocjob| stores the name of
% the document on which the \LaTeX{} compiler was originally invoked.
% The content of |\jobname| cannot be compared
% to filenames specified in the source due to different catcodes.
% The following code rescans |\jobname|, stores the result
% in |\childdocname| and saves a copy in |\childdocjob|:
%    \begin{macrocode}
\edef\childdocname{\scantokens\expandafter{\jobname\noexpand}}
\let\childdocjob\childdocname
%    \end{macrocode}

% \macro{\childdocdisable}
% The macro |\childdocdisable| prevents the main file
% from being processed more than once.
% At this stage, the main document command |\childdocmain|
% is assumed to be called once again where it should do nothing.
% Any subsequent call to it should prevent
% a secondary processing of the main document
% It overwrites the forwarding commands
% |\childdocof| and |\childdocforward|
% with empty macros to prevent further inclusions of the main document:
%    \begin{macrocode}
\newcommand{\childdocdisable}
{
  \renewcommand{\childdocmain}[1]{\renewcommand{\childdocmain}[1]{\endinput}}
  \renewcommand{\childdocof}[1]{}
  \renewcommand{\childdocby}[2][]{}
  \renewcommand{\childdocforward}[2][]{}
  \renewcommand{\childdocdisable}{}
}
%    \end{macrocode}

% \macro{\childdocmain}
% The macro |\childdocmain| is to be called at the top of the main file
% with nothing or the main filename (without extension) as argument.
% First, it breaks loops.
% If the argument is not empty and does not match |\childdocname|
% (which is set by the first inclusion of |childdoc.def|),
% |\ifchilddoc| is set to true, |\includeonly| is applied to the child file
% and |\jobname| is set to the main file
% (for proper handling of |.aux| files):
%    \begin{macrocode}
\newcommand{\childdocmain}[1]
{
  \childdocdisable\childdocmain{}
  \if?#1?\else
    \begingroup
      \def\childdoctmp{#1}
      \ifx\childdoctmp\childdocname
        \def\childdoctmp{}
      \else
        \def\childdoctmp
        {
          \childdoctrue
          \includeonly{\childdocname}
          \def\childdocjob{#1}
          \def\jobname{#1}
        }
      \fi
      \expandafter
    \endgroup
    \childdoctmp
  \fi
}
%    \end{macrocode}

% \macro{\childdocof}
% The command |\childdocof| redirects
% compilation to the main file |#1|.
%    \begin{macrocode}
\newcommand{\childdocof}[1]
{
  \childdocdisable
  \childdoctrue
  \includeonly{\childdocname}
  \def\jobname{#1}
  \def\childdocjob{#1}
  \input{#1}
}
%    \end{macrocode}

% \macro{\childdocby}
% The command |\childdocby| ....
%    \begin{macrocode}
\newcommand{\childdocby}[2][]
{
  \childdocdisable
  \childdoctrue
  \childdocmanualtrue
  \if?#1?\else
    \def\jobname{#2}
  \fi
  \def\childdocjob{#2}
  \input{#2}
  \endinput
}
%    \end{macrocode}

% \macro{\childdocforward}
% The command |\childdocforward| redirects
% compilation to the main file or
% (if the optional argument is given) a child file.
% Parameters are set as if the main file
% or a child file starting with |\childdocof| was compiled.
% Then compilation is handed over to the main file:
%    \begin{macrocode}
\newcommand{\childdocforward}[2][]
{
  \begingroup
    \if?#1?
      \def\childdoctmp
      {
        \def\childdocname{#2}
        \def\childdocjob{#2}
        \def\jobname{#2}
        \input{#2}
        \endinput
      }
    \else
      \def\childdoctmp
      {
        \childdocdisable
        \def\childdocname{#2}
        \childdoctrue
        \includeonly{#2}
        \def\childdocjob{#1}
        \def\jobname{#1}
        \input{#1}
        \endinput
      }
    \fi
    \expandafter
  \endgroup
  \childdoctmp
}
%    \end{macrocode}

% \macro{\childdocforwardprefix}
% The command |\childdocforwardprefix| redirects
% compilation to the main or a child file by means of a pattern.
% The prefix |#1| in the current filename is replaced by |#2|
% and the suffix of the current filename is kept
% (it is assumed that the filename does not contain the substring `|~~~|'
% which is used as a delimiter).
% Compilation is handed over to the new file by |\childdocforward|:
%    \begin{macrocode}
\newcommand{\childdocforwardprefix}[3][]
{
  \begingroup
    \def\childdocextract #2##1~~~{\def\childdoctmp{\childdocforward[#1]{#3##1}}}
    \expandafter\childdocextract\childdocname~~~
    \expandafter
  \endgroup
  \childdoctmp
}
%    \end{macrocode}

% \macro{\childdoc}
% The deprecated macro |\childdoc| is a legacy version of |\childdocmain|:
%    \begin{macrocode}
\newcommand{\childdoc}{\childdocmain}
%    \end{macrocode}

% \macro{\childdocredirect}
% The deprecated macro |\childdocredirect| is a legacy version
% of |\childdocforward| and |\childdocforwardprefix|:
%    \begin{macrocode}
\newcommand{\childdocredirect}[2][]
{
  \begingroup
    \if?#1?
      \def\childdoctmp{\childdocforward{#2}}
    \else
      \def\childdoctmp{\childdocforwardprefix{#1}{#2}}
    \fi
    \expandafter
  \endgroup
  \childdoctmp
}
%    \end{macrocode}

%\iffalse
%</package>
%\fi
%
\endinput
|\\
|\childdocby{|\textit{main}|}|\\
\end{tabular}
\end{center}
%
The directive |\childdocby| is similar to |\childdocof|
described in \secref{sec:include},
but the subsequent selection of content must be done manually.
To that end, both |\ifchilddoc| and |\ifchilddocmanual|
will be true upon processing of a part,
and the name of the part is stored in |\childdocname|.
Note that |\jobname| will be set to the filename of the current part
so that each part receives an individual |.aux| file
that does not interfere with the |.aux| file(s) of the main document.
This behaviour can be altered by the alternative form
|\childdocby[*]{|\textit{main}|}| (with a non-empty optional argument)
which uses the |.aux| file of the main document
by setting |\jobname| to \textit{main}.

%%%%%%%%%%%%%%%%%%%%%%%%%%%%%%%%%%%%%%%%%%%%%%%%%%%%%%%%%%%%%%%%%%%%%%%%%%%%%%%%
\subsection{Driver Development}
\label{sec:driver}

The \textsf{childdoc} mechanism can also be use for the development
of definition files such as \LaTeX{} styles or classes.
This case differs from the above setup with multiple parts
included by |\include| in that no |\includeonly| should be invoked.
This can be achieved by starting the include file
(before |\ProvidesPackage|) with:
%
\begin{center}
\begin{tabular}{l}
|% \iffalse
%
% childdoc.dtx Copyright (C) 2017-2018 Niklas Beisert
%
% This work may be distributed and/or modified under the
% conditions of the LaTeX Project Public License, either version 1.3
% of this license or (at your option) any later version.
% The latest version of this license is in
%   http://www.latex-project.org/lppl.txt
% and version 1.3 or later is part of all distributions of LaTeX
% version 2005/12/01 or later.
%
% This work has the LPPL maintenance status `maintained'.
%
% The Current Maintainer of this work is Niklas Beisert.
%
% This work consists of the files childdoc.dtx and childdoc.ins
% and the derived files childdoc.def and cdocsamp.tex with
% cdocsch1.tex, cdocsch2.tex, cdocsdrf.tex, cdocsfn1.tex, cdocsfn2.tex.
%
%<package>\ifdefined\childdocmain\endinput\fi
%<package>\ProvidesFile{childdoc.def}[2018/12/30 v2.0 child document driver]
%<samplemain>\ProvidesFile{cdocsamp.tex}[2018/12/30 v2.0 sample for childdoc]
%<*driver>
%\ProvidesFile{childdoc.drv}[2018/12/30 v2.0 childdoc reference manual file]
\PassOptionsToClass{10pt,a4paper}{article}
\documentclass{ltxdoc}

\usepackage[margin=35mm]{geometry}
\usepackage{hyperref}
\usepackage{hyperxmp}
\usepackage[usenames]{color}

\hypersetup{colorlinks=true}
\hypersetup{pdfstartview=FitH}
\hypersetup{pdfpagemode=UseNone}
\hypersetup{pdfsource={}}
\hypersetup{pdflang={en-UK}}
\hypersetup{pdfcopyright={Copyright 2017-2018 Niklas Beisert.
  This work may be distributed and/or modified under the
  conditions of the LaTeX Project Public License, either version 1.3
  of this license or (at your option) any later version.}}
\hypersetup{pdflicenseurl={http://www.latex-project.org/lppl.txt}}
\hypersetup{pdfcontactaddress={ETH Zurich, ITP, HIT K,
  Wolfgang-Pauli-Strasse 27}}
\hypersetup{pdfcontactpostcode={8093}}
\hypersetup{pdfcontactcity={Zurich}}
\hypersetup{pdfcontactcountry={Switzerland}}
\hypersetup{pdfcontactemail={nbeisert@itp.phys.ethz.ch}}
\hypersetup{pdfcontacturl={http://people.phys.ethz.ch/\xmptilde nbeisert/}}

\newcommand{\secref}[1]{\hyperref[#1]{section \ref*{#1}}}

\parskip1ex
\parindent0pt
\let\olditemize\itemize
\def\itemize{\olditemize\parskip0pt}

\begin{document}

\title{The \textsf{childdoc} Package}
\hypersetup{pdftitle={The childdoc Package}}
\author{Niklas Beisert\\[2ex]
  Institut f\"ur Theoretische Physik\\
  Eidgen\"ossische Technische Hochschule Z\"urich\\
  Wolfgang-Pauli-Strasse 27, 8093 Z\"urich, Switzerland\\[1ex]
  \href{mailto:nbeisert@itp.phys.ethz.ch}
  {\texttt{nbeisert@itp.phys.ethz.ch}}}
\hypersetup{pdfauthor={Niklas Beisert}}
\hypersetup{pdfsubject={Manual for the LaTeX2e Package childdoc}}
\date{30 December 2018, \textsf{v2.0}}
\maketitle

\begin{abstract}\noindent
\textsf{childdoc} is a \LaTeXe{} package
that enables the direct compilation
of document sections included by |\include|
to individual files.
\end{abstract}

\begingroup
\parskip0ex
\tableofcontents
\endgroup

%%%%%%%%%%%%%%%%%%%%%%%%%%%%%%%%%%%%%%%%%%%%%%%%%%%%%%%%%%%%%%%%%%%%%%%%%%%%%%%%
%%%%%%%%%%%%%%%%%%%%%%%%%%%%%%%%%%%%%%%%%%%%%%%%%%%%%%%%%%%%%%%%%%%%%%%%%%%%%%%%
\section{Introduction}

\LaTeX{} provides a mechanism to structure a large document (such as a book)
into a main file and several child files (containing the chapters)
using the |\include| command.
This mechanism is beneficial for documents
which span hundreds of pages in order to
make the source file(s) more manageable.
Moreover, compilation can be restricted to
selected child files by means of the |\includeonly| command.
The latter feature can be used to reduce the compilation time while editing
(this was significantly more useful in the earlier days of \LaTeX{})
or to generate a smaller document which is easier to navigate.
Another application of |\includeonly| is to generate
documents consisting of selected parts of the complete document.

However, there are a few drawbacks of the plain |\include| mechanism:
\begin{itemize}
\item
The child files cannot be compiled on their own,
they can only be compiled via the main file.
A naive editing environment
(such as a text editor with an option
to have the current file processed by \LaTeX)
may require one to switch to the main file before compiling;
attempting to compile the child file produces errors.
\item
The main file must be modified (each time)
to adjust the |\includeonly| command
to the present needs. This easily leaves the main file in a messy state.
\item
The generated document will always carry the filename
of the main document. This is inconvenient if
several child files are to be compiled and
to be kept for distribution.
\end{itemize}

The present package provides a simple interface
to make child files individually compilable by \LaTeX{}.
Compiling a child file then has the same effect as compiling
the main file with an |\includeonly| command
to select the appropriate child.
Moreover the generated document will carry the name of the child
rather than the main file.
This resolves all three above issues.

This feature is meant to make the editing of books,
thesis documents and lecture notes somewhat more convenient.
However, the package can also be used efficiently for
composing a series of documents (such as exercise sheets)
which are typically distributed individually.
It then assists the author in generating the individual documents
(potentially in different versions)
as well as a document containing the collected series.
Another application is in developing style files
or other kinds of included material
where compilation of the style file could redirect
to a sample or test file.

%%%%%%%%%%%%%%%%%%%%%%%%%%%%%%%%%%%%%%%%%%%%%%%%%%%%%%%%%%%%%%%%%%%%%%%%%%%%%%%%
%%%%%%%%%%%%%%%%%%%%%%%%%%%%%%%%%%%%%%%%%%%%%%%%%%%%%%%%%%%%%%%%%%%%%%%%%%%%%%%%
\section{Usage}

First of all, the package \textsf{childdoc} is \emph{not} a standard
\LaTeXe{} |.sty| style file! Therefore it needs to be invoked in
a non-standard way.

%%%%%%%%%%%%%%%%%%%%%%%%%%%%%%%%%%%%%%%%%%%%%%%%%%%%%%%%%%%%%%%%%%%%%%%%%%%%%%%%
\subsection{Included Files}
\label{sec:include}

%%%%%%%%%%%%%%%%%%%%%%%%%%%%%%%%%%%%%%%%
\DescribeMacro{\childdocmain}
To use the package, add the commands
\begin{center}
\begin{tabular}{l}
|\input{childdoc.def}|\\
|\childdocmain{}|\\
\end{tabular}
\end{center}
at the very top of the main \LaTeX{} file,
in particular \emph{before} the |\documentclass| statement!
The argument of |\childdocmain| should be left empty
(but it must be present).

%%%%%%%%%%%%%%%%%%%%%%%%%%%%%%%%%%%%%%%%
\DescribeMacro{\childdocof}
Furthermore, add the commands
\begin{center}
\begin{tabular}{l}
|\input{childdoc.def}|\\
|\childdocof{|\textit{main}|}|\\
\end{tabular}
\end{center}
at the top of every child file \textit{child}
which is included by |\include{|\textit{child}|}|
from within the main file
(or at least for those files to be compiled individually).
The argument \textit{main} must be the filename of the main file.

There are a couple of
considerations in setting up the main and child documents:

%%%%%%%%%%%%%%%%%%%%%%%%%%%%%%%%%%%%%%%%
\paragraph{Restrictions.}

Please note the following restrictions:
\begin{itemize}
\item
|\childdocmain| must be called with one argument \textit{main}
to ensure compatibility with earlier version of the package.
It must either be empty (|\childdocmain{}|)
or precisely match the filename of the main file in which it is specified.
See \secref{sec:detection} for further information.
\item
The filename \textit{main} must be specified without the |.tex| extension.
\item
The filename \textit{main} is case sensitive
(even in case-insensitive file systems)
due to internal string comparison.
\item
The argument \textit{main} should be fully expanded, it cannot be a macro.
\item
Subdirectories and special characters should be avoided in filenames.
\item
The command |\childdocmain{|\textit{main}|}| must be followed by a whitespace.
It should not be followed immediately by another command
or by a comment mark `|%|'.
This is because the \TeX{} parser reads the token immediately following
the argument of |\childdocmain| and puts it
at the beginning of every child section;
however, a white\-space is ignored.
\end{itemize}

%%%%%%%%%%%%%%%%%%%%%%%%%%%%%%%%%%%%%%%%
\paragraph{Content of Main File.}

It is advisable to place all content in the child files included by |\include|.
Any output contained in the main file will appear in all child documents
unless suppressed manually;
it cannot be suppressed automatically by the |\includeonly| directive
and thus should normally be avoided.
A method to include some content in the main file
by means of conditional processing is described in \secref{sec:conditional}.

%%%%%%%%%%%%%%%%%%%%%%%%%%%%%%%%%%%%%%%%
\paragraph{Page Numbering.}

When only a part of the document is compiled,
the appropriate numbering of pages
(as well as other status parameters)
is determined from the |.aux| files.
The latter contain information from previous passes.
However this information needs to propagate through
all intermediate child documents.
Therefore the page numbering in child documents may well
be inconsistent until the complete document is compiled at least once.

A useful (if unconventional) way to always ensure a consistent
page numbering is to restart the numbering in each child document
and denote the pages by `\textit{child}|.|\textit{page}'
where \textit{child} represents the chapter/section number of the child file.
This can be achieved by the command
|\numberwithin{page}{|\textit{child}|}|
of the \textsf{amsmath} package
where \textit{child} can be |chapter| or |section|
depending on the chosen structuring.
Alternatively, one can modify the macro |\thepage| appropriately
and reset the counter |page| at the start of each child file.

%%%%%%%%%%%%%%%%%%%%%%%%%%%%%%%%%%%%%%%%%%%%%%%%%%%%%%%%%%%%%%%%%%%%%%%%%%%%%%%%
\subsection{Conditional Processing}
\label{sec:conditional}

The package provides a mechanism to compile different versions
of a document. To customise the versions further some conditional processing
can come in handy to distinguish which version is being compiled.
The package provides two macros to describe the compilation context:

%%%%%%%%%%%%%%%%%%%%%%%%%%%%%%%%%%%%%%%%
\DescribeMacro{\ifchilddoc}
The conditional |\ifchilddoc| distinguishes between the compilation of
child documents and the main document:
%
\begin{center}
|\ifchilddoc |\textit{child-code}| |[|\||else |\textit{main-code}]| \||fi|
\end{center}

%%%%%%%%%%%%%%%%%%%%%%%%%%%%%%%%%%%%%%%%
\DescribeMacro{\childdocname}
\DescribeMacro{\childdocjob}
The macro |\childdocname| contains the filename (without extension)
of the main or child file being processed.
Note that |\childdocjob| will always contain the name of the main file.

%%%%%%%%%%%%%%%%%%%%%%%%%%%%%%%%%%%%%%%%
\paragraph{Title Page.}

Conditional processing can be used to include a title or banner page
in the main document when proper precautions are taken.
Importantly, the code in the main file should ensure that the page counter
(as well as other status parameters which are stored in the |.aux| files)
takes the same value after the conditional processing.
Otherwise the page numbers may take divergent values
depending on which part is compiled.

For example, a title page could be declared by:
%
\begin{center}
\begin{tabular}{l}
|\ifchilddoc\||else|\\
|\addtocounter{page}{-1}|\\
\textit{code for title page}\\
|\newpage|\\
|\||fi|
\end{tabular}
\end{center}
%
A banner page for the child documents can be generated by:
%
\begin{center}
\begin{tabular}{l}
|\ifchilddoc|\\
|\addtocounter{page}{-1}|\\
\textit{code for banner page}\\
|\newpage|\\
|\||fi|
\end{tabular}
\end{center}
%
Here one could write a message such as:
\begin{center}
|This is the part \childdocname{} of \childdocjob{}.|
\end{center}

%%%%%%%%%%%%%%%%%%%%%%%%%%%%%%%%%%%%%%%%%%%%%%%%%%%%%%%%%%%%%%%%%%%%%%%%%%%%%%%%
\subsection{Flags}
\label{sec:flags}

The package makes it easy to generate different versions
of the main or child documents.
To this end compilation flags can be defined
and assigned different default values.
They will be particularly useful in conjunction
with the forwarding mechanism described in \secref{sec:forward}.

For example, it may be useful to have a flag |\version|
which can be set to |draft| or |final|.
The document source will contain some conditional code
depending on the value of |\version|.
Suppose further, the flag should default to |final| for the main file
and to |draft| for child files
which is a natural assignment for editing the document.
This is achieved by placing the following code
in the preamble of the main document
(below the |\childdocmain| directive):
%
\begin{center}
\begin{tabular}{l}
|\ifchilddoc|\\
|\providecommand{\version}{draft}|\\
|\||else|\\
|\providecommand{\version}{final}|\\
|\||fi|
\end{tabular}
\end{center}
%
The definition by |\providecommand| makes sure
that previous definitions are not overwritten.
Further statements |\providecommand{\version}{...}|
can thus be added before the above code to override it.

For the main file, one might add a line
(between |\childdocmain| and the above block)
%
\begin{center}
|%\ifchilddoc\||else\providecommand{\version}{draft}\||fi|
\end{center}
%
which can be uncommented to produce a draft version.
Likewise one can add a line to the very top of a child file
(above the |\childdocof{|\textit{main}|}| directive)
%
\begin{center}
|%\providecommand{\version}{final}|
\end{center}
%
which can be uncommented to produce the final version of this child document.

%%%%%%%%%%%%%%%%%%%%%%%%%%%%%%%%%%%%%%%%%%%%%%%%%%%%%%%%%%%%%%%%%%%%%%%%%%%%%%%%
\subsection{Forwarding}
\label{sec:forward}

Different versions of the main or child documents
using compilation flags as described in \secref{sec:flags}
can be (permanently) stored in different files
for convenient compilation, viewing and distribution.
To this end, the package defines a command
to pass on compilation to a different file:

%%%%%%%%%%%%%%%%%%%%%%%%%%%%%%%%%%%%%%%%
\DescribeMacro{\childdocforward}
The command |\childdocforward| redirects processing to
another source file:
%
\begin{center}
\begin{tabular}{l}
|\input{childdoc.def}|\\
|\childdocforward[|\textit{main}|]{|\textit{dest}|}|\\
\end{tabular}
\end{center}
%
The argument \textit{dest} is the destination file
(without extension).
It should be the main file or one of the child files.
Note that further \textsf{childdoc} directives
such as |\childdocof| and |\childdocforward|
in the indicated file will be processed in this form.
The optional argument \textit{main}
passes on directly to the main file \textit{main}
while pretending to compile the child \textit{dest}.
This form behaves as if \textit{dest}
issues |\childdocof{|\textit{main}|}| right away,
and no further \textsf{childdoc} directives will be processed.

%%%%%%%%%%%%%%%%%%%%%%%%%%%%%%%%%%%%%%%%
\DescribeMacro{\...prefix}
In the alternative form |\childdocforwardprefix|,
%
\begin{center}
\begin{tabular}{l}
|\input{childdoc.def}|\\
|\childdocforwardprefix[|\textit{main}|]{|\textit{prefix}|}{|\textit{dest}|}|
\end{tabular}
\end{center}
%
the destination file is determined by a pattern
depending on the current file:
To make this work, the current file must be called
`{\textit{prefix}\hspace{0.2em}\textit{suffix}}'
with \textit{prefix} matching precisely the argument.
Processing is then passed on to the file
`{\textit{dest}\hspace{0.2em}\textit{suffix}}'.
Surely, the same effect is achieved by
directly specifying the
argument `{\textit{dest}\hspace{0.2em}\textit{suffix}}'
in the first form.
However, that requires to set up a different file
for each child. With the alternative form of the command
all these files can have exactly the same content
which simplifies setting them up and maintaining them.

For example, the following file |draft.tex|
with a compilation flag |\version| as described in \secref{sec:flags}
compiles the main document as a draft:
%
\begin{center}
\begin{tabular}{l}
|\def\version{draft}|\\
|\input{childdoc.def}|\\
|\childdocforward{|\textit{main}|}|
\end{tabular}
\end{center}
%
Likewise, the following files |final|\textit{nn}|.tex|
compile the final version of the child document
|child|\textit{nn}|.tex|:
%
\begin{center}
\begin{tabular}{l}
|\def\version{final}|\\
|\input{childdoc.def}|\\
|\childdocforwardprefix{final}{child}|
\end{tabular}
\end{center}
%

Note that when several versions of a main file and/or of each child file
are to be generated, it may be convenient to set up a |Makefile| or
shell script to automatise the process.

%%%%%%%%%%%%%%%%%%%%%%%%%%%%%%%%%%%%%%%%%%%%%%%%%%%%%%%%%%%%%%%%%%%%%%%%%%%%%%%%
\subsection{Command Line Processing}
\label{sec:commandline}

The effect of redirection files can also be achieved by invoking
the \LaTeX{} compiler with a more elaborate command line.
Most conveniently this should be done as part
of a shell script or a |Makefile|.

When using \textsf{childdoc} in the main file, the following
command lines effectively perform a redirection
(note that depending on the shell being used,
backslashes may have to be doubled: `|\|' $\to$ `|\\|'):
%
\begin{center}
|... -jobname "|\textit{target}|" |\\|"|[\textit{flags}]%
|\input{childdoc.def}\childdocforward[|\textit{main}|]{|\textit{dest}|}"|
\end{center}
%
Here \textit{target} is the name of the output file,
\textit{main} is the name of the main file
and \textit{dest} is the name of the main or child file to be processed
(all filenames without extensions).
The optional argument \textit{main} can be omitted
if \textit{main} matches \textit{dest}.
Optionally, compilation \textit{flags} can be defined via |\def| commands.
This command line makes the \TeX{} engine believe
it is compiling the file \textit{target}
whose content is specified as the latter parameter.
The provided code then forwards the processing to
\textit{main} or \textit{dest} as described in \secref{sec:forward}.

%%%%%%%%%%%%%%%%%%%%%%%%%%%%%%%%%%%%%%%%%%%%%%%%%%%%%%%%%%%%%%%%%%%%%%%%%%%%%%%%
\subsection{Include by Input}
\label{sec:input}

Including child documents by |\include| has some restrictions by design.
Most notably, the content of a child document always occupies
its own set of pages; pages cannot be shared between child documents.
Usually, this behaviour makes perfect sense
because each child document contain an essential part of the document.
However, in some situations it may be desirable to compose
a document from a collection of parts
without having mandatory page breaks between then.
For this case, the package
provides a mechanism to include parts
by |\input| which can also be processed individually.
However, by construction this mechanism
requires manual handling of the content to be output.

%%%%%%%%%%%%%%%%%%%%%%%%%%%%%%%%%%%%%%%%
\DescribeMacro{\ifchilddocmanual}
The main file should be prepared as usual, see \secref{sec:include}.
However, the document body must make a distinction
between processing of an individual part and of the main document, e.g.:
%
\begin{center}
\begin{tabular}{l}
|\ifchilddocmanual|\\
|\input{\childdocname}|\\
|\||else|\\
\textit{document body with }|\input{|\textit{part}|}|\\
|\||fi|
\end{tabular}
\end{center}
%
The conditional |\ifchilddocmanual| is true whenever
a part to be included by |\input| is being compiled,
and the name of the part is stored in |\childdocname|.

%%%%%%%%%%%%%%%%%%%%%%%%%%%%%%%%%%%%%%%%
\DescribeMacro{\childdocby}
Each part to be included by |\input| should start with:
%
\begin{center}
\begin{tabular}{l}
|\input{childdoc.def}|\\
|\childdocby{|\textit{main}|}|\\
\end{tabular}
\end{center}
%
The directive |\childdocby| is similar to |\childdocof|
described in \secref{sec:include},
but the subsequent selection of content must be done manually.
To that end, both |\ifchilddoc| and |\ifchilddocmanual|
will be true upon processing of a part,
and the name of the part is stored in |\childdocname|.
Note that |\jobname| will be set to the filename of the current part
so that each part receives an individual |.aux| file
that does not interfere with the |.aux| file(s) of the main document.
This behaviour can be altered by the alternative form
|\childdocby[*]{|\textit{main}|}| (with a non-empty optional argument)
which uses the |.aux| file of the main document
by setting |\jobname| to \textit{main}.

%%%%%%%%%%%%%%%%%%%%%%%%%%%%%%%%%%%%%%%%%%%%%%%%%%%%%%%%%%%%%%%%%%%%%%%%%%%%%%%%
\subsection{Driver Development}
\label{sec:driver}

The \textsf{childdoc} mechanism can also be use for the development
of definition files such as \LaTeX{} styles or classes.
This case differs from the above setup with multiple parts
included by |\include| in that no |\includeonly| should be invoked.
This can be achieved by starting the include file
(before |\ProvidesPackage|) with:
%
\begin{center}
\begin{tabular}{l}
|\input{childdoc.def}|\\
|\childdocforward{|\textit{main}|}|\\
\end{tabular}
\end{center}
%
or alternatively with:
%
\begin{center}
\begin{tabular}{l}
|\input{childdoc.def}|\\
|\childdocby{|\textit{main}|}|\\
\end{tabular}
\end{center}
%
Both forms have slightly different effects as described above.
The main file is prepared as usual, see \secref{sec:include}.

%%%%%%%%%%%%%%%%%%%%%%%%%%%%%%%%%%%%%%%%%%%%%%%%%%%%%%%%%%%%%%%%%%%%%%%%%%%%%%%%
\subsection{Legacy Detection}
\label{sec:detection}

The directive |\childdocmain| in the main file can detect
whether the complete document or merely a child is to be compiled
even without using the directive |\childdocof|.
This method is deprecated because it is less robust
and there is no compelling reason to use it;
it is merely provided for backward compatibility
and it may be removed in future versions.

If the detection mechanism is to be used,
it is mandatory to correctly specify
the filename of the main file as the argument of |\childdocmain|:
%
\begin{center}
\begin{tabular}{l}
|\input{childdoc.def}|\\
|\childdocmain{|\textit{main}|}|\\
\end{tabular}
\end{center}
%
If |\jobname| does not match the argument \textit{main} of |\childdocmain|,
it is assumed that |\jobname| points to the child file to be compiled.
When using |\childdocmain| with the main file specified as argument,
it suffices to start a child file
with just |\input{|\textit{main}|}|
without loading of the package and using |\childdocof|.
If instead all processing is done
with the appropriate \textsf{childdoc} directives,
the argument of \textit{main} of |\childdocmain| can be empty.

An alternative version of the command line processing described
in \secref{sec:commandline} using the detection mechanism reads:
%
\begin{center}
|... -jobname "|\textit{target}|" "|[\textit{flags}]%
[|\def\jobname{|\textit{dest}|}|]|\input{|\textit{main}|}"|
\end{center}

%%%%%%%%%%%%%%%%%%%%%%%%%%%%%%%%%%%%%%%%%%%%%%%%%%%%%%%%%%%%%%%%%%%%%%%%%%%%%%%%
\subsection{Manual Code}
\label{sec:manual}

In case one cannot be certain whether the definitions file |childdoc.def|
is installed on the target \TeX{} distribution
and one prefers not to ship it,
it is conceivable to paste a few relevant commands into the sources.

To that end, drop all statements |\input{childdoc.def}|
and perform the replacements as outlined below.
Instead of |\childdocmain{|\textit{main}|}| add the following code
to the top of the main file:
%
\begin{center}
\begin{tabular}{l}
|\||ifdefined\childdocname\endinput\||fi\newif\ifchilddoc|\\
|\edef\childdocname{\scantokens\expandafter{\jobname\noexpand}}|\\
|\def\childdocmain{|\textit{main}|}\||ifx\childdocmain\childdocname\||else|\\
|\childdoctrue\includeonly{\childdocname}\let\jobname\childdocmain\||fi|\\
\end{tabular}
\end{center}
%
Instead of |\childdocof{|\textit{main}|}| just include the main file
at the top of each child file:
%
\begin{center}
|\input{|\textit{main}|}|
\end{center}
%
A simple redirection |\childdocforward{|\textit{dest}|}| is achieved by:
%
\begin{center}
|\def\jobname{|\textit{dest}|}\input{\jobname}|
\end{center}
%
The redirection with prefix
|\childdocforwardprefix[|\textit{prefix}|]{|\textit{dest}|}|
is accomplished by:
%
\begin{center}
\begin{tabular}{l}
|{\edef\jobname{\scantokens\expandafter{\jobname\noexpand}}|\\
|\def\redirectjob |\textit{prefix}|#1~~~{\gdef\jobname{|\textit{dest}|#1}}|\\
|\expandafter\redirectjob\jobname~~~}\input{\jobname}|
\end{tabular}
\end{center}

In an alternative approach,
child documents can be compiled by a specific command line
without additional code or specific definitions:
%
\begin{center}
|... -jobname "|\textit{target}|" "|[\textit{flags}]%
|\includeonly{|\textit{dest}|}\input{|\textit{main}|}"|
\end{center}
%

%%%%%%%%%%%%%%%%%%%%%%%%%%%%%%%%%%%%%%%%%%%%%%%%%%%%%%%%%%%%%%%%%%%%%%%%%%%%%%%%
%%%%%%%%%%%%%%%%%%%%%%%%%%%%%%%%%%%%%%%%%%%%%%%%%%%%%%%%%%%%%%%%%%%%%%%%%%%%%%%%
\section{Information}

%%%%%%%%%%%%%%%%%%%%%%%%%%%%%%%%%%%%%%%%%%%%%%%%%%%%%%%%%%%%%%%%%%%%%%%%%%%%%%%%
\subsection{Copyright}

Copyright \copyright{} 2017--2018 Niklas Beisert

This work may be distributed and/or modified under the
conditions of the \LaTeX{} Project Public License, either version 1.3
of this license or (at your option) any later version.
The latest version of this license is in
  \url{http://www.latex-project.org/lppl.txt}
and version 1.3 or later is part of all distributions of \LaTeX{}
version 2005/12/01 or later.

This work has the LPPL maintenance status `maintained'.

The Current Maintainer of this work is Niklas Beisert.

This work consists of the files |README.txt|, |childdoc.ins| and |childdoc.dtx|
as well as the derived files |childdoc.def|, |cdocsamp.tex|
with |cdocsch1.tex|, |cdocsch2.tex|, |cdocspt3.tex|, |cdocspt4.tex|,
|cdocsdrf.tex|, |cdocsfn1.tex|, |cdocsfn2.tex|
as well as |childdoc.pdf|.

%%%%%%%%%%%%%%%%%%%%%%%%%%%%%%%%%%%%%%%%%%%%%%%%%%%%%%%%%%%%%%%%%%%%%%%%%%%%%%%%
\subsection{Files and Installation}

The package consists of the files:
%
\begin{center}
\begin{tabular}{ll}
    |README.txt|   & readme file \\
    |childdoc.ins| & installation file \\
    |childdoc.dtx| & source file \\
    |childdoc.def| & definition file \\
    |cdocsamp.tex| & sample main file \\
    |cdocsch1.tex| & sample include file \\
    |cdocsch2.tex| & sample include file \\
    |cdocspt3.tex| & sample part file \\
    |cdocspt4.tex| & sample part file \\
    |cdocsdrf.tex| & sample redirection file \\
    |cdocsfn1.tex| & sample redirection file \\
    |cdocsfn2.tex| & sample redirection file \\
    |childdoc.pdf| & manual
\end{tabular}
\end{center}
%
The distribution consists of the files
|README.txt|, |childdoc.ins| and |childdoc.dtx|.
%
\begin{itemize}
\item
Run (pdf)\LaTeX{} on |childdoc.dtx|
to compile the manual |childdoc.pdf| (this file).
\item
Run \LaTeX{} on |childdoc.ins| to create the definitions file |childdoc.def|
and the sample |cdocsamp.tex| with include files
|cdocsch1.tex|, |cdocsch2.tex|, |cdocspt3.tex|, |cdocspt4.tex|,
|cdocsdrf.tex|, |cdocsfn1.tex|, |cdocsfn2.tex|.
Then copy the file |childdoc.def| to an appropriate directory of your \LaTeX{}
distribution, e.g.\ \textit{texmf-root}|/tex/latex/childdoc|.
\end{itemize}

%%%%%%%%%%%%%%%%%%%%%%%%%%%%%%%%%%%%%%%%%%%%%%%%%%%%%%%%%%%%%%%%%%%%%%%%%%%%%%%%
\subsection{Related CTAN Packages}

There are several other packages which offer a similar functionality:
%
\begin{itemize}
\item
The packages
\href{http://ctan.org/pkg/docmute}{\textsf{docmute}},
\href{http://ctan.org/pkg/includex}{\textsf{includex}} and
\href{http://ctan.org/pkg/standalone}{\textsf{standalone}}
provide commands to include only the document body of
a child file thus allowing both files to be compiled individually.
\item
The packages \href{http://ctan.org/pkg/subdocs}{\textsf{subdocs}}
and \href{http://ctan.org/pkg/subfiles}{\textsf{subfiles}}
provide structures in which the main and child documents can be
encapsulated and allowing them to be compiled individually.
The inclusion mechanism is different from the conventional |\include|.
\item
The package \href{http://ctan.org/pkg/combine}{\textsf{combine}}
is an elaborate solution to combine several documents into one.
\end{itemize}
%
See also the CTAN topic \href{http://ctan.org/topic/subdocs}{\textsf{subdocs}}
for further related packages.
The present package differs from the above solutions in that
a document structure constructed with the conventional |\include| mechanism
just needs two extra commands at the top of every file
such that all constituent files can be compiled individually.

%%%%%%%%%%%%%%%%%%%%%%%%%%%%%%%%%%%%%%%%%%%%%%%%%%%%%%%%%%%%%%%%%%%%%%%%%%%%%%%%
%\subsection{Feature Suggestions}
%
%The following is a list of features which may be useful for future
%versions of this package:
%%
%\begin{itemize}
%\item
%\ldots
%\end{itemize}

%%%%%%%%%%%%%%%%%%%%%%%%%%%%%%%%%%%%%%%%%%%%%%%%%%%%%%%%%%%%%%%%%%%%%%%%%%%%%%%%
\subsection{Revision History}

%%%%%%%%%%%%%%%%%%%%%%%%%%%%%%%%%%%%%%%%
\paragraph{v2.0:} 2018/12/30

\begin{itemize}
\item
immediate forward processing
\item
added |\childdocby| mechanism
\item
manual restructured
\end{itemize}

%%%%%%%%%%%%%%%%%%%%%%%%%%%%%%%%%%%%%%%%
\paragraph{v1.6:} 2018/01/17

\begin{itemize}
\item
application for development of include files
\item
corrections to manual
\end{itemize}

%%%%%%%%%%%%%%%%%%%%%%%%%%%%%%%%%%%%%%%%
\paragraph{v1.5:} 2017/05/21

\begin{itemize}
\item
more complete structuring introduced
\item
|\childdocof| introduced
\item
|\childdoc| renamed to |\childdocmain|
\item
|\childredirect| renamed to |\childdocforward| and |\childdocforwardprefix|
and functionality expanded
\end{itemize}

%%%%%%%%%%%%%%%%%%%%%%%%%%%%%%%%%%%%%%%%
\paragraph{v1.0:} 2017/04/27

\begin{itemize}
\item
manual and install package
\item
first version published on CTAN
\end{itemize}

%%%%%%%%%%%%%%%%%%%%%%%%%%%%%%%%%%%%%%%%
\paragraph{v0.6:} 2017/04/26

\begin{itemize}
\item
redirection mechanism added
\end{itemize}

%%%%%%%%%%%%%%%%%%%%%%%%%%%%%%%%%%%%%%%%
\paragraph{v0.5:} 2017/04/26

\begin{itemize}
\item
functionality in definition file
\end{itemize}


%%%%%%%%%%%%%%%%%%%%%%%%%%%%%%%%%%%%%%%%%%%%%%%%%%%%%%%%%%%%%%%%%%%%%%%%%%%%%%%%
%%%%%%%%%%%%%%%%%%%%%%%%%%%%%%%%%%%%%%%%%%%%%%%%%%%%%%%%%%%%%%%%%%%%%%%%%%%%%%%%
%%%%%%%%%%%%%%%%%%%%%%%%%%%%%%%%%%%%%%%%%%%%%%%%%%%%%%%%%%%%%%%%%%%%%%%%%%%%%%%%
\appendix

\settowidth\MacroIndent{\rmfamily\scriptsize 000\ }

 \DocInput{childdoc.dtx}

\end{document}
%</driver>
% \fi
%
% %%%%%%%%%%%%%%%%%%%%%%%%%%%%%%%%%%%%%%%%%%%%%%%%%%%%%%%%%%%%%%%%%%%%%%%%%%%%%%
% %%%%%%%%%%%%%%%%%%%%%%%%%%%%%%%%%%%%%%%%%%%%%%%%%%%%%%%%%%%%%%%%%%%%%%%%%%%%%%
% \section{Sample}
%\iffalse
%<*samplemain>
%\fi
%
% The following presents a sample document
% with two chapters, two parts, a title page,
% a compile flag as well as three forwarding files to set the flag.
% It consists of eight |.tex| files:
% \begin{center}
% \begin{tabular}{ll}
% |cdocsamp.tex|&main file\\
% |cdocsch1.tex|&include file for chapter 1\\
% |cdocsch2.tex|&include file for chapter 2\\
% |cdocspt3.tex|&include file for part 3\\
% |cdocspt4.tex|&include file for part 4\\
% |cdocsdrf.tex|&forwarding file for main file in draft mode\\
% |cdocsfi1.tex|&forwarding file for final version of chapter 1\\
% |cdocsfi2.tex|&forwarding file for final version of chapter 2\\
% \end{tabular}
% \end{center}
% Each of the eight files can be compiled directly by the \LaTeX{} compiler.
%
% %%%%%%%%%%%%%%%%%%%%%%%%%%%%%%%%%%%%%%
% \paragraph{Main File.}
%
% The main file is called |cdocsamp.tex|.
%
% Load the \textsf{childdoc} definitions and
% declare the filename for the main document:
%    \begin{macrocode}
\input{childdoc.def}
\childdocmain{}
%    \end{macrocode}

% Optional override for |\version| flag:
%    \begin{macrocode}
%%\ifchilddoc\else\providecommand{\version}{draft}\fi
%    \end{macrocode}

% Define the default values for the |\version| flag
% (|final| for the main file and |draft| for childs):
%    \begin{macrocode}
\ifchilddoc
\providecommand{\version}{draft}
\else
\providecommand{\version}{final}
\fi
%    \end{macrocode}

% Load the standard document class:
%    \begin{macrocode}
\documentclass[12pt]{article}
%    \end{macrocode}

% Start the document body:
%    \begin{macrocode}
\begin{document}
%    \end{macrocode}

% Declare a title page.
% Print title, part of document being processed and version flag:
%    \begin{macrocode}
\addtocounter{page}{-1}
\begin{center}
{\LARGE\bfseries{}childdoc example\par}
\vspace{1cm}
\ifchilddoc
\ifchilddocmanual part\else chapter\fi:
`\childdocname' of `\childdocjob'\par
\else
main document: `\childdocjob'\par
\fi
version: \version\par
\end{center}
\newpage
%    \end{macrocode}

% Manually include selected file,
% otherwise process as usual:
%    \begin{macrocode}
\ifchilddocmanual
\section*{part `\childdocname'}
\input{\childdocname}
\else
%    \end{macrocode}

% Include the two chapters:
%    \begin{macrocode}
\include{cdocsch1}
\include{cdocsch2}
%    \end{macrocode}

% Include the two parts unless only chapters should be displayed:
%    \begin{macrocode}
\ifchilddoc\else
\section{part three}
\input{cdocspt3}
\section{part four}
\input{cdocspt4}
\fi
%    \end{macrocode}

% Process as usual until here:
%    \begin{macrocode}
\fi
%    \end{macrocode}

% End of document body:
%    \begin{macrocode}
\end{document}
%    \end{macrocode}
%\iffalse
%</samplemain>
%\fi
%
% %%%%%%%%%%%%%%%%%%%%%%%%%%%%%%%%%%%%%%
% \paragraph{Chapter Include Files.}
%
% The include files are called |cdocsch1.tex| and |cdocsch2.tex|.
%
%\iffalse
%<*samplechap1|samplechap2>
%\fi

% Optional override for |\version| flag:
%    \begin{macrocode}
%%\providecommand{\version}{final}
%    \end{macrocode}

% Include the main document:
%    \begin{macrocode}
\input{childdoc.def}
\childdocof{cdocsamp}
%    \end{macrocode}

%\iffalse
%</samplechap1|samplechap2>
%\fi
%
%\iffalse
%<*samplechap1>
%\fi
% Some text for chapter 1:
%    \begin{macrocode}
\section{one}
some text in chapter one
%    \end{macrocode}

%\iffalse
%</samplechap1>
%\fi
% Some text for chapter 2:
%\iffalse
%<*samplechap2>
%\fi
%    \begin{macrocode}
\section{two}
more text in chapter two
%    \end{macrocode}

%\iffalse
%</samplechap2>
%\fi
%
% %%%%%%%%%%%%%%%%%%%%%%%%%%%%%%%%%%%%%%
% \paragraph{Part Include Files.}
%
% The include files are called |cdocspt3.tex| and |cdocspt4.tex|.
%
%\iffalse
%<*samplepart3|samplepart4>
%\fi

% Optional override for |\version| flag:
%    \begin{macrocode}
%%\providecommand{\version}{final}
%    \end{macrocode}

% Include the main document:
%    \begin{macrocode}
\input{childdoc.def}
\childdocby{cdocsamp}
%    \end{macrocode}

%\iffalse
%</samplepart3|samplepart4>
%\fi
%
%\iffalse
%<*samplepart3>
%\fi
% Some text for part 3:
%    \begin{macrocode}
some text in part three
%    \end{macrocode}

%\iffalse
%</samplepart3>
%\fi
% Some text for part 4:
%\iffalse
%<*samplepart4>
%\fi
%    \begin{macrocode}
more text in part four
%    \end{macrocode}

%\iffalse
%</samplepart4>
%\fi
%
% %%%%%%%%%%%%%%%%%%%%%%%%%%%%%%%%%%%%%%
% \paragraph{Forwarding for a Complete Draft.}
%
% The following forwarding file |cdocsdrf.tex|
% compiles the main document in draft mode:
%\iffalse
%<*sampledraft>
%\fi
%    \begin{macrocode}
\def\version{draft}
\input{childdoc.def}
\childdocforward{cdocsamp}
%    \end{macrocode}

%\iffalse
%</sampledraft>
%\fi
%
% %%%%%%%%%%%%%%%%%%%%%%%%%%%%%%%%%%%%%%
% \paragraph{Forwarding for Final Version of the Chapters.}
%
% The following forwarding files |cdocsfn1.tex| and |cdocsfn2.tex|
% (with identical content)
% compile the final versions of the child documents
% |cdocsch1.tex| and |cdocsch2.tex|, respectively:
%\iffalse
%<*samplefinal>
%\fi
%    \begin{macrocode}
\def\version{final}
\input{childdoc.def}
\childdocforwardprefix[cdocsamp]{cdocsfn}{cdocsch}
%    \end{macrocode}

%\iffalse
%</samplefinal>
%\fi
%
% %%%%%%%%%%%%%%%%%%%%%%%%%%%%%%%%%%%%%%
% \paragraph{Command Line Processing.}
%
% The following three command lines generate the output files
% |cdocscld|, |cdocscl1| and |cdocscl2|
% which should be identical to
% |cdocsdrf|, |cdocsch1| and |cdocsfn2|, respectively:
% \begin{center}
% \begin{tabular}{l}
% |latex -jobname cdocscld \|\\
% |  "\def\version{draft}\input{childdoc.def}\childdocforward{cdocsamp}"|\\
% |latex -jobname cdocscl1 \|\\
% |  "\input{childdoc.def}\childdocforward[cdocsamp]{cdocsch1}"|\\
% |latex -jobname cdocscl2 \|\\
% |  "\def\version{final}\input{childdoc.def}\childdocforward{cdocsch2}"|
% \end{tabular}
% \end{center}
% Note that the trailing backslash on each first line
% merely continues the input to the second line
% (for convenient cut ant paste).
% Furthermore, the command |latex| can be replaced by any
% of its alternative versions such as |pdflatex|.
%
% %%%%%%%%%%%%%%%%%%%%%%%%%%%%%%%%%%%%%%%%%%%%%%%%%%%%%%%%%%%%%%%%%%%%%%%%%%%%%%
% %%%%%%%%%%%%%%%%%%%%%%%%%%%%%%%%%%%%%%%%%%%%%%%%%%%%%%%%%%%%%%%%%%%%%%%%%%%%%%
% \section{Implementation}
%\iffalse
%<*package>
%\fi
%
% This section describes the definitions file |childdoc.def|.

% The definitions cannot be loaded using |\usepackage| or |\RequirePackage|
% which has a mechanism to prevent loading a style file more than once.
% When loading the definitions by means of |\input|
% multiple instances have to be prevented manually:
%\iffalse
%This code needs to be before the `\ProvidesFile' directive
%which is defined at the beginning of this file.
%Therefore it is also placed there and commented out here.
%</package>
%<*discard>
%\fi
%    \begin{macrocode}
\ifdefined\childdocmain\endinput\fi
%    \end{macrocode}
%\iffalse
%</discard>
%<*package>
%\fi
%
% \macro{\ifchilddoc}
% \macro{\ifchilddocmanual}
% The conditional |\ifchilddoc| tells whether a
% child (true) or main (false) document is being compiled.
% The conditional |\ifchilddocmanual| tells whether
% the |\includeonly| mechanism is used (false) or
% the selection of child files must be performed manually (true).
% The definitions initialise to false:
%    \begin{macrocode}
\newif\ifchilddoc
\newif\ifchilddocmanual
%    \end{macrocode}

% \macro{\childdocname}
% \macro{\childdocjob}
% The macro |\childdocname| stores the name of the main document
% to be compiled. The macro |\childdocjob| stores the name of
% the document on which the \LaTeX{} compiler was originally invoked.
% The content of |\jobname| cannot be compared
% to filenames specified in the source due to different catcodes.
% The following code rescans |\jobname|, stores the result
% in |\childdocname| and saves a copy in |\childdocjob|:
%    \begin{macrocode}
\edef\childdocname{\scantokens\expandafter{\jobname\noexpand}}
\let\childdocjob\childdocname
%    \end{macrocode}

% \macro{\childdocdisable}
% The macro |\childdocdisable| prevents the main file
% from being processed more than once.
% At this stage, the main document command |\childdocmain|
% is assumed to be called once again where it should do nothing.
% Any subsequent call to it should prevent
% a secondary processing of the main document
% It overwrites the forwarding commands
% |\childdocof| and |\childdocforward|
% with empty macros to prevent further inclusions of the main document:
%    \begin{macrocode}
\newcommand{\childdocdisable}
{
  \renewcommand{\childdocmain}[1]{\renewcommand{\childdocmain}[1]{\endinput}}
  \renewcommand{\childdocof}[1]{}
  \renewcommand{\childdocby}[2][]{}
  \renewcommand{\childdocforward}[2][]{}
  \renewcommand{\childdocdisable}{}
}
%    \end{macrocode}

% \macro{\childdocmain}
% The macro |\childdocmain| is to be called at the top of the main file
% with nothing or the main filename (without extension) as argument.
% First, it breaks loops.
% If the argument is not empty and does not match |\childdocname|
% (which is set by the first inclusion of |childdoc.def|),
% |\ifchilddoc| is set to true, |\includeonly| is applied to the child file
% and |\jobname| is set to the main file
% (for proper handling of |.aux| files):
%    \begin{macrocode}
\newcommand{\childdocmain}[1]
{
  \childdocdisable\childdocmain{}
  \if?#1?\else
    \begingroup
      \def\childdoctmp{#1}
      \ifx\childdoctmp\childdocname
        \def\childdoctmp{}
      \else
        \def\childdoctmp
        {
          \childdoctrue
          \includeonly{\childdocname}
          \def\childdocjob{#1}
          \def\jobname{#1}
        }
      \fi
      \expandafter
    \endgroup
    \childdoctmp
  \fi
}
%    \end{macrocode}

% \macro{\childdocof}
% The command |\childdocof| redirects
% compilation to the main file |#1|.
%    \begin{macrocode}
\newcommand{\childdocof}[1]
{
  \childdocdisable
  \childdoctrue
  \includeonly{\childdocname}
  \def\jobname{#1}
  \def\childdocjob{#1}
  \input{#1}
}
%    \end{macrocode}

% \macro{\childdocby}
% The command |\childdocby| ....
%    \begin{macrocode}
\newcommand{\childdocby}[2][]
{
  \childdocdisable
  \childdoctrue
  \childdocmanualtrue
  \if?#1?\else
    \def\jobname{#2}
  \fi
  \def\childdocjob{#2}
  \input{#2}
  \endinput
}
%    \end{macrocode}

% \macro{\childdocforward}
% The command |\childdocforward| redirects
% compilation to the main file or
% (if the optional argument is given) a child file.
% Parameters are set as if the main file
% or a child file starting with |\childdocof| was compiled.
% Then compilation is handed over to the main file:
%    \begin{macrocode}
\newcommand{\childdocforward}[2][]
{
  \begingroup
    \if?#1?
      \def\childdoctmp
      {
        \def\childdocname{#2}
        \def\childdocjob{#2}
        \def\jobname{#2}
        \input{#2}
        \endinput
      }
    \else
      \def\childdoctmp
      {
        \childdocdisable
        \def\childdocname{#2}
        \childdoctrue
        \includeonly{#2}
        \def\childdocjob{#1}
        \def\jobname{#1}
        \input{#1}
        \endinput
      }
    \fi
    \expandafter
  \endgroup
  \childdoctmp
}
%    \end{macrocode}

% \macro{\childdocforwardprefix}
% The command |\childdocforwardprefix| redirects
% compilation to the main or a child file by means of a pattern.
% The prefix |#1| in the current filename is replaced by |#2|
% and the suffix of the current filename is kept
% (it is assumed that the filename does not contain the substring `|~~~|'
% which is used as a delimiter).
% Compilation is handed over to the new file by |\childdocforward|:
%    \begin{macrocode}
\newcommand{\childdocforwardprefix}[3][]
{
  \begingroup
    \def\childdocextract #2##1~~~{\def\childdoctmp{\childdocforward[#1]{#3##1}}}
    \expandafter\childdocextract\childdocname~~~
    \expandafter
  \endgroup
  \childdoctmp
}
%    \end{macrocode}

% \macro{\childdoc}
% The deprecated macro |\childdoc| is a legacy version of |\childdocmain|:
%    \begin{macrocode}
\newcommand{\childdoc}{\childdocmain}
%    \end{macrocode}

% \macro{\childdocredirect}
% The deprecated macro |\childdocredirect| is a legacy version
% of |\childdocforward| and |\childdocforwardprefix|:
%    \begin{macrocode}
\newcommand{\childdocredirect}[2][]
{
  \begingroup
    \if?#1?
      \def\childdoctmp{\childdocforward{#2}}
    \else
      \def\childdoctmp{\childdocforwardprefix{#1}{#2}}
    \fi
    \expandafter
  \endgroup
  \childdoctmp
}
%    \end{macrocode}

%\iffalse
%</package>
%\fi
%
\endinput
|\\
|\childdocforward{|\textit{main}|}|\\
\end{tabular}
\end{center}
%
or alternatively with:
%
\begin{center}
\begin{tabular}{l}
|% \iffalse
%
% childdoc.dtx Copyright (C) 2017-2018 Niklas Beisert
%
% This work may be distributed and/or modified under the
% conditions of the LaTeX Project Public License, either version 1.3
% of this license or (at your option) any later version.
% The latest version of this license is in
%   http://www.latex-project.org/lppl.txt
% and version 1.3 or later is part of all distributions of LaTeX
% version 2005/12/01 or later.
%
% This work has the LPPL maintenance status `maintained'.
%
% The Current Maintainer of this work is Niklas Beisert.
%
% This work consists of the files childdoc.dtx and childdoc.ins
% and the derived files childdoc.def and cdocsamp.tex with
% cdocsch1.tex, cdocsch2.tex, cdocsdrf.tex, cdocsfn1.tex, cdocsfn2.tex.
%
%<package>\ifdefined\childdocmain\endinput\fi
%<package>\ProvidesFile{childdoc.def}[2018/12/30 v2.0 child document driver]
%<samplemain>\ProvidesFile{cdocsamp.tex}[2018/12/30 v2.0 sample for childdoc]
%<*driver>
%\ProvidesFile{childdoc.drv}[2018/12/30 v2.0 childdoc reference manual file]
\PassOptionsToClass{10pt,a4paper}{article}
\documentclass{ltxdoc}

\usepackage[margin=35mm]{geometry}
\usepackage{hyperref}
\usepackage{hyperxmp}
\usepackage[usenames]{color}

\hypersetup{colorlinks=true}
\hypersetup{pdfstartview=FitH}
\hypersetup{pdfpagemode=UseNone}
\hypersetup{pdfsource={}}
\hypersetup{pdflang={en-UK}}
\hypersetup{pdfcopyright={Copyright 2017-2018 Niklas Beisert.
  This work may be distributed and/or modified under the
  conditions of the LaTeX Project Public License, either version 1.3
  of this license or (at your option) any later version.}}
\hypersetup{pdflicenseurl={http://www.latex-project.org/lppl.txt}}
\hypersetup{pdfcontactaddress={ETH Zurich, ITP, HIT K,
  Wolfgang-Pauli-Strasse 27}}
\hypersetup{pdfcontactpostcode={8093}}
\hypersetup{pdfcontactcity={Zurich}}
\hypersetup{pdfcontactcountry={Switzerland}}
\hypersetup{pdfcontactemail={nbeisert@itp.phys.ethz.ch}}
\hypersetup{pdfcontacturl={http://people.phys.ethz.ch/\xmptilde nbeisert/}}

\newcommand{\secref}[1]{\hyperref[#1]{section \ref*{#1}}}

\parskip1ex
\parindent0pt
\let\olditemize\itemize
\def\itemize{\olditemize\parskip0pt}

\begin{document}

\title{The \textsf{childdoc} Package}
\hypersetup{pdftitle={The childdoc Package}}
\author{Niklas Beisert\\[2ex]
  Institut f\"ur Theoretische Physik\\
  Eidgen\"ossische Technische Hochschule Z\"urich\\
  Wolfgang-Pauli-Strasse 27, 8093 Z\"urich, Switzerland\\[1ex]
  \href{mailto:nbeisert@itp.phys.ethz.ch}
  {\texttt{nbeisert@itp.phys.ethz.ch}}}
\hypersetup{pdfauthor={Niklas Beisert}}
\hypersetup{pdfsubject={Manual for the LaTeX2e Package childdoc}}
\date{30 December 2018, \textsf{v2.0}}
\maketitle

\begin{abstract}\noindent
\textsf{childdoc} is a \LaTeXe{} package
that enables the direct compilation
of document sections included by |\include|
to individual files.
\end{abstract}

\begingroup
\parskip0ex
\tableofcontents
\endgroup

%%%%%%%%%%%%%%%%%%%%%%%%%%%%%%%%%%%%%%%%%%%%%%%%%%%%%%%%%%%%%%%%%%%%%%%%%%%%%%%%
%%%%%%%%%%%%%%%%%%%%%%%%%%%%%%%%%%%%%%%%%%%%%%%%%%%%%%%%%%%%%%%%%%%%%%%%%%%%%%%%
\section{Introduction}

\LaTeX{} provides a mechanism to structure a large document (such as a book)
into a main file and several child files (containing the chapters)
using the |\include| command.
This mechanism is beneficial for documents
which span hundreds of pages in order to
make the source file(s) more manageable.
Moreover, compilation can be restricted to
selected child files by means of the |\includeonly| command.
The latter feature can be used to reduce the compilation time while editing
(this was significantly more useful in the earlier days of \LaTeX{})
or to generate a smaller document which is easier to navigate.
Another application of |\includeonly| is to generate
documents consisting of selected parts of the complete document.

However, there are a few drawbacks of the plain |\include| mechanism:
\begin{itemize}
\item
The child files cannot be compiled on their own,
they can only be compiled via the main file.
A naive editing environment
(such as a text editor with an option
to have the current file processed by \LaTeX)
may require one to switch to the main file before compiling;
attempting to compile the child file produces errors.
\item
The main file must be modified (each time)
to adjust the |\includeonly| command
to the present needs. This easily leaves the main file in a messy state.
\item
The generated document will always carry the filename
of the main document. This is inconvenient if
several child files are to be compiled and
to be kept for distribution.
\end{itemize}

The present package provides a simple interface
to make child files individually compilable by \LaTeX{}.
Compiling a child file then has the same effect as compiling
the main file with an |\includeonly| command
to select the appropriate child.
Moreover the generated document will carry the name of the child
rather than the main file.
This resolves all three above issues.

This feature is meant to make the editing of books,
thesis documents and lecture notes somewhat more convenient.
However, the package can also be used efficiently for
composing a series of documents (such as exercise sheets)
which are typically distributed individually.
It then assists the author in generating the individual documents
(potentially in different versions)
as well as a document containing the collected series.
Another application is in developing style files
or other kinds of included material
where compilation of the style file could redirect
to a sample or test file.

%%%%%%%%%%%%%%%%%%%%%%%%%%%%%%%%%%%%%%%%%%%%%%%%%%%%%%%%%%%%%%%%%%%%%%%%%%%%%%%%
%%%%%%%%%%%%%%%%%%%%%%%%%%%%%%%%%%%%%%%%%%%%%%%%%%%%%%%%%%%%%%%%%%%%%%%%%%%%%%%%
\section{Usage}

First of all, the package \textsf{childdoc} is \emph{not} a standard
\LaTeXe{} |.sty| style file! Therefore it needs to be invoked in
a non-standard way.

%%%%%%%%%%%%%%%%%%%%%%%%%%%%%%%%%%%%%%%%%%%%%%%%%%%%%%%%%%%%%%%%%%%%%%%%%%%%%%%%
\subsection{Included Files}
\label{sec:include}

%%%%%%%%%%%%%%%%%%%%%%%%%%%%%%%%%%%%%%%%
\DescribeMacro{\childdocmain}
To use the package, add the commands
\begin{center}
\begin{tabular}{l}
|\input{childdoc.def}|\\
|\childdocmain{}|\\
\end{tabular}
\end{center}
at the very top of the main \LaTeX{} file,
in particular \emph{before} the |\documentclass| statement!
The argument of |\childdocmain| should be left empty
(but it must be present).

%%%%%%%%%%%%%%%%%%%%%%%%%%%%%%%%%%%%%%%%
\DescribeMacro{\childdocof}
Furthermore, add the commands
\begin{center}
\begin{tabular}{l}
|\input{childdoc.def}|\\
|\childdocof{|\textit{main}|}|\\
\end{tabular}
\end{center}
at the top of every child file \textit{child}
which is included by |\include{|\textit{child}|}|
from within the main file
(or at least for those files to be compiled individually).
The argument \textit{main} must be the filename of the main file.

There are a couple of
considerations in setting up the main and child documents:

%%%%%%%%%%%%%%%%%%%%%%%%%%%%%%%%%%%%%%%%
\paragraph{Restrictions.}

Please note the following restrictions:
\begin{itemize}
\item
|\childdocmain| must be called with one argument \textit{main}
to ensure compatibility with earlier version of the package.
It must either be empty (|\childdocmain{}|)
or precisely match the filename of the main file in which it is specified.
See \secref{sec:detection} for further information.
\item
The filename \textit{main} must be specified without the |.tex| extension.
\item
The filename \textit{main} is case sensitive
(even in case-insensitive file systems)
due to internal string comparison.
\item
The argument \textit{main} should be fully expanded, it cannot be a macro.
\item
Subdirectories and special characters should be avoided in filenames.
\item
The command |\childdocmain{|\textit{main}|}| must be followed by a whitespace.
It should not be followed immediately by another command
or by a comment mark `|%|'.
This is because the \TeX{} parser reads the token immediately following
the argument of |\childdocmain| and puts it
at the beginning of every child section;
however, a white\-space is ignored.
\end{itemize}

%%%%%%%%%%%%%%%%%%%%%%%%%%%%%%%%%%%%%%%%
\paragraph{Content of Main File.}

It is advisable to place all content in the child files included by |\include|.
Any output contained in the main file will appear in all child documents
unless suppressed manually;
it cannot be suppressed automatically by the |\includeonly| directive
and thus should normally be avoided.
A method to include some content in the main file
by means of conditional processing is described in \secref{sec:conditional}.

%%%%%%%%%%%%%%%%%%%%%%%%%%%%%%%%%%%%%%%%
\paragraph{Page Numbering.}

When only a part of the document is compiled,
the appropriate numbering of pages
(as well as other status parameters)
is determined from the |.aux| files.
The latter contain information from previous passes.
However this information needs to propagate through
all intermediate child documents.
Therefore the page numbering in child documents may well
be inconsistent until the complete document is compiled at least once.

A useful (if unconventional) way to always ensure a consistent
page numbering is to restart the numbering in each child document
and denote the pages by `\textit{child}|.|\textit{page}'
where \textit{child} represents the chapter/section number of the child file.
This can be achieved by the command
|\numberwithin{page}{|\textit{child}|}|
of the \textsf{amsmath} package
where \textit{child} can be |chapter| or |section|
depending on the chosen structuring.
Alternatively, one can modify the macro |\thepage| appropriately
and reset the counter |page| at the start of each child file.

%%%%%%%%%%%%%%%%%%%%%%%%%%%%%%%%%%%%%%%%%%%%%%%%%%%%%%%%%%%%%%%%%%%%%%%%%%%%%%%%
\subsection{Conditional Processing}
\label{sec:conditional}

The package provides a mechanism to compile different versions
of a document. To customise the versions further some conditional processing
can come in handy to distinguish which version is being compiled.
The package provides two macros to describe the compilation context:

%%%%%%%%%%%%%%%%%%%%%%%%%%%%%%%%%%%%%%%%
\DescribeMacro{\ifchilddoc}
The conditional |\ifchilddoc| distinguishes between the compilation of
child documents and the main document:
%
\begin{center}
|\ifchilddoc |\textit{child-code}| |[|\||else |\textit{main-code}]| \||fi|
\end{center}

%%%%%%%%%%%%%%%%%%%%%%%%%%%%%%%%%%%%%%%%
\DescribeMacro{\childdocname}
\DescribeMacro{\childdocjob}
The macro |\childdocname| contains the filename (without extension)
of the main or child file being processed.
Note that |\childdocjob| will always contain the name of the main file.

%%%%%%%%%%%%%%%%%%%%%%%%%%%%%%%%%%%%%%%%
\paragraph{Title Page.}

Conditional processing can be used to include a title or banner page
in the main document when proper precautions are taken.
Importantly, the code in the main file should ensure that the page counter
(as well as other status parameters which are stored in the |.aux| files)
takes the same value after the conditional processing.
Otherwise the page numbers may take divergent values
depending on which part is compiled.

For example, a title page could be declared by:
%
\begin{center}
\begin{tabular}{l}
|\ifchilddoc\||else|\\
|\addtocounter{page}{-1}|\\
\textit{code for title page}\\
|\newpage|\\
|\||fi|
\end{tabular}
\end{center}
%
A banner page for the child documents can be generated by:
%
\begin{center}
\begin{tabular}{l}
|\ifchilddoc|\\
|\addtocounter{page}{-1}|\\
\textit{code for banner page}\\
|\newpage|\\
|\||fi|
\end{tabular}
\end{center}
%
Here one could write a message such as:
\begin{center}
|This is the part \childdocname{} of \childdocjob{}.|
\end{center}

%%%%%%%%%%%%%%%%%%%%%%%%%%%%%%%%%%%%%%%%%%%%%%%%%%%%%%%%%%%%%%%%%%%%%%%%%%%%%%%%
\subsection{Flags}
\label{sec:flags}

The package makes it easy to generate different versions
of the main or child documents.
To this end compilation flags can be defined
and assigned different default values.
They will be particularly useful in conjunction
with the forwarding mechanism described in \secref{sec:forward}.

For example, it may be useful to have a flag |\version|
which can be set to |draft| or |final|.
The document source will contain some conditional code
depending on the value of |\version|.
Suppose further, the flag should default to |final| for the main file
and to |draft| for child files
which is a natural assignment for editing the document.
This is achieved by placing the following code
in the preamble of the main document
(below the |\childdocmain| directive):
%
\begin{center}
\begin{tabular}{l}
|\ifchilddoc|\\
|\providecommand{\version}{draft}|\\
|\||else|\\
|\providecommand{\version}{final}|\\
|\||fi|
\end{tabular}
\end{center}
%
The definition by |\providecommand| makes sure
that previous definitions are not overwritten.
Further statements |\providecommand{\version}{...}|
can thus be added before the above code to override it.

For the main file, one might add a line
(between |\childdocmain| and the above block)
%
\begin{center}
|%\ifchilddoc\||else\providecommand{\version}{draft}\||fi|
\end{center}
%
which can be uncommented to produce a draft version.
Likewise one can add a line to the very top of a child file
(above the |\childdocof{|\textit{main}|}| directive)
%
\begin{center}
|%\providecommand{\version}{final}|
\end{center}
%
which can be uncommented to produce the final version of this child document.

%%%%%%%%%%%%%%%%%%%%%%%%%%%%%%%%%%%%%%%%%%%%%%%%%%%%%%%%%%%%%%%%%%%%%%%%%%%%%%%%
\subsection{Forwarding}
\label{sec:forward}

Different versions of the main or child documents
using compilation flags as described in \secref{sec:flags}
can be (permanently) stored in different files
for convenient compilation, viewing and distribution.
To this end, the package defines a command
to pass on compilation to a different file:

%%%%%%%%%%%%%%%%%%%%%%%%%%%%%%%%%%%%%%%%
\DescribeMacro{\childdocforward}
The command |\childdocforward| redirects processing to
another source file:
%
\begin{center}
\begin{tabular}{l}
|\input{childdoc.def}|\\
|\childdocforward[|\textit{main}|]{|\textit{dest}|}|\\
\end{tabular}
\end{center}
%
The argument \textit{dest} is the destination file
(without extension).
It should be the main file or one of the child files.
Note that further \textsf{childdoc} directives
such as |\childdocof| and |\childdocforward|
in the indicated file will be processed in this form.
The optional argument \textit{main}
passes on directly to the main file \textit{main}
while pretending to compile the child \textit{dest}.
This form behaves as if \textit{dest}
issues |\childdocof{|\textit{main}|}| right away,
and no further \textsf{childdoc} directives will be processed.

%%%%%%%%%%%%%%%%%%%%%%%%%%%%%%%%%%%%%%%%
\DescribeMacro{\...prefix}
In the alternative form |\childdocforwardprefix|,
%
\begin{center}
\begin{tabular}{l}
|\input{childdoc.def}|\\
|\childdocforwardprefix[|\textit{main}|]{|\textit{prefix}|}{|\textit{dest}|}|
\end{tabular}
\end{center}
%
the destination file is determined by a pattern
depending on the current file:
To make this work, the current file must be called
`{\textit{prefix}\hspace{0.2em}\textit{suffix}}'
with \textit{prefix} matching precisely the argument.
Processing is then passed on to the file
`{\textit{dest}\hspace{0.2em}\textit{suffix}}'.
Surely, the same effect is achieved by
directly specifying the
argument `{\textit{dest}\hspace{0.2em}\textit{suffix}}'
in the first form.
However, that requires to set up a different file
for each child. With the alternative form of the command
all these files can have exactly the same content
which simplifies setting them up and maintaining them.

For example, the following file |draft.tex|
with a compilation flag |\version| as described in \secref{sec:flags}
compiles the main document as a draft:
%
\begin{center}
\begin{tabular}{l}
|\def\version{draft}|\\
|\input{childdoc.def}|\\
|\childdocforward{|\textit{main}|}|
\end{tabular}
\end{center}
%
Likewise, the following files |final|\textit{nn}|.tex|
compile the final version of the child document
|child|\textit{nn}|.tex|:
%
\begin{center}
\begin{tabular}{l}
|\def\version{final}|\\
|\input{childdoc.def}|\\
|\childdocforwardprefix{final}{child}|
\end{tabular}
\end{center}
%

Note that when several versions of a main file and/or of each child file
are to be generated, it may be convenient to set up a |Makefile| or
shell script to automatise the process.

%%%%%%%%%%%%%%%%%%%%%%%%%%%%%%%%%%%%%%%%%%%%%%%%%%%%%%%%%%%%%%%%%%%%%%%%%%%%%%%%
\subsection{Command Line Processing}
\label{sec:commandline}

The effect of redirection files can also be achieved by invoking
the \LaTeX{} compiler with a more elaborate command line.
Most conveniently this should be done as part
of a shell script or a |Makefile|.

When using \textsf{childdoc} in the main file, the following
command lines effectively perform a redirection
(note that depending on the shell being used,
backslashes may have to be doubled: `|\|' $\to$ `|\\|'):
%
\begin{center}
|... -jobname "|\textit{target}|" |\\|"|[\textit{flags}]%
|\input{childdoc.def}\childdocforward[|\textit{main}|]{|\textit{dest}|}"|
\end{center}
%
Here \textit{target} is the name of the output file,
\textit{main} is the name of the main file
and \textit{dest} is the name of the main or child file to be processed
(all filenames without extensions).
The optional argument \textit{main} can be omitted
if \textit{main} matches \textit{dest}.
Optionally, compilation \textit{flags} can be defined via |\def| commands.
This command line makes the \TeX{} engine believe
it is compiling the file \textit{target}
whose content is specified as the latter parameter.
The provided code then forwards the processing to
\textit{main} or \textit{dest} as described in \secref{sec:forward}.

%%%%%%%%%%%%%%%%%%%%%%%%%%%%%%%%%%%%%%%%%%%%%%%%%%%%%%%%%%%%%%%%%%%%%%%%%%%%%%%%
\subsection{Include by Input}
\label{sec:input}

Including child documents by |\include| has some restrictions by design.
Most notably, the content of a child document always occupies
its own set of pages; pages cannot be shared between child documents.
Usually, this behaviour makes perfect sense
because each child document contain an essential part of the document.
However, in some situations it may be desirable to compose
a document from a collection of parts
without having mandatory page breaks between then.
For this case, the package
provides a mechanism to include parts
by |\input| which can also be processed individually.
However, by construction this mechanism
requires manual handling of the content to be output.

%%%%%%%%%%%%%%%%%%%%%%%%%%%%%%%%%%%%%%%%
\DescribeMacro{\ifchilddocmanual}
The main file should be prepared as usual, see \secref{sec:include}.
However, the document body must make a distinction
between processing of an individual part and of the main document, e.g.:
%
\begin{center}
\begin{tabular}{l}
|\ifchilddocmanual|\\
|\input{\childdocname}|\\
|\||else|\\
\textit{document body with }|\input{|\textit{part}|}|\\
|\||fi|
\end{tabular}
\end{center}
%
The conditional |\ifchilddocmanual| is true whenever
a part to be included by |\input| is being compiled,
and the name of the part is stored in |\childdocname|.

%%%%%%%%%%%%%%%%%%%%%%%%%%%%%%%%%%%%%%%%
\DescribeMacro{\childdocby}
Each part to be included by |\input| should start with:
%
\begin{center}
\begin{tabular}{l}
|\input{childdoc.def}|\\
|\childdocby{|\textit{main}|}|\\
\end{tabular}
\end{center}
%
The directive |\childdocby| is similar to |\childdocof|
described in \secref{sec:include},
but the subsequent selection of content must be done manually.
To that end, both |\ifchilddoc| and |\ifchilddocmanual|
will be true upon processing of a part,
and the name of the part is stored in |\childdocname|.
Note that |\jobname| will be set to the filename of the current part
so that each part receives an individual |.aux| file
that does not interfere with the |.aux| file(s) of the main document.
This behaviour can be altered by the alternative form
|\childdocby[*]{|\textit{main}|}| (with a non-empty optional argument)
which uses the |.aux| file of the main document
by setting |\jobname| to \textit{main}.

%%%%%%%%%%%%%%%%%%%%%%%%%%%%%%%%%%%%%%%%%%%%%%%%%%%%%%%%%%%%%%%%%%%%%%%%%%%%%%%%
\subsection{Driver Development}
\label{sec:driver}

The \textsf{childdoc} mechanism can also be use for the development
of definition files such as \LaTeX{} styles or classes.
This case differs from the above setup with multiple parts
included by |\include| in that no |\includeonly| should be invoked.
This can be achieved by starting the include file
(before |\ProvidesPackage|) with:
%
\begin{center}
\begin{tabular}{l}
|\input{childdoc.def}|\\
|\childdocforward{|\textit{main}|}|\\
\end{tabular}
\end{center}
%
or alternatively with:
%
\begin{center}
\begin{tabular}{l}
|\input{childdoc.def}|\\
|\childdocby{|\textit{main}|}|\\
\end{tabular}
\end{center}
%
Both forms have slightly different effects as described above.
The main file is prepared as usual, see \secref{sec:include}.

%%%%%%%%%%%%%%%%%%%%%%%%%%%%%%%%%%%%%%%%%%%%%%%%%%%%%%%%%%%%%%%%%%%%%%%%%%%%%%%%
\subsection{Legacy Detection}
\label{sec:detection}

The directive |\childdocmain| in the main file can detect
whether the complete document or merely a child is to be compiled
even without using the directive |\childdocof|.
This method is deprecated because it is less robust
and there is no compelling reason to use it;
it is merely provided for backward compatibility
and it may be removed in future versions.

If the detection mechanism is to be used,
it is mandatory to correctly specify
the filename of the main file as the argument of |\childdocmain|:
%
\begin{center}
\begin{tabular}{l}
|\input{childdoc.def}|\\
|\childdocmain{|\textit{main}|}|\\
\end{tabular}
\end{center}
%
If |\jobname| does not match the argument \textit{main} of |\childdocmain|,
it is assumed that |\jobname| points to the child file to be compiled.
When using |\childdocmain| with the main file specified as argument,
it suffices to start a child file
with just |\input{|\textit{main}|}|
without loading of the package and using |\childdocof|.
If instead all processing is done
with the appropriate \textsf{childdoc} directives,
the argument of \textit{main} of |\childdocmain| can be empty.

An alternative version of the command line processing described
in \secref{sec:commandline} using the detection mechanism reads:
%
\begin{center}
|... -jobname "|\textit{target}|" "|[\textit{flags}]%
[|\def\jobname{|\textit{dest}|}|]|\input{|\textit{main}|}"|
\end{center}

%%%%%%%%%%%%%%%%%%%%%%%%%%%%%%%%%%%%%%%%%%%%%%%%%%%%%%%%%%%%%%%%%%%%%%%%%%%%%%%%
\subsection{Manual Code}
\label{sec:manual}

In case one cannot be certain whether the definitions file |childdoc.def|
is installed on the target \TeX{} distribution
and one prefers not to ship it,
it is conceivable to paste a few relevant commands into the sources.

To that end, drop all statements |\input{childdoc.def}|
and perform the replacements as outlined below.
Instead of |\childdocmain{|\textit{main}|}| add the following code
to the top of the main file:
%
\begin{center}
\begin{tabular}{l}
|\||ifdefined\childdocname\endinput\||fi\newif\ifchilddoc|\\
|\edef\childdocname{\scantokens\expandafter{\jobname\noexpand}}|\\
|\def\childdocmain{|\textit{main}|}\||ifx\childdocmain\childdocname\||else|\\
|\childdoctrue\includeonly{\childdocname}\let\jobname\childdocmain\||fi|\\
\end{tabular}
\end{center}
%
Instead of |\childdocof{|\textit{main}|}| just include the main file
at the top of each child file:
%
\begin{center}
|\input{|\textit{main}|}|
\end{center}
%
A simple redirection |\childdocforward{|\textit{dest}|}| is achieved by:
%
\begin{center}
|\def\jobname{|\textit{dest}|}\input{\jobname}|
\end{center}
%
The redirection with prefix
|\childdocforwardprefix[|\textit{prefix}|]{|\textit{dest}|}|
is accomplished by:
%
\begin{center}
\begin{tabular}{l}
|{\edef\jobname{\scantokens\expandafter{\jobname\noexpand}}|\\
|\def\redirectjob |\textit{prefix}|#1~~~{\gdef\jobname{|\textit{dest}|#1}}|\\
|\expandafter\redirectjob\jobname~~~}\input{\jobname}|
\end{tabular}
\end{center}

In an alternative approach,
child documents can be compiled by a specific command line
without additional code or specific definitions:
%
\begin{center}
|... -jobname "|\textit{target}|" "|[\textit{flags}]%
|\includeonly{|\textit{dest}|}\input{|\textit{main}|}"|
\end{center}
%

%%%%%%%%%%%%%%%%%%%%%%%%%%%%%%%%%%%%%%%%%%%%%%%%%%%%%%%%%%%%%%%%%%%%%%%%%%%%%%%%
%%%%%%%%%%%%%%%%%%%%%%%%%%%%%%%%%%%%%%%%%%%%%%%%%%%%%%%%%%%%%%%%%%%%%%%%%%%%%%%%
\section{Information}

%%%%%%%%%%%%%%%%%%%%%%%%%%%%%%%%%%%%%%%%%%%%%%%%%%%%%%%%%%%%%%%%%%%%%%%%%%%%%%%%
\subsection{Copyright}

Copyright \copyright{} 2017--2018 Niklas Beisert

This work may be distributed and/or modified under the
conditions of the \LaTeX{} Project Public License, either version 1.3
of this license or (at your option) any later version.
The latest version of this license is in
  \url{http://www.latex-project.org/lppl.txt}
and version 1.3 or later is part of all distributions of \LaTeX{}
version 2005/12/01 or later.

This work has the LPPL maintenance status `maintained'.

The Current Maintainer of this work is Niklas Beisert.

This work consists of the files |README.txt|, |childdoc.ins| and |childdoc.dtx|
as well as the derived files |childdoc.def|, |cdocsamp.tex|
with |cdocsch1.tex|, |cdocsch2.tex|, |cdocspt3.tex|, |cdocspt4.tex|,
|cdocsdrf.tex|, |cdocsfn1.tex|, |cdocsfn2.tex|
as well as |childdoc.pdf|.

%%%%%%%%%%%%%%%%%%%%%%%%%%%%%%%%%%%%%%%%%%%%%%%%%%%%%%%%%%%%%%%%%%%%%%%%%%%%%%%%
\subsection{Files and Installation}

The package consists of the files:
%
\begin{center}
\begin{tabular}{ll}
    |README.txt|   & readme file \\
    |childdoc.ins| & installation file \\
    |childdoc.dtx| & source file \\
    |childdoc.def| & definition file \\
    |cdocsamp.tex| & sample main file \\
    |cdocsch1.tex| & sample include file \\
    |cdocsch2.tex| & sample include file \\
    |cdocspt3.tex| & sample part file \\
    |cdocspt4.tex| & sample part file \\
    |cdocsdrf.tex| & sample redirection file \\
    |cdocsfn1.tex| & sample redirection file \\
    |cdocsfn2.tex| & sample redirection file \\
    |childdoc.pdf| & manual
\end{tabular}
\end{center}
%
The distribution consists of the files
|README.txt|, |childdoc.ins| and |childdoc.dtx|.
%
\begin{itemize}
\item
Run (pdf)\LaTeX{} on |childdoc.dtx|
to compile the manual |childdoc.pdf| (this file).
\item
Run \LaTeX{} on |childdoc.ins| to create the definitions file |childdoc.def|
and the sample |cdocsamp.tex| with include files
|cdocsch1.tex|, |cdocsch2.tex|, |cdocspt3.tex|, |cdocspt4.tex|,
|cdocsdrf.tex|, |cdocsfn1.tex|, |cdocsfn2.tex|.
Then copy the file |childdoc.def| to an appropriate directory of your \LaTeX{}
distribution, e.g.\ \textit{texmf-root}|/tex/latex/childdoc|.
\end{itemize}

%%%%%%%%%%%%%%%%%%%%%%%%%%%%%%%%%%%%%%%%%%%%%%%%%%%%%%%%%%%%%%%%%%%%%%%%%%%%%%%%
\subsection{Related CTAN Packages}

There are several other packages which offer a similar functionality:
%
\begin{itemize}
\item
The packages
\href{http://ctan.org/pkg/docmute}{\textsf{docmute}},
\href{http://ctan.org/pkg/includex}{\textsf{includex}} and
\href{http://ctan.org/pkg/standalone}{\textsf{standalone}}
provide commands to include only the document body of
a child file thus allowing both files to be compiled individually.
\item
The packages \href{http://ctan.org/pkg/subdocs}{\textsf{subdocs}}
and \href{http://ctan.org/pkg/subfiles}{\textsf{subfiles}}
provide structures in which the main and child documents can be
encapsulated and allowing them to be compiled individually.
The inclusion mechanism is different from the conventional |\include|.
\item
The package \href{http://ctan.org/pkg/combine}{\textsf{combine}}
is an elaborate solution to combine several documents into one.
\end{itemize}
%
See also the CTAN topic \href{http://ctan.org/topic/subdocs}{\textsf{subdocs}}
for further related packages.
The present package differs from the above solutions in that
a document structure constructed with the conventional |\include| mechanism
just needs two extra commands at the top of every file
such that all constituent files can be compiled individually.

%%%%%%%%%%%%%%%%%%%%%%%%%%%%%%%%%%%%%%%%%%%%%%%%%%%%%%%%%%%%%%%%%%%%%%%%%%%%%%%%
%\subsection{Feature Suggestions}
%
%The following is a list of features which may be useful for future
%versions of this package:
%%
%\begin{itemize}
%\item
%\ldots
%\end{itemize}

%%%%%%%%%%%%%%%%%%%%%%%%%%%%%%%%%%%%%%%%%%%%%%%%%%%%%%%%%%%%%%%%%%%%%%%%%%%%%%%%
\subsection{Revision History}

%%%%%%%%%%%%%%%%%%%%%%%%%%%%%%%%%%%%%%%%
\paragraph{v2.0:} 2018/12/30

\begin{itemize}
\item
immediate forward processing
\item
added |\childdocby| mechanism
\item
manual restructured
\end{itemize}

%%%%%%%%%%%%%%%%%%%%%%%%%%%%%%%%%%%%%%%%
\paragraph{v1.6:} 2018/01/17

\begin{itemize}
\item
application for development of include files
\item
corrections to manual
\end{itemize}

%%%%%%%%%%%%%%%%%%%%%%%%%%%%%%%%%%%%%%%%
\paragraph{v1.5:} 2017/05/21

\begin{itemize}
\item
more complete structuring introduced
\item
|\childdocof| introduced
\item
|\childdoc| renamed to |\childdocmain|
\item
|\childredirect| renamed to |\childdocforward| and |\childdocforwardprefix|
and functionality expanded
\end{itemize}

%%%%%%%%%%%%%%%%%%%%%%%%%%%%%%%%%%%%%%%%
\paragraph{v1.0:} 2017/04/27

\begin{itemize}
\item
manual and install package
\item
first version published on CTAN
\end{itemize}

%%%%%%%%%%%%%%%%%%%%%%%%%%%%%%%%%%%%%%%%
\paragraph{v0.6:} 2017/04/26

\begin{itemize}
\item
redirection mechanism added
\end{itemize}

%%%%%%%%%%%%%%%%%%%%%%%%%%%%%%%%%%%%%%%%
\paragraph{v0.5:} 2017/04/26

\begin{itemize}
\item
functionality in definition file
\end{itemize}


%%%%%%%%%%%%%%%%%%%%%%%%%%%%%%%%%%%%%%%%%%%%%%%%%%%%%%%%%%%%%%%%%%%%%%%%%%%%%%%%
%%%%%%%%%%%%%%%%%%%%%%%%%%%%%%%%%%%%%%%%%%%%%%%%%%%%%%%%%%%%%%%%%%%%%%%%%%%%%%%%
%%%%%%%%%%%%%%%%%%%%%%%%%%%%%%%%%%%%%%%%%%%%%%%%%%%%%%%%%%%%%%%%%%%%%%%%%%%%%%%%
\appendix

\settowidth\MacroIndent{\rmfamily\scriptsize 000\ }

 \DocInput{childdoc.dtx}

\end{document}
%</driver>
% \fi
%
% %%%%%%%%%%%%%%%%%%%%%%%%%%%%%%%%%%%%%%%%%%%%%%%%%%%%%%%%%%%%%%%%%%%%%%%%%%%%%%
% %%%%%%%%%%%%%%%%%%%%%%%%%%%%%%%%%%%%%%%%%%%%%%%%%%%%%%%%%%%%%%%%%%%%%%%%%%%%%%
% \section{Sample}
%\iffalse
%<*samplemain>
%\fi
%
% The following presents a sample document
% with two chapters, two parts, a title page,
% a compile flag as well as three forwarding files to set the flag.
% It consists of eight |.tex| files:
% \begin{center}
% \begin{tabular}{ll}
% |cdocsamp.tex|&main file\\
% |cdocsch1.tex|&include file for chapter 1\\
% |cdocsch2.tex|&include file for chapter 2\\
% |cdocspt3.tex|&include file for part 3\\
% |cdocspt4.tex|&include file for part 4\\
% |cdocsdrf.tex|&forwarding file for main file in draft mode\\
% |cdocsfi1.tex|&forwarding file for final version of chapter 1\\
% |cdocsfi2.tex|&forwarding file for final version of chapter 2\\
% \end{tabular}
% \end{center}
% Each of the eight files can be compiled directly by the \LaTeX{} compiler.
%
% %%%%%%%%%%%%%%%%%%%%%%%%%%%%%%%%%%%%%%
% \paragraph{Main File.}
%
% The main file is called |cdocsamp.tex|.
%
% Load the \textsf{childdoc} definitions and
% declare the filename for the main document:
%    \begin{macrocode}
\input{childdoc.def}
\childdocmain{}
%    \end{macrocode}

% Optional override for |\version| flag:
%    \begin{macrocode}
%%\ifchilddoc\else\providecommand{\version}{draft}\fi
%    \end{macrocode}

% Define the default values for the |\version| flag
% (|final| for the main file and |draft| for childs):
%    \begin{macrocode}
\ifchilddoc
\providecommand{\version}{draft}
\else
\providecommand{\version}{final}
\fi
%    \end{macrocode}

% Load the standard document class:
%    \begin{macrocode}
\documentclass[12pt]{article}
%    \end{macrocode}

% Start the document body:
%    \begin{macrocode}
\begin{document}
%    \end{macrocode}

% Declare a title page.
% Print title, part of document being processed and version flag:
%    \begin{macrocode}
\addtocounter{page}{-1}
\begin{center}
{\LARGE\bfseries{}childdoc example\par}
\vspace{1cm}
\ifchilddoc
\ifchilddocmanual part\else chapter\fi:
`\childdocname' of `\childdocjob'\par
\else
main document: `\childdocjob'\par
\fi
version: \version\par
\end{center}
\newpage
%    \end{macrocode}

% Manually include selected file,
% otherwise process as usual:
%    \begin{macrocode}
\ifchilddocmanual
\section*{part `\childdocname'}
\input{\childdocname}
\else
%    \end{macrocode}

% Include the two chapters:
%    \begin{macrocode}
\include{cdocsch1}
\include{cdocsch2}
%    \end{macrocode}

% Include the two parts unless only chapters should be displayed:
%    \begin{macrocode}
\ifchilddoc\else
\section{part three}
\input{cdocspt3}
\section{part four}
\input{cdocspt4}
\fi
%    \end{macrocode}

% Process as usual until here:
%    \begin{macrocode}
\fi
%    \end{macrocode}

% End of document body:
%    \begin{macrocode}
\end{document}
%    \end{macrocode}
%\iffalse
%</samplemain>
%\fi
%
% %%%%%%%%%%%%%%%%%%%%%%%%%%%%%%%%%%%%%%
% \paragraph{Chapter Include Files.}
%
% The include files are called |cdocsch1.tex| and |cdocsch2.tex|.
%
%\iffalse
%<*samplechap1|samplechap2>
%\fi

% Optional override for |\version| flag:
%    \begin{macrocode}
%%\providecommand{\version}{final}
%    \end{macrocode}

% Include the main document:
%    \begin{macrocode}
\input{childdoc.def}
\childdocof{cdocsamp}
%    \end{macrocode}

%\iffalse
%</samplechap1|samplechap2>
%\fi
%
%\iffalse
%<*samplechap1>
%\fi
% Some text for chapter 1:
%    \begin{macrocode}
\section{one}
some text in chapter one
%    \end{macrocode}

%\iffalse
%</samplechap1>
%\fi
% Some text for chapter 2:
%\iffalse
%<*samplechap2>
%\fi
%    \begin{macrocode}
\section{two}
more text in chapter two
%    \end{macrocode}

%\iffalse
%</samplechap2>
%\fi
%
% %%%%%%%%%%%%%%%%%%%%%%%%%%%%%%%%%%%%%%
% \paragraph{Part Include Files.}
%
% The include files are called |cdocspt3.tex| and |cdocspt4.tex|.
%
%\iffalse
%<*samplepart3|samplepart4>
%\fi

% Optional override for |\version| flag:
%    \begin{macrocode}
%%\providecommand{\version}{final}
%    \end{macrocode}

% Include the main document:
%    \begin{macrocode}
\input{childdoc.def}
\childdocby{cdocsamp}
%    \end{macrocode}

%\iffalse
%</samplepart3|samplepart4>
%\fi
%
%\iffalse
%<*samplepart3>
%\fi
% Some text for part 3:
%    \begin{macrocode}
some text in part three
%    \end{macrocode}

%\iffalse
%</samplepart3>
%\fi
% Some text for part 4:
%\iffalse
%<*samplepart4>
%\fi
%    \begin{macrocode}
more text in part four
%    \end{macrocode}

%\iffalse
%</samplepart4>
%\fi
%
% %%%%%%%%%%%%%%%%%%%%%%%%%%%%%%%%%%%%%%
% \paragraph{Forwarding for a Complete Draft.}
%
% The following forwarding file |cdocsdrf.tex|
% compiles the main document in draft mode:
%\iffalse
%<*sampledraft>
%\fi
%    \begin{macrocode}
\def\version{draft}
\input{childdoc.def}
\childdocforward{cdocsamp}
%    \end{macrocode}

%\iffalse
%</sampledraft>
%\fi
%
% %%%%%%%%%%%%%%%%%%%%%%%%%%%%%%%%%%%%%%
% \paragraph{Forwarding for Final Version of the Chapters.}
%
% The following forwarding files |cdocsfn1.tex| and |cdocsfn2.tex|
% (with identical content)
% compile the final versions of the child documents
% |cdocsch1.tex| and |cdocsch2.tex|, respectively:
%\iffalse
%<*samplefinal>
%\fi
%    \begin{macrocode}
\def\version{final}
\input{childdoc.def}
\childdocforwardprefix[cdocsamp]{cdocsfn}{cdocsch}
%    \end{macrocode}

%\iffalse
%</samplefinal>
%\fi
%
% %%%%%%%%%%%%%%%%%%%%%%%%%%%%%%%%%%%%%%
% \paragraph{Command Line Processing.}
%
% The following three command lines generate the output files
% |cdocscld|, |cdocscl1| and |cdocscl2|
% which should be identical to
% |cdocsdrf|, |cdocsch1| and |cdocsfn2|, respectively:
% \begin{center}
% \begin{tabular}{l}
% |latex -jobname cdocscld \|\\
% |  "\def\version{draft}\input{childdoc.def}\childdocforward{cdocsamp}"|\\
% |latex -jobname cdocscl1 \|\\
% |  "\input{childdoc.def}\childdocforward[cdocsamp]{cdocsch1}"|\\
% |latex -jobname cdocscl2 \|\\
% |  "\def\version{final}\input{childdoc.def}\childdocforward{cdocsch2}"|
% \end{tabular}
% \end{center}
% Note that the trailing backslash on each first line
% merely continues the input to the second line
% (for convenient cut ant paste).
% Furthermore, the command |latex| can be replaced by any
% of its alternative versions such as |pdflatex|.
%
% %%%%%%%%%%%%%%%%%%%%%%%%%%%%%%%%%%%%%%%%%%%%%%%%%%%%%%%%%%%%%%%%%%%%%%%%%%%%%%
% %%%%%%%%%%%%%%%%%%%%%%%%%%%%%%%%%%%%%%%%%%%%%%%%%%%%%%%%%%%%%%%%%%%%%%%%%%%%%%
% \section{Implementation}
%\iffalse
%<*package>
%\fi
%
% This section describes the definitions file |childdoc.def|.

% The definitions cannot be loaded using |\usepackage| or |\RequirePackage|
% which has a mechanism to prevent loading a style file more than once.
% When loading the definitions by means of |\input|
% multiple instances have to be prevented manually:
%\iffalse
%This code needs to be before the `\ProvidesFile' directive
%which is defined at the beginning of this file.
%Therefore it is also placed there and commented out here.
%</package>
%<*discard>
%\fi
%    \begin{macrocode}
\ifdefined\childdocmain\endinput\fi
%    \end{macrocode}
%\iffalse
%</discard>
%<*package>
%\fi
%
% \macro{\ifchilddoc}
% \macro{\ifchilddocmanual}
% The conditional |\ifchilddoc| tells whether a
% child (true) or main (false) document is being compiled.
% The conditional |\ifchilddocmanual| tells whether
% the |\includeonly| mechanism is used (false) or
% the selection of child files must be performed manually (true).
% The definitions initialise to false:
%    \begin{macrocode}
\newif\ifchilddoc
\newif\ifchilddocmanual
%    \end{macrocode}

% \macro{\childdocname}
% \macro{\childdocjob}
% The macro |\childdocname| stores the name of the main document
% to be compiled. The macro |\childdocjob| stores the name of
% the document on which the \LaTeX{} compiler was originally invoked.
% The content of |\jobname| cannot be compared
% to filenames specified in the source due to different catcodes.
% The following code rescans |\jobname|, stores the result
% in |\childdocname| and saves a copy in |\childdocjob|:
%    \begin{macrocode}
\edef\childdocname{\scantokens\expandafter{\jobname\noexpand}}
\let\childdocjob\childdocname
%    \end{macrocode}

% \macro{\childdocdisable}
% The macro |\childdocdisable| prevents the main file
% from being processed more than once.
% At this stage, the main document command |\childdocmain|
% is assumed to be called once again where it should do nothing.
% Any subsequent call to it should prevent
% a secondary processing of the main document
% It overwrites the forwarding commands
% |\childdocof| and |\childdocforward|
% with empty macros to prevent further inclusions of the main document:
%    \begin{macrocode}
\newcommand{\childdocdisable}
{
  \renewcommand{\childdocmain}[1]{\renewcommand{\childdocmain}[1]{\endinput}}
  \renewcommand{\childdocof}[1]{}
  \renewcommand{\childdocby}[2][]{}
  \renewcommand{\childdocforward}[2][]{}
  \renewcommand{\childdocdisable}{}
}
%    \end{macrocode}

% \macro{\childdocmain}
% The macro |\childdocmain| is to be called at the top of the main file
% with nothing or the main filename (without extension) as argument.
% First, it breaks loops.
% If the argument is not empty and does not match |\childdocname|
% (which is set by the first inclusion of |childdoc.def|),
% |\ifchilddoc| is set to true, |\includeonly| is applied to the child file
% and |\jobname| is set to the main file
% (for proper handling of |.aux| files):
%    \begin{macrocode}
\newcommand{\childdocmain}[1]
{
  \childdocdisable\childdocmain{}
  \if?#1?\else
    \begingroup
      \def\childdoctmp{#1}
      \ifx\childdoctmp\childdocname
        \def\childdoctmp{}
      \else
        \def\childdoctmp
        {
          \childdoctrue
          \includeonly{\childdocname}
          \def\childdocjob{#1}
          \def\jobname{#1}
        }
      \fi
      \expandafter
    \endgroup
    \childdoctmp
  \fi
}
%    \end{macrocode}

% \macro{\childdocof}
% The command |\childdocof| redirects
% compilation to the main file |#1|.
%    \begin{macrocode}
\newcommand{\childdocof}[1]
{
  \childdocdisable
  \childdoctrue
  \includeonly{\childdocname}
  \def\jobname{#1}
  \def\childdocjob{#1}
  \input{#1}
}
%    \end{macrocode}

% \macro{\childdocby}
% The command |\childdocby| ....
%    \begin{macrocode}
\newcommand{\childdocby}[2][]
{
  \childdocdisable
  \childdoctrue
  \childdocmanualtrue
  \if?#1?\else
    \def\jobname{#2}
  \fi
  \def\childdocjob{#2}
  \input{#2}
  \endinput
}
%    \end{macrocode}

% \macro{\childdocforward}
% The command |\childdocforward| redirects
% compilation to the main file or
% (if the optional argument is given) a child file.
% Parameters are set as if the main file
% or a child file starting with |\childdocof| was compiled.
% Then compilation is handed over to the main file:
%    \begin{macrocode}
\newcommand{\childdocforward}[2][]
{
  \begingroup
    \if?#1?
      \def\childdoctmp
      {
        \def\childdocname{#2}
        \def\childdocjob{#2}
        \def\jobname{#2}
        \input{#2}
        \endinput
      }
    \else
      \def\childdoctmp
      {
        \childdocdisable
        \def\childdocname{#2}
        \childdoctrue
        \includeonly{#2}
        \def\childdocjob{#1}
        \def\jobname{#1}
        \input{#1}
        \endinput
      }
    \fi
    \expandafter
  \endgroup
  \childdoctmp
}
%    \end{macrocode}

% \macro{\childdocforwardprefix}
% The command |\childdocforwardprefix| redirects
% compilation to the main or a child file by means of a pattern.
% The prefix |#1| in the current filename is replaced by |#2|
% and the suffix of the current filename is kept
% (it is assumed that the filename does not contain the substring `|~~~|'
% which is used as a delimiter).
% Compilation is handed over to the new file by |\childdocforward|:
%    \begin{macrocode}
\newcommand{\childdocforwardprefix}[3][]
{
  \begingroup
    \def\childdocextract #2##1~~~{\def\childdoctmp{\childdocforward[#1]{#3##1}}}
    \expandafter\childdocextract\childdocname~~~
    \expandafter
  \endgroup
  \childdoctmp
}
%    \end{macrocode}

% \macro{\childdoc}
% The deprecated macro |\childdoc| is a legacy version of |\childdocmain|:
%    \begin{macrocode}
\newcommand{\childdoc}{\childdocmain}
%    \end{macrocode}

% \macro{\childdocredirect}
% The deprecated macro |\childdocredirect| is a legacy version
% of |\childdocforward| and |\childdocforwardprefix|:
%    \begin{macrocode}
\newcommand{\childdocredirect}[2][]
{
  \begingroup
    \if?#1?
      \def\childdoctmp{\childdocforward{#2}}
    \else
      \def\childdoctmp{\childdocforwardprefix{#1}{#2}}
    \fi
    \expandafter
  \endgroup
  \childdoctmp
}
%    \end{macrocode}

%\iffalse
%</package>
%\fi
%
\endinput
|\\
|\childdocby{|\textit{main}|}|\\
\end{tabular}
\end{center}
%
Both forms have slightly different effects as described above.
The main file is prepared as usual, see \secref{sec:include}.

%%%%%%%%%%%%%%%%%%%%%%%%%%%%%%%%%%%%%%%%%%%%%%%%%%%%%%%%%%%%%%%%%%%%%%%%%%%%%%%%
\subsection{Legacy Detection}
\label{sec:detection}

The directive |\childdocmain| in the main file can detect
whether the complete document or merely a child is to be compiled
even without using the directive |\childdocof|.
This method is deprecated because it is less robust
and there is no compelling reason to use it;
it is merely provided for backward compatibility
and it may be removed in future versions.

If the detection mechanism is to be used,
it is mandatory to correctly specify
the filename of the main file as the argument of |\childdocmain|:
%
\begin{center}
\begin{tabular}{l}
|% \iffalse
%
% childdoc.dtx Copyright (C) 2017-2018 Niklas Beisert
%
% This work may be distributed and/or modified under the
% conditions of the LaTeX Project Public License, either version 1.3
% of this license or (at your option) any later version.
% The latest version of this license is in
%   http://www.latex-project.org/lppl.txt
% and version 1.3 or later is part of all distributions of LaTeX
% version 2005/12/01 or later.
%
% This work has the LPPL maintenance status `maintained'.
%
% The Current Maintainer of this work is Niklas Beisert.
%
% This work consists of the files childdoc.dtx and childdoc.ins
% and the derived files childdoc.def and cdocsamp.tex with
% cdocsch1.tex, cdocsch2.tex, cdocsdrf.tex, cdocsfn1.tex, cdocsfn2.tex.
%
%<package>\ifdefined\childdocmain\endinput\fi
%<package>\ProvidesFile{childdoc.def}[2018/12/30 v2.0 child document driver]
%<samplemain>\ProvidesFile{cdocsamp.tex}[2018/12/30 v2.0 sample for childdoc]
%<*driver>
%\ProvidesFile{childdoc.drv}[2018/12/30 v2.0 childdoc reference manual file]
\PassOptionsToClass{10pt,a4paper}{article}
\documentclass{ltxdoc}

\usepackage[margin=35mm]{geometry}
\usepackage{hyperref}
\usepackage{hyperxmp}
\usepackage[usenames]{color}

\hypersetup{colorlinks=true}
\hypersetup{pdfstartview=FitH}
\hypersetup{pdfpagemode=UseNone}
\hypersetup{pdfsource={}}
\hypersetup{pdflang={en-UK}}
\hypersetup{pdfcopyright={Copyright 2017-2018 Niklas Beisert.
  This work may be distributed and/or modified under the
  conditions of the LaTeX Project Public License, either version 1.3
  of this license or (at your option) any later version.}}
\hypersetup{pdflicenseurl={http://www.latex-project.org/lppl.txt}}
\hypersetup{pdfcontactaddress={ETH Zurich, ITP, HIT K,
  Wolfgang-Pauli-Strasse 27}}
\hypersetup{pdfcontactpostcode={8093}}
\hypersetup{pdfcontactcity={Zurich}}
\hypersetup{pdfcontactcountry={Switzerland}}
\hypersetup{pdfcontactemail={nbeisert@itp.phys.ethz.ch}}
\hypersetup{pdfcontacturl={http://people.phys.ethz.ch/\xmptilde nbeisert/}}

\newcommand{\secref}[1]{\hyperref[#1]{section \ref*{#1}}}

\parskip1ex
\parindent0pt
\let\olditemize\itemize
\def\itemize{\olditemize\parskip0pt}

\begin{document}

\title{The \textsf{childdoc} Package}
\hypersetup{pdftitle={The childdoc Package}}
\author{Niklas Beisert\\[2ex]
  Institut f\"ur Theoretische Physik\\
  Eidgen\"ossische Technische Hochschule Z\"urich\\
  Wolfgang-Pauli-Strasse 27, 8093 Z\"urich, Switzerland\\[1ex]
  \href{mailto:nbeisert@itp.phys.ethz.ch}
  {\texttt{nbeisert@itp.phys.ethz.ch}}}
\hypersetup{pdfauthor={Niklas Beisert}}
\hypersetup{pdfsubject={Manual for the LaTeX2e Package childdoc}}
\date{30 December 2018, \textsf{v2.0}}
\maketitle

\begin{abstract}\noindent
\textsf{childdoc} is a \LaTeXe{} package
that enables the direct compilation
of document sections included by |\include|
to individual files.
\end{abstract}

\begingroup
\parskip0ex
\tableofcontents
\endgroup

%%%%%%%%%%%%%%%%%%%%%%%%%%%%%%%%%%%%%%%%%%%%%%%%%%%%%%%%%%%%%%%%%%%%%%%%%%%%%%%%
%%%%%%%%%%%%%%%%%%%%%%%%%%%%%%%%%%%%%%%%%%%%%%%%%%%%%%%%%%%%%%%%%%%%%%%%%%%%%%%%
\section{Introduction}

\LaTeX{} provides a mechanism to structure a large document (such as a book)
into a main file and several child files (containing the chapters)
using the |\include| command.
This mechanism is beneficial for documents
which span hundreds of pages in order to
make the source file(s) more manageable.
Moreover, compilation can be restricted to
selected child files by means of the |\includeonly| command.
The latter feature can be used to reduce the compilation time while editing
(this was significantly more useful in the earlier days of \LaTeX{})
or to generate a smaller document which is easier to navigate.
Another application of |\includeonly| is to generate
documents consisting of selected parts of the complete document.

However, there are a few drawbacks of the plain |\include| mechanism:
\begin{itemize}
\item
The child files cannot be compiled on their own,
they can only be compiled via the main file.
A naive editing environment
(such as a text editor with an option
to have the current file processed by \LaTeX)
may require one to switch to the main file before compiling;
attempting to compile the child file produces errors.
\item
The main file must be modified (each time)
to adjust the |\includeonly| command
to the present needs. This easily leaves the main file in a messy state.
\item
The generated document will always carry the filename
of the main document. This is inconvenient if
several child files are to be compiled and
to be kept for distribution.
\end{itemize}

The present package provides a simple interface
to make child files individually compilable by \LaTeX{}.
Compiling a child file then has the same effect as compiling
the main file with an |\includeonly| command
to select the appropriate child.
Moreover the generated document will carry the name of the child
rather than the main file.
This resolves all three above issues.

This feature is meant to make the editing of books,
thesis documents and lecture notes somewhat more convenient.
However, the package can also be used efficiently for
composing a series of documents (such as exercise sheets)
which are typically distributed individually.
It then assists the author in generating the individual documents
(potentially in different versions)
as well as a document containing the collected series.
Another application is in developing style files
or other kinds of included material
where compilation of the style file could redirect
to a sample or test file.

%%%%%%%%%%%%%%%%%%%%%%%%%%%%%%%%%%%%%%%%%%%%%%%%%%%%%%%%%%%%%%%%%%%%%%%%%%%%%%%%
%%%%%%%%%%%%%%%%%%%%%%%%%%%%%%%%%%%%%%%%%%%%%%%%%%%%%%%%%%%%%%%%%%%%%%%%%%%%%%%%
\section{Usage}

First of all, the package \textsf{childdoc} is \emph{not} a standard
\LaTeXe{} |.sty| style file! Therefore it needs to be invoked in
a non-standard way.

%%%%%%%%%%%%%%%%%%%%%%%%%%%%%%%%%%%%%%%%%%%%%%%%%%%%%%%%%%%%%%%%%%%%%%%%%%%%%%%%
\subsection{Included Files}
\label{sec:include}

%%%%%%%%%%%%%%%%%%%%%%%%%%%%%%%%%%%%%%%%
\DescribeMacro{\childdocmain}
To use the package, add the commands
\begin{center}
\begin{tabular}{l}
|\input{childdoc.def}|\\
|\childdocmain{}|\\
\end{tabular}
\end{center}
at the very top of the main \LaTeX{} file,
in particular \emph{before} the |\documentclass| statement!
The argument of |\childdocmain| should be left empty
(but it must be present).

%%%%%%%%%%%%%%%%%%%%%%%%%%%%%%%%%%%%%%%%
\DescribeMacro{\childdocof}
Furthermore, add the commands
\begin{center}
\begin{tabular}{l}
|\input{childdoc.def}|\\
|\childdocof{|\textit{main}|}|\\
\end{tabular}
\end{center}
at the top of every child file \textit{child}
which is included by |\include{|\textit{child}|}|
from within the main file
(or at least for those files to be compiled individually).
The argument \textit{main} must be the filename of the main file.

There are a couple of
considerations in setting up the main and child documents:

%%%%%%%%%%%%%%%%%%%%%%%%%%%%%%%%%%%%%%%%
\paragraph{Restrictions.}

Please note the following restrictions:
\begin{itemize}
\item
|\childdocmain| must be called with one argument \textit{main}
to ensure compatibility with earlier version of the package.
It must either be empty (|\childdocmain{}|)
or precisely match the filename of the main file in which it is specified.
See \secref{sec:detection} for further information.
\item
The filename \textit{main} must be specified without the |.tex| extension.
\item
The filename \textit{main} is case sensitive
(even in case-insensitive file systems)
due to internal string comparison.
\item
The argument \textit{main} should be fully expanded, it cannot be a macro.
\item
Subdirectories and special characters should be avoided in filenames.
\item
The command |\childdocmain{|\textit{main}|}| must be followed by a whitespace.
It should not be followed immediately by another command
or by a comment mark `|%|'.
This is because the \TeX{} parser reads the token immediately following
the argument of |\childdocmain| and puts it
at the beginning of every child section;
however, a white\-space is ignored.
\end{itemize}

%%%%%%%%%%%%%%%%%%%%%%%%%%%%%%%%%%%%%%%%
\paragraph{Content of Main File.}

It is advisable to place all content in the child files included by |\include|.
Any output contained in the main file will appear in all child documents
unless suppressed manually;
it cannot be suppressed automatically by the |\includeonly| directive
and thus should normally be avoided.
A method to include some content in the main file
by means of conditional processing is described in \secref{sec:conditional}.

%%%%%%%%%%%%%%%%%%%%%%%%%%%%%%%%%%%%%%%%
\paragraph{Page Numbering.}

When only a part of the document is compiled,
the appropriate numbering of pages
(as well as other status parameters)
is determined from the |.aux| files.
The latter contain information from previous passes.
However this information needs to propagate through
all intermediate child documents.
Therefore the page numbering in child documents may well
be inconsistent until the complete document is compiled at least once.

A useful (if unconventional) way to always ensure a consistent
page numbering is to restart the numbering in each child document
and denote the pages by `\textit{child}|.|\textit{page}'
where \textit{child} represents the chapter/section number of the child file.
This can be achieved by the command
|\numberwithin{page}{|\textit{child}|}|
of the \textsf{amsmath} package
where \textit{child} can be |chapter| or |section|
depending on the chosen structuring.
Alternatively, one can modify the macro |\thepage| appropriately
and reset the counter |page| at the start of each child file.

%%%%%%%%%%%%%%%%%%%%%%%%%%%%%%%%%%%%%%%%%%%%%%%%%%%%%%%%%%%%%%%%%%%%%%%%%%%%%%%%
\subsection{Conditional Processing}
\label{sec:conditional}

The package provides a mechanism to compile different versions
of a document. To customise the versions further some conditional processing
can come in handy to distinguish which version is being compiled.
The package provides two macros to describe the compilation context:

%%%%%%%%%%%%%%%%%%%%%%%%%%%%%%%%%%%%%%%%
\DescribeMacro{\ifchilddoc}
The conditional |\ifchilddoc| distinguishes between the compilation of
child documents and the main document:
%
\begin{center}
|\ifchilddoc |\textit{child-code}| |[|\||else |\textit{main-code}]| \||fi|
\end{center}

%%%%%%%%%%%%%%%%%%%%%%%%%%%%%%%%%%%%%%%%
\DescribeMacro{\childdocname}
\DescribeMacro{\childdocjob}
The macro |\childdocname| contains the filename (without extension)
of the main or child file being processed.
Note that |\childdocjob| will always contain the name of the main file.

%%%%%%%%%%%%%%%%%%%%%%%%%%%%%%%%%%%%%%%%
\paragraph{Title Page.}

Conditional processing can be used to include a title or banner page
in the main document when proper precautions are taken.
Importantly, the code in the main file should ensure that the page counter
(as well as other status parameters which are stored in the |.aux| files)
takes the same value after the conditional processing.
Otherwise the page numbers may take divergent values
depending on which part is compiled.

For example, a title page could be declared by:
%
\begin{center}
\begin{tabular}{l}
|\ifchilddoc\||else|\\
|\addtocounter{page}{-1}|\\
\textit{code for title page}\\
|\newpage|\\
|\||fi|
\end{tabular}
\end{center}
%
A banner page for the child documents can be generated by:
%
\begin{center}
\begin{tabular}{l}
|\ifchilddoc|\\
|\addtocounter{page}{-1}|\\
\textit{code for banner page}\\
|\newpage|\\
|\||fi|
\end{tabular}
\end{center}
%
Here one could write a message such as:
\begin{center}
|This is the part \childdocname{} of \childdocjob{}.|
\end{center}

%%%%%%%%%%%%%%%%%%%%%%%%%%%%%%%%%%%%%%%%%%%%%%%%%%%%%%%%%%%%%%%%%%%%%%%%%%%%%%%%
\subsection{Flags}
\label{sec:flags}

The package makes it easy to generate different versions
of the main or child documents.
To this end compilation flags can be defined
and assigned different default values.
They will be particularly useful in conjunction
with the forwarding mechanism described in \secref{sec:forward}.

For example, it may be useful to have a flag |\version|
which can be set to |draft| or |final|.
The document source will contain some conditional code
depending on the value of |\version|.
Suppose further, the flag should default to |final| for the main file
and to |draft| for child files
which is a natural assignment for editing the document.
This is achieved by placing the following code
in the preamble of the main document
(below the |\childdocmain| directive):
%
\begin{center}
\begin{tabular}{l}
|\ifchilddoc|\\
|\providecommand{\version}{draft}|\\
|\||else|\\
|\providecommand{\version}{final}|\\
|\||fi|
\end{tabular}
\end{center}
%
The definition by |\providecommand| makes sure
that previous definitions are not overwritten.
Further statements |\providecommand{\version}{...}|
can thus be added before the above code to override it.

For the main file, one might add a line
(between |\childdocmain| and the above block)
%
\begin{center}
|%\ifchilddoc\||else\providecommand{\version}{draft}\||fi|
\end{center}
%
which can be uncommented to produce a draft version.
Likewise one can add a line to the very top of a child file
(above the |\childdocof{|\textit{main}|}| directive)
%
\begin{center}
|%\providecommand{\version}{final}|
\end{center}
%
which can be uncommented to produce the final version of this child document.

%%%%%%%%%%%%%%%%%%%%%%%%%%%%%%%%%%%%%%%%%%%%%%%%%%%%%%%%%%%%%%%%%%%%%%%%%%%%%%%%
\subsection{Forwarding}
\label{sec:forward}

Different versions of the main or child documents
using compilation flags as described in \secref{sec:flags}
can be (permanently) stored in different files
for convenient compilation, viewing and distribution.
To this end, the package defines a command
to pass on compilation to a different file:

%%%%%%%%%%%%%%%%%%%%%%%%%%%%%%%%%%%%%%%%
\DescribeMacro{\childdocforward}
The command |\childdocforward| redirects processing to
another source file:
%
\begin{center}
\begin{tabular}{l}
|\input{childdoc.def}|\\
|\childdocforward[|\textit{main}|]{|\textit{dest}|}|\\
\end{tabular}
\end{center}
%
The argument \textit{dest} is the destination file
(without extension).
It should be the main file or one of the child files.
Note that further \textsf{childdoc} directives
such as |\childdocof| and |\childdocforward|
in the indicated file will be processed in this form.
The optional argument \textit{main}
passes on directly to the main file \textit{main}
while pretending to compile the child \textit{dest}.
This form behaves as if \textit{dest}
issues |\childdocof{|\textit{main}|}| right away,
and no further \textsf{childdoc} directives will be processed.

%%%%%%%%%%%%%%%%%%%%%%%%%%%%%%%%%%%%%%%%
\DescribeMacro{\...prefix}
In the alternative form |\childdocforwardprefix|,
%
\begin{center}
\begin{tabular}{l}
|\input{childdoc.def}|\\
|\childdocforwardprefix[|\textit{main}|]{|\textit{prefix}|}{|\textit{dest}|}|
\end{tabular}
\end{center}
%
the destination file is determined by a pattern
depending on the current file:
To make this work, the current file must be called
`{\textit{prefix}\hspace{0.2em}\textit{suffix}}'
with \textit{prefix} matching precisely the argument.
Processing is then passed on to the file
`{\textit{dest}\hspace{0.2em}\textit{suffix}}'.
Surely, the same effect is achieved by
directly specifying the
argument `{\textit{dest}\hspace{0.2em}\textit{suffix}}'
in the first form.
However, that requires to set up a different file
for each child. With the alternative form of the command
all these files can have exactly the same content
which simplifies setting them up and maintaining them.

For example, the following file |draft.tex|
with a compilation flag |\version| as described in \secref{sec:flags}
compiles the main document as a draft:
%
\begin{center}
\begin{tabular}{l}
|\def\version{draft}|\\
|\input{childdoc.def}|\\
|\childdocforward{|\textit{main}|}|
\end{tabular}
\end{center}
%
Likewise, the following files |final|\textit{nn}|.tex|
compile the final version of the child document
|child|\textit{nn}|.tex|:
%
\begin{center}
\begin{tabular}{l}
|\def\version{final}|\\
|\input{childdoc.def}|\\
|\childdocforwardprefix{final}{child}|
\end{tabular}
\end{center}
%

Note that when several versions of a main file and/or of each child file
are to be generated, it may be convenient to set up a |Makefile| or
shell script to automatise the process.

%%%%%%%%%%%%%%%%%%%%%%%%%%%%%%%%%%%%%%%%%%%%%%%%%%%%%%%%%%%%%%%%%%%%%%%%%%%%%%%%
\subsection{Command Line Processing}
\label{sec:commandline}

The effect of redirection files can also be achieved by invoking
the \LaTeX{} compiler with a more elaborate command line.
Most conveniently this should be done as part
of a shell script or a |Makefile|.

When using \textsf{childdoc} in the main file, the following
command lines effectively perform a redirection
(note that depending on the shell being used,
backslashes may have to be doubled: `|\|' $\to$ `|\\|'):
%
\begin{center}
|... -jobname "|\textit{target}|" |\\|"|[\textit{flags}]%
|\input{childdoc.def}\childdocforward[|\textit{main}|]{|\textit{dest}|}"|
\end{center}
%
Here \textit{target} is the name of the output file,
\textit{main} is the name of the main file
and \textit{dest} is the name of the main or child file to be processed
(all filenames without extensions).
The optional argument \textit{main} can be omitted
if \textit{main} matches \textit{dest}.
Optionally, compilation \textit{flags} can be defined via |\def| commands.
This command line makes the \TeX{} engine believe
it is compiling the file \textit{target}
whose content is specified as the latter parameter.
The provided code then forwards the processing to
\textit{main} or \textit{dest} as described in \secref{sec:forward}.

%%%%%%%%%%%%%%%%%%%%%%%%%%%%%%%%%%%%%%%%%%%%%%%%%%%%%%%%%%%%%%%%%%%%%%%%%%%%%%%%
\subsection{Include by Input}
\label{sec:input}

Including child documents by |\include| has some restrictions by design.
Most notably, the content of a child document always occupies
its own set of pages; pages cannot be shared between child documents.
Usually, this behaviour makes perfect sense
because each child document contain an essential part of the document.
However, in some situations it may be desirable to compose
a document from a collection of parts
without having mandatory page breaks between then.
For this case, the package
provides a mechanism to include parts
by |\input| which can also be processed individually.
However, by construction this mechanism
requires manual handling of the content to be output.

%%%%%%%%%%%%%%%%%%%%%%%%%%%%%%%%%%%%%%%%
\DescribeMacro{\ifchilddocmanual}
The main file should be prepared as usual, see \secref{sec:include}.
However, the document body must make a distinction
between processing of an individual part and of the main document, e.g.:
%
\begin{center}
\begin{tabular}{l}
|\ifchilddocmanual|\\
|\input{\childdocname}|\\
|\||else|\\
\textit{document body with }|\input{|\textit{part}|}|\\
|\||fi|
\end{tabular}
\end{center}
%
The conditional |\ifchilddocmanual| is true whenever
a part to be included by |\input| is being compiled,
and the name of the part is stored in |\childdocname|.

%%%%%%%%%%%%%%%%%%%%%%%%%%%%%%%%%%%%%%%%
\DescribeMacro{\childdocby}
Each part to be included by |\input| should start with:
%
\begin{center}
\begin{tabular}{l}
|\input{childdoc.def}|\\
|\childdocby{|\textit{main}|}|\\
\end{tabular}
\end{center}
%
The directive |\childdocby| is similar to |\childdocof|
described in \secref{sec:include},
but the subsequent selection of content must be done manually.
To that end, both |\ifchilddoc| and |\ifchilddocmanual|
will be true upon processing of a part,
and the name of the part is stored in |\childdocname|.
Note that |\jobname| will be set to the filename of the current part
so that each part receives an individual |.aux| file
that does not interfere with the |.aux| file(s) of the main document.
This behaviour can be altered by the alternative form
|\childdocby[*]{|\textit{main}|}| (with a non-empty optional argument)
which uses the |.aux| file of the main document
by setting |\jobname| to \textit{main}.

%%%%%%%%%%%%%%%%%%%%%%%%%%%%%%%%%%%%%%%%%%%%%%%%%%%%%%%%%%%%%%%%%%%%%%%%%%%%%%%%
\subsection{Driver Development}
\label{sec:driver}

The \textsf{childdoc} mechanism can also be use for the development
of definition files such as \LaTeX{} styles or classes.
This case differs from the above setup with multiple parts
included by |\include| in that no |\includeonly| should be invoked.
This can be achieved by starting the include file
(before |\ProvidesPackage|) with:
%
\begin{center}
\begin{tabular}{l}
|\input{childdoc.def}|\\
|\childdocforward{|\textit{main}|}|\\
\end{tabular}
\end{center}
%
or alternatively with:
%
\begin{center}
\begin{tabular}{l}
|\input{childdoc.def}|\\
|\childdocby{|\textit{main}|}|\\
\end{tabular}
\end{center}
%
Both forms have slightly different effects as described above.
The main file is prepared as usual, see \secref{sec:include}.

%%%%%%%%%%%%%%%%%%%%%%%%%%%%%%%%%%%%%%%%%%%%%%%%%%%%%%%%%%%%%%%%%%%%%%%%%%%%%%%%
\subsection{Legacy Detection}
\label{sec:detection}

The directive |\childdocmain| in the main file can detect
whether the complete document or merely a child is to be compiled
even without using the directive |\childdocof|.
This method is deprecated because it is less robust
and there is no compelling reason to use it;
it is merely provided for backward compatibility
and it may be removed in future versions.

If the detection mechanism is to be used,
it is mandatory to correctly specify
the filename of the main file as the argument of |\childdocmain|:
%
\begin{center}
\begin{tabular}{l}
|\input{childdoc.def}|\\
|\childdocmain{|\textit{main}|}|\\
\end{tabular}
\end{center}
%
If |\jobname| does not match the argument \textit{main} of |\childdocmain|,
it is assumed that |\jobname| points to the child file to be compiled.
When using |\childdocmain| with the main file specified as argument,
it suffices to start a child file
with just |\input{|\textit{main}|}|
without loading of the package and using |\childdocof|.
If instead all processing is done
with the appropriate \textsf{childdoc} directives,
the argument of \textit{main} of |\childdocmain| can be empty.

An alternative version of the command line processing described
in \secref{sec:commandline} using the detection mechanism reads:
%
\begin{center}
|... -jobname "|\textit{target}|" "|[\textit{flags}]%
[|\def\jobname{|\textit{dest}|}|]|\input{|\textit{main}|}"|
\end{center}

%%%%%%%%%%%%%%%%%%%%%%%%%%%%%%%%%%%%%%%%%%%%%%%%%%%%%%%%%%%%%%%%%%%%%%%%%%%%%%%%
\subsection{Manual Code}
\label{sec:manual}

In case one cannot be certain whether the definitions file |childdoc.def|
is installed on the target \TeX{} distribution
and one prefers not to ship it,
it is conceivable to paste a few relevant commands into the sources.

To that end, drop all statements |\input{childdoc.def}|
and perform the replacements as outlined below.
Instead of |\childdocmain{|\textit{main}|}| add the following code
to the top of the main file:
%
\begin{center}
\begin{tabular}{l}
|\||ifdefined\childdocname\endinput\||fi\newif\ifchilddoc|\\
|\edef\childdocname{\scantokens\expandafter{\jobname\noexpand}}|\\
|\def\childdocmain{|\textit{main}|}\||ifx\childdocmain\childdocname\||else|\\
|\childdoctrue\includeonly{\childdocname}\let\jobname\childdocmain\||fi|\\
\end{tabular}
\end{center}
%
Instead of |\childdocof{|\textit{main}|}| just include the main file
at the top of each child file:
%
\begin{center}
|\input{|\textit{main}|}|
\end{center}
%
A simple redirection |\childdocforward{|\textit{dest}|}| is achieved by:
%
\begin{center}
|\def\jobname{|\textit{dest}|}\input{\jobname}|
\end{center}
%
The redirection with prefix
|\childdocforwardprefix[|\textit{prefix}|]{|\textit{dest}|}|
is accomplished by:
%
\begin{center}
\begin{tabular}{l}
|{\edef\jobname{\scantokens\expandafter{\jobname\noexpand}}|\\
|\def\redirectjob |\textit{prefix}|#1~~~{\gdef\jobname{|\textit{dest}|#1}}|\\
|\expandafter\redirectjob\jobname~~~}\input{\jobname}|
\end{tabular}
\end{center}

In an alternative approach,
child documents can be compiled by a specific command line
without additional code or specific definitions:
%
\begin{center}
|... -jobname "|\textit{target}|" "|[\textit{flags}]%
|\includeonly{|\textit{dest}|}\input{|\textit{main}|}"|
\end{center}
%

%%%%%%%%%%%%%%%%%%%%%%%%%%%%%%%%%%%%%%%%%%%%%%%%%%%%%%%%%%%%%%%%%%%%%%%%%%%%%%%%
%%%%%%%%%%%%%%%%%%%%%%%%%%%%%%%%%%%%%%%%%%%%%%%%%%%%%%%%%%%%%%%%%%%%%%%%%%%%%%%%
\section{Information}

%%%%%%%%%%%%%%%%%%%%%%%%%%%%%%%%%%%%%%%%%%%%%%%%%%%%%%%%%%%%%%%%%%%%%%%%%%%%%%%%
\subsection{Copyright}

Copyright \copyright{} 2017--2018 Niklas Beisert

This work may be distributed and/or modified under the
conditions of the \LaTeX{} Project Public License, either version 1.3
of this license or (at your option) any later version.
The latest version of this license is in
  \url{http://www.latex-project.org/lppl.txt}
and version 1.3 or later is part of all distributions of \LaTeX{}
version 2005/12/01 or later.

This work has the LPPL maintenance status `maintained'.

The Current Maintainer of this work is Niklas Beisert.

This work consists of the files |README.txt|, |childdoc.ins| and |childdoc.dtx|
as well as the derived files |childdoc.def|, |cdocsamp.tex|
with |cdocsch1.tex|, |cdocsch2.tex|, |cdocspt3.tex|, |cdocspt4.tex|,
|cdocsdrf.tex|, |cdocsfn1.tex|, |cdocsfn2.tex|
as well as |childdoc.pdf|.

%%%%%%%%%%%%%%%%%%%%%%%%%%%%%%%%%%%%%%%%%%%%%%%%%%%%%%%%%%%%%%%%%%%%%%%%%%%%%%%%
\subsection{Files and Installation}

The package consists of the files:
%
\begin{center}
\begin{tabular}{ll}
    |README.txt|   & readme file \\
    |childdoc.ins| & installation file \\
    |childdoc.dtx| & source file \\
    |childdoc.def| & definition file \\
    |cdocsamp.tex| & sample main file \\
    |cdocsch1.tex| & sample include file \\
    |cdocsch2.tex| & sample include file \\
    |cdocspt3.tex| & sample part file \\
    |cdocspt4.tex| & sample part file \\
    |cdocsdrf.tex| & sample redirection file \\
    |cdocsfn1.tex| & sample redirection file \\
    |cdocsfn2.tex| & sample redirection file \\
    |childdoc.pdf| & manual
\end{tabular}
\end{center}
%
The distribution consists of the files
|README.txt|, |childdoc.ins| and |childdoc.dtx|.
%
\begin{itemize}
\item
Run (pdf)\LaTeX{} on |childdoc.dtx|
to compile the manual |childdoc.pdf| (this file).
\item
Run \LaTeX{} on |childdoc.ins| to create the definitions file |childdoc.def|
and the sample |cdocsamp.tex| with include files
|cdocsch1.tex|, |cdocsch2.tex|, |cdocspt3.tex|, |cdocspt4.tex|,
|cdocsdrf.tex|, |cdocsfn1.tex|, |cdocsfn2.tex|.
Then copy the file |childdoc.def| to an appropriate directory of your \LaTeX{}
distribution, e.g.\ \textit{texmf-root}|/tex/latex/childdoc|.
\end{itemize}

%%%%%%%%%%%%%%%%%%%%%%%%%%%%%%%%%%%%%%%%%%%%%%%%%%%%%%%%%%%%%%%%%%%%%%%%%%%%%%%%
\subsection{Related CTAN Packages}

There are several other packages which offer a similar functionality:
%
\begin{itemize}
\item
The packages
\href{http://ctan.org/pkg/docmute}{\textsf{docmute}},
\href{http://ctan.org/pkg/includex}{\textsf{includex}} and
\href{http://ctan.org/pkg/standalone}{\textsf{standalone}}
provide commands to include only the document body of
a child file thus allowing both files to be compiled individually.
\item
The packages \href{http://ctan.org/pkg/subdocs}{\textsf{subdocs}}
and \href{http://ctan.org/pkg/subfiles}{\textsf{subfiles}}
provide structures in which the main and child documents can be
encapsulated and allowing them to be compiled individually.
The inclusion mechanism is different from the conventional |\include|.
\item
The package \href{http://ctan.org/pkg/combine}{\textsf{combine}}
is an elaborate solution to combine several documents into one.
\end{itemize}
%
See also the CTAN topic \href{http://ctan.org/topic/subdocs}{\textsf{subdocs}}
for further related packages.
The present package differs from the above solutions in that
a document structure constructed with the conventional |\include| mechanism
just needs two extra commands at the top of every file
such that all constituent files can be compiled individually.

%%%%%%%%%%%%%%%%%%%%%%%%%%%%%%%%%%%%%%%%%%%%%%%%%%%%%%%%%%%%%%%%%%%%%%%%%%%%%%%%
%\subsection{Feature Suggestions}
%
%The following is a list of features which may be useful for future
%versions of this package:
%%
%\begin{itemize}
%\item
%\ldots
%\end{itemize}

%%%%%%%%%%%%%%%%%%%%%%%%%%%%%%%%%%%%%%%%%%%%%%%%%%%%%%%%%%%%%%%%%%%%%%%%%%%%%%%%
\subsection{Revision History}

%%%%%%%%%%%%%%%%%%%%%%%%%%%%%%%%%%%%%%%%
\paragraph{v2.0:} 2018/12/30

\begin{itemize}
\item
immediate forward processing
\item
added |\childdocby| mechanism
\item
manual restructured
\end{itemize}

%%%%%%%%%%%%%%%%%%%%%%%%%%%%%%%%%%%%%%%%
\paragraph{v1.6:} 2018/01/17

\begin{itemize}
\item
application for development of include files
\item
corrections to manual
\end{itemize}

%%%%%%%%%%%%%%%%%%%%%%%%%%%%%%%%%%%%%%%%
\paragraph{v1.5:} 2017/05/21

\begin{itemize}
\item
more complete structuring introduced
\item
|\childdocof| introduced
\item
|\childdoc| renamed to |\childdocmain|
\item
|\childredirect| renamed to |\childdocforward| and |\childdocforwardprefix|
and functionality expanded
\end{itemize}

%%%%%%%%%%%%%%%%%%%%%%%%%%%%%%%%%%%%%%%%
\paragraph{v1.0:} 2017/04/27

\begin{itemize}
\item
manual and install package
\item
first version published on CTAN
\end{itemize}

%%%%%%%%%%%%%%%%%%%%%%%%%%%%%%%%%%%%%%%%
\paragraph{v0.6:} 2017/04/26

\begin{itemize}
\item
redirection mechanism added
\end{itemize}

%%%%%%%%%%%%%%%%%%%%%%%%%%%%%%%%%%%%%%%%
\paragraph{v0.5:} 2017/04/26

\begin{itemize}
\item
functionality in definition file
\end{itemize}


%%%%%%%%%%%%%%%%%%%%%%%%%%%%%%%%%%%%%%%%%%%%%%%%%%%%%%%%%%%%%%%%%%%%%%%%%%%%%%%%
%%%%%%%%%%%%%%%%%%%%%%%%%%%%%%%%%%%%%%%%%%%%%%%%%%%%%%%%%%%%%%%%%%%%%%%%%%%%%%%%
%%%%%%%%%%%%%%%%%%%%%%%%%%%%%%%%%%%%%%%%%%%%%%%%%%%%%%%%%%%%%%%%%%%%%%%%%%%%%%%%
\appendix

\settowidth\MacroIndent{\rmfamily\scriptsize 000\ }

 \DocInput{childdoc.dtx}

\end{document}
%</driver>
% \fi
%
% %%%%%%%%%%%%%%%%%%%%%%%%%%%%%%%%%%%%%%%%%%%%%%%%%%%%%%%%%%%%%%%%%%%%%%%%%%%%%%
% %%%%%%%%%%%%%%%%%%%%%%%%%%%%%%%%%%%%%%%%%%%%%%%%%%%%%%%%%%%%%%%%%%%%%%%%%%%%%%
% \section{Sample}
%\iffalse
%<*samplemain>
%\fi
%
% The following presents a sample document
% with two chapters, two parts, a title page,
% a compile flag as well as three forwarding files to set the flag.
% It consists of eight |.tex| files:
% \begin{center}
% \begin{tabular}{ll}
% |cdocsamp.tex|&main file\\
% |cdocsch1.tex|&include file for chapter 1\\
% |cdocsch2.tex|&include file for chapter 2\\
% |cdocspt3.tex|&include file for part 3\\
% |cdocspt4.tex|&include file for part 4\\
% |cdocsdrf.tex|&forwarding file for main file in draft mode\\
% |cdocsfi1.tex|&forwarding file for final version of chapter 1\\
% |cdocsfi2.tex|&forwarding file for final version of chapter 2\\
% \end{tabular}
% \end{center}
% Each of the eight files can be compiled directly by the \LaTeX{} compiler.
%
% %%%%%%%%%%%%%%%%%%%%%%%%%%%%%%%%%%%%%%
% \paragraph{Main File.}
%
% The main file is called |cdocsamp.tex|.
%
% Load the \textsf{childdoc} definitions and
% declare the filename for the main document:
%    \begin{macrocode}
\input{childdoc.def}
\childdocmain{}
%    \end{macrocode}

% Optional override for |\version| flag:
%    \begin{macrocode}
%%\ifchilddoc\else\providecommand{\version}{draft}\fi
%    \end{macrocode}

% Define the default values for the |\version| flag
% (|final| for the main file and |draft| for childs):
%    \begin{macrocode}
\ifchilddoc
\providecommand{\version}{draft}
\else
\providecommand{\version}{final}
\fi
%    \end{macrocode}

% Load the standard document class:
%    \begin{macrocode}
\documentclass[12pt]{article}
%    \end{macrocode}

% Start the document body:
%    \begin{macrocode}
\begin{document}
%    \end{macrocode}

% Declare a title page.
% Print title, part of document being processed and version flag:
%    \begin{macrocode}
\addtocounter{page}{-1}
\begin{center}
{\LARGE\bfseries{}childdoc example\par}
\vspace{1cm}
\ifchilddoc
\ifchilddocmanual part\else chapter\fi:
`\childdocname' of `\childdocjob'\par
\else
main document: `\childdocjob'\par
\fi
version: \version\par
\end{center}
\newpage
%    \end{macrocode}

% Manually include selected file,
% otherwise process as usual:
%    \begin{macrocode}
\ifchilddocmanual
\section*{part `\childdocname'}
\input{\childdocname}
\else
%    \end{macrocode}

% Include the two chapters:
%    \begin{macrocode}
\include{cdocsch1}
\include{cdocsch2}
%    \end{macrocode}

% Include the two parts unless only chapters should be displayed:
%    \begin{macrocode}
\ifchilddoc\else
\section{part three}
\input{cdocspt3}
\section{part four}
\input{cdocspt4}
\fi
%    \end{macrocode}

% Process as usual until here:
%    \begin{macrocode}
\fi
%    \end{macrocode}

% End of document body:
%    \begin{macrocode}
\end{document}
%    \end{macrocode}
%\iffalse
%</samplemain>
%\fi
%
% %%%%%%%%%%%%%%%%%%%%%%%%%%%%%%%%%%%%%%
% \paragraph{Chapter Include Files.}
%
% The include files are called |cdocsch1.tex| and |cdocsch2.tex|.
%
%\iffalse
%<*samplechap1|samplechap2>
%\fi

% Optional override for |\version| flag:
%    \begin{macrocode}
%%\providecommand{\version}{final}
%    \end{macrocode}

% Include the main document:
%    \begin{macrocode}
\input{childdoc.def}
\childdocof{cdocsamp}
%    \end{macrocode}

%\iffalse
%</samplechap1|samplechap2>
%\fi
%
%\iffalse
%<*samplechap1>
%\fi
% Some text for chapter 1:
%    \begin{macrocode}
\section{one}
some text in chapter one
%    \end{macrocode}

%\iffalse
%</samplechap1>
%\fi
% Some text for chapter 2:
%\iffalse
%<*samplechap2>
%\fi
%    \begin{macrocode}
\section{two}
more text in chapter two
%    \end{macrocode}

%\iffalse
%</samplechap2>
%\fi
%
% %%%%%%%%%%%%%%%%%%%%%%%%%%%%%%%%%%%%%%
% \paragraph{Part Include Files.}
%
% The include files are called |cdocspt3.tex| and |cdocspt4.tex|.
%
%\iffalse
%<*samplepart3|samplepart4>
%\fi

% Optional override for |\version| flag:
%    \begin{macrocode}
%%\providecommand{\version}{final}
%    \end{macrocode}

% Include the main document:
%    \begin{macrocode}
\input{childdoc.def}
\childdocby{cdocsamp}
%    \end{macrocode}

%\iffalse
%</samplepart3|samplepart4>
%\fi
%
%\iffalse
%<*samplepart3>
%\fi
% Some text for part 3:
%    \begin{macrocode}
some text in part three
%    \end{macrocode}

%\iffalse
%</samplepart3>
%\fi
% Some text for part 4:
%\iffalse
%<*samplepart4>
%\fi
%    \begin{macrocode}
more text in part four
%    \end{macrocode}

%\iffalse
%</samplepart4>
%\fi
%
% %%%%%%%%%%%%%%%%%%%%%%%%%%%%%%%%%%%%%%
% \paragraph{Forwarding for a Complete Draft.}
%
% The following forwarding file |cdocsdrf.tex|
% compiles the main document in draft mode:
%\iffalse
%<*sampledraft>
%\fi
%    \begin{macrocode}
\def\version{draft}
\input{childdoc.def}
\childdocforward{cdocsamp}
%    \end{macrocode}

%\iffalse
%</sampledraft>
%\fi
%
% %%%%%%%%%%%%%%%%%%%%%%%%%%%%%%%%%%%%%%
% \paragraph{Forwarding for Final Version of the Chapters.}
%
% The following forwarding files |cdocsfn1.tex| and |cdocsfn2.tex|
% (with identical content)
% compile the final versions of the child documents
% |cdocsch1.tex| and |cdocsch2.tex|, respectively:
%\iffalse
%<*samplefinal>
%\fi
%    \begin{macrocode}
\def\version{final}
\input{childdoc.def}
\childdocforwardprefix[cdocsamp]{cdocsfn}{cdocsch}
%    \end{macrocode}

%\iffalse
%</samplefinal>
%\fi
%
% %%%%%%%%%%%%%%%%%%%%%%%%%%%%%%%%%%%%%%
% \paragraph{Command Line Processing.}
%
% The following three command lines generate the output files
% |cdocscld|, |cdocscl1| and |cdocscl2|
% which should be identical to
% |cdocsdrf|, |cdocsch1| and |cdocsfn2|, respectively:
% \begin{center}
% \begin{tabular}{l}
% |latex -jobname cdocscld \|\\
% |  "\def\version{draft}\input{childdoc.def}\childdocforward{cdocsamp}"|\\
% |latex -jobname cdocscl1 \|\\
% |  "\input{childdoc.def}\childdocforward[cdocsamp]{cdocsch1}"|\\
% |latex -jobname cdocscl2 \|\\
% |  "\def\version{final}\input{childdoc.def}\childdocforward{cdocsch2}"|
% \end{tabular}
% \end{center}
% Note that the trailing backslash on each first line
% merely continues the input to the second line
% (for convenient cut ant paste).
% Furthermore, the command |latex| can be replaced by any
% of its alternative versions such as |pdflatex|.
%
% %%%%%%%%%%%%%%%%%%%%%%%%%%%%%%%%%%%%%%%%%%%%%%%%%%%%%%%%%%%%%%%%%%%%%%%%%%%%%%
% %%%%%%%%%%%%%%%%%%%%%%%%%%%%%%%%%%%%%%%%%%%%%%%%%%%%%%%%%%%%%%%%%%%%%%%%%%%%%%
% \section{Implementation}
%\iffalse
%<*package>
%\fi
%
% This section describes the definitions file |childdoc.def|.

% The definitions cannot be loaded using |\usepackage| or |\RequirePackage|
% which has a mechanism to prevent loading a style file more than once.
% When loading the definitions by means of |\input|
% multiple instances have to be prevented manually:
%\iffalse
%This code needs to be before the `\ProvidesFile' directive
%which is defined at the beginning of this file.
%Therefore it is also placed there and commented out here.
%</package>
%<*discard>
%\fi
%    \begin{macrocode}
\ifdefined\childdocmain\endinput\fi
%    \end{macrocode}
%\iffalse
%</discard>
%<*package>
%\fi
%
% \macro{\ifchilddoc}
% \macro{\ifchilddocmanual}
% The conditional |\ifchilddoc| tells whether a
% child (true) or main (false) document is being compiled.
% The conditional |\ifchilddocmanual| tells whether
% the |\includeonly| mechanism is used (false) or
% the selection of child files must be performed manually (true).
% The definitions initialise to false:
%    \begin{macrocode}
\newif\ifchilddoc
\newif\ifchilddocmanual
%    \end{macrocode}

% \macro{\childdocname}
% \macro{\childdocjob}
% The macro |\childdocname| stores the name of the main document
% to be compiled. The macro |\childdocjob| stores the name of
% the document on which the \LaTeX{} compiler was originally invoked.
% The content of |\jobname| cannot be compared
% to filenames specified in the source due to different catcodes.
% The following code rescans |\jobname|, stores the result
% in |\childdocname| and saves a copy in |\childdocjob|:
%    \begin{macrocode}
\edef\childdocname{\scantokens\expandafter{\jobname\noexpand}}
\let\childdocjob\childdocname
%    \end{macrocode}

% \macro{\childdocdisable}
% The macro |\childdocdisable| prevents the main file
% from being processed more than once.
% At this stage, the main document command |\childdocmain|
% is assumed to be called once again where it should do nothing.
% Any subsequent call to it should prevent
% a secondary processing of the main document
% It overwrites the forwarding commands
% |\childdocof| and |\childdocforward|
% with empty macros to prevent further inclusions of the main document:
%    \begin{macrocode}
\newcommand{\childdocdisable}
{
  \renewcommand{\childdocmain}[1]{\renewcommand{\childdocmain}[1]{\endinput}}
  \renewcommand{\childdocof}[1]{}
  \renewcommand{\childdocby}[2][]{}
  \renewcommand{\childdocforward}[2][]{}
  \renewcommand{\childdocdisable}{}
}
%    \end{macrocode}

% \macro{\childdocmain}
% The macro |\childdocmain| is to be called at the top of the main file
% with nothing or the main filename (without extension) as argument.
% First, it breaks loops.
% If the argument is not empty and does not match |\childdocname|
% (which is set by the first inclusion of |childdoc.def|),
% |\ifchilddoc| is set to true, |\includeonly| is applied to the child file
% and |\jobname| is set to the main file
% (for proper handling of |.aux| files):
%    \begin{macrocode}
\newcommand{\childdocmain}[1]
{
  \childdocdisable\childdocmain{}
  \if?#1?\else
    \begingroup
      \def\childdoctmp{#1}
      \ifx\childdoctmp\childdocname
        \def\childdoctmp{}
      \else
        \def\childdoctmp
        {
          \childdoctrue
          \includeonly{\childdocname}
          \def\childdocjob{#1}
          \def\jobname{#1}
        }
      \fi
      \expandafter
    \endgroup
    \childdoctmp
  \fi
}
%    \end{macrocode}

% \macro{\childdocof}
% The command |\childdocof| redirects
% compilation to the main file |#1|.
%    \begin{macrocode}
\newcommand{\childdocof}[1]
{
  \childdocdisable
  \childdoctrue
  \includeonly{\childdocname}
  \def\jobname{#1}
  \def\childdocjob{#1}
  \input{#1}
}
%    \end{macrocode}

% \macro{\childdocby}
% The command |\childdocby| ....
%    \begin{macrocode}
\newcommand{\childdocby}[2][]
{
  \childdocdisable
  \childdoctrue
  \childdocmanualtrue
  \if?#1?\else
    \def\jobname{#2}
  \fi
  \def\childdocjob{#2}
  \input{#2}
  \endinput
}
%    \end{macrocode}

% \macro{\childdocforward}
% The command |\childdocforward| redirects
% compilation to the main file or
% (if the optional argument is given) a child file.
% Parameters are set as if the main file
% or a child file starting with |\childdocof| was compiled.
% Then compilation is handed over to the main file:
%    \begin{macrocode}
\newcommand{\childdocforward}[2][]
{
  \begingroup
    \if?#1?
      \def\childdoctmp
      {
        \def\childdocname{#2}
        \def\childdocjob{#2}
        \def\jobname{#2}
        \input{#2}
        \endinput
      }
    \else
      \def\childdoctmp
      {
        \childdocdisable
        \def\childdocname{#2}
        \childdoctrue
        \includeonly{#2}
        \def\childdocjob{#1}
        \def\jobname{#1}
        \input{#1}
        \endinput
      }
    \fi
    \expandafter
  \endgroup
  \childdoctmp
}
%    \end{macrocode}

% \macro{\childdocforwardprefix}
% The command |\childdocforwardprefix| redirects
% compilation to the main or a child file by means of a pattern.
% The prefix |#1| in the current filename is replaced by |#2|
% and the suffix of the current filename is kept
% (it is assumed that the filename does not contain the substring `|~~~|'
% which is used as a delimiter).
% Compilation is handed over to the new file by |\childdocforward|:
%    \begin{macrocode}
\newcommand{\childdocforwardprefix}[3][]
{
  \begingroup
    \def\childdocextract #2##1~~~{\def\childdoctmp{\childdocforward[#1]{#3##1}}}
    \expandafter\childdocextract\childdocname~~~
    \expandafter
  \endgroup
  \childdoctmp
}
%    \end{macrocode}

% \macro{\childdoc}
% The deprecated macro |\childdoc| is a legacy version of |\childdocmain|:
%    \begin{macrocode}
\newcommand{\childdoc}{\childdocmain}
%    \end{macrocode}

% \macro{\childdocredirect}
% The deprecated macro |\childdocredirect| is a legacy version
% of |\childdocforward| and |\childdocforwardprefix|:
%    \begin{macrocode}
\newcommand{\childdocredirect}[2][]
{
  \begingroup
    \if?#1?
      \def\childdoctmp{\childdocforward{#2}}
    \else
      \def\childdoctmp{\childdocforwardprefix{#1}{#2}}
    \fi
    \expandafter
  \endgroup
  \childdoctmp
}
%    \end{macrocode}

%\iffalse
%</package>
%\fi
%
\endinput
|\\
|\childdocmain{|\textit{main}|}|\\
\end{tabular}
\end{center}
%
If |\jobname| does not match the argument \textit{main} of |\childdocmain|,
it is assumed that |\jobname| points to the child file to be compiled.
When using |\childdocmain| with the main file specified as argument,
it suffices to start a child file
with just |\input{|\textit{main}|}|
without loading of the package and using |\childdocof|.
If instead all processing is done
with the appropriate \textsf{childdoc} directives,
the argument of \textit{main} of |\childdocmain| can be empty.

An alternative version of the command line processing described
in \secref{sec:commandline} using the detection mechanism reads:
%
\begin{center}
|... -jobname "|\textit{target}|" "|[\textit{flags}]%
[|\def\jobname{|\textit{dest}|}|]|\input{|\textit{main}|}"|
\end{center}

%%%%%%%%%%%%%%%%%%%%%%%%%%%%%%%%%%%%%%%%%%%%%%%%%%%%%%%%%%%%%%%%%%%%%%%%%%%%%%%%
\subsection{Manual Code}
\label{sec:manual}

In case one cannot be certain whether the definitions file |childdoc.def|
is installed on the target \TeX{} distribution
and one prefers not to ship it,
it is conceivable to paste a few relevant commands into the sources.

To that end, drop all statements |% \iffalse
%
% childdoc.dtx Copyright (C) 2017-2018 Niklas Beisert
%
% This work may be distributed and/or modified under the
% conditions of the LaTeX Project Public License, either version 1.3
% of this license or (at your option) any later version.
% The latest version of this license is in
%   http://www.latex-project.org/lppl.txt
% and version 1.3 or later is part of all distributions of LaTeX
% version 2005/12/01 or later.
%
% This work has the LPPL maintenance status `maintained'.
%
% The Current Maintainer of this work is Niklas Beisert.
%
% This work consists of the files childdoc.dtx and childdoc.ins
% and the derived files childdoc.def and cdocsamp.tex with
% cdocsch1.tex, cdocsch2.tex, cdocsdrf.tex, cdocsfn1.tex, cdocsfn2.tex.
%
%<package>\ifdefined\childdocmain\endinput\fi
%<package>\ProvidesFile{childdoc.def}[2018/12/30 v2.0 child document driver]
%<samplemain>\ProvidesFile{cdocsamp.tex}[2018/12/30 v2.0 sample for childdoc]
%<*driver>
%\ProvidesFile{childdoc.drv}[2018/12/30 v2.0 childdoc reference manual file]
\PassOptionsToClass{10pt,a4paper}{article}
\documentclass{ltxdoc}

\usepackage[margin=35mm]{geometry}
\usepackage{hyperref}
\usepackage{hyperxmp}
\usepackage[usenames]{color}

\hypersetup{colorlinks=true}
\hypersetup{pdfstartview=FitH}
\hypersetup{pdfpagemode=UseNone}
\hypersetup{pdfsource={}}
\hypersetup{pdflang={en-UK}}
\hypersetup{pdfcopyright={Copyright 2017-2018 Niklas Beisert.
  This work may be distributed and/or modified under the
  conditions of the LaTeX Project Public License, either version 1.3
  of this license or (at your option) any later version.}}
\hypersetup{pdflicenseurl={http://www.latex-project.org/lppl.txt}}
\hypersetup{pdfcontactaddress={ETH Zurich, ITP, HIT K,
  Wolfgang-Pauli-Strasse 27}}
\hypersetup{pdfcontactpostcode={8093}}
\hypersetup{pdfcontactcity={Zurich}}
\hypersetup{pdfcontactcountry={Switzerland}}
\hypersetup{pdfcontactemail={nbeisert@itp.phys.ethz.ch}}
\hypersetup{pdfcontacturl={http://people.phys.ethz.ch/\xmptilde nbeisert/}}

\newcommand{\secref}[1]{\hyperref[#1]{section \ref*{#1}}}

\parskip1ex
\parindent0pt
\let\olditemize\itemize
\def\itemize{\olditemize\parskip0pt}

\begin{document}

\title{The \textsf{childdoc} Package}
\hypersetup{pdftitle={The childdoc Package}}
\author{Niklas Beisert\\[2ex]
  Institut f\"ur Theoretische Physik\\
  Eidgen\"ossische Technische Hochschule Z\"urich\\
  Wolfgang-Pauli-Strasse 27, 8093 Z\"urich, Switzerland\\[1ex]
  \href{mailto:nbeisert@itp.phys.ethz.ch}
  {\texttt{nbeisert@itp.phys.ethz.ch}}}
\hypersetup{pdfauthor={Niklas Beisert}}
\hypersetup{pdfsubject={Manual for the LaTeX2e Package childdoc}}
\date{30 December 2018, \textsf{v2.0}}
\maketitle

\begin{abstract}\noindent
\textsf{childdoc} is a \LaTeXe{} package
that enables the direct compilation
of document sections included by |\include|
to individual files.
\end{abstract}

\begingroup
\parskip0ex
\tableofcontents
\endgroup

%%%%%%%%%%%%%%%%%%%%%%%%%%%%%%%%%%%%%%%%%%%%%%%%%%%%%%%%%%%%%%%%%%%%%%%%%%%%%%%%
%%%%%%%%%%%%%%%%%%%%%%%%%%%%%%%%%%%%%%%%%%%%%%%%%%%%%%%%%%%%%%%%%%%%%%%%%%%%%%%%
\section{Introduction}

\LaTeX{} provides a mechanism to structure a large document (such as a book)
into a main file and several child files (containing the chapters)
using the |\include| command.
This mechanism is beneficial for documents
which span hundreds of pages in order to
make the source file(s) more manageable.
Moreover, compilation can be restricted to
selected child files by means of the |\includeonly| command.
The latter feature can be used to reduce the compilation time while editing
(this was significantly more useful in the earlier days of \LaTeX{})
or to generate a smaller document which is easier to navigate.
Another application of |\includeonly| is to generate
documents consisting of selected parts of the complete document.

However, there are a few drawbacks of the plain |\include| mechanism:
\begin{itemize}
\item
The child files cannot be compiled on their own,
they can only be compiled via the main file.
A naive editing environment
(such as a text editor with an option
to have the current file processed by \LaTeX)
may require one to switch to the main file before compiling;
attempting to compile the child file produces errors.
\item
The main file must be modified (each time)
to adjust the |\includeonly| command
to the present needs. This easily leaves the main file in a messy state.
\item
The generated document will always carry the filename
of the main document. This is inconvenient if
several child files are to be compiled and
to be kept for distribution.
\end{itemize}

The present package provides a simple interface
to make child files individually compilable by \LaTeX{}.
Compiling a child file then has the same effect as compiling
the main file with an |\includeonly| command
to select the appropriate child.
Moreover the generated document will carry the name of the child
rather than the main file.
This resolves all three above issues.

This feature is meant to make the editing of books,
thesis documents and lecture notes somewhat more convenient.
However, the package can also be used efficiently for
composing a series of documents (such as exercise sheets)
which are typically distributed individually.
It then assists the author in generating the individual documents
(potentially in different versions)
as well as a document containing the collected series.
Another application is in developing style files
or other kinds of included material
where compilation of the style file could redirect
to a sample or test file.

%%%%%%%%%%%%%%%%%%%%%%%%%%%%%%%%%%%%%%%%%%%%%%%%%%%%%%%%%%%%%%%%%%%%%%%%%%%%%%%%
%%%%%%%%%%%%%%%%%%%%%%%%%%%%%%%%%%%%%%%%%%%%%%%%%%%%%%%%%%%%%%%%%%%%%%%%%%%%%%%%
\section{Usage}

First of all, the package \textsf{childdoc} is \emph{not} a standard
\LaTeXe{} |.sty| style file! Therefore it needs to be invoked in
a non-standard way.

%%%%%%%%%%%%%%%%%%%%%%%%%%%%%%%%%%%%%%%%%%%%%%%%%%%%%%%%%%%%%%%%%%%%%%%%%%%%%%%%
\subsection{Included Files}
\label{sec:include}

%%%%%%%%%%%%%%%%%%%%%%%%%%%%%%%%%%%%%%%%
\DescribeMacro{\childdocmain}
To use the package, add the commands
\begin{center}
\begin{tabular}{l}
|\input{childdoc.def}|\\
|\childdocmain{}|\\
\end{tabular}
\end{center}
at the very top of the main \LaTeX{} file,
in particular \emph{before} the |\documentclass| statement!
The argument of |\childdocmain| should be left empty
(but it must be present).

%%%%%%%%%%%%%%%%%%%%%%%%%%%%%%%%%%%%%%%%
\DescribeMacro{\childdocof}
Furthermore, add the commands
\begin{center}
\begin{tabular}{l}
|\input{childdoc.def}|\\
|\childdocof{|\textit{main}|}|\\
\end{tabular}
\end{center}
at the top of every child file \textit{child}
which is included by |\include{|\textit{child}|}|
from within the main file
(or at least for those files to be compiled individually).
The argument \textit{main} must be the filename of the main file.

There are a couple of
considerations in setting up the main and child documents:

%%%%%%%%%%%%%%%%%%%%%%%%%%%%%%%%%%%%%%%%
\paragraph{Restrictions.}

Please note the following restrictions:
\begin{itemize}
\item
|\childdocmain| must be called with one argument \textit{main}
to ensure compatibility with earlier version of the package.
It must either be empty (|\childdocmain{}|)
or precisely match the filename of the main file in which it is specified.
See \secref{sec:detection} for further information.
\item
The filename \textit{main} must be specified without the |.tex| extension.
\item
The filename \textit{main} is case sensitive
(even in case-insensitive file systems)
due to internal string comparison.
\item
The argument \textit{main} should be fully expanded, it cannot be a macro.
\item
Subdirectories and special characters should be avoided in filenames.
\item
The command |\childdocmain{|\textit{main}|}| must be followed by a whitespace.
It should not be followed immediately by another command
or by a comment mark `|%|'.
This is because the \TeX{} parser reads the token immediately following
the argument of |\childdocmain| and puts it
at the beginning of every child section;
however, a white\-space is ignored.
\end{itemize}

%%%%%%%%%%%%%%%%%%%%%%%%%%%%%%%%%%%%%%%%
\paragraph{Content of Main File.}

It is advisable to place all content in the child files included by |\include|.
Any output contained in the main file will appear in all child documents
unless suppressed manually;
it cannot be suppressed automatically by the |\includeonly| directive
and thus should normally be avoided.
A method to include some content in the main file
by means of conditional processing is described in \secref{sec:conditional}.

%%%%%%%%%%%%%%%%%%%%%%%%%%%%%%%%%%%%%%%%
\paragraph{Page Numbering.}

When only a part of the document is compiled,
the appropriate numbering of pages
(as well as other status parameters)
is determined from the |.aux| files.
The latter contain information from previous passes.
However this information needs to propagate through
all intermediate child documents.
Therefore the page numbering in child documents may well
be inconsistent until the complete document is compiled at least once.

A useful (if unconventional) way to always ensure a consistent
page numbering is to restart the numbering in each child document
and denote the pages by `\textit{child}|.|\textit{page}'
where \textit{child} represents the chapter/section number of the child file.
This can be achieved by the command
|\numberwithin{page}{|\textit{child}|}|
of the \textsf{amsmath} package
where \textit{child} can be |chapter| or |section|
depending on the chosen structuring.
Alternatively, one can modify the macro |\thepage| appropriately
and reset the counter |page| at the start of each child file.

%%%%%%%%%%%%%%%%%%%%%%%%%%%%%%%%%%%%%%%%%%%%%%%%%%%%%%%%%%%%%%%%%%%%%%%%%%%%%%%%
\subsection{Conditional Processing}
\label{sec:conditional}

The package provides a mechanism to compile different versions
of a document. To customise the versions further some conditional processing
can come in handy to distinguish which version is being compiled.
The package provides two macros to describe the compilation context:

%%%%%%%%%%%%%%%%%%%%%%%%%%%%%%%%%%%%%%%%
\DescribeMacro{\ifchilddoc}
The conditional |\ifchilddoc| distinguishes between the compilation of
child documents and the main document:
%
\begin{center}
|\ifchilddoc |\textit{child-code}| |[|\||else |\textit{main-code}]| \||fi|
\end{center}

%%%%%%%%%%%%%%%%%%%%%%%%%%%%%%%%%%%%%%%%
\DescribeMacro{\childdocname}
\DescribeMacro{\childdocjob}
The macro |\childdocname| contains the filename (without extension)
of the main or child file being processed.
Note that |\childdocjob| will always contain the name of the main file.

%%%%%%%%%%%%%%%%%%%%%%%%%%%%%%%%%%%%%%%%
\paragraph{Title Page.}

Conditional processing can be used to include a title or banner page
in the main document when proper precautions are taken.
Importantly, the code in the main file should ensure that the page counter
(as well as other status parameters which are stored in the |.aux| files)
takes the same value after the conditional processing.
Otherwise the page numbers may take divergent values
depending on which part is compiled.

For example, a title page could be declared by:
%
\begin{center}
\begin{tabular}{l}
|\ifchilddoc\||else|\\
|\addtocounter{page}{-1}|\\
\textit{code for title page}\\
|\newpage|\\
|\||fi|
\end{tabular}
\end{center}
%
A banner page for the child documents can be generated by:
%
\begin{center}
\begin{tabular}{l}
|\ifchilddoc|\\
|\addtocounter{page}{-1}|\\
\textit{code for banner page}\\
|\newpage|\\
|\||fi|
\end{tabular}
\end{center}
%
Here one could write a message such as:
\begin{center}
|This is the part \childdocname{} of \childdocjob{}.|
\end{center}

%%%%%%%%%%%%%%%%%%%%%%%%%%%%%%%%%%%%%%%%%%%%%%%%%%%%%%%%%%%%%%%%%%%%%%%%%%%%%%%%
\subsection{Flags}
\label{sec:flags}

The package makes it easy to generate different versions
of the main or child documents.
To this end compilation flags can be defined
and assigned different default values.
They will be particularly useful in conjunction
with the forwarding mechanism described in \secref{sec:forward}.

For example, it may be useful to have a flag |\version|
which can be set to |draft| or |final|.
The document source will contain some conditional code
depending on the value of |\version|.
Suppose further, the flag should default to |final| for the main file
and to |draft| for child files
which is a natural assignment for editing the document.
This is achieved by placing the following code
in the preamble of the main document
(below the |\childdocmain| directive):
%
\begin{center}
\begin{tabular}{l}
|\ifchilddoc|\\
|\providecommand{\version}{draft}|\\
|\||else|\\
|\providecommand{\version}{final}|\\
|\||fi|
\end{tabular}
\end{center}
%
The definition by |\providecommand| makes sure
that previous definitions are not overwritten.
Further statements |\providecommand{\version}{...}|
can thus be added before the above code to override it.

For the main file, one might add a line
(between |\childdocmain| and the above block)
%
\begin{center}
|%\ifchilddoc\||else\providecommand{\version}{draft}\||fi|
\end{center}
%
which can be uncommented to produce a draft version.
Likewise one can add a line to the very top of a child file
(above the |\childdocof{|\textit{main}|}| directive)
%
\begin{center}
|%\providecommand{\version}{final}|
\end{center}
%
which can be uncommented to produce the final version of this child document.

%%%%%%%%%%%%%%%%%%%%%%%%%%%%%%%%%%%%%%%%%%%%%%%%%%%%%%%%%%%%%%%%%%%%%%%%%%%%%%%%
\subsection{Forwarding}
\label{sec:forward}

Different versions of the main or child documents
using compilation flags as described in \secref{sec:flags}
can be (permanently) stored in different files
for convenient compilation, viewing and distribution.
To this end, the package defines a command
to pass on compilation to a different file:

%%%%%%%%%%%%%%%%%%%%%%%%%%%%%%%%%%%%%%%%
\DescribeMacro{\childdocforward}
The command |\childdocforward| redirects processing to
another source file:
%
\begin{center}
\begin{tabular}{l}
|\input{childdoc.def}|\\
|\childdocforward[|\textit{main}|]{|\textit{dest}|}|\\
\end{tabular}
\end{center}
%
The argument \textit{dest} is the destination file
(without extension).
It should be the main file or one of the child files.
Note that further \textsf{childdoc} directives
such as |\childdocof| and |\childdocforward|
in the indicated file will be processed in this form.
The optional argument \textit{main}
passes on directly to the main file \textit{main}
while pretending to compile the child \textit{dest}.
This form behaves as if \textit{dest}
issues |\childdocof{|\textit{main}|}| right away,
and no further \textsf{childdoc} directives will be processed.

%%%%%%%%%%%%%%%%%%%%%%%%%%%%%%%%%%%%%%%%
\DescribeMacro{\...prefix}
In the alternative form |\childdocforwardprefix|,
%
\begin{center}
\begin{tabular}{l}
|\input{childdoc.def}|\\
|\childdocforwardprefix[|\textit{main}|]{|\textit{prefix}|}{|\textit{dest}|}|
\end{tabular}
\end{center}
%
the destination file is determined by a pattern
depending on the current file:
To make this work, the current file must be called
`{\textit{prefix}\hspace{0.2em}\textit{suffix}}'
with \textit{prefix} matching precisely the argument.
Processing is then passed on to the file
`{\textit{dest}\hspace{0.2em}\textit{suffix}}'.
Surely, the same effect is achieved by
directly specifying the
argument `{\textit{dest}\hspace{0.2em}\textit{suffix}}'
in the first form.
However, that requires to set up a different file
for each child. With the alternative form of the command
all these files can have exactly the same content
which simplifies setting them up and maintaining them.

For example, the following file |draft.tex|
with a compilation flag |\version| as described in \secref{sec:flags}
compiles the main document as a draft:
%
\begin{center}
\begin{tabular}{l}
|\def\version{draft}|\\
|\input{childdoc.def}|\\
|\childdocforward{|\textit{main}|}|
\end{tabular}
\end{center}
%
Likewise, the following files |final|\textit{nn}|.tex|
compile the final version of the child document
|child|\textit{nn}|.tex|:
%
\begin{center}
\begin{tabular}{l}
|\def\version{final}|\\
|\input{childdoc.def}|\\
|\childdocforwardprefix{final}{child}|
\end{tabular}
\end{center}
%

Note that when several versions of a main file and/or of each child file
are to be generated, it may be convenient to set up a |Makefile| or
shell script to automatise the process.

%%%%%%%%%%%%%%%%%%%%%%%%%%%%%%%%%%%%%%%%%%%%%%%%%%%%%%%%%%%%%%%%%%%%%%%%%%%%%%%%
\subsection{Command Line Processing}
\label{sec:commandline}

The effect of redirection files can also be achieved by invoking
the \LaTeX{} compiler with a more elaborate command line.
Most conveniently this should be done as part
of a shell script or a |Makefile|.

When using \textsf{childdoc} in the main file, the following
command lines effectively perform a redirection
(note that depending on the shell being used,
backslashes may have to be doubled: `|\|' $\to$ `|\\|'):
%
\begin{center}
|... -jobname "|\textit{target}|" |\\|"|[\textit{flags}]%
|\input{childdoc.def}\childdocforward[|\textit{main}|]{|\textit{dest}|}"|
\end{center}
%
Here \textit{target} is the name of the output file,
\textit{main} is the name of the main file
and \textit{dest} is the name of the main or child file to be processed
(all filenames without extensions).
The optional argument \textit{main} can be omitted
if \textit{main} matches \textit{dest}.
Optionally, compilation \textit{flags} can be defined via |\def| commands.
This command line makes the \TeX{} engine believe
it is compiling the file \textit{target}
whose content is specified as the latter parameter.
The provided code then forwards the processing to
\textit{main} or \textit{dest} as described in \secref{sec:forward}.

%%%%%%%%%%%%%%%%%%%%%%%%%%%%%%%%%%%%%%%%%%%%%%%%%%%%%%%%%%%%%%%%%%%%%%%%%%%%%%%%
\subsection{Include by Input}
\label{sec:input}

Including child documents by |\include| has some restrictions by design.
Most notably, the content of a child document always occupies
its own set of pages; pages cannot be shared between child documents.
Usually, this behaviour makes perfect sense
because each child document contain an essential part of the document.
However, in some situations it may be desirable to compose
a document from a collection of parts
without having mandatory page breaks between then.
For this case, the package
provides a mechanism to include parts
by |\input| which can also be processed individually.
However, by construction this mechanism
requires manual handling of the content to be output.

%%%%%%%%%%%%%%%%%%%%%%%%%%%%%%%%%%%%%%%%
\DescribeMacro{\ifchilddocmanual}
The main file should be prepared as usual, see \secref{sec:include}.
However, the document body must make a distinction
between processing of an individual part and of the main document, e.g.:
%
\begin{center}
\begin{tabular}{l}
|\ifchilddocmanual|\\
|\input{\childdocname}|\\
|\||else|\\
\textit{document body with }|\input{|\textit{part}|}|\\
|\||fi|
\end{tabular}
\end{center}
%
The conditional |\ifchilddocmanual| is true whenever
a part to be included by |\input| is being compiled,
and the name of the part is stored in |\childdocname|.

%%%%%%%%%%%%%%%%%%%%%%%%%%%%%%%%%%%%%%%%
\DescribeMacro{\childdocby}
Each part to be included by |\input| should start with:
%
\begin{center}
\begin{tabular}{l}
|\input{childdoc.def}|\\
|\childdocby{|\textit{main}|}|\\
\end{tabular}
\end{center}
%
The directive |\childdocby| is similar to |\childdocof|
described in \secref{sec:include},
but the subsequent selection of content must be done manually.
To that end, both |\ifchilddoc| and |\ifchilddocmanual|
will be true upon processing of a part,
and the name of the part is stored in |\childdocname|.
Note that |\jobname| will be set to the filename of the current part
so that each part receives an individual |.aux| file
that does not interfere with the |.aux| file(s) of the main document.
This behaviour can be altered by the alternative form
|\childdocby[*]{|\textit{main}|}| (with a non-empty optional argument)
which uses the |.aux| file of the main document
by setting |\jobname| to \textit{main}.

%%%%%%%%%%%%%%%%%%%%%%%%%%%%%%%%%%%%%%%%%%%%%%%%%%%%%%%%%%%%%%%%%%%%%%%%%%%%%%%%
\subsection{Driver Development}
\label{sec:driver}

The \textsf{childdoc} mechanism can also be use for the development
of definition files such as \LaTeX{} styles or classes.
This case differs from the above setup with multiple parts
included by |\include| in that no |\includeonly| should be invoked.
This can be achieved by starting the include file
(before |\ProvidesPackage|) with:
%
\begin{center}
\begin{tabular}{l}
|\input{childdoc.def}|\\
|\childdocforward{|\textit{main}|}|\\
\end{tabular}
\end{center}
%
or alternatively with:
%
\begin{center}
\begin{tabular}{l}
|\input{childdoc.def}|\\
|\childdocby{|\textit{main}|}|\\
\end{tabular}
\end{center}
%
Both forms have slightly different effects as described above.
The main file is prepared as usual, see \secref{sec:include}.

%%%%%%%%%%%%%%%%%%%%%%%%%%%%%%%%%%%%%%%%%%%%%%%%%%%%%%%%%%%%%%%%%%%%%%%%%%%%%%%%
\subsection{Legacy Detection}
\label{sec:detection}

The directive |\childdocmain| in the main file can detect
whether the complete document or merely a child is to be compiled
even without using the directive |\childdocof|.
This method is deprecated because it is less robust
and there is no compelling reason to use it;
it is merely provided for backward compatibility
and it may be removed in future versions.

If the detection mechanism is to be used,
it is mandatory to correctly specify
the filename of the main file as the argument of |\childdocmain|:
%
\begin{center}
\begin{tabular}{l}
|\input{childdoc.def}|\\
|\childdocmain{|\textit{main}|}|\\
\end{tabular}
\end{center}
%
If |\jobname| does not match the argument \textit{main} of |\childdocmain|,
it is assumed that |\jobname| points to the child file to be compiled.
When using |\childdocmain| with the main file specified as argument,
it suffices to start a child file
with just |\input{|\textit{main}|}|
without loading of the package and using |\childdocof|.
If instead all processing is done
with the appropriate \textsf{childdoc} directives,
the argument of \textit{main} of |\childdocmain| can be empty.

An alternative version of the command line processing described
in \secref{sec:commandline} using the detection mechanism reads:
%
\begin{center}
|... -jobname "|\textit{target}|" "|[\textit{flags}]%
[|\def\jobname{|\textit{dest}|}|]|\input{|\textit{main}|}"|
\end{center}

%%%%%%%%%%%%%%%%%%%%%%%%%%%%%%%%%%%%%%%%%%%%%%%%%%%%%%%%%%%%%%%%%%%%%%%%%%%%%%%%
\subsection{Manual Code}
\label{sec:manual}

In case one cannot be certain whether the definitions file |childdoc.def|
is installed on the target \TeX{} distribution
and one prefers not to ship it,
it is conceivable to paste a few relevant commands into the sources.

To that end, drop all statements |\input{childdoc.def}|
and perform the replacements as outlined below.
Instead of |\childdocmain{|\textit{main}|}| add the following code
to the top of the main file:
%
\begin{center}
\begin{tabular}{l}
|\||ifdefined\childdocname\endinput\||fi\newif\ifchilddoc|\\
|\edef\childdocname{\scantokens\expandafter{\jobname\noexpand}}|\\
|\def\childdocmain{|\textit{main}|}\||ifx\childdocmain\childdocname\||else|\\
|\childdoctrue\includeonly{\childdocname}\let\jobname\childdocmain\||fi|\\
\end{tabular}
\end{center}
%
Instead of |\childdocof{|\textit{main}|}| just include the main file
at the top of each child file:
%
\begin{center}
|\input{|\textit{main}|}|
\end{center}
%
A simple redirection |\childdocforward{|\textit{dest}|}| is achieved by:
%
\begin{center}
|\def\jobname{|\textit{dest}|}\input{\jobname}|
\end{center}
%
The redirection with prefix
|\childdocforwardprefix[|\textit{prefix}|]{|\textit{dest}|}|
is accomplished by:
%
\begin{center}
\begin{tabular}{l}
|{\edef\jobname{\scantokens\expandafter{\jobname\noexpand}}|\\
|\def\redirectjob |\textit{prefix}|#1~~~{\gdef\jobname{|\textit{dest}|#1}}|\\
|\expandafter\redirectjob\jobname~~~}\input{\jobname}|
\end{tabular}
\end{center}

In an alternative approach,
child documents can be compiled by a specific command line
without additional code or specific definitions:
%
\begin{center}
|... -jobname "|\textit{target}|" "|[\textit{flags}]%
|\includeonly{|\textit{dest}|}\input{|\textit{main}|}"|
\end{center}
%

%%%%%%%%%%%%%%%%%%%%%%%%%%%%%%%%%%%%%%%%%%%%%%%%%%%%%%%%%%%%%%%%%%%%%%%%%%%%%%%%
%%%%%%%%%%%%%%%%%%%%%%%%%%%%%%%%%%%%%%%%%%%%%%%%%%%%%%%%%%%%%%%%%%%%%%%%%%%%%%%%
\section{Information}

%%%%%%%%%%%%%%%%%%%%%%%%%%%%%%%%%%%%%%%%%%%%%%%%%%%%%%%%%%%%%%%%%%%%%%%%%%%%%%%%
\subsection{Copyright}

Copyright \copyright{} 2017--2018 Niklas Beisert

This work may be distributed and/or modified under the
conditions of the \LaTeX{} Project Public License, either version 1.3
of this license or (at your option) any later version.
The latest version of this license is in
  \url{http://www.latex-project.org/lppl.txt}
and version 1.3 or later is part of all distributions of \LaTeX{}
version 2005/12/01 or later.

This work has the LPPL maintenance status `maintained'.

The Current Maintainer of this work is Niklas Beisert.

This work consists of the files |README.txt|, |childdoc.ins| and |childdoc.dtx|
as well as the derived files |childdoc.def|, |cdocsamp.tex|
with |cdocsch1.tex|, |cdocsch2.tex|, |cdocspt3.tex|, |cdocspt4.tex|,
|cdocsdrf.tex|, |cdocsfn1.tex|, |cdocsfn2.tex|
as well as |childdoc.pdf|.

%%%%%%%%%%%%%%%%%%%%%%%%%%%%%%%%%%%%%%%%%%%%%%%%%%%%%%%%%%%%%%%%%%%%%%%%%%%%%%%%
\subsection{Files and Installation}

The package consists of the files:
%
\begin{center}
\begin{tabular}{ll}
    |README.txt|   & readme file \\
    |childdoc.ins| & installation file \\
    |childdoc.dtx| & source file \\
    |childdoc.def| & definition file \\
    |cdocsamp.tex| & sample main file \\
    |cdocsch1.tex| & sample include file \\
    |cdocsch2.tex| & sample include file \\
    |cdocspt3.tex| & sample part file \\
    |cdocspt4.tex| & sample part file \\
    |cdocsdrf.tex| & sample redirection file \\
    |cdocsfn1.tex| & sample redirection file \\
    |cdocsfn2.tex| & sample redirection file \\
    |childdoc.pdf| & manual
\end{tabular}
\end{center}
%
The distribution consists of the files
|README.txt|, |childdoc.ins| and |childdoc.dtx|.
%
\begin{itemize}
\item
Run (pdf)\LaTeX{} on |childdoc.dtx|
to compile the manual |childdoc.pdf| (this file).
\item
Run \LaTeX{} on |childdoc.ins| to create the definitions file |childdoc.def|
and the sample |cdocsamp.tex| with include files
|cdocsch1.tex|, |cdocsch2.tex|, |cdocspt3.tex|, |cdocspt4.tex|,
|cdocsdrf.tex|, |cdocsfn1.tex|, |cdocsfn2.tex|.
Then copy the file |childdoc.def| to an appropriate directory of your \LaTeX{}
distribution, e.g.\ \textit{texmf-root}|/tex/latex/childdoc|.
\end{itemize}

%%%%%%%%%%%%%%%%%%%%%%%%%%%%%%%%%%%%%%%%%%%%%%%%%%%%%%%%%%%%%%%%%%%%%%%%%%%%%%%%
\subsection{Related CTAN Packages}

There are several other packages which offer a similar functionality:
%
\begin{itemize}
\item
The packages
\href{http://ctan.org/pkg/docmute}{\textsf{docmute}},
\href{http://ctan.org/pkg/includex}{\textsf{includex}} and
\href{http://ctan.org/pkg/standalone}{\textsf{standalone}}
provide commands to include only the document body of
a child file thus allowing both files to be compiled individually.
\item
The packages \href{http://ctan.org/pkg/subdocs}{\textsf{subdocs}}
and \href{http://ctan.org/pkg/subfiles}{\textsf{subfiles}}
provide structures in which the main and child documents can be
encapsulated and allowing them to be compiled individually.
The inclusion mechanism is different from the conventional |\include|.
\item
The package \href{http://ctan.org/pkg/combine}{\textsf{combine}}
is an elaborate solution to combine several documents into one.
\end{itemize}
%
See also the CTAN topic \href{http://ctan.org/topic/subdocs}{\textsf{subdocs}}
for further related packages.
The present package differs from the above solutions in that
a document structure constructed with the conventional |\include| mechanism
just needs two extra commands at the top of every file
such that all constituent files can be compiled individually.

%%%%%%%%%%%%%%%%%%%%%%%%%%%%%%%%%%%%%%%%%%%%%%%%%%%%%%%%%%%%%%%%%%%%%%%%%%%%%%%%
%\subsection{Feature Suggestions}
%
%The following is a list of features which may be useful for future
%versions of this package:
%%
%\begin{itemize}
%\item
%\ldots
%\end{itemize}

%%%%%%%%%%%%%%%%%%%%%%%%%%%%%%%%%%%%%%%%%%%%%%%%%%%%%%%%%%%%%%%%%%%%%%%%%%%%%%%%
\subsection{Revision History}

%%%%%%%%%%%%%%%%%%%%%%%%%%%%%%%%%%%%%%%%
\paragraph{v2.0:} 2018/12/30

\begin{itemize}
\item
immediate forward processing
\item
added |\childdocby| mechanism
\item
manual restructured
\end{itemize}

%%%%%%%%%%%%%%%%%%%%%%%%%%%%%%%%%%%%%%%%
\paragraph{v1.6:} 2018/01/17

\begin{itemize}
\item
application for development of include files
\item
corrections to manual
\end{itemize}

%%%%%%%%%%%%%%%%%%%%%%%%%%%%%%%%%%%%%%%%
\paragraph{v1.5:} 2017/05/21

\begin{itemize}
\item
more complete structuring introduced
\item
|\childdocof| introduced
\item
|\childdoc| renamed to |\childdocmain|
\item
|\childredirect| renamed to |\childdocforward| and |\childdocforwardprefix|
and functionality expanded
\end{itemize}

%%%%%%%%%%%%%%%%%%%%%%%%%%%%%%%%%%%%%%%%
\paragraph{v1.0:} 2017/04/27

\begin{itemize}
\item
manual and install package
\item
first version published on CTAN
\end{itemize}

%%%%%%%%%%%%%%%%%%%%%%%%%%%%%%%%%%%%%%%%
\paragraph{v0.6:} 2017/04/26

\begin{itemize}
\item
redirection mechanism added
\end{itemize}

%%%%%%%%%%%%%%%%%%%%%%%%%%%%%%%%%%%%%%%%
\paragraph{v0.5:} 2017/04/26

\begin{itemize}
\item
functionality in definition file
\end{itemize}


%%%%%%%%%%%%%%%%%%%%%%%%%%%%%%%%%%%%%%%%%%%%%%%%%%%%%%%%%%%%%%%%%%%%%%%%%%%%%%%%
%%%%%%%%%%%%%%%%%%%%%%%%%%%%%%%%%%%%%%%%%%%%%%%%%%%%%%%%%%%%%%%%%%%%%%%%%%%%%%%%
%%%%%%%%%%%%%%%%%%%%%%%%%%%%%%%%%%%%%%%%%%%%%%%%%%%%%%%%%%%%%%%%%%%%%%%%%%%%%%%%
\appendix

\settowidth\MacroIndent{\rmfamily\scriptsize 000\ }

 \DocInput{childdoc.dtx}

\end{document}
%</driver>
% \fi
%
% %%%%%%%%%%%%%%%%%%%%%%%%%%%%%%%%%%%%%%%%%%%%%%%%%%%%%%%%%%%%%%%%%%%%%%%%%%%%%%
% %%%%%%%%%%%%%%%%%%%%%%%%%%%%%%%%%%%%%%%%%%%%%%%%%%%%%%%%%%%%%%%%%%%%%%%%%%%%%%
% \section{Sample}
%\iffalse
%<*samplemain>
%\fi
%
% The following presents a sample document
% with two chapters, two parts, a title page,
% a compile flag as well as three forwarding files to set the flag.
% It consists of eight |.tex| files:
% \begin{center}
% \begin{tabular}{ll}
% |cdocsamp.tex|&main file\\
% |cdocsch1.tex|&include file for chapter 1\\
% |cdocsch2.tex|&include file for chapter 2\\
% |cdocspt3.tex|&include file for part 3\\
% |cdocspt4.tex|&include file for part 4\\
% |cdocsdrf.tex|&forwarding file for main file in draft mode\\
% |cdocsfi1.tex|&forwarding file for final version of chapter 1\\
% |cdocsfi2.tex|&forwarding file for final version of chapter 2\\
% \end{tabular}
% \end{center}
% Each of the eight files can be compiled directly by the \LaTeX{} compiler.
%
% %%%%%%%%%%%%%%%%%%%%%%%%%%%%%%%%%%%%%%
% \paragraph{Main File.}
%
% The main file is called |cdocsamp.tex|.
%
% Load the \textsf{childdoc} definitions and
% declare the filename for the main document:
%    \begin{macrocode}
\input{childdoc.def}
\childdocmain{}
%    \end{macrocode}

% Optional override for |\version| flag:
%    \begin{macrocode}
%%\ifchilddoc\else\providecommand{\version}{draft}\fi
%    \end{macrocode}

% Define the default values for the |\version| flag
% (|final| for the main file and |draft| for childs):
%    \begin{macrocode}
\ifchilddoc
\providecommand{\version}{draft}
\else
\providecommand{\version}{final}
\fi
%    \end{macrocode}

% Load the standard document class:
%    \begin{macrocode}
\documentclass[12pt]{article}
%    \end{macrocode}

% Start the document body:
%    \begin{macrocode}
\begin{document}
%    \end{macrocode}

% Declare a title page.
% Print title, part of document being processed and version flag:
%    \begin{macrocode}
\addtocounter{page}{-1}
\begin{center}
{\LARGE\bfseries{}childdoc example\par}
\vspace{1cm}
\ifchilddoc
\ifchilddocmanual part\else chapter\fi:
`\childdocname' of `\childdocjob'\par
\else
main document: `\childdocjob'\par
\fi
version: \version\par
\end{center}
\newpage
%    \end{macrocode}

% Manually include selected file,
% otherwise process as usual:
%    \begin{macrocode}
\ifchilddocmanual
\section*{part `\childdocname'}
\input{\childdocname}
\else
%    \end{macrocode}

% Include the two chapters:
%    \begin{macrocode}
\include{cdocsch1}
\include{cdocsch2}
%    \end{macrocode}

% Include the two parts unless only chapters should be displayed:
%    \begin{macrocode}
\ifchilddoc\else
\section{part three}
\input{cdocspt3}
\section{part four}
\input{cdocspt4}
\fi
%    \end{macrocode}

% Process as usual until here:
%    \begin{macrocode}
\fi
%    \end{macrocode}

% End of document body:
%    \begin{macrocode}
\end{document}
%    \end{macrocode}
%\iffalse
%</samplemain>
%\fi
%
% %%%%%%%%%%%%%%%%%%%%%%%%%%%%%%%%%%%%%%
% \paragraph{Chapter Include Files.}
%
% The include files are called |cdocsch1.tex| and |cdocsch2.tex|.
%
%\iffalse
%<*samplechap1|samplechap2>
%\fi

% Optional override for |\version| flag:
%    \begin{macrocode}
%%\providecommand{\version}{final}
%    \end{macrocode}

% Include the main document:
%    \begin{macrocode}
\input{childdoc.def}
\childdocof{cdocsamp}
%    \end{macrocode}

%\iffalse
%</samplechap1|samplechap2>
%\fi
%
%\iffalse
%<*samplechap1>
%\fi
% Some text for chapter 1:
%    \begin{macrocode}
\section{one}
some text in chapter one
%    \end{macrocode}

%\iffalse
%</samplechap1>
%\fi
% Some text for chapter 2:
%\iffalse
%<*samplechap2>
%\fi
%    \begin{macrocode}
\section{two}
more text in chapter two
%    \end{macrocode}

%\iffalse
%</samplechap2>
%\fi
%
% %%%%%%%%%%%%%%%%%%%%%%%%%%%%%%%%%%%%%%
% \paragraph{Part Include Files.}
%
% The include files are called |cdocspt3.tex| and |cdocspt4.tex|.
%
%\iffalse
%<*samplepart3|samplepart4>
%\fi

% Optional override for |\version| flag:
%    \begin{macrocode}
%%\providecommand{\version}{final}
%    \end{macrocode}

% Include the main document:
%    \begin{macrocode}
\input{childdoc.def}
\childdocby{cdocsamp}
%    \end{macrocode}

%\iffalse
%</samplepart3|samplepart4>
%\fi
%
%\iffalse
%<*samplepart3>
%\fi
% Some text for part 3:
%    \begin{macrocode}
some text in part three
%    \end{macrocode}

%\iffalse
%</samplepart3>
%\fi
% Some text for part 4:
%\iffalse
%<*samplepart4>
%\fi
%    \begin{macrocode}
more text in part four
%    \end{macrocode}

%\iffalse
%</samplepart4>
%\fi
%
% %%%%%%%%%%%%%%%%%%%%%%%%%%%%%%%%%%%%%%
% \paragraph{Forwarding for a Complete Draft.}
%
% The following forwarding file |cdocsdrf.tex|
% compiles the main document in draft mode:
%\iffalse
%<*sampledraft>
%\fi
%    \begin{macrocode}
\def\version{draft}
\input{childdoc.def}
\childdocforward{cdocsamp}
%    \end{macrocode}

%\iffalse
%</sampledraft>
%\fi
%
% %%%%%%%%%%%%%%%%%%%%%%%%%%%%%%%%%%%%%%
% \paragraph{Forwarding for Final Version of the Chapters.}
%
% The following forwarding files |cdocsfn1.tex| and |cdocsfn2.tex|
% (with identical content)
% compile the final versions of the child documents
% |cdocsch1.tex| and |cdocsch2.tex|, respectively:
%\iffalse
%<*samplefinal>
%\fi
%    \begin{macrocode}
\def\version{final}
\input{childdoc.def}
\childdocforwardprefix[cdocsamp]{cdocsfn}{cdocsch}
%    \end{macrocode}

%\iffalse
%</samplefinal>
%\fi
%
% %%%%%%%%%%%%%%%%%%%%%%%%%%%%%%%%%%%%%%
% \paragraph{Command Line Processing.}
%
% The following three command lines generate the output files
% |cdocscld|, |cdocscl1| and |cdocscl2|
% which should be identical to
% |cdocsdrf|, |cdocsch1| and |cdocsfn2|, respectively:
% \begin{center}
% \begin{tabular}{l}
% |latex -jobname cdocscld \|\\
% |  "\def\version{draft}\input{childdoc.def}\childdocforward{cdocsamp}"|\\
% |latex -jobname cdocscl1 \|\\
% |  "\input{childdoc.def}\childdocforward[cdocsamp]{cdocsch1}"|\\
% |latex -jobname cdocscl2 \|\\
% |  "\def\version{final}\input{childdoc.def}\childdocforward{cdocsch2}"|
% \end{tabular}
% \end{center}
% Note that the trailing backslash on each first line
% merely continues the input to the second line
% (for convenient cut ant paste).
% Furthermore, the command |latex| can be replaced by any
% of its alternative versions such as |pdflatex|.
%
% %%%%%%%%%%%%%%%%%%%%%%%%%%%%%%%%%%%%%%%%%%%%%%%%%%%%%%%%%%%%%%%%%%%%%%%%%%%%%%
% %%%%%%%%%%%%%%%%%%%%%%%%%%%%%%%%%%%%%%%%%%%%%%%%%%%%%%%%%%%%%%%%%%%%%%%%%%%%%%
% \section{Implementation}
%\iffalse
%<*package>
%\fi
%
% This section describes the definitions file |childdoc.def|.

% The definitions cannot be loaded using |\usepackage| or |\RequirePackage|
% which has a mechanism to prevent loading a style file more than once.
% When loading the definitions by means of |\input|
% multiple instances have to be prevented manually:
%\iffalse
%This code needs to be before the `\ProvidesFile' directive
%which is defined at the beginning of this file.
%Therefore it is also placed there and commented out here.
%</package>
%<*discard>
%\fi
%    \begin{macrocode}
\ifdefined\childdocmain\endinput\fi
%    \end{macrocode}
%\iffalse
%</discard>
%<*package>
%\fi
%
% \macro{\ifchilddoc}
% \macro{\ifchilddocmanual}
% The conditional |\ifchilddoc| tells whether a
% child (true) or main (false) document is being compiled.
% The conditional |\ifchilddocmanual| tells whether
% the |\includeonly| mechanism is used (false) or
% the selection of child files must be performed manually (true).
% The definitions initialise to false:
%    \begin{macrocode}
\newif\ifchilddoc
\newif\ifchilddocmanual
%    \end{macrocode}

% \macro{\childdocname}
% \macro{\childdocjob}
% The macro |\childdocname| stores the name of the main document
% to be compiled. The macro |\childdocjob| stores the name of
% the document on which the \LaTeX{} compiler was originally invoked.
% The content of |\jobname| cannot be compared
% to filenames specified in the source due to different catcodes.
% The following code rescans |\jobname|, stores the result
% in |\childdocname| and saves a copy in |\childdocjob|:
%    \begin{macrocode}
\edef\childdocname{\scantokens\expandafter{\jobname\noexpand}}
\let\childdocjob\childdocname
%    \end{macrocode}

% \macro{\childdocdisable}
% The macro |\childdocdisable| prevents the main file
% from being processed more than once.
% At this stage, the main document command |\childdocmain|
% is assumed to be called once again where it should do nothing.
% Any subsequent call to it should prevent
% a secondary processing of the main document
% It overwrites the forwarding commands
% |\childdocof| and |\childdocforward|
% with empty macros to prevent further inclusions of the main document:
%    \begin{macrocode}
\newcommand{\childdocdisable}
{
  \renewcommand{\childdocmain}[1]{\renewcommand{\childdocmain}[1]{\endinput}}
  \renewcommand{\childdocof}[1]{}
  \renewcommand{\childdocby}[2][]{}
  \renewcommand{\childdocforward}[2][]{}
  \renewcommand{\childdocdisable}{}
}
%    \end{macrocode}

% \macro{\childdocmain}
% The macro |\childdocmain| is to be called at the top of the main file
% with nothing or the main filename (without extension) as argument.
% First, it breaks loops.
% If the argument is not empty and does not match |\childdocname|
% (which is set by the first inclusion of |childdoc.def|),
% |\ifchilddoc| is set to true, |\includeonly| is applied to the child file
% and |\jobname| is set to the main file
% (for proper handling of |.aux| files):
%    \begin{macrocode}
\newcommand{\childdocmain}[1]
{
  \childdocdisable\childdocmain{}
  \if?#1?\else
    \begingroup
      \def\childdoctmp{#1}
      \ifx\childdoctmp\childdocname
        \def\childdoctmp{}
      \else
        \def\childdoctmp
        {
          \childdoctrue
          \includeonly{\childdocname}
          \def\childdocjob{#1}
          \def\jobname{#1}
        }
      \fi
      \expandafter
    \endgroup
    \childdoctmp
  \fi
}
%    \end{macrocode}

% \macro{\childdocof}
% The command |\childdocof| redirects
% compilation to the main file |#1|.
%    \begin{macrocode}
\newcommand{\childdocof}[1]
{
  \childdocdisable
  \childdoctrue
  \includeonly{\childdocname}
  \def\jobname{#1}
  \def\childdocjob{#1}
  \input{#1}
}
%    \end{macrocode}

% \macro{\childdocby}
% The command |\childdocby| ....
%    \begin{macrocode}
\newcommand{\childdocby}[2][]
{
  \childdocdisable
  \childdoctrue
  \childdocmanualtrue
  \if?#1?\else
    \def\jobname{#2}
  \fi
  \def\childdocjob{#2}
  \input{#2}
  \endinput
}
%    \end{macrocode}

% \macro{\childdocforward}
% The command |\childdocforward| redirects
% compilation to the main file or
% (if the optional argument is given) a child file.
% Parameters are set as if the main file
% or a child file starting with |\childdocof| was compiled.
% Then compilation is handed over to the main file:
%    \begin{macrocode}
\newcommand{\childdocforward}[2][]
{
  \begingroup
    \if?#1?
      \def\childdoctmp
      {
        \def\childdocname{#2}
        \def\childdocjob{#2}
        \def\jobname{#2}
        \input{#2}
        \endinput
      }
    \else
      \def\childdoctmp
      {
        \childdocdisable
        \def\childdocname{#2}
        \childdoctrue
        \includeonly{#2}
        \def\childdocjob{#1}
        \def\jobname{#1}
        \input{#1}
        \endinput
      }
    \fi
    \expandafter
  \endgroup
  \childdoctmp
}
%    \end{macrocode}

% \macro{\childdocforwardprefix}
% The command |\childdocforwardprefix| redirects
% compilation to the main or a child file by means of a pattern.
% The prefix |#1| in the current filename is replaced by |#2|
% and the suffix of the current filename is kept
% (it is assumed that the filename does not contain the substring `|~~~|'
% which is used as a delimiter).
% Compilation is handed over to the new file by |\childdocforward|:
%    \begin{macrocode}
\newcommand{\childdocforwardprefix}[3][]
{
  \begingroup
    \def\childdocextract #2##1~~~{\def\childdoctmp{\childdocforward[#1]{#3##1}}}
    \expandafter\childdocextract\childdocname~~~
    \expandafter
  \endgroup
  \childdoctmp
}
%    \end{macrocode}

% \macro{\childdoc}
% The deprecated macro |\childdoc| is a legacy version of |\childdocmain|:
%    \begin{macrocode}
\newcommand{\childdoc}{\childdocmain}
%    \end{macrocode}

% \macro{\childdocredirect}
% The deprecated macro |\childdocredirect| is a legacy version
% of |\childdocforward| and |\childdocforwardprefix|:
%    \begin{macrocode}
\newcommand{\childdocredirect}[2][]
{
  \begingroup
    \if?#1?
      \def\childdoctmp{\childdocforward{#2}}
    \else
      \def\childdoctmp{\childdocforwardprefix{#1}{#2}}
    \fi
    \expandafter
  \endgroup
  \childdoctmp
}
%    \end{macrocode}

%\iffalse
%</package>
%\fi
%
\endinput
|
and perform the replacements as outlined below.
Instead of |\childdocmain{|\textit{main}|}| add the following code
to the top of the main file:
%
\begin{center}
\begin{tabular}{l}
|\||ifdefined\childdocname\endinput\||fi\newif\ifchilddoc|\\
|\edef\childdocname{\scantokens\expandafter{\jobname\noexpand}}|\\
|\def\childdocmain{|\textit{main}|}\||ifx\childdocmain\childdocname\||else|\\
|\childdoctrue\includeonly{\childdocname}\let\jobname\childdocmain\||fi|\\
\end{tabular}
\end{center}
%
Instead of |\childdocof{|\textit{main}|}| just include the main file
at the top of each child file:
%
\begin{center}
|\input{|\textit{main}|}|
\end{center}
%
A simple redirection |\childdocforward{|\textit{dest}|}| is achieved by:
%
\begin{center}
|\def\jobname{|\textit{dest}|}\input{\jobname}|
\end{center}
%
The redirection with prefix
|\childdocforwardprefix[|\textit{prefix}|]{|\textit{dest}|}|
is accomplished by:
%
\begin{center}
\begin{tabular}{l}
|{\edef\jobname{\scantokens\expandafter{\jobname\noexpand}}|\\
|\def\redirectjob |\textit{prefix}|#1~~~{\gdef\jobname{|\textit{dest}|#1}}|\\
|\expandafter\redirectjob\jobname~~~}\input{\jobname}|
\end{tabular}
\end{center}

In an alternative approach,
child documents can be compiled by a specific command line
without additional code or specific definitions:
%
\begin{center}
|... -jobname "|\textit{target}|" "|[\textit{flags}]%
|\includeonly{|\textit{dest}|}\input{|\textit{main}|}"|
\end{center}
%

%%%%%%%%%%%%%%%%%%%%%%%%%%%%%%%%%%%%%%%%%%%%%%%%%%%%%%%%%%%%%%%%%%%%%%%%%%%%%%%%
%%%%%%%%%%%%%%%%%%%%%%%%%%%%%%%%%%%%%%%%%%%%%%%%%%%%%%%%%%%%%%%%%%%%%%%%%%%%%%%%
\section{Information}

%%%%%%%%%%%%%%%%%%%%%%%%%%%%%%%%%%%%%%%%%%%%%%%%%%%%%%%%%%%%%%%%%%%%%%%%%%%%%%%%
\subsection{Copyright}

Copyright \copyright{} 2017--2018 Niklas Beisert

This work may be distributed and/or modified under the
conditions of the \LaTeX{} Project Public License, either version 1.3
of this license or (at your option) any later version.
The latest version of this license is in
  \url{http://www.latex-project.org/lppl.txt}
and version 1.3 or later is part of all distributions of \LaTeX{}
version 2005/12/01 or later.

This work has the LPPL maintenance status `maintained'.

The Current Maintainer of this work is Niklas Beisert.

This work consists of the files |README.txt|, |childdoc.ins| and |childdoc.dtx|
as well as the derived files |childdoc.def|, |cdocsamp.tex|
with |cdocsch1.tex|, |cdocsch2.tex|, |cdocspt3.tex|, |cdocspt4.tex|,
|cdocsdrf.tex|, |cdocsfn1.tex|, |cdocsfn2.tex|
as well as |childdoc.pdf|.

%%%%%%%%%%%%%%%%%%%%%%%%%%%%%%%%%%%%%%%%%%%%%%%%%%%%%%%%%%%%%%%%%%%%%%%%%%%%%%%%
\subsection{Files and Installation}

The package consists of the files:
%
\begin{center}
\begin{tabular}{ll}
    |README.txt|   & readme file \\
    |childdoc.ins| & installation file \\
    |childdoc.dtx| & source file \\
    |childdoc.def| & definition file \\
    |cdocsamp.tex| & sample main file \\
    |cdocsch1.tex| & sample include file \\
    |cdocsch2.tex| & sample include file \\
    |cdocspt3.tex| & sample part file \\
    |cdocspt4.tex| & sample part file \\
    |cdocsdrf.tex| & sample redirection file \\
    |cdocsfn1.tex| & sample redirection file \\
    |cdocsfn2.tex| & sample redirection file \\
    |childdoc.pdf| & manual
\end{tabular}
\end{center}
%
The distribution consists of the files
|README.txt|, |childdoc.ins| and |childdoc.dtx|.
%
\begin{itemize}
\item
Run (pdf)\LaTeX{} on |childdoc.dtx|
to compile the manual |childdoc.pdf| (this file).
\item
Run \LaTeX{} on |childdoc.ins| to create the definitions file |childdoc.def|
and the sample |cdocsamp.tex| with include files
|cdocsch1.tex|, |cdocsch2.tex|, |cdocspt3.tex|, |cdocspt4.tex|,
|cdocsdrf.tex|, |cdocsfn1.tex|, |cdocsfn2.tex|.
Then copy the file |childdoc.def| to an appropriate directory of your \LaTeX{}
distribution, e.g.\ \textit{texmf-root}|/tex/latex/childdoc|.
\end{itemize}

%%%%%%%%%%%%%%%%%%%%%%%%%%%%%%%%%%%%%%%%%%%%%%%%%%%%%%%%%%%%%%%%%%%%%%%%%%%%%%%%
\subsection{Related CTAN Packages}

There are several other packages which offer a similar functionality:
%
\begin{itemize}
\item
The packages
\href{http://ctan.org/pkg/docmute}{\textsf{docmute}},
\href{http://ctan.org/pkg/includex}{\textsf{includex}} and
\href{http://ctan.org/pkg/standalone}{\textsf{standalone}}
provide commands to include only the document body of
a child file thus allowing both files to be compiled individually.
\item
The packages \href{http://ctan.org/pkg/subdocs}{\textsf{subdocs}}
and \href{http://ctan.org/pkg/subfiles}{\textsf{subfiles}}
provide structures in which the main and child documents can be
encapsulated and allowing them to be compiled individually.
The inclusion mechanism is different from the conventional |\include|.
\item
The package \href{http://ctan.org/pkg/combine}{\textsf{combine}}
is an elaborate solution to combine several documents into one.
\end{itemize}
%
See also the CTAN topic \href{http://ctan.org/topic/subdocs}{\textsf{subdocs}}
for further related packages.
The present package differs from the above solutions in that
a document structure constructed with the conventional |\include| mechanism
just needs two extra commands at the top of every file
such that all constituent files can be compiled individually.

%%%%%%%%%%%%%%%%%%%%%%%%%%%%%%%%%%%%%%%%%%%%%%%%%%%%%%%%%%%%%%%%%%%%%%%%%%%%%%%%
%\subsection{Feature Suggestions}
%
%The following is a list of features which may be useful for future
%versions of this package:
%%
%\begin{itemize}
%\item
%\ldots
%\end{itemize}

%%%%%%%%%%%%%%%%%%%%%%%%%%%%%%%%%%%%%%%%%%%%%%%%%%%%%%%%%%%%%%%%%%%%%%%%%%%%%%%%
\subsection{Revision History}

%%%%%%%%%%%%%%%%%%%%%%%%%%%%%%%%%%%%%%%%
\paragraph{v2.0:} 2018/12/30

\begin{itemize}
\item
immediate forward processing
\item
added |\childdocby| mechanism
\item
manual restructured
\end{itemize}

%%%%%%%%%%%%%%%%%%%%%%%%%%%%%%%%%%%%%%%%
\paragraph{v1.6:} 2018/01/17

\begin{itemize}
\item
application for development of include files
\item
corrections to manual
\end{itemize}

%%%%%%%%%%%%%%%%%%%%%%%%%%%%%%%%%%%%%%%%
\paragraph{v1.5:} 2017/05/21

\begin{itemize}
\item
more complete structuring introduced
\item
|\childdocof| introduced
\item
|\childdoc| renamed to |\childdocmain|
\item
|\childredirect| renamed to |\childdocforward| and |\childdocforwardprefix|
and functionality expanded
\end{itemize}

%%%%%%%%%%%%%%%%%%%%%%%%%%%%%%%%%%%%%%%%
\paragraph{v1.0:} 2017/04/27

\begin{itemize}
\item
manual and install package
\item
first version published on CTAN
\end{itemize}

%%%%%%%%%%%%%%%%%%%%%%%%%%%%%%%%%%%%%%%%
\paragraph{v0.6:} 2017/04/26

\begin{itemize}
\item
redirection mechanism added
\end{itemize}

%%%%%%%%%%%%%%%%%%%%%%%%%%%%%%%%%%%%%%%%
\paragraph{v0.5:} 2017/04/26

\begin{itemize}
\item
functionality in definition file
\end{itemize}


%%%%%%%%%%%%%%%%%%%%%%%%%%%%%%%%%%%%%%%%%%%%%%%%%%%%%%%%%%%%%%%%%%%%%%%%%%%%%%%%
%%%%%%%%%%%%%%%%%%%%%%%%%%%%%%%%%%%%%%%%%%%%%%%%%%%%%%%%%%%%%%%%%%%%%%%%%%%%%%%%
%%%%%%%%%%%%%%%%%%%%%%%%%%%%%%%%%%%%%%%%%%%%%%%%%%%%%%%%%%%%%%%%%%%%%%%%%%%%%%%%
\appendix

\settowidth\MacroIndent{\rmfamily\scriptsize 000\ }

 \DocInput{childdoc.dtx}

\end{document}
%</driver>
% \fi
%
% %%%%%%%%%%%%%%%%%%%%%%%%%%%%%%%%%%%%%%%%%%%%%%%%%%%%%%%%%%%%%%%%%%%%%%%%%%%%%%
% %%%%%%%%%%%%%%%%%%%%%%%%%%%%%%%%%%%%%%%%%%%%%%%%%%%%%%%%%%%%%%%%%%%%%%%%%%%%%%
% \section{Sample}
%\iffalse
%<*samplemain>
%\fi
%
% The following presents a sample document
% with two chapters, two parts, a title page,
% a compile flag as well as three forwarding files to set the flag.
% It consists of eight |.tex| files:
% \begin{center}
% \begin{tabular}{ll}
% |cdocsamp.tex|&main file\\
% |cdocsch1.tex|&include file for chapter 1\\
% |cdocsch2.tex|&include file for chapter 2\\
% |cdocspt3.tex|&include file for part 3\\
% |cdocspt4.tex|&include file for part 4\\
% |cdocsdrf.tex|&forwarding file for main file in draft mode\\
% |cdocsfi1.tex|&forwarding file for final version of chapter 1\\
% |cdocsfi2.tex|&forwarding file for final version of chapter 2\\
% \end{tabular}
% \end{center}
% Each of the eight files can be compiled directly by the \LaTeX{} compiler.
%
% %%%%%%%%%%%%%%%%%%%%%%%%%%%%%%%%%%%%%%
% \paragraph{Main File.}
%
% The main file is called |cdocsamp.tex|.
%
% Load the \textsf{childdoc} definitions and
% declare the filename for the main document:
%    \begin{macrocode}
% \iffalse
%
% childdoc.dtx Copyright (C) 2017-2018 Niklas Beisert
%
% This work may be distributed and/or modified under the
% conditions of the LaTeX Project Public License, either version 1.3
% of this license or (at your option) any later version.
% The latest version of this license is in
%   http://www.latex-project.org/lppl.txt
% and version 1.3 or later is part of all distributions of LaTeX
% version 2005/12/01 or later.
%
% This work has the LPPL maintenance status `maintained'.
%
% The Current Maintainer of this work is Niklas Beisert.
%
% This work consists of the files childdoc.dtx and childdoc.ins
% and the derived files childdoc.def and cdocsamp.tex with
% cdocsch1.tex, cdocsch2.tex, cdocsdrf.tex, cdocsfn1.tex, cdocsfn2.tex.
%
%<package>\ifdefined\childdocmain\endinput\fi
%<package>\ProvidesFile{childdoc.def}[2018/12/30 v2.0 child document driver]
%<samplemain>\ProvidesFile{cdocsamp.tex}[2018/12/30 v2.0 sample for childdoc]
%<*driver>
%\ProvidesFile{childdoc.drv}[2018/12/30 v2.0 childdoc reference manual file]
\PassOptionsToClass{10pt,a4paper}{article}
\documentclass{ltxdoc}

\usepackage[margin=35mm]{geometry}
\usepackage{hyperref}
\usepackage{hyperxmp}
\usepackage[usenames]{color}

\hypersetup{colorlinks=true}
\hypersetup{pdfstartview=FitH}
\hypersetup{pdfpagemode=UseNone}
\hypersetup{pdfsource={}}
\hypersetup{pdflang={en-UK}}
\hypersetup{pdfcopyright={Copyright 2017-2018 Niklas Beisert.
  This work may be distributed and/or modified under the
  conditions of the LaTeX Project Public License, either version 1.3
  of this license or (at your option) any later version.}}
\hypersetup{pdflicenseurl={http://www.latex-project.org/lppl.txt}}
\hypersetup{pdfcontactaddress={ETH Zurich, ITP, HIT K,
  Wolfgang-Pauli-Strasse 27}}
\hypersetup{pdfcontactpostcode={8093}}
\hypersetup{pdfcontactcity={Zurich}}
\hypersetup{pdfcontactcountry={Switzerland}}
\hypersetup{pdfcontactemail={nbeisert@itp.phys.ethz.ch}}
\hypersetup{pdfcontacturl={http://people.phys.ethz.ch/\xmptilde nbeisert/}}

\newcommand{\secref}[1]{\hyperref[#1]{section \ref*{#1}}}

\parskip1ex
\parindent0pt
\let\olditemize\itemize
\def\itemize{\olditemize\parskip0pt}

\begin{document}

\title{The \textsf{childdoc} Package}
\hypersetup{pdftitle={The childdoc Package}}
\author{Niklas Beisert\\[2ex]
  Institut f\"ur Theoretische Physik\\
  Eidgen\"ossische Technische Hochschule Z\"urich\\
  Wolfgang-Pauli-Strasse 27, 8093 Z\"urich, Switzerland\\[1ex]
  \href{mailto:nbeisert@itp.phys.ethz.ch}
  {\texttt{nbeisert@itp.phys.ethz.ch}}}
\hypersetup{pdfauthor={Niklas Beisert}}
\hypersetup{pdfsubject={Manual for the LaTeX2e Package childdoc}}
\date{30 December 2018, \textsf{v2.0}}
\maketitle

\begin{abstract}\noindent
\textsf{childdoc} is a \LaTeXe{} package
that enables the direct compilation
of document sections included by |\include|
to individual files.
\end{abstract}

\begingroup
\parskip0ex
\tableofcontents
\endgroup

%%%%%%%%%%%%%%%%%%%%%%%%%%%%%%%%%%%%%%%%%%%%%%%%%%%%%%%%%%%%%%%%%%%%%%%%%%%%%%%%
%%%%%%%%%%%%%%%%%%%%%%%%%%%%%%%%%%%%%%%%%%%%%%%%%%%%%%%%%%%%%%%%%%%%%%%%%%%%%%%%
\section{Introduction}

\LaTeX{} provides a mechanism to structure a large document (such as a book)
into a main file and several child files (containing the chapters)
using the |\include| command.
This mechanism is beneficial for documents
which span hundreds of pages in order to
make the source file(s) more manageable.
Moreover, compilation can be restricted to
selected child files by means of the |\includeonly| command.
The latter feature can be used to reduce the compilation time while editing
(this was significantly more useful in the earlier days of \LaTeX{})
or to generate a smaller document which is easier to navigate.
Another application of |\includeonly| is to generate
documents consisting of selected parts of the complete document.

However, there are a few drawbacks of the plain |\include| mechanism:
\begin{itemize}
\item
The child files cannot be compiled on their own,
they can only be compiled via the main file.
A naive editing environment
(such as a text editor with an option
to have the current file processed by \LaTeX)
may require one to switch to the main file before compiling;
attempting to compile the child file produces errors.
\item
The main file must be modified (each time)
to adjust the |\includeonly| command
to the present needs. This easily leaves the main file in a messy state.
\item
The generated document will always carry the filename
of the main document. This is inconvenient if
several child files are to be compiled and
to be kept for distribution.
\end{itemize}

The present package provides a simple interface
to make child files individually compilable by \LaTeX{}.
Compiling a child file then has the same effect as compiling
the main file with an |\includeonly| command
to select the appropriate child.
Moreover the generated document will carry the name of the child
rather than the main file.
This resolves all three above issues.

This feature is meant to make the editing of books,
thesis documents and lecture notes somewhat more convenient.
However, the package can also be used efficiently for
composing a series of documents (such as exercise sheets)
which are typically distributed individually.
It then assists the author in generating the individual documents
(potentially in different versions)
as well as a document containing the collected series.
Another application is in developing style files
or other kinds of included material
where compilation of the style file could redirect
to a sample or test file.

%%%%%%%%%%%%%%%%%%%%%%%%%%%%%%%%%%%%%%%%%%%%%%%%%%%%%%%%%%%%%%%%%%%%%%%%%%%%%%%%
%%%%%%%%%%%%%%%%%%%%%%%%%%%%%%%%%%%%%%%%%%%%%%%%%%%%%%%%%%%%%%%%%%%%%%%%%%%%%%%%
\section{Usage}

First of all, the package \textsf{childdoc} is \emph{not} a standard
\LaTeXe{} |.sty| style file! Therefore it needs to be invoked in
a non-standard way.

%%%%%%%%%%%%%%%%%%%%%%%%%%%%%%%%%%%%%%%%%%%%%%%%%%%%%%%%%%%%%%%%%%%%%%%%%%%%%%%%
\subsection{Included Files}
\label{sec:include}

%%%%%%%%%%%%%%%%%%%%%%%%%%%%%%%%%%%%%%%%
\DescribeMacro{\childdocmain}
To use the package, add the commands
\begin{center}
\begin{tabular}{l}
|\input{childdoc.def}|\\
|\childdocmain{}|\\
\end{tabular}
\end{center}
at the very top of the main \LaTeX{} file,
in particular \emph{before} the |\documentclass| statement!
The argument of |\childdocmain| should be left empty
(but it must be present).

%%%%%%%%%%%%%%%%%%%%%%%%%%%%%%%%%%%%%%%%
\DescribeMacro{\childdocof}
Furthermore, add the commands
\begin{center}
\begin{tabular}{l}
|\input{childdoc.def}|\\
|\childdocof{|\textit{main}|}|\\
\end{tabular}
\end{center}
at the top of every child file \textit{child}
which is included by |\include{|\textit{child}|}|
from within the main file
(or at least for those files to be compiled individually).
The argument \textit{main} must be the filename of the main file.

There are a couple of
considerations in setting up the main and child documents:

%%%%%%%%%%%%%%%%%%%%%%%%%%%%%%%%%%%%%%%%
\paragraph{Restrictions.}

Please note the following restrictions:
\begin{itemize}
\item
|\childdocmain| must be called with one argument \textit{main}
to ensure compatibility with earlier version of the package.
It must either be empty (|\childdocmain{}|)
or precisely match the filename of the main file in which it is specified.
See \secref{sec:detection} for further information.
\item
The filename \textit{main} must be specified without the |.tex| extension.
\item
The filename \textit{main} is case sensitive
(even in case-insensitive file systems)
due to internal string comparison.
\item
The argument \textit{main} should be fully expanded, it cannot be a macro.
\item
Subdirectories and special characters should be avoided in filenames.
\item
The command |\childdocmain{|\textit{main}|}| must be followed by a whitespace.
It should not be followed immediately by another command
or by a comment mark `|%|'.
This is because the \TeX{} parser reads the token immediately following
the argument of |\childdocmain| and puts it
at the beginning of every child section;
however, a white\-space is ignored.
\end{itemize}

%%%%%%%%%%%%%%%%%%%%%%%%%%%%%%%%%%%%%%%%
\paragraph{Content of Main File.}

It is advisable to place all content in the child files included by |\include|.
Any output contained in the main file will appear in all child documents
unless suppressed manually;
it cannot be suppressed automatically by the |\includeonly| directive
and thus should normally be avoided.
A method to include some content in the main file
by means of conditional processing is described in \secref{sec:conditional}.

%%%%%%%%%%%%%%%%%%%%%%%%%%%%%%%%%%%%%%%%
\paragraph{Page Numbering.}

When only a part of the document is compiled,
the appropriate numbering of pages
(as well as other status parameters)
is determined from the |.aux| files.
The latter contain information from previous passes.
However this information needs to propagate through
all intermediate child documents.
Therefore the page numbering in child documents may well
be inconsistent until the complete document is compiled at least once.

A useful (if unconventional) way to always ensure a consistent
page numbering is to restart the numbering in each child document
and denote the pages by `\textit{child}|.|\textit{page}'
where \textit{child} represents the chapter/section number of the child file.
This can be achieved by the command
|\numberwithin{page}{|\textit{child}|}|
of the \textsf{amsmath} package
where \textit{child} can be |chapter| or |section|
depending on the chosen structuring.
Alternatively, one can modify the macro |\thepage| appropriately
and reset the counter |page| at the start of each child file.

%%%%%%%%%%%%%%%%%%%%%%%%%%%%%%%%%%%%%%%%%%%%%%%%%%%%%%%%%%%%%%%%%%%%%%%%%%%%%%%%
\subsection{Conditional Processing}
\label{sec:conditional}

The package provides a mechanism to compile different versions
of a document. To customise the versions further some conditional processing
can come in handy to distinguish which version is being compiled.
The package provides two macros to describe the compilation context:

%%%%%%%%%%%%%%%%%%%%%%%%%%%%%%%%%%%%%%%%
\DescribeMacro{\ifchilddoc}
The conditional |\ifchilddoc| distinguishes between the compilation of
child documents and the main document:
%
\begin{center}
|\ifchilddoc |\textit{child-code}| |[|\||else |\textit{main-code}]| \||fi|
\end{center}

%%%%%%%%%%%%%%%%%%%%%%%%%%%%%%%%%%%%%%%%
\DescribeMacro{\childdocname}
\DescribeMacro{\childdocjob}
The macro |\childdocname| contains the filename (without extension)
of the main or child file being processed.
Note that |\childdocjob| will always contain the name of the main file.

%%%%%%%%%%%%%%%%%%%%%%%%%%%%%%%%%%%%%%%%
\paragraph{Title Page.}

Conditional processing can be used to include a title or banner page
in the main document when proper precautions are taken.
Importantly, the code in the main file should ensure that the page counter
(as well as other status parameters which are stored in the |.aux| files)
takes the same value after the conditional processing.
Otherwise the page numbers may take divergent values
depending on which part is compiled.

For example, a title page could be declared by:
%
\begin{center}
\begin{tabular}{l}
|\ifchilddoc\||else|\\
|\addtocounter{page}{-1}|\\
\textit{code for title page}\\
|\newpage|\\
|\||fi|
\end{tabular}
\end{center}
%
A banner page for the child documents can be generated by:
%
\begin{center}
\begin{tabular}{l}
|\ifchilddoc|\\
|\addtocounter{page}{-1}|\\
\textit{code for banner page}\\
|\newpage|\\
|\||fi|
\end{tabular}
\end{center}
%
Here one could write a message such as:
\begin{center}
|This is the part \childdocname{} of \childdocjob{}.|
\end{center}

%%%%%%%%%%%%%%%%%%%%%%%%%%%%%%%%%%%%%%%%%%%%%%%%%%%%%%%%%%%%%%%%%%%%%%%%%%%%%%%%
\subsection{Flags}
\label{sec:flags}

The package makes it easy to generate different versions
of the main or child documents.
To this end compilation flags can be defined
and assigned different default values.
They will be particularly useful in conjunction
with the forwarding mechanism described in \secref{sec:forward}.

For example, it may be useful to have a flag |\version|
which can be set to |draft| or |final|.
The document source will contain some conditional code
depending on the value of |\version|.
Suppose further, the flag should default to |final| for the main file
and to |draft| for child files
which is a natural assignment for editing the document.
This is achieved by placing the following code
in the preamble of the main document
(below the |\childdocmain| directive):
%
\begin{center}
\begin{tabular}{l}
|\ifchilddoc|\\
|\providecommand{\version}{draft}|\\
|\||else|\\
|\providecommand{\version}{final}|\\
|\||fi|
\end{tabular}
\end{center}
%
The definition by |\providecommand| makes sure
that previous definitions are not overwritten.
Further statements |\providecommand{\version}{...}|
can thus be added before the above code to override it.

For the main file, one might add a line
(between |\childdocmain| and the above block)
%
\begin{center}
|%\ifchilddoc\||else\providecommand{\version}{draft}\||fi|
\end{center}
%
which can be uncommented to produce a draft version.
Likewise one can add a line to the very top of a child file
(above the |\childdocof{|\textit{main}|}| directive)
%
\begin{center}
|%\providecommand{\version}{final}|
\end{center}
%
which can be uncommented to produce the final version of this child document.

%%%%%%%%%%%%%%%%%%%%%%%%%%%%%%%%%%%%%%%%%%%%%%%%%%%%%%%%%%%%%%%%%%%%%%%%%%%%%%%%
\subsection{Forwarding}
\label{sec:forward}

Different versions of the main or child documents
using compilation flags as described in \secref{sec:flags}
can be (permanently) stored in different files
for convenient compilation, viewing and distribution.
To this end, the package defines a command
to pass on compilation to a different file:

%%%%%%%%%%%%%%%%%%%%%%%%%%%%%%%%%%%%%%%%
\DescribeMacro{\childdocforward}
The command |\childdocforward| redirects processing to
another source file:
%
\begin{center}
\begin{tabular}{l}
|\input{childdoc.def}|\\
|\childdocforward[|\textit{main}|]{|\textit{dest}|}|\\
\end{tabular}
\end{center}
%
The argument \textit{dest} is the destination file
(without extension).
It should be the main file or one of the child files.
Note that further \textsf{childdoc} directives
such as |\childdocof| and |\childdocforward|
in the indicated file will be processed in this form.
The optional argument \textit{main}
passes on directly to the main file \textit{main}
while pretending to compile the child \textit{dest}.
This form behaves as if \textit{dest}
issues |\childdocof{|\textit{main}|}| right away,
and no further \textsf{childdoc} directives will be processed.

%%%%%%%%%%%%%%%%%%%%%%%%%%%%%%%%%%%%%%%%
\DescribeMacro{\...prefix}
In the alternative form |\childdocforwardprefix|,
%
\begin{center}
\begin{tabular}{l}
|\input{childdoc.def}|\\
|\childdocforwardprefix[|\textit{main}|]{|\textit{prefix}|}{|\textit{dest}|}|
\end{tabular}
\end{center}
%
the destination file is determined by a pattern
depending on the current file:
To make this work, the current file must be called
`{\textit{prefix}\hspace{0.2em}\textit{suffix}}'
with \textit{prefix} matching precisely the argument.
Processing is then passed on to the file
`{\textit{dest}\hspace{0.2em}\textit{suffix}}'.
Surely, the same effect is achieved by
directly specifying the
argument `{\textit{dest}\hspace{0.2em}\textit{suffix}}'
in the first form.
However, that requires to set up a different file
for each child. With the alternative form of the command
all these files can have exactly the same content
which simplifies setting them up and maintaining them.

For example, the following file |draft.tex|
with a compilation flag |\version| as described in \secref{sec:flags}
compiles the main document as a draft:
%
\begin{center}
\begin{tabular}{l}
|\def\version{draft}|\\
|\input{childdoc.def}|\\
|\childdocforward{|\textit{main}|}|
\end{tabular}
\end{center}
%
Likewise, the following files |final|\textit{nn}|.tex|
compile the final version of the child document
|child|\textit{nn}|.tex|:
%
\begin{center}
\begin{tabular}{l}
|\def\version{final}|\\
|\input{childdoc.def}|\\
|\childdocforwardprefix{final}{child}|
\end{tabular}
\end{center}
%

Note that when several versions of a main file and/or of each child file
are to be generated, it may be convenient to set up a |Makefile| or
shell script to automatise the process.

%%%%%%%%%%%%%%%%%%%%%%%%%%%%%%%%%%%%%%%%%%%%%%%%%%%%%%%%%%%%%%%%%%%%%%%%%%%%%%%%
\subsection{Command Line Processing}
\label{sec:commandline}

The effect of redirection files can also be achieved by invoking
the \LaTeX{} compiler with a more elaborate command line.
Most conveniently this should be done as part
of a shell script or a |Makefile|.

When using \textsf{childdoc} in the main file, the following
command lines effectively perform a redirection
(note that depending on the shell being used,
backslashes may have to be doubled: `|\|' $\to$ `|\\|'):
%
\begin{center}
|... -jobname "|\textit{target}|" |\\|"|[\textit{flags}]%
|\input{childdoc.def}\childdocforward[|\textit{main}|]{|\textit{dest}|}"|
\end{center}
%
Here \textit{target} is the name of the output file,
\textit{main} is the name of the main file
and \textit{dest} is the name of the main or child file to be processed
(all filenames without extensions).
The optional argument \textit{main} can be omitted
if \textit{main} matches \textit{dest}.
Optionally, compilation \textit{flags} can be defined via |\def| commands.
This command line makes the \TeX{} engine believe
it is compiling the file \textit{target}
whose content is specified as the latter parameter.
The provided code then forwards the processing to
\textit{main} or \textit{dest} as described in \secref{sec:forward}.

%%%%%%%%%%%%%%%%%%%%%%%%%%%%%%%%%%%%%%%%%%%%%%%%%%%%%%%%%%%%%%%%%%%%%%%%%%%%%%%%
\subsection{Include by Input}
\label{sec:input}

Including child documents by |\include| has some restrictions by design.
Most notably, the content of a child document always occupies
its own set of pages; pages cannot be shared between child documents.
Usually, this behaviour makes perfect sense
because each child document contain an essential part of the document.
However, in some situations it may be desirable to compose
a document from a collection of parts
without having mandatory page breaks between then.
For this case, the package
provides a mechanism to include parts
by |\input| which can also be processed individually.
However, by construction this mechanism
requires manual handling of the content to be output.

%%%%%%%%%%%%%%%%%%%%%%%%%%%%%%%%%%%%%%%%
\DescribeMacro{\ifchilddocmanual}
The main file should be prepared as usual, see \secref{sec:include}.
However, the document body must make a distinction
between processing of an individual part and of the main document, e.g.:
%
\begin{center}
\begin{tabular}{l}
|\ifchilddocmanual|\\
|\input{\childdocname}|\\
|\||else|\\
\textit{document body with }|\input{|\textit{part}|}|\\
|\||fi|
\end{tabular}
\end{center}
%
The conditional |\ifchilddocmanual| is true whenever
a part to be included by |\input| is being compiled,
and the name of the part is stored in |\childdocname|.

%%%%%%%%%%%%%%%%%%%%%%%%%%%%%%%%%%%%%%%%
\DescribeMacro{\childdocby}
Each part to be included by |\input| should start with:
%
\begin{center}
\begin{tabular}{l}
|\input{childdoc.def}|\\
|\childdocby{|\textit{main}|}|\\
\end{tabular}
\end{center}
%
The directive |\childdocby| is similar to |\childdocof|
described in \secref{sec:include},
but the subsequent selection of content must be done manually.
To that end, both |\ifchilddoc| and |\ifchilddocmanual|
will be true upon processing of a part,
and the name of the part is stored in |\childdocname|.
Note that |\jobname| will be set to the filename of the current part
so that each part receives an individual |.aux| file
that does not interfere with the |.aux| file(s) of the main document.
This behaviour can be altered by the alternative form
|\childdocby[*]{|\textit{main}|}| (with a non-empty optional argument)
which uses the |.aux| file of the main document
by setting |\jobname| to \textit{main}.

%%%%%%%%%%%%%%%%%%%%%%%%%%%%%%%%%%%%%%%%%%%%%%%%%%%%%%%%%%%%%%%%%%%%%%%%%%%%%%%%
\subsection{Driver Development}
\label{sec:driver}

The \textsf{childdoc} mechanism can also be use for the development
of definition files such as \LaTeX{} styles or classes.
This case differs from the above setup with multiple parts
included by |\include| in that no |\includeonly| should be invoked.
This can be achieved by starting the include file
(before |\ProvidesPackage|) with:
%
\begin{center}
\begin{tabular}{l}
|\input{childdoc.def}|\\
|\childdocforward{|\textit{main}|}|\\
\end{tabular}
\end{center}
%
or alternatively with:
%
\begin{center}
\begin{tabular}{l}
|\input{childdoc.def}|\\
|\childdocby{|\textit{main}|}|\\
\end{tabular}
\end{center}
%
Both forms have slightly different effects as described above.
The main file is prepared as usual, see \secref{sec:include}.

%%%%%%%%%%%%%%%%%%%%%%%%%%%%%%%%%%%%%%%%%%%%%%%%%%%%%%%%%%%%%%%%%%%%%%%%%%%%%%%%
\subsection{Legacy Detection}
\label{sec:detection}

The directive |\childdocmain| in the main file can detect
whether the complete document or merely a child is to be compiled
even without using the directive |\childdocof|.
This method is deprecated because it is less robust
and there is no compelling reason to use it;
it is merely provided for backward compatibility
and it may be removed in future versions.

If the detection mechanism is to be used,
it is mandatory to correctly specify
the filename of the main file as the argument of |\childdocmain|:
%
\begin{center}
\begin{tabular}{l}
|\input{childdoc.def}|\\
|\childdocmain{|\textit{main}|}|\\
\end{tabular}
\end{center}
%
If |\jobname| does not match the argument \textit{main} of |\childdocmain|,
it is assumed that |\jobname| points to the child file to be compiled.
When using |\childdocmain| with the main file specified as argument,
it suffices to start a child file
with just |\input{|\textit{main}|}|
without loading of the package and using |\childdocof|.
If instead all processing is done
with the appropriate \textsf{childdoc} directives,
the argument of \textit{main} of |\childdocmain| can be empty.

An alternative version of the command line processing described
in \secref{sec:commandline} using the detection mechanism reads:
%
\begin{center}
|... -jobname "|\textit{target}|" "|[\textit{flags}]%
[|\def\jobname{|\textit{dest}|}|]|\input{|\textit{main}|}"|
\end{center}

%%%%%%%%%%%%%%%%%%%%%%%%%%%%%%%%%%%%%%%%%%%%%%%%%%%%%%%%%%%%%%%%%%%%%%%%%%%%%%%%
\subsection{Manual Code}
\label{sec:manual}

In case one cannot be certain whether the definitions file |childdoc.def|
is installed on the target \TeX{} distribution
and one prefers not to ship it,
it is conceivable to paste a few relevant commands into the sources.

To that end, drop all statements |\input{childdoc.def}|
and perform the replacements as outlined below.
Instead of |\childdocmain{|\textit{main}|}| add the following code
to the top of the main file:
%
\begin{center}
\begin{tabular}{l}
|\||ifdefined\childdocname\endinput\||fi\newif\ifchilddoc|\\
|\edef\childdocname{\scantokens\expandafter{\jobname\noexpand}}|\\
|\def\childdocmain{|\textit{main}|}\||ifx\childdocmain\childdocname\||else|\\
|\childdoctrue\includeonly{\childdocname}\let\jobname\childdocmain\||fi|\\
\end{tabular}
\end{center}
%
Instead of |\childdocof{|\textit{main}|}| just include the main file
at the top of each child file:
%
\begin{center}
|\input{|\textit{main}|}|
\end{center}
%
A simple redirection |\childdocforward{|\textit{dest}|}| is achieved by:
%
\begin{center}
|\def\jobname{|\textit{dest}|}\input{\jobname}|
\end{center}
%
The redirection with prefix
|\childdocforwardprefix[|\textit{prefix}|]{|\textit{dest}|}|
is accomplished by:
%
\begin{center}
\begin{tabular}{l}
|{\edef\jobname{\scantokens\expandafter{\jobname\noexpand}}|\\
|\def\redirectjob |\textit{prefix}|#1~~~{\gdef\jobname{|\textit{dest}|#1}}|\\
|\expandafter\redirectjob\jobname~~~}\input{\jobname}|
\end{tabular}
\end{center}

In an alternative approach,
child documents can be compiled by a specific command line
without additional code or specific definitions:
%
\begin{center}
|... -jobname "|\textit{target}|" "|[\textit{flags}]%
|\includeonly{|\textit{dest}|}\input{|\textit{main}|}"|
\end{center}
%

%%%%%%%%%%%%%%%%%%%%%%%%%%%%%%%%%%%%%%%%%%%%%%%%%%%%%%%%%%%%%%%%%%%%%%%%%%%%%%%%
%%%%%%%%%%%%%%%%%%%%%%%%%%%%%%%%%%%%%%%%%%%%%%%%%%%%%%%%%%%%%%%%%%%%%%%%%%%%%%%%
\section{Information}

%%%%%%%%%%%%%%%%%%%%%%%%%%%%%%%%%%%%%%%%%%%%%%%%%%%%%%%%%%%%%%%%%%%%%%%%%%%%%%%%
\subsection{Copyright}

Copyright \copyright{} 2017--2018 Niklas Beisert

This work may be distributed and/or modified under the
conditions of the \LaTeX{} Project Public License, either version 1.3
of this license or (at your option) any later version.
The latest version of this license is in
  \url{http://www.latex-project.org/lppl.txt}
and version 1.3 or later is part of all distributions of \LaTeX{}
version 2005/12/01 or later.

This work has the LPPL maintenance status `maintained'.

The Current Maintainer of this work is Niklas Beisert.

This work consists of the files |README.txt|, |childdoc.ins| and |childdoc.dtx|
as well as the derived files |childdoc.def|, |cdocsamp.tex|
with |cdocsch1.tex|, |cdocsch2.tex|, |cdocspt3.tex|, |cdocspt4.tex|,
|cdocsdrf.tex|, |cdocsfn1.tex|, |cdocsfn2.tex|
as well as |childdoc.pdf|.

%%%%%%%%%%%%%%%%%%%%%%%%%%%%%%%%%%%%%%%%%%%%%%%%%%%%%%%%%%%%%%%%%%%%%%%%%%%%%%%%
\subsection{Files and Installation}

The package consists of the files:
%
\begin{center}
\begin{tabular}{ll}
    |README.txt|   & readme file \\
    |childdoc.ins| & installation file \\
    |childdoc.dtx| & source file \\
    |childdoc.def| & definition file \\
    |cdocsamp.tex| & sample main file \\
    |cdocsch1.tex| & sample include file \\
    |cdocsch2.tex| & sample include file \\
    |cdocspt3.tex| & sample part file \\
    |cdocspt4.tex| & sample part file \\
    |cdocsdrf.tex| & sample redirection file \\
    |cdocsfn1.tex| & sample redirection file \\
    |cdocsfn2.tex| & sample redirection file \\
    |childdoc.pdf| & manual
\end{tabular}
\end{center}
%
The distribution consists of the files
|README.txt|, |childdoc.ins| and |childdoc.dtx|.
%
\begin{itemize}
\item
Run (pdf)\LaTeX{} on |childdoc.dtx|
to compile the manual |childdoc.pdf| (this file).
\item
Run \LaTeX{} on |childdoc.ins| to create the definitions file |childdoc.def|
and the sample |cdocsamp.tex| with include files
|cdocsch1.tex|, |cdocsch2.tex|, |cdocspt3.tex|, |cdocspt4.tex|,
|cdocsdrf.tex|, |cdocsfn1.tex|, |cdocsfn2.tex|.
Then copy the file |childdoc.def| to an appropriate directory of your \LaTeX{}
distribution, e.g.\ \textit{texmf-root}|/tex/latex/childdoc|.
\end{itemize}

%%%%%%%%%%%%%%%%%%%%%%%%%%%%%%%%%%%%%%%%%%%%%%%%%%%%%%%%%%%%%%%%%%%%%%%%%%%%%%%%
\subsection{Related CTAN Packages}

There are several other packages which offer a similar functionality:
%
\begin{itemize}
\item
The packages
\href{http://ctan.org/pkg/docmute}{\textsf{docmute}},
\href{http://ctan.org/pkg/includex}{\textsf{includex}} and
\href{http://ctan.org/pkg/standalone}{\textsf{standalone}}
provide commands to include only the document body of
a child file thus allowing both files to be compiled individually.
\item
The packages \href{http://ctan.org/pkg/subdocs}{\textsf{subdocs}}
and \href{http://ctan.org/pkg/subfiles}{\textsf{subfiles}}
provide structures in which the main and child documents can be
encapsulated and allowing them to be compiled individually.
The inclusion mechanism is different from the conventional |\include|.
\item
The package \href{http://ctan.org/pkg/combine}{\textsf{combine}}
is an elaborate solution to combine several documents into one.
\end{itemize}
%
See also the CTAN topic \href{http://ctan.org/topic/subdocs}{\textsf{subdocs}}
for further related packages.
The present package differs from the above solutions in that
a document structure constructed with the conventional |\include| mechanism
just needs two extra commands at the top of every file
such that all constituent files can be compiled individually.

%%%%%%%%%%%%%%%%%%%%%%%%%%%%%%%%%%%%%%%%%%%%%%%%%%%%%%%%%%%%%%%%%%%%%%%%%%%%%%%%
%\subsection{Feature Suggestions}
%
%The following is a list of features which may be useful for future
%versions of this package:
%%
%\begin{itemize}
%\item
%\ldots
%\end{itemize}

%%%%%%%%%%%%%%%%%%%%%%%%%%%%%%%%%%%%%%%%%%%%%%%%%%%%%%%%%%%%%%%%%%%%%%%%%%%%%%%%
\subsection{Revision History}

%%%%%%%%%%%%%%%%%%%%%%%%%%%%%%%%%%%%%%%%
\paragraph{v2.0:} 2018/12/30

\begin{itemize}
\item
immediate forward processing
\item
added |\childdocby| mechanism
\item
manual restructured
\end{itemize}

%%%%%%%%%%%%%%%%%%%%%%%%%%%%%%%%%%%%%%%%
\paragraph{v1.6:} 2018/01/17

\begin{itemize}
\item
application for development of include files
\item
corrections to manual
\end{itemize}

%%%%%%%%%%%%%%%%%%%%%%%%%%%%%%%%%%%%%%%%
\paragraph{v1.5:} 2017/05/21

\begin{itemize}
\item
more complete structuring introduced
\item
|\childdocof| introduced
\item
|\childdoc| renamed to |\childdocmain|
\item
|\childredirect| renamed to |\childdocforward| and |\childdocforwardprefix|
and functionality expanded
\end{itemize}

%%%%%%%%%%%%%%%%%%%%%%%%%%%%%%%%%%%%%%%%
\paragraph{v1.0:} 2017/04/27

\begin{itemize}
\item
manual and install package
\item
first version published on CTAN
\end{itemize}

%%%%%%%%%%%%%%%%%%%%%%%%%%%%%%%%%%%%%%%%
\paragraph{v0.6:} 2017/04/26

\begin{itemize}
\item
redirection mechanism added
\end{itemize}

%%%%%%%%%%%%%%%%%%%%%%%%%%%%%%%%%%%%%%%%
\paragraph{v0.5:} 2017/04/26

\begin{itemize}
\item
functionality in definition file
\end{itemize}


%%%%%%%%%%%%%%%%%%%%%%%%%%%%%%%%%%%%%%%%%%%%%%%%%%%%%%%%%%%%%%%%%%%%%%%%%%%%%%%%
%%%%%%%%%%%%%%%%%%%%%%%%%%%%%%%%%%%%%%%%%%%%%%%%%%%%%%%%%%%%%%%%%%%%%%%%%%%%%%%%
%%%%%%%%%%%%%%%%%%%%%%%%%%%%%%%%%%%%%%%%%%%%%%%%%%%%%%%%%%%%%%%%%%%%%%%%%%%%%%%%
\appendix

\settowidth\MacroIndent{\rmfamily\scriptsize 000\ }

 \DocInput{childdoc.dtx}

\end{document}
%</driver>
% \fi
%
% %%%%%%%%%%%%%%%%%%%%%%%%%%%%%%%%%%%%%%%%%%%%%%%%%%%%%%%%%%%%%%%%%%%%%%%%%%%%%%
% %%%%%%%%%%%%%%%%%%%%%%%%%%%%%%%%%%%%%%%%%%%%%%%%%%%%%%%%%%%%%%%%%%%%%%%%%%%%%%
% \section{Sample}
%\iffalse
%<*samplemain>
%\fi
%
% The following presents a sample document
% with two chapters, two parts, a title page,
% a compile flag as well as three forwarding files to set the flag.
% It consists of eight |.tex| files:
% \begin{center}
% \begin{tabular}{ll}
% |cdocsamp.tex|&main file\\
% |cdocsch1.tex|&include file for chapter 1\\
% |cdocsch2.tex|&include file for chapter 2\\
% |cdocspt3.tex|&include file for part 3\\
% |cdocspt4.tex|&include file for part 4\\
% |cdocsdrf.tex|&forwarding file for main file in draft mode\\
% |cdocsfi1.tex|&forwarding file for final version of chapter 1\\
% |cdocsfi2.tex|&forwarding file for final version of chapter 2\\
% \end{tabular}
% \end{center}
% Each of the eight files can be compiled directly by the \LaTeX{} compiler.
%
% %%%%%%%%%%%%%%%%%%%%%%%%%%%%%%%%%%%%%%
% \paragraph{Main File.}
%
% The main file is called |cdocsamp.tex|.
%
% Load the \textsf{childdoc} definitions and
% declare the filename for the main document:
%    \begin{macrocode}
\input{childdoc.def}
\childdocmain{}
%    \end{macrocode}

% Optional override for |\version| flag:
%    \begin{macrocode}
%%\ifchilddoc\else\providecommand{\version}{draft}\fi
%    \end{macrocode}

% Define the default values for the |\version| flag
% (|final| for the main file and |draft| for childs):
%    \begin{macrocode}
\ifchilddoc
\providecommand{\version}{draft}
\else
\providecommand{\version}{final}
\fi
%    \end{macrocode}

% Load the standard document class:
%    \begin{macrocode}
\documentclass[12pt]{article}
%    \end{macrocode}

% Start the document body:
%    \begin{macrocode}
\begin{document}
%    \end{macrocode}

% Declare a title page.
% Print title, part of document being processed and version flag:
%    \begin{macrocode}
\addtocounter{page}{-1}
\begin{center}
{\LARGE\bfseries{}childdoc example\par}
\vspace{1cm}
\ifchilddoc
\ifchilddocmanual part\else chapter\fi:
`\childdocname' of `\childdocjob'\par
\else
main document: `\childdocjob'\par
\fi
version: \version\par
\end{center}
\newpage
%    \end{macrocode}

% Manually include selected file,
% otherwise process as usual:
%    \begin{macrocode}
\ifchilddocmanual
\section*{part `\childdocname'}
\input{\childdocname}
\else
%    \end{macrocode}

% Include the two chapters:
%    \begin{macrocode}
\include{cdocsch1}
\include{cdocsch2}
%    \end{macrocode}

% Include the two parts unless only chapters should be displayed:
%    \begin{macrocode}
\ifchilddoc\else
\section{part three}
\input{cdocspt3}
\section{part four}
\input{cdocspt4}
\fi
%    \end{macrocode}

% Process as usual until here:
%    \begin{macrocode}
\fi
%    \end{macrocode}

% End of document body:
%    \begin{macrocode}
\end{document}
%    \end{macrocode}
%\iffalse
%</samplemain>
%\fi
%
% %%%%%%%%%%%%%%%%%%%%%%%%%%%%%%%%%%%%%%
% \paragraph{Chapter Include Files.}
%
% The include files are called |cdocsch1.tex| and |cdocsch2.tex|.
%
%\iffalse
%<*samplechap1|samplechap2>
%\fi

% Optional override for |\version| flag:
%    \begin{macrocode}
%%\providecommand{\version}{final}
%    \end{macrocode}

% Include the main document:
%    \begin{macrocode}
\input{childdoc.def}
\childdocof{cdocsamp}
%    \end{macrocode}

%\iffalse
%</samplechap1|samplechap2>
%\fi
%
%\iffalse
%<*samplechap1>
%\fi
% Some text for chapter 1:
%    \begin{macrocode}
\section{one}
some text in chapter one
%    \end{macrocode}

%\iffalse
%</samplechap1>
%\fi
% Some text for chapter 2:
%\iffalse
%<*samplechap2>
%\fi
%    \begin{macrocode}
\section{two}
more text in chapter two
%    \end{macrocode}

%\iffalse
%</samplechap2>
%\fi
%
% %%%%%%%%%%%%%%%%%%%%%%%%%%%%%%%%%%%%%%
% \paragraph{Part Include Files.}
%
% The include files are called |cdocspt3.tex| and |cdocspt4.tex|.
%
%\iffalse
%<*samplepart3|samplepart4>
%\fi

% Optional override for |\version| flag:
%    \begin{macrocode}
%%\providecommand{\version}{final}
%    \end{macrocode}

% Include the main document:
%    \begin{macrocode}
\input{childdoc.def}
\childdocby{cdocsamp}
%    \end{macrocode}

%\iffalse
%</samplepart3|samplepart4>
%\fi
%
%\iffalse
%<*samplepart3>
%\fi
% Some text for part 3:
%    \begin{macrocode}
some text in part three
%    \end{macrocode}

%\iffalse
%</samplepart3>
%\fi
% Some text for part 4:
%\iffalse
%<*samplepart4>
%\fi
%    \begin{macrocode}
more text in part four
%    \end{macrocode}

%\iffalse
%</samplepart4>
%\fi
%
% %%%%%%%%%%%%%%%%%%%%%%%%%%%%%%%%%%%%%%
% \paragraph{Forwarding for a Complete Draft.}
%
% The following forwarding file |cdocsdrf.tex|
% compiles the main document in draft mode:
%\iffalse
%<*sampledraft>
%\fi
%    \begin{macrocode}
\def\version{draft}
\input{childdoc.def}
\childdocforward{cdocsamp}
%    \end{macrocode}

%\iffalse
%</sampledraft>
%\fi
%
% %%%%%%%%%%%%%%%%%%%%%%%%%%%%%%%%%%%%%%
% \paragraph{Forwarding for Final Version of the Chapters.}
%
% The following forwarding files |cdocsfn1.tex| and |cdocsfn2.tex|
% (with identical content)
% compile the final versions of the child documents
% |cdocsch1.tex| and |cdocsch2.tex|, respectively:
%\iffalse
%<*samplefinal>
%\fi
%    \begin{macrocode}
\def\version{final}
\input{childdoc.def}
\childdocforwardprefix[cdocsamp]{cdocsfn}{cdocsch}
%    \end{macrocode}

%\iffalse
%</samplefinal>
%\fi
%
% %%%%%%%%%%%%%%%%%%%%%%%%%%%%%%%%%%%%%%
% \paragraph{Command Line Processing.}
%
% The following three command lines generate the output files
% |cdocscld|, |cdocscl1| and |cdocscl2|
% which should be identical to
% |cdocsdrf|, |cdocsch1| and |cdocsfn2|, respectively:
% \begin{center}
% \begin{tabular}{l}
% |latex -jobname cdocscld \|\\
% |  "\def\version{draft}\input{childdoc.def}\childdocforward{cdocsamp}"|\\
% |latex -jobname cdocscl1 \|\\
% |  "\input{childdoc.def}\childdocforward[cdocsamp]{cdocsch1}"|\\
% |latex -jobname cdocscl2 \|\\
% |  "\def\version{final}\input{childdoc.def}\childdocforward{cdocsch2}"|
% \end{tabular}
% \end{center}
% Note that the trailing backslash on each first line
% merely continues the input to the second line
% (for convenient cut ant paste).
% Furthermore, the command |latex| can be replaced by any
% of its alternative versions such as |pdflatex|.
%
% %%%%%%%%%%%%%%%%%%%%%%%%%%%%%%%%%%%%%%%%%%%%%%%%%%%%%%%%%%%%%%%%%%%%%%%%%%%%%%
% %%%%%%%%%%%%%%%%%%%%%%%%%%%%%%%%%%%%%%%%%%%%%%%%%%%%%%%%%%%%%%%%%%%%%%%%%%%%%%
% \section{Implementation}
%\iffalse
%<*package>
%\fi
%
% This section describes the definitions file |childdoc.def|.

% The definitions cannot be loaded using |\usepackage| or |\RequirePackage|
% which has a mechanism to prevent loading a style file more than once.
% When loading the definitions by means of |\input|
% multiple instances have to be prevented manually:
%\iffalse
%This code needs to be before the `\ProvidesFile' directive
%which is defined at the beginning of this file.
%Therefore it is also placed there and commented out here.
%</package>
%<*discard>
%\fi
%    \begin{macrocode}
\ifdefined\childdocmain\endinput\fi
%    \end{macrocode}
%\iffalse
%</discard>
%<*package>
%\fi
%
% \macro{\ifchilddoc}
% \macro{\ifchilddocmanual}
% The conditional |\ifchilddoc| tells whether a
% child (true) or main (false) document is being compiled.
% The conditional |\ifchilddocmanual| tells whether
% the |\includeonly| mechanism is used (false) or
% the selection of child files must be performed manually (true).
% The definitions initialise to false:
%    \begin{macrocode}
\newif\ifchilddoc
\newif\ifchilddocmanual
%    \end{macrocode}

% \macro{\childdocname}
% \macro{\childdocjob}
% The macro |\childdocname| stores the name of the main document
% to be compiled. The macro |\childdocjob| stores the name of
% the document on which the \LaTeX{} compiler was originally invoked.
% The content of |\jobname| cannot be compared
% to filenames specified in the source due to different catcodes.
% The following code rescans |\jobname|, stores the result
% in |\childdocname| and saves a copy in |\childdocjob|:
%    \begin{macrocode}
\edef\childdocname{\scantokens\expandafter{\jobname\noexpand}}
\let\childdocjob\childdocname
%    \end{macrocode}

% \macro{\childdocdisable}
% The macro |\childdocdisable| prevents the main file
% from being processed more than once.
% At this stage, the main document command |\childdocmain|
% is assumed to be called once again where it should do nothing.
% Any subsequent call to it should prevent
% a secondary processing of the main document
% It overwrites the forwarding commands
% |\childdocof| and |\childdocforward|
% with empty macros to prevent further inclusions of the main document:
%    \begin{macrocode}
\newcommand{\childdocdisable}
{
  \renewcommand{\childdocmain}[1]{\renewcommand{\childdocmain}[1]{\endinput}}
  \renewcommand{\childdocof}[1]{}
  \renewcommand{\childdocby}[2][]{}
  \renewcommand{\childdocforward}[2][]{}
  \renewcommand{\childdocdisable}{}
}
%    \end{macrocode}

% \macro{\childdocmain}
% The macro |\childdocmain| is to be called at the top of the main file
% with nothing or the main filename (without extension) as argument.
% First, it breaks loops.
% If the argument is not empty and does not match |\childdocname|
% (which is set by the first inclusion of |childdoc.def|),
% |\ifchilddoc| is set to true, |\includeonly| is applied to the child file
% and |\jobname| is set to the main file
% (for proper handling of |.aux| files):
%    \begin{macrocode}
\newcommand{\childdocmain}[1]
{
  \childdocdisable\childdocmain{}
  \if?#1?\else
    \begingroup
      \def\childdoctmp{#1}
      \ifx\childdoctmp\childdocname
        \def\childdoctmp{}
      \else
        \def\childdoctmp
        {
          \childdoctrue
          \includeonly{\childdocname}
          \def\childdocjob{#1}
          \def\jobname{#1}
        }
      \fi
      \expandafter
    \endgroup
    \childdoctmp
  \fi
}
%    \end{macrocode}

% \macro{\childdocof}
% The command |\childdocof| redirects
% compilation to the main file |#1|.
%    \begin{macrocode}
\newcommand{\childdocof}[1]
{
  \childdocdisable
  \childdoctrue
  \includeonly{\childdocname}
  \def\jobname{#1}
  \def\childdocjob{#1}
  \input{#1}
}
%    \end{macrocode}

% \macro{\childdocby}
% The command |\childdocby| ....
%    \begin{macrocode}
\newcommand{\childdocby}[2][]
{
  \childdocdisable
  \childdoctrue
  \childdocmanualtrue
  \if?#1?\else
    \def\jobname{#2}
  \fi
  \def\childdocjob{#2}
  \input{#2}
  \endinput
}
%    \end{macrocode}

% \macro{\childdocforward}
% The command |\childdocforward| redirects
% compilation to the main file or
% (if the optional argument is given) a child file.
% Parameters are set as if the main file
% or a child file starting with |\childdocof| was compiled.
% Then compilation is handed over to the main file:
%    \begin{macrocode}
\newcommand{\childdocforward}[2][]
{
  \begingroup
    \if?#1?
      \def\childdoctmp
      {
        \def\childdocname{#2}
        \def\childdocjob{#2}
        \def\jobname{#2}
        \input{#2}
        \endinput
      }
    \else
      \def\childdoctmp
      {
        \childdocdisable
        \def\childdocname{#2}
        \childdoctrue
        \includeonly{#2}
        \def\childdocjob{#1}
        \def\jobname{#1}
        \input{#1}
        \endinput
      }
    \fi
    \expandafter
  \endgroup
  \childdoctmp
}
%    \end{macrocode}

% \macro{\childdocforwardprefix}
% The command |\childdocforwardprefix| redirects
% compilation to the main or a child file by means of a pattern.
% The prefix |#1| in the current filename is replaced by |#2|
% and the suffix of the current filename is kept
% (it is assumed that the filename does not contain the substring `|~~~|'
% which is used as a delimiter).
% Compilation is handed over to the new file by |\childdocforward|:
%    \begin{macrocode}
\newcommand{\childdocforwardprefix}[3][]
{
  \begingroup
    \def\childdocextract #2##1~~~{\def\childdoctmp{\childdocforward[#1]{#3##1}}}
    \expandafter\childdocextract\childdocname~~~
    \expandafter
  \endgroup
  \childdoctmp
}
%    \end{macrocode}

% \macro{\childdoc}
% The deprecated macro |\childdoc| is a legacy version of |\childdocmain|:
%    \begin{macrocode}
\newcommand{\childdoc}{\childdocmain}
%    \end{macrocode}

% \macro{\childdocredirect}
% The deprecated macro |\childdocredirect| is a legacy version
% of |\childdocforward| and |\childdocforwardprefix|:
%    \begin{macrocode}
\newcommand{\childdocredirect}[2][]
{
  \begingroup
    \if?#1?
      \def\childdoctmp{\childdocforward{#2}}
    \else
      \def\childdoctmp{\childdocforwardprefix{#1}{#2}}
    \fi
    \expandafter
  \endgroup
  \childdoctmp
}
%    \end{macrocode}

%\iffalse
%</package>
%\fi
%
\endinput

\childdocmain{}
%    \end{macrocode}

% Optional override for |\version| flag:
%    \begin{macrocode}
%%\ifchilddoc\else\providecommand{\version}{draft}\fi
%    \end{macrocode}

% Define the default values for the |\version| flag
% (|final| for the main file and |draft| for childs):
%    \begin{macrocode}
\ifchilddoc
\providecommand{\version}{draft}
\else
\providecommand{\version}{final}
\fi
%    \end{macrocode}

% Load the standard document class:
%    \begin{macrocode}
\documentclass[12pt]{article}
%    \end{macrocode}

% Start the document body:
%    \begin{macrocode}
\begin{document}
%    \end{macrocode}

% Declare a title page.
% Print title, part of document being processed and version flag:
%    \begin{macrocode}
\addtocounter{page}{-1}
\begin{center}
{\LARGE\bfseries{}childdoc example\par}
\vspace{1cm}
\ifchilddoc
\ifchilddocmanual part\else chapter\fi:
`\childdocname' of `\childdocjob'\par
\else
main document: `\childdocjob'\par
\fi
version: \version\par
\end{center}
\newpage
%    \end{macrocode}

% Manually include selected file,
% otherwise process as usual:
%    \begin{macrocode}
\ifchilddocmanual
\section*{part `\childdocname'}
\input{\childdocname}
\else
%    \end{macrocode}

% Include the two chapters:
%    \begin{macrocode}
\include{cdocsch1}
\include{cdocsch2}
%    \end{macrocode}

% Include the two parts unless only chapters should be displayed:
%    \begin{macrocode}
\ifchilddoc\else
\section{part three}
\input{cdocspt3}
\section{part four}
\input{cdocspt4}
\fi
%    \end{macrocode}

% Process as usual until here:
%    \begin{macrocode}
\fi
%    \end{macrocode}

% End of document body:
%    \begin{macrocode}
\end{document}
%    \end{macrocode}
%\iffalse
%</samplemain>
%\fi
%
% %%%%%%%%%%%%%%%%%%%%%%%%%%%%%%%%%%%%%%
% \paragraph{Chapter Include Files.}
%
% The include files are called |cdocsch1.tex| and |cdocsch2.tex|.
%
%\iffalse
%<*samplechap1|samplechap2>
%\fi

% Optional override for |\version| flag:
%    \begin{macrocode}
%%\providecommand{\version}{final}
%    \end{macrocode}

% Include the main document:
%    \begin{macrocode}
% \iffalse
%
% childdoc.dtx Copyright (C) 2017-2018 Niklas Beisert
%
% This work may be distributed and/or modified under the
% conditions of the LaTeX Project Public License, either version 1.3
% of this license or (at your option) any later version.
% The latest version of this license is in
%   http://www.latex-project.org/lppl.txt
% and version 1.3 or later is part of all distributions of LaTeX
% version 2005/12/01 or later.
%
% This work has the LPPL maintenance status `maintained'.
%
% The Current Maintainer of this work is Niklas Beisert.
%
% This work consists of the files childdoc.dtx and childdoc.ins
% and the derived files childdoc.def and cdocsamp.tex with
% cdocsch1.tex, cdocsch2.tex, cdocsdrf.tex, cdocsfn1.tex, cdocsfn2.tex.
%
%<package>\ifdefined\childdocmain\endinput\fi
%<package>\ProvidesFile{childdoc.def}[2018/12/30 v2.0 child document driver]
%<samplemain>\ProvidesFile{cdocsamp.tex}[2018/12/30 v2.0 sample for childdoc]
%<*driver>
%\ProvidesFile{childdoc.drv}[2018/12/30 v2.0 childdoc reference manual file]
\PassOptionsToClass{10pt,a4paper}{article}
\documentclass{ltxdoc}

\usepackage[margin=35mm]{geometry}
\usepackage{hyperref}
\usepackage{hyperxmp}
\usepackage[usenames]{color}

\hypersetup{colorlinks=true}
\hypersetup{pdfstartview=FitH}
\hypersetup{pdfpagemode=UseNone}
\hypersetup{pdfsource={}}
\hypersetup{pdflang={en-UK}}
\hypersetup{pdfcopyright={Copyright 2017-2018 Niklas Beisert.
  This work may be distributed and/or modified under the
  conditions of the LaTeX Project Public License, either version 1.3
  of this license or (at your option) any later version.}}
\hypersetup{pdflicenseurl={http://www.latex-project.org/lppl.txt}}
\hypersetup{pdfcontactaddress={ETH Zurich, ITP, HIT K,
  Wolfgang-Pauli-Strasse 27}}
\hypersetup{pdfcontactpostcode={8093}}
\hypersetup{pdfcontactcity={Zurich}}
\hypersetup{pdfcontactcountry={Switzerland}}
\hypersetup{pdfcontactemail={nbeisert@itp.phys.ethz.ch}}
\hypersetup{pdfcontacturl={http://people.phys.ethz.ch/\xmptilde nbeisert/}}

\newcommand{\secref}[1]{\hyperref[#1]{section \ref*{#1}}}

\parskip1ex
\parindent0pt
\let\olditemize\itemize
\def\itemize{\olditemize\parskip0pt}

\begin{document}

\title{The \textsf{childdoc} Package}
\hypersetup{pdftitle={The childdoc Package}}
\author{Niklas Beisert\\[2ex]
  Institut f\"ur Theoretische Physik\\
  Eidgen\"ossische Technische Hochschule Z\"urich\\
  Wolfgang-Pauli-Strasse 27, 8093 Z\"urich, Switzerland\\[1ex]
  \href{mailto:nbeisert@itp.phys.ethz.ch}
  {\texttt{nbeisert@itp.phys.ethz.ch}}}
\hypersetup{pdfauthor={Niklas Beisert}}
\hypersetup{pdfsubject={Manual for the LaTeX2e Package childdoc}}
\date{30 December 2018, \textsf{v2.0}}
\maketitle

\begin{abstract}\noindent
\textsf{childdoc} is a \LaTeXe{} package
that enables the direct compilation
of document sections included by |\include|
to individual files.
\end{abstract}

\begingroup
\parskip0ex
\tableofcontents
\endgroup

%%%%%%%%%%%%%%%%%%%%%%%%%%%%%%%%%%%%%%%%%%%%%%%%%%%%%%%%%%%%%%%%%%%%%%%%%%%%%%%%
%%%%%%%%%%%%%%%%%%%%%%%%%%%%%%%%%%%%%%%%%%%%%%%%%%%%%%%%%%%%%%%%%%%%%%%%%%%%%%%%
\section{Introduction}

\LaTeX{} provides a mechanism to structure a large document (such as a book)
into a main file and several child files (containing the chapters)
using the |\include| command.
This mechanism is beneficial for documents
which span hundreds of pages in order to
make the source file(s) more manageable.
Moreover, compilation can be restricted to
selected child files by means of the |\includeonly| command.
The latter feature can be used to reduce the compilation time while editing
(this was significantly more useful in the earlier days of \LaTeX{})
or to generate a smaller document which is easier to navigate.
Another application of |\includeonly| is to generate
documents consisting of selected parts of the complete document.

However, there are a few drawbacks of the plain |\include| mechanism:
\begin{itemize}
\item
The child files cannot be compiled on their own,
they can only be compiled via the main file.
A naive editing environment
(such as a text editor with an option
to have the current file processed by \LaTeX)
may require one to switch to the main file before compiling;
attempting to compile the child file produces errors.
\item
The main file must be modified (each time)
to adjust the |\includeonly| command
to the present needs. This easily leaves the main file in a messy state.
\item
The generated document will always carry the filename
of the main document. This is inconvenient if
several child files are to be compiled and
to be kept for distribution.
\end{itemize}

The present package provides a simple interface
to make child files individually compilable by \LaTeX{}.
Compiling a child file then has the same effect as compiling
the main file with an |\includeonly| command
to select the appropriate child.
Moreover the generated document will carry the name of the child
rather than the main file.
This resolves all three above issues.

This feature is meant to make the editing of books,
thesis documents and lecture notes somewhat more convenient.
However, the package can also be used efficiently for
composing a series of documents (such as exercise sheets)
which are typically distributed individually.
It then assists the author in generating the individual documents
(potentially in different versions)
as well as a document containing the collected series.
Another application is in developing style files
or other kinds of included material
where compilation of the style file could redirect
to a sample or test file.

%%%%%%%%%%%%%%%%%%%%%%%%%%%%%%%%%%%%%%%%%%%%%%%%%%%%%%%%%%%%%%%%%%%%%%%%%%%%%%%%
%%%%%%%%%%%%%%%%%%%%%%%%%%%%%%%%%%%%%%%%%%%%%%%%%%%%%%%%%%%%%%%%%%%%%%%%%%%%%%%%
\section{Usage}

First of all, the package \textsf{childdoc} is \emph{not} a standard
\LaTeXe{} |.sty| style file! Therefore it needs to be invoked in
a non-standard way.

%%%%%%%%%%%%%%%%%%%%%%%%%%%%%%%%%%%%%%%%%%%%%%%%%%%%%%%%%%%%%%%%%%%%%%%%%%%%%%%%
\subsection{Included Files}
\label{sec:include}

%%%%%%%%%%%%%%%%%%%%%%%%%%%%%%%%%%%%%%%%
\DescribeMacro{\childdocmain}
To use the package, add the commands
\begin{center}
\begin{tabular}{l}
|\input{childdoc.def}|\\
|\childdocmain{}|\\
\end{tabular}
\end{center}
at the very top of the main \LaTeX{} file,
in particular \emph{before} the |\documentclass| statement!
The argument of |\childdocmain| should be left empty
(but it must be present).

%%%%%%%%%%%%%%%%%%%%%%%%%%%%%%%%%%%%%%%%
\DescribeMacro{\childdocof}
Furthermore, add the commands
\begin{center}
\begin{tabular}{l}
|\input{childdoc.def}|\\
|\childdocof{|\textit{main}|}|\\
\end{tabular}
\end{center}
at the top of every child file \textit{child}
which is included by |\include{|\textit{child}|}|
from within the main file
(or at least for those files to be compiled individually).
The argument \textit{main} must be the filename of the main file.

There are a couple of
considerations in setting up the main and child documents:

%%%%%%%%%%%%%%%%%%%%%%%%%%%%%%%%%%%%%%%%
\paragraph{Restrictions.}

Please note the following restrictions:
\begin{itemize}
\item
|\childdocmain| must be called with one argument \textit{main}
to ensure compatibility with earlier version of the package.
It must either be empty (|\childdocmain{}|)
or precisely match the filename of the main file in which it is specified.
See \secref{sec:detection} for further information.
\item
The filename \textit{main} must be specified without the |.tex| extension.
\item
The filename \textit{main} is case sensitive
(even in case-insensitive file systems)
due to internal string comparison.
\item
The argument \textit{main} should be fully expanded, it cannot be a macro.
\item
Subdirectories and special characters should be avoided in filenames.
\item
The command |\childdocmain{|\textit{main}|}| must be followed by a whitespace.
It should not be followed immediately by another command
or by a comment mark `|%|'.
This is because the \TeX{} parser reads the token immediately following
the argument of |\childdocmain| and puts it
at the beginning of every child section;
however, a white\-space is ignored.
\end{itemize}

%%%%%%%%%%%%%%%%%%%%%%%%%%%%%%%%%%%%%%%%
\paragraph{Content of Main File.}

It is advisable to place all content in the child files included by |\include|.
Any output contained in the main file will appear in all child documents
unless suppressed manually;
it cannot be suppressed automatically by the |\includeonly| directive
and thus should normally be avoided.
A method to include some content in the main file
by means of conditional processing is described in \secref{sec:conditional}.

%%%%%%%%%%%%%%%%%%%%%%%%%%%%%%%%%%%%%%%%
\paragraph{Page Numbering.}

When only a part of the document is compiled,
the appropriate numbering of pages
(as well as other status parameters)
is determined from the |.aux| files.
The latter contain information from previous passes.
However this information needs to propagate through
all intermediate child documents.
Therefore the page numbering in child documents may well
be inconsistent until the complete document is compiled at least once.

A useful (if unconventional) way to always ensure a consistent
page numbering is to restart the numbering in each child document
and denote the pages by `\textit{child}|.|\textit{page}'
where \textit{child} represents the chapter/section number of the child file.
This can be achieved by the command
|\numberwithin{page}{|\textit{child}|}|
of the \textsf{amsmath} package
where \textit{child} can be |chapter| or |section|
depending on the chosen structuring.
Alternatively, one can modify the macro |\thepage| appropriately
and reset the counter |page| at the start of each child file.

%%%%%%%%%%%%%%%%%%%%%%%%%%%%%%%%%%%%%%%%%%%%%%%%%%%%%%%%%%%%%%%%%%%%%%%%%%%%%%%%
\subsection{Conditional Processing}
\label{sec:conditional}

The package provides a mechanism to compile different versions
of a document. To customise the versions further some conditional processing
can come in handy to distinguish which version is being compiled.
The package provides two macros to describe the compilation context:

%%%%%%%%%%%%%%%%%%%%%%%%%%%%%%%%%%%%%%%%
\DescribeMacro{\ifchilddoc}
The conditional |\ifchilddoc| distinguishes between the compilation of
child documents and the main document:
%
\begin{center}
|\ifchilddoc |\textit{child-code}| |[|\||else |\textit{main-code}]| \||fi|
\end{center}

%%%%%%%%%%%%%%%%%%%%%%%%%%%%%%%%%%%%%%%%
\DescribeMacro{\childdocname}
\DescribeMacro{\childdocjob}
The macro |\childdocname| contains the filename (without extension)
of the main or child file being processed.
Note that |\childdocjob| will always contain the name of the main file.

%%%%%%%%%%%%%%%%%%%%%%%%%%%%%%%%%%%%%%%%
\paragraph{Title Page.}

Conditional processing can be used to include a title or banner page
in the main document when proper precautions are taken.
Importantly, the code in the main file should ensure that the page counter
(as well as other status parameters which are stored in the |.aux| files)
takes the same value after the conditional processing.
Otherwise the page numbers may take divergent values
depending on which part is compiled.

For example, a title page could be declared by:
%
\begin{center}
\begin{tabular}{l}
|\ifchilddoc\||else|\\
|\addtocounter{page}{-1}|\\
\textit{code for title page}\\
|\newpage|\\
|\||fi|
\end{tabular}
\end{center}
%
A banner page for the child documents can be generated by:
%
\begin{center}
\begin{tabular}{l}
|\ifchilddoc|\\
|\addtocounter{page}{-1}|\\
\textit{code for banner page}\\
|\newpage|\\
|\||fi|
\end{tabular}
\end{center}
%
Here one could write a message such as:
\begin{center}
|This is the part \childdocname{} of \childdocjob{}.|
\end{center}

%%%%%%%%%%%%%%%%%%%%%%%%%%%%%%%%%%%%%%%%%%%%%%%%%%%%%%%%%%%%%%%%%%%%%%%%%%%%%%%%
\subsection{Flags}
\label{sec:flags}

The package makes it easy to generate different versions
of the main or child documents.
To this end compilation flags can be defined
and assigned different default values.
They will be particularly useful in conjunction
with the forwarding mechanism described in \secref{sec:forward}.

For example, it may be useful to have a flag |\version|
which can be set to |draft| or |final|.
The document source will contain some conditional code
depending on the value of |\version|.
Suppose further, the flag should default to |final| for the main file
and to |draft| for child files
which is a natural assignment for editing the document.
This is achieved by placing the following code
in the preamble of the main document
(below the |\childdocmain| directive):
%
\begin{center}
\begin{tabular}{l}
|\ifchilddoc|\\
|\providecommand{\version}{draft}|\\
|\||else|\\
|\providecommand{\version}{final}|\\
|\||fi|
\end{tabular}
\end{center}
%
The definition by |\providecommand| makes sure
that previous definitions are not overwritten.
Further statements |\providecommand{\version}{...}|
can thus be added before the above code to override it.

For the main file, one might add a line
(between |\childdocmain| and the above block)
%
\begin{center}
|%\ifchilddoc\||else\providecommand{\version}{draft}\||fi|
\end{center}
%
which can be uncommented to produce a draft version.
Likewise one can add a line to the very top of a child file
(above the |\childdocof{|\textit{main}|}| directive)
%
\begin{center}
|%\providecommand{\version}{final}|
\end{center}
%
which can be uncommented to produce the final version of this child document.

%%%%%%%%%%%%%%%%%%%%%%%%%%%%%%%%%%%%%%%%%%%%%%%%%%%%%%%%%%%%%%%%%%%%%%%%%%%%%%%%
\subsection{Forwarding}
\label{sec:forward}

Different versions of the main or child documents
using compilation flags as described in \secref{sec:flags}
can be (permanently) stored in different files
for convenient compilation, viewing and distribution.
To this end, the package defines a command
to pass on compilation to a different file:

%%%%%%%%%%%%%%%%%%%%%%%%%%%%%%%%%%%%%%%%
\DescribeMacro{\childdocforward}
The command |\childdocforward| redirects processing to
another source file:
%
\begin{center}
\begin{tabular}{l}
|\input{childdoc.def}|\\
|\childdocforward[|\textit{main}|]{|\textit{dest}|}|\\
\end{tabular}
\end{center}
%
The argument \textit{dest} is the destination file
(without extension).
It should be the main file or one of the child files.
Note that further \textsf{childdoc} directives
such as |\childdocof| and |\childdocforward|
in the indicated file will be processed in this form.
The optional argument \textit{main}
passes on directly to the main file \textit{main}
while pretending to compile the child \textit{dest}.
This form behaves as if \textit{dest}
issues |\childdocof{|\textit{main}|}| right away,
and no further \textsf{childdoc} directives will be processed.

%%%%%%%%%%%%%%%%%%%%%%%%%%%%%%%%%%%%%%%%
\DescribeMacro{\...prefix}
In the alternative form |\childdocforwardprefix|,
%
\begin{center}
\begin{tabular}{l}
|\input{childdoc.def}|\\
|\childdocforwardprefix[|\textit{main}|]{|\textit{prefix}|}{|\textit{dest}|}|
\end{tabular}
\end{center}
%
the destination file is determined by a pattern
depending on the current file:
To make this work, the current file must be called
`{\textit{prefix}\hspace{0.2em}\textit{suffix}}'
with \textit{prefix} matching precisely the argument.
Processing is then passed on to the file
`{\textit{dest}\hspace{0.2em}\textit{suffix}}'.
Surely, the same effect is achieved by
directly specifying the
argument `{\textit{dest}\hspace{0.2em}\textit{suffix}}'
in the first form.
However, that requires to set up a different file
for each child. With the alternative form of the command
all these files can have exactly the same content
which simplifies setting them up and maintaining them.

For example, the following file |draft.tex|
with a compilation flag |\version| as described in \secref{sec:flags}
compiles the main document as a draft:
%
\begin{center}
\begin{tabular}{l}
|\def\version{draft}|\\
|\input{childdoc.def}|\\
|\childdocforward{|\textit{main}|}|
\end{tabular}
\end{center}
%
Likewise, the following files |final|\textit{nn}|.tex|
compile the final version of the child document
|child|\textit{nn}|.tex|:
%
\begin{center}
\begin{tabular}{l}
|\def\version{final}|\\
|\input{childdoc.def}|\\
|\childdocforwardprefix{final}{child}|
\end{tabular}
\end{center}
%

Note that when several versions of a main file and/or of each child file
are to be generated, it may be convenient to set up a |Makefile| or
shell script to automatise the process.

%%%%%%%%%%%%%%%%%%%%%%%%%%%%%%%%%%%%%%%%%%%%%%%%%%%%%%%%%%%%%%%%%%%%%%%%%%%%%%%%
\subsection{Command Line Processing}
\label{sec:commandline}

The effect of redirection files can also be achieved by invoking
the \LaTeX{} compiler with a more elaborate command line.
Most conveniently this should be done as part
of a shell script or a |Makefile|.

When using \textsf{childdoc} in the main file, the following
command lines effectively perform a redirection
(note that depending on the shell being used,
backslashes may have to be doubled: `|\|' $\to$ `|\\|'):
%
\begin{center}
|... -jobname "|\textit{target}|" |\\|"|[\textit{flags}]%
|\input{childdoc.def}\childdocforward[|\textit{main}|]{|\textit{dest}|}"|
\end{center}
%
Here \textit{target} is the name of the output file,
\textit{main} is the name of the main file
and \textit{dest} is the name of the main or child file to be processed
(all filenames without extensions).
The optional argument \textit{main} can be omitted
if \textit{main} matches \textit{dest}.
Optionally, compilation \textit{flags} can be defined via |\def| commands.
This command line makes the \TeX{} engine believe
it is compiling the file \textit{target}
whose content is specified as the latter parameter.
The provided code then forwards the processing to
\textit{main} or \textit{dest} as described in \secref{sec:forward}.

%%%%%%%%%%%%%%%%%%%%%%%%%%%%%%%%%%%%%%%%%%%%%%%%%%%%%%%%%%%%%%%%%%%%%%%%%%%%%%%%
\subsection{Include by Input}
\label{sec:input}

Including child documents by |\include| has some restrictions by design.
Most notably, the content of a child document always occupies
its own set of pages; pages cannot be shared between child documents.
Usually, this behaviour makes perfect sense
because each child document contain an essential part of the document.
However, in some situations it may be desirable to compose
a document from a collection of parts
without having mandatory page breaks between then.
For this case, the package
provides a mechanism to include parts
by |\input| which can also be processed individually.
However, by construction this mechanism
requires manual handling of the content to be output.

%%%%%%%%%%%%%%%%%%%%%%%%%%%%%%%%%%%%%%%%
\DescribeMacro{\ifchilddocmanual}
The main file should be prepared as usual, see \secref{sec:include}.
However, the document body must make a distinction
between processing of an individual part and of the main document, e.g.:
%
\begin{center}
\begin{tabular}{l}
|\ifchilddocmanual|\\
|\input{\childdocname}|\\
|\||else|\\
\textit{document body with }|\input{|\textit{part}|}|\\
|\||fi|
\end{tabular}
\end{center}
%
The conditional |\ifchilddocmanual| is true whenever
a part to be included by |\input| is being compiled,
and the name of the part is stored in |\childdocname|.

%%%%%%%%%%%%%%%%%%%%%%%%%%%%%%%%%%%%%%%%
\DescribeMacro{\childdocby}
Each part to be included by |\input| should start with:
%
\begin{center}
\begin{tabular}{l}
|\input{childdoc.def}|\\
|\childdocby{|\textit{main}|}|\\
\end{tabular}
\end{center}
%
The directive |\childdocby| is similar to |\childdocof|
described in \secref{sec:include},
but the subsequent selection of content must be done manually.
To that end, both |\ifchilddoc| and |\ifchilddocmanual|
will be true upon processing of a part,
and the name of the part is stored in |\childdocname|.
Note that |\jobname| will be set to the filename of the current part
so that each part receives an individual |.aux| file
that does not interfere with the |.aux| file(s) of the main document.
This behaviour can be altered by the alternative form
|\childdocby[*]{|\textit{main}|}| (with a non-empty optional argument)
which uses the |.aux| file of the main document
by setting |\jobname| to \textit{main}.

%%%%%%%%%%%%%%%%%%%%%%%%%%%%%%%%%%%%%%%%%%%%%%%%%%%%%%%%%%%%%%%%%%%%%%%%%%%%%%%%
\subsection{Driver Development}
\label{sec:driver}

The \textsf{childdoc} mechanism can also be use for the development
of definition files such as \LaTeX{} styles or classes.
This case differs from the above setup with multiple parts
included by |\include| in that no |\includeonly| should be invoked.
This can be achieved by starting the include file
(before |\ProvidesPackage|) with:
%
\begin{center}
\begin{tabular}{l}
|\input{childdoc.def}|\\
|\childdocforward{|\textit{main}|}|\\
\end{tabular}
\end{center}
%
or alternatively with:
%
\begin{center}
\begin{tabular}{l}
|\input{childdoc.def}|\\
|\childdocby{|\textit{main}|}|\\
\end{tabular}
\end{center}
%
Both forms have slightly different effects as described above.
The main file is prepared as usual, see \secref{sec:include}.

%%%%%%%%%%%%%%%%%%%%%%%%%%%%%%%%%%%%%%%%%%%%%%%%%%%%%%%%%%%%%%%%%%%%%%%%%%%%%%%%
\subsection{Legacy Detection}
\label{sec:detection}

The directive |\childdocmain| in the main file can detect
whether the complete document or merely a child is to be compiled
even without using the directive |\childdocof|.
This method is deprecated because it is less robust
and there is no compelling reason to use it;
it is merely provided for backward compatibility
and it may be removed in future versions.

If the detection mechanism is to be used,
it is mandatory to correctly specify
the filename of the main file as the argument of |\childdocmain|:
%
\begin{center}
\begin{tabular}{l}
|\input{childdoc.def}|\\
|\childdocmain{|\textit{main}|}|\\
\end{tabular}
\end{center}
%
If |\jobname| does not match the argument \textit{main} of |\childdocmain|,
it is assumed that |\jobname| points to the child file to be compiled.
When using |\childdocmain| with the main file specified as argument,
it suffices to start a child file
with just |\input{|\textit{main}|}|
without loading of the package and using |\childdocof|.
If instead all processing is done
with the appropriate \textsf{childdoc} directives,
the argument of \textit{main} of |\childdocmain| can be empty.

An alternative version of the command line processing described
in \secref{sec:commandline} using the detection mechanism reads:
%
\begin{center}
|... -jobname "|\textit{target}|" "|[\textit{flags}]%
[|\def\jobname{|\textit{dest}|}|]|\input{|\textit{main}|}"|
\end{center}

%%%%%%%%%%%%%%%%%%%%%%%%%%%%%%%%%%%%%%%%%%%%%%%%%%%%%%%%%%%%%%%%%%%%%%%%%%%%%%%%
\subsection{Manual Code}
\label{sec:manual}

In case one cannot be certain whether the definitions file |childdoc.def|
is installed on the target \TeX{} distribution
and one prefers not to ship it,
it is conceivable to paste a few relevant commands into the sources.

To that end, drop all statements |\input{childdoc.def}|
and perform the replacements as outlined below.
Instead of |\childdocmain{|\textit{main}|}| add the following code
to the top of the main file:
%
\begin{center}
\begin{tabular}{l}
|\||ifdefined\childdocname\endinput\||fi\newif\ifchilddoc|\\
|\edef\childdocname{\scantokens\expandafter{\jobname\noexpand}}|\\
|\def\childdocmain{|\textit{main}|}\||ifx\childdocmain\childdocname\||else|\\
|\childdoctrue\includeonly{\childdocname}\let\jobname\childdocmain\||fi|\\
\end{tabular}
\end{center}
%
Instead of |\childdocof{|\textit{main}|}| just include the main file
at the top of each child file:
%
\begin{center}
|\input{|\textit{main}|}|
\end{center}
%
A simple redirection |\childdocforward{|\textit{dest}|}| is achieved by:
%
\begin{center}
|\def\jobname{|\textit{dest}|}\input{\jobname}|
\end{center}
%
The redirection with prefix
|\childdocforwardprefix[|\textit{prefix}|]{|\textit{dest}|}|
is accomplished by:
%
\begin{center}
\begin{tabular}{l}
|{\edef\jobname{\scantokens\expandafter{\jobname\noexpand}}|\\
|\def\redirectjob |\textit{prefix}|#1~~~{\gdef\jobname{|\textit{dest}|#1}}|\\
|\expandafter\redirectjob\jobname~~~}\input{\jobname}|
\end{tabular}
\end{center}

In an alternative approach,
child documents can be compiled by a specific command line
without additional code or specific definitions:
%
\begin{center}
|... -jobname "|\textit{target}|" "|[\textit{flags}]%
|\includeonly{|\textit{dest}|}\input{|\textit{main}|}"|
\end{center}
%

%%%%%%%%%%%%%%%%%%%%%%%%%%%%%%%%%%%%%%%%%%%%%%%%%%%%%%%%%%%%%%%%%%%%%%%%%%%%%%%%
%%%%%%%%%%%%%%%%%%%%%%%%%%%%%%%%%%%%%%%%%%%%%%%%%%%%%%%%%%%%%%%%%%%%%%%%%%%%%%%%
\section{Information}

%%%%%%%%%%%%%%%%%%%%%%%%%%%%%%%%%%%%%%%%%%%%%%%%%%%%%%%%%%%%%%%%%%%%%%%%%%%%%%%%
\subsection{Copyright}

Copyright \copyright{} 2017--2018 Niklas Beisert

This work may be distributed and/or modified under the
conditions of the \LaTeX{} Project Public License, either version 1.3
of this license or (at your option) any later version.
The latest version of this license is in
  \url{http://www.latex-project.org/lppl.txt}
and version 1.3 or later is part of all distributions of \LaTeX{}
version 2005/12/01 or later.

This work has the LPPL maintenance status `maintained'.

The Current Maintainer of this work is Niklas Beisert.

This work consists of the files |README.txt|, |childdoc.ins| and |childdoc.dtx|
as well as the derived files |childdoc.def|, |cdocsamp.tex|
with |cdocsch1.tex|, |cdocsch2.tex|, |cdocspt3.tex|, |cdocspt4.tex|,
|cdocsdrf.tex|, |cdocsfn1.tex|, |cdocsfn2.tex|
as well as |childdoc.pdf|.

%%%%%%%%%%%%%%%%%%%%%%%%%%%%%%%%%%%%%%%%%%%%%%%%%%%%%%%%%%%%%%%%%%%%%%%%%%%%%%%%
\subsection{Files and Installation}

The package consists of the files:
%
\begin{center}
\begin{tabular}{ll}
    |README.txt|   & readme file \\
    |childdoc.ins| & installation file \\
    |childdoc.dtx| & source file \\
    |childdoc.def| & definition file \\
    |cdocsamp.tex| & sample main file \\
    |cdocsch1.tex| & sample include file \\
    |cdocsch2.tex| & sample include file \\
    |cdocspt3.tex| & sample part file \\
    |cdocspt4.tex| & sample part file \\
    |cdocsdrf.tex| & sample redirection file \\
    |cdocsfn1.tex| & sample redirection file \\
    |cdocsfn2.tex| & sample redirection file \\
    |childdoc.pdf| & manual
\end{tabular}
\end{center}
%
The distribution consists of the files
|README.txt|, |childdoc.ins| and |childdoc.dtx|.
%
\begin{itemize}
\item
Run (pdf)\LaTeX{} on |childdoc.dtx|
to compile the manual |childdoc.pdf| (this file).
\item
Run \LaTeX{} on |childdoc.ins| to create the definitions file |childdoc.def|
and the sample |cdocsamp.tex| with include files
|cdocsch1.tex|, |cdocsch2.tex|, |cdocspt3.tex|, |cdocspt4.tex|,
|cdocsdrf.tex|, |cdocsfn1.tex|, |cdocsfn2.tex|.
Then copy the file |childdoc.def| to an appropriate directory of your \LaTeX{}
distribution, e.g.\ \textit{texmf-root}|/tex/latex/childdoc|.
\end{itemize}

%%%%%%%%%%%%%%%%%%%%%%%%%%%%%%%%%%%%%%%%%%%%%%%%%%%%%%%%%%%%%%%%%%%%%%%%%%%%%%%%
\subsection{Related CTAN Packages}

There are several other packages which offer a similar functionality:
%
\begin{itemize}
\item
The packages
\href{http://ctan.org/pkg/docmute}{\textsf{docmute}},
\href{http://ctan.org/pkg/includex}{\textsf{includex}} and
\href{http://ctan.org/pkg/standalone}{\textsf{standalone}}
provide commands to include only the document body of
a child file thus allowing both files to be compiled individually.
\item
The packages \href{http://ctan.org/pkg/subdocs}{\textsf{subdocs}}
and \href{http://ctan.org/pkg/subfiles}{\textsf{subfiles}}
provide structures in which the main and child documents can be
encapsulated and allowing them to be compiled individually.
The inclusion mechanism is different from the conventional |\include|.
\item
The package \href{http://ctan.org/pkg/combine}{\textsf{combine}}
is an elaborate solution to combine several documents into one.
\end{itemize}
%
See also the CTAN topic \href{http://ctan.org/topic/subdocs}{\textsf{subdocs}}
for further related packages.
The present package differs from the above solutions in that
a document structure constructed with the conventional |\include| mechanism
just needs two extra commands at the top of every file
such that all constituent files can be compiled individually.

%%%%%%%%%%%%%%%%%%%%%%%%%%%%%%%%%%%%%%%%%%%%%%%%%%%%%%%%%%%%%%%%%%%%%%%%%%%%%%%%
%\subsection{Feature Suggestions}
%
%The following is a list of features which may be useful for future
%versions of this package:
%%
%\begin{itemize}
%\item
%\ldots
%\end{itemize}

%%%%%%%%%%%%%%%%%%%%%%%%%%%%%%%%%%%%%%%%%%%%%%%%%%%%%%%%%%%%%%%%%%%%%%%%%%%%%%%%
\subsection{Revision History}

%%%%%%%%%%%%%%%%%%%%%%%%%%%%%%%%%%%%%%%%
\paragraph{v2.0:} 2018/12/30

\begin{itemize}
\item
immediate forward processing
\item
added |\childdocby| mechanism
\item
manual restructured
\end{itemize}

%%%%%%%%%%%%%%%%%%%%%%%%%%%%%%%%%%%%%%%%
\paragraph{v1.6:} 2018/01/17

\begin{itemize}
\item
application for development of include files
\item
corrections to manual
\end{itemize}

%%%%%%%%%%%%%%%%%%%%%%%%%%%%%%%%%%%%%%%%
\paragraph{v1.5:} 2017/05/21

\begin{itemize}
\item
more complete structuring introduced
\item
|\childdocof| introduced
\item
|\childdoc| renamed to |\childdocmain|
\item
|\childredirect| renamed to |\childdocforward| and |\childdocforwardprefix|
and functionality expanded
\end{itemize}

%%%%%%%%%%%%%%%%%%%%%%%%%%%%%%%%%%%%%%%%
\paragraph{v1.0:} 2017/04/27

\begin{itemize}
\item
manual and install package
\item
first version published on CTAN
\end{itemize}

%%%%%%%%%%%%%%%%%%%%%%%%%%%%%%%%%%%%%%%%
\paragraph{v0.6:} 2017/04/26

\begin{itemize}
\item
redirection mechanism added
\end{itemize}

%%%%%%%%%%%%%%%%%%%%%%%%%%%%%%%%%%%%%%%%
\paragraph{v0.5:} 2017/04/26

\begin{itemize}
\item
functionality in definition file
\end{itemize}


%%%%%%%%%%%%%%%%%%%%%%%%%%%%%%%%%%%%%%%%%%%%%%%%%%%%%%%%%%%%%%%%%%%%%%%%%%%%%%%%
%%%%%%%%%%%%%%%%%%%%%%%%%%%%%%%%%%%%%%%%%%%%%%%%%%%%%%%%%%%%%%%%%%%%%%%%%%%%%%%%
%%%%%%%%%%%%%%%%%%%%%%%%%%%%%%%%%%%%%%%%%%%%%%%%%%%%%%%%%%%%%%%%%%%%%%%%%%%%%%%%
\appendix

\settowidth\MacroIndent{\rmfamily\scriptsize 000\ }

 \DocInput{childdoc.dtx}

\end{document}
%</driver>
% \fi
%
% %%%%%%%%%%%%%%%%%%%%%%%%%%%%%%%%%%%%%%%%%%%%%%%%%%%%%%%%%%%%%%%%%%%%%%%%%%%%%%
% %%%%%%%%%%%%%%%%%%%%%%%%%%%%%%%%%%%%%%%%%%%%%%%%%%%%%%%%%%%%%%%%%%%%%%%%%%%%%%
% \section{Sample}
%\iffalse
%<*samplemain>
%\fi
%
% The following presents a sample document
% with two chapters, two parts, a title page,
% a compile flag as well as three forwarding files to set the flag.
% It consists of eight |.tex| files:
% \begin{center}
% \begin{tabular}{ll}
% |cdocsamp.tex|&main file\\
% |cdocsch1.tex|&include file for chapter 1\\
% |cdocsch2.tex|&include file for chapter 2\\
% |cdocspt3.tex|&include file for part 3\\
% |cdocspt4.tex|&include file for part 4\\
% |cdocsdrf.tex|&forwarding file for main file in draft mode\\
% |cdocsfi1.tex|&forwarding file for final version of chapter 1\\
% |cdocsfi2.tex|&forwarding file for final version of chapter 2\\
% \end{tabular}
% \end{center}
% Each of the eight files can be compiled directly by the \LaTeX{} compiler.
%
% %%%%%%%%%%%%%%%%%%%%%%%%%%%%%%%%%%%%%%
% \paragraph{Main File.}
%
% The main file is called |cdocsamp.tex|.
%
% Load the \textsf{childdoc} definitions and
% declare the filename for the main document:
%    \begin{macrocode}
\input{childdoc.def}
\childdocmain{}
%    \end{macrocode}

% Optional override for |\version| flag:
%    \begin{macrocode}
%%\ifchilddoc\else\providecommand{\version}{draft}\fi
%    \end{macrocode}

% Define the default values for the |\version| flag
% (|final| for the main file and |draft| for childs):
%    \begin{macrocode}
\ifchilddoc
\providecommand{\version}{draft}
\else
\providecommand{\version}{final}
\fi
%    \end{macrocode}

% Load the standard document class:
%    \begin{macrocode}
\documentclass[12pt]{article}
%    \end{macrocode}

% Start the document body:
%    \begin{macrocode}
\begin{document}
%    \end{macrocode}

% Declare a title page.
% Print title, part of document being processed and version flag:
%    \begin{macrocode}
\addtocounter{page}{-1}
\begin{center}
{\LARGE\bfseries{}childdoc example\par}
\vspace{1cm}
\ifchilddoc
\ifchilddocmanual part\else chapter\fi:
`\childdocname' of `\childdocjob'\par
\else
main document: `\childdocjob'\par
\fi
version: \version\par
\end{center}
\newpage
%    \end{macrocode}

% Manually include selected file,
% otherwise process as usual:
%    \begin{macrocode}
\ifchilddocmanual
\section*{part `\childdocname'}
\input{\childdocname}
\else
%    \end{macrocode}

% Include the two chapters:
%    \begin{macrocode}
\include{cdocsch1}
\include{cdocsch2}
%    \end{macrocode}

% Include the two parts unless only chapters should be displayed:
%    \begin{macrocode}
\ifchilddoc\else
\section{part three}
\input{cdocspt3}
\section{part four}
\input{cdocspt4}
\fi
%    \end{macrocode}

% Process as usual until here:
%    \begin{macrocode}
\fi
%    \end{macrocode}

% End of document body:
%    \begin{macrocode}
\end{document}
%    \end{macrocode}
%\iffalse
%</samplemain>
%\fi
%
% %%%%%%%%%%%%%%%%%%%%%%%%%%%%%%%%%%%%%%
% \paragraph{Chapter Include Files.}
%
% The include files are called |cdocsch1.tex| and |cdocsch2.tex|.
%
%\iffalse
%<*samplechap1|samplechap2>
%\fi

% Optional override for |\version| flag:
%    \begin{macrocode}
%%\providecommand{\version}{final}
%    \end{macrocode}

% Include the main document:
%    \begin{macrocode}
\input{childdoc.def}
\childdocof{cdocsamp}
%    \end{macrocode}

%\iffalse
%</samplechap1|samplechap2>
%\fi
%
%\iffalse
%<*samplechap1>
%\fi
% Some text for chapter 1:
%    \begin{macrocode}
\section{one}
some text in chapter one
%    \end{macrocode}

%\iffalse
%</samplechap1>
%\fi
% Some text for chapter 2:
%\iffalse
%<*samplechap2>
%\fi
%    \begin{macrocode}
\section{two}
more text in chapter two
%    \end{macrocode}

%\iffalse
%</samplechap2>
%\fi
%
% %%%%%%%%%%%%%%%%%%%%%%%%%%%%%%%%%%%%%%
% \paragraph{Part Include Files.}
%
% The include files are called |cdocspt3.tex| and |cdocspt4.tex|.
%
%\iffalse
%<*samplepart3|samplepart4>
%\fi

% Optional override for |\version| flag:
%    \begin{macrocode}
%%\providecommand{\version}{final}
%    \end{macrocode}

% Include the main document:
%    \begin{macrocode}
\input{childdoc.def}
\childdocby{cdocsamp}
%    \end{macrocode}

%\iffalse
%</samplepart3|samplepart4>
%\fi
%
%\iffalse
%<*samplepart3>
%\fi
% Some text for part 3:
%    \begin{macrocode}
some text in part three
%    \end{macrocode}

%\iffalse
%</samplepart3>
%\fi
% Some text for part 4:
%\iffalse
%<*samplepart4>
%\fi
%    \begin{macrocode}
more text in part four
%    \end{macrocode}

%\iffalse
%</samplepart4>
%\fi
%
% %%%%%%%%%%%%%%%%%%%%%%%%%%%%%%%%%%%%%%
% \paragraph{Forwarding for a Complete Draft.}
%
% The following forwarding file |cdocsdrf.tex|
% compiles the main document in draft mode:
%\iffalse
%<*sampledraft>
%\fi
%    \begin{macrocode}
\def\version{draft}
\input{childdoc.def}
\childdocforward{cdocsamp}
%    \end{macrocode}

%\iffalse
%</sampledraft>
%\fi
%
% %%%%%%%%%%%%%%%%%%%%%%%%%%%%%%%%%%%%%%
% \paragraph{Forwarding for Final Version of the Chapters.}
%
% The following forwarding files |cdocsfn1.tex| and |cdocsfn2.tex|
% (with identical content)
% compile the final versions of the child documents
% |cdocsch1.tex| and |cdocsch2.tex|, respectively:
%\iffalse
%<*samplefinal>
%\fi
%    \begin{macrocode}
\def\version{final}
\input{childdoc.def}
\childdocforwardprefix[cdocsamp]{cdocsfn}{cdocsch}
%    \end{macrocode}

%\iffalse
%</samplefinal>
%\fi
%
% %%%%%%%%%%%%%%%%%%%%%%%%%%%%%%%%%%%%%%
% \paragraph{Command Line Processing.}
%
% The following three command lines generate the output files
% |cdocscld|, |cdocscl1| and |cdocscl2|
% which should be identical to
% |cdocsdrf|, |cdocsch1| and |cdocsfn2|, respectively:
% \begin{center}
% \begin{tabular}{l}
% |latex -jobname cdocscld \|\\
% |  "\def\version{draft}\input{childdoc.def}\childdocforward{cdocsamp}"|\\
% |latex -jobname cdocscl1 \|\\
% |  "\input{childdoc.def}\childdocforward[cdocsamp]{cdocsch1}"|\\
% |latex -jobname cdocscl2 \|\\
% |  "\def\version{final}\input{childdoc.def}\childdocforward{cdocsch2}"|
% \end{tabular}
% \end{center}
% Note that the trailing backslash on each first line
% merely continues the input to the second line
% (for convenient cut ant paste).
% Furthermore, the command |latex| can be replaced by any
% of its alternative versions such as |pdflatex|.
%
% %%%%%%%%%%%%%%%%%%%%%%%%%%%%%%%%%%%%%%%%%%%%%%%%%%%%%%%%%%%%%%%%%%%%%%%%%%%%%%
% %%%%%%%%%%%%%%%%%%%%%%%%%%%%%%%%%%%%%%%%%%%%%%%%%%%%%%%%%%%%%%%%%%%%%%%%%%%%%%
% \section{Implementation}
%\iffalse
%<*package>
%\fi
%
% This section describes the definitions file |childdoc.def|.

% The definitions cannot be loaded using |\usepackage| or |\RequirePackage|
% which has a mechanism to prevent loading a style file more than once.
% When loading the definitions by means of |\input|
% multiple instances have to be prevented manually:
%\iffalse
%This code needs to be before the `\ProvidesFile' directive
%which is defined at the beginning of this file.
%Therefore it is also placed there and commented out here.
%</package>
%<*discard>
%\fi
%    \begin{macrocode}
\ifdefined\childdocmain\endinput\fi
%    \end{macrocode}
%\iffalse
%</discard>
%<*package>
%\fi
%
% \macro{\ifchilddoc}
% \macro{\ifchilddocmanual}
% The conditional |\ifchilddoc| tells whether a
% child (true) or main (false) document is being compiled.
% The conditional |\ifchilddocmanual| tells whether
% the |\includeonly| mechanism is used (false) or
% the selection of child files must be performed manually (true).
% The definitions initialise to false:
%    \begin{macrocode}
\newif\ifchilddoc
\newif\ifchilddocmanual
%    \end{macrocode}

% \macro{\childdocname}
% \macro{\childdocjob}
% The macro |\childdocname| stores the name of the main document
% to be compiled. The macro |\childdocjob| stores the name of
% the document on which the \LaTeX{} compiler was originally invoked.
% The content of |\jobname| cannot be compared
% to filenames specified in the source due to different catcodes.
% The following code rescans |\jobname|, stores the result
% in |\childdocname| and saves a copy in |\childdocjob|:
%    \begin{macrocode}
\edef\childdocname{\scantokens\expandafter{\jobname\noexpand}}
\let\childdocjob\childdocname
%    \end{macrocode}

% \macro{\childdocdisable}
% The macro |\childdocdisable| prevents the main file
% from being processed more than once.
% At this stage, the main document command |\childdocmain|
% is assumed to be called once again where it should do nothing.
% Any subsequent call to it should prevent
% a secondary processing of the main document
% It overwrites the forwarding commands
% |\childdocof| and |\childdocforward|
% with empty macros to prevent further inclusions of the main document:
%    \begin{macrocode}
\newcommand{\childdocdisable}
{
  \renewcommand{\childdocmain}[1]{\renewcommand{\childdocmain}[1]{\endinput}}
  \renewcommand{\childdocof}[1]{}
  \renewcommand{\childdocby}[2][]{}
  \renewcommand{\childdocforward}[2][]{}
  \renewcommand{\childdocdisable}{}
}
%    \end{macrocode}

% \macro{\childdocmain}
% The macro |\childdocmain| is to be called at the top of the main file
% with nothing or the main filename (without extension) as argument.
% First, it breaks loops.
% If the argument is not empty and does not match |\childdocname|
% (which is set by the first inclusion of |childdoc.def|),
% |\ifchilddoc| is set to true, |\includeonly| is applied to the child file
% and |\jobname| is set to the main file
% (for proper handling of |.aux| files):
%    \begin{macrocode}
\newcommand{\childdocmain}[1]
{
  \childdocdisable\childdocmain{}
  \if?#1?\else
    \begingroup
      \def\childdoctmp{#1}
      \ifx\childdoctmp\childdocname
        \def\childdoctmp{}
      \else
        \def\childdoctmp
        {
          \childdoctrue
          \includeonly{\childdocname}
          \def\childdocjob{#1}
          \def\jobname{#1}
        }
      \fi
      \expandafter
    \endgroup
    \childdoctmp
  \fi
}
%    \end{macrocode}

% \macro{\childdocof}
% The command |\childdocof| redirects
% compilation to the main file |#1|.
%    \begin{macrocode}
\newcommand{\childdocof}[1]
{
  \childdocdisable
  \childdoctrue
  \includeonly{\childdocname}
  \def\jobname{#1}
  \def\childdocjob{#1}
  \input{#1}
}
%    \end{macrocode}

% \macro{\childdocby}
% The command |\childdocby| ....
%    \begin{macrocode}
\newcommand{\childdocby}[2][]
{
  \childdocdisable
  \childdoctrue
  \childdocmanualtrue
  \if?#1?\else
    \def\jobname{#2}
  \fi
  \def\childdocjob{#2}
  \input{#2}
  \endinput
}
%    \end{macrocode}

% \macro{\childdocforward}
% The command |\childdocforward| redirects
% compilation to the main file or
% (if the optional argument is given) a child file.
% Parameters are set as if the main file
% or a child file starting with |\childdocof| was compiled.
% Then compilation is handed over to the main file:
%    \begin{macrocode}
\newcommand{\childdocforward}[2][]
{
  \begingroup
    \if?#1?
      \def\childdoctmp
      {
        \def\childdocname{#2}
        \def\childdocjob{#2}
        \def\jobname{#2}
        \input{#2}
        \endinput
      }
    \else
      \def\childdoctmp
      {
        \childdocdisable
        \def\childdocname{#2}
        \childdoctrue
        \includeonly{#2}
        \def\childdocjob{#1}
        \def\jobname{#1}
        \input{#1}
        \endinput
      }
    \fi
    \expandafter
  \endgroup
  \childdoctmp
}
%    \end{macrocode}

% \macro{\childdocforwardprefix}
% The command |\childdocforwardprefix| redirects
% compilation to the main or a child file by means of a pattern.
% The prefix |#1| in the current filename is replaced by |#2|
% and the suffix of the current filename is kept
% (it is assumed that the filename does not contain the substring `|~~~|'
% which is used as a delimiter).
% Compilation is handed over to the new file by |\childdocforward|:
%    \begin{macrocode}
\newcommand{\childdocforwardprefix}[3][]
{
  \begingroup
    \def\childdocextract #2##1~~~{\def\childdoctmp{\childdocforward[#1]{#3##1}}}
    \expandafter\childdocextract\childdocname~~~
    \expandafter
  \endgroup
  \childdoctmp
}
%    \end{macrocode}

% \macro{\childdoc}
% The deprecated macro |\childdoc| is a legacy version of |\childdocmain|:
%    \begin{macrocode}
\newcommand{\childdoc}{\childdocmain}
%    \end{macrocode}

% \macro{\childdocredirect}
% The deprecated macro |\childdocredirect| is a legacy version
% of |\childdocforward| and |\childdocforwardprefix|:
%    \begin{macrocode}
\newcommand{\childdocredirect}[2][]
{
  \begingroup
    \if?#1?
      \def\childdoctmp{\childdocforward{#2}}
    \else
      \def\childdoctmp{\childdocforwardprefix{#1}{#2}}
    \fi
    \expandafter
  \endgroup
  \childdoctmp
}
%    \end{macrocode}

%\iffalse
%</package>
%\fi
%
\endinput

\childdocof{cdocsamp}
%    \end{macrocode}

%\iffalse
%</samplechap1|samplechap2>
%\fi
%
%\iffalse
%<*samplechap1>
%\fi
% Some text for chapter 1:
%    \begin{macrocode}
\section{one}
some text in chapter one
%    \end{macrocode}

%\iffalse
%</samplechap1>
%\fi
% Some text for chapter 2:
%\iffalse
%<*samplechap2>
%\fi
%    \begin{macrocode}
\section{two}
more text in chapter two
%    \end{macrocode}

%\iffalse
%</samplechap2>
%\fi
%
% %%%%%%%%%%%%%%%%%%%%%%%%%%%%%%%%%%%%%%
% \paragraph{Part Include Files.}
%
% The include files are called |cdocspt3.tex| and |cdocspt4.tex|.
%
%\iffalse
%<*samplepart3|samplepart4>
%\fi

% Optional override for |\version| flag:
%    \begin{macrocode}
%%\providecommand{\version}{final}
%    \end{macrocode}

% Include the main document:
%    \begin{macrocode}
% \iffalse
%
% childdoc.dtx Copyright (C) 2017-2018 Niklas Beisert
%
% This work may be distributed and/or modified under the
% conditions of the LaTeX Project Public License, either version 1.3
% of this license or (at your option) any later version.
% The latest version of this license is in
%   http://www.latex-project.org/lppl.txt
% and version 1.3 or later is part of all distributions of LaTeX
% version 2005/12/01 or later.
%
% This work has the LPPL maintenance status `maintained'.
%
% The Current Maintainer of this work is Niklas Beisert.
%
% This work consists of the files childdoc.dtx and childdoc.ins
% and the derived files childdoc.def and cdocsamp.tex with
% cdocsch1.tex, cdocsch2.tex, cdocsdrf.tex, cdocsfn1.tex, cdocsfn2.tex.
%
%<package>\ifdefined\childdocmain\endinput\fi
%<package>\ProvidesFile{childdoc.def}[2018/12/30 v2.0 child document driver]
%<samplemain>\ProvidesFile{cdocsamp.tex}[2018/12/30 v2.0 sample for childdoc]
%<*driver>
%\ProvidesFile{childdoc.drv}[2018/12/30 v2.0 childdoc reference manual file]
\PassOptionsToClass{10pt,a4paper}{article}
\documentclass{ltxdoc}

\usepackage[margin=35mm]{geometry}
\usepackage{hyperref}
\usepackage{hyperxmp}
\usepackage[usenames]{color}

\hypersetup{colorlinks=true}
\hypersetup{pdfstartview=FitH}
\hypersetup{pdfpagemode=UseNone}
\hypersetup{pdfsource={}}
\hypersetup{pdflang={en-UK}}
\hypersetup{pdfcopyright={Copyright 2017-2018 Niklas Beisert.
  This work may be distributed and/or modified under the
  conditions of the LaTeX Project Public License, either version 1.3
  of this license or (at your option) any later version.}}
\hypersetup{pdflicenseurl={http://www.latex-project.org/lppl.txt}}
\hypersetup{pdfcontactaddress={ETH Zurich, ITP, HIT K,
  Wolfgang-Pauli-Strasse 27}}
\hypersetup{pdfcontactpostcode={8093}}
\hypersetup{pdfcontactcity={Zurich}}
\hypersetup{pdfcontactcountry={Switzerland}}
\hypersetup{pdfcontactemail={nbeisert@itp.phys.ethz.ch}}
\hypersetup{pdfcontacturl={http://people.phys.ethz.ch/\xmptilde nbeisert/}}

\newcommand{\secref}[1]{\hyperref[#1]{section \ref*{#1}}}

\parskip1ex
\parindent0pt
\let\olditemize\itemize
\def\itemize{\olditemize\parskip0pt}

\begin{document}

\title{The \textsf{childdoc} Package}
\hypersetup{pdftitle={The childdoc Package}}
\author{Niklas Beisert\\[2ex]
  Institut f\"ur Theoretische Physik\\
  Eidgen\"ossische Technische Hochschule Z\"urich\\
  Wolfgang-Pauli-Strasse 27, 8093 Z\"urich, Switzerland\\[1ex]
  \href{mailto:nbeisert@itp.phys.ethz.ch}
  {\texttt{nbeisert@itp.phys.ethz.ch}}}
\hypersetup{pdfauthor={Niklas Beisert}}
\hypersetup{pdfsubject={Manual for the LaTeX2e Package childdoc}}
\date{30 December 2018, \textsf{v2.0}}
\maketitle

\begin{abstract}\noindent
\textsf{childdoc} is a \LaTeXe{} package
that enables the direct compilation
of document sections included by |\include|
to individual files.
\end{abstract}

\begingroup
\parskip0ex
\tableofcontents
\endgroup

%%%%%%%%%%%%%%%%%%%%%%%%%%%%%%%%%%%%%%%%%%%%%%%%%%%%%%%%%%%%%%%%%%%%%%%%%%%%%%%%
%%%%%%%%%%%%%%%%%%%%%%%%%%%%%%%%%%%%%%%%%%%%%%%%%%%%%%%%%%%%%%%%%%%%%%%%%%%%%%%%
\section{Introduction}

\LaTeX{} provides a mechanism to structure a large document (such as a book)
into a main file and several child files (containing the chapters)
using the |\include| command.
This mechanism is beneficial for documents
which span hundreds of pages in order to
make the source file(s) more manageable.
Moreover, compilation can be restricted to
selected child files by means of the |\includeonly| command.
The latter feature can be used to reduce the compilation time while editing
(this was significantly more useful in the earlier days of \LaTeX{})
or to generate a smaller document which is easier to navigate.
Another application of |\includeonly| is to generate
documents consisting of selected parts of the complete document.

However, there are a few drawbacks of the plain |\include| mechanism:
\begin{itemize}
\item
The child files cannot be compiled on their own,
they can only be compiled via the main file.
A naive editing environment
(such as a text editor with an option
to have the current file processed by \LaTeX)
may require one to switch to the main file before compiling;
attempting to compile the child file produces errors.
\item
The main file must be modified (each time)
to adjust the |\includeonly| command
to the present needs. This easily leaves the main file in a messy state.
\item
The generated document will always carry the filename
of the main document. This is inconvenient if
several child files are to be compiled and
to be kept for distribution.
\end{itemize}

The present package provides a simple interface
to make child files individually compilable by \LaTeX{}.
Compiling a child file then has the same effect as compiling
the main file with an |\includeonly| command
to select the appropriate child.
Moreover the generated document will carry the name of the child
rather than the main file.
This resolves all three above issues.

This feature is meant to make the editing of books,
thesis documents and lecture notes somewhat more convenient.
However, the package can also be used efficiently for
composing a series of documents (such as exercise sheets)
which are typically distributed individually.
It then assists the author in generating the individual documents
(potentially in different versions)
as well as a document containing the collected series.
Another application is in developing style files
or other kinds of included material
where compilation of the style file could redirect
to a sample or test file.

%%%%%%%%%%%%%%%%%%%%%%%%%%%%%%%%%%%%%%%%%%%%%%%%%%%%%%%%%%%%%%%%%%%%%%%%%%%%%%%%
%%%%%%%%%%%%%%%%%%%%%%%%%%%%%%%%%%%%%%%%%%%%%%%%%%%%%%%%%%%%%%%%%%%%%%%%%%%%%%%%
\section{Usage}

First of all, the package \textsf{childdoc} is \emph{not} a standard
\LaTeXe{} |.sty| style file! Therefore it needs to be invoked in
a non-standard way.

%%%%%%%%%%%%%%%%%%%%%%%%%%%%%%%%%%%%%%%%%%%%%%%%%%%%%%%%%%%%%%%%%%%%%%%%%%%%%%%%
\subsection{Included Files}
\label{sec:include}

%%%%%%%%%%%%%%%%%%%%%%%%%%%%%%%%%%%%%%%%
\DescribeMacro{\childdocmain}
To use the package, add the commands
\begin{center}
\begin{tabular}{l}
|\input{childdoc.def}|\\
|\childdocmain{}|\\
\end{tabular}
\end{center}
at the very top of the main \LaTeX{} file,
in particular \emph{before} the |\documentclass| statement!
The argument of |\childdocmain| should be left empty
(but it must be present).

%%%%%%%%%%%%%%%%%%%%%%%%%%%%%%%%%%%%%%%%
\DescribeMacro{\childdocof}
Furthermore, add the commands
\begin{center}
\begin{tabular}{l}
|\input{childdoc.def}|\\
|\childdocof{|\textit{main}|}|\\
\end{tabular}
\end{center}
at the top of every child file \textit{child}
which is included by |\include{|\textit{child}|}|
from within the main file
(or at least for those files to be compiled individually).
The argument \textit{main} must be the filename of the main file.

There are a couple of
considerations in setting up the main and child documents:

%%%%%%%%%%%%%%%%%%%%%%%%%%%%%%%%%%%%%%%%
\paragraph{Restrictions.}

Please note the following restrictions:
\begin{itemize}
\item
|\childdocmain| must be called with one argument \textit{main}
to ensure compatibility with earlier version of the package.
It must either be empty (|\childdocmain{}|)
or precisely match the filename of the main file in which it is specified.
See \secref{sec:detection} for further information.
\item
The filename \textit{main} must be specified without the |.tex| extension.
\item
The filename \textit{main} is case sensitive
(even in case-insensitive file systems)
due to internal string comparison.
\item
The argument \textit{main} should be fully expanded, it cannot be a macro.
\item
Subdirectories and special characters should be avoided in filenames.
\item
The command |\childdocmain{|\textit{main}|}| must be followed by a whitespace.
It should not be followed immediately by another command
or by a comment mark `|%|'.
This is because the \TeX{} parser reads the token immediately following
the argument of |\childdocmain| and puts it
at the beginning of every child section;
however, a white\-space is ignored.
\end{itemize}

%%%%%%%%%%%%%%%%%%%%%%%%%%%%%%%%%%%%%%%%
\paragraph{Content of Main File.}

It is advisable to place all content in the child files included by |\include|.
Any output contained in the main file will appear in all child documents
unless suppressed manually;
it cannot be suppressed automatically by the |\includeonly| directive
and thus should normally be avoided.
A method to include some content in the main file
by means of conditional processing is described in \secref{sec:conditional}.

%%%%%%%%%%%%%%%%%%%%%%%%%%%%%%%%%%%%%%%%
\paragraph{Page Numbering.}

When only a part of the document is compiled,
the appropriate numbering of pages
(as well as other status parameters)
is determined from the |.aux| files.
The latter contain information from previous passes.
However this information needs to propagate through
all intermediate child documents.
Therefore the page numbering in child documents may well
be inconsistent until the complete document is compiled at least once.

A useful (if unconventional) way to always ensure a consistent
page numbering is to restart the numbering in each child document
and denote the pages by `\textit{child}|.|\textit{page}'
where \textit{child} represents the chapter/section number of the child file.
This can be achieved by the command
|\numberwithin{page}{|\textit{child}|}|
of the \textsf{amsmath} package
where \textit{child} can be |chapter| or |section|
depending on the chosen structuring.
Alternatively, one can modify the macro |\thepage| appropriately
and reset the counter |page| at the start of each child file.

%%%%%%%%%%%%%%%%%%%%%%%%%%%%%%%%%%%%%%%%%%%%%%%%%%%%%%%%%%%%%%%%%%%%%%%%%%%%%%%%
\subsection{Conditional Processing}
\label{sec:conditional}

The package provides a mechanism to compile different versions
of a document. To customise the versions further some conditional processing
can come in handy to distinguish which version is being compiled.
The package provides two macros to describe the compilation context:

%%%%%%%%%%%%%%%%%%%%%%%%%%%%%%%%%%%%%%%%
\DescribeMacro{\ifchilddoc}
The conditional |\ifchilddoc| distinguishes between the compilation of
child documents and the main document:
%
\begin{center}
|\ifchilddoc |\textit{child-code}| |[|\||else |\textit{main-code}]| \||fi|
\end{center}

%%%%%%%%%%%%%%%%%%%%%%%%%%%%%%%%%%%%%%%%
\DescribeMacro{\childdocname}
\DescribeMacro{\childdocjob}
The macro |\childdocname| contains the filename (without extension)
of the main or child file being processed.
Note that |\childdocjob| will always contain the name of the main file.

%%%%%%%%%%%%%%%%%%%%%%%%%%%%%%%%%%%%%%%%
\paragraph{Title Page.}

Conditional processing can be used to include a title or banner page
in the main document when proper precautions are taken.
Importantly, the code in the main file should ensure that the page counter
(as well as other status parameters which are stored in the |.aux| files)
takes the same value after the conditional processing.
Otherwise the page numbers may take divergent values
depending on which part is compiled.

For example, a title page could be declared by:
%
\begin{center}
\begin{tabular}{l}
|\ifchilddoc\||else|\\
|\addtocounter{page}{-1}|\\
\textit{code for title page}\\
|\newpage|\\
|\||fi|
\end{tabular}
\end{center}
%
A banner page for the child documents can be generated by:
%
\begin{center}
\begin{tabular}{l}
|\ifchilddoc|\\
|\addtocounter{page}{-1}|\\
\textit{code for banner page}\\
|\newpage|\\
|\||fi|
\end{tabular}
\end{center}
%
Here one could write a message such as:
\begin{center}
|This is the part \childdocname{} of \childdocjob{}.|
\end{center}

%%%%%%%%%%%%%%%%%%%%%%%%%%%%%%%%%%%%%%%%%%%%%%%%%%%%%%%%%%%%%%%%%%%%%%%%%%%%%%%%
\subsection{Flags}
\label{sec:flags}

The package makes it easy to generate different versions
of the main or child documents.
To this end compilation flags can be defined
and assigned different default values.
They will be particularly useful in conjunction
with the forwarding mechanism described in \secref{sec:forward}.

For example, it may be useful to have a flag |\version|
which can be set to |draft| or |final|.
The document source will contain some conditional code
depending on the value of |\version|.
Suppose further, the flag should default to |final| for the main file
and to |draft| for child files
which is a natural assignment for editing the document.
This is achieved by placing the following code
in the preamble of the main document
(below the |\childdocmain| directive):
%
\begin{center}
\begin{tabular}{l}
|\ifchilddoc|\\
|\providecommand{\version}{draft}|\\
|\||else|\\
|\providecommand{\version}{final}|\\
|\||fi|
\end{tabular}
\end{center}
%
The definition by |\providecommand| makes sure
that previous definitions are not overwritten.
Further statements |\providecommand{\version}{...}|
can thus be added before the above code to override it.

For the main file, one might add a line
(between |\childdocmain| and the above block)
%
\begin{center}
|%\ifchilddoc\||else\providecommand{\version}{draft}\||fi|
\end{center}
%
which can be uncommented to produce a draft version.
Likewise one can add a line to the very top of a child file
(above the |\childdocof{|\textit{main}|}| directive)
%
\begin{center}
|%\providecommand{\version}{final}|
\end{center}
%
which can be uncommented to produce the final version of this child document.

%%%%%%%%%%%%%%%%%%%%%%%%%%%%%%%%%%%%%%%%%%%%%%%%%%%%%%%%%%%%%%%%%%%%%%%%%%%%%%%%
\subsection{Forwarding}
\label{sec:forward}

Different versions of the main or child documents
using compilation flags as described in \secref{sec:flags}
can be (permanently) stored in different files
for convenient compilation, viewing and distribution.
To this end, the package defines a command
to pass on compilation to a different file:

%%%%%%%%%%%%%%%%%%%%%%%%%%%%%%%%%%%%%%%%
\DescribeMacro{\childdocforward}
The command |\childdocforward| redirects processing to
another source file:
%
\begin{center}
\begin{tabular}{l}
|\input{childdoc.def}|\\
|\childdocforward[|\textit{main}|]{|\textit{dest}|}|\\
\end{tabular}
\end{center}
%
The argument \textit{dest} is the destination file
(without extension).
It should be the main file or one of the child files.
Note that further \textsf{childdoc} directives
such as |\childdocof| and |\childdocforward|
in the indicated file will be processed in this form.
The optional argument \textit{main}
passes on directly to the main file \textit{main}
while pretending to compile the child \textit{dest}.
This form behaves as if \textit{dest}
issues |\childdocof{|\textit{main}|}| right away,
and no further \textsf{childdoc} directives will be processed.

%%%%%%%%%%%%%%%%%%%%%%%%%%%%%%%%%%%%%%%%
\DescribeMacro{\...prefix}
In the alternative form |\childdocforwardprefix|,
%
\begin{center}
\begin{tabular}{l}
|\input{childdoc.def}|\\
|\childdocforwardprefix[|\textit{main}|]{|\textit{prefix}|}{|\textit{dest}|}|
\end{tabular}
\end{center}
%
the destination file is determined by a pattern
depending on the current file:
To make this work, the current file must be called
`{\textit{prefix}\hspace{0.2em}\textit{suffix}}'
with \textit{prefix} matching precisely the argument.
Processing is then passed on to the file
`{\textit{dest}\hspace{0.2em}\textit{suffix}}'.
Surely, the same effect is achieved by
directly specifying the
argument `{\textit{dest}\hspace{0.2em}\textit{suffix}}'
in the first form.
However, that requires to set up a different file
for each child. With the alternative form of the command
all these files can have exactly the same content
which simplifies setting them up and maintaining them.

For example, the following file |draft.tex|
with a compilation flag |\version| as described in \secref{sec:flags}
compiles the main document as a draft:
%
\begin{center}
\begin{tabular}{l}
|\def\version{draft}|\\
|\input{childdoc.def}|\\
|\childdocforward{|\textit{main}|}|
\end{tabular}
\end{center}
%
Likewise, the following files |final|\textit{nn}|.tex|
compile the final version of the child document
|child|\textit{nn}|.tex|:
%
\begin{center}
\begin{tabular}{l}
|\def\version{final}|\\
|\input{childdoc.def}|\\
|\childdocforwardprefix{final}{child}|
\end{tabular}
\end{center}
%

Note that when several versions of a main file and/or of each child file
are to be generated, it may be convenient to set up a |Makefile| or
shell script to automatise the process.

%%%%%%%%%%%%%%%%%%%%%%%%%%%%%%%%%%%%%%%%%%%%%%%%%%%%%%%%%%%%%%%%%%%%%%%%%%%%%%%%
\subsection{Command Line Processing}
\label{sec:commandline}

The effect of redirection files can also be achieved by invoking
the \LaTeX{} compiler with a more elaborate command line.
Most conveniently this should be done as part
of a shell script or a |Makefile|.

When using \textsf{childdoc} in the main file, the following
command lines effectively perform a redirection
(note that depending on the shell being used,
backslashes may have to be doubled: `|\|' $\to$ `|\\|'):
%
\begin{center}
|... -jobname "|\textit{target}|" |\\|"|[\textit{flags}]%
|\input{childdoc.def}\childdocforward[|\textit{main}|]{|\textit{dest}|}"|
\end{center}
%
Here \textit{target} is the name of the output file,
\textit{main} is the name of the main file
and \textit{dest} is the name of the main or child file to be processed
(all filenames without extensions).
The optional argument \textit{main} can be omitted
if \textit{main} matches \textit{dest}.
Optionally, compilation \textit{flags} can be defined via |\def| commands.
This command line makes the \TeX{} engine believe
it is compiling the file \textit{target}
whose content is specified as the latter parameter.
The provided code then forwards the processing to
\textit{main} or \textit{dest} as described in \secref{sec:forward}.

%%%%%%%%%%%%%%%%%%%%%%%%%%%%%%%%%%%%%%%%%%%%%%%%%%%%%%%%%%%%%%%%%%%%%%%%%%%%%%%%
\subsection{Include by Input}
\label{sec:input}

Including child documents by |\include| has some restrictions by design.
Most notably, the content of a child document always occupies
its own set of pages; pages cannot be shared between child documents.
Usually, this behaviour makes perfect sense
because each child document contain an essential part of the document.
However, in some situations it may be desirable to compose
a document from a collection of parts
without having mandatory page breaks between then.
For this case, the package
provides a mechanism to include parts
by |\input| which can also be processed individually.
However, by construction this mechanism
requires manual handling of the content to be output.

%%%%%%%%%%%%%%%%%%%%%%%%%%%%%%%%%%%%%%%%
\DescribeMacro{\ifchilddocmanual}
The main file should be prepared as usual, see \secref{sec:include}.
However, the document body must make a distinction
between processing of an individual part and of the main document, e.g.:
%
\begin{center}
\begin{tabular}{l}
|\ifchilddocmanual|\\
|\input{\childdocname}|\\
|\||else|\\
\textit{document body with }|\input{|\textit{part}|}|\\
|\||fi|
\end{tabular}
\end{center}
%
The conditional |\ifchilddocmanual| is true whenever
a part to be included by |\input| is being compiled,
and the name of the part is stored in |\childdocname|.

%%%%%%%%%%%%%%%%%%%%%%%%%%%%%%%%%%%%%%%%
\DescribeMacro{\childdocby}
Each part to be included by |\input| should start with:
%
\begin{center}
\begin{tabular}{l}
|\input{childdoc.def}|\\
|\childdocby{|\textit{main}|}|\\
\end{tabular}
\end{center}
%
The directive |\childdocby| is similar to |\childdocof|
described in \secref{sec:include},
but the subsequent selection of content must be done manually.
To that end, both |\ifchilddoc| and |\ifchilddocmanual|
will be true upon processing of a part,
and the name of the part is stored in |\childdocname|.
Note that |\jobname| will be set to the filename of the current part
so that each part receives an individual |.aux| file
that does not interfere with the |.aux| file(s) of the main document.
This behaviour can be altered by the alternative form
|\childdocby[*]{|\textit{main}|}| (with a non-empty optional argument)
which uses the |.aux| file of the main document
by setting |\jobname| to \textit{main}.

%%%%%%%%%%%%%%%%%%%%%%%%%%%%%%%%%%%%%%%%%%%%%%%%%%%%%%%%%%%%%%%%%%%%%%%%%%%%%%%%
\subsection{Driver Development}
\label{sec:driver}

The \textsf{childdoc} mechanism can also be use for the development
of definition files such as \LaTeX{} styles or classes.
This case differs from the above setup with multiple parts
included by |\include| in that no |\includeonly| should be invoked.
This can be achieved by starting the include file
(before |\ProvidesPackage|) with:
%
\begin{center}
\begin{tabular}{l}
|\input{childdoc.def}|\\
|\childdocforward{|\textit{main}|}|\\
\end{tabular}
\end{center}
%
or alternatively with:
%
\begin{center}
\begin{tabular}{l}
|\input{childdoc.def}|\\
|\childdocby{|\textit{main}|}|\\
\end{tabular}
\end{center}
%
Both forms have slightly different effects as described above.
The main file is prepared as usual, see \secref{sec:include}.

%%%%%%%%%%%%%%%%%%%%%%%%%%%%%%%%%%%%%%%%%%%%%%%%%%%%%%%%%%%%%%%%%%%%%%%%%%%%%%%%
\subsection{Legacy Detection}
\label{sec:detection}

The directive |\childdocmain| in the main file can detect
whether the complete document or merely a child is to be compiled
even without using the directive |\childdocof|.
This method is deprecated because it is less robust
and there is no compelling reason to use it;
it is merely provided for backward compatibility
and it may be removed in future versions.

If the detection mechanism is to be used,
it is mandatory to correctly specify
the filename of the main file as the argument of |\childdocmain|:
%
\begin{center}
\begin{tabular}{l}
|\input{childdoc.def}|\\
|\childdocmain{|\textit{main}|}|\\
\end{tabular}
\end{center}
%
If |\jobname| does not match the argument \textit{main} of |\childdocmain|,
it is assumed that |\jobname| points to the child file to be compiled.
When using |\childdocmain| with the main file specified as argument,
it suffices to start a child file
with just |\input{|\textit{main}|}|
without loading of the package and using |\childdocof|.
If instead all processing is done
with the appropriate \textsf{childdoc} directives,
the argument of \textit{main} of |\childdocmain| can be empty.

An alternative version of the command line processing described
in \secref{sec:commandline} using the detection mechanism reads:
%
\begin{center}
|... -jobname "|\textit{target}|" "|[\textit{flags}]%
[|\def\jobname{|\textit{dest}|}|]|\input{|\textit{main}|}"|
\end{center}

%%%%%%%%%%%%%%%%%%%%%%%%%%%%%%%%%%%%%%%%%%%%%%%%%%%%%%%%%%%%%%%%%%%%%%%%%%%%%%%%
\subsection{Manual Code}
\label{sec:manual}

In case one cannot be certain whether the definitions file |childdoc.def|
is installed on the target \TeX{} distribution
and one prefers not to ship it,
it is conceivable to paste a few relevant commands into the sources.

To that end, drop all statements |\input{childdoc.def}|
and perform the replacements as outlined below.
Instead of |\childdocmain{|\textit{main}|}| add the following code
to the top of the main file:
%
\begin{center}
\begin{tabular}{l}
|\||ifdefined\childdocname\endinput\||fi\newif\ifchilddoc|\\
|\edef\childdocname{\scantokens\expandafter{\jobname\noexpand}}|\\
|\def\childdocmain{|\textit{main}|}\||ifx\childdocmain\childdocname\||else|\\
|\childdoctrue\includeonly{\childdocname}\let\jobname\childdocmain\||fi|\\
\end{tabular}
\end{center}
%
Instead of |\childdocof{|\textit{main}|}| just include the main file
at the top of each child file:
%
\begin{center}
|\input{|\textit{main}|}|
\end{center}
%
A simple redirection |\childdocforward{|\textit{dest}|}| is achieved by:
%
\begin{center}
|\def\jobname{|\textit{dest}|}\input{\jobname}|
\end{center}
%
The redirection with prefix
|\childdocforwardprefix[|\textit{prefix}|]{|\textit{dest}|}|
is accomplished by:
%
\begin{center}
\begin{tabular}{l}
|{\edef\jobname{\scantokens\expandafter{\jobname\noexpand}}|\\
|\def\redirectjob |\textit{prefix}|#1~~~{\gdef\jobname{|\textit{dest}|#1}}|\\
|\expandafter\redirectjob\jobname~~~}\input{\jobname}|
\end{tabular}
\end{center}

In an alternative approach,
child documents can be compiled by a specific command line
without additional code or specific definitions:
%
\begin{center}
|... -jobname "|\textit{target}|" "|[\textit{flags}]%
|\includeonly{|\textit{dest}|}\input{|\textit{main}|}"|
\end{center}
%

%%%%%%%%%%%%%%%%%%%%%%%%%%%%%%%%%%%%%%%%%%%%%%%%%%%%%%%%%%%%%%%%%%%%%%%%%%%%%%%%
%%%%%%%%%%%%%%%%%%%%%%%%%%%%%%%%%%%%%%%%%%%%%%%%%%%%%%%%%%%%%%%%%%%%%%%%%%%%%%%%
\section{Information}

%%%%%%%%%%%%%%%%%%%%%%%%%%%%%%%%%%%%%%%%%%%%%%%%%%%%%%%%%%%%%%%%%%%%%%%%%%%%%%%%
\subsection{Copyright}

Copyright \copyright{} 2017--2018 Niklas Beisert

This work may be distributed and/or modified under the
conditions of the \LaTeX{} Project Public License, either version 1.3
of this license or (at your option) any later version.
The latest version of this license is in
  \url{http://www.latex-project.org/lppl.txt}
and version 1.3 or later is part of all distributions of \LaTeX{}
version 2005/12/01 or later.

This work has the LPPL maintenance status `maintained'.

The Current Maintainer of this work is Niklas Beisert.

This work consists of the files |README.txt|, |childdoc.ins| and |childdoc.dtx|
as well as the derived files |childdoc.def|, |cdocsamp.tex|
with |cdocsch1.tex|, |cdocsch2.tex|, |cdocspt3.tex|, |cdocspt4.tex|,
|cdocsdrf.tex|, |cdocsfn1.tex|, |cdocsfn2.tex|
as well as |childdoc.pdf|.

%%%%%%%%%%%%%%%%%%%%%%%%%%%%%%%%%%%%%%%%%%%%%%%%%%%%%%%%%%%%%%%%%%%%%%%%%%%%%%%%
\subsection{Files and Installation}

The package consists of the files:
%
\begin{center}
\begin{tabular}{ll}
    |README.txt|   & readme file \\
    |childdoc.ins| & installation file \\
    |childdoc.dtx| & source file \\
    |childdoc.def| & definition file \\
    |cdocsamp.tex| & sample main file \\
    |cdocsch1.tex| & sample include file \\
    |cdocsch2.tex| & sample include file \\
    |cdocspt3.tex| & sample part file \\
    |cdocspt4.tex| & sample part file \\
    |cdocsdrf.tex| & sample redirection file \\
    |cdocsfn1.tex| & sample redirection file \\
    |cdocsfn2.tex| & sample redirection file \\
    |childdoc.pdf| & manual
\end{tabular}
\end{center}
%
The distribution consists of the files
|README.txt|, |childdoc.ins| and |childdoc.dtx|.
%
\begin{itemize}
\item
Run (pdf)\LaTeX{} on |childdoc.dtx|
to compile the manual |childdoc.pdf| (this file).
\item
Run \LaTeX{} on |childdoc.ins| to create the definitions file |childdoc.def|
and the sample |cdocsamp.tex| with include files
|cdocsch1.tex|, |cdocsch2.tex|, |cdocspt3.tex|, |cdocspt4.tex|,
|cdocsdrf.tex|, |cdocsfn1.tex|, |cdocsfn2.tex|.
Then copy the file |childdoc.def| to an appropriate directory of your \LaTeX{}
distribution, e.g.\ \textit{texmf-root}|/tex/latex/childdoc|.
\end{itemize}

%%%%%%%%%%%%%%%%%%%%%%%%%%%%%%%%%%%%%%%%%%%%%%%%%%%%%%%%%%%%%%%%%%%%%%%%%%%%%%%%
\subsection{Related CTAN Packages}

There are several other packages which offer a similar functionality:
%
\begin{itemize}
\item
The packages
\href{http://ctan.org/pkg/docmute}{\textsf{docmute}},
\href{http://ctan.org/pkg/includex}{\textsf{includex}} and
\href{http://ctan.org/pkg/standalone}{\textsf{standalone}}
provide commands to include only the document body of
a child file thus allowing both files to be compiled individually.
\item
The packages \href{http://ctan.org/pkg/subdocs}{\textsf{subdocs}}
and \href{http://ctan.org/pkg/subfiles}{\textsf{subfiles}}
provide structures in which the main and child documents can be
encapsulated and allowing them to be compiled individually.
The inclusion mechanism is different from the conventional |\include|.
\item
The package \href{http://ctan.org/pkg/combine}{\textsf{combine}}
is an elaborate solution to combine several documents into one.
\end{itemize}
%
See also the CTAN topic \href{http://ctan.org/topic/subdocs}{\textsf{subdocs}}
for further related packages.
The present package differs from the above solutions in that
a document structure constructed with the conventional |\include| mechanism
just needs two extra commands at the top of every file
such that all constituent files can be compiled individually.

%%%%%%%%%%%%%%%%%%%%%%%%%%%%%%%%%%%%%%%%%%%%%%%%%%%%%%%%%%%%%%%%%%%%%%%%%%%%%%%%
%\subsection{Feature Suggestions}
%
%The following is a list of features which may be useful for future
%versions of this package:
%%
%\begin{itemize}
%\item
%\ldots
%\end{itemize}

%%%%%%%%%%%%%%%%%%%%%%%%%%%%%%%%%%%%%%%%%%%%%%%%%%%%%%%%%%%%%%%%%%%%%%%%%%%%%%%%
\subsection{Revision History}

%%%%%%%%%%%%%%%%%%%%%%%%%%%%%%%%%%%%%%%%
\paragraph{v2.0:} 2018/12/30

\begin{itemize}
\item
immediate forward processing
\item
added |\childdocby| mechanism
\item
manual restructured
\end{itemize}

%%%%%%%%%%%%%%%%%%%%%%%%%%%%%%%%%%%%%%%%
\paragraph{v1.6:} 2018/01/17

\begin{itemize}
\item
application for development of include files
\item
corrections to manual
\end{itemize}

%%%%%%%%%%%%%%%%%%%%%%%%%%%%%%%%%%%%%%%%
\paragraph{v1.5:} 2017/05/21

\begin{itemize}
\item
more complete structuring introduced
\item
|\childdocof| introduced
\item
|\childdoc| renamed to |\childdocmain|
\item
|\childredirect| renamed to |\childdocforward| and |\childdocforwardprefix|
and functionality expanded
\end{itemize}

%%%%%%%%%%%%%%%%%%%%%%%%%%%%%%%%%%%%%%%%
\paragraph{v1.0:} 2017/04/27

\begin{itemize}
\item
manual and install package
\item
first version published on CTAN
\end{itemize}

%%%%%%%%%%%%%%%%%%%%%%%%%%%%%%%%%%%%%%%%
\paragraph{v0.6:} 2017/04/26

\begin{itemize}
\item
redirection mechanism added
\end{itemize}

%%%%%%%%%%%%%%%%%%%%%%%%%%%%%%%%%%%%%%%%
\paragraph{v0.5:} 2017/04/26

\begin{itemize}
\item
functionality in definition file
\end{itemize}


%%%%%%%%%%%%%%%%%%%%%%%%%%%%%%%%%%%%%%%%%%%%%%%%%%%%%%%%%%%%%%%%%%%%%%%%%%%%%%%%
%%%%%%%%%%%%%%%%%%%%%%%%%%%%%%%%%%%%%%%%%%%%%%%%%%%%%%%%%%%%%%%%%%%%%%%%%%%%%%%%
%%%%%%%%%%%%%%%%%%%%%%%%%%%%%%%%%%%%%%%%%%%%%%%%%%%%%%%%%%%%%%%%%%%%%%%%%%%%%%%%
\appendix

\settowidth\MacroIndent{\rmfamily\scriptsize 000\ }

 \DocInput{childdoc.dtx}

\end{document}
%</driver>
% \fi
%
% %%%%%%%%%%%%%%%%%%%%%%%%%%%%%%%%%%%%%%%%%%%%%%%%%%%%%%%%%%%%%%%%%%%%%%%%%%%%%%
% %%%%%%%%%%%%%%%%%%%%%%%%%%%%%%%%%%%%%%%%%%%%%%%%%%%%%%%%%%%%%%%%%%%%%%%%%%%%%%
% \section{Sample}
%\iffalse
%<*samplemain>
%\fi
%
% The following presents a sample document
% with two chapters, two parts, a title page,
% a compile flag as well as three forwarding files to set the flag.
% It consists of eight |.tex| files:
% \begin{center}
% \begin{tabular}{ll}
% |cdocsamp.tex|&main file\\
% |cdocsch1.tex|&include file for chapter 1\\
% |cdocsch2.tex|&include file for chapter 2\\
% |cdocspt3.tex|&include file for part 3\\
% |cdocspt4.tex|&include file for part 4\\
% |cdocsdrf.tex|&forwarding file for main file in draft mode\\
% |cdocsfi1.tex|&forwarding file for final version of chapter 1\\
% |cdocsfi2.tex|&forwarding file for final version of chapter 2\\
% \end{tabular}
% \end{center}
% Each of the eight files can be compiled directly by the \LaTeX{} compiler.
%
% %%%%%%%%%%%%%%%%%%%%%%%%%%%%%%%%%%%%%%
% \paragraph{Main File.}
%
% The main file is called |cdocsamp.tex|.
%
% Load the \textsf{childdoc} definitions and
% declare the filename for the main document:
%    \begin{macrocode}
\input{childdoc.def}
\childdocmain{}
%    \end{macrocode}

% Optional override for |\version| flag:
%    \begin{macrocode}
%%\ifchilddoc\else\providecommand{\version}{draft}\fi
%    \end{macrocode}

% Define the default values for the |\version| flag
% (|final| for the main file and |draft| for childs):
%    \begin{macrocode}
\ifchilddoc
\providecommand{\version}{draft}
\else
\providecommand{\version}{final}
\fi
%    \end{macrocode}

% Load the standard document class:
%    \begin{macrocode}
\documentclass[12pt]{article}
%    \end{macrocode}

% Start the document body:
%    \begin{macrocode}
\begin{document}
%    \end{macrocode}

% Declare a title page.
% Print title, part of document being processed and version flag:
%    \begin{macrocode}
\addtocounter{page}{-1}
\begin{center}
{\LARGE\bfseries{}childdoc example\par}
\vspace{1cm}
\ifchilddoc
\ifchilddocmanual part\else chapter\fi:
`\childdocname' of `\childdocjob'\par
\else
main document: `\childdocjob'\par
\fi
version: \version\par
\end{center}
\newpage
%    \end{macrocode}

% Manually include selected file,
% otherwise process as usual:
%    \begin{macrocode}
\ifchilddocmanual
\section*{part `\childdocname'}
\input{\childdocname}
\else
%    \end{macrocode}

% Include the two chapters:
%    \begin{macrocode}
\include{cdocsch1}
\include{cdocsch2}
%    \end{macrocode}

% Include the two parts unless only chapters should be displayed:
%    \begin{macrocode}
\ifchilddoc\else
\section{part three}
\input{cdocspt3}
\section{part four}
\input{cdocspt4}
\fi
%    \end{macrocode}

% Process as usual until here:
%    \begin{macrocode}
\fi
%    \end{macrocode}

% End of document body:
%    \begin{macrocode}
\end{document}
%    \end{macrocode}
%\iffalse
%</samplemain>
%\fi
%
% %%%%%%%%%%%%%%%%%%%%%%%%%%%%%%%%%%%%%%
% \paragraph{Chapter Include Files.}
%
% The include files are called |cdocsch1.tex| and |cdocsch2.tex|.
%
%\iffalse
%<*samplechap1|samplechap2>
%\fi

% Optional override for |\version| flag:
%    \begin{macrocode}
%%\providecommand{\version}{final}
%    \end{macrocode}

% Include the main document:
%    \begin{macrocode}
\input{childdoc.def}
\childdocof{cdocsamp}
%    \end{macrocode}

%\iffalse
%</samplechap1|samplechap2>
%\fi
%
%\iffalse
%<*samplechap1>
%\fi
% Some text for chapter 1:
%    \begin{macrocode}
\section{one}
some text in chapter one
%    \end{macrocode}

%\iffalse
%</samplechap1>
%\fi
% Some text for chapter 2:
%\iffalse
%<*samplechap2>
%\fi
%    \begin{macrocode}
\section{two}
more text in chapter two
%    \end{macrocode}

%\iffalse
%</samplechap2>
%\fi
%
% %%%%%%%%%%%%%%%%%%%%%%%%%%%%%%%%%%%%%%
% \paragraph{Part Include Files.}
%
% The include files are called |cdocspt3.tex| and |cdocspt4.tex|.
%
%\iffalse
%<*samplepart3|samplepart4>
%\fi

% Optional override for |\version| flag:
%    \begin{macrocode}
%%\providecommand{\version}{final}
%    \end{macrocode}

% Include the main document:
%    \begin{macrocode}
\input{childdoc.def}
\childdocby{cdocsamp}
%    \end{macrocode}

%\iffalse
%</samplepart3|samplepart4>
%\fi
%
%\iffalse
%<*samplepart3>
%\fi
% Some text for part 3:
%    \begin{macrocode}
some text in part three
%    \end{macrocode}

%\iffalse
%</samplepart3>
%\fi
% Some text for part 4:
%\iffalse
%<*samplepart4>
%\fi
%    \begin{macrocode}
more text in part four
%    \end{macrocode}

%\iffalse
%</samplepart4>
%\fi
%
% %%%%%%%%%%%%%%%%%%%%%%%%%%%%%%%%%%%%%%
% \paragraph{Forwarding for a Complete Draft.}
%
% The following forwarding file |cdocsdrf.tex|
% compiles the main document in draft mode:
%\iffalse
%<*sampledraft>
%\fi
%    \begin{macrocode}
\def\version{draft}
\input{childdoc.def}
\childdocforward{cdocsamp}
%    \end{macrocode}

%\iffalse
%</sampledraft>
%\fi
%
% %%%%%%%%%%%%%%%%%%%%%%%%%%%%%%%%%%%%%%
% \paragraph{Forwarding for Final Version of the Chapters.}
%
% The following forwarding files |cdocsfn1.tex| and |cdocsfn2.tex|
% (with identical content)
% compile the final versions of the child documents
% |cdocsch1.tex| and |cdocsch2.tex|, respectively:
%\iffalse
%<*samplefinal>
%\fi
%    \begin{macrocode}
\def\version{final}
\input{childdoc.def}
\childdocforwardprefix[cdocsamp]{cdocsfn}{cdocsch}
%    \end{macrocode}

%\iffalse
%</samplefinal>
%\fi
%
% %%%%%%%%%%%%%%%%%%%%%%%%%%%%%%%%%%%%%%
% \paragraph{Command Line Processing.}
%
% The following three command lines generate the output files
% |cdocscld|, |cdocscl1| and |cdocscl2|
% which should be identical to
% |cdocsdrf|, |cdocsch1| and |cdocsfn2|, respectively:
% \begin{center}
% \begin{tabular}{l}
% |latex -jobname cdocscld \|\\
% |  "\def\version{draft}\input{childdoc.def}\childdocforward{cdocsamp}"|\\
% |latex -jobname cdocscl1 \|\\
% |  "\input{childdoc.def}\childdocforward[cdocsamp]{cdocsch1}"|\\
% |latex -jobname cdocscl2 \|\\
% |  "\def\version{final}\input{childdoc.def}\childdocforward{cdocsch2}"|
% \end{tabular}
% \end{center}
% Note that the trailing backslash on each first line
% merely continues the input to the second line
% (for convenient cut ant paste).
% Furthermore, the command |latex| can be replaced by any
% of its alternative versions such as |pdflatex|.
%
% %%%%%%%%%%%%%%%%%%%%%%%%%%%%%%%%%%%%%%%%%%%%%%%%%%%%%%%%%%%%%%%%%%%%%%%%%%%%%%
% %%%%%%%%%%%%%%%%%%%%%%%%%%%%%%%%%%%%%%%%%%%%%%%%%%%%%%%%%%%%%%%%%%%%%%%%%%%%%%
% \section{Implementation}
%\iffalse
%<*package>
%\fi
%
% This section describes the definitions file |childdoc.def|.

% The definitions cannot be loaded using |\usepackage| or |\RequirePackage|
% which has a mechanism to prevent loading a style file more than once.
% When loading the definitions by means of |\input|
% multiple instances have to be prevented manually:
%\iffalse
%This code needs to be before the `\ProvidesFile' directive
%which is defined at the beginning of this file.
%Therefore it is also placed there and commented out here.
%</package>
%<*discard>
%\fi
%    \begin{macrocode}
\ifdefined\childdocmain\endinput\fi
%    \end{macrocode}
%\iffalse
%</discard>
%<*package>
%\fi
%
% \macro{\ifchilddoc}
% \macro{\ifchilddocmanual}
% The conditional |\ifchilddoc| tells whether a
% child (true) or main (false) document is being compiled.
% The conditional |\ifchilddocmanual| tells whether
% the |\includeonly| mechanism is used (false) or
% the selection of child files must be performed manually (true).
% The definitions initialise to false:
%    \begin{macrocode}
\newif\ifchilddoc
\newif\ifchilddocmanual
%    \end{macrocode}

% \macro{\childdocname}
% \macro{\childdocjob}
% The macro |\childdocname| stores the name of the main document
% to be compiled. The macro |\childdocjob| stores the name of
% the document on which the \LaTeX{} compiler was originally invoked.
% The content of |\jobname| cannot be compared
% to filenames specified in the source due to different catcodes.
% The following code rescans |\jobname|, stores the result
% in |\childdocname| and saves a copy in |\childdocjob|:
%    \begin{macrocode}
\edef\childdocname{\scantokens\expandafter{\jobname\noexpand}}
\let\childdocjob\childdocname
%    \end{macrocode}

% \macro{\childdocdisable}
% The macro |\childdocdisable| prevents the main file
% from being processed more than once.
% At this stage, the main document command |\childdocmain|
% is assumed to be called once again where it should do nothing.
% Any subsequent call to it should prevent
% a secondary processing of the main document
% It overwrites the forwarding commands
% |\childdocof| and |\childdocforward|
% with empty macros to prevent further inclusions of the main document:
%    \begin{macrocode}
\newcommand{\childdocdisable}
{
  \renewcommand{\childdocmain}[1]{\renewcommand{\childdocmain}[1]{\endinput}}
  \renewcommand{\childdocof}[1]{}
  \renewcommand{\childdocby}[2][]{}
  \renewcommand{\childdocforward}[2][]{}
  \renewcommand{\childdocdisable}{}
}
%    \end{macrocode}

% \macro{\childdocmain}
% The macro |\childdocmain| is to be called at the top of the main file
% with nothing or the main filename (without extension) as argument.
% First, it breaks loops.
% If the argument is not empty and does not match |\childdocname|
% (which is set by the first inclusion of |childdoc.def|),
% |\ifchilddoc| is set to true, |\includeonly| is applied to the child file
% and |\jobname| is set to the main file
% (for proper handling of |.aux| files):
%    \begin{macrocode}
\newcommand{\childdocmain}[1]
{
  \childdocdisable\childdocmain{}
  \if?#1?\else
    \begingroup
      \def\childdoctmp{#1}
      \ifx\childdoctmp\childdocname
        \def\childdoctmp{}
      \else
        \def\childdoctmp
        {
          \childdoctrue
          \includeonly{\childdocname}
          \def\childdocjob{#1}
          \def\jobname{#1}
        }
      \fi
      \expandafter
    \endgroup
    \childdoctmp
  \fi
}
%    \end{macrocode}

% \macro{\childdocof}
% The command |\childdocof| redirects
% compilation to the main file |#1|.
%    \begin{macrocode}
\newcommand{\childdocof}[1]
{
  \childdocdisable
  \childdoctrue
  \includeonly{\childdocname}
  \def\jobname{#1}
  \def\childdocjob{#1}
  \input{#1}
}
%    \end{macrocode}

% \macro{\childdocby}
% The command |\childdocby| ....
%    \begin{macrocode}
\newcommand{\childdocby}[2][]
{
  \childdocdisable
  \childdoctrue
  \childdocmanualtrue
  \if?#1?\else
    \def\jobname{#2}
  \fi
  \def\childdocjob{#2}
  \input{#2}
  \endinput
}
%    \end{macrocode}

% \macro{\childdocforward}
% The command |\childdocforward| redirects
% compilation to the main file or
% (if the optional argument is given) a child file.
% Parameters are set as if the main file
% or a child file starting with |\childdocof| was compiled.
% Then compilation is handed over to the main file:
%    \begin{macrocode}
\newcommand{\childdocforward}[2][]
{
  \begingroup
    \if?#1?
      \def\childdoctmp
      {
        \def\childdocname{#2}
        \def\childdocjob{#2}
        \def\jobname{#2}
        \input{#2}
        \endinput
      }
    \else
      \def\childdoctmp
      {
        \childdocdisable
        \def\childdocname{#2}
        \childdoctrue
        \includeonly{#2}
        \def\childdocjob{#1}
        \def\jobname{#1}
        \input{#1}
        \endinput
      }
    \fi
    \expandafter
  \endgroup
  \childdoctmp
}
%    \end{macrocode}

% \macro{\childdocforwardprefix}
% The command |\childdocforwardprefix| redirects
% compilation to the main or a child file by means of a pattern.
% The prefix |#1| in the current filename is replaced by |#2|
% and the suffix of the current filename is kept
% (it is assumed that the filename does not contain the substring `|~~~|'
% which is used as a delimiter).
% Compilation is handed over to the new file by |\childdocforward|:
%    \begin{macrocode}
\newcommand{\childdocforwardprefix}[3][]
{
  \begingroup
    \def\childdocextract #2##1~~~{\def\childdoctmp{\childdocforward[#1]{#3##1}}}
    \expandafter\childdocextract\childdocname~~~
    \expandafter
  \endgroup
  \childdoctmp
}
%    \end{macrocode}

% \macro{\childdoc}
% The deprecated macro |\childdoc| is a legacy version of |\childdocmain|:
%    \begin{macrocode}
\newcommand{\childdoc}{\childdocmain}
%    \end{macrocode}

% \macro{\childdocredirect}
% The deprecated macro |\childdocredirect| is a legacy version
% of |\childdocforward| and |\childdocforwardprefix|:
%    \begin{macrocode}
\newcommand{\childdocredirect}[2][]
{
  \begingroup
    \if?#1?
      \def\childdoctmp{\childdocforward{#2}}
    \else
      \def\childdoctmp{\childdocforwardprefix{#1}{#2}}
    \fi
    \expandafter
  \endgroup
  \childdoctmp
}
%    \end{macrocode}

%\iffalse
%</package>
%\fi
%
\endinput

\childdocby{cdocsamp}
%    \end{macrocode}

%\iffalse
%</samplepart3|samplepart4>
%\fi
%
%\iffalse
%<*samplepart3>
%\fi
% Some text for part 3:
%    \begin{macrocode}
some text in part three
%    \end{macrocode}

%\iffalse
%</samplepart3>
%\fi
% Some text for part 4:
%\iffalse
%<*samplepart4>
%\fi
%    \begin{macrocode}
more text in part four
%    \end{macrocode}

%\iffalse
%</samplepart4>
%\fi
%
% %%%%%%%%%%%%%%%%%%%%%%%%%%%%%%%%%%%%%%
% \paragraph{Forwarding for a Complete Draft.}
%
% The following forwarding file |cdocsdrf.tex|
% compiles the main document in draft mode:
%\iffalse
%<*sampledraft>
%\fi
%    \begin{macrocode}
\def\version{draft}
% \iffalse
%
% childdoc.dtx Copyright (C) 2017-2018 Niklas Beisert
%
% This work may be distributed and/or modified under the
% conditions of the LaTeX Project Public License, either version 1.3
% of this license or (at your option) any later version.
% The latest version of this license is in
%   http://www.latex-project.org/lppl.txt
% and version 1.3 or later is part of all distributions of LaTeX
% version 2005/12/01 or later.
%
% This work has the LPPL maintenance status `maintained'.
%
% The Current Maintainer of this work is Niklas Beisert.
%
% This work consists of the files childdoc.dtx and childdoc.ins
% and the derived files childdoc.def and cdocsamp.tex with
% cdocsch1.tex, cdocsch2.tex, cdocsdrf.tex, cdocsfn1.tex, cdocsfn2.tex.
%
%<package>\ifdefined\childdocmain\endinput\fi
%<package>\ProvidesFile{childdoc.def}[2018/12/30 v2.0 child document driver]
%<samplemain>\ProvidesFile{cdocsamp.tex}[2018/12/30 v2.0 sample for childdoc]
%<*driver>
%\ProvidesFile{childdoc.drv}[2018/12/30 v2.0 childdoc reference manual file]
\PassOptionsToClass{10pt,a4paper}{article}
\documentclass{ltxdoc}

\usepackage[margin=35mm]{geometry}
\usepackage{hyperref}
\usepackage{hyperxmp}
\usepackage[usenames]{color}

\hypersetup{colorlinks=true}
\hypersetup{pdfstartview=FitH}
\hypersetup{pdfpagemode=UseNone}
\hypersetup{pdfsource={}}
\hypersetup{pdflang={en-UK}}
\hypersetup{pdfcopyright={Copyright 2017-2018 Niklas Beisert.
  This work may be distributed and/or modified under the
  conditions of the LaTeX Project Public License, either version 1.3
  of this license or (at your option) any later version.}}
\hypersetup{pdflicenseurl={http://www.latex-project.org/lppl.txt}}
\hypersetup{pdfcontactaddress={ETH Zurich, ITP, HIT K,
  Wolfgang-Pauli-Strasse 27}}
\hypersetup{pdfcontactpostcode={8093}}
\hypersetup{pdfcontactcity={Zurich}}
\hypersetup{pdfcontactcountry={Switzerland}}
\hypersetup{pdfcontactemail={nbeisert@itp.phys.ethz.ch}}
\hypersetup{pdfcontacturl={http://people.phys.ethz.ch/\xmptilde nbeisert/}}

\newcommand{\secref}[1]{\hyperref[#1]{section \ref*{#1}}}

\parskip1ex
\parindent0pt
\let\olditemize\itemize
\def\itemize{\olditemize\parskip0pt}

\begin{document}

\title{The \textsf{childdoc} Package}
\hypersetup{pdftitle={The childdoc Package}}
\author{Niklas Beisert\\[2ex]
  Institut f\"ur Theoretische Physik\\
  Eidgen\"ossische Technische Hochschule Z\"urich\\
  Wolfgang-Pauli-Strasse 27, 8093 Z\"urich, Switzerland\\[1ex]
  \href{mailto:nbeisert@itp.phys.ethz.ch}
  {\texttt{nbeisert@itp.phys.ethz.ch}}}
\hypersetup{pdfauthor={Niklas Beisert}}
\hypersetup{pdfsubject={Manual for the LaTeX2e Package childdoc}}
\date{30 December 2018, \textsf{v2.0}}
\maketitle

\begin{abstract}\noindent
\textsf{childdoc} is a \LaTeXe{} package
that enables the direct compilation
of document sections included by |\include|
to individual files.
\end{abstract}

\begingroup
\parskip0ex
\tableofcontents
\endgroup

%%%%%%%%%%%%%%%%%%%%%%%%%%%%%%%%%%%%%%%%%%%%%%%%%%%%%%%%%%%%%%%%%%%%%%%%%%%%%%%%
%%%%%%%%%%%%%%%%%%%%%%%%%%%%%%%%%%%%%%%%%%%%%%%%%%%%%%%%%%%%%%%%%%%%%%%%%%%%%%%%
\section{Introduction}

\LaTeX{} provides a mechanism to structure a large document (such as a book)
into a main file and several child files (containing the chapters)
using the |\include| command.
This mechanism is beneficial for documents
which span hundreds of pages in order to
make the source file(s) more manageable.
Moreover, compilation can be restricted to
selected child files by means of the |\includeonly| command.
The latter feature can be used to reduce the compilation time while editing
(this was significantly more useful in the earlier days of \LaTeX{})
or to generate a smaller document which is easier to navigate.
Another application of |\includeonly| is to generate
documents consisting of selected parts of the complete document.

However, there are a few drawbacks of the plain |\include| mechanism:
\begin{itemize}
\item
The child files cannot be compiled on their own,
they can only be compiled via the main file.
A naive editing environment
(such as a text editor with an option
to have the current file processed by \LaTeX)
may require one to switch to the main file before compiling;
attempting to compile the child file produces errors.
\item
The main file must be modified (each time)
to adjust the |\includeonly| command
to the present needs. This easily leaves the main file in a messy state.
\item
The generated document will always carry the filename
of the main document. This is inconvenient if
several child files are to be compiled and
to be kept for distribution.
\end{itemize}

The present package provides a simple interface
to make child files individually compilable by \LaTeX{}.
Compiling a child file then has the same effect as compiling
the main file with an |\includeonly| command
to select the appropriate child.
Moreover the generated document will carry the name of the child
rather than the main file.
This resolves all three above issues.

This feature is meant to make the editing of books,
thesis documents and lecture notes somewhat more convenient.
However, the package can also be used efficiently for
composing a series of documents (such as exercise sheets)
which are typically distributed individually.
It then assists the author in generating the individual documents
(potentially in different versions)
as well as a document containing the collected series.
Another application is in developing style files
or other kinds of included material
where compilation of the style file could redirect
to a sample or test file.

%%%%%%%%%%%%%%%%%%%%%%%%%%%%%%%%%%%%%%%%%%%%%%%%%%%%%%%%%%%%%%%%%%%%%%%%%%%%%%%%
%%%%%%%%%%%%%%%%%%%%%%%%%%%%%%%%%%%%%%%%%%%%%%%%%%%%%%%%%%%%%%%%%%%%%%%%%%%%%%%%
\section{Usage}

First of all, the package \textsf{childdoc} is \emph{not} a standard
\LaTeXe{} |.sty| style file! Therefore it needs to be invoked in
a non-standard way.

%%%%%%%%%%%%%%%%%%%%%%%%%%%%%%%%%%%%%%%%%%%%%%%%%%%%%%%%%%%%%%%%%%%%%%%%%%%%%%%%
\subsection{Included Files}
\label{sec:include}

%%%%%%%%%%%%%%%%%%%%%%%%%%%%%%%%%%%%%%%%
\DescribeMacro{\childdocmain}
To use the package, add the commands
\begin{center}
\begin{tabular}{l}
|\input{childdoc.def}|\\
|\childdocmain{}|\\
\end{tabular}
\end{center}
at the very top of the main \LaTeX{} file,
in particular \emph{before} the |\documentclass| statement!
The argument of |\childdocmain| should be left empty
(but it must be present).

%%%%%%%%%%%%%%%%%%%%%%%%%%%%%%%%%%%%%%%%
\DescribeMacro{\childdocof}
Furthermore, add the commands
\begin{center}
\begin{tabular}{l}
|\input{childdoc.def}|\\
|\childdocof{|\textit{main}|}|\\
\end{tabular}
\end{center}
at the top of every child file \textit{child}
which is included by |\include{|\textit{child}|}|
from within the main file
(or at least for those files to be compiled individually).
The argument \textit{main} must be the filename of the main file.

There are a couple of
considerations in setting up the main and child documents:

%%%%%%%%%%%%%%%%%%%%%%%%%%%%%%%%%%%%%%%%
\paragraph{Restrictions.}

Please note the following restrictions:
\begin{itemize}
\item
|\childdocmain| must be called with one argument \textit{main}
to ensure compatibility with earlier version of the package.
It must either be empty (|\childdocmain{}|)
or precisely match the filename of the main file in which it is specified.
See \secref{sec:detection} for further information.
\item
The filename \textit{main} must be specified without the |.tex| extension.
\item
The filename \textit{main} is case sensitive
(even in case-insensitive file systems)
due to internal string comparison.
\item
The argument \textit{main} should be fully expanded, it cannot be a macro.
\item
Subdirectories and special characters should be avoided in filenames.
\item
The command |\childdocmain{|\textit{main}|}| must be followed by a whitespace.
It should not be followed immediately by another command
or by a comment mark `|%|'.
This is because the \TeX{} parser reads the token immediately following
the argument of |\childdocmain| and puts it
at the beginning of every child section;
however, a white\-space is ignored.
\end{itemize}

%%%%%%%%%%%%%%%%%%%%%%%%%%%%%%%%%%%%%%%%
\paragraph{Content of Main File.}

It is advisable to place all content in the child files included by |\include|.
Any output contained in the main file will appear in all child documents
unless suppressed manually;
it cannot be suppressed automatically by the |\includeonly| directive
and thus should normally be avoided.
A method to include some content in the main file
by means of conditional processing is described in \secref{sec:conditional}.

%%%%%%%%%%%%%%%%%%%%%%%%%%%%%%%%%%%%%%%%
\paragraph{Page Numbering.}

When only a part of the document is compiled,
the appropriate numbering of pages
(as well as other status parameters)
is determined from the |.aux| files.
The latter contain information from previous passes.
However this information needs to propagate through
all intermediate child documents.
Therefore the page numbering in child documents may well
be inconsistent until the complete document is compiled at least once.

A useful (if unconventional) way to always ensure a consistent
page numbering is to restart the numbering in each child document
and denote the pages by `\textit{child}|.|\textit{page}'
where \textit{child} represents the chapter/section number of the child file.
This can be achieved by the command
|\numberwithin{page}{|\textit{child}|}|
of the \textsf{amsmath} package
where \textit{child} can be |chapter| or |section|
depending on the chosen structuring.
Alternatively, one can modify the macro |\thepage| appropriately
and reset the counter |page| at the start of each child file.

%%%%%%%%%%%%%%%%%%%%%%%%%%%%%%%%%%%%%%%%%%%%%%%%%%%%%%%%%%%%%%%%%%%%%%%%%%%%%%%%
\subsection{Conditional Processing}
\label{sec:conditional}

The package provides a mechanism to compile different versions
of a document. To customise the versions further some conditional processing
can come in handy to distinguish which version is being compiled.
The package provides two macros to describe the compilation context:

%%%%%%%%%%%%%%%%%%%%%%%%%%%%%%%%%%%%%%%%
\DescribeMacro{\ifchilddoc}
The conditional |\ifchilddoc| distinguishes between the compilation of
child documents and the main document:
%
\begin{center}
|\ifchilddoc |\textit{child-code}| |[|\||else |\textit{main-code}]| \||fi|
\end{center}

%%%%%%%%%%%%%%%%%%%%%%%%%%%%%%%%%%%%%%%%
\DescribeMacro{\childdocname}
\DescribeMacro{\childdocjob}
The macro |\childdocname| contains the filename (without extension)
of the main or child file being processed.
Note that |\childdocjob| will always contain the name of the main file.

%%%%%%%%%%%%%%%%%%%%%%%%%%%%%%%%%%%%%%%%
\paragraph{Title Page.}

Conditional processing can be used to include a title or banner page
in the main document when proper precautions are taken.
Importantly, the code in the main file should ensure that the page counter
(as well as other status parameters which are stored in the |.aux| files)
takes the same value after the conditional processing.
Otherwise the page numbers may take divergent values
depending on which part is compiled.

For example, a title page could be declared by:
%
\begin{center}
\begin{tabular}{l}
|\ifchilddoc\||else|\\
|\addtocounter{page}{-1}|\\
\textit{code for title page}\\
|\newpage|\\
|\||fi|
\end{tabular}
\end{center}
%
A banner page for the child documents can be generated by:
%
\begin{center}
\begin{tabular}{l}
|\ifchilddoc|\\
|\addtocounter{page}{-1}|\\
\textit{code for banner page}\\
|\newpage|\\
|\||fi|
\end{tabular}
\end{center}
%
Here one could write a message such as:
\begin{center}
|This is the part \childdocname{} of \childdocjob{}.|
\end{center}

%%%%%%%%%%%%%%%%%%%%%%%%%%%%%%%%%%%%%%%%%%%%%%%%%%%%%%%%%%%%%%%%%%%%%%%%%%%%%%%%
\subsection{Flags}
\label{sec:flags}

The package makes it easy to generate different versions
of the main or child documents.
To this end compilation flags can be defined
and assigned different default values.
They will be particularly useful in conjunction
with the forwarding mechanism described in \secref{sec:forward}.

For example, it may be useful to have a flag |\version|
which can be set to |draft| or |final|.
The document source will contain some conditional code
depending on the value of |\version|.
Suppose further, the flag should default to |final| for the main file
and to |draft| for child files
which is a natural assignment for editing the document.
This is achieved by placing the following code
in the preamble of the main document
(below the |\childdocmain| directive):
%
\begin{center}
\begin{tabular}{l}
|\ifchilddoc|\\
|\providecommand{\version}{draft}|\\
|\||else|\\
|\providecommand{\version}{final}|\\
|\||fi|
\end{tabular}
\end{center}
%
The definition by |\providecommand| makes sure
that previous definitions are not overwritten.
Further statements |\providecommand{\version}{...}|
can thus be added before the above code to override it.

For the main file, one might add a line
(between |\childdocmain| and the above block)
%
\begin{center}
|%\ifchilddoc\||else\providecommand{\version}{draft}\||fi|
\end{center}
%
which can be uncommented to produce a draft version.
Likewise one can add a line to the very top of a child file
(above the |\childdocof{|\textit{main}|}| directive)
%
\begin{center}
|%\providecommand{\version}{final}|
\end{center}
%
which can be uncommented to produce the final version of this child document.

%%%%%%%%%%%%%%%%%%%%%%%%%%%%%%%%%%%%%%%%%%%%%%%%%%%%%%%%%%%%%%%%%%%%%%%%%%%%%%%%
\subsection{Forwarding}
\label{sec:forward}

Different versions of the main or child documents
using compilation flags as described in \secref{sec:flags}
can be (permanently) stored in different files
for convenient compilation, viewing and distribution.
To this end, the package defines a command
to pass on compilation to a different file:

%%%%%%%%%%%%%%%%%%%%%%%%%%%%%%%%%%%%%%%%
\DescribeMacro{\childdocforward}
The command |\childdocforward| redirects processing to
another source file:
%
\begin{center}
\begin{tabular}{l}
|\input{childdoc.def}|\\
|\childdocforward[|\textit{main}|]{|\textit{dest}|}|\\
\end{tabular}
\end{center}
%
The argument \textit{dest} is the destination file
(without extension).
It should be the main file or one of the child files.
Note that further \textsf{childdoc} directives
such as |\childdocof| and |\childdocforward|
in the indicated file will be processed in this form.
The optional argument \textit{main}
passes on directly to the main file \textit{main}
while pretending to compile the child \textit{dest}.
This form behaves as if \textit{dest}
issues |\childdocof{|\textit{main}|}| right away,
and no further \textsf{childdoc} directives will be processed.

%%%%%%%%%%%%%%%%%%%%%%%%%%%%%%%%%%%%%%%%
\DescribeMacro{\...prefix}
In the alternative form |\childdocforwardprefix|,
%
\begin{center}
\begin{tabular}{l}
|\input{childdoc.def}|\\
|\childdocforwardprefix[|\textit{main}|]{|\textit{prefix}|}{|\textit{dest}|}|
\end{tabular}
\end{center}
%
the destination file is determined by a pattern
depending on the current file:
To make this work, the current file must be called
`{\textit{prefix}\hspace{0.2em}\textit{suffix}}'
with \textit{prefix} matching precisely the argument.
Processing is then passed on to the file
`{\textit{dest}\hspace{0.2em}\textit{suffix}}'.
Surely, the same effect is achieved by
directly specifying the
argument `{\textit{dest}\hspace{0.2em}\textit{suffix}}'
in the first form.
However, that requires to set up a different file
for each child. With the alternative form of the command
all these files can have exactly the same content
which simplifies setting them up and maintaining them.

For example, the following file |draft.tex|
with a compilation flag |\version| as described in \secref{sec:flags}
compiles the main document as a draft:
%
\begin{center}
\begin{tabular}{l}
|\def\version{draft}|\\
|\input{childdoc.def}|\\
|\childdocforward{|\textit{main}|}|
\end{tabular}
\end{center}
%
Likewise, the following files |final|\textit{nn}|.tex|
compile the final version of the child document
|child|\textit{nn}|.tex|:
%
\begin{center}
\begin{tabular}{l}
|\def\version{final}|\\
|\input{childdoc.def}|\\
|\childdocforwardprefix{final}{child}|
\end{tabular}
\end{center}
%

Note that when several versions of a main file and/or of each child file
are to be generated, it may be convenient to set up a |Makefile| or
shell script to automatise the process.

%%%%%%%%%%%%%%%%%%%%%%%%%%%%%%%%%%%%%%%%%%%%%%%%%%%%%%%%%%%%%%%%%%%%%%%%%%%%%%%%
\subsection{Command Line Processing}
\label{sec:commandline}

The effect of redirection files can also be achieved by invoking
the \LaTeX{} compiler with a more elaborate command line.
Most conveniently this should be done as part
of a shell script or a |Makefile|.

When using \textsf{childdoc} in the main file, the following
command lines effectively perform a redirection
(note that depending on the shell being used,
backslashes may have to be doubled: `|\|' $\to$ `|\\|'):
%
\begin{center}
|... -jobname "|\textit{target}|" |\\|"|[\textit{flags}]%
|\input{childdoc.def}\childdocforward[|\textit{main}|]{|\textit{dest}|}"|
\end{center}
%
Here \textit{target} is the name of the output file,
\textit{main} is the name of the main file
and \textit{dest} is the name of the main or child file to be processed
(all filenames without extensions).
The optional argument \textit{main} can be omitted
if \textit{main} matches \textit{dest}.
Optionally, compilation \textit{flags} can be defined via |\def| commands.
This command line makes the \TeX{} engine believe
it is compiling the file \textit{target}
whose content is specified as the latter parameter.
The provided code then forwards the processing to
\textit{main} or \textit{dest} as described in \secref{sec:forward}.

%%%%%%%%%%%%%%%%%%%%%%%%%%%%%%%%%%%%%%%%%%%%%%%%%%%%%%%%%%%%%%%%%%%%%%%%%%%%%%%%
\subsection{Include by Input}
\label{sec:input}

Including child documents by |\include| has some restrictions by design.
Most notably, the content of a child document always occupies
its own set of pages; pages cannot be shared between child documents.
Usually, this behaviour makes perfect sense
because each child document contain an essential part of the document.
However, in some situations it may be desirable to compose
a document from a collection of parts
without having mandatory page breaks between then.
For this case, the package
provides a mechanism to include parts
by |\input| which can also be processed individually.
However, by construction this mechanism
requires manual handling of the content to be output.

%%%%%%%%%%%%%%%%%%%%%%%%%%%%%%%%%%%%%%%%
\DescribeMacro{\ifchilddocmanual}
The main file should be prepared as usual, see \secref{sec:include}.
However, the document body must make a distinction
between processing of an individual part and of the main document, e.g.:
%
\begin{center}
\begin{tabular}{l}
|\ifchilddocmanual|\\
|\input{\childdocname}|\\
|\||else|\\
\textit{document body with }|\input{|\textit{part}|}|\\
|\||fi|
\end{tabular}
\end{center}
%
The conditional |\ifchilddocmanual| is true whenever
a part to be included by |\input| is being compiled,
and the name of the part is stored in |\childdocname|.

%%%%%%%%%%%%%%%%%%%%%%%%%%%%%%%%%%%%%%%%
\DescribeMacro{\childdocby}
Each part to be included by |\input| should start with:
%
\begin{center}
\begin{tabular}{l}
|\input{childdoc.def}|\\
|\childdocby{|\textit{main}|}|\\
\end{tabular}
\end{center}
%
The directive |\childdocby| is similar to |\childdocof|
described in \secref{sec:include},
but the subsequent selection of content must be done manually.
To that end, both |\ifchilddoc| and |\ifchilddocmanual|
will be true upon processing of a part,
and the name of the part is stored in |\childdocname|.
Note that |\jobname| will be set to the filename of the current part
so that each part receives an individual |.aux| file
that does not interfere with the |.aux| file(s) of the main document.
This behaviour can be altered by the alternative form
|\childdocby[*]{|\textit{main}|}| (with a non-empty optional argument)
which uses the |.aux| file of the main document
by setting |\jobname| to \textit{main}.

%%%%%%%%%%%%%%%%%%%%%%%%%%%%%%%%%%%%%%%%%%%%%%%%%%%%%%%%%%%%%%%%%%%%%%%%%%%%%%%%
\subsection{Driver Development}
\label{sec:driver}

The \textsf{childdoc} mechanism can also be use for the development
of definition files such as \LaTeX{} styles or classes.
This case differs from the above setup with multiple parts
included by |\include| in that no |\includeonly| should be invoked.
This can be achieved by starting the include file
(before |\ProvidesPackage|) with:
%
\begin{center}
\begin{tabular}{l}
|\input{childdoc.def}|\\
|\childdocforward{|\textit{main}|}|\\
\end{tabular}
\end{center}
%
or alternatively with:
%
\begin{center}
\begin{tabular}{l}
|\input{childdoc.def}|\\
|\childdocby{|\textit{main}|}|\\
\end{tabular}
\end{center}
%
Both forms have slightly different effects as described above.
The main file is prepared as usual, see \secref{sec:include}.

%%%%%%%%%%%%%%%%%%%%%%%%%%%%%%%%%%%%%%%%%%%%%%%%%%%%%%%%%%%%%%%%%%%%%%%%%%%%%%%%
\subsection{Legacy Detection}
\label{sec:detection}

The directive |\childdocmain| in the main file can detect
whether the complete document or merely a child is to be compiled
even without using the directive |\childdocof|.
This method is deprecated because it is less robust
and there is no compelling reason to use it;
it is merely provided for backward compatibility
and it may be removed in future versions.

If the detection mechanism is to be used,
it is mandatory to correctly specify
the filename of the main file as the argument of |\childdocmain|:
%
\begin{center}
\begin{tabular}{l}
|\input{childdoc.def}|\\
|\childdocmain{|\textit{main}|}|\\
\end{tabular}
\end{center}
%
If |\jobname| does not match the argument \textit{main} of |\childdocmain|,
it is assumed that |\jobname| points to the child file to be compiled.
When using |\childdocmain| with the main file specified as argument,
it suffices to start a child file
with just |\input{|\textit{main}|}|
without loading of the package and using |\childdocof|.
If instead all processing is done
with the appropriate \textsf{childdoc} directives,
the argument of \textit{main} of |\childdocmain| can be empty.

An alternative version of the command line processing described
in \secref{sec:commandline} using the detection mechanism reads:
%
\begin{center}
|... -jobname "|\textit{target}|" "|[\textit{flags}]%
[|\def\jobname{|\textit{dest}|}|]|\input{|\textit{main}|}"|
\end{center}

%%%%%%%%%%%%%%%%%%%%%%%%%%%%%%%%%%%%%%%%%%%%%%%%%%%%%%%%%%%%%%%%%%%%%%%%%%%%%%%%
\subsection{Manual Code}
\label{sec:manual}

In case one cannot be certain whether the definitions file |childdoc.def|
is installed on the target \TeX{} distribution
and one prefers not to ship it,
it is conceivable to paste a few relevant commands into the sources.

To that end, drop all statements |\input{childdoc.def}|
and perform the replacements as outlined below.
Instead of |\childdocmain{|\textit{main}|}| add the following code
to the top of the main file:
%
\begin{center}
\begin{tabular}{l}
|\||ifdefined\childdocname\endinput\||fi\newif\ifchilddoc|\\
|\edef\childdocname{\scantokens\expandafter{\jobname\noexpand}}|\\
|\def\childdocmain{|\textit{main}|}\||ifx\childdocmain\childdocname\||else|\\
|\childdoctrue\includeonly{\childdocname}\let\jobname\childdocmain\||fi|\\
\end{tabular}
\end{center}
%
Instead of |\childdocof{|\textit{main}|}| just include the main file
at the top of each child file:
%
\begin{center}
|\input{|\textit{main}|}|
\end{center}
%
A simple redirection |\childdocforward{|\textit{dest}|}| is achieved by:
%
\begin{center}
|\def\jobname{|\textit{dest}|}\input{\jobname}|
\end{center}
%
The redirection with prefix
|\childdocforwardprefix[|\textit{prefix}|]{|\textit{dest}|}|
is accomplished by:
%
\begin{center}
\begin{tabular}{l}
|{\edef\jobname{\scantokens\expandafter{\jobname\noexpand}}|\\
|\def\redirectjob |\textit{prefix}|#1~~~{\gdef\jobname{|\textit{dest}|#1}}|\\
|\expandafter\redirectjob\jobname~~~}\input{\jobname}|
\end{tabular}
\end{center}

In an alternative approach,
child documents can be compiled by a specific command line
without additional code or specific definitions:
%
\begin{center}
|... -jobname "|\textit{target}|" "|[\textit{flags}]%
|\includeonly{|\textit{dest}|}\input{|\textit{main}|}"|
\end{center}
%

%%%%%%%%%%%%%%%%%%%%%%%%%%%%%%%%%%%%%%%%%%%%%%%%%%%%%%%%%%%%%%%%%%%%%%%%%%%%%%%%
%%%%%%%%%%%%%%%%%%%%%%%%%%%%%%%%%%%%%%%%%%%%%%%%%%%%%%%%%%%%%%%%%%%%%%%%%%%%%%%%
\section{Information}

%%%%%%%%%%%%%%%%%%%%%%%%%%%%%%%%%%%%%%%%%%%%%%%%%%%%%%%%%%%%%%%%%%%%%%%%%%%%%%%%
\subsection{Copyright}

Copyright \copyright{} 2017--2018 Niklas Beisert

This work may be distributed and/or modified under the
conditions of the \LaTeX{} Project Public License, either version 1.3
of this license or (at your option) any later version.
The latest version of this license is in
  \url{http://www.latex-project.org/lppl.txt}
and version 1.3 or later is part of all distributions of \LaTeX{}
version 2005/12/01 or later.

This work has the LPPL maintenance status `maintained'.

The Current Maintainer of this work is Niklas Beisert.

This work consists of the files |README.txt|, |childdoc.ins| and |childdoc.dtx|
as well as the derived files |childdoc.def|, |cdocsamp.tex|
with |cdocsch1.tex|, |cdocsch2.tex|, |cdocspt3.tex|, |cdocspt4.tex|,
|cdocsdrf.tex|, |cdocsfn1.tex|, |cdocsfn2.tex|
as well as |childdoc.pdf|.

%%%%%%%%%%%%%%%%%%%%%%%%%%%%%%%%%%%%%%%%%%%%%%%%%%%%%%%%%%%%%%%%%%%%%%%%%%%%%%%%
\subsection{Files and Installation}

The package consists of the files:
%
\begin{center}
\begin{tabular}{ll}
    |README.txt|   & readme file \\
    |childdoc.ins| & installation file \\
    |childdoc.dtx| & source file \\
    |childdoc.def| & definition file \\
    |cdocsamp.tex| & sample main file \\
    |cdocsch1.tex| & sample include file \\
    |cdocsch2.tex| & sample include file \\
    |cdocspt3.tex| & sample part file \\
    |cdocspt4.tex| & sample part file \\
    |cdocsdrf.tex| & sample redirection file \\
    |cdocsfn1.tex| & sample redirection file \\
    |cdocsfn2.tex| & sample redirection file \\
    |childdoc.pdf| & manual
\end{tabular}
\end{center}
%
The distribution consists of the files
|README.txt|, |childdoc.ins| and |childdoc.dtx|.
%
\begin{itemize}
\item
Run (pdf)\LaTeX{} on |childdoc.dtx|
to compile the manual |childdoc.pdf| (this file).
\item
Run \LaTeX{} on |childdoc.ins| to create the definitions file |childdoc.def|
and the sample |cdocsamp.tex| with include files
|cdocsch1.tex|, |cdocsch2.tex|, |cdocspt3.tex|, |cdocspt4.tex|,
|cdocsdrf.tex|, |cdocsfn1.tex|, |cdocsfn2.tex|.
Then copy the file |childdoc.def| to an appropriate directory of your \LaTeX{}
distribution, e.g.\ \textit{texmf-root}|/tex/latex/childdoc|.
\end{itemize}

%%%%%%%%%%%%%%%%%%%%%%%%%%%%%%%%%%%%%%%%%%%%%%%%%%%%%%%%%%%%%%%%%%%%%%%%%%%%%%%%
\subsection{Related CTAN Packages}

There are several other packages which offer a similar functionality:
%
\begin{itemize}
\item
The packages
\href{http://ctan.org/pkg/docmute}{\textsf{docmute}},
\href{http://ctan.org/pkg/includex}{\textsf{includex}} and
\href{http://ctan.org/pkg/standalone}{\textsf{standalone}}
provide commands to include only the document body of
a child file thus allowing both files to be compiled individually.
\item
The packages \href{http://ctan.org/pkg/subdocs}{\textsf{subdocs}}
and \href{http://ctan.org/pkg/subfiles}{\textsf{subfiles}}
provide structures in which the main and child documents can be
encapsulated and allowing them to be compiled individually.
The inclusion mechanism is different from the conventional |\include|.
\item
The package \href{http://ctan.org/pkg/combine}{\textsf{combine}}
is an elaborate solution to combine several documents into one.
\end{itemize}
%
See also the CTAN topic \href{http://ctan.org/topic/subdocs}{\textsf{subdocs}}
for further related packages.
The present package differs from the above solutions in that
a document structure constructed with the conventional |\include| mechanism
just needs two extra commands at the top of every file
such that all constituent files can be compiled individually.

%%%%%%%%%%%%%%%%%%%%%%%%%%%%%%%%%%%%%%%%%%%%%%%%%%%%%%%%%%%%%%%%%%%%%%%%%%%%%%%%
%\subsection{Feature Suggestions}
%
%The following is a list of features which may be useful for future
%versions of this package:
%%
%\begin{itemize}
%\item
%\ldots
%\end{itemize}

%%%%%%%%%%%%%%%%%%%%%%%%%%%%%%%%%%%%%%%%%%%%%%%%%%%%%%%%%%%%%%%%%%%%%%%%%%%%%%%%
\subsection{Revision History}

%%%%%%%%%%%%%%%%%%%%%%%%%%%%%%%%%%%%%%%%
\paragraph{v2.0:} 2018/12/30

\begin{itemize}
\item
immediate forward processing
\item
added |\childdocby| mechanism
\item
manual restructured
\end{itemize}

%%%%%%%%%%%%%%%%%%%%%%%%%%%%%%%%%%%%%%%%
\paragraph{v1.6:} 2018/01/17

\begin{itemize}
\item
application for development of include files
\item
corrections to manual
\end{itemize}

%%%%%%%%%%%%%%%%%%%%%%%%%%%%%%%%%%%%%%%%
\paragraph{v1.5:} 2017/05/21

\begin{itemize}
\item
more complete structuring introduced
\item
|\childdocof| introduced
\item
|\childdoc| renamed to |\childdocmain|
\item
|\childredirect| renamed to |\childdocforward| and |\childdocforwardprefix|
and functionality expanded
\end{itemize}

%%%%%%%%%%%%%%%%%%%%%%%%%%%%%%%%%%%%%%%%
\paragraph{v1.0:} 2017/04/27

\begin{itemize}
\item
manual and install package
\item
first version published on CTAN
\end{itemize}

%%%%%%%%%%%%%%%%%%%%%%%%%%%%%%%%%%%%%%%%
\paragraph{v0.6:} 2017/04/26

\begin{itemize}
\item
redirection mechanism added
\end{itemize}

%%%%%%%%%%%%%%%%%%%%%%%%%%%%%%%%%%%%%%%%
\paragraph{v0.5:} 2017/04/26

\begin{itemize}
\item
functionality in definition file
\end{itemize}


%%%%%%%%%%%%%%%%%%%%%%%%%%%%%%%%%%%%%%%%%%%%%%%%%%%%%%%%%%%%%%%%%%%%%%%%%%%%%%%%
%%%%%%%%%%%%%%%%%%%%%%%%%%%%%%%%%%%%%%%%%%%%%%%%%%%%%%%%%%%%%%%%%%%%%%%%%%%%%%%%
%%%%%%%%%%%%%%%%%%%%%%%%%%%%%%%%%%%%%%%%%%%%%%%%%%%%%%%%%%%%%%%%%%%%%%%%%%%%%%%%
\appendix

\settowidth\MacroIndent{\rmfamily\scriptsize 000\ }

 \DocInput{childdoc.dtx}

\end{document}
%</driver>
% \fi
%
% %%%%%%%%%%%%%%%%%%%%%%%%%%%%%%%%%%%%%%%%%%%%%%%%%%%%%%%%%%%%%%%%%%%%%%%%%%%%%%
% %%%%%%%%%%%%%%%%%%%%%%%%%%%%%%%%%%%%%%%%%%%%%%%%%%%%%%%%%%%%%%%%%%%%%%%%%%%%%%
% \section{Sample}
%\iffalse
%<*samplemain>
%\fi
%
% The following presents a sample document
% with two chapters, two parts, a title page,
% a compile flag as well as three forwarding files to set the flag.
% It consists of eight |.tex| files:
% \begin{center}
% \begin{tabular}{ll}
% |cdocsamp.tex|&main file\\
% |cdocsch1.tex|&include file for chapter 1\\
% |cdocsch2.tex|&include file for chapter 2\\
% |cdocspt3.tex|&include file for part 3\\
% |cdocspt4.tex|&include file for part 4\\
% |cdocsdrf.tex|&forwarding file for main file in draft mode\\
% |cdocsfi1.tex|&forwarding file for final version of chapter 1\\
% |cdocsfi2.tex|&forwarding file for final version of chapter 2\\
% \end{tabular}
% \end{center}
% Each of the eight files can be compiled directly by the \LaTeX{} compiler.
%
% %%%%%%%%%%%%%%%%%%%%%%%%%%%%%%%%%%%%%%
% \paragraph{Main File.}
%
% The main file is called |cdocsamp.tex|.
%
% Load the \textsf{childdoc} definitions and
% declare the filename for the main document:
%    \begin{macrocode}
\input{childdoc.def}
\childdocmain{}
%    \end{macrocode}

% Optional override for |\version| flag:
%    \begin{macrocode}
%%\ifchilddoc\else\providecommand{\version}{draft}\fi
%    \end{macrocode}

% Define the default values for the |\version| flag
% (|final| for the main file and |draft| for childs):
%    \begin{macrocode}
\ifchilddoc
\providecommand{\version}{draft}
\else
\providecommand{\version}{final}
\fi
%    \end{macrocode}

% Load the standard document class:
%    \begin{macrocode}
\documentclass[12pt]{article}
%    \end{macrocode}

% Start the document body:
%    \begin{macrocode}
\begin{document}
%    \end{macrocode}

% Declare a title page.
% Print title, part of document being processed and version flag:
%    \begin{macrocode}
\addtocounter{page}{-1}
\begin{center}
{\LARGE\bfseries{}childdoc example\par}
\vspace{1cm}
\ifchilddoc
\ifchilddocmanual part\else chapter\fi:
`\childdocname' of `\childdocjob'\par
\else
main document: `\childdocjob'\par
\fi
version: \version\par
\end{center}
\newpage
%    \end{macrocode}

% Manually include selected file,
% otherwise process as usual:
%    \begin{macrocode}
\ifchilddocmanual
\section*{part `\childdocname'}
\input{\childdocname}
\else
%    \end{macrocode}

% Include the two chapters:
%    \begin{macrocode}
\include{cdocsch1}
\include{cdocsch2}
%    \end{macrocode}

% Include the two parts unless only chapters should be displayed:
%    \begin{macrocode}
\ifchilddoc\else
\section{part three}
\input{cdocspt3}
\section{part four}
\input{cdocspt4}
\fi
%    \end{macrocode}

% Process as usual until here:
%    \begin{macrocode}
\fi
%    \end{macrocode}

% End of document body:
%    \begin{macrocode}
\end{document}
%    \end{macrocode}
%\iffalse
%</samplemain>
%\fi
%
% %%%%%%%%%%%%%%%%%%%%%%%%%%%%%%%%%%%%%%
% \paragraph{Chapter Include Files.}
%
% The include files are called |cdocsch1.tex| and |cdocsch2.tex|.
%
%\iffalse
%<*samplechap1|samplechap2>
%\fi

% Optional override for |\version| flag:
%    \begin{macrocode}
%%\providecommand{\version}{final}
%    \end{macrocode}

% Include the main document:
%    \begin{macrocode}
\input{childdoc.def}
\childdocof{cdocsamp}
%    \end{macrocode}

%\iffalse
%</samplechap1|samplechap2>
%\fi
%
%\iffalse
%<*samplechap1>
%\fi
% Some text for chapter 1:
%    \begin{macrocode}
\section{one}
some text in chapter one
%    \end{macrocode}

%\iffalse
%</samplechap1>
%\fi
% Some text for chapter 2:
%\iffalse
%<*samplechap2>
%\fi
%    \begin{macrocode}
\section{two}
more text in chapter two
%    \end{macrocode}

%\iffalse
%</samplechap2>
%\fi
%
% %%%%%%%%%%%%%%%%%%%%%%%%%%%%%%%%%%%%%%
% \paragraph{Part Include Files.}
%
% The include files are called |cdocspt3.tex| and |cdocspt4.tex|.
%
%\iffalse
%<*samplepart3|samplepart4>
%\fi

% Optional override for |\version| flag:
%    \begin{macrocode}
%%\providecommand{\version}{final}
%    \end{macrocode}

% Include the main document:
%    \begin{macrocode}
\input{childdoc.def}
\childdocby{cdocsamp}
%    \end{macrocode}

%\iffalse
%</samplepart3|samplepart4>
%\fi
%
%\iffalse
%<*samplepart3>
%\fi
% Some text for part 3:
%    \begin{macrocode}
some text in part three
%    \end{macrocode}

%\iffalse
%</samplepart3>
%\fi
% Some text for part 4:
%\iffalse
%<*samplepart4>
%\fi
%    \begin{macrocode}
more text in part four
%    \end{macrocode}

%\iffalse
%</samplepart4>
%\fi
%
% %%%%%%%%%%%%%%%%%%%%%%%%%%%%%%%%%%%%%%
% \paragraph{Forwarding for a Complete Draft.}
%
% The following forwarding file |cdocsdrf.tex|
% compiles the main document in draft mode:
%\iffalse
%<*sampledraft>
%\fi
%    \begin{macrocode}
\def\version{draft}
\input{childdoc.def}
\childdocforward{cdocsamp}
%    \end{macrocode}

%\iffalse
%</sampledraft>
%\fi
%
% %%%%%%%%%%%%%%%%%%%%%%%%%%%%%%%%%%%%%%
% \paragraph{Forwarding for Final Version of the Chapters.}
%
% The following forwarding files |cdocsfn1.tex| and |cdocsfn2.tex|
% (with identical content)
% compile the final versions of the child documents
% |cdocsch1.tex| and |cdocsch2.tex|, respectively:
%\iffalse
%<*samplefinal>
%\fi
%    \begin{macrocode}
\def\version{final}
\input{childdoc.def}
\childdocforwardprefix[cdocsamp]{cdocsfn}{cdocsch}
%    \end{macrocode}

%\iffalse
%</samplefinal>
%\fi
%
% %%%%%%%%%%%%%%%%%%%%%%%%%%%%%%%%%%%%%%
% \paragraph{Command Line Processing.}
%
% The following three command lines generate the output files
% |cdocscld|, |cdocscl1| and |cdocscl2|
% which should be identical to
% |cdocsdrf|, |cdocsch1| and |cdocsfn2|, respectively:
% \begin{center}
% \begin{tabular}{l}
% |latex -jobname cdocscld \|\\
% |  "\def\version{draft}\input{childdoc.def}\childdocforward{cdocsamp}"|\\
% |latex -jobname cdocscl1 \|\\
% |  "\input{childdoc.def}\childdocforward[cdocsamp]{cdocsch1}"|\\
% |latex -jobname cdocscl2 \|\\
% |  "\def\version{final}\input{childdoc.def}\childdocforward{cdocsch2}"|
% \end{tabular}
% \end{center}
% Note that the trailing backslash on each first line
% merely continues the input to the second line
% (for convenient cut ant paste).
% Furthermore, the command |latex| can be replaced by any
% of its alternative versions such as |pdflatex|.
%
% %%%%%%%%%%%%%%%%%%%%%%%%%%%%%%%%%%%%%%%%%%%%%%%%%%%%%%%%%%%%%%%%%%%%%%%%%%%%%%
% %%%%%%%%%%%%%%%%%%%%%%%%%%%%%%%%%%%%%%%%%%%%%%%%%%%%%%%%%%%%%%%%%%%%%%%%%%%%%%
% \section{Implementation}
%\iffalse
%<*package>
%\fi
%
% This section describes the definitions file |childdoc.def|.

% The definitions cannot be loaded using |\usepackage| or |\RequirePackage|
% which has a mechanism to prevent loading a style file more than once.
% When loading the definitions by means of |\input|
% multiple instances have to be prevented manually:
%\iffalse
%This code needs to be before the `\ProvidesFile' directive
%which is defined at the beginning of this file.
%Therefore it is also placed there and commented out here.
%</package>
%<*discard>
%\fi
%    \begin{macrocode}
\ifdefined\childdocmain\endinput\fi
%    \end{macrocode}
%\iffalse
%</discard>
%<*package>
%\fi
%
% \macro{\ifchilddoc}
% \macro{\ifchilddocmanual}
% The conditional |\ifchilddoc| tells whether a
% child (true) or main (false) document is being compiled.
% The conditional |\ifchilddocmanual| tells whether
% the |\includeonly| mechanism is used (false) or
% the selection of child files must be performed manually (true).
% The definitions initialise to false:
%    \begin{macrocode}
\newif\ifchilddoc
\newif\ifchilddocmanual
%    \end{macrocode}

% \macro{\childdocname}
% \macro{\childdocjob}
% The macro |\childdocname| stores the name of the main document
% to be compiled. The macro |\childdocjob| stores the name of
% the document on which the \LaTeX{} compiler was originally invoked.
% The content of |\jobname| cannot be compared
% to filenames specified in the source due to different catcodes.
% The following code rescans |\jobname|, stores the result
% in |\childdocname| and saves a copy in |\childdocjob|:
%    \begin{macrocode}
\edef\childdocname{\scantokens\expandafter{\jobname\noexpand}}
\let\childdocjob\childdocname
%    \end{macrocode}

% \macro{\childdocdisable}
% The macro |\childdocdisable| prevents the main file
% from being processed more than once.
% At this stage, the main document command |\childdocmain|
% is assumed to be called once again where it should do nothing.
% Any subsequent call to it should prevent
% a secondary processing of the main document
% It overwrites the forwarding commands
% |\childdocof| and |\childdocforward|
% with empty macros to prevent further inclusions of the main document:
%    \begin{macrocode}
\newcommand{\childdocdisable}
{
  \renewcommand{\childdocmain}[1]{\renewcommand{\childdocmain}[1]{\endinput}}
  \renewcommand{\childdocof}[1]{}
  \renewcommand{\childdocby}[2][]{}
  \renewcommand{\childdocforward}[2][]{}
  \renewcommand{\childdocdisable}{}
}
%    \end{macrocode}

% \macro{\childdocmain}
% The macro |\childdocmain| is to be called at the top of the main file
% with nothing or the main filename (without extension) as argument.
% First, it breaks loops.
% If the argument is not empty and does not match |\childdocname|
% (which is set by the first inclusion of |childdoc.def|),
% |\ifchilddoc| is set to true, |\includeonly| is applied to the child file
% and |\jobname| is set to the main file
% (for proper handling of |.aux| files):
%    \begin{macrocode}
\newcommand{\childdocmain}[1]
{
  \childdocdisable\childdocmain{}
  \if?#1?\else
    \begingroup
      \def\childdoctmp{#1}
      \ifx\childdoctmp\childdocname
        \def\childdoctmp{}
      \else
        \def\childdoctmp
        {
          \childdoctrue
          \includeonly{\childdocname}
          \def\childdocjob{#1}
          \def\jobname{#1}
        }
      \fi
      \expandafter
    \endgroup
    \childdoctmp
  \fi
}
%    \end{macrocode}

% \macro{\childdocof}
% The command |\childdocof| redirects
% compilation to the main file |#1|.
%    \begin{macrocode}
\newcommand{\childdocof}[1]
{
  \childdocdisable
  \childdoctrue
  \includeonly{\childdocname}
  \def\jobname{#1}
  \def\childdocjob{#1}
  \input{#1}
}
%    \end{macrocode}

% \macro{\childdocby}
% The command |\childdocby| ....
%    \begin{macrocode}
\newcommand{\childdocby}[2][]
{
  \childdocdisable
  \childdoctrue
  \childdocmanualtrue
  \if?#1?\else
    \def\jobname{#2}
  \fi
  \def\childdocjob{#2}
  \input{#2}
  \endinput
}
%    \end{macrocode}

% \macro{\childdocforward}
% The command |\childdocforward| redirects
% compilation to the main file or
% (if the optional argument is given) a child file.
% Parameters are set as if the main file
% or a child file starting with |\childdocof| was compiled.
% Then compilation is handed over to the main file:
%    \begin{macrocode}
\newcommand{\childdocforward}[2][]
{
  \begingroup
    \if?#1?
      \def\childdoctmp
      {
        \def\childdocname{#2}
        \def\childdocjob{#2}
        \def\jobname{#2}
        \input{#2}
        \endinput
      }
    \else
      \def\childdoctmp
      {
        \childdocdisable
        \def\childdocname{#2}
        \childdoctrue
        \includeonly{#2}
        \def\childdocjob{#1}
        \def\jobname{#1}
        \input{#1}
        \endinput
      }
    \fi
    \expandafter
  \endgroup
  \childdoctmp
}
%    \end{macrocode}

% \macro{\childdocforwardprefix}
% The command |\childdocforwardprefix| redirects
% compilation to the main or a child file by means of a pattern.
% The prefix |#1| in the current filename is replaced by |#2|
% and the suffix of the current filename is kept
% (it is assumed that the filename does not contain the substring `|~~~|'
% which is used as a delimiter).
% Compilation is handed over to the new file by |\childdocforward|:
%    \begin{macrocode}
\newcommand{\childdocforwardprefix}[3][]
{
  \begingroup
    \def\childdocextract #2##1~~~{\def\childdoctmp{\childdocforward[#1]{#3##1}}}
    \expandafter\childdocextract\childdocname~~~
    \expandafter
  \endgroup
  \childdoctmp
}
%    \end{macrocode}

% \macro{\childdoc}
% The deprecated macro |\childdoc| is a legacy version of |\childdocmain|:
%    \begin{macrocode}
\newcommand{\childdoc}{\childdocmain}
%    \end{macrocode}

% \macro{\childdocredirect}
% The deprecated macro |\childdocredirect| is a legacy version
% of |\childdocforward| and |\childdocforwardprefix|:
%    \begin{macrocode}
\newcommand{\childdocredirect}[2][]
{
  \begingroup
    \if?#1?
      \def\childdoctmp{\childdocforward{#2}}
    \else
      \def\childdoctmp{\childdocforwardprefix{#1}{#2}}
    \fi
    \expandafter
  \endgroup
  \childdoctmp
}
%    \end{macrocode}

%\iffalse
%</package>
%\fi
%
\endinput

\childdocforward{cdocsamp}
%    \end{macrocode}

%\iffalse
%</sampledraft>
%\fi
%
% %%%%%%%%%%%%%%%%%%%%%%%%%%%%%%%%%%%%%%
% \paragraph{Forwarding for Final Version of the Chapters.}
%
% The following forwarding files |cdocsfn1.tex| and |cdocsfn2.tex|
% (with identical content)
% compile the final versions of the child documents
% |cdocsch1.tex| and |cdocsch2.tex|, respectively:
%\iffalse
%<*samplefinal>
%\fi
%    \begin{macrocode}
\def\version{final}
% \iffalse
%
% childdoc.dtx Copyright (C) 2017-2018 Niklas Beisert
%
% This work may be distributed and/or modified under the
% conditions of the LaTeX Project Public License, either version 1.3
% of this license or (at your option) any later version.
% The latest version of this license is in
%   http://www.latex-project.org/lppl.txt
% and version 1.3 or later is part of all distributions of LaTeX
% version 2005/12/01 or later.
%
% This work has the LPPL maintenance status `maintained'.
%
% The Current Maintainer of this work is Niklas Beisert.
%
% This work consists of the files childdoc.dtx and childdoc.ins
% and the derived files childdoc.def and cdocsamp.tex with
% cdocsch1.tex, cdocsch2.tex, cdocsdrf.tex, cdocsfn1.tex, cdocsfn2.tex.
%
%<package>\ifdefined\childdocmain\endinput\fi
%<package>\ProvidesFile{childdoc.def}[2018/12/30 v2.0 child document driver]
%<samplemain>\ProvidesFile{cdocsamp.tex}[2018/12/30 v2.0 sample for childdoc]
%<*driver>
%\ProvidesFile{childdoc.drv}[2018/12/30 v2.0 childdoc reference manual file]
\PassOptionsToClass{10pt,a4paper}{article}
\documentclass{ltxdoc}

\usepackage[margin=35mm]{geometry}
\usepackage{hyperref}
\usepackage{hyperxmp}
\usepackage[usenames]{color}

\hypersetup{colorlinks=true}
\hypersetup{pdfstartview=FitH}
\hypersetup{pdfpagemode=UseNone}
\hypersetup{pdfsource={}}
\hypersetup{pdflang={en-UK}}
\hypersetup{pdfcopyright={Copyright 2017-2018 Niklas Beisert.
  This work may be distributed and/or modified under the
  conditions of the LaTeX Project Public License, either version 1.3
  of this license or (at your option) any later version.}}
\hypersetup{pdflicenseurl={http://www.latex-project.org/lppl.txt}}
\hypersetup{pdfcontactaddress={ETH Zurich, ITP, HIT K,
  Wolfgang-Pauli-Strasse 27}}
\hypersetup{pdfcontactpostcode={8093}}
\hypersetup{pdfcontactcity={Zurich}}
\hypersetup{pdfcontactcountry={Switzerland}}
\hypersetup{pdfcontactemail={nbeisert@itp.phys.ethz.ch}}
\hypersetup{pdfcontacturl={http://people.phys.ethz.ch/\xmptilde nbeisert/}}

\newcommand{\secref}[1]{\hyperref[#1]{section \ref*{#1}}}

\parskip1ex
\parindent0pt
\let\olditemize\itemize
\def\itemize{\olditemize\parskip0pt}

\begin{document}

\title{The \textsf{childdoc} Package}
\hypersetup{pdftitle={The childdoc Package}}
\author{Niklas Beisert\\[2ex]
  Institut f\"ur Theoretische Physik\\
  Eidgen\"ossische Technische Hochschule Z\"urich\\
  Wolfgang-Pauli-Strasse 27, 8093 Z\"urich, Switzerland\\[1ex]
  \href{mailto:nbeisert@itp.phys.ethz.ch}
  {\texttt{nbeisert@itp.phys.ethz.ch}}}
\hypersetup{pdfauthor={Niklas Beisert}}
\hypersetup{pdfsubject={Manual for the LaTeX2e Package childdoc}}
\date{30 December 2018, \textsf{v2.0}}
\maketitle

\begin{abstract}\noindent
\textsf{childdoc} is a \LaTeXe{} package
that enables the direct compilation
of document sections included by |\include|
to individual files.
\end{abstract}

\begingroup
\parskip0ex
\tableofcontents
\endgroup

%%%%%%%%%%%%%%%%%%%%%%%%%%%%%%%%%%%%%%%%%%%%%%%%%%%%%%%%%%%%%%%%%%%%%%%%%%%%%%%%
%%%%%%%%%%%%%%%%%%%%%%%%%%%%%%%%%%%%%%%%%%%%%%%%%%%%%%%%%%%%%%%%%%%%%%%%%%%%%%%%
\section{Introduction}

\LaTeX{} provides a mechanism to structure a large document (such as a book)
into a main file and several child files (containing the chapters)
using the |\include| command.
This mechanism is beneficial for documents
which span hundreds of pages in order to
make the source file(s) more manageable.
Moreover, compilation can be restricted to
selected child files by means of the |\includeonly| command.
The latter feature can be used to reduce the compilation time while editing
(this was significantly more useful in the earlier days of \LaTeX{})
or to generate a smaller document which is easier to navigate.
Another application of |\includeonly| is to generate
documents consisting of selected parts of the complete document.

However, there are a few drawbacks of the plain |\include| mechanism:
\begin{itemize}
\item
The child files cannot be compiled on their own,
they can only be compiled via the main file.
A naive editing environment
(such as a text editor with an option
to have the current file processed by \LaTeX)
may require one to switch to the main file before compiling;
attempting to compile the child file produces errors.
\item
The main file must be modified (each time)
to adjust the |\includeonly| command
to the present needs. This easily leaves the main file in a messy state.
\item
The generated document will always carry the filename
of the main document. This is inconvenient if
several child files are to be compiled and
to be kept for distribution.
\end{itemize}

The present package provides a simple interface
to make child files individually compilable by \LaTeX{}.
Compiling a child file then has the same effect as compiling
the main file with an |\includeonly| command
to select the appropriate child.
Moreover the generated document will carry the name of the child
rather than the main file.
This resolves all three above issues.

This feature is meant to make the editing of books,
thesis documents and lecture notes somewhat more convenient.
However, the package can also be used efficiently for
composing a series of documents (such as exercise sheets)
which are typically distributed individually.
It then assists the author in generating the individual documents
(potentially in different versions)
as well as a document containing the collected series.
Another application is in developing style files
or other kinds of included material
where compilation of the style file could redirect
to a sample or test file.

%%%%%%%%%%%%%%%%%%%%%%%%%%%%%%%%%%%%%%%%%%%%%%%%%%%%%%%%%%%%%%%%%%%%%%%%%%%%%%%%
%%%%%%%%%%%%%%%%%%%%%%%%%%%%%%%%%%%%%%%%%%%%%%%%%%%%%%%%%%%%%%%%%%%%%%%%%%%%%%%%
\section{Usage}

First of all, the package \textsf{childdoc} is \emph{not} a standard
\LaTeXe{} |.sty| style file! Therefore it needs to be invoked in
a non-standard way.

%%%%%%%%%%%%%%%%%%%%%%%%%%%%%%%%%%%%%%%%%%%%%%%%%%%%%%%%%%%%%%%%%%%%%%%%%%%%%%%%
\subsection{Included Files}
\label{sec:include}

%%%%%%%%%%%%%%%%%%%%%%%%%%%%%%%%%%%%%%%%
\DescribeMacro{\childdocmain}
To use the package, add the commands
\begin{center}
\begin{tabular}{l}
|\input{childdoc.def}|\\
|\childdocmain{}|\\
\end{tabular}
\end{center}
at the very top of the main \LaTeX{} file,
in particular \emph{before} the |\documentclass| statement!
The argument of |\childdocmain| should be left empty
(but it must be present).

%%%%%%%%%%%%%%%%%%%%%%%%%%%%%%%%%%%%%%%%
\DescribeMacro{\childdocof}
Furthermore, add the commands
\begin{center}
\begin{tabular}{l}
|\input{childdoc.def}|\\
|\childdocof{|\textit{main}|}|\\
\end{tabular}
\end{center}
at the top of every child file \textit{child}
which is included by |\include{|\textit{child}|}|
from within the main file
(or at least for those files to be compiled individually).
The argument \textit{main} must be the filename of the main file.

There are a couple of
considerations in setting up the main and child documents:

%%%%%%%%%%%%%%%%%%%%%%%%%%%%%%%%%%%%%%%%
\paragraph{Restrictions.}

Please note the following restrictions:
\begin{itemize}
\item
|\childdocmain| must be called with one argument \textit{main}
to ensure compatibility with earlier version of the package.
It must either be empty (|\childdocmain{}|)
or precisely match the filename of the main file in which it is specified.
See \secref{sec:detection} for further information.
\item
The filename \textit{main} must be specified without the |.tex| extension.
\item
The filename \textit{main} is case sensitive
(even in case-insensitive file systems)
due to internal string comparison.
\item
The argument \textit{main} should be fully expanded, it cannot be a macro.
\item
Subdirectories and special characters should be avoided in filenames.
\item
The command |\childdocmain{|\textit{main}|}| must be followed by a whitespace.
It should not be followed immediately by another command
or by a comment mark `|%|'.
This is because the \TeX{} parser reads the token immediately following
the argument of |\childdocmain| and puts it
at the beginning of every child section;
however, a white\-space is ignored.
\end{itemize}

%%%%%%%%%%%%%%%%%%%%%%%%%%%%%%%%%%%%%%%%
\paragraph{Content of Main File.}

It is advisable to place all content in the child files included by |\include|.
Any output contained in the main file will appear in all child documents
unless suppressed manually;
it cannot be suppressed automatically by the |\includeonly| directive
and thus should normally be avoided.
A method to include some content in the main file
by means of conditional processing is described in \secref{sec:conditional}.

%%%%%%%%%%%%%%%%%%%%%%%%%%%%%%%%%%%%%%%%
\paragraph{Page Numbering.}

When only a part of the document is compiled,
the appropriate numbering of pages
(as well as other status parameters)
is determined from the |.aux| files.
The latter contain information from previous passes.
However this information needs to propagate through
all intermediate child documents.
Therefore the page numbering in child documents may well
be inconsistent until the complete document is compiled at least once.

A useful (if unconventional) way to always ensure a consistent
page numbering is to restart the numbering in each child document
and denote the pages by `\textit{child}|.|\textit{page}'
where \textit{child} represents the chapter/section number of the child file.
This can be achieved by the command
|\numberwithin{page}{|\textit{child}|}|
of the \textsf{amsmath} package
where \textit{child} can be |chapter| or |section|
depending on the chosen structuring.
Alternatively, one can modify the macro |\thepage| appropriately
and reset the counter |page| at the start of each child file.

%%%%%%%%%%%%%%%%%%%%%%%%%%%%%%%%%%%%%%%%%%%%%%%%%%%%%%%%%%%%%%%%%%%%%%%%%%%%%%%%
\subsection{Conditional Processing}
\label{sec:conditional}

The package provides a mechanism to compile different versions
of a document. To customise the versions further some conditional processing
can come in handy to distinguish which version is being compiled.
The package provides two macros to describe the compilation context:

%%%%%%%%%%%%%%%%%%%%%%%%%%%%%%%%%%%%%%%%
\DescribeMacro{\ifchilddoc}
The conditional |\ifchilddoc| distinguishes between the compilation of
child documents and the main document:
%
\begin{center}
|\ifchilddoc |\textit{child-code}| |[|\||else |\textit{main-code}]| \||fi|
\end{center}

%%%%%%%%%%%%%%%%%%%%%%%%%%%%%%%%%%%%%%%%
\DescribeMacro{\childdocname}
\DescribeMacro{\childdocjob}
The macro |\childdocname| contains the filename (without extension)
of the main or child file being processed.
Note that |\childdocjob| will always contain the name of the main file.

%%%%%%%%%%%%%%%%%%%%%%%%%%%%%%%%%%%%%%%%
\paragraph{Title Page.}

Conditional processing can be used to include a title or banner page
in the main document when proper precautions are taken.
Importantly, the code in the main file should ensure that the page counter
(as well as other status parameters which are stored in the |.aux| files)
takes the same value after the conditional processing.
Otherwise the page numbers may take divergent values
depending on which part is compiled.

For example, a title page could be declared by:
%
\begin{center}
\begin{tabular}{l}
|\ifchilddoc\||else|\\
|\addtocounter{page}{-1}|\\
\textit{code for title page}\\
|\newpage|\\
|\||fi|
\end{tabular}
\end{center}
%
A banner page for the child documents can be generated by:
%
\begin{center}
\begin{tabular}{l}
|\ifchilddoc|\\
|\addtocounter{page}{-1}|\\
\textit{code for banner page}\\
|\newpage|\\
|\||fi|
\end{tabular}
\end{center}
%
Here one could write a message such as:
\begin{center}
|This is the part \childdocname{} of \childdocjob{}.|
\end{center}

%%%%%%%%%%%%%%%%%%%%%%%%%%%%%%%%%%%%%%%%%%%%%%%%%%%%%%%%%%%%%%%%%%%%%%%%%%%%%%%%
\subsection{Flags}
\label{sec:flags}

The package makes it easy to generate different versions
of the main or child documents.
To this end compilation flags can be defined
and assigned different default values.
They will be particularly useful in conjunction
with the forwarding mechanism described in \secref{sec:forward}.

For example, it may be useful to have a flag |\version|
which can be set to |draft| or |final|.
The document source will contain some conditional code
depending on the value of |\version|.
Suppose further, the flag should default to |final| for the main file
and to |draft| for child files
which is a natural assignment for editing the document.
This is achieved by placing the following code
in the preamble of the main document
(below the |\childdocmain| directive):
%
\begin{center}
\begin{tabular}{l}
|\ifchilddoc|\\
|\providecommand{\version}{draft}|\\
|\||else|\\
|\providecommand{\version}{final}|\\
|\||fi|
\end{tabular}
\end{center}
%
The definition by |\providecommand| makes sure
that previous definitions are not overwritten.
Further statements |\providecommand{\version}{...}|
can thus be added before the above code to override it.

For the main file, one might add a line
(between |\childdocmain| and the above block)
%
\begin{center}
|%\ifchilddoc\||else\providecommand{\version}{draft}\||fi|
\end{center}
%
which can be uncommented to produce a draft version.
Likewise one can add a line to the very top of a child file
(above the |\childdocof{|\textit{main}|}| directive)
%
\begin{center}
|%\providecommand{\version}{final}|
\end{center}
%
which can be uncommented to produce the final version of this child document.

%%%%%%%%%%%%%%%%%%%%%%%%%%%%%%%%%%%%%%%%%%%%%%%%%%%%%%%%%%%%%%%%%%%%%%%%%%%%%%%%
\subsection{Forwarding}
\label{sec:forward}

Different versions of the main or child documents
using compilation flags as described in \secref{sec:flags}
can be (permanently) stored in different files
for convenient compilation, viewing and distribution.
To this end, the package defines a command
to pass on compilation to a different file:

%%%%%%%%%%%%%%%%%%%%%%%%%%%%%%%%%%%%%%%%
\DescribeMacro{\childdocforward}
The command |\childdocforward| redirects processing to
another source file:
%
\begin{center}
\begin{tabular}{l}
|\input{childdoc.def}|\\
|\childdocforward[|\textit{main}|]{|\textit{dest}|}|\\
\end{tabular}
\end{center}
%
The argument \textit{dest} is the destination file
(without extension).
It should be the main file or one of the child files.
Note that further \textsf{childdoc} directives
such as |\childdocof| and |\childdocforward|
in the indicated file will be processed in this form.
The optional argument \textit{main}
passes on directly to the main file \textit{main}
while pretending to compile the child \textit{dest}.
This form behaves as if \textit{dest}
issues |\childdocof{|\textit{main}|}| right away,
and no further \textsf{childdoc} directives will be processed.

%%%%%%%%%%%%%%%%%%%%%%%%%%%%%%%%%%%%%%%%
\DescribeMacro{\...prefix}
In the alternative form |\childdocforwardprefix|,
%
\begin{center}
\begin{tabular}{l}
|\input{childdoc.def}|\\
|\childdocforwardprefix[|\textit{main}|]{|\textit{prefix}|}{|\textit{dest}|}|
\end{tabular}
\end{center}
%
the destination file is determined by a pattern
depending on the current file:
To make this work, the current file must be called
`{\textit{prefix}\hspace{0.2em}\textit{suffix}}'
with \textit{prefix} matching precisely the argument.
Processing is then passed on to the file
`{\textit{dest}\hspace{0.2em}\textit{suffix}}'.
Surely, the same effect is achieved by
directly specifying the
argument `{\textit{dest}\hspace{0.2em}\textit{suffix}}'
in the first form.
However, that requires to set up a different file
for each child. With the alternative form of the command
all these files can have exactly the same content
which simplifies setting them up and maintaining them.

For example, the following file |draft.tex|
with a compilation flag |\version| as described in \secref{sec:flags}
compiles the main document as a draft:
%
\begin{center}
\begin{tabular}{l}
|\def\version{draft}|\\
|\input{childdoc.def}|\\
|\childdocforward{|\textit{main}|}|
\end{tabular}
\end{center}
%
Likewise, the following files |final|\textit{nn}|.tex|
compile the final version of the child document
|child|\textit{nn}|.tex|:
%
\begin{center}
\begin{tabular}{l}
|\def\version{final}|\\
|\input{childdoc.def}|\\
|\childdocforwardprefix{final}{child}|
\end{tabular}
\end{center}
%

Note that when several versions of a main file and/or of each child file
are to be generated, it may be convenient to set up a |Makefile| or
shell script to automatise the process.

%%%%%%%%%%%%%%%%%%%%%%%%%%%%%%%%%%%%%%%%%%%%%%%%%%%%%%%%%%%%%%%%%%%%%%%%%%%%%%%%
\subsection{Command Line Processing}
\label{sec:commandline}

The effect of redirection files can also be achieved by invoking
the \LaTeX{} compiler with a more elaborate command line.
Most conveniently this should be done as part
of a shell script or a |Makefile|.

When using \textsf{childdoc} in the main file, the following
command lines effectively perform a redirection
(note that depending on the shell being used,
backslashes may have to be doubled: `|\|' $\to$ `|\\|'):
%
\begin{center}
|... -jobname "|\textit{target}|" |\\|"|[\textit{flags}]%
|\input{childdoc.def}\childdocforward[|\textit{main}|]{|\textit{dest}|}"|
\end{center}
%
Here \textit{target} is the name of the output file,
\textit{main} is the name of the main file
and \textit{dest} is the name of the main or child file to be processed
(all filenames without extensions).
The optional argument \textit{main} can be omitted
if \textit{main} matches \textit{dest}.
Optionally, compilation \textit{flags} can be defined via |\def| commands.
This command line makes the \TeX{} engine believe
it is compiling the file \textit{target}
whose content is specified as the latter parameter.
The provided code then forwards the processing to
\textit{main} or \textit{dest} as described in \secref{sec:forward}.

%%%%%%%%%%%%%%%%%%%%%%%%%%%%%%%%%%%%%%%%%%%%%%%%%%%%%%%%%%%%%%%%%%%%%%%%%%%%%%%%
\subsection{Include by Input}
\label{sec:input}

Including child documents by |\include| has some restrictions by design.
Most notably, the content of a child document always occupies
its own set of pages; pages cannot be shared between child documents.
Usually, this behaviour makes perfect sense
because each child document contain an essential part of the document.
However, in some situations it may be desirable to compose
a document from a collection of parts
without having mandatory page breaks between then.
For this case, the package
provides a mechanism to include parts
by |\input| which can also be processed individually.
However, by construction this mechanism
requires manual handling of the content to be output.

%%%%%%%%%%%%%%%%%%%%%%%%%%%%%%%%%%%%%%%%
\DescribeMacro{\ifchilddocmanual}
The main file should be prepared as usual, see \secref{sec:include}.
However, the document body must make a distinction
between processing of an individual part and of the main document, e.g.:
%
\begin{center}
\begin{tabular}{l}
|\ifchilddocmanual|\\
|\input{\childdocname}|\\
|\||else|\\
\textit{document body with }|\input{|\textit{part}|}|\\
|\||fi|
\end{tabular}
\end{center}
%
The conditional |\ifchilddocmanual| is true whenever
a part to be included by |\input| is being compiled,
and the name of the part is stored in |\childdocname|.

%%%%%%%%%%%%%%%%%%%%%%%%%%%%%%%%%%%%%%%%
\DescribeMacro{\childdocby}
Each part to be included by |\input| should start with:
%
\begin{center}
\begin{tabular}{l}
|\input{childdoc.def}|\\
|\childdocby{|\textit{main}|}|\\
\end{tabular}
\end{center}
%
The directive |\childdocby| is similar to |\childdocof|
described in \secref{sec:include},
but the subsequent selection of content must be done manually.
To that end, both |\ifchilddoc| and |\ifchilddocmanual|
will be true upon processing of a part,
and the name of the part is stored in |\childdocname|.
Note that |\jobname| will be set to the filename of the current part
so that each part receives an individual |.aux| file
that does not interfere with the |.aux| file(s) of the main document.
This behaviour can be altered by the alternative form
|\childdocby[*]{|\textit{main}|}| (with a non-empty optional argument)
which uses the |.aux| file of the main document
by setting |\jobname| to \textit{main}.

%%%%%%%%%%%%%%%%%%%%%%%%%%%%%%%%%%%%%%%%%%%%%%%%%%%%%%%%%%%%%%%%%%%%%%%%%%%%%%%%
\subsection{Driver Development}
\label{sec:driver}

The \textsf{childdoc} mechanism can also be use for the development
of definition files such as \LaTeX{} styles or classes.
This case differs from the above setup with multiple parts
included by |\include| in that no |\includeonly| should be invoked.
This can be achieved by starting the include file
(before |\ProvidesPackage|) with:
%
\begin{center}
\begin{tabular}{l}
|\input{childdoc.def}|\\
|\childdocforward{|\textit{main}|}|\\
\end{tabular}
\end{center}
%
or alternatively with:
%
\begin{center}
\begin{tabular}{l}
|\input{childdoc.def}|\\
|\childdocby{|\textit{main}|}|\\
\end{tabular}
\end{center}
%
Both forms have slightly different effects as described above.
The main file is prepared as usual, see \secref{sec:include}.

%%%%%%%%%%%%%%%%%%%%%%%%%%%%%%%%%%%%%%%%%%%%%%%%%%%%%%%%%%%%%%%%%%%%%%%%%%%%%%%%
\subsection{Legacy Detection}
\label{sec:detection}

The directive |\childdocmain| in the main file can detect
whether the complete document or merely a child is to be compiled
even without using the directive |\childdocof|.
This method is deprecated because it is less robust
and there is no compelling reason to use it;
it is merely provided for backward compatibility
and it may be removed in future versions.

If the detection mechanism is to be used,
it is mandatory to correctly specify
the filename of the main file as the argument of |\childdocmain|:
%
\begin{center}
\begin{tabular}{l}
|\input{childdoc.def}|\\
|\childdocmain{|\textit{main}|}|\\
\end{tabular}
\end{center}
%
If |\jobname| does not match the argument \textit{main} of |\childdocmain|,
it is assumed that |\jobname| points to the child file to be compiled.
When using |\childdocmain| with the main file specified as argument,
it suffices to start a child file
with just |\input{|\textit{main}|}|
without loading of the package and using |\childdocof|.
If instead all processing is done
with the appropriate \textsf{childdoc} directives,
the argument of \textit{main} of |\childdocmain| can be empty.

An alternative version of the command line processing described
in \secref{sec:commandline} using the detection mechanism reads:
%
\begin{center}
|... -jobname "|\textit{target}|" "|[\textit{flags}]%
[|\def\jobname{|\textit{dest}|}|]|\input{|\textit{main}|}"|
\end{center}

%%%%%%%%%%%%%%%%%%%%%%%%%%%%%%%%%%%%%%%%%%%%%%%%%%%%%%%%%%%%%%%%%%%%%%%%%%%%%%%%
\subsection{Manual Code}
\label{sec:manual}

In case one cannot be certain whether the definitions file |childdoc.def|
is installed on the target \TeX{} distribution
and one prefers not to ship it,
it is conceivable to paste a few relevant commands into the sources.

To that end, drop all statements |\input{childdoc.def}|
and perform the replacements as outlined below.
Instead of |\childdocmain{|\textit{main}|}| add the following code
to the top of the main file:
%
\begin{center}
\begin{tabular}{l}
|\||ifdefined\childdocname\endinput\||fi\newif\ifchilddoc|\\
|\edef\childdocname{\scantokens\expandafter{\jobname\noexpand}}|\\
|\def\childdocmain{|\textit{main}|}\||ifx\childdocmain\childdocname\||else|\\
|\childdoctrue\includeonly{\childdocname}\let\jobname\childdocmain\||fi|\\
\end{tabular}
\end{center}
%
Instead of |\childdocof{|\textit{main}|}| just include the main file
at the top of each child file:
%
\begin{center}
|\input{|\textit{main}|}|
\end{center}
%
A simple redirection |\childdocforward{|\textit{dest}|}| is achieved by:
%
\begin{center}
|\def\jobname{|\textit{dest}|}\input{\jobname}|
\end{center}
%
The redirection with prefix
|\childdocforwardprefix[|\textit{prefix}|]{|\textit{dest}|}|
is accomplished by:
%
\begin{center}
\begin{tabular}{l}
|{\edef\jobname{\scantokens\expandafter{\jobname\noexpand}}|\\
|\def\redirectjob |\textit{prefix}|#1~~~{\gdef\jobname{|\textit{dest}|#1}}|\\
|\expandafter\redirectjob\jobname~~~}\input{\jobname}|
\end{tabular}
\end{center}

In an alternative approach,
child documents can be compiled by a specific command line
without additional code or specific definitions:
%
\begin{center}
|... -jobname "|\textit{target}|" "|[\textit{flags}]%
|\includeonly{|\textit{dest}|}\input{|\textit{main}|}"|
\end{center}
%

%%%%%%%%%%%%%%%%%%%%%%%%%%%%%%%%%%%%%%%%%%%%%%%%%%%%%%%%%%%%%%%%%%%%%%%%%%%%%%%%
%%%%%%%%%%%%%%%%%%%%%%%%%%%%%%%%%%%%%%%%%%%%%%%%%%%%%%%%%%%%%%%%%%%%%%%%%%%%%%%%
\section{Information}

%%%%%%%%%%%%%%%%%%%%%%%%%%%%%%%%%%%%%%%%%%%%%%%%%%%%%%%%%%%%%%%%%%%%%%%%%%%%%%%%
\subsection{Copyright}

Copyright \copyright{} 2017--2018 Niklas Beisert

This work may be distributed and/or modified under the
conditions of the \LaTeX{} Project Public License, either version 1.3
of this license or (at your option) any later version.
The latest version of this license is in
  \url{http://www.latex-project.org/lppl.txt}
and version 1.3 or later is part of all distributions of \LaTeX{}
version 2005/12/01 or later.

This work has the LPPL maintenance status `maintained'.

The Current Maintainer of this work is Niklas Beisert.

This work consists of the files |README.txt|, |childdoc.ins| and |childdoc.dtx|
as well as the derived files |childdoc.def|, |cdocsamp.tex|
with |cdocsch1.tex|, |cdocsch2.tex|, |cdocspt3.tex|, |cdocspt4.tex|,
|cdocsdrf.tex|, |cdocsfn1.tex|, |cdocsfn2.tex|
as well as |childdoc.pdf|.

%%%%%%%%%%%%%%%%%%%%%%%%%%%%%%%%%%%%%%%%%%%%%%%%%%%%%%%%%%%%%%%%%%%%%%%%%%%%%%%%
\subsection{Files and Installation}

The package consists of the files:
%
\begin{center}
\begin{tabular}{ll}
    |README.txt|   & readme file \\
    |childdoc.ins| & installation file \\
    |childdoc.dtx| & source file \\
    |childdoc.def| & definition file \\
    |cdocsamp.tex| & sample main file \\
    |cdocsch1.tex| & sample include file \\
    |cdocsch2.tex| & sample include file \\
    |cdocspt3.tex| & sample part file \\
    |cdocspt4.tex| & sample part file \\
    |cdocsdrf.tex| & sample redirection file \\
    |cdocsfn1.tex| & sample redirection file \\
    |cdocsfn2.tex| & sample redirection file \\
    |childdoc.pdf| & manual
\end{tabular}
\end{center}
%
The distribution consists of the files
|README.txt|, |childdoc.ins| and |childdoc.dtx|.
%
\begin{itemize}
\item
Run (pdf)\LaTeX{} on |childdoc.dtx|
to compile the manual |childdoc.pdf| (this file).
\item
Run \LaTeX{} on |childdoc.ins| to create the definitions file |childdoc.def|
and the sample |cdocsamp.tex| with include files
|cdocsch1.tex|, |cdocsch2.tex|, |cdocspt3.tex|, |cdocspt4.tex|,
|cdocsdrf.tex|, |cdocsfn1.tex|, |cdocsfn2.tex|.
Then copy the file |childdoc.def| to an appropriate directory of your \LaTeX{}
distribution, e.g.\ \textit{texmf-root}|/tex/latex/childdoc|.
\end{itemize}

%%%%%%%%%%%%%%%%%%%%%%%%%%%%%%%%%%%%%%%%%%%%%%%%%%%%%%%%%%%%%%%%%%%%%%%%%%%%%%%%
\subsection{Related CTAN Packages}

There are several other packages which offer a similar functionality:
%
\begin{itemize}
\item
The packages
\href{http://ctan.org/pkg/docmute}{\textsf{docmute}},
\href{http://ctan.org/pkg/includex}{\textsf{includex}} and
\href{http://ctan.org/pkg/standalone}{\textsf{standalone}}
provide commands to include only the document body of
a child file thus allowing both files to be compiled individually.
\item
The packages \href{http://ctan.org/pkg/subdocs}{\textsf{subdocs}}
and \href{http://ctan.org/pkg/subfiles}{\textsf{subfiles}}
provide structures in which the main and child documents can be
encapsulated and allowing them to be compiled individually.
The inclusion mechanism is different from the conventional |\include|.
\item
The package \href{http://ctan.org/pkg/combine}{\textsf{combine}}
is an elaborate solution to combine several documents into one.
\end{itemize}
%
See also the CTAN topic \href{http://ctan.org/topic/subdocs}{\textsf{subdocs}}
for further related packages.
The present package differs from the above solutions in that
a document structure constructed with the conventional |\include| mechanism
just needs two extra commands at the top of every file
such that all constituent files can be compiled individually.

%%%%%%%%%%%%%%%%%%%%%%%%%%%%%%%%%%%%%%%%%%%%%%%%%%%%%%%%%%%%%%%%%%%%%%%%%%%%%%%%
%\subsection{Feature Suggestions}
%
%The following is a list of features which may be useful for future
%versions of this package:
%%
%\begin{itemize}
%\item
%\ldots
%\end{itemize}

%%%%%%%%%%%%%%%%%%%%%%%%%%%%%%%%%%%%%%%%%%%%%%%%%%%%%%%%%%%%%%%%%%%%%%%%%%%%%%%%
\subsection{Revision History}

%%%%%%%%%%%%%%%%%%%%%%%%%%%%%%%%%%%%%%%%
\paragraph{v2.0:} 2018/12/30

\begin{itemize}
\item
immediate forward processing
\item
added |\childdocby| mechanism
\item
manual restructured
\end{itemize}

%%%%%%%%%%%%%%%%%%%%%%%%%%%%%%%%%%%%%%%%
\paragraph{v1.6:} 2018/01/17

\begin{itemize}
\item
application for development of include files
\item
corrections to manual
\end{itemize}

%%%%%%%%%%%%%%%%%%%%%%%%%%%%%%%%%%%%%%%%
\paragraph{v1.5:} 2017/05/21

\begin{itemize}
\item
more complete structuring introduced
\item
|\childdocof| introduced
\item
|\childdoc| renamed to |\childdocmain|
\item
|\childredirect| renamed to |\childdocforward| and |\childdocforwardprefix|
and functionality expanded
\end{itemize}

%%%%%%%%%%%%%%%%%%%%%%%%%%%%%%%%%%%%%%%%
\paragraph{v1.0:} 2017/04/27

\begin{itemize}
\item
manual and install package
\item
first version published on CTAN
\end{itemize}

%%%%%%%%%%%%%%%%%%%%%%%%%%%%%%%%%%%%%%%%
\paragraph{v0.6:} 2017/04/26

\begin{itemize}
\item
redirection mechanism added
\end{itemize}

%%%%%%%%%%%%%%%%%%%%%%%%%%%%%%%%%%%%%%%%
\paragraph{v0.5:} 2017/04/26

\begin{itemize}
\item
functionality in definition file
\end{itemize}


%%%%%%%%%%%%%%%%%%%%%%%%%%%%%%%%%%%%%%%%%%%%%%%%%%%%%%%%%%%%%%%%%%%%%%%%%%%%%%%%
%%%%%%%%%%%%%%%%%%%%%%%%%%%%%%%%%%%%%%%%%%%%%%%%%%%%%%%%%%%%%%%%%%%%%%%%%%%%%%%%
%%%%%%%%%%%%%%%%%%%%%%%%%%%%%%%%%%%%%%%%%%%%%%%%%%%%%%%%%%%%%%%%%%%%%%%%%%%%%%%%
\appendix

\settowidth\MacroIndent{\rmfamily\scriptsize 000\ }

 \DocInput{childdoc.dtx}

\end{document}
%</driver>
% \fi
%
% %%%%%%%%%%%%%%%%%%%%%%%%%%%%%%%%%%%%%%%%%%%%%%%%%%%%%%%%%%%%%%%%%%%%%%%%%%%%%%
% %%%%%%%%%%%%%%%%%%%%%%%%%%%%%%%%%%%%%%%%%%%%%%%%%%%%%%%%%%%%%%%%%%%%%%%%%%%%%%
% \section{Sample}
%\iffalse
%<*samplemain>
%\fi
%
% The following presents a sample document
% with two chapters, two parts, a title page,
% a compile flag as well as three forwarding files to set the flag.
% It consists of eight |.tex| files:
% \begin{center}
% \begin{tabular}{ll}
% |cdocsamp.tex|&main file\\
% |cdocsch1.tex|&include file for chapter 1\\
% |cdocsch2.tex|&include file for chapter 2\\
% |cdocspt3.tex|&include file for part 3\\
% |cdocspt4.tex|&include file for part 4\\
% |cdocsdrf.tex|&forwarding file for main file in draft mode\\
% |cdocsfi1.tex|&forwarding file for final version of chapter 1\\
% |cdocsfi2.tex|&forwarding file for final version of chapter 2\\
% \end{tabular}
% \end{center}
% Each of the eight files can be compiled directly by the \LaTeX{} compiler.
%
% %%%%%%%%%%%%%%%%%%%%%%%%%%%%%%%%%%%%%%
% \paragraph{Main File.}
%
% The main file is called |cdocsamp.tex|.
%
% Load the \textsf{childdoc} definitions and
% declare the filename for the main document:
%    \begin{macrocode}
\input{childdoc.def}
\childdocmain{}
%    \end{macrocode}

% Optional override for |\version| flag:
%    \begin{macrocode}
%%\ifchilddoc\else\providecommand{\version}{draft}\fi
%    \end{macrocode}

% Define the default values for the |\version| flag
% (|final| for the main file and |draft| for childs):
%    \begin{macrocode}
\ifchilddoc
\providecommand{\version}{draft}
\else
\providecommand{\version}{final}
\fi
%    \end{macrocode}

% Load the standard document class:
%    \begin{macrocode}
\documentclass[12pt]{article}
%    \end{macrocode}

% Start the document body:
%    \begin{macrocode}
\begin{document}
%    \end{macrocode}

% Declare a title page.
% Print title, part of document being processed and version flag:
%    \begin{macrocode}
\addtocounter{page}{-1}
\begin{center}
{\LARGE\bfseries{}childdoc example\par}
\vspace{1cm}
\ifchilddoc
\ifchilddocmanual part\else chapter\fi:
`\childdocname' of `\childdocjob'\par
\else
main document: `\childdocjob'\par
\fi
version: \version\par
\end{center}
\newpage
%    \end{macrocode}

% Manually include selected file,
% otherwise process as usual:
%    \begin{macrocode}
\ifchilddocmanual
\section*{part `\childdocname'}
\input{\childdocname}
\else
%    \end{macrocode}

% Include the two chapters:
%    \begin{macrocode}
\include{cdocsch1}
\include{cdocsch2}
%    \end{macrocode}

% Include the two parts unless only chapters should be displayed:
%    \begin{macrocode}
\ifchilddoc\else
\section{part three}
\input{cdocspt3}
\section{part four}
\input{cdocspt4}
\fi
%    \end{macrocode}

% Process as usual until here:
%    \begin{macrocode}
\fi
%    \end{macrocode}

% End of document body:
%    \begin{macrocode}
\end{document}
%    \end{macrocode}
%\iffalse
%</samplemain>
%\fi
%
% %%%%%%%%%%%%%%%%%%%%%%%%%%%%%%%%%%%%%%
% \paragraph{Chapter Include Files.}
%
% The include files are called |cdocsch1.tex| and |cdocsch2.tex|.
%
%\iffalse
%<*samplechap1|samplechap2>
%\fi

% Optional override for |\version| flag:
%    \begin{macrocode}
%%\providecommand{\version}{final}
%    \end{macrocode}

% Include the main document:
%    \begin{macrocode}
\input{childdoc.def}
\childdocof{cdocsamp}
%    \end{macrocode}

%\iffalse
%</samplechap1|samplechap2>
%\fi
%
%\iffalse
%<*samplechap1>
%\fi
% Some text for chapter 1:
%    \begin{macrocode}
\section{one}
some text in chapter one
%    \end{macrocode}

%\iffalse
%</samplechap1>
%\fi
% Some text for chapter 2:
%\iffalse
%<*samplechap2>
%\fi
%    \begin{macrocode}
\section{two}
more text in chapter two
%    \end{macrocode}

%\iffalse
%</samplechap2>
%\fi
%
% %%%%%%%%%%%%%%%%%%%%%%%%%%%%%%%%%%%%%%
% \paragraph{Part Include Files.}
%
% The include files are called |cdocspt3.tex| and |cdocspt4.tex|.
%
%\iffalse
%<*samplepart3|samplepart4>
%\fi

% Optional override for |\version| flag:
%    \begin{macrocode}
%%\providecommand{\version}{final}
%    \end{macrocode}

% Include the main document:
%    \begin{macrocode}
\input{childdoc.def}
\childdocby{cdocsamp}
%    \end{macrocode}

%\iffalse
%</samplepart3|samplepart4>
%\fi
%
%\iffalse
%<*samplepart3>
%\fi
% Some text for part 3:
%    \begin{macrocode}
some text in part three
%    \end{macrocode}

%\iffalse
%</samplepart3>
%\fi
% Some text for part 4:
%\iffalse
%<*samplepart4>
%\fi
%    \begin{macrocode}
more text in part four
%    \end{macrocode}

%\iffalse
%</samplepart4>
%\fi
%
% %%%%%%%%%%%%%%%%%%%%%%%%%%%%%%%%%%%%%%
% \paragraph{Forwarding for a Complete Draft.}
%
% The following forwarding file |cdocsdrf.tex|
% compiles the main document in draft mode:
%\iffalse
%<*sampledraft>
%\fi
%    \begin{macrocode}
\def\version{draft}
\input{childdoc.def}
\childdocforward{cdocsamp}
%    \end{macrocode}

%\iffalse
%</sampledraft>
%\fi
%
% %%%%%%%%%%%%%%%%%%%%%%%%%%%%%%%%%%%%%%
% \paragraph{Forwarding for Final Version of the Chapters.}
%
% The following forwarding files |cdocsfn1.tex| and |cdocsfn2.tex|
% (with identical content)
% compile the final versions of the child documents
% |cdocsch1.tex| and |cdocsch2.tex|, respectively:
%\iffalse
%<*samplefinal>
%\fi
%    \begin{macrocode}
\def\version{final}
\input{childdoc.def}
\childdocforwardprefix[cdocsamp]{cdocsfn}{cdocsch}
%    \end{macrocode}

%\iffalse
%</samplefinal>
%\fi
%
% %%%%%%%%%%%%%%%%%%%%%%%%%%%%%%%%%%%%%%
% \paragraph{Command Line Processing.}
%
% The following three command lines generate the output files
% |cdocscld|, |cdocscl1| and |cdocscl2|
% which should be identical to
% |cdocsdrf|, |cdocsch1| and |cdocsfn2|, respectively:
% \begin{center}
% \begin{tabular}{l}
% |latex -jobname cdocscld \|\\
% |  "\def\version{draft}\input{childdoc.def}\childdocforward{cdocsamp}"|\\
% |latex -jobname cdocscl1 \|\\
% |  "\input{childdoc.def}\childdocforward[cdocsamp]{cdocsch1}"|\\
% |latex -jobname cdocscl2 \|\\
% |  "\def\version{final}\input{childdoc.def}\childdocforward{cdocsch2}"|
% \end{tabular}
% \end{center}
% Note that the trailing backslash on each first line
% merely continues the input to the second line
% (for convenient cut ant paste).
% Furthermore, the command |latex| can be replaced by any
% of its alternative versions such as |pdflatex|.
%
% %%%%%%%%%%%%%%%%%%%%%%%%%%%%%%%%%%%%%%%%%%%%%%%%%%%%%%%%%%%%%%%%%%%%%%%%%%%%%%
% %%%%%%%%%%%%%%%%%%%%%%%%%%%%%%%%%%%%%%%%%%%%%%%%%%%%%%%%%%%%%%%%%%%%%%%%%%%%%%
% \section{Implementation}
%\iffalse
%<*package>
%\fi
%
% This section describes the definitions file |childdoc.def|.

% The definitions cannot be loaded using |\usepackage| or |\RequirePackage|
% which has a mechanism to prevent loading a style file more than once.
% When loading the definitions by means of |\input|
% multiple instances have to be prevented manually:
%\iffalse
%This code needs to be before the `\ProvidesFile' directive
%which is defined at the beginning of this file.
%Therefore it is also placed there and commented out here.
%</package>
%<*discard>
%\fi
%    \begin{macrocode}
\ifdefined\childdocmain\endinput\fi
%    \end{macrocode}
%\iffalse
%</discard>
%<*package>
%\fi
%
% \macro{\ifchilddoc}
% \macro{\ifchilddocmanual}
% The conditional |\ifchilddoc| tells whether a
% child (true) or main (false) document is being compiled.
% The conditional |\ifchilddocmanual| tells whether
% the |\includeonly| mechanism is used (false) or
% the selection of child files must be performed manually (true).
% The definitions initialise to false:
%    \begin{macrocode}
\newif\ifchilddoc
\newif\ifchilddocmanual
%    \end{macrocode}

% \macro{\childdocname}
% \macro{\childdocjob}
% The macro |\childdocname| stores the name of the main document
% to be compiled. The macro |\childdocjob| stores the name of
% the document on which the \LaTeX{} compiler was originally invoked.
% The content of |\jobname| cannot be compared
% to filenames specified in the source due to different catcodes.
% The following code rescans |\jobname|, stores the result
% in |\childdocname| and saves a copy in |\childdocjob|:
%    \begin{macrocode}
\edef\childdocname{\scantokens\expandafter{\jobname\noexpand}}
\let\childdocjob\childdocname
%    \end{macrocode}

% \macro{\childdocdisable}
% The macro |\childdocdisable| prevents the main file
% from being processed more than once.
% At this stage, the main document command |\childdocmain|
% is assumed to be called once again where it should do nothing.
% Any subsequent call to it should prevent
% a secondary processing of the main document
% It overwrites the forwarding commands
% |\childdocof| and |\childdocforward|
% with empty macros to prevent further inclusions of the main document:
%    \begin{macrocode}
\newcommand{\childdocdisable}
{
  \renewcommand{\childdocmain}[1]{\renewcommand{\childdocmain}[1]{\endinput}}
  \renewcommand{\childdocof}[1]{}
  \renewcommand{\childdocby}[2][]{}
  \renewcommand{\childdocforward}[2][]{}
  \renewcommand{\childdocdisable}{}
}
%    \end{macrocode}

% \macro{\childdocmain}
% The macro |\childdocmain| is to be called at the top of the main file
% with nothing or the main filename (without extension) as argument.
% First, it breaks loops.
% If the argument is not empty and does not match |\childdocname|
% (which is set by the first inclusion of |childdoc.def|),
% |\ifchilddoc| is set to true, |\includeonly| is applied to the child file
% and |\jobname| is set to the main file
% (for proper handling of |.aux| files):
%    \begin{macrocode}
\newcommand{\childdocmain}[1]
{
  \childdocdisable\childdocmain{}
  \if?#1?\else
    \begingroup
      \def\childdoctmp{#1}
      \ifx\childdoctmp\childdocname
        \def\childdoctmp{}
      \else
        \def\childdoctmp
        {
          \childdoctrue
          \includeonly{\childdocname}
          \def\childdocjob{#1}
          \def\jobname{#1}
        }
      \fi
      \expandafter
    \endgroup
    \childdoctmp
  \fi
}
%    \end{macrocode}

% \macro{\childdocof}
% The command |\childdocof| redirects
% compilation to the main file |#1|.
%    \begin{macrocode}
\newcommand{\childdocof}[1]
{
  \childdocdisable
  \childdoctrue
  \includeonly{\childdocname}
  \def\jobname{#1}
  \def\childdocjob{#1}
  \input{#1}
}
%    \end{macrocode}

% \macro{\childdocby}
% The command |\childdocby| ....
%    \begin{macrocode}
\newcommand{\childdocby}[2][]
{
  \childdocdisable
  \childdoctrue
  \childdocmanualtrue
  \if?#1?\else
    \def\jobname{#2}
  \fi
  \def\childdocjob{#2}
  \input{#2}
  \endinput
}
%    \end{macrocode}

% \macro{\childdocforward}
% The command |\childdocforward| redirects
% compilation to the main file or
% (if the optional argument is given) a child file.
% Parameters are set as if the main file
% or a child file starting with |\childdocof| was compiled.
% Then compilation is handed over to the main file:
%    \begin{macrocode}
\newcommand{\childdocforward}[2][]
{
  \begingroup
    \if?#1?
      \def\childdoctmp
      {
        \def\childdocname{#2}
        \def\childdocjob{#2}
        \def\jobname{#2}
        \input{#2}
        \endinput
      }
    \else
      \def\childdoctmp
      {
        \childdocdisable
        \def\childdocname{#2}
        \childdoctrue
        \includeonly{#2}
        \def\childdocjob{#1}
        \def\jobname{#1}
        \input{#1}
        \endinput
      }
    \fi
    \expandafter
  \endgroup
  \childdoctmp
}
%    \end{macrocode}

% \macro{\childdocforwardprefix}
% The command |\childdocforwardprefix| redirects
% compilation to the main or a child file by means of a pattern.
% The prefix |#1| in the current filename is replaced by |#2|
% and the suffix of the current filename is kept
% (it is assumed that the filename does not contain the substring `|~~~|'
% which is used as a delimiter).
% Compilation is handed over to the new file by |\childdocforward|:
%    \begin{macrocode}
\newcommand{\childdocforwardprefix}[3][]
{
  \begingroup
    \def\childdocextract #2##1~~~{\def\childdoctmp{\childdocforward[#1]{#3##1}}}
    \expandafter\childdocextract\childdocname~~~
    \expandafter
  \endgroup
  \childdoctmp
}
%    \end{macrocode}

% \macro{\childdoc}
% The deprecated macro |\childdoc| is a legacy version of |\childdocmain|:
%    \begin{macrocode}
\newcommand{\childdoc}{\childdocmain}
%    \end{macrocode}

% \macro{\childdocredirect}
% The deprecated macro |\childdocredirect| is a legacy version
% of |\childdocforward| and |\childdocforwardprefix|:
%    \begin{macrocode}
\newcommand{\childdocredirect}[2][]
{
  \begingroup
    \if?#1?
      \def\childdoctmp{\childdocforward{#2}}
    \else
      \def\childdoctmp{\childdocforwardprefix{#1}{#2}}
    \fi
    \expandafter
  \endgroup
  \childdoctmp
}
%    \end{macrocode}

%\iffalse
%</package>
%\fi
%
\endinput

\childdocforwardprefix[cdocsamp]{cdocsfn}{cdocsch}
%    \end{macrocode}

%\iffalse
%</samplefinal>
%\fi
%
% %%%%%%%%%%%%%%%%%%%%%%%%%%%%%%%%%%%%%%
% \paragraph{Command Line Processing.}
%
% The following three command lines generate the output files
% |cdocscld|, |cdocscl1| and |cdocscl2|
% which should be identical to
% |cdocsdrf|, |cdocsch1| and |cdocsfn2|, respectively:
% \begin{center}
% \begin{tabular}{l}
% |latex -jobname cdocscld \|\\
% |  "\def\version{draft}% \iffalse
%
% childdoc.dtx Copyright (C) 2017-2018 Niklas Beisert
%
% This work may be distributed and/or modified under the
% conditions of the LaTeX Project Public License, either version 1.3
% of this license or (at your option) any later version.
% The latest version of this license is in
%   http://www.latex-project.org/lppl.txt
% and version 1.3 or later is part of all distributions of LaTeX
% version 2005/12/01 or later.
%
% This work has the LPPL maintenance status `maintained'.
%
% The Current Maintainer of this work is Niklas Beisert.
%
% This work consists of the files childdoc.dtx and childdoc.ins
% and the derived files childdoc.def and cdocsamp.tex with
% cdocsch1.tex, cdocsch2.tex, cdocsdrf.tex, cdocsfn1.tex, cdocsfn2.tex.
%
%<package>\ifdefined\childdocmain\endinput\fi
%<package>\ProvidesFile{childdoc.def}[2018/12/30 v2.0 child document driver]
%<samplemain>\ProvidesFile{cdocsamp.tex}[2018/12/30 v2.0 sample for childdoc]
%<*driver>
%\ProvidesFile{childdoc.drv}[2018/12/30 v2.0 childdoc reference manual file]
\PassOptionsToClass{10pt,a4paper}{article}
\documentclass{ltxdoc}

\usepackage[margin=35mm]{geometry}
\usepackage{hyperref}
\usepackage{hyperxmp}
\usepackage[usenames]{color}

\hypersetup{colorlinks=true}
\hypersetup{pdfstartview=FitH}
\hypersetup{pdfpagemode=UseNone}
\hypersetup{pdfsource={}}
\hypersetup{pdflang={en-UK}}
\hypersetup{pdfcopyright={Copyright 2017-2018 Niklas Beisert.
  This work may be distributed and/or modified under the
  conditions of the LaTeX Project Public License, either version 1.3
  of this license or (at your option) any later version.}}
\hypersetup{pdflicenseurl={http://www.latex-project.org/lppl.txt}}
\hypersetup{pdfcontactaddress={ETH Zurich, ITP, HIT K,
  Wolfgang-Pauli-Strasse 27}}
\hypersetup{pdfcontactpostcode={8093}}
\hypersetup{pdfcontactcity={Zurich}}
\hypersetup{pdfcontactcountry={Switzerland}}
\hypersetup{pdfcontactemail={nbeisert@itp.phys.ethz.ch}}
\hypersetup{pdfcontacturl={http://people.phys.ethz.ch/\xmptilde nbeisert/}}

\newcommand{\secref}[1]{\hyperref[#1]{section \ref*{#1}}}

\parskip1ex
\parindent0pt
\let\olditemize\itemize
\def\itemize{\olditemize\parskip0pt}

\begin{document}

\title{The \textsf{childdoc} Package}
\hypersetup{pdftitle={The childdoc Package}}
\author{Niklas Beisert\\[2ex]
  Institut f\"ur Theoretische Physik\\
  Eidgen\"ossische Technische Hochschule Z\"urich\\
  Wolfgang-Pauli-Strasse 27, 8093 Z\"urich, Switzerland\\[1ex]
  \href{mailto:nbeisert@itp.phys.ethz.ch}
  {\texttt{nbeisert@itp.phys.ethz.ch}}}
\hypersetup{pdfauthor={Niklas Beisert}}
\hypersetup{pdfsubject={Manual for the LaTeX2e Package childdoc}}
\date{30 December 2018, \textsf{v2.0}}
\maketitle

\begin{abstract}\noindent
\textsf{childdoc} is a \LaTeXe{} package
that enables the direct compilation
of document sections included by |\include|
to individual files.
\end{abstract}

\begingroup
\parskip0ex
\tableofcontents
\endgroup

%%%%%%%%%%%%%%%%%%%%%%%%%%%%%%%%%%%%%%%%%%%%%%%%%%%%%%%%%%%%%%%%%%%%%%%%%%%%%%%%
%%%%%%%%%%%%%%%%%%%%%%%%%%%%%%%%%%%%%%%%%%%%%%%%%%%%%%%%%%%%%%%%%%%%%%%%%%%%%%%%
\section{Introduction}

\LaTeX{} provides a mechanism to structure a large document (such as a book)
into a main file and several child files (containing the chapters)
using the |\include| command.
This mechanism is beneficial for documents
which span hundreds of pages in order to
make the source file(s) more manageable.
Moreover, compilation can be restricted to
selected child files by means of the |\includeonly| command.
The latter feature can be used to reduce the compilation time while editing
(this was significantly more useful in the earlier days of \LaTeX{})
or to generate a smaller document which is easier to navigate.
Another application of |\includeonly| is to generate
documents consisting of selected parts of the complete document.

However, there are a few drawbacks of the plain |\include| mechanism:
\begin{itemize}
\item
The child files cannot be compiled on their own,
they can only be compiled via the main file.
A naive editing environment
(such as a text editor with an option
to have the current file processed by \LaTeX)
may require one to switch to the main file before compiling;
attempting to compile the child file produces errors.
\item
The main file must be modified (each time)
to adjust the |\includeonly| command
to the present needs. This easily leaves the main file in a messy state.
\item
The generated document will always carry the filename
of the main document. This is inconvenient if
several child files are to be compiled and
to be kept for distribution.
\end{itemize}

The present package provides a simple interface
to make child files individually compilable by \LaTeX{}.
Compiling a child file then has the same effect as compiling
the main file with an |\includeonly| command
to select the appropriate child.
Moreover the generated document will carry the name of the child
rather than the main file.
This resolves all three above issues.

This feature is meant to make the editing of books,
thesis documents and lecture notes somewhat more convenient.
However, the package can also be used efficiently for
composing a series of documents (such as exercise sheets)
which are typically distributed individually.
It then assists the author in generating the individual documents
(potentially in different versions)
as well as a document containing the collected series.
Another application is in developing style files
or other kinds of included material
where compilation of the style file could redirect
to a sample or test file.

%%%%%%%%%%%%%%%%%%%%%%%%%%%%%%%%%%%%%%%%%%%%%%%%%%%%%%%%%%%%%%%%%%%%%%%%%%%%%%%%
%%%%%%%%%%%%%%%%%%%%%%%%%%%%%%%%%%%%%%%%%%%%%%%%%%%%%%%%%%%%%%%%%%%%%%%%%%%%%%%%
\section{Usage}

First of all, the package \textsf{childdoc} is \emph{not} a standard
\LaTeXe{} |.sty| style file! Therefore it needs to be invoked in
a non-standard way.

%%%%%%%%%%%%%%%%%%%%%%%%%%%%%%%%%%%%%%%%%%%%%%%%%%%%%%%%%%%%%%%%%%%%%%%%%%%%%%%%
\subsection{Included Files}
\label{sec:include}

%%%%%%%%%%%%%%%%%%%%%%%%%%%%%%%%%%%%%%%%
\DescribeMacro{\childdocmain}
To use the package, add the commands
\begin{center}
\begin{tabular}{l}
|\input{childdoc.def}|\\
|\childdocmain{}|\\
\end{tabular}
\end{center}
at the very top of the main \LaTeX{} file,
in particular \emph{before} the |\documentclass| statement!
The argument of |\childdocmain| should be left empty
(but it must be present).

%%%%%%%%%%%%%%%%%%%%%%%%%%%%%%%%%%%%%%%%
\DescribeMacro{\childdocof}
Furthermore, add the commands
\begin{center}
\begin{tabular}{l}
|\input{childdoc.def}|\\
|\childdocof{|\textit{main}|}|\\
\end{tabular}
\end{center}
at the top of every child file \textit{child}
which is included by |\include{|\textit{child}|}|
from within the main file
(or at least for those files to be compiled individually).
The argument \textit{main} must be the filename of the main file.

There are a couple of
considerations in setting up the main and child documents:

%%%%%%%%%%%%%%%%%%%%%%%%%%%%%%%%%%%%%%%%
\paragraph{Restrictions.}

Please note the following restrictions:
\begin{itemize}
\item
|\childdocmain| must be called with one argument \textit{main}
to ensure compatibility with earlier version of the package.
It must either be empty (|\childdocmain{}|)
or precisely match the filename of the main file in which it is specified.
See \secref{sec:detection} for further information.
\item
The filename \textit{main} must be specified without the |.tex| extension.
\item
The filename \textit{main} is case sensitive
(even in case-insensitive file systems)
due to internal string comparison.
\item
The argument \textit{main} should be fully expanded, it cannot be a macro.
\item
Subdirectories and special characters should be avoided in filenames.
\item
The command |\childdocmain{|\textit{main}|}| must be followed by a whitespace.
It should not be followed immediately by another command
or by a comment mark `|%|'.
This is because the \TeX{} parser reads the token immediately following
the argument of |\childdocmain| and puts it
at the beginning of every child section;
however, a white\-space is ignored.
\end{itemize}

%%%%%%%%%%%%%%%%%%%%%%%%%%%%%%%%%%%%%%%%
\paragraph{Content of Main File.}

It is advisable to place all content in the child files included by |\include|.
Any output contained in the main file will appear in all child documents
unless suppressed manually;
it cannot be suppressed automatically by the |\includeonly| directive
and thus should normally be avoided.
A method to include some content in the main file
by means of conditional processing is described in \secref{sec:conditional}.

%%%%%%%%%%%%%%%%%%%%%%%%%%%%%%%%%%%%%%%%
\paragraph{Page Numbering.}

When only a part of the document is compiled,
the appropriate numbering of pages
(as well as other status parameters)
is determined from the |.aux| files.
The latter contain information from previous passes.
However this information needs to propagate through
all intermediate child documents.
Therefore the page numbering in child documents may well
be inconsistent until the complete document is compiled at least once.

A useful (if unconventional) way to always ensure a consistent
page numbering is to restart the numbering in each child document
and denote the pages by `\textit{child}|.|\textit{page}'
where \textit{child} represents the chapter/section number of the child file.
This can be achieved by the command
|\numberwithin{page}{|\textit{child}|}|
of the \textsf{amsmath} package
where \textit{child} can be |chapter| or |section|
depending on the chosen structuring.
Alternatively, one can modify the macro |\thepage| appropriately
and reset the counter |page| at the start of each child file.

%%%%%%%%%%%%%%%%%%%%%%%%%%%%%%%%%%%%%%%%%%%%%%%%%%%%%%%%%%%%%%%%%%%%%%%%%%%%%%%%
\subsection{Conditional Processing}
\label{sec:conditional}

The package provides a mechanism to compile different versions
of a document. To customise the versions further some conditional processing
can come in handy to distinguish which version is being compiled.
The package provides two macros to describe the compilation context:

%%%%%%%%%%%%%%%%%%%%%%%%%%%%%%%%%%%%%%%%
\DescribeMacro{\ifchilddoc}
The conditional |\ifchilddoc| distinguishes between the compilation of
child documents and the main document:
%
\begin{center}
|\ifchilddoc |\textit{child-code}| |[|\||else |\textit{main-code}]| \||fi|
\end{center}

%%%%%%%%%%%%%%%%%%%%%%%%%%%%%%%%%%%%%%%%
\DescribeMacro{\childdocname}
\DescribeMacro{\childdocjob}
The macro |\childdocname| contains the filename (without extension)
of the main or child file being processed.
Note that |\childdocjob| will always contain the name of the main file.

%%%%%%%%%%%%%%%%%%%%%%%%%%%%%%%%%%%%%%%%
\paragraph{Title Page.}

Conditional processing can be used to include a title or banner page
in the main document when proper precautions are taken.
Importantly, the code in the main file should ensure that the page counter
(as well as other status parameters which are stored in the |.aux| files)
takes the same value after the conditional processing.
Otherwise the page numbers may take divergent values
depending on which part is compiled.

For example, a title page could be declared by:
%
\begin{center}
\begin{tabular}{l}
|\ifchilddoc\||else|\\
|\addtocounter{page}{-1}|\\
\textit{code for title page}\\
|\newpage|\\
|\||fi|
\end{tabular}
\end{center}
%
A banner page for the child documents can be generated by:
%
\begin{center}
\begin{tabular}{l}
|\ifchilddoc|\\
|\addtocounter{page}{-1}|\\
\textit{code for banner page}\\
|\newpage|\\
|\||fi|
\end{tabular}
\end{center}
%
Here one could write a message such as:
\begin{center}
|This is the part \childdocname{} of \childdocjob{}.|
\end{center}

%%%%%%%%%%%%%%%%%%%%%%%%%%%%%%%%%%%%%%%%%%%%%%%%%%%%%%%%%%%%%%%%%%%%%%%%%%%%%%%%
\subsection{Flags}
\label{sec:flags}

The package makes it easy to generate different versions
of the main or child documents.
To this end compilation flags can be defined
and assigned different default values.
They will be particularly useful in conjunction
with the forwarding mechanism described in \secref{sec:forward}.

For example, it may be useful to have a flag |\version|
which can be set to |draft| or |final|.
The document source will contain some conditional code
depending on the value of |\version|.
Suppose further, the flag should default to |final| for the main file
and to |draft| for child files
which is a natural assignment for editing the document.
This is achieved by placing the following code
in the preamble of the main document
(below the |\childdocmain| directive):
%
\begin{center}
\begin{tabular}{l}
|\ifchilddoc|\\
|\providecommand{\version}{draft}|\\
|\||else|\\
|\providecommand{\version}{final}|\\
|\||fi|
\end{tabular}
\end{center}
%
The definition by |\providecommand| makes sure
that previous definitions are not overwritten.
Further statements |\providecommand{\version}{...}|
can thus be added before the above code to override it.

For the main file, one might add a line
(between |\childdocmain| and the above block)
%
\begin{center}
|%\ifchilddoc\||else\providecommand{\version}{draft}\||fi|
\end{center}
%
which can be uncommented to produce a draft version.
Likewise one can add a line to the very top of a child file
(above the |\childdocof{|\textit{main}|}| directive)
%
\begin{center}
|%\providecommand{\version}{final}|
\end{center}
%
which can be uncommented to produce the final version of this child document.

%%%%%%%%%%%%%%%%%%%%%%%%%%%%%%%%%%%%%%%%%%%%%%%%%%%%%%%%%%%%%%%%%%%%%%%%%%%%%%%%
\subsection{Forwarding}
\label{sec:forward}

Different versions of the main or child documents
using compilation flags as described in \secref{sec:flags}
can be (permanently) stored in different files
for convenient compilation, viewing and distribution.
To this end, the package defines a command
to pass on compilation to a different file:

%%%%%%%%%%%%%%%%%%%%%%%%%%%%%%%%%%%%%%%%
\DescribeMacro{\childdocforward}
The command |\childdocforward| redirects processing to
another source file:
%
\begin{center}
\begin{tabular}{l}
|\input{childdoc.def}|\\
|\childdocforward[|\textit{main}|]{|\textit{dest}|}|\\
\end{tabular}
\end{center}
%
The argument \textit{dest} is the destination file
(without extension).
It should be the main file or one of the child files.
Note that further \textsf{childdoc} directives
such as |\childdocof| and |\childdocforward|
in the indicated file will be processed in this form.
The optional argument \textit{main}
passes on directly to the main file \textit{main}
while pretending to compile the child \textit{dest}.
This form behaves as if \textit{dest}
issues |\childdocof{|\textit{main}|}| right away,
and no further \textsf{childdoc} directives will be processed.

%%%%%%%%%%%%%%%%%%%%%%%%%%%%%%%%%%%%%%%%
\DescribeMacro{\...prefix}
In the alternative form |\childdocforwardprefix|,
%
\begin{center}
\begin{tabular}{l}
|\input{childdoc.def}|\\
|\childdocforwardprefix[|\textit{main}|]{|\textit{prefix}|}{|\textit{dest}|}|
\end{tabular}
\end{center}
%
the destination file is determined by a pattern
depending on the current file:
To make this work, the current file must be called
`{\textit{prefix}\hspace{0.2em}\textit{suffix}}'
with \textit{prefix} matching precisely the argument.
Processing is then passed on to the file
`{\textit{dest}\hspace{0.2em}\textit{suffix}}'.
Surely, the same effect is achieved by
directly specifying the
argument `{\textit{dest}\hspace{0.2em}\textit{suffix}}'
in the first form.
However, that requires to set up a different file
for each child. With the alternative form of the command
all these files can have exactly the same content
which simplifies setting them up and maintaining them.

For example, the following file |draft.tex|
with a compilation flag |\version| as described in \secref{sec:flags}
compiles the main document as a draft:
%
\begin{center}
\begin{tabular}{l}
|\def\version{draft}|\\
|\input{childdoc.def}|\\
|\childdocforward{|\textit{main}|}|
\end{tabular}
\end{center}
%
Likewise, the following files |final|\textit{nn}|.tex|
compile the final version of the child document
|child|\textit{nn}|.tex|:
%
\begin{center}
\begin{tabular}{l}
|\def\version{final}|\\
|\input{childdoc.def}|\\
|\childdocforwardprefix{final}{child}|
\end{tabular}
\end{center}
%

Note that when several versions of a main file and/or of each child file
are to be generated, it may be convenient to set up a |Makefile| or
shell script to automatise the process.

%%%%%%%%%%%%%%%%%%%%%%%%%%%%%%%%%%%%%%%%%%%%%%%%%%%%%%%%%%%%%%%%%%%%%%%%%%%%%%%%
\subsection{Command Line Processing}
\label{sec:commandline}

The effect of redirection files can also be achieved by invoking
the \LaTeX{} compiler with a more elaborate command line.
Most conveniently this should be done as part
of a shell script or a |Makefile|.

When using \textsf{childdoc} in the main file, the following
command lines effectively perform a redirection
(note that depending on the shell being used,
backslashes may have to be doubled: `|\|' $\to$ `|\\|'):
%
\begin{center}
|... -jobname "|\textit{target}|" |\\|"|[\textit{flags}]%
|\input{childdoc.def}\childdocforward[|\textit{main}|]{|\textit{dest}|}"|
\end{center}
%
Here \textit{target} is the name of the output file,
\textit{main} is the name of the main file
and \textit{dest} is the name of the main or child file to be processed
(all filenames without extensions).
The optional argument \textit{main} can be omitted
if \textit{main} matches \textit{dest}.
Optionally, compilation \textit{flags} can be defined via |\def| commands.
This command line makes the \TeX{} engine believe
it is compiling the file \textit{target}
whose content is specified as the latter parameter.
The provided code then forwards the processing to
\textit{main} or \textit{dest} as described in \secref{sec:forward}.

%%%%%%%%%%%%%%%%%%%%%%%%%%%%%%%%%%%%%%%%%%%%%%%%%%%%%%%%%%%%%%%%%%%%%%%%%%%%%%%%
\subsection{Include by Input}
\label{sec:input}

Including child documents by |\include| has some restrictions by design.
Most notably, the content of a child document always occupies
its own set of pages; pages cannot be shared between child documents.
Usually, this behaviour makes perfect sense
because each child document contain an essential part of the document.
However, in some situations it may be desirable to compose
a document from a collection of parts
without having mandatory page breaks between then.
For this case, the package
provides a mechanism to include parts
by |\input| which can also be processed individually.
However, by construction this mechanism
requires manual handling of the content to be output.

%%%%%%%%%%%%%%%%%%%%%%%%%%%%%%%%%%%%%%%%
\DescribeMacro{\ifchilddocmanual}
The main file should be prepared as usual, see \secref{sec:include}.
However, the document body must make a distinction
between processing of an individual part and of the main document, e.g.:
%
\begin{center}
\begin{tabular}{l}
|\ifchilddocmanual|\\
|\input{\childdocname}|\\
|\||else|\\
\textit{document body with }|\input{|\textit{part}|}|\\
|\||fi|
\end{tabular}
\end{center}
%
The conditional |\ifchilddocmanual| is true whenever
a part to be included by |\input| is being compiled,
and the name of the part is stored in |\childdocname|.

%%%%%%%%%%%%%%%%%%%%%%%%%%%%%%%%%%%%%%%%
\DescribeMacro{\childdocby}
Each part to be included by |\input| should start with:
%
\begin{center}
\begin{tabular}{l}
|\input{childdoc.def}|\\
|\childdocby{|\textit{main}|}|\\
\end{tabular}
\end{center}
%
The directive |\childdocby| is similar to |\childdocof|
described in \secref{sec:include},
but the subsequent selection of content must be done manually.
To that end, both |\ifchilddoc| and |\ifchilddocmanual|
will be true upon processing of a part,
and the name of the part is stored in |\childdocname|.
Note that |\jobname| will be set to the filename of the current part
so that each part receives an individual |.aux| file
that does not interfere with the |.aux| file(s) of the main document.
This behaviour can be altered by the alternative form
|\childdocby[*]{|\textit{main}|}| (with a non-empty optional argument)
which uses the |.aux| file of the main document
by setting |\jobname| to \textit{main}.

%%%%%%%%%%%%%%%%%%%%%%%%%%%%%%%%%%%%%%%%%%%%%%%%%%%%%%%%%%%%%%%%%%%%%%%%%%%%%%%%
\subsection{Driver Development}
\label{sec:driver}

The \textsf{childdoc} mechanism can also be use for the development
of definition files such as \LaTeX{} styles or classes.
This case differs from the above setup with multiple parts
included by |\include| in that no |\includeonly| should be invoked.
This can be achieved by starting the include file
(before |\ProvidesPackage|) with:
%
\begin{center}
\begin{tabular}{l}
|\input{childdoc.def}|\\
|\childdocforward{|\textit{main}|}|\\
\end{tabular}
\end{center}
%
or alternatively with:
%
\begin{center}
\begin{tabular}{l}
|\input{childdoc.def}|\\
|\childdocby{|\textit{main}|}|\\
\end{tabular}
\end{center}
%
Both forms have slightly different effects as described above.
The main file is prepared as usual, see \secref{sec:include}.

%%%%%%%%%%%%%%%%%%%%%%%%%%%%%%%%%%%%%%%%%%%%%%%%%%%%%%%%%%%%%%%%%%%%%%%%%%%%%%%%
\subsection{Legacy Detection}
\label{sec:detection}

The directive |\childdocmain| in the main file can detect
whether the complete document or merely a child is to be compiled
even without using the directive |\childdocof|.
This method is deprecated because it is less robust
and there is no compelling reason to use it;
it is merely provided for backward compatibility
and it may be removed in future versions.

If the detection mechanism is to be used,
it is mandatory to correctly specify
the filename of the main file as the argument of |\childdocmain|:
%
\begin{center}
\begin{tabular}{l}
|\input{childdoc.def}|\\
|\childdocmain{|\textit{main}|}|\\
\end{tabular}
\end{center}
%
If |\jobname| does not match the argument \textit{main} of |\childdocmain|,
it is assumed that |\jobname| points to the child file to be compiled.
When using |\childdocmain| with the main file specified as argument,
it suffices to start a child file
with just |\input{|\textit{main}|}|
without loading of the package and using |\childdocof|.
If instead all processing is done
with the appropriate \textsf{childdoc} directives,
the argument of \textit{main} of |\childdocmain| can be empty.

An alternative version of the command line processing described
in \secref{sec:commandline} using the detection mechanism reads:
%
\begin{center}
|... -jobname "|\textit{target}|" "|[\textit{flags}]%
[|\def\jobname{|\textit{dest}|}|]|\input{|\textit{main}|}"|
\end{center}

%%%%%%%%%%%%%%%%%%%%%%%%%%%%%%%%%%%%%%%%%%%%%%%%%%%%%%%%%%%%%%%%%%%%%%%%%%%%%%%%
\subsection{Manual Code}
\label{sec:manual}

In case one cannot be certain whether the definitions file |childdoc.def|
is installed on the target \TeX{} distribution
and one prefers not to ship it,
it is conceivable to paste a few relevant commands into the sources.

To that end, drop all statements |\input{childdoc.def}|
and perform the replacements as outlined below.
Instead of |\childdocmain{|\textit{main}|}| add the following code
to the top of the main file:
%
\begin{center}
\begin{tabular}{l}
|\||ifdefined\childdocname\endinput\||fi\newif\ifchilddoc|\\
|\edef\childdocname{\scantokens\expandafter{\jobname\noexpand}}|\\
|\def\childdocmain{|\textit{main}|}\||ifx\childdocmain\childdocname\||else|\\
|\childdoctrue\includeonly{\childdocname}\let\jobname\childdocmain\||fi|\\
\end{tabular}
\end{center}
%
Instead of |\childdocof{|\textit{main}|}| just include the main file
at the top of each child file:
%
\begin{center}
|\input{|\textit{main}|}|
\end{center}
%
A simple redirection |\childdocforward{|\textit{dest}|}| is achieved by:
%
\begin{center}
|\def\jobname{|\textit{dest}|}\input{\jobname}|
\end{center}
%
The redirection with prefix
|\childdocforwardprefix[|\textit{prefix}|]{|\textit{dest}|}|
is accomplished by:
%
\begin{center}
\begin{tabular}{l}
|{\edef\jobname{\scantokens\expandafter{\jobname\noexpand}}|\\
|\def\redirectjob |\textit{prefix}|#1~~~{\gdef\jobname{|\textit{dest}|#1}}|\\
|\expandafter\redirectjob\jobname~~~}\input{\jobname}|
\end{tabular}
\end{center}

In an alternative approach,
child documents can be compiled by a specific command line
without additional code or specific definitions:
%
\begin{center}
|... -jobname "|\textit{target}|" "|[\textit{flags}]%
|\includeonly{|\textit{dest}|}\input{|\textit{main}|}"|
\end{center}
%

%%%%%%%%%%%%%%%%%%%%%%%%%%%%%%%%%%%%%%%%%%%%%%%%%%%%%%%%%%%%%%%%%%%%%%%%%%%%%%%%
%%%%%%%%%%%%%%%%%%%%%%%%%%%%%%%%%%%%%%%%%%%%%%%%%%%%%%%%%%%%%%%%%%%%%%%%%%%%%%%%
\section{Information}

%%%%%%%%%%%%%%%%%%%%%%%%%%%%%%%%%%%%%%%%%%%%%%%%%%%%%%%%%%%%%%%%%%%%%%%%%%%%%%%%
\subsection{Copyright}

Copyright \copyright{} 2017--2018 Niklas Beisert

This work may be distributed and/or modified under the
conditions of the \LaTeX{} Project Public License, either version 1.3
of this license or (at your option) any later version.
The latest version of this license is in
  \url{http://www.latex-project.org/lppl.txt}
and version 1.3 or later is part of all distributions of \LaTeX{}
version 2005/12/01 or later.

This work has the LPPL maintenance status `maintained'.

The Current Maintainer of this work is Niklas Beisert.

This work consists of the files |README.txt|, |childdoc.ins| and |childdoc.dtx|
as well as the derived files |childdoc.def|, |cdocsamp.tex|
with |cdocsch1.tex|, |cdocsch2.tex|, |cdocspt3.tex|, |cdocspt4.tex|,
|cdocsdrf.tex|, |cdocsfn1.tex|, |cdocsfn2.tex|
as well as |childdoc.pdf|.

%%%%%%%%%%%%%%%%%%%%%%%%%%%%%%%%%%%%%%%%%%%%%%%%%%%%%%%%%%%%%%%%%%%%%%%%%%%%%%%%
\subsection{Files and Installation}

The package consists of the files:
%
\begin{center}
\begin{tabular}{ll}
    |README.txt|   & readme file \\
    |childdoc.ins| & installation file \\
    |childdoc.dtx| & source file \\
    |childdoc.def| & definition file \\
    |cdocsamp.tex| & sample main file \\
    |cdocsch1.tex| & sample include file \\
    |cdocsch2.tex| & sample include file \\
    |cdocspt3.tex| & sample part file \\
    |cdocspt4.tex| & sample part file \\
    |cdocsdrf.tex| & sample redirection file \\
    |cdocsfn1.tex| & sample redirection file \\
    |cdocsfn2.tex| & sample redirection file \\
    |childdoc.pdf| & manual
\end{tabular}
\end{center}
%
The distribution consists of the files
|README.txt|, |childdoc.ins| and |childdoc.dtx|.
%
\begin{itemize}
\item
Run (pdf)\LaTeX{} on |childdoc.dtx|
to compile the manual |childdoc.pdf| (this file).
\item
Run \LaTeX{} on |childdoc.ins| to create the definitions file |childdoc.def|
and the sample |cdocsamp.tex| with include files
|cdocsch1.tex|, |cdocsch2.tex|, |cdocspt3.tex|, |cdocspt4.tex|,
|cdocsdrf.tex|, |cdocsfn1.tex|, |cdocsfn2.tex|.
Then copy the file |childdoc.def| to an appropriate directory of your \LaTeX{}
distribution, e.g.\ \textit{texmf-root}|/tex/latex/childdoc|.
\end{itemize}

%%%%%%%%%%%%%%%%%%%%%%%%%%%%%%%%%%%%%%%%%%%%%%%%%%%%%%%%%%%%%%%%%%%%%%%%%%%%%%%%
\subsection{Related CTAN Packages}

There are several other packages which offer a similar functionality:
%
\begin{itemize}
\item
The packages
\href{http://ctan.org/pkg/docmute}{\textsf{docmute}},
\href{http://ctan.org/pkg/includex}{\textsf{includex}} and
\href{http://ctan.org/pkg/standalone}{\textsf{standalone}}
provide commands to include only the document body of
a child file thus allowing both files to be compiled individually.
\item
The packages \href{http://ctan.org/pkg/subdocs}{\textsf{subdocs}}
and \href{http://ctan.org/pkg/subfiles}{\textsf{subfiles}}
provide structures in which the main and child documents can be
encapsulated and allowing them to be compiled individually.
The inclusion mechanism is different from the conventional |\include|.
\item
The package \href{http://ctan.org/pkg/combine}{\textsf{combine}}
is an elaborate solution to combine several documents into one.
\end{itemize}
%
See also the CTAN topic \href{http://ctan.org/topic/subdocs}{\textsf{subdocs}}
for further related packages.
The present package differs from the above solutions in that
a document structure constructed with the conventional |\include| mechanism
just needs two extra commands at the top of every file
such that all constituent files can be compiled individually.

%%%%%%%%%%%%%%%%%%%%%%%%%%%%%%%%%%%%%%%%%%%%%%%%%%%%%%%%%%%%%%%%%%%%%%%%%%%%%%%%
%\subsection{Feature Suggestions}
%
%The following is a list of features which may be useful for future
%versions of this package:
%%
%\begin{itemize}
%\item
%\ldots
%\end{itemize}

%%%%%%%%%%%%%%%%%%%%%%%%%%%%%%%%%%%%%%%%%%%%%%%%%%%%%%%%%%%%%%%%%%%%%%%%%%%%%%%%
\subsection{Revision History}

%%%%%%%%%%%%%%%%%%%%%%%%%%%%%%%%%%%%%%%%
\paragraph{v2.0:} 2018/12/30

\begin{itemize}
\item
immediate forward processing
\item
added |\childdocby| mechanism
\item
manual restructured
\end{itemize}

%%%%%%%%%%%%%%%%%%%%%%%%%%%%%%%%%%%%%%%%
\paragraph{v1.6:} 2018/01/17

\begin{itemize}
\item
application for development of include files
\item
corrections to manual
\end{itemize}

%%%%%%%%%%%%%%%%%%%%%%%%%%%%%%%%%%%%%%%%
\paragraph{v1.5:} 2017/05/21

\begin{itemize}
\item
more complete structuring introduced
\item
|\childdocof| introduced
\item
|\childdoc| renamed to |\childdocmain|
\item
|\childredirect| renamed to |\childdocforward| and |\childdocforwardprefix|
and functionality expanded
\end{itemize}

%%%%%%%%%%%%%%%%%%%%%%%%%%%%%%%%%%%%%%%%
\paragraph{v1.0:} 2017/04/27

\begin{itemize}
\item
manual and install package
\item
first version published on CTAN
\end{itemize}

%%%%%%%%%%%%%%%%%%%%%%%%%%%%%%%%%%%%%%%%
\paragraph{v0.6:} 2017/04/26

\begin{itemize}
\item
redirection mechanism added
\end{itemize}

%%%%%%%%%%%%%%%%%%%%%%%%%%%%%%%%%%%%%%%%
\paragraph{v0.5:} 2017/04/26

\begin{itemize}
\item
functionality in definition file
\end{itemize}


%%%%%%%%%%%%%%%%%%%%%%%%%%%%%%%%%%%%%%%%%%%%%%%%%%%%%%%%%%%%%%%%%%%%%%%%%%%%%%%%
%%%%%%%%%%%%%%%%%%%%%%%%%%%%%%%%%%%%%%%%%%%%%%%%%%%%%%%%%%%%%%%%%%%%%%%%%%%%%%%%
%%%%%%%%%%%%%%%%%%%%%%%%%%%%%%%%%%%%%%%%%%%%%%%%%%%%%%%%%%%%%%%%%%%%%%%%%%%%%%%%
\appendix

\settowidth\MacroIndent{\rmfamily\scriptsize 000\ }

 \DocInput{childdoc.dtx}

\end{document}
%</driver>
% \fi
%
% %%%%%%%%%%%%%%%%%%%%%%%%%%%%%%%%%%%%%%%%%%%%%%%%%%%%%%%%%%%%%%%%%%%%%%%%%%%%%%
% %%%%%%%%%%%%%%%%%%%%%%%%%%%%%%%%%%%%%%%%%%%%%%%%%%%%%%%%%%%%%%%%%%%%%%%%%%%%%%
% \section{Sample}
%\iffalse
%<*samplemain>
%\fi
%
% The following presents a sample document
% with two chapters, two parts, a title page,
% a compile flag as well as three forwarding files to set the flag.
% It consists of eight |.tex| files:
% \begin{center}
% \begin{tabular}{ll}
% |cdocsamp.tex|&main file\\
% |cdocsch1.tex|&include file for chapter 1\\
% |cdocsch2.tex|&include file for chapter 2\\
% |cdocspt3.tex|&include file for part 3\\
% |cdocspt4.tex|&include file for part 4\\
% |cdocsdrf.tex|&forwarding file for main file in draft mode\\
% |cdocsfi1.tex|&forwarding file for final version of chapter 1\\
% |cdocsfi2.tex|&forwarding file for final version of chapter 2\\
% \end{tabular}
% \end{center}
% Each of the eight files can be compiled directly by the \LaTeX{} compiler.
%
% %%%%%%%%%%%%%%%%%%%%%%%%%%%%%%%%%%%%%%
% \paragraph{Main File.}
%
% The main file is called |cdocsamp.tex|.
%
% Load the \textsf{childdoc} definitions and
% declare the filename for the main document:
%    \begin{macrocode}
\input{childdoc.def}
\childdocmain{}
%    \end{macrocode}

% Optional override for |\version| flag:
%    \begin{macrocode}
%%\ifchilddoc\else\providecommand{\version}{draft}\fi
%    \end{macrocode}

% Define the default values for the |\version| flag
% (|final| for the main file and |draft| for childs):
%    \begin{macrocode}
\ifchilddoc
\providecommand{\version}{draft}
\else
\providecommand{\version}{final}
\fi
%    \end{macrocode}

% Load the standard document class:
%    \begin{macrocode}
\documentclass[12pt]{article}
%    \end{macrocode}

% Start the document body:
%    \begin{macrocode}
\begin{document}
%    \end{macrocode}

% Declare a title page.
% Print title, part of document being processed and version flag:
%    \begin{macrocode}
\addtocounter{page}{-1}
\begin{center}
{\LARGE\bfseries{}childdoc example\par}
\vspace{1cm}
\ifchilddoc
\ifchilddocmanual part\else chapter\fi:
`\childdocname' of `\childdocjob'\par
\else
main document: `\childdocjob'\par
\fi
version: \version\par
\end{center}
\newpage
%    \end{macrocode}

% Manually include selected file,
% otherwise process as usual:
%    \begin{macrocode}
\ifchilddocmanual
\section*{part `\childdocname'}
\input{\childdocname}
\else
%    \end{macrocode}

% Include the two chapters:
%    \begin{macrocode}
\include{cdocsch1}
\include{cdocsch2}
%    \end{macrocode}

% Include the two parts unless only chapters should be displayed:
%    \begin{macrocode}
\ifchilddoc\else
\section{part three}
\input{cdocspt3}
\section{part four}
\input{cdocspt4}
\fi
%    \end{macrocode}

% Process as usual until here:
%    \begin{macrocode}
\fi
%    \end{macrocode}

% End of document body:
%    \begin{macrocode}
\end{document}
%    \end{macrocode}
%\iffalse
%</samplemain>
%\fi
%
% %%%%%%%%%%%%%%%%%%%%%%%%%%%%%%%%%%%%%%
% \paragraph{Chapter Include Files.}
%
% The include files are called |cdocsch1.tex| and |cdocsch2.tex|.
%
%\iffalse
%<*samplechap1|samplechap2>
%\fi

% Optional override for |\version| flag:
%    \begin{macrocode}
%%\providecommand{\version}{final}
%    \end{macrocode}

% Include the main document:
%    \begin{macrocode}
\input{childdoc.def}
\childdocof{cdocsamp}
%    \end{macrocode}

%\iffalse
%</samplechap1|samplechap2>
%\fi
%
%\iffalse
%<*samplechap1>
%\fi
% Some text for chapter 1:
%    \begin{macrocode}
\section{one}
some text in chapter one
%    \end{macrocode}

%\iffalse
%</samplechap1>
%\fi
% Some text for chapter 2:
%\iffalse
%<*samplechap2>
%\fi
%    \begin{macrocode}
\section{two}
more text in chapter two
%    \end{macrocode}

%\iffalse
%</samplechap2>
%\fi
%
% %%%%%%%%%%%%%%%%%%%%%%%%%%%%%%%%%%%%%%
% \paragraph{Part Include Files.}
%
% The include files are called |cdocspt3.tex| and |cdocspt4.tex|.
%
%\iffalse
%<*samplepart3|samplepart4>
%\fi

% Optional override for |\version| flag:
%    \begin{macrocode}
%%\providecommand{\version}{final}
%    \end{macrocode}

% Include the main document:
%    \begin{macrocode}
\input{childdoc.def}
\childdocby{cdocsamp}
%    \end{macrocode}

%\iffalse
%</samplepart3|samplepart4>
%\fi
%
%\iffalse
%<*samplepart3>
%\fi
% Some text for part 3:
%    \begin{macrocode}
some text in part three
%    \end{macrocode}

%\iffalse
%</samplepart3>
%\fi
% Some text for part 4:
%\iffalse
%<*samplepart4>
%\fi
%    \begin{macrocode}
more text in part four
%    \end{macrocode}

%\iffalse
%</samplepart4>
%\fi
%
% %%%%%%%%%%%%%%%%%%%%%%%%%%%%%%%%%%%%%%
% \paragraph{Forwarding for a Complete Draft.}
%
% The following forwarding file |cdocsdrf.tex|
% compiles the main document in draft mode:
%\iffalse
%<*sampledraft>
%\fi
%    \begin{macrocode}
\def\version{draft}
\input{childdoc.def}
\childdocforward{cdocsamp}
%    \end{macrocode}

%\iffalse
%</sampledraft>
%\fi
%
% %%%%%%%%%%%%%%%%%%%%%%%%%%%%%%%%%%%%%%
% \paragraph{Forwarding for Final Version of the Chapters.}
%
% The following forwarding files |cdocsfn1.tex| and |cdocsfn2.tex|
% (with identical content)
% compile the final versions of the child documents
% |cdocsch1.tex| and |cdocsch2.tex|, respectively:
%\iffalse
%<*samplefinal>
%\fi
%    \begin{macrocode}
\def\version{final}
\input{childdoc.def}
\childdocforwardprefix[cdocsamp]{cdocsfn}{cdocsch}
%    \end{macrocode}

%\iffalse
%</samplefinal>
%\fi
%
% %%%%%%%%%%%%%%%%%%%%%%%%%%%%%%%%%%%%%%
% \paragraph{Command Line Processing.}
%
% The following three command lines generate the output files
% |cdocscld|, |cdocscl1| and |cdocscl2|
% which should be identical to
% |cdocsdrf|, |cdocsch1| and |cdocsfn2|, respectively:
% \begin{center}
% \begin{tabular}{l}
% |latex -jobname cdocscld \|\\
% |  "\def\version{draft}\input{childdoc.def}\childdocforward{cdocsamp}"|\\
% |latex -jobname cdocscl1 \|\\
% |  "\input{childdoc.def}\childdocforward[cdocsamp]{cdocsch1}"|\\
% |latex -jobname cdocscl2 \|\\
% |  "\def\version{final}\input{childdoc.def}\childdocforward{cdocsch2}"|
% \end{tabular}
% \end{center}
% Note that the trailing backslash on each first line
% merely continues the input to the second line
% (for convenient cut ant paste).
% Furthermore, the command |latex| can be replaced by any
% of its alternative versions such as |pdflatex|.
%
% %%%%%%%%%%%%%%%%%%%%%%%%%%%%%%%%%%%%%%%%%%%%%%%%%%%%%%%%%%%%%%%%%%%%%%%%%%%%%%
% %%%%%%%%%%%%%%%%%%%%%%%%%%%%%%%%%%%%%%%%%%%%%%%%%%%%%%%%%%%%%%%%%%%%%%%%%%%%%%
% \section{Implementation}
%\iffalse
%<*package>
%\fi
%
% This section describes the definitions file |childdoc.def|.

% The definitions cannot be loaded using |\usepackage| or |\RequirePackage|
% which has a mechanism to prevent loading a style file more than once.
% When loading the definitions by means of |\input|
% multiple instances have to be prevented manually:
%\iffalse
%This code needs to be before the `\ProvidesFile' directive
%which is defined at the beginning of this file.
%Therefore it is also placed there and commented out here.
%</package>
%<*discard>
%\fi
%    \begin{macrocode}
\ifdefined\childdocmain\endinput\fi
%    \end{macrocode}
%\iffalse
%</discard>
%<*package>
%\fi
%
% \macro{\ifchilddoc}
% \macro{\ifchilddocmanual}
% The conditional |\ifchilddoc| tells whether a
% child (true) or main (false) document is being compiled.
% The conditional |\ifchilddocmanual| tells whether
% the |\includeonly| mechanism is used (false) or
% the selection of child files must be performed manually (true).
% The definitions initialise to false:
%    \begin{macrocode}
\newif\ifchilddoc
\newif\ifchilddocmanual
%    \end{macrocode}

% \macro{\childdocname}
% \macro{\childdocjob}
% The macro |\childdocname| stores the name of the main document
% to be compiled. The macro |\childdocjob| stores the name of
% the document on which the \LaTeX{} compiler was originally invoked.
% The content of |\jobname| cannot be compared
% to filenames specified in the source due to different catcodes.
% The following code rescans |\jobname|, stores the result
% in |\childdocname| and saves a copy in |\childdocjob|:
%    \begin{macrocode}
\edef\childdocname{\scantokens\expandafter{\jobname\noexpand}}
\let\childdocjob\childdocname
%    \end{macrocode}

% \macro{\childdocdisable}
% The macro |\childdocdisable| prevents the main file
% from being processed more than once.
% At this stage, the main document command |\childdocmain|
% is assumed to be called once again where it should do nothing.
% Any subsequent call to it should prevent
% a secondary processing of the main document
% It overwrites the forwarding commands
% |\childdocof| and |\childdocforward|
% with empty macros to prevent further inclusions of the main document:
%    \begin{macrocode}
\newcommand{\childdocdisable}
{
  \renewcommand{\childdocmain}[1]{\renewcommand{\childdocmain}[1]{\endinput}}
  \renewcommand{\childdocof}[1]{}
  \renewcommand{\childdocby}[2][]{}
  \renewcommand{\childdocforward}[2][]{}
  \renewcommand{\childdocdisable}{}
}
%    \end{macrocode}

% \macro{\childdocmain}
% The macro |\childdocmain| is to be called at the top of the main file
% with nothing or the main filename (without extension) as argument.
% First, it breaks loops.
% If the argument is not empty and does not match |\childdocname|
% (which is set by the first inclusion of |childdoc.def|),
% |\ifchilddoc| is set to true, |\includeonly| is applied to the child file
% and |\jobname| is set to the main file
% (for proper handling of |.aux| files):
%    \begin{macrocode}
\newcommand{\childdocmain}[1]
{
  \childdocdisable\childdocmain{}
  \if?#1?\else
    \begingroup
      \def\childdoctmp{#1}
      \ifx\childdoctmp\childdocname
        \def\childdoctmp{}
      \else
        \def\childdoctmp
        {
          \childdoctrue
          \includeonly{\childdocname}
          \def\childdocjob{#1}
          \def\jobname{#1}
        }
      \fi
      \expandafter
    \endgroup
    \childdoctmp
  \fi
}
%    \end{macrocode}

% \macro{\childdocof}
% The command |\childdocof| redirects
% compilation to the main file |#1|.
%    \begin{macrocode}
\newcommand{\childdocof}[1]
{
  \childdocdisable
  \childdoctrue
  \includeonly{\childdocname}
  \def\jobname{#1}
  \def\childdocjob{#1}
  \input{#1}
}
%    \end{macrocode}

% \macro{\childdocby}
% The command |\childdocby| ....
%    \begin{macrocode}
\newcommand{\childdocby}[2][]
{
  \childdocdisable
  \childdoctrue
  \childdocmanualtrue
  \if?#1?\else
    \def\jobname{#2}
  \fi
  \def\childdocjob{#2}
  \input{#2}
  \endinput
}
%    \end{macrocode}

% \macro{\childdocforward}
% The command |\childdocforward| redirects
% compilation to the main file or
% (if the optional argument is given) a child file.
% Parameters are set as if the main file
% or a child file starting with |\childdocof| was compiled.
% Then compilation is handed over to the main file:
%    \begin{macrocode}
\newcommand{\childdocforward}[2][]
{
  \begingroup
    \if?#1?
      \def\childdoctmp
      {
        \def\childdocname{#2}
        \def\childdocjob{#2}
        \def\jobname{#2}
        \input{#2}
        \endinput
      }
    \else
      \def\childdoctmp
      {
        \childdocdisable
        \def\childdocname{#2}
        \childdoctrue
        \includeonly{#2}
        \def\childdocjob{#1}
        \def\jobname{#1}
        \input{#1}
        \endinput
      }
    \fi
    \expandafter
  \endgroup
  \childdoctmp
}
%    \end{macrocode}

% \macro{\childdocforwardprefix}
% The command |\childdocforwardprefix| redirects
% compilation to the main or a child file by means of a pattern.
% The prefix |#1| in the current filename is replaced by |#2|
% and the suffix of the current filename is kept
% (it is assumed that the filename does not contain the substring `|~~~|'
% which is used as a delimiter).
% Compilation is handed over to the new file by |\childdocforward|:
%    \begin{macrocode}
\newcommand{\childdocforwardprefix}[3][]
{
  \begingroup
    \def\childdocextract #2##1~~~{\def\childdoctmp{\childdocforward[#1]{#3##1}}}
    \expandafter\childdocextract\childdocname~~~
    \expandafter
  \endgroup
  \childdoctmp
}
%    \end{macrocode}

% \macro{\childdoc}
% The deprecated macro |\childdoc| is a legacy version of |\childdocmain|:
%    \begin{macrocode}
\newcommand{\childdoc}{\childdocmain}
%    \end{macrocode}

% \macro{\childdocredirect}
% The deprecated macro |\childdocredirect| is a legacy version
% of |\childdocforward| and |\childdocforwardprefix|:
%    \begin{macrocode}
\newcommand{\childdocredirect}[2][]
{
  \begingroup
    \if?#1?
      \def\childdoctmp{\childdocforward{#2}}
    \else
      \def\childdoctmp{\childdocforwardprefix{#1}{#2}}
    \fi
    \expandafter
  \endgroup
  \childdoctmp
}
%    \end{macrocode}

%\iffalse
%</package>
%\fi
%
\endinput
\childdocforward{cdocsamp}"|\\
% |latex -jobname cdocscl1 \|\\
% |  "% \iffalse
%
% childdoc.dtx Copyright (C) 2017-2018 Niklas Beisert
%
% This work may be distributed and/or modified under the
% conditions of the LaTeX Project Public License, either version 1.3
% of this license or (at your option) any later version.
% The latest version of this license is in
%   http://www.latex-project.org/lppl.txt
% and version 1.3 or later is part of all distributions of LaTeX
% version 2005/12/01 or later.
%
% This work has the LPPL maintenance status `maintained'.
%
% The Current Maintainer of this work is Niklas Beisert.
%
% This work consists of the files childdoc.dtx and childdoc.ins
% and the derived files childdoc.def and cdocsamp.tex with
% cdocsch1.tex, cdocsch2.tex, cdocsdrf.tex, cdocsfn1.tex, cdocsfn2.tex.
%
%<package>\ifdefined\childdocmain\endinput\fi
%<package>\ProvidesFile{childdoc.def}[2018/12/30 v2.0 child document driver]
%<samplemain>\ProvidesFile{cdocsamp.tex}[2018/12/30 v2.0 sample for childdoc]
%<*driver>
%\ProvidesFile{childdoc.drv}[2018/12/30 v2.0 childdoc reference manual file]
\PassOptionsToClass{10pt,a4paper}{article}
\documentclass{ltxdoc}

\usepackage[margin=35mm]{geometry}
\usepackage{hyperref}
\usepackage{hyperxmp}
\usepackage[usenames]{color}

\hypersetup{colorlinks=true}
\hypersetup{pdfstartview=FitH}
\hypersetup{pdfpagemode=UseNone}
\hypersetup{pdfsource={}}
\hypersetup{pdflang={en-UK}}
\hypersetup{pdfcopyright={Copyright 2017-2018 Niklas Beisert.
  This work may be distributed and/or modified under the
  conditions of the LaTeX Project Public License, either version 1.3
  of this license or (at your option) any later version.}}
\hypersetup{pdflicenseurl={http://www.latex-project.org/lppl.txt}}
\hypersetup{pdfcontactaddress={ETH Zurich, ITP, HIT K,
  Wolfgang-Pauli-Strasse 27}}
\hypersetup{pdfcontactpostcode={8093}}
\hypersetup{pdfcontactcity={Zurich}}
\hypersetup{pdfcontactcountry={Switzerland}}
\hypersetup{pdfcontactemail={nbeisert@itp.phys.ethz.ch}}
\hypersetup{pdfcontacturl={http://people.phys.ethz.ch/\xmptilde nbeisert/}}

\newcommand{\secref}[1]{\hyperref[#1]{section \ref*{#1}}}

\parskip1ex
\parindent0pt
\let\olditemize\itemize
\def\itemize{\olditemize\parskip0pt}

\begin{document}

\title{The \textsf{childdoc} Package}
\hypersetup{pdftitle={The childdoc Package}}
\author{Niklas Beisert\\[2ex]
  Institut f\"ur Theoretische Physik\\
  Eidgen\"ossische Technische Hochschule Z\"urich\\
  Wolfgang-Pauli-Strasse 27, 8093 Z\"urich, Switzerland\\[1ex]
  \href{mailto:nbeisert@itp.phys.ethz.ch}
  {\texttt{nbeisert@itp.phys.ethz.ch}}}
\hypersetup{pdfauthor={Niklas Beisert}}
\hypersetup{pdfsubject={Manual for the LaTeX2e Package childdoc}}
\date{30 December 2018, \textsf{v2.0}}
\maketitle

\begin{abstract}\noindent
\textsf{childdoc} is a \LaTeXe{} package
that enables the direct compilation
of document sections included by |\include|
to individual files.
\end{abstract}

\begingroup
\parskip0ex
\tableofcontents
\endgroup

%%%%%%%%%%%%%%%%%%%%%%%%%%%%%%%%%%%%%%%%%%%%%%%%%%%%%%%%%%%%%%%%%%%%%%%%%%%%%%%%
%%%%%%%%%%%%%%%%%%%%%%%%%%%%%%%%%%%%%%%%%%%%%%%%%%%%%%%%%%%%%%%%%%%%%%%%%%%%%%%%
\section{Introduction}

\LaTeX{} provides a mechanism to structure a large document (such as a book)
into a main file and several child files (containing the chapters)
using the |\include| command.
This mechanism is beneficial for documents
which span hundreds of pages in order to
make the source file(s) more manageable.
Moreover, compilation can be restricted to
selected child files by means of the |\includeonly| command.
The latter feature can be used to reduce the compilation time while editing
(this was significantly more useful in the earlier days of \LaTeX{})
or to generate a smaller document which is easier to navigate.
Another application of |\includeonly| is to generate
documents consisting of selected parts of the complete document.

However, there are a few drawbacks of the plain |\include| mechanism:
\begin{itemize}
\item
The child files cannot be compiled on their own,
they can only be compiled via the main file.
A naive editing environment
(such as a text editor with an option
to have the current file processed by \LaTeX)
may require one to switch to the main file before compiling;
attempting to compile the child file produces errors.
\item
The main file must be modified (each time)
to adjust the |\includeonly| command
to the present needs. This easily leaves the main file in a messy state.
\item
The generated document will always carry the filename
of the main document. This is inconvenient if
several child files are to be compiled and
to be kept for distribution.
\end{itemize}

The present package provides a simple interface
to make child files individually compilable by \LaTeX{}.
Compiling a child file then has the same effect as compiling
the main file with an |\includeonly| command
to select the appropriate child.
Moreover the generated document will carry the name of the child
rather than the main file.
This resolves all three above issues.

This feature is meant to make the editing of books,
thesis documents and lecture notes somewhat more convenient.
However, the package can also be used efficiently for
composing a series of documents (such as exercise sheets)
which are typically distributed individually.
It then assists the author in generating the individual documents
(potentially in different versions)
as well as a document containing the collected series.
Another application is in developing style files
or other kinds of included material
where compilation of the style file could redirect
to a sample or test file.

%%%%%%%%%%%%%%%%%%%%%%%%%%%%%%%%%%%%%%%%%%%%%%%%%%%%%%%%%%%%%%%%%%%%%%%%%%%%%%%%
%%%%%%%%%%%%%%%%%%%%%%%%%%%%%%%%%%%%%%%%%%%%%%%%%%%%%%%%%%%%%%%%%%%%%%%%%%%%%%%%
\section{Usage}

First of all, the package \textsf{childdoc} is \emph{not} a standard
\LaTeXe{} |.sty| style file! Therefore it needs to be invoked in
a non-standard way.

%%%%%%%%%%%%%%%%%%%%%%%%%%%%%%%%%%%%%%%%%%%%%%%%%%%%%%%%%%%%%%%%%%%%%%%%%%%%%%%%
\subsection{Included Files}
\label{sec:include}

%%%%%%%%%%%%%%%%%%%%%%%%%%%%%%%%%%%%%%%%
\DescribeMacro{\childdocmain}
To use the package, add the commands
\begin{center}
\begin{tabular}{l}
|\input{childdoc.def}|\\
|\childdocmain{}|\\
\end{tabular}
\end{center}
at the very top of the main \LaTeX{} file,
in particular \emph{before} the |\documentclass| statement!
The argument of |\childdocmain| should be left empty
(but it must be present).

%%%%%%%%%%%%%%%%%%%%%%%%%%%%%%%%%%%%%%%%
\DescribeMacro{\childdocof}
Furthermore, add the commands
\begin{center}
\begin{tabular}{l}
|\input{childdoc.def}|\\
|\childdocof{|\textit{main}|}|\\
\end{tabular}
\end{center}
at the top of every child file \textit{child}
which is included by |\include{|\textit{child}|}|
from within the main file
(or at least for those files to be compiled individually).
The argument \textit{main} must be the filename of the main file.

There are a couple of
considerations in setting up the main and child documents:

%%%%%%%%%%%%%%%%%%%%%%%%%%%%%%%%%%%%%%%%
\paragraph{Restrictions.}

Please note the following restrictions:
\begin{itemize}
\item
|\childdocmain| must be called with one argument \textit{main}
to ensure compatibility with earlier version of the package.
It must either be empty (|\childdocmain{}|)
or precisely match the filename of the main file in which it is specified.
See \secref{sec:detection} for further information.
\item
The filename \textit{main} must be specified without the |.tex| extension.
\item
The filename \textit{main} is case sensitive
(even in case-insensitive file systems)
due to internal string comparison.
\item
The argument \textit{main} should be fully expanded, it cannot be a macro.
\item
Subdirectories and special characters should be avoided in filenames.
\item
The command |\childdocmain{|\textit{main}|}| must be followed by a whitespace.
It should not be followed immediately by another command
or by a comment mark `|%|'.
This is because the \TeX{} parser reads the token immediately following
the argument of |\childdocmain| and puts it
at the beginning of every child section;
however, a white\-space is ignored.
\end{itemize}

%%%%%%%%%%%%%%%%%%%%%%%%%%%%%%%%%%%%%%%%
\paragraph{Content of Main File.}

It is advisable to place all content in the child files included by |\include|.
Any output contained in the main file will appear in all child documents
unless suppressed manually;
it cannot be suppressed automatically by the |\includeonly| directive
and thus should normally be avoided.
A method to include some content in the main file
by means of conditional processing is described in \secref{sec:conditional}.

%%%%%%%%%%%%%%%%%%%%%%%%%%%%%%%%%%%%%%%%
\paragraph{Page Numbering.}

When only a part of the document is compiled,
the appropriate numbering of pages
(as well as other status parameters)
is determined from the |.aux| files.
The latter contain information from previous passes.
However this information needs to propagate through
all intermediate child documents.
Therefore the page numbering in child documents may well
be inconsistent until the complete document is compiled at least once.

A useful (if unconventional) way to always ensure a consistent
page numbering is to restart the numbering in each child document
and denote the pages by `\textit{child}|.|\textit{page}'
where \textit{child} represents the chapter/section number of the child file.
This can be achieved by the command
|\numberwithin{page}{|\textit{child}|}|
of the \textsf{amsmath} package
where \textit{child} can be |chapter| or |section|
depending on the chosen structuring.
Alternatively, one can modify the macro |\thepage| appropriately
and reset the counter |page| at the start of each child file.

%%%%%%%%%%%%%%%%%%%%%%%%%%%%%%%%%%%%%%%%%%%%%%%%%%%%%%%%%%%%%%%%%%%%%%%%%%%%%%%%
\subsection{Conditional Processing}
\label{sec:conditional}

The package provides a mechanism to compile different versions
of a document. To customise the versions further some conditional processing
can come in handy to distinguish which version is being compiled.
The package provides two macros to describe the compilation context:

%%%%%%%%%%%%%%%%%%%%%%%%%%%%%%%%%%%%%%%%
\DescribeMacro{\ifchilddoc}
The conditional |\ifchilddoc| distinguishes between the compilation of
child documents and the main document:
%
\begin{center}
|\ifchilddoc |\textit{child-code}| |[|\||else |\textit{main-code}]| \||fi|
\end{center}

%%%%%%%%%%%%%%%%%%%%%%%%%%%%%%%%%%%%%%%%
\DescribeMacro{\childdocname}
\DescribeMacro{\childdocjob}
The macro |\childdocname| contains the filename (without extension)
of the main or child file being processed.
Note that |\childdocjob| will always contain the name of the main file.

%%%%%%%%%%%%%%%%%%%%%%%%%%%%%%%%%%%%%%%%
\paragraph{Title Page.}

Conditional processing can be used to include a title or banner page
in the main document when proper precautions are taken.
Importantly, the code in the main file should ensure that the page counter
(as well as other status parameters which are stored in the |.aux| files)
takes the same value after the conditional processing.
Otherwise the page numbers may take divergent values
depending on which part is compiled.

For example, a title page could be declared by:
%
\begin{center}
\begin{tabular}{l}
|\ifchilddoc\||else|\\
|\addtocounter{page}{-1}|\\
\textit{code for title page}\\
|\newpage|\\
|\||fi|
\end{tabular}
\end{center}
%
A banner page for the child documents can be generated by:
%
\begin{center}
\begin{tabular}{l}
|\ifchilddoc|\\
|\addtocounter{page}{-1}|\\
\textit{code for banner page}\\
|\newpage|\\
|\||fi|
\end{tabular}
\end{center}
%
Here one could write a message such as:
\begin{center}
|This is the part \childdocname{} of \childdocjob{}.|
\end{center}

%%%%%%%%%%%%%%%%%%%%%%%%%%%%%%%%%%%%%%%%%%%%%%%%%%%%%%%%%%%%%%%%%%%%%%%%%%%%%%%%
\subsection{Flags}
\label{sec:flags}

The package makes it easy to generate different versions
of the main or child documents.
To this end compilation flags can be defined
and assigned different default values.
They will be particularly useful in conjunction
with the forwarding mechanism described in \secref{sec:forward}.

For example, it may be useful to have a flag |\version|
which can be set to |draft| or |final|.
The document source will contain some conditional code
depending on the value of |\version|.
Suppose further, the flag should default to |final| for the main file
and to |draft| for child files
which is a natural assignment for editing the document.
This is achieved by placing the following code
in the preamble of the main document
(below the |\childdocmain| directive):
%
\begin{center}
\begin{tabular}{l}
|\ifchilddoc|\\
|\providecommand{\version}{draft}|\\
|\||else|\\
|\providecommand{\version}{final}|\\
|\||fi|
\end{tabular}
\end{center}
%
The definition by |\providecommand| makes sure
that previous definitions are not overwritten.
Further statements |\providecommand{\version}{...}|
can thus be added before the above code to override it.

For the main file, one might add a line
(between |\childdocmain| and the above block)
%
\begin{center}
|%\ifchilddoc\||else\providecommand{\version}{draft}\||fi|
\end{center}
%
which can be uncommented to produce a draft version.
Likewise one can add a line to the very top of a child file
(above the |\childdocof{|\textit{main}|}| directive)
%
\begin{center}
|%\providecommand{\version}{final}|
\end{center}
%
which can be uncommented to produce the final version of this child document.

%%%%%%%%%%%%%%%%%%%%%%%%%%%%%%%%%%%%%%%%%%%%%%%%%%%%%%%%%%%%%%%%%%%%%%%%%%%%%%%%
\subsection{Forwarding}
\label{sec:forward}

Different versions of the main or child documents
using compilation flags as described in \secref{sec:flags}
can be (permanently) stored in different files
for convenient compilation, viewing and distribution.
To this end, the package defines a command
to pass on compilation to a different file:

%%%%%%%%%%%%%%%%%%%%%%%%%%%%%%%%%%%%%%%%
\DescribeMacro{\childdocforward}
The command |\childdocforward| redirects processing to
another source file:
%
\begin{center}
\begin{tabular}{l}
|\input{childdoc.def}|\\
|\childdocforward[|\textit{main}|]{|\textit{dest}|}|\\
\end{tabular}
\end{center}
%
The argument \textit{dest} is the destination file
(without extension).
It should be the main file or one of the child files.
Note that further \textsf{childdoc} directives
such as |\childdocof| and |\childdocforward|
in the indicated file will be processed in this form.
The optional argument \textit{main}
passes on directly to the main file \textit{main}
while pretending to compile the child \textit{dest}.
This form behaves as if \textit{dest}
issues |\childdocof{|\textit{main}|}| right away,
and no further \textsf{childdoc} directives will be processed.

%%%%%%%%%%%%%%%%%%%%%%%%%%%%%%%%%%%%%%%%
\DescribeMacro{\...prefix}
In the alternative form |\childdocforwardprefix|,
%
\begin{center}
\begin{tabular}{l}
|\input{childdoc.def}|\\
|\childdocforwardprefix[|\textit{main}|]{|\textit{prefix}|}{|\textit{dest}|}|
\end{tabular}
\end{center}
%
the destination file is determined by a pattern
depending on the current file:
To make this work, the current file must be called
`{\textit{prefix}\hspace{0.2em}\textit{suffix}}'
with \textit{prefix} matching precisely the argument.
Processing is then passed on to the file
`{\textit{dest}\hspace{0.2em}\textit{suffix}}'.
Surely, the same effect is achieved by
directly specifying the
argument `{\textit{dest}\hspace{0.2em}\textit{suffix}}'
in the first form.
However, that requires to set up a different file
for each child. With the alternative form of the command
all these files can have exactly the same content
which simplifies setting them up and maintaining them.

For example, the following file |draft.tex|
with a compilation flag |\version| as described in \secref{sec:flags}
compiles the main document as a draft:
%
\begin{center}
\begin{tabular}{l}
|\def\version{draft}|\\
|\input{childdoc.def}|\\
|\childdocforward{|\textit{main}|}|
\end{tabular}
\end{center}
%
Likewise, the following files |final|\textit{nn}|.tex|
compile the final version of the child document
|child|\textit{nn}|.tex|:
%
\begin{center}
\begin{tabular}{l}
|\def\version{final}|\\
|\input{childdoc.def}|\\
|\childdocforwardprefix{final}{child}|
\end{tabular}
\end{center}
%

Note that when several versions of a main file and/or of each child file
are to be generated, it may be convenient to set up a |Makefile| or
shell script to automatise the process.

%%%%%%%%%%%%%%%%%%%%%%%%%%%%%%%%%%%%%%%%%%%%%%%%%%%%%%%%%%%%%%%%%%%%%%%%%%%%%%%%
\subsection{Command Line Processing}
\label{sec:commandline}

The effect of redirection files can also be achieved by invoking
the \LaTeX{} compiler with a more elaborate command line.
Most conveniently this should be done as part
of a shell script or a |Makefile|.

When using \textsf{childdoc} in the main file, the following
command lines effectively perform a redirection
(note that depending on the shell being used,
backslashes may have to be doubled: `|\|' $\to$ `|\\|'):
%
\begin{center}
|... -jobname "|\textit{target}|" |\\|"|[\textit{flags}]%
|\input{childdoc.def}\childdocforward[|\textit{main}|]{|\textit{dest}|}"|
\end{center}
%
Here \textit{target} is the name of the output file,
\textit{main} is the name of the main file
and \textit{dest} is the name of the main or child file to be processed
(all filenames without extensions).
The optional argument \textit{main} can be omitted
if \textit{main} matches \textit{dest}.
Optionally, compilation \textit{flags} can be defined via |\def| commands.
This command line makes the \TeX{} engine believe
it is compiling the file \textit{target}
whose content is specified as the latter parameter.
The provided code then forwards the processing to
\textit{main} or \textit{dest} as described in \secref{sec:forward}.

%%%%%%%%%%%%%%%%%%%%%%%%%%%%%%%%%%%%%%%%%%%%%%%%%%%%%%%%%%%%%%%%%%%%%%%%%%%%%%%%
\subsection{Include by Input}
\label{sec:input}

Including child documents by |\include| has some restrictions by design.
Most notably, the content of a child document always occupies
its own set of pages; pages cannot be shared between child documents.
Usually, this behaviour makes perfect sense
because each child document contain an essential part of the document.
However, in some situations it may be desirable to compose
a document from a collection of parts
without having mandatory page breaks between then.
For this case, the package
provides a mechanism to include parts
by |\input| which can also be processed individually.
However, by construction this mechanism
requires manual handling of the content to be output.

%%%%%%%%%%%%%%%%%%%%%%%%%%%%%%%%%%%%%%%%
\DescribeMacro{\ifchilddocmanual}
The main file should be prepared as usual, see \secref{sec:include}.
However, the document body must make a distinction
between processing of an individual part and of the main document, e.g.:
%
\begin{center}
\begin{tabular}{l}
|\ifchilddocmanual|\\
|\input{\childdocname}|\\
|\||else|\\
\textit{document body with }|\input{|\textit{part}|}|\\
|\||fi|
\end{tabular}
\end{center}
%
The conditional |\ifchilddocmanual| is true whenever
a part to be included by |\input| is being compiled,
and the name of the part is stored in |\childdocname|.

%%%%%%%%%%%%%%%%%%%%%%%%%%%%%%%%%%%%%%%%
\DescribeMacro{\childdocby}
Each part to be included by |\input| should start with:
%
\begin{center}
\begin{tabular}{l}
|\input{childdoc.def}|\\
|\childdocby{|\textit{main}|}|\\
\end{tabular}
\end{center}
%
The directive |\childdocby| is similar to |\childdocof|
described in \secref{sec:include},
but the subsequent selection of content must be done manually.
To that end, both |\ifchilddoc| and |\ifchilddocmanual|
will be true upon processing of a part,
and the name of the part is stored in |\childdocname|.
Note that |\jobname| will be set to the filename of the current part
so that each part receives an individual |.aux| file
that does not interfere with the |.aux| file(s) of the main document.
This behaviour can be altered by the alternative form
|\childdocby[*]{|\textit{main}|}| (with a non-empty optional argument)
which uses the |.aux| file of the main document
by setting |\jobname| to \textit{main}.

%%%%%%%%%%%%%%%%%%%%%%%%%%%%%%%%%%%%%%%%%%%%%%%%%%%%%%%%%%%%%%%%%%%%%%%%%%%%%%%%
\subsection{Driver Development}
\label{sec:driver}

The \textsf{childdoc} mechanism can also be use for the development
of definition files such as \LaTeX{} styles or classes.
This case differs from the above setup with multiple parts
included by |\include| in that no |\includeonly| should be invoked.
This can be achieved by starting the include file
(before |\ProvidesPackage|) with:
%
\begin{center}
\begin{tabular}{l}
|\input{childdoc.def}|\\
|\childdocforward{|\textit{main}|}|\\
\end{tabular}
\end{center}
%
or alternatively with:
%
\begin{center}
\begin{tabular}{l}
|\input{childdoc.def}|\\
|\childdocby{|\textit{main}|}|\\
\end{tabular}
\end{center}
%
Both forms have slightly different effects as described above.
The main file is prepared as usual, see \secref{sec:include}.

%%%%%%%%%%%%%%%%%%%%%%%%%%%%%%%%%%%%%%%%%%%%%%%%%%%%%%%%%%%%%%%%%%%%%%%%%%%%%%%%
\subsection{Legacy Detection}
\label{sec:detection}

The directive |\childdocmain| in the main file can detect
whether the complete document or merely a child is to be compiled
even without using the directive |\childdocof|.
This method is deprecated because it is less robust
and there is no compelling reason to use it;
it is merely provided for backward compatibility
and it may be removed in future versions.

If the detection mechanism is to be used,
it is mandatory to correctly specify
the filename of the main file as the argument of |\childdocmain|:
%
\begin{center}
\begin{tabular}{l}
|\input{childdoc.def}|\\
|\childdocmain{|\textit{main}|}|\\
\end{tabular}
\end{center}
%
If |\jobname| does not match the argument \textit{main} of |\childdocmain|,
it is assumed that |\jobname| points to the child file to be compiled.
When using |\childdocmain| with the main file specified as argument,
it suffices to start a child file
with just |\input{|\textit{main}|}|
without loading of the package and using |\childdocof|.
If instead all processing is done
with the appropriate \textsf{childdoc} directives,
the argument of \textit{main} of |\childdocmain| can be empty.

An alternative version of the command line processing described
in \secref{sec:commandline} using the detection mechanism reads:
%
\begin{center}
|... -jobname "|\textit{target}|" "|[\textit{flags}]%
[|\def\jobname{|\textit{dest}|}|]|\input{|\textit{main}|}"|
\end{center}

%%%%%%%%%%%%%%%%%%%%%%%%%%%%%%%%%%%%%%%%%%%%%%%%%%%%%%%%%%%%%%%%%%%%%%%%%%%%%%%%
\subsection{Manual Code}
\label{sec:manual}

In case one cannot be certain whether the definitions file |childdoc.def|
is installed on the target \TeX{} distribution
and one prefers not to ship it,
it is conceivable to paste a few relevant commands into the sources.

To that end, drop all statements |\input{childdoc.def}|
and perform the replacements as outlined below.
Instead of |\childdocmain{|\textit{main}|}| add the following code
to the top of the main file:
%
\begin{center}
\begin{tabular}{l}
|\||ifdefined\childdocname\endinput\||fi\newif\ifchilddoc|\\
|\edef\childdocname{\scantokens\expandafter{\jobname\noexpand}}|\\
|\def\childdocmain{|\textit{main}|}\||ifx\childdocmain\childdocname\||else|\\
|\childdoctrue\includeonly{\childdocname}\let\jobname\childdocmain\||fi|\\
\end{tabular}
\end{center}
%
Instead of |\childdocof{|\textit{main}|}| just include the main file
at the top of each child file:
%
\begin{center}
|\input{|\textit{main}|}|
\end{center}
%
A simple redirection |\childdocforward{|\textit{dest}|}| is achieved by:
%
\begin{center}
|\def\jobname{|\textit{dest}|}\input{\jobname}|
\end{center}
%
The redirection with prefix
|\childdocforwardprefix[|\textit{prefix}|]{|\textit{dest}|}|
is accomplished by:
%
\begin{center}
\begin{tabular}{l}
|{\edef\jobname{\scantokens\expandafter{\jobname\noexpand}}|\\
|\def\redirectjob |\textit{prefix}|#1~~~{\gdef\jobname{|\textit{dest}|#1}}|\\
|\expandafter\redirectjob\jobname~~~}\input{\jobname}|
\end{tabular}
\end{center}

In an alternative approach,
child documents can be compiled by a specific command line
without additional code or specific definitions:
%
\begin{center}
|... -jobname "|\textit{target}|" "|[\textit{flags}]%
|\includeonly{|\textit{dest}|}\input{|\textit{main}|}"|
\end{center}
%

%%%%%%%%%%%%%%%%%%%%%%%%%%%%%%%%%%%%%%%%%%%%%%%%%%%%%%%%%%%%%%%%%%%%%%%%%%%%%%%%
%%%%%%%%%%%%%%%%%%%%%%%%%%%%%%%%%%%%%%%%%%%%%%%%%%%%%%%%%%%%%%%%%%%%%%%%%%%%%%%%
\section{Information}

%%%%%%%%%%%%%%%%%%%%%%%%%%%%%%%%%%%%%%%%%%%%%%%%%%%%%%%%%%%%%%%%%%%%%%%%%%%%%%%%
\subsection{Copyright}

Copyright \copyright{} 2017--2018 Niklas Beisert

This work may be distributed and/or modified under the
conditions of the \LaTeX{} Project Public License, either version 1.3
of this license or (at your option) any later version.
The latest version of this license is in
  \url{http://www.latex-project.org/lppl.txt}
and version 1.3 or later is part of all distributions of \LaTeX{}
version 2005/12/01 or later.

This work has the LPPL maintenance status `maintained'.

The Current Maintainer of this work is Niklas Beisert.

This work consists of the files |README.txt|, |childdoc.ins| and |childdoc.dtx|
as well as the derived files |childdoc.def|, |cdocsamp.tex|
with |cdocsch1.tex|, |cdocsch2.tex|, |cdocspt3.tex|, |cdocspt4.tex|,
|cdocsdrf.tex|, |cdocsfn1.tex|, |cdocsfn2.tex|
as well as |childdoc.pdf|.

%%%%%%%%%%%%%%%%%%%%%%%%%%%%%%%%%%%%%%%%%%%%%%%%%%%%%%%%%%%%%%%%%%%%%%%%%%%%%%%%
\subsection{Files and Installation}

The package consists of the files:
%
\begin{center}
\begin{tabular}{ll}
    |README.txt|   & readme file \\
    |childdoc.ins| & installation file \\
    |childdoc.dtx| & source file \\
    |childdoc.def| & definition file \\
    |cdocsamp.tex| & sample main file \\
    |cdocsch1.tex| & sample include file \\
    |cdocsch2.tex| & sample include file \\
    |cdocspt3.tex| & sample part file \\
    |cdocspt4.tex| & sample part file \\
    |cdocsdrf.tex| & sample redirection file \\
    |cdocsfn1.tex| & sample redirection file \\
    |cdocsfn2.tex| & sample redirection file \\
    |childdoc.pdf| & manual
\end{tabular}
\end{center}
%
The distribution consists of the files
|README.txt|, |childdoc.ins| and |childdoc.dtx|.
%
\begin{itemize}
\item
Run (pdf)\LaTeX{} on |childdoc.dtx|
to compile the manual |childdoc.pdf| (this file).
\item
Run \LaTeX{} on |childdoc.ins| to create the definitions file |childdoc.def|
and the sample |cdocsamp.tex| with include files
|cdocsch1.tex|, |cdocsch2.tex|, |cdocspt3.tex|, |cdocspt4.tex|,
|cdocsdrf.tex|, |cdocsfn1.tex|, |cdocsfn2.tex|.
Then copy the file |childdoc.def| to an appropriate directory of your \LaTeX{}
distribution, e.g.\ \textit{texmf-root}|/tex/latex/childdoc|.
\end{itemize}

%%%%%%%%%%%%%%%%%%%%%%%%%%%%%%%%%%%%%%%%%%%%%%%%%%%%%%%%%%%%%%%%%%%%%%%%%%%%%%%%
\subsection{Related CTAN Packages}

There are several other packages which offer a similar functionality:
%
\begin{itemize}
\item
The packages
\href{http://ctan.org/pkg/docmute}{\textsf{docmute}},
\href{http://ctan.org/pkg/includex}{\textsf{includex}} and
\href{http://ctan.org/pkg/standalone}{\textsf{standalone}}
provide commands to include only the document body of
a child file thus allowing both files to be compiled individually.
\item
The packages \href{http://ctan.org/pkg/subdocs}{\textsf{subdocs}}
and \href{http://ctan.org/pkg/subfiles}{\textsf{subfiles}}
provide structures in which the main and child documents can be
encapsulated and allowing them to be compiled individually.
The inclusion mechanism is different from the conventional |\include|.
\item
The package \href{http://ctan.org/pkg/combine}{\textsf{combine}}
is an elaborate solution to combine several documents into one.
\end{itemize}
%
See also the CTAN topic \href{http://ctan.org/topic/subdocs}{\textsf{subdocs}}
for further related packages.
The present package differs from the above solutions in that
a document structure constructed with the conventional |\include| mechanism
just needs two extra commands at the top of every file
such that all constituent files can be compiled individually.

%%%%%%%%%%%%%%%%%%%%%%%%%%%%%%%%%%%%%%%%%%%%%%%%%%%%%%%%%%%%%%%%%%%%%%%%%%%%%%%%
%\subsection{Feature Suggestions}
%
%The following is a list of features which may be useful for future
%versions of this package:
%%
%\begin{itemize}
%\item
%\ldots
%\end{itemize}

%%%%%%%%%%%%%%%%%%%%%%%%%%%%%%%%%%%%%%%%%%%%%%%%%%%%%%%%%%%%%%%%%%%%%%%%%%%%%%%%
\subsection{Revision History}

%%%%%%%%%%%%%%%%%%%%%%%%%%%%%%%%%%%%%%%%
\paragraph{v2.0:} 2018/12/30

\begin{itemize}
\item
immediate forward processing
\item
added |\childdocby| mechanism
\item
manual restructured
\end{itemize}

%%%%%%%%%%%%%%%%%%%%%%%%%%%%%%%%%%%%%%%%
\paragraph{v1.6:} 2018/01/17

\begin{itemize}
\item
application for development of include files
\item
corrections to manual
\end{itemize}

%%%%%%%%%%%%%%%%%%%%%%%%%%%%%%%%%%%%%%%%
\paragraph{v1.5:} 2017/05/21

\begin{itemize}
\item
more complete structuring introduced
\item
|\childdocof| introduced
\item
|\childdoc| renamed to |\childdocmain|
\item
|\childredirect| renamed to |\childdocforward| and |\childdocforwardprefix|
and functionality expanded
\end{itemize}

%%%%%%%%%%%%%%%%%%%%%%%%%%%%%%%%%%%%%%%%
\paragraph{v1.0:} 2017/04/27

\begin{itemize}
\item
manual and install package
\item
first version published on CTAN
\end{itemize}

%%%%%%%%%%%%%%%%%%%%%%%%%%%%%%%%%%%%%%%%
\paragraph{v0.6:} 2017/04/26

\begin{itemize}
\item
redirection mechanism added
\end{itemize}

%%%%%%%%%%%%%%%%%%%%%%%%%%%%%%%%%%%%%%%%
\paragraph{v0.5:} 2017/04/26

\begin{itemize}
\item
functionality in definition file
\end{itemize}


%%%%%%%%%%%%%%%%%%%%%%%%%%%%%%%%%%%%%%%%%%%%%%%%%%%%%%%%%%%%%%%%%%%%%%%%%%%%%%%%
%%%%%%%%%%%%%%%%%%%%%%%%%%%%%%%%%%%%%%%%%%%%%%%%%%%%%%%%%%%%%%%%%%%%%%%%%%%%%%%%
%%%%%%%%%%%%%%%%%%%%%%%%%%%%%%%%%%%%%%%%%%%%%%%%%%%%%%%%%%%%%%%%%%%%%%%%%%%%%%%%
\appendix

\settowidth\MacroIndent{\rmfamily\scriptsize 000\ }

 \DocInput{childdoc.dtx}

\end{document}
%</driver>
% \fi
%
% %%%%%%%%%%%%%%%%%%%%%%%%%%%%%%%%%%%%%%%%%%%%%%%%%%%%%%%%%%%%%%%%%%%%%%%%%%%%%%
% %%%%%%%%%%%%%%%%%%%%%%%%%%%%%%%%%%%%%%%%%%%%%%%%%%%%%%%%%%%%%%%%%%%%%%%%%%%%%%
% \section{Sample}
%\iffalse
%<*samplemain>
%\fi
%
% The following presents a sample document
% with two chapters, two parts, a title page,
% a compile flag as well as three forwarding files to set the flag.
% It consists of eight |.tex| files:
% \begin{center}
% \begin{tabular}{ll}
% |cdocsamp.tex|&main file\\
% |cdocsch1.tex|&include file for chapter 1\\
% |cdocsch2.tex|&include file for chapter 2\\
% |cdocspt3.tex|&include file for part 3\\
% |cdocspt4.tex|&include file for part 4\\
% |cdocsdrf.tex|&forwarding file for main file in draft mode\\
% |cdocsfi1.tex|&forwarding file for final version of chapter 1\\
% |cdocsfi2.tex|&forwarding file for final version of chapter 2\\
% \end{tabular}
% \end{center}
% Each of the eight files can be compiled directly by the \LaTeX{} compiler.
%
% %%%%%%%%%%%%%%%%%%%%%%%%%%%%%%%%%%%%%%
% \paragraph{Main File.}
%
% The main file is called |cdocsamp.tex|.
%
% Load the \textsf{childdoc} definitions and
% declare the filename for the main document:
%    \begin{macrocode}
\input{childdoc.def}
\childdocmain{}
%    \end{macrocode}

% Optional override for |\version| flag:
%    \begin{macrocode}
%%\ifchilddoc\else\providecommand{\version}{draft}\fi
%    \end{macrocode}

% Define the default values for the |\version| flag
% (|final| for the main file and |draft| for childs):
%    \begin{macrocode}
\ifchilddoc
\providecommand{\version}{draft}
\else
\providecommand{\version}{final}
\fi
%    \end{macrocode}

% Load the standard document class:
%    \begin{macrocode}
\documentclass[12pt]{article}
%    \end{macrocode}

% Start the document body:
%    \begin{macrocode}
\begin{document}
%    \end{macrocode}

% Declare a title page.
% Print title, part of document being processed and version flag:
%    \begin{macrocode}
\addtocounter{page}{-1}
\begin{center}
{\LARGE\bfseries{}childdoc example\par}
\vspace{1cm}
\ifchilddoc
\ifchilddocmanual part\else chapter\fi:
`\childdocname' of `\childdocjob'\par
\else
main document: `\childdocjob'\par
\fi
version: \version\par
\end{center}
\newpage
%    \end{macrocode}

% Manually include selected file,
% otherwise process as usual:
%    \begin{macrocode}
\ifchilddocmanual
\section*{part `\childdocname'}
\input{\childdocname}
\else
%    \end{macrocode}

% Include the two chapters:
%    \begin{macrocode}
\include{cdocsch1}
\include{cdocsch2}
%    \end{macrocode}

% Include the two parts unless only chapters should be displayed:
%    \begin{macrocode}
\ifchilddoc\else
\section{part three}
\input{cdocspt3}
\section{part four}
\input{cdocspt4}
\fi
%    \end{macrocode}

% Process as usual until here:
%    \begin{macrocode}
\fi
%    \end{macrocode}

% End of document body:
%    \begin{macrocode}
\end{document}
%    \end{macrocode}
%\iffalse
%</samplemain>
%\fi
%
% %%%%%%%%%%%%%%%%%%%%%%%%%%%%%%%%%%%%%%
% \paragraph{Chapter Include Files.}
%
% The include files are called |cdocsch1.tex| and |cdocsch2.tex|.
%
%\iffalse
%<*samplechap1|samplechap2>
%\fi

% Optional override for |\version| flag:
%    \begin{macrocode}
%%\providecommand{\version}{final}
%    \end{macrocode}

% Include the main document:
%    \begin{macrocode}
\input{childdoc.def}
\childdocof{cdocsamp}
%    \end{macrocode}

%\iffalse
%</samplechap1|samplechap2>
%\fi
%
%\iffalse
%<*samplechap1>
%\fi
% Some text for chapter 1:
%    \begin{macrocode}
\section{one}
some text in chapter one
%    \end{macrocode}

%\iffalse
%</samplechap1>
%\fi
% Some text for chapter 2:
%\iffalse
%<*samplechap2>
%\fi
%    \begin{macrocode}
\section{two}
more text in chapter two
%    \end{macrocode}

%\iffalse
%</samplechap2>
%\fi
%
% %%%%%%%%%%%%%%%%%%%%%%%%%%%%%%%%%%%%%%
% \paragraph{Part Include Files.}
%
% The include files are called |cdocspt3.tex| and |cdocspt4.tex|.
%
%\iffalse
%<*samplepart3|samplepart4>
%\fi

% Optional override for |\version| flag:
%    \begin{macrocode}
%%\providecommand{\version}{final}
%    \end{macrocode}

% Include the main document:
%    \begin{macrocode}
\input{childdoc.def}
\childdocby{cdocsamp}
%    \end{macrocode}

%\iffalse
%</samplepart3|samplepart4>
%\fi
%
%\iffalse
%<*samplepart3>
%\fi
% Some text for part 3:
%    \begin{macrocode}
some text in part three
%    \end{macrocode}

%\iffalse
%</samplepart3>
%\fi
% Some text for part 4:
%\iffalse
%<*samplepart4>
%\fi
%    \begin{macrocode}
more text in part four
%    \end{macrocode}

%\iffalse
%</samplepart4>
%\fi
%
% %%%%%%%%%%%%%%%%%%%%%%%%%%%%%%%%%%%%%%
% \paragraph{Forwarding for a Complete Draft.}
%
% The following forwarding file |cdocsdrf.tex|
% compiles the main document in draft mode:
%\iffalse
%<*sampledraft>
%\fi
%    \begin{macrocode}
\def\version{draft}
\input{childdoc.def}
\childdocforward{cdocsamp}
%    \end{macrocode}

%\iffalse
%</sampledraft>
%\fi
%
% %%%%%%%%%%%%%%%%%%%%%%%%%%%%%%%%%%%%%%
% \paragraph{Forwarding for Final Version of the Chapters.}
%
% The following forwarding files |cdocsfn1.tex| and |cdocsfn2.tex|
% (with identical content)
% compile the final versions of the child documents
% |cdocsch1.tex| and |cdocsch2.tex|, respectively:
%\iffalse
%<*samplefinal>
%\fi
%    \begin{macrocode}
\def\version{final}
\input{childdoc.def}
\childdocforwardprefix[cdocsamp]{cdocsfn}{cdocsch}
%    \end{macrocode}

%\iffalse
%</samplefinal>
%\fi
%
% %%%%%%%%%%%%%%%%%%%%%%%%%%%%%%%%%%%%%%
% \paragraph{Command Line Processing.}
%
% The following three command lines generate the output files
% |cdocscld|, |cdocscl1| and |cdocscl2|
% which should be identical to
% |cdocsdrf|, |cdocsch1| and |cdocsfn2|, respectively:
% \begin{center}
% \begin{tabular}{l}
% |latex -jobname cdocscld \|\\
% |  "\def\version{draft}\input{childdoc.def}\childdocforward{cdocsamp}"|\\
% |latex -jobname cdocscl1 \|\\
% |  "\input{childdoc.def}\childdocforward[cdocsamp]{cdocsch1}"|\\
% |latex -jobname cdocscl2 \|\\
% |  "\def\version{final}\input{childdoc.def}\childdocforward{cdocsch2}"|
% \end{tabular}
% \end{center}
% Note that the trailing backslash on each first line
% merely continues the input to the second line
% (for convenient cut ant paste).
% Furthermore, the command |latex| can be replaced by any
% of its alternative versions such as |pdflatex|.
%
% %%%%%%%%%%%%%%%%%%%%%%%%%%%%%%%%%%%%%%%%%%%%%%%%%%%%%%%%%%%%%%%%%%%%%%%%%%%%%%
% %%%%%%%%%%%%%%%%%%%%%%%%%%%%%%%%%%%%%%%%%%%%%%%%%%%%%%%%%%%%%%%%%%%%%%%%%%%%%%
% \section{Implementation}
%\iffalse
%<*package>
%\fi
%
% This section describes the definitions file |childdoc.def|.

% The definitions cannot be loaded using |\usepackage| or |\RequirePackage|
% which has a mechanism to prevent loading a style file more than once.
% When loading the definitions by means of |\input|
% multiple instances have to be prevented manually:
%\iffalse
%This code needs to be before the `\ProvidesFile' directive
%which is defined at the beginning of this file.
%Therefore it is also placed there and commented out here.
%</package>
%<*discard>
%\fi
%    \begin{macrocode}
\ifdefined\childdocmain\endinput\fi
%    \end{macrocode}
%\iffalse
%</discard>
%<*package>
%\fi
%
% \macro{\ifchilddoc}
% \macro{\ifchilddocmanual}
% The conditional |\ifchilddoc| tells whether a
% child (true) or main (false) document is being compiled.
% The conditional |\ifchilddocmanual| tells whether
% the |\includeonly| mechanism is used (false) or
% the selection of child files must be performed manually (true).
% The definitions initialise to false:
%    \begin{macrocode}
\newif\ifchilddoc
\newif\ifchilddocmanual
%    \end{macrocode}

% \macro{\childdocname}
% \macro{\childdocjob}
% The macro |\childdocname| stores the name of the main document
% to be compiled. The macro |\childdocjob| stores the name of
% the document on which the \LaTeX{} compiler was originally invoked.
% The content of |\jobname| cannot be compared
% to filenames specified in the source due to different catcodes.
% The following code rescans |\jobname|, stores the result
% in |\childdocname| and saves a copy in |\childdocjob|:
%    \begin{macrocode}
\edef\childdocname{\scantokens\expandafter{\jobname\noexpand}}
\let\childdocjob\childdocname
%    \end{macrocode}

% \macro{\childdocdisable}
% The macro |\childdocdisable| prevents the main file
% from being processed more than once.
% At this stage, the main document command |\childdocmain|
% is assumed to be called once again where it should do nothing.
% Any subsequent call to it should prevent
% a secondary processing of the main document
% It overwrites the forwarding commands
% |\childdocof| and |\childdocforward|
% with empty macros to prevent further inclusions of the main document:
%    \begin{macrocode}
\newcommand{\childdocdisable}
{
  \renewcommand{\childdocmain}[1]{\renewcommand{\childdocmain}[1]{\endinput}}
  \renewcommand{\childdocof}[1]{}
  \renewcommand{\childdocby}[2][]{}
  \renewcommand{\childdocforward}[2][]{}
  \renewcommand{\childdocdisable}{}
}
%    \end{macrocode}

% \macro{\childdocmain}
% The macro |\childdocmain| is to be called at the top of the main file
% with nothing or the main filename (without extension) as argument.
% First, it breaks loops.
% If the argument is not empty and does not match |\childdocname|
% (which is set by the first inclusion of |childdoc.def|),
% |\ifchilddoc| is set to true, |\includeonly| is applied to the child file
% and |\jobname| is set to the main file
% (for proper handling of |.aux| files):
%    \begin{macrocode}
\newcommand{\childdocmain}[1]
{
  \childdocdisable\childdocmain{}
  \if?#1?\else
    \begingroup
      \def\childdoctmp{#1}
      \ifx\childdoctmp\childdocname
        \def\childdoctmp{}
      \else
        \def\childdoctmp
        {
          \childdoctrue
          \includeonly{\childdocname}
          \def\childdocjob{#1}
          \def\jobname{#1}
        }
      \fi
      \expandafter
    \endgroup
    \childdoctmp
  \fi
}
%    \end{macrocode}

% \macro{\childdocof}
% The command |\childdocof| redirects
% compilation to the main file |#1|.
%    \begin{macrocode}
\newcommand{\childdocof}[1]
{
  \childdocdisable
  \childdoctrue
  \includeonly{\childdocname}
  \def\jobname{#1}
  \def\childdocjob{#1}
  \input{#1}
}
%    \end{macrocode}

% \macro{\childdocby}
% The command |\childdocby| ....
%    \begin{macrocode}
\newcommand{\childdocby}[2][]
{
  \childdocdisable
  \childdoctrue
  \childdocmanualtrue
  \if?#1?\else
    \def\jobname{#2}
  \fi
  \def\childdocjob{#2}
  \input{#2}
  \endinput
}
%    \end{macrocode}

% \macro{\childdocforward}
% The command |\childdocforward| redirects
% compilation to the main file or
% (if the optional argument is given) a child file.
% Parameters are set as if the main file
% or a child file starting with |\childdocof| was compiled.
% Then compilation is handed over to the main file:
%    \begin{macrocode}
\newcommand{\childdocforward}[2][]
{
  \begingroup
    \if?#1?
      \def\childdoctmp
      {
        \def\childdocname{#2}
        \def\childdocjob{#2}
        \def\jobname{#2}
        \input{#2}
        \endinput
      }
    \else
      \def\childdoctmp
      {
        \childdocdisable
        \def\childdocname{#2}
        \childdoctrue
        \includeonly{#2}
        \def\childdocjob{#1}
        \def\jobname{#1}
        \input{#1}
        \endinput
      }
    \fi
    \expandafter
  \endgroup
  \childdoctmp
}
%    \end{macrocode}

% \macro{\childdocforwardprefix}
% The command |\childdocforwardprefix| redirects
% compilation to the main or a child file by means of a pattern.
% The prefix |#1| in the current filename is replaced by |#2|
% and the suffix of the current filename is kept
% (it is assumed that the filename does not contain the substring `|~~~|'
% which is used as a delimiter).
% Compilation is handed over to the new file by |\childdocforward|:
%    \begin{macrocode}
\newcommand{\childdocforwardprefix}[3][]
{
  \begingroup
    \def\childdocextract #2##1~~~{\def\childdoctmp{\childdocforward[#1]{#3##1}}}
    \expandafter\childdocextract\childdocname~~~
    \expandafter
  \endgroup
  \childdoctmp
}
%    \end{macrocode}

% \macro{\childdoc}
% The deprecated macro |\childdoc| is a legacy version of |\childdocmain|:
%    \begin{macrocode}
\newcommand{\childdoc}{\childdocmain}
%    \end{macrocode}

% \macro{\childdocredirect}
% The deprecated macro |\childdocredirect| is a legacy version
% of |\childdocforward| and |\childdocforwardprefix|:
%    \begin{macrocode}
\newcommand{\childdocredirect}[2][]
{
  \begingroup
    \if?#1?
      \def\childdoctmp{\childdocforward{#2}}
    \else
      \def\childdoctmp{\childdocforwardprefix{#1}{#2}}
    \fi
    \expandafter
  \endgroup
  \childdoctmp
}
%    \end{macrocode}

%\iffalse
%</package>
%\fi
%
\endinput
\childdocforward[cdocsamp]{cdocsch1}"|\\
% |latex -jobname cdocscl2 \|\\
% |  "\def\version{final}% \iffalse
%
% childdoc.dtx Copyright (C) 2017-2018 Niklas Beisert
%
% This work may be distributed and/or modified under the
% conditions of the LaTeX Project Public License, either version 1.3
% of this license or (at your option) any later version.
% The latest version of this license is in
%   http://www.latex-project.org/lppl.txt
% and version 1.3 or later is part of all distributions of LaTeX
% version 2005/12/01 or later.
%
% This work has the LPPL maintenance status `maintained'.
%
% The Current Maintainer of this work is Niklas Beisert.
%
% This work consists of the files childdoc.dtx and childdoc.ins
% and the derived files childdoc.def and cdocsamp.tex with
% cdocsch1.tex, cdocsch2.tex, cdocsdrf.tex, cdocsfn1.tex, cdocsfn2.tex.
%
%<package>\ifdefined\childdocmain\endinput\fi
%<package>\ProvidesFile{childdoc.def}[2018/12/30 v2.0 child document driver]
%<samplemain>\ProvidesFile{cdocsamp.tex}[2018/12/30 v2.0 sample for childdoc]
%<*driver>
%\ProvidesFile{childdoc.drv}[2018/12/30 v2.0 childdoc reference manual file]
\PassOptionsToClass{10pt,a4paper}{article}
\documentclass{ltxdoc}

\usepackage[margin=35mm]{geometry}
\usepackage{hyperref}
\usepackage{hyperxmp}
\usepackage[usenames]{color}

\hypersetup{colorlinks=true}
\hypersetup{pdfstartview=FitH}
\hypersetup{pdfpagemode=UseNone}
\hypersetup{pdfsource={}}
\hypersetup{pdflang={en-UK}}
\hypersetup{pdfcopyright={Copyright 2017-2018 Niklas Beisert.
  This work may be distributed and/or modified under the
  conditions of the LaTeX Project Public License, either version 1.3
  of this license or (at your option) any later version.}}
\hypersetup{pdflicenseurl={http://www.latex-project.org/lppl.txt}}
\hypersetup{pdfcontactaddress={ETH Zurich, ITP, HIT K,
  Wolfgang-Pauli-Strasse 27}}
\hypersetup{pdfcontactpostcode={8093}}
\hypersetup{pdfcontactcity={Zurich}}
\hypersetup{pdfcontactcountry={Switzerland}}
\hypersetup{pdfcontactemail={nbeisert@itp.phys.ethz.ch}}
\hypersetup{pdfcontacturl={http://people.phys.ethz.ch/\xmptilde nbeisert/}}

\newcommand{\secref}[1]{\hyperref[#1]{section \ref*{#1}}}

\parskip1ex
\parindent0pt
\let\olditemize\itemize
\def\itemize{\olditemize\parskip0pt}

\begin{document}

\title{The \textsf{childdoc} Package}
\hypersetup{pdftitle={The childdoc Package}}
\author{Niklas Beisert\\[2ex]
  Institut f\"ur Theoretische Physik\\
  Eidgen\"ossische Technische Hochschule Z\"urich\\
  Wolfgang-Pauli-Strasse 27, 8093 Z\"urich, Switzerland\\[1ex]
  \href{mailto:nbeisert@itp.phys.ethz.ch}
  {\texttt{nbeisert@itp.phys.ethz.ch}}}
\hypersetup{pdfauthor={Niklas Beisert}}
\hypersetup{pdfsubject={Manual for the LaTeX2e Package childdoc}}
\date{30 December 2018, \textsf{v2.0}}
\maketitle

\begin{abstract}\noindent
\textsf{childdoc} is a \LaTeXe{} package
that enables the direct compilation
of document sections included by |\include|
to individual files.
\end{abstract}

\begingroup
\parskip0ex
\tableofcontents
\endgroup

%%%%%%%%%%%%%%%%%%%%%%%%%%%%%%%%%%%%%%%%%%%%%%%%%%%%%%%%%%%%%%%%%%%%%%%%%%%%%%%%
%%%%%%%%%%%%%%%%%%%%%%%%%%%%%%%%%%%%%%%%%%%%%%%%%%%%%%%%%%%%%%%%%%%%%%%%%%%%%%%%
\section{Introduction}

\LaTeX{} provides a mechanism to structure a large document (such as a book)
into a main file and several child files (containing the chapters)
using the |\include| command.
This mechanism is beneficial for documents
which span hundreds of pages in order to
make the source file(s) more manageable.
Moreover, compilation can be restricted to
selected child files by means of the |\includeonly| command.
The latter feature can be used to reduce the compilation time while editing
(this was significantly more useful in the earlier days of \LaTeX{})
or to generate a smaller document which is easier to navigate.
Another application of |\includeonly| is to generate
documents consisting of selected parts of the complete document.

However, there are a few drawbacks of the plain |\include| mechanism:
\begin{itemize}
\item
The child files cannot be compiled on their own,
they can only be compiled via the main file.
A naive editing environment
(such as a text editor with an option
to have the current file processed by \LaTeX)
may require one to switch to the main file before compiling;
attempting to compile the child file produces errors.
\item
The main file must be modified (each time)
to adjust the |\includeonly| command
to the present needs. This easily leaves the main file in a messy state.
\item
The generated document will always carry the filename
of the main document. This is inconvenient if
several child files are to be compiled and
to be kept for distribution.
\end{itemize}

The present package provides a simple interface
to make child files individually compilable by \LaTeX{}.
Compiling a child file then has the same effect as compiling
the main file with an |\includeonly| command
to select the appropriate child.
Moreover the generated document will carry the name of the child
rather than the main file.
This resolves all three above issues.

This feature is meant to make the editing of books,
thesis documents and lecture notes somewhat more convenient.
However, the package can also be used efficiently for
composing a series of documents (such as exercise sheets)
which are typically distributed individually.
It then assists the author in generating the individual documents
(potentially in different versions)
as well as a document containing the collected series.
Another application is in developing style files
or other kinds of included material
where compilation of the style file could redirect
to a sample or test file.

%%%%%%%%%%%%%%%%%%%%%%%%%%%%%%%%%%%%%%%%%%%%%%%%%%%%%%%%%%%%%%%%%%%%%%%%%%%%%%%%
%%%%%%%%%%%%%%%%%%%%%%%%%%%%%%%%%%%%%%%%%%%%%%%%%%%%%%%%%%%%%%%%%%%%%%%%%%%%%%%%
\section{Usage}

First of all, the package \textsf{childdoc} is \emph{not} a standard
\LaTeXe{} |.sty| style file! Therefore it needs to be invoked in
a non-standard way.

%%%%%%%%%%%%%%%%%%%%%%%%%%%%%%%%%%%%%%%%%%%%%%%%%%%%%%%%%%%%%%%%%%%%%%%%%%%%%%%%
\subsection{Included Files}
\label{sec:include}

%%%%%%%%%%%%%%%%%%%%%%%%%%%%%%%%%%%%%%%%
\DescribeMacro{\childdocmain}
To use the package, add the commands
\begin{center}
\begin{tabular}{l}
|\input{childdoc.def}|\\
|\childdocmain{}|\\
\end{tabular}
\end{center}
at the very top of the main \LaTeX{} file,
in particular \emph{before} the |\documentclass| statement!
The argument of |\childdocmain| should be left empty
(but it must be present).

%%%%%%%%%%%%%%%%%%%%%%%%%%%%%%%%%%%%%%%%
\DescribeMacro{\childdocof}
Furthermore, add the commands
\begin{center}
\begin{tabular}{l}
|\input{childdoc.def}|\\
|\childdocof{|\textit{main}|}|\\
\end{tabular}
\end{center}
at the top of every child file \textit{child}
which is included by |\include{|\textit{child}|}|
from within the main file
(or at least for those files to be compiled individually).
The argument \textit{main} must be the filename of the main file.

There are a couple of
considerations in setting up the main and child documents:

%%%%%%%%%%%%%%%%%%%%%%%%%%%%%%%%%%%%%%%%
\paragraph{Restrictions.}

Please note the following restrictions:
\begin{itemize}
\item
|\childdocmain| must be called with one argument \textit{main}
to ensure compatibility with earlier version of the package.
It must either be empty (|\childdocmain{}|)
or precisely match the filename of the main file in which it is specified.
See \secref{sec:detection} for further information.
\item
The filename \textit{main} must be specified without the |.tex| extension.
\item
The filename \textit{main} is case sensitive
(even in case-insensitive file systems)
due to internal string comparison.
\item
The argument \textit{main} should be fully expanded, it cannot be a macro.
\item
Subdirectories and special characters should be avoided in filenames.
\item
The command |\childdocmain{|\textit{main}|}| must be followed by a whitespace.
It should not be followed immediately by another command
or by a comment mark `|%|'.
This is because the \TeX{} parser reads the token immediately following
the argument of |\childdocmain| and puts it
at the beginning of every child section;
however, a white\-space is ignored.
\end{itemize}

%%%%%%%%%%%%%%%%%%%%%%%%%%%%%%%%%%%%%%%%
\paragraph{Content of Main File.}

It is advisable to place all content in the child files included by |\include|.
Any output contained in the main file will appear in all child documents
unless suppressed manually;
it cannot be suppressed automatically by the |\includeonly| directive
and thus should normally be avoided.
A method to include some content in the main file
by means of conditional processing is described in \secref{sec:conditional}.

%%%%%%%%%%%%%%%%%%%%%%%%%%%%%%%%%%%%%%%%
\paragraph{Page Numbering.}

When only a part of the document is compiled,
the appropriate numbering of pages
(as well as other status parameters)
is determined from the |.aux| files.
The latter contain information from previous passes.
However this information needs to propagate through
all intermediate child documents.
Therefore the page numbering in child documents may well
be inconsistent until the complete document is compiled at least once.

A useful (if unconventional) way to always ensure a consistent
page numbering is to restart the numbering in each child document
and denote the pages by `\textit{child}|.|\textit{page}'
where \textit{child} represents the chapter/section number of the child file.
This can be achieved by the command
|\numberwithin{page}{|\textit{child}|}|
of the \textsf{amsmath} package
where \textit{child} can be |chapter| or |section|
depending on the chosen structuring.
Alternatively, one can modify the macro |\thepage| appropriately
and reset the counter |page| at the start of each child file.

%%%%%%%%%%%%%%%%%%%%%%%%%%%%%%%%%%%%%%%%%%%%%%%%%%%%%%%%%%%%%%%%%%%%%%%%%%%%%%%%
\subsection{Conditional Processing}
\label{sec:conditional}

The package provides a mechanism to compile different versions
of a document. To customise the versions further some conditional processing
can come in handy to distinguish which version is being compiled.
The package provides two macros to describe the compilation context:

%%%%%%%%%%%%%%%%%%%%%%%%%%%%%%%%%%%%%%%%
\DescribeMacro{\ifchilddoc}
The conditional |\ifchilddoc| distinguishes between the compilation of
child documents and the main document:
%
\begin{center}
|\ifchilddoc |\textit{child-code}| |[|\||else |\textit{main-code}]| \||fi|
\end{center}

%%%%%%%%%%%%%%%%%%%%%%%%%%%%%%%%%%%%%%%%
\DescribeMacro{\childdocname}
\DescribeMacro{\childdocjob}
The macro |\childdocname| contains the filename (without extension)
of the main or child file being processed.
Note that |\childdocjob| will always contain the name of the main file.

%%%%%%%%%%%%%%%%%%%%%%%%%%%%%%%%%%%%%%%%
\paragraph{Title Page.}

Conditional processing can be used to include a title or banner page
in the main document when proper precautions are taken.
Importantly, the code in the main file should ensure that the page counter
(as well as other status parameters which are stored in the |.aux| files)
takes the same value after the conditional processing.
Otherwise the page numbers may take divergent values
depending on which part is compiled.

For example, a title page could be declared by:
%
\begin{center}
\begin{tabular}{l}
|\ifchilddoc\||else|\\
|\addtocounter{page}{-1}|\\
\textit{code for title page}\\
|\newpage|\\
|\||fi|
\end{tabular}
\end{center}
%
A banner page for the child documents can be generated by:
%
\begin{center}
\begin{tabular}{l}
|\ifchilddoc|\\
|\addtocounter{page}{-1}|\\
\textit{code for banner page}\\
|\newpage|\\
|\||fi|
\end{tabular}
\end{center}
%
Here one could write a message such as:
\begin{center}
|This is the part \childdocname{} of \childdocjob{}.|
\end{center}

%%%%%%%%%%%%%%%%%%%%%%%%%%%%%%%%%%%%%%%%%%%%%%%%%%%%%%%%%%%%%%%%%%%%%%%%%%%%%%%%
\subsection{Flags}
\label{sec:flags}

The package makes it easy to generate different versions
of the main or child documents.
To this end compilation flags can be defined
and assigned different default values.
They will be particularly useful in conjunction
with the forwarding mechanism described in \secref{sec:forward}.

For example, it may be useful to have a flag |\version|
which can be set to |draft| or |final|.
The document source will contain some conditional code
depending on the value of |\version|.
Suppose further, the flag should default to |final| for the main file
and to |draft| for child files
which is a natural assignment for editing the document.
This is achieved by placing the following code
in the preamble of the main document
(below the |\childdocmain| directive):
%
\begin{center}
\begin{tabular}{l}
|\ifchilddoc|\\
|\providecommand{\version}{draft}|\\
|\||else|\\
|\providecommand{\version}{final}|\\
|\||fi|
\end{tabular}
\end{center}
%
The definition by |\providecommand| makes sure
that previous definitions are not overwritten.
Further statements |\providecommand{\version}{...}|
can thus be added before the above code to override it.

For the main file, one might add a line
(between |\childdocmain| and the above block)
%
\begin{center}
|%\ifchilddoc\||else\providecommand{\version}{draft}\||fi|
\end{center}
%
which can be uncommented to produce a draft version.
Likewise one can add a line to the very top of a child file
(above the |\childdocof{|\textit{main}|}| directive)
%
\begin{center}
|%\providecommand{\version}{final}|
\end{center}
%
which can be uncommented to produce the final version of this child document.

%%%%%%%%%%%%%%%%%%%%%%%%%%%%%%%%%%%%%%%%%%%%%%%%%%%%%%%%%%%%%%%%%%%%%%%%%%%%%%%%
\subsection{Forwarding}
\label{sec:forward}

Different versions of the main or child documents
using compilation flags as described in \secref{sec:flags}
can be (permanently) stored in different files
for convenient compilation, viewing and distribution.
To this end, the package defines a command
to pass on compilation to a different file:

%%%%%%%%%%%%%%%%%%%%%%%%%%%%%%%%%%%%%%%%
\DescribeMacro{\childdocforward}
The command |\childdocforward| redirects processing to
another source file:
%
\begin{center}
\begin{tabular}{l}
|\input{childdoc.def}|\\
|\childdocforward[|\textit{main}|]{|\textit{dest}|}|\\
\end{tabular}
\end{center}
%
The argument \textit{dest} is the destination file
(without extension).
It should be the main file or one of the child files.
Note that further \textsf{childdoc} directives
such as |\childdocof| and |\childdocforward|
in the indicated file will be processed in this form.
The optional argument \textit{main}
passes on directly to the main file \textit{main}
while pretending to compile the child \textit{dest}.
This form behaves as if \textit{dest}
issues |\childdocof{|\textit{main}|}| right away,
and no further \textsf{childdoc} directives will be processed.

%%%%%%%%%%%%%%%%%%%%%%%%%%%%%%%%%%%%%%%%
\DescribeMacro{\...prefix}
In the alternative form |\childdocforwardprefix|,
%
\begin{center}
\begin{tabular}{l}
|\input{childdoc.def}|\\
|\childdocforwardprefix[|\textit{main}|]{|\textit{prefix}|}{|\textit{dest}|}|
\end{tabular}
\end{center}
%
the destination file is determined by a pattern
depending on the current file:
To make this work, the current file must be called
`{\textit{prefix}\hspace{0.2em}\textit{suffix}}'
with \textit{prefix} matching precisely the argument.
Processing is then passed on to the file
`{\textit{dest}\hspace{0.2em}\textit{suffix}}'.
Surely, the same effect is achieved by
directly specifying the
argument `{\textit{dest}\hspace{0.2em}\textit{suffix}}'
in the first form.
However, that requires to set up a different file
for each child. With the alternative form of the command
all these files can have exactly the same content
which simplifies setting them up and maintaining them.

For example, the following file |draft.tex|
with a compilation flag |\version| as described in \secref{sec:flags}
compiles the main document as a draft:
%
\begin{center}
\begin{tabular}{l}
|\def\version{draft}|\\
|\input{childdoc.def}|\\
|\childdocforward{|\textit{main}|}|
\end{tabular}
\end{center}
%
Likewise, the following files |final|\textit{nn}|.tex|
compile the final version of the child document
|child|\textit{nn}|.tex|:
%
\begin{center}
\begin{tabular}{l}
|\def\version{final}|\\
|\input{childdoc.def}|\\
|\childdocforwardprefix{final}{child}|
\end{tabular}
\end{center}
%

Note that when several versions of a main file and/or of each child file
are to be generated, it may be convenient to set up a |Makefile| or
shell script to automatise the process.

%%%%%%%%%%%%%%%%%%%%%%%%%%%%%%%%%%%%%%%%%%%%%%%%%%%%%%%%%%%%%%%%%%%%%%%%%%%%%%%%
\subsection{Command Line Processing}
\label{sec:commandline}

The effect of redirection files can also be achieved by invoking
the \LaTeX{} compiler with a more elaborate command line.
Most conveniently this should be done as part
of a shell script or a |Makefile|.

When using \textsf{childdoc} in the main file, the following
command lines effectively perform a redirection
(note that depending on the shell being used,
backslashes may have to be doubled: `|\|' $\to$ `|\\|'):
%
\begin{center}
|... -jobname "|\textit{target}|" |\\|"|[\textit{flags}]%
|\input{childdoc.def}\childdocforward[|\textit{main}|]{|\textit{dest}|}"|
\end{center}
%
Here \textit{target} is the name of the output file,
\textit{main} is the name of the main file
and \textit{dest} is the name of the main or child file to be processed
(all filenames without extensions).
The optional argument \textit{main} can be omitted
if \textit{main} matches \textit{dest}.
Optionally, compilation \textit{flags} can be defined via |\def| commands.
This command line makes the \TeX{} engine believe
it is compiling the file \textit{target}
whose content is specified as the latter parameter.
The provided code then forwards the processing to
\textit{main} or \textit{dest} as described in \secref{sec:forward}.

%%%%%%%%%%%%%%%%%%%%%%%%%%%%%%%%%%%%%%%%%%%%%%%%%%%%%%%%%%%%%%%%%%%%%%%%%%%%%%%%
\subsection{Include by Input}
\label{sec:input}

Including child documents by |\include| has some restrictions by design.
Most notably, the content of a child document always occupies
its own set of pages; pages cannot be shared between child documents.
Usually, this behaviour makes perfect sense
because each child document contain an essential part of the document.
However, in some situations it may be desirable to compose
a document from a collection of parts
without having mandatory page breaks between then.
For this case, the package
provides a mechanism to include parts
by |\input| which can also be processed individually.
However, by construction this mechanism
requires manual handling of the content to be output.

%%%%%%%%%%%%%%%%%%%%%%%%%%%%%%%%%%%%%%%%
\DescribeMacro{\ifchilddocmanual}
The main file should be prepared as usual, see \secref{sec:include}.
However, the document body must make a distinction
between processing of an individual part and of the main document, e.g.:
%
\begin{center}
\begin{tabular}{l}
|\ifchilddocmanual|\\
|\input{\childdocname}|\\
|\||else|\\
\textit{document body with }|\input{|\textit{part}|}|\\
|\||fi|
\end{tabular}
\end{center}
%
The conditional |\ifchilddocmanual| is true whenever
a part to be included by |\input| is being compiled,
and the name of the part is stored in |\childdocname|.

%%%%%%%%%%%%%%%%%%%%%%%%%%%%%%%%%%%%%%%%
\DescribeMacro{\childdocby}
Each part to be included by |\input| should start with:
%
\begin{center}
\begin{tabular}{l}
|\input{childdoc.def}|\\
|\childdocby{|\textit{main}|}|\\
\end{tabular}
\end{center}
%
The directive |\childdocby| is similar to |\childdocof|
described in \secref{sec:include},
but the subsequent selection of content must be done manually.
To that end, both |\ifchilddoc| and |\ifchilddocmanual|
will be true upon processing of a part,
and the name of the part is stored in |\childdocname|.
Note that |\jobname| will be set to the filename of the current part
so that each part receives an individual |.aux| file
that does not interfere with the |.aux| file(s) of the main document.
This behaviour can be altered by the alternative form
|\childdocby[*]{|\textit{main}|}| (with a non-empty optional argument)
which uses the |.aux| file of the main document
by setting |\jobname| to \textit{main}.

%%%%%%%%%%%%%%%%%%%%%%%%%%%%%%%%%%%%%%%%%%%%%%%%%%%%%%%%%%%%%%%%%%%%%%%%%%%%%%%%
\subsection{Driver Development}
\label{sec:driver}

The \textsf{childdoc} mechanism can also be use for the development
of definition files such as \LaTeX{} styles or classes.
This case differs from the above setup with multiple parts
included by |\include| in that no |\includeonly| should be invoked.
This can be achieved by starting the include file
(before |\ProvidesPackage|) with:
%
\begin{center}
\begin{tabular}{l}
|\input{childdoc.def}|\\
|\childdocforward{|\textit{main}|}|\\
\end{tabular}
\end{center}
%
or alternatively with:
%
\begin{center}
\begin{tabular}{l}
|\input{childdoc.def}|\\
|\childdocby{|\textit{main}|}|\\
\end{tabular}
\end{center}
%
Both forms have slightly different effects as described above.
The main file is prepared as usual, see \secref{sec:include}.

%%%%%%%%%%%%%%%%%%%%%%%%%%%%%%%%%%%%%%%%%%%%%%%%%%%%%%%%%%%%%%%%%%%%%%%%%%%%%%%%
\subsection{Legacy Detection}
\label{sec:detection}

The directive |\childdocmain| in the main file can detect
whether the complete document or merely a child is to be compiled
even without using the directive |\childdocof|.
This method is deprecated because it is less robust
and there is no compelling reason to use it;
it is merely provided for backward compatibility
and it may be removed in future versions.

If the detection mechanism is to be used,
it is mandatory to correctly specify
the filename of the main file as the argument of |\childdocmain|:
%
\begin{center}
\begin{tabular}{l}
|\input{childdoc.def}|\\
|\childdocmain{|\textit{main}|}|\\
\end{tabular}
\end{center}
%
If |\jobname| does not match the argument \textit{main} of |\childdocmain|,
it is assumed that |\jobname| points to the child file to be compiled.
When using |\childdocmain| with the main file specified as argument,
it suffices to start a child file
with just |\input{|\textit{main}|}|
without loading of the package and using |\childdocof|.
If instead all processing is done
with the appropriate \textsf{childdoc} directives,
the argument of \textit{main} of |\childdocmain| can be empty.

An alternative version of the command line processing described
in \secref{sec:commandline} using the detection mechanism reads:
%
\begin{center}
|... -jobname "|\textit{target}|" "|[\textit{flags}]%
[|\def\jobname{|\textit{dest}|}|]|\input{|\textit{main}|}"|
\end{center}

%%%%%%%%%%%%%%%%%%%%%%%%%%%%%%%%%%%%%%%%%%%%%%%%%%%%%%%%%%%%%%%%%%%%%%%%%%%%%%%%
\subsection{Manual Code}
\label{sec:manual}

In case one cannot be certain whether the definitions file |childdoc.def|
is installed on the target \TeX{} distribution
and one prefers not to ship it,
it is conceivable to paste a few relevant commands into the sources.

To that end, drop all statements |\input{childdoc.def}|
and perform the replacements as outlined below.
Instead of |\childdocmain{|\textit{main}|}| add the following code
to the top of the main file:
%
\begin{center}
\begin{tabular}{l}
|\||ifdefined\childdocname\endinput\||fi\newif\ifchilddoc|\\
|\edef\childdocname{\scantokens\expandafter{\jobname\noexpand}}|\\
|\def\childdocmain{|\textit{main}|}\||ifx\childdocmain\childdocname\||else|\\
|\childdoctrue\includeonly{\childdocname}\let\jobname\childdocmain\||fi|\\
\end{tabular}
\end{center}
%
Instead of |\childdocof{|\textit{main}|}| just include the main file
at the top of each child file:
%
\begin{center}
|\input{|\textit{main}|}|
\end{center}
%
A simple redirection |\childdocforward{|\textit{dest}|}| is achieved by:
%
\begin{center}
|\def\jobname{|\textit{dest}|}\input{\jobname}|
\end{center}
%
The redirection with prefix
|\childdocforwardprefix[|\textit{prefix}|]{|\textit{dest}|}|
is accomplished by:
%
\begin{center}
\begin{tabular}{l}
|{\edef\jobname{\scantokens\expandafter{\jobname\noexpand}}|\\
|\def\redirectjob |\textit{prefix}|#1~~~{\gdef\jobname{|\textit{dest}|#1}}|\\
|\expandafter\redirectjob\jobname~~~}\input{\jobname}|
\end{tabular}
\end{center}

In an alternative approach,
child documents can be compiled by a specific command line
without additional code or specific definitions:
%
\begin{center}
|... -jobname "|\textit{target}|" "|[\textit{flags}]%
|\includeonly{|\textit{dest}|}\input{|\textit{main}|}"|
\end{center}
%

%%%%%%%%%%%%%%%%%%%%%%%%%%%%%%%%%%%%%%%%%%%%%%%%%%%%%%%%%%%%%%%%%%%%%%%%%%%%%%%%
%%%%%%%%%%%%%%%%%%%%%%%%%%%%%%%%%%%%%%%%%%%%%%%%%%%%%%%%%%%%%%%%%%%%%%%%%%%%%%%%
\section{Information}

%%%%%%%%%%%%%%%%%%%%%%%%%%%%%%%%%%%%%%%%%%%%%%%%%%%%%%%%%%%%%%%%%%%%%%%%%%%%%%%%
\subsection{Copyright}

Copyright \copyright{} 2017--2018 Niklas Beisert

This work may be distributed and/or modified under the
conditions of the \LaTeX{} Project Public License, either version 1.3
of this license or (at your option) any later version.
The latest version of this license is in
  \url{http://www.latex-project.org/lppl.txt}
and version 1.3 or later is part of all distributions of \LaTeX{}
version 2005/12/01 or later.

This work has the LPPL maintenance status `maintained'.

The Current Maintainer of this work is Niklas Beisert.

This work consists of the files |README.txt|, |childdoc.ins| and |childdoc.dtx|
as well as the derived files |childdoc.def|, |cdocsamp.tex|
with |cdocsch1.tex|, |cdocsch2.tex|, |cdocspt3.tex|, |cdocspt4.tex|,
|cdocsdrf.tex|, |cdocsfn1.tex|, |cdocsfn2.tex|
as well as |childdoc.pdf|.

%%%%%%%%%%%%%%%%%%%%%%%%%%%%%%%%%%%%%%%%%%%%%%%%%%%%%%%%%%%%%%%%%%%%%%%%%%%%%%%%
\subsection{Files and Installation}

The package consists of the files:
%
\begin{center}
\begin{tabular}{ll}
    |README.txt|   & readme file \\
    |childdoc.ins| & installation file \\
    |childdoc.dtx| & source file \\
    |childdoc.def| & definition file \\
    |cdocsamp.tex| & sample main file \\
    |cdocsch1.tex| & sample include file \\
    |cdocsch2.tex| & sample include file \\
    |cdocspt3.tex| & sample part file \\
    |cdocspt4.tex| & sample part file \\
    |cdocsdrf.tex| & sample redirection file \\
    |cdocsfn1.tex| & sample redirection file \\
    |cdocsfn2.tex| & sample redirection file \\
    |childdoc.pdf| & manual
\end{tabular}
\end{center}
%
The distribution consists of the files
|README.txt|, |childdoc.ins| and |childdoc.dtx|.
%
\begin{itemize}
\item
Run (pdf)\LaTeX{} on |childdoc.dtx|
to compile the manual |childdoc.pdf| (this file).
\item
Run \LaTeX{} on |childdoc.ins| to create the definitions file |childdoc.def|
and the sample |cdocsamp.tex| with include files
|cdocsch1.tex|, |cdocsch2.tex|, |cdocspt3.tex|, |cdocspt4.tex|,
|cdocsdrf.tex|, |cdocsfn1.tex|, |cdocsfn2.tex|.
Then copy the file |childdoc.def| to an appropriate directory of your \LaTeX{}
distribution, e.g.\ \textit{texmf-root}|/tex/latex/childdoc|.
\end{itemize}

%%%%%%%%%%%%%%%%%%%%%%%%%%%%%%%%%%%%%%%%%%%%%%%%%%%%%%%%%%%%%%%%%%%%%%%%%%%%%%%%
\subsection{Related CTAN Packages}

There are several other packages which offer a similar functionality:
%
\begin{itemize}
\item
The packages
\href{http://ctan.org/pkg/docmute}{\textsf{docmute}},
\href{http://ctan.org/pkg/includex}{\textsf{includex}} and
\href{http://ctan.org/pkg/standalone}{\textsf{standalone}}
provide commands to include only the document body of
a child file thus allowing both files to be compiled individually.
\item
The packages \href{http://ctan.org/pkg/subdocs}{\textsf{subdocs}}
and \href{http://ctan.org/pkg/subfiles}{\textsf{subfiles}}
provide structures in which the main and child documents can be
encapsulated and allowing them to be compiled individually.
The inclusion mechanism is different from the conventional |\include|.
\item
The package \href{http://ctan.org/pkg/combine}{\textsf{combine}}
is an elaborate solution to combine several documents into one.
\end{itemize}
%
See also the CTAN topic \href{http://ctan.org/topic/subdocs}{\textsf{subdocs}}
for further related packages.
The present package differs from the above solutions in that
a document structure constructed with the conventional |\include| mechanism
just needs two extra commands at the top of every file
such that all constituent files can be compiled individually.

%%%%%%%%%%%%%%%%%%%%%%%%%%%%%%%%%%%%%%%%%%%%%%%%%%%%%%%%%%%%%%%%%%%%%%%%%%%%%%%%
%\subsection{Feature Suggestions}
%
%The following is a list of features which may be useful for future
%versions of this package:
%%
%\begin{itemize}
%\item
%\ldots
%\end{itemize}

%%%%%%%%%%%%%%%%%%%%%%%%%%%%%%%%%%%%%%%%%%%%%%%%%%%%%%%%%%%%%%%%%%%%%%%%%%%%%%%%
\subsection{Revision History}

%%%%%%%%%%%%%%%%%%%%%%%%%%%%%%%%%%%%%%%%
\paragraph{v2.0:} 2018/12/30

\begin{itemize}
\item
immediate forward processing
\item
added |\childdocby| mechanism
\item
manual restructured
\end{itemize}

%%%%%%%%%%%%%%%%%%%%%%%%%%%%%%%%%%%%%%%%
\paragraph{v1.6:} 2018/01/17

\begin{itemize}
\item
application for development of include files
\item
corrections to manual
\end{itemize}

%%%%%%%%%%%%%%%%%%%%%%%%%%%%%%%%%%%%%%%%
\paragraph{v1.5:} 2017/05/21

\begin{itemize}
\item
more complete structuring introduced
\item
|\childdocof| introduced
\item
|\childdoc| renamed to |\childdocmain|
\item
|\childredirect| renamed to |\childdocforward| and |\childdocforwardprefix|
and functionality expanded
\end{itemize}

%%%%%%%%%%%%%%%%%%%%%%%%%%%%%%%%%%%%%%%%
\paragraph{v1.0:} 2017/04/27

\begin{itemize}
\item
manual and install package
\item
first version published on CTAN
\end{itemize}

%%%%%%%%%%%%%%%%%%%%%%%%%%%%%%%%%%%%%%%%
\paragraph{v0.6:} 2017/04/26

\begin{itemize}
\item
redirection mechanism added
\end{itemize}

%%%%%%%%%%%%%%%%%%%%%%%%%%%%%%%%%%%%%%%%
\paragraph{v0.5:} 2017/04/26

\begin{itemize}
\item
functionality in definition file
\end{itemize}


%%%%%%%%%%%%%%%%%%%%%%%%%%%%%%%%%%%%%%%%%%%%%%%%%%%%%%%%%%%%%%%%%%%%%%%%%%%%%%%%
%%%%%%%%%%%%%%%%%%%%%%%%%%%%%%%%%%%%%%%%%%%%%%%%%%%%%%%%%%%%%%%%%%%%%%%%%%%%%%%%
%%%%%%%%%%%%%%%%%%%%%%%%%%%%%%%%%%%%%%%%%%%%%%%%%%%%%%%%%%%%%%%%%%%%%%%%%%%%%%%%
\appendix

\settowidth\MacroIndent{\rmfamily\scriptsize 000\ }

 \DocInput{childdoc.dtx}

\end{document}
%</driver>
% \fi
%
% %%%%%%%%%%%%%%%%%%%%%%%%%%%%%%%%%%%%%%%%%%%%%%%%%%%%%%%%%%%%%%%%%%%%%%%%%%%%%%
% %%%%%%%%%%%%%%%%%%%%%%%%%%%%%%%%%%%%%%%%%%%%%%%%%%%%%%%%%%%%%%%%%%%%%%%%%%%%%%
% \section{Sample}
%\iffalse
%<*samplemain>
%\fi
%
% The following presents a sample document
% with two chapters, two parts, a title page,
% a compile flag as well as three forwarding files to set the flag.
% It consists of eight |.tex| files:
% \begin{center}
% \begin{tabular}{ll}
% |cdocsamp.tex|&main file\\
% |cdocsch1.tex|&include file for chapter 1\\
% |cdocsch2.tex|&include file for chapter 2\\
% |cdocspt3.tex|&include file for part 3\\
% |cdocspt4.tex|&include file for part 4\\
% |cdocsdrf.tex|&forwarding file for main file in draft mode\\
% |cdocsfi1.tex|&forwarding file for final version of chapter 1\\
% |cdocsfi2.tex|&forwarding file for final version of chapter 2\\
% \end{tabular}
% \end{center}
% Each of the eight files can be compiled directly by the \LaTeX{} compiler.
%
% %%%%%%%%%%%%%%%%%%%%%%%%%%%%%%%%%%%%%%
% \paragraph{Main File.}
%
% The main file is called |cdocsamp.tex|.
%
% Load the \textsf{childdoc} definitions and
% declare the filename for the main document:
%    \begin{macrocode}
\input{childdoc.def}
\childdocmain{}
%    \end{macrocode}

% Optional override for |\version| flag:
%    \begin{macrocode}
%%\ifchilddoc\else\providecommand{\version}{draft}\fi
%    \end{macrocode}

% Define the default values for the |\version| flag
% (|final| for the main file and |draft| for childs):
%    \begin{macrocode}
\ifchilddoc
\providecommand{\version}{draft}
\else
\providecommand{\version}{final}
\fi
%    \end{macrocode}

% Load the standard document class:
%    \begin{macrocode}
\documentclass[12pt]{article}
%    \end{macrocode}

% Start the document body:
%    \begin{macrocode}
\begin{document}
%    \end{macrocode}

% Declare a title page.
% Print title, part of document being processed and version flag:
%    \begin{macrocode}
\addtocounter{page}{-1}
\begin{center}
{\LARGE\bfseries{}childdoc example\par}
\vspace{1cm}
\ifchilddoc
\ifchilddocmanual part\else chapter\fi:
`\childdocname' of `\childdocjob'\par
\else
main document: `\childdocjob'\par
\fi
version: \version\par
\end{center}
\newpage
%    \end{macrocode}

% Manually include selected file,
% otherwise process as usual:
%    \begin{macrocode}
\ifchilddocmanual
\section*{part `\childdocname'}
\input{\childdocname}
\else
%    \end{macrocode}

% Include the two chapters:
%    \begin{macrocode}
\include{cdocsch1}
\include{cdocsch2}
%    \end{macrocode}

% Include the two parts unless only chapters should be displayed:
%    \begin{macrocode}
\ifchilddoc\else
\section{part three}
\input{cdocspt3}
\section{part four}
\input{cdocspt4}
\fi
%    \end{macrocode}

% Process as usual until here:
%    \begin{macrocode}
\fi
%    \end{macrocode}

% End of document body:
%    \begin{macrocode}
\end{document}
%    \end{macrocode}
%\iffalse
%</samplemain>
%\fi
%
% %%%%%%%%%%%%%%%%%%%%%%%%%%%%%%%%%%%%%%
% \paragraph{Chapter Include Files.}
%
% The include files are called |cdocsch1.tex| and |cdocsch2.tex|.
%
%\iffalse
%<*samplechap1|samplechap2>
%\fi

% Optional override for |\version| flag:
%    \begin{macrocode}
%%\providecommand{\version}{final}
%    \end{macrocode}

% Include the main document:
%    \begin{macrocode}
\input{childdoc.def}
\childdocof{cdocsamp}
%    \end{macrocode}

%\iffalse
%</samplechap1|samplechap2>
%\fi
%
%\iffalse
%<*samplechap1>
%\fi
% Some text for chapter 1:
%    \begin{macrocode}
\section{one}
some text in chapter one
%    \end{macrocode}

%\iffalse
%</samplechap1>
%\fi
% Some text for chapter 2:
%\iffalse
%<*samplechap2>
%\fi
%    \begin{macrocode}
\section{two}
more text in chapter two
%    \end{macrocode}

%\iffalse
%</samplechap2>
%\fi
%
% %%%%%%%%%%%%%%%%%%%%%%%%%%%%%%%%%%%%%%
% \paragraph{Part Include Files.}
%
% The include files are called |cdocspt3.tex| and |cdocspt4.tex|.
%
%\iffalse
%<*samplepart3|samplepart4>
%\fi

% Optional override for |\version| flag:
%    \begin{macrocode}
%%\providecommand{\version}{final}
%    \end{macrocode}

% Include the main document:
%    \begin{macrocode}
\input{childdoc.def}
\childdocby{cdocsamp}
%    \end{macrocode}

%\iffalse
%</samplepart3|samplepart4>
%\fi
%
%\iffalse
%<*samplepart3>
%\fi
% Some text for part 3:
%    \begin{macrocode}
some text in part three
%    \end{macrocode}

%\iffalse
%</samplepart3>
%\fi
% Some text for part 4:
%\iffalse
%<*samplepart4>
%\fi
%    \begin{macrocode}
more text in part four
%    \end{macrocode}

%\iffalse
%</samplepart4>
%\fi
%
% %%%%%%%%%%%%%%%%%%%%%%%%%%%%%%%%%%%%%%
% \paragraph{Forwarding for a Complete Draft.}
%
% The following forwarding file |cdocsdrf.tex|
% compiles the main document in draft mode:
%\iffalse
%<*sampledraft>
%\fi
%    \begin{macrocode}
\def\version{draft}
\input{childdoc.def}
\childdocforward{cdocsamp}
%    \end{macrocode}

%\iffalse
%</sampledraft>
%\fi
%
% %%%%%%%%%%%%%%%%%%%%%%%%%%%%%%%%%%%%%%
% \paragraph{Forwarding for Final Version of the Chapters.}
%
% The following forwarding files |cdocsfn1.tex| and |cdocsfn2.tex|
% (with identical content)
% compile the final versions of the child documents
% |cdocsch1.tex| and |cdocsch2.tex|, respectively:
%\iffalse
%<*samplefinal>
%\fi
%    \begin{macrocode}
\def\version{final}
\input{childdoc.def}
\childdocforwardprefix[cdocsamp]{cdocsfn}{cdocsch}
%    \end{macrocode}

%\iffalse
%</samplefinal>
%\fi
%
% %%%%%%%%%%%%%%%%%%%%%%%%%%%%%%%%%%%%%%
% \paragraph{Command Line Processing.}
%
% The following three command lines generate the output files
% |cdocscld|, |cdocscl1| and |cdocscl2|
% which should be identical to
% |cdocsdrf|, |cdocsch1| and |cdocsfn2|, respectively:
% \begin{center}
% \begin{tabular}{l}
% |latex -jobname cdocscld \|\\
% |  "\def\version{draft}\input{childdoc.def}\childdocforward{cdocsamp}"|\\
% |latex -jobname cdocscl1 \|\\
% |  "\input{childdoc.def}\childdocforward[cdocsamp]{cdocsch1}"|\\
% |latex -jobname cdocscl2 \|\\
% |  "\def\version{final}\input{childdoc.def}\childdocforward{cdocsch2}"|
% \end{tabular}
% \end{center}
% Note that the trailing backslash on each first line
% merely continues the input to the second line
% (for convenient cut ant paste).
% Furthermore, the command |latex| can be replaced by any
% of its alternative versions such as |pdflatex|.
%
% %%%%%%%%%%%%%%%%%%%%%%%%%%%%%%%%%%%%%%%%%%%%%%%%%%%%%%%%%%%%%%%%%%%%%%%%%%%%%%
% %%%%%%%%%%%%%%%%%%%%%%%%%%%%%%%%%%%%%%%%%%%%%%%%%%%%%%%%%%%%%%%%%%%%%%%%%%%%%%
% \section{Implementation}
%\iffalse
%<*package>
%\fi
%
% This section describes the definitions file |childdoc.def|.

% The definitions cannot be loaded using |\usepackage| or |\RequirePackage|
% which has a mechanism to prevent loading a style file more than once.
% When loading the definitions by means of |\input|
% multiple instances have to be prevented manually:
%\iffalse
%This code needs to be before the `\ProvidesFile' directive
%which is defined at the beginning of this file.
%Therefore it is also placed there and commented out here.
%</package>
%<*discard>
%\fi
%    \begin{macrocode}
\ifdefined\childdocmain\endinput\fi
%    \end{macrocode}
%\iffalse
%</discard>
%<*package>
%\fi
%
% \macro{\ifchilddoc}
% \macro{\ifchilddocmanual}
% The conditional |\ifchilddoc| tells whether a
% child (true) or main (false) document is being compiled.
% The conditional |\ifchilddocmanual| tells whether
% the |\includeonly| mechanism is used (false) or
% the selection of child files must be performed manually (true).
% The definitions initialise to false:
%    \begin{macrocode}
\newif\ifchilddoc
\newif\ifchilddocmanual
%    \end{macrocode}

% \macro{\childdocname}
% \macro{\childdocjob}
% The macro |\childdocname| stores the name of the main document
% to be compiled. The macro |\childdocjob| stores the name of
% the document on which the \LaTeX{} compiler was originally invoked.
% The content of |\jobname| cannot be compared
% to filenames specified in the source due to different catcodes.
% The following code rescans |\jobname|, stores the result
% in |\childdocname| and saves a copy in |\childdocjob|:
%    \begin{macrocode}
\edef\childdocname{\scantokens\expandafter{\jobname\noexpand}}
\let\childdocjob\childdocname
%    \end{macrocode}

% \macro{\childdocdisable}
% The macro |\childdocdisable| prevents the main file
% from being processed more than once.
% At this stage, the main document command |\childdocmain|
% is assumed to be called once again where it should do nothing.
% Any subsequent call to it should prevent
% a secondary processing of the main document
% It overwrites the forwarding commands
% |\childdocof| and |\childdocforward|
% with empty macros to prevent further inclusions of the main document:
%    \begin{macrocode}
\newcommand{\childdocdisable}
{
  \renewcommand{\childdocmain}[1]{\renewcommand{\childdocmain}[1]{\endinput}}
  \renewcommand{\childdocof}[1]{}
  \renewcommand{\childdocby}[2][]{}
  \renewcommand{\childdocforward}[2][]{}
  \renewcommand{\childdocdisable}{}
}
%    \end{macrocode}

% \macro{\childdocmain}
% The macro |\childdocmain| is to be called at the top of the main file
% with nothing or the main filename (without extension) as argument.
% First, it breaks loops.
% If the argument is not empty and does not match |\childdocname|
% (which is set by the first inclusion of |childdoc.def|),
% |\ifchilddoc| is set to true, |\includeonly| is applied to the child file
% and |\jobname| is set to the main file
% (for proper handling of |.aux| files):
%    \begin{macrocode}
\newcommand{\childdocmain}[1]
{
  \childdocdisable\childdocmain{}
  \if?#1?\else
    \begingroup
      \def\childdoctmp{#1}
      \ifx\childdoctmp\childdocname
        \def\childdoctmp{}
      \else
        \def\childdoctmp
        {
          \childdoctrue
          \includeonly{\childdocname}
          \def\childdocjob{#1}
          \def\jobname{#1}
        }
      \fi
      \expandafter
    \endgroup
    \childdoctmp
  \fi
}
%    \end{macrocode}

% \macro{\childdocof}
% The command |\childdocof| redirects
% compilation to the main file |#1|.
%    \begin{macrocode}
\newcommand{\childdocof}[1]
{
  \childdocdisable
  \childdoctrue
  \includeonly{\childdocname}
  \def\jobname{#1}
  \def\childdocjob{#1}
  \input{#1}
}
%    \end{macrocode}

% \macro{\childdocby}
% The command |\childdocby| ....
%    \begin{macrocode}
\newcommand{\childdocby}[2][]
{
  \childdocdisable
  \childdoctrue
  \childdocmanualtrue
  \if?#1?\else
    \def\jobname{#2}
  \fi
  \def\childdocjob{#2}
  \input{#2}
  \endinput
}
%    \end{macrocode}

% \macro{\childdocforward}
% The command |\childdocforward| redirects
% compilation to the main file or
% (if the optional argument is given) a child file.
% Parameters are set as if the main file
% or a child file starting with |\childdocof| was compiled.
% Then compilation is handed over to the main file:
%    \begin{macrocode}
\newcommand{\childdocforward}[2][]
{
  \begingroup
    \if?#1?
      \def\childdoctmp
      {
        \def\childdocname{#2}
        \def\childdocjob{#2}
        \def\jobname{#2}
        \input{#2}
        \endinput
      }
    \else
      \def\childdoctmp
      {
        \childdocdisable
        \def\childdocname{#2}
        \childdoctrue
        \includeonly{#2}
        \def\childdocjob{#1}
        \def\jobname{#1}
        \input{#1}
        \endinput
      }
    \fi
    \expandafter
  \endgroup
  \childdoctmp
}
%    \end{macrocode}

% \macro{\childdocforwardprefix}
% The command |\childdocforwardprefix| redirects
% compilation to the main or a child file by means of a pattern.
% The prefix |#1| in the current filename is replaced by |#2|
% and the suffix of the current filename is kept
% (it is assumed that the filename does not contain the substring `|~~~|'
% which is used as a delimiter).
% Compilation is handed over to the new file by |\childdocforward|:
%    \begin{macrocode}
\newcommand{\childdocforwardprefix}[3][]
{
  \begingroup
    \def\childdocextract #2##1~~~{\def\childdoctmp{\childdocforward[#1]{#3##1}}}
    \expandafter\childdocextract\childdocname~~~
    \expandafter
  \endgroup
  \childdoctmp
}
%    \end{macrocode}

% \macro{\childdoc}
% The deprecated macro |\childdoc| is a legacy version of |\childdocmain|:
%    \begin{macrocode}
\newcommand{\childdoc}{\childdocmain}
%    \end{macrocode}

% \macro{\childdocredirect}
% The deprecated macro |\childdocredirect| is a legacy version
% of |\childdocforward| and |\childdocforwardprefix|:
%    \begin{macrocode}
\newcommand{\childdocredirect}[2][]
{
  \begingroup
    \if?#1?
      \def\childdoctmp{\childdocforward{#2}}
    \else
      \def\childdoctmp{\childdocforwardprefix{#1}{#2}}
    \fi
    \expandafter
  \endgroup
  \childdoctmp
}
%    \end{macrocode}

%\iffalse
%</package>
%\fi
%
\endinput
\childdocforward{cdocsch2}"|
% \end{tabular}
% \end{center}
% Note that the trailing backslash on each first line
% merely continues the input to the second line
% (for convenient cut ant paste).
% Furthermore, the command |latex| can be replaced by any
% of its alternative versions such as |pdflatex|.
%
% %%%%%%%%%%%%%%%%%%%%%%%%%%%%%%%%%%%%%%%%%%%%%%%%%%%%%%%%%%%%%%%%%%%%%%%%%%%%%%
% %%%%%%%%%%%%%%%%%%%%%%%%%%%%%%%%%%%%%%%%%%%%%%%%%%%%%%%%%%%%%%%%%%%%%%%%%%%%%%
% \section{Implementation}
%\iffalse
%<*package>
%\fi
%
% This section describes the definitions file |childdoc.def|.

% The definitions cannot be loaded using |\usepackage| or |\RequirePackage|
% which has a mechanism to prevent loading a style file more than once.
% When loading the definitions by means of |\input|
% multiple instances have to be prevented manually:
%\iffalse
%This code needs to be before the `\ProvidesFile' directive
%which is defined at the beginning of this file.
%Therefore it is also placed there and commented out here.
%</package>
%<*discard>
%\fi
%    \begin{macrocode}
\ifdefined\childdocmain\endinput\fi
%    \end{macrocode}
%\iffalse
%</discard>
%<*package>
%\fi
%
% \macro{\ifchilddoc}
% \macro{\ifchilddocmanual}
% The conditional |\ifchilddoc| tells whether a
% child (true) or main (false) document is being compiled.
% The conditional |\ifchilddocmanual| tells whether
% the |\includeonly| mechanism is used (false) or
% the selection of child files must be performed manually (true).
% The definitions initialise to false:
%    \begin{macrocode}
\newif\ifchilddoc
\newif\ifchilddocmanual
%    \end{macrocode}

% \macro{\childdocname}
% \macro{\childdocjob}
% The macro |\childdocname| stores the name of the main document
% to be compiled. The macro |\childdocjob| stores the name of
% the document on which the \LaTeX{} compiler was originally invoked.
% The content of |\jobname| cannot be compared
% to filenames specified in the source due to different catcodes.
% The following code rescans |\jobname|, stores the result
% in |\childdocname| and saves a copy in |\childdocjob|:
%    \begin{macrocode}
\edef\childdocname{\scantokens\expandafter{\jobname\noexpand}}
\let\childdocjob\childdocname
%    \end{macrocode}

% \macro{\childdocdisable}
% The macro |\childdocdisable| prevents the main file
% from being processed more than once.
% At this stage, the main document command |\childdocmain|
% is assumed to be called once again where it should do nothing.
% Any subsequent call to it should prevent
% a secondary processing of the main document
% It overwrites the forwarding commands
% |\childdocof| and |\childdocforward|
% with empty macros to prevent further inclusions of the main document:
%    \begin{macrocode}
\newcommand{\childdocdisable}
{
  \renewcommand{\childdocmain}[1]{\renewcommand{\childdocmain}[1]{\endinput}}
  \renewcommand{\childdocof}[1]{}
  \renewcommand{\childdocby}[2][]{}
  \renewcommand{\childdocforward}[2][]{}
  \renewcommand{\childdocdisable}{}
}
%    \end{macrocode}

% \macro{\childdocmain}
% The macro |\childdocmain| is to be called at the top of the main file
% with nothing or the main filename (without extension) as argument.
% First, it breaks loops.
% If the argument is not empty and does not match |\childdocname|
% (which is set by the first inclusion of |childdoc.def|),
% |\ifchilddoc| is set to true, |\includeonly| is applied to the child file
% and |\jobname| is set to the main file
% (for proper handling of |.aux| files):
%    \begin{macrocode}
\newcommand{\childdocmain}[1]
{
  \childdocdisable\childdocmain{}
  \if?#1?\else
    \begingroup
      \def\childdoctmp{#1}
      \ifx\childdoctmp\childdocname
        \def\childdoctmp{}
      \else
        \def\childdoctmp
        {
          \childdoctrue
          \includeonly{\childdocname}
          \def\childdocjob{#1}
          \def\jobname{#1}
        }
      \fi
      \expandafter
    \endgroup
    \childdoctmp
  \fi
}
%    \end{macrocode}

% \macro{\childdocof}
% The command |\childdocof| redirects
% compilation to the main file |#1|.
%    \begin{macrocode}
\newcommand{\childdocof}[1]
{
  \childdocdisable
  \childdoctrue
  \includeonly{\childdocname}
  \def\jobname{#1}
  \def\childdocjob{#1}
  \input{#1}
}
%    \end{macrocode}

% \macro{\childdocby}
% The command |\childdocby| ....
%    \begin{macrocode}
\newcommand{\childdocby}[2][]
{
  \childdocdisable
  \childdoctrue
  \childdocmanualtrue
  \if?#1?\else
    \def\jobname{#2}
  \fi
  \def\childdocjob{#2}
  \input{#2}
  \endinput
}
%    \end{macrocode}

% \macro{\childdocforward}
% The command |\childdocforward| redirects
% compilation to the main file or
% (if the optional argument is given) a child file.
% Parameters are set as if the main file
% or a child file starting with |\childdocof| was compiled.
% Then compilation is handed over to the main file:
%    \begin{macrocode}
\newcommand{\childdocforward}[2][]
{
  \begingroup
    \if?#1?
      \def\childdoctmp
      {
        \def\childdocname{#2}
        \def\childdocjob{#2}
        \def\jobname{#2}
        \input{#2}
        \endinput
      }
    \else
      \def\childdoctmp
      {
        \childdocdisable
        \def\childdocname{#2}
        \childdoctrue
        \includeonly{#2}
        \def\childdocjob{#1}
        \def\jobname{#1}
        \input{#1}
        \endinput
      }
    \fi
    \expandafter
  \endgroup
  \childdoctmp
}
%    \end{macrocode}

% \macro{\childdocforwardprefix}
% The command |\childdocforwardprefix| redirects
% compilation to the main or a child file by means of a pattern.
% The prefix |#1| in the current filename is replaced by |#2|
% and the suffix of the current filename is kept
% (it is assumed that the filename does not contain the substring `|~~~|'
% which is used as a delimiter).
% Compilation is handed over to the new file by |\childdocforward|:
%    \begin{macrocode}
\newcommand{\childdocforwardprefix}[3][]
{
  \begingroup
    \def\childdocextract #2##1~~~{\def\childdoctmp{\childdocforward[#1]{#3##1}}}
    \expandafter\childdocextract\childdocname~~~
    \expandafter
  \endgroup
  \childdoctmp
}
%    \end{macrocode}

% \macro{\childdoc}
% The deprecated macro |\childdoc| is a legacy version of |\childdocmain|:
%    \begin{macrocode}
\newcommand{\childdoc}{\childdocmain}
%    \end{macrocode}

% \macro{\childdocredirect}
% The deprecated macro |\childdocredirect| is a legacy version
% of |\childdocforward| and |\childdocforwardprefix|:
%    \begin{macrocode}
\newcommand{\childdocredirect}[2][]
{
  \begingroup
    \if?#1?
      \def\childdoctmp{\childdocforward{#2}}
    \else
      \def\childdoctmp{\childdocforwardprefix{#1}{#2}}
    \fi
    \expandafter
  \endgroup
  \childdoctmp
}
%    \end{macrocode}

%\iffalse
%</package>
%\fi
%
\endinput
\childdocforward{cdocsamp}"|\\
% |latex -jobname cdocscl1 \|\\
% |  "% \iffalse
%
% childdoc.dtx Copyright (C) 2017-2018 Niklas Beisert
%
% This work may be distributed and/or modified under the
% conditions of the LaTeX Project Public License, either version 1.3
% of this license or (at your option) any later version.
% The latest version of this license is in
%   http://www.latex-project.org/lppl.txt
% and version 1.3 or later is part of all distributions of LaTeX
% version 2005/12/01 or later.
%
% This work has the LPPL maintenance status `maintained'.
%
% The Current Maintainer of this work is Niklas Beisert.
%
% This work consists of the files childdoc.dtx and childdoc.ins
% and the derived files childdoc.def and cdocsamp.tex with
% cdocsch1.tex, cdocsch2.tex, cdocsdrf.tex, cdocsfn1.tex, cdocsfn2.tex.
%
%<package>\ifdefined\childdocmain\endinput\fi
%<package>\ProvidesFile{childdoc.def}[2018/12/30 v2.0 child document driver]
%<samplemain>\ProvidesFile{cdocsamp.tex}[2018/12/30 v2.0 sample for childdoc]
%<*driver>
%\ProvidesFile{childdoc.drv}[2018/12/30 v2.0 childdoc reference manual file]
\PassOptionsToClass{10pt,a4paper}{article}
\documentclass{ltxdoc}

\usepackage[margin=35mm]{geometry}
\usepackage{hyperref}
\usepackage{hyperxmp}
\usepackage[usenames]{color}

\hypersetup{colorlinks=true}
\hypersetup{pdfstartview=FitH}
\hypersetup{pdfpagemode=UseNone}
\hypersetup{pdfsource={}}
\hypersetup{pdflang={en-UK}}
\hypersetup{pdfcopyright={Copyright 2017-2018 Niklas Beisert.
  This work may be distributed and/or modified under the
  conditions of the LaTeX Project Public License, either version 1.3
  of this license or (at your option) any later version.}}
\hypersetup{pdflicenseurl={http://www.latex-project.org/lppl.txt}}
\hypersetup{pdfcontactaddress={ETH Zurich, ITP, HIT K,
  Wolfgang-Pauli-Strasse 27}}
\hypersetup{pdfcontactpostcode={8093}}
\hypersetup{pdfcontactcity={Zurich}}
\hypersetup{pdfcontactcountry={Switzerland}}
\hypersetup{pdfcontactemail={nbeisert@itp.phys.ethz.ch}}
\hypersetup{pdfcontacturl={http://people.phys.ethz.ch/\xmptilde nbeisert/}}

\newcommand{\secref}[1]{\hyperref[#1]{section \ref*{#1}}}

\parskip1ex
\parindent0pt
\let\olditemize\itemize
\def\itemize{\olditemize\parskip0pt}

\begin{document}

\title{The \textsf{childdoc} Package}
\hypersetup{pdftitle={The childdoc Package}}
\author{Niklas Beisert\\[2ex]
  Institut f\"ur Theoretische Physik\\
  Eidgen\"ossische Technische Hochschule Z\"urich\\
  Wolfgang-Pauli-Strasse 27, 8093 Z\"urich, Switzerland\\[1ex]
  \href{mailto:nbeisert@itp.phys.ethz.ch}
  {\texttt{nbeisert@itp.phys.ethz.ch}}}
\hypersetup{pdfauthor={Niklas Beisert}}
\hypersetup{pdfsubject={Manual for the LaTeX2e Package childdoc}}
\date{30 December 2018, \textsf{v2.0}}
\maketitle

\begin{abstract}\noindent
\textsf{childdoc} is a \LaTeXe{} package
that enables the direct compilation
of document sections included by |\include|
to individual files.
\end{abstract}

\begingroup
\parskip0ex
\tableofcontents
\endgroup

%%%%%%%%%%%%%%%%%%%%%%%%%%%%%%%%%%%%%%%%%%%%%%%%%%%%%%%%%%%%%%%%%%%%%%%%%%%%%%%%
%%%%%%%%%%%%%%%%%%%%%%%%%%%%%%%%%%%%%%%%%%%%%%%%%%%%%%%%%%%%%%%%%%%%%%%%%%%%%%%%
\section{Introduction}

\LaTeX{} provides a mechanism to structure a large document (such as a book)
into a main file and several child files (containing the chapters)
using the |\include| command.
This mechanism is beneficial for documents
which span hundreds of pages in order to
make the source file(s) more manageable.
Moreover, compilation can be restricted to
selected child files by means of the |\includeonly| command.
The latter feature can be used to reduce the compilation time while editing
(this was significantly more useful in the earlier days of \LaTeX{})
or to generate a smaller document which is easier to navigate.
Another application of |\includeonly| is to generate
documents consisting of selected parts of the complete document.

However, there are a few drawbacks of the plain |\include| mechanism:
\begin{itemize}
\item
The child files cannot be compiled on their own,
they can only be compiled via the main file.
A naive editing environment
(such as a text editor with an option
to have the current file processed by \LaTeX)
may require one to switch to the main file before compiling;
attempting to compile the child file produces errors.
\item
The main file must be modified (each time)
to adjust the |\includeonly| command
to the present needs. This easily leaves the main file in a messy state.
\item
The generated document will always carry the filename
of the main document. This is inconvenient if
several child files are to be compiled and
to be kept for distribution.
\end{itemize}

The present package provides a simple interface
to make child files individually compilable by \LaTeX{}.
Compiling a child file then has the same effect as compiling
the main file with an |\includeonly| command
to select the appropriate child.
Moreover the generated document will carry the name of the child
rather than the main file.
This resolves all three above issues.

This feature is meant to make the editing of books,
thesis documents and lecture notes somewhat more convenient.
However, the package can also be used efficiently for
composing a series of documents (such as exercise sheets)
which are typically distributed individually.
It then assists the author in generating the individual documents
(potentially in different versions)
as well as a document containing the collected series.
Another application is in developing style files
or other kinds of included material
where compilation of the style file could redirect
to a sample or test file.

%%%%%%%%%%%%%%%%%%%%%%%%%%%%%%%%%%%%%%%%%%%%%%%%%%%%%%%%%%%%%%%%%%%%%%%%%%%%%%%%
%%%%%%%%%%%%%%%%%%%%%%%%%%%%%%%%%%%%%%%%%%%%%%%%%%%%%%%%%%%%%%%%%%%%%%%%%%%%%%%%
\section{Usage}

First of all, the package \textsf{childdoc} is \emph{not} a standard
\LaTeXe{} |.sty| style file! Therefore it needs to be invoked in
a non-standard way.

%%%%%%%%%%%%%%%%%%%%%%%%%%%%%%%%%%%%%%%%%%%%%%%%%%%%%%%%%%%%%%%%%%%%%%%%%%%%%%%%
\subsection{Included Files}
\label{sec:include}

%%%%%%%%%%%%%%%%%%%%%%%%%%%%%%%%%%%%%%%%
\DescribeMacro{\childdocmain}
To use the package, add the commands
\begin{center}
\begin{tabular}{l}
|% \iffalse
%
% childdoc.dtx Copyright (C) 2017-2018 Niklas Beisert
%
% This work may be distributed and/or modified under the
% conditions of the LaTeX Project Public License, either version 1.3
% of this license or (at your option) any later version.
% The latest version of this license is in
%   http://www.latex-project.org/lppl.txt
% and version 1.3 or later is part of all distributions of LaTeX
% version 2005/12/01 or later.
%
% This work has the LPPL maintenance status `maintained'.
%
% The Current Maintainer of this work is Niklas Beisert.
%
% This work consists of the files childdoc.dtx and childdoc.ins
% and the derived files childdoc.def and cdocsamp.tex with
% cdocsch1.tex, cdocsch2.tex, cdocsdrf.tex, cdocsfn1.tex, cdocsfn2.tex.
%
%<package>\ifdefined\childdocmain\endinput\fi
%<package>\ProvidesFile{childdoc.def}[2018/12/30 v2.0 child document driver]
%<samplemain>\ProvidesFile{cdocsamp.tex}[2018/12/30 v2.0 sample for childdoc]
%<*driver>
%\ProvidesFile{childdoc.drv}[2018/12/30 v2.0 childdoc reference manual file]
\PassOptionsToClass{10pt,a4paper}{article}
\documentclass{ltxdoc}

\usepackage[margin=35mm]{geometry}
\usepackage{hyperref}
\usepackage{hyperxmp}
\usepackage[usenames]{color}

\hypersetup{colorlinks=true}
\hypersetup{pdfstartview=FitH}
\hypersetup{pdfpagemode=UseNone}
\hypersetup{pdfsource={}}
\hypersetup{pdflang={en-UK}}
\hypersetup{pdfcopyright={Copyright 2017-2018 Niklas Beisert.
  This work may be distributed and/or modified under the
  conditions of the LaTeX Project Public License, either version 1.3
  of this license or (at your option) any later version.}}
\hypersetup{pdflicenseurl={http://www.latex-project.org/lppl.txt}}
\hypersetup{pdfcontactaddress={ETH Zurich, ITP, HIT K,
  Wolfgang-Pauli-Strasse 27}}
\hypersetup{pdfcontactpostcode={8093}}
\hypersetup{pdfcontactcity={Zurich}}
\hypersetup{pdfcontactcountry={Switzerland}}
\hypersetup{pdfcontactemail={nbeisert@itp.phys.ethz.ch}}
\hypersetup{pdfcontacturl={http://people.phys.ethz.ch/\xmptilde nbeisert/}}

\newcommand{\secref}[1]{\hyperref[#1]{section \ref*{#1}}}

\parskip1ex
\parindent0pt
\let\olditemize\itemize
\def\itemize{\olditemize\parskip0pt}

\begin{document}

\title{The \textsf{childdoc} Package}
\hypersetup{pdftitle={The childdoc Package}}
\author{Niklas Beisert\\[2ex]
  Institut f\"ur Theoretische Physik\\
  Eidgen\"ossische Technische Hochschule Z\"urich\\
  Wolfgang-Pauli-Strasse 27, 8093 Z\"urich, Switzerland\\[1ex]
  \href{mailto:nbeisert@itp.phys.ethz.ch}
  {\texttt{nbeisert@itp.phys.ethz.ch}}}
\hypersetup{pdfauthor={Niklas Beisert}}
\hypersetup{pdfsubject={Manual for the LaTeX2e Package childdoc}}
\date{30 December 2018, \textsf{v2.0}}
\maketitle

\begin{abstract}\noindent
\textsf{childdoc} is a \LaTeXe{} package
that enables the direct compilation
of document sections included by |\include|
to individual files.
\end{abstract}

\begingroup
\parskip0ex
\tableofcontents
\endgroup

%%%%%%%%%%%%%%%%%%%%%%%%%%%%%%%%%%%%%%%%%%%%%%%%%%%%%%%%%%%%%%%%%%%%%%%%%%%%%%%%
%%%%%%%%%%%%%%%%%%%%%%%%%%%%%%%%%%%%%%%%%%%%%%%%%%%%%%%%%%%%%%%%%%%%%%%%%%%%%%%%
\section{Introduction}

\LaTeX{} provides a mechanism to structure a large document (such as a book)
into a main file and several child files (containing the chapters)
using the |\include| command.
This mechanism is beneficial for documents
which span hundreds of pages in order to
make the source file(s) more manageable.
Moreover, compilation can be restricted to
selected child files by means of the |\includeonly| command.
The latter feature can be used to reduce the compilation time while editing
(this was significantly more useful in the earlier days of \LaTeX{})
or to generate a smaller document which is easier to navigate.
Another application of |\includeonly| is to generate
documents consisting of selected parts of the complete document.

However, there are a few drawbacks of the plain |\include| mechanism:
\begin{itemize}
\item
The child files cannot be compiled on their own,
they can only be compiled via the main file.
A naive editing environment
(such as a text editor with an option
to have the current file processed by \LaTeX)
may require one to switch to the main file before compiling;
attempting to compile the child file produces errors.
\item
The main file must be modified (each time)
to adjust the |\includeonly| command
to the present needs. This easily leaves the main file in a messy state.
\item
The generated document will always carry the filename
of the main document. This is inconvenient if
several child files are to be compiled and
to be kept for distribution.
\end{itemize}

The present package provides a simple interface
to make child files individually compilable by \LaTeX{}.
Compiling a child file then has the same effect as compiling
the main file with an |\includeonly| command
to select the appropriate child.
Moreover the generated document will carry the name of the child
rather than the main file.
This resolves all three above issues.

This feature is meant to make the editing of books,
thesis documents and lecture notes somewhat more convenient.
However, the package can also be used efficiently for
composing a series of documents (such as exercise sheets)
which are typically distributed individually.
It then assists the author in generating the individual documents
(potentially in different versions)
as well as a document containing the collected series.
Another application is in developing style files
or other kinds of included material
where compilation of the style file could redirect
to a sample or test file.

%%%%%%%%%%%%%%%%%%%%%%%%%%%%%%%%%%%%%%%%%%%%%%%%%%%%%%%%%%%%%%%%%%%%%%%%%%%%%%%%
%%%%%%%%%%%%%%%%%%%%%%%%%%%%%%%%%%%%%%%%%%%%%%%%%%%%%%%%%%%%%%%%%%%%%%%%%%%%%%%%
\section{Usage}

First of all, the package \textsf{childdoc} is \emph{not} a standard
\LaTeXe{} |.sty| style file! Therefore it needs to be invoked in
a non-standard way.

%%%%%%%%%%%%%%%%%%%%%%%%%%%%%%%%%%%%%%%%%%%%%%%%%%%%%%%%%%%%%%%%%%%%%%%%%%%%%%%%
\subsection{Included Files}
\label{sec:include}

%%%%%%%%%%%%%%%%%%%%%%%%%%%%%%%%%%%%%%%%
\DescribeMacro{\childdocmain}
To use the package, add the commands
\begin{center}
\begin{tabular}{l}
|\input{childdoc.def}|\\
|\childdocmain{}|\\
\end{tabular}
\end{center}
at the very top of the main \LaTeX{} file,
in particular \emph{before} the |\documentclass| statement!
The argument of |\childdocmain| should be left empty
(but it must be present).

%%%%%%%%%%%%%%%%%%%%%%%%%%%%%%%%%%%%%%%%
\DescribeMacro{\childdocof}
Furthermore, add the commands
\begin{center}
\begin{tabular}{l}
|\input{childdoc.def}|\\
|\childdocof{|\textit{main}|}|\\
\end{tabular}
\end{center}
at the top of every child file \textit{child}
which is included by |\include{|\textit{child}|}|
from within the main file
(or at least for those files to be compiled individually).
The argument \textit{main} must be the filename of the main file.

There are a couple of
considerations in setting up the main and child documents:

%%%%%%%%%%%%%%%%%%%%%%%%%%%%%%%%%%%%%%%%
\paragraph{Restrictions.}

Please note the following restrictions:
\begin{itemize}
\item
|\childdocmain| must be called with one argument \textit{main}
to ensure compatibility with earlier version of the package.
It must either be empty (|\childdocmain{}|)
or precisely match the filename of the main file in which it is specified.
See \secref{sec:detection} for further information.
\item
The filename \textit{main} must be specified without the |.tex| extension.
\item
The filename \textit{main} is case sensitive
(even in case-insensitive file systems)
due to internal string comparison.
\item
The argument \textit{main} should be fully expanded, it cannot be a macro.
\item
Subdirectories and special characters should be avoided in filenames.
\item
The command |\childdocmain{|\textit{main}|}| must be followed by a whitespace.
It should not be followed immediately by another command
or by a comment mark `|%|'.
This is because the \TeX{} parser reads the token immediately following
the argument of |\childdocmain| and puts it
at the beginning of every child section;
however, a white\-space is ignored.
\end{itemize}

%%%%%%%%%%%%%%%%%%%%%%%%%%%%%%%%%%%%%%%%
\paragraph{Content of Main File.}

It is advisable to place all content in the child files included by |\include|.
Any output contained in the main file will appear in all child documents
unless suppressed manually;
it cannot be suppressed automatically by the |\includeonly| directive
and thus should normally be avoided.
A method to include some content in the main file
by means of conditional processing is described in \secref{sec:conditional}.

%%%%%%%%%%%%%%%%%%%%%%%%%%%%%%%%%%%%%%%%
\paragraph{Page Numbering.}

When only a part of the document is compiled,
the appropriate numbering of pages
(as well as other status parameters)
is determined from the |.aux| files.
The latter contain information from previous passes.
However this information needs to propagate through
all intermediate child documents.
Therefore the page numbering in child documents may well
be inconsistent until the complete document is compiled at least once.

A useful (if unconventional) way to always ensure a consistent
page numbering is to restart the numbering in each child document
and denote the pages by `\textit{child}|.|\textit{page}'
where \textit{child} represents the chapter/section number of the child file.
This can be achieved by the command
|\numberwithin{page}{|\textit{child}|}|
of the \textsf{amsmath} package
where \textit{child} can be |chapter| or |section|
depending on the chosen structuring.
Alternatively, one can modify the macro |\thepage| appropriately
and reset the counter |page| at the start of each child file.

%%%%%%%%%%%%%%%%%%%%%%%%%%%%%%%%%%%%%%%%%%%%%%%%%%%%%%%%%%%%%%%%%%%%%%%%%%%%%%%%
\subsection{Conditional Processing}
\label{sec:conditional}

The package provides a mechanism to compile different versions
of a document. To customise the versions further some conditional processing
can come in handy to distinguish which version is being compiled.
The package provides two macros to describe the compilation context:

%%%%%%%%%%%%%%%%%%%%%%%%%%%%%%%%%%%%%%%%
\DescribeMacro{\ifchilddoc}
The conditional |\ifchilddoc| distinguishes between the compilation of
child documents and the main document:
%
\begin{center}
|\ifchilddoc |\textit{child-code}| |[|\||else |\textit{main-code}]| \||fi|
\end{center}

%%%%%%%%%%%%%%%%%%%%%%%%%%%%%%%%%%%%%%%%
\DescribeMacro{\childdocname}
\DescribeMacro{\childdocjob}
The macro |\childdocname| contains the filename (without extension)
of the main or child file being processed.
Note that |\childdocjob| will always contain the name of the main file.

%%%%%%%%%%%%%%%%%%%%%%%%%%%%%%%%%%%%%%%%
\paragraph{Title Page.}

Conditional processing can be used to include a title or banner page
in the main document when proper precautions are taken.
Importantly, the code in the main file should ensure that the page counter
(as well as other status parameters which are stored in the |.aux| files)
takes the same value after the conditional processing.
Otherwise the page numbers may take divergent values
depending on which part is compiled.

For example, a title page could be declared by:
%
\begin{center}
\begin{tabular}{l}
|\ifchilddoc\||else|\\
|\addtocounter{page}{-1}|\\
\textit{code for title page}\\
|\newpage|\\
|\||fi|
\end{tabular}
\end{center}
%
A banner page for the child documents can be generated by:
%
\begin{center}
\begin{tabular}{l}
|\ifchilddoc|\\
|\addtocounter{page}{-1}|\\
\textit{code for banner page}\\
|\newpage|\\
|\||fi|
\end{tabular}
\end{center}
%
Here one could write a message such as:
\begin{center}
|This is the part \childdocname{} of \childdocjob{}.|
\end{center}

%%%%%%%%%%%%%%%%%%%%%%%%%%%%%%%%%%%%%%%%%%%%%%%%%%%%%%%%%%%%%%%%%%%%%%%%%%%%%%%%
\subsection{Flags}
\label{sec:flags}

The package makes it easy to generate different versions
of the main or child documents.
To this end compilation flags can be defined
and assigned different default values.
They will be particularly useful in conjunction
with the forwarding mechanism described in \secref{sec:forward}.

For example, it may be useful to have a flag |\version|
which can be set to |draft| or |final|.
The document source will contain some conditional code
depending on the value of |\version|.
Suppose further, the flag should default to |final| for the main file
and to |draft| for child files
which is a natural assignment for editing the document.
This is achieved by placing the following code
in the preamble of the main document
(below the |\childdocmain| directive):
%
\begin{center}
\begin{tabular}{l}
|\ifchilddoc|\\
|\providecommand{\version}{draft}|\\
|\||else|\\
|\providecommand{\version}{final}|\\
|\||fi|
\end{tabular}
\end{center}
%
The definition by |\providecommand| makes sure
that previous definitions are not overwritten.
Further statements |\providecommand{\version}{...}|
can thus be added before the above code to override it.

For the main file, one might add a line
(between |\childdocmain| and the above block)
%
\begin{center}
|%\ifchilddoc\||else\providecommand{\version}{draft}\||fi|
\end{center}
%
which can be uncommented to produce a draft version.
Likewise one can add a line to the very top of a child file
(above the |\childdocof{|\textit{main}|}| directive)
%
\begin{center}
|%\providecommand{\version}{final}|
\end{center}
%
which can be uncommented to produce the final version of this child document.

%%%%%%%%%%%%%%%%%%%%%%%%%%%%%%%%%%%%%%%%%%%%%%%%%%%%%%%%%%%%%%%%%%%%%%%%%%%%%%%%
\subsection{Forwarding}
\label{sec:forward}

Different versions of the main or child documents
using compilation flags as described in \secref{sec:flags}
can be (permanently) stored in different files
for convenient compilation, viewing and distribution.
To this end, the package defines a command
to pass on compilation to a different file:

%%%%%%%%%%%%%%%%%%%%%%%%%%%%%%%%%%%%%%%%
\DescribeMacro{\childdocforward}
The command |\childdocforward| redirects processing to
another source file:
%
\begin{center}
\begin{tabular}{l}
|\input{childdoc.def}|\\
|\childdocforward[|\textit{main}|]{|\textit{dest}|}|\\
\end{tabular}
\end{center}
%
The argument \textit{dest} is the destination file
(without extension).
It should be the main file or one of the child files.
Note that further \textsf{childdoc} directives
such as |\childdocof| and |\childdocforward|
in the indicated file will be processed in this form.
The optional argument \textit{main}
passes on directly to the main file \textit{main}
while pretending to compile the child \textit{dest}.
This form behaves as if \textit{dest}
issues |\childdocof{|\textit{main}|}| right away,
and no further \textsf{childdoc} directives will be processed.

%%%%%%%%%%%%%%%%%%%%%%%%%%%%%%%%%%%%%%%%
\DescribeMacro{\...prefix}
In the alternative form |\childdocforwardprefix|,
%
\begin{center}
\begin{tabular}{l}
|\input{childdoc.def}|\\
|\childdocforwardprefix[|\textit{main}|]{|\textit{prefix}|}{|\textit{dest}|}|
\end{tabular}
\end{center}
%
the destination file is determined by a pattern
depending on the current file:
To make this work, the current file must be called
`{\textit{prefix}\hspace{0.2em}\textit{suffix}}'
with \textit{prefix} matching precisely the argument.
Processing is then passed on to the file
`{\textit{dest}\hspace{0.2em}\textit{suffix}}'.
Surely, the same effect is achieved by
directly specifying the
argument `{\textit{dest}\hspace{0.2em}\textit{suffix}}'
in the first form.
However, that requires to set up a different file
for each child. With the alternative form of the command
all these files can have exactly the same content
which simplifies setting them up and maintaining them.

For example, the following file |draft.tex|
with a compilation flag |\version| as described in \secref{sec:flags}
compiles the main document as a draft:
%
\begin{center}
\begin{tabular}{l}
|\def\version{draft}|\\
|\input{childdoc.def}|\\
|\childdocforward{|\textit{main}|}|
\end{tabular}
\end{center}
%
Likewise, the following files |final|\textit{nn}|.tex|
compile the final version of the child document
|child|\textit{nn}|.tex|:
%
\begin{center}
\begin{tabular}{l}
|\def\version{final}|\\
|\input{childdoc.def}|\\
|\childdocforwardprefix{final}{child}|
\end{tabular}
\end{center}
%

Note that when several versions of a main file and/or of each child file
are to be generated, it may be convenient to set up a |Makefile| or
shell script to automatise the process.

%%%%%%%%%%%%%%%%%%%%%%%%%%%%%%%%%%%%%%%%%%%%%%%%%%%%%%%%%%%%%%%%%%%%%%%%%%%%%%%%
\subsection{Command Line Processing}
\label{sec:commandline}

The effect of redirection files can also be achieved by invoking
the \LaTeX{} compiler with a more elaborate command line.
Most conveniently this should be done as part
of a shell script or a |Makefile|.

When using \textsf{childdoc} in the main file, the following
command lines effectively perform a redirection
(note that depending on the shell being used,
backslashes may have to be doubled: `|\|' $\to$ `|\\|'):
%
\begin{center}
|... -jobname "|\textit{target}|" |\\|"|[\textit{flags}]%
|\input{childdoc.def}\childdocforward[|\textit{main}|]{|\textit{dest}|}"|
\end{center}
%
Here \textit{target} is the name of the output file,
\textit{main} is the name of the main file
and \textit{dest} is the name of the main or child file to be processed
(all filenames without extensions).
The optional argument \textit{main} can be omitted
if \textit{main} matches \textit{dest}.
Optionally, compilation \textit{flags} can be defined via |\def| commands.
This command line makes the \TeX{} engine believe
it is compiling the file \textit{target}
whose content is specified as the latter parameter.
The provided code then forwards the processing to
\textit{main} or \textit{dest} as described in \secref{sec:forward}.

%%%%%%%%%%%%%%%%%%%%%%%%%%%%%%%%%%%%%%%%%%%%%%%%%%%%%%%%%%%%%%%%%%%%%%%%%%%%%%%%
\subsection{Include by Input}
\label{sec:input}

Including child documents by |\include| has some restrictions by design.
Most notably, the content of a child document always occupies
its own set of pages; pages cannot be shared between child documents.
Usually, this behaviour makes perfect sense
because each child document contain an essential part of the document.
However, in some situations it may be desirable to compose
a document from a collection of parts
without having mandatory page breaks between then.
For this case, the package
provides a mechanism to include parts
by |\input| which can also be processed individually.
However, by construction this mechanism
requires manual handling of the content to be output.

%%%%%%%%%%%%%%%%%%%%%%%%%%%%%%%%%%%%%%%%
\DescribeMacro{\ifchilddocmanual}
The main file should be prepared as usual, see \secref{sec:include}.
However, the document body must make a distinction
between processing of an individual part and of the main document, e.g.:
%
\begin{center}
\begin{tabular}{l}
|\ifchilddocmanual|\\
|\input{\childdocname}|\\
|\||else|\\
\textit{document body with }|\input{|\textit{part}|}|\\
|\||fi|
\end{tabular}
\end{center}
%
The conditional |\ifchilddocmanual| is true whenever
a part to be included by |\input| is being compiled,
and the name of the part is stored in |\childdocname|.

%%%%%%%%%%%%%%%%%%%%%%%%%%%%%%%%%%%%%%%%
\DescribeMacro{\childdocby}
Each part to be included by |\input| should start with:
%
\begin{center}
\begin{tabular}{l}
|\input{childdoc.def}|\\
|\childdocby{|\textit{main}|}|\\
\end{tabular}
\end{center}
%
The directive |\childdocby| is similar to |\childdocof|
described in \secref{sec:include},
but the subsequent selection of content must be done manually.
To that end, both |\ifchilddoc| and |\ifchilddocmanual|
will be true upon processing of a part,
and the name of the part is stored in |\childdocname|.
Note that |\jobname| will be set to the filename of the current part
so that each part receives an individual |.aux| file
that does not interfere with the |.aux| file(s) of the main document.
This behaviour can be altered by the alternative form
|\childdocby[*]{|\textit{main}|}| (with a non-empty optional argument)
which uses the |.aux| file of the main document
by setting |\jobname| to \textit{main}.

%%%%%%%%%%%%%%%%%%%%%%%%%%%%%%%%%%%%%%%%%%%%%%%%%%%%%%%%%%%%%%%%%%%%%%%%%%%%%%%%
\subsection{Driver Development}
\label{sec:driver}

The \textsf{childdoc} mechanism can also be use for the development
of definition files such as \LaTeX{} styles or classes.
This case differs from the above setup with multiple parts
included by |\include| in that no |\includeonly| should be invoked.
This can be achieved by starting the include file
(before |\ProvidesPackage|) with:
%
\begin{center}
\begin{tabular}{l}
|\input{childdoc.def}|\\
|\childdocforward{|\textit{main}|}|\\
\end{tabular}
\end{center}
%
or alternatively with:
%
\begin{center}
\begin{tabular}{l}
|\input{childdoc.def}|\\
|\childdocby{|\textit{main}|}|\\
\end{tabular}
\end{center}
%
Both forms have slightly different effects as described above.
The main file is prepared as usual, see \secref{sec:include}.

%%%%%%%%%%%%%%%%%%%%%%%%%%%%%%%%%%%%%%%%%%%%%%%%%%%%%%%%%%%%%%%%%%%%%%%%%%%%%%%%
\subsection{Legacy Detection}
\label{sec:detection}

The directive |\childdocmain| in the main file can detect
whether the complete document or merely a child is to be compiled
even without using the directive |\childdocof|.
This method is deprecated because it is less robust
and there is no compelling reason to use it;
it is merely provided for backward compatibility
and it may be removed in future versions.

If the detection mechanism is to be used,
it is mandatory to correctly specify
the filename of the main file as the argument of |\childdocmain|:
%
\begin{center}
\begin{tabular}{l}
|\input{childdoc.def}|\\
|\childdocmain{|\textit{main}|}|\\
\end{tabular}
\end{center}
%
If |\jobname| does not match the argument \textit{main} of |\childdocmain|,
it is assumed that |\jobname| points to the child file to be compiled.
When using |\childdocmain| with the main file specified as argument,
it suffices to start a child file
with just |\input{|\textit{main}|}|
without loading of the package and using |\childdocof|.
If instead all processing is done
with the appropriate \textsf{childdoc} directives,
the argument of \textit{main} of |\childdocmain| can be empty.

An alternative version of the command line processing described
in \secref{sec:commandline} using the detection mechanism reads:
%
\begin{center}
|... -jobname "|\textit{target}|" "|[\textit{flags}]%
[|\def\jobname{|\textit{dest}|}|]|\input{|\textit{main}|}"|
\end{center}

%%%%%%%%%%%%%%%%%%%%%%%%%%%%%%%%%%%%%%%%%%%%%%%%%%%%%%%%%%%%%%%%%%%%%%%%%%%%%%%%
\subsection{Manual Code}
\label{sec:manual}

In case one cannot be certain whether the definitions file |childdoc.def|
is installed on the target \TeX{} distribution
and one prefers not to ship it,
it is conceivable to paste a few relevant commands into the sources.

To that end, drop all statements |\input{childdoc.def}|
and perform the replacements as outlined below.
Instead of |\childdocmain{|\textit{main}|}| add the following code
to the top of the main file:
%
\begin{center}
\begin{tabular}{l}
|\||ifdefined\childdocname\endinput\||fi\newif\ifchilddoc|\\
|\edef\childdocname{\scantokens\expandafter{\jobname\noexpand}}|\\
|\def\childdocmain{|\textit{main}|}\||ifx\childdocmain\childdocname\||else|\\
|\childdoctrue\includeonly{\childdocname}\let\jobname\childdocmain\||fi|\\
\end{tabular}
\end{center}
%
Instead of |\childdocof{|\textit{main}|}| just include the main file
at the top of each child file:
%
\begin{center}
|\input{|\textit{main}|}|
\end{center}
%
A simple redirection |\childdocforward{|\textit{dest}|}| is achieved by:
%
\begin{center}
|\def\jobname{|\textit{dest}|}\input{\jobname}|
\end{center}
%
The redirection with prefix
|\childdocforwardprefix[|\textit{prefix}|]{|\textit{dest}|}|
is accomplished by:
%
\begin{center}
\begin{tabular}{l}
|{\edef\jobname{\scantokens\expandafter{\jobname\noexpand}}|\\
|\def\redirectjob |\textit{prefix}|#1~~~{\gdef\jobname{|\textit{dest}|#1}}|\\
|\expandafter\redirectjob\jobname~~~}\input{\jobname}|
\end{tabular}
\end{center}

In an alternative approach,
child documents can be compiled by a specific command line
without additional code or specific definitions:
%
\begin{center}
|... -jobname "|\textit{target}|" "|[\textit{flags}]%
|\includeonly{|\textit{dest}|}\input{|\textit{main}|}"|
\end{center}
%

%%%%%%%%%%%%%%%%%%%%%%%%%%%%%%%%%%%%%%%%%%%%%%%%%%%%%%%%%%%%%%%%%%%%%%%%%%%%%%%%
%%%%%%%%%%%%%%%%%%%%%%%%%%%%%%%%%%%%%%%%%%%%%%%%%%%%%%%%%%%%%%%%%%%%%%%%%%%%%%%%
\section{Information}

%%%%%%%%%%%%%%%%%%%%%%%%%%%%%%%%%%%%%%%%%%%%%%%%%%%%%%%%%%%%%%%%%%%%%%%%%%%%%%%%
\subsection{Copyright}

Copyright \copyright{} 2017--2018 Niklas Beisert

This work may be distributed and/or modified under the
conditions of the \LaTeX{} Project Public License, either version 1.3
of this license or (at your option) any later version.
The latest version of this license is in
  \url{http://www.latex-project.org/lppl.txt}
and version 1.3 or later is part of all distributions of \LaTeX{}
version 2005/12/01 or later.

This work has the LPPL maintenance status `maintained'.

The Current Maintainer of this work is Niklas Beisert.

This work consists of the files |README.txt|, |childdoc.ins| and |childdoc.dtx|
as well as the derived files |childdoc.def|, |cdocsamp.tex|
with |cdocsch1.tex|, |cdocsch2.tex|, |cdocspt3.tex|, |cdocspt4.tex|,
|cdocsdrf.tex|, |cdocsfn1.tex|, |cdocsfn2.tex|
as well as |childdoc.pdf|.

%%%%%%%%%%%%%%%%%%%%%%%%%%%%%%%%%%%%%%%%%%%%%%%%%%%%%%%%%%%%%%%%%%%%%%%%%%%%%%%%
\subsection{Files and Installation}

The package consists of the files:
%
\begin{center}
\begin{tabular}{ll}
    |README.txt|   & readme file \\
    |childdoc.ins| & installation file \\
    |childdoc.dtx| & source file \\
    |childdoc.def| & definition file \\
    |cdocsamp.tex| & sample main file \\
    |cdocsch1.tex| & sample include file \\
    |cdocsch2.tex| & sample include file \\
    |cdocspt3.tex| & sample part file \\
    |cdocspt4.tex| & sample part file \\
    |cdocsdrf.tex| & sample redirection file \\
    |cdocsfn1.tex| & sample redirection file \\
    |cdocsfn2.tex| & sample redirection file \\
    |childdoc.pdf| & manual
\end{tabular}
\end{center}
%
The distribution consists of the files
|README.txt|, |childdoc.ins| and |childdoc.dtx|.
%
\begin{itemize}
\item
Run (pdf)\LaTeX{} on |childdoc.dtx|
to compile the manual |childdoc.pdf| (this file).
\item
Run \LaTeX{} on |childdoc.ins| to create the definitions file |childdoc.def|
and the sample |cdocsamp.tex| with include files
|cdocsch1.tex|, |cdocsch2.tex|, |cdocspt3.tex|, |cdocspt4.tex|,
|cdocsdrf.tex|, |cdocsfn1.tex|, |cdocsfn2.tex|.
Then copy the file |childdoc.def| to an appropriate directory of your \LaTeX{}
distribution, e.g.\ \textit{texmf-root}|/tex/latex/childdoc|.
\end{itemize}

%%%%%%%%%%%%%%%%%%%%%%%%%%%%%%%%%%%%%%%%%%%%%%%%%%%%%%%%%%%%%%%%%%%%%%%%%%%%%%%%
\subsection{Related CTAN Packages}

There are several other packages which offer a similar functionality:
%
\begin{itemize}
\item
The packages
\href{http://ctan.org/pkg/docmute}{\textsf{docmute}},
\href{http://ctan.org/pkg/includex}{\textsf{includex}} and
\href{http://ctan.org/pkg/standalone}{\textsf{standalone}}
provide commands to include only the document body of
a child file thus allowing both files to be compiled individually.
\item
The packages \href{http://ctan.org/pkg/subdocs}{\textsf{subdocs}}
and \href{http://ctan.org/pkg/subfiles}{\textsf{subfiles}}
provide structures in which the main and child documents can be
encapsulated and allowing them to be compiled individually.
The inclusion mechanism is different from the conventional |\include|.
\item
The package \href{http://ctan.org/pkg/combine}{\textsf{combine}}
is an elaborate solution to combine several documents into one.
\end{itemize}
%
See also the CTAN topic \href{http://ctan.org/topic/subdocs}{\textsf{subdocs}}
for further related packages.
The present package differs from the above solutions in that
a document structure constructed with the conventional |\include| mechanism
just needs two extra commands at the top of every file
such that all constituent files can be compiled individually.

%%%%%%%%%%%%%%%%%%%%%%%%%%%%%%%%%%%%%%%%%%%%%%%%%%%%%%%%%%%%%%%%%%%%%%%%%%%%%%%%
%\subsection{Feature Suggestions}
%
%The following is a list of features which may be useful for future
%versions of this package:
%%
%\begin{itemize}
%\item
%\ldots
%\end{itemize}

%%%%%%%%%%%%%%%%%%%%%%%%%%%%%%%%%%%%%%%%%%%%%%%%%%%%%%%%%%%%%%%%%%%%%%%%%%%%%%%%
\subsection{Revision History}

%%%%%%%%%%%%%%%%%%%%%%%%%%%%%%%%%%%%%%%%
\paragraph{v2.0:} 2018/12/30

\begin{itemize}
\item
immediate forward processing
\item
added |\childdocby| mechanism
\item
manual restructured
\end{itemize}

%%%%%%%%%%%%%%%%%%%%%%%%%%%%%%%%%%%%%%%%
\paragraph{v1.6:} 2018/01/17

\begin{itemize}
\item
application for development of include files
\item
corrections to manual
\end{itemize}

%%%%%%%%%%%%%%%%%%%%%%%%%%%%%%%%%%%%%%%%
\paragraph{v1.5:} 2017/05/21

\begin{itemize}
\item
more complete structuring introduced
\item
|\childdocof| introduced
\item
|\childdoc| renamed to |\childdocmain|
\item
|\childredirect| renamed to |\childdocforward| and |\childdocforwardprefix|
and functionality expanded
\end{itemize}

%%%%%%%%%%%%%%%%%%%%%%%%%%%%%%%%%%%%%%%%
\paragraph{v1.0:} 2017/04/27

\begin{itemize}
\item
manual and install package
\item
first version published on CTAN
\end{itemize}

%%%%%%%%%%%%%%%%%%%%%%%%%%%%%%%%%%%%%%%%
\paragraph{v0.6:} 2017/04/26

\begin{itemize}
\item
redirection mechanism added
\end{itemize}

%%%%%%%%%%%%%%%%%%%%%%%%%%%%%%%%%%%%%%%%
\paragraph{v0.5:} 2017/04/26

\begin{itemize}
\item
functionality in definition file
\end{itemize}


%%%%%%%%%%%%%%%%%%%%%%%%%%%%%%%%%%%%%%%%%%%%%%%%%%%%%%%%%%%%%%%%%%%%%%%%%%%%%%%%
%%%%%%%%%%%%%%%%%%%%%%%%%%%%%%%%%%%%%%%%%%%%%%%%%%%%%%%%%%%%%%%%%%%%%%%%%%%%%%%%
%%%%%%%%%%%%%%%%%%%%%%%%%%%%%%%%%%%%%%%%%%%%%%%%%%%%%%%%%%%%%%%%%%%%%%%%%%%%%%%%
\appendix

\settowidth\MacroIndent{\rmfamily\scriptsize 000\ }

 \DocInput{childdoc.dtx}

\end{document}
%</driver>
% \fi
%
% %%%%%%%%%%%%%%%%%%%%%%%%%%%%%%%%%%%%%%%%%%%%%%%%%%%%%%%%%%%%%%%%%%%%%%%%%%%%%%
% %%%%%%%%%%%%%%%%%%%%%%%%%%%%%%%%%%%%%%%%%%%%%%%%%%%%%%%%%%%%%%%%%%%%%%%%%%%%%%
% \section{Sample}
%\iffalse
%<*samplemain>
%\fi
%
% The following presents a sample document
% with two chapters, two parts, a title page,
% a compile flag as well as three forwarding files to set the flag.
% It consists of eight |.tex| files:
% \begin{center}
% \begin{tabular}{ll}
% |cdocsamp.tex|&main file\\
% |cdocsch1.tex|&include file for chapter 1\\
% |cdocsch2.tex|&include file for chapter 2\\
% |cdocspt3.tex|&include file for part 3\\
% |cdocspt4.tex|&include file for part 4\\
% |cdocsdrf.tex|&forwarding file for main file in draft mode\\
% |cdocsfi1.tex|&forwarding file for final version of chapter 1\\
% |cdocsfi2.tex|&forwarding file for final version of chapter 2\\
% \end{tabular}
% \end{center}
% Each of the eight files can be compiled directly by the \LaTeX{} compiler.
%
% %%%%%%%%%%%%%%%%%%%%%%%%%%%%%%%%%%%%%%
% \paragraph{Main File.}
%
% The main file is called |cdocsamp.tex|.
%
% Load the \textsf{childdoc} definitions and
% declare the filename for the main document:
%    \begin{macrocode}
\input{childdoc.def}
\childdocmain{}
%    \end{macrocode}

% Optional override for |\version| flag:
%    \begin{macrocode}
%%\ifchilddoc\else\providecommand{\version}{draft}\fi
%    \end{macrocode}

% Define the default values for the |\version| flag
% (|final| for the main file and |draft| for childs):
%    \begin{macrocode}
\ifchilddoc
\providecommand{\version}{draft}
\else
\providecommand{\version}{final}
\fi
%    \end{macrocode}

% Load the standard document class:
%    \begin{macrocode}
\documentclass[12pt]{article}
%    \end{macrocode}

% Start the document body:
%    \begin{macrocode}
\begin{document}
%    \end{macrocode}

% Declare a title page.
% Print title, part of document being processed and version flag:
%    \begin{macrocode}
\addtocounter{page}{-1}
\begin{center}
{\LARGE\bfseries{}childdoc example\par}
\vspace{1cm}
\ifchilddoc
\ifchilddocmanual part\else chapter\fi:
`\childdocname' of `\childdocjob'\par
\else
main document: `\childdocjob'\par
\fi
version: \version\par
\end{center}
\newpage
%    \end{macrocode}

% Manually include selected file,
% otherwise process as usual:
%    \begin{macrocode}
\ifchilddocmanual
\section*{part `\childdocname'}
\input{\childdocname}
\else
%    \end{macrocode}

% Include the two chapters:
%    \begin{macrocode}
\include{cdocsch1}
\include{cdocsch2}
%    \end{macrocode}

% Include the two parts unless only chapters should be displayed:
%    \begin{macrocode}
\ifchilddoc\else
\section{part three}
\input{cdocspt3}
\section{part four}
\input{cdocspt4}
\fi
%    \end{macrocode}

% Process as usual until here:
%    \begin{macrocode}
\fi
%    \end{macrocode}

% End of document body:
%    \begin{macrocode}
\end{document}
%    \end{macrocode}
%\iffalse
%</samplemain>
%\fi
%
% %%%%%%%%%%%%%%%%%%%%%%%%%%%%%%%%%%%%%%
% \paragraph{Chapter Include Files.}
%
% The include files are called |cdocsch1.tex| and |cdocsch2.tex|.
%
%\iffalse
%<*samplechap1|samplechap2>
%\fi

% Optional override for |\version| flag:
%    \begin{macrocode}
%%\providecommand{\version}{final}
%    \end{macrocode}

% Include the main document:
%    \begin{macrocode}
\input{childdoc.def}
\childdocof{cdocsamp}
%    \end{macrocode}

%\iffalse
%</samplechap1|samplechap2>
%\fi
%
%\iffalse
%<*samplechap1>
%\fi
% Some text for chapter 1:
%    \begin{macrocode}
\section{one}
some text in chapter one
%    \end{macrocode}

%\iffalse
%</samplechap1>
%\fi
% Some text for chapter 2:
%\iffalse
%<*samplechap2>
%\fi
%    \begin{macrocode}
\section{two}
more text in chapter two
%    \end{macrocode}

%\iffalse
%</samplechap2>
%\fi
%
% %%%%%%%%%%%%%%%%%%%%%%%%%%%%%%%%%%%%%%
% \paragraph{Part Include Files.}
%
% The include files are called |cdocspt3.tex| and |cdocspt4.tex|.
%
%\iffalse
%<*samplepart3|samplepart4>
%\fi

% Optional override for |\version| flag:
%    \begin{macrocode}
%%\providecommand{\version}{final}
%    \end{macrocode}

% Include the main document:
%    \begin{macrocode}
\input{childdoc.def}
\childdocby{cdocsamp}
%    \end{macrocode}

%\iffalse
%</samplepart3|samplepart4>
%\fi
%
%\iffalse
%<*samplepart3>
%\fi
% Some text for part 3:
%    \begin{macrocode}
some text in part three
%    \end{macrocode}

%\iffalse
%</samplepart3>
%\fi
% Some text for part 4:
%\iffalse
%<*samplepart4>
%\fi
%    \begin{macrocode}
more text in part four
%    \end{macrocode}

%\iffalse
%</samplepart4>
%\fi
%
% %%%%%%%%%%%%%%%%%%%%%%%%%%%%%%%%%%%%%%
% \paragraph{Forwarding for a Complete Draft.}
%
% The following forwarding file |cdocsdrf.tex|
% compiles the main document in draft mode:
%\iffalse
%<*sampledraft>
%\fi
%    \begin{macrocode}
\def\version{draft}
\input{childdoc.def}
\childdocforward{cdocsamp}
%    \end{macrocode}

%\iffalse
%</sampledraft>
%\fi
%
% %%%%%%%%%%%%%%%%%%%%%%%%%%%%%%%%%%%%%%
% \paragraph{Forwarding for Final Version of the Chapters.}
%
% The following forwarding files |cdocsfn1.tex| and |cdocsfn2.tex|
% (with identical content)
% compile the final versions of the child documents
% |cdocsch1.tex| and |cdocsch2.tex|, respectively:
%\iffalse
%<*samplefinal>
%\fi
%    \begin{macrocode}
\def\version{final}
\input{childdoc.def}
\childdocforwardprefix[cdocsamp]{cdocsfn}{cdocsch}
%    \end{macrocode}

%\iffalse
%</samplefinal>
%\fi
%
% %%%%%%%%%%%%%%%%%%%%%%%%%%%%%%%%%%%%%%
% \paragraph{Command Line Processing.}
%
% The following three command lines generate the output files
% |cdocscld|, |cdocscl1| and |cdocscl2|
% which should be identical to
% |cdocsdrf|, |cdocsch1| and |cdocsfn2|, respectively:
% \begin{center}
% \begin{tabular}{l}
% |latex -jobname cdocscld \|\\
% |  "\def\version{draft}\input{childdoc.def}\childdocforward{cdocsamp}"|\\
% |latex -jobname cdocscl1 \|\\
% |  "\input{childdoc.def}\childdocforward[cdocsamp]{cdocsch1}"|\\
% |latex -jobname cdocscl2 \|\\
% |  "\def\version{final}\input{childdoc.def}\childdocforward{cdocsch2}"|
% \end{tabular}
% \end{center}
% Note that the trailing backslash on each first line
% merely continues the input to the second line
% (for convenient cut ant paste).
% Furthermore, the command |latex| can be replaced by any
% of its alternative versions such as |pdflatex|.
%
% %%%%%%%%%%%%%%%%%%%%%%%%%%%%%%%%%%%%%%%%%%%%%%%%%%%%%%%%%%%%%%%%%%%%%%%%%%%%%%
% %%%%%%%%%%%%%%%%%%%%%%%%%%%%%%%%%%%%%%%%%%%%%%%%%%%%%%%%%%%%%%%%%%%%%%%%%%%%%%
% \section{Implementation}
%\iffalse
%<*package>
%\fi
%
% This section describes the definitions file |childdoc.def|.

% The definitions cannot be loaded using |\usepackage| or |\RequirePackage|
% which has a mechanism to prevent loading a style file more than once.
% When loading the definitions by means of |\input|
% multiple instances have to be prevented manually:
%\iffalse
%This code needs to be before the `\ProvidesFile' directive
%which is defined at the beginning of this file.
%Therefore it is also placed there and commented out here.
%</package>
%<*discard>
%\fi
%    \begin{macrocode}
\ifdefined\childdocmain\endinput\fi
%    \end{macrocode}
%\iffalse
%</discard>
%<*package>
%\fi
%
% \macro{\ifchilddoc}
% \macro{\ifchilddocmanual}
% The conditional |\ifchilddoc| tells whether a
% child (true) or main (false) document is being compiled.
% The conditional |\ifchilddocmanual| tells whether
% the |\includeonly| mechanism is used (false) or
% the selection of child files must be performed manually (true).
% The definitions initialise to false:
%    \begin{macrocode}
\newif\ifchilddoc
\newif\ifchilddocmanual
%    \end{macrocode}

% \macro{\childdocname}
% \macro{\childdocjob}
% The macro |\childdocname| stores the name of the main document
% to be compiled. The macro |\childdocjob| stores the name of
% the document on which the \LaTeX{} compiler was originally invoked.
% The content of |\jobname| cannot be compared
% to filenames specified in the source due to different catcodes.
% The following code rescans |\jobname|, stores the result
% in |\childdocname| and saves a copy in |\childdocjob|:
%    \begin{macrocode}
\edef\childdocname{\scantokens\expandafter{\jobname\noexpand}}
\let\childdocjob\childdocname
%    \end{macrocode}

% \macro{\childdocdisable}
% The macro |\childdocdisable| prevents the main file
% from being processed more than once.
% At this stage, the main document command |\childdocmain|
% is assumed to be called once again where it should do nothing.
% Any subsequent call to it should prevent
% a secondary processing of the main document
% It overwrites the forwarding commands
% |\childdocof| and |\childdocforward|
% with empty macros to prevent further inclusions of the main document:
%    \begin{macrocode}
\newcommand{\childdocdisable}
{
  \renewcommand{\childdocmain}[1]{\renewcommand{\childdocmain}[1]{\endinput}}
  \renewcommand{\childdocof}[1]{}
  \renewcommand{\childdocby}[2][]{}
  \renewcommand{\childdocforward}[2][]{}
  \renewcommand{\childdocdisable}{}
}
%    \end{macrocode}

% \macro{\childdocmain}
% The macro |\childdocmain| is to be called at the top of the main file
% with nothing or the main filename (without extension) as argument.
% First, it breaks loops.
% If the argument is not empty and does not match |\childdocname|
% (which is set by the first inclusion of |childdoc.def|),
% |\ifchilddoc| is set to true, |\includeonly| is applied to the child file
% and |\jobname| is set to the main file
% (for proper handling of |.aux| files):
%    \begin{macrocode}
\newcommand{\childdocmain}[1]
{
  \childdocdisable\childdocmain{}
  \if?#1?\else
    \begingroup
      \def\childdoctmp{#1}
      \ifx\childdoctmp\childdocname
        \def\childdoctmp{}
      \else
        \def\childdoctmp
        {
          \childdoctrue
          \includeonly{\childdocname}
          \def\childdocjob{#1}
          \def\jobname{#1}
        }
      \fi
      \expandafter
    \endgroup
    \childdoctmp
  \fi
}
%    \end{macrocode}

% \macro{\childdocof}
% The command |\childdocof| redirects
% compilation to the main file |#1|.
%    \begin{macrocode}
\newcommand{\childdocof}[1]
{
  \childdocdisable
  \childdoctrue
  \includeonly{\childdocname}
  \def\jobname{#1}
  \def\childdocjob{#1}
  \input{#1}
}
%    \end{macrocode}

% \macro{\childdocby}
% The command |\childdocby| ....
%    \begin{macrocode}
\newcommand{\childdocby}[2][]
{
  \childdocdisable
  \childdoctrue
  \childdocmanualtrue
  \if?#1?\else
    \def\jobname{#2}
  \fi
  \def\childdocjob{#2}
  \input{#2}
  \endinput
}
%    \end{macrocode}

% \macro{\childdocforward}
% The command |\childdocforward| redirects
% compilation to the main file or
% (if the optional argument is given) a child file.
% Parameters are set as if the main file
% or a child file starting with |\childdocof| was compiled.
% Then compilation is handed over to the main file:
%    \begin{macrocode}
\newcommand{\childdocforward}[2][]
{
  \begingroup
    \if?#1?
      \def\childdoctmp
      {
        \def\childdocname{#2}
        \def\childdocjob{#2}
        \def\jobname{#2}
        \input{#2}
        \endinput
      }
    \else
      \def\childdoctmp
      {
        \childdocdisable
        \def\childdocname{#2}
        \childdoctrue
        \includeonly{#2}
        \def\childdocjob{#1}
        \def\jobname{#1}
        \input{#1}
        \endinput
      }
    \fi
    \expandafter
  \endgroup
  \childdoctmp
}
%    \end{macrocode}

% \macro{\childdocforwardprefix}
% The command |\childdocforwardprefix| redirects
% compilation to the main or a child file by means of a pattern.
% The prefix |#1| in the current filename is replaced by |#2|
% and the suffix of the current filename is kept
% (it is assumed that the filename does not contain the substring `|~~~|'
% which is used as a delimiter).
% Compilation is handed over to the new file by |\childdocforward|:
%    \begin{macrocode}
\newcommand{\childdocforwardprefix}[3][]
{
  \begingroup
    \def\childdocextract #2##1~~~{\def\childdoctmp{\childdocforward[#1]{#3##1}}}
    \expandafter\childdocextract\childdocname~~~
    \expandafter
  \endgroup
  \childdoctmp
}
%    \end{macrocode}

% \macro{\childdoc}
% The deprecated macro |\childdoc| is a legacy version of |\childdocmain|:
%    \begin{macrocode}
\newcommand{\childdoc}{\childdocmain}
%    \end{macrocode}

% \macro{\childdocredirect}
% The deprecated macro |\childdocredirect| is a legacy version
% of |\childdocforward| and |\childdocforwardprefix|:
%    \begin{macrocode}
\newcommand{\childdocredirect}[2][]
{
  \begingroup
    \if?#1?
      \def\childdoctmp{\childdocforward{#2}}
    \else
      \def\childdoctmp{\childdocforwardprefix{#1}{#2}}
    \fi
    \expandafter
  \endgroup
  \childdoctmp
}
%    \end{macrocode}

%\iffalse
%</package>
%\fi
%
\endinput
|\\
|\childdocmain{}|\\
\end{tabular}
\end{center}
at the very top of the main \LaTeX{} file,
in particular \emph{before} the |\documentclass| statement!
The argument of |\childdocmain| should be left empty
(but it must be present).

%%%%%%%%%%%%%%%%%%%%%%%%%%%%%%%%%%%%%%%%
\DescribeMacro{\childdocof}
Furthermore, add the commands
\begin{center}
\begin{tabular}{l}
|% \iffalse
%
% childdoc.dtx Copyright (C) 2017-2018 Niklas Beisert
%
% This work may be distributed and/or modified under the
% conditions of the LaTeX Project Public License, either version 1.3
% of this license or (at your option) any later version.
% The latest version of this license is in
%   http://www.latex-project.org/lppl.txt
% and version 1.3 or later is part of all distributions of LaTeX
% version 2005/12/01 or later.
%
% This work has the LPPL maintenance status `maintained'.
%
% The Current Maintainer of this work is Niklas Beisert.
%
% This work consists of the files childdoc.dtx and childdoc.ins
% and the derived files childdoc.def and cdocsamp.tex with
% cdocsch1.tex, cdocsch2.tex, cdocsdrf.tex, cdocsfn1.tex, cdocsfn2.tex.
%
%<package>\ifdefined\childdocmain\endinput\fi
%<package>\ProvidesFile{childdoc.def}[2018/12/30 v2.0 child document driver]
%<samplemain>\ProvidesFile{cdocsamp.tex}[2018/12/30 v2.0 sample for childdoc]
%<*driver>
%\ProvidesFile{childdoc.drv}[2018/12/30 v2.0 childdoc reference manual file]
\PassOptionsToClass{10pt,a4paper}{article}
\documentclass{ltxdoc}

\usepackage[margin=35mm]{geometry}
\usepackage{hyperref}
\usepackage{hyperxmp}
\usepackage[usenames]{color}

\hypersetup{colorlinks=true}
\hypersetup{pdfstartview=FitH}
\hypersetup{pdfpagemode=UseNone}
\hypersetup{pdfsource={}}
\hypersetup{pdflang={en-UK}}
\hypersetup{pdfcopyright={Copyright 2017-2018 Niklas Beisert.
  This work may be distributed and/or modified under the
  conditions of the LaTeX Project Public License, either version 1.3
  of this license or (at your option) any later version.}}
\hypersetup{pdflicenseurl={http://www.latex-project.org/lppl.txt}}
\hypersetup{pdfcontactaddress={ETH Zurich, ITP, HIT K,
  Wolfgang-Pauli-Strasse 27}}
\hypersetup{pdfcontactpostcode={8093}}
\hypersetup{pdfcontactcity={Zurich}}
\hypersetup{pdfcontactcountry={Switzerland}}
\hypersetup{pdfcontactemail={nbeisert@itp.phys.ethz.ch}}
\hypersetup{pdfcontacturl={http://people.phys.ethz.ch/\xmptilde nbeisert/}}

\newcommand{\secref}[1]{\hyperref[#1]{section \ref*{#1}}}

\parskip1ex
\parindent0pt
\let\olditemize\itemize
\def\itemize{\olditemize\parskip0pt}

\begin{document}

\title{The \textsf{childdoc} Package}
\hypersetup{pdftitle={The childdoc Package}}
\author{Niklas Beisert\\[2ex]
  Institut f\"ur Theoretische Physik\\
  Eidgen\"ossische Technische Hochschule Z\"urich\\
  Wolfgang-Pauli-Strasse 27, 8093 Z\"urich, Switzerland\\[1ex]
  \href{mailto:nbeisert@itp.phys.ethz.ch}
  {\texttt{nbeisert@itp.phys.ethz.ch}}}
\hypersetup{pdfauthor={Niklas Beisert}}
\hypersetup{pdfsubject={Manual for the LaTeX2e Package childdoc}}
\date{30 December 2018, \textsf{v2.0}}
\maketitle

\begin{abstract}\noindent
\textsf{childdoc} is a \LaTeXe{} package
that enables the direct compilation
of document sections included by |\include|
to individual files.
\end{abstract}

\begingroup
\parskip0ex
\tableofcontents
\endgroup

%%%%%%%%%%%%%%%%%%%%%%%%%%%%%%%%%%%%%%%%%%%%%%%%%%%%%%%%%%%%%%%%%%%%%%%%%%%%%%%%
%%%%%%%%%%%%%%%%%%%%%%%%%%%%%%%%%%%%%%%%%%%%%%%%%%%%%%%%%%%%%%%%%%%%%%%%%%%%%%%%
\section{Introduction}

\LaTeX{} provides a mechanism to structure a large document (such as a book)
into a main file and several child files (containing the chapters)
using the |\include| command.
This mechanism is beneficial for documents
which span hundreds of pages in order to
make the source file(s) more manageable.
Moreover, compilation can be restricted to
selected child files by means of the |\includeonly| command.
The latter feature can be used to reduce the compilation time while editing
(this was significantly more useful in the earlier days of \LaTeX{})
or to generate a smaller document which is easier to navigate.
Another application of |\includeonly| is to generate
documents consisting of selected parts of the complete document.

However, there are a few drawbacks of the plain |\include| mechanism:
\begin{itemize}
\item
The child files cannot be compiled on their own,
they can only be compiled via the main file.
A naive editing environment
(such as a text editor with an option
to have the current file processed by \LaTeX)
may require one to switch to the main file before compiling;
attempting to compile the child file produces errors.
\item
The main file must be modified (each time)
to adjust the |\includeonly| command
to the present needs. This easily leaves the main file in a messy state.
\item
The generated document will always carry the filename
of the main document. This is inconvenient if
several child files are to be compiled and
to be kept for distribution.
\end{itemize}

The present package provides a simple interface
to make child files individually compilable by \LaTeX{}.
Compiling a child file then has the same effect as compiling
the main file with an |\includeonly| command
to select the appropriate child.
Moreover the generated document will carry the name of the child
rather than the main file.
This resolves all three above issues.

This feature is meant to make the editing of books,
thesis documents and lecture notes somewhat more convenient.
However, the package can also be used efficiently for
composing a series of documents (such as exercise sheets)
which are typically distributed individually.
It then assists the author in generating the individual documents
(potentially in different versions)
as well as a document containing the collected series.
Another application is in developing style files
or other kinds of included material
where compilation of the style file could redirect
to a sample or test file.

%%%%%%%%%%%%%%%%%%%%%%%%%%%%%%%%%%%%%%%%%%%%%%%%%%%%%%%%%%%%%%%%%%%%%%%%%%%%%%%%
%%%%%%%%%%%%%%%%%%%%%%%%%%%%%%%%%%%%%%%%%%%%%%%%%%%%%%%%%%%%%%%%%%%%%%%%%%%%%%%%
\section{Usage}

First of all, the package \textsf{childdoc} is \emph{not} a standard
\LaTeXe{} |.sty| style file! Therefore it needs to be invoked in
a non-standard way.

%%%%%%%%%%%%%%%%%%%%%%%%%%%%%%%%%%%%%%%%%%%%%%%%%%%%%%%%%%%%%%%%%%%%%%%%%%%%%%%%
\subsection{Included Files}
\label{sec:include}

%%%%%%%%%%%%%%%%%%%%%%%%%%%%%%%%%%%%%%%%
\DescribeMacro{\childdocmain}
To use the package, add the commands
\begin{center}
\begin{tabular}{l}
|\input{childdoc.def}|\\
|\childdocmain{}|\\
\end{tabular}
\end{center}
at the very top of the main \LaTeX{} file,
in particular \emph{before} the |\documentclass| statement!
The argument of |\childdocmain| should be left empty
(but it must be present).

%%%%%%%%%%%%%%%%%%%%%%%%%%%%%%%%%%%%%%%%
\DescribeMacro{\childdocof}
Furthermore, add the commands
\begin{center}
\begin{tabular}{l}
|\input{childdoc.def}|\\
|\childdocof{|\textit{main}|}|\\
\end{tabular}
\end{center}
at the top of every child file \textit{child}
which is included by |\include{|\textit{child}|}|
from within the main file
(or at least for those files to be compiled individually).
The argument \textit{main} must be the filename of the main file.

There are a couple of
considerations in setting up the main and child documents:

%%%%%%%%%%%%%%%%%%%%%%%%%%%%%%%%%%%%%%%%
\paragraph{Restrictions.}

Please note the following restrictions:
\begin{itemize}
\item
|\childdocmain| must be called with one argument \textit{main}
to ensure compatibility with earlier version of the package.
It must either be empty (|\childdocmain{}|)
or precisely match the filename of the main file in which it is specified.
See \secref{sec:detection} for further information.
\item
The filename \textit{main} must be specified without the |.tex| extension.
\item
The filename \textit{main} is case sensitive
(even in case-insensitive file systems)
due to internal string comparison.
\item
The argument \textit{main} should be fully expanded, it cannot be a macro.
\item
Subdirectories and special characters should be avoided in filenames.
\item
The command |\childdocmain{|\textit{main}|}| must be followed by a whitespace.
It should not be followed immediately by another command
or by a comment mark `|%|'.
This is because the \TeX{} parser reads the token immediately following
the argument of |\childdocmain| and puts it
at the beginning of every child section;
however, a white\-space is ignored.
\end{itemize}

%%%%%%%%%%%%%%%%%%%%%%%%%%%%%%%%%%%%%%%%
\paragraph{Content of Main File.}

It is advisable to place all content in the child files included by |\include|.
Any output contained in the main file will appear in all child documents
unless suppressed manually;
it cannot be suppressed automatically by the |\includeonly| directive
and thus should normally be avoided.
A method to include some content in the main file
by means of conditional processing is described in \secref{sec:conditional}.

%%%%%%%%%%%%%%%%%%%%%%%%%%%%%%%%%%%%%%%%
\paragraph{Page Numbering.}

When only a part of the document is compiled,
the appropriate numbering of pages
(as well as other status parameters)
is determined from the |.aux| files.
The latter contain information from previous passes.
However this information needs to propagate through
all intermediate child documents.
Therefore the page numbering in child documents may well
be inconsistent until the complete document is compiled at least once.

A useful (if unconventional) way to always ensure a consistent
page numbering is to restart the numbering in each child document
and denote the pages by `\textit{child}|.|\textit{page}'
where \textit{child} represents the chapter/section number of the child file.
This can be achieved by the command
|\numberwithin{page}{|\textit{child}|}|
of the \textsf{amsmath} package
where \textit{child} can be |chapter| or |section|
depending on the chosen structuring.
Alternatively, one can modify the macro |\thepage| appropriately
and reset the counter |page| at the start of each child file.

%%%%%%%%%%%%%%%%%%%%%%%%%%%%%%%%%%%%%%%%%%%%%%%%%%%%%%%%%%%%%%%%%%%%%%%%%%%%%%%%
\subsection{Conditional Processing}
\label{sec:conditional}

The package provides a mechanism to compile different versions
of a document. To customise the versions further some conditional processing
can come in handy to distinguish which version is being compiled.
The package provides two macros to describe the compilation context:

%%%%%%%%%%%%%%%%%%%%%%%%%%%%%%%%%%%%%%%%
\DescribeMacro{\ifchilddoc}
The conditional |\ifchilddoc| distinguishes between the compilation of
child documents and the main document:
%
\begin{center}
|\ifchilddoc |\textit{child-code}| |[|\||else |\textit{main-code}]| \||fi|
\end{center}

%%%%%%%%%%%%%%%%%%%%%%%%%%%%%%%%%%%%%%%%
\DescribeMacro{\childdocname}
\DescribeMacro{\childdocjob}
The macro |\childdocname| contains the filename (without extension)
of the main or child file being processed.
Note that |\childdocjob| will always contain the name of the main file.

%%%%%%%%%%%%%%%%%%%%%%%%%%%%%%%%%%%%%%%%
\paragraph{Title Page.}

Conditional processing can be used to include a title or banner page
in the main document when proper precautions are taken.
Importantly, the code in the main file should ensure that the page counter
(as well as other status parameters which are stored in the |.aux| files)
takes the same value after the conditional processing.
Otherwise the page numbers may take divergent values
depending on which part is compiled.

For example, a title page could be declared by:
%
\begin{center}
\begin{tabular}{l}
|\ifchilddoc\||else|\\
|\addtocounter{page}{-1}|\\
\textit{code for title page}\\
|\newpage|\\
|\||fi|
\end{tabular}
\end{center}
%
A banner page for the child documents can be generated by:
%
\begin{center}
\begin{tabular}{l}
|\ifchilddoc|\\
|\addtocounter{page}{-1}|\\
\textit{code for banner page}\\
|\newpage|\\
|\||fi|
\end{tabular}
\end{center}
%
Here one could write a message such as:
\begin{center}
|This is the part \childdocname{} of \childdocjob{}.|
\end{center}

%%%%%%%%%%%%%%%%%%%%%%%%%%%%%%%%%%%%%%%%%%%%%%%%%%%%%%%%%%%%%%%%%%%%%%%%%%%%%%%%
\subsection{Flags}
\label{sec:flags}

The package makes it easy to generate different versions
of the main or child documents.
To this end compilation flags can be defined
and assigned different default values.
They will be particularly useful in conjunction
with the forwarding mechanism described in \secref{sec:forward}.

For example, it may be useful to have a flag |\version|
which can be set to |draft| or |final|.
The document source will contain some conditional code
depending on the value of |\version|.
Suppose further, the flag should default to |final| for the main file
and to |draft| for child files
which is a natural assignment for editing the document.
This is achieved by placing the following code
in the preamble of the main document
(below the |\childdocmain| directive):
%
\begin{center}
\begin{tabular}{l}
|\ifchilddoc|\\
|\providecommand{\version}{draft}|\\
|\||else|\\
|\providecommand{\version}{final}|\\
|\||fi|
\end{tabular}
\end{center}
%
The definition by |\providecommand| makes sure
that previous definitions are not overwritten.
Further statements |\providecommand{\version}{...}|
can thus be added before the above code to override it.

For the main file, one might add a line
(between |\childdocmain| and the above block)
%
\begin{center}
|%\ifchilddoc\||else\providecommand{\version}{draft}\||fi|
\end{center}
%
which can be uncommented to produce a draft version.
Likewise one can add a line to the very top of a child file
(above the |\childdocof{|\textit{main}|}| directive)
%
\begin{center}
|%\providecommand{\version}{final}|
\end{center}
%
which can be uncommented to produce the final version of this child document.

%%%%%%%%%%%%%%%%%%%%%%%%%%%%%%%%%%%%%%%%%%%%%%%%%%%%%%%%%%%%%%%%%%%%%%%%%%%%%%%%
\subsection{Forwarding}
\label{sec:forward}

Different versions of the main or child documents
using compilation flags as described in \secref{sec:flags}
can be (permanently) stored in different files
for convenient compilation, viewing and distribution.
To this end, the package defines a command
to pass on compilation to a different file:

%%%%%%%%%%%%%%%%%%%%%%%%%%%%%%%%%%%%%%%%
\DescribeMacro{\childdocforward}
The command |\childdocforward| redirects processing to
another source file:
%
\begin{center}
\begin{tabular}{l}
|\input{childdoc.def}|\\
|\childdocforward[|\textit{main}|]{|\textit{dest}|}|\\
\end{tabular}
\end{center}
%
The argument \textit{dest} is the destination file
(without extension).
It should be the main file or one of the child files.
Note that further \textsf{childdoc} directives
such as |\childdocof| and |\childdocforward|
in the indicated file will be processed in this form.
The optional argument \textit{main}
passes on directly to the main file \textit{main}
while pretending to compile the child \textit{dest}.
This form behaves as if \textit{dest}
issues |\childdocof{|\textit{main}|}| right away,
and no further \textsf{childdoc} directives will be processed.

%%%%%%%%%%%%%%%%%%%%%%%%%%%%%%%%%%%%%%%%
\DescribeMacro{\...prefix}
In the alternative form |\childdocforwardprefix|,
%
\begin{center}
\begin{tabular}{l}
|\input{childdoc.def}|\\
|\childdocforwardprefix[|\textit{main}|]{|\textit{prefix}|}{|\textit{dest}|}|
\end{tabular}
\end{center}
%
the destination file is determined by a pattern
depending on the current file:
To make this work, the current file must be called
`{\textit{prefix}\hspace{0.2em}\textit{suffix}}'
with \textit{prefix} matching precisely the argument.
Processing is then passed on to the file
`{\textit{dest}\hspace{0.2em}\textit{suffix}}'.
Surely, the same effect is achieved by
directly specifying the
argument `{\textit{dest}\hspace{0.2em}\textit{suffix}}'
in the first form.
However, that requires to set up a different file
for each child. With the alternative form of the command
all these files can have exactly the same content
which simplifies setting them up and maintaining them.

For example, the following file |draft.tex|
with a compilation flag |\version| as described in \secref{sec:flags}
compiles the main document as a draft:
%
\begin{center}
\begin{tabular}{l}
|\def\version{draft}|\\
|\input{childdoc.def}|\\
|\childdocforward{|\textit{main}|}|
\end{tabular}
\end{center}
%
Likewise, the following files |final|\textit{nn}|.tex|
compile the final version of the child document
|child|\textit{nn}|.tex|:
%
\begin{center}
\begin{tabular}{l}
|\def\version{final}|\\
|\input{childdoc.def}|\\
|\childdocforwardprefix{final}{child}|
\end{tabular}
\end{center}
%

Note that when several versions of a main file and/or of each child file
are to be generated, it may be convenient to set up a |Makefile| or
shell script to automatise the process.

%%%%%%%%%%%%%%%%%%%%%%%%%%%%%%%%%%%%%%%%%%%%%%%%%%%%%%%%%%%%%%%%%%%%%%%%%%%%%%%%
\subsection{Command Line Processing}
\label{sec:commandline}

The effect of redirection files can also be achieved by invoking
the \LaTeX{} compiler with a more elaborate command line.
Most conveniently this should be done as part
of a shell script or a |Makefile|.

When using \textsf{childdoc} in the main file, the following
command lines effectively perform a redirection
(note that depending on the shell being used,
backslashes may have to be doubled: `|\|' $\to$ `|\\|'):
%
\begin{center}
|... -jobname "|\textit{target}|" |\\|"|[\textit{flags}]%
|\input{childdoc.def}\childdocforward[|\textit{main}|]{|\textit{dest}|}"|
\end{center}
%
Here \textit{target} is the name of the output file,
\textit{main} is the name of the main file
and \textit{dest} is the name of the main or child file to be processed
(all filenames without extensions).
The optional argument \textit{main} can be omitted
if \textit{main} matches \textit{dest}.
Optionally, compilation \textit{flags} can be defined via |\def| commands.
This command line makes the \TeX{} engine believe
it is compiling the file \textit{target}
whose content is specified as the latter parameter.
The provided code then forwards the processing to
\textit{main} or \textit{dest} as described in \secref{sec:forward}.

%%%%%%%%%%%%%%%%%%%%%%%%%%%%%%%%%%%%%%%%%%%%%%%%%%%%%%%%%%%%%%%%%%%%%%%%%%%%%%%%
\subsection{Include by Input}
\label{sec:input}

Including child documents by |\include| has some restrictions by design.
Most notably, the content of a child document always occupies
its own set of pages; pages cannot be shared between child documents.
Usually, this behaviour makes perfect sense
because each child document contain an essential part of the document.
However, in some situations it may be desirable to compose
a document from a collection of parts
without having mandatory page breaks between then.
For this case, the package
provides a mechanism to include parts
by |\input| which can also be processed individually.
However, by construction this mechanism
requires manual handling of the content to be output.

%%%%%%%%%%%%%%%%%%%%%%%%%%%%%%%%%%%%%%%%
\DescribeMacro{\ifchilddocmanual}
The main file should be prepared as usual, see \secref{sec:include}.
However, the document body must make a distinction
between processing of an individual part and of the main document, e.g.:
%
\begin{center}
\begin{tabular}{l}
|\ifchilddocmanual|\\
|\input{\childdocname}|\\
|\||else|\\
\textit{document body with }|\input{|\textit{part}|}|\\
|\||fi|
\end{tabular}
\end{center}
%
The conditional |\ifchilddocmanual| is true whenever
a part to be included by |\input| is being compiled,
and the name of the part is stored in |\childdocname|.

%%%%%%%%%%%%%%%%%%%%%%%%%%%%%%%%%%%%%%%%
\DescribeMacro{\childdocby}
Each part to be included by |\input| should start with:
%
\begin{center}
\begin{tabular}{l}
|\input{childdoc.def}|\\
|\childdocby{|\textit{main}|}|\\
\end{tabular}
\end{center}
%
The directive |\childdocby| is similar to |\childdocof|
described in \secref{sec:include},
but the subsequent selection of content must be done manually.
To that end, both |\ifchilddoc| and |\ifchilddocmanual|
will be true upon processing of a part,
and the name of the part is stored in |\childdocname|.
Note that |\jobname| will be set to the filename of the current part
so that each part receives an individual |.aux| file
that does not interfere with the |.aux| file(s) of the main document.
This behaviour can be altered by the alternative form
|\childdocby[*]{|\textit{main}|}| (with a non-empty optional argument)
which uses the |.aux| file of the main document
by setting |\jobname| to \textit{main}.

%%%%%%%%%%%%%%%%%%%%%%%%%%%%%%%%%%%%%%%%%%%%%%%%%%%%%%%%%%%%%%%%%%%%%%%%%%%%%%%%
\subsection{Driver Development}
\label{sec:driver}

The \textsf{childdoc} mechanism can also be use for the development
of definition files such as \LaTeX{} styles or classes.
This case differs from the above setup with multiple parts
included by |\include| in that no |\includeonly| should be invoked.
This can be achieved by starting the include file
(before |\ProvidesPackage|) with:
%
\begin{center}
\begin{tabular}{l}
|\input{childdoc.def}|\\
|\childdocforward{|\textit{main}|}|\\
\end{tabular}
\end{center}
%
or alternatively with:
%
\begin{center}
\begin{tabular}{l}
|\input{childdoc.def}|\\
|\childdocby{|\textit{main}|}|\\
\end{tabular}
\end{center}
%
Both forms have slightly different effects as described above.
The main file is prepared as usual, see \secref{sec:include}.

%%%%%%%%%%%%%%%%%%%%%%%%%%%%%%%%%%%%%%%%%%%%%%%%%%%%%%%%%%%%%%%%%%%%%%%%%%%%%%%%
\subsection{Legacy Detection}
\label{sec:detection}

The directive |\childdocmain| in the main file can detect
whether the complete document or merely a child is to be compiled
even without using the directive |\childdocof|.
This method is deprecated because it is less robust
and there is no compelling reason to use it;
it is merely provided for backward compatibility
and it may be removed in future versions.

If the detection mechanism is to be used,
it is mandatory to correctly specify
the filename of the main file as the argument of |\childdocmain|:
%
\begin{center}
\begin{tabular}{l}
|\input{childdoc.def}|\\
|\childdocmain{|\textit{main}|}|\\
\end{tabular}
\end{center}
%
If |\jobname| does not match the argument \textit{main} of |\childdocmain|,
it is assumed that |\jobname| points to the child file to be compiled.
When using |\childdocmain| with the main file specified as argument,
it suffices to start a child file
with just |\input{|\textit{main}|}|
without loading of the package and using |\childdocof|.
If instead all processing is done
with the appropriate \textsf{childdoc} directives,
the argument of \textit{main} of |\childdocmain| can be empty.

An alternative version of the command line processing described
in \secref{sec:commandline} using the detection mechanism reads:
%
\begin{center}
|... -jobname "|\textit{target}|" "|[\textit{flags}]%
[|\def\jobname{|\textit{dest}|}|]|\input{|\textit{main}|}"|
\end{center}

%%%%%%%%%%%%%%%%%%%%%%%%%%%%%%%%%%%%%%%%%%%%%%%%%%%%%%%%%%%%%%%%%%%%%%%%%%%%%%%%
\subsection{Manual Code}
\label{sec:manual}

In case one cannot be certain whether the definitions file |childdoc.def|
is installed on the target \TeX{} distribution
and one prefers not to ship it,
it is conceivable to paste a few relevant commands into the sources.

To that end, drop all statements |\input{childdoc.def}|
and perform the replacements as outlined below.
Instead of |\childdocmain{|\textit{main}|}| add the following code
to the top of the main file:
%
\begin{center}
\begin{tabular}{l}
|\||ifdefined\childdocname\endinput\||fi\newif\ifchilddoc|\\
|\edef\childdocname{\scantokens\expandafter{\jobname\noexpand}}|\\
|\def\childdocmain{|\textit{main}|}\||ifx\childdocmain\childdocname\||else|\\
|\childdoctrue\includeonly{\childdocname}\let\jobname\childdocmain\||fi|\\
\end{tabular}
\end{center}
%
Instead of |\childdocof{|\textit{main}|}| just include the main file
at the top of each child file:
%
\begin{center}
|\input{|\textit{main}|}|
\end{center}
%
A simple redirection |\childdocforward{|\textit{dest}|}| is achieved by:
%
\begin{center}
|\def\jobname{|\textit{dest}|}\input{\jobname}|
\end{center}
%
The redirection with prefix
|\childdocforwardprefix[|\textit{prefix}|]{|\textit{dest}|}|
is accomplished by:
%
\begin{center}
\begin{tabular}{l}
|{\edef\jobname{\scantokens\expandafter{\jobname\noexpand}}|\\
|\def\redirectjob |\textit{prefix}|#1~~~{\gdef\jobname{|\textit{dest}|#1}}|\\
|\expandafter\redirectjob\jobname~~~}\input{\jobname}|
\end{tabular}
\end{center}

In an alternative approach,
child documents can be compiled by a specific command line
without additional code or specific definitions:
%
\begin{center}
|... -jobname "|\textit{target}|" "|[\textit{flags}]%
|\includeonly{|\textit{dest}|}\input{|\textit{main}|}"|
\end{center}
%

%%%%%%%%%%%%%%%%%%%%%%%%%%%%%%%%%%%%%%%%%%%%%%%%%%%%%%%%%%%%%%%%%%%%%%%%%%%%%%%%
%%%%%%%%%%%%%%%%%%%%%%%%%%%%%%%%%%%%%%%%%%%%%%%%%%%%%%%%%%%%%%%%%%%%%%%%%%%%%%%%
\section{Information}

%%%%%%%%%%%%%%%%%%%%%%%%%%%%%%%%%%%%%%%%%%%%%%%%%%%%%%%%%%%%%%%%%%%%%%%%%%%%%%%%
\subsection{Copyright}

Copyright \copyright{} 2017--2018 Niklas Beisert

This work may be distributed and/or modified under the
conditions of the \LaTeX{} Project Public License, either version 1.3
of this license or (at your option) any later version.
The latest version of this license is in
  \url{http://www.latex-project.org/lppl.txt}
and version 1.3 or later is part of all distributions of \LaTeX{}
version 2005/12/01 or later.

This work has the LPPL maintenance status `maintained'.

The Current Maintainer of this work is Niklas Beisert.

This work consists of the files |README.txt|, |childdoc.ins| and |childdoc.dtx|
as well as the derived files |childdoc.def|, |cdocsamp.tex|
with |cdocsch1.tex|, |cdocsch2.tex|, |cdocspt3.tex|, |cdocspt4.tex|,
|cdocsdrf.tex|, |cdocsfn1.tex|, |cdocsfn2.tex|
as well as |childdoc.pdf|.

%%%%%%%%%%%%%%%%%%%%%%%%%%%%%%%%%%%%%%%%%%%%%%%%%%%%%%%%%%%%%%%%%%%%%%%%%%%%%%%%
\subsection{Files and Installation}

The package consists of the files:
%
\begin{center}
\begin{tabular}{ll}
    |README.txt|   & readme file \\
    |childdoc.ins| & installation file \\
    |childdoc.dtx| & source file \\
    |childdoc.def| & definition file \\
    |cdocsamp.tex| & sample main file \\
    |cdocsch1.tex| & sample include file \\
    |cdocsch2.tex| & sample include file \\
    |cdocspt3.tex| & sample part file \\
    |cdocspt4.tex| & sample part file \\
    |cdocsdrf.tex| & sample redirection file \\
    |cdocsfn1.tex| & sample redirection file \\
    |cdocsfn2.tex| & sample redirection file \\
    |childdoc.pdf| & manual
\end{tabular}
\end{center}
%
The distribution consists of the files
|README.txt|, |childdoc.ins| and |childdoc.dtx|.
%
\begin{itemize}
\item
Run (pdf)\LaTeX{} on |childdoc.dtx|
to compile the manual |childdoc.pdf| (this file).
\item
Run \LaTeX{} on |childdoc.ins| to create the definitions file |childdoc.def|
and the sample |cdocsamp.tex| with include files
|cdocsch1.tex|, |cdocsch2.tex|, |cdocspt3.tex|, |cdocspt4.tex|,
|cdocsdrf.tex|, |cdocsfn1.tex|, |cdocsfn2.tex|.
Then copy the file |childdoc.def| to an appropriate directory of your \LaTeX{}
distribution, e.g.\ \textit{texmf-root}|/tex/latex/childdoc|.
\end{itemize}

%%%%%%%%%%%%%%%%%%%%%%%%%%%%%%%%%%%%%%%%%%%%%%%%%%%%%%%%%%%%%%%%%%%%%%%%%%%%%%%%
\subsection{Related CTAN Packages}

There are several other packages which offer a similar functionality:
%
\begin{itemize}
\item
The packages
\href{http://ctan.org/pkg/docmute}{\textsf{docmute}},
\href{http://ctan.org/pkg/includex}{\textsf{includex}} and
\href{http://ctan.org/pkg/standalone}{\textsf{standalone}}
provide commands to include only the document body of
a child file thus allowing both files to be compiled individually.
\item
The packages \href{http://ctan.org/pkg/subdocs}{\textsf{subdocs}}
and \href{http://ctan.org/pkg/subfiles}{\textsf{subfiles}}
provide structures in which the main and child documents can be
encapsulated and allowing them to be compiled individually.
The inclusion mechanism is different from the conventional |\include|.
\item
The package \href{http://ctan.org/pkg/combine}{\textsf{combine}}
is an elaborate solution to combine several documents into one.
\end{itemize}
%
See also the CTAN topic \href{http://ctan.org/topic/subdocs}{\textsf{subdocs}}
for further related packages.
The present package differs from the above solutions in that
a document structure constructed with the conventional |\include| mechanism
just needs two extra commands at the top of every file
such that all constituent files can be compiled individually.

%%%%%%%%%%%%%%%%%%%%%%%%%%%%%%%%%%%%%%%%%%%%%%%%%%%%%%%%%%%%%%%%%%%%%%%%%%%%%%%%
%\subsection{Feature Suggestions}
%
%The following is a list of features which may be useful for future
%versions of this package:
%%
%\begin{itemize}
%\item
%\ldots
%\end{itemize}

%%%%%%%%%%%%%%%%%%%%%%%%%%%%%%%%%%%%%%%%%%%%%%%%%%%%%%%%%%%%%%%%%%%%%%%%%%%%%%%%
\subsection{Revision History}

%%%%%%%%%%%%%%%%%%%%%%%%%%%%%%%%%%%%%%%%
\paragraph{v2.0:} 2018/12/30

\begin{itemize}
\item
immediate forward processing
\item
added |\childdocby| mechanism
\item
manual restructured
\end{itemize}

%%%%%%%%%%%%%%%%%%%%%%%%%%%%%%%%%%%%%%%%
\paragraph{v1.6:} 2018/01/17

\begin{itemize}
\item
application for development of include files
\item
corrections to manual
\end{itemize}

%%%%%%%%%%%%%%%%%%%%%%%%%%%%%%%%%%%%%%%%
\paragraph{v1.5:} 2017/05/21

\begin{itemize}
\item
more complete structuring introduced
\item
|\childdocof| introduced
\item
|\childdoc| renamed to |\childdocmain|
\item
|\childredirect| renamed to |\childdocforward| and |\childdocforwardprefix|
and functionality expanded
\end{itemize}

%%%%%%%%%%%%%%%%%%%%%%%%%%%%%%%%%%%%%%%%
\paragraph{v1.0:} 2017/04/27

\begin{itemize}
\item
manual and install package
\item
first version published on CTAN
\end{itemize}

%%%%%%%%%%%%%%%%%%%%%%%%%%%%%%%%%%%%%%%%
\paragraph{v0.6:} 2017/04/26

\begin{itemize}
\item
redirection mechanism added
\end{itemize}

%%%%%%%%%%%%%%%%%%%%%%%%%%%%%%%%%%%%%%%%
\paragraph{v0.5:} 2017/04/26

\begin{itemize}
\item
functionality in definition file
\end{itemize}


%%%%%%%%%%%%%%%%%%%%%%%%%%%%%%%%%%%%%%%%%%%%%%%%%%%%%%%%%%%%%%%%%%%%%%%%%%%%%%%%
%%%%%%%%%%%%%%%%%%%%%%%%%%%%%%%%%%%%%%%%%%%%%%%%%%%%%%%%%%%%%%%%%%%%%%%%%%%%%%%%
%%%%%%%%%%%%%%%%%%%%%%%%%%%%%%%%%%%%%%%%%%%%%%%%%%%%%%%%%%%%%%%%%%%%%%%%%%%%%%%%
\appendix

\settowidth\MacroIndent{\rmfamily\scriptsize 000\ }

 \DocInput{childdoc.dtx}

\end{document}
%</driver>
% \fi
%
% %%%%%%%%%%%%%%%%%%%%%%%%%%%%%%%%%%%%%%%%%%%%%%%%%%%%%%%%%%%%%%%%%%%%%%%%%%%%%%
% %%%%%%%%%%%%%%%%%%%%%%%%%%%%%%%%%%%%%%%%%%%%%%%%%%%%%%%%%%%%%%%%%%%%%%%%%%%%%%
% \section{Sample}
%\iffalse
%<*samplemain>
%\fi
%
% The following presents a sample document
% with two chapters, two parts, a title page,
% a compile flag as well as three forwarding files to set the flag.
% It consists of eight |.tex| files:
% \begin{center}
% \begin{tabular}{ll}
% |cdocsamp.tex|&main file\\
% |cdocsch1.tex|&include file for chapter 1\\
% |cdocsch2.tex|&include file for chapter 2\\
% |cdocspt3.tex|&include file for part 3\\
% |cdocspt4.tex|&include file for part 4\\
% |cdocsdrf.tex|&forwarding file for main file in draft mode\\
% |cdocsfi1.tex|&forwarding file for final version of chapter 1\\
% |cdocsfi2.tex|&forwarding file for final version of chapter 2\\
% \end{tabular}
% \end{center}
% Each of the eight files can be compiled directly by the \LaTeX{} compiler.
%
% %%%%%%%%%%%%%%%%%%%%%%%%%%%%%%%%%%%%%%
% \paragraph{Main File.}
%
% The main file is called |cdocsamp.tex|.
%
% Load the \textsf{childdoc} definitions and
% declare the filename for the main document:
%    \begin{macrocode}
\input{childdoc.def}
\childdocmain{}
%    \end{macrocode}

% Optional override for |\version| flag:
%    \begin{macrocode}
%%\ifchilddoc\else\providecommand{\version}{draft}\fi
%    \end{macrocode}

% Define the default values for the |\version| flag
% (|final| for the main file and |draft| for childs):
%    \begin{macrocode}
\ifchilddoc
\providecommand{\version}{draft}
\else
\providecommand{\version}{final}
\fi
%    \end{macrocode}

% Load the standard document class:
%    \begin{macrocode}
\documentclass[12pt]{article}
%    \end{macrocode}

% Start the document body:
%    \begin{macrocode}
\begin{document}
%    \end{macrocode}

% Declare a title page.
% Print title, part of document being processed and version flag:
%    \begin{macrocode}
\addtocounter{page}{-1}
\begin{center}
{\LARGE\bfseries{}childdoc example\par}
\vspace{1cm}
\ifchilddoc
\ifchilddocmanual part\else chapter\fi:
`\childdocname' of `\childdocjob'\par
\else
main document: `\childdocjob'\par
\fi
version: \version\par
\end{center}
\newpage
%    \end{macrocode}

% Manually include selected file,
% otherwise process as usual:
%    \begin{macrocode}
\ifchilddocmanual
\section*{part `\childdocname'}
\input{\childdocname}
\else
%    \end{macrocode}

% Include the two chapters:
%    \begin{macrocode}
\include{cdocsch1}
\include{cdocsch2}
%    \end{macrocode}

% Include the two parts unless only chapters should be displayed:
%    \begin{macrocode}
\ifchilddoc\else
\section{part three}
\input{cdocspt3}
\section{part four}
\input{cdocspt4}
\fi
%    \end{macrocode}

% Process as usual until here:
%    \begin{macrocode}
\fi
%    \end{macrocode}

% End of document body:
%    \begin{macrocode}
\end{document}
%    \end{macrocode}
%\iffalse
%</samplemain>
%\fi
%
% %%%%%%%%%%%%%%%%%%%%%%%%%%%%%%%%%%%%%%
% \paragraph{Chapter Include Files.}
%
% The include files are called |cdocsch1.tex| and |cdocsch2.tex|.
%
%\iffalse
%<*samplechap1|samplechap2>
%\fi

% Optional override for |\version| flag:
%    \begin{macrocode}
%%\providecommand{\version}{final}
%    \end{macrocode}

% Include the main document:
%    \begin{macrocode}
\input{childdoc.def}
\childdocof{cdocsamp}
%    \end{macrocode}

%\iffalse
%</samplechap1|samplechap2>
%\fi
%
%\iffalse
%<*samplechap1>
%\fi
% Some text for chapter 1:
%    \begin{macrocode}
\section{one}
some text in chapter one
%    \end{macrocode}

%\iffalse
%</samplechap1>
%\fi
% Some text for chapter 2:
%\iffalse
%<*samplechap2>
%\fi
%    \begin{macrocode}
\section{two}
more text in chapter two
%    \end{macrocode}

%\iffalse
%</samplechap2>
%\fi
%
% %%%%%%%%%%%%%%%%%%%%%%%%%%%%%%%%%%%%%%
% \paragraph{Part Include Files.}
%
% The include files are called |cdocspt3.tex| and |cdocspt4.tex|.
%
%\iffalse
%<*samplepart3|samplepart4>
%\fi

% Optional override for |\version| flag:
%    \begin{macrocode}
%%\providecommand{\version}{final}
%    \end{macrocode}

% Include the main document:
%    \begin{macrocode}
\input{childdoc.def}
\childdocby{cdocsamp}
%    \end{macrocode}

%\iffalse
%</samplepart3|samplepart4>
%\fi
%
%\iffalse
%<*samplepart3>
%\fi
% Some text for part 3:
%    \begin{macrocode}
some text in part three
%    \end{macrocode}

%\iffalse
%</samplepart3>
%\fi
% Some text for part 4:
%\iffalse
%<*samplepart4>
%\fi
%    \begin{macrocode}
more text in part four
%    \end{macrocode}

%\iffalse
%</samplepart4>
%\fi
%
% %%%%%%%%%%%%%%%%%%%%%%%%%%%%%%%%%%%%%%
% \paragraph{Forwarding for a Complete Draft.}
%
% The following forwarding file |cdocsdrf.tex|
% compiles the main document in draft mode:
%\iffalse
%<*sampledraft>
%\fi
%    \begin{macrocode}
\def\version{draft}
\input{childdoc.def}
\childdocforward{cdocsamp}
%    \end{macrocode}

%\iffalse
%</sampledraft>
%\fi
%
% %%%%%%%%%%%%%%%%%%%%%%%%%%%%%%%%%%%%%%
% \paragraph{Forwarding for Final Version of the Chapters.}
%
% The following forwarding files |cdocsfn1.tex| and |cdocsfn2.tex|
% (with identical content)
% compile the final versions of the child documents
% |cdocsch1.tex| and |cdocsch2.tex|, respectively:
%\iffalse
%<*samplefinal>
%\fi
%    \begin{macrocode}
\def\version{final}
\input{childdoc.def}
\childdocforwardprefix[cdocsamp]{cdocsfn}{cdocsch}
%    \end{macrocode}

%\iffalse
%</samplefinal>
%\fi
%
% %%%%%%%%%%%%%%%%%%%%%%%%%%%%%%%%%%%%%%
% \paragraph{Command Line Processing.}
%
% The following three command lines generate the output files
% |cdocscld|, |cdocscl1| and |cdocscl2|
% which should be identical to
% |cdocsdrf|, |cdocsch1| and |cdocsfn2|, respectively:
% \begin{center}
% \begin{tabular}{l}
% |latex -jobname cdocscld \|\\
% |  "\def\version{draft}\input{childdoc.def}\childdocforward{cdocsamp}"|\\
% |latex -jobname cdocscl1 \|\\
% |  "\input{childdoc.def}\childdocforward[cdocsamp]{cdocsch1}"|\\
% |latex -jobname cdocscl2 \|\\
% |  "\def\version{final}\input{childdoc.def}\childdocforward{cdocsch2}"|
% \end{tabular}
% \end{center}
% Note that the trailing backslash on each first line
% merely continues the input to the second line
% (for convenient cut ant paste).
% Furthermore, the command |latex| can be replaced by any
% of its alternative versions such as |pdflatex|.
%
% %%%%%%%%%%%%%%%%%%%%%%%%%%%%%%%%%%%%%%%%%%%%%%%%%%%%%%%%%%%%%%%%%%%%%%%%%%%%%%
% %%%%%%%%%%%%%%%%%%%%%%%%%%%%%%%%%%%%%%%%%%%%%%%%%%%%%%%%%%%%%%%%%%%%%%%%%%%%%%
% \section{Implementation}
%\iffalse
%<*package>
%\fi
%
% This section describes the definitions file |childdoc.def|.

% The definitions cannot be loaded using |\usepackage| or |\RequirePackage|
% which has a mechanism to prevent loading a style file more than once.
% When loading the definitions by means of |\input|
% multiple instances have to be prevented manually:
%\iffalse
%This code needs to be before the `\ProvidesFile' directive
%which is defined at the beginning of this file.
%Therefore it is also placed there and commented out here.
%</package>
%<*discard>
%\fi
%    \begin{macrocode}
\ifdefined\childdocmain\endinput\fi
%    \end{macrocode}
%\iffalse
%</discard>
%<*package>
%\fi
%
% \macro{\ifchilddoc}
% \macro{\ifchilddocmanual}
% The conditional |\ifchilddoc| tells whether a
% child (true) or main (false) document is being compiled.
% The conditional |\ifchilddocmanual| tells whether
% the |\includeonly| mechanism is used (false) or
% the selection of child files must be performed manually (true).
% The definitions initialise to false:
%    \begin{macrocode}
\newif\ifchilddoc
\newif\ifchilddocmanual
%    \end{macrocode}

% \macro{\childdocname}
% \macro{\childdocjob}
% The macro |\childdocname| stores the name of the main document
% to be compiled. The macro |\childdocjob| stores the name of
% the document on which the \LaTeX{} compiler was originally invoked.
% The content of |\jobname| cannot be compared
% to filenames specified in the source due to different catcodes.
% The following code rescans |\jobname|, stores the result
% in |\childdocname| and saves a copy in |\childdocjob|:
%    \begin{macrocode}
\edef\childdocname{\scantokens\expandafter{\jobname\noexpand}}
\let\childdocjob\childdocname
%    \end{macrocode}

% \macro{\childdocdisable}
% The macro |\childdocdisable| prevents the main file
% from being processed more than once.
% At this stage, the main document command |\childdocmain|
% is assumed to be called once again where it should do nothing.
% Any subsequent call to it should prevent
% a secondary processing of the main document
% It overwrites the forwarding commands
% |\childdocof| and |\childdocforward|
% with empty macros to prevent further inclusions of the main document:
%    \begin{macrocode}
\newcommand{\childdocdisable}
{
  \renewcommand{\childdocmain}[1]{\renewcommand{\childdocmain}[1]{\endinput}}
  \renewcommand{\childdocof}[1]{}
  \renewcommand{\childdocby}[2][]{}
  \renewcommand{\childdocforward}[2][]{}
  \renewcommand{\childdocdisable}{}
}
%    \end{macrocode}

% \macro{\childdocmain}
% The macro |\childdocmain| is to be called at the top of the main file
% with nothing or the main filename (without extension) as argument.
% First, it breaks loops.
% If the argument is not empty and does not match |\childdocname|
% (which is set by the first inclusion of |childdoc.def|),
% |\ifchilddoc| is set to true, |\includeonly| is applied to the child file
% and |\jobname| is set to the main file
% (for proper handling of |.aux| files):
%    \begin{macrocode}
\newcommand{\childdocmain}[1]
{
  \childdocdisable\childdocmain{}
  \if?#1?\else
    \begingroup
      \def\childdoctmp{#1}
      \ifx\childdoctmp\childdocname
        \def\childdoctmp{}
      \else
        \def\childdoctmp
        {
          \childdoctrue
          \includeonly{\childdocname}
          \def\childdocjob{#1}
          \def\jobname{#1}
        }
      \fi
      \expandafter
    \endgroup
    \childdoctmp
  \fi
}
%    \end{macrocode}

% \macro{\childdocof}
% The command |\childdocof| redirects
% compilation to the main file |#1|.
%    \begin{macrocode}
\newcommand{\childdocof}[1]
{
  \childdocdisable
  \childdoctrue
  \includeonly{\childdocname}
  \def\jobname{#1}
  \def\childdocjob{#1}
  \input{#1}
}
%    \end{macrocode}

% \macro{\childdocby}
% The command |\childdocby| ....
%    \begin{macrocode}
\newcommand{\childdocby}[2][]
{
  \childdocdisable
  \childdoctrue
  \childdocmanualtrue
  \if?#1?\else
    \def\jobname{#2}
  \fi
  \def\childdocjob{#2}
  \input{#2}
  \endinput
}
%    \end{macrocode}

% \macro{\childdocforward}
% The command |\childdocforward| redirects
% compilation to the main file or
% (if the optional argument is given) a child file.
% Parameters are set as if the main file
% or a child file starting with |\childdocof| was compiled.
% Then compilation is handed over to the main file:
%    \begin{macrocode}
\newcommand{\childdocforward}[2][]
{
  \begingroup
    \if?#1?
      \def\childdoctmp
      {
        \def\childdocname{#2}
        \def\childdocjob{#2}
        \def\jobname{#2}
        \input{#2}
        \endinput
      }
    \else
      \def\childdoctmp
      {
        \childdocdisable
        \def\childdocname{#2}
        \childdoctrue
        \includeonly{#2}
        \def\childdocjob{#1}
        \def\jobname{#1}
        \input{#1}
        \endinput
      }
    \fi
    \expandafter
  \endgroup
  \childdoctmp
}
%    \end{macrocode}

% \macro{\childdocforwardprefix}
% The command |\childdocforwardprefix| redirects
% compilation to the main or a child file by means of a pattern.
% The prefix |#1| in the current filename is replaced by |#2|
% and the suffix of the current filename is kept
% (it is assumed that the filename does not contain the substring `|~~~|'
% which is used as a delimiter).
% Compilation is handed over to the new file by |\childdocforward|:
%    \begin{macrocode}
\newcommand{\childdocforwardprefix}[3][]
{
  \begingroup
    \def\childdocextract #2##1~~~{\def\childdoctmp{\childdocforward[#1]{#3##1}}}
    \expandafter\childdocextract\childdocname~~~
    \expandafter
  \endgroup
  \childdoctmp
}
%    \end{macrocode}

% \macro{\childdoc}
% The deprecated macro |\childdoc| is a legacy version of |\childdocmain|:
%    \begin{macrocode}
\newcommand{\childdoc}{\childdocmain}
%    \end{macrocode}

% \macro{\childdocredirect}
% The deprecated macro |\childdocredirect| is a legacy version
% of |\childdocforward| and |\childdocforwardprefix|:
%    \begin{macrocode}
\newcommand{\childdocredirect}[2][]
{
  \begingroup
    \if?#1?
      \def\childdoctmp{\childdocforward{#2}}
    \else
      \def\childdoctmp{\childdocforwardprefix{#1}{#2}}
    \fi
    \expandafter
  \endgroup
  \childdoctmp
}
%    \end{macrocode}

%\iffalse
%</package>
%\fi
%
\endinput
|\\
|\childdocof{|\textit{main}|}|\\
\end{tabular}
\end{center}
at the top of every child file \textit{child}
which is included by |\include{|\textit{child}|}|
from within the main file
(or at least for those files to be compiled individually).
The argument \textit{main} must be the filename of the main file.

There are a couple of
considerations in setting up the main and child documents:

%%%%%%%%%%%%%%%%%%%%%%%%%%%%%%%%%%%%%%%%
\paragraph{Restrictions.}

Please note the following restrictions:
\begin{itemize}
\item
|\childdocmain| must be called with one argument \textit{main}
to ensure compatibility with earlier version of the package.
It must either be empty (|\childdocmain{}|)
or precisely match the filename of the main file in which it is specified.
See \secref{sec:detection} for further information.
\item
The filename \textit{main} must be specified without the |.tex| extension.
\item
The filename \textit{main} is case sensitive
(even in case-insensitive file systems)
due to internal string comparison.
\item
The argument \textit{main} should be fully expanded, it cannot be a macro.
\item
Subdirectories and special characters should be avoided in filenames.
\item
The command |\childdocmain{|\textit{main}|}| must be followed by a whitespace.
It should not be followed immediately by another command
or by a comment mark `|%|'.
This is because the \TeX{} parser reads the token immediately following
the argument of |\childdocmain| and puts it
at the beginning of every child section;
however, a white\-space is ignored.
\end{itemize}

%%%%%%%%%%%%%%%%%%%%%%%%%%%%%%%%%%%%%%%%
\paragraph{Content of Main File.}

It is advisable to place all content in the child files included by |\include|.
Any output contained in the main file will appear in all child documents
unless suppressed manually;
it cannot be suppressed automatically by the |\includeonly| directive
and thus should normally be avoided.
A method to include some content in the main file
by means of conditional processing is described in \secref{sec:conditional}.

%%%%%%%%%%%%%%%%%%%%%%%%%%%%%%%%%%%%%%%%
\paragraph{Page Numbering.}

When only a part of the document is compiled,
the appropriate numbering of pages
(as well as other status parameters)
is determined from the |.aux| files.
The latter contain information from previous passes.
However this information needs to propagate through
all intermediate child documents.
Therefore the page numbering in child documents may well
be inconsistent until the complete document is compiled at least once.

A useful (if unconventional) way to always ensure a consistent
page numbering is to restart the numbering in each child document
and denote the pages by `\textit{child}|.|\textit{page}'
where \textit{child} represents the chapter/section number of the child file.
This can be achieved by the command
|\numberwithin{page}{|\textit{child}|}|
of the \textsf{amsmath} package
where \textit{child} can be |chapter| or |section|
depending on the chosen structuring.
Alternatively, one can modify the macro |\thepage| appropriately
and reset the counter |page| at the start of each child file.

%%%%%%%%%%%%%%%%%%%%%%%%%%%%%%%%%%%%%%%%%%%%%%%%%%%%%%%%%%%%%%%%%%%%%%%%%%%%%%%%
\subsection{Conditional Processing}
\label{sec:conditional}

The package provides a mechanism to compile different versions
of a document. To customise the versions further some conditional processing
can come in handy to distinguish which version is being compiled.
The package provides two macros to describe the compilation context:

%%%%%%%%%%%%%%%%%%%%%%%%%%%%%%%%%%%%%%%%
\DescribeMacro{\ifchilddoc}
The conditional |\ifchilddoc| distinguishes between the compilation of
child documents and the main document:
%
\begin{center}
|\ifchilddoc |\textit{child-code}| |[|\||else |\textit{main-code}]| \||fi|
\end{center}

%%%%%%%%%%%%%%%%%%%%%%%%%%%%%%%%%%%%%%%%
\DescribeMacro{\childdocname}
\DescribeMacro{\childdocjob}
The macro |\childdocname| contains the filename (without extension)
of the main or child file being processed.
Note that |\childdocjob| will always contain the name of the main file.

%%%%%%%%%%%%%%%%%%%%%%%%%%%%%%%%%%%%%%%%
\paragraph{Title Page.}

Conditional processing can be used to include a title or banner page
in the main document when proper precautions are taken.
Importantly, the code in the main file should ensure that the page counter
(as well as other status parameters which are stored in the |.aux| files)
takes the same value after the conditional processing.
Otherwise the page numbers may take divergent values
depending on which part is compiled.

For example, a title page could be declared by:
%
\begin{center}
\begin{tabular}{l}
|\ifchilddoc\||else|\\
|\addtocounter{page}{-1}|\\
\textit{code for title page}\\
|\newpage|\\
|\||fi|
\end{tabular}
\end{center}
%
A banner page for the child documents can be generated by:
%
\begin{center}
\begin{tabular}{l}
|\ifchilddoc|\\
|\addtocounter{page}{-1}|\\
\textit{code for banner page}\\
|\newpage|\\
|\||fi|
\end{tabular}
\end{center}
%
Here one could write a message such as:
\begin{center}
|This is the part \childdocname{} of \childdocjob{}.|
\end{center}

%%%%%%%%%%%%%%%%%%%%%%%%%%%%%%%%%%%%%%%%%%%%%%%%%%%%%%%%%%%%%%%%%%%%%%%%%%%%%%%%
\subsection{Flags}
\label{sec:flags}

The package makes it easy to generate different versions
of the main or child documents.
To this end compilation flags can be defined
and assigned different default values.
They will be particularly useful in conjunction
with the forwarding mechanism described in \secref{sec:forward}.

For example, it may be useful to have a flag |\version|
which can be set to |draft| or |final|.
The document source will contain some conditional code
depending on the value of |\version|.
Suppose further, the flag should default to |final| for the main file
and to |draft| for child files
which is a natural assignment for editing the document.
This is achieved by placing the following code
in the preamble of the main document
(below the |\childdocmain| directive):
%
\begin{center}
\begin{tabular}{l}
|\ifchilddoc|\\
|\providecommand{\version}{draft}|\\
|\||else|\\
|\providecommand{\version}{final}|\\
|\||fi|
\end{tabular}
\end{center}
%
The definition by |\providecommand| makes sure
that previous definitions are not overwritten.
Further statements |\providecommand{\version}{...}|
can thus be added before the above code to override it.

For the main file, one might add a line
(between |\childdocmain| and the above block)
%
\begin{center}
|%\ifchilddoc\||else\providecommand{\version}{draft}\||fi|
\end{center}
%
which can be uncommented to produce a draft version.
Likewise one can add a line to the very top of a child file
(above the |\childdocof{|\textit{main}|}| directive)
%
\begin{center}
|%\providecommand{\version}{final}|
\end{center}
%
which can be uncommented to produce the final version of this child document.

%%%%%%%%%%%%%%%%%%%%%%%%%%%%%%%%%%%%%%%%%%%%%%%%%%%%%%%%%%%%%%%%%%%%%%%%%%%%%%%%
\subsection{Forwarding}
\label{sec:forward}

Different versions of the main or child documents
using compilation flags as described in \secref{sec:flags}
can be (permanently) stored in different files
for convenient compilation, viewing and distribution.
To this end, the package defines a command
to pass on compilation to a different file:

%%%%%%%%%%%%%%%%%%%%%%%%%%%%%%%%%%%%%%%%
\DescribeMacro{\childdocforward}
The command |\childdocforward| redirects processing to
another source file:
%
\begin{center}
\begin{tabular}{l}
|% \iffalse
%
% childdoc.dtx Copyright (C) 2017-2018 Niklas Beisert
%
% This work may be distributed and/or modified under the
% conditions of the LaTeX Project Public License, either version 1.3
% of this license or (at your option) any later version.
% The latest version of this license is in
%   http://www.latex-project.org/lppl.txt
% and version 1.3 or later is part of all distributions of LaTeX
% version 2005/12/01 or later.
%
% This work has the LPPL maintenance status `maintained'.
%
% The Current Maintainer of this work is Niklas Beisert.
%
% This work consists of the files childdoc.dtx and childdoc.ins
% and the derived files childdoc.def and cdocsamp.tex with
% cdocsch1.tex, cdocsch2.tex, cdocsdrf.tex, cdocsfn1.tex, cdocsfn2.tex.
%
%<package>\ifdefined\childdocmain\endinput\fi
%<package>\ProvidesFile{childdoc.def}[2018/12/30 v2.0 child document driver]
%<samplemain>\ProvidesFile{cdocsamp.tex}[2018/12/30 v2.0 sample for childdoc]
%<*driver>
%\ProvidesFile{childdoc.drv}[2018/12/30 v2.0 childdoc reference manual file]
\PassOptionsToClass{10pt,a4paper}{article}
\documentclass{ltxdoc}

\usepackage[margin=35mm]{geometry}
\usepackage{hyperref}
\usepackage{hyperxmp}
\usepackage[usenames]{color}

\hypersetup{colorlinks=true}
\hypersetup{pdfstartview=FitH}
\hypersetup{pdfpagemode=UseNone}
\hypersetup{pdfsource={}}
\hypersetup{pdflang={en-UK}}
\hypersetup{pdfcopyright={Copyright 2017-2018 Niklas Beisert.
  This work may be distributed and/or modified under the
  conditions of the LaTeX Project Public License, either version 1.3
  of this license or (at your option) any later version.}}
\hypersetup{pdflicenseurl={http://www.latex-project.org/lppl.txt}}
\hypersetup{pdfcontactaddress={ETH Zurich, ITP, HIT K,
  Wolfgang-Pauli-Strasse 27}}
\hypersetup{pdfcontactpostcode={8093}}
\hypersetup{pdfcontactcity={Zurich}}
\hypersetup{pdfcontactcountry={Switzerland}}
\hypersetup{pdfcontactemail={nbeisert@itp.phys.ethz.ch}}
\hypersetup{pdfcontacturl={http://people.phys.ethz.ch/\xmptilde nbeisert/}}

\newcommand{\secref}[1]{\hyperref[#1]{section \ref*{#1}}}

\parskip1ex
\parindent0pt
\let\olditemize\itemize
\def\itemize{\olditemize\parskip0pt}

\begin{document}

\title{The \textsf{childdoc} Package}
\hypersetup{pdftitle={The childdoc Package}}
\author{Niklas Beisert\\[2ex]
  Institut f\"ur Theoretische Physik\\
  Eidgen\"ossische Technische Hochschule Z\"urich\\
  Wolfgang-Pauli-Strasse 27, 8093 Z\"urich, Switzerland\\[1ex]
  \href{mailto:nbeisert@itp.phys.ethz.ch}
  {\texttt{nbeisert@itp.phys.ethz.ch}}}
\hypersetup{pdfauthor={Niklas Beisert}}
\hypersetup{pdfsubject={Manual for the LaTeX2e Package childdoc}}
\date{30 December 2018, \textsf{v2.0}}
\maketitle

\begin{abstract}\noindent
\textsf{childdoc} is a \LaTeXe{} package
that enables the direct compilation
of document sections included by |\include|
to individual files.
\end{abstract}

\begingroup
\parskip0ex
\tableofcontents
\endgroup

%%%%%%%%%%%%%%%%%%%%%%%%%%%%%%%%%%%%%%%%%%%%%%%%%%%%%%%%%%%%%%%%%%%%%%%%%%%%%%%%
%%%%%%%%%%%%%%%%%%%%%%%%%%%%%%%%%%%%%%%%%%%%%%%%%%%%%%%%%%%%%%%%%%%%%%%%%%%%%%%%
\section{Introduction}

\LaTeX{} provides a mechanism to structure a large document (such as a book)
into a main file and several child files (containing the chapters)
using the |\include| command.
This mechanism is beneficial for documents
which span hundreds of pages in order to
make the source file(s) more manageable.
Moreover, compilation can be restricted to
selected child files by means of the |\includeonly| command.
The latter feature can be used to reduce the compilation time while editing
(this was significantly more useful in the earlier days of \LaTeX{})
or to generate a smaller document which is easier to navigate.
Another application of |\includeonly| is to generate
documents consisting of selected parts of the complete document.

However, there are a few drawbacks of the plain |\include| mechanism:
\begin{itemize}
\item
The child files cannot be compiled on their own,
they can only be compiled via the main file.
A naive editing environment
(such as a text editor with an option
to have the current file processed by \LaTeX)
may require one to switch to the main file before compiling;
attempting to compile the child file produces errors.
\item
The main file must be modified (each time)
to adjust the |\includeonly| command
to the present needs. This easily leaves the main file in a messy state.
\item
The generated document will always carry the filename
of the main document. This is inconvenient if
several child files are to be compiled and
to be kept for distribution.
\end{itemize}

The present package provides a simple interface
to make child files individually compilable by \LaTeX{}.
Compiling a child file then has the same effect as compiling
the main file with an |\includeonly| command
to select the appropriate child.
Moreover the generated document will carry the name of the child
rather than the main file.
This resolves all three above issues.

This feature is meant to make the editing of books,
thesis documents and lecture notes somewhat more convenient.
However, the package can also be used efficiently for
composing a series of documents (such as exercise sheets)
which are typically distributed individually.
It then assists the author in generating the individual documents
(potentially in different versions)
as well as a document containing the collected series.
Another application is in developing style files
or other kinds of included material
where compilation of the style file could redirect
to a sample or test file.

%%%%%%%%%%%%%%%%%%%%%%%%%%%%%%%%%%%%%%%%%%%%%%%%%%%%%%%%%%%%%%%%%%%%%%%%%%%%%%%%
%%%%%%%%%%%%%%%%%%%%%%%%%%%%%%%%%%%%%%%%%%%%%%%%%%%%%%%%%%%%%%%%%%%%%%%%%%%%%%%%
\section{Usage}

First of all, the package \textsf{childdoc} is \emph{not} a standard
\LaTeXe{} |.sty| style file! Therefore it needs to be invoked in
a non-standard way.

%%%%%%%%%%%%%%%%%%%%%%%%%%%%%%%%%%%%%%%%%%%%%%%%%%%%%%%%%%%%%%%%%%%%%%%%%%%%%%%%
\subsection{Included Files}
\label{sec:include}

%%%%%%%%%%%%%%%%%%%%%%%%%%%%%%%%%%%%%%%%
\DescribeMacro{\childdocmain}
To use the package, add the commands
\begin{center}
\begin{tabular}{l}
|\input{childdoc.def}|\\
|\childdocmain{}|\\
\end{tabular}
\end{center}
at the very top of the main \LaTeX{} file,
in particular \emph{before} the |\documentclass| statement!
The argument of |\childdocmain| should be left empty
(but it must be present).

%%%%%%%%%%%%%%%%%%%%%%%%%%%%%%%%%%%%%%%%
\DescribeMacro{\childdocof}
Furthermore, add the commands
\begin{center}
\begin{tabular}{l}
|\input{childdoc.def}|\\
|\childdocof{|\textit{main}|}|\\
\end{tabular}
\end{center}
at the top of every child file \textit{child}
which is included by |\include{|\textit{child}|}|
from within the main file
(or at least for those files to be compiled individually).
The argument \textit{main} must be the filename of the main file.

There are a couple of
considerations in setting up the main and child documents:

%%%%%%%%%%%%%%%%%%%%%%%%%%%%%%%%%%%%%%%%
\paragraph{Restrictions.}

Please note the following restrictions:
\begin{itemize}
\item
|\childdocmain| must be called with one argument \textit{main}
to ensure compatibility with earlier version of the package.
It must either be empty (|\childdocmain{}|)
or precisely match the filename of the main file in which it is specified.
See \secref{sec:detection} for further information.
\item
The filename \textit{main} must be specified without the |.tex| extension.
\item
The filename \textit{main} is case sensitive
(even in case-insensitive file systems)
due to internal string comparison.
\item
The argument \textit{main} should be fully expanded, it cannot be a macro.
\item
Subdirectories and special characters should be avoided in filenames.
\item
The command |\childdocmain{|\textit{main}|}| must be followed by a whitespace.
It should not be followed immediately by another command
or by a comment mark `|%|'.
This is because the \TeX{} parser reads the token immediately following
the argument of |\childdocmain| and puts it
at the beginning of every child section;
however, a white\-space is ignored.
\end{itemize}

%%%%%%%%%%%%%%%%%%%%%%%%%%%%%%%%%%%%%%%%
\paragraph{Content of Main File.}

It is advisable to place all content in the child files included by |\include|.
Any output contained in the main file will appear in all child documents
unless suppressed manually;
it cannot be suppressed automatically by the |\includeonly| directive
and thus should normally be avoided.
A method to include some content in the main file
by means of conditional processing is described in \secref{sec:conditional}.

%%%%%%%%%%%%%%%%%%%%%%%%%%%%%%%%%%%%%%%%
\paragraph{Page Numbering.}

When only a part of the document is compiled,
the appropriate numbering of pages
(as well as other status parameters)
is determined from the |.aux| files.
The latter contain information from previous passes.
However this information needs to propagate through
all intermediate child documents.
Therefore the page numbering in child documents may well
be inconsistent until the complete document is compiled at least once.

A useful (if unconventional) way to always ensure a consistent
page numbering is to restart the numbering in each child document
and denote the pages by `\textit{child}|.|\textit{page}'
where \textit{child} represents the chapter/section number of the child file.
This can be achieved by the command
|\numberwithin{page}{|\textit{child}|}|
of the \textsf{amsmath} package
where \textit{child} can be |chapter| or |section|
depending on the chosen structuring.
Alternatively, one can modify the macro |\thepage| appropriately
and reset the counter |page| at the start of each child file.

%%%%%%%%%%%%%%%%%%%%%%%%%%%%%%%%%%%%%%%%%%%%%%%%%%%%%%%%%%%%%%%%%%%%%%%%%%%%%%%%
\subsection{Conditional Processing}
\label{sec:conditional}

The package provides a mechanism to compile different versions
of a document. To customise the versions further some conditional processing
can come in handy to distinguish which version is being compiled.
The package provides two macros to describe the compilation context:

%%%%%%%%%%%%%%%%%%%%%%%%%%%%%%%%%%%%%%%%
\DescribeMacro{\ifchilddoc}
The conditional |\ifchilddoc| distinguishes between the compilation of
child documents and the main document:
%
\begin{center}
|\ifchilddoc |\textit{child-code}| |[|\||else |\textit{main-code}]| \||fi|
\end{center}

%%%%%%%%%%%%%%%%%%%%%%%%%%%%%%%%%%%%%%%%
\DescribeMacro{\childdocname}
\DescribeMacro{\childdocjob}
The macro |\childdocname| contains the filename (without extension)
of the main or child file being processed.
Note that |\childdocjob| will always contain the name of the main file.

%%%%%%%%%%%%%%%%%%%%%%%%%%%%%%%%%%%%%%%%
\paragraph{Title Page.}

Conditional processing can be used to include a title or banner page
in the main document when proper precautions are taken.
Importantly, the code in the main file should ensure that the page counter
(as well as other status parameters which are stored in the |.aux| files)
takes the same value after the conditional processing.
Otherwise the page numbers may take divergent values
depending on which part is compiled.

For example, a title page could be declared by:
%
\begin{center}
\begin{tabular}{l}
|\ifchilddoc\||else|\\
|\addtocounter{page}{-1}|\\
\textit{code for title page}\\
|\newpage|\\
|\||fi|
\end{tabular}
\end{center}
%
A banner page for the child documents can be generated by:
%
\begin{center}
\begin{tabular}{l}
|\ifchilddoc|\\
|\addtocounter{page}{-1}|\\
\textit{code for banner page}\\
|\newpage|\\
|\||fi|
\end{tabular}
\end{center}
%
Here one could write a message such as:
\begin{center}
|This is the part \childdocname{} of \childdocjob{}.|
\end{center}

%%%%%%%%%%%%%%%%%%%%%%%%%%%%%%%%%%%%%%%%%%%%%%%%%%%%%%%%%%%%%%%%%%%%%%%%%%%%%%%%
\subsection{Flags}
\label{sec:flags}

The package makes it easy to generate different versions
of the main or child documents.
To this end compilation flags can be defined
and assigned different default values.
They will be particularly useful in conjunction
with the forwarding mechanism described in \secref{sec:forward}.

For example, it may be useful to have a flag |\version|
which can be set to |draft| or |final|.
The document source will contain some conditional code
depending on the value of |\version|.
Suppose further, the flag should default to |final| for the main file
and to |draft| for child files
which is a natural assignment for editing the document.
This is achieved by placing the following code
in the preamble of the main document
(below the |\childdocmain| directive):
%
\begin{center}
\begin{tabular}{l}
|\ifchilddoc|\\
|\providecommand{\version}{draft}|\\
|\||else|\\
|\providecommand{\version}{final}|\\
|\||fi|
\end{tabular}
\end{center}
%
The definition by |\providecommand| makes sure
that previous definitions are not overwritten.
Further statements |\providecommand{\version}{...}|
can thus be added before the above code to override it.

For the main file, one might add a line
(between |\childdocmain| and the above block)
%
\begin{center}
|%\ifchilddoc\||else\providecommand{\version}{draft}\||fi|
\end{center}
%
which can be uncommented to produce a draft version.
Likewise one can add a line to the very top of a child file
(above the |\childdocof{|\textit{main}|}| directive)
%
\begin{center}
|%\providecommand{\version}{final}|
\end{center}
%
which can be uncommented to produce the final version of this child document.

%%%%%%%%%%%%%%%%%%%%%%%%%%%%%%%%%%%%%%%%%%%%%%%%%%%%%%%%%%%%%%%%%%%%%%%%%%%%%%%%
\subsection{Forwarding}
\label{sec:forward}

Different versions of the main or child documents
using compilation flags as described in \secref{sec:flags}
can be (permanently) stored in different files
for convenient compilation, viewing and distribution.
To this end, the package defines a command
to pass on compilation to a different file:

%%%%%%%%%%%%%%%%%%%%%%%%%%%%%%%%%%%%%%%%
\DescribeMacro{\childdocforward}
The command |\childdocforward| redirects processing to
another source file:
%
\begin{center}
\begin{tabular}{l}
|\input{childdoc.def}|\\
|\childdocforward[|\textit{main}|]{|\textit{dest}|}|\\
\end{tabular}
\end{center}
%
The argument \textit{dest} is the destination file
(without extension).
It should be the main file or one of the child files.
Note that further \textsf{childdoc} directives
such as |\childdocof| and |\childdocforward|
in the indicated file will be processed in this form.
The optional argument \textit{main}
passes on directly to the main file \textit{main}
while pretending to compile the child \textit{dest}.
This form behaves as if \textit{dest}
issues |\childdocof{|\textit{main}|}| right away,
and no further \textsf{childdoc} directives will be processed.

%%%%%%%%%%%%%%%%%%%%%%%%%%%%%%%%%%%%%%%%
\DescribeMacro{\...prefix}
In the alternative form |\childdocforwardprefix|,
%
\begin{center}
\begin{tabular}{l}
|\input{childdoc.def}|\\
|\childdocforwardprefix[|\textit{main}|]{|\textit{prefix}|}{|\textit{dest}|}|
\end{tabular}
\end{center}
%
the destination file is determined by a pattern
depending on the current file:
To make this work, the current file must be called
`{\textit{prefix}\hspace{0.2em}\textit{suffix}}'
with \textit{prefix} matching precisely the argument.
Processing is then passed on to the file
`{\textit{dest}\hspace{0.2em}\textit{suffix}}'.
Surely, the same effect is achieved by
directly specifying the
argument `{\textit{dest}\hspace{0.2em}\textit{suffix}}'
in the first form.
However, that requires to set up a different file
for each child. With the alternative form of the command
all these files can have exactly the same content
which simplifies setting them up and maintaining them.

For example, the following file |draft.tex|
with a compilation flag |\version| as described in \secref{sec:flags}
compiles the main document as a draft:
%
\begin{center}
\begin{tabular}{l}
|\def\version{draft}|\\
|\input{childdoc.def}|\\
|\childdocforward{|\textit{main}|}|
\end{tabular}
\end{center}
%
Likewise, the following files |final|\textit{nn}|.tex|
compile the final version of the child document
|child|\textit{nn}|.tex|:
%
\begin{center}
\begin{tabular}{l}
|\def\version{final}|\\
|\input{childdoc.def}|\\
|\childdocforwardprefix{final}{child}|
\end{tabular}
\end{center}
%

Note that when several versions of a main file and/or of each child file
are to be generated, it may be convenient to set up a |Makefile| or
shell script to automatise the process.

%%%%%%%%%%%%%%%%%%%%%%%%%%%%%%%%%%%%%%%%%%%%%%%%%%%%%%%%%%%%%%%%%%%%%%%%%%%%%%%%
\subsection{Command Line Processing}
\label{sec:commandline}

The effect of redirection files can also be achieved by invoking
the \LaTeX{} compiler with a more elaborate command line.
Most conveniently this should be done as part
of a shell script or a |Makefile|.

When using \textsf{childdoc} in the main file, the following
command lines effectively perform a redirection
(note that depending on the shell being used,
backslashes may have to be doubled: `|\|' $\to$ `|\\|'):
%
\begin{center}
|... -jobname "|\textit{target}|" |\\|"|[\textit{flags}]%
|\input{childdoc.def}\childdocforward[|\textit{main}|]{|\textit{dest}|}"|
\end{center}
%
Here \textit{target} is the name of the output file,
\textit{main} is the name of the main file
and \textit{dest} is the name of the main or child file to be processed
(all filenames without extensions).
The optional argument \textit{main} can be omitted
if \textit{main} matches \textit{dest}.
Optionally, compilation \textit{flags} can be defined via |\def| commands.
This command line makes the \TeX{} engine believe
it is compiling the file \textit{target}
whose content is specified as the latter parameter.
The provided code then forwards the processing to
\textit{main} or \textit{dest} as described in \secref{sec:forward}.

%%%%%%%%%%%%%%%%%%%%%%%%%%%%%%%%%%%%%%%%%%%%%%%%%%%%%%%%%%%%%%%%%%%%%%%%%%%%%%%%
\subsection{Include by Input}
\label{sec:input}

Including child documents by |\include| has some restrictions by design.
Most notably, the content of a child document always occupies
its own set of pages; pages cannot be shared between child documents.
Usually, this behaviour makes perfect sense
because each child document contain an essential part of the document.
However, in some situations it may be desirable to compose
a document from a collection of parts
without having mandatory page breaks between then.
For this case, the package
provides a mechanism to include parts
by |\input| which can also be processed individually.
However, by construction this mechanism
requires manual handling of the content to be output.

%%%%%%%%%%%%%%%%%%%%%%%%%%%%%%%%%%%%%%%%
\DescribeMacro{\ifchilddocmanual}
The main file should be prepared as usual, see \secref{sec:include}.
However, the document body must make a distinction
between processing of an individual part and of the main document, e.g.:
%
\begin{center}
\begin{tabular}{l}
|\ifchilddocmanual|\\
|\input{\childdocname}|\\
|\||else|\\
\textit{document body with }|\input{|\textit{part}|}|\\
|\||fi|
\end{tabular}
\end{center}
%
The conditional |\ifchilddocmanual| is true whenever
a part to be included by |\input| is being compiled,
and the name of the part is stored in |\childdocname|.

%%%%%%%%%%%%%%%%%%%%%%%%%%%%%%%%%%%%%%%%
\DescribeMacro{\childdocby}
Each part to be included by |\input| should start with:
%
\begin{center}
\begin{tabular}{l}
|\input{childdoc.def}|\\
|\childdocby{|\textit{main}|}|\\
\end{tabular}
\end{center}
%
The directive |\childdocby| is similar to |\childdocof|
described in \secref{sec:include},
but the subsequent selection of content must be done manually.
To that end, both |\ifchilddoc| and |\ifchilddocmanual|
will be true upon processing of a part,
and the name of the part is stored in |\childdocname|.
Note that |\jobname| will be set to the filename of the current part
so that each part receives an individual |.aux| file
that does not interfere with the |.aux| file(s) of the main document.
This behaviour can be altered by the alternative form
|\childdocby[*]{|\textit{main}|}| (with a non-empty optional argument)
which uses the |.aux| file of the main document
by setting |\jobname| to \textit{main}.

%%%%%%%%%%%%%%%%%%%%%%%%%%%%%%%%%%%%%%%%%%%%%%%%%%%%%%%%%%%%%%%%%%%%%%%%%%%%%%%%
\subsection{Driver Development}
\label{sec:driver}

The \textsf{childdoc} mechanism can also be use for the development
of definition files such as \LaTeX{} styles or classes.
This case differs from the above setup with multiple parts
included by |\include| in that no |\includeonly| should be invoked.
This can be achieved by starting the include file
(before |\ProvidesPackage|) with:
%
\begin{center}
\begin{tabular}{l}
|\input{childdoc.def}|\\
|\childdocforward{|\textit{main}|}|\\
\end{tabular}
\end{center}
%
or alternatively with:
%
\begin{center}
\begin{tabular}{l}
|\input{childdoc.def}|\\
|\childdocby{|\textit{main}|}|\\
\end{tabular}
\end{center}
%
Both forms have slightly different effects as described above.
The main file is prepared as usual, see \secref{sec:include}.

%%%%%%%%%%%%%%%%%%%%%%%%%%%%%%%%%%%%%%%%%%%%%%%%%%%%%%%%%%%%%%%%%%%%%%%%%%%%%%%%
\subsection{Legacy Detection}
\label{sec:detection}

The directive |\childdocmain| in the main file can detect
whether the complete document or merely a child is to be compiled
even without using the directive |\childdocof|.
This method is deprecated because it is less robust
and there is no compelling reason to use it;
it is merely provided for backward compatibility
and it may be removed in future versions.

If the detection mechanism is to be used,
it is mandatory to correctly specify
the filename of the main file as the argument of |\childdocmain|:
%
\begin{center}
\begin{tabular}{l}
|\input{childdoc.def}|\\
|\childdocmain{|\textit{main}|}|\\
\end{tabular}
\end{center}
%
If |\jobname| does not match the argument \textit{main} of |\childdocmain|,
it is assumed that |\jobname| points to the child file to be compiled.
When using |\childdocmain| with the main file specified as argument,
it suffices to start a child file
with just |\input{|\textit{main}|}|
without loading of the package and using |\childdocof|.
If instead all processing is done
with the appropriate \textsf{childdoc} directives,
the argument of \textit{main} of |\childdocmain| can be empty.

An alternative version of the command line processing described
in \secref{sec:commandline} using the detection mechanism reads:
%
\begin{center}
|... -jobname "|\textit{target}|" "|[\textit{flags}]%
[|\def\jobname{|\textit{dest}|}|]|\input{|\textit{main}|}"|
\end{center}

%%%%%%%%%%%%%%%%%%%%%%%%%%%%%%%%%%%%%%%%%%%%%%%%%%%%%%%%%%%%%%%%%%%%%%%%%%%%%%%%
\subsection{Manual Code}
\label{sec:manual}

In case one cannot be certain whether the definitions file |childdoc.def|
is installed on the target \TeX{} distribution
and one prefers not to ship it,
it is conceivable to paste a few relevant commands into the sources.

To that end, drop all statements |\input{childdoc.def}|
and perform the replacements as outlined below.
Instead of |\childdocmain{|\textit{main}|}| add the following code
to the top of the main file:
%
\begin{center}
\begin{tabular}{l}
|\||ifdefined\childdocname\endinput\||fi\newif\ifchilddoc|\\
|\edef\childdocname{\scantokens\expandafter{\jobname\noexpand}}|\\
|\def\childdocmain{|\textit{main}|}\||ifx\childdocmain\childdocname\||else|\\
|\childdoctrue\includeonly{\childdocname}\let\jobname\childdocmain\||fi|\\
\end{tabular}
\end{center}
%
Instead of |\childdocof{|\textit{main}|}| just include the main file
at the top of each child file:
%
\begin{center}
|\input{|\textit{main}|}|
\end{center}
%
A simple redirection |\childdocforward{|\textit{dest}|}| is achieved by:
%
\begin{center}
|\def\jobname{|\textit{dest}|}\input{\jobname}|
\end{center}
%
The redirection with prefix
|\childdocforwardprefix[|\textit{prefix}|]{|\textit{dest}|}|
is accomplished by:
%
\begin{center}
\begin{tabular}{l}
|{\edef\jobname{\scantokens\expandafter{\jobname\noexpand}}|\\
|\def\redirectjob |\textit{prefix}|#1~~~{\gdef\jobname{|\textit{dest}|#1}}|\\
|\expandafter\redirectjob\jobname~~~}\input{\jobname}|
\end{tabular}
\end{center}

In an alternative approach,
child documents can be compiled by a specific command line
without additional code or specific definitions:
%
\begin{center}
|... -jobname "|\textit{target}|" "|[\textit{flags}]%
|\includeonly{|\textit{dest}|}\input{|\textit{main}|}"|
\end{center}
%

%%%%%%%%%%%%%%%%%%%%%%%%%%%%%%%%%%%%%%%%%%%%%%%%%%%%%%%%%%%%%%%%%%%%%%%%%%%%%%%%
%%%%%%%%%%%%%%%%%%%%%%%%%%%%%%%%%%%%%%%%%%%%%%%%%%%%%%%%%%%%%%%%%%%%%%%%%%%%%%%%
\section{Information}

%%%%%%%%%%%%%%%%%%%%%%%%%%%%%%%%%%%%%%%%%%%%%%%%%%%%%%%%%%%%%%%%%%%%%%%%%%%%%%%%
\subsection{Copyright}

Copyright \copyright{} 2017--2018 Niklas Beisert

This work may be distributed and/or modified under the
conditions of the \LaTeX{} Project Public License, either version 1.3
of this license or (at your option) any later version.
The latest version of this license is in
  \url{http://www.latex-project.org/lppl.txt}
and version 1.3 or later is part of all distributions of \LaTeX{}
version 2005/12/01 or later.

This work has the LPPL maintenance status `maintained'.

The Current Maintainer of this work is Niklas Beisert.

This work consists of the files |README.txt|, |childdoc.ins| and |childdoc.dtx|
as well as the derived files |childdoc.def|, |cdocsamp.tex|
with |cdocsch1.tex|, |cdocsch2.tex|, |cdocspt3.tex|, |cdocspt4.tex|,
|cdocsdrf.tex|, |cdocsfn1.tex|, |cdocsfn2.tex|
as well as |childdoc.pdf|.

%%%%%%%%%%%%%%%%%%%%%%%%%%%%%%%%%%%%%%%%%%%%%%%%%%%%%%%%%%%%%%%%%%%%%%%%%%%%%%%%
\subsection{Files and Installation}

The package consists of the files:
%
\begin{center}
\begin{tabular}{ll}
    |README.txt|   & readme file \\
    |childdoc.ins| & installation file \\
    |childdoc.dtx| & source file \\
    |childdoc.def| & definition file \\
    |cdocsamp.tex| & sample main file \\
    |cdocsch1.tex| & sample include file \\
    |cdocsch2.tex| & sample include file \\
    |cdocspt3.tex| & sample part file \\
    |cdocspt4.tex| & sample part file \\
    |cdocsdrf.tex| & sample redirection file \\
    |cdocsfn1.tex| & sample redirection file \\
    |cdocsfn2.tex| & sample redirection file \\
    |childdoc.pdf| & manual
\end{tabular}
\end{center}
%
The distribution consists of the files
|README.txt|, |childdoc.ins| and |childdoc.dtx|.
%
\begin{itemize}
\item
Run (pdf)\LaTeX{} on |childdoc.dtx|
to compile the manual |childdoc.pdf| (this file).
\item
Run \LaTeX{} on |childdoc.ins| to create the definitions file |childdoc.def|
and the sample |cdocsamp.tex| with include files
|cdocsch1.tex|, |cdocsch2.tex|, |cdocspt3.tex|, |cdocspt4.tex|,
|cdocsdrf.tex|, |cdocsfn1.tex|, |cdocsfn2.tex|.
Then copy the file |childdoc.def| to an appropriate directory of your \LaTeX{}
distribution, e.g.\ \textit{texmf-root}|/tex/latex/childdoc|.
\end{itemize}

%%%%%%%%%%%%%%%%%%%%%%%%%%%%%%%%%%%%%%%%%%%%%%%%%%%%%%%%%%%%%%%%%%%%%%%%%%%%%%%%
\subsection{Related CTAN Packages}

There are several other packages which offer a similar functionality:
%
\begin{itemize}
\item
The packages
\href{http://ctan.org/pkg/docmute}{\textsf{docmute}},
\href{http://ctan.org/pkg/includex}{\textsf{includex}} and
\href{http://ctan.org/pkg/standalone}{\textsf{standalone}}
provide commands to include only the document body of
a child file thus allowing both files to be compiled individually.
\item
The packages \href{http://ctan.org/pkg/subdocs}{\textsf{subdocs}}
and \href{http://ctan.org/pkg/subfiles}{\textsf{subfiles}}
provide structures in which the main and child documents can be
encapsulated and allowing them to be compiled individually.
The inclusion mechanism is different from the conventional |\include|.
\item
The package \href{http://ctan.org/pkg/combine}{\textsf{combine}}
is an elaborate solution to combine several documents into one.
\end{itemize}
%
See also the CTAN topic \href{http://ctan.org/topic/subdocs}{\textsf{subdocs}}
for further related packages.
The present package differs from the above solutions in that
a document structure constructed with the conventional |\include| mechanism
just needs two extra commands at the top of every file
such that all constituent files can be compiled individually.

%%%%%%%%%%%%%%%%%%%%%%%%%%%%%%%%%%%%%%%%%%%%%%%%%%%%%%%%%%%%%%%%%%%%%%%%%%%%%%%%
%\subsection{Feature Suggestions}
%
%The following is a list of features which may be useful for future
%versions of this package:
%%
%\begin{itemize}
%\item
%\ldots
%\end{itemize}

%%%%%%%%%%%%%%%%%%%%%%%%%%%%%%%%%%%%%%%%%%%%%%%%%%%%%%%%%%%%%%%%%%%%%%%%%%%%%%%%
\subsection{Revision History}

%%%%%%%%%%%%%%%%%%%%%%%%%%%%%%%%%%%%%%%%
\paragraph{v2.0:} 2018/12/30

\begin{itemize}
\item
immediate forward processing
\item
added |\childdocby| mechanism
\item
manual restructured
\end{itemize}

%%%%%%%%%%%%%%%%%%%%%%%%%%%%%%%%%%%%%%%%
\paragraph{v1.6:} 2018/01/17

\begin{itemize}
\item
application for development of include files
\item
corrections to manual
\end{itemize}

%%%%%%%%%%%%%%%%%%%%%%%%%%%%%%%%%%%%%%%%
\paragraph{v1.5:} 2017/05/21

\begin{itemize}
\item
more complete structuring introduced
\item
|\childdocof| introduced
\item
|\childdoc| renamed to |\childdocmain|
\item
|\childredirect| renamed to |\childdocforward| and |\childdocforwardprefix|
and functionality expanded
\end{itemize}

%%%%%%%%%%%%%%%%%%%%%%%%%%%%%%%%%%%%%%%%
\paragraph{v1.0:} 2017/04/27

\begin{itemize}
\item
manual and install package
\item
first version published on CTAN
\end{itemize}

%%%%%%%%%%%%%%%%%%%%%%%%%%%%%%%%%%%%%%%%
\paragraph{v0.6:} 2017/04/26

\begin{itemize}
\item
redirection mechanism added
\end{itemize}

%%%%%%%%%%%%%%%%%%%%%%%%%%%%%%%%%%%%%%%%
\paragraph{v0.5:} 2017/04/26

\begin{itemize}
\item
functionality in definition file
\end{itemize}


%%%%%%%%%%%%%%%%%%%%%%%%%%%%%%%%%%%%%%%%%%%%%%%%%%%%%%%%%%%%%%%%%%%%%%%%%%%%%%%%
%%%%%%%%%%%%%%%%%%%%%%%%%%%%%%%%%%%%%%%%%%%%%%%%%%%%%%%%%%%%%%%%%%%%%%%%%%%%%%%%
%%%%%%%%%%%%%%%%%%%%%%%%%%%%%%%%%%%%%%%%%%%%%%%%%%%%%%%%%%%%%%%%%%%%%%%%%%%%%%%%
\appendix

\settowidth\MacroIndent{\rmfamily\scriptsize 000\ }

 \DocInput{childdoc.dtx}

\end{document}
%</driver>
% \fi
%
% %%%%%%%%%%%%%%%%%%%%%%%%%%%%%%%%%%%%%%%%%%%%%%%%%%%%%%%%%%%%%%%%%%%%%%%%%%%%%%
% %%%%%%%%%%%%%%%%%%%%%%%%%%%%%%%%%%%%%%%%%%%%%%%%%%%%%%%%%%%%%%%%%%%%%%%%%%%%%%
% \section{Sample}
%\iffalse
%<*samplemain>
%\fi
%
% The following presents a sample document
% with two chapters, two parts, a title page,
% a compile flag as well as three forwarding files to set the flag.
% It consists of eight |.tex| files:
% \begin{center}
% \begin{tabular}{ll}
% |cdocsamp.tex|&main file\\
% |cdocsch1.tex|&include file for chapter 1\\
% |cdocsch2.tex|&include file for chapter 2\\
% |cdocspt3.tex|&include file for part 3\\
% |cdocspt4.tex|&include file for part 4\\
% |cdocsdrf.tex|&forwarding file for main file in draft mode\\
% |cdocsfi1.tex|&forwarding file for final version of chapter 1\\
% |cdocsfi2.tex|&forwarding file for final version of chapter 2\\
% \end{tabular}
% \end{center}
% Each of the eight files can be compiled directly by the \LaTeX{} compiler.
%
% %%%%%%%%%%%%%%%%%%%%%%%%%%%%%%%%%%%%%%
% \paragraph{Main File.}
%
% The main file is called |cdocsamp.tex|.
%
% Load the \textsf{childdoc} definitions and
% declare the filename for the main document:
%    \begin{macrocode}
\input{childdoc.def}
\childdocmain{}
%    \end{macrocode}

% Optional override for |\version| flag:
%    \begin{macrocode}
%%\ifchilddoc\else\providecommand{\version}{draft}\fi
%    \end{macrocode}

% Define the default values for the |\version| flag
% (|final| for the main file and |draft| for childs):
%    \begin{macrocode}
\ifchilddoc
\providecommand{\version}{draft}
\else
\providecommand{\version}{final}
\fi
%    \end{macrocode}

% Load the standard document class:
%    \begin{macrocode}
\documentclass[12pt]{article}
%    \end{macrocode}

% Start the document body:
%    \begin{macrocode}
\begin{document}
%    \end{macrocode}

% Declare a title page.
% Print title, part of document being processed and version flag:
%    \begin{macrocode}
\addtocounter{page}{-1}
\begin{center}
{\LARGE\bfseries{}childdoc example\par}
\vspace{1cm}
\ifchilddoc
\ifchilddocmanual part\else chapter\fi:
`\childdocname' of `\childdocjob'\par
\else
main document: `\childdocjob'\par
\fi
version: \version\par
\end{center}
\newpage
%    \end{macrocode}

% Manually include selected file,
% otherwise process as usual:
%    \begin{macrocode}
\ifchilddocmanual
\section*{part `\childdocname'}
\input{\childdocname}
\else
%    \end{macrocode}

% Include the two chapters:
%    \begin{macrocode}
\include{cdocsch1}
\include{cdocsch2}
%    \end{macrocode}

% Include the two parts unless only chapters should be displayed:
%    \begin{macrocode}
\ifchilddoc\else
\section{part three}
\input{cdocspt3}
\section{part four}
\input{cdocspt4}
\fi
%    \end{macrocode}

% Process as usual until here:
%    \begin{macrocode}
\fi
%    \end{macrocode}

% End of document body:
%    \begin{macrocode}
\end{document}
%    \end{macrocode}
%\iffalse
%</samplemain>
%\fi
%
% %%%%%%%%%%%%%%%%%%%%%%%%%%%%%%%%%%%%%%
% \paragraph{Chapter Include Files.}
%
% The include files are called |cdocsch1.tex| and |cdocsch2.tex|.
%
%\iffalse
%<*samplechap1|samplechap2>
%\fi

% Optional override for |\version| flag:
%    \begin{macrocode}
%%\providecommand{\version}{final}
%    \end{macrocode}

% Include the main document:
%    \begin{macrocode}
\input{childdoc.def}
\childdocof{cdocsamp}
%    \end{macrocode}

%\iffalse
%</samplechap1|samplechap2>
%\fi
%
%\iffalse
%<*samplechap1>
%\fi
% Some text for chapter 1:
%    \begin{macrocode}
\section{one}
some text in chapter one
%    \end{macrocode}

%\iffalse
%</samplechap1>
%\fi
% Some text for chapter 2:
%\iffalse
%<*samplechap2>
%\fi
%    \begin{macrocode}
\section{two}
more text in chapter two
%    \end{macrocode}

%\iffalse
%</samplechap2>
%\fi
%
% %%%%%%%%%%%%%%%%%%%%%%%%%%%%%%%%%%%%%%
% \paragraph{Part Include Files.}
%
% The include files are called |cdocspt3.tex| and |cdocspt4.tex|.
%
%\iffalse
%<*samplepart3|samplepart4>
%\fi

% Optional override for |\version| flag:
%    \begin{macrocode}
%%\providecommand{\version}{final}
%    \end{macrocode}

% Include the main document:
%    \begin{macrocode}
\input{childdoc.def}
\childdocby{cdocsamp}
%    \end{macrocode}

%\iffalse
%</samplepart3|samplepart4>
%\fi
%
%\iffalse
%<*samplepart3>
%\fi
% Some text for part 3:
%    \begin{macrocode}
some text in part three
%    \end{macrocode}

%\iffalse
%</samplepart3>
%\fi
% Some text for part 4:
%\iffalse
%<*samplepart4>
%\fi
%    \begin{macrocode}
more text in part four
%    \end{macrocode}

%\iffalse
%</samplepart4>
%\fi
%
% %%%%%%%%%%%%%%%%%%%%%%%%%%%%%%%%%%%%%%
% \paragraph{Forwarding for a Complete Draft.}
%
% The following forwarding file |cdocsdrf.tex|
% compiles the main document in draft mode:
%\iffalse
%<*sampledraft>
%\fi
%    \begin{macrocode}
\def\version{draft}
\input{childdoc.def}
\childdocforward{cdocsamp}
%    \end{macrocode}

%\iffalse
%</sampledraft>
%\fi
%
% %%%%%%%%%%%%%%%%%%%%%%%%%%%%%%%%%%%%%%
% \paragraph{Forwarding for Final Version of the Chapters.}
%
% The following forwarding files |cdocsfn1.tex| and |cdocsfn2.tex|
% (with identical content)
% compile the final versions of the child documents
% |cdocsch1.tex| and |cdocsch2.tex|, respectively:
%\iffalse
%<*samplefinal>
%\fi
%    \begin{macrocode}
\def\version{final}
\input{childdoc.def}
\childdocforwardprefix[cdocsamp]{cdocsfn}{cdocsch}
%    \end{macrocode}

%\iffalse
%</samplefinal>
%\fi
%
% %%%%%%%%%%%%%%%%%%%%%%%%%%%%%%%%%%%%%%
% \paragraph{Command Line Processing.}
%
% The following three command lines generate the output files
% |cdocscld|, |cdocscl1| and |cdocscl2|
% which should be identical to
% |cdocsdrf|, |cdocsch1| and |cdocsfn2|, respectively:
% \begin{center}
% \begin{tabular}{l}
% |latex -jobname cdocscld \|\\
% |  "\def\version{draft}\input{childdoc.def}\childdocforward{cdocsamp}"|\\
% |latex -jobname cdocscl1 \|\\
% |  "\input{childdoc.def}\childdocforward[cdocsamp]{cdocsch1}"|\\
% |latex -jobname cdocscl2 \|\\
% |  "\def\version{final}\input{childdoc.def}\childdocforward{cdocsch2}"|
% \end{tabular}
% \end{center}
% Note that the trailing backslash on each first line
% merely continues the input to the second line
% (for convenient cut ant paste).
% Furthermore, the command |latex| can be replaced by any
% of its alternative versions such as |pdflatex|.
%
% %%%%%%%%%%%%%%%%%%%%%%%%%%%%%%%%%%%%%%%%%%%%%%%%%%%%%%%%%%%%%%%%%%%%%%%%%%%%%%
% %%%%%%%%%%%%%%%%%%%%%%%%%%%%%%%%%%%%%%%%%%%%%%%%%%%%%%%%%%%%%%%%%%%%%%%%%%%%%%
% \section{Implementation}
%\iffalse
%<*package>
%\fi
%
% This section describes the definitions file |childdoc.def|.

% The definitions cannot be loaded using |\usepackage| or |\RequirePackage|
% which has a mechanism to prevent loading a style file more than once.
% When loading the definitions by means of |\input|
% multiple instances have to be prevented manually:
%\iffalse
%This code needs to be before the `\ProvidesFile' directive
%which is defined at the beginning of this file.
%Therefore it is also placed there and commented out here.
%</package>
%<*discard>
%\fi
%    \begin{macrocode}
\ifdefined\childdocmain\endinput\fi
%    \end{macrocode}
%\iffalse
%</discard>
%<*package>
%\fi
%
% \macro{\ifchilddoc}
% \macro{\ifchilddocmanual}
% The conditional |\ifchilddoc| tells whether a
% child (true) or main (false) document is being compiled.
% The conditional |\ifchilddocmanual| tells whether
% the |\includeonly| mechanism is used (false) or
% the selection of child files must be performed manually (true).
% The definitions initialise to false:
%    \begin{macrocode}
\newif\ifchilddoc
\newif\ifchilddocmanual
%    \end{macrocode}

% \macro{\childdocname}
% \macro{\childdocjob}
% The macro |\childdocname| stores the name of the main document
% to be compiled. The macro |\childdocjob| stores the name of
% the document on which the \LaTeX{} compiler was originally invoked.
% The content of |\jobname| cannot be compared
% to filenames specified in the source due to different catcodes.
% The following code rescans |\jobname|, stores the result
% in |\childdocname| and saves a copy in |\childdocjob|:
%    \begin{macrocode}
\edef\childdocname{\scantokens\expandafter{\jobname\noexpand}}
\let\childdocjob\childdocname
%    \end{macrocode}

% \macro{\childdocdisable}
% The macro |\childdocdisable| prevents the main file
% from being processed more than once.
% At this stage, the main document command |\childdocmain|
% is assumed to be called once again where it should do nothing.
% Any subsequent call to it should prevent
% a secondary processing of the main document
% It overwrites the forwarding commands
% |\childdocof| and |\childdocforward|
% with empty macros to prevent further inclusions of the main document:
%    \begin{macrocode}
\newcommand{\childdocdisable}
{
  \renewcommand{\childdocmain}[1]{\renewcommand{\childdocmain}[1]{\endinput}}
  \renewcommand{\childdocof}[1]{}
  \renewcommand{\childdocby}[2][]{}
  \renewcommand{\childdocforward}[2][]{}
  \renewcommand{\childdocdisable}{}
}
%    \end{macrocode}

% \macro{\childdocmain}
% The macro |\childdocmain| is to be called at the top of the main file
% with nothing or the main filename (without extension) as argument.
% First, it breaks loops.
% If the argument is not empty and does not match |\childdocname|
% (which is set by the first inclusion of |childdoc.def|),
% |\ifchilddoc| is set to true, |\includeonly| is applied to the child file
% and |\jobname| is set to the main file
% (for proper handling of |.aux| files):
%    \begin{macrocode}
\newcommand{\childdocmain}[1]
{
  \childdocdisable\childdocmain{}
  \if?#1?\else
    \begingroup
      \def\childdoctmp{#1}
      \ifx\childdoctmp\childdocname
        \def\childdoctmp{}
      \else
        \def\childdoctmp
        {
          \childdoctrue
          \includeonly{\childdocname}
          \def\childdocjob{#1}
          \def\jobname{#1}
        }
      \fi
      \expandafter
    \endgroup
    \childdoctmp
  \fi
}
%    \end{macrocode}

% \macro{\childdocof}
% The command |\childdocof| redirects
% compilation to the main file |#1|.
%    \begin{macrocode}
\newcommand{\childdocof}[1]
{
  \childdocdisable
  \childdoctrue
  \includeonly{\childdocname}
  \def\jobname{#1}
  \def\childdocjob{#1}
  \input{#1}
}
%    \end{macrocode}

% \macro{\childdocby}
% The command |\childdocby| ....
%    \begin{macrocode}
\newcommand{\childdocby}[2][]
{
  \childdocdisable
  \childdoctrue
  \childdocmanualtrue
  \if?#1?\else
    \def\jobname{#2}
  \fi
  \def\childdocjob{#2}
  \input{#2}
  \endinput
}
%    \end{macrocode}

% \macro{\childdocforward}
% The command |\childdocforward| redirects
% compilation to the main file or
% (if the optional argument is given) a child file.
% Parameters are set as if the main file
% or a child file starting with |\childdocof| was compiled.
% Then compilation is handed over to the main file:
%    \begin{macrocode}
\newcommand{\childdocforward}[2][]
{
  \begingroup
    \if?#1?
      \def\childdoctmp
      {
        \def\childdocname{#2}
        \def\childdocjob{#2}
        \def\jobname{#2}
        \input{#2}
        \endinput
      }
    \else
      \def\childdoctmp
      {
        \childdocdisable
        \def\childdocname{#2}
        \childdoctrue
        \includeonly{#2}
        \def\childdocjob{#1}
        \def\jobname{#1}
        \input{#1}
        \endinput
      }
    \fi
    \expandafter
  \endgroup
  \childdoctmp
}
%    \end{macrocode}

% \macro{\childdocforwardprefix}
% The command |\childdocforwardprefix| redirects
% compilation to the main or a child file by means of a pattern.
% The prefix |#1| in the current filename is replaced by |#2|
% and the suffix of the current filename is kept
% (it is assumed that the filename does not contain the substring `|~~~|'
% which is used as a delimiter).
% Compilation is handed over to the new file by |\childdocforward|:
%    \begin{macrocode}
\newcommand{\childdocforwardprefix}[3][]
{
  \begingroup
    \def\childdocextract #2##1~~~{\def\childdoctmp{\childdocforward[#1]{#3##1}}}
    \expandafter\childdocextract\childdocname~~~
    \expandafter
  \endgroup
  \childdoctmp
}
%    \end{macrocode}

% \macro{\childdoc}
% The deprecated macro |\childdoc| is a legacy version of |\childdocmain|:
%    \begin{macrocode}
\newcommand{\childdoc}{\childdocmain}
%    \end{macrocode}

% \macro{\childdocredirect}
% The deprecated macro |\childdocredirect| is a legacy version
% of |\childdocforward| and |\childdocforwardprefix|:
%    \begin{macrocode}
\newcommand{\childdocredirect}[2][]
{
  \begingroup
    \if?#1?
      \def\childdoctmp{\childdocforward{#2}}
    \else
      \def\childdoctmp{\childdocforwardprefix{#1}{#2}}
    \fi
    \expandafter
  \endgroup
  \childdoctmp
}
%    \end{macrocode}

%\iffalse
%</package>
%\fi
%
\endinput
|\\
|\childdocforward[|\textit{main}|]{|\textit{dest}|}|\\
\end{tabular}
\end{center}
%
The argument \textit{dest} is the destination file
(without extension).
It should be the main file or one of the child files.
Note that further \textsf{childdoc} directives
such as |\childdocof| and |\childdocforward|
in the indicated file will be processed in this form.
The optional argument \textit{main}
passes on directly to the main file \textit{main}
while pretending to compile the child \textit{dest}.
This form behaves as if \textit{dest}
issues |\childdocof{|\textit{main}|}| right away,
and no further \textsf{childdoc} directives will be processed.

%%%%%%%%%%%%%%%%%%%%%%%%%%%%%%%%%%%%%%%%
\DescribeMacro{\...prefix}
In the alternative form |\childdocforwardprefix|,
%
\begin{center}
\begin{tabular}{l}
|% \iffalse
%
% childdoc.dtx Copyright (C) 2017-2018 Niklas Beisert
%
% This work may be distributed and/or modified under the
% conditions of the LaTeX Project Public License, either version 1.3
% of this license or (at your option) any later version.
% The latest version of this license is in
%   http://www.latex-project.org/lppl.txt
% and version 1.3 or later is part of all distributions of LaTeX
% version 2005/12/01 or later.
%
% This work has the LPPL maintenance status `maintained'.
%
% The Current Maintainer of this work is Niklas Beisert.
%
% This work consists of the files childdoc.dtx and childdoc.ins
% and the derived files childdoc.def and cdocsamp.tex with
% cdocsch1.tex, cdocsch2.tex, cdocsdrf.tex, cdocsfn1.tex, cdocsfn2.tex.
%
%<package>\ifdefined\childdocmain\endinput\fi
%<package>\ProvidesFile{childdoc.def}[2018/12/30 v2.0 child document driver]
%<samplemain>\ProvidesFile{cdocsamp.tex}[2018/12/30 v2.0 sample for childdoc]
%<*driver>
%\ProvidesFile{childdoc.drv}[2018/12/30 v2.0 childdoc reference manual file]
\PassOptionsToClass{10pt,a4paper}{article}
\documentclass{ltxdoc}

\usepackage[margin=35mm]{geometry}
\usepackage{hyperref}
\usepackage{hyperxmp}
\usepackage[usenames]{color}

\hypersetup{colorlinks=true}
\hypersetup{pdfstartview=FitH}
\hypersetup{pdfpagemode=UseNone}
\hypersetup{pdfsource={}}
\hypersetup{pdflang={en-UK}}
\hypersetup{pdfcopyright={Copyright 2017-2018 Niklas Beisert.
  This work may be distributed and/or modified under the
  conditions of the LaTeX Project Public License, either version 1.3
  of this license or (at your option) any later version.}}
\hypersetup{pdflicenseurl={http://www.latex-project.org/lppl.txt}}
\hypersetup{pdfcontactaddress={ETH Zurich, ITP, HIT K,
  Wolfgang-Pauli-Strasse 27}}
\hypersetup{pdfcontactpostcode={8093}}
\hypersetup{pdfcontactcity={Zurich}}
\hypersetup{pdfcontactcountry={Switzerland}}
\hypersetup{pdfcontactemail={nbeisert@itp.phys.ethz.ch}}
\hypersetup{pdfcontacturl={http://people.phys.ethz.ch/\xmptilde nbeisert/}}

\newcommand{\secref}[1]{\hyperref[#1]{section \ref*{#1}}}

\parskip1ex
\parindent0pt
\let\olditemize\itemize
\def\itemize{\olditemize\parskip0pt}

\begin{document}

\title{The \textsf{childdoc} Package}
\hypersetup{pdftitle={The childdoc Package}}
\author{Niklas Beisert\\[2ex]
  Institut f\"ur Theoretische Physik\\
  Eidgen\"ossische Technische Hochschule Z\"urich\\
  Wolfgang-Pauli-Strasse 27, 8093 Z\"urich, Switzerland\\[1ex]
  \href{mailto:nbeisert@itp.phys.ethz.ch}
  {\texttt{nbeisert@itp.phys.ethz.ch}}}
\hypersetup{pdfauthor={Niklas Beisert}}
\hypersetup{pdfsubject={Manual for the LaTeX2e Package childdoc}}
\date{30 December 2018, \textsf{v2.0}}
\maketitle

\begin{abstract}\noindent
\textsf{childdoc} is a \LaTeXe{} package
that enables the direct compilation
of document sections included by |\include|
to individual files.
\end{abstract}

\begingroup
\parskip0ex
\tableofcontents
\endgroup

%%%%%%%%%%%%%%%%%%%%%%%%%%%%%%%%%%%%%%%%%%%%%%%%%%%%%%%%%%%%%%%%%%%%%%%%%%%%%%%%
%%%%%%%%%%%%%%%%%%%%%%%%%%%%%%%%%%%%%%%%%%%%%%%%%%%%%%%%%%%%%%%%%%%%%%%%%%%%%%%%
\section{Introduction}

\LaTeX{} provides a mechanism to structure a large document (such as a book)
into a main file and several child files (containing the chapters)
using the |\include| command.
This mechanism is beneficial for documents
which span hundreds of pages in order to
make the source file(s) more manageable.
Moreover, compilation can be restricted to
selected child files by means of the |\includeonly| command.
The latter feature can be used to reduce the compilation time while editing
(this was significantly more useful in the earlier days of \LaTeX{})
or to generate a smaller document which is easier to navigate.
Another application of |\includeonly| is to generate
documents consisting of selected parts of the complete document.

However, there are a few drawbacks of the plain |\include| mechanism:
\begin{itemize}
\item
The child files cannot be compiled on their own,
they can only be compiled via the main file.
A naive editing environment
(such as a text editor with an option
to have the current file processed by \LaTeX)
may require one to switch to the main file before compiling;
attempting to compile the child file produces errors.
\item
The main file must be modified (each time)
to adjust the |\includeonly| command
to the present needs. This easily leaves the main file in a messy state.
\item
The generated document will always carry the filename
of the main document. This is inconvenient if
several child files are to be compiled and
to be kept for distribution.
\end{itemize}

The present package provides a simple interface
to make child files individually compilable by \LaTeX{}.
Compiling a child file then has the same effect as compiling
the main file with an |\includeonly| command
to select the appropriate child.
Moreover the generated document will carry the name of the child
rather than the main file.
This resolves all three above issues.

This feature is meant to make the editing of books,
thesis documents and lecture notes somewhat more convenient.
However, the package can also be used efficiently for
composing a series of documents (such as exercise sheets)
which are typically distributed individually.
It then assists the author in generating the individual documents
(potentially in different versions)
as well as a document containing the collected series.
Another application is in developing style files
or other kinds of included material
where compilation of the style file could redirect
to a sample or test file.

%%%%%%%%%%%%%%%%%%%%%%%%%%%%%%%%%%%%%%%%%%%%%%%%%%%%%%%%%%%%%%%%%%%%%%%%%%%%%%%%
%%%%%%%%%%%%%%%%%%%%%%%%%%%%%%%%%%%%%%%%%%%%%%%%%%%%%%%%%%%%%%%%%%%%%%%%%%%%%%%%
\section{Usage}

First of all, the package \textsf{childdoc} is \emph{not} a standard
\LaTeXe{} |.sty| style file! Therefore it needs to be invoked in
a non-standard way.

%%%%%%%%%%%%%%%%%%%%%%%%%%%%%%%%%%%%%%%%%%%%%%%%%%%%%%%%%%%%%%%%%%%%%%%%%%%%%%%%
\subsection{Included Files}
\label{sec:include}

%%%%%%%%%%%%%%%%%%%%%%%%%%%%%%%%%%%%%%%%
\DescribeMacro{\childdocmain}
To use the package, add the commands
\begin{center}
\begin{tabular}{l}
|\input{childdoc.def}|\\
|\childdocmain{}|\\
\end{tabular}
\end{center}
at the very top of the main \LaTeX{} file,
in particular \emph{before} the |\documentclass| statement!
The argument of |\childdocmain| should be left empty
(but it must be present).

%%%%%%%%%%%%%%%%%%%%%%%%%%%%%%%%%%%%%%%%
\DescribeMacro{\childdocof}
Furthermore, add the commands
\begin{center}
\begin{tabular}{l}
|\input{childdoc.def}|\\
|\childdocof{|\textit{main}|}|\\
\end{tabular}
\end{center}
at the top of every child file \textit{child}
which is included by |\include{|\textit{child}|}|
from within the main file
(or at least for those files to be compiled individually).
The argument \textit{main} must be the filename of the main file.

There are a couple of
considerations in setting up the main and child documents:

%%%%%%%%%%%%%%%%%%%%%%%%%%%%%%%%%%%%%%%%
\paragraph{Restrictions.}

Please note the following restrictions:
\begin{itemize}
\item
|\childdocmain| must be called with one argument \textit{main}
to ensure compatibility with earlier version of the package.
It must either be empty (|\childdocmain{}|)
or precisely match the filename of the main file in which it is specified.
See \secref{sec:detection} for further information.
\item
The filename \textit{main} must be specified without the |.tex| extension.
\item
The filename \textit{main} is case sensitive
(even in case-insensitive file systems)
due to internal string comparison.
\item
The argument \textit{main} should be fully expanded, it cannot be a macro.
\item
Subdirectories and special characters should be avoided in filenames.
\item
The command |\childdocmain{|\textit{main}|}| must be followed by a whitespace.
It should not be followed immediately by another command
or by a comment mark `|%|'.
This is because the \TeX{} parser reads the token immediately following
the argument of |\childdocmain| and puts it
at the beginning of every child section;
however, a white\-space is ignored.
\end{itemize}

%%%%%%%%%%%%%%%%%%%%%%%%%%%%%%%%%%%%%%%%
\paragraph{Content of Main File.}

It is advisable to place all content in the child files included by |\include|.
Any output contained in the main file will appear in all child documents
unless suppressed manually;
it cannot be suppressed automatically by the |\includeonly| directive
and thus should normally be avoided.
A method to include some content in the main file
by means of conditional processing is described in \secref{sec:conditional}.

%%%%%%%%%%%%%%%%%%%%%%%%%%%%%%%%%%%%%%%%
\paragraph{Page Numbering.}

When only a part of the document is compiled,
the appropriate numbering of pages
(as well as other status parameters)
is determined from the |.aux| files.
The latter contain information from previous passes.
However this information needs to propagate through
all intermediate child documents.
Therefore the page numbering in child documents may well
be inconsistent until the complete document is compiled at least once.

A useful (if unconventional) way to always ensure a consistent
page numbering is to restart the numbering in each child document
and denote the pages by `\textit{child}|.|\textit{page}'
where \textit{child} represents the chapter/section number of the child file.
This can be achieved by the command
|\numberwithin{page}{|\textit{child}|}|
of the \textsf{amsmath} package
where \textit{child} can be |chapter| or |section|
depending on the chosen structuring.
Alternatively, one can modify the macro |\thepage| appropriately
and reset the counter |page| at the start of each child file.

%%%%%%%%%%%%%%%%%%%%%%%%%%%%%%%%%%%%%%%%%%%%%%%%%%%%%%%%%%%%%%%%%%%%%%%%%%%%%%%%
\subsection{Conditional Processing}
\label{sec:conditional}

The package provides a mechanism to compile different versions
of a document. To customise the versions further some conditional processing
can come in handy to distinguish which version is being compiled.
The package provides two macros to describe the compilation context:

%%%%%%%%%%%%%%%%%%%%%%%%%%%%%%%%%%%%%%%%
\DescribeMacro{\ifchilddoc}
The conditional |\ifchilddoc| distinguishes between the compilation of
child documents and the main document:
%
\begin{center}
|\ifchilddoc |\textit{child-code}| |[|\||else |\textit{main-code}]| \||fi|
\end{center}

%%%%%%%%%%%%%%%%%%%%%%%%%%%%%%%%%%%%%%%%
\DescribeMacro{\childdocname}
\DescribeMacro{\childdocjob}
The macro |\childdocname| contains the filename (without extension)
of the main or child file being processed.
Note that |\childdocjob| will always contain the name of the main file.

%%%%%%%%%%%%%%%%%%%%%%%%%%%%%%%%%%%%%%%%
\paragraph{Title Page.}

Conditional processing can be used to include a title or banner page
in the main document when proper precautions are taken.
Importantly, the code in the main file should ensure that the page counter
(as well as other status parameters which are stored in the |.aux| files)
takes the same value after the conditional processing.
Otherwise the page numbers may take divergent values
depending on which part is compiled.

For example, a title page could be declared by:
%
\begin{center}
\begin{tabular}{l}
|\ifchilddoc\||else|\\
|\addtocounter{page}{-1}|\\
\textit{code for title page}\\
|\newpage|\\
|\||fi|
\end{tabular}
\end{center}
%
A banner page for the child documents can be generated by:
%
\begin{center}
\begin{tabular}{l}
|\ifchilddoc|\\
|\addtocounter{page}{-1}|\\
\textit{code for banner page}\\
|\newpage|\\
|\||fi|
\end{tabular}
\end{center}
%
Here one could write a message such as:
\begin{center}
|This is the part \childdocname{} of \childdocjob{}.|
\end{center}

%%%%%%%%%%%%%%%%%%%%%%%%%%%%%%%%%%%%%%%%%%%%%%%%%%%%%%%%%%%%%%%%%%%%%%%%%%%%%%%%
\subsection{Flags}
\label{sec:flags}

The package makes it easy to generate different versions
of the main or child documents.
To this end compilation flags can be defined
and assigned different default values.
They will be particularly useful in conjunction
with the forwarding mechanism described in \secref{sec:forward}.

For example, it may be useful to have a flag |\version|
which can be set to |draft| or |final|.
The document source will contain some conditional code
depending on the value of |\version|.
Suppose further, the flag should default to |final| for the main file
and to |draft| for child files
which is a natural assignment for editing the document.
This is achieved by placing the following code
in the preamble of the main document
(below the |\childdocmain| directive):
%
\begin{center}
\begin{tabular}{l}
|\ifchilddoc|\\
|\providecommand{\version}{draft}|\\
|\||else|\\
|\providecommand{\version}{final}|\\
|\||fi|
\end{tabular}
\end{center}
%
The definition by |\providecommand| makes sure
that previous definitions are not overwritten.
Further statements |\providecommand{\version}{...}|
can thus be added before the above code to override it.

For the main file, one might add a line
(between |\childdocmain| and the above block)
%
\begin{center}
|%\ifchilddoc\||else\providecommand{\version}{draft}\||fi|
\end{center}
%
which can be uncommented to produce a draft version.
Likewise one can add a line to the very top of a child file
(above the |\childdocof{|\textit{main}|}| directive)
%
\begin{center}
|%\providecommand{\version}{final}|
\end{center}
%
which can be uncommented to produce the final version of this child document.

%%%%%%%%%%%%%%%%%%%%%%%%%%%%%%%%%%%%%%%%%%%%%%%%%%%%%%%%%%%%%%%%%%%%%%%%%%%%%%%%
\subsection{Forwarding}
\label{sec:forward}

Different versions of the main or child documents
using compilation flags as described in \secref{sec:flags}
can be (permanently) stored in different files
for convenient compilation, viewing and distribution.
To this end, the package defines a command
to pass on compilation to a different file:

%%%%%%%%%%%%%%%%%%%%%%%%%%%%%%%%%%%%%%%%
\DescribeMacro{\childdocforward}
The command |\childdocforward| redirects processing to
another source file:
%
\begin{center}
\begin{tabular}{l}
|\input{childdoc.def}|\\
|\childdocforward[|\textit{main}|]{|\textit{dest}|}|\\
\end{tabular}
\end{center}
%
The argument \textit{dest} is the destination file
(without extension).
It should be the main file or one of the child files.
Note that further \textsf{childdoc} directives
such as |\childdocof| and |\childdocforward|
in the indicated file will be processed in this form.
The optional argument \textit{main}
passes on directly to the main file \textit{main}
while pretending to compile the child \textit{dest}.
This form behaves as if \textit{dest}
issues |\childdocof{|\textit{main}|}| right away,
and no further \textsf{childdoc} directives will be processed.

%%%%%%%%%%%%%%%%%%%%%%%%%%%%%%%%%%%%%%%%
\DescribeMacro{\...prefix}
In the alternative form |\childdocforwardprefix|,
%
\begin{center}
\begin{tabular}{l}
|\input{childdoc.def}|\\
|\childdocforwardprefix[|\textit{main}|]{|\textit{prefix}|}{|\textit{dest}|}|
\end{tabular}
\end{center}
%
the destination file is determined by a pattern
depending on the current file:
To make this work, the current file must be called
`{\textit{prefix}\hspace{0.2em}\textit{suffix}}'
with \textit{prefix} matching precisely the argument.
Processing is then passed on to the file
`{\textit{dest}\hspace{0.2em}\textit{suffix}}'.
Surely, the same effect is achieved by
directly specifying the
argument `{\textit{dest}\hspace{0.2em}\textit{suffix}}'
in the first form.
However, that requires to set up a different file
for each child. With the alternative form of the command
all these files can have exactly the same content
which simplifies setting them up and maintaining them.

For example, the following file |draft.tex|
with a compilation flag |\version| as described in \secref{sec:flags}
compiles the main document as a draft:
%
\begin{center}
\begin{tabular}{l}
|\def\version{draft}|\\
|\input{childdoc.def}|\\
|\childdocforward{|\textit{main}|}|
\end{tabular}
\end{center}
%
Likewise, the following files |final|\textit{nn}|.tex|
compile the final version of the child document
|child|\textit{nn}|.tex|:
%
\begin{center}
\begin{tabular}{l}
|\def\version{final}|\\
|\input{childdoc.def}|\\
|\childdocforwardprefix{final}{child}|
\end{tabular}
\end{center}
%

Note that when several versions of a main file and/or of each child file
are to be generated, it may be convenient to set up a |Makefile| or
shell script to automatise the process.

%%%%%%%%%%%%%%%%%%%%%%%%%%%%%%%%%%%%%%%%%%%%%%%%%%%%%%%%%%%%%%%%%%%%%%%%%%%%%%%%
\subsection{Command Line Processing}
\label{sec:commandline}

The effect of redirection files can also be achieved by invoking
the \LaTeX{} compiler with a more elaborate command line.
Most conveniently this should be done as part
of a shell script or a |Makefile|.

When using \textsf{childdoc} in the main file, the following
command lines effectively perform a redirection
(note that depending on the shell being used,
backslashes may have to be doubled: `|\|' $\to$ `|\\|'):
%
\begin{center}
|... -jobname "|\textit{target}|" |\\|"|[\textit{flags}]%
|\input{childdoc.def}\childdocforward[|\textit{main}|]{|\textit{dest}|}"|
\end{center}
%
Here \textit{target} is the name of the output file,
\textit{main} is the name of the main file
and \textit{dest} is the name of the main or child file to be processed
(all filenames without extensions).
The optional argument \textit{main} can be omitted
if \textit{main} matches \textit{dest}.
Optionally, compilation \textit{flags} can be defined via |\def| commands.
This command line makes the \TeX{} engine believe
it is compiling the file \textit{target}
whose content is specified as the latter parameter.
The provided code then forwards the processing to
\textit{main} or \textit{dest} as described in \secref{sec:forward}.

%%%%%%%%%%%%%%%%%%%%%%%%%%%%%%%%%%%%%%%%%%%%%%%%%%%%%%%%%%%%%%%%%%%%%%%%%%%%%%%%
\subsection{Include by Input}
\label{sec:input}

Including child documents by |\include| has some restrictions by design.
Most notably, the content of a child document always occupies
its own set of pages; pages cannot be shared between child documents.
Usually, this behaviour makes perfect sense
because each child document contain an essential part of the document.
However, in some situations it may be desirable to compose
a document from a collection of parts
without having mandatory page breaks between then.
For this case, the package
provides a mechanism to include parts
by |\input| which can also be processed individually.
However, by construction this mechanism
requires manual handling of the content to be output.

%%%%%%%%%%%%%%%%%%%%%%%%%%%%%%%%%%%%%%%%
\DescribeMacro{\ifchilddocmanual}
The main file should be prepared as usual, see \secref{sec:include}.
However, the document body must make a distinction
between processing of an individual part and of the main document, e.g.:
%
\begin{center}
\begin{tabular}{l}
|\ifchilddocmanual|\\
|\input{\childdocname}|\\
|\||else|\\
\textit{document body with }|\input{|\textit{part}|}|\\
|\||fi|
\end{tabular}
\end{center}
%
The conditional |\ifchilddocmanual| is true whenever
a part to be included by |\input| is being compiled,
and the name of the part is stored in |\childdocname|.

%%%%%%%%%%%%%%%%%%%%%%%%%%%%%%%%%%%%%%%%
\DescribeMacro{\childdocby}
Each part to be included by |\input| should start with:
%
\begin{center}
\begin{tabular}{l}
|\input{childdoc.def}|\\
|\childdocby{|\textit{main}|}|\\
\end{tabular}
\end{center}
%
The directive |\childdocby| is similar to |\childdocof|
described in \secref{sec:include},
but the subsequent selection of content must be done manually.
To that end, both |\ifchilddoc| and |\ifchilddocmanual|
will be true upon processing of a part,
and the name of the part is stored in |\childdocname|.
Note that |\jobname| will be set to the filename of the current part
so that each part receives an individual |.aux| file
that does not interfere with the |.aux| file(s) of the main document.
This behaviour can be altered by the alternative form
|\childdocby[*]{|\textit{main}|}| (with a non-empty optional argument)
which uses the |.aux| file of the main document
by setting |\jobname| to \textit{main}.

%%%%%%%%%%%%%%%%%%%%%%%%%%%%%%%%%%%%%%%%%%%%%%%%%%%%%%%%%%%%%%%%%%%%%%%%%%%%%%%%
\subsection{Driver Development}
\label{sec:driver}

The \textsf{childdoc} mechanism can also be use for the development
of definition files such as \LaTeX{} styles or classes.
This case differs from the above setup with multiple parts
included by |\include| in that no |\includeonly| should be invoked.
This can be achieved by starting the include file
(before |\ProvidesPackage|) with:
%
\begin{center}
\begin{tabular}{l}
|\input{childdoc.def}|\\
|\childdocforward{|\textit{main}|}|\\
\end{tabular}
\end{center}
%
or alternatively with:
%
\begin{center}
\begin{tabular}{l}
|\input{childdoc.def}|\\
|\childdocby{|\textit{main}|}|\\
\end{tabular}
\end{center}
%
Both forms have slightly different effects as described above.
The main file is prepared as usual, see \secref{sec:include}.

%%%%%%%%%%%%%%%%%%%%%%%%%%%%%%%%%%%%%%%%%%%%%%%%%%%%%%%%%%%%%%%%%%%%%%%%%%%%%%%%
\subsection{Legacy Detection}
\label{sec:detection}

The directive |\childdocmain| in the main file can detect
whether the complete document or merely a child is to be compiled
even without using the directive |\childdocof|.
This method is deprecated because it is less robust
and there is no compelling reason to use it;
it is merely provided for backward compatibility
and it may be removed in future versions.

If the detection mechanism is to be used,
it is mandatory to correctly specify
the filename of the main file as the argument of |\childdocmain|:
%
\begin{center}
\begin{tabular}{l}
|\input{childdoc.def}|\\
|\childdocmain{|\textit{main}|}|\\
\end{tabular}
\end{center}
%
If |\jobname| does not match the argument \textit{main} of |\childdocmain|,
it is assumed that |\jobname| points to the child file to be compiled.
When using |\childdocmain| with the main file specified as argument,
it suffices to start a child file
with just |\input{|\textit{main}|}|
without loading of the package and using |\childdocof|.
If instead all processing is done
with the appropriate \textsf{childdoc} directives,
the argument of \textit{main} of |\childdocmain| can be empty.

An alternative version of the command line processing described
in \secref{sec:commandline} using the detection mechanism reads:
%
\begin{center}
|... -jobname "|\textit{target}|" "|[\textit{flags}]%
[|\def\jobname{|\textit{dest}|}|]|\input{|\textit{main}|}"|
\end{center}

%%%%%%%%%%%%%%%%%%%%%%%%%%%%%%%%%%%%%%%%%%%%%%%%%%%%%%%%%%%%%%%%%%%%%%%%%%%%%%%%
\subsection{Manual Code}
\label{sec:manual}

In case one cannot be certain whether the definitions file |childdoc.def|
is installed on the target \TeX{} distribution
and one prefers not to ship it,
it is conceivable to paste a few relevant commands into the sources.

To that end, drop all statements |\input{childdoc.def}|
and perform the replacements as outlined below.
Instead of |\childdocmain{|\textit{main}|}| add the following code
to the top of the main file:
%
\begin{center}
\begin{tabular}{l}
|\||ifdefined\childdocname\endinput\||fi\newif\ifchilddoc|\\
|\edef\childdocname{\scantokens\expandafter{\jobname\noexpand}}|\\
|\def\childdocmain{|\textit{main}|}\||ifx\childdocmain\childdocname\||else|\\
|\childdoctrue\includeonly{\childdocname}\let\jobname\childdocmain\||fi|\\
\end{tabular}
\end{center}
%
Instead of |\childdocof{|\textit{main}|}| just include the main file
at the top of each child file:
%
\begin{center}
|\input{|\textit{main}|}|
\end{center}
%
A simple redirection |\childdocforward{|\textit{dest}|}| is achieved by:
%
\begin{center}
|\def\jobname{|\textit{dest}|}\input{\jobname}|
\end{center}
%
The redirection with prefix
|\childdocforwardprefix[|\textit{prefix}|]{|\textit{dest}|}|
is accomplished by:
%
\begin{center}
\begin{tabular}{l}
|{\edef\jobname{\scantokens\expandafter{\jobname\noexpand}}|\\
|\def\redirectjob |\textit{prefix}|#1~~~{\gdef\jobname{|\textit{dest}|#1}}|\\
|\expandafter\redirectjob\jobname~~~}\input{\jobname}|
\end{tabular}
\end{center}

In an alternative approach,
child documents can be compiled by a specific command line
without additional code or specific definitions:
%
\begin{center}
|... -jobname "|\textit{target}|" "|[\textit{flags}]%
|\includeonly{|\textit{dest}|}\input{|\textit{main}|}"|
\end{center}
%

%%%%%%%%%%%%%%%%%%%%%%%%%%%%%%%%%%%%%%%%%%%%%%%%%%%%%%%%%%%%%%%%%%%%%%%%%%%%%%%%
%%%%%%%%%%%%%%%%%%%%%%%%%%%%%%%%%%%%%%%%%%%%%%%%%%%%%%%%%%%%%%%%%%%%%%%%%%%%%%%%
\section{Information}

%%%%%%%%%%%%%%%%%%%%%%%%%%%%%%%%%%%%%%%%%%%%%%%%%%%%%%%%%%%%%%%%%%%%%%%%%%%%%%%%
\subsection{Copyright}

Copyright \copyright{} 2017--2018 Niklas Beisert

This work may be distributed and/or modified under the
conditions of the \LaTeX{} Project Public License, either version 1.3
of this license or (at your option) any later version.
The latest version of this license is in
  \url{http://www.latex-project.org/lppl.txt}
and version 1.3 or later is part of all distributions of \LaTeX{}
version 2005/12/01 or later.

This work has the LPPL maintenance status `maintained'.

The Current Maintainer of this work is Niklas Beisert.

This work consists of the files |README.txt|, |childdoc.ins| and |childdoc.dtx|
as well as the derived files |childdoc.def|, |cdocsamp.tex|
with |cdocsch1.tex|, |cdocsch2.tex|, |cdocspt3.tex|, |cdocspt4.tex|,
|cdocsdrf.tex|, |cdocsfn1.tex|, |cdocsfn2.tex|
as well as |childdoc.pdf|.

%%%%%%%%%%%%%%%%%%%%%%%%%%%%%%%%%%%%%%%%%%%%%%%%%%%%%%%%%%%%%%%%%%%%%%%%%%%%%%%%
\subsection{Files and Installation}

The package consists of the files:
%
\begin{center}
\begin{tabular}{ll}
    |README.txt|   & readme file \\
    |childdoc.ins| & installation file \\
    |childdoc.dtx| & source file \\
    |childdoc.def| & definition file \\
    |cdocsamp.tex| & sample main file \\
    |cdocsch1.tex| & sample include file \\
    |cdocsch2.tex| & sample include file \\
    |cdocspt3.tex| & sample part file \\
    |cdocspt4.tex| & sample part file \\
    |cdocsdrf.tex| & sample redirection file \\
    |cdocsfn1.tex| & sample redirection file \\
    |cdocsfn2.tex| & sample redirection file \\
    |childdoc.pdf| & manual
\end{tabular}
\end{center}
%
The distribution consists of the files
|README.txt|, |childdoc.ins| and |childdoc.dtx|.
%
\begin{itemize}
\item
Run (pdf)\LaTeX{} on |childdoc.dtx|
to compile the manual |childdoc.pdf| (this file).
\item
Run \LaTeX{} on |childdoc.ins| to create the definitions file |childdoc.def|
and the sample |cdocsamp.tex| with include files
|cdocsch1.tex|, |cdocsch2.tex|, |cdocspt3.tex|, |cdocspt4.tex|,
|cdocsdrf.tex|, |cdocsfn1.tex|, |cdocsfn2.tex|.
Then copy the file |childdoc.def| to an appropriate directory of your \LaTeX{}
distribution, e.g.\ \textit{texmf-root}|/tex/latex/childdoc|.
\end{itemize}

%%%%%%%%%%%%%%%%%%%%%%%%%%%%%%%%%%%%%%%%%%%%%%%%%%%%%%%%%%%%%%%%%%%%%%%%%%%%%%%%
\subsection{Related CTAN Packages}

There are several other packages which offer a similar functionality:
%
\begin{itemize}
\item
The packages
\href{http://ctan.org/pkg/docmute}{\textsf{docmute}},
\href{http://ctan.org/pkg/includex}{\textsf{includex}} and
\href{http://ctan.org/pkg/standalone}{\textsf{standalone}}
provide commands to include only the document body of
a child file thus allowing both files to be compiled individually.
\item
The packages \href{http://ctan.org/pkg/subdocs}{\textsf{subdocs}}
and \href{http://ctan.org/pkg/subfiles}{\textsf{subfiles}}
provide structures in which the main and child documents can be
encapsulated and allowing them to be compiled individually.
The inclusion mechanism is different from the conventional |\include|.
\item
The package \href{http://ctan.org/pkg/combine}{\textsf{combine}}
is an elaborate solution to combine several documents into one.
\end{itemize}
%
See also the CTAN topic \href{http://ctan.org/topic/subdocs}{\textsf{subdocs}}
for further related packages.
The present package differs from the above solutions in that
a document structure constructed with the conventional |\include| mechanism
just needs two extra commands at the top of every file
such that all constituent files can be compiled individually.

%%%%%%%%%%%%%%%%%%%%%%%%%%%%%%%%%%%%%%%%%%%%%%%%%%%%%%%%%%%%%%%%%%%%%%%%%%%%%%%%
%\subsection{Feature Suggestions}
%
%The following is a list of features which may be useful for future
%versions of this package:
%%
%\begin{itemize}
%\item
%\ldots
%\end{itemize}

%%%%%%%%%%%%%%%%%%%%%%%%%%%%%%%%%%%%%%%%%%%%%%%%%%%%%%%%%%%%%%%%%%%%%%%%%%%%%%%%
\subsection{Revision History}

%%%%%%%%%%%%%%%%%%%%%%%%%%%%%%%%%%%%%%%%
\paragraph{v2.0:} 2018/12/30

\begin{itemize}
\item
immediate forward processing
\item
added |\childdocby| mechanism
\item
manual restructured
\end{itemize}

%%%%%%%%%%%%%%%%%%%%%%%%%%%%%%%%%%%%%%%%
\paragraph{v1.6:} 2018/01/17

\begin{itemize}
\item
application for development of include files
\item
corrections to manual
\end{itemize}

%%%%%%%%%%%%%%%%%%%%%%%%%%%%%%%%%%%%%%%%
\paragraph{v1.5:} 2017/05/21

\begin{itemize}
\item
more complete structuring introduced
\item
|\childdocof| introduced
\item
|\childdoc| renamed to |\childdocmain|
\item
|\childredirect| renamed to |\childdocforward| and |\childdocforwardprefix|
and functionality expanded
\end{itemize}

%%%%%%%%%%%%%%%%%%%%%%%%%%%%%%%%%%%%%%%%
\paragraph{v1.0:} 2017/04/27

\begin{itemize}
\item
manual and install package
\item
first version published on CTAN
\end{itemize}

%%%%%%%%%%%%%%%%%%%%%%%%%%%%%%%%%%%%%%%%
\paragraph{v0.6:} 2017/04/26

\begin{itemize}
\item
redirection mechanism added
\end{itemize}

%%%%%%%%%%%%%%%%%%%%%%%%%%%%%%%%%%%%%%%%
\paragraph{v0.5:} 2017/04/26

\begin{itemize}
\item
functionality in definition file
\end{itemize}


%%%%%%%%%%%%%%%%%%%%%%%%%%%%%%%%%%%%%%%%%%%%%%%%%%%%%%%%%%%%%%%%%%%%%%%%%%%%%%%%
%%%%%%%%%%%%%%%%%%%%%%%%%%%%%%%%%%%%%%%%%%%%%%%%%%%%%%%%%%%%%%%%%%%%%%%%%%%%%%%%
%%%%%%%%%%%%%%%%%%%%%%%%%%%%%%%%%%%%%%%%%%%%%%%%%%%%%%%%%%%%%%%%%%%%%%%%%%%%%%%%
\appendix

\settowidth\MacroIndent{\rmfamily\scriptsize 000\ }

 \DocInput{childdoc.dtx}

\end{document}
%</driver>
% \fi
%
% %%%%%%%%%%%%%%%%%%%%%%%%%%%%%%%%%%%%%%%%%%%%%%%%%%%%%%%%%%%%%%%%%%%%%%%%%%%%%%
% %%%%%%%%%%%%%%%%%%%%%%%%%%%%%%%%%%%%%%%%%%%%%%%%%%%%%%%%%%%%%%%%%%%%%%%%%%%%%%
% \section{Sample}
%\iffalse
%<*samplemain>
%\fi
%
% The following presents a sample document
% with two chapters, two parts, a title page,
% a compile flag as well as three forwarding files to set the flag.
% It consists of eight |.tex| files:
% \begin{center}
% \begin{tabular}{ll}
% |cdocsamp.tex|&main file\\
% |cdocsch1.tex|&include file for chapter 1\\
% |cdocsch2.tex|&include file for chapter 2\\
% |cdocspt3.tex|&include file for part 3\\
% |cdocspt4.tex|&include file for part 4\\
% |cdocsdrf.tex|&forwarding file for main file in draft mode\\
% |cdocsfi1.tex|&forwarding file for final version of chapter 1\\
% |cdocsfi2.tex|&forwarding file for final version of chapter 2\\
% \end{tabular}
% \end{center}
% Each of the eight files can be compiled directly by the \LaTeX{} compiler.
%
% %%%%%%%%%%%%%%%%%%%%%%%%%%%%%%%%%%%%%%
% \paragraph{Main File.}
%
% The main file is called |cdocsamp.tex|.
%
% Load the \textsf{childdoc} definitions and
% declare the filename for the main document:
%    \begin{macrocode}
\input{childdoc.def}
\childdocmain{}
%    \end{macrocode}

% Optional override for |\version| flag:
%    \begin{macrocode}
%%\ifchilddoc\else\providecommand{\version}{draft}\fi
%    \end{macrocode}

% Define the default values for the |\version| flag
% (|final| for the main file and |draft| for childs):
%    \begin{macrocode}
\ifchilddoc
\providecommand{\version}{draft}
\else
\providecommand{\version}{final}
\fi
%    \end{macrocode}

% Load the standard document class:
%    \begin{macrocode}
\documentclass[12pt]{article}
%    \end{macrocode}

% Start the document body:
%    \begin{macrocode}
\begin{document}
%    \end{macrocode}

% Declare a title page.
% Print title, part of document being processed and version flag:
%    \begin{macrocode}
\addtocounter{page}{-1}
\begin{center}
{\LARGE\bfseries{}childdoc example\par}
\vspace{1cm}
\ifchilddoc
\ifchilddocmanual part\else chapter\fi:
`\childdocname' of `\childdocjob'\par
\else
main document: `\childdocjob'\par
\fi
version: \version\par
\end{center}
\newpage
%    \end{macrocode}

% Manually include selected file,
% otherwise process as usual:
%    \begin{macrocode}
\ifchilddocmanual
\section*{part `\childdocname'}
\input{\childdocname}
\else
%    \end{macrocode}

% Include the two chapters:
%    \begin{macrocode}
\include{cdocsch1}
\include{cdocsch2}
%    \end{macrocode}

% Include the two parts unless only chapters should be displayed:
%    \begin{macrocode}
\ifchilddoc\else
\section{part three}
\input{cdocspt3}
\section{part four}
\input{cdocspt4}
\fi
%    \end{macrocode}

% Process as usual until here:
%    \begin{macrocode}
\fi
%    \end{macrocode}

% End of document body:
%    \begin{macrocode}
\end{document}
%    \end{macrocode}
%\iffalse
%</samplemain>
%\fi
%
% %%%%%%%%%%%%%%%%%%%%%%%%%%%%%%%%%%%%%%
% \paragraph{Chapter Include Files.}
%
% The include files are called |cdocsch1.tex| and |cdocsch2.tex|.
%
%\iffalse
%<*samplechap1|samplechap2>
%\fi

% Optional override for |\version| flag:
%    \begin{macrocode}
%%\providecommand{\version}{final}
%    \end{macrocode}

% Include the main document:
%    \begin{macrocode}
\input{childdoc.def}
\childdocof{cdocsamp}
%    \end{macrocode}

%\iffalse
%</samplechap1|samplechap2>
%\fi
%
%\iffalse
%<*samplechap1>
%\fi
% Some text for chapter 1:
%    \begin{macrocode}
\section{one}
some text in chapter one
%    \end{macrocode}

%\iffalse
%</samplechap1>
%\fi
% Some text for chapter 2:
%\iffalse
%<*samplechap2>
%\fi
%    \begin{macrocode}
\section{two}
more text in chapter two
%    \end{macrocode}

%\iffalse
%</samplechap2>
%\fi
%
% %%%%%%%%%%%%%%%%%%%%%%%%%%%%%%%%%%%%%%
% \paragraph{Part Include Files.}
%
% The include files are called |cdocspt3.tex| and |cdocspt4.tex|.
%
%\iffalse
%<*samplepart3|samplepart4>
%\fi

% Optional override for |\version| flag:
%    \begin{macrocode}
%%\providecommand{\version}{final}
%    \end{macrocode}

% Include the main document:
%    \begin{macrocode}
\input{childdoc.def}
\childdocby{cdocsamp}
%    \end{macrocode}

%\iffalse
%</samplepart3|samplepart4>
%\fi
%
%\iffalse
%<*samplepart3>
%\fi
% Some text for part 3:
%    \begin{macrocode}
some text in part three
%    \end{macrocode}

%\iffalse
%</samplepart3>
%\fi
% Some text for part 4:
%\iffalse
%<*samplepart4>
%\fi
%    \begin{macrocode}
more text in part four
%    \end{macrocode}

%\iffalse
%</samplepart4>
%\fi
%
% %%%%%%%%%%%%%%%%%%%%%%%%%%%%%%%%%%%%%%
% \paragraph{Forwarding for a Complete Draft.}
%
% The following forwarding file |cdocsdrf.tex|
% compiles the main document in draft mode:
%\iffalse
%<*sampledraft>
%\fi
%    \begin{macrocode}
\def\version{draft}
\input{childdoc.def}
\childdocforward{cdocsamp}
%    \end{macrocode}

%\iffalse
%</sampledraft>
%\fi
%
% %%%%%%%%%%%%%%%%%%%%%%%%%%%%%%%%%%%%%%
% \paragraph{Forwarding for Final Version of the Chapters.}
%
% The following forwarding files |cdocsfn1.tex| and |cdocsfn2.tex|
% (with identical content)
% compile the final versions of the child documents
% |cdocsch1.tex| and |cdocsch2.tex|, respectively:
%\iffalse
%<*samplefinal>
%\fi
%    \begin{macrocode}
\def\version{final}
\input{childdoc.def}
\childdocforwardprefix[cdocsamp]{cdocsfn}{cdocsch}
%    \end{macrocode}

%\iffalse
%</samplefinal>
%\fi
%
% %%%%%%%%%%%%%%%%%%%%%%%%%%%%%%%%%%%%%%
% \paragraph{Command Line Processing.}
%
% The following three command lines generate the output files
% |cdocscld|, |cdocscl1| and |cdocscl2|
% which should be identical to
% |cdocsdrf|, |cdocsch1| and |cdocsfn2|, respectively:
% \begin{center}
% \begin{tabular}{l}
% |latex -jobname cdocscld \|\\
% |  "\def\version{draft}\input{childdoc.def}\childdocforward{cdocsamp}"|\\
% |latex -jobname cdocscl1 \|\\
% |  "\input{childdoc.def}\childdocforward[cdocsamp]{cdocsch1}"|\\
% |latex -jobname cdocscl2 \|\\
% |  "\def\version{final}\input{childdoc.def}\childdocforward{cdocsch2}"|
% \end{tabular}
% \end{center}
% Note that the trailing backslash on each first line
% merely continues the input to the second line
% (for convenient cut ant paste).
% Furthermore, the command |latex| can be replaced by any
% of its alternative versions such as |pdflatex|.
%
% %%%%%%%%%%%%%%%%%%%%%%%%%%%%%%%%%%%%%%%%%%%%%%%%%%%%%%%%%%%%%%%%%%%%%%%%%%%%%%
% %%%%%%%%%%%%%%%%%%%%%%%%%%%%%%%%%%%%%%%%%%%%%%%%%%%%%%%%%%%%%%%%%%%%%%%%%%%%%%
% \section{Implementation}
%\iffalse
%<*package>
%\fi
%
% This section describes the definitions file |childdoc.def|.

% The definitions cannot be loaded using |\usepackage| or |\RequirePackage|
% which has a mechanism to prevent loading a style file more than once.
% When loading the definitions by means of |\input|
% multiple instances have to be prevented manually:
%\iffalse
%This code needs to be before the `\ProvidesFile' directive
%which is defined at the beginning of this file.
%Therefore it is also placed there and commented out here.
%</package>
%<*discard>
%\fi
%    \begin{macrocode}
\ifdefined\childdocmain\endinput\fi
%    \end{macrocode}
%\iffalse
%</discard>
%<*package>
%\fi
%
% \macro{\ifchilddoc}
% \macro{\ifchilddocmanual}
% The conditional |\ifchilddoc| tells whether a
% child (true) or main (false) document is being compiled.
% The conditional |\ifchilddocmanual| tells whether
% the |\includeonly| mechanism is used (false) or
% the selection of child files must be performed manually (true).
% The definitions initialise to false:
%    \begin{macrocode}
\newif\ifchilddoc
\newif\ifchilddocmanual
%    \end{macrocode}

% \macro{\childdocname}
% \macro{\childdocjob}
% The macro |\childdocname| stores the name of the main document
% to be compiled. The macro |\childdocjob| stores the name of
% the document on which the \LaTeX{} compiler was originally invoked.
% The content of |\jobname| cannot be compared
% to filenames specified in the source due to different catcodes.
% The following code rescans |\jobname|, stores the result
% in |\childdocname| and saves a copy in |\childdocjob|:
%    \begin{macrocode}
\edef\childdocname{\scantokens\expandafter{\jobname\noexpand}}
\let\childdocjob\childdocname
%    \end{macrocode}

% \macro{\childdocdisable}
% The macro |\childdocdisable| prevents the main file
% from being processed more than once.
% At this stage, the main document command |\childdocmain|
% is assumed to be called once again where it should do nothing.
% Any subsequent call to it should prevent
% a secondary processing of the main document
% It overwrites the forwarding commands
% |\childdocof| and |\childdocforward|
% with empty macros to prevent further inclusions of the main document:
%    \begin{macrocode}
\newcommand{\childdocdisable}
{
  \renewcommand{\childdocmain}[1]{\renewcommand{\childdocmain}[1]{\endinput}}
  \renewcommand{\childdocof}[1]{}
  \renewcommand{\childdocby}[2][]{}
  \renewcommand{\childdocforward}[2][]{}
  \renewcommand{\childdocdisable}{}
}
%    \end{macrocode}

% \macro{\childdocmain}
% The macro |\childdocmain| is to be called at the top of the main file
% with nothing or the main filename (without extension) as argument.
% First, it breaks loops.
% If the argument is not empty and does not match |\childdocname|
% (which is set by the first inclusion of |childdoc.def|),
% |\ifchilddoc| is set to true, |\includeonly| is applied to the child file
% and |\jobname| is set to the main file
% (for proper handling of |.aux| files):
%    \begin{macrocode}
\newcommand{\childdocmain}[1]
{
  \childdocdisable\childdocmain{}
  \if?#1?\else
    \begingroup
      \def\childdoctmp{#1}
      \ifx\childdoctmp\childdocname
        \def\childdoctmp{}
      \else
        \def\childdoctmp
        {
          \childdoctrue
          \includeonly{\childdocname}
          \def\childdocjob{#1}
          \def\jobname{#1}
        }
      \fi
      \expandafter
    \endgroup
    \childdoctmp
  \fi
}
%    \end{macrocode}

% \macro{\childdocof}
% The command |\childdocof| redirects
% compilation to the main file |#1|.
%    \begin{macrocode}
\newcommand{\childdocof}[1]
{
  \childdocdisable
  \childdoctrue
  \includeonly{\childdocname}
  \def\jobname{#1}
  \def\childdocjob{#1}
  \input{#1}
}
%    \end{macrocode}

% \macro{\childdocby}
% The command |\childdocby| ....
%    \begin{macrocode}
\newcommand{\childdocby}[2][]
{
  \childdocdisable
  \childdoctrue
  \childdocmanualtrue
  \if?#1?\else
    \def\jobname{#2}
  \fi
  \def\childdocjob{#2}
  \input{#2}
  \endinput
}
%    \end{macrocode}

% \macro{\childdocforward}
% The command |\childdocforward| redirects
% compilation to the main file or
% (if the optional argument is given) a child file.
% Parameters are set as if the main file
% or a child file starting with |\childdocof| was compiled.
% Then compilation is handed over to the main file:
%    \begin{macrocode}
\newcommand{\childdocforward}[2][]
{
  \begingroup
    \if?#1?
      \def\childdoctmp
      {
        \def\childdocname{#2}
        \def\childdocjob{#2}
        \def\jobname{#2}
        \input{#2}
        \endinput
      }
    \else
      \def\childdoctmp
      {
        \childdocdisable
        \def\childdocname{#2}
        \childdoctrue
        \includeonly{#2}
        \def\childdocjob{#1}
        \def\jobname{#1}
        \input{#1}
        \endinput
      }
    \fi
    \expandafter
  \endgroup
  \childdoctmp
}
%    \end{macrocode}

% \macro{\childdocforwardprefix}
% The command |\childdocforwardprefix| redirects
% compilation to the main or a child file by means of a pattern.
% The prefix |#1| in the current filename is replaced by |#2|
% and the suffix of the current filename is kept
% (it is assumed that the filename does not contain the substring `|~~~|'
% which is used as a delimiter).
% Compilation is handed over to the new file by |\childdocforward|:
%    \begin{macrocode}
\newcommand{\childdocforwardprefix}[3][]
{
  \begingroup
    \def\childdocextract #2##1~~~{\def\childdoctmp{\childdocforward[#1]{#3##1}}}
    \expandafter\childdocextract\childdocname~~~
    \expandafter
  \endgroup
  \childdoctmp
}
%    \end{macrocode}

% \macro{\childdoc}
% The deprecated macro |\childdoc| is a legacy version of |\childdocmain|:
%    \begin{macrocode}
\newcommand{\childdoc}{\childdocmain}
%    \end{macrocode}

% \macro{\childdocredirect}
% The deprecated macro |\childdocredirect| is a legacy version
% of |\childdocforward| and |\childdocforwardprefix|:
%    \begin{macrocode}
\newcommand{\childdocredirect}[2][]
{
  \begingroup
    \if?#1?
      \def\childdoctmp{\childdocforward{#2}}
    \else
      \def\childdoctmp{\childdocforwardprefix{#1}{#2}}
    \fi
    \expandafter
  \endgroup
  \childdoctmp
}
%    \end{macrocode}

%\iffalse
%</package>
%\fi
%
\endinput
|\\
|\childdocforwardprefix[|\textit{main}|]{|\textit{prefix}|}{|\textit{dest}|}|
\end{tabular}
\end{center}
%
the destination file is determined by a pattern
depending on the current file:
To make this work, the current file must be called
`{\textit{prefix}\hspace{0.2em}\textit{suffix}}'
with \textit{prefix} matching precisely the argument.
Processing is then passed on to the file
`{\textit{dest}\hspace{0.2em}\textit{suffix}}'.
Surely, the same effect is achieved by
directly specifying the
argument `{\textit{dest}\hspace{0.2em}\textit{suffix}}'
in the first form.
However, that requires to set up a different file
for each child. With the alternative form of the command
all these files can have exactly the same content
which simplifies setting them up and maintaining them.

For example, the following file |draft.tex|
with a compilation flag |\version| as described in \secref{sec:flags}
compiles the main document as a draft:
%
\begin{center}
\begin{tabular}{l}
|\def\version{draft}|\\
|% \iffalse
%
% childdoc.dtx Copyright (C) 2017-2018 Niklas Beisert
%
% This work may be distributed and/or modified under the
% conditions of the LaTeX Project Public License, either version 1.3
% of this license or (at your option) any later version.
% The latest version of this license is in
%   http://www.latex-project.org/lppl.txt
% and version 1.3 or later is part of all distributions of LaTeX
% version 2005/12/01 or later.
%
% This work has the LPPL maintenance status `maintained'.
%
% The Current Maintainer of this work is Niklas Beisert.
%
% This work consists of the files childdoc.dtx and childdoc.ins
% and the derived files childdoc.def and cdocsamp.tex with
% cdocsch1.tex, cdocsch2.tex, cdocsdrf.tex, cdocsfn1.tex, cdocsfn2.tex.
%
%<package>\ifdefined\childdocmain\endinput\fi
%<package>\ProvidesFile{childdoc.def}[2018/12/30 v2.0 child document driver]
%<samplemain>\ProvidesFile{cdocsamp.tex}[2018/12/30 v2.0 sample for childdoc]
%<*driver>
%\ProvidesFile{childdoc.drv}[2018/12/30 v2.0 childdoc reference manual file]
\PassOptionsToClass{10pt,a4paper}{article}
\documentclass{ltxdoc}

\usepackage[margin=35mm]{geometry}
\usepackage{hyperref}
\usepackage{hyperxmp}
\usepackage[usenames]{color}

\hypersetup{colorlinks=true}
\hypersetup{pdfstartview=FitH}
\hypersetup{pdfpagemode=UseNone}
\hypersetup{pdfsource={}}
\hypersetup{pdflang={en-UK}}
\hypersetup{pdfcopyright={Copyright 2017-2018 Niklas Beisert.
  This work may be distributed and/or modified under the
  conditions of the LaTeX Project Public License, either version 1.3
  of this license or (at your option) any later version.}}
\hypersetup{pdflicenseurl={http://www.latex-project.org/lppl.txt}}
\hypersetup{pdfcontactaddress={ETH Zurich, ITP, HIT K,
  Wolfgang-Pauli-Strasse 27}}
\hypersetup{pdfcontactpostcode={8093}}
\hypersetup{pdfcontactcity={Zurich}}
\hypersetup{pdfcontactcountry={Switzerland}}
\hypersetup{pdfcontactemail={nbeisert@itp.phys.ethz.ch}}
\hypersetup{pdfcontacturl={http://people.phys.ethz.ch/\xmptilde nbeisert/}}

\newcommand{\secref}[1]{\hyperref[#1]{section \ref*{#1}}}

\parskip1ex
\parindent0pt
\let\olditemize\itemize
\def\itemize{\olditemize\parskip0pt}

\begin{document}

\title{The \textsf{childdoc} Package}
\hypersetup{pdftitle={The childdoc Package}}
\author{Niklas Beisert\\[2ex]
  Institut f\"ur Theoretische Physik\\
  Eidgen\"ossische Technische Hochschule Z\"urich\\
  Wolfgang-Pauli-Strasse 27, 8093 Z\"urich, Switzerland\\[1ex]
  \href{mailto:nbeisert@itp.phys.ethz.ch}
  {\texttt{nbeisert@itp.phys.ethz.ch}}}
\hypersetup{pdfauthor={Niklas Beisert}}
\hypersetup{pdfsubject={Manual for the LaTeX2e Package childdoc}}
\date{30 December 2018, \textsf{v2.0}}
\maketitle

\begin{abstract}\noindent
\textsf{childdoc} is a \LaTeXe{} package
that enables the direct compilation
of document sections included by |\include|
to individual files.
\end{abstract}

\begingroup
\parskip0ex
\tableofcontents
\endgroup

%%%%%%%%%%%%%%%%%%%%%%%%%%%%%%%%%%%%%%%%%%%%%%%%%%%%%%%%%%%%%%%%%%%%%%%%%%%%%%%%
%%%%%%%%%%%%%%%%%%%%%%%%%%%%%%%%%%%%%%%%%%%%%%%%%%%%%%%%%%%%%%%%%%%%%%%%%%%%%%%%
\section{Introduction}

\LaTeX{} provides a mechanism to structure a large document (such as a book)
into a main file and several child files (containing the chapters)
using the |\include| command.
This mechanism is beneficial for documents
which span hundreds of pages in order to
make the source file(s) more manageable.
Moreover, compilation can be restricted to
selected child files by means of the |\includeonly| command.
The latter feature can be used to reduce the compilation time while editing
(this was significantly more useful in the earlier days of \LaTeX{})
or to generate a smaller document which is easier to navigate.
Another application of |\includeonly| is to generate
documents consisting of selected parts of the complete document.

However, there are a few drawbacks of the plain |\include| mechanism:
\begin{itemize}
\item
The child files cannot be compiled on their own,
they can only be compiled via the main file.
A naive editing environment
(such as a text editor with an option
to have the current file processed by \LaTeX)
may require one to switch to the main file before compiling;
attempting to compile the child file produces errors.
\item
The main file must be modified (each time)
to adjust the |\includeonly| command
to the present needs. This easily leaves the main file in a messy state.
\item
The generated document will always carry the filename
of the main document. This is inconvenient if
several child files are to be compiled and
to be kept for distribution.
\end{itemize}

The present package provides a simple interface
to make child files individually compilable by \LaTeX{}.
Compiling a child file then has the same effect as compiling
the main file with an |\includeonly| command
to select the appropriate child.
Moreover the generated document will carry the name of the child
rather than the main file.
This resolves all three above issues.

This feature is meant to make the editing of books,
thesis documents and lecture notes somewhat more convenient.
However, the package can also be used efficiently for
composing a series of documents (such as exercise sheets)
which are typically distributed individually.
It then assists the author in generating the individual documents
(potentially in different versions)
as well as a document containing the collected series.
Another application is in developing style files
or other kinds of included material
where compilation of the style file could redirect
to a sample or test file.

%%%%%%%%%%%%%%%%%%%%%%%%%%%%%%%%%%%%%%%%%%%%%%%%%%%%%%%%%%%%%%%%%%%%%%%%%%%%%%%%
%%%%%%%%%%%%%%%%%%%%%%%%%%%%%%%%%%%%%%%%%%%%%%%%%%%%%%%%%%%%%%%%%%%%%%%%%%%%%%%%
\section{Usage}

First of all, the package \textsf{childdoc} is \emph{not} a standard
\LaTeXe{} |.sty| style file! Therefore it needs to be invoked in
a non-standard way.

%%%%%%%%%%%%%%%%%%%%%%%%%%%%%%%%%%%%%%%%%%%%%%%%%%%%%%%%%%%%%%%%%%%%%%%%%%%%%%%%
\subsection{Included Files}
\label{sec:include}

%%%%%%%%%%%%%%%%%%%%%%%%%%%%%%%%%%%%%%%%
\DescribeMacro{\childdocmain}
To use the package, add the commands
\begin{center}
\begin{tabular}{l}
|\input{childdoc.def}|\\
|\childdocmain{}|\\
\end{tabular}
\end{center}
at the very top of the main \LaTeX{} file,
in particular \emph{before} the |\documentclass| statement!
The argument of |\childdocmain| should be left empty
(but it must be present).

%%%%%%%%%%%%%%%%%%%%%%%%%%%%%%%%%%%%%%%%
\DescribeMacro{\childdocof}
Furthermore, add the commands
\begin{center}
\begin{tabular}{l}
|\input{childdoc.def}|\\
|\childdocof{|\textit{main}|}|\\
\end{tabular}
\end{center}
at the top of every child file \textit{child}
which is included by |\include{|\textit{child}|}|
from within the main file
(or at least for those files to be compiled individually).
The argument \textit{main} must be the filename of the main file.

There are a couple of
considerations in setting up the main and child documents:

%%%%%%%%%%%%%%%%%%%%%%%%%%%%%%%%%%%%%%%%
\paragraph{Restrictions.}

Please note the following restrictions:
\begin{itemize}
\item
|\childdocmain| must be called with one argument \textit{main}
to ensure compatibility with earlier version of the package.
It must either be empty (|\childdocmain{}|)
or precisely match the filename of the main file in which it is specified.
See \secref{sec:detection} for further information.
\item
The filename \textit{main} must be specified without the |.tex| extension.
\item
The filename \textit{main} is case sensitive
(even in case-insensitive file systems)
due to internal string comparison.
\item
The argument \textit{main} should be fully expanded, it cannot be a macro.
\item
Subdirectories and special characters should be avoided in filenames.
\item
The command |\childdocmain{|\textit{main}|}| must be followed by a whitespace.
It should not be followed immediately by another command
or by a comment mark `|%|'.
This is because the \TeX{} parser reads the token immediately following
the argument of |\childdocmain| and puts it
at the beginning of every child section;
however, a white\-space is ignored.
\end{itemize}

%%%%%%%%%%%%%%%%%%%%%%%%%%%%%%%%%%%%%%%%
\paragraph{Content of Main File.}

It is advisable to place all content in the child files included by |\include|.
Any output contained in the main file will appear in all child documents
unless suppressed manually;
it cannot be suppressed automatically by the |\includeonly| directive
and thus should normally be avoided.
A method to include some content in the main file
by means of conditional processing is described in \secref{sec:conditional}.

%%%%%%%%%%%%%%%%%%%%%%%%%%%%%%%%%%%%%%%%
\paragraph{Page Numbering.}

When only a part of the document is compiled,
the appropriate numbering of pages
(as well as other status parameters)
is determined from the |.aux| files.
The latter contain information from previous passes.
However this information needs to propagate through
all intermediate child documents.
Therefore the page numbering in child documents may well
be inconsistent until the complete document is compiled at least once.

A useful (if unconventional) way to always ensure a consistent
page numbering is to restart the numbering in each child document
and denote the pages by `\textit{child}|.|\textit{page}'
where \textit{child} represents the chapter/section number of the child file.
This can be achieved by the command
|\numberwithin{page}{|\textit{child}|}|
of the \textsf{amsmath} package
where \textit{child} can be |chapter| or |section|
depending on the chosen structuring.
Alternatively, one can modify the macro |\thepage| appropriately
and reset the counter |page| at the start of each child file.

%%%%%%%%%%%%%%%%%%%%%%%%%%%%%%%%%%%%%%%%%%%%%%%%%%%%%%%%%%%%%%%%%%%%%%%%%%%%%%%%
\subsection{Conditional Processing}
\label{sec:conditional}

The package provides a mechanism to compile different versions
of a document. To customise the versions further some conditional processing
can come in handy to distinguish which version is being compiled.
The package provides two macros to describe the compilation context:

%%%%%%%%%%%%%%%%%%%%%%%%%%%%%%%%%%%%%%%%
\DescribeMacro{\ifchilddoc}
The conditional |\ifchilddoc| distinguishes between the compilation of
child documents and the main document:
%
\begin{center}
|\ifchilddoc |\textit{child-code}| |[|\||else |\textit{main-code}]| \||fi|
\end{center}

%%%%%%%%%%%%%%%%%%%%%%%%%%%%%%%%%%%%%%%%
\DescribeMacro{\childdocname}
\DescribeMacro{\childdocjob}
The macro |\childdocname| contains the filename (without extension)
of the main or child file being processed.
Note that |\childdocjob| will always contain the name of the main file.

%%%%%%%%%%%%%%%%%%%%%%%%%%%%%%%%%%%%%%%%
\paragraph{Title Page.}

Conditional processing can be used to include a title or banner page
in the main document when proper precautions are taken.
Importantly, the code in the main file should ensure that the page counter
(as well as other status parameters which are stored in the |.aux| files)
takes the same value after the conditional processing.
Otherwise the page numbers may take divergent values
depending on which part is compiled.

For example, a title page could be declared by:
%
\begin{center}
\begin{tabular}{l}
|\ifchilddoc\||else|\\
|\addtocounter{page}{-1}|\\
\textit{code for title page}\\
|\newpage|\\
|\||fi|
\end{tabular}
\end{center}
%
A banner page for the child documents can be generated by:
%
\begin{center}
\begin{tabular}{l}
|\ifchilddoc|\\
|\addtocounter{page}{-1}|\\
\textit{code for banner page}\\
|\newpage|\\
|\||fi|
\end{tabular}
\end{center}
%
Here one could write a message such as:
\begin{center}
|This is the part \childdocname{} of \childdocjob{}.|
\end{center}

%%%%%%%%%%%%%%%%%%%%%%%%%%%%%%%%%%%%%%%%%%%%%%%%%%%%%%%%%%%%%%%%%%%%%%%%%%%%%%%%
\subsection{Flags}
\label{sec:flags}

The package makes it easy to generate different versions
of the main or child documents.
To this end compilation flags can be defined
and assigned different default values.
They will be particularly useful in conjunction
with the forwarding mechanism described in \secref{sec:forward}.

For example, it may be useful to have a flag |\version|
which can be set to |draft| or |final|.
The document source will contain some conditional code
depending on the value of |\version|.
Suppose further, the flag should default to |final| for the main file
and to |draft| for child files
which is a natural assignment for editing the document.
This is achieved by placing the following code
in the preamble of the main document
(below the |\childdocmain| directive):
%
\begin{center}
\begin{tabular}{l}
|\ifchilddoc|\\
|\providecommand{\version}{draft}|\\
|\||else|\\
|\providecommand{\version}{final}|\\
|\||fi|
\end{tabular}
\end{center}
%
The definition by |\providecommand| makes sure
that previous definitions are not overwritten.
Further statements |\providecommand{\version}{...}|
can thus be added before the above code to override it.

For the main file, one might add a line
(between |\childdocmain| and the above block)
%
\begin{center}
|%\ifchilddoc\||else\providecommand{\version}{draft}\||fi|
\end{center}
%
which can be uncommented to produce a draft version.
Likewise one can add a line to the very top of a child file
(above the |\childdocof{|\textit{main}|}| directive)
%
\begin{center}
|%\providecommand{\version}{final}|
\end{center}
%
which can be uncommented to produce the final version of this child document.

%%%%%%%%%%%%%%%%%%%%%%%%%%%%%%%%%%%%%%%%%%%%%%%%%%%%%%%%%%%%%%%%%%%%%%%%%%%%%%%%
\subsection{Forwarding}
\label{sec:forward}

Different versions of the main or child documents
using compilation flags as described in \secref{sec:flags}
can be (permanently) stored in different files
for convenient compilation, viewing and distribution.
To this end, the package defines a command
to pass on compilation to a different file:

%%%%%%%%%%%%%%%%%%%%%%%%%%%%%%%%%%%%%%%%
\DescribeMacro{\childdocforward}
The command |\childdocforward| redirects processing to
another source file:
%
\begin{center}
\begin{tabular}{l}
|\input{childdoc.def}|\\
|\childdocforward[|\textit{main}|]{|\textit{dest}|}|\\
\end{tabular}
\end{center}
%
The argument \textit{dest} is the destination file
(without extension).
It should be the main file or one of the child files.
Note that further \textsf{childdoc} directives
such as |\childdocof| and |\childdocforward|
in the indicated file will be processed in this form.
The optional argument \textit{main}
passes on directly to the main file \textit{main}
while pretending to compile the child \textit{dest}.
This form behaves as if \textit{dest}
issues |\childdocof{|\textit{main}|}| right away,
and no further \textsf{childdoc} directives will be processed.

%%%%%%%%%%%%%%%%%%%%%%%%%%%%%%%%%%%%%%%%
\DescribeMacro{\...prefix}
In the alternative form |\childdocforwardprefix|,
%
\begin{center}
\begin{tabular}{l}
|\input{childdoc.def}|\\
|\childdocforwardprefix[|\textit{main}|]{|\textit{prefix}|}{|\textit{dest}|}|
\end{tabular}
\end{center}
%
the destination file is determined by a pattern
depending on the current file:
To make this work, the current file must be called
`{\textit{prefix}\hspace{0.2em}\textit{suffix}}'
with \textit{prefix} matching precisely the argument.
Processing is then passed on to the file
`{\textit{dest}\hspace{0.2em}\textit{suffix}}'.
Surely, the same effect is achieved by
directly specifying the
argument `{\textit{dest}\hspace{0.2em}\textit{suffix}}'
in the first form.
However, that requires to set up a different file
for each child. With the alternative form of the command
all these files can have exactly the same content
which simplifies setting them up and maintaining them.

For example, the following file |draft.tex|
with a compilation flag |\version| as described in \secref{sec:flags}
compiles the main document as a draft:
%
\begin{center}
\begin{tabular}{l}
|\def\version{draft}|\\
|\input{childdoc.def}|\\
|\childdocforward{|\textit{main}|}|
\end{tabular}
\end{center}
%
Likewise, the following files |final|\textit{nn}|.tex|
compile the final version of the child document
|child|\textit{nn}|.tex|:
%
\begin{center}
\begin{tabular}{l}
|\def\version{final}|\\
|\input{childdoc.def}|\\
|\childdocforwardprefix{final}{child}|
\end{tabular}
\end{center}
%

Note that when several versions of a main file and/or of each child file
are to be generated, it may be convenient to set up a |Makefile| or
shell script to automatise the process.

%%%%%%%%%%%%%%%%%%%%%%%%%%%%%%%%%%%%%%%%%%%%%%%%%%%%%%%%%%%%%%%%%%%%%%%%%%%%%%%%
\subsection{Command Line Processing}
\label{sec:commandline}

The effect of redirection files can also be achieved by invoking
the \LaTeX{} compiler with a more elaborate command line.
Most conveniently this should be done as part
of a shell script or a |Makefile|.

When using \textsf{childdoc} in the main file, the following
command lines effectively perform a redirection
(note that depending on the shell being used,
backslashes may have to be doubled: `|\|' $\to$ `|\\|'):
%
\begin{center}
|... -jobname "|\textit{target}|" |\\|"|[\textit{flags}]%
|\input{childdoc.def}\childdocforward[|\textit{main}|]{|\textit{dest}|}"|
\end{center}
%
Here \textit{target} is the name of the output file,
\textit{main} is the name of the main file
and \textit{dest} is the name of the main or child file to be processed
(all filenames without extensions).
The optional argument \textit{main} can be omitted
if \textit{main} matches \textit{dest}.
Optionally, compilation \textit{flags} can be defined via |\def| commands.
This command line makes the \TeX{} engine believe
it is compiling the file \textit{target}
whose content is specified as the latter parameter.
The provided code then forwards the processing to
\textit{main} or \textit{dest} as described in \secref{sec:forward}.

%%%%%%%%%%%%%%%%%%%%%%%%%%%%%%%%%%%%%%%%%%%%%%%%%%%%%%%%%%%%%%%%%%%%%%%%%%%%%%%%
\subsection{Include by Input}
\label{sec:input}

Including child documents by |\include| has some restrictions by design.
Most notably, the content of a child document always occupies
its own set of pages; pages cannot be shared between child documents.
Usually, this behaviour makes perfect sense
because each child document contain an essential part of the document.
However, in some situations it may be desirable to compose
a document from a collection of parts
without having mandatory page breaks between then.
For this case, the package
provides a mechanism to include parts
by |\input| which can also be processed individually.
However, by construction this mechanism
requires manual handling of the content to be output.

%%%%%%%%%%%%%%%%%%%%%%%%%%%%%%%%%%%%%%%%
\DescribeMacro{\ifchilddocmanual}
The main file should be prepared as usual, see \secref{sec:include}.
However, the document body must make a distinction
between processing of an individual part and of the main document, e.g.:
%
\begin{center}
\begin{tabular}{l}
|\ifchilddocmanual|\\
|\input{\childdocname}|\\
|\||else|\\
\textit{document body with }|\input{|\textit{part}|}|\\
|\||fi|
\end{tabular}
\end{center}
%
The conditional |\ifchilddocmanual| is true whenever
a part to be included by |\input| is being compiled,
and the name of the part is stored in |\childdocname|.

%%%%%%%%%%%%%%%%%%%%%%%%%%%%%%%%%%%%%%%%
\DescribeMacro{\childdocby}
Each part to be included by |\input| should start with:
%
\begin{center}
\begin{tabular}{l}
|\input{childdoc.def}|\\
|\childdocby{|\textit{main}|}|\\
\end{tabular}
\end{center}
%
The directive |\childdocby| is similar to |\childdocof|
described in \secref{sec:include},
but the subsequent selection of content must be done manually.
To that end, both |\ifchilddoc| and |\ifchilddocmanual|
will be true upon processing of a part,
and the name of the part is stored in |\childdocname|.
Note that |\jobname| will be set to the filename of the current part
so that each part receives an individual |.aux| file
that does not interfere with the |.aux| file(s) of the main document.
This behaviour can be altered by the alternative form
|\childdocby[*]{|\textit{main}|}| (with a non-empty optional argument)
which uses the |.aux| file of the main document
by setting |\jobname| to \textit{main}.

%%%%%%%%%%%%%%%%%%%%%%%%%%%%%%%%%%%%%%%%%%%%%%%%%%%%%%%%%%%%%%%%%%%%%%%%%%%%%%%%
\subsection{Driver Development}
\label{sec:driver}

The \textsf{childdoc} mechanism can also be use for the development
of definition files such as \LaTeX{} styles or classes.
This case differs from the above setup with multiple parts
included by |\include| in that no |\includeonly| should be invoked.
This can be achieved by starting the include file
(before |\ProvidesPackage|) with:
%
\begin{center}
\begin{tabular}{l}
|\input{childdoc.def}|\\
|\childdocforward{|\textit{main}|}|\\
\end{tabular}
\end{center}
%
or alternatively with:
%
\begin{center}
\begin{tabular}{l}
|\input{childdoc.def}|\\
|\childdocby{|\textit{main}|}|\\
\end{tabular}
\end{center}
%
Both forms have slightly different effects as described above.
The main file is prepared as usual, see \secref{sec:include}.

%%%%%%%%%%%%%%%%%%%%%%%%%%%%%%%%%%%%%%%%%%%%%%%%%%%%%%%%%%%%%%%%%%%%%%%%%%%%%%%%
\subsection{Legacy Detection}
\label{sec:detection}

The directive |\childdocmain| in the main file can detect
whether the complete document or merely a child is to be compiled
even without using the directive |\childdocof|.
This method is deprecated because it is less robust
and there is no compelling reason to use it;
it is merely provided for backward compatibility
and it may be removed in future versions.

If the detection mechanism is to be used,
it is mandatory to correctly specify
the filename of the main file as the argument of |\childdocmain|:
%
\begin{center}
\begin{tabular}{l}
|\input{childdoc.def}|\\
|\childdocmain{|\textit{main}|}|\\
\end{tabular}
\end{center}
%
If |\jobname| does not match the argument \textit{main} of |\childdocmain|,
it is assumed that |\jobname| points to the child file to be compiled.
When using |\childdocmain| with the main file specified as argument,
it suffices to start a child file
with just |\input{|\textit{main}|}|
without loading of the package and using |\childdocof|.
If instead all processing is done
with the appropriate \textsf{childdoc} directives,
the argument of \textit{main} of |\childdocmain| can be empty.

An alternative version of the command line processing described
in \secref{sec:commandline} using the detection mechanism reads:
%
\begin{center}
|... -jobname "|\textit{target}|" "|[\textit{flags}]%
[|\def\jobname{|\textit{dest}|}|]|\input{|\textit{main}|}"|
\end{center}

%%%%%%%%%%%%%%%%%%%%%%%%%%%%%%%%%%%%%%%%%%%%%%%%%%%%%%%%%%%%%%%%%%%%%%%%%%%%%%%%
\subsection{Manual Code}
\label{sec:manual}

In case one cannot be certain whether the definitions file |childdoc.def|
is installed on the target \TeX{} distribution
and one prefers not to ship it,
it is conceivable to paste a few relevant commands into the sources.

To that end, drop all statements |\input{childdoc.def}|
and perform the replacements as outlined below.
Instead of |\childdocmain{|\textit{main}|}| add the following code
to the top of the main file:
%
\begin{center}
\begin{tabular}{l}
|\||ifdefined\childdocname\endinput\||fi\newif\ifchilddoc|\\
|\edef\childdocname{\scantokens\expandafter{\jobname\noexpand}}|\\
|\def\childdocmain{|\textit{main}|}\||ifx\childdocmain\childdocname\||else|\\
|\childdoctrue\includeonly{\childdocname}\let\jobname\childdocmain\||fi|\\
\end{tabular}
\end{center}
%
Instead of |\childdocof{|\textit{main}|}| just include the main file
at the top of each child file:
%
\begin{center}
|\input{|\textit{main}|}|
\end{center}
%
A simple redirection |\childdocforward{|\textit{dest}|}| is achieved by:
%
\begin{center}
|\def\jobname{|\textit{dest}|}\input{\jobname}|
\end{center}
%
The redirection with prefix
|\childdocforwardprefix[|\textit{prefix}|]{|\textit{dest}|}|
is accomplished by:
%
\begin{center}
\begin{tabular}{l}
|{\edef\jobname{\scantokens\expandafter{\jobname\noexpand}}|\\
|\def\redirectjob |\textit{prefix}|#1~~~{\gdef\jobname{|\textit{dest}|#1}}|\\
|\expandafter\redirectjob\jobname~~~}\input{\jobname}|
\end{tabular}
\end{center}

In an alternative approach,
child documents can be compiled by a specific command line
without additional code or specific definitions:
%
\begin{center}
|... -jobname "|\textit{target}|" "|[\textit{flags}]%
|\includeonly{|\textit{dest}|}\input{|\textit{main}|}"|
\end{center}
%

%%%%%%%%%%%%%%%%%%%%%%%%%%%%%%%%%%%%%%%%%%%%%%%%%%%%%%%%%%%%%%%%%%%%%%%%%%%%%%%%
%%%%%%%%%%%%%%%%%%%%%%%%%%%%%%%%%%%%%%%%%%%%%%%%%%%%%%%%%%%%%%%%%%%%%%%%%%%%%%%%
\section{Information}

%%%%%%%%%%%%%%%%%%%%%%%%%%%%%%%%%%%%%%%%%%%%%%%%%%%%%%%%%%%%%%%%%%%%%%%%%%%%%%%%
\subsection{Copyright}

Copyright \copyright{} 2017--2018 Niklas Beisert

This work may be distributed and/or modified under the
conditions of the \LaTeX{} Project Public License, either version 1.3
of this license or (at your option) any later version.
The latest version of this license is in
  \url{http://www.latex-project.org/lppl.txt}
and version 1.3 or later is part of all distributions of \LaTeX{}
version 2005/12/01 or later.

This work has the LPPL maintenance status `maintained'.

The Current Maintainer of this work is Niklas Beisert.

This work consists of the files |README.txt|, |childdoc.ins| and |childdoc.dtx|
as well as the derived files |childdoc.def|, |cdocsamp.tex|
with |cdocsch1.tex|, |cdocsch2.tex|, |cdocspt3.tex|, |cdocspt4.tex|,
|cdocsdrf.tex|, |cdocsfn1.tex|, |cdocsfn2.tex|
as well as |childdoc.pdf|.

%%%%%%%%%%%%%%%%%%%%%%%%%%%%%%%%%%%%%%%%%%%%%%%%%%%%%%%%%%%%%%%%%%%%%%%%%%%%%%%%
\subsection{Files and Installation}

The package consists of the files:
%
\begin{center}
\begin{tabular}{ll}
    |README.txt|   & readme file \\
    |childdoc.ins| & installation file \\
    |childdoc.dtx| & source file \\
    |childdoc.def| & definition file \\
    |cdocsamp.tex| & sample main file \\
    |cdocsch1.tex| & sample include file \\
    |cdocsch2.tex| & sample include file \\
    |cdocspt3.tex| & sample part file \\
    |cdocspt4.tex| & sample part file \\
    |cdocsdrf.tex| & sample redirection file \\
    |cdocsfn1.tex| & sample redirection file \\
    |cdocsfn2.tex| & sample redirection file \\
    |childdoc.pdf| & manual
\end{tabular}
\end{center}
%
The distribution consists of the files
|README.txt|, |childdoc.ins| and |childdoc.dtx|.
%
\begin{itemize}
\item
Run (pdf)\LaTeX{} on |childdoc.dtx|
to compile the manual |childdoc.pdf| (this file).
\item
Run \LaTeX{} on |childdoc.ins| to create the definitions file |childdoc.def|
and the sample |cdocsamp.tex| with include files
|cdocsch1.tex|, |cdocsch2.tex|, |cdocspt3.tex|, |cdocspt4.tex|,
|cdocsdrf.tex|, |cdocsfn1.tex|, |cdocsfn2.tex|.
Then copy the file |childdoc.def| to an appropriate directory of your \LaTeX{}
distribution, e.g.\ \textit{texmf-root}|/tex/latex/childdoc|.
\end{itemize}

%%%%%%%%%%%%%%%%%%%%%%%%%%%%%%%%%%%%%%%%%%%%%%%%%%%%%%%%%%%%%%%%%%%%%%%%%%%%%%%%
\subsection{Related CTAN Packages}

There are several other packages which offer a similar functionality:
%
\begin{itemize}
\item
The packages
\href{http://ctan.org/pkg/docmute}{\textsf{docmute}},
\href{http://ctan.org/pkg/includex}{\textsf{includex}} and
\href{http://ctan.org/pkg/standalone}{\textsf{standalone}}
provide commands to include only the document body of
a child file thus allowing both files to be compiled individually.
\item
The packages \href{http://ctan.org/pkg/subdocs}{\textsf{subdocs}}
and \href{http://ctan.org/pkg/subfiles}{\textsf{subfiles}}
provide structures in which the main and child documents can be
encapsulated and allowing them to be compiled individually.
The inclusion mechanism is different from the conventional |\include|.
\item
The package \href{http://ctan.org/pkg/combine}{\textsf{combine}}
is an elaborate solution to combine several documents into one.
\end{itemize}
%
See also the CTAN topic \href{http://ctan.org/topic/subdocs}{\textsf{subdocs}}
for further related packages.
The present package differs from the above solutions in that
a document structure constructed with the conventional |\include| mechanism
just needs two extra commands at the top of every file
such that all constituent files can be compiled individually.

%%%%%%%%%%%%%%%%%%%%%%%%%%%%%%%%%%%%%%%%%%%%%%%%%%%%%%%%%%%%%%%%%%%%%%%%%%%%%%%%
%\subsection{Feature Suggestions}
%
%The following is a list of features which may be useful for future
%versions of this package:
%%
%\begin{itemize}
%\item
%\ldots
%\end{itemize}

%%%%%%%%%%%%%%%%%%%%%%%%%%%%%%%%%%%%%%%%%%%%%%%%%%%%%%%%%%%%%%%%%%%%%%%%%%%%%%%%
\subsection{Revision History}

%%%%%%%%%%%%%%%%%%%%%%%%%%%%%%%%%%%%%%%%
\paragraph{v2.0:} 2018/12/30

\begin{itemize}
\item
immediate forward processing
\item
added |\childdocby| mechanism
\item
manual restructured
\end{itemize}

%%%%%%%%%%%%%%%%%%%%%%%%%%%%%%%%%%%%%%%%
\paragraph{v1.6:} 2018/01/17

\begin{itemize}
\item
application for development of include files
\item
corrections to manual
\end{itemize}

%%%%%%%%%%%%%%%%%%%%%%%%%%%%%%%%%%%%%%%%
\paragraph{v1.5:} 2017/05/21

\begin{itemize}
\item
more complete structuring introduced
\item
|\childdocof| introduced
\item
|\childdoc| renamed to |\childdocmain|
\item
|\childredirect| renamed to |\childdocforward| and |\childdocforwardprefix|
and functionality expanded
\end{itemize}

%%%%%%%%%%%%%%%%%%%%%%%%%%%%%%%%%%%%%%%%
\paragraph{v1.0:} 2017/04/27

\begin{itemize}
\item
manual and install package
\item
first version published on CTAN
\end{itemize}

%%%%%%%%%%%%%%%%%%%%%%%%%%%%%%%%%%%%%%%%
\paragraph{v0.6:} 2017/04/26

\begin{itemize}
\item
redirection mechanism added
\end{itemize}

%%%%%%%%%%%%%%%%%%%%%%%%%%%%%%%%%%%%%%%%
\paragraph{v0.5:} 2017/04/26

\begin{itemize}
\item
functionality in definition file
\end{itemize}


%%%%%%%%%%%%%%%%%%%%%%%%%%%%%%%%%%%%%%%%%%%%%%%%%%%%%%%%%%%%%%%%%%%%%%%%%%%%%%%%
%%%%%%%%%%%%%%%%%%%%%%%%%%%%%%%%%%%%%%%%%%%%%%%%%%%%%%%%%%%%%%%%%%%%%%%%%%%%%%%%
%%%%%%%%%%%%%%%%%%%%%%%%%%%%%%%%%%%%%%%%%%%%%%%%%%%%%%%%%%%%%%%%%%%%%%%%%%%%%%%%
\appendix

\settowidth\MacroIndent{\rmfamily\scriptsize 000\ }

 \DocInput{childdoc.dtx}

\end{document}
%</driver>
% \fi
%
% %%%%%%%%%%%%%%%%%%%%%%%%%%%%%%%%%%%%%%%%%%%%%%%%%%%%%%%%%%%%%%%%%%%%%%%%%%%%%%
% %%%%%%%%%%%%%%%%%%%%%%%%%%%%%%%%%%%%%%%%%%%%%%%%%%%%%%%%%%%%%%%%%%%%%%%%%%%%%%
% \section{Sample}
%\iffalse
%<*samplemain>
%\fi
%
% The following presents a sample document
% with two chapters, two parts, a title page,
% a compile flag as well as three forwarding files to set the flag.
% It consists of eight |.tex| files:
% \begin{center}
% \begin{tabular}{ll}
% |cdocsamp.tex|&main file\\
% |cdocsch1.tex|&include file for chapter 1\\
% |cdocsch2.tex|&include file for chapter 2\\
% |cdocspt3.tex|&include file for part 3\\
% |cdocspt4.tex|&include file for part 4\\
% |cdocsdrf.tex|&forwarding file for main file in draft mode\\
% |cdocsfi1.tex|&forwarding file for final version of chapter 1\\
% |cdocsfi2.tex|&forwarding file for final version of chapter 2\\
% \end{tabular}
% \end{center}
% Each of the eight files can be compiled directly by the \LaTeX{} compiler.
%
% %%%%%%%%%%%%%%%%%%%%%%%%%%%%%%%%%%%%%%
% \paragraph{Main File.}
%
% The main file is called |cdocsamp.tex|.
%
% Load the \textsf{childdoc} definitions and
% declare the filename for the main document:
%    \begin{macrocode}
\input{childdoc.def}
\childdocmain{}
%    \end{macrocode}

% Optional override for |\version| flag:
%    \begin{macrocode}
%%\ifchilddoc\else\providecommand{\version}{draft}\fi
%    \end{macrocode}

% Define the default values for the |\version| flag
% (|final| for the main file and |draft| for childs):
%    \begin{macrocode}
\ifchilddoc
\providecommand{\version}{draft}
\else
\providecommand{\version}{final}
\fi
%    \end{macrocode}

% Load the standard document class:
%    \begin{macrocode}
\documentclass[12pt]{article}
%    \end{macrocode}

% Start the document body:
%    \begin{macrocode}
\begin{document}
%    \end{macrocode}

% Declare a title page.
% Print title, part of document being processed and version flag:
%    \begin{macrocode}
\addtocounter{page}{-1}
\begin{center}
{\LARGE\bfseries{}childdoc example\par}
\vspace{1cm}
\ifchilddoc
\ifchilddocmanual part\else chapter\fi:
`\childdocname' of `\childdocjob'\par
\else
main document: `\childdocjob'\par
\fi
version: \version\par
\end{center}
\newpage
%    \end{macrocode}

% Manually include selected file,
% otherwise process as usual:
%    \begin{macrocode}
\ifchilddocmanual
\section*{part `\childdocname'}
\input{\childdocname}
\else
%    \end{macrocode}

% Include the two chapters:
%    \begin{macrocode}
\include{cdocsch1}
\include{cdocsch2}
%    \end{macrocode}

% Include the two parts unless only chapters should be displayed:
%    \begin{macrocode}
\ifchilddoc\else
\section{part three}
\input{cdocspt3}
\section{part four}
\input{cdocspt4}
\fi
%    \end{macrocode}

% Process as usual until here:
%    \begin{macrocode}
\fi
%    \end{macrocode}

% End of document body:
%    \begin{macrocode}
\end{document}
%    \end{macrocode}
%\iffalse
%</samplemain>
%\fi
%
% %%%%%%%%%%%%%%%%%%%%%%%%%%%%%%%%%%%%%%
% \paragraph{Chapter Include Files.}
%
% The include files are called |cdocsch1.tex| and |cdocsch2.tex|.
%
%\iffalse
%<*samplechap1|samplechap2>
%\fi

% Optional override for |\version| flag:
%    \begin{macrocode}
%%\providecommand{\version}{final}
%    \end{macrocode}

% Include the main document:
%    \begin{macrocode}
\input{childdoc.def}
\childdocof{cdocsamp}
%    \end{macrocode}

%\iffalse
%</samplechap1|samplechap2>
%\fi
%
%\iffalse
%<*samplechap1>
%\fi
% Some text for chapter 1:
%    \begin{macrocode}
\section{one}
some text in chapter one
%    \end{macrocode}

%\iffalse
%</samplechap1>
%\fi
% Some text for chapter 2:
%\iffalse
%<*samplechap2>
%\fi
%    \begin{macrocode}
\section{two}
more text in chapter two
%    \end{macrocode}

%\iffalse
%</samplechap2>
%\fi
%
% %%%%%%%%%%%%%%%%%%%%%%%%%%%%%%%%%%%%%%
% \paragraph{Part Include Files.}
%
% The include files are called |cdocspt3.tex| and |cdocspt4.tex|.
%
%\iffalse
%<*samplepart3|samplepart4>
%\fi

% Optional override for |\version| flag:
%    \begin{macrocode}
%%\providecommand{\version}{final}
%    \end{macrocode}

% Include the main document:
%    \begin{macrocode}
\input{childdoc.def}
\childdocby{cdocsamp}
%    \end{macrocode}

%\iffalse
%</samplepart3|samplepart4>
%\fi
%
%\iffalse
%<*samplepart3>
%\fi
% Some text for part 3:
%    \begin{macrocode}
some text in part three
%    \end{macrocode}

%\iffalse
%</samplepart3>
%\fi
% Some text for part 4:
%\iffalse
%<*samplepart4>
%\fi
%    \begin{macrocode}
more text in part four
%    \end{macrocode}

%\iffalse
%</samplepart4>
%\fi
%
% %%%%%%%%%%%%%%%%%%%%%%%%%%%%%%%%%%%%%%
% \paragraph{Forwarding for a Complete Draft.}
%
% The following forwarding file |cdocsdrf.tex|
% compiles the main document in draft mode:
%\iffalse
%<*sampledraft>
%\fi
%    \begin{macrocode}
\def\version{draft}
\input{childdoc.def}
\childdocforward{cdocsamp}
%    \end{macrocode}

%\iffalse
%</sampledraft>
%\fi
%
% %%%%%%%%%%%%%%%%%%%%%%%%%%%%%%%%%%%%%%
% \paragraph{Forwarding for Final Version of the Chapters.}
%
% The following forwarding files |cdocsfn1.tex| and |cdocsfn2.tex|
% (with identical content)
% compile the final versions of the child documents
% |cdocsch1.tex| and |cdocsch2.tex|, respectively:
%\iffalse
%<*samplefinal>
%\fi
%    \begin{macrocode}
\def\version{final}
\input{childdoc.def}
\childdocforwardprefix[cdocsamp]{cdocsfn}{cdocsch}
%    \end{macrocode}

%\iffalse
%</samplefinal>
%\fi
%
% %%%%%%%%%%%%%%%%%%%%%%%%%%%%%%%%%%%%%%
% \paragraph{Command Line Processing.}
%
% The following three command lines generate the output files
% |cdocscld|, |cdocscl1| and |cdocscl2|
% which should be identical to
% |cdocsdrf|, |cdocsch1| and |cdocsfn2|, respectively:
% \begin{center}
% \begin{tabular}{l}
% |latex -jobname cdocscld \|\\
% |  "\def\version{draft}\input{childdoc.def}\childdocforward{cdocsamp}"|\\
% |latex -jobname cdocscl1 \|\\
% |  "\input{childdoc.def}\childdocforward[cdocsamp]{cdocsch1}"|\\
% |latex -jobname cdocscl2 \|\\
% |  "\def\version{final}\input{childdoc.def}\childdocforward{cdocsch2}"|
% \end{tabular}
% \end{center}
% Note that the trailing backslash on each first line
% merely continues the input to the second line
% (for convenient cut ant paste).
% Furthermore, the command |latex| can be replaced by any
% of its alternative versions such as |pdflatex|.
%
% %%%%%%%%%%%%%%%%%%%%%%%%%%%%%%%%%%%%%%%%%%%%%%%%%%%%%%%%%%%%%%%%%%%%%%%%%%%%%%
% %%%%%%%%%%%%%%%%%%%%%%%%%%%%%%%%%%%%%%%%%%%%%%%%%%%%%%%%%%%%%%%%%%%%%%%%%%%%%%
% \section{Implementation}
%\iffalse
%<*package>
%\fi
%
% This section describes the definitions file |childdoc.def|.

% The definitions cannot be loaded using |\usepackage| or |\RequirePackage|
% which has a mechanism to prevent loading a style file more than once.
% When loading the definitions by means of |\input|
% multiple instances have to be prevented manually:
%\iffalse
%This code needs to be before the `\ProvidesFile' directive
%which is defined at the beginning of this file.
%Therefore it is also placed there and commented out here.
%</package>
%<*discard>
%\fi
%    \begin{macrocode}
\ifdefined\childdocmain\endinput\fi
%    \end{macrocode}
%\iffalse
%</discard>
%<*package>
%\fi
%
% \macro{\ifchilddoc}
% \macro{\ifchilddocmanual}
% The conditional |\ifchilddoc| tells whether a
% child (true) or main (false) document is being compiled.
% The conditional |\ifchilddocmanual| tells whether
% the |\includeonly| mechanism is used (false) or
% the selection of child files must be performed manually (true).
% The definitions initialise to false:
%    \begin{macrocode}
\newif\ifchilddoc
\newif\ifchilddocmanual
%    \end{macrocode}

% \macro{\childdocname}
% \macro{\childdocjob}
% The macro |\childdocname| stores the name of the main document
% to be compiled. The macro |\childdocjob| stores the name of
% the document on which the \LaTeX{} compiler was originally invoked.
% The content of |\jobname| cannot be compared
% to filenames specified in the source due to different catcodes.
% The following code rescans |\jobname|, stores the result
% in |\childdocname| and saves a copy in |\childdocjob|:
%    \begin{macrocode}
\edef\childdocname{\scantokens\expandafter{\jobname\noexpand}}
\let\childdocjob\childdocname
%    \end{macrocode}

% \macro{\childdocdisable}
% The macro |\childdocdisable| prevents the main file
% from being processed more than once.
% At this stage, the main document command |\childdocmain|
% is assumed to be called once again where it should do nothing.
% Any subsequent call to it should prevent
% a secondary processing of the main document
% It overwrites the forwarding commands
% |\childdocof| and |\childdocforward|
% with empty macros to prevent further inclusions of the main document:
%    \begin{macrocode}
\newcommand{\childdocdisable}
{
  \renewcommand{\childdocmain}[1]{\renewcommand{\childdocmain}[1]{\endinput}}
  \renewcommand{\childdocof}[1]{}
  \renewcommand{\childdocby}[2][]{}
  \renewcommand{\childdocforward}[2][]{}
  \renewcommand{\childdocdisable}{}
}
%    \end{macrocode}

% \macro{\childdocmain}
% The macro |\childdocmain| is to be called at the top of the main file
% with nothing or the main filename (without extension) as argument.
% First, it breaks loops.
% If the argument is not empty and does not match |\childdocname|
% (which is set by the first inclusion of |childdoc.def|),
% |\ifchilddoc| is set to true, |\includeonly| is applied to the child file
% and |\jobname| is set to the main file
% (for proper handling of |.aux| files):
%    \begin{macrocode}
\newcommand{\childdocmain}[1]
{
  \childdocdisable\childdocmain{}
  \if?#1?\else
    \begingroup
      \def\childdoctmp{#1}
      \ifx\childdoctmp\childdocname
        \def\childdoctmp{}
      \else
        \def\childdoctmp
        {
          \childdoctrue
          \includeonly{\childdocname}
          \def\childdocjob{#1}
          \def\jobname{#1}
        }
      \fi
      \expandafter
    \endgroup
    \childdoctmp
  \fi
}
%    \end{macrocode}

% \macro{\childdocof}
% The command |\childdocof| redirects
% compilation to the main file |#1|.
%    \begin{macrocode}
\newcommand{\childdocof}[1]
{
  \childdocdisable
  \childdoctrue
  \includeonly{\childdocname}
  \def\jobname{#1}
  \def\childdocjob{#1}
  \input{#1}
}
%    \end{macrocode}

% \macro{\childdocby}
% The command |\childdocby| ....
%    \begin{macrocode}
\newcommand{\childdocby}[2][]
{
  \childdocdisable
  \childdoctrue
  \childdocmanualtrue
  \if?#1?\else
    \def\jobname{#2}
  \fi
  \def\childdocjob{#2}
  \input{#2}
  \endinput
}
%    \end{macrocode}

% \macro{\childdocforward}
% The command |\childdocforward| redirects
% compilation to the main file or
% (if the optional argument is given) a child file.
% Parameters are set as if the main file
% or a child file starting with |\childdocof| was compiled.
% Then compilation is handed over to the main file:
%    \begin{macrocode}
\newcommand{\childdocforward}[2][]
{
  \begingroup
    \if?#1?
      \def\childdoctmp
      {
        \def\childdocname{#2}
        \def\childdocjob{#2}
        \def\jobname{#2}
        \input{#2}
        \endinput
      }
    \else
      \def\childdoctmp
      {
        \childdocdisable
        \def\childdocname{#2}
        \childdoctrue
        \includeonly{#2}
        \def\childdocjob{#1}
        \def\jobname{#1}
        \input{#1}
        \endinput
      }
    \fi
    \expandafter
  \endgroup
  \childdoctmp
}
%    \end{macrocode}

% \macro{\childdocforwardprefix}
% The command |\childdocforwardprefix| redirects
% compilation to the main or a child file by means of a pattern.
% The prefix |#1| in the current filename is replaced by |#2|
% and the suffix of the current filename is kept
% (it is assumed that the filename does not contain the substring `|~~~|'
% which is used as a delimiter).
% Compilation is handed over to the new file by |\childdocforward|:
%    \begin{macrocode}
\newcommand{\childdocforwardprefix}[3][]
{
  \begingroup
    \def\childdocextract #2##1~~~{\def\childdoctmp{\childdocforward[#1]{#3##1}}}
    \expandafter\childdocextract\childdocname~~~
    \expandafter
  \endgroup
  \childdoctmp
}
%    \end{macrocode}

% \macro{\childdoc}
% The deprecated macro |\childdoc| is a legacy version of |\childdocmain|:
%    \begin{macrocode}
\newcommand{\childdoc}{\childdocmain}
%    \end{macrocode}

% \macro{\childdocredirect}
% The deprecated macro |\childdocredirect| is a legacy version
% of |\childdocforward| and |\childdocforwardprefix|:
%    \begin{macrocode}
\newcommand{\childdocredirect}[2][]
{
  \begingroup
    \if?#1?
      \def\childdoctmp{\childdocforward{#2}}
    \else
      \def\childdoctmp{\childdocforwardprefix{#1}{#2}}
    \fi
    \expandafter
  \endgroup
  \childdoctmp
}
%    \end{macrocode}

%\iffalse
%</package>
%\fi
%
\endinput
|\\
|\childdocforward{|\textit{main}|}|
\end{tabular}
\end{center}
%
Likewise, the following files |final|\textit{nn}|.tex|
compile the final version of the child document
|child|\textit{nn}|.tex|:
%
\begin{center}
\begin{tabular}{l}
|\def\version{final}|\\
|% \iffalse
%
% childdoc.dtx Copyright (C) 2017-2018 Niklas Beisert
%
% This work may be distributed and/or modified under the
% conditions of the LaTeX Project Public License, either version 1.3
% of this license or (at your option) any later version.
% The latest version of this license is in
%   http://www.latex-project.org/lppl.txt
% and version 1.3 or later is part of all distributions of LaTeX
% version 2005/12/01 or later.
%
% This work has the LPPL maintenance status `maintained'.
%
% The Current Maintainer of this work is Niklas Beisert.
%
% This work consists of the files childdoc.dtx and childdoc.ins
% and the derived files childdoc.def and cdocsamp.tex with
% cdocsch1.tex, cdocsch2.tex, cdocsdrf.tex, cdocsfn1.tex, cdocsfn2.tex.
%
%<package>\ifdefined\childdocmain\endinput\fi
%<package>\ProvidesFile{childdoc.def}[2018/12/30 v2.0 child document driver]
%<samplemain>\ProvidesFile{cdocsamp.tex}[2018/12/30 v2.0 sample for childdoc]
%<*driver>
%\ProvidesFile{childdoc.drv}[2018/12/30 v2.0 childdoc reference manual file]
\PassOptionsToClass{10pt,a4paper}{article}
\documentclass{ltxdoc}

\usepackage[margin=35mm]{geometry}
\usepackage{hyperref}
\usepackage{hyperxmp}
\usepackage[usenames]{color}

\hypersetup{colorlinks=true}
\hypersetup{pdfstartview=FitH}
\hypersetup{pdfpagemode=UseNone}
\hypersetup{pdfsource={}}
\hypersetup{pdflang={en-UK}}
\hypersetup{pdfcopyright={Copyright 2017-2018 Niklas Beisert.
  This work may be distributed and/or modified under the
  conditions of the LaTeX Project Public License, either version 1.3
  of this license or (at your option) any later version.}}
\hypersetup{pdflicenseurl={http://www.latex-project.org/lppl.txt}}
\hypersetup{pdfcontactaddress={ETH Zurich, ITP, HIT K,
  Wolfgang-Pauli-Strasse 27}}
\hypersetup{pdfcontactpostcode={8093}}
\hypersetup{pdfcontactcity={Zurich}}
\hypersetup{pdfcontactcountry={Switzerland}}
\hypersetup{pdfcontactemail={nbeisert@itp.phys.ethz.ch}}
\hypersetup{pdfcontacturl={http://people.phys.ethz.ch/\xmptilde nbeisert/}}

\newcommand{\secref}[1]{\hyperref[#1]{section \ref*{#1}}}

\parskip1ex
\parindent0pt
\let\olditemize\itemize
\def\itemize{\olditemize\parskip0pt}

\begin{document}

\title{The \textsf{childdoc} Package}
\hypersetup{pdftitle={The childdoc Package}}
\author{Niklas Beisert\\[2ex]
  Institut f\"ur Theoretische Physik\\
  Eidgen\"ossische Technische Hochschule Z\"urich\\
  Wolfgang-Pauli-Strasse 27, 8093 Z\"urich, Switzerland\\[1ex]
  \href{mailto:nbeisert@itp.phys.ethz.ch}
  {\texttt{nbeisert@itp.phys.ethz.ch}}}
\hypersetup{pdfauthor={Niklas Beisert}}
\hypersetup{pdfsubject={Manual for the LaTeX2e Package childdoc}}
\date{30 December 2018, \textsf{v2.0}}
\maketitle

\begin{abstract}\noindent
\textsf{childdoc} is a \LaTeXe{} package
that enables the direct compilation
of document sections included by |\include|
to individual files.
\end{abstract}

\begingroup
\parskip0ex
\tableofcontents
\endgroup

%%%%%%%%%%%%%%%%%%%%%%%%%%%%%%%%%%%%%%%%%%%%%%%%%%%%%%%%%%%%%%%%%%%%%%%%%%%%%%%%
%%%%%%%%%%%%%%%%%%%%%%%%%%%%%%%%%%%%%%%%%%%%%%%%%%%%%%%%%%%%%%%%%%%%%%%%%%%%%%%%
\section{Introduction}

\LaTeX{} provides a mechanism to structure a large document (such as a book)
into a main file and several child files (containing the chapters)
using the |\include| command.
This mechanism is beneficial for documents
which span hundreds of pages in order to
make the source file(s) more manageable.
Moreover, compilation can be restricted to
selected child files by means of the |\includeonly| command.
The latter feature can be used to reduce the compilation time while editing
(this was significantly more useful in the earlier days of \LaTeX{})
or to generate a smaller document which is easier to navigate.
Another application of |\includeonly| is to generate
documents consisting of selected parts of the complete document.

However, there are a few drawbacks of the plain |\include| mechanism:
\begin{itemize}
\item
The child files cannot be compiled on their own,
they can only be compiled via the main file.
A naive editing environment
(such as a text editor with an option
to have the current file processed by \LaTeX)
may require one to switch to the main file before compiling;
attempting to compile the child file produces errors.
\item
The main file must be modified (each time)
to adjust the |\includeonly| command
to the present needs. This easily leaves the main file in a messy state.
\item
The generated document will always carry the filename
of the main document. This is inconvenient if
several child files are to be compiled and
to be kept for distribution.
\end{itemize}

The present package provides a simple interface
to make child files individually compilable by \LaTeX{}.
Compiling a child file then has the same effect as compiling
the main file with an |\includeonly| command
to select the appropriate child.
Moreover the generated document will carry the name of the child
rather than the main file.
This resolves all three above issues.

This feature is meant to make the editing of books,
thesis documents and lecture notes somewhat more convenient.
However, the package can also be used efficiently for
composing a series of documents (such as exercise sheets)
which are typically distributed individually.
It then assists the author in generating the individual documents
(potentially in different versions)
as well as a document containing the collected series.
Another application is in developing style files
or other kinds of included material
where compilation of the style file could redirect
to a sample or test file.

%%%%%%%%%%%%%%%%%%%%%%%%%%%%%%%%%%%%%%%%%%%%%%%%%%%%%%%%%%%%%%%%%%%%%%%%%%%%%%%%
%%%%%%%%%%%%%%%%%%%%%%%%%%%%%%%%%%%%%%%%%%%%%%%%%%%%%%%%%%%%%%%%%%%%%%%%%%%%%%%%
\section{Usage}

First of all, the package \textsf{childdoc} is \emph{not} a standard
\LaTeXe{} |.sty| style file! Therefore it needs to be invoked in
a non-standard way.

%%%%%%%%%%%%%%%%%%%%%%%%%%%%%%%%%%%%%%%%%%%%%%%%%%%%%%%%%%%%%%%%%%%%%%%%%%%%%%%%
\subsection{Included Files}
\label{sec:include}

%%%%%%%%%%%%%%%%%%%%%%%%%%%%%%%%%%%%%%%%
\DescribeMacro{\childdocmain}
To use the package, add the commands
\begin{center}
\begin{tabular}{l}
|\input{childdoc.def}|\\
|\childdocmain{}|\\
\end{tabular}
\end{center}
at the very top of the main \LaTeX{} file,
in particular \emph{before} the |\documentclass| statement!
The argument of |\childdocmain| should be left empty
(but it must be present).

%%%%%%%%%%%%%%%%%%%%%%%%%%%%%%%%%%%%%%%%
\DescribeMacro{\childdocof}
Furthermore, add the commands
\begin{center}
\begin{tabular}{l}
|\input{childdoc.def}|\\
|\childdocof{|\textit{main}|}|\\
\end{tabular}
\end{center}
at the top of every child file \textit{child}
which is included by |\include{|\textit{child}|}|
from within the main file
(or at least for those files to be compiled individually).
The argument \textit{main} must be the filename of the main file.

There are a couple of
considerations in setting up the main and child documents:

%%%%%%%%%%%%%%%%%%%%%%%%%%%%%%%%%%%%%%%%
\paragraph{Restrictions.}

Please note the following restrictions:
\begin{itemize}
\item
|\childdocmain| must be called with one argument \textit{main}
to ensure compatibility with earlier version of the package.
It must either be empty (|\childdocmain{}|)
or precisely match the filename of the main file in which it is specified.
See \secref{sec:detection} for further information.
\item
The filename \textit{main} must be specified without the |.tex| extension.
\item
The filename \textit{main} is case sensitive
(even in case-insensitive file systems)
due to internal string comparison.
\item
The argument \textit{main} should be fully expanded, it cannot be a macro.
\item
Subdirectories and special characters should be avoided in filenames.
\item
The command |\childdocmain{|\textit{main}|}| must be followed by a whitespace.
It should not be followed immediately by another command
or by a comment mark `|%|'.
This is because the \TeX{} parser reads the token immediately following
the argument of |\childdocmain| and puts it
at the beginning of every child section;
however, a white\-space is ignored.
\end{itemize}

%%%%%%%%%%%%%%%%%%%%%%%%%%%%%%%%%%%%%%%%
\paragraph{Content of Main File.}

It is advisable to place all content in the child files included by |\include|.
Any output contained in the main file will appear in all child documents
unless suppressed manually;
it cannot be suppressed automatically by the |\includeonly| directive
and thus should normally be avoided.
A method to include some content in the main file
by means of conditional processing is described in \secref{sec:conditional}.

%%%%%%%%%%%%%%%%%%%%%%%%%%%%%%%%%%%%%%%%
\paragraph{Page Numbering.}

When only a part of the document is compiled,
the appropriate numbering of pages
(as well as other status parameters)
is determined from the |.aux| files.
The latter contain information from previous passes.
However this information needs to propagate through
all intermediate child documents.
Therefore the page numbering in child documents may well
be inconsistent until the complete document is compiled at least once.

A useful (if unconventional) way to always ensure a consistent
page numbering is to restart the numbering in each child document
and denote the pages by `\textit{child}|.|\textit{page}'
where \textit{child} represents the chapter/section number of the child file.
This can be achieved by the command
|\numberwithin{page}{|\textit{child}|}|
of the \textsf{amsmath} package
where \textit{child} can be |chapter| or |section|
depending on the chosen structuring.
Alternatively, one can modify the macro |\thepage| appropriately
and reset the counter |page| at the start of each child file.

%%%%%%%%%%%%%%%%%%%%%%%%%%%%%%%%%%%%%%%%%%%%%%%%%%%%%%%%%%%%%%%%%%%%%%%%%%%%%%%%
\subsection{Conditional Processing}
\label{sec:conditional}

The package provides a mechanism to compile different versions
of a document. To customise the versions further some conditional processing
can come in handy to distinguish which version is being compiled.
The package provides two macros to describe the compilation context:

%%%%%%%%%%%%%%%%%%%%%%%%%%%%%%%%%%%%%%%%
\DescribeMacro{\ifchilddoc}
The conditional |\ifchilddoc| distinguishes between the compilation of
child documents and the main document:
%
\begin{center}
|\ifchilddoc |\textit{child-code}| |[|\||else |\textit{main-code}]| \||fi|
\end{center}

%%%%%%%%%%%%%%%%%%%%%%%%%%%%%%%%%%%%%%%%
\DescribeMacro{\childdocname}
\DescribeMacro{\childdocjob}
The macro |\childdocname| contains the filename (without extension)
of the main or child file being processed.
Note that |\childdocjob| will always contain the name of the main file.

%%%%%%%%%%%%%%%%%%%%%%%%%%%%%%%%%%%%%%%%
\paragraph{Title Page.}

Conditional processing can be used to include a title or banner page
in the main document when proper precautions are taken.
Importantly, the code in the main file should ensure that the page counter
(as well as other status parameters which are stored in the |.aux| files)
takes the same value after the conditional processing.
Otherwise the page numbers may take divergent values
depending on which part is compiled.

For example, a title page could be declared by:
%
\begin{center}
\begin{tabular}{l}
|\ifchilddoc\||else|\\
|\addtocounter{page}{-1}|\\
\textit{code for title page}\\
|\newpage|\\
|\||fi|
\end{tabular}
\end{center}
%
A banner page for the child documents can be generated by:
%
\begin{center}
\begin{tabular}{l}
|\ifchilddoc|\\
|\addtocounter{page}{-1}|\\
\textit{code for banner page}\\
|\newpage|\\
|\||fi|
\end{tabular}
\end{center}
%
Here one could write a message such as:
\begin{center}
|This is the part \childdocname{} of \childdocjob{}.|
\end{center}

%%%%%%%%%%%%%%%%%%%%%%%%%%%%%%%%%%%%%%%%%%%%%%%%%%%%%%%%%%%%%%%%%%%%%%%%%%%%%%%%
\subsection{Flags}
\label{sec:flags}

The package makes it easy to generate different versions
of the main or child documents.
To this end compilation flags can be defined
and assigned different default values.
They will be particularly useful in conjunction
with the forwarding mechanism described in \secref{sec:forward}.

For example, it may be useful to have a flag |\version|
which can be set to |draft| or |final|.
The document source will contain some conditional code
depending on the value of |\version|.
Suppose further, the flag should default to |final| for the main file
and to |draft| for child files
which is a natural assignment for editing the document.
This is achieved by placing the following code
in the preamble of the main document
(below the |\childdocmain| directive):
%
\begin{center}
\begin{tabular}{l}
|\ifchilddoc|\\
|\providecommand{\version}{draft}|\\
|\||else|\\
|\providecommand{\version}{final}|\\
|\||fi|
\end{tabular}
\end{center}
%
The definition by |\providecommand| makes sure
that previous definitions are not overwritten.
Further statements |\providecommand{\version}{...}|
can thus be added before the above code to override it.

For the main file, one might add a line
(between |\childdocmain| and the above block)
%
\begin{center}
|%\ifchilddoc\||else\providecommand{\version}{draft}\||fi|
\end{center}
%
which can be uncommented to produce a draft version.
Likewise one can add a line to the very top of a child file
(above the |\childdocof{|\textit{main}|}| directive)
%
\begin{center}
|%\providecommand{\version}{final}|
\end{center}
%
which can be uncommented to produce the final version of this child document.

%%%%%%%%%%%%%%%%%%%%%%%%%%%%%%%%%%%%%%%%%%%%%%%%%%%%%%%%%%%%%%%%%%%%%%%%%%%%%%%%
\subsection{Forwarding}
\label{sec:forward}

Different versions of the main or child documents
using compilation flags as described in \secref{sec:flags}
can be (permanently) stored in different files
for convenient compilation, viewing and distribution.
To this end, the package defines a command
to pass on compilation to a different file:

%%%%%%%%%%%%%%%%%%%%%%%%%%%%%%%%%%%%%%%%
\DescribeMacro{\childdocforward}
The command |\childdocforward| redirects processing to
another source file:
%
\begin{center}
\begin{tabular}{l}
|\input{childdoc.def}|\\
|\childdocforward[|\textit{main}|]{|\textit{dest}|}|\\
\end{tabular}
\end{center}
%
The argument \textit{dest} is the destination file
(without extension).
It should be the main file or one of the child files.
Note that further \textsf{childdoc} directives
such as |\childdocof| and |\childdocforward|
in the indicated file will be processed in this form.
The optional argument \textit{main}
passes on directly to the main file \textit{main}
while pretending to compile the child \textit{dest}.
This form behaves as if \textit{dest}
issues |\childdocof{|\textit{main}|}| right away,
and no further \textsf{childdoc} directives will be processed.

%%%%%%%%%%%%%%%%%%%%%%%%%%%%%%%%%%%%%%%%
\DescribeMacro{\...prefix}
In the alternative form |\childdocforwardprefix|,
%
\begin{center}
\begin{tabular}{l}
|\input{childdoc.def}|\\
|\childdocforwardprefix[|\textit{main}|]{|\textit{prefix}|}{|\textit{dest}|}|
\end{tabular}
\end{center}
%
the destination file is determined by a pattern
depending on the current file:
To make this work, the current file must be called
`{\textit{prefix}\hspace{0.2em}\textit{suffix}}'
with \textit{prefix} matching precisely the argument.
Processing is then passed on to the file
`{\textit{dest}\hspace{0.2em}\textit{suffix}}'.
Surely, the same effect is achieved by
directly specifying the
argument `{\textit{dest}\hspace{0.2em}\textit{suffix}}'
in the first form.
However, that requires to set up a different file
for each child. With the alternative form of the command
all these files can have exactly the same content
which simplifies setting them up and maintaining them.

For example, the following file |draft.tex|
with a compilation flag |\version| as described in \secref{sec:flags}
compiles the main document as a draft:
%
\begin{center}
\begin{tabular}{l}
|\def\version{draft}|\\
|\input{childdoc.def}|\\
|\childdocforward{|\textit{main}|}|
\end{tabular}
\end{center}
%
Likewise, the following files |final|\textit{nn}|.tex|
compile the final version of the child document
|child|\textit{nn}|.tex|:
%
\begin{center}
\begin{tabular}{l}
|\def\version{final}|\\
|\input{childdoc.def}|\\
|\childdocforwardprefix{final}{child}|
\end{tabular}
\end{center}
%

Note that when several versions of a main file and/or of each child file
are to be generated, it may be convenient to set up a |Makefile| or
shell script to automatise the process.

%%%%%%%%%%%%%%%%%%%%%%%%%%%%%%%%%%%%%%%%%%%%%%%%%%%%%%%%%%%%%%%%%%%%%%%%%%%%%%%%
\subsection{Command Line Processing}
\label{sec:commandline}

The effect of redirection files can also be achieved by invoking
the \LaTeX{} compiler with a more elaborate command line.
Most conveniently this should be done as part
of a shell script or a |Makefile|.

When using \textsf{childdoc} in the main file, the following
command lines effectively perform a redirection
(note that depending on the shell being used,
backslashes may have to be doubled: `|\|' $\to$ `|\\|'):
%
\begin{center}
|... -jobname "|\textit{target}|" |\\|"|[\textit{flags}]%
|\input{childdoc.def}\childdocforward[|\textit{main}|]{|\textit{dest}|}"|
\end{center}
%
Here \textit{target} is the name of the output file,
\textit{main} is the name of the main file
and \textit{dest} is the name of the main or child file to be processed
(all filenames without extensions).
The optional argument \textit{main} can be omitted
if \textit{main} matches \textit{dest}.
Optionally, compilation \textit{flags} can be defined via |\def| commands.
This command line makes the \TeX{} engine believe
it is compiling the file \textit{target}
whose content is specified as the latter parameter.
The provided code then forwards the processing to
\textit{main} or \textit{dest} as described in \secref{sec:forward}.

%%%%%%%%%%%%%%%%%%%%%%%%%%%%%%%%%%%%%%%%%%%%%%%%%%%%%%%%%%%%%%%%%%%%%%%%%%%%%%%%
\subsection{Include by Input}
\label{sec:input}

Including child documents by |\include| has some restrictions by design.
Most notably, the content of a child document always occupies
its own set of pages; pages cannot be shared between child documents.
Usually, this behaviour makes perfect sense
because each child document contain an essential part of the document.
However, in some situations it may be desirable to compose
a document from a collection of parts
without having mandatory page breaks between then.
For this case, the package
provides a mechanism to include parts
by |\input| which can also be processed individually.
However, by construction this mechanism
requires manual handling of the content to be output.

%%%%%%%%%%%%%%%%%%%%%%%%%%%%%%%%%%%%%%%%
\DescribeMacro{\ifchilddocmanual}
The main file should be prepared as usual, see \secref{sec:include}.
However, the document body must make a distinction
between processing of an individual part and of the main document, e.g.:
%
\begin{center}
\begin{tabular}{l}
|\ifchilddocmanual|\\
|\input{\childdocname}|\\
|\||else|\\
\textit{document body with }|\input{|\textit{part}|}|\\
|\||fi|
\end{tabular}
\end{center}
%
The conditional |\ifchilddocmanual| is true whenever
a part to be included by |\input| is being compiled,
and the name of the part is stored in |\childdocname|.

%%%%%%%%%%%%%%%%%%%%%%%%%%%%%%%%%%%%%%%%
\DescribeMacro{\childdocby}
Each part to be included by |\input| should start with:
%
\begin{center}
\begin{tabular}{l}
|\input{childdoc.def}|\\
|\childdocby{|\textit{main}|}|\\
\end{tabular}
\end{center}
%
The directive |\childdocby| is similar to |\childdocof|
described in \secref{sec:include},
but the subsequent selection of content must be done manually.
To that end, both |\ifchilddoc| and |\ifchilddocmanual|
will be true upon processing of a part,
and the name of the part is stored in |\childdocname|.
Note that |\jobname| will be set to the filename of the current part
so that each part receives an individual |.aux| file
that does not interfere with the |.aux| file(s) of the main document.
This behaviour can be altered by the alternative form
|\childdocby[*]{|\textit{main}|}| (with a non-empty optional argument)
which uses the |.aux| file of the main document
by setting |\jobname| to \textit{main}.

%%%%%%%%%%%%%%%%%%%%%%%%%%%%%%%%%%%%%%%%%%%%%%%%%%%%%%%%%%%%%%%%%%%%%%%%%%%%%%%%
\subsection{Driver Development}
\label{sec:driver}

The \textsf{childdoc} mechanism can also be use for the development
of definition files such as \LaTeX{} styles or classes.
This case differs from the above setup with multiple parts
included by |\include| in that no |\includeonly| should be invoked.
This can be achieved by starting the include file
(before |\ProvidesPackage|) with:
%
\begin{center}
\begin{tabular}{l}
|\input{childdoc.def}|\\
|\childdocforward{|\textit{main}|}|\\
\end{tabular}
\end{center}
%
or alternatively with:
%
\begin{center}
\begin{tabular}{l}
|\input{childdoc.def}|\\
|\childdocby{|\textit{main}|}|\\
\end{tabular}
\end{center}
%
Both forms have slightly different effects as described above.
The main file is prepared as usual, see \secref{sec:include}.

%%%%%%%%%%%%%%%%%%%%%%%%%%%%%%%%%%%%%%%%%%%%%%%%%%%%%%%%%%%%%%%%%%%%%%%%%%%%%%%%
\subsection{Legacy Detection}
\label{sec:detection}

The directive |\childdocmain| in the main file can detect
whether the complete document or merely a child is to be compiled
even without using the directive |\childdocof|.
This method is deprecated because it is less robust
and there is no compelling reason to use it;
it is merely provided for backward compatibility
and it may be removed in future versions.

If the detection mechanism is to be used,
it is mandatory to correctly specify
the filename of the main file as the argument of |\childdocmain|:
%
\begin{center}
\begin{tabular}{l}
|\input{childdoc.def}|\\
|\childdocmain{|\textit{main}|}|\\
\end{tabular}
\end{center}
%
If |\jobname| does not match the argument \textit{main} of |\childdocmain|,
it is assumed that |\jobname| points to the child file to be compiled.
When using |\childdocmain| with the main file specified as argument,
it suffices to start a child file
with just |\input{|\textit{main}|}|
without loading of the package and using |\childdocof|.
If instead all processing is done
with the appropriate \textsf{childdoc} directives,
the argument of \textit{main} of |\childdocmain| can be empty.

An alternative version of the command line processing described
in \secref{sec:commandline} using the detection mechanism reads:
%
\begin{center}
|... -jobname "|\textit{target}|" "|[\textit{flags}]%
[|\def\jobname{|\textit{dest}|}|]|\input{|\textit{main}|}"|
\end{center}

%%%%%%%%%%%%%%%%%%%%%%%%%%%%%%%%%%%%%%%%%%%%%%%%%%%%%%%%%%%%%%%%%%%%%%%%%%%%%%%%
\subsection{Manual Code}
\label{sec:manual}

In case one cannot be certain whether the definitions file |childdoc.def|
is installed on the target \TeX{} distribution
and one prefers not to ship it,
it is conceivable to paste a few relevant commands into the sources.

To that end, drop all statements |\input{childdoc.def}|
and perform the replacements as outlined below.
Instead of |\childdocmain{|\textit{main}|}| add the following code
to the top of the main file:
%
\begin{center}
\begin{tabular}{l}
|\||ifdefined\childdocname\endinput\||fi\newif\ifchilddoc|\\
|\edef\childdocname{\scantokens\expandafter{\jobname\noexpand}}|\\
|\def\childdocmain{|\textit{main}|}\||ifx\childdocmain\childdocname\||else|\\
|\childdoctrue\includeonly{\childdocname}\let\jobname\childdocmain\||fi|\\
\end{tabular}
\end{center}
%
Instead of |\childdocof{|\textit{main}|}| just include the main file
at the top of each child file:
%
\begin{center}
|\input{|\textit{main}|}|
\end{center}
%
A simple redirection |\childdocforward{|\textit{dest}|}| is achieved by:
%
\begin{center}
|\def\jobname{|\textit{dest}|}\input{\jobname}|
\end{center}
%
The redirection with prefix
|\childdocforwardprefix[|\textit{prefix}|]{|\textit{dest}|}|
is accomplished by:
%
\begin{center}
\begin{tabular}{l}
|{\edef\jobname{\scantokens\expandafter{\jobname\noexpand}}|\\
|\def\redirectjob |\textit{prefix}|#1~~~{\gdef\jobname{|\textit{dest}|#1}}|\\
|\expandafter\redirectjob\jobname~~~}\input{\jobname}|
\end{tabular}
\end{center}

In an alternative approach,
child documents can be compiled by a specific command line
without additional code or specific definitions:
%
\begin{center}
|... -jobname "|\textit{target}|" "|[\textit{flags}]%
|\includeonly{|\textit{dest}|}\input{|\textit{main}|}"|
\end{center}
%

%%%%%%%%%%%%%%%%%%%%%%%%%%%%%%%%%%%%%%%%%%%%%%%%%%%%%%%%%%%%%%%%%%%%%%%%%%%%%%%%
%%%%%%%%%%%%%%%%%%%%%%%%%%%%%%%%%%%%%%%%%%%%%%%%%%%%%%%%%%%%%%%%%%%%%%%%%%%%%%%%
\section{Information}

%%%%%%%%%%%%%%%%%%%%%%%%%%%%%%%%%%%%%%%%%%%%%%%%%%%%%%%%%%%%%%%%%%%%%%%%%%%%%%%%
\subsection{Copyright}

Copyright \copyright{} 2017--2018 Niklas Beisert

This work may be distributed and/or modified under the
conditions of the \LaTeX{} Project Public License, either version 1.3
of this license or (at your option) any later version.
The latest version of this license is in
  \url{http://www.latex-project.org/lppl.txt}
and version 1.3 or later is part of all distributions of \LaTeX{}
version 2005/12/01 or later.

This work has the LPPL maintenance status `maintained'.

The Current Maintainer of this work is Niklas Beisert.

This work consists of the files |README.txt|, |childdoc.ins| and |childdoc.dtx|
as well as the derived files |childdoc.def|, |cdocsamp.tex|
with |cdocsch1.tex|, |cdocsch2.tex|, |cdocspt3.tex|, |cdocspt4.tex|,
|cdocsdrf.tex|, |cdocsfn1.tex|, |cdocsfn2.tex|
as well as |childdoc.pdf|.

%%%%%%%%%%%%%%%%%%%%%%%%%%%%%%%%%%%%%%%%%%%%%%%%%%%%%%%%%%%%%%%%%%%%%%%%%%%%%%%%
\subsection{Files and Installation}

The package consists of the files:
%
\begin{center}
\begin{tabular}{ll}
    |README.txt|   & readme file \\
    |childdoc.ins| & installation file \\
    |childdoc.dtx| & source file \\
    |childdoc.def| & definition file \\
    |cdocsamp.tex| & sample main file \\
    |cdocsch1.tex| & sample include file \\
    |cdocsch2.tex| & sample include file \\
    |cdocspt3.tex| & sample part file \\
    |cdocspt4.tex| & sample part file \\
    |cdocsdrf.tex| & sample redirection file \\
    |cdocsfn1.tex| & sample redirection file \\
    |cdocsfn2.tex| & sample redirection file \\
    |childdoc.pdf| & manual
\end{tabular}
\end{center}
%
The distribution consists of the files
|README.txt|, |childdoc.ins| and |childdoc.dtx|.
%
\begin{itemize}
\item
Run (pdf)\LaTeX{} on |childdoc.dtx|
to compile the manual |childdoc.pdf| (this file).
\item
Run \LaTeX{} on |childdoc.ins| to create the definitions file |childdoc.def|
and the sample |cdocsamp.tex| with include files
|cdocsch1.tex|, |cdocsch2.tex|, |cdocspt3.tex|, |cdocspt4.tex|,
|cdocsdrf.tex|, |cdocsfn1.tex|, |cdocsfn2.tex|.
Then copy the file |childdoc.def| to an appropriate directory of your \LaTeX{}
distribution, e.g.\ \textit{texmf-root}|/tex/latex/childdoc|.
\end{itemize}

%%%%%%%%%%%%%%%%%%%%%%%%%%%%%%%%%%%%%%%%%%%%%%%%%%%%%%%%%%%%%%%%%%%%%%%%%%%%%%%%
\subsection{Related CTAN Packages}

There are several other packages which offer a similar functionality:
%
\begin{itemize}
\item
The packages
\href{http://ctan.org/pkg/docmute}{\textsf{docmute}},
\href{http://ctan.org/pkg/includex}{\textsf{includex}} and
\href{http://ctan.org/pkg/standalone}{\textsf{standalone}}
provide commands to include only the document body of
a child file thus allowing both files to be compiled individually.
\item
The packages \href{http://ctan.org/pkg/subdocs}{\textsf{subdocs}}
and \href{http://ctan.org/pkg/subfiles}{\textsf{subfiles}}
provide structures in which the main and child documents can be
encapsulated and allowing them to be compiled individually.
The inclusion mechanism is different from the conventional |\include|.
\item
The package \href{http://ctan.org/pkg/combine}{\textsf{combine}}
is an elaborate solution to combine several documents into one.
\end{itemize}
%
See also the CTAN topic \href{http://ctan.org/topic/subdocs}{\textsf{subdocs}}
for further related packages.
The present package differs from the above solutions in that
a document structure constructed with the conventional |\include| mechanism
just needs two extra commands at the top of every file
such that all constituent files can be compiled individually.

%%%%%%%%%%%%%%%%%%%%%%%%%%%%%%%%%%%%%%%%%%%%%%%%%%%%%%%%%%%%%%%%%%%%%%%%%%%%%%%%
%\subsection{Feature Suggestions}
%
%The following is a list of features which may be useful for future
%versions of this package:
%%
%\begin{itemize}
%\item
%\ldots
%\end{itemize}

%%%%%%%%%%%%%%%%%%%%%%%%%%%%%%%%%%%%%%%%%%%%%%%%%%%%%%%%%%%%%%%%%%%%%%%%%%%%%%%%
\subsection{Revision History}

%%%%%%%%%%%%%%%%%%%%%%%%%%%%%%%%%%%%%%%%
\paragraph{v2.0:} 2018/12/30

\begin{itemize}
\item
immediate forward processing
\item
added |\childdocby| mechanism
\item
manual restructured
\end{itemize}

%%%%%%%%%%%%%%%%%%%%%%%%%%%%%%%%%%%%%%%%
\paragraph{v1.6:} 2018/01/17

\begin{itemize}
\item
application for development of include files
\item
corrections to manual
\end{itemize}

%%%%%%%%%%%%%%%%%%%%%%%%%%%%%%%%%%%%%%%%
\paragraph{v1.5:} 2017/05/21

\begin{itemize}
\item
more complete structuring introduced
\item
|\childdocof| introduced
\item
|\childdoc| renamed to |\childdocmain|
\item
|\childredirect| renamed to |\childdocforward| and |\childdocforwardprefix|
and functionality expanded
\end{itemize}

%%%%%%%%%%%%%%%%%%%%%%%%%%%%%%%%%%%%%%%%
\paragraph{v1.0:} 2017/04/27

\begin{itemize}
\item
manual and install package
\item
first version published on CTAN
\end{itemize}

%%%%%%%%%%%%%%%%%%%%%%%%%%%%%%%%%%%%%%%%
\paragraph{v0.6:} 2017/04/26

\begin{itemize}
\item
redirection mechanism added
\end{itemize}

%%%%%%%%%%%%%%%%%%%%%%%%%%%%%%%%%%%%%%%%
\paragraph{v0.5:} 2017/04/26

\begin{itemize}
\item
functionality in definition file
\end{itemize}


%%%%%%%%%%%%%%%%%%%%%%%%%%%%%%%%%%%%%%%%%%%%%%%%%%%%%%%%%%%%%%%%%%%%%%%%%%%%%%%%
%%%%%%%%%%%%%%%%%%%%%%%%%%%%%%%%%%%%%%%%%%%%%%%%%%%%%%%%%%%%%%%%%%%%%%%%%%%%%%%%
%%%%%%%%%%%%%%%%%%%%%%%%%%%%%%%%%%%%%%%%%%%%%%%%%%%%%%%%%%%%%%%%%%%%%%%%%%%%%%%%
\appendix

\settowidth\MacroIndent{\rmfamily\scriptsize 000\ }

 \DocInput{childdoc.dtx}

\end{document}
%</driver>
% \fi
%
% %%%%%%%%%%%%%%%%%%%%%%%%%%%%%%%%%%%%%%%%%%%%%%%%%%%%%%%%%%%%%%%%%%%%%%%%%%%%%%
% %%%%%%%%%%%%%%%%%%%%%%%%%%%%%%%%%%%%%%%%%%%%%%%%%%%%%%%%%%%%%%%%%%%%%%%%%%%%%%
% \section{Sample}
%\iffalse
%<*samplemain>
%\fi
%
% The following presents a sample document
% with two chapters, two parts, a title page,
% a compile flag as well as three forwarding files to set the flag.
% It consists of eight |.tex| files:
% \begin{center}
% \begin{tabular}{ll}
% |cdocsamp.tex|&main file\\
% |cdocsch1.tex|&include file for chapter 1\\
% |cdocsch2.tex|&include file for chapter 2\\
% |cdocspt3.tex|&include file for part 3\\
% |cdocspt4.tex|&include file for part 4\\
% |cdocsdrf.tex|&forwarding file for main file in draft mode\\
% |cdocsfi1.tex|&forwarding file for final version of chapter 1\\
% |cdocsfi2.tex|&forwarding file for final version of chapter 2\\
% \end{tabular}
% \end{center}
% Each of the eight files can be compiled directly by the \LaTeX{} compiler.
%
% %%%%%%%%%%%%%%%%%%%%%%%%%%%%%%%%%%%%%%
% \paragraph{Main File.}
%
% The main file is called |cdocsamp.tex|.
%
% Load the \textsf{childdoc} definitions and
% declare the filename for the main document:
%    \begin{macrocode}
\input{childdoc.def}
\childdocmain{}
%    \end{macrocode}

% Optional override for |\version| flag:
%    \begin{macrocode}
%%\ifchilddoc\else\providecommand{\version}{draft}\fi
%    \end{macrocode}

% Define the default values for the |\version| flag
% (|final| for the main file and |draft| for childs):
%    \begin{macrocode}
\ifchilddoc
\providecommand{\version}{draft}
\else
\providecommand{\version}{final}
\fi
%    \end{macrocode}

% Load the standard document class:
%    \begin{macrocode}
\documentclass[12pt]{article}
%    \end{macrocode}

% Start the document body:
%    \begin{macrocode}
\begin{document}
%    \end{macrocode}

% Declare a title page.
% Print title, part of document being processed and version flag:
%    \begin{macrocode}
\addtocounter{page}{-1}
\begin{center}
{\LARGE\bfseries{}childdoc example\par}
\vspace{1cm}
\ifchilddoc
\ifchilddocmanual part\else chapter\fi:
`\childdocname' of `\childdocjob'\par
\else
main document: `\childdocjob'\par
\fi
version: \version\par
\end{center}
\newpage
%    \end{macrocode}

% Manually include selected file,
% otherwise process as usual:
%    \begin{macrocode}
\ifchilddocmanual
\section*{part `\childdocname'}
\input{\childdocname}
\else
%    \end{macrocode}

% Include the two chapters:
%    \begin{macrocode}
\include{cdocsch1}
\include{cdocsch2}
%    \end{macrocode}

% Include the two parts unless only chapters should be displayed:
%    \begin{macrocode}
\ifchilddoc\else
\section{part three}
\input{cdocspt3}
\section{part four}
\input{cdocspt4}
\fi
%    \end{macrocode}

% Process as usual until here:
%    \begin{macrocode}
\fi
%    \end{macrocode}

% End of document body:
%    \begin{macrocode}
\end{document}
%    \end{macrocode}
%\iffalse
%</samplemain>
%\fi
%
% %%%%%%%%%%%%%%%%%%%%%%%%%%%%%%%%%%%%%%
% \paragraph{Chapter Include Files.}
%
% The include files are called |cdocsch1.tex| and |cdocsch2.tex|.
%
%\iffalse
%<*samplechap1|samplechap2>
%\fi

% Optional override for |\version| flag:
%    \begin{macrocode}
%%\providecommand{\version}{final}
%    \end{macrocode}

% Include the main document:
%    \begin{macrocode}
\input{childdoc.def}
\childdocof{cdocsamp}
%    \end{macrocode}

%\iffalse
%</samplechap1|samplechap2>
%\fi
%
%\iffalse
%<*samplechap1>
%\fi
% Some text for chapter 1:
%    \begin{macrocode}
\section{one}
some text in chapter one
%    \end{macrocode}

%\iffalse
%</samplechap1>
%\fi
% Some text for chapter 2:
%\iffalse
%<*samplechap2>
%\fi
%    \begin{macrocode}
\section{two}
more text in chapter two
%    \end{macrocode}

%\iffalse
%</samplechap2>
%\fi
%
% %%%%%%%%%%%%%%%%%%%%%%%%%%%%%%%%%%%%%%
% \paragraph{Part Include Files.}
%
% The include files are called |cdocspt3.tex| and |cdocspt4.tex|.
%
%\iffalse
%<*samplepart3|samplepart4>
%\fi

% Optional override for |\version| flag:
%    \begin{macrocode}
%%\providecommand{\version}{final}
%    \end{macrocode}

% Include the main document:
%    \begin{macrocode}
\input{childdoc.def}
\childdocby{cdocsamp}
%    \end{macrocode}

%\iffalse
%</samplepart3|samplepart4>
%\fi
%
%\iffalse
%<*samplepart3>
%\fi
% Some text for part 3:
%    \begin{macrocode}
some text in part three
%    \end{macrocode}

%\iffalse
%</samplepart3>
%\fi
% Some text for part 4:
%\iffalse
%<*samplepart4>
%\fi
%    \begin{macrocode}
more text in part four
%    \end{macrocode}

%\iffalse
%</samplepart4>
%\fi
%
% %%%%%%%%%%%%%%%%%%%%%%%%%%%%%%%%%%%%%%
% \paragraph{Forwarding for a Complete Draft.}
%
% The following forwarding file |cdocsdrf.tex|
% compiles the main document in draft mode:
%\iffalse
%<*sampledraft>
%\fi
%    \begin{macrocode}
\def\version{draft}
\input{childdoc.def}
\childdocforward{cdocsamp}
%    \end{macrocode}

%\iffalse
%</sampledraft>
%\fi
%
% %%%%%%%%%%%%%%%%%%%%%%%%%%%%%%%%%%%%%%
% \paragraph{Forwarding for Final Version of the Chapters.}
%
% The following forwarding files |cdocsfn1.tex| and |cdocsfn2.tex|
% (with identical content)
% compile the final versions of the child documents
% |cdocsch1.tex| and |cdocsch2.tex|, respectively:
%\iffalse
%<*samplefinal>
%\fi
%    \begin{macrocode}
\def\version{final}
\input{childdoc.def}
\childdocforwardprefix[cdocsamp]{cdocsfn}{cdocsch}
%    \end{macrocode}

%\iffalse
%</samplefinal>
%\fi
%
% %%%%%%%%%%%%%%%%%%%%%%%%%%%%%%%%%%%%%%
% \paragraph{Command Line Processing.}
%
% The following three command lines generate the output files
% |cdocscld|, |cdocscl1| and |cdocscl2|
% which should be identical to
% |cdocsdrf|, |cdocsch1| and |cdocsfn2|, respectively:
% \begin{center}
% \begin{tabular}{l}
% |latex -jobname cdocscld \|\\
% |  "\def\version{draft}\input{childdoc.def}\childdocforward{cdocsamp}"|\\
% |latex -jobname cdocscl1 \|\\
% |  "\input{childdoc.def}\childdocforward[cdocsamp]{cdocsch1}"|\\
% |latex -jobname cdocscl2 \|\\
% |  "\def\version{final}\input{childdoc.def}\childdocforward{cdocsch2}"|
% \end{tabular}
% \end{center}
% Note that the trailing backslash on each first line
% merely continues the input to the second line
% (for convenient cut ant paste).
% Furthermore, the command |latex| can be replaced by any
% of its alternative versions such as |pdflatex|.
%
% %%%%%%%%%%%%%%%%%%%%%%%%%%%%%%%%%%%%%%%%%%%%%%%%%%%%%%%%%%%%%%%%%%%%%%%%%%%%%%
% %%%%%%%%%%%%%%%%%%%%%%%%%%%%%%%%%%%%%%%%%%%%%%%%%%%%%%%%%%%%%%%%%%%%%%%%%%%%%%
% \section{Implementation}
%\iffalse
%<*package>
%\fi
%
% This section describes the definitions file |childdoc.def|.

% The definitions cannot be loaded using |\usepackage| or |\RequirePackage|
% which has a mechanism to prevent loading a style file more than once.
% When loading the definitions by means of |\input|
% multiple instances have to be prevented manually:
%\iffalse
%This code needs to be before the `\ProvidesFile' directive
%which is defined at the beginning of this file.
%Therefore it is also placed there and commented out here.
%</package>
%<*discard>
%\fi
%    \begin{macrocode}
\ifdefined\childdocmain\endinput\fi
%    \end{macrocode}
%\iffalse
%</discard>
%<*package>
%\fi
%
% \macro{\ifchilddoc}
% \macro{\ifchilddocmanual}
% The conditional |\ifchilddoc| tells whether a
% child (true) or main (false) document is being compiled.
% The conditional |\ifchilddocmanual| tells whether
% the |\includeonly| mechanism is used (false) or
% the selection of child files must be performed manually (true).
% The definitions initialise to false:
%    \begin{macrocode}
\newif\ifchilddoc
\newif\ifchilddocmanual
%    \end{macrocode}

% \macro{\childdocname}
% \macro{\childdocjob}
% The macro |\childdocname| stores the name of the main document
% to be compiled. The macro |\childdocjob| stores the name of
% the document on which the \LaTeX{} compiler was originally invoked.
% The content of |\jobname| cannot be compared
% to filenames specified in the source due to different catcodes.
% The following code rescans |\jobname|, stores the result
% in |\childdocname| and saves a copy in |\childdocjob|:
%    \begin{macrocode}
\edef\childdocname{\scantokens\expandafter{\jobname\noexpand}}
\let\childdocjob\childdocname
%    \end{macrocode}

% \macro{\childdocdisable}
% The macro |\childdocdisable| prevents the main file
% from being processed more than once.
% At this stage, the main document command |\childdocmain|
% is assumed to be called once again where it should do nothing.
% Any subsequent call to it should prevent
% a secondary processing of the main document
% It overwrites the forwarding commands
% |\childdocof| and |\childdocforward|
% with empty macros to prevent further inclusions of the main document:
%    \begin{macrocode}
\newcommand{\childdocdisable}
{
  \renewcommand{\childdocmain}[1]{\renewcommand{\childdocmain}[1]{\endinput}}
  \renewcommand{\childdocof}[1]{}
  \renewcommand{\childdocby}[2][]{}
  \renewcommand{\childdocforward}[2][]{}
  \renewcommand{\childdocdisable}{}
}
%    \end{macrocode}

% \macro{\childdocmain}
% The macro |\childdocmain| is to be called at the top of the main file
% with nothing or the main filename (without extension) as argument.
% First, it breaks loops.
% If the argument is not empty and does not match |\childdocname|
% (which is set by the first inclusion of |childdoc.def|),
% |\ifchilddoc| is set to true, |\includeonly| is applied to the child file
% and |\jobname| is set to the main file
% (for proper handling of |.aux| files):
%    \begin{macrocode}
\newcommand{\childdocmain}[1]
{
  \childdocdisable\childdocmain{}
  \if?#1?\else
    \begingroup
      \def\childdoctmp{#1}
      \ifx\childdoctmp\childdocname
        \def\childdoctmp{}
      \else
        \def\childdoctmp
        {
          \childdoctrue
          \includeonly{\childdocname}
          \def\childdocjob{#1}
          \def\jobname{#1}
        }
      \fi
      \expandafter
    \endgroup
    \childdoctmp
  \fi
}
%    \end{macrocode}

% \macro{\childdocof}
% The command |\childdocof| redirects
% compilation to the main file |#1|.
%    \begin{macrocode}
\newcommand{\childdocof}[1]
{
  \childdocdisable
  \childdoctrue
  \includeonly{\childdocname}
  \def\jobname{#1}
  \def\childdocjob{#1}
  \input{#1}
}
%    \end{macrocode}

% \macro{\childdocby}
% The command |\childdocby| ....
%    \begin{macrocode}
\newcommand{\childdocby}[2][]
{
  \childdocdisable
  \childdoctrue
  \childdocmanualtrue
  \if?#1?\else
    \def\jobname{#2}
  \fi
  \def\childdocjob{#2}
  \input{#2}
  \endinput
}
%    \end{macrocode}

% \macro{\childdocforward}
% The command |\childdocforward| redirects
% compilation to the main file or
% (if the optional argument is given) a child file.
% Parameters are set as if the main file
% or a child file starting with |\childdocof| was compiled.
% Then compilation is handed over to the main file:
%    \begin{macrocode}
\newcommand{\childdocforward}[2][]
{
  \begingroup
    \if?#1?
      \def\childdoctmp
      {
        \def\childdocname{#2}
        \def\childdocjob{#2}
        \def\jobname{#2}
        \input{#2}
        \endinput
      }
    \else
      \def\childdoctmp
      {
        \childdocdisable
        \def\childdocname{#2}
        \childdoctrue
        \includeonly{#2}
        \def\childdocjob{#1}
        \def\jobname{#1}
        \input{#1}
        \endinput
      }
    \fi
    \expandafter
  \endgroup
  \childdoctmp
}
%    \end{macrocode}

% \macro{\childdocforwardprefix}
% The command |\childdocforwardprefix| redirects
% compilation to the main or a child file by means of a pattern.
% The prefix |#1| in the current filename is replaced by |#2|
% and the suffix of the current filename is kept
% (it is assumed that the filename does not contain the substring `|~~~|'
% which is used as a delimiter).
% Compilation is handed over to the new file by |\childdocforward|:
%    \begin{macrocode}
\newcommand{\childdocforwardprefix}[3][]
{
  \begingroup
    \def\childdocextract #2##1~~~{\def\childdoctmp{\childdocforward[#1]{#3##1}}}
    \expandafter\childdocextract\childdocname~~~
    \expandafter
  \endgroup
  \childdoctmp
}
%    \end{macrocode}

% \macro{\childdoc}
% The deprecated macro |\childdoc| is a legacy version of |\childdocmain|:
%    \begin{macrocode}
\newcommand{\childdoc}{\childdocmain}
%    \end{macrocode}

% \macro{\childdocredirect}
% The deprecated macro |\childdocredirect| is a legacy version
% of |\childdocforward| and |\childdocforwardprefix|:
%    \begin{macrocode}
\newcommand{\childdocredirect}[2][]
{
  \begingroup
    \if?#1?
      \def\childdoctmp{\childdocforward{#2}}
    \else
      \def\childdoctmp{\childdocforwardprefix{#1}{#2}}
    \fi
    \expandafter
  \endgroup
  \childdoctmp
}
%    \end{macrocode}

%\iffalse
%</package>
%\fi
%
\endinput
|\\
|\childdocforwardprefix{final}{child}|
\end{tabular}
\end{center}
%

Note that when several versions of a main file and/or of each child file
are to be generated, it may be convenient to set up a |Makefile| or
shell script to automatise the process.

%%%%%%%%%%%%%%%%%%%%%%%%%%%%%%%%%%%%%%%%%%%%%%%%%%%%%%%%%%%%%%%%%%%%%%%%%%%%%%%%
\subsection{Command Line Processing}
\label{sec:commandline}

The effect of redirection files can also be achieved by invoking
the \LaTeX{} compiler with a more elaborate command line.
Most conveniently this should be done as part
of a shell script or a |Makefile|.

When using \textsf{childdoc} in the main file, the following
command lines effectively perform a redirection
(note that depending on the shell being used,
backslashes may have to be doubled: `|\|' $\to$ `|\\|'):
%
\begin{center}
|... -jobname "|\textit{target}|" |\\|"|[\textit{flags}]%
|% \iffalse
%
% childdoc.dtx Copyright (C) 2017-2018 Niklas Beisert
%
% This work may be distributed and/or modified under the
% conditions of the LaTeX Project Public License, either version 1.3
% of this license or (at your option) any later version.
% The latest version of this license is in
%   http://www.latex-project.org/lppl.txt
% and version 1.3 or later is part of all distributions of LaTeX
% version 2005/12/01 or later.
%
% This work has the LPPL maintenance status `maintained'.
%
% The Current Maintainer of this work is Niklas Beisert.
%
% This work consists of the files childdoc.dtx and childdoc.ins
% and the derived files childdoc.def and cdocsamp.tex with
% cdocsch1.tex, cdocsch2.tex, cdocsdrf.tex, cdocsfn1.tex, cdocsfn2.tex.
%
%<package>\ifdefined\childdocmain\endinput\fi
%<package>\ProvidesFile{childdoc.def}[2018/12/30 v2.0 child document driver]
%<samplemain>\ProvidesFile{cdocsamp.tex}[2018/12/30 v2.0 sample for childdoc]
%<*driver>
%\ProvidesFile{childdoc.drv}[2018/12/30 v2.0 childdoc reference manual file]
\PassOptionsToClass{10pt,a4paper}{article}
\documentclass{ltxdoc}

\usepackage[margin=35mm]{geometry}
\usepackage{hyperref}
\usepackage{hyperxmp}
\usepackage[usenames]{color}

\hypersetup{colorlinks=true}
\hypersetup{pdfstartview=FitH}
\hypersetup{pdfpagemode=UseNone}
\hypersetup{pdfsource={}}
\hypersetup{pdflang={en-UK}}
\hypersetup{pdfcopyright={Copyright 2017-2018 Niklas Beisert.
  This work may be distributed and/or modified under the
  conditions of the LaTeX Project Public License, either version 1.3
  of this license or (at your option) any later version.}}
\hypersetup{pdflicenseurl={http://www.latex-project.org/lppl.txt}}
\hypersetup{pdfcontactaddress={ETH Zurich, ITP, HIT K,
  Wolfgang-Pauli-Strasse 27}}
\hypersetup{pdfcontactpostcode={8093}}
\hypersetup{pdfcontactcity={Zurich}}
\hypersetup{pdfcontactcountry={Switzerland}}
\hypersetup{pdfcontactemail={nbeisert@itp.phys.ethz.ch}}
\hypersetup{pdfcontacturl={http://people.phys.ethz.ch/\xmptilde nbeisert/}}

\newcommand{\secref}[1]{\hyperref[#1]{section \ref*{#1}}}

\parskip1ex
\parindent0pt
\let\olditemize\itemize
\def\itemize{\olditemize\parskip0pt}

\begin{document}

\title{The \textsf{childdoc} Package}
\hypersetup{pdftitle={The childdoc Package}}
\author{Niklas Beisert\\[2ex]
  Institut f\"ur Theoretische Physik\\
  Eidgen\"ossische Technische Hochschule Z\"urich\\
  Wolfgang-Pauli-Strasse 27, 8093 Z\"urich, Switzerland\\[1ex]
  \href{mailto:nbeisert@itp.phys.ethz.ch}
  {\texttt{nbeisert@itp.phys.ethz.ch}}}
\hypersetup{pdfauthor={Niklas Beisert}}
\hypersetup{pdfsubject={Manual for the LaTeX2e Package childdoc}}
\date{30 December 2018, \textsf{v2.0}}
\maketitle

\begin{abstract}\noindent
\textsf{childdoc} is a \LaTeXe{} package
that enables the direct compilation
of document sections included by |\include|
to individual files.
\end{abstract}

\begingroup
\parskip0ex
\tableofcontents
\endgroup

%%%%%%%%%%%%%%%%%%%%%%%%%%%%%%%%%%%%%%%%%%%%%%%%%%%%%%%%%%%%%%%%%%%%%%%%%%%%%%%%
%%%%%%%%%%%%%%%%%%%%%%%%%%%%%%%%%%%%%%%%%%%%%%%%%%%%%%%%%%%%%%%%%%%%%%%%%%%%%%%%
\section{Introduction}

\LaTeX{} provides a mechanism to structure a large document (such as a book)
into a main file and several child files (containing the chapters)
using the |\include| command.
This mechanism is beneficial for documents
which span hundreds of pages in order to
make the source file(s) more manageable.
Moreover, compilation can be restricted to
selected child files by means of the |\includeonly| command.
The latter feature can be used to reduce the compilation time while editing
(this was significantly more useful in the earlier days of \LaTeX{})
or to generate a smaller document which is easier to navigate.
Another application of |\includeonly| is to generate
documents consisting of selected parts of the complete document.

However, there are a few drawbacks of the plain |\include| mechanism:
\begin{itemize}
\item
The child files cannot be compiled on their own,
they can only be compiled via the main file.
A naive editing environment
(such as a text editor with an option
to have the current file processed by \LaTeX)
may require one to switch to the main file before compiling;
attempting to compile the child file produces errors.
\item
The main file must be modified (each time)
to adjust the |\includeonly| command
to the present needs. This easily leaves the main file in a messy state.
\item
The generated document will always carry the filename
of the main document. This is inconvenient if
several child files are to be compiled and
to be kept for distribution.
\end{itemize}

The present package provides a simple interface
to make child files individually compilable by \LaTeX{}.
Compiling a child file then has the same effect as compiling
the main file with an |\includeonly| command
to select the appropriate child.
Moreover the generated document will carry the name of the child
rather than the main file.
This resolves all three above issues.

This feature is meant to make the editing of books,
thesis documents and lecture notes somewhat more convenient.
However, the package can also be used efficiently for
composing a series of documents (such as exercise sheets)
which are typically distributed individually.
It then assists the author in generating the individual documents
(potentially in different versions)
as well as a document containing the collected series.
Another application is in developing style files
or other kinds of included material
where compilation of the style file could redirect
to a sample or test file.

%%%%%%%%%%%%%%%%%%%%%%%%%%%%%%%%%%%%%%%%%%%%%%%%%%%%%%%%%%%%%%%%%%%%%%%%%%%%%%%%
%%%%%%%%%%%%%%%%%%%%%%%%%%%%%%%%%%%%%%%%%%%%%%%%%%%%%%%%%%%%%%%%%%%%%%%%%%%%%%%%
\section{Usage}

First of all, the package \textsf{childdoc} is \emph{not} a standard
\LaTeXe{} |.sty| style file! Therefore it needs to be invoked in
a non-standard way.

%%%%%%%%%%%%%%%%%%%%%%%%%%%%%%%%%%%%%%%%%%%%%%%%%%%%%%%%%%%%%%%%%%%%%%%%%%%%%%%%
\subsection{Included Files}
\label{sec:include}

%%%%%%%%%%%%%%%%%%%%%%%%%%%%%%%%%%%%%%%%
\DescribeMacro{\childdocmain}
To use the package, add the commands
\begin{center}
\begin{tabular}{l}
|\input{childdoc.def}|\\
|\childdocmain{}|\\
\end{tabular}
\end{center}
at the very top of the main \LaTeX{} file,
in particular \emph{before} the |\documentclass| statement!
The argument of |\childdocmain| should be left empty
(but it must be present).

%%%%%%%%%%%%%%%%%%%%%%%%%%%%%%%%%%%%%%%%
\DescribeMacro{\childdocof}
Furthermore, add the commands
\begin{center}
\begin{tabular}{l}
|\input{childdoc.def}|\\
|\childdocof{|\textit{main}|}|\\
\end{tabular}
\end{center}
at the top of every child file \textit{child}
which is included by |\include{|\textit{child}|}|
from within the main file
(or at least for those files to be compiled individually).
The argument \textit{main} must be the filename of the main file.

There are a couple of
considerations in setting up the main and child documents:

%%%%%%%%%%%%%%%%%%%%%%%%%%%%%%%%%%%%%%%%
\paragraph{Restrictions.}

Please note the following restrictions:
\begin{itemize}
\item
|\childdocmain| must be called with one argument \textit{main}
to ensure compatibility with earlier version of the package.
It must either be empty (|\childdocmain{}|)
or precisely match the filename of the main file in which it is specified.
See \secref{sec:detection} for further information.
\item
The filename \textit{main} must be specified without the |.tex| extension.
\item
The filename \textit{main} is case sensitive
(even in case-insensitive file systems)
due to internal string comparison.
\item
The argument \textit{main} should be fully expanded, it cannot be a macro.
\item
Subdirectories and special characters should be avoided in filenames.
\item
The command |\childdocmain{|\textit{main}|}| must be followed by a whitespace.
It should not be followed immediately by another command
or by a comment mark `|%|'.
This is because the \TeX{} parser reads the token immediately following
the argument of |\childdocmain| and puts it
at the beginning of every child section;
however, a white\-space is ignored.
\end{itemize}

%%%%%%%%%%%%%%%%%%%%%%%%%%%%%%%%%%%%%%%%
\paragraph{Content of Main File.}

It is advisable to place all content in the child files included by |\include|.
Any output contained in the main file will appear in all child documents
unless suppressed manually;
it cannot be suppressed automatically by the |\includeonly| directive
and thus should normally be avoided.
A method to include some content in the main file
by means of conditional processing is described in \secref{sec:conditional}.

%%%%%%%%%%%%%%%%%%%%%%%%%%%%%%%%%%%%%%%%
\paragraph{Page Numbering.}

When only a part of the document is compiled,
the appropriate numbering of pages
(as well as other status parameters)
is determined from the |.aux| files.
The latter contain information from previous passes.
However this information needs to propagate through
all intermediate child documents.
Therefore the page numbering in child documents may well
be inconsistent until the complete document is compiled at least once.

A useful (if unconventional) way to always ensure a consistent
page numbering is to restart the numbering in each child document
and denote the pages by `\textit{child}|.|\textit{page}'
where \textit{child} represents the chapter/section number of the child file.
This can be achieved by the command
|\numberwithin{page}{|\textit{child}|}|
of the \textsf{amsmath} package
where \textit{child} can be |chapter| or |section|
depending on the chosen structuring.
Alternatively, one can modify the macro |\thepage| appropriately
and reset the counter |page| at the start of each child file.

%%%%%%%%%%%%%%%%%%%%%%%%%%%%%%%%%%%%%%%%%%%%%%%%%%%%%%%%%%%%%%%%%%%%%%%%%%%%%%%%
\subsection{Conditional Processing}
\label{sec:conditional}

The package provides a mechanism to compile different versions
of a document. To customise the versions further some conditional processing
can come in handy to distinguish which version is being compiled.
The package provides two macros to describe the compilation context:

%%%%%%%%%%%%%%%%%%%%%%%%%%%%%%%%%%%%%%%%
\DescribeMacro{\ifchilddoc}
The conditional |\ifchilddoc| distinguishes between the compilation of
child documents and the main document:
%
\begin{center}
|\ifchilddoc |\textit{child-code}| |[|\||else |\textit{main-code}]| \||fi|
\end{center}

%%%%%%%%%%%%%%%%%%%%%%%%%%%%%%%%%%%%%%%%
\DescribeMacro{\childdocname}
\DescribeMacro{\childdocjob}
The macro |\childdocname| contains the filename (without extension)
of the main or child file being processed.
Note that |\childdocjob| will always contain the name of the main file.

%%%%%%%%%%%%%%%%%%%%%%%%%%%%%%%%%%%%%%%%
\paragraph{Title Page.}

Conditional processing can be used to include a title or banner page
in the main document when proper precautions are taken.
Importantly, the code in the main file should ensure that the page counter
(as well as other status parameters which are stored in the |.aux| files)
takes the same value after the conditional processing.
Otherwise the page numbers may take divergent values
depending on which part is compiled.

For example, a title page could be declared by:
%
\begin{center}
\begin{tabular}{l}
|\ifchilddoc\||else|\\
|\addtocounter{page}{-1}|\\
\textit{code for title page}\\
|\newpage|\\
|\||fi|
\end{tabular}
\end{center}
%
A banner page for the child documents can be generated by:
%
\begin{center}
\begin{tabular}{l}
|\ifchilddoc|\\
|\addtocounter{page}{-1}|\\
\textit{code for banner page}\\
|\newpage|\\
|\||fi|
\end{tabular}
\end{center}
%
Here one could write a message such as:
\begin{center}
|This is the part \childdocname{} of \childdocjob{}.|
\end{center}

%%%%%%%%%%%%%%%%%%%%%%%%%%%%%%%%%%%%%%%%%%%%%%%%%%%%%%%%%%%%%%%%%%%%%%%%%%%%%%%%
\subsection{Flags}
\label{sec:flags}

The package makes it easy to generate different versions
of the main or child documents.
To this end compilation flags can be defined
and assigned different default values.
They will be particularly useful in conjunction
with the forwarding mechanism described in \secref{sec:forward}.

For example, it may be useful to have a flag |\version|
which can be set to |draft| or |final|.
The document source will contain some conditional code
depending on the value of |\version|.
Suppose further, the flag should default to |final| for the main file
and to |draft| for child files
which is a natural assignment for editing the document.
This is achieved by placing the following code
in the preamble of the main document
(below the |\childdocmain| directive):
%
\begin{center}
\begin{tabular}{l}
|\ifchilddoc|\\
|\providecommand{\version}{draft}|\\
|\||else|\\
|\providecommand{\version}{final}|\\
|\||fi|
\end{tabular}
\end{center}
%
The definition by |\providecommand| makes sure
that previous definitions are not overwritten.
Further statements |\providecommand{\version}{...}|
can thus be added before the above code to override it.

For the main file, one might add a line
(between |\childdocmain| and the above block)
%
\begin{center}
|%\ifchilddoc\||else\providecommand{\version}{draft}\||fi|
\end{center}
%
which can be uncommented to produce a draft version.
Likewise one can add a line to the very top of a child file
(above the |\childdocof{|\textit{main}|}| directive)
%
\begin{center}
|%\providecommand{\version}{final}|
\end{center}
%
which can be uncommented to produce the final version of this child document.

%%%%%%%%%%%%%%%%%%%%%%%%%%%%%%%%%%%%%%%%%%%%%%%%%%%%%%%%%%%%%%%%%%%%%%%%%%%%%%%%
\subsection{Forwarding}
\label{sec:forward}

Different versions of the main or child documents
using compilation flags as described in \secref{sec:flags}
can be (permanently) stored in different files
for convenient compilation, viewing and distribution.
To this end, the package defines a command
to pass on compilation to a different file:

%%%%%%%%%%%%%%%%%%%%%%%%%%%%%%%%%%%%%%%%
\DescribeMacro{\childdocforward}
The command |\childdocforward| redirects processing to
another source file:
%
\begin{center}
\begin{tabular}{l}
|\input{childdoc.def}|\\
|\childdocforward[|\textit{main}|]{|\textit{dest}|}|\\
\end{tabular}
\end{center}
%
The argument \textit{dest} is the destination file
(without extension).
It should be the main file or one of the child files.
Note that further \textsf{childdoc} directives
such as |\childdocof| and |\childdocforward|
in the indicated file will be processed in this form.
The optional argument \textit{main}
passes on directly to the main file \textit{main}
while pretending to compile the child \textit{dest}.
This form behaves as if \textit{dest}
issues |\childdocof{|\textit{main}|}| right away,
and no further \textsf{childdoc} directives will be processed.

%%%%%%%%%%%%%%%%%%%%%%%%%%%%%%%%%%%%%%%%
\DescribeMacro{\...prefix}
In the alternative form |\childdocforwardprefix|,
%
\begin{center}
\begin{tabular}{l}
|\input{childdoc.def}|\\
|\childdocforwardprefix[|\textit{main}|]{|\textit{prefix}|}{|\textit{dest}|}|
\end{tabular}
\end{center}
%
the destination file is determined by a pattern
depending on the current file:
To make this work, the current file must be called
`{\textit{prefix}\hspace{0.2em}\textit{suffix}}'
with \textit{prefix} matching precisely the argument.
Processing is then passed on to the file
`{\textit{dest}\hspace{0.2em}\textit{suffix}}'.
Surely, the same effect is achieved by
directly specifying the
argument `{\textit{dest}\hspace{0.2em}\textit{suffix}}'
in the first form.
However, that requires to set up a different file
for each child. With the alternative form of the command
all these files can have exactly the same content
which simplifies setting them up and maintaining them.

For example, the following file |draft.tex|
with a compilation flag |\version| as described in \secref{sec:flags}
compiles the main document as a draft:
%
\begin{center}
\begin{tabular}{l}
|\def\version{draft}|\\
|\input{childdoc.def}|\\
|\childdocforward{|\textit{main}|}|
\end{tabular}
\end{center}
%
Likewise, the following files |final|\textit{nn}|.tex|
compile the final version of the child document
|child|\textit{nn}|.tex|:
%
\begin{center}
\begin{tabular}{l}
|\def\version{final}|\\
|\input{childdoc.def}|\\
|\childdocforwardprefix{final}{child}|
\end{tabular}
\end{center}
%

Note that when several versions of a main file and/or of each child file
are to be generated, it may be convenient to set up a |Makefile| or
shell script to automatise the process.

%%%%%%%%%%%%%%%%%%%%%%%%%%%%%%%%%%%%%%%%%%%%%%%%%%%%%%%%%%%%%%%%%%%%%%%%%%%%%%%%
\subsection{Command Line Processing}
\label{sec:commandline}

The effect of redirection files can also be achieved by invoking
the \LaTeX{} compiler with a more elaborate command line.
Most conveniently this should be done as part
of a shell script or a |Makefile|.

When using \textsf{childdoc} in the main file, the following
command lines effectively perform a redirection
(note that depending on the shell being used,
backslashes may have to be doubled: `|\|' $\to$ `|\\|'):
%
\begin{center}
|... -jobname "|\textit{target}|" |\\|"|[\textit{flags}]%
|\input{childdoc.def}\childdocforward[|\textit{main}|]{|\textit{dest}|}"|
\end{center}
%
Here \textit{target} is the name of the output file,
\textit{main} is the name of the main file
and \textit{dest} is the name of the main or child file to be processed
(all filenames without extensions).
The optional argument \textit{main} can be omitted
if \textit{main} matches \textit{dest}.
Optionally, compilation \textit{flags} can be defined via |\def| commands.
This command line makes the \TeX{} engine believe
it is compiling the file \textit{target}
whose content is specified as the latter parameter.
The provided code then forwards the processing to
\textit{main} or \textit{dest} as described in \secref{sec:forward}.

%%%%%%%%%%%%%%%%%%%%%%%%%%%%%%%%%%%%%%%%%%%%%%%%%%%%%%%%%%%%%%%%%%%%%%%%%%%%%%%%
\subsection{Include by Input}
\label{sec:input}

Including child documents by |\include| has some restrictions by design.
Most notably, the content of a child document always occupies
its own set of pages; pages cannot be shared between child documents.
Usually, this behaviour makes perfect sense
because each child document contain an essential part of the document.
However, in some situations it may be desirable to compose
a document from a collection of parts
without having mandatory page breaks between then.
For this case, the package
provides a mechanism to include parts
by |\input| which can also be processed individually.
However, by construction this mechanism
requires manual handling of the content to be output.

%%%%%%%%%%%%%%%%%%%%%%%%%%%%%%%%%%%%%%%%
\DescribeMacro{\ifchilddocmanual}
The main file should be prepared as usual, see \secref{sec:include}.
However, the document body must make a distinction
between processing of an individual part and of the main document, e.g.:
%
\begin{center}
\begin{tabular}{l}
|\ifchilddocmanual|\\
|\input{\childdocname}|\\
|\||else|\\
\textit{document body with }|\input{|\textit{part}|}|\\
|\||fi|
\end{tabular}
\end{center}
%
The conditional |\ifchilddocmanual| is true whenever
a part to be included by |\input| is being compiled,
and the name of the part is stored in |\childdocname|.

%%%%%%%%%%%%%%%%%%%%%%%%%%%%%%%%%%%%%%%%
\DescribeMacro{\childdocby}
Each part to be included by |\input| should start with:
%
\begin{center}
\begin{tabular}{l}
|\input{childdoc.def}|\\
|\childdocby{|\textit{main}|}|\\
\end{tabular}
\end{center}
%
The directive |\childdocby| is similar to |\childdocof|
described in \secref{sec:include},
but the subsequent selection of content must be done manually.
To that end, both |\ifchilddoc| and |\ifchilddocmanual|
will be true upon processing of a part,
and the name of the part is stored in |\childdocname|.
Note that |\jobname| will be set to the filename of the current part
so that each part receives an individual |.aux| file
that does not interfere with the |.aux| file(s) of the main document.
This behaviour can be altered by the alternative form
|\childdocby[*]{|\textit{main}|}| (with a non-empty optional argument)
which uses the |.aux| file of the main document
by setting |\jobname| to \textit{main}.

%%%%%%%%%%%%%%%%%%%%%%%%%%%%%%%%%%%%%%%%%%%%%%%%%%%%%%%%%%%%%%%%%%%%%%%%%%%%%%%%
\subsection{Driver Development}
\label{sec:driver}

The \textsf{childdoc} mechanism can also be use for the development
of definition files such as \LaTeX{} styles or classes.
This case differs from the above setup with multiple parts
included by |\include| in that no |\includeonly| should be invoked.
This can be achieved by starting the include file
(before |\ProvidesPackage|) with:
%
\begin{center}
\begin{tabular}{l}
|\input{childdoc.def}|\\
|\childdocforward{|\textit{main}|}|\\
\end{tabular}
\end{center}
%
or alternatively with:
%
\begin{center}
\begin{tabular}{l}
|\input{childdoc.def}|\\
|\childdocby{|\textit{main}|}|\\
\end{tabular}
\end{center}
%
Both forms have slightly different effects as described above.
The main file is prepared as usual, see \secref{sec:include}.

%%%%%%%%%%%%%%%%%%%%%%%%%%%%%%%%%%%%%%%%%%%%%%%%%%%%%%%%%%%%%%%%%%%%%%%%%%%%%%%%
\subsection{Legacy Detection}
\label{sec:detection}

The directive |\childdocmain| in the main file can detect
whether the complete document or merely a child is to be compiled
even without using the directive |\childdocof|.
This method is deprecated because it is less robust
and there is no compelling reason to use it;
it is merely provided for backward compatibility
and it may be removed in future versions.

If the detection mechanism is to be used,
it is mandatory to correctly specify
the filename of the main file as the argument of |\childdocmain|:
%
\begin{center}
\begin{tabular}{l}
|\input{childdoc.def}|\\
|\childdocmain{|\textit{main}|}|\\
\end{tabular}
\end{center}
%
If |\jobname| does not match the argument \textit{main} of |\childdocmain|,
it is assumed that |\jobname| points to the child file to be compiled.
When using |\childdocmain| with the main file specified as argument,
it suffices to start a child file
with just |\input{|\textit{main}|}|
without loading of the package and using |\childdocof|.
If instead all processing is done
with the appropriate \textsf{childdoc} directives,
the argument of \textit{main} of |\childdocmain| can be empty.

An alternative version of the command line processing described
in \secref{sec:commandline} using the detection mechanism reads:
%
\begin{center}
|... -jobname "|\textit{target}|" "|[\textit{flags}]%
[|\def\jobname{|\textit{dest}|}|]|\input{|\textit{main}|}"|
\end{center}

%%%%%%%%%%%%%%%%%%%%%%%%%%%%%%%%%%%%%%%%%%%%%%%%%%%%%%%%%%%%%%%%%%%%%%%%%%%%%%%%
\subsection{Manual Code}
\label{sec:manual}

In case one cannot be certain whether the definitions file |childdoc.def|
is installed on the target \TeX{} distribution
and one prefers not to ship it,
it is conceivable to paste a few relevant commands into the sources.

To that end, drop all statements |\input{childdoc.def}|
and perform the replacements as outlined below.
Instead of |\childdocmain{|\textit{main}|}| add the following code
to the top of the main file:
%
\begin{center}
\begin{tabular}{l}
|\||ifdefined\childdocname\endinput\||fi\newif\ifchilddoc|\\
|\edef\childdocname{\scantokens\expandafter{\jobname\noexpand}}|\\
|\def\childdocmain{|\textit{main}|}\||ifx\childdocmain\childdocname\||else|\\
|\childdoctrue\includeonly{\childdocname}\let\jobname\childdocmain\||fi|\\
\end{tabular}
\end{center}
%
Instead of |\childdocof{|\textit{main}|}| just include the main file
at the top of each child file:
%
\begin{center}
|\input{|\textit{main}|}|
\end{center}
%
A simple redirection |\childdocforward{|\textit{dest}|}| is achieved by:
%
\begin{center}
|\def\jobname{|\textit{dest}|}\input{\jobname}|
\end{center}
%
The redirection with prefix
|\childdocforwardprefix[|\textit{prefix}|]{|\textit{dest}|}|
is accomplished by:
%
\begin{center}
\begin{tabular}{l}
|{\edef\jobname{\scantokens\expandafter{\jobname\noexpand}}|\\
|\def\redirectjob |\textit{prefix}|#1~~~{\gdef\jobname{|\textit{dest}|#1}}|\\
|\expandafter\redirectjob\jobname~~~}\input{\jobname}|
\end{tabular}
\end{center}

In an alternative approach,
child documents can be compiled by a specific command line
without additional code or specific definitions:
%
\begin{center}
|... -jobname "|\textit{target}|" "|[\textit{flags}]%
|\includeonly{|\textit{dest}|}\input{|\textit{main}|}"|
\end{center}
%

%%%%%%%%%%%%%%%%%%%%%%%%%%%%%%%%%%%%%%%%%%%%%%%%%%%%%%%%%%%%%%%%%%%%%%%%%%%%%%%%
%%%%%%%%%%%%%%%%%%%%%%%%%%%%%%%%%%%%%%%%%%%%%%%%%%%%%%%%%%%%%%%%%%%%%%%%%%%%%%%%
\section{Information}

%%%%%%%%%%%%%%%%%%%%%%%%%%%%%%%%%%%%%%%%%%%%%%%%%%%%%%%%%%%%%%%%%%%%%%%%%%%%%%%%
\subsection{Copyright}

Copyright \copyright{} 2017--2018 Niklas Beisert

This work may be distributed and/or modified under the
conditions of the \LaTeX{} Project Public License, either version 1.3
of this license or (at your option) any later version.
The latest version of this license is in
  \url{http://www.latex-project.org/lppl.txt}
and version 1.3 or later is part of all distributions of \LaTeX{}
version 2005/12/01 or later.

This work has the LPPL maintenance status `maintained'.

The Current Maintainer of this work is Niklas Beisert.

This work consists of the files |README.txt|, |childdoc.ins| and |childdoc.dtx|
as well as the derived files |childdoc.def|, |cdocsamp.tex|
with |cdocsch1.tex|, |cdocsch2.tex|, |cdocspt3.tex|, |cdocspt4.tex|,
|cdocsdrf.tex|, |cdocsfn1.tex|, |cdocsfn2.tex|
as well as |childdoc.pdf|.

%%%%%%%%%%%%%%%%%%%%%%%%%%%%%%%%%%%%%%%%%%%%%%%%%%%%%%%%%%%%%%%%%%%%%%%%%%%%%%%%
\subsection{Files and Installation}

The package consists of the files:
%
\begin{center}
\begin{tabular}{ll}
    |README.txt|   & readme file \\
    |childdoc.ins| & installation file \\
    |childdoc.dtx| & source file \\
    |childdoc.def| & definition file \\
    |cdocsamp.tex| & sample main file \\
    |cdocsch1.tex| & sample include file \\
    |cdocsch2.tex| & sample include file \\
    |cdocspt3.tex| & sample part file \\
    |cdocspt4.tex| & sample part file \\
    |cdocsdrf.tex| & sample redirection file \\
    |cdocsfn1.tex| & sample redirection file \\
    |cdocsfn2.tex| & sample redirection file \\
    |childdoc.pdf| & manual
\end{tabular}
\end{center}
%
The distribution consists of the files
|README.txt|, |childdoc.ins| and |childdoc.dtx|.
%
\begin{itemize}
\item
Run (pdf)\LaTeX{} on |childdoc.dtx|
to compile the manual |childdoc.pdf| (this file).
\item
Run \LaTeX{} on |childdoc.ins| to create the definitions file |childdoc.def|
and the sample |cdocsamp.tex| with include files
|cdocsch1.tex|, |cdocsch2.tex|, |cdocspt3.tex|, |cdocspt4.tex|,
|cdocsdrf.tex|, |cdocsfn1.tex|, |cdocsfn2.tex|.
Then copy the file |childdoc.def| to an appropriate directory of your \LaTeX{}
distribution, e.g.\ \textit{texmf-root}|/tex/latex/childdoc|.
\end{itemize}

%%%%%%%%%%%%%%%%%%%%%%%%%%%%%%%%%%%%%%%%%%%%%%%%%%%%%%%%%%%%%%%%%%%%%%%%%%%%%%%%
\subsection{Related CTAN Packages}

There are several other packages which offer a similar functionality:
%
\begin{itemize}
\item
The packages
\href{http://ctan.org/pkg/docmute}{\textsf{docmute}},
\href{http://ctan.org/pkg/includex}{\textsf{includex}} and
\href{http://ctan.org/pkg/standalone}{\textsf{standalone}}
provide commands to include only the document body of
a child file thus allowing both files to be compiled individually.
\item
The packages \href{http://ctan.org/pkg/subdocs}{\textsf{subdocs}}
and \href{http://ctan.org/pkg/subfiles}{\textsf{subfiles}}
provide structures in which the main and child documents can be
encapsulated and allowing them to be compiled individually.
The inclusion mechanism is different from the conventional |\include|.
\item
The package \href{http://ctan.org/pkg/combine}{\textsf{combine}}
is an elaborate solution to combine several documents into one.
\end{itemize}
%
See also the CTAN topic \href{http://ctan.org/topic/subdocs}{\textsf{subdocs}}
for further related packages.
The present package differs from the above solutions in that
a document structure constructed with the conventional |\include| mechanism
just needs two extra commands at the top of every file
such that all constituent files can be compiled individually.

%%%%%%%%%%%%%%%%%%%%%%%%%%%%%%%%%%%%%%%%%%%%%%%%%%%%%%%%%%%%%%%%%%%%%%%%%%%%%%%%
%\subsection{Feature Suggestions}
%
%The following is a list of features which may be useful for future
%versions of this package:
%%
%\begin{itemize}
%\item
%\ldots
%\end{itemize}

%%%%%%%%%%%%%%%%%%%%%%%%%%%%%%%%%%%%%%%%%%%%%%%%%%%%%%%%%%%%%%%%%%%%%%%%%%%%%%%%
\subsection{Revision History}

%%%%%%%%%%%%%%%%%%%%%%%%%%%%%%%%%%%%%%%%
\paragraph{v2.0:} 2018/12/30

\begin{itemize}
\item
immediate forward processing
\item
added |\childdocby| mechanism
\item
manual restructured
\end{itemize}

%%%%%%%%%%%%%%%%%%%%%%%%%%%%%%%%%%%%%%%%
\paragraph{v1.6:} 2018/01/17

\begin{itemize}
\item
application for development of include files
\item
corrections to manual
\end{itemize}

%%%%%%%%%%%%%%%%%%%%%%%%%%%%%%%%%%%%%%%%
\paragraph{v1.5:} 2017/05/21

\begin{itemize}
\item
more complete structuring introduced
\item
|\childdocof| introduced
\item
|\childdoc| renamed to |\childdocmain|
\item
|\childredirect| renamed to |\childdocforward| and |\childdocforwardprefix|
and functionality expanded
\end{itemize}

%%%%%%%%%%%%%%%%%%%%%%%%%%%%%%%%%%%%%%%%
\paragraph{v1.0:} 2017/04/27

\begin{itemize}
\item
manual and install package
\item
first version published on CTAN
\end{itemize}

%%%%%%%%%%%%%%%%%%%%%%%%%%%%%%%%%%%%%%%%
\paragraph{v0.6:} 2017/04/26

\begin{itemize}
\item
redirection mechanism added
\end{itemize}

%%%%%%%%%%%%%%%%%%%%%%%%%%%%%%%%%%%%%%%%
\paragraph{v0.5:} 2017/04/26

\begin{itemize}
\item
functionality in definition file
\end{itemize}


%%%%%%%%%%%%%%%%%%%%%%%%%%%%%%%%%%%%%%%%%%%%%%%%%%%%%%%%%%%%%%%%%%%%%%%%%%%%%%%%
%%%%%%%%%%%%%%%%%%%%%%%%%%%%%%%%%%%%%%%%%%%%%%%%%%%%%%%%%%%%%%%%%%%%%%%%%%%%%%%%
%%%%%%%%%%%%%%%%%%%%%%%%%%%%%%%%%%%%%%%%%%%%%%%%%%%%%%%%%%%%%%%%%%%%%%%%%%%%%%%%
\appendix

\settowidth\MacroIndent{\rmfamily\scriptsize 000\ }

 \DocInput{childdoc.dtx}

\end{document}
%</driver>
% \fi
%
% %%%%%%%%%%%%%%%%%%%%%%%%%%%%%%%%%%%%%%%%%%%%%%%%%%%%%%%%%%%%%%%%%%%%%%%%%%%%%%
% %%%%%%%%%%%%%%%%%%%%%%%%%%%%%%%%%%%%%%%%%%%%%%%%%%%%%%%%%%%%%%%%%%%%%%%%%%%%%%
% \section{Sample}
%\iffalse
%<*samplemain>
%\fi
%
% The following presents a sample document
% with two chapters, two parts, a title page,
% a compile flag as well as three forwarding files to set the flag.
% It consists of eight |.tex| files:
% \begin{center}
% \begin{tabular}{ll}
% |cdocsamp.tex|&main file\\
% |cdocsch1.tex|&include file for chapter 1\\
% |cdocsch2.tex|&include file for chapter 2\\
% |cdocspt3.tex|&include file for part 3\\
% |cdocspt4.tex|&include file for part 4\\
% |cdocsdrf.tex|&forwarding file for main file in draft mode\\
% |cdocsfi1.tex|&forwarding file for final version of chapter 1\\
% |cdocsfi2.tex|&forwarding file for final version of chapter 2\\
% \end{tabular}
% \end{center}
% Each of the eight files can be compiled directly by the \LaTeX{} compiler.
%
% %%%%%%%%%%%%%%%%%%%%%%%%%%%%%%%%%%%%%%
% \paragraph{Main File.}
%
% The main file is called |cdocsamp.tex|.
%
% Load the \textsf{childdoc} definitions and
% declare the filename for the main document:
%    \begin{macrocode}
\input{childdoc.def}
\childdocmain{}
%    \end{macrocode}

% Optional override for |\version| flag:
%    \begin{macrocode}
%%\ifchilddoc\else\providecommand{\version}{draft}\fi
%    \end{macrocode}

% Define the default values for the |\version| flag
% (|final| for the main file and |draft| for childs):
%    \begin{macrocode}
\ifchilddoc
\providecommand{\version}{draft}
\else
\providecommand{\version}{final}
\fi
%    \end{macrocode}

% Load the standard document class:
%    \begin{macrocode}
\documentclass[12pt]{article}
%    \end{macrocode}

% Start the document body:
%    \begin{macrocode}
\begin{document}
%    \end{macrocode}

% Declare a title page.
% Print title, part of document being processed and version flag:
%    \begin{macrocode}
\addtocounter{page}{-1}
\begin{center}
{\LARGE\bfseries{}childdoc example\par}
\vspace{1cm}
\ifchilddoc
\ifchilddocmanual part\else chapter\fi:
`\childdocname' of `\childdocjob'\par
\else
main document: `\childdocjob'\par
\fi
version: \version\par
\end{center}
\newpage
%    \end{macrocode}

% Manually include selected file,
% otherwise process as usual:
%    \begin{macrocode}
\ifchilddocmanual
\section*{part `\childdocname'}
\input{\childdocname}
\else
%    \end{macrocode}

% Include the two chapters:
%    \begin{macrocode}
\include{cdocsch1}
\include{cdocsch2}
%    \end{macrocode}

% Include the two parts unless only chapters should be displayed:
%    \begin{macrocode}
\ifchilddoc\else
\section{part three}
\input{cdocspt3}
\section{part four}
\input{cdocspt4}
\fi
%    \end{macrocode}

% Process as usual until here:
%    \begin{macrocode}
\fi
%    \end{macrocode}

% End of document body:
%    \begin{macrocode}
\end{document}
%    \end{macrocode}
%\iffalse
%</samplemain>
%\fi
%
% %%%%%%%%%%%%%%%%%%%%%%%%%%%%%%%%%%%%%%
% \paragraph{Chapter Include Files.}
%
% The include files are called |cdocsch1.tex| and |cdocsch2.tex|.
%
%\iffalse
%<*samplechap1|samplechap2>
%\fi

% Optional override for |\version| flag:
%    \begin{macrocode}
%%\providecommand{\version}{final}
%    \end{macrocode}

% Include the main document:
%    \begin{macrocode}
\input{childdoc.def}
\childdocof{cdocsamp}
%    \end{macrocode}

%\iffalse
%</samplechap1|samplechap2>
%\fi
%
%\iffalse
%<*samplechap1>
%\fi
% Some text for chapter 1:
%    \begin{macrocode}
\section{one}
some text in chapter one
%    \end{macrocode}

%\iffalse
%</samplechap1>
%\fi
% Some text for chapter 2:
%\iffalse
%<*samplechap2>
%\fi
%    \begin{macrocode}
\section{two}
more text in chapter two
%    \end{macrocode}

%\iffalse
%</samplechap2>
%\fi
%
% %%%%%%%%%%%%%%%%%%%%%%%%%%%%%%%%%%%%%%
% \paragraph{Part Include Files.}
%
% The include files are called |cdocspt3.tex| and |cdocspt4.tex|.
%
%\iffalse
%<*samplepart3|samplepart4>
%\fi

% Optional override for |\version| flag:
%    \begin{macrocode}
%%\providecommand{\version}{final}
%    \end{macrocode}

% Include the main document:
%    \begin{macrocode}
\input{childdoc.def}
\childdocby{cdocsamp}
%    \end{macrocode}

%\iffalse
%</samplepart3|samplepart4>
%\fi
%
%\iffalse
%<*samplepart3>
%\fi
% Some text for part 3:
%    \begin{macrocode}
some text in part three
%    \end{macrocode}

%\iffalse
%</samplepart3>
%\fi
% Some text for part 4:
%\iffalse
%<*samplepart4>
%\fi
%    \begin{macrocode}
more text in part four
%    \end{macrocode}

%\iffalse
%</samplepart4>
%\fi
%
% %%%%%%%%%%%%%%%%%%%%%%%%%%%%%%%%%%%%%%
% \paragraph{Forwarding for a Complete Draft.}
%
% The following forwarding file |cdocsdrf.tex|
% compiles the main document in draft mode:
%\iffalse
%<*sampledraft>
%\fi
%    \begin{macrocode}
\def\version{draft}
\input{childdoc.def}
\childdocforward{cdocsamp}
%    \end{macrocode}

%\iffalse
%</sampledraft>
%\fi
%
% %%%%%%%%%%%%%%%%%%%%%%%%%%%%%%%%%%%%%%
% \paragraph{Forwarding for Final Version of the Chapters.}
%
% The following forwarding files |cdocsfn1.tex| and |cdocsfn2.tex|
% (with identical content)
% compile the final versions of the child documents
% |cdocsch1.tex| and |cdocsch2.tex|, respectively:
%\iffalse
%<*samplefinal>
%\fi
%    \begin{macrocode}
\def\version{final}
\input{childdoc.def}
\childdocforwardprefix[cdocsamp]{cdocsfn}{cdocsch}
%    \end{macrocode}

%\iffalse
%</samplefinal>
%\fi
%
% %%%%%%%%%%%%%%%%%%%%%%%%%%%%%%%%%%%%%%
% \paragraph{Command Line Processing.}
%
% The following three command lines generate the output files
% |cdocscld|, |cdocscl1| and |cdocscl2|
% which should be identical to
% |cdocsdrf|, |cdocsch1| and |cdocsfn2|, respectively:
% \begin{center}
% \begin{tabular}{l}
% |latex -jobname cdocscld \|\\
% |  "\def\version{draft}\input{childdoc.def}\childdocforward{cdocsamp}"|\\
% |latex -jobname cdocscl1 \|\\
% |  "\input{childdoc.def}\childdocforward[cdocsamp]{cdocsch1}"|\\
% |latex -jobname cdocscl2 \|\\
% |  "\def\version{final}\input{childdoc.def}\childdocforward{cdocsch2}"|
% \end{tabular}
% \end{center}
% Note that the trailing backslash on each first line
% merely continues the input to the second line
% (for convenient cut ant paste).
% Furthermore, the command |latex| can be replaced by any
% of its alternative versions such as |pdflatex|.
%
% %%%%%%%%%%%%%%%%%%%%%%%%%%%%%%%%%%%%%%%%%%%%%%%%%%%%%%%%%%%%%%%%%%%%%%%%%%%%%%
% %%%%%%%%%%%%%%%%%%%%%%%%%%%%%%%%%%%%%%%%%%%%%%%%%%%%%%%%%%%%%%%%%%%%%%%%%%%%%%
% \section{Implementation}
%\iffalse
%<*package>
%\fi
%
% This section describes the definitions file |childdoc.def|.

% The definitions cannot be loaded using |\usepackage| or |\RequirePackage|
% which has a mechanism to prevent loading a style file more than once.
% When loading the definitions by means of |\input|
% multiple instances have to be prevented manually:
%\iffalse
%This code needs to be before the `\ProvidesFile' directive
%which is defined at the beginning of this file.
%Therefore it is also placed there and commented out here.
%</package>
%<*discard>
%\fi
%    \begin{macrocode}
\ifdefined\childdocmain\endinput\fi
%    \end{macrocode}
%\iffalse
%</discard>
%<*package>
%\fi
%
% \macro{\ifchilddoc}
% \macro{\ifchilddocmanual}
% The conditional |\ifchilddoc| tells whether a
% child (true) or main (false) document is being compiled.
% The conditional |\ifchilddocmanual| tells whether
% the |\includeonly| mechanism is used (false) or
% the selection of child files must be performed manually (true).
% The definitions initialise to false:
%    \begin{macrocode}
\newif\ifchilddoc
\newif\ifchilddocmanual
%    \end{macrocode}

% \macro{\childdocname}
% \macro{\childdocjob}
% The macro |\childdocname| stores the name of the main document
% to be compiled. The macro |\childdocjob| stores the name of
% the document on which the \LaTeX{} compiler was originally invoked.
% The content of |\jobname| cannot be compared
% to filenames specified in the source due to different catcodes.
% The following code rescans |\jobname|, stores the result
% in |\childdocname| and saves a copy in |\childdocjob|:
%    \begin{macrocode}
\edef\childdocname{\scantokens\expandafter{\jobname\noexpand}}
\let\childdocjob\childdocname
%    \end{macrocode}

% \macro{\childdocdisable}
% The macro |\childdocdisable| prevents the main file
% from being processed more than once.
% At this stage, the main document command |\childdocmain|
% is assumed to be called once again where it should do nothing.
% Any subsequent call to it should prevent
% a secondary processing of the main document
% It overwrites the forwarding commands
% |\childdocof| and |\childdocforward|
% with empty macros to prevent further inclusions of the main document:
%    \begin{macrocode}
\newcommand{\childdocdisable}
{
  \renewcommand{\childdocmain}[1]{\renewcommand{\childdocmain}[1]{\endinput}}
  \renewcommand{\childdocof}[1]{}
  \renewcommand{\childdocby}[2][]{}
  \renewcommand{\childdocforward}[2][]{}
  \renewcommand{\childdocdisable}{}
}
%    \end{macrocode}

% \macro{\childdocmain}
% The macro |\childdocmain| is to be called at the top of the main file
% with nothing or the main filename (without extension) as argument.
% First, it breaks loops.
% If the argument is not empty and does not match |\childdocname|
% (which is set by the first inclusion of |childdoc.def|),
% |\ifchilddoc| is set to true, |\includeonly| is applied to the child file
% and |\jobname| is set to the main file
% (for proper handling of |.aux| files):
%    \begin{macrocode}
\newcommand{\childdocmain}[1]
{
  \childdocdisable\childdocmain{}
  \if?#1?\else
    \begingroup
      \def\childdoctmp{#1}
      \ifx\childdoctmp\childdocname
        \def\childdoctmp{}
      \else
        \def\childdoctmp
        {
          \childdoctrue
          \includeonly{\childdocname}
          \def\childdocjob{#1}
          \def\jobname{#1}
        }
      \fi
      \expandafter
    \endgroup
    \childdoctmp
  \fi
}
%    \end{macrocode}

% \macro{\childdocof}
% The command |\childdocof| redirects
% compilation to the main file |#1|.
%    \begin{macrocode}
\newcommand{\childdocof}[1]
{
  \childdocdisable
  \childdoctrue
  \includeonly{\childdocname}
  \def\jobname{#1}
  \def\childdocjob{#1}
  \input{#1}
}
%    \end{macrocode}

% \macro{\childdocby}
% The command |\childdocby| ....
%    \begin{macrocode}
\newcommand{\childdocby}[2][]
{
  \childdocdisable
  \childdoctrue
  \childdocmanualtrue
  \if?#1?\else
    \def\jobname{#2}
  \fi
  \def\childdocjob{#2}
  \input{#2}
  \endinput
}
%    \end{macrocode}

% \macro{\childdocforward}
% The command |\childdocforward| redirects
% compilation to the main file or
% (if the optional argument is given) a child file.
% Parameters are set as if the main file
% or a child file starting with |\childdocof| was compiled.
% Then compilation is handed over to the main file:
%    \begin{macrocode}
\newcommand{\childdocforward}[2][]
{
  \begingroup
    \if?#1?
      \def\childdoctmp
      {
        \def\childdocname{#2}
        \def\childdocjob{#2}
        \def\jobname{#2}
        \input{#2}
        \endinput
      }
    \else
      \def\childdoctmp
      {
        \childdocdisable
        \def\childdocname{#2}
        \childdoctrue
        \includeonly{#2}
        \def\childdocjob{#1}
        \def\jobname{#1}
        \input{#1}
        \endinput
      }
    \fi
    \expandafter
  \endgroup
  \childdoctmp
}
%    \end{macrocode}

% \macro{\childdocforwardprefix}
% The command |\childdocforwardprefix| redirects
% compilation to the main or a child file by means of a pattern.
% The prefix |#1| in the current filename is replaced by |#2|
% and the suffix of the current filename is kept
% (it is assumed that the filename does not contain the substring `|~~~|'
% which is used as a delimiter).
% Compilation is handed over to the new file by |\childdocforward|:
%    \begin{macrocode}
\newcommand{\childdocforwardprefix}[3][]
{
  \begingroup
    \def\childdocextract #2##1~~~{\def\childdoctmp{\childdocforward[#1]{#3##1}}}
    \expandafter\childdocextract\childdocname~~~
    \expandafter
  \endgroup
  \childdoctmp
}
%    \end{macrocode}

% \macro{\childdoc}
% The deprecated macro |\childdoc| is a legacy version of |\childdocmain|:
%    \begin{macrocode}
\newcommand{\childdoc}{\childdocmain}
%    \end{macrocode}

% \macro{\childdocredirect}
% The deprecated macro |\childdocredirect| is a legacy version
% of |\childdocforward| and |\childdocforwardprefix|:
%    \begin{macrocode}
\newcommand{\childdocredirect}[2][]
{
  \begingroup
    \if?#1?
      \def\childdoctmp{\childdocforward{#2}}
    \else
      \def\childdoctmp{\childdocforwardprefix{#1}{#2}}
    \fi
    \expandafter
  \endgroup
  \childdoctmp
}
%    \end{macrocode}

%\iffalse
%</package>
%\fi
%
\endinput
\childdocforward[|\textit{main}|]{|\textit{dest}|}"|
\end{center}
%
Here \textit{target} is the name of the output file,
\textit{main} is the name of the main file
and \textit{dest} is the name of the main or child file to be processed
(all filenames without extensions).
The optional argument \textit{main} can be omitted
if \textit{main} matches \textit{dest}.
Optionally, compilation \textit{flags} can be defined via |\def| commands.
This command line makes the \TeX{} engine believe
it is compiling the file \textit{target}
whose content is specified as the latter parameter.
The provided code then forwards the processing to
\textit{main} or \textit{dest} as described in \secref{sec:forward}.

%%%%%%%%%%%%%%%%%%%%%%%%%%%%%%%%%%%%%%%%%%%%%%%%%%%%%%%%%%%%%%%%%%%%%%%%%%%%%%%%
\subsection{Include by Input}
\label{sec:input}

Including child documents by |\include| has some restrictions by design.
Most notably, the content of a child document always occupies
its own set of pages; pages cannot be shared between child documents.
Usually, this behaviour makes perfect sense
because each child document contain an essential part of the document.
However, in some situations it may be desirable to compose
a document from a collection of parts
without having mandatory page breaks between then.
For this case, the package
provides a mechanism to include parts
by |\input| which can also be processed individually.
However, by construction this mechanism
requires manual handling of the content to be output.

%%%%%%%%%%%%%%%%%%%%%%%%%%%%%%%%%%%%%%%%
\DescribeMacro{\ifchilddocmanual}
The main file should be prepared as usual, see \secref{sec:include}.
However, the document body must make a distinction
between processing of an individual part and of the main document, e.g.:
%
\begin{center}
\begin{tabular}{l}
|\ifchilddocmanual|\\
|\input{\childdocname}|\\
|\||else|\\
\textit{document body with }|\input{|\textit{part}|}|\\
|\||fi|
\end{tabular}
\end{center}
%
The conditional |\ifchilddocmanual| is true whenever
a part to be included by |\input| is being compiled,
and the name of the part is stored in |\childdocname|.

%%%%%%%%%%%%%%%%%%%%%%%%%%%%%%%%%%%%%%%%
\DescribeMacro{\childdocby}
Each part to be included by |\input| should start with:
%
\begin{center}
\begin{tabular}{l}
|% \iffalse
%
% childdoc.dtx Copyright (C) 2017-2018 Niklas Beisert
%
% This work may be distributed and/or modified under the
% conditions of the LaTeX Project Public License, either version 1.3
% of this license or (at your option) any later version.
% The latest version of this license is in
%   http://www.latex-project.org/lppl.txt
% and version 1.3 or later is part of all distributions of LaTeX
% version 2005/12/01 or later.
%
% This work has the LPPL maintenance status `maintained'.
%
% The Current Maintainer of this work is Niklas Beisert.
%
% This work consists of the files childdoc.dtx and childdoc.ins
% and the derived files childdoc.def and cdocsamp.tex with
% cdocsch1.tex, cdocsch2.tex, cdocsdrf.tex, cdocsfn1.tex, cdocsfn2.tex.
%
%<package>\ifdefined\childdocmain\endinput\fi
%<package>\ProvidesFile{childdoc.def}[2018/12/30 v2.0 child document driver]
%<samplemain>\ProvidesFile{cdocsamp.tex}[2018/12/30 v2.0 sample for childdoc]
%<*driver>
%\ProvidesFile{childdoc.drv}[2018/12/30 v2.0 childdoc reference manual file]
\PassOptionsToClass{10pt,a4paper}{article}
\documentclass{ltxdoc}

\usepackage[margin=35mm]{geometry}
\usepackage{hyperref}
\usepackage{hyperxmp}
\usepackage[usenames]{color}

\hypersetup{colorlinks=true}
\hypersetup{pdfstartview=FitH}
\hypersetup{pdfpagemode=UseNone}
\hypersetup{pdfsource={}}
\hypersetup{pdflang={en-UK}}
\hypersetup{pdfcopyright={Copyright 2017-2018 Niklas Beisert.
  This work may be distributed and/or modified under the
  conditions of the LaTeX Project Public License, either version 1.3
  of this license or (at your option) any later version.}}
\hypersetup{pdflicenseurl={http://www.latex-project.org/lppl.txt}}
\hypersetup{pdfcontactaddress={ETH Zurich, ITP, HIT K,
  Wolfgang-Pauli-Strasse 27}}
\hypersetup{pdfcontactpostcode={8093}}
\hypersetup{pdfcontactcity={Zurich}}
\hypersetup{pdfcontactcountry={Switzerland}}
\hypersetup{pdfcontactemail={nbeisert@itp.phys.ethz.ch}}
\hypersetup{pdfcontacturl={http://people.phys.ethz.ch/\xmptilde nbeisert/}}

\newcommand{\secref}[1]{\hyperref[#1]{section \ref*{#1}}}

\parskip1ex
\parindent0pt
\let\olditemize\itemize
\def\itemize{\olditemize\parskip0pt}

\begin{document}

\title{The \textsf{childdoc} Package}
\hypersetup{pdftitle={The childdoc Package}}
\author{Niklas Beisert\\[2ex]
  Institut f\"ur Theoretische Physik\\
  Eidgen\"ossische Technische Hochschule Z\"urich\\
  Wolfgang-Pauli-Strasse 27, 8093 Z\"urich, Switzerland\\[1ex]
  \href{mailto:nbeisert@itp.phys.ethz.ch}
  {\texttt{nbeisert@itp.phys.ethz.ch}}}
\hypersetup{pdfauthor={Niklas Beisert}}
\hypersetup{pdfsubject={Manual for the LaTeX2e Package childdoc}}
\date{30 December 2018, \textsf{v2.0}}
\maketitle

\begin{abstract}\noindent
\textsf{childdoc} is a \LaTeXe{} package
that enables the direct compilation
of document sections included by |\include|
to individual files.
\end{abstract}

\begingroup
\parskip0ex
\tableofcontents
\endgroup

%%%%%%%%%%%%%%%%%%%%%%%%%%%%%%%%%%%%%%%%%%%%%%%%%%%%%%%%%%%%%%%%%%%%%%%%%%%%%%%%
%%%%%%%%%%%%%%%%%%%%%%%%%%%%%%%%%%%%%%%%%%%%%%%%%%%%%%%%%%%%%%%%%%%%%%%%%%%%%%%%
\section{Introduction}

\LaTeX{} provides a mechanism to structure a large document (such as a book)
into a main file and several child files (containing the chapters)
using the |\include| command.
This mechanism is beneficial for documents
which span hundreds of pages in order to
make the source file(s) more manageable.
Moreover, compilation can be restricted to
selected child files by means of the |\includeonly| command.
The latter feature can be used to reduce the compilation time while editing
(this was significantly more useful in the earlier days of \LaTeX{})
or to generate a smaller document which is easier to navigate.
Another application of |\includeonly| is to generate
documents consisting of selected parts of the complete document.

However, there are a few drawbacks of the plain |\include| mechanism:
\begin{itemize}
\item
The child files cannot be compiled on their own,
they can only be compiled via the main file.
A naive editing environment
(such as a text editor with an option
to have the current file processed by \LaTeX)
may require one to switch to the main file before compiling;
attempting to compile the child file produces errors.
\item
The main file must be modified (each time)
to adjust the |\includeonly| command
to the present needs. This easily leaves the main file in a messy state.
\item
The generated document will always carry the filename
of the main document. This is inconvenient if
several child files are to be compiled and
to be kept for distribution.
\end{itemize}

The present package provides a simple interface
to make child files individually compilable by \LaTeX{}.
Compiling a child file then has the same effect as compiling
the main file with an |\includeonly| command
to select the appropriate child.
Moreover the generated document will carry the name of the child
rather than the main file.
This resolves all three above issues.

This feature is meant to make the editing of books,
thesis documents and lecture notes somewhat more convenient.
However, the package can also be used efficiently for
composing a series of documents (such as exercise sheets)
which are typically distributed individually.
It then assists the author in generating the individual documents
(potentially in different versions)
as well as a document containing the collected series.
Another application is in developing style files
or other kinds of included material
where compilation of the style file could redirect
to a sample or test file.

%%%%%%%%%%%%%%%%%%%%%%%%%%%%%%%%%%%%%%%%%%%%%%%%%%%%%%%%%%%%%%%%%%%%%%%%%%%%%%%%
%%%%%%%%%%%%%%%%%%%%%%%%%%%%%%%%%%%%%%%%%%%%%%%%%%%%%%%%%%%%%%%%%%%%%%%%%%%%%%%%
\section{Usage}

First of all, the package \textsf{childdoc} is \emph{not} a standard
\LaTeXe{} |.sty| style file! Therefore it needs to be invoked in
a non-standard way.

%%%%%%%%%%%%%%%%%%%%%%%%%%%%%%%%%%%%%%%%%%%%%%%%%%%%%%%%%%%%%%%%%%%%%%%%%%%%%%%%
\subsection{Included Files}
\label{sec:include}

%%%%%%%%%%%%%%%%%%%%%%%%%%%%%%%%%%%%%%%%
\DescribeMacro{\childdocmain}
To use the package, add the commands
\begin{center}
\begin{tabular}{l}
|\input{childdoc.def}|\\
|\childdocmain{}|\\
\end{tabular}
\end{center}
at the very top of the main \LaTeX{} file,
in particular \emph{before} the |\documentclass| statement!
The argument of |\childdocmain| should be left empty
(but it must be present).

%%%%%%%%%%%%%%%%%%%%%%%%%%%%%%%%%%%%%%%%
\DescribeMacro{\childdocof}
Furthermore, add the commands
\begin{center}
\begin{tabular}{l}
|\input{childdoc.def}|\\
|\childdocof{|\textit{main}|}|\\
\end{tabular}
\end{center}
at the top of every child file \textit{child}
which is included by |\include{|\textit{child}|}|
from within the main file
(or at least for those files to be compiled individually).
The argument \textit{main} must be the filename of the main file.

There are a couple of
considerations in setting up the main and child documents:

%%%%%%%%%%%%%%%%%%%%%%%%%%%%%%%%%%%%%%%%
\paragraph{Restrictions.}

Please note the following restrictions:
\begin{itemize}
\item
|\childdocmain| must be called with one argument \textit{main}
to ensure compatibility with earlier version of the package.
It must either be empty (|\childdocmain{}|)
or precisely match the filename of the main file in which it is specified.
See \secref{sec:detection} for further information.
\item
The filename \textit{main} must be specified without the |.tex| extension.
\item
The filename \textit{main} is case sensitive
(even in case-insensitive file systems)
due to internal string comparison.
\item
The argument \textit{main} should be fully expanded, it cannot be a macro.
\item
Subdirectories and special characters should be avoided in filenames.
\item
The command |\childdocmain{|\textit{main}|}| must be followed by a whitespace.
It should not be followed immediately by another command
or by a comment mark `|%|'.
This is because the \TeX{} parser reads the token immediately following
the argument of |\childdocmain| and puts it
at the beginning of every child section;
however, a white\-space is ignored.
\end{itemize}

%%%%%%%%%%%%%%%%%%%%%%%%%%%%%%%%%%%%%%%%
\paragraph{Content of Main File.}

It is advisable to place all content in the child files included by |\include|.
Any output contained in the main file will appear in all child documents
unless suppressed manually;
it cannot be suppressed automatically by the |\includeonly| directive
and thus should normally be avoided.
A method to include some content in the main file
by means of conditional processing is described in \secref{sec:conditional}.

%%%%%%%%%%%%%%%%%%%%%%%%%%%%%%%%%%%%%%%%
\paragraph{Page Numbering.}

When only a part of the document is compiled,
the appropriate numbering of pages
(as well as other status parameters)
is determined from the |.aux| files.
The latter contain information from previous passes.
However this information needs to propagate through
all intermediate child documents.
Therefore the page numbering in child documents may well
be inconsistent until the complete document is compiled at least once.

A useful (if unconventional) way to always ensure a consistent
page numbering is to restart the numbering in each child document
and denote the pages by `\textit{child}|.|\textit{page}'
where \textit{child} represents the chapter/section number of the child file.
This can be achieved by the command
|\numberwithin{page}{|\textit{child}|}|
of the \textsf{amsmath} package
where \textit{child} can be |chapter| or |section|
depending on the chosen structuring.
Alternatively, one can modify the macro |\thepage| appropriately
and reset the counter |page| at the start of each child file.

%%%%%%%%%%%%%%%%%%%%%%%%%%%%%%%%%%%%%%%%%%%%%%%%%%%%%%%%%%%%%%%%%%%%%%%%%%%%%%%%
\subsection{Conditional Processing}
\label{sec:conditional}

The package provides a mechanism to compile different versions
of a document. To customise the versions further some conditional processing
can come in handy to distinguish which version is being compiled.
The package provides two macros to describe the compilation context:

%%%%%%%%%%%%%%%%%%%%%%%%%%%%%%%%%%%%%%%%
\DescribeMacro{\ifchilddoc}
The conditional |\ifchilddoc| distinguishes between the compilation of
child documents and the main document:
%
\begin{center}
|\ifchilddoc |\textit{child-code}| |[|\||else |\textit{main-code}]| \||fi|
\end{center}

%%%%%%%%%%%%%%%%%%%%%%%%%%%%%%%%%%%%%%%%
\DescribeMacro{\childdocname}
\DescribeMacro{\childdocjob}
The macro |\childdocname| contains the filename (without extension)
of the main or child file being processed.
Note that |\childdocjob| will always contain the name of the main file.

%%%%%%%%%%%%%%%%%%%%%%%%%%%%%%%%%%%%%%%%
\paragraph{Title Page.}

Conditional processing can be used to include a title or banner page
in the main document when proper precautions are taken.
Importantly, the code in the main file should ensure that the page counter
(as well as other status parameters which are stored in the |.aux| files)
takes the same value after the conditional processing.
Otherwise the page numbers may take divergent values
depending on which part is compiled.

For example, a title page could be declared by:
%
\begin{center}
\begin{tabular}{l}
|\ifchilddoc\||else|\\
|\addtocounter{page}{-1}|\\
\textit{code for title page}\\
|\newpage|\\
|\||fi|
\end{tabular}
\end{center}
%
A banner page for the child documents can be generated by:
%
\begin{center}
\begin{tabular}{l}
|\ifchilddoc|\\
|\addtocounter{page}{-1}|\\
\textit{code for banner page}\\
|\newpage|\\
|\||fi|
\end{tabular}
\end{center}
%
Here one could write a message such as:
\begin{center}
|This is the part \childdocname{} of \childdocjob{}.|
\end{center}

%%%%%%%%%%%%%%%%%%%%%%%%%%%%%%%%%%%%%%%%%%%%%%%%%%%%%%%%%%%%%%%%%%%%%%%%%%%%%%%%
\subsection{Flags}
\label{sec:flags}

The package makes it easy to generate different versions
of the main or child documents.
To this end compilation flags can be defined
and assigned different default values.
They will be particularly useful in conjunction
with the forwarding mechanism described in \secref{sec:forward}.

For example, it may be useful to have a flag |\version|
which can be set to |draft| or |final|.
The document source will contain some conditional code
depending on the value of |\version|.
Suppose further, the flag should default to |final| for the main file
and to |draft| for child files
which is a natural assignment for editing the document.
This is achieved by placing the following code
in the preamble of the main document
(below the |\childdocmain| directive):
%
\begin{center}
\begin{tabular}{l}
|\ifchilddoc|\\
|\providecommand{\version}{draft}|\\
|\||else|\\
|\providecommand{\version}{final}|\\
|\||fi|
\end{tabular}
\end{center}
%
The definition by |\providecommand| makes sure
that previous definitions are not overwritten.
Further statements |\providecommand{\version}{...}|
can thus be added before the above code to override it.

For the main file, one might add a line
(between |\childdocmain| and the above block)
%
\begin{center}
|%\ifchilddoc\||else\providecommand{\version}{draft}\||fi|
\end{center}
%
which can be uncommented to produce a draft version.
Likewise one can add a line to the very top of a child file
(above the |\childdocof{|\textit{main}|}| directive)
%
\begin{center}
|%\providecommand{\version}{final}|
\end{center}
%
which can be uncommented to produce the final version of this child document.

%%%%%%%%%%%%%%%%%%%%%%%%%%%%%%%%%%%%%%%%%%%%%%%%%%%%%%%%%%%%%%%%%%%%%%%%%%%%%%%%
\subsection{Forwarding}
\label{sec:forward}

Different versions of the main or child documents
using compilation flags as described in \secref{sec:flags}
can be (permanently) stored in different files
for convenient compilation, viewing and distribution.
To this end, the package defines a command
to pass on compilation to a different file:

%%%%%%%%%%%%%%%%%%%%%%%%%%%%%%%%%%%%%%%%
\DescribeMacro{\childdocforward}
The command |\childdocforward| redirects processing to
another source file:
%
\begin{center}
\begin{tabular}{l}
|\input{childdoc.def}|\\
|\childdocforward[|\textit{main}|]{|\textit{dest}|}|\\
\end{tabular}
\end{center}
%
The argument \textit{dest} is the destination file
(without extension).
It should be the main file or one of the child files.
Note that further \textsf{childdoc} directives
such as |\childdocof| and |\childdocforward|
in the indicated file will be processed in this form.
The optional argument \textit{main}
passes on directly to the main file \textit{main}
while pretending to compile the child \textit{dest}.
This form behaves as if \textit{dest}
issues |\childdocof{|\textit{main}|}| right away,
and no further \textsf{childdoc} directives will be processed.

%%%%%%%%%%%%%%%%%%%%%%%%%%%%%%%%%%%%%%%%
\DescribeMacro{\...prefix}
In the alternative form |\childdocforwardprefix|,
%
\begin{center}
\begin{tabular}{l}
|\input{childdoc.def}|\\
|\childdocforwardprefix[|\textit{main}|]{|\textit{prefix}|}{|\textit{dest}|}|
\end{tabular}
\end{center}
%
the destination file is determined by a pattern
depending on the current file:
To make this work, the current file must be called
`{\textit{prefix}\hspace{0.2em}\textit{suffix}}'
with \textit{prefix} matching precisely the argument.
Processing is then passed on to the file
`{\textit{dest}\hspace{0.2em}\textit{suffix}}'.
Surely, the same effect is achieved by
directly specifying the
argument `{\textit{dest}\hspace{0.2em}\textit{suffix}}'
in the first form.
However, that requires to set up a different file
for each child. With the alternative form of the command
all these files can have exactly the same content
which simplifies setting them up and maintaining them.

For example, the following file |draft.tex|
with a compilation flag |\version| as described in \secref{sec:flags}
compiles the main document as a draft:
%
\begin{center}
\begin{tabular}{l}
|\def\version{draft}|\\
|\input{childdoc.def}|\\
|\childdocforward{|\textit{main}|}|
\end{tabular}
\end{center}
%
Likewise, the following files |final|\textit{nn}|.tex|
compile the final version of the child document
|child|\textit{nn}|.tex|:
%
\begin{center}
\begin{tabular}{l}
|\def\version{final}|\\
|\input{childdoc.def}|\\
|\childdocforwardprefix{final}{child}|
\end{tabular}
\end{center}
%

Note that when several versions of a main file and/or of each child file
are to be generated, it may be convenient to set up a |Makefile| or
shell script to automatise the process.

%%%%%%%%%%%%%%%%%%%%%%%%%%%%%%%%%%%%%%%%%%%%%%%%%%%%%%%%%%%%%%%%%%%%%%%%%%%%%%%%
\subsection{Command Line Processing}
\label{sec:commandline}

The effect of redirection files can also be achieved by invoking
the \LaTeX{} compiler with a more elaborate command line.
Most conveniently this should be done as part
of a shell script or a |Makefile|.

When using \textsf{childdoc} in the main file, the following
command lines effectively perform a redirection
(note that depending on the shell being used,
backslashes may have to be doubled: `|\|' $\to$ `|\\|'):
%
\begin{center}
|... -jobname "|\textit{target}|" |\\|"|[\textit{flags}]%
|\input{childdoc.def}\childdocforward[|\textit{main}|]{|\textit{dest}|}"|
\end{center}
%
Here \textit{target} is the name of the output file,
\textit{main} is the name of the main file
and \textit{dest} is the name of the main or child file to be processed
(all filenames without extensions).
The optional argument \textit{main} can be omitted
if \textit{main} matches \textit{dest}.
Optionally, compilation \textit{flags} can be defined via |\def| commands.
This command line makes the \TeX{} engine believe
it is compiling the file \textit{target}
whose content is specified as the latter parameter.
The provided code then forwards the processing to
\textit{main} or \textit{dest} as described in \secref{sec:forward}.

%%%%%%%%%%%%%%%%%%%%%%%%%%%%%%%%%%%%%%%%%%%%%%%%%%%%%%%%%%%%%%%%%%%%%%%%%%%%%%%%
\subsection{Include by Input}
\label{sec:input}

Including child documents by |\include| has some restrictions by design.
Most notably, the content of a child document always occupies
its own set of pages; pages cannot be shared between child documents.
Usually, this behaviour makes perfect sense
because each child document contain an essential part of the document.
However, in some situations it may be desirable to compose
a document from a collection of parts
without having mandatory page breaks between then.
For this case, the package
provides a mechanism to include parts
by |\input| which can also be processed individually.
However, by construction this mechanism
requires manual handling of the content to be output.

%%%%%%%%%%%%%%%%%%%%%%%%%%%%%%%%%%%%%%%%
\DescribeMacro{\ifchilddocmanual}
The main file should be prepared as usual, see \secref{sec:include}.
However, the document body must make a distinction
between processing of an individual part and of the main document, e.g.:
%
\begin{center}
\begin{tabular}{l}
|\ifchilddocmanual|\\
|\input{\childdocname}|\\
|\||else|\\
\textit{document body with }|\input{|\textit{part}|}|\\
|\||fi|
\end{tabular}
\end{center}
%
The conditional |\ifchilddocmanual| is true whenever
a part to be included by |\input| is being compiled,
and the name of the part is stored in |\childdocname|.

%%%%%%%%%%%%%%%%%%%%%%%%%%%%%%%%%%%%%%%%
\DescribeMacro{\childdocby}
Each part to be included by |\input| should start with:
%
\begin{center}
\begin{tabular}{l}
|\input{childdoc.def}|\\
|\childdocby{|\textit{main}|}|\\
\end{tabular}
\end{center}
%
The directive |\childdocby| is similar to |\childdocof|
described in \secref{sec:include},
but the subsequent selection of content must be done manually.
To that end, both |\ifchilddoc| and |\ifchilddocmanual|
will be true upon processing of a part,
and the name of the part is stored in |\childdocname|.
Note that |\jobname| will be set to the filename of the current part
so that each part receives an individual |.aux| file
that does not interfere with the |.aux| file(s) of the main document.
This behaviour can be altered by the alternative form
|\childdocby[*]{|\textit{main}|}| (with a non-empty optional argument)
which uses the |.aux| file of the main document
by setting |\jobname| to \textit{main}.

%%%%%%%%%%%%%%%%%%%%%%%%%%%%%%%%%%%%%%%%%%%%%%%%%%%%%%%%%%%%%%%%%%%%%%%%%%%%%%%%
\subsection{Driver Development}
\label{sec:driver}

The \textsf{childdoc} mechanism can also be use for the development
of definition files such as \LaTeX{} styles or classes.
This case differs from the above setup with multiple parts
included by |\include| in that no |\includeonly| should be invoked.
This can be achieved by starting the include file
(before |\ProvidesPackage|) with:
%
\begin{center}
\begin{tabular}{l}
|\input{childdoc.def}|\\
|\childdocforward{|\textit{main}|}|\\
\end{tabular}
\end{center}
%
or alternatively with:
%
\begin{center}
\begin{tabular}{l}
|\input{childdoc.def}|\\
|\childdocby{|\textit{main}|}|\\
\end{tabular}
\end{center}
%
Both forms have slightly different effects as described above.
The main file is prepared as usual, see \secref{sec:include}.

%%%%%%%%%%%%%%%%%%%%%%%%%%%%%%%%%%%%%%%%%%%%%%%%%%%%%%%%%%%%%%%%%%%%%%%%%%%%%%%%
\subsection{Legacy Detection}
\label{sec:detection}

The directive |\childdocmain| in the main file can detect
whether the complete document or merely a child is to be compiled
even without using the directive |\childdocof|.
This method is deprecated because it is less robust
and there is no compelling reason to use it;
it is merely provided for backward compatibility
and it may be removed in future versions.

If the detection mechanism is to be used,
it is mandatory to correctly specify
the filename of the main file as the argument of |\childdocmain|:
%
\begin{center}
\begin{tabular}{l}
|\input{childdoc.def}|\\
|\childdocmain{|\textit{main}|}|\\
\end{tabular}
\end{center}
%
If |\jobname| does not match the argument \textit{main} of |\childdocmain|,
it is assumed that |\jobname| points to the child file to be compiled.
When using |\childdocmain| with the main file specified as argument,
it suffices to start a child file
with just |\input{|\textit{main}|}|
without loading of the package and using |\childdocof|.
If instead all processing is done
with the appropriate \textsf{childdoc} directives,
the argument of \textit{main} of |\childdocmain| can be empty.

An alternative version of the command line processing described
in \secref{sec:commandline} using the detection mechanism reads:
%
\begin{center}
|... -jobname "|\textit{target}|" "|[\textit{flags}]%
[|\def\jobname{|\textit{dest}|}|]|\input{|\textit{main}|}"|
\end{center}

%%%%%%%%%%%%%%%%%%%%%%%%%%%%%%%%%%%%%%%%%%%%%%%%%%%%%%%%%%%%%%%%%%%%%%%%%%%%%%%%
\subsection{Manual Code}
\label{sec:manual}

In case one cannot be certain whether the definitions file |childdoc.def|
is installed on the target \TeX{} distribution
and one prefers not to ship it,
it is conceivable to paste a few relevant commands into the sources.

To that end, drop all statements |\input{childdoc.def}|
and perform the replacements as outlined below.
Instead of |\childdocmain{|\textit{main}|}| add the following code
to the top of the main file:
%
\begin{center}
\begin{tabular}{l}
|\||ifdefined\childdocname\endinput\||fi\newif\ifchilddoc|\\
|\edef\childdocname{\scantokens\expandafter{\jobname\noexpand}}|\\
|\def\childdocmain{|\textit{main}|}\||ifx\childdocmain\childdocname\||else|\\
|\childdoctrue\includeonly{\childdocname}\let\jobname\childdocmain\||fi|\\
\end{tabular}
\end{center}
%
Instead of |\childdocof{|\textit{main}|}| just include the main file
at the top of each child file:
%
\begin{center}
|\input{|\textit{main}|}|
\end{center}
%
A simple redirection |\childdocforward{|\textit{dest}|}| is achieved by:
%
\begin{center}
|\def\jobname{|\textit{dest}|}\input{\jobname}|
\end{center}
%
The redirection with prefix
|\childdocforwardprefix[|\textit{prefix}|]{|\textit{dest}|}|
is accomplished by:
%
\begin{center}
\begin{tabular}{l}
|{\edef\jobname{\scantokens\expandafter{\jobname\noexpand}}|\\
|\def\redirectjob |\textit{prefix}|#1~~~{\gdef\jobname{|\textit{dest}|#1}}|\\
|\expandafter\redirectjob\jobname~~~}\input{\jobname}|
\end{tabular}
\end{center}

In an alternative approach,
child documents can be compiled by a specific command line
without additional code or specific definitions:
%
\begin{center}
|... -jobname "|\textit{target}|" "|[\textit{flags}]%
|\includeonly{|\textit{dest}|}\input{|\textit{main}|}"|
\end{center}
%

%%%%%%%%%%%%%%%%%%%%%%%%%%%%%%%%%%%%%%%%%%%%%%%%%%%%%%%%%%%%%%%%%%%%%%%%%%%%%%%%
%%%%%%%%%%%%%%%%%%%%%%%%%%%%%%%%%%%%%%%%%%%%%%%%%%%%%%%%%%%%%%%%%%%%%%%%%%%%%%%%
\section{Information}

%%%%%%%%%%%%%%%%%%%%%%%%%%%%%%%%%%%%%%%%%%%%%%%%%%%%%%%%%%%%%%%%%%%%%%%%%%%%%%%%
\subsection{Copyright}

Copyright \copyright{} 2017--2018 Niklas Beisert

This work may be distributed and/or modified under the
conditions of the \LaTeX{} Project Public License, either version 1.3
of this license or (at your option) any later version.
The latest version of this license is in
  \url{http://www.latex-project.org/lppl.txt}
and version 1.3 or later is part of all distributions of \LaTeX{}
version 2005/12/01 or later.

This work has the LPPL maintenance status `maintained'.

The Current Maintainer of this work is Niklas Beisert.

This work consists of the files |README.txt|, |childdoc.ins| and |childdoc.dtx|
as well as the derived files |childdoc.def|, |cdocsamp.tex|
with |cdocsch1.tex|, |cdocsch2.tex|, |cdocspt3.tex|, |cdocspt4.tex|,
|cdocsdrf.tex|, |cdocsfn1.tex|, |cdocsfn2.tex|
as well as |childdoc.pdf|.

%%%%%%%%%%%%%%%%%%%%%%%%%%%%%%%%%%%%%%%%%%%%%%%%%%%%%%%%%%%%%%%%%%%%%%%%%%%%%%%%
\subsection{Files and Installation}

The package consists of the files:
%
\begin{center}
\begin{tabular}{ll}
    |README.txt|   & readme file \\
    |childdoc.ins| & installation file \\
    |childdoc.dtx| & source file \\
    |childdoc.def| & definition file \\
    |cdocsamp.tex| & sample main file \\
    |cdocsch1.tex| & sample include file \\
    |cdocsch2.tex| & sample include file \\
    |cdocspt3.tex| & sample part file \\
    |cdocspt4.tex| & sample part file \\
    |cdocsdrf.tex| & sample redirection file \\
    |cdocsfn1.tex| & sample redirection file \\
    |cdocsfn2.tex| & sample redirection file \\
    |childdoc.pdf| & manual
\end{tabular}
\end{center}
%
The distribution consists of the files
|README.txt|, |childdoc.ins| and |childdoc.dtx|.
%
\begin{itemize}
\item
Run (pdf)\LaTeX{} on |childdoc.dtx|
to compile the manual |childdoc.pdf| (this file).
\item
Run \LaTeX{} on |childdoc.ins| to create the definitions file |childdoc.def|
and the sample |cdocsamp.tex| with include files
|cdocsch1.tex|, |cdocsch2.tex|, |cdocspt3.tex|, |cdocspt4.tex|,
|cdocsdrf.tex|, |cdocsfn1.tex|, |cdocsfn2.tex|.
Then copy the file |childdoc.def| to an appropriate directory of your \LaTeX{}
distribution, e.g.\ \textit{texmf-root}|/tex/latex/childdoc|.
\end{itemize}

%%%%%%%%%%%%%%%%%%%%%%%%%%%%%%%%%%%%%%%%%%%%%%%%%%%%%%%%%%%%%%%%%%%%%%%%%%%%%%%%
\subsection{Related CTAN Packages}

There are several other packages which offer a similar functionality:
%
\begin{itemize}
\item
The packages
\href{http://ctan.org/pkg/docmute}{\textsf{docmute}},
\href{http://ctan.org/pkg/includex}{\textsf{includex}} and
\href{http://ctan.org/pkg/standalone}{\textsf{standalone}}
provide commands to include only the document body of
a child file thus allowing both files to be compiled individually.
\item
The packages \href{http://ctan.org/pkg/subdocs}{\textsf{subdocs}}
and \href{http://ctan.org/pkg/subfiles}{\textsf{subfiles}}
provide structures in which the main and child documents can be
encapsulated and allowing them to be compiled individually.
The inclusion mechanism is different from the conventional |\include|.
\item
The package \href{http://ctan.org/pkg/combine}{\textsf{combine}}
is an elaborate solution to combine several documents into one.
\end{itemize}
%
See also the CTAN topic \href{http://ctan.org/topic/subdocs}{\textsf{subdocs}}
for further related packages.
The present package differs from the above solutions in that
a document structure constructed with the conventional |\include| mechanism
just needs two extra commands at the top of every file
such that all constituent files can be compiled individually.

%%%%%%%%%%%%%%%%%%%%%%%%%%%%%%%%%%%%%%%%%%%%%%%%%%%%%%%%%%%%%%%%%%%%%%%%%%%%%%%%
%\subsection{Feature Suggestions}
%
%The following is a list of features which may be useful for future
%versions of this package:
%%
%\begin{itemize}
%\item
%\ldots
%\end{itemize}

%%%%%%%%%%%%%%%%%%%%%%%%%%%%%%%%%%%%%%%%%%%%%%%%%%%%%%%%%%%%%%%%%%%%%%%%%%%%%%%%
\subsection{Revision History}

%%%%%%%%%%%%%%%%%%%%%%%%%%%%%%%%%%%%%%%%
\paragraph{v2.0:} 2018/12/30

\begin{itemize}
\item
immediate forward processing
\item
added |\childdocby| mechanism
\item
manual restructured
\end{itemize}

%%%%%%%%%%%%%%%%%%%%%%%%%%%%%%%%%%%%%%%%
\paragraph{v1.6:} 2018/01/17

\begin{itemize}
\item
application for development of include files
\item
corrections to manual
\end{itemize}

%%%%%%%%%%%%%%%%%%%%%%%%%%%%%%%%%%%%%%%%
\paragraph{v1.5:} 2017/05/21

\begin{itemize}
\item
more complete structuring introduced
\item
|\childdocof| introduced
\item
|\childdoc| renamed to |\childdocmain|
\item
|\childredirect| renamed to |\childdocforward| and |\childdocforwardprefix|
and functionality expanded
\end{itemize}

%%%%%%%%%%%%%%%%%%%%%%%%%%%%%%%%%%%%%%%%
\paragraph{v1.0:} 2017/04/27

\begin{itemize}
\item
manual and install package
\item
first version published on CTAN
\end{itemize}

%%%%%%%%%%%%%%%%%%%%%%%%%%%%%%%%%%%%%%%%
\paragraph{v0.6:} 2017/04/26

\begin{itemize}
\item
redirection mechanism added
\end{itemize}

%%%%%%%%%%%%%%%%%%%%%%%%%%%%%%%%%%%%%%%%
\paragraph{v0.5:} 2017/04/26

\begin{itemize}
\item
functionality in definition file
\end{itemize}


%%%%%%%%%%%%%%%%%%%%%%%%%%%%%%%%%%%%%%%%%%%%%%%%%%%%%%%%%%%%%%%%%%%%%%%%%%%%%%%%
%%%%%%%%%%%%%%%%%%%%%%%%%%%%%%%%%%%%%%%%%%%%%%%%%%%%%%%%%%%%%%%%%%%%%%%%%%%%%%%%
%%%%%%%%%%%%%%%%%%%%%%%%%%%%%%%%%%%%%%%%%%%%%%%%%%%%%%%%%%%%%%%%%%%%%%%%%%%%%%%%
\appendix

\settowidth\MacroIndent{\rmfamily\scriptsize 000\ }

 \DocInput{childdoc.dtx}

\end{document}
%</driver>
% \fi
%
% %%%%%%%%%%%%%%%%%%%%%%%%%%%%%%%%%%%%%%%%%%%%%%%%%%%%%%%%%%%%%%%%%%%%%%%%%%%%%%
% %%%%%%%%%%%%%%%%%%%%%%%%%%%%%%%%%%%%%%%%%%%%%%%%%%%%%%%%%%%%%%%%%%%%%%%%%%%%%%
% \section{Sample}
%\iffalse
%<*samplemain>
%\fi
%
% The following presents a sample document
% with two chapters, two parts, a title page,
% a compile flag as well as three forwarding files to set the flag.
% It consists of eight |.tex| files:
% \begin{center}
% \begin{tabular}{ll}
% |cdocsamp.tex|&main file\\
% |cdocsch1.tex|&include file for chapter 1\\
% |cdocsch2.tex|&include file for chapter 2\\
% |cdocspt3.tex|&include file for part 3\\
% |cdocspt4.tex|&include file for part 4\\
% |cdocsdrf.tex|&forwarding file for main file in draft mode\\
% |cdocsfi1.tex|&forwarding file for final version of chapter 1\\
% |cdocsfi2.tex|&forwarding file for final version of chapter 2\\
% \end{tabular}
% \end{center}
% Each of the eight files can be compiled directly by the \LaTeX{} compiler.
%
% %%%%%%%%%%%%%%%%%%%%%%%%%%%%%%%%%%%%%%
% \paragraph{Main File.}
%
% The main file is called |cdocsamp.tex|.
%
% Load the \textsf{childdoc} definitions and
% declare the filename for the main document:
%    \begin{macrocode}
\input{childdoc.def}
\childdocmain{}
%    \end{macrocode}

% Optional override for |\version| flag:
%    \begin{macrocode}
%%\ifchilddoc\else\providecommand{\version}{draft}\fi
%    \end{macrocode}

% Define the default values for the |\version| flag
% (|final| for the main file and |draft| for childs):
%    \begin{macrocode}
\ifchilddoc
\providecommand{\version}{draft}
\else
\providecommand{\version}{final}
\fi
%    \end{macrocode}

% Load the standard document class:
%    \begin{macrocode}
\documentclass[12pt]{article}
%    \end{macrocode}

% Start the document body:
%    \begin{macrocode}
\begin{document}
%    \end{macrocode}

% Declare a title page.
% Print title, part of document being processed and version flag:
%    \begin{macrocode}
\addtocounter{page}{-1}
\begin{center}
{\LARGE\bfseries{}childdoc example\par}
\vspace{1cm}
\ifchilddoc
\ifchilddocmanual part\else chapter\fi:
`\childdocname' of `\childdocjob'\par
\else
main document: `\childdocjob'\par
\fi
version: \version\par
\end{center}
\newpage
%    \end{macrocode}

% Manually include selected file,
% otherwise process as usual:
%    \begin{macrocode}
\ifchilddocmanual
\section*{part `\childdocname'}
\input{\childdocname}
\else
%    \end{macrocode}

% Include the two chapters:
%    \begin{macrocode}
\include{cdocsch1}
\include{cdocsch2}
%    \end{macrocode}

% Include the two parts unless only chapters should be displayed:
%    \begin{macrocode}
\ifchilddoc\else
\section{part three}
\input{cdocspt3}
\section{part four}
\input{cdocspt4}
\fi
%    \end{macrocode}

% Process as usual until here:
%    \begin{macrocode}
\fi
%    \end{macrocode}

% End of document body:
%    \begin{macrocode}
\end{document}
%    \end{macrocode}
%\iffalse
%</samplemain>
%\fi
%
% %%%%%%%%%%%%%%%%%%%%%%%%%%%%%%%%%%%%%%
% \paragraph{Chapter Include Files.}
%
% The include files are called |cdocsch1.tex| and |cdocsch2.tex|.
%
%\iffalse
%<*samplechap1|samplechap2>
%\fi

% Optional override for |\version| flag:
%    \begin{macrocode}
%%\providecommand{\version}{final}
%    \end{macrocode}

% Include the main document:
%    \begin{macrocode}
\input{childdoc.def}
\childdocof{cdocsamp}
%    \end{macrocode}

%\iffalse
%</samplechap1|samplechap2>
%\fi
%
%\iffalse
%<*samplechap1>
%\fi
% Some text for chapter 1:
%    \begin{macrocode}
\section{one}
some text in chapter one
%    \end{macrocode}

%\iffalse
%</samplechap1>
%\fi
% Some text for chapter 2:
%\iffalse
%<*samplechap2>
%\fi
%    \begin{macrocode}
\section{two}
more text in chapter two
%    \end{macrocode}

%\iffalse
%</samplechap2>
%\fi
%
% %%%%%%%%%%%%%%%%%%%%%%%%%%%%%%%%%%%%%%
% \paragraph{Part Include Files.}
%
% The include files are called |cdocspt3.tex| and |cdocspt4.tex|.
%
%\iffalse
%<*samplepart3|samplepart4>
%\fi

% Optional override for |\version| flag:
%    \begin{macrocode}
%%\providecommand{\version}{final}
%    \end{macrocode}

% Include the main document:
%    \begin{macrocode}
\input{childdoc.def}
\childdocby{cdocsamp}
%    \end{macrocode}

%\iffalse
%</samplepart3|samplepart4>
%\fi
%
%\iffalse
%<*samplepart3>
%\fi
% Some text for part 3:
%    \begin{macrocode}
some text in part three
%    \end{macrocode}

%\iffalse
%</samplepart3>
%\fi
% Some text for part 4:
%\iffalse
%<*samplepart4>
%\fi
%    \begin{macrocode}
more text in part four
%    \end{macrocode}

%\iffalse
%</samplepart4>
%\fi
%
% %%%%%%%%%%%%%%%%%%%%%%%%%%%%%%%%%%%%%%
% \paragraph{Forwarding for a Complete Draft.}
%
% The following forwarding file |cdocsdrf.tex|
% compiles the main document in draft mode:
%\iffalse
%<*sampledraft>
%\fi
%    \begin{macrocode}
\def\version{draft}
\input{childdoc.def}
\childdocforward{cdocsamp}
%    \end{macrocode}

%\iffalse
%</sampledraft>
%\fi
%
% %%%%%%%%%%%%%%%%%%%%%%%%%%%%%%%%%%%%%%
% \paragraph{Forwarding for Final Version of the Chapters.}
%
% The following forwarding files |cdocsfn1.tex| and |cdocsfn2.tex|
% (with identical content)
% compile the final versions of the child documents
% |cdocsch1.tex| and |cdocsch2.tex|, respectively:
%\iffalse
%<*samplefinal>
%\fi
%    \begin{macrocode}
\def\version{final}
\input{childdoc.def}
\childdocforwardprefix[cdocsamp]{cdocsfn}{cdocsch}
%    \end{macrocode}

%\iffalse
%</samplefinal>
%\fi
%
% %%%%%%%%%%%%%%%%%%%%%%%%%%%%%%%%%%%%%%
% \paragraph{Command Line Processing.}
%
% The following three command lines generate the output files
% |cdocscld|, |cdocscl1| and |cdocscl2|
% which should be identical to
% |cdocsdrf|, |cdocsch1| and |cdocsfn2|, respectively:
% \begin{center}
% \begin{tabular}{l}
% |latex -jobname cdocscld \|\\
% |  "\def\version{draft}\input{childdoc.def}\childdocforward{cdocsamp}"|\\
% |latex -jobname cdocscl1 \|\\
% |  "\input{childdoc.def}\childdocforward[cdocsamp]{cdocsch1}"|\\
% |latex -jobname cdocscl2 \|\\
% |  "\def\version{final}\input{childdoc.def}\childdocforward{cdocsch2}"|
% \end{tabular}
% \end{center}
% Note that the trailing backslash on each first line
% merely continues the input to the second line
% (for convenient cut ant paste).
% Furthermore, the command |latex| can be replaced by any
% of its alternative versions such as |pdflatex|.
%
% %%%%%%%%%%%%%%%%%%%%%%%%%%%%%%%%%%%%%%%%%%%%%%%%%%%%%%%%%%%%%%%%%%%%%%%%%%%%%%
% %%%%%%%%%%%%%%%%%%%%%%%%%%%%%%%%%%%%%%%%%%%%%%%%%%%%%%%%%%%%%%%%%%%%%%%%%%%%%%
% \section{Implementation}
%\iffalse
%<*package>
%\fi
%
% This section describes the definitions file |childdoc.def|.

% The definitions cannot be loaded using |\usepackage| or |\RequirePackage|
% which has a mechanism to prevent loading a style file more than once.
% When loading the definitions by means of |\input|
% multiple instances have to be prevented manually:
%\iffalse
%This code needs to be before the `\ProvidesFile' directive
%which is defined at the beginning of this file.
%Therefore it is also placed there and commented out here.
%</package>
%<*discard>
%\fi
%    \begin{macrocode}
\ifdefined\childdocmain\endinput\fi
%    \end{macrocode}
%\iffalse
%</discard>
%<*package>
%\fi
%
% \macro{\ifchilddoc}
% \macro{\ifchilddocmanual}
% The conditional |\ifchilddoc| tells whether a
% child (true) or main (false) document is being compiled.
% The conditional |\ifchilddocmanual| tells whether
% the |\includeonly| mechanism is used (false) or
% the selection of child files must be performed manually (true).
% The definitions initialise to false:
%    \begin{macrocode}
\newif\ifchilddoc
\newif\ifchilddocmanual
%    \end{macrocode}

% \macro{\childdocname}
% \macro{\childdocjob}
% The macro |\childdocname| stores the name of the main document
% to be compiled. The macro |\childdocjob| stores the name of
% the document on which the \LaTeX{} compiler was originally invoked.
% The content of |\jobname| cannot be compared
% to filenames specified in the source due to different catcodes.
% The following code rescans |\jobname|, stores the result
% in |\childdocname| and saves a copy in |\childdocjob|:
%    \begin{macrocode}
\edef\childdocname{\scantokens\expandafter{\jobname\noexpand}}
\let\childdocjob\childdocname
%    \end{macrocode}

% \macro{\childdocdisable}
% The macro |\childdocdisable| prevents the main file
% from being processed more than once.
% At this stage, the main document command |\childdocmain|
% is assumed to be called once again where it should do nothing.
% Any subsequent call to it should prevent
% a secondary processing of the main document
% It overwrites the forwarding commands
% |\childdocof| and |\childdocforward|
% with empty macros to prevent further inclusions of the main document:
%    \begin{macrocode}
\newcommand{\childdocdisable}
{
  \renewcommand{\childdocmain}[1]{\renewcommand{\childdocmain}[1]{\endinput}}
  \renewcommand{\childdocof}[1]{}
  \renewcommand{\childdocby}[2][]{}
  \renewcommand{\childdocforward}[2][]{}
  \renewcommand{\childdocdisable}{}
}
%    \end{macrocode}

% \macro{\childdocmain}
% The macro |\childdocmain| is to be called at the top of the main file
% with nothing or the main filename (without extension) as argument.
% First, it breaks loops.
% If the argument is not empty and does not match |\childdocname|
% (which is set by the first inclusion of |childdoc.def|),
% |\ifchilddoc| is set to true, |\includeonly| is applied to the child file
% and |\jobname| is set to the main file
% (for proper handling of |.aux| files):
%    \begin{macrocode}
\newcommand{\childdocmain}[1]
{
  \childdocdisable\childdocmain{}
  \if?#1?\else
    \begingroup
      \def\childdoctmp{#1}
      \ifx\childdoctmp\childdocname
        \def\childdoctmp{}
      \else
        \def\childdoctmp
        {
          \childdoctrue
          \includeonly{\childdocname}
          \def\childdocjob{#1}
          \def\jobname{#1}
        }
      \fi
      \expandafter
    \endgroup
    \childdoctmp
  \fi
}
%    \end{macrocode}

% \macro{\childdocof}
% The command |\childdocof| redirects
% compilation to the main file |#1|.
%    \begin{macrocode}
\newcommand{\childdocof}[1]
{
  \childdocdisable
  \childdoctrue
  \includeonly{\childdocname}
  \def\jobname{#1}
  \def\childdocjob{#1}
  \input{#1}
}
%    \end{macrocode}

% \macro{\childdocby}
% The command |\childdocby| ....
%    \begin{macrocode}
\newcommand{\childdocby}[2][]
{
  \childdocdisable
  \childdoctrue
  \childdocmanualtrue
  \if?#1?\else
    \def\jobname{#2}
  \fi
  \def\childdocjob{#2}
  \input{#2}
  \endinput
}
%    \end{macrocode}

% \macro{\childdocforward}
% The command |\childdocforward| redirects
% compilation to the main file or
% (if the optional argument is given) a child file.
% Parameters are set as if the main file
% or a child file starting with |\childdocof| was compiled.
% Then compilation is handed over to the main file:
%    \begin{macrocode}
\newcommand{\childdocforward}[2][]
{
  \begingroup
    \if?#1?
      \def\childdoctmp
      {
        \def\childdocname{#2}
        \def\childdocjob{#2}
        \def\jobname{#2}
        \input{#2}
        \endinput
      }
    \else
      \def\childdoctmp
      {
        \childdocdisable
        \def\childdocname{#2}
        \childdoctrue
        \includeonly{#2}
        \def\childdocjob{#1}
        \def\jobname{#1}
        \input{#1}
        \endinput
      }
    \fi
    \expandafter
  \endgroup
  \childdoctmp
}
%    \end{macrocode}

% \macro{\childdocforwardprefix}
% The command |\childdocforwardprefix| redirects
% compilation to the main or a child file by means of a pattern.
% The prefix |#1| in the current filename is replaced by |#2|
% and the suffix of the current filename is kept
% (it is assumed that the filename does not contain the substring `|~~~|'
% which is used as a delimiter).
% Compilation is handed over to the new file by |\childdocforward|:
%    \begin{macrocode}
\newcommand{\childdocforwardprefix}[3][]
{
  \begingroup
    \def\childdocextract #2##1~~~{\def\childdoctmp{\childdocforward[#1]{#3##1}}}
    \expandafter\childdocextract\childdocname~~~
    \expandafter
  \endgroup
  \childdoctmp
}
%    \end{macrocode}

% \macro{\childdoc}
% The deprecated macro |\childdoc| is a legacy version of |\childdocmain|:
%    \begin{macrocode}
\newcommand{\childdoc}{\childdocmain}
%    \end{macrocode}

% \macro{\childdocredirect}
% The deprecated macro |\childdocredirect| is a legacy version
% of |\childdocforward| and |\childdocforwardprefix|:
%    \begin{macrocode}
\newcommand{\childdocredirect}[2][]
{
  \begingroup
    \if?#1?
      \def\childdoctmp{\childdocforward{#2}}
    \else
      \def\childdoctmp{\childdocforwardprefix{#1}{#2}}
    \fi
    \expandafter
  \endgroup
  \childdoctmp
}
%    \end{macrocode}

%\iffalse
%</package>
%\fi
%
\endinput
|\\
|\childdocby{|\textit{main}|}|\\
\end{tabular}
\end{center}
%
The directive |\childdocby| is similar to |\childdocof|
described in \secref{sec:include},
but the subsequent selection of content must be done manually.
To that end, both |\ifchilddoc| and |\ifchilddocmanual|
will be true upon processing of a part,
and the name of the part is stored in |\childdocname|.
Note that |\jobname| will be set to the filename of the current part
so that each part receives an individual |.aux| file
that does not interfere with the |.aux| file(s) of the main document.
This behaviour can be altered by the alternative form
|\childdocby[*]{|\textit{main}|}| (with a non-empty optional argument)
which uses the |.aux| file of the main document
by setting |\jobname| to \textit{main}.

%%%%%%%%%%%%%%%%%%%%%%%%%%%%%%%%%%%%%%%%%%%%%%%%%%%%%%%%%%%%%%%%%%%%%%%%%%%%%%%%
\subsection{Driver Development}
\label{sec:driver}

The \textsf{childdoc} mechanism can also be use for the development
of definition files such as \LaTeX{} styles or classes.
This case differs from the above setup with multiple parts
included by |\include| in that no |\includeonly| should be invoked.
This can be achieved by starting the include file
(before |\ProvidesPackage|) with:
%
\begin{center}
\begin{tabular}{l}
|% \iffalse
%
% childdoc.dtx Copyright (C) 2017-2018 Niklas Beisert
%
% This work may be distributed and/or modified under the
% conditions of the LaTeX Project Public License, either version 1.3
% of this license or (at your option) any later version.
% The latest version of this license is in
%   http://www.latex-project.org/lppl.txt
% and version 1.3 or later is part of all distributions of LaTeX
% version 2005/12/01 or later.
%
% This work has the LPPL maintenance status `maintained'.
%
% The Current Maintainer of this work is Niklas Beisert.
%
% This work consists of the files childdoc.dtx and childdoc.ins
% and the derived files childdoc.def and cdocsamp.tex with
% cdocsch1.tex, cdocsch2.tex, cdocsdrf.tex, cdocsfn1.tex, cdocsfn2.tex.
%
%<package>\ifdefined\childdocmain\endinput\fi
%<package>\ProvidesFile{childdoc.def}[2018/12/30 v2.0 child document driver]
%<samplemain>\ProvidesFile{cdocsamp.tex}[2018/12/30 v2.0 sample for childdoc]
%<*driver>
%\ProvidesFile{childdoc.drv}[2018/12/30 v2.0 childdoc reference manual file]
\PassOptionsToClass{10pt,a4paper}{article}
\documentclass{ltxdoc}

\usepackage[margin=35mm]{geometry}
\usepackage{hyperref}
\usepackage{hyperxmp}
\usepackage[usenames]{color}

\hypersetup{colorlinks=true}
\hypersetup{pdfstartview=FitH}
\hypersetup{pdfpagemode=UseNone}
\hypersetup{pdfsource={}}
\hypersetup{pdflang={en-UK}}
\hypersetup{pdfcopyright={Copyright 2017-2018 Niklas Beisert.
  This work may be distributed and/or modified under the
  conditions of the LaTeX Project Public License, either version 1.3
  of this license or (at your option) any later version.}}
\hypersetup{pdflicenseurl={http://www.latex-project.org/lppl.txt}}
\hypersetup{pdfcontactaddress={ETH Zurich, ITP, HIT K,
  Wolfgang-Pauli-Strasse 27}}
\hypersetup{pdfcontactpostcode={8093}}
\hypersetup{pdfcontactcity={Zurich}}
\hypersetup{pdfcontactcountry={Switzerland}}
\hypersetup{pdfcontactemail={nbeisert@itp.phys.ethz.ch}}
\hypersetup{pdfcontacturl={http://people.phys.ethz.ch/\xmptilde nbeisert/}}

\newcommand{\secref}[1]{\hyperref[#1]{section \ref*{#1}}}

\parskip1ex
\parindent0pt
\let\olditemize\itemize
\def\itemize{\olditemize\parskip0pt}

\begin{document}

\title{The \textsf{childdoc} Package}
\hypersetup{pdftitle={The childdoc Package}}
\author{Niklas Beisert\\[2ex]
  Institut f\"ur Theoretische Physik\\
  Eidgen\"ossische Technische Hochschule Z\"urich\\
  Wolfgang-Pauli-Strasse 27, 8093 Z\"urich, Switzerland\\[1ex]
  \href{mailto:nbeisert@itp.phys.ethz.ch}
  {\texttt{nbeisert@itp.phys.ethz.ch}}}
\hypersetup{pdfauthor={Niklas Beisert}}
\hypersetup{pdfsubject={Manual for the LaTeX2e Package childdoc}}
\date{30 December 2018, \textsf{v2.0}}
\maketitle

\begin{abstract}\noindent
\textsf{childdoc} is a \LaTeXe{} package
that enables the direct compilation
of document sections included by |\include|
to individual files.
\end{abstract}

\begingroup
\parskip0ex
\tableofcontents
\endgroup

%%%%%%%%%%%%%%%%%%%%%%%%%%%%%%%%%%%%%%%%%%%%%%%%%%%%%%%%%%%%%%%%%%%%%%%%%%%%%%%%
%%%%%%%%%%%%%%%%%%%%%%%%%%%%%%%%%%%%%%%%%%%%%%%%%%%%%%%%%%%%%%%%%%%%%%%%%%%%%%%%
\section{Introduction}

\LaTeX{} provides a mechanism to structure a large document (such as a book)
into a main file and several child files (containing the chapters)
using the |\include| command.
This mechanism is beneficial for documents
which span hundreds of pages in order to
make the source file(s) more manageable.
Moreover, compilation can be restricted to
selected child files by means of the |\includeonly| command.
The latter feature can be used to reduce the compilation time while editing
(this was significantly more useful in the earlier days of \LaTeX{})
or to generate a smaller document which is easier to navigate.
Another application of |\includeonly| is to generate
documents consisting of selected parts of the complete document.

However, there are a few drawbacks of the plain |\include| mechanism:
\begin{itemize}
\item
The child files cannot be compiled on their own,
they can only be compiled via the main file.
A naive editing environment
(such as a text editor with an option
to have the current file processed by \LaTeX)
may require one to switch to the main file before compiling;
attempting to compile the child file produces errors.
\item
The main file must be modified (each time)
to adjust the |\includeonly| command
to the present needs. This easily leaves the main file in a messy state.
\item
The generated document will always carry the filename
of the main document. This is inconvenient if
several child files are to be compiled and
to be kept for distribution.
\end{itemize}

The present package provides a simple interface
to make child files individually compilable by \LaTeX{}.
Compiling a child file then has the same effect as compiling
the main file with an |\includeonly| command
to select the appropriate child.
Moreover the generated document will carry the name of the child
rather than the main file.
This resolves all three above issues.

This feature is meant to make the editing of books,
thesis documents and lecture notes somewhat more convenient.
However, the package can also be used efficiently for
composing a series of documents (such as exercise sheets)
which are typically distributed individually.
It then assists the author in generating the individual documents
(potentially in different versions)
as well as a document containing the collected series.
Another application is in developing style files
or other kinds of included material
where compilation of the style file could redirect
to a sample or test file.

%%%%%%%%%%%%%%%%%%%%%%%%%%%%%%%%%%%%%%%%%%%%%%%%%%%%%%%%%%%%%%%%%%%%%%%%%%%%%%%%
%%%%%%%%%%%%%%%%%%%%%%%%%%%%%%%%%%%%%%%%%%%%%%%%%%%%%%%%%%%%%%%%%%%%%%%%%%%%%%%%
\section{Usage}

First of all, the package \textsf{childdoc} is \emph{not} a standard
\LaTeXe{} |.sty| style file! Therefore it needs to be invoked in
a non-standard way.

%%%%%%%%%%%%%%%%%%%%%%%%%%%%%%%%%%%%%%%%%%%%%%%%%%%%%%%%%%%%%%%%%%%%%%%%%%%%%%%%
\subsection{Included Files}
\label{sec:include}

%%%%%%%%%%%%%%%%%%%%%%%%%%%%%%%%%%%%%%%%
\DescribeMacro{\childdocmain}
To use the package, add the commands
\begin{center}
\begin{tabular}{l}
|\input{childdoc.def}|\\
|\childdocmain{}|\\
\end{tabular}
\end{center}
at the very top of the main \LaTeX{} file,
in particular \emph{before} the |\documentclass| statement!
The argument of |\childdocmain| should be left empty
(but it must be present).

%%%%%%%%%%%%%%%%%%%%%%%%%%%%%%%%%%%%%%%%
\DescribeMacro{\childdocof}
Furthermore, add the commands
\begin{center}
\begin{tabular}{l}
|\input{childdoc.def}|\\
|\childdocof{|\textit{main}|}|\\
\end{tabular}
\end{center}
at the top of every child file \textit{child}
which is included by |\include{|\textit{child}|}|
from within the main file
(or at least for those files to be compiled individually).
The argument \textit{main} must be the filename of the main file.

There are a couple of
considerations in setting up the main and child documents:

%%%%%%%%%%%%%%%%%%%%%%%%%%%%%%%%%%%%%%%%
\paragraph{Restrictions.}

Please note the following restrictions:
\begin{itemize}
\item
|\childdocmain| must be called with one argument \textit{main}
to ensure compatibility with earlier version of the package.
It must either be empty (|\childdocmain{}|)
or precisely match the filename of the main file in which it is specified.
See \secref{sec:detection} for further information.
\item
The filename \textit{main} must be specified without the |.tex| extension.
\item
The filename \textit{main} is case sensitive
(even in case-insensitive file systems)
due to internal string comparison.
\item
The argument \textit{main} should be fully expanded, it cannot be a macro.
\item
Subdirectories and special characters should be avoided in filenames.
\item
The command |\childdocmain{|\textit{main}|}| must be followed by a whitespace.
It should not be followed immediately by another command
or by a comment mark `|%|'.
This is because the \TeX{} parser reads the token immediately following
the argument of |\childdocmain| and puts it
at the beginning of every child section;
however, a white\-space is ignored.
\end{itemize}

%%%%%%%%%%%%%%%%%%%%%%%%%%%%%%%%%%%%%%%%
\paragraph{Content of Main File.}

It is advisable to place all content in the child files included by |\include|.
Any output contained in the main file will appear in all child documents
unless suppressed manually;
it cannot be suppressed automatically by the |\includeonly| directive
and thus should normally be avoided.
A method to include some content in the main file
by means of conditional processing is described in \secref{sec:conditional}.

%%%%%%%%%%%%%%%%%%%%%%%%%%%%%%%%%%%%%%%%
\paragraph{Page Numbering.}

When only a part of the document is compiled,
the appropriate numbering of pages
(as well as other status parameters)
is determined from the |.aux| files.
The latter contain information from previous passes.
However this information needs to propagate through
all intermediate child documents.
Therefore the page numbering in child documents may well
be inconsistent until the complete document is compiled at least once.

A useful (if unconventional) way to always ensure a consistent
page numbering is to restart the numbering in each child document
and denote the pages by `\textit{child}|.|\textit{page}'
where \textit{child} represents the chapter/section number of the child file.
This can be achieved by the command
|\numberwithin{page}{|\textit{child}|}|
of the \textsf{amsmath} package
where \textit{child} can be |chapter| or |section|
depending on the chosen structuring.
Alternatively, one can modify the macro |\thepage| appropriately
and reset the counter |page| at the start of each child file.

%%%%%%%%%%%%%%%%%%%%%%%%%%%%%%%%%%%%%%%%%%%%%%%%%%%%%%%%%%%%%%%%%%%%%%%%%%%%%%%%
\subsection{Conditional Processing}
\label{sec:conditional}

The package provides a mechanism to compile different versions
of a document. To customise the versions further some conditional processing
can come in handy to distinguish which version is being compiled.
The package provides two macros to describe the compilation context:

%%%%%%%%%%%%%%%%%%%%%%%%%%%%%%%%%%%%%%%%
\DescribeMacro{\ifchilddoc}
The conditional |\ifchilddoc| distinguishes between the compilation of
child documents and the main document:
%
\begin{center}
|\ifchilddoc |\textit{child-code}| |[|\||else |\textit{main-code}]| \||fi|
\end{center}

%%%%%%%%%%%%%%%%%%%%%%%%%%%%%%%%%%%%%%%%
\DescribeMacro{\childdocname}
\DescribeMacro{\childdocjob}
The macro |\childdocname| contains the filename (without extension)
of the main or child file being processed.
Note that |\childdocjob| will always contain the name of the main file.

%%%%%%%%%%%%%%%%%%%%%%%%%%%%%%%%%%%%%%%%
\paragraph{Title Page.}

Conditional processing can be used to include a title or banner page
in the main document when proper precautions are taken.
Importantly, the code in the main file should ensure that the page counter
(as well as other status parameters which are stored in the |.aux| files)
takes the same value after the conditional processing.
Otherwise the page numbers may take divergent values
depending on which part is compiled.

For example, a title page could be declared by:
%
\begin{center}
\begin{tabular}{l}
|\ifchilddoc\||else|\\
|\addtocounter{page}{-1}|\\
\textit{code for title page}\\
|\newpage|\\
|\||fi|
\end{tabular}
\end{center}
%
A banner page for the child documents can be generated by:
%
\begin{center}
\begin{tabular}{l}
|\ifchilddoc|\\
|\addtocounter{page}{-1}|\\
\textit{code for banner page}\\
|\newpage|\\
|\||fi|
\end{tabular}
\end{center}
%
Here one could write a message such as:
\begin{center}
|This is the part \childdocname{} of \childdocjob{}.|
\end{center}

%%%%%%%%%%%%%%%%%%%%%%%%%%%%%%%%%%%%%%%%%%%%%%%%%%%%%%%%%%%%%%%%%%%%%%%%%%%%%%%%
\subsection{Flags}
\label{sec:flags}

The package makes it easy to generate different versions
of the main or child documents.
To this end compilation flags can be defined
and assigned different default values.
They will be particularly useful in conjunction
with the forwarding mechanism described in \secref{sec:forward}.

For example, it may be useful to have a flag |\version|
which can be set to |draft| or |final|.
The document source will contain some conditional code
depending on the value of |\version|.
Suppose further, the flag should default to |final| for the main file
and to |draft| for child files
which is a natural assignment for editing the document.
This is achieved by placing the following code
in the preamble of the main document
(below the |\childdocmain| directive):
%
\begin{center}
\begin{tabular}{l}
|\ifchilddoc|\\
|\providecommand{\version}{draft}|\\
|\||else|\\
|\providecommand{\version}{final}|\\
|\||fi|
\end{tabular}
\end{center}
%
The definition by |\providecommand| makes sure
that previous definitions are not overwritten.
Further statements |\providecommand{\version}{...}|
can thus be added before the above code to override it.

For the main file, one might add a line
(between |\childdocmain| and the above block)
%
\begin{center}
|%\ifchilddoc\||else\providecommand{\version}{draft}\||fi|
\end{center}
%
which can be uncommented to produce a draft version.
Likewise one can add a line to the very top of a child file
(above the |\childdocof{|\textit{main}|}| directive)
%
\begin{center}
|%\providecommand{\version}{final}|
\end{center}
%
which can be uncommented to produce the final version of this child document.

%%%%%%%%%%%%%%%%%%%%%%%%%%%%%%%%%%%%%%%%%%%%%%%%%%%%%%%%%%%%%%%%%%%%%%%%%%%%%%%%
\subsection{Forwarding}
\label{sec:forward}

Different versions of the main or child documents
using compilation flags as described in \secref{sec:flags}
can be (permanently) stored in different files
for convenient compilation, viewing and distribution.
To this end, the package defines a command
to pass on compilation to a different file:

%%%%%%%%%%%%%%%%%%%%%%%%%%%%%%%%%%%%%%%%
\DescribeMacro{\childdocforward}
The command |\childdocforward| redirects processing to
another source file:
%
\begin{center}
\begin{tabular}{l}
|\input{childdoc.def}|\\
|\childdocforward[|\textit{main}|]{|\textit{dest}|}|\\
\end{tabular}
\end{center}
%
The argument \textit{dest} is the destination file
(without extension).
It should be the main file or one of the child files.
Note that further \textsf{childdoc} directives
such as |\childdocof| and |\childdocforward|
in the indicated file will be processed in this form.
The optional argument \textit{main}
passes on directly to the main file \textit{main}
while pretending to compile the child \textit{dest}.
This form behaves as if \textit{dest}
issues |\childdocof{|\textit{main}|}| right away,
and no further \textsf{childdoc} directives will be processed.

%%%%%%%%%%%%%%%%%%%%%%%%%%%%%%%%%%%%%%%%
\DescribeMacro{\...prefix}
In the alternative form |\childdocforwardprefix|,
%
\begin{center}
\begin{tabular}{l}
|\input{childdoc.def}|\\
|\childdocforwardprefix[|\textit{main}|]{|\textit{prefix}|}{|\textit{dest}|}|
\end{tabular}
\end{center}
%
the destination file is determined by a pattern
depending on the current file:
To make this work, the current file must be called
`{\textit{prefix}\hspace{0.2em}\textit{suffix}}'
with \textit{prefix} matching precisely the argument.
Processing is then passed on to the file
`{\textit{dest}\hspace{0.2em}\textit{suffix}}'.
Surely, the same effect is achieved by
directly specifying the
argument `{\textit{dest}\hspace{0.2em}\textit{suffix}}'
in the first form.
However, that requires to set up a different file
for each child. With the alternative form of the command
all these files can have exactly the same content
which simplifies setting them up and maintaining them.

For example, the following file |draft.tex|
with a compilation flag |\version| as described in \secref{sec:flags}
compiles the main document as a draft:
%
\begin{center}
\begin{tabular}{l}
|\def\version{draft}|\\
|\input{childdoc.def}|\\
|\childdocforward{|\textit{main}|}|
\end{tabular}
\end{center}
%
Likewise, the following files |final|\textit{nn}|.tex|
compile the final version of the child document
|child|\textit{nn}|.tex|:
%
\begin{center}
\begin{tabular}{l}
|\def\version{final}|\\
|\input{childdoc.def}|\\
|\childdocforwardprefix{final}{child}|
\end{tabular}
\end{center}
%

Note that when several versions of a main file and/or of each child file
are to be generated, it may be convenient to set up a |Makefile| or
shell script to automatise the process.

%%%%%%%%%%%%%%%%%%%%%%%%%%%%%%%%%%%%%%%%%%%%%%%%%%%%%%%%%%%%%%%%%%%%%%%%%%%%%%%%
\subsection{Command Line Processing}
\label{sec:commandline}

The effect of redirection files can also be achieved by invoking
the \LaTeX{} compiler with a more elaborate command line.
Most conveniently this should be done as part
of a shell script or a |Makefile|.

When using \textsf{childdoc} in the main file, the following
command lines effectively perform a redirection
(note that depending on the shell being used,
backslashes may have to be doubled: `|\|' $\to$ `|\\|'):
%
\begin{center}
|... -jobname "|\textit{target}|" |\\|"|[\textit{flags}]%
|\input{childdoc.def}\childdocforward[|\textit{main}|]{|\textit{dest}|}"|
\end{center}
%
Here \textit{target} is the name of the output file,
\textit{main} is the name of the main file
and \textit{dest} is the name of the main or child file to be processed
(all filenames without extensions).
The optional argument \textit{main} can be omitted
if \textit{main} matches \textit{dest}.
Optionally, compilation \textit{flags} can be defined via |\def| commands.
This command line makes the \TeX{} engine believe
it is compiling the file \textit{target}
whose content is specified as the latter parameter.
The provided code then forwards the processing to
\textit{main} or \textit{dest} as described in \secref{sec:forward}.

%%%%%%%%%%%%%%%%%%%%%%%%%%%%%%%%%%%%%%%%%%%%%%%%%%%%%%%%%%%%%%%%%%%%%%%%%%%%%%%%
\subsection{Include by Input}
\label{sec:input}

Including child documents by |\include| has some restrictions by design.
Most notably, the content of a child document always occupies
its own set of pages; pages cannot be shared between child documents.
Usually, this behaviour makes perfect sense
because each child document contain an essential part of the document.
However, in some situations it may be desirable to compose
a document from a collection of parts
without having mandatory page breaks between then.
For this case, the package
provides a mechanism to include parts
by |\input| which can also be processed individually.
However, by construction this mechanism
requires manual handling of the content to be output.

%%%%%%%%%%%%%%%%%%%%%%%%%%%%%%%%%%%%%%%%
\DescribeMacro{\ifchilddocmanual}
The main file should be prepared as usual, see \secref{sec:include}.
However, the document body must make a distinction
between processing of an individual part and of the main document, e.g.:
%
\begin{center}
\begin{tabular}{l}
|\ifchilddocmanual|\\
|\input{\childdocname}|\\
|\||else|\\
\textit{document body with }|\input{|\textit{part}|}|\\
|\||fi|
\end{tabular}
\end{center}
%
The conditional |\ifchilddocmanual| is true whenever
a part to be included by |\input| is being compiled,
and the name of the part is stored in |\childdocname|.

%%%%%%%%%%%%%%%%%%%%%%%%%%%%%%%%%%%%%%%%
\DescribeMacro{\childdocby}
Each part to be included by |\input| should start with:
%
\begin{center}
\begin{tabular}{l}
|\input{childdoc.def}|\\
|\childdocby{|\textit{main}|}|\\
\end{tabular}
\end{center}
%
The directive |\childdocby| is similar to |\childdocof|
described in \secref{sec:include},
but the subsequent selection of content must be done manually.
To that end, both |\ifchilddoc| and |\ifchilddocmanual|
will be true upon processing of a part,
and the name of the part is stored in |\childdocname|.
Note that |\jobname| will be set to the filename of the current part
so that each part receives an individual |.aux| file
that does not interfere with the |.aux| file(s) of the main document.
This behaviour can be altered by the alternative form
|\childdocby[*]{|\textit{main}|}| (with a non-empty optional argument)
which uses the |.aux| file of the main document
by setting |\jobname| to \textit{main}.

%%%%%%%%%%%%%%%%%%%%%%%%%%%%%%%%%%%%%%%%%%%%%%%%%%%%%%%%%%%%%%%%%%%%%%%%%%%%%%%%
\subsection{Driver Development}
\label{sec:driver}

The \textsf{childdoc} mechanism can also be use for the development
of definition files such as \LaTeX{} styles or classes.
This case differs from the above setup with multiple parts
included by |\include| in that no |\includeonly| should be invoked.
This can be achieved by starting the include file
(before |\ProvidesPackage|) with:
%
\begin{center}
\begin{tabular}{l}
|\input{childdoc.def}|\\
|\childdocforward{|\textit{main}|}|\\
\end{tabular}
\end{center}
%
or alternatively with:
%
\begin{center}
\begin{tabular}{l}
|\input{childdoc.def}|\\
|\childdocby{|\textit{main}|}|\\
\end{tabular}
\end{center}
%
Both forms have slightly different effects as described above.
The main file is prepared as usual, see \secref{sec:include}.

%%%%%%%%%%%%%%%%%%%%%%%%%%%%%%%%%%%%%%%%%%%%%%%%%%%%%%%%%%%%%%%%%%%%%%%%%%%%%%%%
\subsection{Legacy Detection}
\label{sec:detection}

The directive |\childdocmain| in the main file can detect
whether the complete document or merely a child is to be compiled
even without using the directive |\childdocof|.
This method is deprecated because it is less robust
and there is no compelling reason to use it;
it is merely provided for backward compatibility
and it may be removed in future versions.

If the detection mechanism is to be used,
it is mandatory to correctly specify
the filename of the main file as the argument of |\childdocmain|:
%
\begin{center}
\begin{tabular}{l}
|\input{childdoc.def}|\\
|\childdocmain{|\textit{main}|}|\\
\end{tabular}
\end{center}
%
If |\jobname| does not match the argument \textit{main} of |\childdocmain|,
it is assumed that |\jobname| points to the child file to be compiled.
When using |\childdocmain| with the main file specified as argument,
it suffices to start a child file
with just |\input{|\textit{main}|}|
without loading of the package and using |\childdocof|.
If instead all processing is done
with the appropriate \textsf{childdoc} directives,
the argument of \textit{main} of |\childdocmain| can be empty.

An alternative version of the command line processing described
in \secref{sec:commandline} using the detection mechanism reads:
%
\begin{center}
|... -jobname "|\textit{target}|" "|[\textit{flags}]%
[|\def\jobname{|\textit{dest}|}|]|\input{|\textit{main}|}"|
\end{center}

%%%%%%%%%%%%%%%%%%%%%%%%%%%%%%%%%%%%%%%%%%%%%%%%%%%%%%%%%%%%%%%%%%%%%%%%%%%%%%%%
\subsection{Manual Code}
\label{sec:manual}

In case one cannot be certain whether the definitions file |childdoc.def|
is installed on the target \TeX{} distribution
and one prefers not to ship it,
it is conceivable to paste a few relevant commands into the sources.

To that end, drop all statements |\input{childdoc.def}|
and perform the replacements as outlined below.
Instead of |\childdocmain{|\textit{main}|}| add the following code
to the top of the main file:
%
\begin{center}
\begin{tabular}{l}
|\||ifdefined\childdocname\endinput\||fi\newif\ifchilddoc|\\
|\edef\childdocname{\scantokens\expandafter{\jobname\noexpand}}|\\
|\def\childdocmain{|\textit{main}|}\||ifx\childdocmain\childdocname\||else|\\
|\childdoctrue\includeonly{\childdocname}\let\jobname\childdocmain\||fi|\\
\end{tabular}
\end{center}
%
Instead of |\childdocof{|\textit{main}|}| just include the main file
at the top of each child file:
%
\begin{center}
|\input{|\textit{main}|}|
\end{center}
%
A simple redirection |\childdocforward{|\textit{dest}|}| is achieved by:
%
\begin{center}
|\def\jobname{|\textit{dest}|}\input{\jobname}|
\end{center}
%
The redirection with prefix
|\childdocforwardprefix[|\textit{prefix}|]{|\textit{dest}|}|
is accomplished by:
%
\begin{center}
\begin{tabular}{l}
|{\edef\jobname{\scantokens\expandafter{\jobname\noexpand}}|\\
|\def\redirectjob |\textit{prefix}|#1~~~{\gdef\jobname{|\textit{dest}|#1}}|\\
|\expandafter\redirectjob\jobname~~~}\input{\jobname}|
\end{tabular}
\end{center}

In an alternative approach,
child documents can be compiled by a specific command line
without additional code or specific definitions:
%
\begin{center}
|... -jobname "|\textit{target}|" "|[\textit{flags}]%
|\includeonly{|\textit{dest}|}\input{|\textit{main}|}"|
\end{center}
%

%%%%%%%%%%%%%%%%%%%%%%%%%%%%%%%%%%%%%%%%%%%%%%%%%%%%%%%%%%%%%%%%%%%%%%%%%%%%%%%%
%%%%%%%%%%%%%%%%%%%%%%%%%%%%%%%%%%%%%%%%%%%%%%%%%%%%%%%%%%%%%%%%%%%%%%%%%%%%%%%%
\section{Information}

%%%%%%%%%%%%%%%%%%%%%%%%%%%%%%%%%%%%%%%%%%%%%%%%%%%%%%%%%%%%%%%%%%%%%%%%%%%%%%%%
\subsection{Copyright}

Copyright \copyright{} 2017--2018 Niklas Beisert

This work may be distributed and/or modified under the
conditions of the \LaTeX{} Project Public License, either version 1.3
of this license or (at your option) any later version.
The latest version of this license is in
  \url{http://www.latex-project.org/lppl.txt}
and version 1.3 or later is part of all distributions of \LaTeX{}
version 2005/12/01 or later.

This work has the LPPL maintenance status `maintained'.

The Current Maintainer of this work is Niklas Beisert.

This work consists of the files |README.txt|, |childdoc.ins| and |childdoc.dtx|
as well as the derived files |childdoc.def|, |cdocsamp.tex|
with |cdocsch1.tex|, |cdocsch2.tex|, |cdocspt3.tex|, |cdocspt4.tex|,
|cdocsdrf.tex|, |cdocsfn1.tex|, |cdocsfn2.tex|
as well as |childdoc.pdf|.

%%%%%%%%%%%%%%%%%%%%%%%%%%%%%%%%%%%%%%%%%%%%%%%%%%%%%%%%%%%%%%%%%%%%%%%%%%%%%%%%
\subsection{Files and Installation}

The package consists of the files:
%
\begin{center}
\begin{tabular}{ll}
    |README.txt|   & readme file \\
    |childdoc.ins| & installation file \\
    |childdoc.dtx| & source file \\
    |childdoc.def| & definition file \\
    |cdocsamp.tex| & sample main file \\
    |cdocsch1.tex| & sample include file \\
    |cdocsch2.tex| & sample include file \\
    |cdocspt3.tex| & sample part file \\
    |cdocspt4.tex| & sample part file \\
    |cdocsdrf.tex| & sample redirection file \\
    |cdocsfn1.tex| & sample redirection file \\
    |cdocsfn2.tex| & sample redirection file \\
    |childdoc.pdf| & manual
\end{tabular}
\end{center}
%
The distribution consists of the files
|README.txt|, |childdoc.ins| and |childdoc.dtx|.
%
\begin{itemize}
\item
Run (pdf)\LaTeX{} on |childdoc.dtx|
to compile the manual |childdoc.pdf| (this file).
\item
Run \LaTeX{} on |childdoc.ins| to create the definitions file |childdoc.def|
and the sample |cdocsamp.tex| with include files
|cdocsch1.tex|, |cdocsch2.tex|, |cdocspt3.tex|, |cdocspt4.tex|,
|cdocsdrf.tex|, |cdocsfn1.tex|, |cdocsfn2.tex|.
Then copy the file |childdoc.def| to an appropriate directory of your \LaTeX{}
distribution, e.g.\ \textit{texmf-root}|/tex/latex/childdoc|.
\end{itemize}

%%%%%%%%%%%%%%%%%%%%%%%%%%%%%%%%%%%%%%%%%%%%%%%%%%%%%%%%%%%%%%%%%%%%%%%%%%%%%%%%
\subsection{Related CTAN Packages}

There are several other packages which offer a similar functionality:
%
\begin{itemize}
\item
The packages
\href{http://ctan.org/pkg/docmute}{\textsf{docmute}},
\href{http://ctan.org/pkg/includex}{\textsf{includex}} and
\href{http://ctan.org/pkg/standalone}{\textsf{standalone}}
provide commands to include only the document body of
a child file thus allowing both files to be compiled individually.
\item
The packages \href{http://ctan.org/pkg/subdocs}{\textsf{subdocs}}
and \href{http://ctan.org/pkg/subfiles}{\textsf{subfiles}}
provide structures in which the main and child documents can be
encapsulated and allowing them to be compiled individually.
The inclusion mechanism is different from the conventional |\include|.
\item
The package \href{http://ctan.org/pkg/combine}{\textsf{combine}}
is an elaborate solution to combine several documents into one.
\end{itemize}
%
See also the CTAN topic \href{http://ctan.org/topic/subdocs}{\textsf{subdocs}}
for further related packages.
The present package differs from the above solutions in that
a document structure constructed with the conventional |\include| mechanism
just needs two extra commands at the top of every file
such that all constituent files can be compiled individually.

%%%%%%%%%%%%%%%%%%%%%%%%%%%%%%%%%%%%%%%%%%%%%%%%%%%%%%%%%%%%%%%%%%%%%%%%%%%%%%%%
%\subsection{Feature Suggestions}
%
%The following is a list of features which may be useful for future
%versions of this package:
%%
%\begin{itemize}
%\item
%\ldots
%\end{itemize}

%%%%%%%%%%%%%%%%%%%%%%%%%%%%%%%%%%%%%%%%%%%%%%%%%%%%%%%%%%%%%%%%%%%%%%%%%%%%%%%%
\subsection{Revision History}

%%%%%%%%%%%%%%%%%%%%%%%%%%%%%%%%%%%%%%%%
\paragraph{v2.0:} 2018/12/30

\begin{itemize}
\item
immediate forward processing
\item
added |\childdocby| mechanism
\item
manual restructured
\end{itemize}

%%%%%%%%%%%%%%%%%%%%%%%%%%%%%%%%%%%%%%%%
\paragraph{v1.6:} 2018/01/17

\begin{itemize}
\item
application for development of include files
\item
corrections to manual
\end{itemize}

%%%%%%%%%%%%%%%%%%%%%%%%%%%%%%%%%%%%%%%%
\paragraph{v1.5:} 2017/05/21

\begin{itemize}
\item
more complete structuring introduced
\item
|\childdocof| introduced
\item
|\childdoc| renamed to |\childdocmain|
\item
|\childredirect| renamed to |\childdocforward| and |\childdocforwardprefix|
and functionality expanded
\end{itemize}

%%%%%%%%%%%%%%%%%%%%%%%%%%%%%%%%%%%%%%%%
\paragraph{v1.0:} 2017/04/27

\begin{itemize}
\item
manual and install package
\item
first version published on CTAN
\end{itemize}

%%%%%%%%%%%%%%%%%%%%%%%%%%%%%%%%%%%%%%%%
\paragraph{v0.6:} 2017/04/26

\begin{itemize}
\item
redirection mechanism added
\end{itemize}

%%%%%%%%%%%%%%%%%%%%%%%%%%%%%%%%%%%%%%%%
\paragraph{v0.5:} 2017/04/26

\begin{itemize}
\item
functionality in definition file
\end{itemize}


%%%%%%%%%%%%%%%%%%%%%%%%%%%%%%%%%%%%%%%%%%%%%%%%%%%%%%%%%%%%%%%%%%%%%%%%%%%%%%%%
%%%%%%%%%%%%%%%%%%%%%%%%%%%%%%%%%%%%%%%%%%%%%%%%%%%%%%%%%%%%%%%%%%%%%%%%%%%%%%%%
%%%%%%%%%%%%%%%%%%%%%%%%%%%%%%%%%%%%%%%%%%%%%%%%%%%%%%%%%%%%%%%%%%%%%%%%%%%%%%%%
\appendix

\settowidth\MacroIndent{\rmfamily\scriptsize 000\ }

 \DocInput{childdoc.dtx}

\end{document}
%</driver>
% \fi
%
% %%%%%%%%%%%%%%%%%%%%%%%%%%%%%%%%%%%%%%%%%%%%%%%%%%%%%%%%%%%%%%%%%%%%%%%%%%%%%%
% %%%%%%%%%%%%%%%%%%%%%%%%%%%%%%%%%%%%%%%%%%%%%%%%%%%%%%%%%%%%%%%%%%%%%%%%%%%%%%
% \section{Sample}
%\iffalse
%<*samplemain>
%\fi
%
% The following presents a sample document
% with two chapters, two parts, a title page,
% a compile flag as well as three forwarding files to set the flag.
% It consists of eight |.tex| files:
% \begin{center}
% \begin{tabular}{ll}
% |cdocsamp.tex|&main file\\
% |cdocsch1.tex|&include file for chapter 1\\
% |cdocsch2.tex|&include file for chapter 2\\
% |cdocspt3.tex|&include file for part 3\\
% |cdocspt4.tex|&include file for part 4\\
% |cdocsdrf.tex|&forwarding file for main file in draft mode\\
% |cdocsfi1.tex|&forwarding file for final version of chapter 1\\
% |cdocsfi2.tex|&forwarding file for final version of chapter 2\\
% \end{tabular}
% \end{center}
% Each of the eight files can be compiled directly by the \LaTeX{} compiler.
%
% %%%%%%%%%%%%%%%%%%%%%%%%%%%%%%%%%%%%%%
% \paragraph{Main File.}
%
% The main file is called |cdocsamp.tex|.
%
% Load the \textsf{childdoc} definitions and
% declare the filename for the main document:
%    \begin{macrocode}
\input{childdoc.def}
\childdocmain{}
%    \end{macrocode}

% Optional override for |\version| flag:
%    \begin{macrocode}
%%\ifchilddoc\else\providecommand{\version}{draft}\fi
%    \end{macrocode}

% Define the default values for the |\version| flag
% (|final| for the main file and |draft| for childs):
%    \begin{macrocode}
\ifchilddoc
\providecommand{\version}{draft}
\else
\providecommand{\version}{final}
\fi
%    \end{macrocode}

% Load the standard document class:
%    \begin{macrocode}
\documentclass[12pt]{article}
%    \end{macrocode}

% Start the document body:
%    \begin{macrocode}
\begin{document}
%    \end{macrocode}

% Declare a title page.
% Print title, part of document being processed and version flag:
%    \begin{macrocode}
\addtocounter{page}{-1}
\begin{center}
{\LARGE\bfseries{}childdoc example\par}
\vspace{1cm}
\ifchilddoc
\ifchilddocmanual part\else chapter\fi:
`\childdocname' of `\childdocjob'\par
\else
main document: `\childdocjob'\par
\fi
version: \version\par
\end{center}
\newpage
%    \end{macrocode}

% Manually include selected file,
% otherwise process as usual:
%    \begin{macrocode}
\ifchilddocmanual
\section*{part `\childdocname'}
\input{\childdocname}
\else
%    \end{macrocode}

% Include the two chapters:
%    \begin{macrocode}
\include{cdocsch1}
\include{cdocsch2}
%    \end{macrocode}

% Include the two parts unless only chapters should be displayed:
%    \begin{macrocode}
\ifchilddoc\else
\section{part three}
\input{cdocspt3}
\section{part four}
\input{cdocspt4}
\fi
%    \end{macrocode}

% Process as usual until here:
%    \begin{macrocode}
\fi
%    \end{macrocode}

% End of document body:
%    \begin{macrocode}
\end{document}
%    \end{macrocode}
%\iffalse
%</samplemain>
%\fi
%
% %%%%%%%%%%%%%%%%%%%%%%%%%%%%%%%%%%%%%%
% \paragraph{Chapter Include Files.}
%
% The include files are called |cdocsch1.tex| and |cdocsch2.tex|.
%
%\iffalse
%<*samplechap1|samplechap2>
%\fi

% Optional override for |\version| flag:
%    \begin{macrocode}
%%\providecommand{\version}{final}
%    \end{macrocode}

% Include the main document:
%    \begin{macrocode}
\input{childdoc.def}
\childdocof{cdocsamp}
%    \end{macrocode}

%\iffalse
%</samplechap1|samplechap2>
%\fi
%
%\iffalse
%<*samplechap1>
%\fi
% Some text for chapter 1:
%    \begin{macrocode}
\section{one}
some text in chapter one
%    \end{macrocode}

%\iffalse
%</samplechap1>
%\fi
% Some text for chapter 2:
%\iffalse
%<*samplechap2>
%\fi
%    \begin{macrocode}
\section{two}
more text in chapter two
%    \end{macrocode}

%\iffalse
%</samplechap2>
%\fi
%
% %%%%%%%%%%%%%%%%%%%%%%%%%%%%%%%%%%%%%%
% \paragraph{Part Include Files.}
%
% The include files are called |cdocspt3.tex| and |cdocspt4.tex|.
%
%\iffalse
%<*samplepart3|samplepart4>
%\fi

% Optional override for |\version| flag:
%    \begin{macrocode}
%%\providecommand{\version}{final}
%    \end{macrocode}

% Include the main document:
%    \begin{macrocode}
\input{childdoc.def}
\childdocby{cdocsamp}
%    \end{macrocode}

%\iffalse
%</samplepart3|samplepart4>
%\fi
%
%\iffalse
%<*samplepart3>
%\fi
% Some text for part 3:
%    \begin{macrocode}
some text in part three
%    \end{macrocode}

%\iffalse
%</samplepart3>
%\fi
% Some text for part 4:
%\iffalse
%<*samplepart4>
%\fi
%    \begin{macrocode}
more text in part four
%    \end{macrocode}

%\iffalse
%</samplepart4>
%\fi
%
% %%%%%%%%%%%%%%%%%%%%%%%%%%%%%%%%%%%%%%
% \paragraph{Forwarding for a Complete Draft.}
%
% The following forwarding file |cdocsdrf.tex|
% compiles the main document in draft mode:
%\iffalse
%<*sampledraft>
%\fi
%    \begin{macrocode}
\def\version{draft}
\input{childdoc.def}
\childdocforward{cdocsamp}
%    \end{macrocode}

%\iffalse
%</sampledraft>
%\fi
%
% %%%%%%%%%%%%%%%%%%%%%%%%%%%%%%%%%%%%%%
% \paragraph{Forwarding for Final Version of the Chapters.}
%
% The following forwarding files |cdocsfn1.tex| and |cdocsfn2.tex|
% (with identical content)
% compile the final versions of the child documents
% |cdocsch1.tex| and |cdocsch2.tex|, respectively:
%\iffalse
%<*samplefinal>
%\fi
%    \begin{macrocode}
\def\version{final}
\input{childdoc.def}
\childdocforwardprefix[cdocsamp]{cdocsfn}{cdocsch}
%    \end{macrocode}

%\iffalse
%</samplefinal>
%\fi
%
% %%%%%%%%%%%%%%%%%%%%%%%%%%%%%%%%%%%%%%
% \paragraph{Command Line Processing.}
%
% The following three command lines generate the output files
% |cdocscld|, |cdocscl1| and |cdocscl2|
% which should be identical to
% |cdocsdrf|, |cdocsch1| and |cdocsfn2|, respectively:
% \begin{center}
% \begin{tabular}{l}
% |latex -jobname cdocscld \|\\
% |  "\def\version{draft}\input{childdoc.def}\childdocforward{cdocsamp}"|\\
% |latex -jobname cdocscl1 \|\\
% |  "\input{childdoc.def}\childdocforward[cdocsamp]{cdocsch1}"|\\
% |latex -jobname cdocscl2 \|\\
% |  "\def\version{final}\input{childdoc.def}\childdocforward{cdocsch2}"|
% \end{tabular}
% \end{center}
% Note that the trailing backslash on each first line
% merely continues the input to the second line
% (for convenient cut ant paste).
% Furthermore, the command |latex| can be replaced by any
% of its alternative versions such as |pdflatex|.
%
% %%%%%%%%%%%%%%%%%%%%%%%%%%%%%%%%%%%%%%%%%%%%%%%%%%%%%%%%%%%%%%%%%%%%%%%%%%%%%%
% %%%%%%%%%%%%%%%%%%%%%%%%%%%%%%%%%%%%%%%%%%%%%%%%%%%%%%%%%%%%%%%%%%%%%%%%%%%%%%
% \section{Implementation}
%\iffalse
%<*package>
%\fi
%
% This section describes the definitions file |childdoc.def|.

% The definitions cannot be loaded using |\usepackage| or |\RequirePackage|
% which has a mechanism to prevent loading a style file more than once.
% When loading the definitions by means of |\input|
% multiple instances have to be prevented manually:
%\iffalse
%This code needs to be before the `\ProvidesFile' directive
%which is defined at the beginning of this file.
%Therefore it is also placed there and commented out here.
%</package>
%<*discard>
%\fi
%    \begin{macrocode}
\ifdefined\childdocmain\endinput\fi
%    \end{macrocode}
%\iffalse
%</discard>
%<*package>
%\fi
%
% \macro{\ifchilddoc}
% \macro{\ifchilddocmanual}
% The conditional |\ifchilddoc| tells whether a
% child (true) or main (false) document is being compiled.
% The conditional |\ifchilddocmanual| tells whether
% the |\includeonly| mechanism is used (false) or
% the selection of child files must be performed manually (true).
% The definitions initialise to false:
%    \begin{macrocode}
\newif\ifchilddoc
\newif\ifchilddocmanual
%    \end{macrocode}

% \macro{\childdocname}
% \macro{\childdocjob}
% The macro |\childdocname| stores the name of the main document
% to be compiled. The macro |\childdocjob| stores the name of
% the document on which the \LaTeX{} compiler was originally invoked.
% The content of |\jobname| cannot be compared
% to filenames specified in the source due to different catcodes.
% The following code rescans |\jobname|, stores the result
% in |\childdocname| and saves a copy in |\childdocjob|:
%    \begin{macrocode}
\edef\childdocname{\scantokens\expandafter{\jobname\noexpand}}
\let\childdocjob\childdocname
%    \end{macrocode}

% \macro{\childdocdisable}
% The macro |\childdocdisable| prevents the main file
% from being processed more than once.
% At this stage, the main document command |\childdocmain|
% is assumed to be called once again where it should do nothing.
% Any subsequent call to it should prevent
% a secondary processing of the main document
% It overwrites the forwarding commands
% |\childdocof| and |\childdocforward|
% with empty macros to prevent further inclusions of the main document:
%    \begin{macrocode}
\newcommand{\childdocdisable}
{
  \renewcommand{\childdocmain}[1]{\renewcommand{\childdocmain}[1]{\endinput}}
  \renewcommand{\childdocof}[1]{}
  \renewcommand{\childdocby}[2][]{}
  \renewcommand{\childdocforward}[2][]{}
  \renewcommand{\childdocdisable}{}
}
%    \end{macrocode}

% \macro{\childdocmain}
% The macro |\childdocmain| is to be called at the top of the main file
% with nothing or the main filename (without extension) as argument.
% First, it breaks loops.
% If the argument is not empty and does not match |\childdocname|
% (which is set by the first inclusion of |childdoc.def|),
% |\ifchilddoc| is set to true, |\includeonly| is applied to the child file
% and |\jobname| is set to the main file
% (for proper handling of |.aux| files):
%    \begin{macrocode}
\newcommand{\childdocmain}[1]
{
  \childdocdisable\childdocmain{}
  \if?#1?\else
    \begingroup
      \def\childdoctmp{#1}
      \ifx\childdoctmp\childdocname
        \def\childdoctmp{}
      \else
        \def\childdoctmp
        {
          \childdoctrue
          \includeonly{\childdocname}
          \def\childdocjob{#1}
          \def\jobname{#1}
        }
      \fi
      \expandafter
    \endgroup
    \childdoctmp
  \fi
}
%    \end{macrocode}

% \macro{\childdocof}
% The command |\childdocof| redirects
% compilation to the main file |#1|.
%    \begin{macrocode}
\newcommand{\childdocof}[1]
{
  \childdocdisable
  \childdoctrue
  \includeonly{\childdocname}
  \def\jobname{#1}
  \def\childdocjob{#1}
  \input{#1}
}
%    \end{macrocode}

% \macro{\childdocby}
% The command |\childdocby| ....
%    \begin{macrocode}
\newcommand{\childdocby}[2][]
{
  \childdocdisable
  \childdoctrue
  \childdocmanualtrue
  \if?#1?\else
    \def\jobname{#2}
  \fi
  \def\childdocjob{#2}
  \input{#2}
  \endinput
}
%    \end{macrocode}

% \macro{\childdocforward}
% The command |\childdocforward| redirects
% compilation to the main file or
% (if the optional argument is given) a child file.
% Parameters are set as if the main file
% or a child file starting with |\childdocof| was compiled.
% Then compilation is handed over to the main file:
%    \begin{macrocode}
\newcommand{\childdocforward}[2][]
{
  \begingroup
    \if?#1?
      \def\childdoctmp
      {
        \def\childdocname{#2}
        \def\childdocjob{#2}
        \def\jobname{#2}
        \input{#2}
        \endinput
      }
    \else
      \def\childdoctmp
      {
        \childdocdisable
        \def\childdocname{#2}
        \childdoctrue
        \includeonly{#2}
        \def\childdocjob{#1}
        \def\jobname{#1}
        \input{#1}
        \endinput
      }
    \fi
    \expandafter
  \endgroup
  \childdoctmp
}
%    \end{macrocode}

% \macro{\childdocforwardprefix}
% The command |\childdocforwardprefix| redirects
% compilation to the main or a child file by means of a pattern.
% The prefix |#1| in the current filename is replaced by |#2|
% and the suffix of the current filename is kept
% (it is assumed that the filename does not contain the substring `|~~~|'
% which is used as a delimiter).
% Compilation is handed over to the new file by |\childdocforward|:
%    \begin{macrocode}
\newcommand{\childdocforwardprefix}[3][]
{
  \begingroup
    \def\childdocextract #2##1~~~{\def\childdoctmp{\childdocforward[#1]{#3##1}}}
    \expandafter\childdocextract\childdocname~~~
    \expandafter
  \endgroup
  \childdoctmp
}
%    \end{macrocode}

% \macro{\childdoc}
% The deprecated macro |\childdoc| is a legacy version of |\childdocmain|:
%    \begin{macrocode}
\newcommand{\childdoc}{\childdocmain}
%    \end{macrocode}

% \macro{\childdocredirect}
% The deprecated macro |\childdocredirect| is a legacy version
% of |\childdocforward| and |\childdocforwardprefix|:
%    \begin{macrocode}
\newcommand{\childdocredirect}[2][]
{
  \begingroup
    \if?#1?
      \def\childdoctmp{\childdocforward{#2}}
    \else
      \def\childdoctmp{\childdocforwardprefix{#1}{#2}}
    \fi
    \expandafter
  \endgroup
  \childdoctmp
}
%    \end{macrocode}

%\iffalse
%</package>
%\fi
%
\endinput
|\\
|\childdocforward{|\textit{main}|}|\\
\end{tabular}
\end{center}
%
or alternatively with:
%
\begin{center}
\begin{tabular}{l}
|% \iffalse
%
% childdoc.dtx Copyright (C) 2017-2018 Niklas Beisert
%
% This work may be distributed and/or modified under the
% conditions of the LaTeX Project Public License, either version 1.3
% of this license or (at your option) any later version.
% The latest version of this license is in
%   http://www.latex-project.org/lppl.txt
% and version 1.3 or later is part of all distributions of LaTeX
% version 2005/12/01 or later.
%
% This work has the LPPL maintenance status `maintained'.
%
% The Current Maintainer of this work is Niklas Beisert.
%
% This work consists of the files childdoc.dtx and childdoc.ins
% and the derived files childdoc.def and cdocsamp.tex with
% cdocsch1.tex, cdocsch2.tex, cdocsdrf.tex, cdocsfn1.tex, cdocsfn2.tex.
%
%<package>\ifdefined\childdocmain\endinput\fi
%<package>\ProvidesFile{childdoc.def}[2018/12/30 v2.0 child document driver]
%<samplemain>\ProvidesFile{cdocsamp.tex}[2018/12/30 v2.0 sample for childdoc]
%<*driver>
%\ProvidesFile{childdoc.drv}[2018/12/30 v2.0 childdoc reference manual file]
\PassOptionsToClass{10pt,a4paper}{article}
\documentclass{ltxdoc}

\usepackage[margin=35mm]{geometry}
\usepackage{hyperref}
\usepackage{hyperxmp}
\usepackage[usenames]{color}

\hypersetup{colorlinks=true}
\hypersetup{pdfstartview=FitH}
\hypersetup{pdfpagemode=UseNone}
\hypersetup{pdfsource={}}
\hypersetup{pdflang={en-UK}}
\hypersetup{pdfcopyright={Copyright 2017-2018 Niklas Beisert.
  This work may be distributed and/or modified under the
  conditions of the LaTeX Project Public License, either version 1.3
  of this license or (at your option) any later version.}}
\hypersetup{pdflicenseurl={http://www.latex-project.org/lppl.txt}}
\hypersetup{pdfcontactaddress={ETH Zurich, ITP, HIT K,
  Wolfgang-Pauli-Strasse 27}}
\hypersetup{pdfcontactpostcode={8093}}
\hypersetup{pdfcontactcity={Zurich}}
\hypersetup{pdfcontactcountry={Switzerland}}
\hypersetup{pdfcontactemail={nbeisert@itp.phys.ethz.ch}}
\hypersetup{pdfcontacturl={http://people.phys.ethz.ch/\xmptilde nbeisert/}}

\newcommand{\secref}[1]{\hyperref[#1]{section \ref*{#1}}}

\parskip1ex
\parindent0pt
\let\olditemize\itemize
\def\itemize{\olditemize\parskip0pt}

\begin{document}

\title{The \textsf{childdoc} Package}
\hypersetup{pdftitle={The childdoc Package}}
\author{Niklas Beisert\\[2ex]
  Institut f\"ur Theoretische Physik\\
  Eidgen\"ossische Technische Hochschule Z\"urich\\
  Wolfgang-Pauli-Strasse 27, 8093 Z\"urich, Switzerland\\[1ex]
  \href{mailto:nbeisert@itp.phys.ethz.ch}
  {\texttt{nbeisert@itp.phys.ethz.ch}}}
\hypersetup{pdfauthor={Niklas Beisert}}
\hypersetup{pdfsubject={Manual for the LaTeX2e Package childdoc}}
\date{30 December 2018, \textsf{v2.0}}
\maketitle

\begin{abstract}\noindent
\textsf{childdoc} is a \LaTeXe{} package
that enables the direct compilation
of document sections included by |\include|
to individual files.
\end{abstract}

\begingroup
\parskip0ex
\tableofcontents
\endgroup

%%%%%%%%%%%%%%%%%%%%%%%%%%%%%%%%%%%%%%%%%%%%%%%%%%%%%%%%%%%%%%%%%%%%%%%%%%%%%%%%
%%%%%%%%%%%%%%%%%%%%%%%%%%%%%%%%%%%%%%%%%%%%%%%%%%%%%%%%%%%%%%%%%%%%%%%%%%%%%%%%
\section{Introduction}

\LaTeX{} provides a mechanism to structure a large document (such as a book)
into a main file and several child files (containing the chapters)
using the |\include| command.
This mechanism is beneficial for documents
which span hundreds of pages in order to
make the source file(s) more manageable.
Moreover, compilation can be restricted to
selected child files by means of the |\includeonly| command.
The latter feature can be used to reduce the compilation time while editing
(this was significantly more useful in the earlier days of \LaTeX{})
or to generate a smaller document which is easier to navigate.
Another application of |\includeonly| is to generate
documents consisting of selected parts of the complete document.

However, there are a few drawbacks of the plain |\include| mechanism:
\begin{itemize}
\item
The child files cannot be compiled on their own,
they can only be compiled via the main file.
A naive editing environment
(such as a text editor with an option
to have the current file processed by \LaTeX)
may require one to switch to the main file before compiling;
attempting to compile the child file produces errors.
\item
The main file must be modified (each time)
to adjust the |\includeonly| command
to the present needs. This easily leaves the main file in a messy state.
\item
The generated document will always carry the filename
of the main document. This is inconvenient if
several child files are to be compiled and
to be kept for distribution.
\end{itemize}

The present package provides a simple interface
to make child files individually compilable by \LaTeX{}.
Compiling a child file then has the same effect as compiling
the main file with an |\includeonly| command
to select the appropriate child.
Moreover the generated document will carry the name of the child
rather than the main file.
This resolves all three above issues.

This feature is meant to make the editing of books,
thesis documents and lecture notes somewhat more convenient.
However, the package can also be used efficiently for
composing a series of documents (such as exercise sheets)
which are typically distributed individually.
It then assists the author in generating the individual documents
(potentially in different versions)
as well as a document containing the collected series.
Another application is in developing style files
or other kinds of included material
where compilation of the style file could redirect
to a sample or test file.

%%%%%%%%%%%%%%%%%%%%%%%%%%%%%%%%%%%%%%%%%%%%%%%%%%%%%%%%%%%%%%%%%%%%%%%%%%%%%%%%
%%%%%%%%%%%%%%%%%%%%%%%%%%%%%%%%%%%%%%%%%%%%%%%%%%%%%%%%%%%%%%%%%%%%%%%%%%%%%%%%
\section{Usage}

First of all, the package \textsf{childdoc} is \emph{not} a standard
\LaTeXe{} |.sty| style file! Therefore it needs to be invoked in
a non-standard way.

%%%%%%%%%%%%%%%%%%%%%%%%%%%%%%%%%%%%%%%%%%%%%%%%%%%%%%%%%%%%%%%%%%%%%%%%%%%%%%%%
\subsection{Included Files}
\label{sec:include}

%%%%%%%%%%%%%%%%%%%%%%%%%%%%%%%%%%%%%%%%
\DescribeMacro{\childdocmain}
To use the package, add the commands
\begin{center}
\begin{tabular}{l}
|\input{childdoc.def}|\\
|\childdocmain{}|\\
\end{tabular}
\end{center}
at the very top of the main \LaTeX{} file,
in particular \emph{before} the |\documentclass| statement!
The argument of |\childdocmain| should be left empty
(but it must be present).

%%%%%%%%%%%%%%%%%%%%%%%%%%%%%%%%%%%%%%%%
\DescribeMacro{\childdocof}
Furthermore, add the commands
\begin{center}
\begin{tabular}{l}
|\input{childdoc.def}|\\
|\childdocof{|\textit{main}|}|\\
\end{tabular}
\end{center}
at the top of every child file \textit{child}
which is included by |\include{|\textit{child}|}|
from within the main file
(or at least for those files to be compiled individually).
The argument \textit{main} must be the filename of the main file.

There are a couple of
considerations in setting up the main and child documents:

%%%%%%%%%%%%%%%%%%%%%%%%%%%%%%%%%%%%%%%%
\paragraph{Restrictions.}

Please note the following restrictions:
\begin{itemize}
\item
|\childdocmain| must be called with one argument \textit{main}
to ensure compatibility with earlier version of the package.
It must either be empty (|\childdocmain{}|)
or precisely match the filename of the main file in which it is specified.
See \secref{sec:detection} for further information.
\item
The filename \textit{main} must be specified without the |.tex| extension.
\item
The filename \textit{main} is case sensitive
(even in case-insensitive file systems)
due to internal string comparison.
\item
The argument \textit{main} should be fully expanded, it cannot be a macro.
\item
Subdirectories and special characters should be avoided in filenames.
\item
The command |\childdocmain{|\textit{main}|}| must be followed by a whitespace.
It should not be followed immediately by another command
or by a comment mark `|%|'.
This is because the \TeX{} parser reads the token immediately following
the argument of |\childdocmain| and puts it
at the beginning of every child section;
however, a white\-space is ignored.
\end{itemize}

%%%%%%%%%%%%%%%%%%%%%%%%%%%%%%%%%%%%%%%%
\paragraph{Content of Main File.}

It is advisable to place all content in the child files included by |\include|.
Any output contained in the main file will appear in all child documents
unless suppressed manually;
it cannot be suppressed automatically by the |\includeonly| directive
and thus should normally be avoided.
A method to include some content in the main file
by means of conditional processing is described in \secref{sec:conditional}.

%%%%%%%%%%%%%%%%%%%%%%%%%%%%%%%%%%%%%%%%
\paragraph{Page Numbering.}

When only a part of the document is compiled,
the appropriate numbering of pages
(as well as other status parameters)
is determined from the |.aux| files.
The latter contain information from previous passes.
However this information needs to propagate through
all intermediate child documents.
Therefore the page numbering in child documents may well
be inconsistent until the complete document is compiled at least once.

A useful (if unconventional) way to always ensure a consistent
page numbering is to restart the numbering in each child document
and denote the pages by `\textit{child}|.|\textit{page}'
where \textit{child} represents the chapter/section number of the child file.
This can be achieved by the command
|\numberwithin{page}{|\textit{child}|}|
of the \textsf{amsmath} package
where \textit{child} can be |chapter| or |section|
depending on the chosen structuring.
Alternatively, one can modify the macro |\thepage| appropriately
and reset the counter |page| at the start of each child file.

%%%%%%%%%%%%%%%%%%%%%%%%%%%%%%%%%%%%%%%%%%%%%%%%%%%%%%%%%%%%%%%%%%%%%%%%%%%%%%%%
\subsection{Conditional Processing}
\label{sec:conditional}

The package provides a mechanism to compile different versions
of a document. To customise the versions further some conditional processing
can come in handy to distinguish which version is being compiled.
The package provides two macros to describe the compilation context:

%%%%%%%%%%%%%%%%%%%%%%%%%%%%%%%%%%%%%%%%
\DescribeMacro{\ifchilddoc}
The conditional |\ifchilddoc| distinguishes between the compilation of
child documents and the main document:
%
\begin{center}
|\ifchilddoc |\textit{child-code}| |[|\||else |\textit{main-code}]| \||fi|
\end{center}

%%%%%%%%%%%%%%%%%%%%%%%%%%%%%%%%%%%%%%%%
\DescribeMacro{\childdocname}
\DescribeMacro{\childdocjob}
The macro |\childdocname| contains the filename (without extension)
of the main or child file being processed.
Note that |\childdocjob| will always contain the name of the main file.

%%%%%%%%%%%%%%%%%%%%%%%%%%%%%%%%%%%%%%%%
\paragraph{Title Page.}

Conditional processing can be used to include a title or banner page
in the main document when proper precautions are taken.
Importantly, the code in the main file should ensure that the page counter
(as well as other status parameters which are stored in the |.aux| files)
takes the same value after the conditional processing.
Otherwise the page numbers may take divergent values
depending on which part is compiled.

For example, a title page could be declared by:
%
\begin{center}
\begin{tabular}{l}
|\ifchilddoc\||else|\\
|\addtocounter{page}{-1}|\\
\textit{code for title page}\\
|\newpage|\\
|\||fi|
\end{tabular}
\end{center}
%
A banner page for the child documents can be generated by:
%
\begin{center}
\begin{tabular}{l}
|\ifchilddoc|\\
|\addtocounter{page}{-1}|\\
\textit{code for banner page}\\
|\newpage|\\
|\||fi|
\end{tabular}
\end{center}
%
Here one could write a message such as:
\begin{center}
|This is the part \childdocname{} of \childdocjob{}.|
\end{center}

%%%%%%%%%%%%%%%%%%%%%%%%%%%%%%%%%%%%%%%%%%%%%%%%%%%%%%%%%%%%%%%%%%%%%%%%%%%%%%%%
\subsection{Flags}
\label{sec:flags}

The package makes it easy to generate different versions
of the main or child documents.
To this end compilation flags can be defined
and assigned different default values.
They will be particularly useful in conjunction
with the forwarding mechanism described in \secref{sec:forward}.

For example, it may be useful to have a flag |\version|
which can be set to |draft| or |final|.
The document source will contain some conditional code
depending on the value of |\version|.
Suppose further, the flag should default to |final| for the main file
and to |draft| for child files
which is a natural assignment for editing the document.
This is achieved by placing the following code
in the preamble of the main document
(below the |\childdocmain| directive):
%
\begin{center}
\begin{tabular}{l}
|\ifchilddoc|\\
|\providecommand{\version}{draft}|\\
|\||else|\\
|\providecommand{\version}{final}|\\
|\||fi|
\end{tabular}
\end{center}
%
The definition by |\providecommand| makes sure
that previous definitions are not overwritten.
Further statements |\providecommand{\version}{...}|
can thus be added before the above code to override it.

For the main file, one might add a line
(between |\childdocmain| and the above block)
%
\begin{center}
|%\ifchilddoc\||else\providecommand{\version}{draft}\||fi|
\end{center}
%
which can be uncommented to produce a draft version.
Likewise one can add a line to the very top of a child file
(above the |\childdocof{|\textit{main}|}| directive)
%
\begin{center}
|%\providecommand{\version}{final}|
\end{center}
%
which can be uncommented to produce the final version of this child document.

%%%%%%%%%%%%%%%%%%%%%%%%%%%%%%%%%%%%%%%%%%%%%%%%%%%%%%%%%%%%%%%%%%%%%%%%%%%%%%%%
\subsection{Forwarding}
\label{sec:forward}

Different versions of the main or child documents
using compilation flags as described in \secref{sec:flags}
can be (permanently) stored in different files
for convenient compilation, viewing and distribution.
To this end, the package defines a command
to pass on compilation to a different file:

%%%%%%%%%%%%%%%%%%%%%%%%%%%%%%%%%%%%%%%%
\DescribeMacro{\childdocforward}
The command |\childdocforward| redirects processing to
another source file:
%
\begin{center}
\begin{tabular}{l}
|\input{childdoc.def}|\\
|\childdocforward[|\textit{main}|]{|\textit{dest}|}|\\
\end{tabular}
\end{center}
%
The argument \textit{dest} is the destination file
(without extension).
It should be the main file or one of the child files.
Note that further \textsf{childdoc} directives
such as |\childdocof| and |\childdocforward|
in the indicated file will be processed in this form.
The optional argument \textit{main}
passes on directly to the main file \textit{main}
while pretending to compile the child \textit{dest}.
This form behaves as if \textit{dest}
issues |\childdocof{|\textit{main}|}| right away,
and no further \textsf{childdoc} directives will be processed.

%%%%%%%%%%%%%%%%%%%%%%%%%%%%%%%%%%%%%%%%
\DescribeMacro{\...prefix}
In the alternative form |\childdocforwardprefix|,
%
\begin{center}
\begin{tabular}{l}
|\input{childdoc.def}|\\
|\childdocforwardprefix[|\textit{main}|]{|\textit{prefix}|}{|\textit{dest}|}|
\end{tabular}
\end{center}
%
the destination file is determined by a pattern
depending on the current file:
To make this work, the current file must be called
`{\textit{prefix}\hspace{0.2em}\textit{suffix}}'
with \textit{prefix} matching precisely the argument.
Processing is then passed on to the file
`{\textit{dest}\hspace{0.2em}\textit{suffix}}'.
Surely, the same effect is achieved by
directly specifying the
argument `{\textit{dest}\hspace{0.2em}\textit{suffix}}'
in the first form.
However, that requires to set up a different file
for each child. With the alternative form of the command
all these files can have exactly the same content
which simplifies setting them up and maintaining them.

For example, the following file |draft.tex|
with a compilation flag |\version| as described in \secref{sec:flags}
compiles the main document as a draft:
%
\begin{center}
\begin{tabular}{l}
|\def\version{draft}|\\
|\input{childdoc.def}|\\
|\childdocforward{|\textit{main}|}|
\end{tabular}
\end{center}
%
Likewise, the following files |final|\textit{nn}|.tex|
compile the final version of the child document
|child|\textit{nn}|.tex|:
%
\begin{center}
\begin{tabular}{l}
|\def\version{final}|\\
|\input{childdoc.def}|\\
|\childdocforwardprefix{final}{child}|
\end{tabular}
\end{center}
%

Note that when several versions of a main file and/or of each child file
are to be generated, it may be convenient to set up a |Makefile| or
shell script to automatise the process.

%%%%%%%%%%%%%%%%%%%%%%%%%%%%%%%%%%%%%%%%%%%%%%%%%%%%%%%%%%%%%%%%%%%%%%%%%%%%%%%%
\subsection{Command Line Processing}
\label{sec:commandline}

The effect of redirection files can also be achieved by invoking
the \LaTeX{} compiler with a more elaborate command line.
Most conveniently this should be done as part
of a shell script or a |Makefile|.

When using \textsf{childdoc} in the main file, the following
command lines effectively perform a redirection
(note that depending on the shell being used,
backslashes may have to be doubled: `|\|' $\to$ `|\\|'):
%
\begin{center}
|... -jobname "|\textit{target}|" |\\|"|[\textit{flags}]%
|\input{childdoc.def}\childdocforward[|\textit{main}|]{|\textit{dest}|}"|
\end{center}
%
Here \textit{target} is the name of the output file,
\textit{main} is the name of the main file
and \textit{dest} is the name of the main or child file to be processed
(all filenames without extensions).
The optional argument \textit{main} can be omitted
if \textit{main} matches \textit{dest}.
Optionally, compilation \textit{flags} can be defined via |\def| commands.
This command line makes the \TeX{} engine believe
it is compiling the file \textit{target}
whose content is specified as the latter parameter.
The provided code then forwards the processing to
\textit{main} or \textit{dest} as described in \secref{sec:forward}.

%%%%%%%%%%%%%%%%%%%%%%%%%%%%%%%%%%%%%%%%%%%%%%%%%%%%%%%%%%%%%%%%%%%%%%%%%%%%%%%%
\subsection{Include by Input}
\label{sec:input}

Including child documents by |\include| has some restrictions by design.
Most notably, the content of a child document always occupies
its own set of pages; pages cannot be shared between child documents.
Usually, this behaviour makes perfect sense
because each child document contain an essential part of the document.
However, in some situations it may be desirable to compose
a document from a collection of parts
without having mandatory page breaks between then.
For this case, the package
provides a mechanism to include parts
by |\input| which can also be processed individually.
However, by construction this mechanism
requires manual handling of the content to be output.

%%%%%%%%%%%%%%%%%%%%%%%%%%%%%%%%%%%%%%%%
\DescribeMacro{\ifchilddocmanual}
The main file should be prepared as usual, see \secref{sec:include}.
However, the document body must make a distinction
between processing of an individual part and of the main document, e.g.:
%
\begin{center}
\begin{tabular}{l}
|\ifchilddocmanual|\\
|\input{\childdocname}|\\
|\||else|\\
\textit{document body with }|\input{|\textit{part}|}|\\
|\||fi|
\end{tabular}
\end{center}
%
The conditional |\ifchilddocmanual| is true whenever
a part to be included by |\input| is being compiled,
and the name of the part is stored in |\childdocname|.

%%%%%%%%%%%%%%%%%%%%%%%%%%%%%%%%%%%%%%%%
\DescribeMacro{\childdocby}
Each part to be included by |\input| should start with:
%
\begin{center}
\begin{tabular}{l}
|\input{childdoc.def}|\\
|\childdocby{|\textit{main}|}|\\
\end{tabular}
\end{center}
%
The directive |\childdocby| is similar to |\childdocof|
described in \secref{sec:include},
but the subsequent selection of content must be done manually.
To that end, both |\ifchilddoc| and |\ifchilddocmanual|
will be true upon processing of a part,
and the name of the part is stored in |\childdocname|.
Note that |\jobname| will be set to the filename of the current part
so that each part receives an individual |.aux| file
that does not interfere with the |.aux| file(s) of the main document.
This behaviour can be altered by the alternative form
|\childdocby[*]{|\textit{main}|}| (with a non-empty optional argument)
which uses the |.aux| file of the main document
by setting |\jobname| to \textit{main}.

%%%%%%%%%%%%%%%%%%%%%%%%%%%%%%%%%%%%%%%%%%%%%%%%%%%%%%%%%%%%%%%%%%%%%%%%%%%%%%%%
\subsection{Driver Development}
\label{sec:driver}

The \textsf{childdoc} mechanism can also be use for the development
of definition files such as \LaTeX{} styles or classes.
This case differs from the above setup with multiple parts
included by |\include| in that no |\includeonly| should be invoked.
This can be achieved by starting the include file
(before |\ProvidesPackage|) with:
%
\begin{center}
\begin{tabular}{l}
|\input{childdoc.def}|\\
|\childdocforward{|\textit{main}|}|\\
\end{tabular}
\end{center}
%
or alternatively with:
%
\begin{center}
\begin{tabular}{l}
|\input{childdoc.def}|\\
|\childdocby{|\textit{main}|}|\\
\end{tabular}
\end{center}
%
Both forms have slightly different effects as described above.
The main file is prepared as usual, see \secref{sec:include}.

%%%%%%%%%%%%%%%%%%%%%%%%%%%%%%%%%%%%%%%%%%%%%%%%%%%%%%%%%%%%%%%%%%%%%%%%%%%%%%%%
\subsection{Legacy Detection}
\label{sec:detection}

The directive |\childdocmain| in the main file can detect
whether the complete document or merely a child is to be compiled
even without using the directive |\childdocof|.
This method is deprecated because it is less robust
and there is no compelling reason to use it;
it is merely provided for backward compatibility
and it may be removed in future versions.

If the detection mechanism is to be used,
it is mandatory to correctly specify
the filename of the main file as the argument of |\childdocmain|:
%
\begin{center}
\begin{tabular}{l}
|\input{childdoc.def}|\\
|\childdocmain{|\textit{main}|}|\\
\end{tabular}
\end{center}
%
If |\jobname| does not match the argument \textit{main} of |\childdocmain|,
it is assumed that |\jobname| points to the child file to be compiled.
When using |\childdocmain| with the main file specified as argument,
it suffices to start a child file
with just |\input{|\textit{main}|}|
without loading of the package and using |\childdocof|.
If instead all processing is done
with the appropriate \textsf{childdoc} directives,
the argument of \textit{main} of |\childdocmain| can be empty.

An alternative version of the command line processing described
in \secref{sec:commandline} using the detection mechanism reads:
%
\begin{center}
|... -jobname "|\textit{target}|" "|[\textit{flags}]%
[|\def\jobname{|\textit{dest}|}|]|\input{|\textit{main}|}"|
\end{center}

%%%%%%%%%%%%%%%%%%%%%%%%%%%%%%%%%%%%%%%%%%%%%%%%%%%%%%%%%%%%%%%%%%%%%%%%%%%%%%%%
\subsection{Manual Code}
\label{sec:manual}

In case one cannot be certain whether the definitions file |childdoc.def|
is installed on the target \TeX{} distribution
and one prefers not to ship it,
it is conceivable to paste a few relevant commands into the sources.

To that end, drop all statements |\input{childdoc.def}|
and perform the replacements as outlined below.
Instead of |\childdocmain{|\textit{main}|}| add the following code
to the top of the main file:
%
\begin{center}
\begin{tabular}{l}
|\||ifdefined\childdocname\endinput\||fi\newif\ifchilddoc|\\
|\edef\childdocname{\scantokens\expandafter{\jobname\noexpand}}|\\
|\def\childdocmain{|\textit{main}|}\||ifx\childdocmain\childdocname\||else|\\
|\childdoctrue\includeonly{\childdocname}\let\jobname\childdocmain\||fi|\\
\end{tabular}
\end{center}
%
Instead of |\childdocof{|\textit{main}|}| just include the main file
at the top of each child file:
%
\begin{center}
|\input{|\textit{main}|}|
\end{center}
%
A simple redirection |\childdocforward{|\textit{dest}|}| is achieved by:
%
\begin{center}
|\def\jobname{|\textit{dest}|}\input{\jobname}|
\end{center}
%
The redirection with prefix
|\childdocforwardprefix[|\textit{prefix}|]{|\textit{dest}|}|
is accomplished by:
%
\begin{center}
\begin{tabular}{l}
|{\edef\jobname{\scantokens\expandafter{\jobname\noexpand}}|\\
|\def\redirectjob |\textit{prefix}|#1~~~{\gdef\jobname{|\textit{dest}|#1}}|\\
|\expandafter\redirectjob\jobname~~~}\input{\jobname}|
\end{tabular}
\end{center}

In an alternative approach,
child documents can be compiled by a specific command line
without additional code or specific definitions:
%
\begin{center}
|... -jobname "|\textit{target}|" "|[\textit{flags}]%
|\includeonly{|\textit{dest}|}\input{|\textit{main}|}"|
\end{center}
%

%%%%%%%%%%%%%%%%%%%%%%%%%%%%%%%%%%%%%%%%%%%%%%%%%%%%%%%%%%%%%%%%%%%%%%%%%%%%%%%%
%%%%%%%%%%%%%%%%%%%%%%%%%%%%%%%%%%%%%%%%%%%%%%%%%%%%%%%%%%%%%%%%%%%%%%%%%%%%%%%%
\section{Information}

%%%%%%%%%%%%%%%%%%%%%%%%%%%%%%%%%%%%%%%%%%%%%%%%%%%%%%%%%%%%%%%%%%%%%%%%%%%%%%%%
\subsection{Copyright}

Copyright \copyright{} 2017--2018 Niklas Beisert

This work may be distributed and/or modified under the
conditions of the \LaTeX{} Project Public License, either version 1.3
of this license or (at your option) any later version.
The latest version of this license is in
  \url{http://www.latex-project.org/lppl.txt}
and version 1.3 or later is part of all distributions of \LaTeX{}
version 2005/12/01 or later.

This work has the LPPL maintenance status `maintained'.

The Current Maintainer of this work is Niklas Beisert.

This work consists of the files |README.txt|, |childdoc.ins| and |childdoc.dtx|
as well as the derived files |childdoc.def|, |cdocsamp.tex|
with |cdocsch1.tex|, |cdocsch2.tex|, |cdocspt3.tex|, |cdocspt4.tex|,
|cdocsdrf.tex|, |cdocsfn1.tex|, |cdocsfn2.tex|
as well as |childdoc.pdf|.

%%%%%%%%%%%%%%%%%%%%%%%%%%%%%%%%%%%%%%%%%%%%%%%%%%%%%%%%%%%%%%%%%%%%%%%%%%%%%%%%
\subsection{Files and Installation}

The package consists of the files:
%
\begin{center}
\begin{tabular}{ll}
    |README.txt|   & readme file \\
    |childdoc.ins| & installation file \\
    |childdoc.dtx| & source file \\
    |childdoc.def| & definition file \\
    |cdocsamp.tex| & sample main file \\
    |cdocsch1.tex| & sample include file \\
    |cdocsch2.tex| & sample include file \\
    |cdocspt3.tex| & sample part file \\
    |cdocspt4.tex| & sample part file \\
    |cdocsdrf.tex| & sample redirection file \\
    |cdocsfn1.tex| & sample redirection file \\
    |cdocsfn2.tex| & sample redirection file \\
    |childdoc.pdf| & manual
\end{tabular}
\end{center}
%
The distribution consists of the files
|README.txt|, |childdoc.ins| and |childdoc.dtx|.
%
\begin{itemize}
\item
Run (pdf)\LaTeX{} on |childdoc.dtx|
to compile the manual |childdoc.pdf| (this file).
\item
Run \LaTeX{} on |childdoc.ins| to create the definitions file |childdoc.def|
and the sample |cdocsamp.tex| with include files
|cdocsch1.tex|, |cdocsch2.tex|, |cdocspt3.tex|, |cdocspt4.tex|,
|cdocsdrf.tex|, |cdocsfn1.tex|, |cdocsfn2.tex|.
Then copy the file |childdoc.def| to an appropriate directory of your \LaTeX{}
distribution, e.g.\ \textit{texmf-root}|/tex/latex/childdoc|.
\end{itemize}

%%%%%%%%%%%%%%%%%%%%%%%%%%%%%%%%%%%%%%%%%%%%%%%%%%%%%%%%%%%%%%%%%%%%%%%%%%%%%%%%
\subsection{Related CTAN Packages}

There are several other packages which offer a similar functionality:
%
\begin{itemize}
\item
The packages
\href{http://ctan.org/pkg/docmute}{\textsf{docmute}},
\href{http://ctan.org/pkg/includex}{\textsf{includex}} and
\href{http://ctan.org/pkg/standalone}{\textsf{standalone}}
provide commands to include only the document body of
a child file thus allowing both files to be compiled individually.
\item
The packages \href{http://ctan.org/pkg/subdocs}{\textsf{subdocs}}
and \href{http://ctan.org/pkg/subfiles}{\textsf{subfiles}}
provide structures in which the main and child documents can be
encapsulated and allowing them to be compiled individually.
The inclusion mechanism is different from the conventional |\include|.
\item
The package \href{http://ctan.org/pkg/combine}{\textsf{combine}}
is an elaborate solution to combine several documents into one.
\end{itemize}
%
See also the CTAN topic \href{http://ctan.org/topic/subdocs}{\textsf{subdocs}}
for further related packages.
The present package differs from the above solutions in that
a document structure constructed with the conventional |\include| mechanism
just needs two extra commands at the top of every file
such that all constituent files can be compiled individually.

%%%%%%%%%%%%%%%%%%%%%%%%%%%%%%%%%%%%%%%%%%%%%%%%%%%%%%%%%%%%%%%%%%%%%%%%%%%%%%%%
%\subsection{Feature Suggestions}
%
%The following is a list of features which may be useful for future
%versions of this package:
%%
%\begin{itemize}
%\item
%\ldots
%\end{itemize}

%%%%%%%%%%%%%%%%%%%%%%%%%%%%%%%%%%%%%%%%%%%%%%%%%%%%%%%%%%%%%%%%%%%%%%%%%%%%%%%%
\subsection{Revision History}

%%%%%%%%%%%%%%%%%%%%%%%%%%%%%%%%%%%%%%%%
\paragraph{v2.0:} 2018/12/30

\begin{itemize}
\item
immediate forward processing
\item
added |\childdocby| mechanism
\item
manual restructured
\end{itemize}

%%%%%%%%%%%%%%%%%%%%%%%%%%%%%%%%%%%%%%%%
\paragraph{v1.6:} 2018/01/17

\begin{itemize}
\item
application for development of include files
\item
corrections to manual
\end{itemize}

%%%%%%%%%%%%%%%%%%%%%%%%%%%%%%%%%%%%%%%%
\paragraph{v1.5:} 2017/05/21

\begin{itemize}
\item
more complete structuring introduced
\item
|\childdocof| introduced
\item
|\childdoc| renamed to |\childdocmain|
\item
|\childredirect| renamed to |\childdocforward| and |\childdocforwardprefix|
and functionality expanded
\end{itemize}

%%%%%%%%%%%%%%%%%%%%%%%%%%%%%%%%%%%%%%%%
\paragraph{v1.0:} 2017/04/27

\begin{itemize}
\item
manual and install package
\item
first version published on CTAN
\end{itemize}

%%%%%%%%%%%%%%%%%%%%%%%%%%%%%%%%%%%%%%%%
\paragraph{v0.6:} 2017/04/26

\begin{itemize}
\item
redirection mechanism added
\end{itemize}

%%%%%%%%%%%%%%%%%%%%%%%%%%%%%%%%%%%%%%%%
\paragraph{v0.5:} 2017/04/26

\begin{itemize}
\item
functionality in definition file
\end{itemize}


%%%%%%%%%%%%%%%%%%%%%%%%%%%%%%%%%%%%%%%%%%%%%%%%%%%%%%%%%%%%%%%%%%%%%%%%%%%%%%%%
%%%%%%%%%%%%%%%%%%%%%%%%%%%%%%%%%%%%%%%%%%%%%%%%%%%%%%%%%%%%%%%%%%%%%%%%%%%%%%%%
%%%%%%%%%%%%%%%%%%%%%%%%%%%%%%%%%%%%%%%%%%%%%%%%%%%%%%%%%%%%%%%%%%%%%%%%%%%%%%%%
\appendix

\settowidth\MacroIndent{\rmfamily\scriptsize 000\ }

 \DocInput{childdoc.dtx}

\end{document}
%</driver>
% \fi
%
% %%%%%%%%%%%%%%%%%%%%%%%%%%%%%%%%%%%%%%%%%%%%%%%%%%%%%%%%%%%%%%%%%%%%%%%%%%%%%%
% %%%%%%%%%%%%%%%%%%%%%%%%%%%%%%%%%%%%%%%%%%%%%%%%%%%%%%%%%%%%%%%%%%%%%%%%%%%%%%
% \section{Sample}
%\iffalse
%<*samplemain>
%\fi
%
% The following presents a sample document
% with two chapters, two parts, a title page,
% a compile flag as well as three forwarding files to set the flag.
% It consists of eight |.tex| files:
% \begin{center}
% \begin{tabular}{ll}
% |cdocsamp.tex|&main file\\
% |cdocsch1.tex|&include file for chapter 1\\
% |cdocsch2.tex|&include file for chapter 2\\
% |cdocspt3.tex|&include file for part 3\\
% |cdocspt4.tex|&include file for part 4\\
% |cdocsdrf.tex|&forwarding file for main file in draft mode\\
% |cdocsfi1.tex|&forwarding file for final version of chapter 1\\
% |cdocsfi2.tex|&forwarding file for final version of chapter 2\\
% \end{tabular}
% \end{center}
% Each of the eight files can be compiled directly by the \LaTeX{} compiler.
%
% %%%%%%%%%%%%%%%%%%%%%%%%%%%%%%%%%%%%%%
% \paragraph{Main File.}
%
% The main file is called |cdocsamp.tex|.
%
% Load the \textsf{childdoc} definitions and
% declare the filename for the main document:
%    \begin{macrocode}
\input{childdoc.def}
\childdocmain{}
%    \end{macrocode}

% Optional override for |\version| flag:
%    \begin{macrocode}
%%\ifchilddoc\else\providecommand{\version}{draft}\fi
%    \end{macrocode}

% Define the default values for the |\version| flag
% (|final| for the main file and |draft| for childs):
%    \begin{macrocode}
\ifchilddoc
\providecommand{\version}{draft}
\else
\providecommand{\version}{final}
\fi
%    \end{macrocode}

% Load the standard document class:
%    \begin{macrocode}
\documentclass[12pt]{article}
%    \end{macrocode}

% Start the document body:
%    \begin{macrocode}
\begin{document}
%    \end{macrocode}

% Declare a title page.
% Print title, part of document being processed and version flag:
%    \begin{macrocode}
\addtocounter{page}{-1}
\begin{center}
{\LARGE\bfseries{}childdoc example\par}
\vspace{1cm}
\ifchilddoc
\ifchilddocmanual part\else chapter\fi:
`\childdocname' of `\childdocjob'\par
\else
main document: `\childdocjob'\par
\fi
version: \version\par
\end{center}
\newpage
%    \end{macrocode}

% Manually include selected file,
% otherwise process as usual:
%    \begin{macrocode}
\ifchilddocmanual
\section*{part `\childdocname'}
\input{\childdocname}
\else
%    \end{macrocode}

% Include the two chapters:
%    \begin{macrocode}
\include{cdocsch1}
\include{cdocsch2}
%    \end{macrocode}

% Include the two parts unless only chapters should be displayed:
%    \begin{macrocode}
\ifchilddoc\else
\section{part three}
\input{cdocspt3}
\section{part four}
\input{cdocspt4}
\fi
%    \end{macrocode}

% Process as usual until here:
%    \begin{macrocode}
\fi
%    \end{macrocode}

% End of document body:
%    \begin{macrocode}
\end{document}
%    \end{macrocode}
%\iffalse
%</samplemain>
%\fi
%
% %%%%%%%%%%%%%%%%%%%%%%%%%%%%%%%%%%%%%%
% \paragraph{Chapter Include Files.}
%
% The include files are called |cdocsch1.tex| and |cdocsch2.tex|.
%
%\iffalse
%<*samplechap1|samplechap2>
%\fi

% Optional override for |\version| flag:
%    \begin{macrocode}
%%\providecommand{\version}{final}
%    \end{macrocode}

% Include the main document:
%    \begin{macrocode}
\input{childdoc.def}
\childdocof{cdocsamp}
%    \end{macrocode}

%\iffalse
%</samplechap1|samplechap2>
%\fi
%
%\iffalse
%<*samplechap1>
%\fi
% Some text for chapter 1:
%    \begin{macrocode}
\section{one}
some text in chapter one
%    \end{macrocode}

%\iffalse
%</samplechap1>
%\fi
% Some text for chapter 2:
%\iffalse
%<*samplechap2>
%\fi
%    \begin{macrocode}
\section{two}
more text in chapter two
%    \end{macrocode}

%\iffalse
%</samplechap2>
%\fi
%
% %%%%%%%%%%%%%%%%%%%%%%%%%%%%%%%%%%%%%%
% \paragraph{Part Include Files.}
%
% The include files are called |cdocspt3.tex| and |cdocspt4.tex|.
%
%\iffalse
%<*samplepart3|samplepart4>
%\fi

% Optional override for |\version| flag:
%    \begin{macrocode}
%%\providecommand{\version}{final}
%    \end{macrocode}

% Include the main document:
%    \begin{macrocode}
\input{childdoc.def}
\childdocby{cdocsamp}
%    \end{macrocode}

%\iffalse
%</samplepart3|samplepart4>
%\fi
%
%\iffalse
%<*samplepart3>
%\fi
% Some text for part 3:
%    \begin{macrocode}
some text in part three
%    \end{macrocode}

%\iffalse
%</samplepart3>
%\fi
% Some text for part 4:
%\iffalse
%<*samplepart4>
%\fi
%    \begin{macrocode}
more text in part four
%    \end{macrocode}

%\iffalse
%</samplepart4>
%\fi
%
% %%%%%%%%%%%%%%%%%%%%%%%%%%%%%%%%%%%%%%
% \paragraph{Forwarding for a Complete Draft.}
%
% The following forwarding file |cdocsdrf.tex|
% compiles the main document in draft mode:
%\iffalse
%<*sampledraft>
%\fi
%    \begin{macrocode}
\def\version{draft}
\input{childdoc.def}
\childdocforward{cdocsamp}
%    \end{macrocode}

%\iffalse
%</sampledraft>
%\fi
%
% %%%%%%%%%%%%%%%%%%%%%%%%%%%%%%%%%%%%%%
% \paragraph{Forwarding for Final Version of the Chapters.}
%
% The following forwarding files |cdocsfn1.tex| and |cdocsfn2.tex|
% (with identical content)
% compile the final versions of the child documents
% |cdocsch1.tex| and |cdocsch2.tex|, respectively:
%\iffalse
%<*samplefinal>
%\fi
%    \begin{macrocode}
\def\version{final}
\input{childdoc.def}
\childdocforwardprefix[cdocsamp]{cdocsfn}{cdocsch}
%    \end{macrocode}

%\iffalse
%</samplefinal>
%\fi
%
% %%%%%%%%%%%%%%%%%%%%%%%%%%%%%%%%%%%%%%
% \paragraph{Command Line Processing.}
%
% The following three command lines generate the output files
% |cdocscld|, |cdocscl1| and |cdocscl2|
% which should be identical to
% |cdocsdrf|, |cdocsch1| and |cdocsfn2|, respectively:
% \begin{center}
% \begin{tabular}{l}
% |latex -jobname cdocscld \|\\
% |  "\def\version{draft}\input{childdoc.def}\childdocforward{cdocsamp}"|\\
% |latex -jobname cdocscl1 \|\\
% |  "\input{childdoc.def}\childdocforward[cdocsamp]{cdocsch1}"|\\
% |latex -jobname cdocscl2 \|\\
% |  "\def\version{final}\input{childdoc.def}\childdocforward{cdocsch2}"|
% \end{tabular}
% \end{center}
% Note that the trailing backslash on each first line
% merely continues the input to the second line
% (for convenient cut ant paste).
% Furthermore, the command |latex| can be replaced by any
% of its alternative versions such as |pdflatex|.
%
% %%%%%%%%%%%%%%%%%%%%%%%%%%%%%%%%%%%%%%%%%%%%%%%%%%%%%%%%%%%%%%%%%%%%%%%%%%%%%%
% %%%%%%%%%%%%%%%%%%%%%%%%%%%%%%%%%%%%%%%%%%%%%%%%%%%%%%%%%%%%%%%%%%%%%%%%%%%%%%
% \section{Implementation}
%\iffalse
%<*package>
%\fi
%
% This section describes the definitions file |childdoc.def|.

% The definitions cannot be loaded using |\usepackage| or |\RequirePackage|
% which has a mechanism to prevent loading a style file more than once.
% When loading the definitions by means of |\input|
% multiple instances have to be prevented manually:
%\iffalse
%This code needs to be before the `\ProvidesFile' directive
%which is defined at the beginning of this file.
%Therefore it is also placed there and commented out here.
%</package>
%<*discard>
%\fi
%    \begin{macrocode}
\ifdefined\childdocmain\endinput\fi
%    \end{macrocode}
%\iffalse
%</discard>
%<*package>
%\fi
%
% \macro{\ifchilddoc}
% \macro{\ifchilddocmanual}
% The conditional |\ifchilddoc| tells whether a
% child (true) or main (false) document is being compiled.
% The conditional |\ifchilddocmanual| tells whether
% the |\includeonly| mechanism is used (false) or
% the selection of child files must be performed manually (true).
% The definitions initialise to false:
%    \begin{macrocode}
\newif\ifchilddoc
\newif\ifchilddocmanual
%    \end{macrocode}

% \macro{\childdocname}
% \macro{\childdocjob}
% The macro |\childdocname| stores the name of the main document
% to be compiled. The macro |\childdocjob| stores the name of
% the document on which the \LaTeX{} compiler was originally invoked.
% The content of |\jobname| cannot be compared
% to filenames specified in the source due to different catcodes.
% The following code rescans |\jobname|, stores the result
% in |\childdocname| and saves a copy in |\childdocjob|:
%    \begin{macrocode}
\edef\childdocname{\scantokens\expandafter{\jobname\noexpand}}
\let\childdocjob\childdocname
%    \end{macrocode}

% \macro{\childdocdisable}
% The macro |\childdocdisable| prevents the main file
% from being processed more than once.
% At this stage, the main document command |\childdocmain|
% is assumed to be called once again where it should do nothing.
% Any subsequent call to it should prevent
% a secondary processing of the main document
% It overwrites the forwarding commands
% |\childdocof| and |\childdocforward|
% with empty macros to prevent further inclusions of the main document:
%    \begin{macrocode}
\newcommand{\childdocdisable}
{
  \renewcommand{\childdocmain}[1]{\renewcommand{\childdocmain}[1]{\endinput}}
  \renewcommand{\childdocof}[1]{}
  \renewcommand{\childdocby}[2][]{}
  \renewcommand{\childdocforward}[2][]{}
  \renewcommand{\childdocdisable}{}
}
%    \end{macrocode}

% \macro{\childdocmain}
% The macro |\childdocmain| is to be called at the top of the main file
% with nothing or the main filename (without extension) as argument.
% First, it breaks loops.
% If the argument is not empty and does not match |\childdocname|
% (which is set by the first inclusion of |childdoc.def|),
% |\ifchilddoc| is set to true, |\includeonly| is applied to the child file
% and |\jobname| is set to the main file
% (for proper handling of |.aux| files):
%    \begin{macrocode}
\newcommand{\childdocmain}[1]
{
  \childdocdisable\childdocmain{}
  \if?#1?\else
    \begingroup
      \def\childdoctmp{#1}
      \ifx\childdoctmp\childdocname
        \def\childdoctmp{}
      \else
        \def\childdoctmp
        {
          \childdoctrue
          \includeonly{\childdocname}
          \def\childdocjob{#1}
          \def\jobname{#1}
        }
      \fi
      \expandafter
    \endgroup
    \childdoctmp
  \fi
}
%    \end{macrocode}

% \macro{\childdocof}
% The command |\childdocof| redirects
% compilation to the main file |#1|.
%    \begin{macrocode}
\newcommand{\childdocof}[1]
{
  \childdocdisable
  \childdoctrue
  \includeonly{\childdocname}
  \def\jobname{#1}
  \def\childdocjob{#1}
  \input{#1}
}
%    \end{macrocode}

% \macro{\childdocby}
% The command |\childdocby| ....
%    \begin{macrocode}
\newcommand{\childdocby}[2][]
{
  \childdocdisable
  \childdoctrue
  \childdocmanualtrue
  \if?#1?\else
    \def\jobname{#2}
  \fi
  \def\childdocjob{#2}
  \input{#2}
  \endinput
}
%    \end{macrocode}

% \macro{\childdocforward}
% The command |\childdocforward| redirects
% compilation to the main file or
% (if the optional argument is given) a child file.
% Parameters are set as if the main file
% or a child file starting with |\childdocof| was compiled.
% Then compilation is handed over to the main file:
%    \begin{macrocode}
\newcommand{\childdocforward}[2][]
{
  \begingroup
    \if?#1?
      \def\childdoctmp
      {
        \def\childdocname{#2}
        \def\childdocjob{#2}
        \def\jobname{#2}
        \input{#2}
        \endinput
      }
    \else
      \def\childdoctmp
      {
        \childdocdisable
        \def\childdocname{#2}
        \childdoctrue
        \includeonly{#2}
        \def\childdocjob{#1}
        \def\jobname{#1}
        \input{#1}
        \endinput
      }
    \fi
    \expandafter
  \endgroup
  \childdoctmp
}
%    \end{macrocode}

% \macro{\childdocforwardprefix}
% The command |\childdocforwardprefix| redirects
% compilation to the main or a child file by means of a pattern.
% The prefix |#1| in the current filename is replaced by |#2|
% and the suffix of the current filename is kept
% (it is assumed that the filename does not contain the substring `|~~~|'
% which is used as a delimiter).
% Compilation is handed over to the new file by |\childdocforward|:
%    \begin{macrocode}
\newcommand{\childdocforwardprefix}[3][]
{
  \begingroup
    \def\childdocextract #2##1~~~{\def\childdoctmp{\childdocforward[#1]{#3##1}}}
    \expandafter\childdocextract\childdocname~~~
    \expandafter
  \endgroup
  \childdoctmp
}
%    \end{macrocode}

% \macro{\childdoc}
% The deprecated macro |\childdoc| is a legacy version of |\childdocmain|:
%    \begin{macrocode}
\newcommand{\childdoc}{\childdocmain}
%    \end{macrocode}

% \macro{\childdocredirect}
% The deprecated macro |\childdocredirect| is a legacy version
% of |\childdocforward| and |\childdocforwardprefix|:
%    \begin{macrocode}
\newcommand{\childdocredirect}[2][]
{
  \begingroup
    \if?#1?
      \def\childdoctmp{\childdocforward{#2}}
    \else
      \def\childdoctmp{\childdocforwardprefix{#1}{#2}}
    \fi
    \expandafter
  \endgroup
  \childdoctmp
}
%    \end{macrocode}

%\iffalse
%</package>
%\fi
%
\endinput
|\\
|\childdocby{|\textit{main}|}|\\
\end{tabular}
\end{center}
%
Both forms have slightly different effects as described above.
The main file is prepared as usual, see \secref{sec:include}.

%%%%%%%%%%%%%%%%%%%%%%%%%%%%%%%%%%%%%%%%%%%%%%%%%%%%%%%%%%%%%%%%%%%%%%%%%%%%%%%%
\subsection{Legacy Detection}
\label{sec:detection}

The directive |\childdocmain| in the main file can detect
whether the complete document or merely a child is to be compiled
even without using the directive |\childdocof|.
This method is deprecated because it is less robust
and there is no compelling reason to use it;
it is merely provided for backward compatibility
and it may be removed in future versions.

If the detection mechanism is to be used,
it is mandatory to correctly specify
the filename of the main file as the argument of |\childdocmain|:
%
\begin{center}
\begin{tabular}{l}
|% \iffalse
%
% childdoc.dtx Copyright (C) 2017-2018 Niklas Beisert
%
% This work may be distributed and/or modified under the
% conditions of the LaTeX Project Public License, either version 1.3
% of this license or (at your option) any later version.
% The latest version of this license is in
%   http://www.latex-project.org/lppl.txt
% and version 1.3 or later is part of all distributions of LaTeX
% version 2005/12/01 or later.
%
% This work has the LPPL maintenance status `maintained'.
%
% The Current Maintainer of this work is Niklas Beisert.
%
% This work consists of the files childdoc.dtx and childdoc.ins
% and the derived files childdoc.def and cdocsamp.tex with
% cdocsch1.tex, cdocsch2.tex, cdocsdrf.tex, cdocsfn1.tex, cdocsfn2.tex.
%
%<package>\ifdefined\childdocmain\endinput\fi
%<package>\ProvidesFile{childdoc.def}[2018/12/30 v2.0 child document driver]
%<samplemain>\ProvidesFile{cdocsamp.tex}[2018/12/30 v2.0 sample for childdoc]
%<*driver>
%\ProvidesFile{childdoc.drv}[2018/12/30 v2.0 childdoc reference manual file]
\PassOptionsToClass{10pt,a4paper}{article}
\documentclass{ltxdoc}

\usepackage[margin=35mm]{geometry}
\usepackage{hyperref}
\usepackage{hyperxmp}
\usepackage[usenames]{color}

\hypersetup{colorlinks=true}
\hypersetup{pdfstartview=FitH}
\hypersetup{pdfpagemode=UseNone}
\hypersetup{pdfsource={}}
\hypersetup{pdflang={en-UK}}
\hypersetup{pdfcopyright={Copyright 2017-2018 Niklas Beisert.
  This work may be distributed and/or modified under the
  conditions of the LaTeX Project Public License, either version 1.3
  of this license or (at your option) any later version.}}
\hypersetup{pdflicenseurl={http://www.latex-project.org/lppl.txt}}
\hypersetup{pdfcontactaddress={ETH Zurich, ITP, HIT K,
  Wolfgang-Pauli-Strasse 27}}
\hypersetup{pdfcontactpostcode={8093}}
\hypersetup{pdfcontactcity={Zurich}}
\hypersetup{pdfcontactcountry={Switzerland}}
\hypersetup{pdfcontactemail={nbeisert@itp.phys.ethz.ch}}
\hypersetup{pdfcontacturl={http://people.phys.ethz.ch/\xmptilde nbeisert/}}

\newcommand{\secref}[1]{\hyperref[#1]{section \ref*{#1}}}

\parskip1ex
\parindent0pt
\let\olditemize\itemize
\def\itemize{\olditemize\parskip0pt}

\begin{document}

\title{The \textsf{childdoc} Package}
\hypersetup{pdftitle={The childdoc Package}}
\author{Niklas Beisert\\[2ex]
  Institut f\"ur Theoretische Physik\\
  Eidgen\"ossische Technische Hochschule Z\"urich\\
  Wolfgang-Pauli-Strasse 27, 8093 Z\"urich, Switzerland\\[1ex]
  \href{mailto:nbeisert@itp.phys.ethz.ch}
  {\texttt{nbeisert@itp.phys.ethz.ch}}}
\hypersetup{pdfauthor={Niklas Beisert}}
\hypersetup{pdfsubject={Manual for the LaTeX2e Package childdoc}}
\date{30 December 2018, \textsf{v2.0}}
\maketitle

\begin{abstract}\noindent
\textsf{childdoc} is a \LaTeXe{} package
that enables the direct compilation
of document sections included by |\include|
to individual files.
\end{abstract}

\begingroup
\parskip0ex
\tableofcontents
\endgroup

%%%%%%%%%%%%%%%%%%%%%%%%%%%%%%%%%%%%%%%%%%%%%%%%%%%%%%%%%%%%%%%%%%%%%%%%%%%%%%%%
%%%%%%%%%%%%%%%%%%%%%%%%%%%%%%%%%%%%%%%%%%%%%%%%%%%%%%%%%%%%%%%%%%%%%%%%%%%%%%%%
\section{Introduction}

\LaTeX{} provides a mechanism to structure a large document (such as a book)
into a main file and several child files (containing the chapters)
using the |\include| command.
This mechanism is beneficial for documents
which span hundreds of pages in order to
make the source file(s) more manageable.
Moreover, compilation can be restricted to
selected child files by means of the |\includeonly| command.
The latter feature can be used to reduce the compilation time while editing
(this was significantly more useful in the earlier days of \LaTeX{})
or to generate a smaller document which is easier to navigate.
Another application of |\includeonly| is to generate
documents consisting of selected parts of the complete document.

However, there are a few drawbacks of the plain |\include| mechanism:
\begin{itemize}
\item
The child files cannot be compiled on their own,
they can only be compiled via the main file.
A naive editing environment
(such as a text editor with an option
to have the current file processed by \LaTeX)
may require one to switch to the main file before compiling;
attempting to compile the child file produces errors.
\item
The main file must be modified (each time)
to adjust the |\includeonly| command
to the present needs. This easily leaves the main file in a messy state.
\item
The generated document will always carry the filename
of the main document. This is inconvenient if
several child files are to be compiled and
to be kept for distribution.
\end{itemize}

The present package provides a simple interface
to make child files individually compilable by \LaTeX{}.
Compiling a child file then has the same effect as compiling
the main file with an |\includeonly| command
to select the appropriate child.
Moreover the generated document will carry the name of the child
rather than the main file.
This resolves all three above issues.

This feature is meant to make the editing of books,
thesis documents and lecture notes somewhat more convenient.
However, the package can also be used efficiently for
composing a series of documents (such as exercise sheets)
which are typically distributed individually.
It then assists the author in generating the individual documents
(potentially in different versions)
as well as a document containing the collected series.
Another application is in developing style files
or other kinds of included material
where compilation of the style file could redirect
to a sample or test file.

%%%%%%%%%%%%%%%%%%%%%%%%%%%%%%%%%%%%%%%%%%%%%%%%%%%%%%%%%%%%%%%%%%%%%%%%%%%%%%%%
%%%%%%%%%%%%%%%%%%%%%%%%%%%%%%%%%%%%%%%%%%%%%%%%%%%%%%%%%%%%%%%%%%%%%%%%%%%%%%%%
\section{Usage}

First of all, the package \textsf{childdoc} is \emph{not} a standard
\LaTeXe{} |.sty| style file! Therefore it needs to be invoked in
a non-standard way.

%%%%%%%%%%%%%%%%%%%%%%%%%%%%%%%%%%%%%%%%%%%%%%%%%%%%%%%%%%%%%%%%%%%%%%%%%%%%%%%%
\subsection{Included Files}
\label{sec:include}

%%%%%%%%%%%%%%%%%%%%%%%%%%%%%%%%%%%%%%%%
\DescribeMacro{\childdocmain}
To use the package, add the commands
\begin{center}
\begin{tabular}{l}
|\input{childdoc.def}|\\
|\childdocmain{}|\\
\end{tabular}
\end{center}
at the very top of the main \LaTeX{} file,
in particular \emph{before} the |\documentclass| statement!
The argument of |\childdocmain| should be left empty
(but it must be present).

%%%%%%%%%%%%%%%%%%%%%%%%%%%%%%%%%%%%%%%%
\DescribeMacro{\childdocof}
Furthermore, add the commands
\begin{center}
\begin{tabular}{l}
|\input{childdoc.def}|\\
|\childdocof{|\textit{main}|}|\\
\end{tabular}
\end{center}
at the top of every child file \textit{child}
which is included by |\include{|\textit{child}|}|
from within the main file
(or at least for those files to be compiled individually).
The argument \textit{main} must be the filename of the main file.

There are a couple of
considerations in setting up the main and child documents:

%%%%%%%%%%%%%%%%%%%%%%%%%%%%%%%%%%%%%%%%
\paragraph{Restrictions.}

Please note the following restrictions:
\begin{itemize}
\item
|\childdocmain| must be called with one argument \textit{main}
to ensure compatibility with earlier version of the package.
It must either be empty (|\childdocmain{}|)
or precisely match the filename of the main file in which it is specified.
See \secref{sec:detection} for further information.
\item
The filename \textit{main} must be specified without the |.tex| extension.
\item
The filename \textit{main} is case sensitive
(even in case-insensitive file systems)
due to internal string comparison.
\item
The argument \textit{main} should be fully expanded, it cannot be a macro.
\item
Subdirectories and special characters should be avoided in filenames.
\item
The command |\childdocmain{|\textit{main}|}| must be followed by a whitespace.
It should not be followed immediately by another command
or by a comment mark `|%|'.
This is because the \TeX{} parser reads the token immediately following
the argument of |\childdocmain| and puts it
at the beginning of every child section;
however, a white\-space is ignored.
\end{itemize}

%%%%%%%%%%%%%%%%%%%%%%%%%%%%%%%%%%%%%%%%
\paragraph{Content of Main File.}

It is advisable to place all content in the child files included by |\include|.
Any output contained in the main file will appear in all child documents
unless suppressed manually;
it cannot be suppressed automatically by the |\includeonly| directive
and thus should normally be avoided.
A method to include some content in the main file
by means of conditional processing is described in \secref{sec:conditional}.

%%%%%%%%%%%%%%%%%%%%%%%%%%%%%%%%%%%%%%%%
\paragraph{Page Numbering.}

When only a part of the document is compiled,
the appropriate numbering of pages
(as well as other status parameters)
is determined from the |.aux| files.
The latter contain information from previous passes.
However this information needs to propagate through
all intermediate child documents.
Therefore the page numbering in child documents may well
be inconsistent until the complete document is compiled at least once.

A useful (if unconventional) way to always ensure a consistent
page numbering is to restart the numbering in each child document
and denote the pages by `\textit{child}|.|\textit{page}'
where \textit{child} represents the chapter/section number of the child file.
This can be achieved by the command
|\numberwithin{page}{|\textit{child}|}|
of the \textsf{amsmath} package
where \textit{child} can be |chapter| or |section|
depending on the chosen structuring.
Alternatively, one can modify the macro |\thepage| appropriately
and reset the counter |page| at the start of each child file.

%%%%%%%%%%%%%%%%%%%%%%%%%%%%%%%%%%%%%%%%%%%%%%%%%%%%%%%%%%%%%%%%%%%%%%%%%%%%%%%%
\subsection{Conditional Processing}
\label{sec:conditional}

The package provides a mechanism to compile different versions
of a document. To customise the versions further some conditional processing
can come in handy to distinguish which version is being compiled.
The package provides two macros to describe the compilation context:

%%%%%%%%%%%%%%%%%%%%%%%%%%%%%%%%%%%%%%%%
\DescribeMacro{\ifchilddoc}
The conditional |\ifchilddoc| distinguishes between the compilation of
child documents and the main document:
%
\begin{center}
|\ifchilddoc |\textit{child-code}| |[|\||else |\textit{main-code}]| \||fi|
\end{center}

%%%%%%%%%%%%%%%%%%%%%%%%%%%%%%%%%%%%%%%%
\DescribeMacro{\childdocname}
\DescribeMacro{\childdocjob}
The macro |\childdocname| contains the filename (without extension)
of the main or child file being processed.
Note that |\childdocjob| will always contain the name of the main file.

%%%%%%%%%%%%%%%%%%%%%%%%%%%%%%%%%%%%%%%%
\paragraph{Title Page.}

Conditional processing can be used to include a title or banner page
in the main document when proper precautions are taken.
Importantly, the code in the main file should ensure that the page counter
(as well as other status parameters which are stored in the |.aux| files)
takes the same value after the conditional processing.
Otherwise the page numbers may take divergent values
depending on which part is compiled.

For example, a title page could be declared by:
%
\begin{center}
\begin{tabular}{l}
|\ifchilddoc\||else|\\
|\addtocounter{page}{-1}|\\
\textit{code for title page}\\
|\newpage|\\
|\||fi|
\end{tabular}
\end{center}
%
A banner page for the child documents can be generated by:
%
\begin{center}
\begin{tabular}{l}
|\ifchilddoc|\\
|\addtocounter{page}{-1}|\\
\textit{code for banner page}\\
|\newpage|\\
|\||fi|
\end{tabular}
\end{center}
%
Here one could write a message such as:
\begin{center}
|This is the part \childdocname{} of \childdocjob{}.|
\end{center}

%%%%%%%%%%%%%%%%%%%%%%%%%%%%%%%%%%%%%%%%%%%%%%%%%%%%%%%%%%%%%%%%%%%%%%%%%%%%%%%%
\subsection{Flags}
\label{sec:flags}

The package makes it easy to generate different versions
of the main or child documents.
To this end compilation flags can be defined
and assigned different default values.
They will be particularly useful in conjunction
with the forwarding mechanism described in \secref{sec:forward}.

For example, it may be useful to have a flag |\version|
which can be set to |draft| or |final|.
The document source will contain some conditional code
depending on the value of |\version|.
Suppose further, the flag should default to |final| for the main file
and to |draft| for child files
which is a natural assignment for editing the document.
This is achieved by placing the following code
in the preamble of the main document
(below the |\childdocmain| directive):
%
\begin{center}
\begin{tabular}{l}
|\ifchilddoc|\\
|\providecommand{\version}{draft}|\\
|\||else|\\
|\providecommand{\version}{final}|\\
|\||fi|
\end{tabular}
\end{center}
%
The definition by |\providecommand| makes sure
that previous definitions are not overwritten.
Further statements |\providecommand{\version}{...}|
can thus be added before the above code to override it.

For the main file, one might add a line
(between |\childdocmain| and the above block)
%
\begin{center}
|%\ifchilddoc\||else\providecommand{\version}{draft}\||fi|
\end{center}
%
which can be uncommented to produce a draft version.
Likewise one can add a line to the very top of a child file
(above the |\childdocof{|\textit{main}|}| directive)
%
\begin{center}
|%\providecommand{\version}{final}|
\end{center}
%
which can be uncommented to produce the final version of this child document.

%%%%%%%%%%%%%%%%%%%%%%%%%%%%%%%%%%%%%%%%%%%%%%%%%%%%%%%%%%%%%%%%%%%%%%%%%%%%%%%%
\subsection{Forwarding}
\label{sec:forward}

Different versions of the main or child documents
using compilation flags as described in \secref{sec:flags}
can be (permanently) stored in different files
for convenient compilation, viewing and distribution.
To this end, the package defines a command
to pass on compilation to a different file:

%%%%%%%%%%%%%%%%%%%%%%%%%%%%%%%%%%%%%%%%
\DescribeMacro{\childdocforward}
The command |\childdocforward| redirects processing to
another source file:
%
\begin{center}
\begin{tabular}{l}
|\input{childdoc.def}|\\
|\childdocforward[|\textit{main}|]{|\textit{dest}|}|\\
\end{tabular}
\end{center}
%
The argument \textit{dest} is the destination file
(without extension).
It should be the main file or one of the child files.
Note that further \textsf{childdoc} directives
such as |\childdocof| and |\childdocforward|
in the indicated file will be processed in this form.
The optional argument \textit{main}
passes on directly to the main file \textit{main}
while pretending to compile the child \textit{dest}.
This form behaves as if \textit{dest}
issues |\childdocof{|\textit{main}|}| right away,
and no further \textsf{childdoc} directives will be processed.

%%%%%%%%%%%%%%%%%%%%%%%%%%%%%%%%%%%%%%%%
\DescribeMacro{\...prefix}
In the alternative form |\childdocforwardprefix|,
%
\begin{center}
\begin{tabular}{l}
|\input{childdoc.def}|\\
|\childdocforwardprefix[|\textit{main}|]{|\textit{prefix}|}{|\textit{dest}|}|
\end{tabular}
\end{center}
%
the destination file is determined by a pattern
depending on the current file:
To make this work, the current file must be called
`{\textit{prefix}\hspace{0.2em}\textit{suffix}}'
with \textit{prefix} matching precisely the argument.
Processing is then passed on to the file
`{\textit{dest}\hspace{0.2em}\textit{suffix}}'.
Surely, the same effect is achieved by
directly specifying the
argument `{\textit{dest}\hspace{0.2em}\textit{suffix}}'
in the first form.
However, that requires to set up a different file
for each child. With the alternative form of the command
all these files can have exactly the same content
which simplifies setting them up and maintaining them.

For example, the following file |draft.tex|
with a compilation flag |\version| as described in \secref{sec:flags}
compiles the main document as a draft:
%
\begin{center}
\begin{tabular}{l}
|\def\version{draft}|\\
|\input{childdoc.def}|\\
|\childdocforward{|\textit{main}|}|
\end{tabular}
\end{center}
%
Likewise, the following files |final|\textit{nn}|.tex|
compile the final version of the child document
|child|\textit{nn}|.tex|:
%
\begin{center}
\begin{tabular}{l}
|\def\version{final}|\\
|\input{childdoc.def}|\\
|\childdocforwardprefix{final}{child}|
\end{tabular}
\end{center}
%

Note that when several versions of a main file and/or of each child file
are to be generated, it may be convenient to set up a |Makefile| or
shell script to automatise the process.

%%%%%%%%%%%%%%%%%%%%%%%%%%%%%%%%%%%%%%%%%%%%%%%%%%%%%%%%%%%%%%%%%%%%%%%%%%%%%%%%
\subsection{Command Line Processing}
\label{sec:commandline}

The effect of redirection files can also be achieved by invoking
the \LaTeX{} compiler with a more elaborate command line.
Most conveniently this should be done as part
of a shell script or a |Makefile|.

When using \textsf{childdoc} in the main file, the following
command lines effectively perform a redirection
(note that depending on the shell being used,
backslashes may have to be doubled: `|\|' $\to$ `|\\|'):
%
\begin{center}
|... -jobname "|\textit{target}|" |\\|"|[\textit{flags}]%
|\input{childdoc.def}\childdocforward[|\textit{main}|]{|\textit{dest}|}"|
\end{center}
%
Here \textit{target} is the name of the output file,
\textit{main} is the name of the main file
and \textit{dest} is the name of the main or child file to be processed
(all filenames without extensions).
The optional argument \textit{main} can be omitted
if \textit{main} matches \textit{dest}.
Optionally, compilation \textit{flags} can be defined via |\def| commands.
This command line makes the \TeX{} engine believe
it is compiling the file \textit{target}
whose content is specified as the latter parameter.
The provided code then forwards the processing to
\textit{main} or \textit{dest} as described in \secref{sec:forward}.

%%%%%%%%%%%%%%%%%%%%%%%%%%%%%%%%%%%%%%%%%%%%%%%%%%%%%%%%%%%%%%%%%%%%%%%%%%%%%%%%
\subsection{Include by Input}
\label{sec:input}

Including child documents by |\include| has some restrictions by design.
Most notably, the content of a child document always occupies
its own set of pages; pages cannot be shared between child documents.
Usually, this behaviour makes perfect sense
because each child document contain an essential part of the document.
However, in some situations it may be desirable to compose
a document from a collection of parts
without having mandatory page breaks between then.
For this case, the package
provides a mechanism to include parts
by |\input| which can also be processed individually.
However, by construction this mechanism
requires manual handling of the content to be output.

%%%%%%%%%%%%%%%%%%%%%%%%%%%%%%%%%%%%%%%%
\DescribeMacro{\ifchilddocmanual}
The main file should be prepared as usual, see \secref{sec:include}.
However, the document body must make a distinction
between processing of an individual part and of the main document, e.g.:
%
\begin{center}
\begin{tabular}{l}
|\ifchilddocmanual|\\
|\input{\childdocname}|\\
|\||else|\\
\textit{document body with }|\input{|\textit{part}|}|\\
|\||fi|
\end{tabular}
\end{center}
%
The conditional |\ifchilddocmanual| is true whenever
a part to be included by |\input| is being compiled,
and the name of the part is stored in |\childdocname|.

%%%%%%%%%%%%%%%%%%%%%%%%%%%%%%%%%%%%%%%%
\DescribeMacro{\childdocby}
Each part to be included by |\input| should start with:
%
\begin{center}
\begin{tabular}{l}
|\input{childdoc.def}|\\
|\childdocby{|\textit{main}|}|\\
\end{tabular}
\end{center}
%
The directive |\childdocby| is similar to |\childdocof|
described in \secref{sec:include},
but the subsequent selection of content must be done manually.
To that end, both |\ifchilddoc| and |\ifchilddocmanual|
will be true upon processing of a part,
and the name of the part is stored in |\childdocname|.
Note that |\jobname| will be set to the filename of the current part
so that each part receives an individual |.aux| file
that does not interfere with the |.aux| file(s) of the main document.
This behaviour can be altered by the alternative form
|\childdocby[*]{|\textit{main}|}| (with a non-empty optional argument)
which uses the |.aux| file of the main document
by setting |\jobname| to \textit{main}.

%%%%%%%%%%%%%%%%%%%%%%%%%%%%%%%%%%%%%%%%%%%%%%%%%%%%%%%%%%%%%%%%%%%%%%%%%%%%%%%%
\subsection{Driver Development}
\label{sec:driver}

The \textsf{childdoc} mechanism can also be use for the development
of definition files such as \LaTeX{} styles or classes.
This case differs from the above setup with multiple parts
included by |\include| in that no |\includeonly| should be invoked.
This can be achieved by starting the include file
(before |\ProvidesPackage|) with:
%
\begin{center}
\begin{tabular}{l}
|\input{childdoc.def}|\\
|\childdocforward{|\textit{main}|}|\\
\end{tabular}
\end{center}
%
or alternatively with:
%
\begin{center}
\begin{tabular}{l}
|\input{childdoc.def}|\\
|\childdocby{|\textit{main}|}|\\
\end{tabular}
\end{center}
%
Both forms have slightly different effects as described above.
The main file is prepared as usual, see \secref{sec:include}.

%%%%%%%%%%%%%%%%%%%%%%%%%%%%%%%%%%%%%%%%%%%%%%%%%%%%%%%%%%%%%%%%%%%%%%%%%%%%%%%%
\subsection{Legacy Detection}
\label{sec:detection}

The directive |\childdocmain| in the main file can detect
whether the complete document or merely a child is to be compiled
even without using the directive |\childdocof|.
This method is deprecated because it is less robust
and there is no compelling reason to use it;
it is merely provided for backward compatibility
and it may be removed in future versions.

If the detection mechanism is to be used,
it is mandatory to correctly specify
the filename of the main file as the argument of |\childdocmain|:
%
\begin{center}
\begin{tabular}{l}
|\input{childdoc.def}|\\
|\childdocmain{|\textit{main}|}|\\
\end{tabular}
\end{center}
%
If |\jobname| does not match the argument \textit{main} of |\childdocmain|,
it is assumed that |\jobname| points to the child file to be compiled.
When using |\childdocmain| with the main file specified as argument,
it suffices to start a child file
with just |\input{|\textit{main}|}|
without loading of the package and using |\childdocof|.
If instead all processing is done
with the appropriate \textsf{childdoc} directives,
the argument of \textit{main} of |\childdocmain| can be empty.

An alternative version of the command line processing described
in \secref{sec:commandline} using the detection mechanism reads:
%
\begin{center}
|... -jobname "|\textit{target}|" "|[\textit{flags}]%
[|\def\jobname{|\textit{dest}|}|]|\input{|\textit{main}|}"|
\end{center}

%%%%%%%%%%%%%%%%%%%%%%%%%%%%%%%%%%%%%%%%%%%%%%%%%%%%%%%%%%%%%%%%%%%%%%%%%%%%%%%%
\subsection{Manual Code}
\label{sec:manual}

In case one cannot be certain whether the definitions file |childdoc.def|
is installed on the target \TeX{} distribution
and one prefers not to ship it,
it is conceivable to paste a few relevant commands into the sources.

To that end, drop all statements |\input{childdoc.def}|
and perform the replacements as outlined below.
Instead of |\childdocmain{|\textit{main}|}| add the following code
to the top of the main file:
%
\begin{center}
\begin{tabular}{l}
|\||ifdefined\childdocname\endinput\||fi\newif\ifchilddoc|\\
|\edef\childdocname{\scantokens\expandafter{\jobname\noexpand}}|\\
|\def\childdocmain{|\textit{main}|}\||ifx\childdocmain\childdocname\||else|\\
|\childdoctrue\includeonly{\childdocname}\let\jobname\childdocmain\||fi|\\
\end{tabular}
\end{center}
%
Instead of |\childdocof{|\textit{main}|}| just include the main file
at the top of each child file:
%
\begin{center}
|\input{|\textit{main}|}|
\end{center}
%
A simple redirection |\childdocforward{|\textit{dest}|}| is achieved by:
%
\begin{center}
|\def\jobname{|\textit{dest}|}\input{\jobname}|
\end{center}
%
The redirection with prefix
|\childdocforwardprefix[|\textit{prefix}|]{|\textit{dest}|}|
is accomplished by:
%
\begin{center}
\begin{tabular}{l}
|{\edef\jobname{\scantokens\expandafter{\jobname\noexpand}}|\\
|\def\redirectjob |\textit{prefix}|#1~~~{\gdef\jobname{|\textit{dest}|#1}}|\\
|\expandafter\redirectjob\jobname~~~}\input{\jobname}|
\end{tabular}
\end{center}

In an alternative approach,
child documents can be compiled by a specific command line
without additional code or specific definitions:
%
\begin{center}
|... -jobname "|\textit{target}|" "|[\textit{flags}]%
|\includeonly{|\textit{dest}|}\input{|\textit{main}|}"|
\end{center}
%

%%%%%%%%%%%%%%%%%%%%%%%%%%%%%%%%%%%%%%%%%%%%%%%%%%%%%%%%%%%%%%%%%%%%%%%%%%%%%%%%
%%%%%%%%%%%%%%%%%%%%%%%%%%%%%%%%%%%%%%%%%%%%%%%%%%%%%%%%%%%%%%%%%%%%%%%%%%%%%%%%
\section{Information}

%%%%%%%%%%%%%%%%%%%%%%%%%%%%%%%%%%%%%%%%%%%%%%%%%%%%%%%%%%%%%%%%%%%%%%%%%%%%%%%%
\subsection{Copyright}

Copyright \copyright{} 2017--2018 Niklas Beisert

This work may be distributed and/or modified under the
conditions of the \LaTeX{} Project Public License, either version 1.3
of this license or (at your option) any later version.
The latest version of this license is in
  \url{http://www.latex-project.org/lppl.txt}
and version 1.3 or later is part of all distributions of \LaTeX{}
version 2005/12/01 or later.

This work has the LPPL maintenance status `maintained'.

The Current Maintainer of this work is Niklas Beisert.

This work consists of the files |README.txt|, |childdoc.ins| and |childdoc.dtx|
as well as the derived files |childdoc.def|, |cdocsamp.tex|
with |cdocsch1.tex|, |cdocsch2.tex|, |cdocspt3.tex|, |cdocspt4.tex|,
|cdocsdrf.tex|, |cdocsfn1.tex|, |cdocsfn2.tex|
as well as |childdoc.pdf|.

%%%%%%%%%%%%%%%%%%%%%%%%%%%%%%%%%%%%%%%%%%%%%%%%%%%%%%%%%%%%%%%%%%%%%%%%%%%%%%%%
\subsection{Files and Installation}

The package consists of the files:
%
\begin{center}
\begin{tabular}{ll}
    |README.txt|   & readme file \\
    |childdoc.ins| & installation file \\
    |childdoc.dtx| & source file \\
    |childdoc.def| & definition file \\
    |cdocsamp.tex| & sample main file \\
    |cdocsch1.tex| & sample include file \\
    |cdocsch2.tex| & sample include file \\
    |cdocspt3.tex| & sample part file \\
    |cdocspt4.tex| & sample part file \\
    |cdocsdrf.tex| & sample redirection file \\
    |cdocsfn1.tex| & sample redirection file \\
    |cdocsfn2.tex| & sample redirection file \\
    |childdoc.pdf| & manual
\end{tabular}
\end{center}
%
The distribution consists of the files
|README.txt|, |childdoc.ins| and |childdoc.dtx|.
%
\begin{itemize}
\item
Run (pdf)\LaTeX{} on |childdoc.dtx|
to compile the manual |childdoc.pdf| (this file).
\item
Run \LaTeX{} on |childdoc.ins| to create the definitions file |childdoc.def|
and the sample |cdocsamp.tex| with include files
|cdocsch1.tex|, |cdocsch2.tex|, |cdocspt3.tex|, |cdocspt4.tex|,
|cdocsdrf.tex|, |cdocsfn1.tex|, |cdocsfn2.tex|.
Then copy the file |childdoc.def| to an appropriate directory of your \LaTeX{}
distribution, e.g.\ \textit{texmf-root}|/tex/latex/childdoc|.
\end{itemize}

%%%%%%%%%%%%%%%%%%%%%%%%%%%%%%%%%%%%%%%%%%%%%%%%%%%%%%%%%%%%%%%%%%%%%%%%%%%%%%%%
\subsection{Related CTAN Packages}

There are several other packages which offer a similar functionality:
%
\begin{itemize}
\item
The packages
\href{http://ctan.org/pkg/docmute}{\textsf{docmute}},
\href{http://ctan.org/pkg/includex}{\textsf{includex}} and
\href{http://ctan.org/pkg/standalone}{\textsf{standalone}}
provide commands to include only the document body of
a child file thus allowing both files to be compiled individually.
\item
The packages \href{http://ctan.org/pkg/subdocs}{\textsf{subdocs}}
and \href{http://ctan.org/pkg/subfiles}{\textsf{subfiles}}
provide structures in which the main and child documents can be
encapsulated and allowing them to be compiled individually.
The inclusion mechanism is different from the conventional |\include|.
\item
The package \href{http://ctan.org/pkg/combine}{\textsf{combine}}
is an elaborate solution to combine several documents into one.
\end{itemize}
%
See also the CTAN topic \href{http://ctan.org/topic/subdocs}{\textsf{subdocs}}
for further related packages.
The present package differs from the above solutions in that
a document structure constructed with the conventional |\include| mechanism
just needs two extra commands at the top of every file
such that all constituent files can be compiled individually.

%%%%%%%%%%%%%%%%%%%%%%%%%%%%%%%%%%%%%%%%%%%%%%%%%%%%%%%%%%%%%%%%%%%%%%%%%%%%%%%%
%\subsection{Feature Suggestions}
%
%The following is a list of features which may be useful for future
%versions of this package:
%%
%\begin{itemize}
%\item
%\ldots
%\end{itemize}

%%%%%%%%%%%%%%%%%%%%%%%%%%%%%%%%%%%%%%%%%%%%%%%%%%%%%%%%%%%%%%%%%%%%%%%%%%%%%%%%
\subsection{Revision History}

%%%%%%%%%%%%%%%%%%%%%%%%%%%%%%%%%%%%%%%%
\paragraph{v2.0:} 2018/12/30

\begin{itemize}
\item
immediate forward processing
\item
added |\childdocby| mechanism
\item
manual restructured
\end{itemize}

%%%%%%%%%%%%%%%%%%%%%%%%%%%%%%%%%%%%%%%%
\paragraph{v1.6:} 2018/01/17

\begin{itemize}
\item
application for development of include files
\item
corrections to manual
\end{itemize}

%%%%%%%%%%%%%%%%%%%%%%%%%%%%%%%%%%%%%%%%
\paragraph{v1.5:} 2017/05/21

\begin{itemize}
\item
more complete structuring introduced
\item
|\childdocof| introduced
\item
|\childdoc| renamed to |\childdocmain|
\item
|\childredirect| renamed to |\childdocforward| and |\childdocforwardprefix|
and functionality expanded
\end{itemize}

%%%%%%%%%%%%%%%%%%%%%%%%%%%%%%%%%%%%%%%%
\paragraph{v1.0:} 2017/04/27

\begin{itemize}
\item
manual and install package
\item
first version published on CTAN
\end{itemize}

%%%%%%%%%%%%%%%%%%%%%%%%%%%%%%%%%%%%%%%%
\paragraph{v0.6:} 2017/04/26

\begin{itemize}
\item
redirection mechanism added
\end{itemize}

%%%%%%%%%%%%%%%%%%%%%%%%%%%%%%%%%%%%%%%%
\paragraph{v0.5:} 2017/04/26

\begin{itemize}
\item
functionality in definition file
\end{itemize}


%%%%%%%%%%%%%%%%%%%%%%%%%%%%%%%%%%%%%%%%%%%%%%%%%%%%%%%%%%%%%%%%%%%%%%%%%%%%%%%%
%%%%%%%%%%%%%%%%%%%%%%%%%%%%%%%%%%%%%%%%%%%%%%%%%%%%%%%%%%%%%%%%%%%%%%%%%%%%%%%%
%%%%%%%%%%%%%%%%%%%%%%%%%%%%%%%%%%%%%%%%%%%%%%%%%%%%%%%%%%%%%%%%%%%%%%%%%%%%%%%%
\appendix

\settowidth\MacroIndent{\rmfamily\scriptsize 000\ }

 \DocInput{childdoc.dtx}

\end{document}
%</driver>
% \fi
%
% %%%%%%%%%%%%%%%%%%%%%%%%%%%%%%%%%%%%%%%%%%%%%%%%%%%%%%%%%%%%%%%%%%%%%%%%%%%%%%
% %%%%%%%%%%%%%%%%%%%%%%%%%%%%%%%%%%%%%%%%%%%%%%%%%%%%%%%%%%%%%%%%%%%%%%%%%%%%%%
% \section{Sample}
%\iffalse
%<*samplemain>
%\fi
%
% The following presents a sample document
% with two chapters, two parts, a title page,
% a compile flag as well as three forwarding files to set the flag.
% It consists of eight |.tex| files:
% \begin{center}
% \begin{tabular}{ll}
% |cdocsamp.tex|&main file\\
% |cdocsch1.tex|&include file for chapter 1\\
% |cdocsch2.tex|&include file for chapter 2\\
% |cdocspt3.tex|&include file for part 3\\
% |cdocspt4.tex|&include file for part 4\\
% |cdocsdrf.tex|&forwarding file for main file in draft mode\\
% |cdocsfi1.tex|&forwarding file for final version of chapter 1\\
% |cdocsfi2.tex|&forwarding file for final version of chapter 2\\
% \end{tabular}
% \end{center}
% Each of the eight files can be compiled directly by the \LaTeX{} compiler.
%
% %%%%%%%%%%%%%%%%%%%%%%%%%%%%%%%%%%%%%%
% \paragraph{Main File.}
%
% The main file is called |cdocsamp.tex|.
%
% Load the \textsf{childdoc} definitions and
% declare the filename for the main document:
%    \begin{macrocode}
\input{childdoc.def}
\childdocmain{}
%    \end{macrocode}

% Optional override for |\version| flag:
%    \begin{macrocode}
%%\ifchilddoc\else\providecommand{\version}{draft}\fi
%    \end{macrocode}

% Define the default values for the |\version| flag
% (|final| for the main file and |draft| for childs):
%    \begin{macrocode}
\ifchilddoc
\providecommand{\version}{draft}
\else
\providecommand{\version}{final}
\fi
%    \end{macrocode}

% Load the standard document class:
%    \begin{macrocode}
\documentclass[12pt]{article}
%    \end{macrocode}

% Start the document body:
%    \begin{macrocode}
\begin{document}
%    \end{macrocode}

% Declare a title page.
% Print title, part of document being processed and version flag:
%    \begin{macrocode}
\addtocounter{page}{-1}
\begin{center}
{\LARGE\bfseries{}childdoc example\par}
\vspace{1cm}
\ifchilddoc
\ifchilddocmanual part\else chapter\fi:
`\childdocname' of `\childdocjob'\par
\else
main document: `\childdocjob'\par
\fi
version: \version\par
\end{center}
\newpage
%    \end{macrocode}

% Manually include selected file,
% otherwise process as usual:
%    \begin{macrocode}
\ifchilddocmanual
\section*{part `\childdocname'}
\input{\childdocname}
\else
%    \end{macrocode}

% Include the two chapters:
%    \begin{macrocode}
\include{cdocsch1}
\include{cdocsch2}
%    \end{macrocode}

% Include the two parts unless only chapters should be displayed:
%    \begin{macrocode}
\ifchilddoc\else
\section{part three}
\input{cdocspt3}
\section{part four}
\input{cdocspt4}
\fi
%    \end{macrocode}

% Process as usual until here:
%    \begin{macrocode}
\fi
%    \end{macrocode}

% End of document body:
%    \begin{macrocode}
\end{document}
%    \end{macrocode}
%\iffalse
%</samplemain>
%\fi
%
% %%%%%%%%%%%%%%%%%%%%%%%%%%%%%%%%%%%%%%
% \paragraph{Chapter Include Files.}
%
% The include files are called |cdocsch1.tex| and |cdocsch2.tex|.
%
%\iffalse
%<*samplechap1|samplechap2>
%\fi

% Optional override for |\version| flag:
%    \begin{macrocode}
%%\providecommand{\version}{final}
%    \end{macrocode}

% Include the main document:
%    \begin{macrocode}
\input{childdoc.def}
\childdocof{cdocsamp}
%    \end{macrocode}

%\iffalse
%</samplechap1|samplechap2>
%\fi
%
%\iffalse
%<*samplechap1>
%\fi
% Some text for chapter 1:
%    \begin{macrocode}
\section{one}
some text in chapter one
%    \end{macrocode}

%\iffalse
%</samplechap1>
%\fi
% Some text for chapter 2:
%\iffalse
%<*samplechap2>
%\fi
%    \begin{macrocode}
\section{two}
more text in chapter two
%    \end{macrocode}

%\iffalse
%</samplechap2>
%\fi
%
% %%%%%%%%%%%%%%%%%%%%%%%%%%%%%%%%%%%%%%
% \paragraph{Part Include Files.}
%
% The include files are called |cdocspt3.tex| and |cdocspt4.tex|.
%
%\iffalse
%<*samplepart3|samplepart4>
%\fi

% Optional override for |\version| flag:
%    \begin{macrocode}
%%\providecommand{\version}{final}
%    \end{macrocode}

% Include the main document:
%    \begin{macrocode}
\input{childdoc.def}
\childdocby{cdocsamp}
%    \end{macrocode}

%\iffalse
%</samplepart3|samplepart4>
%\fi
%
%\iffalse
%<*samplepart3>
%\fi
% Some text for part 3:
%    \begin{macrocode}
some text in part three
%    \end{macrocode}

%\iffalse
%</samplepart3>
%\fi
% Some text for part 4:
%\iffalse
%<*samplepart4>
%\fi
%    \begin{macrocode}
more text in part four
%    \end{macrocode}

%\iffalse
%</samplepart4>
%\fi
%
% %%%%%%%%%%%%%%%%%%%%%%%%%%%%%%%%%%%%%%
% \paragraph{Forwarding for a Complete Draft.}
%
% The following forwarding file |cdocsdrf.tex|
% compiles the main document in draft mode:
%\iffalse
%<*sampledraft>
%\fi
%    \begin{macrocode}
\def\version{draft}
\input{childdoc.def}
\childdocforward{cdocsamp}
%    \end{macrocode}

%\iffalse
%</sampledraft>
%\fi
%
% %%%%%%%%%%%%%%%%%%%%%%%%%%%%%%%%%%%%%%
% \paragraph{Forwarding for Final Version of the Chapters.}
%
% The following forwarding files |cdocsfn1.tex| and |cdocsfn2.tex|
% (with identical content)
% compile the final versions of the child documents
% |cdocsch1.tex| and |cdocsch2.tex|, respectively:
%\iffalse
%<*samplefinal>
%\fi
%    \begin{macrocode}
\def\version{final}
\input{childdoc.def}
\childdocforwardprefix[cdocsamp]{cdocsfn}{cdocsch}
%    \end{macrocode}

%\iffalse
%</samplefinal>
%\fi
%
% %%%%%%%%%%%%%%%%%%%%%%%%%%%%%%%%%%%%%%
% \paragraph{Command Line Processing.}
%
% The following three command lines generate the output files
% |cdocscld|, |cdocscl1| and |cdocscl2|
% which should be identical to
% |cdocsdrf|, |cdocsch1| and |cdocsfn2|, respectively:
% \begin{center}
% \begin{tabular}{l}
% |latex -jobname cdocscld \|\\
% |  "\def\version{draft}\input{childdoc.def}\childdocforward{cdocsamp}"|\\
% |latex -jobname cdocscl1 \|\\
% |  "\input{childdoc.def}\childdocforward[cdocsamp]{cdocsch1}"|\\
% |latex -jobname cdocscl2 \|\\
% |  "\def\version{final}\input{childdoc.def}\childdocforward{cdocsch2}"|
% \end{tabular}
% \end{center}
% Note that the trailing backslash on each first line
% merely continues the input to the second line
% (for convenient cut ant paste).
% Furthermore, the command |latex| can be replaced by any
% of its alternative versions such as |pdflatex|.
%
% %%%%%%%%%%%%%%%%%%%%%%%%%%%%%%%%%%%%%%%%%%%%%%%%%%%%%%%%%%%%%%%%%%%%%%%%%%%%%%
% %%%%%%%%%%%%%%%%%%%%%%%%%%%%%%%%%%%%%%%%%%%%%%%%%%%%%%%%%%%%%%%%%%%%%%%%%%%%%%
% \section{Implementation}
%\iffalse
%<*package>
%\fi
%
% This section describes the definitions file |childdoc.def|.

% The definitions cannot be loaded using |\usepackage| or |\RequirePackage|
% which has a mechanism to prevent loading a style file more than once.
% When loading the definitions by means of |\input|
% multiple instances have to be prevented manually:
%\iffalse
%This code needs to be before the `\ProvidesFile' directive
%which is defined at the beginning of this file.
%Therefore it is also placed there and commented out here.
%</package>
%<*discard>
%\fi
%    \begin{macrocode}
\ifdefined\childdocmain\endinput\fi
%    \end{macrocode}
%\iffalse
%</discard>
%<*package>
%\fi
%
% \macro{\ifchilddoc}
% \macro{\ifchilddocmanual}
% The conditional |\ifchilddoc| tells whether a
% child (true) or main (false) document is being compiled.
% The conditional |\ifchilddocmanual| tells whether
% the |\includeonly| mechanism is used (false) or
% the selection of child files must be performed manually (true).
% The definitions initialise to false:
%    \begin{macrocode}
\newif\ifchilddoc
\newif\ifchilddocmanual
%    \end{macrocode}

% \macro{\childdocname}
% \macro{\childdocjob}
% The macro |\childdocname| stores the name of the main document
% to be compiled. The macro |\childdocjob| stores the name of
% the document on which the \LaTeX{} compiler was originally invoked.
% The content of |\jobname| cannot be compared
% to filenames specified in the source due to different catcodes.
% The following code rescans |\jobname|, stores the result
% in |\childdocname| and saves a copy in |\childdocjob|:
%    \begin{macrocode}
\edef\childdocname{\scantokens\expandafter{\jobname\noexpand}}
\let\childdocjob\childdocname
%    \end{macrocode}

% \macro{\childdocdisable}
% The macro |\childdocdisable| prevents the main file
% from being processed more than once.
% At this stage, the main document command |\childdocmain|
% is assumed to be called once again where it should do nothing.
% Any subsequent call to it should prevent
% a secondary processing of the main document
% It overwrites the forwarding commands
% |\childdocof| and |\childdocforward|
% with empty macros to prevent further inclusions of the main document:
%    \begin{macrocode}
\newcommand{\childdocdisable}
{
  \renewcommand{\childdocmain}[1]{\renewcommand{\childdocmain}[1]{\endinput}}
  \renewcommand{\childdocof}[1]{}
  \renewcommand{\childdocby}[2][]{}
  \renewcommand{\childdocforward}[2][]{}
  \renewcommand{\childdocdisable}{}
}
%    \end{macrocode}

% \macro{\childdocmain}
% The macro |\childdocmain| is to be called at the top of the main file
% with nothing or the main filename (without extension) as argument.
% First, it breaks loops.
% If the argument is not empty and does not match |\childdocname|
% (which is set by the first inclusion of |childdoc.def|),
% |\ifchilddoc| is set to true, |\includeonly| is applied to the child file
% and |\jobname| is set to the main file
% (for proper handling of |.aux| files):
%    \begin{macrocode}
\newcommand{\childdocmain}[1]
{
  \childdocdisable\childdocmain{}
  \if?#1?\else
    \begingroup
      \def\childdoctmp{#1}
      \ifx\childdoctmp\childdocname
        \def\childdoctmp{}
      \else
        \def\childdoctmp
        {
          \childdoctrue
          \includeonly{\childdocname}
          \def\childdocjob{#1}
          \def\jobname{#1}
        }
      \fi
      \expandafter
    \endgroup
    \childdoctmp
  \fi
}
%    \end{macrocode}

% \macro{\childdocof}
% The command |\childdocof| redirects
% compilation to the main file |#1|.
%    \begin{macrocode}
\newcommand{\childdocof}[1]
{
  \childdocdisable
  \childdoctrue
  \includeonly{\childdocname}
  \def\jobname{#1}
  \def\childdocjob{#1}
  \input{#1}
}
%    \end{macrocode}

% \macro{\childdocby}
% The command |\childdocby| ....
%    \begin{macrocode}
\newcommand{\childdocby}[2][]
{
  \childdocdisable
  \childdoctrue
  \childdocmanualtrue
  \if?#1?\else
    \def\jobname{#2}
  \fi
  \def\childdocjob{#2}
  \input{#2}
  \endinput
}
%    \end{macrocode}

% \macro{\childdocforward}
% The command |\childdocforward| redirects
% compilation to the main file or
% (if the optional argument is given) a child file.
% Parameters are set as if the main file
% or a child file starting with |\childdocof| was compiled.
% Then compilation is handed over to the main file:
%    \begin{macrocode}
\newcommand{\childdocforward}[2][]
{
  \begingroup
    \if?#1?
      \def\childdoctmp
      {
        \def\childdocname{#2}
        \def\childdocjob{#2}
        \def\jobname{#2}
        \input{#2}
        \endinput
      }
    \else
      \def\childdoctmp
      {
        \childdocdisable
        \def\childdocname{#2}
        \childdoctrue
        \includeonly{#2}
        \def\childdocjob{#1}
        \def\jobname{#1}
        \input{#1}
        \endinput
      }
    \fi
    \expandafter
  \endgroup
  \childdoctmp
}
%    \end{macrocode}

% \macro{\childdocforwardprefix}
% The command |\childdocforwardprefix| redirects
% compilation to the main or a child file by means of a pattern.
% The prefix |#1| in the current filename is replaced by |#2|
% and the suffix of the current filename is kept
% (it is assumed that the filename does not contain the substring `|~~~|'
% which is used as a delimiter).
% Compilation is handed over to the new file by |\childdocforward|:
%    \begin{macrocode}
\newcommand{\childdocforwardprefix}[3][]
{
  \begingroup
    \def\childdocextract #2##1~~~{\def\childdoctmp{\childdocforward[#1]{#3##1}}}
    \expandafter\childdocextract\childdocname~~~
    \expandafter
  \endgroup
  \childdoctmp
}
%    \end{macrocode}

% \macro{\childdoc}
% The deprecated macro |\childdoc| is a legacy version of |\childdocmain|:
%    \begin{macrocode}
\newcommand{\childdoc}{\childdocmain}
%    \end{macrocode}

% \macro{\childdocredirect}
% The deprecated macro |\childdocredirect| is a legacy version
% of |\childdocforward| and |\childdocforwardprefix|:
%    \begin{macrocode}
\newcommand{\childdocredirect}[2][]
{
  \begingroup
    \if?#1?
      \def\childdoctmp{\childdocforward{#2}}
    \else
      \def\childdoctmp{\childdocforwardprefix{#1}{#2}}
    \fi
    \expandafter
  \endgroup
  \childdoctmp
}
%    \end{macrocode}

%\iffalse
%</package>
%\fi
%
\endinput
|\\
|\childdocmain{|\textit{main}|}|\\
\end{tabular}
\end{center}
%
If |\jobname| does not match the argument \textit{main} of |\childdocmain|,
it is assumed that |\jobname| points to the child file to be compiled.
When using |\childdocmain| with the main file specified as argument,
it suffices to start a child file
with just |\input{|\textit{main}|}|
without loading of the package and using |\childdocof|.
If instead all processing is done
with the appropriate \textsf{childdoc} directives,
the argument of \textit{main} of |\childdocmain| can be empty.

An alternative version of the command line processing described
in \secref{sec:commandline} using the detection mechanism reads:
%
\begin{center}
|... -jobname "|\textit{target}|" "|[\textit{flags}]%
[|\def\jobname{|\textit{dest}|}|]|\input{|\textit{main}|}"|
\end{center}

%%%%%%%%%%%%%%%%%%%%%%%%%%%%%%%%%%%%%%%%%%%%%%%%%%%%%%%%%%%%%%%%%%%%%%%%%%%%%%%%
\subsection{Manual Code}
\label{sec:manual}

In case one cannot be certain whether the definitions file |childdoc.def|
is installed on the target \TeX{} distribution
and one prefers not to ship it,
it is conceivable to paste a few relevant commands into the sources.

To that end, drop all statements |% \iffalse
%
% childdoc.dtx Copyright (C) 2017-2018 Niklas Beisert
%
% This work may be distributed and/or modified under the
% conditions of the LaTeX Project Public License, either version 1.3
% of this license or (at your option) any later version.
% The latest version of this license is in
%   http://www.latex-project.org/lppl.txt
% and version 1.3 or later is part of all distributions of LaTeX
% version 2005/12/01 or later.
%
% This work has the LPPL maintenance status `maintained'.
%
% The Current Maintainer of this work is Niklas Beisert.
%
% This work consists of the files childdoc.dtx and childdoc.ins
% and the derived files childdoc.def and cdocsamp.tex with
% cdocsch1.tex, cdocsch2.tex, cdocsdrf.tex, cdocsfn1.tex, cdocsfn2.tex.
%
%<package>\ifdefined\childdocmain\endinput\fi
%<package>\ProvidesFile{childdoc.def}[2018/12/30 v2.0 child document driver]
%<samplemain>\ProvidesFile{cdocsamp.tex}[2018/12/30 v2.0 sample for childdoc]
%<*driver>
%\ProvidesFile{childdoc.drv}[2018/12/30 v2.0 childdoc reference manual file]
\PassOptionsToClass{10pt,a4paper}{article}
\documentclass{ltxdoc}

\usepackage[margin=35mm]{geometry}
\usepackage{hyperref}
\usepackage{hyperxmp}
\usepackage[usenames]{color}

\hypersetup{colorlinks=true}
\hypersetup{pdfstartview=FitH}
\hypersetup{pdfpagemode=UseNone}
\hypersetup{pdfsource={}}
\hypersetup{pdflang={en-UK}}
\hypersetup{pdfcopyright={Copyright 2017-2018 Niklas Beisert.
  This work may be distributed and/or modified under the
  conditions of the LaTeX Project Public License, either version 1.3
  of this license or (at your option) any later version.}}
\hypersetup{pdflicenseurl={http://www.latex-project.org/lppl.txt}}
\hypersetup{pdfcontactaddress={ETH Zurich, ITP, HIT K,
  Wolfgang-Pauli-Strasse 27}}
\hypersetup{pdfcontactpostcode={8093}}
\hypersetup{pdfcontactcity={Zurich}}
\hypersetup{pdfcontactcountry={Switzerland}}
\hypersetup{pdfcontactemail={nbeisert@itp.phys.ethz.ch}}
\hypersetup{pdfcontacturl={http://people.phys.ethz.ch/\xmptilde nbeisert/}}

\newcommand{\secref}[1]{\hyperref[#1]{section \ref*{#1}}}

\parskip1ex
\parindent0pt
\let\olditemize\itemize
\def\itemize{\olditemize\parskip0pt}

\begin{document}

\title{The \textsf{childdoc} Package}
\hypersetup{pdftitle={The childdoc Package}}
\author{Niklas Beisert\\[2ex]
  Institut f\"ur Theoretische Physik\\
  Eidgen\"ossische Technische Hochschule Z\"urich\\
  Wolfgang-Pauli-Strasse 27, 8093 Z\"urich, Switzerland\\[1ex]
  \href{mailto:nbeisert@itp.phys.ethz.ch}
  {\texttt{nbeisert@itp.phys.ethz.ch}}}
\hypersetup{pdfauthor={Niklas Beisert}}
\hypersetup{pdfsubject={Manual for the LaTeX2e Package childdoc}}
\date{30 December 2018, \textsf{v2.0}}
\maketitle

\begin{abstract}\noindent
\textsf{childdoc} is a \LaTeXe{} package
that enables the direct compilation
of document sections included by |\include|
to individual files.
\end{abstract}

\begingroup
\parskip0ex
\tableofcontents
\endgroup

%%%%%%%%%%%%%%%%%%%%%%%%%%%%%%%%%%%%%%%%%%%%%%%%%%%%%%%%%%%%%%%%%%%%%%%%%%%%%%%%
%%%%%%%%%%%%%%%%%%%%%%%%%%%%%%%%%%%%%%%%%%%%%%%%%%%%%%%%%%%%%%%%%%%%%%%%%%%%%%%%
\section{Introduction}

\LaTeX{} provides a mechanism to structure a large document (such as a book)
into a main file and several child files (containing the chapters)
using the |\include| command.
This mechanism is beneficial for documents
which span hundreds of pages in order to
make the source file(s) more manageable.
Moreover, compilation can be restricted to
selected child files by means of the |\includeonly| command.
The latter feature can be used to reduce the compilation time while editing
(this was significantly more useful in the earlier days of \LaTeX{})
or to generate a smaller document which is easier to navigate.
Another application of |\includeonly| is to generate
documents consisting of selected parts of the complete document.

However, there are a few drawbacks of the plain |\include| mechanism:
\begin{itemize}
\item
The child files cannot be compiled on their own,
they can only be compiled via the main file.
A naive editing environment
(such as a text editor with an option
to have the current file processed by \LaTeX)
may require one to switch to the main file before compiling;
attempting to compile the child file produces errors.
\item
The main file must be modified (each time)
to adjust the |\includeonly| command
to the present needs. This easily leaves the main file in a messy state.
\item
The generated document will always carry the filename
of the main document. This is inconvenient if
several child files are to be compiled and
to be kept for distribution.
\end{itemize}

The present package provides a simple interface
to make child files individually compilable by \LaTeX{}.
Compiling a child file then has the same effect as compiling
the main file with an |\includeonly| command
to select the appropriate child.
Moreover the generated document will carry the name of the child
rather than the main file.
This resolves all three above issues.

This feature is meant to make the editing of books,
thesis documents and lecture notes somewhat more convenient.
However, the package can also be used efficiently for
composing a series of documents (such as exercise sheets)
which are typically distributed individually.
It then assists the author in generating the individual documents
(potentially in different versions)
as well as a document containing the collected series.
Another application is in developing style files
or other kinds of included material
where compilation of the style file could redirect
to a sample or test file.

%%%%%%%%%%%%%%%%%%%%%%%%%%%%%%%%%%%%%%%%%%%%%%%%%%%%%%%%%%%%%%%%%%%%%%%%%%%%%%%%
%%%%%%%%%%%%%%%%%%%%%%%%%%%%%%%%%%%%%%%%%%%%%%%%%%%%%%%%%%%%%%%%%%%%%%%%%%%%%%%%
\section{Usage}

First of all, the package \textsf{childdoc} is \emph{not} a standard
\LaTeXe{} |.sty| style file! Therefore it needs to be invoked in
a non-standard way.

%%%%%%%%%%%%%%%%%%%%%%%%%%%%%%%%%%%%%%%%%%%%%%%%%%%%%%%%%%%%%%%%%%%%%%%%%%%%%%%%
\subsection{Included Files}
\label{sec:include}

%%%%%%%%%%%%%%%%%%%%%%%%%%%%%%%%%%%%%%%%
\DescribeMacro{\childdocmain}
To use the package, add the commands
\begin{center}
\begin{tabular}{l}
|\input{childdoc.def}|\\
|\childdocmain{}|\\
\end{tabular}
\end{center}
at the very top of the main \LaTeX{} file,
in particular \emph{before} the |\documentclass| statement!
The argument of |\childdocmain| should be left empty
(but it must be present).

%%%%%%%%%%%%%%%%%%%%%%%%%%%%%%%%%%%%%%%%
\DescribeMacro{\childdocof}
Furthermore, add the commands
\begin{center}
\begin{tabular}{l}
|\input{childdoc.def}|\\
|\childdocof{|\textit{main}|}|\\
\end{tabular}
\end{center}
at the top of every child file \textit{child}
which is included by |\include{|\textit{child}|}|
from within the main file
(or at least for those files to be compiled individually).
The argument \textit{main} must be the filename of the main file.

There are a couple of
considerations in setting up the main and child documents:

%%%%%%%%%%%%%%%%%%%%%%%%%%%%%%%%%%%%%%%%
\paragraph{Restrictions.}

Please note the following restrictions:
\begin{itemize}
\item
|\childdocmain| must be called with one argument \textit{main}
to ensure compatibility with earlier version of the package.
It must either be empty (|\childdocmain{}|)
or precisely match the filename of the main file in which it is specified.
See \secref{sec:detection} for further information.
\item
The filename \textit{main} must be specified without the |.tex| extension.
\item
The filename \textit{main} is case sensitive
(even in case-insensitive file systems)
due to internal string comparison.
\item
The argument \textit{main} should be fully expanded, it cannot be a macro.
\item
Subdirectories and special characters should be avoided in filenames.
\item
The command |\childdocmain{|\textit{main}|}| must be followed by a whitespace.
It should not be followed immediately by another command
or by a comment mark `|%|'.
This is because the \TeX{} parser reads the token immediately following
the argument of |\childdocmain| and puts it
at the beginning of every child section;
however, a white\-space is ignored.
\end{itemize}

%%%%%%%%%%%%%%%%%%%%%%%%%%%%%%%%%%%%%%%%
\paragraph{Content of Main File.}

It is advisable to place all content in the child files included by |\include|.
Any output contained in the main file will appear in all child documents
unless suppressed manually;
it cannot be suppressed automatically by the |\includeonly| directive
and thus should normally be avoided.
A method to include some content in the main file
by means of conditional processing is described in \secref{sec:conditional}.

%%%%%%%%%%%%%%%%%%%%%%%%%%%%%%%%%%%%%%%%
\paragraph{Page Numbering.}

When only a part of the document is compiled,
the appropriate numbering of pages
(as well as other status parameters)
is determined from the |.aux| files.
The latter contain information from previous passes.
However this information needs to propagate through
all intermediate child documents.
Therefore the page numbering in child documents may well
be inconsistent until the complete document is compiled at least once.

A useful (if unconventional) way to always ensure a consistent
page numbering is to restart the numbering in each child document
and denote the pages by `\textit{child}|.|\textit{page}'
where \textit{child} represents the chapter/section number of the child file.
This can be achieved by the command
|\numberwithin{page}{|\textit{child}|}|
of the \textsf{amsmath} package
where \textit{child} can be |chapter| or |section|
depending on the chosen structuring.
Alternatively, one can modify the macro |\thepage| appropriately
and reset the counter |page| at the start of each child file.

%%%%%%%%%%%%%%%%%%%%%%%%%%%%%%%%%%%%%%%%%%%%%%%%%%%%%%%%%%%%%%%%%%%%%%%%%%%%%%%%
\subsection{Conditional Processing}
\label{sec:conditional}

The package provides a mechanism to compile different versions
of a document. To customise the versions further some conditional processing
can come in handy to distinguish which version is being compiled.
The package provides two macros to describe the compilation context:

%%%%%%%%%%%%%%%%%%%%%%%%%%%%%%%%%%%%%%%%
\DescribeMacro{\ifchilddoc}
The conditional |\ifchilddoc| distinguishes between the compilation of
child documents and the main document:
%
\begin{center}
|\ifchilddoc |\textit{child-code}| |[|\||else |\textit{main-code}]| \||fi|
\end{center}

%%%%%%%%%%%%%%%%%%%%%%%%%%%%%%%%%%%%%%%%
\DescribeMacro{\childdocname}
\DescribeMacro{\childdocjob}
The macro |\childdocname| contains the filename (without extension)
of the main or child file being processed.
Note that |\childdocjob| will always contain the name of the main file.

%%%%%%%%%%%%%%%%%%%%%%%%%%%%%%%%%%%%%%%%
\paragraph{Title Page.}

Conditional processing can be used to include a title or banner page
in the main document when proper precautions are taken.
Importantly, the code in the main file should ensure that the page counter
(as well as other status parameters which are stored in the |.aux| files)
takes the same value after the conditional processing.
Otherwise the page numbers may take divergent values
depending on which part is compiled.

For example, a title page could be declared by:
%
\begin{center}
\begin{tabular}{l}
|\ifchilddoc\||else|\\
|\addtocounter{page}{-1}|\\
\textit{code for title page}\\
|\newpage|\\
|\||fi|
\end{tabular}
\end{center}
%
A banner page for the child documents can be generated by:
%
\begin{center}
\begin{tabular}{l}
|\ifchilddoc|\\
|\addtocounter{page}{-1}|\\
\textit{code for banner page}\\
|\newpage|\\
|\||fi|
\end{tabular}
\end{center}
%
Here one could write a message such as:
\begin{center}
|This is the part \childdocname{} of \childdocjob{}.|
\end{center}

%%%%%%%%%%%%%%%%%%%%%%%%%%%%%%%%%%%%%%%%%%%%%%%%%%%%%%%%%%%%%%%%%%%%%%%%%%%%%%%%
\subsection{Flags}
\label{sec:flags}

The package makes it easy to generate different versions
of the main or child documents.
To this end compilation flags can be defined
and assigned different default values.
They will be particularly useful in conjunction
with the forwarding mechanism described in \secref{sec:forward}.

For example, it may be useful to have a flag |\version|
which can be set to |draft| or |final|.
The document source will contain some conditional code
depending on the value of |\version|.
Suppose further, the flag should default to |final| for the main file
and to |draft| for child files
which is a natural assignment for editing the document.
This is achieved by placing the following code
in the preamble of the main document
(below the |\childdocmain| directive):
%
\begin{center}
\begin{tabular}{l}
|\ifchilddoc|\\
|\providecommand{\version}{draft}|\\
|\||else|\\
|\providecommand{\version}{final}|\\
|\||fi|
\end{tabular}
\end{center}
%
The definition by |\providecommand| makes sure
that previous definitions are not overwritten.
Further statements |\providecommand{\version}{...}|
can thus be added before the above code to override it.

For the main file, one might add a line
(between |\childdocmain| and the above block)
%
\begin{center}
|%\ifchilddoc\||else\providecommand{\version}{draft}\||fi|
\end{center}
%
which can be uncommented to produce a draft version.
Likewise one can add a line to the very top of a child file
(above the |\childdocof{|\textit{main}|}| directive)
%
\begin{center}
|%\providecommand{\version}{final}|
\end{center}
%
which can be uncommented to produce the final version of this child document.

%%%%%%%%%%%%%%%%%%%%%%%%%%%%%%%%%%%%%%%%%%%%%%%%%%%%%%%%%%%%%%%%%%%%%%%%%%%%%%%%
\subsection{Forwarding}
\label{sec:forward}

Different versions of the main or child documents
using compilation flags as described in \secref{sec:flags}
can be (permanently) stored in different files
for convenient compilation, viewing and distribution.
To this end, the package defines a command
to pass on compilation to a different file:

%%%%%%%%%%%%%%%%%%%%%%%%%%%%%%%%%%%%%%%%
\DescribeMacro{\childdocforward}
The command |\childdocforward| redirects processing to
another source file:
%
\begin{center}
\begin{tabular}{l}
|\input{childdoc.def}|\\
|\childdocforward[|\textit{main}|]{|\textit{dest}|}|\\
\end{tabular}
\end{center}
%
The argument \textit{dest} is the destination file
(without extension).
It should be the main file or one of the child files.
Note that further \textsf{childdoc} directives
such as |\childdocof| and |\childdocforward|
in the indicated file will be processed in this form.
The optional argument \textit{main}
passes on directly to the main file \textit{main}
while pretending to compile the child \textit{dest}.
This form behaves as if \textit{dest}
issues |\childdocof{|\textit{main}|}| right away,
and no further \textsf{childdoc} directives will be processed.

%%%%%%%%%%%%%%%%%%%%%%%%%%%%%%%%%%%%%%%%
\DescribeMacro{\...prefix}
In the alternative form |\childdocforwardprefix|,
%
\begin{center}
\begin{tabular}{l}
|\input{childdoc.def}|\\
|\childdocforwardprefix[|\textit{main}|]{|\textit{prefix}|}{|\textit{dest}|}|
\end{tabular}
\end{center}
%
the destination file is determined by a pattern
depending on the current file:
To make this work, the current file must be called
`{\textit{prefix}\hspace{0.2em}\textit{suffix}}'
with \textit{prefix} matching precisely the argument.
Processing is then passed on to the file
`{\textit{dest}\hspace{0.2em}\textit{suffix}}'.
Surely, the same effect is achieved by
directly specifying the
argument `{\textit{dest}\hspace{0.2em}\textit{suffix}}'
in the first form.
However, that requires to set up a different file
for each child. With the alternative form of the command
all these files can have exactly the same content
which simplifies setting them up and maintaining them.

For example, the following file |draft.tex|
with a compilation flag |\version| as described in \secref{sec:flags}
compiles the main document as a draft:
%
\begin{center}
\begin{tabular}{l}
|\def\version{draft}|\\
|\input{childdoc.def}|\\
|\childdocforward{|\textit{main}|}|
\end{tabular}
\end{center}
%
Likewise, the following files |final|\textit{nn}|.tex|
compile the final version of the child document
|child|\textit{nn}|.tex|:
%
\begin{center}
\begin{tabular}{l}
|\def\version{final}|\\
|\input{childdoc.def}|\\
|\childdocforwardprefix{final}{child}|
\end{tabular}
\end{center}
%

Note that when several versions of a main file and/or of each child file
are to be generated, it may be convenient to set up a |Makefile| or
shell script to automatise the process.

%%%%%%%%%%%%%%%%%%%%%%%%%%%%%%%%%%%%%%%%%%%%%%%%%%%%%%%%%%%%%%%%%%%%%%%%%%%%%%%%
\subsection{Command Line Processing}
\label{sec:commandline}

The effect of redirection files can also be achieved by invoking
the \LaTeX{} compiler with a more elaborate command line.
Most conveniently this should be done as part
of a shell script or a |Makefile|.

When using \textsf{childdoc} in the main file, the following
command lines effectively perform a redirection
(note that depending on the shell being used,
backslashes may have to be doubled: `|\|' $\to$ `|\\|'):
%
\begin{center}
|... -jobname "|\textit{target}|" |\\|"|[\textit{flags}]%
|\input{childdoc.def}\childdocforward[|\textit{main}|]{|\textit{dest}|}"|
\end{center}
%
Here \textit{target} is the name of the output file,
\textit{main} is the name of the main file
and \textit{dest} is the name of the main or child file to be processed
(all filenames without extensions).
The optional argument \textit{main} can be omitted
if \textit{main} matches \textit{dest}.
Optionally, compilation \textit{flags} can be defined via |\def| commands.
This command line makes the \TeX{} engine believe
it is compiling the file \textit{target}
whose content is specified as the latter parameter.
The provided code then forwards the processing to
\textit{main} or \textit{dest} as described in \secref{sec:forward}.

%%%%%%%%%%%%%%%%%%%%%%%%%%%%%%%%%%%%%%%%%%%%%%%%%%%%%%%%%%%%%%%%%%%%%%%%%%%%%%%%
\subsection{Include by Input}
\label{sec:input}

Including child documents by |\include| has some restrictions by design.
Most notably, the content of a child document always occupies
its own set of pages; pages cannot be shared between child documents.
Usually, this behaviour makes perfect sense
because each child document contain an essential part of the document.
However, in some situations it may be desirable to compose
a document from a collection of parts
without having mandatory page breaks between then.
For this case, the package
provides a mechanism to include parts
by |\input| which can also be processed individually.
However, by construction this mechanism
requires manual handling of the content to be output.

%%%%%%%%%%%%%%%%%%%%%%%%%%%%%%%%%%%%%%%%
\DescribeMacro{\ifchilddocmanual}
The main file should be prepared as usual, see \secref{sec:include}.
However, the document body must make a distinction
between processing of an individual part and of the main document, e.g.:
%
\begin{center}
\begin{tabular}{l}
|\ifchilddocmanual|\\
|\input{\childdocname}|\\
|\||else|\\
\textit{document body with }|\input{|\textit{part}|}|\\
|\||fi|
\end{tabular}
\end{center}
%
The conditional |\ifchilddocmanual| is true whenever
a part to be included by |\input| is being compiled,
and the name of the part is stored in |\childdocname|.

%%%%%%%%%%%%%%%%%%%%%%%%%%%%%%%%%%%%%%%%
\DescribeMacro{\childdocby}
Each part to be included by |\input| should start with:
%
\begin{center}
\begin{tabular}{l}
|\input{childdoc.def}|\\
|\childdocby{|\textit{main}|}|\\
\end{tabular}
\end{center}
%
The directive |\childdocby| is similar to |\childdocof|
described in \secref{sec:include},
but the subsequent selection of content must be done manually.
To that end, both |\ifchilddoc| and |\ifchilddocmanual|
will be true upon processing of a part,
and the name of the part is stored in |\childdocname|.
Note that |\jobname| will be set to the filename of the current part
so that each part receives an individual |.aux| file
that does not interfere with the |.aux| file(s) of the main document.
This behaviour can be altered by the alternative form
|\childdocby[*]{|\textit{main}|}| (with a non-empty optional argument)
which uses the |.aux| file of the main document
by setting |\jobname| to \textit{main}.

%%%%%%%%%%%%%%%%%%%%%%%%%%%%%%%%%%%%%%%%%%%%%%%%%%%%%%%%%%%%%%%%%%%%%%%%%%%%%%%%
\subsection{Driver Development}
\label{sec:driver}

The \textsf{childdoc} mechanism can also be use for the development
of definition files such as \LaTeX{} styles or classes.
This case differs from the above setup with multiple parts
included by |\include| in that no |\includeonly| should be invoked.
This can be achieved by starting the include file
(before |\ProvidesPackage|) with:
%
\begin{center}
\begin{tabular}{l}
|\input{childdoc.def}|\\
|\childdocforward{|\textit{main}|}|\\
\end{tabular}
\end{center}
%
or alternatively with:
%
\begin{center}
\begin{tabular}{l}
|\input{childdoc.def}|\\
|\childdocby{|\textit{main}|}|\\
\end{tabular}
\end{center}
%
Both forms have slightly different effects as described above.
The main file is prepared as usual, see \secref{sec:include}.

%%%%%%%%%%%%%%%%%%%%%%%%%%%%%%%%%%%%%%%%%%%%%%%%%%%%%%%%%%%%%%%%%%%%%%%%%%%%%%%%
\subsection{Legacy Detection}
\label{sec:detection}

The directive |\childdocmain| in the main file can detect
whether the complete document or merely a child is to be compiled
even without using the directive |\childdocof|.
This method is deprecated because it is less robust
and there is no compelling reason to use it;
it is merely provided for backward compatibility
and it may be removed in future versions.

If the detection mechanism is to be used,
it is mandatory to correctly specify
the filename of the main file as the argument of |\childdocmain|:
%
\begin{center}
\begin{tabular}{l}
|\input{childdoc.def}|\\
|\childdocmain{|\textit{main}|}|\\
\end{tabular}
\end{center}
%
If |\jobname| does not match the argument \textit{main} of |\childdocmain|,
it is assumed that |\jobname| points to the child file to be compiled.
When using |\childdocmain| with the main file specified as argument,
it suffices to start a child file
with just |\input{|\textit{main}|}|
without loading of the package and using |\childdocof|.
If instead all processing is done
with the appropriate \textsf{childdoc} directives,
the argument of \textit{main} of |\childdocmain| can be empty.

An alternative version of the command line processing described
in \secref{sec:commandline} using the detection mechanism reads:
%
\begin{center}
|... -jobname "|\textit{target}|" "|[\textit{flags}]%
[|\def\jobname{|\textit{dest}|}|]|\input{|\textit{main}|}"|
\end{center}

%%%%%%%%%%%%%%%%%%%%%%%%%%%%%%%%%%%%%%%%%%%%%%%%%%%%%%%%%%%%%%%%%%%%%%%%%%%%%%%%
\subsection{Manual Code}
\label{sec:manual}

In case one cannot be certain whether the definitions file |childdoc.def|
is installed on the target \TeX{} distribution
and one prefers not to ship it,
it is conceivable to paste a few relevant commands into the sources.

To that end, drop all statements |\input{childdoc.def}|
and perform the replacements as outlined below.
Instead of |\childdocmain{|\textit{main}|}| add the following code
to the top of the main file:
%
\begin{center}
\begin{tabular}{l}
|\||ifdefined\childdocname\endinput\||fi\newif\ifchilddoc|\\
|\edef\childdocname{\scantokens\expandafter{\jobname\noexpand}}|\\
|\def\childdocmain{|\textit{main}|}\||ifx\childdocmain\childdocname\||else|\\
|\childdoctrue\includeonly{\childdocname}\let\jobname\childdocmain\||fi|\\
\end{tabular}
\end{center}
%
Instead of |\childdocof{|\textit{main}|}| just include the main file
at the top of each child file:
%
\begin{center}
|\input{|\textit{main}|}|
\end{center}
%
A simple redirection |\childdocforward{|\textit{dest}|}| is achieved by:
%
\begin{center}
|\def\jobname{|\textit{dest}|}\input{\jobname}|
\end{center}
%
The redirection with prefix
|\childdocforwardprefix[|\textit{prefix}|]{|\textit{dest}|}|
is accomplished by:
%
\begin{center}
\begin{tabular}{l}
|{\edef\jobname{\scantokens\expandafter{\jobname\noexpand}}|\\
|\def\redirectjob |\textit{prefix}|#1~~~{\gdef\jobname{|\textit{dest}|#1}}|\\
|\expandafter\redirectjob\jobname~~~}\input{\jobname}|
\end{tabular}
\end{center}

In an alternative approach,
child documents can be compiled by a specific command line
without additional code or specific definitions:
%
\begin{center}
|... -jobname "|\textit{target}|" "|[\textit{flags}]%
|\includeonly{|\textit{dest}|}\input{|\textit{main}|}"|
\end{center}
%

%%%%%%%%%%%%%%%%%%%%%%%%%%%%%%%%%%%%%%%%%%%%%%%%%%%%%%%%%%%%%%%%%%%%%%%%%%%%%%%%
%%%%%%%%%%%%%%%%%%%%%%%%%%%%%%%%%%%%%%%%%%%%%%%%%%%%%%%%%%%%%%%%%%%%%%%%%%%%%%%%
\section{Information}

%%%%%%%%%%%%%%%%%%%%%%%%%%%%%%%%%%%%%%%%%%%%%%%%%%%%%%%%%%%%%%%%%%%%%%%%%%%%%%%%
\subsection{Copyright}

Copyright \copyright{} 2017--2018 Niklas Beisert

This work may be distributed and/or modified under the
conditions of the \LaTeX{} Project Public License, either version 1.3
of this license or (at your option) any later version.
The latest version of this license is in
  \url{http://www.latex-project.org/lppl.txt}
and version 1.3 or later is part of all distributions of \LaTeX{}
version 2005/12/01 or later.

This work has the LPPL maintenance status `maintained'.

The Current Maintainer of this work is Niklas Beisert.

This work consists of the files |README.txt|, |childdoc.ins| and |childdoc.dtx|
as well as the derived files |childdoc.def|, |cdocsamp.tex|
with |cdocsch1.tex|, |cdocsch2.tex|, |cdocspt3.tex|, |cdocspt4.tex|,
|cdocsdrf.tex|, |cdocsfn1.tex|, |cdocsfn2.tex|
as well as |childdoc.pdf|.

%%%%%%%%%%%%%%%%%%%%%%%%%%%%%%%%%%%%%%%%%%%%%%%%%%%%%%%%%%%%%%%%%%%%%%%%%%%%%%%%
\subsection{Files and Installation}

The package consists of the files:
%
\begin{center}
\begin{tabular}{ll}
    |README.txt|   & readme file \\
    |childdoc.ins| & installation file \\
    |childdoc.dtx| & source file \\
    |childdoc.def| & definition file \\
    |cdocsamp.tex| & sample main file \\
    |cdocsch1.tex| & sample include file \\
    |cdocsch2.tex| & sample include file \\
    |cdocspt3.tex| & sample part file \\
    |cdocspt4.tex| & sample part file \\
    |cdocsdrf.tex| & sample redirection file \\
    |cdocsfn1.tex| & sample redirection file \\
    |cdocsfn2.tex| & sample redirection file \\
    |childdoc.pdf| & manual
\end{tabular}
\end{center}
%
The distribution consists of the files
|README.txt|, |childdoc.ins| and |childdoc.dtx|.
%
\begin{itemize}
\item
Run (pdf)\LaTeX{} on |childdoc.dtx|
to compile the manual |childdoc.pdf| (this file).
\item
Run \LaTeX{} on |childdoc.ins| to create the definitions file |childdoc.def|
and the sample |cdocsamp.tex| with include files
|cdocsch1.tex|, |cdocsch2.tex|, |cdocspt3.tex|, |cdocspt4.tex|,
|cdocsdrf.tex|, |cdocsfn1.tex|, |cdocsfn2.tex|.
Then copy the file |childdoc.def| to an appropriate directory of your \LaTeX{}
distribution, e.g.\ \textit{texmf-root}|/tex/latex/childdoc|.
\end{itemize}

%%%%%%%%%%%%%%%%%%%%%%%%%%%%%%%%%%%%%%%%%%%%%%%%%%%%%%%%%%%%%%%%%%%%%%%%%%%%%%%%
\subsection{Related CTAN Packages}

There are several other packages which offer a similar functionality:
%
\begin{itemize}
\item
The packages
\href{http://ctan.org/pkg/docmute}{\textsf{docmute}},
\href{http://ctan.org/pkg/includex}{\textsf{includex}} and
\href{http://ctan.org/pkg/standalone}{\textsf{standalone}}
provide commands to include only the document body of
a child file thus allowing both files to be compiled individually.
\item
The packages \href{http://ctan.org/pkg/subdocs}{\textsf{subdocs}}
and \href{http://ctan.org/pkg/subfiles}{\textsf{subfiles}}
provide structures in which the main and child documents can be
encapsulated and allowing them to be compiled individually.
The inclusion mechanism is different from the conventional |\include|.
\item
The package \href{http://ctan.org/pkg/combine}{\textsf{combine}}
is an elaborate solution to combine several documents into one.
\end{itemize}
%
See also the CTAN topic \href{http://ctan.org/topic/subdocs}{\textsf{subdocs}}
for further related packages.
The present package differs from the above solutions in that
a document structure constructed with the conventional |\include| mechanism
just needs two extra commands at the top of every file
such that all constituent files can be compiled individually.

%%%%%%%%%%%%%%%%%%%%%%%%%%%%%%%%%%%%%%%%%%%%%%%%%%%%%%%%%%%%%%%%%%%%%%%%%%%%%%%%
%\subsection{Feature Suggestions}
%
%The following is a list of features which may be useful for future
%versions of this package:
%%
%\begin{itemize}
%\item
%\ldots
%\end{itemize}

%%%%%%%%%%%%%%%%%%%%%%%%%%%%%%%%%%%%%%%%%%%%%%%%%%%%%%%%%%%%%%%%%%%%%%%%%%%%%%%%
\subsection{Revision History}

%%%%%%%%%%%%%%%%%%%%%%%%%%%%%%%%%%%%%%%%
\paragraph{v2.0:} 2018/12/30

\begin{itemize}
\item
immediate forward processing
\item
added |\childdocby| mechanism
\item
manual restructured
\end{itemize}

%%%%%%%%%%%%%%%%%%%%%%%%%%%%%%%%%%%%%%%%
\paragraph{v1.6:} 2018/01/17

\begin{itemize}
\item
application for development of include files
\item
corrections to manual
\end{itemize}

%%%%%%%%%%%%%%%%%%%%%%%%%%%%%%%%%%%%%%%%
\paragraph{v1.5:} 2017/05/21

\begin{itemize}
\item
more complete structuring introduced
\item
|\childdocof| introduced
\item
|\childdoc| renamed to |\childdocmain|
\item
|\childredirect| renamed to |\childdocforward| and |\childdocforwardprefix|
and functionality expanded
\end{itemize}

%%%%%%%%%%%%%%%%%%%%%%%%%%%%%%%%%%%%%%%%
\paragraph{v1.0:} 2017/04/27

\begin{itemize}
\item
manual and install package
\item
first version published on CTAN
\end{itemize}

%%%%%%%%%%%%%%%%%%%%%%%%%%%%%%%%%%%%%%%%
\paragraph{v0.6:} 2017/04/26

\begin{itemize}
\item
redirection mechanism added
\end{itemize}

%%%%%%%%%%%%%%%%%%%%%%%%%%%%%%%%%%%%%%%%
\paragraph{v0.5:} 2017/04/26

\begin{itemize}
\item
functionality in definition file
\end{itemize}


%%%%%%%%%%%%%%%%%%%%%%%%%%%%%%%%%%%%%%%%%%%%%%%%%%%%%%%%%%%%%%%%%%%%%%%%%%%%%%%%
%%%%%%%%%%%%%%%%%%%%%%%%%%%%%%%%%%%%%%%%%%%%%%%%%%%%%%%%%%%%%%%%%%%%%%%%%%%%%%%%
%%%%%%%%%%%%%%%%%%%%%%%%%%%%%%%%%%%%%%%%%%%%%%%%%%%%%%%%%%%%%%%%%%%%%%%%%%%%%%%%
\appendix

\settowidth\MacroIndent{\rmfamily\scriptsize 000\ }

 \DocInput{childdoc.dtx}

\end{document}
%</driver>
% \fi
%
% %%%%%%%%%%%%%%%%%%%%%%%%%%%%%%%%%%%%%%%%%%%%%%%%%%%%%%%%%%%%%%%%%%%%%%%%%%%%%%
% %%%%%%%%%%%%%%%%%%%%%%%%%%%%%%%%%%%%%%%%%%%%%%%%%%%%%%%%%%%%%%%%%%%%%%%%%%%%%%
% \section{Sample}
%\iffalse
%<*samplemain>
%\fi
%
% The following presents a sample document
% with two chapters, two parts, a title page,
% a compile flag as well as three forwarding files to set the flag.
% It consists of eight |.tex| files:
% \begin{center}
% \begin{tabular}{ll}
% |cdocsamp.tex|&main file\\
% |cdocsch1.tex|&include file for chapter 1\\
% |cdocsch2.tex|&include file for chapter 2\\
% |cdocspt3.tex|&include file for part 3\\
% |cdocspt4.tex|&include file for part 4\\
% |cdocsdrf.tex|&forwarding file for main file in draft mode\\
% |cdocsfi1.tex|&forwarding file for final version of chapter 1\\
% |cdocsfi2.tex|&forwarding file for final version of chapter 2\\
% \end{tabular}
% \end{center}
% Each of the eight files can be compiled directly by the \LaTeX{} compiler.
%
% %%%%%%%%%%%%%%%%%%%%%%%%%%%%%%%%%%%%%%
% \paragraph{Main File.}
%
% The main file is called |cdocsamp.tex|.
%
% Load the \textsf{childdoc} definitions and
% declare the filename for the main document:
%    \begin{macrocode}
\input{childdoc.def}
\childdocmain{}
%    \end{macrocode}

% Optional override for |\version| flag:
%    \begin{macrocode}
%%\ifchilddoc\else\providecommand{\version}{draft}\fi
%    \end{macrocode}

% Define the default values for the |\version| flag
% (|final| for the main file and |draft| for childs):
%    \begin{macrocode}
\ifchilddoc
\providecommand{\version}{draft}
\else
\providecommand{\version}{final}
\fi
%    \end{macrocode}

% Load the standard document class:
%    \begin{macrocode}
\documentclass[12pt]{article}
%    \end{macrocode}

% Start the document body:
%    \begin{macrocode}
\begin{document}
%    \end{macrocode}

% Declare a title page.
% Print title, part of document being processed and version flag:
%    \begin{macrocode}
\addtocounter{page}{-1}
\begin{center}
{\LARGE\bfseries{}childdoc example\par}
\vspace{1cm}
\ifchilddoc
\ifchilddocmanual part\else chapter\fi:
`\childdocname' of `\childdocjob'\par
\else
main document: `\childdocjob'\par
\fi
version: \version\par
\end{center}
\newpage
%    \end{macrocode}

% Manually include selected file,
% otherwise process as usual:
%    \begin{macrocode}
\ifchilddocmanual
\section*{part `\childdocname'}
\input{\childdocname}
\else
%    \end{macrocode}

% Include the two chapters:
%    \begin{macrocode}
\include{cdocsch1}
\include{cdocsch2}
%    \end{macrocode}

% Include the two parts unless only chapters should be displayed:
%    \begin{macrocode}
\ifchilddoc\else
\section{part three}
\input{cdocspt3}
\section{part four}
\input{cdocspt4}
\fi
%    \end{macrocode}

% Process as usual until here:
%    \begin{macrocode}
\fi
%    \end{macrocode}

% End of document body:
%    \begin{macrocode}
\end{document}
%    \end{macrocode}
%\iffalse
%</samplemain>
%\fi
%
% %%%%%%%%%%%%%%%%%%%%%%%%%%%%%%%%%%%%%%
% \paragraph{Chapter Include Files.}
%
% The include files are called |cdocsch1.tex| and |cdocsch2.tex|.
%
%\iffalse
%<*samplechap1|samplechap2>
%\fi

% Optional override for |\version| flag:
%    \begin{macrocode}
%%\providecommand{\version}{final}
%    \end{macrocode}

% Include the main document:
%    \begin{macrocode}
\input{childdoc.def}
\childdocof{cdocsamp}
%    \end{macrocode}

%\iffalse
%</samplechap1|samplechap2>
%\fi
%
%\iffalse
%<*samplechap1>
%\fi
% Some text for chapter 1:
%    \begin{macrocode}
\section{one}
some text in chapter one
%    \end{macrocode}

%\iffalse
%</samplechap1>
%\fi
% Some text for chapter 2:
%\iffalse
%<*samplechap2>
%\fi
%    \begin{macrocode}
\section{two}
more text in chapter two
%    \end{macrocode}

%\iffalse
%</samplechap2>
%\fi
%
% %%%%%%%%%%%%%%%%%%%%%%%%%%%%%%%%%%%%%%
% \paragraph{Part Include Files.}
%
% The include files are called |cdocspt3.tex| and |cdocspt4.tex|.
%
%\iffalse
%<*samplepart3|samplepart4>
%\fi

% Optional override for |\version| flag:
%    \begin{macrocode}
%%\providecommand{\version}{final}
%    \end{macrocode}

% Include the main document:
%    \begin{macrocode}
\input{childdoc.def}
\childdocby{cdocsamp}
%    \end{macrocode}

%\iffalse
%</samplepart3|samplepart4>
%\fi
%
%\iffalse
%<*samplepart3>
%\fi
% Some text for part 3:
%    \begin{macrocode}
some text in part three
%    \end{macrocode}

%\iffalse
%</samplepart3>
%\fi
% Some text for part 4:
%\iffalse
%<*samplepart4>
%\fi
%    \begin{macrocode}
more text in part four
%    \end{macrocode}

%\iffalse
%</samplepart4>
%\fi
%
% %%%%%%%%%%%%%%%%%%%%%%%%%%%%%%%%%%%%%%
% \paragraph{Forwarding for a Complete Draft.}
%
% The following forwarding file |cdocsdrf.tex|
% compiles the main document in draft mode:
%\iffalse
%<*sampledraft>
%\fi
%    \begin{macrocode}
\def\version{draft}
\input{childdoc.def}
\childdocforward{cdocsamp}
%    \end{macrocode}

%\iffalse
%</sampledraft>
%\fi
%
% %%%%%%%%%%%%%%%%%%%%%%%%%%%%%%%%%%%%%%
% \paragraph{Forwarding for Final Version of the Chapters.}
%
% The following forwarding files |cdocsfn1.tex| and |cdocsfn2.tex|
% (with identical content)
% compile the final versions of the child documents
% |cdocsch1.tex| and |cdocsch2.tex|, respectively:
%\iffalse
%<*samplefinal>
%\fi
%    \begin{macrocode}
\def\version{final}
\input{childdoc.def}
\childdocforwardprefix[cdocsamp]{cdocsfn}{cdocsch}
%    \end{macrocode}

%\iffalse
%</samplefinal>
%\fi
%
% %%%%%%%%%%%%%%%%%%%%%%%%%%%%%%%%%%%%%%
% \paragraph{Command Line Processing.}
%
% The following three command lines generate the output files
% |cdocscld|, |cdocscl1| and |cdocscl2|
% which should be identical to
% |cdocsdrf|, |cdocsch1| and |cdocsfn2|, respectively:
% \begin{center}
% \begin{tabular}{l}
% |latex -jobname cdocscld \|\\
% |  "\def\version{draft}\input{childdoc.def}\childdocforward{cdocsamp}"|\\
% |latex -jobname cdocscl1 \|\\
% |  "\input{childdoc.def}\childdocforward[cdocsamp]{cdocsch1}"|\\
% |latex -jobname cdocscl2 \|\\
% |  "\def\version{final}\input{childdoc.def}\childdocforward{cdocsch2}"|
% \end{tabular}
% \end{center}
% Note that the trailing backslash on each first line
% merely continues the input to the second line
% (for convenient cut ant paste).
% Furthermore, the command |latex| can be replaced by any
% of its alternative versions such as |pdflatex|.
%
% %%%%%%%%%%%%%%%%%%%%%%%%%%%%%%%%%%%%%%%%%%%%%%%%%%%%%%%%%%%%%%%%%%%%%%%%%%%%%%
% %%%%%%%%%%%%%%%%%%%%%%%%%%%%%%%%%%%%%%%%%%%%%%%%%%%%%%%%%%%%%%%%%%%%%%%%%%%%%%
% \section{Implementation}
%\iffalse
%<*package>
%\fi
%
% This section describes the definitions file |childdoc.def|.

% The definitions cannot be loaded using |\usepackage| or |\RequirePackage|
% which has a mechanism to prevent loading a style file more than once.
% When loading the definitions by means of |\input|
% multiple instances have to be prevented manually:
%\iffalse
%This code needs to be before the `\ProvidesFile' directive
%which is defined at the beginning of this file.
%Therefore it is also placed there and commented out here.
%</package>
%<*discard>
%\fi
%    \begin{macrocode}
\ifdefined\childdocmain\endinput\fi
%    \end{macrocode}
%\iffalse
%</discard>
%<*package>
%\fi
%
% \macro{\ifchilddoc}
% \macro{\ifchilddocmanual}
% The conditional |\ifchilddoc| tells whether a
% child (true) or main (false) document is being compiled.
% The conditional |\ifchilddocmanual| tells whether
% the |\includeonly| mechanism is used (false) or
% the selection of child files must be performed manually (true).
% The definitions initialise to false:
%    \begin{macrocode}
\newif\ifchilddoc
\newif\ifchilddocmanual
%    \end{macrocode}

% \macro{\childdocname}
% \macro{\childdocjob}
% The macro |\childdocname| stores the name of the main document
% to be compiled. The macro |\childdocjob| stores the name of
% the document on which the \LaTeX{} compiler was originally invoked.
% The content of |\jobname| cannot be compared
% to filenames specified in the source due to different catcodes.
% The following code rescans |\jobname|, stores the result
% in |\childdocname| and saves a copy in |\childdocjob|:
%    \begin{macrocode}
\edef\childdocname{\scantokens\expandafter{\jobname\noexpand}}
\let\childdocjob\childdocname
%    \end{macrocode}

% \macro{\childdocdisable}
% The macro |\childdocdisable| prevents the main file
% from being processed more than once.
% At this stage, the main document command |\childdocmain|
% is assumed to be called once again where it should do nothing.
% Any subsequent call to it should prevent
% a secondary processing of the main document
% It overwrites the forwarding commands
% |\childdocof| and |\childdocforward|
% with empty macros to prevent further inclusions of the main document:
%    \begin{macrocode}
\newcommand{\childdocdisable}
{
  \renewcommand{\childdocmain}[1]{\renewcommand{\childdocmain}[1]{\endinput}}
  \renewcommand{\childdocof}[1]{}
  \renewcommand{\childdocby}[2][]{}
  \renewcommand{\childdocforward}[2][]{}
  \renewcommand{\childdocdisable}{}
}
%    \end{macrocode}

% \macro{\childdocmain}
% The macro |\childdocmain| is to be called at the top of the main file
% with nothing or the main filename (without extension) as argument.
% First, it breaks loops.
% If the argument is not empty and does not match |\childdocname|
% (which is set by the first inclusion of |childdoc.def|),
% |\ifchilddoc| is set to true, |\includeonly| is applied to the child file
% and |\jobname| is set to the main file
% (for proper handling of |.aux| files):
%    \begin{macrocode}
\newcommand{\childdocmain}[1]
{
  \childdocdisable\childdocmain{}
  \if?#1?\else
    \begingroup
      \def\childdoctmp{#1}
      \ifx\childdoctmp\childdocname
        \def\childdoctmp{}
      \else
        \def\childdoctmp
        {
          \childdoctrue
          \includeonly{\childdocname}
          \def\childdocjob{#1}
          \def\jobname{#1}
        }
      \fi
      \expandafter
    \endgroup
    \childdoctmp
  \fi
}
%    \end{macrocode}

% \macro{\childdocof}
% The command |\childdocof| redirects
% compilation to the main file |#1|.
%    \begin{macrocode}
\newcommand{\childdocof}[1]
{
  \childdocdisable
  \childdoctrue
  \includeonly{\childdocname}
  \def\jobname{#1}
  \def\childdocjob{#1}
  \input{#1}
}
%    \end{macrocode}

% \macro{\childdocby}
% The command |\childdocby| ....
%    \begin{macrocode}
\newcommand{\childdocby}[2][]
{
  \childdocdisable
  \childdoctrue
  \childdocmanualtrue
  \if?#1?\else
    \def\jobname{#2}
  \fi
  \def\childdocjob{#2}
  \input{#2}
  \endinput
}
%    \end{macrocode}

% \macro{\childdocforward}
% The command |\childdocforward| redirects
% compilation to the main file or
% (if the optional argument is given) a child file.
% Parameters are set as if the main file
% or a child file starting with |\childdocof| was compiled.
% Then compilation is handed over to the main file:
%    \begin{macrocode}
\newcommand{\childdocforward}[2][]
{
  \begingroup
    \if?#1?
      \def\childdoctmp
      {
        \def\childdocname{#2}
        \def\childdocjob{#2}
        \def\jobname{#2}
        \input{#2}
        \endinput
      }
    \else
      \def\childdoctmp
      {
        \childdocdisable
        \def\childdocname{#2}
        \childdoctrue
        \includeonly{#2}
        \def\childdocjob{#1}
        \def\jobname{#1}
        \input{#1}
        \endinput
      }
    \fi
    \expandafter
  \endgroup
  \childdoctmp
}
%    \end{macrocode}

% \macro{\childdocforwardprefix}
% The command |\childdocforwardprefix| redirects
% compilation to the main or a child file by means of a pattern.
% The prefix |#1| in the current filename is replaced by |#2|
% and the suffix of the current filename is kept
% (it is assumed that the filename does not contain the substring `|~~~|'
% which is used as a delimiter).
% Compilation is handed over to the new file by |\childdocforward|:
%    \begin{macrocode}
\newcommand{\childdocforwardprefix}[3][]
{
  \begingroup
    \def\childdocextract #2##1~~~{\def\childdoctmp{\childdocforward[#1]{#3##1}}}
    \expandafter\childdocextract\childdocname~~~
    \expandafter
  \endgroup
  \childdoctmp
}
%    \end{macrocode}

% \macro{\childdoc}
% The deprecated macro |\childdoc| is a legacy version of |\childdocmain|:
%    \begin{macrocode}
\newcommand{\childdoc}{\childdocmain}
%    \end{macrocode}

% \macro{\childdocredirect}
% The deprecated macro |\childdocredirect| is a legacy version
% of |\childdocforward| and |\childdocforwardprefix|:
%    \begin{macrocode}
\newcommand{\childdocredirect}[2][]
{
  \begingroup
    \if?#1?
      \def\childdoctmp{\childdocforward{#2}}
    \else
      \def\childdoctmp{\childdocforwardprefix{#1}{#2}}
    \fi
    \expandafter
  \endgroup
  \childdoctmp
}
%    \end{macrocode}

%\iffalse
%</package>
%\fi
%
\endinput
|
and perform the replacements as outlined below.
Instead of |\childdocmain{|\textit{main}|}| add the following code
to the top of the main file:
%
\begin{center}
\begin{tabular}{l}
|\||ifdefined\childdocname\endinput\||fi\newif\ifchilddoc|\\
|\edef\childdocname{\scantokens\expandafter{\jobname\noexpand}}|\\
|\def\childdocmain{|\textit{main}|}\||ifx\childdocmain\childdocname\||else|\\
|\childdoctrue\includeonly{\childdocname}\let\jobname\childdocmain\||fi|\\
\end{tabular}
\end{center}
%
Instead of |\childdocof{|\textit{main}|}| just include the main file
at the top of each child file:
%
\begin{center}
|\input{|\textit{main}|}|
\end{center}
%
A simple redirection |\childdocforward{|\textit{dest}|}| is achieved by:
%
\begin{center}
|\def\jobname{|\textit{dest}|}\input{\jobname}|
\end{center}
%
The redirection with prefix
|\childdocforwardprefix[|\textit{prefix}|]{|\textit{dest}|}|
is accomplished by:
%
\begin{center}
\begin{tabular}{l}
|{\edef\jobname{\scantokens\expandafter{\jobname\noexpand}}|\\
|\def\redirectjob |\textit{prefix}|#1~~~{\gdef\jobname{|\textit{dest}|#1}}|\\
|\expandafter\redirectjob\jobname~~~}\input{\jobname}|
\end{tabular}
\end{center}

In an alternative approach,
child documents can be compiled by a specific command line
without additional code or specific definitions:
%
\begin{center}
|... -jobname "|\textit{target}|" "|[\textit{flags}]%
|\includeonly{|\textit{dest}|}\input{|\textit{main}|}"|
\end{center}
%

%%%%%%%%%%%%%%%%%%%%%%%%%%%%%%%%%%%%%%%%%%%%%%%%%%%%%%%%%%%%%%%%%%%%%%%%%%%%%%%%
%%%%%%%%%%%%%%%%%%%%%%%%%%%%%%%%%%%%%%%%%%%%%%%%%%%%%%%%%%%%%%%%%%%%%%%%%%%%%%%%
\section{Information}

%%%%%%%%%%%%%%%%%%%%%%%%%%%%%%%%%%%%%%%%%%%%%%%%%%%%%%%%%%%%%%%%%%%%%%%%%%%%%%%%
\subsection{Copyright}

Copyright \copyright{} 2017--2018 Niklas Beisert

This work may be distributed and/or modified under the
conditions of the \LaTeX{} Project Public License, either version 1.3
of this license or (at your option) any later version.
The latest version of this license is in
  \url{http://www.latex-project.org/lppl.txt}
and version 1.3 or later is part of all distributions of \LaTeX{}
version 2005/12/01 or later.

This work has the LPPL maintenance status `maintained'.

The Current Maintainer of this work is Niklas Beisert.

This work consists of the files |README.txt|, |childdoc.ins| and |childdoc.dtx|
as well as the derived files |childdoc.def|, |cdocsamp.tex|
with |cdocsch1.tex|, |cdocsch2.tex|, |cdocspt3.tex|, |cdocspt4.tex|,
|cdocsdrf.tex|, |cdocsfn1.tex|, |cdocsfn2.tex|
as well as |childdoc.pdf|.

%%%%%%%%%%%%%%%%%%%%%%%%%%%%%%%%%%%%%%%%%%%%%%%%%%%%%%%%%%%%%%%%%%%%%%%%%%%%%%%%
\subsection{Files and Installation}

The package consists of the files:
%
\begin{center}
\begin{tabular}{ll}
    |README.txt|   & readme file \\
    |childdoc.ins| & installation file \\
    |childdoc.dtx| & source file \\
    |childdoc.def| & definition file \\
    |cdocsamp.tex| & sample main file \\
    |cdocsch1.tex| & sample include file \\
    |cdocsch2.tex| & sample include file \\
    |cdocspt3.tex| & sample part file \\
    |cdocspt4.tex| & sample part file \\
    |cdocsdrf.tex| & sample redirection file \\
    |cdocsfn1.tex| & sample redirection file \\
    |cdocsfn2.tex| & sample redirection file \\
    |childdoc.pdf| & manual
\end{tabular}
\end{center}
%
The distribution consists of the files
|README.txt|, |childdoc.ins| and |childdoc.dtx|.
%
\begin{itemize}
\item
Run (pdf)\LaTeX{} on |childdoc.dtx|
to compile the manual |childdoc.pdf| (this file).
\item
Run \LaTeX{} on |childdoc.ins| to create the definitions file |childdoc.def|
and the sample |cdocsamp.tex| with include files
|cdocsch1.tex|, |cdocsch2.tex|, |cdocspt3.tex|, |cdocspt4.tex|,
|cdocsdrf.tex|, |cdocsfn1.tex|, |cdocsfn2.tex|.
Then copy the file |childdoc.def| to an appropriate directory of your \LaTeX{}
distribution, e.g.\ \textit{texmf-root}|/tex/latex/childdoc|.
\end{itemize}

%%%%%%%%%%%%%%%%%%%%%%%%%%%%%%%%%%%%%%%%%%%%%%%%%%%%%%%%%%%%%%%%%%%%%%%%%%%%%%%%
\subsection{Related CTAN Packages}

There are several other packages which offer a similar functionality:
%
\begin{itemize}
\item
The packages
\href{http://ctan.org/pkg/docmute}{\textsf{docmute}},
\href{http://ctan.org/pkg/includex}{\textsf{includex}} and
\href{http://ctan.org/pkg/standalone}{\textsf{standalone}}
provide commands to include only the document body of
a child file thus allowing both files to be compiled individually.
\item
The packages \href{http://ctan.org/pkg/subdocs}{\textsf{subdocs}}
and \href{http://ctan.org/pkg/subfiles}{\textsf{subfiles}}
provide structures in which the main and child documents can be
encapsulated and allowing them to be compiled individually.
The inclusion mechanism is different from the conventional |\include|.
\item
The package \href{http://ctan.org/pkg/combine}{\textsf{combine}}
is an elaborate solution to combine several documents into one.
\end{itemize}
%
See also the CTAN topic \href{http://ctan.org/topic/subdocs}{\textsf{subdocs}}
for further related packages.
The present package differs from the above solutions in that
a document structure constructed with the conventional |\include| mechanism
just needs two extra commands at the top of every file
such that all constituent files can be compiled individually.

%%%%%%%%%%%%%%%%%%%%%%%%%%%%%%%%%%%%%%%%%%%%%%%%%%%%%%%%%%%%%%%%%%%%%%%%%%%%%%%%
%\subsection{Feature Suggestions}
%
%The following is a list of features which may be useful for future
%versions of this package:
%%
%\begin{itemize}
%\item
%\ldots
%\end{itemize}

%%%%%%%%%%%%%%%%%%%%%%%%%%%%%%%%%%%%%%%%%%%%%%%%%%%%%%%%%%%%%%%%%%%%%%%%%%%%%%%%
\subsection{Revision History}

%%%%%%%%%%%%%%%%%%%%%%%%%%%%%%%%%%%%%%%%
\paragraph{v2.0:} 2018/12/30

\begin{itemize}
\item
immediate forward processing
\item
added |\childdocby| mechanism
\item
manual restructured
\end{itemize}

%%%%%%%%%%%%%%%%%%%%%%%%%%%%%%%%%%%%%%%%
\paragraph{v1.6:} 2018/01/17

\begin{itemize}
\item
application for development of include files
\item
corrections to manual
\end{itemize}

%%%%%%%%%%%%%%%%%%%%%%%%%%%%%%%%%%%%%%%%
\paragraph{v1.5:} 2017/05/21

\begin{itemize}
\item
more complete structuring introduced
\item
|\childdocof| introduced
\item
|\childdoc| renamed to |\childdocmain|
\item
|\childredirect| renamed to |\childdocforward| and |\childdocforwardprefix|
and functionality expanded
\end{itemize}

%%%%%%%%%%%%%%%%%%%%%%%%%%%%%%%%%%%%%%%%
\paragraph{v1.0:} 2017/04/27

\begin{itemize}
\item
manual and install package
\item
first version published on CTAN
\end{itemize}

%%%%%%%%%%%%%%%%%%%%%%%%%%%%%%%%%%%%%%%%
\paragraph{v0.6:} 2017/04/26

\begin{itemize}
\item
redirection mechanism added
\end{itemize}

%%%%%%%%%%%%%%%%%%%%%%%%%%%%%%%%%%%%%%%%
\paragraph{v0.5:} 2017/04/26

\begin{itemize}
\item
functionality in definition file
\end{itemize}


%%%%%%%%%%%%%%%%%%%%%%%%%%%%%%%%%%%%%%%%%%%%%%%%%%%%%%%%%%%%%%%%%%%%%%%%%%%%%%%%
%%%%%%%%%%%%%%%%%%%%%%%%%%%%%%%%%%%%%%%%%%%%%%%%%%%%%%%%%%%%%%%%%%%%%%%%%%%%%%%%
%%%%%%%%%%%%%%%%%%%%%%%%%%%%%%%%%%%%%%%%%%%%%%%%%%%%%%%%%%%%%%%%%%%%%%%%%%%%%%%%
\appendix

\settowidth\MacroIndent{\rmfamily\scriptsize 000\ }

 \DocInput{childdoc.dtx}

\end{document}
%</driver>
% \fi
%
% %%%%%%%%%%%%%%%%%%%%%%%%%%%%%%%%%%%%%%%%%%%%%%%%%%%%%%%%%%%%%%%%%%%%%%%%%%%%%%
% %%%%%%%%%%%%%%%%%%%%%%%%%%%%%%%%%%%%%%%%%%%%%%%%%%%%%%%%%%%%%%%%%%%%%%%%%%%%%%
% \section{Sample}
%\iffalse
%<*samplemain>
%\fi
%
% The following presents a sample document
% with two chapters, two parts, a title page,
% a compile flag as well as three forwarding files to set the flag.
% It consists of eight |.tex| files:
% \begin{center}
% \begin{tabular}{ll}
% |cdocsamp.tex|&main file\\
% |cdocsch1.tex|&include file for chapter 1\\
% |cdocsch2.tex|&include file for chapter 2\\
% |cdocspt3.tex|&include file for part 3\\
% |cdocspt4.tex|&include file for part 4\\
% |cdocsdrf.tex|&forwarding file for main file in draft mode\\
% |cdocsfi1.tex|&forwarding file for final version of chapter 1\\
% |cdocsfi2.tex|&forwarding file for final version of chapter 2\\
% \end{tabular}
% \end{center}
% Each of the eight files can be compiled directly by the \LaTeX{} compiler.
%
% %%%%%%%%%%%%%%%%%%%%%%%%%%%%%%%%%%%%%%
% \paragraph{Main File.}
%
% The main file is called |cdocsamp.tex|.
%
% Load the \textsf{childdoc} definitions and
% declare the filename for the main document:
%    \begin{macrocode}
% \iffalse
%
% childdoc.dtx Copyright (C) 2017-2018 Niklas Beisert
%
% This work may be distributed and/or modified under the
% conditions of the LaTeX Project Public License, either version 1.3
% of this license or (at your option) any later version.
% The latest version of this license is in
%   http://www.latex-project.org/lppl.txt
% and version 1.3 or later is part of all distributions of LaTeX
% version 2005/12/01 or later.
%
% This work has the LPPL maintenance status `maintained'.
%
% The Current Maintainer of this work is Niklas Beisert.
%
% This work consists of the files childdoc.dtx and childdoc.ins
% and the derived files childdoc.def and cdocsamp.tex with
% cdocsch1.tex, cdocsch2.tex, cdocsdrf.tex, cdocsfn1.tex, cdocsfn2.tex.
%
%<package>\ifdefined\childdocmain\endinput\fi
%<package>\ProvidesFile{childdoc.def}[2018/12/30 v2.0 child document driver]
%<samplemain>\ProvidesFile{cdocsamp.tex}[2018/12/30 v2.0 sample for childdoc]
%<*driver>
%\ProvidesFile{childdoc.drv}[2018/12/30 v2.0 childdoc reference manual file]
\PassOptionsToClass{10pt,a4paper}{article}
\documentclass{ltxdoc}

\usepackage[margin=35mm]{geometry}
\usepackage{hyperref}
\usepackage{hyperxmp}
\usepackage[usenames]{color}

\hypersetup{colorlinks=true}
\hypersetup{pdfstartview=FitH}
\hypersetup{pdfpagemode=UseNone}
\hypersetup{pdfsource={}}
\hypersetup{pdflang={en-UK}}
\hypersetup{pdfcopyright={Copyright 2017-2018 Niklas Beisert.
  This work may be distributed and/or modified under the
  conditions of the LaTeX Project Public License, either version 1.3
  of this license or (at your option) any later version.}}
\hypersetup{pdflicenseurl={http://www.latex-project.org/lppl.txt}}
\hypersetup{pdfcontactaddress={ETH Zurich, ITP, HIT K,
  Wolfgang-Pauli-Strasse 27}}
\hypersetup{pdfcontactpostcode={8093}}
\hypersetup{pdfcontactcity={Zurich}}
\hypersetup{pdfcontactcountry={Switzerland}}
\hypersetup{pdfcontactemail={nbeisert@itp.phys.ethz.ch}}
\hypersetup{pdfcontacturl={http://people.phys.ethz.ch/\xmptilde nbeisert/}}

\newcommand{\secref}[1]{\hyperref[#1]{section \ref*{#1}}}

\parskip1ex
\parindent0pt
\let\olditemize\itemize
\def\itemize{\olditemize\parskip0pt}

\begin{document}

\title{The \textsf{childdoc} Package}
\hypersetup{pdftitle={The childdoc Package}}
\author{Niklas Beisert\\[2ex]
  Institut f\"ur Theoretische Physik\\
  Eidgen\"ossische Technische Hochschule Z\"urich\\
  Wolfgang-Pauli-Strasse 27, 8093 Z\"urich, Switzerland\\[1ex]
  \href{mailto:nbeisert@itp.phys.ethz.ch}
  {\texttt{nbeisert@itp.phys.ethz.ch}}}
\hypersetup{pdfauthor={Niklas Beisert}}
\hypersetup{pdfsubject={Manual for the LaTeX2e Package childdoc}}
\date{30 December 2018, \textsf{v2.0}}
\maketitle

\begin{abstract}\noindent
\textsf{childdoc} is a \LaTeXe{} package
that enables the direct compilation
of document sections included by |\include|
to individual files.
\end{abstract}

\begingroup
\parskip0ex
\tableofcontents
\endgroup

%%%%%%%%%%%%%%%%%%%%%%%%%%%%%%%%%%%%%%%%%%%%%%%%%%%%%%%%%%%%%%%%%%%%%%%%%%%%%%%%
%%%%%%%%%%%%%%%%%%%%%%%%%%%%%%%%%%%%%%%%%%%%%%%%%%%%%%%%%%%%%%%%%%%%%%%%%%%%%%%%
\section{Introduction}

\LaTeX{} provides a mechanism to structure a large document (such as a book)
into a main file and several child files (containing the chapters)
using the |\include| command.
This mechanism is beneficial for documents
which span hundreds of pages in order to
make the source file(s) more manageable.
Moreover, compilation can be restricted to
selected child files by means of the |\includeonly| command.
The latter feature can be used to reduce the compilation time while editing
(this was significantly more useful in the earlier days of \LaTeX{})
or to generate a smaller document which is easier to navigate.
Another application of |\includeonly| is to generate
documents consisting of selected parts of the complete document.

However, there are a few drawbacks of the plain |\include| mechanism:
\begin{itemize}
\item
The child files cannot be compiled on their own,
they can only be compiled via the main file.
A naive editing environment
(such as a text editor with an option
to have the current file processed by \LaTeX)
may require one to switch to the main file before compiling;
attempting to compile the child file produces errors.
\item
The main file must be modified (each time)
to adjust the |\includeonly| command
to the present needs. This easily leaves the main file in a messy state.
\item
The generated document will always carry the filename
of the main document. This is inconvenient if
several child files are to be compiled and
to be kept for distribution.
\end{itemize}

The present package provides a simple interface
to make child files individually compilable by \LaTeX{}.
Compiling a child file then has the same effect as compiling
the main file with an |\includeonly| command
to select the appropriate child.
Moreover the generated document will carry the name of the child
rather than the main file.
This resolves all three above issues.

This feature is meant to make the editing of books,
thesis documents and lecture notes somewhat more convenient.
However, the package can also be used efficiently for
composing a series of documents (such as exercise sheets)
which are typically distributed individually.
It then assists the author in generating the individual documents
(potentially in different versions)
as well as a document containing the collected series.
Another application is in developing style files
or other kinds of included material
where compilation of the style file could redirect
to a sample or test file.

%%%%%%%%%%%%%%%%%%%%%%%%%%%%%%%%%%%%%%%%%%%%%%%%%%%%%%%%%%%%%%%%%%%%%%%%%%%%%%%%
%%%%%%%%%%%%%%%%%%%%%%%%%%%%%%%%%%%%%%%%%%%%%%%%%%%%%%%%%%%%%%%%%%%%%%%%%%%%%%%%
\section{Usage}

First of all, the package \textsf{childdoc} is \emph{not} a standard
\LaTeXe{} |.sty| style file! Therefore it needs to be invoked in
a non-standard way.

%%%%%%%%%%%%%%%%%%%%%%%%%%%%%%%%%%%%%%%%%%%%%%%%%%%%%%%%%%%%%%%%%%%%%%%%%%%%%%%%
\subsection{Included Files}
\label{sec:include}

%%%%%%%%%%%%%%%%%%%%%%%%%%%%%%%%%%%%%%%%
\DescribeMacro{\childdocmain}
To use the package, add the commands
\begin{center}
\begin{tabular}{l}
|\input{childdoc.def}|\\
|\childdocmain{}|\\
\end{tabular}
\end{center}
at the very top of the main \LaTeX{} file,
in particular \emph{before} the |\documentclass| statement!
The argument of |\childdocmain| should be left empty
(but it must be present).

%%%%%%%%%%%%%%%%%%%%%%%%%%%%%%%%%%%%%%%%
\DescribeMacro{\childdocof}
Furthermore, add the commands
\begin{center}
\begin{tabular}{l}
|\input{childdoc.def}|\\
|\childdocof{|\textit{main}|}|\\
\end{tabular}
\end{center}
at the top of every child file \textit{child}
which is included by |\include{|\textit{child}|}|
from within the main file
(or at least for those files to be compiled individually).
The argument \textit{main} must be the filename of the main file.

There are a couple of
considerations in setting up the main and child documents:

%%%%%%%%%%%%%%%%%%%%%%%%%%%%%%%%%%%%%%%%
\paragraph{Restrictions.}

Please note the following restrictions:
\begin{itemize}
\item
|\childdocmain| must be called with one argument \textit{main}
to ensure compatibility with earlier version of the package.
It must either be empty (|\childdocmain{}|)
or precisely match the filename of the main file in which it is specified.
See \secref{sec:detection} for further information.
\item
The filename \textit{main} must be specified without the |.tex| extension.
\item
The filename \textit{main} is case sensitive
(even in case-insensitive file systems)
due to internal string comparison.
\item
The argument \textit{main} should be fully expanded, it cannot be a macro.
\item
Subdirectories and special characters should be avoided in filenames.
\item
The command |\childdocmain{|\textit{main}|}| must be followed by a whitespace.
It should not be followed immediately by another command
or by a comment mark `|%|'.
This is because the \TeX{} parser reads the token immediately following
the argument of |\childdocmain| and puts it
at the beginning of every child section;
however, a white\-space is ignored.
\end{itemize}

%%%%%%%%%%%%%%%%%%%%%%%%%%%%%%%%%%%%%%%%
\paragraph{Content of Main File.}

It is advisable to place all content in the child files included by |\include|.
Any output contained in the main file will appear in all child documents
unless suppressed manually;
it cannot be suppressed automatically by the |\includeonly| directive
and thus should normally be avoided.
A method to include some content in the main file
by means of conditional processing is described in \secref{sec:conditional}.

%%%%%%%%%%%%%%%%%%%%%%%%%%%%%%%%%%%%%%%%
\paragraph{Page Numbering.}

When only a part of the document is compiled,
the appropriate numbering of pages
(as well as other status parameters)
is determined from the |.aux| files.
The latter contain information from previous passes.
However this information needs to propagate through
all intermediate child documents.
Therefore the page numbering in child documents may well
be inconsistent until the complete document is compiled at least once.

A useful (if unconventional) way to always ensure a consistent
page numbering is to restart the numbering in each child document
and denote the pages by `\textit{child}|.|\textit{page}'
where \textit{child} represents the chapter/section number of the child file.
This can be achieved by the command
|\numberwithin{page}{|\textit{child}|}|
of the \textsf{amsmath} package
where \textit{child} can be |chapter| or |section|
depending on the chosen structuring.
Alternatively, one can modify the macro |\thepage| appropriately
and reset the counter |page| at the start of each child file.

%%%%%%%%%%%%%%%%%%%%%%%%%%%%%%%%%%%%%%%%%%%%%%%%%%%%%%%%%%%%%%%%%%%%%%%%%%%%%%%%
\subsection{Conditional Processing}
\label{sec:conditional}

The package provides a mechanism to compile different versions
of a document. To customise the versions further some conditional processing
can come in handy to distinguish which version is being compiled.
The package provides two macros to describe the compilation context:

%%%%%%%%%%%%%%%%%%%%%%%%%%%%%%%%%%%%%%%%
\DescribeMacro{\ifchilddoc}
The conditional |\ifchilddoc| distinguishes between the compilation of
child documents and the main document:
%
\begin{center}
|\ifchilddoc |\textit{child-code}| |[|\||else |\textit{main-code}]| \||fi|
\end{center}

%%%%%%%%%%%%%%%%%%%%%%%%%%%%%%%%%%%%%%%%
\DescribeMacro{\childdocname}
\DescribeMacro{\childdocjob}
The macro |\childdocname| contains the filename (without extension)
of the main or child file being processed.
Note that |\childdocjob| will always contain the name of the main file.

%%%%%%%%%%%%%%%%%%%%%%%%%%%%%%%%%%%%%%%%
\paragraph{Title Page.}

Conditional processing can be used to include a title or banner page
in the main document when proper precautions are taken.
Importantly, the code in the main file should ensure that the page counter
(as well as other status parameters which are stored in the |.aux| files)
takes the same value after the conditional processing.
Otherwise the page numbers may take divergent values
depending on which part is compiled.

For example, a title page could be declared by:
%
\begin{center}
\begin{tabular}{l}
|\ifchilddoc\||else|\\
|\addtocounter{page}{-1}|\\
\textit{code for title page}\\
|\newpage|\\
|\||fi|
\end{tabular}
\end{center}
%
A banner page for the child documents can be generated by:
%
\begin{center}
\begin{tabular}{l}
|\ifchilddoc|\\
|\addtocounter{page}{-1}|\\
\textit{code for banner page}\\
|\newpage|\\
|\||fi|
\end{tabular}
\end{center}
%
Here one could write a message such as:
\begin{center}
|This is the part \childdocname{} of \childdocjob{}.|
\end{center}

%%%%%%%%%%%%%%%%%%%%%%%%%%%%%%%%%%%%%%%%%%%%%%%%%%%%%%%%%%%%%%%%%%%%%%%%%%%%%%%%
\subsection{Flags}
\label{sec:flags}

The package makes it easy to generate different versions
of the main or child documents.
To this end compilation flags can be defined
and assigned different default values.
They will be particularly useful in conjunction
with the forwarding mechanism described in \secref{sec:forward}.

For example, it may be useful to have a flag |\version|
which can be set to |draft| or |final|.
The document source will contain some conditional code
depending on the value of |\version|.
Suppose further, the flag should default to |final| for the main file
and to |draft| for child files
which is a natural assignment for editing the document.
This is achieved by placing the following code
in the preamble of the main document
(below the |\childdocmain| directive):
%
\begin{center}
\begin{tabular}{l}
|\ifchilddoc|\\
|\providecommand{\version}{draft}|\\
|\||else|\\
|\providecommand{\version}{final}|\\
|\||fi|
\end{tabular}
\end{center}
%
The definition by |\providecommand| makes sure
that previous definitions are not overwritten.
Further statements |\providecommand{\version}{...}|
can thus be added before the above code to override it.

For the main file, one might add a line
(between |\childdocmain| and the above block)
%
\begin{center}
|%\ifchilddoc\||else\providecommand{\version}{draft}\||fi|
\end{center}
%
which can be uncommented to produce a draft version.
Likewise one can add a line to the very top of a child file
(above the |\childdocof{|\textit{main}|}| directive)
%
\begin{center}
|%\providecommand{\version}{final}|
\end{center}
%
which can be uncommented to produce the final version of this child document.

%%%%%%%%%%%%%%%%%%%%%%%%%%%%%%%%%%%%%%%%%%%%%%%%%%%%%%%%%%%%%%%%%%%%%%%%%%%%%%%%
\subsection{Forwarding}
\label{sec:forward}

Different versions of the main or child documents
using compilation flags as described in \secref{sec:flags}
can be (permanently) stored in different files
for convenient compilation, viewing and distribution.
To this end, the package defines a command
to pass on compilation to a different file:

%%%%%%%%%%%%%%%%%%%%%%%%%%%%%%%%%%%%%%%%
\DescribeMacro{\childdocforward}
The command |\childdocforward| redirects processing to
another source file:
%
\begin{center}
\begin{tabular}{l}
|\input{childdoc.def}|\\
|\childdocforward[|\textit{main}|]{|\textit{dest}|}|\\
\end{tabular}
\end{center}
%
The argument \textit{dest} is the destination file
(without extension).
It should be the main file or one of the child files.
Note that further \textsf{childdoc} directives
such as |\childdocof| and |\childdocforward|
in the indicated file will be processed in this form.
The optional argument \textit{main}
passes on directly to the main file \textit{main}
while pretending to compile the child \textit{dest}.
This form behaves as if \textit{dest}
issues |\childdocof{|\textit{main}|}| right away,
and no further \textsf{childdoc} directives will be processed.

%%%%%%%%%%%%%%%%%%%%%%%%%%%%%%%%%%%%%%%%
\DescribeMacro{\...prefix}
In the alternative form |\childdocforwardprefix|,
%
\begin{center}
\begin{tabular}{l}
|\input{childdoc.def}|\\
|\childdocforwardprefix[|\textit{main}|]{|\textit{prefix}|}{|\textit{dest}|}|
\end{tabular}
\end{center}
%
the destination file is determined by a pattern
depending on the current file:
To make this work, the current file must be called
`{\textit{prefix}\hspace{0.2em}\textit{suffix}}'
with \textit{prefix} matching precisely the argument.
Processing is then passed on to the file
`{\textit{dest}\hspace{0.2em}\textit{suffix}}'.
Surely, the same effect is achieved by
directly specifying the
argument `{\textit{dest}\hspace{0.2em}\textit{suffix}}'
in the first form.
However, that requires to set up a different file
for each child. With the alternative form of the command
all these files can have exactly the same content
which simplifies setting them up and maintaining them.

For example, the following file |draft.tex|
with a compilation flag |\version| as described in \secref{sec:flags}
compiles the main document as a draft:
%
\begin{center}
\begin{tabular}{l}
|\def\version{draft}|\\
|\input{childdoc.def}|\\
|\childdocforward{|\textit{main}|}|
\end{tabular}
\end{center}
%
Likewise, the following files |final|\textit{nn}|.tex|
compile the final version of the child document
|child|\textit{nn}|.tex|:
%
\begin{center}
\begin{tabular}{l}
|\def\version{final}|\\
|\input{childdoc.def}|\\
|\childdocforwardprefix{final}{child}|
\end{tabular}
\end{center}
%

Note that when several versions of a main file and/or of each child file
are to be generated, it may be convenient to set up a |Makefile| or
shell script to automatise the process.

%%%%%%%%%%%%%%%%%%%%%%%%%%%%%%%%%%%%%%%%%%%%%%%%%%%%%%%%%%%%%%%%%%%%%%%%%%%%%%%%
\subsection{Command Line Processing}
\label{sec:commandline}

The effect of redirection files can also be achieved by invoking
the \LaTeX{} compiler with a more elaborate command line.
Most conveniently this should be done as part
of a shell script or a |Makefile|.

When using \textsf{childdoc} in the main file, the following
command lines effectively perform a redirection
(note that depending on the shell being used,
backslashes may have to be doubled: `|\|' $\to$ `|\\|'):
%
\begin{center}
|... -jobname "|\textit{target}|" |\\|"|[\textit{flags}]%
|\input{childdoc.def}\childdocforward[|\textit{main}|]{|\textit{dest}|}"|
\end{center}
%
Here \textit{target} is the name of the output file,
\textit{main} is the name of the main file
and \textit{dest} is the name of the main or child file to be processed
(all filenames without extensions).
The optional argument \textit{main} can be omitted
if \textit{main} matches \textit{dest}.
Optionally, compilation \textit{flags} can be defined via |\def| commands.
This command line makes the \TeX{} engine believe
it is compiling the file \textit{target}
whose content is specified as the latter parameter.
The provided code then forwards the processing to
\textit{main} or \textit{dest} as described in \secref{sec:forward}.

%%%%%%%%%%%%%%%%%%%%%%%%%%%%%%%%%%%%%%%%%%%%%%%%%%%%%%%%%%%%%%%%%%%%%%%%%%%%%%%%
\subsection{Include by Input}
\label{sec:input}

Including child documents by |\include| has some restrictions by design.
Most notably, the content of a child document always occupies
its own set of pages; pages cannot be shared between child documents.
Usually, this behaviour makes perfect sense
because each child document contain an essential part of the document.
However, in some situations it may be desirable to compose
a document from a collection of parts
without having mandatory page breaks between then.
For this case, the package
provides a mechanism to include parts
by |\input| which can also be processed individually.
However, by construction this mechanism
requires manual handling of the content to be output.

%%%%%%%%%%%%%%%%%%%%%%%%%%%%%%%%%%%%%%%%
\DescribeMacro{\ifchilddocmanual}
The main file should be prepared as usual, see \secref{sec:include}.
However, the document body must make a distinction
between processing of an individual part and of the main document, e.g.:
%
\begin{center}
\begin{tabular}{l}
|\ifchilddocmanual|\\
|\input{\childdocname}|\\
|\||else|\\
\textit{document body with }|\input{|\textit{part}|}|\\
|\||fi|
\end{tabular}
\end{center}
%
The conditional |\ifchilddocmanual| is true whenever
a part to be included by |\input| is being compiled,
and the name of the part is stored in |\childdocname|.

%%%%%%%%%%%%%%%%%%%%%%%%%%%%%%%%%%%%%%%%
\DescribeMacro{\childdocby}
Each part to be included by |\input| should start with:
%
\begin{center}
\begin{tabular}{l}
|\input{childdoc.def}|\\
|\childdocby{|\textit{main}|}|\\
\end{tabular}
\end{center}
%
The directive |\childdocby| is similar to |\childdocof|
described in \secref{sec:include},
but the subsequent selection of content must be done manually.
To that end, both |\ifchilddoc| and |\ifchilddocmanual|
will be true upon processing of a part,
and the name of the part is stored in |\childdocname|.
Note that |\jobname| will be set to the filename of the current part
so that each part receives an individual |.aux| file
that does not interfere with the |.aux| file(s) of the main document.
This behaviour can be altered by the alternative form
|\childdocby[*]{|\textit{main}|}| (with a non-empty optional argument)
which uses the |.aux| file of the main document
by setting |\jobname| to \textit{main}.

%%%%%%%%%%%%%%%%%%%%%%%%%%%%%%%%%%%%%%%%%%%%%%%%%%%%%%%%%%%%%%%%%%%%%%%%%%%%%%%%
\subsection{Driver Development}
\label{sec:driver}

The \textsf{childdoc} mechanism can also be use for the development
of definition files such as \LaTeX{} styles or classes.
This case differs from the above setup with multiple parts
included by |\include| in that no |\includeonly| should be invoked.
This can be achieved by starting the include file
(before |\ProvidesPackage|) with:
%
\begin{center}
\begin{tabular}{l}
|\input{childdoc.def}|\\
|\childdocforward{|\textit{main}|}|\\
\end{tabular}
\end{center}
%
or alternatively with:
%
\begin{center}
\begin{tabular}{l}
|\input{childdoc.def}|\\
|\childdocby{|\textit{main}|}|\\
\end{tabular}
\end{center}
%
Both forms have slightly different effects as described above.
The main file is prepared as usual, see \secref{sec:include}.

%%%%%%%%%%%%%%%%%%%%%%%%%%%%%%%%%%%%%%%%%%%%%%%%%%%%%%%%%%%%%%%%%%%%%%%%%%%%%%%%
\subsection{Legacy Detection}
\label{sec:detection}

The directive |\childdocmain| in the main file can detect
whether the complete document or merely a child is to be compiled
even without using the directive |\childdocof|.
This method is deprecated because it is less robust
and there is no compelling reason to use it;
it is merely provided for backward compatibility
and it may be removed in future versions.

If the detection mechanism is to be used,
it is mandatory to correctly specify
the filename of the main file as the argument of |\childdocmain|:
%
\begin{center}
\begin{tabular}{l}
|\input{childdoc.def}|\\
|\childdocmain{|\textit{main}|}|\\
\end{tabular}
\end{center}
%
If |\jobname| does not match the argument \textit{main} of |\childdocmain|,
it is assumed that |\jobname| points to the child file to be compiled.
When using |\childdocmain| with the main file specified as argument,
it suffices to start a child file
with just |\input{|\textit{main}|}|
without loading of the package and using |\childdocof|.
If instead all processing is done
with the appropriate \textsf{childdoc} directives,
the argument of \textit{main} of |\childdocmain| can be empty.

An alternative version of the command line processing described
in \secref{sec:commandline} using the detection mechanism reads:
%
\begin{center}
|... -jobname "|\textit{target}|" "|[\textit{flags}]%
[|\def\jobname{|\textit{dest}|}|]|\input{|\textit{main}|}"|
\end{center}

%%%%%%%%%%%%%%%%%%%%%%%%%%%%%%%%%%%%%%%%%%%%%%%%%%%%%%%%%%%%%%%%%%%%%%%%%%%%%%%%
\subsection{Manual Code}
\label{sec:manual}

In case one cannot be certain whether the definitions file |childdoc.def|
is installed on the target \TeX{} distribution
and one prefers not to ship it,
it is conceivable to paste a few relevant commands into the sources.

To that end, drop all statements |\input{childdoc.def}|
and perform the replacements as outlined below.
Instead of |\childdocmain{|\textit{main}|}| add the following code
to the top of the main file:
%
\begin{center}
\begin{tabular}{l}
|\||ifdefined\childdocname\endinput\||fi\newif\ifchilddoc|\\
|\edef\childdocname{\scantokens\expandafter{\jobname\noexpand}}|\\
|\def\childdocmain{|\textit{main}|}\||ifx\childdocmain\childdocname\||else|\\
|\childdoctrue\includeonly{\childdocname}\let\jobname\childdocmain\||fi|\\
\end{tabular}
\end{center}
%
Instead of |\childdocof{|\textit{main}|}| just include the main file
at the top of each child file:
%
\begin{center}
|\input{|\textit{main}|}|
\end{center}
%
A simple redirection |\childdocforward{|\textit{dest}|}| is achieved by:
%
\begin{center}
|\def\jobname{|\textit{dest}|}\input{\jobname}|
\end{center}
%
The redirection with prefix
|\childdocforwardprefix[|\textit{prefix}|]{|\textit{dest}|}|
is accomplished by:
%
\begin{center}
\begin{tabular}{l}
|{\edef\jobname{\scantokens\expandafter{\jobname\noexpand}}|\\
|\def\redirectjob |\textit{prefix}|#1~~~{\gdef\jobname{|\textit{dest}|#1}}|\\
|\expandafter\redirectjob\jobname~~~}\input{\jobname}|
\end{tabular}
\end{center}

In an alternative approach,
child documents can be compiled by a specific command line
without additional code or specific definitions:
%
\begin{center}
|... -jobname "|\textit{target}|" "|[\textit{flags}]%
|\includeonly{|\textit{dest}|}\input{|\textit{main}|}"|
\end{center}
%

%%%%%%%%%%%%%%%%%%%%%%%%%%%%%%%%%%%%%%%%%%%%%%%%%%%%%%%%%%%%%%%%%%%%%%%%%%%%%%%%
%%%%%%%%%%%%%%%%%%%%%%%%%%%%%%%%%%%%%%%%%%%%%%%%%%%%%%%%%%%%%%%%%%%%%%%%%%%%%%%%
\section{Information}

%%%%%%%%%%%%%%%%%%%%%%%%%%%%%%%%%%%%%%%%%%%%%%%%%%%%%%%%%%%%%%%%%%%%%%%%%%%%%%%%
\subsection{Copyright}

Copyright \copyright{} 2017--2018 Niklas Beisert

This work may be distributed and/or modified under the
conditions of the \LaTeX{} Project Public License, either version 1.3
of this license or (at your option) any later version.
The latest version of this license is in
  \url{http://www.latex-project.org/lppl.txt}
and version 1.3 or later is part of all distributions of \LaTeX{}
version 2005/12/01 or later.

This work has the LPPL maintenance status `maintained'.

The Current Maintainer of this work is Niklas Beisert.

This work consists of the files |README.txt|, |childdoc.ins| and |childdoc.dtx|
as well as the derived files |childdoc.def|, |cdocsamp.tex|
with |cdocsch1.tex|, |cdocsch2.tex|, |cdocspt3.tex|, |cdocspt4.tex|,
|cdocsdrf.tex|, |cdocsfn1.tex|, |cdocsfn2.tex|
as well as |childdoc.pdf|.

%%%%%%%%%%%%%%%%%%%%%%%%%%%%%%%%%%%%%%%%%%%%%%%%%%%%%%%%%%%%%%%%%%%%%%%%%%%%%%%%
\subsection{Files and Installation}

The package consists of the files:
%
\begin{center}
\begin{tabular}{ll}
    |README.txt|   & readme file \\
    |childdoc.ins| & installation file \\
    |childdoc.dtx| & source file \\
    |childdoc.def| & definition file \\
    |cdocsamp.tex| & sample main file \\
    |cdocsch1.tex| & sample include file \\
    |cdocsch2.tex| & sample include file \\
    |cdocspt3.tex| & sample part file \\
    |cdocspt4.tex| & sample part file \\
    |cdocsdrf.tex| & sample redirection file \\
    |cdocsfn1.tex| & sample redirection file \\
    |cdocsfn2.tex| & sample redirection file \\
    |childdoc.pdf| & manual
\end{tabular}
\end{center}
%
The distribution consists of the files
|README.txt|, |childdoc.ins| and |childdoc.dtx|.
%
\begin{itemize}
\item
Run (pdf)\LaTeX{} on |childdoc.dtx|
to compile the manual |childdoc.pdf| (this file).
\item
Run \LaTeX{} on |childdoc.ins| to create the definitions file |childdoc.def|
and the sample |cdocsamp.tex| with include files
|cdocsch1.tex|, |cdocsch2.tex|, |cdocspt3.tex|, |cdocspt4.tex|,
|cdocsdrf.tex|, |cdocsfn1.tex|, |cdocsfn2.tex|.
Then copy the file |childdoc.def| to an appropriate directory of your \LaTeX{}
distribution, e.g.\ \textit{texmf-root}|/tex/latex/childdoc|.
\end{itemize}

%%%%%%%%%%%%%%%%%%%%%%%%%%%%%%%%%%%%%%%%%%%%%%%%%%%%%%%%%%%%%%%%%%%%%%%%%%%%%%%%
\subsection{Related CTAN Packages}

There are several other packages which offer a similar functionality:
%
\begin{itemize}
\item
The packages
\href{http://ctan.org/pkg/docmute}{\textsf{docmute}},
\href{http://ctan.org/pkg/includex}{\textsf{includex}} and
\href{http://ctan.org/pkg/standalone}{\textsf{standalone}}
provide commands to include only the document body of
a child file thus allowing both files to be compiled individually.
\item
The packages \href{http://ctan.org/pkg/subdocs}{\textsf{subdocs}}
and \href{http://ctan.org/pkg/subfiles}{\textsf{subfiles}}
provide structures in which the main and child documents can be
encapsulated and allowing them to be compiled individually.
The inclusion mechanism is different from the conventional |\include|.
\item
The package \href{http://ctan.org/pkg/combine}{\textsf{combine}}
is an elaborate solution to combine several documents into one.
\end{itemize}
%
See also the CTAN topic \href{http://ctan.org/topic/subdocs}{\textsf{subdocs}}
for further related packages.
The present package differs from the above solutions in that
a document structure constructed with the conventional |\include| mechanism
just needs two extra commands at the top of every file
such that all constituent files can be compiled individually.

%%%%%%%%%%%%%%%%%%%%%%%%%%%%%%%%%%%%%%%%%%%%%%%%%%%%%%%%%%%%%%%%%%%%%%%%%%%%%%%%
%\subsection{Feature Suggestions}
%
%The following is a list of features which may be useful for future
%versions of this package:
%%
%\begin{itemize}
%\item
%\ldots
%\end{itemize}

%%%%%%%%%%%%%%%%%%%%%%%%%%%%%%%%%%%%%%%%%%%%%%%%%%%%%%%%%%%%%%%%%%%%%%%%%%%%%%%%
\subsection{Revision History}

%%%%%%%%%%%%%%%%%%%%%%%%%%%%%%%%%%%%%%%%
\paragraph{v2.0:} 2018/12/30

\begin{itemize}
\item
immediate forward processing
\item
added |\childdocby| mechanism
\item
manual restructured
\end{itemize}

%%%%%%%%%%%%%%%%%%%%%%%%%%%%%%%%%%%%%%%%
\paragraph{v1.6:} 2018/01/17

\begin{itemize}
\item
application for development of include files
\item
corrections to manual
\end{itemize}

%%%%%%%%%%%%%%%%%%%%%%%%%%%%%%%%%%%%%%%%
\paragraph{v1.5:} 2017/05/21

\begin{itemize}
\item
more complete structuring introduced
\item
|\childdocof| introduced
\item
|\childdoc| renamed to |\childdocmain|
\item
|\childredirect| renamed to |\childdocforward| and |\childdocforwardprefix|
and functionality expanded
\end{itemize}

%%%%%%%%%%%%%%%%%%%%%%%%%%%%%%%%%%%%%%%%
\paragraph{v1.0:} 2017/04/27

\begin{itemize}
\item
manual and install package
\item
first version published on CTAN
\end{itemize}

%%%%%%%%%%%%%%%%%%%%%%%%%%%%%%%%%%%%%%%%
\paragraph{v0.6:} 2017/04/26

\begin{itemize}
\item
redirection mechanism added
\end{itemize}

%%%%%%%%%%%%%%%%%%%%%%%%%%%%%%%%%%%%%%%%
\paragraph{v0.5:} 2017/04/26

\begin{itemize}
\item
functionality in definition file
\end{itemize}


%%%%%%%%%%%%%%%%%%%%%%%%%%%%%%%%%%%%%%%%%%%%%%%%%%%%%%%%%%%%%%%%%%%%%%%%%%%%%%%%
%%%%%%%%%%%%%%%%%%%%%%%%%%%%%%%%%%%%%%%%%%%%%%%%%%%%%%%%%%%%%%%%%%%%%%%%%%%%%%%%
%%%%%%%%%%%%%%%%%%%%%%%%%%%%%%%%%%%%%%%%%%%%%%%%%%%%%%%%%%%%%%%%%%%%%%%%%%%%%%%%
\appendix

\settowidth\MacroIndent{\rmfamily\scriptsize 000\ }

 \DocInput{childdoc.dtx}

\end{document}
%</driver>
% \fi
%
% %%%%%%%%%%%%%%%%%%%%%%%%%%%%%%%%%%%%%%%%%%%%%%%%%%%%%%%%%%%%%%%%%%%%%%%%%%%%%%
% %%%%%%%%%%%%%%%%%%%%%%%%%%%%%%%%%%%%%%%%%%%%%%%%%%%%%%%%%%%%%%%%%%%%%%%%%%%%%%
% \section{Sample}
%\iffalse
%<*samplemain>
%\fi
%
% The following presents a sample document
% with two chapters, two parts, a title page,
% a compile flag as well as three forwarding files to set the flag.
% It consists of eight |.tex| files:
% \begin{center}
% \begin{tabular}{ll}
% |cdocsamp.tex|&main file\\
% |cdocsch1.tex|&include file for chapter 1\\
% |cdocsch2.tex|&include file for chapter 2\\
% |cdocspt3.tex|&include file for part 3\\
% |cdocspt4.tex|&include file for part 4\\
% |cdocsdrf.tex|&forwarding file for main file in draft mode\\
% |cdocsfi1.tex|&forwarding file for final version of chapter 1\\
% |cdocsfi2.tex|&forwarding file for final version of chapter 2\\
% \end{tabular}
% \end{center}
% Each of the eight files can be compiled directly by the \LaTeX{} compiler.
%
% %%%%%%%%%%%%%%%%%%%%%%%%%%%%%%%%%%%%%%
% \paragraph{Main File.}
%
% The main file is called |cdocsamp.tex|.
%
% Load the \textsf{childdoc} definitions and
% declare the filename for the main document:
%    \begin{macrocode}
\input{childdoc.def}
\childdocmain{}
%    \end{macrocode}

% Optional override for |\version| flag:
%    \begin{macrocode}
%%\ifchilddoc\else\providecommand{\version}{draft}\fi
%    \end{macrocode}

% Define the default values for the |\version| flag
% (|final| for the main file and |draft| for childs):
%    \begin{macrocode}
\ifchilddoc
\providecommand{\version}{draft}
\else
\providecommand{\version}{final}
\fi
%    \end{macrocode}

% Load the standard document class:
%    \begin{macrocode}
\documentclass[12pt]{article}
%    \end{macrocode}

% Start the document body:
%    \begin{macrocode}
\begin{document}
%    \end{macrocode}

% Declare a title page.
% Print title, part of document being processed and version flag:
%    \begin{macrocode}
\addtocounter{page}{-1}
\begin{center}
{\LARGE\bfseries{}childdoc example\par}
\vspace{1cm}
\ifchilddoc
\ifchilddocmanual part\else chapter\fi:
`\childdocname' of `\childdocjob'\par
\else
main document: `\childdocjob'\par
\fi
version: \version\par
\end{center}
\newpage
%    \end{macrocode}

% Manually include selected file,
% otherwise process as usual:
%    \begin{macrocode}
\ifchilddocmanual
\section*{part `\childdocname'}
\input{\childdocname}
\else
%    \end{macrocode}

% Include the two chapters:
%    \begin{macrocode}
\include{cdocsch1}
\include{cdocsch2}
%    \end{macrocode}

% Include the two parts unless only chapters should be displayed:
%    \begin{macrocode}
\ifchilddoc\else
\section{part three}
\input{cdocspt3}
\section{part four}
\input{cdocspt4}
\fi
%    \end{macrocode}

% Process as usual until here:
%    \begin{macrocode}
\fi
%    \end{macrocode}

% End of document body:
%    \begin{macrocode}
\end{document}
%    \end{macrocode}
%\iffalse
%</samplemain>
%\fi
%
% %%%%%%%%%%%%%%%%%%%%%%%%%%%%%%%%%%%%%%
% \paragraph{Chapter Include Files.}
%
% The include files are called |cdocsch1.tex| and |cdocsch2.tex|.
%
%\iffalse
%<*samplechap1|samplechap2>
%\fi

% Optional override for |\version| flag:
%    \begin{macrocode}
%%\providecommand{\version}{final}
%    \end{macrocode}

% Include the main document:
%    \begin{macrocode}
\input{childdoc.def}
\childdocof{cdocsamp}
%    \end{macrocode}

%\iffalse
%</samplechap1|samplechap2>
%\fi
%
%\iffalse
%<*samplechap1>
%\fi
% Some text for chapter 1:
%    \begin{macrocode}
\section{one}
some text in chapter one
%    \end{macrocode}

%\iffalse
%</samplechap1>
%\fi
% Some text for chapter 2:
%\iffalse
%<*samplechap2>
%\fi
%    \begin{macrocode}
\section{two}
more text in chapter two
%    \end{macrocode}

%\iffalse
%</samplechap2>
%\fi
%
% %%%%%%%%%%%%%%%%%%%%%%%%%%%%%%%%%%%%%%
% \paragraph{Part Include Files.}
%
% The include files are called |cdocspt3.tex| and |cdocspt4.tex|.
%
%\iffalse
%<*samplepart3|samplepart4>
%\fi

% Optional override for |\version| flag:
%    \begin{macrocode}
%%\providecommand{\version}{final}
%    \end{macrocode}

% Include the main document:
%    \begin{macrocode}
\input{childdoc.def}
\childdocby{cdocsamp}
%    \end{macrocode}

%\iffalse
%</samplepart3|samplepart4>
%\fi
%
%\iffalse
%<*samplepart3>
%\fi
% Some text for part 3:
%    \begin{macrocode}
some text in part three
%    \end{macrocode}

%\iffalse
%</samplepart3>
%\fi
% Some text for part 4:
%\iffalse
%<*samplepart4>
%\fi
%    \begin{macrocode}
more text in part four
%    \end{macrocode}

%\iffalse
%</samplepart4>
%\fi
%
% %%%%%%%%%%%%%%%%%%%%%%%%%%%%%%%%%%%%%%
% \paragraph{Forwarding for a Complete Draft.}
%
% The following forwarding file |cdocsdrf.tex|
% compiles the main document in draft mode:
%\iffalse
%<*sampledraft>
%\fi
%    \begin{macrocode}
\def\version{draft}
\input{childdoc.def}
\childdocforward{cdocsamp}
%    \end{macrocode}

%\iffalse
%</sampledraft>
%\fi
%
% %%%%%%%%%%%%%%%%%%%%%%%%%%%%%%%%%%%%%%
% \paragraph{Forwarding for Final Version of the Chapters.}
%
% The following forwarding files |cdocsfn1.tex| and |cdocsfn2.tex|
% (with identical content)
% compile the final versions of the child documents
% |cdocsch1.tex| and |cdocsch2.tex|, respectively:
%\iffalse
%<*samplefinal>
%\fi
%    \begin{macrocode}
\def\version{final}
\input{childdoc.def}
\childdocforwardprefix[cdocsamp]{cdocsfn}{cdocsch}
%    \end{macrocode}

%\iffalse
%</samplefinal>
%\fi
%
% %%%%%%%%%%%%%%%%%%%%%%%%%%%%%%%%%%%%%%
% \paragraph{Command Line Processing.}
%
% The following three command lines generate the output files
% |cdocscld|, |cdocscl1| and |cdocscl2|
% which should be identical to
% |cdocsdrf|, |cdocsch1| and |cdocsfn2|, respectively:
% \begin{center}
% \begin{tabular}{l}
% |latex -jobname cdocscld \|\\
% |  "\def\version{draft}\input{childdoc.def}\childdocforward{cdocsamp}"|\\
% |latex -jobname cdocscl1 \|\\
% |  "\input{childdoc.def}\childdocforward[cdocsamp]{cdocsch1}"|\\
% |latex -jobname cdocscl2 \|\\
% |  "\def\version{final}\input{childdoc.def}\childdocforward{cdocsch2}"|
% \end{tabular}
% \end{center}
% Note that the trailing backslash on each first line
% merely continues the input to the second line
% (for convenient cut ant paste).
% Furthermore, the command |latex| can be replaced by any
% of its alternative versions such as |pdflatex|.
%
% %%%%%%%%%%%%%%%%%%%%%%%%%%%%%%%%%%%%%%%%%%%%%%%%%%%%%%%%%%%%%%%%%%%%%%%%%%%%%%
% %%%%%%%%%%%%%%%%%%%%%%%%%%%%%%%%%%%%%%%%%%%%%%%%%%%%%%%%%%%%%%%%%%%%%%%%%%%%%%
% \section{Implementation}
%\iffalse
%<*package>
%\fi
%
% This section describes the definitions file |childdoc.def|.

% The definitions cannot be loaded using |\usepackage| or |\RequirePackage|
% which has a mechanism to prevent loading a style file more than once.
% When loading the definitions by means of |\input|
% multiple instances have to be prevented manually:
%\iffalse
%This code needs to be before the `\ProvidesFile' directive
%which is defined at the beginning of this file.
%Therefore it is also placed there and commented out here.
%</package>
%<*discard>
%\fi
%    \begin{macrocode}
\ifdefined\childdocmain\endinput\fi
%    \end{macrocode}
%\iffalse
%</discard>
%<*package>
%\fi
%
% \macro{\ifchilddoc}
% \macro{\ifchilddocmanual}
% The conditional |\ifchilddoc| tells whether a
% child (true) or main (false) document is being compiled.
% The conditional |\ifchilddocmanual| tells whether
% the |\includeonly| mechanism is used (false) or
% the selection of child files must be performed manually (true).
% The definitions initialise to false:
%    \begin{macrocode}
\newif\ifchilddoc
\newif\ifchilddocmanual
%    \end{macrocode}

% \macro{\childdocname}
% \macro{\childdocjob}
% The macro |\childdocname| stores the name of the main document
% to be compiled. The macro |\childdocjob| stores the name of
% the document on which the \LaTeX{} compiler was originally invoked.
% The content of |\jobname| cannot be compared
% to filenames specified in the source due to different catcodes.
% The following code rescans |\jobname|, stores the result
% in |\childdocname| and saves a copy in |\childdocjob|:
%    \begin{macrocode}
\edef\childdocname{\scantokens\expandafter{\jobname\noexpand}}
\let\childdocjob\childdocname
%    \end{macrocode}

% \macro{\childdocdisable}
% The macro |\childdocdisable| prevents the main file
% from being processed more than once.
% At this stage, the main document command |\childdocmain|
% is assumed to be called once again where it should do nothing.
% Any subsequent call to it should prevent
% a secondary processing of the main document
% It overwrites the forwarding commands
% |\childdocof| and |\childdocforward|
% with empty macros to prevent further inclusions of the main document:
%    \begin{macrocode}
\newcommand{\childdocdisable}
{
  \renewcommand{\childdocmain}[1]{\renewcommand{\childdocmain}[1]{\endinput}}
  \renewcommand{\childdocof}[1]{}
  \renewcommand{\childdocby}[2][]{}
  \renewcommand{\childdocforward}[2][]{}
  \renewcommand{\childdocdisable}{}
}
%    \end{macrocode}

% \macro{\childdocmain}
% The macro |\childdocmain| is to be called at the top of the main file
% with nothing or the main filename (without extension) as argument.
% First, it breaks loops.
% If the argument is not empty and does not match |\childdocname|
% (which is set by the first inclusion of |childdoc.def|),
% |\ifchilddoc| is set to true, |\includeonly| is applied to the child file
% and |\jobname| is set to the main file
% (for proper handling of |.aux| files):
%    \begin{macrocode}
\newcommand{\childdocmain}[1]
{
  \childdocdisable\childdocmain{}
  \if?#1?\else
    \begingroup
      \def\childdoctmp{#1}
      \ifx\childdoctmp\childdocname
        \def\childdoctmp{}
      \else
        \def\childdoctmp
        {
          \childdoctrue
          \includeonly{\childdocname}
          \def\childdocjob{#1}
          \def\jobname{#1}
        }
      \fi
      \expandafter
    \endgroup
    \childdoctmp
  \fi
}
%    \end{macrocode}

% \macro{\childdocof}
% The command |\childdocof| redirects
% compilation to the main file |#1|.
%    \begin{macrocode}
\newcommand{\childdocof}[1]
{
  \childdocdisable
  \childdoctrue
  \includeonly{\childdocname}
  \def\jobname{#1}
  \def\childdocjob{#1}
  \input{#1}
}
%    \end{macrocode}

% \macro{\childdocby}
% The command |\childdocby| ....
%    \begin{macrocode}
\newcommand{\childdocby}[2][]
{
  \childdocdisable
  \childdoctrue
  \childdocmanualtrue
  \if?#1?\else
    \def\jobname{#2}
  \fi
  \def\childdocjob{#2}
  \input{#2}
  \endinput
}
%    \end{macrocode}

% \macro{\childdocforward}
% The command |\childdocforward| redirects
% compilation to the main file or
% (if the optional argument is given) a child file.
% Parameters are set as if the main file
% or a child file starting with |\childdocof| was compiled.
% Then compilation is handed over to the main file:
%    \begin{macrocode}
\newcommand{\childdocforward}[2][]
{
  \begingroup
    \if?#1?
      \def\childdoctmp
      {
        \def\childdocname{#2}
        \def\childdocjob{#2}
        \def\jobname{#2}
        \input{#2}
        \endinput
      }
    \else
      \def\childdoctmp
      {
        \childdocdisable
        \def\childdocname{#2}
        \childdoctrue
        \includeonly{#2}
        \def\childdocjob{#1}
        \def\jobname{#1}
        \input{#1}
        \endinput
      }
    \fi
    \expandafter
  \endgroup
  \childdoctmp
}
%    \end{macrocode}

% \macro{\childdocforwardprefix}
% The command |\childdocforwardprefix| redirects
% compilation to the main or a child file by means of a pattern.
% The prefix |#1| in the current filename is replaced by |#2|
% and the suffix of the current filename is kept
% (it is assumed that the filename does not contain the substring `|~~~|'
% which is used as a delimiter).
% Compilation is handed over to the new file by |\childdocforward|:
%    \begin{macrocode}
\newcommand{\childdocforwardprefix}[3][]
{
  \begingroup
    \def\childdocextract #2##1~~~{\def\childdoctmp{\childdocforward[#1]{#3##1}}}
    \expandafter\childdocextract\childdocname~~~
    \expandafter
  \endgroup
  \childdoctmp
}
%    \end{macrocode}

% \macro{\childdoc}
% The deprecated macro |\childdoc| is a legacy version of |\childdocmain|:
%    \begin{macrocode}
\newcommand{\childdoc}{\childdocmain}
%    \end{macrocode}

% \macro{\childdocredirect}
% The deprecated macro |\childdocredirect| is a legacy version
% of |\childdocforward| and |\childdocforwardprefix|:
%    \begin{macrocode}
\newcommand{\childdocredirect}[2][]
{
  \begingroup
    \if?#1?
      \def\childdoctmp{\childdocforward{#2}}
    \else
      \def\childdoctmp{\childdocforwardprefix{#1}{#2}}
    \fi
    \expandafter
  \endgroup
  \childdoctmp
}
%    \end{macrocode}

%\iffalse
%</package>
%\fi
%
\endinput

\childdocmain{}
%    \end{macrocode}

% Optional override for |\version| flag:
%    \begin{macrocode}
%%\ifchilddoc\else\providecommand{\version}{draft}\fi
%    \end{macrocode}

% Define the default values for the |\version| flag
% (|final| for the main file and |draft| for childs):
%    \begin{macrocode}
\ifchilddoc
\providecommand{\version}{draft}
\else
\providecommand{\version}{final}
\fi
%    \end{macrocode}

% Load the standard document class:
%    \begin{macrocode}
\documentclass[12pt]{article}
%    \end{macrocode}

% Start the document body:
%    \begin{macrocode}
\begin{document}
%    \end{macrocode}

% Declare a title page.
% Print title, part of document being processed and version flag:
%    \begin{macrocode}
\addtocounter{page}{-1}
\begin{center}
{\LARGE\bfseries{}childdoc example\par}
\vspace{1cm}
\ifchilddoc
\ifchilddocmanual part\else chapter\fi:
`\childdocname' of `\childdocjob'\par
\else
main document: `\childdocjob'\par
\fi
version: \version\par
\end{center}
\newpage
%    \end{macrocode}

% Manually include selected file,
% otherwise process as usual:
%    \begin{macrocode}
\ifchilddocmanual
\section*{part `\childdocname'}
\input{\childdocname}
\else
%    \end{macrocode}

% Include the two chapters:
%    \begin{macrocode}
\include{cdocsch1}
\include{cdocsch2}
%    \end{macrocode}

% Include the two parts unless only chapters should be displayed:
%    \begin{macrocode}
\ifchilddoc\else
\section{part three}
\input{cdocspt3}
\section{part four}
\input{cdocspt4}
\fi
%    \end{macrocode}

% Process as usual until here:
%    \begin{macrocode}
\fi
%    \end{macrocode}

% End of document body:
%    \begin{macrocode}
\end{document}
%    \end{macrocode}
%\iffalse
%</samplemain>
%\fi
%
% %%%%%%%%%%%%%%%%%%%%%%%%%%%%%%%%%%%%%%
% \paragraph{Chapter Include Files.}
%
% The include files are called |cdocsch1.tex| and |cdocsch2.tex|.
%
%\iffalse
%<*samplechap1|samplechap2>
%\fi

% Optional override for |\version| flag:
%    \begin{macrocode}
%%\providecommand{\version}{final}
%    \end{macrocode}

% Include the main document:
%    \begin{macrocode}
% \iffalse
%
% childdoc.dtx Copyright (C) 2017-2018 Niklas Beisert
%
% This work may be distributed and/or modified under the
% conditions of the LaTeX Project Public License, either version 1.3
% of this license or (at your option) any later version.
% The latest version of this license is in
%   http://www.latex-project.org/lppl.txt
% and version 1.3 or later is part of all distributions of LaTeX
% version 2005/12/01 or later.
%
% This work has the LPPL maintenance status `maintained'.
%
% The Current Maintainer of this work is Niklas Beisert.
%
% This work consists of the files childdoc.dtx and childdoc.ins
% and the derived files childdoc.def and cdocsamp.tex with
% cdocsch1.tex, cdocsch2.tex, cdocsdrf.tex, cdocsfn1.tex, cdocsfn2.tex.
%
%<package>\ifdefined\childdocmain\endinput\fi
%<package>\ProvidesFile{childdoc.def}[2018/12/30 v2.0 child document driver]
%<samplemain>\ProvidesFile{cdocsamp.tex}[2018/12/30 v2.0 sample for childdoc]
%<*driver>
%\ProvidesFile{childdoc.drv}[2018/12/30 v2.0 childdoc reference manual file]
\PassOptionsToClass{10pt,a4paper}{article}
\documentclass{ltxdoc}

\usepackage[margin=35mm]{geometry}
\usepackage{hyperref}
\usepackage{hyperxmp}
\usepackage[usenames]{color}

\hypersetup{colorlinks=true}
\hypersetup{pdfstartview=FitH}
\hypersetup{pdfpagemode=UseNone}
\hypersetup{pdfsource={}}
\hypersetup{pdflang={en-UK}}
\hypersetup{pdfcopyright={Copyright 2017-2018 Niklas Beisert.
  This work may be distributed and/or modified under the
  conditions of the LaTeX Project Public License, either version 1.3
  of this license or (at your option) any later version.}}
\hypersetup{pdflicenseurl={http://www.latex-project.org/lppl.txt}}
\hypersetup{pdfcontactaddress={ETH Zurich, ITP, HIT K,
  Wolfgang-Pauli-Strasse 27}}
\hypersetup{pdfcontactpostcode={8093}}
\hypersetup{pdfcontactcity={Zurich}}
\hypersetup{pdfcontactcountry={Switzerland}}
\hypersetup{pdfcontactemail={nbeisert@itp.phys.ethz.ch}}
\hypersetup{pdfcontacturl={http://people.phys.ethz.ch/\xmptilde nbeisert/}}

\newcommand{\secref}[1]{\hyperref[#1]{section \ref*{#1}}}

\parskip1ex
\parindent0pt
\let\olditemize\itemize
\def\itemize{\olditemize\parskip0pt}

\begin{document}

\title{The \textsf{childdoc} Package}
\hypersetup{pdftitle={The childdoc Package}}
\author{Niklas Beisert\\[2ex]
  Institut f\"ur Theoretische Physik\\
  Eidgen\"ossische Technische Hochschule Z\"urich\\
  Wolfgang-Pauli-Strasse 27, 8093 Z\"urich, Switzerland\\[1ex]
  \href{mailto:nbeisert@itp.phys.ethz.ch}
  {\texttt{nbeisert@itp.phys.ethz.ch}}}
\hypersetup{pdfauthor={Niklas Beisert}}
\hypersetup{pdfsubject={Manual for the LaTeX2e Package childdoc}}
\date{30 December 2018, \textsf{v2.0}}
\maketitle

\begin{abstract}\noindent
\textsf{childdoc} is a \LaTeXe{} package
that enables the direct compilation
of document sections included by |\include|
to individual files.
\end{abstract}

\begingroup
\parskip0ex
\tableofcontents
\endgroup

%%%%%%%%%%%%%%%%%%%%%%%%%%%%%%%%%%%%%%%%%%%%%%%%%%%%%%%%%%%%%%%%%%%%%%%%%%%%%%%%
%%%%%%%%%%%%%%%%%%%%%%%%%%%%%%%%%%%%%%%%%%%%%%%%%%%%%%%%%%%%%%%%%%%%%%%%%%%%%%%%
\section{Introduction}

\LaTeX{} provides a mechanism to structure a large document (such as a book)
into a main file and several child files (containing the chapters)
using the |\include| command.
This mechanism is beneficial for documents
which span hundreds of pages in order to
make the source file(s) more manageable.
Moreover, compilation can be restricted to
selected child files by means of the |\includeonly| command.
The latter feature can be used to reduce the compilation time while editing
(this was significantly more useful in the earlier days of \LaTeX{})
or to generate a smaller document which is easier to navigate.
Another application of |\includeonly| is to generate
documents consisting of selected parts of the complete document.

However, there are a few drawbacks of the plain |\include| mechanism:
\begin{itemize}
\item
The child files cannot be compiled on their own,
they can only be compiled via the main file.
A naive editing environment
(such as a text editor with an option
to have the current file processed by \LaTeX)
may require one to switch to the main file before compiling;
attempting to compile the child file produces errors.
\item
The main file must be modified (each time)
to adjust the |\includeonly| command
to the present needs. This easily leaves the main file in a messy state.
\item
The generated document will always carry the filename
of the main document. This is inconvenient if
several child files are to be compiled and
to be kept for distribution.
\end{itemize}

The present package provides a simple interface
to make child files individually compilable by \LaTeX{}.
Compiling a child file then has the same effect as compiling
the main file with an |\includeonly| command
to select the appropriate child.
Moreover the generated document will carry the name of the child
rather than the main file.
This resolves all three above issues.

This feature is meant to make the editing of books,
thesis documents and lecture notes somewhat more convenient.
However, the package can also be used efficiently for
composing a series of documents (such as exercise sheets)
which are typically distributed individually.
It then assists the author in generating the individual documents
(potentially in different versions)
as well as a document containing the collected series.
Another application is in developing style files
or other kinds of included material
where compilation of the style file could redirect
to a sample or test file.

%%%%%%%%%%%%%%%%%%%%%%%%%%%%%%%%%%%%%%%%%%%%%%%%%%%%%%%%%%%%%%%%%%%%%%%%%%%%%%%%
%%%%%%%%%%%%%%%%%%%%%%%%%%%%%%%%%%%%%%%%%%%%%%%%%%%%%%%%%%%%%%%%%%%%%%%%%%%%%%%%
\section{Usage}

First of all, the package \textsf{childdoc} is \emph{not} a standard
\LaTeXe{} |.sty| style file! Therefore it needs to be invoked in
a non-standard way.

%%%%%%%%%%%%%%%%%%%%%%%%%%%%%%%%%%%%%%%%%%%%%%%%%%%%%%%%%%%%%%%%%%%%%%%%%%%%%%%%
\subsection{Included Files}
\label{sec:include}

%%%%%%%%%%%%%%%%%%%%%%%%%%%%%%%%%%%%%%%%
\DescribeMacro{\childdocmain}
To use the package, add the commands
\begin{center}
\begin{tabular}{l}
|\input{childdoc.def}|\\
|\childdocmain{}|\\
\end{tabular}
\end{center}
at the very top of the main \LaTeX{} file,
in particular \emph{before} the |\documentclass| statement!
The argument of |\childdocmain| should be left empty
(but it must be present).

%%%%%%%%%%%%%%%%%%%%%%%%%%%%%%%%%%%%%%%%
\DescribeMacro{\childdocof}
Furthermore, add the commands
\begin{center}
\begin{tabular}{l}
|\input{childdoc.def}|\\
|\childdocof{|\textit{main}|}|\\
\end{tabular}
\end{center}
at the top of every child file \textit{child}
which is included by |\include{|\textit{child}|}|
from within the main file
(or at least for those files to be compiled individually).
The argument \textit{main} must be the filename of the main file.

There are a couple of
considerations in setting up the main and child documents:

%%%%%%%%%%%%%%%%%%%%%%%%%%%%%%%%%%%%%%%%
\paragraph{Restrictions.}

Please note the following restrictions:
\begin{itemize}
\item
|\childdocmain| must be called with one argument \textit{main}
to ensure compatibility with earlier version of the package.
It must either be empty (|\childdocmain{}|)
or precisely match the filename of the main file in which it is specified.
See \secref{sec:detection} for further information.
\item
The filename \textit{main} must be specified without the |.tex| extension.
\item
The filename \textit{main} is case sensitive
(even in case-insensitive file systems)
due to internal string comparison.
\item
The argument \textit{main} should be fully expanded, it cannot be a macro.
\item
Subdirectories and special characters should be avoided in filenames.
\item
The command |\childdocmain{|\textit{main}|}| must be followed by a whitespace.
It should not be followed immediately by another command
or by a comment mark `|%|'.
This is because the \TeX{} parser reads the token immediately following
the argument of |\childdocmain| and puts it
at the beginning of every child section;
however, a white\-space is ignored.
\end{itemize}

%%%%%%%%%%%%%%%%%%%%%%%%%%%%%%%%%%%%%%%%
\paragraph{Content of Main File.}

It is advisable to place all content in the child files included by |\include|.
Any output contained in the main file will appear in all child documents
unless suppressed manually;
it cannot be suppressed automatically by the |\includeonly| directive
and thus should normally be avoided.
A method to include some content in the main file
by means of conditional processing is described in \secref{sec:conditional}.

%%%%%%%%%%%%%%%%%%%%%%%%%%%%%%%%%%%%%%%%
\paragraph{Page Numbering.}

When only a part of the document is compiled,
the appropriate numbering of pages
(as well as other status parameters)
is determined from the |.aux| files.
The latter contain information from previous passes.
However this information needs to propagate through
all intermediate child documents.
Therefore the page numbering in child documents may well
be inconsistent until the complete document is compiled at least once.

A useful (if unconventional) way to always ensure a consistent
page numbering is to restart the numbering in each child document
and denote the pages by `\textit{child}|.|\textit{page}'
where \textit{child} represents the chapter/section number of the child file.
This can be achieved by the command
|\numberwithin{page}{|\textit{child}|}|
of the \textsf{amsmath} package
where \textit{child} can be |chapter| or |section|
depending on the chosen structuring.
Alternatively, one can modify the macro |\thepage| appropriately
and reset the counter |page| at the start of each child file.

%%%%%%%%%%%%%%%%%%%%%%%%%%%%%%%%%%%%%%%%%%%%%%%%%%%%%%%%%%%%%%%%%%%%%%%%%%%%%%%%
\subsection{Conditional Processing}
\label{sec:conditional}

The package provides a mechanism to compile different versions
of a document. To customise the versions further some conditional processing
can come in handy to distinguish which version is being compiled.
The package provides two macros to describe the compilation context:

%%%%%%%%%%%%%%%%%%%%%%%%%%%%%%%%%%%%%%%%
\DescribeMacro{\ifchilddoc}
The conditional |\ifchilddoc| distinguishes between the compilation of
child documents and the main document:
%
\begin{center}
|\ifchilddoc |\textit{child-code}| |[|\||else |\textit{main-code}]| \||fi|
\end{center}

%%%%%%%%%%%%%%%%%%%%%%%%%%%%%%%%%%%%%%%%
\DescribeMacro{\childdocname}
\DescribeMacro{\childdocjob}
The macro |\childdocname| contains the filename (without extension)
of the main or child file being processed.
Note that |\childdocjob| will always contain the name of the main file.

%%%%%%%%%%%%%%%%%%%%%%%%%%%%%%%%%%%%%%%%
\paragraph{Title Page.}

Conditional processing can be used to include a title or banner page
in the main document when proper precautions are taken.
Importantly, the code in the main file should ensure that the page counter
(as well as other status parameters which are stored in the |.aux| files)
takes the same value after the conditional processing.
Otherwise the page numbers may take divergent values
depending on which part is compiled.

For example, a title page could be declared by:
%
\begin{center}
\begin{tabular}{l}
|\ifchilddoc\||else|\\
|\addtocounter{page}{-1}|\\
\textit{code for title page}\\
|\newpage|\\
|\||fi|
\end{tabular}
\end{center}
%
A banner page for the child documents can be generated by:
%
\begin{center}
\begin{tabular}{l}
|\ifchilddoc|\\
|\addtocounter{page}{-1}|\\
\textit{code for banner page}\\
|\newpage|\\
|\||fi|
\end{tabular}
\end{center}
%
Here one could write a message such as:
\begin{center}
|This is the part \childdocname{} of \childdocjob{}.|
\end{center}

%%%%%%%%%%%%%%%%%%%%%%%%%%%%%%%%%%%%%%%%%%%%%%%%%%%%%%%%%%%%%%%%%%%%%%%%%%%%%%%%
\subsection{Flags}
\label{sec:flags}

The package makes it easy to generate different versions
of the main or child documents.
To this end compilation flags can be defined
and assigned different default values.
They will be particularly useful in conjunction
with the forwarding mechanism described in \secref{sec:forward}.

For example, it may be useful to have a flag |\version|
which can be set to |draft| or |final|.
The document source will contain some conditional code
depending on the value of |\version|.
Suppose further, the flag should default to |final| for the main file
and to |draft| for child files
which is a natural assignment for editing the document.
This is achieved by placing the following code
in the preamble of the main document
(below the |\childdocmain| directive):
%
\begin{center}
\begin{tabular}{l}
|\ifchilddoc|\\
|\providecommand{\version}{draft}|\\
|\||else|\\
|\providecommand{\version}{final}|\\
|\||fi|
\end{tabular}
\end{center}
%
The definition by |\providecommand| makes sure
that previous definitions are not overwritten.
Further statements |\providecommand{\version}{...}|
can thus be added before the above code to override it.

For the main file, one might add a line
(between |\childdocmain| and the above block)
%
\begin{center}
|%\ifchilddoc\||else\providecommand{\version}{draft}\||fi|
\end{center}
%
which can be uncommented to produce a draft version.
Likewise one can add a line to the very top of a child file
(above the |\childdocof{|\textit{main}|}| directive)
%
\begin{center}
|%\providecommand{\version}{final}|
\end{center}
%
which can be uncommented to produce the final version of this child document.

%%%%%%%%%%%%%%%%%%%%%%%%%%%%%%%%%%%%%%%%%%%%%%%%%%%%%%%%%%%%%%%%%%%%%%%%%%%%%%%%
\subsection{Forwarding}
\label{sec:forward}

Different versions of the main or child documents
using compilation flags as described in \secref{sec:flags}
can be (permanently) stored in different files
for convenient compilation, viewing and distribution.
To this end, the package defines a command
to pass on compilation to a different file:

%%%%%%%%%%%%%%%%%%%%%%%%%%%%%%%%%%%%%%%%
\DescribeMacro{\childdocforward}
The command |\childdocforward| redirects processing to
another source file:
%
\begin{center}
\begin{tabular}{l}
|\input{childdoc.def}|\\
|\childdocforward[|\textit{main}|]{|\textit{dest}|}|\\
\end{tabular}
\end{center}
%
The argument \textit{dest} is the destination file
(without extension).
It should be the main file or one of the child files.
Note that further \textsf{childdoc} directives
such as |\childdocof| and |\childdocforward|
in the indicated file will be processed in this form.
The optional argument \textit{main}
passes on directly to the main file \textit{main}
while pretending to compile the child \textit{dest}.
This form behaves as if \textit{dest}
issues |\childdocof{|\textit{main}|}| right away,
and no further \textsf{childdoc} directives will be processed.

%%%%%%%%%%%%%%%%%%%%%%%%%%%%%%%%%%%%%%%%
\DescribeMacro{\...prefix}
In the alternative form |\childdocforwardprefix|,
%
\begin{center}
\begin{tabular}{l}
|\input{childdoc.def}|\\
|\childdocforwardprefix[|\textit{main}|]{|\textit{prefix}|}{|\textit{dest}|}|
\end{tabular}
\end{center}
%
the destination file is determined by a pattern
depending on the current file:
To make this work, the current file must be called
`{\textit{prefix}\hspace{0.2em}\textit{suffix}}'
with \textit{prefix} matching precisely the argument.
Processing is then passed on to the file
`{\textit{dest}\hspace{0.2em}\textit{suffix}}'.
Surely, the same effect is achieved by
directly specifying the
argument `{\textit{dest}\hspace{0.2em}\textit{suffix}}'
in the first form.
However, that requires to set up a different file
for each child. With the alternative form of the command
all these files can have exactly the same content
which simplifies setting them up and maintaining them.

For example, the following file |draft.tex|
with a compilation flag |\version| as described in \secref{sec:flags}
compiles the main document as a draft:
%
\begin{center}
\begin{tabular}{l}
|\def\version{draft}|\\
|\input{childdoc.def}|\\
|\childdocforward{|\textit{main}|}|
\end{tabular}
\end{center}
%
Likewise, the following files |final|\textit{nn}|.tex|
compile the final version of the child document
|child|\textit{nn}|.tex|:
%
\begin{center}
\begin{tabular}{l}
|\def\version{final}|\\
|\input{childdoc.def}|\\
|\childdocforwardprefix{final}{child}|
\end{tabular}
\end{center}
%

Note that when several versions of a main file and/or of each child file
are to be generated, it may be convenient to set up a |Makefile| or
shell script to automatise the process.

%%%%%%%%%%%%%%%%%%%%%%%%%%%%%%%%%%%%%%%%%%%%%%%%%%%%%%%%%%%%%%%%%%%%%%%%%%%%%%%%
\subsection{Command Line Processing}
\label{sec:commandline}

The effect of redirection files can also be achieved by invoking
the \LaTeX{} compiler with a more elaborate command line.
Most conveniently this should be done as part
of a shell script or a |Makefile|.

When using \textsf{childdoc} in the main file, the following
command lines effectively perform a redirection
(note that depending on the shell being used,
backslashes may have to be doubled: `|\|' $\to$ `|\\|'):
%
\begin{center}
|... -jobname "|\textit{target}|" |\\|"|[\textit{flags}]%
|\input{childdoc.def}\childdocforward[|\textit{main}|]{|\textit{dest}|}"|
\end{center}
%
Here \textit{target} is the name of the output file,
\textit{main} is the name of the main file
and \textit{dest} is the name of the main or child file to be processed
(all filenames without extensions).
The optional argument \textit{main} can be omitted
if \textit{main} matches \textit{dest}.
Optionally, compilation \textit{flags} can be defined via |\def| commands.
This command line makes the \TeX{} engine believe
it is compiling the file \textit{target}
whose content is specified as the latter parameter.
The provided code then forwards the processing to
\textit{main} or \textit{dest} as described in \secref{sec:forward}.

%%%%%%%%%%%%%%%%%%%%%%%%%%%%%%%%%%%%%%%%%%%%%%%%%%%%%%%%%%%%%%%%%%%%%%%%%%%%%%%%
\subsection{Include by Input}
\label{sec:input}

Including child documents by |\include| has some restrictions by design.
Most notably, the content of a child document always occupies
its own set of pages; pages cannot be shared between child documents.
Usually, this behaviour makes perfect sense
because each child document contain an essential part of the document.
However, in some situations it may be desirable to compose
a document from a collection of parts
without having mandatory page breaks between then.
For this case, the package
provides a mechanism to include parts
by |\input| which can also be processed individually.
However, by construction this mechanism
requires manual handling of the content to be output.

%%%%%%%%%%%%%%%%%%%%%%%%%%%%%%%%%%%%%%%%
\DescribeMacro{\ifchilddocmanual}
The main file should be prepared as usual, see \secref{sec:include}.
However, the document body must make a distinction
between processing of an individual part and of the main document, e.g.:
%
\begin{center}
\begin{tabular}{l}
|\ifchilddocmanual|\\
|\input{\childdocname}|\\
|\||else|\\
\textit{document body with }|\input{|\textit{part}|}|\\
|\||fi|
\end{tabular}
\end{center}
%
The conditional |\ifchilddocmanual| is true whenever
a part to be included by |\input| is being compiled,
and the name of the part is stored in |\childdocname|.

%%%%%%%%%%%%%%%%%%%%%%%%%%%%%%%%%%%%%%%%
\DescribeMacro{\childdocby}
Each part to be included by |\input| should start with:
%
\begin{center}
\begin{tabular}{l}
|\input{childdoc.def}|\\
|\childdocby{|\textit{main}|}|\\
\end{tabular}
\end{center}
%
The directive |\childdocby| is similar to |\childdocof|
described in \secref{sec:include},
but the subsequent selection of content must be done manually.
To that end, both |\ifchilddoc| and |\ifchilddocmanual|
will be true upon processing of a part,
and the name of the part is stored in |\childdocname|.
Note that |\jobname| will be set to the filename of the current part
so that each part receives an individual |.aux| file
that does not interfere with the |.aux| file(s) of the main document.
This behaviour can be altered by the alternative form
|\childdocby[*]{|\textit{main}|}| (with a non-empty optional argument)
which uses the |.aux| file of the main document
by setting |\jobname| to \textit{main}.

%%%%%%%%%%%%%%%%%%%%%%%%%%%%%%%%%%%%%%%%%%%%%%%%%%%%%%%%%%%%%%%%%%%%%%%%%%%%%%%%
\subsection{Driver Development}
\label{sec:driver}

The \textsf{childdoc} mechanism can also be use for the development
of definition files such as \LaTeX{} styles or classes.
This case differs from the above setup with multiple parts
included by |\include| in that no |\includeonly| should be invoked.
This can be achieved by starting the include file
(before |\ProvidesPackage|) with:
%
\begin{center}
\begin{tabular}{l}
|\input{childdoc.def}|\\
|\childdocforward{|\textit{main}|}|\\
\end{tabular}
\end{center}
%
or alternatively with:
%
\begin{center}
\begin{tabular}{l}
|\input{childdoc.def}|\\
|\childdocby{|\textit{main}|}|\\
\end{tabular}
\end{center}
%
Both forms have slightly different effects as described above.
The main file is prepared as usual, see \secref{sec:include}.

%%%%%%%%%%%%%%%%%%%%%%%%%%%%%%%%%%%%%%%%%%%%%%%%%%%%%%%%%%%%%%%%%%%%%%%%%%%%%%%%
\subsection{Legacy Detection}
\label{sec:detection}

The directive |\childdocmain| in the main file can detect
whether the complete document or merely a child is to be compiled
even without using the directive |\childdocof|.
This method is deprecated because it is less robust
and there is no compelling reason to use it;
it is merely provided for backward compatibility
and it may be removed in future versions.

If the detection mechanism is to be used,
it is mandatory to correctly specify
the filename of the main file as the argument of |\childdocmain|:
%
\begin{center}
\begin{tabular}{l}
|\input{childdoc.def}|\\
|\childdocmain{|\textit{main}|}|\\
\end{tabular}
\end{center}
%
If |\jobname| does not match the argument \textit{main} of |\childdocmain|,
it is assumed that |\jobname| points to the child file to be compiled.
When using |\childdocmain| with the main file specified as argument,
it suffices to start a child file
with just |\input{|\textit{main}|}|
without loading of the package and using |\childdocof|.
If instead all processing is done
with the appropriate \textsf{childdoc} directives,
the argument of \textit{main} of |\childdocmain| can be empty.

An alternative version of the command line processing described
in \secref{sec:commandline} using the detection mechanism reads:
%
\begin{center}
|... -jobname "|\textit{target}|" "|[\textit{flags}]%
[|\def\jobname{|\textit{dest}|}|]|\input{|\textit{main}|}"|
\end{center}

%%%%%%%%%%%%%%%%%%%%%%%%%%%%%%%%%%%%%%%%%%%%%%%%%%%%%%%%%%%%%%%%%%%%%%%%%%%%%%%%
\subsection{Manual Code}
\label{sec:manual}

In case one cannot be certain whether the definitions file |childdoc.def|
is installed on the target \TeX{} distribution
and one prefers not to ship it,
it is conceivable to paste a few relevant commands into the sources.

To that end, drop all statements |\input{childdoc.def}|
and perform the replacements as outlined below.
Instead of |\childdocmain{|\textit{main}|}| add the following code
to the top of the main file:
%
\begin{center}
\begin{tabular}{l}
|\||ifdefined\childdocname\endinput\||fi\newif\ifchilddoc|\\
|\edef\childdocname{\scantokens\expandafter{\jobname\noexpand}}|\\
|\def\childdocmain{|\textit{main}|}\||ifx\childdocmain\childdocname\||else|\\
|\childdoctrue\includeonly{\childdocname}\let\jobname\childdocmain\||fi|\\
\end{tabular}
\end{center}
%
Instead of |\childdocof{|\textit{main}|}| just include the main file
at the top of each child file:
%
\begin{center}
|\input{|\textit{main}|}|
\end{center}
%
A simple redirection |\childdocforward{|\textit{dest}|}| is achieved by:
%
\begin{center}
|\def\jobname{|\textit{dest}|}\input{\jobname}|
\end{center}
%
The redirection with prefix
|\childdocforwardprefix[|\textit{prefix}|]{|\textit{dest}|}|
is accomplished by:
%
\begin{center}
\begin{tabular}{l}
|{\edef\jobname{\scantokens\expandafter{\jobname\noexpand}}|\\
|\def\redirectjob |\textit{prefix}|#1~~~{\gdef\jobname{|\textit{dest}|#1}}|\\
|\expandafter\redirectjob\jobname~~~}\input{\jobname}|
\end{tabular}
\end{center}

In an alternative approach,
child documents can be compiled by a specific command line
without additional code or specific definitions:
%
\begin{center}
|... -jobname "|\textit{target}|" "|[\textit{flags}]%
|\includeonly{|\textit{dest}|}\input{|\textit{main}|}"|
\end{center}
%

%%%%%%%%%%%%%%%%%%%%%%%%%%%%%%%%%%%%%%%%%%%%%%%%%%%%%%%%%%%%%%%%%%%%%%%%%%%%%%%%
%%%%%%%%%%%%%%%%%%%%%%%%%%%%%%%%%%%%%%%%%%%%%%%%%%%%%%%%%%%%%%%%%%%%%%%%%%%%%%%%
\section{Information}

%%%%%%%%%%%%%%%%%%%%%%%%%%%%%%%%%%%%%%%%%%%%%%%%%%%%%%%%%%%%%%%%%%%%%%%%%%%%%%%%
\subsection{Copyright}

Copyright \copyright{} 2017--2018 Niklas Beisert

This work may be distributed and/or modified under the
conditions of the \LaTeX{} Project Public License, either version 1.3
of this license or (at your option) any later version.
The latest version of this license is in
  \url{http://www.latex-project.org/lppl.txt}
and version 1.3 or later is part of all distributions of \LaTeX{}
version 2005/12/01 or later.

This work has the LPPL maintenance status `maintained'.

The Current Maintainer of this work is Niklas Beisert.

This work consists of the files |README.txt|, |childdoc.ins| and |childdoc.dtx|
as well as the derived files |childdoc.def|, |cdocsamp.tex|
with |cdocsch1.tex|, |cdocsch2.tex|, |cdocspt3.tex|, |cdocspt4.tex|,
|cdocsdrf.tex|, |cdocsfn1.tex|, |cdocsfn2.tex|
as well as |childdoc.pdf|.

%%%%%%%%%%%%%%%%%%%%%%%%%%%%%%%%%%%%%%%%%%%%%%%%%%%%%%%%%%%%%%%%%%%%%%%%%%%%%%%%
\subsection{Files and Installation}

The package consists of the files:
%
\begin{center}
\begin{tabular}{ll}
    |README.txt|   & readme file \\
    |childdoc.ins| & installation file \\
    |childdoc.dtx| & source file \\
    |childdoc.def| & definition file \\
    |cdocsamp.tex| & sample main file \\
    |cdocsch1.tex| & sample include file \\
    |cdocsch2.tex| & sample include file \\
    |cdocspt3.tex| & sample part file \\
    |cdocspt4.tex| & sample part file \\
    |cdocsdrf.tex| & sample redirection file \\
    |cdocsfn1.tex| & sample redirection file \\
    |cdocsfn2.tex| & sample redirection file \\
    |childdoc.pdf| & manual
\end{tabular}
\end{center}
%
The distribution consists of the files
|README.txt|, |childdoc.ins| and |childdoc.dtx|.
%
\begin{itemize}
\item
Run (pdf)\LaTeX{} on |childdoc.dtx|
to compile the manual |childdoc.pdf| (this file).
\item
Run \LaTeX{} on |childdoc.ins| to create the definitions file |childdoc.def|
and the sample |cdocsamp.tex| with include files
|cdocsch1.tex|, |cdocsch2.tex|, |cdocspt3.tex|, |cdocspt4.tex|,
|cdocsdrf.tex|, |cdocsfn1.tex|, |cdocsfn2.tex|.
Then copy the file |childdoc.def| to an appropriate directory of your \LaTeX{}
distribution, e.g.\ \textit{texmf-root}|/tex/latex/childdoc|.
\end{itemize}

%%%%%%%%%%%%%%%%%%%%%%%%%%%%%%%%%%%%%%%%%%%%%%%%%%%%%%%%%%%%%%%%%%%%%%%%%%%%%%%%
\subsection{Related CTAN Packages}

There are several other packages which offer a similar functionality:
%
\begin{itemize}
\item
The packages
\href{http://ctan.org/pkg/docmute}{\textsf{docmute}},
\href{http://ctan.org/pkg/includex}{\textsf{includex}} and
\href{http://ctan.org/pkg/standalone}{\textsf{standalone}}
provide commands to include only the document body of
a child file thus allowing both files to be compiled individually.
\item
The packages \href{http://ctan.org/pkg/subdocs}{\textsf{subdocs}}
and \href{http://ctan.org/pkg/subfiles}{\textsf{subfiles}}
provide structures in which the main and child documents can be
encapsulated and allowing them to be compiled individually.
The inclusion mechanism is different from the conventional |\include|.
\item
The package \href{http://ctan.org/pkg/combine}{\textsf{combine}}
is an elaborate solution to combine several documents into one.
\end{itemize}
%
See also the CTAN topic \href{http://ctan.org/topic/subdocs}{\textsf{subdocs}}
for further related packages.
The present package differs from the above solutions in that
a document structure constructed with the conventional |\include| mechanism
just needs two extra commands at the top of every file
such that all constituent files can be compiled individually.

%%%%%%%%%%%%%%%%%%%%%%%%%%%%%%%%%%%%%%%%%%%%%%%%%%%%%%%%%%%%%%%%%%%%%%%%%%%%%%%%
%\subsection{Feature Suggestions}
%
%The following is a list of features which may be useful for future
%versions of this package:
%%
%\begin{itemize}
%\item
%\ldots
%\end{itemize}

%%%%%%%%%%%%%%%%%%%%%%%%%%%%%%%%%%%%%%%%%%%%%%%%%%%%%%%%%%%%%%%%%%%%%%%%%%%%%%%%
\subsection{Revision History}

%%%%%%%%%%%%%%%%%%%%%%%%%%%%%%%%%%%%%%%%
\paragraph{v2.0:} 2018/12/30

\begin{itemize}
\item
immediate forward processing
\item
added |\childdocby| mechanism
\item
manual restructured
\end{itemize}

%%%%%%%%%%%%%%%%%%%%%%%%%%%%%%%%%%%%%%%%
\paragraph{v1.6:} 2018/01/17

\begin{itemize}
\item
application for development of include files
\item
corrections to manual
\end{itemize}

%%%%%%%%%%%%%%%%%%%%%%%%%%%%%%%%%%%%%%%%
\paragraph{v1.5:} 2017/05/21

\begin{itemize}
\item
more complete structuring introduced
\item
|\childdocof| introduced
\item
|\childdoc| renamed to |\childdocmain|
\item
|\childredirect| renamed to |\childdocforward| and |\childdocforwardprefix|
and functionality expanded
\end{itemize}

%%%%%%%%%%%%%%%%%%%%%%%%%%%%%%%%%%%%%%%%
\paragraph{v1.0:} 2017/04/27

\begin{itemize}
\item
manual and install package
\item
first version published on CTAN
\end{itemize}

%%%%%%%%%%%%%%%%%%%%%%%%%%%%%%%%%%%%%%%%
\paragraph{v0.6:} 2017/04/26

\begin{itemize}
\item
redirection mechanism added
\end{itemize}

%%%%%%%%%%%%%%%%%%%%%%%%%%%%%%%%%%%%%%%%
\paragraph{v0.5:} 2017/04/26

\begin{itemize}
\item
functionality in definition file
\end{itemize}


%%%%%%%%%%%%%%%%%%%%%%%%%%%%%%%%%%%%%%%%%%%%%%%%%%%%%%%%%%%%%%%%%%%%%%%%%%%%%%%%
%%%%%%%%%%%%%%%%%%%%%%%%%%%%%%%%%%%%%%%%%%%%%%%%%%%%%%%%%%%%%%%%%%%%%%%%%%%%%%%%
%%%%%%%%%%%%%%%%%%%%%%%%%%%%%%%%%%%%%%%%%%%%%%%%%%%%%%%%%%%%%%%%%%%%%%%%%%%%%%%%
\appendix

\settowidth\MacroIndent{\rmfamily\scriptsize 000\ }

 \DocInput{childdoc.dtx}

\end{document}
%</driver>
% \fi
%
% %%%%%%%%%%%%%%%%%%%%%%%%%%%%%%%%%%%%%%%%%%%%%%%%%%%%%%%%%%%%%%%%%%%%%%%%%%%%%%
% %%%%%%%%%%%%%%%%%%%%%%%%%%%%%%%%%%%%%%%%%%%%%%%%%%%%%%%%%%%%%%%%%%%%%%%%%%%%%%
% \section{Sample}
%\iffalse
%<*samplemain>
%\fi
%
% The following presents a sample document
% with two chapters, two parts, a title page,
% a compile flag as well as three forwarding files to set the flag.
% It consists of eight |.tex| files:
% \begin{center}
% \begin{tabular}{ll}
% |cdocsamp.tex|&main file\\
% |cdocsch1.tex|&include file for chapter 1\\
% |cdocsch2.tex|&include file for chapter 2\\
% |cdocspt3.tex|&include file for part 3\\
% |cdocspt4.tex|&include file for part 4\\
% |cdocsdrf.tex|&forwarding file for main file in draft mode\\
% |cdocsfi1.tex|&forwarding file for final version of chapter 1\\
% |cdocsfi2.tex|&forwarding file for final version of chapter 2\\
% \end{tabular}
% \end{center}
% Each of the eight files can be compiled directly by the \LaTeX{} compiler.
%
% %%%%%%%%%%%%%%%%%%%%%%%%%%%%%%%%%%%%%%
% \paragraph{Main File.}
%
% The main file is called |cdocsamp.tex|.
%
% Load the \textsf{childdoc} definitions and
% declare the filename for the main document:
%    \begin{macrocode}
\input{childdoc.def}
\childdocmain{}
%    \end{macrocode}

% Optional override for |\version| flag:
%    \begin{macrocode}
%%\ifchilddoc\else\providecommand{\version}{draft}\fi
%    \end{macrocode}

% Define the default values for the |\version| flag
% (|final| for the main file and |draft| for childs):
%    \begin{macrocode}
\ifchilddoc
\providecommand{\version}{draft}
\else
\providecommand{\version}{final}
\fi
%    \end{macrocode}

% Load the standard document class:
%    \begin{macrocode}
\documentclass[12pt]{article}
%    \end{macrocode}

% Start the document body:
%    \begin{macrocode}
\begin{document}
%    \end{macrocode}

% Declare a title page.
% Print title, part of document being processed and version flag:
%    \begin{macrocode}
\addtocounter{page}{-1}
\begin{center}
{\LARGE\bfseries{}childdoc example\par}
\vspace{1cm}
\ifchilddoc
\ifchilddocmanual part\else chapter\fi:
`\childdocname' of `\childdocjob'\par
\else
main document: `\childdocjob'\par
\fi
version: \version\par
\end{center}
\newpage
%    \end{macrocode}

% Manually include selected file,
% otherwise process as usual:
%    \begin{macrocode}
\ifchilddocmanual
\section*{part `\childdocname'}
\input{\childdocname}
\else
%    \end{macrocode}

% Include the two chapters:
%    \begin{macrocode}
\include{cdocsch1}
\include{cdocsch2}
%    \end{macrocode}

% Include the two parts unless only chapters should be displayed:
%    \begin{macrocode}
\ifchilddoc\else
\section{part three}
\input{cdocspt3}
\section{part four}
\input{cdocspt4}
\fi
%    \end{macrocode}

% Process as usual until here:
%    \begin{macrocode}
\fi
%    \end{macrocode}

% End of document body:
%    \begin{macrocode}
\end{document}
%    \end{macrocode}
%\iffalse
%</samplemain>
%\fi
%
% %%%%%%%%%%%%%%%%%%%%%%%%%%%%%%%%%%%%%%
% \paragraph{Chapter Include Files.}
%
% The include files are called |cdocsch1.tex| and |cdocsch2.tex|.
%
%\iffalse
%<*samplechap1|samplechap2>
%\fi

% Optional override for |\version| flag:
%    \begin{macrocode}
%%\providecommand{\version}{final}
%    \end{macrocode}

% Include the main document:
%    \begin{macrocode}
\input{childdoc.def}
\childdocof{cdocsamp}
%    \end{macrocode}

%\iffalse
%</samplechap1|samplechap2>
%\fi
%
%\iffalse
%<*samplechap1>
%\fi
% Some text for chapter 1:
%    \begin{macrocode}
\section{one}
some text in chapter one
%    \end{macrocode}

%\iffalse
%</samplechap1>
%\fi
% Some text for chapter 2:
%\iffalse
%<*samplechap2>
%\fi
%    \begin{macrocode}
\section{two}
more text in chapter two
%    \end{macrocode}

%\iffalse
%</samplechap2>
%\fi
%
% %%%%%%%%%%%%%%%%%%%%%%%%%%%%%%%%%%%%%%
% \paragraph{Part Include Files.}
%
% The include files are called |cdocspt3.tex| and |cdocspt4.tex|.
%
%\iffalse
%<*samplepart3|samplepart4>
%\fi

% Optional override for |\version| flag:
%    \begin{macrocode}
%%\providecommand{\version}{final}
%    \end{macrocode}

% Include the main document:
%    \begin{macrocode}
\input{childdoc.def}
\childdocby{cdocsamp}
%    \end{macrocode}

%\iffalse
%</samplepart3|samplepart4>
%\fi
%
%\iffalse
%<*samplepart3>
%\fi
% Some text for part 3:
%    \begin{macrocode}
some text in part three
%    \end{macrocode}

%\iffalse
%</samplepart3>
%\fi
% Some text for part 4:
%\iffalse
%<*samplepart4>
%\fi
%    \begin{macrocode}
more text in part four
%    \end{macrocode}

%\iffalse
%</samplepart4>
%\fi
%
% %%%%%%%%%%%%%%%%%%%%%%%%%%%%%%%%%%%%%%
% \paragraph{Forwarding for a Complete Draft.}
%
% The following forwarding file |cdocsdrf.tex|
% compiles the main document in draft mode:
%\iffalse
%<*sampledraft>
%\fi
%    \begin{macrocode}
\def\version{draft}
\input{childdoc.def}
\childdocforward{cdocsamp}
%    \end{macrocode}

%\iffalse
%</sampledraft>
%\fi
%
% %%%%%%%%%%%%%%%%%%%%%%%%%%%%%%%%%%%%%%
% \paragraph{Forwarding for Final Version of the Chapters.}
%
% The following forwarding files |cdocsfn1.tex| and |cdocsfn2.tex|
% (with identical content)
% compile the final versions of the child documents
% |cdocsch1.tex| and |cdocsch2.tex|, respectively:
%\iffalse
%<*samplefinal>
%\fi
%    \begin{macrocode}
\def\version{final}
\input{childdoc.def}
\childdocforwardprefix[cdocsamp]{cdocsfn}{cdocsch}
%    \end{macrocode}

%\iffalse
%</samplefinal>
%\fi
%
% %%%%%%%%%%%%%%%%%%%%%%%%%%%%%%%%%%%%%%
% \paragraph{Command Line Processing.}
%
% The following three command lines generate the output files
% |cdocscld|, |cdocscl1| and |cdocscl2|
% which should be identical to
% |cdocsdrf|, |cdocsch1| and |cdocsfn2|, respectively:
% \begin{center}
% \begin{tabular}{l}
% |latex -jobname cdocscld \|\\
% |  "\def\version{draft}\input{childdoc.def}\childdocforward{cdocsamp}"|\\
% |latex -jobname cdocscl1 \|\\
% |  "\input{childdoc.def}\childdocforward[cdocsamp]{cdocsch1}"|\\
% |latex -jobname cdocscl2 \|\\
% |  "\def\version{final}\input{childdoc.def}\childdocforward{cdocsch2}"|
% \end{tabular}
% \end{center}
% Note that the trailing backslash on each first line
% merely continues the input to the second line
% (for convenient cut ant paste).
% Furthermore, the command |latex| can be replaced by any
% of its alternative versions such as |pdflatex|.
%
% %%%%%%%%%%%%%%%%%%%%%%%%%%%%%%%%%%%%%%%%%%%%%%%%%%%%%%%%%%%%%%%%%%%%%%%%%%%%%%
% %%%%%%%%%%%%%%%%%%%%%%%%%%%%%%%%%%%%%%%%%%%%%%%%%%%%%%%%%%%%%%%%%%%%%%%%%%%%%%
% \section{Implementation}
%\iffalse
%<*package>
%\fi
%
% This section describes the definitions file |childdoc.def|.

% The definitions cannot be loaded using |\usepackage| or |\RequirePackage|
% which has a mechanism to prevent loading a style file more than once.
% When loading the definitions by means of |\input|
% multiple instances have to be prevented manually:
%\iffalse
%This code needs to be before the `\ProvidesFile' directive
%which is defined at the beginning of this file.
%Therefore it is also placed there and commented out here.
%</package>
%<*discard>
%\fi
%    \begin{macrocode}
\ifdefined\childdocmain\endinput\fi
%    \end{macrocode}
%\iffalse
%</discard>
%<*package>
%\fi
%
% \macro{\ifchilddoc}
% \macro{\ifchilddocmanual}
% The conditional |\ifchilddoc| tells whether a
% child (true) or main (false) document is being compiled.
% The conditional |\ifchilddocmanual| tells whether
% the |\includeonly| mechanism is used (false) or
% the selection of child files must be performed manually (true).
% The definitions initialise to false:
%    \begin{macrocode}
\newif\ifchilddoc
\newif\ifchilddocmanual
%    \end{macrocode}

% \macro{\childdocname}
% \macro{\childdocjob}
% The macro |\childdocname| stores the name of the main document
% to be compiled. The macro |\childdocjob| stores the name of
% the document on which the \LaTeX{} compiler was originally invoked.
% The content of |\jobname| cannot be compared
% to filenames specified in the source due to different catcodes.
% The following code rescans |\jobname|, stores the result
% in |\childdocname| and saves a copy in |\childdocjob|:
%    \begin{macrocode}
\edef\childdocname{\scantokens\expandafter{\jobname\noexpand}}
\let\childdocjob\childdocname
%    \end{macrocode}

% \macro{\childdocdisable}
% The macro |\childdocdisable| prevents the main file
% from being processed more than once.
% At this stage, the main document command |\childdocmain|
% is assumed to be called once again where it should do nothing.
% Any subsequent call to it should prevent
% a secondary processing of the main document
% It overwrites the forwarding commands
% |\childdocof| and |\childdocforward|
% with empty macros to prevent further inclusions of the main document:
%    \begin{macrocode}
\newcommand{\childdocdisable}
{
  \renewcommand{\childdocmain}[1]{\renewcommand{\childdocmain}[1]{\endinput}}
  \renewcommand{\childdocof}[1]{}
  \renewcommand{\childdocby}[2][]{}
  \renewcommand{\childdocforward}[2][]{}
  \renewcommand{\childdocdisable}{}
}
%    \end{macrocode}

% \macro{\childdocmain}
% The macro |\childdocmain| is to be called at the top of the main file
% with nothing or the main filename (without extension) as argument.
% First, it breaks loops.
% If the argument is not empty and does not match |\childdocname|
% (which is set by the first inclusion of |childdoc.def|),
% |\ifchilddoc| is set to true, |\includeonly| is applied to the child file
% and |\jobname| is set to the main file
% (for proper handling of |.aux| files):
%    \begin{macrocode}
\newcommand{\childdocmain}[1]
{
  \childdocdisable\childdocmain{}
  \if?#1?\else
    \begingroup
      \def\childdoctmp{#1}
      \ifx\childdoctmp\childdocname
        \def\childdoctmp{}
      \else
        \def\childdoctmp
        {
          \childdoctrue
          \includeonly{\childdocname}
          \def\childdocjob{#1}
          \def\jobname{#1}
        }
      \fi
      \expandafter
    \endgroup
    \childdoctmp
  \fi
}
%    \end{macrocode}

% \macro{\childdocof}
% The command |\childdocof| redirects
% compilation to the main file |#1|.
%    \begin{macrocode}
\newcommand{\childdocof}[1]
{
  \childdocdisable
  \childdoctrue
  \includeonly{\childdocname}
  \def\jobname{#1}
  \def\childdocjob{#1}
  \input{#1}
}
%    \end{macrocode}

% \macro{\childdocby}
% The command |\childdocby| ....
%    \begin{macrocode}
\newcommand{\childdocby}[2][]
{
  \childdocdisable
  \childdoctrue
  \childdocmanualtrue
  \if?#1?\else
    \def\jobname{#2}
  \fi
  \def\childdocjob{#2}
  \input{#2}
  \endinput
}
%    \end{macrocode}

% \macro{\childdocforward}
% The command |\childdocforward| redirects
% compilation to the main file or
% (if the optional argument is given) a child file.
% Parameters are set as if the main file
% or a child file starting with |\childdocof| was compiled.
% Then compilation is handed over to the main file:
%    \begin{macrocode}
\newcommand{\childdocforward}[2][]
{
  \begingroup
    \if?#1?
      \def\childdoctmp
      {
        \def\childdocname{#2}
        \def\childdocjob{#2}
        \def\jobname{#2}
        \input{#2}
        \endinput
      }
    \else
      \def\childdoctmp
      {
        \childdocdisable
        \def\childdocname{#2}
        \childdoctrue
        \includeonly{#2}
        \def\childdocjob{#1}
        \def\jobname{#1}
        \input{#1}
        \endinput
      }
    \fi
    \expandafter
  \endgroup
  \childdoctmp
}
%    \end{macrocode}

% \macro{\childdocforwardprefix}
% The command |\childdocforwardprefix| redirects
% compilation to the main or a child file by means of a pattern.
% The prefix |#1| in the current filename is replaced by |#2|
% and the suffix of the current filename is kept
% (it is assumed that the filename does not contain the substring `|~~~|'
% which is used as a delimiter).
% Compilation is handed over to the new file by |\childdocforward|:
%    \begin{macrocode}
\newcommand{\childdocforwardprefix}[3][]
{
  \begingroup
    \def\childdocextract #2##1~~~{\def\childdoctmp{\childdocforward[#1]{#3##1}}}
    \expandafter\childdocextract\childdocname~~~
    \expandafter
  \endgroup
  \childdoctmp
}
%    \end{macrocode}

% \macro{\childdoc}
% The deprecated macro |\childdoc| is a legacy version of |\childdocmain|:
%    \begin{macrocode}
\newcommand{\childdoc}{\childdocmain}
%    \end{macrocode}

% \macro{\childdocredirect}
% The deprecated macro |\childdocredirect| is a legacy version
% of |\childdocforward| and |\childdocforwardprefix|:
%    \begin{macrocode}
\newcommand{\childdocredirect}[2][]
{
  \begingroup
    \if?#1?
      \def\childdoctmp{\childdocforward{#2}}
    \else
      \def\childdoctmp{\childdocforwardprefix{#1}{#2}}
    \fi
    \expandafter
  \endgroup
  \childdoctmp
}
%    \end{macrocode}

%\iffalse
%</package>
%\fi
%
\endinput

\childdocof{cdocsamp}
%    \end{macrocode}

%\iffalse
%</samplechap1|samplechap2>
%\fi
%
%\iffalse
%<*samplechap1>
%\fi
% Some text for chapter 1:
%    \begin{macrocode}
\section{one}
some text in chapter one
%    \end{macrocode}

%\iffalse
%</samplechap1>
%\fi
% Some text for chapter 2:
%\iffalse
%<*samplechap2>
%\fi
%    \begin{macrocode}
\section{two}
more text in chapter two
%    \end{macrocode}

%\iffalse
%</samplechap2>
%\fi
%
% %%%%%%%%%%%%%%%%%%%%%%%%%%%%%%%%%%%%%%
% \paragraph{Part Include Files.}
%
% The include files are called |cdocspt3.tex| and |cdocspt4.tex|.
%
%\iffalse
%<*samplepart3|samplepart4>
%\fi

% Optional override for |\version| flag:
%    \begin{macrocode}
%%\providecommand{\version}{final}
%    \end{macrocode}

% Include the main document:
%    \begin{macrocode}
% \iffalse
%
% childdoc.dtx Copyright (C) 2017-2018 Niklas Beisert
%
% This work may be distributed and/or modified under the
% conditions of the LaTeX Project Public License, either version 1.3
% of this license or (at your option) any later version.
% The latest version of this license is in
%   http://www.latex-project.org/lppl.txt
% and version 1.3 or later is part of all distributions of LaTeX
% version 2005/12/01 or later.
%
% This work has the LPPL maintenance status `maintained'.
%
% The Current Maintainer of this work is Niklas Beisert.
%
% This work consists of the files childdoc.dtx and childdoc.ins
% and the derived files childdoc.def and cdocsamp.tex with
% cdocsch1.tex, cdocsch2.tex, cdocsdrf.tex, cdocsfn1.tex, cdocsfn2.tex.
%
%<package>\ifdefined\childdocmain\endinput\fi
%<package>\ProvidesFile{childdoc.def}[2018/12/30 v2.0 child document driver]
%<samplemain>\ProvidesFile{cdocsamp.tex}[2018/12/30 v2.0 sample for childdoc]
%<*driver>
%\ProvidesFile{childdoc.drv}[2018/12/30 v2.0 childdoc reference manual file]
\PassOptionsToClass{10pt,a4paper}{article}
\documentclass{ltxdoc}

\usepackage[margin=35mm]{geometry}
\usepackage{hyperref}
\usepackage{hyperxmp}
\usepackage[usenames]{color}

\hypersetup{colorlinks=true}
\hypersetup{pdfstartview=FitH}
\hypersetup{pdfpagemode=UseNone}
\hypersetup{pdfsource={}}
\hypersetup{pdflang={en-UK}}
\hypersetup{pdfcopyright={Copyright 2017-2018 Niklas Beisert.
  This work may be distributed and/or modified under the
  conditions of the LaTeX Project Public License, either version 1.3
  of this license or (at your option) any later version.}}
\hypersetup{pdflicenseurl={http://www.latex-project.org/lppl.txt}}
\hypersetup{pdfcontactaddress={ETH Zurich, ITP, HIT K,
  Wolfgang-Pauli-Strasse 27}}
\hypersetup{pdfcontactpostcode={8093}}
\hypersetup{pdfcontactcity={Zurich}}
\hypersetup{pdfcontactcountry={Switzerland}}
\hypersetup{pdfcontactemail={nbeisert@itp.phys.ethz.ch}}
\hypersetup{pdfcontacturl={http://people.phys.ethz.ch/\xmptilde nbeisert/}}

\newcommand{\secref}[1]{\hyperref[#1]{section \ref*{#1}}}

\parskip1ex
\parindent0pt
\let\olditemize\itemize
\def\itemize{\olditemize\parskip0pt}

\begin{document}

\title{The \textsf{childdoc} Package}
\hypersetup{pdftitle={The childdoc Package}}
\author{Niklas Beisert\\[2ex]
  Institut f\"ur Theoretische Physik\\
  Eidgen\"ossische Technische Hochschule Z\"urich\\
  Wolfgang-Pauli-Strasse 27, 8093 Z\"urich, Switzerland\\[1ex]
  \href{mailto:nbeisert@itp.phys.ethz.ch}
  {\texttt{nbeisert@itp.phys.ethz.ch}}}
\hypersetup{pdfauthor={Niklas Beisert}}
\hypersetup{pdfsubject={Manual for the LaTeX2e Package childdoc}}
\date{30 December 2018, \textsf{v2.0}}
\maketitle

\begin{abstract}\noindent
\textsf{childdoc} is a \LaTeXe{} package
that enables the direct compilation
of document sections included by |\include|
to individual files.
\end{abstract}

\begingroup
\parskip0ex
\tableofcontents
\endgroup

%%%%%%%%%%%%%%%%%%%%%%%%%%%%%%%%%%%%%%%%%%%%%%%%%%%%%%%%%%%%%%%%%%%%%%%%%%%%%%%%
%%%%%%%%%%%%%%%%%%%%%%%%%%%%%%%%%%%%%%%%%%%%%%%%%%%%%%%%%%%%%%%%%%%%%%%%%%%%%%%%
\section{Introduction}

\LaTeX{} provides a mechanism to structure a large document (such as a book)
into a main file and several child files (containing the chapters)
using the |\include| command.
This mechanism is beneficial for documents
which span hundreds of pages in order to
make the source file(s) more manageable.
Moreover, compilation can be restricted to
selected child files by means of the |\includeonly| command.
The latter feature can be used to reduce the compilation time while editing
(this was significantly more useful in the earlier days of \LaTeX{})
or to generate a smaller document which is easier to navigate.
Another application of |\includeonly| is to generate
documents consisting of selected parts of the complete document.

However, there are a few drawbacks of the plain |\include| mechanism:
\begin{itemize}
\item
The child files cannot be compiled on their own,
they can only be compiled via the main file.
A naive editing environment
(such as a text editor with an option
to have the current file processed by \LaTeX)
may require one to switch to the main file before compiling;
attempting to compile the child file produces errors.
\item
The main file must be modified (each time)
to adjust the |\includeonly| command
to the present needs. This easily leaves the main file in a messy state.
\item
The generated document will always carry the filename
of the main document. This is inconvenient if
several child files are to be compiled and
to be kept for distribution.
\end{itemize}

The present package provides a simple interface
to make child files individually compilable by \LaTeX{}.
Compiling a child file then has the same effect as compiling
the main file with an |\includeonly| command
to select the appropriate child.
Moreover the generated document will carry the name of the child
rather than the main file.
This resolves all three above issues.

This feature is meant to make the editing of books,
thesis documents and lecture notes somewhat more convenient.
However, the package can also be used efficiently for
composing a series of documents (such as exercise sheets)
which are typically distributed individually.
It then assists the author in generating the individual documents
(potentially in different versions)
as well as a document containing the collected series.
Another application is in developing style files
or other kinds of included material
where compilation of the style file could redirect
to a sample or test file.

%%%%%%%%%%%%%%%%%%%%%%%%%%%%%%%%%%%%%%%%%%%%%%%%%%%%%%%%%%%%%%%%%%%%%%%%%%%%%%%%
%%%%%%%%%%%%%%%%%%%%%%%%%%%%%%%%%%%%%%%%%%%%%%%%%%%%%%%%%%%%%%%%%%%%%%%%%%%%%%%%
\section{Usage}

First of all, the package \textsf{childdoc} is \emph{not} a standard
\LaTeXe{} |.sty| style file! Therefore it needs to be invoked in
a non-standard way.

%%%%%%%%%%%%%%%%%%%%%%%%%%%%%%%%%%%%%%%%%%%%%%%%%%%%%%%%%%%%%%%%%%%%%%%%%%%%%%%%
\subsection{Included Files}
\label{sec:include}

%%%%%%%%%%%%%%%%%%%%%%%%%%%%%%%%%%%%%%%%
\DescribeMacro{\childdocmain}
To use the package, add the commands
\begin{center}
\begin{tabular}{l}
|\input{childdoc.def}|\\
|\childdocmain{}|\\
\end{tabular}
\end{center}
at the very top of the main \LaTeX{} file,
in particular \emph{before} the |\documentclass| statement!
The argument of |\childdocmain| should be left empty
(but it must be present).

%%%%%%%%%%%%%%%%%%%%%%%%%%%%%%%%%%%%%%%%
\DescribeMacro{\childdocof}
Furthermore, add the commands
\begin{center}
\begin{tabular}{l}
|\input{childdoc.def}|\\
|\childdocof{|\textit{main}|}|\\
\end{tabular}
\end{center}
at the top of every child file \textit{child}
which is included by |\include{|\textit{child}|}|
from within the main file
(or at least for those files to be compiled individually).
The argument \textit{main} must be the filename of the main file.

There are a couple of
considerations in setting up the main and child documents:

%%%%%%%%%%%%%%%%%%%%%%%%%%%%%%%%%%%%%%%%
\paragraph{Restrictions.}

Please note the following restrictions:
\begin{itemize}
\item
|\childdocmain| must be called with one argument \textit{main}
to ensure compatibility with earlier version of the package.
It must either be empty (|\childdocmain{}|)
or precisely match the filename of the main file in which it is specified.
See \secref{sec:detection} for further information.
\item
The filename \textit{main} must be specified without the |.tex| extension.
\item
The filename \textit{main} is case sensitive
(even in case-insensitive file systems)
due to internal string comparison.
\item
The argument \textit{main} should be fully expanded, it cannot be a macro.
\item
Subdirectories and special characters should be avoided in filenames.
\item
The command |\childdocmain{|\textit{main}|}| must be followed by a whitespace.
It should not be followed immediately by another command
or by a comment mark `|%|'.
This is because the \TeX{} parser reads the token immediately following
the argument of |\childdocmain| and puts it
at the beginning of every child section;
however, a white\-space is ignored.
\end{itemize}

%%%%%%%%%%%%%%%%%%%%%%%%%%%%%%%%%%%%%%%%
\paragraph{Content of Main File.}

It is advisable to place all content in the child files included by |\include|.
Any output contained in the main file will appear in all child documents
unless suppressed manually;
it cannot be suppressed automatically by the |\includeonly| directive
and thus should normally be avoided.
A method to include some content in the main file
by means of conditional processing is described in \secref{sec:conditional}.

%%%%%%%%%%%%%%%%%%%%%%%%%%%%%%%%%%%%%%%%
\paragraph{Page Numbering.}

When only a part of the document is compiled,
the appropriate numbering of pages
(as well as other status parameters)
is determined from the |.aux| files.
The latter contain information from previous passes.
However this information needs to propagate through
all intermediate child documents.
Therefore the page numbering in child documents may well
be inconsistent until the complete document is compiled at least once.

A useful (if unconventional) way to always ensure a consistent
page numbering is to restart the numbering in each child document
and denote the pages by `\textit{child}|.|\textit{page}'
where \textit{child} represents the chapter/section number of the child file.
This can be achieved by the command
|\numberwithin{page}{|\textit{child}|}|
of the \textsf{amsmath} package
where \textit{child} can be |chapter| or |section|
depending on the chosen structuring.
Alternatively, one can modify the macro |\thepage| appropriately
and reset the counter |page| at the start of each child file.

%%%%%%%%%%%%%%%%%%%%%%%%%%%%%%%%%%%%%%%%%%%%%%%%%%%%%%%%%%%%%%%%%%%%%%%%%%%%%%%%
\subsection{Conditional Processing}
\label{sec:conditional}

The package provides a mechanism to compile different versions
of a document. To customise the versions further some conditional processing
can come in handy to distinguish which version is being compiled.
The package provides two macros to describe the compilation context:

%%%%%%%%%%%%%%%%%%%%%%%%%%%%%%%%%%%%%%%%
\DescribeMacro{\ifchilddoc}
The conditional |\ifchilddoc| distinguishes between the compilation of
child documents and the main document:
%
\begin{center}
|\ifchilddoc |\textit{child-code}| |[|\||else |\textit{main-code}]| \||fi|
\end{center}

%%%%%%%%%%%%%%%%%%%%%%%%%%%%%%%%%%%%%%%%
\DescribeMacro{\childdocname}
\DescribeMacro{\childdocjob}
The macro |\childdocname| contains the filename (without extension)
of the main or child file being processed.
Note that |\childdocjob| will always contain the name of the main file.

%%%%%%%%%%%%%%%%%%%%%%%%%%%%%%%%%%%%%%%%
\paragraph{Title Page.}

Conditional processing can be used to include a title or banner page
in the main document when proper precautions are taken.
Importantly, the code in the main file should ensure that the page counter
(as well as other status parameters which are stored in the |.aux| files)
takes the same value after the conditional processing.
Otherwise the page numbers may take divergent values
depending on which part is compiled.

For example, a title page could be declared by:
%
\begin{center}
\begin{tabular}{l}
|\ifchilddoc\||else|\\
|\addtocounter{page}{-1}|\\
\textit{code for title page}\\
|\newpage|\\
|\||fi|
\end{tabular}
\end{center}
%
A banner page for the child documents can be generated by:
%
\begin{center}
\begin{tabular}{l}
|\ifchilddoc|\\
|\addtocounter{page}{-1}|\\
\textit{code for banner page}\\
|\newpage|\\
|\||fi|
\end{tabular}
\end{center}
%
Here one could write a message such as:
\begin{center}
|This is the part \childdocname{} of \childdocjob{}.|
\end{center}

%%%%%%%%%%%%%%%%%%%%%%%%%%%%%%%%%%%%%%%%%%%%%%%%%%%%%%%%%%%%%%%%%%%%%%%%%%%%%%%%
\subsection{Flags}
\label{sec:flags}

The package makes it easy to generate different versions
of the main or child documents.
To this end compilation flags can be defined
and assigned different default values.
They will be particularly useful in conjunction
with the forwarding mechanism described in \secref{sec:forward}.

For example, it may be useful to have a flag |\version|
which can be set to |draft| or |final|.
The document source will contain some conditional code
depending on the value of |\version|.
Suppose further, the flag should default to |final| for the main file
and to |draft| for child files
which is a natural assignment for editing the document.
This is achieved by placing the following code
in the preamble of the main document
(below the |\childdocmain| directive):
%
\begin{center}
\begin{tabular}{l}
|\ifchilddoc|\\
|\providecommand{\version}{draft}|\\
|\||else|\\
|\providecommand{\version}{final}|\\
|\||fi|
\end{tabular}
\end{center}
%
The definition by |\providecommand| makes sure
that previous definitions are not overwritten.
Further statements |\providecommand{\version}{...}|
can thus be added before the above code to override it.

For the main file, one might add a line
(between |\childdocmain| and the above block)
%
\begin{center}
|%\ifchilddoc\||else\providecommand{\version}{draft}\||fi|
\end{center}
%
which can be uncommented to produce a draft version.
Likewise one can add a line to the very top of a child file
(above the |\childdocof{|\textit{main}|}| directive)
%
\begin{center}
|%\providecommand{\version}{final}|
\end{center}
%
which can be uncommented to produce the final version of this child document.

%%%%%%%%%%%%%%%%%%%%%%%%%%%%%%%%%%%%%%%%%%%%%%%%%%%%%%%%%%%%%%%%%%%%%%%%%%%%%%%%
\subsection{Forwarding}
\label{sec:forward}

Different versions of the main or child documents
using compilation flags as described in \secref{sec:flags}
can be (permanently) stored in different files
for convenient compilation, viewing and distribution.
To this end, the package defines a command
to pass on compilation to a different file:

%%%%%%%%%%%%%%%%%%%%%%%%%%%%%%%%%%%%%%%%
\DescribeMacro{\childdocforward}
The command |\childdocforward| redirects processing to
another source file:
%
\begin{center}
\begin{tabular}{l}
|\input{childdoc.def}|\\
|\childdocforward[|\textit{main}|]{|\textit{dest}|}|\\
\end{tabular}
\end{center}
%
The argument \textit{dest} is the destination file
(without extension).
It should be the main file or one of the child files.
Note that further \textsf{childdoc} directives
such as |\childdocof| and |\childdocforward|
in the indicated file will be processed in this form.
The optional argument \textit{main}
passes on directly to the main file \textit{main}
while pretending to compile the child \textit{dest}.
This form behaves as if \textit{dest}
issues |\childdocof{|\textit{main}|}| right away,
and no further \textsf{childdoc} directives will be processed.

%%%%%%%%%%%%%%%%%%%%%%%%%%%%%%%%%%%%%%%%
\DescribeMacro{\...prefix}
In the alternative form |\childdocforwardprefix|,
%
\begin{center}
\begin{tabular}{l}
|\input{childdoc.def}|\\
|\childdocforwardprefix[|\textit{main}|]{|\textit{prefix}|}{|\textit{dest}|}|
\end{tabular}
\end{center}
%
the destination file is determined by a pattern
depending on the current file:
To make this work, the current file must be called
`{\textit{prefix}\hspace{0.2em}\textit{suffix}}'
with \textit{prefix} matching precisely the argument.
Processing is then passed on to the file
`{\textit{dest}\hspace{0.2em}\textit{suffix}}'.
Surely, the same effect is achieved by
directly specifying the
argument `{\textit{dest}\hspace{0.2em}\textit{suffix}}'
in the first form.
However, that requires to set up a different file
for each child. With the alternative form of the command
all these files can have exactly the same content
which simplifies setting them up and maintaining them.

For example, the following file |draft.tex|
with a compilation flag |\version| as described in \secref{sec:flags}
compiles the main document as a draft:
%
\begin{center}
\begin{tabular}{l}
|\def\version{draft}|\\
|\input{childdoc.def}|\\
|\childdocforward{|\textit{main}|}|
\end{tabular}
\end{center}
%
Likewise, the following files |final|\textit{nn}|.tex|
compile the final version of the child document
|child|\textit{nn}|.tex|:
%
\begin{center}
\begin{tabular}{l}
|\def\version{final}|\\
|\input{childdoc.def}|\\
|\childdocforwardprefix{final}{child}|
\end{tabular}
\end{center}
%

Note that when several versions of a main file and/or of each child file
are to be generated, it may be convenient to set up a |Makefile| or
shell script to automatise the process.

%%%%%%%%%%%%%%%%%%%%%%%%%%%%%%%%%%%%%%%%%%%%%%%%%%%%%%%%%%%%%%%%%%%%%%%%%%%%%%%%
\subsection{Command Line Processing}
\label{sec:commandline}

The effect of redirection files can also be achieved by invoking
the \LaTeX{} compiler with a more elaborate command line.
Most conveniently this should be done as part
of a shell script or a |Makefile|.

When using \textsf{childdoc} in the main file, the following
command lines effectively perform a redirection
(note that depending on the shell being used,
backslashes may have to be doubled: `|\|' $\to$ `|\\|'):
%
\begin{center}
|... -jobname "|\textit{target}|" |\\|"|[\textit{flags}]%
|\input{childdoc.def}\childdocforward[|\textit{main}|]{|\textit{dest}|}"|
\end{center}
%
Here \textit{target} is the name of the output file,
\textit{main} is the name of the main file
and \textit{dest} is the name of the main or child file to be processed
(all filenames without extensions).
The optional argument \textit{main} can be omitted
if \textit{main} matches \textit{dest}.
Optionally, compilation \textit{flags} can be defined via |\def| commands.
This command line makes the \TeX{} engine believe
it is compiling the file \textit{target}
whose content is specified as the latter parameter.
The provided code then forwards the processing to
\textit{main} or \textit{dest} as described in \secref{sec:forward}.

%%%%%%%%%%%%%%%%%%%%%%%%%%%%%%%%%%%%%%%%%%%%%%%%%%%%%%%%%%%%%%%%%%%%%%%%%%%%%%%%
\subsection{Include by Input}
\label{sec:input}

Including child documents by |\include| has some restrictions by design.
Most notably, the content of a child document always occupies
its own set of pages; pages cannot be shared between child documents.
Usually, this behaviour makes perfect sense
because each child document contain an essential part of the document.
However, in some situations it may be desirable to compose
a document from a collection of parts
without having mandatory page breaks between then.
For this case, the package
provides a mechanism to include parts
by |\input| which can also be processed individually.
However, by construction this mechanism
requires manual handling of the content to be output.

%%%%%%%%%%%%%%%%%%%%%%%%%%%%%%%%%%%%%%%%
\DescribeMacro{\ifchilddocmanual}
The main file should be prepared as usual, see \secref{sec:include}.
However, the document body must make a distinction
between processing of an individual part and of the main document, e.g.:
%
\begin{center}
\begin{tabular}{l}
|\ifchilddocmanual|\\
|\input{\childdocname}|\\
|\||else|\\
\textit{document body with }|\input{|\textit{part}|}|\\
|\||fi|
\end{tabular}
\end{center}
%
The conditional |\ifchilddocmanual| is true whenever
a part to be included by |\input| is being compiled,
and the name of the part is stored in |\childdocname|.

%%%%%%%%%%%%%%%%%%%%%%%%%%%%%%%%%%%%%%%%
\DescribeMacro{\childdocby}
Each part to be included by |\input| should start with:
%
\begin{center}
\begin{tabular}{l}
|\input{childdoc.def}|\\
|\childdocby{|\textit{main}|}|\\
\end{tabular}
\end{center}
%
The directive |\childdocby| is similar to |\childdocof|
described in \secref{sec:include},
but the subsequent selection of content must be done manually.
To that end, both |\ifchilddoc| and |\ifchilddocmanual|
will be true upon processing of a part,
and the name of the part is stored in |\childdocname|.
Note that |\jobname| will be set to the filename of the current part
so that each part receives an individual |.aux| file
that does not interfere with the |.aux| file(s) of the main document.
This behaviour can be altered by the alternative form
|\childdocby[*]{|\textit{main}|}| (with a non-empty optional argument)
which uses the |.aux| file of the main document
by setting |\jobname| to \textit{main}.

%%%%%%%%%%%%%%%%%%%%%%%%%%%%%%%%%%%%%%%%%%%%%%%%%%%%%%%%%%%%%%%%%%%%%%%%%%%%%%%%
\subsection{Driver Development}
\label{sec:driver}

The \textsf{childdoc} mechanism can also be use for the development
of definition files such as \LaTeX{} styles or classes.
This case differs from the above setup with multiple parts
included by |\include| in that no |\includeonly| should be invoked.
This can be achieved by starting the include file
(before |\ProvidesPackage|) with:
%
\begin{center}
\begin{tabular}{l}
|\input{childdoc.def}|\\
|\childdocforward{|\textit{main}|}|\\
\end{tabular}
\end{center}
%
or alternatively with:
%
\begin{center}
\begin{tabular}{l}
|\input{childdoc.def}|\\
|\childdocby{|\textit{main}|}|\\
\end{tabular}
\end{center}
%
Both forms have slightly different effects as described above.
The main file is prepared as usual, see \secref{sec:include}.

%%%%%%%%%%%%%%%%%%%%%%%%%%%%%%%%%%%%%%%%%%%%%%%%%%%%%%%%%%%%%%%%%%%%%%%%%%%%%%%%
\subsection{Legacy Detection}
\label{sec:detection}

The directive |\childdocmain| in the main file can detect
whether the complete document or merely a child is to be compiled
even without using the directive |\childdocof|.
This method is deprecated because it is less robust
and there is no compelling reason to use it;
it is merely provided for backward compatibility
and it may be removed in future versions.

If the detection mechanism is to be used,
it is mandatory to correctly specify
the filename of the main file as the argument of |\childdocmain|:
%
\begin{center}
\begin{tabular}{l}
|\input{childdoc.def}|\\
|\childdocmain{|\textit{main}|}|\\
\end{tabular}
\end{center}
%
If |\jobname| does not match the argument \textit{main} of |\childdocmain|,
it is assumed that |\jobname| points to the child file to be compiled.
When using |\childdocmain| with the main file specified as argument,
it suffices to start a child file
with just |\input{|\textit{main}|}|
without loading of the package and using |\childdocof|.
If instead all processing is done
with the appropriate \textsf{childdoc} directives,
the argument of \textit{main} of |\childdocmain| can be empty.

An alternative version of the command line processing described
in \secref{sec:commandline} using the detection mechanism reads:
%
\begin{center}
|... -jobname "|\textit{target}|" "|[\textit{flags}]%
[|\def\jobname{|\textit{dest}|}|]|\input{|\textit{main}|}"|
\end{center}

%%%%%%%%%%%%%%%%%%%%%%%%%%%%%%%%%%%%%%%%%%%%%%%%%%%%%%%%%%%%%%%%%%%%%%%%%%%%%%%%
\subsection{Manual Code}
\label{sec:manual}

In case one cannot be certain whether the definitions file |childdoc.def|
is installed on the target \TeX{} distribution
and one prefers not to ship it,
it is conceivable to paste a few relevant commands into the sources.

To that end, drop all statements |\input{childdoc.def}|
and perform the replacements as outlined below.
Instead of |\childdocmain{|\textit{main}|}| add the following code
to the top of the main file:
%
\begin{center}
\begin{tabular}{l}
|\||ifdefined\childdocname\endinput\||fi\newif\ifchilddoc|\\
|\edef\childdocname{\scantokens\expandafter{\jobname\noexpand}}|\\
|\def\childdocmain{|\textit{main}|}\||ifx\childdocmain\childdocname\||else|\\
|\childdoctrue\includeonly{\childdocname}\let\jobname\childdocmain\||fi|\\
\end{tabular}
\end{center}
%
Instead of |\childdocof{|\textit{main}|}| just include the main file
at the top of each child file:
%
\begin{center}
|\input{|\textit{main}|}|
\end{center}
%
A simple redirection |\childdocforward{|\textit{dest}|}| is achieved by:
%
\begin{center}
|\def\jobname{|\textit{dest}|}\input{\jobname}|
\end{center}
%
The redirection with prefix
|\childdocforwardprefix[|\textit{prefix}|]{|\textit{dest}|}|
is accomplished by:
%
\begin{center}
\begin{tabular}{l}
|{\edef\jobname{\scantokens\expandafter{\jobname\noexpand}}|\\
|\def\redirectjob |\textit{prefix}|#1~~~{\gdef\jobname{|\textit{dest}|#1}}|\\
|\expandafter\redirectjob\jobname~~~}\input{\jobname}|
\end{tabular}
\end{center}

In an alternative approach,
child documents can be compiled by a specific command line
without additional code or specific definitions:
%
\begin{center}
|... -jobname "|\textit{target}|" "|[\textit{flags}]%
|\includeonly{|\textit{dest}|}\input{|\textit{main}|}"|
\end{center}
%

%%%%%%%%%%%%%%%%%%%%%%%%%%%%%%%%%%%%%%%%%%%%%%%%%%%%%%%%%%%%%%%%%%%%%%%%%%%%%%%%
%%%%%%%%%%%%%%%%%%%%%%%%%%%%%%%%%%%%%%%%%%%%%%%%%%%%%%%%%%%%%%%%%%%%%%%%%%%%%%%%
\section{Information}

%%%%%%%%%%%%%%%%%%%%%%%%%%%%%%%%%%%%%%%%%%%%%%%%%%%%%%%%%%%%%%%%%%%%%%%%%%%%%%%%
\subsection{Copyright}

Copyright \copyright{} 2017--2018 Niklas Beisert

This work may be distributed and/or modified under the
conditions of the \LaTeX{} Project Public License, either version 1.3
of this license or (at your option) any later version.
The latest version of this license is in
  \url{http://www.latex-project.org/lppl.txt}
and version 1.3 or later is part of all distributions of \LaTeX{}
version 2005/12/01 or later.

This work has the LPPL maintenance status `maintained'.

The Current Maintainer of this work is Niklas Beisert.

This work consists of the files |README.txt|, |childdoc.ins| and |childdoc.dtx|
as well as the derived files |childdoc.def|, |cdocsamp.tex|
with |cdocsch1.tex|, |cdocsch2.tex|, |cdocspt3.tex|, |cdocspt4.tex|,
|cdocsdrf.tex|, |cdocsfn1.tex|, |cdocsfn2.tex|
as well as |childdoc.pdf|.

%%%%%%%%%%%%%%%%%%%%%%%%%%%%%%%%%%%%%%%%%%%%%%%%%%%%%%%%%%%%%%%%%%%%%%%%%%%%%%%%
\subsection{Files and Installation}

The package consists of the files:
%
\begin{center}
\begin{tabular}{ll}
    |README.txt|   & readme file \\
    |childdoc.ins| & installation file \\
    |childdoc.dtx| & source file \\
    |childdoc.def| & definition file \\
    |cdocsamp.tex| & sample main file \\
    |cdocsch1.tex| & sample include file \\
    |cdocsch2.tex| & sample include file \\
    |cdocspt3.tex| & sample part file \\
    |cdocspt4.tex| & sample part file \\
    |cdocsdrf.tex| & sample redirection file \\
    |cdocsfn1.tex| & sample redirection file \\
    |cdocsfn2.tex| & sample redirection file \\
    |childdoc.pdf| & manual
\end{tabular}
\end{center}
%
The distribution consists of the files
|README.txt|, |childdoc.ins| and |childdoc.dtx|.
%
\begin{itemize}
\item
Run (pdf)\LaTeX{} on |childdoc.dtx|
to compile the manual |childdoc.pdf| (this file).
\item
Run \LaTeX{} on |childdoc.ins| to create the definitions file |childdoc.def|
and the sample |cdocsamp.tex| with include files
|cdocsch1.tex|, |cdocsch2.tex|, |cdocspt3.tex|, |cdocspt4.tex|,
|cdocsdrf.tex|, |cdocsfn1.tex|, |cdocsfn2.tex|.
Then copy the file |childdoc.def| to an appropriate directory of your \LaTeX{}
distribution, e.g.\ \textit{texmf-root}|/tex/latex/childdoc|.
\end{itemize}

%%%%%%%%%%%%%%%%%%%%%%%%%%%%%%%%%%%%%%%%%%%%%%%%%%%%%%%%%%%%%%%%%%%%%%%%%%%%%%%%
\subsection{Related CTAN Packages}

There are several other packages which offer a similar functionality:
%
\begin{itemize}
\item
The packages
\href{http://ctan.org/pkg/docmute}{\textsf{docmute}},
\href{http://ctan.org/pkg/includex}{\textsf{includex}} and
\href{http://ctan.org/pkg/standalone}{\textsf{standalone}}
provide commands to include only the document body of
a child file thus allowing both files to be compiled individually.
\item
The packages \href{http://ctan.org/pkg/subdocs}{\textsf{subdocs}}
and \href{http://ctan.org/pkg/subfiles}{\textsf{subfiles}}
provide structures in which the main and child documents can be
encapsulated and allowing them to be compiled individually.
The inclusion mechanism is different from the conventional |\include|.
\item
The package \href{http://ctan.org/pkg/combine}{\textsf{combine}}
is an elaborate solution to combine several documents into one.
\end{itemize}
%
See also the CTAN topic \href{http://ctan.org/topic/subdocs}{\textsf{subdocs}}
for further related packages.
The present package differs from the above solutions in that
a document structure constructed with the conventional |\include| mechanism
just needs two extra commands at the top of every file
such that all constituent files can be compiled individually.

%%%%%%%%%%%%%%%%%%%%%%%%%%%%%%%%%%%%%%%%%%%%%%%%%%%%%%%%%%%%%%%%%%%%%%%%%%%%%%%%
%\subsection{Feature Suggestions}
%
%The following is a list of features which may be useful for future
%versions of this package:
%%
%\begin{itemize}
%\item
%\ldots
%\end{itemize}

%%%%%%%%%%%%%%%%%%%%%%%%%%%%%%%%%%%%%%%%%%%%%%%%%%%%%%%%%%%%%%%%%%%%%%%%%%%%%%%%
\subsection{Revision History}

%%%%%%%%%%%%%%%%%%%%%%%%%%%%%%%%%%%%%%%%
\paragraph{v2.0:} 2018/12/30

\begin{itemize}
\item
immediate forward processing
\item
added |\childdocby| mechanism
\item
manual restructured
\end{itemize}

%%%%%%%%%%%%%%%%%%%%%%%%%%%%%%%%%%%%%%%%
\paragraph{v1.6:} 2018/01/17

\begin{itemize}
\item
application for development of include files
\item
corrections to manual
\end{itemize}

%%%%%%%%%%%%%%%%%%%%%%%%%%%%%%%%%%%%%%%%
\paragraph{v1.5:} 2017/05/21

\begin{itemize}
\item
more complete structuring introduced
\item
|\childdocof| introduced
\item
|\childdoc| renamed to |\childdocmain|
\item
|\childredirect| renamed to |\childdocforward| and |\childdocforwardprefix|
and functionality expanded
\end{itemize}

%%%%%%%%%%%%%%%%%%%%%%%%%%%%%%%%%%%%%%%%
\paragraph{v1.0:} 2017/04/27

\begin{itemize}
\item
manual and install package
\item
first version published on CTAN
\end{itemize}

%%%%%%%%%%%%%%%%%%%%%%%%%%%%%%%%%%%%%%%%
\paragraph{v0.6:} 2017/04/26

\begin{itemize}
\item
redirection mechanism added
\end{itemize}

%%%%%%%%%%%%%%%%%%%%%%%%%%%%%%%%%%%%%%%%
\paragraph{v0.5:} 2017/04/26

\begin{itemize}
\item
functionality in definition file
\end{itemize}


%%%%%%%%%%%%%%%%%%%%%%%%%%%%%%%%%%%%%%%%%%%%%%%%%%%%%%%%%%%%%%%%%%%%%%%%%%%%%%%%
%%%%%%%%%%%%%%%%%%%%%%%%%%%%%%%%%%%%%%%%%%%%%%%%%%%%%%%%%%%%%%%%%%%%%%%%%%%%%%%%
%%%%%%%%%%%%%%%%%%%%%%%%%%%%%%%%%%%%%%%%%%%%%%%%%%%%%%%%%%%%%%%%%%%%%%%%%%%%%%%%
\appendix

\settowidth\MacroIndent{\rmfamily\scriptsize 000\ }

 \DocInput{childdoc.dtx}

\end{document}
%</driver>
% \fi
%
% %%%%%%%%%%%%%%%%%%%%%%%%%%%%%%%%%%%%%%%%%%%%%%%%%%%%%%%%%%%%%%%%%%%%%%%%%%%%%%
% %%%%%%%%%%%%%%%%%%%%%%%%%%%%%%%%%%%%%%%%%%%%%%%%%%%%%%%%%%%%%%%%%%%%%%%%%%%%%%
% \section{Sample}
%\iffalse
%<*samplemain>
%\fi
%
% The following presents a sample document
% with two chapters, two parts, a title page,
% a compile flag as well as three forwarding files to set the flag.
% It consists of eight |.tex| files:
% \begin{center}
% \begin{tabular}{ll}
% |cdocsamp.tex|&main file\\
% |cdocsch1.tex|&include file for chapter 1\\
% |cdocsch2.tex|&include file for chapter 2\\
% |cdocspt3.tex|&include file for part 3\\
% |cdocspt4.tex|&include file for part 4\\
% |cdocsdrf.tex|&forwarding file for main file in draft mode\\
% |cdocsfi1.tex|&forwarding file for final version of chapter 1\\
% |cdocsfi2.tex|&forwarding file for final version of chapter 2\\
% \end{tabular}
% \end{center}
% Each of the eight files can be compiled directly by the \LaTeX{} compiler.
%
% %%%%%%%%%%%%%%%%%%%%%%%%%%%%%%%%%%%%%%
% \paragraph{Main File.}
%
% The main file is called |cdocsamp.tex|.
%
% Load the \textsf{childdoc} definitions and
% declare the filename for the main document:
%    \begin{macrocode}
\input{childdoc.def}
\childdocmain{}
%    \end{macrocode}

% Optional override for |\version| flag:
%    \begin{macrocode}
%%\ifchilddoc\else\providecommand{\version}{draft}\fi
%    \end{macrocode}

% Define the default values for the |\version| flag
% (|final| for the main file and |draft| for childs):
%    \begin{macrocode}
\ifchilddoc
\providecommand{\version}{draft}
\else
\providecommand{\version}{final}
\fi
%    \end{macrocode}

% Load the standard document class:
%    \begin{macrocode}
\documentclass[12pt]{article}
%    \end{macrocode}

% Start the document body:
%    \begin{macrocode}
\begin{document}
%    \end{macrocode}

% Declare a title page.
% Print title, part of document being processed and version flag:
%    \begin{macrocode}
\addtocounter{page}{-1}
\begin{center}
{\LARGE\bfseries{}childdoc example\par}
\vspace{1cm}
\ifchilddoc
\ifchilddocmanual part\else chapter\fi:
`\childdocname' of `\childdocjob'\par
\else
main document: `\childdocjob'\par
\fi
version: \version\par
\end{center}
\newpage
%    \end{macrocode}

% Manually include selected file,
% otherwise process as usual:
%    \begin{macrocode}
\ifchilddocmanual
\section*{part `\childdocname'}
\input{\childdocname}
\else
%    \end{macrocode}

% Include the two chapters:
%    \begin{macrocode}
\include{cdocsch1}
\include{cdocsch2}
%    \end{macrocode}

% Include the two parts unless only chapters should be displayed:
%    \begin{macrocode}
\ifchilddoc\else
\section{part three}
\input{cdocspt3}
\section{part four}
\input{cdocspt4}
\fi
%    \end{macrocode}

% Process as usual until here:
%    \begin{macrocode}
\fi
%    \end{macrocode}

% End of document body:
%    \begin{macrocode}
\end{document}
%    \end{macrocode}
%\iffalse
%</samplemain>
%\fi
%
% %%%%%%%%%%%%%%%%%%%%%%%%%%%%%%%%%%%%%%
% \paragraph{Chapter Include Files.}
%
% The include files are called |cdocsch1.tex| and |cdocsch2.tex|.
%
%\iffalse
%<*samplechap1|samplechap2>
%\fi

% Optional override for |\version| flag:
%    \begin{macrocode}
%%\providecommand{\version}{final}
%    \end{macrocode}

% Include the main document:
%    \begin{macrocode}
\input{childdoc.def}
\childdocof{cdocsamp}
%    \end{macrocode}

%\iffalse
%</samplechap1|samplechap2>
%\fi
%
%\iffalse
%<*samplechap1>
%\fi
% Some text for chapter 1:
%    \begin{macrocode}
\section{one}
some text in chapter one
%    \end{macrocode}

%\iffalse
%</samplechap1>
%\fi
% Some text for chapter 2:
%\iffalse
%<*samplechap2>
%\fi
%    \begin{macrocode}
\section{two}
more text in chapter two
%    \end{macrocode}

%\iffalse
%</samplechap2>
%\fi
%
% %%%%%%%%%%%%%%%%%%%%%%%%%%%%%%%%%%%%%%
% \paragraph{Part Include Files.}
%
% The include files are called |cdocspt3.tex| and |cdocspt4.tex|.
%
%\iffalse
%<*samplepart3|samplepart4>
%\fi

% Optional override for |\version| flag:
%    \begin{macrocode}
%%\providecommand{\version}{final}
%    \end{macrocode}

% Include the main document:
%    \begin{macrocode}
\input{childdoc.def}
\childdocby{cdocsamp}
%    \end{macrocode}

%\iffalse
%</samplepart3|samplepart4>
%\fi
%
%\iffalse
%<*samplepart3>
%\fi
% Some text for part 3:
%    \begin{macrocode}
some text in part three
%    \end{macrocode}

%\iffalse
%</samplepart3>
%\fi
% Some text for part 4:
%\iffalse
%<*samplepart4>
%\fi
%    \begin{macrocode}
more text in part four
%    \end{macrocode}

%\iffalse
%</samplepart4>
%\fi
%
% %%%%%%%%%%%%%%%%%%%%%%%%%%%%%%%%%%%%%%
% \paragraph{Forwarding for a Complete Draft.}
%
% The following forwarding file |cdocsdrf.tex|
% compiles the main document in draft mode:
%\iffalse
%<*sampledraft>
%\fi
%    \begin{macrocode}
\def\version{draft}
\input{childdoc.def}
\childdocforward{cdocsamp}
%    \end{macrocode}

%\iffalse
%</sampledraft>
%\fi
%
% %%%%%%%%%%%%%%%%%%%%%%%%%%%%%%%%%%%%%%
% \paragraph{Forwarding for Final Version of the Chapters.}
%
% The following forwarding files |cdocsfn1.tex| and |cdocsfn2.tex|
% (with identical content)
% compile the final versions of the child documents
% |cdocsch1.tex| and |cdocsch2.tex|, respectively:
%\iffalse
%<*samplefinal>
%\fi
%    \begin{macrocode}
\def\version{final}
\input{childdoc.def}
\childdocforwardprefix[cdocsamp]{cdocsfn}{cdocsch}
%    \end{macrocode}

%\iffalse
%</samplefinal>
%\fi
%
% %%%%%%%%%%%%%%%%%%%%%%%%%%%%%%%%%%%%%%
% \paragraph{Command Line Processing.}
%
% The following three command lines generate the output files
% |cdocscld|, |cdocscl1| and |cdocscl2|
% which should be identical to
% |cdocsdrf|, |cdocsch1| and |cdocsfn2|, respectively:
% \begin{center}
% \begin{tabular}{l}
% |latex -jobname cdocscld \|\\
% |  "\def\version{draft}\input{childdoc.def}\childdocforward{cdocsamp}"|\\
% |latex -jobname cdocscl1 \|\\
% |  "\input{childdoc.def}\childdocforward[cdocsamp]{cdocsch1}"|\\
% |latex -jobname cdocscl2 \|\\
% |  "\def\version{final}\input{childdoc.def}\childdocforward{cdocsch2}"|
% \end{tabular}
% \end{center}
% Note that the trailing backslash on each first line
% merely continues the input to the second line
% (for convenient cut ant paste).
% Furthermore, the command |latex| can be replaced by any
% of its alternative versions such as |pdflatex|.
%
% %%%%%%%%%%%%%%%%%%%%%%%%%%%%%%%%%%%%%%%%%%%%%%%%%%%%%%%%%%%%%%%%%%%%%%%%%%%%%%
% %%%%%%%%%%%%%%%%%%%%%%%%%%%%%%%%%%%%%%%%%%%%%%%%%%%%%%%%%%%%%%%%%%%%%%%%%%%%%%
% \section{Implementation}
%\iffalse
%<*package>
%\fi
%
% This section describes the definitions file |childdoc.def|.

% The definitions cannot be loaded using |\usepackage| or |\RequirePackage|
% which has a mechanism to prevent loading a style file more than once.
% When loading the definitions by means of |\input|
% multiple instances have to be prevented manually:
%\iffalse
%This code needs to be before the `\ProvidesFile' directive
%which is defined at the beginning of this file.
%Therefore it is also placed there and commented out here.
%</package>
%<*discard>
%\fi
%    \begin{macrocode}
\ifdefined\childdocmain\endinput\fi
%    \end{macrocode}
%\iffalse
%</discard>
%<*package>
%\fi
%
% \macro{\ifchilddoc}
% \macro{\ifchilddocmanual}
% The conditional |\ifchilddoc| tells whether a
% child (true) or main (false) document is being compiled.
% The conditional |\ifchilddocmanual| tells whether
% the |\includeonly| mechanism is used (false) or
% the selection of child files must be performed manually (true).
% The definitions initialise to false:
%    \begin{macrocode}
\newif\ifchilddoc
\newif\ifchilddocmanual
%    \end{macrocode}

% \macro{\childdocname}
% \macro{\childdocjob}
% The macro |\childdocname| stores the name of the main document
% to be compiled. The macro |\childdocjob| stores the name of
% the document on which the \LaTeX{} compiler was originally invoked.
% The content of |\jobname| cannot be compared
% to filenames specified in the source due to different catcodes.
% The following code rescans |\jobname|, stores the result
% in |\childdocname| and saves a copy in |\childdocjob|:
%    \begin{macrocode}
\edef\childdocname{\scantokens\expandafter{\jobname\noexpand}}
\let\childdocjob\childdocname
%    \end{macrocode}

% \macro{\childdocdisable}
% The macro |\childdocdisable| prevents the main file
% from being processed more than once.
% At this stage, the main document command |\childdocmain|
% is assumed to be called once again where it should do nothing.
% Any subsequent call to it should prevent
% a secondary processing of the main document
% It overwrites the forwarding commands
% |\childdocof| and |\childdocforward|
% with empty macros to prevent further inclusions of the main document:
%    \begin{macrocode}
\newcommand{\childdocdisable}
{
  \renewcommand{\childdocmain}[1]{\renewcommand{\childdocmain}[1]{\endinput}}
  \renewcommand{\childdocof}[1]{}
  \renewcommand{\childdocby}[2][]{}
  \renewcommand{\childdocforward}[2][]{}
  \renewcommand{\childdocdisable}{}
}
%    \end{macrocode}

% \macro{\childdocmain}
% The macro |\childdocmain| is to be called at the top of the main file
% with nothing or the main filename (without extension) as argument.
% First, it breaks loops.
% If the argument is not empty and does not match |\childdocname|
% (which is set by the first inclusion of |childdoc.def|),
% |\ifchilddoc| is set to true, |\includeonly| is applied to the child file
% and |\jobname| is set to the main file
% (for proper handling of |.aux| files):
%    \begin{macrocode}
\newcommand{\childdocmain}[1]
{
  \childdocdisable\childdocmain{}
  \if?#1?\else
    \begingroup
      \def\childdoctmp{#1}
      \ifx\childdoctmp\childdocname
        \def\childdoctmp{}
      \else
        \def\childdoctmp
        {
          \childdoctrue
          \includeonly{\childdocname}
          \def\childdocjob{#1}
          \def\jobname{#1}
        }
      \fi
      \expandafter
    \endgroup
    \childdoctmp
  \fi
}
%    \end{macrocode}

% \macro{\childdocof}
% The command |\childdocof| redirects
% compilation to the main file |#1|.
%    \begin{macrocode}
\newcommand{\childdocof}[1]
{
  \childdocdisable
  \childdoctrue
  \includeonly{\childdocname}
  \def\jobname{#1}
  \def\childdocjob{#1}
  \input{#1}
}
%    \end{macrocode}

% \macro{\childdocby}
% The command |\childdocby| ....
%    \begin{macrocode}
\newcommand{\childdocby}[2][]
{
  \childdocdisable
  \childdoctrue
  \childdocmanualtrue
  \if?#1?\else
    \def\jobname{#2}
  \fi
  \def\childdocjob{#2}
  \input{#2}
  \endinput
}
%    \end{macrocode}

% \macro{\childdocforward}
% The command |\childdocforward| redirects
% compilation to the main file or
% (if the optional argument is given) a child file.
% Parameters are set as if the main file
% or a child file starting with |\childdocof| was compiled.
% Then compilation is handed over to the main file:
%    \begin{macrocode}
\newcommand{\childdocforward}[2][]
{
  \begingroup
    \if?#1?
      \def\childdoctmp
      {
        \def\childdocname{#2}
        \def\childdocjob{#2}
        \def\jobname{#2}
        \input{#2}
        \endinput
      }
    \else
      \def\childdoctmp
      {
        \childdocdisable
        \def\childdocname{#2}
        \childdoctrue
        \includeonly{#2}
        \def\childdocjob{#1}
        \def\jobname{#1}
        \input{#1}
        \endinput
      }
    \fi
    \expandafter
  \endgroup
  \childdoctmp
}
%    \end{macrocode}

% \macro{\childdocforwardprefix}
% The command |\childdocforwardprefix| redirects
% compilation to the main or a child file by means of a pattern.
% The prefix |#1| in the current filename is replaced by |#2|
% and the suffix of the current filename is kept
% (it is assumed that the filename does not contain the substring `|~~~|'
% which is used as a delimiter).
% Compilation is handed over to the new file by |\childdocforward|:
%    \begin{macrocode}
\newcommand{\childdocforwardprefix}[3][]
{
  \begingroup
    \def\childdocextract #2##1~~~{\def\childdoctmp{\childdocforward[#1]{#3##1}}}
    \expandafter\childdocextract\childdocname~~~
    \expandafter
  \endgroup
  \childdoctmp
}
%    \end{macrocode}

% \macro{\childdoc}
% The deprecated macro |\childdoc| is a legacy version of |\childdocmain|:
%    \begin{macrocode}
\newcommand{\childdoc}{\childdocmain}
%    \end{macrocode}

% \macro{\childdocredirect}
% The deprecated macro |\childdocredirect| is a legacy version
% of |\childdocforward| and |\childdocforwardprefix|:
%    \begin{macrocode}
\newcommand{\childdocredirect}[2][]
{
  \begingroup
    \if?#1?
      \def\childdoctmp{\childdocforward{#2}}
    \else
      \def\childdoctmp{\childdocforwardprefix{#1}{#2}}
    \fi
    \expandafter
  \endgroup
  \childdoctmp
}
%    \end{macrocode}

%\iffalse
%</package>
%\fi
%
\endinput

\childdocby{cdocsamp}
%    \end{macrocode}

%\iffalse
%</samplepart3|samplepart4>
%\fi
%
%\iffalse
%<*samplepart3>
%\fi
% Some text for part 3:
%    \begin{macrocode}
some text in part three
%    \end{macrocode}

%\iffalse
%</samplepart3>
%\fi
% Some text for part 4:
%\iffalse
%<*samplepart4>
%\fi
%    \begin{macrocode}
more text in part four
%    \end{macrocode}

%\iffalse
%</samplepart4>
%\fi
%
% %%%%%%%%%%%%%%%%%%%%%%%%%%%%%%%%%%%%%%
% \paragraph{Forwarding for a Complete Draft.}
%
% The following forwarding file |cdocsdrf.tex|
% compiles the main document in draft mode:
%\iffalse
%<*sampledraft>
%\fi
%    \begin{macrocode}
\def\version{draft}
% \iffalse
%
% childdoc.dtx Copyright (C) 2017-2018 Niklas Beisert
%
% This work may be distributed and/or modified under the
% conditions of the LaTeX Project Public License, either version 1.3
% of this license or (at your option) any later version.
% The latest version of this license is in
%   http://www.latex-project.org/lppl.txt
% and version 1.3 or later is part of all distributions of LaTeX
% version 2005/12/01 or later.
%
% This work has the LPPL maintenance status `maintained'.
%
% The Current Maintainer of this work is Niklas Beisert.
%
% This work consists of the files childdoc.dtx and childdoc.ins
% and the derived files childdoc.def and cdocsamp.tex with
% cdocsch1.tex, cdocsch2.tex, cdocsdrf.tex, cdocsfn1.tex, cdocsfn2.tex.
%
%<package>\ifdefined\childdocmain\endinput\fi
%<package>\ProvidesFile{childdoc.def}[2018/12/30 v2.0 child document driver]
%<samplemain>\ProvidesFile{cdocsamp.tex}[2018/12/30 v2.0 sample for childdoc]
%<*driver>
%\ProvidesFile{childdoc.drv}[2018/12/30 v2.0 childdoc reference manual file]
\PassOptionsToClass{10pt,a4paper}{article}
\documentclass{ltxdoc}

\usepackage[margin=35mm]{geometry}
\usepackage{hyperref}
\usepackage{hyperxmp}
\usepackage[usenames]{color}

\hypersetup{colorlinks=true}
\hypersetup{pdfstartview=FitH}
\hypersetup{pdfpagemode=UseNone}
\hypersetup{pdfsource={}}
\hypersetup{pdflang={en-UK}}
\hypersetup{pdfcopyright={Copyright 2017-2018 Niklas Beisert.
  This work may be distributed and/or modified under the
  conditions of the LaTeX Project Public License, either version 1.3
  of this license or (at your option) any later version.}}
\hypersetup{pdflicenseurl={http://www.latex-project.org/lppl.txt}}
\hypersetup{pdfcontactaddress={ETH Zurich, ITP, HIT K,
  Wolfgang-Pauli-Strasse 27}}
\hypersetup{pdfcontactpostcode={8093}}
\hypersetup{pdfcontactcity={Zurich}}
\hypersetup{pdfcontactcountry={Switzerland}}
\hypersetup{pdfcontactemail={nbeisert@itp.phys.ethz.ch}}
\hypersetup{pdfcontacturl={http://people.phys.ethz.ch/\xmptilde nbeisert/}}

\newcommand{\secref}[1]{\hyperref[#1]{section \ref*{#1}}}

\parskip1ex
\parindent0pt
\let\olditemize\itemize
\def\itemize{\olditemize\parskip0pt}

\begin{document}

\title{The \textsf{childdoc} Package}
\hypersetup{pdftitle={The childdoc Package}}
\author{Niklas Beisert\\[2ex]
  Institut f\"ur Theoretische Physik\\
  Eidgen\"ossische Technische Hochschule Z\"urich\\
  Wolfgang-Pauli-Strasse 27, 8093 Z\"urich, Switzerland\\[1ex]
  \href{mailto:nbeisert@itp.phys.ethz.ch}
  {\texttt{nbeisert@itp.phys.ethz.ch}}}
\hypersetup{pdfauthor={Niklas Beisert}}
\hypersetup{pdfsubject={Manual for the LaTeX2e Package childdoc}}
\date{30 December 2018, \textsf{v2.0}}
\maketitle

\begin{abstract}\noindent
\textsf{childdoc} is a \LaTeXe{} package
that enables the direct compilation
of document sections included by |\include|
to individual files.
\end{abstract}

\begingroup
\parskip0ex
\tableofcontents
\endgroup

%%%%%%%%%%%%%%%%%%%%%%%%%%%%%%%%%%%%%%%%%%%%%%%%%%%%%%%%%%%%%%%%%%%%%%%%%%%%%%%%
%%%%%%%%%%%%%%%%%%%%%%%%%%%%%%%%%%%%%%%%%%%%%%%%%%%%%%%%%%%%%%%%%%%%%%%%%%%%%%%%
\section{Introduction}

\LaTeX{} provides a mechanism to structure a large document (such as a book)
into a main file and several child files (containing the chapters)
using the |\include| command.
This mechanism is beneficial for documents
which span hundreds of pages in order to
make the source file(s) more manageable.
Moreover, compilation can be restricted to
selected child files by means of the |\includeonly| command.
The latter feature can be used to reduce the compilation time while editing
(this was significantly more useful in the earlier days of \LaTeX{})
or to generate a smaller document which is easier to navigate.
Another application of |\includeonly| is to generate
documents consisting of selected parts of the complete document.

However, there are a few drawbacks of the plain |\include| mechanism:
\begin{itemize}
\item
The child files cannot be compiled on their own,
they can only be compiled via the main file.
A naive editing environment
(such as a text editor with an option
to have the current file processed by \LaTeX)
may require one to switch to the main file before compiling;
attempting to compile the child file produces errors.
\item
The main file must be modified (each time)
to adjust the |\includeonly| command
to the present needs. This easily leaves the main file in a messy state.
\item
The generated document will always carry the filename
of the main document. This is inconvenient if
several child files are to be compiled and
to be kept for distribution.
\end{itemize}

The present package provides a simple interface
to make child files individually compilable by \LaTeX{}.
Compiling a child file then has the same effect as compiling
the main file with an |\includeonly| command
to select the appropriate child.
Moreover the generated document will carry the name of the child
rather than the main file.
This resolves all three above issues.

This feature is meant to make the editing of books,
thesis documents and lecture notes somewhat more convenient.
However, the package can also be used efficiently for
composing a series of documents (such as exercise sheets)
which are typically distributed individually.
It then assists the author in generating the individual documents
(potentially in different versions)
as well as a document containing the collected series.
Another application is in developing style files
or other kinds of included material
where compilation of the style file could redirect
to a sample or test file.

%%%%%%%%%%%%%%%%%%%%%%%%%%%%%%%%%%%%%%%%%%%%%%%%%%%%%%%%%%%%%%%%%%%%%%%%%%%%%%%%
%%%%%%%%%%%%%%%%%%%%%%%%%%%%%%%%%%%%%%%%%%%%%%%%%%%%%%%%%%%%%%%%%%%%%%%%%%%%%%%%
\section{Usage}

First of all, the package \textsf{childdoc} is \emph{not} a standard
\LaTeXe{} |.sty| style file! Therefore it needs to be invoked in
a non-standard way.

%%%%%%%%%%%%%%%%%%%%%%%%%%%%%%%%%%%%%%%%%%%%%%%%%%%%%%%%%%%%%%%%%%%%%%%%%%%%%%%%
\subsection{Included Files}
\label{sec:include}

%%%%%%%%%%%%%%%%%%%%%%%%%%%%%%%%%%%%%%%%
\DescribeMacro{\childdocmain}
To use the package, add the commands
\begin{center}
\begin{tabular}{l}
|\input{childdoc.def}|\\
|\childdocmain{}|\\
\end{tabular}
\end{center}
at the very top of the main \LaTeX{} file,
in particular \emph{before} the |\documentclass| statement!
The argument of |\childdocmain| should be left empty
(but it must be present).

%%%%%%%%%%%%%%%%%%%%%%%%%%%%%%%%%%%%%%%%
\DescribeMacro{\childdocof}
Furthermore, add the commands
\begin{center}
\begin{tabular}{l}
|\input{childdoc.def}|\\
|\childdocof{|\textit{main}|}|\\
\end{tabular}
\end{center}
at the top of every child file \textit{child}
which is included by |\include{|\textit{child}|}|
from within the main file
(or at least for those files to be compiled individually).
The argument \textit{main} must be the filename of the main file.

There are a couple of
considerations in setting up the main and child documents:

%%%%%%%%%%%%%%%%%%%%%%%%%%%%%%%%%%%%%%%%
\paragraph{Restrictions.}

Please note the following restrictions:
\begin{itemize}
\item
|\childdocmain| must be called with one argument \textit{main}
to ensure compatibility with earlier version of the package.
It must either be empty (|\childdocmain{}|)
or precisely match the filename of the main file in which it is specified.
See \secref{sec:detection} for further information.
\item
The filename \textit{main} must be specified without the |.tex| extension.
\item
The filename \textit{main} is case sensitive
(even in case-insensitive file systems)
due to internal string comparison.
\item
The argument \textit{main} should be fully expanded, it cannot be a macro.
\item
Subdirectories and special characters should be avoided in filenames.
\item
The command |\childdocmain{|\textit{main}|}| must be followed by a whitespace.
It should not be followed immediately by another command
or by a comment mark `|%|'.
This is because the \TeX{} parser reads the token immediately following
the argument of |\childdocmain| and puts it
at the beginning of every child section;
however, a white\-space is ignored.
\end{itemize}

%%%%%%%%%%%%%%%%%%%%%%%%%%%%%%%%%%%%%%%%
\paragraph{Content of Main File.}

It is advisable to place all content in the child files included by |\include|.
Any output contained in the main file will appear in all child documents
unless suppressed manually;
it cannot be suppressed automatically by the |\includeonly| directive
and thus should normally be avoided.
A method to include some content in the main file
by means of conditional processing is described in \secref{sec:conditional}.

%%%%%%%%%%%%%%%%%%%%%%%%%%%%%%%%%%%%%%%%
\paragraph{Page Numbering.}

When only a part of the document is compiled,
the appropriate numbering of pages
(as well as other status parameters)
is determined from the |.aux| files.
The latter contain information from previous passes.
However this information needs to propagate through
all intermediate child documents.
Therefore the page numbering in child documents may well
be inconsistent until the complete document is compiled at least once.

A useful (if unconventional) way to always ensure a consistent
page numbering is to restart the numbering in each child document
and denote the pages by `\textit{child}|.|\textit{page}'
where \textit{child} represents the chapter/section number of the child file.
This can be achieved by the command
|\numberwithin{page}{|\textit{child}|}|
of the \textsf{amsmath} package
where \textit{child} can be |chapter| or |section|
depending on the chosen structuring.
Alternatively, one can modify the macro |\thepage| appropriately
and reset the counter |page| at the start of each child file.

%%%%%%%%%%%%%%%%%%%%%%%%%%%%%%%%%%%%%%%%%%%%%%%%%%%%%%%%%%%%%%%%%%%%%%%%%%%%%%%%
\subsection{Conditional Processing}
\label{sec:conditional}

The package provides a mechanism to compile different versions
of a document. To customise the versions further some conditional processing
can come in handy to distinguish which version is being compiled.
The package provides two macros to describe the compilation context:

%%%%%%%%%%%%%%%%%%%%%%%%%%%%%%%%%%%%%%%%
\DescribeMacro{\ifchilddoc}
The conditional |\ifchilddoc| distinguishes between the compilation of
child documents and the main document:
%
\begin{center}
|\ifchilddoc |\textit{child-code}| |[|\||else |\textit{main-code}]| \||fi|
\end{center}

%%%%%%%%%%%%%%%%%%%%%%%%%%%%%%%%%%%%%%%%
\DescribeMacro{\childdocname}
\DescribeMacro{\childdocjob}
The macro |\childdocname| contains the filename (without extension)
of the main or child file being processed.
Note that |\childdocjob| will always contain the name of the main file.

%%%%%%%%%%%%%%%%%%%%%%%%%%%%%%%%%%%%%%%%
\paragraph{Title Page.}

Conditional processing can be used to include a title or banner page
in the main document when proper precautions are taken.
Importantly, the code in the main file should ensure that the page counter
(as well as other status parameters which are stored in the |.aux| files)
takes the same value after the conditional processing.
Otherwise the page numbers may take divergent values
depending on which part is compiled.

For example, a title page could be declared by:
%
\begin{center}
\begin{tabular}{l}
|\ifchilddoc\||else|\\
|\addtocounter{page}{-1}|\\
\textit{code for title page}\\
|\newpage|\\
|\||fi|
\end{tabular}
\end{center}
%
A banner page for the child documents can be generated by:
%
\begin{center}
\begin{tabular}{l}
|\ifchilddoc|\\
|\addtocounter{page}{-1}|\\
\textit{code for banner page}\\
|\newpage|\\
|\||fi|
\end{tabular}
\end{center}
%
Here one could write a message such as:
\begin{center}
|This is the part \childdocname{} of \childdocjob{}.|
\end{center}

%%%%%%%%%%%%%%%%%%%%%%%%%%%%%%%%%%%%%%%%%%%%%%%%%%%%%%%%%%%%%%%%%%%%%%%%%%%%%%%%
\subsection{Flags}
\label{sec:flags}

The package makes it easy to generate different versions
of the main or child documents.
To this end compilation flags can be defined
and assigned different default values.
They will be particularly useful in conjunction
with the forwarding mechanism described in \secref{sec:forward}.

For example, it may be useful to have a flag |\version|
which can be set to |draft| or |final|.
The document source will contain some conditional code
depending on the value of |\version|.
Suppose further, the flag should default to |final| for the main file
and to |draft| for child files
which is a natural assignment for editing the document.
This is achieved by placing the following code
in the preamble of the main document
(below the |\childdocmain| directive):
%
\begin{center}
\begin{tabular}{l}
|\ifchilddoc|\\
|\providecommand{\version}{draft}|\\
|\||else|\\
|\providecommand{\version}{final}|\\
|\||fi|
\end{tabular}
\end{center}
%
The definition by |\providecommand| makes sure
that previous definitions are not overwritten.
Further statements |\providecommand{\version}{...}|
can thus be added before the above code to override it.

For the main file, one might add a line
(between |\childdocmain| and the above block)
%
\begin{center}
|%\ifchilddoc\||else\providecommand{\version}{draft}\||fi|
\end{center}
%
which can be uncommented to produce a draft version.
Likewise one can add a line to the very top of a child file
(above the |\childdocof{|\textit{main}|}| directive)
%
\begin{center}
|%\providecommand{\version}{final}|
\end{center}
%
which can be uncommented to produce the final version of this child document.

%%%%%%%%%%%%%%%%%%%%%%%%%%%%%%%%%%%%%%%%%%%%%%%%%%%%%%%%%%%%%%%%%%%%%%%%%%%%%%%%
\subsection{Forwarding}
\label{sec:forward}

Different versions of the main or child documents
using compilation flags as described in \secref{sec:flags}
can be (permanently) stored in different files
for convenient compilation, viewing and distribution.
To this end, the package defines a command
to pass on compilation to a different file:

%%%%%%%%%%%%%%%%%%%%%%%%%%%%%%%%%%%%%%%%
\DescribeMacro{\childdocforward}
The command |\childdocforward| redirects processing to
another source file:
%
\begin{center}
\begin{tabular}{l}
|\input{childdoc.def}|\\
|\childdocforward[|\textit{main}|]{|\textit{dest}|}|\\
\end{tabular}
\end{center}
%
The argument \textit{dest} is the destination file
(without extension).
It should be the main file or one of the child files.
Note that further \textsf{childdoc} directives
such as |\childdocof| and |\childdocforward|
in the indicated file will be processed in this form.
The optional argument \textit{main}
passes on directly to the main file \textit{main}
while pretending to compile the child \textit{dest}.
This form behaves as if \textit{dest}
issues |\childdocof{|\textit{main}|}| right away,
and no further \textsf{childdoc} directives will be processed.

%%%%%%%%%%%%%%%%%%%%%%%%%%%%%%%%%%%%%%%%
\DescribeMacro{\...prefix}
In the alternative form |\childdocforwardprefix|,
%
\begin{center}
\begin{tabular}{l}
|\input{childdoc.def}|\\
|\childdocforwardprefix[|\textit{main}|]{|\textit{prefix}|}{|\textit{dest}|}|
\end{tabular}
\end{center}
%
the destination file is determined by a pattern
depending on the current file:
To make this work, the current file must be called
`{\textit{prefix}\hspace{0.2em}\textit{suffix}}'
with \textit{prefix} matching precisely the argument.
Processing is then passed on to the file
`{\textit{dest}\hspace{0.2em}\textit{suffix}}'.
Surely, the same effect is achieved by
directly specifying the
argument `{\textit{dest}\hspace{0.2em}\textit{suffix}}'
in the first form.
However, that requires to set up a different file
for each child. With the alternative form of the command
all these files can have exactly the same content
which simplifies setting them up and maintaining them.

For example, the following file |draft.tex|
with a compilation flag |\version| as described in \secref{sec:flags}
compiles the main document as a draft:
%
\begin{center}
\begin{tabular}{l}
|\def\version{draft}|\\
|\input{childdoc.def}|\\
|\childdocforward{|\textit{main}|}|
\end{tabular}
\end{center}
%
Likewise, the following files |final|\textit{nn}|.tex|
compile the final version of the child document
|child|\textit{nn}|.tex|:
%
\begin{center}
\begin{tabular}{l}
|\def\version{final}|\\
|\input{childdoc.def}|\\
|\childdocforwardprefix{final}{child}|
\end{tabular}
\end{center}
%

Note that when several versions of a main file and/or of each child file
are to be generated, it may be convenient to set up a |Makefile| or
shell script to automatise the process.

%%%%%%%%%%%%%%%%%%%%%%%%%%%%%%%%%%%%%%%%%%%%%%%%%%%%%%%%%%%%%%%%%%%%%%%%%%%%%%%%
\subsection{Command Line Processing}
\label{sec:commandline}

The effect of redirection files can also be achieved by invoking
the \LaTeX{} compiler with a more elaborate command line.
Most conveniently this should be done as part
of a shell script or a |Makefile|.

When using \textsf{childdoc} in the main file, the following
command lines effectively perform a redirection
(note that depending on the shell being used,
backslashes may have to be doubled: `|\|' $\to$ `|\\|'):
%
\begin{center}
|... -jobname "|\textit{target}|" |\\|"|[\textit{flags}]%
|\input{childdoc.def}\childdocforward[|\textit{main}|]{|\textit{dest}|}"|
\end{center}
%
Here \textit{target} is the name of the output file,
\textit{main} is the name of the main file
and \textit{dest} is the name of the main or child file to be processed
(all filenames without extensions).
The optional argument \textit{main} can be omitted
if \textit{main} matches \textit{dest}.
Optionally, compilation \textit{flags} can be defined via |\def| commands.
This command line makes the \TeX{} engine believe
it is compiling the file \textit{target}
whose content is specified as the latter parameter.
The provided code then forwards the processing to
\textit{main} or \textit{dest} as described in \secref{sec:forward}.

%%%%%%%%%%%%%%%%%%%%%%%%%%%%%%%%%%%%%%%%%%%%%%%%%%%%%%%%%%%%%%%%%%%%%%%%%%%%%%%%
\subsection{Include by Input}
\label{sec:input}

Including child documents by |\include| has some restrictions by design.
Most notably, the content of a child document always occupies
its own set of pages; pages cannot be shared between child documents.
Usually, this behaviour makes perfect sense
because each child document contain an essential part of the document.
However, in some situations it may be desirable to compose
a document from a collection of parts
without having mandatory page breaks between then.
For this case, the package
provides a mechanism to include parts
by |\input| which can also be processed individually.
However, by construction this mechanism
requires manual handling of the content to be output.

%%%%%%%%%%%%%%%%%%%%%%%%%%%%%%%%%%%%%%%%
\DescribeMacro{\ifchilddocmanual}
The main file should be prepared as usual, see \secref{sec:include}.
However, the document body must make a distinction
between processing of an individual part and of the main document, e.g.:
%
\begin{center}
\begin{tabular}{l}
|\ifchilddocmanual|\\
|\input{\childdocname}|\\
|\||else|\\
\textit{document body with }|\input{|\textit{part}|}|\\
|\||fi|
\end{tabular}
\end{center}
%
The conditional |\ifchilddocmanual| is true whenever
a part to be included by |\input| is being compiled,
and the name of the part is stored in |\childdocname|.

%%%%%%%%%%%%%%%%%%%%%%%%%%%%%%%%%%%%%%%%
\DescribeMacro{\childdocby}
Each part to be included by |\input| should start with:
%
\begin{center}
\begin{tabular}{l}
|\input{childdoc.def}|\\
|\childdocby{|\textit{main}|}|\\
\end{tabular}
\end{center}
%
The directive |\childdocby| is similar to |\childdocof|
described in \secref{sec:include},
but the subsequent selection of content must be done manually.
To that end, both |\ifchilddoc| and |\ifchilddocmanual|
will be true upon processing of a part,
and the name of the part is stored in |\childdocname|.
Note that |\jobname| will be set to the filename of the current part
so that each part receives an individual |.aux| file
that does not interfere with the |.aux| file(s) of the main document.
This behaviour can be altered by the alternative form
|\childdocby[*]{|\textit{main}|}| (with a non-empty optional argument)
which uses the |.aux| file of the main document
by setting |\jobname| to \textit{main}.

%%%%%%%%%%%%%%%%%%%%%%%%%%%%%%%%%%%%%%%%%%%%%%%%%%%%%%%%%%%%%%%%%%%%%%%%%%%%%%%%
\subsection{Driver Development}
\label{sec:driver}

The \textsf{childdoc} mechanism can also be use for the development
of definition files such as \LaTeX{} styles or classes.
This case differs from the above setup with multiple parts
included by |\include| in that no |\includeonly| should be invoked.
This can be achieved by starting the include file
(before |\ProvidesPackage|) with:
%
\begin{center}
\begin{tabular}{l}
|\input{childdoc.def}|\\
|\childdocforward{|\textit{main}|}|\\
\end{tabular}
\end{center}
%
or alternatively with:
%
\begin{center}
\begin{tabular}{l}
|\input{childdoc.def}|\\
|\childdocby{|\textit{main}|}|\\
\end{tabular}
\end{center}
%
Both forms have slightly different effects as described above.
The main file is prepared as usual, see \secref{sec:include}.

%%%%%%%%%%%%%%%%%%%%%%%%%%%%%%%%%%%%%%%%%%%%%%%%%%%%%%%%%%%%%%%%%%%%%%%%%%%%%%%%
\subsection{Legacy Detection}
\label{sec:detection}

The directive |\childdocmain| in the main file can detect
whether the complete document or merely a child is to be compiled
even without using the directive |\childdocof|.
This method is deprecated because it is less robust
and there is no compelling reason to use it;
it is merely provided for backward compatibility
and it may be removed in future versions.

If the detection mechanism is to be used,
it is mandatory to correctly specify
the filename of the main file as the argument of |\childdocmain|:
%
\begin{center}
\begin{tabular}{l}
|\input{childdoc.def}|\\
|\childdocmain{|\textit{main}|}|\\
\end{tabular}
\end{center}
%
If |\jobname| does not match the argument \textit{main} of |\childdocmain|,
it is assumed that |\jobname| points to the child file to be compiled.
When using |\childdocmain| with the main file specified as argument,
it suffices to start a child file
with just |\input{|\textit{main}|}|
without loading of the package and using |\childdocof|.
If instead all processing is done
with the appropriate \textsf{childdoc} directives,
the argument of \textit{main} of |\childdocmain| can be empty.

An alternative version of the command line processing described
in \secref{sec:commandline} using the detection mechanism reads:
%
\begin{center}
|... -jobname "|\textit{target}|" "|[\textit{flags}]%
[|\def\jobname{|\textit{dest}|}|]|\input{|\textit{main}|}"|
\end{center}

%%%%%%%%%%%%%%%%%%%%%%%%%%%%%%%%%%%%%%%%%%%%%%%%%%%%%%%%%%%%%%%%%%%%%%%%%%%%%%%%
\subsection{Manual Code}
\label{sec:manual}

In case one cannot be certain whether the definitions file |childdoc.def|
is installed on the target \TeX{} distribution
and one prefers not to ship it,
it is conceivable to paste a few relevant commands into the sources.

To that end, drop all statements |\input{childdoc.def}|
and perform the replacements as outlined below.
Instead of |\childdocmain{|\textit{main}|}| add the following code
to the top of the main file:
%
\begin{center}
\begin{tabular}{l}
|\||ifdefined\childdocname\endinput\||fi\newif\ifchilddoc|\\
|\edef\childdocname{\scantokens\expandafter{\jobname\noexpand}}|\\
|\def\childdocmain{|\textit{main}|}\||ifx\childdocmain\childdocname\||else|\\
|\childdoctrue\includeonly{\childdocname}\let\jobname\childdocmain\||fi|\\
\end{tabular}
\end{center}
%
Instead of |\childdocof{|\textit{main}|}| just include the main file
at the top of each child file:
%
\begin{center}
|\input{|\textit{main}|}|
\end{center}
%
A simple redirection |\childdocforward{|\textit{dest}|}| is achieved by:
%
\begin{center}
|\def\jobname{|\textit{dest}|}\input{\jobname}|
\end{center}
%
The redirection with prefix
|\childdocforwardprefix[|\textit{prefix}|]{|\textit{dest}|}|
is accomplished by:
%
\begin{center}
\begin{tabular}{l}
|{\edef\jobname{\scantokens\expandafter{\jobname\noexpand}}|\\
|\def\redirectjob |\textit{prefix}|#1~~~{\gdef\jobname{|\textit{dest}|#1}}|\\
|\expandafter\redirectjob\jobname~~~}\input{\jobname}|
\end{tabular}
\end{center}

In an alternative approach,
child documents can be compiled by a specific command line
without additional code or specific definitions:
%
\begin{center}
|... -jobname "|\textit{target}|" "|[\textit{flags}]%
|\includeonly{|\textit{dest}|}\input{|\textit{main}|}"|
\end{center}
%

%%%%%%%%%%%%%%%%%%%%%%%%%%%%%%%%%%%%%%%%%%%%%%%%%%%%%%%%%%%%%%%%%%%%%%%%%%%%%%%%
%%%%%%%%%%%%%%%%%%%%%%%%%%%%%%%%%%%%%%%%%%%%%%%%%%%%%%%%%%%%%%%%%%%%%%%%%%%%%%%%
\section{Information}

%%%%%%%%%%%%%%%%%%%%%%%%%%%%%%%%%%%%%%%%%%%%%%%%%%%%%%%%%%%%%%%%%%%%%%%%%%%%%%%%
\subsection{Copyright}

Copyright \copyright{} 2017--2018 Niklas Beisert

This work may be distributed and/or modified under the
conditions of the \LaTeX{} Project Public License, either version 1.3
of this license or (at your option) any later version.
The latest version of this license is in
  \url{http://www.latex-project.org/lppl.txt}
and version 1.3 or later is part of all distributions of \LaTeX{}
version 2005/12/01 or later.

This work has the LPPL maintenance status `maintained'.

The Current Maintainer of this work is Niklas Beisert.

This work consists of the files |README.txt|, |childdoc.ins| and |childdoc.dtx|
as well as the derived files |childdoc.def|, |cdocsamp.tex|
with |cdocsch1.tex|, |cdocsch2.tex|, |cdocspt3.tex|, |cdocspt4.tex|,
|cdocsdrf.tex|, |cdocsfn1.tex|, |cdocsfn2.tex|
as well as |childdoc.pdf|.

%%%%%%%%%%%%%%%%%%%%%%%%%%%%%%%%%%%%%%%%%%%%%%%%%%%%%%%%%%%%%%%%%%%%%%%%%%%%%%%%
\subsection{Files and Installation}

The package consists of the files:
%
\begin{center}
\begin{tabular}{ll}
    |README.txt|   & readme file \\
    |childdoc.ins| & installation file \\
    |childdoc.dtx| & source file \\
    |childdoc.def| & definition file \\
    |cdocsamp.tex| & sample main file \\
    |cdocsch1.tex| & sample include file \\
    |cdocsch2.tex| & sample include file \\
    |cdocspt3.tex| & sample part file \\
    |cdocspt4.tex| & sample part file \\
    |cdocsdrf.tex| & sample redirection file \\
    |cdocsfn1.tex| & sample redirection file \\
    |cdocsfn2.tex| & sample redirection file \\
    |childdoc.pdf| & manual
\end{tabular}
\end{center}
%
The distribution consists of the files
|README.txt|, |childdoc.ins| and |childdoc.dtx|.
%
\begin{itemize}
\item
Run (pdf)\LaTeX{} on |childdoc.dtx|
to compile the manual |childdoc.pdf| (this file).
\item
Run \LaTeX{} on |childdoc.ins| to create the definitions file |childdoc.def|
and the sample |cdocsamp.tex| with include files
|cdocsch1.tex|, |cdocsch2.tex|, |cdocspt3.tex|, |cdocspt4.tex|,
|cdocsdrf.tex|, |cdocsfn1.tex|, |cdocsfn2.tex|.
Then copy the file |childdoc.def| to an appropriate directory of your \LaTeX{}
distribution, e.g.\ \textit{texmf-root}|/tex/latex/childdoc|.
\end{itemize}

%%%%%%%%%%%%%%%%%%%%%%%%%%%%%%%%%%%%%%%%%%%%%%%%%%%%%%%%%%%%%%%%%%%%%%%%%%%%%%%%
\subsection{Related CTAN Packages}

There are several other packages which offer a similar functionality:
%
\begin{itemize}
\item
The packages
\href{http://ctan.org/pkg/docmute}{\textsf{docmute}},
\href{http://ctan.org/pkg/includex}{\textsf{includex}} and
\href{http://ctan.org/pkg/standalone}{\textsf{standalone}}
provide commands to include only the document body of
a child file thus allowing both files to be compiled individually.
\item
The packages \href{http://ctan.org/pkg/subdocs}{\textsf{subdocs}}
and \href{http://ctan.org/pkg/subfiles}{\textsf{subfiles}}
provide structures in which the main and child documents can be
encapsulated and allowing them to be compiled individually.
The inclusion mechanism is different from the conventional |\include|.
\item
The package \href{http://ctan.org/pkg/combine}{\textsf{combine}}
is an elaborate solution to combine several documents into one.
\end{itemize}
%
See also the CTAN topic \href{http://ctan.org/topic/subdocs}{\textsf{subdocs}}
for further related packages.
The present package differs from the above solutions in that
a document structure constructed with the conventional |\include| mechanism
just needs two extra commands at the top of every file
such that all constituent files can be compiled individually.

%%%%%%%%%%%%%%%%%%%%%%%%%%%%%%%%%%%%%%%%%%%%%%%%%%%%%%%%%%%%%%%%%%%%%%%%%%%%%%%%
%\subsection{Feature Suggestions}
%
%The following is a list of features which may be useful for future
%versions of this package:
%%
%\begin{itemize}
%\item
%\ldots
%\end{itemize}

%%%%%%%%%%%%%%%%%%%%%%%%%%%%%%%%%%%%%%%%%%%%%%%%%%%%%%%%%%%%%%%%%%%%%%%%%%%%%%%%
\subsection{Revision History}

%%%%%%%%%%%%%%%%%%%%%%%%%%%%%%%%%%%%%%%%
\paragraph{v2.0:} 2018/12/30

\begin{itemize}
\item
immediate forward processing
\item
added |\childdocby| mechanism
\item
manual restructured
\end{itemize}

%%%%%%%%%%%%%%%%%%%%%%%%%%%%%%%%%%%%%%%%
\paragraph{v1.6:} 2018/01/17

\begin{itemize}
\item
application for development of include files
\item
corrections to manual
\end{itemize}

%%%%%%%%%%%%%%%%%%%%%%%%%%%%%%%%%%%%%%%%
\paragraph{v1.5:} 2017/05/21

\begin{itemize}
\item
more complete structuring introduced
\item
|\childdocof| introduced
\item
|\childdoc| renamed to |\childdocmain|
\item
|\childredirect| renamed to |\childdocforward| and |\childdocforwardprefix|
and functionality expanded
\end{itemize}

%%%%%%%%%%%%%%%%%%%%%%%%%%%%%%%%%%%%%%%%
\paragraph{v1.0:} 2017/04/27

\begin{itemize}
\item
manual and install package
\item
first version published on CTAN
\end{itemize}

%%%%%%%%%%%%%%%%%%%%%%%%%%%%%%%%%%%%%%%%
\paragraph{v0.6:} 2017/04/26

\begin{itemize}
\item
redirection mechanism added
\end{itemize}

%%%%%%%%%%%%%%%%%%%%%%%%%%%%%%%%%%%%%%%%
\paragraph{v0.5:} 2017/04/26

\begin{itemize}
\item
functionality in definition file
\end{itemize}


%%%%%%%%%%%%%%%%%%%%%%%%%%%%%%%%%%%%%%%%%%%%%%%%%%%%%%%%%%%%%%%%%%%%%%%%%%%%%%%%
%%%%%%%%%%%%%%%%%%%%%%%%%%%%%%%%%%%%%%%%%%%%%%%%%%%%%%%%%%%%%%%%%%%%%%%%%%%%%%%%
%%%%%%%%%%%%%%%%%%%%%%%%%%%%%%%%%%%%%%%%%%%%%%%%%%%%%%%%%%%%%%%%%%%%%%%%%%%%%%%%
\appendix

\settowidth\MacroIndent{\rmfamily\scriptsize 000\ }

 \DocInput{childdoc.dtx}

\end{document}
%</driver>
% \fi
%
% %%%%%%%%%%%%%%%%%%%%%%%%%%%%%%%%%%%%%%%%%%%%%%%%%%%%%%%%%%%%%%%%%%%%%%%%%%%%%%
% %%%%%%%%%%%%%%%%%%%%%%%%%%%%%%%%%%%%%%%%%%%%%%%%%%%%%%%%%%%%%%%%%%%%%%%%%%%%%%
% \section{Sample}
%\iffalse
%<*samplemain>
%\fi
%
% The following presents a sample document
% with two chapters, two parts, a title page,
% a compile flag as well as three forwarding files to set the flag.
% It consists of eight |.tex| files:
% \begin{center}
% \begin{tabular}{ll}
% |cdocsamp.tex|&main file\\
% |cdocsch1.tex|&include file for chapter 1\\
% |cdocsch2.tex|&include file for chapter 2\\
% |cdocspt3.tex|&include file for part 3\\
% |cdocspt4.tex|&include file for part 4\\
% |cdocsdrf.tex|&forwarding file for main file in draft mode\\
% |cdocsfi1.tex|&forwarding file for final version of chapter 1\\
% |cdocsfi2.tex|&forwarding file for final version of chapter 2\\
% \end{tabular}
% \end{center}
% Each of the eight files can be compiled directly by the \LaTeX{} compiler.
%
% %%%%%%%%%%%%%%%%%%%%%%%%%%%%%%%%%%%%%%
% \paragraph{Main File.}
%
% The main file is called |cdocsamp.tex|.
%
% Load the \textsf{childdoc} definitions and
% declare the filename for the main document:
%    \begin{macrocode}
\input{childdoc.def}
\childdocmain{}
%    \end{macrocode}

% Optional override for |\version| flag:
%    \begin{macrocode}
%%\ifchilddoc\else\providecommand{\version}{draft}\fi
%    \end{macrocode}

% Define the default values for the |\version| flag
% (|final| for the main file and |draft| for childs):
%    \begin{macrocode}
\ifchilddoc
\providecommand{\version}{draft}
\else
\providecommand{\version}{final}
\fi
%    \end{macrocode}

% Load the standard document class:
%    \begin{macrocode}
\documentclass[12pt]{article}
%    \end{macrocode}

% Start the document body:
%    \begin{macrocode}
\begin{document}
%    \end{macrocode}

% Declare a title page.
% Print title, part of document being processed and version flag:
%    \begin{macrocode}
\addtocounter{page}{-1}
\begin{center}
{\LARGE\bfseries{}childdoc example\par}
\vspace{1cm}
\ifchilddoc
\ifchilddocmanual part\else chapter\fi:
`\childdocname' of `\childdocjob'\par
\else
main document: `\childdocjob'\par
\fi
version: \version\par
\end{center}
\newpage
%    \end{macrocode}

% Manually include selected file,
% otherwise process as usual:
%    \begin{macrocode}
\ifchilddocmanual
\section*{part `\childdocname'}
\input{\childdocname}
\else
%    \end{macrocode}

% Include the two chapters:
%    \begin{macrocode}
\include{cdocsch1}
\include{cdocsch2}
%    \end{macrocode}

% Include the two parts unless only chapters should be displayed:
%    \begin{macrocode}
\ifchilddoc\else
\section{part three}
\input{cdocspt3}
\section{part four}
\input{cdocspt4}
\fi
%    \end{macrocode}

% Process as usual until here:
%    \begin{macrocode}
\fi
%    \end{macrocode}

% End of document body:
%    \begin{macrocode}
\end{document}
%    \end{macrocode}
%\iffalse
%</samplemain>
%\fi
%
% %%%%%%%%%%%%%%%%%%%%%%%%%%%%%%%%%%%%%%
% \paragraph{Chapter Include Files.}
%
% The include files are called |cdocsch1.tex| and |cdocsch2.tex|.
%
%\iffalse
%<*samplechap1|samplechap2>
%\fi

% Optional override for |\version| flag:
%    \begin{macrocode}
%%\providecommand{\version}{final}
%    \end{macrocode}

% Include the main document:
%    \begin{macrocode}
\input{childdoc.def}
\childdocof{cdocsamp}
%    \end{macrocode}

%\iffalse
%</samplechap1|samplechap2>
%\fi
%
%\iffalse
%<*samplechap1>
%\fi
% Some text for chapter 1:
%    \begin{macrocode}
\section{one}
some text in chapter one
%    \end{macrocode}

%\iffalse
%</samplechap1>
%\fi
% Some text for chapter 2:
%\iffalse
%<*samplechap2>
%\fi
%    \begin{macrocode}
\section{two}
more text in chapter two
%    \end{macrocode}

%\iffalse
%</samplechap2>
%\fi
%
% %%%%%%%%%%%%%%%%%%%%%%%%%%%%%%%%%%%%%%
% \paragraph{Part Include Files.}
%
% The include files are called |cdocspt3.tex| and |cdocspt4.tex|.
%
%\iffalse
%<*samplepart3|samplepart4>
%\fi

% Optional override for |\version| flag:
%    \begin{macrocode}
%%\providecommand{\version}{final}
%    \end{macrocode}

% Include the main document:
%    \begin{macrocode}
\input{childdoc.def}
\childdocby{cdocsamp}
%    \end{macrocode}

%\iffalse
%</samplepart3|samplepart4>
%\fi
%
%\iffalse
%<*samplepart3>
%\fi
% Some text for part 3:
%    \begin{macrocode}
some text in part three
%    \end{macrocode}

%\iffalse
%</samplepart3>
%\fi
% Some text for part 4:
%\iffalse
%<*samplepart4>
%\fi
%    \begin{macrocode}
more text in part four
%    \end{macrocode}

%\iffalse
%</samplepart4>
%\fi
%
% %%%%%%%%%%%%%%%%%%%%%%%%%%%%%%%%%%%%%%
% \paragraph{Forwarding for a Complete Draft.}
%
% The following forwarding file |cdocsdrf.tex|
% compiles the main document in draft mode:
%\iffalse
%<*sampledraft>
%\fi
%    \begin{macrocode}
\def\version{draft}
\input{childdoc.def}
\childdocforward{cdocsamp}
%    \end{macrocode}

%\iffalse
%</sampledraft>
%\fi
%
% %%%%%%%%%%%%%%%%%%%%%%%%%%%%%%%%%%%%%%
% \paragraph{Forwarding for Final Version of the Chapters.}
%
% The following forwarding files |cdocsfn1.tex| and |cdocsfn2.tex|
% (with identical content)
% compile the final versions of the child documents
% |cdocsch1.tex| and |cdocsch2.tex|, respectively:
%\iffalse
%<*samplefinal>
%\fi
%    \begin{macrocode}
\def\version{final}
\input{childdoc.def}
\childdocforwardprefix[cdocsamp]{cdocsfn}{cdocsch}
%    \end{macrocode}

%\iffalse
%</samplefinal>
%\fi
%
% %%%%%%%%%%%%%%%%%%%%%%%%%%%%%%%%%%%%%%
% \paragraph{Command Line Processing.}
%
% The following three command lines generate the output files
% |cdocscld|, |cdocscl1| and |cdocscl2|
% which should be identical to
% |cdocsdrf|, |cdocsch1| and |cdocsfn2|, respectively:
% \begin{center}
% \begin{tabular}{l}
% |latex -jobname cdocscld \|\\
% |  "\def\version{draft}\input{childdoc.def}\childdocforward{cdocsamp}"|\\
% |latex -jobname cdocscl1 \|\\
% |  "\input{childdoc.def}\childdocforward[cdocsamp]{cdocsch1}"|\\
% |latex -jobname cdocscl2 \|\\
% |  "\def\version{final}\input{childdoc.def}\childdocforward{cdocsch2}"|
% \end{tabular}
% \end{center}
% Note that the trailing backslash on each first line
% merely continues the input to the second line
% (for convenient cut ant paste).
% Furthermore, the command |latex| can be replaced by any
% of its alternative versions such as |pdflatex|.
%
% %%%%%%%%%%%%%%%%%%%%%%%%%%%%%%%%%%%%%%%%%%%%%%%%%%%%%%%%%%%%%%%%%%%%%%%%%%%%%%
% %%%%%%%%%%%%%%%%%%%%%%%%%%%%%%%%%%%%%%%%%%%%%%%%%%%%%%%%%%%%%%%%%%%%%%%%%%%%%%
% \section{Implementation}
%\iffalse
%<*package>
%\fi
%
% This section describes the definitions file |childdoc.def|.

% The definitions cannot be loaded using |\usepackage| or |\RequirePackage|
% which has a mechanism to prevent loading a style file more than once.
% When loading the definitions by means of |\input|
% multiple instances have to be prevented manually:
%\iffalse
%This code needs to be before the `\ProvidesFile' directive
%which is defined at the beginning of this file.
%Therefore it is also placed there and commented out here.
%</package>
%<*discard>
%\fi
%    \begin{macrocode}
\ifdefined\childdocmain\endinput\fi
%    \end{macrocode}
%\iffalse
%</discard>
%<*package>
%\fi
%
% \macro{\ifchilddoc}
% \macro{\ifchilddocmanual}
% The conditional |\ifchilddoc| tells whether a
% child (true) or main (false) document is being compiled.
% The conditional |\ifchilddocmanual| tells whether
% the |\includeonly| mechanism is used (false) or
% the selection of child files must be performed manually (true).
% The definitions initialise to false:
%    \begin{macrocode}
\newif\ifchilddoc
\newif\ifchilddocmanual
%    \end{macrocode}

% \macro{\childdocname}
% \macro{\childdocjob}
% The macro |\childdocname| stores the name of the main document
% to be compiled. The macro |\childdocjob| stores the name of
% the document on which the \LaTeX{} compiler was originally invoked.
% The content of |\jobname| cannot be compared
% to filenames specified in the source due to different catcodes.
% The following code rescans |\jobname|, stores the result
% in |\childdocname| and saves a copy in |\childdocjob|:
%    \begin{macrocode}
\edef\childdocname{\scantokens\expandafter{\jobname\noexpand}}
\let\childdocjob\childdocname
%    \end{macrocode}

% \macro{\childdocdisable}
% The macro |\childdocdisable| prevents the main file
% from being processed more than once.
% At this stage, the main document command |\childdocmain|
% is assumed to be called once again where it should do nothing.
% Any subsequent call to it should prevent
% a secondary processing of the main document
% It overwrites the forwarding commands
% |\childdocof| and |\childdocforward|
% with empty macros to prevent further inclusions of the main document:
%    \begin{macrocode}
\newcommand{\childdocdisable}
{
  \renewcommand{\childdocmain}[1]{\renewcommand{\childdocmain}[1]{\endinput}}
  \renewcommand{\childdocof}[1]{}
  \renewcommand{\childdocby}[2][]{}
  \renewcommand{\childdocforward}[2][]{}
  \renewcommand{\childdocdisable}{}
}
%    \end{macrocode}

% \macro{\childdocmain}
% The macro |\childdocmain| is to be called at the top of the main file
% with nothing or the main filename (without extension) as argument.
% First, it breaks loops.
% If the argument is not empty and does not match |\childdocname|
% (which is set by the first inclusion of |childdoc.def|),
% |\ifchilddoc| is set to true, |\includeonly| is applied to the child file
% and |\jobname| is set to the main file
% (for proper handling of |.aux| files):
%    \begin{macrocode}
\newcommand{\childdocmain}[1]
{
  \childdocdisable\childdocmain{}
  \if?#1?\else
    \begingroup
      \def\childdoctmp{#1}
      \ifx\childdoctmp\childdocname
        \def\childdoctmp{}
      \else
        \def\childdoctmp
        {
          \childdoctrue
          \includeonly{\childdocname}
          \def\childdocjob{#1}
          \def\jobname{#1}
        }
      \fi
      \expandafter
    \endgroup
    \childdoctmp
  \fi
}
%    \end{macrocode}

% \macro{\childdocof}
% The command |\childdocof| redirects
% compilation to the main file |#1|.
%    \begin{macrocode}
\newcommand{\childdocof}[1]
{
  \childdocdisable
  \childdoctrue
  \includeonly{\childdocname}
  \def\jobname{#1}
  \def\childdocjob{#1}
  \input{#1}
}
%    \end{macrocode}

% \macro{\childdocby}
% The command |\childdocby| ....
%    \begin{macrocode}
\newcommand{\childdocby}[2][]
{
  \childdocdisable
  \childdoctrue
  \childdocmanualtrue
  \if?#1?\else
    \def\jobname{#2}
  \fi
  \def\childdocjob{#2}
  \input{#2}
  \endinput
}
%    \end{macrocode}

% \macro{\childdocforward}
% The command |\childdocforward| redirects
% compilation to the main file or
% (if the optional argument is given) a child file.
% Parameters are set as if the main file
% or a child file starting with |\childdocof| was compiled.
% Then compilation is handed over to the main file:
%    \begin{macrocode}
\newcommand{\childdocforward}[2][]
{
  \begingroup
    \if?#1?
      \def\childdoctmp
      {
        \def\childdocname{#2}
        \def\childdocjob{#2}
        \def\jobname{#2}
        \input{#2}
        \endinput
      }
    \else
      \def\childdoctmp
      {
        \childdocdisable
        \def\childdocname{#2}
        \childdoctrue
        \includeonly{#2}
        \def\childdocjob{#1}
        \def\jobname{#1}
        \input{#1}
        \endinput
      }
    \fi
    \expandafter
  \endgroup
  \childdoctmp
}
%    \end{macrocode}

% \macro{\childdocforwardprefix}
% The command |\childdocforwardprefix| redirects
% compilation to the main or a child file by means of a pattern.
% The prefix |#1| in the current filename is replaced by |#2|
% and the suffix of the current filename is kept
% (it is assumed that the filename does not contain the substring `|~~~|'
% which is used as a delimiter).
% Compilation is handed over to the new file by |\childdocforward|:
%    \begin{macrocode}
\newcommand{\childdocforwardprefix}[3][]
{
  \begingroup
    \def\childdocextract #2##1~~~{\def\childdoctmp{\childdocforward[#1]{#3##1}}}
    \expandafter\childdocextract\childdocname~~~
    \expandafter
  \endgroup
  \childdoctmp
}
%    \end{macrocode}

% \macro{\childdoc}
% The deprecated macro |\childdoc| is a legacy version of |\childdocmain|:
%    \begin{macrocode}
\newcommand{\childdoc}{\childdocmain}
%    \end{macrocode}

% \macro{\childdocredirect}
% The deprecated macro |\childdocredirect| is a legacy version
% of |\childdocforward| and |\childdocforwardprefix|:
%    \begin{macrocode}
\newcommand{\childdocredirect}[2][]
{
  \begingroup
    \if?#1?
      \def\childdoctmp{\childdocforward{#2}}
    \else
      \def\childdoctmp{\childdocforwardprefix{#1}{#2}}
    \fi
    \expandafter
  \endgroup
  \childdoctmp
}
%    \end{macrocode}

%\iffalse
%</package>
%\fi
%
\endinput

\childdocforward{cdocsamp}
%    \end{macrocode}

%\iffalse
%</sampledraft>
%\fi
%
% %%%%%%%%%%%%%%%%%%%%%%%%%%%%%%%%%%%%%%
% \paragraph{Forwarding for Final Version of the Chapters.}
%
% The following forwarding files |cdocsfn1.tex| and |cdocsfn2.tex|
% (with identical content)
% compile the final versions of the child documents
% |cdocsch1.tex| and |cdocsch2.tex|, respectively:
%\iffalse
%<*samplefinal>
%\fi
%    \begin{macrocode}
\def\version{final}
% \iffalse
%
% childdoc.dtx Copyright (C) 2017-2018 Niklas Beisert
%
% This work may be distributed and/or modified under the
% conditions of the LaTeX Project Public License, either version 1.3
% of this license or (at your option) any later version.
% The latest version of this license is in
%   http://www.latex-project.org/lppl.txt
% and version 1.3 or later is part of all distributions of LaTeX
% version 2005/12/01 or later.
%
% This work has the LPPL maintenance status `maintained'.
%
% The Current Maintainer of this work is Niklas Beisert.
%
% This work consists of the files childdoc.dtx and childdoc.ins
% and the derived files childdoc.def and cdocsamp.tex with
% cdocsch1.tex, cdocsch2.tex, cdocsdrf.tex, cdocsfn1.tex, cdocsfn2.tex.
%
%<package>\ifdefined\childdocmain\endinput\fi
%<package>\ProvidesFile{childdoc.def}[2018/12/30 v2.0 child document driver]
%<samplemain>\ProvidesFile{cdocsamp.tex}[2018/12/30 v2.0 sample for childdoc]
%<*driver>
%\ProvidesFile{childdoc.drv}[2018/12/30 v2.0 childdoc reference manual file]
\PassOptionsToClass{10pt,a4paper}{article}
\documentclass{ltxdoc}

\usepackage[margin=35mm]{geometry}
\usepackage{hyperref}
\usepackage{hyperxmp}
\usepackage[usenames]{color}

\hypersetup{colorlinks=true}
\hypersetup{pdfstartview=FitH}
\hypersetup{pdfpagemode=UseNone}
\hypersetup{pdfsource={}}
\hypersetup{pdflang={en-UK}}
\hypersetup{pdfcopyright={Copyright 2017-2018 Niklas Beisert.
  This work may be distributed and/or modified under the
  conditions of the LaTeX Project Public License, either version 1.3
  of this license or (at your option) any later version.}}
\hypersetup{pdflicenseurl={http://www.latex-project.org/lppl.txt}}
\hypersetup{pdfcontactaddress={ETH Zurich, ITP, HIT K,
  Wolfgang-Pauli-Strasse 27}}
\hypersetup{pdfcontactpostcode={8093}}
\hypersetup{pdfcontactcity={Zurich}}
\hypersetup{pdfcontactcountry={Switzerland}}
\hypersetup{pdfcontactemail={nbeisert@itp.phys.ethz.ch}}
\hypersetup{pdfcontacturl={http://people.phys.ethz.ch/\xmptilde nbeisert/}}

\newcommand{\secref}[1]{\hyperref[#1]{section \ref*{#1}}}

\parskip1ex
\parindent0pt
\let\olditemize\itemize
\def\itemize{\olditemize\parskip0pt}

\begin{document}

\title{The \textsf{childdoc} Package}
\hypersetup{pdftitle={The childdoc Package}}
\author{Niklas Beisert\\[2ex]
  Institut f\"ur Theoretische Physik\\
  Eidgen\"ossische Technische Hochschule Z\"urich\\
  Wolfgang-Pauli-Strasse 27, 8093 Z\"urich, Switzerland\\[1ex]
  \href{mailto:nbeisert@itp.phys.ethz.ch}
  {\texttt{nbeisert@itp.phys.ethz.ch}}}
\hypersetup{pdfauthor={Niklas Beisert}}
\hypersetup{pdfsubject={Manual for the LaTeX2e Package childdoc}}
\date{30 December 2018, \textsf{v2.0}}
\maketitle

\begin{abstract}\noindent
\textsf{childdoc} is a \LaTeXe{} package
that enables the direct compilation
of document sections included by |\include|
to individual files.
\end{abstract}

\begingroup
\parskip0ex
\tableofcontents
\endgroup

%%%%%%%%%%%%%%%%%%%%%%%%%%%%%%%%%%%%%%%%%%%%%%%%%%%%%%%%%%%%%%%%%%%%%%%%%%%%%%%%
%%%%%%%%%%%%%%%%%%%%%%%%%%%%%%%%%%%%%%%%%%%%%%%%%%%%%%%%%%%%%%%%%%%%%%%%%%%%%%%%
\section{Introduction}

\LaTeX{} provides a mechanism to structure a large document (such as a book)
into a main file and several child files (containing the chapters)
using the |\include| command.
This mechanism is beneficial for documents
which span hundreds of pages in order to
make the source file(s) more manageable.
Moreover, compilation can be restricted to
selected child files by means of the |\includeonly| command.
The latter feature can be used to reduce the compilation time while editing
(this was significantly more useful in the earlier days of \LaTeX{})
or to generate a smaller document which is easier to navigate.
Another application of |\includeonly| is to generate
documents consisting of selected parts of the complete document.

However, there are a few drawbacks of the plain |\include| mechanism:
\begin{itemize}
\item
The child files cannot be compiled on their own,
they can only be compiled via the main file.
A naive editing environment
(such as a text editor with an option
to have the current file processed by \LaTeX)
may require one to switch to the main file before compiling;
attempting to compile the child file produces errors.
\item
The main file must be modified (each time)
to adjust the |\includeonly| command
to the present needs. This easily leaves the main file in a messy state.
\item
The generated document will always carry the filename
of the main document. This is inconvenient if
several child files are to be compiled and
to be kept for distribution.
\end{itemize}

The present package provides a simple interface
to make child files individually compilable by \LaTeX{}.
Compiling a child file then has the same effect as compiling
the main file with an |\includeonly| command
to select the appropriate child.
Moreover the generated document will carry the name of the child
rather than the main file.
This resolves all three above issues.

This feature is meant to make the editing of books,
thesis documents and lecture notes somewhat more convenient.
However, the package can also be used efficiently for
composing a series of documents (such as exercise sheets)
which are typically distributed individually.
It then assists the author in generating the individual documents
(potentially in different versions)
as well as a document containing the collected series.
Another application is in developing style files
or other kinds of included material
where compilation of the style file could redirect
to a sample or test file.

%%%%%%%%%%%%%%%%%%%%%%%%%%%%%%%%%%%%%%%%%%%%%%%%%%%%%%%%%%%%%%%%%%%%%%%%%%%%%%%%
%%%%%%%%%%%%%%%%%%%%%%%%%%%%%%%%%%%%%%%%%%%%%%%%%%%%%%%%%%%%%%%%%%%%%%%%%%%%%%%%
\section{Usage}

First of all, the package \textsf{childdoc} is \emph{not} a standard
\LaTeXe{} |.sty| style file! Therefore it needs to be invoked in
a non-standard way.

%%%%%%%%%%%%%%%%%%%%%%%%%%%%%%%%%%%%%%%%%%%%%%%%%%%%%%%%%%%%%%%%%%%%%%%%%%%%%%%%
\subsection{Included Files}
\label{sec:include}

%%%%%%%%%%%%%%%%%%%%%%%%%%%%%%%%%%%%%%%%
\DescribeMacro{\childdocmain}
To use the package, add the commands
\begin{center}
\begin{tabular}{l}
|\input{childdoc.def}|\\
|\childdocmain{}|\\
\end{tabular}
\end{center}
at the very top of the main \LaTeX{} file,
in particular \emph{before} the |\documentclass| statement!
The argument of |\childdocmain| should be left empty
(but it must be present).

%%%%%%%%%%%%%%%%%%%%%%%%%%%%%%%%%%%%%%%%
\DescribeMacro{\childdocof}
Furthermore, add the commands
\begin{center}
\begin{tabular}{l}
|\input{childdoc.def}|\\
|\childdocof{|\textit{main}|}|\\
\end{tabular}
\end{center}
at the top of every child file \textit{child}
which is included by |\include{|\textit{child}|}|
from within the main file
(or at least for those files to be compiled individually).
The argument \textit{main} must be the filename of the main file.

There are a couple of
considerations in setting up the main and child documents:

%%%%%%%%%%%%%%%%%%%%%%%%%%%%%%%%%%%%%%%%
\paragraph{Restrictions.}

Please note the following restrictions:
\begin{itemize}
\item
|\childdocmain| must be called with one argument \textit{main}
to ensure compatibility with earlier version of the package.
It must either be empty (|\childdocmain{}|)
or precisely match the filename of the main file in which it is specified.
See \secref{sec:detection} for further information.
\item
The filename \textit{main} must be specified without the |.tex| extension.
\item
The filename \textit{main} is case sensitive
(even in case-insensitive file systems)
due to internal string comparison.
\item
The argument \textit{main} should be fully expanded, it cannot be a macro.
\item
Subdirectories and special characters should be avoided in filenames.
\item
The command |\childdocmain{|\textit{main}|}| must be followed by a whitespace.
It should not be followed immediately by another command
or by a comment mark `|%|'.
This is because the \TeX{} parser reads the token immediately following
the argument of |\childdocmain| and puts it
at the beginning of every child section;
however, a white\-space is ignored.
\end{itemize}

%%%%%%%%%%%%%%%%%%%%%%%%%%%%%%%%%%%%%%%%
\paragraph{Content of Main File.}

It is advisable to place all content in the child files included by |\include|.
Any output contained in the main file will appear in all child documents
unless suppressed manually;
it cannot be suppressed automatically by the |\includeonly| directive
and thus should normally be avoided.
A method to include some content in the main file
by means of conditional processing is described in \secref{sec:conditional}.

%%%%%%%%%%%%%%%%%%%%%%%%%%%%%%%%%%%%%%%%
\paragraph{Page Numbering.}

When only a part of the document is compiled,
the appropriate numbering of pages
(as well as other status parameters)
is determined from the |.aux| files.
The latter contain information from previous passes.
However this information needs to propagate through
all intermediate child documents.
Therefore the page numbering in child documents may well
be inconsistent until the complete document is compiled at least once.

A useful (if unconventional) way to always ensure a consistent
page numbering is to restart the numbering in each child document
and denote the pages by `\textit{child}|.|\textit{page}'
where \textit{child} represents the chapter/section number of the child file.
This can be achieved by the command
|\numberwithin{page}{|\textit{child}|}|
of the \textsf{amsmath} package
where \textit{child} can be |chapter| or |section|
depending on the chosen structuring.
Alternatively, one can modify the macro |\thepage| appropriately
and reset the counter |page| at the start of each child file.

%%%%%%%%%%%%%%%%%%%%%%%%%%%%%%%%%%%%%%%%%%%%%%%%%%%%%%%%%%%%%%%%%%%%%%%%%%%%%%%%
\subsection{Conditional Processing}
\label{sec:conditional}

The package provides a mechanism to compile different versions
of a document. To customise the versions further some conditional processing
can come in handy to distinguish which version is being compiled.
The package provides two macros to describe the compilation context:

%%%%%%%%%%%%%%%%%%%%%%%%%%%%%%%%%%%%%%%%
\DescribeMacro{\ifchilddoc}
The conditional |\ifchilddoc| distinguishes between the compilation of
child documents and the main document:
%
\begin{center}
|\ifchilddoc |\textit{child-code}| |[|\||else |\textit{main-code}]| \||fi|
\end{center}

%%%%%%%%%%%%%%%%%%%%%%%%%%%%%%%%%%%%%%%%
\DescribeMacro{\childdocname}
\DescribeMacro{\childdocjob}
The macro |\childdocname| contains the filename (without extension)
of the main or child file being processed.
Note that |\childdocjob| will always contain the name of the main file.

%%%%%%%%%%%%%%%%%%%%%%%%%%%%%%%%%%%%%%%%
\paragraph{Title Page.}

Conditional processing can be used to include a title or banner page
in the main document when proper precautions are taken.
Importantly, the code in the main file should ensure that the page counter
(as well as other status parameters which are stored in the |.aux| files)
takes the same value after the conditional processing.
Otherwise the page numbers may take divergent values
depending on which part is compiled.

For example, a title page could be declared by:
%
\begin{center}
\begin{tabular}{l}
|\ifchilddoc\||else|\\
|\addtocounter{page}{-1}|\\
\textit{code for title page}\\
|\newpage|\\
|\||fi|
\end{tabular}
\end{center}
%
A banner page for the child documents can be generated by:
%
\begin{center}
\begin{tabular}{l}
|\ifchilddoc|\\
|\addtocounter{page}{-1}|\\
\textit{code for banner page}\\
|\newpage|\\
|\||fi|
\end{tabular}
\end{center}
%
Here one could write a message such as:
\begin{center}
|This is the part \childdocname{} of \childdocjob{}.|
\end{center}

%%%%%%%%%%%%%%%%%%%%%%%%%%%%%%%%%%%%%%%%%%%%%%%%%%%%%%%%%%%%%%%%%%%%%%%%%%%%%%%%
\subsection{Flags}
\label{sec:flags}

The package makes it easy to generate different versions
of the main or child documents.
To this end compilation flags can be defined
and assigned different default values.
They will be particularly useful in conjunction
with the forwarding mechanism described in \secref{sec:forward}.

For example, it may be useful to have a flag |\version|
which can be set to |draft| or |final|.
The document source will contain some conditional code
depending on the value of |\version|.
Suppose further, the flag should default to |final| for the main file
and to |draft| for child files
which is a natural assignment for editing the document.
This is achieved by placing the following code
in the preamble of the main document
(below the |\childdocmain| directive):
%
\begin{center}
\begin{tabular}{l}
|\ifchilddoc|\\
|\providecommand{\version}{draft}|\\
|\||else|\\
|\providecommand{\version}{final}|\\
|\||fi|
\end{tabular}
\end{center}
%
The definition by |\providecommand| makes sure
that previous definitions are not overwritten.
Further statements |\providecommand{\version}{...}|
can thus be added before the above code to override it.

For the main file, one might add a line
(between |\childdocmain| and the above block)
%
\begin{center}
|%\ifchilddoc\||else\providecommand{\version}{draft}\||fi|
\end{center}
%
which can be uncommented to produce a draft version.
Likewise one can add a line to the very top of a child file
(above the |\childdocof{|\textit{main}|}| directive)
%
\begin{center}
|%\providecommand{\version}{final}|
\end{center}
%
which can be uncommented to produce the final version of this child document.

%%%%%%%%%%%%%%%%%%%%%%%%%%%%%%%%%%%%%%%%%%%%%%%%%%%%%%%%%%%%%%%%%%%%%%%%%%%%%%%%
\subsection{Forwarding}
\label{sec:forward}

Different versions of the main or child documents
using compilation flags as described in \secref{sec:flags}
can be (permanently) stored in different files
for convenient compilation, viewing and distribution.
To this end, the package defines a command
to pass on compilation to a different file:

%%%%%%%%%%%%%%%%%%%%%%%%%%%%%%%%%%%%%%%%
\DescribeMacro{\childdocforward}
The command |\childdocforward| redirects processing to
another source file:
%
\begin{center}
\begin{tabular}{l}
|\input{childdoc.def}|\\
|\childdocforward[|\textit{main}|]{|\textit{dest}|}|\\
\end{tabular}
\end{center}
%
The argument \textit{dest} is the destination file
(without extension).
It should be the main file or one of the child files.
Note that further \textsf{childdoc} directives
such as |\childdocof| and |\childdocforward|
in the indicated file will be processed in this form.
The optional argument \textit{main}
passes on directly to the main file \textit{main}
while pretending to compile the child \textit{dest}.
This form behaves as if \textit{dest}
issues |\childdocof{|\textit{main}|}| right away,
and no further \textsf{childdoc} directives will be processed.

%%%%%%%%%%%%%%%%%%%%%%%%%%%%%%%%%%%%%%%%
\DescribeMacro{\...prefix}
In the alternative form |\childdocforwardprefix|,
%
\begin{center}
\begin{tabular}{l}
|\input{childdoc.def}|\\
|\childdocforwardprefix[|\textit{main}|]{|\textit{prefix}|}{|\textit{dest}|}|
\end{tabular}
\end{center}
%
the destination file is determined by a pattern
depending on the current file:
To make this work, the current file must be called
`{\textit{prefix}\hspace{0.2em}\textit{suffix}}'
with \textit{prefix} matching precisely the argument.
Processing is then passed on to the file
`{\textit{dest}\hspace{0.2em}\textit{suffix}}'.
Surely, the same effect is achieved by
directly specifying the
argument `{\textit{dest}\hspace{0.2em}\textit{suffix}}'
in the first form.
However, that requires to set up a different file
for each child. With the alternative form of the command
all these files can have exactly the same content
which simplifies setting them up and maintaining them.

For example, the following file |draft.tex|
with a compilation flag |\version| as described in \secref{sec:flags}
compiles the main document as a draft:
%
\begin{center}
\begin{tabular}{l}
|\def\version{draft}|\\
|\input{childdoc.def}|\\
|\childdocforward{|\textit{main}|}|
\end{tabular}
\end{center}
%
Likewise, the following files |final|\textit{nn}|.tex|
compile the final version of the child document
|child|\textit{nn}|.tex|:
%
\begin{center}
\begin{tabular}{l}
|\def\version{final}|\\
|\input{childdoc.def}|\\
|\childdocforwardprefix{final}{child}|
\end{tabular}
\end{center}
%

Note that when several versions of a main file and/or of each child file
are to be generated, it may be convenient to set up a |Makefile| or
shell script to automatise the process.

%%%%%%%%%%%%%%%%%%%%%%%%%%%%%%%%%%%%%%%%%%%%%%%%%%%%%%%%%%%%%%%%%%%%%%%%%%%%%%%%
\subsection{Command Line Processing}
\label{sec:commandline}

The effect of redirection files can also be achieved by invoking
the \LaTeX{} compiler with a more elaborate command line.
Most conveniently this should be done as part
of a shell script or a |Makefile|.

When using \textsf{childdoc} in the main file, the following
command lines effectively perform a redirection
(note that depending on the shell being used,
backslashes may have to be doubled: `|\|' $\to$ `|\\|'):
%
\begin{center}
|... -jobname "|\textit{target}|" |\\|"|[\textit{flags}]%
|\input{childdoc.def}\childdocforward[|\textit{main}|]{|\textit{dest}|}"|
\end{center}
%
Here \textit{target} is the name of the output file,
\textit{main} is the name of the main file
and \textit{dest} is the name of the main or child file to be processed
(all filenames without extensions).
The optional argument \textit{main} can be omitted
if \textit{main} matches \textit{dest}.
Optionally, compilation \textit{flags} can be defined via |\def| commands.
This command line makes the \TeX{} engine believe
it is compiling the file \textit{target}
whose content is specified as the latter parameter.
The provided code then forwards the processing to
\textit{main} or \textit{dest} as described in \secref{sec:forward}.

%%%%%%%%%%%%%%%%%%%%%%%%%%%%%%%%%%%%%%%%%%%%%%%%%%%%%%%%%%%%%%%%%%%%%%%%%%%%%%%%
\subsection{Include by Input}
\label{sec:input}

Including child documents by |\include| has some restrictions by design.
Most notably, the content of a child document always occupies
its own set of pages; pages cannot be shared between child documents.
Usually, this behaviour makes perfect sense
because each child document contain an essential part of the document.
However, in some situations it may be desirable to compose
a document from a collection of parts
without having mandatory page breaks between then.
For this case, the package
provides a mechanism to include parts
by |\input| which can also be processed individually.
However, by construction this mechanism
requires manual handling of the content to be output.

%%%%%%%%%%%%%%%%%%%%%%%%%%%%%%%%%%%%%%%%
\DescribeMacro{\ifchilddocmanual}
The main file should be prepared as usual, see \secref{sec:include}.
However, the document body must make a distinction
between processing of an individual part and of the main document, e.g.:
%
\begin{center}
\begin{tabular}{l}
|\ifchilddocmanual|\\
|\input{\childdocname}|\\
|\||else|\\
\textit{document body with }|\input{|\textit{part}|}|\\
|\||fi|
\end{tabular}
\end{center}
%
The conditional |\ifchilddocmanual| is true whenever
a part to be included by |\input| is being compiled,
and the name of the part is stored in |\childdocname|.

%%%%%%%%%%%%%%%%%%%%%%%%%%%%%%%%%%%%%%%%
\DescribeMacro{\childdocby}
Each part to be included by |\input| should start with:
%
\begin{center}
\begin{tabular}{l}
|\input{childdoc.def}|\\
|\childdocby{|\textit{main}|}|\\
\end{tabular}
\end{center}
%
The directive |\childdocby| is similar to |\childdocof|
described in \secref{sec:include},
but the subsequent selection of content must be done manually.
To that end, both |\ifchilddoc| and |\ifchilddocmanual|
will be true upon processing of a part,
and the name of the part is stored in |\childdocname|.
Note that |\jobname| will be set to the filename of the current part
so that each part receives an individual |.aux| file
that does not interfere with the |.aux| file(s) of the main document.
This behaviour can be altered by the alternative form
|\childdocby[*]{|\textit{main}|}| (with a non-empty optional argument)
which uses the |.aux| file of the main document
by setting |\jobname| to \textit{main}.

%%%%%%%%%%%%%%%%%%%%%%%%%%%%%%%%%%%%%%%%%%%%%%%%%%%%%%%%%%%%%%%%%%%%%%%%%%%%%%%%
\subsection{Driver Development}
\label{sec:driver}

The \textsf{childdoc} mechanism can also be use for the development
of definition files such as \LaTeX{} styles or classes.
This case differs from the above setup with multiple parts
included by |\include| in that no |\includeonly| should be invoked.
This can be achieved by starting the include file
(before |\ProvidesPackage|) with:
%
\begin{center}
\begin{tabular}{l}
|\input{childdoc.def}|\\
|\childdocforward{|\textit{main}|}|\\
\end{tabular}
\end{center}
%
or alternatively with:
%
\begin{center}
\begin{tabular}{l}
|\input{childdoc.def}|\\
|\childdocby{|\textit{main}|}|\\
\end{tabular}
\end{center}
%
Both forms have slightly different effects as described above.
The main file is prepared as usual, see \secref{sec:include}.

%%%%%%%%%%%%%%%%%%%%%%%%%%%%%%%%%%%%%%%%%%%%%%%%%%%%%%%%%%%%%%%%%%%%%%%%%%%%%%%%
\subsection{Legacy Detection}
\label{sec:detection}

The directive |\childdocmain| in the main file can detect
whether the complete document or merely a child is to be compiled
even without using the directive |\childdocof|.
This method is deprecated because it is less robust
and there is no compelling reason to use it;
it is merely provided for backward compatibility
and it may be removed in future versions.

If the detection mechanism is to be used,
it is mandatory to correctly specify
the filename of the main file as the argument of |\childdocmain|:
%
\begin{center}
\begin{tabular}{l}
|\input{childdoc.def}|\\
|\childdocmain{|\textit{main}|}|\\
\end{tabular}
\end{center}
%
If |\jobname| does not match the argument \textit{main} of |\childdocmain|,
it is assumed that |\jobname| points to the child file to be compiled.
When using |\childdocmain| with the main file specified as argument,
it suffices to start a child file
with just |\input{|\textit{main}|}|
without loading of the package and using |\childdocof|.
If instead all processing is done
with the appropriate \textsf{childdoc} directives,
the argument of \textit{main} of |\childdocmain| can be empty.

An alternative version of the command line processing described
in \secref{sec:commandline} using the detection mechanism reads:
%
\begin{center}
|... -jobname "|\textit{target}|" "|[\textit{flags}]%
[|\def\jobname{|\textit{dest}|}|]|\input{|\textit{main}|}"|
\end{center}

%%%%%%%%%%%%%%%%%%%%%%%%%%%%%%%%%%%%%%%%%%%%%%%%%%%%%%%%%%%%%%%%%%%%%%%%%%%%%%%%
\subsection{Manual Code}
\label{sec:manual}

In case one cannot be certain whether the definitions file |childdoc.def|
is installed on the target \TeX{} distribution
and one prefers not to ship it,
it is conceivable to paste a few relevant commands into the sources.

To that end, drop all statements |\input{childdoc.def}|
and perform the replacements as outlined below.
Instead of |\childdocmain{|\textit{main}|}| add the following code
to the top of the main file:
%
\begin{center}
\begin{tabular}{l}
|\||ifdefined\childdocname\endinput\||fi\newif\ifchilddoc|\\
|\edef\childdocname{\scantokens\expandafter{\jobname\noexpand}}|\\
|\def\childdocmain{|\textit{main}|}\||ifx\childdocmain\childdocname\||else|\\
|\childdoctrue\includeonly{\childdocname}\let\jobname\childdocmain\||fi|\\
\end{tabular}
\end{center}
%
Instead of |\childdocof{|\textit{main}|}| just include the main file
at the top of each child file:
%
\begin{center}
|\input{|\textit{main}|}|
\end{center}
%
A simple redirection |\childdocforward{|\textit{dest}|}| is achieved by:
%
\begin{center}
|\def\jobname{|\textit{dest}|}\input{\jobname}|
\end{center}
%
The redirection with prefix
|\childdocforwardprefix[|\textit{prefix}|]{|\textit{dest}|}|
is accomplished by:
%
\begin{center}
\begin{tabular}{l}
|{\edef\jobname{\scantokens\expandafter{\jobname\noexpand}}|\\
|\def\redirectjob |\textit{prefix}|#1~~~{\gdef\jobname{|\textit{dest}|#1}}|\\
|\expandafter\redirectjob\jobname~~~}\input{\jobname}|
\end{tabular}
\end{center}

In an alternative approach,
child documents can be compiled by a specific command line
without additional code or specific definitions:
%
\begin{center}
|... -jobname "|\textit{target}|" "|[\textit{flags}]%
|\includeonly{|\textit{dest}|}\input{|\textit{main}|}"|
\end{center}
%

%%%%%%%%%%%%%%%%%%%%%%%%%%%%%%%%%%%%%%%%%%%%%%%%%%%%%%%%%%%%%%%%%%%%%%%%%%%%%%%%
%%%%%%%%%%%%%%%%%%%%%%%%%%%%%%%%%%%%%%%%%%%%%%%%%%%%%%%%%%%%%%%%%%%%%%%%%%%%%%%%
\section{Information}

%%%%%%%%%%%%%%%%%%%%%%%%%%%%%%%%%%%%%%%%%%%%%%%%%%%%%%%%%%%%%%%%%%%%%%%%%%%%%%%%
\subsection{Copyright}

Copyright \copyright{} 2017--2018 Niklas Beisert

This work may be distributed and/or modified under the
conditions of the \LaTeX{} Project Public License, either version 1.3
of this license or (at your option) any later version.
The latest version of this license is in
  \url{http://www.latex-project.org/lppl.txt}
and version 1.3 or later is part of all distributions of \LaTeX{}
version 2005/12/01 or later.

This work has the LPPL maintenance status `maintained'.

The Current Maintainer of this work is Niklas Beisert.

This work consists of the files |README.txt|, |childdoc.ins| and |childdoc.dtx|
as well as the derived files |childdoc.def|, |cdocsamp.tex|
with |cdocsch1.tex|, |cdocsch2.tex|, |cdocspt3.tex|, |cdocspt4.tex|,
|cdocsdrf.tex|, |cdocsfn1.tex|, |cdocsfn2.tex|
as well as |childdoc.pdf|.

%%%%%%%%%%%%%%%%%%%%%%%%%%%%%%%%%%%%%%%%%%%%%%%%%%%%%%%%%%%%%%%%%%%%%%%%%%%%%%%%
\subsection{Files and Installation}

The package consists of the files:
%
\begin{center}
\begin{tabular}{ll}
    |README.txt|   & readme file \\
    |childdoc.ins| & installation file \\
    |childdoc.dtx| & source file \\
    |childdoc.def| & definition file \\
    |cdocsamp.tex| & sample main file \\
    |cdocsch1.tex| & sample include file \\
    |cdocsch2.tex| & sample include file \\
    |cdocspt3.tex| & sample part file \\
    |cdocspt4.tex| & sample part file \\
    |cdocsdrf.tex| & sample redirection file \\
    |cdocsfn1.tex| & sample redirection file \\
    |cdocsfn2.tex| & sample redirection file \\
    |childdoc.pdf| & manual
\end{tabular}
\end{center}
%
The distribution consists of the files
|README.txt|, |childdoc.ins| and |childdoc.dtx|.
%
\begin{itemize}
\item
Run (pdf)\LaTeX{} on |childdoc.dtx|
to compile the manual |childdoc.pdf| (this file).
\item
Run \LaTeX{} on |childdoc.ins| to create the definitions file |childdoc.def|
and the sample |cdocsamp.tex| with include files
|cdocsch1.tex|, |cdocsch2.tex|, |cdocspt3.tex|, |cdocspt4.tex|,
|cdocsdrf.tex|, |cdocsfn1.tex|, |cdocsfn2.tex|.
Then copy the file |childdoc.def| to an appropriate directory of your \LaTeX{}
distribution, e.g.\ \textit{texmf-root}|/tex/latex/childdoc|.
\end{itemize}

%%%%%%%%%%%%%%%%%%%%%%%%%%%%%%%%%%%%%%%%%%%%%%%%%%%%%%%%%%%%%%%%%%%%%%%%%%%%%%%%
\subsection{Related CTAN Packages}

There are several other packages which offer a similar functionality:
%
\begin{itemize}
\item
The packages
\href{http://ctan.org/pkg/docmute}{\textsf{docmute}},
\href{http://ctan.org/pkg/includex}{\textsf{includex}} and
\href{http://ctan.org/pkg/standalone}{\textsf{standalone}}
provide commands to include only the document body of
a child file thus allowing both files to be compiled individually.
\item
The packages \href{http://ctan.org/pkg/subdocs}{\textsf{subdocs}}
and \href{http://ctan.org/pkg/subfiles}{\textsf{subfiles}}
provide structures in which the main and child documents can be
encapsulated and allowing them to be compiled individually.
The inclusion mechanism is different from the conventional |\include|.
\item
The package \href{http://ctan.org/pkg/combine}{\textsf{combine}}
is an elaborate solution to combine several documents into one.
\end{itemize}
%
See also the CTAN topic \href{http://ctan.org/topic/subdocs}{\textsf{subdocs}}
for further related packages.
The present package differs from the above solutions in that
a document structure constructed with the conventional |\include| mechanism
just needs two extra commands at the top of every file
such that all constituent files can be compiled individually.

%%%%%%%%%%%%%%%%%%%%%%%%%%%%%%%%%%%%%%%%%%%%%%%%%%%%%%%%%%%%%%%%%%%%%%%%%%%%%%%%
%\subsection{Feature Suggestions}
%
%The following is a list of features which may be useful for future
%versions of this package:
%%
%\begin{itemize}
%\item
%\ldots
%\end{itemize}

%%%%%%%%%%%%%%%%%%%%%%%%%%%%%%%%%%%%%%%%%%%%%%%%%%%%%%%%%%%%%%%%%%%%%%%%%%%%%%%%
\subsection{Revision History}

%%%%%%%%%%%%%%%%%%%%%%%%%%%%%%%%%%%%%%%%
\paragraph{v2.0:} 2018/12/30

\begin{itemize}
\item
immediate forward processing
\item
added |\childdocby| mechanism
\item
manual restructured
\end{itemize}

%%%%%%%%%%%%%%%%%%%%%%%%%%%%%%%%%%%%%%%%
\paragraph{v1.6:} 2018/01/17

\begin{itemize}
\item
application for development of include files
\item
corrections to manual
\end{itemize}

%%%%%%%%%%%%%%%%%%%%%%%%%%%%%%%%%%%%%%%%
\paragraph{v1.5:} 2017/05/21

\begin{itemize}
\item
more complete structuring introduced
\item
|\childdocof| introduced
\item
|\childdoc| renamed to |\childdocmain|
\item
|\childredirect| renamed to |\childdocforward| and |\childdocforwardprefix|
and functionality expanded
\end{itemize}

%%%%%%%%%%%%%%%%%%%%%%%%%%%%%%%%%%%%%%%%
\paragraph{v1.0:} 2017/04/27

\begin{itemize}
\item
manual and install package
\item
first version published on CTAN
\end{itemize}

%%%%%%%%%%%%%%%%%%%%%%%%%%%%%%%%%%%%%%%%
\paragraph{v0.6:} 2017/04/26

\begin{itemize}
\item
redirection mechanism added
\end{itemize}

%%%%%%%%%%%%%%%%%%%%%%%%%%%%%%%%%%%%%%%%
\paragraph{v0.5:} 2017/04/26

\begin{itemize}
\item
functionality in definition file
\end{itemize}


%%%%%%%%%%%%%%%%%%%%%%%%%%%%%%%%%%%%%%%%%%%%%%%%%%%%%%%%%%%%%%%%%%%%%%%%%%%%%%%%
%%%%%%%%%%%%%%%%%%%%%%%%%%%%%%%%%%%%%%%%%%%%%%%%%%%%%%%%%%%%%%%%%%%%%%%%%%%%%%%%
%%%%%%%%%%%%%%%%%%%%%%%%%%%%%%%%%%%%%%%%%%%%%%%%%%%%%%%%%%%%%%%%%%%%%%%%%%%%%%%%
\appendix

\settowidth\MacroIndent{\rmfamily\scriptsize 000\ }

 \DocInput{childdoc.dtx}

\end{document}
%</driver>
% \fi
%
% %%%%%%%%%%%%%%%%%%%%%%%%%%%%%%%%%%%%%%%%%%%%%%%%%%%%%%%%%%%%%%%%%%%%%%%%%%%%%%
% %%%%%%%%%%%%%%%%%%%%%%%%%%%%%%%%%%%%%%%%%%%%%%%%%%%%%%%%%%%%%%%%%%%%%%%%%%%%%%
% \section{Sample}
%\iffalse
%<*samplemain>
%\fi
%
% The following presents a sample document
% with two chapters, two parts, a title page,
% a compile flag as well as three forwarding files to set the flag.
% It consists of eight |.tex| files:
% \begin{center}
% \begin{tabular}{ll}
% |cdocsamp.tex|&main file\\
% |cdocsch1.tex|&include file for chapter 1\\
% |cdocsch2.tex|&include file for chapter 2\\
% |cdocspt3.tex|&include file for part 3\\
% |cdocspt4.tex|&include file for part 4\\
% |cdocsdrf.tex|&forwarding file for main file in draft mode\\
% |cdocsfi1.tex|&forwarding file for final version of chapter 1\\
% |cdocsfi2.tex|&forwarding file for final version of chapter 2\\
% \end{tabular}
% \end{center}
% Each of the eight files can be compiled directly by the \LaTeX{} compiler.
%
% %%%%%%%%%%%%%%%%%%%%%%%%%%%%%%%%%%%%%%
% \paragraph{Main File.}
%
% The main file is called |cdocsamp.tex|.
%
% Load the \textsf{childdoc} definitions and
% declare the filename for the main document:
%    \begin{macrocode}
\input{childdoc.def}
\childdocmain{}
%    \end{macrocode}

% Optional override for |\version| flag:
%    \begin{macrocode}
%%\ifchilddoc\else\providecommand{\version}{draft}\fi
%    \end{macrocode}

% Define the default values for the |\version| flag
% (|final| for the main file and |draft| for childs):
%    \begin{macrocode}
\ifchilddoc
\providecommand{\version}{draft}
\else
\providecommand{\version}{final}
\fi
%    \end{macrocode}

% Load the standard document class:
%    \begin{macrocode}
\documentclass[12pt]{article}
%    \end{macrocode}

% Start the document body:
%    \begin{macrocode}
\begin{document}
%    \end{macrocode}

% Declare a title page.
% Print title, part of document being processed and version flag:
%    \begin{macrocode}
\addtocounter{page}{-1}
\begin{center}
{\LARGE\bfseries{}childdoc example\par}
\vspace{1cm}
\ifchilddoc
\ifchilddocmanual part\else chapter\fi:
`\childdocname' of `\childdocjob'\par
\else
main document: `\childdocjob'\par
\fi
version: \version\par
\end{center}
\newpage
%    \end{macrocode}

% Manually include selected file,
% otherwise process as usual:
%    \begin{macrocode}
\ifchilddocmanual
\section*{part `\childdocname'}
\input{\childdocname}
\else
%    \end{macrocode}

% Include the two chapters:
%    \begin{macrocode}
\include{cdocsch1}
\include{cdocsch2}
%    \end{macrocode}

% Include the two parts unless only chapters should be displayed:
%    \begin{macrocode}
\ifchilddoc\else
\section{part three}
\input{cdocspt3}
\section{part four}
\input{cdocspt4}
\fi
%    \end{macrocode}

% Process as usual until here:
%    \begin{macrocode}
\fi
%    \end{macrocode}

% End of document body:
%    \begin{macrocode}
\end{document}
%    \end{macrocode}
%\iffalse
%</samplemain>
%\fi
%
% %%%%%%%%%%%%%%%%%%%%%%%%%%%%%%%%%%%%%%
% \paragraph{Chapter Include Files.}
%
% The include files are called |cdocsch1.tex| and |cdocsch2.tex|.
%
%\iffalse
%<*samplechap1|samplechap2>
%\fi

% Optional override for |\version| flag:
%    \begin{macrocode}
%%\providecommand{\version}{final}
%    \end{macrocode}

% Include the main document:
%    \begin{macrocode}
\input{childdoc.def}
\childdocof{cdocsamp}
%    \end{macrocode}

%\iffalse
%</samplechap1|samplechap2>
%\fi
%
%\iffalse
%<*samplechap1>
%\fi
% Some text for chapter 1:
%    \begin{macrocode}
\section{one}
some text in chapter one
%    \end{macrocode}

%\iffalse
%</samplechap1>
%\fi
% Some text for chapter 2:
%\iffalse
%<*samplechap2>
%\fi
%    \begin{macrocode}
\section{two}
more text in chapter two
%    \end{macrocode}

%\iffalse
%</samplechap2>
%\fi
%
% %%%%%%%%%%%%%%%%%%%%%%%%%%%%%%%%%%%%%%
% \paragraph{Part Include Files.}
%
% The include files are called |cdocspt3.tex| and |cdocspt4.tex|.
%
%\iffalse
%<*samplepart3|samplepart4>
%\fi

% Optional override for |\version| flag:
%    \begin{macrocode}
%%\providecommand{\version}{final}
%    \end{macrocode}

% Include the main document:
%    \begin{macrocode}
\input{childdoc.def}
\childdocby{cdocsamp}
%    \end{macrocode}

%\iffalse
%</samplepart3|samplepart4>
%\fi
%
%\iffalse
%<*samplepart3>
%\fi
% Some text for part 3:
%    \begin{macrocode}
some text in part three
%    \end{macrocode}

%\iffalse
%</samplepart3>
%\fi
% Some text for part 4:
%\iffalse
%<*samplepart4>
%\fi
%    \begin{macrocode}
more text in part four
%    \end{macrocode}

%\iffalse
%</samplepart4>
%\fi
%
% %%%%%%%%%%%%%%%%%%%%%%%%%%%%%%%%%%%%%%
% \paragraph{Forwarding for a Complete Draft.}
%
% The following forwarding file |cdocsdrf.tex|
% compiles the main document in draft mode:
%\iffalse
%<*sampledraft>
%\fi
%    \begin{macrocode}
\def\version{draft}
\input{childdoc.def}
\childdocforward{cdocsamp}
%    \end{macrocode}

%\iffalse
%</sampledraft>
%\fi
%
% %%%%%%%%%%%%%%%%%%%%%%%%%%%%%%%%%%%%%%
% \paragraph{Forwarding for Final Version of the Chapters.}
%
% The following forwarding files |cdocsfn1.tex| and |cdocsfn2.tex|
% (with identical content)
% compile the final versions of the child documents
% |cdocsch1.tex| and |cdocsch2.tex|, respectively:
%\iffalse
%<*samplefinal>
%\fi
%    \begin{macrocode}
\def\version{final}
\input{childdoc.def}
\childdocforwardprefix[cdocsamp]{cdocsfn}{cdocsch}
%    \end{macrocode}

%\iffalse
%</samplefinal>
%\fi
%
% %%%%%%%%%%%%%%%%%%%%%%%%%%%%%%%%%%%%%%
% \paragraph{Command Line Processing.}
%
% The following three command lines generate the output files
% |cdocscld|, |cdocscl1| and |cdocscl2|
% which should be identical to
% |cdocsdrf|, |cdocsch1| and |cdocsfn2|, respectively:
% \begin{center}
% \begin{tabular}{l}
% |latex -jobname cdocscld \|\\
% |  "\def\version{draft}\input{childdoc.def}\childdocforward{cdocsamp}"|\\
% |latex -jobname cdocscl1 \|\\
% |  "\input{childdoc.def}\childdocforward[cdocsamp]{cdocsch1}"|\\
% |latex -jobname cdocscl2 \|\\
% |  "\def\version{final}\input{childdoc.def}\childdocforward{cdocsch2}"|
% \end{tabular}
% \end{center}
% Note that the trailing backslash on each first line
% merely continues the input to the second line
% (for convenient cut ant paste).
% Furthermore, the command |latex| can be replaced by any
% of its alternative versions such as |pdflatex|.
%
% %%%%%%%%%%%%%%%%%%%%%%%%%%%%%%%%%%%%%%%%%%%%%%%%%%%%%%%%%%%%%%%%%%%%%%%%%%%%%%
% %%%%%%%%%%%%%%%%%%%%%%%%%%%%%%%%%%%%%%%%%%%%%%%%%%%%%%%%%%%%%%%%%%%%%%%%%%%%%%
% \section{Implementation}
%\iffalse
%<*package>
%\fi
%
% This section describes the definitions file |childdoc.def|.

% The definitions cannot be loaded using |\usepackage| or |\RequirePackage|
% which has a mechanism to prevent loading a style file more than once.
% When loading the definitions by means of |\input|
% multiple instances have to be prevented manually:
%\iffalse
%This code needs to be before the `\ProvidesFile' directive
%which is defined at the beginning of this file.
%Therefore it is also placed there and commented out here.
%</package>
%<*discard>
%\fi
%    \begin{macrocode}
\ifdefined\childdocmain\endinput\fi
%    \end{macrocode}
%\iffalse
%</discard>
%<*package>
%\fi
%
% \macro{\ifchilddoc}
% \macro{\ifchilddocmanual}
% The conditional |\ifchilddoc| tells whether a
% child (true) or main (false) document is being compiled.
% The conditional |\ifchilddocmanual| tells whether
% the |\includeonly| mechanism is used (false) or
% the selection of child files must be performed manually (true).
% The definitions initialise to false:
%    \begin{macrocode}
\newif\ifchilddoc
\newif\ifchilddocmanual
%    \end{macrocode}

% \macro{\childdocname}
% \macro{\childdocjob}
% The macro |\childdocname| stores the name of the main document
% to be compiled. The macro |\childdocjob| stores the name of
% the document on which the \LaTeX{} compiler was originally invoked.
% The content of |\jobname| cannot be compared
% to filenames specified in the source due to different catcodes.
% The following code rescans |\jobname|, stores the result
% in |\childdocname| and saves a copy in |\childdocjob|:
%    \begin{macrocode}
\edef\childdocname{\scantokens\expandafter{\jobname\noexpand}}
\let\childdocjob\childdocname
%    \end{macrocode}

% \macro{\childdocdisable}
% The macro |\childdocdisable| prevents the main file
% from being processed more than once.
% At this stage, the main document command |\childdocmain|
% is assumed to be called once again where it should do nothing.
% Any subsequent call to it should prevent
% a secondary processing of the main document
% It overwrites the forwarding commands
% |\childdocof| and |\childdocforward|
% with empty macros to prevent further inclusions of the main document:
%    \begin{macrocode}
\newcommand{\childdocdisable}
{
  \renewcommand{\childdocmain}[1]{\renewcommand{\childdocmain}[1]{\endinput}}
  \renewcommand{\childdocof}[1]{}
  \renewcommand{\childdocby}[2][]{}
  \renewcommand{\childdocforward}[2][]{}
  \renewcommand{\childdocdisable}{}
}
%    \end{macrocode}

% \macro{\childdocmain}
% The macro |\childdocmain| is to be called at the top of the main file
% with nothing or the main filename (without extension) as argument.
% First, it breaks loops.
% If the argument is not empty and does not match |\childdocname|
% (which is set by the first inclusion of |childdoc.def|),
% |\ifchilddoc| is set to true, |\includeonly| is applied to the child file
% and |\jobname| is set to the main file
% (for proper handling of |.aux| files):
%    \begin{macrocode}
\newcommand{\childdocmain}[1]
{
  \childdocdisable\childdocmain{}
  \if?#1?\else
    \begingroup
      \def\childdoctmp{#1}
      \ifx\childdoctmp\childdocname
        \def\childdoctmp{}
      \else
        \def\childdoctmp
        {
          \childdoctrue
          \includeonly{\childdocname}
          \def\childdocjob{#1}
          \def\jobname{#1}
        }
      \fi
      \expandafter
    \endgroup
    \childdoctmp
  \fi
}
%    \end{macrocode}

% \macro{\childdocof}
% The command |\childdocof| redirects
% compilation to the main file |#1|.
%    \begin{macrocode}
\newcommand{\childdocof}[1]
{
  \childdocdisable
  \childdoctrue
  \includeonly{\childdocname}
  \def\jobname{#1}
  \def\childdocjob{#1}
  \input{#1}
}
%    \end{macrocode}

% \macro{\childdocby}
% The command |\childdocby| ....
%    \begin{macrocode}
\newcommand{\childdocby}[2][]
{
  \childdocdisable
  \childdoctrue
  \childdocmanualtrue
  \if?#1?\else
    \def\jobname{#2}
  \fi
  \def\childdocjob{#2}
  \input{#2}
  \endinput
}
%    \end{macrocode}

% \macro{\childdocforward}
% The command |\childdocforward| redirects
% compilation to the main file or
% (if the optional argument is given) a child file.
% Parameters are set as if the main file
% or a child file starting with |\childdocof| was compiled.
% Then compilation is handed over to the main file:
%    \begin{macrocode}
\newcommand{\childdocforward}[2][]
{
  \begingroup
    \if?#1?
      \def\childdoctmp
      {
        \def\childdocname{#2}
        \def\childdocjob{#2}
        \def\jobname{#2}
        \input{#2}
        \endinput
      }
    \else
      \def\childdoctmp
      {
        \childdocdisable
        \def\childdocname{#2}
        \childdoctrue
        \includeonly{#2}
        \def\childdocjob{#1}
        \def\jobname{#1}
        \input{#1}
        \endinput
      }
    \fi
    \expandafter
  \endgroup
  \childdoctmp
}
%    \end{macrocode}

% \macro{\childdocforwardprefix}
% The command |\childdocforwardprefix| redirects
% compilation to the main or a child file by means of a pattern.
% The prefix |#1| in the current filename is replaced by |#2|
% and the suffix of the current filename is kept
% (it is assumed that the filename does not contain the substring `|~~~|'
% which is used as a delimiter).
% Compilation is handed over to the new file by |\childdocforward|:
%    \begin{macrocode}
\newcommand{\childdocforwardprefix}[3][]
{
  \begingroup
    \def\childdocextract #2##1~~~{\def\childdoctmp{\childdocforward[#1]{#3##1}}}
    \expandafter\childdocextract\childdocname~~~
    \expandafter
  \endgroup
  \childdoctmp
}
%    \end{macrocode}

% \macro{\childdoc}
% The deprecated macro |\childdoc| is a legacy version of |\childdocmain|:
%    \begin{macrocode}
\newcommand{\childdoc}{\childdocmain}
%    \end{macrocode}

% \macro{\childdocredirect}
% The deprecated macro |\childdocredirect| is a legacy version
% of |\childdocforward| and |\childdocforwardprefix|:
%    \begin{macrocode}
\newcommand{\childdocredirect}[2][]
{
  \begingroup
    \if?#1?
      \def\childdoctmp{\childdocforward{#2}}
    \else
      \def\childdoctmp{\childdocforwardprefix{#1}{#2}}
    \fi
    \expandafter
  \endgroup
  \childdoctmp
}
%    \end{macrocode}

%\iffalse
%</package>
%\fi
%
\endinput

\childdocforwardprefix[cdocsamp]{cdocsfn}{cdocsch}
%    \end{macrocode}

%\iffalse
%</samplefinal>
%\fi
%
% %%%%%%%%%%%%%%%%%%%%%%%%%%%%%%%%%%%%%%
% \paragraph{Command Line Processing.}
%
% The following three command lines generate the output files
% |cdocscld|, |cdocscl1| and |cdocscl2|
% which should be identical to
% |cdocsdrf|, |cdocsch1| and |cdocsfn2|, respectively:
% \begin{center}
% \begin{tabular}{l}
% |latex -jobname cdocscld \|\\
% |  "\def\version{draft}% \iffalse
%
% childdoc.dtx Copyright (C) 2017-2018 Niklas Beisert
%
% This work may be distributed and/or modified under the
% conditions of the LaTeX Project Public License, either version 1.3
% of this license or (at your option) any later version.
% The latest version of this license is in
%   http://www.latex-project.org/lppl.txt
% and version 1.3 or later is part of all distributions of LaTeX
% version 2005/12/01 or later.
%
% This work has the LPPL maintenance status `maintained'.
%
% The Current Maintainer of this work is Niklas Beisert.
%
% This work consists of the files childdoc.dtx and childdoc.ins
% and the derived files childdoc.def and cdocsamp.tex with
% cdocsch1.tex, cdocsch2.tex, cdocsdrf.tex, cdocsfn1.tex, cdocsfn2.tex.
%
%<package>\ifdefined\childdocmain\endinput\fi
%<package>\ProvidesFile{childdoc.def}[2018/12/30 v2.0 child document driver]
%<samplemain>\ProvidesFile{cdocsamp.tex}[2018/12/30 v2.0 sample for childdoc]
%<*driver>
%\ProvidesFile{childdoc.drv}[2018/12/30 v2.0 childdoc reference manual file]
\PassOptionsToClass{10pt,a4paper}{article}
\documentclass{ltxdoc}

\usepackage[margin=35mm]{geometry}
\usepackage{hyperref}
\usepackage{hyperxmp}
\usepackage[usenames]{color}

\hypersetup{colorlinks=true}
\hypersetup{pdfstartview=FitH}
\hypersetup{pdfpagemode=UseNone}
\hypersetup{pdfsource={}}
\hypersetup{pdflang={en-UK}}
\hypersetup{pdfcopyright={Copyright 2017-2018 Niklas Beisert.
  This work may be distributed and/or modified under the
  conditions of the LaTeX Project Public License, either version 1.3
  of this license or (at your option) any later version.}}
\hypersetup{pdflicenseurl={http://www.latex-project.org/lppl.txt}}
\hypersetup{pdfcontactaddress={ETH Zurich, ITP, HIT K,
  Wolfgang-Pauli-Strasse 27}}
\hypersetup{pdfcontactpostcode={8093}}
\hypersetup{pdfcontactcity={Zurich}}
\hypersetup{pdfcontactcountry={Switzerland}}
\hypersetup{pdfcontactemail={nbeisert@itp.phys.ethz.ch}}
\hypersetup{pdfcontacturl={http://people.phys.ethz.ch/\xmptilde nbeisert/}}

\newcommand{\secref}[1]{\hyperref[#1]{section \ref*{#1}}}

\parskip1ex
\parindent0pt
\let\olditemize\itemize
\def\itemize{\olditemize\parskip0pt}

\begin{document}

\title{The \textsf{childdoc} Package}
\hypersetup{pdftitle={The childdoc Package}}
\author{Niklas Beisert\\[2ex]
  Institut f\"ur Theoretische Physik\\
  Eidgen\"ossische Technische Hochschule Z\"urich\\
  Wolfgang-Pauli-Strasse 27, 8093 Z\"urich, Switzerland\\[1ex]
  \href{mailto:nbeisert@itp.phys.ethz.ch}
  {\texttt{nbeisert@itp.phys.ethz.ch}}}
\hypersetup{pdfauthor={Niklas Beisert}}
\hypersetup{pdfsubject={Manual for the LaTeX2e Package childdoc}}
\date{30 December 2018, \textsf{v2.0}}
\maketitle

\begin{abstract}\noindent
\textsf{childdoc} is a \LaTeXe{} package
that enables the direct compilation
of document sections included by |\include|
to individual files.
\end{abstract}

\begingroup
\parskip0ex
\tableofcontents
\endgroup

%%%%%%%%%%%%%%%%%%%%%%%%%%%%%%%%%%%%%%%%%%%%%%%%%%%%%%%%%%%%%%%%%%%%%%%%%%%%%%%%
%%%%%%%%%%%%%%%%%%%%%%%%%%%%%%%%%%%%%%%%%%%%%%%%%%%%%%%%%%%%%%%%%%%%%%%%%%%%%%%%
\section{Introduction}

\LaTeX{} provides a mechanism to structure a large document (such as a book)
into a main file and several child files (containing the chapters)
using the |\include| command.
This mechanism is beneficial for documents
which span hundreds of pages in order to
make the source file(s) more manageable.
Moreover, compilation can be restricted to
selected child files by means of the |\includeonly| command.
The latter feature can be used to reduce the compilation time while editing
(this was significantly more useful in the earlier days of \LaTeX{})
or to generate a smaller document which is easier to navigate.
Another application of |\includeonly| is to generate
documents consisting of selected parts of the complete document.

However, there are a few drawbacks of the plain |\include| mechanism:
\begin{itemize}
\item
The child files cannot be compiled on their own,
they can only be compiled via the main file.
A naive editing environment
(such as a text editor with an option
to have the current file processed by \LaTeX)
may require one to switch to the main file before compiling;
attempting to compile the child file produces errors.
\item
The main file must be modified (each time)
to adjust the |\includeonly| command
to the present needs. This easily leaves the main file in a messy state.
\item
The generated document will always carry the filename
of the main document. This is inconvenient if
several child files are to be compiled and
to be kept for distribution.
\end{itemize}

The present package provides a simple interface
to make child files individually compilable by \LaTeX{}.
Compiling a child file then has the same effect as compiling
the main file with an |\includeonly| command
to select the appropriate child.
Moreover the generated document will carry the name of the child
rather than the main file.
This resolves all three above issues.

This feature is meant to make the editing of books,
thesis documents and lecture notes somewhat more convenient.
However, the package can also be used efficiently for
composing a series of documents (such as exercise sheets)
which are typically distributed individually.
It then assists the author in generating the individual documents
(potentially in different versions)
as well as a document containing the collected series.
Another application is in developing style files
or other kinds of included material
where compilation of the style file could redirect
to a sample or test file.

%%%%%%%%%%%%%%%%%%%%%%%%%%%%%%%%%%%%%%%%%%%%%%%%%%%%%%%%%%%%%%%%%%%%%%%%%%%%%%%%
%%%%%%%%%%%%%%%%%%%%%%%%%%%%%%%%%%%%%%%%%%%%%%%%%%%%%%%%%%%%%%%%%%%%%%%%%%%%%%%%
\section{Usage}

First of all, the package \textsf{childdoc} is \emph{not} a standard
\LaTeXe{} |.sty| style file! Therefore it needs to be invoked in
a non-standard way.

%%%%%%%%%%%%%%%%%%%%%%%%%%%%%%%%%%%%%%%%%%%%%%%%%%%%%%%%%%%%%%%%%%%%%%%%%%%%%%%%
\subsection{Included Files}
\label{sec:include}

%%%%%%%%%%%%%%%%%%%%%%%%%%%%%%%%%%%%%%%%
\DescribeMacro{\childdocmain}
To use the package, add the commands
\begin{center}
\begin{tabular}{l}
|\input{childdoc.def}|\\
|\childdocmain{}|\\
\end{tabular}
\end{center}
at the very top of the main \LaTeX{} file,
in particular \emph{before} the |\documentclass| statement!
The argument of |\childdocmain| should be left empty
(but it must be present).

%%%%%%%%%%%%%%%%%%%%%%%%%%%%%%%%%%%%%%%%
\DescribeMacro{\childdocof}
Furthermore, add the commands
\begin{center}
\begin{tabular}{l}
|\input{childdoc.def}|\\
|\childdocof{|\textit{main}|}|\\
\end{tabular}
\end{center}
at the top of every child file \textit{child}
which is included by |\include{|\textit{child}|}|
from within the main file
(or at least for those files to be compiled individually).
The argument \textit{main} must be the filename of the main file.

There are a couple of
considerations in setting up the main and child documents:

%%%%%%%%%%%%%%%%%%%%%%%%%%%%%%%%%%%%%%%%
\paragraph{Restrictions.}

Please note the following restrictions:
\begin{itemize}
\item
|\childdocmain| must be called with one argument \textit{main}
to ensure compatibility with earlier version of the package.
It must either be empty (|\childdocmain{}|)
or precisely match the filename of the main file in which it is specified.
See \secref{sec:detection} for further information.
\item
The filename \textit{main} must be specified without the |.tex| extension.
\item
The filename \textit{main} is case sensitive
(even in case-insensitive file systems)
due to internal string comparison.
\item
The argument \textit{main} should be fully expanded, it cannot be a macro.
\item
Subdirectories and special characters should be avoided in filenames.
\item
The command |\childdocmain{|\textit{main}|}| must be followed by a whitespace.
It should not be followed immediately by another command
or by a comment mark `|%|'.
This is because the \TeX{} parser reads the token immediately following
the argument of |\childdocmain| and puts it
at the beginning of every child section;
however, a white\-space is ignored.
\end{itemize}

%%%%%%%%%%%%%%%%%%%%%%%%%%%%%%%%%%%%%%%%
\paragraph{Content of Main File.}

It is advisable to place all content in the child files included by |\include|.
Any output contained in the main file will appear in all child documents
unless suppressed manually;
it cannot be suppressed automatically by the |\includeonly| directive
and thus should normally be avoided.
A method to include some content in the main file
by means of conditional processing is described in \secref{sec:conditional}.

%%%%%%%%%%%%%%%%%%%%%%%%%%%%%%%%%%%%%%%%
\paragraph{Page Numbering.}

When only a part of the document is compiled,
the appropriate numbering of pages
(as well as other status parameters)
is determined from the |.aux| files.
The latter contain information from previous passes.
However this information needs to propagate through
all intermediate child documents.
Therefore the page numbering in child documents may well
be inconsistent until the complete document is compiled at least once.

A useful (if unconventional) way to always ensure a consistent
page numbering is to restart the numbering in each child document
and denote the pages by `\textit{child}|.|\textit{page}'
where \textit{child} represents the chapter/section number of the child file.
This can be achieved by the command
|\numberwithin{page}{|\textit{child}|}|
of the \textsf{amsmath} package
where \textit{child} can be |chapter| or |section|
depending on the chosen structuring.
Alternatively, one can modify the macro |\thepage| appropriately
and reset the counter |page| at the start of each child file.

%%%%%%%%%%%%%%%%%%%%%%%%%%%%%%%%%%%%%%%%%%%%%%%%%%%%%%%%%%%%%%%%%%%%%%%%%%%%%%%%
\subsection{Conditional Processing}
\label{sec:conditional}

The package provides a mechanism to compile different versions
of a document. To customise the versions further some conditional processing
can come in handy to distinguish which version is being compiled.
The package provides two macros to describe the compilation context:

%%%%%%%%%%%%%%%%%%%%%%%%%%%%%%%%%%%%%%%%
\DescribeMacro{\ifchilddoc}
The conditional |\ifchilddoc| distinguishes between the compilation of
child documents and the main document:
%
\begin{center}
|\ifchilddoc |\textit{child-code}| |[|\||else |\textit{main-code}]| \||fi|
\end{center}

%%%%%%%%%%%%%%%%%%%%%%%%%%%%%%%%%%%%%%%%
\DescribeMacro{\childdocname}
\DescribeMacro{\childdocjob}
The macro |\childdocname| contains the filename (without extension)
of the main or child file being processed.
Note that |\childdocjob| will always contain the name of the main file.

%%%%%%%%%%%%%%%%%%%%%%%%%%%%%%%%%%%%%%%%
\paragraph{Title Page.}

Conditional processing can be used to include a title or banner page
in the main document when proper precautions are taken.
Importantly, the code in the main file should ensure that the page counter
(as well as other status parameters which are stored in the |.aux| files)
takes the same value after the conditional processing.
Otherwise the page numbers may take divergent values
depending on which part is compiled.

For example, a title page could be declared by:
%
\begin{center}
\begin{tabular}{l}
|\ifchilddoc\||else|\\
|\addtocounter{page}{-1}|\\
\textit{code for title page}\\
|\newpage|\\
|\||fi|
\end{tabular}
\end{center}
%
A banner page for the child documents can be generated by:
%
\begin{center}
\begin{tabular}{l}
|\ifchilddoc|\\
|\addtocounter{page}{-1}|\\
\textit{code for banner page}\\
|\newpage|\\
|\||fi|
\end{tabular}
\end{center}
%
Here one could write a message such as:
\begin{center}
|This is the part \childdocname{} of \childdocjob{}.|
\end{center}

%%%%%%%%%%%%%%%%%%%%%%%%%%%%%%%%%%%%%%%%%%%%%%%%%%%%%%%%%%%%%%%%%%%%%%%%%%%%%%%%
\subsection{Flags}
\label{sec:flags}

The package makes it easy to generate different versions
of the main or child documents.
To this end compilation flags can be defined
and assigned different default values.
They will be particularly useful in conjunction
with the forwarding mechanism described in \secref{sec:forward}.

For example, it may be useful to have a flag |\version|
which can be set to |draft| or |final|.
The document source will contain some conditional code
depending on the value of |\version|.
Suppose further, the flag should default to |final| for the main file
and to |draft| for child files
which is a natural assignment for editing the document.
This is achieved by placing the following code
in the preamble of the main document
(below the |\childdocmain| directive):
%
\begin{center}
\begin{tabular}{l}
|\ifchilddoc|\\
|\providecommand{\version}{draft}|\\
|\||else|\\
|\providecommand{\version}{final}|\\
|\||fi|
\end{tabular}
\end{center}
%
The definition by |\providecommand| makes sure
that previous definitions are not overwritten.
Further statements |\providecommand{\version}{...}|
can thus be added before the above code to override it.

For the main file, one might add a line
(between |\childdocmain| and the above block)
%
\begin{center}
|%\ifchilddoc\||else\providecommand{\version}{draft}\||fi|
\end{center}
%
which can be uncommented to produce a draft version.
Likewise one can add a line to the very top of a child file
(above the |\childdocof{|\textit{main}|}| directive)
%
\begin{center}
|%\providecommand{\version}{final}|
\end{center}
%
which can be uncommented to produce the final version of this child document.

%%%%%%%%%%%%%%%%%%%%%%%%%%%%%%%%%%%%%%%%%%%%%%%%%%%%%%%%%%%%%%%%%%%%%%%%%%%%%%%%
\subsection{Forwarding}
\label{sec:forward}

Different versions of the main or child documents
using compilation flags as described in \secref{sec:flags}
can be (permanently) stored in different files
for convenient compilation, viewing and distribution.
To this end, the package defines a command
to pass on compilation to a different file:

%%%%%%%%%%%%%%%%%%%%%%%%%%%%%%%%%%%%%%%%
\DescribeMacro{\childdocforward}
The command |\childdocforward| redirects processing to
another source file:
%
\begin{center}
\begin{tabular}{l}
|\input{childdoc.def}|\\
|\childdocforward[|\textit{main}|]{|\textit{dest}|}|\\
\end{tabular}
\end{center}
%
The argument \textit{dest} is the destination file
(without extension).
It should be the main file or one of the child files.
Note that further \textsf{childdoc} directives
such as |\childdocof| and |\childdocforward|
in the indicated file will be processed in this form.
The optional argument \textit{main}
passes on directly to the main file \textit{main}
while pretending to compile the child \textit{dest}.
This form behaves as if \textit{dest}
issues |\childdocof{|\textit{main}|}| right away,
and no further \textsf{childdoc} directives will be processed.

%%%%%%%%%%%%%%%%%%%%%%%%%%%%%%%%%%%%%%%%
\DescribeMacro{\...prefix}
In the alternative form |\childdocforwardprefix|,
%
\begin{center}
\begin{tabular}{l}
|\input{childdoc.def}|\\
|\childdocforwardprefix[|\textit{main}|]{|\textit{prefix}|}{|\textit{dest}|}|
\end{tabular}
\end{center}
%
the destination file is determined by a pattern
depending on the current file:
To make this work, the current file must be called
`{\textit{prefix}\hspace{0.2em}\textit{suffix}}'
with \textit{prefix} matching precisely the argument.
Processing is then passed on to the file
`{\textit{dest}\hspace{0.2em}\textit{suffix}}'.
Surely, the same effect is achieved by
directly specifying the
argument `{\textit{dest}\hspace{0.2em}\textit{suffix}}'
in the first form.
However, that requires to set up a different file
for each child. With the alternative form of the command
all these files can have exactly the same content
which simplifies setting them up and maintaining them.

For example, the following file |draft.tex|
with a compilation flag |\version| as described in \secref{sec:flags}
compiles the main document as a draft:
%
\begin{center}
\begin{tabular}{l}
|\def\version{draft}|\\
|\input{childdoc.def}|\\
|\childdocforward{|\textit{main}|}|
\end{tabular}
\end{center}
%
Likewise, the following files |final|\textit{nn}|.tex|
compile the final version of the child document
|child|\textit{nn}|.tex|:
%
\begin{center}
\begin{tabular}{l}
|\def\version{final}|\\
|\input{childdoc.def}|\\
|\childdocforwardprefix{final}{child}|
\end{tabular}
\end{center}
%

Note that when several versions of a main file and/or of each child file
are to be generated, it may be convenient to set up a |Makefile| or
shell script to automatise the process.

%%%%%%%%%%%%%%%%%%%%%%%%%%%%%%%%%%%%%%%%%%%%%%%%%%%%%%%%%%%%%%%%%%%%%%%%%%%%%%%%
\subsection{Command Line Processing}
\label{sec:commandline}

The effect of redirection files can also be achieved by invoking
the \LaTeX{} compiler with a more elaborate command line.
Most conveniently this should be done as part
of a shell script or a |Makefile|.

When using \textsf{childdoc} in the main file, the following
command lines effectively perform a redirection
(note that depending on the shell being used,
backslashes may have to be doubled: `|\|' $\to$ `|\\|'):
%
\begin{center}
|... -jobname "|\textit{target}|" |\\|"|[\textit{flags}]%
|\input{childdoc.def}\childdocforward[|\textit{main}|]{|\textit{dest}|}"|
\end{center}
%
Here \textit{target} is the name of the output file,
\textit{main} is the name of the main file
and \textit{dest} is the name of the main or child file to be processed
(all filenames without extensions).
The optional argument \textit{main} can be omitted
if \textit{main} matches \textit{dest}.
Optionally, compilation \textit{flags} can be defined via |\def| commands.
This command line makes the \TeX{} engine believe
it is compiling the file \textit{target}
whose content is specified as the latter parameter.
The provided code then forwards the processing to
\textit{main} or \textit{dest} as described in \secref{sec:forward}.

%%%%%%%%%%%%%%%%%%%%%%%%%%%%%%%%%%%%%%%%%%%%%%%%%%%%%%%%%%%%%%%%%%%%%%%%%%%%%%%%
\subsection{Include by Input}
\label{sec:input}

Including child documents by |\include| has some restrictions by design.
Most notably, the content of a child document always occupies
its own set of pages; pages cannot be shared between child documents.
Usually, this behaviour makes perfect sense
because each child document contain an essential part of the document.
However, in some situations it may be desirable to compose
a document from a collection of parts
without having mandatory page breaks between then.
For this case, the package
provides a mechanism to include parts
by |\input| which can also be processed individually.
However, by construction this mechanism
requires manual handling of the content to be output.

%%%%%%%%%%%%%%%%%%%%%%%%%%%%%%%%%%%%%%%%
\DescribeMacro{\ifchilddocmanual}
The main file should be prepared as usual, see \secref{sec:include}.
However, the document body must make a distinction
between processing of an individual part and of the main document, e.g.:
%
\begin{center}
\begin{tabular}{l}
|\ifchilddocmanual|\\
|\input{\childdocname}|\\
|\||else|\\
\textit{document body with }|\input{|\textit{part}|}|\\
|\||fi|
\end{tabular}
\end{center}
%
The conditional |\ifchilddocmanual| is true whenever
a part to be included by |\input| is being compiled,
and the name of the part is stored in |\childdocname|.

%%%%%%%%%%%%%%%%%%%%%%%%%%%%%%%%%%%%%%%%
\DescribeMacro{\childdocby}
Each part to be included by |\input| should start with:
%
\begin{center}
\begin{tabular}{l}
|\input{childdoc.def}|\\
|\childdocby{|\textit{main}|}|\\
\end{tabular}
\end{center}
%
The directive |\childdocby| is similar to |\childdocof|
described in \secref{sec:include},
but the subsequent selection of content must be done manually.
To that end, both |\ifchilddoc| and |\ifchilddocmanual|
will be true upon processing of a part,
and the name of the part is stored in |\childdocname|.
Note that |\jobname| will be set to the filename of the current part
so that each part receives an individual |.aux| file
that does not interfere with the |.aux| file(s) of the main document.
This behaviour can be altered by the alternative form
|\childdocby[*]{|\textit{main}|}| (with a non-empty optional argument)
which uses the |.aux| file of the main document
by setting |\jobname| to \textit{main}.

%%%%%%%%%%%%%%%%%%%%%%%%%%%%%%%%%%%%%%%%%%%%%%%%%%%%%%%%%%%%%%%%%%%%%%%%%%%%%%%%
\subsection{Driver Development}
\label{sec:driver}

The \textsf{childdoc} mechanism can also be use for the development
of definition files such as \LaTeX{} styles or classes.
This case differs from the above setup with multiple parts
included by |\include| in that no |\includeonly| should be invoked.
This can be achieved by starting the include file
(before |\ProvidesPackage|) with:
%
\begin{center}
\begin{tabular}{l}
|\input{childdoc.def}|\\
|\childdocforward{|\textit{main}|}|\\
\end{tabular}
\end{center}
%
or alternatively with:
%
\begin{center}
\begin{tabular}{l}
|\input{childdoc.def}|\\
|\childdocby{|\textit{main}|}|\\
\end{tabular}
\end{center}
%
Both forms have slightly different effects as described above.
The main file is prepared as usual, see \secref{sec:include}.

%%%%%%%%%%%%%%%%%%%%%%%%%%%%%%%%%%%%%%%%%%%%%%%%%%%%%%%%%%%%%%%%%%%%%%%%%%%%%%%%
\subsection{Legacy Detection}
\label{sec:detection}

The directive |\childdocmain| in the main file can detect
whether the complete document or merely a child is to be compiled
even without using the directive |\childdocof|.
This method is deprecated because it is less robust
and there is no compelling reason to use it;
it is merely provided for backward compatibility
and it may be removed in future versions.

If the detection mechanism is to be used,
it is mandatory to correctly specify
the filename of the main file as the argument of |\childdocmain|:
%
\begin{center}
\begin{tabular}{l}
|\input{childdoc.def}|\\
|\childdocmain{|\textit{main}|}|\\
\end{tabular}
\end{center}
%
If |\jobname| does not match the argument \textit{main} of |\childdocmain|,
it is assumed that |\jobname| points to the child file to be compiled.
When using |\childdocmain| with the main file specified as argument,
it suffices to start a child file
with just |\input{|\textit{main}|}|
without loading of the package and using |\childdocof|.
If instead all processing is done
with the appropriate \textsf{childdoc} directives,
the argument of \textit{main} of |\childdocmain| can be empty.

An alternative version of the command line processing described
in \secref{sec:commandline} using the detection mechanism reads:
%
\begin{center}
|... -jobname "|\textit{target}|" "|[\textit{flags}]%
[|\def\jobname{|\textit{dest}|}|]|\input{|\textit{main}|}"|
\end{center}

%%%%%%%%%%%%%%%%%%%%%%%%%%%%%%%%%%%%%%%%%%%%%%%%%%%%%%%%%%%%%%%%%%%%%%%%%%%%%%%%
\subsection{Manual Code}
\label{sec:manual}

In case one cannot be certain whether the definitions file |childdoc.def|
is installed on the target \TeX{} distribution
and one prefers not to ship it,
it is conceivable to paste a few relevant commands into the sources.

To that end, drop all statements |\input{childdoc.def}|
and perform the replacements as outlined below.
Instead of |\childdocmain{|\textit{main}|}| add the following code
to the top of the main file:
%
\begin{center}
\begin{tabular}{l}
|\||ifdefined\childdocname\endinput\||fi\newif\ifchilddoc|\\
|\edef\childdocname{\scantokens\expandafter{\jobname\noexpand}}|\\
|\def\childdocmain{|\textit{main}|}\||ifx\childdocmain\childdocname\||else|\\
|\childdoctrue\includeonly{\childdocname}\let\jobname\childdocmain\||fi|\\
\end{tabular}
\end{center}
%
Instead of |\childdocof{|\textit{main}|}| just include the main file
at the top of each child file:
%
\begin{center}
|\input{|\textit{main}|}|
\end{center}
%
A simple redirection |\childdocforward{|\textit{dest}|}| is achieved by:
%
\begin{center}
|\def\jobname{|\textit{dest}|}\input{\jobname}|
\end{center}
%
The redirection with prefix
|\childdocforwardprefix[|\textit{prefix}|]{|\textit{dest}|}|
is accomplished by:
%
\begin{center}
\begin{tabular}{l}
|{\edef\jobname{\scantokens\expandafter{\jobname\noexpand}}|\\
|\def\redirectjob |\textit{prefix}|#1~~~{\gdef\jobname{|\textit{dest}|#1}}|\\
|\expandafter\redirectjob\jobname~~~}\input{\jobname}|
\end{tabular}
\end{center}

In an alternative approach,
child documents can be compiled by a specific command line
without additional code or specific definitions:
%
\begin{center}
|... -jobname "|\textit{target}|" "|[\textit{flags}]%
|\includeonly{|\textit{dest}|}\input{|\textit{main}|}"|
\end{center}
%

%%%%%%%%%%%%%%%%%%%%%%%%%%%%%%%%%%%%%%%%%%%%%%%%%%%%%%%%%%%%%%%%%%%%%%%%%%%%%%%%
%%%%%%%%%%%%%%%%%%%%%%%%%%%%%%%%%%%%%%%%%%%%%%%%%%%%%%%%%%%%%%%%%%%%%%%%%%%%%%%%
\section{Information}

%%%%%%%%%%%%%%%%%%%%%%%%%%%%%%%%%%%%%%%%%%%%%%%%%%%%%%%%%%%%%%%%%%%%%%%%%%%%%%%%
\subsection{Copyright}

Copyright \copyright{} 2017--2018 Niklas Beisert

This work may be distributed and/or modified under the
conditions of the \LaTeX{} Project Public License, either version 1.3
of this license or (at your option) any later version.
The latest version of this license is in
  \url{http://www.latex-project.org/lppl.txt}
and version 1.3 or later is part of all distributions of \LaTeX{}
version 2005/12/01 or later.

This work has the LPPL maintenance status `maintained'.

The Current Maintainer of this work is Niklas Beisert.

This work consists of the files |README.txt|, |childdoc.ins| and |childdoc.dtx|
as well as the derived files |childdoc.def|, |cdocsamp.tex|
with |cdocsch1.tex|, |cdocsch2.tex|, |cdocspt3.tex|, |cdocspt4.tex|,
|cdocsdrf.tex|, |cdocsfn1.tex|, |cdocsfn2.tex|
as well as |childdoc.pdf|.

%%%%%%%%%%%%%%%%%%%%%%%%%%%%%%%%%%%%%%%%%%%%%%%%%%%%%%%%%%%%%%%%%%%%%%%%%%%%%%%%
\subsection{Files and Installation}

The package consists of the files:
%
\begin{center}
\begin{tabular}{ll}
    |README.txt|   & readme file \\
    |childdoc.ins| & installation file \\
    |childdoc.dtx| & source file \\
    |childdoc.def| & definition file \\
    |cdocsamp.tex| & sample main file \\
    |cdocsch1.tex| & sample include file \\
    |cdocsch2.tex| & sample include file \\
    |cdocspt3.tex| & sample part file \\
    |cdocspt4.tex| & sample part file \\
    |cdocsdrf.tex| & sample redirection file \\
    |cdocsfn1.tex| & sample redirection file \\
    |cdocsfn2.tex| & sample redirection file \\
    |childdoc.pdf| & manual
\end{tabular}
\end{center}
%
The distribution consists of the files
|README.txt|, |childdoc.ins| and |childdoc.dtx|.
%
\begin{itemize}
\item
Run (pdf)\LaTeX{} on |childdoc.dtx|
to compile the manual |childdoc.pdf| (this file).
\item
Run \LaTeX{} on |childdoc.ins| to create the definitions file |childdoc.def|
and the sample |cdocsamp.tex| with include files
|cdocsch1.tex|, |cdocsch2.tex|, |cdocspt3.tex|, |cdocspt4.tex|,
|cdocsdrf.tex|, |cdocsfn1.tex|, |cdocsfn2.tex|.
Then copy the file |childdoc.def| to an appropriate directory of your \LaTeX{}
distribution, e.g.\ \textit{texmf-root}|/tex/latex/childdoc|.
\end{itemize}

%%%%%%%%%%%%%%%%%%%%%%%%%%%%%%%%%%%%%%%%%%%%%%%%%%%%%%%%%%%%%%%%%%%%%%%%%%%%%%%%
\subsection{Related CTAN Packages}

There are several other packages which offer a similar functionality:
%
\begin{itemize}
\item
The packages
\href{http://ctan.org/pkg/docmute}{\textsf{docmute}},
\href{http://ctan.org/pkg/includex}{\textsf{includex}} and
\href{http://ctan.org/pkg/standalone}{\textsf{standalone}}
provide commands to include only the document body of
a child file thus allowing both files to be compiled individually.
\item
The packages \href{http://ctan.org/pkg/subdocs}{\textsf{subdocs}}
and \href{http://ctan.org/pkg/subfiles}{\textsf{subfiles}}
provide structures in which the main and child documents can be
encapsulated and allowing them to be compiled individually.
The inclusion mechanism is different from the conventional |\include|.
\item
The package \href{http://ctan.org/pkg/combine}{\textsf{combine}}
is an elaborate solution to combine several documents into one.
\end{itemize}
%
See also the CTAN topic \href{http://ctan.org/topic/subdocs}{\textsf{subdocs}}
for further related packages.
The present package differs from the above solutions in that
a document structure constructed with the conventional |\include| mechanism
just needs two extra commands at the top of every file
such that all constituent files can be compiled individually.

%%%%%%%%%%%%%%%%%%%%%%%%%%%%%%%%%%%%%%%%%%%%%%%%%%%%%%%%%%%%%%%%%%%%%%%%%%%%%%%%
%\subsection{Feature Suggestions}
%
%The following is a list of features which may be useful for future
%versions of this package:
%%
%\begin{itemize}
%\item
%\ldots
%\end{itemize}

%%%%%%%%%%%%%%%%%%%%%%%%%%%%%%%%%%%%%%%%%%%%%%%%%%%%%%%%%%%%%%%%%%%%%%%%%%%%%%%%
\subsection{Revision History}

%%%%%%%%%%%%%%%%%%%%%%%%%%%%%%%%%%%%%%%%
\paragraph{v2.0:} 2018/12/30

\begin{itemize}
\item
immediate forward processing
\item
added |\childdocby| mechanism
\item
manual restructured
\end{itemize}

%%%%%%%%%%%%%%%%%%%%%%%%%%%%%%%%%%%%%%%%
\paragraph{v1.6:} 2018/01/17

\begin{itemize}
\item
application for development of include files
\item
corrections to manual
\end{itemize}

%%%%%%%%%%%%%%%%%%%%%%%%%%%%%%%%%%%%%%%%
\paragraph{v1.5:} 2017/05/21

\begin{itemize}
\item
more complete structuring introduced
\item
|\childdocof| introduced
\item
|\childdoc| renamed to |\childdocmain|
\item
|\childredirect| renamed to |\childdocforward| and |\childdocforwardprefix|
and functionality expanded
\end{itemize}

%%%%%%%%%%%%%%%%%%%%%%%%%%%%%%%%%%%%%%%%
\paragraph{v1.0:} 2017/04/27

\begin{itemize}
\item
manual and install package
\item
first version published on CTAN
\end{itemize}

%%%%%%%%%%%%%%%%%%%%%%%%%%%%%%%%%%%%%%%%
\paragraph{v0.6:} 2017/04/26

\begin{itemize}
\item
redirection mechanism added
\end{itemize}

%%%%%%%%%%%%%%%%%%%%%%%%%%%%%%%%%%%%%%%%
\paragraph{v0.5:} 2017/04/26

\begin{itemize}
\item
functionality in definition file
\end{itemize}


%%%%%%%%%%%%%%%%%%%%%%%%%%%%%%%%%%%%%%%%%%%%%%%%%%%%%%%%%%%%%%%%%%%%%%%%%%%%%%%%
%%%%%%%%%%%%%%%%%%%%%%%%%%%%%%%%%%%%%%%%%%%%%%%%%%%%%%%%%%%%%%%%%%%%%%%%%%%%%%%%
%%%%%%%%%%%%%%%%%%%%%%%%%%%%%%%%%%%%%%%%%%%%%%%%%%%%%%%%%%%%%%%%%%%%%%%%%%%%%%%%
\appendix

\settowidth\MacroIndent{\rmfamily\scriptsize 000\ }

 \DocInput{childdoc.dtx}

\end{document}
%</driver>
% \fi
%
% %%%%%%%%%%%%%%%%%%%%%%%%%%%%%%%%%%%%%%%%%%%%%%%%%%%%%%%%%%%%%%%%%%%%%%%%%%%%%%
% %%%%%%%%%%%%%%%%%%%%%%%%%%%%%%%%%%%%%%%%%%%%%%%%%%%%%%%%%%%%%%%%%%%%%%%%%%%%%%
% \section{Sample}
%\iffalse
%<*samplemain>
%\fi
%
% The following presents a sample document
% with two chapters, two parts, a title page,
% a compile flag as well as three forwarding files to set the flag.
% It consists of eight |.tex| files:
% \begin{center}
% \begin{tabular}{ll}
% |cdocsamp.tex|&main file\\
% |cdocsch1.tex|&include file for chapter 1\\
% |cdocsch2.tex|&include file for chapter 2\\
% |cdocspt3.tex|&include file for part 3\\
% |cdocspt4.tex|&include file for part 4\\
% |cdocsdrf.tex|&forwarding file for main file in draft mode\\
% |cdocsfi1.tex|&forwarding file for final version of chapter 1\\
% |cdocsfi2.tex|&forwarding file for final version of chapter 2\\
% \end{tabular}
% \end{center}
% Each of the eight files can be compiled directly by the \LaTeX{} compiler.
%
% %%%%%%%%%%%%%%%%%%%%%%%%%%%%%%%%%%%%%%
% \paragraph{Main File.}
%
% The main file is called |cdocsamp.tex|.
%
% Load the \textsf{childdoc} definitions and
% declare the filename for the main document:
%    \begin{macrocode}
\input{childdoc.def}
\childdocmain{}
%    \end{macrocode}

% Optional override for |\version| flag:
%    \begin{macrocode}
%%\ifchilddoc\else\providecommand{\version}{draft}\fi
%    \end{macrocode}

% Define the default values for the |\version| flag
% (|final| for the main file and |draft| for childs):
%    \begin{macrocode}
\ifchilddoc
\providecommand{\version}{draft}
\else
\providecommand{\version}{final}
\fi
%    \end{macrocode}

% Load the standard document class:
%    \begin{macrocode}
\documentclass[12pt]{article}
%    \end{macrocode}

% Start the document body:
%    \begin{macrocode}
\begin{document}
%    \end{macrocode}

% Declare a title page.
% Print title, part of document being processed and version flag:
%    \begin{macrocode}
\addtocounter{page}{-1}
\begin{center}
{\LARGE\bfseries{}childdoc example\par}
\vspace{1cm}
\ifchilddoc
\ifchilddocmanual part\else chapter\fi:
`\childdocname' of `\childdocjob'\par
\else
main document: `\childdocjob'\par
\fi
version: \version\par
\end{center}
\newpage
%    \end{macrocode}

% Manually include selected file,
% otherwise process as usual:
%    \begin{macrocode}
\ifchilddocmanual
\section*{part `\childdocname'}
\input{\childdocname}
\else
%    \end{macrocode}

% Include the two chapters:
%    \begin{macrocode}
\include{cdocsch1}
\include{cdocsch2}
%    \end{macrocode}

% Include the two parts unless only chapters should be displayed:
%    \begin{macrocode}
\ifchilddoc\else
\section{part three}
\input{cdocspt3}
\section{part four}
\input{cdocspt4}
\fi
%    \end{macrocode}

% Process as usual until here:
%    \begin{macrocode}
\fi
%    \end{macrocode}

% End of document body:
%    \begin{macrocode}
\end{document}
%    \end{macrocode}
%\iffalse
%</samplemain>
%\fi
%
% %%%%%%%%%%%%%%%%%%%%%%%%%%%%%%%%%%%%%%
% \paragraph{Chapter Include Files.}
%
% The include files are called |cdocsch1.tex| and |cdocsch2.tex|.
%
%\iffalse
%<*samplechap1|samplechap2>
%\fi

% Optional override for |\version| flag:
%    \begin{macrocode}
%%\providecommand{\version}{final}
%    \end{macrocode}

% Include the main document:
%    \begin{macrocode}
\input{childdoc.def}
\childdocof{cdocsamp}
%    \end{macrocode}

%\iffalse
%</samplechap1|samplechap2>
%\fi
%
%\iffalse
%<*samplechap1>
%\fi
% Some text for chapter 1:
%    \begin{macrocode}
\section{one}
some text in chapter one
%    \end{macrocode}

%\iffalse
%</samplechap1>
%\fi
% Some text for chapter 2:
%\iffalse
%<*samplechap2>
%\fi
%    \begin{macrocode}
\section{two}
more text in chapter two
%    \end{macrocode}

%\iffalse
%</samplechap2>
%\fi
%
% %%%%%%%%%%%%%%%%%%%%%%%%%%%%%%%%%%%%%%
% \paragraph{Part Include Files.}
%
% The include files are called |cdocspt3.tex| and |cdocspt4.tex|.
%
%\iffalse
%<*samplepart3|samplepart4>
%\fi

% Optional override for |\version| flag:
%    \begin{macrocode}
%%\providecommand{\version}{final}
%    \end{macrocode}

% Include the main document:
%    \begin{macrocode}
\input{childdoc.def}
\childdocby{cdocsamp}
%    \end{macrocode}

%\iffalse
%</samplepart3|samplepart4>
%\fi
%
%\iffalse
%<*samplepart3>
%\fi
% Some text for part 3:
%    \begin{macrocode}
some text in part three
%    \end{macrocode}

%\iffalse
%</samplepart3>
%\fi
% Some text for part 4:
%\iffalse
%<*samplepart4>
%\fi
%    \begin{macrocode}
more text in part four
%    \end{macrocode}

%\iffalse
%</samplepart4>
%\fi
%
% %%%%%%%%%%%%%%%%%%%%%%%%%%%%%%%%%%%%%%
% \paragraph{Forwarding for a Complete Draft.}
%
% The following forwarding file |cdocsdrf.tex|
% compiles the main document in draft mode:
%\iffalse
%<*sampledraft>
%\fi
%    \begin{macrocode}
\def\version{draft}
\input{childdoc.def}
\childdocforward{cdocsamp}
%    \end{macrocode}

%\iffalse
%</sampledraft>
%\fi
%
% %%%%%%%%%%%%%%%%%%%%%%%%%%%%%%%%%%%%%%
% \paragraph{Forwarding for Final Version of the Chapters.}
%
% The following forwarding files |cdocsfn1.tex| and |cdocsfn2.tex|
% (with identical content)
% compile the final versions of the child documents
% |cdocsch1.tex| and |cdocsch2.tex|, respectively:
%\iffalse
%<*samplefinal>
%\fi
%    \begin{macrocode}
\def\version{final}
\input{childdoc.def}
\childdocforwardprefix[cdocsamp]{cdocsfn}{cdocsch}
%    \end{macrocode}

%\iffalse
%</samplefinal>
%\fi
%
% %%%%%%%%%%%%%%%%%%%%%%%%%%%%%%%%%%%%%%
% \paragraph{Command Line Processing.}
%
% The following three command lines generate the output files
% |cdocscld|, |cdocscl1| and |cdocscl2|
% which should be identical to
% |cdocsdrf|, |cdocsch1| and |cdocsfn2|, respectively:
% \begin{center}
% \begin{tabular}{l}
% |latex -jobname cdocscld \|\\
% |  "\def\version{draft}\input{childdoc.def}\childdocforward{cdocsamp}"|\\
% |latex -jobname cdocscl1 \|\\
% |  "\input{childdoc.def}\childdocforward[cdocsamp]{cdocsch1}"|\\
% |latex -jobname cdocscl2 \|\\
% |  "\def\version{final}\input{childdoc.def}\childdocforward{cdocsch2}"|
% \end{tabular}
% \end{center}
% Note that the trailing backslash on each first line
% merely continues the input to the second line
% (for convenient cut ant paste).
% Furthermore, the command |latex| can be replaced by any
% of its alternative versions such as |pdflatex|.
%
% %%%%%%%%%%%%%%%%%%%%%%%%%%%%%%%%%%%%%%%%%%%%%%%%%%%%%%%%%%%%%%%%%%%%%%%%%%%%%%
% %%%%%%%%%%%%%%%%%%%%%%%%%%%%%%%%%%%%%%%%%%%%%%%%%%%%%%%%%%%%%%%%%%%%%%%%%%%%%%
% \section{Implementation}
%\iffalse
%<*package>
%\fi
%
% This section describes the definitions file |childdoc.def|.

% The definitions cannot be loaded using |\usepackage| or |\RequirePackage|
% which has a mechanism to prevent loading a style file more than once.
% When loading the definitions by means of |\input|
% multiple instances have to be prevented manually:
%\iffalse
%This code needs to be before the `\ProvidesFile' directive
%which is defined at the beginning of this file.
%Therefore it is also placed there and commented out here.
%</package>
%<*discard>
%\fi
%    \begin{macrocode}
\ifdefined\childdocmain\endinput\fi
%    \end{macrocode}
%\iffalse
%</discard>
%<*package>
%\fi
%
% \macro{\ifchilddoc}
% \macro{\ifchilddocmanual}
% The conditional |\ifchilddoc| tells whether a
% child (true) or main (false) document is being compiled.
% The conditional |\ifchilddocmanual| tells whether
% the |\includeonly| mechanism is used (false) or
% the selection of child files must be performed manually (true).
% The definitions initialise to false:
%    \begin{macrocode}
\newif\ifchilddoc
\newif\ifchilddocmanual
%    \end{macrocode}

% \macro{\childdocname}
% \macro{\childdocjob}
% The macro |\childdocname| stores the name of the main document
% to be compiled. The macro |\childdocjob| stores the name of
% the document on which the \LaTeX{} compiler was originally invoked.
% The content of |\jobname| cannot be compared
% to filenames specified in the source due to different catcodes.
% The following code rescans |\jobname|, stores the result
% in |\childdocname| and saves a copy in |\childdocjob|:
%    \begin{macrocode}
\edef\childdocname{\scantokens\expandafter{\jobname\noexpand}}
\let\childdocjob\childdocname
%    \end{macrocode}

% \macro{\childdocdisable}
% The macro |\childdocdisable| prevents the main file
% from being processed more than once.
% At this stage, the main document command |\childdocmain|
% is assumed to be called once again where it should do nothing.
% Any subsequent call to it should prevent
% a secondary processing of the main document
% It overwrites the forwarding commands
% |\childdocof| and |\childdocforward|
% with empty macros to prevent further inclusions of the main document:
%    \begin{macrocode}
\newcommand{\childdocdisable}
{
  \renewcommand{\childdocmain}[1]{\renewcommand{\childdocmain}[1]{\endinput}}
  \renewcommand{\childdocof}[1]{}
  \renewcommand{\childdocby}[2][]{}
  \renewcommand{\childdocforward}[2][]{}
  \renewcommand{\childdocdisable}{}
}
%    \end{macrocode}

% \macro{\childdocmain}
% The macro |\childdocmain| is to be called at the top of the main file
% with nothing or the main filename (without extension) as argument.
% First, it breaks loops.
% If the argument is not empty and does not match |\childdocname|
% (which is set by the first inclusion of |childdoc.def|),
% |\ifchilddoc| is set to true, |\includeonly| is applied to the child file
% and |\jobname| is set to the main file
% (for proper handling of |.aux| files):
%    \begin{macrocode}
\newcommand{\childdocmain}[1]
{
  \childdocdisable\childdocmain{}
  \if?#1?\else
    \begingroup
      \def\childdoctmp{#1}
      \ifx\childdoctmp\childdocname
        \def\childdoctmp{}
      \else
        \def\childdoctmp
        {
          \childdoctrue
          \includeonly{\childdocname}
          \def\childdocjob{#1}
          \def\jobname{#1}
        }
      \fi
      \expandafter
    \endgroup
    \childdoctmp
  \fi
}
%    \end{macrocode}

% \macro{\childdocof}
% The command |\childdocof| redirects
% compilation to the main file |#1|.
%    \begin{macrocode}
\newcommand{\childdocof}[1]
{
  \childdocdisable
  \childdoctrue
  \includeonly{\childdocname}
  \def\jobname{#1}
  \def\childdocjob{#1}
  \input{#1}
}
%    \end{macrocode}

% \macro{\childdocby}
% The command |\childdocby| ....
%    \begin{macrocode}
\newcommand{\childdocby}[2][]
{
  \childdocdisable
  \childdoctrue
  \childdocmanualtrue
  \if?#1?\else
    \def\jobname{#2}
  \fi
  \def\childdocjob{#2}
  \input{#2}
  \endinput
}
%    \end{macrocode}

% \macro{\childdocforward}
% The command |\childdocforward| redirects
% compilation to the main file or
% (if the optional argument is given) a child file.
% Parameters are set as if the main file
% or a child file starting with |\childdocof| was compiled.
% Then compilation is handed over to the main file:
%    \begin{macrocode}
\newcommand{\childdocforward}[2][]
{
  \begingroup
    \if?#1?
      \def\childdoctmp
      {
        \def\childdocname{#2}
        \def\childdocjob{#2}
        \def\jobname{#2}
        \input{#2}
        \endinput
      }
    \else
      \def\childdoctmp
      {
        \childdocdisable
        \def\childdocname{#2}
        \childdoctrue
        \includeonly{#2}
        \def\childdocjob{#1}
        \def\jobname{#1}
        \input{#1}
        \endinput
      }
    \fi
    \expandafter
  \endgroup
  \childdoctmp
}
%    \end{macrocode}

% \macro{\childdocforwardprefix}
% The command |\childdocforwardprefix| redirects
% compilation to the main or a child file by means of a pattern.
% The prefix |#1| in the current filename is replaced by |#2|
% and the suffix of the current filename is kept
% (it is assumed that the filename does not contain the substring `|~~~|'
% which is used as a delimiter).
% Compilation is handed over to the new file by |\childdocforward|:
%    \begin{macrocode}
\newcommand{\childdocforwardprefix}[3][]
{
  \begingroup
    \def\childdocextract #2##1~~~{\def\childdoctmp{\childdocforward[#1]{#3##1}}}
    \expandafter\childdocextract\childdocname~~~
    \expandafter
  \endgroup
  \childdoctmp
}
%    \end{macrocode}

% \macro{\childdoc}
% The deprecated macro |\childdoc| is a legacy version of |\childdocmain|:
%    \begin{macrocode}
\newcommand{\childdoc}{\childdocmain}
%    \end{macrocode}

% \macro{\childdocredirect}
% The deprecated macro |\childdocredirect| is a legacy version
% of |\childdocforward| and |\childdocforwardprefix|:
%    \begin{macrocode}
\newcommand{\childdocredirect}[2][]
{
  \begingroup
    \if?#1?
      \def\childdoctmp{\childdocforward{#2}}
    \else
      \def\childdoctmp{\childdocforwardprefix{#1}{#2}}
    \fi
    \expandafter
  \endgroup
  \childdoctmp
}
%    \end{macrocode}

%\iffalse
%</package>
%\fi
%
\endinput
\childdocforward{cdocsamp}"|\\
% |latex -jobname cdocscl1 \|\\
% |  "% \iffalse
%
% childdoc.dtx Copyright (C) 2017-2018 Niklas Beisert
%
% This work may be distributed and/or modified under the
% conditions of the LaTeX Project Public License, either version 1.3
% of this license or (at your option) any later version.
% The latest version of this license is in
%   http://www.latex-project.org/lppl.txt
% and version 1.3 or later is part of all distributions of LaTeX
% version 2005/12/01 or later.
%
% This work has the LPPL maintenance status `maintained'.
%
% The Current Maintainer of this work is Niklas Beisert.
%
% This work consists of the files childdoc.dtx and childdoc.ins
% and the derived files childdoc.def and cdocsamp.tex with
% cdocsch1.tex, cdocsch2.tex, cdocsdrf.tex, cdocsfn1.tex, cdocsfn2.tex.
%
%<package>\ifdefined\childdocmain\endinput\fi
%<package>\ProvidesFile{childdoc.def}[2018/12/30 v2.0 child document driver]
%<samplemain>\ProvidesFile{cdocsamp.tex}[2018/12/30 v2.0 sample for childdoc]
%<*driver>
%\ProvidesFile{childdoc.drv}[2018/12/30 v2.0 childdoc reference manual file]
\PassOptionsToClass{10pt,a4paper}{article}
\documentclass{ltxdoc}

\usepackage[margin=35mm]{geometry}
\usepackage{hyperref}
\usepackage{hyperxmp}
\usepackage[usenames]{color}

\hypersetup{colorlinks=true}
\hypersetup{pdfstartview=FitH}
\hypersetup{pdfpagemode=UseNone}
\hypersetup{pdfsource={}}
\hypersetup{pdflang={en-UK}}
\hypersetup{pdfcopyright={Copyright 2017-2018 Niklas Beisert.
  This work may be distributed and/or modified under the
  conditions of the LaTeX Project Public License, either version 1.3
  of this license or (at your option) any later version.}}
\hypersetup{pdflicenseurl={http://www.latex-project.org/lppl.txt}}
\hypersetup{pdfcontactaddress={ETH Zurich, ITP, HIT K,
  Wolfgang-Pauli-Strasse 27}}
\hypersetup{pdfcontactpostcode={8093}}
\hypersetup{pdfcontactcity={Zurich}}
\hypersetup{pdfcontactcountry={Switzerland}}
\hypersetup{pdfcontactemail={nbeisert@itp.phys.ethz.ch}}
\hypersetup{pdfcontacturl={http://people.phys.ethz.ch/\xmptilde nbeisert/}}

\newcommand{\secref}[1]{\hyperref[#1]{section \ref*{#1}}}

\parskip1ex
\parindent0pt
\let\olditemize\itemize
\def\itemize{\olditemize\parskip0pt}

\begin{document}

\title{The \textsf{childdoc} Package}
\hypersetup{pdftitle={The childdoc Package}}
\author{Niklas Beisert\\[2ex]
  Institut f\"ur Theoretische Physik\\
  Eidgen\"ossische Technische Hochschule Z\"urich\\
  Wolfgang-Pauli-Strasse 27, 8093 Z\"urich, Switzerland\\[1ex]
  \href{mailto:nbeisert@itp.phys.ethz.ch}
  {\texttt{nbeisert@itp.phys.ethz.ch}}}
\hypersetup{pdfauthor={Niklas Beisert}}
\hypersetup{pdfsubject={Manual for the LaTeX2e Package childdoc}}
\date{30 December 2018, \textsf{v2.0}}
\maketitle

\begin{abstract}\noindent
\textsf{childdoc} is a \LaTeXe{} package
that enables the direct compilation
of document sections included by |\include|
to individual files.
\end{abstract}

\begingroup
\parskip0ex
\tableofcontents
\endgroup

%%%%%%%%%%%%%%%%%%%%%%%%%%%%%%%%%%%%%%%%%%%%%%%%%%%%%%%%%%%%%%%%%%%%%%%%%%%%%%%%
%%%%%%%%%%%%%%%%%%%%%%%%%%%%%%%%%%%%%%%%%%%%%%%%%%%%%%%%%%%%%%%%%%%%%%%%%%%%%%%%
\section{Introduction}

\LaTeX{} provides a mechanism to structure a large document (such as a book)
into a main file and several child files (containing the chapters)
using the |\include| command.
This mechanism is beneficial for documents
which span hundreds of pages in order to
make the source file(s) more manageable.
Moreover, compilation can be restricted to
selected child files by means of the |\includeonly| command.
The latter feature can be used to reduce the compilation time while editing
(this was significantly more useful in the earlier days of \LaTeX{})
or to generate a smaller document which is easier to navigate.
Another application of |\includeonly| is to generate
documents consisting of selected parts of the complete document.

However, there are a few drawbacks of the plain |\include| mechanism:
\begin{itemize}
\item
The child files cannot be compiled on their own,
they can only be compiled via the main file.
A naive editing environment
(such as a text editor with an option
to have the current file processed by \LaTeX)
may require one to switch to the main file before compiling;
attempting to compile the child file produces errors.
\item
The main file must be modified (each time)
to adjust the |\includeonly| command
to the present needs. This easily leaves the main file in a messy state.
\item
The generated document will always carry the filename
of the main document. This is inconvenient if
several child files are to be compiled and
to be kept for distribution.
\end{itemize}

The present package provides a simple interface
to make child files individually compilable by \LaTeX{}.
Compiling a child file then has the same effect as compiling
the main file with an |\includeonly| command
to select the appropriate child.
Moreover the generated document will carry the name of the child
rather than the main file.
This resolves all three above issues.

This feature is meant to make the editing of books,
thesis documents and lecture notes somewhat more convenient.
However, the package can also be used efficiently for
composing a series of documents (such as exercise sheets)
which are typically distributed individually.
It then assists the author in generating the individual documents
(potentially in different versions)
as well as a document containing the collected series.
Another application is in developing style files
or other kinds of included material
where compilation of the style file could redirect
to a sample or test file.

%%%%%%%%%%%%%%%%%%%%%%%%%%%%%%%%%%%%%%%%%%%%%%%%%%%%%%%%%%%%%%%%%%%%%%%%%%%%%%%%
%%%%%%%%%%%%%%%%%%%%%%%%%%%%%%%%%%%%%%%%%%%%%%%%%%%%%%%%%%%%%%%%%%%%%%%%%%%%%%%%
\section{Usage}

First of all, the package \textsf{childdoc} is \emph{not} a standard
\LaTeXe{} |.sty| style file! Therefore it needs to be invoked in
a non-standard way.

%%%%%%%%%%%%%%%%%%%%%%%%%%%%%%%%%%%%%%%%%%%%%%%%%%%%%%%%%%%%%%%%%%%%%%%%%%%%%%%%
\subsection{Included Files}
\label{sec:include}

%%%%%%%%%%%%%%%%%%%%%%%%%%%%%%%%%%%%%%%%
\DescribeMacro{\childdocmain}
To use the package, add the commands
\begin{center}
\begin{tabular}{l}
|\input{childdoc.def}|\\
|\childdocmain{}|\\
\end{tabular}
\end{center}
at the very top of the main \LaTeX{} file,
in particular \emph{before} the |\documentclass| statement!
The argument of |\childdocmain| should be left empty
(but it must be present).

%%%%%%%%%%%%%%%%%%%%%%%%%%%%%%%%%%%%%%%%
\DescribeMacro{\childdocof}
Furthermore, add the commands
\begin{center}
\begin{tabular}{l}
|\input{childdoc.def}|\\
|\childdocof{|\textit{main}|}|\\
\end{tabular}
\end{center}
at the top of every child file \textit{child}
which is included by |\include{|\textit{child}|}|
from within the main file
(or at least for those files to be compiled individually).
The argument \textit{main} must be the filename of the main file.

There are a couple of
considerations in setting up the main and child documents:

%%%%%%%%%%%%%%%%%%%%%%%%%%%%%%%%%%%%%%%%
\paragraph{Restrictions.}

Please note the following restrictions:
\begin{itemize}
\item
|\childdocmain| must be called with one argument \textit{main}
to ensure compatibility with earlier version of the package.
It must either be empty (|\childdocmain{}|)
or precisely match the filename of the main file in which it is specified.
See \secref{sec:detection} for further information.
\item
The filename \textit{main} must be specified without the |.tex| extension.
\item
The filename \textit{main} is case sensitive
(even in case-insensitive file systems)
due to internal string comparison.
\item
The argument \textit{main} should be fully expanded, it cannot be a macro.
\item
Subdirectories and special characters should be avoided in filenames.
\item
The command |\childdocmain{|\textit{main}|}| must be followed by a whitespace.
It should not be followed immediately by another command
or by a comment mark `|%|'.
This is because the \TeX{} parser reads the token immediately following
the argument of |\childdocmain| and puts it
at the beginning of every child section;
however, a white\-space is ignored.
\end{itemize}

%%%%%%%%%%%%%%%%%%%%%%%%%%%%%%%%%%%%%%%%
\paragraph{Content of Main File.}

It is advisable to place all content in the child files included by |\include|.
Any output contained in the main file will appear in all child documents
unless suppressed manually;
it cannot be suppressed automatically by the |\includeonly| directive
and thus should normally be avoided.
A method to include some content in the main file
by means of conditional processing is described in \secref{sec:conditional}.

%%%%%%%%%%%%%%%%%%%%%%%%%%%%%%%%%%%%%%%%
\paragraph{Page Numbering.}

When only a part of the document is compiled,
the appropriate numbering of pages
(as well as other status parameters)
is determined from the |.aux| files.
The latter contain information from previous passes.
However this information needs to propagate through
all intermediate child documents.
Therefore the page numbering in child documents may well
be inconsistent until the complete document is compiled at least once.

A useful (if unconventional) way to always ensure a consistent
page numbering is to restart the numbering in each child document
and denote the pages by `\textit{child}|.|\textit{page}'
where \textit{child} represents the chapter/section number of the child file.
This can be achieved by the command
|\numberwithin{page}{|\textit{child}|}|
of the \textsf{amsmath} package
where \textit{child} can be |chapter| or |section|
depending on the chosen structuring.
Alternatively, one can modify the macro |\thepage| appropriately
and reset the counter |page| at the start of each child file.

%%%%%%%%%%%%%%%%%%%%%%%%%%%%%%%%%%%%%%%%%%%%%%%%%%%%%%%%%%%%%%%%%%%%%%%%%%%%%%%%
\subsection{Conditional Processing}
\label{sec:conditional}

The package provides a mechanism to compile different versions
of a document. To customise the versions further some conditional processing
can come in handy to distinguish which version is being compiled.
The package provides two macros to describe the compilation context:

%%%%%%%%%%%%%%%%%%%%%%%%%%%%%%%%%%%%%%%%
\DescribeMacro{\ifchilddoc}
The conditional |\ifchilddoc| distinguishes between the compilation of
child documents and the main document:
%
\begin{center}
|\ifchilddoc |\textit{child-code}| |[|\||else |\textit{main-code}]| \||fi|
\end{center}

%%%%%%%%%%%%%%%%%%%%%%%%%%%%%%%%%%%%%%%%
\DescribeMacro{\childdocname}
\DescribeMacro{\childdocjob}
The macro |\childdocname| contains the filename (without extension)
of the main or child file being processed.
Note that |\childdocjob| will always contain the name of the main file.

%%%%%%%%%%%%%%%%%%%%%%%%%%%%%%%%%%%%%%%%
\paragraph{Title Page.}

Conditional processing can be used to include a title or banner page
in the main document when proper precautions are taken.
Importantly, the code in the main file should ensure that the page counter
(as well as other status parameters which are stored in the |.aux| files)
takes the same value after the conditional processing.
Otherwise the page numbers may take divergent values
depending on which part is compiled.

For example, a title page could be declared by:
%
\begin{center}
\begin{tabular}{l}
|\ifchilddoc\||else|\\
|\addtocounter{page}{-1}|\\
\textit{code for title page}\\
|\newpage|\\
|\||fi|
\end{tabular}
\end{center}
%
A banner page for the child documents can be generated by:
%
\begin{center}
\begin{tabular}{l}
|\ifchilddoc|\\
|\addtocounter{page}{-1}|\\
\textit{code for banner page}\\
|\newpage|\\
|\||fi|
\end{tabular}
\end{center}
%
Here one could write a message such as:
\begin{center}
|This is the part \childdocname{} of \childdocjob{}.|
\end{center}

%%%%%%%%%%%%%%%%%%%%%%%%%%%%%%%%%%%%%%%%%%%%%%%%%%%%%%%%%%%%%%%%%%%%%%%%%%%%%%%%
\subsection{Flags}
\label{sec:flags}

The package makes it easy to generate different versions
of the main or child documents.
To this end compilation flags can be defined
and assigned different default values.
They will be particularly useful in conjunction
with the forwarding mechanism described in \secref{sec:forward}.

For example, it may be useful to have a flag |\version|
which can be set to |draft| or |final|.
The document source will contain some conditional code
depending on the value of |\version|.
Suppose further, the flag should default to |final| for the main file
and to |draft| for child files
which is a natural assignment for editing the document.
This is achieved by placing the following code
in the preamble of the main document
(below the |\childdocmain| directive):
%
\begin{center}
\begin{tabular}{l}
|\ifchilddoc|\\
|\providecommand{\version}{draft}|\\
|\||else|\\
|\providecommand{\version}{final}|\\
|\||fi|
\end{tabular}
\end{center}
%
The definition by |\providecommand| makes sure
that previous definitions are not overwritten.
Further statements |\providecommand{\version}{...}|
can thus be added before the above code to override it.

For the main file, one might add a line
(between |\childdocmain| and the above block)
%
\begin{center}
|%\ifchilddoc\||else\providecommand{\version}{draft}\||fi|
\end{center}
%
which can be uncommented to produce a draft version.
Likewise one can add a line to the very top of a child file
(above the |\childdocof{|\textit{main}|}| directive)
%
\begin{center}
|%\providecommand{\version}{final}|
\end{center}
%
which can be uncommented to produce the final version of this child document.

%%%%%%%%%%%%%%%%%%%%%%%%%%%%%%%%%%%%%%%%%%%%%%%%%%%%%%%%%%%%%%%%%%%%%%%%%%%%%%%%
\subsection{Forwarding}
\label{sec:forward}

Different versions of the main or child documents
using compilation flags as described in \secref{sec:flags}
can be (permanently) stored in different files
for convenient compilation, viewing and distribution.
To this end, the package defines a command
to pass on compilation to a different file:

%%%%%%%%%%%%%%%%%%%%%%%%%%%%%%%%%%%%%%%%
\DescribeMacro{\childdocforward}
The command |\childdocforward| redirects processing to
another source file:
%
\begin{center}
\begin{tabular}{l}
|\input{childdoc.def}|\\
|\childdocforward[|\textit{main}|]{|\textit{dest}|}|\\
\end{tabular}
\end{center}
%
The argument \textit{dest} is the destination file
(without extension).
It should be the main file or one of the child files.
Note that further \textsf{childdoc} directives
such as |\childdocof| and |\childdocforward|
in the indicated file will be processed in this form.
The optional argument \textit{main}
passes on directly to the main file \textit{main}
while pretending to compile the child \textit{dest}.
This form behaves as if \textit{dest}
issues |\childdocof{|\textit{main}|}| right away,
and no further \textsf{childdoc} directives will be processed.

%%%%%%%%%%%%%%%%%%%%%%%%%%%%%%%%%%%%%%%%
\DescribeMacro{\...prefix}
In the alternative form |\childdocforwardprefix|,
%
\begin{center}
\begin{tabular}{l}
|\input{childdoc.def}|\\
|\childdocforwardprefix[|\textit{main}|]{|\textit{prefix}|}{|\textit{dest}|}|
\end{tabular}
\end{center}
%
the destination file is determined by a pattern
depending on the current file:
To make this work, the current file must be called
`{\textit{prefix}\hspace{0.2em}\textit{suffix}}'
with \textit{prefix} matching precisely the argument.
Processing is then passed on to the file
`{\textit{dest}\hspace{0.2em}\textit{suffix}}'.
Surely, the same effect is achieved by
directly specifying the
argument `{\textit{dest}\hspace{0.2em}\textit{suffix}}'
in the first form.
However, that requires to set up a different file
for each child. With the alternative form of the command
all these files can have exactly the same content
which simplifies setting them up and maintaining them.

For example, the following file |draft.tex|
with a compilation flag |\version| as described in \secref{sec:flags}
compiles the main document as a draft:
%
\begin{center}
\begin{tabular}{l}
|\def\version{draft}|\\
|\input{childdoc.def}|\\
|\childdocforward{|\textit{main}|}|
\end{tabular}
\end{center}
%
Likewise, the following files |final|\textit{nn}|.tex|
compile the final version of the child document
|child|\textit{nn}|.tex|:
%
\begin{center}
\begin{tabular}{l}
|\def\version{final}|\\
|\input{childdoc.def}|\\
|\childdocforwardprefix{final}{child}|
\end{tabular}
\end{center}
%

Note that when several versions of a main file and/or of each child file
are to be generated, it may be convenient to set up a |Makefile| or
shell script to automatise the process.

%%%%%%%%%%%%%%%%%%%%%%%%%%%%%%%%%%%%%%%%%%%%%%%%%%%%%%%%%%%%%%%%%%%%%%%%%%%%%%%%
\subsection{Command Line Processing}
\label{sec:commandline}

The effect of redirection files can also be achieved by invoking
the \LaTeX{} compiler with a more elaborate command line.
Most conveniently this should be done as part
of a shell script or a |Makefile|.

When using \textsf{childdoc} in the main file, the following
command lines effectively perform a redirection
(note that depending on the shell being used,
backslashes may have to be doubled: `|\|' $\to$ `|\\|'):
%
\begin{center}
|... -jobname "|\textit{target}|" |\\|"|[\textit{flags}]%
|\input{childdoc.def}\childdocforward[|\textit{main}|]{|\textit{dest}|}"|
\end{center}
%
Here \textit{target} is the name of the output file,
\textit{main} is the name of the main file
and \textit{dest} is the name of the main or child file to be processed
(all filenames without extensions).
The optional argument \textit{main} can be omitted
if \textit{main} matches \textit{dest}.
Optionally, compilation \textit{flags} can be defined via |\def| commands.
This command line makes the \TeX{} engine believe
it is compiling the file \textit{target}
whose content is specified as the latter parameter.
The provided code then forwards the processing to
\textit{main} or \textit{dest} as described in \secref{sec:forward}.

%%%%%%%%%%%%%%%%%%%%%%%%%%%%%%%%%%%%%%%%%%%%%%%%%%%%%%%%%%%%%%%%%%%%%%%%%%%%%%%%
\subsection{Include by Input}
\label{sec:input}

Including child documents by |\include| has some restrictions by design.
Most notably, the content of a child document always occupies
its own set of pages; pages cannot be shared between child documents.
Usually, this behaviour makes perfect sense
because each child document contain an essential part of the document.
However, in some situations it may be desirable to compose
a document from a collection of parts
without having mandatory page breaks between then.
For this case, the package
provides a mechanism to include parts
by |\input| which can also be processed individually.
However, by construction this mechanism
requires manual handling of the content to be output.

%%%%%%%%%%%%%%%%%%%%%%%%%%%%%%%%%%%%%%%%
\DescribeMacro{\ifchilddocmanual}
The main file should be prepared as usual, see \secref{sec:include}.
However, the document body must make a distinction
between processing of an individual part and of the main document, e.g.:
%
\begin{center}
\begin{tabular}{l}
|\ifchilddocmanual|\\
|\input{\childdocname}|\\
|\||else|\\
\textit{document body with }|\input{|\textit{part}|}|\\
|\||fi|
\end{tabular}
\end{center}
%
The conditional |\ifchilddocmanual| is true whenever
a part to be included by |\input| is being compiled,
and the name of the part is stored in |\childdocname|.

%%%%%%%%%%%%%%%%%%%%%%%%%%%%%%%%%%%%%%%%
\DescribeMacro{\childdocby}
Each part to be included by |\input| should start with:
%
\begin{center}
\begin{tabular}{l}
|\input{childdoc.def}|\\
|\childdocby{|\textit{main}|}|\\
\end{tabular}
\end{center}
%
The directive |\childdocby| is similar to |\childdocof|
described in \secref{sec:include},
but the subsequent selection of content must be done manually.
To that end, both |\ifchilddoc| and |\ifchilddocmanual|
will be true upon processing of a part,
and the name of the part is stored in |\childdocname|.
Note that |\jobname| will be set to the filename of the current part
so that each part receives an individual |.aux| file
that does not interfere with the |.aux| file(s) of the main document.
This behaviour can be altered by the alternative form
|\childdocby[*]{|\textit{main}|}| (with a non-empty optional argument)
which uses the |.aux| file of the main document
by setting |\jobname| to \textit{main}.

%%%%%%%%%%%%%%%%%%%%%%%%%%%%%%%%%%%%%%%%%%%%%%%%%%%%%%%%%%%%%%%%%%%%%%%%%%%%%%%%
\subsection{Driver Development}
\label{sec:driver}

The \textsf{childdoc} mechanism can also be use for the development
of definition files such as \LaTeX{} styles or classes.
This case differs from the above setup with multiple parts
included by |\include| in that no |\includeonly| should be invoked.
This can be achieved by starting the include file
(before |\ProvidesPackage|) with:
%
\begin{center}
\begin{tabular}{l}
|\input{childdoc.def}|\\
|\childdocforward{|\textit{main}|}|\\
\end{tabular}
\end{center}
%
or alternatively with:
%
\begin{center}
\begin{tabular}{l}
|\input{childdoc.def}|\\
|\childdocby{|\textit{main}|}|\\
\end{tabular}
\end{center}
%
Both forms have slightly different effects as described above.
The main file is prepared as usual, see \secref{sec:include}.

%%%%%%%%%%%%%%%%%%%%%%%%%%%%%%%%%%%%%%%%%%%%%%%%%%%%%%%%%%%%%%%%%%%%%%%%%%%%%%%%
\subsection{Legacy Detection}
\label{sec:detection}

The directive |\childdocmain| in the main file can detect
whether the complete document or merely a child is to be compiled
even without using the directive |\childdocof|.
This method is deprecated because it is less robust
and there is no compelling reason to use it;
it is merely provided for backward compatibility
and it may be removed in future versions.

If the detection mechanism is to be used,
it is mandatory to correctly specify
the filename of the main file as the argument of |\childdocmain|:
%
\begin{center}
\begin{tabular}{l}
|\input{childdoc.def}|\\
|\childdocmain{|\textit{main}|}|\\
\end{tabular}
\end{center}
%
If |\jobname| does not match the argument \textit{main} of |\childdocmain|,
it is assumed that |\jobname| points to the child file to be compiled.
When using |\childdocmain| with the main file specified as argument,
it suffices to start a child file
with just |\input{|\textit{main}|}|
without loading of the package and using |\childdocof|.
If instead all processing is done
with the appropriate \textsf{childdoc} directives,
the argument of \textit{main} of |\childdocmain| can be empty.

An alternative version of the command line processing described
in \secref{sec:commandline} using the detection mechanism reads:
%
\begin{center}
|... -jobname "|\textit{target}|" "|[\textit{flags}]%
[|\def\jobname{|\textit{dest}|}|]|\input{|\textit{main}|}"|
\end{center}

%%%%%%%%%%%%%%%%%%%%%%%%%%%%%%%%%%%%%%%%%%%%%%%%%%%%%%%%%%%%%%%%%%%%%%%%%%%%%%%%
\subsection{Manual Code}
\label{sec:manual}

In case one cannot be certain whether the definitions file |childdoc.def|
is installed on the target \TeX{} distribution
and one prefers not to ship it,
it is conceivable to paste a few relevant commands into the sources.

To that end, drop all statements |\input{childdoc.def}|
and perform the replacements as outlined below.
Instead of |\childdocmain{|\textit{main}|}| add the following code
to the top of the main file:
%
\begin{center}
\begin{tabular}{l}
|\||ifdefined\childdocname\endinput\||fi\newif\ifchilddoc|\\
|\edef\childdocname{\scantokens\expandafter{\jobname\noexpand}}|\\
|\def\childdocmain{|\textit{main}|}\||ifx\childdocmain\childdocname\||else|\\
|\childdoctrue\includeonly{\childdocname}\let\jobname\childdocmain\||fi|\\
\end{tabular}
\end{center}
%
Instead of |\childdocof{|\textit{main}|}| just include the main file
at the top of each child file:
%
\begin{center}
|\input{|\textit{main}|}|
\end{center}
%
A simple redirection |\childdocforward{|\textit{dest}|}| is achieved by:
%
\begin{center}
|\def\jobname{|\textit{dest}|}\input{\jobname}|
\end{center}
%
The redirection with prefix
|\childdocforwardprefix[|\textit{prefix}|]{|\textit{dest}|}|
is accomplished by:
%
\begin{center}
\begin{tabular}{l}
|{\edef\jobname{\scantokens\expandafter{\jobname\noexpand}}|\\
|\def\redirectjob |\textit{prefix}|#1~~~{\gdef\jobname{|\textit{dest}|#1}}|\\
|\expandafter\redirectjob\jobname~~~}\input{\jobname}|
\end{tabular}
\end{center}

In an alternative approach,
child documents can be compiled by a specific command line
without additional code or specific definitions:
%
\begin{center}
|... -jobname "|\textit{target}|" "|[\textit{flags}]%
|\includeonly{|\textit{dest}|}\input{|\textit{main}|}"|
\end{center}
%

%%%%%%%%%%%%%%%%%%%%%%%%%%%%%%%%%%%%%%%%%%%%%%%%%%%%%%%%%%%%%%%%%%%%%%%%%%%%%%%%
%%%%%%%%%%%%%%%%%%%%%%%%%%%%%%%%%%%%%%%%%%%%%%%%%%%%%%%%%%%%%%%%%%%%%%%%%%%%%%%%
\section{Information}

%%%%%%%%%%%%%%%%%%%%%%%%%%%%%%%%%%%%%%%%%%%%%%%%%%%%%%%%%%%%%%%%%%%%%%%%%%%%%%%%
\subsection{Copyright}

Copyright \copyright{} 2017--2018 Niklas Beisert

This work may be distributed and/or modified under the
conditions of the \LaTeX{} Project Public License, either version 1.3
of this license or (at your option) any later version.
The latest version of this license is in
  \url{http://www.latex-project.org/lppl.txt}
and version 1.3 or later is part of all distributions of \LaTeX{}
version 2005/12/01 or later.

This work has the LPPL maintenance status `maintained'.

The Current Maintainer of this work is Niklas Beisert.

This work consists of the files |README.txt|, |childdoc.ins| and |childdoc.dtx|
as well as the derived files |childdoc.def|, |cdocsamp.tex|
with |cdocsch1.tex|, |cdocsch2.tex|, |cdocspt3.tex|, |cdocspt4.tex|,
|cdocsdrf.tex|, |cdocsfn1.tex|, |cdocsfn2.tex|
as well as |childdoc.pdf|.

%%%%%%%%%%%%%%%%%%%%%%%%%%%%%%%%%%%%%%%%%%%%%%%%%%%%%%%%%%%%%%%%%%%%%%%%%%%%%%%%
\subsection{Files and Installation}

The package consists of the files:
%
\begin{center}
\begin{tabular}{ll}
    |README.txt|   & readme file \\
    |childdoc.ins| & installation file \\
    |childdoc.dtx| & source file \\
    |childdoc.def| & definition file \\
    |cdocsamp.tex| & sample main file \\
    |cdocsch1.tex| & sample include file \\
    |cdocsch2.tex| & sample include file \\
    |cdocspt3.tex| & sample part file \\
    |cdocspt4.tex| & sample part file \\
    |cdocsdrf.tex| & sample redirection file \\
    |cdocsfn1.tex| & sample redirection file \\
    |cdocsfn2.tex| & sample redirection file \\
    |childdoc.pdf| & manual
\end{tabular}
\end{center}
%
The distribution consists of the files
|README.txt|, |childdoc.ins| and |childdoc.dtx|.
%
\begin{itemize}
\item
Run (pdf)\LaTeX{} on |childdoc.dtx|
to compile the manual |childdoc.pdf| (this file).
\item
Run \LaTeX{} on |childdoc.ins| to create the definitions file |childdoc.def|
and the sample |cdocsamp.tex| with include files
|cdocsch1.tex|, |cdocsch2.tex|, |cdocspt3.tex|, |cdocspt4.tex|,
|cdocsdrf.tex|, |cdocsfn1.tex|, |cdocsfn2.tex|.
Then copy the file |childdoc.def| to an appropriate directory of your \LaTeX{}
distribution, e.g.\ \textit{texmf-root}|/tex/latex/childdoc|.
\end{itemize}

%%%%%%%%%%%%%%%%%%%%%%%%%%%%%%%%%%%%%%%%%%%%%%%%%%%%%%%%%%%%%%%%%%%%%%%%%%%%%%%%
\subsection{Related CTAN Packages}

There are several other packages which offer a similar functionality:
%
\begin{itemize}
\item
The packages
\href{http://ctan.org/pkg/docmute}{\textsf{docmute}},
\href{http://ctan.org/pkg/includex}{\textsf{includex}} and
\href{http://ctan.org/pkg/standalone}{\textsf{standalone}}
provide commands to include only the document body of
a child file thus allowing both files to be compiled individually.
\item
The packages \href{http://ctan.org/pkg/subdocs}{\textsf{subdocs}}
and \href{http://ctan.org/pkg/subfiles}{\textsf{subfiles}}
provide structures in which the main and child documents can be
encapsulated and allowing them to be compiled individually.
The inclusion mechanism is different from the conventional |\include|.
\item
The package \href{http://ctan.org/pkg/combine}{\textsf{combine}}
is an elaborate solution to combine several documents into one.
\end{itemize}
%
See also the CTAN topic \href{http://ctan.org/topic/subdocs}{\textsf{subdocs}}
for further related packages.
The present package differs from the above solutions in that
a document structure constructed with the conventional |\include| mechanism
just needs two extra commands at the top of every file
such that all constituent files can be compiled individually.

%%%%%%%%%%%%%%%%%%%%%%%%%%%%%%%%%%%%%%%%%%%%%%%%%%%%%%%%%%%%%%%%%%%%%%%%%%%%%%%%
%\subsection{Feature Suggestions}
%
%The following is a list of features which may be useful for future
%versions of this package:
%%
%\begin{itemize}
%\item
%\ldots
%\end{itemize}

%%%%%%%%%%%%%%%%%%%%%%%%%%%%%%%%%%%%%%%%%%%%%%%%%%%%%%%%%%%%%%%%%%%%%%%%%%%%%%%%
\subsection{Revision History}

%%%%%%%%%%%%%%%%%%%%%%%%%%%%%%%%%%%%%%%%
\paragraph{v2.0:} 2018/12/30

\begin{itemize}
\item
immediate forward processing
\item
added |\childdocby| mechanism
\item
manual restructured
\end{itemize}

%%%%%%%%%%%%%%%%%%%%%%%%%%%%%%%%%%%%%%%%
\paragraph{v1.6:} 2018/01/17

\begin{itemize}
\item
application for development of include files
\item
corrections to manual
\end{itemize}

%%%%%%%%%%%%%%%%%%%%%%%%%%%%%%%%%%%%%%%%
\paragraph{v1.5:} 2017/05/21

\begin{itemize}
\item
more complete structuring introduced
\item
|\childdocof| introduced
\item
|\childdoc| renamed to |\childdocmain|
\item
|\childredirect| renamed to |\childdocforward| and |\childdocforwardprefix|
and functionality expanded
\end{itemize}

%%%%%%%%%%%%%%%%%%%%%%%%%%%%%%%%%%%%%%%%
\paragraph{v1.0:} 2017/04/27

\begin{itemize}
\item
manual and install package
\item
first version published on CTAN
\end{itemize}

%%%%%%%%%%%%%%%%%%%%%%%%%%%%%%%%%%%%%%%%
\paragraph{v0.6:} 2017/04/26

\begin{itemize}
\item
redirection mechanism added
\end{itemize}

%%%%%%%%%%%%%%%%%%%%%%%%%%%%%%%%%%%%%%%%
\paragraph{v0.5:} 2017/04/26

\begin{itemize}
\item
functionality in definition file
\end{itemize}


%%%%%%%%%%%%%%%%%%%%%%%%%%%%%%%%%%%%%%%%%%%%%%%%%%%%%%%%%%%%%%%%%%%%%%%%%%%%%%%%
%%%%%%%%%%%%%%%%%%%%%%%%%%%%%%%%%%%%%%%%%%%%%%%%%%%%%%%%%%%%%%%%%%%%%%%%%%%%%%%%
%%%%%%%%%%%%%%%%%%%%%%%%%%%%%%%%%%%%%%%%%%%%%%%%%%%%%%%%%%%%%%%%%%%%%%%%%%%%%%%%
\appendix

\settowidth\MacroIndent{\rmfamily\scriptsize 000\ }

 \DocInput{childdoc.dtx}

\end{document}
%</driver>
% \fi
%
% %%%%%%%%%%%%%%%%%%%%%%%%%%%%%%%%%%%%%%%%%%%%%%%%%%%%%%%%%%%%%%%%%%%%%%%%%%%%%%
% %%%%%%%%%%%%%%%%%%%%%%%%%%%%%%%%%%%%%%%%%%%%%%%%%%%%%%%%%%%%%%%%%%%%%%%%%%%%%%
% \section{Sample}
%\iffalse
%<*samplemain>
%\fi
%
% The following presents a sample document
% with two chapters, two parts, a title page,
% a compile flag as well as three forwarding files to set the flag.
% It consists of eight |.tex| files:
% \begin{center}
% \begin{tabular}{ll}
% |cdocsamp.tex|&main file\\
% |cdocsch1.tex|&include file for chapter 1\\
% |cdocsch2.tex|&include file for chapter 2\\
% |cdocspt3.tex|&include file for part 3\\
% |cdocspt4.tex|&include file for part 4\\
% |cdocsdrf.tex|&forwarding file for main file in draft mode\\
% |cdocsfi1.tex|&forwarding file for final version of chapter 1\\
% |cdocsfi2.tex|&forwarding file for final version of chapter 2\\
% \end{tabular}
% \end{center}
% Each of the eight files can be compiled directly by the \LaTeX{} compiler.
%
% %%%%%%%%%%%%%%%%%%%%%%%%%%%%%%%%%%%%%%
% \paragraph{Main File.}
%
% The main file is called |cdocsamp.tex|.
%
% Load the \textsf{childdoc} definitions and
% declare the filename for the main document:
%    \begin{macrocode}
\input{childdoc.def}
\childdocmain{}
%    \end{macrocode}

% Optional override for |\version| flag:
%    \begin{macrocode}
%%\ifchilddoc\else\providecommand{\version}{draft}\fi
%    \end{macrocode}

% Define the default values for the |\version| flag
% (|final| for the main file and |draft| for childs):
%    \begin{macrocode}
\ifchilddoc
\providecommand{\version}{draft}
\else
\providecommand{\version}{final}
\fi
%    \end{macrocode}

% Load the standard document class:
%    \begin{macrocode}
\documentclass[12pt]{article}
%    \end{macrocode}

% Start the document body:
%    \begin{macrocode}
\begin{document}
%    \end{macrocode}

% Declare a title page.
% Print title, part of document being processed and version flag:
%    \begin{macrocode}
\addtocounter{page}{-1}
\begin{center}
{\LARGE\bfseries{}childdoc example\par}
\vspace{1cm}
\ifchilddoc
\ifchilddocmanual part\else chapter\fi:
`\childdocname' of `\childdocjob'\par
\else
main document: `\childdocjob'\par
\fi
version: \version\par
\end{center}
\newpage
%    \end{macrocode}

% Manually include selected file,
% otherwise process as usual:
%    \begin{macrocode}
\ifchilddocmanual
\section*{part `\childdocname'}
\input{\childdocname}
\else
%    \end{macrocode}

% Include the two chapters:
%    \begin{macrocode}
\include{cdocsch1}
\include{cdocsch2}
%    \end{macrocode}

% Include the two parts unless only chapters should be displayed:
%    \begin{macrocode}
\ifchilddoc\else
\section{part three}
\input{cdocspt3}
\section{part four}
\input{cdocspt4}
\fi
%    \end{macrocode}

% Process as usual until here:
%    \begin{macrocode}
\fi
%    \end{macrocode}

% End of document body:
%    \begin{macrocode}
\end{document}
%    \end{macrocode}
%\iffalse
%</samplemain>
%\fi
%
% %%%%%%%%%%%%%%%%%%%%%%%%%%%%%%%%%%%%%%
% \paragraph{Chapter Include Files.}
%
% The include files are called |cdocsch1.tex| and |cdocsch2.tex|.
%
%\iffalse
%<*samplechap1|samplechap2>
%\fi

% Optional override for |\version| flag:
%    \begin{macrocode}
%%\providecommand{\version}{final}
%    \end{macrocode}

% Include the main document:
%    \begin{macrocode}
\input{childdoc.def}
\childdocof{cdocsamp}
%    \end{macrocode}

%\iffalse
%</samplechap1|samplechap2>
%\fi
%
%\iffalse
%<*samplechap1>
%\fi
% Some text for chapter 1:
%    \begin{macrocode}
\section{one}
some text in chapter one
%    \end{macrocode}

%\iffalse
%</samplechap1>
%\fi
% Some text for chapter 2:
%\iffalse
%<*samplechap2>
%\fi
%    \begin{macrocode}
\section{two}
more text in chapter two
%    \end{macrocode}

%\iffalse
%</samplechap2>
%\fi
%
% %%%%%%%%%%%%%%%%%%%%%%%%%%%%%%%%%%%%%%
% \paragraph{Part Include Files.}
%
% The include files are called |cdocspt3.tex| and |cdocspt4.tex|.
%
%\iffalse
%<*samplepart3|samplepart4>
%\fi

% Optional override for |\version| flag:
%    \begin{macrocode}
%%\providecommand{\version}{final}
%    \end{macrocode}

% Include the main document:
%    \begin{macrocode}
\input{childdoc.def}
\childdocby{cdocsamp}
%    \end{macrocode}

%\iffalse
%</samplepart3|samplepart4>
%\fi
%
%\iffalse
%<*samplepart3>
%\fi
% Some text for part 3:
%    \begin{macrocode}
some text in part three
%    \end{macrocode}

%\iffalse
%</samplepart3>
%\fi
% Some text for part 4:
%\iffalse
%<*samplepart4>
%\fi
%    \begin{macrocode}
more text in part four
%    \end{macrocode}

%\iffalse
%</samplepart4>
%\fi
%
% %%%%%%%%%%%%%%%%%%%%%%%%%%%%%%%%%%%%%%
% \paragraph{Forwarding for a Complete Draft.}
%
% The following forwarding file |cdocsdrf.tex|
% compiles the main document in draft mode:
%\iffalse
%<*sampledraft>
%\fi
%    \begin{macrocode}
\def\version{draft}
\input{childdoc.def}
\childdocforward{cdocsamp}
%    \end{macrocode}

%\iffalse
%</sampledraft>
%\fi
%
% %%%%%%%%%%%%%%%%%%%%%%%%%%%%%%%%%%%%%%
% \paragraph{Forwarding for Final Version of the Chapters.}
%
% The following forwarding files |cdocsfn1.tex| and |cdocsfn2.tex|
% (with identical content)
% compile the final versions of the child documents
% |cdocsch1.tex| and |cdocsch2.tex|, respectively:
%\iffalse
%<*samplefinal>
%\fi
%    \begin{macrocode}
\def\version{final}
\input{childdoc.def}
\childdocforwardprefix[cdocsamp]{cdocsfn}{cdocsch}
%    \end{macrocode}

%\iffalse
%</samplefinal>
%\fi
%
% %%%%%%%%%%%%%%%%%%%%%%%%%%%%%%%%%%%%%%
% \paragraph{Command Line Processing.}
%
% The following three command lines generate the output files
% |cdocscld|, |cdocscl1| and |cdocscl2|
% which should be identical to
% |cdocsdrf|, |cdocsch1| and |cdocsfn2|, respectively:
% \begin{center}
% \begin{tabular}{l}
% |latex -jobname cdocscld \|\\
% |  "\def\version{draft}\input{childdoc.def}\childdocforward{cdocsamp}"|\\
% |latex -jobname cdocscl1 \|\\
% |  "\input{childdoc.def}\childdocforward[cdocsamp]{cdocsch1}"|\\
% |latex -jobname cdocscl2 \|\\
% |  "\def\version{final}\input{childdoc.def}\childdocforward{cdocsch2}"|
% \end{tabular}
% \end{center}
% Note that the trailing backslash on each first line
% merely continues the input to the second line
% (for convenient cut ant paste).
% Furthermore, the command |latex| can be replaced by any
% of its alternative versions such as |pdflatex|.
%
% %%%%%%%%%%%%%%%%%%%%%%%%%%%%%%%%%%%%%%%%%%%%%%%%%%%%%%%%%%%%%%%%%%%%%%%%%%%%%%
% %%%%%%%%%%%%%%%%%%%%%%%%%%%%%%%%%%%%%%%%%%%%%%%%%%%%%%%%%%%%%%%%%%%%%%%%%%%%%%
% \section{Implementation}
%\iffalse
%<*package>
%\fi
%
% This section describes the definitions file |childdoc.def|.

% The definitions cannot be loaded using |\usepackage| or |\RequirePackage|
% which has a mechanism to prevent loading a style file more than once.
% When loading the definitions by means of |\input|
% multiple instances have to be prevented manually:
%\iffalse
%This code needs to be before the `\ProvidesFile' directive
%which is defined at the beginning of this file.
%Therefore it is also placed there and commented out here.
%</package>
%<*discard>
%\fi
%    \begin{macrocode}
\ifdefined\childdocmain\endinput\fi
%    \end{macrocode}
%\iffalse
%</discard>
%<*package>
%\fi
%
% \macro{\ifchilddoc}
% \macro{\ifchilddocmanual}
% The conditional |\ifchilddoc| tells whether a
% child (true) or main (false) document is being compiled.
% The conditional |\ifchilddocmanual| tells whether
% the |\includeonly| mechanism is used (false) or
% the selection of child files must be performed manually (true).
% The definitions initialise to false:
%    \begin{macrocode}
\newif\ifchilddoc
\newif\ifchilddocmanual
%    \end{macrocode}

% \macro{\childdocname}
% \macro{\childdocjob}
% The macro |\childdocname| stores the name of the main document
% to be compiled. The macro |\childdocjob| stores the name of
% the document on which the \LaTeX{} compiler was originally invoked.
% The content of |\jobname| cannot be compared
% to filenames specified in the source due to different catcodes.
% The following code rescans |\jobname|, stores the result
% in |\childdocname| and saves a copy in |\childdocjob|:
%    \begin{macrocode}
\edef\childdocname{\scantokens\expandafter{\jobname\noexpand}}
\let\childdocjob\childdocname
%    \end{macrocode}

% \macro{\childdocdisable}
% The macro |\childdocdisable| prevents the main file
% from being processed more than once.
% At this stage, the main document command |\childdocmain|
% is assumed to be called once again where it should do nothing.
% Any subsequent call to it should prevent
% a secondary processing of the main document
% It overwrites the forwarding commands
% |\childdocof| and |\childdocforward|
% with empty macros to prevent further inclusions of the main document:
%    \begin{macrocode}
\newcommand{\childdocdisable}
{
  \renewcommand{\childdocmain}[1]{\renewcommand{\childdocmain}[1]{\endinput}}
  \renewcommand{\childdocof}[1]{}
  \renewcommand{\childdocby}[2][]{}
  \renewcommand{\childdocforward}[2][]{}
  \renewcommand{\childdocdisable}{}
}
%    \end{macrocode}

% \macro{\childdocmain}
% The macro |\childdocmain| is to be called at the top of the main file
% with nothing or the main filename (without extension) as argument.
% First, it breaks loops.
% If the argument is not empty and does not match |\childdocname|
% (which is set by the first inclusion of |childdoc.def|),
% |\ifchilddoc| is set to true, |\includeonly| is applied to the child file
% and |\jobname| is set to the main file
% (for proper handling of |.aux| files):
%    \begin{macrocode}
\newcommand{\childdocmain}[1]
{
  \childdocdisable\childdocmain{}
  \if?#1?\else
    \begingroup
      \def\childdoctmp{#1}
      \ifx\childdoctmp\childdocname
        \def\childdoctmp{}
      \else
        \def\childdoctmp
        {
          \childdoctrue
          \includeonly{\childdocname}
          \def\childdocjob{#1}
          \def\jobname{#1}
        }
      \fi
      \expandafter
    \endgroup
    \childdoctmp
  \fi
}
%    \end{macrocode}

% \macro{\childdocof}
% The command |\childdocof| redirects
% compilation to the main file |#1|.
%    \begin{macrocode}
\newcommand{\childdocof}[1]
{
  \childdocdisable
  \childdoctrue
  \includeonly{\childdocname}
  \def\jobname{#1}
  \def\childdocjob{#1}
  \input{#1}
}
%    \end{macrocode}

% \macro{\childdocby}
% The command |\childdocby| ....
%    \begin{macrocode}
\newcommand{\childdocby}[2][]
{
  \childdocdisable
  \childdoctrue
  \childdocmanualtrue
  \if?#1?\else
    \def\jobname{#2}
  \fi
  \def\childdocjob{#2}
  \input{#2}
  \endinput
}
%    \end{macrocode}

% \macro{\childdocforward}
% The command |\childdocforward| redirects
% compilation to the main file or
% (if the optional argument is given) a child file.
% Parameters are set as if the main file
% or a child file starting with |\childdocof| was compiled.
% Then compilation is handed over to the main file:
%    \begin{macrocode}
\newcommand{\childdocforward}[2][]
{
  \begingroup
    \if?#1?
      \def\childdoctmp
      {
        \def\childdocname{#2}
        \def\childdocjob{#2}
        \def\jobname{#2}
        \input{#2}
        \endinput
      }
    \else
      \def\childdoctmp
      {
        \childdocdisable
        \def\childdocname{#2}
        \childdoctrue
        \includeonly{#2}
        \def\childdocjob{#1}
        \def\jobname{#1}
        \input{#1}
        \endinput
      }
    \fi
    \expandafter
  \endgroup
  \childdoctmp
}
%    \end{macrocode}

% \macro{\childdocforwardprefix}
% The command |\childdocforwardprefix| redirects
% compilation to the main or a child file by means of a pattern.
% The prefix |#1| in the current filename is replaced by |#2|
% and the suffix of the current filename is kept
% (it is assumed that the filename does not contain the substring `|~~~|'
% which is used as a delimiter).
% Compilation is handed over to the new file by |\childdocforward|:
%    \begin{macrocode}
\newcommand{\childdocforwardprefix}[3][]
{
  \begingroup
    \def\childdocextract #2##1~~~{\def\childdoctmp{\childdocforward[#1]{#3##1}}}
    \expandafter\childdocextract\childdocname~~~
    \expandafter
  \endgroup
  \childdoctmp
}
%    \end{macrocode}

% \macro{\childdoc}
% The deprecated macro |\childdoc| is a legacy version of |\childdocmain|:
%    \begin{macrocode}
\newcommand{\childdoc}{\childdocmain}
%    \end{macrocode}

% \macro{\childdocredirect}
% The deprecated macro |\childdocredirect| is a legacy version
% of |\childdocforward| and |\childdocforwardprefix|:
%    \begin{macrocode}
\newcommand{\childdocredirect}[2][]
{
  \begingroup
    \if?#1?
      \def\childdoctmp{\childdocforward{#2}}
    \else
      \def\childdoctmp{\childdocforwardprefix{#1}{#2}}
    \fi
    \expandafter
  \endgroup
  \childdoctmp
}
%    \end{macrocode}

%\iffalse
%</package>
%\fi
%
\endinput
\childdocforward[cdocsamp]{cdocsch1}"|\\
% |latex -jobname cdocscl2 \|\\
% |  "\def\version{final}% \iffalse
%
% childdoc.dtx Copyright (C) 2017-2018 Niklas Beisert
%
% This work may be distributed and/or modified under the
% conditions of the LaTeX Project Public License, either version 1.3
% of this license or (at your option) any later version.
% The latest version of this license is in
%   http://www.latex-project.org/lppl.txt
% and version 1.3 or later is part of all distributions of LaTeX
% version 2005/12/01 or later.
%
% This work has the LPPL maintenance status `maintained'.
%
% The Current Maintainer of this work is Niklas Beisert.
%
% This work consists of the files childdoc.dtx and childdoc.ins
% and the derived files childdoc.def and cdocsamp.tex with
% cdocsch1.tex, cdocsch2.tex, cdocsdrf.tex, cdocsfn1.tex, cdocsfn2.tex.
%
%<package>\ifdefined\childdocmain\endinput\fi
%<package>\ProvidesFile{childdoc.def}[2018/12/30 v2.0 child document driver]
%<samplemain>\ProvidesFile{cdocsamp.tex}[2018/12/30 v2.0 sample for childdoc]
%<*driver>
%\ProvidesFile{childdoc.drv}[2018/12/30 v2.0 childdoc reference manual file]
\PassOptionsToClass{10pt,a4paper}{article}
\documentclass{ltxdoc}

\usepackage[margin=35mm]{geometry}
\usepackage{hyperref}
\usepackage{hyperxmp}
\usepackage[usenames]{color}

\hypersetup{colorlinks=true}
\hypersetup{pdfstartview=FitH}
\hypersetup{pdfpagemode=UseNone}
\hypersetup{pdfsource={}}
\hypersetup{pdflang={en-UK}}
\hypersetup{pdfcopyright={Copyright 2017-2018 Niklas Beisert.
  This work may be distributed and/or modified under the
  conditions of the LaTeX Project Public License, either version 1.3
  of this license or (at your option) any later version.}}
\hypersetup{pdflicenseurl={http://www.latex-project.org/lppl.txt}}
\hypersetup{pdfcontactaddress={ETH Zurich, ITP, HIT K,
  Wolfgang-Pauli-Strasse 27}}
\hypersetup{pdfcontactpostcode={8093}}
\hypersetup{pdfcontactcity={Zurich}}
\hypersetup{pdfcontactcountry={Switzerland}}
\hypersetup{pdfcontactemail={nbeisert@itp.phys.ethz.ch}}
\hypersetup{pdfcontacturl={http://people.phys.ethz.ch/\xmptilde nbeisert/}}

\newcommand{\secref}[1]{\hyperref[#1]{section \ref*{#1}}}

\parskip1ex
\parindent0pt
\let\olditemize\itemize
\def\itemize{\olditemize\parskip0pt}

\begin{document}

\title{The \textsf{childdoc} Package}
\hypersetup{pdftitle={The childdoc Package}}
\author{Niklas Beisert\\[2ex]
  Institut f\"ur Theoretische Physik\\
  Eidgen\"ossische Technische Hochschule Z\"urich\\
  Wolfgang-Pauli-Strasse 27, 8093 Z\"urich, Switzerland\\[1ex]
  \href{mailto:nbeisert@itp.phys.ethz.ch}
  {\texttt{nbeisert@itp.phys.ethz.ch}}}
\hypersetup{pdfauthor={Niklas Beisert}}
\hypersetup{pdfsubject={Manual for the LaTeX2e Package childdoc}}
\date{30 December 2018, \textsf{v2.0}}
\maketitle

\begin{abstract}\noindent
\textsf{childdoc} is a \LaTeXe{} package
that enables the direct compilation
of document sections included by |\include|
to individual files.
\end{abstract}

\begingroup
\parskip0ex
\tableofcontents
\endgroup

%%%%%%%%%%%%%%%%%%%%%%%%%%%%%%%%%%%%%%%%%%%%%%%%%%%%%%%%%%%%%%%%%%%%%%%%%%%%%%%%
%%%%%%%%%%%%%%%%%%%%%%%%%%%%%%%%%%%%%%%%%%%%%%%%%%%%%%%%%%%%%%%%%%%%%%%%%%%%%%%%
\section{Introduction}

\LaTeX{} provides a mechanism to structure a large document (such as a book)
into a main file and several child files (containing the chapters)
using the |\include| command.
This mechanism is beneficial for documents
which span hundreds of pages in order to
make the source file(s) more manageable.
Moreover, compilation can be restricted to
selected child files by means of the |\includeonly| command.
The latter feature can be used to reduce the compilation time while editing
(this was significantly more useful in the earlier days of \LaTeX{})
or to generate a smaller document which is easier to navigate.
Another application of |\includeonly| is to generate
documents consisting of selected parts of the complete document.

However, there are a few drawbacks of the plain |\include| mechanism:
\begin{itemize}
\item
The child files cannot be compiled on their own,
they can only be compiled via the main file.
A naive editing environment
(such as a text editor with an option
to have the current file processed by \LaTeX)
may require one to switch to the main file before compiling;
attempting to compile the child file produces errors.
\item
The main file must be modified (each time)
to adjust the |\includeonly| command
to the present needs. This easily leaves the main file in a messy state.
\item
The generated document will always carry the filename
of the main document. This is inconvenient if
several child files are to be compiled and
to be kept for distribution.
\end{itemize}

The present package provides a simple interface
to make child files individually compilable by \LaTeX{}.
Compiling a child file then has the same effect as compiling
the main file with an |\includeonly| command
to select the appropriate child.
Moreover the generated document will carry the name of the child
rather than the main file.
This resolves all three above issues.

This feature is meant to make the editing of books,
thesis documents and lecture notes somewhat more convenient.
However, the package can also be used efficiently for
composing a series of documents (such as exercise sheets)
which are typically distributed individually.
It then assists the author in generating the individual documents
(potentially in different versions)
as well as a document containing the collected series.
Another application is in developing style files
or other kinds of included material
where compilation of the style file could redirect
to a sample or test file.

%%%%%%%%%%%%%%%%%%%%%%%%%%%%%%%%%%%%%%%%%%%%%%%%%%%%%%%%%%%%%%%%%%%%%%%%%%%%%%%%
%%%%%%%%%%%%%%%%%%%%%%%%%%%%%%%%%%%%%%%%%%%%%%%%%%%%%%%%%%%%%%%%%%%%%%%%%%%%%%%%
\section{Usage}

First of all, the package \textsf{childdoc} is \emph{not} a standard
\LaTeXe{} |.sty| style file! Therefore it needs to be invoked in
a non-standard way.

%%%%%%%%%%%%%%%%%%%%%%%%%%%%%%%%%%%%%%%%%%%%%%%%%%%%%%%%%%%%%%%%%%%%%%%%%%%%%%%%
\subsection{Included Files}
\label{sec:include}

%%%%%%%%%%%%%%%%%%%%%%%%%%%%%%%%%%%%%%%%
\DescribeMacro{\childdocmain}
To use the package, add the commands
\begin{center}
\begin{tabular}{l}
|\input{childdoc.def}|\\
|\childdocmain{}|\\
\end{tabular}
\end{center}
at the very top of the main \LaTeX{} file,
in particular \emph{before} the |\documentclass| statement!
The argument of |\childdocmain| should be left empty
(but it must be present).

%%%%%%%%%%%%%%%%%%%%%%%%%%%%%%%%%%%%%%%%
\DescribeMacro{\childdocof}
Furthermore, add the commands
\begin{center}
\begin{tabular}{l}
|\input{childdoc.def}|\\
|\childdocof{|\textit{main}|}|\\
\end{tabular}
\end{center}
at the top of every child file \textit{child}
which is included by |\include{|\textit{child}|}|
from within the main file
(or at least for those files to be compiled individually).
The argument \textit{main} must be the filename of the main file.

There are a couple of
considerations in setting up the main and child documents:

%%%%%%%%%%%%%%%%%%%%%%%%%%%%%%%%%%%%%%%%
\paragraph{Restrictions.}

Please note the following restrictions:
\begin{itemize}
\item
|\childdocmain| must be called with one argument \textit{main}
to ensure compatibility with earlier version of the package.
It must either be empty (|\childdocmain{}|)
or precisely match the filename of the main file in which it is specified.
See \secref{sec:detection} for further information.
\item
The filename \textit{main} must be specified without the |.tex| extension.
\item
The filename \textit{main} is case sensitive
(even in case-insensitive file systems)
due to internal string comparison.
\item
The argument \textit{main} should be fully expanded, it cannot be a macro.
\item
Subdirectories and special characters should be avoided in filenames.
\item
The command |\childdocmain{|\textit{main}|}| must be followed by a whitespace.
It should not be followed immediately by another command
or by a comment mark `|%|'.
This is because the \TeX{} parser reads the token immediately following
the argument of |\childdocmain| and puts it
at the beginning of every child section;
however, a white\-space is ignored.
\end{itemize}

%%%%%%%%%%%%%%%%%%%%%%%%%%%%%%%%%%%%%%%%
\paragraph{Content of Main File.}

It is advisable to place all content in the child files included by |\include|.
Any output contained in the main file will appear in all child documents
unless suppressed manually;
it cannot be suppressed automatically by the |\includeonly| directive
and thus should normally be avoided.
A method to include some content in the main file
by means of conditional processing is described in \secref{sec:conditional}.

%%%%%%%%%%%%%%%%%%%%%%%%%%%%%%%%%%%%%%%%
\paragraph{Page Numbering.}

When only a part of the document is compiled,
the appropriate numbering of pages
(as well as other status parameters)
is determined from the |.aux| files.
The latter contain information from previous passes.
However this information needs to propagate through
all intermediate child documents.
Therefore the page numbering in child documents may well
be inconsistent until the complete document is compiled at least once.

A useful (if unconventional) way to always ensure a consistent
page numbering is to restart the numbering in each child document
and denote the pages by `\textit{child}|.|\textit{page}'
where \textit{child} represents the chapter/section number of the child file.
This can be achieved by the command
|\numberwithin{page}{|\textit{child}|}|
of the \textsf{amsmath} package
where \textit{child} can be |chapter| or |section|
depending on the chosen structuring.
Alternatively, one can modify the macro |\thepage| appropriately
and reset the counter |page| at the start of each child file.

%%%%%%%%%%%%%%%%%%%%%%%%%%%%%%%%%%%%%%%%%%%%%%%%%%%%%%%%%%%%%%%%%%%%%%%%%%%%%%%%
\subsection{Conditional Processing}
\label{sec:conditional}

The package provides a mechanism to compile different versions
of a document. To customise the versions further some conditional processing
can come in handy to distinguish which version is being compiled.
The package provides two macros to describe the compilation context:

%%%%%%%%%%%%%%%%%%%%%%%%%%%%%%%%%%%%%%%%
\DescribeMacro{\ifchilddoc}
The conditional |\ifchilddoc| distinguishes between the compilation of
child documents and the main document:
%
\begin{center}
|\ifchilddoc |\textit{child-code}| |[|\||else |\textit{main-code}]| \||fi|
\end{center}

%%%%%%%%%%%%%%%%%%%%%%%%%%%%%%%%%%%%%%%%
\DescribeMacro{\childdocname}
\DescribeMacro{\childdocjob}
The macro |\childdocname| contains the filename (without extension)
of the main or child file being processed.
Note that |\childdocjob| will always contain the name of the main file.

%%%%%%%%%%%%%%%%%%%%%%%%%%%%%%%%%%%%%%%%
\paragraph{Title Page.}

Conditional processing can be used to include a title or banner page
in the main document when proper precautions are taken.
Importantly, the code in the main file should ensure that the page counter
(as well as other status parameters which are stored in the |.aux| files)
takes the same value after the conditional processing.
Otherwise the page numbers may take divergent values
depending on which part is compiled.

For example, a title page could be declared by:
%
\begin{center}
\begin{tabular}{l}
|\ifchilddoc\||else|\\
|\addtocounter{page}{-1}|\\
\textit{code for title page}\\
|\newpage|\\
|\||fi|
\end{tabular}
\end{center}
%
A banner page for the child documents can be generated by:
%
\begin{center}
\begin{tabular}{l}
|\ifchilddoc|\\
|\addtocounter{page}{-1}|\\
\textit{code for banner page}\\
|\newpage|\\
|\||fi|
\end{tabular}
\end{center}
%
Here one could write a message such as:
\begin{center}
|This is the part \childdocname{} of \childdocjob{}.|
\end{center}

%%%%%%%%%%%%%%%%%%%%%%%%%%%%%%%%%%%%%%%%%%%%%%%%%%%%%%%%%%%%%%%%%%%%%%%%%%%%%%%%
\subsection{Flags}
\label{sec:flags}

The package makes it easy to generate different versions
of the main or child documents.
To this end compilation flags can be defined
and assigned different default values.
They will be particularly useful in conjunction
with the forwarding mechanism described in \secref{sec:forward}.

For example, it may be useful to have a flag |\version|
which can be set to |draft| or |final|.
The document source will contain some conditional code
depending on the value of |\version|.
Suppose further, the flag should default to |final| for the main file
and to |draft| for child files
which is a natural assignment for editing the document.
This is achieved by placing the following code
in the preamble of the main document
(below the |\childdocmain| directive):
%
\begin{center}
\begin{tabular}{l}
|\ifchilddoc|\\
|\providecommand{\version}{draft}|\\
|\||else|\\
|\providecommand{\version}{final}|\\
|\||fi|
\end{tabular}
\end{center}
%
The definition by |\providecommand| makes sure
that previous definitions are not overwritten.
Further statements |\providecommand{\version}{...}|
can thus be added before the above code to override it.

For the main file, one might add a line
(between |\childdocmain| and the above block)
%
\begin{center}
|%\ifchilddoc\||else\providecommand{\version}{draft}\||fi|
\end{center}
%
which can be uncommented to produce a draft version.
Likewise one can add a line to the very top of a child file
(above the |\childdocof{|\textit{main}|}| directive)
%
\begin{center}
|%\providecommand{\version}{final}|
\end{center}
%
which can be uncommented to produce the final version of this child document.

%%%%%%%%%%%%%%%%%%%%%%%%%%%%%%%%%%%%%%%%%%%%%%%%%%%%%%%%%%%%%%%%%%%%%%%%%%%%%%%%
\subsection{Forwarding}
\label{sec:forward}

Different versions of the main or child documents
using compilation flags as described in \secref{sec:flags}
can be (permanently) stored in different files
for convenient compilation, viewing and distribution.
To this end, the package defines a command
to pass on compilation to a different file:

%%%%%%%%%%%%%%%%%%%%%%%%%%%%%%%%%%%%%%%%
\DescribeMacro{\childdocforward}
The command |\childdocforward| redirects processing to
another source file:
%
\begin{center}
\begin{tabular}{l}
|\input{childdoc.def}|\\
|\childdocforward[|\textit{main}|]{|\textit{dest}|}|\\
\end{tabular}
\end{center}
%
The argument \textit{dest} is the destination file
(without extension).
It should be the main file or one of the child files.
Note that further \textsf{childdoc} directives
such as |\childdocof| and |\childdocforward|
in the indicated file will be processed in this form.
The optional argument \textit{main}
passes on directly to the main file \textit{main}
while pretending to compile the child \textit{dest}.
This form behaves as if \textit{dest}
issues |\childdocof{|\textit{main}|}| right away,
and no further \textsf{childdoc} directives will be processed.

%%%%%%%%%%%%%%%%%%%%%%%%%%%%%%%%%%%%%%%%
\DescribeMacro{\...prefix}
In the alternative form |\childdocforwardprefix|,
%
\begin{center}
\begin{tabular}{l}
|\input{childdoc.def}|\\
|\childdocforwardprefix[|\textit{main}|]{|\textit{prefix}|}{|\textit{dest}|}|
\end{tabular}
\end{center}
%
the destination file is determined by a pattern
depending on the current file:
To make this work, the current file must be called
`{\textit{prefix}\hspace{0.2em}\textit{suffix}}'
with \textit{prefix} matching precisely the argument.
Processing is then passed on to the file
`{\textit{dest}\hspace{0.2em}\textit{suffix}}'.
Surely, the same effect is achieved by
directly specifying the
argument `{\textit{dest}\hspace{0.2em}\textit{suffix}}'
in the first form.
However, that requires to set up a different file
for each child. With the alternative form of the command
all these files can have exactly the same content
which simplifies setting them up and maintaining them.

For example, the following file |draft.tex|
with a compilation flag |\version| as described in \secref{sec:flags}
compiles the main document as a draft:
%
\begin{center}
\begin{tabular}{l}
|\def\version{draft}|\\
|\input{childdoc.def}|\\
|\childdocforward{|\textit{main}|}|
\end{tabular}
\end{center}
%
Likewise, the following files |final|\textit{nn}|.tex|
compile the final version of the child document
|child|\textit{nn}|.tex|:
%
\begin{center}
\begin{tabular}{l}
|\def\version{final}|\\
|\input{childdoc.def}|\\
|\childdocforwardprefix{final}{child}|
\end{tabular}
\end{center}
%

Note that when several versions of a main file and/or of each child file
are to be generated, it may be convenient to set up a |Makefile| or
shell script to automatise the process.

%%%%%%%%%%%%%%%%%%%%%%%%%%%%%%%%%%%%%%%%%%%%%%%%%%%%%%%%%%%%%%%%%%%%%%%%%%%%%%%%
\subsection{Command Line Processing}
\label{sec:commandline}

The effect of redirection files can also be achieved by invoking
the \LaTeX{} compiler with a more elaborate command line.
Most conveniently this should be done as part
of a shell script or a |Makefile|.

When using \textsf{childdoc} in the main file, the following
command lines effectively perform a redirection
(note that depending on the shell being used,
backslashes may have to be doubled: `|\|' $\to$ `|\\|'):
%
\begin{center}
|... -jobname "|\textit{target}|" |\\|"|[\textit{flags}]%
|\input{childdoc.def}\childdocforward[|\textit{main}|]{|\textit{dest}|}"|
\end{center}
%
Here \textit{target} is the name of the output file,
\textit{main} is the name of the main file
and \textit{dest} is the name of the main or child file to be processed
(all filenames without extensions).
The optional argument \textit{main} can be omitted
if \textit{main} matches \textit{dest}.
Optionally, compilation \textit{flags} can be defined via |\def| commands.
This command line makes the \TeX{} engine believe
it is compiling the file \textit{target}
whose content is specified as the latter parameter.
The provided code then forwards the processing to
\textit{main} or \textit{dest} as described in \secref{sec:forward}.

%%%%%%%%%%%%%%%%%%%%%%%%%%%%%%%%%%%%%%%%%%%%%%%%%%%%%%%%%%%%%%%%%%%%%%%%%%%%%%%%
\subsection{Include by Input}
\label{sec:input}

Including child documents by |\include| has some restrictions by design.
Most notably, the content of a child document always occupies
its own set of pages; pages cannot be shared between child documents.
Usually, this behaviour makes perfect sense
because each child document contain an essential part of the document.
However, in some situations it may be desirable to compose
a document from a collection of parts
without having mandatory page breaks between then.
For this case, the package
provides a mechanism to include parts
by |\input| which can also be processed individually.
However, by construction this mechanism
requires manual handling of the content to be output.

%%%%%%%%%%%%%%%%%%%%%%%%%%%%%%%%%%%%%%%%
\DescribeMacro{\ifchilddocmanual}
The main file should be prepared as usual, see \secref{sec:include}.
However, the document body must make a distinction
between processing of an individual part and of the main document, e.g.:
%
\begin{center}
\begin{tabular}{l}
|\ifchilddocmanual|\\
|\input{\childdocname}|\\
|\||else|\\
\textit{document body with }|\input{|\textit{part}|}|\\
|\||fi|
\end{tabular}
\end{center}
%
The conditional |\ifchilddocmanual| is true whenever
a part to be included by |\input| is being compiled,
and the name of the part is stored in |\childdocname|.

%%%%%%%%%%%%%%%%%%%%%%%%%%%%%%%%%%%%%%%%
\DescribeMacro{\childdocby}
Each part to be included by |\input| should start with:
%
\begin{center}
\begin{tabular}{l}
|\input{childdoc.def}|\\
|\childdocby{|\textit{main}|}|\\
\end{tabular}
\end{center}
%
The directive |\childdocby| is similar to |\childdocof|
described in \secref{sec:include},
but the subsequent selection of content must be done manually.
To that end, both |\ifchilddoc| and |\ifchilddocmanual|
will be true upon processing of a part,
and the name of the part is stored in |\childdocname|.
Note that |\jobname| will be set to the filename of the current part
so that each part receives an individual |.aux| file
that does not interfere with the |.aux| file(s) of the main document.
This behaviour can be altered by the alternative form
|\childdocby[*]{|\textit{main}|}| (with a non-empty optional argument)
which uses the |.aux| file of the main document
by setting |\jobname| to \textit{main}.

%%%%%%%%%%%%%%%%%%%%%%%%%%%%%%%%%%%%%%%%%%%%%%%%%%%%%%%%%%%%%%%%%%%%%%%%%%%%%%%%
\subsection{Driver Development}
\label{sec:driver}

The \textsf{childdoc} mechanism can also be use for the development
of definition files such as \LaTeX{} styles or classes.
This case differs from the above setup with multiple parts
included by |\include| in that no |\includeonly| should be invoked.
This can be achieved by starting the include file
(before |\ProvidesPackage|) with:
%
\begin{center}
\begin{tabular}{l}
|\input{childdoc.def}|\\
|\childdocforward{|\textit{main}|}|\\
\end{tabular}
\end{center}
%
or alternatively with:
%
\begin{center}
\begin{tabular}{l}
|\input{childdoc.def}|\\
|\childdocby{|\textit{main}|}|\\
\end{tabular}
\end{center}
%
Both forms have slightly different effects as described above.
The main file is prepared as usual, see \secref{sec:include}.

%%%%%%%%%%%%%%%%%%%%%%%%%%%%%%%%%%%%%%%%%%%%%%%%%%%%%%%%%%%%%%%%%%%%%%%%%%%%%%%%
\subsection{Legacy Detection}
\label{sec:detection}

The directive |\childdocmain| in the main file can detect
whether the complete document or merely a child is to be compiled
even without using the directive |\childdocof|.
This method is deprecated because it is less robust
and there is no compelling reason to use it;
it is merely provided for backward compatibility
and it may be removed in future versions.

If the detection mechanism is to be used,
it is mandatory to correctly specify
the filename of the main file as the argument of |\childdocmain|:
%
\begin{center}
\begin{tabular}{l}
|\input{childdoc.def}|\\
|\childdocmain{|\textit{main}|}|\\
\end{tabular}
\end{center}
%
If |\jobname| does not match the argument \textit{main} of |\childdocmain|,
it is assumed that |\jobname| points to the child file to be compiled.
When using |\childdocmain| with the main file specified as argument,
it suffices to start a child file
with just |\input{|\textit{main}|}|
without loading of the package and using |\childdocof|.
If instead all processing is done
with the appropriate \textsf{childdoc} directives,
the argument of \textit{main} of |\childdocmain| can be empty.

An alternative version of the command line processing described
in \secref{sec:commandline} using the detection mechanism reads:
%
\begin{center}
|... -jobname "|\textit{target}|" "|[\textit{flags}]%
[|\def\jobname{|\textit{dest}|}|]|\input{|\textit{main}|}"|
\end{center}

%%%%%%%%%%%%%%%%%%%%%%%%%%%%%%%%%%%%%%%%%%%%%%%%%%%%%%%%%%%%%%%%%%%%%%%%%%%%%%%%
\subsection{Manual Code}
\label{sec:manual}

In case one cannot be certain whether the definitions file |childdoc.def|
is installed on the target \TeX{} distribution
and one prefers not to ship it,
it is conceivable to paste a few relevant commands into the sources.

To that end, drop all statements |\input{childdoc.def}|
and perform the replacements as outlined below.
Instead of |\childdocmain{|\textit{main}|}| add the following code
to the top of the main file:
%
\begin{center}
\begin{tabular}{l}
|\||ifdefined\childdocname\endinput\||fi\newif\ifchilddoc|\\
|\edef\childdocname{\scantokens\expandafter{\jobname\noexpand}}|\\
|\def\childdocmain{|\textit{main}|}\||ifx\childdocmain\childdocname\||else|\\
|\childdoctrue\includeonly{\childdocname}\let\jobname\childdocmain\||fi|\\
\end{tabular}
\end{center}
%
Instead of |\childdocof{|\textit{main}|}| just include the main file
at the top of each child file:
%
\begin{center}
|\input{|\textit{main}|}|
\end{center}
%
A simple redirection |\childdocforward{|\textit{dest}|}| is achieved by:
%
\begin{center}
|\def\jobname{|\textit{dest}|}\input{\jobname}|
\end{center}
%
The redirection with prefix
|\childdocforwardprefix[|\textit{prefix}|]{|\textit{dest}|}|
is accomplished by:
%
\begin{center}
\begin{tabular}{l}
|{\edef\jobname{\scantokens\expandafter{\jobname\noexpand}}|\\
|\def\redirectjob |\textit{prefix}|#1~~~{\gdef\jobname{|\textit{dest}|#1}}|\\
|\expandafter\redirectjob\jobname~~~}\input{\jobname}|
\end{tabular}
\end{center}

In an alternative approach,
child documents can be compiled by a specific command line
without additional code or specific definitions:
%
\begin{center}
|... -jobname "|\textit{target}|" "|[\textit{flags}]%
|\includeonly{|\textit{dest}|}\input{|\textit{main}|}"|
\end{center}
%

%%%%%%%%%%%%%%%%%%%%%%%%%%%%%%%%%%%%%%%%%%%%%%%%%%%%%%%%%%%%%%%%%%%%%%%%%%%%%%%%
%%%%%%%%%%%%%%%%%%%%%%%%%%%%%%%%%%%%%%%%%%%%%%%%%%%%%%%%%%%%%%%%%%%%%%%%%%%%%%%%
\section{Information}

%%%%%%%%%%%%%%%%%%%%%%%%%%%%%%%%%%%%%%%%%%%%%%%%%%%%%%%%%%%%%%%%%%%%%%%%%%%%%%%%
\subsection{Copyright}

Copyright \copyright{} 2017--2018 Niklas Beisert

This work may be distributed and/or modified under the
conditions of the \LaTeX{} Project Public License, either version 1.3
of this license or (at your option) any later version.
The latest version of this license is in
  \url{http://www.latex-project.org/lppl.txt}
and version 1.3 or later is part of all distributions of \LaTeX{}
version 2005/12/01 or later.

This work has the LPPL maintenance status `maintained'.

The Current Maintainer of this work is Niklas Beisert.

This work consists of the files |README.txt|, |childdoc.ins| and |childdoc.dtx|
as well as the derived files |childdoc.def|, |cdocsamp.tex|
with |cdocsch1.tex|, |cdocsch2.tex|, |cdocspt3.tex|, |cdocspt4.tex|,
|cdocsdrf.tex|, |cdocsfn1.tex|, |cdocsfn2.tex|
as well as |childdoc.pdf|.

%%%%%%%%%%%%%%%%%%%%%%%%%%%%%%%%%%%%%%%%%%%%%%%%%%%%%%%%%%%%%%%%%%%%%%%%%%%%%%%%
\subsection{Files and Installation}

The package consists of the files:
%
\begin{center}
\begin{tabular}{ll}
    |README.txt|   & readme file \\
    |childdoc.ins| & installation file \\
    |childdoc.dtx| & source file \\
    |childdoc.def| & definition file \\
    |cdocsamp.tex| & sample main file \\
    |cdocsch1.tex| & sample include file \\
    |cdocsch2.tex| & sample include file \\
    |cdocspt3.tex| & sample part file \\
    |cdocspt4.tex| & sample part file \\
    |cdocsdrf.tex| & sample redirection file \\
    |cdocsfn1.tex| & sample redirection file \\
    |cdocsfn2.tex| & sample redirection file \\
    |childdoc.pdf| & manual
\end{tabular}
\end{center}
%
The distribution consists of the files
|README.txt|, |childdoc.ins| and |childdoc.dtx|.
%
\begin{itemize}
\item
Run (pdf)\LaTeX{} on |childdoc.dtx|
to compile the manual |childdoc.pdf| (this file).
\item
Run \LaTeX{} on |childdoc.ins| to create the definitions file |childdoc.def|
and the sample |cdocsamp.tex| with include files
|cdocsch1.tex|, |cdocsch2.tex|, |cdocspt3.tex|, |cdocspt4.tex|,
|cdocsdrf.tex|, |cdocsfn1.tex|, |cdocsfn2.tex|.
Then copy the file |childdoc.def| to an appropriate directory of your \LaTeX{}
distribution, e.g.\ \textit{texmf-root}|/tex/latex/childdoc|.
\end{itemize}

%%%%%%%%%%%%%%%%%%%%%%%%%%%%%%%%%%%%%%%%%%%%%%%%%%%%%%%%%%%%%%%%%%%%%%%%%%%%%%%%
\subsection{Related CTAN Packages}

There are several other packages which offer a similar functionality:
%
\begin{itemize}
\item
The packages
\href{http://ctan.org/pkg/docmute}{\textsf{docmute}},
\href{http://ctan.org/pkg/includex}{\textsf{includex}} and
\href{http://ctan.org/pkg/standalone}{\textsf{standalone}}
provide commands to include only the document body of
a child file thus allowing both files to be compiled individually.
\item
The packages \href{http://ctan.org/pkg/subdocs}{\textsf{subdocs}}
and \href{http://ctan.org/pkg/subfiles}{\textsf{subfiles}}
provide structures in which the main and child documents can be
encapsulated and allowing them to be compiled individually.
The inclusion mechanism is different from the conventional |\include|.
\item
The package \href{http://ctan.org/pkg/combine}{\textsf{combine}}
is an elaborate solution to combine several documents into one.
\end{itemize}
%
See also the CTAN topic \href{http://ctan.org/topic/subdocs}{\textsf{subdocs}}
for further related packages.
The present package differs from the above solutions in that
a document structure constructed with the conventional |\include| mechanism
just needs two extra commands at the top of every file
such that all constituent files can be compiled individually.

%%%%%%%%%%%%%%%%%%%%%%%%%%%%%%%%%%%%%%%%%%%%%%%%%%%%%%%%%%%%%%%%%%%%%%%%%%%%%%%%
%\subsection{Feature Suggestions}
%
%The following is a list of features which may be useful for future
%versions of this package:
%%
%\begin{itemize}
%\item
%\ldots
%\end{itemize}

%%%%%%%%%%%%%%%%%%%%%%%%%%%%%%%%%%%%%%%%%%%%%%%%%%%%%%%%%%%%%%%%%%%%%%%%%%%%%%%%
\subsection{Revision History}

%%%%%%%%%%%%%%%%%%%%%%%%%%%%%%%%%%%%%%%%
\paragraph{v2.0:} 2018/12/30

\begin{itemize}
\item
immediate forward processing
\item
added |\childdocby| mechanism
\item
manual restructured
\end{itemize}

%%%%%%%%%%%%%%%%%%%%%%%%%%%%%%%%%%%%%%%%
\paragraph{v1.6:} 2018/01/17

\begin{itemize}
\item
application for development of include files
\item
corrections to manual
\end{itemize}

%%%%%%%%%%%%%%%%%%%%%%%%%%%%%%%%%%%%%%%%
\paragraph{v1.5:} 2017/05/21

\begin{itemize}
\item
more complete structuring introduced
\item
|\childdocof| introduced
\item
|\childdoc| renamed to |\childdocmain|
\item
|\childredirect| renamed to |\childdocforward| and |\childdocforwardprefix|
and functionality expanded
\end{itemize}

%%%%%%%%%%%%%%%%%%%%%%%%%%%%%%%%%%%%%%%%
\paragraph{v1.0:} 2017/04/27

\begin{itemize}
\item
manual and install package
\item
first version published on CTAN
\end{itemize}

%%%%%%%%%%%%%%%%%%%%%%%%%%%%%%%%%%%%%%%%
\paragraph{v0.6:} 2017/04/26

\begin{itemize}
\item
redirection mechanism added
\end{itemize}

%%%%%%%%%%%%%%%%%%%%%%%%%%%%%%%%%%%%%%%%
\paragraph{v0.5:} 2017/04/26

\begin{itemize}
\item
functionality in definition file
\end{itemize}


%%%%%%%%%%%%%%%%%%%%%%%%%%%%%%%%%%%%%%%%%%%%%%%%%%%%%%%%%%%%%%%%%%%%%%%%%%%%%%%%
%%%%%%%%%%%%%%%%%%%%%%%%%%%%%%%%%%%%%%%%%%%%%%%%%%%%%%%%%%%%%%%%%%%%%%%%%%%%%%%%
%%%%%%%%%%%%%%%%%%%%%%%%%%%%%%%%%%%%%%%%%%%%%%%%%%%%%%%%%%%%%%%%%%%%%%%%%%%%%%%%
\appendix

\settowidth\MacroIndent{\rmfamily\scriptsize 000\ }

 \DocInput{childdoc.dtx}

\end{document}
%</driver>
% \fi
%
% %%%%%%%%%%%%%%%%%%%%%%%%%%%%%%%%%%%%%%%%%%%%%%%%%%%%%%%%%%%%%%%%%%%%%%%%%%%%%%
% %%%%%%%%%%%%%%%%%%%%%%%%%%%%%%%%%%%%%%%%%%%%%%%%%%%%%%%%%%%%%%%%%%%%%%%%%%%%%%
% \section{Sample}
%\iffalse
%<*samplemain>
%\fi
%
% The following presents a sample document
% with two chapters, two parts, a title page,
% a compile flag as well as three forwarding files to set the flag.
% It consists of eight |.tex| files:
% \begin{center}
% \begin{tabular}{ll}
% |cdocsamp.tex|&main file\\
% |cdocsch1.tex|&include file for chapter 1\\
% |cdocsch2.tex|&include file for chapter 2\\
% |cdocspt3.tex|&include file for part 3\\
% |cdocspt4.tex|&include file for part 4\\
% |cdocsdrf.tex|&forwarding file for main file in draft mode\\
% |cdocsfi1.tex|&forwarding file for final version of chapter 1\\
% |cdocsfi2.tex|&forwarding file for final version of chapter 2\\
% \end{tabular}
% \end{center}
% Each of the eight files can be compiled directly by the \LaTeX{} compiler.
%
% %%%%%%%%%%%%%%%%%%%%%%%%%%%%%%%%%%%%%%
% \paragraph{Main File.}
%
% The main file is called |cdocsamp.tex|.
%
% Load the \textsf{childdoc} definitions and
% declare the filename for the main document:
%    \begin{macrocode}
\input{childdoc.def}
\childdocmain{}
%    \end{macrocode}

% Optional override for |\version| flag:
%    \begin{macrocode}
%%\ifchilddoc\else\providecommand{\version}{draft}\fi
%    \end{macrocode}

% Define the default values for the |\version| flag
% (|final| for the main file and |draft| for childs):
%    \begin{macrocode}
\ifchilddoc
\providecommand{\version}{draft}
\else
\providecommand{\version}{final}
\fi
%    \end{macrocode}

% Load the standard document class:
%    \begin{macrocode}
\documentclass[12pt]{article}
%    \end{macrocode}

% Start the document body:
%    \begin{macrocode}
\begin{document}
%    \end{macrocode}

% Declare a title page.
% Print title, part of document being processed and version flag:
%    \begin{macrocode}
\addtocounter{page}{-1}
\begin{center}
{\LARGE\bfseries{}childdoc example\par}
\vspace{1cm}
\ifchilddoc
\ifchilddocmanual part\else chapter\fi:
`\childdocname' of `\childdocjob'\par
\else
main document: `\childdocjob'\par
\fi
version: \version\par
\end{center}
\newpage
%    \end{macrocode}

% Manually include selected file,
% otherwise process as usual:
%    \begin{macrocode}
\ifchilddocmanual
\section*{part `\childdocname'}
\input{\childdocname}
\else
%    \end{macrocode}

% Include the two chapters:
%    \begin{macrocode}
\include{cdocsch1}
\include{cdocsch2}
%    \end{macrocode}

% Include the two parts unless only chapters should be displayed:
%    \begin{macrocode}
\ifchilddoc\else
\section{part three}
\input{cdocspt3}
\section{part four}
\input{cdocspt4}
\fi
%    \end{macrocode}

% Process as usual until here:
%    \begin{macrocode}
\fi
%    \end{macrocode}

% End of document body:
%    \begin{macrocode}
\end{document}
%    \end{macrocode}
%\iffalse
%</samplemain>
%\fi
%
% %%%%%%%%%%%%%%%%%%%%%%%%%%%%%%%%%%%%%%
% \paragraph{Chapter Include Files.}
%
% The include files are called |cdocsch1.tex| and |cdocsch2.tex|.
%
%\iffalse
%<*samplechap1|samplechap2>
%\fi

% Optional override for |\version| flag:
%    \begin{macrocode}
%%\providecommand{\version}{final}
%    \end{macrocode}

% Include the main document:
%    \begin{macrocode}
\input{childdoc.def}
\childdocof{cdocsamp}
%    \end{macrocode}

%\iffalse
%</samplechap1|samplechap2>
%\fi
%
%\iffalse
%<*samplechap1>
%\fi
% Some text for chapter 1:
%    \begin{macrocode}
\section{one}
some text in chapter one
%    \end{macrocode}

%\iffalse
%</samplechap1>
%\fi
% Some text for chapter 2:
%\iffalse
%<*samplechap2>
%\fi
%    \begin{macrocode}
\section{two}
more text in chapter two
%    \end{macrocode}

%\iffalse
%</samplechap2>
%\fi
%
% %%%%%%%%%%%%%%%%%%%%%%%%%%%%%%%%%%%%%%
% \paragraph{Part Include Files.}
%
% The include files are called |cdocspt3.tex| and |cdocspt4.tex|.
%
%\iffalse
%<*samplepart3|samplepart4>
%\fi

% Optional override for |\version| flag:
%    \begin{macrocode}
%%\providecommand{\version}{final}
%    \end{macrocode}

% Include the main document:
%    \begin{macrocode}
\input{childdoc.def}
\childdocby{cdocsamp}
%    \end{macrocode}

%\iffalse
%</samplepart3|samplepart4>
%\fi
%
%\iffalse
%<*samplepart3>
%\fi
% Some text for part 3:
%    \begin{macrocode}
some text in part three
%    \end{macrocode}

%\iffalse
%</samplepart3>
%\fi
% Some text for part 4:
%\iffalse
%<*samplepart4>
%\fi
%    \begin{macrocode}
more text in part four
%    \end{macrocode}

%\iffalse
%</samplepart4>
%\fi
%
% %%%%%%%%%%%%%%%%%%%%%%%%%%%%%%%%%%%%%%
% \paragraph{Forwarding for a Complete Draft.}
%
% The following forwarding file |cdocsdrf.tex|
% compiles the main document in draft mode:
%\iffalse
%<*sampledraft>
%\fi
%    \begin{macrocode}
\def\version{draft}
\input{childdoc.def}
\childdocforward{cdocsamp}
%    \end{macrocode}

%\iffalse
%</sampledraft>
%\fi
%
% %%%%%%%%%%%%%%%%%%%%%%%%%%%%%%%%%%%%%%
% \paragraph{Forwarding for Final Version of the Chapters.}
%
% The following forwarding files |cdocsfn1.tex| and |cdocsfn2.tex|
% (with identical content)
% compile the final versions of the child documents
% |cdocsch1.tex| and |cdocsch2.tex|, respectively:
%\iffalse
%<*samplefinal>
%\fi
%    \begin{macrocode}
\def\version{final}
\input{childdoc.def}
\childdocforwardprefix[cdocsamp]{cdocsfn}{cdocsch}
%    \end{macrocode}

%\iffalse
%</samplefinal>
%\fi
%
% %%%%%%%%%%%%%%%%%%%%%%%%%%%%%%%%%%%%%%
% \paragraph{Command Line Processing.}
%
% The following three command lines generate the output files
% |cdocscld|, |cdocscl1| and |cdocscl2|
% which should be identical to
% |cdocsdrf|, |cdocsch1| and |cdocsfn2|, respectively:
% \begin{center}
% \begin{tabular}{l}
% |latex -jobname cdocscld \|\\
% |  "\def\version{draft}\input{childdoc.def}\childdocforward{cdocsamp}"|\\
% |latex -jobname cdocscl1 \|\\
% |  "\input{childdoc.def}\childdocforward[cdocsamp]{cdocsch1}"|\\
% |latex -jobname cdocscl2 \|\\
% |  "\def\version{final}\input{childdoc.def}\childdocforward{cdocsch2}"|
% \end{tabular}
% \end{center}
% Note that the trailing backslash on each first line
% merely continues the input to the second line
% (for convenient cut ant paste).
% Furthermore, the command |latex| can be replaced by any
% of its alternative versions such as |pdflatex|.
%
% %%%%%%%%%%%%%%%%%%%%%%%%%%%%%%%%%%%%%%%%%%%%%%%%%%%%%%%%%%%%%%%%%%%%%%%%%%%%%%
% %%%%%%%%%%%%%%%%%%%%%%%%%%%%%%%%%%%%%%%%%%%%%%%%%%%%%%%%%%%%%%%%%%%%%%%%%%%%%%
% \section{Implementation}
%\iffalse
%<*package>
%\fi
%
% This section describes the definitions file |childdoc.def|.

% The definitions cannot be loaded using |\usepackage| or |\RequirePackage|
% which has a mechanism to prevent loading a style file more than once.
% When loading the definitions by means of |\input|
% multiple instances have to be prevented manually:
%\iffalse
%This code needs to be before the `\ProvidesFile' directive
%which is defined at the beginning of this file.
%Therefore it is also placed there and commented out here.
%</package>
%<*discard>
%\fi
%    \begin{macrocode}
\ifdefined\childdocmain\endinput\fi
%    \end{macrocode}
%\iffalse
%</discard>
%<*package>
%\fi
%
% \macro{\ifchilddoc}
% \macro{\ifchilddocmanual}
% The conditional |\ifchilddoc| tells whether a
% child (true) or main (false) document is being compiled.
% The conditional |\ifchilddocmanual| tells whether
% the |\includeonly| mechanism is used (false) or
% the selection of child files must be performed manually (true).
% The definitions initialise to false:
%    \begin{macrocode}
\newif\ifchilddoc
\newif\ifchilddocmanual
%    \end{macrocode}

% \macro{\childdocname}
% \macro{\childdocjob}
% The macro |\childdocname| stores the name of the main document
% to be compiled. The macro |\childdocjob| stores the name of
% the document on which the \LaTeX{} compiler was originally invoked.
% The content of |\jobname| cannot be compared
% to filenames specified in the source due to different catcodes.
% The following code rescans |\jobname|, stores the result
% in |\childdocname| and saves a copy in |\childdocjob|:
%    \begin{macrocode}
\edef\childdocname{\scantokens\expandafter{\jobname\noexpand}}
\let\childdocjob\childdocname
%    \end{macrocode}

% \macro{\childdocdisable}
% The macro |\childdocdisable| prevents the main file
% from being processed more than once.
% At this stage, the main document command |\childdocmain|
% is assumed to be called once again where it should do nothing.
% Any subsequent call to it should prevent
% a secondary processing of the main document
% It overwrites the forwarding commands
% |\childdocof| and |\childdocforward|
% with empty macros to prevent further inclusions of the main document:
%    \begin{macrocode}
\newcommand{\childdocdisable}
{
  \renewcommand{\childdocmain}[1]{\renewcommand{\childdocmain}[1]{\endinput}}
  \renewcommand{\childdocof}[1]{}
  \renewcommand{\childdocby}[2][]{}
  \renewcommand{\childdocforward}[2][]{}
  \renewcommand{\childdocdisable}{}
}
%    \end{macrocode}

% \macro{\childdocmain}
% The macro |\childdocmain| is to be called at the top of the main file
% with nothing or the main filename (without extension) as argument.
% First, it breaks loops.
% If the argument is not empty and does not match |\childdocname|
% (which is set by the first inclusion of |childdoc.def|),
% |\ifchilddoc| is set to true, |\includeonly| is applied to the child file
% and |\jobname| is set to the main file
% (for proper handling of |.aux| files):
%    \begin{macrocode}
\newcommand{\childdocmain}[1]
{
  \childdocdisable\childdocmain{}
  \if?#1?\else
    \begingroup
      \def\childdoctmp{#1}
      \ifx\childdoctmp\childdocname
        \def\childdoctmp{}
      \else
        \def\childdoctmp
        {
          \childdoctrue
          \includeonly{\childdocname}
          \def\childdocjob{#1}
          \def\jobname{#1}
        }
      \fi
      \expandafter
    \endgroup
    \childdoctmp
  \fi
}
%    \end{macrocode}

% \macro{\childdocof}
% The command |\childdocof| redirects
% compilation to the main file |#1|.
%    \begin{macrocode}
\newcommand{\childdocof}[1]
{
  \childdocdisable
  \childdoctrue
  \includeonly{\childdocname}
  \def\jobname{#1}
  \def\childdocjob{#1}
  \input{#1}
}
%    \end{macrocode}

% \macro{\childdocby}
% The command |\childdocby| ....
%    \begin{macrocode}
\newcommand{\childdocby}[2][]
{
  \childdocdisable
  \childdoctrue
  \childdocmanualtrue
  \if?#1?\else
    \def\jobname{#2}
  \fi
  \def\childdocjob{#2}
  \input{#2}
  \endinput
}
%    \end{macrocode}

% \macro{\childdocforward}
% The command |\childdocforward| redirects
% compilation to the main file or
% (if the optional argument is given) a child file.
% Parameters are set as if the main file
% or a child file starting with |\childdocof| was compiled.
% Then compilation is handed over to the main file:
%    \begin{macrocode}
\newcommand{\childdocforward}[2][]
{
  \begingroup
    \if?#1?
      \def\childdoctmp
      {
        \def\childdocname{#2}
        \def\childdocjob{#2}
        \def\jobname{#2}
        \input{#2}
        \endinput
      }
    \else
      \def\childdoctmp
      {
        \childdocdisable
        \def\childdocname{#2}
        \childdoctrue
        \includeonly{#2}
        \def\childdocjob{#1}
        \def\jobname{#1}
        \input{#1}
        \endinput
      }
    \fi
    \expandafter
  \endgroup
  \childdoctmp
}
%    \end{macrocode}

% \macro{\childdocforwardprefix}
% The command |\childdocforwardprefix| redirects
% compilation to the main or a child file by means of a pattern.
% The prefix |#1| in the current filename is replaced by |#2|
% and the suffix of the current filename is kept
% (it is assumed that the filename does not contain the substring `|~~~|'
% which is used as a delimiter).
% Compilation is handed over to the new file by |\childdocforward|:
%    \begin{macrocode}
\newcommand{\childdocforwardprefix}[3][]
{
  \begingroup
    \def\childdocextract #2##1~~~{\def\childdoctmp{\childdocforward[#1]{#3##1}}}
    \expandafter\childdocextract\childdocname~~~
    \expandafter
  \endgroup
  \childdoctmp
}
%    \end{macrocode}

% \macro{\childdoc}
% The deprecated macro |\childdoc| is a legacy version of |\childdocmain|:
%    \begin{macrocode}
\newcommand{\childdoc}{\childdocmain}
%    \end{macrocode}

% \macro{\childdocredirect}
% The deprecated macro |\childdocredirect| is a legacy version
% of |\childdocforward| and |\childdocforwardprefix|:
%    \begin{macrocode}
\newcommand{\childdocredirect}[2][]
{
  \begingroup
    \if?#1?
      \def\childdoctmp{\childdocforward{#2}}
    \else
      \def\childdoctmp{\childdocforwardprefix{#1}{#2}}
    \fi
    \expandafter
  \endgroup
  \childdoctmp
}
%    \end{macrocode}

%\iffalse
%</package>
%\fi
%
\endinput
\childdocforward{cdocsch2}"|
% \end{tabular}
% \end{center}
% Note that the trailing backslash on each first line
% merely continues the input to the second line
% (for convenient cut ant paste).
% Furthermore, the command |latex| can be replaced by any
% of its alternative versions such as |pdflatex|.
%
% %%%%%%%%%%%%%%%%%%%%%%%%%%%%%%%%%%%%%%%%%%%%%%%%%%%%%%%%%%%%%%%%%%%%%%%%%%%%%%
% %%%%%%%%%%%%%%%%%%%%%%%%%%%%%%%%%%%%%%%%%%%%%%%%%%%%%%%%%%%%%%%%%%%%%%%%%%%%%%
% \section{Implementation}
%\iffalse
%<*package>
%\fi
%
% This section describes the definitions file |childdoc.def|.

% The definitions cannot be loaded using |\usepackage| or |\RequirePackage|
% which has a mechanism to prevent loading a style file more than once.
% When loading the definitions by means of |\input|
% multiple instances have to be prevented manually:
%\iffalse
%This code needs to be before the `\ProvidesFile' directive
%which is defined at the beginning of this file.
%Therefore it is also placed there and commented out here.
%</package>
%<*discard>
%\fi
%    \begin{macrocode}
\ifdefined\childdocmain\endinput\fi
%    \end{macrocode}
%\iffalse
%</discard>
%<*package>
%\fi
%
% \macro{\ifchilddoc}
% \macro{\ifchilddocmanual}
% The conditional |\ifchilddoc| tells whether a
% child (true) or main (false) document is being compiled.
% The conditional |\ifchilddocmanual| tells whether
% the |\includeonly| mechanism is used (false) or
% the selection of child files must be performed manually (true).
% The definitions initialise to false:
%    \begin{macrocode}
\newif\ifchilddoc
\newif\ifchilddocmanual
%    \end{macrocode}

% \macro{\childdocname}
% \macro{\childdocjob}
% The macro |\childdocname| stores the name of the main document
% to be compiled. The macro |\childdocjob| stores the name of
% the document on which the \LaTeX{} compiler was originally invoked.
% The content of |\jobname| cannot be compared
% to filenames specified in the source due to different catcodes.
% The following code rescans |\jobname|, stores the result
% in |\childdocname| and saves a copy in |\childdocjob|:
%    \begin{macrocode}
\edef\childdocname{\scantokens\expandafter{\jobname\noexpand}}
\let\childdocjob\childdocname
%    \end{macrocode}

% \macro{\childdocdisable}
% The macro |\childdocdisable| prevents the main file
% from being processed more than once.
% At this stage, the main document command |\childdocmain|
% is assumed to be called once again where it should do nothing.
% Any subsequent call to it should prevent
% a secondary processing of the main document
% It overwrites the forwarding commands
% |\childdocof| and |\childdocforward|
% with empty macros to prevent further inclusions of the main document:
%    \begin{macrocode}
\newcommand{\childdocdisable}
{
  \renewcommand{\childdocmain}[1]{\renewcommand{\childdocmain}[1]{\endinput}}
  \renewcommand{\childdocof}[1]{}
  \renewcommand{\childdocby}[2][]{}
  \renewcommand{\childdocforward}[2][]{}
  \renewcommand{\childdocdisable}{}
}
%    \end{macrocode}

% \macro{\childdocmain}
% The macro |\childdocmain| is to be called at the top of the main file
% with nothing or the main filename (without extension) as argument.
% First, it breaks loops.
% If the argument is not empty and does not match |\childdocname|
% (which is set by the first inclusion of |childdoc.def|),
% |\ifchilddoc| is set to true, |\includeonly| is applied to the child file
% and |\jobname| is set to the main file
% (for proper handling of |.aux| files):
%    \begin{macrocode}
\newcommand{\childdocmain}[1]
{
  \childdocdisable\childdocmain{}
  \if?#1?\else
    \begingroup
      \def\childdoctmp{#1}
      \ifx\childdoctmp\childdocname
        \def\childdoctmp{}
      \else
        \def\childdoctmp
        {
          \childdoctrue
          \includeonly{\childdocname}
          \def\childdocjob{#1}
          \def\jobname{#1}
        }
      \fi
      \expandafter
    \endgroup
    \childdoctmp
  \fi
}
%    \end{macrocode}

% \macro{\childdocof}
% The command |\childdocof| redirects
% compilation to the main file |#1|.
%    \begin{macrocode}
\newcommand{\childdocof}[1]
{
  \childdocdisable
  \childdoctrue
  \includeonly{\childdocname}
  \def\jobname{#1}
  \def\childdocjob{#1}
  \input{#1}
}
%    \end{macrocode}

% \macro{\childdocby}
% The command |\childdocby| ....
%    \begin{macrocode}
\newcommand{\childdocby}[2][]
{
  \childdocdisable
  \childdoctrue
  \childdocmanualtrue
  \if?#1?\else
    \def\jobname{#2}
  \fi
  \def\childdocjob{#2}
  \input{#2}
  \endinput
}
%    \end{macrocode}

% \macro{\childdocforward}
% The command |\childdocforward| redirects
% compilation to the main file or
% (if the optional argument is given) a child file.
% Parameters are set as if the main file
% or a child file starting with |\childdocof| was compiled.
% Then compilation is handed over to the main file:
%    \begin{macrocode}
\newcommand{\childdocforward}[2][]
{
  \begingroup
    \if?#1?
      \def\childdoctmp
      {
        \def\childdocname{#2}
        \def\childdocjob{#2}
        \def\jobname{#2}
        \input{#2}
        \endinput
      }
    \else
      \def\childdoctmp
      {
        \childdocdisable
        \def\childdocname{#2}
        \childdoctrue
        \includeonly{#2}
        \def\childdocjob{#1}
        \def\jobname{#1}
        \input{#1}
        \endinput
      }
    \fi
    \expandafter
  \endgroup
  \childdoctmp
}
%    \end{macrocode}

% \macro{\childdocforwardprefix}
% The command |\childdocforwardprefix| redirects
% compilation to the main or a child file by means of a pattern.
% The prefix |#1| in the current filename is replaced by |#2|
% and the suffix of the current filename is kept
% (it is assumed that the filename does not contain the substring `|~~~|'
% which is used as a delimiter).
% Compilation is handed over to the new file by |\childdocforward|:
%    \begin{macrocode}
\newcommand{\childdocforwardprefix}[3][]
{
  \begingroup
    \def\childdocextract #2##1~~~{\def\childdoctmp{\childdocforward[#1]{#3##1}}}
    \expandafter\childdocextract\childdocname~~~
    \expandafter
  \endgroup
  \childdoctmp
}
%    \end{macrocode}

% \macro{\childdoc}
% The deprecated macro |\childdoc| is a legacy version of |\childdocmain|:
%    \begin{macrocode}
\newcommand{\childdoc}{\childdocmain}
%    \end{macrocode}

% \macro{\childdocredirect}
% The deprecated macro |\childdocredirect| is a legacy version
% of |\childdocforward| and |\childdocforwardprefix|:
%    \begin{macrocode}
\newcommand{\childdocredirect}[2][]
{
  \begingroup
    \if?#1?
      \def\childdoctmp{\childdocforward{#2}}
    \else
      \def\childdoctmp{\childdocforwardprefix{#1}{#2}}
    \fi
    \expandafter
  \endgroup
  \childdoctmp
}
%    \end{macrocode}

%\iffalse
%</package>
%\fi
%
\endinput
\childdocforward[cdocsamp]{cdocsch1}"|\\
% |latex -jobname cdocscl2 \|\\
% |  "\def\version{final}% \iffalse
%
% childdoc.dtx Copyright (C) 2017-2018 Niklas Beisert
%
% This work may be distributed and/or modified under the
% conditions of the LaTeX Project Public License, either version 1.3
% of this license or (at your option) any later version.
% The latest version of this license is in
%   http://www.latex-project.org/lppl.txt
% and version 1.3 or later is part of all distributions of LaTeX
% version 2005/12/01 or later.
%
% This work has the LPPL maintenance status `maintained'.
%
% The Current Maintainer of this work is Niklas Beisert.
%
% This work consists of the files childdoc.dtx and childdoc.ins
% and the derived files childdoc.def and cdocsamp.tex with
% cdocsch1.tex, cdocsch2.tex, cdocsdrf.tex, cdocsfn1.tex, cdocsfn2.tex.
%
%<package>\ifdefined\childdocmain\endinput\fi
%<package>\ProvidesFile{childdoc.def}[2018/12/30 v2.0 child document driver]
%<samplemain>\ProvidesFile{cdocsamp.tex}[2018/12/30 v2.0 sample for childdoc]
%<*driver>
%\ProvidesFile{childdoc.drv}[2018/12/30 v2.0 childdoc reference manual file]
\PassOptionsToClass{10pt,a4paper}{article}
\documentclass{ltxdoc}

\usepackage[margin=35mm]{geometry}
\usepackage{hyperref}
\usepackage{hyperxmp}
\usepackage[usenames]{color}

\hypersetup{colorlinks=true}
\hypersetup{pdfstartview=FitH}
\hypersetup{pdfpagemode=UseNone}
\hypersetup{pdfsource={}}
\hypersetup{pdflang={en-UK}}
\hypersetup{pdfcopyright={Copyright 2017-2018 Niklas Beisert.
  This work may be distributed and/or modified under the
  conditions of the LaTeX Project Public License, either version 1.3
  of this license or (at your option) any later version.}}
\hypersetup{pdflicenseurl={http://www.latex-project.org/lppl.txt}}
\hypersetup{pdfcontactaddress={ETH Zurich, ITP, HIT K,
  Wolfgang-Pauli-Strasse 27}}
\hypersetup{pdfcontactpostcode={8093}}
\hypersetup{pdfcontactcity={Zurich}}
\hypersetup{pdfcontactcountry={Switzerland}}
\hypersetup{pdfcontactemail={nbeisert@itp.phys.ethz.ch}}
\hypersetup{pdfcontacturl={http://people.phys.ethz.ch/\xmptilde nbeisert/}}

\newcommand{\secref}[1]{\hyperref[#1]{section \ref*{#1}}}

\parskip1ex
\parindent0pt
\let\olditemize\itemize
\def\itemize{\olditemize\parskip0pt}

\begin{document}

\title{The \textsf{childdoc} Package}
\hypersetup{pdftitle={The childdoc Package}}
\author{Niklas Beisert\\[2ex]
  Institut f\"ur Theoretische Physik\\
  Eidgen\"ossische Technische Hochschule Z\"urich\\
  Wolfgang-Pauli-Strasse 27, 8093 Z\"urich, Switzerland\\[1ex]
  \href{mailto:nbeisert@itp.phys.ethz.ch}
  {\texttt{nbeisert@itp.phys.ethz.ch}}}
\hypersetup{pdfauthor={Niklas Beisert}}
\hypersetup{pdfsubject={Manual for the LaTeX2e Package childdoc}}
\date{30 December 2018, \textsf{v2.0}}
\maketitle

\begin{abstract}\noindent
\textsf{childdoc} is a \LaTeXe{} package
that enables the direct compilation
of document sections included by |\include|
to individual files.
\end{abstract}

\begingroup
\parskip0ex
\tableofcontents
\endgroup

%%%%%%%%%%%%%%%%%%%%%%%%%%%%%%%%%%%%%%%%%%%%%%%%%%%%%%%%%%%%%%%%%%%%%%%%%%%%%%%%
%%%%%%%%%%%%%%%%%%%%%%%%%%%%%%%%%%%%%%%%%%%%%%%%%%%%%%%%%%%%%%%%%%%%%%%%%%%%%%%%
\section{Introduction}

\LaTeX{} provides a mechanism to structure a large document (such as a book)
into a main file and several child files (containing the chapters)
using the |\include| command.
This mechanism is beneficial for documents
which span hundreds of pages in order to
make the source file(s) more manageable.
Moreover, compilation can be restricted to
selected child files by means of the |\includeonly| command.
The latter feature can be used to reduce the compilation time while editing
(this was significantly more useful in the earlier days of \LaTeX{})
or to generate a smaller document which is easier to navigate.
Another application of |\includeonly| is to generate
documents consisting of selected parts of the complete document.

However, there are a few drawbacks of the plain |\include| mechanism:
\begin{itemize}
\item
The child files cannot be compiled on their own,
they can only be compiled via the main file.
A naive editing environment
(such as a text editor with an option
to have the current file processed by \LaTeX)
may require one to switch to the main file before compiling;
attempting to compile the child file produces errors.
\item
The main file must be modified (each time)
to adjust the |\includeonly| command
to the present needs. This easily leaves the main file in a messy state.
\item
The generated document will always carry the filename
of the main document. This is inconvenient if
several child files are to be compiled and
to be kept for distribution.
\end{itemize}

The present package provides a simple interface
to make child files individually compilable by \LaTeX{}.
Compiling a child file then has the same effect as compiling
the main file with an |\includeonly| command
to select the appropriate child.
Moreover the generated document will carry the name of the child
rather than the main file.
This resolves all three above issues.

This feature is meant to make the editing of books,
thesis documents and lecture notes somewhat more convenient.
However, the package can also be used efficiently for
composing a series of documents (such as exercise sheets)
which are typically distributed individually.
It then assists the author in generating the individual documents
(potentially in different versions)
as well as a document containing the collected series.
Another application is in developing style files
or other kinds of included material
where compilation of the style file could redirect
to a sample or test file.

%%%%%%%%%%%%%%%%%%%%%%%%%%%%%%%%%%%%%%%%%%%%%%%%%%%%%%%%%%%%%%%%%%%%%%%%%%%%%%%%
%%%%%%%%%%%%%%%%%%%%%%%%%%%%%%%%%%%%%%%%%%%%%%%%%%%%%%%%%%%%%%%%%%%%%%%%%%%%%%%%
\section{Usage}

First of all, the package \textsf{childdoc} is \emph{not} a standard
\LaTeXe{} |.sty| style file! Therefore it needs to be invoked in
a non-standard way.

%%%%%%%%%%%%%%%%%%%%%%%%%%%%%%%%%%%%%%%%%%%%%%%%%%%%%%%%%%%%%%%%%%%%%%%%%%%%%%%%
\subsection{Included Files}
\label{sec:include}

%%%%%%%%%%%%%%%%%%%%%%%%%%%%%%%%%%%%%%%%
\DescribeMacro{\childdocmain}
To use the package, add the commands
\begin{center}
\begin{tabular}{l}
|% \iffalse
%
% childdoc.dtx Copyright (C) 2017-2018 Niklas Beisert
%
% This work may be distributed and/or modified under the
% conditions of the LaTeX Project Public License, either version 1.3
% of this license or (at your option) any later version.
% The latest version of this license is in
%   http://www.latex-project.org/lppl.txt
% and version 1.3 or later is part of all distributions of LaTeX
% version 2005/12/01 or later.
%
% This work has the LPPL maintenance status `maintained'.
%
% The Current Maintainer of this work is Niklas Beisert.
%
% This work consists of the files childdoc.dtx and childdoc.ins
% and the derived files childdoc.def and cdocsamp.tex with
% cdocsch1.tex, cdocsch2.tex, cdocsdrf.tex, cdocsfn1.tex, cdocsfn2.tex.
%
%<package>\ifdefined\childdocmain\endinput\fi
%<package>\ProvidesFile{childdoc.def}[2018/12/30 v2.0 child document driver]
%<samplemain>\ProvidesFile{cdocsamp.tex}[2018/12/30 v2.0 sample for childdoc]
%<*driver>
%\ProvidesFile{childdoc.drv}[2018/12/30 v2.0 childdoc reference manual file]
\PassOptionsToClass{10pt,a4paper}{article}
\documentclass{ltxdoc}

\usepackage[margin=35mm]{geometry}
\usepackage{hyperref}
\usepackage{hyperxmp}
\usepackage[usenames]{color}

\hypersetup{colorlinks=true}
\hypersetup{pdfstartview=FitH}
\hypersetup{pdfpagemode=UseNone}
\hypersetup{pdfsource={}}
\hypersetup{pdflang={en-UK}}
\hypersetup{pdfcopyright={Copyright 2017-2018 Niklas Beisert.
  This work may be distributed and/or modified under the
  conditions of the LaTeX Project Public License, either version 1.3
  of this license or (at your option) any later version.}}
\hypersetup{pdflicenseurl={http://www.latex-project.org/lppl.txt}}
\hypersetup{pdfcontactaddress={ETH Zurich, ITP, HIT K,
  Wolfgang-Pauli-Strasse 27}}
\hypersetup{pdfcontactpostcode={8093}}
\hypersetup{pdfcontactcity={Zurich}}
\hypersetup{pdfcontactcountry={Switzerland}}
\hypersetup{pdfcontactemail={nbeisert@itp.phys.ethz.ch}}
\hypersetup{pdfcontacturl={http://people.phys.ethz.ch/\xmptilde nbeisert/}}

\newcommand{\secref}[1]{\hyperref[#1]{section \ref*{#1}}}

\parskip1ex
\parindent0pt
\let\olditemize\itemize
\def\itemize{\olditemize\parskip0pt}

\begin{document}

\title{The \textsf{childdoc} Package}
\hypersetup{pdftitle={The childdoc Package}}
\author{Niklas Beisert\\[2ex]
  Institut f\"ur Theoretische Physik\\
  Eidgen\"ossische Technische Hochschule Z\"urich\\
  Wolfgang-Pauli-Strasse 27, 8093 Z\"urich, Switzerland\\[1ex]
  \href{mailto:nbeisert@itp.phys.ethz.ch}
  {\texttt{nbeisert@itp.phys.ethz.ch}}}
\hypersetup{pdfauthor={Niklas Beisert}}
\hypersetup{pdfsubject={Manual for the LaTeX2e Package childdoc}}
\date{30 December 2018, \textsf{v2.0}}
\maketitle

\begin{abstract}\noindent
\textsf{childdoc} is a \LaTeXe{} package
that enables the direct compilation
of document sections included by |\include|
to individual files.
\end{abstract}

\begingroup
\parskip0ex
\tableofcontents
\endgroup

%%%%%%%%%%%%%%%%%%%%%%%%%%%%%%%%%%%%%%%%%%%%%%%%%%%%%%%%%%%%%%%%%%%%%%%%%%%%%%%%
%%%%%%%%%%%%%%%%%%%%%%%%%%%%%%%%%%%%%%%%%%%%%%%%%%%%%%%%%%%%%%%%%%%%%%%%%%%%%%%%
\section{Introduction}

\LaTeX{} provides a mechanism to structure a large document (such as a book)
into a main file and several child files (containing the chapters)
using the |\include| command.
This mechanism is beneficial for documents
which span hundreds of pages in order to
make the source file(s) more manageable.
Moreover, compilation can be restricted to
selected child files by means of the |\includeonly| command.
The latter feature can be used to reduce the compilation time while editing
(this was significantly more useful in the earlier days of \LaTeX{})
or to generate a smaller document which is easier to navigate.
Another application of |\includeonly| is to generate
documents consisting of selected parts of the complete document.

However, there are a few drawbacks of the plain |\include| mechanism:
\begin{itemize}
\item
The child files cannot be compiled on their own,
they can only be compiled via the main file.
A naive editing environment
(such as a text editor with an option
to have the current file processed by \LaTeX)
may require one to switch to the main file before compiling;
attempting to compile the child file produces errors.
\item
The main file must be modified (each time)
to adjust the |\includeonly| command
to the present needs. This easily leaves the main file in a messy state.
\item
The generated document will always carry the filename
of the main document. This is inconvenient if
several child files are to be compiled and
to be kept for distribution.
\end{itemize}

The present package provides a simple interface
to make child files individually compilable by \LaTeX{}.
Compiling a child file then has the same effect as compiling
the main file with an |\includeonly| command
to select the appropriate child.
Moreover the generated document will carry the name of the child
rather than the main file.
This resolves all three above issues.

This feature is meant to make the editing of books,
thesis documents and lecture notes somewhat more convenient.
However, the package can also be used efficiently for
composing a series of documents (such as exercise sheets)
which are typically distributed individually.
It then assists the author in generating the individual documents
(potentially in different versions)
as well as a document containing the collected series.
Another application is in developing style files
or other kinds of included material
where compilation of the style file could redirect
to a sample or test file.

%%%%%%%%%%%%%%%%%%%%%%%%%%%%%%%%%%%%%%%%%%%%%%%%%%%%%%%%%%%%%%%%%%%%%%%%%%%%%%%%
%%%%%%%%%%%%%%%%%%%%%%%%%%%%%%%%%%%%%%%%%%%%%%%%%%%%%%%%%%%%%%%%%%%%%%%%%%%%%%%%
\section{Usage}

First of all, the package \textsf{childdoc} is \emph{not} a standard
\LaTeXe{} |.sty| style file! Therefore it needs to be invoked in
a non-standard way.

%%%%%%%%%%%%%%%%%%%%%%%%%%%%%%%%%%%%%%%%%%%%%%%%%%%%%%%%%%%%%%%%%%%%%%%%%%%%%%%%
\subsection{Included Files}
\label{sec:include}

%%%%%%%%%%%%%%%%%%%%%%%%%%%%%%%%%%%%%%%%
\DescribeMacro{\childdocmain}
To use the package, add the commands
\begin{center}
\begin{tabular}{l}
|\input{childdoc.def}|\\
|\childdocmain{}|\\
\end{tabular}
\end{center}
at the very top of the main \LaTeX{} file,
in particular \emph{before} the |\documentclass| statement!
The argument of |\childdocmain| should be left empty
(but it must be present).

%%%%%%%%%%%%%%%%%%%%%%%%%%%%%%%%%%%%%%%%
\DescribeMacro{\childdocof}
Furthermore, add the commands
\begin{center}
\begin{tabular}{l}
|\input{childdoc.def}|\\
|\childdocof{|\textit{main}|}|\\
\end{tabular}
\end{center}
at the top of every child file \textit{child}
which is included by |\include{|\textit{child}|}|
from within the main file
(or at least for those files to be compiled individually).
The argument \textit{main} must be the filename of the main file.

There are a couple of
considerations in setting up the main and child documents:

%%%%%%%%%%%%%%%%%%%%%%%%%%%%%%%%%%%%%%%%
\paragraph{Restrictions.}

Please note the following restrictions:
\begin{itemize}
\item
|\childdocmain| must be called with one argument \textit{main}
to ensure compatibility with earlier version of the package.
It must either be empty (|\childdocmain{}|)
or precisely match the filename of the main file in which it is specified.
See \secref{sec:detection} for further information.
\item
The filename \textit{main} must be specified without the |.tex| extension.
\item
The filename \textit{main} is case sensitive
(even in case-insensitive file systems)
due to internal string comparison.
\item
The argument \textit{main} should be fully expanded, it cannot be a macro.
\item
Subdirectories and special characters should be avoided in filenames.
\item
The command |\childdocmain{|\textit{main}|}| must be followed by a whitespace.
It should not be followed immediately by another command
or by a comment mark `|%|'.
This is because the \TeX{} parser reads the token immediately following
the argument of |\childdocmain| and puts it
at the beginning of every child section;
however, a white\-space is ignored.
\end{itemize}

%%%%%%%%%%%%%%%%%%%%%%%%%%%%%%%%%%%%%%%%
\paragraph{Content of Main File.}

It is advisable to place all content in the child files included by |\include|.
Any output contained in the main file will appear in all child documents
unless suppressed manually;
it cannot be suppressed automatically by the |\includeonly| directive
and thus should normally be avoided.
A method to include some content in the main file
by means of conditional processing is described in \secref{sec:conditional}.

%%%%%%%%%%%%%%%%%%%%%%%%%%%%%%%%%%%%%%%%
\paragraph{Page Numbering.}

When only a part of the document is compiled,
the appropriate numbering of pages
(as well as other status parameters)
is determined from the |.aux| files.
The latter contain information from previous passes.
However this information needs to propagate through
all intermediate child documents.
Therefore the page numbering in child documents may well
be inconsistent until the complete document is compiled at least once.

A useful (if unconventional) way to always ensure a consistent
page numbering is to restart the numbering in each child document
and denote the pages by `\textit{child}|.|\textit{page}'
where \textit{child} represents the chapter/section number of the child file.
This can be achieved by the command
|\numberwithin{page}{|\textit{child}|}|
of the \textsf{amsmath} package
where \textit{child} can be |chapter| or |section|
depending on the chosen structuring.
Alternatively, one can modify the macro |\thepage| appropriately
and reset the counter |page| at the start of each child file.

%%%%%%%%%%%%%%%%%%%%%%%%%%%%%%%%%%%%%%%%%%%%%%%%%%%%%%%%%%%%%%%%%%%%%%%%%%%%%%%%
\subsection{Conditional Processing}
\label{sec:conditional}

The package provides a mechanism to compile different versions
of a document. To customise the versions further some conditional processing
can come in handy to distinguish which version is being compiled.
The package provides two macros to describe the compilation context:

%%%%%%%%%%%%%%%%%%%%%%%%%%%%%%%%%%%%%%%%
\DescribeMacro{\ifchilddoc}
The conditional |\ifchilddoc| distinguishes between the compilation of
child documents and the main document:
%
\begin{center}
|\ifchilddoc |\textit{child-code}| |[|\||else |\textit{main-code}]| \||fi|
\end{center}

%%%%%%%%%%%%%%%%%%%%%%%%%%%%%%%%%%%%%%%%
\DescribeMacro{\childdocname}
\DescribeMacro{\childdocjob}
The macro |\childdocname| contains the filename (without extension)
of the main or child file being processed.
Note that |\childdocjob| will always contain the name of the main file.

%%%%%%%%%%%%%%%%%%%%%%%%%%%%%%%%%%%%%%%%
\paragraph{Title Page.}

Conditional processing can be used to include a title or banner page
in the main document when proper precautions are taken.
Importantly, the code in the main file should ensure that the page counter
(as well as other status parameters which are stored in the |.aux| files)
takes the same value after the conditional processing.
Otherwise the page numbers may take divergent values
depending on which part is compiled.

For example, a title page could be declared by:
%
\begin{center}
\begin{tabular}{l}
|\ifchilddoc\||else|\\
|\addtocounter{page}{-1}|\\
\textit{code for title page}\\
|\newpage|\\
|\||fi|
\end{tabular}
\end{center}
%
A banner page for the child documents can be generated by:
%
\begin{center}
\begin{tabular}{l}
|\ifchilddoc|\\
|\addtocounter{page}{-1}|\\
\textit{code for banner page}\\
|\newpage|\\
|\||fi|
\end{tabular}
\end{center}
%
Here one could write a message such as:
\begin{center}
|This is the part \childdocname{} of \childdocjob{}.|
\end{center}

%%%%%%%%%%%%%%%%%%%%%%%%%%%%%%%%%%%%%%%%%%%%%%%%%%%%%%%%%%%%%%%%%%%%%%%%%%%%%%%%
\subsection{Flags}
\label{sec:flags}

The package makes it easy to generate different versions
of the main or child documents.
To this end compilation flags can be defined
and assigned different default values.
They will be particularly useful in conjunction
with the forwarding mechanism described in \secref{sec:forward}.

For example, it may be useful to have a flag |\version|
which can be set to |draft| or |final|.
The document source will contain some conditional code
depending on the value of |\version|.
Suppose further, the flag should default to |final| for the main file
and to |draft| for child files
which is a natural assignment for editing the document.
This is achieved by placing the following code
in the preamble of the main document
(below the |\childdocmain| directive):
%
\begin{center}
\begin{tabular}{l}
|\ifchilddoc|\\
|\providecommand{\version}{draft}|\\
|\||else|\\
|\providecommand{\version}{final}|\\
|\||fi|
\end{tabular}
\end{center}
%
The definition by |\providecommand| makes sure
that previous definitions are not overwritten.
Further statements |\providecommand{\version}{...}|
can thus be added before the above code to override it.

For the main file, one might add a line
(between |\childdocmain| and the above block)
%
\begin{center}
|%\ifchilddoc\||else\providecommand{\version}{draft}\||fi|
\end{center}
%
which can be uncommented to produce a draft version.
Likewise one can add a line to the very top of a child file
(above the |\childdocof{|\textit{main}|}| directive)
%
\begin{center}
|%\providecommand{\version}{final}|
\end{center}
%
which can be uncommented to produce the final version of this child document.

%%%%%%%%%%%%%%%%%%%%%%%%%%%%%%%%%%%%%%%%%%%%%%%%%%%%%%%%%%%%%%%%%%%%%%%%%%%%%%%%
\subsection{Forwarding}
\label{sec:forward}

Different versions of the main or child documents
using compilation flags as described in \secref{sec:flags}
can be (permanently) stored in different files
for convenient compilation, viewing and distribution.
To this end, the package defines a command
to pass on compilation to a different file:

%%%%%%%%%%%%%%%%%%%%%%%%%%%%%%%%%%%%%%%%
\DescribeMacro{\childdocforward}
The command |\childdocforward| redirects processing to
another source file:
%
\begin{center}
\begin{tabular}{l}
|\input{childdoc.def}|\\
|\childdocforward[|\textit{main}|]{|\textit{dest}|}|\\
\end{tabular}
\end{center}
%
The argument \textit{dest} is the destination file
(without extension).
It should be the main file or one of the child files.
Note that further \textsf{childdoc} directives
such as |\childdocof| and |\childdocforward|
in the indicated file will be processed in this form.
The optional argument \textit{main}
passes on directly to the main file \textit{main}
while pretending to compile the child \textit{dest}.
This form behaves as if \textit{dest}
issues |\childdocof{|\textit{main}|}| right away,
and no further \textsf{childdoc} directives will be processed.

%%%%%%%%%%%%%%%%%%%%%%%%%%%%%%%%%%%%%%%%
\DescribeMacro{\...prefix}
In the alternative form |\childdocforwardprefix|,
%
\begin{center}
\begin{tabular}{l}
|\input{childdoc.def}|\\
|\childdocforwardprefix[|\textit{main}|]{|\textit{prefix}|}{|\textit{dest}|}|
\end{tabular}
\end{center}
%
the destination file is determined by a pattern
depending on the current file:
To make this work, the current file must be called
`{\textit{prefix}\hspace{0.2em}\textit{suffix}}'
with \textit{prefix} matching precisely the argument.
Processing is then passed on to the file
`{\textit{dest}\hspace{0.2em}\textit{suffix}}'.
Surely, the same effect is achieved by
directly specifying the
argument `{\textit{dest}\hspace{0.2em}\textit{suffix}}'
in the first form.
However, that requires to set up a different file
for each child. With the alternative form of the command
all these files can have exactly the same content
which simplifies setting them up and maintaining them.

For example, the following file |draft.tex|
with a compilation flag |\version| as described in \secref{sec:flags}
compiles the main document as a draft:
%
\begin{center}
\begin{tabular}{l}
|\def\version{draft}|\\
|\input{childdoc.def}|\\
|\childdocforward{|\textit{main}|}|
\end{tabular}
\end{center}
%
Likewise, the following files |final|\textit{nn}|.tex|
compile the final version of the child document
|child|\textit{nn}|.tex|:
%
\begin{center}
\begin{tabular}{l}
|\def\version{final}|\\
|\input{childdoc.def}|\\
|\childdocforwardprefix{final}{child}|
\end{tabular}
\end{center}
%

Note that when several versions of a main file and/or of each child file
are to be generated, it may be convenient to set up a |Makefile| or
shell script to automatise the process.

%%%%%%%%%%%%%%%%%%%%%%%%%%%%%%%%%%%%%%%%%%%%%%%%%%%%%%%%%%%%%%%%%%%%%%%%%%%%%%%%
\subsection{Command Line Processing}
\label{sec:commandline}

The effect of redirection files can also be achieved by invoking
the \LaTeX{} compiler with a more elaborate command line.
Most conveniently this should be done as part
of a shell script or a |Makefile|.

When using \textsf{childdoc} in the main file, the following
command lines effectively perform a redirection
(note that depending on the shell being used,
backslashes may have to be doubled: `|\|' $\to$ `|\\|'):
%
\begin{center}
|... -jobname "|\textit{target}|" |\\|"|[\textit{flags}]%
|\input{childdoc.def}\childdocforward[|\textit{main}|]{|\textit{dest}|}"|
\end{center}
%
Here \textit{target} is the name of the output file,
\textit{main} is the name of the main file
and \textit{dest} is the name of the main or child file to be processed
(all filenames without extensions).
The optional argument \textit{main} can be omitted
if \textit{main} matches \textit{dest}.
Optionally, compilation \textit{flags} can be defined via |\def| commands.
This command line makes the \TeX{} engine believe
it is compiling the file \textit{target}
whose content is specified as the latter parameter.
The provided code then forwards the processing to
\textit{main} or \textit{dest} as described in \secref{sec:forward}.

%%%%%%%%%%%%%%%%%%%%%%%%%%%%%%%%%%%%%%%%%%%%%%%%%%%%%%%%%%%%%%%%%%%%%%%%%%%%%%%%
\subsection{Include by Input}
\label{sec:input}

Including child documents by |\include| has some restrictions by design.
Most notably, the content of a child document always occupies
its own set of pages; pages cannot be shared between child documents.
Usually, this behaviour makes perfect sense
because each child document contain an essential part of the document.
However, in some situations it may be desirable to compose
a document from a collection of parts
without having mandatory page breaks between then.
For this case, the package
provides a mechanism to include parts
by |\input| which can also be processed individually.
However, by construction this mechanism
requires manual handling of the content to be output.

%%%%%%%%%%%%%%%%%%%%%%%%%%%%%%%%%%%%%%%%
\DescribeMacro{\ifchilddocmanual}
The main file should be prepared as usual, see \secref{sec:include}.
However, the document body must make a distinction
between processing of an individual part and of the main document, e.g.:
%
\begin{center}
\begin{tabular}{l}
|\ifchilddocmanual|\\
|\input{\childdocname}|\\
|\||else|\\
\textit{document body with }|\input{|\textit{part}|}|\\
|\||fi|
\end{tabular}
\end{center}
%
The conditional |\ifchilddocmanual| is true whenever
a part to be included by |\input| is being compiled,
and the name of the part is stored in |\childdocname|.

%%%%%%%%%%%%%%%%%%%%%%%%%%%%%%%%%%%%%%%%
\DescribeMacro{\childdocby}
Each part to be included by |\input| should start with:
%
\begin{center}
\begin{tabular}{l}
|\input{childdoc.def}|\\
|\childdocby{|\textit{main}|}|\\
\end{tabular}
\end{center}
%
The directive |\childdocby| is similar to |\childdocof|
described in \secref{sec:include},
but the subsequent selection of content must be done manually.
To that end, both |\ifchilddoc| and |\ifchilddocmanual|
will be true upon processing of a part,
and the name of the part is stored in |\childdocname|.
Note that |\jobname| will be set to the filename of the current part
so that each part receives an individual |.aux| file
that does not interfere with the |.aux| file(s) of the main document.
This behaviour can be altered by the alternative form
|\childdocby[*]{|\textit{main}|}| (with a non-empty optional argument)
which uses the |.aux| file of the main document
by setting |\jobname| to \textit{main}.

%%%%%%%%%%%%%%%%%%%%%%%%%%%%%%%%%%%%%%%%%%%%%%%%%%%%%%%%%%%%%%%%%%%%%%%%%%%%%%%%
\subsection{Driver Development}
\label{sec:driver}

The \textsf{childdoc} mechanism can also be use for the development
of definition files such as \LaTeX{} styles or classes.
This case differs from the above setup with multiple parts
included by |\include| in that no |\includeonly| should be invoked.
This can be achieved by starting the include file
(before |\ProvidesPackage|) with:
%
\begin{center}
\begin{tabular}{l}
|\input{childdoc.def}|\\
|\childdocforward{|\textit{main}|}|\\
\end{tabular}
\end{center}
%
or alternatively with:
%
\begin{center}
\begin{tabular}{l}
|\input{childdoc.def}|\\
|\childdocby{|\textit{main}|}|\\
\end{tabular}
\end{center}
%
Both forms have slightly different effects as described above.
The main file is prepared as usual, see \secref{sec:include}.

%%%%%%%%%%%%%%%%%%%%%%%%%%%%%%%%%%%%%%%%%%%%%%%%%%%%%%%%%%%%%%%%%%%%%%%%%%%%%%%%
\subsection{Legacy Detection}
\label{sec:detection}

The directive |\childdocmain| in the main file can detect
whether the complete document or merely a child is to be compiled
even without using the directive |\childdocof|.
This method is deprecated because it is less robust
and there is no compelling reason to use it;
it is merely provided for backward compatibility
and it may be removed in future versions.

If the detection mechanism is to be used,
it is mandatory to correctly specify
the filename of the main file as the argument of |\childdocmain|:
%
\begin{center}
\begin{tabular}{l}
|\input{childdoc.def}|\\
|\childdocmain{|\textit{main}|}|\\
\end{tabular}
\end{center}
%
If |\jobname| does not match the argument \textit{main} of |\childdocmain|,
it is assumed that |\jobname| points to the child file to be compiled.
When using |\childdocmain| with the main file specified as argument,
it suffices to start a child file
with just |\input{|\textit{main}|}|
without loading of the package and using |\childdocof|.
If instead all processing is done
with the appropriate \textsf{childdoc} directives,
the argument of \textit{main} of |\childdocmain| can be empty.

An alternative version of the command line processing described
in \secref{sec:commandline} using the detection mechanism reads:
%
\begin{center}
|... -jobname "|\textit{target}|" "|[\textit{flags}]%
[|\def\jobname{|\textit{dest}|}|]|\input{|\textit{main}|}"|
\end{center}

%%%%%%%%%%%%%%%%%%%%%%%%%%%%%%%%%%%%%%%%%%%%%%%%%%%%%%%%%%%%%%%%%%%%%%%%%%%%%%%%
\subsection{Manual Code}
\label{sec:manual}

In case one cannot be certain whether the definitions file |childdoc.def|
is installed on the target \TeX{} distribution
and one prefers not to ship it,
it is conceivable to paste a few relevant commands into the sources.

To that end, drop all statements |\input{childdoc.def}|
and perform the replacements as outlined below.
Instead of |\childdocmain{|\textit{main}|}| add the following code
to the top of the main file:
%
\begin{center}
\begin{tabular}{l}
|\||ifdefined\childdocname\endinput\||fi\newif\ifchilddoc|\\
|\edef\childdocname{\scantokens\expandafter{\jobname\noexpand}}|\\
|\def\childdocmain{|\textit{main}|}\||ifx\childdocmain\childdocname\||else|\\
|\childdoctrue\includeonly{\childdocname}\let\jobname\childdocmain\||fi|\\
\end{tabular}
\end{center}
%
Instead of |\childdocof{|\textit{main}|}| just include the main file
at the top of each child file:
%
\begin{center}
|\input{|\textit{main}|}|
\end{center}
%
A simple redirection |\childdocforward{|\textit{dest}|}| is achieved by:
%
\begin{center}
|\def\jobname{|\textit{dest}|}\input{\jobname}|
\end{center}
%
The redirection with prefix
|\childdocforwardprefix[|\textit{prefix}|]{|\textit{dest}|}|
is accomplished by:
%
\begin{center}
\begin{tabular}{l}
|{\edef\jobname{\scantokens\expandafter{\jobname\noexpand}}|\\
|\def\redirectjob |\textit{prefix}|#1~~~{\gdef\jobname{|\textit{dest}|#1}}|\\
|\expandafter\redirectjob\jobname~~~}\input{\jobname}|
\end{tabular}
\end{center}

In an alternative approach,
child documents can be compiled by a specific command line
without additional code or specific definitions:
%
\begin{center}
|... -jobname "|\textit{target}|" "|[\textit{flags}]%
|\includeonly{|\textit{dest}|}\input{|\textit{main}|}"|
\end{center}
%

%%%%%%%%%%%%%%%%%%%%%%%%%%%%%%%%%%%%%%%%%%%%%%%%%%%%%%%%%%%%%%%%%%%%%%%%%%%%%%%%
%%%%%%%%%%%%%%%%%%%%%%%%%%%%%%%%%%%%%%%%%%%%%%%%%%%%%%%%%%%%%%%%%%%%%%%%%%%%%%%%
\section{Information}

%%%%%%%%%%%%%%%%%%%%%%%%%%%%%%%%%%%%%%%%%%%%%%%%%%%%%%%%%%%%%%%%%%%%%%%%%%%%%%%%
\subsection{Copyright}

Copyright \copyright{} 2017--2018 Niklas Beisert

This work may be distributed and/or modified under the
conditions of the \LaTeX{} Project Public License, either version 1.3
of this license or (at your option) any later version.
The latest version of this license is in
  \url{http://www.latex-project.org/lppl.txt}
and version 1.3 or later is part of all distributions of \LaTeX{}
version 2005/12/01 or later.

This work has the LPPL maintenance status `maintained'.

The Current Maintainer of this work is Niklas Beisert.

This work consists of the files |README.txt|, |childdoc.ins| and |childdoc.dtx|
as well as the derived files |childdoc.def|, |cdocsamp.tex|
with |cdocsch1.tex|, |cdocsch2.tex|, |cdocspt3.tex|, |cdocspt4.tex|,
|cdocsdrf.tex|, |cdocsfn1.tex|, |cdocsfn2.tex|
as well as |childdoc.pdf|.

%%%%%%%%%%%%%%%%%%%%%%%%%%%%%%%%%%%%%%%%%%%%%%%%%%%%%%%%%%%%%%%%%%%%%%%%%%%%%%%%
\subsection{Files and Installation}

The package consists of the files:
%
\begin{center}
\begin{tabular}{ll}
    |README.txt|   & readme file \\
    |childdoc.ins| & installation file \\
    |childdoc.dtx| & source file \\
    |childdoc.def| & definition file \\
    |cdocsamp.tex| & sample main file \\
    |cdocsch1.tex| & sample include file \\
    |cdocsch2.tex| & sample include file \\
    |cdocspt3.tex| & sample part file \\
    |cdocspt4.tex| & sample part file \\
    |cdocsdrf.tex| & sample redirection file \\
    |cdocsfn1.tex| & sample redirection file \\
    |cdocsfn2.tex| & sample redirection file \\
    |childdoc.pdf| & manual
\end{tabular}
\end{center}
%
The distribution consists of the files
|README.txt|, |childdoc.ins| and |childdoc.dtx|.
%
\begin{itemize}
\item
Run (pdf)\LaTeX{} on |childdoc.dtx|
to compile the manual |childdoc.pdf| (this file).
\item
Run \LaTeX{} on |childdoc.ins| to create the definitions file |childdoc.def|
and the sample |cdocsamp.tex| with include files
|cdocsch1.tex|, |cdocsch2.tex|, |cdocspt3.tex|, |cdocspt4.tex|,
|cdocsdrf.tex|, |cdocsfn1.tex|, |cdocsfn2.tex|.
Then copy the file |childdoc.def| to an appropriate directory of your \LaTeX{}
distribution, e.g.\ \textit{texmf-root}|/tex/latex/childdoc|.
\end{itemize}

%%%%%%%%%%%%%%%%%%%%%%%%%%%%%%%%%%%%%%%%%%%%%%%%%%%%%%%%%%%%%%%%%%%%%%%%%%%%%%%%
\subsection{Related CTAN Packages}

There are several other packages which offer a similar functionality:
%
\begin{itemize}
\item
The packages
\href{http://ctan.org/pkg/docmute}{\textsf{docmute}},
\href{http://ctan.org/pkg/includex}{\textsf{includex}} and
\href{http://ctan.org/pkg/standalone}{\textsf{standalone}}
provide commands to include only the document body of
a child file thus allowing both files to be compiled individually.
\item
The packages \href{http://ctan.org/pkg/subdocs}{\textsf{subdocs}}
and \href{http://ctan.org/pkg/subfiles}{\textsf{subfiles}}
provide structures in which the main and child documents can be
encapsulated and allowing them to be compiled individually.
The inclusion mechanism is different from the conventional |\include|.
\item
The package \href{http://ctan.org/pkg/combine}{\textsf{combine}}
is an elaborate solution to combine several documents into one.
\end{itemize}
%
See also the CTAN topic \href{http://ctan.org/topic/subdocs}{\textsf{subdocs}}
for further related packages.
The present package differs from the above solutions in that
a document structure constructed with the conventional |\include| mechanism
just needs two extra commands at the top of every file
such that all constituent files can be compiled individually.

%%%%%%%%%%%%%%%%%%%%%%%%%%%%%%%%%%%%%%%%%%%%%%%%%%%%%%%%%%%%%%%%%%%%%%%%%%%%%%%%
%\subsection{Feature Suggestions}
%
%The following is a list of features which may be useful for future
%versions of this package:
%%
%\begin{itemize}
%\item
%\ldots
%\end{itemize}

%%%%%%%%%%%%%%%%%%%%%%%%%%%%%%%%%%%%%%%%%%%%%%%%%%%%%%%%%%%%%%%%%%%%%%%%%%%%%%%%
\subsection{Revision History}

%%%%%%%%%%%%%%%%%%%%%%%%%%%%%%%%%%%%%%%%
\paragraph{v2.0:} 2018/12/30

\begin{itemize}
\item
immediate forward processing
\item
added |\childdocby| mechanism
\item
manual restructured
\end{itemize}

%%%%%%%%%%%%%%%%%%%%%%%%%%%%%%%%%%%%%%%%
\paragraph{v1.6:} 2018/01/17

\begin{itemize}
\item
application for development of include files
\item
corrections to manual
\end{itemize}

%%%%%%%%%%%%%%%%%%%%%%%%%%%%%%%%%%%%%%%%
\paragraph{v1.5:} 2017/05/21

\begin{itemize}
\item
more complete structuring introduced
\item
|\childdocof| introduced
\item
|\childdoc| renamed to |\childdocmain|
\item
|\childredirect| renamed to |\childdocforward| and |\childdocforwardprefix|
and functionality expanded
\end{itemize}

%%%%%%%%%%%%%%%%%%%%%%%%%%%%%%%%%%%%%%%%
\paragraph{v1.0:} 2017/04/27

\begin{itemize}
\item
manual and install package
\item
first version published on CTAN
\end{itemize}

%%%%%%%%%%%%%%%%%%%%%%%%%%%%%%%%%%%%%%%%
\paragraph{v0.6:} 2017/04/26

\begin{itemize}
\item
redirection mechanism added
\end{itemize}

%%%%%%%%%%%%%%%%%%%%%%%%%%%%%%%%%%%%%%%%
\paragraph{v0.5:} 2017/04/26

\begin{itemize}
\item
functionality in definition file
\end{itemize}


%%%%%%%%%%%%%%%%%%%%%%%%%%%%%%%%%%%%%%%%%%%%%%%%%%%%%%%%%%%%%%%%%%%%%%%%%%%%%%%%
%%%%%%%%%%%%%%%%%%%%%%%%%%%%%%%%%%%%%%%%%%%%%%%%%%%%%%%%%%%%%%%%%%%%%%%%%%%%%%%%
%%%%%%%%%%%%%%%%%%%%%%%%%%%%%%%%%%%%%%%%%%%%%%%%%%%%%%%%%%%%%%%%%%%%%%%%%%%%%%%%
\appendix

\settowidth\MacroIndent{\rmfamily\scriptsize 000\ }

 \DocInput{childdoc.dtx}

\end{document}
%</driver>
% \fi
%
% %%%%%%%%%%%%%%%%%%%%%%%%%%%%%%%%%%%%%%%%%%%%%%%%%%%%%%%%%%%%%%%%%%%%%%%%%%%%%%
% %%%%%%%%%%%%%%%%%%%%%%%%%%%%%%%%%%%%%%%%%%%%%%%%%%%%%%%%%%%%%%%%%%%%%%%%%%%%%%
% \section{Sample}
%\iffalse
%<*samplemain>
%\fi
%
% The following presents a sample document
% with two chapters, two parts, a title page,
% a compile flag as well as three forwarding files to set the flag.
% It consists of eight |.tex| files:
% \begin{center}
% \begin{tabular}{ll}
% |cdocsamp.tex|&main file\\
% |cdocsch1.tex|&include file for chapter 1\\
% |cdocsch2.tex|&include file for chapter 2\\
% |cdocspt3.tex|&include file for part 3\\
% |cdocspt4.tex|&include file for part 4\\
% |cdocsdrf.tex|&forwarding file for main file in draft mode\\
% |cdocsfi1.tex|&forwarding file for final version of chapter 1\\
% |cdocsfi2.tex|&forwarding file for final version of chapter 2\\
% \end{tabular}
% \end{center}
% Each of the eight files can be compiled directly by the \LaTeX{} compiler.
%
% %%%%%%%%%%%%%%%%%%%%%%%%%%%%%%%%%%%%%%
% \paragraph{Main File.}
%
% The main file is called |cdocsamp.tex|.
%
% Load the \textsf{childdoc} definitions and
% declare the filename for the main document:
%    \begin{macrocode}
\input{childdoc.def}
\childdocmain{}
%    \end{macrocode}

% Optional override for |\version| flag:
%    \begin{macrocode}
%%\ifchilddoc\else\providecommand{\version}{draft}\fi
%    \end{macrocode}

% Define the default values for the |\version| flag
% (|final| for the main file and |draft| for childs):
%    \begin{macrocode}
\ifchilddoc
\providecommand{\version}{draft}
\else
\providecommand{\version}{final}
\fi
%    \end{macrocode}

% Load the standard document class:
%    \begin{macrocode}
\documentclass[12pt]{article}
%    \end{macrocode}

% Start the document body:
%    \begin{macrocode}
\begin{document}
%    \end{macrocode}

% Declare a title page.
% Print title, part of document being processed and version flag:
%    \begin{macrocode}
\addtocounter{page}{-1}
\begin{center}
{\LARGE\bfseries{}childdoc example\par}
\vspace{1cm}
\ifchilddoc
\ifchilddocmanual part\else chapter\fi:
`\childdocname' of `\childdocjob'\par
\else
main document: `\childdocjob'\par
\fi
version: \version\par
\end{center}
\newpage
%    \end{macrocode}

% Manually include selected file,
% otherwise process as usual:
%    \begin{macrocode}
\ifchilddocmanual
\section*{part `\childdocname'}
\input{\childdocname}
\else
%    \end{macrocode}

% Include the two chapters:
%    \begin{macrocode}
\include{cdocsch1}
\include{cdocsch2}
%    \end{macrocode}

% Include the two parts unless only chapters should be displayed:
%    \begin{macrocode}
\ifchilddoc\else
\section{part three}
\input{cdocspt3}
\section{part four}
\input{cdocspt4}
\fi
%    \end{macrocode}

% Process as usual until here:
%    \begin{macrocode}
\fi
%    \end{macrocode}

% End of document body:
%    \begin{macrocode}
\end{document}
%    \end{macrocode}
%\iffalse
%</samplemain>
%\fi
%
% %%%%%%%%%%%%%%%%%%%%%%%%%%%%%%%%%%%%%%
% \paragraph{Chapter Include Files.}
%
% The include files are called |cdocsch1.tex| and |cdocsch2.tex|.
%
%\iffalse
%<*samplechap1|samplechap2>
%\fi

% Optional override for |\version| flag:
%    \begin{macrocode}
%%\providecommand{\version}{final}
%    \end{macrocode}

% Include the main document:
%    \begin{macrocode}
\input{childdoc.def}
\childdocof{cdocsamp}
%    \end{macrocode}

%\iffalse
%</samplechap1|samplechap2>
%\fi
%
%\iffalse
%<*samplechap1>
%\fi
% Some text for chapter 1:
%    \begin{macrocode}
\section{one}
some text in chapter one
%    \end{macrocode}

%\iffalse
%</samplechap1>
%\fi
% Some text for chapter 2:
%\iffalse
%<*samplechap2>
%\fi
%    \begin{macrocode}
\section{two}
more text in chapter two
%    \end{macrocode}

%\iffalse
%</samplechap2>
%\fi
%
% %%%%%%%%%%%%%%%%%%%%%%%%%%%%%%%%%%%%%%
% \paragraph{Part Include Files.}
%
% The include files are called |cdocspt3.tex| and |cdocspt4.tex|.
%
%\iffalse
%<*samplepart3|samplepart4>
%\fi

% Optional override for |\version| flag:
%    \begin{macrocode}
%%\providecommand{\version}{final}
%    \end{macrocode}

% Include the main document:
%    \begin{macrocode}
\input{childdoc.def}
\childdocby{cdocsamp}
%    \end{macrocode}

%\iffalse
%</samplepart3|samplepart4>
%\fi
%
%\iffalse
%<*samplepart3>
%\fi
% Some text for part 3:
%    \begin{macrocode}
some text in part three
%    \end{macrocode}

%\iffalse
%</samplepart3>
%\fi
% Some text for part 4:
%\iffalse
%<*samplepart4>
%\fi
%    \begin{macrocode}
more text in part four
%    \end{macrocode}

%\iffalse
%</samplepart4>
%\fi
%
% %%%%%%%%%%%%%%%%%%%%%%%%%%%%%%%%%%%%%%
% \paragraph{Forwarding for a Complete Draft.}
%
% The following forwarding file |cdocsdrf.tex|
% compiles the main document in draft mode:
%\iffalse
%<*sampledraft>
%\fi
%    \begin{macrocode}
\def\version{draft}
\input{childdoc.def}
\childdocforward{cdocsamp}
%    \end{macrocode}

%\iffalse
%</sampledraft>
%\fi
%
% %%%%%%%%%%%%%%%%%%%%%%%%%%%%%%%%%%%%%%
% \paragraph{Forwarding for Final Version of the Chapters.}
%
% The following forwarding files |cdocsfn1.tex| and |cdocsfn2.tex|
% (with identical content)
% compile the final versions of the child documents
% |cdocsch1.tex| and |cdocsch2.tex|, respectively:
%\iffalse
%<*samplefinal>
%\fi
%    \begin{macrocode}
\def\version{final}
\input{childdoc.def}
\childdocforwardprefix[cdocsamp]{cdocsfn}{cdocsch}
%    \end{macrocode}

%\iffalse
%</samplefinal>
%\fi
%
% %%%%%%%%%%%%%%%%%%%%%%%%%%%%%%%%%%%%%%
% \paragraph{Command Line Processing.}
%
% The following three command lines generate the output files
% |cdocscld|, |cdocscl1| and |cdocscl2|
% which should be identical to
% |cdocsdrf|, |cdocsch1| and |cdocsfn2|, respectively:
% \begin{center}
% \begin{tabular}{l}
% |latex -jobname cdocscld \|\\
% |  "\def\version{draft}\input{childdoc.def}\childdocforward{cdocsamp}"|\\
% |latex -jobname cdocscl1 \|\\
% |  "\input{childdoc.def}\childdocforward[cdocsamp]{cdocsch1}"|\\
% |latex -jobname cdocscl2 \|\\
% |  "\def\version{final}\input{childdoc.def}\childdocforward{cdocsch2}"|
% \end{tabular}
% \end{center}
% Note that the trailing backslash on each first line
% merely continues the input to the second line
% (for convenient cut ant paste).
% Furthermore, the command |latex| can be replaced by any
% of its alternative versions such as |pdflatex|.
%
% %%%%%%%%%%%%%%%%%%%%%%%%%%%%%%%%%%%%%%%%%%%%%%%%%%%%%%%%%%%%%%%%%%%%%%%%%%%%%%
% %%%%%%%%%%%%%%%%%%%%%%%%%%%%%%%%%%%%%%%%%%%%%%%%%%%%%%%%%%%%%%%%%%%%%%%%%%%%%%
% \section{Implementation}
%\iffalse
%<*package>
%\fi
%
% This section describes the definitions file |childdoc.def|.

% The definitions cannot be loaded using |\usepackage| or |\RequirePackage|
% which has a mechanism to prevent loading a style file more than once.
% When loading the definitions by means of |\input|
% multiple instances have to be prevented manually:
%\iffalse
%This code needs to be before the `\ProvidesFile' directive
%which is defined at the beginning of this file.
%Therefore it is also placed there and commented out here.
%</package>
%<*discard>
%\fi
%    \begin{macrocode}
\ifdefined\childdocmain\endinput\fi
%    \end{macrocode}
%\iffalse
%</discard>
%<*package>
%\fi
%
% \macro{\ifchilddoc}
% \macro{\ifchilddocmanual}
% The conditional |\ifchilddoc| tells whether a
% child (true) or main (false) document is being compiled.
% The conditional |\ifchilddocmanual| tells whether
% the |\includeonly| mechanism is used (false) or
% the selection of child files must be performed manually (true).
% The definitions initialise to false:
%    \begin{macrocode}
\newif\ifchilddoc
\newif\ifchilddocmanual
%    \end{macrocode}

% \macro{\childdocname}
% \macro{\childdocjob}
% The macro |\childdocname| stores the name of the main document
% to be compiled. The macro |\childdocjob| stores the name of
% the document on which the \LaTeX{} compiler was originally invoked.
% The content of |\jobname| cannot be compared
% to filenames specified in the source due to different catcodes.
% The following code rescans |\jobname|, stores the result
% in |\childdocname| and saves a copy in |\childdocjob|:
%    \begin{macrocode}
\edef\childdocname{\scantokens\expandafter{\jobname\noexpand}}
\let\childdocjob\childdocname
%    \end{macrocode}

% \macro{\childdocdisable}
% The macro |\childdocdisable| prevents the main file
% from being processed more than once.
% At this stage, the main document command |\childdocmain|
% is assumed to be called once again where it should do nothing.
% Any subsequent call to it should prevent
% a secondary processing of the main document
% It overwrites the forwarding commands
% |\childdocof| and |\childdocforward|
% with empty macros to prevent further inclusions of the main document:
%    \begin{macrocode}
\newcommand{\childdocdisable}
{
  \renewcommand{\childdocmain}[1]{\renewcommand{\childdocmain}[1]{\endinput}}
  \renewcommand{\childdocof}[1]{}
  \renewcommand{\childdocby}[2][]{}
  \renewcommand{\childdocforward}[2][]{}
  \renewcommand{\childdocdisable}{}
}
%    \end{macrocode}

% \macro{\childdocmain}
% The macro |\childdocmain| is to be called at the top of the main file
% with nothing or the main filename (without extension) as argument.
% First, it breaks loops.
% If the argument is not empty and does not match |\childdocname|
% (which is set by the first inclusion of |childdoc.def|),
% |\ifchilddoc| is set to true, |\includeonly| is applied to the child file
% and |\jobname| is set to the main file
% (for proper handling of |.aux| files):
%    \begin{macrocode}
\newcommand{\childdocmain}[1]
{
  \childdocdisable\childdocmain{}
  \if?#1?\else
    \begingroup
      \def\childdoctmp{#1}
      \ifx\childdoctmp\childdocname
        \def\childdoctmp{}
      \else
        \def\childdoctmp
        {
          \childdoctrue
          \includeonly{\childdocname}
          \def\childdocjob{#1}
          \def\jobname{#1}
        }
      \fi
      \expandafter
    \endgroup
    \childdoctmp
  \fi
}
%    \end{macrocode}

% \macro{\childdocof}
% The command |\childdocof| redirects
% compilation to the main file |#1|.
%    \begin{macrocode}
\newcommand{\childdocof}[1]
{
  \childdocdisable
  \childdoctrue
  \includeonly{\childdocname}
  \def\jobname{#1}
  \def\childdocjob{#1}
  \input{#1}
}
%    \end{macrocode}

% \macro{\childdocby}
% The command |\childdocby| ....
%    \begin{macrocode}
\newcommand{\childdocby}[2][]
{
  \childdocdisable
  \childdoctrue
  \childdocmanualtrue
  \if?#1?\else
    \def\jobname{#2}
  \fi
  \def\childdocjob{#2}
  \input{#2}
  \endinput
}
%    \end{macrocode}

% \macro{\childdocforward}
% The command |\childdocforward| redirects
% compilation to the main file or
% (if the optional argument is given) a child file.
% Parameters are set as if the main file
% or a child file starting with |\childdocof| was compiled.
% Then compilation is handed over to the main file:
%    \begin{macrocode}
\newcommand{\childdocforward}[2][]
{
  \begingroup
    \if?#1?
      \def\childdoctmp
      {
        \def\childdocname{#2}
        \def\childdocjob{#2}
        \def\jobname{#2}
        \input{#2}
        \endinput
      }
    \else
      \def\childdoctmp
      {
        \childdocdisable
        \def\childdocname{#2}
        \childdoctrue
        \includeonly{#2}
        \def\childdocjob{#1}
        \def\jobname{#1}
        \input{#1}
        \endinput
      }
    \fi
    \expandafter
  \endgroup
  \childdoctmp
}
%    \end{macrocode}

% \macro{\childdocforwardprefix}
% The command |\childdocforwardprefix| redirects
% compilation to the main or a child file by means of a pattern.
% The prefix |#1| in the current filename is replaced by |#2|
% and the suffix of the current filename is kept
% (it is assumed that the filename does not contain the substring `|~~~|'
% which is used as a delimiter).
% Compilation is handed over to the new file by |\childdocforward|:
%    \begin{macrocode}
\newcommand{\childdocforwardprefix}[3][]
{
  \begingroup
    \def\childdocextract #2##1~~~{\def\childdoctmp{\childdocforward[#1]{#3##1}}}
    \expandafter\childdocextract\childdocname~~~
    \expandafter
  \endgroup
  \childdoctmp
}
%    \end{macrocode}

% \macro{\childdoc}
% The deprecated macro |\childdoc| is a legacy version of |\childdocmain|:
%    \begin{macrocode}
\newcommand{\childdoc}{\childdocmain}
%    \end{macrocode}

% \macro{\childdocredirect}
% The deprecated macro |\childdocredirect| is a legacy version
% of |\childdocforward| and |\childdocforwardprefix|:
%    \begin{macrocode}
\newcommand{\childdocredirect}[2][]
{
  \begingroup
    \if?#1?
      \def\childdoctmp{\childdocforward{#2}}
    \else
      \def\childdoctmp{\childdocforwardprefix{#1}{#2}}
    \fi
    \expandafter
  \endgroup
  \childdoctmp
}
%    \end{macrocode}

%\iffalse
%</package>
%\fi
%
\endinput
|\\
|\childdocmain{}|\\
\end{tabular}
\end{center}
at the very top of the main \LaTeX{} file,
in particular \emph{before} the |\documentclass| statement!
The argument of |\childdocmain| should be left empty
(but it must be present).

%%%%%%%%%%%%%%%%%%%%%%%%%%%%%%%%%%%%%%%%
\DescribeMacro{\childdocof}
Furthermore, add the commands
\begin{center}
\begin{tabular}{l}
|% \iffalse
%
% childdoc.dtx Copyright (C) 2017-2018 Niklas Beisert
%
% This work may be distributed and/or modified under the
% conditions of the LaTeX Project Public License, either version 1.3
% of this license or (at your option) any later version.
% The latest version of this license is in
%   http://www.latex-project.org/lppl.txt
% and version 1.3 or later is part of all distributions of LaTeX
% version 2005/12/01 or later.
%
% This work has the LPPL maintenance status `maintained'.
%
% The Current Maintainer of this work is Niklas Beisert.
%
% This work consists of the files childdoc.dtx and childdoc.ins
% and the derived files childdoc.def and cdocsamp.tex with
% cdocsch1.tex, cdocsch2.tex, cdocsdrf.tex, cdocsfn1.tex, cdocsfn2.tex.
%
%<package>\ifdefined\childdocmain\endinput\fi
%<package>\ProvidesFile{childdoc.def}[2018/12/30 v2.0 child document driver]
%<samplemain>\ProvidesFile{cdocsamp.tex}[2018/12/30 v2.0 sample for childdoc]
%<*driver>
%\ProvidesFile{childdoc.drv}[2018/12/30 v2.0 childdoc reference manual file]
\PassOptionsToClass{10pt,a4paper}{article}
\documentclass{ltxdoc}

\usepackage[margin=35mm]{geometry}
\usepackage{hyperref}
\usepackage{hyperxmp}
\usepackage[usenames]{color}

\hypersetup{colorlinks=true}
\hypersetup{pdfstartview=FitH}
\hypersetup{pdfpagemode=UseNone}
\hypersetup{pdfsource={}}
\hypersetup{pdflang={en-UK}}
\hypersetup{pdfcopyright={Copyright 2017-2018 Niklas Beisert.
  This work may be distributed and/or modified under the
  conditions of the LaTeX Project Public License, either version 1.3
  of this license or (at your option) any later version.}}
\hypersetup{pdflicenseurl={http://www.latex-project.org/lppl.txt}}
\hypersetup{pdfcontactaddress={ETH Zurich, ITP, HIT K,
  Wolfgang-Pauli-Strasse 27}}
\hypersetup{pdfcontactpostcode={8093}}
\hypersetup{pdfcontactcity={Zurich}}
\hypersetup{pdfcontactcountry={Switzerland}}
\hypersetup{pdfcontactemail={nbeisert@itp.phys.ethz.ch}}
\hypersetup{pdfcontacturl={http://people.phys.ethz.ch/\xmptilde nbeisert/}}

\newcommand{\secref}[1]{\hyperref[#1]{section \ref*{#1}}}

\parskip1ex
\parindent0pt
\let\olditemize\itemize
\def\itemize{\olditemize\parskip0pt}

\begin{document}

\title{The \textsf{childdoc} Package}
\hypersetup{pdftitle={The childdoc Package}}
\author{Niklas Beisert\\[2ex]
  Institut f\"ur Theoretische Physik\\
  Eidgen\"ossische Technische Hochschule Z\"urich\\
  Wolfgang-Pauli-Strasse 27, 8093 Z\"urich, Switzerland\\[1ex]
  \href{mailto:nbeisert@itp.phys.ethz.ch}
  {\texttt{nbeisert@itp.phys.ethz.ch}}}
\hypersetup{pdfauthor={Niklas Beisert}}
\hypersetup{pdfsubject={Manual for the LaTeX2e Package childdoc}}
\date{30 December 2018, \textsf{v2.0}}
\maketitle

\begin{abstract}\noindent
\textsf{childdoc} is a \LaTeXe{} package
that enables the direct compilation
of document sections included by |\include|
to individual files.
\end{abstract}

\begingroup
\parskip0ex
\tableofcontents
\endgroup

%%%%%%%%%%%%%%%%%%%%%%%%%%%%%%%%%%%%%%%%%%%%%%%%%%%%%%%%%%%%%%%%%%%%%%%%%%%%%%%%
%%%%%%%%%%%%%%%%%%%%%%%%%%%%%%%%%%%%%%%%%%%%%%%%%%%%%%%%%%%%%%%%%%%%%%%%%%%%%%%%
\section{Introduction}

\LaTeX{} provides a mechanism to structure a large document (such as a book)
into a main file and several child files (containing the chapters)
using the |\include| command.
This mechanism is beneficial for documents
which span hundreds of pages in order to
make the source file(s) more manageable.
Moreover, compilation can be restricted to
selected child files by means of the |\includeonly| command.
The latter feature can be used to reduce the compilation time while editing
(this was significantly more useful in the earlier days of \LaTeX{})
or to generate a smaller document which is easier to navigate.
Another application of |\includeonly| is to generate
documents consisting of selected parts of the complete document.

However, there are a few drawbacks of the plain |\include| mechanism:
\begin{itemize}
\item
The child files cannot be compiled on their own,
they can only be compiled via the main file.
A naive editing environment
(such as a text editor with an option
to have the current file processed by \LaTeX)
may require one to switch to the main file before compiling;
attempting to compile the child file produces errors.
\item
The main file must be modified (each time)
to adjust the |\includeonly| command
to the present needs. This easily leaves the main file in a messy state.
\item
The generated document will always carry the filename
of the main document. This is inconvenient if
several child files are to be compiled and
to be kept for distribution.
\end{itemize}

The present package provides a simple interface
to make child files individually compilable by \LaTeX{}.
Compiling a child file then has the same effect as compiling
the main file with an |\includeonly| command
to select the appropriate child.
Moreover the generated document will carry the name of the child
rather than the main file.
This resolves all three above issues.

This feature is meant to make the editing of books,
thesis documents and lecture notes somewhat more convenient.
However, the package can also be used efficiently for
composing a series of documents (such as exercise sheets)
which are typically distributed individually.
It then assists the author in generating the individual documents
(potentially in different versions)
as well as a document containing the collected series.
Another application is in developing style files
or other kinds of included material
where compilation of the style file could redirect
to a sample or test file.

%%%%%%%%%%%%%%%%%%%%%%%%%%%%%%%%%%%%%%%%%%%%%%%%%%%%%%%%%%%%%%%%%%%%%%%%%%%%%%%%
%%%%%%%%%%%%%%%%%%%%%%%%%%%%%%%%%%%%%%%%%%%%%%%%%%%%%%%%%%%%%%%%%%%%%%%%%%%%%%%%
\section{Usage}

First of all, the package \textsf{childdoc} is \emph{not} a standard
\LaTeXe{} |.sty| style file! Therefore it needs to be invoked in
a non-standard way.

%%%%%%%%%%%%%%%%%%%%%%%%%%%%%%%%%%%%%%%%%%%%%%%%%%%%%%%%%%%%%%%%%%%%%%%%%%%%%%%%
\subsection{Included Files}
\label{sec:include}

%%%%%%%%%%%%%%%%%%%%%%%%%%%%%%%%%%%%%%%%
\DescribeMacro{\childdocmain}
To use the package, add the commands
\begin{center}
\begin{tabular}{l}
|\input{childdoc.def}|\\
|\childdocmain{}|\\
\end{tabular}
\end{center}
at the very top of the main \LaTeX{} file,
in particular \emph{before} the |\documentclass| statement!
The argument of |\childdocmain| should be left empty
(but it must be present).

%%%%%%%%%%%%%%%%%%%%%%%%%%%%%%%%%%%%%%%%
\DescribeMacro{\childdocof}
Furthermore, add the commands
\begin{center}
\begin{tabular}{l}
|\input{childdoc.def}|\\
|\childdocof{|\textit{main}|}|\\
\end{tabular}
\end{center}
at the top of every child file \textit{child}
which is included by |\include{|\textit{child}|}|
from within the main file
(or at least for those files to be compiled individually).
The argument \textit{main} must be the filename of the main file.

There are a couple of
considerations in setting up the main and child documents:

%%%%%%%%%%%%%%%%%%%%%%%%%%%%%%%%%%%%%%%%
\paragraph{Restrictions.}

Please note the following restrictions:
\begin{itemize}
\item
|\childdocmain| must be called with one argument \textit{main}
to ensure compatibility with earlier version of the package.
It must either be empty (|\childdocmain{}|)
or precisely match the filename of the main file in which it is specified.
See \secref{sec:detection} for further information.
\item
The filename \textit{main} must be specified without the |.tex| extension.
\item
The filename \textit{main} is case sensitive
(even in case-insensitive file systems)
due to internal string comparison.
\item
The argument \textit{main} should be fully expanded, it cannot be a macro.
\item
Subdirectories and special characters should be avoided in filenames.
\item
The command |\childdocmain{|\textit{main}|}| must be followed by a whitespace.
It should not be followed immediately by another command
or by a comment mark `|%|'.
This is because the \TeX{} parser reads the token immediately following
the argument of |\childdocmain| and puts it
at the beginning of every child section;
however, a white\-space is ignored.
\end{itemize}

%%%%%%%%%%%%%%%%%%%%%%%%%%%%%%%%%%%%%%%%
\paragraph{Content of Main File.}

It is advisable to place all content in the child files included by |\include|.
Any output contained in the main file will appear in all child documents
unless suppressed manually;
it cannot be suppressed automatically by the |\includeonly| directive
and thus should normally be avoided.
A method to include some content in the main file
by means of conditional processing is described in \secref{sec:conditional}.

%%%%%%%%%%%%%%%%%%%%%%%%%%%%%%%%%%%%%%%%
\paragraph{Page Numbering.}

When only a part of the document is compiled,
the appropriate numbering of pages
(as well as other status parameters)
is determined from the |.aux| files.
The latter contain information from previous passes.
However this information needs to propagate through
all intermediate child documents.
Therefore the page numbering in child documents may well
be inconsistent until the complete document is compiled at least once.

A useful (if unconventional) way to always ensure a consistent
page numbering is to restart the numbering in each child document
and denote the pages by `\textit{child}|.|\textit{page}'
where \textit{child} represents the chapter/section number of the child file.
This can be achieved by the command
|\numberwithin{page}{|\textit{child}|}|
of the \textsf{amsmath} package
where \textit{child} can be |chapter| or |section|
depending on the chosen structuring.
Alternatively, one can modify the macro |\thepage| appropriately
and reset the counter |page| at the start of each child file.

%%%%%%%%%%%%%%%%%%%%%%%%%%%%%%%%%%%%%%%%%%%%%%%%%%%%%%%%%%%%%%%%%%%%%%%%%%%%%%%%
\subsection{Conditional Processing}
\label{sec:conditional}

The package provides a mechanism to compile different versions
of a document. To customise the versions further some conditional processing
can come in handy to distinguish which version is being compiled.
The package provides two macros to describe the compilation context:

%%%%%%%%%%%%%%%%%%%%%%%%%%%%%%%%%%%%%%%%
\DescribeMacro{\ifchilddoc}
The conditional |\ifchilddoc| distinguishes between the compilation of
child documents and the main document:
%
\begin{center}
|\ifchilddoc |\textit{child-code}| |[|\||else |\textit{main-code}]| \||fi|
\end{center}

%%%%%%%%%%%%%%%%%%%%%%%%%%%%%%%%%%%%%%%%
\DescribeMacro{\childdocname}
\DescribeMacro{\childdocjob}
The macro |\childdocname| contains the filename (without extension)
of the main or child file being processed.
Note that |\childdocjob| will always contain the name of the main file.

%%%%%%%%%%%%%%%%%%%%%%%%%%%%%%%%%%%%%%%%
\paragraph{Title Page.}

Conditional processing can be used to include a title or banner page
in the main document when proper precautions are taken.
Importantly, the code in the main file should ensure that the page counter
(as well as other status parameters which are stored in the |.aux| files)
takes the same value after the conditional processing.
Otherwise the page numbers may take divergent values
depending on which part is compiled.

For example, a title page could be declared by:
%
\begin{center}
\begin{tabular}{l}
|\ifchilddoc\||else|\\
|\addtocounter{page}{-1}|\\
\textit{code for title page}\\
|\newpage|\\
|\||fi|
\end{tabular}
\end{center}
%
A banner page for the child documents can be generated by:
%
\begin{center}
\begin{tabular}{l}
|\ifchilddoc|\\
|\addtocounter{page}{-1}|\\
\textit{code for banner page}\\
|\newpage|\\
|\||fi|
\end{tabular}
\end{center}
%
Here one could write a message such as:
\begin{center}
|This is the part \childdocname{} of \childdocjob{}.|
\end{center}

%%%%%%%%%%%%%%%%%%%%%%%%%%%%%%%%%%%%%%%%%%%%%%%%%%%%%%%%%%%%%%%%%%%%%%%%%%%%%%%%
\subsection{Flags}
\label{sec:flags}

The package makes it easy to generate different versions
of the main or child documents.
To this end compilation flags can be defined
and assigned different default values.
They will be particularly useful in conjunction
with the forwarding mechanism described in \secref{sec:forward}.

For example, it may be useful to have a flag |\version|
which can be set to |draft| or |final|.
The document source will contain some conditional code
depending on the value of |\version|.
Suppose further, the flag should default to |final| for the main file
and to |draft| for child files
which is a natural assignment for editing the document.
This is achieved by placing the following code
in the preamble of the main document
(below the |\childdocmain| directive):
%
\begin{center}
\begin{tabular}{l}
|\ifchilddoc|\\
|\providecommand{\version}{draft}|\\
|\||else|\\
|\providecommand{\version}{final}|\\
|\||fi|
\end{tabular}
\end{center}
%
The definition by |\providecommand| makes sure
that previous definitions are not overwritten.
Further statements |\providecommand{\version}{...}|
can thus be added before the above code to override it.

For the main file, one might add a line
(between |\childdocmain| and the above block)
%
\begin{center}
|%\ifchilddoc\||else\providecommand{\version}{draft}\||fi|
\end{center}
%
which can be uncommented to produce a draft version.
Likewise one can add a line to the very top of a child file
(above the |\childdocof{|\textit{main}|}| directive)
%
\begin{center}
|%\providecommand{\version}{final}|
\end{center}
%
which can be uncommented to produce the final version of this child document.

%%%%%%%%%%%%%%%%%%%%%%%%%%%%%%%%%%%%%%%%%%%%%%%%%%%%%%%%%%%%%%%%%%%%%%%%%%%%%%%%
\subsection{Forwarding}
\label{sec:forward}

Different versions of the main or child documents
using compilation flags as described in \secref{sec:flags}
can be (permanently) stored in different files
for convenient compilation, viewing and distribution.
To this end, the package defines a command
to pass on compilation to a different file:

%%%%%%%%%%%%%%%%%%%%%%%%%%%%%%%%%%%%%%%%
\DescribeMacro{\childdocforward}
The command |\childdocforward| redirects processing to
another source file:
%
\begin{center}
\begin{tabular}{l}
|\input{childdoc.def}|\\
|\childdocforward[|\textit{main}|]{|\textit{dest}|}|\\
\end{tabular}
\end{center}
%
The argument \textit{dest} is the destination file
(without extension).
It should be the main file or one of the child files.
Note that further \textsf{childdoc} directives
such as |\childdocof| and |\childdocforward|
in the indicated file will be processed in this form.
The optional argument \textit{main}
passes on directly to the main file \textit{main}
while pretending to compile the child \textit{dest}.
This form behaves as if \textit{dest}
issues |\childdocof{|\textit{main}|}| right away,
and no further \textsf{childdoc} directives will be processed.

%%%%%%%%%%%%%%%%%%%%%%%%%%%%%%%%%%%%%%%%
\DescribeMacro{\...prefix}
In the alternative form |\childdocforwardprefix|,
%
\begin{center}
\begin{tabular}{l}
|\input{childdoc.def}|\\
|\childdocforwardprefix[|\textit{main}|]{|\textit{prefix}|}{|\textit{dest}|}|
\end{tabular}
\end{center}
%
the destination file is determined by a pattern
depending on the current file:
To make this work, the current file must be called
`{\textit{prefix}\hspace{0.2em}\textit{suffix}}'
with \textit{prefix} matching precisely the argument.
Processing is then passed on to the file
`{\textit{dest}\hspace{0.2em}\textit{suffix}}'.
Surely, the same effect is achieved by
directly specifying the
argument `{\textit{dest}\hspace{0.2em}\textit{suffix}}'
in the first form.
However, that requires to set up a different file
for each child. With the alternative form of the command
all these files can have exactly the same content
which simplifies setting them up and maintaining them.

For example, the following file |draft.tex|
with a compilation flag |\version| as described in \secref{sec:flags}
compiles the main document as a draft:
%
\begin{center}
\begin{tabular}{l}
|\def\version{draft}|\\
|\input{childdoc.def}|\\
|\childdocforward{|\textit{main}|}|
\end{tabular}
\end{center}
%
Likewise, the following files |final|\textit{nn}|.tex|
compile the final version of the child document
|child|\textit{nn}|.tex|:
%
\begin{center}
\begin{tabular}{l}
|\def\version{final}|\\
|\input{childdoc.def}|\\
|\childdocforwardprefix{final}{child}|
\end{tabular}
\end{center}
%

Note that when several versions of a main file and/or of each child file
are to be generated, it may be convenient to set up a |Makefile| or
shell script to automatise the process.

%%%%%%%%%%%%%%%%%%%%%%%%%%%%%%%%%%%%%%%%%%%%%%%%%%%%%%%%%%%%%%%%%%%%%%%%%%%%%%%%
\subsection{Command Line Processing}
\label{sec:commandline}

The effect of redirection files can also be achieved by invoking
the \LaTeX{} compiler with a more elaborate command line.
Most conveniently this should be done as part
of a shell script or a |Makefile|.

When using \textsf{childdoc} in the main file, the following
command lines effectively perform a redirection
(note that depending on the shell being used,
backslashes may have to be doubled: `|\|' $\to$ `|\\|'):
%
\begin{center}
|... -jobname "|\textit{target}|" |\\|"|[\textit{flags}]%
|\input{childdoc.def}\childdocforward[|\textit{main}|]{|\textit{dest}|}"|
\end{center}
%
Here \textit{target} is the name of the output file,
\textit{main} is the name of the main file
and \textit{dest} is the name of the main or child file to be processed
(all filenames without extensions).
The optional argument \textit{main} can be omitted
if \textit{main} matches \textit{dest}.
Optionally, compilation \textit{flags} can be defined via |\def| commands.
This command line makes the \TeX{} engine believe
it is compiling the file \textit{target}
whose content is specified as the latter parameter.
The provided code then forwards the processing to
\textit{main} or \textit{dest} as described in \secref{sec:forward}.

%%%%%%%%%%%%%%%%%%%%%%%%%%%%%%%%%%%%%%%%%%%%%%%%%%%%%%%%%%%%%%%%%%%%%%%%%%%%%%%%
\subsection{Include by Input}
\label{sec:input}

Including child documents by |\include| has some restrictions by design.
Most notably, the content of a child document always occupies
its own set of pages; pages cannot be shared between child documents.
Usually, this behaviour makes perfect sense
because each child document contain an essential part of the document.
However, in some situations it may be desirable to compose
a document from a collection of parts
without having mandatory page breaks between then.
For this case, the package
provides a mechanism to include parts
by |\input| which can also be processed individually.
However, by construction this mechanism
requires manual handling of the content to be output.

%%%%%%%%%%%%%%%%%%%%%%%%%%%%%%%%%%%%%%%%
\DescribeMacro{\ifchilddocmanual}
The main file should be prepared as usual, see \secref{sec:include}.
However, the document body must make a distinction
between processing of an individual part and of the main document, e.g.:
%
\begin{center}
\begin{tabular}{l}
|\ifchilddocmanual|\\
|\input{\childdocname}|\\
|\||else|\\
\textit{document body with }|\input{|\textit{part}|}|\\
|\||fi|
\end{tabular}
\end{center}
%
The conditional |\ifchilddocmanual| is true whenever
a part to be included by |\input| is being compiled,
and the name of the part is stored in |\childdocname|.

%%%%%%%%%%%%%%%%%%%%%%%%%%%%%%%%%%%%%%%%
\DescribeMacro{\childdocby}
Each part to be included by |\input| should start with:
%
\begin{center}
\begin{tabular}{l}
|\input{childdoc.def}|\\
|\childdocby{|\textit{main}|}|\\
\end{tabular}
\end{center}
%
The directive |\childdocby| is similar to |\childdocof|
described in \secref{sec:include},
but the subsequent selection of content must be done manually.
To that end, both |\ifchilddoc| and |\ifchilddocmanual|
will be true upon processing of a part,
and the name of the part is stored in |\childdocname|.
Note that |\jobname| will be set to the filename of the current part
so that each part receives an individual |.aux| file
that does not interfere with the |.aux| file(s) of the main document.
This behaviour can be altered by the alternative form
|\childdocby[*]{|\textit{main}|}| (with a non-empty optional argument)
which uses the |.aux| file of the main document
by setting |\jobname| to \textit{main}.

%%%%%%%%%%%%%%%%%%%%%%%%%%%%%%%%%%%%%%%%%%%%%%%%%%%%%%%%%%%%%%%%%%%%%%%%%%%%%%%%
\subsection{Driver Development}
\label{sec:driver}

The \textsf{childdoc} mechanism can also be use for the development
of definition files such as \LaTeX{} styles or classes.
This case differs from the above setup with multiple parts
included by |\include| in that no |\includeonly| should be invoked.
This can be achieved by starting the include file
(before |\ProvidesPackage|) with:
%
\begin{center}
\begin{tabular}{l}
|\input{childdoc.def}|\\
|\childdocforward{|\textit{main}|}|\\
\end{tabular}
\end{center}
%
or alternatively with:
%
\begin{center}
\begin{tabular}{l}
|\input{childdoc.def}|\\
|\childdocby{|\textit{main}|}|\\
\end{tabular}
\end{center}
%
Both forms have slightly different effects as described above.
The main file is prepared as usual, see \secref{sec:include}.

%%%%%%%%%%%%%%%%%%%%%%%%%%%%%%%%%%%%%%%%%%%%%%%%%%%%%%%%%%%%%%%%%%%%%%%%%%%%%%%%
\subsection{Legacy Detection}
\label{sec:detection}

The directive |\childdocmain| in the main file can detect
whether the complete document or merely a child is to be compiled
even without using the directive |\childdocof|.
This method is deprecated because it is less robust
and there is no compelling reason to use it;
it is merely provided for backward compatibility
and it may be removed in future versions.

If the detection mechanism is to be used,
it is mandatory to correctly specify
the filename of the main file as the argument of |\childdocmain|:
%
\begin{center}
\begin{tabular}{l}
|\input{childdoc.def}|\\
|\childdocmain{|\textit{main}|}|\\
\end{tabular}
\end{center}
%
If |\jobname| does not match the argument \textit{main} of |\childdocmain|,
it is assumed that |\jobname| points to the child file to be compiled.
When using |\childdocmain| with the main file specified as argument,
it suffices to start a child file
with just |\input{|\textit{main}|}|
without loading of the package and using |\childdocof|.
If instead all processing is done
with the appropriate \textsf{childdoc} directives,
the argument of \textit{main} of |\childdocmain| can be empty.

An alternative version of the command line processing described
in \secref{sec:commandline} using the detection mechanism reads:
%
\begin{center}
|... -jobname "|\textit{target}|" "|[\textit{flags}]%
[|\def\jobname{|\textit{dest}|}|]|\input{|\textit{main}|}"|
\end{center}

%%%%%%%%%%%%%%%%%%%%%%%%%%%%%%%%%%%%%%%%%%%%%%%%%%%%%%%%%%%%%%%%%%%%%%%%%%%%%%%%
\subsection{Manual Code}
\label{sec:manual}

In case one cannot be certain whether the definitions file |childdoc.def|
is installed on the target \TeX{} distribution
and one prefers not to ship it,
it is conceivable to paste a few relevant commands into the sources.

To that end, drop all statements |\input{childdoc.def}|
and perform the replacements as outlined below.
Instead of |\childdocmain{|\textit{main}|}| add the following code
to the top of the main file:
%
\begin{center}
\begin{tabular}{l}
|\||ifdefined\childdocname\endinput\||fi\newif\ifchilddoc|\\
|\edef\childdocname{\scantokens\expandafter{\jobname\noexpand}}|\\
|\def\childdocmain{|\textit{main}|}\||ifx\childdocmain\childdocname\||else|\\
|\childdoctrue\includeonly{\childdocname}\let\jobname\childdocmain\||fi|\\
\end{tabular}
\end{center}
%
Instead of |\childdocof{|\textit{main}|}| just include the main file
at the top of each child file:
%
\begin{center}
|\input{|\textit{main}|}|
\end{center}
%
A simple redirection |\childdocforward{|\textit{dest}|}| is achieved by:
%
\begin{center}
|\def\jobname{|\textit{dest}|}\input{\jobname}|
\end{center}
%
The redirection with prefix
|\childdocforwardprefix[|\textit{prefix}|]{|\textit{dest}|}|
is accomplished by:
%
\begin{center}
\begin{tabular}{l}
|{\edef\jobname{\scantokens\expandafter{\jobname\noexpand}}|\\
|\def\redirectjob |\textit{prefix}|#1~~~{\gdef\jobname{|\textit{dest}|#1}}|\\
|\expandafter\redirectjob\jobname~~~}\input{\jobname}|
\end{tabular}
\end{center}

In an alternative approach,
child documents can be compiled by a specific command line
without additional code or specific definitions:
%
\begin{center}
|... -jobname "|\textit{target}|" "|[\textit{flags}]%
|\includeonly{|\textit{dest}|}\input{|\textit{main}|}"|
\end{center}
%

%%%%%%%%%%%%%%%%%%%%%%%%%%%%%%%%%%%%%%%%%%%%%%%%%%%%%%%%%%%%%%%%%%%%%%%%%%%%%%%%
%%%%%%%%%%%%%%%%%%%%%%%%%%%%%%%%%%%%%%%%%%%%%%%%%%%%%%%%%%%%%%%%%%%%%%%%%%%%%%%%
\section{Information}

%%%%%%%%%%%%%%%%%%%%%%%%%%%%%%%%%%%%%%%%%%%%%%%%%%%%%%%%%%%%%%%%%%%%%%%%%%%%%%%%
\subsection{Copyright}

Copyright \copyright{} 2017--2018 Niklas Beisert

This work may be distributed and/or modified under the
conditions of the \LaTeX{} Project Public License, either version 1.3
of this license or (at your option) any later version.
The latest version of this license is in
  \url{http://www.latex-project.org/lppl.txt}
and version 1.3 or later is part of all distributions of \LaTeX{}
version 2005/12/01 or later.

This work has the LPPL maintenance status `maintained'.

The Current Maintainer of this work is Niklas Beisert.

This work consists of the files |README.txt|, |childdoc.ins| and |childdoc.dtx|
as well as the derived files |childdoc.def|, |cdocsamp.tex|
with |cdocsch1.tex|, |cdocsch2.tex|, |cdocspt3.tex|, |cdocspt4.tex|,
|cdocsdrf.tex|, |cdocsfn1.tex|, |cdocsfn2.tex|
as well as |childdoc.pdf|.

%%%%%%%%%%%%%%%%%%%%%%%%%%%%%%%%%%%%%%%%%%%%%%%%%%%%%%%%%%%%%%%%%%%%%%%%%%%%%%%%
\subsection{Files and Installation}

The package consists of the files:
%
\begin{center}
\begin{tabular}{ll}
    |README.txt|   & readme file \\
    |childdoc.ins| & installation file \\
    |childdoc.dtx| & source file \\
    |childdoc.def| & definition file \\
    |cdocsamp.tex| & sample main file \\
    |cdocsch1.tex| & sample include file \\
    |cdocsch2.tex| & sample include file \\
    |cdocspt3.tex| & sample part file \\
    |cdocspt4.tex| & sample part file \\
    |cdocsdrf.tex| & sample redirection file \\
    |cdocsfn1.tex| & sample redirection file \\
    |cdocsfn2.tex| & sample redirection file \\
    |childdoc.pdf| & manual
\end{tabular}
\end{center}
%
The distribution consists of the files
|README.txt|, |childdoc.ins| and |childdoc.dtx|.
%
\begin{itemize}
\item
Run (pdf)\LaTeX{} on |childdoc.dtx|
to compile the manual |childdoc.pdf| (this file).
\item
Run \LaTeX{} on |childdoc.ins| to create the definitions file |childdoc.def|
and the sample |cdocsamp.tex| with include files
|cdocsch1.tex|, |cdocsch2.tex|, |cdocspt3.tex|, |cdocspt4.tex|,
|cdocsdrf.tex|, |cdocsfn1.tex|, |cdocsfn2.tex|.
Then copy the file |childdoc.def| to an appropriate directory of your \LaTeX{}
distribution, e.g.\ \textit{texmf-root}|/tex/latex/childdoc|.
\end{itemize}

%%%%%%%%%%%%%%%%%%%%%%%%%%%%%%%%%%%%%%%%%%%%%%%%%%%%%%%%%%%%%%%%%%%%%%%%%%%%%%%%
\subsection{Related CTAN Packages}

There are several other packages which offer a similar functionality:
%
\begin{itemize}
\item
The packages
\href{http://ctan.org/pkg/docmute}{\textsf{docmute}},
\href{http://ctan.org/pkg/includex}{\textsf{includex}} and
\href{http://ctan.org/pkg/standalone}{\textsf{standalone}}
provide commands to include only the document body of
a child file thus allowing both files to be compiled individually.
\item
The packages \href{http://ctan.org/pkg/subdocs}{\textsf{subdocs}}
and \href{http://ctan.org/pkg/subfiles}{\textsf{subfiles}}
provide structures in which the main and child documents can be
encapsulated and allowing them to be compiled individually.
The inclusion mechanism is different from the conventional |\include|.
\item
The package \href{http://ctan.org/pkg/combine}{\textsf{combine}}
is an elaborate solution to combine several documents into one.
\end{itemize}
%
See also the CTAN topic \href{http://ctan.org/topic/subdocs}{\textsf{subdocs}}
for further related packages.
The present package differs from the above solutions in that
a document structure constructed with the conventional |\include| mechanism
just needs two extra commands at the top of every file
such that all constituent files can be compiled individually.

%%%%%%%%%%%%%%%%%%%%%%%%%%%%%%%%%%%%%%%%%%%%%%%%%%%%%%%%%%%%%%%%%%%%%%%%%%%%%%%%
%\subsection{Feature Suggestions}
%
%The following is a list of features which may be useful for future
%versions of this package:
%%
%\begin{itemize}
%\item
%\ldots
%\end{itemize}

%%%%%%%%%%%%%%%%%%%%%%%%%%%%%%%%%%%%%%%%%%%%%%%%%%%%%%%%%%%%%%%%%%%%%%%%%%%%%%%%
\subsection{Revision History}

%%%%%%%%%%%%%%%%%%%%%%%%%%%%%%%%%%%%%%%%
\paragraph{v2.0:} 2018/12/30

\begin{itemize}
\item
immediate forward processing
\item
added |\childdocby| mechanism
\item
manual restructured
\end{itemize}

%%%%%%%%%%%%%%%%%%%%%%%%%%%%%%%%%%%%%%%%
\paragraph{v1.6:} 2018/01/17

\begin{itemize}
\item
application for development of include files
\item
corrections to manual
\end{itemize}

%%%%%%%%%%%%%%%%%%%%%%%%%%%%%%%%%%%%%%%%
\paragraph{v1.5:} 2017/05/21

\begin{itemize}
\item
more complete structuring introduced
\item
|\childdocof| introduced
\item
|\childdoc| renamed to |\childdocmain|
\item
|\childredirect| renamed to |\childdocforward| and |\childdocforwardprefix|
and functionality expanded
\end{itemize}

%%%%%%%%%%%%%%%%%%%%%%%%%%%%%%%%%%%%%%%%
\paragraph{v1.0:} 2017/04/27

\begin{itemize}
\item
manual and install package
\item
first version published on CTAN
\end{itemize}

%%%%%%%%%%%%%%%%%%%%%%%%%%%%%%%%%%%%%%%%
\paragraph{v0.6:} 2017/04/26

\begin{itemize}
\item
redirection mechanism added
\end{itemize}

%%%%%%%%%%%%%%%%%%%%%%%%%%%%%%%%%%%%%%%%
\paragraph{v0.5:} 2017/04/26

\begin{itemize}
\item
functionality in definition file
\end{itemize}


%%%%%%%%%%%%%%%%%%%%%%%%%%%%%%%%%%%%%%%%%%%%%%%%%%%%%%%%%%%%%%%%%%%%%%%%%%%%%%%%
%%%%%%%%%%%%%%%%%%%%%%%%%%%%%%%%%%%%%%%%%%%%%%%%%%%%%%%%%%%%%%%%%%%%%%%%%%%%%%%%
%%%%%%%%%%%%%%%%%%%%%%%%%%%%%%%%%%%%%%%%%%%%%%%%%%%%%%%%%%%%%%%%%%%%%%%%%%%%%%%%
\appendix

\settowidth\MacroIndent{\rmfamily\scriptsize 000\ }

 \DocInput{childdoc.dtx}

\end{document}
%</driver>
% \fi
%
% %%%%%%%%%%%%%%%%%%%%%%%%%%%%%%%%%%%%%%%%%%%%%%%%%%%%%%%%%%%%%%%%%%%%%%%%%%%%%%
% %%%%%%%%%%%%%%%%%%%%%%%%%%%%%%%%%%%%%%%%%%%%%%%%%%%%%%%%%%%%%%%%%%%%%%%%%%%%%%
% \section{Sample}
%\iffalse
%<*samplemain>
%\fi
%
% The following presents a sample document
% with two chapters, two parts, a title page,
% a compile flag as well as three forwarding files to set the flag.
% It consists of eight |.tex| files:
% \begin{center}
% \begin{tabular}{ll}
% |cdocsamp.tex|&main file\\
% |cdocsch1.tex|&include file for chapter 1\\
% |cdocsch2.tex|&include file for chapter 2\\
% |cdocspt3.tex|&include file for part 3\\
% |cdocspt4.tex|&include file for part 4\\
% |cdocsdrf.tex|&forwarding file for main file in draft mode\\
% |cdocsfi1.tex|&forwarding file for final version of chapter 1\\
% |cdocsfi2.tex|&forwarding file for final version of chapter 2\\
% \end{tabular}
% \end{center}
% Each of the eight files can be compiled directly by the \LaTeX{} compiler.
%
% %%%%%%%%%%%%%%%%%%%%%%%%%%%%%%%%%%%%%%
% \paragraph{Main File.}
%
% The main file is called |cdocsamp.tex|.
%
% Load the \textsf{childdoc} definitions and
% declare the filename for the main document:
%    \begin{macrocode}
\input{childdoc.def}
\childdocmain{}
%    \end{macrocode}

% Optional override for |\version| flag:
%    \begin{macrocode}
%%\ifchilddoc\else\providecommand{\version}{draft}\fi
%    \end{macrocode}

% Define the default values for the |\version| flag
% (|final| for the main file and |draft| for childs):
%    \begin{macrocode}
\ifchilddoc
\providecommand{\version}{draft}
\else
\providecommand{\version}{final}
\fi
%    \end{macrocode}

% Load the standard document class:
%    \begin{macrocode}
\documentclass[12pt]{article}
%    \end{macrocode}

% Start the document body:
%    \begin{macrocode}
\begin{document}
%    \end{macrocode}

% Declare a title page.
% Print title, part of document being processed and version flag:
%    \begin{macrocode}
\addtocounter{page}{-1}
\begin{center}
{\LARGE\bfseries{}childdoc example\par}
\vspace{1cm}
\ifchilddoc
\ifchilddocmanual part\else chapter\fi:
`\childdocname' of `\childdocjob'\par
\else
main document: `\childdocjob'\par
\fi
version: \version\par
\end{center}
\newpage
%    \end{macrocode}

% Manually include selected file,
% otherwise process as usual:
%    \begin{macrocode}
\ifchilddocmanual
\section*{part `\childdocname'}
\input{\childdocname}
\else
%    \end{macrocode}

% Include the two chapters:
%    \begin{macrocode}
\include{cdocsch1}
\include{cdocsch2}
%    \end{macrocode}

% Include the two parts unless only chapters should be displayed:
%    \begin{macrocode}
\ifchilddoc\else
\section{part three}
\input{cdocspt3}
\section{part four}
\input{cdocspt4}
\fi
%    \end{macrocode}

% Process as usual until here:
%    \begin{macrocode}
\fi
%    \end{macrocode}

% End of document body:
%    \begin{macrocode}
\end{document}
%    \end{macrocode}
%\iffalse
%</samplemain>
%\fi
%
% %%%%%%%%%%%%%%%%%%%%%%%%%%%%%%%%%%%%%%
% \paragraph{Chapter Include Files.}
%
% The include files are called |cdocsch1.tex| and |cdocsch2.tex|.
%
%\iffalse
%<*samplechap1|samplechap2>
%\fi

% Optional override for |\version| flag:
%    \begin{macrocode}
%%\providecommand{\version}{final}
%    \end{macrocode}

% Include the main document:
%    \begin{macrocode}
\input{childdoc.def}
\childdocof{cdocsamp}
%    \end{macrocode}

%\iffalse
%</samplechap1|samplechap2>
%\fi
%
%\iffalse
%<*samplechap1>
%\fi
% Some text for chapter 1:
%    \begin{macrocode}
\section{one}
some text in chapter one
%    \end{macrocode}

%\iffalse
%</samplechap1>
%\fi
% Some text for chapter 2:
%\iffalse
%<*samplechap2>
%\fi
%    \begin{macrocode}
\section{two}
more text in chapter two
%    \end{macrocode}

%\iffalse
%</samplechap2>
%\fi
%
% %%%%%%%%%%%%%%%%%%%%%%%%%%%%%%%%%%%%%%
% \paragraph{Part Include Files.}
%
% The include files are called |cdocspt3.tex| and |cdocspt4.tex|.
%
%\iffalse
%<*samplepart3|samplepart4>
%\fi

% Optional override for |\version| flag:
%    \begin{macrocode}
%%\providecommand{\version}{final}
%    \end{macrocode}

% Include the main document:
%    \begin{macrocode}
\input{childdoc.def}
\childdocby{cdocsamp}
%    \end{macrocode}

%\iffalse
%</samplepart3|samplepart4>
%\fi
%
%\iffalse
%<*samplepart3>
%\fi
% Some text for part 3:
%    \begin{macrocode}
some text in part three
%    \end{macrocode}

%\iffalse
%</samplepart3>
%\fi
% Some text for part 4:
%\iffalse
%<*samplepart4>
%\fi
%    \begin{macrocode}
more text in part four
%    \end{macrocode}

%\iffalse
%</samplepart4>
%\fi
%
% %%%%%%%%%%%%%%%%%%%%%%%%%%%%%%%%%%%%%%
% \paragraph{Forwarding for a Complete Draft.}
%
% The following forwarding file |cdocsdrf.tex|
% compiles the main document in draft mode:
%\iffalse
%<*sampledraft>
%\fi
%    \begin{macrocode}
\def\version{draft}
\input{childdoc.def}
\childdocforward{cdocsamp}
%    \end{macrocode}

%\iffalse
%</sampledraft>
%\fi
%
% %%%%%%%%%%%%%%%%%%%%%%%%%%%%%%%%%%%%%%
% \paragraph{Forwarding for Final Version of the Chapters.}
%
% The following forwarding files |cdocsfn1.tex| and |cdocsfn2.tex|
% (with identical content)
% compile the final versions of the child documents
% |cdocsch1.tex| and |cdocsch2.tex|, respectively:
%\iffalse
%<*samplefinal>
%\fi
%    \begin{macrocode}
\def\version{final}
\input{childdoc.def}
\childdocforwardprefix[cdocsamp]{cdocsfn}{cdocsch}
%    \end{macrocode}

%\iffalse
%</samplefinal>
%\fi
%
% %%%%%%%%%%%%%%%%%%%%%%%%%%%%%%%%%%%%%%
% \paragraph{Command Line Processing.}
%
% The following three command lines generate the output files
% |cdocscld|, |cdocscl1| and |cdocscl2|
% which should be identical to
% |cdocsdrf|, |cdocsch1| and |cdocsfn2|, respectively:
% \begin{center}
% \begin{tabular}{l}
% |latex -jobname cdocscld \|\\
% |  "\def\version{draft}\input{childdoc.def}\childdocforward{cdocsamp}"|\\
% |latex -jobname cdocscl1 \|\\
% |  "\input{childdoc.def}\childdocforward[cdocsamp]{cdocsch1}"|\\
% |latex -jobname cdocscl2 \|\\
% |  "\def\version{final}\input{childdoc.def}\childdocforward{cdocsch2}"|
% \end{tabular}
% \end{center}
% Note that the trailing backslash on each first line
% merely continues the input to the second line
% (for convenient cut ant paste).
% Furthermore, the command |latex| can be replaced by any
% of its alternative versions such as |pdflatex|.
%
% %%%%%%%%%%%%%%%%%%%%%%%%%%%%%%%%%%%%%%%%%%%%%%%%%%%%%%%%%%%%%%%%%%%%%%%%%%%%%%
% %%%%%%%%%%%%%%%%%%%%%%%%%%%%%%%%%%%%%%%%%%%%%%%%%%%%%%%%%%%%%%%%%%%%%%%%%%%%%%
% \section{Implementation}
%\iffalse
%<*package>
%\fi
%
% This section describes the definitions file |childdoc.def|.

% The definitions cannot be loaded using |\usepackage| or |\RequirePackage|
% which has a mechanism to prevent loading a style file more than once.
% When loading the definitions by means of |\input|
% multiple instances have to be prevented manually:
%\iffalse
%This code needs to be before the `\ProvidesFile' directive
%which is defined at the beginning of this file.
%Therefore it is also placed there and commented out here.
%</package>
%<*discard>
%\fi
%    \begin{macrocode}
\ifdefined\childdocmain\endinput\fi
%    \end{macrocode}
%\iffalse
%</discard>
%<*package>
%\fi
%
% \macro{\ifchilddoc}
% \macro{\ifchilddocmanual}
% The conditional |\ifchilddoc| tells whether a
% child (true) or main (false) document is being compiled.
% The conditional |\ifchilddocmanual| tells whether
% the |\includeonly| mechanism is used (false) or
% the selection of child files must be performed manually (true).
% The definitions initialise to false:
%    \begin{macrocode}
\newif\ifchilddoc
\newif\ifchilddocmanual
%    \end{macrocode}

% \macro{\childdocname}
% \macro{\childdocjob}
% The macro |\childdocname| stores the name of the main document
% to be compiled. The macro |\childdocjob| stores the name of
% the document on which the \LaTeX{} compiler was originally invoked.
% The content of |\jobname| cannot be compared
% to filenames specified in the source due to different catcodes.
% The following code rescans |\jobname|, stores the result
% in |\childdocname| and saves a copy in |\childdocjob|:
%    \begin{macrocode}
\edef\childdocname{\scantokens\expandafter{\jobname\noexpand}}
\let\childdocjob\childdocname
%    \end{macrocode}

% \macro{\childdocdisable}
% The macro |\childdocdisable| prevents the main file
% from being processed more than once.
% At this stage, the main document command |\childdocmain|
% is assumed to be called once again where it should do nothing.
% Any subsequent call to it should prevent
% a secondary processing of the main document
% It overwrites the forwarding commands
% |\childdocof| and |\childdocforward|
% with empty macros to prevent further inclusions of the main document:
%    \begin{macrocode}
\newcommand{\childdocdisable}
{
  \renewcommand{\childdocmain}[1]{\renewcommand{\childdocmain}[1]{\endinput}}
  \renewcommand{\childdocof}[1]{}
  \renewcommand{\childdocby}[2][]{}
  \renewcommand{\childdocforward}[2][]{}
  \renewcommand{\childdocdisable}{}
}
%    \end{macrocode}

% \macro{\childdocmain}
% The macro |\childdocmain| is to be called at the top of the main file
% with nothing or the main filename (without extension) as argument.
% First, it breaks loops.
% If the argument is not empty and does not match |\childdocname|
% (which is set by the first inclusion of |childdoc.def|),
% |\ifchilddoc| is set to true, |\includeonly| is applied to the child file
% and |\jobname| is set to the main file
% (for proper handling of |.aux| files):
%    \begin{macrocode}
\newcommand{\childdocmain}[1]
{
  \childdocdisable\childdocmain{}
  \if?#1?\else
    \begingroup
      \def\childdoctmp{#1}
      \ifx\childdoctmp\childdocname
        \def\childdoctmp{}
      \else
        \def\childdoctmp
        {
          \childdoctrue
          \includeonly{\childdocname}
          \def\childdocjob{#1}
          \def\jobname{#1}
        }
      \fi
      \expandafter
    \endgroup
    \childdoctmp
  \fi
}
%    \end{macrocode}

% \macro{\childdocof}
% The command |\childdocof| redirects
% compilation to the main file |#1|.
%    \begin{macrocode}
\newcommand{\childdocof}[1]
{
  \childdocdisable
  \childdoctrue
  \includeonly{\childdocname}
  \def\jobname{#1}
  \def\childdocjob{#1}
  \input{#1}
}
%    \end{macrocode}

% \macro{\childdocby}
% The command |\childdocby| ....
%    \begin{macrocode}
\newcommand{\childdocby}[2][]
{
  \childdocdisable
  \childdoctrue
  \childdocmanualtrue
  \if?#1?\else
    \def\jobname{#2}
  \fi
  \def\childdocjob{#2}
  \input{#2}
  \endinput
}
%    \end{macrocode}

% \macro{\childdocforward}
% The command |\childdocforward| redirects
% compilation to the main file or
% (if the optional argument is given) a child file.
% Parameters are set as if the main file
% or a child file starting with |\childdocof| was compiled.
% Then compilation is handed over to the main file:
%    \begin{macrocode}
\newcommand{\childdocforward}[2][]
{
  \begingroup
    \if?#1?
      \def\childdoctmp
      {
        \def\childdocname{#2}
        \def\childdocjob{#2}
        \def\jobname{#2}
        \input{#2}
        \endinput
      }
    \else
      \def\childdoctmp
      {
        \childdocdisable
        \def\childdocname{#2}
        \childdoctrue
        \includeonly{#2}
        \def\childdocjob{#1}
        \def\jobname{#1}
        \input{#1}
        \endinput
      }
    \fi
    \expandafter
  \endgroup
  \childdoctmp
}
%    \end{macrocode}

% \macro{\childdocforwardprefix}
% The command |\childdocforwardprefix| redirects
% compilation to the main or a child file by means of a pattern.
% The prefix |#1| in the current filename is replaced by |#2|
% and the suffix of the current filename is kept
% (it is assumed that the filename does not contain the substring `|~~~|'
% which is used as a delimiter).
% Compilation is handed over to the new file by |\childdocforward|:
%    \begin{macrocode}
\newcommand{\childdocforwardprefix}[3][]
{
  \begingroup
    \def\childdocextract #2##1~~~{\def\childdoctmp{\childdocforward[#1]{#3##1}}}
    \expandafter\childdocextract\childdocname~~~
    \expandafter
  \endgroup
  \childdoctmp
}
%    \end{macrocode}

% \macro{\childdoc}
% The deprecated macro |\childdoc| is a legacy version of |\childdocmain|:
%    \begin{macrocode}
\newcommand{\childdoc}{\childdocmain}
%    \end{macrocode}

% \macro{\childdocredirect}
% The deprecated macro |\childdocredirect| is a legacy version
% of |\childdocforward| and |\childdocforwardprefix|:
%    \begin{macrocode}
\newcommand{\childdocredirect}[2][]
{
  \begingroup
    \if?#1?
      \def\childdoctmp{\childdocforward{#2}}
    \else
      \def\childdoctmp{\childdocforwardprefix{#1}{#2}}
    \fi
    \expandafter
  \endgroup
  \childdoctmp
}
%    \end{macrocode}

%\iffalse
%</package>
%\fi
%
\endinput
|\\
|\childdocof{|\textit{main}|}|\\
\end{tabular}
\end{center}
at the top of every child file \textit{child}
which is included by |\include{|\textit{child}|}|
from within the main file
(or at least for those files to be compiled individually).
The argument \textit{main} must be the filename of the main file.

There are a couple of
considerations in setting up the main and child documents:

%%%%%%%%%%%%%%%%%%%%%%%%%%%%%%%%%%%%%%%%
\paragraph{Restrictions.}

Please note the following restrictions:
\begin{itemize}
\item
|\childdocmain| must be called with one argument \textit{main}
to ensure compatibility with earlier version of the package.
It must either be empty (|\childdocmain{}|)
or precisely match the filename of the main file in which it is specified.
See \secref{sec:detection} for further information.
\item
The filename \textit{main} must be specified without the |.tex| extension.
\item
The filename \textit{main} is case sensitive
(even in case-insensitive file systems)
due to internal string comparison.
\item
The argument \textit{main} should be fully expanded, it cannot be a macro.
\item
Subdirectories and special characters should be avoided in filenames.
\item
The command |\childdocmain{|\textit{main}|}| must be followed by a whitespace.
It should not be followed immediately by another command
or by a comment mark `|%|'.
This is because the \TeX{} parser reads the token immediately following
the argument of |\childdocmain| and puts it
at the beginning of every child section;
however, a white\-space is ignored.
\end{itemize}

%%%%%%%%%%%%%%%%%%%%%%%%%%%%%%%%%%%%%%%%
\paragraph{Content of Main File.}

It is advisable to place all content in the child files included by |\include|.
Any output contained in the main file will appear in all child documents
unless suppressed manually;
it cannot be suppressed automatically by the |\includeonly| directive
and thus should normally be avoided.
A method to include some content in the main file
by means of conditional processing is described in \secref{sec:conditional}.

%%%%%%%%%%%%%%%%%%%%%%%%%%%%%%%%%%%%%%%%
\paragraph{Page Numbering.}

When only a part of the document is compiled,
the appropriate numbering of pages
(as well as other status parameters)
is determined from the |.aux| files.
The latter contain information from previous passes.
However this information needs to propagate through
all intermediate child documents.
Therefore the page numbering in child documents may well
be inconsistent until the complete document is compiled at least once.

A useful (if unconventional) way to always ensure a consistent
page numbering is to restart the numbering in each child document
and denote the pages by `\textit{child}|.|\textit{page}'
where \textit{child} represents the chapter/section number of the child file.
This can be achieved by the command
|\numberwithin{page}{|\textit{child}|}|
of the \textsf{amsmath} package
where \textit{child} can be |chapter| or |section|
depending on the chosen structuring.
Alternatively, one can modify the macro |\thepage| appropriately
and reset the counter |page| at the start of each child file.

%%%%%%%%%%%%%%%%%%%%%%%%%%%%%%%%%%%%%%%%%%%%%%%%%%%%%%%%%%%%%%%%%%%%%%%%%%%%%%%%
\subsection{Conditional Processing}
\label{sec:conditional}

The package provides a mechanism to compile different versions
of a document. To customise the versions further some conditional processing
can come in handy to distinguish which version is being compiled.
The package provides two macros to describe the compilation context:

%%%%%%%%%%%%%%%%%%%%%%%%%%%%%%%%%%%%%%%%
\DescribeMacro{\ifchilddoc}
The conditional |\ifchilddoc| distinguishes between the compilation of
child documents and the main document:
%
\begin{center}
|\ifchilddoc |\textit{child-code}| |[|\||else |\textit{main-code}]| \||fi|
\end{center}

%%%%%%%%%%%%%%%%%%%%%%%%%%%%%%%%%%%%%%%%
\DescribeMacro{\childdocname}
\DescribeMacro{\childdocjob}
The macro |\childdocname| contains the filename (without extension)
of the main or child file being processed.
Note that |\childdocjob| will always contain the name of the main file.

%%%%%%%%%%%%%%%%%%%%%%%%%%%%%%%%%%%%%%%%
\paragraph{Title Page.}

Conditional processing can be used to include a title or banner page
in the main document when proper precautions are taken.
Importantly, the code in the main file should ensure that the page counter
(as well as other status parameters which are stored in the |.aux| files)
takes the same value after the conditional processing.
Otherwise the page numbers may take divergent values
depending on which part is compiled.

For example, a title page could be declared by:
%
\begin{center}
\begin{tabular}{l}
|\ifchilddoc\||else|\\
|\addtocounter{page}{-1}|\\
\textit{code for title page}\\
|\newpage|\\
|\||fi|
\end{tabular}
\end{center}
%
A banner page for the child documents can be generated by:
%
\begin{center}
\begin{tabular}{l}
|\ifchilddoc|\\
|\addtocounter{page}{-1}|\\
\textit{code for banner page}\\
|\newpage|\\
|\||fi|
\end{tabular}
\end{center}
%
Here one could write a message such as:
\begin{center}
|This is the part \childdocname{} of \childdocjob{}.|
\end{center}

%%%%%%%%%%%%%%%%%%%%%%%%%%%%%%%%%%%%%%%%%%%%%%%%%%%%%%%%%%%%%%%%%%%%%%%%%%%%%%%%
\subsection{Flags}
\label{sec:flags}

The package makes it easy to generate different versions
of the main or child documents.
To this end compilation flags can be defined
and assigned different default values.
They will be particularly useful in conjunction
with the forwarding mechanism described in \secref{sec:forward}.

For example, it may be useful to have a flag |\version|
which can be set to |draft| or |final|.
The document source will contain some conditional code
depending on the value of |\version|.
Suppose further, the flag should default to |final| for the main file
and to |draft| for child files
which is a natural assignment for editing the document.
This is achieved by placing the following code
in the preamble of the main document
(below the |\childdocmain| directive):
%
\begin{center}
\begin{tabular}{l}
|\ifchilddoc|\\
|\providecommand{\version}{draft}|\\
|\||else|\\
|\providecommand{\version}{final}|\\
|\||fi|
\end{tabular}
\end{center}
%
The definition by |\providecommand| makes sure
that previous definitions are not overwritten.
Further statements |\providecommand{\version}{...}|
can thus be added before the above code to override it.

For the main file, one might add a line
(between |\childdocmain| and the above block)
%
\begin{center}
|%\ifchilddoc\||else\providecommand{\version}{draft}\||fi|
\end{center}
%
which can be uncommented to produce a draft version.
Likewise one can add a line to the very top of a child file
(above the |\childdocof{|\textit{main}|}| directive)
%
\begin{center}
|%\providecommand{\version}{final}|
\end{center}
%
which can be uncommented to produce the final version of this child document.

%%%%%%%%%%%%%%%%%%%%%%%%%%%%%%%%%%%%%%%%%%%%%%%%%%%%%%%%%%%%%%%%%%%%%%%%%%%%%%%%
\subsection{Forwarding}
\label{sec:forward}

Different versions of the main or child documents
using compilation flags as described in \secref{sec:flags}
can be (permanently) stored in different files
for convenient compilation, viewing and distribution.
To this end, the package defines a command
to pass on compilation to a different file:

%%%%%%%%%%%%%%%%%%%%%%%%%%%%%%%%%%%%%%%%
\DescribeMacro{\childdocforward}
The command |\childdocforward| redirects processing to
another source file:
%
\begin{center}
\begin{tabular}{l}
|% \iffalse
%
% childdoc.dtx Copyright (C) 2017-2018 Niklas Beisert
%
% This work may be distributed and/or modified under the
% conditions of the LaTeX Project Public License, either version 1.3
% of this license or (at your option) any later version.
% The latest version of this license is in
%   http://www.latex-project.org/lppl.txt
% and version 1.3 or later is part of all distributions of LaTeX
% version 2005/12/01 or later.
%
% This work has the LPPL maintenance status `maintained'.
%
% The Current Maintainer of this work is Niklas Beisert.
%
% This work consists of the files childdoc.dtx and childdoc.ins
% and the derived files childdoc.def and cdocsamp.tex with
% cdocsch1.tex, cdocsch2.tex, cdocsdrf.tex, cdocsfn1.tex, cdocsfn2.tex.
%
%<package>\ifdefined\childdocmain\endinput\fi
%<package>\ProvidesFile{childdoc.def}[2018/12/30 v2.0 child document driver]
%<samplemain>\ProvidesFile{cdocsamp.tex}[2018/12/30 v2.0 sample for childdoc]
%<*driver>
%\ProvidesFile{childdoc.drv}[2018/12/30 v2.0 childdoc reference manual file]
\PassOptionsToClass{10pt,a4paper}{article}
\documentclass{ltxdoc}

\usepackage[margin=35mm]{geometry}
\usepackage{hyperref}
\usepackage{hyperxmp}
\usepackage[usenames]{color}

\hypersetup{colorlinks=true}
\hypersetup{pdfstartview=FitH}
\hypersetup{pdfpagemode=UseNone}
\hypersetup{pdfsource={}}
\hypersetup{pdflang={en-UK}}
\hypersetup{pdfcopyright={Copyright 2017-2018 Niklas Beisert.
  This work may be distributed and/or modified under the
  conditions of the LaTeX Project Public License, either version 1.3
  of this license or (at your option) any later version.}}
\hypersetup{pdflicenseurl={http://www.latex-project.org/lppl.txt}}
\hypersetup{pdfcontactaddress={ETH Zurich, ITP, HIT K,
  Wolfgang-Pauli-Strasse 27}}
\hypersetup{pdfcontactpostcode={8093}}
\hypersetup{pdfcontactcity={Zurich}}
\hypersetup{pdfcontactcountry={Switzerland}}
\hypersetup{pdfcontactemail={nbeisert@itp.phys.ethz.ch}}
\hypersetup{pdfcontacturl={http://people.phys.ethz.ch/\xmptilde nbeisert/}}

\newcommand{\secref}[1]{\hyperref[#1]{section \ref*{#1}}}

\parskip1ex
\parindent0pt
\let\olditemize\itemize
\def\itemize{\olditemize\parskip0pt}

\begin{document}

\title{The \textsf{childdoc} Package}
\hypersetup{pdftitle={The childdoc Package}}
\author{Niklas Beisert\\[2ex]
  Institut f\"ur Theoretische Physik\\
  Eidgen\"ossische Technische Hochschule Z\"urich\\
  Wolfgang-Pauli-Strasse 27, 8093 Z\"urich, Switzerland\\[1ex]
  \href{mailto:nbeisert@itp.phys.ethz.ch}
  {\texttt{nbeisert@itp.phys.ethz.ch}}}
\hypersetup{pdfauthor={Niklas Beisert}}
\hypersetup{pdfsubject={Manual for the LaTeX2e Package childdoc}}
\date{30 December 2018, \textsf{v2.0}}
\maketitle

\begin{abstract}\noindent
\textsf{childdoc} is a \LaTeXe{} package
that enables the direct compilation
of document sections included by |\include|
to individual files.
\end{abstract}

\begingroup
\parskip0ex
\tableofcontents
\endgroup

%%%%%%%%%%%%%%%%%%%%%%%%%%%%%%%%%%%%%%%%%%%%%%%%%%%%%%%%%%%%%%%%%%%%%%%%%%%%%%%%
%%%%%%%%%%%%%%%%%%%%%%%%%%%%%%%%%%%%%%%%%%%%%%%%%%%%%%%%%%%%%%%%%%%%%%%%%%%%%%%%
\section{Introduction}

\LaTeX{} provides a mechanism to structure a large document (such as a book)
into a main file and several child files (containing the chapters)
using the |\include| command.
This mechanism is beneficial for documents
which span hundreds of pages in order to
make the source file(s) more manageable.
Moreover, compilation can be restricted to
selected child files by means of the |\includeonly| command.
The latter feature can be used to reduce the compilation time while editing
(this was significantly more useful in the earlier days of \LaTeX{})
or to generate a smaller document which is easier to navigate.
Another application of |\includeonly| is to generate
documents consisting of selected parts of the complete document.

However, there are a few drawbacks of the plain |\include| mechanism:
\begin{itemize}
\item
The child files cannot be compiled on their own,
they can only be compiled via the main file.
A naive editing environment
(such as a text editor with an option
to have the current file processed by \LaTeX)
may require one to switch to the main file before compiling;
attempting to compile the child file produces errors.
\item
The main file must be modified (each time)
to adjust the |\includeonly| command
to the present needs. This easily leaves the main file in a messy state.
\item
The generated document will always carry the filename
of the main document. This is inconvenient if
several child files are to be compiled and
to be kept for distribution.
\end{itemize}

The present package provides a simple interface
to make child files individually compilable by \LaTeX{}.
Compiling a child file then has the same effect as compiling
the main file with an |\includeonly| command
to select the appropriate child.
Moreover the generated document will carry the name of the child
rather than the main file.
This resolves all three above issues.

This feature is meant to make the editing of books,
thesis documents and lecture notes somewhat more convenient.
However, the package can also be used efficiently for
composing a series of documents (such as exercise sheets)
which are typically distributed individually.
It then assists the author in generating the individual documents
(potentially in different versions)
as well as a document containing the collected series.
Another application is in developing style files
or other kinds of included material
where compilation of the style file could redirect
to a sample or test file.

%%%%%%%%%%%%%%%%%%%%%%%%%%%%%%%%%%%%%%%%%%%%%%%%%%%%%%%%%%%%%%%%%%%%%%%%%%%%%%%%
%%%%%%%%%%%%%%%%%%%%%%%%%%%%%%%%%%%%%%%%%%%%%%%%%%%%%%%%%%%%%%%%%%%%%%%%%%%%%%%%
\section{Usage}

First of all, the package \textsf{childdoc} is \emph{not} a standard
\LaTeXe{} |.sty| style file! Therefore it needs to be invoked in
a non-standard way.

%%%%%%%%%%%%%%%%%%%%%%%%%%%%%%%%%%%%%%%%%%%%%%%%%%%%%%%%%%%%%%%%%%%%%%%%%%%%%%%%
\subsection{Included Files}
\label{sec:include}

%%%%%%%%%%%%%%%%%%%%%%%%%%%%%%%%%%%%%%%%
\DescribeMacro{\childdocmain}
To use the package, add the commands
\begin{center}
\begin{tabular}{l}
|\input{childdoc.def}|\\
|\childdocmain{}|\\
\end{tabular}
\end{center}
at the very top of the main \LaTeX{} file,
in particular \emph{before} the |\documentclass| statement!
The argument of |\childdocmain| should be left empty
(but it must be present).

%%%%%%%%%%%%%%%%%%%%%%%%%%%%%%%%%%%%%%%%
\DescribeMacro{\childdocof}
Furthermore, add the commands
\begin{center}
\begin{tabular}{l}
|\input{childdoc.def}|\\
|\childdocof{|\textit{main}|}|\\
\end{tabular}
\end{center}
at the top of every child file \textit{child}
which is included by |\include{|\textit{child}|}|
from within the main file
(or at least for those files to be compiled individually).
The argument \textit{main} must be the filename of the main file.

There are a couple of
considerations in setting up the main and child documents:

%%%%%%%%%%%%%%%%%%%%%%%%%%%%%%%%%%%%%%%%
\paragraph{Restrictions.}

Please note the following restrictions:
\begin{itemize}
\item
|\childdocmain| must be called with one argument \textit{main}
to ensure compatibility with earlier version of the package.
It must either be empty (|\childdocmain{}|)
or precisely match the filename of the main file in which it is specified.
See \secref{sec:detection} for further information.
\item
The filename \textit{main} must be specified without the |.tex| extension.
\item
The filename \textit{main} is case sensitive
(even in case-insensitive file systems)
due to internal string comparison.
\item
The argument \textit{main} should be fully expanded, it cannot be a macro.
\item
Subdirectories and special characters should be avoided in filenames.
\item
The command |\childdocmain{|\textit{main}|}| must be followed by a whitespace.
It should not be followed immediately by another command
or by a comment mark `|%|'.
This is because the \TeX{} parser reads the token immediately following
the argument of |\childdocmain| and puts it
at the beginning of every child section;
however, a white\-space is ignored.
\end{itemize}

%%%%%%%%%%%%%%%%%%%%%%%%%%%%%%%%%%%%%%%%
\paragraph{Content of Main File.}

It is advisable to place all content in the child files included by |\include|.
Any output contained in the main file will appear in all child documents
unless suppressed manually;
it cannot be suppressed automatically by the |\includeonly| directive
and thus should normally be avoided.
A method to include some content in the main file
by means of conditional processing is described in \secref{sec:conditional}.

%%%%%%%%%%%%%%%%%%%%%%%%%%%%%%%%%%%%%%%%
\paragraph{Page Numbering.}

When only a part of the document is compiled,
the appropriate numbering of pages
(as well as other status parameters)
is determined from the |.aux| files.
The latter contain information from previous passes.
However this information needs to propagate through
all intermediate child documents.
Therefore the page numbering in child documents may well
be inconsistent until the complete document is compiled at least once.

A useful (if unconventional) way to always ensure a consistent
page numbering is to restart the numbering in each child document
and denote the pages by `\textit{child}|.|\textit{page}'
where \textit{child} represents the chapter/section number of the child file.
This can be achieved by the command
|\numberwithin{page}{|\textit{child}|}|
of the \textsf{amsmath} package
where \textit{child} can be |chapter| or |section|
depending on the chosen structuring.
Alternatively, one can modify the macro |\thepage| appropriately
and reset the counter |page| at the start of each child file.

%%%%%%%%%%%%%%%%%%%%%%%%%%%%%%%%%%%%%%%%%%%%%%%%%%%%%%%%%%%%%%%%%%%%%%%%%%%%%%%%
\subsection{Conditional Processing}
\label{sec:conditional}

The package provides a mechanism to compile different versions
of a document. To customise the versions further some conditional processing
can come in handy to distinguish which version is being compiled.
The package provides two macros to describe the compilation context:

%%%%%%%%%%%%%%%%%%%%%%%%%%%%%%%%%%%%%%%%
\DescribeMacro{\ifchilddoc}
The conditional |\ifchilddoc| distinguishes between the compilation of
child documents and the main document:
%
\begin{center}
|\ifchilddoc |\textit{child-code}| |[|\||else |\textit{main-code}]| \||fi|
\end{center}

%%%%%%%%%%%%%%%%%%%%%%%%%%%%%%%%%%%%%%%%
\DescribeMacro{\childdocname}
\DescribeMacro{\childdocjob}
The macro |\childdocname| contains the filename (without extension)
of the main or child file being processed.
Note that |\childdocjob| will always contain the name of the main file.

%%%%%%%%%%%%%%%%%%%%%%%%%%%%%%%%%%%%%%%%
\paragraph{Title Page.}

Conditional processing can be used to include a title or banner page
in the main document when proper precautions are taken.
Importantly, the code in the main file should ensure that the page counter
(as well as other status parameters which are stored in the |.aux| files)
takes the same value after the conditional processing.
Otherwise the page numbers may take divergent values
depending on which part is compiled.

For example, a title page could be declared by:
%
\begin{center}
\begin{tabular}{l}
|\ifchilddoc\||else|\\
|\addtocounter{page}{-1}|\\
\textit{code for title page}\\
|\newpage|\\
|\||fi|
\end{tabular}
\end{center}
%
A banner page for the child documents can be generated by:
%
\begin{center}
\begin{tabular}{l}
|\ifchilddoc|\\
|\addtocounter{page}{-1}|\\
\textit{code for banner page}\\
|\newpage|\\
|\||fi|
\end{tabular}
\end{center}
%
Here one could write a message such as:
\begin{center}
|This is the part \childdocname{} of \childdocjob{}.|
\end{center}

%%%%%%%%%%%%%%%%%%%%%%%%%%%%%%%%%%%%%%%%%%%%%%%%%%%%%%%%%%%%%%%%%%%%%%%%%%%%%%%%
\subsection{Flags}
\label{sec:flags}

The package makes it easy to generate different versions
of the main or child documents.
To this end compilation flags can be defined
and assigned different default values.
They will be particularly useful in conjunction
with the forwarding mechanism described in \secref{sec:forward}.

For example, it may be useful to have a flag |\version|
which can be set to |draft| or |final|.
The document source will contain some conditional code
depending on the value of |\version|.
Suppose further, the flag should default to |final| for the main file
and to |draft| for child files
which is a natural assignment for editing the document.
This is achieved by placing the following code
in the preamble of the main document
(below the |\childdocmain| directive):
%
\begin{center}
\begin{tabular}{l}
|\ifchilddoc|\\
|\providecommand{\version}{draft}|\\
|\||else|\\
|\providecommand{\version}{final}|\\
|\||fi|
\end{tabular}
\end{center}
%
The definition by |\providecommand| makes sure
that previous definitions are not overwritten.
Further statements |\providecommand{\version}{...}|
can thus be added before the above code to override it.

For the main file, one might add a line
(between |\childdocmain| and the above block)
%
\begin{center}
|%\ifchilddoc\||else\providecommand{\version}{draft}\||fi|
\end{center}
%
which can be uncommented to produce a draft version.
Likewise one can add a line to the very top of a child file
(above the |\childdocof{|\textit{main}|}| directive)
%
\begin{center}
|%\providecommand{\version}{final}|
\end{center}
%
which can be uncommented to produce the final version of this child document.

%%%%%%%%%%%%%%%%%%%%%%%%%%%%%%%%%%%%%%%%%%%%%%%%%%%%%%%%%%%%%%%%%%%%%%%%%%%%%%%%
\subsection{Forwarding}
\label{sec:forward}

Different versions of the main or child documents
using compilation flags as described in \secref{sec:flags}
can be (permanently) stored in different files
for convenient compilation, viewing and distribution.
To this end, the package defines a command
to pass on compilation to a different file:

%%%%%%%%%%%%%%%%%%%%%%%%%%%%%%%%%%%%%%%%
\DescribeMacro{\childdocforward}
The command |\childdocforward| redirects processing to
another source file:
%
\begin{center}
\begin{tabular}{l}
|\input{childdoc.def}|\\
|\childdocforward[|\textit{main}|]{|\textit{dest}|}|\\
\end{tabular}
\end{center}
%
The argument \textit{dest} is the destination file
(without extension).
It should be the main file or one of the child files.
Note that further \textsf{childdoc} directives
such as |\childdocof| and |\childdocforward|
in the indicated file will be processed in this form.
The optional argument \textit{main}
passes on directly to the main file \textit{main}
while pretending to compile the child \textit{dest}.
This form behaves as if \textit{dest}
issues |\childdocof{|\textit{main}|}| right away,
and no further \textsf{childdoc} directives will be processed.

%%%%%%%%%%%%%%%%%%%%%%%%%%%%%%%%%%%%%%%%
\DescribeMacro{\...prefix}
In the alternative form |\childdocforwardprefix|,
%
\begin{center}
\begin{tabular}{l}
|\input{childdoc.def}|\\
|\childdocforwardprefix[|\textit{main}|]{|\textit{prefix}|}{|\textit{dest}|}|
\end{tabular}
\end{center}
%
the destination file is determined by a pattern
depending on the current file:
To make this work, the current file must be called
`{\textit{prefix}\hspace{0.2em}\textit{suffix}}'
with \textit{prefix} matching precisely the argument.
Processing is then passed on to the file
`{\textit{dest}\hspace{0.2em}\textit{suffix}}'.
Surely, the same effect is achieved by
directly specifying the
argument `{\textit{dest}\hspace{0.2em}\textit{suffix}}'
in the first form.
However, that requires to set up a different file
for each child. With the alternative form of the command
all these files can have exactly the same content
which simplifies setting them up and maintaining them.

For example, the following file |draft.tex|
with a compilation flag |\version| as described in \secref{sec:flags}
compiles the main document as a draft:
%
\begin{center}
\begin{tabular}{l}
|\def\version{draft}|\\
|\input{childdoc.def}|\\
|\childdocforward{|\textit{main}|}|
\end{tabular}
\end{center}
%
Likewise, the following files |final|\textit{nn}|.tex|
compile the final version of the child document
|child|\textit{nn}|.tex|:
%
\begin{center}
\begin{tabular}{l}
|\def\version{final}|\\
|\input{childdoc.def}|\\
|\childdocforwardprefix{final}{child}|
\end{tabular}
\end{center}
%

Note that when several versions of a main file and/or of each child file
are to be generated, it may be convenient to set up a |Makefile| or
shell script to automatise the process.

%%%%%%%%%%%%%%%%%%%%%%%%%%%%%%%%%%%%%%%%%%%%%%%%%%%%%%%%%%%%%%%%%%%%%%%%%%%%%%%%
\subsection{Command Line Processing}
\label{sec:commandline}

The effect of redirection files can also be achieved by invoking
the \LaTeX{} compiler with a more elaborate command line.
Most conveniently this should be done as part
of a shell script or a |Makefile|.

When using \textsf{childdoc} in the main file, the following
command lines effectively perform a redirection
(note that depending on the shell being used,
backslashes may have to be doubled: `|\|' $\to$ `|\\|'):
%
\begin{center}
|... -jobname "|\textit{target}|" |\\|"|[\textit{flags}]%
|\input{childdoc.def}\childdocforward[|\textit{main}|]{|\textit{dest}|}"|
\end{center}
%
Here \textit{target} is the name of the output file,
\textit{main} is the name of the main file
and \textit{dest} is the name of the main or child file to be processed
(all filenames without extensions).
The optional argument \textit{main} can be omitted
if \textit{main} matches \textit{dest}.
Optionally, compilation \textit{flags} can be defined via |\def| commands.
This command line makes the \TeX{} engine believe
it is compiling the file \textit{target}
whose content is specified as the latter parameter.
The provided code then forwards the processing to
\textit{main} or \textit{dest} as described in \secref{sec:forward}.

%%%%%%%%%%%%%%%%%%%%%%%%%%%%%%%%%%%%%%%%%%%%%%%%%%%%%%%%%%%%%%%%%%%%%%%%%%%%%%%%
\subsection{Include by Input}
\label{sec:input}

Including child documents by |\include| has some restrictions by design.
Most notably, the content of a child document always occupies
its own set of pages; pages cannot be shared between child documents.
Usually, this behaviour makes perfect sense
because each child document contain an essential part of the document.
However, in some situations it may be desirable to compose
a document from a collection of parts
without having mandatory page breaks between then.
For this case, the package
provides a mechanism to include parts
by |\input| which can also be processed individually.
However, by construction this mechanism
requires manual handling of the content to be output.

%%%%%%%%%%%%%%%%%%%%%%%%%%%%%%%%%%%%%%%%
\DescribeMacro{\ifchilddocmanual}
The main file should be prepared as usual, see \secref{sec:include}.
However, the document body must make a distinction
between processing of an individual part and of the main document, e.g.:
%
\begin{center}
\begin{tabular}{l}
|\ifchilddocmanual|\\
|\input{\childdocname}|\\
|\||else|\\
\textit{document body with }|\input{|\textit{part}|}|\\
|\||fi|
\end{tabular}
\end{center}
%
The conditional |\ifchilddocmanual| is true whenever
a part to be included by |\input| is being compiled,
and the name of the part is stored in |\childdocname|.

%%%%%%%%%%%%%%%%%%%%%%%%%%%%%%%%%%%%%%%%
\DescribeMacro{\childdocby}
Each part to be included by |\input| should start with:
%
\begin{center}
\begin{tabular}{l}
|\input{childdoc.def}|\\
|\childdocby{|\textit{main}|}|\\
\end{tabular}
\end{center}
%
The directive |\childdocby| is similar to |\childdocof|
described in \secref{sec:include},
but the subsequent selection of content must be done manually.
To that end, both |\ifchilddoc| and |\ifchilddocmanual|
will be true upon processing of a part,
and the name of the part is stored in |\childdocname|.
Note that |\jobname| will be set to the filename of the current part
so that each part receives an individual |.aux| file
that does not interfere with the |.aux| file(s) of the main document.
This behaviour can be altered by the alternative form
|\childdocby[*]{|\textit{main}|}| (with a non-empty optional argument)
which uses the |.aux| file of the main document
by setting |\jobname| to \textit{main}.

%%%%%%%%%%%%%%%%%%%%%%%%%%%%%%%%%%%%%%%%%%%%%%%%%%%%%%%%%%%%%%%%%%%%%%%%%%%%%%%%
\subsection{Driver Development}
\label{sec:driver}

The \textsf{childdoc} mechanism can also be use for the development
of definition files such as \LaTeX{} styles or classes.
This case differs from the above setup with multiple parts
included by |\include| in that no |\includeonly| should be invoked.
This can be achieved by starting the include file
(before |\ProvidesPackage|) with:
%
\begin{center}
\begin{tabular}{l}
|\input{childdoc.def}|\\
|\childdocforward{|\textit{main}|}|\\
\end{tabular}
\end{center}
%
or alternatively with:
%
\begin{center}
\begin{tabular}{l}
|\input{childdoc.def}|\\
|\childdocby{|\textit{main}|}|\\
\end{tabular}
\end{center}
%
Both forms have slightly different effects as described above.
The main file is prepared as usual, see \secref{sec:include}.

%%%%%%%%%%%%%%%%%%%%%%%%%%%%%%%%%%%%%%%%%%%%%%%%%%%%%%%%%%%%%%%%%%%%%%%%%%%%%%%%
\subsection{Legacy Detection}
\label{sec:detection}

The directive |\childdocmain| in the main file can detect
whether the complete document or merely a child is to be compiled
even without using the directive |\childdocof|.
This method is deprecated because it is less robust
and there is no compelling reason to use it;
it is merely provided for backward compatibility
and it may be removed in future versions.

If the detection mechanism is to be used,
it is mandatory to correctly specify
the filename of the main file as the argument of |\childdocmain|:
%
\begin{center}
\begin{tabular}{l}
|\input{childdoc.def}|\\
|\childdocmain{|\textit{main}|}|\\
\end{tabular}
\end{center}
%
If |\jobname| does not match the argument \textit{main} of |\childdocmain|,
it is assumed that |\jobname| points to the child file to be compiled.
When using |\childdocmain| with the main file specified as argument,
it suffices to start a child file
with just |\input{|\textit{main}|}|
without loading of the package and using |\childdocof|.
If instead all processing is done
with the appropriate \textsf{childdoc} directives,
the argument of \textit{main} of |\childdocmain| can be empty.

An alternative version of the command line processing described
in \secref{sec:commandline} using the detection mechanism reads:
%
\begin{center}
|... -jobname "|\textit{target}|" "|[\textit{flags}]%
[|\def\jobname{|\textit{dest}|}|]|\input{|\textit{main}|}"|
\end{center}

%%%%%%%%%%%%%%%%%%%%%%%%%%%%%%%%%%%%%%%%%%%%%%%%%%%%%%%%%%%%%%%%%%%%%%%%%%%%%%%%
\subsection{Manual Code}
\label{sec:manual}

In case one cannot be certain whether the definitions file |childdoc.def|
is installed on the target \TeX{} distribution
and one prefers not to ship it,
it is conceivable to paste a few relevant commands into the sources.

To that end, drop all statements |\input{childdoc.def}|
and perform the replacements as outlined below.
Instead of |\childdocmain{|\textit{main}|}| add the following code
to the top of the main file:
%
\begin{center}
\begin{tabular}{l}
|\||ifdefined\childdocname\endinput\||fi\newif\ifchilddoc|\\
|\edef\childdocname{\scantokens\expandafter{\jobname\noexpand}}|\\
|\def\childdocmain{|\textit{main}|}\||ifx\childdocmain\childdocname\||else|\\
|\childdoctrue\includeonly{\childdocname}\let\jobname\childdocmain\||fi|\\
\end{tabular}
\end{center}
%
Instead of |\childdocof{|\textit{main}|}| just include the main file
at the top of each child file:
%
\begin{center}
|\input{|\textit{main}|}|
\end{center}
%
A simple redirection |\childdocforward{|\textit{dest}|}| is achieved by:
%
\begin{center}
|\def\jobname{|\textit{dest}|}\input{\jobname}|
\end{center}
%
The redirection with prefix
|\childdocforwardprefix[|\textit{prefix}|]{|\textit{dest}|}|
is accomplished by:
%
\begin{center}
\begin{tabular}{l}
|{\edef\jobname{\scantokens\expandafter{\jobname\noexpand}}|\\
|\def\redirectjob |\textit{prefix}|#1~~~{\gdef\jobname{|\textit{dest}|#1}}|\\
|\expandafter\redirectjob\jobname~~~}\input{\jobname}|
\end{tabular}
\end{center}

In an alternative approach,
child documents can be compiled by a specific command line
without additional code or specific definitions:
%
\begin{center}
|... -jobname "|\textit{target}|" "|[\textit{flags}]%
|\includeonly{|\textit{dest}|}\input{|\textit{main}|}"|
\end{center}
%

%%%%%%%%%%%%%%%%%%%%%%%%%%%%%%%%%%%%%%%%%%%%%%%%%%%%%%%%%%%%%%%%%%%%%%%%%%%%%%%%
%%%%%%%%%%%%%%%%%%%%%%%%%%%%%%%%%%%%%%%%%%%%%%%%%%%%%%%%%%%%%%%%%%%%%%%%%%%%%%%%
\section{Information}

%%%%%%%%%%%%%%%%%%%%%%%%%%%%%%%%%%%%%%%%%%%%%%%%%%%%%%%%%%%%%%%%%%%%%%%%%%%%%%%%
\subsection{Copyright}

Copyright \copyright{} 2017--2018 Niklas Beisert

This work may be distributed and/or modified under the
conditions of the \LaTeX{} Project Public License, either version 1.3
of this license or (at your option) any later version.
The latest version of this license is in
  \url{http://www.latex-project.org/lppl.txt}
and version 1.3 or later is part of all distributions of \LaTeX{}
version 2005/12/01 or later.

This work has the LPPL maintenance status `maintained'.

The Current Maintainer of this work is Niklas Beisert.

This work consists of the files |README.txt|, |childdoc.ins| and |childdoc.dtx|
as well as the derived files |childdoc.def|, |cdocsamp.tex|
with |cdocsch1.tex|, |cdocsch2.tex|, |cdocspt3.tex|, |cdocspt4.tex|,
|cdocsdrf.tex|, |cdocsfn1.tex|, |cdocsfn2.tex|
as well as |childdoc.pdf|.

%%%%%%%%%%%%%%%%%%%%%%%%%%%%%%%%%%%%%%%%%%%%%%%%%%%%%%%%%%%%%%%%%%%%%%%%%%%%%%%%
\subsection{Files and Installation}

The package consists of the files:
%
\begin{center}
\begin{tabular}{ll}
    |README.txt|   & readme file \\
    |childdoc.ins| & installation file \\
    |childdoc.dtx| & source file \\
    |childdoc.def| & definition file \\
    |cdocsamp.tex| & sample main file \\
    |cdocsch1.tex| & sample include file \\
    |cdocsch2.tex| & sample include file \\
    |cdocspt3.tex| & sample part file \\
    |cdocspt4.tex| & sample part file \\
    |cdocsdrf.tex| & sample redirection file \\
    |cdocsfn1.tex| & sample redirection file \\
    |cdocsfn2.tex| & sample redirection file \\
    |childdoc.pdf| & manual
\end{tabular}
\end{center}
%
The distribution consists of the files
|README.txt|, |childdoc.ins| and |childdoc.dtx|.
%
\begin{itemize}
\item
Run (pdf)\LaTeX{} on |childdoc.dtx|
to compile the manual |childdoc.pdf| (this file).
\item
Run \LaTeX{} on |childdoc.ins| to create the definitions file |childdoc.def|
and the sample |cdocsamp.tex| with include files
|cdocsch1.tex|, |cdocsch2.tex|, |cdocspt3.tex|, |cdocspt4.tex|,
|cdocsdrf.tex|, |cdocsfn1.tex|, |cdocsfn2.tex|.
Then copy the file |childdoc.def| to an appropriate directory of your \LaTeX{}
distribution, e.g.\ \textit{texmf-root}|/tex/latex/childdoc|.
\end{itemize}

%%%%%%%%%%%%%%%%%%%%%%%%%%%%%%%%%%%%%%%%%%%%%%%%%%%%%%%%%%%%%%%%%%%%%%%%%%%%%%%%
\subsection{Related CTAN Packages}

There are several other packages which offer a similar functionality:
%
\begin{itemize}
\item
The packages
\href{http://ctan.org/pkg/docmute}{\textsf{docmute}},
\href{http://ctan.org/pkg/includex}{\textsf{includex}} and
\href{http://ctan.org/pkg/standalone}{\textsf{standalone}}
provide commands to include only the document body of
a child file thus allowing both files to be compiled individually.
\item
The packages \href{http://ctan.org/pkg/subdocs}{\textsf{subdocs}}
and \href{http://ctan.org/pkg/subfiles}{\textsf{subfiles}}
provide structures in which the main and child documents can be
encapsulated and allowing them to be compiled individually.
The inclusion mechanism is different from the conventional |\include|.
\item
The package \href{http://ctan.org/pkg/combine}{\textsf{combine}}
is an elaborate solution to combine several documents into one.
\end{itemize}
%
See also the CTAN topic \href{http://ctan.org/topic/subdocs}{\textsf{subdocs}}
for further related packages.
The present package differs from the above solutions in that
a document structure constructed with the conventional |\include| mechanism
just needs two extra commands at the top of every file
such that all constituent files can be compiled individually.

%%%%%%%%%%%%%%%%%%%%%%%%%%%%%%%%%%%%%%%%%%%%%%%%%%%%%%%%%%%%%%%%%%%%%%%%%%%%%%%%
%\subsection{Feature Suggestions}
%
%The following is a list of features which may be useful for future
%versions of this package:
%%
%\begin{itemize}
%\item
%\ldots
%\end{itemize}

%%%%%%%%%%%%%%%%%%%%%%%%%%%%%%%%%%%%%%%%%%%%%%%%%%%%%%%%%%%%%%%%%%%%%%%%%%%%%%%%
\subsection{Revision History}

%%%%%%%%%%%%%%%%%%%%%%%%%%%%%%%%%%%%%%%%
\paragraph{v2.0:} 2018/12/30

\begin{itemize}
\item
immediate forward processing
\item
added |\childdocby| mechanism
\item
manual restructured
\end{itemize}

%%%%%%%%%%%%%%%%%%%%%%%%%%%%%%%%%%%%%%%%
\paragraph{v1.6:} 2018/01/17

\begin{itemize}
\item
application for development of include files
\item
corrections to manual
\end{itemize}

%%%%%%%%%%%%%%%%%%%%%%%%%%%%%%%%%%%%%%%%
\paragraph{v1.5:} 2017/05/21

\begin{itemize}
\item
more complete structuring introduced
\item
|\childdocof| introduced
\item
|\childdoc| renamed to |\childdocmain|
\item
|\childredirect| renamed to |\childdocforward| and |\childdocforwardprefix|
and functionality expanded
\end{itemize}

%%%%%%%%%%%%%%%%%%%%%%%%%%%%%%%%%%%%%%%%
\paragraph{v1.0:} 2017/04/27

\begin{itemize}
\item
manual and install package
\item
first version published on CTAN
\end{itemize}

%%%%%%%%%%%%%%%%%%%%%%%%%%%%%%%%%%%%%%%%
\paragraph{v0.6:} 2017/04/26

\begin{itemize}
\item
redirection mechanism added
\end{itemize}

%%%%%%%%%%%%%%%%%%%%%%%%%%%%%%%%%%%%%%%%
\paragraph{v0.5:} 2017/04/26

\begin{itemize}
\item
functionality in definition file
\end{itemize}


%%%%%%%%%%%%%%%%%%%%%%%%%%%%%%%%%%%%%%%%%%%%%%%%%%%%%%%%%%%%%%%%%%%%%%%%%%%%%%%%
%%%%%%%%%%%%%%%%%%%%%%%%%%%%%%%%%%%%%%%%%%%%%%%%%%%%%%%%%%%%%%%%%%%%%%%%%%%%%%%%
%%%%%%%%%%%%%%%%%%%%%%%%%%%%%%%%%%%%%%%%%%%%%%%%%%%%%%%%%%%%%%%%%%%%%%%%%%%%%%%%
\appendix

\settowidth\MacroIndent{\rmfamily\scriptsize 000\ }

 \DocInput{childdoc.dtx}

\end{document}
%</driver>
% \fi
%
% %%%%%%%%%%%%%%%%%%%%%%%%%%%%%%%%%%%%%%%%%%%%%%%%%%%%%%%%%%%%%%%%%%%%%%%%%%%%%%
% %%%%%%%%%%%%%%%%%%%%%%%%%%%%%%%%%%%%%%%%%%%%%%%%%%%%%%%%%%%%%%%%%%%%%%%%%%%%%%
% \section{Sample}
%\iffalse
%<*samplemain>
%\fi
%
% The following presents a sample document
% with two chapters, two parts, a title page,
% a compile flag as well as three forwarding files to set the flag.
% It consists of eight |.tex| files:
% \begin{center}
% \begin{tabular}{ll}
% |cdocsamp.tex|&main file\\
% |cdocsch1.tex|&include file for chapter 1\\
% |cdocsch2.tex|&include file for chapter 2\\
% |cdocspt3.tex|&include file for part 3\\
% |cdocspt4.tex|&include file for part 4\\
% |cdocsdrf.tex|&forwarding file for main file in draft mode\\
% |cdocsfi1.tex|&forwarding file for final version of chapter 1\\
% |cdocsfi2.tex|&forwarding file for final version of chapter 2\\
% \end{tabular}
% \end{center}
% Each of the eight files can be compiled directly by the \LaTeX{} compiler.
%
% %%%%%%%%%%%%%%%%%%%%%%%%%%%%%%%%%%%%%%
% \paragraph{Main File.}
%
% The main file is called |cdocsamp.tex|.
%
% Load the \textsf{childdoc} definitions and
% declare the filename for the main document:
%    \begin{macrocode}
\input{childdoc.def}
\childdocmain{}
%    \end{macrocode}

% Optional override for |\version| flag:
%    \begin{macrocode}
%%\ifchilddoc\else\providecommand{\version}{draft}\fi
%    \end{macrocode}

% Define the default values for the |\version| flag
% (|final| for the main file and |draft| for childs):
%    \begin{macrocode}
\ifchilddoc
\providecommand{\version}{draft}
\else
\providecommand{\version}{final}
\fi
%    \end{macrocode}

% Load the standard document class:
%    \begin{macrocode}
\documentclass[12pt]{article}
%    \end{macrocode}

% Start the document body:
%    \begin{macrocode}
\begin{document}
%    \end{macrocode}

% Declare a title page.
% Print title, part of document being processed and version flag:
%    \begin{macrocode}
\addtocounter{page}{-1}
\begin{center}
{\LARGE\bfseries{}childdoc example\par}
\vspace{1cm}
\ifchilddoc
\ifchilddocmanual part\else chapter\fi:
`\childdocname' of `\childdocjob'\par
\else
main document: `\childdocjob'\par
\fi
version: \version\par
\end{center}
\newpage
%    \end{macrocode}

% Manually include selected file,
% otherwise process as usual:
%    \begin{macrocode}
\ifchilddocmanual
\section*{part `\childdocname'}
\input{\childdocname}
\else
%    \end{macrocode}

% Include the two chapters:
%    \begin{macrocode}
\include{cdocsch1}
\include{cdocsch2}
%    \end{macrocode}

% Include the two parts unless only chapters should be displayed:
%    \begin{macrocode}
\ifchilddoc\else
\section{part three}
\input{cdocspt3}
\section{part four}
\input{cdocspt4}
\fi
%    \end{macrocode}

% Process as usual until here:
%    \begin{macrocode}
\fi
%    \end{macrocode}

% End of document body:
%    \begin{macrocode}
\end{document}
%    \end{macrocode}
%\iffalse
%</samplemain>
%\fi
%
% %%%%%%%%%%%%%%%%%%%%%%%%%%%%%%%%%%%%%%
% \paragraph{Chapter Include Files.}
%
% The include files are called |cdocsch1.tex| and |cdocsch2.tex|.
%
%\iffalse
%<*samplechap1|samplechap2>
%\fi

% Optional override for |\version| flag:
%    \begin{macrocode}
%%\providecommand{\version}{final}
%    \end{macrocode}

% Include the main document:
%    \begin{macrocode}
\input{childdoc.def}
\childdocof{cdocsamp}
%    \end{macrocode}

%\iffalse
%</samplechap1|samplechap2>
%\fi
%
%\iffalse
%<*samplechap1>
%\fi
% Some text for chapter 1:
%    \begin{macrocode}
\section{one}
some text in chapter one
%    \end{macrocode}

%\iffalse
%</samplechap1>
%\fi
% Some text for chapter 2:
%\iffalse
%<*samplechap2>
%\fi
%    \begin{macrocode}
\section{two}
more text in chapter two
%    \end{macrocode}

%\iffalse
%</samplechap2>
%\fi
%
% %%%%%%%%%%%%%%%%%%%%%%%%%%%%%%%%%%%%%%
% \paragraph{Part Include Files.}
%
% The include files are called |cdocspt3.tex| and |cdocspt4.tex|.
%
%\iffalse
%<*samplepart3|samplepart4>
%\fi

% Optional override for |\version| flag:
%    \begin{macrocode}
%%\providecommand{\version}{final}
%    \end{macrocode}

% Include the main document:
%    \begin{macrocode}
\input{childdoc.def}
\childdocby{cdocsamp}
%    \end{macrocode}

%\iffalse
%</samplepart3|samplepart4>
%\fi
%
%\iffalse
%<*samplepart3>
%\fi
% Some text for part 3:
%    \begin{macrocode}
some text in part three
%    \end{macrocode}

%\iffalse
%</samplepart3>
%\fi
% Some text for part 4:
%\iffalse
%<*samplepart4>
%\fi
%    \begin{macrocode}
more text in part four
%    \end{macrocode}

%\iffalse
%</samplepart4>
%\fi
%
% %%%%%%%%%%%%%%%%%%%%%%%%%%%%%%%%%%%%%%
% \paragraph{Forwarding for a Complete Draft.}
%
% The following forwarding file |cdocsdrf.tex|
% compiles the main document in draft mode:
%\iffalse
%<*sampledraft>
%\fi
%    \begin{macrocode}
\def\version{draft}
\input{childdoc.def}
\childdocforward{cdocsamp}
%    \end{macrocode}

%\iffalse
%</sampledraft>
%\fi
%
% %%%%%%%%%%%%%%%%%%%%%%%%%%%%%%%%%%%%%%
% \paragraph{Forwarding for Final Version of the Chapters.}
%
% The following forwarding files |cdocsfn1.tex| and |cdocsfn2.tex|
% (with identical content)
% compile the final versions of the child documents
% |cdocsch1.tex| and |cdocsch2.tex|, respectively:
%\iffalse
%<*samplefinal>
%\fi
%    \begin{macrocode}
\def\version{final}
\input{childdoc.def}
\childdocforwardprefix[cdocsamp]{cdocsfn}{cdocsch}
%    \end{macrocode}

%\iffalse
%</samplefinal>
%\fi
%
% %%%%%%%%%%%%%%%%%%%%%%%%%%%%%%%%%%%%%%
% \paragraph{Command Line Processing.}
%
% The following three command lines generate the output files
% |cdocscld|, |cdocscl1| and |cdocscl2|
% which should be identical to
% |cdocsdrf|, |cdocsch1| and |cdocsfn2|, respectively:
% \begin{center}
% \begin{tabular}{l}
% |latex -jobname cdocscld \|\\
% |  "\def\version{draft}\input{childdoc.def}\childdocforward{cdocsamp}"|\\
% |latex -jobname cdocscl1 \|\\
% |  "\input{childdoc.def}\childdocforward[cdocsamp]{cdocsch1}"|\\
% |latex -jobname cdocscl2 \|\\
% |  "\def\version{final}\input{childdoc.def}\childdocforward{cdocsch2}"|
% \end{tabular}
% \end{center}
% Note that the trailing backslash on each first line
% merely continues the input to the second line
% (for convenient cut ant paste).
% Furthermore, the command |latex| can be replaced by any
% of its alternative versions such as |pdflatex|.
%
% %%%%%%%%%%%%%%%%%%%%%%%%%%%%%%%%%%%%%%%%%%%%%%%%%%%%%%%%%%%%%%%%%%%%%%%%%%%%%%
% %%%%%%%%%%%%%%%%%%%%%%%%%%%%%%%%%%%%%%%%%%%%%%%%%%%%%%%%%%%%%%%%%%%%%%%%%%%%%%
% \section{Implementation}
%\iffalse
%<*package>
%\fi
%
% This section describes the definitions file |childdoc.def|.

% The definitions cannot be loaded using |\usepackage| or |\RequirePackage|
% which has a mechanism to prevent loading a style file more than once.
% When loading the definitions by means of |\input|
% multiple instances have to be prevented manually:
%\iffalse
%This code needs to be before the `\ProvidesFile' directive
%which is defined at the beginning of this file.
%Therefore it is also placed there and commented out here.
%</package>
%<*discard>
%\fi
%    \begin{macrocode}
\ifdefined\childdocmain\endinput\fi
%    \end{macrocode}
%\iffalse
%</discard>
%<*package>
%\fi
%
% \macro{\ifchilddoc}
% \macro{\ifchilddocmanual}
% The conditional |\ifchilddoc| tells whether a
% child (true) or main (false) document is being compiled.
% The conditional |\ifchilddocmanual| tells whether
% the |\includeonly| mechanism is used (false) or
% the selection of child files must be performed manually (true).
% The definitions initialise to false:
%    \begin{macrocode}
\newif\ifchilddoc
\newif\ifchilddocmanual
%    \end{macrocode}

% \macro{\childdocname}
% \macro{\childdocjob}
% The macro |\childdocname| stores the name of the main document
% to be compiled. The macro |\childdocjob| stores the name of
% the document on which the \LaTeX{} compiler was originally invoked.
% The content of |\jobname| cannot be compared
% to filenames specified in the source due to different catcodes.
% The following code rescans |\jobname|, stores the result
% in |\childdocname| and saves a copy in |\childdocjob|:
%    \begin{macrocode}
\edef\childdocname{\scantokens\expandafter{\jobname\noexpand}}
\let\childdocjob\childdocname
%    \end{macrocode}

% \macro{\childdocdisable}
% The macro |\childdocdisable| prevents the main file
% from being processed more than once.
% At this stage, the main document command |\childdocmain|
% is assumed to be called once again where it should do nothing.
% Any subsequent call to it should prevent
% a secondary processing of the main document
% It overwrites the forwarding commands
% |\childdocof| and |\childdocforward|
% with empty macros to prevent further inclusions of the main document:
%    \begin{macrocode}
\newcommand{\childdocdisable}
{
  \renewcommand{\childdocmain}[1]{\renewcommand{\childdocmain}[1]{\endinput}}
  \renewcommand{\childdocof}[1]{}
  \renewcommand{\childdocby}[2][]{}
  \renewcommand{\childdocforward}[2][]{}
  \renewcommand{\childdocdisable}{}
}
%    \end{macrocode}

% \macro{\childdocmain}
% The macro |\childdocmain| is to be called at the top of the main file
% with nothing or the main filename (without extension) as argument.
% First, it breaks loops.
% If the argument is not empty and does not match |\childdocname|
% (which is set by the first inclusion of |childdoc.def|),
% |\ifchilddoc| is set to true, |\includeonly| is applied to the child file
% and |\jobname| is set to the main file
% (for proper handling of |.aux| files):
%    \begin{macrocode}
\newcommand{\childdocmain}[1]
{
  \childdocdisable\childdocmain{}
  \if?#1?\else
    \begingroup
      \def\childdoctmp{#1}
      \ifx\childdoctmp\childdocname
        \def\childdoctmp{}
      \else
        \def\childdoctmp
        {
          \childdoctrue
          \includeonly{\childdocname}
          \def\childdocjob{#1}
          \def\jobname{#1}
        }
      \fi
      \expandafter
    \endgroup
    \childdoctmp
  \fi
}
%    \end{macrocode}

% \macro{\childdocof}
% The command |\childdocof| redirects
% compilation to the main file |#1|.
%    \begin{macrocode}
\newcommand{\childdocof}[1]
{
  \childdocdisable
  \childdoctrue
  \includeonly{\childdocname}
  \def\jobname{#1}
  \def\childdocjob{#1}
  \input{#1}
}
%    \end{macrocode}

% \macro{\childdocby}
% The command |\childdocby| ....
%    \begin{macrocode}
\newcommand{\childdocby}[2][]
{
  \childdocdisable
  \childdoctrue
  \childdocmanualtrue
  \if?#1?\else
    \def\jobname{#2}
  \fi
  \def\childdocjob{#2}
  \input{#2}
  \endinput
}
%    \end{macrocode}

% \macro{\childdocforward}
% The command |\childdocforward| redirects
% compilation to the main file or
% (if the optional argument is given) a child file.
% Parameters are set as if the main file
% or a child file starting with |\childdocof| was compiled.
% Then compilation is handed over to the main file:
%    \begin{macrocode}
\newcommand{\childdocforward}[2][]
{
  \begingroup
    \if?#1?
      \def\childdoctmp
      {
        \def\childdocname{#2}
        \def\childdocjob{#2}
        \def\jobname{#2}
        \input{#2}
        \endinput
      }
    \else
      \def\childdoctmp
      {
        \childdocdisable
        \def\childdocname{#2}
        \childdoctrue
        \includeonly{#2}
        \def\childdocjob{#1}
        \def\jobname{#1}
        \input{#1}
        \endinput
      }
    \fi
    \expandafter
  \endgroup
  \childdoctmp
}
%    \end{macrocode}

% \macro{\childdocforwardprefix}
% The command |\childdocforwardprefix| redirects
% compilation to the main or a child file by means of a pattern.
% The prefix |#1| in the current filename is replaced by |#2|
% and the suffix of the current filename is kept
% (it is assumed that the filename does not contain the substring `|~~~|'
% which is used as a delimiter).
% Compilation is handed over to the new file by |\childdocforward|:
%    \begin{macrocode}
\newcommand{\childdocforwardprefix}[3][]
{
  \begingroup
    \def\childdocextract #2##1~~~{\def\childdoctmp{\childdocforward[#1]{#3##1}}}
    \expandafter\childdocextract\childdocname~~~
    \expandafter
  \endgroup
  \childdoctmp
}
%    \end{macrocode}

% \macro{\childdoc}
% The deprecated macro |\childdoc| is a legacy version of |\childdocmain|:
%    \begin{macrocode}
\newcommand{\childdoc}{\childdocmain}
%    \end{macrocode}

% \macro{\childdocredirect}
% The deprecated macro |\childdocredirect| is a legacy version
% of |\childdocforward| and |\childdocforwardprefix|:
%    \begin{macrocode}
\newcommand{\childdocredirect}[2][]
{
  \begingroup
    \if?#1?
      \def\childdoctmp{\childdocforward{#2}}
    \else
      \def\childdoctmp{\childdocforwardprefix{#1}{#2}}
    \fi
    \expandafter
  \endgroup
  \childdoctmp
}
%    \end{macrocode}

%\iffalse
%</package>
%\fi
%
\endinput
|\\
|\childdocforward[|\textit{main}|]{|\textit{dest}|}|\\
\end{tabular}
\end{center}
%
The argument \textit{dest} is the destination file
(without extension).
It should be the main file or one of the child files.
Note that further \textsf{childdoc} directives
such as |\childdocof| and |\childdocforward|
in the indicated file will be processed in this form.
The optional argument \textit{main}
passes on directly to the main file \textit{main}
while pretending to compile the child \textit{dest}.
This form behaves as if \textit{dest}
issues |\childdocof{|\textit{main}|}| right away,
and no further \textsf{childdoc} directives will be processed.

%%%%%%%%%%%%%%%%%%%%%%%%%%%%%%%%%%%%%%%%
\DescribeMacro{\...prefix}
In the alternative form |\childdocforwardprefix|,
%
\begin{center}
\begin{tabular}{l}
|% \iffalse
%
% childdoc.dtx Copyright (C) 2017-2018 Niklas Beisert
%
% This work may be distributed and/or modified under the
% conditions of the LaTeX Project Public License, either version 1.3
% of this license or (at your option) any later version.
% The latest version of this license is in
%   http://www.latex-project.org/lppl.txt
% and version 1.3 or later is part of all distributions of LaTeX
% version 2005/12/01 or later.
%
% This work has the LPPL maintenance status `maintained'.
%
% The Current Maintainer of this work is Niklas Beisert.
%
% This work consists of the files childdoc.dtx and childdoc.ins
% and the derived files childdoc.def and cdocsamp.tex with
% cdocsch1.tex, cdocsch2.tex, cdocsdrf.tex, cdocsfn1.tex, cdocsfn2.tex.
%
%<package>\ifdefined\childdocmain\endinput\fi
%<package>\ProvidesFile{childdoc.def}[2018/12/30 v2.0 child document driver]
%<samplemain>\ProvidesFile{cdocsamp.tex}[2018/12/30 v2.0 sample for childdoc]
%<*driver>
%\ProvidesFile{childdoc.drv}[2018/12/30 v2.0 childdoc reference manual file]
\PassOptionsToClass{10pt,a4paper}{article}
\documentclass{ltxdoc}

\usepackage[margin=35mm]{geometry}
\usepackage{hyperref}
\usepackage{hyperxmp}
\usepackage[usenames]{color}

\hypersetup{colorlinks=true}
\hypersetup{pdfstartview=FitH}
\hypersetup{pdfpagemode=UseNone}
\hypersetup{pdfsource={}}
\hypersetup{pdflang={en-UK}}
\hypersetup{pdfcopyright={Copyright 2017-2018 Niklas Beisert.
  This work may be distributed and/or modified under the
  conditions of the LaTeX Project Public License, either version 1.3
  of this license or (at your option) any later version.}}
\hypersetup{pdflicenseurl={http://www.latex-project.org/lppl.txt}}
\hypersetup{pdfcontactaddress={ETH Zurich, ITP, HIT K,
  Wolfgang-Pauli-Strasse 27}}
\hypersetup{pdfcontactpostcode={8093}}
\hypersetup{pdfcontactcity={Zurich}}
\hypersetup{pdfcontactcountry={Switzerland}}
\hypersetup{pdfcontactemail={nbeisert@itp.phys.ethz.ch}}
\hypersetup{pdfcontacturl={http://people.phys.ethz.ch/\xmptilde nbeisert/}}

\newcommand{\secref}[1]{\hyperref[#1]{section \ref*{#1}}}

\parskip1ex
\parindent0pt
\let\olditemize\itemize
\def\itemize{\olditemize\parskip0pt}

\begin{document}

\title{The \textsf{childdoc} Package}
\hypersetup{pdftitle={The childdoc Package}}
\author{Niklas Beisert\\[2ex]
  Institut f\"ur Theoretische Physik\\
  Eidgen\"ossische Technische Hochschule Z\"urich\\
  Wolfgang-Pauli-Strasse 27, 8093 Z\"urich, Switzerland\\[1ex]
  \href{mailto:nbeisert@itp.phys.ethz.ch}
  {\texttt{nbeisert@itp.phys.ethz.ch}}}
\hypersetup{pdfauthor={Niklas Beisert}}
\hypersetup{pdfsubject={Manual for the LaTeX2e Package childdoc}}
\date{30 December 2018, \textsf{v2.0}}
\maketitle

\begin{abstract}\noindent
\textsf{childdoc} is a \LaTeXe{} package
that enables the direct compilation
of document sections included by |\include|
to individual files.
\end{abstract}

\begingroup
\parskip0ex
\tableofcontents
\endgroup

%%%%%%%%%%%%%%%%%%%%%%%%%%%%%%%%%%%%%%%%%%%%%%%%%%%%%%%%%%%%%%%%%%%%%%%%%%%%%%%%
%%%%%%%%%%%%%%%%%%%%%%%%%%%%%%%%%%%%%%%%%%%%%%%%%%%%%%%%%%%%%%%%%%%%%%%%%%%%%%%%
\section{Introduction}

\LaTeX{} provides a mechanism to structure a large document (such as a book)
into a main file and several child files (containing the chapters)
using the |\include| command.
This mechanism is beneficial for documents
which span hundreds of pages in order to
make the source file(s) more manageable.
Moreover, compilation can be restricted to
selected child files by means of the |\includeonly| command.
The latter feature can be used to reduce the compilation time while editing
(this was significantly more useful in the earlier days of \LaTeX{})
or to generate a smaller document which is easier to navigate.
Another application of |\includeonly| is to generate
documents consisting of selected parts of the complete document.

However, there are a few drawbacks of the plain |\include| mechanism:
\begin{itemize}
\item
The child files cannot be compiled on their own,
they can only be compiled via the main file.
A naive editing environment
(such as a text editor with an option
to have the current file processed by \LaTeX)
may require one to switch to the main file before compiling;
attempting to compile the child file produces errors.
\item
The main file must be modified (each time)
to adjust the |\includeonly| command
to the present needs. This easily leaves the main file in a messy state.
\item
The generated document will always carry the filename
of the main document. This is inconvenient if
several child files are to be compiled and
to be kept for distribution.
\end{itemize}

The present package provides a simple interface
to make child files individually compilable by \LaTeX{}.
Compiling a child file then has the same effect as compiling
the main file with an |\includeonly| command
to select the appropriate child.
Moreover the generated document will carry the name of the child
rather than the main file.
This resolves all three above issues.

This feature is meant to make the editing of books,
thesis documents and lecture notes somewhat more convenient.
However, the package can also be used efficiently for
composing a series of documents (such as exercise sheets)
which are typically distributed individually.
It then assists the author in generating the individual documents
(potentially in different versions)
as well as a document containing the collected series.
Another application is in developing style files
or other kinds of included material
where compilation of the style file could redirect
to a sample or test file.

%%%%%%%%%%%%%%%%%%%%%%%%%%%%%%%%%%%%%%%%%%%%%%%%%%%%%%%%%%%%%%%%%%%%%%%%%%%%%%%%
%%%%%%%%%%%%%%%%%%%%%%%%%%%%%%%%%%%%%%%%%%%%%%%%%%%%%%%%%%%%%%%%%%%%%%%%%%%%%%%%
\section{Usage}

First of all, the package \textsf{childdoc} is \emph{not} a standard
\LaTeXe{} |.sty| style file! Therefore it needs to be invoked in
a non-standard way.

%%%%%%%%%%%%%%%%%%%%%%%%%%%%%%%%%%%%%%%%%%%%%%%%%%%%%%%%%%%%%%%%%%%%%%%%%%%%%%%%
\subsection{Included Files}
\label{sec:include}

%%%%%%%%%%%%%%%%%%%%%%%%%%%%%%%%%%%%%%%%
\DescribeMacro{\childdocmain}
To use the package, add the commands
\begin{center}
\begin{tabular}{l}
|\input{childdoc.def}|\\
|\childdocmain{}|\\
\end{tabular}
\end{center}
at the very top of the main \LaTeX{} file,
in particular \emph{before} the |\documentclass| statement!
The argument of |\childdocmain| should be left empty
(but it must be present).

%%%%%%%%%%%%%%%%%%%%%%%%%%%%%%%%%%%%%%%%
\DescribeMacro{\childdocof}
Furthermore, add the commands
\begin{center}
\begin{tabular}{l}
|\input{childdoc.def}|\\
|\childdocof{|\textit{main}|}|\\
\end{tabular}
\end{center}
at the top of every child file \textit{child}
which is included by |\include{|\textit{child}|}|
from within the main file
(or at least for those files to be compiled individually).
The argument \textit{main} must be the filename of the main file.

There are a couple of
considerations in setting up the main and child documents:

%%%%%%%%%%%%%%%%%%%%%%%%%%%%%%%%%%%%%%%%
\paragraph{Restrictions.}

Please note the following restrictions:
\begin{itemize}
\item
|\childdocmain| must be called with one argument \textit{main}
to ensure compatibility with earlier version of the package.
It must either be empty (|\childdocmain{}|)
or precisely match the filename of the main file in which it is specified.
See \secref{sec:detection} for further information.
\item
The filename \textit{main} must be specified without the |.tex| extension.
\item
The filename \textit{main} is case sensitive
(even in case-insensitive file systems)
due to internal string comparison.
\item
The argument \textit{main} should be fully expanded, it cannot be a macro.
\item
Subdirectories and special characters should be avoided in filenames.
\item
The command |\childdocmain{|\textit{main}|}| must be followed by a whitespace.
It should not be followed immediately by another command
or by a comment mark `|%|'.
This is because the \TeX{} parser reads the token immediately following
the argument of |\childdocmain| and puts it
at the beginning of every child section;
however, a white\-space is ignored.
\end{itemize}

%%%%%%%%%%%%%%%%%%%%%%%%%%%%%%%%%%%%%%%%
\paragraph{Content of Main File.}

It is advisable to place all content in the child files included by |\include|.
Any output contained in the main file will appear in all child documents
unless suppressed manually;
it cannot be suppressed automatically by the |\includeonly| directive
and thus should normally be avoided.
A method to include some content in the main file
by means of conditional processing is described in \secref{sec:conditional}.

%%%%%%%%%%%%%%%%%%%%%%%%%%%%%%%%%%%%%%%%
\paragraph{Page Numbering.}

When only a part of the document is compiled,
the appropriate numbering of pages
(as well as other status parameters)
is determined from the |.aux| files.
The latter contain information from previous passes.
However this information needs to propagate through
all intermediate child documents.
Therefore the page numbering in child documents may well
be inconsistent until the complete document is compiled at least once.

A useful (if unconventional) way to always ensure a consistent
page numbering is to restart the numbering in each child document
and denote the pages by `\textit{child}|.|\textit{page}'
where \textit{child} represents the chapter/section number of the child file.
This can be achieved by the command
|\numberwithin{page}{|\textit{child}|}|
of the \textsf{amsmath} package
where \textit{child} can be |chapter| or |section|
depending on the chosen structuring.
Alternatively, one can modify the macro |\thepage| appropriately
and reset the counter |page| at the start of each child file.

%%%%%%%%%%%%%%%%%%%%%%%%%%%%%%%%%%%%%%%%%%%%%%%%%%%%%%%%%%%%%%%%%%%%%%%%%%%%%%%%
\subsection{Conditional Processing}
\label{sec:conditional}

The package provides a mechanism to compile different versions
of a document. To customise the versions further some conditional processing
can come in handy to distinguish which version is being compiled.
The package provides two macros to describe the compilation context:

%%%%%%%%%%%%%%%%%%%%%%%%%%%%%%%%%%%%%%%%
\DescribeMacro{\ifchilddoc}
The conditional |\ifchilddoc| distinguishes between the compilation of
child documents and the main document:
%
\begin{center}
|\ifchilddoc |\textit{child-code}| |[|\||else |\textit{main-code}]| \||fi|
\end{center}

%%%%%%%%%%%%%%%%%%%%%%%%%%%%%%%%%%%%%%%%
\DescribeMacro{\childdocname}
\DescribeMacro{\childdocjob}
The macro |\childdocname| contains the filename (without extension)
of the main or child file being processed.
Note that |\childdocjob| will always contain the name of the main file.

%%%%%%%%%%%%%%%%%%%%%%%%%%%%%%%%%%%%%%%%
\paragraph{Title Page.}

Conditional processing can be used to include a title or banner page
in the main document when proper precautions are taken.
Importantly, the code in the main file should ensure that the page counter
(as well as other status parameters which are stored in the |.aux| files)
takes the same value after the conditional processing.
Otherwise the page numbers may take divergent values
depending on which part is compiled.

For example, a title page could be declared by:
%
\begin{center}
\begin{tabular}{l}
|\ifchilddoc\||else|\\
|\addtocounter{page}{-1}|\\
\textit{code for title page}\\
|\newpage|\\
|\||fi|
\end{tabular}
\end{center}
%
A banner page for the child documents can be generated by:
%
\begin{center}
\begin{tabular}{l}
|\ifchilddoc|\\
|\addtocounter{page}{-1}|\\
\textit{code for banner page}\\
|\newpage|\\
|\||fi|
\end{tabular}
\end{center}
%
Here one could write a message such as:
\begin{center}
|This is the part \childdocname{} of \childdocjob{}.|
\end{center}

%%%%%%%%%%%%%%%%%%%%%%%%%%%%%%%%%%%%%%%%%%%%%%%%%%%%%%%%%%%%%%%%%%%%%%%%%%%%%%%%
\subsection{Flags}
\label{sec:flags}

The package makes it easy to generate different versions
of the main or child documents.
To this end compilation flags can be defined
and assigned different default values.
They will be particularly useful in conjunction
with the forwarding mechanism described in \secref{sec:forward}.

For example, it may be useful to have a flag |\version|
which can be set to |draft| or |final|.
The document source will contain some conditional code
depending on the value of |\version|.
Suppose further, the flag should default to |final| for the main file
and to |draft| for child files
which is a natural assignment for editing the document.
This is achieved by placing the following code
in the preamble of the main document
(below the |\childdocmain| directive):
%
\begin{center}
\begin{tabular}{l}
|\ifchilddoc|\\
|\providecommand{\version}{draft}|\\
|\||else|\\
|\providecommand{\version}{final}|\\
|\||fi|
\end{tabular}
\end{center}
%
The definition by |\providecommand| makes sure
that previous definitions are not overwritten.
Further statements |\providecommand{\version}{...}|
can thus be added before the above code to override it.

For the main file, one might add a line
(between |\childdocmain| and the above block)
%
\begin{center}
|%\ifchilddoc\||else\providecommand{\version}{draft}\||fi|
\end{center}
%
which can be uncommented to produce a draft version.
Likewise one can add a line to the very top of a child file
(above the |\childdocof{|\textit{main}|}| directive)
%
\begin{center}
|%\providecommand{\version}{final}|
\end{center}
%
which can be uncommented to produce the final version of this child document.

%%%%%%%%%%%%%%%%%%%%%%%%%%%%%%%%%%%%%%%%%%%%%%%%%%%%%%%%%%%%%%%%%%%%%%%%%%%%%%%%
\subsection{Forwarding}
\label{sec:forward}

Different versions of the main or child documents
using compilation flags as described in \secref{sec:flags}
can be (permanently) stored in different files
for convenient compilation, viewing and distribution.
To this end, the package defines a command
to pass on compilation to a different file:

%%%%%%%%%%%%%%%%%%%%%%%%%%%%%%%%%%%%%%%%
\DescribeMacro{\childdocforward}
The command |\childdocforward| redirects processing to
another source file:
%
\begin{center}
\begin{tabular}{l}
|\input{childdoc.def}|\\
|\childdocforward[|\textit{main}|]{|\textit{dest}|}|\\
\end{tabular}
\end{center}
%
The argument \textit{dest} is the destination file
(without extension).
It should be the main file or one of the child files.
Note that further \textsf{childdoc} directives
such as |\childdocof| and |\childdocforward|
in the indicated file will be processed in this form.
The optional argument \textit{main}
passes on directly to the main file \textit{main}
while pretending to compile the child \textit{dest}.
This form behaves as if \textit{dest}
issues |\childdocof{|\textit{main}|}| right away,
and no further \textsf{childdoc} directives will be processed.

%%%%%%%%%%%%%%%%%%%%%%%%%%%%%%%%%%%%%%%%
\DescribeMacro{\...prefix}
In the alternative form |\childdocforwardprefix|,
%
\begin{center}
\begin{tabular}{l}
|\input{childdoc.def}|\\
|\childdocforwardprefix[|\textit{main}|]{|\textit{prefix}|}{|\textit{dest}|}|
\end{tabular}
\end{center}
%
the destination file is determined by a pattern
depending on the current file:
To make this work, the current file must be called
`{\textit{prefix}\hspace{0.2em}\textit{suffix}}'
with \textit{prefix} matching precisely the argument.
Processing is then passed on to the file
`{\textit{dest}\hspace{0.2em}\textit{suffix}}'.
Surely, the same effect is achieved by
directly specifying the
argument `{\textit{dest}\hspace{0.2em}\textit{suffix}}'
in the first form.
However, that requires to set up a different file
for each child. With the alternative form of the command
all these files can have exactly the same content
which simplifies setting them up and maintaining them.

For example, the following file |draft.tex|
with a compilation flag |\version| as described in \secref{sec:flags}
compiles the main document as a draft:
%
\begin{center}
\begin{tabular}{l}
|\def\version{draft}|\\
|\input{childdoc.def}|\\
|\childdocforward{|\textit{main}|}|
\end{tabular}
\end{center}
%
Likewise, the following files |final|\textit{nn}|.tex|
compile the final version of the child document
|child|\textit{nn}|.tex|:
%
\begin{center}
\begin{tabular}{l}
|\def\version{final}|\\
|\input{childdoc.def}|\\
|\childdocforwardprefix{final}{child}|
\end{tabular}
\end{center}
%

Note that when several versions of a main file and/or of each child file
are to be generated, it may be convenient to set up a |Makefile| or
shell script to automatise the process.

%%%%%%%%%%%%%%%%%%%%%%%%%%%%%%%%%%%%%%%%%%%%%%%%%%%%%%%%%%%%%%%%%%%%%%%%%%%%%%%%
\subsection{Command Line Processing}
\label{sec:commandline}

The effect of redirection files can also be achieved by invoking
the \LaTeX{} compiler with a more elaborate command line.
Most conveniently this should be done as part
of a shell script or a |Makefile|.

When using \textsf{childdoc} in the main file, the following
command lines effectively perform a redirection
(note that depending on the shell being used,
backslashes may have to be doubled: `|\|' $\to$ `|\\|'):
%
\begin{center}
|... -jobname "|\textit{target}|" |\\|"|[\textit{flags}]%
|\input{childdoc.def}\childdocforward[|\textit{main}|]{|\textit{dest}|}"|
\end{center}
%
Here \textit{target} is the name of the output file,
\textit{main} is the name of the main file
and \textit{dest} is the name of the main or child file to be processed
(all filenames without extensions).
The optional argument \textit{main} can be omitted
if \textit{main} matches \textit{dest}.
Optionally, compilation \textit{flags} can be defined via |\def| commands.
This command line makes the \TeX{} engine believe
it is compiling the file \textit{target}
whose content is specified as the latter parameter.
The provided code then forwards the processing to
\textit{main} or \textit{dest} as described in \secref{sec:forward}.

%%%%%%%%%%%%%%%%%%%%%%%%%%%%%%%%%%%%%%%%%%%%%%%%%%%%%%%%%%%%%%%%%%%%%%%%%%%%%%%%
\subsection{Include by Input}
\label{sec:input}

Including child documents by |\include| has some restrictions by design.
Most notably, the content of a child document always occupies
its own set of pages; pages cannot be shared between child documents.
Usually, this behaviour makes perfect sense
because each child document contain an essential part of the document.
However, in some situations it may be desirable to compose
a document from a collection of parts
without having mandatory page breaks between then.
For this case, the package
provides a mechanism to include parts
by |\input| which can also be processed individually.
However, by construction this mechanism
requires manual handling of the content to be output.

%%%%%%%%%%%%%%%%%%%%%%%%%%%%%%%%%%%%%%%%
\DescribeMacro{\ifchilddocmanual}
The main file should be prepared as usual, see \secref{sec:include}.
However, the document body must make a distinction
between processing of an individual part and of the main document, e.g.:
%
\begin{center}
\begin{tabular}{l}
|\ifchilddocmanual|\\
|\input{\childdocname}|\\
|\||else|\\
\textit{document body with }|\input{|\textit{part}|}|\\
|\||fi|
\end{tabular}
\end{center}
%
The conditional |\ifchilddocmanual| is true whenever
a part to be included by |\input| is being compiled,
and the name of the part is stored in |\childdocname|.

%%%%%%%%%%%%%%%%%%%%%%%%%%%%%%%%%%%%%%%%
\DescribeMacro{\childdocby}
Each part to be included by |\input| should start with:
%
\begin{center}
\begin{tabular}{l}
|\input{childdoc.def}|\\
|\childdocby{|\textit{main}|}|\\
\end{tabular}
\end{center}
%
The directive |\childdocby| is similar to |\childdocof|
described in \secref{sec:include},
but the subsequent selection of content must be done manually.
To that end, both |\ifchilddoc| and |\ifchilddocmanual|
will be true upon processing of a part,
and the name of the part is stored in |\childdocname|.
Note that |\jobname| will be set to the filename of the current part
so that each part receives an individual |.aux| file
that does not interfere with the |.aux| file(s) of the main document.
This behaviour can be altered by the alternative form
|\childdocby[*]{|\textit{main}|}| (with a non-empty optional argument)
which uses the |.aux| file of the main document
by setting |\jobname| to \textit{main}.

%%%%%%%%%%%%%%%%%%%%%%%%%%%%%%%%%%%%%%%%%%%%%%%%%%%%%%%%%%%%%%%%%%%%%%%%%%%%%%%%
\subsection{Driver Development}
\label{sec:driver}

The \textsf{childdoc} mechanism can also be use for the development
of definition files such as \LaTeX{} styles or classes.
This case differs from the above setup with multiple parts
included by |\include| in that no |\includeonly| should be invoked.
This can be achieved by starting the include file
(before |\ProvidesPackage|) with:
%
\begin{center}
\begin{tabular}{l}
|\input{childdoc.def}|\\
|\childdocforward{|\textit{main}|}|\\
\end{tabular}
\end{center}
%
or alternatively with:
%
\begin{center}
\begin{tabular}{l}
|\input{childdoc.def}|\\
|\childdocby{|\textit{main}|}|\\
\end{tabular}
\end{center}
%
Both forms have slightly different effects as described above.
The main file is prepared as usual, see \secref{sec:include}.

%%%%%%%%%%%%%%%%%%%%%%%%%%%%%%%%%%%%%%%%%%%%%%%%%%%%%%%%%%%%%%%%%%%%%%%%%%%%%%%%
\subsection{Legacy Detection}
\label{sec:detection}

The directive |\childdocmain| in the main file can detect
whether the complete document or merely a child is to be compiled
even without using the directive |\childdocof|.
This method is deprecated because it is less robust
and there is no compelling reason to use it;
it is merely provided for backward compatibility
and it may be removed in future versions.

If the detection mechanism is to be used,
it is mandatory to correctly specify
the filename of the main file as the argument of |\childdocmain|:
%
\begin{center}
\begin{tabular}{l}
|\input{childdoc.def}|\\
|\childdocmain{|\textit{main}|}|\\
\end{tabular}
\end{center}
%
If |\jobname| does not match the argument \textit{main} of |\childdocmain|,
it is assumed that |\jobname| points to the child file to be compiled.
When using |\childdocmain| with the main file specified as argument,
it suffices to start a child file
with just |\input{|\textit{main}|}|
without loading of the package and using |\childdocof|.
If instead all processing is done
with the appropriate \textsf{childdoc} directives,
the argument of \textit{main} of |\childdocmain| can be empty.

An alternative version of the command line processing described
in \secref{sec:commandline} using the detection mechanism reads:
%
\begin{center}
|... -jobname "|\textit{target}|" "|[\textit{flags}]%
[|\def\jobname{|\textit{dest}|}|]|\input{|\textit{main}|}"|
\end{center}

%%%%%%%%%%%%%%%%%%%%%%%%%%%%%%%%%%%%%%%%%%%%%%%%%%%%%%%%%%%%%%%%%%%%%%%%%%%%%%%%
\subsection{Manual Code}
\label{sec:manual}

In case one cannot be certain whether the definitions file |childdoc.def|
is installed on the target \TeX{} distribution
and one prefers not to ship it,
it is conceivable to paste a few relevant commands into the sources.

To that end, drop all statements |\input{childdoc.def}|
and perform the replacements as outlined below.
Instead of |\childdocmain{|\textit{main}|}| add the following code
to the top of the main file:
%
\begin{center}
\begin{tabular}{l}
|\||ifdefined\childdocname\endinput\||fi\newif\ifchilddoc|\\
|\edef\childdocname{\scantokens\expandafter{\jobname\noexpand}}|\\
|\def\childdocmain{|\textit{main}|}\||ifx\childdocmain\childdocname\||else|\\
|\childdoctrue\includeonly{\childdocname}\let\jobname\childdocmain\||fi|\\
\end{tabular}
\end{center}
%
Instead of |\childdocof{|\textit{main}|}| just include the main file
at the top of each child file:
%
\begin{center}
|\input{|\textit{main}|}|
\end{center}
%
A simple redirection |\childdocforward{|\textit{dest}|}| is achieved by:
%
\begin{center}
|\def\jobname{|\textit{dest}|}\input{\jobname}|
\end{center}
%
The redirection with prefix
|\childdocforwardprefix[|\textit{prefix}|]{|\textit{dest}|}|
is accomplished by:
%
\begin{center}
\begin{tabular}{l}
|{\edef\jobname{\scantokens\expandafter{\jobname\noexpand}}|\\
|\def\redirectjob |\textit{prefix}|#1~~~{\gdef\jobname{|\textit{dest}|#1}}|\\
|\expandafter\redirectjob\jobname~~~}\input{\jobname}|
\end{tabular}
\end{center}

In an alternative approach,
child documents can be compiled by a specific command line
without additional code or specific definitions:
%
\begin{center}
|... -jobname "|\textit{target}|" "|[\textit{flags}]%
|\includeonly{|\textit{dest}|}\input{|\textit{main}|}"|
\end{center}
%

%%%%%%%%%%%%%%%%%%%%%%%%%%%%%%%%%%%%%%%%%%%%%%%%%%%%%%%%%%%%%%%%%%%%%%%%%%%%%%%%
%%%%%%%%%%%%%%%%%%%%%%%%%%%%%%%%%%%%%%%%%%%%%%%%%%%%%%%%%%%%%%%%%%%%%%%%%%%%%%%%
\section{Information}

%%%%%%%%%%%%%%%%%%%%%%%%%%%%%%%%%%%%%%%%%%%%%%%%%%%%%%%%%%%%%%%%%%%%%%%%%%%%%%%%
\subsection{Copyright}

Copyright \copyright{} 2017--2018 Niklas Beisert

This work may be distributed and/or modified under the
conditions of the \LaTeX{} Project Public License, either version 1.3
of this license or (at your option) any later version.
The latest version of this license is in
  \url{http://www.latex-project.org/lppl.txt}
and version 1.3 or later is part of all distributions of \LaTeX{}
version 2005/12/01 or later.

This work has the LPPL maintenance status `maintained'.

The Current Maintainer of this work is Niklas Beisert.

This work consists of the files |README.txt|, |childdoc.ins| and |childdoc.dtx|
as well as the derived files |childdoc.def|, |cdocsamp.tex|
with |cdocsch1.tex|, |cdocsch2.tex|, |cdocspt3.tex|, |cdocspt4.tex|,
|cdocsdrf.tex|, |cdocsfn1.tex|, |cdocsfn2.tex|
as well as |childdoc.pdf|.

%%%%%%%%%%%%%%%%%%%%%%%%%%%%%%%%%%%%%%%%%%%%%%%%%%%%%%%%%%%%%%%%%%%%%%%%%%%%%%%%
\subsection{Files and Installation}

The package consists of the files:
%
\begin{center}
\begin{tabular}{ll}
    |README.txt|   & readme file \\
    |childdoc.ins| & installation file \\
    |childdoc.dtx| & source file \\
    |childdoc.def| & definition file \\
    |cdocsamp.tex| & sample main file \\
    |cdocsch1.tex| & sample include file \\
    |cdocsch2.tex| & sample include file \\
    |cdocspt3.tex| & sample part file \\
    |cdocspt4.tex| & sample part file \\
    |cdocsdrf.tex| & sample redirection file \\
    |cdocsfn1.tex| & sample redirection file \\
    |cdocsfn2.tex| & sample redirection file \\
    |childdoc.pdf| & manual
\end{tabular}
\end{center}
%
The distribution consists of the files
|README.txt|, |childdoc.ins| and |childdoc.dtx|.
%
\begin{itemize}
\item
Run (pdf)\LaTeX{} on |childdoc.dtx|
to compile the manual |childdoc.pdf| (this file).
\item
Run \LaTeX{} on |childdoc.ins| to create the definitions file |childdoc.def|
and the sample |cdocsamp.tex| with include files
|cdocsch1.tex|, |cdocsch2.tex|, |cdocspt3.tex|, |cdocspt4.tex|,
|cdocsdrf.tex|, |cdocsfn1.tex|, |cdocsfn2.tex|.
Then copy the file |childdoc.def| to an appropriate directory of your \LaTeX{}
distribution, e.g.\ \textit{texmf-root}|/tex/latex/childdoc|.
\end{itemize}

%%%%%%%%%%%%%%%%%%%%%%%%%%%%%%%%%%%%%%%%%%%%%%%%%%%%%%%%%%%%%%%%%%%%%%%%%%%%%%%%
\subsection{Related CTAN Packages}

There are several other packages which offer a similar functionality:
%
\begin{itemize}
\item
The packages
\href{http://ctan.org/pkg/docmute}{\textsf{docmute}},
\href{http://ctan.org/pkg/includex}{\textsf{includex}} and
\href{http://ctan.org/pkg/standalone}{\textsf{standalone}}
provide commands to include only the document body of
a child file thus allowing both files to be compiled individually.
\item
The packages \href{http://ctan.org/pkg/subdocs}{\textsf{subdocs}}
and \href{http://ctan.org/pkg/subfiles}{\textsf{subfiles}}
provide structures in which the main and child documents can be
encapsulated and allowing them to be compiled individually.
The inclusion mechanism is different from the conventional |\include|.
\item
The package \href{http://ctan.org/pkg/combine}{\textsf{combine}}
is an elaborate solution to combine several documents into one.
\end{itemize}
%
See also the CTAN topic \href{http://ctan.org/topic/subdocs}{\textsf{subdocs}}
for further related packages.
The present package differs from the above solutions in that
a document structure constructed with the conventional |\include| mechanism
just needs two extra commands at the top of every file
such that all constituent files can be compiled individually.

%%%%%%%%%%%%%%%%%%%%%%%%%%%%%%%%%%%%%%%%%%%%%%%%%%%%%%%%%%%%%%%%%%%%%%%%%%%%%%%%
%\subsection{Feature Suggestions}
%
%The following is a list of features which may be useful for future
%versions of this package:
%%
%\begin{itemize}
%\item
%\ldots
%\end{itemize}

%%%%%%%%%%%%%%%%%%%%%%%%%%%%%%%%%%%%%%%%%%%%%%%%%%%%%%%%%%%%%%%%%%%%%%%%%%%%%%%%
\subsection{Revision History}

%%%%%%%%%%%%%%%%%%%%%%%%%%%%%%%%%%%%%%%%
\paragraph{v2.0:} 2018/12/30

\begin{itemize}
\item
immediate forward processing
\item
added |\childdocby| mechanism
\item
manual restructured
\end{itemize}

%%%%%%%%%%%%%%%%%%%%%%%%%%%%%%%%%%%%%%%%
\paragraph{v1.6:} 2018/01/17

\begin{itemize}
\item
application for development of include files
\item
corrections to manual
\end{itemize}

%%%%%%%%%%%%%%%%%%%%%%%%%%%%%%%%%%%%%%%%
\paragraph{v1.5:} 2017/05/21

\begin{itemize}
\item
more complete structuring introduced
\item
|\childdocof| introduced
\item
|\childdoc| renamed to |\childdocmain|
\item
|\childredirect| renamed to |\childdocforward| and |\childdocforwardprefix|
and functionality expanded
\end{itemize}

%%%%%%%%%%%%%%%%%%%%%%%%%%%%%%%%%%%%%%%%
\paragraph{v1.0:} 2017/04/27

\begin{itemize}
\item
manual and install package
\item
first version published on CTAN
\end{itemize}

%%%%%%%%%%%%%%%%%%%%%%%%%%%%%%%%%%%%%%%%
\paragraph{v0.6:} 2017/04/26

\begin{itemize}
\item
redirection mechanism added
\end{itemize}

%%%%%%%%%%%%%%%%%%%%%%%%%%%%%%%%%%%%%%%%
\paragraph{v0.5:} 2017/04/26

\begin{itemize}
\item
functionality in definition file
\end{itemize}


%%%%%%%%%%%%%%%%%%%%%%%%%%%%%%%%%%%%%%%%%%%%%%%%%%%%%%%%%%%%%%%%%%%%%%%%%%%%%%%%
%%%%%%%%%%%%%%%%%%%%%%%%%%%%%%%%%%%%%%%%%%%%%%%%%%%%%%%%%%%%%%%%%%%%%%%%%%%%%%%%
%%%%%%%%%%%%%%%%%%%%%%%%%%%%%%%%%%%%%%%%%%%%%%%%%%%%%%%%%%%%%%%%%%%%%%%%%%%%%%%%
\appendix

\settowidth\MacroIndent{\rmfamily\scriptsize 000\ }

 \DocInput{childdoc.dtx}

\end{document}
%</driver>
% \fi
%
% %%%%%%%%%%%%%%%%%%%%%%%%%%%%%%%%%%%%%%%%%%%%%%%%%%%%%%%%%%%%%%%%%%%%%%%%%%%%%%
% %%%%%%%%%%%%%%%%%%%%%%%%%%%%%%%%%%%%%%%%%%%%%%%%%%%%%%%%%%%%%%%%%%%%%%%%%%%%%%
% \section{Sample}
%\iffalse
%<*samplemain>
%\fi
%
% The following presents a sample document
% with two chapters, two parts, a title page,
% a compile flag as well as three forwarding files to set the flag.
% It consists of eight |.tex| files:
% \begin{center}
% \begin{tabular}{ll}
% |cdocsamp.tex|&main file\\
% |cdocsch1.tex|&include file for chapter 1\\
% |cdocsch2.tex|&include file for chapter 2\\
% |cdocspt3.tex|&include file for part 3\\
% |cdocspt4.tex|&include file for part 4\\
% |cdocsdrf.tex|&forwarding file for main file in draft mode\\
% |cdocsfi1.tex|&forwarding file for final version of chapter 1\\
% |cdocsfi2.tex|&forwarding file for final version of chapter 2\\
% \end{tabular}
% \end{center}
% Each of the eight files can be compiled directly by the \LaTeX{} compiler.
%
% %%%%%%%%%%%%%%%%%%%%%%%%%%%%%%%%%%%%%%
% \paragraph{Main File.}
%
% The main file is called |cdocsamp.tex|.
%
% Load the \textsf{childdoc} definitions and
% declare the filename for the main document:
%    \begin{macrocode}
\input{childdoc.def}
\childdocmain{}
%    \end{macrocode}

% Optional override for |\version| flag:
%    \begin{macrocode}
%%\ifchilddoc\else\providecommand{\version}{draft}\fi
%    \end{macrocode}

% Define the default values for the |\version| flag
% (|final| for the main file and |draft| for childs):
%    \begin{macrocode}
\ifchilddoc
\providecommand{\version}{draft}
\else
\providecommand{\version}{final}
\fi
%    \end{macrocode}

% Load the standard document class:
%    \begin{macrocode}
\documentclass[12pt]{article}
%    \end{macrocode}

% Start the document body:
%    \begin{macrocode}
\begin{document}
%    \end{macrocode}

% Declare a title page.
% Print title, part of document being processed and version flag:
%    \begin{macrocode}
\addtocounter{page}{-1}
\begin{center}
{\LARGE\bfseries{}childdoc example\par}
\vspace{1cm}
\ifchilddoc
\ifchilddocmanual part\else chapter\fi:
`\childdocname' of `\childdocjob'\par
\else
main document: `\childdocjob'\par
\fi
version: \version\par
\end{center}
\newpage
%    \end{macrocode}

% Manually include selected file,
% otherwise process as usual:
%    \begin{macrocode}
\ifchilddocmanual
\section*{part `\childdocname'}
\input{\childdocname}
\else
%    \end{macrocode}

% Include the two chapters:
%    \begin{macrocode}
\include{cdocsch1}
\include{cdocsch2}
%    \end{macrocode}

% Include the two parts unless only chapters should be displayed:
%    \begin{macrocode}
\ifchilddoc\else
\section{part three}
\input{cdocspt3}
\section{part four}
\input{cdocspt4}
\fi
%    \end{macrocode}

% Process as usual until here:
%    \begin{macrocode}
\fi
%    \end{macrocode}

% End of document body:
%    \begin{macrocode}
\end{document}
%    \end{macrocode}
%\iffalse
%</samplemain>
%\fi
%
% %%%%%%%%%%%%%%%%%%%%%%%%%%%%%%%%%%%%%%
% \paragraph{Chapter Include Files.}
%
% The include files are called |cdocsch1.tex| and |cdocsch2.tex|.
%
%\iffalse
%<*samplechap1|samplechap2>
%\fi

% Optional override for |\version| flag:
%    \begin{macrocode}
%%\providecommand{\version}{final}
%    \end{macrocode}

% Include the main document:
%    \begin{macrocode}
\input{childdoc.def}
\childdocof{cdocsamp}
%    \end{macrocode}

%\iffalse
%</samplechap1|samplechap2>
%\fi
%
%\iffalse
%<*samplechap1>
%\fi
% Some text for chapter 1:
%    \begin{macrocode}
\section{one}
some text in chapter one
%    \end{macrocode}

%\iffalse
%</samplechap1>
%\fi
% Some text for chapter 2:
%\iffalse
%<*samplechap2>
%\fi
%    \begin{macrocode}
\section{two}
more text in chapter two
%    \end{macrocode}

%\iffalse
%</samplechap2>
%\fi
%
% %%%%%%%%%%%%%%%%%%%%%%%%%%%%%%%%%%%%%%
% \paragraph{Part Include Files.}
%
% The include files are called |cdocspt3.tex| and |cdocspt4.tex|.
%
%\iffalse
%<*samplepart3|samplepart4>
%\fi

% Optional override for |\version| flag:
%    \begin{macrocode}
%%\providecommand{\version}{final}
%    \end{macrocode}

% Include the main document:
%    \begin{macrocode}
\input{childdoc.def}
\childdocby{cdocsamp}
%    \end{macrocode}

%\iffalse
%</samplepart3|samplepart4>
%\fi
%
%\iffalse
%<*samplepart3>
%\fi
% Some text for part 3:
%    \begin{macrocode}
some text in part three
%    \end{macrocode}

%\iffalse
%</samplepart3>
%\fi
% Some text for part 4:
%\iffalse
%<*samplepart4>
%\fi
%    \begin{macrocode}
more text in part four
%    \end{macrocode}

%\iffalse
%</samplepart4>
%\fi
%
% %%%%%%%%%%%%%%%%%%%%%%%%%%%%%%%%%%%%%%
% \paragraph{Forwarding for a Complete Draft.}
%
% The following forwarding file |cdocsdrf.tex|
% compiles the main document in draft mode:
%\iffalse
%<*sampledraft>
%\fi
%    \begin{macrocode}
\def\version{draft}
\input{childdoc.def}
\childdocforward{cdocsamp}
%    \end{macrocode}

%\iffalse
%</sampledraft>
%\fi
%
% %%%%%%%%%%%%%%%%%%%%%%%%%%%%%%%%%%%%%%
% \paragraph{Forwarding for Final Version of the Chapters.}
%
% The following forwarding files |cdocsfn1.tex| and |cdocsfn2.tex|
% (with identical content)
% compile the final versions of the child documents
% |cdocsch1.tex| and |cdocsch2.tex|, respectively:
%\iffalse
%<*samplefinal>
%\fi
%    \begin{macrocode}
\def\version{final}
\input{childdoc.def}
\childdocforwardprefix[cdocsamp]{cdocsfn}{cdocsch}
%    \end{macrocode}

%\iffalse
%</samplefinal>
%\fi
%
% %%%%%%%%%%%%%%%%%%%%%%%%%%%%%%%%%%%%%%
% \paragraph{Command Line Processing.}
%
% The following three command lines generate the output files
% |cdocscld|, |cdocscl1| and |cdocscl2|
% which should be identical to
% |cdocsdrf|, |cdocsch1| and |cdocsfn2|, respectively:
% \begin{center}
% \begin{tabular}{l}
% |latex -jobname cdocscld \|\\
% |  "\def\version{draft}\input{childdoc.def}\childdocforward{cdocsamp}"|\\
% |latex -jobname cdocscl1 \|\\
% |  "\input{childdoc.def}\childdocforward[cdocsamp]{cdocsch1}"|\\
% |latex -jobname cdocscl2 \|\\
% |  "\def\version{final}\input{childdoc.def}\childdocforward{cdocsch2}"|
% \end{tabular}
% \end{center}
% Note that the trailing backslash on each first line
% merely continues the input to the second line
% (for convenient cut ant paste).
% Furthermore, the command |latex| can be replaced by any
% of its alternative versions such as |pdflatex|.
%
% %%%%%%%%%%%%%%%%%%%%%%%%%%%%%%%%%%%%%%%%%%%%%%%%%%%%%%%%%%%%%%%%%%%%%%%%%%%%%%
% %%%%%%%%%%%%%%%%%%%%%%%%%%%%%%%%%%%%%%%%%%%%%%%%%%%%%%%%%%%%%%%%%%%%%%%%%%%%%%
% \section{Implementation}
%\iffalse
%<*package>
%\fi
%
% This section describes the definitions file |childdoc.def|.

% The definitions cannot be loaded using |\usepackage| or |\RequirePackage|
% which has a mechanism to prevent loading a style file more than once.
% When loading the definitions by means of |\input|
% multiple instances have to be prevented manually:
%\iffalse
%This code needs to be before the `\ProvidesFile' directive
%which is defined at the beginning of this file.
%Therefore it is also placed there and commented out here.
%</package>
%<*discard>
%\fi
%    \begin{macrocode}
\ifdefined\childdocmain\endinput\fi
%    \end{macrocode}
%\iffalse
%</discard>
%<*package>
%\fi
%
% \macro{\ifchilddoc}
% \macro{\ifchilddocmanual}
% The conditional |\ifchilddoc| tells whether a
% child (true) or main (false) document is being compiled.
% The conditional |\ifchilddocmanual| tells whether
% the |\includeonly| mechanism is used (false) or
% the selection of child files must be performed manually (true).
% The definitions initialise to false:
%    \begin{macrocode}
\newif\ifchilddoc
\newif\ifchilddocmanual
%    \end{macrocode}

% \macro{\childdocname}
% \macro{\childdocjob}
% The macro |\childdocname| stores the name of the main document
% to be compiled. The macro |\childdocjob| stores the name of
% the document on which the \LaTeX{} compiler was originally invoked.
% The content of |\jobname| cannot be compared
% to filenames specified in the source due to different catcodes.
% The following code rescans |\jobname|, stores the result
% in |\childdocname| and saves a copy in |\childdocjob|:
%    \begin{macrocode}
\edef\childdocname{\scantokens\expandafter{\jobname\noexpand}}
\let\childdocjob\childdocname
%    \end{macrocode}

% \macro{\childdocdisable}
% The macro |\childdocdisable| prevents the main file
% from being processed more than once.
% At this stage, the main document command |\childdocmain|
% is assumed to be called once again where it should do nothing.
% Any subsequent call to it should prevent
% a secondary processing of the main document
% It overwrites the forwarding commands
% |\childdocof| and |\childdocforward|
% with empty macros to prevent further inclusions of the main document:
%    \begin{macrocode}
\newcommand{\childdocdisable}
{
  \renewcommand{\childdocmain}[1]{\renewcommand{\childdocmain}[1]{\endinput}}
  \renewcommand{\childdocof}[1]{}
  \renewcommand{\childdocby}[2][]{}
  \renewcommand{\childdocforward}[2][]{}
  \renewcommand{\childdocdisable}{}
}
%    \end{macrocode}

% \macro{\childdocmain}
% The macro |\childdocmain| is to be called at the top of the main file
% with nothing or the main filename (without extension) as argument.
% First, it breaks loops.
% If the argument is not empty and does not match |\childdocname|
% (which is set by the first inclusion of |childdoc.def|),
% |\ifchilddoc| is set to true, |\includeonly| is applied to the child file
% and |\jobname| is set to the main file
% (for proper handling of |.aux| files):
%    \begin{macrocode}
\newcommand{\childdocmain}[1]
{
  \childdocdisable\childdocmain{}
  \if?#1?\else
    \begingroup
      \def\childdoctmp{#1}
      \ifx\childdoctmp\childdocname
        \def\childdoctmp{}
      \else
        \def\childdoctmp
        {
          \childdoctrue
          \includeonly{\childdocname}
          \def\childdocjob{#1}
          \def\jobname{#1}
        }
      \fi
      \expandafter
    \endgroup
    \childdoctmp
  \fi
}
%    \end{macrocode}

% \macro{\childdocof}
% The command |\childdocof| redirects
% compilation to the main file |#1|.
%    \begin{macrocode}
\newcommand{\childdocof}[1]
{
  \childdocdisable
  \childdoctrue
  \includeonly{\childdocname}
  \def\jobname{#1}
  \def\childdocjob{#1}
  \input{#1}
}
%    \end{macrocode}

% \macro{\childdocby}
% The command |\childdocby| ....
%    \begin{macrocode}
\newcommand{\childdocby}[2][]
{
  \childdocdisable
  \childdoctrue
  \childdocmanualtrue
  \if?#1?\else
    \def\jobname{#2}
  \fi
  \def\childdocjob{#2}
  \input{#2}
  \endinput
}
%    \end{macrocode}

% \macro{\childdocforward}
% The command |\childdocforward| redirects
% compilation to the main file or
% (if the optional argument is given) a child file.
% Parameters are set as if the main file
% or a child file starting with |\childdocof| was compiled.
% Then compilation is handed over to the main file:
%    \begin{macrocode}
\newcommand{\childdocforward}[2][]
{
  \begingroup
    \if?#1?
      \def\childdoctmp
      {
        \def\childdocname{#2}
        \def\childdocjob{#2}
        \def\jobname{#2}
        \input{#2}
        \endinput
      }
    \else
      \def\childdoctmp
      {
        \childdocdisable
        \def\childdocname{#2}
        \childdoctrue
        \includeonly{#2}
        \def\childdocjob{#1}
        \def\jobname{#1}
        \input{#1}
        \endinput
      }
    \fi
    \expandafter
  \endgroup
  \childdoctmp
}
%    \end{macrocode}

% \macro{\childdocforwardprefix}
% The command |\childdocforwardprefix| redirects
% compilation to the main or a child file by means of a pattern.
% The prefix |#1| in the current filename is replaced by |#2|
% and the suffix of the current filename is kept
% (it is assumed that the filename does not contain the substring `|~~~|'
% which is used as a delimiter).
% Compilation is handed over to the new file by |\childdocforward|:
%    \begin{macrocode}
\newcommand{\childdocforwardprefix}[3][]
{
  \begingroup
    \def\childdocextract #2##1~~~{\def\childdoctmp{\childdocforward[#1]{#3##1}}}
    \expandafter\childdocextract\childdocname~~~
    \expandafter
  \endgroup
  \childdoctmp
}
%    \end{macrocode}

% \macro{\childdoc}
% The deprecated macro |\childdoc| is a legacy version of |\childdocmain|:
%    \begin{macrocode}
\newcommand{\childdoc}{\childdocmain}
%    \end{macrocode}

% \macro{\childdocredirect}
% The deprecated macro |\childdocredirect| is a legacy version
% of |\childdocforward| and |\childdocforwardprefix|:
%    \begin{macrocode}
\newcommand{\childdocredirect}[2][]
{
  \begingroup
    \if?#1?
      \def\childdoctmp{\childdocforward{#2}}
    \else
      \def\childdoctmp{\childdocforwardprefix{#1}{#2}}
    \fi
    \expandafter
  \endgroup
  \childdoctmp
}
%    \end{macrocode}

%\iffalse
%</package>
%\fi
%
\endinput
|\\
|\childdocforwardprefix[|\textit{main}|]{|\textit{prefix}|}{|\textit{dest}|}|
\end{tabular}
\end{center}
%
the destination file is determined by a pattern
depending on the current file:
To make this work, the current file must be called
`{\textit{prefix}\hspace{0.2em}\textit{suffix}}'
with \textit{prefix} matching precisely the argument.
Processing is then passed on to the file
`{\textit{dest}\hspace{0.2em}\textit{suffix}}'.
Surely, the same effect is achieved by
directly specifying the
argument `{\textit{dest}\hspace{0.2em}\textit{suffix}}'
in the first form.
However, that requires to set up a different file
for each child. With the alternative form of the command
all these files can have exactly the same content
which simplifies setting them up and maintaining them.

For example, the following file |draft.tex|
with a compilation flag |\version| as described in \secref{sec:flags}
compiles the main document as a draft:
%
\begin{center}
\begin{tabular}{l}
|\def\version{draft}|\\
|% \iffalse
%
% childdoc.dtx Copyright (C) 2017-2018 Niklas Beisert
%
% This work may be distributed and/or modified under the
% conditions of the LaTeX Project Public License, either version 1.3
% of this license or (at your option) any later version.
% The latest version of this license is in
%   http://www.latex-project.org/lppl.txt
% and version 1.3 or later is part of all distributions of LaTeX
% version 2005/12/01 or later.
%
% This work has the LPPL maintenance status `maintained'.
%
% The Current Maintainer of this work is Niklas Beisert.
%
% This work consists of the files childdoc.dtx and childdoc.ins
% and the derived files childdoc.def and cdocsamp.tex with
% cdocsch1.tex, cdocsch2.tex, cdocsdrf.tex, cdocsfn1.tex, cdocsfn2.tex.
%
%<package>\ifdefined\childdocmain\endinput\fi
%<package>\ProvidesFile{childdoc.def}[2018/12/30 v2.0 child document driver]
%<samplemain>\ProvidesFile{cdocsamp.tex}[2018/12/30 v2.0 sample for childdoc]
%<*driver>
%\ProvidesFile{childdoc.drv}[2018/12/30 v2.0 childdoc reference manual file]
\PassOptionsToClass{10pt,a4paper}{article}
\documentclass{ltxdoc}

\usepackage[margin=35mm]{geometry}
\usepackage{hyperref}
\usepackage{hyperxmp}
\usepackage[usenames]{color}

\hypersetup{colorlinks=true}
\hypersetup{pdfstartview=FitH}
\hypersetup{pdfpagemode=UseNone}
\hypersetup{pdfsource={}}
\hypersetup{pdflang={en-UK}}
\hypersetup{pdfcopyright={Copyright 2017-2018 Niklas Beisert.
  This work may be distributed and/or modified under the
  conditions of the LaTeX Project Public License, either version 1.3
  of this license or (at your option) any later version.}}
\hypersetup{pdflicenseurl={http://www.latex-project.org/lppl.txt}}
\hypersetup{pdfcontactaddress={ETH Zurich, ITP, HIT K,
  Wolfgang-Pauli-Strasse 27}}
\hypersetup{pdfcontactpostcode={8093}}
\hypersetup{pdfcontactcity={Zurich}}
\hypersetup{pdfcontactcountry={Switzerland}}
\hypersetup{pdfcontactemail={nbeisert@itp.phys.ethz.ch}}
\hypersetup{pdfcontacturl={http://people.phys.ethz.ch/\xmptilde nbeisert/}}

\newcommand{\secref}[1]{\hyperref[#1]{section \ref*{#1}}}

\parskip1ex
\parindent0pt
\let\olditemize\itemize
\def\itemize{\olditemize\parskip0pt}

\begin{document}

\title{The \textsf{childdoc} Package}
\hypersetup{pdftitle={The childdoc Package}}
\author{Niklas Beisert\\[2ex]
  Institut f\"ur Theoretische Physik\\
  Eidgen\"ossische Technische Hochschule Z\"urich\\
  Wolfgang-Pauli-Strasse 27, 8093 Z\"urich, Switzerland\\[1ex]
  \href{mailto:nbeisert@itp.phys.ethz.ch}
  {\texttt{nbeisert@itp.phys.ethz.ch}}}
\hypersetup{pdfauthor={Niklas Beisert}}
\hypersetup{pdfsubject={Manual for the LaTeX2e Package childdoc}}
\date{30 December 2018, \textsf{v2.0}}
\maketitle

\begin{abstract}\noindent
\textsf{childdoc} is a \LaTeXe{} package
that enables the direct compilation
of document sections included by |\include|
to individual files.
\end{abstract}

\begingroup
\parskip0ex
\tableofcontents
\endgroup

%%%%%%%%%%%%%%%%%%%%%%%%%%%%%%%%%%%%%%%%%%%%%%%%%%%%%%%%%%%%%%%%%%%%%%%%%%%%%%%%
%%%%%%%%%%%%%%%%%%%%%%%%%%%%%%%%%%%%%%%%%%%%%%%%%%%%%%%%%%%%%%%%%%%%%%%%%%%%%%%%
\section{Introduction}

\LaTeX{} provides a mechanism to structure a large document (such as a book)
into a main file and several child files (containing the chapters)
using the |\include| command.
This mechanism is beneficial for documents
which span hundreds of pages in order to
make the source file(s) more manageable.
Moreover, compilation can be restricted to
selected child files by means of the |\includeonly| command.
The latter feature can be used to reduce the compilation time while editing
(this was significantly more useful in the earlier days of \LaTeX{})
or to generate a smaller document which is easier to navigate.
Another application of |\includeonly| is to generate
documents consisting of selected parts of the complete document.

However, there are a few drawbacks of the plain |\include| mechanism:
\begin{itemize}
\item
The child files cannot be compiled on their own,
they can only be compiled via the main file.
A naive editing environment
(such as a text editor with an option
to have the current file processed by \LaTeX)
may require one to switch to the main file before compiling;
attempting to compile the child file produces errors.
\item
The main file must be modified (each time)
to adjust the |\includeonly| command
to the present needs. This easily leaves the main file in a messy state.
\item
The generated document will always carry the filename
of the main document. This is inconvenient if
several child files are to be compiled and
to be kept for distribution.
\end{itemize}

The present package provides a simple interface
to make child files individually compilable by \LaTeX{}.
Compiling a child file then has the same effect as compiling
the main file with an |\includeonly| command
to select the appropriate child.
Moreover the generated document will carry the name of the child
rather than the main file.
This resolves all three above issues.

This feature is meant to make the editing of books,
thesis documents and lecture notes somewhat more convenient.
However, the package can also be used efficiently for
composing a series of documents (such as exercise sheets)
which are typically distributed individually.
It then assists the author in generating the individual documents
(potentially in different versions)
as well as a document containing the collected series.
Another application is in developing style files
or other kinds of included material
where compilation of the style file could redirect
to a sample or test file.

%%%%%%%%%%%%%%%%%%%%%%%%%%%%%%%%%%%%%%%%%%%%%%%%%%%%%%%%%%%%%%%%%%%%%%%%%%%%%%%%
%%%%%%%%%%%%%%%%%%%%%%%%%%%%%%%%%%%%%%%%%%%%%%%%%%%%%%%%%%%%%%%%%%%%%%%%%%%%%%%%
\section{Usage}

First of all, the package \textsf{childdoc} is \emph{not} a standard
\LaTeXe{} |.sty| style file! Therefore it needs to be invoked in
a non-standard way.

%%%%%%%%%%%%%%%%%%%%%%%%%%%%%%%%%%%%%%%%%%%%%%%%%%%%%%%%%%%%%%%%%%%%%%%%%%%%%%%%
\subsection{Included Files}
\label{sec:include}

%%%%%%%%%%%%%%%%%%%%%%%%%%%%%%%%%%%%%%%%
\DescribeMacro{\childdocmain}
To use the package, add the commands
\begin{center}
\begin{tabular}{l}
|\input{childdoc.def}|\\
|\childdocmain{}|\\
\end{tabular}
\end{center}
at the very top of the main \LaTeX{} file,
in particular \emph{before} the |\documentclass| statement!
The argument of |\childdocmain| should be left empty
(but it must be present).

%%%%%%%%%%%%%%%%%%%%%%%%%%%%%%%%%%%%%%%%
\DescribeMacro{\childdocof}
Furthermore, add the commands
\begin{center}
\begin{tabular}{l}
|\input{childdoc.def}|\\
|\childdocof{|\textit{main}|}|\\
\end{tabular}
\end{center}
at the top of every child file \textit{child}
which is included by |\include{|\textit{child}|}|
from within the main file
(or at least for those files to be compiled individually).
The argument \textit{main} must be the filename of the main file.

There are a couple of
considerations in setting up the main and child documents:

%%%%%%%%%%%%%%%%%%%%%%%%%%%%%%%%%%%%%%%%
\paragraph{Restrictions.}

Please note the following restrictions:
\begin{itemize}
\item
|\childdocmain| must be called with one argument \textit{main}
to ensure compatibility with earlier version of the package.
It must either be empty (|\childdocmain{}|)
or precisely match the filename of the main file in which it is specified.
See \secref{sec:detection} for further information.
\item
The filename \textit{main} must be specified without the |.tex| extension.
\item
The filename \textit{main} is case sensitive
(even in case-insensitive file systems)
due to internal string comparison.
\item
The argument \textit{main} should be fully expanded, it cannot be a macro.
\item
Subdirectories and special characters should be avoided in filenames.
\item
The command |\childdocmain{|\textit{main}|}| must be followed by a whitespace.
It should not be followed immediately by another command
or by a comment mark `|%|'.
This is because the \TeX{} parser reads the token immediately following
the argument of |\childdocmain| and puts it
at the beginning of every child section;
however, a white\-space is ignored.
\end{itemize}

%%%%%%%%%%%%%%%%%%%%%%%%%%%%%%%%%%%%%%%%
\paragraph{Content of Main File.}

It is advisable to place all content in the child files included by |\include|.
Any output contained in the main file will appear in all child documents
unless suppressed manually;
it cannot be suppressed automatically by the |\includeonly| directive
and thus should normally be avoided.
A method to include some content in the main file
by means of conditional processing is described in \secref{sec:conditional}.

%%%%%%%%%%%%%%%%%%%%%%%%%%%%%%%%%%%%%%%%
\paragraph{Page Numbering.}

When only a part of the document is compiled,
the appropriate numbering of pages
(as well as other status parameters)
is determined from the |.aux| files.
The latter contain information from previous passes.
However this information needs to propagate through
all intermediate child documents.
Therefore the page numbering in child documents may well
be inconsistent until the complete document is compiled at least once.

A useful (if unconventional) way to always ensure a consistent
page numbering is to restart the numbering in each child document
and denote the pages by `\textit{child}|.|\textit{page}'
where \textit{child} represents the chapter/section number of the child file.
This can be achieved by the command
|\numberwithin{page}{|\textit{child}|}|
of the \textsf{amsmath} package
where \textit{child} can be |chapter| or |section|
depending on the chosen structuring.
Alternatively, one can modify the macro |\thepage| appropriately
and reset the counter |page| at the start of each child file.

%%%%%%%%%%%%%%%%%%%%%%%%%%%%%%%%%%%%%%%%%%%%%%%%%%%%%%%%%%%%%%%%%%%%%%%%%%%%%%%%
\subsection{Conditional Processing}
\label{sec:conditional}

The package provides a mechanism to compile different versions
of a document. To customise the versions further some conditional processing
can come in handy to distinguish which version is being compiled.
The package provides two macros to describe the compilation context:

%%%%%%%%%%%%%%%%%%%%%%%%%%%%%%%%%%%%%%%%
\DescribeMacro{\ifchilddoc}
The conditional |\ifchilddoc| distinguishes between the compilation of
child documents and the main document:
%
\begin{center}
|\ifchilddoc |\textit{child-code}| |[|\||else |\textit{main-code}]| \||fi|
\end{center}

%%%%%%%%%%%%%%%%%%%%%%%%%%%%%%%%%%%%%%%%
\DescribeMacro{\childdocname}
\DescribeMacro{\childdocjob}
The macro |\childdocname| contains the filename (without extension)
of the main or child file being processed.
Note that |\childdocjob| will always contain the name of the main file.

%%%%%%%%%%%%%%%%%%%%%%%%%%%%%%%%%%%%%%%%
\paragraph{Title Page.}

Conditional processing can be used to include a title or banner page
in the main document when proper precautions are taken.
Importantly, the code in the main file should ensure that the page counter
(as well as other status parameters which are stored in the |.aux| files)
takes the same value after the conditional processing.
Otherwise the page numbers may take divergent values
depending on which part is compiled.

For example, a title page could be declared by:
%
\begin{center}
\begin{tabular}{l}
|\ifchilddoc\||else|\\
|\addtocounter{page}{-1}|\\
\textit{code for title page}\\
|\newpage|\\
|\||fi|
\end{tabular}
\end{center}
%
A banner page for the child documents can be generated by:
%
\begin{center}
\begin{tabular}{l}
|\ifchilddoc|\\
|\addtocounter{page}{-1}|\\
\textit{code for banner page}\\
|\newpage|\\
|\||fi|
\end{tabular}
\end{center}
%
Here one could write a message such as:
\begin{center}
|This is the part \childdocname{} of \childdocjob{}.|
\end{center}

%%%%%%%%%%%%%%%%%%%%%%%%%%%%%%%%%%%%%%%%%%%%%%%%%%%%%%%%%%%%%%%%%%%%%%%%%%%%%%%%
\subsection{Flags}
\label{sec:flags}

The package makes it easy to generate different versions
of the main or child documents.
To this end compilation flags can be defined
and assigned different default values.
They will be particularly useful in conjunction
with the forwarding mechanism described in \secref{sec:forward}.

For example, it may be useful to have a flag |\version|
which can be set to |draft| or |final|.
The document source will contain some conditional code
depending on the value of |\version|.
Suppose further, the flag should default to |final| for the main file
and to |draft| for child files
which is a natural assignment for editing the document.
This is achieved by placing the following code
in the preamble of the main document
(below the |\childdocmain| directive):
%
\begin{center}
\begin{tabular}{l}
|\ifchilddoc|\\
|\providecommand{\version}{draft}|\\
|\||else|\\
|\providecommand{\version}{final}|\\
|\||fi|
\end{tabular}
\end{center}
%
The definition by |\providecommand| makes sure
that previous definitions are not overwritten.
Further statements |\providecommand{\version}{...}|
can thus be added before the above code to override it.

For the main file, one might add a line
(between |\childdocmain| and the above block)
%
\begin{center}
|%\ifchilddoc\||else\providecommand{\version}{draft}\||fi|
\end{center}
%
which can be uncommented to produce a draft version.
Likewise one can add a line to the very top of a child file
(above the |\childdocof{|\textit{main}|}| directive)
%
\begin{center}
|%\providecommand{\version}{final}|
\end{center}
%
which can be uncommented to produce the final version of this child document.

%%%%%%%%%%%%%%%%%%%%%%%%%%%%%%%%%%%%%%%%%%%%%%%%%%%%%%%%%%%%%%%%%%%%%%%%%%%%%%%%
\subsection{Forwarding}
\label{sec:forward}

Different versions of the main or child documents
using compilation flags as described in \secref{sec:flags}
can be (permanently) stored in different files
for convenient compilation, viewing and distribution.
To this end, the package defines a command
to pass on compilation to a different file:

%%%%%%%%%%%%%%%%%%%%%%%%%%%%%%%%%%%%%%%%
\DescribeMacro{\childdocforward}
The command |\childdocforward| redirects processing to
another source file:
%
\begin{center}
\begin{tabular}{l}
|\input{childdoc.def}|\\
|\childdocforward[|\textit{main}|]{|\textit{dest}|}|\\
\end{tabular}
\end{center}
%
The argument \textit{dest} is the destination file
(without extension).
It should be the main file or one of the child files.
Note that further \textsf{childdoc} directives
such as |\childdocof| and |\childdocforward|
in the indicated file will be processed in this form.
The optional argument \textit{main}
passes on directly to the main file \textit{main}
while pretending to compile the child \textit{dest}.
This form behaves as if \textit{dest}
issues |\childdocof{|\textit{main}|}| right away,
and no further \textsf{childdoc} directives will be processed.

%%%%%%%%%%%%%%%%%%%%%%%%%%%%%%%%%%%%%%%%
\DescribeMacro{\...prefix}
In the alternative form |\childdocforwardprefix|,
%
\begin{center}
\begin{tabular}{l}
|\input{childdoc.def}|\\
|\childdocforwardprefix[|\textit{main}|]{|\textit{prefix}|}{|\textit{dest}|}|
\end{tabular}
\end{center}
%
the destination file is determined by a pattern
depending on the current file:
To make this work, the current file must be called
`{\textit{prefix}\hspace{0.2em}\textit{suffix}}'
with \textit{prefix} matching precisely the argument.
Processing is then passed on to the file
`{\textit{dest}\hspace{0.2em}\textit{suffix}}'.
Surely, the same effect is achieved by
directly specifying the
argument `{\textit{dest}\hspace{0.2em}\textit{suffix}}'
in the first form.
However, that requires to set up a different file
for each child. With the alternative form of the command
all these files can have exactly the same content
which simplifies setting them up and maintaining them.

For example, the following file |draft.tex|
with a compilation flag |\version| as described in \secref{sec:flags}
compiles the main document as a draft:
%
\begin{center}
\begin{tabular}{l}
|\def\version{draft}|\\
|\input{childdoc.def}|\\
|\childdocforward{|\textit{main}|}|
\end{tabular}
\end{center}
%
Likewise, the following files |final|\textit{nn}|.tex|
compile the final version of the child document
|child|\textit{nn}|.tex|:
%
\begin{center}
\begin{tabular}{l}
|\def\version{final}|\\
|\input{childdoc.def}|\\
|\childdocforwardprefix{final}{child}|
\end{tabular}
\end{center}
%

Note that when several versions of a main file and/or of each child file
are to be generated, it may be convenient to set up a |Makefile| or
shell script to automatise the process.

%%%%%%%%%%%%%%%%%%%%%%%%%%%%%%%%%%%%%%%%%%%%%%%%%%%%%%%%%%%%%%%%%%%%%%%%%%%%%%%%
\subsection{Command Line Processing}
\label{sec:commandline}

The effect of redirection files can also be achieved by invoking
the \LaTeX{} compiler with a more elaborate command line.
Most conveniently this should be done as part
of a shell script or a |Makefile|.

When using \textsf{childdoc} in the main file, the following
command lines effectively perform a redirection
(note that depending on the shell being used,
backslashes may have to be doubled: `|\|' $\to$ `|\\|'):
%
\begin{center}
|... -jobname "|\textit{target}|" |\\|"|[\textit{flags}]%
|\input{childdoc.def}\childdocforward[|\textit{main}|]{|\textit{dest}|}"|
\end{center}
%
Here \textit{target} is the name of the output file,
\textit{main} is the name of the main file
and \textit{dest} is the name of the main or child file to be processed
(all filenames without extensions).
The optional argument \textit{main} can be omitted
if \textit{main} matches \textit{dest}.
Optionally, compilation \textit{flags} can be defined via |\def| commands.
This command line makes the \TeX{} engine believe
it is compiling the file \textit{target}
whose content is specified as the latter parameter.
The provided code then forwards the processing to
\textit{main} or \textit{dest} as described in \secref{sec:forward}.

%%%%%%%%%%%%%%%%%%%%%%%%%%%%%%%%%%%%%%%%%%%%%%%%%%%%%%%%%%%%%%%%%%%%%%%%%%%%%%%%
\subsection{Include by Input}
\label{sec:input}

Including child documents by |\include| has some restrictions by design.
Most notably, the content of a child document always occupies
its own set of pages; pages cannot be shared between child documents.
Usually, this behaviour makes perfect sense
because each child document contain an essential part of the document.
However, in some situations it may be desirable to compose
a document from a collection of parts
without having mandatory page breaks between then.
For this case, the package
provides a mechanism to include parts
by |\input| which can also be processed individually.
However, by construction this mechanism
requires manual handling of the content to be output.

%%%%%%%%%%%%%%%%%%%%%%%%%%%%%%%%%%%%%%%%
\DescribeMacro{\ifchilddocmanual}
The main file should be prepared as usual, see \secref{sec:include}.
However, the document body must make a distinction
between processing of an individual part and of the main document, e.g.:
%
\begin{center}
\begin{tabular}{l}
|\ifchilddocmanual|\\
|\input{\childdocname}|\\
|\||else|\\
\textit{document body with }|\input{|\textit{part}|}|\\
|\||fi|
\end{tabular}
\end{center}
%
The conditional |\ifchilddocmanual| is true whenever
a part to be included by |\input| is being compiled,
and the name of the part is stored in |\childdocname|.

%%%%%%%%%%%%%%%%%%%%%%%%%%%%%%%%%%%%%%%%
\DescribeMacro{\childdocby}
Each part to be included by |\input| should start with:
%
\begin{center}
\begin{tabular}{l}
|\input{childdoc.def}|\\
|\childdocby{|\textit{main}|}|\\
\end{tabular}
\end{center}
%
The directive |\childdocby| is similar to |\childdocof|
described in \secref{sec:include},
but the subsequent selection of content must be done manually.
To that end, both |\ifchilddoc| and |\ifchilddocmanual|
will be true upon processing of a part,
and the name of the part is stored in |\childdocname|.
Note that |\jobname| will be set to the filename of the current part
so that each part receives an individual |.aux| file
that does not interfere with the |.aux| file(s) of the main document.
This behaviour can be altered by the alternative form
|\childdocby[*]{|\textit{main}|}| (with a non-empty optional argument)
which uses the |.aux| file of the main document
by setting |\jobname| to \textit{main}.

%%%%%%%%%%%%%%%%%%%%%%%%%%%%%%%%%%%%%%%%%%%%%%%%%%%%%%%%%%%%%%%%%%%%%%%%%%%%%%%%
\subsection{Driver Development}
\label{sec:driver}

The \textsf{childdoc} mechanism can also be use for the development
of definition files such as \LaTeX{} styles or classes.
This case differs from the above setup with multiple parts
included by |\include| in that no |\includeonly| should be invoked.
This can be achieved by starting the include file
(before |\ProvidesPackage|) with:
%
\begin{center}
\begin{tabular}{l}
|\input{childdoc.def}|\\
|\childdocforward{|\textit{main}|}|\\
\end{tabular}
\end{center}
%
or alternatively with:
%
\begin{center}
\begin{tabular}{l}
|\input{childdoc.def}|\\
|\childdocby{|\textit{main}|}|\\
\end{tabular}
\end{center}
%
Both forms have slightly different effects as described above.
The main file is prepared as usual, see \secref{sec:include}.

%%%%%%%%%%%%%%%%%%%%%%%%%%%%%%%%%%%%%%%%%%%%%%%%%%%%%%%%%%%%%%%%%%%%%%%%%%%%%%%%
\subsection{Legacy Detection}
\label{sec:detection}

The directive |\childdocmain| in the main file can detect
whether the complete document or merely a child is to be compiled
even without using the directive |\childdocof|.
This method is deprecated because it is less robust
and there is no compelling reason to use it;
it is merely provided for backward compatibility
and it may be removed in future versions.

If the detection mechanism is to be used,
it is mandatory to correctly specify
the filename of the main file as the argument of |\childdocmain|:
%
\begin{center}
\begin{tabular}{l}
|\input{childdoc.def}|\\
|\childdocmain{|\textit{main}|}|\\
\end{tabular}
\end{center}
%
If |\jobname| does not match the argument \textit{main} of |\childdocmain|,
it is assumed that |\jobname| points to the child file to be compiled.
When using |\childdocmain| with the main file specified as argument,
it suffices to start a child file
with just |\input{|\textit{main}|}|
without loading of the package and using |\childdocof|.
If instead all processing is done
with the appropriate \textsf{childdoc} directives,
the argument of \textit{main} of |\childdocmain| can be empty.

An alternative version of the command line processing described
in \secref{sec:commandline} using the detection mechanism reads:
%
\begin{center}
|... -jobname "|\textit{target}|" "|[\textit{flags}]%
[|\def\jobname{|\textit{dest}|}|]|\input{|\textit{main}|}"|
\end{center}

%%%%%%%%%%%%%%%%%%%%%%%%%%%%%%%%%%%%%%%%%%%%%%%%%%%%%%%%%%%%%%%%%%%%%%%%%%%%%%%%
\subsection{Manual Code}
\label{sec:manual}

In case one cannot be certain whether the definitions file |childdoc.def|
is installed on the target \TeX{} distribution
and one prefers not to ship it,
it is conceivable to paste a few relevant commands into the sources.

To that end, drop all statements |\input{childdoc.def}|
and perform the replacements as outlined below.
Instead of |\childdocmain{|\textit{main}|}| add the following code
to the top of the main file:
%
\begin{center}
\begin{tabular}{l}
|\||ifdefined\childdocname\endinput\||fi\newif\ifchilddoc|\\
|\edef\childdocname{\scantokens\expandafter{\jobname\noexpand}}|\\
|\def\childdocmain{|\textit{main}|}\||ifx\childdocmain\childdocname\||else|\\
|\childdoctrue\includeonly{\childdocname}\let\jobname\childdocmain\||fi|\\
\end{tabular}
\end{center}
%
Instead of |\childdocof{|\textit{main}|}| just include the main file
at the top of each child file:
%
\begin{center}
|\input{|\textit{main}|}|
\end{center}
%
A simple redirection |\childdocforward{|\textit{dest}|}| is achieved by:
%
\begin{center}
|\def\jobname{|\textit{dest}|}\input{\jobname}|
\end{center}
%
The redirection with prefix
|\childdocforwardprefix[|\textit{prefix}|]{|\textit{dest}|}|
is accomplished by:
%
\begin{center}
\begin{tabular}{l}
|{\edef\jobname{\scantokens\expandafter{\jobname\noexpand}}|\\
|\def\redirectjob |\textit{prefix}|#1~~~{\gdef\jobname{|\textit{dest}|#1}}|\\
|\expandafter\redirectjob\jobname~~~}\input{\jobname}|
\end{tabular}
\end{center}

In an alternative approach,
child documents can be compiled by a specific command line
without additional code or specific definitions:
%
\begin{center}
|... -jobname "|\textit{target}|" "|[\textit{flags}]%
|\includeonly{|\textit{dest}|}\input{|\textit{main}|}"|
\end{center}
%

%%%%%%%%%%%%%%%%%%%%%%%%%%%%%%%%%%%%%%%%%%%%%%%%%%%%%%%%%%%%%%%%%%%%%%%%%%%%%%%%
%%%%%%%%%%%%%%%%%%%%%%%%%%%%%%%%%%%%%%%%%%%%%%%%%%%%%%%%%%%%%%%%%%%%%%%%%%%%%%%%
\section{Information}

%%%%%%%%%%%%%%%%%%%%%%%%%%%%%%%%%%%%%%%%%%%%%%%%%%%%%%%%%%%%%%%%%%%%%%%%%%%%%%%%
\subsection{Copyright}

Copyright \copyright{} 2017--2018 Niklas Beisert

This work may be distributed and/or modified under the
conditions of the \LaTeX{} Project Public License, either version 1.3
of this license or (at your option) any later version.
The latest version of this license is in
  \url{http://www.latex-project.org/lppl.txt}
and version 1.3 or later is part of all distributions of \LaTeX{}
version 2005/12/01 or later.

This work has the LPPL maintenance status `maintained'.

The Current Maintainer of this work is Niklas Beisert.

This work consists of the files |README.txt|, |childdoc.ins| and |childdoc.dtx|
as well as the derived files |childdoc.def|, |cdocsamp.tex|
with |cdocsch1.tex|, |cdocsch2.tex|, |cdocspt3.tex|, |cdocspt4.tex|,
|cdocsdrf.tex|, |cdocsfn1.tex|, |cdocsfn2.tex|
as well as |childdoc.pdf|.

%%%%%%%%%%%%%%%%%%%%%%%%%%%%%%%%%%%%%%%%%%%%%%%%%%%%%%%%%%%%%%%%%%%%%%%%%%%%%%%%
\subsection{Files and Installation}

The package consists of the files:
%
\begin{center}
\begin{tabular}{ll}
    |README.txt|   & readme file \\
    |childdoc.ins| & installation file \\
    |childdoc.dtx| & source file \\
    |childdoc.def| & definition file \\
    |cdocsamp.tex| & sample main file \\
    |cdocsch1.tex| & sample include file \\
    |cdocsch2.tex| & sample include file \\
    |cdocspt3.tex| & sample part file \\
    |cdocspt4.tex| & sample part file \\
    |cdocsdrf.tex| & sample redirection file \\
    |cdocsfn1.tex| & sample redirection file \\
    |cdocsfn2.tex| & sample redirection file \\
    |childdoc.pdf| & manual
\end{tabular}
\end{center}
%
The distribution consists of the files
|README.txt|, |childdoc.ins| and |childdoc.dtx|.
%
\begin{itemize}
\item
Run (pdf)\LaTeX{} on |childdoc.dtx|
to compile the manual |childdoc.pdf| (this file).
\item
Run \LaTeX{} on |childdoc.ins| to create the definitions file |childdoc.def|
and the sample |cdocsamp.tex| with include files
|cdocsch1.tex|, |cdocsch2.tex|, |cdocspt3.tex|, |cdocspt4.tex|,
|cdocsdrf.tex|, |cdocsfn1.tex|, |cdocsfn2.tex|.
Then copy the file |childdoc.def| to an appropriate directory of your \LaTeX{}
distribution, e.g.\ \textit{texmf-root}|/tex/latex/childdoc|.
\end{itemize}

%%%%%%%%%%%%%%%%%%%%%%%%%%%%%%%%%%%%%%%%%%%%%%%%%%%%%%%%%%%%%%%%%%%%%%%%%%%%%%%%
\subsection{Related CTAN Packages}

There are several other packages which offer a similar functionality:
%
\begin{itemize}
\item
The packages
\href{http://ctan.org/pkg/docmute}{\textsf{docmute}},
\href{http://ctan.org/pkg/includex}{\textsf{includex}} and
\href{http://ctan.org/pkg/standalone}{\textsf{standalone}}
provide commands to include only the document body of
a child file thus allowing both files to be compiled individually.
\item
The packages \href{http://ctan.org/pkg/subdocs}{\textsf{subdocs}}
and \href{http://ctan.org/pkg/subfiles}{\textsf{subfiles}}
provide structures in which the main and child documents can be
encapsulated and allowing them to be compiled individually.
The inclusion mechanism is different from the conventional |\include|.
\item
The package \href{http://ctan.org/pkg/combine}{\textsf{combine}}
is an elaborate solution to combine several documents into one.
\end{itemize}
%
See also the CTAN topic \href{http://ctan.org/topic/subdocs}{\textsf{subdocs}}
for further related packages.
The present package differs from the above solutions in that
a document structure constructed with the conventional |\include| mechanism
just needs two extra commands at the top of every file
such that all constituent files can be compiled individually.

%%%%%%%%%%%%%%%%%%%%%%%%%%%%%%%%%%%%%%%%%%%%%%%%%%%%%%%%%%%%%%%%%%%%%%%%%%%%%%%%
%\subsection{Feature Suggestions}
%
%The following is a list of features which may be useful for future
%versions of this package:
%%
%\begin{itemize}
%\item
%\ldots
%\end{itemize}

%%%%%%%%%%%%%%%%%%%%%%%%%%%%%%%%%%%%%%%%%%%%%%%%%%%%%%%%%%%%%%%%%%%%%%%%%%%%%%%%
\subsection{Revision History}

%%%%%%%%%%%%%%%%%%%%%%%%%%%%%%%%%%%%%%%%
\paragraph{v2.0:} 2018/12/30

\begin{itemize}
\item
immediate forward processing
\item
added |\childdocby| mechanism
\item
manual restructured
\end{itemize}

%%%%%%%%%%%%%%%%%%%%%%%%%%%%%%%%%%%%%%%%
\paragraph{v1.6:} 2018/01/17

\begin{itemize}
\item
application for development of include files
\item
corrections to manual
\end{itemize}

%%%%%%%%%%%%%%%%%%%%%%%%%%%%%%%%%%%%%%%%
\paragraph{v1.5:} 2017/05/21

\begin{itemize}
\item
more complete structuring introduced
\item
|\childdocof| introduced
\item
|\childdoc| renamed to |\childdocmain|
\item
|\childredirect| renamed to |\childdocforward| and |\childdocforwardprefix|
and functionality expanded
\end{itemize}

%%%%%%%%%%%%%%%%%%%%%%%%%%%%%%%%%%%%%%%%
\paragraph{v1.0:} 2017/04/27

\begin{itemize}
\item
manual and install package
\item
first version published on CTAN
\end{itemize}

%%%%%%%%%%%%%%%%%%%%%%%%%%%%%%%%%%%%%%%%
\paragraph{v0.6:} 2017/04/26

\begin{itemize}
\item
redirection mechanism added
\end{itemize}

%%%%%%%%%%%%%%%%%%%%%%%%%%%%%%%%%%%%%%%%
\paragraph{v0.5:} 2017/04/26

\begin{itemize}
\item
functionality in definition file
\end{itemize}


%%%%%%%%%%%%%%%%%%%%%%%%%%%%%%%%%%%%%%%%%%%%%%%%%%%%%%%%%%%%%%%%%%%%%%%%%%%%%%%%
%%%%%%%%%%%%%%%%%%%%%%%%%%%%%%%%%%%%%%%%%%%%%%%%%%%%%%%%%%%%%%%%%%%%%%%%%%%%%%%%
%%%%%%%%%%%%%%%%%%%%%%%%%%%%%%%%%%%%%%%%%%%%%%%%%%%%%%%%%%%%%%%%%%%%%%%%%%%%%%%%
\appendix

\settowidth\MacroIndent{\rmfamily\scriptsize 000\ }

 \DocInput{childdoc.dtx}

\end{document}
%</driver>
% \fi
%
% %%%%%%%%%%%%%%%%%%%%%%%%%%%%%%%%%%%%%%%%%%%%%%%%%%%%%%%%%%%%%%%%%%%%%%%%%%%%%%
% %%%%%%%%%%%%%%%%%%%%%%%%%%%%%%%%%%%%%%%%%%%%%%%%%%%%%%%%%%%%%%%%%%%%%%%%%%%%%%
% \section{Sample}
%\iffalse
%<*samplemain>
%\fi
%
% The following presents a sample document
% with two chapters, two parts, a title page,
% a compile flag as well as three forwarding files to set the flag.
% It consists of eight |.tex| files:
% \begin{center}
% \begin{tabular}{ll}
% |cdocsamp.tex|&main file\\
% |cdocsch1.tex|&include file for chapter 1\\
% |cdocsch2.tex|&include file for chapter 2\\
% |cdocspt3.tex|&include file for part 3\\
% |cdocspt4.tex|&include file for part 4\\
% |cdocsdrf.tex|&forwarding file for main file in draft mode\\
% |cdocsfi1.tex|&forwarding file for final version of chapter 1\\
% |cdocsfi2.tex|&forwarding file for final version of chapter 2\\
% \end{tabular}
% \end{center}
% Each of the eight files can be compiled directly by the \LaTeX{} compiler.
%
% %%%%%%%%%%%%%%%%%%%%%%%%%%%%%%%%%%%%%%
% \paragraph{Main File.}
%
% The main file is called |cdocsamp.tex|.
%
% Load the \textsf{childdoc} definitions and
% declare the filename for the main document:
%    \begin{macrocode}
\input{childdoc.def}
\childdocmain{}
%    \end{macrocode}

% Optional override for |\version| flag:
%    \begin{macrocode}
%%\ifchilddoc\else\providecommand{\version}{draft}\fi
%    \end{macrocode}

% Define the default values for the |\version| flag
% (|final| for the main file and |draft| for childs):
%    \begin{macrocode}
\ifchilddoc
\providecommand{\version}{draft}
\else
\providecommand{\version}{final}
\fi
%    \end{macrocode}

% Load the standard document class:
%    \begin{macrocode}
\documentclass[12pt]{article}
%    \end{macrocode}

% Start the document body:
%    \begin{macrocode}
\begin{document}
%    \end{macrocode}

% Declare a title page.
% Print title, part of document being processed and version flag:
%    \begin{macrocode}
\addtocounter{page}{-1}
\begin{center}
{\LARGE\bfseries{}childdoc example\par}
\vspace{1cm}
\ifchilddoc
\ifchilddocmanual part\else chapter\fi:
`\childdocname' of `\childdocjob'\par
\else
main document: `\childdocjob'\par
\fi
version: \version\par
\end{center}
\newpage
%    \end{macrocode}

% Manually include selected file,
% otherwise process as usual:
%    \begin{macrocode}
\ifchilddocmanual
\section*{part `\childdocname'}
\input{\childdocname}
\else
%    \end{macrocode}

% Include the two chapters:
%    \begin{macrocode}
\include{cdocsch1}
\include{cdocsch2}
%    \end{macrocode}

% Include the two parts unless only chapters should be displayed:
%    \begin{macrocode}
\ifchilddoc\else
\section{part three}
\input{cdocspt3}
\section{part four}
\input{cdocspt4}
\fi
%    \end{macrocode}

% Process as usual until here:
%    \begin{macrocode}
\fi
%    \end{macrocode}

% End of document body:
%    \begin{macrocode}
\end{document}
%    \end{macrocode}
%\iffalse
%</samplemain>
%\fi
%
% %%%%%%%%%%%%%%%%%%%%%%%%%%%%%%%%%%%%%%
% \paragraph{Chapter Include Files.}
%
% The include files are called |cdocsch1.tex| and |cdocsch2.tex|.
%
%\iffalse
%<*samplechap1|samplechap2>
%\fi

% Optional override for |\version| flag:
%    \begin{macrocode}
%%\providecommand{\version}{final}
%    \end{macrocode}

% Include the main document:
%    \begin{macrocode}
\input{childdoc.def}
\childdocof{cdocsamp}
%    \end{macrocode}

%\iffalse
%</samplechap1|samplechap2>
%\fi
%
%\iffalse
%<*samplechap1>
%\fi
% Some text for chapter 1:
%    \begin{macrocode}
\section{one}
some text in chapter one
%    \end{macrocode}

%\iffalse
%</samplechap1>
%\fi
% Some text for chapter 2:
%\iffalse
%<*samplechap2>
%\fi
%    \begin{macrocode}
\section{two}
more text in chapter two
%    \end{macrocode}

%\iffalse
%</samplechap2>
%\fi
%
% %%%%%%%%%%%%%%%%%%%%%%%%%%%%%%%%%%%%%%
% \paragraph{Part Include Files.}
%
% The include files are called |cdocspt3.tex| and |cdocspt4.tex|.
%
%\iffalse
%<*samplepart3|samplepart4>
%\fi

% Optional override for |\version| flag:
%    \begin{macrocode}
%%\providecommand{\version}{final}
%    \end{macrocode}

% Include the main document:
%    \begin{macrocode}
\input{childdoc.def}
\childdocby{cdocsamp}
%    \end{macrocode}

%\iffalse
%</samplepart3|samplepart4>
%\fi
%
%\iffalse
%<*samplepart3>
%\fi
% Some text for part 3:
%    \begin{macrocode}
some text in part three
%    \end{macrocode}

%\iffalse
%</samplepart3>
%\fi
% Some text for part 4:
%\iffalse
%<*samplepart4>
%\fi
%    \begin{macrocode}
more text in part four
%    \end{macrocode}

%\iffalse
%</samplepart4>
%\fi
%
% %%%%%%%%%%%%%%%%%%%%%%%%%%%%%%%%%%%%%%
% \paragraph{Forwarding for a Complete Draft.}
%
% The following forwarding file |cdocsdrf.tex|
% compiles the main document in draft mode:
%\iffalse
%<*sampledraft>
%\fi
%    \begin{macrocode}
\def\version{draft}
\input{childdoc.def}
\childdocforward{cdocsamp}
%    \end{macrocode}

%\iffalse
%</sampledraft>
%\fi
%
% %%%%%%%%%%%%%%%%%%%%%%%%%%%%%%%%%%%%%%
% \paragraph{Forwarding for Final Version of the Chapters.}
%
% The following forwarding files |cdocsfn1.tex| and |cdocsfn2.tex|
% (with identical content)
% compile the final versions of the child documents
% |cdocsch1.tex| and |cdocsch2.tex|, respectively:
%\iffalse
%<*samplefinal>
%\fi
%    \begin{macrocode}
\def\version{final}
\input{childdoc.def}
\childdocforwardprefix[cdocsamp]{cdocsfn}{cdocsch}
%    \end{macrocode}

%\iffalse
%</samplefinal>
%\fi
%
% %%%%%%%%%%%%%%%%%%%%%%%%%%%%%%%%%%%%%%
% \paragraph{Command Line Processing.}
%
% The following three command lines generate the output files
% |cdocscld|, |cdocscl1| and |cdocscl2|
% which should be identical to
% |cdocsdrf|, |cdocsch1| and |cdocsfn2|, respectively:
% \begin{center}
% \begin{tabular}{l}
% |latex -jobname cdocscld \|\\
% |  "\def\version{draft}\input{childdoc.def}\childdocforward{cdocsamp}"|\\
% |latex -jobname cdocscl1 \|\\
% |  "\input{childdoc.def}\childdocforward[cdocsamp]{cdocsch1}"|\\
% |latex -jobname cdocscl2 \|\\
% |  "\def\version{final}\input{childdoc.def}\childdocforward{cdocsch2}"|
% \end{tabular}
% \end{center}
% Note that the trailing backslash on each first line
% merely continues the input to the second line
% (for convenient cut ant paste).
% Furthermore, the command |latex| can be replaced by any
% of its alternative versions such as |pdflatex|.
%
% %%%%%%%%%%%%%%%%%%%%%%%%%%%%%%%%%%%%%%%%%%%%%%%%%%%%%%%%%%%%%%%%%%%%%%%%%%%%%%
% %%%%%%%%%%%%%%%%%%%%%%%%%%%%%%%%%%%%%%%%%%%%%%%%%%%%%%%%%%%%%%%%%%%%%%%%%%%%%%
% \section{Implementation}
%\iffalse
%<*package>
%\fi
%
% This section describes the definitions file |childdoc.def|.

% The definitions cannot be loaded using |\usepackage| or |\RequirePackage|
% which has a mechanism to prevent loading a style file more than once.
% When loading the definitions by means of |\input|
% multiple instances have to be prevented manually:
%\iffalse
%This code needs to be before the `\ProvidesFile' directive
%which is defined at the beginning of this file.
%Therefore it is also placed there and commented out here.
%</package>
%<*discard>
%\fi
%    \begin{macrocode}
\ifdefined\childdocmain\endinput\fi
%    \end{macrocode}
%\iffalse
%</discard>
%<*package>
%\fi
%
% \macro{\ifchilddoc}
% \macro{\ifchilddocmanual}
% The conditional |\ifchilddoc| tells whether a
% child (true) or main (false) document is being compiled.
% The conditional |\ifchilddocmanual| tells whether
% the |\includeonly| mechanism is used (false) or
% the selection of child files must be performed manually (true).
% The definitions initialise to false:
%    \begin{macrocode}
\newif\ifchilddoc
\newif\ifchilddocmanual
%    \end{macrocode}

% \macro{\childdocname}
% \macro{\childdocjob}
% The macro |\childdocname| stores the name of the main document
% to be compiled. The macro |\childdocjob| stores the name of
% the document on which the \LaTeX{} compiler was originally invoked.
% The content of |\jobname| cannot be compared
% to filenames specified in the source due to different catcodes.
% The following code rescans |\jobname|, stores the result
% in |\childdocname| and saves a copy in |\childdocjob|:
%    \begin{macrocode}
\edef\childdocname{\scantokens\expandafter{\jobname\noexpand}}
\let\childdocjob\childdocname
%    \end{macrocode}

% \macro{\childdocdisable}
% The macro |\childdocdisable| prevents the main file
% from being processed more than once.
% At this stage, the main document command |\childdocmain|
% is assumed to be called once again where it should do nothing.
% Any subsequent call to it should prevent
% a secondary processing of the main document
% It overwrites the forwarding commands
% |\childdocof| and |\childdocforward|
% with empty macros to prevent further inclusions of the main document:
%    \begin{macrocode}
\newcommand{\childdocdisable}
{
  \renewcommand{\childdocmain}[1]{\renewcommand{\childdocmain}[1]{\endinput}}
  \renewcommand{\childdocof}[1]{}
  \renewcommand{\childdocby}[2][]{}
  \renewcommand{\childdocforward}[2][]{}
  \renewcommand{\childdocdisable}{}
}
%    \end{macrocode}

% \macro{\childdocmain}
% The macro |\childdocmain| is to be called at the top of the main file
% with nothing or the main filename (without extension) as argument.
% First, it breaks loops.
% If the argument is not empty and does not match |\childdocname|
% (which is set by the first inclusion of |childdoc.def|),
% |\ifchilddoc| is set to true, |\includeonly| is applied to the child file
% and |\jobname| is set to the main file
% (for proper handling of |.aux| files):
%    \begin{macrocode}
\newcommand{\childdocmain}[1]
{
  \childdocdisable\childdocmain{}
  \if?#1?\else
    \begingroup
      \def\childdoctmp{#1}
      \ifx\childdoctmp\childdocname
        \def\childdoctmp{}
      \else
        \def\childdoctmp
        {
          \childdoctrue
          \includeonly{\childdocname}
          \def\childdocjob{#1}
          \def\jobname{#1}
        }
      \fi
      \expandafter
    \endgroup
    \childdoctmp
  \fi
}
%    \end{macrocode}

% \macro{\childdocof}
% The command |\childdocof| redirects
% compilation to the main file |#1|.
%    \begin{macrocode}
\newcommand{\childdocof}[1]
{
  \childdocdisable
  \childdoctrue
  \includeonly{\childdocname}
  \def\jobname{#1}
  \def\childdocjob{#1}
  \input{#1}
}
%    \end{macrocode}

% \macro{\childdocby}
% The command |\childdocby| ....
%    \begin{macrocode}
\newcommand{\childdocby}[2][]
{
  \childdocdisable
  \childdoctrue
  \childdocmanualtrue
  \if?#1?\else
    \def\jobname{#2}
  \fi
  \def\childdocjob{#2}
  \input{#2}
  \endinput
}
%    \end{macrocode}

% \macro{\childdocforward}
% The command |\childdocforward| redirects
% compilation to the main file or
% (if the optional argument is given) a child file.
% Parameters are set as if the main file
% or a child file starting with |\childdocof| was compiled.
% Then compilation is handed over to the main file:
%    \begin{macrocode}
\newcommand{\childdocforward}[2][]
{
  \begingroup
    \if?#1?
      \def\childdoctmp
      {
        \def\childdocname{#2}
        \def\childdocjob{#2}
        \def\jobname{#2}
        \input{#2}
        \endinput
      }
    \else
      \def\childdoctmp
      {
        \childdocdisable
        \def\childdocname{#2}
        \childdoctrue
        \includeonly{#2}
        \def\childdocjob{#1}
        \def\jobname{#1}
        \input{#1}
        \endinput
      }
    \fi
    \expandafter
  \endgroup
  \childdoctmp
}
%    \end{macrocode}

% \macro{\childdocforwardprefix}
% The command |\childdocforwardprefix| redirects
% compilation to the main or a child file by means of a pattern.
% The prefix |#1| in the current filename is replaced by |#2|
% and the suffix of the current filename is kept
% (it is assumed that the filename does not contain the substring `|~~~|'
% which is used as a delimiter).
% Compilation is handed over to the new file by |\childdocforward|:
%    \begin{macrocode}
\newcommand{\childdocforwardprefix}[3][]
{
  \begingroup
    \def\childdocextract #2##1~~~{\def\childdoctmp{\childdocforward[#1]{#3##1}}}
    \expandafter\childdocextract\childdocname~~~
    \expandafter
  \endgroup
  \childdoctmp
}
%    \end{macrocode}

% \macro{\childdoc}
% The deprecated macro |\childdoc| is a legacy version of |\childdocmain|:
%    \begin{macrocode}
\newcommand{\childdoc}{\childdocmain}
%    \end{macrocode}

% \macro{\childdocredirect}
% The deprecated macro |\childdocredirect| is a legacy version
% of |\childdocforward| and |\childdocforwardprefix|:
%    \begin{macrocode}
\newcommand{\childdocredirect}[2][]
{
  \begingroup
    \if?#1?
      \def\childdoctmp{\childdocforward{#2}}
    \else
      \def\childdoctmp{\childdocforwardprefix{#1}{#2}}
    \fi
    \expandafter
  \endgroup
  \childdoctmp
}
%    \end{macrocode}

%\iffalse
%</package>
%\fi
%
\endinput
|\\
|\childdocforward{|\textit{main}|}|
\end{tabular}
\end{center}
%
Likewise, the following files |final|\textit{nn}|.tex|
compile the final version of the child document
|child|\textit{nn}|.tex|:
%
\begin{center}
\begin{tabular}{l}
|\def\version{final}|\\
|% \iffalse
%
% childdoc.dtx Copyright (C) 2017-2018 Niklas Beisert
%
% This work may be distributed and/or modified under the
% conditions of the LaTeX Project Public License, either version 1.3
% of this license or (at your option) any later version.
% The latest version of this license is in
%   http://www.latex-project.org/lppl.txt
% and version 1.3 or later is part of all distributions of LaTeX
% version 2005/12/01 or later.
%
% This work has the LPPL maintenance status `maintained'.
%
% The Current Maintainer of this work is Niklas Beisert.
%
% This work consists of the files childdoc.dtx and childdoc.ins
% and the derived files childdoc.def and cdocsamp.tex with
% cdocsch1.tex, cdocsch2.tex, cdocsdrf.tex, cdocsfn1.tex, cdocsfn2.tex.
%
%<package>\ifdefined\childdocmain\endinput\fi
%<package>\ProvidesFile{childdoc.def}[2018/12/30 v2.0 child document driver]
%<samplemain>\ProvidesFile{cdocsamp.tex}[2018/12/30 v2.0 sample for childdoc]
%<*driver>
%\ProvidesFile{childdoc.drv}[2018/12/30 v2.0 childdoc reference manual file]
\PassOptionsToClass{10pt,a4paper}{article}
\documentclass{ltxdoc}

\usepackage[margin=35mm]{geometry}
\usepackage{hyperref}
\usepackage{hyperxmp}
\usepackage[usenames]{color}

\hypersetup{colorlinks=true}
\hypersetup{pdfstartview=FitH}
\hypersetup{pdfpagemode=UseNone}
\hypersetup{pdfsource={}}
\hypersetup{pdflang={en-UK}}
\hypersetup{pdfcopyright={Copyright 2017-2018 Niklas Beisert.
  This work may be distributed and/or modified under the
  conditions of the LaTeX Project Public License, either version 1.3
  of this license or (at your option) any later version.}}
\hypersetup{pdflicenseurl={http://www.latex-project.org/lppl.txt}}
\hypersetup{pdfcontactaddress={ETH Zurich, ITP, HIT K,
  Wolfgang-Pauli-Strasse 27}}
\hypersetup{pdfcontactpostcode={8093}}
\hypersetup{pdfcontactcity={Zurich}}
\hypersetup{pdfcontactcountry={Switzerland}}
\hypersetup{pdfcontactemail={nbeisert@itp.phys.ethz.ch}}
\hypersetup{pdfcontacturl={http://people.phys.ethz.ch/\xmptilde nbeisert/}}

\newcommand{\secref}[1]{\hyperref[#1]{section \ref*{#1}}}

\parskip1ex
\parindent0pt
\let\olditemize\itemize
\def\itemize{\olditemize\parskip0pt}

\begin{document}

\title{The \textsf{childdoc} Package}
\hypersetup{pdftitle={The childdoc Package}}
\author{Niklas Beisert\\[2ex]
  Institut f\"ur Theoretische Physik\\
  Eidgen\"ossische Technische Hochschule Z\"urich\\
  Wolfgang-Pauli-Strasse 27, 8093 Z\"urich, Switzerland\\[1ex]
  \href{mailto:nbeisert@itp.phys.ethz.ch}
  {\texttt{nbeisert@itp.phys.ethz.ch}}}
\hypersetup{pdfauthor={Niklas Beisert}}
\hypersetup{pdfsubject={Manual for the LaTeX2e Package childdoc}}
\date{30 December 2018, \textsf{v2.0}}
\maketitle

\begin{abstract}\noindent
\textsf{childdoc} is a \LaTeXe{} package
that enables the direct compilation
of document sections included by |\include|
to individual files.
\end{abstract}

\begingroup
\parskip0ex
\tableofcontents
\endgroup

%%%%%%%%%%%%%%%%%%%%%%%%%%%%%%%%%%%%%%%%%%%%%%%%%%%%%%%%%%%%%%%%%%%%%%%%%%%%%%%%
%%%%%%%%%%%%%%%%%%%%%%%%%%%%%%%%%%%%%%%%%%%%%%%%%%%%%%%%%%%%%%%%%%%%%%%%%%%%%%%%
\section{Introduction}

\LaTeX{} provides a mechanism to structure a large document (such as a book)
into a main file and several child files (containing the chapters)
using the |\include| command.
This mechanism is beneficial for documents
which span hundreds of pages in order to
make the source file(s) more manageable.
Moreover, compilation can be restricted to
selected child files by means of the |\includeonly| command.
The latter feature can be used to reduce the compilation time while editing
(this was significantly more useful in the earlier days of \LaTeX{})
or to generate a smaller document which is easier to navigate.
Another application of |\includeonly| is to generate
documents consisting of selected parts of the complete document.

However, there are a few drawbacks of the plain |\include| mechanism:
\begin{itemize}
\item
The child files cannot be compiled on their own,
they can only be compiled via the main file.
A naive editing environment
(such as a text editor with an option
to have the current file processed by \LaTeX)
may require one to switch to the main file before compiling;
attempting to compile the child file produces errors.
\item
The main file must be modified (each time)
to adjust the |\includeonly| command
to the present needs. This easily leaves the main file in a messy state.
\item
The generated document will always carry the filename
of the main document. This is inconvenient if
several child files are to be compiled and
to be kept for distribution.
\end{itemize}

The present package provides a simple interface
to make child files individually compilable by \LaTeX{}.
Compiling a child file then has the same effect as compiling
the main file with an |\includeonly| command
to select the appropriate child.
Moreover the generated document will carry the name of the child
rather than the main file.
This resolves all three above issues.

This feature is meant to make the editing of books,
thesis documents and lecture notes somewhat more convenient.
However, the package can also be used efficiently for
composing a series of documents (such as exercise sheets)
which are typically distributed individually.
It then assists the author in generating the individual documents
(potentially in different versions)
as well as a document containing the collected series.
Another application is in developing style files
or other kinds of included material
where compilation of the style file could redirect
to a sample or test file.

%%%%%%%%%%%%%%%%%%%%%%%%%%%%%%%%%%%%%%%%%%%%%%%%%%%%%%%%%%%%%%%%%%%%%%%%%%%%%%%%
%%%%%%%%%%%%%%%%%%%%%%%%%%%%%%%%%%%%%%%%%%%%%%%%%%%%%%%%%%%%%%%%%%%%%%%%%%%%%%%%
\section{Usage}

First of all, the package \textsf{childdoc} is \emph{not} a standard
\LaTeXe{} |.sty| style file! Therefore it needs to be invoked in
a non-standard way.

%%%%%%%%%%%%%%%%%%%%%%%%%%%%%%%%%%%%%%%%%%%%%%%%%%%%%%%%%%%%%%%%%%%%%%%%%%%%%%%%
\subsection{Included Files}
\label{sec:include}

%%%%%%%%%%%%%%%%%%%%%%%%%%%%%%%%%%%%%%%%
\DescribeMacro{\childdocmain}
To use the package, add the commands
\begin{center}
\begin{tabular}{l}
|\input{childdoc.def}|\\
|\childdocmain{}|\\
\end{tabular}
\end{center}
at the very top of the main \LaTeX{} file,
in particular \emph{before} the |\documentclass| statement!
The argument of |\childdocmain| should be left empty
(but it must be present).

%%%%%%%%%%%%%%%%%%%%%%%%%%%%%%%%%%%%%%%%
\DescribeMacro{\childdocof}
Furthermore, add the commands
\begin{center}
\begin{tabular}{l}
|\input{childdoc.def}|\\
|\childdocof{|\textit{main}|}|\\
\end{tabular}
\end{center}
at the top of every child file \textit{child}
which is included by |\include{|\textit{child}|}|
from within the main file
(or at least for those files to be compiled individually).
The argument \textit{main} must be the filename of the main file.

There are a couple of
considerations in setting up the main and child documents:

%%%%%%%%%%%%%%%%%%%%%%%%%%%%%%%%%%%%%%%%
\paragraph{Restrictions.}

Please note the following restrictions:
\begin{itemize}
\item
|\childdocmain| must be called with one argument \textit{main}
to ensure compatibility with earlier version of the package.
It must either be empty (|\childdocmain{}|)
or precisely match the filename of the main file in which it is specified.
See \secref{sec:detection} for further information.
\item
The filename \textit{main} must be specified without the |.tex| extension.
\item
The filename \textit{main} is case sensitive
(even in case-insensitive file systems)
due to internal string comparison.
\item
The argument \textit{main} should be fully expanded, it cannot be a macro.
\item
Subdirectories and special characters should be avoided in filenames.
\item
The command |\childdocmain{|\textit{main}|}| must be followed by a whitespace.
It should not be followed immediately by another command
or by a comment mark `|%|'.
This is because the \TeX{} parser reads the token immediately following
the argument of |\childdocmain| and puts it
at the beginning of every child section;
however, a white\-space is ignored.
\end{itemize}

%%%%%%%%%%%%%%%%%%%%%%%%%%%%%%%%%%%%%%%%
\paragraph{Content of Main File.}

It is advisable to place all content in the child files included by |\include|.
Any output contained in the main file will appear in all child documents
unless suppressed manually;
it cannot be suppressed automatically by the |\includeonly| directive
and thus should normally be avoided.
A method to include some content in the main file
by means of conditional processing is described in \secref{sec:conditional}.

%%%%%%%%%%%%%%%%%%%%%%%%%%%%%%%%%%%%%%%%
\paragraph{Page Numbering.}

When only a part of the document is compiled,
the appropriate numbering of pages
(as well as other status parameters)
is determined from the |.aux| files.
The latter contain information from previous passes.
However this information needs to propagate through
all intermediate child documents.
Therefore the page numbering in child documents may well
be inconsistent until the complete document is compiled at least once.

A useful (if unconventional) way to always ensure a consistent
page numbering is to restart the numbering in each child document
and denote the pages by `\textit{child}|.|\textit{page}'
where \textit{child} represents the chapter/section number of the child file.
This can be achieved by the command
|\numberwithin{page}{|\textit{child}|}|
of the \textsf{amsmath} package
where \textit{child} can be |chapter| or |section|
depending on the chosen structuring.
Alternatively, one can modify the macro |\thepage| appropriately
and reset the counter |page| at the start of each child file.

%%%%%%%%%%%%%%%%%%%%%%%%%%%%%%%%%%%%%%%%%%%%%%%%%%%%%%%%%%%%%%%%%%%%%%%%%%%%%%%%
\subsection{Conditional Processing}
\label{sec:conditional}

The package provides a mechanism to compile different versions
of a document. To customise the versions further some conditional processing
can come in handy to distinguish which version is being compiled.
The package provides two macros to describe the compilation context:

%%%%%%%%%%%%%%%%%%%%%%%%%%%%%%%%%%%%%%%%
\DescribeMacro{\ifchilddoc}
The conditional |\ifchilddoc| distinguishes between the compilation of
child documents and the main document:
%
\begin{center}
|\ifchilddoc |\textit{child-code}| |[|\||else |\textit{main-code}]| \||fi|
\end{center}

%%%%%%%%%%%%%%%%%%%%%%%%%%%%%%%%%%%%%%%%
\DescribeMacro{\childdocname}
\DescribeMacro{\childdocjob}
The macro |\childdocname| contains the filename (without extension)
of the main or child file being processed.
Note that |\childdocjob| will always contain the name of the main file.

%%%%%%%%%%%%%%%%%%%%%%%%%%%%%%%%%%%%%%%%
\paragraph{Title Page.}

Conditional processing can be used to include a title or banner page
in the main document when proper precautions are taken.
Importantly, the code in the main file should ensure that the page counter
(as well as other status parameters which are stored in the |.aux| files)
takes the same value after the conditional processing.
Otherwise the page numbers may take divergent values
depending on which part is compiled.

For example, a title page could be declared by:
%
\begin{center}
\begin{tabular}{l}
|\ifchilddoc\||else|\\
|\addtocounter{page}{-1}|\\
\textit{code for title page}\\
|\newpage|\\
|\||fi|
\end{tabular}
\end{center}
%
A banner page for the child documents can be generated by:
%
\begin{center}
\begin{tabular}{l}
|\ifchilddoc|\\
|\addtocounter{page}{-1}|\\
\textit{code for banner page}\\
|\newpage|\\
|\||fi|
\end{tabular}
\end{center}
%
Here one could write a message such as:
\begin{center}
|This is the part \childdocname{} of \childdocjob{}.|
\end{center}

%%%%%%%%%%%%%%%%%%%%%%%%%%%%%%%%%%%%%%%%%%%%%%%%%%%%%%%%%%%%%%%%%%%%%%%%%%%%%%%%
\subsection{Flags}
\label{sec:flags}

The package makes it easy to generate different versions
of the main or child documents.
To this end compilation flags can be defined
and assigned different default values.
They will be particularly useful in conjunction
with the forwarding mechanism described in \secref{sec:forward}.

For example, it may be useful to have a flag |\version|
which can be set to |draft| or |final|.
The document source will contain some conditional code
depending on the value of |\version|.
Suppose further, the flag should default to |final| for the main file
and to |draft| for child files
which is a natural assignment for editing the document.
This is achieved by placing the following code
in the preamble of the main document
(below the |\childdocmain| directive):
%
\begin{center}
\begin{tabular}{l}
|\ifchilddoc|\\
|\providecommand{\version}{draft}|\\
|\||else|\\
|\providecommand{\version}{final}|\\
|\||fi|
\end{tabular}
\end{center}
%
The definition by |\providecommand| makes sure
that previous definitions are not overwritten.
Further statements |\providecommand{\version}{...}|
can thus be added before the above code to override it.

For the main file, one might add a line
(between |\childdocmain| and the above block)
%
\begin{center}
|%\ifchilddoc\||else\providecommand{\version}{draft}\||fi|
\end{center}
%
which can be uncommented to produce a draft version.
Likewise one can add a line to the very top of a child file
(above the |\childdocof{|\textit{main}|}| directive)
%
\begin{center}
|%\providecommand{\version}{final}|
\end{center}
%
which can be uncommented to produce the final version of this child document.

%%%%%%%%%%%%%%%%%%%%%%%%%%%%%%%%%%%%%%%%%%%%%%%%%%%%%%%%%%%%%%%%%%%%%%%%%%%%%%%%
\subsection{Forwarding}
\label{sec:forward}

Different versions of the main or child documents
using compilation flags as described in \secref{sec:flags}
can be (permanently) stored in different files
for convenient compilation, viewing and distribution.
To this end, the package defines a command
to pass on compilation to a different file:

%%%%%%%%%%%%%%%%%%%%%%%%%%%%%%%%%%%%%%%%
\DescribeMacro{\childdocforward}
The command |\childdocforward| redirects processing to
another source file:
%
\begin{center}
\begin{tabular}{l}
|\input{childdoc.def}|\\
|\childdocforward[|\textit{main}|]{|\textit{dest}|}|\\
\end{tabular}
\end{center}
%
The argument \textit{dest} is the destination file
(without extension).
It should be the main file or one of the child files.
Note that further \textsf{childdoc} directives
such as |\childdocof| and |\childdocforward|
in the indicated file will be processed in this form.
The optional argument \textit{main}
passes on directly to the main file \textit{main}
while pretending to compile the child \textit{dest}.
This form behaves as if \textit{dest}
issues |\childdocof{|\textit{main}|}| right away,
and no further \textsf{childdoc} directives will be processed.

%%%%%%%%%%%%%%%%%%%%%%%%%%%%%%%%%%%%%%%%
\DescribeMacro{\...prefix}
In the alternative form |\childdocforwardprefix|,
%
\begin{center}
\begin{tabular}{l}
|\input{childdoc.def}|\\
|\childdocforwardprefix[|\textit{main}|]{|\textit{prefix}|}{|\textit{dest}|}|
\end{tabular}
\end{center}
%
the destination file is determined by a pattern
depending on the current file:
To make this work, the current file must be called
`{\textit{prefix}\hspace{0.2em}\textit{suffix}}'
with \textit{prefix} matching precisely the argument.
Processing is then passed on to the file
`{\textit{dest}\hspace{0.2em}\textit{suffix}}'.
Surely, the same effect is achieved by
directly specifying the
argument `{\textit{dest}\hspace{0.2em}\textit{suffix}}'
in the first form.
However, that requires to set up a different file
for each child. With the alternative form of the command
all these files can have exactly the same content
which simplifies setting them up and maintaining them.

For example, the following file |draft.tex|
with a compilation flag |\version| as described in \secref{sec:flags}
compiles the main document as a draft:
%
\begin{center}
\begin{tabular}{l}
|\def\version{draft}|\\
|\input{childdoc.def}|\\
|\childdocforward{|\textit{main}|}|
\end{tabular}
\end{center}
%
Likewise, the following files |final|\textit{nn}|.tex|
compile the final version of the child document
|child|\textit{nn}|.tex|:
%
\begin{center}
\begin{tabular}{l}
|\def\version{final}|\\
|\input{childdoc.def}|\\
|\childdocforwardprefix{final}{child}|
\end{tabular}
\end{center}
%

Note that when several versions of a main file and/or of each child file
are to be generated, it may be convenient to set up a |Makefile| or
shell script to automatise the process.

%%%%%%%%%%%%%%%%%%%%%%%%%%%%%%%%%%%%%%%%%%%%%%%%%%%%%%%%%%%%%%%%%%%%%%%%%%%%%%%%
\subsection{Command Line Processing}
\label{sec:commandline}

The effect of redirection files can also be achieved by invoking
the \LaTeX{} compiler with a more elaborate command line.
Most conveniently this should be done as part
of a shell script or a |Makefile|.

When using \textsf{childdoc} in the main file, the following
command lines effectively perform a redirection
(note that depending on the shell being used,
backslashes may have to be doubled: `|\|' $\to$ `|\\|'):
%
\begin{center}
|... -jobname "|\textit{target}|" |\\|"|[\textit{flags}]%
|\input{childdoc.def}\childdocforward[|\textit{main}|]{|\textit{dest}|}"|
\end{center}
%
Here \textit{target} is the name of the output file,
\textit{main} is the name of the main file
and \textit{dest} is the name of the main or child file to be processed
(all filenames without extensions).
The optional argument \textit{main} can be omitted
if \textit{main} matches \textit{dest}.
Optionally, compilation \textit{flags} can be defined via |\def| commands.
This command line makes the \TeX{} engine believe
it is compiling the file \textit{target}
whose content is specified as the latter parameter.
The provided code then forwards the processing to
\textit{main} or \textit{dest} as described in \secref{sec:forward}.

%%%%%%%%%%%%%%%%%%%%%%%%%%%%%%%%%%%%%%%%%%%%%%%%%%%%%%%%%%%%%%%%%%%%%%%%%%%%%%%%
\subsection{Include by Input}
\label{sec:input}

Including child documents by |\include| has some restrictions by design.
Most notably, the content of a child document always occupies
its own set of pages; pages cannot be shared between child documents.
Usually, this behaviour makes perfect sense
because each child document contain an essential part of the document.
However, in some situations it may be desirable to compose
a document from a collection of parts
without having mandatory page breaks between then.
For this case, the package
provides a mechanism to include parts
by |\input| which can also be processed individually.
However, by construction this mechanism
requires manual handling of the content to be output.

%%%%%%%%%%%%%%%%%%%%%%%%%%%%%%%%%%%%%%%%
\DescribeMacro{\ifchilddocmanual}
The main file should be prepared as usual, see \secref{sec:include}.
However, the document body must make a distinction
between processing of an individual part and of the main document, e.g.:
%
\begin{center}
\begin{tabular}{l}
|\ifchilddocmanual|\\
|\input{\childdocname}|\\
|\||else|\\
\textit{document body with }|\input{|\textit{part}|}|\\
|\||fi|
\end{tabular}
\end{center}
%
The conditional |\ifchilddocmanual| is true whenever
a part to be included by |\input| is being compiled,
and the name of the part is stored in |\childdocname|.

%%%%%%%%%%%%%%%%%%%%%%%%%%%%%%%%%%%%%%%%
\DescribeMacro{\childdocby}
Each part to be included by |\input| should start with:
%
\begin{center}
\begin{tabular}{l}
|\input{childdoc.def}|\\
|\childdocby{|\textit{main}|}|\\
\end{tabular}
\end{center}
%
The directive |\childdocby| is similar to |\childdocof|
described in \secref{sec:include},
but the subsequent selection of content must be done manually.
To that end, both |\ifchilddoc| and |\ifchilddocmanual|
will be true upon processing of a part,
and the name of the part is stored in |\childdocname|.
Note that |\jobname| will be set to the filename of the current part
so that each part receives an individual |.aux| file
that does not interfere with the |.aux| file(s) of the main document.
This behaviour can be altered by the alternative form
|\childdocby[*]{|\textit{main}|}| (with a non-empty optional argument)
which uses the |.aux| file of the main document
by setting |\jobname| to \textit{main}.

%%%%%%%%%%%%%%%%%%%%%%%%%%%%%%%%%%%%%%%%%%%%%%%%%%%%%%%%%%%%%%%%%%%%%%%%%%%%%%%%
\subsection{Driver Development}
\label{sec:driver}

The \textsf{childdoc} mechanism can also be use for the development
of definition files such as \LaTeX{} styles or classes.
This case differs from the above setup with multiple parts
included by |\include| in that no |\includeonly| should be invoked.
This can be achieved by starting the include file
(before |\ProvidesPackage|) with:
%
\begin{center}
\begin{tabular}{l}
|\input{childdoc.def}|\\
|\childdocforward{|\textit{main}|}|\\
\end{tabular}
\end{center}
%
or alternatively with:
%
\begin{center}
\begin{tabular}{l}
|\input{childdoc.def}|\\
|\childdocby{|\textit{main}|}|\\
\end{tabular}
\end{center}
%
Both forms have slightly different effects as described above.
The main file is prepared as usual, see \secref{sec:include}.

%%%%%%%%%%%%%%%%%%%%%%%%%%%%%%%%%%%%%%%%%%%%%%%%%%%%%%%%%%%%%%%%%%%%%%%%%%%%%%%%
\subsection{Legacy Detection}
\label{sec:detection}

The directive |\childdocmain| in the main file can detect
whether the complete document or merely a child is to be compiled
even without using the directive |\childdocof|.
This method is deprecated because it is less robust
and there is no compelling reason to use it;
it is merely provided for backward compatibility
and it may be removed in future versions.

If the detection mechanism is to be used,
it is mandatory to correctly specify
the filename of the main file as the argument of |\childdocmain|:
%
\begin{center}
\begin{tabular}{l}
|\input{childdoc.def}|\\
|\childdocmain{|\textit{main}|}|\\
\end{tabular}
\end{center}
%
If |\jobname| does not match the argument \textit{main} of |\childdocmain|,
it is assumed that |\jobname| points to the child file to be compiled.
When using |\childdocmain| with the main file specified as argument,
it suffices to start a child file
with just |\input{|\textit{main}|}|
without loading of the package and using |\childdocof|.
If instead all processing is done
with the appropriate \textsf{childdoc} directives,
the argument of \textit{main} of |\childdocmain| can be empty.

An alternative version of the command line processing described
in \secref{sec:commandline} using the detection mechanism reads:
%
\begin{center}
|... -jobname "|\textit{target}|" "|[\textit{flags}]%
[|\def\jobname{|\textit{dest}|}|]|\input{|\textit{main}|}"|
\end{center}

%%%%%%%%%%%%%%%%%%%%%%%%%%%%%%%%%%%%%%%%%%%%%%%%%%%%%%%%%%%%%%%%%%%%%%%%%%%%%%%%
\subsection{Manual Code}
\label{sec:manual}

In case one cannot be certain whether the definitions file |childdoc.def|
is installed on the target \TeX{} distribution
and one prefers not to ship it,
it is conceivable to paste a few relevant commands into the sources.

To that end, drop all statements |\input{childdoc.def}|
and perform the replacements as outlined below.
Instead of |\childdocmain{|\textit{main}|}| add the following code
to the top of the main file:
%
\begin{center}
\begin{tabular}{l}
|\||ifdefined\childdocname\endinput\||fi\newif\ifchilddoc|\\
|\edef\childdocname{\scantokens\expandafter{\jobname\noexpand}}|\\
|\def\childdocmain{|\textit{main}|}\||ifx\childdocmain\childdocname\||else|\\
|\childdoctrue\includeonly{\childdocname}\let\jobname\childdocmain\||fi|\\
\end{tabular}
\end{center}
%
Instead of |\childdocof{|\textit{main}|}| just include the main file
at the top of each child file:
%
\begin{center}
|\input{|\textit{main}|}|
\end{center}
%
A simple redirection |\childdocforward{|\textit{dest}|}| is achieved by:
%
\begin{center}
|\def\jobname{|\textit{dest}|}\input{\jobname}|
\end{center}
%
The redirection with prefix
|\childdocforwardprefix[|\textit{prefix}|]{|\textit{dest}|}|
is accomplished by:
%
\begin{center}
\begin{tabular}{l}
|{\edef\jobname{\scantokens\expandafter{\jobname\noexpand}}|\\
|\def\redirectjob |\textit{prefix}|#1~~~{\gdef\jobname{|\textit{dest}|#1}}|\\
|\expandafter\redirectjob\jobname~~~}\input{\jobname}|
\end{tabular}
\end{center}

In an alternative approach,
child documents can be compiled by a specific command line
without additional code or specific definitions:
%
\begin{center}
|... -jobname "|\textit{target}|" "|[\textit{flags}]%
|\includeonly{|\textit{dest}|}\input{|\textit{main}|}"|
\end{center}
%

%%%%%%%%%%%%%%%%%%%%%%%%%%%%%%%%%%%%%%%%%%%%%%%%%%%%%%%%%%%%%%%%%%%%%%%%%%%%%%%%
%%%%%%%%%%%%%%%%%%%%%%%%%%%%%%%%%%%%%%%%%%%%%%%%%%%%%%%%%%%%%%%%%%%%%%%%%%%%%%%%
\section{Information}

%%%%%%%%%%%%%%%%%%%%%%%%%%%%%%%%%%%%%%%%%%%%%%%%%%%%%%%%%%%%%%%%%%%%%%%%%%%%%%%%
\subsection{Copyright}

Copyright \copyright{} 2017--2018 Niklas Beisert

This work may be distributed and/or modified under the
conditions of the \LaTeX{} Project Public License, either version 1.3
of this license or (at your option) any later version.
The latest version of this license is in
  \url{http://www.latex-project.org/lppl.txt}
and version 1.3 or later is part of all distributions of \LaTeX{}
version 2005/12/01 or later.

This work has the LPPL maintenance status `maintained'.

The Current Maintainer of this work is Niklas Beisert.

This work consists of the files |README.txt|, |childdoc.ins| and |childdoc.dtx|
as well as the derived files |childdoc.def|, |cdocsamp.tex|
with |cdocsch1.tex|, |cdocsch2.tex|, |cdocspt3.tex|, |cdocspt4.tex|,
|cdocsdrf.tex|, |cdocsfn1.tex|, |cdocsfn2.tex|
as well as |childdoc.pdf|.

%%%%%%%%%%%%%%%%%%%%%%%%%%%%%%%%%%%%%%%%%%%%%%%%%%%%%%%%%%%%%%%%%%%%%%%%%%%%%%%%
\subsection{Files and Installation}

The package consists of the files:
%
\begin{center}
\begin{tabular}{ll}
    |README.txt|   & readme file \\
    |childdoc.ins| & installation file \\
    |childdoc.dtx| & source file \\
    |childdoc.def| & definition file \\
    |cdocsamp.tex| & sample main file \\
    |cdocsch1.tex| & sample include file \\
    |cdocsch2.tex| & sample include file \\
    |cdocspt3.tex| & sample part file \\
    |cdocspt4.tex| & sample part file \\
    |cdocsdrf.tex| & sample redirection file \\
    |cdocsfn1.tex| & sample redirection file \\
    |cdocsfn2.tex| & sample redirection file \\
    |childdoc.pdf| & manual
\end{tabular}
\end{center}
%
The distribution consists of the files
|README.txt|, |childdoc.ins| and |childdoc.dtx|.
%
\begin{itemize}
\item
Run (pdf)\LaTeX{} on |childdoc.dtx|
to compile the manual |childdoc.pdf| (this file).
\item
Run \LaTeX{} on |childdoc.ins| to create the definitions file |childdoc.def|
and the sample |cdocsamp.tex| with include files
|cdocsch1.tex|, |cdocsch2.tex|, |cdocspt3.tex|, |cdocspt4.tex|,
|cdocsdrf.tex|, |cdocsfn1.tex|, |cdocsfn2.tex|.
Then copy the file |childdoc.def| to an appropriate directory of your \LaTeX{}
distribution, e.g.\ \textit{texmf-root}|/tex/latex/childdoc|.
\end{itemize}

%%%%%%%%%%%%%%%%%%%%%%%%%%%%%%%%%%%%%%%%%%%%%%%%%%%%%%%%%%%%%%%%%%%%%%%%%%%%%%%%
\subsection{Related CTAN Packages}

There are several other packages which offer a similar functionality:
%
\begin{itemize}
\item
The packages
\href{http://ctan.org/pkg/docmute}{\textsf{docmute}},
\href{http://ctan.org/pkg/includex}{\textsf{includex}} and
\href{http://ctan.org/pkg/standalone}{\textsf{standalone}}
provide commands to include only the document body of
a child file thus allowing both files to be compiled individually.
\item
The packages \href{http://ctan.org/pkg/subdocs}{\textsf{subdocs}}
and \href{http://ctan.org/pkg/subfiles}{\textsf{subfiles}}
provide structures in which the main and child documents can be
encapsulated and allowing them to be compiled individually.
The inclusion mechanism is different from the conventional |\include|.
\item
The package \href{http://ctan.org/pkg/combine}{\textsf{combine}}
is an elaborate solution to combine several documents into one.
\end{itemize}
%
See also the CTAN topic \href{http://ctan.org/topic/subdocs}{\textsf{subdocs}}
for further related packages.
The present package differs from the above solutions in that
a document structure constructed with the conventional |\include| mechanism
just needs two extra commands at the top of every file
such that all constituent files can be compiled individually.

%%%%%%%%%%%%%%%%%%%%%%%%%%%%%%%%%%%%%%%%%%%%%%%%%%%%%%%%%%%%%%%%%%%%%%%%%%%%%%%%
%\subsection{Feature Suggestions}
%
%The following is a list of features which may be useful for future
%versions of this package:
%%
%\begin{itemize}
%\item
%\ldots
%\end{itemize}

%%%%%%%%%%%%%%%%%%%%%%%%%%%%%%%%%%%%%%%%%%%%%%%%%%%%%%%%%%%%%%%%%%%%%%%%%%%%%%%%
\subsection{Revision History}

%%%%%%%%%%%%%%%%%%%%%%%%%%%%%%%%%%%%%%%%
\paragraph{v2.0:} 2018/12/30

\begin{itemize}
\item
immediate forward processing
\item
added |\childdocby| mechanism
\item
manual restructured
\end{itemize}

%%%%%%%%%%%%%%%%%%%%%%%%%%%%%%%%%%%%%%%%
\paragraph{v1.6:} 2018/01/17

\begin{itemize}
\item
application for development of include files
\item
corrections to manual
\end{itemize}

%%%%%%%%%%%%%%%%%%%%%%%%%%%%%%%%%%%%%%%%
\paragraph{v1.5:} 2017/05/21

\begin{itemize}
\item
more complete structuring introduced
\item
|\childdocof| introduced
\item
|\childdoc| renamed to |\childdocmain|
\item
|\childredirect| renamed to |\childdocforward| and |\childdocforwardprefix|
and functionality expanded
\end{itemize}

%%%%%%%%%%%%%%%%%%%%%%%%%%%%%%%%%%%%%%%%
\paragraph{v1.0:} 2017/04/27

\begin{itemize}
\item
manual and install package
\item
first version published on CTAN
\end{itemize}

%%%%%%%%%%%%%%%%%%%%%%%%%%%%%%%%%%%%%%%%
\paragraph{v0.6:} 2017/04/26

\begin{itemize}
\item
redirection mechanism added
\end{itemize}

%%%%%%%%%%%%%%%%%%%%%%%%%%%%%%%%%%%%%%%%
\paragraph{v0.5:} 2017/04/26

\begin{itemize}
\item
functionality in definition file
\end{itemize}


%%%%%%%%%%%%%%%%%%%%%%%%%%%%%%%%%%%%%%%%%%%%%%%%%%%%%%%%%%%%%%%%%%%%%%%%%%%%%%%%
%%%%%%%%%%%%%%%%%%%%%%%%%%%%%%%%%%%%%%%%%%%%%%%%%%%%%%%%%%%%%%%%%%%%%%%%%%%%%%%%
%%%%%%%%%%%%%%%%%%%%%%%%%%%%%%%%%%%%%%%%%%%%%%%%%%%%%%%%%%%%%%%%%%%%%%%%%%%%%%%%
\appendix

\settowidth\MacroIndent{\rmfamily\scriptsize 000\ }

 \DocInput{childdoc.dtx}

\end{document}
%</driver>
% \fi
%
% %%%%%%%%%%%%%%%%%%%%%%%%%%%%%%%%%%%%%%%%%%%%%%%%%%%%%%%%%%%%%%%%%%%%%%%%%%%%%%
% %%%%%%%%%%%%%%%%%%%%%%%%%%%%%%%%%%%%%%%%%%%%%%%%%%%%%%%%%%%%%%%%%%%%%%%%%%%%%%
% \section{Sample}
%\iffalse
%<*samplemain>
%\fi
%
% The following presents a sample document
% with two chapters, two parts, a title page,
% a compile flag as well as three forwarding files to set the flag.
% It consists of eight |.tex| files:
% \begin{center}
% \begin{tabular}{ll}
% |cdocsamp.tex|&main file\\
% |cdocsch1.tex|&include file for chapter 1\\
% |cdocsch2.tex|&include file for chapter 2\\
% |cdocspt3.tex|&include file for part 3\\
% |cdocspt4.tex|&include file for part 4\\
% |cdocsdrf.tex|&forwarding file for main file in draft mode\\
% |cdocsfi1.tex|&forwarding file for final version of chapter 1\\
% |cdocsfi2.tex|&forwarding file for final version of chapter 2\\
% \end{tabular}
% \end{center}
% Each of the eight files can be compiled directly by the \LaTeX{} compiler.
%
% %%%%%%%%%%%%%%%%%%%%%%%%%%%%%%%%%%%%%%
% \paragraph{Main File.}
%
% The main file is called |cdocsamp.tex|.
%
% Load the \textsf{childdoc} definitions and
% declare the filename for the main document:
%    \begin{macrocode}
\input{childdoc.def}
\childdocmain{}
%    \end{macrocode}

% Optional override for |\version| flag:
%    \begin{macrocode}
%%\ifchilddoc\else\providecommand{\version}{draft}\fi
%    \end{macrocode}

% Define the default values for the |\version| flag
% (|final| for the main file and |draft| for childs):
%    \begin{macrocode}
\ifchilddoc
\providecommand{\version}{draft}
\else
\providecommand{\version}{final}
\fi
%    \end{macrocode}

% Load the standard document class:
%    \begin{macrocode}
\documentclass[12pt]{article}
%    \end{macrocode}

% Start the document body:
%    \begin{macrocode}
\begin{document}
%    \end{macrocode}

% Declare a title page.
% Print title, part of document being processed and version flag:
%    \begin{macrocode}
\addtocounter{page}{-1}
\begin{center}
{\LARGE\bfseries{}childdoc example\par}
\vspace{1cm}
\ifchilddoc
\ifchilddocmanual part\else chapter\fi:
`\childdocname' of `\childdocjob'\par
\else
main document: `\childdocjob'\par
\fi
version: \version\par
\end{center}
\newpage
%    \end{macrocode}

% Manually include selected file,
% otherwise process as usual:
%    \begin{macrocode}
\ifchilddocmanual
\section*{part `\childdocname'}
\input{\childdocname}
\else
%    \end{macrocode}

% Include the two chapters:
%    \begin{macrocode}
\include{cdocsch1}
\include{cdocsch2}
%    \end{macrocode}

% Include the two parts unless only chapters should be displayed:
%    \begin{macrocode}
\ifchilddoc\else
\section{part three}
\input{cdocspt3}
\section{part four}
\input{cdocspt4}
\fi
%    \end{macrocode}

% Process as usual until here:
%    \begin{macrocode}
\fi
%    \end{macrocode}

% End of document body:
%    \begin{macrocode}
\end{document}
%    \end{macrocode}
%\iffalse
%</samplemain>
%\fi
%
% %%%%%%%%%%%%%%%%%%%%%%%%%%%%%%%%%%%%%%
% \paragraph{Chapter Include Files.}
%
% The include files are called |cdocsch1.tex| and |cdocsch2.tex|.
%
%\iffalse
%<*samplechap1|samplechap2>
%\fi

% Optional override for |\version| flag:
%    \begin{macrocode}
%%\providecommand{\version}{final}
%    \end{macrocode}

% Include the main document:
%    \begin{macrocode}
\input{childdoc.def}
\childdocof{cdocsamp}
%    \end{macrocode}

%\iffalse
%</samplechap1|samplechap2>
%\fi
%
%\iffalse
%<*samplechap1>
%\fi
% Some text for chapter 1:
%    \begin{macrocode}
\section{one}
some text in chapter one
%    \end{macrocode}

%\iffalse
%</samplechap1>
%\fi
% Some text for chapter 2:
%\iffalse
%<*samplechap2>
%\fi
%    \begin{macrocode}
\section{two}
more text in chapter two
%    \end{macrocode}

%\iffalse
%</samplechap2>
%\fi
%
% %%%%%%%%%%%%%%%%%%%%%%%%%%%%%%%%%%%%%%
% \paragraph{Part Include Files.}
%
% The include files are called |cdocspt3.tex| and |cdocspt4.tex|.
%
%\iffalse
%<*samplepart3|samplepart4>
%\fi

% Optional override for |\version| flag:
%    \begin{macrocode}
%%\providecommand{\version}{final}
%    \end{macrocode}

% Include the main document:
%    \begin{macrocode}
\input{childdoc.def}
\childdocby{cdocsamp}
%    \end{macrocode}

%\iffalse
%</samplepart3|samplepart4>
%\fi
%
%\iffalse
%<*samplepart3>
%\fi
% Some text for part 3:
%    \begin{macrocode}
some text in part three
%    \end{macrocode}

%\iffalse
%</samplepart3>
%\fi
% Some text for part 4:
%\iffalse
%<*samplepart4>
%\fi
%    \begin{macrocode}
more text in part four
%    \end{macrocode}

%\iffalse
%</samplepart4>
%\fi
%
% %%%%%%%%%%%%%%%%%%%%%%%%%%%%%%%%%%%%%%
% \paragraph{Forwarding for a Complete Draft.}
%
% The following forwarding file |cdocsdrf.tex|
% compiles the main document in draft mode:
%\iffalse
%<*sampledraft>
%\fi
%    \begin{macrocode}
\def\version{draft}
\input{childdoc.def}
\childdocforward{cdocsamp}
%    \end{macrocode}

%\iffalse
%</sampledraft>
%\fi
%
% %%%%%%%%%%%%%%%%%%%%%%%%%%%%%%%%%%%%%%
% \paragraph{Forwarding for Final Version of the Chapters.}
%
% The following forwarding files |cdocsfn1.tex| and |cdocsfn2.tex|
% (with identical content)
% compile the final versions of the child documents
% |cdocsch1.tex| and |cdocsch2.tex|, respectively:
%\iffalse
%<*samplefinal>
%\fi
%    \begin{macrocode}
\def\version{final}
\input{childdoc.def}
\childdocforwardprefix[cdocsamp]{cdocsfn}{cdocsch}
%    \end{macrocode}

%\iffalse
%</samplefinal>
%\fi
%
% %%%%%%%%%%%%%%%%%%%%%%%%%%%%%%%%%%%%%%
% \paragraph{Command Line Processing.}
%
% The following three command lines generate the output files
% |cdocscld|, |cdocscl1| and |cdocscl2|
% which should be identical to
% |cdocsdrf|, |cdocsch1| and |cdocsfn2|, respectively:
% \begin{center}
% \begin{tabular}{l}
% |latex -jobname cdocscld \|\\
% |  "\def\version{draft}\input{childdoc.def}\childdocforward{cdocsamp}"|\\
% |latex -jobname cdocscl1 \|\\
% |  "\input{childdoc.def}\childdocforward[cdocsamp]{cdocsch1}"|\\
% |latex -jobname cdocscl2 \|\\
% |  "\def\version{final}\input{childdoc.def}\childdocforward{cdocsch2}"|
% \end{tabular}
% \end{center}
% Note that the trailing backslash on each first line
% merely continues the input to the second line
% (for convenient cut ant paste).
% Furthermore, the command |latex| can be replaced by any
% of its alternative versions such as |pdflatex|.
%
% %%%%%%%%%%%%%%%%%%%%%%%%%%%%%%%%%%%%%%%%%%%%%%%%%%%%%%%%%%%%%%%%%%%%%%%%%%%%%%
% %%%%%%%%%%%%%%%%%%%%%%%%%%%%%%%%%%%%%%%%%%%%%%%%%%%%%%%%%%%%%%%%%%%%%%%%%%%%%%
% \section{Implementation}
%\iffalse
%<*package>
%\fi
%
% This section describes the definitions file |childdoc.def|.

% The definitions cannot be loaded using |\usepackage| or |\RequirePackage|
% which has a mechanism to prevent loading a style file more than once.
% When loading the definitions by means of |\input|
% multiple instances have to be prevented manually:
%\iffalse
%This code needs to be before the `\ProvidesFile' directive
%which is defined at the beginning of this file.
%Therefore it is also placed there and commented out here.
%</package>
%<*discard>
%\fi
%    \begin{macrocode}
\ifdefined\childdocmain\endinput\fi
%    \end{macrocode}
%\iffalse
%</discard>
%<*package>
%\fi
%
% \macro{\ifchilddoc}
% \macro{\ifchilddocmanual}
% The conditional |\ifchilddoc| tells whether a
% child (true) or main (false) document is being compiled.
% The conditional |\ifchilddocmanual| tells whether
% the |\includeonly| mechanism is used (false) or
% the selection of child files must be performed manually (true).
% The definitions initialise to false:
%    \begin{macrocode}
\newif\ifchilddoc
\newif\ifchilddocmanual
%    \end{macrocode}

% \macro{\childdocname}
% \macro{\childdocjob}
% The macro |\childdocname| stores the name of the main document
% to be compiled. The macro |\childdocjob| stores the name of
% the document on which the \LaTeX{} compiler was originally invoked.
% The content of |\jobname| cannot be compared
% to filenames specified in the source due to different catcodes.
% The following code rescans |\jobname|, stores the result
% in |\childdocname| and saves a copy in |\childdocjob|:
%    \begin{macrocode}
\edef\childdocname{\scantokens\expandafter{\jobname\noexpand}}
\let\childdocjob\childdocname
%    \end{macrocode}

% \macro{\childdocdisable}
% The macro |\childdocdisable| prevents the main file
% from being processed more than once.
% At this stage, the main document command |\childdocmain|
% is assumed to be called once again where it should do nothing.
% Any subsequent call to it should prevent
% a secondary processing of the main document
% It overwrites the forwarding commands
% |\childdocof| and |\childdocforward|
% with empty macros to prevent further inclusions of the main document:
%    \begin{macrocode}
\newcommand{\childdocdisable}
{
  \renewcommand{\childdocmain}[1]{\renewcommand{\childdocmain}[1]{\endinput}}
  \renewcommand{\childdocof}[1]{}
  \renewcommand{\childdocby}[2][]{}
  \renewcommand{\childdocforward}[2][]{}
  \renewcommand{\childdocdisable}{}
}
%    \end{macrocode}

% \macro{\childdocmain}
% The macro |\childdocmain| is to be called at the top of the main file
% with nothing or the main filename (without extension) as argument.
% First, it breaks loops.
% If the argument is not empty and does not match |\childdocname|
% (which is set by the first inclusion of |childdoc.def|),
% |\ifchilddoc| is set to true, |\includeonly| is applied to the child file
% and |\jobname| is set to the main file
% (for proper handling of |.aux| files):
%    \begin{macrocode}
\newcommand{\childdocmain}[1]
{
  \childdocdisable\childdocmain{}
  \if?#1?\else
    \begingroup
      \def\childdoctmp{#1}
      \ifx\childdoctmp\childdocname
        \def\childdoctmp{}
      \else
        \def\childdoctmp
        {
          \childdoctrue
          \includeonly{\childdocname}
          \def\childdocjob{#1}
          \def\jobname{#1}
        }
      \fi
      \expandafter
    \endgroup
    \childdoctmp
  \fi
}
%    \end{macrocode}

% \macro{\childdocof}
% The command |\childdocof| redirects
% compilation to the main file |#1|.
%    \begin{macrocode}
\newcommand{\childdocof}[1]
{
  \childdocdisable
  \childdoctrue
  \includeonly{\childdocname}
  \def\jobname{#1}
  \def\childdocjob{#1}
  \input{#1}
}
%    \end{macrocode}

% \macro{\childdocby}
% The command |\childdocby| ....
%    \begin{macrocode}
\newcommand{\childdocby}[2][]
{
  \childdocdisable
  \childdoctrue
  \childdocmanualtrue
  \if?#1?\else
    \def\jobname{#2}
  \fi
  \def\childdocjob{#2}
  \input{#2}
  \endinput
}
%    \end{macrocode}

% \macro{\childdocforward}
% The command |\childdocforward| redirects
% compilation to the main file or
% (if the optional argument is given) a child file.
% Parameters are set as if the main file
% or a child file starting with |\childdocof| was compiled.
% Then compilation is handed over to the main file:
%    \begin{macrocode}
\newcommand{\childdocforward}[2][]
{
  \begingroup
    \if?#1?
      \def\childdoctmp
      {
        \def\childdocname{#2}
        \def\childdocjob{#2}
        \def\jobname{#2}
        \input{#2}
        \endinput
      }
    \else
      \def\childdoctmp
      {
        \childdocdisable
        \def\childdocname{#2}
        \childdoctrue
        \includeonly{#2}
        \def\childdocjob{#1}
        \def\jobname{#1}
        \input{#1}
        \endinput
      }
    \fi
    \expandafter
  \endgroup
  \childdoctmp
}
%    \end{macrocode}

% \macro{\childdocforwardprefix}
% The command |\childdocforwardprefix| redirects
% compilation to the main or a child file by means of a pattern.
% The prefix |#1| in the current filename is replaced by |#2|
% and the suffix of the current filename is kept
% (it is assumed that the filename does not contain the substring `|~~~|'
% which is used as a delimiter).
% Compilation is handed over to the new file by |\childdocforward|:
%    \begin{macrocode}
\newcommand{\childdocforwardprefix}[3][]
{
  \begingroup
    \def\childdocextract #2##1~~~{\def\childdoctmp{\childdocforward[#1]{#3##1}}}
    \expandafter\childdocextract\childdocname~~~
    \expandafter
  \endgroup
  \childdoctmp
}
%    \end{macrocode}

% \macro{\childdoc}
% The deprecated macro |\childdoc| is a legacy version of |\childdocmain|:
%    \begin{macrocode}
\newcommand{\childdoc}{\childdocmain}
%    \end{macrocode}

% \macro{\childdocredirect}
% The deprecated macro |\childdocredirect| is a legacy version
% of |\childdocforward| and |\childdocforwardprefix|:
%    \begin{macrocode}
\newcommand{\childdocredirect}[2][]
{
  \begingroup
    \if?#1?
      \def\childdoctmp{\childdocforward{#2}}
    \else
      \def\childdoctmp{\childdocforwardprefix{#1}{#2}}
    \fi
    \expandafter
  \endgroup
  \childdoctmp
}
%    \end{macrocode}

%\iffalse
%</package>
%\fi
%
\endinput
|\\
|\childdocforwardprefix{final}{child}|
\end{tabular}
\end{center}
%

Note that when several versions of a main file and/or of each child file
are to be generated, it may be convenient to set up a |Makefile| or
shell script to automatise the process.

%%%%%%%%%%%%%%%%%%%%%%%%%%%%%%%%%%%%%%%%%%%%%%%%%%%%%%%%%%%%%%%%%%%%%%%%%%%%%%%%
\subsection{Command Line Processing}
\label{sec:commandline}

The effect of redirection files can also be achieved by invoking
the \LaTeX{} compiler with a more elaborate command line.
Most conveniently this should be done as part
of a shell script or a |Makefile|.

When using \textsf{childdoc} in the main file, the following
command lines effectively perform a redirection
(note that depending on the shell being used,
backslashes may have to be doubled: `|\|' $\to$ `|\\|'):
%
\begin{center}
|... -jobname "|\textit{target}|" |\\|"|[\textit{flags}]%
|% \iffalse
%
% childdoc.dtx Copyright (C) 2017-2018 Niklas Beisert
%
% This work may be distributed and/or modified under the
% conditions of the LaTeX Project Public License, either version 1.3
% of this license or (at your option) any later version.
% The latest version of this license is in
%   http://www.latex-project.org/lppl.txt
% and version 1.3 or later is part of all distributions of LaTeX
% version 2005/12/01 or later.
%
% This work has the LPPL maintenance status `maintained'.
%
% The Current Maintainer of this work is Niklas Beisert.
%
% This work consists of the files childdoc.dtx and childdoc.ins
% and the derived files childdoc.def and cdocsamp.tex with
% cdocsch1.tex, cdocsch2.tex, cdocsdrf.tex, cdocsfn1.tex, cdocsfn2.tex.
%
%<package>\ifdefined\childdocmain\endinput\fi
%<package>\ProvidesFile{childdoc.def}[2018/12/30 v2.0 child document driver]
%<samplemain>\ProvidesFile{cdocsamp.tex}[2018/12/30 v2.0 sample for childdoc]
%<*driver>
%\ProvidesFile{childdoc.drv}[2018/12/30 v2.0 childdoc reference manual file]
\PassOptionsToClass{10pt,a4paper}{article}
\documentclass{ltxdoc}

\usepackage[margin=35mm]{geometry}
\usepackage{hyperref}
\usepackage{hyperxmp}
\usepackage[usenames]{color}

\hypersetup{colorlinks=true}
\hypersetup{pdfstartview=FitH}
\hypersetup{pdfpagemode=UseNone}
\hypersetup{pdfsource={}}
\hypersetup{pdflang={en-UK}}
\hypersetup{pdfcopyright={Copyright 2017-2018 Niklas Beisert.
  This work may be distributed and/or modified under the
  conditions of the LaTeX Project Public License, either version 1.3
  of this license or (at your option) any later version.}}
\hypersetup{pdflicenseurl={http://www.latex-project.org/lppl.txt}}
\hypersetup{pdfcontactaddress={ETH Zurich, ITP, HIT K,
  Wolfgang-Pauli-Strasse 27}}
\hypersetup{pdfcontactpostcode={8093}}
\hypersetup{pdfcontactcity={Zurich}}
\hypersetup{pdfcontactcountry={Switzerland}}
\hypersetup{pdfcontactemail={nbeisert@itp.phys.ethz.ch}}
\hypersetup{pdfcontacturl={http://people.phys.ethz.ch/\xmptilde nbeisert/}}

\newcommand{\secref}[1]{\hyperref[#1]{section \ref*{#1}}}

\parskip1ex
\parindent0pt
\let\olditemize\itemize
\def\itemize{\olditemize\parskip0pt}

\begin{document}

\title{The \textsf{childdoc} Package}
\hypersetup{pdftitle={The childdoc Package}}
\author{Niklas Beisert\\[2ex]
  Institut f\"ur Theoretische Physik\\
  Eidgen\"ossische Technische Hochschule Z\"urich\\
  Wolfgang-Pauli-Strasse 27, 8093 Z\"urich, Switzerland\\[1ex]
  \href{mailto:nbeisert@itp.phys.ethz.ch}
  {\texttt{nbeisert@itp.phys.ethz.ch}}}
\hypersetup{pdfauthor={Niklas Beisert}}
\hypersetup{pdfsubject={Manual for the LaTeX2e Package childdoc}}
\date{30 December 2018, \textsf{v2.0}}
\maketitle

\begin{abstract}\noindent
\textsf{childdoc} is a \LaTeXe{} package
that enables the direct compilation
of document sections included by |\include|
to individual files.
\end{abstract}

\begingroup
\parskip0ex
\tableofcontents
\endgroup

%%%%%%%%%%%%%%%%%%%%%%%%%%%%%%%%%%%%%%%%%%%%%%%%%%%%%%%%%%%%%%%%%%%%%%%%%%%%%%%%
%%%%%%%%%%%%%%%%%%%%%%%%%%%%%%%%%%%%%%%%%%%%%%%%%%%%%%%%%%%%%%%%%%%%%%%%%%%%%%%%
\section{Introduction}

\LaTeX{} provides a mechanism to structure a large document (such as a book)
into a main file and several child files (containing the chapters)
using the |\include| command.
This mechanism is beneficial for documents
which span hundreds of pages in order to
make the source file(s) more manageable.
Moreover, compilation can be restricted to
selected child files by means of the |\includeonly| command.
The latter feature can be used to reduce the compilation time while editing
(this was significantly more useful in the earlier days of \LaTeX{})
or to generate a smaller document which is easier to navigate.
Another application of |\includeonly| is to generate
documents consisting of selected parts of the complete document.

However, there are a few drawbacks of the plain |\include| mechanism:
\begin{itemize}
\item
The child files cannot be compiled on their own,
they can only be compiled via the main file.
A naive editing environment
(such as a text editor with an option
to have the current file processed by \LaTeX)
may require one to switch to the main file before compiling;
attempting to compile the child file produces errors.
\item
The main file must be modified (each time)
to adjust the |\includeonly| command
to the present needs. This easily leaves the main file in a messy state.
\item
The generated document will always carry the filename
of the main document. This is inconvenient if
several child files are to be compiled and
to be kept for distribution.
\end{itemize}

The present package provides a simple interface
to make child files individually compilable by \LaTeX{}.
Compiling a child file then has the same effect as compiling
the main file with an |\includeonly| command
to select the appropriate child.
Moreover the generated document will carry the name of the child
rather than the main file.
This resolves all three above issues.

This feature is meant to make the editing of books,
thesis documents and lecture notes somewhat more convenient.
However, the package can also be used efficiently for
composing a series of documents (such as exercise sheets)
which are typically distributed individually.
It then assists the author in generating the individual documents
(potentially in different versions)
as well as a document containing the collected series.
Another application is in developing style files
or other kinds of included material
where compilation of the style file could redirect
to a sample or test file.

%%%%%%%%%%%%%%%%%%%%%%%%%%%%%%%%%%%%%%%%%%%%%%%%%%%%%%%%%%%%%%%%%%%%%%%%%%%%%%%%
%%%%%%%%%%%%%%%%%%%%%%%%%%%%%%%%%%%%%%%%%%%%%%%%%%%%%%%%%%%%%%%%%%%%%%%%%%%%%%%%
\section{Usage}

First of all, the package \textsf{childdoc} is \emph{not} a standard
\LaTeXe{} |.sty| style file! Therefore it needs to be invoked in
a non-standard way.

%%%%%%%%%%%%%%%%%%%%%%%%%%%%%%%%%%%%%%%%%%%%%%%%%%%%%%%%%%%%%%%%%%%%%%%%%%%%%%%%
\subsection{Included Files}
\label{sec:include}

%%%%%%%%%%%%%%%%%%%%%%%%%%%%%%%%%%%%%%%%
\DescribeMacro{\childdocmain}
To use the package, add the commands
\begin{center}
\begin{tabular}{l}
|\input{childdoc.def}|\\
|\childdocmain{}|\\
\end{tabular}
\end{center}
at the very top of the main \LaTeX{} file,
in particular \emph{before} the |\documentclass| statement!
The argument of |\childdocmain| should be left empty
(but it must be present).

%%%%%%%%%%%%%%%%%%%%%%%%%%%%%%%%%%%%%%%%
\DescribeMacro{\childdocof}
Furthermore, add the commands
\begin{center}
\begin{tabular}{l}
|\input{childdoc.def}|\\
|\childdocof{|\textit{main}|}|\\
\end{tabular}
\end{center}
at the top of every child file \textit{child}
which is included by |\include{|\textit{child}|}|
from within the main file
(or at least for those files to be compiled individually).
The argument \textit{main} must be the filename of the main file.

There are a couple of
considerations in setting up the main and child documents:

%%%%%%%%%%%%%%%%%%%%%%%%%%%%%%%%%%%%%%%%
\paragraph{Restrictions.}

Please note the following restrictions:
\begin{itemize}
\item
|\childdocmain| must be called with one argument \textit{main}
to ensure compatibility with earlier version of the package.
It must either be empty (|\childdocmain{}|)
or precisely match the filename of the main file in which it is specified.
See \secref{sec:detection} for further information.
\item
The filename \textit{main} must be specified without the |.tex| extension.
\item
The filename \textit{main} is case sensitive
(even in case-insensitive file systems)
due to internal string comparison.
\item
The argument \textit{main} should be fully expanded, it cannot be a macro.
\item
Subdirectories and special characters should be avoided in filenames.
\item
The command |\childdocmain{|\textit{main}|}| must be followed by a whitespace.
It should not be followed immediately by another command
or by a comment mark `|%|'.
This is because the \TeX{} parser reads the token immediately following
the argument of |\childdocmain| and puts it
at the beginning of every child section;
however, a white\-space is ignored.
\end{itemize}

%%%%%%%%%%%%%%%%%%%%%%%%%%%%%%%%%%%%%%%%
\paragraph{Content of Main File.}

It is advisable to place all content in the child files included by |\include|.
Any output contained in the main file will appear in all child documents
unless suppressed manually;
it cannot be suppressed automatically by the |\includeonly| directive
and thus should normally be avoided.
A method to include some content in the main file
by means of conditional processing is described in \secref{sec:conditional}.

%%%%%%%%%%%%%%%%%%%%%%%%%%%%%%%%%%%%%%%%
\paragraph{Page Numbering.}

When only a part of the document is compiled,
the appropriate numbering of pages
(as well as other status parameters)
is determined from the |.aux| files.
The latter contain information from previous passes.
However this information needs to propagate through
all intermediate child documents.
Therefore the page numbering in child documents may well
be inconsistent until the complete document is compiled at least once.

A useful (if unconventional) way to always ensure a consistent
page numbering is to restart the numbering in each child document
and denote the pages by `\textit{child}|.|\textit{page}'
where \textit{child} represents the chapter/section number of the child file.
This can be achieved by the command
|\numberwithin{page}{|\textit{child}|}|
of the \textsf{amsmath} package
where \textit{child} can be |chapter| or |section|
depending on the chosen structuring.
Alternatively, one can modify the macro |\thepage| appropriately
and reset the counter |page| at the start of each child file.

%%%%%%%%%%%%%%%%%%%%%%%%%%%%%%%%%%%%%%%%%%%%%%%%%%%%%%%%%%%%%%%%%%%%%%%%%%%%%%%%
\subsection{Conditional Processing}
\label{sec:conditional}

The package provides a mechanism to compile different versions
of a document. To customise the versions further some conditional processing
can come in handy to distinguish which version is being compiled.
The package provides two macros to describe the compilation context:

%%%%%%%%%%%%%%%%%%%%%%%%%%%%%%%%%%%%%%%%
\DescribeMacro{\ifchilddoc}
The conditional |\ifchilddoc| distinguishes between the compilation of
child documents and the main document:
%
\begin{center}
|\ifchilddoc |\textit{child-code}| |[|\||else |\textit{main-code}]| \||fi|
\end{center}

%%%%%%%%%%%%%%%%%%%%%%%%%%%%%%%%%%%%%%%%
\DescribeMacro{\childdocname}
\DescribeMacro{\childdocjob}
The macro |\childdocname| contains the filename (without extension)
of the main or child file being processed.
Note that |\childdocjob| will always contain the name of the main file.

%%%%%%%%%%%%%%%%%%%%%%%%%%%%%%%%%%%%%%%%
\paragraph{Title Page.}

Conditional processing can be used to include a title or banner page
in the main document when proper precautions are taken.
Importantly, the code in the main file should ensure that the page counter
(as well as other status parameters which are stored in the |.aux| files)
takes the same value after the conditional processing.
Otherwise the page numbers may take divergent values
depending on which part is compiled.

For example, a title page could be declared by:
%
\begin{center}
\begin{tabular}{l}
|\ifchilddoc\||else|\\
|\addtocounter{page}{-1}|\\
\textit{code for title page}\\
|\newpage|\\
|\||fi|
\end{tabular}
\end{center}
%
A banner page for the child documents can be generated by:
%
\begin{center}
\begin{tabular}{l}
|\ifchilddoc|\\
|\addtocounter{page}{-1}|\\
\textit{code for banner page}\\
|\newpage|\\
|\||fi|
\end{tabular}
\end{center}
%
Here one could write a message such as:
\begin{center}
|This is the part \childdocname{} of \childdocjob{}.|
\end{center}

%%%%%%%%%%%%%%%%%%%%%%%%%%%%%%%%%%%%%%%%%%%%%%%%%%%%%%%%%%%%%%%%%%%%%%%%%%%%%%%%
\subsection{Flags}
\label{sec:flags}

The package makes it easy to generate different versions
of the main or child documents.
To this end compilation flags can be defined
and assigned different default values.
They will be particularly useful in conjunction
with the forwarding mechanism described in \secref{sec:forward}.

For example, it may be useful to have a flag |\version|
which can be set to |draft| or |final|.
The document source will contain some conditional code
depending on the value of |\version|.
Suppose further, the flag should default to |final| for the main file
and to |draft| for child files
which is a natural assignment for editing the document.
This is achieved by placing the following code
in the preamble of the main document
(below the |\childdocmain| directive):
%
\begin{center}
\begin{tabular}{l}
|\ifchilddoc|\\
|\providecommand{\version}{draft}|\\
|\||else|\\
|\providecommand{\version}{final}|\\
|\||fi|
\end{tabular}
\end{center}
%
The definition by |\providecommand| makes sure
that previous definitions are not overwritten.
Further statements |\providecommand{\version}{...}|
can thus be added before the above code to override it.

For the main file, one might add a line
(between |\childdocmain| and the above block)
%
\begin{center}
|%\ifchilddoc\||else\providecommand{\version}{draft}\||fi|
\end{center}
%
which can be uncommented to produce a draft version.
Likewise one can add a line to the very top of a child file
(above the |\childdocof{|\textit{main}|}| directive)
%
\begin{center}
|%\providecommand{\version}{final}|
\end{center}
%
which can be uncommented to produce the final version of this child document.

%%%%%%%%%%%%%%%%%%%%%%%%%%%%%%%%%%%%%%%%%%%%%%%%%%%%%%%%%%%%%%%%%%%%%%%%%%%%%%%%
\subsection{Forwarding}
\label{sec:forward}

Different versions of the main or child documents
using compilation flags as described in \secref{sec:flags}
can be (permanently) stored in different files
for convenient compilation, viewing and distribution.
To this end, the package defines a command
to pass on compilation to a different file:

%%%%%%%%%%%%%%%%%%%%%%%%%%%%%%%%%%%%%%%%
\DescribeMacro{\childdocforward}
The command |\childdocforward| redirects processing to
another source file:
%
\begin{center}
\begin{tabular}{l}
|\input{childdoc.def}|\\
|\childdocforward[|\textit{main}|]{|\textit{dest}|}|\\
\end{tabular}
\end{center}
%
The argument \textit{dest} is the destination file
(without extension).
It should be the main file or one of the child files.
Note that further \textsf{childdoc} directives
such as |\childdocof| and |\childdocforward|
in the indicated file will be processed in this form.
The optional argument \textit{main}
passes on directly to the main file \textit{main}
while pretending to compile the child \textit{dest}.
This form behaves as if \textit{dest}
issues |\childdocof{|\textit{main}|}| right away,
and no further \textsf{childdoc} directives will be processed.

%%%%%%%%%%%%%%%%%%%%%%%%%%%%%%%%%%%%%%%%
\DescribeMacro{\...prefix}
In the alternative form |\childdocforwardprefix|,
%
\begin{center}
\begin{tabular}{l}
|\input{childdoc.def}|\\
|\childdocforwardprefix[|\textit{main}|]{|\textit{prefix}|}{|\textit{dest}|}|
\end{tabular}
\end{center}
%
the destination file is determined by a pattern
depending on the current file:
To make this work, the current file must be called
`{\textit{prefix}\hspace{0.2em}\textit{suffix}}'
with \textit{prefix} matching precisely the argument.
Processing is then passed on to the file
`{\textit{dest}\hspace{0.2em}\textit{suffix}}'.
Surely, the same effect is achieved by
directly specifying the
argument `{\textit{dest}\hspace{0.2em}\textit{suffix}}'
in the first form.
However, that requires to set up a different file
for each child. With the alternative form of the command
all these files can have exactly the same content
which simplifies setting them up and maintaining them.

For example, the following file |draft.tex|
with a compilation flag |\version| as described in \secref{sec:flags}
compiles the main document as a draft:
%
\begin{center}
\begin{tabular}{l}
|\def\version{draft}|\\
|\input{childdoc.def}|\\
|\childdocforward{|\textit{main}|}|
\end{tabular}
\end{center}
%
Likewise, the following files |final|\textit{nn}|.tex|
compile the final version of the child document
|child|\textit{nn}|.tex|:
%
\begin{center}
\begin{tabular}{l}
|\def\version{final}|\\
|\input{childdoc.def}|\\
|\childdocforwardprefix{final}{child}|
\end{tabular}
\end{center}
%

Note that when several versions of a main file and/or of each child file
are to be generated, it may be convenient to set up a |Makefile| or
shell script to automatise the process.

%%%%%%%%%%%%%%%%%%%%%%%%%%%%%%%%%%%%%%%%%%%%%%%%%%%%%%%%%%%%%%%%%%%%%%%%%%%%%%%%
\subsection{Command Line Processing}
\label{sec:commandline}

The effect of redirection files can also be achieved by invoking
the \LaTeX{} compiler with a more elaborate command line.
Most conveniently this should be done as part
of a shell script or a |Makefile|.

When using \textsf{childdoc} in the main file, the following
command lines effectively perform a redirection
(note that depending on the shell being used,
backslashes may have to be doubled: `|\|' $\to$ `|\\|'):
%
\begin{center}
|... -jobname "|\textit{target}|" |\\|"|[\textit{flags}]%
|\input{childdoc.def}\childdocforward[|\textit{main}|]{|\textit{dest}|}"|
\end{center}
%
Here \textit{target} is the name of the output file,
\textit{main} is the name of the main file
and \textit{dest} is the name of the main or child file to be processed
(all filenames without extensions).
The optional argument \textit{main} can be omitted
if \textit{main} matches \textit{dest}.
Optionally, compilation \textit{flags} can be defined via |\def| commands.
This command line makes the \TeX{} engine believe
it is compiling the file \textit{target}
whose content is specified as the latter parameter.
The provided code then forwards the processing to
\textit{main} or \textit{dest} as described in \secref{sec:forward}.

%%%%%%%%%%%%%%%%%%%%%%%%%%%%%%%%%%%%%%%%%%%%%%%%%%%%%%%%%%%%%%%%%%%%%%%%%%%%%%%%
\subsection{Include by Input}
\label{sec:input}

Including child documents by |\include| has some restrictions by design.
Most notably, the content of a child document always occupies
its own set of pages; pages cannot be shared between child documents.
Usually, this behaviour makes perfect sense
because each child document contain an essential part of the document.
However, in some situations it may be desirable to compose
a document from a collection of parts
without having mandatory page breaks between then.
For this case, the package
provides a mechanism to include parts
by |\input| which can also be processed individually.
However, by construction this mechanism
requires manual handling of the content to be output.

%%%%%%%%%%%%%%%%%%%%%%%%%%%%%%%%%%%%%%%%
\DescribeMacro{\ifchilddocmanual}
The main file should be prepared as usual, see \secref{sec:include}.
However, the document body must make a distinction
between processing of an individual part and of the main document, e.g.:
%
\begin{center}
\begin{tabular}{l}
|\ifchilddocmanual|\\
|\input{\childdocname}|\\
|\||else|\\
\textit{document body with }|\input{|\textit{part}|}|\\
|\||fi|
\end{tabular}
\end{center}
%
The conditional |\ifchilddocmanual| is true whenever
a part to be included by |\input| is being compiled,
and the name of the part is stored in |\childdocname|.

%%%%%%%%%%%%%%%%%%%%%%%%%%%%%%%%%%%%%%%%
\DescribeMacro{\childdocby}
Each part to be included by |\input| should start with:
%
\begin{center}
\begin{tabular}{l}
|\input{childdoc.def}|\\
|\childdocby{|\textit{main}|}|\\
\end{tabular}
\end{center}
%
The directive |\childdocby| is similar to |\childdocof|
described in \secref{sec:include},
but the subsequent selection of content must be done manually.
To that end, both |\ifchilddoc| and |\ifchilddocmanual|
will be true upon processing of a part,
and the name of the part is stored in |\childdocname|.
Note that |\jobname| will be set to the filename of the current part
so that each part receives an individual |.aux| file
that does not interfere with the |.aux| file(s) of the main document.
This behaviour can be altered by the alternative form
|\childdocby[*]{|\textit{main}|}| (with a non-empty optional argument)
which uses the |.aux| file of the main document
by setting |\jobname| to \textit{main}.

%%%%%%%%%%%%%%%%%%%%%%%%%%%%%%%%%%%%%%%%%%%%%%%%%%%%%%%%%%%%%%%%%%%%%%%%%%%%%%%%
\subsection{Driver Development}
\label{sec:driver}

The \textsf{childdoc} mechanism can also be use for the development
of definition files such as \LaTeX{} styles or classes.
This case differs from the above setup with multiple parts
included by |\include| in that no |\includeonly| should be invoked.
This can be achieved by starting the include file
(before |\ProvidesPackage|) with:
%
\begin{center}
\begin{tabular}{l}
|\input{childdoc.def}|\\
|\childdocforward{|\textit{main}|}|\\
\end{tabular}
\end{center}
%
or alternatively with:
%
\begin{center}
\begin{tabular}{l}
|\input{childdoc.def}|\\
|\childdocby{|\textit{main}|}|\\
\end{tabular}
\end{center}
%
Both forms have slightly different effects as described above.
The main file is prepared as usual, see \secref{sec:include}.

%%%%%%%%%%%%%%%%%%%%%%%%%%%%%%%%%%%%%%%%%%%%%%%%%%%%%%%%%%%%%%%%%%%%%%%%%%%%%%%%
\subsection{Legacy Detection}
\label{sec:detection}

The directive |\childdocmain| in the main file can detect
whether the complete document or merely a child is to be compiled
even without using the directive |\childdocof|.
This method is deprecated because it is less robust
and there is no compelling reason to use it;
it is merely provided for backward compatibility
and it may be removed in future versions.

If the detection mechanism is to be used,
it is mandatory to correctly specify
the filename of the main file as the argument of |\childdocmain|:
%
\begin{center}
\begin{tabular}{l}
|\input{childdoc.def}|\\
|\childdocmain{|\textit{main}|}|\\
\end{tabular}
\end{center}
%
If |\jobname| does not match the argument \textit{main} of |\childdocmain|,
it is assumed that |\jobname| points to the child file to be compiled.
When using |\childdocmain| with the main file specified as argument,
it suffices to start a child file
with just |\input{|\textit{main}|}|
without loading of the package and using |\childdocof|.
If instead all processing is done
with the appropriate \textsf{childdoc} directives,
the argument of \textit{main} of |\childdocmain| can be empty.

An alternative version of the command line processing described
in \secref{sec:commandline} using the detection mechanism reads:
%
\begin{center}
|... -jobname "|\textit{target}|" "|[\textit{flags}]%
[|\def\jobname{|\textit{dest}|}|]|\input{|\textit{main}|}"|
\end{center}

%%%%%%%%%%%%%%%%%%%%%%%%%%%%%%%%%%%%%%%%%%%%%%%%%%%%%%%%%%%%%%%%%%%%%%%%%%%%%%%%
\subsection{Manual Code}
\label{sec:manual}

In case one cannot be certain whether the definitions file |childdoc.def|
is installed on the target \TeX{} distribution
and one prefers not to ship it,
it is conceivable to paste a few relevant commands into the sources.

To that end, drop all statements |\input{childdoc.def}|
and perform the replacements as outlined below.
Instead of |\childdocmain{|\textit{main}|}| add the following code
to the top of the main file:
%
\begin{center}
\begin{tabular}{l}
|\||ifdefined\childdocname\endinput\||fi\newif\ifchilddoc|\\
|\edef\childdocname{\scantokens\expandafter{\jobname\noexpand}}|\\
|\def\childdocmain{|\textit{main}|}\||ifx\childdocmain\childdocname\||else|\\
|\childdoctrue\includeonly{\childdocname}\let\jobname\childdocmain\||fi|\\
\end{tabular}
\end{center}
%
Instead of |\childdocof{|\textit{main}|}| just include the main file
at the top of each child file:
%
\begin{center}
|\input{|\textit{main}|}|
\end{center}
%
A simple redirection |\childdocforward{|\textit{dest}|}| is achieved by:
%
\begin{center}
|\def\jobname{|\textit{dest}|}\input{\jobname}|
\end{center}
%
The redirection with prefix
|\childdocforwardprefix[|\textit{prefix}|]{|\textit{dest}|}|
is accomplished by:
%
\begin{center}
\begin{tabular}{l}
|{\edef\jobname{\scantokens\expandafter{\jobname\noexpand}}|\\
|\def\redirectjob |\textit{prefix}|#1~~~{\gdef\jobname{|\textit{dest}|#1}}|\\
|\expandafter\redirectjob\jobname~~~}\input{\jobname}|
\end{tabular}
\end{center}

In an alternative approach,
child documents can be compiled by a specific command line
without additional code or specific definitions:
%
\begin{center}
|... -jobname "|\textit{target}|" "|[\textit{flags}]%
|\includeonly{|\textit{dest}|}\input{|\textit{main}|}"|
\end{center}
%

%%%%%%%%%%%%%%%%%%%%%%%%%%%%%%%%%%%%%%%%%%%%%%%%%%%%%%%%%%%%%%%%%%%%%%%%%%%%%%%%
%%%%%%%%%%%%%%%%%%%%%%%%%%%%%%%%%%%%%%%%%%%%%%%%%%%%%%%%%%%%%%%%%%%%%%%%%%%%%%%%
\section{Information}

%%%%%%%%%%%%%%%%%%%%%%%%%%%%%%%%%%%%%%%%%%%%%%%%%%%%%%%%%%%%%%%%%%%%%%%%%%%%%%%%
\subsection{Copyright}

Copyright \copyright{} 2017--2018 Niklas Beisert

This work may be distributed and/or modified under the
conditions of the \LaTeX{} Project Public License, either version 1.3
of this license or (at your option) any later version.
The latest version of this license is in
  \url{http://www.latex-project.org/lppl.txt}
and version 1.3 or later is part of all distributions of \LaTeX{}
version 2005/12/01 or later.

This work has the LPPL maintenance status `maintained'.

The Current Maintainer of this work is Niklas Beisert.

This work consists of the files |README.txt|, |childdoc.ins| and |childdoc.dtx|
as well as the derived files |childdoc.def|, |cdocsamp.tex|
with |cdocsch1.tex|, |cdocsch2.tex|, |cdocspt3.tex|, |cdocspt4.tex|,
|cdocsdrf.tex|, |cdocsfn1.tex|, |cdocsfn2.tex|
as well as |childdoc.pdf|.

%%%%%%%%%%%%%%%%%%%%%%%%%%%%%%%%%%%%%%%%%%%%%%%%%%%%%%%%%%%%%%%%%%%%%%%%%%%%%%%%
\subsection{Files and Installation}

The package consists of the files:
%
\begin{center}
\begin{tabular}{ll}
    |README.txt|   & readme file \\
    |childdoc.ins| & installation file \\
    |childdoc.dtx| & source file \\
    |childdoc.def| & definition file \\
    |cdocsamp.tex| & sample main file \\
    |cdocsch1.tex| & sample include file \\
    |cdocsch2.tex| & sample include file \\
    |cdocspt3.tex| & sample part file \\
    |cdocspt4.tex| & sample part file \\
    |cdocsdrf.tex| & sample redirection file \\
    |cdocsfn1.tex| & sample redirection file \\
    |cdocsfn2.tex| & sample redirection file \\
    |childdoc.pdf| & manual
\end{tabular}
\end{center}
%
The distribution consists of the files
|README.txt|, |childdoc.ins| and |childdoc.dtx|.
%
\begin{itemize}
\item
Run (pdf)\LaTeX{} on |childdoc.dtx|
to compile the manual |childdoc.pdf| (this file).
\item
Run \LaTeX{} on |childdoc.ins| to create the definitions file |childdoc.def|
and the sample |cdocsamp.tex| with include files
|cdocsch1.tex|, |cdocsch2.tex|, |cdocspt3.tex|, |cdocspt4.tex|,
|cdocsdrf.tex|, |cdocsfn1.tex|, |cdocsfn2.tex|.
Then copy the file |childdoc.def| to an appropriate directory of your \LaTeX{}
distribution, e.g.\ \textit{texmf-root}|/tex/latex/childdoc|.
\end{itemize}

%%%%%%%%%%%%%%%%%%%%%%%%%%%%%%%%%%%%%%%%%%%%%%%%%%%%%%%%%%%%%%%%%%%%%%%%%%%%%%%%
\subsection{Related CTAN Packages}

There are several other packages which offer a similar functionality:
%
\begin{itemize}
\item
The packages
\href{http://ctan.org/pkg/docmute}{\textsf{docmute}},
\href{http://ctan.org/pkg/includex}{\textsf{includex}} and
\href{http://ctan.org/pkg/standalone}{\textsf{standalone}}
provide commands to include only the document body of
a child file thus allowing both files to be compiled individually.
\item
The packages \href{http://ctan.org/pkg/subdocs}{\textsf{subdocs}}
and \href{http://ctan.org/pkg/subfiles}{\textsf{subfiles}}
provide structures in which the main and child documents can be
encapsulated and allowing them to be compiled individually.
The inclusion mechanism is different from the conventional |\include|.
\item
The package \href{http://ctan.org/pkg/combine}{\textsf{combine}}
is an elaborate solution to combine several documents into one.
\end{itemize}
%
See also the CTAN topic \href{http://ctan.org/topic/subdocs}{\textsf{subdocs}}
for further related packages.
The present package differs from the above solutions in that
a document structure constructed with the conventional |\include| mechanism
just needs two extra commands at the top of every file
such that all constituent files can be compiled individually.

%%%%%%%%%%%%%%%%%%%%%%%%%%%%%%%%%%%%%%%%%%%%%%%%%%%%%%%%%%%%%%%%%%%%%%%%%%%%%%%%
%\subsection{Feature Suggestions}
%
%The following is a list of features which may be useful for future
%versions of this package:
%%
%\begin{itemize}
%\item
%\ldots
%\end{itemize}

%%%%%%%%%%%%%%%%%%%%%%%%%%%%%%%%%%%%%%%%%%%%%%%%%%%%%%%%%%%%%%%%%%%%%%%%%%%%%%%%
\subsection{Revision History}

%%%%%%%%%%%%%%%%%%%%%%%%%%%%%%%%%%%%%%%%
\paragraph{v2.0:} 2018/12/30

\begin{itemize}
\item
immediate forward processing
\item
added |\childdocby| mechanism
\item
manual restructured
\end{itemize}

%%%%%%%%%%%%%%%%%%%%%%%%%%%%%%%%%%%%%%%%
\paragraph{v1.6:} 2018/01/17

\begin{itemize}
\item
application for development of include files
\item
corrections to manual
\end{itemize}

%%%%%%%%%%%%%%%%%%%%%%%%%%%%%%%%%%%%%%%%
\paragraph{v1.5:} 2017/05/21

\begin{itemize}
\item
more complete structuring introduced
\item
|\childdocof| introduced
\item
|\childdoc| renamed to |\childdocmain|
\item
|\childredirect| renamed to |\childdocforward| and |\childdocforwardprefix|
and functionality expanded
\end{itemize}

%%%%%%%%%%%%%%%%%%%%%%%%%%%%%%%%%%%%%%%%
\paragraph{v1.0:} 2017/04/27

\begin{itemize}
\item
manual and install package
\item
first version published on CTAN
\end{itemize}

%%%%%%%%%%%%%%%%%%%%%%%%%%%%%%%%%%%%%%%%
\paragraph{v0.6:} 2017/04/26

\begin{itemize}
\item
redirection mechanism added
\end{itemize}

%%%%%%%%%%%%%%%%%%%%%%%%%%%%%%%%%%%%%%%%
\paragraph{v0.5:} 2017/04/26

\begin{itemize}
\item
functionality in definition file
\end{itemize}


%%%%%%%%%%%%%%%%%%%%%%%%%%%%%%%%%%%%%%%%%%%%%%%%%%%%%%%%%%%%%%%%%%%%%%%%%%%%%%%%
%%%%%%%%%%%%%%%%%%%%%%%%%%%%%%%%%%%%%%%%%%%%%%%%%%%%%%%%%%%%%%%%%%%%%%%%%%%%%%%%
%%%%%%%%%%%%%%%%%%%%%%%%%%%%%%%%%%%%%%%%%%%%%%%%%%%%%%%%%%%%%%%%%%%%%%%%%%%%%%%%
\appendix

\settowidth\MacroIndent{\rmfamily\scriptsize 000\ }

 \DocInput{childdoc.dtx}

\end{document}
%</driver>
% \fi
%
% %%%%%%%%%%%%%%%%%%%%%%%%%%%%%%%%%%%%%%%%%%%%%%%%%%%%%%%%%%%%%%%%%%%%%%%%%%%%%%
% %%%%%%%%%%%%%%%%%%%%%%%%%%%%%%%%%%%%%%%%%%%%%%%%%%%%%%%%%%%%%%%%%%%%%%%%%%%%%%
% \section{Sample}
%\iffalse
%<*samplemain>
%\fi
%
% The following presents a sample document
% with two chapters, two parts, a title page,
% a compile flag as well as three forwarding files to set the flag.
% It consists of eight |.tex| files:
% \begin{center}
% \begin{tabular}{ll}
% |cdocsamp.tex|&main file\\
% |cdocsch1.tex|&include file for chapter 1\\
% |cdocsch2.tex|&include file for chapter 2\\
% |cdocspt3.tex|&include file for part 3\\
% |cdocspt4.tex|&include file for part 4\\
% |cdocsdrf.tex|&forwarding file for main file in draft mode\\
% |cdocsfi1.tex|&forwarding file for final version of chapter 1\\
% |cdocsfi2.tex|&forwarding file for final version of chapter 2\\
% \end{tabular}
% \end{center}
% Each of the eight files can be compiled directly by the \LaTeX{} compiler.
%
% %%%%%%%%%%%%%%%%%%%%%%%%%%%%%%%%%%%%%%
% \paragraph{Main File.}
%
% The main file is called |cdocsamp.tex|.
%
% Load the \textsf{childdoc} definitions and
% declare the filename for the main document:
%    \begin{macrocode}
\input{childdoc.def}
\childdocmain{}
%    \end{macrocode}

% Optional override for |\version| flag:
%    \begin{macrocode}
%%\ifchilddoc\else\providecommand{\version}{draft}\fi
%    \end{macrocode}

% Define the default values for the |\version| flag
% (|final| for the main file and |draft| for childs):
%    \begin{macrocode}
\ifchilddoc
\providecommand{\version}{draft}
\else
\providecommand{\version}{final}
\fi
%    \end{macrocode}

% Load the standard document class:
%    \begin{macrocode}
\documentclass[12pt]{article}
%    \end{macrocode}

% Start the document body:
%    \begin{macrocode}
\begin{document}
%    \end{macrocode}

% Declare a title page.
% Print title, part of document being processed and version flag:
%    \begin{macrocode}
\addtocounter{page}{-1}
\begin{center}
{\LARGE\bfseries{}childdoc example\par}
\vspace{1cm}
\ifchilddoc
\ifchilddocmanual part\else chapter\fi:
`\childdocname' of `\childdocjob'\par
\else
main document: `\childdocjob'\par
\fi
version: \version\par
\end{center}
\newpage
%    \end{macrocode}

% Manually include selected file,
% otherwise process as usual:
%    \begin{macrocode}
\ifchilddocmanual
\section*{part `\childdocname'}
\input{\childdocname}
\else
%    \end{macrocode}

% Include the two chapters:
%    \begin{macrocode}
\include{cdocsch1}
\include{cdocsch2}
%    \end{macrocode}

% Include the two parts unless only chapters should be displayed:
%    \begin{macrocode}
\ifchilddoc\else
\section{part three}
\input{cdocspt3}
\section{part four}
\input{cdocspt4}
\fi
%    \end{macrocode}

% Process as usual until here:
%    \begin{macrocode}
\fi
%    \end{macrocode}

% End of document body:
%    \begin{macrocode}
\end{document}
%    \end{macrocode}
%\iffalse
%</samplemain>
%\fi
%
% %%%%%%%%%%%%%%%%%%%%%%%%%%%%%%%%%%%%%%
% \paragraph{Chapter Include Files.}
%
% The include files are called |cdocsch1.tex| and |cdocsch2.tex|.
%
%\iffalse
%<*samplechap1|samplechap2>
%\fi

% Optional override for |\version| flag:
%    \begin{macrocode}
%%\providecommand{\version}{final}
%    \end{macrocode}

% Include the main document:
%    \begin{macrocode}
\input{childdoc.def}
\childdocof{cdocsamp}
%    \end{macrocode}

%\iffalse
%</samplechap1|samplechap2>
%\fi
%
%\iffalse
%<*samplechap1>
%\fi
% Some text for chapter 1:
%    \begin{macrocode}
\section{one}
some text in chapter one
%    \end{macrocode}

%\iffalse
%</samplechap1>
%\fi
% Some text for chapter 2:
%\iffalse
%<*samplechap2>
%\fi
%    \begin{macrocode}
\section{two}
more text in chapter two
%    \end{macrocode}

%\iffalse
%</samplechap2>
%\fi
%
% %%%%%%%%%%%%%%%%%%%%%%%%%%%%%%%%%%%%%%
% \paragraph{Part Include Files.}
%
% The include files are called |cdocspt3.tex| and |cdocspt4.tex|.
%
%\iffalse
%<*samplepart3|samplepart4>
%\fi

% Optional override for |\version| flag:
%    \begin{macrocode}
%%\providecommand{\version}{final}
%    \end{macrocode}

% Include the main document:
%    \begin{macrocode}
\input{childdoc.def}
\childdocby{cdocsamp}
%    \end{macrocode}

%\iffalse
%</samplepart3|samplepart4>
%\fi
%
%\iffalse
%<*samplepart3>
%\fi
% Some text for part 3:
%    \begin{macrocode}
some text in part three
%    \end{macrocode}

%\iffalse
%</samplepart3>
%\fi
% Some text for part 4:
%\iffalse
%<*samplepart4>
%\fi
%    \begin{macrocode}
more text in part four
%    \end{macrocode}

%\iffalse
%</samplepart4>
%\fi
%
% %%%%%%%%%%%%%%%%%%%%%%%%%%%%%%%%%%%%%%
% \paragraph{Forwarding for a Complete Draft.}
%
% The following forwarding file |cdocsdrf.tex|
% compiles the main document in draft mode:
%\iffalse
%<*sampledraft>
%\fi
%    \begin{macrocode}
\def\version{draft}
\input{childdoc.def}
\childdocforward{cdocsamp}
%    \end{macrocode}

%\iffalse
%</sampledraft>
%\fi
%
% %%%%%%%%%%%%%%%%%%%%%%%%%%%%%%%%%%%%%%
% \paragraph{Forwarding for Final Version of the Chapters.}
%
% The following forwarding files |cdocsfn1.tex| and |cdocsfn2.tex|
% (with identical content)
% compile the final versions of the child documents
% |cdocsch1.tex| and |cdocsch2.tex|, respectively:
%\iffalse
%<*samplefinal>
%\fi
%    \begin{macrocode}
\def\version{final}
\input{childdoc.def}
\childdocforwardprefix[cdocsamp]{cdocsfn}{cdocsch}
%    \end{macrocode}

%\iffalse
%</samplefinal>
%\fi
%
% %%%%%%%%%%%%%%%%%%%%%%%%%%%%%%%%%%%%%%
% \paragraph{Command Line Processing.}
%
% The following three command lines generate the output files
% |cdocscld|, |cdocscl1| and |cdocscl2|
% which should be identical to
% |cdocsdrf|, |cdocsch1| and |cdocsfn2|, respectively:
% \begin{center}
% \begin{tabular}{l}
% |latex -jobname cdocscld \|\\
% |  "\def\version{draft}\input{childdoc.def}\childdocforward{cdocsamp}"|\\
% |latex -jobname cdocscl1 \|\\
% |  "\input{childdoc.def}\childdocforward[cdocsamp]{cdocsch1}"|\\
% |latex -jobname cdocscl2 \|\\
% |  "\def\version{final}\input{childdoc.def}\childdocforward{cdocsch2}"|
% \end{tabular}
% \end{center}
% Note that the trailing backslash on each first line
% merely continues the input to the second line
% (for convenient cut ant paste).
% Furthermore, the command |latex| can be replaced by any
% of its alternative versions such as |pdflatex|.
%
% %%%%%%%%%%%%%%%%%%%%%%%%%%%%%%%%%%%%%%%%%%%%%%%%%%%%%%%%%%%%%%%%%%%%%%%%%%%%%%
% %%%%%%%%%%%%%%%%%%%%%%%%%%%%%%%%%%%%%%%%%%%%%%%%%%%%%%%%%%%%%%%%%%%%%%%%%%%%%%
% \section{Implementation}
%\iffalse
%<*package>
%\fi
%
% This section describes the definitions file |childdoc.def|.

% The definitions cannot be loaded using |\usepackage| or |\RequirePackage|
% which has a mechanism to prevent loading a style file more than once.
% When loading the definitions by means of |\input|
% multiple instances have to be prevented manually:
%\iffalse
%This code needs to be before the `\ProvidesFile' directive
%which is defined at the beginning of this file.
%Therefore it is also placed there and commented out here.
%</package>
%<*discard>
%\fi
%    \begin{macrocode}
\ifdefined\childdocmain\endinput\fi
%    \end{macrocode}
%\iffalse
%</discard>
%<*package>
%\fi
%
% \macro{\ifchilddoc}
% \macro{\ifchilddocmanual}
% The conditional |\ifchilddoc| tells whether a
% child (true) or main (false) document is being compiled.
% The conditional |\ifchilddocmanual| tells whether
% the |\includeonly| mechanism is used (false) or
% the selection of child files must be performed manually (true).
% The definitions initialise to false:
%    \begin{macrocode}
\newif\ifchilddoc
\newif\ifchilddocmanual
%    \end{macrocode}

% \macro{\childdocname}
% \macro{\childdocjob}
% The macro |\childdocname| stores the name of the main document
% to be compiled. The macro |\childdocjob| stores the name of
% the document on which the \LaTeX{} compiler was originally invoked.
% The content of |\jobname| cannot be compared
% to filenames specified in the source due to different catcodes.
% The following code rescans |\jobname|, stores the result
% in |\childdocname| and saves a copy in |\childdocjob|:
%    \begin{macrocode}
\edef\childdocname{\scantokens\expandafter{\jobname\noexpand}}
\let\childdocjob\childdocname
%    \end{macrocode}

% \macro{\childdocdisable}
% The macro |\childdocdisable| prevents the main file
% from being processed more than once.
% At this stage, the main document command |\childdocmain|
% is assumed to be called once again where it should do nothing.
% Any subsequent call to it should prevent
% a secondary processing of the main document
% It overwrites the forwarding commands
% |\childdocof| and |\childdocforward|
% with empty macros to prevent further inclusions of the main document:
%    \begin{macrocode}
\newcommand{\childdocdisable}
{
  \renewcommand{\childdocmain}[1]{\renewcommand{\childdocmain}[1]{\endinput}}
  \renewcommand{\childdocof}[1]{}
  \renewcommand{\childdocby}[2][]{}
  \renewcommand{\childdocforward}[2][]{}
  \renewcommand{\childdocdisable}{}
}
%    \end{macrocode}

% \macro{\childdocmain}
% The macro |\childdocmain| is to be called at the top of the main file
% with nothing or the main filename (without extension) as argument.
% First, it breaks loops.
% If the argument is not empty and does not match |\childdocname|
% (which is set by the first inclusion of |childdoc.def|),
% |\ifchilddoc| is set to true, |\includeonly| is applied to the child file
% and |\jobname| is set to the main file
% (for proper handling of |.aux| files):
%    \begin{macrocode}
\newcommand{\childdocmain}[1]
{
  \childdocdisable\childdocmain{}
  \if?#1?\else
    \begingroup
      \def\childdoctmp{#1}
      \ifx\childdoctmp\childdocname
        \def\childdoctmp{}
      \else
        \def\childdoctmp
        {
          \childdoctrue
          \includeonly{\childdocname}
          \def\childdocjob{#1}
          \def\jobname{#1}
        }
      \fi
      \expandafter
    \endgroup
    \childdoctmp
  \fi
}
%    \end{macrocode}

% \macro{\childdocof}
% The command |\childdocof| redirects
% compilation to the main file |#1|.
%    \begin{macrocode}
\newcommand{\childdocof}[1]
{
  \childdocdisable
  \childdoctrue
  \includeonly{\childdocname}
  \def\jobname{#1}
  \def\childdocjob{#1}
  \input{#1}
}
%    \end{macrocode}

% \macro{\childdocby}
% The command |\childdocby| ....
%    \begin{macrocode}
\newcommand{\childdocby}[2][]
{
  \childdocdisable
  \childdoctrue
  \childdocmanualtrue
  \if?#1?\else
    \def\jobname{#2}
  \fi
  \def\childdocjob{#2}
  \input{#2}
  \endinput
}
%    \end{macrocode}

% \macro{\childdocforward}
% The command |\childdocforward| redirects
% compilation to the main file or
% (if the optional argument is given) a child file.
% Parameters are set as if the main file
% or a child file starting with |\childdocof| was compiled.
% Then compilation is handed over to the main file:
%    \begin{macrocode}
\newcommand{\childdocforward}[2][]
{
  \begingroup
    \if?#1?
      \def\childdoctmp
      {
        \def\childdocname{#2}
        \def\childdocjob{#2}
        \def\jobname{#2}
        \input{#2}
        \endinput
      }
    \else
      \def\childdoctmp
      {
        \childdocdisable
        \def\childdocname{#2}
        \childdoctrue
        \includeonly{#2}
        \def\childdocjob{#1}
        \def\jobname{#1}
        \input{#1}
        \endinput
      }
    \fi
    \expandafter
  \endgroup
  \childdoctmp
}
%    \end{macrocode}

% \macro{\childdocforwardprefix}
% The command |\childdocforwardprefix| redirects
% compilation to the main or a child file by means of a pattern.
% The prefix |#1| in the current filename is replaced by |#2|
% and the suffix of the current filename is kept
% (it is assumed that the filename does not contain the substring `|~~~|'
% which is used as a delimiter).
% Compilation is handed over to the new file by |\childdocforward|:
%    \begin{macrocode}
\newcommand{\childdocforwardprefix}[3][]
{
  \begingroup
    \def\childdocextract #2##1~~~{\def\childdoctmp{\childdocforward[#1]{#3##1}}}
    \expandafter\childdocextract\childdocname~~~
    \expandafter
  \endgroup
  \childdoctmp
}
%    \end{macrocode}

% \macro{\childdoc}
% The deprecated macro |\childdoc| is a legacy version of |\childdocmain|:
%    \begin{macrocode}
\newcommand{\childdoc}{\childdocmain}
%    \end{macrocode}

% \macro{\childdocredirect}
% The deprecated macro |\childdocredirect| is a legacy version
% of |\childdocforward| and |\childdocforwardprefix|:
%    \begin{macrocode}
\newcommand{\childdocredirect}[2][]
{
  \begingroup
    \if?#1?
      \def\childdoctmp{\childdocforward{#2}}
    \else
      \def\childdoctmp{\childdocforwardprefix{#1}{#2}}
    \fi
    \expandafter
  \endgroup
  \childdoctmp
}
%    \end{macrocode}

%\iffalse
%</package>
%\fi
%
\endinput
\childdocforward[|\textit{main}|]{|\textit{dest}|}"|
\end{center}
%
Here \textit{target} is the name of the output file,
\textit{main} is the name of the main file
and \textit{dest} is the name of the main or child file to be processed
(all filenames without extensions).
The optional argument \textit{main} can be omitted
if \textit{main} matches \textit{dest}.
Optionally, compilation \textit{flags} can be defined via |\def| commands.
This command line makes the \TeX{} engine believe
it is compiling the file \textit{target}
whose content is specified as the latter parameter.
The provided code then forwards the processing to
\textit{main} or \textit{dest} as described in \secref{sec:forward}.

%%%%%%%%%%%%%%%%%%%%%%%%%%%%%%%%%%%%%%%%%%%%%%%%%%%%%%%%%%%%%%%%%%%%%%%%%%%%%%%%
\subsection{Include by Input}
\label{sec:input}

Including child documents by |\include| has some restrictions by design.
Most notably, the content of a child document always occupies
its own set of pages; pages cannot be shared between child documents.
Usually, this behaviour makes perfect sense
because each child document contain an essential part of the document.
However, in some situations it may be desirable to compose
a document from a collection of parts
without having mandatory page breaks between then.
For this case, the package
provides a mechanism to include parts
by |\input| which can also be processed individually.
However, by construction this mechanism
requires manual handling of the content to be output.

%%%%%%%%%%%%%%%%%%%%%%%%%%%%%%%%%%%%%%%%
\DescribeMacro{\ifchilddocmanual}
The main file should be prepared as usual, see \secref{sec:include}.
However, the document body must make a distinction
between processing of an individual part and of the main document, e.g.:
%
\begin{center}
\begin{tabular}{l}
|\ifchilddocmanual|\\
|\input{\childdocname}|\\
|\||else|\\
\textit{document body with }|\input{|\textit{part}|}|\\
|\||fi|
\end{tabular}
\end{center}
%
The conditional |\ifchilddocmanual| is true whenever
a part to be included by |\input| is being compiled,
and the name of the part is stored in |\childdocname|.

%%%%%%%%%%%%%%%%%%%%%%%%%%%%%%%%%%%%%%%%
\DescribeMacro{\childdocby}
Each part to be included by |\input| should start with:
%
\begin{center}
\begin{tabular}{l}
|% \iffalse
%
% childdoc.dtx Copyright (C) 2017-2018 Niklas Beisert
%
% This work may be distributed and/or modified under the
% conditions of the LaTeX Project Public License, either version 1.3
% of this license or (at your option) any later version.
% The latest version of this license is in
%   http://www.latex-project.org/lppl.txt
% and version 1.3 or later is part of all distributions of LaTeX
% version 2005/12/01 or later.
%
% This work has the LPPL maintenance status `maintained'.
%
% The Current Maintainer of this work is Niklas Beisert.
%
% This work consists of the files childdoc.dtx and childdoc.ins
% and the derived files childdoc.def and cdocsamp.tex with
% cdocsch1.tex, cdocsch2.tex, cdocsdrf.tex, cdocsfn1.tex, cdocsfn2.tex.
%
%<package>\ifdefined\childdocmain\endinput\fi
%<package>\ProvidesFile{childdoc.def}[2018/12/30 v2.0 child document driver]
%<samplemain>\ProvidesFile{cdocsamp.tex}[2018/12/30 v2.0 sample for childdoc]
%<*driver>
%\ProvidesFile{childdoc.drv}[2018/12/30 v2.0 childdoc reference manual file]
\PassOptionsToClass{10pt,a4paper}{article}
\documentclass{ltxdoc}

\usepackage[margin=35mm]{geometry}
\usepackage{hyperref}
\usepackage{hyperxmp}
\usepackage[usenames]{color}

\hypersetup{colorlinks=true}
\hypersetup{pdfstartview=FitH}
\hypersetup{pdfpagemode=UseNone}
\hypersetup{pdfsource={}}
\hypersetup{pdflang={en-UK}}
\hypersetup{pdfcopyright={Copyright 2017-2018 Niklas Beisert.
  This work may be distributed and/or modified under the
  conditions of the LaTeX Project Public License, either version 1.3
  of this license or (at your option) any later version.}}
\hypersetup{pdflicenseurl={http://www.latex-project.org/lppl.txt}}
\hypersetup{pdfcontactaddress={ETH Zurich, ITP, HIT K,
  Wolfgang-Pauli-Strasse 27}}
\hypersetup{pdfcontactpostcode={8093}}
\hypersetup{pdfcontactcity={Zurich}}
\hypersetup{pdfcontactcountry={Switzerland}}
\hypersetup{pdfcontactemail={nbeisert@itp.phys.ethz.ch}}
\hypersetup{pdfcontacturl={http://people.phys.ethz.ch/\xmptilde nbeisert/}}

\newcommand{\secref}[1]{\hyperref[#1]{section \ref*{#1}}}

\parskip1ex
\parindent0pt
\let\olditemize\itemize
\def\itemize{\olditemize\parskip0pt}

\begin{document}

\title{The \textsf{childdoc} Package}
\hypersetup{pdftitle={The childdoc Package}}
\author{Niklas Beisert\\[2ex]
  Institut f\"ur Theoretische Physik\\
  Eidgen\"ossische Technische Hochschule Z\"urich\\
  Wolfgang-Pauli-Strasse 27, 8093 Z\"urich, Switzerland\\[1ex]
  \href{mailto:nbeisert@itp.phys.ethz.ch}
  {\texttt{nbeisert@itp.phys.ethz.ch}}}
\hypersetup{pdfauthor={Niklas Beisert}}
\hypersetup{pdfsubject={Manual for the LaTeX2e Package childdoc}}
\date{30 December 2018, \textsf{v2.0}}
\maketitle

\begin{abstract}\noindent
\textsf{childdoc} is a \LaTeXe{} package
that enables the direct compilation
of document sections included by |\include|
to individual files.
\end{abstract}

\begingroup
\parskip0ex
\tableofcontents
\endgroup

%%%%%%%%%%%%%%%%%%%%%%%%%%%%%%%%%%%%%%%%%%%%%%%%%%%%%%%%%%%%%%%%%%%%%%%%%%%%%%%%
%%%%%%%%%%%%%%%%%%%%%%%%%%%%%%%%%%%%%%%%%%%%%%%%%%%%%%%%%%%%%%%%%%%%%%%%%%%%%%%%
\section{Introduction}

\LaTeX{} provides a mechanism to structure a large document (such as a book)
into a main file and several child files (containing the chapters)
using the |\include| command.
This mechanism is beneficial for documents
which span hundreds of pages in order to
make the source file(s) more manageable.
Moreover, compilation can be restricted to
selected child files by means of the |\includeonly| command.
The latter feature can be used to reduce the compilation time while editing
(this was significantly more useful in the earlier days of \LaTeX{})
or to generate a smaller document which is easier to navigate.
Another application of |\includeonly| is to generate
documents consisting of selected parts of the complete document.

However, there are a few drawbacks of the plain |\include| mechanism:
\begin{itemize}
\item
The child files cannot be compiled on their own,
they can only be compiled via the main file.
A naive editing environment
(such as a text editor with an option
to have the current file processed by \LaTeX)
may require one to switch to the main file before compiling;
attempting to compile the child file produces errors.
\item
The main file must be modified (each time)
to adjust the |\includeonly| command
to the present needs. This easily leaves the main file in a messy state.
\item
The generated document will always carry the filename
of the main document. This is inconvenient if
several child files are to be compiled and
to be kept for distribution.
\end{itemize}

The present package provides a simple interface
to make child files individually compilable by \LaTeX{}.
Compiling a child file then has the same effect as compiling
the main file with an |\includeonly| command
to select the appropriate child.
Moreover the generated document will carry the name of the child
rather than the main file.
This resolves all three above issues.

This feature is meant to make the editing of books,
thesis documents and lecture notes somewhat more convenient.
However, the package can also be used efficiently for
composing a series of documents (such as exercise sheets)
which are typically distributed individually.
It then assists the author in generating the individual documents
(potentially in different versions)
as well as a document containing the collected series.
Another application is in developing style files
or other kinds of included material
where compilation of the style file could redirect
to a sample or test file.

%%%%%%%%%%%%%%%%%%%%%%%%%%%%%%%%%%%%%%%%%%%%%%%%%%%%%%%%%%%%%%%%%%%%%%%%%%%%%%%%
%%%%%%%%%%%%%%%%%%%%%%%%%%%%%%%%%%%%%%%%%%%%%%%%%%%%%%%%%%%%%%%%%%%%%%%%%%%%%%%%
\section{Usage}

First of all, the package \textsf{childdoc} is \emph{not} a standard
\LaTeXe{} |.sty| style file! Therefore it needs to be invoked in
a non-standard way.

%%%%%%%%%%%%%%%%%%%%%%%%%%%%%%%%%%%%%%%%%%%%%%%%%%%%%%%%%%%%%%%%%%%%%%%%%%%%%%%%
\subsection{Included Files}
\label{sec:include}

%%%%%%%%%%%%%%%%%%%%%%%%%%%%%%%%%%%%%%%%
\DescribeMacro{\childdocmain}
To use the package, add the commands
\begin{center}
\begin{tabular}{l}
|\input{childdoc.def}|\\
|\childdocmain{}|\\
\end{tabular}
\end{center}
at the very top of the main \LaTeX{} file,
in particular \emph{before} the |\documentclass| statement!
The argument of |\childdocmain| should be left empty
(but it must be present).

%%%%%%%%%%%%%%%%%%%%%%%%%%%%%%%%%%%%%%%%
\DescribeMacro{\childdocof}
Furthermore, add the commands
\begin{center}
\begin{tabular}{l}
|\input{childdoc.def}|\\
|\childdocof{|\textit{main}|}|\\
\end{tabular}
\end{center}
at the top of every child file \textit{child}
which is included by |\include{|\textit{child}|}|
from within the main file
(or at least for those files to be compiled individually).
The argument \textit{main} must be the filename of the main file.

There are a couple of
considerations in setting up the main and child documents:

%%%%%%%%%%%%%%%%%%%%%%%%%%%%%%%%%%%%%%%%
\paragraph{Restrictions.}

Please note the following restrictions:
\begin{itemize}
\item
|\childdocmain| must be called with one argument \textit{main}
to ensure compatibility with earlier version of the package.
It must either be empty (|\childdocmain{}|)
or precisely match the filename of the main file in which it is specified.
See \secref{sec:detection} for further information.
\item
The filename \textit{main} must be specified without the |.tex| extension.
\item
The filename \textit{main} is case sensitive
(even in case-insensitive file systems)
due to internal string comparison.
\item
The argument \textit{main} should be fully expanded, it cannot be a macro.
\item
Subdirectories and special characters should be avoided in filenames.
\item
The command |\childdocmain{|\textit{main}|}| must be followed by a whitespace.
It should not be followed immediately by another command
or by a comment mark `|%|'.
This is because the \TeX{} parser reads the token immediately following
the argument of |\childdocmain| and puts it
at the beginning of every child section;
however, a white\-space is ignored.
\end{itemize}

%%%%%%%%%%%%%%%%%%%%%%%%%%%%%%%%%%%%%%%%
\paragraph{Content of Main File.}

It is advisable to place all content in the child files included by |\include|.
Any output contained in the main file will appear in all child documents
unless suppressed manually;
it cannot be suppressed automatically by the |\includeonly| directive
and thus should normally be avoided.
A method to include some content in the main file
by means of conditional processing is described in \secref{sec:conditional}.

%%%%%%%%%%%%%%%%%%%%%%%%%%%%%%%%%%%%%%%%
\paragraph{Page Numbering.}

When only a part of the document is compiled,
the appropriate numbering of pages
(as well as other status parameters)
is determined from the |.aux| files.
The latter contain information from previous passes.
However this information needs to propagate through
all intermediate child documents.
Therefore the page numbering in child documents may well
be inconsistent until the complete document is compiled at least once.

A useful (if unconventional) way to always ensure a consistent
page numbering is to restart the numbering in each child document
and denote the pages by `\textit{child}|.|\textit{page}'
where \textit{child} represents the chapter/section number of the child file.
This can be achieved by the command
|\numberwithin{page}{|\textit{child}|}|
of the \textsf{amsmath} package
where \textit{child} can be |chapter| or |section|
depending on the chosen structuring.
Alternatively, one can modify the macro |\thepage| appropriately
and reset the counter |page| at the start of each child file.

%%%%%%%%%%%%%%%%%%%%%%%%%%%%%%%%%%%%%%%%%%%%%%%%%%%%%%%%%%%%%%%%%%%%%%%%%%%%%%%%
\subsection{Conditional Processing}
\label{sec:conditional}

The package provides a mechanism to compile different versions
of a document. To customise the versions further some conditional processing
can come in handy to distinguish which version is being compiled.
The package provides two macros to describe the compilation context:

%%%%%%%%%%%%%%%%%%%%%%%%%%%%%%%%%%%%%%%%
\DescribeMacro{\ifchilddoc}
The conditional |\ifchilddoc| distinguishes between the compilation of
child documents and the main document:
%
\begin{center}
|\ifchilddoc |\textit{child-code}| |[|\||else |\textit{main-code}]| \||fi|
\end{center}

%%%%%%%%%%%%%%%%%%%%%%%%%%%%%%%%%%%%%%%%
\DescribeMacro{\childdocname}
\DescribeMacro{\childdocjob}
The macro |\childdocname| contains the filename (without extension)
of the main or child file being processed.
Note that |\childdocjob| will always contain the name of the main file.

%%%%%%%%%%%%%%%%%%%%%%%%%%%%%%%%%%%%%%%%
\paragraph{Title Page.}

Conditional processing can be used to include a title or banner page
in the main document when proper precautions are taken.
Importantly, the code in the main file should ensure that the page counter
(as well as other status parameters which are stored in the |.aux| files)
takes the same value after the conditional processing.
Otherwise the page numbers may take divergent values
depending on which part is compiled.

For example, a title page could be declared by:
%
\begin{center}
\begin{tabular}{l}
|\ifchilddoc\||else|\\
|\addtocounter{page}{-1}|\\
\textit{code for title page}\\
|\newpage|\\
|\||fi|
\end{tabular}
\end{center}
%
A banner page for the child documents can be generated by:
%
\begin{center}
\begin{tabular}{l}
|\ifchilddoc|\\
|\addtocounter{page}{-1}|\\
\textit{code for banner page}\\
|\newpage|\\
|\||fi|
\end{tabular}
\end{center}
%
Here one could write a message such as:
\begin{center}
|This is the part \childdocname{} of \childdocjob{}.|
\end{center}

%%%%%%%%%%%%%%%%%%%%%%%%%%%%%%%%%%%%%%%%%%%%%%%%%%%%%%%%%%%%%%%%%%%%%%%%%%%%%%%%
\subsection{Flags}
\label{sec:flags}

The package makes it easy to generate different versions
of the main or child documents.
To this end compilation flags can be defined
and assigned different default values.
They will be particularly useful in conjunction
with the forwarding mechanism described in \secref{sec:forward}.

For example, it may be useful to have a flag |\version|
which can be set to |draft| or |final|.
The document source will contain some conditional code
depending on the value of |\version|.
Suppose further, the flag should default to |final| for the main file
and to |draft| for child files
which is a natural assignment for editing the document.
This is achieved by placing the following code
in the preamble of the main document
(below the |\childdocmain| directive):
%
\begin{center}
\begin{tabular}{l}
|\ifchilddoc|\\
|\providecommand{\version}{draft}|\\
|\||else|\\
|\providecommand{\version}{final}|\\
|\||fi|
\end{tabular}
\end{center}
%
The definition by |\providecommand| makes sure
that previous definitions are not overwritten.
Further statements |\providecommand{\version}{...}|
can thus be added before the above code to override it.

For the main file, one might add a line
(between |\childdocmain| and the above block)
%
\begin{center}
|%\ifchilddoc\||else\providecommand{\version}{draft}\||fi|
\end{center}
%
which can be uncommented to produce a draft version.
Likewise one can add a line to the very top of a child file
(above the |\childdocof{|\textit{main}|}| directive)
%
\begin{center}
|%\providecommand{\version}{final}|
\end{center}
%
which can be uncommented to produce the final version of this child document.

%%%%%%%%%%%%%%%%%%%%%%%%%%%%%%%%%%%%%%%%%%%%%%%%%%%%%%%%%%%%%%%%%%%%%%%%%%%%%%%%
\subsection{Forwarding}
\label{sec:forward}

Different versions of the main or child documents
using compilation flags as described in \secref{sec:flags}
can be (permanently) stored in different files
for convenient compilation, viewing and distribution.
To this end, the package defines a command
to pass on compilation to a different file:

%%%%%%%%%%%%%%%%%%%%%%%%%%%%%%%%%%%%%%%%
\DescribeMacro{\childdocforward}
The command |\childdocforward| redirects processing to
another source file:
%
\begin{center}
\begin{tabular}{l}
|\input{childdoc.def}|\\
|\childdocforward[|\textit{main}|]{|\textit{dest}|}|\\
\end{tabular}
\end{center}
%
The argument \textit{dest} is the destination file
(without extension).
It should be the main file or one of the child files.
Note that further \textsf{childdoc} directives
such as |\childdocof| and |\childdocforward|
in the indicated file will be processed in this form.
The optional argument \textit{main}
passes on directly to the main file \textit{main}
while pretending to compile the child \textit{dest}.
This form behaves as if \textit{dest}
issues |\childdocof{|\textit{main}|}| right away,
and no further \textsf{childdoc} directives will be processed.

%%%%%%%%%%%%%%%%%%%%%%%%%%%%%%%%%%%%%%%%
\DescribeMacro{\...prefix}
In the alternative form |\childdocforwardprefix|,
%
\begin{center}
\begin{tabular}{l}
|\input{childdoc.def}|\\
|\childdocforwardprefix[|\textit{main}|]{|\textit{prefix}|}{|\textit{dest}|}|
\end{tabular}
\end{center}
%
the destination file is determined by a pattern
depending on the current file:
To make this work, the current file must be called
`{\textit{prefix}\hspace{0.2em}\textit{suffix}}'
with \textit{prefix} matching precisely the argument.
Processing is then passed on to the file
`{\textit{dest}\hspace{0.2em}\textit{suffix}}'.
Surely, the same effect is achieved by
directly specifying the
argument `{\textit{dest}\hspace{0.2em}\textit{suffix}}'
in the first form.
However, that requires to set up a different file
for each child. With the alternative form of the command
all these files can have exactly the same content
which simplifies setting them up and maintaining them.

For example, the following file |draft.tex|
with a compilation flag |\version| as described in \secref{sec:flags}
compiles the main document as a draft:
%
\begin{center}
\begin{tabular}{l}
|\def\version{draft}|\\
|\input{childdoc.def}|\\
|\childdocforward{|\textit{main}|}|
\end{tabular}
\end{center}
%
Likewise, the following files |final|\textit{nn}|.tex|
compile the final version of the child document
|child|\textit{nn}|.tex|:
%
\begin{center}
\begin{tabular}{l}
|\def\version{final}|\\
|\input{childdoc.def}|\\
|\childdocforwardprefix{final}{child}|
\end{tabular}
\end{center}
%

Note that when several versions of a main file and/or of each child file
are to be generated, it may be convenient to set up a |Makefile| or
shell script to automatise the process.

%%%%%%%%%%%%%%%%%%%%%%%%%%%%%%%%%%%%%%%%%%%%%%%%%%%%%%%%%%%%%%%%%%%%%%%%%%%%%%%%
\subsection{Command Line Processing}
\label{sec:commandline}

The effect of redirection files can also be achieved by invoking
the \LaTeX{} compiler with a more elaborate command line.
Most conveniently this should be done as part
of a shell script or a |Makefile|.

When using \textsf{childdoc} in the main file, the following
command lines effectively perform a redirection
(note that depending on the shell being used,
backslashes may have to be doubled: `|\|' $\to$ `|\\|'):
%
\begin{center}
|... -jobname "|\textit{target}|" |\\|"|[\textit{flags}]%
|\input{childdoc.def}\childdocforward[|\textit{main}|]{|\textit{dest}|}"|
\end{center}
%
Here \textit{target} is the name of the output file,
\textit{main} is the name of the main file
and \textit{dest} is the name of the main or child file to be processed
(all filenames without extensions).
The optional argument \textit{main} can be omitted
if \textit{main} matches \textit{dest}.
Optionally, compilation \textit{flags} can be defined via |\def| commands.
This command line makes the \TeX{} engine believe
it is compiling the file \textit{target}
whose content is specified as the latter parameter.
The provided code then forwards the processing to
\textit{main} or \textit{dest} as described in \secref{sec:forward}.

%%%%%%%%%%%%%%%%%%%%%%%%%%%%%%%%%%%%%%%%%%%%%%%%%%%%%%%%%%%%%%%%%%%%%%%%%%%%%%%%
\subsection{Include by Input}
\label{sec:input}

Including child documents by |\include| has some restrictions by design.
Most notably, the content of a child document always occupies
its own set of pages; pages cannot be shared between child documents.
Usually, this behaviour makes perfect sense
because each child document contain an essential part of the document.
However, in some situations it may be desirable to compose
a document from a collection of parts
without having mandatory page breaks between then.
For this case, the package
provides a mechanism to include parts
by |\input| which can also be processed individually.
However, by construction this mechanism
requires manual handling of the content to be output.

%%%%%%%%%%%%%%%%%%%%%%%%%%%%%%%%%%%%%%%%
\DescribeMacro{\ifchilddocmanual}
The main file should be prepared as usual, see \secref{sec:include}.
However, the document body must make a distinction
between processing of an individual part and of the main document, e.g.:
%
\begin{center}
\begin{tabular}{l}
|\ifchilddocmanual|\\
|\input{\childdocname}|\\
|\||else|\\
\textit{document body with }|\input{|\textit{part}|}|\\
|\||fi|
\end{tabular}
\end{center}
%
The conditional |\ifchilddocmanual| is true whenever
a part to be included by |\input| is being compiled,
and the name of the part is stored in |\childdocname|.

%%%%%%%%%%%%%%%%%%%%%%%%%%%%%%%%%%%%%%%%
\DescribeMacro{\childdocby}
Each part to be included by |\input| should start with:
%
\begin{center}
\begin{tabular}{l}
|\input{childdoc.def}|\\
|\childdocby{|\textit{main}|}|\\
\end{tabular}
\end{center}
%
The directive |\childdocby| is similar to |\childdocof|
described in \secref{sec:include},
but the subsequent selection of content must be done manually.
To that end, both |\ifchilddoc| and |\ifchilddocmanual|
will be true upon processing of a part,
and the name of the part is stored in |\childdocname|.
Note that |\jobname| will be set to the filename of the current part
so that each part receives an individual |.aux| file
that does not interfere with the |.aux| file(s) of the main document.
This behaviour can be altered by the alternative form
|\childdocby[*]{|\textit{main}|}| (with a non-empty optional argument)
which uses the |.aux| file of the main document
by setting |\jobname| to \textit{main}.

%%%%%%%%%%%%%%%%%%%%%%%%%%%%%%%%%%%%%%%%%%%%%%%%%%%%%%%%%%%%%%%%%%%%%%%%%%%%%%%%
\subsection{Driver Development}
\label{sec:driver}

The \textsf{childdoc} mechanism can also be use for the development
of definition files such as \LaTeX{} styles or classes.
This case differs from the above setup with multiple parts
included by |\include| in that no |\includeonly| should be invoked.
This can be achieved by starting the include file
(before |\ProvidesPackage|) with:
%
\begin{center}
\begin{tabular}{l}
|\input{childdoc.def}|\\
|\childdocforward{|\textit{main}|}|\\
\end{tabular}
\end{center}
%
or alternatively with:
%
\begin{center}
\begin{tabular}{l}
|\input{childdoc.def}|\\
|\childdocby{|\textit{main}|}|\\
\end{tabular}
\end{center}
%
Both forms have slightly different effects as described above.
The main file is prepared as usual, see \secref{sec:include}.

%%%%%%%%%%%%%%%%%%%%%%%%%%%%%%%%%%%%%%%%%%%%%%%%%%%%%%%%%%%%%%%%%%%%%%%%%%%%%%%%
\subsection{Legacy Detection}
\label{sec:detection}

The directive |\childdocmain| in the main file can detect
whether the complete document or merely a child is to be compiled
even without using the directive |\childdocof|.
This method is deprecated because it is less robust
and there is no compelling reason to use it;
it is merely provided for backward compatibility
and it may be removed in future versions.

If the detection mechanism is to be used,
it is mandatory to correctly specify
the filename of the main file as the argument of |\childdocmain|:
%
\begin{center}
\begin{tabular}{l}
|\input{childdoc.def}|\\
|\childdocmain{|\textit{main}|}|\\
\end{tabular}
\end{center}
%
If |\jobname| does not match the argument \textit{main} of |\childdocmain|,
it is assumed that |\jobname| points to the child file to be compiled.
When using |\childdocmain| with the main file specified as argument,
it suffices to start a child file
with just |\input{|\textit{main}|}|
without loading of the package and using |\childdocof|.
If instead all processing is done
with the appropriate \textsf{childdoc} directives,
the argument of \textit{main} of |\childdocmain| can be empty.

An alternative version of the command line processing described
in \secref{sec:commandline} using the detection mechanism reads:
%
\begin{center}
|... -jobname "|\textit{target}|" "|[\textit{flags}]%
[|\def\jobname{|\textit{dest}|}|]|\input{|\textit{main}|}"|
\end{center}

%%%%%%%%%%%%%%%%%%%%%%%%%%%%%%%%%%%%%%%%%%%%%%%%%%%%%%%%%%%%%%%%%%%%%%%%%%%%%%%%
\subsection{Manual Code}
\label{sec:manual}

In case one cannot be certain whether the definitions file |childdoc.def|
is installed on the target \TeX{} distribution
and one prefers not to ship it,
it is conceivable to paste a few relevant commands into the sources.

To that end, drop all statements |\input{childdoc.def}|
and perform the replacements as outlined below.
Instead of |\childdocmain{|\textit{main}|}| add the following code
to the top of the main file:
%
\begin{center}
\begin{tabular}{l}
|\||ifdefined\childdocname\endinput\||fi\newif\ifchilddoc|\\
|\edef\childdocname{\scantokens\expandafter{\jobname\noexpand}}|\\
|\def\childdocmain{|\textit{main}|}\||ifx\childdocmain\childdocname\||else|\\
|\childdoctrue\includeonly{\childdocname}\let\jobname\childdocmain\||fi|\\
\end{tabular}
\end{center}
%
Instead of |\childdocof{|\textit{main}|}| just include the main file
at the top of each child file:
%
\begin{center}
|\input{|\textit{main}|}|
\end{center}
%
A simple redirection |\childdocforward{|\textit{dest}|}| is achieved by:
%
\begin{center}
|\def\jobname{|\textit{dest}|}\input{\jobname}|
\end{center}
%
The redirection with prefix
|\childdocforwardprefix[|\textit{prefix}|]{|\textit{dest}|}|
is accomplished by:
%
\begin{center}
\begin{tabular}{l}
|{\edef\jobname{\scantokens\expandafter{\jobname\noexpand}}|\\
|\def\redirectjob |\textit{prefix}|#1~~~{\gdef\jobname{|\textit{dest}|#1}}|\\
|\expandafter\redirectjob\jobname~~~}\input{\jobname}|
\end{tabular}
\end{center}

In an alternative approach,
child documents can be compiled by a specific command line
without additional code or specific definitions:
%
\begin{center}
|... -jobname "|\textit{target}|" "|[\textit{flags}]%
|\includeonly{|\textit{dest}|}\input{|\textit{main}|}"|
\end{center}
%

%%%%%%%%%%%%%%%%%%%%%%%%%%%%%%%%%%%%%%%%%%%%%%%%%%%%%%%%%%%%%%%%%%%%%%%%%%%%%%%%
%%%%%%%%%%%%%%%%%%%%%%%%%%%%%%%%%%%%%%%%%%%%%%%%%%%%%%%%%%%%%%%%%%%%%%%%%%%%%%%%
\section{Information}

%%%%%%%%%%%%%%%%%%%%%%%%%%%%%%%%%%%%%%%%%%%%%%%%%%%%%%%%%%%%%%%%%%%%%%%%%%%%%%%%
\subsection{Copyright}

Copyright \copyright{} 2017--2018 Niklas Beisert

This work may be distributed and/or modified under the
conditions of the \LaTeX{} Project Public License, either version 1.3
of this license or (at your option) any later version.
The latest version of this license is in
  \url{http://www.latex-project.org/lppl.txt}
and version 1.3 or later is part of all distributions of \LaTeX{}
version 2005/12/01 or later.

This work has the LPPL maintenance status `maintained'.

The Current Maintainer of this work is Niklas Beisert.

This work consists of the files |README.txt|, |childdoc.ins| and |childdoc.dtx|
as well as the derived files |childdoc.def|, |cdocsamp.tex|
with |cdocsch1.tex|, |cdocsch2.tex|, |cdocspt3.tex|, |cdocspt4.tex|,
|cdocsdrf.tex|, |cdocsfn1.tex|, |cdocsfn2.tex|
as well as |childdoc.pdf|.

%%%%%%%%%%%%%%%%%%%%%%%%%%%%%%%%%%%%%%%%%%%%%%%%%%%%%%%%%%%%%%%%%%%%%%%%%%%%%%%%
\subsection{Files and Installation}

The package consists of the files:
%
\begin{center}
\begin{tabular}{ll}
    |README.txt|   & readme file \\
    |childdoc.ins| & installation file \\
    |childdoc.dtx| & source file \\
    |childdoc.def| & definition file \\
    |cdocsamp.tex| & sample main file \\
    |cdocsch1.tex| & sample include file \\
    |cdocsch2.tex| & sample include file \\
    |cdocspt3.tex| & sample part file \\
    |cdocspt4.tex| & sample part file \\
    |cdocsdrf.tex| & sample redirection file \\
    |cdocsfn1.tex| & sample redirection file \\
    |cdocsfn2.tex| & sample redirection file \\
    |childdoc.pdf| & manual
\end{tabular}
\end{center}
%
The distribution consists of the files
|README.txt|, |childdoc.ins| and |childdoc.dtx|.
%
\begin{itemize}
\item
Run (pdf)\LaTeX{} on |childdoc.dtx|
to compile the manual |childdoc.pdf| (this file).
\item
Run \LaTeX{} on |childdoc.ins| to create the definitions file |childdoc.def|
and the sample |cdocsamp.tex| with include files
|cdocsch1.tex|, |cdocsch2.tex|, |cdocspt3.tex|, |cdocspt4.tex|,
|cdocsdrf.tex|, |cdocsfn1.tex|, |cdocsfn2.tex|.
Then copy the file |childdoc.def| to an appropriate directory of your \LaTeX{}
distribution, e.g.\ \textit{texmf-root}|/tex/latex/childdoc|.
\end{itemize}

%%%%%%%%%%%%%%%%%%%%%%%%%%%%%%%%%%%%%%%%%%%%%%%%%%%%%%%%%%%%%%%%%%%%%%%%%%%%%%%%
\subsection{Related CTAN Packages}

There are several other packages which offer a similar functionality:
%
\begin{itemize}
\item
The packages
\href{http://ctan.org/pkg/docmute}{\textsf{docmute}},
\href{http://ctan.org/pkg/includex}{\textsf{includex}} and
\href{http://ctan.org/pkg/standalone}{\textsf{standalone}}
provide commands to include only the document body of
a child file thus allowing both files to be compiled individually.
\item
The packages \href{http://ctan.org/pkg/subdocs}{\textsf{subdocs}}
and \href{http://ctan.org/pkg/subfiles}{\textsf{subfiles}}
provide structures in which the main and child documents can be
encapsulated and allowing them to be compiled individually.
The inclusion mechanism is different from the conventional |\include|.
\item
The package \href{http://ctan.org/pkg/combine}{\textsf{combine}}
is an elaborate solution to combine several documents into one.
\end{itemize}
%
See also the CTAN topic \href{http://ctan.org/topic/subdocs}{\textsf{subdocs}}
for further related packages.
The present package differs from the above solutions in that
a document structure constructed with the conventional |\include| mechanism
just needs two extra commands at the top of every file
such that all constituent files can be compiled individually.

%%%%%%%%%%%%%%%%%%%%%%%%%%%%%%%%%%%%%%%%%%%%%%%%%%%%%%%%%%%%%%%%%%%%%%%%%%%%%%%%
%\subsection{Feature Suggestions}
%
%The following is a list of features which may be useful for future
%versions of this package:
%%
%\begin{itemize}
%\item
%\ldots
%\end{itemize}

%%%%%%%%%%%%%%%%%%%%%%%%%%%%%%%%%%%%%%%%%%%%%%%%%%%%%%%%%%%%%%%%%%%%%%%%%%%%%%%%
\subsection{Revision History}

%%%%%%%%%%%%%%%%%%%%%%%%%%%%%%%%%%%%%%%%
\paragraph{v2.0:} 2018/12/30

\begin{itemize}
\item
immediate forward processing
\item
added |\childdocby| mechanism
\item
manual restructured
\end{itemize}

%%%%%%%%%%%%%%%%%%%%%%%%%%%%%%%%%%%%%%%%
\paragraph{v1.6:} 2018/01/17

\begin{itemize}
\item
application for development of include files
\item
corrections to manual
\end{itemize}

%%%%%%%%%%%%%%%%%%%%%%%%%%%%%%%%%%%%%%%%
\paragraph{v1.5:} 2017/05/21

\begin{itemize}
\item
more complete structuring introduced
\item
|\childdocof| introduced
\item
|\childdoc| renamed to |\childdocmain|
\item
|\childredirect| renamed to |\childdocforward| and |\childdocforwardprefix|
and functionality expanded
\end{itemize}

%%%%%%%%%%%%%%%%%%%%%%%%%%%%%%%%%%%%%%%%
\paragraph{v1.0:} 2017/04/27

\begin{itemize}
\item
manual and install package
\item
first version published on CTAN
\end{itemize}

%%%%%%%%%%%%%%%%%%%%%%%%%%%%%%%%%%%%%%%%
\paragraph{v0.6:} 2017/04/26

\begin{itemize}
\item
redirection mechanism added
\end{itemize}

%%%%%%%%%%%%%%%%%%%%%%%%%%%%%%%%%%%%%%%%
\paragraph{v0.5:} 2017/04/26

\begin{itemize}
\item
functionality in definition file
\end{itemize}


%%%%%%%%%%%%%%%%%%%%%%%%%%%%%%%%%%%%%%%%%%%%%%%%%%%%%%%%%%%%%%%%%%%%%%%%%%%%%%%%
%%%%%%%%%%%%%%%%%%%%%%%%%%%%%%%%%%%%%%%%%%%%%%%%%%%%%%%%%%%%%%%%%%%%%%%%%%%%%%%%
%%%%%%%%%%%%%%%%%%%%%%%%%%%%%%%%%%%%%%%%%%%%%%%%%%%%%%%%%%%%%%%%%%%%%%%%%%%%%%%%
\appendix

\settowidth\MacroIndent{\rmfamily\scriptsize 000\ }

 \DocInput{childdoc.dtx}

\end{document}
%</driver>
% \fi
%
% %%%%%%%%%%%%%%%%%%%%%%%%%%%%%%%%%%%%%%%%%%%%%%%%%%%%%%%%%%%%%%%%%%%%%%%%%%%%%%
% %%%%%%%%%%%%%%%%%%%%%%%%%%%%%%%%%%%%%%%%%%%%%%%%%%%%%%%%%%%%%%%%%%%%%%%%%%%%%%
% \section{Sample}
%\iffalse
%<*samplemain>
%\fi
%
% The following presents a sample document
% with two chapters, two parts, a title page,
% a compile flag as well as three forwarding files to set the flag.
% It consists of eight |.tex| files:
% \begin{center}
% \begin{tabular}{ll}
% |cdocsamp.tex|&main file\\
% |cdocsch1.tex|&include file for chapter 1\\
% |cdocsch2.tex|&include file for chapter 2\\
% |cdocspt3.tex|&include file for part 3\\
% |cdocspt4.tex|&include file for part 4\\
% |cdocsdrf.tex|&forwarding file for main file in draft mode\\
% |cdocsfi1.tex|&forwarding file for final version of chapter 1\\
% |cdocsfi2.tex|&forwarding file for final version of chapter 2\\
% \end{tabular}
% \end{center}
% Each of the eight files can be compiled directly by the \LaTeX{} compiler.
%
% %%%%%%%%%%%%%%%%%%%%%%%%%%%%%%%%%%%%%%
% \paragraph{Main File.}
%
% The main file is called |cdocsamp.tex|.
%
% Load the \textsf{childdoc} definitions and
% declare the filename for the main document:
%    \begin{macrocode}
\input{childdoc.def}
\childdocmain{}
%    \end{macrocode}

% Optional override for |\version| flag:
%    \begin{macrocode}
%%\ifchilddoc\else\providecommand{\version}{draft}\fi
%    \end{macrocode}

% Define the default values for the |\version| flag
% (|final| for the main file and |draft| for childs):
%    \begin{macrocode}
\ifchilddoc
\providecommand{\version}{draft}
\else
\providecommand{\version}{final}
\fi
%    \end{macrocode}

% Load the standard document class:
%    \begin{macrocode}
\documentclass[12pt]{article}
%    \end{macrocode}

% Start the document body:
%    \begin{macrocode}
\begin{document}
%    \end{macrocode}

% Declare a title page.
% Print title, part of document being processed and version flag:
%    \begin{macrocode}
\addtocounter{page}{-1}
\begin{center}
{\LARGE\bfseries{}childdoc example\par}
\vspace{1cm}
\ifchilddoc
\ifchilddocmanual part\else chapter\fi:
`\childdocname' of `\childdocjob'\par
\else
main document: `\childdocjob'\par
\fi
version: \version\par
\end{center}
\newpage
%    \end{macrocode}

% Manually include selected file,
% otherwise process as usual:
%    \begin{macrocode}
\ifchilddocmanual
\section*{part `\childdocname'}
\input{\childdocname}
\else
%    \end{macrocode}

% Include the two chapters:
%    \begin{macrocode}
\include{cdocsch1}
\include{cdocsch2}
%    \end{macrocode}

% Include the two parts unless only chapters should be displayed:
%    \begin{macrocode}
\ifchilddoc\else
\section{part three}
\input{cdocspt3}
\section{part four}
\input{cdocspt4}
\fi
%    \end{macrocode}

% Process as usual until here:
%    \begin{macrocode}
\fi
%    \end{macrocode}

% End of document body:
%    \begin{macrocode}
\end{document}
%    \end{macrocode}
%\iffalse
%</samplemain>
%\fi
%
% %%%%%%%%%%%%%%%%%%%%%%%%%%%%%%%%%%%%%%
% \paragraph{Chapter Include Files.}
%
% The include files are called |cdocsch1.tex| and |cdocsch2.tex|.
%
%\iffalse
%<*samplechap1|samplechap2>
%\fi

% Optional override for |\version| flag:
%    \begin{macrocode}
%%\providecommand{\version}{final}
%    \end{macrocode}

% Include the main document:
%    \begin{macrocode}
\input{childdoc.def}
\childdocof{cdocsamp}
%    \end{macrocode}

%\iffalse
%</samplechap1|samplechap2>
%\fi
%
%\iffalse
%<*samplechap1>
%\fi
% Some text for chapter 1:
%    \begin{macrocode}
\section{one}
some text in chapter one
%    \end{macrocode}

%\iffalse
%</samplechap1>
%\fi
% Some text for chapter 2:
%\iffalse
%<*samplechap2>
%\fi
%    \begin{macrocode}
\section{two}
more text in chapter two
%    \end{macrocode}

%\iffalse
%</samplechap2>
%\fi
%
% %%%%%%%%%%%%%%%%%%%%%%%%%%%%%%%%%%%%%%
% \paragraph{Part Include Files.}
%
% The include files are called |cdocspt3.tex| and |cdocspt4.tex|.
%
%\iffalse
%<*samplepart3|samplepart4>
%\fi

% Optional override for |\version| flag:
%    \begin{macrocode}
%%\providecommand{\version}{final}
%    \end{macrocode}

% Include the main document:
%    \begin{macrocode}
\input{childdoc.def}
\childdocby{cdocsamp}
%    \end{macrocode}

%\iffalse
%</samplepart3|samplepart4>
%\fi
%
%\iffalse
%<*samplepart3>
%\fi
% Some text for part 3:
%    \begin{macrocode}
some text in part three
%    \end{macrocode}

%\iffalse
%</samplepart3>
%\fi
% Some text for part 4:
%\iffalse
%<*samplepart4>
%\fi
%    \begin{macrocode}
more text in part four
%    \end{macrocode}

%\iffalse
%</samplepart4>
%\fi
%
% %%%%%%%%%%%%%%%%%%%%%%%%%%%%%%%%%%%%%%
% \paragraph{Forwarding for a Complete Draft.}
%
% The following forwarding file |cdocsdrf.tex|
% compiles the main document in draft mode:
%\iffalse
%<*sampledraft>
%\fi
%    \begin{macrocode}
\def\version{draft}
\input{childdoc.def}
\childdocforward{cdocsamp}
%    \end{macrocode}

%\iffalse
%</sampledraft>
%\fi
%
% %%%%%%%%%%%%%%%%%%%%%%%%%%%%%%%%%%%%%%
% \paragraph{Forwarding for Final Version of the Chapters.}
%
% The following forwarding files |cdocsfn1.tex| and |cdocsfn2.tex|
% (with identical content)
% compile the final versions of the child documents
% |cdocsch1.tex| and |cdocsch2.tex|, respectively:
%\iffalse
%<*samplefinal>
%\fi
%    \begin{macrocode}
\def\version{final}
\input{childdoc.def}
\childdocforwardprefix[cdocsamp]{cdocsfn}{cdocsch}
%    \end{macrocode}

%\iffalse
%</samplefinal>
%\fi
%
% %%%%%%%%%%%%%%%%%%%%%%%%%%%%%%%%%%%%%%
% \paragraph{Command Line Processing.}
%
% The following three command lines generate the output files
% |cdocscld|, |cdocscl1| and |cdocscl2|
% which should be identical to
% |cdocsdrf|, |cdocsch1| and |cdocsfn2|, respectively:
% \begin{center}
% \begin{tabular}{l}
% |latex -jobname cdocscld \|\\
% |  "\def\version{draft}\input{childdoc.def}\childdocforward{cdocsamp}"|\\
% |latex -jobname cdocscl1 \|\\
% |  "\input{childdoc.def}\childdocforward[cdocsamp]{cdocsch1}"|\\
% |latex -jobname cdocscl2 \|\\
% |  "\def\version{final}\input{childdoc.def}\childdocforward{cdocsch2}"|
% \end{tabular}
% \end{center}
% Note that the trailing backslash on each first line
% merely continues the input to the second line
% (for convenient cut ant paste).
% Furthermore, the command |latex| can be replaced by any
% of its alternative versions such as |pdflatex|.
%
% %%%%%%%%%%%%%%%%%%%%%%%%%%%%%%%%%%%%%%%%%%%%%%%%%%%%%%%%%%%%%%%%%%%%%%%%%%%%%%
% %%%%%%%%%%%%%%%%%%%%%%%%%%%%%%%%%%%%%%%%%%%%%%%%%%%%%%%%%%%%%%%%%%%%%%%%%%%%%%
% \section{Implementation}
%\iffalse
%<*package>
%\fi
%
% This section describes the definitions file |childdoc.def|.

% The definitions cannot be loaded using |\usepackage| or |\RequirePackage|
% which has a mechanism to prevent loading a style file more than once.
% When loading the definitions by means of |\input|
% multiple instances have to be prevented manually:
%\iffalse
%This code needs to be before the `\ProvidesFile' directive
%which is defined at the beginning of this file.
%Therefore it is also placed there and commented out here.
%</package>
%<*discard>
%\fi
%    \begin{macrocode}
\ifdefined\childdocmain\endinput\fi
%    \end{macrocode}
%\iffalse
%</discard>
%<*package>
%\fi
%
% \macro{\ifchilddoc}
% \macro{\ifchilddocmanual}
% The conditional |\ifchilddoc| tells whether a
% child (true) or main (false) document is being compiled.
% The conditional |\ifchilddocmanual| tells whether
% the |\includeonly| mechanism is used (false) or
% the selection of child files must be performed manually (true).
% The definitions initialise to false:
%    \begin{macrocode}
\newif\ifchilddoc
\newif\ifchilddocmanual
%    \end{macrocode}

% \macro{\childdocname}
% \macro{\childdocjob}
% The macro |\childdocname| stores the name of the main document
% to be compiled. The macro |\childdocjob| stores the name of
% the document on which the \LaTeX{} compiler was originally invoked.
% The content of |\jobname| cannot be compared
% to filenames specified in the source due to different catcodes.
% The following code rescans |\jobname|, stores the result
% in |\childdocname| and saves a copy in |\childdocjob|:
%    \begin{macrocode}
\edef\childdocname{\scantokens\expandafter{\jobname\noexpand}}
\let\childdocjob\childdocname
%    \end{macrocode}

% \macro{\childdocdisable}
% The macro |\childdocdisable| prevents the main file
% from being processed more than once.
% At this stage, the main document command |\childdocmain|
% is assumed to be called once again where it should do nothing.
% Any subsequent call to it should prevent
% a secondary processing of the main document
% It overwrites the forwarding commands
% |\childdocof| and |\childdocforward|
% with empty macros to prevent further inclusions of the main document:
%    \begin{macrocode}
\newcommand{\childdocdisable}
{
  \renewcommand{\childdocmain}[1]{\renewcommand{\childdocmain}[1]{\endinput}}
  \renewcommand{\childdocof}[1]{}
  \renewcommand{\childdocby}[2][]{}
  \renewcommand{\childdocforward}[2][]{}
  \renewcommand{\childdocdisable}{}
}
%    \end{macrocode}

% \macro{\childdocmain}
% The macro |\childdocmain| is to be called at the top of the main file
% with nothing or the main filename (without extension) as argument.
% First, it breaks loops.
% If the argument is not empty and does not match |\childdocname|
% (which is set by the first inclusion of |childdoc.def|),
% |\ifchilddoc| is set to true, |\includeonly| is applied to the child file
% and |\jobname| is set to the main file
% (for proper handling of |.aux| files):
%    \begin{macrocode}
\newcommand{\childdocmain}[1]
{
  \childdocdisable\childdocmain{}
  \if?#1?\else
    \begingroup
      \def\childdoctmp{#1}
      \ifx\childdoctmp\childdocname
        \def\childdoctmp{}
      \else
        \def\childdoctmp
        {
          \childdoctrue
          \includeonly{\childdocname}
          \def\childdocjob{#1}
          \def\jobname{#1}
        }
      \fi
      \expandafter
    \endgroup
    \childdoctmp
  \fi
}
%    \end{macrocode}

% \macro{\childdocof}
% The command |\childdocof| redirects
% compilation to the main file |#1|.
%    \begin{macrocode}
\newcommand{\childdocof}[1]
{
  \childdocdisable
  \childdoctrue
  \includeonly{\childdocname}
  \def\jobname{#1}
  \def\childdocjob{#1}
  \input{#1}
}
%    \end{macrocode}

% \macro{\childdocby}
% The command |\childdocby| ....
%    \begin{macrocode}
\newcommand{\childdocby}[2][]
{
  \childdocdisable
  \childdoctrue
  \childdocmanualtrue
  \if?#1?\else
    \def\jobname{#2}
  \fi
  \def\childdocjob{#2}
  \input{#2}
  \endinput
}
%    \end{macrocode}

% \macro{\childdocforward}
% The command |\childdocforward| redirects
% compilation to the main file or
% (if the optional argument is given) a child file.
% Parameters are set as if the main file
% or a child file starting with |\childdocof| was compiled.
% Then compilation is handed over to the main file:
%    \begin{macrocode}
\newcommand{\childdocforward}[2][]
{
  \begingroup
    \if?#1?
      \def\childdoctmp
      {
        \def\childdocname{#2}
        \def\childdocjob{#2}
        \def\jobname{#2}
        \input{#2}
        \endinput
      }
    \else
      \def\childdoctmp
      {
        \childdocdisable
        \def\childdocname{#2}
        \childdoctrue
        \includeonly{#2}
        \def\childdocjob{#1}
        \def\jobname{#1}
        \input{#1}
        \endinput
      }
    \fi
    \expandafter
  \endgroup
  \childdoctmp
}
%    \end{macrocode}

% \macro{\childdocforwardprefix}
% The command |\childdocforwardprefix| redirects
% compilation to the main or a child file by means of a pattern.
% The prefix |#1| in the current filename is replaced by |#2|
% and the suffix of the current filename is kept
% (it is assumed that the filename does not contain the substring `|~~~|'
% which is used as a delimiter).
% Compilation is handed over to the new file by |\childdocforward|:
%    \begin{macrocode}
\newcommand{\childdocforwardprefix}[3][]
{
  \begingroup
    \def\childdocextract #2##1~~~{\def\childdoctmp{\childdocforward[#1]{#3##1}}}
    \expandafter\childdocextract\childdocname~~~
    \expandafter
  \endgroup
  \childdoctmp
}
%    \end{macrocode}

% \macro{\childdoc}
% The deprecated macro |\childdoc| is a legacy version of |\childdocmain|:
%    \begin{macrocode}
\newcommand{\childdoc}{\childdocmain}
%    \end{macrocode}

% \macro{\childdocredirect}
% The deprecated macro |\childdocredirect| is a legacy version
% of |\childdocforward| and |\childdocforwardprefix|:
%    \begin{macrocode}
\newcommand{\childdocredirect}[2][]
{
  \begingroup
    \if?#1?
      \def\childdoctmp{\childdocforward{#2}}
    \else
      \def\childdoctmp{\childdocforwardprefix{#1}{#2}}
    \fi
    \expandafter
  \endgroup
  \childdoctmp
}
%    \end{macrocode}

%\iffalse
%</package>
%\fi
%
\endinput
|\\
|\childdocby{|\textit{main}|}|\\
\end{tabular}
\end{center}
%
The directive |\childdocby| is similar to |\childdocof|
described in \secref{sec:include},
but the subsequent selection of content must be done manually.
To that end, both |\ifchilddoc| and |\ifchilddocmanual|
will be true upon processing of a part,
and the name of the part is stored in |\childdocname|.
Note that |\jobname| will be set to the filename of the current part
so that each part receives an individual |.aux| file
that does not interfere with the |.aux| file(s) of the main document.
This behaviour can be altered by the alternative form
|\childdocby[*]{|\textit{main}|}| (with a non-empty optional argument)
which uses the |.aux| file of the main document
by setting |\jobname| to \textit{main}.

%%%%%%%%%%%%%%%%%%%%%%%%%%%%%%%%%%%%%%%%%%%%%%%%%%%%%%%%%%%%%%%%%%%%%%%%%%%%%%%%
\subsection{Driver Development}
\label{sec:driver}

The \textsf{childdoc} mechanism can also be use for the development
of definition files such as \LaTeX{} styles or classes.
This case differs from the above setup with multiple parts
included by |\include| in that no |\includeonly| should be invoked.
This can be achieved by starting the include file
(before |\ProvidesPackage|) with:
%
\begin{center}
\begin{tabular}{l}
|% \iffalse
%
% childdoc.dtx Copyright (C) 2017-2018 Niklas Beisert
%
% This work may be distributed and/or modified under the
% conditions of the LaTeX Project Public License, either version 1.3
% of this license or (at your option) any later version.
% The latest version of this license is in
%   http://www.latex-project.org/lppl.txt
% and version 1.3 or later is part of all distributions of LaTeX
% version 2005/12/01 or later.
%
% This work has the LPPL maintenance status `maintained'.
%
% The Current Maintainer of this work is Niklas Beisert.
%
% This work consists of the files childdoc.dtx and childdoc.ins
% and the derived files childdoc.def and cdocsamp.tex with
% cdocsch1.tex, cdocsch2.tex, cdocsdrf.tex, cdocsfn1.tex, cdocsfn2.tex.
%
%<package>\ifdefined\childdocmain\endinput\fi
%<package>\ProvidesFile{childdoc.def}[2018/12/30 v2.0 child document driver]
%<samplemain>\ProvidesFile{cdocsamp.tex}[2018/12/30 v2.0 sample for childdoc]
%<*driver>
%\ProvidesFile{childdoc.drv}[2018/12/30 v2.0 childdoc reference manual file]
\PassOptionsToClass{10pt,a4paper}{article}
\documentclass{ltxdoc}

\usepackage[margin=35mm]{geometry}
\usepackage{hyperref}
\usepackage{hyperxmp}
\usepackage[usenames]{color}

\hypersetup{colorlinks=true}
\hypersetup{pdfstartview=FitH}
\hypersetup{pdfpagemode=UseNone}
\hypersetup{pdfsource={}}
\hypersetup{pdflang={en-UK}}
\hypersetup{pdfcopyright={Copyright 2017-2018 Niklas Beisert.
  This work may be distributed and/or modified under the
  conditions of the LaTeX Project Public License, either version 1.3
  of this license or (at your option) any later version.}}
\hypersetup{pdflicenseurl={http://www.latex-project.org/lppl.txt}}
\hypersetup{pdfcontactaddress={ETH Zurich, ITP, HIT K,
  Wolfgang-Pauli-Strasse 27}}
\hypersetup{pdfcontactpostcode={8093}}
\hypersetup{pdfcontactcity={Zurich}}
\hypersetup{pdfcontactcountry={Switzerland}}
\hypersetup{pdfcontactemail={nbeisert@itp.phys.ethz.ch}}
\hypersetup{pdfcontacturl={http://people.phys.ethz.ch/\xmptilde nbeisert/}}

\newcommand{\secref}[1]{\hyperref[#1]{section \ref*{#1}}}

\parskip1ex
\parindent0pt
\let\olditemize\itemize
\def\itemize{\olditemize\parskip0pt}

\begin{document}

\title{The \textsf{childdoc} Package}
\hypersetup{pdftitle={The childdoc Package}}
\author{Niklas Beisert\\[2ex]
  Institut f\"ur Theoretische Physik\\
  Eidgen\"ossische Technische Hochschule Z\"urich\\
  Wolfgang-Pauli-Strasse 27, 8093 Z\"urich, Switzerland\\[1ex]
  \href{mailto:nbeisert@itp.phys.ethz.ch}
  {\texttt{nbeisert@itp.phys.ethz.ch}}}
\hypersetup{pdfauthor={Niklas Beisert}}
\hypersetup{pdfsubject={Manual for the LaTeX2e Package childdoc}}
\date{30 December 2018, \textsf{v2.0}}
\maketitle

\begin{abstract}\noindent
\textsf{childdoc} is a \LaTeXe{} package
that enables the direct compilation
of document sections included by |\include|
to individual files.
\end{abstract}

\begingroup
\parskip0ex
\tableofcontents
\endgroup

%%%%%%%%%%%%%%%%%%%%%%%%%%%%%%%%%%%%%%%%%%%%%%%%%%%%%%%%%%%%%%%%%%%%%%%%%%%%%%%%
%%%%%%%%%%%%%%%%%%%%%%%%%%%%%%%%%%%%%%%%%%%%%%%%%%%%%%%%%%%%%%%%%%%%%%%%%%%%%%%%
\section{Introduction}

\LaTeX{} provides a mechanism to structure a large document (such as a book)
into a main file and several child files (containing the chapters)
using the |\include| command.
This mechanism is beneficial for documents
which span hundreds of pages in order to
make the source file(s) more manageable.
Moreover, compilation can be restricted to
selected child files by means of the |\includeonly| command.
The latter feature can be used to reduce the compilation time while editing
(this was significantly more useful in the earlier days of \LaTeX{})
or to generate a smaller document which is easier to navigate.
Another application of |\includeonly| is to generate
documents consisting of selected parts of the complete document.

However, there are a few drawbacks of the plain |\include| mechanism:
\begin{itemize}
\item
The child files cannot be compiled on their own,
they can only be compiled via the main file.
A naive editing environment
(such as a text editor with an option
to have the current file processed by \LaTeX)
may require one to switch to the main file before compiling;
attempting to compile the child file produces errors.
\item
The main file must be modified (each time)
to adjust the |\includeonly| command
to the present needs. This easily leaves the main file in a messy state.
\item
The generated document will always carry the filename
of the main document. This is inconvenient if
several child files are to be compiled and
to be kept for distribution.
\end{itemize}

The present package provides a simple interface
to make child files individually compilable by \LaTeX{}.
Compiling a child file then has the same effect as compiling
the main file with an |\includeonly| command
to select the appropriate child.
Moreover the generated document will carry the name of the child
rather than the main file.
This resolves all three above issues.

This feature is meant to make the editing of books,
thesis documents and lecture notes somewhat more convenient.
However, the package can also be used efficiently for
composing a series of documents (such as exercise sheets)
which are typically distributed individually.
It then assists the author in generating the individual documents
(potentially in different versions)
as well as a document containing the collected series.
Another application is in developing style files
or other kinds of included material
where compilation of the style file could redirect
to a sample or test file.

%%%%%%%%%%%%%%%%%%%%%%%%%%%%%%%%%%%%%%%%%%%%%%%%%%%%%%%%%%%%%%%%%%%%%%%%%%%%%%%%
%%%%%%%%%%%%%%%%%%%%%%%%%%%%%%%%%%%%%%%%%%%%%%%%%%%%%%%%%%%%%%%%%%%%%%%%%%%%%%%%
\section{Usage}

First of all, the package \textsf{childdoc} is \emph{not} a standard
\LaTeXe{} |.sty| style file! Therefore it needs to be invoked in
a non-standard way.

%%%%%%%%%%%%%%%%%%%%%%%%%%%%%%%%%%%%%%%%%%%%%%%%%%%%%%%%%%%%%%%%%%%%%%%%%%%%%%%%
\subsection{Included Files}
\label{sec:include}

%%%%%%%%%%%%%%%%%%%%%%%%%%%%%%%%%%%%%%%%
\DescribeMacro{\childdocmain}
To use the package, add the commands
\begin{center}
\begin{tabular}{l}
|\input{childdoc.def}|\\
|\childdocmain{}|\\
\end{tabular}
\end{center}
at the very top of the main \LaTeX{} file,
in particular \emph{before} the |\documentclass| statement!
The argument of |\childdocmain| should be left empty
(but it must be present).

%%%%%%%%%%%%%%%%%%%%%%%%%%%%%%%%%%%%%%%%
\DescribeMacro{\childdocof}
Furthermore, add the commands
\begin{center}
\begin{tabular}{l}
|\input{childdoc.def}|\\
|\childdocof{|\textit{main}|}|\\
\end{tabular}
\end{center}
at the top of every child file \textit{child}
which is included by |\include{|\textit{child}|}|
from within the main file
(or at least for those files to be compiled individually).
The argument \textit{main} must be the filename of the main file.

There are a couple of
considerations in setting up the main and child documents:

%%%%%%%%%%%%%%%%%%%%%%%%%%%%%%%%%%%%%%%%
\paragraph{Restrictions.}

Please note the following restrictions:
\begin{itemize}
\item
|\childdocmain| must be called with one argument \textit{main}
to ensure compatibility with earlier version of the package.
It must either be empty (|\childdocmain{}|)
or precisely match the filename of the main file in which it is specified.
See \secref{sec:detection} for further information.
\item
The filename \textit{main} must be specified without the |.tex| extension.
\item
The filename \textit{main} is case sensitive
(even in case-insensitive file systems)
due to internal string comparison.
\item
The argument \textit{main} should be fully expanded, it cannot be a macro.
\item
Subdirectories and special characters should be avoided in filenames.
\item
The command |\childdocmain{|\textit{main}|}| must be followed by a whitespace.
It should not be followed immediately by another command
or by a comment mark `|%|'.
This is because the \TeX{} parser reads the token immediately following
the argument of |\childdocmain| and puts it
at the beginning of every child section;
however, a white\-space is ignored.
\end{itemize}

%%%%%%%%%%%%%%%%%%%%%%%%%%%%%%%%%%%%%%%%
\paragraph{Content of Main File.}

It is advisable to place all content in the child files included by |\include|.
Any output contained in the main file will appear in all child documents
unless suppressed manually;
it cannot be suppressed automatically by the |\includeonly| directive
and thus should normally be avoided.
A method to include some content in the main file
by means of conditional processing is described in \secref{sec:conditional}.

%%%%%%%%%%%%%%%%%%%%%%%%%%%%%%%%%%%%%%%%
\paragraph{Page Numbering.}

When only a part of the document is compiled,
the appropriate numbering of pages
(as well as other status parameters)
is determined from the |.aux| files.
The latter contain information from previous passes.
However this information needs to propagate through
all intermediate child documents.
Therefore the page numbering in child documents may well
be inconsistent until the complete document is compiled at least once.

A useful (if unconventional) way to always ensure a consistent
page numbering is to restart the numbering in each child document
and denote the pages by `\textit{child}|.|\textit{page}'
where \textit{child} represents the chapter/section number of the child file.
This can be achieved by the command
|\numberwithin{page}{|\textit{child}|}|
of the \textsf{amsmath} package
where \textit{child} can be |chapter| or |section|
depending on the chosen structuring.
Alternatively, one can modify the macro |\thepage| appropriately
and reset the counter |page| at the start of each child file.

%%%%%%%%%%%%%%%%%%%%%%%%%%%%%%%%%%%%%%%%%%%%%%%%%%%%%%%%%%%%%%%%%%%%%%%%%%%%%%%%
\subsection{Conditional Processing}
\label{sec:conditional}

The package provides a mechanism to compile different versions
of a document. To customise the versions further some conditional processing
can come in handy to distinguish which version is being compiled.
The package provides two macros to describe the compilation context:

%%%%%%%%%%%%%%%%%%%%%%%%%%%%%%%%%%%%%%%%
\DescribeMacro{\ifchilddoc}
The conditional |\ifchilddoc| distinguishes between the compilation of
child documents and the main document:
%
\begin{center}
|\ifchilddoc |\textit{child-code}| |[|\||else |\textit{main-code}]| \||fi|
\end{center}

%%%%%%%%%%%%%%%%%%%%%%%%%%%%%%%%%%%%%%%%
\DescribeMacro{\childdocname}
\DescribeMacro{\childdocjob}
The macro |\childdocname| contains the filename (without extension)
of the main or child file being processed.
Note that |\childdocjob| will always contain the name of the main file.

%%%%%%%%%%%%%%%%%%%%%%%%%%%%%%%%%%%%%%%%
\paragraph{Title Page.}

Conditional processing can be used to include a title or banner page
in the main document when proper precautions are taken.
Importantly, the code in the main file should ensure that the page counter
(as well as other status parameters which are stored in the |.aux| files)
takes the same value after the conditional processing.
Otherwise the page numbers may take divergent values
depending on which part is compiled.

For example, a title page could be declared by:
%
\begin{center}
\begin{tabular}{l}
|\ifchilddoc\||else|\\
|\addtocounter{page}{-1}|\\
\textit{code for title page}\\
|\newpage|\\
|\||fi|
\end{tabular}
\end{center}
%
A banner page for the child documents can be generated by:
%
\begin{center}
\begin{tabular}{l}
|\ifchilddoc|\\
|\addtocounter{page}{-1}|\\
\textit{code for banner page}\\
|\newpage|\\
|\||fi|
\end{tabular}
\end{center}
%
Here one could write a message such as:
\begin{center}
|This is the part \childdocname{} of \childdocjob{}.|
\end{center}

%%%%%%%%%%%%%%%%%%%%%%%%%%%%%%%%%%%%%%%%%%%%%%%%%%%%%%%%%%%%%%%%%%%%%%%%%%%%%%%%
\subsection{Flags}
\label{sec:flags}

The package makes it easy to generate different versions
of the main or child documents.
To this end compilation flags can be defined
and assigned different default values.
They will be particularly useful in conjunction
with the forwarding mechanism described in \secref{sec:forward}.

For example, it may be useful to have a flag |\version|
which can be set to |draft| or |final|.
The document source will contain some conditional code
depending on the value of |\version|.
Suppose further, the flag should default to |final| for the main file
and to |draft| for child files
which is a natural assignment for editing the document.
This is achieved by placing the following code
in the preamble of the main document
(below the |\childdocmain| directive):
%
\begin{center}
\begin{tabular}{l}
|\ifchilddoc|\\
|\providecommand{\version}{draft}|\\
|\||else|\\
|\providecommand{\version}{final}|\\
|\||fi|
\end{tabular}
\end{center}
%
The definition by |\providecommand| makes sure
that previous definitions are not overwritten.
Further statements |\providecommand{\version}{...}|
can thus be added before the above code to override it.

For the main file, one might add a line
(between |\childdocmain| and the above block)
%
\begin{center}
|%\ifchilddoc\||else\providecommand{\version}{draft}\||fi|
\end{center}
%
which can be uncommented to produce a draft version.
Likewise one can add a line to the very top of a child file
(above the |\childdocof{|\textit{main}|}| directive)
%
\begin{center}
|%\providecommand{\version}{final}|
\end{center}
%
which can be uncommented to produce the final version of this child document.

%%%%%%%%%%%%%%%%%%%%%%%%%%%%%%%%%%%%%%%%%%%%%%%%%%%%%%%%%%%%%%%%%%%%%%%%%%%%%%%%
\subsection{Forwarding}
\label{sec:forward}

Different versions of the main or child documents
using compilation flags as described in \secref{sec:flags}
can be (permanently) stored in different files
for convenient compilation, viewing and distribution.
To this end, the package defines a command
to pass on compilation to a different file:

%%%%%%%%%%%%%%%%%%%%%%%%%%%%%%%%%%%%%%%%
\DescribeMacro{\childdocforward}
The command |\childdocforward| redirects processing to
another source file:
%
\begin{center}
\begin{tabular}{l}
|\input{childdoc.def}|\\
|\childdocforward[|\textit{main}|]{|\textit{dest}|}|\\
\end{tabular}
\end{center}
%
The argument \textit{dest} is the destination file
(without extension).
It should be the main file or one of the child files.
Note that further \textsf{childdoc} directives
such as |\childdocof| and |\childdocforward|
in the indicated file will be processed in this form.
The optional argument \textit{main}
passes on directly to the main file \textit{main}
while pretending to compile the child \textit{dest}.
This form behaves as if \textit{dest}
issues |\childdocof{|\textit{main}|}| right away,
and no further \textsf{childdoc} directives will be processed.

%%%%%%%%%%%%%%%%%%%%%%%%%%%%%%%%%%%%%%%%
\DescribeMacro{\...prefix}
In the alternative form |\childdocforwardprefix|,
%
\begin{center}
\begin{tabular}{l}
|\input{childdoc.def}|\\
|\childdocforwardprefix[|\textit{main}|]{|\textit{prefix}|}{|\textit{dest}|}|
\end{tabular}
\end{center}
%
the destination file is determined by a pattern
depending on the current file:
To make this work, the current file must be called
`{\textit{prefix}\hspace{0.2em}\textit{suffix}}'
with \textit{prefix} matching precisely the argument.
Processing is then passed on to the file
`{\textit{dest}\hspace{0.2em}\textit{suffix}}'.
Surely, the same effect is achieved by
directly specifying the
argument `{\textit{dest}\hspace{0.2em}\textit{suffix}}'
in the first form.
However, that requires to set up a different file
for each child. With the alternative form of the command
all these files can have exactly the same content
which simplifies setting them up and maintaining them.

For example, the following file |draft.tex|
with a compilation flag |\version| as described in \secref{sec:flags}
compiles the main document as a draft:
%
\begin{center}
\begin{tabular}{l}
|\def\version{draft}|\\
|\input{childdoc.def}|\\
|\childdocforward{|\textit{main}|}|
\end{tabular}
\end{center}
%
Likewise, the following files |final|\textit{nn}|.tex|
compile the final version of the child document
|child|\textit{nn}|.tex|:
%
\begin{center}
\begin{tabular}{l}
|\def\version{final}|\\
|\input{childdoc.def}|\\
|\childdocforwardprefix{final}{child}|
\end{tabular}
\end{center}
%

Note that when several versions of a main file and/or of each child file
are to be generated, it may be convenient to set up a |Makefile| or
shell script to automatise the process.

%%%%%%%%%%%%%%%%%%%%%%%%%%%%%%%%%%%%%%%%%%%%%%%%%%%%%%%%%%%%%%%%%%%%%%%%%%%%%%%%
\subsection{Command Line Processing}
\label{sec:commandline}

The effect of redirection files can also be achieved by invoking
the \LaTeX{} compiler with a more elaborate command line.
Most conveniently this should be done as part
of a shell script or a |Makefile|.

When using \textsf{childdoc} in the main file, the following
command lines effectively perform a redirection
(note that depending on the shell being used,
backslashes may have to be doubled: `|\|' $\to$ `|\\|'):
%
\begin{center}
|... -jobname "|\textit{target}|" |\\|"|[\textit{flags}]%
|\input{childdoc.def}\childdocforward[|\textit{main}|]{|\textit{dest}|}"|
\end{center}
%
Here \textit{target} is the name of the output file,
\textit{main} is the name of the main file
and \textit{dest} is the name of the main or child file to be processed
(all filenames without extensions).
The optional argument \textit{main} can be omitted
if \textit{main} matches \textit{dest}.
Optionally, compilation \textit{flags} can be defined via |\def| commands.
This command line makes the \TeX{} engine believe
it is compiling the file \textit{target}
whose content is specified as the latter parameter.
The provided code then forwards the processing to
\textit{main} or \textit{dest} as described in \secref{sec:forward}.

%%%%%%%%%%%%%%%%%%%%%%%%%%%%%%%%%%%%%%%%%%%%%%%%%%%%%%%%%%%%%%%%%%%%%%%%%%%%%%%%
\subsection{Include by Input}
\label{sec:input}

Including child documents by |\include| has some restrictions by design.
Most notably, the content of a child document always occupies
its own set of pages; pages cannot be shared between child documents.
Usually, this behaviour makes perfect sense
because each child document contain an essential part of the document.
However, in some situations it may be desirable to compose
a document from a collection of parts
without having mandatory page breaks between then.
For this case, the package
provides a mechanism to include parts
by |\input| which can also be processed individually.
However, by construction this mechanism
requires manual handling of the content to be output.

%%%%%%%%%%%%%%%%%%%%%%%%%%%%%%%%%%%%%%%%
\DescribeMacro{\ifchilddocmanual}
The main file should be prepared as usual, see \secref{sec:include}.
However, the document body must make a distinction
between processing of an individual part and of the main document, e.g.:
%
\begin{center}
\begin{tabular}{l}
|\ifchilddocmanual|\\
|\input{\childdocname}|\\
|\||else|\\
\textit{document body with }|\input{|\textit{part}|}|\\
|\||fi|
\end{tabular}
\end{center}
%
The conditional |\ifchilddocmanual| is true whenever
a part to be included by |\input| is being compiled,
and the name of the part is stored in |\childdocname|.

%%%%%%%%%%%%%%%%%%%%%%%%%%%%%%%%%%%%%%%%
\DescribeMacro{\childdocby}
Each part to be included by |\input| should start with:
%
\begin{center}
\begin{tabular}{l}
|\input{childdoc.def}|\\
|\childdocby{|\textit{main}|}|\\
\end{tabular}
\end{center}
%
The directive |\childdocby| is similar to |\childdocof|
described in \secref{sec:include},
but the subsequent selection of content must be done manually.
To that end, both |\ifchilddoc| and |\ifchilddocmanual|
will be true upon processing of a part,
and the name of the part is stored in |\childdocname|.
Note that |\jobname| will be set to the filename of the current part
so that each part receives an individual |.aux| file
that does not interfere with the |.aux| file(s) of the main document.
This behaviour can be altered by the alternative form
|\childdocby[*]{|\textit{main}|}| (with a non-empty optional argument)
which uses the |.aux| file of the main document
by setting |\jobname| to \textit{main}.

%%%%%%%%%%%%%%%%%%%%%%%%%%%%%%%%%%%%%%%%%%%%%%%%%%%%%%%%%%%%%%%%%%%%%%%%%%%%%%%%
\subsection{Driver Development}
\label{sec:driver}

The \textsf{childdoc} mechanism can also be use for the development
of definition files such as \LaTeX{} styles or classes.
This case differs from the above setup with multiple parts
included by |\include| in that no |\includeonly| should be invoked.
This can be achieved by starting the include file
(before |\ProvidesPackage|) with:
%
\begin{center}
\begin{tabular}{l}
|\input{childdoc.def}|\\
|\childdocforward{|\textit{main}|}|\\
\end{tabular}
\end{center}
%
or alternatively with:
%
\begin{center}
\begin{tabular}{l}
|\input{childdoc.def}|\\
|\childdocby{|\textit{main}|}|\\
\end{tabular}
\end{center}
%
Both forms have slightly different effects as described above.
The main file is prepared as usual, see \secref{sec:include}.

%%%%%%%%%%%%%%%%%%%%%%%%%%%%%%%%%%%%%%%%%%%%%%%%%%%%%%%%%%%%%%%%%%%%%%%%%%%%%%%%
\subsection{Legacy Detection}
\label{sec:detection}

The directive |\childdocmain| in the main file can detect
whether the complete document or merely a child is to be compiled
even without using the directive |\childdocof|.
This method is deprecated because it is less robust
and there is no compelling reason to use it;
it is merely provided for backward compatibility
and it may be removed in future versions.

If the detection mechanism is to be used,
it is mandatory to correctly specify
the filename of the main file as the argument of |\childdocmain|:
%
\begin{center}
\begin{tabular}{l}
|\input{childdoc.def}|\\
|\childdocmain{|\textit{main}|}|\\
\end{tabular}
\end{center}
%
If |\jobname| does not match the argument \textit{main} of |\childdocmain|,
it is assumed that |\jobname| points to the child file to be compiled.
When using |\childdocmain| with the main file specified as argument,
it suffices to start a child file
with just |\input{|\textit{main}|}|
without loading of the package and using |\childdocof|.
If instead all processing is done
with the appropriate \textsf{childdoc} directives,
the argument of \textit{main} of |\childdocmain| can be empty.

An alternative version of the command line processing described
in \secref{sec:commandline} using the detection mechanism reads:
%
\begin{center}
|... -jobname "|\textit{target}|" "|[\textit{flags}]%
[|\def\jobname{|\textit{dest}|}|]|\input{|\textit{main}|}"|
\end{center}

%%%%%%%%%%%%%%%%%%%%%%%%%%%%%%%%%%%%%%%%%%%%%%%%%%%%%%%%%%%%%%%%%%%%%%%%%%%%%%%%
\subsection{Manual Code}
\label{sec:manual}

In case one cannot be certain whether the definitions file |childdoc.def|
is installed on the target \TeX{} distribution
and one prefers not to ship it,
it is conceivable to paste a few relevant commands into the sources.

To that end, drop all statements |\input{childdoc.def}|
and perform the replacements as outlined below.
Instead of |\childdocmain{|\textit{main}|}| add the following code
to the top of the main file:
%
\begin{center}
\begin{tabular}{l}
|\||ifdefined\childdocname\endinput\||fi\newif\ifchilddoc|\\
|\edef\childdocname{\scantokens\expandafter{\jobname\noexpand}}|\\
|\def\childdocmain{|\textit{main}|}\||ifx\childdocmain\childdocname\||else|\\
|\childdoctrue\includeonly{\childdocname}\let\jobname\childdocmain\||fi|\\
\end{tabular}
\end{center}
%
Instead of |\childdocof{|\textit{main}|}| just include the main file
at the top of each child file:
%
\begin{center}
|\input{|\textit{main}|}|
\end{center}
%
A simple redirection |\childdocforward{|\textit{dest}|}| is achieved by:
%
\begin{center}
|\def\jobname{|\textit{dest}|}\input{\jobname}|
\end{center}
%
The redirection with prefix
|\childdocforwardprefix[|\textit{prefix}|]{|\textit{dest}|}|
is accomplished by:
%
\begin{center}
\begin{tabular}{l}
|{\edef\jobname{\scantokens\expandafter{\jobname\noexpand}}|\\
|\def\redirectjob |\textit{prefix}|#1~~~{\gdef\jobname{|\textit{dest}|#1}}|\\
|\expandafter\redirectjob\jobname~~~}\input{\jobname}|
\end{tabular}
\end{center}

In an alternative approach,
child documents can be compiled by a specific command line
without additional code or specific definitions:
%
\begin{center}
|... -jobname "|\textit{target}|" "|[\textit{flags}]%
|\includeonly{|\textit{dest}|}\input{|\textit{main}|}"|
\end{center}
%

%%%%%%%%%%%%%%%%%%%%%%%%%%%%%%%%%%%%%%%%%%%%%%%%%%%%%%%%%%%%%%%%%%%%%%%%%%%%%%%%
%%%%%%%%%%%%%%%%%%%%%%%%%%%%%%%%%%%%%%%%%%%%%%%%%%%%%%%%%%%%%%%%%%%%%%%%%%%%%%%%
\section{Information}

%%%%%%%%%%%%%%%%%%%%%%%%%%%%%%%%%%%%%%%%%%%%%%%%%%%%%%%%%%%%%%%%%%%%%%%%%%%%%%%%
\subsection{Copyright}

Copyright \copyright{} 2017--2018 Niklas Beisert

This work may be distributed and/or modified under the
conditions of the \LaTeX{} Project Public License, either version 1.3
of this license or (at your option) any later version.
The latest version of this license is in
  \url{http://www.latex-project.org/lppl.txt}
and version 1.3 or later is part of all distributions of \LaTeX{}
version 2005/12/01 or later.

This work has the LPPL maintenance status `maintained'.

The Current Maintainer of this work is Niklas Beisert.

This work consists of the files |README.txt|, |childdoc.ins| and |childdoc.dtx|
as well as the derived files |childdoc.def|, |cdocsamp.tex|
with |cdocsch1.tex|, |cdocsch2.tex|, |cdocspt3.tex|, |cdocspt4.tex|,
|cdocsdrf.tex|, |cdocsfn1.tex|, |cdocsfn2.tex|
as well as |childdoc.pdf|.

%%%%%%%%%%%%%%%%%%%%%%%%%%%%%%%%%%%%%%%%%%%%%%%%%%%%%%%%%%%%%%%%%%%%%%%%%%%%%%%%
\subsection{Files and Installation}

The package consists of the files:
%
\begin{center}
\begin{tabular}{ll}
    |README.txt|   & readme file \\
    |childdoc.ins| & installation file \\
    |childdoc.dtx| & source file \\
    |childdoc.def| & definition file \\
    |cdocsamp.tex| & sample main file \\
    |cdocsch1.tex| & sample include file \\
    |cdocsch2.tex| & sample include file \\
    |cdocspt3.tex| & sample part file \\
    |cdocspt4.tex| & sample part file \\
    |cdocsdrf.tex| & sample redirection file \\
    |cdocsfn1.tex| & sample redirection file \\
    |cdocsfn2.tex| & sample redirection file \\
    |childdoc.pdf| & manual
\end{tabular}
\end{center}
%
The distribution consists of the files
|README.txt|, |childdoc.ins| and |childdoc.dtx|.
%
\begin{itemize}
\item
Run (pdf)\LaTeX{} on |childdoc.dtx|
to compile the manual |childdoc.pdf| (this file).
\item
Run \LaTeX{} on |childdoc.ins| to create the definitions file |childdoc.def|
and the sample |cdocsamp.tex| with include files
|cdocsch1.tex|, |cdocsch2.tex|, |cdocspt3.tex|, |cdocspt4.tex|,
|cdocsdrf.tex|, |cdocsfn1.tex|, |cdocsfn2.tex|.
Then copy the file |childdoc.def| to an appropriate directory of your \LaTeX{}
distribution, e.g.\ \textit{texmf-root}|/tex/latex/childdoc|.
\end{itemize}

%%%%%%%%%%%%%%%%%%%%%%%%%%%%%%%%%%%%%%%%%%%%%%%%%%%%%%%%%%%%%%%%%%%%%%%%%%%%%%%%
\subsection{Related CTAN Packages}

There are several other packages which offer a similar functionality:
%
\begin{itemize}
\item
The packages
\href{http://ctan.org/pkg/docmute}{\textsf{docmute}},
\href{http://ctan.org/pkg/includex}{\textsf{includex}} and
\href{http://ctan.org/pkg/standalone}{\textsf{standalone}}
provide commands to include only the document body of
a child file thus allowing both files to be compiled individually.
\item
The packages \href{http://ctan.org/pkg/subdocs}{\textsf{subdocs}}
and \href{http://ctan.org/pkg/subfiles}{\textsf{subfiles}}
provide structures in which the main and child documents can be
encapsulated and allowing them to be compiled individually.
The inclusion mechanism is different from the conventional |\include|.
\item
The package \href{http://ctan.org/pkg/combine}{\textsf{combine}}
is an elaborate solution to combine several documents into one.
\end{itemize}
%
See also the CTAN topic \href{http://ctan.org/topic/subdocs}{\textsf{subdocs}}
for further related packages.
The present package differs from the above solutions in that
a document structure constructed with the conventional |\include| mechanism
just needs two extra commands at the top of every file
such that all constituent files can be compiled individually.

%%%%%%%%%%%%%%%%%%%%%%%%%%%%%%%%%%%%%%%%%%%%%%%%%%%%%%%%%%%%%%%%%%%%%%%%%%%%%%%%
%\subsection{Feature Suggestions}
%
%The following is a list of features which may be useful for future
%versions of this package:
%%
%\begin{itemize}
%\item
%\ldots
%\end{itemize}

%%%%%%%%%%%%%%%%%%%%%%%%%%%%%%%%%%%%%%%%%%%%%%%%%%%%%%%%%%%%%%%%%%%%%%%%%%%%%%%%
\subsection{Revision History}

%%%%%%%%%%%%%%%%%%%%%%%%%%%%%%%%%%%%%%%%
\paragraph{v2.0:} 2018/12/30

\begin{itemize}
\item
immediate forward processing
\item
added |\childdocby| mechanism
\item
manual restructured
\end{itemize}

%%%%%%%%%%%%%%%%%%%%%%%%%%%%%%%%%%%%%%%%
\paragraph{v1.6:} 2018/01/17

\begin{itemize}
\item
application for development of include files
\item
corrections to manual
\end{itemize}

%%%%%%%%%%%%%%%%%%%%%%%%%%%%%%%%%%%%%%%%
\paragraph{v1.5:} 2017/05/21

\begin{itemize}
\item
more complete structuring introduced
\item
|\childdocof| introduced
\item
|\childdoc| renamed to |\childdocmain|
\item
|\childredirect| renamed to |\childdocforward| and |\childdocforwardprefix|
and functionality expanded
\end{itemize}

%%%%%%%%%%%%%%%%%%%%%%%%%%%%%%%%%%%%%%%%
\paragraph{v1.0:} 2017/04/27

\begin{itemize}
\item
manual and install package
\item
first version published on CTAN
\end{itemize}

%%%%%%%%%%%%%%%%%%%%%%%%%%%%%%%%%%%%%%%%
\paragraph{v0.6:} 2017/04/26

\begin{itemize}
\item
redirection mechanism added
\end{itemize}

%%%%%%%%%%%%%%%%%%%%%%%%%%%%%%%%%%%%%%%%
\paragraph{v0.5:} 2017/04/26

\begin{itemize}
\item
functionality in definition file
\end{itemize}


%%%%%%%%%%%%%%%%%%%%%%%%%%%%%%%%%%%%%%%%%%%%%%%%%%%%%%%%%%%%%%%%%%%%%%%%%%%%%%%%
%%%%%%%%%%%%%%%%%%%%%%%%%%%%%%%%%%%%%%%%%%%%%%%%%%%%%%%%%%%%%%%%%%%%%%%%%%%%%%%%
%%%%%%%%%%%%%%%%%%%%%%%%%%%%%%%%%%%%%%%%%%%%%%%%%%%%%%%%%%%%%%%%%%%%%%%%%%%%%%%%
\appendix

\settowidth\MacroIndent{\rmfamily\scriptsize 000\ }

 \DocInput{childdoc.dtx}

\end{document}
%</driver>
% \fi
%
% %%%%%%%%%%%%%%%%%%%%%%%%%%%%%%%%%%%%%%%%%%%%%%%%%%%%%%%%%%%%%%%%%%%%%%%%%%%%%%
% %%%%%%%%%%%%%%%%%%%%%%%%%%%%%%%%%%%%%%%%%%%%%%%%%%%%%%%%%%%%%%%%%%%%%%%%%%%%%%
% \section{Sample}
%\iffalse
%<*samplemain>
%\fi
%
% The following presents a sample document
% with two chapters, two parts, a title page,
% a compile flag as well as three forwarding files to set the flag.
% It consists of eight |.tex| files:
% \begin{center}
% \begin{tabular}{ll}
% |cdocsamp.tex|&main file\\
% |cdocsch1.tex|&include file for chapter 1\\
% |cdocsch2.tex|&include file for chapter 2\\
% |cdocspt3.tex|&include file for part 3\\
% |cdocspt4.tex|&include file for part 4\\
% |cdocsdrf.tex|&forwarding file for main file in draft mode\\
% |cdocsfi1.tex|&forwarding file for final version of chapter 1\\
% |cdocsfi2.tex|&forwarding file for final version of chapter 2\\
% \end{tabular}
% \end{center}
% Each of the eight files can be compiled directly by the \LaTeX{} compiler.
%
% %%%%%%%%%%%%%%%%%%%%%%%%%%%%%%%%%%%%%%
% \paragraph{Main File.}
%
% The main file is called |cdocsamp.tex|.
%
% Load the \textsf{childdoc} definitions and
% declare the filename for the main document:
%    \begin{macrocode}
\input{childdoc.def}
\childdocmain{}
%    \end{macrocode}

% Optional override for |\version| flag:
%    \begin{macrocode}
%%\ifchilddoc\else\providecommand{\version}{draft}\fi
%    \end{macrocode}

% Define the default values for the |\version| flag
% (|final| for the main file and |draft| for childs):
%    \begin{macrocode}
\ifchilddoc
\providecommand{\version}{draft}
\else
\providecommand{\version}{final}
\fi
%    \end{macrocode}

% Load the standard document class:
%    \begin{macrocode}
\documentclass[12pt]{article}
%    \end{macrocode}

% Start the document body:
%    \begin{macrocode}
\begin{document}
%    \end{macrocode}

% Declare a title page.
% Print title, part of document being processed and version flag:
%    \begin{macrocode}
\addtocounter{page}{-1}
\begin{center}
{\LARGE\bfseries{}childdoc example\par}
\vspace{1cm}
\ifchilddoc
\ifchilddocmanual part\else chapter\fi:
`\childdocname' of `\childdocjob'\par
\else
main document: `\childdocjob'\par
\fi
version: \version\par
\end{center}
\newpage
%    \end{macrocode}

% Manually include selected file,
% otherwise process as usual:
%    \begin{macrocode}
\ifchilddocmanual
\section*{part `\childdocname'}
\input{\childdocname}
\else
%    \end{macrocode}

% Include the two chapters:
%    \begin{macrocode}
\include{cdocsch1}
\include{cdocsch2}
%    \end{macrocode}

% Include the two parts unless only chapters should be displayed:
%    \begin{macrocode}
\ifchilddoc\else
\section{part three}
\input{cdocspt3}
\section{part four}
\input{cdocspt4}
\fi
%    \end{macrocode}

% Process as usual until here:
%    \begin{macrocode}
\fi
%    \end{macrocode}

% End of document body:
%    \begin{macrocode}
\end{document}
%    \end{macrocode}
%\iffalse
%</samplemain>
%\fi
%
% %%%%%%%%%%%%%%%%%%%%%%%%%%%%%%%%%%%%%%
% \paragraph{Chapter Include Files.}
%
% The include files are called |cdocsch1.tex| and |cdocsch2.tex|.
%
%\iffalse
%<*samplechap1|samplechap2>
%\fi

% Optional override for |\version| flag:
%    \begin{macrocode}
%%\providecommand{\version}{final}
%    \end{macrocode}

% Include the main document:
%    \begin{macrocode}
\input{childdoc.def}
\childdocof{cdocsamp}
%    \end{macrocode}

%\iffalse
%</samplechap1|samplechap2>
%\fi
%
%\iffalse
%<*samplechap1>
%\fi
% Some text for chapter 1:
%    \begin{macrocode}
\section{one}
some text in chapter one
%    \end{macrocode}

%\iffalse
%</samplechap1>
%\fi
% Some text for chapter 2:
%\iffalse
%<*samplechap2>
%\fi
%    \begin{macrocode}
\section{two}
more text in chapter two
%    \end{macrocode}

%\iffalse
%</samplechap2>
%\fi
%
% %%%%%%%%%%%%%%%%%%%%%%%%%%%%%%%%%%%%%%
% \paragraph{Part Include Files.}
%
% The include files are called |cdocspt3.tex| and |cdocspt4.tex|.
%
%\iffalse
%<*samplepart3|samplepart4>
%\fi

% Optional override for |\version| flag:
%    \begin{macrocode}
%%\providecommand{\version}{final}
%    \end{macrocode}

% Include the main document:
%    \begin{macrocode}
\input{childdoc.def}
\childdocby{cdocsamp}
%    \end{macrocode}

%\iffalse
%</samplepart3|samplepart4>
%\fi
%
%\iffalse
%<*samplepart3>
%\fi
% Some text for part 3:
%    \begin{macrocode}
some text in part three
%    \end{macrocode}

%\iffalse
%</samplepart3>
%\fi
% Some text for part 4:
%\iffalse
%<*samplepart4>
%\fi
%    \begin{macrocode}
more text in part four
%    \end{macrocode}

%\iffalse
%</samplepart4>
%\fi
%
% %%%%%%%%%%%%%%%%%%%%%%%%%%%%%%%%%%%%%%
% \paragraph{Forwarding for a Complete Draft.}
%
% The following forwarding file |cdocsdrf.tex|
% compiles the main document in draft mode:
%\iffalse
%<*sampledraft>
%\fi
%    \begin{macrocode}
\def\version{draft}
\input{childdoc.def}
\childdocforward{cdocsamp}
%    \end{macrocode}

%\iffalse
%</sampledraft>
%\fi
%
% %%%%%%%%%%%%%%%%%%%%%%%%%%%%%%%%%%%%%%
% \paragraph{Forwarding for Final Version of the Chapters.}
%
% The following forwarding files |cdocsfn1.tex| and |cdocsfn2.tex|
% (with identical content)
% compile the final versions of the child documents
% |cdocsch1.tex| and |cdocsch2.tex|, respectively:
%\iffalse
%<*samplefinal>
%\fi
%    \begin{macrocode}
\def\version{final}
\input{childdoc.def}
\childdocforwardprefix[cdocsamp]{cdocsfn}{cdocsch}
%    \end{macrocode}

%\iffalse
%</samplefinal>
%\fi
%
% %%%%%%%%%%%%%%%%%%%%%%%%%%%%%%%%%%%%%%
% \paragraph{Command Line Processing.}
%
% The following three command lines generate the output files
% |cdocscld|, |cdocscl1| and |cdocscl2|
% which should be identical to
% |cdocsdrf|, |cdocsch1| and |cdocsfn2|, respectively:
% \begin{center}
% \begin{tabular}{l}
% |latex -jobname cdocscld \|\\
% |  "\def\version{draft}\input{childdoc.def}\childdocforward{cdocsamp}"|\\
% |latex -jobname cdocscl1 \|\\
% |  "\input{childdoc.def}\childdocforward[cdocsamp]{cdocsch1}"|\\
% |latex -jobname cdocscl2 \|\\
% |  "\def\version{final}\input{childdoc.def}\childdocforward{cdocsch2}"|
% \end{tabular}
% \end{center}
% Note that the trailing backslash on each first line
% merely continues the input to the second line
% (for convenient cut ant paste).
% Furthermore, the command |latex| can be replaced by any
% of its alternative versions such as |pdflatex|.
%
% %%%%%%%%%%%%%%%%%%%%%%%%%%%%%%%%%%%%%%%%%%%%%%%%%%%%%%%%%%%%%%%%%%%%%%%%%%%%%%
% %%%%%%%%%%%%%%%%%%%%%%%%%%%%%%%%%%%%%%%%%%%%%%%%%%%%%%%%%%%%%%%%%%%%%%%%%%%%%%
% \section{Implementation}
%\iffalse
%<*package>
%\fi
%
% This section describes the definitions file |childdoc.def|.

% The definitions cannot be loaded using |\usepackage| or |\RequirePackage|
% which has a mechanism to prevent loading a style file more than once.
% When loading the definitions by means of |\input|
% multiple instances have to be prevented manually:
%\iffalse
%This code needs to be before the `\ProvidesFile' directive
%which is defined at the beginning of this file.
%Therefore it is also placed there and commented out here.
%</package>
%<*discard>
%\fi
%    \begin{macrocode}
\ifdefined\childdocmain\endinput\fi
%    \end{macrocode}
%\iffalse
%</discard>
%<*package>
%\fi
%
% \macro{\ifchilddoc}
% \macro{\ifchilddocmanual}
% The conditional |\ifchilddoc| tells whether a
% child (true) or main (false) document is being compiled.
% The conditional |\ifchilddocmanual| tells whether
% the |\includeonly| mechanism is used (false) or
% the selection of child files must be performed manually (true).
% The definitions initialise to false:
%    \begin{macrocode}
\newif\ifchilddoc
\newif\ifchilddocmanual
%    \end{macrocode}

% \macro{\childdocname}
% \macro{\childdocjob}
% The macro |\childdocname| stores the name of the main document
% to be compiled. The macro |\childdocjob| stores the name of
% the document on which the \LaTeX{} compiler was originally invoked.
% The content of |\jobname| cannot be compared
% to filenames specified in the source due to different catcodes.
% The following code rescans |\jobname|, stores the result
% in |\childdocname| and saves a copy in |\childdocjob|:
%    \begin{macrocode}
\edef\childdocname{\scantokens\expandafter{\jobname\noexpand}}
\let\childdocjob\childdocname
%    \end{macrocode}

% \macro{\childdocdisable}
% The macro |\childdocdisable| prevents the main file
% from being processed more than once.
% At this stage, the main document command |\childdocmain|
% is assumed to be called once again where it should do nothing.
% Any subsequent call to it should prevent
% a secondary processing of the main document
% It overwrites the forwarding commands
% |\childdocof| and |\childdocforward|
% with empty macros to prevent further inclusions of the main document:
%    \begin{macrocode}
\newcommand{\childdocdisable}
{
  \renewcommand{\childdocmain}[1]{\renewcommand{\childdocmain}[1]{\endinput}}
  \renewcommand{\childdocof}[1]{}
  \renewcommand{\childdocby}[2][]{}
  \renewcommand{\childdocforward}[2][]{}
  \renewcommand{\childdocdisable}{}
}
%    \end{macrocode}

% \macro{\childdocmain}
% The macro |\childdocmain| is to be called at the top of the main file
% with nothing or the main filename (without extension) as argument.
% First, it breaks loops.
% If the argument is not empty and does not match |\childdocname|
% (which is set by the first inclusion of |childdoc.def|),
% |\ifchilddoc| is set to true, |\includeonly| is applied to the child file
% and |\jobname| is set to the main file
% (for proper handling of |.aux| files):
%    \begin{macrocode}
\newcommand{\childdocmain}[1]
{
  \childdocdisable\childdocmain{}
  \if?#1?\else
    \begingroup
      \def\childdoctmp{#1}
      \ifx\childdoctmp\childdocname
        \def\childdoctmp{}
      \else
        \def\childdoctmp
        {
          \childdoctrue
          \includeonly{\childdocname}
          \def\childdocjob{#1}
          \def\jobname{#1}
        }
      \fi
      \expandafter
    \endgroup
    \childdoctmp
  \fi
}
%    \end{macrocode}

% \macro{\childdocof}
% The command |\childdocof| redirects
% compilation to the main file |#1|.
%    \begin{macrocode}
\newcommand{\childdocof}[1]
{
  \childdocdisable
  \childdoctrue
  \includeonly{\childdocname}
  \def\jobname{#1}
  \def\childdocjob{#1}
  \input{#1}
}
%    \end{macrocode}

% \macro{\childdocby}
% The command |\childdocby| ....
%    \begin{macrocode}
\newcommand{\childdocby}[2][]
{
  \childdocdisable
  \childdoctrue
  \childdocmanualtrue
  \if?#1?\else
    \def\jobname{#2}
  \fi
  \def\childdocjob{#2}
  \input{#2}
  \endinput
}
%    \end{macrocode}

% \macro{\childdocforward}
% The command |\childdocforward| redirects
% compilation to the main file or
% (if the optional argument is given) a child file.
% Parameters are set as if the main file
% or a child file starting with |\childdocof| was compiled.
% Then compilation is handed over to the main file:
%    \begin{macrocode}
\newcommand{\childdocforward}[2][]
{
  \begingroup
    \if?#1?
      \def\childdoctmp
      {
        \def\childdocname{#2}
        \def\childdocjob{#2}
        \def\jobname{#2}
        \input{#2}
        \endinput
      }
    \else
      \def\childdoctmp
      {
        \childdocdisable
        \def\childdocname{#2}
        \childdoctrue
        \includeonly{#2}
        \def\childdocjob{#1}
        \def\jobname{#1}
        \input{#1}
        \endinput
      }
    \fi
    \expandafter
  \endgroup
  \childdoctmp
}
%    \end{macrocode}

% \macro{\childdocforwardprefix}
% The command |\childdocforwardprefix| redirects
% compilation to the main or a child file by means of a pattern.
% The prefix |#1| in the current filename is replaced by |#2|
% and the suffix of the current filename is kept
% (it is assumed that the filename does not contain the substring `|~~~|'
% which is used as a delimiter).
% Compilation is handed over to the new file by |\childdocforward|:
%    \begin{macrocode}
\newcommand{\childdocforwardprefix}[3][]
{
  \begingroup
    \def\childdocextract #2##1~~~{\def\childdoctmp{\childdocforward[#1]{#3##1}}}
    \expandafter\childdocextract\childdocname~~~
    \expandafter
  \endgroup
  \childdoctmp
}
%    \end{macrocode}

% \macro{\childdoc}
% The deprecated macro |\childdoc| is a legacy version of |\childdocmain|:
%    \begin{macrocode}
\newcommand{\childdoc}{\childdocmain}
%    \end{macrocode}

% \macro{\childdocredirect}
% The deprecated macro |\childdocredirect| is a legacy version
% of |\childdocforward| and |\childdocforwardprefix|:
%    \begin{macrocode}
\newcommand{\childdocredirect}[2][]
{
  \begingroup
    \if?#1?
      \def\childdoctmp{\childdocforward{#2}}
    \else
      \def\childdoctmp{\childdocforwardprefix{#1}{#2}}
    \fi
    \expandafter
  \endgroup
  \childdoctmp
}
%    \end{macrocode}

%\iffalse
%</package>
%\fi
%
\endinput
|\\
|\childdocforward{|\textit{main}|}|\\
\end{tabular}
\end{center}
%
or alternatively with:
%
\begin{center}
\begin{tabular}{l}
|% \iffalse
%
% childdoc.dtx Copyright (C) 2017-2018 Niklas Beisert
%
% This work may be distributed and/or modified under the
% conditions of the LaTeX Project Public License, either version 1.3
% of this license or (at your option) any later version.
% The latest version of this license is in
%   http://www.latex-project.org/lppl.txt
% and version 1.3 or later is part of all distributions of LaTeX
% version 2005/12/01 or later.
%
% This work has the LPPL maintenance status `maintained'.
%
% The Current Maintainer of this work is Niklas Beisert.
%
% This work consists of the files childdoc.dtx and childdoc.ins
% and the derived files childdoc.def and cdocsamp.tex with
% cdocsch1.tex, cdocsch2.tex, cdocsdrf.tex, cdocsfn1.tex, cdocsfn2.tex.
%
%<package>\ifdefined\childdocmain\endinput\fi
%<package>\ProvidesFile{childdoc.def}[2018/12/30 v2.0 child document driver]
%<samplemain>\ProvidesFile{cdocsamp.tex}[2018/12/30 v2.0 sample for childdoc]
%<*driver>
%\ProvidesFile{childdoc.drv}[2018/12/30 v2.0 childdoc reference manual file]
\PassOptionsToClass{10pt,a4paper}{article}
\documentclass{ltxdoc}

\usepackage[margin=35mm]{geometry}
\usepackage{hyperref}
\usepackage{hyperxmp}
\usepackage[usenames]{color}

\hypersetup{colorlinks=true}
\hypersetup{pdfstartview=FitH}
\hypersetup{pdfpagemode=UseNone}
\hypersetup{pdfsource={}}
\hypersetup{pdflang={en-UK}}
\hypersetup{pdfcopyright={Copyright 2017-2018 Niklas Beisert.
  This work may be distributed and/or modified under the
  conditions of the LaTeX Project Public License, either version 1.3
  of this license or (at your option) any later version.}}
\hypersetup{pdflicenseurl={http://www.latex-project.org/lppl.txt}}
\hypersetup{pdfcontactaddress={ETH Zurich, ITP, HIT K,
  Wolfgang-Pauli-Strasse 27}}
\hypersetup{pdfcontactpostcode={8093}}
\hypersetup{pdfcontactcity={Zurich}}
\hypersetup{pdfcontactcountry={Switzerland}}
\hypersetup{pdfcontactemail={nbeisert@itp.phys.ethz.ch}}
\hypersetup{pdfcontacturl={http://people.phys.ethz.ch/\xmptilde nbeisert/}}

\newcommand{\secref}[1]{\hyperref[#1]{section \ref*{#1}}}

\parskip1ex
\parindent0pt
\let\olditemize\itemize
\def\itemize{\olditemize\parskip0pt}

\begin{document}

\title{The \textsf{childdoc} Package}
\hypersetup{pdftitle={The childdoc Package}}
\author{Niklas Beisert\\[2ex]
  Institut f\"ur Theoretische Physik\\
  Eidgen\"ossische Technische Hochschule Z\"urich\\
  Wolfgang-Pauli-Strasse 27, 8093 Z\"urich, Switzerland\\[1ex]
  \href{mailto:nbeisert@itp.phys.ethz.ch}
  {\texttt{nbeisert@itp.phys.ethz.ch}}}
\hypersetup{pdfauthor={Niklas Beisert}}
\hypersetup{pdfsubject={Manual for the LaTeX2e Package childdoc}}
\date{30 December 2018, \textsf{v2.0}}
\maketitle

\begin{abstract}\noindent
\textsf{childdoc} is a \LaTeXe{} package
that enables the direct compilation
of document sections included by |\include|
to individual files.
\end{abstract}

\begingroup
\parskip0ex
\tableofcontents
\endgroup

%%%%%%%%%%%%%%%%%%%%%%%%%%%%%%%%%%%%%%%%%%%%%%%%%%%%%%%%%%%%%%%%%%%%%%%%%%%%%%%%
%%%%%%%%%%%%%%%%%%%%%%%%%%%%%%%%%%%%%%%%%%%%%%%%%%%%%%%%%%%%%%%%%%%%%%%%%%%%%%%%
\section{Introduction}

\LaTeX{} provides a mechanism to structure a large document (such as a book)
into a main file and several child files (containing the chapters)
using the |\include| command.
This mechanism is beneficial for documents
which span hundreds of pages in order to
make the source file(s) more manageable.
Moreover, compilation can be restricted to
selected child files by means of the |\includeonly| command.
The latter feature can be used to reduce the compilation time while editing
(this was significantly more useful in the earlier days of \LaTeX{})
or to generate a smaller document which is easier to navigate.
Another application of |\includeonly| is to generate
documents consisting of selected parts of the complete document.

However, there are a few drawbacks of the plain |\include| mechanism:
\begin{itemize}
\item
The child files cannot be compiled on their own,
they can only be compiled via the main file.
A naive editing environment
(such as a text editor with an option
to have the current file processed by \LaTeX)
may require one to switch to the main file before compiling;
attempting to compile the child file produces errors.
\item
The main file must be modified (each time)
to adjust the |\includeonly| command
to the present needs. This easily leaves the main file in a messy state.
\item
The generated document will always carry the filename
of the main document. This is inconvenient if
several child files are to be compiled and
to be kept for distribution.
\end{itemize}

The present package provides a simple interface
to make child files individually compilable by \LaTeX{}.
Compiling a child file then has the same effect as compiling
the main file with an |\includeonly| command
to select the appropriate child.
Moreover the generated document will carry the name of the child
rather than the main file.
This resolves all three above issues.

This feature is meant to make the editing of books,
thesis documents and lecture notes somewhat more convenient.
However, the package can also be used efficiently for
composing a series of documents (such as exercise sheets)
which are typically distributed individually.
It then assists the author in generating the individual documents
(potentially in different versions)
as well as a document containing the collected series.
Another application is in developing style files
or other kinds of included material
where compilation of the style file could redirect
to a sample or test file.

%%%%%%%%%%%%%%%%%%%%%%%%%%%%%%%%%%%%%%%%%%%%%%%%%%%%%%%%%%%%%%%%%%%%%%%%%%%%%%%%
%%%%%%%%%%%%%%%%%%%%%%%%%%%%%%%%%%%%%%%%%%%%%%%%%%%%%%%%%%%%%%%%%%%%%%%%%%%%%%%%
\section{Usage}

First of all, the package \textsf{childdoc} is \emph{not} a standard
\LaTeXe{} |.sty| style file! Therefore it needs to be invoked in
a non-standard way.

%%%%%%%%%%%%%%%%%%%%%%%%%%%%%%%%%%%%%%%%%%%%%%%%%%%%%%%%%%%%%%%%%%%%%%%%%%%%%%%%
\subsection{Included Files}
\label{sec:include}

%%%%%%%%%%%%%%%%%%%%%%%%%%%%%%%%%%%%%%%%
\DescribeMacro{\childdocmain}
To use the package, add the commands
\begin{center}
\begin{tabular}{l}
|\input{childdoc.def}|\\
|\childdocmain{}|\\
\end{tabular}
\end{center}
at the very top of the main \LaTeX{} file,
in particular \emph{before} the |\documentclass| statement!
The argument of |\childdocmain| should be left empty
(but it must be present).

%%%%%%%%%%%%%%%%%%%%%%%%%%%%%%%%%%%%%%%%
\DescribeMacro{\childdocof}
Furthermore, add the commands
\begin{center}
\begin{tabular}{l}
|\input{childdoc.def}|\\
|\childdocof{|\textit{main}|}|\\
\end{tabular}
\end{center}
at the top of every child file \textit{child}
which is included by |\include{|\textit{child}|}|
from within the main file
(or at least for those files to be compiled individually).
The argument \textit{main} must be the filename of the main file.

There are a couple of
considerations in setting up the main and child documents:

%%%%%%%%%%%%%%%%%%%%%%%%%%%%%%%%%%%%%%%%
\paragraph{Restrictions.}

Please note the following restrictions:
\begin{itemize}
\item
|\childdocmain| must be called with one argument \textit{main}
to ensure compatibility with earlier version of the package.
It must either be empty (|\childdocmain{}|)
or precisely match the filename of the main file in which it is specified.
See \secref{sec:detection} for further information.
\item
The filename \textit{main} must be specified without the |.tex| extension.
\item
The filename \textit{main} is case sensitive
(even in case-insensitive file systems)
due to internal string comparison.
\item
The argument \textit{main} should be fully expanded, it cannot be a macro.
\item
Subdirectories and special characters should be avoided in filenames.
\item
The command |\childdocmain{|\textit{main}|}| must be followed by a whitespace.
It should not be followed immediately by another command
or by a comment mark `|%|'.
This is because the \TeX{} parser reads the token immediately following
the argument of |\childdocmain| and puts it
at the beginning of every child section;
however, a white\-space is ignored.
\end{itemize}

%%%%%%%%%%%%%%%%%%%%%%%%%%%%%%%%%%%%%%%%
\paragraph{Content of Main File.}

It is advisable to place all content in the child files included by |\include|.
Any output contained in the main file will appear in all child documents
unless suppressed manually;
it cannot be suppressed automatically by the |\includeonly| directive
and thus should normally be avoided.
A method to include some content in the main file
by means of conditional processing is described in \secref{sec:conditional}.

%%%%%%%%%%%%%%%%%%%%%%%%%%%%%%%%%%%%%%%%
\paragraph{Page Numbering.}

When only a part of the document is compiled,
the appropriate numbering of pages
(as well as other status parameters)
is determined from the |.aux| files.
The latter contain information from previous passes.
However this information needs to propagate through
all intermediate child documents.
Therefore the page numbering in child documents may well
be inconsistent until the complete document is compiled at least once.

A useful (if unconventional) way to always ensure a consistent
page numbering is to restart the numbering in each child document
and denote the pages by `\textit{child}|.|\textit{page}'
where \textit{child} represents the chapter/section number of the child file.
This can be achieved by the command
|\numberwithin{page}{|\textit{child}|}|
of the \textsf{amsmath} package
where \textit{child} can be |chapter| or |section|
depending on the chosen structuring.
Alternatively, one can modify the macro |\thepage| appropriately
and reset the counter |page| at the start of each child file.

%%%%%%%%%%%%%%%%%%%%%%%%%%%%%%%%%%%%%%%%%%%%%%%%%%%%%%%%%%%%%%%%%%%%%%%%%%%%%%%%
\subsection{Conditional Processing}
\label{sec:conditional}

The package provides a mechanism to compile different versions
of a document. To customise the versions further some conditional processing
can come in handy to distinguish which version is being compiled.
The package provides two macros to describe the compilation context:

%%%%%%%%%%%%%%%%%%%%%%%%%%%%%%%%%%%%%%%%
\DescribeMacro{\ifchilddoc}
The conditional |\ifchilddoc| distinguishes between the compilation of
child documents and the main document:
%
\begin{center}
|\ifchilddoc |\textit{child-code}| |[|\||else |\textit{main-code}]| \||fi|
\end{center}

%%%%%%%%%%%%%%%%%%%%%%%%%%%%%%%%%%%%%%%%
\DescribeMacro{\childdocname}
\DescribeMacro{\childdocjob}
The macro |\childdocname| contains the filename (without extension)
of the main or child file being processed.
Note that |\childdocjob| will always contain the name of the main file.

%%%%%%%%%%%%%%%%%%%%%%%%%%%%%%%%%%%%%%%%
\paragraph{Title Page.}

Conditional processing can be used to include a title or banner page
in the main document when proper precautions are taken.
Importantly, the code in the main file should ensure that the page counter
(as well as other status parameters which are stored in the |.aux| files)
takes the same value after the conditional processing.
Otherwise the page numbers may take divergent values
depending on which part is compiled.

For example, a title page could be declared by:
%
\begin{center}
\begin{tabular}{l}
|\ifchilddoc\||else|\\
|\addtocounter{page}{-1}|\\
\textit{code for title page}\\
|\newpage|\\
|\||fi|
\end{tabular}
\end{center}
%
A banner page for the child documents can be generated by:
%
\begin{center}
\begin{tabular}{l}
|\ifchilddoc|\\
|\addtocounter{page}{-1}|\\
\textit{code for banner page}\\
|\newpage|\\
|\||fi|
\end{tabular}
\end{center}
%
Here one could write a message such as:
\begin{center}
|This is the part \childdocname{} of \childdocjob{}.|
\end{center}

%%%%%%%%%%%%%%%%%%%%%%%%%%%%%%%%%%%%%%%%%%%%%%%%%%%%%%%%%%%%%%%%%%%%%%%%%%%%%%%%
\subsection{Flags}
\label{sec:flags}

The package makes it easy to generate different versions
of the main or child documents.
To this end compilation flags can be defined
and assigned different default values.
They will be particularly useful in conjunction
with the forwarding mechanism described in \secref{sec:forward}.

For example, it may be useful to have a flag |\version|
which can be set to |draft| or |final|.
The document source will contain some conditional code
depending on the value of |\version|.
Suppose further, the flag should default to |final| for the main file
and to |draft| for child files
which is a natural assignment for editing the document.
This is achieved by placing the following code
in the preamble of the main document
(below the |\childdocmain| directive):
%
\begin{center}
\begin{tabular}{l}
|\ifchilddoc|\\
|\providecommand{\version}{draft}|\\
|\||else|\\
|\providecommand{\version}{final}|\\
|\||fi|
\end{tabular}
\end{center}
%
The definition by |\providecommand| makes sure
that previous definitions are not overwritten.
Further statements |\providecommand{\version}{...}|
can thus be added before the above code to override it.

For the main file, one might add a line
(between |\childdocmain| and the above block)
%
\begin{center}
|%\ifchilddoc\||else\providecommand{\version}{draft}\||fi|
\end{center}
%
which can be uncommented to produce a draft version.
Likewise one can add a line to the very top of a child file
(above the |\childdocof{|\textit{main}|}| directive)
%
\begin{center}
|%\providecommand{\version}{final}|
\end{center}
%
which can be uncommented to produce the final version of this child document.

%%%%%%%%%%%%%%%%%%%%%%%%%%%%%%%%%%%%%%%%%%%%%%%%%%%%%%%%%%%%%%%%%%%%%%%%%%%%%%%%
\subsection{Forwarding}
\label{sec:forward}

Different versions of the main or child documents
using compilation flags as described in \secref{sec:flags}
can be (permanently) stored in different files
for convenient compilation, viewing and distribution.
To this end, the package defines a command
to pass on compilation to a different file:

%%%%%%%%%%%%%%%%%%%%%%%%%%%%%%%%%%%%%%%%
\DescribeMacro{\childdocforward}
The command |\childdocforward| redirects processing to
another source file:
%
\begin{center}
\begin{tabular}{l}
|\input{childdoc.def}|\\
|\childdocforward[|\textit{main}|]{|\textit{dest}|}|\\
\end{tabular}
\end{center}
%
The argument \textit{dest} is the destination file
(without extension).
It should be the main file or one of the child files.
Note that further \textsf{childdoc} directives
such as |\childdocof| and |\childdocforward|
in the indicated file will be processed in this form.
The optional argument \textit{main}
passes on directly to the main file \textit{main}
while pretending to compile the child \textit{dest}.
This form behaves as if \textit{dest}
issues |\childdocof{|\textit{main}|}| right away,
and no further \textsf{childdoc} directives will be processed.

%%%%%%%%%%%%%%%%%%%%%%%%%%%%%%%%%%%%%%%%
\DescribeMacro{\...prefix}
In the alternative form |\childdocforwardprefix|,
%
\begin{center}
\begin{tabular}{l}
|\input{childdoc.def}|\\
|\childdocforwardprefix[|\textit{main}|]{|\textit{prefix}|}{|\textit{dest}|}|
\end{tabular}
\end{center}
%
the destination file is determined by a pattern
depending on the current file:
To make this work, the current file must be called
`{\textit{prefix}\hspace{0.2em}\textit{suffix}}'
with \textit{prefix} matching precisely the argument.
Processing is then passed on to the file
`{\textit{dest}\hspace{0.2em}\textit{suffix}}'.
Surely, the same effect is achieved by
directly specifying the
argument `{\textit{dest}\hspace{0.2em}\textit{suffix}}'
in the first form.
However, that requires to set up a different file
for each child. With the alternative form of the command
all these files can have exactly the same content
which simplifies setting them up and maintaining them.

For example, the following file |draft.tex|
with a compilation flag |\version| as described in \secref{sec:flags}
compiles the main document as a draft:
%
\begin{center}
\begin{tabular}{l}
|\def\version{draft}|\\
|\input{childdoc.def}|\\
|\childdocforward{|\textit{main}|}|
\end{tabular}
\end{center}
%
Likewise, the following files |final|\textit{nn}|.tex|
compile the final version of the child document
|child|\textit{nn}|.tex|:
%
\begin{center}
\begin{tabular}{l}
|\def\version{final}|\\
|\input{childdoc.def}|\\
|\childdocforwardprefix{final}{child}|
\end{tabular}
\end{center}
%

Note that when several versions of a main file and/or of each child file
are to be generated, it may be convenient to set up a |Makefile| or
shell script to automatise the process.

%%%%%%%%%%%%%%%%%%%%%%%%%%%%%%%%%%%%%%%%%%%%%%%%%%%%%%%%%%%%%%%%%%%%%%%%%%%%%%%%
\subsection{Command Line Processing}
\label{sec:commandline}

The effect of redirection files can also be achieved by invoking
the \LaTeX{} compiler with a more elaborate command line.
Most conveniently this should be done as part
of a shell script or a |Makefile|.

When using \textsf{childdoc} in the main file, the following
command lines effectively perform a redirection
(note that depending on the shell being used,
backslashes may have to be doubled: `|\|' $\to$ `|\\|'):
%
\begin{center}
|... -jobname "|\textit{target}|" |\\|"|[\textit{flags}]%
|\input{childdoc.def}\childdocforward[|\textit{main}|]{|\textit{dest}|}"|
\end{center}
%
Here \textit{target} is the name of the output file,
\textit{main} is the name of the main file
and \textit{dest} is the name of the main or child file to be processed
(all filenames without extensions).
The optional argument \textit{main} can be omitted
if \textit{main} matches \textit{dest}.
Optionally, compilation \textit{flags} can be defined via |\def| commands.
This command line makes the \TeX{} engine believe
it is compiling the file \textit{target}
whose content is specified as the latter parameter.
The provided code then forwards the processing to
\textit{main} or \textit{dest} as described in \secref{sec:forward}.

%%%%%%%%%%%%%%%%%%%%%%%%%%%%%%%%%%%%%%%%%%%%%%%%%%%%%%%%%%%%%%%%%%%%%%%%%%%%%%%%
\subsection{Include by Input}
\label{sec:input}

Including child documents by |\include| has some restrictions by design.
Most notably, the content of a child document always occupies
its own set of pages; pages cannot be shared between child documents.
Usually, this behaviour makes perfect sense
because each child document contain an essential part of the document.
However, in some situations it may be desirable to compose
a document from a collection of parts
without having mandatory page breaks between then.
For this case, the package
provides a mechanism to include parts
by |\input| which can also be processed individually.
However, by construction this mechanism
requires manual handling of the content to be output.

%%%%%%%%%%%%%%%%%%%%%%%%%%%%%%%%%%%%%%%%
\DescribeMacro{\ifchilddocmanual}
The main file should be prepared as usual, see \secref{sec:include}.
However, the document body must make a distinction
between processing of an individual part and of the main document, e.g.:
%
\begin{center}
\begin{tabular}{l}
|\ifchilddocmanual|\\
|\input{\childdocname}|\\
|\||else|\\
\textit{document body with }|\input{|\textit{part}|}|\\
|\||fi|
\end{tabular}
\end{center}
%
The conditional |\ifchilddocmanual| is true whenever
a part to be included by |\input| is being compiled,
and the name of the part is stored in |\childdocname|.

%%%%%%%%%%%%%%%%%%%%%%%%%%%%%%%%%%%%%%%%
\DescribeMacro{\childdocby}
Each part to be included by |\input| should start with:
%
\begin{center}
\begin{tabular}{l}
|\input{childdoc.def}|\\
|\childdocby{|\textit{main}|}|\\
\end{tabular}
\end{center}
%
The directive |\childdocby| is similar to |\childdocof|
described in \secref{sec:include},
but the subsequent selection of content must be done manually.
To that end, both |\ifchilddoc| and |\ifchilddocmanual|
will be true upon processing of a part,
and the name of the part is stored in |\childdocname|.
Note that |\jobname| will be set to the filename of the current part
so that each part receives an individual |.aux| file
that does not interfere with the |.aux| file(s) of the main document.
This behaviour can be altered by the alternative form
|\childdocby[*]{|\textit{main}|}| (with a non-empty optional argument)
which uses the |.aux| file of the main document
by setting |\jobname| to \textit{main}.

%%%%%%%%%%%%%%%%%%%%%%%%%%%%%%%%%%%%%%%%%%%%%%%%%%%%%%%%%%%%%%%%%%%%%%%%%%%%%%%%
\subsection{Driver Development}
\label{sec:driver}

The \textsf{childdoc} mechanism can also be use for the development
of definition files such as \LaTeX{} styles or classes.
This case differs from the above setup with multiple parts
included by |\include| in that no |\includeonly| should be invoked.
This can be achieved by starting the include file
(before |\ProvidesPackage|) with:
%
\begin{center}
\begin{tabular}{l}
|\input{childdoc.def}|\\
|\childdocforward{|\textit{main}|}|\\
\end{tabular}
\end{center}
%
or alternatively with:
%
\begin{center}
\begin{tabular}{l}
|\input{childdoc.def}|\\
|\childdocby{|\textit{main}|}|\\
\end{tabular}
\end{center}
%
Both forms have slightly different effects as described above.
The main file is prepared as usual, see \secref{sec:include}.

%%%%%%%%%%%%%%%%%%%%%%%%%%%%%%%%%%%%%%%%%%%%%%%%%%%%%%%%%%%%%%%%%%%%%%%%%%%%%%%%
\subsection{Legacy Detection}
\label{sec:detection}

The directive |\childdocmain| in the main file can detect
whether the complete document or merely a child is to be compiled
even without using the directive |\childdocof|.
This method is deprecated because it is less robust
and there is no compelling reason to use it;
it is merely provided for backward compatibility
and it may be removed in future versions.

If the detection mechanism is to be used,
it is mandatory to correctly specify
the filename of the main file as the argument of |\childdocmain|:
%
\begin{center}
\begin{tabular}{l}
|\input{childdoc.def}|\\
|\childdocmain{|\textit{main}|}|\\
\end{tabular}
\end{center}
%
If |\jobname| does not match the argument \textit{main} of |\childdocmain|,
it is assumed that |\jobname| points to the child file to be compiled.
When using |\childdocmain| with the main file specified as argument,
it suffices to start a child file
with just |\input{|\textit{main}|}|
without loading of the package and using |\childdocof|.
If instead all processing is done
with the appropriate \textsf{childdoc} directives,
the argument of \textit{main} of |\childdocmain| can be empty.

An alternative version of the command line processing described
in \secref{sec:commandline} using the detection mechanism reads:
%
\begin{center}
|... -jobname "|\textit{target}|" "|[\textit{flags}]%
[|\def\jobname{|\textit{dest}|}|]|\input{|\textit{main}|}"|
\end{center}

%%%%%%%%%%%%%%%%%%%%%%%%%%%%%%%%%%%%%%%%%%%%%%%%%%%%%%%%%%%%%%%%%%%%%%%%%%%%%%%%
\subsection{Manual Code}
\label{sec:manual}

In case one cannot be certain whether the definitions file |childdoc.def|
is installed on the target \TeX{} distribution
and one prefers not to ship it,
it is conceivable to paste a few relevant commands into the sources.

To that end, drop all statements |\input{childdoc.def}|
and perform the replacements as outlined below.
Instead of |\childdocmain{|\textit{main}|}| add the following code
to the top of the main file:
%
\begin{center}
\begin{tabular}{l}
|\||ifdefined\childdocname\endinput\||fi\newif\ifchilddoc|\\
|\edef\childdocname{\scantokens\expandafter{\jobname\noexpand}}|\\
|\def\childdocmain{|\textit{main}|}\||ifx\childdocmain\childdocname\||else|\\
|\childdoctrue\includeonly{\childdocname}\let\jobname\childdocmain\||fi|\\
\end{tabular}
\end{center}
%
Instead of |\childdocof{|\textit{main}|}| just include the main file
at the top of each child file:
%
\begin{center}
|\input{|\textit{main}|}|
\end{center}
%
A simple redirection |\childdocforward{|\textit{dest}|}| is achieved by:
%
\begin{center}
|\def\jobname{|\textit{dest}|}\input{\jobname}|
\end{center}
%
The redirection with prefix
|\childdocforwardprefix[|\textit{prefix}|]{|\textit{dest}|}|
is accomplished by:
%
\begin{center}
\begin{tabular}{l}
|{\edef\jobname{\scantokens\expandafter{\jobname\noexpand}}|\\
|\def\redirectjob |\textit{prefix}|#1~~~{\gdef\jobname{|\textit{dest}|#1}}|\\
|\expandafter\redirectjob\jobname~~~}\input{\jobname}|
\end{tabular}
\end{center}

In an alternative approach,
child documents can be compiled by a specific command line
without additional code or specific definitions:
%
\begin{center}
|... -jobname "|\textit{target}|" "|[\textit{flags}]%
|\includeonly{|\textit{dest}|}\input{|\textit{main}|}"|
\end{center}
%

%%%%%%%%%%%%%%%%%%%%%%%%%%%%%%%%%%%%%%%%%%%%%%%%%%%%%%%%%%%%%%%%%%%%%%%%%%%%%%%%
%%%%%%%%%%%%%%%%%%%%%%%%%%%%%%%%%%%%%%%%%%%%%%%%%%%%%%%%%%%%%%%%%%%%%%%%%%%%%%%%
\section{Information}

%%%%%%%%%%%%%%%%%%%%%%%%%%%%%%%%%%%%%%%%%%%%%%%%%%%%%%%%%%%%%%%%%%%%%%%%%%%%%%%%
\subsection{Copyright}

Copyright \copyright{} 2017--2018 Niklas Beisert

This work may be distributed and/or modified under the
conditions of the \LaTeX{} Project Public License, either version 1.3
of this license or (at your option) any later version.
The latest version of this license is in
  \url{http://www.latex-project.org/lppl.txt}
and version 1.3 or later is part of all distributions of \LaTeX{}
version 2005/12/01 or later.

This work has the LPPL maintenance status `maintained'.

The Current Maintainer of this work is Niklas Beisert.

This work consists of the files |README.txt|, |childdoc.ins| and |childdoc.dtx|
as well as the derived files |childdoc.def|, |cdocsamp.tex|
with |cdocsch1.tex|, |cdocsch2.tex|, |cdocspt3.tex|, |cdocspt4.tex|,
|cdocsdrf.tex|, |cdocsfn1.tex|, |cdocsfn2.tex|
as well as |childdoc.pdf|.

%%%%%%%%%%%%%%%%%%%%%%%%%%%%%%%%%%%%%%%%%%%%%%%%%%%%%%%%%%%%%%%%%%%%%%%%%%%%%%%%
\subsection{Files and Installation}

The package consists of the files:
%
\begin{center}
\begin{tabular}{ll}
    |README.txt|   & readme file \\
    |childdoc.ins| & installation file \\
    |childdoc.dtx| & source file \\
    |childdoc.def| & definition file \\
    |cdocsamp.tex| & sample main file \\
    |cdocsch1.tex| & sample include file \\
    |cdocsch2.tex| & sample include file \\
    |cdocspt3.tex| & sample part file \\
    |cdocspt4.tex| & sample part file \\
    |cdocsdrf.tex| & sample redirection file \\
    |cdocsfn1.tex| & sample redirection file \\
    |cdocsfn2.tex| & sample redirection file \\
    |childdoc.pdf| & manual
\end{tabular}
\end{center}
%
The distribution consists of the files
|README.txt|, |childdoc.ins| and |childdoc.dtx|.
%
\begin{itemize}
\item
Run (pdf)\LaTeX{} on |childdoc.dtx|
to compile the manual |childdoc.pdf| (this file).
\item
Run \LaTeX{} on |childdoc.ins| to create the definitions file |childdoc.def|
and the sample |cdocsamp.tex| with include files
|cdocsch1.tex|, |cdocsch2.tex|, |cdocspt3.tex|, |cdocspt4.tex|,
|cdocsdrf.tex|, |cdocsfn1.tex|, |cdocsfn2.tex|.
Then copy the file |childdoc.def| to an appropriate directory of your \LaTeX{}
distribution, e.g.\ \textit{texmf-root}|/tex/latex/childdoc|.
\end{itemize}

%%%%%%%%%%%%%%%%%%%%%%%%%%%%%%%%%%%%%%%%%%%%%%%%%%%%%%%%%%%%%%%%%%%%%%%%%%%%%%%%
\subsection{Related CTAN Packages}

There are several other packages which offer a similar functionality:
%
\begin{itemize}
\item
The packages
\href{http://ctan.org/pkg/docmute}{\textsf{docmute}},
\href{http://ctan.org/pkg/includex}{\textsf{includex}} and
\href{http://ctan.org/pkg/standalone}{\textsf{standalone}}
provide commands to include only the document body of
a child file thus allowing both files to be compiled individually.
\item
The packages \href{http://ctan.org/pkg/subdocs}{\textsf{subdocs}}
and \href{http://ctan.org/pkg/subfiles}{\textsf{subfiles}}
provide structures in which the main and child documents can be
encapsulated and allowing them to be compiled individually.
The inclusion mechanism is different from the conventional |\include|.
\item
The package \href{http://ctan.org/pkg/combine}{\textsf{combine}}
is an elaborate solution to combine several documents into one.
\end{itemize}
%
See also the CTAN topic \href{http://ctan.org/topic/subdocs}{\textsf{subdocs}}
for further related packages.
The present package differs from the above solutions in that
a document structure constructed with the conventional |\include| mechanism
just needs two extra commands at the top of every file
such that all constituent files can be compiled individually.

%%%%%%%%%%%%%%%%%%%%%%%%%%%%%%%%%%%%%%%%%%%%%%%%%%%%%%%%%%%%%%%%%%%%%%%%%%%%%%%%
%\subsection{Feature Suggestions}
%
%The following is a list of features which may be useful for future
%versions of this package:
%%
%\begin{itemize}
%\item
%\ldots
%\end{itemize}

%%%%%%%%%%%%%%%%%%%%%%%%%%%%%%%%%%%%%%%%%%%%%%%%%%%%%%%%%%%%%%%%%%%%%%%%%%%%%%%%
\subsection{Revision History}

%%%%%%%%%%%%%%%%%%%%%%%%%%%%%%%%%%%%%%%%
\paragraph{v2.0:} 2018/12/30

\begin{itemize}
\item
immediate forward processing
\item
added |\childdocby| mechanism
\item
manual restructured
\end{itemize}

%%%%%%%%%%%%%%%%%%%%%%%%%%%%%%%%%%%%%%%%
\paragraph{v1.6:} 2018/01/17

\begin{itemize}
\item
application for development of include files
\item
corrections to manual
\end{itemize}

%%%%%%%%%%%%%%%%%%%%%%%%%%%%%%%%%%%%%%%%
\paragraph{v1.5:} 2017/05/21

\begin{itemize}
\item
more complete structuring introduced
\item
|\childdocof| introduced
\item
|\childdoc| renamed to |\childdocmain|
\item
|\childredirect| renamed to |\childdocforward| and |\childdocforwardprefix|
and functionality expanded
\end{itemize}

%%%%%%%%%%%%%%%%%%%%%%%%%%%%%%%%%%%%%%%%
\paragraph{v1.0:} 2017/04/27

\begin{itemize}
\item
manual and install package
\item
first version published on CTAN
\end{itemize}

%%%%%%%%%%%%%%%%%%%%%%%%%%%%%%%%%%%%%%%%
\paragraph{v0.6:} 2017/04/26

\begin{itemize}
\item
redirection mechanism added
\end{itemize}

%%%%%%%%%%%%%%%%%%%%%%%%%%%%%%%%%%%%%%%%
\paragraph{v0.5:} 2017/04/26

\begin{itemize}
\item
functionality in definition file
\end{itemize}


%%%%%%%%%%%%%%%%%%%%%%%%%%%%%%%%%%%%%%%%%%%%%%%%%%%%%%%%%%%%%%%%%%%%%%%%%%%%%%%%
%%%%%%%%%%%%%%%%%%%%%%%%%%%%%%%%%%%%%%%%%%%%%%%%%%%%%%%%%%%%%%%%%%%%%%%%%%%%%%%%
%%%%%%%%%%%%%%%%%%%%%%%%%%%%%%%%%%%%%%%%%%%%%%%%%%%%%%%%%%%%%%%%%%%%%%%%%%%%%%%%
\appendix

\settowidth\MacroIndent{\rmfamily\scriptsize 000\ }

 \DocInput{childdoc.dtx}

\end{document}
%</driver>
% \fi
%
% %%%%%%%%%%%%%%%%%%%%%%%%%%%%%%%%%%%%%%%%%%%%%%%%%%%%%%%%%%%%%%%%%%%%%%%%%%%%%%
% %%%%%%%%%%%%%%%%%%%%%%%%%%%%%%%%%%%%%%%%%%%%%%%%%%%%%%%%%%%%%%%%%%%%%%%%%%%%%%
% \section{Sample}
%\iffalse
%<*samplemain>
%\fi
%
% The following presents a sample document
% with two chapters, two parts, a title page,
% a compile flag as well as three forwarding files to set the flag.
% It consists of eight |.tex| files:
% \begin{center}
% \begin{tabular}{ll}
% |cdocsamp.tex|&main file\\
% |cdocsch1.tex|&include file for chapter 1\\
% |cdocsch2.tex|&include file for chapter 2\\
% |cdocspt3.tex|&include file for part 3\\
% |cdocspt4.tex|&include file for part 4\\
% |cdocsdrf.tex|&forwarding file for main file in draft mode\\
% |cdocsfi1.tex|&forwarding file for final version of chapter 1\\
% |cdocsfi2.tex|&forwarding file for final version of chapter 2\\
% \end{tabular}
% \end{center}
% Each of the eight files can be compiled directly by the \LaTeX{} compiler.
%
% %%%%%%%%%%%%%%%%%%%%%%%%%%%%%%%%%%%%%%
% \paragraph{Main File.}
%
% The main file is called |cdocsamp.tex|.
%
% Load the \textsf{childdoc} definitions and
% declare the filename for the main document:
%    \begin{macrocode}
\input{childdoc.def}
\childdocmain{}
%    \end{macrocode}

% Optional override for |\version| flag:
%    \begin{macrocode}
%%\ifchilddoc\else\providecommand{\version}{draft}\fi
%    \end{macrocode}

% Define the default values for the |\version| flag
% (|final| for the main file and |draft| for childs):
%    \begin{macrocode}
\ifchilddoc
\providecommand{\version}{draft}
\else
\providecommand{\version}{final}
\fi
%    \end{macrocode}

% Load the standard document class:
%    \begin{macrocode}
\documentclass[12pt]{article}
%    \end{macrocode}

% Start the document body:
%    \begin{macrocode}
\begin{document}
%    \end{macrocode}

% Declare a title page.
% Print title, part of document being processed and version flag:
%    \begin{macrocode}
\addtocounter{page}{-1}
\begin{center}
{\LARGE\bfseries{}childdoc example\par}
\vspace{1cm}
\ifchilddoc
\ifchilddocmanual part\else chapter\fi:
`\childdocname' of `\childdocjob'\par
\else
main document: `\childdocjob'\par
\fi
version: \version\par
\end{center}
\newpage
%    \end{macrocode}

% Manually include selected file,
% otherwise process as usual:
%    \begin{macrocode}
\ifchilddocmanual
\section*{part `\childdocname'}
\input{\childdocname}
\else
%    \end{macrocode}

% Include the two chapters:
%    \begin{macrocode}
\include{cdocsch1}
\include{cdocsch2}
%    \end{macrocode}

% Include the two parts unless only chapters should be displayed:
%    \begin{macrocode}
\ifchilddoc\else
\section{part three}
\input{cdocspt3}
\section{part four}
\input{cdocspt4}
\fi
%    \end{macrocode}

% Process as usual until here:
%    \begin{macrocode}
\fi
%    \end{macrocode}

% End of document body:
%    \begin{macrocode}
\end{document}
%    \end{macrocode}
%\iffalse
%</samplemain>
%\fi
%
% %%%%%%%%%%%%%%%%%%%%%%%%%%%%%%%%%%%%%%
% \paragraph{Chapter Include Files.}
%
% The include files are called |cdocsch1.tex| and |cdocsch2.tex|.
%
%\iffalse
%<*samplechap1|samplechap2>
%\fi

% Optional override for |\version| flag:
%    \begin{macrocode}
%%\providecommand{\version}{final}
%    \end{macrocode}

% Include the main document:
%    \begin{macrocode}
\input{childdoc.def}
\childdocof{cdocsamp}
%    \end{macrocode}

%\iffalse
%</samplechap1|samplechap2>
%\fi
%
%\iffalse
%<*samplechap1>
%\fi
% Some text for chapter 1:
%    \begin{macrocode}
\section{one}
some text in chapter one
%    \end{macrocode}

%\iffalse
%</samplechap1>
%\fi
% Some text for chapter 2:
%\iffalse
%<*samplechap2>
%\fi
%    \begin{macrocode}
\section{two}
more text in chapter two
%    \end{macrocode}

%\iffalse
%</samplechap2>
%\fi
%
% %%%%%%%%%%%%%%%%%%%%%%%%%%%%%%%%%%%%%%
% \paragraph{Part Include Files.}
%
% The include files are called |cdocspt3.tex| and |cdocspt4.tex|.
%
%\iffalse
%<*samplepart3|samplepart4>
%\fi

% Optional override for |\version| flag:
%    \begin{macrocode}
%%\providecommand{\version}{final}
%    \end{macrocode}

% Include the main document:
%    \begin{macrocode}
\input{childdoc.def}
\childdocby{cdocsamp}
%    \end{macrocode}

%\iffalse
%</samplepart3|samplepart4>
%\fi
%
%\iffalse
%<*samplepart3>
%\fi
% Some text for part 3:
%    \begin{macrocode}
some text in part three
%    \end{macrocode}

%\iffalse
%</samplepart3>
%\fi
% Some text for part 4:
%\iffalse
%<*samplepart4>
%\fi
%    \begin{macrocode}
more text in part four
%    \end{macrocode}

%\iffalse
%</samplepart4>
%\fi
%
% %%%%%%%%%%%%%%%%%%%%%%%%%%%%%%%%%%%%%%
% \paragraph{Forwarding for a Complete Draft.}
%
% The following forwarding file |cdocsdrf.tex|
% compiles the main document in draft mode:
%\iffalse
%<*sampledraft>
%\fi
%    \begin{macrocode}
\def\version{draft}
\input{childdoc.def}
\childdocforward{cdocsamp}
%    \end{macrocode}

%\iffalse
%</sampledraft>
%\fi
%
% %%%%%%%%%%%%%%%%%%%%%%%%%%%%%%%%%%%%%%
% \paragraph{Forwarding for Final Version of the Chapters.}
%
% The following forwarding files |cdocsfn1.tex| and |cdocsfn2.tex|
% (with identical content)
% compile the final versions of the child documents
% |cdocsch1.tex| and |cdocsch2.tex|, respectively:
%\iffalse
%<*samplefinal>
%\fi
%    \begin{macrocode}
\def\version{final}
\input{childdoc.def}
\childdocforwardprefix[cdocsamp]{cdocsfn}{cdocsch}
%    \end{macrocode}

%\iffalse
%</samplefinal>
%\fi
%
% %%%%%%%%%%%%%%%%%%%%%%%%%%%%%%%%%%%%%%
% \paragraph{Command Line Processing.}
%
% The following three command lines generate the output files
% |cdocscld|, |cdocscl1| and |cdocscl2|
% which should be identical to
% |cdocsdrf|, |cdocsch1| and |cdocsfn2|, respectively:
% \begin{center}
% \begin{tabular}{l}
% |latex -jobname cdocscld \|\\
% |  "\def\version{draft}\input{childdoc.def}\childdocforward{cdocsamp}"|\\
% |latex -jobname cdocscl1 \|\\
% |  "\input{childdoc.def}\childdocforward[cdocsamp]{cdocsch1}"|\\
% |latex -jobname cdocscl2 \|\\
% |  "\def\version{final}\input{childdoc.def}\childdocforward{cdocsch2}"|
% \end{tabular}
% \end{center}
% Note that the trailing backslash on each first line
% merely continues the input to the second line
% (for convenient cut ant paste).
% Furthermore, the command |latex| can be replaced by any
% of its alternative versions such as |pdflatex|.
%
% %%%%%%%%%%%%%%%%%%%%%%%%%%%%%%%%%%%%%%%%%%%%%%%%%%%%%%%%%%%%%%%%%%%%%%%%%%%%%%
% %%%%%%%%%%%%%%%%%%%%%%%%%%%%%%%%%%%%%%%%%%%%%%%%%%%%%%%%%%%%%%%%%%%%%%%%%%%%%%
% \section{Implementation}
%\iffalse
%<*package>
%\fi
%
% This section describes the definitions file |childdoc.def|.

% The definitions cannot be loaded using |\usepackage| or |\RequirePackage|
% which has a mechanism to prevent loading a style file more than once.
% When loading the definitions by means of |\input|
% multiple instances have to be prevented manually:
%\iffalse
%This code needs to be before the `\ProvidesFile' directive
%which is defined at the beginning of this file.
%Therefore it is also placed there and commented out here.
%</package>
%<*discard>
%\fi
%    \begin{macrocode}
\ifdefined\childdocmain\endinput\fi
%    \end{macrocode}
%\iffalse
%</discard>
%<*package>
%\fi
%
% \macro{\ifchilddoc}
% \macro{\ifchilddocmanual}
% The conditional |\ifchilddoc| tells whether a
% child (true) or main (false) document is being compiled.
% The conditional |\ifchilddocmanual| tells whether
% the |\includeonly| mechanism is used (false) or
% the selection of child files must be performed manually (true).
% The definitions initialise to false:
%    \begin{macrocode}
\newif\ifchilddoc
\newif\ifchilddocmanual
%    \end{macrocode}

% \macro{\childdocname}
% \macro{\childdocjob}
% The macro |\childdocname| stores the name of the main document
% to be compiled. The macro |\childdocjob| stores the name of
% the document on which the \LaTeX{} compiler was originally invoked.
% The content of |\jobname| cannot be compared
% to filenames specified in the source due to different catcodes.
% The following code rescans |\jobname|, stores the result
% in |\childdocname| and saves a copy in |\childdocjob|:
%    \begin{macrocode}
\edef\childdocname{\scantokens\expandafter{\jobname\noexpand}}
\let\childdocjob\childdocname
%    \end{macrocode}

% \macro{\childdocdisable}
% The macro |\childdocdisable| prevents the main file
% from being processed more than once.
% At this stage, the main document command |\childdocmain|
% is assumed to be called once again where it should do nothing.
% Any subsequent call to it should prevent
% a secondary processing of the main document
% It overwrites the forwarding commands
% |\childdocof| and |\childdocforward|
% with empty macros to prevent further inclusions of the main document:
%    \begin{macrocode}
\newcommand{\childdocdisable}
{
  \renewcommand{\childdocmain}[1]{\renewcommand{\childdocmain}[1]{\endinput}}
  \renewcommand{\childdocof}[1]{}
  \renewcommand{\childdocby}[2][]{}
  \renewcommand{\childdocforward}[2][]{}
  \renewcommand{\childdocdisable}{}
}
%    \end{macrocode}

% \macro{\childdocmain}
% The macro |\childdocmain| is to be called at the top of the main file
% with nothing or the main filename (without extension) as argument.
% First, it breaks loops.
% If the argument is not empty and does not match |\childdocname|
% (which is set by the first inclusion of |childdoc.def|),
% |\ifchilddoc| is set to true, |\includeonly| is applied to the child file
% and |\jobname| is set to the main file
% (for proper handling of |.aux| files):
%    \begin{macrocode}
\newcommand{\childdocmain}[1]
{
  \childdocdisable\childdocmain{}
  \if?#1?\else
    \begingroup
      \def\childdoctmp{#1}
      \ifx\childdoctmp\childdocname
        \def\childdoctmp{}
      \else
        \def\childdoctmp
        {
          \childdoctrue
          \includeonly{\childdocname}
          \def\childdocjob{#1}
          \def\jobname{#1}
        }
      \fi
      \expandafter
    \endgroup
    \childdoctmp
  \fi
}
%    \end{macrocode}

% \macro{\childdocof}
% The command |\childdocof| redirects
% compilation to the main file |#1|.
%    \begin{macrocode}
\newcommand{\childdocof}[1]
{
  \childdocdisable
  \childdoctrue
  \includeonly{\childdocname}
  \def\jobname{#1}
  \def\childdocjob{#1}
  \input{#1}
}
%    \end{macrocode}

% \macro{\childdocby}
% The command |\childdocby| ....
%    \begin{macrocode}
\newcommand{\childdocby}[2][]
{
  \childdocdisable
  \childdoctrue
  \childdocmanualtrue
  \if?#1?\else
    \def\jobname{#2}
  \fi
  \def\childdocjob{#2}
  \input{#2}
  \endinput
}
%    \end{macrocode}

% \macro{\childdocforward}
% The command |\childdocforward| redirects
% compilation to the main file or
% (if the optional argument is given) a child file.
% Parameters are set as if the main file
% or a child file starting with |\childdocof| was compiled.
% Then compilation is handed over to the main file:
%    \begin{macrocode}
\newcommand{\childdocforward}[2][]
{
  \begingroup
    \if?#1?
      \def\childdoctmp
      {
        \def\childdocname{#2}
        \def\childdocjob{#2}
        \def\jobname{#2}
        \input{#2}
        \endinput
      }
    \else
      \def\childdoctmp
      {
        \childdocdisable
        \def\childdocname{#2}
        \childdoctrue
        \includeonly{#2}
        \def\childdocjob{#1}
        \def\jobname{#1}
        \input{#1}
        \endinput
      }
    \fi
    \expandafter
  \endgroup
  \childdoctmp
}
%    \end{macrocode}

% \macro{\childdocforwardprefix}
% The command |\childdocforwardprefix| redirects
% compilation to the main or a child file by means of a pattern.
% The prefix |#1| in the current filename is replaced by |#2|
% and the suffix of the current filename is kept
% (it is assumed that the filename does not contain the substring `|~~~|'
% which is used as a delimiter).
% Compilation is handed over to the new file by |\childdocforward|:
%    \begin{macrocode}
\newcommand{\childdocforwardprefix}[3][]
{
  \begingroup
    \def\childdocextract #2##1~~~{\def\childdoctmp{\childdocforward[#1]{#3##1}}}
    \expandafter\childdocextract\childdocname~~~
    \expandafter
  \endgroup
  \childdoctmp
}
%    \end{macrocode}

% \macro{\childdoc}
% The deprecated macro |\childdoc| is a legacy version of |\childdocmain|:
%    \begin{macrocode}
\newcommand{\childdoc}{\childdocmain}
%    \end{macrocode}

% \macro{\childdocredirect}
% The deprecated macro |\childdocredirect| is a legacy version
% of |\childdocforward| and |\childdocforwardprefix|:
%    \begin{macrocode}
\newcommand{\childdocredirect}[2][]
{
  \begingroup
    \if?#1?
      \def\childdoctmp{\childdocforward{#2}}
    \else
      \def\childdoctmp{\childdocforwardprefix{#1}{#2}}
    \fi
    \expandafter
  \endgroup
  \childdoctmp
}
%    \end{macrocode}

%\iffalse
%</package>
%\fi
%
\endinput
|\\
|\childdocby{|\textit{main}|}|\\
\end{tabular}
\end{center}
%
Both forms have slightly different effects as described above.
The main file is prepared as usual, see \secref{sec:include}.

%%%%%%%%%%%%%%%%%%%%%%%%%%%%%%%%%%%%%%%%%%%%%%%%%%%%%%%%%%%%%%%%%%%%%%%%%%%%%%%%
\subsection{Legacy Detection}
\label{sec:detection}

The directive |\childdocmain| in the main file can detect
whether the complete document or merely a child is to be compiled
even without using the directive |\childdocof|.
This method is deprecated because it is less robust
and there is no compelling reason to use it;
it is merely provided for backward compatibility
and it may be removed in future versions.

If the detection mechanism is to be used,
it is mandatory to correctly specify
the filename of the main file as the argument of |\childdocmain|:
%
\begin{center}
\begin{tabular}{l}
|% \iffalse
%
% childdoc.dtx Copyright (C) 2017-2018 Niklas Beisert
%
% This work may be distributed and/or modified under the
% conditions of the LaTeX Project Public License, either version 1.3
% of this license or (at your option) any later version.
% The latest version of this license is in
%   http://www.latex-project.org/lppl.txt
% and version 1.3 or later is part of all distributions of LaTeX
% version 2005/12/01 or later.
%
% This work has the LPPL maintenance status `maintained'.
%
% The Current Maintainer of this work is Niklas Beisert.
%
% This work consists of the files childdoc.dtx and childdoc.ins
% and the derived files childdoc.def and cdocsamp.tex with
% cdocsch1.tex, cdocsch2.tex, cdocsdrf.tex, cdocsfn1.tex, cdocsfn2.tex.
%
%<package>\ifdefined\childdocmain\endinput\fi
%<package>\ProvidesFile{childdoc.def}[2018/12/30 v2.0 child document driver]
%<samplemain>\ProvidesFile{cdocsamp.tex}[2018/12/30 v2.0 sample for childdoc]
%<*driver>
%\ProvidesFile{childdoc.drv}[2018/12/30 v2.0 childdoc reference manual file]
\PassOptionsToClass{10pt,a4paper}{article}
\documentclass{ltxdoc}

\usepackage[margin=35mm]{geometry}
\usepackage{hyperref}
\usepackage{hyperxmp}
\usepackage[usenames]{color}

\hypersetup{colorlinks=true}
\hypersetup{pdfstartview=FitH}
\hypersetup{pdfpagemode=UseNone}
\hypersetup{pdfsource={}}
\hypersetup{pdflang={en-UK}}
\hypersetup{pdfcopyright={Copyright 2017-2018 Niklas Beisert.
  This work may be distributed and/or modified under the
  conditions of the LaTeX Project Public License, either version 1.3
  of this license or (at your option) any later version.}}
\hypersetup{pdflicenseurl={http://www.latex-project.org/lppl.txt}}
\hypersetup{pdfcontactaddress={ETH Zurich, ITP, HIT K,
  Wolfgang-Pauli-Strasse 27}}
\hypersetup{pdfcontactpostcode={8093}}
\hypersetup{pdfcontactcity={Zurich}}
\hypersetup{pdfcontactcountry={Switzerland}}
\hypersetup{pdfcontactemail={nbeisert@itp.phys.ethz.ch}}
\hypersetup{pdfcontacturl={http://people.phys.ethz.ch/\xmptilde nbeisert/}}

\newcommand{\secref}[1]{\hyperref[#1]{section \ref*{#1}}}

\parskip1ex
\parindent0pt
\let\olditemize\itemize
\def\itemize{\olditemize\parskip0pt}

\begin{document}

\title{The \textsf{childdoc} Package}
\hypersetup{pdftitle={The childdoc Package}}
\author{Niklas Beisert\\[2ex]
  Institut f\"ur Theoretische Physik\\
  Eidgen\"ossische Technische Hochschule Z\"urich\\
  Wolfgang-Pauli-Strasse 27, 8093 Z\"urich, Switzerland\\[1ex]
  \href{mailto:nbeisert@itp.phys.ethz.ch}
  {\texttt{nbeisert@itp.phys.ethz.ch}}}
\hypersetup{pdfauthor={Niklas Beisert}}
\hypersetup{pdfsubject={Manual for the LaTeX2e Package childdoc}}
\date{30 December 2018, \textsf{v2.0}}
\maketitle

\begin{abstract}\noindent
\textsf{childdoc} is a \LaTeXe{} package
that enables the direct compilation
of document sections included by |\include|
to individual files.
\end{abstract}

\begingroup
\parskip0ex
\tableofcontents
\endgroup

%%%%%%%%%%%%%%%%%%%%%%%%%%%%%%%%%%%%%%%%%%%%%%%%%%%%%%%%%%%%%%%%%%%%%%%%%%%%%%%%
%%%%%%%%%%%%%%%%%%%%%%%%%%%%%%%%%%%%%%%%%%%%%%%%%%%%%%%%%%%%%%%%%%%%%%%%%%%%%%%%
\section{Introduction}

\LaTeX{} provides a mechanism to structure a large document (such as a book)
into a main file and several child files (containing the chapters)
using the |\include| command.
This mechanism is beneficial for documents
which span hundreds of pages in order to
make the source file(s) more manageable.
Moreover, compilation can be restricted to
selected child files by means of the |\includeonly| command.
The latter feature can be used to reduce the compilation time while editing
(this was significantly more useful in the earlier days of \LaTeX{})
or to generate a smaller document which is easier to navigate.
Another application of |\includeonly| is to generate
documents consisting of selected parts of the complete document.

However, there are a few drawbacks of the plain |\include| mechanism:
\begin{itemize}
\item
The child files cannot be compiled on their own,
they can only be compiled via the main file.
A naive editing environment
(such as a text editor with an option
to have the current file processed by \LaTeX)
may require one to switch to the main file before compiling;
attempting to compile the child file produces errors.
\item
The main file must be modified (each time)
to adjust the |\includeonly| command
to the present needs. This easily leaves the main file in a messy state.
\item
The generated document will always carry the filename
of the main document. This is inconvenient if
several child files are to be compiled and
to be kept for distribution.
\end{itemize}

The present package provides a simple interface
to make child files individually compilable by \LaTeX{}.
Compiling a child file then has the same effect as compiling
the main file with an |\includeonly| command
to select the appropriate child.
Moreover the generated document will carry the name of the child
rather than the main file.
This resolves all three above issues.

This feature is meant to make the editing of books,
thesis documents and lecture notes somewhat more convenient.
However, the package can also be used efficiently for
composing a series of documents (such as exercise sheets)
which are typically distributed individually.
It then assists the author in generating the individual documents
(potentially in different versions)
as well as a document containing the collected series.
Another application is in developing style files
or other kinds of included material
where compilation of the style file could redirect
to a sample or test file.

%%%%%%%%%%%%%%%%%%%%%%%%%%%%%%%%%%%%%%%%%%%%%%%%%%%%%%%%%%%%%%%%%%%%%%%%%%%%%%%%
%%%%%%%%%%%%%%%%%%%%%%%%%%%%%%%%%%%%%%%%%%%%%%%%%%%%%%%%%%%%%%%%%%%%%%%%%%%%%%%%
\section{Usage}

First of all, the package \textsf{childdoc} is \emph{not} a standard
\LaTeXe{} |.sty| style file! Therefore it needs to be invoked in
a non-standard way.

%%%%%%%%%%%%%%%%%%%%%%%%%%%%%%%%%%%%%%%%%%%%%%%%%%%%%%%%%%%%%%%%%%%%%%%%%%%%%%%%
\subsection{Included Files}
\label{sec:include}

%%%%%%%%%%%%%%%%%%%%%%%%%%%%%%%%%%%%%%%%
\DescribeMacro{\childdocmain}
To use the package, add the commands
\begin{center}
\begin{tabular}{l}
|\input{childdoc.def}|\\
|\childdocmain{}|\\
\end{tabular}
\end{center}
at the very top of the main \LaTeX{} file,
in particular \emph{before} the |\documentclass| statement!
The argument of |\childdocmain| should be left empty
(but it must be present).

%%%%%%%%%%%%%%%%%%%%%%%%%%%%%%%%%%%%%%%%
\DescribeMacro{\childdocof}
Furthermore, add the commands
\begin{center}
\begin{tabular}{l}
|\input{childdoc.def}|\\
|\childdocof{|\textit{main}|}|\\
\end{tabular}
\end{center}
at the top of every child file \textit{child}
which is included by |\include{|\textit{child}|}|
from within the main file
(or at least for those files to be compiled individually).
The argument \textit{main} must be the filename of the main file.

There are a couple of
considerations in setting up the main and child documents:

%%%%%%%%%%%%%%%%%%%%%%%%%%%%%%%%%%%%%%%%
\paragraph{Restrictions.}

Please note the following restrictions:
\begin{itemize}
\item
|\childdocmain| must be called with one argument \textit{main}
to ensure compatibility with earlier version of the package.
It must either be empty (|\childdocmain{}|)
or precisely match the filename of the main file in which it is specified.
See \secref{sec:detection} for further information.
\item
The filename \textit{main} must be specified without the |.tex| extension.
\item
The filename \textit{main} is case sensitive
(even in case-insensitive file systems)
due to internal string comparison.
\item
The argument \textit{main} should be fully expanded, it cannot be a macro.
\item
Subdirectories and special characters should be avoided in filenames.
\item
The command |\childdocmain{|\textit{main}|}| must be followed by a whitespace.
It should not be followed immediately by another command
or by a comment mark `|%|'.
This is because the \TeX{} parser reads the token immediately following
the argument of |\childdocmain| and puts it
at the beginning of every child section;
however, a white\-space is ignored.
\end{itemize}

%%%%%%%%%%%%%%%%%%%%%%%%%%%%%%%%%%%%%%%%
\paragraph{Content of Main File.}

It is advisable to place all content in the child files included by |\include|.
Any output contained in the main file will appear in all child documents
unless suppressed manually;
it cannot be suppressed automatically by the |\includeonly| directive
and thus should normally be avoided.
A method to include some content in the main file
by means of conditional processing is described in \secref{sec:conditional}.

%%%%%%%%%%%%%%%%%%%%%%%%%%%%%%%%%%%%%%%%
\paragraph{Page Numbering.}

When only a part of the document is compiled,
the appropriate numbering of pages
(as well as other status parameters)
is determined from the |.aux| files.
The latter contain information from previous passes.
However this information needs to propagate through
all intermediate child documents.
Therefore the page numbering in child documents may well
be inconsistent until the complete document is compiled at least once.

A useful (if unconventional) way to always ensure a consistent
page numbering is to restart the numbering in each child document
and denote the pages by `\textit{child}|.|\textit{page}'
where \textit{child} represents the chapter/section number of the child file.
This can be achieved by the command
|\numberwithin{page}{|\textit{child}|}|
of the \textsf{amsmath} package
where \textit{child} can be |chapter| or |section|
depending on the chosen structuring.
Alternatively, one can modify the macro |\thepage| appropriately
and reset the counter |page| at the start of each child file.

%%%%%%%%%%%%%%%%%%%%%%%%%%%%%%%%%%%%%%%%%%%%%%%%%%%%%%%%%%%%%%%%%%%%%%%%%%%%%%%%
\subsection{Conditional Processing}
\label{sec:conditional}

The package provides a mechanism to compile different versions
of a document. To customise the versions further some conditional processing
can come in handy to distinguish which version is being compiled.
The package provides two macros to describe the compilation context:

%%%%%%%%%%%%%%%%%%%%%%%%%%%%%%%%%%%%%%%%
\DescribeMacro{\ifchilddoc}
The conditional |\ifchilddoc| distinguishes between the compilation of
child documents and the main document:
%
\begin{center}
|\ifchilddoc |\textit{child-code}| |[|\||else |\textit{main-code}]| \||fi|
\end{center}

%%%%%%%%%%%%%%%%%%%%%%%%%%%%%%%%%%%%%%%%
\DescribeMacro{\childdocname}
\DescribeMacro{\childdocjob}
The macro |\childdocname| contains the filename (without extension)
of the main or child file being processed.
Note that |\childdocjob| will always contain the name of the main file.

%%%%%%%%%%%%%%%%%%%%%%%%%%%%%%%%%%%%%%%%
\paragraph{Title Page.}

Conditional processing can be used to include a title or banner page
in the main document when proper precautions are taken.
Importantly, the code in the main file should ensure that the page counter
(as well as other status parameters which are stored in the |.aux| files)
takes the same value after the conditional processing.
Otherwise the page numbers may take divergent values
depending on which part is compiled.

For example, a title page could be declared by:
%
\begin{center}
\begin{tabular}{l}
|\ifchilddoc\||else|\\
|\addtocounter{page}{-1}|\\
\textit{code for title page}\\
|\newpage|\\
|\||fi|
\end{tabular}
\end{center}
%
A banner page for the child documents can be generated by:
%
\begin{center}
\begin{tabular}{l}
|\ifchilddoc|\\
|\addtocounter{page}{-1}|\\
\textit{code for banner page}\\
|\newpage|\\
|\||fi|
\end{tabular}
\end{center}
%
Here one could write a message such as:
\begin{center}
|This is the part \childdocname{} of \childdocjob{}.|
\end{center}

%%%%%%%%%%%%%%%%%%%%%%%%%%%%%%%%%%%%%%%%%%%%%%%%%%%%%%%%%%%%%%%%%%%%%%%%%%%%%%%%
\subsection{Flags}
\label{sec:flags}

The package makes it easy to generate different versions
of the main or child documents.
To this end compilation flags can be defined
and assigned different default values.
They will be particularly useful in conjunction
with the forwarding mechanism described in \secref{sec:forward}.

For example, it may be useful to have a flag |\version|
which can be set to |draft| or |final|.
The document source will contain some conditional code
depending on the value of |\version|.
Suppose further, the flag should default to |final| for the main file
and to |draft| for child files
which is a natural assignment for editing the document.
This is achieved by placing the following code
in the preamble of the main document
(below the |\childdocmain| directive):
%
\begin{center}
\begin{tabular}{l}
|\ifchilddoc|\\
|\providecommand{\version}{draft}|\\
|\||else|\\
|\providecommand{\version}{final}|\\
|\||fi|
\end{tabular}
\end{center}
%
The definition by |\providecommand| makes sure
that previous definitions are not overwritten.
Further statements |\providecommand{\version}{...}|
can thus be added before the above code to override it.

For the main file, one might add a line
(between |\childdocmain| and the above block)
%
\begin{center}
|%\ifchilddoc\||else\providecommand{\version}{draft}\||fi|
\end{center}
%
which can be uncommented to produce a draft version.
Likewise one can add a line to the very top of a child file
(above the |\childdocof{|\textit{main}|}| directive)
%
\begin{center}
|%\providecommand{\version}{final}|
\end{center}
%
which can be uncommented to produce the final version of this child document.

%%%%%%%%%%%%%%%%%%%%%%%%%%%%%%%%%%%%%%%%%%%%%%%%%%%%%%%%%%%%%%%%%%%%%%%%%%%%%%%%
\subsection{Forwarding}
\label{sec:forward}

Different versions of the main or child documents
using compilation flags as described in \secref{sec:flags}
can be (permanently) stored in different files
for convenient compilation, viewing and distribution.
To this end, the package defines a command
to pass on compilation to a different file:

%%%%%%%%%%%%%%%%%%%%%%%%%%%%%%%%%%%%%%%%
\DescribeMacro{\childdocforward}
The command |\childdocforward| redirects processing to
another source file:
%
\begin{center}
\begin{tabular}{l}
|\input{childdoc.def}|\\
|\childdocforward[|\textit{main}|]{|\textit{dest}|}|\\
\end{tabular}
\end{center}
%
The argument \textit{dest} is the destination file
(without extension).
It should be the main file or one of the child files.
Note that further \textsf{childdoc} directives
such as |\childdocof| and |\childdocforward|
in the indicated file will be processed in this form.
The optional argument \textit{main}
passes on directly to the main file \textit{main}
while pretending to compile the child \textit{dest}.
This form behaves as if \textit{dest}
issues |\childdocof{|\textit{main}|}| right away,
and no further \textsf{childdoc} directives will be processed.

%%%%%%%%%%%%%%%%%%%%%%%%%%%%%%%%%%%%%%%%
\DescribeMacro{\...prefix}
In the alternative form |\childdocforwardprefix|,
%
\begin{center}
\begin{tabular}{l}
|\input{childdoc.def}|\\
|\childdocforwardprefix[|\textit{main}|]{|\textit{prefix}|}{|\textit{dest}|}|
\end{tabular}
\end{center}
%
the destination file is determined by a pattern
depending on the current file:
To make this work, the current file must be called
`{\textit{prefix}\hspace{0.2em}\textit{suffix}}'
with \textit{prefix} matching precisely the argument.
Processing is then passed on to the file
`{\textit{dest}\hspace{0.2em}\textit{suffix}}'.
Surely, the same effect is achieved by
directly specifying the
argument `{\textit{dest}\hspace{0.2em}\textit{suffix}}'
in the first form.
However, that requires to set up a different file
for each child. With the alternative form of the command
all these files can have exactly the same content
which simplifies setting them up and maintaining them.

For example, the following file |draft.tex|
with a compilation flag |\version| as described in \secref{sec:flags}
compiles the main document as a draft:
%
\begin{center}
\begin{tabular}{l}
|\def\version{draft}|\\
|\input{childdoc.def}|\\
|\childdocforward{|\textit{main}|}|
\end{tabular}
\end{center}
%
Likewise, the following files |final|\textit{nn}|.tex|
compile the final version of the child document
|child|\textit{nn}|.tex|:
%
\begin{center}
\begin{tabular}{l}
|\def\version{final}|\\
|\input{childdoc.def}|\\
|\childdocforwardprefix{final}{child}|
\end{tabular}
\end{center}
%

Note that when several versions of a main file and/or of each child file
are to be generated, it may be convenient to set up a |Makefile| or
shell script to automatise the process.

%%%%%%%%%%%%%%%%%%%%%%%%%%%%%%%%%%%%%%%%%%%%%%%%%%%%%%%%%%%%%%%%%%%%%%%%%%%%%%%%
\subsection{Command Line Processing}
\label{sec:commandline}

The effect of redirection files can also be achieved by invoking
the \LaTeX{} compiler with a more elaborate command line.
Most conveniently this should be done as part
of a shell script or a |Makefile|.

When using \textsf{childdoc} in the main file, the following
command lines effectively perform a redirection
(note that depending on the shell being used,
backslashes may have to be doubled: `|\|' $\to$ `|\\|'):
%
\begin{center}
|... -jobname "|\textit{target}|" |\\|"|[\textit{flags}]%
|\input{childdoc.def}\childdocforward[|\textit{main}|]{|\textit{dest}|}"|
\end{center}
%
Here \textit{target} is the name of the output file,
\textit{main} is the name of the main file
and \textit{dest} is the name of the main or child file to be processed
(all filenames without extensions).
The optional argument \textit{main} can be omitted
if \textit{main} matches \textit{dest}.
Optionally, compilation \textit{flags} can be defined via |\def| commands.
This command line makes the \TeX{} engine believe
it is compiling the file \textit{target}
whose content is specified as the latter parameter.
The provided code then forwards the processing to
\textit{main} or \textit{dest} as described in \secref{sec:forward}.

%%%%%%%%%%%%%%%%%%%%%%%%%%%%%%%%%%%%%%%%%%%%%%%%%%%%%%%%%%%%%%%%%%%%%%%%%%%%%%%%
\subsection{Include by Input}
\label{sec:input}

Including child documents by |\include| has some restrictions by design.
Most notably, the content of a child document always occupies
its own set of pages; pages cannot be shared between child documents.
Usually, this behaviour makes perfect sense
because each child document contain an essential part of the document.
However, in some situations it may be desirable to compose
a document from a collection of parts
without having mandatory page breaks between then.
For this case, the package
provides a mechanism to include parts
by |\input| which can also be processed individually.
However, by construction this mechanism
requires manual handling of the content to be output.

%%%%%%%%%%%%%%%%%%%%%%%%%%%%%%%%%%%%%%%%
\DescribeMacro{\ifchilddocmanual}
The main file should be prepared as usual, see \secref{sec:include}.
However, the document body must make a distinction
between processing of an individual part and of the main document, e.g.:
%
\begin{center}
\begin{tabular}{l}
|\ifchilddocmanual|\\
|\input{\childdocname}|\\
|\||else|\\
\textit{document body with }|\input{|\textit{part}|}|\\
|\||fi|
\end{tabular}
\end{center}
%
The conditional |\ifchilddocmanual| is true whenever
a part to be included by |\input| is being compiled,
and the name of the part is stored in |\childdocname|.

%%%%%%%%%%%%%%%%%%%%%%%%%%%%%%%%%%%%%%%%
\DescribeMacro{\childdocby}
Each part to be included by |\input| should start with:
%
\begin{center}
\begin{tabular}{l}
|\input{childdoc.def}|\\
|\childdocby{|\textit{main}|}|\\
\end{tabular}
\end{center}
%
The directive |\childdocby| is similar to |\childdocof|
described in \secref{sec:include},
but the subsequent selection of content must be done manually.
To that end, both |\ifchilddoc| and |\ifchilddocmanual|
will be true upon processing of a part,
and the name of the part is stored in |\childdocname|.
Note that |\jobname| will be set to the filename of the current part
so that each part receives an individual |.aux| file
that does not interfere with the |.aux| file(s) of the main document.
This behaviour can be altered by the alternative form
|\childdocby[*]{|\textit{main}|}| (with a non-empty optional argument)
which uses the |.aux| file of the main document
by setting |\jobname| to \textit{main}.

%%%%%%%%%%%%%%%%%%%%%%%%%%%%%%%%%%%%%%%%%%%%%%%%%%%%%%%%%%%%%%%%%%%%%%%%%%%%%%%%
\subsection{Driver Development}
\label{sec:driver}

The \textsf{childdoc} mechanism can also be use for the development
of definition files such as \LaTeX{} styles or classes.
This case differs from the above setup with multiple parts
included by |\include| in that no |\includeonly| should be invoked.
This can be achieved by starting the include file
(before |\ProvidesPackage|) with:
%
\begin{center}
\begin{tabular}{l}
|\input{childdoc.def}|\\
|\childdocforward{|\textit{main}|}|\\
\end{tabular}
\end{center}
%
or alternatively with:
%
\begin{center}
\begin{tabular}{l}
|\input{childdoc.def}|\\
|\childdocby{|\textit{main}|}|\\
\end{tabular}
\end{center}
%
Both forms have slightly different effects as described above.
The main file is prepared as usual, see \secref{sec:include}.

%%%%%%%%%%%%%%%%%%%%%%%%%%%%%%%%%%%%%%%%%%%%%%%%%%%%%%%%%%%%%%%%%%%%%%%%%%%%%%%%
\subsection{Legacy Detection}
\label{sec:detection}

The directive |\childdocmain| in the main file can detect
whether the complete document or merely a child is to be compiled
even without using the directive |\childdocof|.
This method is deprecated because it is less robust
and there is no compelling reason to use it;
it is merely provided for backward compatibility
and it may be removed in future versions.

If the detection mechanism is to be used,
it is mandatory to correctly specify
the filename of the main file as the argument of |\childdocmain|:
%
\begin{center}
\begin{tabular}{l}
|\input{childdoc.def}|\\
|\childdocmain{|\textit{main}|}|\\
\end{tabular}
\end{center}
%
If |\jobname| does not match the argument \textit{main} of |\childdocmain|,
it is assumed that |\jobname| points to the child file to be compiled.
When using |\childdocmain| with the main file specified as argument,
it suffices to start a child file
with just |\input{|\textit{main}|}|
without loading of the package and using |\childdocof|.
If instead all processing is done
with the appropriate \textsf{childdoc} directives,
the argument of \textit{main} of |\childdocmain| can be empty.

An alternative version of the command line processing described
in \secref{sec:commandline} using the detection mechanism reads:
%
\begin{center}
|... -jobname "|\textit{target}|" "|[\textit{flags}]%
[|\def\jobname{|\textit{dest}|}|]|\input{|\textit{main}|}"|
\end{center}

%%%%%%%%%%%%%%%%%%%%%%%%%%%%%%%%%%%%%%%%%%%%%%%%%%%%%%%%%%%%%%%%%%%%%%%%%%%%%%%%
\subsection{Manual Code}
\label{sec:manual}

In case one cannot be certain whether the definitions file |childdoc.def|
is installed on the target \TeX{} distribution
and one prefers not to ship it,
it is conceivable to paste a few relevant commands into the sources.

To that end, drop all statements |\input{childdoc.def}|
and perform the replacements as outlined below.
Instead of |\childdocmain{|\textit{main}|}| add the following code
to the top of the main file:
%
\begin{center}
\begin{tabular}{l}
|\||ifdefined\childdocname\endinput\||fi\newif\ifchilddoc|\\
|\edef\childdocname{\scantokens\expandafter{\jobname\noexpand}}|\\
|\def\childdocmain{|\textit{main}|}\||ifx\childdocmain\childdocname\||else|\\
|\childdoctrue\includeonly{\childdocname}\let\jobname\childdocmain\||fi|\\
\end{tabular}
\end{center}
%
Instead of |\childdocof{|\textit{main}|}| just include the main file
at the top of each child file:
%
\begin{center}
|\input{|\textit{main}|}|
\end{center}
%
A simple redirection |\childdocforward{|\textit{dest}|}| is achieved by:
%
\begin{center}
|\def\jobname{|\textit{dest}|}\input{\jobname}|
\end{center}
%
The redirection with prefix
|\childdocforwardprefix[|\textit{prefix}|]{|\textit{dest}|}|
is accomplished by:
%
\begin{center}
\begin{tabular}{l}
|{\edef\jobname{\scantokens\expandafter{\jobname\noexpand}}|\\
|\def\redirectjob |\textit{prefix}|#1~~~{\gdef\jobname{|\textit{dest}|#1}}|\\
|\expandafter\redirectjob\jobname~~~}\input{\jobname}|
\end{tabular}
\end{center}

In an alternative approach,
child documents can be compiled by a specific command line
without additional code or specific definitions:
%
\begin{center}
|... -jobname "|\textit{target}|" "|[\textit{flags}]%
|\includeonly{|\textit{dest}|}\input{|\textit{main}|}"|
\end{center}
%

%%%%%%%%%%%%%%%%%%%%%%%%%%%%%%%%%%%%%%%%%%%%%%%%%%%%%%%%%%%%%%%%%%%%%%%%%%%%%%%%
%%%%%%%%%%%%%%%%%%%%%%%%%%%%%%%%%%%%%%%%%%%%%%%%%%%%%%%%%%%%%%%%%%%%%%%%%%%%%%%%
\section{Information}

%%%%%%%%%%%%%%%%%%%%%%%%%%%%%%%%%%%%%%%%%%%%%%%%%%%%%%%%%%%%%%%%%%%%%%%%%%%%%%%%
\subsection{Copyright}

Copyright \copyright{} 2017--2018 Niklas Beisert

This work may be distributed and/or modified under the
conditions of the \LaTeX{} Project Public License, either version 1.3
of this license or (at your option) any later version.
The latest version of this license is in
  \url{http://www.latex-project.org/lppl.txt}
and version 1.3 or later is part of all distributions of \LaTeX{}
version 2005/12/01 or later.

This work has the LPPL maintenance status `maintained'.

The Current Maintainer of this work is Niklas Beisert.

This work consists of the files |README.txt|, |childdoc.ins| and |childdoc.dtx|
as well as the derived files |childdoc.def|, |cdocsamp.tex|
with |cdocsch1.tex|, |cdocsch2.tex|, |cdocspt3.tex|, |cdocspt4.tex|,
|cdocsdrf.tex|, |cdocsfn1.tex|, |cdocsfn2.tex|
as well as |childdoc.pdf|.

%%%%%%%%%%%%%%%%%%%%%%%%%%%%%%%%%%%%%%%%%%%%%%%%%%%%%%%%%%%%%%%%%%%%%%%%%%%%%%%%
\subsection{Files and Installation}

The package consists of the files:
%
\begin{center}
\begin{tabular}{ll}
    |README.txt|   & readme file \\
    |childdoc.ins| & installation file \\
    |childdoc.dtx| & source file \\
    |childdoc.def| & definition file \\
    |cdocsamp.tex| & sample main file \\
    |cdocsch1.tex| & sample include file \\
    |cdocsch2.tex| & sample include file \\
    |cdocspt3.tex| & sample part file \\
    |cdocspt4.tex| & sample part file \\
    |cdocsdrf.tex| & sample redirection file \\
    |cdocsfn1.tex| & sample redirection file \\
    |cdocsfn2.tex| & sample redirection file \\
    |childdoc.pdf| & manual
\end{tabular}
\end{center}
%
The distribution consists of the files
|README.txt|, |childdoc.ins| and |childdoc.dtx|.
%
\begin{itemize}
\item
Run (pdf)\LaTeX{} on |childdoc.dtx|
to compile the manual |childdoc.pdf| (this file).
\item
Run \LaTeX{} on |childdoc.ins| to create the definitions file |childdoc.def|
and the sample |cdocsamp.tex| with include files
|cdocsch1.tex|, |cdocsch2.tex|, |cdocspt3.tex|, |cdocspt4.tex|,
|cdocsdrf.tex|, |cdocsfn1.tex|, |cdocsfn2.tex|.
Then copy the file |childdoc.def| to an appropriate directory of your \LaTeX{}
distribution, e.g.\ \textit{texmf-root}|/tex/latex/childdoc|.
\end{itemize}

%%%%%%%%%%%%%%%%%%%%%%%%%%%%%%%%%%%%%%%%%%%%%%%%%%%%%%%%%%%%%%%%%%%%%%%%%%%%%%%%
\subsection{Related CTAN Packages}

There are several other packages which offer a similar functionality:
%
\begin{itemize}
\item
The packages
\href{http://ctan.org/pkg/docmute}{\textsf{docmute}},
\href{http://ctan.org/pkg/includex}{\textsf{includex}} and
\href{http://ctan.org/pkg/standalone}{\textsf{standalone}}
provide commands to include only the document body of
a child file thus allowing both files to be compiled individually.
\item
The packages \href{http://ctan.org/pkg/subdocs}{\textsf{subdocs}}
and \href{http://ctan.org/pkg/subfiles}{\textsf{subfiles}}
provide structures in which the main and child documents can be
encapsulated and allowing them to be compiled individually.
The inclusion mechanism is different from the conventional |\include|.
\item
The package \href{http://ctan.org/pkg/combine}{\textsf{combine}}
is an elaborate solution to combine several documents into one.
\end{itemize}
%
See also the CTAN topic \href{http://ctan.org/topic/subdocs}{\textsf{subdocs}}
for further related packages.
The present package differs from the above solutions in that
a document structure constructed with the conventional |\include| mechanism
just needs two extra commands at the top of every file
such that all constituent files can be compiled individually.

%%%%%%%%%%%%%%%%%%%%%%%%%%%%%%%%%%%%%%%%%%%%%%%%%%%%%%%%%%%%%%%%%%%%%%%%%%%%%%%%
%\subsection{Feature Suggestions}
%
%The following is a list of features which may be useful for future
%versions of this package:
%%
%\begin{itemize}
%\item
%\ldots
%\end{itemize}

%%%%%%%%%%%%%%%%%%%%%%%%%%%%%%%%%%%%%%%%%%%%%%%%%%%%%%%%%%%%%%%%%%%%%%%%%%%%%%%%
\subsection{Revision History}

%%%%%%%%%%%%%%%%%%%%%%%%%%%%%%%%%%%%%%%%
\paragraph{v2.0:} 2018/12/30

\begin{itemize}
\item
immediate forward processing
\item
added |\childdocby| mechanism
\item
manual restructured
\end{itemize}

%%%%%%%%%%%%%%%%%%%%%%%%%%%%%%%%%%%%%%%%
\paragraph{v1.6:} 2018/01/17

\begin{itemize}
\item
application for development of include files
\item
corrections to manual
\end{itemize}

%%%%%%%%%%%%%%%%%%%%%%%%%%%%%%%%%%%%%%%%
\paragraph{v1.5:} 2017/05/21

\begin{itemize}
\item
more complete structuring introduced
\item
|\childdocof| introduced
\item
|\childdoc| renamed to |\childdocmain|
\item
|\childredirect| renamed to |\childdocforward| and |\childdocforwardprefix|
and functionality expanded
\end{itemize}

%%%%%%%%%%%%%%%%%%%%%%%%%%%%%%%%%%%%%%%%
\paragraph{v1.0:} 2017/04/27

\begin{itemize}
\item
manual and install package
\item
first version published on CTAN
\end{itemize}

%%%%%%%%%%%%%%%%%%%%%%%%%%%%%%%%%%%%%%%%
\paragraph{v0.6:} 2017/04/26

\begin{itemize}
\item
redirection mechanism added
\end{itemize}

%%%%%%%%%%%%%%%%%%%%%%%%%%%%%%%%%%%%%%%%
\paragraph{v0.5:} 2017/04/26

\begin{itemize}
\item
functionality in definition file
\end{itemize}


%%%%%%%%%%%%%%%%%%%%%%%%%%%%%%%%%%%%%%%%%%%%%%%%%%%%%%%%%%%%%%%%%%%%%%%%%%%%%%%%
%%%%%%%%%%%%%%%%%%%%%%%%%%%%%%%%%%%%%%%%%%%%%%%%%%%%%%%%%%%%%%%%%%%%%%%%%%%%%%%%
%%%%%%%%%%%%%%%%%%%%%%%%%%%%%%%%%%%%%%%%%%%%%%%%%%%%%%%%%%%%%%%%%%%%%%%%%%%%%%%%
\appendix

\settowidth\MacroIndent{\rmfamily\scriptsize 000\ }

 \DocInput{childdoc.dtx}

\end{document}
%</driver>
% \fi
%
% %%%%%%%%%%%%%%%%%%%%%%%%%%%%%%%%%%%%%%%%%%%%%%%%%%%%%%%%%%%%%%%%%%%%%%%%%%%%%%
% %%%%%%%%%%%%%%%%%%%%%%%%%%%%%%%%%%%%%%%%%%%%%%%%%%%%%%%%%%%%%%%%%%%%%%%%%%%%%%
% \section{Sample}
%\iffalse
%<*samplemain>
%\fi
%
% The following presents a sample document
% with two chapters, two parts, a title page,
% a compile flag as well as three forwarding files to set the flag.
% It consists of eight |.tex| files:
% \begin{center}
% \begin{tabular}{ll}
% |cdocsamp.tex|&main file\\
% |cdocsch1.tex|&include file for chapter 1\\
% |cdocsch2.tex|&include file for chapter 2\\
% |cdocspt3.tex|&include file for part 3\\
% |cdocspt4.tex|&include file for part 4\\
% |cdocsdrf.tex|&forwarding file for main file in draft mode\\
% |cdocsfi1.tex|&forwarding file for final version of chapter 1\\
% |cdocsfi2.tex|&forwarding file for final version of chapter 2\\
% \end{tabular}
% \end{center}
% Each of the eight files can be compiled directly by the \LaTeX{} compiler.
%
% %%%%%%%%%%%%%%%%%%%%%%%%%%%%%%%%%%%%%%
% \paragraph{Main File.}
%
% The main file is called |cdocsamp.tex|.
%
% Load the \textsf{childdoc} definitions and
% declare the filename for the main document:
%    \begin{macrocode}
\input{childdoc.def}
\childdocmain{}
%    \end{macrocode}

% Optional override for |\version| flag:
%    \begin{macrocode}
%%\ifchilddoc\else\providecommand{\version}{draft}\fi
%    \end{macrocode}

% Define the default values for the |\version| flag
% (|final| for the main file and |draft| for childs):
%    \begin{macrocode}
\ifchilddoc
\providecommand{\version}{draft}
\else
\providecommand{\version}{final}
\fi
%    \end{macrocode}

% Load the standard document class:
%    \begin{macrocode}
\documentclass[12pt]{article}
%    \end{macrocode}

% Start the document body:
%    \begin{macrocode}
\begin{document}
%    \end{macrocode}

% Declare a title page.
% Print title, part of document being processed and version flag:
%    \begin{macrocode}
\addtocounter{page}{-1}
\begin{center}
{\LARGE\bfseries{}childdoc example\par}
\vspace{1cm}
\ifchilddoc
\ifchilddocmanual part\else chapter\fi:
`\childdocname' of `\childdocjob'\par
\else
main document: `\childdocjob'\par
\fi
version: \version\par
\end{center}
\newpage
%    \end{macrocode}

% Manually include selected file,
% otherwise process as usual:
%    \begin{macrocode}
\ifchilddocmanual
\section*{part `\childdocname'}
\input{\childdocname}
\else
%    \end{macrocode}

% Include the two chapters:
%    \begin{macrocode}
\include{cdocsch1}
\include{cdocsch2}
%    \end{macrocode}

% Include the two parts unless only chapters should be displayed:
%    \begin{macrocode}
\ifchilddoc\else
\section{part three}
\input{cdocspt3}
\section{part four}
\input{cdocspt4}
\fi
%    \end{macrocode}

% Process as usual until here:
%    \begin{macrocode}
\fi
%    \end{macrocode}

% End of document body:
%    \begin{macrocode}
\end{document}
%    \end{macrocode}
%\iffalse
%</samplemain>
%\fi
%
% %%%%%%%%%%%%%%%%%%%%%%%%%%%%%%%%%%%%%%
% \paragraph{Chapter Include Files.}
%
% The include files are called |cdocsch1.tex| and |cdocsch2.tex|.
%
%\iffalse
%<*samplechap1|samplechap2>
%\fi

% Optional override for |\version| flag:
%    \begin{macrocode}
%%\providecommand{\version}{final}
%    \end{macrocode}

% Include the main document:
%    \begin{macrocode}
\input{childdoc.def}
\childdocof{cdocsamp}
%    \end{macrocode}

%\iffalse
%</samplechap1|samplechap2>
%\fi
%
%\iffalse
%<*samplechap1>
%\fi
% Some text for chapter 1:
%    \begin{macrocode}
\section{one}
some text in chapter one
%    \end{macrocode}

%\iffalse
%</samplechap1>
%\fi
% Some text for chapter 2:
%\iffalse
%<*samplechap2>
%\fi
%    \begin{macrocode}
\section{two}
more text in chapter two
%    \end{macrocode}

%\iffalse
%</samplechap2>
%\fi
%
% %%%%%%%%%%%%%%%%%%%%%%%%%%%%%%%%%%%%%%
% \paragraph{Part Include Files.}
%
% The include files are called |cdocspt3.tex| and |cdocspt4.tex|.
%
%\iffalse
%<*samplepart3|samplepart4>
%\fi

% Optional override for |\version| flag:
%    \begin{macrocode}
%%\providecommand{\version}{final}
%    \end{macrocode}

% Include the main document:
%    \begin{macrocode}
\input{childdoc.def}
\childdocby{cdocsamp}
%    \end{macrocode}

%\iffalse
%</samplepart3|samplepart4>
%\fi
%
%\iffalse
%<*samplepart3>
%\fi
% Some text for part 3:
%    \begin{macrocode}
some text in part three
%    \end{macrocode}

%\iffalse
%</samplepart3>
%\fi
% Some text for part 4:
%\iffalse
%<*samplepart4>
%\fi
%    \begin{macrocode}
more text in part four
%    \end{macrocode}

%\iffalse
%</samplepart4>
%\fi
%
% %%%%%%%%%%%%%%%%%%%%%%%%%%%%%%%%%%%%%%
% \paragraph{Forwarding for a Complete Draft.}
%
% The following forwarding file |cdocsdrf.tex|
% compiles the main document in draft mode:
%\iffalse
%<*sampledraft>
%\fi
%    \begin{macrocode}
\def\version{draft}
\input{childdoc.def}
\childdocforward{cdocsamp}
%    \end{macrocode}

%\iffalse
%</sampledraft>
%\fi
%
% %%%%%%%%%%%%%%%%%%%%%%%%%%%%%%%%%%%%%%
% \paragraph{Forwarding for Final Version of the Chapters.}
%
% The following forwarding files |cdocsfn1.tex| and |cdocsfn2.tex|
% (with identical content)
% compile the final versions of the child documents
% |cdocsch1.tex| and |cdocsch2.tex|, respectively:
%\iffalse
%<*samplefinal>
%\fi
%    \begin{macrocode}
\def\version{final}
\input{childdoc.def}
\childdocforwardprefix[cdocsamp]{cdocsfn}{cdocsch}
%    \end{macrocode}

%\iffalse
%</samplefinal>
%\fi
%
% %%%%%%%%%%%%%%%%%%%%%%%%%%%%%%%%%%%%%%
% \paragraph{Command Line Processing.}
%
% The following three command lines generate the output files
% |cdocscld|, |cdocscl1| and |cdocscl2|
% which should be identical to
% |cdocsdrf|, |cdocsch1| and |cdocsfn2|, respectively:
% \begin{center}
% \begin{tabular}{l}
% |latex -jobname cdocscld \|\\
% |  "\def\version{draft}\input{childdoc.def}\childdocforward{cdocsamp}"|\\
% |latex -jobname cdocscl1 \|\\
% |  "\input{childdoc.def}\childdocforward[cdocsamp]{cdocsch1}"|\\
% |latex -jobname cdocscl2 \|\\
% |  "\def\version{final}\input{childdoc.def}\childdocforward{cdocsch2}"|
% \end{tabular}
% \end{center}
% Note that the trailing backslash on each first line
% merely continues the input to the second line
% (for convenient cut ant paste).
% Furthermore, the command |latex| can be replaced by any
% of its alternative versions such as |pdflatex|.
%
% %%%%%%%%%%%%%%%%%%%%%%%%%%%%%%%%%%%%%%%%%%%%%%%%%%%%%%%%%%%%%%%%%%%%%%%%%%%%%%
% %%%%%%%%%%%%%%%%%%%%%%%%%%%%%%%%%%%%%%%%%%%%%%%%%%%%%%%%%%%%%%%%%%%%%%%%%%%%%%
% \section{Implementation}
%\iffalse
%<*package>
%\fi
%
% This section describes the definitions file |childdoc.def|.

% The definitions cannot be loaded using |\usepackage| or |\RequirePackage|
% which has a mechanism to prevent loading a style file more than once.
% When loading the definitions by means of |\input|
% multiple instances have to be prevented manually:
%\iffalse
%This code needs to be before the `\ProvidesFile' directive
%which is defined at the beginning of this file.
%Therefore it is also placed there and commented out here.
%</package>
%<*discard>
%\fi
%    \begin{macrocode}
\ifdefined\childdocmain\endinput\fi
%    \end{macrocode}
%\iffalse
%</discard>
%<*package>
%\fi
%
% \macro{\ifchilddoc}
% \macro{\ifchilddocmanual}
% The conditional |\ifchilddoc| tells whether a
% child (true) or main (false) document is being compiled.
% The conditional |\ifchilddocmanual| tells whether
% the |\includeonly| mechanism is used (false) or
% the selection of child files must be performed manually (true).
% The definitions initialise to false:
%    \begin{macrocode}
\newif\ifchilddoc
\newif\ifchilddocmanual
%    \end{macrocode}

% \macro{\childdocname}
% \macro{\childdocjob}
% The macro |\childdocname| stores the name of the main document
% to be compiled. The macro |\childdocjob| stores the name of
% the document on which the \LaTeX{} compiler was originally invoked.
% The content of |\jobname| cannot be compared
% to filenames specified in the source due to different catcodes.
% The following code rescans |\jobname|, stores the result
% in |\childdocname| and saves a copy in |\childdocjob|:
%    \begin{macrocode}
\edef\childdocname{\scantokens\expandafter{\jobname\noexpand}}
\let\childdocjob\childdocname
%    \end{macrocode}

% \macro{\childdocdisable}
% The macro |\childdocdisable| prevents the main file
% from being processed more than once.
% At this stage, the main document command |\childdocmain|
% is assumed to be called once again where it should do nothing.
% Any subsequent call to it should prevent
% a secondary processing of the main document
% It overwrites the forwarding commands
% |\childdocof| and |\childdocforward|
% with empty macros to prevent further inclusions of the main document:
%    \begin{macrocode}
\newcommand{\childdocdisable}
{
  \renewcommand{\childdocmain}[1]{\renewcommand{\childdocmain}[1]{\endinput}}
  \renewcommand{\childdocof}[1]{}
  \renewcommand{\childdocby}[2][]{}
  \renewcommand{\childdocforward}[2][]{}
  \renewcommand{\childdocdisable}{}
}
%    \end{macrocode}

% \macro{\childdocmain}
% The macro |\childdocmain| is to be called at the top of the main file
% with nothing or the main filename (without extension) as argument.
% First, it breaks loops.
% If the argument is not empty and does not match |\childdocname|
% (which is set by the first inclusion of |childdoc.def|),
% |\ifchilddoc| is set to true, |\includeonly| is applied to the child file
% and |\jobname| is set to the main file
% (for proper handling of |.aux| files):
%    \begin{macrocode}
\newcommand{\childdocmain}[1]
{
  \childdocdisable\childdocmain{}
  \if?#1?\else
    \begingroup
      \def\childdoctmp{#1}
      \ifx\childdoctmp\childdocname
        \def\childdoctmp{}
      \else
        \def\childdoctmp
        {
          \childdoctrue
          \includeonly{\childdocname}
          \def\childdocjob{#1}
          \def\jobname{#1}
        }
      \fi
      \expandafter
    \endgroup
    \childdoctmp
  \fi
}
%    \end{macrocode}

% \macro{\childdocof}
% The command |\childdocof| redirects
% compilation to the main file |#1|.
%    \begin{macrocode}
\newcommand{\childdocof}[1]
{
  \childdocdisable
  \childdoctrue
  \includeonly{\childdocname}
  \def\jobname{#1}
  \def\childdocjob{#1}
  \input{#1}
}
%    \end{macrocode}

% \macro{\childdocby}
% The command |\childdocby| ....
%    \begin{macrocode}
\newcommand{\childdocby}[2][]
{
  \childdocdisable
  \childdoctrue
  \childdocmanualtrue
  \if?#1?\else
    \def\jobname{#2}
  \fi
  \def\childdocjob{#2}
  \input{#2}
  \endinput
}
%    \end{macrocode}

% \macro{\childdocforward}
% The command |\childdocforward| redirects
% compilation to the main file or
% (if the optional argument is given) a child file.
% Parameters are set as if the main file
% or a child file starting with |\childdocof| was compiled.
% Then compilation is handed over to the main file:
%    \begin{macrocode}
\newcommand{\childdocforward}[2][]
{
  \begingroup
    \if?#1?
      \def\childdoctmp
      {
        \def\childdocname{#2}
        \def\childdocjob{#2}
        \def\jobname{#2}
        \input{#2}
        \endinput
      }
    \else
      \def\childdoctmp
      {
        \childdocdisable
        \def\childdocname{#2}
        \childdoctrue
        \includeonly{#2}
        \def\childdocjob{#1}
        \def\jobname{#1}
        \input{#1}
        \endinput
      }
    \fi
    \expandafter
  \endgroup
  \childdoctmp
}
%    \end{macrocode}

% \macro{\childdocforwardprefix}
% The command |\childdocforwardprefix| redirects
% compilation to the main or a child file by means of a pattern.
% The prefix |#1| in the current filename is replaced by |#2|
% and the suffix of the current filename is kept
% (it is assumed that the filename does not contain the substring `|~~~|'
% which is used as a delimiter).
% Compilation is handed over to the new file by |\childdocforward|:
%    \begin{macrocode}
\newcommand{\childdocforwardprefix}[3][]
{
  \begingroup
    \def\childdocextract #2##1~~~{\def\childdoctmp{\childdocforward[#1]{#3##1}}}
    \expandafter\childdocextract\childdocname~~~
    \expandafter
  \endgroup
  \childdoctmp
}
%    \end{macrocode}

% \macro{\childdoc}
% The deprecated macro |\childdoc| is a legacy version of |\childdocmain|:
%    \begin{macrocode}
\newcommand{\childdoc}{\childdocmain}
%    \end{macrocode}

% \macro{\childdocredirect}
% The deprecated macro |\childdocredirect| is a legacy version
% of |\childdocforward| and |\childdocforwardprefix|:
%    \begin{macrocode}
\newcommand{\childdocredirect}[2][]
{
  \begingroup
    \if?#1?
      \def\childdoctmp{\childdocforward{#2}}
    \else
      \def\childdoctmp{\childdocforwardprefix{#1}{#2}}
    \fi
    \expandafter
  \endgroup
  \childdoctmp
}
%    \end{macrocode}

%\iffalse
%</package>
%\fi
%
\endinput
|\\
|\childdocmain{|\textit{main}|}|\\
\end{tabular}
\end{center}
%
If |\jobname| does not match the argument \textit{main} of |\childdocmain|,
it is assumed that |\jobname| points to the child file to be compiled.
When using |\childdocmain| with the main file specified as argument,
it suffices to start a child file
with just |\input{|\textit{main}|}|
without loading of the package and using |\childdocof|.
If instead all processing is done
with the appropriate \textsf{childdoc} directives,
the argument of \textit{main} of |\childdocmain| can be empty.

An alternative version of the command line processing described
in \secref{sec:commandline} using the detection mechanism reads:
%
\begin{center}
|... -jobname "|\textit{target}|" "|[\textit{flags}]%
[|\def\jobname{|\textit{dest}|}|]|\input{|\textit{main}|}"|
\end{center}

%%%%%%%%%%%%%%%%%%%%%%%%%%%%%%%%%%%%%%%%%%%%%%%%%%%%%%%%%%%%%%%%%%%%%%%%%%%%%%%%
\subsection{Manual Code}
\label{sec:manual}

In case one cannot be certain whether the definitions file |childdoc.def|
is installed on the target \TeX{} distribution
and one prefers not to ship it,
it is conceivable to paste a few relevant commands into the sources.

To that end, drop all statements |% \iffalse
%
% childdoc.dtx Copyright (C) 2017-2018 Niklas Beisert
%
% This work may be distributed and/or modified under the
% conditions of the LaTeX Project Public License, either version 1.3
% of this license or (at your option) any later version.
% The latest version of this license is in
%   http://www.latex-project.org/lppl.txt
% and version 1.3 or later is part of all distributions of LaTeX
% version 2005/12/01 or later.
%
% This work has the LPPL maintenance status `maintained'.
%
% The Current Maintainer of this work is Niklas Beisert.
%
% This work consists of the files childdoc.dtx and childdoc.ins
% and the derived files childdoc.def and cdocsamp.tex with
% cdocsch1.tex, cdocsch2.tex, cdocsdrf.tex, cdocsfn1.tex, cdocsfn2.tex.
%
%<package>\ifdefined\childdocmain\endinput\fi
%<package>\ProvidesFile{childdoc.def}[2018/12/30 v2.0 child document driver]
%<samplemain>\ProvidesFile{cdocsamp.tex}[2018/12/30 v2.0 sample for childdoc]
%<*driver>
%\ProvidesFile{childdoc.drv}[2018/12/30 v2.0 childdoc reference manual file]
\PassOptionsToClass{10pt,a4paper}{article}
\documentclass{ltxdoc}

\usepackage[margin=35mm]{geometry}
\usepackage{hyperref}
\usepackage{hyperxmp}
\usepackage[usenames]{color}

\hypersetup{colorlinks=true}
\hypersetup{pdfstartview=FitH}
\hypersetup{pdfpagemode=UseNone}
\hypersetup{pdfsource={}}
\hypersetup{pdflang={en-UK}}
\hypersetup{pdfcopyright={Copyright 2017-2018 Niklas Beisert.
  This work may be distributed and/or modified under the
  conditions of the LaTeX Project Public License, either version 1.3
  of this license or (at your option) any later version.}}
\hypersetup{pdflicenseurl={http://www.latex-project.org/lppl.txt}}
\hypersetup{pdfcontactaddress={ETH Zurich, ITP, HIT K,
  Wolfgang-Pauli-Strasse 27}}
\hypersetup{pdfcontactpostcode={8093}}
\hypersetup{pdfcontactcity={Zurich}}
\hypersetup{pdfcontactcountry={Switzerland}}
\hypersetup{pdfcontactemail={nbeisert@itp.phys.ethz.ch}}
\hypersetup{pdfcontacturl={http://people.phys.ethz.ch/\xmptilde nbeisert/}}

\newcommand{\secref}[1]{\hyperref[#1]{section \ref*{#1}}}

\parskip1ex
\parindent0pt
\let\olditemize\itemize
\def\itemize{\olditemize\parskip0pt}

\begin{document}

\title{The \textsf{childdoc} Package}
\hypersetup{pdftitle={The childdoc Package}}
\author{Niklas Beisert\\[2ex]
  Institut f\"ur Theoretische Physik\\
  Eidgen\"ossische Technische Hochschule Z\"urich\\
  Wolfgang-Pauli-Strasse 27, 8093 Z\"urich, Switzerland\\[1ex]
  \href{mailto:nbeisert@itp.phys.ethz.ch}
  {\texttt{nbeisert@itp.phys.ethz.ch}}}
\hypersetup{pdfauthor={Niklas Beisert}}
\hypersetup{pdfsubject={Manual for the LaTeX2e Package childdoc}}
\date{30 December 2018, \textsf{v2.0}}
\maketitle

\begin{abstract}\noindent
\textsf{childdoc} is a \LaTeXe{} package
that enables the direct compilation
of document sections included by |\include|
to individual files.
\end{abstract}

\begingroup
\parskip0ex
\tableofcontents
\endgroup

%%%%%%%%%%%%%%%%%%%%%%%%%%%%%%%%%%%%%%%%%%%%%%%%%%%%%%%%%%%%%%%%%%%%%%%%%%%%%%%%
%%%%%%%%%%%%%%%%%%%%%%%%%%%%%%%%%%%%%%%%%%%%%%%%%%%%%%%%%%%%%%%%%%%%%%%%%%%%%%%%
\section{Introduction}

\LaTeX{} provides a mechanism to structure a large document (such as a book)
into a main file and several child files (containing the chapters)
using the |\include| command.
This mechanism is beneficial for documents
which span hundreds of pages in order to
make the source file(s) more manageable.
Moreover, compilation can be restricted to
selected child files by means of the |\includeonly| command.
The latter feature can be used to reduce the compilation time while editing
(this was significantly more useful in the earlier days of \LaTeX{})
or to generate a smaller document which is easier to navigate.
Another application of |\includeonly| is to generate
documents consisting of selected parts of the complete document.

However, there are a few drawbacks of the plain |\include| mechanism:
\begin{itemize}
\item
The child files cannot be compiled on their own,
they can only be compiled via the main file.
A naive editing environment
(such as a text editor with an option
to have the current file processed by \LaTeX)
may require one to switch to the main file before compiling;
attempting to compile the child file produces errors.
\item
The main file must be modified (each time)
to adjust the |\includeonly| command
to the present needs. This easily leaves the main file in a messy state.
\item
The generated document will always carry the filename
of the main document. This is inconvenient if
several child files are to be compiled and
to be kept for distribution.
\end{itemize}

The present package provides a simple interface
to make child files individually compilable by \LaTeX{}.
Compiling a child file then has the same effect as compiling
the main file with an |\includeonly| command
to select the appropriate child.
Moreover the generated document will carry the name of the child
rather than the main file.
This resolves all three above issues.

This feature is meant to make the editing of books,
thesis documents and lecture notes somewhat more convenient.
However, the package can also be used efficiently for
composing a series of documents (such as exercise sheets)
which are typically distributed individually.
It then assists the author in generating the individual documents
(potentially in different versions)
as well as a document containing the collected series.
Another application is in developing style files
or other kinds of included material
where compilation of the style file could redirect
to a sample or test file.

%%%%%%%%%%%%%%%%%%%%%%%%%%%%%%%%%%%%%%%%%%%%%%%%%%%%%%%%%%%%%%%%%%%%%%%%%%%%%%%%
%%%%%%%%%%%%%%%%%%%%%%%%%%%%%%%%%%%%%%%%%%%%%%%%%%%%%%%%%%%%%%%%%%%%%%%%%%%%%%%%
\section{Usage}

First of all, the package \textsf{childdoc} is \emph{not} a standard
\LaTeXe{} |.sty| style file! Therefore it needs to be invoked in
a non-standard way.

%%%%%%%%%%%%%%%%%%%%%%%%%%%%%%%%%%%%%%%%%%%%%%%%%%%%%%%%%%%%%%%%%%%%%%%%%%%%%%%%
\subsection{Included Files}
\label{sec:include}

%%%%%%%%%%%%%%%%%%%%%%%%%%%%%%%%%%%%%%%%
\DescribeMacro{\childdocmain}
To use the package, add the commands
\begin{center}
\begin{tabular}{l}
|\input{childdoc.def}|\\
|\childdocmain{}|\\
\end{tabular}
\end{center}
at the very top of the main \LaTeX{} file,
in particular \emph{before} the |\documentclass| statement!
The argument of |\childdocmain| should be left empty
(but it must be present).

%%%%%%%%%%%%%%%%%%%%%%%%%%%%%%%%%%%%%%%%
\DescribeMacro{\childdocof}
Furthermore, add the commands
\begin{center}
\begin{tabular}{l}
|\input{childdoc.def}|\\
|\childdocof{|\textit{main}|}|\\
\end{tabular}
\end{center}
at the top of every child file \textit{child}
which is included by |\include{|\textit{child}|}|
from within the main file
(or at least for those files to be compiled individually).
The argument \textit{main} must be the filename of the main file.

There are a couple of
considerations in setting up the main and child documents:

%%%%%%%%%%%%%%%%%%%%%%%%%%%%%%%%%%%%%%%%
\paragraph{Restrictions.}

Please note the following restrictions:
\begin{itemize}
\item
|\childdocmain| must be called with one argument \textit{main}
to ensure compatibility with earlier version of the package.
It must either be empty (|\childdocmain{}|)
or precisely match the filename of the main file in which it is specified.
See \secref{sec:detection} for further information.
\item
The filename \textit{main} must be specified without the |.tex| extension.
\item
The filename \textit{main} is case sensitive
(even in case-insensitive file systems)
due to internal string comparison.
\item
The argument \textit{main} should be fully expanded, it cannot be a macro.
\item
Subdirectories and special characters should be avoided in filenames.
\item
The command |\childdocmain{|\textit{main}|}| must be followed by a whitespace.
It should not be followed immediately by another command
or by a comment mark `|%|'.
This is because the \TeX{} parser reads the token immediately following
the argument of |\childdocmain| and puts it
at the beginning of every child section;
however, a white\-space is ignored.
\end{itemize}

%%%%%%%%%%%%%%%%%%%%%%%%%%%%%%%%%%%%%%%%
\paragraph{Content of Main File.}

It is advisable to place all content in the child files included by |\include|.
Any output contained in the main file will appear in all child documents
unless suppressed manually;
it cannot be suppressed automatically by the |\includeonly| directive
and thus should normally be avoided.
A method to include some content in the main file
by means of conditional processing is described in \secref{sec:conditional}.

%%%%%%%%%%%%%%%%%%%%%%%%%%%%%%%%%%%%%%%%
\paragraph{Page Numbering.}

When only a part of the document is compiled,
the appropriate numbering of pages
(as well as other status parameters)
is determined from the |.aux| files.
The latter contain information from previous passes.
However this information needs to propagate through
all intermediate child documents.
Therefore the page numbering in child documents may well
be inconsistent until the complete document is compiled at least once.

A useful (if unconventional) way to always ensure a consistent
page numbering is to restart the numbering in each child document
and denote the pages by `\textit{child}|.|\textit{page}'
where \textit{child} represents the chapter/section number of the child file.
This can be achieved by the command
|\numberwithin{page}{|\textit{child}|}|
of the \textsf{amsmath} package
where \textit{child} can be |chapter| or |section|
depending on the chosen structuring.
Alternatively, one can modify the macro |\thepage| appropriately
and reset the counter |page| at the start of each child file.

%%%%%%%%%%%%%%%%%%%%%%%%%%%%%%%%%%%%%%%%%%%%%%%%%%%%%%%%%%%%%%%%%%%%%%%%%%%%%%%%
\subsection{Conditional Processing}
\label{sec:conditional}

The package provides a mechanism to compile different versions
of a document. To customise the versions further some conditional processing
can come in handy to distinguish which version is being compiled.
The package provides two macros to describe the compilation context:

%%%%%%%%%%%%%%%%%%%%%%%%%%%%%%%%%%%%%%%%
\DescribeMacro{\ifchilddoc}
The conditional |\ifchilddoc| distinguishes between the compilation of
child documents and the main document:
%
\begin{center}
|\ifchilddoc |\textit{child-code}| |[|\||else |\textit{main-code}]| \||fi|
\end{center}

%%%%%%%%%%%%%%%%%%%%%%%%%%%%%%%%%%%%%%%%
\DescribeMacro{\childdocname}
\DescribeMacro{\childdocjob}
The macro |\childdocname| contains the filename (without extension)
of the main or child file being processed.
Note that |\childdocjob| will always contain the name of the main file.

%%%%%%%%%%%%%%%%%%%%%%%%%%%%%%%%%%%%%%%%
\paragraph{Title Page.}

Conditional processing can be used to include a title or banner page
in the main document when proper precautions are taken.
Importantly, the code in the main file should ensure that the page counter
(as well as other status parameters which are stored in the |.aux| files)
takes the same value after the conditional processing.
Otherwise the page numbers may take divergent values
depending on which part is compiled.

For example, a title page could be declared by:
%
\begin{center}
\begin{tabular}{l}
|\ifchilddoc\||else|\\
|\addtocounter{page}{-1}|\\
\textit{code for title page}\\
|\newpage|\\
|\||fi|
\end{tabular}
\end{center}
%
A banner page for the child documents can be generated by:
%
\begin{center}
\begin{tabular}{l}
|\ifchilddoc|\\
|\addtocounter{page}{-1}|\\
\textit{code for banner page}\\
|\newpage|\\
|\||fi|
\end{tabular}
\end{center}
%
Here one could write a message such as:
\begin{center}
|This is the part \childdocname{} of \childdocjob{}.|
\end{center}

%%%%%%%%%%%%%%%%%%%%%%%%%%%%%%%%%%%%%%%%%%%%%%%%%%%%%%%%%%%%%%%%%%%%%%%%%%%%%%%%
\subsection{Flags}
\label{sec:flags}

The package makes it easy to generate different versions
of the main or child documents.
To this end compilation flags can be defined
and assigned different default values.
They will be particularly useful in conjunction
with the forwarding mechanism described in \secref{sec:forward}.

For example, it may be useful to have a flag |\version|
which can be set to |draft| or |final|.
The document source will contain some conditional code
depending on the value of |\version|.
Suppose further, the flag should default to |final| for the main file
and to |draft| for child files
which is a natural assignment for editing the document.
This is achieved by placing the following code
in the preamble of the main document
(below the |\childdocmain| directive):
%
\begin{center}
\begin{tabular}{l}
|\ifchilddoc|\\
|\providecommand{\version}{draft}|\\
|\||else|\\
|\providecommand{\version}{final}|\\
|\||fi|
\end{tabular}
\end{center}
%
The definition by |\providecommand| makes sure
that previous definitions are not overwritten.
Further statements |\providecommand{\version}{...}|
can thus be added before the above code to override it.

For the main file, one might add a line
(between |\childdocmain| and the above block)
%
\begin{center}
|%\ifchilddoc\||else\providecommand{\version}{draft}\||fi|
\end{center}
%
which can be uncommented to produce a draft version.
Likewise one can add a line to the very top of a child file
(above the |\childdocof{|\textit{main}|}| directive)
%
\begin{center}
|%\providecommand{\version}{final}|
\end{center}
%
which can be uncommented to produce the final version of this child document.

%%%%%%%%%%%%%%%%%%%%%%%%%%%%%%%%%%%%%%%%%%%%%%%%%%%%%%%%%%%%%%%%%%%%%%%%%%%%%%%%
\subsection{Forwarding}
\label{sec:forward}

Different versions of the main or child documents
using compilation flags as described in \secref{sec:flags}
can be (permanently) stored in different files
for convenient compilation, viewing and distribution.
To this end, the package defines a command
to pass on compilation to a different file:

%%%%%%%%%%%%%%%%%%%%%%%%%%%%%%%%%%%%%%%%
\DescribeMacro{\childdocforward}
The command |\childdocforward| redirects processing to
another source file:
%
\begin{center}
\begin{tabular}{l}
|\input{childdoc.def}|\\
|\childdocforward[|\textit{main}|]{|\textit{dest}|}|\\
\end{tabular}
\end{center}
%
The argument \textit{dest} is the destination file
(without extension).
It should be the main file or one of the child files.
Note that further \textsf{childdoc} directives
such as |\childdocof| and |\childdocforward|
in the indicated file will be processed in this form.
The optional argument \textit{main}
passes on directly to the main file \textit{main}
while pretending to compile the child \textit{dest}.
This form behaves as if \textit{dest}
issues |\childdocof{|\textit{main}|}| right away,
and no further \textsf{childdoc} directives will be processed.

%%%%%%%%%%%%%%%%%%%%%%%%%%%%%%%%%%%%%%%%
\DescribeMacro{\...prefix}
In the alternative form |\childdocforwardprefix|,
%
\begin{center}
\begin{tabular}{l}
|\input{childdoc.def}|\\
|\childdocforwardprefix[|\textit{main}|]{|\textit{prefix}|}{|\textit{dest}|}|
\end{tabular}
\end{center}
%
the destination file is determined by a pattern
depending on the current file:
To make this work, the current file must be called
`{\textit{prefix}\hspace{0.2em}\textit{suffix}}'
with \textit{prefix} matching precisely the argument.
Processing is then passed on to the file
`{\textit{dest}\hspace{0.2em}\textit{suffix}}'.
Surely, the same effect is achieved by
directly specifying the
argument `{\textit{dest}\hspace{0.2em}\textit{suffix}}'
in the first form.
However, that requires to set up a different file
for each child. With the alternative form of the command
all these files can have exactly the same content
which simplifies setting them up and maintaining them.

For example, the following file |draft.tex|
with a compilation flag |\version| as described in \secref{sec:flags}
compiles the main document as a draft:
%
\begin{center}
\begin{tabular}{l}
|\def\version{draft}|\\
|\input{childdoc.def}|\\
|\childdocforward{|\textit{main}|}|
\end{tabular}
\end{center}
%
Likewise, the following files |final|\textit{nn}|.tex|
compile the final version of the child document
|child|\textit{nn}|.tex|:
%
\begin{center}
\begin{tabular}{l}
|\def\version{final}|\\
|\input{childdoc.def}|\\
|\childdocforwardprefix{final}{child}|
\end{tabular}
\end{center}
%

Note that when several versions of a main file and/or of each child file
are to be generated, it may be convenient to set up a |Makefile| or
shell script to automatise the process.

%%%%%%%%%%%%%%%%%%%%%%%%%%%%%%%%%%%%%%%%%%%%%%%%%%%%%%%%%%%%%%%%%%%%%%%%%%%%%%%%
\subsection{Command Line Processing}
\label{sec:commandline}

The effect of redirection files can also be achieved by invoking
the \LaTeX{} compiler with a more elaborate command line.
Most conveniently this should be done as part
of a shell script or a |Makefile|.

When using \textsf{childdoc} in the main file, the following
command lines effectively perform a redirection
(note that depending on the shell being used,
backslashes may have to be doubled: `|\|' $\to$ `|\\|'):
%
\begin{center}
|... -jobname "|\textit{target}|" |\\|"|[\textit{flags}]%
|\input{childdoc.def}\childdocforward[|\textit{main}|]{|\textit{dest}|}"|
\end{center}
%
Here \textit{target} is the name of the output file,
\textit{main} is the name of the main file
and \textit{dest} is the name of the main or child file to be processed
(all filenames without extensions).
The optional argument \textit{main} can be omitted
if \textit{main} matches \textit{dest}.
Optionally, compilation \textit{flags} can be defined via |\def| commands.
This command line makes the \TeX{} engine believe
it is compiling the file \textit{target}
whose content is specified as the latter parameter.
The provided code then forwards the processing to
\textit{main} or \textit{dest} as described in \secref{sec:forward}.

%%%%%%%%%%%%%%%%%%%%%%%%%%%%%%%%%%%%%%%%%%%%%%%%%%%%%%%%%%%%%%%%%%%%%%%%%%%%%%%%
\subsection{Include by Input}
\label{sec:input}

Including child documents by |\include| has some restrictions by design.
Most notably, the content of a child document always occupies
its own set of pages; pages cannot be shared between child documents.
Usually, this behaviour makes perfect sense
because each child document contain an essential part of the document.
However, in some situations it may be desirable to compose
a document from a collection of parts
without having mandatory page breaks between then.
For this case, the package
provides a mechanism to include parts
by |\input| which can also be processed individually.
However, by construction this mechanism
requires manual handling of the content to be output.

%%%%%%%%%%%%%%%%%%%%%%%%%%%%%%%%%%%%%%%%
\DescribeMacro{\ifchilddocmanual}
The main file should be prepared as usual, see \secref{sec:include}.
However, the document body must make a distinction
between processing of an individual part and of the main document, e.g.:
%
\begin{center}
\begin{tabular}{l}
|\ifchilddocmanual|\\
|\input{\childdocname}|\\
|\||else|\\
\textit{document body with }|\input{|\textit{part}|}|\\
|\||fi|
\end{tabular}
\end{center}
%
The conditional |\ifchilddocmanual| is true whenever
a part to be included by |\input| is being compiled,
and the name of the part is stored in |\childdocname|.

%%%%%%%%%%%%%%%%%%%%%%%%%%%%%%%%%%%%%%%%
\DescribeMacro{\childdocby}
Each part to be included by |\input| should start with:
%
\begin{center}
\begin{tabular}{l}
|\input{childdoc.def}|\\
|\childdocby{|\textit{main}|}|\\
\end{tabular}
\end{center}
%
The directive |\childdocby| is similar to |\childdocof|
described in \secref{sec:include},
but the subsequent selection of content must be done manually.
To that end, both |\ifchilddoc| and |\ifchilddocmanual|
will be true upon processing of a part,
and the name of the part is stored in |\childdocname|.
Note that |\jobname| will be set to the filename of the current part
so that each part receives an individual |.aux| file
that does not interfere with the |.aux| file(s) of the main document.
This behaviour can be altered by the alternative form
|\childdocby[*]{|\textit{main}|}| (with a non-empty optional argument)
which uses the |.aux| file of the main document
by setting |\jobname| to \textit{main}.

%%%%%%%%%%%%%%%%%%%%%%%%%%%%%%%%%%%%%%%%%%%%%%%%%%%%%%%%%%%%%%%%%%%%%%%%%%%%%%%%
\subsection{Driver Development}
\label{sec:driver}

The \textsf{childdoc} mechanism can also be use for the development
of definition files such as \LaTeX{} styles or classes.
This case differs from the above setup with multiple parts
included by |\include| in that no |\includeonly| should be invoked.
This can be achieved by starting the include file
(before |\ProvidesPackage|) with:
%
\begin{center}
\begin{tabular}{l}
|\input{childdoc.def}|\\
|\childdocforward{|\textit{main}|}|\\
\end{tabular}
\end{center}
%
or alternatively with:
%
\begin{center}
\begin{tabular}{l}
|\input{childdoc.def}|\\
|\childdocby{|\textit{main}|}|\\
\end{tabular}
\end{center}
%
Both forms have slightly different effects as described above.
The main file is prepared as usual, see \secref{sec:include}.

%%%%%%%%%%%%%%%%%%%%%%%%%%%%%%%%%%%%%%%%%%%%%%%%%%%%%%%%%%%%%%%%%%%%%%%%%%%%%%%%
\subsection{Legacy Detection}
\label{sec:detection}

The directive |\childdocmain| in the main file can detect
whether the complete document or merely a child is to be compiled
even without using the directive |\childdocof|.
This method is deprecated because it is less robust
and there is no compelling reason to use it;
it is merely provided for backward compatibility
and it may be removed in future versions.

If the detection mechanism is to be used,
it is mandatory to correctly specify
the filename of the main file as the argument of |\childdocmain|:
%
\begin{center}
\begin{tabular}{l}
|\input{childdoc.def}|\\
|\childdocmain{|\textit{main}|}|\\
\end{tabular}
\end{center}
%
If |\jobname| does not match the argument \textit{main} of |\childdocmain|,
it is assumed that |\jobname| points to the child file to be compiled.
When using |\childdocmain| with the main file specified as argument,
it suffices to start a child file
with just |\input{|\textit{main}|}|
without loading of the package and using |\childdocof|.
If instead all processing is done
with the appropriate \textsf{childdoc} directives,
the argument of \textit{main} of |\childdocmain| can be empty.

An alternative version of the command line processing described
in \secref{sec:commandline} using the detection mechanism reads:
%
\begin{center}
|... -jobname "|\textit{target}|" "|[\textit{flags}]%
[|\def\jobname{|\textit{dest}|}|]|\input{|\textit{main}|}"|
\end{center}

%%%%%%%%%%%%%%%%%%%%%%%%%%%%%%%%%%%%%%%%%%%%%%%%%%%%%%%%%%%%%%%%%%%%%%%%%%%%%%%%
\subsection{Manual Code}
\label{sec:manual}

In case one cannot be certain whether the definitions file |childdoc.def|
is installed on the target \TeX{} distribution
and one prefers not to ship it,
it is conceivable to paste a few relevant commands into the sources.

To that end, drop all statements |\input{childdoc.def}|
and perform the replacements as outlined below.
Instead of |\childdocmain{|\textit{main}|}| add the following code
to the top of the main file:
%
\begin{center}
\begin{tabular}{l}
|\||ifdefined\childdocname\endinput\||fi\newif\ifchilddoc|\\
|\edef\childdocname{\scantokens\expandafter{\jobname\noexpand}}|\\
|\def\childdocmain{|\textit{main}|}\||ifx\childdocmain\childdocname\||else|\\
|\childdoctrue\includeonly{\childdocname}\let\jobname\childdocmain\||fi|\\
\end{tabular}
\end{center}
%
Instead of |\childdocof{|\textit{main}|}| just include the main file
at the top of each child file:
%
\begin{center}
|\input{|\textit{main}|}|
\end{center}
%
A simple redirection |\childdocforward{|\textit{dest}|}| is achieved by:
%
\begin{center}
|\def\jobname{|\textit{dest}|}\input{\jobname}|
\end{center}
%
The redirection with prefix
|\childdocforwardprefix[|\textit{prefix}|]{|\textit{dest}|}|
is accomplished by:
%
\begin{center}
\begin{tabular}{l}
|{\edef\jobname{\scantokens\expandafter{\jobname\noexpand}}|\\
|\def\redirectjob |\textit{prefix}|#1~~~{\gdef\jobname{|\textit{dest}|#1}}|\\
|\expandafter\redirectjob\jobname~~~}\input{\jobname}|
\end{tabular}
\end{center}

In an alternative approach,
child documents can be compiled by a specific command line
without additional code or specific definitions:
%
\begin{center}
|... -jobname "|\textit{target}|" "|[\textit{flags}]%
|\includeonly{|\textit{dest}|}\input{|\textit{main}|}"|
\end{center}
%

%%%%%%%%%%%%%%%%%%%%%%%%%%%%%%%%%%%%%%%%%%%%%%%%%%%%%%%%%%%%%%%%%%%%%%%%%%%%%%%%
%%%%%%%%%%%%%%%%%%%%%%%%%%%%%%%%%%%%%%%%%%%%%%%%%%%%%%%%%%%%%%%%%%%%%%%%%%%%%%%%
\section{Information}

%%%%%%%%%%%%%%%%%%%%%%%%%%%%%%%%%%%%%%%%%%%%%%%%%%%%%%%%%%%%%%%%%%%%%%%%%%%%%%%%
\subsection{Copyright}

Copyright \copyright{} 2017--2018 Niklas Beisert

This work may be distributed and/or modified under the
conditions of the \LaTeX{} Project Public License, either version 1.3
of this license or (at your option) any later version.
The latest version of this license is in
  \url{http://www.latex-project.org/lppl.txt}
and version 1.3 or later is part of all distributions of \LaTeX{}
version 2005/12/01 or later.

This work has the LPPL maintenance status `maintained'.

The Current Maintainer of this work is Niklas Beisert.

This work consists of the files |README.txt|, |childdoc.ins| and |childdoc.dtx|
as well as the derived files |childdoc.def|, |cdocsamp.tex|
with |cdocsch1.tex|, |cdocsch2.tex|, |cdocspt3.tex|, |cdocspt4.tex|,
|cdocsdrf.tex|, |cdocsfn1.tex|, |cdocsfn2.tex|
as well as |childdoc.pdf|.

%%%%%%%%%%%%%%%%%%%%%%%%%%%%%%%%%%%%%%%%%%%%%%%%%%%%%%%%%%%%%%%%%%%%%%%%%%%%%%%%
\subsection{Files and Installation}

The package consists of the files:
%
\begin{center}
\begin{tabular}{ll}
    |README.txt|   & readme file \\
    |childdoc.ins| & installation file \\
    |childdoc.dtx| & source file \\
    |childdoc.def| & definition file \\
    |cdocsamp.tex| & sample main file \\
    |cdocsch1.tex| & sample include file \\
    |cdocsch2.tex| & sample include file \\
    |cdocspt3.tex| & sample part file \\
    |cdocspt4.tex| & sample part file \\
    |cdocsdrf.tex| & sample redirection file \\
    |cdocsfn1.tex| & sample redirection file \\
    |cdocsfn2.tex| & sample redirection file \\
    |childdoc.pdf| & manual
\end{tabular}
\end{center}
%
The distribution consists of the files
|README.txt|, |childdoc.ins| and |childdoc.dtx|.
%
\begin{itemize}
\item
Run (pdf)\LaTeX{} on |childdoc.dtx|
to compile the manual |childdoc.pdf| (this file).
\item
Run \LaTeX{} on |childdoc.ins| to create the definitions file |childdoc.def|
and the sample |cdocsamp.tex| with include files
|cdocsch1.tex|, |cdocsch2.tex|, |cdocspt3.tex|, |cdocspt4.tex|,
|cdocsdrf.tex|, |cdocsfn1.tex|, |cdocsfn2.tex|.
Then copy the file |childdoc.def| to an appropriate directory of your \LaTeX{}
distribution, e.g.\ \textit{texmf-root}|/tex/latex/childdoc|.
\end{itemize}

%%%%%%%%%%%%%%%%%%%%%%%%%%%%%%%%%%%%%%%%%%%%%%%%%%%%%%%%%%%%%%%%%%%%%%%%%%%%%%%%
\subsection{Related CTAN Packages}

There are several other packages which offer a similar functionality:
%
\begin{itemize}
\item
The packages
\href{http://ctan.org/pkg/docmute}{\textsf{docmute}},
\href{http://ctan.org/pkg/includex}{\textsf{includex}} and
\href{http://ctan.org/pkg/standalone}{\textsf{standalone}}
provide commands to include only the document body of
a child file thus allowing both files to be compiled individually.
\item
The packages \href{http://ctan.org/pkg/subdocs}{\textsf{subdocs}}
and \href{http://ctan.org/pkg/subfiles}{\textsf{subfiles}}
provide structures in which the main and child documents can be
encapsulated and allowing them to be compiled individually.
The inclusion mechanism is different from the conventional |\include|.
\item
The package \href{http://ctan.org/pkg/combine}{\textsf{combine}}
is an elaborate solution to combine several documents into one.
\end{itemize}
%
See also the CTAN topic \href{http://ctan.org/topic/subdocs}{\textsf{subdocs}}
for further related packages.
The present package differs from the above solutions in that
a document structure constructed with the conventional |\include| mechanism
just needs two extra commands at the top of every file
such that all constituent files can be compiled individually.

%%%%%%%%%%%%%%%%%%%%%%%%%%%%%%%%%%%%%%%%%%%%%%%%%%%%%%%%%%%%%%%%%%%%%%%%%%%%%%%%
%\subsection{Feature Suggestions}
%
%The following is a list of features which may be useful for future
%versions of this package:
%%
%\begin{itemize}
%\item
%\ldots
%\end{itemize}

%%%%%%%%%%%%%%%%%%%%%%%%%%%%%%%%%%%%%%%%%%%%%%%%%%%%%%%%%%%%%%%%%%%%%%%%%%%%%%%%
\subsection{Revision History}

%%%%%%%%%%%%%%%%%%%%%%%%%%%%%%%%%%%%%%%%
\paragraph{v2.0:} 2018/12/30

\begin{itemize}
\item
immediate forward processing
\item
added |\childdocby| mechanism
\item
manual restructured
\end{itemize}

%%%%%%%%%%%%%%%%%%%%%%%%%%%%%%%%%%%%%%%%
\paragraph{v1.6:} 2018/01/17

\begin{itemize}
\item
application for development of include files
\item
corrections to manual
\end{itemize}

%%%%%%%%%%%%%%%%%%%%%%%%%%%%%%%%%%%%%%%%
\paragraph{v1.5:} 2017/05/21

\begin{itemize}
\item
more complete structuring introduced
\item
|\childdocof| introduced
\item
|\childdoc| renamed to |\childdocmain|
\item
|\childredirect| renamed to |\childdocforward| and |\childdocforwardprefix|
and functionality expanded
\end{itemize}

%%%%%%%%%%%%%%%%%%%%%%%%%%%%%%%%%%%%%%%%
\paragraph{v1.0:} 2017/04/27

\begin{itemize}
\item
manual and install package
\item
first version published on CTAN
\end{itemize}

%%%%%%%%%%%%%%%%%%%%%%%%%%%%%%%%%%%%%%%%
\paragraph{v0.6:} 2017/04/26

\begin{itemize}
\item
redirection mechanism added
\end{itemize}

%%%%%%%%%%%%%%%%%%%%%%%%%%%%%%%%%%%%%%%%
\paragraph{v0.5:} 2017/04/26

\begin{itemize}
\item
functionality in definition file
\end{itemize}


%%%%%%%%%%%%%%%%%%%%%%%%%%%%%%%%%%%%%%%%%%%%%%%%%%%%%%%%%%%%%%%%%%%%%%%%%%%%%%%%
%%%%%%%%%%%%%%%%%%%%%%%%%%%%%%%%%%%%%%%%%%%%%%%%%%%%%%%%%%%%%%%%%%%%%%%%%%%%%%%%
%%%%%%%%%%%%%%%%%%%%%%%%%%%%%%%%%%%%%%%%%%%%%%%%%%%%%%%%%%%%%%%%%%%%%%%%%%%%%%%%
\appendix

\settowidth\MacroIndent{\rmfamily\scriptsize 000\ }

 \DocInput{childdoc.dtx}

\end{document}
%</driver>
% \fi
%
% %%%%%%%%%%%%%%%%%%%%%%%%%%%%%%%%%%%%%%%%%%%%%%%%%%%%%%%%%%%%%%%%%%%%%%%%%%%%%%
% %%%%%%%%%%%%%%%%%%%%%%%%%%%%%%%%%%%%%%%%%%%%%%%%%%%%%%%%%%%%%%%%%%%%%%%%%%%%%%
% \section{Sample}
%\iffalse
%<*samplemain>
%\fi
%
% The following presents a sample document
% with two chapters, two parts, a title page,
% a compile flag as well as three forwarding files to set the flag.
% It consists of eight |.tex| files:
% \begin{center}
% \begin{tabular}{ll}
% |cdocsamp.tex|&main file\\
% |cdocsch1.tex|&include file for chapter 1\\
% |cdocsch2.tex|&include file for chapter 2\\
% |cdocspt3.tex|&include file for part 3\\
% |cdocspt4.tex|&include file for part 4\\
% |cdocsdrf.tex|&forwarding file for main file in draft mode\\
% |cdocsfi1.tex|&forwarding file for final version of chapter 1\\
% |cdocsfi2.tex|&forwarding file for final version of chapter 2\\
% \end{tabular}
% \end{center}
% Each of the eight files can be compiled directly by the \LaTeX{} compiler.
%
% %%%%%%%%%%%%%%%%%%%%%%%%%%%%%%%%%%%%%%
% \paragraph{Main File.}
%
% The main file is called |cdocsamp.tex|.
%
% Load the \textsf{childdoc} definitions and
% declare the filename for the main document:
%    \begin{macrocode}
\input{childdoc.def}
\childdocmain{}
%    \end{macrocode}

% Optional override for |\version| flag:
%    \begin{macrocode}
%%\ifchilddoc\else\providecommand{\version}{draft}\fi
%    \end{macrocode}

% Define the default values for the |\version| flag
% (|final| for the main file and |draft| for childs):
%    \begin{macrocode}
\ifchilddoc
\providecommand{\version}{draft}
\else
\providecommand{\version}{final}
\fi
%    \end{macrocode}

% Load the standard document class:
%    \begin{macrocode}
\documentclass[12pt]{article}
%    \end{macrocode}

% Start the document body:
%    \begin{macrocode}
\begin{document}
%    \end{macrocode}

% Declare a title page.
% Print title, part of document being processed and version flag:
%    \begin{macrocode}
\addtocounter{page}{-1}
\begin{center}
{\LARGE\bfseries{}childdoc example\par}
\vspace{1cm}
\ifchilddoc
\ifchilddocmanual part\else chapter\fi:
`\childdocname' of `\childdocjob'\par
\else
main document: `\childdocjob'\par
\fi
version: \version\par
\end{center}
\newpage
%    \end{macrocode}

% Manually include selected file,
% otherwise process as usual:
%    \begin{macrocode}
\ifchilddocmanual
\section*{part `\childdocname'}
\input{\childdocname}
\else
%    \end{macrocode}

% Include the two chapters:
%    \begin{macrocode}
\include{cdocsch1}
\include{cdocsch2}
%    \end{macrocode}

% Include the two parts unless only chapters should be displayed:
%    \begin{macrocode}
\ifchilddoc\else
\section{part three}
\input{cdocspt3}
\section{part four}
\input{cdocspt4}
\fi
%    \end{macrocode}

% Process as usual until here:
%    \begin{macrocode}
\fi
%    \end{macrocode}

% End of document body:
%    \begin{macrocode}
\end{document}
%    \end{macrocode}
%\iffalse
%</samplemain>
%\fi
%
% %%%%%%%%%%%%%%%%%%%%%%%%%%%%%%%%%%%%%%
% \paragraph{Chapter Include Files.}
%
% The include files are called |cdocsch1.tex| and |cdocsch2.tex|.
%
%\iffalse
%<*samplechap1|samplechap2>
%\fi

% Optional override for |\version| flag:
%    \begin{macrocode}
%%\providecommand{\version}{final}
%    \end{macrocode}

% Include the main document:
%    \begin{macrocode}
\input{childdoc.def}
\childdocof{cdocsamp}
%    \end{macrocode}

%\iffalse
%</samplechap1|samplechap2>
%\fi
%
%\iffalse
%<*samplechap1>
%\fi
% Some text for chapter 1:
%    \begin{macrocode}
\section{one}
some text in chapter one
%    \end{macrocode}

%\iffalse
%</samplechap1>
%\fi
% Some text for chapter 2:
%\iffalse
%<*samplechap2>
%\fi
%    \begin{macrocode}
\section{two}
more text in chapter two
%    \end{macrocode}

%\iffalse
%</samplechap2>
%\fi
%
% %%%%%%%%%%%%%%%%%%%%%%%%%%%%%%%%%%%%%%
% \paragraph{Part Include Files.}
%
% The include files are called |cdocspt3.tex| and |cdocspt4.tex|.
%
%\iffalse
%<*samplepart3|samplepart4>
%\fi

% Optional override for |\version| flag:
%    \begin{macrocode}
%%\providecommand{\version}{final}
%    \end{macrocode}

% Include the main document:
%    \begin{macrocode}
\input{childdoc.def}
\childdocby{cdocsamp}
%    \end{macrocode}

%\iffalse
%</samplepart3|samplepart4>
%\fi
%
%\iffalse
%<*samplepart3>
%\fi
% Some text for part 3:
%    \begin{macrocode}
some text in part three
%    \end{macrocode}

%\iffalse
%</samplepart3>
%\fi
% Some text for part 4:
%\iffalse
%<*samplepart4>
%\fi
%    \begin{macrocode}
more text in part four
%    \end{macrocode}

%\iffalse
%</samplepart4>
%\fi
%
% %%%%%%%%%%%%%%%%%%%%%%%%%%%%%%%%%%%%%%
% \paragraph{Forwarding for a Complete Draft.}
%
% The following forwarding file |cdocsdrf.tex|
% compiles the main document in draft mode:
%\iffalse
%<*sampledraft>
%\fi
%    \begin{macrocode}
\def\version{draft}
\input{childdoc.def}
\childdocforward{cdocsamp}
%    \end{macrocode}

%\iffalse
%</sampledraft>
%\fi
%
% %%%%%%%%%%%%%%%%%%%%%%%%%%%%%%%%%%%%%%
% \paragraph{Forwarding for Final Version of the Chapters.}
%
% The following forwarding files |cdocsfn1.tex| and |cdocsfn2.tex|
% (with identical content)
% compile the final versions of the child documents
% |cdocsch1.tex| and |cdocsch2.tex|, respectively:
%\iffalse
%<*samplefinal>
%\fi
%    \begin{macrocode}
\def\version{final}
\input{childdoc.def}
\childdocforwardprefix[cdocsamp]{cdocsfn}{cdocsch}
%    \end{macrocode}

%\iffalse
%</samplefinal>
%\fi
%
% %%%%%%%%%%%%%%%%%%%%%%%%%%%%%%%%%%%%%%
% \paragraph{Command Line Processing.}
%
% The following three command lines generate the output files
% |cdocscld|, |cdocscl1| and |cdocscl2|
% which should be identical to
% |cdocsdrf|, |cdocsch1| and |cdocsfn2|, respectively:
% \begin{center}
% \begin{tabular}{l}
% |latex -jobname cdocscld \|\\
% |  "\def\version{draft}\input{childdoc.def}\childdocforward{cdocsamp}"|\\
% |latex -jobname cdocscl1 \|\\
% |  "\input{childdoc.def}\childdocforward[cdocsamp]{cdocsch1}"|\\
% |latex -jobname cdocscl2 \|\\
% |  "\def\version{final}\input{childdoc.def}\childdocforward{cdocsch2}"|
% \end{tabular}
% \end{center}
% Note that the trailing backslash on each first line
% merely continues the input to the second line
% (for convenient cut ant paste).
% Furthermore, the command |latex| can be replaced by any
% of its alternative versions such as |pdflatex|.
%
% %%%%%%%%%%%%%%%%%%%%%%%%%%%%%%%%%%%%%%%%%%%%%%%%%%%%%%%%%%%%%%%%%%%%%%%%%%%%%%
% %%%%%%%%%%%%%%%%%%%%%%%%%%%%%%%%%%%%%%%%%%%%%%%%%%%%%%%%%%%%%%%%%%%%%%%%%%%%%%
% \section{Implementation}
%\iffalse
%<*package>
%\fi
%
% This section describes the definitions file |childdoc.def|.

% The definitions cannot be loaded using |\usepackage| or |\RequirePackage|
% which has a mechanism to prevent loading a style file more than once.
% When loading the definitions by means of |\input|
% multiple instances have to be prevented manually:
%\iffalse
%This code needs to be before the `\ProvidesFile' directive
%which is defined at the beginning of this file.
%Therefore it is also placed there and commented out here.
%</package>
%<*discard>
%\fi
%    \begin{macrocode}
\ifdefined\childdocmain\endinput\fi
%    \end{macrocode}
%\iffalse
%</discard>
%<*package>
%\fi
%
% \macro{\ifchilddoc}
% \macro{\ifchilddocmanual}
% The conditional |\ifchilddoc| tells whether a
% child (true) or main (false) document is being compiled.
% The conditional |\ifchilddocmanual| tells whether
% the |\includeonly| mechanism is used (false) or
% the selection of child files must be performed manually (true).
% The definitions initialise to false:
%    \begin{macrocode}
\newif\ifchilddoc
\newif\ifchilddocmanual
%    \end{macrocode}

% \macro{\childdocname}
% \macro{\childdocjob}
% The macro |\childdocname| stores the name of the main document
% to be compiled. The macro |\childdocjob| stores the name of
% the document on which the \LaTeX{} compiler was originally invoked.
% The content of |\jobname| cannot be compared
% to filenames specified in the source due to different catcodes.
% The following code rescans |\jobname|, stores the result
% in |\childdocname| and saves a copy in |\childdocjob|:
%    \begin{macrocode}
\edef\childdocname{\scantokens\expandafter{\jobname\noexpand}}
\let\childdocjob\childdocname
%    \end{macrocode}

% \macro{\childdocdisable}
% The macro |\childdocdisable| prevents the main file
% from being processed more than once.
% At this stage, the main document command |\childdocmain|
% is assumed to be called once again where it should do nothing.
% Any subsequent call to it should prevent
% a secondary processing of the main document
% It overwrites the forwarding commands
% |\childdocof| and |\childdocforward|
% with empty macros to prevent further inclusions of the main document:
%    \begin{macrocode}
\newcommand{\childdocdisable}
{
  \renewcommand{\childdocmain}[1]{\renewcommand{\childdocmain}[1]{\endinput}}
  \renewcommand{\childdocof}[1]{}
  \renewcommand{\childdocby}[2][]{}
  \renewcommand{\childdocforward}[2][]{}
  \renewcommand{\childdocdisable}{}
}
%    \end{macrocode}

% \macro{\childdocmain}
% The macro |\childdocmain| is to be called at the top of the main file
% with nothing or the main filename (without extension) as argument.
% First, it breaks loops.
% If the argument is not empty and does not match |\childdocname|
% (which is set by the first inclusion of |childdoc.def|),
% |\ifchilddoc| is set to true, |\includeonly| is applied to the child file
% and |\jobname| is set to the main file
% (for proper handling of |.aux| files):
%    \begin{macrocode}
\newcommand{\childdocmain}[1]
{
  \childdocdisable\childdocmain{}
  \if?#1?\else
    \begingroup
      \def\childdoctmp{#1}
      \ifx\childdoctmp\childdocname
        \def\childdoctmp{}
      \else
        \def\childdoctmp
        {
          \childdoctrue
          \includeonly{\childdocname}
          \def\childdocjob{#1}
          \def\jobname{#1}
        }
      \fi
      \expandafter
    \endgroup
    \childdoctmp
  \fi
}
%    \end{macrocode}

% \macro{\childdocof}
% The command |\childdocof| redirects
% compilation to the main file |#1|.
%    \begin{macrocode}
\newcommand{\childdocof}[1]
{
  \childdocdisable
  \childdoctrue
  \includeonly{\childdocname}
  \def\jobname{#1}
  \def\childdocjob{#1}
  \input{#1}
}
%    \end{macrocode}

% \macro{\childdocby}
% The command |\childdocby| ....
%    \begin{macrocode}
\newcommand{\childdocby}[2][]
{
  \childdocdisable
  \childdoctrue
  \childdocmanualtrue
  \if?#1?\else
    \def\jobname{#2}
  \fi
  \def\childdocjob{#2}
  \input{#2}
  \endinput
}
%    \end{macrocode}

% \macro{\childdocforward}
% The command |\childdocforward| redirects
% compilation to the main file or
% (if the optional argument is given) a child file.
% Parameters are set as if the main file
% or a child file starting with |\childdocof| was compiled.
% Then compilation is handed over to the main file:
%    \begin{macrocode}
\newcommand{\childdocforward}[2][]
{
  \begingroup
    \if?#1?
      \def\childdoctmp
      {
        \def\childdocname{#2}
        \def\childdocjob{#2}
        \def\jobname{#2}
        \input{#2}
        \endinput
      }
    \else
      \def\childdoctmp
      {
        \childdocdisable
        \def\childdocname{#2}
        \childdoctrue
        \includeonly{#2}
        \def\childdocjob{#1}
        \def\jobname{#1}
        \input{#1}
        \endinput
      }
    \fi
    \expandafter
  \endgroup
  \childdoctmp
}
%    \end{macrocode}

% \macro{\childdocforwardprefix}
% The command |\childdocforwardprefix| redirects
% compilation to the main or a child file by means of a pattern.
% The prefix |#1| in the current filename is replaced by |#2|
% and the suffix of the current filename is kept
% (it is assumed that the filename does not contain the substring `|~~~|'
% which is used as a delimiter).
% Compilation is handed over to the new file by |\childdocforward|:
%    \begin{macrocode}
\newcommand{\childdocforwardprefix}[3][]
{
  \begingroup
    \def\childdocextract #2##1~~~{\def\childdoctmp{\childdocforward[#1]{#3##1}}}
    \expandafter\childdocextract\childdocname~~~
    \expandafter
  \endgroup
  \childdoctmp
}
%    \end{macrocode}

% \macro{\childdoc}
% The deprecated macro |\childdoc| is a legacy version of |\childdocmain|:
%    \begin{macrocode}
\newcommand{\childdoc}{\childdocmain}
%    \end{macrocode}

% \macro{\childdocredirect}
% The deprecated macro |\childdocredirect| is a legacy version
% of |\childdocforward| and |\childdocforwardprefix|:
%    \begin{macrocode}
\newcommand{\childdocredirect}[2][]
{
  \begingroup
    \if?#1?
      \def\childdoctmp{\childdocforward{#2}}
    \else
      \def\childdoctmp{\childdocforwardprefix{#1}{#2}}
    \fi
    \expandafter
  \endgroup
  \childdoctmp
}
%    \end{macrocode}

%\iffalse
%</package>
%\fi
%
\endinput
|
and perform the replacements as outlined below.
Instead of |\childdocmain{|\textit{main}|}| add the following code
to the top of the main file:
%
\begin{center}
\begin{tabular}{l}
|\||ifdefined\childdocname\endinput\||fi\newif\ifchilddoc|\\
|\edef\childdocname{\scantokens\expandafter{\jobname\noexpand}}|\\
|\def\childdocmain{|\textit{main}|}\||ifx\childdocmain\childdocname\||else|\\
|\childdoctrue\includeonly{\childdocname}\let\jobname\childdocmain\||fi|\\
\end{tabular}
\end{center}
%
Instead of |\childdocof{|\textit{main}|}| just include the main file
at the top of each child file:
%
\begin{center}
|\input{|\textit{main}|}|
\end{center}
%
A simple redirection |\childdocforward{|\textit{dest}|}| is achieved by:
%
\begin{center}
|\def\jobname{|\textit{dest}|}\input{\jobname}|
\end{center}
%
The redirection with prefix
|\childdocforwardprefix[|\textit{prefix}|]{|\textit{dest}|}|
is accomplished by:
%
\begin{center}
\begin{tabular}{l}
|{\edef\jobname{\scantokens\expandafter{\jobname\noexpand}}|\\
|\def\redirectjob |\textit{prefix}|#1~~~{\gdef\jobname{|\textit{dest}|#1}}|\\
|\expandafter\redirectjob\jobname~~~}\input{\jobname}|
\end{tabular}
\end{center}

In an alternative approach,
child documents can be compiled by a specific command line
without additional code or specific definitions:
%
\begin{center}
|... -jobname "|\textit{target}|" "|[\textit{flags}]%
|\includeonly{|\textit{dest}|}\input{|\textit{main}|}"|
\end{center}
%

%%%%%%%%%%%%%%%%%%%%%%%%%%%%%%%%%%%%%%%%%%%%%%%%%%%%%%%%%%%%%%%%%%%%%%%%%%%%%%%%
%%%%%%%%%%%%%%%%%%%%%%%%%%%%%%%%%%%%%%%%%%%%%%%%%%%%%%%%%%%%%%%%%%%%%%%%%%%%%%%%
\section{Information}

%%%%%%%%%%%%%%%%%%%%%%%%%%%%%%%%%%%%%%%%%%%%%%%%%%%%%%%%%%%%%%%%%%%%%%%%%%%%%%%%
\subsection{Copyright}

Copyright \copyright{} 2017--2018 Niklas Beisert

This work may be distributed and/or modified under the
conditions of the \LaTeX{} Project Public License, either version 1.3
of this license or (at your option) any later version.
The latest version of this license is in
  \url{http://www.latex-project.org/lppl.txt}
and version 1.3 or later is part of all distributions of \LaTeX{}
version 2005/12/01 or later.

This work has the LPPL maintenance status `maintained'.

The Current Maintainer of this work is Niklas Beisert.

This work consists of the files |README.txt|, |childdoc.ins| and |childdoc.dtx|
as well as the derived files |childdoc.def|, |cdocsamp.tex|
with |cdocsch1.tex|, |cdocsch2.tex|, |cdocspt3.tex|, |cdocspt4.tex|,
|cdocsdrf.tex|, |cdocsfn1.tex|, |cdocsfn2.tex|
as well as |childdoc.pdf|.

%%%%%%%%%%%%%%%%%%%%%%%%%%%%%%%%%%%%%%%%%%%%%%%%%%%%%%%%%%%%%%%%%%%%%%%%%%%%%%%%
\subsection{Files and Installation}

The package consists of the files:
%
\begin{center}
\begin{tabular}{ll}
    |README.txt|   & readme file \\
    |childdoc.ins| & installation file \\
    |childdoc.dtx| & source file \\
    |childdoc.def| & definition file \\
    |cdocsamp.tex| & sample main file \\
    |cdocsch1.tex| & sample include file \\
    |cdocsch2.tex| & sample include file \\
    |cdocspt3.tex| & sample part file \\
    |cdocspt4.tex| & sample part file \\
    |cdocsdrf.tex| & sample redirection file \\
    |cdocsfn1.tex| & sample redirection file \\
    |cdocsfn2.tex| & sample redirection file \\
    |childdoc.pdf| & manual
\end{tabular}
\end{center}
%
The distribution consists of the files
|README.txt|, |childdoc.ins| and |childdoc.dtx|.
%
\begin{itemize}
\item
Run (pdf)\LaTeX{} on |childdoc.dtx|
to compile the manual |childdoc.pdf| (this file).
\item
Run \LaTeX{} on |childdoc.ins| to create the definitions file |childdoc.def|
and the sample |cdocsamp.tex| with include files
|cdocsch1.tex|, |cdocsch2.tex|, |cdocspt3.tex|, |cdocspt4.tex|,
|cdocsdrf.tex|, |cdocsfn1.tex|, |cdocsfn2.tex|.
Then copy the file |childdoc.def| to an appropriate directory of your \LaTeX{}
distribution, e.g.\ \textit{texmf-root}|/tex/latex/childdoc|.
\end{itemize}

%%%%%%%%%%%%%%%%%%%%%%%%%%%%%%%%%%%%%%%%%%%%%%%%%%%%%%%%%%%%%%%%%%%%%%%%%%%%%%%%
\subsection{Related CTAN Packages}

There are several other packages which offer a similar functionality:
%
\begin{itemize}
\item
The packages
\href{http://ctan.org/pkg/docmute}{\textsf{docmute}},
\href{http://ctan.org/pkg/includex}{\textsf{includex}} and
\href{http://ctan.org/pkg/standalone}{\textsf{standalone}}
provide commands to include only the document body of
a child file thus allowing both files to be compiled individually.
\item
The packages \href{http://ctan.org/pkg/subdocs}{\textsf{subdocs}}
and \href{http://ctan.org/pkg/subfiles}{\textsf{subfiles}}
provide structures in which the main and child documents can be
encapsulated and allowing them to be compiled individually.
The inclusion mechanism is different from the conventional |\include|.
\item
The package \href{http://ctan.org/pkg/combine}{\textsf{combine}}
is an elaborate solution to combine several documents into one.
\end{itemize}
%
See also the CTAN topic \href{http://ctan.org/topic/subdocs}{\textsf{subdocs}}
for further related packages.
The present package differs from the above solutions in that
a document structure constructed with the conventional |\include| mechanism
just needs two extra commands at the top of every file
such that all constituent files can be compiled individually.

%%%%%%%%%%%%%%%%%%%%%%%%%%%%%%%%%%%%%%%%%%%%%%%%%%%%%%%%%%%%%%%%%%%%%%%%%%%%%%%%
%\subsection{Feature Suggestions}
%
%The following is a list of features which may be useful for future
%versions of this package:
%%
%\begin{itemize}
%\item
%\ldots
%\end{itemize}

%%%%%%%%%%%%%%%%%%%%%%%%%%%%%%%%%%%%%%%%%%%%%%%%%%%%%%%%%%%%%%%%%%%%%%%%%%%%%%%%
\subsection{Revision History}

%%%%%%%%%%%%%%%%%%%%%%%%%%%%%%%%%%%%%%%%
\paragraph{v2.0:} 2018/12/30

\begin{itemize}
\item
immediate forward processing
\item
added |\childdocby| mechanism
\item
manual restructured
\end{itemize}

%%%%%%%%%%%%%%%%%%%%%%%%%%%%%%%%%%%%%%%%
\paragraph{v1.6:} 2018/01/17

\begin{itemize}
\item
application for development of include files
\item
corrections to manual
\end{itemize}

%%%%%%%%%%%%%%%%%%%%%%%%%%%%%%%%%%%%%%%%
\paragraph{v1.5:} 2017/05/21

\begin{itemize}
\item
more complete structuring introduced
\item
|\childdocof| introduced
\item
|\childdoc| renamed to |\childdocmain|
\item
|\childredirect| renamed to |\childdocforward| and |\childdocforwardprefix|
and functionality expanded
\end{itemize}

%%%%%%%%%%%%%%%%%%%%%%%%%%%%%%%%%%%%%%%%
\paragraph{v1.0:} 2017/04/27

\begin{itemize}
\item
manual and install package
\item
first version published on CTAN
\end{itemize}

%%%%%%%%%%%%%%%%%%%%%%%%%%%%%%%%%%%%%%%%
\paragraph{v0.6:} 2017/04/26

\begin{itemize}
\item
redirection mechanism added
\end{itemize}

%%%%%%%%%%%%%%%%%%%%%%%%%%%%%%%%%%%%%%%%
\paragraph{v0.5:} 2017/04/26

\begin{itemize}
\item
functionality in definition file
\end{itemize}


%%%%%%%%%%%%%%%%%%%%%%%%%%%%%%%%%%%%%%%%%%%%%%%%%%%%%%%%%%%%%%%%%%%%%%%%%%%%%%%%
%%%%%%%%%%%%%%%%%%%%%%%%%%%%%%%%%%%%%%%%%%%%%%%%%%%%%%%%%%%%%%%%%%%%%%%%%%%%%%%%
%%%%%%%%%%%%%%%%%%%%%%%%%%%%%%%%%%%%%%%%%%%%%%%%%%%%%%%%%%%%%%%%%%%%%%%%%%%%%%%%
\appendix

\settowidth\MacroIndent{\rmfamily\scriptsize 000\ }

 \DocInput{childdoc.dtx}

\end{document}
%</driver>
% \fi
%
% %%%%%%%%%%%%%%%%%%%%%%%%%%%%%%%%%%%%%%%%%%%%%%%%%%%%%%%%%%%%%%%%%%%%%%%%%%%%%%
% %%%%%%%%%%%%%%%%%%%%%%%%%%%%%%%%%%%%%%%%%%%%%%%%%%%%%%%%%%%%%%%%%%%%%%%%%%%%%%
% \section{Sample}
%\iffalse
%<*samplemain>
%\fi
%
% The following presents a sample document
% with two chapters, two parts, a title page,
% a compile flag as well as three forwarding files to set the flag.
% It consists of eight |.tex| files:
% \begin{center}
% \begin{tabular}{ll}
% |cdocsamp.tex|&main file\\
% |cdocsch1.tex|&include file for chapter 1\\
% |cdocsch2.tex|&include file for chapter 2\\
% |cdocspt3.tex|&include file for part 3\\
% |cdocspt4.tex|&include file for part 4\\
% |cdocsdrf.tex|&forwarding file for main file in draft mode\\
% |cdocsfi1.tex|&forwarding file for final version of chapter 1\\
% |cdocsfi2.tex|&forwarding file for final version of chapter 2\\
% \end{tabular}
% \end{center}
% Each of the eight files can be compiled directly by the \LaTeX{} compiler.
%
% %%%%%%%%%%%%%%%%%%%%%%%%%%%%%%%%%%%%%%
% \paragraph{Main File.}
%
% The main file is called |cdocsamp.tex|.
%
% Load the \textsf{childdoc} definitions and
% declare the filename for the main document:
%    \begin{macrocode}
% \iffalse
%
% childdoc.dtx Copyright (C) 2017-2018 Niklas Beisert
%
% This work may be distributed and/or modified under the
% conditions of the LaTeX Project Public License, either version 1.3
% of this license or (at your option) any later version.
% The latest version of this license is in
%   http://www.latex-project.org/lppl.txt
% and version 1.3 or later is part of all distributions of LaTeX
% version 2005/12/01 or later.
%
% This work has the LPPL maintenance status `maintained'.
%
% The Current Maintainer of this work is Niklas Beisert.
%
% This work consists of the files childdoc.dtx and childdoc.ins
% and the derived files childdoc.def and cdocsamp.tex with
% cdocsch1.tex, cdocsch2.tex, cdocsdrf.tex, cdocsfn1.tex, cdocsfn2.tex.
%
%<package>\ifdefined\childdocmain\endinput\fi
%<package>\ProvidesFile{childdoc.def}[2018/12/30 v2.0 child document driver]
%<samplemain>\ProvidesFile{cdocsamp.tex}[2018/12/30 v2.0 sample for childdoc]
%<*driver>
%\ProvidesFile{childdoc.drv}[2018/12/30 v2.0 childdoc reference manual file]
\PassOptionsToClass{10pt,a4paper}{article}
\documentclass{ltxdoc}

\usepackage[margin=35mm]{geometry}
\usepackage{hyperref}
\usepackage{hyperxmp}
\usepackage[usenames]{color}

\hypersetup{colorlinks=true}
\hypersetup{pdfstartview=FitH}
\hypersetup{pdfpagemode=UseNone}
\hypersetup{pdfsource={}}
\hypersetup{pdflang={en-UK}}
\hypersetup{pdfcopyright={Copyright 2017-2018 Niklas Beisert.
  This work may be distributed and/or modified under the
  conditions of the LaTeX Project Public License, either version 1.3
  of this license or (at your option) any later version.}}
\hypersetup{pdflicenseurl={http://www.latex-project.org/lppl.txt}}
\hypersetup{pdfcontactaddress={ETH Zurich, ITP, HIT K,
  Wolfgang-Pauli-Strasse 27}}
\hypersetup{pdfcontactpostcode={8093}}
\hypersetup{pdfcontactcity={Zurich}}
\hypersetup{pdfcontactcountry={Switzerland}}
\hypersetup{pdfcontactemail={nbeisert@itp.phys.ethz.ch}}
\hypersetup{pdfcontacturl={http://people.phys.ethz.ch/\xmptilde nbeisert/}}

\newcommand{\secref}[1]{\hyperref[#1]{section \ref*{#1}}}

\parskip1ex
\parindent0pt
\let\olditemize\itemize
\def\itemize{\olditemize\parskip0pt}

\begin{document}

\title{The \textsf{childdoc} Package}
\hypersetup{pdftitle={The childdoc Package}}
\author{Niklas Beisert\\[2ex]
  Institut f\"ur Theoretische Physik\\
  Eidgen\"ossische Technische Hochschule Z\"urich\\
  Wolfgang-Pauli-Strasse 27, 8093 Z\"urich, Switzerland\\[1ex]
  \href{mailto:nbeisert@itp.phys.ethz.ch}
  {\texttt{nbeisert@itp.phys.ethz.ch}}}
\hypersetup{pdfauthor={Niklas Beisert}}
\hypersetup{pdfsubject={Manual for the LaTeX2e Package childdoc}}
\date{30 December 2018, \textsf{v2.0}}
\maketitle

\begin{abstract}\noindent
\textsf{childdoc} is a \LaTeXe{} package
that enables the direct compilation
of document sections included by |\include|
to individual files.
\end{abstract}

\begingroup
\parskip0ex
\tableofcontents
\endgroup

%%%%%%%%%%%%%%%%%%%%%%%%%%%%%%%%%%%%%%%%%%%%%%%%%%%%%%%%%%%%%%%%%%%%%%%%%%%%%%%%
%%%%%%%%%%%%%%%%%%%%%%%%%%%%%%%%%%%%%%%%%%%%%%%%%%%%%%%%%%%%%%%%%%%%%%%%%%%%%%%%
\section{Introduction}

\LaTeX{} provides a mechanism to structure a large document (such as a book)
into a main file and several child files (containing the chapters)
using the |\include| command.
This mechanism is beneficial for documents
which span hundreds of pages in order to
make the source file(s) more manageable.
Moreover, compilation can be restricted to
selected child files by means of the |\includeonly| command.
The latter feature can be used to reduce the compilation time while editing
(this was significantly more useful in the earlier days of \LaTeX{})
or to generate a smaller document which is easier to navigate.
Another application of |\includeonly| is to generate
documents consisting of selected parts of the complete document.

However, there are a few drawbacks of the plain |\include| mechanism:
\begin{itemize}
\item
The child files cannot be compiled on their own,
they can only be compiled via the main file.
A naive editing environment
(such as a text editor with an option
to have the current file processed by \LaTeX)
may require one to switch to the main file before compiling;
attempting to compile the child file produces errors.
\item
The main file must be modified (each time)
to adjust the |\includeonly| command
to the present needs. This easily leaves the main file in a messy state.
\item
The generated document will always carry the filename
of the main document. This is inconvenient if
several child files are to be compiled and
to be kept for distribution.
\end{itemize}

The present package provides a simple interface
to make child files individually compilable by \LaTeX{}.
Compiling a child file then has the same effect as compiling
the main file with an |\includeonly| command
to select the appropriate child.
Moreover the generated document will carry the name of the child
rather than the main file.
This resolves all three above issues.

This feature is meant to make the editing of books,
thesis documents and lecture notes somewhat more convenient.
However, the package can also be used efficiently for
composing a series of documents (such as exercise sheets)
which are typically distributed individually.
It then assists the author in generating the individual documents
(potentially in different versions)
as well as a document containing the collected series.
Another application is in developing style files
or other kinds of included material
where compilation of the style file could redirect
to a sample or test file.

%%%%%%%%%%%%%%%%%%%%%%%%%%%%%%%%%%%%%%%%%%%%%%%%%%%%%%%%%%%%%%%%%%%%%%%%%%%%%%%%
%%%%%%%%%%%%%%%%%%%%%%%%%%%%%%%%%%%%%%%%%%%%%%%%%%%%%%%%%%%%%%%%%%%%%%%%%%%%%%%%
\section{Usage}

First of all, the package \textsf{childdoc} is \emph{not} a standard
\LaTeXe{} |.sty| style file! Therefore it needs to be invoked in
a non-standard way.

%%%%%%%%%%%%%%%%%%%%%%%%%%%%%%%%%%%%%%%%%%%%%%%%%%%%%%%%%%%%%%%%%%%%%%%%%%%%%%%%
\subsection{Included Files}
\label{sec:include}

%%%%%%%%%%%%%%%%%%%%%%%%%%%%%%%%%%%%%%%%
\DescribeMacro{\childdocmain}
To use the package, add the commands
\begin{center}
\begin{tabular}{l}
|\input{childdoc.def}|\\
|\childdocmain{}|\\
\end{tabular}
\end{center}
at the very top of the main \LaTeX{} file,
in particular \emph{before} the |\documentclass| statement!
The argument of |\childdocmain| should be left empty
(but it must be present).

%%%%%%%%%%%%%%%%%%%%%%%%%%%%%%%%%%%%%%%%
\DescribeMacro{\childdocof}
Furthermore, add the commands
\begin{center}
\begin{tabular}{l}
|\input{childdoc.def}|\\
|\childdocof{|\textit{main}|}|\\
\end{tabular}
\end{center}
at the top of every child file \textit{child}
which is included by |\include{|\textit{child}|}|
from within the main file
(or at least for those files to be compiled individually).
The argument \textit{main} must be the filename of the main file.

There are a couple of
considerations in setting up the main and child documents:

%%%%%%%%%%%%%%%%%%%%%%%%%%%%%%%%%%%%%%%%
\paragraph{Restrictions.}

Please note the following restrictions:
\begin{itemize}
\item
|\childdocmain| must be called with one argument \textit{main}
to ensure compatibility with earlier version of the package.
It must either be empty (|\childdocmain{}|)
or precisely match the filename of the main file in which it is specified.
See \secref{sec:detection} for further information.
\item
The filename \textit{main} must be specified without the |.tex| extension.
\item
The filename \textit{main} is case sensitive
(even in case-insensitive file systems)
due to internal string comparison.
\item
The argument \textit{main} should be fully expanded, it cannot be a macro.
\item
Subdirectories and special characters should be avoided in filenames.
\item
The command |\childdocmain{|\textit{main}|}| must be followed by a whitespace.
It should not be followed immediately by another command
or by a comment mark `|%|'.
This is because the \TeX{} parser reads the token immediately following
the argument of |\childdocmain| and puts it
at the beginning of every child section;
however, a white\-space is ignored.
\end{itemize}

%%%%%%%%%%%%%%%%%%%%%%%%%%%%%%%%%%%%%%%%
\paragraph{Content of Main File.}

It is advisable to place all content in the child files included by |\include|.
Any output contained in the main file will appear in all child documents
unless suppressed manually;
it cannot be suppressed automatically by the |\includeonly| directive
and thus should normally be avoided.
A method to include some content in the main file
by means of conditional processing is described in \secref{sec:conditional}.

%%%%%%%%%%%%%%%%%%%%%%%%%%%%%%%%%%%%%%%%
\paragraph{Page Numbering.}

When only a part of the document is compiled,
the appropriate numbering of pages
(as well as other status parameters)
is determined from the |.aux| files.
The latter contain information from previous passes.
However this information needs to propagate through
all intermediate child documents.
Therefore the page numbering in child documents may well
be inconsistent until the complete document is compiled at least once.

A useful (if unconventional) way to always ensure a consistent
page numbering is to restart the numbering in each child document
and denote the pages by `\textit{child}|.|\textit{page}'
where \textit{child} represents the chapter/section number of the child file.
This can be achieved by the command
|\numberwithin{page}{|\textit{child}|}|
of the \textsf{amsmath} package
where \textit{child} can be |chapter| or |section|
depending on the chosen structuring.
Alternatively, one can modify the macro |\thepage| appropriately
and reset the counter |page| at the start of each child file.

%%%%%%%%%%%%%%%%%%%%%%%%%%%%%%%%%%%%%%%%%%%%%%%%%%%%%%%%%%%%%%%%%%%%%%%%%%%%%%%%
\subsection{Conditional Processing}
\label{sec:conditional}

The package provides a mechanism to compile different versions
of a document. To customise the versions further some conditional processing
can come in handy to distinguish which version is being compiled.
The package provides two macros to describe the compilation context:

%%%%%%%%%%%%%%%%%%%%%%%%%%%%%%%%%%%%%%%%
\DescribeMacro{\ifchilddoc}
The conditional |\ifchilddoc| distinguishes between the compilation of
child documents and the main document:
%
\begin{center}
|\ifchilddoc |\textit{child-code}| |[|\||else |\textit{main-code}]| \||fi|
\end{center}

%%%%%%%%%%%%%%%%%%%%%%%%%%%%%%%%%%%%%%%%
\DescribeMacro{\childdocname}
\DescribeMacro{\childdocjob}
The macro |\childdocname| contains the filename (without extension)
of the main or child file being processed.
Note that |\childdocjob| will always contain the name of the main file.

%%%%%%%%%%%%%%%%%%%%%%%%%%%%%%%%%%%%%%%%
\paragraph{Title Page.}

Conditional processing can be used to include a title or banner page
in the main document when proper precautions are taken.
Importantly, the code in the main file should ensure that the page counter
(as well as other status parameters which are stored in the |.aux| files)
takes the same value after the conditional processing.
Otherwise the page numbers may take divergent values
depending on which part is compiled.

For example, a title page could be declared by:
%
\begin{center}
\begin{tabular}{l}
|\ifchilddoc\||else|\\
|\addtocounter{page}{-1}|\\
\textit{code for title page}\\
|\newpage|\\
|\||fi|
\end{tabular}
\end{center}
%
A banner page for the child documents can be generated by:
%
\begin{center}
\begin{tabular}{l}
|\ifchilddoc|\\
|\addtocounter{page}{-1}|\\
\textit{code for banner page}\\
|\newpage|\\
|\||fi|
\end{tabular}
\end{center}
%
Here one could write a message such as:
\begin{center}
|This is the part \childdocname{} of \childdocjob{}.|
\end{center}

%%%%%%%%%%%%%%%%%%%%%%%%%%%%%%%%%%%%%%%%%%%%%%%%%%%%%%%%%%%%%%%%%%%%%%%%%%%%%%%%
\subsection{Flags}
\label{sec:flags}

The package makes it easy to generate different versions
of the main or child documents.
To this end compilation flags can be defined
and assigned different default values.
They will be particularly useful in conjunction
with the forwarding mechanism described in \secref{sec:forward}.

For example, it may be useful to have a flag |\version|
which can be set to |draft| or |final|.
The document source will contain some conditional code
depending on the value of |\version|.
Suppose further, the flag should default to |final| for the main file
and to |draft| for child files
which is a natural assignment for editing the document.
This is achieved by placing the following code
in the preamble of the main document
(below the |\childdocmain| directive):
%
\begin{center}
\begin{tabular}{l}
|\ifchilddoc|\\
|\providecommand{\version}{draft}|\\
|\||else|\\
|\providecommand{\version}{final}|\\
|\||fi|
\end{tabular}
\end{center}
%
The definition by |\providecommand| makes sure
that previous definitions are not overwritten.
Further statements |\providecommand{\version}{...}|
can thus be added before the above code to override it.

For the main file, one might add a line
(between |\childdocmain| and the above block)
%
\begin{center}
|%\ifchilddoc\||else\providecommand{\version}{draft}\||fi|
\end{center}
%
which can be uncommented to produce a draft version.
Likewise one can add a line to the very top of a child file
(above the |\childdocof{|\textit{main}|}| directive)
%
\begin{center}
|%\providecommand{\version}{final}|
\end{center}
%
which can be uncommented to produce the final version of this child document.

%%%%%%%%%%%%%%%%%%%%%%%%%%%%%%%%%%%%%%%%%%%%%%%%%%%%%%%%%%%%%%%%%%%%%%%%%%%%%%%%
\subsection{Forwarding}
\label{sec:forward}

Different versions of the main or child documents
using compilation flags as described in \secref{sec:flags}
can be (permanently) stored in different files
for convenient compilation, viewing and distribution.
To this end, the package defines a command
to pass on compilation to a different file:

%%%%%%%%%%%%%%%%%%%%%%%%%%%%%%%%%%%%%%%%
\DescribeMacro{\childdocforward}
The command |\childdocforward| redirects processing to
another source file:
%
\begin{center}
\begin{tabular}{l}
|\input{childdoc.def}|\\
|\childdocforward[|\textit{main}|]{|\textit{dest}|}|\\
\end{tabular}
\end{center}
%
The argument \textit{dest} is the destination file
(without extension).
It should be the main file or one of the child files.
Note that further \textsf{childdoc} directives
such as |\childdocof| and |\childdocforward|
in the indicated file will be processed in this form.
The optional argument \textit{main}
passes on directly to the main file \textit{main}
while pretending to compile the child \textit{dest}.
This form behaves as if \textit{dest}
issues |\childdocof{|\textit{main}|}| right away,
and no further \textsf{childdoc} directives will be processed.

%%%%%%%%%%%%%%%%%%%%%%%%%%%%%%%%%%%%%%%%
\DescribeMacro{\...prefix}
In the alternative form |\childdocforwardprefix|,
%
\begin{center}
\begin{tabular}{l}
|\input{childdoc.def}|\\
|\childdocforwardprefix[|\textit{main}|]{|\textit{prefix}|}{|\textit{dest}|}|
\end{tabular}
\end{center}
%
the destination file is determined by a pattern
depending on the current file:
To make this work, the current file must be called
`{\textit{prefix}\hspace{0.2em}\textit{suffix}}'
with \textit{prefix} matching precisely the argument.
Processing is then passed on to the file
`{\textit{dest}\hspace{0.2em}\textit{suffix}}'.
Surely, the same effect is achieved by
directly specifying the
argument `{\textit{dest}\hspace{0.2em}\textit{suffix}}'
in the first form.
However, that requires to set up a different file
for each child. With the alternative form of the command
all these files can have exactly the same content
which simplifies setting them up and maintaining them.

For example, the following file |draft.tex|
with a compilation flag |\version| as described in \secref{sec:flags}
compiles the main document as a draft:
%
\begin{center}
\begin{tabular}{l}
|\def\version{draft}|\\
|\input{childdoc.def}|\\
|\childdocforward{|\textit{main}|}|
\end{tabular}
\end{center}
%
Likewise, the following files |final|\textit{nn}|.tex|
compile the final version of the child document
|child|\textit{nn}|.tex|:
%
\begin{center}
\begin{tabular}{l}
|\def\version{final}|\\
|\input{childdoc.def}|\\
|\childdocforwardprefix{final}{child}|
\end{tabular}
\end{center}
%

Note that when several versions of a main file and/or of each child file
are to be generated, it may be convenient to set up a |Makefile| or
shell script to automatise the process.

%%%%%%%%%%%%%%%%%%%%%%%%%%%%%%%%%%%%%%%%%%%%%%%%%%%%%%%%%%%%%%%%%%%%%%%%%%%%%%%%
\subsection{Command Line Processing}
\label{sec:commandline}

The effect of redirection files can also be achieved by invoking
the \LaTeX{} compiler with a more elaborate command line.
Most conveniently this should be done as part
of a shell script or a |Makefile|.

When using \textsf{childdoc} in the main file, the following
command lines effectively perform a redirection
(note that depending on the shell being used,
backslashes may have to be doubled: `|\|' $\to$ `|\\|'):
%
\begin{center}
|... -jobname "|\textit{target}|" |\\|"|[\textit{flags}]%
|\input{childdoc.def}\childdocforward[|\textit{main}|]{|\textit{dest}|}"|
\end{center}
%
Here \textit{target} is the name of the output file,
\textit{main} is the name of the main file
and \textit{dest} is the name of the main or child file to be processed
(all filenames without extensions).
The optional argument \textit{main} can be omitted
if \textit{main} matches \textit{dest}.
Optionally, compilation \textit{flags} can be defined via |\def| commands.
This command line makes the \TeX{} engine believe
it is compiling the file \textit{target}
whose content is specified as the latter parameter.
The provided code then forwards the processing to
\textit{main} or \textit{dest} as described in \secref{sec:forward}.

%%%%%%%%%%%%%%%%%%%%%%%%%%%%%%%%%%%%%%%%%%%%%%%%%%%%%%%%%%%%%%%%%%%%%%%%%%%%%%%%
\subsection{Include by Input}
\label{sec:input}

Including child documents by |\include| has some restrictions by design.
Most notably, the content of a child document always occupies
its own set of pages; pages cannot be shared between child documents.
Usually, this behaviour makes perfect sense
because each child document contain an essential part of the document.
However, in some situations it may be desirable to compose
a document from a collection of parts
without having mandatory page breaks between then.
For this case, the package
provides a mechanism to include parts
by |\input| which can also be processed individually.
However, by construction this mechanism
requires manual handling of the content to be output.

%%%%%%%%%%%%%%%%%%%%%%%%%%%%%%%%%%%%%%%%
\DescribeMacro{\ifchilddocmanual}
The main file should be prepared as usual, see \secref{sec:include}.
However, the document body must make a distinction
between processing of an individual part and of the main document, e.g.:
%
\begin{center}
\begin{tabular}{l}
|\ifchilddocmanual|\\
|\input{\childdocname}|\\
|\||else|\\
\textit{document body with }|\input{|\textit{part}|}|\\
|\||fi|
\end{tabular}
\end{center}
%
The conditional |\ifchilddocmanual| is true whenever
a part to be included by |\input| is being compiled,
and the name of the part is stored in |\childdocname|.

%%%%%%%%%%%%%%%%%%%%%%%%%%%%%%%%%%%%%%%%
\DescribeMacro{\childdocby}
Each part to be included by |\input| should start with:
%
\begin{center}
\begin{tabular}{l}
|\input{childdoc.def}|\\
|\childdocby{|\textit{main}|}|\\
\end{tabular}
\end{center}
%
The directive |\childdocby| is similar to |\childdocof|
described in \secref{sec:include},
but the subsequent selection of content must be done manually.
To that end, both |\ifchilddoc| and |\ifchilddocmanual|
will be true upon processing of a part,
and the name of the part is stored in |\childdocname|.
Note that |\jobname| will be set to the filename of the current part
so that each part receives an individual |.aux| file
that does not interfere with the |.aux| file(s) of the main document.
This behaviour can be altered by the alternative form
|\childdocby[*]{|\textit{main}|}| (with a non-empty optional argument)
which uses the |.aux| file of the main document
by setting |\jobname| to \textit{main}.

%%%%%%%%%%%%%%%%%%%%%%%%%%%%%%%%%%%%%%%%%%%%%%%%%%%%%%%%%%%%%%%%%%%%%%%%%%%%%%%%
\subsection{Driver Development}
\label{sec:driver}

The \textsf{childdoc} mechanism can also be use for the development
of definition files such as \LaTeX{} styles or classes.
This case differs from the above setup with multiple parts
included by |\include| in that no |\includeonly| should be invoked.
This can be achieved by starting the include file
(before |\ProvidesPackage|) with:
%
\begin{center}
\begin{tabular}{l}
|\input{childdoc.def}|\\
|\childdocforward{|\textit{main}|}|\\
\end{tabular}
\end{center}
%
or alternatively with:
%
\begin{center}
\begin{tabular}{l}
|\input{childdoc.def}|\\
|\childdocby{|\textit{main}|}|\\
\end{tabular}
\end{center}
%
Both forms have slightly different effects as described above.
The main file is prepared as usual, see \secref{sec:include}.

%%%%%%%%%%%%%%%%%%%%%%%%%%%%%%%%%%%%%%%%%%%%%%%%%%%%%%%%%%%%%%%%%%%%%%%%%%%%%%%%
\subsection{Legacy Detection}
\label{sec:detection}

The directive |\childdocmain| in the main file can detect
whether the complete document or merely a child is to be compiled
even without using the directive |\childdocof|.
This method is deprecated because it is less robust
and there is no compelling reason to use it;
it is merely provided for backward compatibility
and it may be removed in future versions.

If the detection mechanism is to be used,
it is mandatory to correctly specify
the filename of the main file as the argument of |\childdocmain|:
%
\begin{center}
\begin{tabular}{l}
|\input{childdoc.def}|\\
|\childdocmain{|\textit{main}|}|\\
\end{tabular}
\end{center}
%
If |\jobname| does not match the argument \textit{main} of |\childdocmain|,
it is assumed that |\jobname| points to the child file to be compiled.
When using |\childdocmain| with the main file specified as argument,
it suffices to start a child file
with just |\input{|\textit{main}|}|
without loading of the package and using |\childdocof|.
If instead all processing is done
with the appropriate \textsf{childdoc} directives,
the argument of \textit{main} of |\childdocmain| can be empty.

An alternative version of the command line processing described
in \secref{sec:commandline} using the detection mechanism reads:
%
\begin{center}
|... -jobname "|\textit{target}|" "|[\textit{flags}]%
[|\def\jobname{|\textit{dest}|}|]|\input{|\textit{main}|}"|
\end{center}

%%%%%%%%%%%%%%%%%%%%%%%%%%%%%%%%%%%%%%%%%%%%%%%%%%%%%%%%%%%%%%%%%%%%%%%%%%%%%%%%
\subsection{Manual Code}
\label{sec:manual}

In case one cannot be certain whether the definitions file |childdoc.def|
is installed on the target \TeX{} distribution
and one prefers not to ship it,
it is conceivable to paste a few relevant commands into the sources.

To that end, drop all statements |\input{childdoc.def}|
and perform the replacements as outlined below.
Instead of |\childdocmain{|\textit{main}|}| add the following code
to the top of the main file:
%
\begin{center}
\begin{tabular}{l}
|\||ifdefined\childdocname\endinput\||fi\newif\ifchilddoc|\\
|\edef\childdocname{\scantokens\expandafter{\jobname\noexpand}}|\\
|\def\childdocmain{|\textit{main}|}\||ifx\childdocmain\childdocname\||else|\\
|\childdoctrue\includeonly{\childdocname}\let\jobname\childdocmain\||fi|\\
\end{tabular}
\end{center}
%
Instead of |\childdocof{|\textit{main}|}| just include the main file
at the top of each child file:
%
\begin{center}
|\input{|\textit{main}|}|
\end{center}
%
A simple redirection |\childdocforward{|\textit{dest}|}| is achieved by:
%
\begin{center}
|\def\jobname{|\textit{dest}|}\input{\jobname}|
\end{center}
%
The redirection with prefix
|\childdocforwardprefix[|\textit{prefix}|]{|\textit{dest}|}|
is accomplished by:
%
\begin{center}
\begin{tabular}{l}
|{\edef\jobname{\scantokens\expandafter{\jobname\noexpand}}|\\
|\def\redirectjob |\textit{prefix}|#1~~~{\gdef\jobname{|\textit{dest}|#1}}|\\
|\expandafter\redirectjob\jobname~~~}\input{\jobname}|
\end{tabular}
\end{center}

In an alternative approach,
child documents can be compiled by a specific command line
without additional code or specific definitions:
%
\begin{center}
|... -jobname "|\textit{target}|" "|[\textit{flags}]%
|\includeonly{|\textit{dest}|}\input{|\textit{main}|}"|
\end{center}
%

%%%%%%%%%%%%%%%%%%%%%%%%%%%%%%%%%%%%%%%%%%%%%%%%%%%%%%%%%%%%%%%%%%%%%%%%%%%%%%%%
%%%%%%%%%%%%%%%%%%%%%%%%%%%%%%%%%%%%%%%%%%%%%%%%%%%%%%%%%%%%%%%%%%%%%%%%%%%%%%%%
\section{Information}

%%%%%%%%%%%%%%%%%%%%%%%%%%%%%%%%%%%%%%%%%%%%%%%%%%%%%%%%%%%%%%%%%%%%%%%%%%%%%%%%
\subsection{Copyright}

Copyright \copyright{} 2017--2018 Niklas Beisert

This work may be distributed and/or modified under the
conditions of the \LaTeX{} Project Public License, either version 1.3
of this license or (at your option) any later version.
The latest version of this license is in
  \url{http://www.latex-project.org/lppl.txt}
and version 1.3 or later is part of all distributions of \LaTeX{}
version 2005/12/01 or later.

This work has the LPPL maintenance status `maintained'.

The Current Maintainer of this work is Niklas Beisert.

This work consists of the files |README.txt|, |childdoc.ins| and |childdoc.dtx|
as well as the derived files |childdoc.def|, |cdocsamp.tex|
with |cdocsch1.tex|, |cdocsch2.tex|, |cdocspt3.tex|, |cdocspt4.tex|,
|cdocsdrf.tex|, |cdocsfn1.tex|, |cdocsfn2.tex|
as well as |childdoc.pdf|.

%%%%%%%%%%%%%%%%%%%%%%%%%%%%%%%%%%%%%%%%%%%%%%%%%%%%%%%%%%%%%%%%%%%%%%%%%%%%%%%%
\subsection{Files and Installation}

The package consists of the files:
%
\begin{center}
\begin{tabular}{ll}
    |README.txt|   & readme file \\
    |childdoc.ins| & installation file \\
    |childdoc.dtx| & source file \\
    |childdoc.def| & definition file \\
    |cdocsamp.tex| & sample main file \\
    |cdocsch1.tex| & sample include file \\
    |cdocsch2.tex| & sample include file \\
    |cdocspt3.tex| & sample part file \\
    |cdocspt4.tex| & sample part file \\
    |cdocsdrf.tex| & sample redirection file \\
    |cdocsfn1.tex| & sample redirection file \\
    |cdocsfn2.tex| & sample redirection file \\
    |childdoc.pdf| & manual
\end{tabular}
\end{center}
%
The distribution consists of the files
|README.txt|, |childdoc.ins| and |childdoc.dtx|.
%
\begin{itemize}
\item
Run (pdf)\LaTeX{} on |childdoc.dtx|
to compile the manual |childdoc.pdf| (this file).
\item
Run \LaTeX{} on |childdoc.ins| to create the definitions file |childdoc.def|
and the sample |cdocsamp.tex| with include files
|cdocsch1.tex|, |cdocsch2.tex|, |cdocspt3.tex|, |cdocspt4.tex|,
|cdocsdrf.tex|, |cdocsfn1.tex|, |cdocsfn2.tex|.
Then copy the file |childdoc.def| to an appropriate directory of your \LaTeX{}
distribution, e.g.\ \textit{texmf-root}|/tex/latex/childdoc|.
\end{itemize}

%%%%%%%%%%%%%%%%%%%%%%%%%%%%%%%%%%%%%%%%%%%%%%%%%%%%%%%%%%%%%%%%%%%%%%%%%%%%%%%%
\subsection{Related CTAN Packages}

There are several other packages which offer a similar functionality:
%
\begin{itemize}
\item
The packages
\href{http://ctan.org/pkg/docmute}{\textsf{docmute}},
\href{http://ctan.org/pkg/includex}{\textsf{includex}} and
\href{http://ctan.org/pkg/standalone}{\textsf{standalone}}
provide commands to include only the document body of
a child file thus allowing both files to be compiled individually.
\item
The packages \href{http://ctan.org/pkg/subdocs}{\textsf{subdocs}}
and \href{http://ctan.org/pkg/subfiles}{\textsf{subfiles}}
provide structures in which the main and child documents can be
encapsulated and allowing them to be compiled individually.
The inclusion mechanism is different from the conventional |\include|.
\item
The package \href{http://ctan.org/pkg/combine}{\textsf{combine}}
is an elaborate solution to combine several documents into one.
\end{itemize}
%
See also the CTAN topic \href{http://ctan.org/topic/subdocs}{\textsf{subdocs}}
for further related packages.
The present package differs from the above solutions in that
a document structure constructed with the conventional |\include| mechanism
just needs two extra commands at the top of every file
such that all constituent files can be compiled individually.

%%%%%%%%%%%%%%%%%%%%%%%%%%%%%%%%%%%%%%%%%%%%%%%%%%%%%%%%%%%%%%%%%%%%%%%%%%%%%%%%
%\subsection{Feature Suggestions}
%
%The following is a list of features which may be useful for future
%versions of this package:
%%
%\begin{itemize}
%\item
%\ldots
%\end{itemize}

%%%%%%%%%%%%%%%%%%%%%%%%%%%%%%%%%%%%%%%%%%%%%%%%%%%%%%%%%%%%%%%%%%%%%%%%%%%%%%%%
\subsection{Revision History}

%%%%%%%%%%%%%%%%%%%%%%%%%%%%%%%%%%%%%%%%
\paragraph{v2.0:} 2018/12/30

\begin{itemize}
\item
immediate forward processing
\item
added |\childdocby| mechanism
\item
manual restructured
\end{itemize}

%%%%%%%%%%%%%%%%%%%%%%%%%%%%%%%%%%%%%%%%
\paragraph{v1.6:} 2018/01/17

\begin{itemize}
\item
application for development of include files
\item
corrections to manual
\end{itemize}

%%%%%%%%%%%%%%%%%%%%%%%%%%%%%%%%%%%%%%%%
\paragraph{v1.5:} 2017/05/21

\begin{itemize}
\item
more complete structuring introduced
\item
|\childdocof| introduced
\item
|\childdoc| renamed to |\childdocmain|
\item
|\childredirect| renamed to |\childdocforward| and |\childdocforwardprefix|
and functionality expanded
\end{itemize}

%%%%%%%%%%%%%%%%%%%%%%%%%%%%%%%%%%%%%%%%
\paragraph{v1.0:} 2017/04/27

\begin{itemize}
\item
manual and install package
\item
first version published on CTAN
\end{itemize}

%%%%%%%%%%%%%%%%%%%%%%%%%%%%%%%%%%%%%%%%
\paragraph{v0.6:} 2017/04/26

\begin{itemize}
\item
redirection mechanism added
\end{itemize}

%%%%%%%%%%%%%%%%%%%%%%%%%%%%%%%%%%%%%%%%
\paragraph{v0.5:} 2017/04/26

\begin{itemize}
\item
functionality in definition file
\end{itemize}


%%%%%%%%%%%%%%%%%%%%%%%%%%%%%%%%%%%%%%%%%%%%%%%%%%%%%%%%%%%%%%%%%%%%%%%%%%%%%%%%
%%%%%%%%%%%%%%%%%%%%%%%%%%%%%%%%%%%%%%%%%%%%%%%%%%%%%%%%%%%%%%%%%%%%%%%%%%%%%%%%
%%%%%%%%%%%%%%%%%%%%%%%%%%%%%%%%%%%%%%%%%%%%%%%%%%%%%%%%%%%%%%%%%%%%%%%%%%%%%%%%
\appendix

\settowidth\MacroIndent{\rmfamily\scriptsize 000\ }

 \DocInput{childdoc.dtx}

\end{document}
%</driver>
% \fi
%
% %%%%%%%%%%%%%%%%%%%%%%%%%%%%%%%%%%%%%%%%%%%%%%%%%%%%%%%%%%%%%%%%%%%%%%%%%%%%%%
% %%%%%%%%%%%%%%%%%%%%%%%%%%%%%%%%%%%%%%%%%%%%%%%%%%%%%%%%%%%%%%%%%%%%%%%%%%%%%%
% \section{Sample}
%\iffalse
%<*samplemain>
%\fi
%
% The following presents a sample document
% with two chapters, two parts, a title page,
% a compile flag as well as three forwarding files to set the flag.
% It consists of eight |.tex| files:
% \begin{center}
% \begin{tabular}{ll}
% |cdocsamp.tex|&main file\\
% |cdocsch1.tex|&include file for chapter 1\\
% |cdocsch2.tex|&include file for chapter 2\\
% |cdocspt3.tex|&include file for part 3\\
% |cdocspt4.tex|&include file for part 4\\
% |cdocsdrf.tex|&forwarding file for main file in draft mode\\
% |cdocsfi1.tex|&forwarding file for final version of chapter 1\\
% |cdocsfi2.tex|&forwarding file for final version of chapter 2\\
% \end{tabular}
% \end{center}
% Each of the eight files can be compiled directly by the \LaTeX{} compiler.
%
% %%%%%%%%%%%%%%%%%%%%%%%%%%%%%%%%%%%%%%
% \paragraph{Main File.}
%
% The main file is called |cdocsamp.tex|.
%
% Load the \textsf{childdoc} definitions and
% declare the filename for the main document:
%    \begin{macrocode}
\input{childdoc.def}
\childdocmain{}
%    \end{macrocode}

% Optional override for |\version| flag:
%    \begin{macrocode}
%%\ifchilddoc\else\providecommand{\version}{draft}\fi
%    \end{macrocode}

% Define the default values for the |\version| flag
% (|final| for the main file and |draft| for childs):
%    \begin{macrocode}
\ifchilddoc
\providecommand{\version}{draft}
\else
\providecommand{\version}{final}
\fi
%    \end{macrocode}

% Load the standard document class:
%    \begin{macrocode}
\documentclass[12pt]{article}
%    \end{macrocode}

% Start the document body:
%    \begin{macrocode}
\begin{document}
%    \end{macrocode}

% Declare a title page.
% Print title, part of document being processed and version flag:
%    \begin{macrocode}
\addtocounter{page}{-1}
\begin{center}
{\LARGE\bfseries{}childdoc example\par}
\vspace{1cm}
\ifchilddoc
\ifchilddocmanual part\else chapter\fi:
`\childdocname' of `\childdocjob'\par
\else
main document: `\childdocjob'\par
\fi
version: \version\par
\end{center}
\newpage
%    \end{macrocode}

% Manually include selected file,
% otherwise process as usual:
%    \begin{macrocode}
\ifchilddocmanual
\section*{part `\childdocname'}
\input{\childdocname}
\else
%    \end{macrocode}

% Include the two chapters:
%    \begin{macrocode}
\include{cdocsch1}
\include{cdocsch2}
%    \end{macrocode}

% Include the two parts unless only chapters should be displayed:
%    \begin{macrocode}
\ifchilddoc\else
\section{part three}
\input{cdocspt3}
\section{part four}
\input{cdocspt4}
\fi
%    \end{macrocode}

% Process as usual until here:
%    \begin{macrocode}
\fi
%    \end{macrocode}

% End of document body:
%    \begin{macrocode}
\end{document}
%    \end{macrocode}
%\iffalse
%</samplemain>
%\fi
%
% %%%%%%%%%%%%%%%%%%%%%%%%%%%%%%%%%%%%%%
% \paragraph{Chapter Include Files.}
%
% The include files are called |cdocsch1.tex| and |cdocsch2.tex|.
%
%\iffalse
%<*samplechap1|samplechap2>
%\fi

% Optional override for |\version| flag:
%    \begin{macrocode}
%%\providecommand{\version}{final}
%    \end{macrocode}

% Include the main document:
%    \begin{macrocode}
\input{childdoc.def}
\childdocof{cdocsamp}
%    \end{macrocode}

%\iffalse
%</samplechap1|samplechap2>
%\fi
%
%\iffalse
%<*samplechap1>
%\fi
% Some text for chapter 1:
%    \begin{macrocode}
\section{one}
some text in chapter one
%    \end{macrocode}

%\iffalse
%</samplechap1>
%\fi
% Some text for chapter 2:
%\iffalse
%<*samplechap2>
%\fi
%    \begin{macrocode}
\section{two}
more text in chapter two
%    \end{macrocode}

%\iffalse
%</samplechap2>
%\fi
%
% %%%%%%%%%%%%%%%%%%%%%%%%%%%%%%%%%%%%%%
% \paragraph{Part Include Files.}
%
% The include files are called |cdocspt3.tex| and |cdocspt4.tex|.
%
%\iffalse
%<*samplepart3|samplepart4>
%\fi

% Optional override for |\version| flag:
%    \begin{macrocode}
%%\providecommand{\version}{final}
%    \end{macrocode}

% Include the main document:
%    \begin{macrocode}
\input{childdoc.def}
\childdocby{cdocsamp}
%    \end{macrocode}

%\iffalse
%</samplepart3|samplepart4>
%\fi
%
%\iffalse
%<*samplepart3>
%\fi
% Some text for part 3:
%    \begin{macrocode}
some text in part three
%    \end{macrocode}

%\iffalse
%</samplepart3>
%\fi
% Some text for part 4:
%\iffalse
%<*samplepart4>
%\fi
%    \begin{macrocode}
more text in part four
%    \end{macrocode}

%\iffalse
%</samplepart4>
%\fi
%
% %%%%%%%%%%%%%%%%%%%%%%%%%%%%%%%%%%%%%%
% \paragraph{Forwarding for a Complete Draft.}
%
% The following forwarding file |cdocsdrf.tex|
% compiles the main document in draft mode:
%\iffalse
%<*sampledraft>
%\fi
%    \begin{macrocode}
\def\version{draft}
\input{childdoc.def}
\childdocforward{cdocsamp}
%    \end{macrocode}

%\iffalse
%</sampledraft>
%\fi
%
% %%%%%%%%%%%%%%%%%%%%%%%%%%%%%%%%%%%%%%
% \paragraph{Forwarding for Final Version of the Chapters.}
%
% The following forwarding files |cdocsfn1.tex| and |cdocsfn2.tex|
% (with identical content)
% compile the final versions of the child documents
% |cdocsch1.tex| and |cdocsch2.tex|, respectively:
%\iffalse
%<*samplefinal>
%\fi
%    \begin{macrocode}
\def\version{final}
\input{childdoc.def}
\childdocforwardprefix[cdocsamp]{cdocsfn}{cdocsch}
%    \end{macrocode}

%\iffalse
%</samplefinal>
%\fi
%
% %%%%%%%%%%%%%%%%%%%%%%%%%%%%%%%%%%%%%%
% \paragraph{Command Line Processing.}
%
% The following three command lines generate the output files
% |cdocscld|, |cdocscl1| and |cdocscl2|
% which should be identical to
% |cdocsdrf|, |cdocsch1| and |cdocsfn2|, respectively:
% \begin{center}
% \begin{tabular}{l}
% |latex -jobname cdocscld \|\\
% |  "\def\version{draft}\input{childdoc.def}\childdocforward{cdocsamp}"|\\
% |latex -jobname cdocscl1 \|\\
% |  "\input{childdoc.def}\childdocforward[cdocsamp]{cdocsch1}"|\\
% |latex -jobname cdocscl2 \|\\
% |  "\def\version{final}\input{childdoc.def}\childdocforward{cdocsch2}"|
% \end{tabular}
% \end{center}
% Note that the trailing backslash on each first line
% merely continues the input to the second line
% (for convenient cut ant paste).
% Furthermore, the command |latex| can be replaced by any
% of its alternative versions such as |pdflatex|.
%
% %%%%%%%%%%%%%%%%%%%%%%%%%%%%%%%%%%%%%%%%%%%%%%%%%%%%%%%%%%%%%%%%%%%%%%%%%%%%%%
% %%%%%%%%%%%%%%%%%%%%%%%%%%%%%%%%%%%%%%%%%%%%%%%%%%%%%%%%%%%%%%%%%%%%%%%%%%%%%%
% \section{Implementation}
%\iffalse
%<*package>
%\fi
%
% This section describes the definitions file |childdoc.def|.

% The definitions cannot be loaded using |\usepackage| or |\RequirePackage|
% which has a mechanism to prevent loading a style file more than once.
% When loading the definitions by means of |\input|
% multiple instances have to be prevented manually:
%\iffalse
%This code needs to be before the `\ProvidesFile' directive
%which is defined at the beginning of this file.
%Therefore it is also placed there and commented out here.
%</package>
%<*discard>
%\fi
%    \begin{macrocode}
\ifdefined\childdocmain\endinput\fi
%    \end{macrocode}
%\iffalse
%</discard>
%<*package>
%\fi
%
% \macro{\ifchilddoc}
% \macro{\ifchilddocmanual}
% The conditional |\ifchilddoc| tells whether a
% child (true) or main (false) document is being compiled.
% The conditional |\ifchilddocmanual| tells whether
% the |\includeonly| mechanism is used (false) or
% the selection of child files must be performed manually (true).
% The definitions initialise to false:
%    \begin{macrocode}
\newif\ifchilddoc
\newif\ifchilddocmanual
%    \end{macrocode}

% \macro{\childdocname}
% \macro{\childdocjob}
% The macro |\childdocname| stores the name of the main document
% to be compiled. The macro |\childdocjob| stores the name of
% the document on which the \LaTeX{} compiler was originally invoked.
% The content of |\jobname| cannot be compared
% to filenames specified in the source due to different catcodes.
% The following code rescans |\jobname|, stores the result
% in |\childdocname| and saves a copy in |\childdocjob|:
%    \begin{macrocode}
\edef\childdocname{\scantokens\expandafter{\jobname\noexpand}}
\let\childdocjob\childdocname
%    \end{macrocode}

% \macro{\childdocdisable}
% The macro |\childdocdisable| prevents the main file
% from being processed more than once.
% At this stage, the main document command |\childdocmain|
% is assumed to be called once again where it should do nothing.
% Any subsequent call to it should prevent
% a secondary processing of the main document
% It overwrites the forwarding commands
% |\childdocof| and |\childdocforward|
% with empty macros to prevent further inclusions of the main document:
%    \begin{macrocode}
\newcommand{\childdocdisable}
{
  \renewcommand{\childdocmain}[1]{\renewcommand{\childdocmain}[1]{\endinput}}
  \renewcommand{\childdocof}[1]{}
  \renewcommand{\childdocby}[2][]{}
  \renewcommand{\childdocforward}[2][]{}
  \renewcommand{\childdocdisable}{}
}
%    \end{macrocode}

% \macro{\childdocmain}
% The macro |\childdocmain| is to be called at the top of the main file
% with nothing or the main filename (without extension) as argument.
% First, it breaks loops.
% If the argument is not empty and does not match |\childdocname|
% (which is set by the first inclusion of |childdoc.def|),
% |\ifchilddoc| is set to true, |\includeonly| is applied to the child file
% and |\jobname| is set to the main file
% (for proper handling of |.aux| files):
%    \begin{macrocode}
\newcommand{\childdocmain}[1]
{
  \childdocdisable\childdocmain{}
  \if?#1?\else
    \begingroup
      \def\childdoctmp{#1}
      \ifx\childdoctmp\childdocname
        \def\childdoctmp{}
      \else
        \def\childdoctmp
        {
          \childdoctrue
          \includeonly{\childdocname}
          \def\childdocjob{#1}
          \def\jobname{#1}
        }
      \fi
      \expandafter
    \endgroup
    \childdoctmp
  \fi
}
%    \end{macrocode}

% \macro{\childdocof}
% The command |\childdocof| redirects
% compilation to the main file |#1|.
%    \begin{macrocode}
\newcommand{\childdocof}[1]
{
  \childdocdisable
  \childdoctrue
  \includeonly{\childdocname}
  \def\jobname{#1}
  \def\childdocjob{#1}
  \input{#1}
}
%    \end{macrocode}

% \macro{\childdocby}
% The command |\childdocby| ....
%    \begin{macrocode}
\newcommand{\childdocby}[2][]
{
  \childdocdisable
  \childdoctrue
  \childdocmanualtrue
  \if?#1?\else
    \def\jobname{#2}
  \fi
  \def\childdocjob{#2}
  \input{#2}
  \endinput
}
%    \end{macrocode}

% \macro{\childdocforward}
% The command |\childdocforward| redirects
% compilation to the main file or
% (if the optional argument is given) a child file.
% Parameters are set as if the main file
% or a child file starting with |\childdocof| was compiled.
% Then compilation is handed over to the main file:
%    \begin{macrocode}
\newcommand{\childdocforward}[2][]
{
  \begingroup
    \if?#1?
      \def\childdoctmp
      {
        \def\childdocname{#2}
        \def\childdocjob{#2}
        \def\jobname{#2}
        \input{#2}
        \endinput
      }
    \else
      \def\childdoctmp
      {
        \childdocdisable
        \def\childdocname{#2}
        \childdoctrue
        \includeonly{#2}
        \def\childdocjob{#1}
        \def\jobname{#1}
        \input{#1}
        \endinput
      }
    \fi
    \expandafter
  \endgroup
  \childdoctmp
}
%    \end{macrocode}

% \macro{\childdocforwardprefix}
% The command |\childdocforwardprefix| redirects
% compilation to the main or a child file by means of a pattern.
% The prefix |#1| in the current filename is replaced by |#2|
% and the suffix of the current filename is kept
% (it is assumed that the filename does not contain the substring `|~~~|'
% which is used as a delimiter).
% Compilation is handed over to the new file by |\childdocforward|:
%    \begin{macrocode}
\newcommand{\childdocforwardprefix}[3][]
{
  \begingroup
    \def\childdocextract #2##1~~~{\def\childdoctmp{\childdocforward[#1]{#3##1}}}
    \expandafter\childdocextract\childdocname~~~
    \expandafter
  \endgroup
  \childdoctmp
}
%    \end{macrocode}

% \macro{\childdoc}
% The deprecated macro |\childdoc| is a legacy version of |\childdocmain|:
%    \begin{macrocode}
\newcommand{\childdoc}{\childdocmain}
%    \end{macrocode}

% \macro{\childdocredirect}
% The deprecated macro |\childdocredirect| is a legacy version
% of |\childdocforward| and |\childdocforwardprefix|:
%    \begin{macrocode}
\newcommand{\childdocredirect}[2][]
{
  \begingroup
    \if?#1?
      \def\childdoctmp{\childdocforward{#2}}
    \else
      \def\childdoctmp{\childdocforwardprefix{#1}{#2}}
    \fi
    \expandafter
  \endgroup
  \childdoctmp
}
%    \end{macrocode}

%\iffalse
%</package>
%\fi
%
\endinput

\childdocmain{}
%    \end{macrocode}

% Optional override for |\version| flag:
%    \begin{macrocode}
%%\ifchilddoc\else\providecommand{\version}{draft}\fi
%    \end{macrocode}

% Define the default values for the |\version| flag
% (|final| for the main file and |draft| for childs):
%    \begin{macrocode}
\ifchilddoc
\providecommand{\version}{draft}
\else
\providecommand{\version}{final}
\fi
%    \end{macrocode}

% Load the standard document class:
%    \begin{macrocode}
\documentclass[12pt]{article}
%    \end{macrocode}

% Start the document body:
%    \begin{macrocode}
\begin{document}
%    \end{macrocode}

% Declare a title page.
% Print title, part of document being processed and version flag:
%    \begin{macrocode}
\addtocounter{page}{-1}
\begin{center}
{\LARGE\bfseries{}childdoc example\par}
\vspace{1cm}
\ifchilddoc
\ifchilddocmanual part\else chapter\fi:
`\childdocname' of `\childdocjob'\par
\else
main document: `\childdocjob'\par
\fi
version: \version\par
\end{center}
\newpage
%    \end{macrocode}

% Manually include selected file,
% otherwise process as usual:
%    \begin{macrocode}
\ifchilddocmanual
\section*{part `\childdocname'}
\input{\childdocname}
\else
%    \end{macrocode}

% Include the two chapters:
%    \begin{macrocode}
\include{cdocsch1}
\include{cdocsch2}
%    \end{macrocode}

% Include the two parts unless only chapters should be displayed:
%    \begin{macrocode}
\ifchilddoc\else
\section{part three}
\input{cdocspt3}
\section{part four}
\input{cdocspt4}
\fi
%    \end{macrocode}

% Process as usual until here:
%    \begin{macrocode}
\fi
%    \end{macrocode}

% End of document body:
%    \begin{macrocode}
\end{document}
%    \end{macrocode}
%\iffalse
%</samplemain>
%\fi
%
% %%%%%%%%%%%%%%%%%%%%%%%%%%%%%%%%%%%%%%
% \paragraph{Chapter Include Files.}
%
% The include files are called |cdocsch1.tex| and |cdocsch2.tex|.
%
%\iffalse
%<*samplechap1|samplechap2>
%\fi

% Optional override for |\version| flag:
%    \begin{macrocode}
%%\providecommand{\version}{final}
%    \end{macrocode}

% Include the main document:
%    \begin{macrocode}
% \iffalse
%
% childdoc.dtx Copyright (C) 2017-2018 Niklas Beisert
%
% This work may be distributed and/or modified under the
% conditions of the LaTeX Project Public License, either version 1.3
% of this license or (at your option) any later version.
% The latest version of this license is in
%   http://www.latex-project.org/lppl.txt
% and version 1.3 or later is part of all distributions of LaTeX
% version 2005/12/01 or later.
%
% This work has the LPPL maintenance status `maintained'.
%
% The Current Maintainer of this work is Niklas Beisert.
%
% This work consists of the files childdoc.dtx and childdoc.ins
% and the derived files childdoc.def and cdocsamp.tex with
% cdocsch1.tex, cdocsch2.tex, cdocsdrf.tex, cdocsfn1.tex, cdocsfn2.tex.
%
%<package>\ifdefined\childdocmain\endinput\fi
%<package>\ProvidesFile{childdoc.def}[2018/12/30 v2.0 child document driver]
%<samplemain>\ProvidesFile{cdocsamp.tex}[2018/12/30 v2.0 sample for childdoc]
%<*driver>
%\ProvidesFile{childdoc.drv}[2018/12/30 v2.0 childdoc reference manual file]
\PassOptionsToClass{10pt,a4paper}{article}
\documentclass{ltxdoc}

\usepackage[margin=35mm]{geometry}
\usepackage{hyperref}
\usepackage{hyperxmp}
\usepackage[usenames]{color}

\hypersetup{colorlinks=true}
\hypersetup{pdfstartview=FitH}
\hypersetup{pdfpagemode=UseNone}
\hypersetup{pdfsource={}}
\hypersetup{pdflang={en-UK}}
\hypersetup{pdfcopyright={Copyright 2017-2018 Niklas Beisert.
  This work may be distributed and/or modified under the
  conditions of the LaTeX Project Public License, either version 1.3
  of this license or (at your option) any later version.}}
\hypersetup{pdflicenseurl={http://www.latex-project.org/lppl.txt}}
\hypersetup{pdfcontactaddress={ETH Zurich, ITP, HIT K,
  Wolfgang-Pauli-Strasse 27}}
\hypersetup{pdfcontactpostcode={8093}}
\hypersetup{pdfcontactcity={Zurich}}
\hypersetup{pdfcontactcountry={Switzerland}}
\hypersetup{pdfcontactemail={nbeisert@itp.phys.ethz.ch}}
\hypersetup{pdfcontacturl={http://people.phys.ethz.ch/\xmptilde nbeisert/}}

\newcommand{\secref}[1]{\hyperref[#1]{section \ref*{#1}}}

\parskip1ex
\parindent0pt
\let\olditemize\itemize
\def\itemize{\olditemize\parskip0pt}

\begin{document}

\title{The \textsf{childdoc} Package}
\hypersetup{pdftitle={The childdoc Package}}
\author{Niklas Beisert\\[2ex]
  Institut f\"ur Theoretische Physik\\
  Eidgen\"ossische Technische Hochschule Z\"urich\\
  Wolfgang-Pauli-Strasse 27, 8093 Z\"urich, Switzerland\\[1ex]
  \href{mailto:nbeisert@itp.phys.ethz.ch}
  {\texttt{nbeisert@itp.phys.ethz.ch}}}
\hypersetup{pdfauthor={Niklas Beisert}}
\hypersetup{pdfsubject={Manual for the LaTeX2e Package childdoc}}
\date{30 December 2018, \textsf{v2.0}}
\maketitle

\begin{abstract}\noindent
\textsf{childdoc} is a \LaTeXe{} package
that enables the direct compilation
of document sections included by |\include|
to individual files.
\end{abstract}

\begingroup
\parskip0ex
\tableofcontents
\endgroup

%%%%%%%%%%%%%%%%%%%%%%%%%%%%%%%%%%%%%%%%%%%%%%%%%%%%%%%%%%%%%%%%%%%%%%%%%%%%%%%%
%%%%%%%%%%%%%%%%%%%%%%%%%%%%%%%%%%%%%%%%%%%%%%%%%%%%%%%%%%%%%%%%%%%%%%%%%%%%%%%%
\section{Introduction}

\LaTeX{} provides a mechanism to structure a large document (such as a book)
into a main file and several child files (containing the chapters)
using the |\include| command.
This mechanism is beneficial for documents
which span hundreds of pages in order to
make the source file(s) more manageable.
Moreover, compilation can be restricted to
selected child files by means of the |\includeonly| command.
The latter feature can be used to reduce the compilation time while editing
(this was significantly more useful in the earlier days of \LaTeX{})
or to generate a smaller document which is easier to navigate.
Another application of |\includeonly| is to generate
documents consisting of selected parts of the complete document.

However, there are a few drawbacks of the plain |\include| mechanism:
\begin{itemize}
\item
The child files cannot be compiled on their own,
they can only be compiled via the main file.
A naive editing environment
(such as a text editor with an option
to have the current file processed by \LaTeX)
may require one to switch to the main file before compiling;
attempting to compile the child file produces errors.
\item
The main file must be modified (each time)
to adjust the |\includeonly| command
to the present needs. This easily leaves the main file in a messy state.
\item
The generated document will always carry the filename
of the main document. This is inconvenient if
several child files are to be compiled and
to be kept for distribution.
\end{itemize}

The present package provides a simple interface
to make child files individually compilable by \LaTeX{}.
Compiling a child file then has the same effect as compiling
the main file with an |\includeonly| command
to select the appropriate child.
Moreover the generated document will carry the name of the child
rather than the main file.
This resolves all three above issues.

This feature is meant to make the editing of books,
thesis documents and lecture notes somewhat more convenient.
However, the package can also be used efficiently for
composing a series of documents (such as exercise sheets)
which are typically distributed individually.
It then assists the author in generating the individual documents
(potentially in different versions)
as well as a document containing the collected series.
Another application is in developing style files
or other kinds of included material
where compilation of the style file could redirect
to a sample or test file.

%%%%%%%%%%%%%%%%%%%%%%%%%%%%%%%%%%%%%%%%%%%%%%%%%%%%%%%%%%%%%%%%%%%%%%%%%%%%%%%%
%%%%%%%%%%%%%%%%%%%%%%%%%%%%%%%%%%%%%%%%%%%%%%%%%%%%%%%%%%%%%%%%%%%%%%%%%%%%%%%%
\section{Usage}

First of all, the package \textsf{childdoc} is \emph{not} a standard
\LaTeXe{} |.sty| style file! Therefore it needs to be invoked in
a non-standard way.

%%%%%%%%%%%%%%%%%%%%%%%%%%%%%%%%%%%%%%%%%%%%%%%%%%%%%%%%%%%%%%%%%%%%%%%%%%%%%%%%
\subsection{Included Files}
\label{sec:include}

%%%%%%%%%%%%%%%%%%%%%%%%%%%%%%%%%%%%%%%%
\DescribeMacro{\childdocmain}
To use the package, add the commands
\begin{center}
\begin{tabular}{l}
|\input{childdoc.def}|\\
|\childdocmain{}|\\
\end{tabular}
\end{center}
at the very top of the main \LaTeX{} file,
in particular \emph{before} the |\documentclass| statement!
The argument of |\childdocmain| should be left empty
(but it must be present).

%%%%%%%%%%%%%%%%%%%%%%%%%%%%%%%%%%%%%%%%
\DescribeMacro{\childdocof}
Furthermore, add the commands
\begin{center}
\begin{tabular}{l}
|\input{childdoc.def}|\\
|\childdocof{|\textit{main}|}|\\
\end{tabular}
\end{center}
at the top of every child file \textit{child}
which is included by |\include{|\textit{child}|}|
from within the main file
(or at least for those files to be compiled individually).
The argument \textit{main} must be the filename of the main file.

There are a couple of
considerations in setting up the main and child documents:

%%%%%%%%%%%%%%%%%%%%%%%%%%%%%%%%%%%%%%%%
\paragraph{Restrictions.}

Please note the following restrictions:
\begin{itemize}
\item
|\childdocmain| must be called with one argument \textit{main}
to ensure compatibility with earlier version of the package.
It must either be empty (|\childdocmain{}|)
or precisely match the filename of the main file in which it is specified.
See \secref{sec:detection} for further information.
\item
The filename \textit{main} must be specified without the |.tex| extension.
\item
The filename \textit{main} is case sensitive
(even in case-insensitive file systems)
due to internal string comparison.
\item
The argument \textit{main} should be fully expanded, it cannot be a macro.
\item
Subdirectories and special characters should be avoided in filenames.
\item
The command |\childdocmain{|\textit{main}|}| must be followed by a whitespace.
It should not be followed immediately by another command
or by a comment mark `|%|'.
This is because the \TeX{} parser reads the token immediately following
the argument of |\childdocmain| and puts it
at the beginning of every child section;
however, a white\-space is ignored.
\end{itemize}

%%%%%%%%%%%%%%%%%%%%%%%%%%%%%%%%%%%%%%%%
\paragraph{Content of Main File.}

It is advisable to place all content in the child files included by |\include|.
Any output contained in the main file will appear in all child documents
unless suppressed manually;
it cannot be suppressed automatically by the |\includeonly| directive
and thus should normally be avoided.
A method to include some content in the main file
by means of conditional processing is described in \secref{sec:conditional}.

%%%%%%%%%%%%%%%%%%%%%%%%%%%%%%%%%%%%%%%%
\paragraph{Page Numbering.}

When only a part of the document is compiled,
the appropriate numbering of pages
(as well as other status parameters)
is determined from the |.aux| files.
The latter contain information from previous passes.
However this information needs to propagate through
all intermediate child documents.
Therefore the page numbering in child documents may well
be inconsistent until the complete document is compiled at least once.

A useful (if unconventional) way to always ensure a consistent
page numbering is to restart the numbering in each child document
and denote the pages by `\textit{child}|.|\textit{page}'
where \textit{child} represents the chapter/section number of the child file.
This can be achieved by the command
|\numberwithin{page}{|\textit{child}|}|
of the \textsf{amsmath} package
where \textit{child} can be |chapter| or |section|
depending on the chosen structuring.
Alternatively, one can modify the macro |\thepage| appropriately
and reset the counter |page| at the start of each child file.

%%%%%%%%%%%%%%%%%%%%%%%%%%%%%%%%%%%%%%%%%%%%%%%%%%%%%%%%%%%%%%%%%%%%%%%%%%%%%%%%
\subsection{Conditional Processing}
\label{sec:conditional}

The package provides a mechanism to compile different versions
of a document. To customise the versions further some conditional processing
can come in handy to distinguish which version is being compiled.
The package provides two macros to describe the compilation context:

%%%%%%%%%%%%%%%%%%%%%%%%%%%%%%%%%%%%%%%%
\DescribeMacro{\ifchilddoc}
The conditional |\ifchilddoc| distinguishes between the compilation of
child documents and the main document:
%
\begin{center}
|\ifchilddoc |\textit{child-code}| |[|\||else |\textit{main-code}]| \||fi|
\end{center}

%%%%%%%%%%%%%%%%%%%%%%%%%%%%%%%%%%%%%%%%
\DescribeMacro{\childdocname}
\DescribeMacro{\childdocjob}
The macro |\childdocname| contains the filename (without extension)
of the main or child file being processed.
Note that |\childdocjob| will always contain the name of the main file.

%%%%%%%%%%%%%%%%%%%%%%%%%%%%%%%%%%%%%%%%
\paragraph{Title Page.}

Conditional processing can be used to include a title or banner page
in the main document when proper precautions are taken.
Importantly, the code in the main file should ensure that the page counter
(as well as other status parameters which are stored in the |.aux| files)
takes the same value after the conditional processing.
Otherwise the page numbers may take divergent values
depending on which part is compiled.

For example, a title page could be declared by:
%
\begin{center}
\begin{tabular}{l}
|\ifchilddoc\||else|\\
|\addtocounter{page}{-1}|\\
\textit{code for title page}\\
|\newpage|\\
|\||fi|
\end{tabular}
\end{center}
%
A banner page for the child documents can be generated by:
%
\begin{center}
\begin{tabular}{l}
|\ifchilddoc|\\
|\addtocounter{page}{-1}|\\
\textit{code for banner page}\\
|\newpage|\\
|\||fi|
\end{tabular}
\end{center}
%
Here one could write a message such as:
\begin{center}
|This is the part \childdocname{} of \childdocjob{}.|
\end{center}

%%%%%%%%%%%%%%%%%%%%%%%%%%%%%%%%%%%%%%%%%%%%%%%%%%%%%%%%%%%%%%%%%%%%%%%%%%%%%%%%
\subsection{Flags}
\label{sec:flags}

The package makes it easy to generate different versions
of the main or child documents.
To this end compilation flags can be defined
and assigned different default values.
They will be particularly useful in conjunction
with the forwarding mechanism described in \secref{sec:forward}.

For example, it may be useful to have a flag |\version|
which can be set to |draft| or |final|.
The document source will contain some conditional code
depending on the value of |\version|.
Suppose further, the flag should default to |final| for the main file
and to |draft| for child files
which is a natural assignment for editing the document.
This is achieved by placing the following code
in the preamble of the main document
(below the |\childdocmain| directive):
%
\begin{center}
\begin{tabular}{l}
|\ifchilddoc|\\
|\providecommand{\version}{draft}|\\
|\||else|\\
|\providecommand{\version}{final}|\\
|\||fi|
\end{tabular}
\end{center}
%
The definition by |\providecommand| makes sure
that previous definitions are not overwritten.
Further statements |\providecommand{\version}{...}|
can thus be added before the above code to override it.

For the main file, one might add a line
(between |\childdocmain| and the above block)
%
\begin{center}
|%\ifchilddoc\||else\providecommand{\version}{draft}\||fi|
\end{center}
%
which can be uncommented to produce a draft version.
Likewise one can add a line to the very top of a child file
(above the |\childdocof{|\textit{main}|}| directive)
%
\begin{center}
|%\providecommand{\version}{final}|
\end{center}
%
which can be uncommented to produce the final version of this child document.

%%%%%%%%%%%%%%%%%%%%%%%%%%%%%%%%%%%%%%%%%%%%%%%%%%%%%%%%%%%%%%%%%%%%%%%%%%%%%%%%
\subsection{Forwarding}
\label{sec:forward}

Different versions of the main or child documents
using compilation flags as described in \secref{sec:flags}
can be (permanently) stored in different files
for convenient compilation, viewing and distribution.
To this end, the package defines a command
to pass on compilation to a different file:

%%%%%%%%%%%%%%%%%%%%%%%%%%%%%%%%%%%%%%%%
\DescribeMacro{\childdocforward}
The command |\childdocforward| redirects processing to
another source file:
%
\begin{center}
\begin{tabular}{l}
|\input{childdoc.def}|\\
|\childdocforward[|\textit{main}|]{|\textit{dest}|}|\\
\end{tabular}
\end{center}
%
The argument \textit{dest} is the destination file
(without extension).
It should be the main file or one of the child files.
Note that further \textsf{childdoc} directives
such as |\childdocof| and |\childdocforward|
in the indicated file will be processed in this form.
The optional argument \textit{main}
passes on directly to the main file \textit{main}
while pretending to compile the child \textit{dest}.
This form behaves as if \textit{dest}
issues |\childdocof{|\textit{main}|}| right away,
and no further \textsf{childdoc} directives will be processed.

%%%%%%%%%%%%%%%%%%%%%%%%%%%%%%%%%%%%%%%%
\DescribeMacro{\...prefix}
In the alternative form |\childdocforwardprefix|,
%
\begin{center}
\begin{tabular}{l}
|\input{childdoc.def}|\\
|\childdocforwardprefix[|\textit{main}|]{|\textit{prefix}|}{|\textit{dest}|}|
\end{tabular}
\end{center}
%
the destination file is determined by a pattern
depending on the current file:
To make this work, the current file must be called
`{\textit{prefix}\hspace{0.2em}\textit{suffix}}'
with \textit{prefix} matching precisely the argument.
Processing is then passed on to the file
`{\textit{dest}\hspace{0.2em}\textit{suffix}}'.
Surely, the same effect is achieved by
directly specifying the
argument `{\textit{dest}\hspace{0.2em}\textit{suffix}}'
in the first form.
However, that requires to set up a different file
for each child. With the alternative form of the command
all these files can have exactly the same content
which simplifies setting them up and maintaining them.

For example, the following file |draft.tex|
with a compilation flag |\version| as described in \secref{sec:flags}
compiles the main document as a draft:
%
\begin{center}
\begin{tabular}{l}
|\def\version{draft}|\\
|\input{childdoc.def}|\\
|\childdocforward{|\textit{main}|}|
\end{tabular}
\end{center}
%
Likewise, the following files |final|\textit{nn}|.tex|
compile the final version of the child document
|child|\textit{nn}|.tex|:
%
\begin{center}
\begin{tabular}{l}
|\def\version{final}|\\
|\input{childdoc.def}|\\
|\childdocforwardprefix{final}{child}|
\end{tabular}
\end{center}
%

Note that when several versions of a main file and/or of each child file
are to be generated, it may be convenient to set up a |Makefile| or
shell script to automatise the process.

%%%%%%%%%%%%%%%%%%%%%%%%%%%%%%%%%%%%%%%%%%%%%%%%%%%%%%%%%%%%%%%%%%%%%%%%%%%%%%%%
\subsection{Command Line Processing}
\label{sec:commandline}

The effect of redirection files can also be achieved by invoking
the \LaTeX{} compiler with a more elaborate command line.
Most conveniently this should be done as part
of a shell script or a |Makefile|.

When using \textsf{childdoc} in the main file, the following
command lines effectively perform a redirection
(note that depending on the shell being used,
backslashes may have to be doubled: `|\|' $\to$ `|\\|'):
%
\begin{center}
|... -jobname "|\textit{target}|" |\\|"|[\textit{flags}]%
|\input{childdoc.def}\childdocforward[|\textit{main}|]{|\textit{dest}|}"|
\end{center}
%
Here \textit{target} is the name of the output file,
\textit{main} is the name of the main file
and \textit{dest} is the name of the main or child file to be processed
(all filenames without extensions).
The optional argument \textit{main} can be omitted
if \textit{main} matches \textit{dest}.
Optionally, compilation \textit{flags} can be defined via |\def| commands.
This command line makes the \TeX{} engine believe
it is compiling the file \textit{target}
whose content is specified as the latter parameter.
The provided code then forwards the processing to
\textit{main} or \textit{dest} as described in \secref{sec:forward}.

%%%%%%%%%%%%%%%%%%%%%%%%%%%%%%%%%%%%%%%%%%%%%%%%%%%%%%%%%%%%%%%%%%%%%%%%%%%%%%%%
\subsection{Include by Input}
\label{sec:input}

Including child documents by |\include| has some restrictions by design.
Most notably, the content of a child document always occupies
its own set of pages; pages cannot be shared between child documents.
Usually, this behaviour makes perfect sense
because each child document contain an essential part of the document.
However, in some situations it may be desirable to compose
a document from a collection of parts
without having mandatory page breaks between then.
For this case, the package
provides a mechanism to include parts
by |\input| which can also be processed individually.
However, by construction this mechanism
requires manual handling of the content to be output.

%%%%%%%%%%%%%%%%%%%%%%%%%%%%%%%%%%%%%%%%
\DescribeMacro{\ifchilddocmanual}
The main file should be prepared as usual, see \secref{sec:include}.
However, the document body must make a distinction
between processing of an individual part and of the main document, e.g.:
%
\begin{center}
\begin{tabular}{l}
|\ifchilddocmanual|\\
|\input{\childdocname}|\\
|\||else|\\
\textit{document body with }|\input{|\textit{part}|}|\\
|\||fi|
\end{tabular}
\end{center}
%
The conditional |\ifchilddocmanual| is true whenever
a part to be included by |\input| is being compiled,
and the name of the part is stored in |\childdocname|.

%%%%%%%%%%%%%%%%%%%%%%%%%%%%%%%%%%%%%%%%
\DescribeMacro{\childdocby}
Each part to be included by |\input| should start with:
%
\begin{center}
\begin{tabular}{l}
|\input{childdoc.def}|\\
|\childdocby{|\textit{main}|}|\\
\end{tabular}
\end{center}
%
The directive |\childdocby| is similar to |\childdocof|
described in \secref{sec:include},
but the subsequent selection of content must be done manually.
To that end, both |\ifchilddoc| and |\ifchilddocmanual|
will be true upon processing of a part,
and the name of the part is stored in |\childdocname|.
Note that |\jobname| will be set to the filename of the current part
so that each part receives an individual |.aux| file
that does not interfere with the |.aux| file(s) of the main document.
This behaviour can be altered by the alternative form
|\childdocby[*]{|\textit{main}|}| (with a non-empty optional argument)
which uses the |.aux| file of the main document
by setting |\jobname| to \textit{main}.

%%%%%%%%%%%%%%%%%%%%%%%%%%%%%%%%%%%%%%%%%%%%%%%%%%%%%%%%%%%%%%%%%%%%%%%%%%%%%%%%
\subsection{Driver Development}
\label{sec:driver}

The \textsf{childdoc} mechanism can also be use for the development
of definition files such as \LaTeX{} styles or classes.
This case differs from the above setup with multiple parts
included by |\include| in that no |\includeonly| should be invoked.
This can be achieved by starting the include file
(before |\ProvidesPackage|) with:
%
\begin{center}
\begin{tabular}{l}
|\input{childdoc.def}|\\
|\childdocforward{|\textit{main}|}|\\
\end{tabular}
\end{center}
%
or alternatively with:
%
\begin{center}
\begin{tabular}{l}
|\input{childdoc.def}|\\
|\childdocby{|\textit{main}|}|\\
\end{tabular}
\end{center}
%
Both forms have slightly different effects as described above.
The main file is prepared as usual, see \secref{sec:include}.

%%%%%%%%%%%%%%%%%%%%%%%%%%%%%%%%%%%%%%%%%%%%%%%%%%%%%%%%%%%%%%%%%%%%%%%%%%%%%%%%
\subsection{Legacy Detection}
\label{sec:detection}

The directive |\childdocmain| in the main file can detect
whether the complete document or merely a child is to be compiled
even without using the directive |\childdocof|.
This method is deprecated because it is less robust
and there is no compelling reason to use it;
it is merely provided for backward compatibility
and it may be removed in future versions.

If the detection mechanism is to be used,
it is mandatory to correctly specify
the filename of the main file as the argument of |\childdocmain|:
%
\begin{center}
\begin{tabular}{l}
|\input{childdoc.def}|\\
|\childdocmain{|\textit{main}|}|\\
\end{tabular}
\end{center}
%
If |\jobname| does not match the argument \textit{main} of |\childdocmain|,
it is assumed that |\jobname| points to the child file to be compiled.
When using |\childdocmain| with the main file specified as argument,
it suffices to start a child file
with just |\input{|\textit{main}|}|
without loading of the package and using |\childdocof|.
If instead all processing is done
with the appropriate \textsf{childdoc} directives,
the argument of \textit{main} of |\childdocmain| can be empty.

An alternative version of the command line processing described
in \secref{sec:commandline} using the detection mechanism reads:
%
\begin{center}
|... -jobname "|\textit{target}|" "|[\textit{flags}]%
[|\def\jobname{|\textit{dest}|}|]|\input{|\textit{main}|}"|
\end{center}

%%%%%%%%%%%%%%%%%%%%%%%%%%%%%%%%%%%%%%%%%%%%%%%%%%%%%%%%%%%%%%%%%%%%%%%%%%%%%%%%
\subsection{Manual Code}
\label{sec:manual}

In case one cannot be certain whether the definitions file |childdoc.def|
is installed on the target \TeX{} distribution
and one prefers not to ship it,
it is conceivable to paste a few relevant commands into the sources.

To that end, drop all statements |\input{childdoc.def}|
and perform the replacements as outlined below.
Instead of |\childdocmain{|\textit{main}|}| add the following code
to the top of the main file:
%
\begin{center}
\begin{tabular}{l}
|\||ifdefined\childdocname\endinput\||fi\newif\ifchilddoc|\\
|\edef\childdocname{\scantokens\expandafter{\jobname\noexpand}}|\\
|\def\childdocmain{|\textit{main}|}\||ifx\childdocmain\childdocname\||else|\\
|\childdoctrue\includeonly{\childdocname}\let\jobname\childdocmain\||fi|\\
\end{tabular}
\end{center}
%
Instead of |\childdocof{|\textit{main}|}| just include the main file
at the top of each child file:
%
\begin{center}
|\input{|\textit{main}|}|
\end{center}
%
A simple redirection |\childdocforward{|\textit{dest}|}| is achieved by:
%
\begin{center}
|\def\jobname{|\textit{dest}|}\input{\jobname}|
\end{center}
%
The redirection with prefix
|\childdocforwardprefix[|\textit{prefix}|]{|\textit{dest}|}|
is accomplished by:
%
\begin{center}
\begin{tabular}{l}
|{\edef\jobname{\scantokens\expandafter{\jobname\noexpand}}|\\
|\def\redirectjob |\textit{prefix}|#1~~~{\gdef\jobname{|\textit{dest}|#1}}|\\
|\expandafter\redirectjob\jobname~~~}\input{\jobname}|
\end{tabular}
\end{center}

In an alternative approach,
child documents can be compiled by a specific command line
without additional code or specific definitions:
%
\begin{center}
|... -jobname "|\textit{target}|" "|[\textit{flags}]%
|\includeonly{|\textit{dest}|}\input{|\textit{main}|}"|
\end{center}
%

%%%%%%%%%%%%%%%%%%%%%%%%%%%%%%%%%%%%%%%%%%%%%%%%%%%%%%%%%%%%%%%%%%%%%%%%%%%%%%%%
%%%%%%%%%%%%%%%%%%%%%%%%%%%%%%%%%%%%%%%%%%%%%%%%%%%%%%%%%%%%%%%%%%%%%%%%%%%%%%%%
\section{Information}

%%%%%%%%%%%%%%%%%%%%%%%%%%%%%%%%%%%%%%%%%%%%%%%%%%%%%%%%%%%%%%%%%%%%%%%%%%%%%%%%
\subsection{Copyright}

Copyright \copyright{} 2017--2018 Niklas Beisert

This work may be distributed and/or modified under the
conditions of the \LaTeX{} Project Public License, either version 1.3
of this license or (at your option) any later version.
The latest version of this license is in
  \url{http://www.latex-project.org/lppl.txt}
and version 1.3 or later is part of all distributions of \LaTeX{}
version 2005/12/01 or later.

This work has the LPPL maintenance status `maintained'.

The Current Maintainer of this work is Niklas Beisert.

This work consists of the files |README.txt|, |childdoc.ins| and |childdoc.dtx|
as well as the derived files |childdoc.def|, |cdocsamp.tex|
with |cdocsch1.tex|, |cdocsch2.tex|, |cdocspt3.tex|, |cdocspt4.tex|,
|cdocsdrf.tex|, |cdocsfn1.tex|, |cdocsfn2.tex|
as well as |childdoc.pdf|.

%%%%%%%%%%%%%%%%%%%%%%%%%%%%%%%%%%%%%%%%%%%%%%%%%%%%%%%%%%%%%%%%%%%%%%%%%%%%%%%%
\subsection{Files and Installation}

The package consists of the files:
%
\begin{center}
\begin{tabular}{ll}
    |README.txt|   & readme file \\
    |childdoc.ins| & installation file \\
    |childdoc.dtx| & source file \\
    |childdoc.def| & definition file \\
    |cdocsamp.tex| & sample main file \\
    |cdocsch1.tex| & sample include file \\
    |cdocsch2.tex| & sample include file \\
    |cdocspt3.tex| & sample part file \\
    |cdocspt4.tex| & sample part file \\
    |cdocsdrf.tex| & sample redirection file \\
    |cdocsfn1.tex| & sample redirection file \\
    |cdocsfn2.tex| & sample redirection file \\
    |childdoc.pdf| & manual
\end{tabular}
\end{center}
%
The distribution consists of the files
|README.txt|, |childdoc.ins| and |childdoc.dtx|.
%
\begin{itemize}
\item
Run (pdf)\LaTeX{} on |childdoc.dtx|
to compile the manual |childdoc.pdf| (this file).
\item
Run \LaTeX{} on |childdoc.ins| to create the definitions file |childdoc.def|
and the sample |cdocsamp.tex| with include files
|cdocsch1.tex|, |cdocsch2.tex|, |cdocspt3.tex|, |cdocspt4.tex|,
|cdocsdrf.tex|, |cdocsfn1.tex|, |cdocsfn2.tex|.
Then copy the file |childdoc.def| to an appropriate directory of your \LaTeX{}
distribution, e.g.\ \textit{texmf-root}|/tex/latex/childdoc|.
\end{itemize}

%%%%%%%%%%%%%%%%%%%%%%%%%%%%%%%%%%%%%%%%%%%%%%%%%%%%%%%%%%%%%%%%%%%%%%%%%%%%%%%%
\subsection{Related CTAN Packages}

There are several other packages which offer a similar functionality:
%
\begin{itemize}
\item
The packages
\href{http://ctan.org/pkg/docmute}{\textsf{docmute}},
\href{http://ctan.org/pkg/includex}{\textsf{includex}} and
\href{http://ctan.org/pkg/standalone}{\textsf{standalone}}
provide commands to include only the document body of
a child file thus allowing both files to be compiled individually.
\item
The packages \href{http://ctan.org/pkg/subdocs}{\textsf{subdocs}}
and \href{http://ctan.org/pkg/subfiles}{\textsf{subfiles}}
provide structures in which the main and child documents can be
encapsulated and allowing them to be compiled individually.
The inclusion mechanism is different from the conventional |\include|.
\item
The package \href{http://ctan.org/pkg/combine}{\textsf{combine}}
is an elaborate solution to combine several documents into one.
\end{itemize}
%
See also the CTAN topic \href{http://ctan.org/topic/subdocs}{\textsf{subdocs}}
for further related packages.
The present package differs from the above solutions in that
a document structure constructed with the conventional |\include| mechanism
just needs two extra commands at the top of every file
such that all constituent files can be compiled individually.

%%%%%%%%%%%%%%%%%%%%%%%%%%%%%%%%%%%%%%%%%%%%%%%%%%%%%%%%%%%%%%%%%%%%%%%%%%%%%%%%
%\subsection{Feature Suggestions}
%
%The following is a list of features which may be useful for future
%versions of this package:
%%
%\begin{itemize}
%\item
%\ldots
%\end{itemize}

%%%%%%%%%%%%%%%%%%%%%%%%%%%%%%%%%%%%%%%%%%%%%%%%%%%%%%%%%%%%%%%%%%%%%%%%%%%%%%%%
\subsection{Revision History}

%%%%%%%%%%%%%%%%%%%%%%%%%%%%%%%%%%%%%%%%
\paragraph{v2.0:} 2018/12/30

\begin{itemize}
\item
immediate forward processing
\item
added |\childdocby| mechanism
\item
manual restructured
\end{itemize}

%%%%%%%%%%%%%%%%%%%%%%%%%%%%%%%%%%%%%%%%
\paragraph{v1.6:} 2018/01/17

\begin{itemize}
\item
application for development of include files
\item
corrections to manual
\end{itemize}

%%%%%%%%%%%%%%%%%%%%%%%%%%%%%%%%%%%%%%%%
\paragraph{v1.5:} 2017/05/21

\begin{itemize}
\item
more complete structuring introduced
\item
|\childdocof| introduced
\item
|\childdoc| renamed to |\childdocmain|
\item
|\childredirect| renamed to |\childdocforward| and |\childdocforwardprefix|
and functionality expanded
\end{itemize}

%%%%%%%%%%%%%%%%%%%%%%%%%%%%%%%%%%%%%%%%
\paragraph{v1.0:} 2017/04/27

\begin{itemize}
\item
manual and install package
\item
first version published on CTAN
\end{itemize}

%%%%%%%%%%%%%%%%%%%%%%%%%%%%%%%%%%%%%%%%
\paragraph{v0.6:} 2017/04/26

\begin{itemize}
\item
redirection mechanism added
\end{itemize}

%%%%%%%%%%%%%%%%%%%%%%%%%%%%%%%%%%%%%%%%
\paragraph{v0.5:} 2017/04/26

\begin{itemize}
\item
functionality in definition file
\end{itemize}


%%%%%%%%%%%%%%%%%%%%%%%%%%%%%%%%%%%%%%%%%%%%%%%%%%%%%%%%%%%%%%%%%%%%%%%%%%%%%%%%
%%%%%%%%%%%%%%%%%%%%%%%%%%%%%%%%%%%%%%%%%%%%%%%%%%%%%%%%%%%%%%%%%%%%%%%%%%%%%%%%
%%%%%%%%%%%%%%%%%%%%%%%%%%%%%%%%%%%%%%%%%%%%%%%%%%%%%%%%%%%%%%%%%%%%%%%%%%%%%%%%
\appendix

\settowidth\MacroIndent{\rmfamily\scriptsize 000\ }

 \DocInput{childdoc.dtx}

\end{document}
%</driver>
% \fi
%
% %%%%%%%%%%%%%%%%%%%%%%%%%%%%%%%%%%%%%%%%%%%%%%%%%%%%%%%%%%%%%%%%%%%%%%%%%%%%%%
% %%%%%%%%%%%%%%%%%%%%%%%%%%%%%%%%%%%%%%%%%%%%%%%%%%%%%%%%%%%%%%%%%%%%%%%%%%%%%%
% \section{Sample}
%\iffalse
%<*samplemain>
%\fi
%
% The following presents a sample document
% with two chapters, two parts, a title page,
% a compile flag as well as three forwarding files to set the flag.
% It consists of eight |.tex| files:
% \begin{center}
% \begin{tabular}{ll}
% |cdocsamp.tex|&main file\\
% |cdocsch1.tex|&include file for chapter 1\\
% |cdocsch2.tex|&include file for chapter 2\\
% |cdocspt3.tex|&include file for part 3\\
% |cdocspt4.tex|&include file for part 4\\
% |cdocsdrf.tex|&forwarding file for main file in draft mode\\
% |cdocsfi1.tex|&forwarding file for final version of chapter 1\\
% |cdocsfi2.tex|&forwarding file for final version of chapter 2\\
% \end{tabular}
% \end{center}
% Each of the eight files can be compiled directly by the \LaTeX{} compiler.
%
% %%%%%%%%%%%%%%%%%%%%%%%%%%%%%%%%%%%%%%
% \paragraph{Main File.}
%
% The main file is called |cdocsamp.tex|.
%
% Load the \textsf{childdoc} definitions and
% declare the filename for the main document:
%    \begin{macrocode}
\input{childdoc.def}
\childdocmain{}
%    \end{macrocode}

% Optional override for |\version| flag:
%    \begin{macrocode}
%%\ifchilddoc\else\providecommand{\version}{draft}\fi
%    \end{macrocode}

% Define the default values for the |\version| flag
% (|final| for the main file and |draft| for childs):
%    \begin{macrocode}
\ifchilddoc
\providecommand{\version}{draft}
\else
\providecommand{\version}{final}
\fi
%    \end{macrocode}

% Load the standard document class:
%    \begin{macrocode}
\documentclass[12pt]{article}
%    \end{macrocode}

% Start the document body:
%    \begin{macrocode}
\begin{document}
%    \end{macrocode}

% Declare a title page.
% Print title, part of document being processed and version flag:
%    \begin{macrocode}
\addtocounter{page}{-1}
\begin{center}
{\LARGE\bfseries{}childdoc example\par}
\vspace{1cm}
\ifchilddoc
\ifchilddocmanual part\else chapter\fi:
`\childdocname' of `\childdocjob'\par
\else
main document: `\childdocjob'\par
\fi
version: \version\par
\end{center}
\newpage
%    \end{macrocode}

% Manually include selected file,
% otherwise process as usual:
%    \begin{macrocode}
\ifchilddocmanual
\section*{part `\childdocname'}
\input{\childdocname}
\else
%    \end{macrocode}

% Include the two chapters:
%    \begin{macrocode}
\include{cdocsch1}
\include{cdocsch2}
%    \end{macrocode}

% Include the two parts unless only chapters should be displayed:
%    \begin{macrocode}
\ifchilddoc\else
\section{part three}
\input{cdocspt3}
\section{part four}
\input{cdocspt4}
\fi
%    \end{macrocode}

% Process as usual until here:
%    \begin{macrocode}
\fi
%    \end{macrocode}

% End of document body:
%    \begin{macrocode}
\end{document}
%    \end{macrocode}
%\iffalse
%</samplemain>
%\fi
%
% %%%%%%%%%%%%%%%%%%%%%%%%%%%%%%%%%%%%%%
% \paragraph{Chapter Include Files.}
%
% The include files are called |cdocsch1.tex| and |cdocsch2.tex|.
%
%\iffalse
%<*samplechap1|samplechap2>
%\fi

% Optional override for |\version| flag:
%    \begin{macrocode}
%%\providecommand{\version}{final}
%    \end{macrocode}

% Include the main document:
%    \begin{macrocode}
\input{childdoc.def}
\childdocof{cdocsamp}
%    \end{macrocode}

%\iffalse
%</samplechap1|samplechap2>
%\fi
%
%\iffalse
%<*samplechap1>
%\fi
% Some text for chapter 1:
%    \begin{macrocode}
\section{one}
some text in chapter one
%    \end{macrocode}

%\iffalse
%</samplechap1>
%\fi
% Some text for chapter 2:
%\iffalse
%<*samplechap2>
%\fi
%    \begin{macrocode}
\section{two}
more text in chapter two
%    \end{macrocode}

%\iffalse
%</samplechap2>
%\fi
%
% %%%%%%%%%%%%%%%%%%%%%%%%%%%%%%%%%%%%%%
% \paragraph{Part Include Files.}
%
% The include files are called |cdocspt3.tex| and |cdocspt4.tex|.
%
%\iffalse
%<*samplepart3|samplepart4>
%\fi

% Optional override for |\version| flag:
%    \begin{macrocode}
%%\providecommand{\version}{final}
%    \end{macrocode}

% Include the main document:
%    \begin{macrocode}
\input{childdoc.def}
\childdocby{cdocsamp}
%    \end{macrocode}

%\iffalse
%</samplepart3|samplepart4>
%\fi
%
%\iffalse
%<*samplepart3>
%\fi
% Some text for part 3:
%    \begin{macrocode}
some text in part three
%    \end{macrocode}

%\iffalse
%</samplepart3>
%\fi
% Some text for part 4:
%\iffalse
%<*samplepart4>
%\fi
%    \begin{macrocode}
more text in part four
%    \end{macrocode}

%\iffalse
%</samplepart4>
%\fi
%
% %%%%%%%%%%%%%%%%%%%%%%%%%%%%%%%%%%%%%%
% \paragraph{Forwarding for a Complete Draft.}
%
% The following forwarding file |cdocsdrf.tex|
% compiles the main document in draft mode:
%\iffalse
%<*sampledraft>
%\fi
%    \begin{macrocode}
\def\version{draft}
\input{childdoc.def}
\childdocforward{cdocsamp}
%    \end{macrocode}

%\iffalse
%</sampledraft>
%\fi
%
% %%%%%%%%%%%%%%%%%%%%%%%%%%%%%%%%%%%%%%
% \paragraph{Forwarding for Final Version of the Chapters.}
%
% The following forwarding files |cdocsfn1.tex| and |cdocsfn2.tex|
% (with identical content)
% compile the final versions of the child documents
% |cdocsch1.tex| and |cdocsch2.tex|, respectively:
%\iffalse
%<*samplefinal>
%\fi
%    \begin{macrocode}
\def\version{final}
\input{childdoc.def}
\childdocforwardprefix[cdocsamp]{cdocsfn}{cdocsch}
%    \end{macrocode}

%\iffalse
%</samplefinal>
%\fi
%
% %%%%%%%%%%%%%%%%%%%%%%%%%%%%%%%%%%%%%%
% \paragraph{Command Line Processing.}
%
% The following three command lines generate the output files
% |cdocscld|, |cdocscl1| and |cdocscl2|
% which should be identical to
% |cdocsdrf|, |cdocsch1| and |cdocsfn2|, respectively:
% \begin{center}
% \begin{tabular}{l}
% |latex -jobname cdocscld \|\\
% |  "\def\version{draft}\input{childdoc.def}\childdocforward{cdocsamp}"|\\
% |latex -jobname cdocscl1 \|\\
% |  "\input{childdoc.def}\childdocforward[cdocsamp]{cdocsch1}"|\\
% |latex -jobname cdocscl2 \|\\
% |  "\def\version{final}\input{childdoc.def}\childdocforward{cdocsch2}"|
% \end{tabular}
% \end{center}
% Note that the trailing backslash on each first line
% merely continues the input to the second line
% (for convenient cut ant paste).
% Furthermore, the command |latex| can be replaced by any
% of its alternative versions such as |pdflatex|.
%
% %%%%%%%%%%%%%%%%%%%%%%%%%%%%%%%%%%%%%%%%%%%%%%%%%%%%%%%%%%%%%%%%%%%%%%%%%%%%%%
% %%%%%%%%%%%%%%%%%%%%%%%%%%%%%%%%%%%%%%%%%%%%%%%%%%%%%%%%%%%%%%%%%%%%%%%%%%%%%%
% \section{Implementation}
%\iffalse
%<*package>
%\fi
%
% This section describes the definitions file |childdoc.def|.

% The definitions cannot be loaded using |\usepackage| or |\RequirePackage|
% which has a mechanism to prevent loading a style file more than once.
% When loading the definitions by means of |\input|
% multiple instances have to be prevented manually:
%\iffalse
%This code needs to be before the `\ProvidesFile' directive
%which is defined at the beginning of this file.
%Therefore it is also placed there and commented out here.
%</package>
%<*discard>
%\fi
%    \begin{macrocode}
\ifdefined\childdocmain\endinput\fi
%    \end{macrocode}
%\iffalse
%</discard>
%<*package>
%\fi
%
% \macro{\ifchilddoc}
% \macro{\ifchilddocmanual}
% The conditional |\ifchilddoc| tells whether a
% child (true) or main (false) document is being compiled.
% The conditional |\ifchilddocmanual| tells whether
% the |\includeonly| mechanism is used (false) or
% the selection of child files must be performed manually (true).
% The definitions initialise to false:
%    \begin{macrocode}
\newif\ifchilddoc
\newif\ifchilddocmanual
%    \end{macrocode}

% \macro{\childdocname}
% \macro{\childdocjob}
% The macro |\childdocname| stores the name of the main document
% to be compiled. The macro |\childdocjob| stores the name of
% the document on which the \LaTeX{} compiler was originally invoked.
% The content of |\jobname| cannot be compared
% to filenames specified in the source due to different catcodes.
% The following code rescans |\jobname|, stores the result
% in |\childdocname| and saves a copy in |\childdocjob|:
%    \begin{macrocode}
\edef\childdocname{\scantokens\expandafter{\jobname\noexpand}}
\let\childdocjob\childdocname
%    \end{macrocode}

% \macro{\childdocdisable}
% The macro |\childdocdisable| prevents the main file
% from being processed more than once.
% At this stage, the main document command |\childdocmain|
% is assumed to be called once again where it should do nothing.
% Any subsequent call to it should prevent
% a secondary processing of the main document
% It overwrites the forwarding commands
% |\childdocof| and |\childdocforward|
% with empty macros to prevent further inclusions of the main document:
%    \begin{macrocode}
\newcommand{\childdocdisable}
{
  \renewcommand{\childdocmain}[1]{\renewcommand{\childdocmain}[1]{\endinput}}
  \renewcommand{\childdocof}[1]{}
  \renewcommand{\childdocby}[2][]{}
  \renewcommand{\childdocforward}[2][]{}
  \renewcommand{\childdocdisable}{}
}
%    \end{macrocode}

% \macro{\childdocmain}
% The macro |\childdocmain| is to be called at the top of the main file
% with nothing or the main filename (without extension) as argument.
% First, it breaks loops.
% If the argument is not empty and does not match |\childdocname|
% (which is set by the first inclusion of |childdoc.def|),
% |\ifchilddoc| is set to true, |\includeonly| is applied to the child file
% and |\jobname| is set to the main file
% (for proper handling of |.aux| files):
%    \begin{macrocode}
\newcommand{\childdocmain}[1]
{
  \childdocdisable\childdocmain{}
  \if?#1?\else
    \begingroup
      \def\childdoctmp{#1}
      \ifx\childdoctmp\childdocname
        \def\childdoctmp{}
      \else
        \def\childdoctmp
        {
          \childdoctrue
          \includeonly{\childdocname}
          \def\childdocjob{#1}
          \def\jobname{#1}
        }
      \fi
      \expandafter
    \endgroup
    \childdoctmp
  \fi
}
%    \end{macrocode}

% \macro{\childdocof}
% The command |\childdocof| redirects
% compilation to the main file |#1|.
%    \begin{macrocode}
\newcommand{\childdocof}[1]
{
  \childdocdisable
  \childdoctrue
  \includeonly{\childdocname}
  \def\jobname{#1}
  \def\childdocjob{#1}
  \input{#1}
}
%    \end{macrocode}

% \macro{\childdocby}
% The command |\childdocby| ....
%    \begin{macrocode}
\newcommand{\childdocby}[2][]
{
  \childdocdisable
  \childdoctrue
  \childdocmanualtrue
  \if?#1?\else
    \def\jobname{#2}
  \fi
  \def\childdocjob{#2}
  \input{#2}
  \endinput
}
%    \end{macrocode}

% \macro{\childdocforward}
% The command |\childdocforward| redirects
% compilation to the main file or
% (if the optional argument is given) a child file.
% Parameters are set as if the main file
% or a child file starting with |\childdocof| was compiled.
% Then compilation is handed over to the main file:
%    \begin{macrocode}
\newcommand{\childdocforward}[2][]
{
  \begingroup
    \if?#1?
      \def\childdoctmp
      {
        \def\childdocname{#2}
        \def\childdocjob{#2}
        \def\jobname{#2}
        \input{#2}
        \endinput
      }
    \else
      \def\childdoctmp
      {
        \childdocdisable
        \def\childdocname{#2}
        \childdoctrue
        \includeonly{#2}
        \def\childdocjob{#1}
        \def\jobname{#1}
        \input{#1}
        \endinput
      }
    \fi
    \expandafter
  \endgroup
  \childdoctmp
}
%    \end{macrocode}

% \macro{\childdocforwardprefix}
% The command |\childdocforwardprefix| redirects
% compilation to the main or a child file by means of a pattern.
% The prefix |#1| in the current filename is replaced by |#2|
% and the suffix of the current filename is kept
% (it is assumed that the filename does not contain the substring `|~~~|'
% which is used as a delimiter).
% Compilation is handed over to the new file by |\childdocforward|:
%    \begin{macrocode}
\newcommand{\childdocforwardprefix}[3][]
{
  \begingroup
    \def\childdocextract #2##1~~~{\def\childdoctmp{\childdocforward[#1]{#3##1}}}
    \expandafter\childdocextract\childdocname~~~
    \expandafter
  \endgroup
  \childdoctmp
}
%    \end{macrocode}

% \macro{\childdoc}
% The deprecated macro |\childdoc| is a legacy version of |\childdocmain|:
%    \begin{macrocode}
\newcommand{\childdoc}{\childdocmain}
%    \end{macrocode}

% \macro{\childdocredirect}
% The deprecated macro |\childdocredirect| is a legacy version
% of |\childdocforward| and |\childdocforwardprefix|:
%    \begin{macrocode}
\newcommand{\childdocredirect}[2][]
{
  \begingroup
    \if?#1?
      \def\childdoctmp{\childdocforward{#2}}
    \else
      \def\childdoctmp{\childdocforwardprefix{#1}{#2}}
    \fi
    \expandafter
  \endgroup
  \childdoctmp
}
%    \end{macrocode}

%\iffalse
%</package>
%\fi
%
\endinput

\childdocof{cdocsamp}
%    \end{macrocode}

%\iffalse
%</samplechap1|samplechap2>
%\fi
%
%\iffalse
%<*samplechap1>
%\fi
% Some text for chapter 1:
%    \begin{macrocode}
\section{one}
some text in chapter one
%    \end{macrocode}

%\iffalse
%</samplechap1>
%\fi
% Some text for chapter 2:
%\iffalse
%<*samplechap2>
%\fi
%    \begin{macrocode}
\section{two}
more text in chapter two
%    \end{macrocode}

%\iffalse
%</samplechap2>
%\fi
%
% %%%%%%%%%%%%%%%%%%%%%%%%%%%%%%%%%%%%%%
% \paragraph{Part Include Files.}
%
% The include files are called |cdocspt3.tex| and |cdocspt4.tex|.
%
%\iffalse
%<*samplepart3|samplepart4>
%\fi

% Optional override for |\version| flag:
%    \begin{macrocode}
%%\providecommand{\version}{final}
%    \end{macrocode}

% Include the main document:
%    \begin{macrocode}
% \iffalse
%
% childdoc.dtx Copyright (C) 2017-2018 Niklas Beisert
%
% This work may be distributed and/or modified under the
% conditions of the LaTeX Project Public License, either version 1.3
% of this license or (at your option) any later version.
% The latest version of this license is in
%   http://www.latex-project.org/lppl.txt
% and version 1.3 or later is part of all distributions of LaTeX
% version 2005/12/01 or later.
%
% This work has the LPPL maintenance status `maintained'.
%
% The Current Maintainer of this work is Niklas Beisert.
%
% This work consists of the files childdoc.dtx and childdoc.ins
% and the derived files childdoc.def and cdocsamp.tex with
% cdocsch1.tex, cdocsch2.tex, cdocsdrf.tex, cdocsfn1.tex, cdocsfn2.tex.
%
%<package>\ifdefined\childdocmain\endinput\fi
%<package>\ProvidesFile{childdoc.def}[2018/12/30 v2.0 child document driver]
%<samplemain>\ProvidesFile{cdocsamp.tex}[2018/12/30 v2.0 sample for childdoc]
%<*driver>
%\ProvidesFile{childdoc.drv}[2018/12/30 v2.0 childdoc reference manual file]
\PassOptionsToClass{10pt,a4paper}{article}
\documentclass{ltxdoc}

\usepackage[margin=35mm]{geometry}
\usepackage{hyperref}
\usepackage{hyperxmp}
\usepackage[usenames]{color}

\hypersetup{colorlinks=true}
\hypersetup{pdfstartview=FitH}
\hypersetup{pdfpagemode=UseNone}
\hypersetup{pdfsource={}}
\hypersetup{pdflang={en-UK}}
\hypersetup{pdfcopyright={Copyright 2017-2018 Niklas Beisert.
  This work may be distributed and/or modified under the
  conditions of the LaTeX Project Public License, either version 1.3
  of this license or (at your option) any later version.}}
\hypersetup{pdflicenseurl={http://www.latex-project.org/lppl.txt}}
\hypersetup{pdfcontactaddress={ETH Zurich, ITP, HIT K,
  Wolfgang-Pauli-Strasse 27}}
\hypersetup{pdfcontactpostcode={8093}}
\hypersetup{pdfcontactcity={Zurich}}
\hypersetup{pdfcontactcountry={Switzerland}}
\hypersetup{pdfcontactemail={nbeisert@itp.phys.ethz.ch}}
\hypersetup{pdfcontacturl={http://people.phys.ethz.ch/\xmptilde nbeisert/}}

\newcommand{\secref}[1]{\hyperref[#1]{section \ref*{#1}}}

\parskip1ex
\parindent0pt
\let\olditemize\itemize
\def\itemize{\olditemize\parskip0pt}

\begin{document}

\title{The \textsf{childdoc} Package}
\hypersetup{pdftitle={The childdoc Package}}
\author{Niklas Beisert\\[2ex]
  Institut f\"ur Theoretische Physik\\
  Eidgen\"ossische Technische Hochschule Z\"urich\\
  Wolfgang-Pauli-Strasse 27, 8093 Z\"urich, Switzerland\\[1ex]
  \href{mailto:nbeisert@itp.phys.ethz.ch}
  {\texttt{nbeisert@itp.phys.ethz.ch}}}
\hypersetup{pdfauthor={Niklas Beisert}}
\hypersetup{pdfsubject={Manual for the LaTeX2e Package childdoc}}
\date{30 December 2018, \textsf{v2.0}}
\maketitle

\begin{abstract}\noindent
\textsf{childdoc} is a \LaTeXe{} package
that enables the direct compilation
of document sections included by |\include|
to individual files.
\end{abstract}

\begingroup
\parskip0ex
\tableofcontents
\endgroup

%%%%%%%%%%%%%%%%%%%%%%%%%%%%%%%%%%%%%%%%%%%%%%%%%%%%%%%%%%%%%%%%%%%%%%%%%%%%%%%%
%%%%%%%%%%%%%%%%%%%%%%%%%%%%%%%%%%%%%%%%%%%%%%%%%%%%%%%%%%%%%%%%%%%%%%%%%%%%%%%%
\section{Introduction}

\LaTeX{} provides a mechanism to structure a large document (such as a book)
into a main file and several child files (containing the chapters)
using the |\include| command.
This mechanism is beneficial for documents
which span hundreds of pages in order to
make the source file(s) more manageable.
Moreover, compilation can be restricted to
selected child files by means of the |\includeonly| command.
The latter feature can be used to reduce the compilation time while editing
(this was significantly more useful in the earlier days of \LaTeX{})
or to generate a smaller document which is easier to navigate.
Another application of |\includeonly| is to generate
documents consisting of selected parts of the complete document.

However, there are a few drawbacks of the plain |\include| mechanism:
\begin{itemize}
\item
The child files cannot be compiled on their own,
they can only be compiled via the main file.
A naive editing environment
(such as a text editor with an option
to have the current file processed by \LaTeX)
may require one to switch to the main file before compiling;
attempting to compile the child file produces errors.
\item
The main file must be modified (each time)
to adjust the |\includeonly| command
to the present needs. This easily leaves the main file in a messy state.
\item
The generated document will always carry the filename
of the main document. This is inconvenient if
several child files are to be compiled and
to be kept for distribution.
\end{itemize}

The present package provides a simple interface
to make child files individually compilable by \LaTeX{}.
Compiling a child file then has the same effect as compiling
the main file with an |\includeonly| command
to select the appropriate child.
Moreover the generated document will carry the name of the child
rather than the main file.
This resolves all three above issues.

This feature is meant to make the editing of books,
thesis documents and lecture notes somewhat more convenient.
However, the package can also be used efficiently for
composing a series of documents (such as exercise sheets)
which are typically distributed individually.
It then assists the author in generating the individual documents
(potentially in different versions)
as well as a document containing the collected series.
Another application is in developing style files
or other kinds of included material
where compilation of the style file could redirect
to a sample or test file.

%%%%%%%%%%%%%%%%%%%%%%%%%%%%%%%%%%%%%%%%%%%%%%%%%%%%%%%%%%%%%%%%%%%%%%%%%%%%%%%%
%%%%%%%%%%%%%%%%%%%%%%%%%%%%%%%%%%%%%%%%%%%%%%%%%%%%%%%%%%%%%%%%%%%%%%%%%%%%%%%%
\section{Usage}

First of all, the package \textsf{childdoc} is \emph{not} a standard
\LaTeXe{} |.sty| style file! Therefore it needs to be invoked in
a non-standard way.

%%%%%%%%%%%%%%%%%%%%%%%%%%%%%%%%%%%%%%%%%%%%%%%%%%%%%%%%%%%%%%%%%%%%%%%%%%%%%%%%
\subsection{Included Files}
\label{sec:include}

%%%%%%%%%%%%%%%%%%%%%%%%%%%%%%%%%%%%%%%%
\DescribeMacro{\childdocmain}
To use the package, add the commands
\begin{center}
\begin{tabular}{l}
|\input{childdoc.def}|\\
|\childdocmain{}|\\
\end{tabular}
\end{center}
at the very top of the main \LaTeX{} file,
in particular \emph{before} the |\documentclass| statement!
The argument of |\childdocmain| should be left empty
(but it must be present).

%%%%%%%%%%%%%%%%%%%%%%%%%%%%%%%%%%%%%%%%
\DescribeMacro{\childdocof}
Furthermore, add the commands
\begin{center}
\begin{tabular}{l}
|\input{childdoc.def}|\\
|\childdocof{|\textit{main}|}|\\
\end{tabular}
\end{center}
at the top of every child file \textit{child}
which is included by |\include{|\textit{child}|}|
from within the main file
(or at least for those files to be compiled individually).
The argument \textit{main} must be the filename of the main file.

There are a couple of
considerations in setting up the main and child documents:

%%%%%%%%%%%%%%%%%%%%%%%%%%%%%%%%%%%%%%%%
\paragraph{Restrictions.}

Please note the following restrictions:
\begin{itemize}
\item
|\childdocmain| must be called with one argument \textit{main}
to ensure compatibility with earlier version of the package.
It must either be empty (|\childdocmain{}|)
or precisely match the filename of the main file in which it is specified.
See \secref{sec:detection} for further information.
\item
The filename \textit{main} must be specified without the |.tex| extension.
\item
The filename \textit{main} is case sensitive
(even in case-insensitive file systems)
due to internal string comparison.
\item
The argument \textit{main} should be fully expanded, it cannot be a macro.
\item
Subdirectories and special characters should be avoided in filenames.
\item
The command |\childdocmain{|\textit{main}|}| must be followed by a whitespace.
It should not be followed immediately by another command
or by a comment mark `|%|'.
This is because the \TeX{} parser reads the token immediately following
the argument of |\childdocmain| and puts it
at the beginning of every child section;
however, a white\-space is ignored.
\end{itemize}

%%%%%%%%%%%%%%%%%%%%%%%%%%%%%%%%%%%%%%%%
\paragraph{Content of Main File.}

It is advisable to place all content in the child files included by |\include|.
Any output contained in the main file will appear in all child documents
unless suppressed manually;
it cannot be suppressed automatically by the |\includeonly| directive
and thus should normally be avoided.
A method to include some content in the main file
by means of conditional processing is described in \secref{sec:conditional}.

%%%%%%%%%%%%%%%%%%%%%%%%%%%%%%%%%%%%%%%%
\paragraph{Page Numbering.}

When only a part of the document is compiled,
the appropriate numbering of pages
(as well as other status parameters)
is determined from the |.aux| files.
The latter contain information from previous passes.
However this information needs to propagate through
all intermediate child documents.
Therefore the page numbering in child documents may well
be inconsistent until the complete document is compiled at least once.

A useful (if unconventional) way to always ensure a consistent
page numbering is to restart the numbering in each child document
and denote the pages by `\textit{child}|.|\textit{page}'
where \textit{child} represents the chapter/section number of the child file.
This can be achieved by the command
|\numberwithin{page}{|\textit{child}|}|
of the \textsf{amsmath} package
where \textit{child} can be |chapter| or |section|
depending on the chosen structuring.
Alternatively, one can modify the macro |\thepage| appropriately
and reset the counter |page| at the start of each child file.

%%%%%%%%%%%%%%%%%%%%%%%%%%%%%%%%%%%%%%%%%%%%%%%%%%%%%%%%%%%%%%%%%%%%%%%%%%%%%%%%
\subsection{Conditional Processing}
\label{sec:conditional}

The package provides a mechanism to compile different versions
of a document. To customise the versions further some conditional processing
can come in handy to distinguish which version is being compiled.
The package provides two macros to describe the compilation context:

%%%%%%%%%%%%%%%%%%%%%%%%%%%%%%%%%%%%%%%%
\DescribeMacro{\ifchilddoc}
The conditional |\ifchilddoc| distinguishes between the compilation of
child documents and the main document:
%
\begin{center}
|\ifchilddoc |\textit{child-code}| |[|\||else |\textit{main-code}]| \||fi|
\end{center}

%%%%%%%%%%%%%%%%%%%%%%%%%%%%%%%%%%%%%%%%
\DescribeMacro{\childdocname}
\DescribeMacro{\childdocjob}
The macro |\childdocname| contains the filename (without extension)
of the main or child file being processed.
Note that |\childdocjob| will always contain the name of the main file.

%%%%%%%%%%%%%%%%%%%%%%%%%%%%%%%%%%%%%%%%
\paragraph{Title Page.}

Conditional processing can be used to include a title or banner page
in the main document when proper precautions are taken.
Importantly, the code in the main file should ensure that the page counter
(as well as other status parameters which are stored in the |.aux| files)
takes the same value after the conditional processing.
Otherwise the page numbers may take divergent values
depending on which part is compiled.

For example, a title page could be declared by:
%
\begin{center}
\begin{tabular}{l}
|\ifchilddoc\||else|\\
|\addtocounter{page}{-1}|\\
\textit{code for title page}\\
|\newpage|\\
|\||fi|
\end{tabular}
\end{center}
%
A banner page for the child documents can be generated by:
%
\begin{center}
\begin{tabular}{l}
|\ifchilddoc|\\
|\addtocounter{page}{-1}|\\
\textit{code for banner page}\\
|\newpage|\\
|\||fi|
\end{tabular}
\end{center}
%
Here one could write a message such as:
\begin{center}
|This is the part \childdocname{} of \childdocjob{}.|
\end{center}

%%%%%%%%%%%%%%%%%%%%%%%%%%%%%%%%%%%%%%%%%%%%%%%%%%%%%%%%%%%%%%%%%%%%%%%%%%%%%%%%
\subsection{Flags}
\label{sec:flags}

The package makes it easy to generate different versions
of the main or child documents.
To this end compilation flags can be defined
and assigned different default values.
They will be particularly useful in conjunction
with the forwarding mechanism described in \secref{sec:forward}.

For example, it may be useful to have a flag |\version|
which can be set to |draft| or |final|.
The document source will contain some conditional code
depending on the value of |\version|.
Suppose further, the flag should default to |final| for the main file
and to |draft| for child files
which is a natural assignment for editing the document.
This is achieved by placing the following code
in the preamble of the main document
(below the |\childdocmain| directive):
%
\begin{center}
\begin{tabular}{l}
|\ifchilddoc|\\
|\providecommand{\version}{draft}|\\
|\||else|\\
|\providecommand{\version}{final}|\\
|\||fi|
\end{tabular}
\end{center}
%
The definition by |\providecommand| makes sure
that previous definitions are not overwritten.
Further statements |\providecommand{\version}{...}|
can thus be added before the above code to override it.

For the main file, one might add a line
(between |\childdocmain| and the above block)
%
\begin{center}
|%\ifchilddoc\||else\providecommand{\version}{draft}\||fi|
\end{center}
%
which can be uncommented to produce a draft version.
Likewise one can add a line to the very top of a child file
(above the |\childdocof{|\textit{main}|}| directive)
%
\begin{center}
|%\providecommand{\version}{final}|
\end{center}
%
which can be uncommented to produce the final version of this child document.

%%%%%%%%%%%%%%%%%%%%%%%%%%%%%%%%%%%%%%%%%%%%%%%%%%%%%%%%%%%%%%%%%%%%%%%%%%%%%%%%
\subsection{Forwarding}
\label{sec:forward}

Different versions of the main or child documents
using compilation flags as described in \secref{sec:flags}
can be (permanently) stored in different files
for convenient compilation, viewing and distribution.
To this end, the package defines a command
to pass on compilation to a different file:

%%%%%%%%%%%%%%%%%%%%%%%%%%%%%%%%%%%%%%%%
\DescribeMacro{\childdocforward}
The command |\childdocforward| redirects processing to
another source file:
%
\begin{center}
\begin{tabular}{l}
|\input{childdoc.def}|\\
|\childdocforward[|\textit{main}|]{|\textit{dest}|}|\\
\end{tabular}
\end{center}
%
The argument \textit{dest} is the destination file
(without extension).
It should be the main file or one of the child files.
Note that further \textsf{childdoc} directives
such as |\childdocof| and |\childdocforward|
in the indicated file will be processed in this form.
The optional argument \textit{main}
passes on directly to the main file \textit{main}
while pretending to compile the child \textit{dest}.
This form behaves as if \textit{dest}
issues |\childdocof{|\textit{main}|}| right away,
and no further \textsf{childdoc} directives will be processed.

%%%%%%%%%%%%%%%%%%%%%%%%%%%%%%%%%%%%%%%%
\DescribeMacro{\...prefix}
In the alternative form |\childdocforwardprefix|,
%
\begin{center}
\begin{tabular}{l}
|\input{childdoc.def}|\\
|\childdocforwardprefix[|\textit{main}|]{|\textit{prefix}|}{|\textit{dest}|}|
\end{tabular}
\end{center}
%
the destination file is determined by a pattern
depending on the current file:
To make this work, the current file must be called
`{\textit{prefix}\hspace{0.2em}\textit{suffix}}'
with \textit{prefix} matching precisely the argument.
Processing is then passed on to the file
`{\textit{dest}\hspace{0.2em}\textit{suffix}}'.
Surely, the same effect is achieved by
directly specifying the
argument `{\textit{dest}\hspace{0.2em}\textit{suffix}}'
in the first form.
However, that requires to set up a different file
for each child. With the alternative form of the command
all these files can have exactly the same content
which simplifies setting them up and maintaining them.

For example, the following file |draft.tex|
with a compilation flag |\version| as described in \secref{sec:flags}
compiles the main document as a draft:
%
\begin{center}
\begin{tabular}{l}
|\def\version{draft}|\\
|\input{childdoc.def}|\\
|\childdocforward{|\textit{main}|}|
\end{tabular}
\end{center}
%
Likewise, the following files |final|\textit{nn}|.tex|
compile the final version of the child document
|child|\textit{nn}|.tex|:
%
\begin{center}
\begin{tabular}{l}
|\def\version{final}|\\
|\input{childdoc.def}|\\
|\childdocforwardprefix{final}{child}|
\end{tabular}
\end{center}
%

Note that when several versions of a main file and/or of each child file
are to be generated, it may be convenient to set up a |Makefile| or
shell script to automatise the process.

%%%%%%%%%%%%%%%%%%%%%%%%%%%%%%%%%%%%%%%%%%%%%%%%%%%%%%%%%%%%%%%%%%%%%%%%%%%%%%%%
\subsection{Command Line Processing}
\label{sec:commandline}

The effect of redirection files can also be achieved by invoking
the \LaTeX{} compiler with a more elaborate command line.
Most conveniently this should be done as part
of a shell script or a |Makefile|.

When using \textsf{childdoc} in the main file, the following
command lines effectively perform a redirection
(note that depending on the shell being used,
backslashes may have to be doubled: `|\|' $\to$ `|\\|'):
%
\begin{center}
|... -jobname "|\textit{target}|" |\\|"|[\textit{flags}]%
|\input{childdoc.def}\childdocforward[|\textit{main}|]{|\textit{dest}|}"|
\end{center}
%
Here \textit{target} is the name of the output file,
\textit{main} is the name of the main file
and \textit{dest} is the name of the main or child file to be processed
(all filenames without extensions).
The optional argument \textit{main} can be omitted
if \textit{main} matches \textit{dest}.
Optionally, compilation \textit{flags} can be defined via |\def| commands.
This command line makes the \TeX{} engine believe
it is compiling the file \textit{target}
whose content is specified as the latter parameter.
The provided code then forwards the processing to
\textit{main} or \textit{dest} as described in \secref{sec:forward}.

%%%%%%%%%%%%%%%%%%%%%%%%%%%%%%%%%%%%%%%%%%%%%%%%%%%%%%%%%%%%%%%%%%%%%%%%%%%%%%%%
\subsection{Include by Input}
\label{sec:input}

Including child documents by |\include| has some restrictions by design.
Most notably, the content of a child document always occupies
its own set of pages; pages cannot be shared between child documents.
Usually, this behaviour makes perfect sense
because each child document contain an essential part of the document.
However, in some situations it may be desirable to compose
a document from a collection of parts
without having mandatory page breaks between then.
For this case, the package
provides a mechanism to include parts
by |\input| which can also be processed individually.
However, by construction this mechanism
requires manual handling of the content to be output.

%%%%%%%%%%%%%%%%%%%%%%%%%%%%%%%%%%%%%%%%
\DescribeMacro{\ifchilddocmanual}
The main file should be prepared as usual, see \secref{sec:include}.
However, the document body must make a distinction
between processing of an individual part and of the main document, e.g.:
%
\begin{center}
\begin{tabular}{l}
|\ifchilddocmanual|\\
|\input{\childdocname}|\\
|\||else|\\
\textit{document body with }|\input{|\textit{part}|}|\\
|\||fi|
\end{tabular}
\end{center}
%
The conditional |\ifchilddocmanual| is true whenever
a part to be included by |\input| is being compiled,
and the name of the part is stored in |\childdocname|.

%%%%%%%%%%%%%%%%%%%%%%%%%%%%%%%%%%%%%%%%
\DescribeMacro{\childdocby}
Each part to be included by |\input| should start with:
%
\begin{center}
\begin{tabular}{l}
|\input{childdoc.def}|\\
|\childdocby{|\textit{main}|}|\\
\end{tabular}
\end{center}
%
The directive |\childdocby| is similar to |\childdocof|
described in \secref{sec:include},
but the subsequent selection of content must be done manually.
To that end, both |\ifchilddoc| and |\ifchilddocmanual|
will be true upon processing of a part,
and the name of the part is stored in |\childdocname|.
Note that |\jobname| will be set to the filename of the current part
so that each part receives an individual |.aux| file
that does not interfere with the |.aux| file(s) of the main document.
This behaviour can be altered by the alternative form
|\childdocby[*]{|\textit{main}|}| (with a non-empty optional argument)
which uses the |.aux| file of the main document
by setting |\jobname| to \textit{main}.

%%%%%%%%%%%%%%%%%%%%%%%%%%%%%%%%%%%%%%%%%%%%%%%%%%%%%%%%%%%%%%%%%%%%%%%%%%%%%%%%
\subsection{Driver Development}
\label{sec:driver}

The \textsf{childdoc} mechanism can also be use for the development
of definition files such as \LaTeX{} styles or classes.
This case differs from the above setup with multiple parts
included by |\include| in that no |\includeonly| should be invoked.
This can be achieved by starting the include file
(before |\ProvidesPackage|) with:
%
\begin{center}
\begin{tabular}{l}
|\input{childdoc.def}|\\
|\childdocforward{|\textit{main}|}|\\
\end{tabular}
\end{center}
%
or alternatively with:
%
\begin{center}
\begin{tabular}{l}
|\input{childdoc.def}|\\
|\childdocby{|\textit{main}|}|\\
\end{tabular}
\end{center}
%
Both forms have slightly different effects as described above.
The main file is prepared as usual, see \secref{sec:include}.

%%%%%%%%%%%%%%%%%%%%%%%%%%%%%%%%%%%%%%%%%%%%%%%%%%%%%%%%%%%%%%%%%%%%%%%%%%%%%%%%
\subsection{Legacy Detection}
\label{sec:detection}

The directive |\childdocmain| in the main file can detect
whether the complete document or merely a child is to be compiled
even without using the directive |\childdocof|.
This method is deprecated because it is less robust
and there is no compelling reason to use it;
it is merely provided for backward compatibility
and it may be removed in future versions.

If the detection mechanism is to be used,
it is mandatory to correctly specify
the filename of the main file as the argument of |\childdocmain|:
%
\begin{center}
\begin{tabular}{l}
|\input{childdoc.def}|\\
|\childdocmain{|\textit{main}|}|\\
\end{tabular}
\end{center}
%
If |\jobname| does not match the argument \textit{main} of |\childdocmain|,
it is assumed that |\jobname| points to the child file to be compiled.
When using |\childdocmain| with the main file specified as argument,
it suffices to start a child file
with just |\input{|\textit{main}|}|
without loading of the package and using |\childdocof|.
If instead all processing is done
with the appropriate \textsf{childdoc} directives,
the argument of \textit{main} of |\childdocmain| can be empty.

An alternative version of the command line processing described
in \secref{sec:commandline} using the detection mechanism reads:
%
\begin{center}
|... -jobname "|\textit{target}|" "|[\textit{flags}]%
[|\def\jobname{|\textit{dest}|}|]|\input{|\textit{main}|}"|
\end{center}

%%%%%%%%%%%%%%%%%%%%%%%%%%%%%%%%%%%%%%%%%%%%%%%%%%%%%%%%%%%%%%%%%%%%%%%%%%%%%%%%
\subsection{Manual Code}
\label{sec:manual}

In case one cannot be certain whether the definitions file |childdoc.def|
is installed on the target \TeX{} distribution
and one prefers not to ship it,
it is conceivable to paste a few relevant commands into the sources.

To that end, drop all statements |\input{childdoc.def}|
and perform the replacements as outlined below.
Instead of |\childdocmain{|\textit{main}|}| add the following code
to the top of the main file:
%
\begin{center}
\begin{tabular}{l}
|\||ifdefined\childdocname\endinput\||fi\newif\ifchilddoc|\\
|\edef\childdocname{\scantokens\expandafter{\jobname\noexpand}}|\\
|\def\childdocmain{|\textit{main}|}\||ifx\childdocmain\childdocname\||else|\\
|\childdoctrue\includeonly{\childdocname}\let\jobname\childdocmain\||fi|\\
\end{tabular}
\end{center}
%
Instead of |\childdocof{|\textit{main}|}| just include the main file
at the top of each child file:
%
\begin{center}
|\input{|\textit{main}|}|
\end{center}
%
A simple redirection |\childdocforward{|\textit{dest}|}| is achieved by:
%
\begin{center}
|\def\jobname{|\textit{dest}|}\input{\jobname}|
\end{center}
%
The redirection with prefix
|\childdocforwardprefix[|\textit{prefix}|]{|\textit{dest}|}|
is accomplished by:
%
\begin{center}
\begin{tabular}{l}
|{\edef\jobname{\scantokens\expandafter{\jobname\noexpand}}|\\
|\def\redirectjob |\textit{prefix}|#1~~~{\gdef\jobname{|\textit{dest}|#1}}|\\
|\expandafter\redirectjob\jobname~~~}\input{\jobname}|
\end{tabular}
\end{center}

In an alternative approach,
child documents can be compiled by a specific command line
without additional code or specific definitions:
%
\begin{center}
|... -jobname "|\textit{target}|" "|[\textit{flags}]%
|\includeonly{|\textit{dest}|}\input{|\textit{main}|}"|
\end{center}
%

%%%%%%%%%%%%%%%%%%%%%%%%%%%%%%%%%%%%%%%%%%%%%%%%%%%%%%%%%%%%%%%%%%%%%%%%%%%%%%%%
%%%%%%%%%%%%%%%%%%%%%%%%%%%%%%%%%%%%%%%%%%%%%%%%%%%%%%%%%%%%%%%%%%%%%%%%%%%%%%%%
\section{Information}

%%%%%%%%%%%%%%%%%%%%%%%%%%%%%%%%%%%%%%%%%%%%%%%%%%%%%%%%%%%%%%%%%%%%%%%%%%%%%%%%
\subsection{Copyright}

Copyright \copyright{} 2017--2018 Niklas Beisert

This work may be distributed and/or modified under the
conditions of the \LaTeX{} Project Public License, either version 1.3
of this license or (at your option) any later version.
The latest version of this license is in
  \url{http://www.latex-project.org/lppl.txt}
and version 1.3 or later is part of all distributions of \LaTeX{}
version 2005/12/01 or later.

This work has the LPPL maintenance status `maintained'.

The Current Maintainer of this work is Niklas Beisert.

This work consists of the files |README.txt|, |childdoc.ins| and |childdoc.dtx|
as well as the derived files |childdoc.def|, |cdocsamp.tex|
with |cdocsch1.tex|, |cdocsch2.tex|, |cdocspt3.tex|, |cdocspt4.tex|,
|cdocsdrf.tex|, |cdocsfn1.tex|, |cdocsfn2.tex|
as well as |childdoc.pdf|.

%%%%%%%%%%%%%%%%%%%%%%%%%%%%%%%%%%%%%%%%%%%%%%%%%%%%%%%%%%%%%%%%%%%%%%%%%%%%%%%%
\subsection{Files and Installation}

The package consists of the files:
%
\begin{center}
\begin{tabular}{ll}
    |README.txt|   & readme file \\
    |childdoc.ins| & installation file \\
    |childdoc.dtx| & source file \\
    |childdoc.def| & definition file \\
    |cdocsamp.tex| & sample main file \\
    |cdocsch1.tex| & sample include file \\
    |cdocsch2.tex| & sample include file \\
    |cdocspt3.tex| & sample part file \\
    |cdocspt4.tex| & sample part file \\
    |cdocsdrf.tex| & sample redirection file \\
    |cdocsfn1.tex| & sample redirection file \\
    |cdocsfn2.tex| & sample redirection file \\
    |childdoc.pdf| & manual
\end{tabular}
\end{center}
%
The distribution consists of the files
|README.txt|, |childdoc.ins| and |childdoc.dtx|.
%
\begin{itemize}
\item
Run (pdf)\LaTeX{} on |childdoc.dtx|
to compile the manual |childdoc.pdf| (this file).
\item
Run \LaTeX{} on |childdoc.ins| to create the definitions file |childdoc.def|
and the sample |cdocsamp.tex| with include files
|cdocsch1.tex|, |cdocsch2.tex|, |cdocspt3.tex|, |cdocspt4.tex|,
|cdocsdrf.tex|, |cdocsfn1.tex|, |cdocsfn2.tex|.
Then copy the file |childdoc.def| to an appropriate directory of your \LaTeX{}
distribution, e.g.\ \textit{texmf-root}|/tex/latex/childdoc|.
\end{itemize}

%%%%%%%%%%%%%%%%%%%%%%%%%%%%%%%%%%%%%%%%%%%%%%%%%%%%%%%%%%%%%%%%%%%%%%%%%%%%%%%%
\subsection{Related CTAN Packages}

There are several other packages which offer a similar functionality:
%
\begin{itemize}
\item
The packages
\href{http://ctan.org/pkg/docmute}{\textsf{docmute}},
\href{http://ctan.org/pkg/includex}{\textsf{includex}} and
\href{http://ctan.org/pkg/standalone}{\textsf{standalone}}
provide commands to include only the document body of
a child file thus allowing both files to be compiled individually.
\item
The packages \href{http://ctan.org/pkg/subdocs}{\textsf{subdocs}}
and \href{http://ctan.org/pkg/subfiles}{\textsf{subfiles}}
provide structures in which the main and child documents can be
encapsulated and allowing them to be compiled individually.
The inclusion mechanism is different from the conventional |\include|.
\item
The package \href{http://ctan.org/pkg/combine}{\textsf{combine}}
is an elaborate solution to combine several documents into one.
\end{itemize}
%
See also the CTAN topic \href{http://ctan.org/topic/subdocs}{\textsf{subdocs}}
for further related packages.
The present package differs from the above solutions in that
a document structure constructed with the conventional |\include| mechanism
just needs two extra commands at the top of every file
such that all constituent files can be compiled individually.

%%%%%%%%%%%%%%%%%%%%%%%%%%%%%%%%%%%%%%%%%%%%%%%%%%%%%%%%%%%%%%%%%%%%%%%%%%%%%%%%
%\subsection{Feature Suggestions}
%
%The following is a list of features which may be useful for future
%versions of this package:
%%
%\begin{itemize}
%\item
%\ldots
%\end{itemize}

%%%%%%%%%%%%%%%%%%%%%%%%%%%%%%%%%%%%%%%%%%%%%%%%%%%%%%%%%%%%%%%%%%%%%%%%%%%%%%%%
\subsection{Revision History}

%%%%%%%%%%%%%%%%%%%%%%%%%%%%%%%%%%%%%%%%
\paragraph{v2.0:} 2018/12/30

\begin{itemize}
\item
immediate forward processing
\item
added |\childdocby| mechanism
\item
manual restructured
\end{itemize}

%%%%%%%%%%%%%%%%%%%%%%%%%%%%%%%%%%%%%%%%
\paragraph{v1.6:} 2018/01/17

\begin{itemize}
\item
application for development of include files
\item
corrections to manual
\end{itemize}

%%%%%%%%%%%%%%%%%%%%%%%%%%%%%%%%%%%%%%%%
\paragraph{v1.5:} 2017/05/21

\begin{itemize}
\item
more complete structuring introduced
\item
|\childdocof| introduced
\item
|\childdoc| renamed to |\childdocmain|
\item
|\childredirect| renamed to |\childdocforward| and |\childdocforwardprefix|
and functionality expanded
\end{itemize}

%%%%%%%%%%%%%%%%%%%%%%%%%%%%%%%%%%%%%%%%
\paragraph{v1.0:} 2017/04/27

\begin{itemize}
\item
manual and install package
\item
first version published on CTAN
\end{itemize}

%%%%%%%%%%%%%%%%%%%%%%%%%%%%%%%%%%%%%%%%
\paragraph{v0.6:} 2017/04/26

\begin{itemize}
\item
redirection mechanism added
\end{itemize}

%%%%%%%%%%%%%%%%%%%%%%%%%%%%%%%%%%%%%%%%
\paragraph{v0.5:} 2017/04/26

\begin{itemize}
\item
functionality in definition file
\end{itemize}


%%%%%%%%%%%%%%%%%%%%%%%%%%%%%%%%%%%%%%%%%%%%%%%%%%%%%%%%%%%%%%%%%%%%%%%%%%%%%%%%
%%%%%%%%%%%%%%%%%%%%%%%%%%%%%%%%%%%%%%%%%%%%%%%%%%%%%%%%%%%%%%%%%%%%%%%%%%%%%%%%
%%%%%%%%%%%%%%%%%%%%%%%%%%%%%%%%%%%%%%%%%%%%%%%%%%%%%%%%%%%%%%%%%%%%%%%%%%%%%%%%
\appendix

\settowidth\MacroIndent{\rmfamily\scriptsize 000\ }

 \DocInput{childdoc.dtx}

\end{document}
%</driver>
% \fi
%
% %%%%%%%%%%%%%%%%%%%%%%%%%%%%%%%%%%%%%%%%%%%%%%%%%%%%%%%%%%%%%%%%%%%%%%%%%%%%%%
% %%%%%%%%%%%%%%%%%%%%%%%%%%%%%%%%%%%%%%%%%%%%%%%%%%%%%%%%%%%%%%%%%%%%%%%%%%%%%%
% \section{Sample}
%\iffalse
%<*samplemain>
%\fi
%
% The following presents a sample document
% with two chapters, two parts, a title page,
% a compile flag as well as three forwarding files to set the flag.
% It consists of eight |.tex| files:
% \begin{center}
% \begin{tabular}{ll}
% |cdocsamp.tex|&main file\\
% |cdocsch1.tex|&include file for chapter 1\\
% |cdocsch2.tex|&include file for chapter 2\\
% |cdocspt3.tex|&include file for part 3\\
% |cdocspt4.tex|&include file for part 4\\
% |cdocsdrf.tex|&forwarding file for main file in draft mode\\
% |cdocsfi1.tex|&forwarding file for final version of chapter 1\\
% |cdocsfi2.tex|&forwarding file for final version of chapter 2\\
% \end{tabular}
% \end{center}
% Each of the eight files can be compiled directly by the \LaTeX{} compiler.
%
% %%%%%%%%%%%%%%%%%%%%%%%%%%%%%%%%%%%%%%
% \paragraph{Main File.}
%
% The main file is called |cdocsamp.tex|.
%
% Load the \textsf{childdoc} definitions and
% declare the filename for the main document:
%    \begin{macrocode}
\input{childdoc.def}
\childdocmain{}
%    \end{macrocode}

% Optional override for |\version| flag:
%    \begin{macrocode}
%%\ifchilddoc\else\providecommand{\version}{draft}\fi
%    \end{macrocode}

% Define the default values for the |\version| flag
% (|final| for the main file and |draft| for childs):
%    \begin{macrocode}
\ifchilddoc
\providecommand{\version}{draft}
\else
\providecommand{\version}{final}
\fi
%    \end{macrocode}

% Load the standard document class:
%    \begin{macrocode}
\documentclass[12pt]{article}
%    \end{macrocode}

% Start the document body:
%    \begin{macrocode}
\begin{document}
%    \end{macrocode}

% Declare a title page.
% Print title, part of document being processed and version flag:
%    \begin{macrocode}
\addtocounter{page}{-1}
\begin{center}
{\LARGE\bfseries{}childdoc example\par}
\vspace{1cm}
\ifchilddoc
\ifchilddocmanual part\else chapter\fi:
`\childdocname' of `\childdocjob'\par
\else
main document: `\childdocjob'\par
\fi
version: \version\par
\end{center}
\newpage
%    \end{macrocode}

% Manually include selected file,
% otherwise process as usual:
%    \begin{macrocode}
\ifchilddocmanual
\section*{part `\childdocname'}
\input{\childdocname}
\else
%    \end{macrocode}

% Include the two chapters:
%    \begin{macrocode}
\include{cdocsch1}
\include{cdocsch2}
%    \end{macrocode}

% Include the two parts unless only chapters should be displayed:
%    \begin{macrocode}
\ifchilddoc\else
\section{part three}
\input{cdocspt3}
\section{part four}
\input{cdocspt4}
\fi
%    \end{macrocode}

% Process as usual until here:
%    \begin{macrocode}
\fi
%    \end{macrocode}

% End of document body:
%    \begin{macrocode}
\end{document}
%    \end{macrocode}
%\iffalse
%</samplemain>
%\fi
%
% %%%%%%%%%%%%%%%%%%%%%%%%%%%%%%%%%%%%%%
% \paragraph{Chapter Include Files.}
%
% The include files are called |cdocsch1.tex| and |cdocsch2.tex|.
%
%\iffalse
%<*samplechap1|samplechap2>
%\fi

% Optional override for |\version| flag:
%    \begin{macrocode}
%%\providecommand{\version}{final}
%    \end{macrocode}

% Include the main document:
%    \begin{macrocode}
\input{childdoc.def}
\childdocof{cdocsamp}
%    \end{macrocode}

%\iffalse
%</samplechap1|samplechap2>
%\fi
%
%\iffalse
%<*samplechap1>
%\fi
% Some text for chapter 1:
%    \begin{macrocode}
\section{one}
some text in chapter one
%    \end{macrocode}

%\iffalse
%</samplechap1>
%\fi
% Some text for chapter 2:
%\iffalse
%<*samplechap2>
%\fi
%    \begin{macrocode}
\section{two}
more text in chapter two
%    \end{macrocode}

%\iffalse
%</samplechap2>
%\fi
%
% %%%%%%%%%%%%%%%%%%%%%%%%%%%%%%%%%%%%%%
% \paragraph{Part Include Files.}
%
% The include files are called |cdocspt3.tex| and |cdocspt4.tex|.
%
%\iffalse
%<*samplepart3|samplepart4>
%\fi

% Optional override for |\version| flag:
%    \begin{macrocode}
%%\providecommand{\version}{final}
%    \end{macrocode}

% Include the main document:
%    \begin{macrocode}
\input{childdoc.def}
\childdocby{cdocsamp}
%    \end{macrocode}

%\iffalse
%</samplepart3|samplepart4>
%\fi
%
%\iffalse
%<*samplepart3>
%\fi
% Some text for part 3:
%    \begin{macrocode}
some text in part three
%    \end{macrocode}

%\iffalse
%</samplepart3>
%\fi
% Some text for part 4:
%\iffalse
%<*samplepart4>
%\fi
%    \begin{macrocode}
more text in part four
%    \end{macrocode}

%\iffalse
%</samplepart4>
%\fi
%
% %%%%%%%%%%%%%%%%%%%%%%%%%%%%%%%%%%%%%%
% \paragraph{Forwarding for a Complete Draft.}
%
% The following forwarding file |cdocsdrf.tex|
% compiles the main document in draft mode:
%\iffalse
%<*sampledraft>
%\fi
%    \begin{macrocode}
\def\version{draft}
\input{childdoc.def}
\childdocforward{cdocsamp}
%    \end{macrocode}

%\iffalse
%</sampledraft>
%\fi
%
% %%%%%%%%%%%%%%%%%%%%%%%%%%%%%%%%%%%%%%
% \paragraph{Forwarding for Final Version of the Chapters.}
%
% The following forwarding files |cdocsfn1.tex| and |cdocsfn2.tex|
% (with identical content)
% compile the final versions of the child documents
% |cdocsch1.tex| and |cdocsch2.tex|, respectively:
%\iffalse
%<*samplefinal>
%\fi
%    \begin{macrocode}
\def\version{final}
\input{childdoc.def}
\childdocforwardprefix[cdocsamp]{cdocsfn}{cdocsch}
%    \end{macrocode}

%\iffalse
%</samplefinal>
%\fi
%
% %%%%%%%%%%%%%%%%%%%%%%%%%%%%%%%%%%%%%%
% \paragraph{Command Line Processing.}
%
% The following three command lines generate the output files
% |cdocscld|, |cdocscl1| and |cdocscl2|
% which should be identical to
% |cdocsdrf|, |cdocsch1| and |cdocsfn2|, respectively:
% \begin{center}
% \begin{tabular}{l}
% |latex -jobname cdocscld \|\\
% |  "\def\version{draft}\input{childdoc.def}\childdocforward{cdocsamp}"|\\
% |latex -jobname cdocscl1 \|\\
% |  "\input{childdoc.def}\childdocforward[cdocsamp]{cdocsch1}"|\\
% |latex -jobname cdocscl2 \|\\
% |  "\def\version{final}\input{childdoc.def}\childdocforward{cdocsch2}"|
% \end{tabular}
% \end{center}
% Note that the trailing backslash on each first line
% merely continues the input to the second line
% (for convenient cut ant paste).
% Furthermore, the command |latex| can be replaced by any
% of its alternative versions such as |pdflatex|.
%
% %%%%%%%%%%%%%%%%%%%%%%%%%%%%%%%%%%%%%%%%%%%%%%%%%%%%%%%%%%%%%%%%%%%%%%%%%%%%%%
% %%%%%%%%%%%%%%%%%%%%%%%%%%%%%%%%%%%%%%%%%%%%%%%%%%%%%%%%%%%%%%%%%%%%%%%%%%%%%%
% \section{Implementation}
%\iffalse
%<*package>
%\fi
%
% This section describes the definitions file |childdoc.def|.

% The definitions cannot be loaded using |\usepackage| or |\RequirePackage|
% which has a mechanism to prevent loading a style file more than once.
% When loading the definitions by means of |\input|
% multiple instances have to be prevented manually:
%\iffalse
%This code needs to be before the `\ProvidesFile' directive
%which is defined at the beginning of this file.
%Therefore it is also placed there and commented out here.
%</package>
%<*discard>
%\fi
%    \begin{macrocode}
\ifdefined\childdocmain\endinput\fi
%    \end{macrocode}
%\iffalse
%</discard>
%<*package>
%\fi
%
% \macro{\ifchilddoc}
% \macro{\ifchilddocmanual}
% The conditional |\ifchilddoc| tells whether a
% child (true) or main (false) document is being compiled.
% The conditional |\ifchilddocmanual| tells whether
% the |\includeonly| mechanism is used (false) or
% the selection of child files must be performed manually (true).
% The definitions initialise to false:
%    \begin{macrocode}
\newif\ifchilddoc
\newif\ifchilddocmanual
%    \end{macrocode}

% \macro{\childdocname}
% \macro{\childdocjob}
% The macro |\childdocname| stores the name of the main document
% to be compiled. The macro |\childdocjob| stores the name of
% the document on which the \LaTeX{} compiler was originally invoked.
% The content of |\jobname| cannot be compared
% to filenames specified in the source due to different catcodes.
% The following code rescans |\jobname|, stores the result
% in |\childdocname| and saves a copy in |\childdocjob|:
%    \begin{macrocode}
\edef\childdocname{\scantokens\expandafter{\jobname\noexpand}}
\let\childdocjob\childdocname
%    \end{macrocode}

% \macro{\childdocdisable}
% The macro |\childdocdisable| prevents the main file
% from being processed more than once.
% At this stage, the main document command |\childdocmain|
% is assumed to be called once again where it should do nothing.
% Any subsequent call to it should prevent
% a secondary processing of the main document
% It overwrites the forwarding commands
% |\childdocof| and |\childdocforward|
% with empty macros to prevent further inclusions of the main document:
%    \begin{macrocode}
\newcommand{\childdocdisable}
{
  \renewcommand{\childdocmain}[1]{\renewcommand{\childdocmain}[1]{\endinput}}
  \renewcommand{\childdocof}[1]{}
  \renewcommand{\childdocby}[2][]{}
  \renewcommand{\childdocforward}[2][]{}
  \renewcommand{\childdocdisable}{}
}
%    \end{macrocode}

% \macro{\childdocmain}
% The macro |\childdocmain| is to be called at the top of the main file
% with nothing or the main filename (without extension) as argument.
% First, it breaks loops.
% If the argument is not empty and does not match |\childdocname|
% (which is set by the first inclusion of |childdoc.def|),
% |\ifchilddoc| is set to true, |\includeonly| is applied to the child file
% and |\jobname| is set to the main file
% (for proper handling of |.aux| files):
%    \begin{macrocode}
\newcommand{\childdocmain}[1]
{
  \childdocdisable\childdocmain{}
  \if?#1?\else
    \begingroup
      \def\childdoctmp{#1}
      \ifx\childdoctmp\childdocname
        \def\childdoctmp{}
      \else
        \def\childdoctmp
        {
          \childdoctrue
          \includeonly{\childdocname}
          \def\childdocjob{#1}
          \def\jobname{#1}
        }
      \fi
      \expandafter
    \endgroup
    \childdoctmp
  \fi
}
%    \end{macrocode}

% \macro{\childdocof}
% The command |\childdocof| redirects
% compilation to the main file |#1|.
%    \begin{macrocode}
\newcommand{\childdocof}[1]
{
  \childdocdisable
  \childdoctrue
  \includeonly{\childdocname}
  \def\jobname{#1}
  \def\childdocjob{#1}
  \input{#1}
}
%    \end{macrocode}

% \macro{\childdocby}
% The command |\childdocby| ....
%    \begin{macrocode}
\newcommand{\childdocby}[2][]
{
  \childdocdisable
  \childdoctrue
  \childdocmanualtrue
  \if?#1?\else
    \def\jobname{#2}
  \fi
  \def\childdocjob{#2}
  \input{#2}
  \endinput
}
%    \end{macrocode}

% \macro{\childdocforward}
% The command |\childdocforward| redirects
% compilation to the main file or
% (if the optional argument is given) a child file.
% Parameters are set as if the main file
% or a child file starting with |\childdocof| was compiled.
% Then compilation is handed over to the main file:
%    \begin{macrocode}
\newcommand{\childdocforward}[2][]
{
  \begingroup
    \if?#1?
      \def\childdoctmp
      {
        \def\childdocname{#2}
        \def\childdocjob{#2}
        \def\jobname{#2}
        \input{#2}
        \endinput
      }
    \else
      \def\childdoctmp
      {
        \childdocdisable
        \def\childdocname{#2}
        \childdoctrue
        \includeonly{#2}
        \def\childdocjob{#1}
        \def\jobname{#1}
        \input{#1}
        \endinput
      }
    \fi
    \expandafter
  \endgroup
  \childdoctmp
}
%    \end{macrocode}

% \macro{\childdocforwardprefix}
% The command |\childdocforwardprefix| redirects
% compilation to the main or a child file by means of a pattern.
% The prefix |#1| in the current filename is replaced by |#2|
% and the suffix of the current filename is kept
% (it is assumed that the filename does not contain the substring `|~~~|'
% which is used as a delimiter).
% Compilation is handed over to the new file by |\childdocforward|:
%    \begin{macrocode}
\newcommand{\childdocforwardprefix}[3][]
{
  \begingroup
    \def\childdocextract #2##1~~~{\def\childdoctmp{\childdocforward[#1]{#3##1}}}
    \expandafter\childdocextract\childdocname~~~
    \expandafter
  \endgroup
  \childdoctmp
}
%    \end{macrocode}

% \macro{\childdoc}
% The deprecated macro |\childdoc| is a legacy version of |\childdocmain|:
%    \begin{macrocode}
\newcommand{\childdoc}{\childdocmain}
%    \end{macrocode}

% \macro{\childdocredirect}
% The deprecated macro |\childdocredirect| is a legacy version
% of |\childdocforward| and |\childdocforwardprefix|:
%    \begin{macrocode}
\newcommand{\childdocredirect}[2][]
{
  \begingroup
    \if?#1?
      \def\childdoctmp{\childdocforward{#2}}
    \else
      \def\childdoctmp{\childdocforwardprefix{#1}{#2}}
    \fi
    \expandafter
  \endgroup
  \childdoctmp
}
%    \end{macrocode}

%\iffalse
%</package>
%\fi
%
\endinput

\childdocby{cdocsamp}
%    \end{macrocode}

%\iffalse
%</samplepart3|samplepart4>
%\fi
%
%\iffalse
%<*samplepart3>
%\fi
% Some text for part 3:
%    \begin{macrocode}
some text in part three
%    \end{macrocode}

%\iffalse
%</samplepart3>
%\fi
% Some text for part 4:
%\iffalse
%<*samplepart4>
%\fi
%    \begin{macrocode}
more text in part four
%    \end{macrocode}

%\iffalse
%</samplepart4>
%\fi
%
% %%%%%%%%%%%%%%%%%%%%%%%%%%%%%%%%%%%%%%
% \paragraph{Forwarding for a Complete Draft.}
%
% The following forwarding file |cdocsdrf.tex|
% compiles the main document in draft mode:
%\iffalse
%<*sampledraft>
%\fi
%    \begin{macrocode}
\def\version{draft}
% \iffalse
%
% childdoc.dtx Copyright (C) 2017-2018 Niklas Beisert
%
% This work may be distributed and/or modified under the
% conditions of the LaTeX Project Public License, either version 1.3
% of this license or (at your option) any later version.
% The latest version of this license is in
%   http://www.latex-project.org/lppl.txt
% and version 1.3 or later is part of all distributions of LaTeX
% version 2005/12/01 or later.
%
% This work has the LPPL maintenance status `maintained'.
%
% The Current Maintainer of this work is Niklas Beisert.
%
% This work consists of the files childdoc.dtx and childdoc.ins
% and the derived files childdoc.def and cdocsamp.tex with
% cdocsch1.tex, cdocsch2.tex, cdocsdrf.tex, cdocsfn1.tex, cdocsfn2.tex.
%
%<package>\ifdefined\childdocmain\endinput\fi
%<package>\ProvidesFile{childdoc.def}[2018/12/30 v2.0 child document driver]
%<samplemain>\ProvidesFile{cdocsamp.tex}[2018/12/30 v2.0 sample for childdoc]
%<*driver>
%\ProvidesFile{childdoc.drv}[2018/12/30 v2.0 childdoc reference manual file]
\PassOptionsToClass{10pt,a4paper}{article}
\documentclass{ltxdoc}

\usepackage[margin=35mm]{geometry}
\usepackage{hyperref}
\usepackage{hyperxmp}
\usepackage[usenames]{color}

\hypersetup{colorlinks=true}
\hypersetup{pdfstartview=FitH}
\hypersetup{pdfpagemode=UseNone}
\hypersetup{pdfsource={}}
\hypersetup{pdflang={en-UK}}
\hypersetup{pdfcopyright={Copyright 2017-2018 Niklas Beisert.
  This work may be distributed and/or modified under the
  conditions of the LaTeX Project Public License, either version 1.3
  of this license or (at your option) any later version.}}
\hypersetup{pdflicenseurl={http://www.latex-project.org/lppl.txt}}
\hypersetup{pdfcontactaddress={ETH Zurich, ITP, HIT K,
  Wolfgang-Pauli-Strasse 27}}
\hypersetup{pdfcontactpostcode={8093}}
\hypersetup{pdfcontactcity={Zurich}}
\hypersetup{pdfcontactcountry={Switzerland}}
\hypersetup{pdfcontactemail={nbeisert@itp.phys.ethz.ch}}
\hypersetup{pdfcontacturl={http://people.phys.ethz.ch/\xmptilde nbeisert/}}

\newcommand{\secref}[1]{\hyperref[#1]{section \ref*{#1}}}

\parskip1ex
\parindent0pt
\let\olditemize\itemize
\def\itemize{\olditemize\parskip0pt}

\begin{document}

\title{The \textsf{childdoc} Package}
\hypersetup{pdftitle={The childdoc Package}}
\author{Niklas Beisert\\[2ex]
  Institut f\"ur Theoretische Physik\\
  Eidgen\"ossische Technische Hochschule Z\"urich\\
  Wolfgang-Pauli-Strasse 27, 8093 Z\"urich, Switzerland\\[1ex]
  \href{mailto:nbeisert@itp.phys.ethz.ch}
  {\texttt{nbeisert@itp.phys.ethz.ch}}}
\hypersetup{pdfauthor={Niklas Beisert}}
\hypersetup{pdfsubject={Manual for the LaTeX2e Package childdoc}}
\date{30 December 2018, \textsf{v2.0}}
\maketitle

\begin{abstract}\noindent
\textsf{childdoc} is a \LaTeXe{} package
that enables the direct compilation
of document sections included by |\include|
to individual files.
\end{abstract}

\begingroup
\parskip0ex
\tableofcontents
\endgroup

%%%%%%%%%%%%%%%%%%%%%%%%%%%%%%%%%%%%%%%%%%%%%%%%%%%%%%%%%%%%%%%%%%%%%%%%%%%%%%%%
%%%%%%%%%%%%%%%%%%%%%%%%%%%%%%%%%%%%%%%%%%%%%%%%%%%%%%%%%%%%%%%%%%%%%%%%%%%%%%%%
\section{Introduction}

\LaTeX{} provides a mechanism to structure a large document (such as a book)
into a main file and several child files (containing the chapters)
using the |\include| command.
This mechanism is beneficial for documents
which span hundreds of pages in order to
make the source file(s) more manageable.
Moreover, compilation can be restricted to
selected child files by means of the |\includeonly| command.
The latter feature can be used to reduce the compilation time while editing
(this was significantly more useful in the earlier days of \LaTeX{})
or to generate a smaller document which is easier to navigate.
Another application of |\includeonly| is to generate
documents consisting of selected parts of the complete document.

However, there are a few drawbacks of the plain |\include| mechanism:
\begin{itemize}
\item
The child files cannot be compiled on their own,
they can only be compiled via the main file.
A naive editing environment
(such as a text editor with an option
to have the current file processed by \LaTeX)
may require one to switch to the main file before compiling;
attempting to compile the child file produces errors.
\item
The main file must be modified (each time)
to adjust the |\includeonly| command
to the present needs. This easily leaves the main file in a messy state.
\item
The generated document will always carry the filename
of the main document. This is inconvenient if
several child files are to be compiled and
to be kept for distribution.
\end{itemize}

The present package provides a simple interface
to make child files individually compilable by \LaTeX{}.
Compiling a child file then has the same effect as compiling
the main file with an |\includeonly| command
to select the appropriate child.
Moreover the generated document will carry the name of the child
rather than the main file.
This resolves all three above issues.

This feature is meant to make the editing of books,
thesis documents and lecture notes somewhat more convenient.
However, the package can also be used efficiently for
composing a series of documents (such as exercise sheets)
which are typically distributed individually.
It then assists the author in generating the individual documents
(potentially in different versions)
as well as a document containing the collected series.
Another application is in developing style files
or other kinds of included material
where compilation of the style file could redirect
to a sample or test file.

%%%%%%%%%%%%%%%%%%%%%%%%%%%%%%%%%%%%%%%%%%%%%%%%%%%%%%%%%%%%%%%%%%%%%%%%%%%%%%%%
%%%%%%%%%%%%%%%%%%%%%%%%%%%%%%%%%%%%%%%%%%%%%%%%%%%%%%%%%%%%%%%%%%%%%%%%%%%%%%%%
\section{Usage}

First of all, the package \textsf{childdoc} is \emph{not} a standard
\LaTeXe{} |.sty| style file! Therefore it needs to be invoked in
a non-standard way.

%%%%%%%%%%%%%%%%%%%%%%%%%%%%%%%%%%%%%%%%%%%%%%%%%%%%%%%%%%%%%%%%%%%%%%%%%%%%%%%%
\subsection{Included Files}
\label{sec:include}

%%%%%%%%%%%%%%%%%%%%%%%%%%%%%%%%%%%%%%%%
\DescribeMacro{\childdocmain}
To use the package, add the commands
\begin{center}
\begin{tabular}{l}
|\input{childdoc.def}|\\
|\childdocmain{}|\\
\end{tabular}
\end{center}
at the very top of the main \LaTeX{} file,
in particular \emph{before} the |\documentclass| statement!
The argument of |\childdocmain| should be left empty
(but it must be present).

%%%%%%%%%%%%%%%%%%%%%%%%%%%%%%%%%%%%%%%%
\DescribeMacro{\childdocof}
Furthermore, add the commands
\begin{center}
\begin{tabular}{l}
|\input{childdoc.def}|\\
|\childdocof{|\textit{main}|}|\\
\end{tabular}
\end{center}
at the top of every child file \textit{child}
which is included by |\include{|\textit{child}|}|
from within the main file
(or at least for those files to be compiled individually).
The argument \textit{main} must be the filename of the main file.

There are a couple of
considerations in setting up the main and child documents:

%%%%%%%%%%%%%%%%%%%%%%%%%%%%%%%%%%%%%%%%
\paragraph{Restrictions.}

Please note the following restrictions:
\begin{itemize}
\item
|\childdocmain| must be called with one argument \textit{main}
to ensure compatibility with earlier version of the package.
It must either be empty (|\childdocmain{}|)
or precisely match the filename of the main file in which it is specified.
See \secref{sec:detection} for further information.
\item
The filename \textit{main} must be specified without the |.tex| extension.
\item
The filename \textit{main} is case sensitive
(even in case-insensitive file systems)
due to internal string comparison.
\item
The argument \textit{main} should be fully expanded, it cannot be a macro.
\item
Subdirectories and special characters should be avoided in filenames.
\item
The command |\childdocmain{|\textit{main}|}| must be followed by a whitespace.
It should not be followed immediately by another command
or by a comment mark `|%|'.
This is because the \TeX{} parser reads the token immediately following
the argument of |\childdocmain| and puts it
at the beginning of every child section;
however, a white\-space is ignored.
\end{itemize}

%%%%%%%%%%%%%%%%%%%%%%%%%%%%%%%%%%%%%%%%
\paragraph{Content of Main File.}

It is advisable to place all content in the child files included by |\include|.
Any output contained in the main file will appear in all child documents
unless suppressed manually;
it cannot be suppressed automatically by the |\includeonly| directive
and thus should normally be avoided.
A method to include some content in the main file
by means of conditional processing is described in \secref{sec:conditional}.

%%%%%%%%%%%%%%%%%%%%%%%%%%%%%%%%%%%%%%%%
\paragraph{Page Numbering.}

When only a part of the document is compiled,
the appropriate numbering of pages
(as well as other status parameters)
is determined from the |.aux| files.
The latter contain information from previous passes.
However this information needs to propagate through
all intermediate child documents.
Therefore the page numbering in child documents may well
be inconsistent until the complete document is compiled at least once.

A useful (if unconventional) way to always ensure a consistent
page numbering is to restart the numbering in each child document
and denote the pages by `\textit{child}|.|\textit{page}'
where \textit{child} represents the chapter/section number of the child file.
This can be achieved by the command
|\numberwithin{page}{|\textit{child}|}|
of the \textsf{amsmath} package
where \textit{child} can be |chapter| or |section|
depending on the chosen structuring.
Alternatively, one can modify the macro |\thepage| appropriately
and reset the counter |page| at the start of each child file.

%%%%%%%%%%%%%%%%%%%%%%%%%%%%%%%%%%%%%%%%%%%%%%%%%%%%%%%%%%%%%%%%%%%%%%%%%%%%%%%%
\subsection{Conditional Processing}
\label{sec:conditional}

The package provides a mechanism to compile different versions
of a document. To customise the versions further some conditional processing
can come in handy to distinguish which version is being compiled.
The package provides two macros to describe the compilation context:

%%%%%%%%%%%%%%%%%%%%%%%%%%%%%%%%%%%%%%%%
\DescribeMacro{\ifchilddoc}
The conditional |\ifchilddoc| distinguishes between the compilation of
child documents and the main document:
%
\begin{center}
|\ifchilddoc |\textit{child-code}| |[|\||else |\textit{main-code}]| \||fi|
\end{center}

%%%%%%%%%%%%%%%%%%%%%%%%%%%%%%%%%%%%%%%%
\DescribeMacro{\childdocname}
\DescribeMacro{\childdocjob}
The macro |\childdocname| contains the filename (without extension)
of the main or child file being processed.
Note that |\childdocjob| will always contain the name of the main file.

%%%%%%%%%%%%%%%%%%%%%%%%%%%%%%%%%%%%%%%%
\paragraph{Title Page.}

Conditional processing can be used to include a title or banner page
in the main document when proper precautions are taken.
Importantly, the code in the main file should ensure that the page counter
(as well as other status parameters which are stored in the |.aux| files)
takes the same value after the conditional processing.
Otherwise the page numbers may take divergent values
depending on which part is compiled.

For example, a title page could be declared by:
%
\begin{center}
\begin{tabular}{l}
|\ifchilddoc\||else|\\
|\addtocounter{page}{-1}|\\
\textit{code for title page}\\
|\newpage|\\
|\||fi|
\end{tabular}
\end{center}
%
A banner page for the child documents can be generated by:
%
\begin{center}
\begin{tabular}{l}
|\ifchilddoc|\\
|\addtocounter{page}{-1}|\\
\textit{code for banner page}\\
|\newpage|\\
|\||fi|
\end{tabular}
\end{center}
%
Here one could write a message such as:
\begin{center}
|This is the part \childdocname{} of \childdocjob{}.|
\end{center}

%%%%%%%%%%%%%%%%%%%%%%%%%%%%%%%%%%%%%%%%%%%%%%%%%%%%%%%%%%%%%%%%%%%%%%%%%%%%%%%%
\subsection{Flags}
\label{sec:flags}

The package makes it easy to generate different versions
of the main or child documents.
To this end compilation flags can be defined
and assigned different default values.
They will be particularly useful in conjunction
with the forwarding mechanism described in \secref{sec:forward}.

For example, it may be useful to have a flag |\version|
which can be set to |draft| or |final|.
The document source will contain some conditional code
depending on the value of |\version|.
Suppose further, the flag should default to |final| for the main file
and to |draft| for child files
which is a natural assignment for editing the document.
This is achieved by placing the following code
in the preamble of the main document
(below the |\childdocmain| directive):
%
\begin{center}
\begin{tabular}{l}
|\ifchilddoc|\\
|\providecommand{\version}{draft}|\\
|\||else|\\
|\providecommand{\version}{final}|\\
|\||fi|
\end{tabular}
\end{center}
%
The definition by |\providecommand| makes sure
that previous definitions are not overwritten.
Further statements |\providecommand{\version}{...}|
can thus be added before the above code to override it.

For the main file, one might add a line
(between |\childdocmain| and the above block)
%
\begin{center}
|%\ifchilddoc\||else\providecommand{\version}{draft}\||fi|
\end{center}
%
which can be uncommented to produce a draft version.
Likewise one can add a line to the very top of a child file
(above the |\childdocof{|\textit{main}|}| directive)
%
\begin{center}
|%\providecommand{\version}{final}|
\end{center}
%
which can be uncommented to produce the final version of this child document.

%%%%%%%%%%%%%%%%%%%%%%%%%%%%%%%%%%%%%%%%%%%%%%%%%%%%%%%%%%%%%%%%%%%%%%%%%%%%%%%%
\subsection{Forwarding}
\label{sec:forward}

Different versions of the main or child documents
using compilation flags as described in \secref{sec:flags}
can be (permanently) stored in different files
for convenient compilation, viewing and distribution.
To this end, the package defines a command
to pass on compilation to a different file:

%%%%%%%%%%%%%%%%%%%%%%%%%%%%%%%%%%%%%%%%
\DescribeMacro{\childdocforward}
The command |\childdocforward| redirects processing to
another source file:
%
\begin{center}
\begin{tabular}{l}
|\input{childdoc.def}|\\
|\childdocforward[|\textit{main}|]{|\textit{dest}|}|\\
\end{tabular}
\end{center}
%
The argument \textit{dest} is the destination file
(without extension).
It should be the main file or one of the child files.
Note that further \textsf{childdoc} directives
such as |\childdocof| and |\childdocforward|
in the indicated file will be processed in this form.
The optional argument \textit{main}
passes on directly to the main file \textit{main}
while pretending to compile the child \textit{dest}.
This form behaves as if \textit{dest}
issues |\childdocof{|\textit{main}|}| right away,
and no further \textsf{childdoc} directives will be processed.

%%%%%%%%%%%%%%%%%%%%%%%%%%%%%%%%%%%%%%%%
\DescribeMacro{\...prefix}
In the alternative form |\childdocforwardprefix|,
%
\begin{center}
\begin{tabular}{l}
|\input{childdoc.def}|\\
|\childdocforwardprefix[|\textit{main}|]{|\textit{prefix}|}{|\textit{dest}|}|
\end{tabular}
\end{center}
%
the destination file is determined by a pattern
depending on the current file:
To make this work, the current file must be called
`{\textit{prefix}\hspace{0.2em}\textit{suffix}}'
with \textit{prefix} matching precisely the argument.
Processing is then passed on to the file
`{\textit{dest}\hspace{0.2em}\textit{suffix}}'.
Surely, the same effect is achieved by
directly specifying the
argument `{\textit{dest}\hspace{0.2em}\textit{suffix}}'
in the first form.
However, that requires to set up a different file
for each child. With the alternative form of the command
all these files can have exactly the same content
which simplifies setting them up and maintaining them.

For example, the following file |draft.tex|
with a compilation flag |\version| as described in \secref{sec:flags}
compiles the main document as a draft:
%
\begin{center}
\begin{tabular}{l}
|\def\version{draft}|\\
|\input{childdoc.def}|\\
|\childdocforward{|\textit{main}|}|
\end{tabular}
\end{center}
%
Likewise, the following files |final|\textit{nn}|.tex|
compile the final version of the child document
|child|\textit{nn}|.tex|:
%
\begin{center}
\begin{tabular}{l}
|\def\version{final}|\\
|\input{childdoc.def}|\\
|\childdocforwardprefix{final}{child}|
\end{tabular}
\end{center}
%

Note that when several versions of a main file and/or of each child file
are to be generated, it may be convenient to set up a |Makefile| or
shell script to automatise the process.

%%%%%%%%%%%%%%%%%%%%%%%%%%%%%%%%%%%%%%%%%%%%%%%%%%%%%%%%%%%%%%%%%%%%%%%%%%%%%%%%
\subsection{Command Line Processing}
\label{sec:commandline}

The effect of redirection files can also be achieved by invoking
the \LaTeX{} compiler with a more elaborate command line.
Most conveniently this should be done as part
of a shell script or a |Makefile|.

When using \textsf{childdoc} in the main file, the following
command lines effectively perform a redirection
(note that depending on the shell being used,
backslashes may have to be doubled: `|\|' $\to$ `|\\|'):
%
\begin{center}
|... -jobname "|\textit{target}|" |\\|"|[\textit{flags}]%
|\input{childdoc.def}\childdocforward[|\textit{main}|]{|\textit{dest}|}"|
\end{center}
%
Here \textit{target} is the name of the output file,
\textit{main} is the name of the main file
and \textit{dest} is the name of the main or child file to be processed
(all filenames without extensions).
The optional argument \textit{main} can be omitted
if \textit{main} matches \textit{dest}.
Optionally, compilation \textit{flags} can be defined via |\def| commands.
This command line makes the \TeX{} engine believe
it is compiling the file \textit{target}
whose content is specified as the latter parameter.
The provided code then forwards the processing to
\textit{main} or \textit{dest} as described in \secref{sec:forward}.

%%%%%%%%%%%%%%%%%%%%%%%%%%%%%%%%%%%%%%%%%%%%%%%%%%%%%%%%%%%%%%%%%%%%%%%%%%%%%%%%
\subsection{Include by Input}
\label{sec:input}

Including child documents by |\include| has some restrictions by design.
Most notably, the content of a child document always occupies
its own set of pages; pages cannot be shared between child documents.
Usually, this behaviour makes perfect sense
because each child document contain an essential part of the document.
However, in some situations it may be desirable to compose
a document from a collection of parts
without having mandatory page breaks between then.
For this case, the package
provides a mechanism to include parts
by |\input| which can also be processed individually.
However, by construction this mechanism
requires manual handling of the content to be output.

%%%%%%%%%%%%%%%%%%%%%%%%%%%%%%%%%%%%%%%%
\DescribeMacro{\ifchilddocmanual}
The main file should be prepared as usual, see \secref{sec:include}.
However, the document body must make a distinction
between processing of an individual part and of the main document, e.g.:
%
\begin{center}
\begin{tabular}{l}
|\ifchilddocmanual|\\
|\input{\childdocname}|\\
|\||else|\\
\textit{document body with }|\input{|\textit{part}|}|\\
|\||fi|
\end{tabular}
\end{center}
%
The conditional |\ifchilddocmanual| is true whenever
a part to be included by |\input| is being compiled,
and the name of the part is stored in |\childdocname|.

%%%%%%%%%%%%%%%%%%%%%%%%%%%%%%%%%%%%%%%%
\DescribeMacro{\childdocby}
Each part to be included by |\input| should start with:
%
\begin{center}
\begin{tabular}{l}
|\input{childdoc.def}|\\
|\childdocby{|\textit{main}|}|\\
\end{tabular}
\end{center}
%
The directive |\childdocby| is similar to |\childdocof|
described in \secref{sec:include},
but the subsequent selection of content must be done manually.
To that end, both |\ifchilddoc| and |\ifchilddocmanual|
will be true upon processing of a part,
and the name of the part is stored in |\childdocname|.
Note that |\jobname| will be set to the filename of the current part
so that each part receives an individual |.aux| file
that does not interfere with the |.aux| file(s) of the main document.
This behaviour can be altered by the alternative form
|\childdocby[*]{|\textit{main}|}| (with a non-empty optional argument)
which uses the |.aux| file of the main document
by setting |\jobname| to \textit{main}.

%%%%%%%%%%%%%%%%%%%%%%%%%%%%%%%%%%%%%%%%%%%%%%%%%%%%%%%%%%%%%%%%%%%%%%%%%%%%%%%%
\subsection{Driver Development}
\label{sec:driver}

The \textsf{childdoc} mechanism can also be use for the development
of definition files such as \LaTeX{} styles or classes.
This case differs from the above setup with multiple parts
included by |\include| in that no |\includeonly| should be invoked.
This can be achieved by starting the include file
(before |\ProvidesPackage|) with:
%
\begin{center}
\begin{tabular}{l}
|\input{childdoc.def}|\\
|\childdocforward{|\textit{main}|}|\\
\end{tabular}
\end{center}
%
or alternatively with:
%
\begin{center}
\begin{tabular}{l}
|\input{childdoc.def}|\\
|\childdocby{|\textit{main}|}|\\
\end{tabular}
\end{center}
%
Both forms have slightly different effects as described above.
The main file is prepared as usual, see \secref{sec:include}.

%%%%%%%%%%%%%%%%%%%%%%%%%%%%%%%%%%%%%%%%%%%%%%%%%%%%%%%%%%%%%%%%%%%%%%%%%%%%%%%%
\subsection{Legacy Detection}
\label{sec:detection}

The directive |\childdocmain| in the main file can detect
whether the complete document or merely a child is to be compiled
even without using the directive |\childdocof|.
This method is deprecated because it is less robust
and there is no compelling reason to use it;
it is merely provided for backward compatibility
and it may be removed in future versions.

If the detection mechanism is to be used,
it is mandatory to correctly specify
the filename of the main file as the argument of |\childdocmain|:
%
\begin{center}
\begin{tabular}{l}
|\input{childdoc.def}|\\
|\childdocmain{|\textit{main}|}|\\
\end{tabular}
\end{center}
%
If |\jobname| does not match the argument \textit{main} of |\childdocmain|,
it is assumed that |\jobname| points to the child file to be compiled.
When using |\childdocmain| with the main file specified as argument,
it suffices to start a child file
with just |\input{|\textit{main}|}|
without loading of the package and using |\childdocof|.
If instead all processing is done
with the appropriate \textsf{childdoc} directives,
the argument of \textit{main} of |\childdocmain| can be empty.

An alternative version of the command line processing described
in \secref{sec:commandline} using the detection mechanism reads:
%
\begin{center}
|... -jobname "|\textit{target}|" "|[\textit{flags}]%
[|\def\jobname{|\textit{dest}|}|]|\input{|\textit{main}|}"|
\end{center}

%%%%%%%%%%%%%%%%%%%%%%%%%%%%%%%%%%%%%%%%%%%%%%%%%%%%%%%%%%%%%%%%%%%%%%%%%%%%%%%%
\subsection{Manual Code}
\label{sec:manual}

In case one cannot be certain whether the definitions file |childdoc.def|
is installed on the target \TeX{} distribution
and one prefers not to ship it,
it is conceivable to paste a few relevant commands into the sources.

To that end, drop all statements |\input{childdoc.def}|
and perform the replacements as outlined below.
Instead of |\childdocmain{|\textit{main}|}| add the following code
to the top of the main file:
%
\begin{center}
\begin{tabular}{l}
|\||ifdefined\childdocname\endinput\||fi\newif\ifchilddoc|\\
|\edef\childdocname{\scantokens\expandafter{\jobname\noexpand}}|\\
|\def\childdocmain{|\textit{main}|}\||ifx\childdocmain\childdocname\||else|\\
|\childdoctrue\includeonly{\childdocname}\let\jobname\childdocmain\||fi|\\
\end{tabular}
\end{center}
%
Instead of |\childdocof{|\textit{main}|}| just include the main file
at the top of each child file:
%
\begin{center}
|\input{|\textit{main}|}|
\end{center}
%
A simple redirection |\childdocforward{|\textit{dest}|}| is achieved by:
%
\begin{center}
|\def\jobname{|\textit{dest}|}\input{\jobname}|
\end{center}
%
The redirection with prefix
|\childdocforwardprefix[|\textit{prefix}|]{|\textit{dest}|}|
is accomplished by:
%
\begin{center}
\begin{tabular}{l}
|{\edef\jobname{\scantokens\expandafter{\jobname\noexpand}}|\\
|\def\redirectjob |\textit{prefix}|#1~~~{\gdef\jobname{|\textit{dest}|#1}}|\\
|\expandafter\redirectjob\jobname~~~}\input{\jobname}|
\end{tabular}
\end{center}

In an alternative approach,
child documents can be compiled by a specific command line
without additional code or specific definitions:
%
\begin{center}
|... -jobname "|\textit{target}|" "|[\textit{flags}]%
|\includeonly{|\textit{dest}|}\input{|\textit{main}|}"|
\end{center}
%

%%%%%%%%%%%%%%%%%%%%%%%%%%%%%%%%%%%%%%%%%%%%%%%%%%%%%%%%%%%%%%%%%%%%%%%%%%%%%%%%
%%%%%%%%%%%%%%%%%%%%%%%%%%%%%%%%%%%%%%%%%%%%%%%%%%%%%%%%%%%%%%%%%%%%%%%%%%%%%%%%
\section{Information}

%%%%%%%%%%%%%%%%%%%%%%%%%%%%%%%%%%%%%%%%%%%%%%%%%%%%%%%%%%%%%%%%%%%%%%%%%%%%%%%%
\subsection{Copyright}

Copyright \copyright{} 2017--2018 Niklas Beisert

This work may be distributed and/or modified under the
conditions of the \LaTeX{} Project Public License, either version 1.3
of this license or (at your option) any later version.
The latest version of this license is in
  \url{http://www.latex-project.org/lppl.txt}
and version 1.3 or later is part of all distributions of \LaTeX{}
version 2005/12/01 or later.

This work has the LPPL maintenance status `maintained'.

The Current Maintainer of this work is Niklas Beisert.

This work consists of the files |README.txt|, |childdoc.ins| and |childdoc.dtx|
as well as the derived files |childdoc.def|, |cdocsamp.tex|
with |cdocsch1.tex|, |cdocsch2.tex|, |cdocspt3.tex|, |cdocspt4.tex|,
|cdocsdrf.tex|, |cdocsfn1.tex|, |cdocsfn2.tex|
as well as |childdoc.pdf|.

%%%%%%%%%%%%%%%%%%%%%%%%%%%%%%%%%%%%%%%%%%%%%%%%%%%%%%%%%%%%%%%%%%%%%%%%%%%%%%%%
\subsection{Files and Installation}

The package consists of the files:
%
\begin{center}
\begin{tabular}{ll}
    |README.txt|   & readme file \\
    |childdoc.ins| & installation file \\
    |childdoc.dtx| & source file \\
    |childdoc.def| & definition file \\
    |cdocsamp.tex| & sample main file \\
    |cdocsch1.tex| & sample include file \\
    |cdocsch2.tex| & sample include file \\
    |cdocspt3.tex| & sample part file \\
    |cdocspt4.tex| & sample part file \\
    |cdocsdrf.tex| & sample redirection file \\
    |cdocsfn1.tex| & sample redirection file \\
    |cdocsfn2.tex| & sample redirection file \\
    |childdoc.pdf| & manual
\end{tabular}
\end{center}
%
The distribution consists of the files
|README.txt|, |childdoc.ins| and |childdoc.dtx|.
%
\begin{itemize}
\item
Run (pdf)\LaTeX{} on |childdoc.dtx|
to compile the manual |childdoc.pdf| (this file).
\item
Run \LaTeX{} on |childdoc.ins| to create the definitions file |childdoc.def|
and the sample |cdocsamp.tex| with include files
|cdocsch1.tex|, |cdocsch2.tex|, |cdocspt3.tex|, |cdocspt4.tex|,
|cdocsdrf.tex|, |cdocsfn1.tex|, |cdocsfn2.tex|.
Then copy the file |childdoc.def| to an appropriate directory of your \LaTeX{}
distribution, e.g.\ \textit{texmf-root}|/tex/latex/childdoc|.
\end{itemize}

%%%%%%%%%%%%%%%%%%%%%%%%%%%%%%%%%%%%%%%%%%%%%%%%%%%%%%%%%%%%%%%%%%%%%%%%%%%%%%%%
\subsection{Related CTAN Packages}

There are several other packages which offer a similar functionality:
%
\begin{itemize}
\item
The packages
\href{http://ctan.org/pkg/docmute}{\textsf{docmute}},
\href{http://ctan.org/pkg/includex}{\textsf{includex}} and
\href{http://ctan.org/pkg/standalone}{\textsf{standalone}}
provide commands to include only the document body of
a child file thus allowing both files to be compiled individually.
\item
The packages \href{http://ctan.org/pkg/subdocs}{\textsf{subdocs}}
and \href{http://ctan.org/pkg/subfiles}{\textsf{subfiles}}
provide structures in which the main and child documents can be
encapsulated and allowing them to be compiled individually.
The inclusion mechanism is different from the conventional |\include|.
\item
The package \href{http://ctan.org/pkg/combine}{\textsf{combine}}
is an elaborate solution to combine several documents into one.
\end{itemize}
%
See also the CTAN topic \href{http://ctan.org/topic/subdocs}{\textsf{subdocs}}
for further related packages.
The present package differs from the above solutions in that
a document structure constructed with the conventional |\include| mechanism
just needs two extra commands at the top of every file
such that all constituent files can be compiled individually.

%%%%%%%%%%%%%%%%%%%%%%%%%%%%%%%%%%%%%%%%%%%%%%%%%%%%%%%%%%%%%%%%%%%%%%%%%%%%%%%%
%\subsection{Feature Suggestions}
%
%The following is a list of features which may be useful for future
%versions of this package:
%%
%\begin{itemize}
%\item
%\ldots
%\end{itemize}

%%%%%%%%%%%%%%%%%%%%%%%%%%%%%%%%%%%%%%%%%%%%%%%%%%%%%%%%%%%%%%%%%%%%%%%%%%%%%%%%
\subsection{Revision History}

%%%%%%%%%%%%%%%%%%%%%%%%%%%%%%%%%%%%%%%%
\paragraph{v2.0:} 2018/12/30

\begin{itemize}
\item
immediate forward processing
\item
added |\childdocby| mechanism
\item
manual restructured
\end{itemize}

%%%%%%%%%%%%%%%%%%%%%%%%%%%%%%%%%%%%%%%%
\paragraph{v1.6:} 2018/01/17

\begin{itemize}
\item
application for development of include files
\item
corrections to manual
\end{itemize}

%%%%%%%%%%%%%%%%%%%%%%%%%%%%%%%%%%%%%%%%
\paragraph{v1.5:} 2017/05/21

\begin{itemize}
\item
more complete structuring introduced
\item
|\childdocof| introduced
\item
|\childdoc| renamed to |\childdocmain|
\item
|\childredirect| renamed to |\childdocforward| and |\childdocforwardprefix|
and functionality expanded
\end{itemize}

%%%%%%%%%%%%%%%%%%%%%%%%%%%%%%%%%%%%%%%%
\paragraph{v1.0:} 2017/04/27

\begin{itemize}
\item
manual and install package
\item
first version published on CTAN
\end{itemize}

%%%%%%%%%%%%%%%%%%%%%%%%%%%%%%%%%%%%%%%%
\paragraph{v0.6:} 2017/04/26

\begin{itemize}
\item
redirection mechanism added
\end{itemize}

%%%%%%%%%%%%%%%%%%%%%%%%%%%%%%%%%%%%%%%%
\paragraph{v0.5:} 2017/04/26

\begin{itemize}
\item
functionality in definition file
\end{itemize}


%%%%%%%%%%%%%%%%%%%%%%%%%%%%%%%%%%%%%%%%%%%%%%%%%%%%%%%%%%%%%%%%%%%%%%%%%%%%%%%%
%%%%%%%%%%%%%%%%%%%%%%%%%%%%%%%%%%%%%%%%%%%%%%%%%%%%%%%%%%%%%%%%%%%%%%%%%%%%%%%%
%%%%%%%%%%%%%%%%%%%%%%%%%%%%%%%%%%%%%%%%%%%%%%%%%%%%%%%%%%%%%%%%%%%%%%%%%%%%%%%%
\appendix

\settowidth\MacroIndent{\rmfamily\scriptsize 000\ }

 \DocInput{childdoc.dtx}

\end{document}
%</driver>
% \fi
%
% %%%%%%%%%%%%%%%%%%%%%%%%%%%%%%%%%%%%%%%%%%%%%%%%%%%%%%%%%%%%%%%%%%%%%%%%%%%%%%
% %%%%%%%%%%%%%%%%%%%%%%%%%%%%%%%%%%%%%%%%%%%%%%%%%%%%%%%%%%%%%%%%%%%%%%%%%%%%%%
% \section{Sample}
%\iffalse
%<*samplemain>
%\fi
%
% The following presents a sample document
% with two chapters, two parts, a title page,
% a compile flag as well as three forwarding files to set the flag.
% It consists of eight |.tex| files:
% \begin{center}
% \begin{tabular}{ll}
% |cdocsamp.tex|&main file\\
% |cdocsch1.tex|&include file for chapter 1\\
% |cdocsch2.tex|&include file for chapter 2\\
% |cdocspt3.tex|&include file for part 3\\
% |cdocspt4.tex|&include file for part 4\\
% |cdocsdrf.tex|&forwarding file for main file in draft mode\\
% |cdocsfi1.tex|&forwarding file for final version of chapter 1\\
% |cdocsfi2.tex|&forwarding file for final version of chapter 2\\
% \end{tabular}
% \end{center}
% Each of the eight files can be compiled directly by the \LaTeX{} compiler.
%
% %%%%%%%%%%%%%%%%%%%%%%%%%%%%%%%%%%%%%%
% \paragraph{Main File.}
%
% The main file is called |cdocsamp.tex|.
%
% Load the \textsf{childdoc} definitions and
% declare the filename for the main document:
%    \begin{macrocode}
\input{childdoc.def}
\childdocmain{}
%    \end{macrocode}

% Optional override for |\version| flag:
%    \begin{macrocode}
%%\ifchilddoc\else\providecommand{\version}{draft}\fi
%    \end{macrocode}

% Define the default values for the |\version| flag
% (|final| for the main file and |draft| for childs):
%    \begin{macrocode}
\ifchilddoc
\providecommand{\version}{draft}
\else
\providecommand{\version}{final}
\fi
%    \end{macrocode}

% Load the standard document class:
%    \begin{macrocode}
\documentclass[12pt]{article}
%    \end{macrocode}

% Start the document body:
%    \begin{macrocode}
\begin{document}
%    \end{macrocode}

% Declare a title page.
% Print title, part of document being processed and version flag:
%    \begin{macrocode}
\addtocounter{page}{-1}
\begin{center}
{\LARGE\bfseries{}childdoc example\par}
\vspace{1cm}
\ifchilddoc
\ifchilddocmanual part\else chapter\fi:
`\childdocname' of `\childdocjob'\par
\else
main document: `\childdocjob'\par
\fi
version: \version\par
\end{center}
\newpage
%    \end{macrocode}

% Manually include selected file,
% otherwise process as usual:
%    \begin{macrocode}
\ifchilddocmanual
\section*{part `\childdocname'}
\input{\childdocname}
\else
%    \end{macrocode}

% Include the two chapters:
%    \begin{macrocode}
\include{cdocsch1}
\include{cdocsch2}
%    \end{macrocode}

% Include the two parts unless only chapters should be displayed:
%    \begin{macrocode}
\ifchilddoc\else
\section{part three}
\input{cdocspt3}
\section{part four}
\input{cdocspt4}
\fi
%    \end{macrocode}

% Process as usual until here:
%    \begin{macrocode}
\fi
%    \end{macrocode}

% End of document body:
%    \begin{macrocode}
\end{document}
%    \end{macrocode}
%\iffalse
%</samplemain>
%\fi
%
% %%%%%%%%%%%%%%%%%%%%%%%%%%%%%%%%%%%%%%
% \paragraph{Chapter Include Files.}
%
% The include files are called |cdocsch1.tex| and |cdocsch2.tex|.
%
%\iffalse
%<*samplechap1|samplechap2>
%\fi

% Optional override for |\version| flag:
%    \begin{macrocode}
%%\providecommand{\version}{final}
%    \end{macrocode}

% Include the main document:
%    \begin{macrocode}
\input{childdoc.def}
\childdocof{cdocsamp}
%    \end{macrocode}

%\iffalse
%</samplechap1|samplechap2>
%\fi
%
%\iffalse
%<*samplechap1>
%\fi
% Some text for chapter 1:
%    \begin{macrocode}
\section{one}
some text in chapter one
%    \end{macrocode}

%\iffalse
%</samplechap1>
%\fi
% Some text for chapter 2:
%\iffalse
%<*samplechap2>
%\fi
%    \begin{macrocode}
\section{two}
more text in chapter two
%    \end{macrocode}

%\iffalse
%</samplechap2>
%\fi
%
% %%%%%%%%%%%%%%%%%%%%%%%%%%%%%%%%%%%%%%
% \paragraph{Part Include Files.}
%
% The include files are called |cdocspt3.tex| and |cdocspt4.tex|.
%
%\iffalse
%<*samplepart3|samplepart4>
%\fi

% Optional override for |\version| flag:
%    \begin{macrocode}
%%\providecommand{\version}{final}
%    \end{macrocode}

% Include the main document:
%    \begin{macrocode}
\input{childdoc.def}
\childdocby{cdocsamp}
%    \end{macrocode}

%\iffalse
%</samplepart3|samplepart4>
%\fi
%
%\iffalse
%<*samplepart3>
%\fi
% Some text for part 3:
%    \begin{macrocode}
some text in part three
%    \end{macrocode}

%\iffalse
%</samplepart3>
%\fi
% Some text for part 4:
%\iffalse
%<*samplepart4>
%\fi
%    \begin{macrocode}
more text in part four
%    \end{macrocode}

%\iffalse
%</samplepart4>
%\fi
%
% %%%%%%%%%%%%%%%%%%%%%%%%%%%%%%%%%%%%%%
% \paragraph{Forwarding for a Complete Draft.}
%
% The following forwarding file |cdocsdrf.tex|
% compiles the main document in draft mode:
%\iffalse
%<*sampledraft>
%\fi
%    \begin{macrocode}
\def\version{draft}
\input{childdoc.def}
\childdocforward{cdocsamp}
%    \end{macrocode}

%\iffalse
%</sampledraft>
%\fi
%
% %%%%%%%%%%%%%%%%%%%%%%%%%%%%%%%%%%%%%%
% \paragraph{Forwarding for Final Version of the Chapters.}
%
% The following forwarding files |cdocsfn1.tex| and |cdocsfn2.tex|
% (with identical content)
% compile the final versions of the child documents
% |cdocsch1.tex| and |cdocsch2.tex|, respectively:
%\iffalse
%<*samplefinal>
%\fi
%    \begin{macrocode}
\def\version{final}
\input{childdoc.def}
\childdocforwardprefix[cdocsamp]{cdocsfn}{cdocsch}
%    \end{macrocode}

%\iffalse
%</samplefinal>
%\fi
%
% %%%%%%%%%%%%%%%%%%%%%%%%%%%%%%%%%%%%%%
% \paragraph{Command Line Processing.}
%
% The following three command lines generate the output files
% |cdocscld|, |cdocscl1| and |cdocscl2|
% which should be identical to
% |cdocsdrf|, |cdocsch1| and |cdocsfn2|, respectively:
% \begin{center}
% \begin{tabular}{l}
% |latex -jobname cdocscld \|\\
% |  "\def\version{draft}\input{childdoc.def}\childdocforward{cdocsamp}"|\\
% |latex -jobname cdocscl1 \|\\
% |  "\input{childdoc.def}\childdocforward[cdocsamp]{cdocsch1}"|\\
% |latex -jobname cdocscl2 \|\\
% |  "\def\version{final}\input{childdoc.def}\childdocforward{cdocsch2}"|
% \end{tabular}
% \end{center}
% Note that the trailing backslash on each first line
% merely continues the input to the second line
% (for convenient cut ant paste).
% Furthermore, the command |latex| can be replaced by any
% of its alternative versions such as |pdflatex|.
%
% %%%%%%%%%%%%%%%%%%%%%%%%%%%%%%%%%%%%%%%%%%%%%%%%%%%%%%%%%%%%%%%%%%%%%%%%%%%%%%
% %%%%%%%%%%%%%%%%%%%%%%%%%%%%%%%%%%%%%%%%%%%%%%%%%%%%%%%%%%%%%%%%%%%%%%%%%%%%%%
% \section{Implementation}
%\iffalse
%<*package>
%\fi
%
% This section describes the definitions file |childdoc.def|.

% The definitions cannot be loaded using |\usepackage| or |\RequirePackage|
% which has a mechanism to prevent loading a style file more than once.
% When loading the definitions by means of |\input|
% multiple instances have to be prevented manually:
%\iffalse
%This code needs to be before the `\ProvidesFile' directive
%which is defined at the beginning of this file.
%Therefore it is also placed there and commented out here.
%</package>
%<*discard>
%\fi
%    \begin{macrocode}
\ifdefined\childdocmain\endinput\fi
%    \end{macrocode}
%\iffalse
%</discard>
%<*package>
%\fi
%
% \macro{\ifchilddoc}
% \macro{\ifchilddocmanual}
% The conditional |\ifchilddoc| tells whether a
% child (true) or main (false) document is being compiled.
% The conditional |\ifchilddocmanual| tells whether
% the |\includeonly| mechanism is used (false) or
% the selection of child files must be performed manually (true).
% The definitions initialise to false:
%    \begin{macrocode}
\newif\ifchilddoc
\newif\ifchilddocmanual
%    \end{macrocode}

% \macro{\childdocname}
% \macro{\childdocjob}
% The macro |\childdocname| stores the name of the main document
% to be compiled. The macro |\childdocjob| stores the name of
% the document on which the \LaTeX{} compiler was originally invoked.
% The content of |\jobname| cannot be compared
% to filenames specified in the source due to different catcodes.
% The following code rescans |\jobname|, stores the result
% in |\childdocname| and saves a copy in |\childdocjob|:
%    \begin{macrocode}
\edef\childdocname{\scantokens\expandafter{\jobname\noexpand}}
\let\childdocjob\childdocname
%    \end{macrocode}

% \macro{\childdocdisable}
% The macro |\childdocdisable| prevents the main file
% from being processed more than once.
% At this stage, the main document command |\childdocmain|
% is assumed to be called once again where it should do nothing.
% Any subsequent call to it should prevent
% a secondary processing of the main document
% It overwrites the forwarding commands
% |\childdocof| and |\childdocforward|
% with empty macros to prevent further inclusions of the main document:
%    \begin{macrocode}
\newcommand{\childdocdisable}
{
  \renewcommand{\childdocmain}[1]{\renewcommand{\childdocmain}[1]{\endinput}}
  \renewcommand{\childdocof}[1]{}
  \renewcommand{\childdocby}[2][]{}
  \renewcommand{\childdocforward}[2][]{}
  \renewcommand{\childdocdisable}{}
}
%    \end{macrocode}

% \macro{\childdocmain}
% The macro |\childdocmain| is to be called at the top of the main file
% with nothing or the main filename (without extension) as argument.
% First, it breaks loops.
% If the argument is not empty and does not match |\childdocname|
% (which is set by the first inclusion of |childdoc.def|),
% |\ifchilddoc| is set to true, |\includeonly| is applied to the child file
% and |\jobname| is set to the main file
% (for proper handling of |.aux| files):
%    \begin{macrocode}
\newcommand{\childdocmain}[1]
{
  \childdocdisable\childdocmain{}
  \if?#1?\else
    \begingroup
      \def\childdoctmp{#1}
      \ifx\childdoctmp\childdocname
        \def\childdoctmp{}
      \else
        \def\childdoctmp
        {
          \childdoctrue
          \includeonly{\childdocname}
          \def\childdocjob{#1}
          \def\jobname{#1}
        }
      \fi
      \expandafter
    \endgroup
    \childdoctmp
  \fi
}
%    \end{macrocode}

% \macro{\childdocof}
% The command |\childdocof| redirects
% compilation to the main file |#1|.
%    \begin{macrocode}
\newcommand{\childdocof}[1]
{
  \childdocdisable
  \childdoctrue
  \includeonly{\childdocname}
  \def\jobname{#1}
  \def\childdocjob{#1}
  \input{#1}
}
%    \end{macrocode}

% \macro{\childdocby}
% The command |\childdocby| ....
%    \begin{macrocode}
\newcommand{\childdocby}[2][]
{
  \childdocdisable
  \childdoctrue
  \childdocmanualtrue
  \if?#1?\else
    \def\jobname{#2}
  \fi
  \def\childdocjob{#2}
  \input{#2}
  \endinput
}
%    \end{macrocode}

% \macro{\childdocforward}
% The command |\childdocforward| redirects
% compilation to the main file or
% (if the optional argument is given) a child file.
% Parameters are set as if the main file
% or a child file starting with |\childdocof| was compiled.
% Then compilation is handed over to the main file:
%    \begin{macrocode}
\newcommand{\childdocforward}[2][]
{
  \begingroup
    \if?#1?
      \def\childdoctmp
      {
        \def\childdocname{#2}
        \def\childdocjob{#2}
        \def\jobname{#2}
        \input{#2}
        \endinput
      }
    \else
      \def\childdoctmp
      {
        \childdocdisable
        \def\childdocname{#2}
        \childdoctrue
        \includeonly{#2}
        \def\childdocjob{#1}
        \def\jobname{#1}
        \input{#1}
        \endinput
      }
    \fi
    \expandafter
  \endgroup
  \childdoctmp
}
%    \end{macrocode}

% \macro{\childdocforwardprefix}
% The command |\childdocforwardprefix| redirects
% compilation to the main or a child file by means of a pattern.
% The prefix |#1| in the current filename is replaced by |#2|
% and the suffix of the current filename is kept
% (it is assumed that the filename does not contain the substring `|~~~|'
% which is used as a delimiter).
% Compilation is handed over to the new file by |\childdocforward|:
%    \begin{macrocode}
\newcommand{\childdocforwardprefix}[3][]
{
  \begingroup
    \def\childdocextract #2##1~~~{\def\childdoctmp{\childdocforward[#1]{#3##1}}}
    \expandafter\childdocextract\childdocname~~~
    \expandafter
  \endgroup
  \childdoctmp
}
%    \end{macrocode}

% \macro{\childdoc}
% The deprecated macro |\childdoc| is a legacy version of |\childdocmain|:
%    \begin{macrocode}
\newcommand{\childdoc}{\childdocmain}
%    \end{macrocode}

% \macro{\childdocredirect}
% The deprecated macro |\childdocredirect| is a legacy version
% of |\childdocforward| and |\childdocforwardprefix|:
%    \begin{macrocode}
\newcommand{\childdocredirect}[2][]
{
  \begingroup
    \if?#1?
      \def\childdoctmp{\childdocforward{#2}}
    \else
      \def\childdoctmp{\childdocforwardprefix{#1}{#2}}
    \fi
    \expandafter
  \endgroup
  \childdoctmp
}
%    \end{macrocode}

%\iffalse
%</package>
%\fi
%
\endinput

\childdocforward{cdocsamp}
%    \end{macrocode}

%\iffalse
%</sampledraft>
%\fi
%
% %%%%%%%%%%%%%%%%%%%%%%%%%%%%%%%%%%%%%%
% \paragraph{Forwarding for Final Version of the Chapters.}
%
% The following forwarding files |cdocsfn1.tex| and |cdocsfn2.tex|
% (with identical content)
% compile the final versions of the child documents
% |cdocsch1.tex| and |cdocsch2.tex|, respectively:
%\iffalse
%<*samplefinal>
%\fi
%    \begin{macrocode}
\def\version{final}
% \iffalse
%
% childdoc.dtx Copyright (C) 2017-2018 Niklas Beisert
%
% This work may be distributed and/or modified under the
% conditions of the LaTeX Project Public License, either version 1.3
% of this license or (at your option) any later version.
% The latest version of this license is in
%   http://www.latex-project.org/lppl.txt
% and version 1.3 or later is part of all distributions of LaTeX
% version 2005/12/01 or later.
%
% This work has the LPPL maintenance status `maintained'.
%
% The Current Maintainer of this work is Niklas Beisert.
%
% This work consists of the files childdoc.dtx and childdoc.ins
% and the derived files childdoc.def and cdocsamp.tex with
% cdocsch1.tex, cdocsch2.tex, cdocsdrf.tex, cdocsfn1.tex, cdocsfn2.tex.
%
%<package>\ifdefined\childdocmain\endinput\fi
%<package>\ProvidesFile{childdoc.def}[2018/12/30 v2.0 child document driver]
%<samplemain>\ProvidesFile{cdocsamp.tex}[2018/12/30 v2.0 sample for childdoc]
%<*driver>
%\ProvidesFile{childdoc.drv}[2018/12/30 v2.0 childdoc reference manual file]
\PassOptionsToClass{10pt,a4paper}{article}
\documentclass{ltxdoc}

\usepackage[margin=35mm]{geometry}
\usepackage{hyperref}
\usepackage{hyperxmp}
\usepackage[usenames]{color}

\hypersetup{colorlinks=true}
\hypersetup{pdfstartview=FitH}
\hypersetup{pdfpagemode=UseNone}
\hypersetup{pdfsource={}}
\hypersetup{pdflang={en-UK}}
\hypersetup{pdfcopyright={Copyright 2017-2018 Niklas Beisert.
  This work may be distributed and/or modified under the
  conditions of the LaTeX Project Public License, either version 1.3
  of this license or (at your option) any later version.}}
\hypersetup{pdflicenseurl={http://www.latex-project.org/lppl.txt}}
\hypersetup{pdfcontactaddress={ETH Zurich, ITP, HIT K,
  Wolfgang-Pauli-Strasse 27}}
\hypersetup{pdfcontactpostcode={8093}}
\hypersetup{pdfcontactcity={Zurich}}
\hypersetup{pdfcontactcountry={Switzerland}}
\hypersetup{pdfcontactemail={nbeisert@itp.phys.ethz.ch}}
\hypersetup{pdfcontacturl={http://people.phys.ethz.ch/\xmptilde nbeisert/}}

\newcommand{\secref}[1]{\hyperref[#1]{section \ref*{#1}}}

\parskip1ex
\parindent0pt
\let\olditemize\itemize
\def\itemize{\olditemize\parskip0pt}

\begin{document}

\title{The \textsf{childdoc} Package}
\hypersetup{pdftitle={The childdoc Package}}
\author{Niklas Beisert\\[2ex]
  Institut f\"ur Theoretische Physik\\
  Eidgen\"ossische Technische Hochschule Z\"urich\\
  Wolfgang-Pauli-Strasse 27, 8093 Z\"urich, Switzerland\\[1ex]
  \href{mailto:nbeisert@itp.phys.ethz.ch}
  {\texttt{nbeisert@itp.phys.ethz.ch}}}
\hypersetup{pdfauthor={Niklas Beisert}}
\hypersetup{pdfsubject={Manual for the LaTeX2e Package childdoc}}
\date{30 December 2018, \textsf{v2.0}}
\maketitle

\begin{abstract}\noindent
\textsf{childdoc} is a \LaTeXe{} package
that enables the direct compilation
of document sections included by |\include|
to individual files.
\end{abstract}

\begingroup
\parskip0ex
\tableofcontents
\endgroup

%%%%%%%%%%%%%%%%%%%%%%%%%%%%%%%%%%%%%%%%%%%%%%%%%%%%%%%%%%%%%%%%%%%%%%%%%%%%%%%%
%%%%%%%%%%%%%%%%%%%%%%%%%%%%%%%%%%%%%%%%%%%%%%%%%%%%%%%%%%%%%%%%%%%%%%%%%%%%%%%%
\section{Introduction}

\LaTeX{} provides a mechanism to structure a large document (such as a book)
into a main file and several child files (containing the chapters)
using the |\include| command.
This mechanism is beneficial for documents
which span hundreds of pages in order to
make the source file(s) more manageable.
Moreover, compilation can be restricted to
selected child files by means of the |\includeonly| command.
The latter feature can be used to reduce the compilation time while editing
(this was significantly more useful in the earlier days of \LaTeX{})
or to generate a smaller document which is easier to navigate.
Another application of |\includeonly| is to generate
documents consisting of selected parts of the complete document.

However, there are a few drawbacks of the plain |\include| mechanism:
\begin{itemize}
\item
The child files cannot be compiled on their own,
they can only be compiled via the main file.
A naive editing environment
(such as a text editor with an option
to have the current file processed by \LaTeX)
may require one to switch to the main file before compiling;
attempting to compile the child file produces errors.
\item
The main file must be modified (each time)
to adjust the |\includeonly| command
to the present needs. This easily leaves the main file in a messy state.
\item
The generated document will always carry the filename
of the main document. This is inconvenient if
several child files are to be compiled and
to be kept for distribution.
\end{itemize}

The present package provides a simple interface
to make child files individually compilable by \LaTeX{}.
Compiling a child file then has the same effect as compiling
the main file with an |\includeonly| command
to select the appropriate child.
Moreover the generated document will carry the name of the child
rather than the main file.
This resolves all three above issues.

This feature is meant to make the editing of books,
thesis documents and lecture notes somewhat more convenient.
However, the package can also be used efficiently for
composing a series of documents (such as exercise sheets)
which are typically distributed individually.
It then assists the author in generating the individual documents
(potentially in different versions)
as well as a document containing the collected series.
Another application is in developing style files
or other kinds of included material
where compilation of the style file could redirect
to a sample or test file.

%%%%%%%%%%%%%%%%%%%%%%%%%%%%%%%%%%%%%%%%%%%%%%%%%%%%%%%%%%%%%%%%%%%%%%%%%%%%%%%%
%%%%%%%%%%%%%%%%%%%%%%%%%%%%%%%%%%%%%%%%%%%%%%%%%%%%%%%%%%%%%%%%%%%%%%%%%%%%%%%%
\section{Usage}

First of all, the package \textsf{childdoc} is \emph{not} a standard
\LaTeXe{} |.sty| style file! Therefore it needs to be invoked in
a non-standard way.

%%%%%%%%%%%%%%%%%%%%%%%%%%%%%%%%%%%%%%%%%%%%%%%%%%%%%%%%%%%%%%%%%%%%%%%%%%%%%%%%
\subsection{Included Files}
\label{sec:include}

%%%%%%%%%%%%%%%%%%%%%%%%%%%%%%%%%%%%%%%%
\DescribeMacro{\childdocmain}
To use the package, add the commands
\begin{center}
\begin{tabular}{l}
|\input{childdoc.def}|\\
|\childdocmain{}|\\
\end{tabular}
\end{center}
at the very top of the main \LaTeX{} file,
in particular \emph{before} the |\documentclass| statement!
The argument of |\childdocmain| should be left empty
(but it must be present).

%%%%%%%%%%%%%%%%%%%%%%%%%%%%%%%%%%%%%%%%
\DescribeMacro{\childdocof}
Furthermore, add the commands
\begin{center}
\begin{tabular}{l}
|\input{childdoc.def}|\\
|\childdocof{|\textit{main}|}|\\
\end{tabular}
\end{center}
at the top of every child file \textit{child}
which is included by |\include{|\textit{child}|}|
from within the main file
(or at least for those files to be compiled individually).
The argument \textit{main} must be the filename of the main file.

There are a couple of
considerations in setting up the main and child documents:

%%%%%%%%%%%%%%%%%%%%%%%%%%%%%%%%%%%%%%%%
\paragraph{Restrictions.}

Please note the following restrictions:
\begin{itemize}
\item
|\childdocmain| must be called with one argument \textit{main}
to ensure compatibility with earlier version of the package.
It must either be empty (|\childdocmain{}|)
or precisely match the filename of the main file in which it is specified.
See \secref{sec:detection} for further information.
\item
The filename \textit{main} must be specified without the |.tex| extension.
\item
The filename \textit{main} is case sensitive
(even in case-insensitive file systems)
due to internal string comparison.
\item
The argument \textit{main} should be fully expanded, it cannot be a macro.
\item
Subdirectories and special characters should be avoided in filenames.
\item
The command |\childdocmain{|\textit{main}|}| must be followed by a whitespace.
It should not be followed immediately by another command
or by a comment mark `|%|'.
This is because the \TeX{} parser reads the token immediately following
the argument of |\childdocmain| and puts it
at the beginning of every child section;
however, a white\-space is ignored.
\end{itemize}

%%%%%%%%%%%%%%%%%%%%%%%%%%%%%%%%%%%%%%%%
\paragraph{Content of Main File.}

It is advisable to place all content in the child files included by |\include|.
Any output contained in the main file will appear in all child documents
unless suppressed manually;
it cannot be suppressed automatically by the |\includeonly| directive
and thus should normally be avoided.
A method to include some content in the main file
by means of conditional processing is described in \secref{sec:conditional}.

%%%%%%%%%%%%%%%%%%%%%%%%%%%%%%%%%%%%%%%%
\paragraph{Page Numbering.}

When only a part of the document is compiled,
the appropriate numbering of pages
(as well as other status parameters)
is determined from the |.aux| files.
The latter contain information from previous passes.
However this information needs to propagate through
all intermediate child documents.
Therefore the page numbering in child documents may well
be inconsistent until the complete document is compiled at least once.

A useful (if unconventional) way to always ensure a consistent
page numbering is to restart the numbering in each child document
and denote the pages by `\textit{child}|.|\textit{page}'
where \textit{child} represents the chapter/section number of the child file.
This can be achieved by the command
|\numberwithin{page}{|\textit{child}|}|
of the \textsf{amsmath} package
where \textit{child} can be |chapter| or |section|
depending on the chosen structuring.
Alternatively, one can modify the macro |\thepage| appropriately
and reset the counter |page| at the start of each child file.

%%%%%%%%%%%%%%%%%%%%%%%%%%%%%%%%%%%%%%%%%%%%%%%%%%%%%%%%%%%%%%%%%%%%%%%%%%%%%%%%
\subsection{Conditional Processing}
\label{sec:conditional}

The package provides a mechanism to compile different versions
of a document. To customise the versions further some conditional processing
can come in handy to distinguish which version is being compiled.
The package provides two macros to describe the compilation context:

%%%%%%%%%%%%%%%%%%%%%%%%%%%%%%%%%%%%%%%%
\DescribeMacro{\ifchilddoc}
The conditional |\ifchilddoc| distinguishes between the compilation of
child documents and the main document:
%
\begin{center}
|\ifchilddoc |\textit{child-code}| |[|\||else |\textit{main-code}]| \||fi|
\end{center}

%%%%%%%%%%%%%%%%%%%%%%%%%%%%%%%%%%%%%%%%
\DescribeMacro{\childdocname}
\DescribeMacro{\childdocjob}
The macro |\childdocname| contains the filename (without extension)
of the main or child file being processed.
Note that |\childdocjob| will always contain the name of the main file.

%%%%%%%%%%%%%%%%%%%%%%%%%%%%%%%%%%%%%%%%
\paragraph{Title Page.}

Conditional processing can be used to include a title or banner page
in the main document when proper precautions are taken.
Importantly, the code in the main file should ensure that the page counter
(as well as other status parameters which are stored in the |.aux| files)
takes the same value after the conditional processing.
Otherwise the page numbers may take divergent values
depending on which part is compiled.

For example, a title page could be declared by:
%
\begin{center}
\begin{tabular}{l}
|\ifchilddoc\||else|\\
|\addtocounter{page}{-1}|\\
\textit{code for title page}\\
|\newpage|\\
|\||fi|
\end{tabular}
\end{center}
%
A banner page for the child documents can be generated by:
%
\begin{center}
\begin{tabular}{l}
|\ifchilddoc|\\
|\addtocounter{page}{-1}|\\
\textit{code for banner page}\\
|\newpage|\\
|\||fi|
\end{tabular}
\end{center}
%
Here one could write a message such as:
\begin{center}
|This is the part \childdocname{} of \childdocjob{}.|
\end{center}

%%%%%%%%%%%%%%%%%%%%%%%%%%%%%%%%%%%%%%%%%%%%%%%%%%%%%%%%%%%%%%%%%%%%%%%%%%%%%%%%
\subsection{Flags}
\label{sec:flags}

The package makes it easy to generate different versions
of the main or child documents.
To this end compilation flags can be defined
and assigned different default values.
They will be particularly useful in conjunction
with the forwarding mechanism described in \secref{sec:forward}.

For example, it may be useful to have a flag |\version|
which can be set to |draft| or |final|.
The document source will contain some conditional code
depending on the value of |\version|.
Suppose further, the flag should default to |final| for the main file
and to |draft| for child files
which is a natural assignment for editing the document.
This is achieved by placing the following code
in the preamble of the main document
(below the |\childdocmain| directive):
%
\begin{center}
\begin{tabular}{l}
|\ifchilddoc|\\
|\providecommand{\version}{draft}|\\
|\||else|\\
|\providecommand{\version}{final}|\\
|\||fi|
\end{tabular}
\end{center}
%
The definition by |\providecommand| makes sure
that previous definitions are not overwritten.
Further statements |\providecommand{\version}{...}|
can thus be added before the above code to override it.

For the main file, one might add a line
(between |\childdocmain| and the above block)
%
\begin{center}
|%\ifchilddoc\||else\providecommand{\version}{draft}\||fi|
\end{center}
%
which can be uncommented to produce a draft version.
Likewise one can add a line to the very top of a child file
(above the |\childdocof{|\textit{main}|}| directive)
%
\begin{center}
|%\providecommand{\version}{final}|
\end{center}
%
which can be uncommented to produce the final version of this child document.

%%%%%%%%%%%%%%%%%%%%%%%%%%%%%%%%%%%%%%%%%%%%%%%%%%%%%%%%%%%%%%%%%%%%%%%%%%%%%%%%
\subsection{Forwarding}
\label{sec:forward}

Different versions of the main or child documents
using compilation flags as described in \secref{sec:flags}
can be (permanently) stored in different files
for convenient compilation, viewing and distribution.
To this end, the package defines a command
to pass on compilation to a different file:

%%%%%%%%%%%%%%%%%%%%%%%%%%%%%%%%%%%%%%%%
\DescribeMacro{\childdocforward}
The command |\childdocforward| redirects processing to
another source file:
%
\begin{center}
\begin{tabular}{l}
|\input{childdoc.def}|\\
|\childdocforward[|\textit{main}|]{|\textit{dest}|}|\\
\end{tabular}
\end{center}
%
The argument \textit{dest} is the destination file
(without extension).
It should be the main file or one of the child files.
Note that further \textsf{childdoc} directives
such as |\childdocof| and |\childdocforward|
in the indicated file will be processed in this form.
The optional argument \textit{main}
passes on directly to the main file \textit{main}
while pretending to compile the child \textit{dest}.
This form behaves as if \textit{dest}
issues |\childdocof{|\textit{main}|}| right away,
and no further \textsf{childdoc} directives will be processed.

%%%%%%%%%%%%%%%%%%%%%%%%%%%%%%%%%%%%%%%%
\DescribeMacro{\...prefix}
In the alternative form |\childdocforwardprefix|,
%
\begin{center}
\begin{tabular}{l}
|\input{childdoc.def}|\\
|\childdocforwardprefix[|\textit{main}|]{|\textit{prefix}|}{|\textit{dest}|}|
\end{tabular}
\end{center}
%
the destination file is determined by a pattern
depending on the current file:
To make this work, the current file must be called
`{\textit{prefix}\hspace{0.2em}\textit{suffix}}'
with \textit{prefix} matching precisely the argument.
Processing is then passed on to the file
`{\textit{dest}\hspace{0.2em}\textit{suffix}}'.
Surely, the same effect is achieved by
directly specifying the
argument `{\textit{dest}\hspace{0.2em}\textit{suffix}}'
in the first form.
However, that requires to set up a different file
for each child. With the alternative form of the command
all these files can have exactly the same content
which simplifies setting them up and maintaining them.

For example, the following file |draft.tex|
with a compilation flag |\version| as described in \secref{sec:flags}
compiles the main document as a draft:
%
\begin{center}
\begin{tabular}{l}
|\def\version{draft}|\\
|\input{childdoc.def}|\\
|\childdocforward{|\textit{main}|}|
\end{tabular}
\end{center}
%
Likewise, the following files |final|\textit{nn}|.tex|
compile the final version of the child document
|child|\textit{nn}|.tex|:
%
\begin{center}
\begin{tabular}{l}
|\def\version{final}|\\
|\input{childdoc.def}|\\
|\childdocforwardprefix{final}{child}|
\end{tabular}
\end{center}
%

Note that when several versions of a main file and/or of each child file
are to be generated, it may be convenient to set up a |Makefile| or
shell script to automatise the process.

%%%%%%%%%%%%%%%%%%%%%%%%%%%%%%%%%%%%%%%%%%%%%%%%%%%%%%%%%%%%%%%%%%%%%%%%%%%%%%%%
\subsection{Command Line Processing}
\label{sec:commandline}

The effect of redirection files can also be achieved by invoking
the \LaTeX{} compiler with a more elaborate command line.
Most conveniently this should be done as part
of a shell script or a |Makefile|.

When using \textsf{childdoc} in the main file, the following
command lines effectively perform a redirection
(note that depending on the shell being used,
backslashes may have to be doubled: `|\|' $\to$ `|\\|'):
%
\begin{center}
|... -jobname "|\textit{target}|" |\\|"|[\textit{flags}]%
|\input{childdoc.def}\childdocforward[|\textit{main}|]{|\textit{dest}|}"|
\end{center}
%
Here \textit{target} is the name of the output file,
\textit{main} is the name of the main file
and \textit{dest} is the name of the main or child file to be processed
(all filenames without extensions).
The optional argument \textit{main} can be omitted
if \textit{main} matches \textit{dest}.
Optionally, compilation \textit{flags} can be defined via |\def| commands.
This command line makes the \TeX{} engine believe
it is compiling the file \textit{target}
whose content is specified as the latter parameter.
The provided code then forwards the processing to
\textit{main} or \textit{dest} as described in \secref{sec:forward}.

%%%%%%%%%%%%%%%%%%%%%%%%%%%%%%%%%%%%%%%%%%%%%%%%%%%%%%%%%%%%%%%%%%%%%%%%%%%%%%%%
\subsection{Include by Input}
\label{sec:input}

Including child documents by |\include| has some restrictions by design.
Most notably, the content of a child document always occupies
its own set of pages; pages cannot be shared between child documents.
Usually, this behaviour makes perfect sense
because each child document contain an essential part of the document.
However, in some situations it may be desirable to compose
a document from a collection of parts
without having mandatory page breaks between then.
For this case, the package
provides a mechanism to include parts
by |\input| which can also be processed individually.
However, by construction this mechanism
requires manual handling of the content to be output.

%%%%%%%%%%%%%%%%%%%%%%%%%%%%%%%%%%%%%%%%
\DescribeMacro{\ifchilddocmanual}
The main file should be prepared as usual, see \secref{sec:include}.
However, the document body must make a distinction
between processing of an individual part and of the main document, e.g.:
%
\begin{center}
\begin{tabular}{l}
|\ifchilddocmanual|\\
|\input{\childdocname}|\\
|\||else|\\
\textit{document body with }|\input{|\textit{part}|}|\\
|\||fi|
\end{tabular}
\end{center}
%
The conditional |\ifchilddocmanual| is true whenever
a part to be included by |\input| is being compiled,
and the name of the part is stored in |\childdocname|.

%%%%%%%%%%%%%%%%%%%%%%%%%%%%%%%%%%%%%%%%
\DescribeMacro{\childdocby}
Each part to be included by |\input| should start with:
%
\begin{center}
\begin{tabular}{l}
|\input{childdoc.def}|\\
|\childdocby{|\textit{main}|}|\\
\end{tabular}
\end{center}
%
The directive |\childdocby| is similar to |\childdocof|
described in \secref{sec:include},
but the subsequent selection of content must be done manually.
To that end, both |\ifchilddoc| and |\ifchilddocmanual|
will be true upon processing of a part,
and the name of the part is stored in |\childdocname|.
Note that |\jobname| will be set to the filename of the current part
so that each part receives an individual |.aux| file
that does not interfere with the |.aux| file(s) of the main document.
This behaviour can be altered by the alternative form
|\childdocby[*]{|\textit{main}|}| (with a non-empty optional argument)
which uses the |.aux| file of the main document
by setting |\jobname| to \textit{main}.

%%%%%%%%%%%%%%%%%%%%%%%%%%%%%%%%%%%%%%%%%%%%%%%%%%%%%%%%%%%%%%%%%%%%%%%%%%%%%%%%
\subsection{Driver Development}
\label{sec:driver}

The \textsf{childdoc} mechanism can also be use for the development
of definition files such as \LaTeX{} styles or classes.
This case differs from the above setup with multiple parts
included by |\include| in that no |\includeonly| should be invoked.
This can be achieved by starting the include file
(before |\ProvidesPackage|) with:
%
\begin{center}
\begin{tabular}{l}
|\input{childdoc.def}|\\
|\childdocforward{|\textit{main}|}|\\
\end{tabular}
\end{center}
%
or alternatively with:
%
\begin{center}
\begin{tabular}{l}
|\input{childdoc.def}|\\
|\childdocby{|\textit{main}|}|\\
\end{tabular}
\end{center}
%
Both forms have slightly different effects as described above.
The main file is prepared as usual, see \secref{sec:include}.

%%%%%%%%%%%%%%%%%%%%%%%%%%%%%%%%%%%%%%%%%%%%%%%%%%%%%%%%%%%%%%%%%%%%%%%%%%%%%%%%
\subsection{Legacy Detection}
\label{sec:detection}

The directive |\childdocmain| in the main file can detect
whether the complete document or merely a child is to be compiled
even without using the directive |\childdocof|.
This method is deprecated because it is less robust
and there is no compelling reason to use it;
it is merely provided for backward compatibility
and it may be removed in future versions.

If the detection mechanism is to be used,
it is mandatory to correctly specify
the filename of the main file as the argument of |\childdocmain|:
%
\begin{center}
\begin{tabular}{l}
|\input{childdoc.def}|\\
|\childdocmain{|\textit{main}|}|\\
\end{tabular}
\end{center}
%
If |\jobname| does not match the argument \textit{main} of |\childdocmain|,
it is assumed that |\jobname| points to the child file to be compiled.
When using |\childdocmain| with the main file specified as argument,
it suffices to start a child file
with just |\input{|\textit{main}|}|
without loading of the package and using |\childdocof|.
If instead all processing is done
with the appropriate \textsf{childdoc} directives,
the argument of \textit{main} of |\childdocmain| can be empty.

An alternative version of the command line processing described
in \secref{sec:commandline} using the detection mechanism reads:
%
\begin{center}
|... -jobname "|\textit{target}|" "|[\textit{flags}]%
[|\def\jobname{|\textit{dest}|}|]|\input{|\textit{main}|}"|
\end{center}

%%%%%%%%%%%%%%%%%%%%%%%%%%%%%%%%%%%%%%%%%%%%%%%%%%%%%%%%%%%%%%%%%%%%%%%%%%%%%%%%
\subsection{Manual Code}
\label{sec:manual}

In case one cannot be certain whether the definitions file |childdoc.def|
is installed on the target \TeX{} distribution
and one prefers not to ship it,
it is conceivable to paste a few relevant commands into the sources.

To that end, drop all statements |\input{childdoc.def}|
and perform the replacements as outlined below.
Instead of |\childdocmain{|\textit{main}|}| add the following code
to the top of the main file:
%
\begin{center}
\begin{tabular}{l}
|\||ifdefined\childdocname\endinput\||fi\newif\ifchilddoc|\\
|\edef\childdocname{\scantokens\expandafter{\jobname\noexpand}}|\\
|\def\childdocmain{|\textit{main}|}\||ifx\childdocmain\childdocname\||else|\\
|\childdoctrue\includeonly{\childdocname}\let\jobname\childdocmain\||fi|\\
\end{tabular}
\end{center}
%
Instead of |\childdocof{|\textit{main}|}| just include the main file
at the top of each child file:
%
\begin{center}
|\input{|\textit{main}|}|
\end{center}
%
A simple redirection |\childdocforward{|\textit{dest}|}| is achieved by:
%
\begin{center}
|\def\jobname{|\textit{dest}|}\input{\jobname}|
\end{center}
%
The redirection with prefix
|\childdocforwardprefix[|\textit{prefix}|]{|\textit{dest}|}|
is accomplished by:
%
\begin{center}
\begin{tabular}{l}
|{\edef\jobname{\scantokens\expandafter{\jobname\noexpand}}|\\
|\def\redirectjob |\textit{prefix}|#1~~~{\gdef\jobname{|\textit{dest}|#1}}|\\
|\expandafter\redirectjob\jobname~~~}\input{\jobname}|
\end{tabular}
\end{center}

In an alternative approach,
child documents can be compiled by a specific command line
without additional code or specific definitions:
%
\begin{center}
|... -jobname "|\textit{target}|" "|[\textit{flags}]%
|\includeonly{|\textit{dest}|}\input{|\textit{main}|}"|
\end{center}
%

%%%%%%%%%%%%%%%%%%%%%%%%%%%%%%%%%%%%%%%%%%%%%%%%%%%%%%%%%%%%%%%%%%%%%%%%%%%%%%%%
%%%%%%%%%%%%%%%%%%%%%%%%%%%%%%%%%%%%%%%%%%%%%%%%%%%%%%%%%%%%%%%%%%%%%%%%%%%%%%%%
\section{Information}

%%%%%%%%%%%%%%%%%%%%%%%%%%%%%%%%%%%%%%%%%%%%%%%%%%%%%%%%%%%%%%%%%%%%%%%%%%%%%%%%
\subsection{Copyright}

Copyright \copyright{} 2017--2018 Niklas Beisert

This work may be distributed and/or modified under the
conditions of the \LaTeX{} Project Public License, either version 1.3
of this license or (at your option) any later version.
The latest version of this license is in
  \url{http://www.latex-project.org/lppl.txt}
and version 1.3 or later is part of all distributions of \LaTeX{}
version 2005/12/01 or later.

This work has the LPPL maintenance status `maintained'.

The Current Maintainer of this work is Niklas Beisert.

This work consists of the files |README.txt|, |childdoc.ins| and |childdoc.dtx|
as well as the derived files |childdoc.def|, |cdocsamp.tex|
with |cdocsch1.tex|, |cdocsch2.tex|, |cdocspt3.tex|, |cdocspt4.tex|,
|cdocsdrf.tex|, |cdocsfn1.tex|, |cdocsfn2.tex|
as well as |childdoc.pdf|.

%%%%%%%%%%%%%%%%%%%%%%%%%%%%%%%%%%%%%%%%%%%%%%%%%%%%%%%%%%%%%%%%%%%%%%%%%%%%%%%%
\subsection{Files and Installation}

The package consists of the files:
%
\begin{center}
\begin{tabular}{ll}
    |README.txt|   & readme file \\
    |childdoc.ins| & installation file \\
    |childdoc.dtx| & source file \\
    |childdoc.def| & definition file \\
    |cdocsamp.tex| & sample main file \\
    |cdocsch1.tex| & sample include file \\
    |cdocsch2.tex| & sample include file \\
    |cdocspt3.tex| & sample part file \\
    |cdocspt4.tex| & sample part file \\
    |cdocsdrf.tex| & sample redirection file \\
    |cdocsfn1.tex| & sample redirection file \\
    |cdocsfn2.tex| & sample redirection file \\
    |childdoc.pdf| & manual
\end{tabular}
\end{center}
%
The distribution consists of the files
|README.txt|, |childdoc.ins| and |childdoc.dtx|.
%
\begin{itemize}
\item
Run (pdf)\LaTeX{} on |childdoc.dtx|
to compile the manual |childdoc.pdf| (this file).
\item
Run \LaTeX{} on |childdoc.ins| to create the definitions file |childdoc.def|
and the sample |cdocsamp.tex| with include files
|cdocsch1.tex|, |cdocsch2.tex|, |cdocspt3.tex|, |cdocspt4.tex|,
|cdocsdrf.tex|, |cdocsfn1.tex|, |cdocsfn2.tex|.
Then copy the file |childdoc.def| to an appropriate directory of your \LaTeX{}
distribution, e.g.\ \textit{texmf-root}|/tex/latex/childdoc|.
\end{itemize}

%%%%%%%%%%%%%%%%%%%%%%%%%%%%%%%%%%%%%%%%%%%%%%%%%%%%%%%%%%%%%%%%%%%%%%%%%%%%%%%%
\subsection{Related CTAN Packages}

There are several other packages which offer a similar functionality:
%
\begin{itemize}
\item
The packages
\href{http://ctan.org/pkg/docmute}{\textsf{docmute}},
\href{http://ctan.org/pkg/includex}{\textsf{includex}} and
\href{http://ctan.org/pkg/standalone}{\textsf{standalone}}
provide commands to include only the document body of
a child file thus allowing both files to be compiled individually.
\item
The packages \href{http://ctan.org/pkg/subdocs}{\textsf{subdocs}}
and \href{http://ctan.org/pkg/subfiles}{\textsf{subfiles}}
provide structures in which the main and child documents can be
encapsulated and allowing them to be compiled individually.
The inclusion mechanism is different from the conventional |\include|.
\item
The package \href{http://ctan.org/pkg/combine}{\textsf{combine}}
is an elaborate solution to combine several documents into one.
\end{itemize}
%
See also the CTAN topic \href{http://ctan.org/topic/subdocs}{\textsf{subdocs}}
for further related packages.
The present package differs from the above solutions in that
a document structure constructed with the conventional |\include| mechanism
just needs two extra commands at the top of every file
such that all constituent files can be compiled individually.

%%%%%%%%%%%%%%%%%%%%%%%%%%%%%%%%%%%%%%%%%%%%%%%%%%%%%%%%%%%%%%%%%%%%%%%%%%%%%%%%
%\subsection{Feature Suggestions}
%
%The following is a list of features which may be useful for future
%versions of this package:
%%
%\begin{itemize}
%\item
%\ldots
%\end{itemize}

%%%%%%%%%%%%%%%%%%%%%%%%%%%%%%%%%%%%%%%%%%%%%%%%%%%%%%%%%%%%%%%%%%%%%%%%%%%%%%%%
\subsection{Revision History}

%%%%%%%%%%%%%%%%%%%%%%%%%%%%%%%%%%%%%%%%
\paragraph{v2.0:} 2018/12/30

\begin{itemize}
\item
immediate forward processing
\item
added |\childdocby| mechanism
\item
manual restructured
\end{itemize}

%%%%%%%%%%%%%%%%%%%%%%%%%%%%%%%%%%%%%%%%
\paragraph{v1.6:} 2018/01/17

\begin{itemize}
\item
application for development of include files
\item
corrections to manual
\end{itemize}

%%%%%%%%%%%%%%%%%%%%%%%%%%%%%%%%%%%%%%%%
\paragraph{v1.5:} 2017/05/21

\begin{itemize}
\item
more complete structuring introduced
\item
|\childdocof| introduced
\item
|\childdoc| renamed to |\childdocmain|
\item
|\childredirect| renamed to |\childdocforward| and |\childdocforwardprefix|
and functionality expanded
\end{itemize}

%%%%%%%%%%%%%%%%%%%%%%%%%%%%%%%%%%%%%%%%
\paragraph{v1.0:} 2017/04/27

\begin{itemize}
\item
manual and install package
\item
first version published on CTAN
\end{itemize}

%%%%%%%%%%%%%%%%%%%%%%%%%%%%%%%%%%%%%%%%
\paragraph{v0.6:} 2017/04/26

\begin{itemize}
\item
redirection mechanism added
\end{itemize}

%%%%%%%%%%%%%%%%%%%%%%%%%%%%%%%%%%%%%%%%
\paragraph{v0.5:} 2017/04/26

\begin{itemize}
\item
functionality in definition file
\end{itemize}


%%%%%%%%%%%%%%%%%%%%%%%%%%%%%%%%%%%%%%%%%%%%%%%%%%%%%%%%%%%%%%%%%%%%%%%%%%%%%%%%
%%%%%%%%%%%%%%%%%%%%%%%%%%%%%%%%%%%%%%%%%%%%%%%%%%%%%%%%%%%%%%%%%%%%%%%%%%%%%%%%
%%%%%%%%%%%%%%%%%%%%%%%%%%%%%%%%%%%%%%%%%%%%%%%%%%%%%%%%%%%%%%%%%%%%%%%%%%%%%%%%
\appendix

\settowidth\MacroIndent{\rmfamily\scriptsize 000\ }

 \DocInput{childdoc.dtx}

\end{document}
%</driver>
% \fi
%
% %%%%%%%%%%%%%%%%%%%%%%%%%%%%%%%%%%%%%%%%%%%%%%%%%%%%%%%%%%%%%%%%%%%%%%%%%%%%%%
% %%%%%%%%%%%%%%%%%%%%%%%%%%%%%%%%%%%%%%%%%%%%%%%%%%%%%%%%%%%%%%%%%%%%%%%%%%%%%%
% \section{Sample}
%\iffalse
%<*samplemain>
%\fi
%
% The following presents a sample document
% with two chapters, two parts, a title page,
% a compile flag as well as three forwarding files to set the flag.
% It consists of eight |.tex| files:
% \begin{center}
% \begin{tabular}{ll}
% |cdocsamp.tex|&main file\\
% |cdocsch1.tex|&include file for chapter 1\\
% |cdocsch2.tex|&include file for chapter 2\\
% |cdocspt3.tex|&include file for part 3\\
% |cdocspt4.tex|&include file for part 4\\
% |cdocsdrf.tex|&forwarding file for main file in draft mode\\
% |cdocsfi1.tex|&forwarding file for final version of chapter 1\\
% |cdocsfi2.tex|&forwarding file for final version of chapter 2\\
% \end{tabular}
% \end{center}
% Each of the eight files can be compiled directly by the \LaTeX{} compiler.
%
% %%%%%%%%%%%%%%%%%%%%%%%%%%%%%%%%%%%%%%
% \paragraph{Main File.}
%
% The main file is called |cdocsamp.tex|.
%
% Load the \textsf{childdoc} definitions and
% declare the filename for the main document:
%    \begin{macrocode}
\input{childdoc.def}
\childdocmain{}
%    \end{macrocode}

% Optional override for |\version| flag:
%    \begin{macrocode}
%%\ifchilddoc\else\providecommand{\version}{draft}\fi
%    \end{macrocode}

% Define the default values for the |\version| flag
% (|final| for the main file and |draft| for childs):
%    \begin{macrocode}
\ifchilddoc
\providecommand{\version}{draft}
\else
\providecommand{\version}{final}
\fi
%    \end{macrocode}

% Load the standard document class:
%    \begin{macrocode}
\documentclass[12pt]{article}
%    \end{macrocode}

% Start the document body:
%    \begin{macrocode}
\begin{document}
%    \end{macrocode}

% Declare a title page.
% Print title, part of document being processed and version flag:
%    \begin{macrocode}
\addtocounter{page}{-1}
\begin{center}
{\LARGE\bfseries{}childdoc example\par}
\vspace{1cm}
\ifchilddoc
\ifchilddocmanual part\else chapter\fi:
`\childdocname' of `\childdocjob'\par
\else
main document: `\childdocjob'\par
\fi
version: \version\par
\end{center}
\newpage
%    \end{macrocode}

% Manually include selected file,
% otherwise process as usual:
%    \begin{macrocode}
\ifchilddocmanual
\section*{part `\childdocname'}
\input{\childdocname}
\else
%    \end{macrocode}

% Include the two chapters:
%    \begin{macrocode}
\include{cdocsch1}
\include{cdocsch2}
%    \end{macrocode}

% Include the two parts unless only chapters should be displayed:
%    \begin{macrocode}
\ifchilddoc\else
\section{part three}
\input{cdocspt3}
\section{part four}
\input{cdocspt4}
\fi
%    \end{macrocode}

% Process as usual until here:
%    \begin{macrocode}
\fi
%    \end{macrocode}

% End of document body:
%    \begin{macrocode}
\end{document}
%    \end{macrocode}
%\iffalse
%</samplemain>
%\fi
%
% %%%%%%%%%%%%%%%%%%%%%%%%%%%%%%%%%%%%%%
% \paragraph{Chapter Include Files.}
%
% The include files are called |cdocsch1.tex| and |cdocsch2.tex|.
%
%\iffalse
%<*samplechap1|samplechap2>
%\fi

% Optional override for |\version| flag:
%    \begin{macrocode}
%%\providecommand{\version}{final}
%    \end{macrocode}

% Include the main document:
%    \begin{macrocode}
\input{childdoc.def}
\childdocof{cdocsamp}
%    \end{macrocode}

%\iffalse
%</samplechap1|samplechap2>
%\fi
%
%\iffalse
%<*samplechap1>
%\fi
% Some text for chapter 1:
%    \begin{macrocode}
\section{one}
some text in chapter one
%    \end{macrocode}

%\iffalse
%</samplechap1>
%\fi
% Some text for chapter 2:
%\iffalse
%<*samplechap2>
%\fi
%    \begin{macrocode}
\section{two}
more text in chapter two
%    \end{macrocode}

%\iffalse
%</samplechap2>
%\fi
%
% %%%%%%%%%%%%%%%%%%%%%%%%%%%%%%%%%%%%%%
% \paragraph{Part Include Files.}
%
% The include files are called |cdocspt3.tex| and |cdocspt4.tex|.
%
%\iffalse
%<*samplepart3|samplepart4>
%\fi

% Optional override for |\version| flag:
%    \begin{macrocode}
%%\providecommand{\version}{final}
%    \end{macrocode}

% Include the main document:
%    \begin{macrocode}
\input{childdoc.def}
\childdocby{cdocsamp}
%    \end{macrocode}

%\iffalse
%</samplepart3|samplepart4>
%\fi
%
%\iffalse
%<*samplepart3>
%\fi
% Some text for part 3:
%    \begin{macrocode}
some text in part three
%    \end{macrocode}

%\iffalse
%</samplepart3>
%\fi
% Some text for part 4:
%\iffalse
%<*samplepart4>
%\fi
%    \begin{macrocode}
more text in part four
%    \end{macrocode}

%\iffalse
%</samplepart4>
%\fi
%
% %%%%%%%%%%%%%%%%%%%%%%%%%%%%%%%%%%%%%%
% \paragraph{Forwarding for a Complete Draft.}
%
% The following forwarding file |cdocsdrf.tex|
% compiles the main document in draft mode:
%\iffalse
%<*sampledraft>
%\fi
%    \begin{macrocode}
\def\version{draft}
\input{childdoc.def}
\childdocforward{cdocsamp}
%    \end{macrocode}

%\iffalse
%</sampledraft>
%\fi
%
% %%%%%%%%%%%%%%%%%%%%%%%%%%%%%%%%%%%%%%
% \paragraph{Forwarding for Final Version of the Chapters.}
%
% The following forwarding files |cdocsfn1.tex| and |cdocsfn2.tex|
% (with identical content)
% compile the final versions of the child documents
% |cdocsch1.tex| and |cdocsch2.tex|, respectively:
%\iffalse
%<*samplefinal>
%\fi
%    \begin{macrocode}
\def\version{final}
\input{childdoc.def}
\childdocforwardprefix[cdocsamp]{cdocsfn}{cdocsch}
%    \end{macrocode}

%\iffalse
%</samplefinal>
%\fi
%
% %%%%%%%%%%%%%%%%%%%%%%%%%%%%%%%%%%%%%%
% \paragraph{Command Line Processing.}
%
% The following three command lines generate the output files
% |cdocscld|, |cdocscl1| and |cdocscl2|
% which should be identical to
% |cdocsdrf|, |cdocsch1| and |cdocsfn2|, respectively:
% \begin{center}
% \begin{tabular}{l}
% |latex -jobname cdocscld \|\\
% |  "\def\version{draft}\input{childdoc.def}\childdocforward{cdocsamp}"|\\
% |latex -jobname cdocscl1 \|\\
% |  "\input{childdoc.def}\childdocforward[cdocsamp]{cdocsch1}"|\\
% |latex -jobname cdocscl2 \|\\
% |  "\def\version{final}\input{childdoc.def}\childdocforward{cdocsch2}"|
% \end{tabular}
% \end{center}
% Note that the trailing backslash on each first line
% merely continues the input to the second line
% (for convenient cut ant paste).
% Furthermore, the command |latex| can be replaced by any
% of its alternative versions such as |pdflatex|.
%
% %%%%%%%%%%%%%%%%%%%%%%%%%%%%%%%%%%%%%%%%%%%%%%%%%%%%%%%%%%%%%%%%%%%%%%%%%%%%%%
% %%%%%%%%%%%%%%%%%%%%%%%%%%%%%%%%%%%%%%%%%%%%%%%%%%%%%%%%%%%%%%%%%%%%%%%%%%%%%%
% \section{Implementation}
%\iffalse
%<*package>
%\fi
%
% This section describes the definitions file |childdoc.def|.

% The definitions cannot be loaded using |\usepackage| or |\RequirePackage|
% which has a mechanism to prevent loading a style file more than once.
% When loading the definitions by means of |\input|
% multiple instances have to be prevented manually:
%\iffalse
%This code needs to be before the `\ProvidesFile' directive
%which is defined at the beginning of this file.
%Therefore it is also placed there and commented out here.
%</package>
%<*discard>
%\fi
%    \begin{macrocode}
\ifdefined\childdocmain\endinput\fi
%    \end{macrocode}
%\iffalse
%</discard>
%<*package>
%\fi
%
% \macro{\ifchilddoc}
% \macro{\ifchilddocmanual}
% The conditional |\ifchilddoc| tells whether a
% child (true) or main (false) document is being compiled.
% The conditional |\ifchilddocmanual| tells whether
% the |\includeonly| mechanism is used (false) or
% the selection of child files must be performed manually (true).
% The definitions initialise to false:
%    \begin{macrocode}
\newif\ifchilddoc
\newif\ifchilddocmanual
%    \end{macrocode}

% \macro{\childdocname}
% \macro{\childdocjob}
% The macro |\childdocname| stores the name of the main document
% to be compiled. The macro |\childdocjob| stores the name of
% the document on which the \LaTeX{} compiler was originally invoked.
% The content of |\jobname| cannot be compared
% to filenames specified in the source due to different catcodes.
% The following code rescans |\jobname|, stores the result
% in |\childdocname| and saves a copy in |\childdocjob|:
%    \begin{macrocode}
\edef\childdocname{\scantokens\expandafter{\jobname\noexpand}}
\let\childdocjob\childdocname
%    \end{macrocode}

% \macro{\childdocdisable}
% The macro |\childdocdisable| prevents the main file
% from being processed more than once.
% At this stage, the main document command |\childdocmain|
% is assumed to be called once again where it should do nothing.
% Any subsequent call to it should prevent
% a secondary processing of the main document
% It overwrites the forwarding commands
% |\childdocof| and |\childdocforward|
% with empty macros to prevent further inclusions of the main document:
%    \begin{macrocode}
\newcommand{\childdocdisable}
{
  \renewcommand{\childdocmain}[1]{\renewcommand{\childdocmain}[1]{\endinput}}
  \renewcommand{\childdocof}[1]{}
  \renewcommand{\childdocby}[2][]{}
  \renewcommand{\childdocforward}[2][]{}
  \renewcommand{\childdocdisable}{}
}
%    \end{macrocode}

% \macro{\childdocmain}
% The macro |\childdocmain| is to be called at the top of the main file
% with nothing or the main filename (without extension) as argument.
% First, it breaks loops.
% If the argument is not empty and does not match |\childdocname|
% (which is set by the first inclusion of |childdoc.def|),
% |\ifchilddoc| is set to true, |\includeonly| is applied to the child file
% and |\jobname| is set to the main file
% (for proper handling of |.aux| files):
%    \begin{macrocode}
\newcommand{\childdocmain}[1]
{
  \childdocdisable\childdocmain{}
  \if?#1?\else
    \begingroup
      \def\childdoctmp{#1}
      \ifx\childdoctmp\childdocname
        \def\childdoctmp{}
      \else
        \def\childdoctmp
        {
          \childdoctrue
          \includeonly{\childdocname}
          \def\childdocjob{#1}
          \def\jobname{#1}
        }
      \fi
      \expandafter
    \endgroup
    \childdoctmp
  \fi
}
%    \end{macrocode}

% \macro{\childdocof}
% The command |\childdocof| redirects
% compilation to the main file |#1|.
%    \begin{macrocode}
\newcommand{\childdocof}[1]
{
  \childdocdisable
  \childdoctrue
  \includeonly{\childdocname}
  \def\jobname{#1}
  \def\childdocjob{#1}
  \input{#1}
}
%    \end{macrocode}

% \macro{\childdocby}
% The command |\childdocby| ....
%    \begin{macrocode}
\newcommand{\childdocby}[2][]
{
  \childdocdisable
  \childdoctrue
  \childdocmanualtrue
  \if?#1?\else
    \def\jobname{#2}
  \fi
  \def\childdocjob{#2}
  \input{#2}
  \endinput
}
%    \end{macrocode}

% \macro{\childdocforward}
% The command |\childdocforward| redirects
% compilation to the main file or
% (if the optional argument is given) a child file.
% Parameters are set as if the main file
% or a child file starting with |\childdocof| was compiled.
% Then compilation is handed over to the main file:
%    \begin{macrocode}
\newcommand{\childdocforward}[2][]
{
  \begingroup
    \if?#1?
      \def\childdoctmp
      {
        \def\childdocname{#2}
        \def\childdocjob{#2}
        \def\jobname{#2}
        \input{#2}
        \endinput
      }
    \else
      \def\childdoctmp
      {
        \childdocdisable
        \def\childdocname{#2}
        \childdoctrue
        \includeonly{#2}
        \def\childdocjob{#1}
        \def\jobname{#1}
        \input{#1}
        \endinput
      }
    \fi
    \expandafter
  \endgroup
  \childdoctmp
}
%    \end{macrocode}

% \macro{\childdocforwardprefix}
% The command |\childdocforwardprefix| redirects
% compilation to the main or a child file by means of a pattern.
% The prefix |#1| in the current filename is replaced by |#2|
% and the suffix of the current filename is kept
% (it is assumed that the filename does not contain the substring `|~~~|'
% which is used as a delimiter).
% Compilation is handed over to the new file by |\childdocforward|:
%    \begin{macrocode}
\newcommand{\childdocforwardprefix}[3][]
{
  \begingroup
    \def\childdocextract #2##1~~~{\def\childdoctmp{\childdocforward[#1]{#3##1}}}
    \expandafter\childdocextract\childdocname~~~
    \expandafter
  \endgroup
  \childdoctmp
}
%    \end{macrocode}

% \macro{\childdoc}
% The deprecated macro |\childdoc| is a legacy version of |\childdocmain|:
%    \begin{macrocode}
\newcommand{\childdoc}{\childdocmain}
%    \end{macrocode}

% \macro{\childdocredirect}
% The deprecated macro |\childdocredirect| is a legacy version
% of |\childdocforward| and |\childdocforwardprefix|:
%    \begin{macrocode}
\newcommand{\childdocredirect}[2][]
{
  \begingroup
    \if?#1?
      \def\childdoctmp{\childdocforward{#2}}
    \else
      \def\childdoctmp{\childdocforwardprefix{#1}{#2}}
    \fi
    \expandafter
  \endgroup
  \childdoctmp
}
%    \end{macrocode}

%\iffalse
%</package>
%\fi
%
\endinput

\childdocforwardprefix[cdocsamp]{cdocsfn}{cdocsch}
%    \end{macrocode}

%\iffalse
%</samplefinal>
%\fi
%
% %%%%%%%%%%%%%%%%%%%%%%%%%%%%%%%%%%%%%%
% \paragraph{Command Line Processing.}
%
% The following three command lines generate the output files
% |cdocscld|, |cdocscl1| and |cdocscl2|
% which should be identical to
% |cdocsdrf|, |cdocsch1| and |cdocsfn2|, respectively:
% \begin{center}
% \begin{tabular}{l}
% |latex -jobname cdocscld \|\\
% |  "\def\version{draft}% \iffalse
%
% childdoc.dtx Copyright (C) 2017-2018 Niklas Beisert
%
% This work may be distributed and/or modified under the
% conditions of the LaTeX Project Public License, either version 1.3
% of this license or (at your option) any later version.
% The latest version of this license is in
%   http://www.latex-project.org/lppl.txt
% and version 1.3 or later is part of all distributions of LaTeX
% version 2005/12/01 or later.
%
% This work has the LPPL maintenance status `maintained'.
%
% The Current Maintainer of this work is Niklas Beisert.
%
% This work consists of the files childdoc.dtx and childdoc.ins
% and the derived files childdoc.def and cdocsamp.tex with
% cdocsch1.tex, cdocsch2.tex, cdocsdrf.tex, cdocsfn1.tex, cdocsfn2.tex.
%
%<package>\ifdefined\childdocmain\endinput\fi
%<package>\ProvidesFile{childdoc.def}[2018/12/30 v2.0 child document driver]
%<samplemain>\ProvidesFile{cdocsamp.tex}[2018/12/30 v2.0 sample for childdoc]
%<*driver>
%\ProvidesFile{childdoc.drv}[2018/12/30 v2.0 childdoc reference manual file]
\PassOptionsToClass{10pt,a4paper}{article}
\documentclass{ltxdoc}

\usepackage[margin=35mm]{geometry}
\usepackage{hyperref}
\usepackage{hyperxmp}
\usepackage[usenames]{color}

\hypersetup{colorlinks=true}
\hypersetup{pdfstartview=FitH}
\hypersetup{pdfpagemode=UseNone}
\hypersetup{pdfsource={}}
\hypersetup{pdflang={en-UK}}
\hypersetup{pdfcopyright={Copyright 2017-2018 Niklas Beisert.
  This work may be distributed and/or modified under the
  conditions of the LaTeX Project Public License, either version 1.3
  of this license or (at your option) any later version.}}
\hypersetup{pdflicenseurl={http://www.latex-project.org/lppl.txt}}
\hypersetup{pdfcontactaddress={ETH Zurich, ITP, HIT K,
  Wolfgang-Pauli-Strasse 27}}
\hypersetup{pdfcontactpostcode={8093}}
\hypersetup{pdfcontactcity={Zurich}}
\hypersetup{pdfcontactcountry={Switzerland}}
\hypersetup{pdfcontactemail={nbeisert@itp.phys.ethz.ch}}
\hypersetup{pdfcontacturl={http://people.phys.ethz.ch/\xmptilde nbeisert/}}

\newcommand{\secref}[1]{\hyperref[#1]{section \ref*{#1}}}

\parskip1ex
\parindent0pt
\let\olditemize\itemize
\def\itemize{\olditemize\parskip0pt}

\begin{document}

\title{The \textsf{childdoc} Package}
\hypersetup{pdftitle={The childdoc Package}}
\author{Niklas Beisert\\[2ex]
  Institut f\"ur Theoretische Physik\\
  Eidgen\"ossische Technische Hochschule Z\"urich\\
  Wolfgang-Pauli-Strasse 27, 8093 Z\"urich, Switzerland\\[1ex]
  \href{mailto:nbeisert@itp.phys.ethz.ch}
  {\texttt{nbeisert@itp.phys.ethz.ch}}}
\hypersetup{pdfauthor={Niklas Beisert}}
\hypersetup{pdfsubject={Manual for the LaTeX2e Package childdoc}}
\date{30 December 2018, \textsf{v2.0}}
\maketitle

\begin{abstract}\noindent
\textsf{childdoc} is a \LaTeXe{} package
that enables the direct compilation
of document sections included by |\include|
to individual files.
\end{abstract}

\begingroup
\parskip0ex
\tableofcontents
\endgroup

%%%%%%%%%%%%%%%%%%%%%%%%%%%%%%%%%%%%%%%%%%%%%%%%%%%%%%%%%%%%%%%%%%%%%%%%%%%%%%%%
%%%%%%%%%%%%%%%%%%%%%%%%%%%%%%%%%%%%%%%%%%%%%%%%%%%%%%%%%%%%%%%%%%%%%%%%%%%%%%%%
\section{Introduction}

\LaTeX{} provides a mechanism to structure a large document (such as a book)
into a main file and several child files (containing the chapters)
using the |\include| command.
This mechanism is beneficial for documents
which span hundreds of pages in order to
make the source file(s) more manageable.
Moreover, compilation can be restricted to
selected child files by means of the |\includeonly| command.
The latter feature can be used to reduce the compilation time while editing
(this was significantly more useful in the earlier days of \LaTeX{})
or to generate a smaller document which is easier to navigate.
Another application of |\includeonly| is to generate
documents consisting of selected parts of the complete document.

However, there are a few drawbacks of the plain |\include| mechanism:
\begin{itemize}
\item
The child files cannot be compiled on their own,
they can only be compiled via the main file.
A naive editing environment
(such as a text editor with an option
to have the current file processed by \LaTeX)
may require one to switch to the main file before compiling;
attempting to compile the child file produces errors.
\item
The main file must be modified (each time)
to adjust the |\includeonly| command
to the present needs. This easily leaves the main file in a messy state.
\item
The generated document will always carry the filename
of the main document. This is inconvenient if
several child files are to be compiled and
to be kept for distribution.
\end{itemize}

The present package provides a simple interface
to make child files individually compilable by \LaTeX{}.
Compiling a child file then has the same effect as compiling
the main file with an |\includeonly| command
to select the appropriate child.
Moreover the generated document will carry the name of the child
rather than the main file.
This resolves all three above issues.

This feature is meant to make the editing of books,
thesis documents and lecture notes somewhat more convenient.
However, the package can also be used efficiently for
composing a series of documents (such as exercise sheets)
which are typically distributed individually.
It then assists the author in generating the individual documents
(potentially in different versions)
as well as a document containing the collected series.
Another application is in developing style files
or other kinds of included material
where compilation of the style file could redirect
to a sample or test file.

%%%%%%%%%%%%%%%%%%%%%%%%%%%%%%%%%%%%%%%%%%%%%%%%%%%%%%%%%%%%%%%%%%%%%%%%%%%%%%%%
%%%%%%%%%%%%%%%%%%%%%%%%%%%%%%%%%%%%%%%%%%%%%%%%%%%%%%%%%%%%%%%%%%%%%%%%%%%%%%%%
\section{Usage}

First of all, the package \textsf{childdoc} is \emph{not} a standard
\LaTeXe{} |.sty| style file! Therefore it needs to be invoked in
a non-standard way.

%%%%%%%%%%%%%%%%%%%%%%%%%%%%%%%%%%%%%%%%%%%%%%%%%%%%%%%%%%%%%%%%%%%%%%%%%%%%%%%%
\subsection{Included Files}
\label{sec:include}

%%%%%%%%%%%%%%%%%%%%%%%%%%%%%%%%%%%%%%%%
\DescribeMacro{\childdocmain}
To use the package, add the commands
\begin{center}
\begin{tabular}{l}
|\input{childdoc.def}|\\
|\childdocmain{}|\\
\end{tabular}
\end{center}
at the very top of the main \LaTeX{} file,
in particular \emph{before} the |\documentclass| statement!
The argument of |\childdocmain| should be left empty
(but it must be present).

%%%%%%%%%%%%%%%%%%%%%%%%%%%%%%%%%%%%%%%%
\DescribeMacro{\childdocof}
Furthermore, add the commands
\begin{center}
\begin{tabular}{l}
|\input{childdoc.def}|\\
|\childdocof{|\textit{main}|}|\\
\end{tabular}
\end{center}
at the top of every child file \textit{child}
which is included by |\include{|\textit{child}|}|
from within the main file
(or at least for those files to be compiled individually).
The argument \textit{main} must be the filename of the main file.

There are a couple of
considerations in setting up the main and child documents:

%%%%%%%%%%%%%%%%%%%%%%%%%%%%%%%%%%%%%%%%
\paragraph{Restrictions.}

Please note the following restrictions:
\begin{itemize}
\item
|\childdocmain| must be called with one argument \textit{main}
to ensure compatibility with earlier version of the package.
It must either be empty (|\childdocmain{}|)
or precisely match the filename of the main file in which it is specified.
See \secref{sec:detection} for further information.
\item
The filename \textit{main} must be specified without the |.tex| extension.
\item
The filename \textit{main} is case sensitive
(even in case-insensitive file systems)
due to internal string comparison.
\item
The argument \textit{main} should be fully expanded, it cannot be a macro.
\item
Subdirectories and special characters should be avoided in filenames.
\item
The command |\childdocmain{|\textit{main}|}| must be followed by a whitespace.
It should not be followed immediately by another command
or by a comment mark `|%|'.
This is because the \TeX{} parser reads the token immediately following
the argument of |\childdocmain| and puts it
at the beginning of every child section;
however, a white\-space is ignored.
\end{itemize}

%%%%%%%%%%%%%%%%%%%%%%%%%%%%%%%%%%%%%%%%
\paragraph{Content of Main File.}

It is advisable to place all content in the child files included by |\include|.
Any output contained in the main file will appear in all child documents
unless suppressed manually;
it cannot be suppressed automatically by the |\includeonly| directive
and thus should normally be avoided.
A method to include some content in the main file
by means of conditional processing is described in \secref{sec:conditional}.

%%%%%%%%%%%%%%%%%%%%%%%%%%%%%%%%%%%%%%%%
\paragraph{Page Numbering.}

When only a part of the document is compiled,
the appropriate numbering of pages
(as well as other status parameters)
is determined from the |.aux| files.
The latter contain information from previous passes.
However this information needs to propagate through
all intermediate child documents.
Therefore the page numbering in child documents may well
be inconsistent until the complete document is compiled at least once.

A useful (if unconventional) way to always ensure a consistent
page numbering is to restart the numbering in each child document
and denote the pages by `\textit{child}|.|\textit{page}'
where \textit{child} represents the chapter/section number of the child file.
This can be achieved by the command
|\numberwithin{page}{|\textit{child}|}|
of the \textsf{amsmath} package
where \textit{child} can be |chapter| or |section|
depending on the chosen structuring.
Alternatively, one can modify the macro |\thepage| appropriately
and reset the counter |page| at the start of each child file.

%%%%%%%%%%%%%%%%%%%%%%%%%%%%%%%%%%%%%%%%%%%%%%%%%%%%%%%%%%%%%%%%%%%%%%%%%%%%%%%%
\subsection{Conditional Processing}
\label{sec:conditional}

The package provides a mechanism to compile different versions
of a document. To customise the versions further some conditional processing
can come in handy to distinguish which version is being compiled.
The package provides two macros to describe the compilation context:

%%%%%%%%%%%%%%%%%%%%%%%%%%%%%%%%%%%%%%%%
\DescribeMacro{\ifchilddoc}
The conditional |\ifchilddoc| distinguishes between the compilation of
child documents and the main document:
%
\begin{center}
|\ifchilddoc |\textit{child-code}| |[|\||else |\textit{main-code}]| \||fi|
\end{center}

%%%%%%%%%%%%%%%%%%%%%%%%%%%%%%%%%%%%%%%%
\DescribeMacro{\childdocname}
\DescribeMacro{\childdocjob}
The macro |\childdocname| contains the filename (without extension)
of the main or child file being processed.
Note that |\childdocjob| will always contain the name of the main file.

%%%%%%%%%%%%%%%%%%%%%%%%%%%%%%%%%%%%%%%%
\paragraph{Title Page.}

Conditional processing can be used to include a title or banner page
in the main document when proper precautions are taken.
Importantly, the code in the main file should ensure that the page counter
(as well as other status parameters which are stored in the |.aux| files)
takes the same value after the conditional processing.
Otherwise the page numbers may take divergent values
depending on which part is compiled.

For example, a title page could be declared by:
%
\begin{center}
\begin{tabular}{l}
|\ifchilddoc\||else|\\
|\addtocounter{page}{-1}|\\
\textit{code for title page}\\
|\newpage|\\
|\||fi|
\end{tabular}
\end{center}
%
A banner page for the child documents can be generated by:
%
\begin{center}
\begin{tabular}{l}
|\ifchilddoc|\\
|\addtocounter{page}{-1}|\\
\textit{code for banner page}\\
|\newpage|\\
|\||fi|
\end{tabular}
\end{center}
%
Here one could write a message such as:
\begin{center}
|This is the part \childdocname{} of \childdocjob{}.|
\end{center}

%%%%%%%%%%%%%%%%%%%%%%%%%%%%%%%%%%%%%%%%%%%%%%%%%%%%%%%%%%%%%%%%%%%%%%%%%%%%%%%%
\subsection{Flags}
\label{sec:flags}

The package makes it easy to generate different versions
of the main or child documents.
To this end compilation flags can be defined
and assigned different default values.
They will be particularly useful in conjunction
with the forwarding mechanism described in \secref{sec:forward}.

For example, it may be useful to have a flag |\version|
which can be set to |draft| or |final|.
The document source will contain some conditional code
depending on the value of |\version|.
Suppose further, the flag should default to |final| for the main file
and to |draft| for child files
which is a natural assignment for editing the document.
This is achieved by placing the following code
in the preamble of the main document
(below the |\childdocmain| directive):
%
\begin{center}
\begin{tabular}{l}
|\ifchilddoc|\\
|\providecommand{\version}{draft}|\\
|\||else|\\
|\providecommand{\version}{final}|\\
|\||fi|
\end{tabular}
\end{center}
%
The definition by |\providecommand| makes sure
that previous definitions are not overwritten.
Further statements |\providecommand{\version}{...}|
can thus be added before the above code to override it.

For the main file, one might add a line
(between |\childdocmain| and the above block)
%
\begin{center}
|%\ifchilddoc\||else\providecommand{\version}{draft}\||fi|
\end{center}
%
which can be uncommented to produce a draft version.
Likewise one can add a line to the very top of a child file
(above the |\childdocof{|\textit{main}|}| directive)
%
\begin{center}
|%\providecommand{\version}{final}|
\end{center}
%
which can be uncommented to produce the final version of this child document.

%%%%%%%%%%%%%%%%%%%%%%%%%%%%%%%%%%%%%%%%%%%%%%%%%%%%%%%%%%%%%%%%%%%%%%%%%%%%%%%%
\subsection{Forwarding}
\label{sec:forward}

Different versions of the main or child documents
using compilation flags as described in \secref{sec:flags}
can be (permanently) stored in different files
for convenient compilation, viewing and distribution.
To this end, the package defines a command
to pass on compilation to a different file:

%%%%%%%%%%%%%%%%%%%%%%%%%%%%%%%%%%%%%%%%
\DescribeMacro{\childdocforward}
The command |\childdocforward| redirects processing to
another source file:
%
\begin{center}
\begin{tabular}{l}
|\input{childdoc.def}|\\
|\childdocforward[|\textit{main}|]{|\textit{dest}|}|\\
\end{tabular}
\end{center}
%
The argument \textit{dest} is the destination file
(without extension).
It should be the main file or one of the child files.
Note that further \textsf{childdoc} directives
such as |\childdocof| and |\childdocforward|
in the indicated file will be processed in this form.
The optional argument \textit{main}
passes on directly to the main file \textit{main}
while pretending to compile the child \textit{dest}.
This form behaves as if \textit{dest}
issues |\childdocof{|\textit{main}|}| right away,
and no further \textsf{childdoc} directives will be processed.

%%%%%%%%%%%%%%%%%%%%%%%%%%%%%%%%%%%%%%%%
\DescribeMacro{\...prefix}
In the alternative form |\childdocforwardprefix|,
%
\begin{center}
\begin{tabular}{l}
|\input{childdoc.def}|\\
|\childdocforwardprefix[|\textit{main}|]{|\textit{prefix}|}{|\textit{dest}|}|
\end{tabular}
\end{center}
%
the destination file is determined by a pattern
depending on the current file:
To make this work, the current file must be called
`{\textit{prefix}\hspace{0.2em}\textit{suffix}}'
with \textit{prefix} matching precisely the argument.
Processing is then passed on to the file
`{\textit{dest}\hspace{0.2em}\textit{suffix}}'.
Surely, the same effect is achieved by
directly specifying the
argument `{\textit{dest}\hspace{0.2em}\textit{suffix}}'
in the first form.
However, that requires to set up a different file
for each child. With the alternative form of the command
all these files can have exactly the same content
which simplifies setting them up and maintaining them.

For example, the following file |draft.tex|
with a compilation flag |\version| as described in \secref{sec:flags}
compiles the main document as a draft:
%
\begin{center}
\begin{tabular}{l}
|\def\version{draft}|\\
|\input{childdoc.def}|\\
|\childdocforward{|\textit{main}|}|
\end{tabular}
\end{center}
%
Likewise, the following files |final|\textit{nn}|.tex|
compile the final version of the child document
|child|\textit{nn}|.tex|:
%
\begin{center}
\begin{tabular}{l}
|\def\version{final}|\\
|\input{childdoc.def}|\\
|\childdocforwardprefix{final}{child}|
\end{tabular}
\end{center}
%

Note that when several versions of a main file and/or of each child file
are to be generated, it may be convenient to set up a |Makefile| or
shell script to automatise the process.

%%%%%%%%%%%%%%%%%%%%%%%%%%%%%%%%%%%%%%%%%%%%%%%%%%%%%%%%%%%%%%%%%%%%%%%%%%%%%%%%
\subsection{Command Line Processing}
\label{sec:commandline}

The effect of redirection files can also be achieved by invoking
the \LaTeX{} compiler with a more elaborate command line.
Most conveniently this should be done as part
of a shell script or a |Makefile|.

When using \textsf{childdoc} in the main file, the following
command lines effectively perform a redirection
(note that depending on the shell being used,
backslashes may have to be doubled: `|\|' $\to$ `|\\|'):
%
\begin{center}
|... -jobname "|\textit{target}|" |\\|"|[\textit{flags}]%
|\input{childdoc.def}\childdocforward[|\textit{main}|]{|\textit{dest}|}"|
\end{center}
%
Here \textit{target} is the name of the output file,
\textit{main} is the name of the main file
and \textit{dest} is the name of the main or child file to be processed
(all filenames without extensions).
The optional argument \textit{main} can be omitted
if \textit{main} matches \textit{dest}.
Optionally, compilation \textit{flags} can be defined via |\def| commands.
This command line makes the \TeX{} engine believe
it is compiling the file \textit{target}
whose content is specified as the latter parameter.
The provided code then forwards the processing to
\textit{main} or \textit{dest} as described in \secref{sec:forward}.

%%%%%%%%%%%%%%%%%%%%%%%%%%%%%%%%%%%%%%%%%%%%%%%%%%%%%%%%%%%%%%%%%%%%%%%%%%%%%%%%
\subsection{Include by Input}
\label{sec:input}

Including child documents by |\include| has some restrictions by design.
Most notably, the content of a child document always occupies
its own set of pages; pages cannot be shared between child documents.
Usually, this behaviour makes perfect sense
because each child document contain an essential part of the document.
However, in some situations it may be desirable to compose
a document from a collection of parts
without having mandatory page breaks between then.
For this case, the package
provides a mechanism to include parts
by |\input| which can also be processed individually.
However, by construction this mechanism
requires manual handling of the content to be output.

%%%%%%%%%%%%%%%%%%%%%%%%%%%%%%%%%%%%%%%%
\DescribeMacro{\ifchilddocmanual}
The main file should be prepared as usual, see \secref{sec:include}.
However, the document body must make a distinction
between processing of an individual part and of the main document, e.g.:
%
\begin{center}
\begin{tabular}{l}
|\ifchilddocmanual|\\
|\input{\childdocname}|\\
|\||else|\\
\textit{document body with }|\input{|\textit{part}|}|\\
|\||fi|
\end{tabular}
\end{center}
%
The conditional |\ifchilddocmanual| is true whenever
a part to be included by |\input| is being compiled,
and the name of the part is stored in |\childdocname|.

%%%%%%%%%%%%%%%%%%%%%%%%%%%%%%%%%%%%%%%%
\DescribeMacro{\childdocby}
Each part to be included by |\input| should start with:
%
\begin{center}
\begin{tabular}{l}
|\input{childdoc.def}|\\
|\childdocby{|\textit{main}|}|\\
\end{tabular}
\end{center}
%
The directive |\childdocby| is similar to |\childdocof|
described in \secref{sec:include},
but the subsequent selection of content must be done manually.
To that end, both |\ifchilddoc| and |\ifchilddocmanual|
will be true upon processing of a part,
and the name of the part is stored in |\childdocname|.
Note that |\jobname| will be set to the filename of the current part
so that each part receives an individual |.aux| file
that does not interfere with the |.aux| file(s) of the main document.
This behaviour can be altered by the alternative form
|\childdocby[*]{|\textit{main}|}| (with a non-empty optional argument)
which uses the |.aux| file of the main document
by setting |\jobname| to \textit{main}.

%%%%%%%%%%%%%%%%%%%%%%%%%%%%%%%%%%%%%%%%%%%%%%%%%%%%%%%%%%%%%%%%%%%%%%%%%%%%%%%%
\subsection{Driver Development}
\label{sec:driver}

The \textsf{childdoc} mechanism can also be use for the development
of definition files such as \LaTeX{} styles or classes.
This case differs from the above setup with multiple parts
included by |\include| in that no |\includeonly| should be invoked.
This can be achieved by starting the include file
(before |\ProvidesPackage|) with:
%
\begin{center}
\begin{tabular}{l}
|\input{childdoc.def}|\\
|\childdocforward{|\textit{main}|}|\\
\end{tabular}
\end{center}
%
or alternatively with:
%
\begin{center}
\begin{tabular}{l}
|\input{childdoc.def}|\\
|\childdocby{|\textit{main}|}|\\
\end{tabular}
\end{center}
%
Both forms have slightly different effects as described above.
The main file is prepared as usual, see \secref{sec:include}.

%%%%%%%%%%%%%%%%%%%%%%%%%%%%%%%%%%%%%%%%%%%%%%%%%%%%%%%%%%%%%%%%%%%%%%%%%%%%%%%%
\subsection{Legacy Detection}
\label{sec:detection}

The directive |\childdocmain| in the main file can detect
whether the complete document or merely a child is to be compiled
even without using the directive |\childdocof|.
This method is deprecated because it is less robust
and there is no compelling reason to use it;
it is merely provided for backward compatibility
and it may be removed in future versions.

If the detection mechanism is to be used,
it is mandatory to correctly specify
the filename of the main file as the argument of |\childdocmain|:
%
\begin{center}
\begin{tabular}{l}
|\input{childdoc.def}|\\
|\childdocmain{|\textit{main}|}|\\
\end{tabular}
\end{center}
%
If |\jobname| does not match the argument \textit{main} of |\childdocmain|,
it is assumed that |\jobname| points to the child file to be compiled.
When using |\childdocmain| with the main file specified as argument,
it suffices to start a child file
with just |\input{|\textit{main}|}|
without loading of the package and using |\childdocof|.
If instead all processing is done
with the appropriate \textsf{childdoc} directives,
the argument of \textit{main} of |\childdocmain| can be empty.

An alternative version of the command line processing described
in \secref{sec:commandline} using the detection mechanism reads:
%
\begin{center}
|... -jobname "|\textit{target}|" "|[\textit{flags}]%
[|\def\jobname{|\textit{dest}|}|]|\input{|\textit{main}|}"|
\end{center}

%%%%%%%%%%%%%%%%%%%%%%%%%%%%%%%%%%%%%%%%%%%%%%%%%%%%%%%%%%%%%%%%%%%%%%%%%%%%%%%%
\subsection{Manual Code}
\label{sec:manual}

In case one cannot be certain whether the definitions file |childdoc.def|
is installed on the target \TeX{} distribution
and one prefers not to ship it,
it is conceivable to paste a few relevant commands into the sources.

To that end, drop all statements |\input{childdoc.def}|
and perform the replacements as outlined below.
Instead of |\childdocmain{|\textit{main}|}| add the following code
to the top of the main file:
%
\begin{center}
\begin{tabular}{l}
|\||ifdefined\childdocname\endinput\||fi\newif\ifchilddoc|\\
|\edef\childdocname{\scantokens\expandafter{\jobname\noexpand}}|\\
|\def\childdocmain{|\textit{main}|}\||ifx\childdocmain\childdocname\||else|\\
|\childdoctrue\includeonly{\childdocname}\let\jobname\childdocmain\||fi|\\
\end{tabular}
\end{center}
%
Instead of |\childdocof{|\textit{main}|}| just include the main file
at the top of each child file:
%
\begin{center}
|\input{|\textit{main}|}|
\end{center}
%
A simple redirection |\childdocforward{|\textit{dest}|}| is achieved by:
%
\begin{center}
|\def\jobname{|\textit{dest}|}\input{\jobname}|
\end{center}
%
The redirection with prefix
|\childdocforwardprefix[|\textit{prefix}|]{|\textit{dest}|}|
is accomplished by:
%
\begin{center}
\begin{tabular}{l}
|{\edef\jobname{\scantokens\expandafter{\jobname\noexpand}}|\\
|\def\redirectjob |\textit{prefix}|#1~~~{\gdef\jobname{|\textit{dest}|#1}}|\\
|\expandafter\redirectjob\jobname~~~}\input{\jobname}|
\end{tabular}
\end{center}

In an alternative approach,
child documents can be compiled by a specific command line
without additional code or specific definitions:
%
\begin{center}
|... -jobname "|\textit{target}|" "|[\textit{flags}]%
|\includeonly{|\textit{dest}|}\input{|\textit{main}|}"|
\end{center}
%

%%%%%%%%%%%%%%%%%%%%%%%%%%%%%%%%%%%%%%%%%%%%%%%%%%%%%%%%%%%%%%%%%%%%%%%%%%%%%%%%
%%%%%%%%%%%%%%%%%%%%%%%%%%%%%%%%%%%%%%%%%%%%%%%%%%%%%%%%%%%%%%%%%%%%%%%%%%%%%%%%
\section{Information}

%%%%%%%%%%%%%%%%%%%%%%%%%%%%%%%%%%%%%%%%%%%%%%%%%%%%%%%%%%%%%%%%%%%%%%%%%%%%%%%%
\subsection{Copyright}

Copyright \copyright{} 2017--2018 Niklas Beisert

This work may be distributed and/or modified under the
conditions of the \LaTeX{} Project Public License, either version 1.3
of this license or (at your option) any later version.
The latest version of this license is in
  \url{http://www.latex-project.org/lppl.txt}
and version 1.3 or later is part of all distributions of \LaTeX{}
version 2005/12/01 or later.

This work has the LPPL maintenance status `maintained'.

The Current Maintainer of this work is Niklas Beisert.

This work consists of the files |README.txt|, |childdoc.ins| and |childdoc.dtx|
as well as the derived files |childdoc.def|, |cdocsamp.tex|
with |cdocsch1.tex|, |cdocsch2.tex|, |cdocspt3.tex|, |cdocspt4.tex|,
|cdocsdrf.tex|, |cdocsfn1.tex|, |cdocsfn2.tex|
as well as |childdoc.pdf|.

%%%%%%%%%%%%%%%%%%%%%%%%%%%%%%%%%%%%%%%%%%%%%%%%%%%%%%%%%%%%%%%%%%%%%%%%%%%%%%%%
\subsection{Files and Installation}

The package consists of the files:
%
\begin{center}
\begin{tabular}{ll}
    |README.txt|   & readme file \\
    |childdoc.ins| & installation file \\
    |childdoc.dtx| & source file \\
    |childdoc.def| & definition file \\
    |cdocsamp.tex| & sample main file \\
    |cdocsch1.tex| & sample include file \\
    |cdocsch2.tex| & sample include file \\
    |cdocspt3.tex| & sample part file \\
    |cdocspt4.tex| & sample part file \\
    |cdocsdrf.tex| & sample redirection file \\
    |cdocsfn1.tex| & sample redirection file \\
    |cdocsfn2.tex| & sample redirection file \\
    |childdoc.pdf| & manual
\end{tabular}
\end{center}
%
The distribution consists of the files
|README.txt|, |childdoc.ins| and |childdoc.dtx|.
%
\begin{itemize}
\item
Run (pdf)\LaTeX{} on |childdoc.dtx|
to compile the manual |childdoc.pdf| (this file).
\item
Run \LaTeX{} on |childdoc.ins| to create the definitions file |childdoc.def|
and the sample |cdocsamp.tex| with include files
|cdocsch1.tex|, |cdocsch2.tex|, |cdocspt3.tex|, |cdocspt4.tex|,
|cdocsdrf.tex|, |cdocsfn1.tex|, |cdocsfn2.tex|.
Then copy the file |childdoc.def| to an appropriate directory of your \LaTeX{}
distribution, e.g.\ \textit{texmf-root}|/tex/latex/childdoc|.
\end{itemize}

%%%%%%%%%%%%%%%%%%%%%%%%%%%%%%%%%%%%%%%%%%%%%%%%%%%%%%%%%%%%%%%%%%%%%%%%%%%%%%%%
\subsection{Related CTAN Packages}

There are several other packages which offer a similar functionality:
%
\begin{itemize}
\item
The packages
\href{http://ctan.org/pkg/docmute}{\textsf{docmute}},
\href{http://ctan.org/pkg/includex}{\textsf{includex}} and
\href{http://ctan.org/pkg/standalone}{\textsf{standalone}}
provide commands to include only the document body of
a child file thus allowing both files to be compiled individually.
\item
The packages \href{http://ctan.org/pkg/subdocs}{\textsf{subdocs}}
and \href{http://ctan.org/pkg/subfiles}{\textsf{subfiles}}
provide structures in which the main and child documents can be
encapsulated and allowing them to be compiled individually.
The inclusion mechanism is different from the conventional |\include|.
\item
The package \href{http://ctan.org/pkg/combine}{\textsf{combine}}
is an elaborate solution to combine several documents into one.
\end{itemize}
%
See also the CTAN topic \href{http://ctan.org/topic/subdocs}{\textsf{subdocs}}
for further related packages.
The present package differs from the above solutions in that
a document structure constructed with the conventional |\include| mechanism
just needs two extra commands at the top of every file
such that all constituent files can be compiled individually.

%%%%%%%%%%%%%%%%%%%%%%%%%%%%%%%%%%%%%%%%%%%%%%%%%%%%%%%%%%%%%%%%%%%%%%%%%%%%%%%%
%\subsection{Feature Suggestions}
%
%The following is a list of features which may be useful for future
%versions of this package:
%%
%\begin{itemize}
%\item
%\ldots
%\end{itemize}

%%%%%%%%%%%%%%%%%%%%%%%%%%%%%%%%%%%%%%%%%%%%%%%%%%%%%%%%%%%%%%%%%%%%%%%%%%%%%%%%
\subsection{Revision History}

%%%%%%%%%%%%%%%%%%%%%%%%%%%%%%%%%%%%%%%%
\paragraph{v2.0:} 2018/12/30

\begin{itemize}
\item
immediate forward processing
\item
added |\childdocby| mechanism
\item
manual restructured
\end{itemize}

%%%%%%%%%%%%%%%%%%%%%%%%%%%%%%%%%%%%%%%%
\paragraph{v1.6:} 2018/01/17

\begin{itemize}
\item
application for development of include files
\item
corrections to manual
\end{itemize}

%%%%%%%%%%%%%%%%%%%%%%%%%%%%%%%%%%%%%%%%
\paragraph{v1.5:} 2017/05/21

\begin{itemize}
\item
more complete structuring introduced
\item
|\childdocof| introduced
\item
|\childdoc| renamed to |\childdocmain|
\item
|\childredirect| renamed to |\childdocforward| and |\childdocforwardprefix|
and functionality expanded
\end{itemize}

%%%%%%%%%%%%%%%%%%%%%%%%%%%%%%%%%%%%%%%%
\paragraph{v1.0:} 2017/04/27

\begin{itemize}
\item
manual and install package
\item
first version published on CTAN
\end{itemize}

%%%%%%%%%%%%%%%%%%%%%%%%%%%%%%%%%%%%%%%%
\paragraph{v0.6:} 2017/04/26

\begin{itemize}
\item
redirection mechanism added
\end{itemize}

%%%%%%%%%%%%%%%%%%%%%%%%%%%%%%%%%%%%%%%%
\paragraph{v0.5:} 2017/04/26

\begin{itemize}
\item
functionality in definition file
\end{itemize}


%%%%%%%%%%%%%%%%%%%%%%%%%%%%%%%%%%%%%%%%%%%%%%%%%%%%%%%%%%%%%%%%%%%%%%%%%%%%%%%%
%%%%%%%%%%%%%%%%%%%%%%%%%%%%%%%%%%%%%%%%%%%%%%%%%%%%%%%%%%%%%%%%%%%%%%%%%%%%%%%%
%%%%%%%%%%%%%%%%%%%%%%%%%%%%%%%%%%%%%%%%%%%%%%%%%%%%%%%%%%%%%%%%%%%%%%%%%%%%%%%%
\appendix

\settowidth\MacroIndent{\rmfamily\scriptsize 000\ }

 \DocInput{childdoc.dtx}

\end{document}
%</driver>
% \fi
%
% %%%%%%%%%%%%%%%%%%%%%%%%%%%%%%%%%%%%%%%%%%%%%%%%%%%%%%%%%%%%%%%%%%%%%%%%%%%%%%
% %%%%%%%%%%%%%%%%%%%%%%%%%%%%%%%%%%%%%%%%%%%%%%%%%%%%%%%%%%%%%%%%%%%%%%%%%%%%%%
% \section{Sample}
%\iffalse
%<*samplemain>
%\fi
%
% The following presents a sample document
% with two chapters, two parts, a title page,
% a compile flag as well as three forwarding files to set the flag.
% It consists of eight |.tex| files:
% \begin{center}
% \begin{tabular}{ll}
% |cdocsamp.tex|&main file\\
% |cdocsch1.tex|&include file for chapter 1\\
% |cdocsch2.tex|&include file for chapter 2\\
% |cdocspt3.tex|&include file for part 3\\
% |cdocspt4.tex|&include file for part 4\\
% |cdocsdrf.tex|&forwarding file for main file in draft mode\\
% |cdocsfi1.tex|&forwarding file for final version of chapter 1\\
% |cdocsfi2.tex|&forwarding file for final version of chapter 2\\
% \end{tabular}
% \end{center}
% Each of the eight files can be compiled directly by the \LaTeX{} compiler.
%
% %%%%%%%%%%%%%%%%%%%%%%%%%%%%%%%%%%%%%%
% \paragraph{Main File.}
%
% The main file is called |cdocsamp.tex|.
%
% Load the \textsf{childdoc} definitions and
% declare the filename for the main document:
%    \begin{macrocode}
\input{childdoc.def}
\childdocmain{}
%    \end{macrocode}

% Optional override for |\version| flag:
%    \begin{macrocode}
%%\ifchilddoc\else\providecommand{\version}{draft}\fi
%    \end{macrocode}

% Define the default values for the |\version| flag
% (|final| for the main file and |draft| for childs):
%    \begin{macrocode}
\ifchilddoc
\providecommand{\version}{draft}
\else
\providecommand{\version}{final}
\fi
%    \end{macrocode}

% Load the standard document class:
%    \begin{macrocode}
\documentclass[12pt]{article}
%    \end{macrocode}

% Start the document body:
%    \begin{macrocode}
\begin{document}
%    \end{macrocode}

% Declare a title page.
% Print title, part of document being processed and version flag:
%    \begin{macrocode}
\addtocounter{page}{-1}
\begin{center}
{\LARGE\bfseries{}childdoc example\par}
\vspace{1cm}
\ifchilddoc
\ifchilddocmanual part\else chapter\fi:
`\childdocname' of `\childdocjob'\par
\else
main document: `\childdocjob'\par
\fi
version: \version\par
\end{center}
\newpage
%    \end{macrocode}

% Manually include selected file,
% otherwise process as usual:
%    \begin{macrocode}
\ifchilddocmanual
\section*{part `\childdocname'}
\input{\childdocname}
\else
%    \end{macrocode}

% Include the two chapters:
%    \begin{macrocode}
\include{cdocsch1}
\include{cdocsch2}
%    \end{macrocode}

% Include the two parts unless only chapters should be displayed:
%    \begin{macrocode}
\ifchilddoc\else
\section{part three}
\input{cdocspt3}
\section{part four}
\input{cdocspt4}
\fi
%    \end{macrocode}

% Process as usual until here:
%    \begin{macrocode}
\fi
%    \end{macrocode}

% End of document body:
%    \begin{macrocode}
\end{document}
%    \end{macrocode}
%\iffalse
%</samplemain>
%\fi
%
% %%%%%%%%%%%%%%%%%%%%%%%%%%%%%%%%%%%%%%
% \paragraph{Chapter Include Files.}
%
% The include files are called |cdocsch1.tex| and |cdocsch2.tex|.
%
%\iffalse
%<*samplechap1|samplechap2>
%\fi

% Optional override for |\version| flag:
%    \begin{macrocode}
%%\providecommand{\version}{final}
%    \end{macrocode}

% Include the main document:
%    \begin{macrocode}
\input{childdoc.def}
\childdocof{cdocsamp}
%    \end{macrocode}

%\iffalse
%</samplechap1|samplechap2>
%\fi
%
%\iffalse
%<*samplechap1>
%\fi
% Some text for chapter 1:
%    \begin{macrocode}
\section{one}
some text in chapter one
%    \end{macrocode}

%\iffalse
%</samplechap1>
%\fi
% Some text for chapter 2:
%\iffalse
%<*samplechap2>
%\fi
%    \begin{macrocode}
\section{two}
more text in chapter two
%    \end{macrocode}

%\iffalse
%</samplechap2>
%\fi
%
% %%%%%%%%%%%%%%%%%%%%%%%%%%%%%%%%%%%%%%
% \paragraph{Part Include Files.}
%
% The include files are called |cdocspt3.tex| and |cdocspt4.tex|.
%
%\iffalse
%<*samplepart3|samplepart4>
%\fi

% Optional override for |\version| flag:
%    \begin{macrocode}
%%\providecommand{\version}{final}
%    \end{macrocode}

% Include the main document:
%    \begin{macrocode}
\input{childdoc.def}
\childdocby{cdocsamp}
%    \end{macrocode}

%\iffalse
%</samplepart3|samplepart4>
%\fi
%
%\iffalse
%<*samplepart3>
%\fi
% Some text for part 3:
%    \begin{macrocode}
some text in part three
%    \end{macrocode}

%\iffalse
%</samplepart3>
%\fi
% Some text for part 4:
%\iffalse
%<*samplepart4>
%\fi
%    \begin{macrocode}
more text in part four
%    \end{macrocode}

%\iffalse
%</samplepart4>
%\fi
%
% %%%%%%%%%%%%%%%%%%%%%%%%%%%%%%%%%%%%%%
% \paragraph{Forwarding for a Complete Draft.}
%
% The following forwarding file |cdocsdrf.tex|
% compiles the main document in draft mode:
%\iffalse
%<*sampledraft>
%\fi
%    \begin{macrocode}
\def\version{draft}
\input{childdoc.def}
\childdocforward{cdocsamp}
%    \end{macrocode}

%\iffalse
%</sampledraft>
%\fi
%
% %%%%%%%%%%%%%%%%%%%%%%%%%%%%%%%%%%%%%%
% \paragraph{Forwarding for Final Version of the Chapters.}
%
% The following forwarding files |cdocsfn1.tex| and |cdocsfn2.tex|
% (with identical content)
% compile the final versions of the child documents
% |cdocsch1.tex| and |cdocsch2.tex|, respectively:
%\iffalse
%<*samplefinal>
%\fi
%    \begin{macrocode}
\def\version{final}
\input{childdoc.def}
\childdocforwardprefix[cdocsamp]{cdocsfn}{cdocsch}
%    \end{macrocode}

%\iffalse
%</samplefinal>
%\fi
%
% %%%%%%%%%%%%%%%%%%%%%%%%%%%%%%%%%%%%%%
% \paragraph{Command Line Processing.}
%
% The following three command lines generate the output files
% |cdocscld|, |cdocscl1| and |cdocscl2|
% which should be identical to
% |cdocsdrf|, |cdocsch1| and |cdocsfn2|, respectively:
% \begin{center}
% \begin{tabular}{l}
% |latex -jobname cdocscld \|\\
% |  "\def\version{draft}\input{childdoc.def}\childdocforward{cdocsamp}"|\\
% |latex -jobname cdocscl1 \|\\
% |  "\input{childdoc.def}\childdocforward[cdocsamp]{cdocsch1}"|\\
% |latex -jobname cdocscl2 \|\\
% |  "\def\version{final}\input{childdoc.def}\childdocforward{cdocsch2}"|
% \end{tabular}
% \end{center}
% Note that the trailing backslash on each first line
% merely continues the input to the second line
% (for convenient cut ant paste).
% Furthermore, the command |latex| can be replaced by any
% of its alternative versions such as |pdflatex|.
%
% %%%%%%%%%%%%%%%%%%%%%%%%%%%%%%%%%%%%%%%%%%%%%%%%%%%%%%%%%%%%%%%%%%%%%%%%%%%%%%
% %%%%%%%%%%%%%%%%%%%%%%%%%%%%%%%%%%%%%%%%%%%%%%%%%%%%%%%%%%%%%%%%%%%%%%%%%%%%%%
% \section{Implementation}
%\iffalse
%<*package>
%\fi
%
% This section describes the definitions file |childdoc.def|.

% The definitions cannot be loaded using |\usepackage| or |\RequirePackage|
% which has a mechanism to prevent loading a style file more than once.
% When loading the definitions by means of |\input|
% multiple instances have to be prevented manually:
%\iffalse
%This code needs to be before the `\ProvidesFile' directive
%which is defined at the beginning of this file.
%Therefore it is also placed there and commented out here.
%</package>
%<*discard>
%\fi
%    \begin{macrocode}
\ifdefined\childdocmain\endinput\fi
%    \end{macrocode}
%\iffalse
%</discard>
%<*package>
%\fi
%
% \macro{\ifchilddoc}
% \macro{\ifchilddocmanual}
% The conditional |\ifchilddoc| tells whether a
% child (true) or main (false) document is being compiled.
% The conditional |\ifchilddocmanual| tells whether
% the |\includeonly| mechanism is used (false) or
% the selection of child files must be performed manually (true).
% The definitions initialise to false:
%    \begin{macrocode}
\newif\ifchilddoc
\newif\ifchilddocmanual
%    \end{macrocode}

% \macro{\childdocname}
% \macro{\childdocjob}
% The macro |\childdocname| stores the name of the main document
% to be compiled. The macro |\childdocjob| stores the name of
% the document on which the \LaTeX{} compiler was originally invoked.
% The content of |\jobname| cannot be compared
% to filenames specified in the source due to different catcodes.
% The following code rescans |\jobname|, stores the result
% in |\childdocname| and saves a copy in |\childdocjob|:
%    \begin{macrocode}
\edef\childdocname{\scantokens\expandafter{\jobname\noexpand}}
\let\childdocjob\childdocname
%    \end{macrocode}

% \macro{\childdocdisable}
% The macro |\childdocdisable| prevents the main file
% from being processed more than once.
% At this stage, the main document command |\childdocmain|
% is assumed to be called once again where it should do nothing.
% Any subsequent call to it should prevent
% a secondary processing of the main document
% It overwrites the forwarding commands
% |\childdocof| and |\childdocforward|
% with empty macros to prevent further inclusions of the main document:
%    \begin{macrocode}
\newcommand{\childdocdisable}
{
  \renewcommand{\childdocmain}[1]{\renewcommand{\childdocmain}[1]{\endinput}}
  \renewcommand{\childdocof}[1]{}
  \renewcommand{\childdocby}[2][]{}
  \renewcommand{\childdocforward}[2][]{}
  \renewcommand{\childdocdisable}{}
}
%    \end{macrocode}

% \macro{\childdocmain}
% The macro |\childdocmain| is to be called at the top of the main file
% with nothing or the main filename (without extension) as argument.
% First, it breaks loops.
% If the argument is not empty and does not match |\childdocname|
% (which is set by the first inclusion of |childdoc.def|),
% |\ifchilddoc| is set to true, |\includeonly| is applied to the child file
% and |\jobname| is set to the main file
% (for proper handling of |.aux| files):
%    \begin{macrocode}
\newcommand{\childdocmain}[1]
{
  \childdocdisable\childdocmain{}
  \if?#1?\else
    \begingroup
      \def\childdoctmp{#1}
      \ifx\childdoctmp\childdocname
        \def\childdoctmp{}
      \else
        \def\childdoctmp
        {
          \childdoctrue
          \includeonly{\childdocname}
          \def\childdocjob{#1}
          \def\jobname{#1}
        }
      \fi
      \expandafter
    \endgroup
    \childdoctmp
  \fi
}
%    \end{macrocode}

% \macro{\childdocof}
% The command |\childdocof| redirects
% compilation to the main file |#1|.
%    \begin{macrocode}
\newcommand{\childdocof}[1]
{
  \childdocdisable
  \childdoctrue
  \includeonly{\childdocname}
  \def\jobname{#1}
  \def\childdocjob{#1}
  \input{#1}
}
%    \end{macrocode}

% \macro{\childdocby}
% The command |\childdocby| ....
%    \begin{macrocode}
\newcommand{\childdocby}[2][]
{
  \childdocdisable
  \childdoctrue
  \childdocmanualtrue
  \if?#1?\else
    \def\jobname{#2}
  \fi
  \def\childdocjob{#2}
  \input{#2}
  \endinput
}
%    \end{macrocode}

% \macro{\childdocforward}
% The command |\childdocforward| redirects
% compilation to the main file or
% (if the optional argument is given) a child file.
% Parameters are set as if the main file
% or a child file starting with |\childdocof| was compiled.
% Then compilation is handed over to the main file:
%    \begin{macrocode}
\newcommand{\childdocforward}[2][]
{
  \begingroup
    \if?#1?
      \def\childdoctmp
      {
        \def\childdocname{#2}
        \def\childdocjob{#2}
        \def\jobname{#2}
        \input{#2}
        \endinput
      }
    \else
      \def\childdoctmp
      {
        \childdocdisable
        \def\childdocname{#2}
        \childdoctrue
        \includeonly{#2}
        \def\childdocjob{#1}
        \def\jobname{#1}
        \input{#1}
        \endinput
      }
    \fi
    \expandafter
  \endgroup
  \childdoctmp
}
%    \end{macrocode}

% \macro{\childdocforwardprefix}
% The command |\childdocforwardprefix| redirects
% compilation to the main or a child file by means of a pattern.
% The prefix |#1| in the current filename is replaced by |#2|
% and the suffix of the current filename is kept
% (it is assumed that the filename does not contain the substring `|~~~|'
% which is used as a delimiter).
% Compilation is handed over to the new file by |\childdocforward|:
%    \begin{macrocode}
\newcommand{\childdocforwardprefix}[3][]
{
  \begingroup
    \def\childdocextract #2##1~~~{\def\childdoctmp{\childdocforward[#1]{#3##1}}}
    \expandafter\childdocextract\childdocname~~~
    \expandafter
  \endgroup
  \childdoctmp
}
%    \end{macrocode}

% \macro{\childdoc}
% The deprecated macro |\childdoc| is a legacy version of |\childdocmain|:
%    \begin{macrocode}
\newcommand{\childdoc}{\childdocmain}
%    \end{macrocode}

% \macro{\childdocredirect}
% The deprecated macro |\childdocredirect| is a legacy version
% of |\childdocforward| and |\childdocforwardprefix|:
%    \begin{macrocode}
\newcommand{\childdocredirect}[2][]
{
  \begingroup
    \if?#1?
      \def\childdoctmp{\childdocforward{#2}}
    \else
      \def\childdoctmp{\childdocforwardprefix{#1}{#2}}
    \fi
    \expandafter
  \endgroup
  \childdoctmp
}
%    \end{macrocode}

%\iffalse
%</package>
%\fi
%
\endinput
\childdocforward{cdocsamp}"|\\
% |latex -jobname cdocscl1 \|\\
% |  "% \iffalse
%
% childdoc.dtx Copyright (C) 2017-2018 Niklas Beisert
%
% This work may be distributed and/or modified under the
% conditions of the LaTeX Project Public License, either version 1.3
% of this license or (at your option) any later version.
% The latest version of this license is in
%   http://www.latex-project.org/lppl.txt
% and version 1.3 or later is part of all distributions of LaTeX
% version 2005/12/01 or later.
%
% This work has the LPPL maintenance status `maintained'.
%
% The Current Maintainer of this work is Niklas Beisert.
%
% This work consists of the files childdoc.dtx and childdoc.ins
% and the derived files childdoc.def and cdocsamp.tex with
% cdocsch1.tex, cdocsch2.tex, cdocsdrf.tex, cdocsfn1.tex, cdocsfn2.tex.
%
%<package>\ifdefined\childdocmain\endinput\fi
%<package>\ProvidesFile{childdoc.def}[2018/12/30 v2.0 child document driver]
%<samplemain>\ProvidesFile{cdocsamp.tex}[2018/12/30 v2.0 sample for childdoc]
%<*driver>
%\ProvidesFile{childdoc.drv}[2018/12/30 v2.0 childdoc reference manual file]
\PassOptionsToClass{10pt,a4paper}{article}
\documentclass{ltxdoc}

\usepackage[margin=35mm]{geometry}
\usepackage{hyperref}
\usepackage{hyperxmp}
\usepackage[usenames]{color}

\hypersetup{colorlinks=true}
\hypersetup{pdfstartview=FitH}
\hypersetup{pdfpagemode=UseNone}
\hypersetup{pdfsource={}}
\hypersetup{pdflang={en-UK}}
\hypersetup{pdfcopyright={Copyright 2017-2018 Niklas Beisert.
  This work may be distributed and/or modified under the
  conditions of the LaTeX Project Public License, either version 1.3
  of this license or (at your option) any later version.}}
\hypersetup{pdflicenseurl={http://www.latex-project.org/lppl.txt}}
\hypersetup{pdfcontactaddress={ETH Zurich, ITP, HIT K,
  Wolfgang-Pauli-Strasse 27}}
\hypersetup{pdfcontactpostcode={8093}}
\hypersetup{pdfcontactcity={Zurich}}
\hypersetup{pdfcontactcountry={Switzerland}}
\hypersetup{pdfcontactemail={nbeisert@itp.phys.ethz.ch}}
\hypersetup{pdfcontacturl={http://people.phys.ethz.ch/\xmptilde nbeisert/}}

\newcommand{\secref}[1]{\hyperref[#1]{section \ref*{#1}}}

\parskip1ex
\parindent0pt
\let\olditemize\itemize
\def\itemize{\olditemize\parskip0pt}

\begin{document}

\title{The \textsf{childdoc} Package}
\hypersetup{pdftitle={The childdoc Package}}
\author{Niklas Beisert\\[2ex]
  Institut f\"ur Theoretische Physik\\
  Eidgen\"ossische Technische Hochschule Z\"urich\\
  Wolfgang-Pauli-Strasse 27, 8093 Z\"urich, Switzerland\\[1ex]
  \href{mailto:nbeisert@itp.phys.ethz.ch}
  {\texttt{nbeisert@itp.phys.ethz.ch}}}
\hypersetup{pdfauthor={Niklas Beisert}}
\hypersetup{pdfsubject={Manual for the LaTeX2e Package childdoc}}
\date{30 December 2018, \textsf{v2.0}}
\maketitle

\begin{abstract}\noindent
\textsf{childdoc} is a \LaTeXe{} package
that enables the direct compilation
of document sections included by |\include|
to individual files.
\end{abstract}

\begingroup
\parskip0ex
\tableofcontents
\endgroup

%%%%%%%%%%%%%%%%%%%%%%%%%%%%%%%%%%%%%%%%%%%%%%%%%%%%%%%%%%%%%%%%%%%%%%%%%%%%%%%%
%%%%%%%%%%%%%%%%%%%%%%%%%%%%%%%%%%%%%%%%%%%%%%%%%%%%%%%%%%%%%%%%%%%%%%%%%%%%%%%%
\section{Introduction}

\LaTeX{} provides a mechanism to structure a large document (such as a book)
into a main file and several child files (containing the chapters)
using the |\include| command.
This mechanism is beneficial for documents
which span hundreds of pages in order to
make the source file(s) more manageable.
Moreover, compilation can be restricted to
selected child files by means of the |\includeonly| command.
The latter feature can be used to reduce the compilation time while editing
(this was significantly more useful in the earlier days of \LaTeX{})
or to generate a smaller document which is easier to navigate.
Another application of |\includeonly| is to generate
documents consisting of selected parts of the complete document.

However, there are a few drawbacks of the plain |\include| mechanism:
\begin{itemize}
\item
The child files cannot be compiled on their own,
they can only be compiled via the main file.
A naive editing environment
(such as a text editor with an option
to have the current file processed by \LaTeX)
may require one to switch to the main file before compiling;
attempting to compile the child file produces errors.
\item
The main file must be modified (each time)
to adjust the |\includeonly| command
to the present needs. This easily leaves the main file in a messy state.
\item
The generated document will always carry the filename
of the main document. This is inconvenient if
several child files are to be compiled and
to be kept for distribution.
\end{itemize}

The present package provides a simple interface
to make child files individually compilable by \LaTeX{}.
Compiling a child file then has the same effect as compiling
the main file with an |\includeonly| command
to select the appropriate child.
Moreover the generated document will carry the name of the child
rather than the main file.
This resolves all three above issues.

This feature is meant to make the editing of books,
thesis documents and lecture notes somewhat more convenient.
However, the package can also be used efficiently for
composing a series of documents (such as exercise sheets)
which are typically distributed individually.
It then assists the author in generating the individual documents
(potentially in different versions)
as well as a document containing the collected series.
Another application is in developing style files
or other kinds of included material
where compilation of the style file could redirect
to a sample or test file.

%%%%%%%%%%%%%%%%%%%%%%%%%%%%%%%%%%%%%%%%%%%%%%%%%%%%%%%%%%%%%%%%%%%%%%%%%%%%%%%%
%%%%%%%%%%%%%%%%%%%%%%%%%%%%%%%%%%%%%%%%%%%%%%%%%%%%%%%%%%%%%%%%%%%%%%%%%%%%%%%%
\section{Usage}

First of all, the package \textsf{childdoc} is \emph{not} a standard
\LaTeXe{} |.sty| style file! Therefore it needs to be invoked in
a non-standard way.

%%%%%%%%%%%%%%%%%%%%%%%%%%%%%%%%%%%%%%%%%%%%%%%%%%%%%%%%%%%%%%%%%%%%%%%%%%%%%%%%
\subsection{Included Files}
\label{sec:include}

%%%%%%%%%%%%%%%%%%%%%%%%%%%%%%%%%%%%%%%%
\DescribeMacro{\childdocmain}
To use the package, add the commands
\begin{center}
\begin{tabular}{l}
|\input{childdoc.def}|\\
|\childdocmain{}|\\
\end{tabular}
\end{center}
at the very top of the main \LaTeX{} file,
in particular \emph{before} the |\documentclass| statement!
The argument of |\childdocmain| should be left empty
(but it must be present).

%%%%%%%%%%%%%%%%%%%%%%%%%%%%%%%%%%%%%%%%
\DescribeMacro{\childdocof}
Furthermore, add the commands
\begin{center}
\begin{tabular}{l}
|\input{childdoc.def}|\\
|\childdocof{|\textit{main}|}|\\
\end{tabular}
\end{center}
at the top of every child file \textit{child}
which is included by |\include{|\textit{child}|}|
from within the main file
(or at least for those files to be compiled individually).
The argument \textit{main} must be the filename of the main file.

There are a couple of
considerations in setting up the main and child documents:

%%%%%%%%%%%%%%%%%%%%%%%%%%%%%%%%%%%%%%%%
\paragraph{Restrictions.}

Please note the following restrictions:
\begin{itemize}
\item
|\childdocmain| must be called with one argument \textit{main}
to ensure compatibility with earlier version of the package.
It must either be empty (|\childdocmain{}|)
or precisely match the filename of the main file in which it is specified.
See \secref{sec:detection} for further information.
\item
The filename \textit{main} must be specified without the |.tex| extension.
\item
The filename \textit{main} is case sensitive
(even in case-insensitive file systems)
due to internal string comparison.
\item
The argument \textit{main} should be fully expanded, it cannot be a macro.
\item
Subdirectories and special characters should be avoided in filenames.
\item
The command |\childdocmain{|\textit{main}|}| must be followed by a whitespace.
It should not be followed immediately by another command
or by a comment mark `|%|'.
This is because the \TeX{} parser reads the token immediately following
the argument of |\childdocmain| and puts it
at the beginning of every child section;
however, a white\-space is ignored.
\end{itemize}

%%%%%%%%%%%%%%%%%%%%%%%%%%%%%%%%%%%%%%%%
\paragraph{Content of Main File.}

It is advisable to place all content in the child files included by |\include|.
Any output contained in the main file will appear in all child documents
unless suppressed manually;
it cannot be suppressed automatically by the |\includeonly| directive
and thus should normally be avoided.
A method to include some content in the main file
by means of conditional processing is described in \secref{sec:conditional}.

%%%%%%%%%%%%%%%%%%%%%%%%%%%%%%%%%%%%%%%%
\paragraph{Page Numbering.}

When only a part of the document is compiled,
the appropriate numbering of pages
(as well as other status parameters)
is determined from the |.aux| files.
The latter contain information from previous passes.
However this information needs to propagate through
all intermediate child documents.
Therefore the page numbering in child documents may well
be inconsistent until the complete document is compiled at least once.

A useful (if unconventional) way to always ensure a consistent
page numbering is to restart the numbering in each child document
and denote the pages by `\textit{child}|.|\textit{page}'
where \textit{child} represents the chapter/section number of the child file.
This can be achieved by the command
|\numberwithin{page}{|\textit{child}|}|
of the \textsf{amsmath} package
where \textit{child} can be |chapter| or |section|
depending on the chosen structuring.
Alternatively, one can modify the macro |\thepage| appropriately
and reset the counter |page| at the start of each child file.

%%%%%%%%%%%%%%%%%%%%%%%%%%%%%%%%%%%%%%%%%%%%%%%%%%%%%%%%%%%%%%%%%%%%%%%%%%%%%%%%
\subsection{Conditional Processing}
\label{sec:conditional}

The package provides a mechanism to compile different versions
of a document. To customise the versions further some conditional processing
can come in handy to distinguish which version is being compiled.
The package provides two macros to describe the compilation context:

%%%%%%%%%%%%%%%%%%%%%%%%%%%%%%%%%%%%%%%%
\DescribeMacro{\ifchilddoc}
The conditional |\ifchilddoc| distinguishes between the compilation of
child documents and the main document:
%
\begin{center}
|\ifchilddoc |\textit{child-code}| |[|\||else |\textit{main-code}]| \||fi|
\end{center}

%%%%%%%%%%%%%%%%%%%%%%%%%%%%%%%%%%%%%%%%
\DescribeMacro{\childdocname}
\DescribeMacro{\childdocjob}
The macro |\childdocname| contains the filename (without extension)
of the main or child file being processed.
Note that |\childdocjob| will always contain the name of the main file.

%%%%%%%%%%%%%%%%%%%%%%%%%%%%%%%%%%%%%%%%
\paragraph{Title Page.}

Conditional processing can be used to include a title or banner page
in the main document when proper precautions are taken.
Importantly, the code in the main file should ensure that the page counter
(as well as other status parameters which are stored in the |.aux| files)
takes the same value after the conditional processing.
Otherwise the page numbers may take divergent values
depending on which part is compiled.

For example, a title page could be declared by:
%
\begin{center}
\begin{tabular}{l}
|\ifchilddoc\||else|\\
|\addtocounter{page}{-1}|\\
\textit{code for title page}\\
|\newpage|\\
|\||fi|
\end{tabular}
\end{center}
%
A banner page for the child documents can be generated by:
%
\begin{center}
\begin{tabular}{l}
|\ifchilddoc|\\
|\addtocounter{page}{-1}|\\
\textit{code for banner page}\\
|\newpage|\\
|\||fi|
\end{tabular}
\end{center}
%
Here one could write a message such as:
\begin{center}
|This is the part \childdocname{} of \childdocjob{}.|
\end{center}

%%%%%%%%%%%%%%%%%%%%%%%%%%%%%%%%%%%%%%%%%%%%%%%%%%%%%%%%%%%%%%%%%%%%%%%%%%%%%%%%
\subsection{Flags}
\label{sec:flags}

The package makes it easy to generate different versions
of the main or child documents.
To this end compilation flags can be defined
and assigned different default values.
They will be particularly useful in conjunction
with the forwarding mechanism described in \secref{sec:forward}.

For example, it may be useful to have a flag |\version|
which can be set to |draft| or |final|.
The document source will contain some conditional code
depending on the value of |\version|.
Suppose further, the flag should default to |final| for the main file
and to |draft| for child files
which is a natural assignment for editing the document.
This is achieved by placing the following code
in the preamble of the main document
(below the |\childdocmain| directive):
%
\begin{center}
\begin{tabular}{l}
|\ifchilddoc|\\
|\providecommand{\version}{draft}|\\
|\||else|\\
|\providecommand{\version}{final}|\\
|\||fi|
\end{tabular}
\end{center}
%
The definition by |\providecommand| makes sure
that previous definitions are not overwritten.
Further statements |\providecommand{\version}{...}|
can thus be added before the above code to override it.

For the main file, one might add a line
(between |\childdocmain| and the above block)
%
\begin{center}
|%\ifchilddoc\||else\providecommand{\version}{draft}\||fi|
\end{center}
%
which can be uncommented to produce a draft version.
Likewise one can add a line to the very top of a child file
(above the |\childdocof{|\textit{main}|}| directive)
%
\begin{center}
|%\providecommand{\version}{final}|
\end{center}
%
which can be uncommented to produce the final version of this child document.

%%%%%%%%%%%%%%%%%%%%%%%%%%%%%%%%%%%%%%%%%%%%%%%%%%%%%%%%%%%%%%%%%%%%%%%%%%%%%%%%
\subsection{Forwarding}
\label{sec:forward}

Different versions of the main or child documents
using compilation flags as described in \secref{sec:flags}
can be (permanently) stored in different files
for convenient compilation, viewing and distribution.
To this end, the package defines a command
to pass on compilation to a different file:

%%%%%%%%%%%%%%%%%%%%%%%%%%%%%%%%%%%%%%%%
\DescribeMacro{\childdocforward}
The command |\childdocforward| redirects processing to
another source file:
%
\begin{center}
\begin{tabular}{l}
|\input{childdoc.def}|\\
|\childdocforward[|\textit{main}|]{|\textit{dest}|}|\\
\end{tabular}
\end{center}
%
The argument \textit{dest} is the destination file
(without extension).
It should be the main file or one of the child files.
Note that further \textsf{childdoc} directives
such as |\childdocof| and |\childdocforward|
in the indicated file will be processed in this form.
The optional argument \textit{main}
passes on directly to the main file \textit{main}
while pretending to compile the child \textit{dest}.
This form behaves as if \textit{dest}
issues |\childdocof{|\textit{main}|}| right away,
and no further \textsf{childdoc} directives will be processed.

%%%%%%%%%%%%%%%%%%%%%%%%%%%%%%%%%%%%%%%%
\DescribeMacro{\...prefix}
In the alternative form |\childdocforwardprefix|,
%
\begin{center}
\begin{tabular}{l}
|\input{childdoc.def}|\\
|\childdocforwardprefix[|\textit{main}|]{|\textit{prefix}|}{|\textit{dest}|}|
\end{tabular}
\end{center}
%
the destination file is determined by a pattern
depending on the current file:
To make this work, the current file must be called
`{\textit{prefix}\hspace{0.2em}\textit{suffix}}'
with \textit{prefix} matching precisely the argument.
Processing is then passed on to the file
`{\textit{dest}\hspace{0.2em}\textit{suffix}}'.
Surely, the same effect is achieved by
directly specifying the
argument `{\textit{dest}\hspace{0.2em}\textit{suffix}}'
in the first form.
However, that requires to set up a different file
for each child. With the alternative form of the command
all these files can have exactly the same content
which simplifies setting them up and maintaining them.

For example, the following file |draft.tex|
with a compilation flag |\version| as described in \secref{sec:flags}
compiles the main document as a draft:
%
\begin{center}
\begin{tabular}{l}
|\def\version{draft}|\\
|\input{childdoc.def}|\\
|\childdocforward{|\textit{main}|}|
\end{tabular}
\end{center}
%
Likewise, the following files |final|\textit{nn}|.tex|
compile the final version of the child document
|child|\textit{nn}|.tex|:
%
\begin{center}
\begin{tabular}{l}
|\def\version{final}|\\
|\input{childdoc.def}|\\
|\childdocforwardprefix{final}{child}|
\end{tabular}
\end{center}
%

Note that when several versions of a main file and/or of each child file
are to be generated, it may be convenient to set up a |Makefile| or
shell script to automatise the process.

%%%%%%%%%%%%%%%%%%%%%%%%%%%%%%%%%%%%%%%%%%%%%%%%%%%%%%%%%%%%%%%%%%%%%%%%%%%%%%%%
\subsection{Command Line Processing}
\label{sec:commandline}

The effect of redirection files can also be achieved by invoking
the \LaTeX{} compiler with a more elaborate command line.
Most conveniently this should be done as part
of a shell script or a |Makefile|.

When using \textsf{childdoc} in the main file, the following
command lines effectively perform a redirection
(note that depending on the shell being used,
backslashes may have to be doubled: `|\|' $\to$ `|\\|'):
%
\begin{center}
|... -jobname "|\textit{target}|" |\\|"|[\textit{flags}]%
|\input{childdoc.def}\childdocforward[|\textit{main}|]{|\textit{dest}|}"|
\end{center}
%
Here \textit{target} is the name of the output file,
\textit{main} is the name of the main file
and \textit{dest} is the name of the main or child file to be processed
(all filenames without extensions).
The optional argument \textit{main} can be omitted
if \textit{main} matches \textit{dest}.
Optionally, compilation \textit{flags} can be defined via |\def| commands.
This command line makes the \TeX{} engine believe
it is compiling the file \textit{target}
whose content is specified as the latter parameter.
The provided code then forwards the processing to
\textit{main} or \textit{dest} as described in \secref{sec:forward}.

%%%%%%%%%%%%%%%%%%%%%%%%%%%%%%%%%%%%%%%%%%%%%%%%%%%%%%%%%%%%%%%%%%%%%%%%%%%%%%%%
\subsection{Include by Input}
\label{sec:input}

Including child documents by |\include| has some restrictions by design.
Most notably, the content of a child document always occupies
its own set of pages; pages cannot be shared between child documents.
Usually, this behaviour makes perfect sense
because each child document contain an essential part of the document.
However, in some situations it may be desirable to compose
a document from a collection of parts
without having mandatory page breaks between then.
For this case, the package
provides a mechanism to include parts
by |\input| which can also be processed individually.
However, by construction this mechanism
requires manual handling of the content to be output.

%%%%%%%%%%%%%%%%%%%%%%%%%%%%%%%%%%%%%%%%
\DescribeMacro{\ifchilddocmanual}
The main file should be prepared as usual, see \secref{sec:include}.
However, the document body must make a distinction
between processing of an individual part and of the main document, e.g.:
%
\begin{center}
\begin{tabular}{l}
|\ifchilddocmanual|\\
|\input{\childdocname}|\\
|\||else|\\
\textit{document body with }|\input{|\textit{part}|}|\\
|\||fi|
\end{tabular}
\end{center}
%
The conditional |\ifchilddocmanual| is true whenever
a part to be included by |\input| is being compiled,
and the name of the part is stored in |\childdocname|.

%%%%%%%%%%%%%%%%%%%%%%%%%%%%%%%%%%%%%%%%
\DescribeMacro{\childdocby}
Each part to be included by |\input| should start with:
%
\begin{center}
\begin{tabular}{l}
|\input{childdoc.def}|\\
|\childdocby{|\textit{main}|}|\\
\end{tabular}
\end{center}
%
The directive |\childdocby| is similar to |\childdocof|
described in \secref{sec:include},
but the subsequent selection of content must be done manually.
To that end, both |\ifchilddoc| and |\ifchilddocmanual|
will be true upon processing of a part,
and the name of the part is stored in |\childdocname|.
Note that |\jobname| will be set to the filename of the current part
so that each part receives an individual |.aux| file
that does not interfere with the |.aux| file(s) of the main document.
This behaviour can be altered by the alternative form
|\childdocby[*]{|\textit{main}|}| (with a non-empty optional argument)
which uses the |.aux| file of the main document
by setting |\jobname| to \textit{main}.

%%%%%%%%%%%%%%%%%%%%%%%%%%%%%%%%%%%%%%%%%%%%%%%%%%%%%%%%%%%%%%%%%%%%%%%%%%%%%%%%
\subsection{Driver Development}
\label{sec:driver}

The \textsf{childdoc} mechanism can also be use for the development
of definition files such as \LaTeX{} styles or classes.
This case differs from the above setup with multiple parts
included by |\include| in that no |\includeonly| should be invoked.
This can be achieved by starting the include file
(before |\ProvidesPackage|) with:
%
\begin{center}
\begin{tabular}{l}
|\input{childdoc.def}|\\
|\childdocforward{|\textit{main}|}|\\
\end{tabular}
\end{center}
%
or alternatively with:
%
\begin{center}
\begin{tabular}{l}
|\input{childdoc.def}|\\
|\childdocby{|\textit{main}|}|\\
\end{tabular}
\end{center}
%
Both forms have slightly different effects as described above.
The main file is prepared as usual, see \secref{sec:include}.

%%%%%%%%%%%%%%%%%%%%%%%%%%%%%%%%%%%%%%%%%%%%%%%%%%%%%%%%%%%%%%%%%%%%%%%%%%%%%%%%
\subsection{Legacy Detection}
\label{sec:detection}

The directive |\childdocmain| in the main file can detect
whether the complete document or merely a child is to be compiled
even without using the directive |\childdocof|.
This method is deprecated because it is less robust
and there is no compelling reason to use it;
it is merely provided for backward compatibility
and it may be removed in future versions.

If the detection mechanism is to be used,
it is mandatory to correctly specify
the filename of the main file as the argument of |\childdocmain|:
%
\begin{center}
\begin{tabular}{l}
|\input{childdoc.def}|\\
|\childdocmain{|\textit{main}|}|\\
\end{tabular}
\end{center}
%
If |\jobname| does not match the argument \textit{main} of |\childdocmain|,
it is assumed that |\jobname| points to the child file to be compiled.
When using |\childdocmain| with the main file specified as argument,
it suffices to start a child file
with just |\input{|\textit{main}|}|
without loading of the package and using |\childdocof|.
If instead all processing is done
with the appropriate \textsf{childdoc} directives,
the argument of \textit{main} of |\childdocmain| can be empty.

An alternative version of the command line processing described
in \secref{sec:commandline} using the detection mechanism reads:
%
\begin{center}
|... -jobname "|\textit{target}|" "|[\textit{flags}]%
[|\def\jobname{|\textit{dest}|}|]|\input{|\textit{main}|}"|
\end{center}

%%%%%%%%%%%%%%%%%%%%%%%%%%%%%%%%%%%%%%%%%%%%%%%%%%%%%%%%%%%%%%%%%%%%%%%%%%%%%%%%
\subsection{Manual Code}
\label{sec:manual}

In case one cannot be certain whether the definitions file |childdoc.def|
is installed on the target \TeX{} distribution
and one prefers not to ship it,
it is conceivable to paste a few relevant commands into the sources.

To that end, drop all statements |\input{childdoc.def}|
and perform the replacements as outlined below.
Instead of |\childdocmain{|\textit{main}|}| add the following code
to the top of the main file:
%
\begin{center}
\begin{tabular}{l}
|\||ifdefined\childdocname\endinput\||fi\newif\ifchilddoc|\\
|\edef\childdocname{\scantokens\expandafter{\jobname\noexpand}}|\\
|\def\childdocmain{|\textit{main}|}\||ifx\childdocmain\childdocname\||else|\\
|\childdoctrue\includeonly{\childdocname}\let\jobname\childdocmain\||fi|\\
\end{tabular}
\end{center}
%
Instead of |\childdocof{|\textit{main}|}| just include the main file
at the top of each child file:
%
\begin{center}
|\input{|\textit{main}|}|
\end{center}
%
A simple redirection |\childdocforward{|\textit{dest}|}| is achieved by:
%
\begin{center}
|\def\jobname{|\textit{dest}|}\input{\jobname}|
\end{center}
%
The redirection with prefix
|\childdocforwardprefix[|\textit{prefix}|]{|\textit{dest}|}|
is accomplished by:
%
\begin{center}
\begin{tabular}{l}
|{\edef\jobname{\scantokens\expandafter{\jobname\noexpand}}|\\
|\def\redirectjob |\textit{prefix}|#1~~~{\gdef\jobname{|\textit{dest}|#1}}|\\
|\expandafter\redirectjob\jobname~~~}\input{\jobname}|
\end{tabular}
\end{center}

In an alternative approach,
child documents can be compiled by a specific command line
without additional code or specific definitions:
%
\begin{center}
|... -jobname "|\textit{target}|" "|[\textit{flags}]%
|\includeonly{|\textit{dest}|}\input{|\textit{main}|}"|
\end{center}
%

%%%%%%%%%%%%%%%%%%%%%%%%%%%%%%%%%%%%%%%%%%%%%%%%%%%%%%%%%%%%%%%%%%%%%%%%%%%%%%%%
%%%%%%%%%%%%%%%%%%%%%%%%%%%%%%%%%%%%%%%%%%%%%%%%%%%%%%%%%%%%%%%%%%%%%%%%%%%%%%%%
\section{Information}

%%%%%%%%%%%%%%%%%%%%%%%%%%%%%%%%%%%%%%%%%%%%%%%%%%%%%%%%%%%%%%%%%%%%%%%%%%%%%%%%
\subsection{Copyright}

Copyright \copyright{} 2017--2018 Niklas Beisert

This work may be distributed and/or modified under the
conditions of the \LaTeX{} Project Public License, either version 1.3
of this license or (at your option) any later version.
The latest version of this license is in
  \url{http://www.latex-project.org/lppl.txt}
and version 1.3 or later is part of all distributions of \LaTeX{}
version 2005/12/01 or later.

This work has the LPPL maintenance status `maintained'.

The Current Maintainer of this work is Niklas Beisert.

This work consists of the files |README.txt|, |childdoc.ins| and |childdoc.dtx|
as well as the derived files |childdoc.def|, |cdocsamp.tex|
with |cdocsch1.tex|, |cdocsch2.tex|, |cdocspt3.tex|, |cdocspt4.tex|,
|cdocsdrf.tex|, |cdocsfn1.tex|, |cdocsfn2.tex|
as well as |childdoc.pdf|.

%%%%%%%%%%%%%%%%%%%%%%%%%%%%%%%%%%%%%%%%%%%%%%%%%%%%%%%%%%%%%%%%%%%%%%%%%%%%%%%%
\subsection{Files and Installation}

The package consists of the files:
%
\begin{center}
\begin{tabular}{ll}
    |README.txt|   & readme file \\
    |childdoc.ins| & installation file \\
    |childdoc.dtx| & source file \\
    |childdoc.def| & definition file \\
    |cdocsamp.tex| & sample main file \\
    |cdocsch1.tex| & sample include file \\
    |cdocsch2.tex| & sample include file \\
    |cdocspt3.tex| & sample part file \\
    |cdocspt4.tex| & sample part file \\
    |cdocsdrf.tex| & sample redirection file \\
    |cdocsfn1.tex| & sample redirection file \\
    |cdocsfn2.tex| & sample redirection file \\
    |childdoc.pdf| & manual
\end{tabular}
\end{center}
%
The distribution consists of the files
|README.txt|, |childdoc.ins| and |childdoc.dtx|.
%
\begin{itemize}
\item
Run (pdf)\LaTeX{} on |childdoc.dtx|
to compile the manual |childdoc.pdf| (this file).
\item
Run \LaTeX{} on |childdoc.ins| to create the definitions file |childdoc.def|
and the sample |cdocsamp.tex| with include files
|cdocsch1.tex|, |cdocsch2.tex|, |cdocspt3.tex|, |cdocspt4.tex|,
|cdocsdrf.tex|, |cdocsfn1.tex|, |cdocsfn2.tex|.
Then copy the file |childdoc.def| to an appropriate directory of your \LaTeX{}
distribution, e.g.\ \textit{texmf-root}|/tex/latex/childdoc|.
\end{itemize}

%%%%%%%%%%%%%%%%%%%%%%%%%%%%%%%%%%%%%%%%%%%%%%%%%%%%%%%%%%%%%%%%%%%%%%%%%%%%%%%%
\subsection{Related CTAN Packages}

There are several other packages which offer a similar functionality:
%
\begin{itemize}
\item
The packages
\href{http://ctan.org/pkg/docmute}{\textsf{docmute}},
\href{http://ctan.org/pkg/includex}{\textsf{includex}} and
\href{http://ctan.org/pkg/standalone}{\textsf{standalone}}
provide commands to include only the document body of
a child file thus allowing both files to be compiled individually.
\item
The packages \href{http://ctan.org/pkg/subdocs}{\textsf{subdocs}}
and \href{http://ctan.org/pkg/subfiles}{\textsf{subfiles}}
provide structures in which the main and child documents can be
encapsulated and allowing them to be compiled individually.
The inclusion mechanism is different from the conventional |\include|.
\item
The package \href{http://ctan.org/pkg/combine}{\textsf{combine}}
is an elaborate solution to combine several documents into one.
\end{itemize}
%
See also the CTAN topic \href{http://ctan.org/topic/subdocs}{\textsf{subdocs}}
for further related packages.
The present package differs from the above solutions in that
a document structure constructed with the conventional |\include| mechanism
just needs two extra commands at the top of every file
such that all constituent files can be compiled individually.

%%%%%%%%%%%%%%%%%%%%%%%%%%%%%%%%%%%%%%%%%%%%%%%%%%%%%%%%%%%%%%%%%%%%%%%%%%%%%%%%
%\subsection{Feature Suggestions}
%
%The following is a list of features which may be useful for future
%versions of this package:
%%
%\begin{itemize}
%\item
%\ldots
%\end{itemize}

%%%%%%%%%%%%%%%%%%%%%%%%%%%%%%%%%%%%%%%%%%%%%%%%%%%%%%%%%%%%%%%%%%%%%%%%%%%%%%%%
\subsection{Revision History}

%%%%%%%%%%%%%%%%%%%%%%%%%%%%%%%%%%%%%%%%
\paragraph{v2.0:} 2018/12/30

\begin{itemize}
\item
immediate forward processing
\item
added |\childdocby| mechanism
\item
manual restructured
\end{itemize}

%%%%%%%%%%%%%%%%%%%%%%%%%%%%%%%%%%%%%%%%
\paragraph{v1.6:} 2018/01/17

\begin{itemize}
\item
application for development of include files
\item
corrections to manual
\end{itemize}

%%%%%%%%%%%%%%%%%%%%%%%%%%%%%%%%%%%%%%%%
\paragraph{v1.5:} 2017/05/21

\begin{itemize}
\item
more complete structuring introduced
\item
|\childdocof| introduced
\item
|\childdoc| renamed to |\childdocmain|
\item
|\childredirect| renamed to |\childdocforward| and |\childdocforwardprefix|
and functionality expanded
\end{itemize}

%%%%%%%%%%%%%%%%%%%%%%%%%%%%%%%%%%%%%%%%
\paragraph{v1.0:} 2017/04/27

\begin{itemize}
\item
manual and install package
\item
first version published on CTAN
\end{itemize}

%%%%%%%%%%%%%%%%%%%%%%%%%%%%%%%%%%%%%%%%
\paragraph{v0.6:} 2017/04/26

\begin{itemize}
\item
redirection mechanism added
\end{itemize}

%%%%%%%%%%%%%%%%%%%%%%%%%%%%%%%%%%%%%%%%
\paragraph{v0.5:} 2017/04/26

\begin{itemize}
\item
functionality in definition file
\end{itemize}


%%%%%%%%%%%%%%%%%%%%%%%%%%%%%%%%%%%%%%%%%%%%%%%%%%%%%%%%%%%%%%%%%%%%%%%%%%%%%%%%
%%%%%%%%%%%%%%%%%%%%%%%%%%%%%%%%%%%%%%%%%%%%%%%%%%%%%%%%%%%%%%%%%%%%%%%%%%%%%%%%
%%%%%%%%%%%%%%%%%%%%%%%%%%%%%%%%%%%%%%%%%%%%%%%%%%%%%%%%%%%%%%%%%%%%%%%%%%%%%%%%
\appendix

\settowidth\MacroIndent{\rmfamily\scriptsize 000\ }

 \DocInput{childdoc.dtx}

\end{document}
%</driver>
% \fi
%
% %%%%%%%%%%%%%%%%%%%%%%%%%%%%%%%%%%%%%%%%%%%%%%%%%%%%%%%%%%%%%%%%%%%%%%%%%%%%%%
% %%%%%%%%%%%%%%%%%%%%%%%%%%%%%%%%%%%%%%%%%%%%%%%%%%%%%%%%%%%%%%%%%%%%%%%%%%%%%%
% \section{Sample}
%\iffalse
%<*samplemain>
%\fi
%
% The following presents a sample document
% with two chapters, two parts, a title page,
% a compile flag as well as three forwarding files to set the flag.
% It consists of eight |.tex| files:
% \begin{center}
% \begin{tabular}{ll}
% |cdocsamp.tex|&main file\\
% |cdocsch1.tex|&include file for chapter 1\\
% |cdocsch2.tex|&include file for chapter 2\\
% |cdocspt3.tex|&include file for part 3\\
% |cdocspt4.tex|&include file for part 4\\
% |cdocsdrf.tex|&forwarding file for main file in draft mode\\
% |cdocsfi1.tex|&forwarding file for final version of chapter 1\\
% |cdocsfi2.tex|&forwarding file for final version of chapter 2\\
% \end{tabular}
% \end{center}
% Each of the eight files can be compiled directly by the \LaTeX{} compiler.
%
% %%%%%%%%%%%%%%%%%%%%%%%%%%%%%%%%%%%%%%
% \paragraph{Main File.}
%
% The main file is called |cdocsamp.tex|.
%
% Load the \textsf{childdoc} definitions and
% declare the filename for the main document:
%    \begin{macrocode}
\input{childdoc.def}
\childdocmain{}
%    \end{macrocode}

% Optional override for |\version| flag:
%    \begin{macrocode}
%%\ifchilddoc\else\providecommand{\version}{draft}\fi
%    \end{macrocode}

% Define the default values for the |\version| flag
% (|final| for the main file and |draft| for childs):
%    \begin{macrocode}
\ifchilddoc
\providecommand{\version}{draft}
\else
\providecommand{\version}{final}
\fi
%    \end{macrocode}

% Load the standard document class:
%    \begin{macrocode}
\documentclass[12pt]{article}
%    \end{macrocode}

% Start the document body:
%    \begin{macrocode}
\begin{document}
%    \end{macrocode}

% Declare a title page.
% Print title, part of document being processed and version flag:
%    \begin{macrocode}
\addtocounter{page}{-1}
\begin{center}
{\LARGE\bfseries{}childdoc example\par}
\vspace{1cm}
\ifchilddoc
\ifchilddocmanual part\else chapter\fi:
`\childdocname' of `\childdocjob'\par
\else
main document: `\childdocjob'\par
\fi
version: \version\par
\end{center}
\newpage
%    \end{macrocode}

% Manually include selected file,
% otherwise process as usual:
%    \begin{macrocode}
\ifchilddocmanual
\section*{part `\childdocname'}
\input{\childdocname}
\else
%    \end{macrocode}

% Include the two chapters:
%    \begin{macrocode}
\include{cdocsch1}
\include{cdocsch2}
%    \end{macrocode}

% Include the two parts unless only chapters should be displayed:
%    \begin{macrocode}
\ifchilddoc\else
\section{part three}
\input{cdocspt3}
\section{part four}
\input{cdocspt4}
\fi
%    \end{macrocode}

% Process as usual until here:
%    \begin{macrocode}
\fi
%    \end{macrocode}

% End of document body:
%    \begin{macrocode}
\end{document}
%    \end{macrocode}
%\iffalse
%</samplemain>
%\fi
%
% %%%%%%%%%%%%%%%%%%%%%%%%%%%%%%%%%%%%%%
% \paragraph{Chapter Include Files.}
%
% The include files are called |cdocsch1.tex| and |cdocsch2.tex|.
%
%\iffalse
%<*samplechap1|samplechap2>
%\fi

% Optional override for |\version| flag:
%    \begin{macrocode}
%%\providecommand{\version}{final}
%    \end{macrocode}

% Include the main document:
%    \begin{macrocode}
\input{childdoc.def}
\childdocof{cdocsamp}
%    \end{macrocode}

%\iffalse
%</samplechap1|samplechap2>
%\fi
%
%\iffalse
%<*samplechap1>
%\fi
% Some text for chapter 1:
%    \begin{macrocode}
\section{one}
some text in chapter one
%    \end{macrocode}

%\iffalse
%</samplechap1>
%\fi
% Some text for chapter 2:
%\iffalse
%<*samplechap2>
%\fi
%    \begin{macrocode}
\section{two}
more text in chapter two
%    \end{macrocode}

%\iffalse
%</samplechap2>
%\fi
%
% %%%%%%%%%%%%%%%%%%%%%%%%%%%%%%%%%%%%%%
% \paragraph{Part Include Files.}
%
% The include files are called |cdocspt3.tex| and |cdocspt4.tex|.
%
%\iffalse
%<*samplepart3|samplepart4>
%\fi

% Optional override for |\version| flag:
%    \begin{macrocode}
%%\providecommand{\version}{final}
%    \end{macrocode}

% Include the main document:
%    \begin{macrocode}
\input{childdoc.def}
\childdocby{cdocsamp}
%    \end{macrocode}

%\iffalse
%</samplepart3|samplepart4>
%\fi
%
%\iffalse
%<*samplepart3>
%\fi
% Some text for part 3:
%    \begin{macrocode}
some text in part three
%    \end{macrocode}

%\iffalse
%</samplepart3>
%\fi
% Some text for part 4:
%\iffalse
%<*samplepart4>
%\fi
%    \begin{macrocode}
more text in part four
%    \end{macrocode}

%\iffalse
%</samplepart4>
%\fi
%
% %%%%%%%%%%%%%%%%%%%%%%%%%%%%%%%%%%%%%%
% \paragraph{Forwarding for a Complete Draft.}
%
% The following forwarding file |cdocsdrf.tex|
% compiles the main document in draft mode:
%\iffalse
%<*sampledraft>
%\fi
%    \begin{macrocode}
\def\version{draft}
\input{childdoc.def}
\childdocforward{cdocsamp}
%    \end{macrocode}

%\iffalse
%</sampledraft>
%\fi
%
% %%%%%%%%%%%%%%%%%%%%%%%%%%%%%%%%%%%%%%
% \paragraph{Forwarding for Final Version of the Chapters.}
%
% The following forwarding files |cdocsfn1.tex| and |cdocsfn2.tex|
% (with identical content)
% compile the final versions of the child documents
% |cdocsch1.tex| and |cdocsch2.tex|, respectively:
%\iffalse
%<*samplefinal>
%\fi
%    \begin{macrocode}
\def\version{final}
\input{childdoc.def}
\childdocforwardprefix[cdocsamp]{cdocsfn}{cdocsch}
%    \end{macrocode}

%\iffalse
%</samplefinal>
%\fi
%
% %%%%%%%%%%%%%%%%%%%%%%%%%%%%%%%%%%%%%%
% \paragraph{Command Line Processing.}
%
% The following three command lines generate the output files
% |cdocscld|, |cdocscl1| and |cdocscl2|
% which should be identical to
% |cdocsdrf|, |cdocsch1| and |cdocsfn2|, respectively:
% \begin{center}
% \begin{tabular}{l}
% |latex -jobname cdocscld \|\\
% |  "\def\version{draft}\input{childdoc.def}\childdocforward{cdocsamp}"|\\
% |latex -jobname cdocscl1 \|\\
% |  "\input{childdoc.def}\childdocforward[cdocsamp]{cdocsch1}"|\\
% |latex -jobname cdocscl2 \|\\
% |  "\def\version{final}\input{childdoc.def}\childdocforward{cdocsch2}"|
% \end{tabular}
% \end{center}
% Note that the trailing backslash on each first line
% merely continues the input to the second line
% (for convenient cut ant paste).
% Furthermore, the command |latex| can be replaced by any
% of its alternative versions such as |pdflatex|.
%
% %%%%%%%%%%%%%%%%%%%%%%%%%%%%%%%%%%%%%%%%%%%%%%%%%%%%%%%%%%%%%%%%%%%%%%%%%%%%%%
% %%%%%%%%%%%%%%%%%%%%%%%%%%%%%%%%%%%%%%%%%%%%%%%%%%%%%%%%%%%%%%%%%%%%%%%%%%%%%%
% \section{Implementation}
%\iffalse
%<*package>
%\fi
%
% This section describes the definitions file |childdoc.def|.

% The definitions cannot be loaded using |\usepackage| or |\RequirePackage|
% which has a mechanism to prevent loading a style file more than once.
% When loading the definitions by means of |\input|
% multiple instances have to be prevented manually:
%\iffalse
%This code needs to be before the `\ProvidesFile' directive
%which is defined at the beginning of this file.
%Therefore it is also placed there and commented out here.
%</package>
%<*discard>
%\fi
%    \begin{macrocode}
\ifdefined\childdocmain\endinput\fi
%    \end{macrocode}
%\iffalse
%</discard>
%<*package>
%\fi
%
% \macro{\ifchilddoc}
% \macro{\ifchilddocmanual}
% The conditional |\ifchilddoc| tells whether a
% child (true) or main (false) document is being compiled.
% The conditional |\ifchilddocmanual| tells whether
% the |\includeonly| mechanism is used (false) or
% the selection of child files must be performed manually (true).
% The definitions initialise to false:
%    \begin{macrocode}
\newif\ifchilddoc
\newif\ifchilddocmanual
%    \end{macrocode}

% \macro{\childdocname}
% \macro{\childdocjob}
% The macro |\childdocname| stores the name of the main document
% to be compiled. The macro |\childdocjob| stores the name of
% the document on which the \LaTeX{} compiler was originally invoked.
% The content of |\jobname| cannot be compared
% to filenames specified in the source due to different catcodes.
% The following code rescans |\jobname|, stores the result
% in |\childdocname| and saves a copy in |\childdocjob|:
%    \begin{macrocode}
\edef\childdocname{\scantokens\expandafter{\jobname\noexpand}}
\let\childdocjob\childdocname
%    \end{macrocode}

% \macro{\childdocdisable}
% The macro |\childdocdisable| prevents the main file
% from being processed more than once.
% At this stage, the main document command |\childdocmain|
% is assumed to be called once again where it should do nothing.
% Any subsequent call to it should prevent
% a secondary processing of the main document
% It overwrites the forwarding commands
% |\childdocof| and |\childdocforward|
% with empty macros to prevent further inclusions of the main document:
%    \begin{macrocode}
\newcommand{\childdocdisable}
{
  \renewcommand{\childdocmain}[1]{\renewcommand{\childdocmain}[1]{\endinput}}
  \renewcommand{\childdocof}[1]{}
  \renewcommand{\childdocby}[2][]{}
  \renewcommand{\childdocforward}[2][]{}
  \renewcommand{\childdocdisable}{}
}
%    \end{macrocode}

% \macro{\childdocmain}
% The macro |\childdocmain| is to be called at the top of the main file
% with nothing or the main filename (without extension) as argument.
% First, it breaks loops.
% If the argument is not empty and does not match |\childdocname|
% (which is set by the first inclusion of |childdoc.def|),
% |\ifchilddoc| is set to true, |\includeonly| is applied to the child file
% and |\jobname| is set to the main file
% (for proper handling of |.aux| files):
%    \begin{macrocode}
\newcommand{\childdocmain}[1]
{
  \childdocdisable\childdocmain{}
  \if?#1?\else
    \begingroup
      \def\childdoctmp{#1}
      \ifx\childdoctmp\childdocname
        \def\childdoctmp{}
      \else
        \def\childdoctmp
        {
          \childdoctrue
          \includeonly{\childdocname}
          \def\childdocjob{#1}
          \def\jobname{#1}
        }
      \fi
      \expandafter
    \endgroup
    \childdoctmp
  \fi
}
%    \end{macrocode}

% \macro{\childdocof}
% The command |\childdocof| redirects
% compilation to the main file |#1|.
%    \begin{macrocode}
\newcommand{\childdocof}[1]
{
  \childdocdisable
  \childdoctrue
  \includeonly{\childdocname}
  \def\jobname{#1}
  \def\childdocjob{#1}
  \input{#1}
}
%    \end{macrocode}

% \macro{\childdocby}
% The command |\childdocby| ....
%    \begin{macrocode}
\newcommand{\childdocby}[2][]
{
  \childdocdisable
  \childdoctrue
  \childdocmanualtrue
  \if?#1?\else
    \def\jobname{#2}
  \fi
  \def\childdocjob{#2}
  \input{#2}
  \endinput
}
%    \end{macrocode}

% \macro{\childdocforward}
% The command |\childdocforward| redirects
% compilation to the main file or
% (if the optional argument is given) a child file.
% Parameters are set as if the main file
% or a child file starting with |\childdocof| was compiled.
% Then compilation is handed over to the main file:
%    \begin{macrocode}
\newcommand{\childdocforward}[2][]
{
  \begingroup
    \if?#1?
      \def\childdoctmp
      {
        \def\childdocname{#2}
        \def\childdocjob{#2}
        \def\jobname{#2}
        \input{#2}
        \endinput
      }
    \else
      \def\childdoctmp
      {
        \childdocdisable
        \def\childdocname{#2}
        \childdoctrue
        \includeonly{#2}
        \def\childdocjob{#1}
        \def\jobname{#1}
        \input{#1}
        \endinput
      }
    \fi
    \expandafter
  \endgroup
  \childdoctmp
}
%    \end{macrocode}

% \macro{\childdocforwardprefix}
% The command |\childdocforwardprefix| redirects
% compilation to the main or a child file by means of a pattern.
% The prefix |#1| in the current filename is replaced by |#2|
% and the suffix of the current filename is kept
% (it is assumed that the filename does not contain the substring `|~~~|'
% which is used as a delimiter).
% Compilation is handed over to the new file by |\childdocforward|:
%    \begin{macrocode}
\newcommand{\childdocforwardprefix}[3][]
{
  \begingroup
    \def\childdocextract #2##1~~~{\def\childdoctmp{\childdocforward[#1]{#3##1}}}
    \expandafter\childdocextract\childdocname~~~
    \expandafter
  \endgroup
  \childdoctmp
}
%    \end{macrocode}

% \macro{\childdoc}
% The deprecated macro |\childdoc| is a legacy version of |\childdocmain|:
%    \begin{macrocode}
\newcommand{\childdoc}{\childdocmain}
%    \end{macrocode}

% \macro{\childdocredirect}
% The deprecated macro |\childdocredirect| is a legacy version
% of |\childdocforward| and |\childdocforwardprefix|:
%    \begin{macrocode}
\newcommand{\childdocredirect}[2][]
{
  \begingroup
    \if?#1?
      \def\childdoctmp{\childdocforward{#2}}
    \else
      \def\childdoctmp{\childdocforwardprefix{#1}{#2}}
    \fi
    \expandafter
  \endgroup
  \childdoctmp
}
%    \end{macrocode}

%\iffalse
%</package>
%\fi
%
\endinput
\childdocforward[cdocsamp]{cdocsch1}"|\\
% |latex -jobname cdocscl2 \|\\
% |  "\def\version{final}% \iffalse
%
% childdoc.dtx Copyright (C) 2017-2018 Niklas Beisert
%
% This work may be distributed and/or modified under the
% conditions of the LaTeX Project Public License, either version 1.3
% of this license or (at your option) any later version.
% The latest version of this license is in
%   http://www.latex-project.org/lppl.txt
% and version 1.3 or later is part of all distributions of LaTeX
% version 2005/12/01 or later.
%
% This work has the LPPL maintenance status `maintained'.
%
% The Current Maintainer of this work is Niklas Beisert.
%
% This work consists of the files childdoc.dtx and childdoc.ins
% and the derived files childdoc.def and cdocsamp.tex with
% cdocsch1.tex, cdocsch2.tex, cdocsdrf.tex, cdocsfn1.tex, cdocsfn2.tex.
%
%<package>\ifdefined\childdocmain\endinput\fi
%<package>\ProvidesFile{childdoc.def}[2018/12/30 v2.0 child document driver]
%<samplemain>\ProvidesFile{cdocsamp.tex}[2018/12/30 v2.0 sample for childdoc]
%<*driver>
%\ProvidesFile{childdoc.drv}[2018/12/30 v2.0 childdoc reference manual file]
\PassOptionsToClass{10pt,a4paper}{article}
\documentclass{ltxdoc}

\usepackage[margin=35mm]{geometry}
\usepackage{hyperref}
\usepackage{hyperxmp}
\usepackage[usenames]{color}

\hypersetup{colorlinks=true}
\hypersetup{pdfstartview=FitH}
\hypersetup{pdfpagemode=UseNone}
\hypersetup{pdfsource={}}
\hypersetup{pdflang={en-UK}}
\hypersetup{pdfcopyright={Copyright 2017-2018 Niklas Beisert.
  This work may be distributed and/or modified under the
  conditions of the LaTeX Project Public License, either version 1.3
  of this license or (at your option) any later version.}}
\hypersetup{pdflicenseurl={http://www.latex-project.org/lppl.txt}}
\hypersetup{pdfcontactaddress={ETH Zurich, ITP, HIT K,
  Wolfgang-Pauli-Strasse 27}}
\hypersetup{pdfcontactpostcode={8093}}
\hypersetup{pdfcontactcity={Zurich}}
\hypersetup{pdfcontactcountry={Switzerland}}
\hypersetup{pdfcontactemail={nbeisert@itp.phys.ethz.ch}}
\hypersetup{pdfcontacturl={http://people.phys.ethz.ch/\xmptilde nbeisert/}}

\newcommand{\secref}[1]{\hyperref[#1]{section \ref*{#1}}}

\parskip1ex
\parindent0pt
\let\olditemize\itemize
\def\itemize{\olditemize\parskip0pt}

\begin{document}

\title{The \textsf{childdoc} Package}
\hypersetup{pdftitle={The childdoc Package}}
\author{Niklas Beisert\\[2ex]
  Institut f\"ur Theoretische Physik\\
  Eidgen\"ossische Technische Hochschule Z\"urich\\
  Wolfgang-Pauli-Strasse 27, 8093 Z\"urich, Switzerland\\[1ex]
  \href{mailto:nbeisert@itp.phys.ethz.ch}
  {\texttt{nbeisert@itp.phys.ethz.ch}}}
\hypersetup{pdfauthor={Niklas Beisert}}
\hypersetup{pdfsubject={Manual for the LaTeX2e Package childdoc}}
\date{30 December 2018, \textsf{v2.0}}
\maketitle

\begin{abstract}\noindent
\textsf{childdoc} is a \LaTeXe{} package
that enables the direct compilation
of document sections included by |\include|
to individual files.
\end{abstract}

\begingroup
\parskip0ex
\tableofcontents
\endgroup

%%%%%%%%%%%%%%%%%%%%%%%%%%%%%%%%%%%%%%%%%%%%%%%%%%%%%%%%%%%%%%%%%%%%%%%%%%%%%%%%
%%%%%%%%%%%%%%%%%%%%%%%%%%%%%%%%%%%%%%%%%%%%%%%%%%%%%%%%%%%%%%%%%%%%%%%%%%%%%%%%
\section{Introduction}

\LaTeX{} provides a mechanism to structure a large document (such as a book)
into a main file and several child files (containing the chapters)
using the |\include| command.
This mechanism is beneficial for documents
which span hundreds of pages in order to
make the source file(s) more manageable.
Moreover, compilation can be restricted to
selected child files by means of the |\includeonly| command.
The latter feature can be used to reduce the compilation time while editing
(this was significantly more useful in the earlier days of \LaTeX{})
or to generate a smaller document which is easier to navigate.
Another application of |\includeonly| is to generate
documents consisting of selected parts of the complete document.

However, there are a few drawbacks of the plain |\include| mechanism:
\begin{itemize}
\item
The child files cannot be compiled on their own,
they can only be compiled via the main file.
A naive editing environment
(such as a text editor with an option
to have the current file processed by \LaTeX)
may require one to switch to the main file before compiling;
attempting to compile the child file produces errors.
\item
The main file must be modified (each time)
to adjust the |\includeonly| command
to the present needs. This easily leaves the main file in a messy state.
\item
The generated document will always carry the filename
of the main document. This is inconvenient if
several child files are to be compiled and
to be kept for distribution.
\end{itemize}

The present package provides a simple interface
to make child files individually compilable by \LaTeX{}.
Compiling a child file then has the same effect as compiling
the main file with an |\includeonly| command
to select the appropriate child.
Moreover the generated document will carry the name of the child
rather than the main file.
This resolves all three above issues.

This feature is meant to make the editing of books,
thesis documents and lecture notes somewhat more convenient.
However, the package can also be used efficiently for
composing a series of documents (such as exercise sheets)
which are typically distributed individually.
It then assists the author in generating the individual documents
(potentially in different versions)
as well as a document containing the collected series.
Another application is in developing style files
or other kinds of included material
where compilation of the style file could redirect
to a sample or test file.

%%%%%%%%%%%%%%%%%%%%%%%%%%%%%%%%%%%%%%%%%%%%%%%%%%%%%%%%%%%%%%%%%%%%%%%%%%%%%%%%
%%%%%%%%%%%%%%%%%%%%%%%%%%%%%%%%%%%%%%%%%%%%%%%%%%%%%%%%%%%%%%%%%%%%%%%%%%%%%%%%
\section{Usage}

First of all, the package \textsf{childdoc} is \emph{not} a standard
\LaTeXe{} |.sty| style file! Therefore it needs to be invoked in
a non-standard way.

%%%%%%%%%%%%%%%%%%%%%%%%%%%%%%%%%%%%%%%%%%%%%%%%%%%%%%%%%%%%%%%%%%%%%%%%%%%%%%%%
\subsection{Included Files}
\label{sec:include}

%%%%%%%%%%%%%%%%%%%%%%%%%%%%%%%%%%%%%%%%
\DescribeMacro{\childdocmain}
To use the package, add the commands
\begin{center}
\begin{tabular}{l}
|\input{childdoc.def}|\\
|\childdocmain{}|\\
\end{tabular}
\end{center}
at the very top of the main \LaTeX{} file,
in particular \emph{before} the |\documentclass| statement!
The argument of |\childdocmain| should be left empty
(but it must be present).

%%%%%%%%%%%%%%%%%%%%%%%%%%%%%%%%%%%%%%%%
\DescribeMacro{\childdocof}
Furthermore, add the commands
\begin{center}
\begin{tabular}{l}
|\input{childdoc.def}|\\
|\childdocof{|\textit{main}|}|\\
\end{tabular}
\end{center}
at the top of every child file \textit{child}
which is included by |\include{|\textit{child}|}|
from within the main file
(or at least for those files to be compiled individually).
The argument \textit{main} must be the filename of the main file.

There are a couple of
considerations in setting up the main and child documents:

%%%%%%%%%%%%%%%%%%%%%%%%%%%%%%%%%%%%%%%%
\paragraph{Restrictions.}

Please note the following restrictions:
\begin{itemize}
\item
|\childdocmain| must be called with one argument \textit{main}
to ensure compatibility with earlier version of the package.
It must either be empty (|\childdocmain{}|)
or precisely match the filename of the main file in which it is specified.
See \secref{sec:detection} for further information.
\item
The filename \textit{main} must be specified without the |.tex| extension.
\item
The filename \textit{main} is case sensitive
(even in case-insensitive file systems)
due to internal string comparison.
\item
The argument \textit{main} should be fully expanded, it cannot be a macro.
\item
Subdirectories and special characters should be avoided in filenames.
\item
The command |\childdocmain{|\textit{main}|}| must be followed by a whitespace.
It should not be followed immediately by another command
or by a comment mark `|%|'.
This is because the \TeX{} parser reads the token immediately following
the argument of |\childdocmain| and puts it
at the beginning of every child section;
however, a white\-space is ignored.
\end{itemize}

%%%%%%%%%%%%%%%%%%%%%%%%%%%%%%%%%%%%%%%%
\paragraph{Content of Main File.}

It is advisable to place all content in the child files included by |\include|.
Any output contained in the main file will appear in all child documents
unless suppressed manually;
it cannot be suppressed automatically by the |\includeonly| directive
and thus should normally be avoided.
A method to include some content in the main file
by means of conditional processing is described in \secref{sec:conditional}.

%%%%%%%%%%%%%%%%%%%%%%%%%%%%%%%%%%%%%%%%
\paragraph{Page Numbering.}

When only a part of the document is compiled,
the appropriate numbering of pages
(as well as other status parameters)
is determined from the |.aux| files.
The latter contain information from previous passes.
However this information needs to propagate through
all intermediate child documents.
Therefore the page numbering in child documents may well
be inconsistent until the complete document is compiled at least once.

A useful (if unconventional) way to always ensure a consistent
page numbering is to restart the numbering in each child document
and denote the pages by `\textit{child}|.|\textit{page}'
where \textit{child} represents the chapter/section number of the child file.
This can be achieved by the command
|\numberwithin{page}{|\textit{child}|}|
of the \textsf{amsmath} package
where \textit{child} can be |chapter| or |section|
depending on the chosen structuring.
Alternatively, one can modify the macro |\thepage| appropriately
and reset the counter |page| at the start of each child file.

%%%%%%%%%%%%%%%%%%%%%%%%%%%%%%%%%%%%%%%%%%%%%%%%%%%%%%%%%%%%%%%%%%%%%%%%%%%%%%%%
\subsection{Conditional Processing}
\label{sec:conditional}

The package provides a mechanism to compile different versions
of a document. To customise the versions further some conditional processing
can come in handy to distinguish which version is being compiled.
The package provides two macros to describe the compilation context:

%%%%%%%%%%%%%%%%%%%%%%%%%%%%%%%%%%%%%%%%
\DescribeMacro{\ifchilddoc}
The conditional |\ifchilddoc| distinguishes between the compilation of
child documents and the main document:
%
\begin{center}
|\ifchilddoc |\textit{child-code}| |[|\||else |\textit{main-code}]| \||fi|
\end{center}

%%%%%%%%%%%%%%%%%%%%%%%%%%%%%%%%%%%%%%%%
\DescribeMacro{\childdocname}
\DescribeMacro{\childdocjob}
The macro |\childdocname| contains the filename (without extension)
of the main or child file being processed.
Note that |\childdocjob| will always contain the name of the main file.

%%%%%%%%%%%%%%%%%%%%%%%%%%%%%%%%%%%%%%%%
\paragraph{Title Page.}

Conditional processing can be used to include a title or banner page
in the main document when proper precautions are taken.
Importantly, the code in the main file should ensure that the page counter
(as well as other status parameters which are stored in the |.aux| files)
takes the same value after the conditional processing.
Otherwise the page numbers may take divergent values
depending on which part is compiled.

For example, a title page could be declared by:
%
\begin{center}
\begin{tabular}{l}
|\ifchilddoc\||else|\\
|\addtocounter{page}{-1}|\\
\textit{code for title page}\\
|\newpage|\\
|\||fi|
\end{tabular}
\end{center}
%
A banner page for the child documents can be generated by:
%
\begin{center}
\begin{tabular}{l}
|\ifchilddoc|\\
|\addtocounter{page}{-1}|\\
\textit{code for banner page}\\
|\newpage|\\
|\||fi|
\end{tabular}
\end{center}
%
Here one could write a message such as:
\begin{center}
|This is the part \childdocname{} of \childdocjob{}.|
\end{center}

%%%%%%%%%%%%%%%%%%%%%%%%%%%%%%%%%%%%%%%%%%%%%%%%%%%%%%%%%%%%%%%%%%%%%%%%%%%%%%%%
\subsection{Flags}
\label{sec:flags}

The package makes it easy to generate different versions
of the main or child documents.
To this end compilation flags can be defined
and assigned different default values.
They will be particularly useful in conjunction
with the forwarding mechanism described in \secref{sec:forward}.

For example, it may be useful to have a flag |\version|
which can be set to |draft| or |final|.
The document source will contain some conditional code
depending on the value of |\version|.
Suppose further, the flag should default to |final| for the main file
and to |draft| for child files
which is a natural assignment for editing the document.
This is achieved by placing the following code
in the preamble of the main document
(below the |\childdocmain| directive):
%
\begin{center}
\begin{tabular}{l}
|\ifchilddoc|\\
|\providecommand{\version}{draft}|\\
|\||else|\\
|\providecommand{\version}{final}|\\
|\||fi|
\end{tabular}
\end{center}
%
The definition by |\providecommand| makes sure
that previous definitions are not overwritten.
Further statements |\providecommand{\version}{...}|
can thus be added before the above code to override it.

For the main file, one might add a line
(between |\childdocmain| and the above block)
%
\begin{center}
|%\ifchilddoc\||else\providecommand{\version}{draft}\||fi|
\end{center}
%
which can be uncommented to produce a draft version.
Likewise one can add a line to the very top of a child file
(above the |\childdocof{|\textit{main}|}| directive)
%
\begin{center}
|%\providecommand{\version}{final}|
\end{center}
%
which can be uncommented to produce the final version of this child document.

%%%%%%%%%%%%%%%%%%%%%%%%%%%%%%%%%%%%%%%%%%%%%%%%%%%%%%%%%%%%%%%%%%%%%%%%%%%%%%%%
\subsection{Forwarding}
\label{sec:forward}

Different versions of the main or child documents
using compilation flags as described in \secref{sec:flags}
can be (permanently) stored in different files
for convenient compilation, viewing and distribution.
To this end, the package defines a command
to pass on compilation to a different file:

%%%%%%%%%%%%%%%%%%%%%%%%%%%%%%%%%%%%%%%%
\DescribeMacro{\childdocforward}
The command |\childdocforward| redirects processing to
another source file:
%
\begin{center}
\begin{tabular}{l}
|\input{childdoc.def}|\\
|\childdocforward[|\textit{main}|]{|\textit{dest}|}|\\
\end{tabular}
\end{center}
%
The argument \textit{dest} is the destination file
(without extension).
It should be the main file or one of the child files.
Note that further \textsf{childdoc} directives
such as |\childdocof| and |\childdocforward|
in the indicated file will be processed in this form.
The optional argument \textit{main}
passes on directly to the main file \textit{main}
while pretending to compile the child \textit{dest}.
This form behaves as if \textit{dest}
issues |\childdocof{|\textit{main}|}| right away,
and no further \textsf{childdoc} directives will be processed.

%%%%%%%%%%%%%%%%%%%%%%%%%%%%%%%%%%%%%%%%
\DescribeMacro{\...prefix}
In the alternative form |\childdocforwardprefix|,
%
\begin{center}
\begin{tabular}{l}
|\input{childdoc.def}|\\
|\childdocforwardprefix[|\textit{main}|]{|\textit{prefix}|}{|\textit{dest}|}|
\end{tabular}
\end{center}
%
the destination file is determined by a pattern
depending on the current file:
To make this work, the current file must be called
`{\textit{prefix}\hspace{0.2em}\textit{suffix}}'
with \textit{prefix} matching precisely the argument.
Processing is then passed on to the file
`{\textit{dest}\hspace{0.2em}\textit{suffix}}'.
Surely, the same effect is achieved by
directly specifying the
argument `{\textit{dest}\hspace{0.2em}\textit{suffix}}'
in the first form.
However, that requires to set up a different file
for each child. With the alternative form of the command
all these files can have exactly the same content
which simplifies setting them up and maintaining them.

For example, the following file |draft.tex|
with a compilation flag |\version| as described in \secref{sec:flags}
compiles the main document as a draft:
%
\begin{center}
\begin{tabular}{l}
|\def\version{draft}|\\
|\input{childdoc.def}|\\
|\childdocforward{|\textit{main}|}|
\end{tabular}
\end{center}
%
Likewise, the following files |final|\textit{nn}|.tex|
compile the final version of the child document
|child|\textit{nn}|.tex|:
%
\begin{center}
\begin{tabular}{l}
|\def\version{final}|\\
|\input{childdoc.def}|\\
|\childdocforwardprefix{final}{child}|
\end{tabular}
\end{center}
%

Note that when several versions of a main file and/or of each child file
are to be generated, it may be convenient to set up a |Makefile| or
shell script to automatise the process.

%%%%%%%%%%%%%%%%%%%%%%%%%%%%%%%%%%%%%%%%%%%%%%%%%%%%%%%%%%%%%%%%%%%%%%%%%%%%%%%%
\subsection{Command Line Processing}
\label{sec:commandline}

The effect of redirection files can also be achieved by invoking
the \LaTeX{} compiler with a more elaborate command line.
Most conveniently this should be done as part
of a shell script or a |Makefile|.

When using \textsf{childdoc} in the main file, the following
command lines effectively perform a redirection
(note that depending on the shell being used,
backslashes may have to be doubled: `|\|' $\to$ `|\\|'):
%
\begin{center}
|... -jobname "|\textit{target}|" |\\|"|[\textit{flags}]%
|\input{childdoc.def}\childdocforward[|\textit{main}|]{|\textit{dest}|}"|
\end{center}
%
Here \textit{target} is the name of the output file,
\textit{main} is the name of the main file
and \textit{dest} is the name of the main or child file to be processed
(all filenames without extensions).
The optional argument \textit{main} can be omitted
if \textit{main} matches \textit{dest}.
Optionally, compilation \textit{flags} can be defined via |\def| commands.
This command line makes the \TeX{} engine believe
it is compiling the file \textit{target}
whose content is specified as the latter parameter.
The provided code then forwards the processing to
\textit{main} or \textit{dest} as described in \secref{sec:forward}.

%%%%%%%%%%%%%%%%%%%%%%%%%%%%%%%%%%%%%%%%%%%%%%%%%%%%%%%%%%%%%%%%%%%%%%%%%%%%%%%%
\subsection{Include by Input}
\label{sec:input}

Including child documents by |\include| has some restrictions by design.
Most notably, the content of a child document always occupies
its own set of pages; pages cannot be shared between child documents.
Usually, this behaviour makes perfect sense
because each child document contain an essential part of the document.
However, in some situations it may be desirable to compose
a document from a collection of parts
without having mandatory page breaks between then.
For this case, the package
provides a mechanism to include parts
by |\input| which can also be processed individually.
However, by construction this mechanism
requires manual handling of the content to be output.

%%%%%%%%%%%%%%%%%%%%%%%%%%%%%%%%%%%%%%%%
\DescribeMacro{\ifchilddocmanual}
The main file should be prepared as usual, see \secref{sec:include}.
However, the document body must make a distinction
between processing of an individual part and of the main document, e.g.:
%
\begin{center}
\begin{tabular}{l}
|\ifchilddocmanual|\\
|\input{\childdocname}|\\
|\||else|\\
\textit{document body with }|\input{|\textit{part}|}|\\
|\||fi|
\end{tabular}
\end{center}
%
The conditional |\ifchilddocmanual| is true whenever
a part to be included by |\input| is being compiled,
and the name of the part is stored in |\childdocname|.

%%%%%%%%%%%%%%%%%%%%%%%%%%%%%%%%%%%%%%%%
\DescribeMacro{\childdocby}
Each part to be included by |\input| should start with:
%
\begin{center}
\begin{tabular}{l}
|\input{childdoc.def}|\\
|\childdocby{|\textit{main}|}|\\
\end{tabular}
\end{center}
%
The directive |\childdocby| is similar to |\childdocof|
described in \secref{sec:include},
but the subsequent selection of content must be done manually.
To that end, both |\ifchilddoc| and |\ifchilddocmanual|
will be true upon processing of a part,
and the name of the part is stored in |\childdocname|.
Note that |\jobname| will be set to the filename of the current part
so that each part receives an individual |.aux| file
that does not interfere with the |.aux| file(s) of the main document.
This behaviour can be altered by the alternative form
|\childdocby[*]{|\textit{main}|}| (with a non-empty optional argument)
which uses the |.aux| file of the main document
by setting |\jobname| to \textit{main}.

%%%%%%%%%%%%%%%%%%%%%%%%%%%%%%%%%%%%%%%%%%%%%%%%%%%%%%%%%%%%%%%%%%%%%%%%%%%%%%%%
\subsection{Driver Development}
\label{sec:driver}

The \textsf{childdoc} mechanism can also be use for the development
of definition files such as \LaTeX{} styles or classes.
This case differs from the above setup with multiple parts
included by |\include| in that no |\includeonly| should be invoked.
This can be achieved by starting the include file
(before |\ProvidesPackage|) with:
%
\begin{center}
\begin{tabular}{l}
|\input{childdoc.def}|\\
|\childdocforward{|\textit{main}|}|\\
\end{tabular}
\end{center}
%
or alternatively with:
%
\begin{center}
\begin{tabular}{l}
|\input{childdoc.def}|\\
|\childdocby{|\textit{main}|}|\\
\end{tabular}
\end{center}
%
Both forms have slightly different effects as described above.
The main file is prepared as usual, see \secref{sec:include}.

%%%%%%%%%%%%%%%%%%%%%%%%%%%%%%%%%%%%%%%%%%%%%%%%%%%%%%%%%%%%%%%%%%%%%%%%%%%%%%%%
\subsection{Legacy Detection}
\label{sec:detection}

The directive |\childdocmain| in the main file can detect
whether the complete document or merely a child is to be compiled
even without using the directive |\childdocof|.
This method is deprecated because it is less robust
and there is no compelling reason to use it;
it is merely provided for backward compatibility
and it may be removed in future versions.

If the detection mechanism is to be used,
it is mandatory to correctly specify
the filename of the main file as the argument of |\childdocmain|:
%
\begin{center}
\begin{tabular}{l}
|\input{childdoc.def}|\\
|\childdocmain{|\textit{main}|}|\\
\end{tabular}
\end{center}
%
If |\jobname| does not match the argument \textit{main} of |\childdocmain|,
it is assumed that |\jobname| points to the child file to be compiled.
When using |\childdocmain| with the main file specified as argument,
it suffices to start a child file
with just |\input{|\textit{main}|}|
without loading of the package and using |\childdocof|.
If instead all processing is done
with the appropriate \textsf{childdoc} directives,
the argument of \textit{main} of |\childdocmain| can be empty.

An alternative version of the command line processing described
in \secref{sec:commandline} using the detection mechanism reads:
%
\begin{center}
|... -jobname "|\textit{target}|" "|[\textit{flags}]%
[|\def\jobname{|\textit{dest}|}|]|\input{|\textit{main}|}"|
\end{center}

%%%%%%%%%%%%%%%%%%%%%%%%%%%%%%%%%%%%%%%%%%%%%%%%%%%%%%%%%%%%%%%%%%%%%%%%%%%%%%%%
\subsection{Manual Code}
\label{sec:manual}

In case one cannot be certain whether the definitions file |childdoc.def|
is installed on the target \TeX{} distribution
and one prefers not to ship it,
it is conceivable to paste a few relevant commands into the sources.

To that end, drop all statements |\input{childdoc.def}|
and perform the replacements as outlined below.
Instead of |\childdocmain{|\textit{main}|}| add the following code
to the top of the main file:
%
\begin{center}
\begin{tabular}{l}
|\||ifdefined\childdocname\endinput\||fi\newif\ifchilddoc|\\
|\edef\childdocname{\scantokens\expandafter{\jobname\noexpand}}|\\
|\def\childdocmain{|\textit{main}|}\||ifx\childdocmain\childdocname\||else|\\
|\childdoctrue\includeonly{\childdocname}\let\jobname\childdocmain\||fi|\\
\end{tabular}
\end{center}
%
Instead of |\childdocof{|\textit{main}|}| just include the main file
at the top of each child file:
%
\begin{center}
|\input{|\textit{main}|}|
\end{center}
%
A simple redirection |\childdocforward{|\textit{dest}|}| is achieved by:
%
\begin{center}
|\def\jobname{|\textit{dest}|}\input{\jobname}|
\end{center}
%
The redirection with prefix
|\childdocforwardprefix[|\textit{prefix}|]{|\textit{dest}|}|
is accomplished by:
%
\begin{center}
\begin{tabular}{l}
|{\edef\jobname{\scantokens\expandafter{\jobname\noexpand}}|\\
|\def\redirectjob |\textit{prefix}|#1~~~{\gdef\jobname{|\textit{dest}|#1}}|\\
|\expandafter\redirectjob\jobname~~~}\input{\jobname}|
\end{tabular}
\end{center}

In an alternative approach,
child documents can be compiled by a specific command line
without additional code or specific definitions:
%
\begin{center}
|... -jobname "|\textit{target}|" "|[\textit{flags}]%
|\includeonly{|\textit{dest}|}\input{|\textit{main}|}"|
\end{center}
%

%%%%%%%%%%%%%%%%%%%%%%%%%%%%%%%%%%%%%%%%%%%%%%%%%%%%%%%%%%%%%%%%%%%%%%%%%%%%%%%%
%%%%%%%%%%%%%%%%%%%%%%%%%%%%%%%%%%%%%%%%%%%%%%%%%%%%%%%%%%%%%%%%%%%%%%%%%%%%%%%%
\section{Information}

%%%%%%%%%%%%%%%%%%%%%%%%%%%%%%%%%%%%%%%%%%%%%%%%%%%%%%%%%%%%%%%%%%%%%%%%%%%%%%%%
\subsection{Copyright}

Copyright \copyright{} 2017--2018 Niklas Beisert

This work may be distributed and/or modified under the
conditions of the \LaTeX{} Project Public License, either version 1.3
of this license or (at your option) any later version.
The latest version of this license is in
  \url{http://www.latex-project.org/lppl.txt}
and version 1.3 or later is part of all distributions of \LaTeX{}
version 2005/12/01 or later.

This work has the LPPL maintenance status `maintained'.

The Current Maintainer of this work is Niklas Beisert.

This work consists of the files |README.txt|, |childdoc.ins| and |childdoc.dtx|
as well as the derived files |childdoc.def|, |cdocsamp.tex|
with |cdocsch1.tex|, |cdocsch2.tex|, |cdocspt3.tex|, |cdocspt4.tex|,
|cdocsdrf.tex|, |cdocsfn1.tex|, |cdocsfn2.tex|
as well as |childdoc.pdf|.

%%%%%%%%%%%%%%%%%%%%%%%%%%%%%%%%%%%%%%%%%%%%%%%%%%%%%%%%%%%%%%%%%%%%%%%%%%%%%%%%
\subsection{Files and Installation}

The package consists of the files:
%
\begin{center}
\begin{tabular}{ll}
    |README.txt|   & readme file \\
    |childdoc.ins| & installation file \\
    |childdoc.dtx| & source file \\
    |childdoc.def| & definition file \\
    |cdocsamp.tex| & sample main file \\
    |cdocsch1.tex| & sample include file \\
    |cdocsch2.tex| & sample include file \\
    |cdocspt3.tex| & sample part file \\
    |cdocspt4.tex| & sample part file \\
    |cdocsdrf.tex| & sample redirection file \\
    |cdocsfn1.tex| & sample redirection file \\
    |cdocsfn2.tex| & sample redirection file \\
    |childdoc.pdf| & manual
\end{tabular}
\end{center}
%
The distribution consists of the files
|README.txt|, |childdoc.ins| and |childdoc.dtx|.
%
\begin{itemize}
\item
Run (pdf)\LaTeX{} on |childdoc.dtx|
to compile the manual |childdoc.pdf| (this file).
\item
Run \LaTeX{} on |childdoc.ins| to create the definitions file |childdoc.def|
and the sample |cdocsamp.tex| with include files
|cdocsch1.tex|, |cdocsch2.tex|, |cdocspt3.tex|, |cdocspt4.tex|,
|cdocsdrf.tex|, |cdocsfn1.tex|, |cdocsfn2.tex|.
Then copy the file |childdoc.def| to an appropriate directory of your \LaTeX{}
distribution, e.g.\ \textit{texmf-root}|/tex/latex/childdoc|.
\end{itemize}

%%%%%%%%%%%%%%%%%%%%%%%%%%%%%%%%%%%%%%%%%%%%%%%%%%%%%%%%%%%%%%%%%%%%%%%%%%%%%%%%
\subsection{Related CTAN Packages}

There are several other packages which offer a similar functionality:
%
\begin{itemize}
\item
The packages
\href{http://ctan.org/pkg/docmute}{\textsf{docmute}},
\href{http://ctan.org/pkg/includex}{\textsf{includex}} and
\href{http://ctan.org/pkg/standalone}{\textsf{standalone}}
provide commands to include only the document body of
a child file thus allowing both files to be compiled individually.
\item
The packages \href{http://ctan.org/pkg/subdocs}{\textsf{subdocs}}
and \href{http://ctan.org/pkg/subfiles}{\textsf{subfiles}}
provide structures in which the main and child documents can be
encapsulated and allowing them to be compiled individually.
The inclusion mechanism is different from the conventional |\include|.
\item
The package \href{http://ctan.org/pkg/combine}{\textsf{combine}}
is an elaborate solution to combine several documents into one.
\end{itemize}
%
See also the CTAN topic \href{http://ctan.org/topic/subdocs}{\textsf{subdocs}}
for further related packages.
The present package differs from the above solutions in that
a document structure constructed with the conventional |\include| mechanism
just needs two extra commands at the top of every file
such that all constituent files can be compiled individually.

%%%%%%%%%%%%%%%%%%%%%%%%%%%%%%%%%%%%%%%%%%%%%%%%%%%%%%%%%%%%%%%%%%%%%%%%%%%%%%%%
%\subsection{Feature Suggestions}
%
%The following is a list of features which may be useful for future
%versions of this package:
%%
%\begin{itemize}
%\item
%\ldots
%\end{itemize}

%%%%%%%%%%%%%%%%%%%%%%%%%%%%%%%%%%%%%%%%%%%%%%%%%%%%%%%%%%%%%%%%%%%%%%%%%%%%%%%%
\subsection{Revision History}

%%%%%%%%%%%%%%%%%%%%%%%%%%%%%%%%%%%%%%%%
\paragraph{v2.0:} 2018/12/30

\begin{itemize}
\item
immediate forward processing
\item
added |\childdocby| mechanism
\item
manual restructured
\end{itemize}

%%%%%%%%%%%%%%%%%%%%%%%%%%%%%%%%%%%%%%%%
\paragraph{v1.6:} 2018/01/17

\begin{itemize}
\item
application for development of include files
\item
corrections to manual
\end{itemize}

%%%%%%%%%%%%%%%%%%%%%%%%%%%%%%%%%%%%%%%%
\paragraph{v1.5:} 2017/05/21

\begin{itemize}
\item
more complete structuring introduced
\item
|\childdocof| introduced
\item
|\childdoc| renamed to |\childdocmain|
\item
|\childredirect| renamed to |\childdocforward| and |\childdocforwardprefix|
and functionality expanded
\end{itemize}

%%%%%%%%%%%%%%%%%%%%%%%%%%%%%%%%%%%%%%%%
\paragraph{v1.0:} 2017/04/27

\begin{itemize}
\item
manual and install package
\item
first version published on CTAN
\end{itemize}

%%%%%%%%%%%%%%%%%%%%%%%%%%%%%%%%%%%%%%%%
\paragraph{v0.6:} 2017/04/26

\begin{itemize}
\item
redirection mechanism added
\end{itemize}

%%%%%%%%%%%%%%%%%%%%%%%%%%%%%%%%%%%%%%%%
\paragraph{v0.5:} 2017/04/26

\begin{itemize}
\item
functionality in definition file
\end{itemize}


%%%%%%%%%%%%%%%%%%%%%%%%%%%%%%%%%%%%%%%%%%%%%%%%%%%%%%%%%%%%%%%%%%%%%%%%%%%%%%%%
%%%%%%%%%%%%%%%%%%%%%%%%%%%%%%%%%%%%%%%%%%%%%%%%%%%%%%%%%%%%%%%%%%%%%%%%%%%%%%%%
%%%%%%%%%%%%%%%%%%%%%%%%%%%%%%%%%%%%%%%%%%%%%%%%%%%%%%%%%%%%%%%%%%%%%%%%%%%%%%%%
\appendix

\settowidth\MacroIndent{\rmfamily\scriptsize 000\ }

 \DocInput{childdoc.dtx}

\end{document}
%</driver>
% \fi
%
% %%%%%%%%%%%%%%%%%%%%%%%%%%%%%%%%%%%%%%%%%%%%%%%%%%%%%%%%%%%%%%%%%%%%%%%%%%%%%%
% %%%%%%%%%%%%%%%%%%%%%%%%%%%%%%%%%%%%%%%%%%%%%%%%%%%%%%%%%%%%%%%%%%%%%%%%%%%%%%
% \section{Sample}
%\iffalse
%<*samplemain>
%\fi
%
% The following presents a sample document
% with two chapters, two parts, a title page,
% a compile flag as well as three forwarding files to set the flag.
% It consists of eight |.tex| files:
% \begin{center}
% \begin{tabular}{ll}
% |cdocsamp.tex|&main file\\
% |cdocsch1.tex|&include file for chapter 1\\
% |cdocsch2.tex|&include file for chapter 2\\
% |cdocspt3.tex|&include file for part 3\\
% |cdocspt4.tex|&include file for part 4\\
% |cdocsdrf.tex|&forwarding file for main file in draft mode\\
% |cdocsfi1.tex|&forwarding file for final version of chapter 1\\
% |cdocsfi2.tex|&forwarding file for final version of chapter 2\\
% \end{tabular}
% \end{center}
% Each of the eight files can be compiled directly by the \LaTeX{} compiler.
%
% %%%%%%%%%%%%%%%%%%%%%%%%%%%%%%%%%%%%%%
% \paragraph{Main File.}
%
% The main file is called |cdocsamp.tex|.
%
% Load the \textsf{childdoc} definitions and
% declare the filename for the main document:
%    \begin{macrocode}
\input{childdoc.def}
\childdocmain{}
%    \end{macrocode}

% Optional override for |\version| flag:
%    \begin{macrocode}
%%\ifchilddoc\else\providecommand{\version}{draft}\fi
%    \end{macrocode}

% Define the default values for the |\version| flag
% (|final| for the main file and |draft| for childs):
%    \begin{macrocode}
\ifchilddoc
\providecommand{\version}{draft}
\else
\providecommand{\version}{final}
\fi
%    \end{macrocode}

% Load the standard document class:
%    \begin{macrocode}
\documentclass[12pt]{article}
%    \end{macrocode}

% Start the document body:
%    \begin{macrocode}
\begin{document}
%    \end{macrocode}

% Declare a title page.
% Print title, part of document being processed and version flag:
%    \begin{macrocode}
\addtocounter{page}{-1}
\begin{center}
{\LARGE\bfseries{}childdoc example\par}
\vspace{1cm}
\ifchilddoc
\ifchilddocmanual part\else chapter\fi:
`\childdocname' of `\childdocjob'\par
\else
main document: `\childdocjob'\par
\fi
version: \version\par
\end{center}
\newpage
%    \end{macrocode}

% Manually include selected file,
% otherwise process as usual:
%    \begin{macrocode}
\ifchilddocmanual
\section*{part `\childdocname'}
\input{\childdocname}
\else
%    \end{macrocode}

% Include the two chapters:
%    \begin{macrocode}
\include{cdocsch1}
\include{cdocsch2}
%    \end{macrocode}

% Include the two parts unless only chapters should be displayed:
%    \begin{macrocode}
\ifchilddoc\else
\section{part three}
\input{cdocspt3}
\section{part four}
\input{cdocspt4}
\fi
%    \end{macrocode}

% Process as usual until here:
%    \begin{macrocode}
\fi
%    \end{macrocode}

% End of document body:
%    \begin{macrocode}
\end{document}
%    \end{macrocode}
%\iffalse
%</samplemain>
%\fi
%
% %%%%%%%%%%%%%%%%%%%%%%%%%%%%%%%%%%%%%%
% \paragraph{Chapter Include Files.}
%
% The include files are called |cdocsch1.tex| and |cdocsch2.tex|.
%
%\iffalse
%<*samplechap1|samplechap2>
%\fi

% Optional override for |\version| flag:
%    \begin{macrocode}
%%\providecommand{\version}{final}
%    \end{macrocode}

% Include the main document:
%    \begin{macrocode}
\input{childdoc.def}
\childdocof{cdocsamp}
%    \end{macrocode}

%\iffalse
%</samplechap1|samplechap2>
%\fi
%
%\iffalse
%<*samplechap1>
%\fi
% Some text for chapter 1:
%    \begin{macrocode}
\section{one}
some text in chapter one
%    \end{macrocode}

%\iffalse
%</samplechap1>
%\fi
% Some text for chapter 2:
%\iffalse
%<*samplechap2>
%\fi
%    \begin{macrocode}
\section{two}
more text in chapter two
%    \end{macrocode}

%\iffalse
%</samplechap2>
%\fi
%
% %%%%%%%%%%%%%%%%%%%%%%%%%%%%%%%%%%%%%%
% \paragraph{Part Include Files.}
%
% The include files are called |cdocspt3.tex| and |cdocspt4.tex|.
%
%\iffalse
%<*samplepart3|samplepart4>
%\fi

% Optional override for |\version| flag:
%    \begin{macrocode}
%%\providecommand{\version}{final}
%    \end{macrocode}

% Include the main document:
%    \begin{macrocode}
\input{childdoc.def}
\childdocby{cdocsamp}
%    \end{macrocode}

%\iffalse
%</samplepart3|samplepart4>
%\fi
%
%\iffalse
%<*samplepart3>
%\fi
% Some text for part 3:
%    \begin{macrocode}
some text in part three
%    \end{macrocode}

%\iffalse
%</samplepart3>
%\fi
% Some text for part 4:
%\iffalse
%<*samplepart4>
%\fi
%    \begin{macrocode}
more text in part four
%    \end{macrocode}

%\iffalse
%</samplepart4>
%\fi
%
% %%%%%%%%%%%%%%%%%%%%%%%%%%%%%%%%%%%%%%
% \paragraph{Forwarding for a Complete Draft.}
%
% The following forwarding file |cdocsdrf.tex|
% compiles the main document in draft mode:
%\iffalse
%<*sampledraft>
%\fi
%    \begin{macrocode}
\def\version{draft}
\input{childdoc.def}
\childdocforward{cdocsamp}
%    \end{macrocode}

%\iffalse
%</sampledraft>
%\fi
%
% %%%%%%%%%%%%%%%%%%%%%%%%%%%%%%%%%%%%%%
% \paragraph{Forwarding for Final Version of the Chapters.}
%
% The following forwarding files |cdocsfn1.tex| and |cdocsfn2.tex|
% (with identical content)
% compile the final versions of the child documents
% |cdocsch1.tex| and |cdocsch2.tex|, respectively:
%\iffalse
%<*samplefinal>
%\fi
%    \begin{macrocode}
\def\version{final}
\input{childdoc.def}
\childdocforwardprefix[cdocsamp]{cdocsfn}{cdocsch}
%    \end{macrocode}

%\iffalse
%</samplefinal>
%\fi
%
% %%%%%%%%%%%%%%%%%%%%%%%%%%%%%%%%%%%%%%
% \paragraph{Command Line Processing.}
%
% The following three command lines generate the output files
% |cdocscld|, |cdocscl1| and |cdocscl2|
% which should be identical to
% |cdocsdrf|, |cdocsch1| and |cdocsfn2|, respectively:
% \begin{center}
% \begin{tabular}{l}
% |latex -jobname cdocscld \|\\
% |  "\def\version{draft}\input{childdoc.def}\childdocforward{cdocsamp}"|\\
% |latex -jobname cdocscl1 \|\\
% |  "\input{childdoc.def}\childdocforward[cdocsamp]{cdocsch1}"|\\
% |latex -jobname cdocscl2 \|\\
% |  "\def\version{final}\input{childdoc.def}\childdocforward{cdocsch2}"|
% \end{tabular}
% \end{center}
% Note that the trailing backslash on each first line
% merely continues the input to the second line
% (for convenient cut ant paste).
% Furthermore, the command |latex| can be replaced by any
% of its alternative versions such as |pdflatex|.
%
% %%%%%%%%%%%%%%%%%%%%%%%%%%%%%%%%%%%%%%%%%%%%%%%%%%%%%%%%%%%%%%%%%%%%%%%%%%%%%%
% %%%%%%%%%%%%%%%%%%%%%%%%%%%%%%%%%%%%%%%%%%%%%%%%%%%%%%%%%%%%%%%%%%%%%%%%%%%%%%
% \section{Implementation}
%\iffalse
%<*package>
%\fi
%
% This section describes the definitions file |childdoc.def|.

% The definitions cannot be loaded using |\usepackage| or |\RequirePackage|
% which has a mechanism to prevent loading a style file more than once.
% When loading the definitions by means of |\input|
% multiple instances have to be prevented manually:
%\iffalse
%This code needs to be before the `\ProvidesFile' directive
%which is defined at the beginning of this file.
%Therefore it is also placed there and commented out here.
%</package>
%<*discard>
%\fi
%    \begin{macrocode}
\ifdefined\childdocmain\endinput\fi
%    \end{macrocode}
%\iffalse
%</discard>
%<*package>
%\fi
%
% \macro{\ifchilddoc}
% \macro{\ifchilddocmanual}
% The conditional |\ifchilddoc| tells whether a
% child (true) or main (false) document is being compiled.
% The conditional |\ifchilddocmanual| tells whether
% the |\includeonly| mechanism is used (false) or
% the selection of child files must be performed manually (true).
% The definitions initialise to false:
%    \begin{macrocode}
\newif\ifchilddoc
\newif\ifchilddocmanual
%    \end{macrocode}

% \macro{\childdocname}
% \macro{\childdocjob}
% The macro |\childdocname| stores the name of the main document
% to be compiled. The macro |\childdocjob| stores the name of
% the document on which the \LaTeX{} compiler was originally invoked.
% The content of |\jobname| cannot be compared
% to filenames specified in the source due to different catcodes.
% The following code rescans |\jobname|, stores the result
% in |\childdocname| and saves a copy in |\childdocjob|:
%    \begin{macrocode}
\edef\childdocname{\scantokens\expandafter{\jobname\noexpand}}
\let\childdocjob\childdocname
%    \end{macrocode}

% \macro{\childdocdisable}
% The macro |\childdocdisable| prevents the main file
% from being processed more than once.
% At this stage, the main document command |\childdocmain|
% is assumed to be called once again where it should do nothing.
% Any subsequent call to it should prevent
% a secondary processing of the main document
% It overwrites the forwarding commands
% |\childdocof| and |\childdocforward|
% with empty macros to prevent further inclusions of the main document:
%    \begin{macrocode}
\newcommand{\childdocdisable}
{
  \renewcommand{\childdocmain}[1]{\renewcommand{\childdocmain}[1]{\endinput}}
  \renewcommand{\childdocof}[1]{}
  \renewcommand{\childdocby}[2][]{}
  \renewcommand{\childdocforward}[2][]{}
  \renewcommand{\childdocdisable}{}
}
%    \end{macrocode}

% \macro{\childdocmain}
% The macro |\childdocmain| is to be called at the top of the main file
% with nothing or the main filename (without extension) as argument.
% First, it breaks loops.
% If the argument is not empty and does not match |\childdocname|
% (which is set by the first inclusion of |childdoc.def|),
% |\ifchilddoc| is set to true, |\includeonly| is applied to the child file
% and |\jobname| is set to the main file
% (for proper handling of |.aux| files):
%    \begin{macrocode}
\newcommand{\childdocmain}[1]
{
  \childdocdisable\childdocmain{}
  \if?#1?\else
    \begingroup
      \def\childdoctmp{#1}
      \ifx\childdoctmp\childdocname
        \def\childdoctmp{}
      \else
        \def\childdoctmp
        {
          \childdoctrue
          \includeonly{\childdocname}
          \def\childdocjob{#1}
          \def\jobname{#1}
        }
      \fi
      \expandafter
    \endgroup
    \childdoctmp
  \fi
}
%    \end{macrocode}

% \macro{\childdocof}
% The command |\childdocof| redirects
% compilation to the main file |#1|.
%    \begin{macrocode}
\newcommand{\childdocof}[1]
{
  \childdocdisable
  \childdoctrue
  \includeonly{\childdocname}
  \def\jobname{#1}
  \def\childdocjob{#1}
  \input{#1}
}
%    \end{macrocode}

% \macro{\childdocby}
% The command |\childdocby| ....
%    \begin{macrocode}
\newcommand{\childdocby}[2][]
{
  \childdocdisable
  \childdoctrue
  \childdocmanualtrue
  \if?#1?\else
    \def\jobname{#2}
  \fi
  \def\childdocjob{#2}
  \input{#2}
  \endinput
}
%    \end{macrocode}

% \macro{\childdocforward}
% The command |\childdocforward| redirects
% compilation to the main file or
% (if the optional argument is given) a child file.
% Parameters are set as if the main file
% or a child file starting with |\childdocof| was compiled.
% Then compilation is handed over to the main file:
%    \begin{macrocode}
\newcommand{\childdocforward}[2][]
{
  \begingroup
    \if?#1?
      \def\childdoctmp
      {
        \def\childdocname{#2}
        \def\childdocjob{#2}
        \def\jobname{#2}
        \input{#2}
        \endinput
      }
    \else
      \def\childdoctmp
      {
        \childdocdisable
        \def\childdocname{#2}
        \childdoctrue
        \includeonly{#2}
        \def\childdocjob{#1}
        \def\jobname{#1}
        \input{#1}
        \endinput
      }
    \fi
    \expandafter
  \endgroup
  \childdoctmp
}
%    \end{macrocode}

% \macro{\childdocforwardprefix}
% The command |\childdocforwardprefix| redirects
% compilation to the main or a child file by means of a pattern.
% The prefix |#1| in the current filename is replaced by |#2|
% and the suffix of the current filename is kept
% (it is assumed that the filename does not contain the substring `|~~~|'
% which is used as a delimiter).
% Compilation is handed over to the new file by |\childdocforward|:
%    \begin{macrocode}
\newcommand{\childdocforwardprefix}[3][]
{
  \begingroup
    \def\childdocextract #2##1~~~{\def\childdoctmp{\childdocforward[#1]{#3##1}}}
    \expandafter\childdocextract\childdocname~~~
    \expandafter
  \endgroup
  \childdoctmp
}
%    \end{macrocode}

% \macro{\childdoc}
% The deprecated macro |\childdoc| is a legacy version of |\childdocmain|:
%    \begin{macrocode}
\newcommand{\childdoc}{\childdocmain}
%    \end{macrocode}

% \macro{\childdocredirect}
% The deprecated macro |\childdocredirect| is a legacy version
% of |\childdocforward| and |\childdocforwardprefix|:
%    \begin{macrocode}
\newcommand{\childdocredirect}[2][]
{
  \begingroup
    \if?#1?
      \def\childdoctmp{\childdocforward{#2}}
    \else
      \def\childdoctmp{\childdocforwardprefix{#1}{#2}}
    \fi
    \expandafter
  \endgroup
  \childdoctmp
}
%    \end{macrocode}

%\iffalse
%</package>
%\fi
%
\endinput
\childdocforward{cdocsch2}"|
% \end{tabular}
% \end{center}
% Note that the trailing backslash on each first line
% merely continues the input to the second line
% (for convenient cut ant paste).
% Furthermore, the command |latex| can be replaced by any
% of its alternative versions such as |pdflatex|.
%
% %%%%%%%%%%%%%%%%%%%%%%%%%%%%%%%%%%%%%%%%%%%%%%%%%%%%%%%%%%%%%%%%%%%%%%%%%%%%%%
% %%%%%%%%%%%%%%%%%%%%%%%%%%%%%%%%%%%%%%%%%%%%%%%%%%%%%%%%%%%%%%%%%%%%%%%%%%%%%%
% \section{Implementation}
%\iffalse
%<*package>
%\fi
%
% This section describes the definitions file |childdoc.def|.

% The definitions cannot be loaded using |\usepackage| or |\RequirePackage|
% which has a mechanism to prevent loading a style file more than once.
% When loading the definitions by means of |\input|
% multiple instances have to be prevented manually:
%\iffalse
%This code needs to be before the `\ProvidesFile' directive
%which is defined at the beginning of this file.
%Therefore it is also placed there and commented out here.
%</package>
%<*discard>
%\fi
%    \begin{macrocode}
\ifdefined\childdocmain\endinput\fi
%    \end{macrocode}
%\iffalse
%</discard>
%<*package>
%\fi
%
% \macro{\ifchilddoc}
% \macro{\ifchilddocmanual}
% The conditional |\ifchilddoc| tells whether a
% child (true) or main (false) document is being compiled.
% The conditional |\ifchilddocmanual| tells whether
% the |\includeonly| mechanism is used (false) or
% the selection of child files must be performed manually (true).
% The definitions initialise to false:
%    \begin{macrocode}
\newif\ifchilddoc
\newif\ifchilddocmanual
%    \end{macrocode}

% \macro{\childdocname}
% \macro{\childdocjob}
% The macro |\childdocname| stores the name of the main document
% to be compiled. The macro |\childdocjob| stores the name of
% the document on which the \LaTeX{} compiler was originally invoked.
% The content of |\jobname| cannot be compared
% to filenames specified in the source due to different catcodes.
% The following code rescans |\jobname|, stores the result
% in |\childdocname| and saves a copy in |\childdocjob|:
%    \begin{macrocode}
\edef\childdocname{\scantokens\expandafter{\jobname\noexpand}}
\let\childdocjob\childdocname
%    \end{macrocode}

% \macro{\childdocdisable}
% The macro |\childdocdisable| prevents the main file
% from being processed more than once.
% At this stage, the main document command |\childdocmain|
% is assumed to be called once again where it should do nothing.
% Any subsequent call to it should prevent
% a secondary processing of the main document
% It overwrites the forwarding commands
% |\childdocof| and |\childdocforward|
% with empty macros to prevent further inclusions of the main document:
%    \begin{macrocode}
\newcommand{\childdocdisable}
{
  \renewcommand{\childdocmain}[1]{\renewcommand{\childdocmain}[1]{\endinput}}
  \renewcommand{\childdocof}[1]{}
  \renewcommand{\childdocby}[2][]{}
  \renewcommand{\childdocforward}[2][]{}
  \renewcommand{\childdocdisable}{}
}
%    \end{macrocode}

% \macro{\childdocmain}
% The macro |\childdocmain| is to be called at the top of the main file
% with nothing or the main filename (without extension) as argument.
% First, it breaks loops.
% If the argument is not empty and does not match |\childdocname|
% (which is set by the first inclusion of |childdoc.def|),
% |\ifchilddoc| is set to true, |\includeonly| is applied to the child file
% and |\jobname| is set to the main file
% (for proper handling of |.aux| files):
%    \begin{macrocode}
\newcommand{\childdocmain}[1]
{
  \childdocdisable\childdocmain{}
  \if?#1?\else
    \begingroup
      \def\childdoctmp{#1}
      \ifx\childdoctmp\childdocname
        \def\childdoctmp{}
      \else
        \def\childdoctmp
        {
          \childdoctrue
          \includeonly{\childdocname}
          \def\childdocjob{#1}
          \def\jobname{#1}
        }
      \fi
      \expandafter
    \endgroup
    \childdoctmp
  \fi
}
%    \end{macrocode}

% \macro{\childdocof}
% The command |\childdocof| redirects
% compilation to the main file |#1|.
%    \begin{macrocode}
\newcommand{\childdocof}[1]
{
  \childdocdisable
  \childdoctrue
  \includeonly{\childdocname}
  \def\jobname{#1}
  \def\childdocjob{#1}
  \input{#1}
}
%    \end{macrocode}

% \macro{\childdocby}
% The command |\childdocby| ....
%    \begin{macrocode}
\newcommand{\childdocby}[2][]
{
  \childdocdisable
  \childdoctrue
  \childdocmanualtrue
  \if?#1?\else
    \def\jobname{#2}
  \fi
  \def\childdocjob{#2}
  \input{#2}
  \endinput
}
%    \end{macrocode}

% \macro{\childdocforward}
% The command |\childdocforward| redirects
% compilation to the main file or
% (if the optional argument is given) a child file.
% Parameters are set as if the main file
% or a child file starting with |\childdocof| was compiled.
% Then compilation is handed over to the main file:
%    \begin{macrocode}
\newcommand{\childdocforward}[2][]
{
  \begingroup
    \if?#1?
      \def\childdoctmp
      {
        \def\childdocname{#2}
        \def\childdocjob{#2}
        \def\jobname{#2}
        \input{#2}
        \endinput
      }
    \else
      \def\childdoctmp
      {
        \childdocdisable
        \def\childdocname{#2}
        \childdoctrue
        \includeonly{#2}
        \def\childdocjob{#1}
        \def\jobname{#1}
        \input{#1}
        \endinput
      }
    \fi
    \expandafter
  \endgroup
  \childdoctmp
}
%    \end{macrocode}

% \macro{\childdocforwardprefix}
% The command |\childdocforwardprefix| redirects
% compilation to the main or a child file by means of a pattern.
% The prefix |#1| in the current filename is replaced by |#2|
% and the suffix of the current filename is kept
% (it is assumed that the filename does not contain the substring `|~~~|'
% which is used as a delimiter).
% Compilation is handed over to the new file by |\childdocforward|:
%    \begin{macrocode}
\newcommand{\childdocforwardprefix}[3][]
{
  \begingroup
    \def\childdocextract #2##1~~~{\def\childdoctmp{\childdocforward[#1]{#3##1}}}
    \expandafter\childdocextract\childdocname~~~
    \expandafter
  \endgroup
  \childdoctmp
}
%    \end{macrocode}

% \macro{\childdoc}
% The deprecated macro |\childdoc| is a legacy version of |\childdocmain|:
%    \begin{macrocode}
\newcommand{\childdoc}{\childdocmain}
%    \end{macrocode}

% \macro{\childdocredirect}
% The deprecated macro |\childdocredirect| is a legacy version
% of |\childdocforward| and |\childdocforwardprefix|:
%    \begin{macrocode}
\newcommand{\childdocredirect}[2][]
{
  \begingroup
    \if?#1?
      \def\childdoctmp{\childdocforward{#2}}
    \else
      \def\childdoctmp{\childdocforwardprefix{#1}{#2}}
    \fi
    \expandafter
  \endgroup
  \childdoctmp
}
%    \end{macrocode}

%\iffalse
%</package>
%\fi
%
\endinput
\childdocforward{cdocsch2}"|
% \end{tabular}
% \end{center}
% Note that the trailing backslash on each first line
% merely continues the input to the second line
% (for convenient cut ant paste).
% Furthermore, the command |latex| can be replaced by any
% of its alternative versions such as |pdflatex|.
%
% %%%%%%%%%%%%%%%%%%%%%%%%%%%%%%%%%%%%%%%%%%%%%%%%%%%%%%%%%%%%%%%%%%%%%%%%%%%%%%
% %%%%%%%%%%%%%%%%%%%%%%%%%%%%%%%%%%%%%%%%%%%%%%%%%%%%%%%%%%%%%%%%%%%%%%%%%%%%%%
% \section{Implementation}
%\iffalse
%<*package>
%\fi
%
% This section describes the definitions file |childdoc.def|.

% The definitions cannot be loaded using |\usepackage| or |\RequirePackage|
% which has a mechanism to prevent loading a style file more than once.
% When loading the definitions by means of |\input|
% multiple instances have to be prevented manually:
%\iffalse
%This code needs to be before the `\ProvidesFile' directive
%which is defined at the beginning of this file.
%Therefore it is also placed there and commented out here.
%</package>
%<*discard>
%\fi
%    \begin{macrocode}
\ifdefined\childdocmain\endinput\fi
%    \end{macrocode}
%\iffalse
%</discard>
%<*package>
%\fi
%
% \macro{\ifchilddoc}
% \macro{\ifchilddocmanual}
% The conditional |\ifchilddoc| tells whether a
% child (true) or main (false) document is being compiled.
% The conditional |\ifchilddocmanual| tells whether
% the |\includeonly| mechanism is used (false) or
% the selection of child files must be performed manually (true).
% The definitions initialise to false:
%    \begin{macrocode}
\newif\ifchilddoc
\newif\ifchilddocmanual
%    \end{macrocode}

% \macro{\childdocname}
% \macro{\childdocjob}
% The macro |\childdocname| stores the name of the main document
% to be compiled. The macro |\childdocjob| stores the name of
% the document on which the \LaTeX{} compiler was originally invoked.
% The content of |\jobname| cannot be compared
% to filenames specified in the source due to different catcodes.
% The following code rescans |\jobname|, stores the result
% in |\childdocname| and saves a copy in |\childdocjob|:
%    \begin{macrocode}
\edef\childdocname{\scantokens\expandafter{\jobname\noexpand}}
\let\childdocjob\childdocname
%    \end{macrocode}

% \macro{\childdocdisable}
% The macro |\childdocdisable| prevents the main file
% from being processed more than once.
% At this stage, the main document command |\childdocmain|
% is assumed to be called once again where it should do nothing.
% Any subsequent call to it should prevent
% a secondary processing of the main document
% It overwrites the forwarding commands
% |\childdocof| and |\childdocforward|
% with empty macros to prevent further inclusions of the main document:
%    \begin{macrocode}
\newcommand{\childdocdisable}
{
  \renewcommand{\childdocmain}[1]{\renewcommand{\childdocmain}[1]{\endinput}}
  \renewcommand{\childdocof}[1]{}
  \renewcommand{\childdocby}[2][]{}
  \renewcommand{\childdocforward}[2][]{}
  \renewcommand{\childdocdisable}{}
}
%    \end{macrocode}

% \macro{\childdocmain}
% The macro |\childdocmain| is to be called at the top of the main file
% with nothing or the main filename (without extension) as argument.
% First, it breaks loops.
% If the argument is not empty and does not match |\childdocname|
% (which is set by the first inclusion of |childdoc.def|),
% |\ifchilddoc| is set to true, |\includeonly| is applied to the child file
% and |\jobname| is set to the main file
% (for proper handling of |.aux| files):
%    \begin{macrocode}
\newcommand{\childdocmain}[1]
{
  \childdocdisable\childdocmain{}
  \if?#1?\else
    \begingroup
      \def\childdoctmp{#1}
      \ifx\childdoctmp\childdocname
        \def\childdoctmp{}
      \else
        \def\childdoctmp
        {
          \childdoctrue
          \includeonly{\childdocname}
          \def\childdocjob{#1}
          \def\jobname{#1}
        }
      \fi
      \expandafter
    \endgroup
    \childdoctmp
  \fi
}
%    \end{macrocode}

% \macro{\childdocof}
% The command |\childdocof| redirects
% compilation to the main file |#1|.
%    \begin{macrocode}
\newcommand{\childdocof}[1]
{
  \childdocdisable
  \childdoctrue
  \includeonly{\childdocname}
  \def\jobname{#1}
  \def\childdocjob{#1}
  \input{#1}
}
%    \end{macrocode}

% \macro{\childdocby}
% The command |\childdocby| ....
%    \begin{macrocode}
\newcommand{\childdocby}[2][]
{
  \childdocdisable
  \childdoctrue
  \childdocmanualtrue
  \if?#1?\else
    \def\jobname{#2}
  \fi
  \def\childdocjob{#2}
  \input{#2}
  \endinput
}
%    \end{macrocode}

% \macro{\childdocforward}
% The command |\childdocforward| redirects
% compilation to the main file or
% (if the optional argument is given) a child file.
% Parameters are set as if the main file
% or a child file starting with |\childdocof| was compiled.
% Then compilation is handed over to the main file:
%    \begin{macrocode}
\newcommand{\childdocforward}[2][]
{
  \begingroup
    \if?#1?
      \def\childdoctmp
      {
        \def\childdocname{#2}
        \def\childdocjob{#2}
        \def\jobname{#2}
        \input{#2}
        \endinput
      }
    \else
      \def\childdoctmp
      {
        \childdocdisable
        \def\childdocname{#2}
        \childdoctrue
        \includeonly{#2}
        \def\childdocjob{#1}
        \def\jobname{#1}
        \input{#1}
        \endinput
      }
    \fi
    \expandafter
  \endgroup
  \childdoctmp
}
%    \end{macrocode}

% \macro{\childdocforwardprefix}
% The command |\childdocforwardprefix| redirects
% compilation to the main or a child file by means of a pattern.
% The prefix |#1| in the current filename is replaced by |#2|
% and the suffix of the current filename is kept
% (it is assumed that the filename does not contain the substring `|~~~|'
% which is used as a delimiter).
% Compilation is handed over to the new file by |\childdocforward|:
%    \begin{macrocode}
\newcommand{\childdocforwardprefix}[3][]
{
  \begingroup
    \def\childdocextract #2##1~~~{\def\childdoctmp{\childdocforward[#1]{#3##1}}}
    \expandafter\childdocextract\childdocname~~~
    \expandafter
  \endgroup
  \childdoctmp
}
%    \end{macrocode}

% \macro{\childdoc}
% The deprecated macro |\childdoc| is a legacy version of |\childdocmain|:
%    \begin{macrocode}
\newcommand{\childdoc}{\childdocmain}
%    \end{macrocode}

% \macro{\childdocredirect}
% The deprecated macro |\childdocredirect| is a legacy version
% of |\childdocforward| and |\childdocforwardprefix|:
%    \begin{macrocode}
\newcommand{\childdocredirect}[2][]
{
  \begingroup
    \if?#1?
      \def\childdoctmp{\childdocforward{#2}}
    \else
      \def\childdoctmp{\childdocforwardprefix{#1}{#2}}
    \fi
    \expandafter
  \endgroup
  \childdoctmp
}
%    \end{macrocode}

%\iffalse
%</package>
%\fi
%
\endinput

\childdocmain{exfserm}
%    \end{macrocode}
% The parameter of |\childdocmain| must match the main file name |exfserm|.

% %%%%%%%%%%%%%%%%%%%%%%%%%%%%%%%%%%%%%%
% \paragraph{Compilation Switches.}
%
% Define compilation switches and declare their default settings.
% |\printsol| controls whether solutions should be printed or not;
% by default solutions are activated for compilation of a part,
% but not for the complete document.
% |\draftver| controls whether the final version of
% the document is to be compiled;
% concretely this affects the compilations of metapost figures, see below:
%    \begin{macrocode}
\providecommand{\draftver}{y}
\ifchilddoc
\providecommand{\printsol}{y}
\else
\providecommand{\printsol}{n}
\fi
%    \end{macrocode}

% %%%%%%%%%%%%%%%%%%%%%%%%%%%%%%%%%%%%%%
% \paragraph{Preamble.}
%
% Standard document class:
%    \begin{macrocode}
\documentclass[12pt]{article}
%    \end{macrocode}

%    \begin{macrocode}
% \textsf{graphicx} package to display license logo:
\RequirePackage{graphicx}
%    \end{macrocode}

% %%%%%%%%%%%%%%%%%%%%%%%%%%%%%%%%%%%%%%
% \paragraph{hyperref Package.}
%
% Use the \textsf{hyperref} package.
% Declare some options, e.g.\ use bookmarks only for complete document:
%    \begin{macrocode}
\PassOptionsToPackage{bookmarks=\ifchilddoc false\else true\fi}{hyperref}
\PassOptionsToPackage{bookmarksopen=true}{hyperref}
\RequirePackage{hyperref}
%    \end{macrocode}
% Use the \textsf{hyperxmp} package for inclusion of
% copyright metadata in the compiled pdf files:
%    \begin{macrocode}
\IfFileExists{hyperxmp.sty}{\RequirePackage{hyperxmp}}{}
%    \end{macrocode}

% %%%%%%%%%%%%%%%%%%%%%%%%%%%%%%%%%%%%%%
% \paragraph{exframe Package.}
%
% Invoke \textsf{exframe} with extended data and styles:
%    \begin{macrocode}
\RequirePackage[extdata,extstyle]{exframe}
%    \end{macrocode}

% Set solutions switch and declare two-sided layout only
% if no solutions are printed:
%    \begin{macrocode}
\if\printsol n
\exercisesetup{solutions=false}
\exercisesetup{twoside=true}
\else
\exercisesetup{solutions=true}
\exercisesetup{twoside=false}
\fi
%    \end{macrocode}

% Might want to display some metadata (only for partial compile):
%    \begin{macrocode}
%%\if\printsol n\else\showprobleminfo{author,source,recycle}\fi
%    \end{macrocode}

% Set some options. Automatically assign labels to problems.
% Include sheets and problems in table of contents.
% Separate solutions by horizontal rules.
% Count problems, equations and pages by sheet
% (unless compiling single problem).
% Collect solutions below each problem.
% Write pdf metadata for sheet when compiling single sheet.
%    \begin{macrocode}
\exercisesetup{autolabelproblem}
\exercisestyle{contents,solutionsep}
\ifchilddocmanual\else
\exercisestyle{pagebysheet,problembysheet,equationbysheet,sheetequation}
\fi
\exercisestyle{solutionbelow={problem}}
\ifchilddoc\ifchilddocmanual\else\exercisesetup{pdfdata=sheet}\fi\fi
%    \end{macrocode}

% %%%%%%%%%%%%%%%%%%%%%%%%%%%%%%%%%%%%%%
% \paragraph{Layout Definitions.}
%
% Set page dimensions and layout:
%    \begin{macrocode}
\RequirePackage[a4paper,margin=2.5cm]{geometry}
\pagestyle{plain}
%    \end{macrocode}
% Remove paragraph indentation:
%    \begin{macrocode}
\setlength\parindent{0pt}
\setlength\parskip{\smallskipamount}
%    \end{macrocode}
% Show overfull lines:
%    \begin{macrocode}
\setlength\overfullrule{5pt}
%    \end{macrocode}
% Define turn page over mark; hide when printing solutions:
%    \begin{macrocode}
\newcommand{\turnover}{\ifsolutions\else\vfill%
  \hfill{\mathversion{bold}$\longrightarrow$}\newpage\fi}
%    \end{macrocode}

% %%%%%%%%%%%%%%%%%%%%%%%%%%%%%%%%%%%%%%
% \paragraph{Sheet Banner.}
%
% Use standard sheet banner;
% show sheet |editdate| below banner (if declared);
% when compiling single sheet, display sheet due date instead:
%    \begin{macrocode}
\exercisestyle{plainheader}
\exerciseconfig{composeheaderbelowright}
 {\sheetdataempty{editdate}{}{version: \getsheetdata{editdate}}}
\ifchilddoc\ifsolutions\else
 \exerciseconfig{composeheaderbelowright}
  {\sheetdataempty{due}{}{due: \getsheetdata{due}}}
\fi\fi
%    \end{macrocode}

% %%%%%%%%%%%%%%%%%%%%%%%%%%%%%%%%%%%%%%
% \paragraph{Lorem.}
%
% Define a macro |\lorem| to write out some paragraph of text
% for the example:
%    \begin{macrocode}
\def\lorem{Lorem ipsum dolor sit amet, consectetur adipisici elit,
  sed eiusmod tempor incidunt ut labore et dolore magna aliqua.
  Ut enim ad minim veniam, quis nostrud exercitation ullamco laboris
  nisi ut aliquid ex ea commodi consequat.
  Quis aute iure reprehenderit in voluptate velit esse
  cillum dolore eu fugiat nulla pariatur.
  Excepteur sint obcaecat cupiditat non proident,
  sunt in culpa qui officia deserunt mollit anim id est laborum.\par}
%    \end{macrocode}

% %%%%%%%%%%%%%%%%%%%%%%%%%%%%%%%%%%%%%%
% \paragraph{metapost Setup.}
%
% Use package \textsf{mpostinl} (if available)
% to include some metapost figures within the source of the problems:
%    \begin{macrocode}
\IfFileExists{mpostinl.sty}{\RequirePackage{mpostinl}}{}
\ifdefined\mpostsetup
%    \end{macrocode}

% Setup \textsf{mpostinl}.
% Use checksums to invoke metapost only when figures change.
% Process all figures individually and immediately when in
% draft mode (avoids a second compilation pass).
% Number figures within sheet to provide a stable numbering
% upon insertion/deletion of new figures or partial compilation.
% Do not warn about unused figures when preparing without solutions
% (figures for solutions should be declared outside |solution| environments):
%    \begin{macrocode}
\mpostsetup{checksum}
\if\draftver y\mpostsetup{now,nowall}\fi
\ifchilddocmanual\else\mpostsetup{numberwithin={sheet}}\fi
\ifsolutions\else\mpostsetup{warnunused=false}\fi
%    \end{macrocode}

% Global metapost definitions.
% Define some latex macro to demonstrate
% usage of latex for typesetting labels.
% Define some global metapost variables,
% |paths| is an array of path variables,
% |xu| serves as a length unit to
% scale individual figures
% (use |interim xu:=...;| to set scale for local figure only):
%    \begin{macrocode}
\mpostsetup{globaldef=true}
\begin{mposttex}
\def\figure{figure}
\end{mposttex}
\begin{mpostdef}
path paths[];
newinternal numeric xu;
xu:=1cm;
\end{mpostdef}
\mpostsetup{globaldef=false}
%    \end{macrocode}

% Close optional \textsf{mpostinl} processing:
%    \begin{macrocode}
\fi
%    \end{macrocode}

% %%%%%%%%%%%%%%%%%%%%%%%%%%%%%%%%%%%%%%
% \paragraph{Document Data.}
%
% Set document data for series of problem sheets:
%    \begin{macrocode}
\exercisedata{course={exframe package samples}}
\exercisedata{instructor={N.\ Beisert}}
\exercisedata{institution={exframe academy}}
\exercisedata{period={spring 2019}}
\exercisedata{date={2019}}
%    \end{macrocode}

% Assemble some entries from given data:
%    \begin{macrocode}
\exercisedata{material={\ifsolutions\getexerciseconfig{termsolutions}%
  \else\getexerciseconfig{termsheets}\fi}}
\exerciseconfig{composetitlesheet}[2]{\exerciseifempty{#2}%
  {\ifsolutions\getexerciseconfig{termsolutions}\else%
   \getexerciseconfig{termsheet}\fi\ #1}%
  {\ifsolutions\getexerciseconfig{termsolutions}\fi\ #2}}
\exerciseconfig{composemetasheet}[2]{\if\draftver yDRAFT: \fi%
  \getexercisedata{course},
  \getexerciseconfig{composetitlesheet}{#1}{#2}}
%    \end{macrocode}

% Set general purpose metadata:
%    \begin{macrocode}
\exercisedata{title={\if\draftver yDRAFT: \fi%
  \getexercisedata{course}, \getexercisedata{material}}}
\exercisedata{author={Niklas Beisert, \getexercisedata{institution}}}
\exercisedata{subject={lecture series at \getexercisedata{institution},
  \getexercisedata{period}}}
%    \end{macrocode}

% Redefine some terms:
%    \begin{macrocode}
\exerciseconfig{termsheet}{sheet}
\exerciseconfig{termsheets}{sample sheets}
\exerciseconfig{termsolution}{Solution}
\exerciseconfig{termsolutions}{solutions}
%    \end{macrocode}

% %%%%%%%%%%%%%%%%%%%%%%%%%%%%%%%%%%%%%%
% \paragraph{License.}
%
% It is always useful to specify a copyright line
% and a license for the document.
% Define some restrictive default text.
%    \begin{macrocode}
\def\copyrightmessage
  {This document as well as its parts is protected by copyright.}
\def\licensemessage
  {Reproduction of any part of this work in any form
  without prior written consent of \getexercisedata{institution}
  is not permissible.}
%    \end{macrocode}

% Apply Creative Commons BY-SA license
% under certain conditions (no solutions, final version):
%    \begin{macrocode}
\ifsolutions\else\if\draftver y\else
\def\licensecctype{by-sa}
\def\licenseccname{Attribution-ShareAlike \licenseccver{} International}
\fi\fi
%    \end{macrocode}

% Write applicable Creative Commons license information.
% Use Creative Commons logos included in package \textsf{doclicense}:
%    \begin{macrocode}
\ifdefined\licensecctype
\ifdefined\licenseccver\else\def\licenseccver{4.0}\fi
\def\licenseurl
  {https://creativecommons.org/licenses/\licensecctype/\licenseccver}
\def\licensemessage{\texorpdfstring
  {This work is licensed under the Creative Commons ``\licenseccname'' License
  (CC \MakeUppercase{\licensecctype} \licenseccver).\par
  \IfFileExists{doclicense.sty}{
   \begin{center}\includegraphics{doclicense-CC-\licensecctype}\end{center}}{}
  To view a copy of this license, visit: \url{\licenseurl}}
  {This work is licensed under the Creative Commons \licenseccname{} License.}}
\fi
%    \end{macrocode}

% Write copyright and license as pdf metadata
% (using package \textsf{hyperxmp} if available):
%    \begin{macrocode}
\ifdefined\xmptilde
\hypersetup{pdfcopyright={Copyright \getexercisedata{date}
  \getexercisedata{author}. \copyrightmessage{} \licensemessage}}
\ifdefined\sourceurl\hypersetup{pdflicenseurl={\sourceurl}}\else
\ifdefined\licenseurl\hypersetup{pdflicenseurl={\licenseurl}}\fi\fi
\hypersetup{keeppdfinfo=true}
\hypersetup{pdfsource={}}
\XMPLangAlt{en}{pdfcopyright={Copyright \getexercisedata{date}
  \getexercisedata{author}. \copyrightmessage{} \licensemessage}}
\fi
%    \end{macrocode}

% %%%%%%%%%%%%%%%%%%%%%%%%%%%%%%%%%%%%%%
% \paragraph{Body.}
%
% Start document body:
%    \begin{macrocode}
\begin{document}
%    \end{macrocode}

% %%%%%%%%%%%%%%%%%%%%%%%%%%%%%%%%%%%%%%
% \paragraph{Single Problem Display.}
%
% The following code handles the compilation
% of individual problems from their own source file.
% Make sure to leave the conditional before issuing |\end{document}|:
%    \begin{macrocode}
\def\tmp{}
\ifchilddocmanual
\def\tmp{\end{document}}
\input{\childdocname}
\fi\tmp
%    \end{macrocode}

% %%%%%%%%%%%%%%%%%%%%%%%%%%%%%%%%%%%%%%
% \paragraph{Frontmatter.}
%
% Do not print frontmatter for individual sheets.
% Define a plain page counter for frontmatter:
%    \begin{macrocode}
\setcounter{section}{-1}
\begingroup\ifchilddoc\else
\renewcommand{\thepage}{\arabic{page}}
%    \end{macrocode}

% Prepare a title page to display some relevant data:
%    \begin{macrocode}
\pdfbookmark[1]{Title Page}{title}
\thispagestyle{empty}
\vspace*{\fill}
\begin{center}
\begingroup\bfseries\LARGE\getexercisedata{course}\par\endgroup
\vspace{0.5cm}
\begingroup\large\getexercisedata{material}\par\endgroup
\vspace{0.5cm}
\begingroup\large\getexercisedata{institution},
  \getexercisedata{period}\par\endgroup
\vspace{2cm}
\begingroup\scshape\Large\getexercisedata{instructor}\par\endgroup
\end{center}
\vspace*{\fill}\vspace*{\fill}
\newpage
%    \end{macrocode}

% Prepare a copyright and license page using data specified above:
%    \begin{macrocode}
\phantomsection\pdfbookmark[1]{Copyright}{copyright}
\thispagestyle{empty}
\vspace*{\fill}\vspace*{\fill}
\begin{center}
\begin{minipage}{11cm}\raggedright
{\copyright} \getexercisedata{date} \getexercisedata{author}
\par\medskip
\copyrightmessage{}
\licensemessage
\ifdefined\sourcemessage\par\medskip\sourcemessage\fi
\ifdefined\attributionmessage\par\medskip\attributionmessage\fi
\end{minipage}\end{center}
\vspace*{\fill}\vspace*{\fill}\vspace*{\fill}
\newpage
%    \end{macrocode}

% Print table of contents:
%    \begin{macrocode}
\makeatletter\renewcommand\@pnumwidth{2.4em}\makeatother
\setcounter{tocdepth}{2}
\phantomsection\pdfbookmark[1]{Contents}{contents}
{\parskip0pt\tableofcontents}
\exercisecleardoublepage\setcounter{page}{1}
%    \end{macrocode}

% End of frontmatter:
%    \begin{macrocode}
\fi\endgroup
%    \end{macrocode}

% %%%%%%%%%%%%%%%%%%%%%%%%%%%%%%%%%%%%%%
% \paragraph{Include Sheets.}
%
% Include problem sheets:
%    \begin{macrocode}
\include{exfser01}
\include{exfser02}
\include{exfser03}
%    \end{macrocode}

% Include sheet to collect unused problems
% (only process for individual sheets, i.e.\ itself):
%    \begin{macrocode}
\def\jobnameunused{exfseraa}
\ifx\childdocname\jobnameunused\include{\jobnameunused}\fi
%    \end{macrocode}

% %%%%%%%%%%%%%%%%%%%%%%%%%%%%%%%%%%%%%%
% \paragraph{End.}
%
% End of document body:
%    \begin{macrocode}
\end{document}
%    \end{macrocode}
%
%\iffalse
%</samplemultimain>
%\fi
%
% %%%%%%%%%%%%%%%%%%%%%%%%%%%%%%%%%%%%%%%%%%%%%%%%%%%%%%%%%%%%%%%%%%%%%%%%%%%%%%
% %%%%%%%%%%%%%%%%%%%%%%%%%%%%%%%%%%%%%%%%%%%%%%%%%%%%%%%%%%%%%%%%%%%%%%%%%%%%%%
% \subsection{Sheet File}
% \label{sec:samplemultisheets}
%\iffalse
%<*samplemultisheet1|samplemultisheet2|samplemultisheet3|samplemultisheeta>
%\fi
%
% Provide some source files
% |exfser01.tex|, |exfser02.tex|, |exfser03.tex|, |exfseraa.tex|
% for problem sheets.
%
% %%%%%%%%%%%%%%%%%%%%%%%%%%%%%%%%%%%%%%
% \paragraph{childdoc Mechanism.}
%
% Instruct the package \textsf{childdoc} to compile
% only the present sheet if source is compiled by latex:
%    \begin{macrocode}
%%\providecommand{\printsol}{n}
% \iffalse
%
% childdoc.dtx Copyright (C) 2017-2018 Niklas Beisert
%
% This work may be distributed and/or modified under the
% conditions of the LaTeX Project Public License, either version 1.3
% of this license or (at your option) any later version.
% The latest version of this license is in
%   http://www.latex-project.org/lppl.txt
% and version 1.3 or later is part of all distributions of LaTeX
% version 2005/12/01 or later.
%
% This work has the LPPL maintenance status `maintained'.
%
% The Current Maintainer of this work is Niklas Beisert.
%
% This work consists of the files childdoc.dtx and childdoc.ins
% and the derived files childdoc.def and cdocsamp.tex with
% cdocsch1.tex, cdocsch2.tex, cdocsdrf.tex, cdocsfn1.tex, cdocsfn2.tex.
%
%<package>\ifdefined\childdocmain\endinput\fi
%<package>\ProvidesFile{childdoc.def}[2018/12/30 v2.0 child document driver]
%<samplemain>\ProvidesFile{cdocsamp.tex}[2018/12/30 v2.0 sample for childdoc]
%<*driver>
%\ProvidesFile{childdoc.drv}[2018/12/30 v2.0 childdoc reference manual file]
\PassOptionsToClass{10pt,a4paper}{article}
\documentclass{ltxdoc}

\usepackage[margin=35mm]{geometry}
\usepackage{hyperref}
\usepackage{hyperxmp}
\usepackage[usenames]{color}

\hypersetup{colorlinks=true}
\hypersetup{pdfstartview=FitH}
\hypersetup{pdfpagemode=UseNone}
\hypersetup{pdfsource={}}
\hypersetup{pdflang={en-UK}}
\hypersetup{pdfcopyright={Copyright 2017-2018 Niklas Beisert.
  This work may be distributed and/or modified under the
  conditions of the LaTeX Project Public License, either version 1.3
  of this license or (at your option) any later version.}}
\hypersetup{pdflicenseurl={http://www.latex-project.org/lppl.txt}}
\hypersetup{pdfcontactaddress={ETH Zurich, ITP, HIT K,
  Wolfgang-Pauli-Strasse 27}}
\hypersetup{pdfcontactpostcode={8093}}
\hypersetup{pdfcontactcity={Zurich}}
\hypersetup{pdfcontactcountry={Switzerland}}
\hypersetup{pdfcontactemail={nbeisert@itp.phys.ethz.ch}}
\hypersetup{pdfcontacturl={http://people.phys.ethz.ch/\xmptilde nbeisert/}}

\newcommand{\secref}[1]{\hyperref[#1]{section \ref*{#1}}}

\parskip1ex
\parindent0pt
\let\olditemize\itemize
\def\itemize{\olditemize\parskip0pt}

\begin{document}

\title{The \textsf{childdoc} Package}
\hypersetup{pdftitle={The childdoc Package}}
\author{Niklas Beisert\\[2ex]
  Institut f\"ur Theoretische Physik\\
  Eidgen\"ossische Technische Hochschule Z\"urich\\
  Wolfgang-Pauli-Strasse 27, 8093 Z\"urich, Switzerland\\[1ex]
  \href{mailto:nbeisert@itp.phys.ethz.ch}
  {\texttt{nbeisert@itp.phys.ethz.ch}}}
\hypersetup{pdfauthor={Niklas Beisert}}
\hypersetup{pdfsubject={Manual for the LaTeX2e Package childdoc}}
\date{30 December 2018, \textsf{v2.0}}
\maketitle

\begin{abstract}\noindent
\textsf{childdoc} is a \LaTeXe{} package
that enables the direct compilation
of document sections included by |\include|
to individual files.
\end{abstract}

\begingroup
\parskip0ex
\tableofcontents
\endgroup

%%%%%%%%%%%%%%%%%%%%%%%%%%%%%%%%%%%%%%%%%%%%%%%%%%%%%%%%%%%%%%%%%%%%%%%%%%%%%%%%
%%%%%%%%%%%%%%%%%%%%%%%%%%%%%%%%%%%%%%%%%%%%%%%%%%%%%%%%%%%%%%%%%%%%%%%%%%%%%%%%
\section{Introduction}

\LaTeX{} provides a mechanism to structure a large document (such as a book)
into a main file and several child files (containing the chapters)
using the |\include| command.
This mechanism is beneficial for documents
which span hundreds of pages in order to
make the source file(s) more manageable.
Moreover, compilation can be restricted to
selected child files by means of the |\includeonly| command.
The latter feature can be used to reduce the compilation time while editing
(this was significantly more useful in the earlier days of \LaTeX{})
or to generate a smaller document which is easier to navigate.
Another application of |\includeonly| is to generate
documents consisting of selected parts of the complete document.

However, there are a few drawbacks of the plain |\include| mechanism:
\begin{itemize}
\item
The child files cannot be compiled on their own,
they can only be compiled via the main file.
A naive editing environment
(such as a text editor with an option
to have the current file processed by \LaTeX)
may require one to switch to the main file before compiling;
attempting to compile the child file produces errors.
\item
The main file must be modified (each time)
to adjust the |\includeonly| command
to the present needs. This easily leaves the main file in a messy state.
\item
The generated document will always carry the filename
of the main document. This is inconvenient if
several child files are to be compiled and
to be kept for distribution.
\end{itemize}

The present package provides a simple interface
to make child files individually compilable by \LaTeX{}.
Compiling a child file then has the same effect as compiling
the main file with an |\includeonly| command
to select the appropriate child.
Moreover the generated document will carry the name of the child
rather than the main file.
This resolves all three above issues.

This feature is meant to make the editing of books,
thesis documents and lecture notes somewhat more convenient.
However, the package can also be used efficiently for
composing a series of documents (such as exercise sheets)
which are typically distributed individually.
It then assists the author in generating the individual documents
(potentially in different versions)
as well as a document containing the collected series.
Another application is in developing style files
or other kinds of included material
where compilation of the style file could redirect
to a sample or test file.

%%%%%%%%%%%%%%%%%%%%%%%%%%%%%%%%%%%%%%%%%%%%%%%%%%%%%%%%%%%%%%%%%%%%%%%%%%%%%%%%
%%%%%%%%%%%%%%%%%%%%%%%%%%%%%%%%%%%%%%%%%%%%%%%%%%%%%%%%%%%%%%%%%%%%%%%%%%%%%%%%
\section{Usage}

First of all, the package \textsf{childdoc} is \emph{not} a standard
\LaTeXe{} |.sty| style file! Therefore it needs to be invoked in
a non-standard way.

%%%%%%%%%%%%%%%%%%%%%%%%%%%%%%%%%%%%%%%%%%%%%%%%%%%%%%%%%%%%%%%%%%%%%%%%%%%%%%%%
\subsection{Included Files}
\label{sec:include}

%%%%%%%%%%%%%%%%%%%%%%%%%%%%%%%%%%%%%%%%
\DescribeMacro{\childdocmain}
To use the package, add the commands
\begin{center}
\begin{tabular}{l}
|% \iffalse
%
% childdoc.dtx Copyright (C) 2017-2018 Niklas Beisert
%
% This work may be distributed and/or modified under the
% conditions of the LaTeX Project Public License, either version 1.3
% of this license or (at your option) any later version.
% The latest version of this license is in
%   http://www.latex-project.org/lppl.txt
% and version 1.3 or later is part of all distributions of LaTeX
% version 2005/12/01 or later.
%
% This work has the LPPL maintenance status `maintained'.
%
% The Current Maintainer of this work is Niklas Beisert.
%
% This work consists of the files childdoc.dtx and childdoc.ins
% and the derived files childdoc.def and cdocsamp.tex with
% cdocsch1.tex, cdocsch2.tex, cdocsdrf.tex, cdocsfn1.tex, cdocsfn2.tex.
%
%<package>\ifdefined\childdocmain\endinput\fi
%<package>\ProvidesFile{childdoc.def}[2018/12/30 v2.0 child document driver]
%<samplemain>\ProvidesFile{cdocsamp.tex}[2018/12/30 v2.0 sample for childdoc]
%<*driver>
%\ProvidesFile{childdoc.drv}[2018/12/30 v2.0 childdoc reference manual file]
\PassOptionsToClass{10pt,a4paper}{article}
\documentclass{ltxdoc}

\usepackage[margin=35mm]{geometry}
\usepackage{hyperref}
\usepackage{hyperxmp}
\usepackage[usenames]{color}

\hypersetup{colorlinks=true}
\hypersetup{pdfstartview=FitH}
\hypersetup{pdfpagemode=UseNone}
\hypersetup{pdfsource={}}
\hypersetup{pdflang={en-UK}}
\hypersetup{pdfcopyright={Copyright 2017-2018 Niklas Beisert.
  This work may be distributed and/or modified under the
  conditions of the LaTeX Project Public License, either version 1.3
  of this license or (at your option) any later version.}}
\hypersetup{pdflicenseurl={http://www.latex-project.org/lppl.txt}}
\hypersetup{pdfcontactaddress={ETH Zurich, ITP, HIT K,
  Wolfgang-Pauli-Strasse 27}}
\hypersetup{pdfcontactpostcode={8093}}
\hypersetup{pdfcontactcity={Zurich}}
\hypersetup{pdfcontactcountry={Switzerland}}
\hypersetup{pdfcontactemail={nbeisert@itp.phys.ethz.ch}}
\hypersetup{pdfcontacturl={http://people.phys.ethz.ch/\xmptilde nbeisert/}}

\newcommand{\secref}[1]{\hyperref[#1]{section \ref*{#1}}}

\parskip1ex
\parindent0pt
\let\olditemize\itemize
\def\itemize{\olditemize\parskip0pt}

\begin{document}

\title{The \textsf{childdoc} Package}
\hypersetup{pdftitle={The childdoc Package}}
\author{Niklas Beisert\\[2ex]
  Institut f\"ur Theoretische Physik\\
  Eidgen\"ossische Technische Hochschule Z\"urich\\
  Wolfgang-Pauli-Strasse 27, 8093 Z\"urich, Switzerland\\[1ex]
  \href{mailto:nbeisert@itp.phys.ethz.ch}
  {\texttt{nbeisert@itp.phys.ethz.ch}}}
\hypersetup{pdfauthor={Niklas Beisert}}
\hypersetup{pdfsubject={Manual for the LaTeX2e Package childdoc}}
\date{30 December 2018, \textsf{v2.0}}
\maketitle

\begin{abstract}\noindent
\textsf{childdoc} is a \LaTeXe{} package
that enables the direct compilation
of document sections included by |\include|
to individual files.
\end{abstract}

\begingroup
\parskip0ex
\tableofcontents
\endgroup

%%%%%%%%%%%%%%%%%%%%%%%%%%%%%%%%%%%%%%%%%%%%%%%%%%%%%%%%%%%%%%%%%%%%%%%%%%%%%%%%
%%%%%%%%%%%%%%%%%%%%%%%%%%%%%%%%%%%%%%%%%%%%%%%%%%%%%%%%%%%%%%%%%%%%%%%%%%%%%%%%
\section{Introduction}

\LaTeX{} provides a mechanism to structure a large document (such as a book)
into a main file and several child files (containing the chapters)
using the |\include| command.
This mechanism is beneficial for documents
which span hundreds of pages in order to
make the source file(s) more manageable.
Moreover, compilation can be restricted to
selected child files by means of the |\includeonly| command.
The latter feature can be used to reduce the compilation time while editing
(this was significantly more useful in the earlier days of \LaTeX{})
or to generate a smaller document which is easier to navigate.
Another application of |\includeonly| is to generate
documents consisting of selected parts of the complete document.

However, there are a few drawbacks of the plain |\include| mechanism:
\begin{itemize}
\item
The child files cannot be compiled on their own,
they can only be compiled via the main file.
A naive editing environment
(such as a text editor with an option
to have the current file processed by \LaTeX)
may require one to switch to the main file before compiling;
attempting to compile the child file produces errors.
\item
The main file must be modified (each time)
to adjust the |\includeonly| command
to the present needs. This easily leaves the main file in a messy state.
\item
The generated document will always carry the filename
of the main document. This is inconvenient if
several child files are to be compiled and
to be kept for distribution.
\end{itemize}

The present package provides a simple interface
to make child files individually compilable by \LaTeX{}.
Compiling a child file then has the same effect as compiling
the main file with an |\includeonly| command
to select the appropriate child.
Moreover the generated document will carry the name of the child
rather than the main file.
This resolves all three above issues.

This feature is meant to make the editing of books,
thesis documents and lecture notes somewhat more convenient.
However, the package can also be used efficiently for
composing a series of documents (such as exercise sheets)
which are typically distributed individually.
It then assists the author in generating the individual documents
(potentially in different versions)
as well as a document containing the collected series.
Another application is in developing style files
or other kinds of included material
where compilation of the style file could redirect
to a sample or test file.

%%%%%%%%%%%%%%%%%%%%%%%%%%%%%%%%%%%%%%%%%%%%%%%%%%%%%%%%%%%%%%%%%%%%%%%%%%%%%%%%
%%%%%%%%%%%%%%%%%%%%%%%%%%%%%%%%%%%%%%%%%%%%%%%%%%%%%%%%%%%%%%%%%%%%%%%%%%%%%%%%
\section{Usage}

First of all, the package \textsf{childdoc} is \emph{not} a standard
\LaTeXe{} |.sty| style file! Therefore it needs to be invoked in
a non-standard way.

%%%%%%%%%%%%%%%%%%%%%%%%%%%%%%%%%%%%%%%%%%%%%%%%%%%%%%%%%%%%%%%%%%%%%%%%%%%%%%%%
\subsection{Included Files}
\label{sec:include}

%%%%%%%%%%%%%%%%%%%%%%%%%%%%%%%%%%%%%%%%
\DescribeMacro{\childdocmain}
To use the package, add the commands
\begin{center}
\begin{tabular}{l}
|% \iffalse
%
% childdoc.dtx Copyright (C) 2017-2018 Niklas Beisert
%
% This work may be distributed and/or modified under the
% conditions of the LaTeX Project Public License, either version 1.3
% of this license or (at your option) any later version.
% The latest version of this license is in
%   http://www.latex-project.org/lppl.txt
% and version 1.3 or later is part of all distributions of LaTeX
% version 2005/12/01 or later.
%
% This work has the LPPL maintenance status `maintained'.
%
% The Current Maintainer of this work is Niklas Beisert.
%
% This work consists of the files childdoc.dtx and childdoc.ins
% and the derived files childdoc.def and cdocsamp.tex with
% cdocsch1.tex, cdocsch2.tex, cdocsdrf.tex, cdocsfn1.tex, cdocsfn2.tex.
%
%<package>\ifdefined\childdocmain\endinput\fi
%<package>\ProvidesFile{childdoc.def}[2018/12/30 v2.0 child document driver]
%<samplemain>\ProvidesFile{cdocsamp.tex}[2018/12/30 v2.0 sample for childdoc]
%<*driver>
%\ProvidesFile{childdoc.drv}[2018/12/30 v2.0 childdoc reference manual file]
\PassOptionsToClass{10pt,a4paper}{article}
\documentclass{ltxdoc}

\usepackage[margin=35mm]{geometry}
\usepackage{hyperref}
\usepackage{hyperxmp}
\usepackage[usenames]{color}

\hypersetup{colorlinks=true}
\hypersetup{pdfstartview=FitH}
\hypersetup{pdfpagemode=UseNone}
\hypersetup{pdfsource={}}
\hypersetup{pdflang={en-UK}}
\hypersetup{pdfcopyright={Copyright 2017-2018 Niklas Beisert.
  This work may be distributed and/or modified under the
  conditions of the LaTeX Project Public License, either version 1.3
  of this license or (at your option) any later version.}}
\hypersetup{pdflicenseurl={http://www.latex-project.org/lppl.txt}}
\hypersetup{pdfcontactaddress={ETH Zurich, ITP, HIT K,
  Wolfgang-Pauli-Strasse 27}}
\hypersetup{pdfcontactpostcode={8093}}
\hypersetup{pdfcontactcity={Zurich}}
\hypersetup{pdfcontactcountry={Switzerland}}
\hypersetup{pdfcontactemail={nbeisert@itp.phys.ethz.ch}}
\hypersetup{pdfcontacturl={http://people.phys.ethz.ch/\xmptilde nbeisert/}}

\newcommand{\secref}[1]{\hyperref[#1]{section \ref*{#1}}}

\parskip1ex
\parindent0pt
\let\olditemize\itemize
\def\itemize{\olditemize\parskip0pt}

\begin{document}

\title{The \textsf{childdoc} Package}
\hypersetup{pdftitle={The childdoc Package}}
\author{Niklas Beisert\\[2ex]
  Institut f\"ur Theoretische Physik\\
  Eidgen\"ossische Technische Hochschule Z\"urich\\
  Wolfgang-Pauli-Strasse 27, 8093 Z\"urich, Switzerland\\[1ex]
  \href{mailto:nbeisert@itp.phys.ethz.ch}
  {\texttt{nbeisert@itp.phys.ethz.ch}}}
\hypersetup{pdfauthor={Niklas Beisert}}
\hypersetup{pdfsubject={Manual for the LaTeX2e Package childdoc}}
\date{30 December 2018, \textsf{v2.0}}
\maketitle

\begin{abstract}\noindent
\textsf{childdoc} is a \LaTeXe{} package
that enables the direct compilation
of document sections included by |\include|
to individual files.
\end{abstract}

\begingroup
\parskip0ex
\tableofcontents
\endgroup

%%%%%%%%%%%%%%%%%%%%%%%%%%%%%%%%%%%%%%%%%%%%%%%%%%%%%%%%%%%%%%%%%%%%%%%%%%%%%%%%
%%%%%%%%%%%%%%%%%%%%%%%%%%%%%%%%%%%%%%%%%%%%%%%%%%%%%%%%%%%%%%%%%%%%%%%%%%%%%%%%
\section{Introduction}

\LaTeX{} provides a mechanism to structure a large document (such as a book)
into a main file and several child files (containing the chapters)
using the |\include| command.
This mechanism is beneficial for documents
which span hundreds of pages in order to
make the source file(s) more manageable.
Moreover, compilation can be restricted to
selected child files by means of the |\includeonly| command.
The latter feature can be used to reduce the compilation time while editing
(this was significantly more useful in the earlier days of \LaTeX{})
or to generate a smaller document which is easier to navigate.
Another application of |\includeonly| is to generate
documents consisting of selected parts of the complete document.

However, there are a few drawbacks of the plain |\include| mechanism:
\begin{itemize}
\item
The child files cannot be compiled on their own,
they can only be compiled via the main file.
A naive editing environment
(such as a text editor with an option
to have the current file processed by \LaTeX)
may require one to switch to the main file before compiling;
attempting to compile the child file produces errors.
\item
The main file must be modified (each time)
to adjust the |\includeonly| command
to the present needs. This easily leaves the main file in a messy state.
\item
The generated document will always carry the filename
of the main document. This is inconvenient if
several child files are to be compiled and
to be kept for distribution.
\end{itemize}

The present package provides a simple interface
to make child files individually compilable by \LaTeX{}.
Compiling a child file then has the same effect as compiling
the main file with an |\includeonly| command
to select the appropriate child.
Moreover the generated document will carry the name of the child
rather than the main file.
This resolves all three above issues.

This feature is meant to make the editing of books,
thesis documents and lecture notes somewhat more convenient.
However, the package can also be used efficiently for
composing a series of documents (such as exercise sheets)
which are typically distributed individually.
It then assists the author in generating the individual documents
(potentially in different versions)
as well as a document containing the collected series.
Another application is in developing style files
or other kinds of included material
where compilation of the style file could redirect
to a sample or test file.

%%%%%%%%%%%%%%%%%%%%%%%%%%%%%%%%%%%%%%%%%%%%%%%%%%%%%%%%%%%%%%%%%%%%%%%%%%%%%%%%
%%%%%%%%%%%%%%%%%%%%%%%%%%%%%%%%%%%%%%%%%%%%%%%%%%%%%%%%%%%%%%%%%%%%%%%%%%%%%%%%
\section{Usage}

First of all, the package \textsf{childdoc} is \emph{not} a standard
\LaTeXe{} |.sty| style file! Therefore it needs to be invoked in
a non-standard way.

%%%%%%%%%%%%%%%%%%%%%%%%%%%%%%%%%%%%%%%%%%%%%%%%%%%%%%%%%%%%%%%%%%%%%%%%%%%%%%%%
\subsection{Included Files}
\label{sec:include}

%%%%%%%%%%%%%%%%%%%%%%%%%%%%%%%%%%%%%%%%
\DescribeMacro{\childdocmain}
To use the package, add the commands
\begin{center}
\begin{tabular}{l}
|\input{childdoc.def}|\\
|\childdocmain{}|\\
\end{tabular}
\end{center}
at the very top of the main \LaTeX{} file,
in particular \emph{before} the |\documentclass| statement!
The argument of |\childdocmain| should be left empty
(but it must be present).

%%%%%%%%%%%%%%%%%%%%%%%%%%%%%%%%%%%%%%%%
\DescribeMacro{\childdocof}
Furthermore, add the commands
\begin{center}
\begin{tabular}{l}
|\input{childdoc.def}|\\
|\childdocof{|\textit{main}|}|\\
\end{tabular}
\end{center}
at the top of every child file \textit{child}
which is included by |\include{|\textit{child}|}|
from within the main file
(or at least for those files to be compiled individually).
The argument \textit{main} must be the filename of the main file.

There are a couple of
considerations in setting up the main and child documents:

%%%%%%%%%%%%%%%%%%%%%%%%%%%%%%%%%%%%%%%%
\paragraph{Restrictions.}

Please note the following restrictions:
\begin{itemize}
\item
|\childdocmain| must be called with one argument \textit{main}
to ensure compatibility with earlier version of the package.
It must either be empty (|\childdocmain{}|)
or precisely match the filename of the main file in which it is specified.
See \secref{sec:detection} for further information.
\item
The filename \textit{main} must be specified without the |.tex| extension.
\item
The filename \textit{main} is case sensitive
(even in case-insensitive file systems)
due to internal string comparison.
\item
The argument \textit{main} should be fully expanded, it cannot be a macro.
\item
Subdirectories and special characters should be avoided in filenames.
\item
The command |\childdocmain{|\textit{main}|}| must be followed by a whitespace.
It should not be followed immediately by another command
or by a comment mark `|%|'.
This is because the \TeX{} parser reads the token immediately following
the argument of |\childdocmain| and puts it
at the beginning of every child section;
however, a white\-space is ignored.
\end{itemize}

%%%%%%%%%%%%%%%%%%%%%%%%%%%%%%%%%%%%%%%%
\paragraph{Content of Main File.}

It is advisable to place all content in the child files included by |\include|.
Any output contained in the main file will appear in all child documents
unless suppressed manually;
it cannot be suppressed automatically by the |\includeonly| directive
and thus should normally be avoided.
A method to include some content in the main file
by means of conditional processing is described in \secref{sec:conditional}.

%%%%%%%%%%%%%%%%%%%%%%%%%%%%%%%%%%%%%%%%
\paragraph{Page Numbering.}

When only a part of the document is compiled,
the appropriate numbering of pages
(as well as other status parameters)
is determined from the |.aux| files.
The latter contain information from previous passes.
However this information needs to propagate through
all intermediate child documents.
Therefore the page numbering in child documents may well
be inconsistent until the complete document is compiled at least once.

A useful (if unconventional) way to always ensure a consistent
page numbering is to restart the numbering in each child document
and denote the pages by `\textit{child}|.|\textit{page}'
where \textit{child} represents the chapter/section number of the child file.
This can be achieved by the command
|\numberwithin{page}{|\textit{child}|}|
of the \textsf{amsmath} package
where \textit{child} can be |chapter| or |section|
depending on the chosen structuring.
Alternatively, one can modify the macro |\thepage| appropriately
and reset the counter |page| at the start of each child file.

%%%%%%%%%%%%%%%%%%%%%%%%%%%%%%%%%%%%%%%%%%%%%%%%%%%%%%%%%%%%%%%%%%%%%%%%%%%%%%%%
\subsection{Conditional Processing}
\label{sec:conditional}

The package provides a mechanism to compile different versions
of a document. To customise the versions further some conditional processing
can come in handy to distinguish which version is being compiled.
The package provides two macros to describe the compilation context:

%%%%%%%%%%%%%%%%%%%%%%%%%%%%%%%%%%%%%%%%
\DescribeMacro{\ifchilddoc}
The conditional |\ifchilddoc| distinguishes between the compilation of
child documents and the main document:
%
\begin{center}
|\ifchilddoc |\textit{child-code}| |[|\||else |\textit{main-code}]| \||fi|
\end{center}

%%%%%%%%%%%%%%%%%%%%%%%%%%%%%%%%%%%%%%%%
\DescribeMacro{\childdocname}
\DescribeMacro{\childdocjob}
The macro |\childdocname| contains the filename (without extension)
of the main or child file being processed.
Note that |\childdocjob| will always contain the name of the main file.

%%%%%%%%%%%%%%%%%%%%%%%%%%%%%%%%%%%%%%%%
\paragraph{Title Page.}

Conditional processing can be used to include a title or banner page
in the main document when proper precautions are taken.
Importantly, the code in the main file should ensure that the page counter
(as well as other status parameters which are stored in the |.aux| files)
takes the same value after the conditional processing.
Otherwise the page numbers may take divergent values
depending on which part is compiled.

For example, a title page could be declared by:
%
\begin{center}
\begin{tabular}{l}
|\ifchilddoc\||else|\\
|\addtocounter{page}{-1}|\\
\textit{code for title page}\\
|\newpage|\\
|\||fi|
\end{tabular}
\end{center}
%
A banner page for the child documents can be generated by:
%
\begin{center}
\begin{tabular}{l}
|\ifchilddoc|\\
|\addtocounter{page}{-1}|\\
\textit{code for banner page}\\
|\newpage|\\
|\||fi|
\end{tabular}
\end{center}
%
Here one could write a message such as:
\begin{center}
|This is the part \childdocname{} of \childdocjob{}.|
\end{center}

%%%%%%%%%%%%%%%%%%%%%%%%%%%%%%%%%%%%%%%%%%%%%%%%%%%%%%%%%%%%%%%%%%%%%%%%%%%%%%%%
\subsection{Flags}
\label{sec:flags}

The package makes it easy to generate different versions
of the main or child documents.
To this end compilation flags can be defined
and assigned different default values.
They will be particularly useful in conjunction
with the forwarding mechanism described in \secref{sec:forward}.

For example, it may be useful to have a flag |\version|
which can be set to |draft| or |final|.
The document source will contain some conditional code
depending on the value of |\version|.
Suppose further, the flag should default to |final| for the main file
and to |draft| for child files
which is a natural assignment for editing the document.
This is achieved by placing the following code
in the preamble of the main document
(below the |\childdocmain| directive):
%
\begin{center}
\begin{tabular}{l}
|\ifchilddoc|\\
|\providecommand{\version}{draft}|\\
|\||else|\\
|\providecommand{\version}{final}|\\
|\||fi|
\end{tabular}
\end{center}
%
The definition by |\providecommand| makes sure
that previous definitions are not overwritten.
Further statements |\providecommand{\version}{...}|
can thus be added before the above code to override it.

For the main file, one might add a line
(between |\childdocmain| and the above block)
%
\begin{center}
|%\ifchilddoc\||else\providecommand{\version}{draft}\||fi|
\end{center}
%
which can be uncommented to produce a draft version.
Likewise one can add a line to the very top of a child file
(above the |\childdocof{|\textit{main}|}| directive)
%
\begin{center}
|%\providecommand{\version}{final}|
\end{center}
%
which can be uncommented to produce the final version of this child document.

%%%%%%%%%%%%%%%%%%%%%%%%%%%%%%%%%%%%%%%%%%%%%%%%%%%%%%%%%%%%%%%%%%%%%%%%%%%%%%%%
\subsection{Forwarding}
\label{sec:forward}

Different versions of the main or child documents
using compilation flags as described in \secref{sec:flags}
can be (permanently) stored in different files
for convenient compilation, viewing and distribution.
To this end, the package defines a command
to pass on compilation to a different file:

%%%%%%%%%%%%%%%%%%%%%%%%%%%%%%%%%%%%%%%%
\DescribeMacro{\childdocforward}
The command |\childdocforward| redirects processing to
another source file:
%
\begin{center}
\begin{tabular}{l}
|\input{childdoc.def}|\\
|\childdocforward[|\textit{main}|]{|\textit{dest}|}|\\
\end{tabular}
\end{center}
%
The argument \textit{dest} is the destination file
(without extension).
It should be the main file or one of the child files.
Note that further \textsf{childdoc} directives
such as |\childdocof| and |\childdocforward|
in the indicated file will be processed in this form.
The optional argument \textit{main}
passes on directly to the main file \textit{main}
while pretending to compile the child \textit{dest}.
This form behaves as if \textit{dest}
issues |\childdocof{|\textit{main}|}| right away,
and no further \textsf{childdoc} directives will be processed.

%%%%%%%%%%%%%%%%%%%%%%%%%%%%%%%%%%%%%%%%
\DescribeMacro{\...prefix}
In the alternative form |\childdocforwardprefix|,
%
\begin{center}
\begin{tabular}{l}
|\input{childdoc.def}|\\
|\childdocforwardprefix[|\textit{main}|]{|\textit{prefix}|}{|\textit{dest}|}|
\end{tabular}
\end{center}
%
the destination file is determined by a pattern
depending on the current file:
To make this work, the current file must be called
`{\textit{prefix}\hspace{0.2em}\textit{suffix}}'
with \textit{prefix} matching precisely the argument.
Processing is then passed on to the file
`{\textit{dest}\hspace{0.2em}\textit{suffix}}'.
Surely, the same effect is achieved by
directly specifying the
argument `{\textit{dest}\hspace{0.2em}\textit{suffix}}'
in the first form.
However, that requires to set up a different file
for each child. With the alternative form of the command
all these files can have exactly the same content
which simplifies setting them up and maintaining them.

For example, the following file |draft.tex|
with a compilation flag |\version| as described in \secref{sec:flags}
compiles the main document as a draft:
%
\begin{center}
\begin{tabular}{l}
|\def\version{draft}|\\
|\input{childdoc.def}|\\
|\childdocforward{|\textit{main}|}|
\end{tabular}
\end{center}
%
Likewise, the following files |final|\textit{nn}|.tex|
compile the final version of the child document
|child|\textit{nn}|.tex|:
%
\begin{center}
\begin{tabular}{l}
|\def\version{final}|\\
|\input{childdoc.def}|\\
|\childdocforwardprefix{final}{child}|
\end{tabular}
\end{center}
%

Note that when several versions of a main file and/or of each child file
are to be generated, it may be convenient to set up a |Makefile| or
shell script to automatise the process.

%%%%%%%%%%%%%%%%%%%%%%%%%%%%%%%%%%%%%%%%%%%%%%%%%%%%%%%%%%%%%%%%%%%%%%%%%%%%%%%%
\subsection{Command Line Processing}
\label{sec:commandline}

The effect of redirection files can also be achieved by invoking
the \LaTeX{} compiler with a more elaborate command line.
Most conveniently this should be done as part
of a shell script or a |Makefile|.

When using \textsf{childdoc} in the main file, the following
command lines effectively perform a redirection
(note that depending on the shell being used,
backslashes may have to be doubled: `|\|' $\to$ `|\\|'):
%
\begin{center}
|... -jobname "|\textit{target}|" |\\|"|[\textit{flags}]%
|\input{childdoc.def}\childdocforward[|\textit{main}|]{|\textit{dest}|}"|
\end{center}
%
Here \textit{target} is the name of the output file,
\textit{main} is the name of the main file
and \textit{dest} is the name of the main or child file to be processed
(all filenames without extensions).
The optional argument \textit{main} can be omitted
if \textit{main} matches \textit{dest}.
Optionally, compilation \textit{flags} can be defined via |\def| commands.
This command line makes the \TeX{} engine believe
it is compiling the file \textit{target}
whose content is specified as the latter parameter.
The provided code then forwards the processing to
\textit{main} or \textit{dest} as described in \secref{sec:forward}.

%%%%%%%%%%%%%%%%%%%%%%%%%%%%%%%%%%%%%%%%%%%%%%%%%%%%%%%%%%%%%%%%%%%%%%%%%%%%%%%%
\subsection{Include by Input}
\label{sec:input}

Including child documents by |\include| has some restrictions by design.
Most notably, the content of a child document always occupies
its own set of pages; pages cannot be shared between child documents.
Usually, this behaviour makes perfect sense
because each child document contain an essential part of the document.
However, in some situations it may be desirable to compose
a document from a collection of parts
without having mandatory page breaks between then.
For this case, the package
provides a mechanism to include parts
by |\input| which can also be processed individually.
However, by construction this mechanism
requires manual handling of the content to be output.

%%%%%%%%%%%%%%%%%%%%%%%%%%%%%%%%%%%%%%%%
\DescribeMacro{\ifchilddocmanual}
The main file should be prepared as usual, see \secref{sec:include}.
However, the document body must make a distinction
between processing of an individual part and of the main document, e.g.:
%
\begin{center}
\begin{tabular}{l}
|\ifchilddocmanual|\\
|\input{\childdocname}|\\
|\||else|\\
\textit{document body with }|\input{|\textit{part}|}|\\
|\||fi|
\end{tabular}
\end{center}
%
The conditional |\ifchilddocmanual| is true whenever
a part to be included by |\input| is being compiled,
and the name of the part is stored in |\childdocname|.

%%%%%%%%%%%%%%%%%%%%%%%%%%%%%%%%%%%%%%%%
\DescribeMacro{\childdocby}
Each part to be included by |\input| should start with:
%
\begin{center}
\begin{tabular}{l}
|\input{childdoc.def}|\\
|\childdocby{|\textit{main}|}|\\
\end{tabular}
\end{center}
%
The directive |\childdocby| is similar to |\childdocof|
described in \secref{sec:include},
but the subsequent selection of content must be done manually.
To that end, both |\ifchilddoc| and |\ifchilddocmanual|
will be true upon processing of a part,
and the name of the part is stored in |\childdocname|.
Note that |\jobname| will be set to the filename of the current part
so that each part receives an individual |.aux| file
that does not interfere with the |.aux| file(s) of the main document.
This behaviour can be altered by the alternative form
|\childdocby[*]{|\textit{main}|}| (with a non-empty optional argument)
which uses the |.aux| file of the main document
by setting |\jobname| to \textit{main}.

%%%%%%%%%%%%%%%%%%%%%%%%%%%%%%%%%%%%%%%%%%%%%%%%%%%%%%%%%%%%%%%%%%%%%%%%%%%%%%%%
\subsection{Driver Development}
\label{sec:driver}

The \textsf{childdoc} mechanism can also be use for the development
of definition files such as \LaTeX{} styles or classes.
This case differs from the above setup with multiple parts
included by |\include| in that no |\includeonly| should be invoked.
This can be achieved by starting the include file
(before |\ProvidesPackage|) with:
%
\begin{center}
\begin{tabular}{l}
|\input{childdoc.def}|\\
|\childdocforward{|\textit{main}|}|\\
\end{tabular}
\end{center}
%
or alternatively with:
%
\begin{center}
\begin{tabular}{l}
|\input{childdoc.def}|\\
|\childdocby{|\textit{main}|}|\\
\end{tabular}
\end{center}
%
Both forms have slightly different effects as described above.
The main file is prepared as usual, see \secref{sec:include}.

%%%%%%%%%%%%%%%%%%%%%%%%%%%%%%%%%%%%%%%%%%%%%%%%%%%%%%%%%%%%%%%%%%%%%%%%%%%%%%%%
\subsection{Legacy Detection}
\label{sec:detection}

The directive |\childdocmain| in the main file can detect
whether the complete document or merely a child is to be compiled
even without using the directive |\childdocof|.
This method is deprecated because it is less robust
and there is no compelling reason to use it;
it is merely provided for backward compatibility
and it may be removed in future versions.

If the detection mechanism is to be used,
it is mandatory to correctly specify
the filename of the main file as the argument of |\childdocmain|:
%
\begin{center}
\begin{tabular}{l}
|\input{childdoc.def}|\\
|\childdocmain{|\textit{main}|}|\\
\end{tabular}
\end{center}
%
If |\jobname| does not match the argument \textit{main} of |\childdocmain|,
it is assumed that |\jobname| points to the child file to be compiled.
When using |\childdocmain| with the main file specified as argument,
it suffices to start a child file
with just |\input{|\textit{main}|}|
without loading of the package and using |\childdocof|.
If instead all processing is done
with the appropriate \textsf{childdoc} directives,
the argument of \textit{main} of |\childdocmain| can be empty.

An alternative version of the command line processing described
in \secref{sec:commandline} using the detection mechanism reads:
%
\begin{center}
|... -jobname "|\textit{target}|" "|[\textit{flags}]%
[|\def\jobname{|\textit{dest}|}|]|\input{|\textit{main}|}"|
\end{center}

%%%%%%%%%%%%%%%%%%%%%%%%%%%%%%%%%%%%%%%%%%%%%%%%%%%%%%%%%%%%%%%%%%%%%%%%%%%%%%%%
\subsection{Manual Code}
\label{sec:manual}

In case one cannot be certain whether the definitions file |childdoc.def|
is installed on the target \TeX{} distribution
and one prefers not to ship it,
it is conceivable to paste a few relevant commands into the sources.

To that end, drop all statements |\input{childdoc.def}|
and perform the replacements as outlined below.
Instead of |\childdocmain{|\textit{main}|}| add the following code
to the top of the main file:
%
\begin{center}
\begin{tabular}{l}
|\||ifdefined\childdocname\endinput\||fi\newif\ifchilddoc|\\
|\edef\childdocname{\scantokens\expandafter{\jobname\noexpand}}|\\
|\def\childdocmain{|\textit{main}|}\||ifx\childdocmain\childdocname\||else|\\
|\childdoctrue\includeonly{\childdocname}\let\jobname\childdocmain\||fi|\\
\end{tabular}
\end{center}
%
Instead of |\childdocof{|\textit{main}|}| just include the main file
at the top of each child file:
%
\begin{center}
|\input{|\textit{main}|}|
\end{center}
%
A simple redirection |\childdocforward{|\textit{dest}|}| is achieved by:
%
\begin{center}
|\def\jobname{|\textit{dest}|}\input{\jobname}|
\end{center}
%
The redirection with prefix
|\childdocforwardprefix[|\textit{prefix}|]{|\textit{dest}|}|
is accomplished by:
%
\begin{center}
\begin{tabular}{l}
|{\edef\jobname{\scantokens\expandafter{\jobname\noexpand}}|\\
|\def\redirectjob |\textit{prefix}|#1~~~{\gdef\jobname{|\textit{dest}|#1}}|\\
|\expandafter\redirectjob\jobname~~~}\input{\jobname}|
\end{tabular}
\end{center}

In an alternative approach,
child documents can be compiled by a specific command line
without additional code or specific definitions:
%
\begin{center}
|... -jobname "|\textit{target}|" "|[\textit{flags}]%
|\includeonly{|\textit{dest}|}\input{|\textit{main}|}"|
\end{center}
%

%%%%%%%%%%%%%%%%%%%%%%%%%%%%%%%%%%%%%%%%%%%%%%%%%%%%%%%%%%%%%%%%%%%%%%%%%%%%%%%%
%%%%%%%%%%%%%%%%%%%%%%%%%%%%%%%%%%%%%%%%%%%%%%%%%%%%%%%%%%%%%%%%%%%%%%%%%%%%%%%%
\section{Information}

%%%%%%%%%%%%%%%%%%%%%%%%%%%%%%%%%%%%%%%%%%%%%%%%%%%%%%%%%%%%%%%%%%%%%%%%%%%%%%%%
\subsection{Copyright}

Copyright \copyright{} 2017--2018 Niklas Beisert

This work may be distributed and/or modified under the
conditions of the \LaTeX{} Project Public License, either version 1.3
of this license or (at your option) any later version.
The latest version of this license is in
  \url{http://www.latex-project.org/lppl.txt}
and version 1.3 or later is part of all distributions of \LaTeX{}
version 2005/12/01 or later.

This work has the LPPL maintenance status `maintained'.

The Current Maintainer of this work is Niklas Beisert.

This work consists of the files |README.txt|, |childdoc.ins| and |childdoc.dtx|
as well as the derived files |childdoc.def|, |cdocsamp.tex|
with |cdocsch1.tex|, |cdocsch2.tex|, |cdocspt3.tex|, |cdocspt4.tex|,
|cdocsdrf.tex|, |cdocsfn1.tex|, |cdocsfn2.tex|
as well as |childdoc.pdf|.

%%%%%%%%%%%%%%%%%%%%%%%%%%%%%%%%%%%%%%%%%%%%%%%%%%%%%%%%%%%%%%%%%%%%%%%%%%%%%%%%
\subsection{Files and Installation}

The package consists of the files:
%
\begin{center}
\begin{tabular}{ll}
    |README.txt|   & readme file \\
    |childdoc.ins| & installation file \\
    |childdoc.dtx| & source file \\
    |childdoc.def| & definition file \\
    |cdocsamp.tex| & sample main file \\
    |cdocsch1.tex| & sample include file \\
    |cdocsch2.tex| & sample include file \\
    |cdocspt3.tex| & sample part file \\
    |cdocspt4.tex| & sample part file \\
    |cdocsdrf.tex| & sample redirection file \\
    |cdocsfn1.tex| & sample redirection file \\
    |cdocsfn2.tex| & sample redirection file \\
    |childdoc.pdf| & manual
\end{tabular}
\end{center}
%
The distribution consists of the files
|README.txt|, |childdoc.ins| and |childdoc.dtx|.
%
\begin{itemize}
\item
Run (pdf)\LaTeX{} on |childdoc.dtx|
to compile the manual |childdoc.pdf| (this file).
\item
Run \LaTeX{} on |childdoc.ins| to create the definitions file |childdoc.def|
and the sample |cdocsamp.tex| with include files
|cdocsch1.tex|, |cdocsch2.tex|, |cdocspt3.tex|, |cdocspt4.tex|,
|cdocsdrf.tex|, |cdocsfn1.tex|, |cdocsfn2.tex|.
Then copy the file |childdoc.def| to an appropriate directory of your \LaTeX{}
distribution, e.g.\ \textit{texmf-root}|/tex/latex/childdoc|.
\end{itemize}

%%%%%%%%%%%%%%%%%%%%%%%%%%%%%%%%%%%%%%%%%%%%%%%%%%%%%%%%%%%%%%%%%%%%%%%%%%%%%%%%
\subsection{Related CTAN Packages}

There are several other packages which offer a similar functionality:
%
\begin{itemize}
\item
The packages
\href{http://ctan.org/pkg/docmute}{\textsf{docmute}},
\href{http://ctan.org/pkg/includex}{\textsf{includex}} and
\href{http://ctan.org/pkg/standalone}{\textsf{standalone}}
provide commands to include only the document body of
a child file thus allowing both files to be compiled individually.
\item
The packages \href{http://ctan.org/pkg/subdocs}{\textsf{subdocs}}
and \href{http://ctan.org/pkg/subfiles}{\textsf{subfiles}}
provide structures in which the main and child documents can be
encapsulated and allowing them to be compiled individually.
The inclusion mechanism is different from the conventional |\include|.
\item
The package \href{http://ctan.org/pkg/combine}{\textsf{combine}}
is an elaborate solution to combine several documents into one.
\end{itemize}
%
See also the CTAN topic \href{http://ctan.org/topic/subdocs}{\textsf{subdocs}}
for further related packages.
The present package differs from the above solutions in that
a document structure constructed with the conventional |\include| mechanism
just needs two extra commands at the top of every file
such that all constituent files can be compiled individually.

%%%%%%%%%%%%%%%%%%%%%%%%%%%%%%%%%%%%%%%%%%%%%%%%%%%%%%%%%%%%%%%%%%%%%%%%%%%%%%%%
%\subsection{Feature Suggestions}
%
%The following is a list of features which may be useful for future
%versions of this package:
%%
%\begin{itemize}
%\item
%\ldots
%\end{itemize}

%%%%%%%%%%%%%%%%%%%%%%%%%%%%%%%%%%%%%%%%%%%%%%%%%%%%%%%%%%%%%%%%%%%%%%%%%%%%%%%%
\subsection{Revision History}

%%%%%%%%%%%%%%%%%%%%%%%%%%%%%%%%%%%%%%%%
\paragraph{v2.0:} 2018/12/30

\begin{itemize}
\item
immediate forward processing
\item
added |\childdocby| mechanism
\item
manual restructured
\end{itemize}

%%%%%%%%%%%%%%%%%%%%%%%%%%%%%%%%%%%%%%%%
\paragraph{v1.6:} 2018/01/17

\begin{itemize}
\item
application for development of include files
\item
corrections to manual
\end{itemize}

%%%%%%%%%%%%%%%%%%%%%%%%%%%%%%%%%%%%%%%%
\paragraph{v1.5:} 2017/05/21

\begin{itemize}
\item
more complete structuring introduced
\item
|\childdocof| introduced
\item
|\childdoc| renamed to |\childdocmain|
\item
|\childredirect| renamed to |\childdocforward| and |\childdocforwardprefix|
and functionality expanded
\end{itemize}

%%%%%%%%%%%%%%%%%%%%%%%%%%%%%%%%%%%%%%%%
\paragraph{v1.0:} 2017/04/27

\begin{itemize}
\item
manual and install package
\item
first version published on CTAN
\end{itemize}

%%%%%%%%%%%%%%%%%%%%%%%%%%%%%%%%%%%%%%%%
\paragraph{v0.6:} 2017/04/26

\begin{itemize}
\item
redirection mechanism added
\end{itemize}

%%%%%%%%%%%%%%%%%%%%%%%%%%%%%%%%%%%%%%%%
\paragraph{v0.5:} 2017/04/26

\begin{itemize}
\item
functionality in definition file
\end{itemize}


%%%%%%%%%%%%%%%%%%%%%%%%%%%%%%%%%%%%%%%%%%%%%%%%%%%%%%%%%%%%%%%%%%%%%%%%%%%%%%%%
%%%%%%%%%%%%%%%%%%%%%%%%%%%%%%%%%%%%%%%%%%%%%%%%%%%%%%%%%%%%%%%%%%%%%%%%%%%%%%%%
%%%%%%%%%%%%%%%%%%%%%%%%%%%%%%%%%%%%%%%%%%%%%%%%%%%%%%%%%%%%%%%%%%%%%%%%%%%%%%%%
\appendix

\settowidth\MacroIndent{\rmfamily\scriptsize 000\ }

 \DocInput{childdoc.dtx}

\end{document}
%</driver>
% \fi
%
% %%%%%%%%%%%%%%%%%%%%%%%%%%%%%%%%%%%%%%%%%%%%%%%%%%%%%%%%%%%%%%%%%%%%%%%%%%%%%%
% %%%%%%%%%%%%%%%%%%%%%%%%%%%%%%%%%%%%%%%%%%%%%%%%%%%%%%%%%%%%%%%%%%%%%%%%%%%%%%
% \section{Sample}
%\iffalse
%<*samplemain>
%\fi
%
% The following presents a sample document
% with two chapters, two parts, a title page,
% a compile flag as well as three forwarding files to set the flag.
% It consists of eight |.tex| files:
% \begin{center}
% \begin{tabular}{ll}
% |cdocsamp.tex|&main file\\
% |cdocsch1.tex|&include file for chapter 1\\
% |cdocsch2.tex|&include file for chapter 2\\
% |cdocspt3.tex|&include file for part 3\\
% |cdocspt4.tex|&include file for part 4\\
% |cdocsdrf.tex|&forwarding file for main file in draft mode\\
% |cdocsfi1.tex|&forwarding file for final version of chapter 1\\
% |cdocsfi2.tex|&forwarding file for final version of chapter 2\\
% \end{tabular}
% \end{center}
% Each of the eight files can be compiled directly by the \LaTeX{} compiler.
%
% %%%%%%%%%%%%%%%%%%%%%%%%%%%%%%%%%%%%%%
% \paragraph{Main File.}
%
% The main file is called |cdocsamp.tex|.
%
% Load the \textsf{childdoc} definitions and
% declare the filename for the main document:
%    \begin{macrocode}
\input{childdoc.def}
\childdocmain{}
%    \end{macrocode}

% Optional override for |\version| flag:
%    \begin{macrocode}
%%\ifchilddoc\else\providecommand{\version}{draft}\fi
%    \end{macrocode}

% Define the default values for the |\version| flag
% (|final| for the main file and |draft| for childs):
%    \begin{macrocode}
\ifchilddoc
\providecommand{\version}{draft}
\else
\providecommand{\version}{final}
\fi
%    \end{macrocode}

% Load the standard document class:
%    \begin{macrocode}
\documentclass[12pt]{article}
%    \end{macrocode}

% Start the document body:
%    \begin{macrocode}
\begin{document}
%    \end{macrocode}

% Declare a title page.
% Print title, part of document being processed and version flag:
%    \begin{macrocode}
\addtocounter{page}{-1}
\begin{center}
{\LARGE\bfseries{}childdoc example\par}
\vspace{1cm}
\ifchilddoc
\ifchilddocmanual part\else chapter\fi:
`\childdocname' of `\childdocjob'\par
\else
main document: `\childdocjob'\par
\fi
version: \version\par
\end{center}
\newpage
%    \end{macrocode}

% Manually include selected file,
% otherwise process as usual:
%    \begin{macrocode}
\ifchilddocmanual
\section*{part `\childdocname'}
\input{\childdocname}
\else
%    \end{macrocode}

% Include the two chapters:
%    \begin{macrocode}
\include{cdocsch1}
\include{cdocsch2}
%    \end{macrocode}

% Include the two parts unless only chapters should be displayed:
%    \begin{macrocode}
\ifchilddoc\else
\section{part three}
\input{cdocspt3}
\section{part four}
\input{cdocspt4}
\fi
%    \end{macrocode}

% Process as usual until here:
%    \begin{macrocode}
\fi
%    \end{macrocode}

% End of document body:
%    \begin{macrocode}
\end{document}
%    \end{macrocode}
%\iffalse
%</samplemain>
%\fi
%
% %%%%%%%%%%%%%%%%%%%%%%%%%%%%%%%%%%%%%%
% \paragraph{Chapter Include Files.}
%
% The include files are called |cdocsch1.tex| and |cdocsch2.tex|.
%
%\iffalse
%<*samplechap1|samplechap2>
%\fi

% Optional override for |\version| flag:
%    \begin{macrocode}
%%\providecommand{\version}{final}
%    \end{macrocode}

% Include the main document:
%    \begin{macrocode}
\input{childdoc.def}
\childdocof{cdocsamp}
%    \end{macrocode}

%\iffalse
%</samplechap1|samplechap2>
%\fi
%
%\iffalse
%<*samplechap1>
%\fi
% Some text for chapter 1:
%    \begin{macrocode}
\section{one}
some text in chapter one
%    \end{macrocode}

%\iffalse
%</samplechap1>
%\fi
% Some text for chapter 2:
%\iffalse
%<*samplechap2>
%\fi
%    \begin{macrocode}
\section{two}
more text in chapter two
%    \end{macrocode}

%\iffalse
%</samplechap2>
%\fi
%
% %%%%%%%%%%%%%%%%%%%%%%%%%%%%%%%%%%%%%%
% \paragraph{Part Include Files.}
%
% The include files are called |cdocspt3.tex| and |cdocspt4.tex|.
%
%\iffalse
%<*samplepart3|samplepart4>
%\fi

% Optional override for |\version| flag:
%    \begin{macrocode}
%%\providecommand{\version}{final}
%    \end{macrocode}

% Include the main document:
%    \begin{macrocode}
\input{childdoc.def}
\childdocby{cdocsamp}
%    \end{macrocode}

%\iffalse
%</samplepart3|samplepart4>
%\fi
%
%\iffalse
%<*samplepart3>
%\fi
% Some text for part 3:
%    \begin{macrocode}
some text in part three
%    \end{macrocode}

%\iffalse
%</samplepart3>
%\fi
% Some text for part 4:
%\iffalse
%<*samplepart4>
%\fi
%    \begin{macrocode}
more text in part four
%    \end{macrocode}

%\iffalse
%</samplepart4>
%\fi
%
% %%%%%%%%%%%%%%%%%%%%%%%%%%%%%%%%%%%%%%
% \paragraph{Forwarding for a Complete Draft.}
%
% The following forwarding file |cdocsdrf.tex|
% compiles the main document in draft mode:
%\iffalse
%<*sampledraft>
%\fi
%    \begin{macrocode}
\def\version{draft}
\input{childdoc.def}
\childdocforward{cdocsamp}
%    \end{macrocode}

%\iffalse
%</sampledraft>
%\fi
%
% %%%%%%%%%%%%%%%%%%%%%%%%%%%%%%%%%%%%%%
% \paragraph{Forwarding for Final Version of the Chapters.}
%
% The following forwarding files |cdocsfn1.tex| and |cdocsfn2.tex|
% (with identical content)
% compile the final versions of the child documents
% |cdocsch1.tex| and |cdocsch2.tex|, respectively:
%\iffalse
%<*samplefinal>
%\fi
%    \begin{macrocode}
\def\version{final}
\input{childdoc.def}
\childdocforwardprefix[cdocsamp]{cdocsfn}{cdocsch}
%    \end{macrocode}

%\iffalse
%</samplefinal>
%\fi
%
% %%%%%%%%%%%%%%%%%%%%%%%%%%%%%%%%%%%%%%
% \paragraph{Command Line Processing.}
%
% The following three command lines generate the output files
% |cdocscld|, |cdocscl1| and |cdocscl2|
% which should be identical to
% |cdocsdrf|, |cdocsch1| and |cdocsfn2|, respectively:
% \begin{center}
% \begin{tabular}{l}
% |latex -jobname cdocscld \|\\
% |  "\def\version{draft}\input{childdoc.def}\childdocforward{cdocsamp}"|\\
% |latex -jobname cdocscl1 \|\\
% |  "\input{childdoc.def}\childdocforward[cdocsamp]{cdocsch1}"|\\
% |latex -jobname cdocscl2 \|\\
% |  "\def\version{final}\input{childdoc.def}\childdocforward{cdocsch2}"|
% \end{tabular}
% \end{center}
% Note that the trailing backslash on each first line
% merely continues the input to the second line
% (for convenient cut ant paste).
% Furthermore, the command |latex| can be replaced by any
% of its alternative versions such as |pdflatex|.
%
% %%%%%%%%%%%%%%%%%%%%%%%%%%%%%%%%%%%%%%%%%%%%%%%%%%%%%%%%%%%%%%%%%%%%%%%%%%%%%%
% %%%%%%%%%%%%%%%%%%%%%%%%%%%%%%%%%%%%%%%%%%%%%%%%%%%%%%%%%%%%%%%%%%%%%%%%%%%%%%
% \section{Implementation}
%\iffalse
%<*package>
%\fi
%
% This section describes the definitions file |childdoc.def|.

% The definitions cannot be loaded using |\usepackage| or |\RequirePackage|
% which has a mechanism to prevent loading a style file more than once.
% When loading the definitions by means of |\input|
% multiple instances have to be prevented manually:
%\iffalse
%This code needs to be before the `\ProvidesFile' directive
%which is defined at the beginning of this file.
%Therefore it is also placed there and commented out here.
%</package>
%<*discard>
%\fi
%    \begin{macrocode}
\ifdefined\childdocmain\endinput\fi
%    \end{macrocode}
%\iffalse
%</discard>
%<*package>
%\fi
%
% \macro{\ifchilddoc}
% \macro{\ifchilddocmanual}
% The conditional |\ifchilddoc| tells whether a
% child (true) or main (false) document is being compiled.
% The conditional |\ifchilddocmanual| tells whether
% the |\includeonly| mechanism is used (false) or
% the selection of child files must be performed manually (true).
% The definitions initialise to false:
%    \begin{macrocode}
\newif\ifchilddoc
\newif\ifchilddocmanual
%    \end{macrocode}

% \macro{\childdocname}
% \macro{\childdocjob}
% The macro |\childdocname| stores the name of the main document
% to be compiled. The macro |\childdocjob| stores the name of
% the document on which the \LaTeX{} compiler was originally invoked.
% The content of |\jobname| cannot be compared
% to filenames specified in the source due to different catcodes.
% The following code rescans |\jobname|, stores the result
% in |\childdocname| and saves a copy in |\childdocjob|:
%    \begin{macrocode}
\edef\childdocname{\scantokens\expandafter{\jobname\noexpand}}
\let\childdocjob\childdocname
%    \end{macrocode}

% \macro{\childdocdisable}
% The macro |\childdocdisable| prevents the main file
% from being processed more than once.
% At this stage, the main document command |\childdocmain|
% is assumed to be called once again where it should do nothing.
% Any subsequent call to it should prevent
% a secondary processing of the main document
% It overwrites the forwarding commands
% |\childdocof| and |\childdocforward|
% with empty macros to prevent further inclusions of the main document:
%    \begin{macrocode}
\newcommand{\childdocdisable}
{
  \renewcommand{\childdocmain}[1]{\renewcommand{\childdocmain}[1]{\endinput}}
  \renewcommand{\childdocof}[1]{}
  \renewcommand{\childdocby}[2][]{}
  \renewcommand{\childdocforward}[2][]{}
  \renewcommand{\childdocdisable}{}
}
%    \end{macrocode}

% \macro{\childdocmain}
% The macro |\childdocmain| is to be called at the top of the main file
% with nothing or the main filename (without extension) as argument.
% First, it breaks loops.
% If the argument is not empty and does not match |\childdocname|
% (which is set by the first inclusion of |childdoc.def|),
% |\ifchilddoc| is set to true, |\includeonly| is applied to the child file
% and |\jobname| is set to the main file
% (for proper handling of |.aux| files):
%    \begin{macrocode}
\newcommand{\childdocmain}[1]
{
  \childdocdisable\childdocmain{}
  \if?#1?\else
    \begingroup
      \def\childdoctmp{#1}
      \ifx\childdoctmp\childdocname
        \def\childdoctmp{}
      \else
        \def\childdoctmp
        {
          \childdoctrue
          \includeonly{\childdocname}
          \def\childdocjob{#1}
          \def\jobname{#1}
        }
      \fi
      \expandafter
    \endgroup
    \childdoctmp
  \fi
}
%    \end{macrocode}

% \macro{\childdocof}
% The command |\childdocof| redirects
% compilation to the main file |#1|.
%    \begin{macrocode}
\newcommand{\childdocof}[1]
{
  \childdocdisable
  \childdoctrue
  \includeonly{\childdocname}
  \def\jobname{#1}
  \def\childdocjob{#1}
  \input{#1}
}
%    \end{macrocode}

% \macro{\childdocby}
% The command |\childdocby| ....
%    \begin{macrocode}
\newcommand{\childdocby}[2][]
{
  \childdocdisable
  \childdoctrue
  \childdocmanualtrue
  \if?#1?\else
    \def\jobname{#2}
  \fi
  \def\childdocjob{#2}
  \input{#2}
  \endinput
}
%    \end{macrocode}

% \macro{\childdocforward}
% The command |\childdocforward| redirects
% compilation to the main file or
% (if the optional argument is given) a child file.
% Parameters are set as if the main file
% or a child file starting with |\childdocof| was compiled.
% Then compilation is handed over to the main file:
%    \begin{macrocode}
\newcommand{\childdocforward}[2][]
{
  \begingroup
    \if?#1?
      \def\childdoctmp
      {
        \def\childdocname{#2}
        \def\childdocjob{#2}
        \def\jobname{#2}
        \input{#2}
        \endinput
      }
    \else
      \def\childdoctmp
      {
        \childdocdisable
        \def\childdocname{#2}
        \childdoctrue
        \includeonly{#2}
        \def\childdocjob{#1}
        \def\jobname{#1}
        \input{#1}
        \endinput
      }
    \fi
    \expandafter
  \endgroup
  \childdoctmp
}
%    \end{macrocode}

% \macro{\childdocforwardprefix}
% The command |\childdocforwardprefix| redirects
% compilation to the main or a child file by means of a pattern.
% The prefix |#1| in the current filename is replaced by |#2|
% and the suffix of the current filename is kept
% (it is assumed that the filename does not contain the substring `|~~~|'
% which is used as a delimiter).
% Compilation is handed over to the new file by |\childdocforward|:
%    \begin{macrocode}
\newcommand{\childdocforwardprefix}[3][]
{
  \begingroup
    \def\childdocextract #2##1~~~{\def\childdoctmp{\childdocforward[#1]{#3##1}}}
    \expandafter\childdocextract\childdocname~~~
    \expandafter
  \endgroup
  \childdoctmp
}
%    \end{macrocode}

% \macro{\childdoc}
% The deprecated macro |\childdoc| is a legacy version of |\childdocmain|:
%    \begin{macrocode}
\newcommand{\childdoc}{\childdocmain}
%    \end{macrocode}

% \macro{\childdocredirect}
% The deprecated macro |\childdocredirect| is a legacy version
% of |\childdocforward| and |\childdocforwardprefix|:
%    \begin{macrocode}
\newcommand{\childdocredirect}[2][]
{
  \begingroup
    \if?#1?
      \def\childdoctmp{\childdocforward{#2}}
    \else
      \def\childdoctmp{\childdocforwardprefix{#1}{#2}}
    \fi
    \expandafter
  \endgroup
  \childdoctmp
}
%    \end{macrocode}

%\iffalse
%</package>
%\fi
%
\endinput
|\\
|\childdocmain{}|\\
\end{tabular}
\end{center}
at the very top of the main \LaTeX{} file,
in particular \emph{before} the |\documentclass| statement!
The argument of |\childdocmain| should be left empty
(but it must be present).

%%%%%%%%%%%%%%%%%%%%%%%%%%%%%%%%%%%%%%%%
\DescribeMacro{\childdocof}
Furthermore, add the commands
\begin{center}
\begin{tabular}{l}
|% \iffalse
%
% childdoc.dtx Copyright (C) 2017-2018 Niklas Beisert
%
% This work may be distributed and/or modified under the
% conditions of the LaTeX Project Public License, either version 1.3
% of this license or (at your option) any later version.
% The latest version of this license is in
%   http://www.latex-project.org/lppl.txt
% and version 1.3 or later is part of all distributions of LaTeX
% version 2005/12/01 or later.
%
% This work has the LPPL maintenance status `maintained'.
%
% The Current Maintainer of this work is Niklas Beisert.
%
% This work consists of the files childdoc.dtx and childdoc.ins
% and the derived files childdoc.def and cdocsamp.tex with
% cdocsch1.tex, cdocsch2.tex, cdocsdrf.tex, cdocsfn1.tex, cdocsfn2.tex.
%
%<package>\ifdefined\childdocmain\endinput\fi
%<package>\ProvidesFile{childdoc.def}[2018/12/30 v2.0 child document driver]
%<samplemain>\ProvidesFile{cdocsamp.tex}[2018/12/30 v2.0 sample for childdoc]
%<*driver>
%\ProvidesFile{childdoc.drv}[2018/12/30 v2.0 childdoc reference manual file]
\PassOptionsToClass{10pt,a4paper}{article}
\documentclass{ltxdoc}

\usepackage[margin=35mm]{geometry}
\usepackage{hyperref}
\usepackage{hyperxmp}
\usepackage[usenames]{color}

\hypersetup{colorlinks=true}
\hypersetup{pdfstartview=FitH}
\hypersetup{pdfpagemode=UseNone}
\hypersetup{pdfsource={}}
\hypersetup{pdflang={en-UK}}
\hypersetup{pdfcopyright={Copyright 2017-2018 Niklas Beisert.
  This work may be distributed and/or modified under the
  conditions of the LaTeX Project Public License, either version 1.3
  of this license or (at your option) any later version.}}
\hypersetup{pdflicenseurl={http://www.latex-project.org/lppl.txt}}
\hypersetup{pdfcontactaddress={ETH Zurich, ITP, HIT K,
  Wolfgang-Pauli-Strasse 27}}
\hypersetup{pdfcontactpostcode={8093}}
\hypersetup{pdfcontactcity={Zurich}}
\hypersetup{pdfcontactcountry={Switzerland}}
\hypersetup{pdfcontactemail={nbeisert@itp.phys.ethz.ch}}
\hypersetup{pdfcontacturl={http://people.phys.ethz.ch/\xmptilde nbeisert/}}

\newcommand{\secref}[1]{\hyperref[#1]{section \ref*{#1}}}

\parskip1ex
\parindent0pt
\let\olditemize\itemize
\def\itemize{\olditemize\parskip0pt}

\begin{document}

\title{The \textsf{childdoc} Package}
\hypersetup{pdftitle={The childdoc Package}}
\author{Niklas Beisert\\[2ex]
  Institut f\"ur Theoretische Physik\\
  Eidgen\"ossische Technische Hochschule Z\"urich\\
  Wolfgang-Pauli-Strasse 27, 8093 Z\"urich, Switzerland\\[1ex]
  \href{mailto:nbeisert@itp.phys.ethz.ch}
  {\texttt{nbeisert@itp.phys.ethz.ch}}}
\hypersetup{pdfauthor={Niklas Beisert}}
\hypersetup{pdfsubject={Manual for the LaTeX2e Package childdoc}}
\date{30 December 2018, \textsf{v2.0}}
\maketitle

\begin{abstract}\noindent
\textsf{childdoc} is a \LaTeXe{} package
that enables the direct compilation
of document sections included by |\include|
to individual files.
\end{abstract}

\begingroup
\parskip0ex
\tableofcontents
\endgroup

%%%%%%%%%%%%%%%%%%%%%%%%%%%%%%%%%%%%%%%%%%%%%%%%%%%%%%%%%%%%%%%%%%%%%%%%%%%%%%%%
%%%%%%%%%%%%%%%%%%%%%%%%%%%%%%%%%%%%%%%%%%%%%%%%%%%%%%%%%%%%%%%%%%%%%%%%%%%%%%%%
\section{Introduction}

\LaTeX{} provides a mechanism to structure a large document (such as a book)
into a main file and several child files (containing the chapters)
using the |\include| command.
This mechanism is beneficial for documents
which span hundreds of pages in order to
make the source file(s) more manageable.
Moreover, compilation can be restricted to
selected child files by means of the |\includeonly| command.
The latter feature can be used to reduce the compilation time while editing
(this was significantly more useful in the earlier days of \LaTeX{})
or to generate a smaller document which is easier to navigate.
Another application of |\includeonly| is to generate
documents consisting of selected parts of the complete document.

However, there are a few drawbacks of the plain |\include| mechanism:
\begin{itemize}
\item
The child files cannot be compiled on their own,
they can only be compiled via the main file.
A naive editing environment
(such as a text editor with an option
to have the current file processed by \LaTeX)
may require one to switch to the main file before compiling;
attempting to compile the child file produces errors.
\item
The main file must be modified (each time)
to adjust the |\includeonly| command
to the present needs. This easily leaves the main file in a messy state.
\item
The generated document will always carry the filename
of the main document. This is inconvenient if
several child files are to be compiled and
to be kept for distribution.
\end{itemize}

The present package provides a simple interface
to make child files individually compilable by \LaTeX{}.
Compiling a child file then has the same effect as compiling
the main file with an |\includeonly| command
to select the appropriate child.
Moreover the generated document will carry the name of the child
rather than the main file.
This resolves all three above issues.

This feature is meant to make the editing of books,
thesis documents and lecture notes somewhat more convenient.
However, the package can also be used efficiently for
composing a series of documents (such as exercise sheets)
which are typically distributed individually.
It then assists the author in generating the individual documents
(potentially in different versions)
as well as a document containing the collected series.
Another application is in developing style files
or other kinds of included material
where compilation of the style file could redirect
to a sample or test file.

%%%%%%%%%%%%%%%%%%%%%%%%%%%%%%%%%%%%%%%%%%%%%%%%%%%%%%%%%%%%%%%%%%%%%%%%%%%%%%%%
%%%%%%%%%%%%%%%%%%%%%%%%%%%%%%%%%%%%%%%%%%%%%%%%%%%%%%%%%%%%%%%%%%%%%%%%%%%%%%%%
\section{Usage}

First of all, the package \textsf{childdoc} is \emph{not} a standard
\LaTeXe{} |.sty| style file! Therefore it needs to be invoked in
a non-standard way.

%%%%%%%%%%%%%%%%%%%%%%%%%%%%%%%%%%%%%%%%%%%%%%%%%%%%%%%%%%%%%%%%%%%%%%%%%%%%%%%%
\subsection{Included Files}
\label{sec:include}

%%%%%%%%%%%%%%%%%%%%%%%%%%%%%%%%%%%%%%%%
\DescribeMacro{\childdocmain}
To use the package, add the commands
\begin{center}
\begin{tabular}{l}
|\input{childdoc.def}|\\
|\childdocmain{}|\\
\end{tabular}
\end{center}
at the very top of the main \LaTeX{} file,
in particular \emph{before} the |\documentclass| statement!
The argument of |\childdocmain| should be left empty
(but it must be present).

%%%%%%%%%%%%%%%%%%%%%%%%%%%%%%%%%%%%%%%%
\DescribeMacro{\childdocof}
Furthermore, add the commands
\begin{center}
\begin{tabular}{l}
|\input{childdoc.def}|\\
|\childdocof{|\textit{main}|}|\\
\end{tabular}
\end{center}
at the top of every child file \textit{child}
which is included by |\include{|\textit{child}|}|
from within the main file
(or at least for those files to be compiled individually).
The argument \textit{main} must be the filename of the main file.

There are a couple of
considerations in setting up the main and child documents:

%%%%%%%%%%%%%%%%%%%%%%%%%%%%%%%%%%%%%%%%
\paragraph{Restrictions.}

Please note the following restrictions:
\begin{itemize}
\item
|\childdocmain| must be called with one argument \textit{main}
to ensure compatibility with earlier version of the package.
It must either be empty (|\childdocmain{}|)
or precisely match the filename of the main file in which it is specified.
See \secref{sec:detection} for further information.
\item
The filename \textit{main} must be specified without the |.tex| extension.
\item
The filename \textit{main} is case sensitive
(even in case-insensitive file systems)
due to internal string comparison.
\item
The argument \textit{main} should be fully expanded, it cannot be a macro.
\item
Subdirectories and special characters should be avoided in filenames.
\item
The command |\childdocmain{|\textit{main}|}| must be followed by a whitespace.
It should not be followed immediately by another command
or by a comment mark `|%|'.
This is because the \TeX{} parser reads the token immediately following
the argument of |\childdocmain| and puts it
at the beginning of every child section;
however, a white\-space is ignored.
\end{itemize}

%%%%%%%%%%%%%%%%%%%%%%%%%%%%%%%%%%%%%%%%
\paragraph{Content of Main File.}

It is advisable to place all content in the child files included by |\include|.
Any output contained in the main file will appear in all child documents
unless suppressed manually;
it cannot be suppressed automatically by the |\includeonly| directive
and thus should normally be avoided.
A method to include some content in the main file
by means of conditional processing is described in \secref{sec:conditional}.

%%%%%%%%%%%%%%%%%%%%%%%%%%%%%%%%%%%%%%%%
\paragraph{Page Numbering.}

When only a part of the document is compiled,
the appropriate numbering of pages
(as well as other status parameters)
is determined from the |.aux| files.
The latter contain information from previous passes.
However this information needs to propagate through
all intermediate child documents.
Therefore the page numbering in child documents may well
be inconsistent until the complete document is compiled at least once.

A useful (if unconventional) way to always ensure a consistent
page numbering is to restart the numbering in each child document
and denote the pages by `\textit{child}|.|\textit{page}'
where \textit{child} represents the chapter/section number of the child file.
This can be achieved by the command
|\numberwithin{page}{|\textit{child}|}|
of the \textsf{amsmath} package
where \textit{child} can be |chapter| or |section|
depending on the chosen structuring.
Alternatively, one can modify the macro |\thepage| appropriately
and reset the counter |page| at the start of each child file.

%%%%%%%%%%%%%%%%%%%%%%%%%%%%%%%%%%%%%%%%%%%%%%%%%%%%%%%%%%%%%%%%%%%%%%%%%%%%%%%%
\subsection{Conditional Processing}
\label{sec:conditional}

The package provides a mechanism to compile different versions
of a document. To customise the versions further some conditional processing
can come in handy to distinguish which version is being compiled.
The package provides two macros to describe the compilation context:

%%%%%%%%%%%%%%%%%%%%%%%%%%%%%%%%%%%%%%%%
\DescribeMacro{\ifchilddoc}
The conditional |\ifchilddoc| distinguishes between the compilation of
child documents and the main document:
%
\begin{center}
|\ifchilddoc |\textit{child-code}| |[|\||else |\textit{main-code}]| \||fi|
\end{center}

%%%%%%%%%%%%%%%%%%%%%%%%%%%%%%%%%%%%%%%%
\DescribeMacro{\childdocname}
\DescribeMacro{\childdocjob}
The macro |\childdocname| contains the filename (without extension)
of the main or child file being processed.
Note that |\childdocjob| will always contain the name of the main file.

%%%%%%%%%%%%%%%%%%%%%%%%%%%%%%%%%%%%%%%%
\paragraph{Title Page.}

Conditional processing can be used to include a title or banner page
in the main document when proper precautions are taken.
Importantly, the code in the main file should ensure that the page counter
(as well as other status parameters which are stored in the |.aux| files)
takes the same value after the conditional processing.
Otherwise the page numbers may take divergent values
depending on which part is compiled.

For example, a title page could be declared by:
%
\begin{center}
\begin{tabular}{l}
|\ifchilddoc\||else|\\
|\addtocounter{page}{-1}|\\
\textit{code for title page}\\
|\newpage|\\
|\||fi|
\end{tabular}
\end{center}
%
A banner page for the child documents can be generated by:
%
\begin{center}
\begin{tabular}{l}
|\ifchilddoc|\\
|\addtocounter{page}{-1}|\\
\textit{code for banner page}\\
|\newpage|\\
|\||fi|
\end{tabular}
\end{center}
%
Here one could write a message such as:
\begin{center}
|This is the part \childdocname{} of \childdocjob{}.|
\end{center}

%%%%%%%%%%%%%%%%%%%%%%%%%%%%%%%%%%%%%%%%%%%%%%%%%%%%%%%%%%%%%%%%%%%%%%%%%%%%%%%%
\subsection{Flags}
\label{sec:flags}

The package makes it easy to generate different versions
of the main or child documents.
To this end compilation flags can be defined
and assigned different default values.
They will be particularly useful in conjunction
with the forwarding mechanism described in \secref{sec:forward}.

For example, it may be useful to have a flag |\version|
which can be set to |draft| or |final|.
The document source will contain some conditional code
depending on the value of |\version|.
Suppose further, the flag should default to |final| for the main file
and to |draft| for child files
which is a natural assignment for editing the document.
This is achieved by placing the following code
in the preamble of the main document
(below the |\childdocmain| directive):
%
\begin{center}
\begin{tabular}{l}
|\ifchilddoc|\\
|\providecommand{\version}{draft}|\\
|\||else|\\
|\providecommand{\version}{final}|\\
|\||fi|
\end{tabular}
\end{center}
%
The definition by |\providecommand| makes sure
that previous definitions are not overwritten.
Further statements |\providecommand{\version}{...}|
can thus be added before the above code to override it.

For the main file, one might add a line
(between |\childdocmain| and the above block)
%
\begin{center}
|%\ifchilddoc\||else\providecommand{\version}{draft}\||fi|
\end{center}
%
which can be uncommented to produce a draft version.
Likewise one can add a line to the very top of a child file
(above the |\childdocof{|\textit{main}|}| directive)
%
\begin{center}
|%\providecommand{\version}{final}|
\end{center}
%
which can be uncommented to produce the final version of this child document.

%%%%%%%%%%%%%%%%%%%%%%%%%%%%%%%%%%%%%%%%%%%%%%%%%%%%%%%%%%%%%%%%%%%%%%%%%%%%%%%%
\subsection{Forwarding}
\label{sec:forward}

Different versions of the main or child documents
using compilation flags as described in \secref{sec:flags}
can be (permanently) stored in different files
for convenient compilation, viewing and distribution.
To this end, the package defines a command
to pass on compilation to a different file:

%%%%%%%%%%%%%%%%%%%%%%%%%%%%%%%%%%%%%%%%
\DescribeMacro{\childdocforward}
The command |\childdocforward| redirects processing to
another source file:
%
\begin{center}
\begin{tabular}{l}
|\input{childdoc.def}|\\
|\childdocforward[|\textit{main}|]{|\textit{dest}|}|\\
\end{tabular}
\end{center}
%
The argument \textit{dest} is the destination file
(without extension).
It should be the main file or one of the child files.
Note that further \textsf{childdoc} directives
such as |\childdocof| and |\childdocforward|
in the indicated file will be processed in this form.
The optional argument \textit{main}
passes on directly to the main file \textit{main}
while pretending to compile the child \textit{dest}.
This form behaves as if \textit{dest}
issues |\childdocof{|\textit{main}|}| right away,
and no further \textsf{childdoc} directives will be processed.

%%%%%%%%%%%%%%%%%%%%%%%%%%%%%%%%%%%%%%%%
\DescribeMacro{\...prefix}
In the alternative form |\childdocforwardprefix|,
%
\begin{center}
\begin{tabular}{l}
|\input{childdoc.def}|\\
|\childdocforwardprefix[|\textit{main}|]{|\textit{prefix}|}{|\textit{dest}|}|
\end{tabular}
\end{center}
%
the destination file is determined by a pattern
depending on the current file:
To make this work, the current file must be called
`{\textit{prefix}\hspace{0.2em}\textit{suffix}}'
with \textit{prefix} matching precisely the argument.
Processing is then passed on to the file
`{\textit{dest}\hspace{0.2em}\textit{suffix}}'.
Surely, the same effect is achieved by
directly specifying the
argument `{\textit{dest}\hspace{0.2em}\textit{suffix}}'
in the first form.
However, that requires to set up a different file
for each child. With the alternative form of the command
all these files can have exactly the same content
which simplifies setting them up and maintaining them.

For example, the following file |draft.tex|
with a compilation flag |\version| as described in \secref{sec:flags}
compiles the main document as a draft:
%
\begin{center}
\begin{tabular}{l}
|\def\version{draft}|\\
|\input{childdoc.def}|\\
|\childdocforward{|\textit{main}|}|
\end{tabular}
\end{center}
%
Likewise, the following files |final|\textit{nn}|.tex|
compile the final version of the child document
|child|\textit{nn}|.tex|:
%
\begin{center}
\begin{tabular}{l}
|\def\version{final}|\\
|\input{childdoc.def}|\\
|\childdocforwardprefix{final}{child}|
\end{tabular}
\end{center}
%

Note that when several versions of a main file and/or of each child file
are to be generated, it may be convenient to set up a |Makefile| or
shell script to automatise the process.

%%%%%%%%%%%%%%%%%%%%%%%%%%%%%%%%%%%%%%%%%%%%%%%%%%%%%%%%%%%%%%%%%%%%%%%%%%%%%%%%
\subsection{Command Line Processing}
\label{sec:commandline}

The effect of redirection files can also be achieved by invoking
the \LaTeX{} compiler with a more elaborate command line.
Most conveniently this should be done as part
of a shell script or a |Makefile|.

When using \textsf{childdoc} in the main file, the following
command lines effectively perform a redirection
(note that depending on the shell being used,
backslashes may have to be doubled: `|\|' $\to$ `|\\|'):
%
\begin{center}
|... -jobname "|\textit{target}|" |\\|"|[\textit{flags}]%
|\input{childdoc.def}\childdocforward[|\textit{main}|]{|\textit{dest}|}"|
\end{center}
%
Here \textit{target} is the name of the output file,
\textit{main} is the name of the main file
and \textit{dest} is the name of the main or child file to be processed
(all filenames without extensions).
The optional argument \textit{main} can be omitted
if \textit{main} matches \textit{dest}.
Optionally, compilation \textit{flags} can be defined via |\def| commands.
This command line makes the \TeX{} engine believe
it is compiling the file \textit{target}
whose content is specified as the latter parameter.
The provided code then forwards the processing to
\textit{main} or \textit{dest} as described in \secref{sec:forward}.

%%%%%%%%%%%%%%%%%%%%%%%%%%%%%%%%%%%%%%%%%%%%%%%%%%%%%%%%%%%%%%%%%%%%%%%%%%%%%%%%
\subsection{Include by Input}
\label{sec:input}

Including child documents by |\include| has some restrictions by design.
Most notably, the content of a child document always occupies
its own set of pages; pages cannot be shared between child documents.
Usually, this behaviour makes perfect sense
because each child document contain an essential part of the document.
However, in some situations it may be desirable to compose
a document from a collection of parts
without having mandatory page breaks between then.
For this case, the package
provides a mechanism to include parts
by |\input| which can also be processed individually.
However, by construction this mechanism
requires manual handling of the content to be output.

%%%%%%%%%%%%%%%%%%%%%%%%%%%%%%%%%%%%%%%%
\DescribeMacro{\ifchilddocmanual}
The main file should be prepared as usual, see \secref{sec:include}.
However, the document body must make a distinction
between processing of an individual part and of the main document, e.g.:
%
\begin{center}
\begin{tabular}{l}
|\ifchilddocmanual|\\
|\input{\childdocname}|\\
|\||else|\\
\textit{document body with }|\input{|\textit{part}|}|\\
|\||fi|
\end{tabular}
\end{center}
%
The conditional |\ifchilddocmanual| is true whenever
a part to be included by |\input| is being compiled,
and the name of the part is stored in |\childdocname|.

%%%%%%%%%%%%%%%%%%%%%%%%%%%%%%%%%%%%%%%%
\DescribeMacro{\childdocby}
Each part to be included by |\input| should start with:
%
\begin{center}
\begin{tabular}{l}
|\input{childdoc.def}|\\
|\childdocby{|\textit{main}|}|\\
\end{tabular}
\end{center}
%
The directive |\childdocby| is similar to |\childdocof|
described in \secref{sec:include},
but the subsequent selection of content must be done manually.
To that end, both |\ifchilddoc| and |\ifchilddocmanual|
will be true upon processing of a part,
and the name of the part is stored in |\childdocname|.
Note that |\jobname| will be set to the filename of the current part
so that each part receives an individual |.aux| file
that does not interfere with the |.aux| file(s) of the main document.
This behaviour can be altered by the alternative form
|\childdocby[*]{|\textit{main}|}| (with a non-empty optional argument)
which uses the |.aux| file of the main document
by setting |\jobname| to \textit{main}.

%%%%%%%%%%%%%%%%%%%%%%%%%%%%%%%%%%%%%%%%%%%%%%%%%%%%%%%%%%%%%%%%%%%%%%%%%%%%%%%%
\subsection{Driver Development}
\label{sec:driver}

The \textsf{childdoc} mechanism can also be use for the development
of definition files such as \LaTeX{} styles or classes.
This case differs from the above setup with multiple parts
included by |\include| in that no |\includeonly| should be invoked.
This can be achieved by starting the include file
(before |\ProvidesPackage|) with:
%
\begin{center}
\begin{tabular}{l}
|\input{childdoc.def}|\\
|\childdocforward{|\textit{main}|}|\\
\end{tabular}
\end{center}
%
or alternatively with:
%
\begin{center}
\begin{tabular}{l}
|\input{childdoc.def}|\\
|\childdocby{|\textit{main}|}|\\
\end{tabular}
\end{center}
%
Both forms have slightly different effects as described above.
The main file is prepared as usual, see \secref{sec:include}.

%%%%%%%%%%%%%%%%%%%%%%%%%%%%%%%%%%%%%%%%%%%%%%%%%%%%%%%%%%%%%%%%%%%%%%%%%%%%%%%%
\subsection{Legacy Detection}
\label{sec:detection}

The directive |\childdocmain| in the main file can detect
whether the complete document or merely a child is to be compiled
even without using the directive |\childdocof|.
This method is deprecated because it is less robust
and there is no compelling reason to use it;
it is merely provided for backward compatibility
and it may be removed in future versions.

If the detection mechanism is to be used,
it is mandatory to correctly specify
the filename of the main file as the argument of |\childdocmain|:
%
\begin{center}
\begin{tabular}{l}
|\input{childdoc.def}|\\
|\childdocmain{|\textit{main}|}|\\
\end{tabular}
\end{center}
%
If |\jobname| does not match the argument \textit{main} of |\childdocmain|,
it is assumed that |\jobname| points to the child file to be compiled.
When using |\childdocmain| with the main file specified as argument,
it suffices to start a child file
with just |\input{|\textit{main}|}|
without loading of the package and using |\childdocof|.
If instead all processing is done
with the appropriate \textsf{childdoc} directives,
the argument of \textit{main} of |\childdocmain| can be empty.

An alternative version of the command line processing described
in \secref{sec:commandline} using the detection mechanism reads:
%
\begin{center}
|... -jobname "|\textit{target}|" "|[\textit{flags}]%
[|\def\jobname{|\textit{dest}|}|]|\input{|\textit{main}|}"|
\end{center}

%%%%%%%%%%%%%%%%%%%%%%%%%%%%%%%%%%%%%%%%%%%%%%%%%%%%%%%%%%%%%%%%%%%%%%%%%%%%%%%%
\subsection{Manual Code}
\label{sec:manual}

In case one cannot be certain whether the definitions file |childdoc.def|
is installed on the target \TeX{} distribution
and one prefers not to ship it,
it is conceivable to paste a few relevant commands into the sources.

To that end, drop all statements |\input{childdoc.def}|
and perform the replacements as outlined below.
Instead of |\childdocmain{|\textit{main}|}| add the following code
to the top of the main file:
%
\begin{center}
\begin{tabular}{l}
|\||ifdefined\childdocname\endinput\||fi\newif\ifchilddoc|\\
|\edef\childdocname{\scantokens\expandafter{\jobname\noexpand}}|\\
|\def\childdocmain{|\textit{main}|}\||ifx\childdocmain\childdocname\||else|\\
|\childdoctrue\includeonly{\childdocname}\let\jobname\childdocmain\||fi|\\
\end{tabular}
\end{center}
%
Instead of |\childdocof{|\textit{main}|}| just include the main file
at the top of each child file:
%
\begin{center}
|\input{|\textit{main}|}|
\end{center}
%
A simple redirection |\childdocforward{|\textit{dest}|}| is achieved by:
%
\begin{center}
|\def\jobname{|\textit{dest}|}\input{\jobname}|
\end{center}
%
The redirection with prefix
|\childdocforwardprefix[|\textit{prefix}|]{|\textit{dest}|}|
is accomplished by:
%
\begin{center}
\begin{tabular}{l}
|{\edef\jobname{\scantokens\expandafter{\jobname\noexpand}}|\\
|\def\redirectjob |\textit{prefix}|#1~~~{\gdef\jobname{|\textit{dest}|#1}}|\\
|\expandafter\redirectjob\jobname~~~}\input{\jobname}|
\end{tabular}
\end{center}

In an alternative approach,
child documents can be compiled by a specific command line
without additional code or specific definitions:
%
\begin{center}
|... -jobname "|\textit{target}|" "|[\textit{flags}]%
|\includeonly{|\textit{dest}|}\input{|\textit{main}|}"|
\end{center}
%

%%%%%%%%%%%%%%%%%%%%%%%%%%%%%%%%%%%%%%%%%%%%%%%%%%%%%%%%%%%%%%%%%%%%%%%%%%%%%%%%
%%%%%%%%%%%%%%%%%%%%%%%%%%%%%%%%%%%%%%%%%%%%%%%%%%%%%%%%%%%%%%%%%%%%%%%%%%%%%%%%
\section{Information}

%%%%%%%%%%%%%%%%%%%%%%%%%%%%%%%%%%%%%%%%%%%%%%%%%%%%%%%%%%%%%%%%%%%%%%%%%%%%%%%%
\subsection{Copyright}

Copyright \copyright{} 2017--2018 Niklas Beisert

This work may be distributed and/or modified under the
conditions of the \LaTeX{} Project Public License, either version 1.3
of this license or (at your option) any later version.
The latest version of this license is in
  \url{http://www.latex-project.org/lppl.txt}
and version 1.3 or later is part of all distributions of \LaTeX{}
version 2005/12/01 or later.

This work has the LPPL maintenance status `maintained'.

The Current Maintainer of this work is Niklas Beisert.

This work consists of the files |README.txt|, |childdoc.ins| and |childdoc.dtx|
as well as the derived files |childdoc.def|, |cdocsamp.tex|
with |cdocsch1.tex|, |cdocsch2.tex|, |cdocspt3.tex|, |cdocspt4.tex|,
|cdocsdrf.tex|, |cdocsfn1.tex|, |cdocsfn2.tex|
as well as |childdoc.pdf|.

%%%%%%%%%%%%%%%%%%%%%%%%%%%%%%%%%%%%%%%%%%%%%%%%%%%%%%%%%%%%%%%%%%%%%%%%%%%%%%%%
\subsection{Files and Installation}

The package consists of the files:
%
\begin{center}
\begin{tabular}{ll}
    |README.txt|   & readme file \\
    |childdoc.ins| & installation file \\
    |childdoc.dtx| & source file \\
    |childdoc.def| & definition file \\
    |cdocsamp.tex| & sample main file \\
    |cdocsch1.tex| & sample include file \\
    |cdocsch2.tex| & sample include file \\
    |cdocspt3.tex| & sample part file \\
    |cdocspt4.tex| & sample part file \\
    |cdocsdrf.tex| & sample redirection file \\
    |cdocsfn1.tex| & sample redirection file \\
    |cdocsfn2.tex| & sample redirection file \\
    |childdoc.pdf| & manual
\end{tabular}
\end{center}
%
The distribution consists of the files
|README.txt|, |childdoc.ins| and |childdoc.dtx|.
%
\begin{itemize}
\item
Run (pdf)\LaTeX{} on |childdoc.dtx|
to compile the manual |childdoc.pdf| (this file).
\item
Run \LaTeX{} on |childdoc.ins| to create the definitions file |childdoc.def|
and the sample |cdocsamp.tex| with include files
|cdocsch1.tex|, |cdocsch2.tex|, |cdocspt3.tex|, |cdocspt4.tex|,
|cdocsdrf.tex|, |cdocsfn1.tex|, |cdocsfn2.tex|.
Then copy the file |childdoc.def| to an appropriate directory of your \LaTeX{}
distribution, e.g.\ \textit{texmf-root}|/tex/latex/childdoc|.
\end{itemize}

%%%%%%%%%%%%%%%%%%%%%%%%%%%%%%%%%%%%%%%%%%%%%%%%%%%%%%%%%%%%%%%%%%%%%%%%%%%%%%%%
\subsection{Related CTAN Packages}

There are several other packages which offer a similar functionality:
%
\begin{itemize}
\item
The packages
\href{http://ctan.org/pkg/docmute}{\textsf{docmute}},
\href{http://ctan.org/pkg/includex}{\textsf{includex}} and
\href{http://ctan.org/pkg/standalone}{\textsf{standalone}}
provide commands to include only the document body of
a child file thus allowing both files to be compiled individually.
\item
The packages \href{http://ctan.org/pkg/subdocs}{\textsf{subdocs}}
and \href{http://ctan.org/pkg/subfiles}{\textsf{subfiles}}
provide structures in which the main and child documents can be
encapsulated and allowing them to be compiled individually.
The inclusion mechanism is different from the conventional |\include|.
\item
The package \href{http://ctan.org/pkg/combine}{\textsf{combine}}
is an elaborate solution to combine several documents into one.
\end{itemize}
%
See also the CTAN topic \href{http://ctan.org/topic/subdocs}{\textsf{subdocs}}
for further related packages.
The present package differs from the above solutions in that
a document structure constructed with the conventional |\include| mechanism
just needs two extra commands at the top of every file
such that all constituent files can be compiled individually.

%%%%%%%%%%%%%%%%%%%%%%%%%%%%%%%%%%%%%%%%%%%%%%%%%%%%%%%%%%%%%%%%%%%%%%%%%%%%%%%%
%\subsection{Feature Suggestions}
%
%The following is a list of features which may be useful for future
%versions of this package:
%%
%\begin{itemize}
%\item
%\ldots
%\end{itemize}

%%%%%%%%%%%%%%%%%%%%%%%%%%%%%%%%%%%%%%%%%%%%%%%%%%%%%%%%%%%%%%%%%%%%%%%%%%%%%%%%
\subsection{Revision History}

%%%%%%%%%%%%%%%%%%%%%%%%%%%%%%%%%%%%%%%%
\paragraph{v2.0:} 2018/12/30

\begin{itemize}
\item
immediate forward processing
\item
added |\childdocby| mechanism
\item
manual restructured
\end{itemize}

%%%%%%%%%%%%%%%%%%%%%%%%%%%%%%%%%%%%%%%%
\paragraph{v1.6:} 2018/01/17

\begin{itemize}
\item
application for development of include files
\item
corrections to manual
\end{itemize}

%%%%%%%%%%%%%%%%%%%%%%%%%%%%%%%%%%%%%%%%
\paragraph{v1.5:} 2017/05/21

\begin{itemize}
\item
more complete structuring introduced
\item
|\childdocof| introduced
\item
|\childdoc| renamed to |\childdocmain|
\item
|\childredirect| renamed to |\childdocforward| and |\childdocforwardprefix|
and functionality expanded
\end{itemize}

%%%%%%%%%%%%%%%%%%%%%%%%%%%%%%%%%%%%%%%%
\paragraph{v1.0:} 2017/04/27

\begin{itemize}
\item
manual and install package
\item
first version published on CTAN
\end{itemize}

%%%%%%%%%%%%%%%%%%%%%%%%%%%%%%%%%%%%%%%%
\paragraph{v0.6:} 2017/04/26

\begin{itemize}
\item
redirection mechanism added
\end{itemize}

%%%%%%%%%%%%%%%%%%%%%%%%%%%%%%%%%%%%%%%%
\paragraph{v0.5:} 2017/04/26

\begin{itemize}
\item
functionality in definition file
\end{itemize}


%%%%%%%%%%%%%%%%%%%%%%%%%%%%%%%%%%%%%%%%%%%%%%%%%%%%%%%%%%%%%%%%%%%%%%%%%%%%%%%%
%%%%%%%%%%%%%%%%%%%%%%%%%%%%%%%%%%%%%%%%%%%%%%%%%%%%%%%%%%%%%%%%%%%%%%%%%%%%%%%%
%%%%%%%%%%%%%%%%%%%%%%%%%%%%%%%%%%%%%%%%%%%%%%%%%%%%%%%%%%%%%%%%%%%%%%%%%%%%%%%%
\appendix

\settowidth\MacroIndent{\rmfamily\scriptsize 000\ }

 \DocInput{childdoc.dtx}

\end{document}
%</driver>
% \fi
%
% %%%%%%%%%%%%%%%%%%%%%%%%%%%%%%%%%%%%%%%%%%%%%%%%%%%%%%%%%%%%%%%%%%%%%%%%%%%%%%
% %%%%%%%%%%%%%%%%%%%%%%%%%%%%%%%%%%%%%%%%%%%%%%%%%%%%%%%%%%%%%%%%%%%%%%%%%%%%%%
% \section{Sample}
%\iffalse
%<*samplemain>
%\fi
%
% The following presents a sample document
% with two chapters, two parts, a title page,
% a compile flag as well as three forwarding files to set the flag.
% It consists of eight |.tex| files:
% \begin{center}
% \begin{tabular}{ll}
% |cdocsamp.tex|&main file\\
% |cdocsch1.tex|&include file for chapter 1\\
% |cdocsch2.tex|&include file for chapter 2\\
% |cdocspt3.tex|&include file for part 3\\
% |cdocspt4.tex|&include file for part 4\\
% |cdocsdrf.tex|&forwarding file for main file in draft mode\\
% |cdocsfi1.tex|&forwarding file for final version of chapter 1\\
% |cdocsfi2.tex|&forwarding file for final version of chapter 2\\
% \end{tabular}
% \end{center}
% Each of the eight files can be compiled directly by the \LaTeX{} compiler.
%
% %%%%%%%%%%%%%%%%%%%%%%%%%%%%%%%%%%%%%%
% \paragraph{Main File.}
%
% The main file is called |cdocsamp.tex|.
%
% Load the \textsf{childdoc} definitions and
% declare the filename for the main document:
%    \begin{macrocode}
\input{childdoc.def}
\childdocmain{}
%    \end{macrocode}

% Optional override for |\version| flag:
%    \begin{macrocode}
%%\ifchilddoc\else\providecommand{\version}{draft}\fi
%    \end{macrocode}

% Define the default values for the |\version| flag
% (|final| for the main file and |draft| for childs):
%    \begin{macrocode}
\ifchilddoc
\providecommand{\version}{draft}
\else
\providecommand{\version}{final}
\fi
%    \end{macrocode}

% Load the standard document class:
%    \begin{macrocode}
\documentclass[12pt]{article}
%    \end{macrocode}

% Start the document body:
%    \begin{macrocode}
\begin{document}
%    \end{macrocode}

% Declare a title page.
% Print title, part of document being processed and version flag:
%    \begin{macrocode}
\addtocounter{page}{-1}
\begin{center}
{\LARGE\bfseries{}childdoc example\par}
\vspace{1cm}
\ifchilddoc
\ifchilddocmanual part\else chapter\fi:
`\childdocname' of `\childdocjob'\par
\else
main document: `\childdocjob'\par
\fi
version: \version\par
\end{center}
\newpage
%    \end{macrocode}

% Manually include selected file,
% otherwise process as usual:
%    \begin{macrocode}
\ifchilddocmanual
\section*{part `\childdocname'}
\input{\childdocname}
\else
%    \end{macrocode}

% Include the two chapters:
%    \begin{macrocode}
\include{cdocsch1}
\include{cdocsch2}
%    \end{macrocode}

% Include the two parts unless only chapters should be displayed:
%    \begin{macrocode}
\ifchilddoc\else
\section{part three}
\input{cdocspt3}
\section{part four}
\input{cdocspt4}
\fi
%    \end{macrocode}

% Process as usual until here:
%    \begin{macrocode}
\fi
%    \end{macrocode}

% End of document body:
%    \begin{macrocode}
\end{document}
%    \end{macrocode}
%\iffalse
%</samplemain>
%\fi
%
% %%%%%%%%%%%%%%%%%%%%%%%%%%%%%%%%%%%%%%
% \paragraph{Chapter Include Files.}
%
% The include files are called |cdocsch1.tex| and |cdocsch2.tex|.
%
%\iffalse
%<*samplechap1|samplechap2>
%\fi

% Optional override for |\version| flag:
%    \begin{macrocode}
%%\providecommand{\version}{final}
%    \end{macrocode}

% Include the main document:
%    \begin{macrocode}
\input{childdoc.def}
\childdocof{cdocsamp}
%    \end{macrocode}

%\iffalse
%</samplechap1|samplechap2>
%\fi
%
%\iffalse
%<*samplechap1>
%\fi
% Some text for chapter 1:
%    \begin{macrocode}
\section{one}
some text in chapter one
%    \end{macrocode}

%\iffalse
%</samplechap1>
%\fi
% Some text for chapter 2:
%\iffalse
%<*samplechap2>
%\fi
%    \begin{macrocode}
\section{two}
more text in chapter two
%    \end{macrocode}

%\iffalse
%</samplechap2>
%\fi
%
% %%%%%%%%%%%%%%%%%%%%%%%%%%%%%%%%%%%%%%
% \paragraph{Part Include Files.}
%
% The include files are called |cdocspt3.tex| and |cdocspt4.tex|.
%
%\iffalse
%<*samplepart3|samplepart4>
%\fi

% Optional override for |\version| flag:
%    \begin{macrocode}
%%\providecommand{\version}{final}
%    \end{macrocode}

% Include the main document:
%    \begin{macrocode}
\input{childdoc.def}
\childdocby{cdocsamp}
%    \end{macrocode}

%\iffalse
%</samplepart3|samplepart4>
%\fi
%
%\iffalse
%<*samplepart3>
%\fi
% Some text for part 3:
%    \begin{macrocode}
some text in part three
%    \end{macrocode}

%\iffalse
%</samplepart3>
%\fi
% Some text for part 4:
%\iffalse
%<*samplepart4>
%\fi
%    \begin{macrocode}
more text in part four
%    \end{macrocode}

%\iffalse
%</samplepart4>
%\fi
%
% %%%%%%%%%%%%%%%%%%%%%%%%%%%%%%%%%%%%%%
% \paragraph{Forwarding for a Complete Draft.}
%
% The following forwarding file |cdocsdrf.tex|
% compiles the main document in draft mode:
%\iffalse
%<*sampledraft>
%\fi
%    \begin{macrocode}
\def\version{draft}
\input{childdoc.def}
\childdocforward{cdocsamp}
%    \end{macrocode}

%\iffalse
%</sampledraft>
%\fi
%
% %%%%%%%%%%%%%%%%%%%%%%%%%%%%%%%%%%%%%%
% \paragraph{Forwarding for Final Version of the Chapters.}
%
% The following forwarding files |cdocsfn1.tex| and |cdocsfn2.tex|
% (with identical content)
% compile the final versions of the child documents
% |cdocsch1.tex| and |cdocsch2.tex|, respectively:
%\iffalse
%<*samplefinal>
%\fi
%    \begin{macrocode}
\def\version{final}
\input{childdoc.def}
\childdocforwardprefix[cdocsamp]{cdocsfn}{cdocsch}
%    \end{macrocode}

%\iffalse
%</samplefinal>
%\fi
%
% %%%%%%%%%%%%%%%%%%%%%%%%%%%%%%%%%%%%%%
% \paragraph{Command Line Processing.}
%
% The following three command lines generate the output files
% |cdocscld|, |cdocscl1| and |cdocscl2|
% which should be identical to
% |cdocsdrf|, |cdocsch1| and |cdocsfn2|, respectively:
% \begin{center}
% \begin{tabular}{l}
% |latex -jobname cdocscld \|\\
% |  "\def\version{draft}\input{childdoc.def}\childdocforward{cdocsamp}"|\\
% |latex -jobname cdocscl1 \|\\
% |  "\input{childdoc.def}\childdocforward[cdocsamp]{cdocsch1}"|\\
% |latex -jobname cdocscl2 \|\\
% |  "\def\version{final}\input{childdoc.def}\childdocforward{cdocsch2}"|
% \end{tabular}
% \end{center}
% Note that the trailing backslash on each first line
% merely continues the input to the second line
% (for convenient cut ant paste).
% Furthermore, the command |latex| can be replaced by any
% of its alternative versions such as |pdflatex|.
%
% %%%%%%%%%%%%%%%%%%%%%%%%%%%%%%%%%%%%%%%%%%%%%%%%%%%%%%%%%%%%%%%%%%%%%%%%%%%%%%
% %%%%%%%%%%%%%%%%%%%%%%%%%%%%%%%%%%%%%%%%%%%%%%%%%%%%%%%%%%%%%%%%%%%%%%%%%%%%%%
% \section{Implementation}
%\iffalse
%<*package>
%\fi
%
% This section describes the definitions file |childdoc.def|.

% The definitions cannot be loaded using |\usepackage| or |\RequirePackage|
% which has a mechanism to prevent loading a style file more than once.
% When loading the definitions by means of |\input|
% multiple instances have to be prevented manually:
%\iffalse
%This code needs to be before the `\ProvidesFile' directive
%which is defined at the beginning of this file.
%Therefore it is also placed there and commented out here.
%</package>
%<*discard>
%\fi
%    \begin{macrocode}
\ifdefined\childdocmain\endinput\fi
%    \end{macrocode}
%\iffalse
%</discard>
%<*package>
%\fi
%
% \macro{\ifchilddoc}
% \macro{\ifchilddocmanual}
% The conditional |\ifchilddoc| tells whether a
% child (true) or main (false) document is being compiled.
% The conditional |\ifchilddocmanual| tells whether
% the |\includeonly| mechanism is used (false) or
% the selection of child files must be performed manually (true).
% The definitions initialise to false:
%    \begin{macrocode}
\newif\ifchilddoc
\newif\ifchilddocmanual
%    \end{macrocode}

% \macro{\childdocname}
% \macro{\childdocjob}
% The macro |\childdocname| stores the name of the main document
% to be compiled. The macro |\childdocjob| stores the name of
% the document on which the \LaTeX{} compiler was originally invoked.
% The content of |\jobname| cannot be compared
% to filenames specified in the source due to different catcodes.
% The following code rescans |\jobname|, stores the result
% in |\childdocname| and saves a copy in |\childdocjob|:
%    \begin{macrocode}
\edef\childdocname{\scantokens\expandafter{\jobname\noexpand}}
\let\childdocjob\childdocname
%    \end{macrocode}

% \macro{\childdocdisable}
% The macro |\childdocdisable| prevents the main file
% from being processed more than once.
% At this stage, the main document command |\childdocmain|
% is assumed to be called once again where it should do nothing.
% Any subsequent call to it should prevent
% a secondary processing of the main document
% It overwrites the forwarding commands
% |\childdocof| and |\childdocforward|
% with empty macros to prevent further inclusions of the main document:
%    \begin{macrocode}
\newcommand{\childdocdisable}
{
  \renewcommand{\childdocmain}[1]{\renewcommand{\childdocmain}[1]{\endinput}}
  \renewcommand{\childdocof}[1]{}
  \renewcommand{\childdocby}[2][]{}
  \renewcommand{\childdocforward}[2][]{}
  \renewcommand{\childdocdisable}{}
}
%    \end{macrocode}

% \macro{\childdocmain}
% The macro |\childdocmain| is to be called at the top of the main file
% with nothing or the main filename (without extension) as argument.
% First, it breaks loops.
% If the argument is not empty and does not match |\childdocname|
% (which is set by the first inclusion of |childdoc.def|),
% |\ifchilddoc| is set to true, |\includeonly| is applied to the child file
% and |\jobname| is set to the main file
% (for proper handling of |.aux| files):
%    \begin{macrocode}
\newcommand{\childdocmain}[1]
{
  \childdocdisable\childdocmain{}
  \if?#1?\else
    \begingroup
      \def\childdoctmp{#1}
      \ifx\childdoctmp\childdocname
        \def\childdoctmp{}
      \else
        \def\childdoctmp
        {
          \childdoctrue
          \includeonly{\childdocname}
          \def\childdocjob{#1}
          \def\jobname{#1}
        }
      \fi
      \expandafter
    \endgroup
    \childdoctmp
  \fi
}
%    \end{macrocode}

% \macro{\childdocof}
% The command |\childdocof| redirects
% compilation to the main file |#1|.
%    \begin{macrocode}
\newcommand{\childdocof}[1]
{
  \childdocdisable
  \childdoctrue
  \includeonly{\childdocname}
  \def\jobname{#1}
  \def\childdocjob{#1}
  \input{#1}
}
%    \end{macrocode}

% \macro{\childdocby}
% The command |\childdocby| ....
%    \begin{macrocode}
\newcommand{\childdocby}[2][]
{
  \childdocdisable
  \childdoctrue
  \childdocmanualtrue
  \if?#1?\else
    \def\jobname{#2}
  \fi
  \def\childdocjob{#2}
  \input{#2}
  \endinput
}
%    \end{macrocode}

% \macro{\childdocforward}
% The command |\childdocforward| redirects
% compilation to the main file or
% (if the optional argument is given) a child file.
% Parameters are set as if the main file
% or a child file starting with |\childdocof| was compiled.
% Then compilation is handed over to the main file:
%    \begin{macrocode}
\newcommand{\childdocforward}[2][]
{
  \begingroup
    \if?#1?
      \def\childdoctmp
      {
        \def\childdocname{#2}
        \def\childdocjob{#2}
        \def\jobname{#2}
        \input{#2}
        \endinput
      }
    \else
      \def\childdoctmp
      {
        \childdocdisable
        \def\childdocname{#2}
        \childdoctrue
        \includeonly{#2}
        \def\childdocjob{#1}
        \def\jobname{#1}
        \input{#1}
        \endinput
      }
    \fi
    \expandafter
  \endgroup
  \childdoctmp
}
%    \end{macrocode}

% \macro{\childdocforwardprefix}
% The command |\childdocforwardprefix| redirects
% compilation to the main or a child file by means of a pattern.
% The prefix |#1| in the current filename is replaced by |#2|
% and the suffix of the current filename is kept
% (it is assumed that the filename does not contain the substring `|~~~|'
% which is used as a delimiter).
% Compilation is handed over to the new file by |\childdocforward|:
%    \begin{macrocode}
\newcommand{\childdocforwardprefix}[3][]
{
  \begingroup
    \def\childdocextract #2##1~~~{\def\childdoctmp{\childdocforward[#1]{#3##1}}}
    \expandafter\childdocextract\childdocname~~~
    \expandafter
  \endgroup
  \childdoctmp
}
%    \end{macrocode}

% \macro{\childdoc}
% The deprecated macro |\childdoc| is a legacy version of |\childdocmain|:
%    \begin{macrocode}
\newcommand{\childdoc}{\childdocmain}
%    \end{macrocode}

% \macro{\childdocredirect}
% The deprecated macro |\childdocredirect| is a legacy version
% of |\childdocforward| and |\childdocforwardprefix|:
%    \begin{macrocode}
\newcommand{\childdocredirect}[2][]
{
  \begingroup
    \if?#1?
      \def\childdoctmp{\childdocforward{#2}}
    \else
      \def\childdoctmp{\childdocforwardprefix{#1}{#2}}
    \fi
    \expandafter
  \endgroup
  \childdoctmp
}
%    \end{macrocode}

%\iffalse
%</package>
%\fi
%
\endinput
|\\
|\childdocof{|\textit{main}|}|\\
\end{tabular}
\end{center}
at the top of every child file \textit{child}
which is included by |\include{|\textit{child}|}|
from within the main file
(or at least for those files to be compiled individually).
The argument \textit{main} must be the filename of the main file.

There are a couple of
considerations in setting up the main and child documents:

%%%%%%%%%%%%%%%%%%%%%%%%%%%%%%%%%%%%%%%%
\paragraph{Restrictions.}

Please note the following restrictions:
\begin{itemize}
\item
|\childdocmain| must be called with one argument \textit{main}
to ensure compatibility with earlier version of the package.
It must either be empty (|\childdocmain{}|)
or precisely match the filename of the main file in which it is specified.
See \secref{sec:detection} for further information.
\item
The filename \textit{main} must be specified without the |.tex| extension.
\item
The filename \textit{main} is case sensitive
(even in case-insensitive file systems)
due to internal string comparison.
\item
The argument \textit{main} should be fully expanded, it cannot be a macro.
\item
Subdirectories and special characters should be avoided in filenames.
\item
The command |\childdocmain{|\textit{main}|}| must be followed by a whitespace.
It should not be followed immediately by another command
or by a comment mark `|%|'.
This is because the \TeX{} parser reads the token immediately following
the argument of |\childdocmain| and puts it
at the beginning of every child section;
however, a white\-space is ignored.
\end{itemize}

%%%%%%%%%%%%%%%%%%%%%%%%%%%%%%%%%%%%%%%%
\paragraph{Content of Main File.}

It is advisable to place all content in the child files included by |\include|.
Any output contained in the main file will appear in all child documents
unless suppressed manually;
it cannot be suppressed automatically by the |\includeonly| directive
and thus should normally be avoided.
A method to include some content in the main file
by means of conditional processing is described in \secref{sec:conditional}.

%%%%%%%%%%%%%%%%%%%%%%%%%%%%%%%%%%%%%%%%
\paragraph{Page Numbering.}

When only a part of the document is compiled,
the appropriate numbering of pages
(as well as other status parameters)
is determined from the |.aux| files.
The latter contain information from previous passes.
However this information needs to propagate through
all intermediate child documents.
Therefore the page numbering in child documents may well
be inconsistent until the complete document is compiled at least once.

A useful (if unconventional) way to always ensure a consistent
page numbering is to restart the numbering in each child document
and denote the pages by `\textit{child}|.|\textit{page}'
where \textit{child} represents the chapter/section number of the child file.
This can be achieved by the command
|\numberwithin{page}{|\textit{child}|}|
of the \textsf{amsmath} package
where \textit{child} can be |chapter| or |section|
depending on the chosen structuring.
Alternatively, one can modify the macro |\thepage| appropriately
and reset the counter |page| at the start of each child file.

%%%%%%%%%%%%%%%%%%%%%%%%%%%%%%%%%%%%%%%%%%%%%%%%%%%%%%%%%%%%%%%%%%%%%%%%%%%%%%%%
\subsection{Conditional Processing}
\label{sec:conditional}

The package provides a mechanism to compile different versions
of a document. To customise the versions further some conditional processing
can come in handy to distinguish which version is being compiled.
The package provides two macros to describe the compilation context:

%%%%%%%%%%%%%%%%%%%%%%%%%%%%%%%%%%%%%%%%
\DescribeMacro{\ifchilddoc}
The conditional |\ifchilddoc| distinguishes between the compilation of
child documents and the main document:
%
\begin{center}
|\ifchilddoc |\textit{child-code}| |[|\||else |\textit{main-code}]| \||fi|
\end{center}

%%%%%%%%%%%%%%%%%%%%%%%%%%%%%%%%%%%%%%%%
\DescribeMacro{\childdocname}
\DescribeMacro{\childdocjob}
The macro |\childdocname| contains the filename (without extension)
of the main or child file being processed.
Note that |\childdocjob| will always contain the name of the main file.

%%%%%%%%%%%%%%%%%%%%%%%%%%%%%%%%%%%%%%%%
\paragraph{Title Page.}

Conditional processing can be used to include a title or banner page
in the main document when proper precautions are taken.
Importantly, the code in the main file should ensure that the page counter
(as well as other status parameters which are stored in the |.aux| files)
takes the same value after the conditional processing.
Otherwise the page numbers may take divergent values
depending on which part is compiled.

For example, a title page could be declared by:
%
\begin{center}
\begin{tabular}{l}
|\ifchilddoc\||else|\\
|\addtocounter{page}{-1}|\\
\textit{code for title page}\\
|\newpage|\\
|\||fi|
\end{tabular}
\end{center}
%
A banner page for the child documents can be generated by:
%
\begin{center}
\begin{tabular}{l}
|\ifchilddoc|\\
|\addtocounter{page}{-1}|\\
\textit{code for banner page}\\
|\newpage|\\
|\||fi|
\end{tabular}
\end{center}
%
Here one could write a message such as:
\begin{center}
|This is the part \childdocname{} of \childdocjob{}.|
\end{center}

%%%%%%%%%%%%%%%%%%%%%%%%%%%%%%%%%%%%%%%%%%%%%%%%%%%%%%%%%%%%%%%%%%%%%%%%%%%%%%%%
\subsection{Flags}
\label{sec:flags}

The package makes it easy to generate different versions
of the main or child documents.
To this end compilation flags can be defined
and assigned different default values.
They will be particularly useful in conjunction
with the forwarding mechanism described in \secref{sec:forward}.

For example, it may be useful to have a flag |\version|
which can be set to |draft| or |final|.
The document source will contain some conditional code
depending on the value of |\version|.
Suppose further, the flag should default to |final| for the main file
and to |draft| for child files
which is a natural assignment for editing the document.
This is achieved by placing the following code
in the preamble of the main document
(below the |\childdocmain| directive):
%
\begin{center}
\begin{tabular}{l}
|\ifchilddoc|\\
|\providecommand{\version}{draft}|\\
|\||else|\\
|\providecommand{\version}{final}|\\
|\||fi|
\end{tabular}
\end{center}
%
The definition by |\providecommand| makes sure
that previous definitions are not overwritten.
Further statements |\providecommand{\version}{...}|
can thus be added before the above code to override it.

For the main file, one might add a line
(between |\childdocmain| and the above block)
%
\begin{center}
|%\ifchilddoc\||else\providecommand{\version}{draft}\||fi|
\end{center}
%
which can be uncommented to produce a draft version.
Likewise one can add a line to the very top of a child file
(above the |\childdocof{|\textit{main}|}| directive)
%
\begin{center}
|%\providecommand{\version}{final}|
\end{center}
%
which can be uncommented to produce the final version of this child document.

%%%%%%%%%%%%%%%%%%%%%%%%%%%%%%%%%%%%%%%%%%%%%%%%%%%%%%%%%%%%%%%%%%%%%%%%%%%%%%%%
\subsection{Forwarding}
\label{sec:forward}

Different versions of the main or child documents
using compilation flags as described in \secref{sec:flags}
can be (permanently) stored in different files
for convenient compilation, viewing and distribution.
To this end, the package defines a command
to pass on compilation to a different file:

%%%%%%%%%%%%%%%%%%%%%%%%%%%%%%%%%%%%%%%%
\DescribeMacro{\childdocforward}
The command |\childdocforward| redirects processing to
another source file:
%
\begin{center}
\begin{tabular}{l}
|% \iffalse
%
% childdoc.dtx Copyright (C) 2017-2018 Niklas Beisert
%
% This work may be distributed and/or modified under the
% conditions of the LaTeX Project Public License, either version 1.3
% of this license or (at your option) any later version.
% The latest version of this license is in
%   http://www.latex-project.org/lppl.txt
% and version 1.3 or later is part of all distributions of LaTeX
% version 2005/12/01 or later.
%
% This work has the LPPL maintenance status `maintained'.
%
% The Current Maintainer of this work is Niklas Beisert.
%
% This work consists of the files childdoc.dtx and childdoc.ins
% and the derived files childdoc.def and cdocsamp.tex with
% cdocsch1.tex, cdocsch2.tex, cdocsdrf.tex, cdocsfn1.tex, cdocsfn2.tex.
%
%<package>\ifdefined\childdocmain\endinput\fi
%<package>\ProvidesFile{childdoc.def}[2018/12/30 v2.0 child document driver]
%<samplemain>\ProvidesFile{cdocsamp.tex}[2018/12/30 v2.0 sample for childdoc]
%<*driver>
%\ProvidesFile{childdoc.drv}[2018/12/30 v2.0 childdoc reference manual file]
\PassOptionsToClass{10pt,a4paper}{article}
\documentclass{ltxdoc}

\usepackage[margin=35mm]{geometry}
\usepackage{hyperref}
\usepackage{hyperxmp}
\usepackage[usenames]{color}

\hypersetup{colorlinks=true}
\hypersetup{pdfstartview=FitH}
\hypersetup{pdfpagemode=UseNone}
\hypersetup{pdfsource={}}
\hypersetup{pdflang={en-UK}}
\hypersetup{pdfcopyright={Copyright 2017-2018 Niklas Beisert.
  This work may be distributed and/or modified under the
  conditions of the LaTeX Project Public License, either version 1.3
  of this license or (at your option) any later version.}}
\hypersetup{pdflicenseurl={http://www.latex-project.org/lppl.txt}}
\hypersetup{pdfcontactaddress={ETH Zurich, ITP, HIT K,
  Wolfgang-Pauli-Strasse 27}}
\hypersetup{pdfcontactpostcode={8093}}
\hypersetup{pdfcontactcity={Zurich}}
\hypersetup{pdfcontactcountry={Switzerland}}
\hypersetup{pdfcontactemail={nbeisert@itp.phys.ethz.ch}}
\hypersetup{pdfcontacturl={http://people.phys.ethz.ch/\xmptilde nbeisert/}}

\newcommand{\secref}[1]{\hyperref[#1]{section \ref*{#1}}}

\parskip1ex
\parindent0pt
\let\olditemize\itemize
\def\itemize{\olditemize\parskip0pt}

\begin{document}

\title{The \textsf{childdoc} Package}
\hypersetup{pdftitle={The childdoc Package}}
\author{Niklas Beisert\\[2ex]
  Institut f\"ur Theoretische Physik\\
  Eidgen\"ossische Technische Hochschule Z\"urich\\
  Wolfgang-Pauli-Strasse 27, 8093 Z\"urich, Switzerland\\[1ex]
  \href{mailto:nbeisert@itp.phys.ethz.ch}
  {\texttt{nbeisert@itp.phys.ethz.ch}}}
\hypersetup{pdfauthor={Niklas Beisert}}
\hypersetup{pdfsubject={Manual for the LaTeX2e Package childdoc}}
\date{30 December 2018, \textsf{v2.0}}
\maketitle

\begin{abstract}\noindent
\textsf{childdoc} is a \LaTeXe{} package
that enables the direct compilation
of document sections included by |\include|
to individual files.
\end{abstract}

\begingroup
\parskip0ex
\tableofcontents
\endgroup

%%%%%%%%%%%%%%%%%%%%%%%%%%%%%%%%%%%%%%%%%%%%%%%%%%%%%%%%%%%%%%%%%%%%%%%%%%%%%%%%
%%%%%%%%%%%%%%%%%%%%%%%%%%%%%%%%%%%%%%%%%%%%%%%%%%%%%%%%%%%%%%%%%%%%%%%%%%%%%%%%
\section{Introduction}

\LaTeX{} provides a mechanism to structure a large document (such as a book)
into a main file and several child files (containing the chapters)
using the |\include| command.
This mechanism is beneficial for documents
which span hundreds of pages in order to
make the source file(s) more manageable.
Moreover, compilation can be restricted to
selected child files by means of the |\includeonly| command.
The latter feature can be used to reduce the compilation time while editing
(this was significantly more useful in the earlier days of \LaTeX{})
or to generate a smaller document which is easier to navigate.
Another application of |\includeonly| is to generate
documents consisting of selected parts of the complete document.

However, there are a few drawbacks of the plain |\include| mechanism:
\begin{itemize}
\item
The child files cannot be compiled on their own,
they can only be compiled via the main file.
A naive editing environment
(such as a text editor with an option
to have the current file processed by \LaTeX)
may require one to switch to the main file before compiling;
attempting to compile the child file produces errors.
\item
The main file must be modified (each time)
to adjust the |\includeonly| command
to the present needs. This easily leaves the main file in a messy state.
\item
The generated document will always carry the filename
of the main document. This is inconvenient if
several child files are to be compiled and
to be kept for distribution.
\end{itemize}

The present package provides a simple interface
to make child files individually compilable by \LaTeX{}.
Compiling a child file then has the same effect as compiling
the main file with an |\includeonly| command
to select the appropriate child.
Moreover the generated document will carry the name of the child
rather than the main file.
This resolves all three above issues.

This feature is meant to make the editing of books,
thesis documents and lecture notes somewhat more convenient.
However, the package can also be used efficiently for
composing a series of documents (such as exercise sheets)
which are typically distributed individually.
It then assists the author in generating the individual documents
(potentially in different versions)
as well as a document containing the collected series.
Another application is in developing style files
or other kinds of included material
where compilation of the style file could redirect
to a sample or test file.

%%%%%%%%%%%%%%%%%%%%%%%%%%%%%%%%%%%%%%%%%%%%%%%%%%%%%%%%%%%%%%%%%%%%%%%%%%%%%%%%
%%%%%%%%%%%%%%%%%%%%%%%%%%%%%%%%%%%%%%%%%%%%%%%%%%%%%%%%%%%%%%%%%%%%%%%%%%%%%%%%
\section{Usage}

First of all, the package \textsf{childdoc} is \emph{not} a standard
\LaTeXe{} |.sty| style file! Therefore it needs to be invoked in
a non-standard way.

%%%%%%%%%%%%%%%%%%%%%%%%%%%%%%%%%%%%%%%%%%%%%%%%%%%%%%%%%%%%%%%%%%%%%%%%%%%%%%%%
\subsection{Included Files}
\label{sec:include}

%%%%%%%%%%%%%%%%%%%%%%%%%%%%%%%%%%%%%%%%
\DescribeMacro{\childdocmain}
To use the package, add the commands
\begin{center}
\begin{tabular}{l}
|\input{childdoc.def}|\\
|\childdocmain{}|\\
\end{tabular}
\end{center}
at the very top of the main \LaTeX{} file,
in particular \emph{before} the |\documentclass| statement!
The argument of |\childdocmain| should be left empty
(but it must be present).

%%%%%%%%%%%%%%%%%%%%%%%%%%%%%%%%%%%%%%%%
\DescribeMacro{\childdocof}
Furthermore, add the commands
\begin{center}
\begin{tabular}{l}
|\input{childdoc.def}|\\
|\childdocof{|\textit{main}|}|\\
\end{tabular}
\end{center}
at the top of every child file \textit{child}
which is included by |\include{|\textit{child}|}|
from within the main file
(or at least for those files to be compiled individually).
The argument \textit{main} must be the filename of the main file.

There are a couple of
considerations in setting up the main and child documents:

%%%%%%%%%%%%%%%%%%%%%%%%%%%%%%%%%%%%%%%%
\paragraph{Restrictions.}

Please note the following restrictions:
\begin{itemize}
\item
|\childdocmain| must be called with one argument \textit{main}
to ensure compatibility with earlier version of the package.
It must either be empty (|\childdocmain{}|)
or precisely match the filename of the main file in which it is specified.
See \secref{sec:detection} for further information.
\item
The filename \textit{main} must be specified without the |.tex| extension.
\item
The filename \textit{main} is case sensitive
(even in case-insensitive file systems)
due to internal string comparison.
\item
The argument \textit{main} should be fully expanded, it cannot be a macro.
\item
Subdirectories and special characters should be avoided in filenames.
\item
The command |\childdocmain{|\textit{main}|}| must be followed by a whitespace.
It should not be followed immediately by another command
or by a comment mark `|%|'.
This is because the \TeX{} parser reads the token immediately following
the argument of |\childdocmain| and puts it
at the beginning of every child section;
however, a white\-space is ignored.
\end{itemize}

%%%%%%%%%%%%%%%%%%%%%%%%%%%%%%%%%%%%%%%%
\paragraph{Content of Main File.}

It is advisable to place all content in the child files included by |\include|.
Any output contained in the main file will appear in all child documents
unless suppressed manually;
it cannot be suppressed automatically by the |\includeonly| directive
and thus should normally be avoided.
A method to include some content in the main file
by means of conditional processing is described in \secref{sec:conditional}.

%%%%%%%%%%%%%%%%%%%%%%%%%%%%%%%%%%%%%%%%
\paragraph{Page Numbering.}

When only a part of the document is compiled,
the appropriate numbering of pages
(as well as other status parameters)
is determined from the |.aux| files.
The latter contain information from previous passes.
However this information needs to propagate through
all intermediate child documents.
Therefore the page numbering in child documents may well
be inconsistent until the complete document is compiled at least once.

A useful (if unconventional) way to always ensure a consistent
page numbering is to restart the numbering in each child document
and denote the pages by `\textit{child}|.|\textit{page}'
where \textit{child} represents the chapter/section number of the child file.
This can be achieved by the command
|\numberwithin{page}{|\textit{child}|}|
of the \textsf{amsmath} package
where \textit{child} can be |chapter| or |section|
depending on the chosen structuring.
Alternatively, one can modify the macro |\thepage| appropriately
and reset the counter |page| at the start of each child file.

%%%%%%%%%%%%%%%%%%%%%%%%%%%%%%%%%%%%%%%%%%%%%%%%%%%%%%%%%%%%%%%%%%%%%%%%%%%%%%%%
\subsection{Conditional Processing}
\label{sec:conditional}

The package provides a mechanism to compile different versions
of a document. To customise the versions further some conditional processing
can come in handy to distinguish which version is being compiled.
The package provides two macros to describe the compilation context:

%%%%%%%%%%%%%%%%%%%%%%%%%%%%%%%%%%%%%%%%
\DescribeMacro{\ifchilddoc}
The conditional |\ifchilddoc| distinguishes between the compilation of
child documents and the main document:
%
\begin{center}
|\ifchilddoc |\textit{child-code}| |[|\||else |\textit{main-code}]| \||fi|
\end{center}

%%%%%%%%%%%%%%%%%%%%%%%%%%%%%%%%%%%%%%%%
\DescribeMacro{\childdocname}
\DescribeMacro{\childdocjob}
The macro |\childdocname| contains the filename (without extension)
of the main or child file being processed.
Note that |\childdocjob| will always contain the name of the main file.

%%%%%%%%%%%%%%%%%%%%%%%%%%%%%%%%%%%%%%%%
\paragraph{Title Page.}

Conditional processing can be used to include a title or banner page
in the main document when proper precautions are taken.
Importantly, the code in the main file should ensure that the page counter
(as well as other status parameters which are stored in the |.aux| files)
takes the same value after the conditional processing.
Otherwise the page numbers may take divergent values
depending on which part is compiled.

For example, a title page could be declared by:
%
\begin{center}
\begin{tabular}{l}
|\ifchilddoc\||else|\\
|\addtocounter{page}{-1}|\\
\textit{code for title page}\\
|\newpage|\\
|\||fi|
\end{tabular}
\end{center}
%
A banner page for the child documents can be generated by:
%
\begin{center}
\begin{tabular}{l}
|\ifchilddoc|\\
|\addtocounter{page}{-1}|\\
\textit{code for banner page}\\
|\newpage|\\
|\||fi|
\end{tabular}
\end{center}
%
Here one could write a message such as:
\begin{center}
|This is the part \childdocname{} of \childdocjob{}.|
\end{center}

%%%%%%%%%%%%%%%%%%%%%%%%%%%%%%%%%%%%%%%%%%%%%%%%%%%%%%%%%%%%%%%%%%%%%%%%%%%%%%%%
\subsection{Flags}
\label{sec:flags}

The package makes it easy to generate different versions
of the main or child documents.
To this end compilation flags can be defined
and assigned different default values.
They will be particularly useful in conjunction
with the forwarding mechanism described in \secref{sec:forward}.

For example, it may be useful to have a flag |\version|
which can be set to |draft| or |final|.
The document source will contain some conditional code
depending on the value of |\version|.
Suppose further, the flag should default to |final| for the main file
and to |draft| for child files
which is a natural assignment for editing the document.
This is achieved by placing the following code
in the preamble of the main document
(below the |\childdocmain| directive):
%
\begin{center}
\begin{tabular}{l}
|\ifchilddoc|\\
|\providecommand{\version}{draft}|\\
|\||else|\\
|\providecommand{\version}{final}|\\
|\||fi|
\end{tabular}
\end{center}
%
The definition by |\providecommand| makes sure
that previous definitions are not overwritten.
Further statements |\providecommand{\version}{...}|
can thus be added before the above code to override it.

For the main file, one might add a line
(between |\childdocmain| and the above block)
%
\begin{center}
|%\ifchilddoc\||else\providecommand{\version}{draft}\||fi|
\end{center}
%
which can be uncommented to produce a draft version.
Likewise one can add a line to the very top of a child file
(above the |\childdocof{|\textit{main}|}| directive)
%
\begin{center}
|%\providecommand{\version}{final}|
\end{center}
%
which can be uncommented to produce the final version of this child document.

%%%%%%%%%%%%%%%%%%%%%%%%%%%%%%%%%%%%%%%%%%%%%%%%%%%%%%%%%%%%%%%%%%%%%%%%%%%%%%%%
\subsection{Forwarding}
\label{sec:forward}

Different versions of the main or child documents
using compilation flags as described in \secref{sec:flags}
can be (permanently) stored in different files
for convenient compilation, viewing and distribution.
To this end, the package defines a command
to pass on compilation to a different file:

%%%%%%%%%%%%%%%%%%%%%%%%%%%%%%%%%%%%%%%%
\DescribeMacro{\childdocforward}
The command |\childdocforward| redirects processing to
another source file:
%
\begin{center}
\begin{tabular}{l}
|\input{childdoc.def}|\\
|\childdocforward[|\textit{main}|]{|\textit{dest}|}|\\
\end{tabular}
\end{center}
%
The argument \textit{dest} is the destination file
(without extension).
It should be the main file or one of the child files.
Note that further \textsf{childdoc} directives
such as |\childdocof| and |\childdocforward|
in the indicated file will be processed in this form.
The optional argument \textit{main}
passes on directly to the main file \textit{main}
while pretending to compile the child \textit{dest}.
This form behaves as if \textit{dest}
issues |\childdocof{|\textit{main}|}| right away,
and no further \textsf{childdoc} directives will be processed.

%%%%%%%%%%%%%%%%%%%%%%%%%%%%%%%%%%%%%%%%
\DescribeMacro{\...prefix}
In the alternative form |\childdocforwardprefix|,
%
\begin{center}
\begin{tabular}{l}
|\input{childdoc.def}|\\
|\childdocforwardprefix[|\textit{main}|]{|\textit{prefix}|}{|\textit{dest}|}|
\end{tabular}
\end{center}
%
the destination file is determined by a pattern
depending on the current file:
To make this work, the current file must be called
`{\textit{prefix}\hspace{0.2em}\textit{suffix}}'
with \textit{prefix} matching precisely the argument.
Processing is then passed on to the file
`{\textit{dest}\hspace{0.2em}\textit{suffix}}'.
Surely, the same effect is achieved by
directly specifying the
argument `{\textit{dest}\hspace{0.2em}\textit{suffix}}'
in the first form.
However, that requires to set up a different file
for each child. With the alternative form of the command
all these files can have exactly the same content
which simplifies setting them up and maintaining them.

For example, the following file |draft.tex|
with a compilation flag |\version| as described in \secref{sec:flags}
compiles the main document as a draft:
%
\begin{center}
\begin{tabular}{l}
|\def\version{draft}|\\
|\input{childdoc.def}|\\
|\childdocforward{|\textit{main}|}|
\end{tabular}
\end{center}
%
Likewise, the following files |final|\textit{nn}|.tex|
compile the final version of the child document
|child|\textit{nn}|.tex|:
%
\begin{center}
\begin{tabular}{l}
|\def\version{final}|\\
|\input{childdoc.def}|\\
|\childdocforwardprefix{final}{child}|
\end{tabular}
\end{center}
%

Note that when several versions of a main file and/or of each child file
are to be generated, it may be convenient to set up a |Makefile| or
shell script to automatise the process.

%%%%%%%%%%%%%%%%%%%%%%%%%%%%%%%%%%%%%%%%%%%%%%%%%%%%%%%%%%%%%%%%%%%%%%%%%%%%%%%%
\subsection{Command Line Processing}
\label{sec:commandline}

The effect of redirection files can also be achieved by invoking
the \LaTeX{} compiler with a more elaborate command line.
Most conveniently this should be done as part
of a shell script or a |Makefile|.

When using \textsf{childdoc} in the main file, the following
command lines effectively perform a redirection
(note that depending on the shell being used,
backslashes may have to be doubled: `|\|' $\to$ `|\\|'):
%
\begin{center}
|... -jobname "|\textit{target}|" |\\|"|[\textit{flags}]%
|\input{childdoc.def}\childdocforward[|\textit{main}|]{|\textit{dest}|}"|
\end{center}
%
Here \textit{target} is the name of the output file,
\textit{main} is the name of the main file
and \textit{dest} is the name of the main or child file to be processed
(all filenames without extensions).
The optional argument \textit{main} can be omitted
if \textit{main} matches \textit{dest}.
Optionally, compilation \textit{flags} can be defined via |\def| commands.
This command line makes the \TeX{} engine believe
it is compiling the file \textit{target}
whose content is specified as the latter parameter.
The provided code then forwards the processing to
\textit{main} or \textit{dest} as described in \secref{sec:forward}.

%%%%%%%%%%%%%%%%%%%%%%%%%%%%%%%%%%%%%%%%%%%%%%%%%%%%%%%%%%%%%%%%%%%%%%%%%%%%%%%%
\subsection{Include by Input}
\label{sec:input}

Including child documents by |\include| has some restrictions by design.
Most notably, the content of a child document always occupies
its own set of pages; pages cannot be shared between child documents.
Usually, this behaviour makes perfect sense
because each child document contain an essential part of the document.
However, in some situations it may be desirable to compose
a document from a collection of parts
without having mandatory page breaks between then.
For this case, the package
provides a mechanism to include parts
by |\input| which can also be processed individually.
However, by construction this mechanism
requires manual handling of the content to be output.

%%%%%%%%%%%%%%%%%%%%%%%%%%%%%%%%%%%%%%%%
\DescribeMacro{\ifchilddocmanual}
The main file should be prepared as usual, see \secref{sec:include}.
However, the document body must make a distinction
between processing of an individual part and of the main document, e.g.:
%
\begin{center}
\begin{tabular}{l}
|\ifchilddocmanual|\\
|\input{\childdocname}|\\
|\||else|\\
\textit{document body with }|\input{|\textit{part}|}|\\
|\||fi|
\end{tabular}
\end{center}
%
The conditional |\ifchilddocmanual| is true whenever
a part to be included by |\input| is being compiled,
and the name of the part is stored in |\childdocname|.

%%%%%%%%%%%%%%%%%%%%%%%%%%%%%%%%%%%%%%%%
\DescribeMacro{\childdocby}
Each part to be included by |\input| should start with:
%
\begin{center}
\begin{tabular}{l}
|\input{childdoc.def}|\\
|\childdocby{|\textit{main}|}|\\
\end{tabular}
\end{center}
%
The directive |\childdocby| is similar to |\childdocof|
described in \secref{sec:include},
but the subsequent selection of content must be done manually.
To that end, both |\ifchilddoc| and |\ifchilddocmanual|
will be true upon processing of a part,
and the name of the part is stored in |\childdocname|.
Note that |\jobname| will be set to the filename of the current part
so that each part receives an individual |.aux| file
that does not interfere with the |.aux| file(s) of the main document.
This behaviour can be altered by the alternative form
|\childdocby[*]{|\textit{main}|}| (with a non-empty optional argument)
which uses the |.aux| file of the main document
by setting |\jobname| to \textit{main}.

%%%%%%%%%%%%%%%%%%%%%%%%%%%%%%%%%%%%%%%%%%%%%%%%%%%%%%%%%%%%%%%%%%%%%%%%%%%%%%%%
\subsection{Driver Development}
\label{sec:driver}

The \textsf{childdoc} mechanism can also be use for the development
of definition files such as \LaTeX{} styles or classes.
This case differs from the above setup with multiple parts
included by |\include| in that no |\includeonly| should be invoked.
This can be achieved by starting the include file
(before |\ProvidesPackage|) with:
%
\begin{center}
\begin{tabular}{l}
|\input{childdoc.def}|\\
|\childdocforward{|\textit{main}|}|\\
\end{tabular}
\end{center}
%
or alternatively with:
%
\begin{center}
\begin{tabular}{l}
|\input{childdoc.def}|\\
|\childdocby{|\textit{main}|}|\\
\end{tabular}
\end{center}
%
Both forms have slightly different effects as described above.
The main file is prepared as usual, see \secref{sec:include}.

%%%%%%%%%%%%%%%%%%%%%%%%%%%%%%%%%%%%%%%%%%%%%%%%%%%%%%%%%%%%%%%%%%%%%%%%%%%%%%%%
\subsection{Legacy Detection}
\label{sec:detection}

The directive |\childdocmain| in the main file can detect
whether the complete document or merely a child is to be compiled
even without using the directive |\childdocof|.
This method is deprecated because it is less robust
and there is no compelling reason to use it;
it is merely provided for backward compatibility
and it may be removed in future versions.

If the detection mechanism is to be used,
it is mandatory to correctly specify
the filename of the main file as the argument of |\childdocmain|:
%
\begin{center}
\begin{tabular}{l}
|\input{childdoc.def}|\\
|\childdocmain{|\textit{main}|}|\\
\end{tabular}
\end{center}
%
If |\jobname| does not match the argument \textit{main} of |\childdocmain|,
it is assumed that |\jobname| points to the child file to be compiled.
When using |\childdocmain| with the main file specified as argument,
it suffices to start a child file
with just |\input{|\textit{main}|}|
without loading of the package and using |\childdocof|.
If instead all processing is done
with the appropriate \textsf{childdoc} directives,
the argument of \textit{main} of |\childdocmain| can be empty.

An alternative version of the command line processing described
in \secref{sec:commandline} using the detection mechanism reads:
%
\begin{center}
|... -jobname "|\textit{target}|" "|[\textit{flags}]%
[|\def\jobname{|\textit{dest}|}|]|\input{|\textit{main}|}"|
\end{center}

%%%%%%%%%%%%%%%%%%%%%%%%%%%%%%%%%%%%%%%%%%%%%%%%%%%%%%%%%%%%%%%%%%%%%%%%%%%%%%%%
\subsection{Manual Code}
\label{sec:manual}

In case one cannot be certain whether the definitions file |childdoc.def|
is installed on the target \TeX{} distribution
and one prefers not to ship it,
it is conceivable to paste a few relevant commands into the sources.

To that end, drop all statements |\input{childdoc.def}|
and perform the replacements as outlined below.
Instead of |\childdocmain{|\textit{main}|}| add the following code
to the top of the main file:
%
\begin{center}
\begin{tabular}{l}
|\||ifdefined\childdocname\endinput\||fi\newif\ifchilddoc|\\
|\edef\childdocname{\scantokens\expandafter{\jobname\noexpand}}|\\
|\def\childdocmain{|\textit{main}|}\||ifx\childdocmain\childdocname\||else|\\
|\childdoctrue\includeonly{\childdocname}\let\jobname\childdocmain\||fi|\\
\end{tabular}
\end{center}
%
Instead of |\childdocof{|\textit{main}|}| just include the main file
at the top of each child file:
%
\begin{center}
|\input{|\textit{main}|}|
\end{center}
%
A simple redirection |\childdocforward{|\textit{dest}|}| is achieved by:
%
\begin{center}
|\def\jobname{|\textit{dest}|}\input{\jobname}|
\end{center}
%
The redirection with prefix
|\childdocforwardprefix[|\textit{prefix}|]{|\textit{dest}|}|
is accomplished by:
%
\begin{center}
\begin{tabular}{l}
|{\edef\jobname{\scantokens\expandafter{\jobname\noexpand}}|\\
|\def\redirectjob |\textit{prefix}|#1~~~{\gdef\jobname{|\textit{dest}|#1}}|\\
|\expandafter\redirectjob\jobname~~~}\input{\jobname}|
\end{tabular}
\end{center}

In an alternative approach,
child documents can be compiled by a specific command line
without additional code or specific definitions:
%
\begin{center}
|... -jobname "|\textit{target}|" "|[\textit{flags}]%
|\includeonly{|\textit{dest}|}\input{|\textit{main}|}"|
\end{center}
%

%%%%%%%%%%%%%%%%%%%%%%%%%%%%%%%%%%%%%%%%%%%%%%%%%%%%%%%%%%%%%%%%%%%%%%%%%%%%%%%%
%%%%%%%%%%%%%%%%%%%%%%%%%%%%%%%%%%%%%%%%%%%%%%%%%%%%%%%%%%%%%%%%%%%%%%%%%%%%%%%%
\section{Information}

%%%%%%%%%%%%%%%%%%%%%%%%%%%%%%%%%%%%%%%%%%%%%%%%%%%%%%%%%%%%%%%%%%%%%%%%%%%%%%%%
\subsection{Copyright}

Copyright \copyright{} 2017--2018 Niklas Beisert

This work may be distributed and/or modified under the
conditions of the \LaTeX{} Project Public License, either version 1.3
of this license or (at your option) any later version.
The latest version of this license is in
  \url{http://www.latex-project.org/lppl.txt}
and version 1.3 or later is part of all distributions of \LaTeX{}
version 2005/12/01 or later.

This work has the LPPL maintenance status `maintained'.

The Current Maintainer of this work is Niklas Beisert.

This work consists of the files |README.txt|, |childdoc.ins| and |childdoc.dtx|
as well as the derived files |childdoc.def|, |cdocsamp.tex|
with |cdocsch1.tex|, |cdocsch2.tex|, |cdocspt3.tex|, |cdocspt4.tex|,
|cdocsdrf.tex|, |cdocsfn1.tex|, |cdocsfn2.tex|
as well as |childdoc.pdf|.

%%%%%%%%%%%%%%%%%%%%%%%%%%%%%%%%%%%%%%%%%%%%%%%%%%%%%%%%%%%%%%%%%%%%%%%%%%%%%%%%
\subsection{Files and Installation}

The package consists of the files:
%
\begin{center}
\begin{tabular}{ll}
    |README.txt|   & readme file \\
    |childdoc.ins| & installation file \\
    |childdoc.dtx| & source file \\
    |childdoc.def| & definition file \\
    |cdocsamp.tex| & sample main file \\
    |cdocsch1.tex| & sample include file \\
    |cdocsch2.tex| & sample include file \\
    |cdocspt3.tex| & sample part file \\
    |cdocspt4.tex| & sample part file \\
    |cdocsdrf.tex| & sample redirection file \\
    |cdocsfn1.tex| & sample redirection file \\
    |cdocsfn2.tex| & sample redirection file \\
    |childdoc.pdf| & manual
\end{tabular}
\end{center}
%
The distribution consists of the files
|README.txt|, |childdoc.ins| and |childdoc.dtx|.
%
\begin{itemize}
\item
Run (pdf)\LaTeX{} on |childdoc.dtx|
to compile the manual |childdoc.pdf| (this file).
\item
Run \LaTeX{} on |childdoc.ins| to create the definitions file |childdoc.def|
and the sample |cdocsamp.tex| with include files
|cdocsch1.tex|, |cdocsch2.tex|, |cdocspt3.tex|, |cdocspt4.tex|,
|cdocsdrf.tex|, |cdocsfn1.tex|, |cdocsfn2.tex|.
Then copy the file |childdoc.def| to an appropriate directory of your \LaTeX{}
distribution, e.g.\ \textit{texmf-root}|/tex/latex/childdoc|.
\end{itemize}

%%%%%%%%%%%%%%%%%%%%%%%%%%%%%%%%%%%%%%%%%%%%%%%%%%%%%%%%%%%%%%%%%%%%%%%%%%%%%%%%
\subsection{Related CTAN Packages}

There are several other packages which offer a similar functionality:
%
\begin{itemize}
\item
The packages
\href{http://ctan.org/pkg/docmute}{\textsf{docmute}},
\href{http://ctan.org/pkg/includex}{\textsf{includex}} and
\href{http://ctan.org/pkg/standalone}{\textsf{standalone}}
provide commands to include only the document body of
a child file thus allowing both files to be compiled individually.
\item
The packages \href{http://ctan.org/pkg/subdocs}{\textsf{subdocs}}
and \href{http://ctan.org/pkg/subfiles}{\textsf{subfiles}}
provide structures in which the main and child documents can be
encapsulated and allowing them to be compiled individually.
The inclusion mechanism is different from the conventional |\include|.
\item
The package \href{http://ctan.org/pkg/combine}{\textsf{combine}}
is an elaborate solution to combine several documents into one.
\end{itemize}
%
See also the CTAN topic \href{http://ctan.org/topic/subdocs}{\textsf{subdocs}}
for further related packages.
The present package differs from the above solutions in that
a document structure constructed with the conventional |\include| mechanism
just needs two extra commands at the top of every file
such that all constituent files can be compiled individually.

%%%%%%%%%%%%%%%%%%%%%%%%%%%%%%%%%%%%%%%%%%%%%%%%%%%%%%%%%%%%%%%%%%%%%%%%%%%%%%%%
%\subsection{Feature Suggestions}
%
%The following is a list of features which may be useful for future
%versions of this package:
%%
%\begin{itemize}
%\item
%\ldots
%\end{itemize}

%%%%%%%%%%%%%%%%%%%%%%%%%%%%%%%%%%%%%%%%%%%%%%%%%%%%%%%%%%%%%%%%%%%%%%%%%%%%%%%%
\subsection{Revision History}

%%%%%%%%%%%%%%%%%%%%%%%%%%%%%%%%%%%%%%%%
\paragraph{v2.0:} 2018/12/30

\begin{itemize}
\item
immediate forward processing
\item
added |\childdocby| mechanism
\item
manual restructured
\end{itemize}

%%%%%%%%%%%%%%%%%%%%%%%%%%%%%%%%%%%%%%%%
\paragraph{v1.6:} 2018/01/17

\begin{itemize}
\item
application for development of include files
\item
corrections to manual
\end{itemize}

%%%%%%%%%%%%%%%%%%%%%%%%%%%%%%%%%%%%%%%%
\paragraph{v1.5:} 2017/05/21

\begin{itemize}
\item
more complete structuring introduced
\item
|\childdocof| introduced
\item
|\childdoc| renamed to |\childdocmain|
\item
|\childredirect| renamed to |\childdocforward| and |\childdocforwardprefix|
and functionality expanded
\end{itemize}

%%%%%%%%%%%%%%%%%%%%%%%%%%%%%%%%%%%%%%%%
\paragraph{v1.0:} 2017/04/27

\begin{itemize}
\item
manual and install package
\item
first version published on CTAN
\end{itemize}

%%%%%%%%%%%%%%%%%%%%%%%%%%%%%%%%%%%%%%%%
\paragraph{v0.6:} 2017/04/26

\begin{itemize}
\item
redirection mechanism added
\end{itemize}

%%%%%%%%%%%%%%%%%%%%%%%%%%%%%%%%%%%%%%%%
\paragraph{v0.5:} 2017/04/26

\begin{itemize}
\item
functionality in definition file
\end{itemize}


%%%%%%%%%%%%%%%%%%%%%%%%%%%%%%%%%%%%%%%%%%%%%%%%%%%%%%%%%%%%%%%%%%%%%%%%%%%%%%%%
%%%%%%%%%%%%%%%%%%%%%%%%%%%%%%%%%%%%%%%%%%%%%%%%%%%%%%%%%%%%%%%%%%%%%%%%%%%%%%%%
%%%%%%%%%%%%%%%%%%%%%%%%%%%%%%%%%%%%%%%%%%%%%%%%%%%%%%%%%%%%%%%%%%%%%%%%%%%%%%%%
\appendix

\settowidth\MacroIndent{\rmfamily\scriptsize 000\ }

 \DocInput{childdoc.dtx}

\end{document}
%</driver>
% \fi
%
% %%%%%%%%%%%%%%%%%%%%%%%%%%%%%%%%%%%%%%%%%%%%%%%%%%%%%%%%%%%%%%%%%%%%%%%%%%%%%%
% %%%%%%%%%%%%%%%%%%%%%%%%%%%%%%%%%%%%%%%%%%%%%%%%%%%%%%%%%%%%%%%%%%%%%%%%%%%%%%
% \section{Sample}
%\iffalse
%<*samplemain>
%\fi
%
% The following presents a sample document
% with two chapters, two parts, a title page,
% a compile flag as well as three forwarding files to set the flag.
% It consists of eight |.tex| files:
% \begin{center}
% \begin{tabular}{ll}
% |cdocsamp.tex|&main file\\
% |cdocsch1.tex|&include file for chapter 1\\
% |cdocsch2.tex|&include file for chapter 2\\
% |cdocspt3.tex|&include file for part 3\\
% |cdocspt4.tex|&include file for part 4\\
% |cdocsdrf.tex|&forwarding file for main file in draft mode\\
% |cdocsfi1.tex|&forwarding file for final version of chapter 1\\
% |cdocsfi2.tex|&forwarding file for final version of chapter 2\\
% \end{tabular}
% \end{center}
% Each of the eight files can be compiled directly by the \LaTeX{} compiler.
%
% %%%%%%%%%%%%%%%%%%%%%%%%%%%%%%%%%%%%%%
% \paragraph{Main File.}
%
% The main file is called |cdocsamp.tex|.
%
% Load the \textsf{childdoc} definitions and
% declare the filename for the main document:
%    \begin{macrocode}
\input{childdoc.def}
\childdocmain{}
%    \end{macrocode}

% Optional override for |\version| flag:
%    \begin{macrocode}
%%\ifchilddoc\else\providecommand{\version}{draft}\fi
%    \end{macrocode}

% Define the default values for the |\version| flag
% (|final| for the main file and |draft| for childs):
%    \begin{macrocode}
\ifchilddoc
\providecommand{\version}{draft}
\else
\providecommand{\version}{final}
\fi
%    \end{macrocode}

% Load the standard document class:
%    \begin{macrocode}
\documentclass[12pt]{article}
%    \end{macrocode}

% Start the document body:
%    \begin{macrocode}
\begin{document}
%    \end{macrocode}

% Declare a title page.
% Print title, part of document being processed and version flag:
%    \begin{macrocode}
\addtocounter{page}{-1}
\begin{center}
{\LARGE\bfseries{}childdoc example\par}
\vspace{1cm}
\ifchilddoc
\ifchilddocmanual part\else chapter\fi:
`\childdocname' of `\childdocjob'\par
\else
main document: `\childdocjob'\par
\fi
version: \version\par
\end{center}
\newpage
%    \end{macrocode}

% Manually include selected file,
% otherwise process as usual:
%    \begin{macrocode}
\ifchilddocmanual
\section*{part `\childdocname'}
\input{\childdocname}
\else
%    \end{macrocode}

% Include the two chapters:
%    \begin{macrocode}
\include{cdocsch1}
\include{cdocsch2}
%    \end{macrocode}

% Include the two parts unless only chapters should be displayed:
%    \begin{macrocode}
\ifchilddoc\else
\section{part three}
\input{cdocspt3}
\section{part four}
\input{cdocspt4}
\fi
%    \end{macrocode}

% Process as usual until here:
%    \begin{macrocode}
\fi
%    \end{macrocode}

% End of document body:
%    \begin{macrocode}
\end{document}
%    \end{macrocode}
%\iffalse
%</samplemain>
%\fi
%
% %%%%%%%%%%%%%%%%%%%%%%%%%%%%%%%%%%%%%%
% \paragraph{Chapter Include Files.}
%
% The include files are called |cdocsch1.tex| and |cdocsch2.tex|.
%
%\iffalse
%<*samplechap1|samplechap2>
%\fi

% Optional override for |\version| flag:
%    \begin{macrocode}
%%\providecommand{\version}{final}
%    \end{macrocode}

% Include the main document:
%    \begin{macrocode}
\input{childdoc.def}
\childdocof{cdocsamp}
%    \end{macrocode}

%\iffalse
%</samplechap1|samplechap2>
%\fi
%
%\iffalse
%<*samplechap1>
%\fi
% Some text for chapter 1:
%    \begin{macrocode}
\section{one}
some text in chapter one
%    \end{macrocode}

%\iffalse
%</samplechap1>
%\fi
% Some text for chapter 2:
%\iffalse
%<*samplechap2>
%\fi
%    \begin{macrocode}
\section{two}
more text in chapter two
%    \end{macrocode}

%\iffalse
%</samplechap2>
%\fi
%
% %%%%%%%%%%%%%%%%%%%%%%%%%%%%%%%%%%%%%%
% \paragraph{Part Include Files.}
%
% The include files are called |cdocspt3.tex| and |cdocspt4.tex|.
%
%\iffalse
%<*samplepart3|samplepart4>
%\fi

% Optional override for |\version| flag:
%    \begin{macrocode}
%%\providecommand{\version}{final}
%    \end{macrocode}

% Include the main document:
%    \begin{macrocode}
\input{childdoc.def}
\childdocby{cdocsamp}
%    \end{macrocode}

%\iffalse
%</samplepart3|samplepart4>
%\fi
%
%\iffalse
%<*samplepart3>
%\fi
% Some text for part 3:
%    \begin{macrocode}
some text in part three
%    \end{macrocode}

%\iffalse
%</samplepart3>
%\fi
% Some text for part 4:
%\iffalse
%<*samplepart4>
%\fi
%    \begin{macrocode}
more text in part four
%    \end{macrocode}

%\iffalse
%</samplepart4>
%\fi
%
% %%%%%%%%%%%%%%%%%%%%%%%%%%%%%%%%%%%%%%
% \paragraph{Forwarding for a Complete Draft.}
%
% The following forwarding file |cdocsdrf.tex|
% compiles the main document in draft mode:
%\iffalse
%<*sampledraft>
%\fi
%    \begin{macrocode}
\def\version{draft}
\input{childdoc.def}
\childdocforward{cdocsamp}
%    \end{macrocode}

%\iffalse
%</sampledraft>
%\fi
%
% %%%%%%%%%%%%%%%%%%%%%%%%%%%%%%%%%%%%%%
% \paragraph{Forwarding for Final Version of the Chapters.}
%
% The following forwarding files |cdocsfn1.tex| and |cdocsfn2.tex|
% (with identical content)
% compile the final versions of the child documents
% |cdocsch1.tex| and |cdocsch2.tex|, respectively:
%\iffalse
%<*samplefinal>
%\fi
%    \begin{macrocode}
\def\version{final}
\input{childdoc.def}
\childdocforwardprefix[cdocsamp]{cdocsfn}{cdocsch}
%    \end{macrocode}

%\iffalse
%</samplefinal>
%\fi
%
% %%%%%%%%%%%%%%%%%%%%%%%%%%%%%%%%%%%%%%
% \paragraph{Command Line Processing.}
%
% The following three command lines generate the output files
% |cdocscld|, |cdocscl1| and |cdocscl2|
% which should be identical to
% |cdocsdrf|, |cdocsch1| and |cdocsfn2|, respectively:
% \begin{center}
% \begin{tabular}{l}
% |latex -jobname cdocscld \|\\
% |  "\def\version{draft}\input{childdoc.def}\childdocforward{cdocsamp}"|\\
% |latex -jobname cdocscl1 \|\\
% |  "\input{childdoc.def}\childdocforward[cdocsamp]{cdocsch1}"|\\
% |latex -jobname cdocscl2 \|\\
% |  "\def\version{final}\input{childdoc.def}\childdocforward{cdocsch2}"|
% \end{tabular}
% \end{center}
% Note that the trailing backslash on each first line
% merely continues the input to the second line
% (for convenient cut ant paste).
% Furthermore, the command |latex| can be replaced by any
% of its alternative versions such as |pdflatex|.
%
% %%%%%%%%%%%%%%%%%%%%%%%%%%%%%%%%%%%%%%%%%%%%%%%%%%%%%%%%%%%%%%%%%%%%%%%%%%%%%%
% %%%%%%%%%%%%%%%%%%%%%%%%%%%%%%%%%%%%%%%%%%%%%%%%%%%%%%%%%%%%%%%%%%%%%%%%%%%%%%
% \section{Implementation}
%\iffalse
%<*package>
%\fi
%
% This section describes the definitions file |childdoc.def|.

% The definitions cannot be loaded using |\usepackage| or |\RequirePackage|
% which has a mechanism to prevent loading a style file more than once.
% When loading the definitions by means of |\input|
% multiple instances have to be prevented manually:
%\iffalse
%This code needs to be before the `\ProvidesFile' directive
%which is defined at the beginning of this file.
%Therefore it is also placed there and commented out here.
%</package>
%<*discard>
%\fi
%    \begin{macrocode}
\ifdefined\childdocmain\endinput\fi
%    \end{macrocode}
%\iffalse
%</discard>
%<*package>
%\fi
%
% \macro{\ifchilddoc}
% \macro{\ifchilddocmanual}
% The conditional |\ifchilddoc| tells whether a
% child (true) or main (false) document is being compiled.
% The conditional |\ifchilddocmanual| tells whether
% the |\includeonly| mechanism is used (false) or
% the selection of child files must be performed manually (true).
% The definitions initialise to false:
%    \begin{macrocode}
\newif\ifchilddoc
\newif\ifchilddocmanual
%    \end{macrocode}

% \macro{\childdocname}
% \macro{\childdocjob}
% The macro |\childdocname| stores the name of the main document
% to be compiled. The macro |\childdocjob| stores the name of
% the document on which the \LaTeX{} compiler was originally invoked.
% The content of |\jobname| cannot be compared
% to filenames specified in the source due to different catcodes.
% The following code rescans |\jobname|, stores the result
% in |\childdocname| and saves a copy in |\childdocjob|:
%    \begin{macrocode}
\edef\childdocname{\scantokens\expandafter{\jobname\noexpand}}
\let\childdocjob\childdocname
%    \end{macrocode}

% \macro{\childdocdisable}
% The macro |\childdocdisable| prevents the main file
% from being processed more than once.
% At this stage, the main document command |\childdocmain|
% is assumed to be called once again where it should do nothing.
% Any subsequent call to it should prevent
% a secondary processing of the main document
% It overwrites the forwarding commands
% |\childdocof| and |\childdocforward|
% with empty macros to prevent further inclusions of the main document:
%    \begin{macrocode}
\newcommand{\childdocdisable}
{
  \renewcommand{\childdocmain}[1]{\renewcommand{\childdocmain}[1]{\endinput}}
  \renewcommand{\childdocof}[1]{}
  \renewcommand{\childdocby}[2][]{}
  \renewcommand{\childdocforward}[2][]{}
  \renewcommand{\childdocdisable}{}
}
%    \end{macrocode}

% \macro{\childdocmain}
% The macro |\childdocmain| is to be called at the top of the main file
% with nothing or the main filename (without extension) as argument.
% First, it breaks loops.
% If the argument is not empty and does not match |\childdocname|
% (which is set by the first inclusion of |childdoc.def|),
% |\ifchilddoc| is set to true, |\includeonly| is applied to the child file
% and |\jobname| is set to the main file
% (for proper handling of |.aux| files):
%    \begin{macrocode}
\newcommand{\childdocmain}[1]
{
  \childdocdisable\childdocmain{}
  \if?#1?\else
    \begingroup
      \def\childdoctmp{#1}
      \ifx\childdoctmp\childdocname
        \def\childdoctmp{}
      \else
        \def\childdoctmp
        {
          \childdoctrue
          \includeonly{\childdocname}
          \def\childdocjob{#1}
          \def\jobname{#1}
        }
      \fi
      \expandafter
    \endgroup
    \childdoctmp
  \fi
}
%    \end{macrocode}

% \macro{\childdocof}
% The command |\childdocof| redirects
% compilation to the main file |#1|.
%    \begin{macrocode}
\newcommand{\childdocof}[1]
{
  \childdocdisable
  \childdoctrue
  \includeonly{\childdocname}
  \def\jobname{#1}
  \def\childdocjob{#1}
  \input{#1}
}
%    \end{macrocode}

% \macro{\childdocby}
% The command |\childdocby| ....
%    \begin{macrocode}
\newcommand{\childdocby}[2][]
{
  \childdocdisable
  \childdoctrue
  \childdocmanualtrue
  \if?#1?\else
    \def\jobname{#2}
  \fi
  \def\childdocjob{#2}
  \input{#2}
  \endinput
}
%    \end{macrocode}

% \macro{\childdocforward}
% The command |\childdocforward| redirects
% compilation to the main file or
% (if the optional argument is given) a child file.
% Parameters are set as if the main file
% or a child file starting with |\childdocof| was compiled.
% Then compilation is handed over to the main file:
%    \begin{macrocode}
\newcommand{\childdocforward}[2][]
{
  \begingroup
    \if?#1?
      \def\childdoctmp
      {
        \def\childdocname{#2}
        \def\childdocjob{#2}
        \def\jobname{#2}
        \input{#2}
        \endinput
      }
    \else
      \def\childdoctmp
      {
        \childdocdisable
        \def\childdocname{#2}
        \childdoctrue
        \includeonly{#2}
        \def\childdocjob{#1}
        \def\jobname{#1}
        \input{#1}
        \endinput
      }
    \fi
    \expandafter
  \endgroup
  \childdoctmp
}
%    \end{macrocode}

% \macro{\childdocforwardprefix}
% The command |\childdocforwardprefix| redirects
% compilation to the main or a child file by means of a pattern.
% The prefix |#1| in the current filename is replaced by |#2|
% and the suffix of the current filename is kept
% (it is assumed that the filename does not contain the substring `|~~~|'
% which is used as a delimiter).
% Compilation is handed over to the new file by |\childdocforward|:
%    \begin{macrocode}
\newcommand{\childdocforwardprefix}[3][]
{
  \begingroup
    \def\childdocextract #2##1~~~{\def\childdoctmp{\childdocforward[#1]{#3##1}}}
    \expandafter\childdocextract\childdocname~~~
    \expandafter
  \endgroup
  \childdoctmp
}
%    \end{macrocode}

% \macro{\childdoc}
% The deprecated macro |\childdoc| is a legacy version of |\childdocmain|:
%    \begin{macrocode}
\newcommand{\childdoc}{\childdocmain}
%    \end{macrocode}

% \macro{\childdocredirect}
% The deprecated macro |\childdocredirect| is a legacy version
% of |\childdocforward| and |\childdocforwardprefix|:
%    \begin{macrocode}
\newcommand{\childdocredirect}[2][]
{
  \begingroup
    \if?#1?
      \def\childdoctmp{\childdocforward{#2}}
    \else
      \def\childdoctmp{\childdocforwardprefix{#1}{#2}}
    \fi
    \expandafter
  \endgroup
  \childdoctmp
}
%    \end{macrocode}

%\iffalse
%</package>
%\fi
%
\endinput
|\\
|\childdocforward[|\textit{main}|]{|\textit{dest}|}|\\
\end{tabular}
\end{center}
%
The argument \textit{dest} is the destination file
(without extension).
It should be the main file or one of the child files.
Note that further \textsf{childdoc} directives
such as |\childdocof| and |\childdocforward|
in the indicated file will be processed in this form.
The optional argument \textit{main}
passes on directly to the main file \textit{main}
while pretending to compile the child \textit{dest}.
This form behaves as if \textit{dest}
issues |\childdocof{|\textit{main}|}| right away,
and no further \textsf{childdoc} directives will be processed.

%%%%%%%%%%%%%%%%%%%%%%%%%%%%%%%%%%%%%%%%
\DescribeMacro{\...prefix}
In the alternative form |\childdocforwardprefix|,
%
\begin{center}
\begin{tabular}{l}
|% \iffalse
%
% childdoc.dtx Copyright (C) 2017-2018 Niklas Beisert
%
% This work may be distributed and/or modified under the
% conditions of the LaTeX Project Public License, either version 1.3
% of this license or (at your option) any later version.
% The latest version of this license is in
%   http://www.latex-project.org/lppl.txt
% and version 1.3 or later is part of all distributions of LaTeX
% version 2005/12/01 or later.
%
% This work has the LPPL maintenance status `maintained'.
%
% The Current Maintainer of this work is Niklas Beisert.
%
% This work consists of the files childdoc.dtx and childdoc.ins
% and the derived files childdoc.def and cdocsamp.tex with
% cdocsch1.tex, cdocsch2.tex, cdocsdrf.tex, cdocsfn1.tex, cdocsfn2.tex.
%
%<package>\ifdefined\childdocmain\endinput\fi
%<package>\ProvidesFile{childdoc.def}[2018/12/30 v2.0 child document driver]
%<samplemain>\ProvidesFile{cdocsamp.tex}[2018/12/30 v2.0 sample for childdoc]
%<*driver>
%\ProvidesFile{childdoc.drv}[2018/12/30 v2.0 childdoc reference manual file]
\PassOptionsToClass{10pt,a4paper}{article}
\documentclass{ltxdoc}

\usepackage[margin=35mm]{geometry}
\usepackage{hyperref}
\usepackage{hyperxmp}
\usepackage[usenames]{color}

\hypersetup{colorlinks=true}
\hypersetup{pdfstartview=FitH}
\hypersetup{pdfpagemode=UseNone}
\hypersetup{pdfsource={}}
\hypersetup{pdflang={en-UK}}
\hypersetup{pdfcopyright={Copyright 2017-2018 Niklas Beisert.
  This work may be distributed and/or modified under the
  conditions of the LaTeX Project Public License, either version 1.3
  of this license or (at your option) any later version.}}
\hypersetup{pdflicenseurl={http://www.latex-project.org/lppl.txt}}
\hypersetup{pdfcontactaddress={ETH Zurich, ITP, HIT K,
  Wolfgang-Pauli-Strasse 27}}
\hypersetup{pdfcontactpostcode={8093}}
\hypersetup{pdfcontactcity={Zurich}}
\hypersetup{pdfcontactcountry={Switzerland}}
\hypersetup{pdfcontactemail={nbeisert@itp.phys.ethz.ch}}
\hypersetup{pdfcontacturl={http://people.phys.ethz.ch/\xmptilde nbeisert/}}

\newcommand{\secref}[1]{\hyperref[#1]{section \ref*{#1}}}

\parskip1ex
\parindent0pt
\let\olditemize\itemize
\def\itemize{\olditemize\parskip0pt}

\begin{document}

\title{The \textsf{childdoc} Package}
\hypersetup{pdftitle={The childdoc Package}}
\author{Niklas Beisert\\[2ex]
  Institut f\"ur Theoretische Physik\\
  Eidgen\"ossische Technische Hochschule Z\"urich\\
  Wolfgang-Pauli-Strasse 27, 8093 Z\"urich, Switzerland\\[1ex]
  \href{mailto:nbeisert@itp.phys.ethz.ch}
  {\texttt{nbeisert@itp.phys.ethz.ch}}}
\hypersetup{pdfauthor={Niklas Beisert}}
\hypersetup{pdfsubject={Manual for the LaTeX2e Package childdoc}}
\date{30 December 2018, \textsf{v2.0}}
\maketitle

\begin{abstract}\noindent
\textsf{childdoc} is a \LaTeXe{} package
that enables the direct compilation
of document sections included by |\include|
to individual files.
\end{abstract}

\begingroup
\parskip0ex
\tableofcontents
\endgroup

%%%%%%%%%%%%%%%%%%%%%%%%%%%%%%%%%%%%%%%%%%%%%%%%%%%%%%%%%%%%%%%%%%%%%%%%%%%%%%%%
%%%%%%%%%%%%%%%%%%%%%%%%%%%%%%%%%%%%%%%%%%%%%%%%%%%%%%%%%%%%%%%%%%%%%%%%%%%%%%%%
\section{Introduction}

\LaTeX{} provides a mechanism to structure a large document (such as a book)
into a main file and several child files (containing the chapters)
using the |\include| command.
This mechanism is beneficial for documents
which span hundreds of pages in order to
make the source file(s) more manageable.
Moreover, compilation can be restricted to
selected child files by means of the |\includeonly| command.
The latter feature can be used to reduce the compilation time while editing
(this was significantly more useful in the earlier days of \LaTeX{})
or to generate a smaller document which is easier to navigate.
Another application of |\includeonly| is to generate
documents consisting of selected parts of the complete document.

However, there are a few drawbacks of the plain |\include| mechanism:
\begin{itemize}
\item
The child files cannot be compiled on their own,
they can only be compiled via the main file.
A naive editing environment
(such as a text editor with an option
to have the current file processed by \LaTeX)
may require one to switch to the main file before compiling;
attempting to compile the child file produces errors.
\item
The main file must be modified (each time)
to adjust the |\includeonly| command
to the present needs. This easily leaves the main file in a messy state.
\item
The generated document will always carry the filename
of the main document. This is inconvenient if
several child files are to be compiled and
to be kept for distribution.
\end{itemize}

The present package provides a simple interface
to make child files individually compilable by \LaTeX{}.
Compiling a child file then has the same effect as compiling
the main file with an |\includeonly| command
to select the appropriate child.
Moreover the generated document will carry the name of the child
rather than the main file.
This resolves all three above issues.

This feature is meant to make the editing of books,
thesis documents and lecture notes somewhat more convenient.
However, the package can also be used efficiently for
composing a series of documents (such as exercise sheets)
which are typically distributed individually.
It then assists the author in generating the individual documents
(potentially in different versions)
as well as a document containing the collected series.
Another application is in developing style files
or other kinds of included material
where compilation of the style file could redirect
to a sample or test file.

%%%%%%%%%%%%%%%%%%%%%%%%%%%%%%%%%%%%%%%%%%%%%%%%%%%%%%%%%%%%%%%%%%%%%%%%%%%%%%%%
%%%%%%%%%%%%%%%%%%%%%%%%%%%%%%%%%%%%%%%%%%%%%%%%%%%%%%%%%%%%%%%%%%%%%%%%%%%%%%%%
\section{Usage}

First of all, the package \textsf{childdoc} is \emph{not} a standard
\LaTeXe{} |.sty| style file! Therefore it needs to be invoked in
a non-standard way.

%%%%%%%%%%%%%%%%%%%%%%%%%%%%%%%%%%%%%%%%%%%%%%%%%%%%%%%%%%%%%%%%%%%%%%%%%%%%%%%%
\subsection{Included Files}
\label{sec:include}

%%%%%%%%%%%%%%%%%%%%%%%%%%%%%%%%%%%%%%%%
\DescribeMacro{\childdocmain}
To use the package, add the commands
\begin{center}
\begin{tabular}{l}
|\input{childdoc.def}|\\
|\childdocmain{}|\\
\end{tabular}
\end{center}
at the very top of the main \LaTeX{} file,
in particular \emph{before} the |\documentclass| statement!
The argument of |\childdocmain| should be left empty
(but it must be present).

%%%%%%%%%%%%%%%%%%%%%%%%%%%%%%%%%%%%%%%%
\DescribeMacro{\childdocof}
Furthermore, add the commands
\begin{center}
\begin{tabular}{l}
|\input{childdoc.def}|\\
|\childdocof{|\textit{main}|}|\\
\end{tabular}
\end{center}
at the top of every child file \textit{child}
which is included by |\include{|\textit{child}|}|
from within the main file
(or at least for those files to be compiled individually).
The argument \textit{main} must be the filename of the main file.

There are a couple of
considerations in setting up the main and child documents:

%%%%%%%%%%%%%%%%%%%%%%%%%%%%%%%%%%%%%%%%
\paragraph{Restrictions.}

Please note the following restrictions:
\begin{itemize}
\item
|\childdocmain| must be called with one argument \textit{main}
to ensure compatibility with earlier version of the package.
It must either be empty (|\childdocmain{}|)
or precisely match the filename of the main file in which it is specified.
See \secref{sec:detection} for further information.
\item
The filename \textit{main} must be specified without the |.tex| extension.
\item
The filename \textit{main} is case sensitive
(even in case-insensitive file systems)
due to internal string comparison.
\item
The argument \textit{main} should be fully expanded, it cannot be a macro.
\item
Subdirectories and special characters should be avoided in filenames.
\item
The command |\childdocmain{|\textit{main}|}| must be followed by a whitespace.
It should not be followed immediately by another command
or by a comment mark `|%|'.
This is because the \TeX{} parser reads the token immediately following
the argument of |\childdocmain| and puts it
at the beginning of every child section;
however, a white\-space is ignored.
\end{itemize}

%%%%%%%%%%%%%%%%%%%%%%%%%%%%%%%%%%%%%%%%
\paragraph{Content of Main File.}

It is advisable to place all content in the child files included by |\include|.
Any output contained in the main file will appear in all child documents
unless suppressed manually;
it cannot be suppressed automatically by the |\includeonly| directive
and thus should normally be avoided.
A method to include some content in the main file
by means of conditional processing is described in \secref{sec:conditional}.

%%%%%%%%%%%%%%%%%%%%%%%%%%%%%%%%%%%%%%%%
\paragraph{Page Numbering.}

When only a part of the document is compiled,
the appropriate numbering of pages
(as well as other status parameters)
is determined from the |.aux| files.
The latter contain information from previous passes.
However this information needs to propagate through
all intermediate child documents.
Therefore the page numbering in child documents may well
be inconsistent until the complete document is compiled at least once.

A useful (if unconventional) way to always ensure a consistent
page numbering is to restart the numbering in each child document
and denote the pages by `\textit{child}|.|\textit{page}'
where \textit{child} represents the chapter/section number of the child file.
This can be achieved by the command
|\numberwithin{page}{|\textit{child}|}|
of the \textsf{amsmath} package
where \textit{child} can be |chapter| or |section|
depending on the chosen structuring.
Alternatively, one can modify the macro |\thepage| appropriately
and reset the counter |page| at the start of each child file.

%%%%%%%%%%%%%%%%%%%%%%%%%%%%%%%%%%%%%%%%%%%%%%%%%%%%%%%%%%%%%%%%%%%%%%%%%%%%%%%%
\subsection{Conditional Processing}
\label{sec:conditional}

The package provides a mechanism to compile different versions
of a document. To customise the versions further some conditional processing
can come in handy to distinguish which version is being compiled.
The package provides two macros to describe the compilation context:

%%%%%%%%%%%%%%%%%%%%%%%%%%%%%%%%%%%%%%%%
\DescribeMacro{\ifchilddoc}
The conditional |\ifchilddoc| distinguishes between the compilation of
child documents and the main document:
%
\begin{center}
|\ifchilddoc |\textit{child-code}| |[|\||else |\textit{main-code}]| \||fi|
\end{center}

%%%%%%%%%%%%%%%%%%%%%%%%%%%%%%%%%%%%%%%%
\DescribeMacro{\childdocname}
\DescribeMacro{\childdocjob}
The macro |\childdocname| contains the filename (without extension)
of the main or child file being processed.
Note that |\childdocjob| will always contain the name of the main file.

%%%%%%%%%%%%%%%%%%%%%%%%%%%%%%%%%%%%%%%%
\paragraph{Title Page.}

Conditional processing can be used to include a title or banner page
in the main document when proper precautions are taken.
Importantly, the code in the main file should ensure that the page counter
(as well as other status parameters which are stored in the |.aux| files)
takes the same value after the conditional processing.
Otherwise the page numbers may take divergent values
depending on which part is compiled.

For example, a title page could be declared by:
%
\begin{center}
\begin{tabular}{l}
|\ifchilddoc\||else|\\
|\addtocounter{page}{-1}|\\
\textit{code for title page}\\
|\newpage|\\
|\||fi|
\end{tabular}
\end{center}
%
A banner page for the child documents can be generated by:
%
\begin{center}
\begin{tabular}{l}
|\ifchilddoc|\\
|\addtocounter{page}{-1}|\\
\textit{code for banner page}\\
|\newpage|\\
|\||fi|
\end{tabular}
\end{center}
%
Here one could write a message such as:
\begin{center}
|This is the part \childdocname{} of \childdocjob{}.|
\end{center}

%%%%%%%%%%%%%%%%%%%%%%%%%%%%%%%%%%%%%%%%%%%%%%%%%%%%%%%%%%%%%%%%%%%%%%%%%%%%%%%%
\subsection{Flags}
\label{sec:flags}

The package makes it easy to generate different versions
of the main or child documents.
To this end compilation flags can be defined
and assigned different default values.
They will be particularly useful in conjunction
with the forwarding mechanism described in \secref{sec:forward}.

For example, it may be useful to have a flag |\version|
which can be set to |draft| or |final|.
The document source will contain some conditional code
depending on the value of |\version|.
Suppose further, the flag should default to |final| for the main file
and to |draft| for child files
which is a natural assignment for editing the document.
This is achieved by placing the following code
in the preamble of the main document
(below the |\childdocmain| directive):
%
\begin{center}
\begin{tabular}{l}
|\ifchilddoc|\\
|\providecommand{\version}{draft}|\\
|\||else|\\
|\providecommand{\version}{final}|\\
|\||fi|
\end{tabular}
\end{center}
%
The definition by |\providecommand| makes sure
that previous definitions are not overwritten.
Further statements |\providecommand{\version}{...}|
can thus be added before the above code to override it.

For the main file, one might add a line
(between |\childdocmain| and the above block)
%
\begin{center}
|%\ifchilddoc\||else\providecommand{\version}{draft}\||fi|
\end{center}
%
which can be uncommented to produce a draft version.
Likewise one can add a line to the very top of a child file
(above the |\childdocof{|\textit{main}|}| directive)
%
\begin{center}
|%\providecommand{\version}{final}|
\end{center}
%
which can be uncommented to produce the final version of this child document.

%%%%%%%%%%%%%%%%%%%%%%%%%%%%%%%%%%%%%%%%%%%%%%%%%%%%%%%%%%%%%%%%%%%%%%%%%%%%%%%%
\subsection{Forwarding}
\label{sec:forward}

Different versions of the main or child documents
using compilation flags as described in \secref{sec:flags}
can be (permanently) stored in different files
for convenient compilation, viewing and distribution.
To this end, the package defines a command
to pass on compilation to a different file:

%%%%%%%%%%%%%%%%%%%%%%%%%%%%%%%%%%%%%%%%
\DescribeMacro{\childdocforward}
The command |\childdocforward| redirects processing to
another source file:
%
\begin{center}
\begin{tabular}{l}
|\input{childdoc.def}|\\
|\childdocforward[|\textit{main}|]{|\textit{dest}|}|\\
\end{tabular}
\end{center}
%
The argument \textit{dest} is the destination file
(without extension).
It should be the main file or one of the child files.
Note that further \textsf{childdoc} directives
such as |\childdocof| and |\childdocforward|
in the indicated file will be processed in this form.
The optional argument \textit{main}
passes on directly to the main file \textit{main}
while pretending to compile the child \textit{dest}.
This form behaves as if \textit{dest}
issues |\childdocof{|\textit{main}|}| right away,
and no further \textsf{childdoc} directives will be processed.

%%%%%%%%%%%%%%%%%%%%%%%%%%%%%%%%%%%%%%%%
\DescribeMacro{\...prefix}
In the alternative form |\childdocforwardprefix|,
%
\begin{center}
\begin{tabular}{l}
|\input{childdoc.def}|\\
|\childdocforwardprefix[|\textit{main}|]{|\textit{prefix}|}{|\textit{dest}|}|
\end{tabular}
\end{center}
%
the destination file is determined by a pattern
depending on the current file:
To make this work, the current file must be called
`{\textit{prefix}\hspace{0.2em}\textit{suffix}}'
with \textit{prefix} matching precisely the argument.
Processing is then passed on to the file
`{\textit{dest}\hspace{0.2em}\textit{suffix}}'.
Surely, the same effect is achieved by
directly specifying the
argument `{\textit{dest}\hspace{0.2em}\textit{suffix}}'
in the first form.
However, that requires to set up a different file
for each child. With the alternative form of the command
all these files can have exactly the same content
which simplifies setting them up and maintaining them.

For example, the following file |draft.tex|
with a compilation flag |\version| as described in \secref{sec:flags}
compiles the main document as a draft:
%
\begin{center}
\begin{tabular}{l}
|\def\version{draft}|\\
|\input{childdoc.def}|\\
|\childdocforward{|\textit{main}|}|
\end{tabular}
\end{center}
%
Likewise, the following files |final|\textit{nn}|.tex|
compile the final version of the child document
|child|\textit{nn}|.tex|:
%
\begin{center}
\begin{tabular}{l}
|\def\version{final}|\\
|\input{childdoc.def}|\\
|\childdocforwardprefix{final}{child}|
\end{tabular}
\end{center}
%

Note that when several versions of a main file and/or of each child file
are to be generated, it may be convenient to set up a |Makefile| or
shell script to automatise the process.

%%%%%%%%%%%%%%%%%%%%%%%%%%%%%%%%%%%%%%%%%%%%%%%%%%%%%%%%%%%%%%%%%%%%%%%%%%%%%%%%
\subsection{Command Line Processing}
\label{sec:commandline}

The effect of redirection files can also be achieved by invoking
the \LaTeX{} compiler with a more elaborate command line.
Most conveniently this should be done as part
of a shell script or a |Makefile|.

When using \textsf{childdoc} in the main file, the following
command lines effectively perform a redirection
(note that depending on the shell being used,
backslashes may have to be doubled: `|\|' $\to$ `|\\|'):
%
\begin{center}
|... -jobname "|\textit{target}|" |\\|"|[\textit{flags}]%
|\input{childdoc.def}\childdocforward[|\textit{main}|]{|\textit{dest}|}"|
\end{center}
%
Here \textit{target} is the name of the output file,
\textit{main} is the name of the main file
and \textit{dest} is the name of the main or child file to be processed
(all filenames without extensions).
The optional argument \textit{main} can be omitted
if \textit{main} matches \textit{dest}.
Optionally, compilation \textit{flags} can be defined via |\def| commands.
This command line makes the \TeX{} engine believe
it is compiling the file \textit{target}
whose content is specified as the latter parameter.
The provided code then forwards the processing to
\textit{main} or \textit{dest} as described in \secref{sec:forward}.

%%%%%%%%%%%%%%%%%%%%%%%%%%%%%%%%%%%%%%%%%%%%%%%%%%%%%%%%%%%%%%%%%%%%%%%%%%%%%%%%
\subsection{Include by Input}
\label{sec:input}

Including child documents by |\include| has some restrictions by design.
Most notably, the content of a child document always occupies
its own set of pages; pages cannot be shared between child documents.
Usually, this behaviour makes perfect sense
because each child document contain an essential part of the document.
However, in some situations it may be desirable to compose
a document from a collection of parts
without having mandatory page breaks between then.
For this case, the package
provides a mechanism to include parts
by |\input| which can also be processed individually.
However, by construction this mechanism
requires manual handling of the content to be output.

%%%%%%%%%%%%%%%%%%%%%%%%%%%%%%%%%%%%%%%%
\DescribeMacro{\ifchilddocmanual}
The main file should be prepared as usual, see \secref{sec:include}.
However, the document body must make a distinction
between processing of an individual part and of the main document, e.g.:
%
\begin{center}
\begin{tabular}{l}
|\ifchilddocmanual|\\
|\input{\childdocname}|\\
|\||else|\\
\textit{document body with }|\input{|\textit{part}|}|\\
|\||fi|
\end{tabular}
\end{center}
%
The conditional |\ifchilddocmanual| is true whenever
a part to be included by |\input| is being compiled,
and the name of the part is stored in |\childdocname|.

%%%%%%%%%%%%%%%%%%%%%%%%%%%%%%%%%%%%%%%%
\DescribeMacro{\childdocby}
Each part to be included by |\input| should start with:
%
\begin{center}
\begin{tabular}{l}
|\input{childdoc.def}|\\
|\childdocby{|\textit{main}|}|\\
\end{tabular}
\end{center}
%
The directive |\childdocby| is similar to |\childdocof|
described in \secref{sec:include},
but the subsequent selection of content must be done manually.
To that end, both |\ifchilddoc| and |\ifchilddocmanual|
will be true upon processing of a part,
and the name of the part is stored in |\childdocname|.
Note that |\jobname| will be set to the filename of the current part
so that each part receives an individual |.aux| file
that does not interfere with the |.aux| file(s) of the main document.
This behaviour can be altered by the alternative form
|\childdocby[*]{|\textit{main}|}| (with a non-empty optional argument)
which uses the |.aux| file of the main document
by setting |\jobname| to \textit{main}.

%%%%%%%%%%%%%%%%%%%%%%%%%%%%%%%%%%%%%%%%%%%%%%%%%%%%%%%%%%%%%%%%%%%%%%%%%%%%%%%%
\subsection{Driver Development}
\label{sec:driver}

The \textsf{childdoc} mechanism can also be use for the development
of definition files such as \LaTeX{} styles or classes.
This case differs from the above setup with multiple parts
included by |\include| in that no |\includeonly| should be invoked.
This can be achieved by starting the include file
(before |\ProvidesPackage|) with:
%
\begin{center}
\begin{tabular}{l}
|\input{childdoc.def}|\\
|\childdocforward{|\textit{main}|}|\\
\end{tabular}
\end{center}
%
or alternatively with:
%
\begin{center}
\begin{tabular}{l}
|\input{childdoc.def}|\\
|\childdocby{|\textit{main}|}|\\
\end{tabular}
\end{center}
%
Both forms have slightly different effects as described above.
The main file is prepared as usual, see \secref{sec:include}.

%%%%%%%%%%%%%%%%%%%%%%%%%%%%%%%%%%%%%%%%%%%%%%%%%%%%%%%%%%%%%%%%%%%%%%%%%%%%%%%%
\subsection{Legacy Detection}
\label{sec:detection}

The directive |\childdocmain| in the main file can detect
whether the complete document or merely a child is to be compiled
even without using the directive |\childdocof|.
This method is deprecated because it is less robust
and there is no compelling reason to use it;
it is merely provided for backward compatibility
and it may be removed in future versions.

If the detection mechanism is to be used,
it is mandatory to correctly specify
the filename of the main file as the argument of |\childdocmain|:
%
\begin{center}
\begin{tabular}{l}
|\input{childdoc.def}|\\
|\childdocmain{|\textit{main}|}|\\
\end{tabular}
\end{center}
%
If |\jobname| does not match the argument \textit{main} of |\childdocmain|,
it is assumed that |\jobname| points to the child file to be compiled.
When using |\childdocmain| with the main file specified as argument,
it suffices to start a child file
with just |\input{|\textit{main}|}|
without loading of the package and using |\childdocof|.
If instead all processing is done
with the appropriate \textsf{childdoc} directives,
the argument of \textit{main} of |\childdocmain| can be empty.

An alternative version of the command line processing described
in \secref{sec:commandline} using the detection mechanism reads:
%
\begin{center}
|... -jobname "|\textit{target}|" "|[\textit{flags}]%
[|\def\jobname{|\textit{dest}|}|]|\input{|\textit{main}|}"|
\end{center}

%%%%%%%%%%%%%%%%%%%%%%%%%%%%%%%%%%%%%%%%%%%%%%%%%%%%%%%%%%%%%%%%%%%%%%%%%%%%%%%%
\subsection{Manual Code}
\label{sec:manual}

In case one cannot be certain whether the definitions file |childdoc.def|
is installed on the target \TeX{} distribution
and one prefers not to ship it,
it is conceivable to paste a few relevant commands into the sources.

To that end, drop all statements |\input{childdoc.def}|
and perform the replacements as outlined below.
Instead of |\childdocmain{|\textit{main}|}| add the following code
to the top of the main file:
%
\begin{center}
\begin{tabular}{l}
|\||ifdefined\childdocname\endinput\||fi\newif\ifchilddoc|\\
|\edef\childdocname{\scantokens\expandafter{\jobname\noexpand}}|\\
|\def\childdocmain{|\textit{main}|}\||ifx\childdocmain\childdocname\||else|\\
|\childdoctrue\includeonly{\childdocname}\let\jobname\childdocmain\||fi|\\
\end{tabular}
\end{center}
%
Instead of |\childdocof{|\textit{main}|}| just include the main file
at the top of each child file:
%
\begin{center}
|\input{|\textit{main}|}|
\end{center}
%
A simple redirection |\childdocforward{|\textit{dest}|}| is achieved by:
%
\begin{center}
|\def\jobname{|\textit{dest}|}\input{\jobname}|
\end{center}
%
The redirection with prefix
|\childdocforwardprefix[|\textit{prefix}|]{|\textit{dest}|}|
is accomplished by:
%
\begin{center}
\begin{tabular}{l}
|{\edef\jobname{\scantokens\expandafter{\jobname\noexpand}}|\\
|\def\redirectjob |\textit{prefix}|#1~~~{\gdef\jobname{|\textit{dest}|#1}}|\\
|\expandafter\redirectjob\jobname~~~}\input{\jobname}|
\end{tabular}
\end{center}

In an alternative approach,
child documents can be compiled by a specific command line
without additional code or specific definitions:
%
\begin{center}
|... -jobname "|\textit{target}|" "|[\textit{flags}]%
|\includeonly{|\textit{dest}|}\input{|\textit{main}|}"|
\end{center}
%

%%%%%%%%%%%%%%%%%%%%%%%%%%%%%%%%%%%%%%%%%%%%%%%%%%%%%%%%%%%%%%%%%%%%%%%%%%%%%%%%
%%%%%%%%%%%%%%%%%%%%%%%%%%%%%%%%%%%%%%%%%%%%%%%%%%%%%%%%%%%%%%%%%%%%%%%%%%%%%%%%
\section{Information}

%%%%%%%%%%%%%%%%%%%%%%%%%%%%%%%%%%%%%%%%%%%%%%%%%%%%%%%%%%%%%%%%%%%%%%%%%%%%%%%%
\subsection{Copyright}

Copyright \copyright{} 2017--2018 Niklas Beisert

This work may be distributed and/or modified under the
conditions of the \LaTeX{} Project Public License, either version 1.3
of this license or (at your option) any later version.
The latest version of this license is in
  \url{http://www.latex-project.org/lppl.txt}
and version 1.3 or later is part of all distributions of \LaTeX{}
version 2005/12/01 or later.

This work has the LPPL maintenance status `maintained'.

The Current Maintainer of this work is Niklas Beisert.

This work consists of the files |README.txt|, |childdoc.ins| and |childdoc.dtx|
as well as the derived files |childdoc.def|, |cdocsamp.tex|
with |cdocsch1.tex|, |cdocsch2.tex|, |cdocspt3.tex|, |cdocspt4.tex|,
|cdocsdrf.tex|, |cdocsfn1.tex|, |cdocsfn2.tex|
as well as |childdoc.pdf|.

%%%%%%%%%%%%%%%%%%%%%%%%%%%%%%%%%%%%%%%%%%%%%%%%%%%%%%%%%%%%%%%%%%%%%%%%%%%%%%%%
\subsection{Files and Installation}

The package consists of the files:
%
\begin{center}
\begin{tabular}{ll}
    |README.txt|   & readme file \\
    |childdoc.ins| & installation file \\
    |childdoc.dtx| & source file \\
    |childdoc.def| & definition file \\
    |cdocsamp.tex| & sample main file \\
    |cdocsch1.tex| & sample include file \\
    |cdocsch2.tex| & sample include file \\
    |cdocspt3.tex| & sample part file \\
    |cdocspt4.tex| & sample part file \\
    |cdocsdrf.tex| & sample redirection file \\
    |cdocsfn1.tex| & sample redirection file \\
    |cdocsfn2.tex| & sample redirection file \\
    |childdoc.pdf| & manual
\end{tabular}
\end{center}
%
The distribution consists of the files
|README.txt|, |childdoc.ins| and |childdoc.dtx|.
%
\begin{itemize}
\item
Run (pdf)\LaTeX{} on |childdoc.dtx|
to compile the manual |childdoc.pdf| (this file).
\item
Run \LaTeX{} on |childdoc.ins| to create the definitions file |childdoc.def|
and the sample |cdocsamp.tex| with include files
|cdocsch1.tex|, |cdocsch2.tex|, |cdocspt3.tex|, |cdocspt4.tex|,
|cdocsdrf.tex|, |cdocsfn1.tex|, |cdocsfn2.tex|.
Then copy the file |childdoc.def| to an appropriate directory of your \LaTeX{}
distribution, e.g.\ \textit{texmf-root}|/tex/latex/childdoc|.
\end{itemize}

%%%%%%%%%%%%%%%%%%%%%%%%%%%%%%%%%%%%%%%%%%%%%%%%%%%%%%%%%%%%%%%%%%%%%%%%%%%%%%%%
\subsection{Related CTAN Packages}

There are several other packages which offer a similar functionality:
%
\begin{itemize}
\item
The packages
\href{http://ctan.org/pkg/docmute}{\textsf{docmute}},
\href{http://ctan.org/pkg/includex}{\textsf{includex}} and
\href{http://ctan.org/pkg/standalone}{\textsf{standalone}}
provide commands to include only the document body of
a child file thus allowing both files to be compiled individually.
\item
The packages \href{http://ctan.org/pkg/subdocs}{\textsf{subdocs}}
and \href{http://ctan.org/pkg/subfiles}{\textsf{subfiles}}
provide structures in which the main and child documents can be
encapsulated and allowing them to be compiled individually.
The inclusion mechanism is different from the conventional |\include|.
\item
The package \href{http://ctan.org/pkg/combine}{\textsf{combine}}
is an elaborate solution to combine several documents into one.
\end{itemize}
%
See also the CTAN topic \href{http://ctan.org/topic/subdocs}{\textsf{subdocs}}
for further related packages.
The present package differs from the above solutions in that
a document structure constructed with the conventional |\include| mechanism
just needs two extra commands at the top of every file
such that all constituent files can be compiled individually.

%%%%%%%%%%%%%%%%%%%%%%%%%%%%%%%%%%%%%%%%%%%%%%%%%%%%%%%%%%%%%%%%%%%%%%%%%%%%%%%%
%\subsection{Feature Suggestions}
%
%The following is a list of features which may be useful for future
%versions of this package:
%%
%\begin{itemize}
%\item
%\ldots
%\end{itemize}

%%%%%%%%%%%%%%%%%%%%%%%%%%%%%%%%%%%%%%%%%%%%%%%%%%%%%%%%%%%%%%%%%%%%%%%%%%%%%%%%
\subsection{Revision History}

%%%%%%%%%%%%%%%%%%%%%%%%%%%%%%%%%%%%%%%%
\paragraph{v2.0:} 2018/12/30

\begin{itemize}
\item
immediate forward processing
\item
added |\childdocby| mechanism
\item
manual restructured
\end{itemize}

%%%%%%%%%%%%%%%%%%%%%%%%%%%%%%%%%%%%%%%%
\paragraph{v1.6:} 2018/01/17

\begin{itemize}
\item
application for development of include files
\item
corrections to manual
\end{itemize}

%%%%%%%%%%%%%%%%%%%%%%%%%%%%%%%%%%%%%%%%
\paragraph{v1.5:} 2017/05/21

\begin{itemize}
\item
more complete structuring introduced
\item
|\childdocof| introduced
\item
|\childdoc| renamed to |\childdocmain|
\item
|\childredirect| renamed to |\childdocforward| and |\childdocforwardprefix|
and functionality expanded
\end{itemize}

%%%%%%%%%%%%%%%%%%%%%%%%%%%%%%%%%%%%%%%%
\paragraph{v1.0:} 2017/04/27

\begin{itemize}
\item
manual and install package
\item
first version published on CTAN
\end{itemize}

%%%%%%%%%%%%%%%%%%%%%%%%%%%%%%%%%%%%%%%%
\paragraph{v0.6:} 2017/04/26

\begin{itemize}
\item
redirection mechanism added
\end{itemize}

%%%%%%%%%%%%%%%%%%%%%%%%%%%%%%%%%%%%%%%%
\paragraph{v0.5:} 2017/04/26

\begin{itemize}
\item
functionality in definition file
\end{itemize}


%%%%%%%%%%%%%%%%%%%%%%%%%%%%%%%%%%%%%%%%%%%%%%%%%%%%%%%%%%%%%%%%%%%%%%%%%%%%%%%%
%%%%%%%%%%%%%%%%%%%%%%%%%%%%%%%%%%%%%%%%%%%%%%%%%%%%%%%%%%%%%%%%%%%%%%%%%%%%%%%%
%%%%%%%%%%%%%%%%%%%%%%%%%%%%%%%%%%%%%%%%%%%%%%%%%%%%%%%%%%%%%%%%%%%%%%%%%%%%%%%%
\appendix

\settowidth\MacroIndent{\rmfamily\scriptsize 000\ }

 \DocInput{childdoc.dtx}

\end{document}
%</driver>
% \fi
%
% %%%%%%%%%%%%%%%%%%%%%%%%%%%%%%%%%%%%%%%%%%%%%%%%%%%%%%%%%%%%%%%%%%%%%%%%%%%%%%
% %%%%%%%%%%%%%%%%%%%%%%%%%%%%%%%%%%%%%%%%%%%%%%%%%%%%%%%%%%%%%%%%%%%%%%%%%%%%%%
% \section{Sample}
%\iffalse
%<*samplemain>
%\fi
%
% The following presents a sample document
% with two chapters, two parts, a title page,
% a compile flag as well as three forwarding files to set the flag.
% It consists of eight |.tex| files:
% \begin{center}
% \begin{tabular}{ll}
% |cdocsamp.tex|&main file\\
% |cdocsch1.tex|&include file for chapter 1\\
% |cdocsch2.tex|&include file for chapter 2\\
% |cdocspt3.tex|&include file for part 3\\
% |cdocspt4.tex|&include file for part 4\\
% |cdocsdrf.tex|&forwarding file for main file in draft mode\\
% |cdocsfi1.tex|&forwarding file for final version of chapter 1\\
% |cdocsfi2.tex|&forwarding file for final version of chapter 2\\
% \end{tabular}
% \end{center}
% Each of the eight files can be compiled directly by the \LaTeX{} compiler.
%
% %%%%%%%%%%%%%%%%%%%%%%%%%%%%%%%%%%%%%%
% \paragraph{Main File.}
%
% The main file is called |cdocsamp.tex|.
%
% Load the \textsf{childdoc} definitions and
% declare the filename for the main document:
%    \begin{macrocode}
\input{childdoc.def}
\childdocmain{}
%    \end{macrocode}

% Optional override for |\version| flag:
%    \begin{macrocode}
%%\ifchilddoc\else\providecommand{\version}{draft}\fi
%    \end{macrocode}

% Define the default values for the |\version| flag
% (|final| for the main file and |draft| for childs):
%    \begin{macrocode}
\ifchilddoc
\providecommand{\version}{draft}
\else
\providecommand{\version}{final}
\fi
%    \end{macrocode}

% Load the standard document class:
%    \begin{macrocode}
\documentclass[12pt]{article}
%    \end{macrocode}

% Start the document body:
%    \begin{macrocode}
\begin{document}
%    \end{macrocode}

% Declare a title page.
% Print title, part of document being processed and version flag:
%    \begin{macrocode}
\addtocounter{page}{-1}
\begin{center}
{\LARGE\bfseries{}childdoc example\par}
\vspace{1cm}
\ifchilddoc
\ifchilddocmanual part\else chapter\fi:
`\childdocname' of `\childdocjob'\par
\else
main document: `\childdocjob'\par
\fi
version: \version\par
\end{center}
\newpage
%    \end{macrocode}

% Manually include selected file,
% otherwise process as usual:
%    \begin{macrocode}
\ifchilddocmanual
\section*{part `\childdocname'}
\input{\childdocname}
\else
%    \end{macrocode}

% Include the two chapters:
%    \begin{macrocode}
\include{cdocsch1}
\include{cdocsch2}
%    \end{macrocode}

% Include the two parts unless only chapters should be displayed:
%    \begin{macrocode}
\ifchilddoc\else
\section{part three}
\input{cdocspt3}
\section{part four}
\input{cdocspt4}
\fi
%    \end{macrocode}

% Process as usual until here:
%    \begin{macrocode}
\fi
%    \end{macrocode}

% End of document body:
%    \begin{macrocode}
\end{document}
%    \end{macrocode}
%\iffalse
%</samplemain>
%\fi
%
% %%%%%%%%%%%%%%%%%%%%%%%%%%%%%%%%%%%%%%
% \paragraph{Chapter Include Files.}
%
% The include files are called |cdocsch1.tex| and |cdocsch2.tex|.
%
%\iffalse
%<*samplechap1|samplechap2>
%\fi

% Optional override for |\version| flag:
%    \begin{macrocode}
%%\providecommand{\version}{final}
%    \end{macrocode}

% Include the main document:
%    \begin{macrocode}
\input{childdoc.def}
\childdocof{cdocsamp}
%    \end{macrocode}

%\iffalse
%</samplechap1|samplechap2>
%\fi
%
%\iffalse
%<*samplechap1>
%\fi
% Some text for chapter 1:
%    \begin{macrocode}
\section{one}
some text in chapter one
%    \end{macrocode}

%\iffalse
%</samplechap1>
%\fi
% Some text for chapter 2:
%\iffalse
%<*samplechap2>
%\fi
%    \begin{macrocode}
\section{two}
more text in chapter two
%    \end{macrocode}

%\iffalse
%</samplechap2>
%\fi
%
% %%%%%%%%%%%%%%%%%%%%%%%%%%%%%%%%%%%%%%
% \paragraph{Part Include Files.}
%
% The include files are called |cdocspt3.tex| and |cdocspt4.tex|.
%
%\iffalse
%<*samplepart3|samplepart4>
%\fi

% Optional override for |\version| flag:
%    \begin{macrocode}
%%\providecommand{\version}{final}
%    \end{macrocode}

% Include the main document:
%    \begin{macrocode}
\input{childdoc.def}
\childdocby{cdocsamp}
%    \end{macrocode}

%\iffalse
%</samplepart3|samplepart4>
%\fi
%
%\iffalse
%<*samplepart3>
%\fi
% Some text for part 3:
%    \begin{macrocode}
some text in part three
%    \end{macrocode}

%\iffalse
%</samplepart3>
%\fi
% Some text for part 4:
%\iffalse
%<*samplepart4>
%\fi
%    \begin{macrocode}
more text in part four
%    \end{macrocode}

%\iffalse
%</samplepart4>
%\fi
%
% %%%%%%%%%%%%%%%%%%%%%%%%%%%%%%%%%%%%%%
% \paragraph{Forwarding for a Complete Draft.}
%
% The following forwarding file |cdocsdrf.tex|
% compiles the main document in draft mode:
%\iffalse
%<*sampledraft>
%\fi
%    \begin{macrocode}
\def\version{draft}
\input{childdoc.def}
\childdocforward{cdocsamp}
%    \end{macrocode}

%\iffalse
%</sampledraft>
%\fi
%
% %%%%%%%%%%%%%%%%%%%%%%%%%%%%%%%%%%%%%%
% \paragraph{Forwarding for Final Version of the Chapters.}
%
% The following forwarding files |cdocsfn1.tex| and |cdocsfn2.tex|
% (with identical content)
% compile the final versions of the child documents
% |cdocsch1.tex| and |cdocsch2.tex|, respectively:
%\iffalse
%<*samplefinal>
%\fi
%    \begin{macrocode}
\def\version{final}
\input{childdoc.def}
\childdocforwardprefix[cdocsamp]{cdocsfn}{cdocsch}
%    \end{macrocode}

%\iffalse
%</samplefinal>
%\fi
%
% %%%%%%%%%%%%%%%%%%%%%%%%%%%%%%%%%%%%%%
% \paragraph{Command Line Processing.}
%
% The following three command lines generate the output files
% |cdocscld|, |cdocscl1| and |cdocscl2|
% which should be identical to
% |cdocsdrf|, |cdocsch1| and |cdocsfn2|, respectively:
% \begin{center}
% \begin{tabular}{l}
% |latex -jobname cdocscld \|\\
% |  "\def\version{draft}\input{childdoc.def}\childdocforward{cdocsamp}"|\\
% |latex -jobname cdocscl1 \|\\
% |  "\input{childdoc.def}\childdocforward[cdocsamp]{cdocsch1}"|\\
% |latex -jobname cdocscl2 \|\\
% |  "\def\version{final}\input{childdoc.def}\childdocforward{cdocsch2}"|
% \end{tabular}
% \end{center}
% Note that the trailing backslash on each first line
% merely continues the input to the second line
% (for convenient cut ant paste).
% Furthermore, the command |latex| can be replaced by any
% of its alternative versions such as |pdflatex|.
%
% %%%%%%%%%%%%%%%%%%%%%%%%%%%%%%%%%%%%%%%%%%%%%%%%%%%%%%%%%%%%%%%%%%%%%%%%%%%%%%
% %%%%%%%%%%%%%%%%%%%%%%%%%%%%%%%%%%%%%%%%%%%%%%%%%%%%%%%%%%%%%%%%%%%%%%%%%%%%%%
% \section{Implementation}
%\iffalse
%<*package>
%\fi
%
% This section describes the definitions file |childdoc.def|.

% The definitions cannot be loaded using |\usepackage| or |\RequirePackage|
% which has a mechanism to prevent loading a style file more than once.
% When loading the definitions by means of |\input|
% multiple instances have to be prevented manually:
%\iffalse
%This code needs to be before the `\ProvidesFile' directive
%which is defined at the beginning of this file.
%Therefore it is also placed there and commented out here.
%</package>
%<*discard>
%\fi
%    \begin{macrocode}
\ifdefined\childdocmain\endinput\fi
%    \end{macrocode}
%\iffalse
%</discard>
%<*package>
%\fi
%
% \macro{\ifchilddoc}
% \macro{\ifchilddocmanual}
% The conditional |\ifchilddoc| tells whether a
% child (true) or main (false) document is being compiled.
% The conditional |\ifchilddocmanual| tells whether
% the |\includeonly| mechanism is used (false) or
% the selection of child files must be performed manually (true).
% The definitions initialise to false:
%    \begin{macrocode}
\newif\ifchilddoc
\newif\ifchilddocmanual
%    \end{macrocode}

% \macro{\childdocname}
% \macro{\childdocjob}
% The macro |\childdocname| stores the name of the main document
% to be compiled. The macro |\childdocjob| stores the name of
% the document on which the \LaTeX{} compiler was originally invoked.
% The content of |\jobname| cannot be compared
% to filenames specified in the source due to different catcodes.
% The following code rescans |\jobname|, stores the result
% in |\childdocname| and saves a copy in |\childdocjob|:
%    \begin{macrocode}
\edef\childdocname{\scantokens\expandafter{\jobname\noexpand}}
\let\childdocjob\childdocname
%    \end{macrocode}

% \macro{\childdocdisable}
% The macro |\childdocdisable| prevents the main file
% from being processed more than once.
% At this stage, the main document command |\childdocmain|
% is assumed to be called once again where it should do nothing.
% Any subsequent call to it should prevent
% a secondary processing of the main document
% It overwrites the forwarding commands
% |\childdocof| and |\childdocforward|
% with empty macros to prevent further inclusions of the main document:
%    \begin{macrocode}
\newcommand{\childdocdisable}
{
  \renewcommand{\childdocmain}[1]{\renewcommand{\childdocmain}[1]{\endinput}}
  \renewcommand{\childdocof}[1]{}
  \renewcommand{\childdocby}[2][]{}
  \renewcommand{\childdocforward}[2][]{}
  \renewcommand{\childdocdisable}{}
}
%    \end{macrocode}

% \macro{\childdocmain}
% The macro |\childdocmain| is to be called at the top of the main file
% with nothing or the main filename (without extension) as argument.
% First, it breaks loops.
% If the argument is not empty and does not match |\childdocname|
% (which is set by the first inclusion of |childdoc.def|),
% |\ifchilddoc| is set to true, |\includeonly| is applied to the child file
% and |\jobname| is set to the main file
% (for proper handling of |.aux| files):
%    \begin{macrocode}
\newcommand{\childdocmain}[1]
{
  \childdocdisable\childdocmain{}
  \if?#1?\else
    \begingroup
      \def\childdoctmp{#1}
      \ifx\childdoctmp\childdocname
        \def\childdoctmp{}
      \else
        \def\childdoctmp
        {
          \childdoctrue
          \includeonly{\childdocname}
          \def\childdocjob{#1}
          \def\jobname{#1}
        }
      \fi
      \expandafter
    \endgroup
    \childdoctmp
  \fi
}
%    \end{macrocode}

% \macro{\childdocof}
% The command |\childdocof| redirects
% compilation to the main file |#1|.
%    \begin{macrocode}
\newcommand{\childdocof}[1]
{
  \childdocdisable
  \childdoctrue
  \includeonly{\childdocname}
  \def\jobname{#1}
  \def\childdocjob{#1}
  \input{#1}
}
%    \end{macrocode}

% \macro{\childdocby}
% The command |\childdocby| ....
%    \begin{macrocode}
\newcommand{\childdocby}[2][]
{
  \childdocdisable
  \childdoctrue
  \childdocmanualtrue
  \if?#1?\else
    \def\jobname{#2}
  \fi
  \def\childdocjob{#2}
  \input{#2}
  \endinput
}
%    \end{macrocode}

% \macro{\childdocforward}
% The command |\childdocforward| redirects
% compilation to the main file or
% (if the optional argument is given) a child file.
% Parameters are set as if the main file
% or a child file starting with |\childdocof| was compiled.
% Then compilation is handed over to the main file:
%    \begin{macrocode}
\newcommand{\childdocforward}[2][]
{
  \begingroup
    \if?#1?
      \def\childdoctmp
      {
        \def\childdocname{#2}
        \def\childdocjob{#2}
        \def\jobname{#2}
        \input{#2}
        \endinput
      }
    \else
      \def\childdoctmp
      {
        \childdocdisable
        \def\childdocname{#2}
        \childdoctrue
        \includeonly{#2}
        \def\childdocjob{#1}
        \def\jobname{#1}
        \input{#1}
        \endinput
      }
    \fi
    \expandafter
  \endgroup
  \childdoctmp
}
%    \end{macrocode}

% \macro{\childdocforwardprefix}
% The command |\childdocforwardprefix| redirects
% compilation to the main or a child file by means of a pattern.
% The prefix |#1| in the current filename is replaced by |#2|
% and the suffix of the current filename is kept
% (it is assumed that the filename does not contain the substring `|~~~|'
% which is used as a delimiter).
% Compilation is handed over to the new file by |\childdocforward|:
%    \begin{macrocode}
\newcommand{\childdocforwardprefix}[3][]
{
  \begingroup
    \def\childdocextract #2##1~~~{\def\childdoctmp{\childdocforward[#1]{#3##1}}}
    \expandafter\childdocextract\childdocname~~~
    \expandafter
  \endgroup
  \childdoctmp
}
%    \end{macrocode}

% \macro{\childdoc}
% The deprecated macro |\childdoc| is a legacy version of |\childdocmain|:
%    \begin{macrocode}
\newcommand{\childdoc}{\childdocmain}
%    \end{macrocode}

% \macro{\childdocredirect}
% The deprecated macro |\childdocredirect| is a legacy version
% of |\childdocforward| and |\childdocforwardprefix|:
%    \begin{macrocode}
\newcommand{\childdocredirect}[2][]
{
  \begingroup
    \if?#1?
      \def\childdoctmp{\childdocforward{#2}}
    \else
      \def\childdoctmp{\childdocforwardprefix{#1}{#2}}
    \fi
    \expandafter
  \endgroup
  \childdoctmp
}
%    \end{macrocode}

%\iffalse
%</package>
%\fi
%
\endinput
|\\
|\childdocforwardprefix[|\textit{main}|]{|\textit{prefix}|}{|\textit{dest}|}|
\end{tabular}
\end{center}
%
the destination file is determined by a pattern
depending on the current file:
To make this work, the current file must be called
`{\textit{prefix}\hspace{0.2em}\textit{suffix}}'
with \textit{prefix} matching precisely the argument.
Processing is then passed on to the file
`{\textit{dest}\hspace{0.2em}\textit{suffix}}'.
Surely, the same effect is achieved by
directly specifying the
argument `{\textit{dest}\hspace{0.2em}\textit{suffix}}'
in the first form.
However, that requires to set up a different file
for each child. With the alternative form of the command
all these files can have exactly the same content
which simplifies setting them up and maintaining them.

For example, the following file |draft.tex|
with a compilation flag |\version| as described in \secref{sec:flags}
compiles the main document as a draft:
%
\begin{center}
\begin{tabular}{l}
|\def\version{draft}|\\
|% \iffalse
%
% childdoc.dtx Copyright (C) 2017-2018 Niklas Beisert
%
% This work may be distributed and/or modified under the
% conditions of the LaTeX Project Public License, either version 1.3
% of this license or (at your option) any later version.
% The latest version of this license is in
%   http://www.latex-project.org/lppl.txt
% and version 1.3 or later is part of all distributions of LaTeX
% version 2005/12/01 or later.
%
% This work has the LPPL maintenance status `maintained'.
%
% The Current Maintainer of this work is Niklas Beisert.
%
% This work consists of the files childdoc.dtx and childdoc.ins
% and the derived files childdoc.def and cdocsamp.tex with
% cdocsch1.tex, cdocsch2.tex, cdocsdrf.tex, cdocsfn1.tex, cdocsfn2.tex.
%
%<package>\ifdefined\childdocmain\endinput\fi
%<package>\ProvidesFile{childdoc.def}[2018/12/30 v2.0 child document driver]
%<samplemain>\ProvidesFile{cdocsamp.tex}[2018/12/30 v2.0 sample for childdoc]
%<*driver>
%\ProvidesFile{childdoc.drv}[2018/12/30 v2.0 childdoc reference manual file]
\PassOptionsToClass{10pt,a4paper}{article}
\documentclass{ltxdoc}

\usepackage[margin=35mm]{geometry}
\usepackage{hyperref}
\usepackage{hyperxmp}
\usepackage[usenames]{color}

\hypersetup{colorlinks=true}
\hypersetup{pdfstartview=FitH}
\hypersetup{pdfpagemode=UseNone}
\hypersetup{pdfsource={}}
\hypersetup{pdflang={en-UK}}
\hypersetup{pdfcopyright={Copyright 2017-2018 Niklas Beisert.
  This work may be distributed and/or modified under the
  conditions of the LaTeX Project Public License, either version 1.3
  of this license or (at your option) any later version.}}
\hypersetup{pdflicenseurl={http://www.latex-project.org/lppl.txt}}
\hypersetup{pdfcontactaddress={ETH Zurich, ITP, HIT K,
  Wolfgang-Pauli-Strasse 27}}
\hypersetup{pdfcontactpostcode={8093}}
\hypersetup{pdfcontactcity={Zurich}}
\hypersetup{pdfcontactcountry={Switzerland}}
\hypersetup{pdfcontactemail={nbeisert@itp.phys.ethz.ch}}
\hypersetup{pdfcontacturl={http://people.phys.ethz.ch/\xmptilde nbeisert/}}

\newcommand{\secref}[1]{\hyperref[#1]{section \ref*{#1}}}

\parskip1ex
\parindent0pt
\let\olditemize\itemize
\def\itemize{\olditemize\parskip0pt}

\begin{document}

\title{The \textsf{childdoc} Package}
\hypersetup{pdftitle={The childdoc Package}}
\author{Niklas Beisert\\[2ex]
  Institut f\"ur Theoretische Physik\\
  Eidgen\"ossische Technische Hochschule Z\"urich\\
  Wolfgang-Pauli-Strasse 27, 8093 Z\"urich, Switzerland\\[1ex]
  \href{mailto:nbeisert@itp.phys.ethz.ch}
  {\texttt{nbeisert@itp.phys.ethz.ch}}}
\hypersetup{pdfauthor={Niklas Beisert}}
\hypersetup{pdfsubject={Manual for the LaTeX2e Package childdoc}}
\date{30 December 2018, \textsf{v2.0}}
\maketitle

\begin{abstract}\noindent
\textsf{childdoc} is a \LaTeXe{} package
that enables the direct compilation
of document sections included by |\include|
to individual files.
\end{abstract}

\begingroup
\parskip0ex
\tableofcontents
\endgroup

%%%%%%%%%%%%%%%%%%%%%%%%%%%%%%%%%%%%%%%%%%%%%%%%%%%%%%%%%%%%%%%%%%%%%%%%%%%%%%%%
%%%%%%%%%%%%%%%%%%%%%%%%%%%%%%%%%%%%%%%%%%%%%%%%%%%%%%%%%%%%%%%%%%%%%%%%%%%%%%%%
\section{Introduction}

\LaTeX{} provides a mechanism to structure a large document (such as a book)
into a main file and several child files (containing the chapters)
using the |\include| command.
This mechanism is beneficial for documents
which span hundreds of pages in order to
make the source file(s) more manageable.
Moreover, compilation can be restricted to
selected child files by means of the |\includeonly| command.
The latter feature can be used to reduce the compilation time while editing
(this was significantly more useful in the earlier days of \LaTeX{})
or to generate a smaller document which is easier to navigate.
Another application of |\includeonly| is to generate
documents consisting of selected parts of the complete document.

However, there are a few drawbacks of the plain |\include| mechanism:
\begin{itemize}
\item
The child files cannot be compiled on their own,
they can only be compiled via the main file.
A naive editing environment
(such as a text editor with an option
to have the current file processed by \LaTeX)
may require one to switch to the main file before compiling;
attempting to compile the child file produces errors.
\item
The main file must be modified (each time)
to adjust the |\includeonly| command
to the present needs. This easily leaves the main file in a messy state.
\item
The generated document will always carry the filename
of the main document. This is inconvenient if
several child files are to be compiled and
to be kept for distribution.
\end{itemize}

The present package provides a simple interface
to make child files individually compilable by \LaTeX{}.
Compiling a child file then has the same effect as compiling
the main file with an |\includeonly| command
to select the appropriate child.
Moreover the generated document will carry the name of the child
rather than the main file.
This resolves all three above issues.

This feature is meant to make the editing of books,
thesis documents and lecture notes somewhat more convenient.
However, the package can also be used efficiently for
composing a series of documents (such as exercise sheets)
which are typically distributed individually.
It then assists the author in generating the individual documents
(potentially in different versions)
as well as a document containing the collected series.
Another application is in developing style files
or other kinds of included material
where compilation of the style file could redirect
to a sample or test file.

%%%%%%%%%%%%%%%%%%%%%%%%%%%%%%%%%%%%%%%%%%%%%%%%%%%%%%%%%%%%%%%%%%%%%%%%%%%%%%%%
%%%%%%%%%%%%%%%%%%%%%%%%%%%%%%%%%%%%%%%%%%%%%%%%%%%%%%%%%%%%%%%%%%%%%%%%%%%%%%%%
\section{Usage}

First of all, the package \textsf{childdoc} is \emph{not} a standard
\LaTeXe{} |.sty| style file! Therefore it needs to be invoked in
a non-standard way.

%%%%%%%%%%%%%%%%%%%%%%%%%%%%%%%%%%%%%%%%%%%%%%%%%%%%%%%%%%%%%%%%%%%%%%%%%%%%%%%%
\subsection{Included Files}
\label{sec:include}

%%%%%%%%%%%%%%%%%%%%%%%%%%%%%%%%%%%%%%%%
\DescribeMacro{\childdocmain}
To use the package, add the commands
\begin{center}
\begin{tabular}{l}
|\input{childdoc.def}|\\
|\childdocmain{}|\\
\end{tabular}
\end{center}
at the very top of the main \LaTeX{} file,
in particular \emph{before} the |\documentclass| statement!
The argument of |\childdocmain| should be left empty
(but it must be present).

%%%%%%%%%%%%%%%%%%%%%%%%%%%%%%%%%%%%%%%%
\DescribeMacro{\childdocof}
Furthermore, add the commands
\begin{center}
\begin{tabular}{l}
|\input{childdoc.def}|\\
|\childdocof{|\textit{main}|}|\\
\end{tabular}
\end{center}
at the top of every child file \textit{child}
which is included by |\include{|\textit{child}|}|
from within the main file
(or at least for those files to be compiled individually).
The argument \textit{main} must be the filename of the main file.

There are a couple of
considerations in setting up the main and child documents:

%%%%%%%%%%%%%%%%%%%%%%%%%%%%%%%%%%%%%%%%
\paragraph{Restrictions.}

Please note the following restrictions:
\begin{itemize}
\item
|\childdocmain| must be called with one argument \textit{main}
to ensure compatibility with earlier version of the package.
It must either be empty (|\childdocmain{}|)
or precisely match the filename of the main file in which it is specified.
See \secref{sec:detection} for further information.
\item
The filename \textit{main} must be specified without the |.tex| extension.
\item
The filename \textit{main} is case sensitive
(even in case-insensitive file systems)
due to internal string comparison.
\item
The argument \textit{main} should be fully expanded, it cannot be a macro.
\item
Subdirectories and special characters should be avoided in filenames.
\item
The command |\childdocmain{|\textit{main}|}| must be followed by a whitespace.
It should not be followed immediately by another command
or by a comment mark `|%|'.
This is because the \TeX{} parser reads the token immediately following
the argument of |\childdocmain| and puts it
at the beginning of every child section;
however, a white\-space is ignored.
\end{itemize}

%%%%%%%%%%%%%%%%%%%%%%%%%%%%%%%%%%%%%%%%
\paragraph{Content of Main File.}

It is advisable to place all content in the child files included by |\include|.
Any output contained in the main file will appear in all child documents
unless suppressed manually;
it cannot be suppressed automatically by the |\includeonly| directive
and thus should normally be avoided.
A method to include some content in the main file
by means of conditional processing is described in \secref{sec:conditional}.

%%%%%%%%%%%%%%%%%%%%%%%%%%%%%%%%%%%%%%%%
\paragraph{Page Numbering.}

When only a part of the document is compiled,
the appropriate numbering of pages
(as well as other status parameters)
is determined from the |.aux| files.
The latter contain information from previous passes.
However this information needs to propagate through
all intermediate child documents.
Therefore the page numbering in child documents may well
be inconsistent until the complete document is compiled at least once.

A useful (if unconventional) way to always ensure a consistent
page numbering is to restart the numbering in each child document
and denote the pages by `\textit{child}|.|\textit{page}'
where \textit{child} represents the chapter/section number of the child file.
This can be achieved by the command
|\numberwithin{page}{|\textit{child}|}|
of the \textsf{amsmath} package
where \textit{child} can be |chapter| or |section|
depending on the chosen structuring.
Alternatively, one can modify the macro |\thepage| appropriately
and reset the counter |page| at the start of each child file.

%%%%%%%%%%%%%%%%%%%%%%%%%%%%%%%%%%%%%%%%%%%%%%%%%%%%%%%%%%%%%%%%%%%%%%%%%%%%%%%%
\subsection{Conditional Processing}
\label{sec:conditional}

The package provides a mechanism to compile different versions
of a document. To customise the versions further some conditional processing
can come in handy to distinguish which version is being compiled.
The package provides two macros to describe the compilation context:

%%%%%%%%%%%%%%%%%%%%%%%%%%%%%%%%%%%%%%%%
\DescribeMacro{\ifchilddoc}
The conditional |\ifchilddoc| distinguishes between the compilation of
child documents and the main document:
%
\begin{center}
|\ifchilddoc |\textit{child-code}| |[|\||else |\textit{main-code}]| \||fi|
\end{center}

%%%%%%%%%%%%%%%%%%%%%%%%%%%%%%%%%%%%%%%%
\DescribeMacro{\childdocname}
\DescribeMacro{\childdocjob}
The macro |\childdocname| contains the filename (without extension)
of the main or child file being processed.
Note that |\childdocjob| will always contain the name of the main file.

%%%%%%%%%%%%%%%%%%%%%%%%%%%%%%%%%%%%%%%%
\paragraph{Title Page.}

Conditional processing can be used to include a title or banner page
in the main document when proper precautions are taken.
Importantly, the code in the main file should ensure that the page counter
(as well as other status parameters which are stored in the |.aux| files)
takes the same value after the conditional processing.
Otherwise the page numbers may take divergent values
depending on which part is compiled.

For example, a title page could be declared by:
%
\begin{center}
\begin{tabular}{l}
|\ifchilddoc\||else|\\
|\addtocounter{page}{-1}|\\
\textit{code for title page}\\
|\newpage|\\
|\||fi|
\end{tabular}
\end{center}
%
A banner page for the child documents can be generated by:
%
\begin{center}
\begin{tabular}{l}
|\ifchilddoc|\\
|\addtocounter{page}{-1}|\\
\textit{code for banner page}\\
|\newpage|\\
|\||fi|
\end{tabular}
\end{center}
%
Here one could write a message such as:
\begin{center}
|This is the part \childdocname{} of \childdocjob{}.|
\end{center}

%%%%%%%%%%%%%%%%%%%%%%%%%%%%%%%%%%%%%%%%%%%%%%%%%%%%%%%%%%%%%%%%%%%%%%%%%%%%%%%%
\subsection{Flags}
\label{sec:flags}

The package makes it easy to generate different versions
of the main or child documents.
To this end compilation flags can be defined
and assigned different default values.
They will be particularly useful in conjunction
with the forwarding mechanism described in \secref{sec:forward}.

For example, it may be useful to have a flag |\version|
which can be set to |draft| or |final|.
The document source will contain some conditional code
depending on the value of |\version|.
Suppose further, the flag should default to |final| for the main file
and to |draft| for child files
which is a natural assignment for editing the document.
This is achieved by placing the following code
in the preamble of the main document
(below the |\childdocmain| directive):
%
\begin{center}
\begin{tabular}{l}
|\ifchilddoc|\\
|\providecommand{\version}{draft}|\\
|\||else|\\
|\providecommand{\version}{final}|\\
|\||fi|
\end{tabular}
\end{center}
%
The definition by |\providecommand| makes sure
that previous definitions are not overwritten.
Further statements |\providecommand{\version}{...}|
can thus be added before the above code to override it.

For the main file, one might add a line
(between |\childdocmain| and the above block)
%
\begin{center}
|%\ifchilddoc\||else\providecommand{\version}{draft}\||fi|
\end{center}
%
which can be uncommented to produce a draft version.
Likewise one can add a line to the very top of a child file
(above the |\childdocof{|\textit{main}|}| directive)
%
\begin{center}
|%\providecommand{\version}{final}|
\end{center}
%
which can be uncommented to produce the final version of this child document.

%%%%%%%%%%%%%%%%%%%%%%%%%%%%%%%%%%%%%%%%%%%%%%%%%%%%%%%%%%%%%%%%%%%%%%%%%%%%%%%%
\subsection{Forwarding}
\label{sec:forward}

Different versions of the main or child documents
using compilation flags as described in \secref{sec:flags}
can be (permanently) stored in different files
for convenient compilation, viewing and distribution.
To this end, the package defines a command
to pass on compilation to a different file:

%%%%%%%%%%%%%%%%%%%%%%%%%%%%%%%%%%%%%%%%
\DescribeMacro{\childdocforward}
The command |\childdocforward| redirects processing to
another source file:
%
\begin{center}
\begin{tabular}{l}
|\input{childdoc.def}|\\
|\childdocforward[|\textit{main}|]{|\textit{dest}|}|\\
\end{tabular}
\end{center}
%
The argument \textit{dest} is the destination file
(without extension).
It should be the main file or one of the child files.
Note that further \textsf{childdoc} directives
such as |\childdocof| and |\childdocforward|
in the indicated file will be processed in this form.
The optional argument \textit{main}
passes on directly to the main file \textit{main}
while pretending to compile the child \textit{dest}.
This form behaves as if \textit{dest}
issues |\childdocof{|\textit{main}|}| right away,
and no further \textsf{childdoc} directives will be processed.

%%%%%%%%%%%%%%%%%%%%%%%%%%%%%%%%%%%%%%%%
\DescribeMacro{\...prefix}
In the alternative form |\childdocforwardprefix|,
%
\begin{center}
\begin{tabular}{l}
|\input{childdoc.def}|\\
|\childdocforwardprefix[|\textit{main}|]{|\textit{prefix}|}{|\textit{dest}|}|
\end{tabular}
\end{center}
%
the destination file is determined by a pattern
depending on the current file:
To make this work, the current file must be called
`{\textit{prefix}\hspace{0.2em}\textit{suffix}}'
with \textit{prefix} matching precisely the argument.
Processing is then passed on to the file
`{\textit{dest}\hspace{0.2em}\textit{suffix}}'.
Surely, the same effect is achieved by
directly specifying the
argument `{\textit{dest}\hspace{0.2em}\textit{suffix}}'
in the first form.
However, that requires to set up a different file
for each child. With the alternative form of the command
all these files can have exactly the same content
which simplifies setting them up and maintaining them.

For example, the following file |draft.tex|
with a compilation flag |\version| as described in \secref{sec:flags}
compiles the main document as a draft:
%
\begin{center}
\begin{tabular}{l}
|\def\version{draft}|\\
|\input{childdoc.def}|\\
|\childdocforward{|\textit{main}|}|
\end{tabular}
\end{center}
%
Likewise, the following files |final|\textit{nn}|.tex|
compile the final version of the child document
|child|\textit{nn}|.tex|:
%
\begin{center}
\begin{tabular}{l}
|\def\version{final}|\\
|\input{childdoc.def}|\\
|\childdocforwardprefix{final}{child}|
\end{tabular}
\end{center}
%

Note that when several versions of a main file and/or of each child file
are to be generated, it may be convenient to set up a |Makefile| or
shell script to automatise the process.

%%%%%%%%%%%%%%%%%%%%%%%%%%%%%%%%%%%%%%%%%%%%%%%%%%%%%%%%%%%%%%%%%%%%%%%%%%%%%%%%
\subsection{Command Line Processing}
\label{sec:commandline}

The effect of redirection files can also be achieved by invoking
the \LaTeX{} compiler with a more elaborate command line.
Most conveniently this should be done as part
of a shell script or a |Makefile|.

When using \textsf{childdoc} in the main file, the following
command lines effectively perform a redirection
(note that depending on the shell being used,
backslashes may have to be doubled: `|\|' $\to$ `|\\|'):
%
\begin{center}
|... -jobname "|\textit{target}|" |\\|"|[\textit{flags}]%
|\input{childdoc.def}\childdocforward[|\textit{main}|]{|\textit{dest}|}"|
\end{center}
%
Here \textit{target} is the name of the output file,
\textit{main} is the name of the main file
and \textit{dest} is the name of the main or child file to be processed
(all filenames without extensions).
The optional argument \textit{main} can be omitted
if \textit{main} matches \textit{dest}.
Optionally, compilation \textit{flags} can be defined via |\def| commands.
This command line makes the \TeX{} engine believe
it is compiling the file \textit{target}
whose content is specified as the latter parameter.
The provided code then forwards the processing to
\textit{main} or \textit{dest} as described in \secref{sec:forward}.

%%%%%%%%%%%%%%%%%%%%%%%%%%%%%%%%%%%%%%%%%%%%%%%%%%%%%%%%%%%%%%%%%%%%%%%%%%%%%%%%
\subsection{Include by Input}
\label{sec:input}

Including child documents by |\include| has some restrictions by design.
Most notably, the content of a child document always occupies
its own set of pages; pages cannot be shared between child documents.
Usually, this behaviour makes perfect sense
because each child document contain an essential part of the document.
However, in some situations it may be desirable to compose
a document from a collection of parts
without having mandatory page breaks between then.
For this case, the package
provides a mechanism to include parts
by |\input| which can also be processed individually.
However, by construction this mechanism
requires manual handling of the content to be output.

%%%%%%%%%%%%%%%%%%%%%%%%%%%%%%%%%%%%%%%%
\DescribeMacro{\ifchilddocmanual}
The main file should be prepared as usual, see \secref{sec:include}.
However, the document body must make a distinction
between processing of an individual part and of the main document, e.g.:
%
\begin{center}
\begin{tabular}{l}
|\ifchilddocmanual|\\
|\input{\childdocname}|\\
|\||else|\\
\textit{document body with }|\input{|\textit{part}|}|\\
|\||fi|
\end{tabular}
\end{center}
%
The conditional |\ifchilddocmanual| is true whenever
a part to be included by |\input| is being compiled,
and the name of the part is stored in |\childdocname|.

%%%%%%%%%%%%%%%%%%%%%%%%%%%%%%%%%%%%%%%%
\DescribeMacro{\childdocby}
Each part to be included by |\input| should start with:
%
\begin{center}
\begin{tabular}{l}
|\input{childdoc.def}|\\
|\childdocby{|\textit{main}|}|\\
\end{tabular}
\end{center}
%
The directive |\childdocby| is similar to |\childdocof|
described in \secref{sec:include},
but the subsequent selection of content must be done manually.
To that end, both |\ifchilddoc| and |\ifchilddocmanual|
will be true upon processing of a part,
and the name of the part is stored in |\childdocname|.
Note that |\jobname| will be set to the filename of the current part
so that each part receives an individual |.aux| file
that does not interfere with the |.aux| file(s) of the main document.
This behaviour can be altered by the alternative form
|\childdocby[*]{|\textit{main}|}| (with a non-empty optional argument)
which uses the |.aux| file of the main document
by setting |\jobname| to \textit{main}.

%%%%%%%%%%%%%%%%%%%%%%%%%%%%%%%%%%%%%%%%%%%%%%%%%%%%%%%%%%%%%%%%%%%%%%%%%%%%%%%%
\subsection{Driver Development}
\label{sec:driver}

The \textsf{childdoc} mechanism can also be use for the development
of definition files such as \LaTeX{} styles or classes.
This case differs from the above setup with multiple parts
included by |\include| in that no |\includeonly| should be invoked.
This can be achieved by starting the include file
(before |\ProvidesPackage|) with:
%
\begin{center}
\begin{tabular}{l}
|\input{childdoc.def}|\\
|\childdocforward{|\textit{main}|}|\\
\end{tabular}
\end{center}
%
or alternatively with:
%
\begin{center}
\begin{tabular}{l}
|\input{childdoc.def}|\\
|\childdocby{|\textit{main}|}|\\
\end{tabular}
\end{center}
%
Both forms have slightly different effects as described above.
The main file is prepared as usual, see \secref{sec:include}.

%%%%%%%%%%%%%%%%%%%%%%%%%%%%%%%%%%%%%%%%%%%%%%%%%%%%%%%%%%%%%%%%%%%%%%%%%%%%%%%%
\subsection{Legacy Detection}
\label{sec:detection}

The directive |\childdocmain| in the main file can detect
whether the complete document or merely a child is to be compiled
even without using the directive |\childdocof|.
This method is deprecated because it is less robust
and there is no compelling reason to use it;
it is merely provided for backward compatibility
and it may be removed in future versions.

If the detection mechanism is to be used,
it is mandatory to correctly specify
the filename of the main file as the argument of |\childdocmain|:
%
\begin{center}
\begin{tabular}{l}
|\input{childdoc.def}|\\
|\childdocmain{|\textit{main}|}|\\
\end{tabular}
\end{center}
%
If |\jobname| does not match the argument \textit{main} of |\childdocmain|,
it is assumed that |\jobname| points to the child file to be compiled.
When using |\childdocmain| with the main file specified as argument,
it suffices to start a child file
with just |\input{|\textit{main}|}|
without loading of the package and using |\childdocof|.
If instead all processing is done
with the appropriate \textsf{childdoc} directives,
the argument of \textit{main} of |\childdocmain| can be empty.

An alternative version of the command line processing described
in \secref{sec:commandline} using the detection mechanism reads:
%
\begin{center}
|... -jobname "|\textit{target}|" "|[\textit{flags}]%
[|\def\jobname{|\textit{dest}|}|]|\input{|\textit{main}|}"|
\end{center}

%%%%%%%%%%%%%%%%%%%%%%%%%%%%%%%%%%%%%%%%%%%%%%%%%%%%%%%%%%%%%%%%%%%%%%%%%%%%%%%%
\subsection{Manual Code}
\label{sec:manual}

In case one cannot be certain whether the definitions file |childdoc.def|
is installed on the target \TeX{} distribution
and one prefers not to ship it,
it is conceivable to paste a few relevant commands into the sources.

To that end, drop all statements |\input{childdoc.def}|
and perform the replacements as outlined below.
Instead of |\childdocmain{|\textit{main}|}| add the following code
to the top of the main file:
%
\begin{center}
\begin{tabular}{l}
|\||ifdefined\childdocname\endinput\||fi\newif\ifchilddoc|\\
|\edef\childdocname{\scantokens\expandafter{\jobname\noexpand}}|\\
|\def\childdocmain{|\textit{main}|}\||ifx\childdocmain\childdocname\||else|\\
|\childdoctrue\includeonly{\childdocname}\let\jobname\childdocmain\||fi|\\
\end{tabular}
\end{center}
%
Instead of |\childdocof{|\textit{main}|}| just include the main file
at the top of each child file:
%
\begin{center}
|\input{|\textit{main}|}|
\end{center}
%
A simple redirection |\childdocforward{|\textit{dest}|}| is achieved by:
%
\begin{center}
|\def\jobname{|\textit{dest}|}\input{\jobname}|
\end{center}
%
The redirection with prefix
|\childdocforwardprefix[|\textit{prefix}|]{|\textit{dest}|}|
is accomplished by:
%
\begin{center}
\begin{tabular}{l}
|{\edef\jobname{\scantokens\expandafter{\jobname\noexpand}}|\\
|\def\redirectjob |\textit{prefix}|#1~~~{\gdef\jobname{|\textit{dest}|#1}}|\\
|\expandafter\redirectjob\jobname~~~}\input{\jobname}|
\end{tabular}
\end{center}

In an alternative approach,
child documents can be compiled by a specific command line
without additional code or specific definitions:
%
\begin{center}
|... -jobname "|\textit{target}|" "|[\textit{flags}]%
|\includeonly{|\textit{dest}|}\input{|\textit{main}|}"|
\end{center}
%

%%%%%%%%%%%%%%%%%%%%%%%%%%%%%%%%%%%%%%%%%%%%%%%%%%%%%%%%%%%%%%%%%%%%%%%%%%%%%%%%
%%%%%%%%%%%%%%%%%%%%%%%%%%%%%%%%%%%%%%%%%%%%%%%%%%%%%%%%%%%%%%%%%%%%%%%%%%%%%%%%
\section{Information}

%%%%%%%%%%%%%%%%%%%%%%%%%%%%%%%%%%%%%%%%%%%%%%%%%%%%%%%%%%%%%%%%%%%%%%%%%%%%%%%%
\subsection{Copyright}

Copyright \copyright{} 2017--2018 Niklas Beisert

This work may be distributed and/or modified under the
conditions of the \LaTeX{} Project Public License, either version 1.3
of this license or (at your option) any later version.
The latest version of this license is in
  \url{http://www.latex-project.org/lppl.txt}
and version 1.3 or later is part of all distributions of \LaTeX{}
version 2005/12/01 or later.

This work has the LPPL maintenance status `maintained'.

The Current Maintainer of this work is Niklas Beisert.

This work consists of the files |README.txt|, |childdoc.ins| and |childdoc.dtx|
as well as the derived files |childdoc.def|, |cdocsamp.tex|
with |cdocsch1.tex|, |cdocsch2.tex|, |cdocspt3.tex|, |cdocspt4.tex|,
|cdocsdrf.tex|, |cdocsfn1.tex|, |cdocsfn2.tex|
as well as |childdoc.pdf|.

%%%%%%%%%%%%%%%%%%%%%%%%%%%%%%%%%%%%%%%%%%%%%%%%%%%%%%%%%%%%%%%%%%%%%%%%%%%%%%%%
\subsection{Files and Installation}

The package consists of the files:
%
\begin{center}
\begin{tabular}{ll}
    |README.txt|   & readme file \\
    |childdoc.ins| & installation file \\
    |childdoc.dtx| & source file \\
    |childdoc.def| & definition file \\
    |cdocsamp.tex| & sample main file \\
    |cdocsch1.tex| & sample include file \\
    |cdocsch2.tex| & sample include file \\
    |cdocspt3.tex| & sample part file \\
    |cdocspt4.tex| & sample part file \\
    |cdocsdrf.tex| & sample redirection file \\
    |cdocsfn1.tex| & sample redirection file \\
    |cdocsfn2.tex| & sample redirection file \\
    |childdoc.pdf| & manual
\end{tabular}
\end{center}
%
The distribution consists of the files
|README.txt|, |childdoc.ins| and |childdoc.dtx|.
%
\begin{itemize}
\item
Run (pdf)\LaTeX{} on |childdoc.dtx|
to compile the manual |childdoc.pdf| (this file).
\item
Run \LaTeX{} on |childdoc.ins| to create the definitions file |childdoc.def|
and the sample |cdocsamp.tex| with include files
|cdocsch1.tex|, |cdocsch2.tex|, |cdocspt3.tex|, |cdocspt4.tex|,
|cdocsdrf.tex|, |cdocsfn1.tex|, |cdocsfn2.tex|.
Then copy the file |childdoc.def| to an appropriate directory of your \LaTeX{}
distribution, e.g.\ \textit{texmf-root}|/tex/latex/childdoc|.
\end{itemize}

%%%%%%%%%%%%%%%%%%%%%%%%%%%%%%%%%%%%%%%%%%%%%%%%%%%%%%%%%%%%%%%%%%%%%%%%%%%%%%%%
\subsection{Related CTAN Packages}

There are several other packages which offer a similar functionality:
%
\begin{itemize}
\item
The packages
\href{http://ctan.org/pkg/docmute}{\textsf{docmute}},
\href{http://ctan.org/pkg/includex}{\textsf{includex}} and
\href{http://ctan.org/pkg/standalone}{\textsf{standalone}}
provide commands to include only the document body of
a child file thus allowing both files to be compiled individually.
\item
The packages \href{http://ctan.org/pkg/subdocs}{\textsf{subdocs}}
and \href{http://ctan.org/pkg/subfiles}{\textsf{subfiles}}
provide structures in which the main and child documents can be
encapsulated and allowing them to be compiled individually.
The inclusion mechanism is different from the conventional |\include|.
\item
The package \href{http://ctan.org/pkg/combine}{\textsf{combine}}
is an elaborate solution to combine several documents into one.
\end{itemize}
%
See also the CTAN topic \href{http://ctan.org/topic/subdocs}{\textsf{subdocs}}
for further related packages.
The present package differs from the above solutions in that
a document structure constructed with the conventional |\include| mechanism
just needs two extra commands at the top of every file
such that all constituent files can be compiled individually.

%%%%%%%%%%%%%%%%%%%%%%%%%%%%%%%%%%%%%%%%%%%%%%%%%%%%%%%%%%%%%%%%%%%%%%%%%%%%%%%%
%\subsection{Feature Suggestions}
%
%The following is a list of features which may be useful for future
%versions of this package:
%%
%\begin{itemize}
%\item
%\ldots
%\end{itemize}

%%%%%%%%%%%%%%%%%%%%%%%%%%%%%%%%%%%%%%%%%%%%%%%%%%%%%%%%%%%%%%%%%%%%%%%%%%%%%%%%
\subsection{Revision History}

%%%%%%%%%%%%%%%%%%%%%%%%%%%%%%%%%%%%%%%%
\paragraph{v2.0:} 2018/12/30

\begin{itemize}
\item
immediate forward processing
\item
added |\childdocby| mechanism
\item
manual restructured
\end{itemize}

%%%%%%%%%%%%%%%%%%%%%%%%%%%%%%%%%%%%%%%%
\paragraph{v1.6:} 2018/01/17

\begin{itemize}
\item
application for development of include files
\item
corrections to manual
\end{itemize}

%%%%%%%%%%%%%%%%%%%%%%%%%%%%%%%%%%%%%%%%
\paragraph{v1.5:} 2017/05/21

\begin{itemize}
\item
more complete structuring introduced
\item
|\childdocof| introduced
\item
|\childdoc| renamed to |\childdocmain|
\item
|\childredirect| renamed to |\childdocforward| and |\childdocforwardprefix|
and functionality expanded
\end{itemize}

%%%%%%%%%%%%%%%%%%%%%%%%%%%%%%%%%%%%%%%%
\paragraph{v1.0:} 2017/04/27

\begin{itemize}
\item
manual and install package
\item
first version published on CTAN
\end{itemize}

%%%%%%%%%%%%%%%%%%%%%%%%%%%%%%%%%%%%%%%%
\paragraph{v0.6:} 2017/04/26

\begin{itemize}
\item
redirection mechanism added
\end{itemize}

%%%%%%%%%%%%%%%%%%%%%%%%%%%%%%%%%%%%%%%%
\paragraph{v0.5:} 2017/04/26

\begin{itemize}
\item
functionality in definition file
\end{itemize}


%%%%%%%%%%%%%%%%%%%%%%%%%%%%%%%%%%%%%%%%%%%%%%%%%%%%%%%%%%%%%%%%%%%%%%%%%%%%%%%%
%%%%%%%%%%%%%%%%%%%%%%%%%%%%%%%%%%%%%%%%%%%%%%%%%%%%%%%%%%%%%%%%%%%%%%%%%%%%%%%%
%%%%%%%%%%%%%%%%%%%%%%%%%%%%%%%%%%%%%%%%%%%%%%%%%%%%%%%%%%%%%%%%%%%%%%%%%%%%%%%%
\appendix

\settowidth\MacroIndent{\rmfamily\scriptsize 000\ }

 \DocInput{childdoc.dtx}

\end{document}
%</driver>
% \fi
%
% %%%%%%%%%%%%%%%%%%%%%%%%%%%%%%%%%%%%%%%%%%%%%%%%%%%%%%%%%%%%%%%%%%%%%%%%%%%%%%
% %%%%%%%%%%%%%%%%%%%%%%%%%%%%%%%%%%%%%%%%%%%%%%%%%%%%%%%%%%%%%%%%%%%%%%%%%%%%%%
% \section{Sample}
%\iffalse
%<*samplemain>
%\fi
%
% The following presents a sample document
% with two chapters, two parts, a title page,
% a compile flag as well as three forwarding files to set the flag.
% It consists of eight |.tex| files:
% \begin{center}
% \begin{tabular}{ll}
% |cdocsamp.tex|&main file\\
% |cdocsch1.tex|&include file for chapter 1\\
% |cdocsch2.tex|&include file for chapter 2\\
% |cdocspt3.tex|&include file for part 3\\
% |cdocspt4.tex|&include file for part 4\\
% |cdocsdrf.tex|&forwarding file for main file in draft mode\\
% |cdocsfi1.tex|&forwarding file for final version of chapter 1\\
% |cdocsfi2.tex|&forwarding file for final version of chapter 2\\
% \end{tabular}
% \end{center}
% Each of the eight files can be compiled directly by the \LaTeX{} compiler.
%
% %%%%%%%%%%%%%%%%%%%%%%%%%%%%%%%%%%%%%%
% \paragraph{Main File.}
%
% The main file is called |cdocsamp.tex|.
%
% Load the \textsf{childdoc} definitions and
% declare the filename for the main document:
%    \begin{macrocode}
\input{childdoc.def}
\childdocmain{}
%    \end{macrocode}

% Optional override for |\version| flag:
%    \begin{macrocode}
%%\ifchilddoc\else\providecommand{\version}{draft}\fi
%    \end{macrocode}

% Define the default values for the |\version| flag
% (|final| for the main file and |draft| for childs):
%    \begin{macrocode}
\ifchilddoc
\providecommand{\version}{draft}
\else
\providecommand{\version}{final}
\fi
%    \end{macrocode}

% Load the standard document class:
%    \begin{macrocode}
\documentclass[12pt]{article}
%    \end{macrocode}

% Start the document body:
%    \begin{macrocode}
\begin{document}
%    \end{macrocode}

% Declare a title page.
% Print title, part of document being processed and version flag:
%    \begin{macrocode}
\addtocounter{page}{-1}
\begin{center}
{\LARGE\bfseries{}childdoc example\par}
\vspace{1cm}
\ifchilddoc
\ifchilddocmanual part\else chapter\fi:
`\childdocname' of `\childdocjob'\par
\else
main document: `\childdocjob'\par
\fi
version: \version\par
\end{center}
\newpage
%    \end{macrocode}

% Manually include selected file,
% otherwise process as usual:
%    \begin{macrocode}
\ifchilddocmanual
\section*{part `\childdocname'}
\input{\childdocname}
\else
%    \end{macrocode}

% Include the two chapters:
%    \begin{macrocode}
\include{cdocsch1}
\include{cdocsch2}
%    \end{macrocode}

% Include the two parts unless only chapters should be displayed:
%    \begin{macrocode}
\ifchilddoc\else
\section{part three}
\input{cdocspt3}
\section{part four}
\input{cdocspt4}
\fi
%    \end{macrocode}

% Process as usual until here:
%    \begin{macrocode}
\fi
%    \end{macrocode}

% End of document body:
%    \begin{macrocode}
\end{document}
%    \end{macrocode}
%\iffalse
%</samplemain>
%\fi
%
% %%%%%%%%%%%%%%%%%%%%%%%%%%%%%%%%%%%%%%
% \paragraph{Chapter Include Files.}
%
% The include files are called |cdocsch1.tex| and |cdocsch2.tex|.
%
%\iffalse
%<*samplechap1|samplechap2>
%\fi

% Optional override for |\version| flag:
%    \begin{macrocode}
%%\providecommand{\version}{final}
%    \end{macrocode}

% Include the main document:
%    \begin{macrocode}
\input{childdoc.def}
\childdocof{cdocsamp}
%    \end{macrocode}

%\iffalse
%</samplechap1|samplechap2>
%\fi
%
%\iffalse
%<*samplechap1>
%\fi
% Some text for chapter 1:
%    \begin{macrocode}
\section{one}
some text in chapter one
%    \end{macrocode}

%\iffalse
%</samplechap1>
%\fi
% Some text for chapter 2:
%\iffalse
%<*samplechap2>
%\fi
%    \begin{macrocode}
\section{two}
more text in chapter two
%    \end{macrocode}

%\iffalse
%</samplechap2>
%\fi
%
% %%%%%%%%%%%%%%%%%%%%%%%%%%%%%%%%%%%%%%
% \paragraph{Part Include Files.}
%
% The include files are called |cdocspt3.tex| and |cdocspt4.tex|.
%
%\iffalse
%<*samplepart3|samplepart4>
%\fi

% Optional override for |\version| flag:
%    \begin{macrocode}
%%\providecommand{\version}{final}
%    \end{macrocode}

% Include the main document:
%    \begin{macrocode}
\input{childdoc.def}
\childdocby{cdocsamp}
%    \end{macrocode}

%\iffalse
%</samplepart3|samplepart4>
%\fi
%
%\iffalse
%<*samplepart3>
%\fi
% Some text for part 3:
%    \begin{macrocode}
some text in part three
%    \end{macrocode}

%\iffalse
%</samplepart3>
%\fi
% Some text for part 4:
%\iffalse
%<*samplepart4>
%\fi
%    \begin{macrocode}
more text in part four
%    \end{macrocode}

%\iffalse
%</samplepart4>
%\fi
%
% %%%%%%%%%%%%%%%%%%%%%%%%%%%%%%%%%%%%%%
% \paragraph{Forwarding for a Complete Draft.}
%
% The following forwarding file |cdocsdrf.tex|
% compiles the main document in draft mode:
%\iffalse
%<*sampledraft>
%\fi
%    \begin{macrocode}
\def\version{draft}
\input{childdoc.def}
\childdocforward{cdocsamp}
%    \end{macrocode}

%\iffalse
%</sampledraft>
%\fi
%
% %%%%%%%%%%%%%%%%%%%%%%%%%%%%%%%%%%%%%%
% \paragraph{Forwarding for Final Version of the Chapters.}
%
% The following forwarding files |cdocsfn1.tex| and |cdocsfn2.tex|
% (with identical content)
% compile the final versions of the child documents
% |cdocsch1.tex| and |cdocsch2.tex|, respectively:
%\iffalse
%<*samplefinal>
%\fi
%    \begin{macrocode}
\def\version{final}
\input{childdoc.def}
\childdocforwardprefix[cdocsamp]{cdocsfn}{cdocsch}
%    \end{macrocode}

%\iffalse
%</samplefinal>
%\fi
%
% %%%%%%%%%%%%%%%%%%%%%%%%%%%%%%%%%%%%%%
% \paragraph{Command Line Processing.}
%
% The following three command lines generate the output files
% |cdocscld|, |cdocscl1| and |cdocscl2|
% which should be identical to
% |cdocsdrf|, |cdocsch1| and |cdocsfn2|, respectively:
% \begin{center}
% \begin{tabular}{l}
% |latex -jobname cdocscld \|\\
% |  "\def\version{draft}\input{childdoc.def}\childdocforward{cdocsamp}"|\\
% |latex -jobname cdocscl1 \|\\
% |  "\input{childdoc.def}\childdocforward[cdocsamp]{cdocsch1}"|\\
% |latex -jobname cdocscl2 \|\\
% |  "\def\version{final}\input{childdoc.def}\childdocforward{cdocsch2}"|
% \end{tabular}
% \end{center}
% Note that the trailing backslash on each first line
% merely continues the input to the second line
% (for convenient cut ant paste).
% Furthermore, the command |latex| can be replaced by any
% of its alternative versions such as |pdflatex|.
%
% %%%%%%%%%%%%%%%%%%%%%%%%%%%%%%%%%%%%%%%%%%%%%%%%%%%%%%%%%%%%%%%%%%%%%%%%%%%%%%
% %%%%%%%%%%%%%%%%%%%%%%%%%%%%%%%%%%%%%%%%%%%%%%%%%%%%%%%%%%%%%%%%%%%%%%%%%%%%%%
% \section{Implementation}
%\iffalse
%<*package>
%\fi
%
% This section describes the definitions file |childdoc.def|.

% The definitions cannot be loaded using |\usepackage| or |\RequirePackage|
% which has a mechanism to prevent loading a style file more than once.
% When loading the definitions by means of |\input|
% multiple instances have to be prevented manually:
%\iffalse
%This code needs to be before the `\ProvidesFile' directive
%which is defined at the beginning of this file.
%Therefore it is also placed there and commented out here.
%</package>
%<*discard>
%\fi
%    \begin{macrocode}
\ifdefined\childdocmain\endinput\fi
%    \end{macrocode}
%\iffalse
%</discard>
%<*package>
%\fi
%
% \macro{\ifchilddoc}
% \macro{\ifchilddocmanual}
% The conditional |\ifchilddoc| tells whether a
% child (true) or main (false) document is being compiled.
% The conditional |\ifchilddocmanual| tells whether
% the |\includeonly| mechanism is used (false) or
% the selection of child files must be performed manually (true).
% The definitions initialise to false:
%    \begin{macrocode}
\newif\ifchilddoc
\newif\ifchilddocmanual
%    \end{macrocode}

% \macro{\childdocname}
% \macro{\childdocjob}
% The macro |\childdocname| stores the name of the main document
% to be compiled. The macro |\childdocjob| stores the name of
% the document on which the \LaTeX{} compiler was originally invoked.
% The content of |\jobname| cannot be compared
% to filenames specified in the source due to different catcodes.
% The following code rescans |\jobname|, stores the result
% in |\childdocname| and saves a copy in |\childdocjob|:
%    \begin{macrocode}
\edef\childdocname{\scantokens\expandafter{\jobname\noexpand}}
\let\childdocjob\childdocname
%    \end{macrocode}

% \macro{\childdocdisable}
% The macro |\childdocdisable| prevents the main file
% from being processed more than once.
% At this stage, the main document command |\childdocmain|
% is assumed to be called once again where it should do nothing.
% Any subsequent call to it should prevent
% a secondary processing of the main document
% It overwrites the forwarding commands
% |\childdocof| and |\childdocforward|
% with empty macros to prevent further inclusions of the main document:
%    \begin{macrocode}
\newcommand{\childdocdisable}
{
  \renewcommand{\childdocmain}[1]{\renewcommand{\childdocmain}[1]{\endinput}}
  \renewcommand{\childdocof}[1]{}
  \renewcommand{\childdocby}[2][]{}
  \renewcommand{\childdocforward}[2][]{}
  \renewcommand{\childdocdisable}{}
}
%    \end{macrocode}

% \macro{\childdocmain}
% The macro |\childdocmain| is to be called at the top of the main file
% with nothing or the main filename (without extension) as argument.
% First, it breaks loops.
% If the argument is not empty and does not match |\childdocname|
% (which is set by the first inclusion of |childdoc.def|),
% |\ifchilddoc| is set to true, |\includeonly| is applied to the child file
% and |\jobname| is set to the main file
% (for proper handling of |.aux| files):
%    \begin{macrocode}
\newcommand{\childdocmain}[1]
{
  \childdocdisable\childdocmain{}
  \if?#1?\else
    \begingroup
      \def\childdoctmp{#1}
      \ifx\childdoctmp\childdocname
        \def\childdoctmp{}
      \else
        \def\childdoctmp
        {
          \childdoctrue
          \includeonly{\childdocname}
          \def\childdocjob{#1}
          \def\jobname{#1}
        }
      \fi
      \expandafter
    \endgroup
    \childdoctmp
  \fi
}
%    \end{macrocode}

% \macro{\childdocof}
% The command |\childdocof| redirects
% compilation to the main file |#1|.
%    \begin{macrocode}
\newcommand{\childdocof}[1]
{
  \childdocdisable
  \childdoctrue
  \includeonly{\childdocname}
  \def\jobname{#1}
  \def\childdocjob{#1}
  \input{#1}
}
%    \end{macrocode}

% \macro{\childdocby}
% The command |\childdocby| ....
%    \begin{macrocode}
\newcommand{\childdocby}[2][]
{
  \childdocdisable
  \childdoctrue
  \childdocmanualtrue
  \if?#1?\else
    \def\jobname{#2}
  \fi
  \def\childdocjob{#2}
  \input{#2}
  \endinput
}
%    \end{macrocode}

% \macro{\childdocforward}
% The command |\childdocforward| redirects
% compilation to the main file or
% (if the optional argument is given) a child file.
% Parameters are set as if the main file
% or a child file starting with |\childdocof| was compiled.
% Then compilation is handed over to the main file:
%    \begin{macrocode}
\newcommand{\childdocforward}[2][]
{
  \begingroup
    \if?#1?
      \def\childdoctmp
      {
        \def\childdocname{#2}
        \def\childdocjob{#2}
        \def\jobname{#2}
        \input{#2}
        \endinput
      }
    \else
      \def\childdoctmp
      {
        \childdocdisable
        \def\childdocname{#2}
        \childdoctrue
        \includeonly{#2}
        \def\childdocjob{#1}
        \def\jobname{#1}
        \input{#1}
        \endinput
      }
    \fi
    \expandafter
  \endgroup
  \childdoctmp
}
%    \end{macrocode}

% \macro{\childdocforwardprefix}
% The command |\childdocforwardprefix| redirects
% compilation to the main or a child file by means of a pattern.
% The prefix |#1| in the current filename is replaced by |#2|
% and the suffix of the current filename is kept
% (it is assumed that the filename does not contain the substring `|~~~|'
% which is used as a delimiter).
% Compilation is handed over to the new file by |\childdocforward|:
%    \begin{macrocode}
\newcommand{\childdocforwardprefix}[3][]
{
  \begingroup
    \def\childdocextract #2##1~~~{\def\childdoctmp{\childdocforward[#1]{#3##1}}}
    \expandafter\childdocextract\childdocname~~~
    \expandafter
  \endgroup
  \childdoctmp
}
%    \end{macrocode}

% \macro{\childdoc}
% The deprecated macro |\childdoc| is a legacy version of |\childdocmain|:
%    \begin{macrocode}
\newcommand{\childdoc}{\childdocmain}
%    \end{macrocode}

% \macro{\childdocredirect}
% The deprecated macro |\childdocredirect| is a legacy version
% of |\childdocforward| and |\childdocforwardprefix|:
%    \begin{macrocode}
\newcommand{\childdocredirect}[2][]
{
  \begingroup
    \if?#1?
      \def\childdoctmp{\childdocforward{#2}}
    \else
      \def\childdoctmp{\childdocforwardprefix{#1}{#2}}
    \fi
    \expandafter
  \endgroup
  \childdoctmp
}
%    \end{macrocode}

%\iffalse
%</package>
%\fi
%
\endinput
|\\
|\childdocforward{|\textit{main}|}|
\end{tabular}
\end{center}
%
Likewise, the following files |final|\textit{nn}|.tex|
compile the final version of the child document
|child|\textit{nn}|.tex|:
%
\begin{center}
\begin{tabular}{l}
|\def\version{final}|\\
|% \iffalse
%
% childdoc.dtx Copyright (C) 2017-2018 Niklas Beisert
%
% This work may be distributed and/or modified under the
% conditions of the LaTeX Project Public License, either version 1.3
% of this license or (at your option) any later version.
% The latest version of this license is in
%   http://www.latex-project.org/lppl.txt
% and version 1.3 or later is part of all distributions of LaTeX
% version 2005/12/01 or later.
%
% This work has the LPPL maintenance status `maintained'.
%
% The Current Maintainer of this work is Niklas Beisert.
%
% This work consists of the files childdoc.dtx and childdoc.ins
% and the derived files childdoc.def and cdocsamp.tex with
% cdocsch1.tex, cdocsch2.tex, cdocsdrf.tex, cdocsfn1.tex, cdocsfn2.tex.
%
%<package>\ifdefined\childdocmain\endinput\fi
%<package>\ProvidesFile{childdoc.def}[2018/12/30 v2.0 child document driver]
%<samplemain>\ProvidesFile{cdocsamp.tex}[2018/12/30 v2.0 sample for childdoc]
%<*driver>
%\ProvidesFile{childdoc.drv}[2018/12/30 v2.0 childdoc reference manual file]
\PassOptionsToClass{10pt,a4paper}{article}
\documentclass{ltxdoc}

\usepackage[margin=35mm]{geometry}
\usepackage{hyperref}
\usepackage{hyperxmp}
\usepackage[usenames]{color}

\hypersetup{colorlinks=true}
\hypersetup{pdfstartview=FitH}
\hypersetup{pdfpagemode=UseNone}
\hypersetup{pdfsource={}}
\hypersetup{pdflang={en-UK}}
\hypersetup{pdfcopyright={Copyright 2017-2018 Niklas Beisert.
  This work may be distributed and/or modified under the
  conditions of the LaTeX Project Public License, either version 1.3
  of this license or (at your option) any later version.}}
\hypersetup{pdflicenseurl={http://www.latex-project.org/lppl.txt}}
\hypersetup{pdfcontactaddress={ETH Zurich, ITP, HIT K,
  Wolfgang-Pauli-Strasse 27}}
\hypersetup{pdfcontactpostcode={8093}}
\hypersetup{pdfcontactcity={Zurich}}
\hypersetup{pdfcontactcountry={Switzerland}}
\hypersetup{pdfcontactemail={nbeisert@itp.phys.ethz.ch}}
\hypersetup{pdfcontacturl={http://people.phys.ethz.ch/\xmptilde nbeisert/}}

\newcommand{\secref}[1]{\hyperref[#1]{section \ref*{#1}}}

\parskip1ex
\parindent0pt
\let\olditemize\itemize
\def\itemize{\olditemize\parskip0pt}

\begin{document}

\title{The \textsf{childdoc} Package}
\hypersetup{pdftitle={The childdoc Package}}
\author{Niklas Beisert\\[2ex]
  Institut f\"ur Theoretische Physik\\
  Eidgen\"ossische Technische Hochschule Z\"urich\\
  Wolfgang-Pauli-Strasse 27, 8093 Z\"urich, Switzerland\\[1ex]
  \href{mailto:nbeisert@itp.phys.ethz.ch}
  {\texttt{nbeisert@itp.phys.ethz.ch}}}
\hypersetup{pdfauthor={Niklas Beisert}}
\hypersetup{pdfsubject={Manual for the LaTeX2e Package childdoc}}
\date{30 December 2018, \textsf{v2.0}}
\maketitle

\begin{abstract}\noindent
\textsf{childdoc} is a \LaTeXe{} package
that enables the direct compilation
of document sections included by |\include|
to individual files.
\end{abstract}

\begingroup
\parskip0ex
\tableofcontents
\endgroup

%%%%%%%%%%%%%%%%%%%%%%%%%%%%%%%%%%%%%%%%%%%%%%%%%%%%%%%%%%%%%%%%%%%%%%%%%%%%%%%%
%%%%%%%%%%%%%%%%%%%%%%%%%%%%%%%%%%%%%%%%%%%%%%%%%%%%%%%%%%%%%%%%%%%%%%%%%%%%%%%%
\section{Introduction}

\LaTeX{} provides a mechanism to structure a large document (such as a book)
into a main file and several child files (containing the chapters)
using the |\include| command.
This mechanism is beneficial for documents
which span hundreds of pages in order to
make the source file(s) more manageable.
Moreover, compilation can be restricted to
selected child files by means of the |\includeonly| command.
The latter feature can be used to reduce the compilation time while editing
(this was significantly more useful in the earlier days of \LaTeX{})
or to generate a smaller document which is easier to navigate.
Another application of |\includeonly| is to generate
documents consisting of selected parts of the complete document.

However, there are a few drawbacks of the plain |\include| mechanism:
\begin{itemize}
\item
The child files cannot be compiled on their own,
they can only be compiled via the main file.
A naive editing environment
(such as a text editor with an option
to have the current file processed by \LaTeX)
may require one to switch to the main file before compiling;
attempting to compile the child file produces errors.
\item
The main file must be modified (each time)
to adjust the |\includeonly| command
to the present needs. This easily leaves the main file in a messy state.
\item
The generated document will always carry the filename
of the main document. This is inconvenient if
several child files are to be compiled and
to be kept for distribution.
\end{itemize}

The present package provides a simple interface
to make child files individually compilable by \LaTeX{}.
Compiling a child file then has the same effect as compiling
the main file with an |\includeonly| command
to select the appropriate child.
Moreover the generated document will carry the name of the child
rather than the main file.
This resolves all three above issues.

This feature is meant to make the editing of books,
thesis documents and lecture notes somewhat more convenient.
However, the package can also be used efficiently for
composing a series of documents (such as exercise sheets)
which are typically distributed individually.
It then assists the author in generating the individual documents
(potentially in different versions)
as well as a document containing the collected series.
Another application is in developing style files
or other kinds of included material
where compilation of the style file could redirect
to a sample or test file.

%%%%%%%%%%%%%%%%%%%%%%%%%%%%%%%%%%%%%%%%%%%%%%%%%%%%%%%%%%%%%%%%%%%%%%%%%%%%%%%%
%%%%%%%%%%%%%%%%%%%%%%%%%%%%%%%%%%%%%%%%%%%%%%%%%%%%%%%%%%%%%%%%%%%%%%%%%%%%%%%%
\section{Usage}

First of all, the package \textsf{childdoc} is \emph{not} a standard
\LaTeXe{} |.sty| style file! Therefore it needs to be invoked in
a non-standard way.

%%%%%%%%%%%%%%%%%%%%%%%%%%%%%%%%%%%%%%%%%%%%%%%%%%%%%%%%%%%%%%%%%%%%%%%%%%%%%%%%
\subsection{Included Files}
\label{sec:include}

%%%%%%%%%%%%%%%%%%%%%%%%%%%%%%%%%%%%%%%%
\DescribeMacro{\childdocmain}
To use the package, add the commands
\begin{center}
\begin{tabular}{l}
|\input{childdoc.def}|\\
|\childdocmain{}|\\
\end{tabular}
\end{center}
at the very top of the main \LaTeX{} file,
in particular \emph{before} the |\documentclass| statement!
The argument of |\childdocmain| should be left empty
(but it must be present).

%%%%%%%%%%%%%%%%%%%%%%%%%%%%%%%%%%%%%%%%
\DescribeMacro{\childdocof}
Furthermore, add the commands
\begin{center}
\begin{tabular}{l}
|\input{childdoc.def}|\\
|\childdocof{|\textit{main}|}|\\
\end{tabular}
\end{center}
at the top of every child file \textit{child}
which is included by |\include{|\textit{child}|}|
from within the main file
(or at least for those files to be compiled individually).
The argument \textit{main} must be the filename of the main file.

There are a couple of
considerations in setting up the main and child documents:

%%%%%%%%%%%%%%%%%%%%%%%%%%%%%%%%%%%%%%%%
\paragraph{Restrictions.}

Please note the following restrictions:
\begin{itemize}
\item
|\childdocmain| must be called with one argument \textit{main}
to ensure compatibility with earlier version of the package.
It must either be empty (|\childdocmain{}|)
or precisely match the filename of the main file in which it is specified.
See \secref{sec:detection} for further information.
\item
The filename \textit{main} must be specified without the |.tex| extension.
\item
The filename \textit{main} is case sensitive
(even in case-insensitive file systems)
due to internal string comparison.
\item
The argument \textit{main} should be fully expanded, it cannot be a macro.
\item
Subdirectories and special characters should be avoided in filenames.
\item
The command |\childdocmain{|\textit{main}|}| must be followed by a whitespace.
It should not be followed immediately by another command
or by a comment mark `|%|'.
This is because the \TeX{} parser reads the token immediately following
the argument of |\childdocmain| and puts it
at the beginning of every child section;
however, a white\-space is ignored.
\end{itemize}

%%%%%%%%%%%%%%%%%%%%%%%%%%%%%%%%%%%%%%%%
\paragraph{Content of Main File.}

It is advisable to place all content in the child files included by |\include|.
Any output contained in the main file will appear in all child documents
unless suppressed manually;
it cannot be suppressed automatically by the |\includeonly| directive
and thus should normally be avoided.
A method to include some content in the main file
by means of conditional processing is described in \secref{sec:conditional}.

%%%%%%%%%%%%%%%%%%%%%%%%%%%%%%%%%%%%%%%%
\paragraph{Page Numbering.}

When only a part of the document is compiled,
the appropriate numbering of pages
(as well as other status parameters)
is determined from the |.aux| files.
The latter contain information from previous passes.
However this information needs to propagate through
all intermediate child documents.
Therefore the page numbering in child documents may well
be inconsistent until the complete document is compiled at least once.

A useful (if unconventional) way to always ensure a consistent
page numbering is to restart the numbering in each child document
and denote the pages by `\textit{child}|.|\textit{page}'
where \textit{child} represents the chapter/section number of the child file.
This can be achieved by the command
|\numberwithin{page}{|\textit{child}|}|
of the \textsf{amsmath} package
where \textit{child} can be |chapter| or |section|
depending on the chosen structuring.
Alternatively, one can modify the macro |\thepage| appropriately
and reset the counter |page| at the start of each child file.

%%%%%%%%%%%%%%%%%%%%%%%%%%%%%%%%%%%%%%%%%%%%%%%%%%%%%%%%%%%%%%%%%%%%%%%%%%%%%%%%
\subsection{Conditional Processing}
\label{sec:conditional}

The package provides a mechanism to compile different versions
of a document. To customise the versions further some conditional processing
can come in handy to distinguish which version is being compiled.
The package provides two macros to describe the compilation context:

%%%%%%%%%%%%%%%%%%%%%%%%%%%%%%%%%%%%%%%%
\DescribeMacro{\ifchilddoc}
The conditional |\ifchilddoc| distinguishes between the compilation of
child documents and the main document:
%
\begin{center}
|\ifchilddoc |\textit{child-code}| |[|\||else |\textit{main-code}]| \||fi|
\end{center}

%%%%%%%%%%%%%%%%%%%%%%%%%%%%%%%%%%%%%%%%
\DescribeMacro{\childdocname}
\DescribeMacro{\childdocjob}
The macro |\childdocname| contains the filename (without extension)
of the main or child file being processed.
Note that |\childdocjob| will always contain the name of the main file.

%%%%%%%%%%%%%%%%%%%%%%%%%%%%%%%%%%%%%%%%
\paragraph{Title Page.}

Conditional processing can be used to include a title or banner page
in the main document when proper precautions are taken.
Importantly, the code in the main file should ensure that the page counter
(as well as other status parameters which are stored in the |.aux| files)
takes the same value after the conditional processing.
Otherwise the page numbers may take divergent values
depending on which part is compiled.

For example, a title page could be declared by:
%
\begin{center}
\begin{tabular}{l}
|\ifchilddoc\||else|\\
|\addtocounter{page}{-1}|\\
\textit{code for title page}\\
|\newpage|\\
|\||fi|
\end{tabular}
\end{center}
%
A banner page for the child documents can be generated by:
%
\begin{center}
\begin{tabular}{l}
|\ifchilddoc|\\
|\addtocounter{page}{-1}|\\
\textit{code for banner page}\\
|\newpage|\\
|\||fi|
\end{tabular}
\end{center}
%
Here one could write a message such as:
\begin{center}
|This is the part \childdocname{} of \childdocjob{}.|
\end{center}

%%%%%%%%%%%%%%%%%%%%%%%%%%%%%%%%%%%%%%%%%%%%%%%%%%%%%%%%%%%%%%%%%%%%%%%%%%%%%%%%
\subsection{Flags}
\label{sec:flags}

The package makes it easy to generate different versions
of the main or child documents.
To this end compilation flags can be defined
and assigned different default values.
They will be particularly useful in conjunction
with the forwarding mechanism described in \secref{sec:forward}.

For example, it may be useful to have a flag |\version|
which can be set to |draft| or |final|.
The document source will contain some conditional code
depending on the value of |\version|.
Suppose further, the flag should default to |final| for the main file
and to |draft| for child files
which is a natural assignment for editing the document.
This is achieved by placing the following code
in the preamble of the main document
(below the |\childdocmain| directive):
%
\begin{center}
\begin{tabular}{l}
|\ifchilddoc|\\
|\providecommand{\version}{draft}|\\
|\||else|\\
|\providecommand{\version}{final}|\\
|\||fi|
\end{tabular}
\end{center}
%
The definition by |\providecommand| makes sure
that previous definitions are not overwritten.
Further statements |\providecommand{\version}{...}|
can thus be added before the above code to override it.

For the main file, one might add a line
(between |\childdocmain| and the above block)
%
\begin{center}
|%\ifchilddoc\||else\providecommand{\version}{draft}\||fi|
\end{center}
%
which can be uncommented to produce a draft version.
Likewise one can add a line to the very top of a child file
(above the |\childdocof{|\textit{main}|}| directive)
%
\begin{center}
|%\providecommand{\version}{final}|
\end{center}
%
which can be uncommented to produce the final version of this child document.

%%%%%%%%%%%%%%%%%%%%%%%%%%%%%%%%%%%%%%%%%%%%%%%%%%%%%%%%%%%%%%%%%%%%%%%%%%%%%%%%
\subsection{Forwarding}
\label{sec:forward}

Different versions of the main or child documents
using compilation flags as described in \secref{sec:flags}
can be (permanently) stored in different files
for convenient compilation, viewing and distribution.
To this end, the package defines a command
to pass on compilation to a different file:

%%%%%%%%%%%%%%%%%%%%%%%%%%%%%%%%%%%%%%%%
\DescribeMacro{\childdocforward}
The command |\childdocforward| redirects processing to
another source file:
%
\begin{center}
\begin{tabular}{l}
|\input{childdoc.def}|\\
|\childdocforward[|\textit{main}|]{|\textit{dest}|}|\\
\end{tabular}
\end{center}
%
The argument \textit{dest} is the destination file
(without extension).
It should be the main file or one of the child files.
Note that further \textsf{childdoc} directives
such as |\childdocof| and |\childdocforward|
in the indicated file will be processed in this form.
The optional argument \textit{main}
passes on directly to the main file \textit{main}
while pretending to compile the child \textit{dest}.
This form behaves as if \textit{dest}
issues |\childdocof{|\textit{main}|}| right away,
and no further \textsf{childdoc} directives will be processed.

%%%%%%%%%%%%%%%%%%%%%%%%%%%%%%%%%%%%%%%%
\DescribeMacro{\...prefix}
In the alternative form |\childdocforwardprefix|,
%
\begin{center}
\begin{tabular}{l}
|\input{childdoc.def}|\\
|\childdocforwardprefix[|\textit{main}|]{|\textit{prefix}|}{|\textit{dest}|}|
\end{tabular}
\end{center}
%
the destination file is determined by a pattern
depending on the current file:
To make this work, the current file must be called
`{\textit{prefix}\hspace{0.2em}\textit{suffix}}'
with \textit{prefix} matching precisely the argument.
Processing is then passed on to the file
`{\textit{dest}\hspace{0.2em}\textit{suffix}}'.
Surely, the same effect is achieved by
directly specifying the
argument `{\textit{dest}\hspace{0.2em}\textit{suffix}}'
in the first form.
However, that requires to set up a different file
for each child. With the alternative form of the command
all these files can have exactly the same content
which simplifies setting them up and maintaining them.

For example, the following file |draft.tex|
with a compilation flag |\version| as described in \secref{sec:flags}
compiles the main document as a draft:
%
\begin{center}
\begin{tabular}{l}
|\def\version{draft}|\\
|\input{childdoc.def}|\\
|\childdocforward{|\textit{main}|}|
\end{tabular}
\end{center}
%
Likewise, the following files |final|\textit{nn}|.tex|
compile the final version of the child document
|child|\textit{nn}|.tex|:
%
\begin{center}
\begin{tabular}{l}
|\def\version{final}|\\
|\input{childdoc.def}|\\
|\childdocforwardprefix{final}{child}|
\end{tabular}
\end{center}
%

Note that when several versions of a main file and/or of each child file
are to be generated, it may be convenient to set up a |Makefile| or
shell script to automatise the process.

%%%%%%%%%%%%%%%%%%%%%%%%%%%%%%%%%%%%%%%%%%%%%%%%%%%%%%%%%%%%%%%%%%%%%%%%%%%%%%%%
\subsection{Command Line Processing}
\label{sec:commandline}

The effect of redirection files can also be achieved by invoking
the \LaTeX{} compiler with a more elaborate command line.
Most conveniently this should be done as part
of a shell script or a |Makefile|.

When using \textsf{childdoc} in the main file, the following
command lines effectively perform a redirection
(note that depending on the shell being used,
backslashes may have to be doubled: `|\|' $\to$ `|\\|'):
%
\begin{center}
|... -jobname "|\textit{target}|" |\\|"|[\textit{flags}]%
|\input{childdoc.def}\childdocforward[|\textit{main}|]{|\textit{dest}|}"|
\end{center}
%
Here \textit{target} is the name of the output file,
\textit{main} is the name of the main file
and \textit{dest} is the name of the main or child file to be processed
(all filenames without extensions).
The optional argument \textit{main} can be omitted
if \textit{main} matches \textit{dest}.
Optionally, compilation \textit{flags} can be defined via |\def| commands.
This command line makes the \TeX{} engine believe
it is compiling the file \textit{target}
whose content is specified as the latter parameter.
The provided code then forwards the processing to
\textit{main} or \textit{dest} as described in \secref{sec:forward}.

%%%%%%%%%%%%%%%%%%%%%%%%%%%%%%%%%%%%%%%%%%%%%%%%%%%%%%%%%%%%%%%%%%%%%%%%%%%%%%%%
\subsection{Include by Input}
\label{sec:input}

Including child documents by |\include| has some restrictions by design.
Most notably, the content of a child document always occupies
its own set of pages; pages cannot be shared between child documents.
Usually, this behaviour makes perfect sense
because each child document contain an essential part of the document.
However, in some situations it may be desirable to compose
a document from a collection of parts
without having mandatory page breaks between then.
For this case, the package
provides a mechanism to include parts
by |\input| which can also be processed individually.
However, by construction this mechanism
requires manual handling of the content to be output.

%%%%%%%%%%%%%%%%%%%%%%%%%%%%%%%%%%%%%%%%
\DescribeMacro{\ifchilddocmanual}
The main file should be prepared as usual, see \secref{sec:include}.
However, the document body must make a distinction
between processing of an individual part and of the main document, e.g.:
%
\begin{center}
\begin{tabular}{l}
|\ifchilddocmanual|\\
|\input{\childdocname}|\\
|\||else|\\
\textit{document body with }|\input{|\textit{part}|}|\\
|\||fi|
\end{tabular}
\end{center}
%
The conditional |\ifchilddocmanual| is true whenever
a part to be included by |\input| is being compiled,
and the name of the part is stored in |\childdocname|.

%%%%%%%%%%%%%%%%%%%%%%%%%%%%%%%%%%%%%%%%
\DescribeMacro{\childdocby}
Each part to be included by |\input| should start with:
%
\begin{center}
\begin{tabular}{l}
|\input{childdoc.def}|\\
|\childdocby{|\textit{main}|}|\\
\end{tabular}
\end{center}
%
The directive |\childdocby| is similar to |\childdocof|
described in \secref{sec:include},
but the subsequent selection of content must be done manually.
To that end, both |\ifchilddoc| and |\ifchilddocmanual|
will be true upon processing of a part,
and the name of the part is stored in |\childdocname|.
Note that |\jobname| will be set to the filename of the current part
so that each part receives an individual |.aux| file
that does not interfere with the |.aux| file(s) of the main document.
This behaviour can be altered by the alternative form
|\childdocby[*]{|\textit{main}|}| (with a non-empty optional argument)
which uses the |.aux| file of the main document
by setting |\jobname| to \textit{main}.

%%%%%%%%%%%%%%%%%%%%%%%%%%%%%%%%%%%%%%%%%%%%%%%%%%%%%%%%%%%%%%%%%%%%%%%%%%%%%%%%
\subsection{Driver Development}
\label{sec:driver}

The \textsf{childdoc} mechanism can also be use for the development
of definition files such as \LaTeX{} styles or classes.
This case differs from the above setup with multiple parts
included by |\include| in that no |\includeonly| should be invoked.
This can be achieved by starting the include file
(before |\ProvidesPackage|) with:
%
\begin{center}
\begin{tabular}{l}
|\input{childdoc.def}|\\
|\childdocforward{|\textit{main}|}|\\
\end{tabular}
\end{center}
%
or alternatively with:
%
\begin{center}
\begin{tabular}{l}
|\input{childdoc.def}|\\
|\childdocby{|\textit{main}|}|\\
\end{tabular}
\end{center}
%
Both forms have slightly different effects as described above.
The main file is prepared as usual, see \secref{sec:include}.

%%%%%%%%%%%%%%%%%%%%%%%%%%%%%%%%%%%%%%%%%%%%%%%%%%%%%%%%%%%%%%%%%%%%%%%%%%%%%%%%
\subsection{Legacy Detection}
\label{sec:detection}

The directive |\childdocmain| in the main file can detect
whether the complete document or merely a child is to be compiled
even without using the directive |\childdocof|.
This method is deprecated because it is less robust
and there is no compelling reason to use it;
it is merely provided for backward compatibility
and it may be removed in future versions.

If the detection mechanism is to be used,
it is mandatory to correctly specify
the filename of the main file as the argument of |\childdocmain|:
%
\begin{center}
\begin{tabular}{l}
|\input{childdoc.def}|\\
|\childdocmain{|\textit{main}|}|\\
\end{tabular}
\end{center}
%
If |\jobname| does not match the argument \textit{main} of |\childdocmain|,
it is assumed that |\jobname| points to the child file to be compiled.
When using |\childdocmain| with the main file specified as argument,
it suffices to start a child file
with just |\input{|\textit{main}|}|
without loading of the package and using |\childdocof|.
If instead all processing is done
with the appropriate \textsf{childdoc} directives,
the argument of \textit{main} of |\childdocmain| can be empty.

An alternative version of the command line processing described
in \secref{sec:commandline} using the detection mechanism reads:
%
\begin{center}
|... -jobname "|\textit{target}|" "|[\textit{flags}]%
[|\def\jobname{|\textit{dest}|}|]|\input{|\textit{main}|}"|
\end{center}

%%%%%%%%%%%%%%%%%%%%%%%%%%%%%%%%%%%%%%%%%%%%%%%%%%%%%%%%%%%%%%%%%%%%%%%%%%%%%%%%
\subsection{Manual Code}
\label{sec:manual}

In case one cannot be certain whether the definitions file |childdoc.def|
is installed on the target \TeX{} distribution
and one prefers not to ship it,
it is conceivable to paste a few relevant commands into the sources.

To that end, drop all statements |\input{childdoc.def}|
and perform the replacements as outlined below.
Instead of |\childdocmain{|\textit{main}|}| add the following code
to the top of the main file:
%
\begin{center}
\begin{tabular}{l}
|\||ifdefined\childdocname\endinput\||fi\newif\ifchilddoc|\\
|\edef\childdocname{\scantokens\expandafter{\jobname\noexpand}}|\\
|\def\childdocmain{|\textit{main}|}\||ifx\childdocmain\childdocname\||else|\\
|\childdoctrue\includeonly{\childdocname}\let\jobname\childdocmain\||fi|\\
\end{tabular}
\end{center}
%
Instead of |\childdocof{|\textit{main}|}| just include the main file
at the top of each child file:
%
\begin{center}
|\input{|\textit{main}|}|
\end{center}
%
A simple redirection |\childdocforward{|\textit{dest}|}| is achieved by:
%
\begin{center}
|\def\jobname{|\textit{dest}|}\input{\jobname}|
\end{center}
%
The redirection with prefix
|\childdocforwardprefix[|\textit{prefix}|]{|\textit{dest}|}|
is accomplished by:
%
\begin{center}
\begin{tabular}{l}
|{\edef\jobname{\scantokens\expandafter{\jobname\noexpand}}|\\
|\def\redirectjob |\textit{prefix}|#1~~~{\gdef\jobname{|\textit{dest}|#1}}|\\
|\expandafter\redirectjob\jobname~~~}\input{\jobname}|
\end{tabular}
\end{center}

In an alternative approach,
child documents can be compiled by a specific command line
without additional code or specific definitions:
%
\begin{center}
|... -jobname "|\textit{target}|" "|[\textit{flags}]%
|\includeonly{|\textit{dest}|}\input{|\textit{main}|}"|
\end{center}
%

%%%%%%%%%%%%%%%%%%%%%%%%%%%%%%%%%%%%%%%%%%%%%%%%%%%%%%%%%%%%%%%%%%%%%%%%%%%%%%%%
%%%%%%%%%%%%%%%%%%%%%%%%%%%%%%%%%%%%%%%%%%%%%%%%%%%%%%%%%%%%%%%%%%%%%%%%%%%%%%%%
\section{Information}

%%%%%%%%%%%%%%%%%%%%%%%%%%%%%%%%%%%%%%%%%%%%%%%%%%%%%%%%%%%%%%%%%%%%%%%%%%%%%%%%
\subsection{Copyright}

Copyright \copyright{} 2017--2018 Niklas Beisert

This work may be distributed and/or modified under the
conditions of the \LaTeX{} Project Public License, either version 1.3
of this license or (at your option) any later version.
The latest version of this license is in
  \url{http://www.latex-project.org/lppl.txt}
and version 1.3 or later is part of all distributions of \LaTeX{}
version 2005/12/01 or later.

This work has the LPPL maintenance status `maintained'.

The Current Maintainer of this work is Niklas Beisert.

This work consists of the files |README.txt|, |childdoc.ins| and |childdoc.dtx|
as well as the derived files |childdoc.def|, |cdocsamp.tex|
with |cdocsch1.tex|, |cdocsch2.tex|, |cdocspt3.tex|, |cdocspt4.tex|,
|cdocsdrf.tex|, |cdocsfn1.tex|, |cdocsfn2.tex|
as well as |childdoc.pdf|.

%%%%%%%%%%%%%%%%%%%%%%%%%%%%%%%%%%%%%%%%%%%%%%%%%%%%%%%%%%%%%%%%%%%%%%%%%%%%%%%%
\subsection{Files and Installation}

The package consists of the files:
%
\begin{center}
\begin{tabular}{ll}
    |README.txt|   & readme file \\
    |childdoc.ins| & installation file \\
    |childdoc.dtx| & source file \\
    |childdoc.def| & definition file \\
    |cdocsamp.tex| & sample main file \\
    |cdocsch1.tex| & sample include file \\
    |cdocsch2.tex| & sample include file \\
    |cdocspt3.tex| & sample part file \\
    |cdocspt4.tex| & sample part file \\
    |cdocsdrf.tex| & sample redirection file \\
    |cdocsfn1.tex| & sample redirection file \\
    |cdocsfn2.tex| & sample redirection file \\
    |childdoc.pdf| & manual
\end{tabular}
\end{center}
%
The distribution consists of the files
|README.txt|, |childdoc.ins| and |childdoc.dtx|.
%
\begin{itemize}
\item
Run (pdf)\LaTeX{} on |childdoc.dtx|
to compile the manual |childdoc.pdf| (this file).
\item
Run \LaTeX{} on |childdoc.ins| to create the definitions file |childdoc.def|
and the sample |cdocsamp.tex| with include files
|cdocsch1.tex|, |cdocsch2.tex|, |cdocspt3.tex|, |cdocspt4.tex|,
|cdocsdrf.tex|, |cdocsfn1.tex|, |cdocsfn2.tex|.
Then copy the file |childdoc.def| to an appropriate directory of your \LaTeX{}
distribution, e.g.\ \textit{texmf-root}|/tex/latex/childdoc|.
\end{itemize}

%%%%%%%%%%%%%%%%%%%%%%%%%%%%%%%%%%%%%%%%%%%%%%%%%%%%%%%%%%%%%%%%%%%%%%%%%%%%%%%%
\subsection{Related CTAN Packages}

There are several other packages which offer a similar functionality:
%
\begin{itemize}
\item
The packages
\href{http://ctan.org/pkg/docmute}{\textsf{docmute}},
\href{http://ctan.org/pkg/includex}{\textsf{includex}} and
\href{http://ctan.org/pkg/standalone}{\textsf{standalone}}
provide commands to include only the document body of
a child file thus allowing both files to be compiled individually.
\item
The packages \href{http://ctan.org/pkg/subdocs}{\textsf{subdocs}}
and \href{http://ctan.org/pkg/subfiles}{\textsf{subfiles}}
provide structures in which the main and child documents can be
encapsulated and allowing them to be compiled individually.
The inclusion mechanism is different from the conventional |\include|.
\item
The package \href{http://ctan.org/pkg/combine}{\textsf{combine}}
is an elaborate solution to combine several documents into one.
\end{itemize}
%
See also the CTAN topic \href{http://ctan.org/topic/subdocs}{\textsf{subdocs}}
for further related packages.
The present package differs from the above solutions in that
a document structure constructed with the conventional |\include| mechanism
just needs two extra commands at the top of every file
such that all constituent files can be compiled individually.

%%%%%%%%%%%%%%%%%%%%%%%%%%%%%%%%%%%%%%%%%%%%%%%%%%%%%%%%%%%%%%%%%%%%%%%%%%%%%%%%
%\subsection{Feature Suggestions}
%
%The following is a list of features which may be useful for future
%versions of this package:
%%
%\begin{itemize}
%\item
%\ldots
%\end{itemize}

%%%%%%%%%%%%%%%%%%%%%%%%%%%%%%%%%%%%%%%%%%%%%%%%%%%%%%%%%%%%%%%%%%%%%%%%%%%%%%%%
\subsection{Revision History}

%%%%%%%%%%%%%%%%%%%%%%%%%%%%%%%%%%%%%%%%
\paragraph{v2.0:} 2018/12/30

\begin{itemize}
\item
immediate forward processing
\item
added |\childdocby| mechanism
\item
manual restructured
\end{itemize}

%%%%%%%%%%%%%%%%%%%%%%%%%%%%%%%%%%%%%%%%
\paragraph{v1.6:} 2018/01/17

\begin{itemize}
\item
application for development of include files
\item
corrections to manual
\end{itemize}

%%%%%%%%%%%%%%%%%%%%%%%%%%%%%%%%%%%%%%%%
\paragraph{v1.5:} 2017/05/21

\begin{itemize}
\item
more complete structuring introduced
\item
|\childdocof| introduced
\item
|\childdoc| renamed to |\childdocmain|
\item
|\childredirect| renamed to |\childdocforward| and |\childdocforwardprefix|
and functionality expanded
\end{itemize}

%%%%%%%%%%%%%%%%%%%%%%%%%%%%%%%%%%%%%%%%
\paragraph{v1.0:} 2017/04/27

\begin{itemize}
\item
manual and install package
\item
first version published on CTAN
\end{itemize}

%%%%%%%%%%%%%%%%%%%%%%%%%%%%%%%%%%%%%%%%
\paragraph{v0.6:} 2017/04/26

\begin{itemize}
\item
redirection mechanism added
\end{itemize}

%%%%%%%%%%%%%%%%%%%%%%%%%%%%%%%%%%%%%%%%
\paragraph{v0.5:} 2017/04/26

\begin{itemize}
\item
functionality in definition file
\end{itemize}


%%%%%%%%%%%%%%%%%%%%%%%%%%%%%%%%%%%%%%%%%%%%%%%%%%%%%%%%%%%%%%%%%%%%%%%%%%%%%%%%
%%%%%%%%%%%%%%%%%%%%%%%%%%%%%%%%%%%%%%%%%%%%%%%%%%%%%%%%%%%%%%%%%%%%%%%%%%%%%%%%
%%%%%%%%%%%%%%%%%%%%%%%%%%%%%%%%%%%%%%%%%%%%%%%%%%%%%%%%%%%%%%%%%%%%%%%%%%%%%%%%
\appendix

\settowidth\MacroIndent{\rmfamily\scriptsize 000\ }

 \DocInput{childdoc.dtx}

\end{document}
%</driver>
% \fi
%
% %%%%%%%%%%%%%%%%%%%%%%%%%%%%%%%%%%%%%%%%%%%%%%%%%%%%%%%%%%%%%%%%%%%%%%%%%%%%%%
% %%%%%%%%%%%%%%%%%%%%%%%%%%%%%%%%%%%%%%%%%%%%%%%%%%%%%%%%%%%%%%%%%%%%%%%%%%%%%%
% \section{Sample}
%\iffalse
%<*samplemain>
%\fi
%
% The following presents a sample document
% with two chapters, two parts, a title page,
% a compile flag as well as three forwarding files to set the flag.
% It consists of eight |.tex| files:
% \begin{center}
% \begin{tabular}{ll}
% |cdocsamp.tex|&main file\\
% |cdocsch1.tex|&include file for chapter 1\\
% |cdocsch2.tex|&include file for chapter 2\\
% |cdocspt3.tex|&include file for part 3\\
% |cdocspt4.tex|&include file for part 4\\
% |cdocsdrf.tex|&forwarding file for main file in draft mode\\
% |cdocsfi1.tex|&forwarding file for final version of chapter 1\\
% |cdocsfi2.tex|&forwarding file for final version of chapter 2\\
% \end{tabular}
% \end{center}
% Each of the eight files can be compiled directly by the \LaTeX{} compiler.
%
% %%%%%%%%%%%%%%%%%%%%%%%%%%%%%%%%%%%%%%
% \paragraph{Main File.}
%
% The main file is called |cdocsamp.tex|.
%
% Load the \textsf{childdoc} definitions and
% declare the filename for the main document:
%    \begin{macrocode}
\input{childdoc.def}
\childdocmain{}
%    \end{macrocode}

% Optional override for |\version| flag:
%    \begin{macrocode}
%%\ifchilddoc\else\providecommand{\version}{draft}\fi
%    \end{macrocode}

% Define the default values for the |\version| flag
% (|final| for the main file and |draft| for childs):
%    \begin{macrocode}
\ifchilddoc
\providecommand{\version}{draft}
\else
\providecommand{\version}{final}
\fi
%    \end{macrocode}

% Load the standard document class:
%    \begin{macrocode}
\documentclass[12pt]{article}
%    \end{macrocode}

% Start the document body:
%    \begin{macrocode}
\begin{document}
%    \end{macrocode}

% Declare a title page.
% Print title, part of document being processed and version flag:
%    \begin{macrocode}
\addtocounter{page}{-1}
\begin{center}
{\LARGE\bfseries{}childdoc example\par}
\vspace{1cm}
\ifchilddoc
\ifchilddocmanual part\else chapter\fi:
`\childdocname' of `\childdocjob'\par
\else
main document: `\childdocjob'\par
\fi
version: \version\par
\end{center}
\newpage
%    \end{macrocode}

% Manually include selected file,
% otherwise process as usual:
%    \begin{macrocode}
\ifchilddocmanual
\section*{part `\childdocname'}
\input{\childdocname}
\else
%    \end{macrocode}

% Include the two chapters:
%    \begin{macrocode}
\include{cdocsch1}
\include{cdocsch2}
%    \end{macrocode}

% Include the two parts unless only chapters should be displayed:
%    \begin{macrocode}
\ifchilddoc\else
\section{part three}
\input{cdocspt3}
\section{part four}
\input{cdocspt4}
\fi
%    \end{macrocode}

% Process as usual until here:
%    \begin{macrocode}
\fi
%    \end{macrocode}

% End of document body:
%    \begin{macrocode}
\end{document}
%    \end{macrocode}
%\iffalse
%</samplemain>
%\fi
%
% %%%%%%%%%%%%%%%%%%%%%%%%%%%%%%%%%%%%%%
% \paragraph{Chapter Include Files.}
%
% The include files are called |cdocsch1.tex| and |cdocsch2.tex|.
%
%\iffalse
%<*samplechap1|samplechap2>
%\fi

% Optional override for |\version| flag:
%    \begin{macrocode}
%%\providecommand{\version}{final}
%    \end{macrocode}

% Include the main document:
%    \begin{macrocode}
\input{childdoc.def}
\childdocof{cdocsamp}
%    \end{macrocode}

%\iffalse
%</samplechap1|samplechap2>
%\fi
%
%\iffalse
%<*samplechap1>
%\fi
% Some text for chapter 1:
%    \begin{macrocode}
\section{one}
some text in chapter one
%    \end{macrocode}

%\iffalse
%</samplechap1>
%\fi
% Some text for chapter 2:
%\iffalse
%<*samplechap2>
%\fi
%    \begin{macrocode}
\section{two}
more text in chapter two
%    \end{macrocode}

%\iffalse
%</samplechap2>
%\fi
%
% %%%%%%%%%%%%%%%%%%%%%%%%%%%%%%%%%%%%%%
% \paragraph{Part Include Files.}
%
% The include files are called |cdocspt3.tex| and |cdocspt4.tex|.
%
%\iffalse
%<*samplepart3|samplepart4>
%\fi

% Optional override for |\version| flag:
%    \begin{macrocode}
%%\providecommand{\version}{final}
%    \end{macrocode}

% Include the main document:
%    \begin{macrocode}
\input{childdoc.def}
\childdocby{cdocsamp}
%    \end{macrocode}

%\iffalse
%</samplepart3|samplepart4>
%\fi
%
%\iffalse
%<*samplepart3>
%\fi
% Some text for part 3:
%    \begin{macrocode}
some text in part three
%    \end{macrocode}

%\iffalse
%</samplepart3>
%\fi
% Some text for part 4:
%\iffalse
%<*samplepart4>
%\fi
%    \begin{macrocode}
more text in part four
%    \end{macrocode}

%\iffalse
%</samplepart4>
%\fi
%
% %%%%%%%%%%%%%%%%%%%%%%%%%%%%%%%%%%%%%%
% \paragraph{Forwarding for a Complete Draft.}
%
% The following forwarding file |cdocsdrf.tex|
% compiles the main document in draft mode:
%\iffalse
%<*sampledraft>
%\fi
%    \begin{macrocode}
\def\version{draft}
\input{childdoc.def}
\childdocforward{cdocsamp}
%    \end{macrocode}

%\iffalse
%</sampledraft>
%\fi
%
% %%%%%%%%%%%%%%%%%%%%%%%%%%%%%%%%%%%%%%
% \paragraph{Forwarding for Final Version of the Chapters.}
%
% The following forwarding files |cdocsfn1.tex| and |cdocsfn2.tex|
% (with identical content)
% compile the final versions of the child documents
% |cdocsch1.tex| and |cdocsch2.tex|, respectively:
%\iffalse
%<*samplefinal>
%\fi
%    \begin{macrocode}
\def\version{final}
\input{childdoc.def}
\childdocforwardprefix[cdocsamp]{cdocsfn}{cdocsch}
%    \end{macrocode}

%\iffalse
%</samplefinal>
%\fi
%
% %%%%%%%%%%%%%%%%%%%%%%%%%%%%%%%%%%%%%%
% \paragraph{Command Line Processing.}
%
% The following three command lines generate the output files
% |cdocscld|, |cdocscl1| and |cdocscl2|
% which should be identical to
% |cdocsdrf|, |cdocsch1| and |cdocsfn2|, respectively:
% \begin{center}
% \begin{tabular}{l}
% |latex -jobname cdocscld \|\\
% |  "\def\version{draft}\input{childdoc.def}\childdocforward{cdocsamp}"|\\
% |latex -jobname cdocscl1 \|\\
% |  "\input{childdoc.def}\childdocforward[cdocsamp]{cdocsch1}"|\\
% |latex -jobname cdocscl2 \|\\
% |  "\def\version{final}\input{childdoc.def}\childdocforward{cdocsch2}"|
% \end{tabular}
% \end{center}
% Note that the trailing backslash on each first line
% merely continues the input to the second line
% (for convenient cut ant paste).
% Furthermore, the command |latex| can be replaced by any
% of its alternative versions such as |pdflatex|.
%
% %%%%%%%%%%%%%%%%%%%%%%%%%%%%%%%%%%%%%%%%%%%%%%%%%%%%%%%%%%%%%%%%%%%%%%%%%%%%%%
% %%%%%%%%%%%%%%%%%%%%%%%%%%%%%%%%%%%%%%%%%%%%%%%%%%%%%%%%%%%%%%%%%%%%%%%%%%%%%%
% \section{Implementation}
%\iffalse
%<*package>
%\fi
%
% This section describes the definitions file |childdoc.def|.

% The definitions cannot be loaded using |\usepackage| or |\RequirePackage|
% which has a mechanism to prevent loading a style file more than once.
% When loading the definitions by means of |\input|
% multiple instances have to be prevented manually:
%\iffalse
%This code needs to be before the `\ProvidesFile' directive
%which is defined at the beginning of this file.
%Therefore it is also placed there and commented out here.
%</package>
%<*discard>
%\fi
%    \begin{macrocode}
\ifdefined\childdocmain\endinput\fi
%    \end{macrocode}
%\iffalse
%</discard>
%<*package>
%\fi
%
% \macro{\ifchilddoc}
% \macro{\ifchilddocmanual}
% The conditional |\ifchilddoc| tells whether a
% child (true) or main (false) document is being compiled.
% The conditional |\ifchilddocmanual| tells whether
% the |\includeonly| mechanism is used (false) or
% the selection of child files must be performed manually (true).
% The definitions initialise to false:
%    \begin{macrocode}
\newif\ifchilddoc
\newif\ifchilddocmanual
%    \end{macrocode}

% \macro{\childdocname}
% \macro{\childdocjob}
% The macro |\childdocname| stores the name of the main document
% to be compiled. The macro |\childdocjob| stores the name of
% the document on which the \LaTeX{} compiler was originally invoked.
% The content of |\jobname| cannot be compared
% to filenames specified in the source due to different catcodes.
% The following code rescans |\jobname|, stores the result
% in |\childdocname| and saves a copy in |\childdocjob|:
%    \begin{macrocode}
\edef\childdocname{\scantokens\expandafter{\jobname\noexpand}}
\let\childdocjob\childdocname
%    \end{macrocode}

% \macro{\childdocdisable}
% The macro |\childdocdisable| prevents the main file
% from being processed more than once.
% At this stage, the main document command |\childdocmain|
% is assumed to be called once again where it should do nothing.
% Any subsequent call to it should prevent
% a secondary processing of the main document
% It overwrites the forwarding commands
% |\childdocof| and |\childdocforward|
% with empty macros to prevent further inclusions of the main document:
%    \begin{macrocode}
\newcommand{\childdocdisable}
{
  \renewcommand{\childdocmain}[1]{\renewcommand{\childdocmain}[1]{\endinput}}
  \renewcommand{\childdocof}[1]{}
  \renewcommand{\childdocby}[2][]{}
  \renewcommand{\childdocforward}[2][]{}
  \renewcommand{\childdocdisable}{}
}
%    \end{macrocode}

% \macro{\childdocmain}
% The macro |\childdocmain| is to be called at the top of the main file
% with nothing or the main filename (without extension) as argument.
% First, it breaks loops.
% If the argument is not empty and does not match |\childdocname|
% (which is set by the first inclusion of |childdoc.def|),
% |\ifchilddoc| is set to true, |\includeonly| is applied to the child file
% and |\jobname| is set to the main file
% (for proper handling of |.aux| files):
%    \begin{macrocode}
\newcommand{\childdocmain}[1]
{
  \childdocdisable\childdocmain{}
  \if?#1?\else
    \begingroup
      \def\childdoctmp{#1}
      \ifx\childdoctmp\childdocname
        \def\childdoctmp{}
      \else
        \def\childdoctmp
        {
          \childdoctrue
          \includeonly{\childdocname}
          \def\childdocjob{#1}
          \def\jobname{#1}
        }
      \fi
      \expandafter
    \endgroup
    \childdoctmp
  \fi
}
%    \end{macrocode}

% \macro{\childdocof}
% The command |\childdocof| redirects
% compilation to the main file |#1|.
%    \begin{macrocode}
\newcommand{\childdocof}[1]
{
  \childdocdisable
  \childdoctrue
  \includeonly{\childdocname}
  \def\jobname{#1}
  \def\childdocjob{#1}
  \input{#1}
}
%    \end{macrocode}

% \macro{\childdocby}
% The command |\childdocby| ....
%    \begin{macrocode}
\newcommand{\childdocby}[2][]
{
  \childdocdisable
  \childdoctrue
  \childdocmanualtrue
  \if?#1?\else
    \def\jobname{#2}
  \fi
  \def\childdocjob{#2}
  \input{#2}
  \endinput
}
%    \end{macrocode}

% \macro{\childdocforward}
% The command |\childdocforward| redirects
% compilation to the main file or
% (if the optional argument is given) a child file.
% Parameters are set as if the main file
% or a child file starting with |\childdocof| was compiled.
% Then compilation is handed over to the main file:
%    \begin{macrocode}
\newcommand{\childdocforward}[2][]
{
  \begingroup
    \if?#1?
      \def\childdoctmp
      {
        \def\childdocname{#2}
        \def\childdocjob{#2}
        \def\jobname{#2}
        \input{#2}
        \endinput
      }
    \else
      \def\childdoctmp
      {
        \childdocdisable
        \def\childdocname{#2}
        \childdoctrue
        \includeonly{#2}
        \def\childdocjob{#1}
        \def\jobname{#1}
        \input{#1}
        \endinput
      }
    \fi
    \expandafter
  \endgroup
  \childdoctmp
}
%    \end{macrocode}

% \macro{\childdocforwardprefix}
% The command |\childdocforwardprefix| redirects
% compilation to the main or a child file by means of a pattern.
% The prefix |#1| in the current filename is replaced by |#2|
% and the suffix of the current filename is kept
% (it is assumed that the filename does not contain the substring `|~~~|'
% which is used as a delimiter).
% Compilation is handed over to the new file by |\childdocforward|:
%    \begin{macrocode}
\newcommand{\childdocforwardprefix}[3][]
{
  \begingroup
    \def\childdocextract #2##1~~~{\def\childdoctmp{\childdocforward[#1]{#3##1}}}
    \expandafter\childdocextract\childdocname~~~
    \expandafter
  \endgroup
  \childdoctmp
}
%    \end{macrocode}

% \macro{\childdoc}
% The deprecated macro |\childdoc| is a legacy version of |\childdocmain|:
%    \begin{macrocode}
\newcommand{\childdoc}{\childdocmain}
%    \end{macrocode}

% \macro{\childdocredirect}
% The deprecated macro |\childdocredirect| is a legacy version
% of |\childdocforward| and |\childdocforwardprefix|:
%    \begin{macrocode}
\newcommand{\childdocredirect}[2][]
{
  \begingroup
    \if?#1?
      \def\childdoctmp{\childdocforward{#2}}
    \else
      \def\childdoctmp{\childdocforwardprefix{#1}{#2}}
    \fi
    \expandafter
  \endgroup
  \childdoctmp
}
%    \end{macrocode}

%\iffalse
%</package>
%\fi
%
\endinput
|\\
|\childdocforwardprefix{final}{child}|
\end{tabular}
\end{center}
%

Note that when several versions of a main file and/or of each child file
are to be generated, it may be convenient to set up a |Makefile| or
shell script to automatise the process.

%%%%%%%%%%%%%%%%%%%%%%%%%%%%%%%%%%%%%%%%%%%%%%%%%%%%%%%%%%%%%%%%%%%%%%%%%%%%%%%%
\subsection{Command Line Processing}
\label{sec:commandline}

The effect of redirection files can also be achieved by invoking
the \LaTeX{} compiler with a more elaborate command line.
Most conveniently this should be done as part
of a shell script or a |Makefile|.

When using \textsf{childdoc} in the main file, the following
command lines effectively perform a redirection
(note that depending on the shell being used,
backslashes may have to be doubled: `|\|' $\to$ `|\\|'):
%
\begin{center}
|... -jobname "|\textit{target}|" |\\|"|[\textit{flags}]%
|% \iffalse
%
% childdoc.dtx Copyright (C) 2017-2018 Niklas Beisert
%
% This work may be distributed and/or modified under the
% conditions of the LaTeX Project Public License, either version 1.3
% of this license or (at your option) any later version.
% The latest version of this license is in
%   http://www.latex-project.org/lppl.txt
% and version 1.3 or later is part of all distributions of LaTeX
% version 2005/12/01 or later.
%
% This work has the LPPL maintenance status `maintained'.
%
% The Current Maintainer of this work is Niklas Beisert.
%
% This work consists of the files childdoc.dtx and childdoc.ins
% and the derived files childdoc.def and cdocsamp.tex with
% cdocsch1.tex, cdocsch2.tex, cdocsdrf.tex, cdocsfn1.tex, cdocsfn2.tex.
%
%<package>\ifdefined\childdocmain\endinput\fi
%<package>\ProvidesFile{childdoc.def}[2018/12/30 v2.0 child document driver]
%<samplemain>\ProvidesFile{cdocsamp.tex}[2018/12/30 v2.0 sample for childdoc]
%<*driver>
%\ProvidesFile{childdoc.drv}[2018/12/30 v2.0 childdoc reference manual file]
\PassOptionsToClass{10pt,a4paper}{article}
\documentclass{ltxdoc}

\usepackage[margin=35mm]{geometry}
\usepackage{hyperref}
\usepackage{hyperxmp}
\usepackage[usenames]{color}

\hypersetup{colorlinks=true}
\hypersetup{pdfstartview=FitH}
\hypersetup{pdfpagemode=UseNone}
\hypersetup{pdfsource={}}
\hypersetup{pdflang={en-UK}}
\hypersetup{pdfcopyright={Copyright 2017-2018 Niklas Beisert.
  This work may be distributed and/or modified under the
  conditions of the LaTeX Project Public License, either version 1.3
  of this license or (at your option) any later version.}}
\hypersetup{pdflicenseurl={http://www.latex-project.org/lppl.txt}}
\hypersetup{pdfcontactaddress={ETH Zurich, ITP, HIT K,
  Wolfgang-Pauli-Strasse 27}}
\hypersetup{pdfcontactpostcode={8093}}
\hypersetup{pdfcontactcity={Zurich}}
\hypersetup{pdfcontactcountry={Switzerland}}
\hypersetup{pdfcontactemail={nbeisert@itp.phys.ethz.ch}}
\hypersetup{pdfcontacturl={http://people.phys.ethz.ch/\xmptilde nbeisert/}}

\newcommand{\secref}[1]{\hyperref[#1]{section \ref*{#1}}}

\parskip1ex
\parindent0pt
\let\olditemize\itemize
\def\itemize{\olditemize\parskip0pt}

\begin{document}

\title{The \textsf{childdoc} Package}
\hypersetup{pdftitle={The childdoc Package}}
\author{Niklas Beisert\\[2ex]
  Institut f\"ur Theoretische Physik\\
  Eidgen\"ossische Technische Hochschule Z\"urich\\
  Wolfgang-Pauli-Strasse 27, 8093 Z\"urich, Switzerland\\[1ex]
  \href{mailto:nbeisert@itp.phys.ethz.ch}
  {\texttt{nbeisert@itp.phys.ethz.ch}}}
\hypersetup{pdfauthor={Niklas Beisert}}
\hypersetup{pdfsubject={Manual for the LaTeX2e Package childdoc}}
\date{30 December 2018, \textsf{v2.0}}
\maketitle

\begin{abstract}\noindent
\textsf{childdoc} is a \LaTeXe{} package
that enables the direct compilation
of document sections included by |\include|
to individual files.
\end{abstract}

\begingroup
\parskip0ex
\tableofcontents
\endgroup

%%%%%%%%%%%%%%%%%%%%%%%%%%%%%%%%%%%%%%%%%%%%%%%%%%%%%%%%%%%%%%%%%%%%%%%%%%%%%%%%
%%%%%%%%%%%%%%%%%%%%%%%%%%%%%%%%%%%%%%%%%%%%%%%%%%%%%%%%%%%%%%%%%%%%%%%%%%%%%%%%
\section{Introduction}

\LaTeX{} provides a mechanism to structure a large document (such as a book)
into a main file and several child files (containing the chapters)
using the |\include| command.
This mechanism is beneficial for documents
which span hundreds of pages in order to
make the source file(s) more manageable.
Moreover, compilation can be restricted to
selected child files by means of the |\includeonly| command.
The latter feature can be used to reduce the compilation time while editing
(this was significantly more useful in the earlier days of \LaTeX{})
or to generate a smaller document which is easier to navigate.
Another application of |\includeonly| is to generate
documents consisting of selected parts of the complete document.

However, there are a few drawbacks of the plain |\include| mechanism:
\begin{itemize}
\item
The child files cannot be compiled on their own,
they can only be compiled via the main file.
A naive editing environment
(such as a text editor with an option
to have the current file processed by \LaTeX)
may require one to switch to the main file before compiling;
attempting to compile the child file produces errors.
\item
The main file must be modified (each time)
to adjust the |\includeonly| command
to the present needs. This easily leaves the main file in a messy state.
\item
The generated document will always carry the filename
of the main document. This is inconvenient if
several child files are to be compiled and
to be kept for distribution.
\end{itemize}

The present package provides a simple interface
to make child files individually compilable by \LaTeX{}.
Compiling a child file then has the same effect as compiling
the main file with an |\includeonly| command
to select the appropriate child.
Moreover the generated document will carry the name of the child
rather than the main file.
This resolves all three above issues.

This feature is meant to make the editing of books,
thesis documents and lecture notes somewhat more convenient.
However, the package can also be used efficiently for
composing a series of documents (such as exercise sheets)
which are typically distributed individually.
It then assists the author in generating the individual documents
(potentially in different versions)
as well as a document containing the collected series.
Another application is in developing style files
or other kinds of included material
where compilation of the style file could redirect
to a sample or test file.

%%%%%%%%%%%%%%%%%%%%%%%%%%%%%%%%%%%%%%%%%%%%%%%%%%%%%%%%%%%%%%%%%%%%%%%%%%%%%%%%
%%%%%%%%%%%%%%%%%%%%%%%%%%%%%%%%%%%%%%%%%%%%%%%%%%%%%%%%%%%%%%%%%%%%%%%%%%%%%%%%
\section{Usage}

First of all, the package \textsf{childdoc} is \emph{not} a standard
\LaTeXe{} |.sty| style file! Therefore it needs to be invoked in
a non-standard way.

%%%%%%%%%%%%%%%%%%%%%%%%%%%%%%%%%%%%%%%%%%%%%%%%%%%%%%%%%%%%%%%%%%%%%%%%%%%%%%%%
\subsection{Included Files}
\label{sec:include}

%%%%%%%%%%%%%%%%%%%%%%%%%%%%%%%%%%%%%%%%
\DescribeMacro{\childdocmain}
To use the package, add the commands
\begin{center}
\begin{tabular}{l}
|\input{childdoc.def}|\\
|\childdocmain{}|\\
\end{tabular}
\end{center}
at the very top of the main \LaTeX{} file,
in particular \emph{before} the |\documentclass| statement!
The argument of |\childdocmain| should be left empty
(but it must be present).

%%%%%%%%%%%%%%%%%%%%%%%%%%%%%%%%%%%%%%%%
\DescribeMacro{\childdocof}
Furthermore, add the commands
\begin{center}
\begin{tabular}{l}
|\input{childdoc.def}|\\
|\childdocof{|\textit{main}|}|\\
\end{tabular}
\end{center}
at the top of every child file \textit{child}
which is included by |\include{|\textit{child}|}|
from within the main file
(or at least for those files to be compiled individually).
The argument \textit{main} must be the filename of the main file.

There are a couple of
considerations in setting up the main and child documents:

%%%%%%%%%%%%%%%%%%%%%%%%%%%%%%%%%%%%%%%%
\paragraph{Restrictions.}

Please note the following restrictions:
\begin{itemize}
\item
|\childdocmain| must be called with one argument \textit{main}
to ensure compatibility with earlier version of the package.
It must either be empty (|\childdocmain{}|)
or precisely match the filename of the main file in which it is specified.
See \secref{sec:detection} for further information.
\item
The filename \textit{main} must be specified without the |.tex| extension.
\item
The filename \textit{main} is case sensitive
(even in case-insensitive file systems)
due to internal string comparison.
\item
The argument \textit{main} should be fully expanded, it cannot be a macro.
\item
Subdirectories and special characters should be avoided in filenames.
\item
The command |\childdocmain{|\textit{main}|}| must be followed by a whitespace.
It should not be followed immediately by another command
or by a comment mark `|%|'.
This is because the \TeX{} parser reads the token immediately following
the argument of |\childdocmain| and puts it
at the beginning of every child section;
however, a white\-space is ignored.
\end{itemize}

%%%%%%%%%%%%%%%%%%%%%%%%%%%%%%%%%%%%%%%%
\paragraph{Content of Main File.}

It is advisable to place all content in the child files included by |\include|.
Any output contained in the main file will appear in all child documents
unless suppressed manually;
it cannot be suppressed automatically by the |\includeonly| directive
and thus should normally be avoided.
A method to include some content in the main file
by means of conditional processing is described in \secref{sec:conditional}.

%%%%%%%%%%%%%%%%%%%%%%%%%%%%%%%%%%%%%%%%
\paragraph{Page Numbering.}

When only a part of the document is compiled,
the appropriate numbering of pages
(as well as other status parameters)
is determined from the |.aux| files.
The latter contain information from previous passes.
However this information needs to propagate through
all intermediate child documents.
Therefore the page numbering in child documents may well
be inconsistent until the complete document is compiled at least once.

A useful (if unconventional) way to always ensure a consistent
page numbering is to restart the numbering in each child document
and denote the pages by `\textit{child}|.|\textit{page}'
where \textit{child} represents the chapter/section number of the child file.
This can be achieved by the command
|\numberwithin{page}{|\textit{child}|}|
of the \textsf{amsmath} package
where \textit{child} can be |chapter| or |section|
depending on the chosen structuring.
Alternatively, one can modify the macro |\thepage| appropriately
and reset the counter |page| at the start of each child file.

%%%%%%%%%%%%%%%%%%%%%%%%%%%%%%%%%%%%%%%%%%%%%%%%%%%%%%%%%%%%%%%%%%%%%%%%%%%%%%%%
\subsection{Conditional Processing}
\label{sec:conditional}

The package provides a mechanism to compile different versions
of a document. To customise the versions further some conditional processing
can come in handy to distinguish which version is being compiled.
The package provides two macros to describe the compilation context:

%%%%%%%%%%%%%%%%%%%%%%%%%%%%%%%%%%%%%%%%
\DescribeMacro{\ifchilddoc}
The conditional |\ifchilddoc| distinguishes between the compilation of
child documents and the main document:
%
\begin{center}
|\ifchilddoc |\textit{child-code}| |[|\||else |\textit{main-code}]| \||fi|
\end{center}

%%%%%%%%%%%%%%%%%%%%%%%%%%%%%%%%%%%%%%%%
\DescribeMacro{\childdocname}
\DescribeMacro{\childdocjob}
The macro |\childdocname| contains the filename (without extension)
of the main or child file being processed.
Note that |\childdocjob| will always contain the name of the main file.

%%%%%%%%%%%%%%%%%%%%%%%%%%%%%%%%%%%%%%%%
\paragraph{Title Page.}

Conditional processing can be used to include a title or banner page
in the main document when proper precautions are taken.
Importantly, the code in the main file should ensure that the page counter
(as well as other status parameters which are stored in the |.aux| files)
takes the same value after the conditional processing.
Otherwise the page numbers may take divergent values
depending on which part is compiled.

For example, a title page could be declared by:
%
\begin{center}
\begin{tabular}{l}
|\ifchilddoc\||else|\\
|\addtocounter{page}{-1}|\\
\textit{code for title page}\\
|\newpage|\\
|\||fi|
\end{tabular}
\end{center}
%
A banner page for the child documents can be generated by:
%
\begin{center}
\begin{tabular}{l}
|\ifchilddoc|\\
|\addtocounter{page}{-1}|\\
\textit{code for banner page}\\
|\newpage|\\
|\||fi|
\end{tabular}
\end{center}
%
Here one could write a message such as:
\begin{center}
|This is the part \childdocname{} of \childdocjob{}.|
\end{center}

%%%%%%%%%%%%%%%%%%%%%%%%%%%%%%%%%%%%%%%%%%%%%%%%%%%%%%%%%%%%%%%%%%%%%%%%%%%%%%%%
\subsection{Flags}
\label{sec:flags}

The package makes it easy to generate different versions
of the main or child documents.
To this end compilation flags can be defined
and assigned different default values.
They will be particularly useful in conjunction
with the forwarding mechanism described in \secref{sec:forward}.

For example, it may be useful to have a flag |\version|
which can be set to |draft| or |final|.
The document source will contain some conditional code
depending on the value of |\version|.
Suppose further, the flag should default to |final| for the main file
and to |draft| for child files
which is a natural assignment for editing the document.
This is achieved by placing the following code
in the preamble of the main document
(below the |\childdocmain| directive):
%
\begin{center}
\begin{tabular}{l}
|\ifchilddoc|\\
|\providecommand{\version}{draft}|\\
|\||else|\\
|\providecommand{\version}{final}|\\
|\||fi|
\end{tabular}
\end{center}
%
The definition by |\providecommand| makes sure
that previous definitions are not overwritten.
Further statements |\providecommand{\version}{...}|
can thus be added before the above code to override it.

For the main file, one might add a line
(between |\childdocmain| and the above block)
%
\begin{center}
|%\ifchilddoc\||else\providecommand{\version}{draft}\||fi|
\end{center}
%
which can be uncommented to produce a draft version.
Likewise one can add a line to the very top of a child file
(above the |\childdocof{|\textit{main}|}| directive)
%
\begin{center}
|%\providecommand{\version}{final}|
\end{center}
%
which can be uncommented to produce the final version of this child document.

%%%%%%%%%%%%%%%%%%%%%%%%%%%%%%%%%%%%%%%%%%%%%%%%%%%%%%%%%%%%%%%%%%%%%%%%%%%%%%%%
\subsection{Forwarding}
\label{sec:forward}

Different versions of the main or child documents
using compilation flags as described in \secref{sec:flags}
can be (permanently) stored in different files
for convenient compilation, viewing and distribution.
To this end, the package defines a command
to pass on compilation to a different file:

%%%%%%%%%%%%%%%%%%%%%%%%%%%%%%%%%%%%%%%%
\DescribeMacro{\childdocforward}
The command |\childdocforward| redirects processing to
another source file:
%
\begin{center}
\begin{tabular}{l}
|\input{childdoc.def}|\\
|\childdocforward[|\textit{main}|]{|\textit{dest}|}|\\
\end{tabular}
\end{center}
%
The argument \textit{dest} is the destination file
(without extension).
It should be the main file or one of the child files.
Note that further \textsf{childdoc} directives
such as |\childdocof| and |\childdocforward|
in the indicated file will be processed in this form.
The optional argument \textit{main}
passes on directly to the main file \textit{main}
while pretending to compile the child \textit{dest}.
This form behaves as if \textit{dest}
issues |\childdocof{|\textit{main}|}| right away,
and no further \textsf{childdoc} directives will be processed.

%%%%%%%%%%%%%%%%%%%%%%%%%%%%%%%%%%%%%%%%
\DescribeMacro{\...prefix}
In the alternative form |\childdocforwardprefix|,
%
\begin{center}
\begin{tabular}{l}
|\input{childdoc.def}|\\
|\childdocforwardprefix[|\textit{main}|]{|\textit{prefix}|}{|\textit{dest}|}|
\end{tabular}
\end{center}
%
the destination file is determined by a pattern
depending on the current file:
To make this work, the current file must be called
`{\textit{prefix}\hspace{0.2em}\textit{suffix}}'
with \textit{prefix} matching precisely the argument.
Processing is then passed on to the file
`{\textit{dest}\hspace{0.2em}\textit{suffix}}'.
Surely, the same effect is achieved by
directly specifying the
argument `{\textit{dest}\hspace{0.2em}\textit{suffix}}'
in the first form.
However, that requires to set up a different file
for each child. With the alternative form of the command
all these files can have exactly the same content
which simplifies setting them up and maintaining them.

For example, the following file |draft.tex|
with a compilation flag |\version| as described in \secref{sec:flags}
compiles the main document as a draft:
%
\begin{center}
\begin{tabular}{l}
|\def\version{draft}|\\
|\input{childdoc.def}|\\
|\childdocforward{|\textit{main}|}|
\end{tabular}
\end{center}
%
Likewise, the following files |final|\textit{nn}|.tex|
compile the final version of the child document
|child|\textit{nn}|.tex|:
%
\begin{center}
\begin{tabular}{l}
|\def\version{final}|\\
|\input{childdoc.def}|\\
|\childdocforwardprefix{final}{child}|
\end{tabular}
\end{center}
%

Note that when several versions of a main file and/or of each child file
are to be generated, it may be convenient to set up a |Makefile| or
shell script to automatise the process.

%%%%%%%%%%%%%%%%%%%%%%%%%%%%%%%%%%%%%%%%%%%%%%%%%%%%%%%%%%%%%%%%%%%%%%%%%%%%%%%%
\subsection{Command Line Processing}
\label{sec:commandline}

The effect of redirection files can also be achieved by invoking
the \LaTeX{} compiler with a more elaborate command line.
Most conveniently this should be done as part
of a shell script or a |Makefile|.

When using \textsf{childdoc} in the main file, the following
command lines effectively perform a redirection
(note that depending on the shell being used,
backslashes may have to be doubled: `|\|' $\to$ `|\\|'):
%
\begin{center}
|... -jobname "|\textit{target}|" |\\|"|[\textit{flags}]%
|\input{childdoc.def}\childdocforward[|\textit{main}|]{|\textit{dest}|}"|
\end{center}
%
Here \textit{target} is the name of the output file,
\textit{main} is the name of the main file
and \textit{dest} is the name of the main or child file to be processed
(all filenames without extensions).
The optional argument \textit{main} can be omitted
if \textit{main} matches \textit{dest}.
Optionally, compilation \textit{flags} can be defined via |\def| commands.
This command line makes the \TeX{} engine believe
it is compiling the file \textit{target}
whose content is specified as the latter parameter.
The provided code then forwards the processing to
\textit{main} or \textit{dest} as described in \secref{sec:forward}.

%%%%%%%%%%%%%%%%%%%%%%%%%%%%%%%%%%%%%%%%%%%%%%%%%%%%%%%%%%%%%%%%%%%%%%%%%%%%%%%%
\subsection{Include by Input}
\label{sec:input}

Including child documents by |\include| has some restrictions by design.
Most notably, the content of a child document always occupies
its own set of pages; pages cannot be shared between child documents.
Usually, this behaviour makes perfect sense
because each child document contain an essential part of the document.
However, in some situations it may be desirable to compose
a document from a collection of parts
without having mandatory page breaks between then.
For this case, the package
provides a mechanism to include parts
by |\input| which can also be processed individually.
However, by construction this mechanism
requires manual handling of the content to be output.

%%%%%%%%%%%%%%%%%%%%%%%%%%%%%%%%%%%%%%%%
\DescribeMacro{\ifchilddocmanual}
The main file should be prepared as usual, see \secref{sec:include}.
However, the document body must make a distinction
between processing of an individual part and of the main document, e.g.:
%
\begin{center}
\begin{tabular}{l}
|\ifchilddocmanual|\\
|\input{\childdocname}|\\
|\||else|\\
\textit{document body with }|\input{|\textit{part}|}|\\
|\||fi|
\end{tabular}
\end{center}
%
The conditional |\ifchilddocmanual| is true whenever
a part to be included by |\input| is being compiled,
and the name of the part is stored in |\childdocname|.

%%%%%%%%%%%%%%%%%%%%%%%%%%%%%%%%%%%%%%%%
\DescribeMacro{\childdocby}
Each part to be included by |\input| should start with:
%
\begin{center}
\begin{tabular}{l}
|\input{childdoc.def}|\\
|\childdocby{|\textit{main}|}|\\
\end{tabular}
\end{center}
%
The directive |\childdocby| is similar to |\childdocof|
described in \secref{sec:include},
but the subsequent selection of content must be done manually.
To that end, both |\ifchilddoc| and |\ifchilddocmanual|
will be true upon processing of a part,
and the name of the part is stored in |\childdocname|.
Note that |\jobname| will be set to the filename of the current part
so that each part receives an individual |.aux| file
that does not interfere with the |.aux| file(s) of the main document.
This behaviour can be altered by the alternative form
|\childdocby[*]{|\textit{main}|}| (with a non-empty optional argument)
which uses the |.aux| file of the main document
by setting |\jobname| to \textit{main}.

%%%%%%%%%%%%%%%%%%%%%%%%%%%%%%%%%%%%%%%%%%%%%%%%%%%%%%%%%%%%%%%%%%%%%%%%%%%%%%%%
\subsection{Driver Development}
\label{sec:driver}

The \textsf{childdoc} mechanism can also be use for the development
of definition files such as \LaTeX{} styles or classes.
This case differs from the above setup with multiple parts
included by |\include| in that no |\includeonly| should be invoked.
This can be achieved by starting the include file
(before |\ProvidesPackage|) with:
%
\begin{center}
\begin{tabular}{l}
|\input{childdoc.def}|\\
|\childdocforward{|\textit{main}|}|\\
\end{tabular}
\end{center}
%
or alternatively with:
%
\begin{center}
\begin{tabular}{l}
|\input{childdoc.def}|\\
|\childdocby{|\textit{main}|}|\\
\end{tabular}
\end{center}
%
Both forms have slightly different effects as described above.
The main file is prepared as usual, see \secref{sec:include}.

%%%%%%%%%%%%%%%%%%%%%%%%%%%%%%%%%%%%%%%%%%%%%%%%%%%%%%%%%%%%%%%%%%%%%%%%%%%%%%%%
\subsection{Legacy Detection}
\label{sec:detection}

The directive |\childdocmain| in the main file can detect
whether the complete document or merely a child is to be compiled
even without using the directive |\childdocof|.
This method is deprecated because it is less robust
and there is no compelling reason to use it;
it is merely provided for backward compatibility
and it may be removed in future versions.

If the detection mechanism is to be used,
it is mandatory to correctly specify
the filename of the main file as the argument of |\childdocmain|:
%
\begin{center}
\begin{tabular}{l}
|\input{childdoc.def}|\\
|\childdocmain{|\textit{main}|}|\\
\end{tabular}
\end{center}
%
If |\jobname| does not match the argument \textit{main} of |\childdocmain|,
it is assumed that |\jobname| points to the child file to be compiled.
When using |\childdocmain| with the main file specified as argument,
it suffices to start a child file
with just |\input{|\textit{main}|}|
without loading of the package and using |\childdocof|.
If instead all processing is done
with the appropriate \textsf{childdoc} directives,
the argument of \textit{main} of |\childdocmain| can be empty.

An alternative version of the command line processing described
in \secref{sec:commandline} using the detection mechanism reads:
%
\begin{center}
|... -jobname "|\textit{target}|" "|[\textit{flags}]%
[|\def\jobname{|\textit{dest}|}|]|\input{|\textit{main}|}"|
\end{center}

%%%%%%%%%%%%%%%%%%%%%%%%%%%%%%%%%%%%%%%%%%%%%%%%%%%%%%%%%%%%%%%%%%%%%%%%%%%%%%%%
\subsection{Manual Code}
\label{sec:manual}

In case one cannot be certain whether the definitions file |childdoc.def|
is installed on the target \TeX{} distribution
and one prefers not to ship it,
it is conceivable to paste a few relevant commands into the sources.

To that end, drop all statements |\input{childdoc.def}|
and perform the replacements as outlined below.
Instead of |\childdocmain{|\textit{main}|}| add the following code
to the top of the main file:
%
\begin{center}
\begin{tabular}{l}
|\||ifdefined\childdocname\endinput\||fi\newif\ifchilddoc|\\
|\edef\childdocname{\scantokens\expandafter{\jobname\noexpand}}|\\
|\def\childdocmain{|\textit{main}|}\||ifx\childdocmain\childdocname\||else|\\
|\childdoctrue\includeonly{\childdocname}\let\jobname\childdocmain\||fi|\\
\end{tabular}
\end{center}
%
Instead of |\childdocof{|\textit{main}|}| just include the main file
at the top of each child file:
%
\begin{center}
|\input{|\textit{main}|}|
\end{center}
%
A simple redirection |\childdocforward{|\textit{dest}|}| is achieved by:
%
\begin{center}
|\def\jobname{|\textit{dest}|}\input{\jobname}|
\end{center}
%
The redirection with prefix
|\childdocforwardprefix[|\textit{prefix}|]{|\textit{dest}|}|
is accomplished by:
%
\begin{center}
\begin{tabular}{l}
|{\edef\jobname{\scantokens\expandafter{\jobname\noexpand}}|\\
|\def\redirectjob |\textit{prefix}|#1~~~{\gdef\jobname{|\textit{dest}|#1}}|\\
|\expandafter\redirectjob\jobname~~~}\input{\jobname}|
\end{tabular}
\end{center}

In an alternative approach,
child documents can be compiled by a specific command line
without additional code or specific definitions:
%
\begin{center}
|... -jobname "|\textit{target}|" "|[\textit{flags}]%
|\includeonly{|\textit{dest}|}\input{|\textit{main}|}"|
\end{center}
%

%%%%%%%%%%%%%%%%%%%%%%%%%%%%%%%%%%%%%%%%%%%%%%%%%%%%%%%%%%%%%%%%%%%%%%%%%%%%%%%%
%%%%%%%%%%%%%%%%%%%%%%%%%%%%%%%%%%%%%%%%%%%%%%%%%%%%%%%%%%%%%%%%%%%%%%%%%%%%%%%%
\section{Information}

%%%%%%%%%%%%%%%%%%%%%%%%%%%%%%%%%%%%%%%%%%%%%%%%%%%%%%%%%%%%%%%%%%%%%%%%%%%%%%%%
\subsection{Copyright}

Copyright \copyright{} 2017--2018 Niklas Beisert

This work may be distributed and/or modified under the
conditions of the \LaTeX{} Project Public License, either version 1.3
of this license or (at your option) any later version.
The latest version of this license is in
  \url{http://www.latex-project.org/lppl.txt}
and version 1.3 or later is part of all distributions of \LaTeX{}
version 2005/12/01 or later.

This work has the LPPL maintenance status `maintained'.

The Current Maintainer of this work is Niklas Beisert.

This work consists of the files |README.txt|, |childdoc.ins| and |childdoc.dtx|
as well as the derived files |childdoc.def|, |cdocsamp.tex|
with |cdocsch1.tex|, |cdocsch2.tex|, |cdocspt3.tex|, |cdocspt4.tex|,
|cdocsdrf.tex|, |cdocsfn1.tex|, |cdocsfn2.tex|
as well as |childdoc.pdf|.

%%%%%%%%%%%%%%%%%%%%%%%%%%%%%%%%%%%%%%%%%%%%%%%%%%%%%%%%%%%%%%%%%%%%%%%%%%%%%%%%
\subsection{Files and Installation}

The package consists of the files:
%
\begin{center}
\begin{tabular}{ll}
    |README.txt|   & readme file \\
    |childdoc.ins| & installation file \\
    |childdoc.dtx| & source file \\
    |childdoc.def| & definition file \\
    |cdocsamp.tex| & sample main file \\
    |cdocsch1.tex| & sample include file \\
    |cdocsch2.tex| & sample include file \\
    |cdocspt3.tex| & sample part file \\
    |cdocspt4.tex| & sample part file \\
    |cdocsdrf.tex| & sample redirection file \\
    |cdocsfn1.tex| & sample redirection file \\
    |cdocsfn2.tex| & sample redirection file \\
    |childdoc.pdf| & manual
\end{tabular}
\end{center}
%
The distribution consists of the files
|README.txt|, |childdoc.ins| and |childdoc.dtx|.
%
\begin{itemize}
\item
Run (pdf)\LaTeX{} on |childdoc.dtx|
to compile the manual |childdoc.pdf| (this file).
\item
Run \LaTeX{} on |childdoc.ins| to create the definitions file |childdoc.def|
and the sample |cdocsamp.tex| with include files
|cdocsch1.tex|, |cdocsch2.tex|, |cdocspt3.tex|, |cdocspt4.tex|,
|cdocsdrf.tex|, |cdocsfn1.tex|, |cdocsfn2.tex|.
Then copy the file |childdoc.def| to an appropriate directory of your \LaTeX{}
distribution, e.g.\ \textit{texmf-root}|/tex/latex/childdoc|.
\end{itemize}

%%%%%%%%%%%%%%%%%%%%%%%%%%%%%%%%%%%%%%%%%%%%%%%%%%%%%%%%%%%%%%%%%%%%%%%%%%%%%%%%
\subsection{Related CTAN Packages}

There are several other packages which offer a similar functionality:
%
\begin{itemize}
\item
The packages
\href{http://ctan.org/pkg/docmute}{\textsf{docmute}},
\href{http://ctan.org/pkg/includex}{\textsf{includex}} and
\href{http://ctan.org/pkg/standalone}{\textsf{standalone}}
provide commands to include only the document body of
a child file thus allowing both files to be compiled individually.
\item
The packages \href{http://ctan.org/pkg/subdocs}{\textsf{subdocs}}
and \href{http://ctan.org/pkg/subfiles}{\textsf{subfiles}}
provide structures in which the main and child documents can be
encapsulated and allowing them to be compiled individually.
The inclusion mechanism is different from the conventional |\include|.
\item
The package \href{http://ctan.org/pkg/combine}{\textsf{combine}}
is an elaborate solution to combine several documents into one.
\end{itemize}
%
See also the CTAN topic \href{http://ctan.org/topic/subdocs}{\textsf{subdocs}}
for further related packages.
The present package differs from the above solutions in that
a document structure constructed with the conventional |\include| mechanism
just needs two extra commands at the top of every file
such that all constituent files can be compiled individually.

%%%%%%%%%%%%%%%%%%%%%%%%%%%%%%%%%%%%%%%%%%%%%%%%%%%%%%%%%%%%%%%%%%%%%%%%%%%%%%%%
%\subsection{Feature Suggestions}
%
%The following is a list of features which may be useful for future
%versions of this package:
%%
%\begin{itemize}
%\item
%\ldots
%\end{itemize}

%%%%%%%%%%%%%%%%%%%%%%%%%%%%%%%%%%%%%%%%%%%%%%%%%%%%%%%%%%%%%%%%%%%%%%%%%%%%%%%%
\subsection{Revision History}

%%%%%%%%%%%%%%%%%%%%%%%%%%%%%%%%%%%%%%%%
\paragraph{v2.0:} 2018/12/30

\begin{itemize}
\item
immediate forward processing
\item
added |\childdocby| mechanism
\item
manual restructured
\end{itemize}

%%%%%%%%%%%%%%%%%%%%%%%%%%%%%%%%%%%%%%%%
\paragraph{v1.6:} 2018/01/17

\begin{itemize}
\item
application for development of include files
\item
corrections to manual
\end{itemize}

%%%%%%%%%%%%%%%%%%%%%%%%%%%%%%%%%%%%%%%%
\paragraph{v1.5:} 2017/05/21

\begin{itemize}
\item
more complete structuring introduced
\item
|\childdocof| introduced
\item
|\childdoc| renamed to |\childdocmain|
\item
|\childredirect| renamed to |\childdocforward| and |\childdocforwardprefix|
and functionality expanded
\end{itemize}

%%%%%%%%%%%%%%%%%%%%%%%%%%%%%%%%%%%%%%%%
\paragraph{v1.0:} 2017/04/27

\begin{itemize}
\item
manual and install package
\item
first version published on CTAN
\end{itemize}

%%%%%%%%%%%%%%%%%%%%%%%%%%%%%%%%%%%%%%%%
\paragraph{v0.6:} 2017/04/26

\begin{itemize}
\item
redirection mechanism added
\end{itemize}

%%%%%%%%%%%%%%%%%%%%%%%%%%%%%%%%%%%%%%%%
\paragraph{v0.5:} 2017/04/26

\begin{itemize}
\item
functionality in definition file
\end{itemize}


%%%%%%%%%%%%%%%%%%%%%%%%%%%%%%%%%%%%%%%%%%%%%%%%%%%%%%%%%%%%%%%%%%%%%%%%%%%%%%%%
%%%%%%%%%%%%%%%%%%%%%%%%%%%%%%%%%%%%%%%%%%%%%%%%%%%%%%%%%%%%%%%%%%%%%%%%%%%%%%%%
%%%%%%%%%%%%%%%%%%%%%%%%%%%%%%%%%%%%%%%%%%%%%%%%%%%%%%%%%%%%%%%%%%%%%%%%%%%%%%%%
\appendix

\settowidth\MacroIndent{\rmfamily\scriptsize 000\ }

 \DocInput{childdoc.dtx}

\end{document}
%</driver>
% \fi
%
% %%%%%%%%%%%%%%%%%%%%%%%%%%%%%%%%%%%%%%%%%%%%%%%%%%%%%%%%%%%%%%%%%%%%%%%%%%%%%%
% %%%%%%%%%%%%%%%%%%%%%%%%%%%%%%%%%%%%%%%%%%%%%%%%%%%%%%%%%%%%%%%%%%%%%%%%%%%%%%
% \section{Sample}
%\iffalse
%<*samplemain>
%\fi
%
% The following presents a sample document
% with two chapters, two parts, a title page,
% a compile flag as well as three forwarding files to set the flag.
% It consists of eight |.tex| files:
% \begin{center}
% \begin{tabular}{ll}
% |cdocsamp.tex|&main file\\
% |cdocsch1.tex|&include file for chapter 1\\
% |cdocsch2.tex|&include file for chapter 2\\
% |cdocspt3.tex|&include file for part 3\\
% |cdocspt4.tex|&include file for part 4\\
% |cdocsdrf.tex|&forwarding file for main file in draft mode\\
% |cdocsfi1.tex|&forwarding file for final version of chapter 1\\
% |cdocsfi2.tex|&forwarding file for final version of chapter 2\\
% \end{tabular}
% \end{center}
% Each of the eight files can be compiled directly by the \LaTeX{} compiler.
%
% %%%%%%%%%%%%%%%%%%%%%%%%%%%%%%%%%%%%%%
% \paragraph{Main File.}
%
% The main file is called |cdocsamp.tex|.
%
% Load the \textsf{childdoc} definitions and
% declare the filename for the main document:
%    \begin{macrocode}
\input{childdoc.def}
\childdocmain{}
%    \end{macrocode}

% Optional override for |\version| flag:
%    \begin{macrocode}
%%\ifchilddoc\else\providecommand{\version}{draft}\fi
%    \end{macrocode}

% Define the default values for the |\version| flag
% (|final| for the main file and |draft| for childs):
%    \begin{macrocode}
\ifchilddoc
\providecommand{\version}{draft}
\else
\providecommand{\version}{final}
\fi
%    \end{macrocode}

% Load the standard document class:
%    \begin{macrocode}
\documentclass[12pt]{article}
%    \end{macrocode}

% Start the document body:
%    \begin{macrocode}
\begin{document}
%    \end{macrocode}

% Declare a title page.
% Print title, part of document being processed and version flag:
%    \begin{macrocode}
\addtocounter{page}{-1}
\begin{center}
{\LARGE\bfseries{}childdoc example\par}
\vspace{1cm}
\ifchilddoc
\ifchilddocmanual part\else chapter\fi:
`\childdocname' of `\childdocjob'\par
\else
main document: `\childdocjob'\par
\fi
version: \version\par
\end{center}
\newpage
%    \end{macrocode}

% Manually include selected file,
% otherwise process as usual:
%    \begin{macrocode}
\ifchilddocmanual
\section*{part `\childdocname'}
\input{\childdocname}
\else
%    \end{macrocode}

% Include the two chapters:
%    \begin{macrocode}
\include{cdocsch1}
\include{cdocsch2}
%    \end{macrocode}

% Include the two parts unless only chapters should be displayed:
%    \begin{macrocode}
\ifchilddoc\else
\section{part three}
\input{cdocspt3}
\section{part four}
\input{cdocspt4}
\fi
%    \end{macrocode}

% Process as usual until here:
%    \begin{macrocode}
\fi
%    \end{macrocode}

% End of document body:
%    \begin{macrocode}
\end{document}
%    \end{macrocode}
%\iffalse
%</samplemain>
%\fi
%
% %%%%%%%%%%%%%%%%%%%%%%%%%%%%%%%%%%%%%%
% \paragraph{Chapter Include Files.}
%
% The include files are called |cdocsch1.tex| and |cdocsch2.tex|.
%
%\iffalse
%<*samplechap1|samplechap2>
%\fi

% Optional override for |\version| flag:
%    \begin{macrocode}
%%\providecommand{\version}{final}
%    \end{macrocode}

% Include the main document:
%    \begin{macrocode}
\input{childdoc.def}
\childdocof{cdocsamp}
%    \end{macrocode}

%\iffalse
%</samplechap1|samplechap2>
%\fi
%
%\iffalse
%<*samplechap1>
%\fi
% Some text for chapter 1:
%    \begin{macrocode}
\section{one}
some text in chapter one
%    \end{macrocode}

%\iffalse
%</samplechap1>
%\fi
% Some text for chapter 2:
%\iffalse
%<*samplechap2>
%\fi
%    \begin{macrocode}
\section{two}
more text in chapter two
%    \end{macrocode}

%\iffalse
%</samplechap2>
%\fi
%
% %%%%%%%%%%%%%%%%%%%%%%%%%%%%%%%%%%%%%%
% \paragraph{Part Include Files.}
%
% The include files are called |cdocspt3.tex| and |cdocspt4.tex|.
%
%\iffalse
%<*samplepart3|samplepart4>
%\fi

% Optional override for |\version| flag:
%    \begin{macrocode}
%%\providecommand{\version}{final}
%    \end{macrocode}

% Include the main document:
%    \begin{macrocode}
\input{childdoc.def}
\childdocby{cdocsamp}
%    \end{macrocode}

%\iffalse
%</samplepart3|samplepart4>
%\fi
%
%\iffalse
%<*samplepart3>
%\fi
% Some text for part 3:
%    \begin{macrocode}
some text in part three
%    \end{macrocode}

%\iffalse
%</samplepart3>
%\fi
% Some text for part 4:
%\iffalse
%<*samplepart4>
%\fi
%    \begin{macrocode}
more text in part four
%    \end{macrocode}

%\iffalse
%</samplepart4>
%\fi
%
% %%%%%%%%%%%%%%%%%%%%%%%%%%%%%%%%%%%%%%
% \paragraph{Forwarding for a Complete Draft.}
%
% The following forwarding file |cdocsdrf.tex|
% compiles the main document in draft mode:
%\iffalse
%<*sampledraft>
%\fi
%    \begin{macrocode}
\def\version{draft}
\input{childdoc.def}
\childdocforward{cdocsamp}
%    \end{macrocode}

%\iffalse
%</sampledraft>
%\fi
%
% %%%%%%%%%%%%%%%%%%%%%%%%%%%%%%%%%%%%%%
% \paragraph{Forwarding for Final Version of the Chapters.}
%
% The following forwarding files |cdocsfn1.tex| and |cdocsfn2.tex|
% (with identical content)
% compile the final versions of the child documents
% |cdocsch1.tex| and |cdocsch2.tex|, respectively:
%\iffalse
%<*samplefinal>
%\fi
%    \begin{macrocode}
\def\version{final}
\input{childdoc.def}
\childdocforwardprefix[cdocsamp]{cdocsfn}{cdocsch}
%    \end{macrocode}

%\iffalse
%</samplefinal>
%\fi
%
% %%%%%%%%%%%%%%%%%%%%%%%%%%%%%%%%%%%%%%
% \paragraph{Command Line Processing.}
%
% The following three command lines generate the output files
% |cdocscld|, |cdocscl1| and |cdocscl2|
% which should be identical to
% |cdocsdrf|, |cdocsch1| and |cdocsfn2|, respectively:
% \begin{center}
% \begin{tabular}{l}
% |latex -jobname cdocscld \|\\
% |  "\def\version{draft}\input{childdoc.def}\childdocforward{cdocsamp}"|\\
% |latex -jobname cdocscl1 \|\\
% |  "\input{childdoc.def}\childdocforward[cdocsamp]{cdocsch1}"|\\
% |latex -jobname cdocscl2 \|\\
% |  "\def\version{final}\input{childdoc.def}\childdocforward{cdocsch2}"|
% \end{tabular}
% \end{center}
% Note that the trailing backslash on each first line
% merely continues the input to the second line
% (for convenient cut ant paste).
% Furthermore, the command |latex| can be replaced by any
% of its alternative versions such as |pdflatex|.
%
% %%%%%%%%%%%%%%%%%%%%%%%%%%%%%%%%%%%%%%%%%%%%%%%%%%%%%%%%%%%%%%%%%%%%%%%%%%%%%%
% %%%%%%%%%%%%%%%%%%%%%%%%%%%%%%%%%%%%%%%%%%%%%%%%%%%%%%%%%%%%%%%%%%%%%%%%%%%%%%
% \section{Implementation}
%\iffalse
%<*package>
%\fi
%
% This section describes the definitions file |childdoc.def|.

% The definitions cannot be loaded using |\usepackage| or |\RequirePackage|
% which has a mechanism to prevent loading a style file more than once.
% When loading the definitions by means of |\input|
% multiple instances have to be prevented manually:
%\iffalse
%This code needs to be before the `\ProvidesFile' directive
%which is defined at the beginning of this file.
%Therefore it is also placed there and commented out here.
%</package>
%<*discard>
%\fi
%    \begin{macrocode}
\ifdefined\childdocmain\endinput\fi
%    \end{macrocode}
%\iffalse
%</discard>
%<*package>
%\fi
%
% \macro{\ifchilddoc}
% \macro{\ifchilddocmanual}
% The conditional |\ifchilddoc| tells whether a
% child (true) or main (false) document is being compiled.
% The conditional |\ifchilddocmanual| tells whether
% the |\includeonly| mechanism is used (false) or
% the selection of child files must be performed manually (true).
% The definitions initialise to false:
%    \begin{macrocode}
\newif\ifchilddoc
\newif\ifchilddocmanual
%    \end{macrocode}

% \macro{\childdocname}
% \macro{\childdocjob}
% The macro |\childdocname| stores the name of the main document
% to be compiled. The macro |\childdocjob| stores the name of
% the document on which the \LaTeX{} compiler was originally invoked.
% The content of |\jobname| cannot be compared
% to filenames specified in the source due to different catcodes.
% The following code rescans |\jobname|, stores the result
% in |\childdocname| and saves a copy in |\childdocjob|:
%    \begin{macrocode}
\edef\childdocname{\scantokens\expandafter{\jobname\noexpand}}
\let\childdocjob\childdocname
%    \end{macrocode}

% \macro{\childdocdisable}
% The macro |\childdocdisable| prevents the main file
% from being processed more than once.
% At this stage, the main document command |\childdocmain|
% is assumed to be called once again where it should do nothing.
% Any subsequent call to it should prevent
% a secondary processing of the main document
% It overwrites the forwarding commands
% |\childdocof| and |\childdocforward|
% with empty macros to prevent further inclusions of the main document:
%    \begin{macrocode}
\newcommand{\childdocdisable}
{
  \renewcommand{\childdocmain}[1]{\renewcommand{\childdocmain}[1]{\endinput}}
  \renewcommand{\childdocof}[1]{}
  \renewcommand{\childdocby}[2][]{}
  \renewcommand{\childdocforward}[2][]{}
  \renewcommand{\childdocdisable}{}
}
%    \end{macrocode}

% \macro{\childdocmain}
% The macro |\childdocmain| is to be called at the top of the main file
% with nothing or the main filename (without extension) as argument.
% First, it breaks loops.
% If the argument is not empty and does not match |\childdocname|
% (which is set by the first inclusion of |childdoc.def|),
% |\ifchilddoc| is set to true, |\includeonly| is applied to the child file
% and |\jobname| is set to the main file
% (for proper handling of |.aux| files):
%    \begin{macrocode}
\newcommand{\childdocmain}[1]
{
  \childdocdisable\childdocmain{}
  \if?#1?\else
    \begingroup
      \def\childdoctmp{#1}
      \ifx\childdoctmp\childdocname
        \def\childdoctmp{}
      \else
        \def\childdoctmp
        {
          \childdoctrue
          \includeonly{\childdocname}
          \def\childdocjob{#1}
          \def\jobname{#1}
        }
      \fi
      \expandafter
    \endgroup
    \childdoctmp
  \fi
}
%    \end{macrocode}

% \macro{\childdocof}
% The command |\childdocof| redirects
% compilation to the main file |#1|.
%    \begin{macrocode}
\newcommand{\childdocof}[1]
{
  \childdocdisable
  \childdoctrue
  \includeonly{\childdocname}
  \def\jobname{#1}
  \def\childdocjob{#1}
  \input{#1}
}
%    \end{macrocode}

% \macro{\childdocby}
% The command |\childdocby| ....
%    \begin{macrocode}
\newcommand{\childdocby}[2][]
{
  \childdocdisable
  \childdoctrue
  \childdocmanualtrue
  \if?#1?\else
    \def\jobname{#2}
  \fi
  \def\childdocjob{#2}
  \input{#2}
  \endinput
}
%    \end{macrocode}

% \macro{\childdocforward}
% The command |\childdocforward| redirects
% compilation to the main file or
% (if the optional argument is given) a child file.
% Parameters are set as if the main file
% or a child file starting with |\childdocof| was compiled.
% Then compilation is handed over to the main file:
%    \begin{macrocode}
\newcommand{\childdocforward}[2][]
{
  \begingroup
    \if?#1?
      \def\childdoctmp
      {
        \def\childdocname{#2}
        \def\childdocjob{#2}
        \def\jobname{#2}
        \input{#2}
        \endinput
      }
    \else
      \def\childdoctmp
      {
        \childdocdisable
        \def\childdocname{#2}
        \childdoctrue
        \includeonly{#2}
        \def\childdocjob{#1}
        \def\jobname{#1}
        \input{#1}
        \endinput
      }
    \fi
    \expandafter
  \endgroup
  \childdoctmp
}
%    \end{macrocode}

% \macro{\childdocforwardprefix}
% The command |\childdocforwardprefix| redirects
% compilation to the main or a child file by means of a pattern.
% The prefix |#1| in the current filename is replaced by |#2|
% and the suffix of the current filename is kept
% (it is assumed that the filename does not contain the substring `|~~~|'
% which is used as a delimiter).
% Compilation is handed over to the new file by |\childdocforward|:
%    \begin{macrocode}
\newcommand{\childdocforwardprefix}[3][]
{
  \begingroup
    \def\childdocextract #2##1~~~{\def\childdoctmp{\childdocforward[#1]{#3##1}}}
    \expandafter\childdocextract\childdocname~~~
    \expandafter
  \endgroup
  \childdoctmp
}
%    \end{macrocode}

% \macro{\childdoc}
% The deprecated macro |\childdoc| is a legacy version of |\childdocmain|:
%    \begin{macrocode}
\newcommand{\childdoc}{\childdocmain}
%    \end{macrocode}

% \macro{\childdocredirect}
% The deprecated macro |\childdocredirect| is a legacy version
% of |\childdocforward| and |\childdocforwardprefix|:
%    \begin{macrocode}
\newcommand{\childdocredirect}[2][]
{
  \begingroup
    \if?#1?
      \def\childdoctmp{\childdocforward{#2}}
    \else
      \def\childdoctmp{\childdocforwardprefix{#1}{#2}}
    \fi
    \expandafter
  \endgroup
  \childdoctmp
}
%    \end{macrocode}

%\iffalse
%</package>
%\fi
%
\endinput
\childdocforward[|\textit{main}|]{|\textit{dest}|}"|
\end{center}
%
Here \textit{target} is the name of the output file,
\textit{main} is the name of the main file
and \textit{dest} is the name of the main or child file to be processed
(all filenames without extensions).
The optional argument \textit{main} can be omitted
if \textit{main} matches \textit{dest}.
Optionally, compilation \textit{flags} can be defined via |\def| commands.
This command line makes the \TeX{} engine believe
it is compiling the file \textit{target}
whose content is specified as the latter parameter.
The provided code then forwards the processing to
\textit{main} or \textit{dest} as described in \secref{sec:forward}.

%%%%%%%%%%%%%%%%%%%%%%%%%%%%%%%%%%%%%%%%%%%%%%%%%%%%%%%%%%%%%%%%%%%%%%%%%%%%%%%%
\subsection{Include by Input}
\label{sec:input}

Including child documents by |\include| has some restrictions by design.
Most notably, the content of a child document always occupies
its own set of pages; pages cannot be shared between child documents.
Usually, this behaviour makes perfect sense
because each child document contain an essential part of the document.
However, in some situations it may be desirable to compose
a document from a collection of parts
without having mandatory page breaks between then.
For this case, the package
provides a mechanism to include parts
by |\input| which can also be processed individually.
However, by construction this mechanism
requires manual handling of the content to be output.

%%%%%%%%%%%%%%%%%%%%%%%%%%%%%%%%%%%%%%%%
\DescribeMacro{\ifchilddocmanual}
The main file should be prepared as usual, see \secref{sec:include}.
However, the document body must make a distinction
between processing of an individual part and of the main document, e.g.:
%
\begin{center}
\begin{tabular}{l}
|\ifchilddocmanual|\\
|\input{\childdocname}|\\
|\||else|\\
\textit{document body with }|\input{|\textit{part}|}|\\
|\||fi|
\end{tabular}
\end{center}
%
The conditional |\ifchilddocmanual| is true whenever
a part to be included by |\input| is being compiled,
and the name of the part is stored in |\childdocname|.

%%%%%%%%%%%%%%%%%%%%%%%%%%%%%%%%%%%%%%%%
\DescribeMacro{\childdocby}
Each part to be included by |\input| should start with:
%
\begin{center}
\begin{tabular}{l}
|% \iffalse
%
% childdoc.dtx Copyright (C) 2017-2018 Niklas Beisert
%
% This work may be distributed and/or modified under the
% conditions of the LaTeX Project Public License, either version 1.3
% of this license or (at your option) any later version.
% The latest version of this license is in
%   http://www.latex-project.org/lppl.txt
% and version 1.3 or later is part of all distributions of LaTeX
% version 2005/12/01 or later.
%
% This work has the LPPL maintenance status `maintained'.
%
% The Current Maintainer of this work is Niklas Beisert.
%
% This work consists of the files childdoc.dtx and childdoc.ins
% and the derived files childdoc.def and cdocsamp.tex with
% cdocsch1.tex, cdocsch2.tex, cdocsdrf.tex, cdocsfn1.tex, cdocsfn2.tex.
%
%<package>\ifdefined\childdocmain\endinput\fi
%<package>\ProvidesFile{childdoc.def}[2018/12/30 v2.0 child document driver]
%<samplemain>\ProvidesFile{cdocsamp.tex}[2018/12/30 v2.0 sample for childdoc]
%<*driver>
%\ProvidesFile{childdoc.drv}[2018/12/30 v2.0 childdoc reference manual file]
\PassOptionsToClass{10pt,a4paper}{article}
\documentclass{ltxdoc}

\usepackage[margin=35mm]{geometry}
\usepackage{hyperref}
\usepackage{hyperxmp}
\usepackage[usenames]{color}

\hypersetup{colorlinks=true}
\hypersetup{pdfstartview=FitH}
\hypersetup{pdfpagemode=UseNone}
\hypersetup{pdfsource={}}
\hypersetup{pdflang={en-UK}}
\hypersetup{pdfcopyright={Copyright 2017-2018 Niklas Beisert.
  This work may be distributed and/or modified under the
  conditions of the LaTeX Project Public License, either version 1.3
  of this license or (at your option) any later version.}}
\hypersetup{pdflicenseurl={http://www.latex-project.org/lppl.txt}}
\hypersetup{pdfcontactaddress={ETH Zurich, ITP, HIT K,
  Wolfgang-Pauli-Strasse 27}}
\hypersetup{pdfcontactpostcode={8093}}
\hypersetup{pdfcontactcity={Zurich}}
\hypersetup{pdfcontactcountry={Switzerland}}
\hypersetup{pdfcontactemail={nbeisert@itp.phys.ethz.ch}}
\hypersetup{pdfcontacturl={http://people.phys.ethz.ch/\xmptilde nbeisert/}}

\newcommand{\secref}[1]{\hyperref[#1]{section \ref*{#1}}}

\parskip1ex
\parindent0pt
\let\olditemize\itemize
\def\itemize{\olditemize\parskip0pt}

\begin{document}

\title{The \textsf{childdoc} Package}
\hypersetup{pdftitle={The childdoc Package}}
\author{Niklas Beisert\\[2ex]
  Institut f\"ur Theoretische Physik\\
  Eidgen\"ossische Technische Hochschule Z\"urich\\
  Wolfgang-Pauli-Strasse 27, 8093 Z\"urich, Switzerland\\[1ex]
  \href{mailto:nbeisert@itp.phys.ethz.ch}
  {\texttt{nbeisert@itp.phys.ethz.ch}}}
\hypersetup{pdfauthor={Niklas Beisert}}
\hypersetup{pdfsubject={Manual for the LaTeX2e Package childdoc}}
\date{30 December 2018, \textsf{v2.0}}
\maketitle

\begin{abstract}\noindent
\textsf{childdoc} is a \LaTeXe{} package
that enables the direct compilation
of document sections included by |\include|
to individual files.
\end{abstract}

\begingroup
\parskip0ex
\tableofcontents
\endgroup

%%%%%%%%%%%%%%%%%%%%%%%%%%%%%%%%%%%%%%%%%%%%%%%%%%%%%%%%%%%%%%%%%%%%%%%%%%%%%%%%
%%%%%%%%%%%%%%%%%%%%%%%%%%%%%%%%%%%%%%%%%%%%%%%%%%%%%%%%%%%%%%%%%%%%%%%%%%%%%%%%
\section{Introduction}

\LaTeX{} provides a mechanism to structure a large document (such as a book)
into a main file and several child files (containing the chapters)
using the |\include| command.
This mechanism is beneficial for documents
which span hundreds of pages in order to
make the source file(s) more manageable.
Moreover, compilation can be restricted to
selected child files by means of the |\includeonly| command.
The latter feature can be used to reduce the compilation time while editing
(this was significantly more useful in the earlier days of \LaTeX{})
or to generate a smaller document which is easier to navigate.
Another application of |\includeonly| is to generate
documents consisting of selected parts of the complete document.

However, there are a few drawbacks of the plain |\include| mechanism:
\begin{itemize}
\item
The child files cannot be compiled on their own,
they can only be compiled via the main file.
A naive editing environment
(such as a text editor with an option
to have the current file processed by \LaTeX)
may require one to switch to the main file before compiling;
attempting to compile the child file produces errors.
\item
The main file must be modified (each time)
to adjust the |\includeonly| command
to the present needs. This easily leaves the main file in a messy state.
\item
The generated document will always carry the filename
of the main document. This is inconvenient if
several child files are to be compiled and
to be kept for distribution.
\end{itemize}

The present package provides a simple interface
to make child files individually compilable by \LaTeX{}.
Compiling a child file then has the same effect as compiling
the main file with an |\includeonly| command
to select the appropriate child.
Moreover the generated document will carry the name of the child
rather than the main file.
This resolves all three above issues.

This feature is meant to make the editing of books,
thesis documents and lecture notes somewhat more convenient.
However, the package can also be used efficiently for
composing a series of documents (such as exercise sheets)
which are typically distributed individually.
It then assists the author in generating the individual documents
(potentially in different versions)
as well as a document containing the collected series.
Another application is in developing style files
or other kinds of included material
where compilation of the style file could redirect
to a sample or test file.

%%%%%%%%%%%%%%%%%%%%%%%%%%%%%%%%%%%%%%%%%%%%%%%%%%%%%%%%%%%%%%%%%%%%%%%%%%%%%%%%
%%%%%%%%%%%%%%%%%%%%%%%%%%%%%%%%%%%%%%%%%%%%%%%%%%%%%%%%%%%%%%%%%%%%%%%%%%%%%%%%
\section{Usage}

First of all, the package \textsf{childdoc} is \emph{not} a standard
\LaTeXe{} |.sty| style file! Therefore it needs to be invoked in
a non-standard way.

%%%%%%%%%%%%%%%%%%%%%%%%%%%%%%%%%%%%%%%%%%%%%%%%%%%%%%%%%%%%%%%%%%%%%%%%%%%%%%%%
\subsection{Included Files}
\label{sec:include}

%%%%%%%%%%%%%%%%%%%%%%%%%%%%%%%%%%%%%%%%
\DescribeMacro{\childdocmain}
To use the package, add the commands
\begin{center}
\begin{tabular}{l}
|\input{childdoc.def}|\\
|\childdocmain{}|\\
\end{tabular}
\end{center}
at the very top of the main \LaTeX{} file,
in particular \emph{before} the |\documentclass| statement!
The argument of |\childdocmain| should be left empty
(but it must be present).

%%%%%%%%%%%%%%%%%%%%%%%%%%%%%%%%%%%%%%%%
\DescribeMacro{\childdocof}
Furthermore, add the commands
\begin{center}
\begin{tabular}{l}
|\input{childdoc.def}|\\
|\childdocof{|\textit{main}|}|\\
\end{tabular}
\end{center}
at the top of every child file \textit{child}
which is included by |\include{|\textit{child}|}|
from within the main file
(or at least for those files to be compiled individually).
The argument \textit{main} must be the filename of the main file.

There are a couple of
considerations in setting up the main and child documents:

%%%%%%%%%%%%%%%%%%%%%%%%%%%%%%%%%%%%%%%%
\paragraph{Restrictions.}

Please note the following restrictions:
\begin{itemize}
\item
|\childdocmain| must be called with one argument \textit{main}
to ensure compatibility with earlier version of the package.
It must either be empty (|\childdocmain{}|)
or precisely match the filename of the main file in which it is specified.
See \secref{sec:detection} for further information.
\item
The filename \textit{main} must be specified without the |.tex| extension.
\item
The filename \textit{main} is case sensitive
(even in case-insensitive file systems)
due to internal string comparison.
\item
The argument \textit{main} should be fully expanded, it cannot be a macro.
\item
Subdirectories and special characters should be avoided in filenames.
\item
The command |\childdocmain{|\textit{main}|}| must be followed by a whitespace.
It should not be followed immediately by another command
or by a comment mark `|%|'.
This is because the \TeX{} parser reads the token immediately following
the argument of |\childdocmain| and puts it
at the beginning of every child section;
however, a white\-space is ignored.
\end{itemize}

%%%%%%%%%%%%%%%%%%%%%%%%%%%%%%%%%%%%%%%%
\paragraph{Content of Main File.}

It is advisable to place all content in the child files included by |\include|.
Any output contained in the main file will appear in all child documents
unless suppressed manually;
it cannot be suppressed automatically by the |\includeonly| directive
and thus should normally be avoided.
A method to include some content in the main file
by means of conditional processing is described in \secref{sec:conditional}.

%%%%%%%%%%%%%%%%%%%%%%%%%%%%%%%%%%%%%%%%
\paragraph{Page Numbering.}

When only a part of the document is compiled,
the appropriate numbering of pages
(as well as other status parameters)
is determined from the |.aux| files.
The latter contain information from previous passes.
However this information needs to propagate through
all intermediate child documents.
Therefore the page numbering in child documents may well
be inconsistent until the complete document is compiled at least once.

A useful (if unconventional) way to always ensure a consistent
page numbering is to restart the numbering in each child document
and denote the pages by `\textit{child}|.|\textit{page}'
where \textit{child} represents the chapter/section number of the child file.
This can be achieved by the command
|\numberwithin{page}{|\textit{child}|}|
of the \textsf{amsmath} package
where \textit{child} can be |chapter| or |section|
depending on the chosen structuring.
Alternatively, one can modify the macro |\thepage| appropriately
and reset the counter |page| at the start of each child file.

%%%%%%%%%%%%%%%%%%%%%%%%%%%%%%%%%%%%%%%%%%%%%%%%%%%%%%%%%%%%%%%%%%%%%%%%%%%%%%%%
\subsection{Conditional Processing}
\label{sec:conditional}

The package provides a mechanism to compile different versions
of a document. To customise the versions further some conditional processing
can come in handy to distinguish which version is being compiled.
The package provides two macros to describe the compilation context:

%%%%%%%%%%%%%%%%%%%%%%%%%%%%%%%%%%%%%%%%
\DescribeMacro{\ifchilddoc}
The conditional |\ifchilddoc| distinguishes between the compilation of
child documents and the main document:
%
\begin{center}
|\ifchilddoc |\textit{child-code}| |[|\||else |\textit{main-code}]| \||fi|
\end{center}

%%%%%%%%%%%%%%%%%%%%%%%%%%%%%%%%%%%%%%%%
\DescribeMacro{\childdocname}
\DescribeMacro{\childdocjob}
The macro |\childdocname| contains the filename (without extension)
of the main or child file being processed.
Note that |\childdocjob| will always contain the name of the main file.

%%%%%%%%%%%%%%%%%%%%%%%%%%%%%%%%%%%%%%%%
\paragraph{Title Page.}

Conditional processing can be used to include a title or banner page
in the main document when proper precautions are taken.
Importantly, the code in the main file should ensure that the page counter
(as well as other status parameters which are stored in the |.aux| files)
takes the same value after the conditional processing.
Otherwise the page numbers may take divergent values
depending on which part is compiled.

For example, a title page could be declared by:
%
\begin{center}
\begin{tabular}{l}
|\ifchilddoc\||else|\\
|\addtocounter{page}{-1}|\\
\textit{code for title page}\\
|\newpage|\\
|\||fi|
\end{tabular}
\end{center}
%
A banner page for the child documents can be generated by:
%
\begin{center}
\begin{tabular}{l}
|\ifchilddoc|\\
|\addtocounter{page}{-1}|\\
\textit{code for banner page}\\
|\newpage|\\
|\||fi|
\end{tabular}
\end{center}
%
Here one could write a message such as:
\begin{center}
|This is the part \childdocname{} of \childdocjob{}.|
\end{center}

%%%%%%%%%%%%%%%%%%%%%%%%%%%%%%%%%%%%%%%%%%%%%%%%%%%%%%%%%%%%%%%%%%%%%%%%%%%%%%%%
\subsection{Flags}
\label{sec:flags}

The package makes it easy to generate different versions
of the main or child documents.
To this end compilation flags can be defined
and assigned different default values.
They will be particularly useful in conjunction
with the forwarding mechanism described in \secref{sec:forward}.

For example, it may be useful to have a flag |\version|
which can be set to |draft| or |final|.
The document source will contain some conditional code
depending on the value of |\version|.
Suppose further, the flag should default to |final| for the main file
and to |draft| for child files
which is a natural assignment for editing the document.
This is achieved by placing the following code
in the preamble of the main document
(below the |\childdocmain| directive):
%
\begin{center}
\begin{tabular}{l}
|\ifchilddoc|\\
|\providecommand{\version}{draft}|\\
|\||else|\\
|\providecommand{\version}{final}|\\
|\||fi|
\end{tabular}
\end{center}
%
The definition by |\providecommand| makes sure
that previous definitions are not overwritten.
Further statements |\providecommand{\version}{...}|
can thus be added before the above code to override it.

For the main file, one might add a line
(between |\childdocmain| and the above block)
%
\begin{center}
|%\ifchilddoc\||else\providecommand{\version}{draft}\||fi|
\end{center}
%
which can be uncommented to produce a draft version.
Likewise one can add a line to the very top of a child file
(above the |\childdocof{|\textit{main}|}| directive)
%
\begin{center}
|%\providecommand{\version}{final}|
\end{center}
%
which can be uncommented to produce the final version of this child document.

%%%%%%%%%%%%%%%%%%%%%%%%%%%%%%%%%%%%%%%%%%%%%%%%%%%%%%%%%%%%%%%%%%%%%%%%%%%%%%%%
\subsection{Forwarding}
\label{sec:forward}

Different versions of the main or child documents
using compilation flags as described in \secref{sec:flags}
can be (permanently) stored in different files
for convenient compilation, viewing and distribution.
To this end, the package defines a command
to pass on compilation to a different file:

%%%%%%%%%%%%%%%%%%%%%%%%%%%%%%%%%%%%%%%%
\DescribeMacro{\childdocforward}
The command |\childdocforward| redirects processing to
another source file:
%
\begin{center}
\begin{tabular}{l}
|\input{childdoc.def}|\\
|\childdocforward[|\textit{main}|]{|\textit{dest}|}|\\
\end{tabular}
\end{center}
%
The argument \textit{dest} is the destination file
(without extension).
It should be the main file or one of the child files.
Note that further \textsf{childdoc} directives
such as |\childdocof| and |\childdocforward|
in the indicated file will be processed in this form.
The optional argument \textit{main}
passes on directly to the main file \textit{main}
while pretending to compile the child \textit{dest}.
This form behaves as if \textit{dest}
issues |\childdocof{|\textit{main}|}| right away,
and no further \textsf{childdoc} directives will be processed.

%%%%%%%%%%%%%%%%%%%%%%%%%%%%%%%%%%%%%%%%
\DescribeMacro{\...prefix}
In the alternative form |\childdocforwardprefix|,
%
\begin{center}
\begin{tabular}{l}
|\input{childdoc.def}|\\
|\childdocforwardprefix[|\textit{main}|]{|\textit{prefix}|}{|\textit{dest}|}|
\end{tabular}
\end{center}
%
the destination file is determined by a pattern
depending on the current file:
To make this work, the current file must be called
`{\textit{prefix}\hspace{0.2em}\textit{suffix}}'
with \textit{prefix} matching precisely the argument.
Processing is then passed on to the file
`{\textit{dest}\hspace{0.2em}\textit{suffix}}'.
Surely, the same effect is achieved by
directly specifying the
argument `{\textit{dest}\hspace{0.2em}\textit{suffix}}'
in the first form.
However, that requires to set up a different file
for each child. With the alternative form of the command
all these files can have exactly the same content
which simplifies setting them up and maintaining them.

For example, the following file |draft.tex|
with a compilation flag |\version| as described in \secref{sec:flags}
compiles the main document as a draft:
%
\begin{center}
\begin{tabular}{l}
|\def\version{draft}|\\
|\input{childdoc.def}|\\
|\childdocforward{|\textit{main}|}|
\end{tabular}
\end{center}
%
Likewise, the following files |final|\textit{nn}|.tex|
compile the final version of the child document
|child|\textit{nn}|.tex|:
%
\begin{center}
\begin{tabular}{l}
|\def\version{final}|\\
|\input{childdoc.def}|\\
|\childdocforwardprefix{final}{child}|
\end{tabular}
\end{center}
%

Note that when several versions of a main file and/or of each child file
are to be generated, it may be convenient to set up a |Makefile| or
shell script to automatise the process.

%%%%%%%%%%%%%%%%%%%%%%%%%%%%%%%%%%%%%%%%%%%%%%%%%%%%%%%%%%%%%%%%%%%%%%%%%%%%%%%%
\subsection{Command Line Processing}
\label{sec:commandline}

The effect of redirection files can also be achieved by invoking
the \LaTeX{} compiler with a more elaborate command line.
Most conveniently this should be done as part
of a shell script or a |Makefile|.

When using \textsf{childdoc} in the main file, the following
command lines effectively perform a redirection
(note that depending on the shell being used,
backslashes may have to be doubled: `|\|' $\to$ `|\\|'):
%
\begin{center}
|... -jobname "|\textit{target}|" |\\|"|[\textit{flags}]%
|\input{childdoc.def}\childdocforward[|\textit{main}|]{|\textit{dest}|}"|
\end{center}
%
Here \textit{target} is the name of the output file,
\textit{main} is the name of the main file
and \textit{dest} is the name of the main or child file to be processed
(all filenames without extensions).
The optional argument \textit{main} can be omitted
if \textit{main} matches \textit{dest}.
Optionally, compilation \textit{flags} can be defined via |\def| commands.
This command line makes the \TeX{} engine believe
it is compiling the file \textit{target}
whose content is specified as the latter parameter.
The provided code then forwards the processing to
\textit{main} or \textit{dest} as described in \secref{sec:forward}.

%%%%%%%%%%%%%%%%%%%%%%%%%%%%%%%%%%%%%%%%%%%%%%%%%%%%%%%%%%%%%%%%%%%%%%%%%%%%%%%%
\subsection{Include by Input}
\label{sec:input}

Including child documents by |\include| has some restrictions by design.
Most notably, the content of a child document always occupies
its own set of pages; pages cannot be shared between child documents.
Usually, this behaviour makes perfect sense
because each child document contain an essential part of the document.
However, in some situations it may be desirable to compose
a document from a collection of parts
without having mandatory page breaks between then.
For this case, the package
provides a mechanism to include parts
by |\input| which can also be processed individually.
However, by construction this mechanism
requires manual handling of the content to be output.

%%%%%%%%%%%%%%%%%%%%%%%%%%%%%%%%%%%%%%%%
\DescribeMacro{\ifchilddocmanual}
The main file should be prepared as usual, see \secref{sec:include}.
However, the document body must make a distinction
between processing of an individual part and of the main document, e.g.:
%
\begin{center}
\begin{tabular}{l}
|\ifchilddocmanual|\\
|\input{\childdocname}|\\
|\||else|\\
\textit{document body with }|\input{|\textit{part}|}|\\
|\||fi|
\end{tabular}
\end{center}
%
The conditional |\ifchilddocmanual| is true whenever
a part to be included by |\input| is being compiled,
and the name of the part is stored in |\childdocname|.

%%%%%%%%%%%%%%%%%%%%%%%%%%%%%%%%%%%%%%%%
\DescribeMacro{\childdocby}
Each part to be included by |\input| should start with:
%
\begin{center}
\begin{tabular}{l}
|\input{childdoc.def}|\\
|\childdocby{|\textit{main}|}|\\
\end{tabular}
\end{center}
%
The directive |\childdocby| is similar to |\childdocof|
described in \secref{sec:include},
but the subsequent selection of content must be done manually.
To that end, both |\ifchilddoc| and |\ifchilddocmanual|
will be true upon processing of a part,
and the name of the part is stored in |\childdocname|.
Note that |\jobname| will be set to the filename of the current part
so that each part receives an individual |.aux| file
that does not interfere with the |.aux| file(s) of the main document.
This behaviour can be altered by the alternative form
|\childdocby[*]{|\textit{main}|}| (with a non-empty optional argument)
which uses the |.aux| file of the main document
by setting |\jobname| to \textit{main}.

%%%%%%%%%%%%%%%%%%%%%%%%%%%%%%%%%%%%%%%%%%%%%%%%%%%%%%%%%%%%%%%%%%%%%%%%%%%%%%%%
\subsection{Driver Development}
\label{sec:driver}

The \textsf{childdoc} mechanism can also be use for the development
of definition files such as \LaTeX{} styles or classes.
This case differs from the above setup with multiple parts
included by |\include| in that no |\includeonly| should be invoked.
This can be achieved by starting the include file
(before |\ProvidesPackage|) with:
%
\begin{center}
\begin{tabular}{l}
|\input{childdoc.def}|\\
|\childdocforward{|\textit{main}|}|\\
\end{tabular}
\end{center}
%
or alternatively with:
%
\begin{center}
\begin{tabular}{l}
|\input{childdoc.def}|\\
|\childdocby{|\textit{main}|}|\\
\end{tabular}
\end{center}
%
Both forms have slightly different effects as described above.
The main file is prepared as usual, see \secref{sec:include}.

%%%%%%%%%%%%%%%%%%%%%%%%%%%%%%%%%%%%%%%%%%%%%%%%%%%%%%%%%%%%%%%%%%%%%%%%%%%%%%%%
\subsection{Legacy Detection}
\label{sec:detection}

The directive |\childdocmain| in the main file can detect
whether the complete document or merely a child is to be compiled
even without using the directive |\childdocof|.
This method is deprecated because it is less robust
and there is no compelling reason to use it;
it is merely provided for backward compatibility
and it may be removed in future versions.

If the detection mechanism is to be used,
it is mandatory to correctly specify
the filename of the main file as the argument of |\childdocmain|:
%
\begin{center}
\begin{tabular}{l}
|\input{childdoc.def}|\\
|\childdocmain{|\textit{main}|}|\\
\end{tabular}
\end{center}
%
If |\jobname| does not match the argument \textit{main} of |\childdocmain|,
it is assumed that |\jobname| points to the child file to be compiled.
When using |\childdocmain| with the main file specified as argument,
it suffices to start a child file
with just |\input{|\textit{main}|}|
without loading of the package and using |\childdocof|.
If instead all processing is done
with the appropriate \textsf{childdoc} directives,
the argument of \textit{main} of |\childdocmain| can be empty.

An alternative version of the command line processing described
in \secref{sec:commandline} using the detection mechanism reads:
%
\begin{center}
|... -jobname "|\textit{target}|" "|[\textit{flags}]%
[|\def\jobname{|\textit{dest}|}|]|\input{|\textit{main}|}"|
\end{center}

%%%%%%%%%%%%%%%%%%%%%%%%%%%%%%%%%%%%%%%%%%%%%%%%%%%%%%%%%%%%%%%%%%%%%%%%%%%%%%%%
\subsection{Manual Code}
\label{sec:manual}

In case one cannot be certain whether the definitions file |childdoc.def|
is installed on the target \TeX{} distribution
and one prefers not to ship it,
it is conceivable to paste a few relevant commands into the sources.

To that end, drop all statements |\input{childdoc.def}|
and perform the replacements as outlined below.
Instead of |\childdocmain{|\textit{main}|}| add the following code
to the top of the main file:
%
\begin{center}
\begin{tabular}{l}
|\||ifdefined\childdocname\endinput\||fi\newif\ifchilddoc|\\
|\edef\childdocname{\scantokens\expandafter{\jobname\noexpand}}|\\
|\def\childdocmain{|\textit{main}|}\||ifx\childdocmain\childdocname\||else|\\
|\childdoctrue\includeonly{\childdocname}\let\jobname\childdocmain\||fi|\\
\end{tabular}
\end{center}
%
Instead of |\childdocof{|\textit{main}|}| just include the main file
at the top of each child file:
%
\begin{center}
|\input{|\textit{main}|}|
\end{center}
%
A simple redirection |\childdocforward{|\textit{dest}|}| is achieved by:
%
\begin{center}
|\def\jobname{|\textit{dest}|}\input{\jobname}|
\end{center}
%
The redirection with prefix
|\childdocforwardprefix[|\textit{prefix}|]{|\textit{dest}|}|
is accomplished by:
%
\begin{center}
\begin{tabular}{l}
|{\edef\jobname{\scantokens\expandafter{\jobname\noexpand}}|\\
|\def\redirectjob |\textit{prefix}|#1~~~{\gdef\jobname{|\textit{dest}|#1}}|\\
|\expandafter\redirectjob\jobname~~~}\input{\jobname}|
\end{tabular}
\end{center}

In an alternative approach,
child documents can be compiled by a specific command line
without additional code or specific definitions:
%
\begin{center}
|... -jobname "|\textit{target}|" "|[\textit{flags}]%
|\includeonly{|\textit{dest}|}\input{|\textit{main}|}"|
\end{center}
%

%%%%%%%%%%%%%%%%%%%%%%%%%%%%%%%%%%%%%%%%%%%%%%%%%%%%%%%%%%%%%%%%%%%%%%%%%%%%%%%%
%%%%%%%%%%%%%%%%%%%%%%%%%%%%%%%%%%%%%%%%%%%%%%%%%%%%%%%%%%%%%%%%%%%%%%%%%%%%%%%%
\section{Information}

%%%%%%%%%%%%%%%%%%%%%%%%%%%%%%%%%%%%%%%%%%%%%%%%%%%%%%%%%%%%%%%%%%%%%%%%%%%%%%%%
\subsection{Copyright}

Copyright \copyright{} 2017--2018 Niklas Beisert

This work may be distributed and/or modified under the
conditions of the \LaTeX{} Project Public License, either version 1.3
of this license or (at your option) any later version.
The latest version of this license is in
  \url{http://www.latex-project.org/lppl.txt}
and version 1.3 or later is part of all distributions of \LaTeX{}
version 2005/12/01 or later.

This work has the LPPL maintenance status `maintained'.

The Current Maintainer of this work is Niklas Beisert.

This work consists of the files |README.txt|, |childdoc.ins| and |childdoc.dtx|
as well as the derived files |childdoc.def|, |cdocsamp.tex|
with |cdocsch1.tex|, |cdocsch2.tex|, |cdocspt3.tex|, |cdocspt4.tex|,
|cdocsdrf.tex|, |cdocsfn1.tex|, |cdocsfn2.tex|
as well as |childdoc.pdf|.

%%%%%%%%%%%%%%%%%%%%%%%%%%%%%%%%%%%%%%%%%%%%%%%%%%%%%%%%%%%%%%%%%%%%%%%%%%%%%%%%
\subsection{Files and Installation}

The package consists of the files:
%
\begin{center}
\begin{tabular}{ll}
    |README.txt|   & readme file \\
    |childdoc.ins| & installation file \\
    |childdoc.dtx| & source file \\
    |childdoc.def| & definition file \\
    |cdocsamp.tex| & sample main file \\
    |cdocsch1.tex| & sample include file \\
    |cdocsch2.tex| & sample include file \\
    |cdocspt3.tex| & sample part file \\
    |cdocspt4.tex| & sample part file \\
    |cdocsdrf.tex| & sample redirection file \\
    |cdocsfn1.tex| & sample redirection file \\
    |cdocsfn2.tex| & sample redirection file \\
    |childdoc.pdf| & manual
\end{tabular}
\end{center}
%
The distribution consists of the files
|README.txt|, |childdoc.ins| and |childdoc.dtx|.
%
\begin{itemize}
\item
Run (pdf)\LaTeX{} on |childdoc.dtx|
to compile the manual |childdoc.pdf| (this file).
\item
Run \LaTeX{} on |childdoc.ins| to create the definitions file |childdoc.def|
and the sample |cdocsamp.tex| with include files
|cdocsch1.tex|, |cdocsch2.tex|, |cdocspt3.tex|, |cdocspt4.tex|,
|cdocsdrf.tex|, |cdocsfn1.tex|, |cdocsfn2.tex|.
Then copy the file |childdoc.def| to an appropriate directory of your \LaTeX{}
distribution, e.g.\ \textit{texmf-root}|/tex/latex/childdoc|.
\end{itemize}

%%%%%%%%%%%%%%%%%%%%%%%%%%%%%%%%%%%%%%%%%%%%%%%%%%%%%%%%%%%%%%%%%%%%%%%%%%%%%%%%
\subsection{Related CTAN Packages}

There are several other packages which offer a similar functionality:
%
\begin{itemize}
\item
The packages
\href{http://ctan.org/pkg/docmute}{\textsf{docmute}},
\href{http://ctan.org/pkg/includex}{\textsf{includex}} and
\href{http://ctan.org/pkg/standalone}{\textsf{standalone}}
provide commands to include only the document body of
a child file thus allowing both files to be compiled individually.
\item
The packages \href{http://ctan.org/pkg/subdocs}{\textsf{subdocs}}
and \href{http://ctan.org/pkg/subfiles}{\textsf{subfiles}}
provide structures in which the main and child documents can be
encapsulated and allowing them to be compiled individually.
The inclusion mechanism is different from the conventional |\include|.
\item
The package \href{http://ctan.org/pkg/combine}{\textsf{combine}}
is an elaborate solution to combine several documents into one.
\end{itemize}
%
See also the CTAN topic \href{http://ctan.org/topic/subdocs}{\textsf{subdocs}}
for further related packages.
The present package differs from the above solutions in that
a document structure constructed with the conventional |\include| mechanism
just needs two extra commands at the top of every file
such that all constituent files can be compiled individually.

%%%%%%%%%%%%%%%%%%%%%%%%%%%%%%%%%%%%%%%%%%%%%%%%%%%%%%%%%%%%%%%%%%%%%%%%%%%%%%%%
%\subsection{Feature Suggestions}
%
%The following is a list of features which may be useful for future
%versions of this package:
%%
%\begin{itemize}
%\item
%\ldots
%\end{itemize}

%%%%%%%%%%%%%%%%%%%%%%%%%%%%%%%%%%%%%%%%%%%%%%%%%%%%%%%%%%%%%%%%%%%%%%%%%%%%%%%%
\subsection{Revision History}

%%%%%%%%%%%%%%%%%%%%%%%%%%%%%%%%%%%%%%%%
\paragraph{v2.0:} 2018/12/30

\begin{itemize}
\item
immediate forward processing
\item
added |\childdocby| mechanism
\item
manual restructured
\end{itemize}

%%%%%%%%%%%%%%%%%%%%%%%%%%%%%%%%%%%%%%%%
\paragraph{v1.6:} 2018/01/17

\begin{itemize}
\item
application for development of include files
\item
corrections to manual
\end{itemize}

%%%%%%%%%%%%%%%%%%%%%%%%%%%%%%%%%%%%%%%%
\paragraph{v1.5:} 2017/05/21

\begin{itemize}
\item
more complete structuring introduced
\item
|\childdocof| introduced
\item
|\childdoc| renamed to |\childdocmain|
\item
|\childredirect| renamed to |\childdocforward| and |\childdocforwardprefix|
and functionality expanded
\end{itemize}

%%%%%%%%%%%%%%%%%%%%%%%%%%%%%%%%%%%%%%%%
\paragraph{v1.0:} 2017/04/27

\begin{itemize}
\item
manual and install package
\item
first version published on CTAN
\end{itemize}

%%%%%%%%%%%%%%%%%%%%%%%%%%%%%%%%%%%%%%%%
\paragraph{v0.6:} 2017/04/26

\begin{itemize}
\item
redirection mechanism added
\end{itemize}

%%%%%%%%%%%%%%%%%%%%%%%%%%%%%%%%%%%%%%%%
\paragraph{v0.5:} 2017/04/26

\begin{itemize}
\item
functionality in definition file
\end{itemize}


%%%%%%%%%%%%%%%%%%%%%%%%%%%%%%%%%%%%%%%%%%%%%%%%%%%%%%%%%%%%%%%%%%%%%%%%%%%%%%%%
%%%%%%%%%%%%%%%%%%%%%%%%%%%%%%%%%%%%%%%%%%%%%%%%%%%%%%%%%%%%%%%%%%%%%%%%%%%%%%%%
%%%%%%%%%%%%%%%%%%%%%%%%%%%%%%%%%%%%%%%%%%%%%%%%%%%%%%%%%%%%%%%%%%%%%%%%%%%%%%%%
\appendix

\settowidth\MacroIndent{\rmfamily\scriptsize 000\ }

 \DocInput{childdoc.dtx}

\end{document}
%</driver>
% \fi
%
% %%%%%%%%%%%%%%%%%%%%%%%%%%%%%%%%%%%%%%%%%%%%%%%%%%%%%%%%%%%%%%%%%%%%%%%%%%%%%%
% %%%%%%%%%%%%%%%%%%%%%%%%%%%%%%%%%%%%%%%%%%%%%%%%%%%%%%%%%%%%%%%%%%%%%%%%%%%%%%
% \section{Sample}
%\iffalse
%<*samplemain>
%\fi
%
% The following presents a sample document
% with two chapters, two parts, a title page,
% a compile flag as well as three forwarding files to set the flag.
% It consists of eight |.tex| files:
% \begin{center}
% \begin{tabular}{ll}
% |cdocsamp.tex|&main file\\
% |cdocsch1.tex|&include file for chapter 1\\
% |cdocsch2.tex|&include file for chapter 2\\
% |cdocspt3.tex|&include file for part 3\\
% |cdocspt4.tex|&include file for part 4\\
% |cdocsdrf.tex|&forwarding file for main file in draft mode\\
% |cdocsfi1.tex|&forwarding file for final version of chapter 1\\
% |cdocsfi2.tex|&forwarding file for final version of chapter 2\\
% \end{tabular}
% \end{center}
% Each of the eight files can be compiled directly by the \LaTeX{} compiler.
%
% %%%%%%%%%%%%%%%%%%%%%%%%%%%%%%%%%%%%%%
% \paragraph{Main File.}
%
% The main file is called |cdocsamp.tex|.
%
% Load the \textsf{childdoc} definitions and
% declare the filename for the main document:
%    \begin{macrocode}
\input{childdoc.def}
\childdocmain{}
%    \end{macrocode}

% Optional override for |\version| flag:
%    \begin{macrocode}
%%\ifchilddoc\else\providecommand{\version}{draft}\fi
%    \end{macrocode}

% Define the default values for the |\version| flag
% (|final| for the main file and |draft| for childs):
%    \begin{macrocode}
\ifchilddoc
\providecommand{\version}{draft}
\else
\providecommand{\version}{final}
\fi
%    \end{macrocode}

% Load the standard document class:
%    \begin{macrocode}
\documentclass[12pt]{article}
%    \end{macrocode}

% Start the document body:
%    \begin{macrocode}
\begin{document}
%    \end{macrocode}

% Declare a title page.
% Print title, part of document being processed and version flag:
%    \begin{macrocode}
\addtocounter{page}{-1}
\begin{center}
{\LARGE\bfseries{}childdoc example\par}
\vspace{1cm}
\ifchilddoc
\ifchilddocmanual part\else chapter\fi:
`\childdocname' of `\childdocjob'\par
\else
main document: `\childdocjob'\par
\fi
version: \version\par
\end{center}
\newpage
%    \end{macrocode}

% Manually include selected file,
% otherwise process as usual:
%    \begin{macrocode}
\ifchilddocmanual
\section*{part `\childdocname'}
\input{\childdocname}
\else
%    \end{macrocode}

% Include the two chapters:
%    \begin{macrocode}
\include{cdocsch1}
\include{cdocsch2}
%    \end{macrocode}

% Include the two parts unless only chapters should be displayed:
%    \begin{macrocode}
\ifchilddoc\else
\section{part three}
\input{cdocspt3}
\section{part four}
\input{cdocspt4}
\fi
%    \end{macrocode}

% Process as usual until here:
%    \begin{macrocode}
\fi
%    \end{macrocode}

% End of document body:
%    \begin{macrocode}
\end{document}
%    \end{macrocode}
%\iffalse
%</samplemain>
%\fi
%
% %%%%%%%%%%%%%%%%%%%%%%%%%%%%%%%%%%%%%%
% \paragraph{Chapter Include Files.}
%
% The include files are called |cdocsch1.tex| and |cdocsch2.tex|.
%
%\iffalse
%<*samplechap1|samplechap2>
%\fi

% Optional override for |\version| flag:
%    \begin{macrocode}
%%\providecommand{\version}{final}
%    \end{macrocode}

% Include the main document:
%    \begin{macrocode}
\input{childdoc.def}
\childdocof{cdocsamp}
%    \end{macrocode}

%\iffalse
%</samplechap1|samplechap2>
%\fi
%
%\iffalse
%<*samplechap1>
%\fi
% Some text for chapter 1:
%    \begin{macrocode}
\section{one}
some text in chapter one
%    \end{macrocode}

%\iffalse
%</samplechap1>
%\fi
% Some text for chapter 2:
%\iffalse
%<*samplechap2>
%\fi
%    \begin{macrocode}
\section{two}
more text in chapter two
%    \end{macrocode}

%\iffalse
%</samplechap2>
%\fi
%
% %%%%%%%%%%%%%%%%%%%%%%%%%%%%%%%%%%%%%%
% \paragraph{Part Include Files.}
%
% The include files are called |cdocspt3.tex| and |cdocspt4.tex|.
%
%\iffalse
%<*samplepart3|samplepart4>
%\fi

% Optional override for |\version| flag:
%    \begin{macrocode}
%%\providecommand{\version}{final}
%    \end{macrocode}

% Include the main document:
%    \begin{macrocode}
\input{childdoc.def}
\childdocby{cdocsamp}
%    \end{macrocode}

%\iffalse
%</samplepart3|samplepart4>
%\fi
%
%\iffalse
%<*samplepart3>
%\fi
% Some text for part 3:
%    \begin{macrocode}
some text in part three
%    \end{macrocode}

%\iffalse
%</samplepart3>
%\fi
% Some text for part 4:
%\iffalse
%<*samplepart4>
%\fi
%    \begin{macrocode}
more text in part four
%    \end{macrocode}

%\iffalse
%</samplepart4>
%\fi
%
% %%%%%%%%%%%%%%%%%%%%%%%%%%%%%%%%%%%%%%
% \paragraph{Forwarding for a Complete Draft.}
%
% The following forwarding file |cdocsdrf.tex|
% compiles the main document in draft mode:
%\iffalse
%<*sampledraft>
%\fi
%    \begin{macrocode}
\def\version{draft}
\input{childdoc.def}
\childdocforward{cdocsamp}
%    \end{macrocode}

%\iffalse
%</sampledraft>
%\fi
%
% %%%%%%%%%%%%%%%%%%%%%%%%%%%%%%%%%%%%%%
% \paragraph{Forwarding for Final Version of the Chapters.}
%
% The following forwarding files |cdocsfn1.tex| and |cdocsfn2.tex|
% (with identical content)
% compile the final versions of the child documents
% |cdocsch1.tex| and |cdocsch2.tex|, respectively:
%\iffalse
%<*samplefinal>
%\fi
%    \begin{macrocode}
\def\version{final}
\input{childdoc.def}
\childdocforwardprefix[cdocsamp]{cdocsfn}{cdocsch}
%    \end{macrocode}

%\iffalse
%</samplefinal>
%\fi
%
% %%%%%%%%%%%%%%%%%%%%%%%%%%%%%%%%%%%%%%
% \paragraph{Command Line Processing.}
%
% The following three command lines generate the output files
% |cdocscld|, |cdocscl1| and |cdocscl2|
% which should be identical to
% |cdocsdrf|, |cdocsch1| and |cdocsfn2|, respectively:
% \begin{center}
% \begin{tabular}{l}
% |latex -jobname cdocscld \|\\
% |  "\def\version{draft}\input{childdoc.def}\childdocforward{cdocsamp}"|\\
% |latex -jobname cdocscl1 \|\\
% |  "\input{childdoc.def}\childdocforward[cdocsamp]{cdocsch1}"|\\
% |latex -jobname cdocscl2 \|\\
% |  "\def\version{final}\input{childdoc.def}\childdocforward{cdocsch2}"|
% \end{tabular}
% \end{center}
% Note that the trailing backslash on each first line
% merely continues the input to the second line
% (for convenient cut ant paste).
% Furthermore, the command |latex| can be replaced by any
% of its alternative versions such as |pdflatex|.
%
% %%%%%%%%%%%%%%%%%%%%%%%%%%%%%%%%%%%%%%%%%%%%%%%%%%%%%%%%%%%%%%%%%%%%%%%%%%%%%%
% %%%%%%%%%%%%%%%%%%%%%%%%%%%%%%%%%%%%%%%%%%%%%%%%%%%%%%%%%%%%%%%%%%%%%%%%%%%%%%
% \section{Implementation}
%\iffalse
%<*package>
%\fi
%
% This section describes the definitions file |childdoc.def|.

% The definitions cannot be loaded using |\usepackage| or |\RequirePackage|
% which has a mechanism to prevent loading a style file more than once.
% When loading the definitions by means of |\input|
% multiple instances have to be prevented manually:
%\iffalse
%This code needs to be before the `\ProvidesFile' directive
%which is defined at the beginning of this file.
%Therefore it is also placed there and commented out here.
%</package>
%<*discard>
%\fi
%    \begin{macrocode}
\ifdefined\childdocmain\endinput\fi
%    \end{macrocode}
%\iffalse
%</discard>
%<*package>
%\fi
%
% \macro{\ifchilddoc}
% \macro{\ifchilddocmanual}
% The conditional |\ifchilddoc| tells whether a
% child (true) or main (false) document is being compiled.
% The conditional |\ifchilddocmanual| tells whether
% the |\includeonly| mechanism is used (false) or
% the selection of child files must be performed manually (true).
% The definitions initialise to false:
%    \begin{macrocode}
\newif\ifchilddoc
\newif\ifchilddocmanual
%    \end{macrocode}

% \macro{\childdocname}
% \macro{\childdocjob}
% The macro |\childdocname| stores the name of the main document
% to be compiled. The macro |\childdocjob| stores the name of
% the document on which the \LaTeX{} compiler was originally invoked.
% The content of |\jobname| cannot be compared
% to filenames specified in the source due to different catcodes.
% The following code rescans |\jobname|, stores the result
% in |\childdocname| and saves a copy in |\childdocjob|:
%    \begin{macrocode}
\edef\childdocname{\scantokens\expandafter{\jobname\noexpand}}
\let\childdocjob\childdocname
%    \end{macrocode}

% \macro{\childdocdisable}
% The macro |\childdocdisable| prevents the main file
% from being processed more than once.
% At this stage, the main document command |\childdocmain|
% is assumed to be called once again where it should do nothing.
% Any subsequent call to it should prevent
% a secondary processing of the main document
% It overwrites the forwarding commands
% |\childdocof| and |\childdocforward|
% with empty macros to prevent further inclusions of the main document:
%    \begin{macrocode}
\newcommand{\childdocdisable}
{
  \renewcommand{\childdocmain}[1]{\renewcommand{\childdocmain}[1]{\endinput}}
  \renewcommand{\childdocof}[1]{}
  \renewcommand{\childdocby}[2][]{}
  \renewcommand{\childdocforward}[2][]{}
  \renewcommand{\childdocdisable}{}
}
%    \end{macrocode}

% \macro{\childdocmain}
% The macro |\childdocmain| is to be called at the top of the main file
% with nothing or the main filename (without extension) as argument.
% First, it breaks loops.
% If the argument is not empty and does not match |\childdocname|
% (which is set by the first inclusion of |childdoc.def|),
% |\ifchilddoc| is set to true, |\includeonly| is applied to the child file
% and |\jobname| is set to the main file
% (for proper handling of |.aux| files):
%    \begin{macrocode}
\newcommand{\childdocmain}[1]
{
  \childdocdisable\childdocmain{}
  \if?#1?\else
    \begingroup
      \def\childdoctmp{#1}
      \ifx\childdoctmp\childdocname
        \def\childdoctmp{}
      \else
        \def\childdoctmp
        {
          \childdoctrue
          \includeonly{\childdocname}
          \def\childdocjob{#1}
          \def\jobname{#1}
        }
      \fi
      \expandafter
    \endgroup
    \childdoctmp
  \fi
}
%    \end{macrocode}

% \macro{\childdocof}
% The command |\childdocof| redirects
% compilation to the main file |#1|.
%    \begin{macrocode}
\newcommand{\childdocof}[1]
{
  \childdocdisable
  \childdoctrue
  \includeonly{\childdocname}
  \def\jobname{#1}
  \def\childdocjob{#1}
  \input{#1}
}
%    \end{macrocode}

% \macro{\childdocby}
% The command |\childdocby| ....
%    \begin{macrocode}
\newcommand{\childdocby}[2][]
{
  \childdocdisable
  \childdoctrue
  \childdocmanualtrue
  \if?#1?\else
    \def\jobname{#2}
  \fi
  \def\childdocjob{#2}
  \input{#2}
  \endinput
}
%    \end{macrocode}

% \macro{\childdocforward}
% The command |\childdocforward| redirects
% compilation to the main file or
% (if the optional argument is given) a child file.
% Parameters are set as if the main file
% or a child file starting with |\childdocof| was compiled.
% Then compilation is handed over to the main file:
%    \begin{macrocode}
\newcommand{\childdocforward}[2][]
{
  \begingroup
    \if?#1?
      \def\childdoctmp
      {
        \def\childdocname{#2}
        \def\childdocjob{#2}
        \def\jobname{#2}
        \input{#2}
        \endinput
      }
    \else
      \def\childdoctmp
      {
        \childdocdisable
        \def\childdocname{#2}
        \childdoctrue
        \includeonly{#2}
        \def\childdocjob{#1}
        \def\jobname{#1}
        \input{#1}
        \endinput
      }
    \fi
    \expandafter
  \endgroup
  \childdoctmp
}
%    \end{macrocode}

% \macro{\childdocforwardprefix}
% The command |\childdocforwardprefix| redirects
% compilation to the main or a child file by means of a pattern.
% The prefix |#1| in the current filename is replaced by |#2|
% and the suffix of the current filename is kept
% (it is assumed that the filename does not contain the substring `|~~~|'
% which is used as a delimiter).
% Compilation is handed over to the new file by |\childdocforward|:
%    \begin{macrocode}
\newcommand{\childdocforwardprefix}[3][]
{
  \begingroup
    \def\childdocextract #2##1~~~{\def\childdoctmp{\childdocforward[#1]{#3##1}}}
    \expandafter\childdocextract\childdocname~~~
    \expandafter
  \endgroup
  \childdoctmp
}
%    \end{macrocode}

% \macro{\childdoc}
% The deprecated macro |\childdoc| is a legacy version of |\childdocmain|:
%    \begin{macrocode}
\newcommand{\childdoc}{\childdocmain}
%    \end{macrocode}

% \macro{\childdocredirect}
% The deprecated macro |\childdocredirect| is a legacy version
% of |\childdocforward| and |\childdocforwardprefix|:
%    \begin{macrocode}
\newcommand{\childdocredirect}[2][]
{
  \begingroup
    \if?#1?
      \def\childdoctmp{\childdocforward{#2}}
    \else
      \def\childdoctmp{\childdocforwardprefix{#1}{#2}}
    \fi
    \expandafter
  \endgroup
  \childdoctmp
}
%    \end{macrocode}

%\iffalse
%</package>
%\fi
%
\endinput
|\\
|\childdocby{|\textit{main}|}|\\
\end{tabular}
\end{center}
%
The directive |\childdocby| is similar to |\childdocof|
described in \secref{sec:include},
but the subsequent selection of content must be done manually.
To that end, both |\ifchilddoc| and |\ifchilddocmanual|
will be true upon processing of a part,
and the name of the part is stored in |\childdocname|.
Note that |\jobname| will be set to the filename of the current part
so that each part receives an individual |.aux| file
that does not interfere with the |.aux| file(s) of the main document.
This behaviour can be altered by the alternative form
|\childdocby[*]{|\textit{main}|}| (with a non-empty optional argument)
which uses the |.aux| file of the main document
by setting |\jobname| to \textit{main}.

%%%%%%%%%%%%%%%%%%%%%%%%%%%%%%%%%%%%%%%%%%%%%%%%%%%%%%%%%%%%%%%%%%%%%%%%%%%%%%%%
\subsection{Driver Development}
\label{sec:driver}

The \textsf{childdoc} mechanism can also be use for the development
of definition files such as \LaTeX{} styles or classes.
This case differs from the above setup with multiple parts
included by |\include| in that no |\includeonly| should be invoked.
This can be achieved by starting the include file
(before |\ProvidesPackage|) with:
%
\begin{center}
\begin{tabular}{l}
|% \iffalse
%
% childdoc.dtx Copyright (C) 2017-2018 Niklas Beisert
%
% This work may be distributed and/or modified under the
% conditions of the LaTeX Project Public License, either version 1.3
% of this license or (at your option) any later version.
% The latest version of this license is in
%   http://www.latex-project.org/lppl.txt
% and version 1.3 or later is part of all distributions of LaTeX
% version 2005/12/01 or later.
%
% This work has the LPPL maintenance status `maintained'.
%
% The Current Maintainer of this work is Niklas Beisert.
%
% This work consists of the files childdoc.dtx and childdoc.ins
% and the derived files childdoc.def and cdocsamp.tex with
% cdocsch1.tex, cdocsch2.tex, cdocsdrf.tex, cdocsfn1.tex, cdocsfn2.tex.
%
%<package>\ifdefined\childdocmain\endinput\fi
%<package>\ProvidesFile{childdoc.def}[2018/12/30 v2.0 child document driver]
%<samplemain>\ProvidesFile{cdocsamp.tex}[2018/12/30 v2.0 sample for childdoc]
%<*driver>
%\ProvidesFile{childdoc.drv}[2018/12/30 v2.0 childdoc reference manual file]
\PassOptionsToClass{10pt,a4paper}{article}
\documentclass{ltxdoc}

\usepackage[margin=35mm]{geometry}
\usepackage{hyperref}
\usepackage{hyperxmp}
\usepackage[usenames]{color}

\hypersetup{colorlinks=true}
\hypersetup{pdfstartview=FitH}
\hypersetup{pdfpagemode=UseNone}
\hypersetup{pdfsource={}}
\hypersetup{pdflang={en-UK}}
\hypersetup{pdfcopyright={Copyright 2017-2018 Niklas Beisert.
  This work may be distributed and/or modified under the
  conditions of the LaTeX Project Public License, either version 1.3
  of this license or (at your option) any later version.}}
\hypersetup{pdflicenseurl={http://www.latex-project.org/lppl.txt}}
\hypersetup{pdfcontactaddress={ETH Zurich, ITP, HIT K,
  Wolfgang-Pauli-Strasse 27}}
\hypersetup{pdfcontactpostcode={8093}}
\hypersetup{pdfcontactcity={Zurich}}
\hypersetup{pdfcontactcountry={Switzerland}}
\hypersetup{pdfcontactemail={nbeisert@itp.phys.ethz.ch}}
\hypersetup{pdfcontacturl={http://people.phys.ethz.ch/\xmptilde nbeisert/}}

\newcommand{\secref}[1]{\hyperref[#1]{section \ref*{#1}}}

\parskip1ex
\parindent0pt
\let\olditemize\itemize
\def\itemize{\olditemize\parskip0pt}

\begin{document}

\title{The \textsf{childdoc} Package}
\hypersetup{pdftitle={The childdoc Package}}
\author{Niklas Beisert\\[2ex]
  Institut f\"ur Theoretische Physik\\
  Eidgen\"ossische Technische Hochschule Z\"urich\\
  Wolfgang-Pauli-Strasse 27, 8093 Z\"urich, Switzerland\\[1ex]
  \href{mailto:nbeisert@itp.phys.ethz.ch}
  {\texttt{nbeisert@itp.phys.ethz.ch}}}
\hypersetup{pdfauthor={Niklas Beisert}}
\hypersetup{pdfsubject={Manual for the LaTeX2e Package childdoc}}
\date{30 December 2018, \textsf{v2.0}}
\maketitle

\begin{abstract}\noindent
\textsf{childdoc} is a \LaTeXe{} package
that enables the direct compilation
of document sections included by |\include|
to individual files.
\end{abstract}

\begingroup
\parskip0ex
\tableofcontents
\endgroup

%%%%%%%%%%%%%%%%%%%%%%%%%%%%%%%%%%%%%%%%%%%%%%%%%%%%%%%%%%%%%%%%%%%%%%%%%%%%%%%%
%%%%%%%%%%%%%%%%%%%%%%%%%%%%%%%%%%%%%%%%%%%%%%%%%%%%%%%%%%%%%%%%%%%%%%%%%%%%%%%%
\section{Introduction}

\LaTeX{} provides a mechanism to structure a large document (such as a book)
into a main file and several child files (containing the chapters)
using the |\include| command.
This mechanism is beneficial for documents
which span hundreds of pages in order to
make the source file(s) more manageable.
Moreover, compilation can be restricted to
selected child files by means of the |\includeonly| command.
The latter feature can be used to reduce the compilation time while editing
(this was significantly more useful in the earlier days of \LaTeX{})
or to generate a smaller document which is easier to navigate.
Another application of |\includeonly| is to generate
documents consisting of selected parts of the complete document.

However, there are a few drawbacks of the plain |\include| mechanism:
\begin{itemize}
\item
The child files cannot be compiled on their own,
they can only be compiled via the main file.
A naive editing environment
(such as a text editor with an option
to have the current file processed by \LaTeX)
may require one to switch to the main file before compiling;
attempting to compile the child file produces errors.
\item
The main file must be modified (each time)
to adjust the |\includeonly| command
to the present needs. This easily leaves the main file in a messy state.
\item
The generated document will always carry the filename
of the main document. This is inconvenient if
several child files are to be compiled and
to be kept for distribution.
\end{itemize}

The present package provides a simple interface
to make child files individually compilable by \LaTeX{}.
Compiling a child file then has the same effect as compiling
the main file with an |\includeonly| command
to select the appropriate child.
Moreover the generated document will carry the name of the child
rather than the main file.
This resolves all three above issues.

This feature is meant to make the editing of books,
thesis documents and lecture notes somewhat more convenient.
However, the package can also be used efficiently for
composing a series of documents (such as exercise sheets)
which are typically distributed individually.
It then assists the author in generating the individual documents
(potentially in different versions)
as well as a document containing the collected series.
Another application is in developing style files
or other kinds of included material
where compilation of the style file could redirect
to a sample or test file.

%%%%%%%%%%%%%%%%%%%%%%%%%%%%%%%%%%%%%%%%%%%%%%%%%%%%%%%%%%%%%%%%%%%%%%%%%%%%%%%%
%%%%%%%%%%%%%%%%%%%%%%%%%%%%%%%%%%%%%%%%%%%%%%%%%%%%%%%%%%%%%%%%%%%%%%%%%%%%%%%%
\section{Usage}

First of all, the package \textsf{childdoc} is \emph{not} a standard
\LaTeXe{} |.sty| style file! Therefore it needs to be invoked in
a non-standard way.

%%%%%%%%%%%%%%%%%%%%%%%%%%%%%%%%%%%%%%%%%%%%%%%%%%%%%%%%%%%%%%%%%%%%%%%%%%%%%%%%
\subsection{Included Files}
\label{sec:include}

%%%%%%%%%%%%%%%%%%%%%%%%%%%%%%%%%%%%%%%%
\DescribeMacro{\childdocmain}
To use the package, add the commands
\begin{center}
\begin{tabular}{l}
|\input{childdoc.def}|\\
|\childdocmain{}|\\
\end{tabular}
\end{center}
at the very top of the main \LaTeX{} file,
in particular \emph{before} the |\documentclass| statement!
The argument of |\childdocmain| should be left empty
(but it must be present).

%%%%%%%%%%%%%%%%%%%%%%%%%%%%%%%%%%%%%%%%
\DescribeMacro{\childdocof}
Furthermore, add the commands
\begin{center}
\begin{tabular}{l}
|\input{childdoc.def}|\\
|\childdocof{|\textit{main}|}|\\
\end{tabular}
\end{center}
at the top of every child file \textit{child}
which is included by |\include{|\textit{child}|}|
from within the main file
(or at least for those files to be compiled individually).
The argument \textit{main} must be the filename of the main file.

There are a couple of
considerations in setting up the main and child documents:

%%%%%%%%%%%%%%%%%%%%%%%%%%%%%%%%%%%%%%%%
\paragraph{Restrictions.}

Please note the following restrictions:
\begin{itemize}
\item
|\childdocmain| must be called with one argument \textit{main}
to ensure compatibility with earlier version of the package.
It must either be empty (|\childdocmain{}|)
or precisely match the filename of the main file in which it is specified.
See \secref{sec:detection} for further information.
\item
The filename \textit{main} must be specified without the |.tex| extension.
\item
The filename \textit{main} is case sensitive
(even in case-insensitive file systems)
due to internal string comparison.
\item
The argument \textit{main} should be fully expanded, it cannot be a macro.
\item
Subdirectories and special characters should be avoided in filenames.
\item
The command |\childdocmain{|\textit{main}|}| must be followed by a whitespace.
It should not be followed immediately by another command
or by a comment mark `|%|'.
This is because the \TeX{} parser reads the token immediately following
the argument of |\childdocmain| and puts it
at the beginning of every child section;
however, a white\-space is ignored.
\end{itemize}

%%%%%%%%%%%%%%%%%%%%%%%%%%%%%%%%%%%%%%%%
\paragraph{Content of Main File.}

It is advisable to place all content in the child files included by |\include|.
Any output contained in the main file will appear in all child documents
unless suppressed manually;
it cannot be suppressed automatically by the |\includeonly| directive
and thus should normally be avoided.
A method to include some content in the main file
by means of conditional processing is described in \secref{sec:conditional}.

%%%%%%%%%%%%%%%%%%%%%%%%%%%%%%%%%%%%%%%%
\paragraph{Page Numbering.}

When only a part of the document is compiled,
the appropriate numbering of pages
(as well as other status parameters)
is determined from the |.aux| files.
The latter contain information from previous passes.
However this information needs to propagate through
all intermediate child documents.
Therefore the page numbering in child documents may well
be inconsistent until the complete document is compiled at least once.

A useful (if unconventional) way to always ensure a consistent
page numbering is to restart the numbering in each child document
and denote the pages by `\textit{child}|.|\textit{page}'
where \textit{child} represents the chapter/section number of the child file.
This can be achieved by the command
|\numberwithin{page}{|\textit{child}|}|
of the \textsf{amsmath} package
where \textit{child} can be |chapter| or |section|
depending on the chosen structuring.
Alternatively, one can modify the macro |\thepage| appropriately
and reset the counter |page| at the start of each child file.

%%%%%%%%%%%%%%%%%%%%%%%%%%%%%%%%%%%%%%%%%%%%%%%%%%%%%%%%%%%%%%%%%%%%%%%%%%%%%%%%
\subsection{Conditional Processing}
\label{sec:conditional}

The package provides a mechanism to compile different versions
of a document. To customise the versions further some conditional processing
can come in handy to distinguish which version is being compiled.
The package provides two macros to describe the compilation context:

%%%%%%%%%%%%%%%%%%%%%%%%%%%%%%%%%%%%%%%%
\DescribeMacro{\ifchilddoc}
The conditional |\ifchilddoc| distinguishes between the compilation of
child documents and the main document:
%
\begin{center}
|\ifchilddoc |\textit{child-code}| |[|\||else |\textit{main-code}]| \||fi|
\end{center}

%%%%%%%%%%%%%%%%%%%%%%%%%%%%%%%%%%%%%%%%
\DescribeMacro{\childdocname}
\DescribeMacro{\childdocjob}
The macro |\childdocname| contains the filename (without extension)
of the main or child file being processed.
Note that |\childdocjob| will always contain the name of the main file.

%%%%%%%%%%%%%%%%%%%%%%%%%%%%%%%%%%%%%%%%
\paragraph{Title Page.}

Conditional processing can be used to include a title or banner page
in the main document when proper precautions are taken.
Importantly, the code in the main file should ensure that the page counter
(as well as other status parameters which are stored in the |.aux| files)
takes the same value after the conditional processing.
Otherwise the page numbers may take divergent values
depending on which part is compiled.

For example, a title page could be declared by:
%
\begin{center}
\begin{tabular}{l}
|\ifchilddoc\||else|\\
|\addtocounter{page}{-1}|\\
\textit{code for title page}\\
|\newpage|\\
|\||fi|
\end{tabular}
\end{center}
%
A banner page for the child documents can be generated by:
%
\begin{center}
\begin{tabular}{l}
|\ifchilddoc|\\
|\addtocounter{page}{-1}|\\
\textit{code for banner page}\\
|\newpage|\\
|\||fi|
\end{tabular}
\end{center}
%
Here one could write a message such as:
\begin{center}
|This is the part \childdocname{} of \childdocjob{}.|
\end{center}

%%%%%%%%%%%%%%%%%%%%%%%%%%%%%%%%%%%%%%%%%%%%%%%%%%%%%%%%%%%%%%%%%%%%%%%%%%%%%%%%
\subsection{Flags}
\label{sec:flags}

The package makes it easy to generate different versions
of the main or child documents.
To this end compilation flags can be defined
and assigned different default values.
They will be particularly useful in conjunction
with the forwarding mechanism described in \secref{sec:forward}.

For example, it may be useful to have a flag |\version|
which can be set to |draft| or |final|.
The document source will contain some conditional code
depending on the value of |\version|.
Suppose further, the flag should default to |final| for the main file
and to |draft| for child files
which is a natural assignment for editing the document.
This is achieved by placing the following code
in the preamble of the main document
(below the |\childdocmain| directive):
%
\begin{center}
\begin{tabular}{l}
|\ifchilddoc|\\
|\providecommand{\version}{draft}|\\
|\||else|\\
|\providecommand{\version}{final}|\\
|\||fi|
\end{tabular}
\end{center}
%
The definition by |\providecommand| makes sure
that previous definitions are not overwritten.
Further statements |\providecommand{\version}{...}|
can thus be added before the above code to override it.

For the main file, one might add a line
(between |\childdocmain| and the above block)
%
\begin{center}
|%\ifchilddoc\||else\providecommand{\version}{draft}\||fi|
\end{center}
%
which can be uncommented to produce a draft version.
Likewise one can add a line to the very top of a child file
(above the |\childdocof{|\textit{main}|}| directive)
%
\begin{center}
|%\providecommand{\version}{final}|
\end{center}
%
which can be uncommented to produce the final version of this child document.

%%%%%%%%%%%%%%%%%%%%%%%%%%%%%%%%%%%%%%%%%%%%%%%%%%%%%%%%%%%%%%%%%%%%%%%%%%%%%%%%
\subsection{Forwarding}
\label{sec:forward}

Different versions of the main or child documents
using compilation flags as described in \secref{sec:flags}
can be (permanently) stored in different files
for convenient compilation, viewing and distribution.
To this end, the package defines a command
to pass on compilation to a different file:

%%%%%%%%%%%%%%%%%%%%%%%%%%%%%%%%%%%%%%%%
\DescribeMacro{\childdocforward}
The command |\childdocforward| redirects processing to
another source file:
%
\begin{center}
\begin{tabular}{l}
|\input{childdoc.def}|\\
|\childdocforward[|\textit{main}|]{|\textit{dest}|}|\\
\end{tabular}
\end{center}
%
The argument \textit{dest} is the destination file
(without extension).
It should be the main file or one of the child files.
Note that further \textsf{childdoc} directives
such as |\childdocof| and |\childdocforward|
in the indicated file will be processed in this form.
The optional argument \textit{main}
passes on directly to the main file \textit{main}
while pretending to compile the child \textit{dest}.
This form behaves as if \textit{dest}
issues |\childdocof{|\textit{main}|}| right away,
and no further \textsf{childdoc} directives will be processed.

%%%%%%%%%%%%%%%%%%%%%%%%%%%%%%%%%%%%%%%%
\DescribeMacro{\...prefix}
In the alternative form |\childdocforwardprefix|,
%
\begin{center}
\begin{tabular}{l}
|\input{childdoc.def}|\\
|\childdocforwardprefix[|\textit{main}|]{|\textit{prefix}|}{|\textit{dest}|}|
\end{tabular}
\end{center}
%
the destination file is determined by a pattern
depending on the current file:
To make this work, the current file must be called
`{\textit{prefix}\hspace{0.2em}\textit{suffix}}'
with \textit{prefix} matching precisely the argument.
Processing is then passed on to the file
`{\textit{dest}\hspace{0.2em}\textit{suffix}}'.
Surely, the same effect is achieved by
directly specifying the
argument `{\textit{dest}\hspace{0.2em}\textit{suffix}}'
in the first form.
However, that requires to set up a different file
for each child. With the alternative form of the command
all these files can have exactly the same content
which simplifies setting them up and maintaining them.

For example, the following file |draft.tex|
with a compilation flag |\version| as described in \secref{sec:flags}
compiles the main document as a draft:
%
\begin{center}
\begin{tabular}{l}
|\def\version{draft}|\\
|\input{childdoc.def}|\\
|\childdocforward{|\textit{main}|}|
\end{tabular}
\end{center}
%
Likewise, the following files |final|\textit{nn}|.tex|
compile the final version of the child document
|child|\textit{nn}|.tex|:
%
\begin{center}
\begin{tabular}{l}
|\def\version{final}|\\
|\input{childdoc.def}|\\
|\childdocforwardprefix{final}{child}|
\end{tabular}
\end{center}
%

Note that when several versions of a main file and/or of each child file
are to be generated, it may be convenient to set up a |Makefile| or
shell script to automatise the process.

%%%%%%%%%%%%%%%%%%%%%%%%%%%%%%%%%%%%%%%%%%%%%%%%%%%%%%%%%%%%%%%%%%%%%%%%%%%%%%%%
\subsection{Command Line Processing}
\label{sec:commandline}

The effect of redirection files can also be achieved by invoking
the \LaTeX{} compiler with a more elaborate command line.
Most conveniently this should be done as part
of a shell script or a |Makefile|.

When using \textsf{childdoc} in the main file, the following
command lines effectively perform a redirection
(note that depending on the shell being used,
backslashes may have to be doubled: `|\|' $\to$ `|\\|'):
%
\begin{center}
|... -jobname "|\textit{target}|" |\\|"|[\textit{flags}]%
|\input{childdoc.def}\childdocforward[|\textit{main}|]{|\textit{dest}|}"|
\end{center}
%
Here \textit{target} is the name of the output file,
\textit{main} is the name of the main file
and \textit{dest} is the name of the main or child file to be processed
(all filenames without extensions).
The optional argument \textit{main} can be omitted
if \textit{main} matches \textit{dest}.
Optionally, compilation \textit{flags} can be defined via |\def| commands.
This command line makes the \TeX{} engine believe
it is compiling the file \textit{target}
whose content is specified as the latter parameter.
The provided code then forwards the processing to
\textit{main} or \textit{dest} as described in \secref{sec:forward}.

%%%%%%%%%%%%%%%%%%%%%%%%%%%%%%%%%%%%%%%%%%%%%%%%%%%%%%%%%%%%%%%%%%%%%%%%%%%%%%%%
\subsection{Include by Input}
\label{sec:input}

Including child documents by |\include| has some restrictions by design.
Most notably, the content of a child document always occupies
its own set of pages; pages cannot be shared between child documents.
Usually, this behaviour makes perfect sense
because each child document contain an essential part of the document.
However, in some situations it may be desirable to compose
a document from a collection of parts
without having mandatory page breaks between then.
For this case, the package
provides a mechanism to include parts
by |\input| which can also be processed individually.
However, by construction this mechanism
requires manual handling of the content to be output.

%%%%%%%%%%%%%%%%%%%%%%%%%%%%%%%%%%%%%%%%
\DescribeMacro{\ifchilddocmanual}
The main file should be prepared as usual, see \secref{sec:include}.
However, the document body must make a distinction
between processing of an individual part and of the main document, e.g.:
%
\begin{center}
\begin{tabular}{l}
|\ifchilddocmanual|\\
|\input{\childdocname}|\\
|\||else|\\
\textit{document body with }|\input{|\textit{part}|}|\\
|\||fi|
\end{tabular}
\end{center}
%
The conditional |\ifchilddocmanual| is true whenever
a part to be included by |\input| is being compiled,
and the name of the part is stored in |\childdocname|.

%%%%%%%%%%%%%%%%%%%%%%%%%%%%%%%%%%%%%%%%
\DescribeMacro{\childdocby}
Each part to be included by |\input| should start with:
%
\begin{center}
\begin{tabular}{l}
|\input{childdoc.def}|\\
|\childdocby{|\textit{main}|}|\\
\end{tabular}
\end{center}
%
The directive |\childdocby| is similar to |\childdocof|
described in \secref{sec:include},
but the subsequent selection of content must be done manually.
To that end, both |\ifchilddoc| and |\ifchilddocmanual|
will be true upon processing of a part,
and the name of the part is stored in |\childdocname|.
Note that |\jobname| will be set to the filename of the current part
so that each part receives an individual |.aux| file
that does not interfere with the |.aux| file(s) of the main document.
This behaviour can be altered by the alternative form
|\childdocby[*]{|\textit{main}|}| (with a non-empty optional argument)
which uses the |.aux| file of the main document
by setting |\jobname| to \textit{main}.

%%%%%%%%%%%%%%%%%%%%%%%%%%%%%%%%%%%%%%%%%%%%%%%%%%%%%%%%%%%%%%%%%%%%%%%%%%%%%%%%
\subsection{Driver Development}
\label{sec:driver}

The \textsf{childdoc} mechanism can also be use for the development
of definition files such as \LaTeX{} styles or classes.
This case differs from the above setup with multiple parts
included by |\include| in that no |\includeonly| should be invoked.
This can be achieved by starting the include file
(before |\ProvidesPackage|) with:
%
\begin{center}
\begin{tabular}{l}
|\input{childdoc.def}|\\
|\childdocforward{|\textit{main}|}|\\
\end{tabular}
\end{center}
%
or alternatively with:
%
\begin{center}
\begin{tabular}{l}
|\input{childdoc.def}|\\
|\childdocby{|\textit{main}|}|\\
\end{tabular}
\end{center}
%
Both forms have slightly different effects as described above.
The main file is prepared as usual, see \secref{sec:include}.

%%%%%%%%%%%%%%%%%%%%%%%%%%%%%%%%%%%%%%%%%%%%%%%%%%%%%%%%%%%%%%%%%%%%%%%%%%%%%%%%
\subsection{Legacy Detection}
\label{sec:detection}

The directive |\childdocmain| in the main file can detect
whether the complete document or merely a child is to be compiled
even without using the directive |\childdocof|.
This method is deprecated because it is less robust
and there is no compelling reason to use it;
it is merely provided for backward compatibility
and it may be removed in future versions.

If the detection mechanism is to be used,
it is mandatory to correctly specify
the filename of the main file as the argument of |\childdocmain|:
%
\begin{center}
\begin{tabular}{l}
|\input{childdoc.def}|\\
|\childdocmain{|\textit{main}|}|\\
\end{tabular}
\end{center}
%
If |\jobname| does not match the argument \textit{main} of |\childdocmain|,
it is assumed that |\jobname| points to the child file to be compiled.
When using |\childdocmain| with the main file specified as argument,
it suffices to start a child file
with just |\input{|\textit{main}|}|
without loading of the package and using |\childdocof|.
If instead all processing is done
with the appropriate \textsf{childdoc} directives,
the argument of \textit{main} of |\childdocmain| can be empty.

An alternative version of the command line processing described
in \secref{sec:commandline} using the detection mechanism reads:
%
\begin{center}
|... -jobname "|\textit{target}|" "|[\textit{flags}]%
[|\def\jobname{|\textit{dest}|}|]|\input{|\textit{main}|}"|
\end{center}

%%%%%%%%%%%%%%%%%%%%%%%%%%%%%%%%%%%%%%%%%%%%%%%%%%%%%%%%%%%%%%%%%%%%%%%%%%%%%%%%
\subsection{Manual Code}
\label{sec:manual}

In case one cannot be certain whether the definitions file |childdoc.def|
is installed on the target \TeX{} distribution
and one prefers not to ship it,
it is conceivable to paste a few relevant commands into the sources.

To that end, drop all statements |\input{childdoc.def}|
and perform the replacements as outlined below.
Instead of |\childdocmain{|\textit{main}|}| add the following code
to the top of the main file:
%
\begin{center}
\begin{tabular}{l}
|\||ifdefined\childdocname\endinput\||fi\newif\ifchilddoc|\\
|\edef\childdocname{\scantokens\expandafter{\jobname\noexpand}}|\\
|\def\childdocmain{|\textit{main}|}\||ifx\childdocmain\childdocname\||else|\\
|\childdoctrue\includeonly{\childdocname}\let\jobname\childdocmain\||fi|\\
\end{tabular}
\end{center}
%
Instead of |\childdocof{|\textit{main}|}| just include the main file
at the top of each child file:
%
\begin{center}
|\input{|\textit{main}|}|
\end{center}
%
A simple redirection |\childdocforward{|\textit{dest}|}| is achieved by:
%
\begin{center}
|\def\jobname{|\textit{dest}|}\input{\jobname}|
\end{center}
%
The redirection with prefix
|\childdocforwardprefix[|\textit{prefix}|]{|\textit{dest}|}|
is accomplished by:
%
\begin{center}
\begin{tabular}{l}
|{\edef\jobname{\scantokens\expandafter{\jobname\noexpand}}|\\
|\def\redirectjob |\textit{prefix}|#1~~~{\gdef\jobname{|\textit{dest}|#1}}|\\
|\expandafter\redirectjob\jobname~~~}\input{\jobname}|
\end{tabular}
\end{center}

In an alternative approach,
child documents can be compiled by a specific command line
without additional code or specific definitions:
%
\begin{center}
|... -jobname "|\textit{target}|" "|[\textit{flags}]%
|\includeonly{|\textit{dest}|}\input{|\textit{main}|}"|
\end{center}
%

%%%%%%%%%%%%%%%%%%%%%%%%%%%%%%%%%%%%%%%%%%%%%%%%%%%%%%%%%%%%%%%%%%%%%%%%%%%%%%%%
%%%%%%%%%%%%%%%%%%%%%%%%%%%%%%%%%%%%%%%%%%%%%%%%%%%%%%%%%%%%%%%%%%%%%%%%%%%%%%%%
\section{Information}

%%%%%%%%%%%%%%%%%%%%%%%%%%%%%%%%%%%%%%%%%%%%%%%%%%%%%%%%%%%%%%%%%%%%%%%%%%%%%%%%
\subsection{Copyright}

Copyright \copyright{} 2017--2018 Niklas Beisert

This work may be distributed and/or modified under the
conditions of the \LaTeX{} Project Public License, either version 1.3
of this license or (at your option) any later version.
The latest version of this license is in
  \url{http://www.latex-project.org/lppl.txt}
and version 1.3 or later is part of all distributions of \LaTeX{}
version 2005/12/01 or later.

This work has the LPPL maintenance status `maintained'.

The Current Maintainer of this work is Niklas Beisert.

This work consists of the files |README.txt|, |childdoc.ins| and |childdoc.dtx|
as well as the derived files |childdoc.def|, |cdocsamp.tex|
with |cdocsch1.tex|, |cdocsch2.tex|, |cdocspt3.tex|, |cdocspt4.tex|,
|cdocsdrf.tex|, |cdocsfn1.tex|, |cdocsfn2.tex|
as well as |childdoc.pdf|.

%%%%%%%%%%%%%%%%%%%%%%%%%%%%%%%%%%%%%%%%%%%%%%%%%%%%%%%%%%%%%%%%%%%%%%%%%%%%%%%%
\subsection{Files and Installation}

The package consists of the files:
%
\begin{center}
\begin{tabular}{ll}
    |README.txt|   & readme file \\
    |childdoc.ins| & installation file \\
    |childdoc.dtx| & source file \\
    |childdoc.def| & definition file \\
    |cdocsamp.tex| & sample main file \\
    |cdocsch1.tex| & sample include file \\
    |cdocsch2.tex| & sample include file \\
    |cdocspt3.tex| & sample part file \\
    |cdocspt4.tex| & sample part file \\
    |cdocsdrf.tex| & sample redirection file \\
    |cdocsfn1.tex| & sample redirection file \\
    |cdocsfn2.tex| & sample redirection file \\
    |childdoc.pdf| & manual
\end{tabular}
\end{center}
%
The distribution consists of the files
|README.txt|, |childdoc.ins| and |childdoc.dtx|.
%
\begin{itemize}
\item
Run (pdf)\LaTeX{} on |childdoc.dtx|
to compile the manual |childdoc.pdf| (this file).
\item
Run \LaTeX{} on |childdoc.ins| to create the definitions file |childdoc.def|
and the sample |cdocsamp.tex| with include files
|cdocsch1.tex|, |cdocsch2.tex|, |cdocspt3.tex|, |cdocspt4.tex|,
|cdocsdrf.tex|, |cdocsfn1.tex|, |cdocsfn2.tex|.
Then copy the file |childdoc.def| to an appropriate directory of your \LaTeX{}
distribution, e.g.\ \textit{texmf-root}|/tex/latex/childdoc|.
\end{itemize}

%%%%%%%%%%%%%%%%%%%%%%%%%%%%%%%%%%%%%%%%%%%%%%%%%%%%%%%%%%%%%%%%%%%%%%%%%%%%%%%%
\subsection{Related CTAN Packages}

There are several other packages which offer a similar functionality:
%
\begin{itemize}
\item
The packages
\href{http://ctan.org/pkg/docmute}{\textsf{docmute}},
\href{http://ctan.org/pkg/includex}{\textsf{includex}} and
\href{http://ctan.org/pkg/standalone}{\textsf{standalone}}
provide commands to include only the document body of
a child file thus allowing both files to be compiled individually.
\item
The packages \href{http://ctan.org/pkg/subdocs}{\textsf{subdocs}}
and \href{http://ctan.org/pkg/subfiles}{\textsf{subfiles}}
provide structures in which the main and child documents can be
encapsulated and allowing them to be compiled individually.
The inclusion mechanism is different from the conventional |\include|.
\item
The package \href{http://ctan.org/pkg/combine}{\textsf{combine}}
is an elaborate solution to combine several documents into one.
\end{itemize}
%
See also the CTAN topic \href{http://ctan.org/topic/subdocs}{\textsf{subdocs}}
for further related packages.
The present package differs from the above solutions in that
a document structure constructed with the conventional |\include| mechanism
just needs two extra commands at the top of every file
such that all constituent files can be compiled individually.

%%%%%%%%%%%%%%%%%%%%%%%%%%%%%%%%%%%%%%%%%%%%%%%%%%%%%%%%%%%%%%%%%%%%%%%%%%%%%%%%
%\subsection{Feature Suggestions}
%
%The following is a list of features which may be useful for future
%versions of this package:
%%
%\begin{itemize}
%\item
%\ldots
%\end{itemize}

%%%%%%%%%%%%%%%%%%%%%%%%%%%%%%%%%%%%%%%%%%%%%%%%%%%%%%%%%%%%%%%%%%%%%%%%%%%%%%%%
\subsection{Revision History}

%%%%%%%%%%%%%%%%%%%%%%%%%%%%%%%%%%%%%%%%
\paragraph{v2.0:} 2018/12/30

\begin{itemize}
\item
immediate forward processing
\item
added |\childdocby| mechanism
\item
manual restructured
\end{itemize}

%%%%%%%%%%%%%%%%%%%%%%%%%%%%%%%%%%%%%%%%
\paragraph{v1.6:} 2018/01/17

\begin{itemize}
\item
application for development of include files
\item
corrections to manual
\end{itemize}

%%%%%%%%%%%%%%%%%%%%%%%%%%%%%%%%%%%%%%%%
\paragraph{v1.5:} 2017/05/21

\begin{itemize}
\item
more complete structuring introduced
\item
|\childdocof| introduced
\item
|\childdoc| renamed to |\childdocmain|
\item
|\childredirect| renamed to |\childdocforward| and |\childdocforwardprefix|
and functionality expanded
\end{itemize}

%%%%%%%%%%%%%%%%%%%%%%%%%%%%%%%%%%%%%%%%
\paragraph{v1.0:} 2017/04/27

\begin{itemize}
\item
manual and install package
\item
first version published on CTAN
\end{itemize}

%%%%%%%%%%%%%%%%%%%%%%%%%%%%%%%%%%%%%%%%
\paragraph{v0.6:} 2017/04/26

\begin{itemize}
\item
redirection mechanism added
\end{itemize}

%%%%%%%%%%%%%%%%%%%%%%%%%%%%%%%%%%%%%%%%
\paragraph{v0.5:} 2017/04/26

\begin{itemize}
\item
functionality in definition file
\end{itemize}


%%%%%%%%%%%%%%%%%%%%%%%%%%%%%%%%%%%%%%%%%%%%%%%%%%%%%%%%%%%%%%%%%%%%%%%%%%%%%%%%
%%%%%%%%%%%%%%%%%%%%%%%%%%%%%%%%%%%%%%%%%%%%%%%%%%%%%%%%%%%%%%%%%%%%%%%%%%%%%%%%
%%%%%%%%%%%%%%%%%%%%%%%%%%%%%%%%%%%%%%%%%%%%%%%%%%%%%%%%%%%%%%%%%%%%%%%%%%%%%%%%
\appendix

\settowidth\MacroIndent{\rmfamily\scriptsize 000\ }

 \DocInput{childdoc.dtx}

\end{document}
%</driver>
% \fi
%
% %%%%%%%%%%%%%%%%%%%%%%%%%%%%%%%%%%%%%%%%%%%%%%%%%%%%%%%%%%%%%%%%%%%%%%%%%%%%%%
% %%%%%%%%%%%%%%%%%%%%%%%%%%%%%%%%%%%%%%%%%%%%%%%%%%%%%%%%%%%%%%%%%%%%%%%%%%%%%%
% \section{Sample}
%\iffalse
%<*samplemain>
%\fi
%
% The following presents a sample document
% with two chapters, two parts, a title page,
% a compile flag as well as three forwarding files to set the flag.
% It consists of eight |.tex| files:
% \begin{center}
% \begin{tabular}{ll}
% |cdocsamp.tex|&main file\\
% |cdocsch1.tex|&include file for chapter 1\\
% |cdocsch2.tex|&include file for chapter 2\\
% |cdocspt3.tex|&include file for part 3\\
% |cdocspt4.tex|&include file for part 4\\
% |cdocsdrf.tex|&forwarding file for main file in draft mode\\
% |cdocsfi1.tex|&forwarding file for final version of chapter 1\\
% |cdocsfi2.tex|&forwarding file for final version of chapter 2\\
% \end{tabular}
% \end{center}
% Each of the eight files can be compiled directly by the \LaTeX{} compiler.
%
% %%%%%%%%%%%%%%%%%%%%%%%%%%%%%%%%%%%%%%
% \paragraph{Main File.}
%
% The main file is called |cdocsamp.tex|.
%
% Load the \textsf{childdoc} definitions and
% declare the filename for the main document:
%    \begin{macrocode}
\input{childdoc.def}
\childdocmain{}
%    \end{macrocode}

% Optional override for |\version| flag:
%    \begin{macrocode}
%%\ifchilddoc\else\providecommand{\version}{draft}\fi
%    \end{macrocode}

% Define the default values for the |\version| flag
% (|final| for the main file and |draft| for childs):
%    \begin{macrocode}
\ifchilddoc
\providecommand{\version}{draft}
\else
\providecommand{\version}{final}
\fi
%    \end{macrocode}

% Load the standard document class:
%    \begin{macrocode}
\documentclass[12pt]{article}
%    \end{macrocode}

% Start the document body:
%    \begin{macrocode}
\begin{document}
%    \end{macrocode}

% Declare a title page.
% Print title, part of document being processed and version flag:
%    \begin{macrocode}
\addtocounter{page}{-1}
\begin{center}
{\LARGE\bfseries{}childdoc example\par}
\vspace{1cm}
\ifchilddoc
\ifchilddocmanual part\else chapter\fi:
`\childdocname' of `\childdocjob'\par
\else
main document: `\childdocjob'\par
\fi
version: \version\par
\end{center}
\newpage
%    \end{macrocode}

% Manually include selected file,
% otherwise process as usual:
%    \begin{macrocode}
\ifchilddocmanual
\section*{part `\childdocname'}
\input{\childdocname}
\else
%    \end{macrocode}

% Include the two chapters:
%    \begin{macrocode}
\include{cdocsch1}
\include{cdocsch2}
%    \end{macrocode}

% Include the two parts unless only chapters should be displayed:
%    \begin{macrocode}
\ifchilddoc\else
\section{part three}
\input{cdocspt3}
\section{part four}
\input{cdocspt4}
\fi
%    \end{macrocode}

% Process as usual until here:
%    \begin{macrocode}
\fi
%    \end{macrocode}

% End of document body:
%    \begin{macrocode}
\end{document}
%    \end{macrocode}
%\iffalse
%</samplemain>
%\fi
%
% %%%%%%%%%%%%%%%%%%%%%%%%%%%%%%%%%%%%%%
% \paragraph{Chapter Include Files.}
%
% The include files are called |cdocsch1.tex| and |cdocsch2.tex|.
%
%\iffalse
%<*samplechap1|samplechap2>
%\fi

% Optional override for |\version| flag:
%    \begin{macrocode}
%%\providecommand{\version}{final}
%    \end{macrocode}

% Include the main document:
%    \begin{macrocode}
\input{childdoc.def}
\childdocof{cdocsamp}
%    \end{macrocode}

%\iffalse
%</samplechap1|samplechap2>
%\fi
%
%\iffalse
%<*samplechap1>
%\fi
% Some text for chapter 1:
%    \begin{macrocode}
\section{one}
some text in chapter one
%    \end{macrocode}

%\iffalse
%</samplechap1>
%\fi
% Some text for chapter 2:
%\iffalse
%<*samplechap2>
%\fi
%    \begin{macrocode}
\section{two}
more text in chapter two
%    \end{macrocode}

%\iffalse
%</samplechap2>
%\fi
%
% %%%%%%%%%%%%%%%%%%%%%%%%%%%%%%%%%%%%%%
% \paragraph{Part Include Files.}
%
% The include files are called |cdocspt3.tex| and |cdocspt4.tex|.
%
%\iffalse
%<*samplepart3|samplepart4>
%\fi

% Optional override for |\version| flag:
%    \begin{macrocode}
%%\providecommand{\version}{final}
%    \end{macrocode}

% Include the main document:
%    \begin{macrocode}
\input{childdoc.def}
\childdocby{cdocsamp}
%    \end{macrocode}

%\iffalse
%</samplepart3|samplepart4>
%\fi
%
%\iffalse
%<*samplepart3>
%\fi
% Some text for part 3:
%    \begin{macrocode}
some text in part three
%    \end{macrocode}

%\iffalse
%</samplepart3>
%\fi
% Some text for part 4:
%\iffalse
%<*samplepart4>
%\fi
%    \begin{macrocode}
more text in part four
%    \end{macrocode}

%\iffalse
%</samplepart4>
%\fi
%
% %%%%%%%%%%%%%%%%%%%%%%%%%%%%%%%%%%%%%%
% \paragraph{Forwarding for a Complete Draft.}
%
% The following forwarding file |cdocsdrf.tex|
% compiles the main document in draft mode:
%\iffalse
%<*sampledraft>
%\fi
%    \begin{macrocode}
\def\version{draft}
\input{childdoc.def}
\childdocforward{cdocsamp}
%    \end{macrocode}

%\iffalse
%</sampledraft>
%\fi
%
% %%%%%%%%%%%%%%%%%%%%%%%%%%%%%%%%%%%%%%
% \paragraph{Forwarding for Final Version of the Chapters.}
%
% The following forwarding files |cdocsfn1.tex| and |cdocsfn2.tex|
% (with identical content)
% compile the final versions of the child documents
% |cdocsch1.tex| and |cdocsch2.tex|, respectively:
%\iffalse
%<*samplefinal>
%\fi
%    \begin{macrocode}
\def\version{final}
\input{childdoc.def}
\childdocforwardprefix[cdocsamp]{cdocsfn}{cdocsch}
%    \end{macrocode}

%\iffalse
%</samplefinal>
%\fi
%
% %%%%%%%%%%%%%%%%%%%%%%%%%%%%%%%%%%%%%%
% \paragraph{Command Line Processing.}
%
% The following three command lines generate the output files
% |cdocscld|, |cdocscl1| and |cdocscl2|
% which should be identical to
% |cdocsdrf|, |cdocsch1| and |cdocsfn2|, respectively:
% \begin{center}
% \begin{tabular}{l}
% |latex -jobname cdocscld \|\\
% |  "\def\version{draft}\input{childdoc.def}\childdocforward{cdocsamp}"|\\
% |latex -jobname cdocscl1 \|\\
% |  "\input{childdoc.def}\childdocforward[cdocsamp]{cdocsch1}"|\\
% |latex -jobname cdocscl2 \|\\
% |  "\def\version{final}\input{childdoc.def}\childdocforward{cdocsch2}"|
% \end{tabular}
% \end{center}
% Note that the trailing backslash on each first line
% merely continues the input to the second line
% (for convenient cut ant paste).
% Furthermore, the command |latex| can be replaced by any
% of its alternative versions such as |pdflatex|.
%
% %%%%%%%%%%%%%%%%%%%%%%%%%%%%%%%%%%%%%%%%%%%%%%%%%%%%%%%%%%%%%%%%%%%%%%%%%%%%%%
% %%%%%%%%%%%%%%%%%%%%%%%%%%%%%%%%%%%%%%%%%%%%%%%%%%%%%%%%%%%%%%%%%%%%%%%%%%%%%%
% \section{Implementation}
%\iffalse
%<*package>
%\fi
%
% This section describes the definitions file |childdoc.def|.

% The definitions cannot be loaded using |\usepackage| or |\RequirePackage|
% which has a mechanism to prevent loading a style file more than once.
% When loading the definitions by means of |\input|
% multiple instances have to be prevented manually:
%\iffalse
%This code needs to be before the `\ProvidesFile' directive
%which is defined at the beginning of this file.
%Therefore it is also placed there and commented out here.
%</package>
%<*discard>
%\fi
%    \begin{macrocode}
\ifdefined\childdocmain\endinput\fi
%    \end{macrocode}
%\iffalse
%</discard>
%<*package>
%\fi
%
% \macro{\ifchilddoc}
% \macro{\ifchilddocmanual}
% The conditional |\ifchilddoc| tells whether a
% child (true) or main (false) document is being compiled.
% The conditional |\ifchilddocmanual| tells whether
% the |\includeonly| mechanism is used (false) or
% the selection of child files must be performed manually (true).
% The definitions initialise to false:
%    \begin{macrocode}
\newif\ifchilddoc
\newif\ifchilddocmanual
%    \end{macrocode}

% \macro{\childdocname}
% \macro{\childdocjob}
% The macro |\childdocname| stores the name of the main document
% to be compiled. The macro |\childdocjob| stores the name of
% the document on which the \LaTeX{} compiler was originally invoked.
% The content of |\jobname| cannot be compared
% to filenames specified in the source due to different catcodes.
% The following code rescans |\jobname|, stores the result
% in |\childdocname| and saves a copy in |\childdocjob|:
%    \begin{macrocode}
\edef\childdocname{\scantokens\expandafter{\jobname\noexpand}}
\let\childdocjob\childdocname
%    \end{macrocode}

% \macro{\childdocdisable}
% The macro |\childdocdisable| prevents the main file
% from being processed more than once.
% At this stage, the main document command |\childdocmain|
% is assumed to be called once again where it should do nothing.
% Any subsequent call to it should prevent
% a secondary processing of the main document
% It overwrites the forwarding commands
% |\childdocof| and |\childdocforward|
% with empty macros to prevent further inclusions of the main document:
%    \begin{macrocode}
\newcommand{\childdocdisable}
{
  \renewcommand{\childdocmain}[1]{\renewcommand{\childdocmain}[1]{\endinput}}
  \renewcommand{\childdocof}[1]{}
  \renewcommand{\childdocby}[2][]{}
  \renewcommand{\childdocforward}[2][]{}
  \renewcommand{\childdocdisable}{}
}
%    \end{macrocode}

% \macro{\childdocmain}
% The macro |\childdocmain| is to be called at the top of the main file
% with nothing or the main filename (without extension) as argument.
% First, it breaks loops.
% If the argument is not empty and does not match |\childdocname|
% (which is set by the first inclusion of |childdoc.def|),
% |\ifchilddoc| is set to true, |\includeonly| is applied to the child file
% and |\jobname| is set to the main file
% (for proper handling of |.aux| files):
%    \begin{macrocode}
\newcommand{\childdocmain}[1]
{
  \childdocdisable\childdocmain{}
  \if?#1?\else
    \begingroup
      \def\childdoctmp{#1}
      \ifx\childdoctmp\childdocname
        \def\childdoctmp{}
      \else
        \def\childdoctmp
        {
          \childdoctrue
          \includeonly{\childdocname}
          \def\childdocjob{#1}
          \def\jobname{#1}
        }
      \fi
      \expandafter
    \endgroup
    \childdoctmp
  \fi
}
%    \end{macrocode}

% \macro{\childdocof}
% The command |\childdocof| redirects
% compilation to the main file |#1|.
%    \begin{macrocode}
\newcommand{\childdocof}[1]
{
  \childdocdisable
  \childdoctrue
  \includeonly{\childdocname}
  \def\jobname{#1}
  \def\childdocjob{#1}
  \input{#1}
}
%    \end{macrocode}

% \macro{\childdocby}
% The command |\childdocby| ....
%    \begin{macrocode}
\newcommand{\childdocby}[2][]
{
  \childdocdisable
  \childdoctrue
  \childdocmanualtrue
  \if?#1?\else
    \def\jobname{#2}
  \fi
  \def\childdocjob{#2}
  \input{#2}
  \endinput
}
%    \end{macrocode}

% \macro{\childdocforward}
% The command |\childdocforward| redirects
% compilation to the main file or
% (if the optional argument is given) a child file.
% Parameters are set as if the main file
% or a child file starting with |\childdocof| was compiled.
% Then compilation is handed over to the main file:
%    \begin{macrocode}
\newcommand{\childdocforward}[2][]
{
  \begingroup
    \if?#1?
      \def\childdoctmp
      {
        \def\childdocname{#2}
        \def\childdocjob{#2}
        \def\jobname{#2}
        \input{#2}
        \endinput
      }
    \else
      \def\childdoctmp
      {
        \childdocdisable
        \def\childdocname{#2}
        \childdoctrue
        \includeonly{#2}
        \def\childdocjob{#1}
        \def\jobname{#1}
        \input{#1}
        \endinput
      }
    \fi
    \expandafter
  \endgroup
  \childdoctmp
}
%    \end{macrocode}

% \macro{\childdocforwardprefix}
% The command |\childdocforwardprefix| redirects
% compilation to the main or a child file by means of a pattern.
% The prefix |#1| in the current filename is replaced by |#2|
% and the suffix of the current filename is kept
% (it is assumed that the filename does not contain the substring `|~~~|'
% which is used as a delimiter).
% Compilation is handed over to the new file by |\childdocforward|:
%    \begin{macrocode}
\newcommand{\childdocforwardprefix}[3][]
{
  \begingroup
    \def\childdocextract #2##1~~~{\def\childdoctmp{\childdocforward[#1]{#3##1}}}
    \expandafter\childdocextract\childdocname~~~
    \expandafter
  \endgroup
  \childdoctmp
}
%    \end{macrocode}

% \macro{\childdoc}
% The deprecated macro |\childdoc| is a legacy version of |\childdocmain|:
%    \begin{macrocode}
\newcommand{\childdoc}{\childdocmain}
%    \end{macrocode}

% \macro{\childdocredirect}
% The deprecated macro |\childdocredirect| is a legacy version
% of |\childdocforward| and |\childdocforwardprefix|:
%    \begin{macrocode}
\newcommand{\childdocredirect}[2][]
{
  \begingroup
    \if?#1?
      \def\childdoctmp{\childdocforward{#2}}
    \else
      \def\childdoctmp{\childdocforwardprefix{#1}{#2}}
    \fi
    \expandafter
  \endgroup
  \childdoctmp
}
%    \end{macrocode}

%\iffalse
%</package>
%\fi
%
\endinput
|\\
|\childdocforward{|\textit{main}|}|\\
\end{tabular}
\end{center}
%
or alternatively with:
%
\begin{center}
\begin{tabular}{l}
|% \iffalse
%
% childdoc.dtx Copyright (C) 2017-2018 Niklas Beisert
%
% This work may be distributed and/or modified under the
% conditions of the LaTeX Project Public License, either version 1.3
% of this license or (at your option) any later version.
% The latest version of this license is in
%   http://www.latex-project.org/lppl.txt
% and version 1.3 or later is part of all distributions of LaTeX
% version 2005/12/01 or later.
%
% This work has the LPPL maintenance status `maintained'.
%
% The Current Maintainer of this work is Niklas Beisert.
%
% This work consists of the files childdoc.dtx and childdoc.ins
% and the derived files childdoc.def and cdocsamp.tex with
% cdocsch1.tex, cdocsch2.tex, cdocsdrf.tex, cdocsfn1.tex, cdocsfn2.tex.
%
%<package>\ifdefined\childdocmain\endinput\fi
%<package>\ProvidesFile{childdoc.def}[2018/12/30 v2.0 child document driver]
%<samplemain>\ProvidesFile{cdocsamp.tex}[2018/12/30 v2.0 sample for childdoc]
%<*driver>
%\ProvidesFile{childdoc.drv}[2018/12/30 v2.0 childdoc reference manual file]
\PassOptionsToClass{10pt,a4paper}{article}
\documentclass{ltxdoc}

\usepackage[margin=35mm]{geometry}
\usepackage{hyperref}
\usepackage{hyperxmp}
\usepackage[usenames]{color}

\hypersetup{colorlinks=true}
\hypersetup{pdfstartview=FitH}
\hypersetup{pdfpagemode=UseNone}
\hypersetup{pdfsource={}}
\hypersetup{pdflang={en-UK}}
\hypersetup{pdfcopyright={Copyright 2017-2018 Niklas Beisert.
  This work may be distributed and/or modified under the
  conditions of the LaTeX Project Public License, either version 1.3
  of this license or (at your option) any later version.}}
\hypersetup{pdflicenseurl={http://www.latex-project.org/lppl.txt}}
\hypersetup{pdfcontactaddress={ETH Zurich, ITP, HIT K,
  Wolfgang-Pauli-Strasse 27}}
\hypersetup{pdfcontactpostcode={8093}}
\hypersetup{pdfcontactcity={Zurich}}
\hypersetup{pdfcontactcountry={Switzerland}}
\hypersetup{pdfcontactemail={nbeisert@itp.phys.ethz.ch}}
\hypersetup{pdfcontacturl={http://people.phys.ethz.ch/\xmptilde nbeisert/}}

\newcommand{\secref}[1]{\hyperref[#1]{section \ref*{#1}}}

\parskip1ex
\parindent0pt
\let\olditemize\itemize
\def\itemize{\olditemize\parskip0pt}

\begin{document}

\title{The \textsf{childdoc} Package}
\hypersetup{pdftitle={The childdoc Package}}
\author{Niklas Beisert\\[2ex]
  Institut f\"ur Theoretische Physik\\
  Eidgen\"ossische Technische Hochschule Z\"urich\\
  Wolfgang-Pauli-Strasse 27, 8093 Z\"urich, Switzerland\\[1ex]
  \href{mailto:nbeisert@itp.phys.ethz.ch}
  {\texttt{nbeisert@itp.phys.ethz.ch}}}
\hypersetup{pdfauthor={Niklas Beisert}}
\hypersetup{pdfsubject={Manual for the LaTeX2e Package childdoc}}
\date{30 December 2018, \textsf{v2.0}}
\maketitle

\begin{abstract}\noindent
\textsf{childdoc} is a \LaTeXe{} package
that enables the direct compilation
of document sections included by |\include|
to individual files.
\end{abstract}

\begingroup
\parskip0ex
\tableofcontents
\endgroup

%%%%%%%%%%%%%%%%%%%%%%%%%%%%%%%%%%%%%%%%%%%%%%%%%%%%%%%%%%%%%%%%%%%%%%%%%%%%%%%%
%%%%%%%%%%%%%%%%%%%%%%%%%%%%%%%%%%%%%%%%%%%%%%%%%%%%%%%%%%%%%%%%%%%%%%%%%%%%%%%%
\section{Introduction}

\LaTeX{} provides a mechanism to structure a large document (such as a book)
into a main file and several child files (containing the chapters)
using the |\include| command.
This mechanism is beneficial for documents
which span hundreds of pages in order to
make the source file(s) more manageable.
Moreover, compilation can be restricted to
selected child files by means of the |\includeonly| command.
The latter feature can be used to reduce the compilation time while editing
(this was significantly more useful in the earlier days of \LaTeX{})
or to generate a smaller document which is easier to navigate.
Another application of |\includeonly| is to generate
documents consisting of selected parts of the complete document.

However, there are a few drawbacks of the plain |\include| mechanism:
\begin{itemize}
\item
The child files cannot be compiled on their own,
they can only be compiled via the main file.
A naive editing environment
(such as a text editor with an option
to have the current file processed by \LaTeX)
may require one to switch to the main file before compiling;
attempting to compile the child file produces errors.
\item
The main file must be modified (each time)
to adjust the |\includeonly| command
to the present needs. This easily leaves the main file in a messy state.
\item
The generated document will always carry the filename
of the main document. This is inconvenient if
several child files are to be compiled and
to be kept for distribution.
\end{itemize}

The present package provides a simple interface
to make child files individually compilable by \LaTeX{}.
Compiling a child file then has the same effect as compiling
the main file with an |\includeonly| command
to select the appropriate child.
Moreover the generated document will carry the name of the child
rather than the main file.
This resolves all three above issues.

This feature is meant to make the editing of books,
thesis documents and lecture notes somewhat more convenient.
However, the package can also be used efficiently for
composing a series of documents (such as exercise sheets)
which are typically distributed individually.
It then assists the author in generating the individual documents
(potentially in different versions)
as well as a document containing the collected series.
Another application is in developing style files
or other kinds of included material
where compilation of the style file could redirect
to a sample or test file.

%%%%%%%%%%%%%%%%%%%%%%%%%%%%%%%%%%%%%%%%%%%%%%%%%%%%%%%%%%%%%%%%%%%%%%%%%%%%%%%%
%%%%%%%%%%%%%%%%%%%%%%%%%%%%%%%%%%%%%%%%%%%%%%%%%%%%%%%%%%%%%%%%%%%%%%%%%%%%%%%%
\section{Usage}

First of all, the package \textsf{childdoc} is \emph{not} a standard
\LaTeXe{} |.sty| style file! Therefore it needs to be invoked in
a non-standard way.

%%%%%%%%%%%%%%%%%%%%%%%%%%%%%%%%%%%%%%%%%%%%%%%%%%%%%%%%%%%%%%%%%%%%%%%%%%%%%%%%
\subsection{Included Files}
\label{sec:include}

%%%%%%%%%%%%%%%%%%%%%%%%%%%%%%%%%%%%%%%%
\DescribeMacro{\childdocmain}
To use the package, add the commands
\begin{center}
\begin{tabular}{l}
|\input{childdoc.def}|\\
|\childdocmain{}|\\
\end{tabular}
\end{center}
at the very top of the main \LaTeX{} file,
in particular \emph{before} the |\documentclass| statement!
The argument of |\childdocmain| should be left empty
(but it must be present).

%%%%%%%%%%%%%%%%%%%%%%%%%%%%%%%%%%%%%%%%
\DescribeMacro{\childdocof}
Furthermore, add the commands
\begin{center}
\begin{tabular}{l}
|\input{childdoc.def}|\\
|\childdocof{|\textit{main}|}|\\
\end{tabular}
\end{center}
at the top of every child file \textit{child}
which is included by |\include{|\textit{child}|}|
from within the main file
(or at least for those files to be compiled individually).
The argument \textit{main} must be the filename of the main file.

There are a couple of
considerations in setting up the main and child documents:

%%%%%%%%%%%%%%%%%%%%%%%%%%%%%%%%%%%%%%%%
\paragraph{Restrictions.}

Please note the following restrictions:
\begin{itemize}
\item
|\childdocmain| must be called with one argument \textit{main}
to ensure compatibility with earlier version of the package.
It must either be empty (|\childdocmain{}|)
or precisely match the filename of the main file in which it is specified.
See \secref{sec:detection} for further information.
\item
The filename \textit{main} must be specified without the |.tex| extension.
\item
The filename \textit{main} is case sensitive
(even in case-insensitive file systems)
due to internal string comparison.
\item
The argument \textit{main} should be fully expanded, it cannot be a macro.
\item
Subdirectories and special characters should be avoided in filenames.
\item
The command |\childdocmain{|\textit{main}|}| must be followed by a whitespace.
It should not be followed immediately by another command
or by a comment mark `|%|'.
This is because the \TeX{} parser reads the token immediately following
the argument of |\childdocmain| and puts it
at the beginning of every child section;
however, a white\-space is ignored.
\end{itemize}

%%%%%%%%%%%%%%%%%%%%%%%%%%%%%%%%%%%%%%%%
\paragraph{Content of Main File.}

It is advisable to place all content in the child files included by |\include|.
Any output contained in the main file will appear in all child documents
unless suppressed manually;
it cannot be suppressed automatically by the |\includeonly| directive
and thus should normally be avoided.
A method to include some content in the main file
by means of conditional processing is described in \secref{sec:conditional}.

%%%%%%%%%%%%%%%%%%%%%%%%%%%%%%%%%%%%%%%%
\paragraph{Page Numbering.}

When only a part of the document is compiled,
the appropriate numbering of pages
(as well as other status parameters)
is determined from the |.aux| files.
The latter contain information from previous passes.
However this information needs to propagate through
all intermediate child documents.
Therefore the page numbering in child documents may well
be inconsistent until the complete document is compiled at least once.

A useful (if unconventional) way to always ensure a consistent
page numbering is to restart the numbering in each child document
and denote the pages by `\textit{child}|.|\textit{page}'
where \textit{child} represents the chapter/section number of the child file.
This can be achieved by the command
|\numberwithin{page}{|\textit{child}|}|
of the \textsf{amsmath} package
where \textit{child} can be |chapter| or |section|
depending on the chosen structuring.
Alternatively, one can modify the macro |\thepage| appropriately
and reset the counter |page| at the start of each child file.

%%%%%%%%%%%%%%%%%%%%%%%%%%%%%%%%%%%%%%%%%%%%%%%%%%%%%%%%%%%%%%%%%%%%%%%%%%%%%%%%
\subsection{Conditional Processing}
\label{sec:conditional}

The package provides a mechanism to compile different versions
of a document. To customise the versions further some conditional processing
can come in handy to distinguish which version is being compiled.
The package provides two macros to describe the compilation context:

%%%%%%%%%%%%%%%%%%%%%%%%%%%%%%%%%%%%%%%%
\DescribeMacro{\ifchilddoc}
The conditional |\ifchilddoc| distinguishes between the compilation of
child documents and the main document:
%
\begin{center}
|\ifchilddoc |\textit{child-code}| |[|\||else |\textit{main-code}]| \||fi|
\end{center}

%%%%%%%%%%%%%%%%%%%%%%%%%%%%%%%%%%%%%%%%
\DescribeMacro{\childdocname}
\DescribeMacro{\childdocjob}
The macro |\childdocname| contains the filename (without extension)
of the main or child file being processed.
Note that |\childdocjob| will always contain the name of the main file.

%%%%%%%%%%%%%%%%%%%%%%%%%%%%%%%%%%%%%%%%
\paragraph{Title Page.}

Conditional processing can be used to include a title or banner page
in the main document when proper precautions are taken.
Importantly, the code in the main file should ensure that the page counter
(as well as other status parameters which are stored in the |.aux| files)
takes the same value after the conditional processing.
Otherwise the page numbers may take divergent values
depending on which part is compiled.

For example, a title page could be declared by:
%
\begin{center}
\begin{tabular}{l}
|\ifchilddoc\||else|\\
|\addtocounter{page}{-1}|\\
\textit{code for title page}\\
|\newpage|\\
|\||fi|
\end{tabular}
\end{center}
%
A banner page for the child documents can be generated by:
%
\begin{center}
\begin{tabular}{l}
|\ifchilddoc|\\
|\addtocounter{page}{-1}|\\
\textit{code for banner page}\\
|\newpage|\\
|\||fi|
\end{tabular}
\end{center}
%
Here one could write a message such as:
\begin{center}
|This is the part \childdocname{} of \childdocjob{}.|
\end{center}

%%%%%%%%%%%%%%%%%%%%%%%%%%%%%%%%%%%%%%%%%%%%%%%%%%%%%%%%%%%%%%%%%%%%%%%%%%%%%%%%
\subsection{Flags}
\label{sec:flags}

The package makes it easy to generate different versions
of the main or child documents.
To this end compilation flags can be defined
and assigned different default values.
They will be particularly useful in conjunction
with the forwarding mechanism described in \secref{sec:forward}.

For example, it may be useful to have a flag |\version|
which can be set to |draft| or |final|.
The document source will contain some conditional code
depending on the value of |\version|.
Suppose further, the flag should default to |final| for the main file
and to |draft| for child files
which is a natural assignment for editing the document.
This is achieved by placing the following code
in the preamble of the main document
(below the |\childdocmain| directive):
%
\begin{center}
\begin{tabular}{l}
|\ifchilddoc|\\
|\providecommand{\version}{draft}|\\
|\||else|\\
|\providecommand{\version}{final}|\\
|\||fi|
\end{tabular}
\end{center}
%
The definition by |\providecommand| makes sure
that previous definitions are not overwritten.
Further statements |\providecommand{\version}{...}|
can thus be added before the above code to override it.

For the main file, one might add a line
(between |\childdocmain| and the above block)
%
\begin{center}
|%\ifchilddoc\||else\providecommand{\version}{draft}\||fi|
\end{center}
%
which can be uncommented to produce a draft version.
Likewise one can add a line to the very top of a child file
(above the |\childdocof{|\textit{main}|}| directive)
%
\begin{center}
|%\providecommand{\version}{final}|
\end{center}
%
which can be uncommented to produce the final version of this child document.

%%%%%%%%%%%%%%%%%%%%%%%%%%%%%%%%%%%%%%%%%%%%%%%%%%%%%%%%%%%%%%%%%%%%%%%%%%%%%%%%
\subsection{Forwarding}
\label{sec:forward}

Different versions of the main or child documents
using compilation flags as described in \secref{sec:flags}
can be (permanently) stored in different files
for convenient compilation, viewing and distribution.
To this end, the package defines a command
to pass on compilation to a different file:

%%%%%%%%%%%%%%%%%%%%%%%%%%%%%%%%%%%%%%%%
\DescribeMacro{\childdocforward}
The command |\childdocforward| redirects processing to
another source file:
%
\begin{center}
\begin{tabular}{l}
|\input{childdoc.def}|\\
|\childdocforward[|\textit{main}|]{|\textit{dest}|}|\\
\end{tabular}
\end{center}
%
The argument \textit{dest} is the destination file
(without extension).
It should be the main file or one of the child files.
Note that further \textsf{childdoc} directives
such as |\childdocof| and |\childdocforward|
in the indicated file will be processed in this form.
The optional argument \textit{main}
passes on directly to the main file \textit{main}
while pretending to compile the child \textit{dest}.
This form behaves as if \textit{dest}
issues |\childdocof{|\textit{main}|}| right away,
and no further \textsf{childdoc} directives will be processed.

%%%%%%%%%%%%%%%%%%%%%%%%%%%%%%%%%%%%%%%%
\DescribeMacro{\...prefix}
In the alternative form |\childdocforwardprefix|,
%
\begin{center}
\begin{tabular}{l}
|\input{childdoc.def}|\\
|\childdocforwardprefix[|\textit{main}|]{|\textit{prefix}|}{|\textit{dest}|}|
\end{tabular}
\end{center}
%
the destination file is determined by a pattern
depending on the current file:
To make this work, the current file must be called
`{\textit{prefix}\hspace{0.2em}\textit{suffix}}'
with \textit{prefix} matching precisely the argument.
Processing is then passed on to the file
`{\textit{dest}\hspace{0.2em}\textit{suffix}}'.
Surely, the same effect is achieved by
directly specifying the
argument `{\textit{dest}\hspace{0.2em}\textit{suffix}}'
in the first form.
However, that requires to set up a different file
for each child. With the alternative form of the command
all these files can have exactly the same content
which simplifies setting them up and maintaining them.

For example, the following file |draft.tex|
with a compilation flag |\version| as described in \secref{sec:flags}
compiles the main document as a draft:
%
\begin{center}
\begin{tabular}{l}
|\def\version{draft}|\\
|\input{childdoc.def}|\\
|\childdocforward{|\textit{main}|}|
\end{tabular}
\end{center}
%
Likewise, the following files |final|\textit{nn}|.tex|
compile the final version of the child document
|child|\textit{nn}|.tex|:
%
\begin{center}
\begin{tabular}{l}
|\def\version{final}|\\
|\input{childdoc.def}|\\
|\childdocforwardprefix{final}{child}|
\end{tabular}
\end{center}
%

Note that when several versions of a main file and/or of each child file
are to be generated, it may be convenient to set up a |Makefile| or
shell script to automatise the process.

%%%%%%%%%%%%%%%%%%%%%%%%%%%%%%%%%%%%%%%%%%%%%%%%%%%%%%%%%%%%%%%%%%%%%%%%%%%%%%%%
\subsection{Command Line Processing}
\label{sec:commandline}

The effect of redirection files can also be achieved by invoking
the \LaTeX{} compiler with a more elaborate command line.
Most conveniently this should be done as part
of a shell script or a |Makefile|.

When using \textsf{childdoc} in the main file, the following
command lines effectively perform a redirection
(note that depending on the shell being used,
backslashes may have to be doubled: `|\|' $\to$ `|\\|'):
%
\begin{center}
|... -jobname "|\textit{target}|" |\\|"|[\textit{flags}]%
|\input{childdoc.def}\childdocforward[|\textit{main}|]{|\textit{dest}|}"|
\end{center}
%
Here \textit{target} is the name of the output file,
\textit{main} is the name of the main file
and \textit{dest} is the name of the main or child file to be processed
(all filenames without extensions).
The optional argument \textit{main} can be omitted
if \textit{main} matches \textit{dest}.
Optionally, compilation \textit{flags} can be defined via |\def| commands.
This command line makes the \TeX{} engine believe
it is compiling the file \textit{target}
whose content is specified as the latter parameter.
The provided code then forwards the processing to
\textit{main} or \textit{dest} as described in \secref{sec:forward}.

%%%%%%%%%%%%%%%%%%%%%%%%%%%%%%%%%%%%%%%%%%%%%%%%%%%%%%%%%%%%%%%%%%%%%%%%%%%%%%%%
\subsection{Include by Input}
\label{sec:input}

Including child documents by |\include| has some restrictions by design.
Most notably, the content of a child document always occupies
its own set of pages; pages cannot be shared between child documents.
Usually, this behaviour makes perfect sense
because each child document contain an essential part of the document.
However, in some situations it may be desirable to compose
a document from a collection of parts
without having mandatory page breaks between then.
For this case, the package
provides a mechanism to include parts
by |\input| which can also be processed individually.
However, by construction this mechanism
requires manual handling of the content to be output.

%%%%%%%%%%%%%%%%%%%%%%%%%%%%%%%%%%%%%%%%
\DescribeMacro{\ifchilddocmanual}
The main file should be prepared as usual, see \secref{sec:include}.
However, the document body must make a distinction
between processing of an individual part and of the main document, e.g.:
%
\begin{center}
\begin{tabular}{l}
|\ifchilddocmanual|\\
|\input{\childdocname}|\\
|\||else|\\
\textit{document body with }|\input{|\textit{part}|}|\\
|\||fi|
\end{tabular}
\end{center}
%
The conditional |\ifchilddocmanual| is true whenever
a part to be included by |\input| is being compiled,
and the name of the part is stored in |\childdocname|.

%%%%%%%%%%%%%%%%%%%%%%%%%%%%%%%%%%%%%%%%
\DescribeMacro{\childdocby}
Each part to be included by |\input| should start with:
%
\begin{center}
\begin{tabular}{l}
|\input{childdoc.def}|\\
|\childdocby{|\textit{main}|}|\\
\end{tabular}
\end{center}
%
The directive |\childdocby| is similar to |\childdocof|
described in \secref{sec:include},
but the subsequent selection of content must be done manually.
To that end, both |\ifchilddoc| and |\ifchilddocmanual|
will be true upon processing of a part,
and the name of the part is stored in |\childdocname|.
Note that |\jobname| will be set to the filename of the current part
so that each part receives an individual |.aux| file
that does not interfere with the |.aux| file(s) of the main document.
This behaviour can be altered by the alternative form
|\childdocby[*]{|\textit{main}|}| (with a non-empty optional argument)
which uses the |.aux| file of the main document
by setting |\jobname| to \textit{main}.

%%%%%%%%%%%%%%%%%%%%%%%%%%%%%%%%%%%%%%%%%%%%%%%%%%%%%%%%%%%%%%%%%%%%%%%%%%%%%%%%
\subsection{Driver Development}
\label{sec:driver}

The \textsf{childdoc} mechanism can also be use for the development
of definition files such as \LaTeX{} styles or classes.
This case differs from the above setup with multiple parts
included by |\include| in that no |\includeonly| should be invoked.
This can be achieved by starting the include file
(before |\ProvidesPackage|) with:
%
\begin{center}
\begin{tabular}{l}
|\input{childdoc.def}|\\
|\childdocforward{|\textit{main}|}|\\
\end{tabular}
\end{center}
%
or alternatively with:
%
\begin{center}
\begin{tabular}{l}
|\input{childdoc.def}|\\
|\childdocby{|\textit{main}|}|\\
\end{tabular}
\end{center}
%
Both forms have slightly different effects as described above.
The main file is prepared as usual, see \secref{sec:include}.

%%%%%%%%%%%%%%%%%%%%%%%%%%%%%%%%%%%%%%%%%%%%%%%%%%%%%%%%%%%%%%%%%%%%%%%%%%%%%%%%
\subsection{Legacy Detection}
\label{sec:detection}

The directive |\childdocmain| in the main file can detect
whether the complete document or merely a child is to be compiled
even without using the directive |\childdocof|.
This method is deprecated because it is less robust
and there is no compelling reason to use it;
it is merely provided for backward compatibility
and it may be removed in future versions.

If the detection mechanism is to be used,
it is mandatory to correctly specify
the filename of the main file as the argument of |\childdocmain|:
%
\begin{center}
\begin{tabular}{l}
|\input{childdoc.def}|\\
|\childdocmain{|\textit{main}|}|\\
\end{tabular}
\end{center}
%
If |\jobname| does not match the argument \textit{main} of |\childdocmain|,
it is assumed that |\jobname| points to the child file to be compiled.
When using |\childdocmain| with the main file specified as argument,
it suffices to start a child file
with just |\input{|\textit{main}|}|
without loading of the package and using |\childdocof|.
If instead all processing is done
with the appropriate \textsf{childdoc} directives,
the argument of \textit{main} of |\childdocmain| can be empty.

An alternative version of the command line processing described
in \secref{sec:commandline} using the detection mechanism reads:
%
\begin{center}
|... -jobname "|\textit{target}|" "|[\textit{flags}]%
[|\def\jobname{|\textit{dest}|}|]|\input{|\textit{main}|}"|
\end{center}

%%%%%%%%%%%%%%%%%%%%%%%%%%%%%%%%%%%%%%%%%%%%%%%%%%%%%%%%%%%%%%%%%%%%%%%%%%%%%%%%
\subsection{Manual Code}
\label{sec:manual}

In case one cannot be certain whether the definitions file |childdoc.def|
is installed on the target \TeX{} distribution
and one prefers not to ship it,
it is conceivable to paste a few relevant commands into the sources.

To that end, drop all statements |\input{childdoc.def}|
and perform the replacements as outlined below.
Instead of |\childdocmain{|\textit{main}|}| add the following code
to the top of the main file:
%
\begin{center}
\begin{tabular}{l}
|\||ifdefined\childdocname\endinput\||fi\newif\ifchilddoc|\\
|\edef\childdocname{\scantokens\expandafter{\jobname\noexpand}}|\\
|\def\childdocmain{|\textit{main}|}\||ifx\childdocmain\childdocname\||else|\\
|\childdoctrue\includeonly{\childdocname}\let\jobname\childdocmain\||fi|\\
\end{tabular}
\end{center}
%
Instead of |\childdocof{|\textit{main}|}| just include the main file
at the top of each child file:
%
\begin{center}
|\input{|\textit{main}|}|
\end{center}
%
A simple redirection |\childdocforward{|\textit{dest}|}| is achieved by:
%
\begin{center}
|\def\jobname{|\textit{dest}|}\input{\jobname}|
\end{center}
%
The redirection with prefix
|\childdocforwardprefix[|\textit{prefix}|]{|\textit{dest}|}|
is accomplished by:
%
\begin{center}
\begin{tabular}{l}
|{\edef\jobname{\scantokens\expandafter{\jobname\noexpand}}|\\
|\def\redirectjob |\textit{prefix}|#1~~~{\gdef\jobname{|\textit{dest}|#1}}|\\
|\expandafter\redirectjob\jobname~~~}\input{\jobname}|
\end{tabular}
\end{center}

In an alternative approach,
child documents can be compiled by a specific command line
without additional code or specific definitions:
%
\begin{center}
|... -jobname "|\textit{target}|" "|[\textit{flags}]%
|\includeonly{|\textit{dest}|}\input{|\textit{main}|}"|
\end{center}
%

%%%%%%%%%%%%%%%%%%%%%%%%%%%%%%%%%%%%%%%%%%%%%%%%%%%%%%%%%%%%%%%%%%%%%%%%%%%%%%%%
%%%%%%%%%%%%%%%%%%%%%%%%%%%%%%%%%%%%%%%%%%%%%%%%%%%%%%%%%%%%%%%%%%%%%%%%%%%%%%%%
\section{Information}

%%%%%%%%%%%%%%%%%%%%%%%%%%%%%%%%%%%%%%%%%%%%%%%%%%%%%%%%%%%%%%%%%%%%%%%%%%%%%%%%
\subsection{Copyright}

Copyright \copyright{} 2017--2018 Niklas Beisert

This work may be distributed and/or modified under the
conditions of the \LaTeX{} Project Public License, either version 1.3
of this license or (at your option) any later version.
The latest version of this license is in
  \url{http://www.latex-project.org/lppl.txt}
and version 1.3 or later is part of all distributions of \LaTeX{}
version 2005/12/01 or later.

This work has the LPPL maintenance status `maintained'.

The Current Maintainer of this work is Niklas Beisert.

This work consists of the files |README.txt|, |childdoc.ins| and |childdoc.dtx|
as well as the derived files |childdoc.def|, |cdocsamp.tex|
with |cdocsch1.tex|, |cdocsch2.tex|, |cdocspt3.tex|, |cdocspt4.tex|,
|cdocsdrf.tex|, |cdocsfn1.tex|, |cdocsfn2.tex|
as well as |childdoc.pdf|.

%%%%%%%%%%%%%%%%%%%%%%%%%%%%%%%%%%%%%%%%%%%%%%%%%%%%%%%%%%%%%%%%%%%%%%%%%%%%%%%%
\subsection{Files and Installation}

The package consists of the files:
%
\begin{center}
\begin{tabular}{ll}
    |README.txt|   & readme file \\
    |childdoc.ins| & installation file \\
    |childdoc.dtx| & source file \\
    |childdoc.def| & definition file \\
    |cdocsamp.tex| & sample main file \\
    |cdocsch1.tex| & sample include file \\
    |cdocsch2.tex| & sample include file \\
    |cdocspt3.tex| & sample part file \\
    |cdocspt4.tex| & sample part file \\
    |cdocsdrf.tex| & sample redirection file \\
    |cdocsfn1.tex| & sample redirection file \\
    |cdocsfn2.tex| & sample redirection file \\
    |childdoc.pdf| & manual
\end{tabular}
\end{center}
%
The distribution consists of the files
|README.txt|, |childdoc.ins| and |childdoc.dtx|.
%
\begin{itemize}
\item
Run (pdf)\LaTeX{} on |childdoc.dtx|
to compile the manual |childdoc.pdf| (this file).
\item
Run \LaTeX{} on |childdoc.ins| to create the definitions file |childdoc.def|
and the sample |cdocsamp.tex| with include files
|cdocsch1.tex|, |cdocsch2.tex|, |cdocspt3.tex|, |cdocspt4.tex|,
|cdocsdrf.tex|, |cdocsfn1.tex|, |cdocsfn2.tex|.
Then copy the file |childdoc.def| to an appropriate directory of your \LaTeX{}
distribution, e.g.\ \textit{texmf-root}|/tex/latex/childdoc|.
\end{itemize}

%%%%%%%%%%%%%%%%%%%%%%%%%%%%%%%%%%%%%%%%%%%%%%%%%%%%%%%%%%%%%%%%%%%%%%%%%%%%%%%%
\subsection{Related CTAN Packages}

There are several other packages which offer a similar functionality:
%
\begin{itemize}
\item
The packages
\href{http://ctan.org/pkg/docmute}{\textsf{docmute}},
\href{http://ctan.org/pkg/includex}{\textsf{includex}} and
\href{http://ctan.org/pkg/standalone}{\textsf{standalone}}
provide commands to include only the document body of
a child file thus allowing both files to be compiled individually.
\item
The packages \href{http://ctan.org/pkg/subdocs}{\textsf{subdocs}}
and \href{http://ctan.org/pkg/subfiles}{\textsf{subfiles}}
provide structures in which the main and child documents can be
encapsulated and allowing them to be compiled individually.
The inclusion mechanism is different from the conventional |\include|.
\item
The package \href{http://ctan.org/pkg/combine}{\textsf{combine}}
is an elaborate solution to combine several documents into one.
\end{itemize}
%
See also the CTAN topic \href{http://ctan.org/topic/subdocs}{\textsf{subdocs}}
for further related packages.
The present package differs from the above solutions in that
a document structure constructed with the conventional |\include| mechanism
just needs two extra commands at the top of every file
such that all constituent files can be compiled individually.

%%%%%%%%%%%%%%%%%%%%%%%%%%%%%%%%%%%%%%%%%%%%%%%%%%%%%%%%%%%%%%%%%%%%%%%%%%%%%%%%
%\subsection{Feature Suggestions}
%
%The following is a list of features which may be useful for future
%versions of this package:
%%
%\begin{itemize}
%\item
%\ldots
%\end{itemize}

%%%%%%%%%%%%%%%%%%%%%%%%%%%%%%%%%%%%%%%%%%%%%%%%%%%%%%%%%%%%%%%%%%%%%%%%%%%%%%%%
\subsection{Revision History}

%%%%%%%%%%%%%%%%%%%%%%%%%%%%%%%%%%%%%%%%
\paragraph{v2.0:} 2018/12/30

\begin{itemize}
\item
immediate forward processing
\item
added |\childdocby| mechanism
\item
manual restructured
\end{itemize}

%%%%%%%%%%%%%%%%%%%%%%%%%%%%%%%%%%%%%%%%
\paragraph{v1.6:} 2018/01/17

\begin{itemize}
\item
application for development of include files
\item
corrections to manual
\end{itemize}

%%%%%%%%%%%%%%%%%%%%%%%%%%%%%%%%%%%%%%%%
\paragraph{v1.5:} 2017/05/21

\begin{itemize}
\item
more complete structuring introduced
\item
|\childdocof| introduced
\item
|\childdoc| renamed to |\childdocmain|
\item
|\childredirect| renamed to |\childdocforward| and |\childdocforwardprefix|
and functionality expanded
\end{itemize}

%%%%%%%%%%%%%%%%%%%%%%%%%%%%%%%%%%%%%%%%
\paragraph{v1.0:} 2017/04/27

\begin{itemize}
\item
manual and install package
\item
first version published on CTAN
\end{itemize}

%%%%%%%%%%%%%%%%%%%%%%%%%%%%%%%%%%%%%%%%
\paragraph{v0.6:} 2017/04/26

\begin{itemize}
\item
redirection mechanism added
\end{itemize}

%%%%%%%%%%%%%%%%%%%%%%%%%%%%%%%%%%%%%%%%
\paragraph{v0.5:} 2017/04/26

\begin{itemize}
\item
functionality in definition file
\end{itemize}


%%%%%%%%%%%%%%%%%%%%%%%%%%%%%%%%%%%%%%%%%%%%%%%%%%%%%%%%%%%%%%%%%%%%%%%%%%%%%%%%
%%%%%%%%%%%%%%%%%%%%%%%%%%%%%%%%%%%%%%%%%%%%%%%%%%%%%%%%%%%%%%%%%%%%%%%%%%%%%%%%
%%%%%%%%%%%%%%%%%%%%%%%%%%%%%%%%%%%%%%%%%%%%%%%%%%%%%%%%%%%%%%%%%%%%%%%%%%%%%%%%
\appendix

\settowidth\MacroIndent{\rmfamily\scriptsize 000\ }

 \DocInput{childdoc.dtx}

\end{document}
%</driver>
% \fi
%
% %%%%%%%%%%%%%%%%%%%%%%%%%%%%%%%%%%%%%%%%%%%%%%%%%%%%%%%%%%%%%%%%%%%%%%%%%%%%%%
% %%%%%%%%%%%%%%%%%%%%%%%%%%%%%%%%%%%%%%%%%%%%%%%%%%%%%%%%%%%%%%%%%%%%%%%%%%%%%%
% \section{Sample}
%\iffalse
%<*samplemain>
%\fi
%
% The following presents a sample document
% with two chapters, two parts, a title page,
% a compile flag as well as three forwarding files to set the flag.
% It consists of eight |.tex| files:
% \begin{center}
% \begin{tabular}{ll}
% |cdocsamp.tex|&main file\\
% |cdocsch1.tex|&include file for chapter 1\\
% |cdocsch2.tex|&include file for chapter 2\\
% |cdocspt3.tex|&include file for part 3\\
% |cdocspt4.tex|&include file for part 4\\
% |cdocsdrf.tex|&forwarding file for main file in draft mode\\
% |cdocsfi1.tex|&forwarding file for final version of chapter 1\\
% |cdocsfi2.tex|&forwarding file for final version of chapter 2\\
% \end{tabular}
% \end{center}
% Each of the eight files can be compiled directly by the \LaTeX{} compiler.
%
% %%%%%%%%%%%%%%%%%%%%%%%%%%%%%%%%%%%%%%
% \paragraph{Main File.}
%
% The main file is called |cdocsamp.tex|.
%
% Load the \textsf{childdoc} definitions and
% declare the filename for the main document:
%    \begin{macrocode}
\input{childdoc.def}
\childdocmain{}
%    \end{macrocode}

% Optional override for |\version| flag:
%    \begin{macrocode}
%%\ifchilddoc\else\providecommand{\version}{draft}\fi
%    \end{macrocode}

% Define the default values for the |\version| flag
% (|final| for the main file and |draft| for childs):
%    \begin{macrocode}
\ifchilddoc
\providecommand{\version}{draft}
\else
\providecommand{\version}{final}
\fi
%    \end{macrocode}

% Load the standard document class:
%    \begin{macrocode}
\documentclass[12pt]{article}
%    \end{macrocode}

% Start the document body:
%    \begin{macrocode}
\begin{document}
%    \end{macrocode}

% Declare a title page.
% Print title, part of document being processed and version flag:
%    \begin{macrocode}
\addtocounter{page}{-1}
\begin{center}
{\LARGE\bfseries{}childdoc example\par}
\vspace{1cm}
\ifchilddoc
\ifchilddocmanual part\else chapter\fi:
`\childdocname' of `\childdocjob'\par
\else
main document: `\childdocjob'\par
\fi
version: \version\par
\end{center}
\newpage
%    \end{macrocode}

% Manually include selected file,
% otherwise process as usual:
%    \begin{macrocode}
\ifchilddocmanual
\section*{part `\childdocname'}
\input{\childdocname}
\else
%    \end{macrocode}

% Include the two chapters:
%    \begin{macrocode}
\include{cdocsch1}
\include{cdocsch2}
%    \end{macrocode}

% Include the two parts unless only chapters should be displayed:
%    \begin{macrocode}
\ifchilddoc\else
\section{part three}
\input{cdocspt3}
\section{part four}
\input{cdocspt4}
\fi
%    \end{macrocode}

% Process as usual until here:
%    \begin{macrocode}
\fi
%    \end{macrocode}

% End of document body:
%    \begin{macrocode}
\end{document}
%    \end{macrocode}
%\iffalse
%</samplemain>
%\fi
%
% %%%%%%%%%%%%%%%%%%%%%%%%%%%%%%%%%%%%%%
% \paragraph{Chapter Include Files.}
%
% The include files are called |cdocsch1.tex| and |cdocsch2.tex|.
%
%\iffalse
%<*samplechap1|samplechap2>
%\fi

% Optional override for |\version| flag:
%    \begin{macrocode}
%%\providecommand{\version}{final}
%    \end{macrocode}

% Include the main document:
%    \begin{macrocode}
\input{childdoc.def}
\childdocof{cdocsamp}
%    \end{macrocode}

%\iffalse
%</samplechap1|samplechap2>
%\fi
%
%\iffalse
%<*samplechap1>
%\fi
% Some text for chapter 1:
%    \begin{macrocode}
\section{one}
some text in chapter one
%    \end{macrocode}

%\iffalse
%</samplechap1>
%\fi
% Some text for chapter 2:
%\iffalse
%<*samplechap2>
%\fi
%    \begin{macrocode}
\section{two}
more text in chapter two
%    \end{macrocode}

%\iffalse
%</samplechap2>
%\fi
%
% %%%%%%%%%%%%%%%%%%%%%%%%%%%%%%%%%%%%%%
% \paragraph{Part Include Files.}
%
% The include files are called |cdocspt3.tex| and |cdocspt4.tex|.
%
%\iffalse
%<*samplepart3|samplepart4>
%\fi

% Optional override for |\version| flag:
%    \begin{macrocode}
%%\providecommand{\version}{final}
%    \end{macrocode}

% Include the main document:
%    \begin{macrocode}
\input{childdoc.def}
\childdocby{cdocsamp}
%    \end{macrocode}

%\iffalse
%</samplepart3|samplepart4>
%\fi
%
%\iffalse
%<*samplepart3>
%\fi
% Some text for part 3:
%    \begin{macrocode}
some text in part three
%    \end{macrocode}

%\iffalse
%</samplepart3>
%\fi
% Some text for part 4:
%\iffalse
%<*samplepart4>
%\fi
%    \begin{macrocode}
more text in part four
%    \end{macrocode}

%\iffalse
%</samplepart4>
%\fi
%
% %%%%%%%%%%%%%%%%%%%%%%%%%%%%%%%%%%%%%%
% \paragraph{Forwarding for a Complete Draft.}
%
% The following forwarding file |cdocsdrf.tex|
% compiles the main document in draft mode:
%\iffalse
%<*sampledraft>
%\fi
%    \begin{macrocode}
\def\version{draft}
\input{childdoc.def}
\childdocforward{cdocsamp}
%    \end{macrocode}

%\iffalse
%</sampledraft>
%\fi
%
% %%%%%%%%%%%%%%%%%%%%%%%%%%%%%%%%%%%%%%
% \paragraph{Forwarding for Final Version of the Chapters.}
%
% The following forwarding files |cdocsfn1.tex| and |cdocsfn2.tex|
% (with identical content)
% compile the final versions of the child documents
% |cdocsch1.tex| and |cdocsch2.tex|, respectively:
%\iffalse
%<*samplefinal>
%\fi
%    \begin{macrocode}
\def\version{final}
\input{childdoc.def}
\childdocforwardprefix[cdocsamp]{cdocsfn}{cdocsch}
%    \end{macrocode}

%\iffalse
%</samplefinal>
%\fi
%
% %%%%%%%%%%%%%%%%%%%%%%%%%%%%%%%%%%%%%%
% \paragraph{Command Line Processing.}
%
% The following three command lines generate the output files
% |cdocscld|, |cdocscl1| and |cdocscl2|
% which should be identical to
% |cdocsdrf|, |cdocsch1| and |cdocsfn2|, respectively:
% \begin{center}
% \begin{tabular}{l}
% |latex -jobname cdocscld \|\\
% |  "\def\version{draft}\input{childdoc.def}\childdocforward{cdocsamp}"|\\
% |latex -jobname cdocscl1 \|\\
% |  "\input{childdoc.def}\childdocforward[cdocsamp]{cdocsch1}"|\\
% |latex -jobname cdocscl2 \|\\
% |  "\def\version{final}\input{childdoc.def}\childdocforward{cdocsch2}"|
% \end{tabular}
% \end{center}
% Note that the trailing backslash on each first line
% merely continues the input to the second line
% (for convenient cut ant paste).
% Furthermore, the command |latex| can be replaced by any
% of its alternative versions such as |pdflatex|.
%
% %%%%%%%%%%%%%%%%%%%%%%%%%%%%%%%%%%%%%%%%%%%%%%%%%%%%%%%%%%%%%%%%%%%%%%%%%%%%%%
% %%%%%%%%%%%%%%%%%%%%%%%%%%%%%%%%%%%%%%%%%%%%%%%%%%%%%%%%%%%%%%%%%%%%%%%%%%%%%%
% \section{Implementation}
%\iffalse
%<*package>
%\fi
%
% This section describes the definitions file |childdoc.def|.

% The definitions cannot be loaded using |\usepackage| or |\RequirePackage|
% which has a mechanism to prevent loading a style file more than once.
% When loading the definitions by means of |\input|
% multiple instances have to be prevented manually:
%\iffalse
%This code needs to be before the `\ProvidesFile' directive
%which is defined at the beginning of this file.
%Therefore it is also placed there and commented out here.
%</package>
%<*discard>
%\fi
%    \begin{macrocode}
\ifdefined\childdocmain\endinput\fi
%    \end{macrocode}
%\iffalse
%</discard>
%<*package>
%\fi
%
% \macro{\ifchilddoc}
% \macro{\ifchilddocmanual}
% The conditional |\ifchilddoc| tells whether a
% child (true) or main (false) document is being compiled.
% The conditional |\ifchilddocmanual| tells whether
% the |\includeonly| mechanism is used (false) or
% the selection of child files must be performed manually (true).
% The definitions initialise to false:
%    \begin{macrocode}
\newif\ifchilddoc
\newif\ifchilddocmanual
%    \end{macrocode}

% \macro{\childdocname}
% \macro{\childdocjob}
% The macro |\childdocname| stores the name of the main document
% to be compiled. The macro |\childdocjob| stores the name of
% the document on which the \LaTeX{} compiler was originally invoked.
% The content of |\jobname| cannot be compared
% to filenames specified in the source due to different catcodes.
% The following code rescans |\jobname|, stores the result
% in |\childdocname| and saves a copy in |\childdocjob|:
%    \begin{macrocode}
\edef\childdocname{\scantokens\expandafter{\jobname\noexpand}}
\let\childdocjob\childdocname
%    \end{macrocode}

% \macro{\childdocdisable}
% The macro |\childdocdisable| prevents the main file
% from being processed more than once.
% At this stage, the main document command |\childdocmain|
% is assumed to be called once again where it should do nothing.
% Any subsequent call to it should prevent
% a secondary processing of the main document
% It overwrites the forwarding commands
% |\childdocof| and |\childdocforward|
% with empty macros to prevent further inclusions of the main document:
%    \begin{macrocode}
\newcommand{\childdocdisable}
{
  \renewcommand{\childdocmain}[1]{\renewcommand{\childdocmain}[1]{\endinput}}
  \renewcommand{\childdocof}[1]{}
  \renewcommand{\childdocby}[2][]{}
  \renewcommand{\childdocforward}[2][]{}
  \renewcommand{\childdocdisable}{}
}
%    \end{macrocode}

% \macro{\childdocmain}
% The macro |\childdocmain| is to be called at the top of the main file
% with nothing or the main filename (without extension) as argument.
% First, it breaks loops.
% If the argument is not empty and does not match |\childdocname|
% (which is set by the first inclusion of |childdoc.def|),
% |\ifchilddoc| is set to true, |\includeonly| is applied to the child file
% and |\jobname| is set to the main file
% (for proper handling of |.aux| files):
%    \begin{macrocode}
\newcommand{\childdocmain}[1]
{
  \childdocdisable\childdocmain{}
  \if?#1?\else
    \begingroup
      \def\childdoctmp{#1}
      \ifx\childdoctmp\childdocname
        \def\childdoctmp{}
      \else
        \def\childdoctmp
        {
          \childdoctrue
          \includeonly{\childdocname}
          \def\childdocjob{#1}
          \def\jobname{#1}
        }
      \fi
      \expandafter
    \endgroup
    \childdoctmp
  \fi
}
%    \end{macrocode}

% \macro{\childdocof}
% The command |\childdocof| redirects
% compilation to the main file |#1|.
%    \begin{macrocode}
\newcommand{\childdocof}[1]
{
  \childdocdisable
  \childdoctrue
  \includeonly{\childdocname}
  \def\jobname{#1}
  \def\childdocjob{#1}
  \input{#1}
}
%    \end{macrocode}

% \macro{\childdocby}
% The command |\childdocby| ....
%    \begin{macrocode}
\newcommand{\childdocby}[2][]
{
  \childdocdisable
  \childdoctrue
  \childdocmanualtrue
  \if?#1?\else
    \def\jobname{#2}
  \fi
  \def\childdocjob{#2}
  \input{#2}
  \endinput
}
%    \end{macrocode}

% \macro{\childdocforward}
% The command |\childdocforward| redirects
% compilation to the main file or
% (if the optional argument is given) a child file.
% Parameters are set as if the main file
% or a child file starting with |\childdocof| was compiled.
% Then compilation is handed over to the main file:
%    \begin{macrocode}
\newcommand{\childdocforward}[2][]
{
  \begingroup
    \if?#1?
      \def\childdoctmp
      {
        \def\childdocname{#2}
        \def\childdocjob{#2}
        \def\jobname{#2}
        \input{#2}
        \endinput
      }
    \else
      \def\childdoctmp
      {
        \childdocdisable
        \def\childdocname{#2}
        \childdoctrue
        \includeonly{#2}
        \def\childdocjob{#1}
        \def\jobname{#1}
        \input{#1}
        \endinput
      }
    \fi
    \expandafter
  \endgroup
  \childdoctmp
}
%    \end{macrocode}

% \macro{\childdocforwardprefix}
% The command |\childdocforwardprefix| redirects
% compilation to the main or a child file by means of a pattern.
% The prefix |#1| in the current filename is replaced by |#2|
% and the suffix of the current filename is kept
% (it is assumed that the filename does not contain the substring `|~~~|'
% which is used as a delimiter).
% Compilation is handed over to the new file by |\childdocforward|:
%    \begin{macrocode}
\newcommand{\childdocforwardprefix}[3][]
{
  \begingroup
    \def\childdocextract #2##1~~~{\def\childdoctmp{\childdocforward[#1]{#3##1}}}
    \expandafter\childdocextract\childdocname~~~
    \expandafter
  \endgroup
  \childdoctmp
}
%    \end{macrocode}

% \macro{\childdoc}
% The deprecated macro |\childdoc| is a legacy version of |\childdocmain|:
%    \begin{macrocode}
\newcommand{\childdoc}{\childdocmain}
%    \end{macrocode}

% \macro{\childdocredirect}
% The deprecated macro |\childdocredirect| is a legacy version
% of |\childdocforward| and |\childdocforwardprefix|:
%    \begin{macrocode}
\newcommand{\childdocredirect}[2][]
{
  \begingroup
    \if?#1?
      \def\childdoctmp{\childdocforward{#2}}
    \else
      \def\childdoctmp{\childdocforwardprefix{#1}{#2}}
    \fi
    \expandafter
  \endgroup
  \childdoctmp
}
%    \end{macrocode}

%\iffalse
%</package>
%\fi
%
\endinput
|\\
|\childdocby{|\textit{main}|}|\\
\end{tabular}
\end{center}
%
Both forms have slightly different effects as described above.
The main file is prepared as usual, see \secref{sec:include}.

%%%%%%%%%%%%%%%%%%%%%%%%%%%%%%%%%%%%%%%%%%%%%%%%%%%%%%%%%%%%%%%%%%%%%%%%%%%%%%%%
\subsection{Legacy Detection}
\label{sec:detection}

The directive |\childdocmain| in the main file can detect
whether the complete document or merely a child is to be compiled
even without using the directive |\childdocof|.
This method is deprecated because it is less robust
and there is no compelling reason to use it;
it is merely provided for backward compatibility
and it may be removed in future versions.

If the detection mechanism is to be used,
it is mandatory to correctly specify
the filename of the main file as the argument of |\childdocmain|:
%
\begin{center}
\begin{tabular}{l}
|% \iffalse
%
% childdoc.dtx Copyright (C) 2017-2018 Niklas Beisert
%
% This work may be distributed and/or modified under the
% conditions of the LaTeX Project Public License, either version 1.3
% of this license or (at your option) any later version.
% The latest version of this license is in
%   http://www.latex-project.org/lppl.txt
% and version 1.3 or later is part of all distributions of LaTeX
% version 2005/12/01 or later.
%
% This work has the LPPL maintenance status `maintained'.
%
% The Current Maintainer of this work is Niklas Beisert.
%
% This work consists of the files childdoc.dtx and childdoc.ins
% and the derived files childdoc.def and cdocsamp.tex with
% cdocsch1.tex, cdocsch2.tex, cdocsdrf.tex, cdocsfn1.tex, cdocsfn2.tex.
%
%<package>\ifdefined\childdocmain\endinput\fi
%<package>\ProvidesFile{childdoc.def}[2018/12/30 v2.0 child document driver]
%<samplemain>\ProvidesFile{cdocsamp.tex}[2018/12/30 v2.0 sample for childdoc]
%<*driver>
%\ProvidesFile{childdoc.drv}[2018/12/30 v2.0 childdoc reference manual file]
\PassOptionsToClass{10pt,a4paper}{article}
\documentclass{ltxdoc}

\usepackage[margin=35mm]{geometry}
\usepackage{hyperref}
\usepackage{hyperxmp}
\usepackage[usenames]{color}

\hypersetup{colorlinks=true}
\hypersetup{pdfstartview=FitH}
\hypersetup{pdfpagemode=UseNone}
\hypersetup{pdfsource={}}
\hypersetup{pdflang={en-UK}}
\hypersetup{pdfcopyright={Copyright 2017-2018 Niklas Beisert.
  This work may be distributed and/or modified under the
  conditions of the LaTeX Project Public License, either version 1.3
  of this license or (at your option) any later version.}}
\hypersetup{pdflicenseurl={http://www.latex-project.org/lppl.txt}}
\hypersetup{pdfcontactaddress={ETH Zurich, ITP, HIT K,
  Wolfgang-Pauli-Strasse 27}}
\hypersetup{pdfcontactpostcode={8093}}
\hypersetup{pdfcontactcity={Zurich}}
\hypersetup{pdfcontactcountry={Switzerland}}
\hypersetup{pdfcontactemail={nbeisert@itp.phys.ethz.ch}}
\hypersetup{pdfcontacturl={http://people.phys.ethz.ch/\xmptilde nbeisert/}}

\newcommand{\secref}[1]{\hyperref[#1]{section \ref*{#1}}}

\parskip1ex
\parindent0pt
\let\olditemize\itemize
\def\itemize{\olditemize\parskip0pt}

\begin{document}

\title{The \textsf{childdoc} Package}
\hypersetup{pdftitle={The childdoc Package}}
\author{Niklas Beisert\\[2ex]
  Institut f\"ur Theoretische Physik\\
  Eidgen\"ossische Technische Hochschule Z\"urich\\
  Wolfgang-Pauli-Strasse 27, 8093 Z\"urich, Switzerland\\[1ex]
  \href{mailto:nbeisert@itp.phys.ethz.ch}
  {\texttt{nbeisert@itp.phys.ethz.ch}}}
\hypersetup{pdfauthor={Niklas Beisert}}
\hypersetup{pdfsubject={Manual for the LaTeX2e Package childdoc}}
\date{30 December 2018, \textsf{v2.0}}
\maketitle

\begin{abstract}\noindent
\textsf{childdoc} is a \LaTeXe{} package
that enables the direct compilation
of document sections included by |\include|
to individual files.
\end{abstract}

\begingroup
\parskip0ex
\tableofcontents
\endgroup

%%%%%%%%%%%%%%%%%%%%%%%%%%%%%%%%%%%%%%%%%%%%%%%%%%%%%%%%%%%%%%%%%%%%%%%%%%%%%%%%
%%%%%%%%%%%%%%%%%%%%%%%%%%%%%%%%%%%%%%%%%%%%%%%%%%%%%%%%%%%%%%%%%%%%%%%%%%%%%%%%
\section{Introduction}

\LaTeX{} provides a mechanism to structure a large document (such as a book)
into a main file and several child files (containing the chapters)
using the |\include| command.
This mechanism is beneficial for documents
which span hundreds of pages in order to
make the source file(s) more manageable.
Moreover, compilation can be restricted to
selected child files by means of the |\includeonly| command.
The latter feature can be used to reduce the compilation time while editing
(this was significantly more useful in the earlier days of \LaTeX{})
or to generate a smaller document which is easier to navigate.
Another application of |\includeonly| is to generate
documents consisting of selected parts of the complete document.

However, there are a few drawbacks of the plain |\include| mechanism:
\begin{itemize}
\item
The child files cannot be compiled on their own,
they can only be compiled via the main file.
A naive editing environment
(such as a text editor with an option
to have the current file processed by \LaTeX)
may require one to switch to the main file before compiling;
attempting to compile the child file produces errors.
\item
The main file must be modified (each time)
to adjust the |\includeonly| command
to the present needs. This easily leaves the main file in a messy state.
\item
The generated document will always carry the filename
of the main document. This is inconvenient if
several child files are to be compiled and
to be kept for distribution.
\end{itemize}

The present package provides a simple interface
to make child files individually compilable by \LaTeX{}.
Compiling a child file then has the same effect as compiling
the main file with an |\includeonly| command
to select the appropriate child.
Moreover the generated document will carry the name of the child
rather than the main file.
This resolves all three above issues.

This feature is meant to make the editing of books,
thesis documents and lecture notes somewhat more convenient.
However, the package can also be used efficiently for
composing a series of documents (such as exercise sheets)
which are typically distributed individually.
It then assists the author in generating the individual documents
(potentially in different versions)
as well as a document containing the collected series.
Another application is in developing style files
or other kinds of included material
where compilation of the style file could redirect
to a sample or test file.

%%%%%%%%%%%%%%%%%%%%%%%%%%%%%%%%%%%%%%%%%%%%%%%%%%%%%%%%%%%%%%%%%%%%%%%%%%%%%%%%
%%%%%%%%%%%%%%%%%%%%%%%%%%%%%%%%%%%%%%%%%%%%%%%%%%%%%%%%%%%%%%%%%%%%%%%%%%%%%%%%
\section{Usage}

First of all, the package \textsf{childdoc} is \emph{not} a standard
\LaTeXe{} |.sty| style file! Therefore it needs to be invoked in
a non-standard way.

%%%%%%%%%%%%%%%%%%%%%%%%%%%%%%%%%%%%%%%%%%%%%%%%%%%%%%%%%%%%%%%%%%%%%%%%%%%%%%%%
\subsection{Included Files}
\label{sec:include}

%%%%%%%%%%%%%%%%%%%%%%%%%%%%%%%%%%%%%%%%
\DescribeMacro{\childdocmain}
To use the package, add the commands
\begin{center}
\begin{tabular}{l}
|\input{childdoc.def}|\\
|\childdocmain{}|\\
\end{tabular}
\end{center}
at the very top of the main \LaTeX{} file,
in particular \emph{before} the |\documentclass| statement!
The argument of |\childdocmain| should be left empty
(but it must be present).

%%%%%%%%%%%%%%%%%%%%%%%%%%%%%%%%%%%%%%%%
\DescribeMacro{\childdocof}
Furthermore, add the commands
\begin{center}
\begin{tabular}{l}
|\input{childdoc.def}|\\
|\childdocof{|\textit{main}|}|\\
\end{tabular}
\end{center}
at the top of every child file \textit{child}
which is included by |\include{|\textit{child}|}|
from within the main file
(or at least for those files to be compiled individually).
The argument \textit{main} must be the filename of the main file.

There are a couple of
considerations in setting up the main and child documents:

%%%%%%%%%%%%%%%%%%%%%%%%%%%%%%%%%%%%%%%%
\paragraph{Restrictions.}

Please note the following restrictions:
\begin{itemize}
\item
|\childdocmain| must be called with one argument \textit{main}
to ensure compatibility with earlier version of the package.
It must either be empty (|\childdocmain{}|)
or precisely match the filename of the main file in which it is specified.
See \secref{sec:detection} for further information.
\item
The filename \textit{main} must be specified without the |.tex| extension.
\item
The filename \textit{main} is case sensitive
(even in case-insensitive file systems)
due to internal string comparison.
\item
The argument \textit{main} should be fully expanded, it cannot be a macro.
\item
Subdirectories and special characters should be avoided in filenames.
\item
The command |\childdocmain{|\textit{main}|}| must be followed by a whitespace.
It should not be followed immediately by another command
or by a comment mark `|%|'.
This is because the \TeX{} parser reads the token immediately following
the argument of |\childdocmain| and puts it
at the beginning of every child section;
however, a white\-space is ignored.
\end{itemize}

%%%%%%%%%%%%%%%%%%%%%%%%%%%%%%%%%%%%%%%%
\paragraph{Content of Main File.}

It is advisable to place all content in the child files included by |\include|.
Any output contained in the main file will appear in all child documents
unless suppressed manually;
it cannot be suppressed automatically by the |\includeonly| directive
and thus should normally be avoided.
A method to include some content in the main file
by means of conditional processing is described in \secref{sec:conditional}.

%%%%%%%%%%%%%%%%%%%%%%%%%%%%%%%%%%%%%%%%
\paragraph{Page Numbering.}

When only a part of the document is compiled,
the appropriate numbering of pages
(as well as other status parameters)
is determined from the |.aux| files.
The latter contain information from previous passes.
However this information needs to propagate through
all intermediate child documents.
Therefore the page numbering in child documents may well
be inconsistent until the complete document is compiled at least once.

A useful (if unconventional) way to always ensure a consistent
page numbering is to restart the numbering in each child document
and denote the pages by `\textit{child}|.|\textit{page}'
where \textit{child} represents the chapter/section number of the child file.
This can be achieved by the command
|\numberwithin{page}{|\textit{child}|}|
of the \textsf{amsmath} package
where \textit{child} can be |chapter| or |section|
depending on the chosen structuring.
Alternatively, one can modify the macro |\thepage| appropriately
and reset the counter |page| at the start of each child file.

%%%%%%%%%%%%%%%%%%%%%%%%%%%%%%%%%%%%%%%%%%%%%%%%%%%%%%%%%%%%%%%%%%%%%%%%%%%%%%%%
\subsection{Conditional Processing}
\label{sec:conditional}

The package provides a mechanism to compile different versions
of a document. To customise the versions further some conditional processing
can come in handy to distinguish which version is being compiled.
The package provides two macros to describe the compilation context:

%%%%%%%%%%%%%%%%%%%%%%%%%%%%%%%%%%%%%%%%
\DescribeMacro{\ifchilddoc}
The conditional |\ifchilddoc| distinguishes between the compilation of
child documents and the main document:
%
\begin{center}
|\ifchilddoc |\textit{child-code}| |[|\||else |\textit{main-code}]| \||fi|
\end{center}

%%%%%%%%%%%%%%%%%%%%%%%%%%%%%%%%%%%%%%%%
\DescribeMacro{\childdocname}
\DescribeMacro{\childdocjob}
The macro |\childdocname| contains the filename (without extension)
of the main or child file being processed.
Note that |\childdocjob| will always contain the name of the main file.

%%%%%%%%%%%%%%%%%%%%%%%%%%%%%%%%%%%%%%%%
\paragraph{Title Page.}

Conditional processing can be used to include a title or banner page
in the main document when proper precautions are taken.
Importantly, the code in the main file should ensure that the page counter
(as well as other status parameters which are stored in the |.aux| files)
takes the same value after the conditional processing.
Otherwise the page numbers may take divergent values
depending on which part is compiled.

For example, a title page could be declared by:
%
\begin{center}
\begin{tabular}{l}
|\ifchilddoc\||else|\\
|\addtocounter{page}{-1}|\\
\textit{code for title page}\\
|\newpage|\\
|\||fi|
\end{tabular}
\end{center}
%
A banner page for the child documents can be generated by:
%
\begin{center}
\begin{tabular}{l}
|\ifchilddoc|\\
|\addtocounter{page}{-1}|\\
\textit{code for banner page}\\
|\newpage|\\
|\||fi|
\end{tabular}
\end{center}
%
Here one could write a message such as:
\begin{center}
|This is the part \childdocname{} of \childdocjob{}.|
\end{center}

%%%%%%%%%%%%%%%%%%%%%%%%%%%%%%%%%%%%%%%%%%%%%%%%%%%%%%%%%%%%%%%%%%%%%%%%%%%%%%%%
\subsection{Flags}
\label{sec:flags}

The package makes it easy to generate different versions
of the main or child documents.
To this end compilation flags can be defined
and assigned different default values.
They will be particularly useful in conjunction
with the forwarding mechanism described in \secref{sec:forward}.

For example, it may be useful to have a flag |\version|
which can be set to |draft| or |final|.
The document source will contain some conditional code
depending on the value of |\version|.
Suppose further, the flag should default to |final| for the main file
and to |draft| for child files
which is a natural assignment for editing the document.
This is achieved by placing the following code
in the preamble of the main document
(below the |\childdocmain| directive):
%
\begin{center}
\begin{tabular}{l}
|\ifchilddoc|\\
|\providecommand{\version}{draft}|\\
|\||else|\\
|\providecommand{\version}{final}|\\
|\||fi|
\end{tabular}
\end{center}
%
The definition by |\providecommand| makes sure
that previous definitions are not overwritten.
Further statements |\providecommand{\version}{...}|
can thus be added before the above code to override it.

For the main file, one might add a line
(between |\childdocmain| and the above block)
%
\begin{center}
|%\ifchilddoc\||else\providecommand{\version}{draft}\||fi|
\end{center}
%
which can be uncommented to produce a draft version.
Likewise one can add a line to the very top of a child file
(above the |\childdocof{|\textit{main}|}| directive)
%
\begin{center}
|%\providecommand{\version}{final}|
\end{center}
%
which can be uncommented to produce the final version of this child document.

%%%%%%%%%%%%%%%%%%%%%%%%%%%%%%%%%%%%%%%%%%%%%%%%%%%%%%%%%%%%%%%%%%%%%%%%%%%%%%%%
\subsection{Forwarding}
\label{sec:forward}

Different versions of the main or child documents
using compilation flags as described in \secref{sec:flags}
can be (permanently) stored in different files
for convenient compilation, viewing and distribution.
To this end, the package defines a command
to pass on compilation to a different file:

%%%%%%%%%%%%%%%%%%%%%%%%%%%%%%%%%%%%%%%%
\DescribeMacro{\childdocforward}
The command |\childdocforward| redirects processing to
another source file:
%
\begin{center}
\begin{tabular}{l}
|\input{childdoc.def}|\\
|\childdocforward[|\textit{main}|]{|\textit{dest}|}|\\
\end{tabular}
\end{center}
%
The argument \textit{dest} is the destination file
(without extension).
It should be the main file or one of the child files.
Note that further \textsf{childdoc} directives
such as |\childdocof| and |\childdocforward|
in the indicated file will be processed in this form.
The optional argument \textit{main}
passes on directly to the main file \textit{main}
while pretending to compile the child \textit{dest}.
This form behaves as if \textit{dest}
issues |\childdocof{|\textit{main}|}| right away,
and no further \textsf{childdoc} directives will be processed.

%%%%%%%%%%%%%%%%%%%%%%%%%%%%%%%%%%%%%%%%
\DescribeMacro{\...prefix}
In the alternative form |\childdocforwardprefix|,
%
\begin{center}
\begin{tabular}{l}
|\input{childdoc.def}|\\
|\childdocforwardprefix[|\textit{main}|]{|\textit{prefix}|}{|\textit{dest}|}|
\end{tabular}
\end{center}
%
the destination file is determined by a pattern
depending on the current file:
To make this work, the current file must be called
`{\textit{prefix}\hspace{0.2em}\textit{suffix}}'
with \textit{prefix} matching precisely the argument.
Processing is then passed on to the file
`{\textit{dest}\hspace{0.2em}\textit{suffix}}'.
Surely, the same effect is achieved by
directly specifying the
argument `{\textit{dest}\hspace{0.2em}\textit{suffix}}'
in the first form.
However, that requires to set up a different file
for each child. With the alternative form of the command
all these files can have exactly the same content
which simplifies setting them up and maintaining them.

For example, the following file |draft.tex|
with a compilation flag |\version| as described in \secref{sec:flags}
compiles the main document as a draft:
%
\begin{center}
\begin{tabular}{l}
|\def\version{draft}|\\
|\input{childdoc.def}|\\
|\childdocforward{|\textit{main}|}|
\end{tabular}
\end{center}
%
Likewise, the following files |final|\textit{nn}|.tex|
compile the final version of the child document
|child|\textit{nn}|.tex|:
%
\begin{center}
\begin{tabular}{l}
|\def\version{final}|\\
|\input{childdoc.def}|\\
|\childdocforwardprefix{final}{child}|
\end{tabular}
\end{center}
%

Note that when several versions of a main file and/or of each child file
are to be generated, it may be convenient to set up a |Makefile| or
shell script to automatise the process.

%%%%%%%%%%%%%%%%%%%%%%%%%%%%%%%%%%%%%%%%%%%%%%%%%%%%%%%%%%%%%%%%%%%%%%%%%%%%%%%%
\subsection{Command Line Processing}
\label{sec:commandline}

The effect of redirection files can also be achieved by invoking
the \LaTeX{} compiler with a more elaborate command line.
Most conveniently this should be done as part
of a shell script or a |Makefile|.

When using \textsf{childdoc} in the main file, the following
command lines effectively perform a redirection
(note that depending on the shell being used,
backslashes may have to be doubled: `|\|' $\to$ `|\\|'):
%
\begin{center}
|... -jobname "|\textit{target}|" |\\|"|[\textit{flags}]%
|\input{childdoc.def}\childdocforward[|\textit{main}|]{|\textit{dest}|}"|
\end{center}
%
Here \textit{target} is the name of the output file,
\textit{main} is the name of the main file
and \textit{dest} is the name of the main or child file to be processed
(all filenames without extensions).
The optional argument \textit{main} can be omitted
if \textit{main} matches \textit{dest}.
Optionally, compilation \textit{flags} can be defined via |\def| commands.
This command line makes the \TeX{} engine believe
it is compiling the file \textit{target}
whose content is specified as the latter parameter.
The provided code then forwards the processing to
\textit{main} or \textit{dest} as described in \secref{sec:forward}.

%%%%%%%%%%%%%%%%%%%%%%%%%%%%%%%%%%%%%%%%%%%%%%%%%%%%%%%%%%%%%%%%%%%%%%%%%%%%%%%%
\subsection{Include by Input}
\label{sec:input}

Including child documents by |\include| has some restrictions by design.
Most notably, the content of a child document always occupies
its own set of pages; pages cannot be shared between child documents.
Usually, this behaviour makes perfect sense
because each child document contain an essential part of the document.
However, in some situations it may be desirable to compose
a document from a collection of parts
without having mandatory page breaks between then.
For this case, the package
provides a mechanism to include parts
by |\input| which can also be processed individually.
However, by construction this mechanism
requires manual handling of the content to be output.

%%%%%%%%%%%%%%%%%%%%%%%%%%%%%%%%%%%%%%%%
\DescribeMacro{\ifchilddocmanual}
The main file should be prepared as usual, see \secref{sec:include}.
However, the document body must make a distinction
between processing of an individual part and of the main document, e.g.:
%
\begin{center}
\begin{tabular}{l}
|\ifchilddocmanual|\\
|\input{\childdocname}|\\
|\||else|\\
\textit{document body with }|\input{|\textit{part}|}|\\
|\||fi|
\end{tabular}
\end{center}
%
The conditional |\ifchilddocmanual| is true whenever
a part to be included by |\input| is being compiled,
and the name of the part is stored in |\childdocname|.

%%%%%%%%%%%%%%%%%%%%%%%%%%%%%%%%%%%%%%%%
\DescribeMacro{\childdocby}
Each part to be included by |\input| should start with:
%
\begin{center}
\begin{tabular}{l}
|\input{childdoc.def}|\\
|\childdocby{|\textit{main}|}|\\
\end{tabular}
\end{center}
%
The directive |\childdocby| is similar to |\childdocof|
described in \secref{sec:include},
but the subsequent selection of content must be done manually.
To that end, both |\ifchilddoc| and |\ifchilddocmanual|
will be true upon processing of a part,
and the name of the part is stored in |\childdocname|.
Note that |\jobname| will be set to the filename of the current part
so that each part receives an individual |.aux| file
that does not interfere with the |.aux| file(s) of the main document.
This behaviour can be altered by the alternative form
|\childdocby[*]{|\textit{main}|}| (with a non-empty optional argument)
which uses the |.aux| file of the main document
by setting |\jobname| to \textit{main}.

%%%%%%%%%%%%%%%%%%%%%%%%%%%%%%%%%%%%%%%%%%%%%%%%%%%%%%%%%%%%%%%%%%%%%%%%%%%%%%%%
\subsection{Driver Development}
\label{sec:driver}

The \textsf{childdoc} mechanism can also be use for the development
of definition files such as \LaTeX{} styles or classes.
This case differs from the above setup with multiple parts
included by |\include| in that no |\includeonly| should be invoked.
This can be achieved by starting the include file
(before |\ProvidesPackage|) with:
%
\begin{center}
\begin{tabular}{l}
|\input{childdoc.def}|\\
|\childdocforward{|\textit{main}|}|\\
\end{tabular}
\end{center}
%
or alternatively with:
%
\begin{center}
\begin{tabular}{l}
|\input{childdoc.def}|\\
|\childdocby{|\textit{main}|}|\\
\end{tabular}
\end{center}
%
Both forms have slightly different effects as described above.
The main file is prepared as usual, see \secref{sec:include}.

%%%%%%%%%%%%%%%%%%%%%%%%%%%%%%%%%%%%%%%%%%%%%%%%%%%%%%%%%%%%%%%%%%%%%%%%%%%%%%%%
\subsection{Legacy Detection}
\label{sec:detection}

The directive |\childdocmain| in the main file can detect
whether the complete document or merely a child is to be compiled
even without using the directive |\childdocof|.
This method is deprecated because it is less robust
and there is no compelling reason to use it;
it is merely provided for backward compatibility
and it may be removed in future versions.

If the detection mechanism is to be used,
it is mandatory to correctly specify
the filename of the main file as the argument of |\childdocmain|:
%
\begin{center}
\begin{tabular}{l}
|\input{childdoc.def}|\\
|\childdocmain{|\textit{main}|}|\\
\end{tabular}
\end{center}
%
If |\jobname| does not match the argument \textit{main} of |\childdocmain|,
it is assumed that |\jobname| points to the child file to be compiled.
When using |\childdocmain| with the main file specified as argument,
it suffices to start a child file
with just |\input{|\textit{main}|}|
without loading of the package and using |\childdocof|.
If instead all processing is done
with the appropriate \textsf{childdoc} directives,
the argument of \textit{main} of |\childdocmain| can be empty.

An alternative version of the command line processing described
in \secref{sec:commandline} using the detection mechanism reads:
%
\begin{center}
|... -jobname "|\textit{target}|" "|[\textit{flags}]%
[|\def\jobname{|\textit{dest}|}|]|\input{|\textit{main}|}"|
\end{center}

%%%%%%%%%%%%%%%%%%%%%%%%%%%%%%%%%%%%%%%%%%%%%%%%%%%%%%%%%%%%%%%%%%%%%%%%%%%%%%%%
\subsection{Manual Code}
\label{sec:manual}

In case one cannot be certain whether the definitions file |childdoc.def|
is installed on the target \TeX{} distribution
and one prefers not to ship it,
it is conceivable to paste a few relevant commands into the sources.

To that end, drop all statements |\input{childdoc.def}|
and perform the replacements as outlined below.
Instead of |\childdocmain{|\textit{main}|}| add the following code
to the top of the main file:
%
\begin{center}
\begin{tabular}{l}
|\||ifdefined\childdocname\endinput\||fi\newif\ifchilddoc|\\
|\edef\childdocname{\scantokens\expandafter{\jobname\noexpand}}|\\
|\def\childdocmain{|\textit{main}|}\||ifx\childdocmain\childdocname\||else|\\
|\childdoctrue\includeonly{\childdocname}\let\jobname\childdocmain\||fi|\\
\end{tabular}
\end{center}
%
Instead of |\childdocof{|\textit{main}|}| just include the main file
at the top of each child file:
%
\begin{center}
|\input{|\textit{main}|}|
\end{center}
%
A simple redirection |\childdocforward{|\textit{dest}|}| is achieved by:
%
\begin{center}
|\def\jobname{|\textit{dest}|}\input{\jobname}|
\end{center}
%
The redirection with prefix
|\childdocforwardprefix[|\textit{prefix}|]{|\textit{dest}|}|
is accomplished by:
%
\begin{center}
\begin{tabular}{l}
|{\edef\jobname{\scantokens\expandafter{\jobname\noexpand}}|\\
|\def\redirectjob |\textit{prefix}|#1~~~{\gdef\jobname{|\textit{dest}|#1}}|\\
|\expandafter\redirectjob\jobname~~~}\input{\jobname}|
\end{tabular}
\end{center}

In an alternative approach,
child documents can be compiled by a specific command line
without additional code or specific definitions:
%
\begin{center}
|... -jobname "|\textit{target}|" "|[\textit{flags}]%
|\includeonly{|\textit{dest}|}\input{|\textit{main}|}"|
\end{center}
%

%%%%%%%%%%%%%%%%%%%%%%%%%%%%%%%%%%%%%%%%%%%%%%%%%%%%%%%%%%%%%%%%%%%%%%%%%%%%%%%%
%%%%%%%%%%%%%%%%%%%%%%%%%%%%%%%%%%%%%%%%%%%%%%%%%%%%%%%%%%%%%%%%%%%%%%%%%%%%%%%%
\section{Information}

%%%%%%%%%%%%%%%%%%%%%%%%%%%%%%%%%%%%%%%%%%%%%%%%%%%%%%%%%%%%%%%%%%%%%%%%%%%%%%%%
\subsection{Copyright}

Copyright \copyright{} 2017--2018 Niklas Beisert

This work may be distributed and/or modified under the
conditions of the \LaTeX{} Project Public License, either version 1.3
of this license or (at your option) any later version.
The latest version of this license is in
  \url{http://www.latex-project.org/lppl.txt}
and version 1.3 or later is part of all distributions of \LaTeX{}
version 2005/12/01 or later.

This work has the LPPL maintenance status `maintained'.

The Current Maintainer of this work is Niklas Beisert.

This work consists of the files |README.txt|, |childdoc.ins| and |childdoc.dtx|
as well as the derived files |childdoc.def|, |cdocsamp.tex|
with |cdocsch1.tex|, |cdocsch2.tex|, |cdocspt3.tex|, |cdocspt4.tex|,
|cdocsdrf.tex|, |cdocsfn1.tex|, |cdocsfn2.tex|
as well as |childdoc.pdf|.

%%%%%%%%%%%%%%%%%%%%%%%%%%%%%%%%%%%%%%%%%%%%%%%%%%%%%%%%%%%%%%%%%%%%%%%%%%%%%%%%
\subsection{Files and Installation}

The package consists of the files:
%
\begin{center}
\begin{tabular}{ll}
    |README.txt|   & readme file \\
    |childdoc.ins| & installation file \\
    |childdoc.dtx| & source file \\
    |childdoc.def| & definition file \\
    |cdocsamp.tex| & sample main file \\
    |cdocsch1.tex| & sample include file \\
    |cdocsch2.tex| & sample include file \\
    |cdocspt3.tex| & sample part file \\
    |cdocspt4.tex| & sample part file \\
    |cdocsdrf.tex| & sample redirection file \\
    |cdocsfn1.tex| & sample redirection file \\
    |cdocsfn2.tex| & sample redirection file \\
    |childdoc.pdf| & manual
\end{tabular}
\end{center}
%
The distribution consists of the files
|README.txt|, |childdoc.ins| and |childdoc.dtx|.
%
\begin{itemize}
\item
Run (pdf)\LaTeX{} on |childdoc.dtx|
to compile the manual |childdoc.pdf| (this file).
\item
Run \LaTeX{} on |childdoc.ins| to create the definitions file |childdoc.def|
and the sample |cdocsamp.tex| with include files
|cdocsch1.tex|, |cdocsch2.tex|, |cdocspt3.tex|, |cdocspt4.tex|,
|cdocsdrf.tex|, |cdocsfn1.tex|, |cdocsfn2.tex|.
Then copy the file |childdoc.def| to an appropriate directory of your \LaTeX{}
distribution, e.g.\ \textit{texmf-root}|/tex/latex/childdoc|.
\end{itemize}

%%%%%%%%%%%%%%%%%%%%%%%%%%%%%%%%%%%%%%%%%%%%%%%%%%%%%%%%%%%%%%%%%%%%%%%%%%%%%%%%
\subsection{Related CTAN Packages}

There are several other packages which offer a similar functionality:
%
\begin{itemize}
\item
The packages
\href{http://ctan.org/pkg/docmute}{\textsf{docmute}},
\href{http://ctan.org/pkg/includex}{\textsf{includex}} and
\href{http://ctan.org/pkg/standalone}{\textsf{standalone}}
provide commands to include only the document body of
a child file thus allowing both files to be compiled individually.
\item
The packages \href{http://ctan.org/pkg/subdocs}{\textsf{subdocs}}
and \href{http://ctan.org/pkg/subfiles}{\textsf{subfiles}}
provide structures in which the main and child documents can be
encapsulated and allowing them to be compiled individually.
The inclusion mechanism is different from the conventional |\include|.
\item
The package \href{http://ctan.org/pkg/combine}{\textsf{combine}}
is an elaborate solution to combine several documents into one.
\end{itemize}
%
See also the CTAN topic \href{http://ctan.org/topic/subdocs}{\textsf{subdocs}}
for further related packages.
The present package differs from the above solutions in that
a document structure constructed with the conventional |\include| mechanism
just needs two extra commands at the top of every file
such that all constituent files can be compiled individually.

%%%%%%%%%%%%%%%%%%%%%%%%%%%%%%%%%%%%%%%%%%%%%%%%%%%%%%%%%%%%%%%%%%%%%%%%%%%%%%%%
%\subsection{Feature Suggestions}
%
%The following is a list of features which may be useful for future
%versions of this package:
%%
%\begin{itemize}
%\item
%\ldots
%\end{itemize}

%%%%%%%%%%%%%%%%%%%%%%%%%%%%%%%%%%%%%%%%%%%%%%%%%%%%%%%%%%%%%%%%%%%%%%%%%%%%%%%%
\subsection{Revision History}

%%%%%%%%%%%%%%%%%%%%%%%%%%%%%%%%%%%%%%%%
\paragraph{v2.0:} 2018/12/30

\begin{itemize}
\item
immediate forward processing
\item
added |\childdocby| mechanism
\item
manual restructured
\end{itemize}

%%%%%%%%%%%%%%%%%%%%%%%%%%%%%%%%%%%%%%%%
\paragraph{v1.6:} 2018/01/17

\begin{itemize}
\item
application for development of include files
\item
corrections to manual
\end{itemize}

%%%%%%%%%%%%%%%%%%%%%%%%%%%%%%%%%%%%%%%%
\paragraph{v1.5:} 2017/05/21

\begin{itemize}
\item
more complete structuring introduced
\item
|\childdocof| introduced
\item
|\childdoc| renamed to |\childdocmain|
\item
|\childredirect| renamed to |\childdocforward| and |\childdocforwardprefix|
and functionality expanded
\end{itemize}

%%%%%%%%%%%%%%%%%%%%%%%%%%%%%%%%%%%%%%%%
\paragraph{v1.0:} 2017/04/27

\begin{itemize}
\item
manual and install package
\item
first version published on CTAN
\end{itemize}

%%%%%%%%%%%%%%%%%%%%%%%%%%%%%%%%%%%%%%%%
\paragraph{v0.6:} 2017/04/26

\begin{itemize}
\item
redirection mechanism added
\end{itemize}

%%%%%%%%%%%%%%%%%%%%%%%%%%%%%%%%%%%%%%%%
\paragraph{v0.5:} 2017/04/26

\begin{itemize}
\item
functionality in definition file
\end{itemize}


%%%%%%%%%%%%%%%%%%%%%%%%%%%%%%%%%%%%%%%%%%%%%%%%%%%%%%%%%%%%%%%%%%%%%%%%%%%%%%%%
%%%%%%%%%%%%%%%%%%%%%%%%%%%%%%%%%%%%%%%%%%%%%%%%%%%%%%%%%%%%%%%%%%%%%%%%%%%%%%%%
%%%%%%%%%%%%%%%%%%%%%%%%%%%%%%%%%%%%%%%%%%%%%%%%%%%%%%%%%%%%%%%%%%%%%%%%%%%%%%%%
\appendix

\settowidth\MacroIndent{\rmfamily\scriptsize 000\ }

 \DocInput{childdoc.dtx}

\end{document}
%</driver>
% \fi
%
% %%%%%%%%%%%%%%%%%%%%%%%%%%%%%%%%%%%%%%%%%%%%%%%%%%%%%%%%%%%%%%%%%%%%%%%%%%%%%%
% %%%%%%%%%%%%%%%%%%%%%%%%%%%%%%%%%%%%%%%%%%%%%%%%%%%%%%%%%%%%%%%%%%%%%%%%%%%%%%
% \section{Sample}
%\iffalse
%<*samplemain>
%\fi
%
% The following presents a sample document
% with two chapters, two parts, a title page,
% a compile flag as well as three forwarding files to set the flag.
% It consists of eight |.tex| files:
% \begin{center}
% \begin{tabular}{ll}
% |cdocsamp.tex|&main file\\
% |cdocsch1.tex|&include file for chapter 1\\
% |cdocsch2.tex|&include file for chapter 2\\
% |cdocspt3.tex|&include file for part 3\\
% |cdocspt4.tex|&include file for part 4\\
% |cdocsdrf.tex|&forwarding file for main file in draft mode\\
% |cdocsfi1.tex|&forwarding file for final version of chapter 1\\
% |cdocsfi2.tex|&forwarding file for final version of chapter 2\\
% \end{tabular}
% \end{center}
% Each of the eight files can be compiled directly by the \LaTeX{} compiler.
%
% %%%%%%%%%%%%%%%%%%%%%%%%%%%%%%%%%%%%%%
% \paragraph{Main File.}
%
% The main file is called |cdocsamp.tex|.
%
% Load the \textsf{childdoc} definitions and
% declare the filename for the main document:
%    \begin{macrocode}
\input{childdoc.def}
\childdocmain{}
%    \end{macrocode}

% Optional override for |\version| flag:
%    \begin{macrocode}
%%\ifchilddoc\else\providecommand{\version}{draft}\fi
%    \end{macrocode}

% Define the default values for the |\version| flag
% (|final| for the main file and |draft| for childs):
%    \begin{macrocode}
\ifchilddoc
\providecommand{\version}{draft}
\else
\providecommand{\version}{final}
\fi
%    \end{macrocode}

% Load the standard document class:
%    \begin{macrocode}
\documentclass[12pt]{article}
%    \end{macrocode}

% Start the document body:
%    \begin{macrocode}
\begin{document}
%    \end{macrocode}

% Declare a title page.
% Print title, part of document being processed and version flag:
%    \begin{macrocode}
\addtocounter{page}{-1}
\begin{center}
{\LARGE\bfseries{}childdoc example\par}
\vspace{1cm}
\ifchilddoc
\ifchilddocmanual part\else chapter\fi:
`\childdocname' of `\childdocjob'\par
\else
main document: `\childdocjob'\par
\fi
version: \version\par
\end{center}
\newpage
%    \end{macrocode}

% Manually include selected file,
% otherwise process as usual:
%    \begin{macrocode}
\ifchilddocmanual
\section*{part `\childdocname'}
\input{\childdocname}
\else
%    \end{macrocode}

% Include the two chapters:
%    \begin{macrocode}
\include{cdocsch1}
\include{cdocsch2}
%    \end{macrocode}

% Include the two parts unless only chapters should be displayed:
%    \begin{macrocode}
\ifchilddoc\else
\section{part three}
\input{cdocspt3}
\section{part four}
\input{cdocspt4}
\fi
%    \end{macrocode}

% Process as usual until here:
%    \begin{macrocode}
\fi
%    \end{macrocode}

% End of document body:
%    \begin{macrocode}
\end{document}
%    \end{macrocode}
%\iffalse
%</samplemain>
%\fi
%
% %%%%%%%%%%%%%%%%%%%%%%%%%%%%%%%%%%%%%%
% \paragraph{Chapter Include Files.}
%
% The include files are called |cdocsch1.tex| and |cdocsch2.tex|.
%
%\iffalse
%<*samplechap1|samplechap2>
%\fi

% Optional override for |\version| flag:
%    \begin{macrocode}
%%\providecommand{\version}{final}
%    \end{macrocode}

% Include the main document:
%    \begin{macrocode}
\input{childdoc.def}
\childdocof{cdocsamp}
%    \end{macrocode}

%\iffalse
%</samplechap1|samplechap2>
%\fi
%
%\iffalse
%<*samplechap1>
%\fi
% Some text for chapter 1:
%    \begin{macrocode}
\section{one}
some text in chapter one
%    \end{macrocode}

%\iffalse
%</samplechap1>
%\fi
% Some text for chapter 2:
%\iffalse
%<*samplechap2>
%\fi
%    \begin{macrocode}
\section{two}
more text in chapter two
%    \end{macrocode}

%\iffalse
%</samplechap2>
%\fi
%
% %%%%%%%%%%%%%%%%%%%%%%%%%%%%%%%%%%%%%%
% \paragraph{Part Include Files.}
%
% The include files are called |cdocspt3.tex| and |cdocspt4.tex|.
%
%\iffalse
%<*samplepart3|samplepart4>
%\fi

% Optional override for |\version| flag:
%    \begin{macrocode}
%%\providecommand{\version}{final}
%    \end{macrocode}

% Include the main document:
%    \begin{macrocode}
\input{childdoc.def}
\childdocby{cdocsamp}
%    \end{macrocode}

%\iffalse
%</samplepart3|samplepart4>
%\fi
%
%\iffalse
%<*samplepart3>
%\fi
% Some text for part 3:
%    \begin{macrocode}
some text in part three
%    \end{macrocode}

%\iffalse
%</samplepart3>
%\fi
% Some text for part 4:
%\iffalse
%<*samplepart4>
%\fi
%    \begin{macrocode}
more text in part four
%    \end{macrocode}

%\iffalse
%</samplepart4>
%\fi
%
% %%%%%%%%%%%%%%%%%%%%%%%%%%%%%%%%%%%%%%
% \paragraph{Forwarding for a Complete Draft.}
%
% The following forwarding file |cdocsdrf.tex|
% compiles the main document in draft mode:
%\iffalse
%<*sampledraft>
%\fi
%    \begin{macrocode}
\def\version{draft}
\input{childdoc.def}
\childdocforward{cdocsamp}
%    \end{macrocode}

%\iffalse
%</sampledraft>
%\fi
%
% %%%%%%%%%%%%%%%%%%%%%%%%%%%%%%%%%%%%%%
% \paragraph{Forwarding for Final Version of the Chapters.}
%
% The following forwarding files |cdocsfn1.tex| and |cdocsfn2.tex|
% (with identical content)
% compile the final versions of the child documents
% |cdocsch1.tex| and |cdocsch2.tex|, respectively:
%\iffalse
%<*samplefinal>
%\fi
%    \begin{macrocode}
\def\version{final}
\input{childdoc.def}
\childdocforwardprefix[cdocsamp]{cdocsfn}{cdocsch}
%    \end{macrocode}

%\iffalse
%</samplefinal>
%\fi
%
% %%%%%%%%%%%%%%%%%%%%%%%%%%%%%%%%%%%%%%
% \paragraph{Command Line Processing.}
%
% The following three command lines generate the output files
% |cdocscld|, |cdocscl1| and |cdocscl2|
% which should be identical to
% |cdocsdrf|, |cdocsch1| and |cdocsfn2|, respectively:
% \begin{center}
% \begin{tabular}{l}
% |latex -jobname cdocscld \|\\
% |  "\def\version{draft}\input{childdoc.def}\childdocforward{cdocsamp}"|\\
% |latex -jobname cdocscl1 \|\\
% |  "\input{childdoc.def}\childdocforward[cdocsamp]{cdocsch1}"|\\
% |latex -jobname cdocscl2 \|\\
% |  "\def\version{final}\input{childdoc.def}\childdocforward{cdocsch2}"|
% \end{tabular}
% \end{center}
% Note that the trailing backslash on each first line
% merely continues the input to the second line
% (for convenient cut ant paste).
% Furthermore, the command |latex| can be replaced by any
% of its alternative versions such as |pdflatex|.
%
% %%%%%%%%%%%%%%%%%%%%%%%%%%%%%%%%%%%%%%%%%%%%%%%%%%%%%%%%%%%%%%%%%%%%%%%%%%%%%%
% %%%%%%%%%%%%%%%%%%%%%%%%%%%%%%%%%%%%%%%%%%%%%%%%%%%%%%%%%%%%%%%%%%%%%%%%%%%%%%
% \section{Implementation}
%\iffalse
%<*package>
%\fi
%
% This section describes the definitions file |childdoc.def|.

% The definitions cannot be loaded using |\usepackage| or |\RequirePackage|
% which has a mechanism to prevent loading a style file more than once.
% When loading the definitions by means of |\input|
% multiple instances have to be prevented manually:
%\iffalse
%This code needs to be before the `\ProvidesFile' directive
%which is defined at the beginning of this file.
%Therefore it is also placed there and commented out here.
%</package>
%<*discard>
%\fi
%    \begin{macrocode}
\ifdefined\childdocmain\endinput\fi
%    \end{macrocode}
%\iffalse
%</discard>
%<*package>
%\fi
%
% \macro{\ifchilddoc}
% \macro{\ifchilddocmanual}
% The conditional |\ifchilddoc| tells whether a
% child (true) or main (false) document is being compiled.
% The conditional |\ifchilddocmanual| tells whether
% the |\includeonly| mechanism is used (false) or
% the selection of child files must be performed manually (true).
% The definitions initialise to false:
%    \begin{macrocode}
\newif\ifchilddoc
\newif\ifchilddocmanual
%    \end{macrocode}

% \macro{\childdocname}
% \macro{\childdocjob}
% The macro |\childdocname| stores the name of the main document
% to be compiled. The macro |\childdocjob| stores the name of
% the document on which the \LaTeX{} compiler was originally invoked.
% The content of |\jobname| cannot be compared
% to filenames specified in the source due to different catcodes.
% The following code rescans |\jobname|, stores the result
% in |\childdocname| and saves a copy in |\childdocjob|:
%    \begin{macrocode}
\edef\childdocname{\scantokens\expandafter{\jobname\noexpand}}
\let\childdocjob\childdocname
%    \end{macrocode}

% \macro{\childdocdisable}
% The macro |\childdocdisable| prevents the main file
% from being processed more than once.
% At this stage, the main document command |\childdocmain|
% is assumed to be called once again where it should do nothing.
% Any subsequent call to it should prevent
% a secondary processing of the main document
% It overwrites the forwarding commands
% |\childdocof| and |\childdocforward|
% with empty macros to prevent further inclusions of the main document:
%    \begin{macrocode}
\newcommand{\childdocdisable}
{
  \renewcommand{\childdocmain}[1]{\renewcommand{\childdocmain}[1]{\endinput}}
  \renewcommand{\childdocof}[1]{}
  \renewcommand{\childdocby}[2][]{}
  \renewcommand{\childdocforward}[2][]{}
  \renewcommand{\childdocdisable}{}
}
%    \end{macrocode}

% \macro{\childdocmain}
% The macro |\childdocmain| is to be called at the top of the main file
% with nothing or the main filename (without extension) as argument.
% First, it breaks loops.
% If the argument is not empty and does not match |\childdocname|
% (which is set by the first inclusion of |childdoc.def|),
% |\ifchilddoc| is set to true, |\includeonly| is applied to the child file
% and |\jobname| is set to the main file
% (for proper handling of |.aux| files):
%    \begin{macrocode}
\newcommand{\childdocmain}[1]
{
  \childdocdisable\childdocmain{}
  \if?#1?\else
    \begingroup
      \def\childdoctmp{#1}
      \ifx\childdoctmp\childdocname
        \def\childdoctmp{}
      \else
        \def\childdoctmp
        {
          \childdoctrue
          \includeonly{\childdocname}
          \def\childdocjob{#1}
          \def\jobname{#1}
        }
      \fi
      \expandafter
    \endgroup
    \childdoctmp
  \fi
}
%    \end{macrocode}

% \macro{\childdocof}
% The command |\childdocof| redirects
% compilation to the main file |#1|.
%    \begin{macrocode}
\newcommand{\childdocof}[1]
{
  \childdocdisable
  \childdoctrue
  \includeonly{\childdocname}
  \def\jobname{#1}
  \def\childdocjob{#1}
  \input{#1}
}
%    \end{macrocode}

% \macro{\childdocby}
% The command |\childdocby| ....
%    \begin{macrocode}
\newcommand{\childdocby}[2][]
{
  \childdocdisable
  \childdoctrue
  \childdocmanualtrue
  \if?#1?\else
    \def\jobname{#2}
  \fi
  \def\childdocjob{#2}
  \input{#2}
  \endinput
}
%    \end{macrocode}

% \macro{\childdocforward}
% The command |\childdocforward| redirects
% compilation to the main file or
% (if the optional argument is given) a child file.
% Parameters are set as if the main file
% or a child file starting with |\childdocof| was compiled.
% Then compilation is handed over to the main file:
%    \begin{macrocode}
\newcommand{\childdocforward}[2][]
{
  \begingroup
    \if?#1?
      \def\childdoctmp
      {
        \def\childdocname{#2}
        \def\childdocjob{#2}
        \def\jobname{#2}
        \input{#2}
        \endinput
      }
    \else
      \def\childdoctmp
      {
        \childdocdisable
        \def\childdocname{#2}
        \childdoctrue
        \includeonly{#2}
        \def\childdocjob{#1}
        \def\jobname{#1}
        \input{#1}
        \endinput
      }
    \fi
    \expandafter
  \endgroup
  \childdoctmp
}
%    \end{macrocode}

% \macro{\childdocforwardprefix}
% The command |\childdocforwardprefix| redirects
% compilation to the main or a child file by means of a pattern.
% The prefix |#1| in the current filename is replaced by |#2|
% and the suffix of the current filename is kept
% (it is assumed that the filename does not contain the substring `|~~~|'
% which is used as a delimiter).
% Compilation is handed over to the new file by |\childdocforward|:
%    \begin{macrocode}
\newcommand{\childdocforwardprefix}[3][]
{
  \begingroup
    \def\childdocextract #2##1~~~{\def\childdoctmp{\childdocforward[#1]{#3##1}}}
    \expandafter\childdocextract\childdocname~~~
    \expandafter
  \endgroup
  \childdoctmp
}
%    \end{macrocode}

% \macro{\childdoc}
% The deprecated macro |\childdoc| is a legacy version of |\childdocmain|:
%    \begin{macrocode}
\newcommand{\childdoc}{\childdocmain}
%    \end{macrocode}

% \macro{\childdocredirect}
% The deprecated macro |\childdocredirect| is a legacy version
% of |\childdocforward| and |\childdocforwardprefix|:
%    \begin{macrocode}
\newcommand{\childdocredirect}[2][]
{
  \begingroup
    \if?#1?
      \def\childdoctmp{\childdocforward{#2}}
    \else
      \def\childdoctmp{\childdocforwardprefix{#1}{#2}}
    \fi
    \expandafter
  \endgroup
  \childdoctmp
}
%    \end{macrocode}

%\iffalse
%</package>
%\fi
%
\endinput
|\\
|\childdocmain{|\textit{main}|}|\\
\end{tabular}
\end{center}
%
If |\jobname| does not match the argument \textit{main} of |\childdocmain|,
it is assumed that |\jobname| points to the child file to be compiled.
When using |\childdocmain| with the main file specified as argument,
it suffices to start a child file
with just |\input{|\textit{main}|}|
without loading of the package and using |\childdocof|.
If instead all processing is done
with the appropriate \textsf{childdoc} directives,
the argument of \textit{main} of |\childdocmain| can be empty.

An alternative version of the command line processing described
in \secref{sec:commandline} using the detection mechanism reads:
%
\begin{center}
|... -jobname "|\textit{target}|" "|[\textit{flags}]%
[|\def\jobname{|\textit{dest}|}|]|\input{|\textit{main}|}"|
\end{center}

%%%%%%%%%%%%%%%%%%%%%%%%%%%%%%%%%%%%%%%%%%%%%%%%%%%%%%%%%%%%%%%%%%%%%%%%%%%%%%%%
\subsection{Manual Code}
\label{sec:manual}

In case one cannot be certain whether the definitions file |childdoc.def|
is installed on the target \TeX{} distribution
and one prefers not to ship it,
it is conceivable to paste a few relevant commands into the sources.

To that end, drop all statements |% \iffalse
%
% childdoc.dtx Copyright (C) 2017-2018 Niklas Beisert
%
% This work may be distributed and/or modified under the
% conditions of the LaTeX Project Public License, either version 1.3
% of this license or (at your option) any later version.
% The latest version of this license is in
%   http://www.latex-project.org/lppl.txt
% and version 1.3 or later is part of all distributions of LaTeX
% version 2005/12/01 or later.
%
% This work has the LPPL maintenance status `maintained'.
%
% The Current Maintainer of this work is Niklas Beisert.
%
% This work consists of the files childdoc.dtx and childdoc.ins
% and the derived files childdoc.def and cdocsamp.tex with
% cdocsch1.tex, cdocsch2.tex, cdocsdrf.tex, cdocsfn1.tex, cdocsfn2.tex.
%
%<package>\ifdefined\childdocmain\endinput\fi
%<package>\ProvidesFile{childdoc.def}[2018/12/30 v2.0 child document driver]
%<samplemain>\ProvidesFile{cdocsamp.tex}[2018/12/30 v2.0 sample for childdoc]
%<*driver>
%\ProvidesFile{childdoc.drv}[2018/12/30 v2.0 childdoc reference manual file]
\PassOptionsToClass{10pt,a4paper}{article}
\documentclass{ltxdoc}

\usepackage[margin=35mm]{geometry}
\usepackage{hyperref}
\usepackage{hyperxmp}
\usepackage[usenames]{color}

\hypersetup{colorlinks=true}
\hypersetup{pdfstartview=FitH}
\hypersetup{pdfpagemode=UseNone}
\hypersetup{pdfsource={}}
\hypersetup{pdflang={en-UK}}
\hypersetup{pdfcopyright={Copyright 2017-2018 Niklas Beisert.
  This work may be distributed and/or modified under the
  conditions of the LaTeX Project Public License, either version 1.3
  of this license or (at your option) any later version.}}
\hypersetup{pdflicenseurl={http://www.latex-project.org/lppl.txt}}
\hypersetup{pdfcontactaddress={ETH Zurich, ITP, HIT K,
  Wolfgang-Pauli-Strasse 27}}
\hypersetup{pdfcontactpostcode={8093}}
\hypersetup{pdfcontactcity={Zurich}}
\hypersetup{pdfcontactcountry={Switzerland}}
\hypersetup{pdfcontactemail={nbeisert@itp.phys.ethz.ch}}
\hypersetup{pdfcontacturl={http://people.phys.ethz.ch/\xmptilde nbeisert/}}

\newcommand{\secref}[1]{\hyperref[#1]{section \ref*{#1}}}

\parskip1ex
\parindent0pt
\let\olditemize\itemize
\def\itemize{\olditemize\parskip0pt}

\begin{document}

\title{The \textsf{childdoc} Package}
\hypersetup{pdftitle={The childdoc Package}}
\author{Niklas Beisert\\[2ex]
  Institut f\"ur Theoretische Physik\\
  Eidgen\"ossische Technische Hochschule Z\"urich\\
  Wolfgang-Pauli-Strasse 27, 8093 Z\"urich, Switzerland\\[1ex]
  \href{mailto:nbeisert@itp.phys.ethz.ch}
  {\texttt{nbeisert@itp.phys.ethz.ch}}}
\hypersetup{pdfauthor={Niklas Beisert}}
\hypersetup{pdfsubject={Manual for the LaTeX2e Package childdoc}}
\date{30 December 2018, \textsf{v2.0}}
\maketitle

\begin{abstract}\noindent
\textsf{childdoc} is a \LaTeXe{} package
that enables the direct compilation
of document sections included by |\include|
to individual files.
\end{abstract}

\begingroup
\parskip0ex
\tableofcontents
\endgroup

%%%%%%%%%%%%%%%%%%%%%%%%%%%%%%%%%%%%%%%%%%%%%%%%%%%%%%%%%%%%%%%%%%%%%%%%%%%%%%%%
%%%%%%%%%%%%%%%%%%%%%%%%%%%%%%%%%%%%%%%%%%%%%%%%%%%%%%%%%%%%%%%%%%%%%%%%%%%%%%%%
\section{Introduction}

\LaTeX{} provides a mechanism to structure a large document (such as a book)
into a main file and several child files (containing the chapters)
using the |\include| command.
This mechanism is beneficial for documents
which span hundreds of pages in order to
make the source file(s) more manageable.
Moreover, compilation can be restricted to
selected child files by means of the |\includeonly| command.
The latter feature can be used to reduce the compilation time while editing
(this was significantly more useful in the earlier days of \LaTeX{})
or to generate a smaller document which is easier to navigate.
Another application of |\includeonly| is to generate
documents consisting of selected parts of the complete document.

However, there are a few drawbacks of the plain |\include| mechanism:
\begin{itemize}
\item
The child files cannot be compiled on their own,
they can only be compiled via the main file.
A naive editing environment
(such as a text editor with an option
to have the current file processed by \LaTeX)
may require one to switch to the main file before compiling;
attempting to compile the child file produces errors.
\item
The main file must be modified (each time)
to adjust the |\includeonly| command
to the present needs. This easily leaves the main file in a messy state.
\item
The generated document will always carry the filename
of the main document. This is inconvenient if
several child files are to be compiled and
to be kept for distribution.
\end{itemize}

The present package provides a simple interface
to make child files individually compilable by \LaTeX{}.
Compiling a child file then has the same effect as compiling
the main file with an |\includeonly| command
to select the appropriate child.
Moreover the generated document will carry the name of the child
rather than the main file.
This resolves all three above issues.

This feature is meant to make the editing of books,
thesis documents and lecture notes somewhat more convenient.
However, the package can also be used efficiently for
composing a series of documents (such as exercise sheets)
which are typically distributed individually.
It then assists the author in generating the individual documents
(potentially in different versions)
as well as a document containing the collected series.
Another application is in developing style files
or other kinds of included material
where compilation of the style file could redirect
to a sample or test file.

%%%%%%%%%%%%%%%%%%%%%%%%%%%%%%%%%%%%%%%%%%%%%%%%%%%%%%%%%%%%%%%%%%%%%%%%%%%%%%%%
%%%%%%%%%%%%%%%%%%%%%%%%%%%%%%%%%%%%%%%%%%%%%%%%%%%%%%%%%%%%%%%%%%%%%%%%%%%%%%%%
\section{Usage}

First of all, the package \textsf{childdoc} is \emph{not} a standard
\LaTeXe{} |.sty| style file! Therefore it needs to be invoked in
a non-standard way.

%%%%%%%%%%%%%%%%%%%%%%%%%%%%%%%%%%%%%%%%%%%%%%%%%%%%%%%%%%%%%%%%%%%%%%%%%%%%%%%%
\subsection{Included Files}
\label{sec:include}

%%%%%%%%%%%%%%%%%%%%%%%%%%%%%%%%%%%%%%%%
\DescribeMacro{\childdocmain}
To use the package, add the commands
\begin{center}
\begin{tabular}{l}
|\input{childdoc.def}|\\
|\childdocmain{}|\\
\end{tabular}
\end{center}
at the very top of the main \LaTeX{} file,
in particular \emph{before} the |\documentclass| statement!
The argument of |\childdocmain| should be left empty
(but it must be present).

%%%%%%%%%%%%%%%%%%%%%%%%%%%%%%%%%%%%%%%%
\DescribeMacro{\childdocof}
Furthermore, add the commands
\begin{center}
\begin{tabular}{l}
|\input{childdoc.def}|\\
|\childdocof{|\textit{main}|}|\\
\end{tabular}
\end{center}
at the top of every child file \textit{child}
which is included by |\include{|\textit{child}|}|
from within the main file
(or at least for those files to be compiled individually).
The argument \textit{main} must be the filename of the main file.

There are a couple of
considerations in setting up the main and child documents:

%%%%%%%%%%%%%%%%%%%%%%%%%%%%%%%%%%%%%%%%
\paragraph{Restrictions.}

Please note the following restrictions:
\begin{itemize}
\item
|\childdocmain| must be called with one argument \textit{main}
to ensure compatibility with earlier version of the package.
It must either be empty (|\childdocmain{}|)
or precisely match the filename of the main file in which it is specified.
See \secref{sec:detection} for further information.
\item
The filename \textit{main} must be specified without the |.tex| extension.
\item
The filename \textit{main} is case sensitive
(even in case-insensitive file systems)
due to internal string comparison.
\item
The argument \textit{main} should be fully expanded, it cannot be a macro.
\item
Subdirectories and special characters should be avoided in filenames.
\item
The command |\childdocmain{|\textit{main}|}| must be followed by a whitespace.
It should not be followed immediately by another command
or by a comment mark `|%|'.
This is because the \TeX{} parser reads the token immediately following
the argument of |\childdocmain| and puts it
at the beginning of every child section;
however, a white\-space is ignored.
\end{itemize}

%%%%%%%%%%%%%%%%%%%%%%%%%%%%%%%%%%%%%%%%
\paragraph{Content of Main File.}

It is advisable to place all content in the child files included by |\include|.
Any output contained in the main file will appear in all child documents
unless suppressed manually;
it cannot be suppressed automatically by the |\includeonly| directive
and thus should normally be avoided.
A method to include some content in the main file
by means of conditional processing is described in \secref{sec:conditional}.

%%%%%%%%%%%%%%%%%%%%%%%%%%%%%%%%%%%%%%%%
\paragraph{Page Numbering.}

When only a part of the document is compiled,
the appropriate numbering of pages
(as well as other status parameters)
is determined from the |.aux| files.
The latter contain information from previous passes.
However this information needs to propagate through
all intermediate child documents.
Therefore the page numbering in child documents may well
be inconsistent until the complete document is compiled at least once.

A useful (if unconventional) way to always ensure a consistent
page numbering is to restart the numbering in each child document
and denote the pages by `\textit{child}|.|\textit{page}'
where \textit{child} represents the chapter/section number of the child file.
This can be achieved by the command
|\numberwithin{page}{|\textit{child}|}|
of the \textsf{amsmath} package
where \textit{child} can be |chapter| or |section|
depending on the chosen structuring.
Alternatively, one can modify the macro |\thepage| appropriately
and reset the counter |page| at the start of each child file.

%%%%%%%%%%%%%%%%%%%%%%%%%%%%%%%%%%%%%%%%%%%%%%%%%%%%%%%%%%%%%%%%%%%%%%%%%%%%%%%%
\subsection{Conditional Processing}
\label{sec:conditional}

The package provides a mechanism to compile different versions
of a document. To customise the versions further some conditional processing
can come in handy to distinguish which version is being compiled.
The package provides two macros to describe the compilation context:

%%%%%%%%%%%%%%%%%%%%%%%%%%%%%%%%%%%%%%%%
\DescribeMacro{\ifchilddoc}
The conditional |\ifchilddoc| distinguishes between the compilation of
child documents and the main document:
%
\begin{center}
|\ifchilddoc |\textit{child-code}| |[|\||else |\textit{main-code}]| \||fi|
\end{center}

%%%%%%%%%%%%%%%%%%%%%%%%%%%%%%%%%%%%%%%%
\DescribeMacro{\childdocname}
\DescribeMacro{\childdocjob}
The macro |\childdocname| contains the filename (without extension)
of the main or child file being processed.
Note that |\childdocjob| will always contain the name of the main file.

%%%%%%%%%%%%%%%%%%%%%%%%%%%%%%%%%%%%%%%%
\paragraph{Title Page.}

Conditional processing can be used to include a title or banner page
in the main document when proper precautions are taken.
Importantly, the code in the main file should ensure that the page counter
(as well as other status parameters which are stored in the |.aux| files)
takes the same value after the conditional processing.
Otherwise the page numbers may take divergent values
depending on which part is compiled.

For example, a title page could be declared by:
%
\begin{center}
\begin{tabular}{l}
|\ifchilddoc\||else|\\
|\addtocounter{page}{-1}|\\
\textit{code for title page}\\
|\newpage|\\
|\||fi|
\end{tabular}
\end{center}
%
A banner page for the child documents can be generated by:
%
\begin{center}
\begin{tabular}{l}
|\ifchilddoc|\\
|\addtocounter{page}{-1}|\\
\textit{code for banner page}\\
|\newpage|\\
|\||fi|
\end{tabular}
\end{center}
%
Here one could write a message such as:
\begin{center}
|This is the part \childdocname{} of \childdocjob{}.|
\end{center}

%%%%%%%%%%%%%%%%%%%%%%%%%%%%%%%%%%%%%%%%%%%%%%%%%%%%%%%%%%%%%%%%%%%%%%%%%%%%%%%%
\subsection{Flags}
\label{sec:flags}

The package makes it easy to generate different versions
of the main or child documents.
To this end compilation flags can be defined
and assigned different default values.
They will be particularly useful in conjunction
with the forwarding mechanism described in \secref{sec:forward}.

For example, it may be useful to have a flag |\version|
which can be set to |draft| or |final|.
The document source will contain some conditional code
depending on the value of |\version|.
Suppose further, the flag should default to |final| for the main file
and to |draft| for child files
which is a natural assignment for editing the document.
This is achieved by placing the following code
in the preamble of the main document
(below the |\childdocmain| directive):
%
\begin{center}
\begin{tabular}{l}
|\ifchilddoc|\\
|\providecommand{\version}{draft}|\\
|\||else|\\
|\providecommand{\version}{final}|\\
|\||fi|
\end{tabular}
\end{center}
%
The definition by |\providecommand| makes sure
that previous definitions are not overwritten.
Further statements |\providecommand{\version}{...}|
can thus be added before the above code to override it.

For the main file, one might add a line
(between |\childdocmain| and the above block)
%
\begin{center}
|%\ifchilddoc\||else\providecommand{\version}{draft}\||fi|
\end{center}
%
which can be uncommented to produce a draft version.
Likewise one can add a line to the very top of a child file
(above the |\childdocof{|\textit{main}|}| directive)
%
\begin{center}
|%\providecommand{\version}{final}|
\end{center}
%
which can be uncommented to produce the final version of this child document.

%%%%%%%%%%%%%%%%%%%%%%%%%%%%%%%%%%%%%%%%%%%%%%%%%%%%%%%%%%%%%%%%%%%%%%%%%%%%%%%%
\subsection{Forwarding}
\label{sec:forward}

Different versions of the main or child documents
using compilation flags as described in \secref{sec:flags}
can be (permanently) stored in different files
for convenient compilation, viewing and distribution.
To this end, the package defines a command
to pass on compilation to a different file:

%%%%%%%%%%%%%%%%%%%%%%%%%%%%%%%%%%%%%%%%
\DescribeMacro{\childdocforward}
The command |\childdocforward| redirects processing to
another source file:
%
\begin{center}
\begin{tabular}{l}
|\input{childdoc.def}|\\
|\childdocforward[|\textit{main}|]{|\textit{dest}|}|\\
\end{tabular}
\end{center}
%
The argument \textit{dest} is the destination file
(without extension).
It should be the main file or one of the child files.
Note that further \textsf{childdoc} directives
such as |\childdocof| and |\childdocforward|
in the indicated file will be processed in this form.
The optional argument \textit{main}
passes on directly to the main file \textit{main}
while pretending to compile the child \textit{dest}.
This form behaves as if \textit{dest}
issues |\childdocof{|\textit{main}|}| right away,
and no further \textsf{childdoc} directives will be processed.

%%%%%%%%%%%%%%%%%%%%%%%%%%%%%%%%%%%%%%%%
\DescribeMacro{\...prefix}
In the alternative form |\childdocforwardprefix|,
%
\begin{center}
\begin{tabular}{l}
|\input{childdoc.def}|\\
|\childdocforwardprefix[|\textit{main}|]{|\textit{prefix}|}{|\textit{dest}|}|
\end{tabular}
\end{center}
%
the destination file is determined by a pattern
depending on the current file:
To make this work, the current file must be called
`{\textit{prefix}\hspace{0.2em}\textit{suffix}}'
with \textit{prefix} matching precisely the argument.
Processing is then passed on to the file
`{\textit{dest}\hspace{0.2em}\textit{suffix}}'.
Surely, the same effect is achieved by
directly specifying the
argument `{\textit{dest}\hspace{0.2em}\textit{suffix}}'
in the first form.
However, that requires to set up a different file
for each child. With the alternative form of the command
all these files can have exactly the same content
which simplifies setting them up and maintaining them.

For example, the following file |draft.tex|
with a compilation flag |\version| as described in \secref{sec:flags}
compiles the main document as a draft:
%
\begin{center}
\begin{tabular}{l}
|\def\version{draft}|\\
|\input{childdoc.def}|\\
|\childdocforward{|\textit{main}|}|
\end{tabular}
\end{center}
%
Likewise, the following files |final|\textit{nn}|.tex|
compile the final version of the child document
|child|\textit{nn}|.tex|:
%
\begin{center}
\begin{tabular}{l}
|\def\version{final}|\\
|\input{childdoc.def}|\\
|\childdocforwardprefix{final}{child}|
\end{tabular}
\end{center}
%

Note that when several versions of a main file and/or of each child file
are to be generated, it may be convenient to set up a |Makefile| or
shell script to automatise the process.

%%%%%%%%%%%%%%%%%%%%%%%%%%%%%%%%%%%%%%%%%%%%%%%%%%%%%%%%%%%%%%%%%%%%%%%%%%%%%%%%
\subsection{Command Line Processing}
\label{sec:commandline}

The effect of redirection files can also be achieved by invoking
the \LaTeX{} compiler with a more elaborate command line.
Most conveniently this should be done as part
of a shell script or a |Makefile|.

When using \textsf{childdoc} in the main file, the following
command lines effectively perform a redirection
(note that depending on the shell being used,
backslashes may have to be doubled: `|\|' $\to$ `|\\|'):
%
\begin{center}
|... -jobname "|\textit{target}|" |\\|"|[\textit{flags}]%
|\input{childdoc.def}\childdocforward[|\textit{main}|]{|\textit{dest}|}"|
\end{center}
%
Here \textit{target} is the name of the output file,
\textit{main} is the name of the main file
and \textit{dest} is the name of the main or child file to be processed
(all filenames without extensions).
The optional argument \textit{main} can be omitted
if \textit{main} matches \textit{dest}.
Optionally, compilation \textit{flags} can be defined via |\def| commands.
This command line makes the \TeX{} engine believe
it is compiling the file \textit{target}
whose content is specified as the latter parameter.
The provided code then forwards the processing to
\textit{main} or \textit{dest} as described in \secref{sec:forward}.

%%%%%%%%%%%%%%%%%%%%%%%%%%%%%%%%%%%%%%%%%%%%%%%%%%%%%%%%%%%%%%%%%%%%%%%%%%%%%%%%
\subsection{Include by Input}
\label{sec:input}

Including child documents by |\include| has some restrictions by design.
Most notably, the content of a child document always occupies
its own set of pages; pages cannot be shared between child documents.
Usually, this behaviour makes perfect sense
because each child document contain an essential part of the document.
However, in some situations it may be desirable to compose
a document from a collection of parts
without having mandatory page breaks between then.
For this case, the package
provides a mechanism to include parts
by |\input| which can also be processed individually.
However, by construction this mechanism
requires manual handling of the content to be output.

%%%%%%%%%%%%%%%%%%%%%%%%%%%%%%%%%%%%%%%%
\DescribeMacro{\ifchilddocmanual}
The main file should be prepared as usual, see \secref{sec:include}.
However, the document body must make a distinction
between processing of an individual part and of the main document, e.g.:
%
\begin{center}
\begin{tabular}{l}
|\ifchilddocmanual|\\
|\input{\childdocname}|\\
|\||else|\\
\textit{document body with }|\input{|\textit{part}|}|\\
|\||fi|
\end{tabular}
\end{center}
%
The conditional |\ifchilddocmanual| is true whenever
a part to be included by |\input| is being compiled,
and the name of the part is stored in |\childdocname|.

%%%%%%%%%%%%%%%%%%%%%%%%%%%%%%%%%%%%%%%%
\DescribeMacro{\childdocby}
Each part to be included by |\input| should start with:
%
\begin{center}
\begin{tabular}{l}
|\input{childdoc.def}|\\
|\childdocby{|\textit{main}|}|\\
\end{tabular}
\end{center}
%
The directive |\childdocby| is similar to |\childdocof|
described in \secref{sec:include},
but the subsequent selection of content must be done manually.
To that end, both |\ifchilddoc| and |\ifchilddocmanual|
will be true upon processing of a part,
and the name of the part is stored in |\childdocname|.
Note that |\jobname| will be set to the filename of the current part
so that each part receives an individual |.aux| file
that does not interfere with the |.aux| file(s) of the main document.
This behaviour can be altered by the alternative form
|\childdocby[*]{|\textit{main}|}| (with a non-empty optional argument)
which uses the |.aux| file of the main document
by setting |\jobname| to \textit{main}.

%%%%%%%%%%%%%%%%%%%%%%%%%%%%%%%%%%%%%%%%%%%%%%%%%%%%%%%%%%%%%%%%%%%%%%%%%%%%%%%%
\subsection{Driver Development}
\label{sec:driver}

The \textsf{childdoc} mechanism can also be use for the development
of definition files such as \LaTeX{} styles or classes.
This case differs from the above setup with multiple parts
included by |\include| in that no |\includeonly| should be invoked.
This can be achieved by starting the include file
(before |\ProvidesPackage|) with:
%
\begin{center}
\begin{tabular}{l}
|\input{childdoc.def}|\\
|\childdocforward{|\textit{main}|}|\\
\end{tabular}
\end{center}
%
or alternatively with:
%
\begin{center}
\begin{tabular}{l}
|\input{childdoc.def}|\\
|\childdocby{|\textit{main}|}|\\
\end{tabular}
\end{center}
%
Both forms have slightly different effects as described above.
The main file is prepared as usual, see \secref{sec:include}.

%%%%%%%%%%%%%%%%%%%%%%%%%%%%%%%%%%%%%%%%%%%%%%%%%%%%%%%%%%%%%%%%%%%%%%%%%%%%%%%%
\subsection{Legacy Detection}
\label{sec:detection}

The directive |\childdocmain| in the main file can detect
whether the complete document or merely a child is to be compiled
even without using the directive |\childdocof|.
This method is deprecated because it is less robust
and there is no compelling reason to use it;
it is merely provided for backward compatibility
and it may be removed in future versions.

If the detection mechanism is to be used,
it is mandatory to correctly specify
the filename of the main file as the argument of |\childdocmain|:
%
\begin{center}
\begin{tabular}{l}
|\input{childdoc.def}|\\
|\childdocmain{|\textit{main}|}|\\
\end{tabular}
\end{center}
%
If |\jobname| does not match the argument \textit{main} of |\childdocmain|,
it is assumed that |\jobname| points to the child file to be compiled.
When using |\childdocmain| with the main file specified as argument,
it suffices to start a child file
with just |\input{|\textit{main}|}|
without loading of the package and using |\childdocof|.
If instead all processing is done
with the appropriate \textsf{childdoc} directives,
the argument of \textit{main} of |\childdocmain| can be empty.

An alternative version of the command line processing described
in \secref{sec:commandline} using the detection mechanism reads:
%
\begin{center}
|... -jobname "|\textit{target}|" "|[\textit{flags}]%
[|\def\jobname{|\textit{dest}|}|]|\input{|\textit{main}|}"|
\end{center}

%%%%%%%%%%%%%%%%%%%%%%%%%%%%%%%%%%%%%%%%%%%%%%%%%%%%%%%%%%%%%%%%%%%%%%%%%%%%%%%%
\subsection{Manual Code}
\label{sec:manual}

In case one cannot be certain whether the definitions file |childdoc.def|
is installed on the target \TeX{} distribution
and one prefers not to ship it,
it is conceivable to paste a few relevant commands into the sources.

To that end, drop all statements |\input{childdoc.def}|
and perform the replacements as outlined below.
Instead of |\childdocmain{|\textit{main}|}| add the following code
to the top of the main file:
%
\begin{center}
\begin{tabular}{l}
|\||ifdefined\childdocname\endinput\||fi\newif\ifchilddoc|\\
|\edef\childdocname{\scantokens\expandafter{\jobname\noexpand}}|\\
|\def\childdocmain{|\textit{main}|}\||ifx\childdocmain\childdocname\||else|\\
|\childdoctrue\includeonly{\childdocname}\let\jobname\childdocmain\||fi|\\
\end{tabular}
\end{center}
%
Instead of |\childdocof{|\textit{main}|}| just include the main file
at the top of each child file:
%
\begin{center}
|\input{|\textit{main}|}|
\end{center}
%
A simple redirection |\childdocforward{|\textit{dest}|}| is achieved by:
%
\begin{center}
|\def\jobname{|\textit{dest}|}\input{\jobname}|
\end{center}
%
The redirection with prefix
|\childdocforwardprefix[|\textit{prefix}|]{|\textit{dest}|}|
is accomplished by:
%
\begin{center}
\begin{tabular}{l}
|{\edef\jobname{\scantokens\expandafter{\jobname\noexpand}}|\\
|\def\redirectjob |\textit{prefix}|#1~~~{\gdef\jobname{|\textit{dest}|#1}}|\\
|\expandafter\redirectjob\jobname~~~}\input{\jobname}|
\end{tabular}
\end{center}

In an alternative approach,
child documents can be compiled by a specific command line
without additional code or specific definitions:
%
\begin{center}
|... -jobname "|\textit{target}|" "|[\textit{flags}]%
|\includeonly{|\textit{dest}|}\input{|\textit{main}|}"|
\end{center}
%

%%%%%%%%%%%%%%%%%%%%%%%%%%%%%%%%%%%%%%%%%%%%%%%%%%%%%%%%%%%%%%%%%%%%%%%%%%%%%%%%
%%%%%%%%%%%%%%%%%%%%%%%%%%%%%%%%%%%%%%%%%%%%%%%%%%%%%%%%%%%%%%%%%%%%%%%%%%%%%%%%
\section{Information}

%%%%%%%%%%%%%%%%%%%%%%%%%%%%%%%%%%%%%%%%%%%%%%%%%%%%%%%%%%%%%%%%%%%%%%%%%%%%%%%%
\subsection{Copyright}

Copyright \copyright{} 2017--2018 Niklas Beisert

This work may be distributed and/or modified under the
conditions of the \LaTeX{} Project Public License, either version 1.3
of this license or (at your option) any later version.
The latest version of this license is in
  \url{http://www.latex-project.org/lppl.txt}
and version 1.3 or later is part of all distributions of \LaTeX{}
version 2005/12/01 or later.

This work has the LPPL maintenance status `maintained'.

The Current Maintainer of this work is Niklas Beisert.

This work consists of the files |README.txt|, |childdoc.ins| and |childdoc.dtx|
as well as the derived files |childdoc.def|, |cdocsamp.tex|
with |cdocsch1.tex|, |cdocsch2.tex|, |cdocspt3.tex|, |cdocspt4.tex|,
|cdocsdrf.tex|, |cdocsfn1.tex|, |cdocsfn2.tex|
as well as |childdoc.pdf|.

%%%%%%%%%%%%%%%%%%%%%%%%%%%%%%%%%%%%%%%%%%%%%%%%%%%%%%%%%%%%%%%%%%%%%%%%%%%%%%%%
\subsection{Files and Installation}

The package consists of the files:
%
\begin{center}
\begin{tabular}{ll}
    |README.txt|   & readme file \\
    |childdoc.ins| & installation file \\
    |childdoc.dtx| & source file \\
    |childdoc.def| & definition file \\
    |cdocsamp.tex| & sample main file \\
    |cdocsch1.tex| & sample include file \\
    |cdocsch2.tex| & sample include file \\
    |cdocspt3.tex| & sample part file \\
    |cdocspt4.tex| & sample part file \\
    |cdocsdrf.tex| & sample redirection file \\
    |cdocsfn1.tex| & sample redirection file \\
    |cdocsfn2.tex| & sample redirection file \\
    |childdoc.pdf| & manual
\end{tabular}
\end{center}
%
The distribution consists of the files
|README.txt|, |childdoc.ins| and |childdoc.dtx|.
%
\begin{itemize}
\item
Run (pdf)\LaTeX{} on |childdoc.dtx|
to compile the manual |childdoc.pdf| (this file).
\item
Run \LaTeX{} on |childdoc.ins| to create the definitions file |childdoc.def|
and the sample |cdocsamp.tex| with include files
|cdocsch1.tex|, |cdocsch2.tex|, |cdocspt3.tex|, |cdocspt4.tex|,
|cdocsdrf.tex|, |cdocsfn1.tex|, |cdocsfn2.tex|.
Then copy the file |childdoc.def| to an appropriate directory of your \LaTeX{}
distribution, e.g.\ \textit{texmf-root}|/tex/latex/childdoc|.
\end{itemize}

%%%%%%%%%%%%%%%%%%%%%%%%%%%%%%%%%%%%%%%%%%%%%%%%%%%%%%%%%%%%%%%%%%%%%%%%%%%%%%%%
\subsection{Related CTAN Packages}

There are several other packages which offer a similar functionality:
%
\begin{itemize}
\item
The packages
\href{http://ctan.org/pkg/docmute}{\textsf{docmute}},
\href{http://ctan.org/pkg/includex}{\textsf{includex}} and
\href{http://ctan.org/pkg/standalone}{\textsf{standalone}}
provide commands to include only the document body of
a child file thus allowing both files to be compiled individually.
\item
The packages \href{http://ctan.org/pkg/subdocs}{\textsf{subdocs}}
and \href{http://ctan.org/pkg/subfiles}{\textsf{subfiles}}
provide structures in which the main and child documents can be
encapsulated and allowing them to be compiled individually.
The inclusion mechanism is different from the conventional |\include|.
\item
The package \href{http://ctan.org/pkg/combine}{\textsf{combine}}
is an elaborate solution to combine several documents into one.
\end{itemize}
%
See also the CTAN topic \href{http://ctan.org/topic/subdocs}{\textsf{subdocs}}
for further related packages.
The present package differs from the above solutions in that
a document structure constructed with the conventional |\include| mechanism
just needs two extra commands at the top of every file
such that all constituent files can be compiled individually.

%%%%%%%%%%%%%%%%%%%%%%%%%%%%%%%%%%%%%%%%%%%%%%%%%%%%%%%%%%%%%%%%%%%%%%%%%%%%%%%%
%\subsection{Feature Suggestions}
%
%The following is a list of features which may be useful for future
%versions of this package:
%%
%\begin{itemize}
%\item
%\ldots
%\end{itemize}

%%%%%%%%%%%%%%%%%%%%%%%%%%%%%%%%%%%%%%%%%%%%%%%%%%%%%%%%%%%%%%%%%%%%%%%%%%%%%%%%
\subsection{Revision History}

%%%%%%%%%%%%%%%%%%%%%%%%%%%%%%%%%%%%%%%%
\paragraph{v2.0:} 2018/12/30

\begin{itemize}
\item
immediate forward processing
\item
added |\childdocby| mechanism
\item
manual restructured
\end{itemize}

%%%%%%%%%%%%%%%%%%%%%%%%%%%%%%%%%%%%%%%%
\paragraph{v1.6:} 2018/01/17

\begin{itemize}
\item
application for development of include files
\item
corrections to manual
\end{itemize}

%%%%%%%%%%%%%%%%%%%%%%%%%%%%%%%%%%%%%%%%
\paragraph{v1.5:} 2017/05/21

\begin{itemize}
\item
more complete structuring introduced
\item
|\childdocof| introduced
\item
|\childdoc| renamed to |\childdocmain|
\item
|\childredirect| renamed to |\childdocforward| and |\childdocforwardprefix|
and functionality expanded
\end{itemize}

%%%%%%%%%%%%%%%%%%%%%%%%%%%%%%%%%%%%%%%%
\paragraph{v1.0:} 2017/04/27

\begin{itemize}
\item
manual and install package
\item
first version published on CTAN
\end{itemize}

%%%%%%%%%%%%%%%%%%%%%%%%%%%%%%%%%%%%%%%%
\paragraph{v0.6:} 2017/04/26

\begin{itemize}
\item
redirection mechanism added
\end{itemize}

%%%%%%%%%%%%%%%%%%%%%%%%%%%%%%%%%%%%%%%%
\paragraph{v0.5:} 2017/04/26

\begin{itemize}
\item
functionality in definition file
\end{itemize}


%%%%%%%%%%%%%%%%%%%%%%%%%%%%%%%%%%%%%%%%%%%%%%%%%%%%%%%%%%%%%%%%%%%%%%%%%%%%%%%%
%%%%%%%%%%%%%%%%%%%%%%%%%%%%%%%%%%%%%%%%%%%%%%%%%%%%%%%%%%%%%%%%%%%%%%%%%%%%%%%%
%%%%%%%%%%%%%%%%%%%%%%%%%%%%%%%%%%%%%%%%%%%%%%%%%%%%%%%%%%%%%%%%%%%%%%%%%%%%%%%%
\appendix

\settowidth\MacroIndent{\rmfamily\scriptsize 000\ }

 \DocInput{childdoc.dtx}

\end{document}
%</driver>
% \fi
%
% %%%%%%%%%%%%%%%%%%%%%%%%%%%%%%%%%%%%%%%%%%%%%%%%%%%%%%%%%%%%%%%%%%%%%%%%%%%%%%
% %%%%%%%%%%%%%%%%%%%%%%%%%%%%%%%%%%%%%%%%%%%%%%%%%%%%%%%%%%%%%%%%%%%%%%%%%%%%%%
% \section{Sample}
%\iffalse
%<*samplemain>
%\fi
%
% The following presents a sample document
% with two chapters, two parts, a title page,
% a compile flag as well as three forwarding files to set the flag.
% It consists of eight |.tex| files:
% \begin{center}
% \begin{tabular}{ll}
% |cdocsamp.tex|&main file\\
% |cdocsch1.tex|&include file for chapter 1\\
% |cdocsch2.tex|&include file for chapter 2\\
% |cdocspt3.tex|&include file for part 3\\
% |cdocspt4.tex|&include file for part 4\\
% |cdocsdrf.tex|&forwarding file for main file in draft mode\\
% |cdocsfi1.tex|&forwarding file for final version of chapter 1\\
% |cdocsfi2.tex|&forwarding file for final version of chapter 2\\
% \end{tabular}
% \end{center}
% Each of the eight files can be compiled directly by the \LaTeX{} compiler.
%
% %%%%%%%%%%%%%%%%%%%%%%%%%%%%%%%%%%%%%%
% \paragraph{Main File.}
%
% The main file is called |cdocsamp.tex|.
%
% Load the \textsf{childdoc} definitions and
% declare the filename for the main document:
%    \begin{macrocode}
\input{childdoc.def}
\childdocmain{}
%    \end{macrocode}

% Optional override for |\version| flag:
%    \begin{macrocode}
%%\ifchilddoc\else\providecommand{\version}{draft}\fi
%    \end{macrocode}

% Define the default values for the |\version| flag
% (|final| for the main file and |draft| for childs):
%    \begin{macrocode}
\ifchilddoc
\providecommand{\version}{draft}
\else
\providecommand{\version}{final}
\fi
%    \end{macrocode}

% Load the standard document class:
%    \begin{macrocode}
\documentclass[12pt]{article}
%    \end{macrocode}

% Start the document body:
%    \begin{macrocode}
\begin{document}
%    \end{macrocode}

% Declare a title page.
% Print title, part of document being processed and version flag:
%    \begin{macrocode}
\addtocounter{page}{-1}
\begin{center}
{\LARGE\bfseries{}childdoc example\par}
\vspace{1cm}
\ifchilddoc
\ifchilddocmanual part\else chapter\fi:
`\childdocname' of `\childdocjob'\par
\else
main document: `\childdocjob'\par
\fi
version: \version\par
\end{center}
\newpage
%    \end{macrocode}

% Manually include selected file,
% otherwise process as usual:
%    \begin{macrocode}
\ifchilddocmanual
\section*{part `\childdocname'}
\input{\childdocname}
\else
%    \end{macrocode}

% Include the two chapters:
%    \begin{macrocode}
\include{cdocsch1}
\include{cdocsch2}
%    \end{macrocode}

% Include the two parts unless only chapters should be displayed:
%    \begin{macrocode}
\ifchilddoc\else
\section{part three}
\input{cdocspt3}
\section{part four}
\input{cdocspt4}
\fi
%    \end{macrocode}

% Process as usual until here:
%    \begin{macrocode}
\fi
%    \end{macrocode}

% End of document body:
%    \begin{macrocode}
\end{document}
%    \end{macrocode}
%\iffalse
%</samplemain>
%\fi
%
% %%%%%%%%%%%%%%%%%%%%%%%%%%%%%%%%%%%%%%
% \paragraph{Chapter Include Files.}
%
% The include files are called |cdocsch1.tex| and |cdocsch2.tex|.
%
%\iffalse
%<*samplechap1|samplechap2>
%\fi

% Optional override for |\version| flag:
%    \begin{macrocode}
%%\providecommand{\version}{final}
%    \end{macrocode}

% Include the main document:
%    \begin{macrocode}
\input{childdoc.def}
\childdocof{cdocsamp}
%    \end{macrocode}

%\iffalse
%</samplechap1|samplechap2>
%\fi
%
%\iffalse
%<*samplechap1>
%\fi
% Some text for chapter 1:
%    \begin{macrocode}
\section{one}
some text in chapter one
%    \end{macrocode}

%\iffalse
%</samplechap1>
%\fi
% Some text for chapter 2:
%\iffalse
%<*samplechap2>
%\fi
%    \begin{macrocode}
\section{two}
more text in chapter two
%    \end{macrocode}

%\iffalse
%</samplechap2>
%\fi
%
% %%%%%%%%%%%%%%%%%%%%%%%%%%%%%%%%%%%%%%
% \paragraph{Part Include Files.}
%
% The include files are called |cdocspt3.tex| and |cdocspt4.tex|.
%
%\iffalse
%<*samplepart3|samplepart4>
%\fi

% Optional override for |\version| flag:
%    \begin{macrocode}
%%\providecommand{\version}{final}
%    \end{macrocode}

% Include the main document:
%    \begin{macrocode}
\input{childdoc.def}
\childdocby{cdocsamp}
%    \end{macrocode}

%\iffalse
%</samplepart3|samplepart4>
%\fi
%
%\iffalse
%<*samplepart3>
%\fi
% Some text for part 3:
%    \begin{macrocode}
some text in part three
%    \end{macrocode}

%\iffalse
%</samplepart3>
%\fi
% Some text for part 4:
%\iffalse
%<*samplepart4>
%\fi
%    \begin{macrocode}
more text in part four
%    \end{macrocode}

%\iffalse
%</samplepart4>
%\fi
%
% %%%%%%%%%%%%%%%%%%%%%%%%%%%%%%%%%%%%%%
% \paragraph{Forwarding for a Complete Draft.}
%
% The following forwarding file |cdocsdrf.tex|
% compiles the main document in draft mode:
%\iffalse
%<*sampledraft>
%\fi
%    \begin{macrocode}
\def\version{draft}
\input{childdoc.def}
\childdocforward{cdocsamp}
%    \end{macrocode}

%\iffalse
%</sampledraft>
%\fi
%
% %%%%%%%%%%%%%%%%%%%%%%%%%%%%%%%%%%%%%%
% \paragraph{Forwarding for Final Version of the Chapters.}
%
% The following forwarding files |cdocsfn1.tex| and |cdocsfn2.tex|
% (with identical content)
% compile the final versions of the child documents
% |cdocsch1.tex| and |cdocsch2.tex|, respectively:
%\iffalse
%<*samplefinal>
%\fi
%    \begin{macrocode}
\def\version{final}
\input{childdoc.def}
\childdocforwardprefix[cdocsamp]{cdocsfn}{cdocsch}
%    \end{macrocode}

%\iffalse
%</samplefinal>
%\fi
%
% %%%%%%%%%%%%%%%%%%%%%%%%%%%%%%%%%%%%%%
% \paragraph{Command Line Processing.}
%
% The following three command lines generate the output files
% |cdocscld|, |cdocscl1| and |cdocscl2|
% which should be identical to
% |cdocsdrf|, |cdocsch1| and |cdocsfn2|, respectively:
% \begin{center}
% \begin{tabular}{l}
% |latex -jobname cdocscld \|\\
% |  "\def\version{draft}\input{childdoc.def}\childdocforward{cdocsamp}"|\\
% |latex -jobname cdocscl1 \|\\
% |  "\input{childdoc.def}\childdocforward[cdocsamp]{cdocsch1}"|\\
% |latex -jobname cdocscl2 \|\\
% |  "\def\version{final}\input{childdoc.def}\childdocforward{cdocsch2}"|
% \end{tabular}
% \end{center}
% Note that the trailing backslash on each first line
% merely continues the input to the second line
% (for convenient cut ant paste).
% Furthermore, the command |latex| can be replaced by any
% of its alternative versions such as |pdflatex|.
%
% %%%%%%%%%%%%%%%%%%%%%%%%%%%%%%%%%%%%%%%%%%%%%%%%%%%%%%%%%%%%%%%%%%%%%%%%%%%%%%
% %%%%%%%%%%%%%%%%%%%%%%%%%%%%%%%%%%%%%%%%%%%%%%%%%%%%%%%%%%%%%%%%%%%%%%%%%%%%%%
% \section{Implementation}
%\iffalse
%<*package>
%\fi
%
% This section describes the definitions file |childdoc.def|.

% The definitions cannot be loaded using |\usepackage| or |\RequirePackage|
% which has a mechanism to prevent loading a style file more than once.
% When loading the definitions by means of |\input|
% multiple instances have to be prevented manually:
%\iffalse
%This code needs to be before the `\ProvidesFile' directive
%which is defined at the beginning of this file.
%Therefore it is also placed there and commented out here.
%</package>
%<*discard>
%\fi
%    \begin{macrocode}
\ifdefined\childdocmain\endinput\fi
%    \end{macrocode}
%\iffalse
%</discard>
%<*package>
%\fi
%
% \macro{\ifchilddoc}
% \macro{\ifchilddocmanual}
% The conditional |\ifchilddoc| tells whether a
% child (true) or main (false) document is being compiled.
% The conditional |\ifchilddocmanual| tells whether
% the |\includeonly| mechanism is used (false) or
% the selection of child files must be performed manually (true).
% The definitions initialise to false:
%    \begin{macrocode}
\newif\ifchilddoc
\newif\ifchilddocmanual
%    \end{macrocode}

% \macro{\childdocname}
% \macro{\childdocjob}
% The macro |\childdocname| stores the name of the main document
% to be compiled. The macro |\childdocjob| stores the name of
% the document on which the \LaTeX{} compiler was originally invoked.
% The content of |\jobname| cannot be compared
% to filenames specified in the source due to different catcodes.
% The following code rescans |\jobname|, stores the result
% in |\childdocname| and saves a copy in |\childdocjob|:
%    \begin{macrocode}
\edef\childdocname{\scantokens\expandafter{\jobname\noexpand}}
\let\childdocjob\childdocname
%    \end{macrocode}

% \macro{\childdocdisable}
% The macro |\childdocdisable| prevents the main file
% from being processed more than once.
% At this stage, the main document command |\childdocmain|
% is assumed to be called once again where it should do nothing.
% Any subsequent call to it should prevent
% a secondary processing of the main document
% It overwrites the forwarding commands
% |\childdocof| and |\childdocforward|
% with empty macros to prevent further inclusions of the main document:
%    \begin{macrocode}
\newcommand{\childdocdisable}
{
  \renewcommand{\childdocmain}[1]{\renewcommand{\childdocmain}[1]{\endinput}}
  \renewcommand{\childdocof}[1]{}
  \renewcommand{\childdocby}[2][]{}
  \renewcommand{\childdocforward}[2][]{}
  \renewcommand{\childdocdisable}{}
}
%    \end{macrocode}

% \macro{\childdocmain}
% The macro |\childdocmain| is to be called at the top of the main file
% with nothing or the main filename (without extension) as argument.
% First, it breaks loops.
% If the argument is not empty and does not match |\childdocname|
% (which is set by the first inclusion of |childdoc.def|),
% |\ifchilddoc| is set to true, |\includeonly| is applied to the child file
% and |\jobname| is set to the main file
% (for proper handling of |.aux| files):
%    \begin{macrocode}
\newcommand{\childdocmain}[1]
{
  \childdocdisable\childdocmain{}
  \if?#1?\else
    \begingroup
      \def\childdoctmp{#1}
      \ifx\childdoctmp\childdocname
        \def\childdoctmp{}
      \else
        \def\childdoctmp
        {
          \childdoctrue
          \includeonly{\childdocname}
          \def\childdocjob{#1}
          \def\jobname{#1}
        }
      \fi
      \expandafter
    \endgroup
    \childdoctmp
  \fi
}
%    \end{macrocode}

% \macro{\childdocof}
% The command |\childdocof| redirects
% compilation to the main file |#1|.
%    \begin{macrocode}
\newcommand{\childdocof}[1]
{
  \childdocdisable
  \childdoctrue
  \includeonly{\childdocname}
  \def\jobname{#1}
  \def\childdocjob{#1}
  \input{#1}
}
%    \end{macrocode}

% \macro{\childdocby}
% The command |\childdocby| ....
%    \begin{macrocode}
\newcommand{\childdocby}[2][]
{
  \childdocdisable
  \childdoctrue
  \childdocmanualtrue
  \if?#1?\else
    \def\jobname{#2}
  \fi
  \def\childdocjob{#2}
  \input{#2}
  \endinput
}
%    \end{macrocode}

% \macro{\childdocforward}
% The command |\childdocforward| redirects
% compilation to the main file or
% (if the optional argument is given) a child file.
% Parameters are set as if the main file
% or a child file starting with |\childdocof| was compiled.
% Then compilation is handed over to the main file:
%    \begin{macrocode}
\newcommand{\childdocforward}[2][]
{
  \begingroup
    \if?#1?
      \def\childdoctmp
      {
        \def\childdocname{#2}
        \def\childdocjob{#2}
        \def\jobname{#2}
        \input{#2}
        \endinput
      }
    \else
      \def\childdoctmp
      {
        \childdocdisable
        \def\childdocname{#2}
        \childdoctrue
        \includeonly{#2}
        \def\childdocjob{#1}
        \def\jobname{#1}
        \input{#1}
        \endinput
      }
    \fi
    \expandafter
  \endgroup
  \childdoctmp
}
%    \end{macrocode}

% \macro{\childdocforwardprefix}
% The command |\childdocforwardprefix| redirects
% compilation to the main or a child file by means of a pattern.
% The prefix |#1| in the current filename is replaced by |#2|
% and the suffix of the current filename is kept
% (it is assumed that the filename does not contain the substring `|~~~|'
% which is used as a delimiter).
% Compilation is handed over to the new file by |\childdocforward|:
%    \begin{macrocode}
\newcommand{\childdocforwardprefix}[3][]
{
  \begingroup
    \def\childdocextract #2##1~~~{\def\childdoctmp{\childdocforward[#1]{#3##1}}}
    \expandafter\childdocextract\childdocname~~~
    \expandafter
  \endgroup
  \childdoctmp
}
%    \end{macrocode}

% \macro{\childdoc}
% The deprecated macro |\childdoc| is a legacy version of |\childdocmain|:
%    \begin{macrocode}
\newcommand{\childdoc}{\childdocmain}
%    \end{macrocode}

% \macro{\childdocredirect}
% The deprecated macro |\childdocredirect| is a legacy version
% of |\childdocforward| and |\childdocforwardprefix|:
%    \begin{macrocode}
\newcommand{\childdocredirect}[2][]
{
  \begingroup
    \if?#1?
      \def\childdoctmp{\childdocforward{#2}}
    \else
      \def\childdoctmp{\childdocforwardprefix{#1}{#2}}
    \fi
    \expandafter
  \endgroup
  \childdoctmp
}
%    \end{macrocode}

%\iffalse
%</package>
%\fi
%
\endinput
|
and perform the replacements as outlined below.
Instead of |\childdocmain{|\textit{main}|}| add the following code
to the top of the main file:
%
\begin{center}
\begin{tabular}{l}
|\||ifdefined\childdocname\endinput\||fi\newif\ifchilddoc|\\
|\edef\childdocname{\scantokens\expandafter{\jobname\noexpand}}|\\
|\def\childdocmain{|\textit{main}|}\||ifx\childdocmain\childdocname\||else|\\
|\childdoctrue\includeonly{\childdocname}\let\jobname\childdocmain\||fi|\\
\end{tabular}
\end{center}
%
Instead of |\childdocof{|\textit{main}|}| just include the main file
at the top of each child file:
%
\begin{center}
|\input{|\textit{main}|}|
\end{center}
%
A simple redirection |\childdocforward{|\textit{dest}|}| is achieved by:
%
\begin{center}
|\def\jobname{|\textit{dest}|}\input{\jobname}|
\end{center}
%
The redirection with prefix
|\childdocforwardprefix[|\textit{prefix}|]{|\textit{dest}|}|
is accomplished by:
%
\begin{center}
\begin{tabular}{l}
|{\edef\jobname{\scantokens\expandafter{\jobname\noexpand}}|\\
|\def\redirectjob |\textit{prefix}|#1~~~{\gdef\jobname{|\textit{dest}|#1}}|\\
|\expandafter\redirectjob\jobname~~~}\input{\jobname}|
\end{tabular}
\end{center}

In an alternative approach,
child documents can be compiled by a specific command line
without additional code or specific definitions:
%
\begin{center}
|... -jobname "|\textit{target}|" "|[\textit{flags}]%
|\includeonly{|\textit{dest}|}\input{|\textit{main}|}"|
\end{center}
%

%%%%%%%%%%%%%%%%%%%%%%%%%%%%%%%%%%%%%%%%%%%%%%%%%%%%%%%%%%%%%%%%%%%%%%%%%%%%%%%%
%%%%%%%%%%%%%%%%%%%%%%%%%%%%%%%%%%%%%%%%%%%%%%%%%%%%%%%%%%%%%%%%%%%%%%%%%%%%%%%%
\section{Information}

%%%%%%%%%%%%%%%%%%%%%%%%%%%%%%%%%%%%%%%%%%%%%%%%%%%%%%%%%%%%%%%%%%%%%%%%%%%%%%%%
\subsection{Copyright}

Copyright \copyright{} 2017--2018 Niklas Beisert

This work may be distributed and/or modified under the
conditions of the \LaTeX{} Project Public License, either version 1.3
of this license or (at your option) any later version.
The latest version of this license is in
  \url{http://www.latex-project.org/lppl.txt}
and version 1.3 or later is part of all distributions of \LaTeX{}
version 2005/12/01 or later.

This work has the LPPL maintenance status `maintained'.

The Current Maintainer of this work is Niklas Beisert.

This work consists of the files |README.txt|, |childdoc.ins| and |childdoc.dtx|
as well as the derived files |childdoc.def|, |cdocsamp.tex|
with |cdocsch1.tex|, |cdocsch2.tex|, |cdocspt3.tex|, |cdocspt4.tex|,
|cdocsdrf.tex|, |cdocsfn1.tex|, |cdocsfn2.tex|
as well as |childdoc.pdf|.

%%%%%%%%%%%%%%%%%%%%%%%%%%%%%%%%%%%%%%%%%%%%%%%%%%%%%%%%%%%%%%%%%%%%%%%%%%%%%%%%
\subsection{Files and Installation}

The package consists of the files:
%
\begin{center}
\begin{tabular}{ll}
    |README.txt|   & readme file \\
    |childdoc.ins| & installation file \\
    |childdoc.dtx| & source file \\
    |childdoc.def| & definition file \\
    |cdocsamp.tex| & sample main file \\
    |cdocsch1.tex| & sample include file \\
    |cdocsch2.tex| & sample include file \\
    |cdocspt3.tex| & sample part file \\
    |cdocspt4.tex| & sample part file \\
    |cdocsdrf.tex| & sample redirection file \\
    |cdocsfn1.tex| & sample redirection file \\
    |cdocsfn2.tex| & sample redirection file \\
    |childdoc.pdf| & manual
\end{tabular}
\end{center}
%
The distribution consists of the files
|README.txt|, |childdoc.ins| and |childdoc.dtx|.
%
\begin{itemize}
\item
Run (pdf)\LaTeX{} on |childdoc.dtx|
to compile the manual |childdoc.pdf| (this file).
\item
Run \LaTeX{} on |childdoc.ins| to create the definitions file |childdoc.def|
and the sample |cdocsamp.tex| with include files
|cdocsch1.tex|, |cdocsch2.tex|, |cdocspt3.tex|, |cdocspt4.tex|,
|cdocsdrf.tex|, |cdocsfn1.tex|, |cdocsfn2.tex|.
Then copy the file |childdoc.def| to an appropriate directory of your \LaTeX{}
distribution, e.g.\ \textit{texmf-root}|/tex/latex/childdoc|.
\end{itemize}

%%%%%%%%%%%%%%%%%%%%%%%%%%%%%%%%%%%%%%%%%%%%%%%%%%%%%%%%%%%%%%%%%%%%%%%%%%%%%%%%
\subsection{Related CTAN Packages}

There are several other packages which offer a similar functionality:
%
\begin{itemize}
\item
The packages
\href{http://ctan.org/pkg/docmute}{\textsf{docmute}},
\href{http://ctan.org/pkg/includex}{\textsf{includex}} and
\href{http://ctan.org/pkg/standalone}{\textsf{standalone}}
provide commands to include only the document body of
a child file thus allowing both files to be compiled individually.
\item
The packages \href{http://ctan.org/pkg/subdocs}{\textsf{subdocs}}
and \href{http://ctan.org/pkg/subfiles}{\textsf{subfiles}}
provide structures in which the main and child documents can be
encapsulated and allowing them to be compiled individually.
The inclusion mechanism is different from the conventional |\include|.
\item
The package \href{http://ctan.org/pkg/combine}{\textsf{combine}}
is an elaborate solution to combine several documents into one.
\end{itemize}
%
See also the CTAN topic \href{http://ctan.org/topic/subdocs}{\textsf{subdocs}}
for further related packages.
The present package differs from the above solutions in that
a document structure constructed with the conventional |\include| mechanism
just needs two extra commands at the top of every file
such that all constituent files can be compiled individually.

%%%%%%%%%%%%%%%%%%%%%%%%%%%%%%%%%%%%%%%%%%%%%%%%%%%%%%%%%%%%%%%%%%%%%%%%%%%%%%%%
%\subsection{Feature Suggestions}
%
%The following is a list of features which may be useful for future
%versions of this package:
%%
%\begin{itemize}
%\item
%\ldots
%\end{itemize}

%%%%%%%%%%%%%%%%%%%%%%%%%%%%%%%%%%%%%%%%%%%%%%%%%%%%%%%%%%%%%%%%%%%%%%%%%%%%%%%%
\subsection{Revision History}

%%%%%%%%%%%%%%%%%%%%%%%%%%%%%%%%%%%%%%%%
\paragraph{v2.0:} 2018/12/30

\begin{itemize}
\item
immediate forward processing
\item
added |\childdocby| mechanism
\item
manual restructured
\end{itemize}

%%%%%%%%%%%%%%%%%%%%%%%%%%%%%%%%%%%%%%%%
\paragraph{v1.6:} 2018/01/17

\begin{itemize}
\item
application for development of include files
\item
corrections to manual
\end{itemize}

%%%%%%%%%%%%%%%%%%%%%%%%%%%%%%%%%%%%%%%%
\paragraph{v1.5:} 2017/05/21

\begin{itemize}
\item
more complete structuring introduced
\item
|\childdocof| introduced
\item
|\childdoc| renamed to |\childdocmain|
\item
|\childredirect| renamed to |\childdocforward| and |\childdocforwardprefix|
and functionality expanded
\end{itemize}

%%%%%%%%%%%%%%%%%%%%%%%%%%%%%%%%%%%%%%%%
\paragraph{v1.0:} 2017/04/27

\begin{itemize}
\item
manual and install package
\item
first version published on CTAN
\end{itemize}

%%%%%%%%%%%%%%%%%%%%%%%%%%%%%%%%%%%%%%%%
\paragraph{v0.6:} 2017/04/26

\begin{itemize}
\item
redirection mechanism added
\end{itemize}

%%%%%%%%%%%%%%%%%%%%%%%%%%%%%%%%%%%%%%%%
\paragraph{v0.5:} 2017/04/26

\begin{itemize}
\item
functionality in definition file
\end{itemize}


%%%%%%%%%%%%%%%%%%%%%%%%%%%%%%%%%%%%%%%%%%%%%%%%%%%%%%%%%%%%%%%%%%%%%%%%%%%%%%%%
%%%%%%%%%%%%%%%%%%%%%%%%%%%%%%%%%%%%%%%%%%%%%%%%%%%%%%%%%%%%%%%%%%%%%%%%%%%%%%%%
%%%%%%%%%%%%%%%%%%%%%%%%%%%%%%%%%%%%%%%%%%%%%%%%%%%%%%%%%%%%%%%%%%%%%%%%%%%%%%%%
\appendix

\settowidth\MacroIndent{\rmfamily\scriptsize 000\ }

 \DocInput{childdoc.dtx}

\end{document}
%</driver>
% \fi
%
% %%%%%%%%%%%%%%%%%%%%%%%%%%%%%%%%%%%%%%%%%%%%%%%%%%%%%%%%%%%%%%%%%%%%%%%%%%%%%%
% %%%%%%%%%%%%%%%%%%%%%%%%%%%%%%%%%%%%%%%%%%%%%%%%%%%%%%%%%%%%%%%%%%%%%%%%%%%%%%
% \section{Sample}
%\iffalse
%<*samplemain>
%\fi
%
% The following presents a sample document
% with two chapters, two parts, a title page,
% a compile flag as well as three forwarding files to set the flag.
% It consists of eight |.tex| files:
% \begin{center}
% \begin{tabular}{ll}
% |cdocsamp.tex|&main file\\
% |cdocsch1.tex|&include file for chapter 1\\
% |cdocsch2.tex|&include file for chapter 2\\
% |cdocspt3.tex|&include file for part 3\\
% |cdocspt4.tex|&include file for part 4\\
% |cdocsdrf.tex|&forwarding file for main file in draft mode\\
% |cdocsfi1.tex|&forwarding file for final version of chapter 1\\
% |cdocsfi2.tex|&forwarding file for final version of chapter 2\\
% \end{tabular}
% \end{center}
% Each of the eight files can be compiled directly by the \LaTeX{} compiler.
%
% %%%%%%%%%%%%%%%%%%%%%%%%%%%%%%%%%%%%%%
% \paragraph{Main File.}
%
% The main file is called |cdocsamp.tex|.
%
% Load the \textsf{childdoc} definitions and
% declare the filename for the main document:
%    \begin{macrocode}
% \iffalse
%
% childdoc.dtx Copyright (C) 2017-2018 Niklas Beisert
%
% This work may be distributed and/or modified under the
% conditions of the LaTeX Project Public License, either version 1.3
% of this license or (at your option) any later version.
% The latest version of this license is in
%   http://www.latex-project.org/lppl.txt
% and version 1.3 or later is part of all distributions of LaTeX
% version 2005/12/01 or later.
%
% This work has the LPPL maintenance status `maintained'.
%
% The Current Maintainer of this work is Niklas Beisert.
%
% This work consists of the files childdoc.dtx and childdoc.ins
% and the derived files childdoc.def and cdocsamp.tex with
% cdocsch1.tex, cdocsch2.tex, cdocsdrf.tex, cdocsfn1.tex, cdocsfn2.tex.
%
%<package>\ifdefined\childdocmain\endinput\fi
%<package>\ProvidesFile{childdoc.def}[2018/12/30 v2.0 child document driver]
%<samplemain>\ProvidesFile{cdocsamp.tex}[2018/12/30 v2.0 sample for childdoc]
%<*driver>
%\ProvidesFile{childdoc.drv}[2018/12/30 v2.0 childdoc reference manual file]
\PassOptionsToClass{10pt,a4paper}{article}
\documentclass{ltxdoc}

\usepackage[margin=35mm]{geometry}
\usepackage{hyperref}
\usepackage{hyperxmp}
\usepackage[usenames]{color}

\hypersetup{colorlinks=true}
\hypersetup{pdfstartview=FitH}
\hypersetup{pdfpagemode=UseNone}
\hypersetup{pdfsource={}}
\hypersetup{pdflang={en-UK}}
\hypersetup{pdfcopyright={Copyright 2017-2018 Niklas Beisert.
  This work may be distributed and/or modified under the
  conditions of the LaTeX Project Public License, either version 1.3
  of this license or (at your option) any later version.}}
\hypersetup{pdflicenseurl={http://www.latex-project.org/lppl.txt}}
\hypersetup{pdfcontactaddress={ETH Zurich, ITP, HIT K,
  Wolfgang-Pauli-Strasse 27}}
\hypersetup{pdfcontactpostcode={8093}}
\hypersetup{pdfcontactcity={Zurich}}
\hypersetup{pdfcontactcountry={Switzerland}}
\hypersetup{pdfcontactemail={nbeisert@itp.phys.ethz.ch}}
\hypersetup{pdfcontacturl={http://people.phys.ethz.ch/\xmptilde nbeisert/}}

\newcommand{\secref}[1]{\hyperref[#1]{section \ref*{#1}}}

\parskip1ex
\parindent0pt
\let\olditemize\itemize
\def\itemize{\olditemize\parskip0pt}

\begin{document}

\title{The \textsf{childdoc} Package}
\hypersetup{pdftitle={The childdoc Package}}
\author{Niklas Beisert\\[2ex]
  Institut f\"ur Theoretische Physik\\
  Eidgen\"ossische Technische Hochschule Z\"urich\\
  Wolfgang-Pauli-Strasse 27, 8093 Z\"urich, Switzerland\\[1ex]
  \href{mailto:nbeisert@itp.phys.ethz.ch}
  {\texttt{nbeisert@itp.phys.ethz.ch}}}
\hypersetup{pdfauthor={Niklas Beisert}}
\hypersetup{pdfsubject={Manual for the LaTeX2e Package childdoc}}
\date{30 December 2018, \textsf{v2.0}}
\maketitle

\begin{abstract}\noindent
\textsf{childdoc} is a \LaTeXe{} package
that enables the direct compilation
of document sections included by |\include|
to individual files.
\end{abstract}

\begingroup
\parskip0ex
\tableofcontents
\endgroup

%%%%%%%%%%%%%%%%%%%%%%%%%%%%%%%%%%%%%%%%%%%%%%%%%%%%%%%%%%%%%%%%%%%%%%%%%%%%%%%%
%%%%%%%%%%%%%%%%%%%%%%%%%%%%%%%%%%%%%%%%%%%%%%%%%%%%%%%%%%%%%%%%%%%%%%%%%%%%%%%%
\section{Introduction}

\LaTeX{} provides a mechanism to structure a large document (such as a book)
into a main file and several child files (containing the chapters)
using the |\include| command.
This mechanism is beneficial for documents
which span hundreds of pages in order to
make the source file(s) more manageable.
Moreover, compilation can be restricted to
selected child files by means of the |\includeonly| command.
The latter feature can be used to reduce the compilation time while editing
(this was significantly more useful in the earlier days of \LaTeX{})
or to generate a smaller document which is easier to navigate.
Another application of |\includeonly| is to generate
documents consisting of selected parts of the complete document.

However, there are a few drawbacks of the plain |\include| mechanism:
\begin{itemize}
\item
The child files cannot be compiled on their own,
they can only be compiled via the main file.
A naive editing environment
(such as a text editor with an option
to have the current file processed by \LaTeX)
may require one to switch to the main file before compiling;
attempting to compile the child file produces errors.
\item
The main file must be modified (each time)
to adjust the |\includeonly| command
to the present needs. This easily leaves the main file in a messy state.
\item
The generated document will always carry the filename
of the main document. This is inconvenient if
several child files are to be compiled and
to be kept for distribution.
\end{itemize}

The present package provides a simple interface
to make child files individually compilable by \LaTeX{}.
Compiling a child file then has the same effect as compiling
the main file with an |\includeonly| command
to select the appropriate child.
Moreover the generated document will carry the name of the child
rather than the main file.
This resolves all three above issues.

This feature is meant to make the editing of books,
thesis documents and lecture notes somewhat more convenient.
However, the package can also be used efficiently for
composing a series of documents (such as exercise sheets)
which are typically distributed individually.
It then assists the author in generating the individual documents
(potentially in different versions)
as well as a document containing the collected series.
Another application is in developing style files
or other kinds of included material
where compilation of the style file could redirect
to a sample or test file.

%%%%%%%%%%%%%%%%%%%%%%%%%%%%%%%%%%%%%%%%%%%%%%%%%%%%%%%%%%%%%%%%%%%%%%%%%%%%%%%%
%%%%%%%%%%%%%%%%%%%%%%%%%%%%%%%%%%%%%%%%%%%%%%%%%%%%%%%%%%%%%%%%%%%%%%%%%%%%%%%%
\section{Usage}

First of all, the package \textsf{childdoc} is \emph{not} a standard
\LaTeXe{} |.sty| style file! Therefore it needs to be invoked in
a non-standard way.

%%%%%%%%%%%%%%%%%%%%%%%%%%%%%%%%%%%%%%%%%%%%%%%%%%%%%%%%%%%%%%%%%%%%%%%%%%%%%%%%
\subsection{Included Files}
\label{sec:include}

%%%%%%%%%%%%%%%%%%%%%%%%%%%%%%%%%%%%%%%%
\DescribeMacro{\childdocmain}
To use the package, add the commands
\begin{center}
\begin{tabular}{l}
|\input{childdoc.def}|\\
|\childdocmain{}|\\
\end{tabular}
\end{center}
at the very top of the main \LaTeX{} file,
in particular \emph{before} the |\documentclass| statement!
The argument of |\childdocmain| should be left empty
(but it must be present).

%%%%%%%%%%%%%%%%%%%%%%%%%%%%%%%%%%%%%%%%
\DescribeMacro{\childdocof}
Furthermore, add the commands
\begin{center}
\begin{tabular}{l}
|\input{childdoc.def}|\\
|\childdocof{|\textit{main}|}|\\
\end{tabular}
\end{center}
at the top of every child file \textit{child}
which is included by |\include{|\textit{child}|}|
from within the main file
(or at least for those files to be compiled individually).
The argument \textit{main} must be the filename of the main file.

There are a couple of
considerations in setting up the main and child documents:

%%%%%%%%%%%%%%%%%%%%%%%%%%%%%%%%%%%%%%%%
\paragraph{Restrictions.}

Please note the following restrictions:
\begin{itemize}
\item
|\childdocmain| must be called with one argument \textit{main}
to ensure compatibility with earlier version of the package.
It must either be empty (|\childdocmain{}|)
or precisely match the filename of the main file in which it is specified.
See \secref{sec:detection} for further information.
\item
The filename \textit{main} must be specified without the |.tex| extension.
\item
The filename \textit{main} is case sensitive
(even in case-insensitive file systems)
due to internal string comparison.
\item
The argument \textit{main} should be fully expanded, it cannot be a macro.
\item
Subdirectories and special characters should be avoided in filenames.
\item
The command |\childdocmain{|\textit{main}|}| must be followed by a whitespace.
It should not be followed immediately by another command
or by a comment mark `|%|'.
This is because the \TeX{} parser reads the token immediately following
the argument of |\childdocmain| and puts it
at the beginning of every child section;
however, a white\-space is ignored.
\end{itemize}

%%%%%%%%%%%%%%%%%%%%%%%%%%%%%%%%%%%%%%%%
\paragraph{Content of Main File.}

It is advisable to place all content in the child files included by |\include|.
Any output contained in the main file will appear in all child documents
unless suppressed manually;
it cannot be suppressed automatically by the |\includeonly| directive
and thus should normally be avoided.
A method to include some content in the main file
by means of conditional processing is described in \secref{sec:conditional}.

%%%%%%%%%%%%%%%%%%%%%%%%%%%%%%%%%%%%%%%%
\paragraph{Page Numbering.}

When only a part of the document is compiled,
the appropriate numbering of pages
(as well as other status parameters)
is determined from the |.aux| files.
The latter contain information from previous passes.
However this information needs to propagate through
all intermediate child documents.
Therefore the page numbering in child documents may well
be inconsistent until the complete document is compiled at least once.

A useful (if unconventional) way to always ensure a consistent
page numbering is to restart the numbering in each child document
and denote the pages by `\textit{child}|.|\textit{page}'
where \textit{child} represents the chapter/section number of the child file.
This can be achieved by the command
|\numberwithin{page}{|\textit{child}|}|
of the \textsf{amsmath} package
where \textit{child} can be |chapter| or |section|
depending on the chosen structuring.
Alternatively, one can modify the macro |\thepage| appropriately
and reset the counter |page| at the start of each child file.

%%%%%%%%%%%%%%%%%%%%%%%%%%%%%%%%%%%%%%%%%%%%%%%%%%%%%%%%%%%%%%%%%%%%%%%%%%%%%%%%
\subsection{Conditional Processing}
\label{sec:conditional}

The package provides a mechanism to compile different versions
of a document. To customise the versions further some conditional processing
can come in handy to distinguish which version is being compiled.
The package provides two macros to describe the compilation context:

%%%%%%%%%%%%%%%%%%%%%%%%%%%%%%%%%%%%%%%%
\DescribeMacro{\ifchilddoc}
The conditional |\ifchilddoc| distinguishes between the compilation of
child documents and the main document:
%
\begin{center}
|\ifchilddoc |\textit{child-code}| |[|\||else |\textit{main-code}]| \||fi|
\end{center}

%%%%%%%%%%%%%%%%%%%%%%%%%%%%%%%%%%%%%%%%
\DescribeMacro{\childdocname}
\DescribeMacro{\childdocjob}
The macro |\childdocname| contains the filename (without extension)
of the main or child file being processed.
Note that |\childdocjob| will always contain the name of the main file.

%%%%%%%%%%%%%%%%%%%%%%%%%%%%%%%%%%%%%%%%
\paragraph{Title Page.}

Conditional processing can be used to include a title or banner page
in the main document when proper precautions are taken.
Importantly, the code in the main file should ensure that the page counter
(as well as other status parameters which are stored in the |.aux| files)
takes the same value after the conditional processing.
Otherwise the page numbers may take divergent values
depending on which part is compiled.

For example, a title page could be declared by:
%
\begin{center}
\begin{tabular}{l}
|\ifchilddoc\||else|\\
|\addtocounter{page}{-1}|\\
\textit{code for title page}\\
|\newpage|\\
|\||fi|
\end{tabular}
\end{center}
%
A banner page for the child documents can be generated by:
%
\begin{center}
\begin{tabular}{l}
|\ifchilddoc|\\
|\addtocounter{page}{-1}|\\
\textit{code for banner page}\\
|\newpage|\\
|\||fi|
\end{tabular}
\end{center}
%
Here one could write a message such as:
\begin{center}
|This is the part \childdocname{} of \childdocjob{}.|
\end{center}

%%%%%%%%%%%%%%%%%%%%%%%%%%%%%%%%%%%%%%%%%%%%%%%%%%%%%%%%%%%%%%%%%%%%%%%%%%%%%%%%
\subsection{Flags}
\label{sec:flags}

The package makes it easy to generate different versions
of the main or child documents.
To this end compilation flags can be defined
and assigned different default values.
They will be particularly useful in conjunction
with the forwarding mechanism described in \secref{sec:forward}.

For example, it may be useful to have a flag |\version|
which can be set to |draft| or |final|.
The document source will contain some conditional code
depending on the value of |\version|.
Suppose further, the flag should default to |final| for the main file
and to |draft| for child files
which is a natural assignment for editing the document.
This is achieved by placing the following code
in the preamble of the main document
(below the |\childdocmain| directive):
%
\begin{center}
\begin{tabular}{l}
|\ifchilddoc|\\
|\providecommand{\version}{draft}|\\
|\||else|\\
|\providecommand{\version}{final}|\\
|\||fi|
\end{tabular}
\end{center}
%
The definition by |\providecommand| makes sure
that previous definitions are not overwritten.
Further statements |\providecommand{\version}{...}|
can thus be added before the above code to override it.

For the main file, one might add a line
(between |\childdocmain| and the above block)
%
\begin{center}
|%\ifchilddoc\||else\providecommand{\version}{draft}\||fi|
\end{center}
%
which can be uncommented to produce a draft version.
Likewise one can add a line to the very top of a child file
(above the |\childdocof{|\textit{main}|}| directive)
%
\begin{center}
|%\providecommand{\version}{final}|
\end{center}
%
which can be uncommented to produce the final version of this child document.

%%%%%%%%%%%%%%%%%%%%%%%%%%%%%%%%%%%%%%%%%%%%%%%%%%%%%%%%%%%%%%%%%%%%%%%%%%%%%%%%
\subsection{Forwarding}
\label{sec:forward}

Different versions of the main or child documents
using compilation flags as described in \secref{sec:flags}
can be (permanently) stored in different files
for convenient compilation, viewing and distribution.
To this end, the package defines a command
to pass on compilation to a different file:

%%%%%%%%%%%%%%%%%%%%%%%%%%%%%%%%%%%%%%%%
\DescribeMacro{\childdocforward}
The command |\childdocforward| redirects processing to
another source file:
%
\begin{center}
\begin{tabular}{l}
|\input{childdoc.def}|\\
|\childdocforward[|\textit{main}|]{|\textit{dest}|}|\\
\end{tabular}
\end{center}
%
The argument \textit{dest} is the destination file
(without extension).
It should be the main file or one of the child files.
Note that further \textsf{childdoc} directives
such as |\childdocof| and |\childdocforward|
in the indicated file will be processed in this form.
The optional argument \textit{main}
passes on directly to the main file \textit{main}
while pretending to compile the child \textit{dest}.
This form behaves as if \textit{dest}
issues |\childdocof{|\textit{main}|}| right away,
and no further \textsf{childdoc} directives will be processed.

%%%%%%%%%%%%%%%%%%%%%%%%%%%%%%%%%%%%%%%%
\DescribeMacro{\...prefix}
In the alternative form |\childdocforwardprefix|,
%
\begin{center}
\begin{tabular}{l}
|\input{childdoc.def}|\\
|\childdocforwardprefix[|\textit{main}|]{|\textit{prefix}|}{|\textit{dest}|}|
\end{tabular}
\end{center}
%
the destination file is determined by a pattern
depending on the current file:
To make this work, the current file must be called
`{\textit{prefix}\hspace{0.2em}\textit{suffix}}'
with \textit{prefix} matching precisely the argument.
Processing is then passed on to the file
`{\textit{dest}\hspace{0.2em}\textit{suffix}}'.
Surely, the same effect is achieved by
directly specifying the
argument `{\textit{dest}\hspace{0.2em}\textit{suffix}}'
in the first form.
However, that requires to set up a different file
for each child. With the alternative form of the command
all these files can have exactly the same content
which simplifies setting them up and maintaining them.

For example, the following file |draft.tex|
with a compilation flag |\version| as described in \secref{sec:flags}
compiles the main document as a draft:
%
\begin{center}
\begin{tabular}{l}
|\def\version{draft}|\\
|\input{childdoc.def}|\\
|\childdocforward{|\textit{main}|}|
\end{tabular}
\end{center}
%
Likewise, the following files |final|\textit{nn}|.tex|
compile the final version of the child document
|child|\textit{nn}|.tex|:
%
\begin{center}
\begin{tabular}{l}
|\def\version{final}|\\
|\input{childdoc.def}|\\
|\childdocforwardprefix{final}{child}|
\end{tabular}
\end{center}
%

Note that when several versions of a main file and/or of each child file
are to be generated, it may be convenient to set up a |Makefile| or
shell script to automatise the process.

%%%%%%%%%%%%%%%%%%%%%%%%%%%%%%%%%%%%%%%%%%%%%%%%%%%%%%%%%%%%%%%%%%%%%%%%%%%%%%%%
\subsection{Command Line Processing}
\label{sec:commandline}

The effect of redirection files can also be achieved by invoking
the \LaTeX{} compiler with a more elaborate command line.
Most conveniently this should be done as part
of a shell script or a |Makefile|.

When using \textsf{childdoc} in the main file, the following
command lines effectively perform a redirection
(note that depending on the shell being used,
backslashes may have to be doubled: `|\|' $\to$ `|\\|'):
%
\begin{center}
|... -jobname "|\textit{target}|" |\\|"|[\textit{flags}]%
|\input{childdoc.def}\childdocforward[|\textit{main}|]{|\textit{dest}|}"|
\end{center}
%
Here \textit{target} is the name of the output file,
\textit{main} is the name of the main file
and \textit{dest} is the name of the main or child file to be processed
(all filenames without extensions).
The optional argument \textit{main} can be omitted
if \textit{main} matches \textit{dest}.
Optionally, compilation \textit{flags} can be defined via |\def| commands.
This command line makes the \TeX{} engine believe
it is compiling the file \textit{target}
whose content is specified as the latter parameter.
The provided code then forwards the processing to
\textit{main} or \textit{dest} as described in \secref{sec:forward}.

%%%%%%%%%%%%%%%%%%%%%%%%%%%%%%%%%%%%%%%%%%%%%%%%%%%%%%%%%%%%%%%%%%%%%%%%%%%%%%%%
\subsection{Include by Input}
\label{sec:input}

Including child documents by |\include| has some restrictions by design.
Most notably, the content of a child document always occupies
its own set of pages; pages cannot be shared between child documents.
Usually, this behaviour makes perfect sense
because each child document contain an essential part of the document.
However, in some situations it may be desirable to compose
a document from a collection of parts
without having mandatory page breaks between then.
For this case, the package
provides a mechanism to include parts
by |\input| which can also be processed individually.
However, by construction this mechanism
requires manual handling of the content to be output.

%%%%%%%%%%%%%%%%%%%%%%%%%%%%%%%%%%%%%%%%
\DescribeMacro{\ifchilddocmanual}
The main file should be prepared as usual, see \secref{sec:include}.
However, the document body must make a distinction
between processing of an individual part and of the main document, e.g.:
%
\begin{center}
\begin{tabular}{l}
|\ifchilddocmanual|\\
|\input{\childdocname}|\\
|\||else|\\
\textit{document body with }|\input{|\textit{part}|}|\\
|\||fi|
\end{tabular}
\end{center}
%
The conditional |\ifchilddocmanual| is true whenever
a part to be included by |\input| is being compiled,
and the name of the part is stored in |\childdocname|.

%%%%%%%%%%%%%%%%%%%%%%%%%%%%%%%%%%%%%%%%
\DescribeMacro{\childdocby}
Each part to be included by |\input| should start with:
%
\begin{center}
\begin{tabular}{l}
|\input{childdoc.def}|\\
|\childdocby{|\textit{main}|}|\\
\end{tabular}
\end{center}
%
The directive |\childdocby| is similar to |\childdocof|
described in \secref{sec:include},
but the subsequent selection of content must be done manually.
To that end, both |\ifchilddoc| and |\ifchilddocmanual|
will be true upon processing of a part,
and the name of the part is stored in |\childdocname|.
Note that |\jobname| will be set to the filename of the current part
so that each part receives an individual |.aux| file
that does not interfere with the |.aux| file(s) of the main document.
This behaviour can be altered by the alternative form
|\childdocby[*]{|\textit{main}|}| (with a non-empty optional argument)
which uses the |.aux| file of the main document
by setting |\jobname| to \textit{main}.

%%%%%%%%%%%%%%%%%%%%%%%%%%%%%%%%%%%%%%%%%%%%%%%%%%%%%%%%%%%%%%%%%%%%%%%%%%%%%%%%
\subsection{Driver Development}
\label{sec:driver}

The \textsf{childdoc} mechanism can also be use for the development
of definition files such as \LaTeX{} styles or classes.
This case differs from the above setup with multiple parts
included by |\include| in that no |\includeonly| should be invoked.
This can be achieved by starting the include file
(before |\ProvidesPackage|) with:
%
\begin{center}
\begin{tabular}{l}
|\input{childdoc.def}|\\
|\childdocforward{|\textit{main}|}|\\
\end{tabular}
\end{center}
%
or alternatively with:
%
\begin{center}
\begin{tabular}{l}
|\input{childdoc.def}|\\
|\childdocby{|\textit{main}|}|\\
\end{tabular}
\end{center}
%
Both forms have slightly different effects as described above.
The main file is prepared as usual, see \secref{sec:include}.

%%%%%%%%%%%%%%%%%%%%%%%%%%%%%%%%%%%%%%%%%%%%%%%%%%%%%%%%%%%%%%%%%%%%%%%%%%%%%%%%
\subsection{Legacy Detection}
\label{sec:detection}

The directive |\childdocmain| in the main file can detect
whether the complete document or merely a child is to be compiled
even without using the directive |\childdocof|.
This method is deprecated because it is less robust
and there is no compelling reason to use it;
it is merely provided for backward compatibility
and it may be removed in future versions.

If the detection mechanism is to be used,
it is mandatory to correctly specify
the filename of the main file as the argument of |\childdocmain|:
%
\begin{center}
\begin{tabular}{l}
|\input{childdoc.def}|\\
|\childdocmain{|\textit{main}|}|\\
\end{tabular}
\end{center}
%
If |\jobname| does not match the argument \textit{main} of |\childdocmain|,
it is assumed that |\jobname| points to the child file to be compiled.
When using |\childdocmain| with the main file specified as argument,
it suffices to start a child file
with just |\input{|\textit{main}|}|
without loading of the package and using |\childdocof|.
If instead all processing is done
with the appropriate \textsf{childdoc} directives,
the argument of \textit{main} of |\childdocmain| can be empty.

An alternative version of the command line processing described
in \secref{sec:commandline} using the detection mechanism reads:
%
\begin{center}
|... -jobname "|\textit{target}|" "|[\textit{flags}]%
[|\def\jobname{|\textit{dest}|}|]|\input{|\textit{main}|}"|
\end{center}

%%%%%%%%%%%%%%%%%%%%%%%%%%%%%%%%%%%%%%%%%%%%%%%%%%%%%%%%%%%%%%%%%%%%%%%%%%%%%%%%
\subsection{Manual Code}
\label{sec:manual}

In case one cannot be certain whether the definitions file |childdoc.def|
is installed on the target \TeX{} distribution
and one prefers not to ship it,
it is conceivable to paste a few relevant commands into the sources.

To that end, drop all statements |\input{childdoc.def}|
and perform the replacements as outlined below.
Instead of |\childdocmain{|\textit{main}|}| add the following code
to the top of the main file:
%
\begin{center}
\begin{tabular}{l}
|\||ifdefined\childdocname\endinput\||fi\newif\ifchilddoc|\\
|\edef\childdocname{\scantokens\expandafter{\jobname\noexpand}}|\\
|\def\childdocmain{|\textit{main}|}\||ifx\childdocmain\childdocname\||else|\\
|\childdoctrue\includeonly{\childdocname}\let\jobname\childdocmain\||fi|\\
\end{tabular}
\end{center}
%
Instead of |\childdocof{|\textit{main}|}| just include the main file
at the top of each child file:
%
\begin{center}
|\input{|\textit{main}|}|
\end{center}
%
A simple redirection |\childdocforward{|\textit{dest}|}| is achieved by:
%
\begin{center}
|\def\jobname{|\textit{dest}|}\input{\jobname}|
\end{center}
%
The redirection with prefix
|\childdocforwardprefix[|\textit{prefix}|]{|\textit{dest}|}|
is accomplished by:
%
\begin{center}
\begin{tabular}{l}
|{\edef\jobname{\scantokens\expandafter{\jobname\noexpand}}|\\
|\def\redirectjob |\textit{prefix}|#1~~~{\gdef\jobname{|\textit{dest}|#1}}|\\
|\expandafter\redirectjob\jobname~~~}\input{\jobname}|
\end{tabular}
\end{center}

In an alternative approach,
child documents can be compiled by a specific command line
without additional code or specific definitions:
%
\begin{center}
|... -jobname "|\textit{target}|" "|[\textit{flags}]%
|\includeonly{|\textit{dest}|}\input{|\textit{main}|}"|
\end{center}
%

%%%%%%%%%%%%%%%%%%%%%%%%%%%%%%%%%%%%%%%%%%%%%%%%%%%%%%%%%%%%%%%%%%%%%%%%%%%%%%%%
%%%%%%%%%%%%%%%%%%%%%%%%%%%%%%%%%%%%%%%%%%%%%%%%%%%%%%%%%%%%%%%%%%%%%%%%%%%%%%%%
\section{Information}

%%%%%%%%%%%%%%%%%%%%%%%%%%%%%%%%%%%%%%%%%%%%%%%%%%%%%%%%%%%%%%%%%%%%%%%%%%%%%%%%
\subsection{Copyright}

Copyright \copyright{} 2017--2018 Niklas Beisert

This work may be distributed and/or modified under the
conditions of the \LaTeX{} Project Public License, either version 1.3
of this license or (at your option) any later version.
The latest version of this license is in
  \url{http://www.latex-project.org/lppl.txt}
and version 1.3 or later is part of all distributions of \LaTeX{}
version 2005/12/01 or later.

This work has the LPPL maintenance status `maintained'.

The Current Maintainer of this work is Niklas Beisert.

This work consists of the files |README.txt|, |childdoc.ins| and |childdoc.dtx|
as well as the derived files |childdoc.def|, |cdocsamp.tex|
with |cdocsch1.tex|, |cdocsch2.tex|, |cdocspt3.tex|, |cdocspt4.tex|,
|cdocsdrf.tex|, |cdocsfn1.tex|, |cdocsfn2.tex|
as well as |childdoc.pdf|.

%%%%%%%%%%%%%%%%%%%%%%%%%%%%%%%%%%%%%%%%%%%%%%%%%%%%%%%%%%%%%%%%%%%%%%%%%%%%%%%%
\subsection{Files and Installation}

The package consists of the files:
%
\begin{center}
\begin{tabular}{ll}
    |README.txt|   & readme file \\
    |childdoc.ins| & installation file \\
    |childdoc.dtx| & source file \\
    |childdoc.def| & definition file \\
    |cdocsamp.tex| & sample main file \\
    |cdocsch1.tex| & sample include file \\
    |cdocsch2.tex| & sample include file \\
    |cdocspt3.tex| & sample part file \\
    |cdocspt4.tex| & sample part file \\
    |cdocsdrf.tex| & sample redirection file \\
    |cdocsfn1.tex| & sample redirection file \\
    |cdocsfn2.tex| & sample redirection file \\
    |childdoc.pdf| & manual
\end{tabular}
\end{center}
%
The distribution consists of the files
|README.txt|, |childdoc.ins| and |childdoc.dtx|.
%
\begin{itemize}
\item
Run (pdf)\LaTeX{} on |childdoc.dtx|
to compile the manual |childdoc.pdf| (this file).
\item
Run \LaTeX{} on |childdoc.ins| to create the definitions file |childdoc.def|
and the sample |cdocsamp.tex| with include files
|cdocsch1.tex|, |cdocsch2.tex|, |cdocspt3.tex|, |cdocspt4.tex|,
|cdocsdrf.tex|, |cdocsfn1.tex|, |cdocsfn2.tex|.
Then copy the file |childdoc.def| to an appropriate directory of your \LaTeX{}
distribution, e.g.\ \textit{texmf-root}|/tex/latex/childdoc|.
\end{itemize}

%%%%%%%%%%%%%%%%%%%%%%%%%%%%%%%%%%%%%%%%%%%%%%%%%%%%%%%%%%%%%%%%%%%%%%%%%%%%%%%%
\subsection{Related CTAN Packages}

There are several other packages which offer a similar functionality:
%
\begin{itemize}
\item
The packages
\href{http://ctan.org/pkg/docmute}{\textsf{docmute}},
\href{http://ctan.org/pkg/includex}{\textsf{includex}} and
\href{http://ctan.org/pkg/standalone}{\textsf{standalone}}
provide commands to include only the document body of
a child file thus allowing both files to be compiled individually.
\item
The packages \href{http://ctan.org/pkg/subdocs}{\textsf{subdocs}}
and \href{http://ctan.org/pkg/subfiles}{\textsf{subfiles}}
provide structures in which the main and child documents can be
encapsulated and allowing them to be compiled individually.
The inclusion mechanism is different from the conventional |\include|.
\item
The package \href{http://ctan.org/pkg/combine}{\textsf{combine}}
is an elaborate solution to combine several documents into one.
\end{itemize}
%
See also the CTAN topic \href{http://ctan.org/topic/subdocs}{\textsf{subdocs}}
for further related packages.
The present package differs from the above solutions in that
a document structure constructed with the conventional |\include| mechanism
just needs two extra commands at the top of every file
such that all constituent files can be compiled individually.

%%%%%%%%%%%%%%%%%%%%%%%%%%%%%%%%%%%%%%%%%%%%%%%%%%%%%%%%%%%%%%%%%%%%%%%%%%%%%%%%
%\subsection{Feature Suggestions}
%
%The following is a list of features which may be useful for future
%versions of this package:
%%
%\begin{itemize}
%\item
%\ldots
%\end{itemize}

%%%%%%%%%%%%%%%%%%%%%%%%%%%%%%%%%%%%%%%%%%%%%%%%%%%%%%%%%%%%%%%%%%%%%%%%%%%%%%%%
\subsection{Revision History}

%%%%%%%%%%%%%%%%%%%%%%%%%%%%%%%%%%%%%%%%
\paragraph{v2.0:} 2018/12/30

\begin{itemize}
\item
immediate forward processing
\item
added |\childdocby| mechanism
\item
manual restructured
\end{itemize}

%%%%%%%%%%%%%%%%%%%%%%%%%%%%%%%%%%%%%%%%
\paragraph{v1.6:} 2018/01/17

\begin{itemize}
\item
application for development of include files
\item
corrections to manual
\end{itemize}

%%%%%%%%%%%%%%%%%%%%%%%%%%%%%%%%%%%%%%%%
\paragraph{v1.5:} 2017/05/21

\begin{itemize}
\item
more complete structuring introduced
\item
|\childdocof| introduced
\item
|\childdoc| renamed to |\childdocmain|
\item
|\childredirect| renamed to |\childdocforward| and |\childdocforwardprefix|
and functionality expanded
\end{itemize}

%%%%%%%%%%%%%%%%%%%%%%%%%%%%%%%%%%%%%%%%
\paragraph{v1.0:} 2017/04/27

\begin{itemize}
\item
manual and install package
\item
first version published on CTAN
\end{itemize}

%%%%%%%%%%%%%%%%%%%%%%%%%%%%%%%%%%%%%%%%
\paragraph{v0.6:} 2017/04/26

\begin{itemize}
\item
redirection mechanism added
\end{itemize}

%%%%%%%%%%%%%%%%%%%%%%%%%%%%%%%%%%%%%%%%
\paragraph{v0.5:} 2017/04/26

\begin{itemize}
\item
functionality in definition file
\end{itemize}


%%%%%%%%%%%%%%%%%%%%%%%%%%%%%%%%%%%%%%%%%%%%%%%%%%%%%%%%%%%%%%%%%%%%%%%%%%%%%%%%
%%%%%%%%%%%%%%%%%%%%%%%%%%%%%%%%%%%%%%%%%%%%%%%%%%%%%%%%%%%%%%%%%%%%%%%%%%%%%%%%
%%%%%%%%%%%%%%%%%%%%%%%%%%%%%%%%%%%%%%%%%%%%%%%%%%%%%%%%%%%%%%%%%%%%%%%%%%%%%%%%
\appendix

\settowidth\MacroIndent{\rmfamily\scriptsize 000\ }

 \DocInput{childdoc.dtx}

\end{document}
%</driver>
% \fi
%
% %%%%%%%%%%%%%%%%%%%%%%%%%%%%%%%%%%%%%%%%%%%%%%%%%%%%%%%%%%%%%%%%%%%%%%%%%%%%%%
% %%%%%%%%%%%%%%%%%%%%%%%%%%%%%%%%%%%%%%%%%%%%%%%%%%%%%%%%%%%%%%%%%%%%%%%%%%%%%%
% \section{Sample}
%\iffalse
%<*samplemain>
%\fi
%
% The following presents a sample document
% with two chapters, two parts, a title page,
% a compile flag as well as three forwarding files to set the flag.
% It consists of eight |.tex| files:
% \begin{center}
% \begin{tabular}{ll}
% |cdocsamp.tex|&main file\\
% |cdocsch1.tex|&include file for chapter 1\\
% |cdocsch2.tex|&include file for chapter 2\\
% |cdocspt3.tex|&include file for part 3\\
% |cdocspt4.tex|&include file for part 4\\
% |cdocsdrf.tex|&forwarding file for main file in draft mode\\
% |cdocsfi1.tex|&forwarding file for final version of chapter 1\\
% |cdocsfi2.tex|&forwarding file for final version of chapter 2\\
% \end{tabular}
% \end{center}
% Each of the eight files can be compiled directly by the \LaTeX{} compiler.
%
% %%%%%%%%%%%%%%%%%%%%%%%%%%%%%%%%%%%%%%
% \paragraph{Main File.}
%
% The main file is called |cdocsamp.tex|.
%
% Load the \textsf{childdoc} definitions and
% declare the filename for the main document:
%    \begin{macrocode}
\input{childdoc.def}
\childdocmain{}
%    \end{macrocode}

% Optional override for |\version| flag:
%    \begin{macrocode}
%%\ifchilddoc\else\providecommand{\version}{draft}\fi
%    \end{macrocode}

% Define the default values for the |\version| flag
% (|final| for the main file and |draft| for childs):
%    \begin{macrocode}
\ifchilddoc
\providecommand{\version}{draft}
\else
\providecommand{\version}{final}
\fi
%    \end{macrocode}

% Load the standard document class:
%    \begin{macrocode}
\documentclass[12pt]{article}
%    \end{macrocode}

% Start the document body:
%    \begin{macrocode}
\begin{document}
%    \end{macrocode}

% Declare a title page.
% Print title, part of document being processed and version flag:
%    \begin{macrocode}
\addtocounter{page}{-1}
\begin{center}
{\LARGE\bfseries{}childdoc example\par}
\vspace{1cm}
\ifchilddoc
\ifchilddocmanual part\else chapter\fi:
`\childdocname' of `\childdocjob'\par
\else
main document: `\childdocjob'\par
\fi
version: \version\par
\end{center}
\newpage
%    \end{macrocode}

% Manually include selected file,
% otherwise process as usual:
%    \begin{macrocode}
\ifchilddocmanual
\section*{part `\childdocname'}
\input{\childdocname}
\else
%    \end{macrocode}

% Include the two chapters:
%    \begin{macrocode}
\include{cdocsch1}
\include{cdocsch2}
%    \end{macrocode}

% Include the two parts unless only chapters should be displayed:
%    \begin{macrocode}
\ifchilddoc\else
\section{part three}
\input{cdocspt3}
\section{part four}
\input{cdocspt4}
\fi
%    \end{macrocode}

% Process as usual until here:
%    \begin{macrocode}
\fi
%    \end{macrocode}

% End of document body:
%    \begin{macrocode}
\end{document}
%    \end{macrocode}
%\iffalse
%</samplemain>
%\fi
%
% %%%%%%%%%%%%%%%%%%%%%%%%%%%%%%%%%%%%%%
% \paragraph{Chapter Include Files.}
%
% The include files are called |cdocsch1.tex| and |cdocsch2.tex|.
%
%\iffalse
%<*samplechap1|samplechap2>
%\fi

% Optional override for |\version| flag:
%    \begin{macrocode}
%%\providecommand{\version}{final}
%    \end{macrocode}

% Include the main document:
%    \begin{macrocode}
\input{childdoc.def}
\childdocof{cdocsamp}
%    \end{macrocode}

%\iffalse
%</samplechap1|samplechap2>
%\fi
%
%\iffalse
%<*samplechap1>
%\fi
% Some text for chapter 1:
%    \begin{macrocode}
\section{one}
some text in chapter one
%    \end{macrocode}

%\iffalse
%</samplechap1>
%\fi
% Some text for chapter 2:
%\iffalse
%<*samplechap2>
%\fi
%    \begin{macrocode}
\section{two}
more text in chapter two
%    \end{macrocode}

%\iffalse
%</samplechap2>
%\fi
%
% %%%%%%%%%%%%%%%%%%%%%%%%%%%%%%%%%%%%%%
% \paragraph{Part Include Files.}
%
% The include files are called |cdocspt3.tex| and |cdocspt4.tex|.
%
%\iffalse
%<*samplepart3|samplepart4>
%\fi

% Optional override for |\version| flag:
%    \begin{macrocode}
%%\providecommand{\version}{final}
%    \end{macrocode}

% Include the main document:
%    \begin{macrocode}
\input{childdoc.def}
\childdocby{cdocsamp}
%    \end{macrocode}

%\iffalse
%</samplepart3|samplepart4>
%\fi
%
%\iffalse
%<*samplepart3>
%\fi
% Some text for part 3:
%    \begin{macrocode}
some text in part three
%    \end{macrocode}

%\iffalse
%</samplepart3>
%\fi
% Some text for part 4:
%\iffalse
%<*samplepart4>
%\fi
%    \begin{macrocode}
more text in part four
%    \end{macrocode}

%\iffalse
%</samplepart4>
%\fi
%
% %%%%%%%%%%%%%%%%%%%%%%%%%%%%%%%%%%%%%%
% \paragraph{Forwarding for a Complete Draft.}
%
% The following forwarding file |cdocsdrf.tex|
% compiles the main document in draft mode:
%\iffalse
%<*sampledraft>
%\fi
%    \begin{macrocode}
\def\version{draft}
\input{childdoc.def}
\childdocforward{cdocsamp}
%    \end{macrocode}

%\iffalse
%</sampledraft>
%\fi
%
% %%%%%%%%%%%%%%%%%%%%%%%%%%%%%%%%%%%%%%
% \paragraph{Forwarding for Final Version of the Chapters.}
%
% The following forwarding files |cdocsfn1.tex| and |cdocsfn2.tex|
% (with identical content)
% compile the final versions of the child documents
% |cdocsch1.tex| and |cdocsch2.tex|, respectively:
%\iffalse
%<*samplefinal>
%\fi
%    \begin{macrocode}
\def\version{final}
\input{childdoc.def}
\childdocforwardprefix[cdocsamp]{cdocsfn}{cdocsch}
%    \end{macrocode}

%\iffalse
%</samplefinal>
%\fi
%
% %%%%%%%%%%%%%%%%%%%%%%%%%%%%%%%%%%%%%%
% \paragraph{Command Line Processing.}
%
% The following three command lines generate the output files
% |cdocscld|, |cdocscl1| and |cdocscl2|
% which should be identical to
% |cdocsdrf|, |cdocsch1| and |cdocsfn2|, respectively:
% \begin{center}
% \begin{tabular}{l}
% |latex -jobname cdocscld \|\\
% |  "\def\version{draft}\input{childdoc.def}\childdocforward{cdocsamp}"|\\
% |latex -jobname cdocscl1 \|\\
% |  "\input{childdoc.def}\childdocforward[cdocsamp]{cdocsch1}"|\\
% |latex -jobname cdocscl2 \|\\
% |  "\def\version{final}\input{childdoc.def}\childdocforward{cdocsch2}"|
% \end{tabular}
% \end{center}
% Note that the trailing backslash on each first line
% merely continues the input to the second line
% (for convenient cut ant paste).
% Furthermore, the command |latex| can be replaced by any
% of its alternative versions such as |pdflatex|.
%
% %%%%%%%%%%%%%%%%%%%%%%%%%%%%%%%%%%%%%%%%%%%%%%%%%%%%%%%%%%%%%%%%%%%%%%%%%%%%%%
% %%%%%%%%%%%%%%%%%%%%%%%%%%%%%%%%%%%%%%%%%%%%%%%%%%%%%%%%%%%%%%%%%%%%%%%%%%%%%%
% \section{Implementation}
%\iffalse
%<*package>
%\fi
%
% This section describes the definitions file |childdoc.def|.

% The definitions cannot be loaded using |\usepackage| or |\RequirePackage|
% which has a mechanism to prevent loading a style file more than once.
% When loading the definitions by means of |\input|
% multiple instances have to be prevented manually:
%\iffalse
%This code needs to be before the `\ProvidesFile' directive
%which is defined at the beginning of this file.
%Therefore it is also placed there and commented out here.
%</package>
%<*discard>
%\fi
%    \begin{macrocode}
\ifdefined\childdocmain\endinput\fi
%    \end{macrocode}
%\iffalse
%</discard>
%<*package>
%\fi
%
% \macro{\ifchilddoc}
% \macro{\ifchilddocmanual}
% The conditional |\ifchilddoc| tells whether a
% child (true) or main (false) document is being compiled.
% The conditional |\ifchilddocmanual| tells whether
% the |\includeonly| mechanism is used (false) or
% the selection of child files must be performed manually (true).
% The definitions initialise to false:
%    \begin{macrocode}
\newif\ifchilddoc
\newif\ifchilddocmanual
%    \end{macrocode}

% \macro{\childdocname}
% \macro{\childdocjob}
% The macro |\childdocname| stores the name of the main document
% to be compiled. The macro |\childdocjob| stores the name of
% the document on which the \LaTeX{} compiler was originally invoked.
% The content of |\jobname| cannot be compared
% to filenames specified in the source due to different catcodes.
% The following code rescans |\jobname|, stores the result
% in |\childdocname| and saves a copy in |\childdocjob|:
%    \begin{macrocode}
\edef\childdocname{\scantokens\expandafter{\jobname\noexpand}}
\let\childdocjob\childdocname
%    \end{macrocode}

% \macro{\childdocdisable}
% The macro |\childdocdisable| prevents the main file
% from being processed more than once.
% At this stage, the main document command |\childdocmain|
% is assumed to be called once again where it should do nothing.
% Any subsequent call to it should prevent
% a secondary processing of the main document
% It overwrites the forwarding commands
% |\childdocof| and |\childdocforward|
% with empty macros to prevent further inclusions of the main document:
%    \begin{macrocode}
\newcommand{\childdocdisable}
{
  \renewcommand{\childdocmain}[1]{\renewcommand{\childdocmain}[1]{\endinput}}
  \renewcommand{\childdocof}[1]{}
  \renewcommand{\childdocby}[2][]{}
  \renewcommand{\childdocforward}[2][]{}
  \renewcommand{\childdocdisable}{}
}
%    \end{macrocode}

% \macro{\childdocmain}
% The macro |\childdocmain| is to be called at the top of the main file
% with nothing or the main filename (without extension) as argument.
% First, it breaks loops.
% If the argument is not empty and does not match |\childdocname|
% (which is set by the first inclusion of |childdoc.def|),
% |\ifchilddoc| is set to true, |\includeonly| is applied to the child file
% and |\jobname| is set to the main file
% (for proper handling of |.aux| files):
%    \begin{macrocode}
\newcommand{\childdocmain}[1]
{
  \childdocdisable\childdocmain{}
  \if?#1?\else
    \begingroup
      \def\childdoctmp{#1}
      \ifx\childdoctmp\childdocname
        \def\childdoctmp{}
      \else
        \def\childdoctmp
        {
          \childdoctrue
          \includeonly{\childdocname}
          \def\childdocjob{#1}
          \def\jobname{#1}
        }
      \fi
      \expandafter
    \endgroup
    \childdoctmp
  \fi
}
%    \end{macrocode}

% \macro{\childdocof}
% The command |\childdocof| redirects
% compilation to the main file |#1|.
%    \begin{macrocode}
\newcommand{\childdocof}[1]
{
  \childdocdisable
  \childdoctrue
  \includeonly{\childdocname}
  \def\jobname{#1}
  \def\childdocjob{#1}
  \input{#1}
}
%    \end{macrocode}

% \macro{\childdocby}
% The command |\childdocby| ....
%    \begin{macrocode}
\newcommand{\childdocby}[2][]
{
  \childdocdisable
  \childdoctrue
  \childdocmanualtrue
  \if?#1?\else
    \def\jobname{#2}
  \fi
  \def\childdocjob{#2}
  \input{#2}
  \endinput
}
%    \end{macrocode}

% \macro{\childdocforward}
% The command |\childdocforward| redirects
% compilation to the main file or
% (if the optional argument is given) a child file.
% Parameters are set as if the main file
% or a child file starting with |\childdocof| was compiled.
% Then compilation is handed over to the main file:
%    \begin{macrocode}
\newcommand{\childdocforward}[2][]
{
  \begingroup
    \if?#1?
      \def\childdoctmp
      {
        \def\childdocname{#2}
        \def\childdocjob{#2}
        \def\jobname{#2}
        \input{#2}
        \endinput
      }
    \else
      \def\childdoctmp
      {
        \childdocdisable
        \def\childdocname{#2}
        \childdoctrue
        \includeonly{#2}
        \def\childdocjob{#1}
        \def\jobname{#1}
        \input{#1}
        \endinput
      }
    \fi
    \expandafter
  \endgroup
  \childdoctmp
}
%    \end{macrocode}

% \macro{\childdocforwardprefix}
% The command |\childdocforwardprefix| redirects
% compilation to the main or a child file by means of a pattern.
% The prefix |#1| in the current filename is replaced by |#2|
% and the suffix of the current filename is kept
% (it is assumed that the filename does not contain the substring `|~~~|'
% which is used as a delimiter).
% Compilation is handed over to the new file by |\childdocforward|:
%    \begin{macrocode}
\newcommand{\childdocforwardprefix}[3][]
{
  \begingroup
    \def\childdocextract #2##1~~~{\def\childdoctmp{\childdocforward[#1]{#3##1}}}
    \expandafter\childdocextract\childdocname~~~
    \expandafter
  \endgroup
  \childdoctmp
}
%    \end{macrocode}

% \macro{\childdoc}
% The deprecated macro |\childdoc| is a legacy version of |\childdocmain|:
%    \begin{macrocode}
\newcommand{\childdoc}{\childdocmain}
%    \end{macrocode}

% \macro{\childdocredirect}
% The deprecated macro |\childdocredirect| is a legacy version
% of |\childdocforward| and |\childdocforwardprefix|:
%    \begin{macrocode}
\newcommand{\childdocredirect}[2][]
{
  \begingroup
    \if?#1?
      \def\childdoctmp{\childdocforward{#2}}
    \else
      \def\childdoctmp{\childdocforwardprefix{#1}{#2}}
    \fi
    \expandafter
  \endgroup
  \childdoctmp
}
%    \end{macrocode}

%\iffalse
%</package>
%\fi
%
\endinput

\childdocmain{}
%    \end{macrocode}

% Optional override for |\version| flag:
%    \begin{macrocode}
%%\ifchilddoc\else\providecommand{\version}{draft}\fi
%    \end{macrocode}

% Define the default values for the |\version| flag
% (|final| for the main file and |draft| for childs):
%    \begin{macrocode}
\ifchilddoc
\providecommand{\version}{draft}
\else
\providecommand{\version}{final}
\fi
%    \end{macrocode}

% Load the standard document class:
%    \begin{macrocode}
\documentclass[12pt]{article}
%    \end{macrocode}

% Start the document body:
%    \begin{macrocode}
\begin{document}
%    \end{macrocode}

% Declare a title page.
% Print title, part of document being processed and version flag:
%    \begin{macrocode}
\addtocounter{page}{-1}
\begin{center}
{\LARGE\bfseries{}childdoc example\par}
\vspace{1cm}
\ifchilddoc
\ifchilddocmanual part\else chapter\fi:
`\childdocname' of `\childdocjob'\par
\else
main document: `\childdocjob'\par
\fi
version: \version\par
\end{center}
\newpage
%    \end{macrocode}

% Manually include selected file,
% otherwise process as usual:
%    \begin{macrocode}
\ifchilddocmanual
\section*{part `\childdocname'}
\input{\childdocname}
\else
%    \end{macrocode}

% Include the two chapters:
%    \begin{macrocode}
\include{cdocsch1}
\include{cdocsch2}
%    \end{macrocode}

% Include the two parts unless only chapters should be displayed:
%    \begin{macrocode}
\ifchilddoc\else
\section{part three}
\input{cdocspt3}
\section{part four}
\input{cdocspt4}
\fi
%    \end{macrocode}

% Process as usual until here:
%    \begin{macrocode}
\fi
%    \end{macrocode}

% End of document body:
%    \begin{macrocode}
\end{document}
%    \end{macrocode}
%\iffalse
%</samplemain>
%\fi
%
% %%%%%%%%%%%%%%%%%%%%%%%%%%%%%%%%%%%%%%
% \paragraph{Chapter Include Files.}
%
% The include files are called |cdocsch1.tex| and |cdocsch2.tex|.
%
%\iffalse
%<*samplechap1|samplechap2>
%\fi

% Optional override for |\version| flag:
%    \begin{macrocode}
%%\providecommand{\version}{final}
%    \end{macrocode}

% Include the main document:
%    \begin{macrocode}
% \iffalse
%
% childdoc.dtx Copyright (C) 2017-2018 Niklas Beisert
%
% This work may be distributed and/or modified under the
% conditions of the LaTeX Project Public License, either version 1.3
% of this license or (at your option) any later version.
% The latest version of this license is in
%   http://www.latex-project.org/lppl.txt
% and version 1.3 or later is part of all distributions of LaTeX
% version 2005/12/01 or later.
%
% This work has the LPPL maintenance status `maintained'.
%
% The Current Maintainer of this work is Niklas Beisert.
%
% This work consists of the files childdoc.dtx and childdoc.ins
% and the derived files childdoc.def and cdocsamp.tex with
% cdocsch1.tex, cdocsch2.tex, cdocsdrf.tex, cdocsfn1.tex, cdocsfn2.tex.
%
%<package>\ifdefined\childdocmain\endinput\fi
%<package>\ProvidesFile{childdoc.def}[2018/12/30 v2.0 child document driver]
%<samplemain>\ProvidesFile{cdocsamp.tex}[2018/12/30 v2.0 sample for childdoc]
%<*driver>
%\ProvidesFile{childdoc.drv}[2018/12/30 v2.0 childdoc reference manual file]
\PassOptionsToClass{10pt,a4paper}{article}
\documentclass{ltxdoc}

\usepackage[margin=35mm]{geometry}
\usepackage{hyperref}
\usepackage{hyperxmp}
\usepackage[usenames]{color}

\hypersetup{colorlinks=true}
\hypersetup{pdfstartview=FitH}
\hypersetup{pdfpagemode=UseNone}
\hypersetup{pdfsource={}}
\hypersetup{pdflang={en-UK}}
\hypersetup{pdfcopyright={Copyright 2017-2018 Niklas Beisert.
  This work may be distributed and/or modified under the
  conditions of the LaTeX Project Public License, either version 1.3
  of this license or (at your option) any later version.}}
\hypersetup{pdflicenseurl={http://www.latex-project.org/lppl.txt}}
\hypersetup{pdfcontactaddress={ETH Zurich, ITP, HIT K,
  Wolfgang-Pauli-Strasse 27}}
\hypersetup{pdfcontactpostcode={8093}}
\hypersetup{pdfcontactcity={Zurich}}
\hypersetup{pdfcontactcountry={Switzerland}}
\hypersetup{pdfcontactemail={nbeisert@itp.phys.ethz.ch}}
\hypersetup{pdfcontacturl={http://people.phys.ethz.ch/\xmptilde nbeisert/}}

\newcommand{\secref}[1]{\hyperref[#1]{section \ref*{#1}}}

\parskip1ex
\parindent0pt
\let\olditemize\itemize
\def\itemize{\olditemize\parskip0pt}

\begin{document}

\title{The \textsf{childdoc} Package}
\hypersetup{pdftitle={The childdoc Package}}
\author{Niklas Beisert\\[2ex]
  Institut f\"ur Theoretische Physik\\
  Eidgen\"ossische Technische Hochschule Z\"urich\\
  Wolfgang-Pauli-Strasse 27, 8093 Z\"urich, Switzerland\\[1ex]
  \href{mailto:nbeisert@itp.phys.ethz.ch}
  {\texttt{nbeisert@itp.phys.ethz.ch}}}
\hypersetup{pdfauthor={Niklas Beisert}}
\hypersetup{pdfsubject={Manual for the LaTeX2e Package childdoc}}
\date{30 December 2018, \textsf{v2.0}}
\maketitle

\begin{abstract}\noindent
\textsf{childdoc} is a \LaTeXe{} package
that enables the direct compilation
of document sections included by |\include|
to individual files.
\end{abstract}

\begingroup
\parskip0ex
\tableofcontents
\endgroup

%%%%%%%%%%%%%%%%%%%%%%%%%%%%%%%%%%%%%%%%%%%%%%%%%%%%%%%%%%%%%%%%%%%%%%%%%%%%%%%%
%%%%%%%%%%%%%%%%%%%%%%%%%%%%%%%%%%%%%%%%%%%%%%%%%%%%%%%%%%%%%%%%%%%%%%%%%%%%%%%%
\section{Introduction}

\LaTeX{} provides a mechanism to structure a large document (such as a book)
into a main file and several child files (containing the chapters)
using the |\include| command.
This mechanism is beneficial for documents
which span hundreds of pages in order to
make the source file(s) more manageable.
Moreover, compilation can be restricted to
selected child files by means of the |\includeonly| command.
The latter feature can be used to reduce the compilation time while editing
(this was significantly more useful in the earlier days of \LaTeX{})
or to generate a smaller document which is easier to navigate.
Another application of |\includeonly| is to generate
documents consisting of selected parts of the complete document.

However, there are a few drawbacks of the plain |\include| mechanism:
\begin{itemize}
\item
The child files cannot be compiled on their own,
they can only be compiled via the main file.
A naive editing environment
(such as a text editor with an option
to have the current file processed by \LaTeX)
may require one to switch to the main file before compiling;
attempting to compile the child file produces errors.
\item
The main file must be modified (each time)
to adjust the |\includeonly| command
to the present needs. This easily leaves the main file in a messy state.
\item
The generated document will always carry the filename
of the main document. This is inconvenient if
several child files are to be compiled and
to be kept for distribution.
\end{itemize}

The present package provides a simple interface
to make child files individually compilable by \LaTeX{}.
Compiling a child file then has the same effect as compiling
the main file with an |\includeonly| command
to select the appropriate child.
Moreover the generated document will carry the name of the child
rather than the main file.
This resolves all three above issues.

This feature is meant to make the editing of books,
thesis documents and lecture notes somewhat more convenient.
However, the package can also be used efficiently for
composing a series of documents (such as exercise sheets)
which are typically distributed individually.
It then assists the author in generating the individual documents
(potentially in different versions)
as well as a document containing the collected series.
Another application is in developing style files
or other kinds of included material
where compilation of the style file could redirect
to a sample or test file.

%%%%%%%%%%%%%%%%%%%%%%%%%%%%%%%%%%%%%%%%%%%%%%%%%%%%%%%%%%%%%%%%%%%%%%%%%%%%%%%%
%%%%%%%%%%%%%%%%%%%%%%%%%%%%%%%%%%%%%%%%%%%%%%%%%%%%%%%%%%%%%%%%%%%%%%%%%%%%%%%%
\section{Usage}

First of all, the package \textsf{childdoc} is \emph{not} a standard
\LaTeXe{} |.sty| style file! Therefore it needs to be invoked in
a non-standard way.

%%%%%%%%%%%%%%%%%%%%%%%%%%%%%%%%%%%%%%%%%%%%%%%%%%%%%%%%%%%%%%%%%%%%%%%%%%%%%%%%
\subsection{Included Files}
\label{sec:include}

%%%%%%%%%%%%%%%%%%%%%%%%%%%%%%%%%%%%%%%%
\DescribeMacro{\childdocmain}
To use the package, add the commands
\begin{center}
\begin{tabular}{l}
|\input{childdoc.def}|\\
|\childdocmain{}|\\
\end{tabular}
\end{center}
at the very top of the main \LaTeX{} file,
in particular \emph{before} the |\documentclass| statement!
The argument of |\childdocmain| should be left empty
(but it must be present).

%%%%%%%%%%%%%%%%%%%%%%%%%%%%%%%%%%%%%%%%
\DescribeMacro{\childdocof}
Furthermore, add the commands
\begin{center}
\begin{tabular}{l}
|\input{childdoc.def}|\\
|\childdocof{|\textit{main}|}|\\
\end{tabular}
\end{center}
at the top of every child file \textit{child}
which is included by |\include{|\textit{child}|}|
from within the main file
(or at least for those files to be compiled individually).
The argument \textit{main} must be the filename of the main file.

There are a couple of
considerations in setting up the main and child documents:

%%%%%%%%%%%%%%%%%%%%%%%%%%%%%%%%%%%%%%%%
\paragraph{Restrictions.}

Please note the following restrictions:
\begin{itemize}
\item
|\childdocmain| must be called with one argument \textit{main}
to ensure compatibility with earlier version of the package.
It must either be empty (|\childdocmain{}|)
or precisely match the filename of the main file in which it is specified.
See \secref{sec:detection} for further information.
\item
The filename \textit{main} must be specified without the |.tex| extension.
\item
The filename \textit{main} is case sensitive
(even in case-insensitive file systems)
due to internal string comparison.
\item
The argument \textit{main} should be fully expanded, it cannot be a macro.
\item
Subdirectories and special characters should be avoided in filenames.
\item
The command |\childdocmain{|\textit{main}|}| must be followed by a whitespace.
It should not be followed immediately by another command
or by a comment mark `|%|'.
This is because the \TeX{} parser reads the token immediately following
the argument of |\childdocmain| and puts it
at the beginning of every child section;
however, a white\-space is ignored.
\end{itemize}

%%%%%%%%%%%%%%%%%%%%%%%%%%%%%%%%%%%%%%%%
\paragraph{Content of Main File.}

It is advisable to place all content in the child files included by |\include|.
Any output contained in the main file will appear in all child documents
unless suppressed manually;
it cannot be suppressed automatically by the |\includeonly| directive
and thus should normally be avoided.
A method to include some content in the main file
by means of conditional processing is described in \secref{sec:conditional}.

%%%%%%%%%%%%%%%%%%%%%%%%%%%%%%%%%%%%%%%%
\paragraph{Page Numbering.}

When only a part of the document is compiled,
the appropriate numbering of pages
(as well as other status parameters)
is determined from the |.aux| files.
The latter contain information from previous passes.
However this information needs to propagate through
all intermediate child documents.
Therefore the page numbering in child documents may well
be inconsistent until the complete document is compiled at least once.

A useful (if unconventional) way to always ensure a consistent
page numbering is to restart the numbering in each child document
and denote the pages by `\textit{child}|.|\textit{page}'
where \textit{child} represents the chapter/section number of the child file.
This can be achieved by the command
|\numberwithin{page}{|\textit{child}|}|
of the \textsf{amsmath} package
where \textit{child} can be |chapter| or |section|
depending on the chosen structuring.
Alternatively, one can modify the macro |\thepage| appropriately
and reset the counter |page| at the start of each child file.

%%%%%%%%%%%%%%%%%%%%%%%%%%%%%%%%%%%%%%%%%%%%%%%%%%%%%%%%%%%%%%%%%%%%%%%%%%%%%%%%
\subsection{Conditional Processing}
\label{sec:conditional}

The package provides a mechanism to compile different versions
of a document. To customise the versions further some conditional processing
can come in handy to distinguish which version is being compiled.
The package provides two macros to describe the compilation context:

%%%%%%%%%%%%%%%%%%%%%%%%%%%%%%%%%%%%%%%%
\DescribeMacro{\ifchilddoc}
The conditional |\ifchilddoc| distinguishes between the compilation of
child documents and the main document:
%
\begin{center}
|\ifchilddoc |\textit{child-code}| |[|\||else |\textit{main-code}]| \||fi|
\end{center}

%%%%%%%%%%%%%%%%%%%%%%%%%%%%%%%%%%%%%%%%
\DescribeMacro{\childdocname}
\DescribeMacro{\childdocjob}
The macro |\childdocname| contains the filename (without extension)
of the main or child file being processed.
Note that |\childdocjob| will always contain the name of the main file.

%%%%%%%%%%%%%%%%%%%%%%%%%%%%%%%%%%%%%%%%
\paragraph{Title Page.}

Conditional processing can be used to include a title or banner page
in the main document when proper precautions are taken.
Importantly, the code in the main file should ensure that the page counter
(as well as other status parameters which are stored in the |.aux| files)
takes the same value after the conditional processing.
Otherwise the page numbers may take divergent values
depending on which part is compiled.

For example, a title page could be declared by:
%
\begin{center}
\begin{tabular}{l}
|\ifchilddoc\||else|\\
|\addtocounter{page}{-1}|\\
\textit{code for title page}\\
|\newpage|\\
|\||fi|
\end{tabular}
\end{center}
%
A banner page for the child documents can be generated by:
%
\begin{center}
\begin{tabular}{l}
|\ifchilddoc|\\
|\addtocounter{page}{-1}|\\
\textit{code for banner page}\\
|\newpage|\\
|\||fi|
\end{tabular}
\end{center}
%
Here one could write a message such as:
\begin{center}
|This is the part \childdocname{} of \childdocjob{}.|
\end{center}

%%%%%%%%%%%%%%%%%%%%%%%%%%%%%%%%%%%%%%%%%%%%%%%%%%%%%%%%%%%%%%%%%%%%%%%%%%%%%%%%
\subsection{Flags}
\label{sec:flags}

The package makes it easy to generate different versions
of the main or child documents.
To this end compilation flags can be defined
and assigned different default values.
They will be particularly useful in conjunction
with the forwarding mechanism described in \secref{sec:forward}.

For example, it may be useful to have a flag |\version|
which can be set to |draft| or |final|.
The document source will contain some conditional code
depending on the value of |\version|.
Suppose further, the flag should default to |final| for the main file
and to |draft| for child files
which is a natural assignment for editing the document.
This is achieved by placing the following code
in the preamble of the main document
(below the |\childdocmain| directive):
%
\begin{center}
\begin{tabular}{l}
|\ifchilddoc|\\
|\providecommand{\version}{draft}|\\
|\||else|\\
|\providecommand{\version}{final}|\\
|\||fi|
\end{tabular}
\end{center}
%
The definition by |\providecommand| makes sure
that previous definitions are not overwritten.
Further statements |\providecommand{\version}{...}|
can thus be added before the above code to override it.

For the main file, one might add a line
(between |\childdocmain| and the above block)
%
\begin{center}
|%\ifchilddoc\||else\providecommand{\version}{draft}\||fi|
\end{center}
%
which can be uncommented to produce a draft version.
Likewise one can add a line to the very top of a child file
(above the |\childdocof{|\textit{main}|}| directive)
%
\begin{center}
|%\providecommand{\version}{final}|
\end{center}
%
which can be uncommented to produce the final version of this child document.

%%%%%%%%%%%%%%%%%%%%%%%%%%%%%%%%%%%%%%%%%%%%%%%%%%%%%%%%%%%%%%%%%%%%%%%%%%%%%%%%
\subsection{Forwarding}
\label{sec:forward}

Different versions of the main or child documents
using compilation flags as described in \secref{sec:flags}
can be (permanently) stored in different files
for convenient compilation, viewing and distribution.
To this end, the package defines a command
to pass on compilation to a different file:

%%%%%%%%%%%%%%%%%%%%%%%%%%%%%%%%%%%%%%%%
\DescribeMacro{\childdocforward}
The command |\childdocforward| redirects processing to
another source file:
%
\begin{center}
\begin{tabular}{l}
|\input{childdoc.def}|\\
|\childdocforward[|\textit{main}|]{|\textit{dest}|}|\\
\end{tabular}
\end{center}
%
The argument \textit{dest} is the destination file
(without extension).
It should be the main file or one of the child files.
Note that further \textsf{childdoc} directives
such as |\childdocof| and |\childdocforward|
in the indicated file will be processed in this form.
The optional argument \textit{main}
passes on directly to the main file \textit{main}
while pretending to compile the child \textit{dest}.
This form behaves as if \textit{dest}
issues |\childdocof{|\textit{main}|}| right away,
and no further \textsf{childdoc} directives will be processed.

%%%%%%%%%%%%%%%%%%%%%%%%%%%%%%%%%%%%%%%%
\DescribeMacro{\...prefix}
In the alternative form |\childdocforwardprefix|,
%
\begin{center}
\begin{tabular}{l}
|\input{childdoc.def}|\\
|\childdocforwardprefix[|\textit{main}|]{|\textit{prefix}|}{|\textit{dest}|}|
\end{tabular}
\end{center}
%
the destination file is determined by a pattern
depending on the current file:
To make this work, the current file must be called
`{\textit{prefix}\hspace{0.2em}\textit{suffix}}'
with \textit{prefix} matching precisely the argument.
Processing is then passed on to the file
`{\textit{dest}\hspace{0.2em}\textit{suffix}}'.
Surely, the same effect is achieved by
directly specifying the
argument `{\textit{dest}\hspace{0.2em}\textit{suffix}}'
in the first form.
However, that requires to set up a different file
for each child. With the alternative form of the command
all these files can have exactly the same content
which simplifies setting them up and maintaining them.

For example, the following file |draft.tex|
with a compilation flag |\version| as described in \secref{sec:flags}
compiles the main document as a draft:
%
\begin{center}
\begin{tabular}{l}
|\def\version{draft}|\\
|\input{childdoc.def}|\\
|\childdocforward{|\textit{main}|}|
\end{tabular}
\end{center}
%
Likewise, the following files |final|\textit{nn}|.tex|
compile the final version of the child document
|child|\textit{nn}|.tex|:
%
\begin{center}
\begin{tabular}{l}
|\def\version{final}|\\
|\input{childdoc.def}|\\
|\childdocforwardprefix{final}{child}|
\end{tabular}
\end{center}
%

Note that when several versions of a main file and/or of each child file
are to be generated, it may be convenient to set up a |Makefile| or
shell script to automatise the process.

%%%%%%%%%%%%%%%%%%%%%%%%%%%%%%%%%%%%%%%%%%%%%%%%%%%%%%%%%%%%%%%%%%%%%%%%%%%%%%%%
\subsection{Command Line Processing}
\label{sec:commandline}

The effect of redirection files can also be achieved by invoking
the \LaTeX{} compiler with a more elaborate command line.
Most conveniently this should be done as part
of a shell script or a |Makefile|.

When using \textsf{childdoc} in the main file, the following
command lines effectively perform a redirection
(note that depending on the shell being used,
backslashes may have to be doubled: `|\|' $\to$ `|\\|'):
%
\begin{center}
|... -jobname "|\textit{target}|" |\\|"|[\textit{flags}]%
|\input{childdoc.def}\childdocforward[|\textit{main}|]{|\textit{dest}|}"|
\end{center}
%
Here \textit{target} is the name of the output file,
\textit{main} is the name of the main file
and \textit{dest} is the name of the main or child file to be processed
(all filenames without extensions).
The optional argument \textit{main} can be omitted
if \textit{main} matches \textit{dest}.
Optionally, compilation \textit{flags} can be defined via |\def| commands.
This command line makes the \TeX{} engine believe
it is compiling the file \textit{target}
whose content is specified as the latter parameter.
The provided code then forwards the processing to
\textit{main} or \textit{dest} as described in \secref{sec:forward}.

%%%%%%%%%%%%%%%%%%%%%%%%%%%%%%%%%%%%%%%%%%%%%%%%%%%%%%%%%%%%%%%%%%%%%%%%%%%%%%%%
\subsection{Include by Input}
\label{sec:input}

Including child documents by |\include| has some restrictions by design.
Most notably, the content of a child document always occupies
its own set of pages; pages cannot be shared between child documents.
Usually, this behaviour makes perfect sense
because each child document contain an essential part of the document.
However, in some situations it may be desirable to compose
a document from a collection of parts
without having mandatory page breaks between then.
For this case, the package
provides a mechanism to include parts
by |\input| which can also be processed individually.
However, by construction this mechanism
requires manual handling of the content to be output.

%%%%%%%%%%%%%%%%%%%%%%%%%%%%%%%%%%%%%%%%
\DescribeMacro{\ifchilddocmanual}
The main file should be prepared as usual, see \secref{sec:include}.
However, the document body must make a distinction
between processing of an individual part and of the main document, e.g.:
%
\begin{center}
\begin{tabular}{l}
|\ifchilddocmanual|\\
|\input{\childdocname}|\\
|\||else|\\
\textit{document body with }|\input{|\textit{part}|}|\\
|\||fi|
\end{tabular}
\end{center}
%
The conditional |\ifchilddocmanual| is true whenever
a part to be included by |\input| is being compiled,
and the name of the part is stored in |\childdocname|.

%%%%%%%%%%%%%%%%%%%%%%%%%%%%%%%%%%%%%%%%
\DescribeMacro{\childdocby}
Each part to be included by |\input| should start with:
%
\begin{center}
\begin{tabular}{l}
|\input{childdoc.def}|\\
|\childdocby{|\textit{main}|}|\\
\end{tabular}
\end{center}
%
The directive |\childdocby| is similar to |\childdocof|
described in \secref{sec:include},
but the subsequent selection of content must be done manually.
To that end, both |\ifchilddoc| and |\ifchilddocmanual|
will be true upon processing of a part,
and the name of the part is stored in |\childdocname|.
Note that |\jobname| will be set to the filename of the current part
so that each part receives an individual |.aux| file
that does not interfere with the |.aux| file(s) of the main document.
This behaviour can be altered by the alternative form
|\childdocby[*]{|\textit{main}|}| (with a non-empty optional argument)
which uses the |.aux| file of the main document
by setting |\jobname| to \textit{main}.

%%%%%%%%%%%%%%%%%%%%%%%%%%%%%%%%%%%%%%%%%%%%%%%%%%%%%%%%%%%%%%%%%%%%%%%%%%%%%%%%
\subsection{Driver Development}
\label{sec:driver}

The \textsf{childdoc} mechanism can also be use for the development
of definition files such as \LaTeX{} styles or classes.
This case differs from the above setup with multiple parts
included by |\include| in that no |\includeonly| should be invoked.
This can be achieved by starting the include file
(before |\ProvidesPackage|) with:
%
\begin{center}
\begin{tabular}{l}
|\input{childdoc.def}|\\
|\childdocforward{|\textit{main}|}|\\
\end{tabular}
\end{center}
%
or alternatively with:
%
\begin{center}
\begin{tabular}{l}
|\input{childdoc.def}|\\
|\childdocby{|\textit{main}|}|\\
\end{tabular}
\end{center}
%
Both forms have slightly different effects as described above.
The main file is prepared as usual, see \secref{sec:include}.

%%%%%%%%%%%%%%%%%%%%%%%%%%%%%%%%%%%%%%%%%%%%%%%%%%%%%%%%%%%%%%%%%%%%%%%%%%%%%%%%
\subsection{Legacy Detection}
\label{sec:detection}

The directive |\childdocmain| in the main file can detect
whether the complete document or merely a child is to be compiled
even without using the directive |\childdocof|.
This method is deprecated because it is less robust
and there is no compelling reason to use it;
it is merely provided for backward compatibility
and it may be removed in future versions.

If the detection mechanism is to be used,
it is mandatory to correctly specify
the filename of the main file as the argument of |\childdocmain|:
%
\begin{center}
\begin{tabular}{l}
|\input{childdoc.def}|\\
|\childdocmain{|\textit{main}|}|\\
\end{tabular}
\end{center}
%
If |\jobname| does not match the argument \textit{main} of |\childdocmain|,
it is assumed that |\jobname| points to the child file to be compiled.
When using |\childdocmain| with the main file specified as argument,
it suffices to start a child file
with just |\input{|\textit{main}|}|
without loading of the package and using |\childdocof|.
If instead all processing is done
with the appropriate \textsf{childdoc} directives,
the argument of \textit{main} of |\childdocmain| can be empty.

An alternative version of the command line processing described
in \secref{sec:commandline} using the detection mechanism reads:
%
\begin{center}
|... -jobname "|\textit{target}|" "|[\textit{flags}]%
[|\def\jobname{|\textit{dest}|}|]|\input{|\textit{main}|}"|
\end{center}

%%%%%%%%%%%%%%%%%%%%%%%%%%%%%%%%%%%%%%%%%%%%%%%%%%%%%%%%%%%%%%%%%%%%%%%%%%%%%%%%
\subsection{Manual Code}
\label{sec:manual}

In case one cannot be certain whether the definitions file |childdoc.def|
is installed on the target \TeX{} distribution
and one prefers not to ship it,
it is conceivable to paste a few relevant commands into the sources.

To that end, drop all statements |\input{childdoc.def}|
and perform the replacements as outlined below.
Instead of |\childdocmain{|\textit{main}|}| add the following code
to the top of the main file:
%
\begin{center}
\begin{tabular}{l}
|\||ifdefined\childdocname\endinput\||fi\newif\ifchilddoc|\\
|\edef\childdocname{\scantokens\expandafter{\jobname\noexpand}}|\\
|\def\childdocmain{|\textit{main}|}\||ifx\childdocmain\childdocname\||else|\\
|\childdoctrue\includeonly{\childdocname}\let\jobname\childdocmain\||fi|\\
\end{tabular}
\end{center}
%
Instead of |\childdocof{|\textit{main}|}| just include the main file
at the top of each child file:
%
\begin{center}
|\input{|\textit{main}|}|
\end{center}
%
A simple redirection |\childdocforward{|\textit{dest}|}| is achieved by:
%
\begin{center}
|\def\jobname{|\textit{dest}|}\input{\jobname}|
\end{center}
%
The redirection with prefix
|\childdocforwardprefix[|\textit{prefix}|]{|\textit{dest}|}|
is accomplished by:
%
\begin{center}
\begin{tabular}{l}
|{\edef\jobname{\scantokens\expandafter{\jobname\noexpand}}|\\
|\def\redirectjob |\textit{prefix}|#1~~~{\gdef\jobname{|\textit{dest}|#1}}|\\
|\expandafter\redirectjob\jobname~~~}\input{\jobname}|
\end{tabular}
\end{center}

In an alternative approach,
child documents can be compiled by a specific command line
without additional code or specific definitions:
%
\begin{center}
|... -jobname "|\textit{target}|" "|[\textit{flags}]%
|\includeonly{|\textit{dest}|}\input{|\textit{main}|}"|
\end{center}
%

%%%%%%%%%%%%%%%%%%%%%%%%%%%%%%%%%%%%%%%%%%%%%%%%%%%%%%%%%%%%%%%%%%%%%%%%%%%%%%%%
%%%%%%%%%%%%%%%%%%%%%%%%%%%%%%%%%%%%%%%%%%%%%%%%%%%%%%%%%%%%%%%%%%%%%%%%%%%%%%%%
\section{Information}

%%%%%%%%%%%%%%%%%%%%%%%%%%%%%%%%%%%%%%%%%%%%%%%%%%%%%%%%%%%%%%%%%%%%%%%%%%%%%%%%
\subsection{Copyright}

Copyright \copyright{} 2017--2018 Niklas Beisert

This work may be distributed and/or modified under the
conditions of the \LaTeX{} Project Public License, either version 1.3
of this license or (at your option) any later version.
The latest version of this license is in
  \url{http://www.latex-project.org/lppl.txt}
and version 1.3 or later is part of all distributions of \LaTeX{}
version 2005/12/01 or later.

This work has the LPPL maintenance status `maintained'.

The Current Maintainer of this work is Niklas Beisert.

This work consists of the files |README.txt|, |childdoc.ins| and |childdoc.dtx|
as well as the derived files |childdoc.def|, |cdocsamp.tex|
with |cdocsch1.tex|, |cdocsch2.tex|, |cdocspt3.tex|, |cdocspt4.tex|,
|cdocsdrf.tex|, |cdocsfn1.tex|, |cdocsfn2.tex|
as well as |childdoc.pdf|.

%%%%%%%%%%%%%%%%%%%%%%%%%%%%%%%%%%%%%%%%%%%%%%%%%%%%%%%%%%%%%%%%%%%%%%%%%%%%%%%%
\subsection{Files and Installation}

The package consists of the files:
%
\begin{center}
\begin{tabular}{ll}
    |README.txt|   & readme file \\
    |childdoc.ins| & installation file \\
    |childdoc.dtx| & source file \\
    |childdoc.def| & definition file \\
    |cdocsamp.tex| & sample main file \\
    |cdocsch1.tex| & sample include file \\
    |cdocsch2.tex| & sample include file \\
    |cdocspt3.tex| & sample part file \\
    |cdocspt4.tex| & sample part file \\
    |cdocsdrf.tex| & sample redirection file \\
    |cdocsfn1.tex| & sample redirection file \\
    |cdocsfn2.tex| & sample redirection file \\
    |childdoc.pdf| & manual
\end{tabular}
\end{center}
%
The distribution consists of the files
|README.txt|, |childdoc.ins| and |childdoc.dtx|.
%
\begin{itemize}
\item
Run (pdf)\LaTeX{} on |childdoc.dtx|
to compile the manual |childdoc.pdf| (this file).
\item
Run \LaTeX{} on |childdoc.ins| to create the definitions file |childdoc.def|
and the sample |cdocsamp.tex| with include files
|cdocsch1.tex|, |cdocsch2.tex|, |cdocspt3.tex|, |cdocspt4.tex|,
|cdocsdrf.tex|, |cdocsfn1.tex|, |cdocsfn2.tex|.
Then copy the file |childdoc.def| to an appropriate directory of your \LaTeX{}
distribution, e.g.\ \textit{texmf-root}|/tex/latex/childdoc|.
\end{itemize}

%%%%%%%%%%%%%%%%%%%%%%%%%%%%%%%%%%%%%%%%%%%%%%%%%%%%%%%%%%%%%%%%%%%%%%%%%%%%%%%%
\subsection{Related CTAN Packages}

There are several other packages which offer a similar functionality:
%
\begin{itemize}
\item
The packages
\href{http://ctan.org/pkg/docmute}{\textsf{docmute}},
\href{http://ctan.org/pkg/includex}{\textsf{includex}} and
\href{http://ctan.org/pkg/standalone}{\textsf{standalone}}
provide commands to include only the document body of
a child file thus allowing both files to be compiled individually.
\item
The packages \href{http://ctan.org/pkg/subdocs}{\textsf{subdocs}}
and \href{http://ctan.org/pkg/subfiles}{\textsf{subfiles}}
provide structures in which the main and child documents can be
encapsulated and allowing them to be compiled individually.
The inclusion mechanism is different from the conventional |\include|.
\item
The package \href{http://ctan.org/pkg/combine}{\textsf{combine}}
is an elaborate solution to combine several documents into one.
\end{itemize}
%
See also the CTAN topic \href{http://ctan.org/topic/subdocs}{\textsf{subdocs}}
for further related packages.
The present package differs from the above solutions in that
a document structure constructed with the conventional |\include| mechanism
just needs two extra commands at the top of every file
such that all constituent files can be compiled individually.

%%%%%%%%%%%%%%%%%%%%%%%%%%%%%%%%%%%%%%%%%%%%%%%%%%%%%%%%%%%%%%%%%%%%%%%%%%%%%%%%
%\subsection{Feature Suggestions}
%
%The following is a list of features which may be useful for future
%versions of this package:
%%
%\begin{itemize}
%\item
%\ldots
%\end{itemize}

%%%%%%%%%%%%%%%%%%%%%%%%%%%%%%%%%%%%%%%%%%%%%%%%%%%%%%%%%%%%%%%%%%%%%%%%%%%%%%%%
\subsection{Revision History}

%%%%%%%%%%%%%%%%%%%%%%%%%%%%%%%%%%%%%%%%
\paragraph{v2.0:} 2018/12/30

\begin{itemize}
\item
immediate forward processing
\item
added |\childdocby| mechanism
\item
manual restructured
\end{itemize}

%%%%%%%%%%%%%%%%%%%%%%%%%%%%%%%%%%%%%%%%
\paragraph{v1.6:} 2018/01/17

\begin{itemize}
\item
application for development of include files
\item
corrections to manual
\end{itemize}

%%%%%%%%%%%%%%%%%%%%%%%%%%%%%%%%%%%%%%%%
\paragraph{v1.5:} 2017/05/21

\begin{itemize}
\item
more complete structuring introduced
\item
|\childdocof| introduced
\item
|\childdoc| renamed to |\childdocmain|
\item
|\childredirect| renamed to |\childdocforward| and |\childdocforwardprefix|
and functionality expanded
\end{itemize}

%%%%%%%%%%%%%%%%%%%%%%%%%%%%%%%%%%%%%%%%
\paragraph{v1.0:} 2017/04/27

\begin{itemize}
\item
manual and install package
\item
first version published on CTAN
\end{itemize}

%%%%%%%%%%%%%%%%%%%%%%%%%%%%%%%%%%%%%%%%
\paragraph{v0.6:} 2017/04/26

\begin{itemize}
\item
redirection mechanism added
\end{itemize}

%%%%%%%%%%%%%%%%%%%%%%%%%%%%%%%%%%%%%%%%
\paragraph{v0.5:} 2017/04/26

\begin{itemize}
\item
functionality in definition file
\end{itemize}


%%%%%%%%%%%%%%%%%%%%%%%%%%%%%%%%%%%%%%%%%%%%%%%%%%%%%%%%%%%%%%%%%%%%%%%%%%%%%%%%
%%%%%%%%%%%%%%%%%%%%%%%%%%%%%%%%%%%%%%%%%%%%%%%%%%%%%%%%%%%%%%%%%%%%%%%%%%%%%%%%
%%%%%%%%%%%%%%%%%%%%%%%%%%%%%%%%%%%%%%%%%%%%%%%%%%%%%%%%%%%%%%%%%%%%%%%%%%%%%%%%
\appendix

\settowidth\MacroIndent{\rmfamily\scriptsize 000\ }

 \DocInput{childdoc.dtx}

\end{document}
%</driver>
% \fi
%
% %%%%%%%%%%%%%%%%%%%%%%%%%%%%%%%%%%%%%%%%%%%%%%%%%%%%%%%%%%%%%%%%%%%%%%%%%%%%%%
% %%%%%%%%%%%%%%%%%%%%%%%%%%%%%%%%%%%%%%%%%%%%%%%%%%%%%%%%%%%%%%%%%%%%%%%%%%%%%%
% \section{Sample}
%\iffalse
%<*samplemain>
%\fi
%
% The following presents a sample document
% with two chapters, two parts, a title page,
% a compile flag as well as three forwarding files to set the flag.
% It consists of eight |.tex| files:
% \begin{center}
% \begin{tabular}{ll}
% |cdocsamp.tex|&main file\\
% |cdocsch1.tex|&include file for chapter 1\\
% |cdocsch2.tex|&include file for chapter 2\\
% |cdocspt3.tex|&include file for part 3\\
% |cdocspt4.tex|&include file for part 4\\
% |cdocsdrf.tex|&forwarding file for main file in draft mode\\
% |cdocsfi1.tex|&forwarding file for final version of chapter 1\\
% |cdocsfi2.tex|&forwarding file for final version of chapter 2\\
% \end{tabular}
% \end{center}
% Each of the eight files can be compiled directly by the \LaTeX{} compiler.
%
% %%%%%%%%%%%%%%%%%%%%%%%%%%%%%%%%%%%%%%
% \paragraph{Main File.}
%
% The main file is called |cdocsamp.tex|.
%
% Load the \textsf{childdoc} definitions and
% declare the filename for the main document:
%    \begin{macrocode}
\input{childdoc.def}
\childdocmain{}
%    \end{macrocode}

% Optional override for |\version| flag:
%    \begin{macrocode}
%%\ifchilddoc\else\providecommand{\version}{draft}\fi
%    \end{macrocode}

% Define the default values for the |\version| flag
% (|final| for the main file and |draft| for childs):
%    \begin{macrocode}
\ifchilddoc
\providecommand{\version}{draft}
\else
\providecommand{\version}{final}
\fi
%    \end{macrocode}

% Load the standard document class:
%    \begin{macrocode}
\documentclass[12pt]{article}
%    \end{macrocode}

% Start the document body:
%    \begin{macrocode}
\begin{document}
%    \end{macrocode}

% Declare a title page.
% Print title, part of document being processed and version flag:
%    \begin{macrocode}
\addtocounter{page}{-1}
\begin{center}
{\LARGE\bfseries{}childdoc example\par}
\vspace{1cm}
\ifchilddoc
\ifchilddocmanual part\else chapter\fi:
`\childdocname' of `\childdocjob'\par
\else
main document: `\childdocjob'\par
\fi
version: \version\par
\end{center}
\newpage
%    \end{macrocode}

% Manually include selected file,
% otherwise process as usual:
%    \begin{macrocode}
\ifchilddocmanual
\section*{part `\childdocname'}
\input{\childdocname}
\else
%    \end{macrocode}

% Include the two chapters:
%    \begin{macrocode}
\include{cdocsch1}
\include{cdocsch2}
%    \end{macrocode}

% Include the two parts unless only chapters should be displayed:
%    \begin{macrocode}
\ifchilddoc\else
\section{part three}
\input{cdocspt3}
\section{part four}
\input{cdocspt4}
\fi
%    \end{macrocode}

% Process as usual until here:
%    \begin{macrocode}
\fi
%    \end{macrocode}

% End of document body:
%    \begin{macrocode}
\end{document}
%    \end{macrocode}
%\iffalse
%</samplemain>
%\fi
%
% %%%%%%%%%%%%%%%%%%%%%%%%%%%%%%%%%%%%%%
% \paragraph{Chapter Include Files.}
%
% The include files are called |cdocsch1.tex| and |cdocsch2.tex|.
%
%\iffalse
%<*samplechap1|samplechap2>
%\fi

% Optional override for |\version| flag:
%    \begin{macrocode}
%%\providecommand{\version}{final}
%    \end{macrocode}

% Include the main document:
%    \begin{macrocode}
\input{childdoc.def}
\childdocof{cdocsamp}
%    \end{macrocode}

%\iffalse
%</samplechap1|samplechap2>
%\fi
%
%\iffalse
%<*samplechap1>
%\fi
% Some text for chapter 1:
%    \begin{macrocode}
\section{one}
some text in chapter one
%    \end{macrocode}

%\iffalse
%</samplechap1>
%\fi
% Some text for chapter 2:
%\iffalse
%<*samplechap2>
%\fi
%    \begin{macrocode}
\section{two}
more text in chapter two
%    \end{macrocode}

%\iffalse
%</samplechap2>
%\fi
%
% %%%%%%%%%%%%%%%%%%%%%%%%%%%%%%%%%%%%%%
% \paragraph{Part Include Files.}
%
% The include files are called |cdocspt3.tex| and |cdocspt4.tex|.
%
%\iffalse
%<*samplepart3|samplepart4>
%\fi

% Optional override for |\version| flag:
%    \begin{macrocode}
%%\providecommand{\version}{final}
%    \end{macrocode}

% Include the main document:
%    \begin{macrocode}
\input{childdoc.def}
\childdocby{cdocsamp}
%    \end{macrocode}

%\iffalse
%</samplepart3|samplepart4>
%\fi
%
%\iffalse
%<*samplepart3>
%\fi
% Some text for part 3:
%    \begin{macrocode}
some text in part three
%    \end{macrocode}

%\iffalse
%</samplepart3>
%\fi
% Some text for part 4:
%\iffalse
%<*samplepart4>
%\fi
%    \begin{macrocode}
more text in part four
%    \end{macrocode}

%\iffalse
%</samplepart4>
%\fi
%
% %%%%%%%%%%%%%%%%%%%%%%%%%%%%%%%%%%%%%%
% \paragraph{Forwarding for a Complete Draft.}
%
% The following forwarding file |cdocsdrf.tex|
% compiles the main document in draft mode:
%\iffalse
%<*sampledraft>
%\fi
%    \begin{macrocode}
\def\version{draft}
\input{childdoc.def}
\childdocforward{cdocsamp}
%    \end{macrocode}

%\iffalse
%</sampledraft>
%\fi
%
% %%%%%%%%%%%%%%%%%%%%%%%%%%%%%%%%%%%%%%
% \paragraph{Forwarding for Final Version of the Chapters.}
%
% The following forwarding files |cdocsfn1.tex| and |cdocsfn2.tex|
% (with identical content)
% compile the final versions of the child documents
% |cdocsch1.tex| and |cdocsch2.tex|, respectively:
%\iffalse
%<*samplefinal>
%\fi
%    \begin{macrocode}
\def\version{final}
\input{childdoc.def}
\childdocforwardprefix[cdocsamp]{cdocsfn}{cdocsch}
%    \end{macrocode}

%\iffalse
%</samplefinal>
%\fi
%
% %%%%%%%%%%%%%%%%%%%%%%%%%%%%%%%%%%%%%%
% \paragraph{Command Line Processing.}
%
% The following three command lines generate the output files
% |cdocscld|, |cdocscl1| and |cdocscl2|
% which should be identical to
% |cdocsdrf|, |cdocsch1| and |cdocsfn2|, respectively:
% \begin{center}
% \begin{tabular}{l}
% |latex -jobname cdocscld \|\\
% |  "\def\version{draft}\input{childdoc.def}\childdocforward{cdocsamp}"|\\
% |latex -jobname cdocscl1 \|\\
% |  "\input{childdoc.def}\childdocforward[cdocsamp]{cdocsch1}"|\\
% |latex -jobname cdocscl2 \|\\
% |  "\def\version{final}\input{childdoc.def}\childdocforward{cdocsch2}"|
% \end{tabular}
% \end{center}
% Note that the trailing backslash on each first line
% merely continues the input to the second line
% (for convenient cut ant paste).
% Furthermore, the command |latex| can be replaced by any
% of its alternative versions such as |pdflatex|.
%
% %%%%%%%%%%%%%%%%%%%%%%%%%%%%%%%%%%%%%%%%%%%%%%%%%%%%%%%%%%%%%%%%%%%%%%%%%%%%%%
% %%%%%%%%%%%%%%%%%%%%%%%%%%%%%%%%%%%%%%%%%%%%%%%%%%%%%%%%%%%%%%%%%%%%%%%%%%%%%%
% \section{Implementation}
%\iffalse
%<*package>
%\fi
%
% This section describes the definitions file |childdoc.def|.

% The definitions cannot be loaded using |\usepackage| or |\RequirePackage|
% which has a mechanism to prevent loading a style file more than once.
% When loading the definitions by means of |\input|
% multiple instances have to be prevented manually:
%\iffalse
%This code needs to be before the `\ProvidesFile' directive
%which is defined at the beginning of this file.
%Therefore it is also placed there and commented out here.
%</package>
%<*discard>
%\fi
%    \begin{macrocode}
\ifdefined\childdocmain\endinput\fi
%    \end{macrocode}
%\iffalse
%</discard>
%<*package>
%\fi
%
% \macro{\ifchilddoc}
% \macro{\ifchilddocmanual}
% The conditional |\ifchilddoc| tells whether a
% child (true) or main (false) document is being compiled.
% The conditional |\ifchilddocmanual| tells whether
% the |\includeonly| mechanism is used (false) or
% the selection of child files must be performed manually (true).
% The definitions initialise to false:
%    \begin{macrocode}
\newif\ifchilddoc
\newif\ifchilddocmanual
%    \end{macrocode}

% \macro{\childdocname}
% \macro{\childdocjob}
% The macro |\childdocname| stores the name of the main document
% to be compiled. The macro |\childdocjob| stores the name of
% the document on which the \LaTeX{} compiler was originally invoked.
% The content of |\jobname| cannot be compared
% to filenames specified in the source due to different catcodes.
% The following code rescans |\jobname|, stores the result
% in |\childdocname| and saves a copy in |\childdocjob|:
%    \begin{macrocode}
\edef\childdocname{\scantokens\expandafter{\jobname\noexpand}}
\let\childdocjob\childdocname
%    \end{macrocode}

% \macro{\childdocdisable}
% The macro |\childdocdisable| prevents the main file
% from being processed more than once.
% At this stage, the main document command |\childdocmain|
% is assumed to be called once again where it should do nothing.
% Any subsequent call to it should prevent
% a secondary processing of the main document
% It overwrites the forwarding commands
% |\childdocof| and |\childdocforward|
% with empty macros to prevent further inclusions of the main document:
%    \begin{macrocode}
\newcommand{\childdocdisable}
{
  \renewcommand{\childdocmain}[1]{\renewcommand{\childdocmain}[1]{\endinput}}
  \renewcommand{\childdocof}[1]{}
  \renewcommand{\childdocby}[2][]{}
  \renewcommand{\childdocforward}[2][]{}
  \renewcommand{\childdocdisable}{}
}
%    \end{macrocode}

% \macro{\childdocmain}
% The macro |\childdocmain| is to be called at the top of the main file
% with nothing or the main filename (without extension) as argument.
% First, it breaks loops.
% If the argument is not empty and does not match |\childdocname|
% (which is set by the first inclusion of |childdoc.def|),
% |\ifchilddoc| is set to true, |\includeonly| is applied to the child file
% and |\jobname| is set to the main file
% (for proper handling of |.aux| files):
%    \begin{macrocode}
\newcommand{\childdocmain}[1]
{
  \childdocdisable\childdocmain{}
  \if?#1?\else
    \begingroup
      \def\childdoctmp{#1}
      \ifx\childdoctmp\childdocname
        \def\childdoctmp{}
      \else
        \def\childdoctmp
        {
          \childdoctrue
          \includeonly{\childdocname}
          \def\childdocjob{#1}
          \def\jobname{#1}
        }
      \fi
      \expandafter
    \endgroup
    \childdoctmp
  \fi
}
%    \end{macrocode}

% \macro{\childdocof}
% The command |\childdocof| redirects
% compilation to the main file |#1|.
%    \begin{macrocode}
\newcommand{\childdocof}[1]
{
  \childdocdisable
  \childdoctrue
  \includeonly{\childdocname}
  \def\jobname{#1}
  \def\childdocjob{#1}
  \input{#1}
}
%    \end{macrocode}

% \macro{\childdocby}
% The command |\childdocby| ....
%    \begin{macrocode}
\newcommand{\childdocby}[2][]
{
  \childdocdisable
  \childdoctrue
  \childdocmanualtrue
  \if?#1?\else
    \def\jobname{#2}
  \fi
  \def\childdocjob{#2}
  \input{#2}
  \endinput
}
%    \end{macrocode}

% \macro{\childdocforward}
% The command |\childdocforward| redirects
% compilation to the main file or
% (if the optional argument is given) a child file.
% Parameters are set as if the main file
% or a child file starting with |\childdocof| was compiled.
% Then compilation is handed over to the main file:
%    \begin{macrocode}
\newcommand{\childdocforward}[2][]
{
  \begingroup
    \if?#1?
      \def\childdoctmp
      {
        \def\childdocname{#2}
        \def\childdocjob{#2}
        \def\jobname{#2}
        \input{#2}
        \endinput
      }
    \else
      \def\childdoctmp
      {
        \childdocdisable
        \def\childdocname{#2}
        \childdoctrue
        \includeonly{#2}
        \def\childdocjob{#1}
        \def\jobname{#1}
        \input{#1}
        \endinput
      }
    \fi
    \expandafter
  \endgroup
  \childdoctmp
}
%    \end{macrocode}

% \macro{\childdocforwardprefix}
% The command |\childdocforwardprefix| redirects
% compilation to the main or a child file by means of a pattern.
% The prefix |#1| in the current filename is replaced by |#2|
% and the suffix of the current filename is kept
% (it is assumed that the filename does not contain the substring `|~~~|'
% which is used as a delimiter).
% Compilation is handed over to the new file by |\childdocforward|:
%    \begin{macrocode}
\newcommand{\childdocforwardprefix}[3][]
{
  \begingroup
    \def\childdocextract #2##1~~~{\def\childdoctmp{\childdocforward[#1]{#3##1}}}
    \expandafter\childdocextract\childdocname~~~
    \expandafter
  \endgroup
  \childdoctmp
}
%    \end{macrocode}

% \macro{\childdoc}
% The deprecated macro |\childdoc| is a legacy version of |\childdocmain|:
%    \begin{macrocode}
\newcommand{\childdoc}{\childdocmain}
%    \end{macrocode}

% \macro{\childdocredirect}
% The deprecated macro |\childdocredirect| is a legacy version
% of |\childdocforward| and |\childdocforwardprefix|:
%    \begin{macrocode}
\newcommand{\childdocredirect}[2][]
{
  \begingroup
    \if?#1?
      \def\childdoctmp{\childdocforward{#2}}
    \else
      \def\childdoctmp{\childdocforwardprefix{#1}{#2}}
    \fi
    \expandafter
  \endgroup
  \childdoctmp
}
%    \end{macrocode}

%\iffalse
%</package>
%\fi
%
\endinput

\childdocof{cdocsamp}
%    \end{macrocode}

%\iffalse
%</samplechap1|samplechap2>
%\fi
%
%\iffalse
%<*samplechap1>
%\fi
% Some text for chapter 1:
%    \begin{macrocode}
\section{one}
some text in chapter one
%    \end{macrocode}

%\iffalse
%</samplechap1>
%\fi
% Some text for chapter 2:
%\iffalse
%<*samplechap2>
%\fi
%    \begin{macrocode}
\section{two}
more text in chapter two
%    \end{macrocode}

%\iffalse
%</samplechap2>
%\fi
%
% %%%%%%%%%%%%%%%%%%%%%%%%%%%%%%%%%%%%%%
% \paragraph{Part Include Files.}
%
% The include files are called |cdocspt3.tex| and |cdocspt4.tex|.
%
%\iffalse
%<*samplepart3|samplepart4>
%\fi

% Optional override for |\version| flag:
%    \begin{macrocode}
%%\providecommand{\version}{final}
%    \end{macrocode}

% Include the main document:
%    \begin{macrocode}
% \iffalse
%
% childdoc.dtx Copyright (C) 2017-2018 Niklas Beisert
%
% This work may be distributed and/or modified under the
% conditions of the LaTeX Project Public License, either version 1.3
% of this license or (at your option) any later version.
% The latest version of this license is in
%   http://www.latex-project.org/lppl.txt
% and version 1.3 or later is part of all distributions of LaTeX
% version 2005/12/01 or later.
%
% This work has the LPPL maintenance status `maintained'.
%
% The Current Maintainer of this work is Niklas Beisert.
%
% This work consists of the files childdoc.dtx and childdoc.ins
% and the derived files childdoc.def and cdocsamp.tex with
% cdocsch1.tex, cdocsch2.tex, cdocsdrf.tex, cdocsfn1.tex, cdocsfn2.tex.
%
%<package>\ifdefined\childdocmain\endinput\fi
%<package>\ProvidesFile{childdoc.def}[2018/12/30 v2.0 child document driver]
%<samplemain>\ProvidesFile{cdocsamp.tex}[2018/12/30 v2.0 sample for childdoc]
%<*driver>
%\ProvidesFile{childdoc.drv}[2018/12/30 v2.0 childdoc reference manual file]
\PassOptionsToClass{10pt,a4paper}{article}
\documentclass{ltxdoc}

\usepackage[margin=35mm]{geometry}
\usepackage{hyperref}
\usepackage{hyperxmp}
\usepackage[usenames]{color}

\hypersetup{colorlinks=true}
\hypersetup{pdfstartview=FitH}
\hypersetup{pdfpagemode=UseNone}
\hypersetup{pdfsource={}}
\hypersetup{pdflang={en-UK}}
\hypersetup{pdfcopyright={Copyright 2017-2018 Niklas Beisert.
  This work may be distributed and/or modified under the
  conditions of the LaTeX Project Public License, either version 1.3
  of this license or (at your option) any later version.}}
\hypersetup{pdflicenseurl={http://www.latex-project.org/lppl.txt}}
\hypersetup{pdfcontactaddress={ETH Zurich, ITP, HIT K,
  Wolfgang-Pauli-Strasse 27}}
\hypersetup{pdfcontactpostcode={8093}}
\hypersetup{pdfcontactcity={Zurich}}
\hypersetup{pdfcontactcountry={Switzerland}}
\hypersetup{pdfcontactemail={nbeisert@itp.phys.ethz.ch}}
\hypersetup{pdfcontacturl={http://people.phys.ethz.ch/\xmptilde nbeisert/}}

\newcommand{\secref}[1]{\hyperref[#1]{section \ref*{#1}}}

\parskip1ex
\parindent0pt
\let\olditemize\itemize
\def\itemize{\olditemize\parskip0pt}

\begin{document}

\title{The \textsf{childdoc} Package}
\hypersetup{pdftitle={The childdoc Package}}
\author{Niklas Beisert\\[2ex]
  Institut f\"ur Theoretische Physik\\
  Eidgen\"ossische Technische Hochschule Z\"urich\\
  Wolfgang-Pauli-Strasse 27, 8093 Z\"urich, Switzerland\\[1ex]
  \href{mailto:nbeisert@itp.phys.ethz.ch}
  {\texttt{nbeisert@itp.phys.ethz.ch}}}
\hypersetup{pdfauthor={Niklas Beisert}}
\hypersetup{pdfsubject={Manual for the LaTeX2e Package childdoc}}
\date{30 December 2018, \textsf{v2.0}}
\maketitle

\begin{abstract}\noindent
\textsf{childdoc} is a \LaTeXe{} package
that enables the direct compilation
of document sections included by |\include|
to individual files.
\end{abstract}

\begingroup
\parskip0ex
\tableofcontents
\endgroup

%%%%%%%%%%%%%%%%%%%%%%%%%%%%%%%%%%%%%%%%%%%%%%%%%%%%%%%%%%%%%%%%%%%%%%%%%%%%%%%%
%%%%%%%%%%%%%%%%%%%%%%%%%%%%%%%%%%%%%%%%%%%%%%%%%%%%%%%%%%%%%%%%%%%%%%%%%%%%%%%%
\section{Introduction}

\LaTeX{} provides a mechanism to structure a large document (such as a book)
into a main file and several child files (containing the chapters)
using the |\include| command.
This mechanism is beneficial for documents
which span hundreds of pages in order to
make the source file(s) more manageable.
Moreover, compilation can be restricted to
selected child files by means of the |\includeonly| command.
The latter feature can be used to reduce the compilation time while editing
(this was significantly more useful in the earlier days of \LaTeX{})
or to generate a smaller document which is easier to navigate.
Another application of |\includeonly| is to generate
documents consisting of selected parts of the complete document.

However, there are a few drawbacks of the plain |\include| mechanism:
\begin{itemize}
\item
The child files cannot be compiled on their own,
they can only be compiled via the main file.
A naive editing environment
(such as a text editor with an option
to have the current file processed by \LaTeX)
may require one to switch to the main file before compiling;
attempting to compile the child file produces errors.
\item
The main file must be modified (each time)
to adjust the |\includeonly| command
to the present needs. This easily leaves the main file in a messy state.
\item
The generated document will always carry the filename
of the main document. This is inconvenient if
several child files are to be compiled and
to be kept for distribution.
\end{itemize}

The present package provides a simple interface
to make child files individually compilable by \LaTeX{}.
Compiling a child file then has the same effect as compiling
the main file with an |\includeonly| command
to select the appropriate child.
Moreover the generated document will carry the name of the child
rather than the main file.
This resolves all three above issues.

This feature is meant to make the editing of books,
thesis documents and lecture notes somewhat more convenient.
However, the package can also be used efficiently for
composing a series of documents (such as exercise sheets)
which are typically distributed individually.
It then assists the author in generating the individual documents
(potentially in different versions)
as well as a document containing the collected series.
Another application is in developing style files
or other kinds of included material
where compilation of the style file could redirect
to a sample or test file.

%%%%%%%%%%%%%%%%%%%%%%%%%%%%%%%%%%%%%%%%%%%%%%%%%%%%%%%%%%%%%%%%%%%%%%%%%%%%%%%%
%%%%%%%%%%%%%%%%%%%%%%%%%%%%%%%%%%%%%%%%%%%%%%%%%%%%%%%%%%%%%%%%%%%%%%%%%%%%%%%%
\section{Usage}

First of all, the package \textsf{childdoc} is \emph{not} a standard
\LaTeXe{} |.sty| style file! Therefore it needs to be invoked in
a non-standard way.

%%%%%%%%%%%%%%%%%%%%%%%%%%%%%%%%%%%%%%%%%%%%%%%%%%%%%%%%%%%%%%%%%%%%%%%%%%%%%%%%
\subsection{Included Files}
\label{sec:include}

%%%%%%%%%%%%%%%%%%%%%%%%%%%%%%%%%%%%%%%%
\DescribeMacro{\childdocmain}
To use the package, add the commands
\begin{center}
\begin{tabular}{l}
|\input{childdoc.def}|\\
|\childdocmain{}|\\
\end{tabular}
\end{center}
at the very top of the main \LaTeX{} file,
in particular \emph{before} the |\documentclass| statement!
The argument of |\childdocmain| should be left empty
(but it must be present).

%%%%%%%%%%%%%%%%%%%%%%%%%%%%%%%%%%%%%%%%
\DescribeMacro{\childdocof}
Furthermore, add the commands
\begin{center}
\begin{tabular}{l}
|\input{childdoc.def}|\\
|\childdocof{|\textit{main}|}|\\
\end{tabular}
\end{center}
at the top of every child file \textit{child}
which is included by |\include{|\textit{child}|}|
from within the main file
(or at least for those files to be compiled individually).
The argument \textit{main} must be the filename of the main file.

There are a couple of
considerations in setting up the main and child documents:

%%%%%%%%%%%%%%%%%%%%%%%%%%%%%%%%%%%%%%%%
\paragraph{Restrictions.}

Please note the following restrictions:
\begin{itemize}
\item
|\childdocmain| must be called with one argument \textit{main}
to ensure compatibility with earlier version of the package.
It must either be empty (|\childdocmain{}|)
or precisely match the filename of the main file in which it is specified.
See \secref{sec:detection} for further information.
\item
The filename \textit{main} must be specified without the |.tex| extension.
\item
The filename \textit{main} is case sensitive
(even in case-insensitive file systems)
due to internal string comparison.
\item
The argument \textit{main} should be fully expanded, it cannot be a macro.
\item
Subdirectories and special characters should be avoided in filenames.
\item
The command |\childdocmain{|\textit{main}|}| must be followed by a whitespace.
It should not be followed immediately by another command
or by a comment mark `|%|'.
This is because the \TeX{} parser reads the token immediately following
the argument of |\childdocmain| and puts it
at the beginning of every child section;
however, a white\-space is ignored.
\end{itemize}

%%%%%%%%%%%%%%%%%%%%%%%%%%%%%%%%%%%%%%%%
\paragraph{Content of Main File.}

It is advisable to place all content in the child files included by |\include|.
Any output contained in the main file will appear in all child documents
unless suppressed manually;
it cannot be suppressed automatically by the |\includeonly| directive
and thus should normally be avoided.
A method to include some content in the main file
by means of conditional processing is described in \secref{sec:conditional}.

%%%%%%%%%%%%%%%%%%%%%%%%%%%%%%%%%%%%%%%%
\paragraph{Page Numbering.}

When only a part of the document is compiled,
the appropriate numbering of pages
(as well as other status parameters)
is determined from the |.aux| files.
The latter contain information from previous passes.
However this information needs to propagate through
all intermediate child documents.
Therefore the page numbering in child documents may well
be inconsistent until the complete document is compiled at least once.

A useful (if unconventional) way to always ensure a consistent
page numbering is to restart the numbering in each child document
and denote the pages by `\textit{child}|.|\textit{page}'
where \textit{child} represents the chapter/section number of the child file.
This can be achieved by the command
|\numberwithin{page}{|\textit{child}|}|
of the \textsf{amsmath} package
where \textit{child} can be |chapter| or |section|
depending on the chosen structuring.
Alternatively, one can modify the macro |\thepage| appropriately
and reset the counter |page| at the start of each child file.

%%%%%%%%%%%%%%%%%%%%%%%%%%%%%%%%%%%%%%%%%%%%%%%%%%%%%%%%%%%%%%%%%%%%%%%%%%%%%%%%
\subsection{Conditional Processing}
\label{sec:conditional}

The package provides a mechanism to compile different versions
of a document. To customise the versions further some conditional processing
can come in handy to distinguish which version is being compiled.
The package provides two macros to describe the compilation context:

%%%%%%%%%%%%%%%%%%%%%%%%%%%%%%%%%%%%%%%%
\DescribeMacro{\ifchilddoc}
The conditional |\ifchilddoc| distinguishes between the compilation of
child documents and the main document:
%
\begin{center}
|\ifchilddoc |\textit{child-code}| |[|\||else |\textit{main-code}]| \||fi|
\end{center}

%%%%%%%%%%%%%%%%%%%%%%%%%%%%%%%%%%%%%%%%
\DescribeMacro{\childdocname}
\DescribeMacro{\childdocjob}
The macro |\childdocname| contains the filename (without extension)
of the main or child file being processed.
Note that |\childdocjob| will always contain the name of the main file.

%%%%%%%%%%%%%%%%%%%%%%%%%%%%%%%%%%%%%%%%
\paragraph{Title Page.}

Conditional processing can be used to include a title or banner page
in the main document when proper precautions are taken.
Importantly, the code in the main file should ensure that the page counter
(as well as other status parameters which are stored in the |.aux| files)
takes the same value after the conditional processing.
Otherwise the page numbers may take divergent values
depending on which part is compiled.

For example, a title page could be declared by:
%
\begin{center}
\begin{tabular}{l}
|\ifchilddoc\||else|\\
|\addtocounter{page}{-1}|\\
\textit{code for title page}\\
|\newpage|\\
|\||fi|
\end{tabular}
\end{center}
%
A banner page for the child documents can be generated by:
%
\begin{center}
\begin{tabular}{l}
|\ifchilddoc|\\
|\addtocounter{page}{-1}|\\
\textit{code for banner page}\\
|\newpage|\\
|\||fi|
\end{tabular}
\end{center}
%
Here one could write a message such as:
\begin{center}
|This is the part \childdocname{} of \childdocjob{}.|
\end{center}

%%%%%%%%%%%%%%%%%%%%%%%%%%%%%%%%%%%%%%%%%%%%%%%%%%%%%%%%%%%%%%%%%%%%%%%%%%%%%%%%
\subsection{Flags}
\label{sec:flags}

The package makes it easy to generate different versions
of the main or child documents.
To this end compilation flags can be defined
and assigned different default values.
They will be particularly useful in conjunction
with the forwarding mechanism described in \secref{sec:forward}.

For example, it may be useful to have a flag |\version|
which can be set to |draft| or |final|.
The document source will contain some conditional code
depending on the value of |\version|.
Suppose further, the flag should default to |final| for the main file
and to |draft| for child files
which is a natural assignment for editing the document.
This is achieved by placing the following code
in the preamble of the main document
(below the |\childdocmain| directive):
%
\begin{center}
\begin{tabular}{l}
|\ifchilddoc|\\
|\providecommand{\version}{draft}|\\
|\||else|\\
|\providecommand{\version}{final}|\\
|\||fi|
\end{tabular}
\end{center}
%
The definition by |\providecommand| makes sure
that previous definitions are not overwritten.
Further statements |\providecommand{\version}{...}|
can thus be added before the above code to override it.

For the main file, one might add a line
(between |\childdocmain| and the above block)
%
\begin{center}
|%\ifchilddoc\||else\providecommand{\version}{draft}\||fi|
\end{center}
%
which can be uncommented to produce a draft version.
Likewise one can add a line to the very top of a child file
(above the |\childdocof{|\textit{main}|}| directive)
%
\begin{center}
|%\providecommand{\version}{final}|
\end{center}
%
which can be uncommented to produce the final version of this child document.

%%%%%%%%%%%%%%%%%%%%%%%%%%%%%%%%%%%%%%%%%%%%%%%%%%%%%%%%%%%%%%%%%%%%%%%%%%%%%%%%
\subsection{Forwarding}
\label{sec:forward}

Different versions of the main or child documents
using compilation flags as described in \secref{sec:flags}
can be (permanently) stored in different files
for convenient compilation, viewing and distribution.
To this end, the package defines a command
to pass on compilation to a different file:

%%%%%%%%%%%%%%%%%%%%%%%%%%%%%%%%%%%%%%%%
\DescribeMacro{\childdocforward}
The command |\childdocforward| redirects processing to
another source file:
%
\begin{center}
\begin{tabular}{l}
|\input{childdoc.def}|\\
|\childdocforward[|\textit{main}|]{|\textit{dest}|}|\\
\end{tabular}
\end{center}
%
The argument \textit{dest} is the destination file
(without extension).
It should be the main file or one of the child files.
Note that further \textsf{childdoc} directives
such as |\childdocof| and |\childdocforward|
in the indicated file will be processed in this form.
The optional argument \textit{main}
passes on directly to the main file \textit{main}
while pretending to compile the child \textit{dest}.
This form behaves as if \textit{dest}
issues |\childdocof{|\textit{main}|}| right away,
and no further \textsf{childdoc} directives will be processed.

%%%%%%%%%%%%%%%%%%%%%%%%%%%%%%%%%%%%%%%%
\DescribeMacro{\...prefix}
In the alternative form |\childdocforwardprefix|,
%
\begin{center}
\begin{tabular}{l}
|\input{childdoc.def}|\\
|\childdocforwardprefix[|\textit{main}|]{|\textit{prefix}|}{|\textit{dest}|}|
\end{tabular}
\end{center}
%
the destination file is determined by a pattern
depending on the current file:
To make this work, the current file must be called
`{\textit{prefix}\hspace{0.2em}\textit{suffix}}'
with \textit{prefix} matching precisely the argument.
Processing is then passed on to the file
`{\textit{dest}\hspace{0.2em}\textit{suffix}}'.
Surely, the same effect is achieved by
directly specifying the
argument `{\textit{dest}\hspace{0.2em}\textit{suffix}}'
in the first form.
However, that requires to set up a different file
for each child. With the alternative form of the command
all these files can have exactly the same content
which simplifies setting them up and maintaining them.

For example, the following file |draft.tex|
with a compilation flag |\version| as described in \secref{sec:flags}
compiles the main document as a draft:
%
\begin{center}
\begin{tabular}{l}
|\def\version{draft}|\\
|\input{childdoc.def}|\\
|\childdocforward{|\textit{main}|}|
\end{tabular}
\end{center}
%
Likewise, the following files |final|\textit{nn}|.tex|
compile the final version of the child document
|child|\textit{nn}|.tex|:
%
\begin{center}
\begin{tabular}{l}
|\def\version{final}|\\
|\input{childdoc.def}|\\
|\childdocforwardprefix{final}{child}|
\end{tabular}
\end{center}
%

Note that when several versions of a main file and/or of each child file
are to be generated, it may be convenient to set up a |Makefile| or
shell script to automatise the process.

%%%%%%%%%%%%%%%%%%%%%%%%%%%%%%%%%%%%%%%%%%%%%%%%%%%%%%%%%%%%%%%%%%%%%%%%%%%%%%%%
\subsection{Command Line Processing}
\label{sec:commandline}

The effect of redirection files can also be achieved by invoking
the \LaTeX{} compiler with a more elaborate command line.
Most conveniently this should be done as part
of a shell script or a |Makefile|.

When using \textsf{childdoc} in the main file, the following
command lines effectively perform a redirection
(note that depending on the shell being used,
backslashes may have to be doubled: `|\|' $\to$ `|\\|'):
%
\begin{center}
|... -jobname "|\textit{target}|" |\\|"|[\textit{flags}]%
|\input{childdoc.def}\childdocforward[|\textit{main}|]{|\textit{dest}|}"|
\end{center}
%
Here \textit{target} is the name of the output file,
\textit{main} is the name of the main file
and \textit{dest} is the name of the main or child file to be processed
(all filenames without extensions).
The optional argument \textit{main} can be omitted
if \textit{main} matches \textit{dest}.
Optionally, compilation \textit{flags} can be defined via |\def| commands.
This command line makes the \TeX{} engine believe
it is compiling the file \textit{target}
whose content is specified as the latter parameter.
The provided code then forwards the processing to
\textit{main} or \textit{dest} as described in \secref{sec:forward}.

%%%%%%%%%%%%%%%%%%%%%%%%%%%%%%%%%%%%%%%%%%%%%%%%%%%%%%%%%%%%%%%%%%%%%%%%%%%%%%%%
\subsection{Include by Input}
\label{sec:input}

Including child documents by |\include| has some restrictions by design.
Most notably, the content of a child document always occupies
its own set of pages; pages cannot be shared between child documents.
Usually, this behaviour makes perfect sense
because each child document contain an essential part of the document.
However, in some situations it may be desirable to compose
a document from a collection of parts
without having mandatory page breaks between then.
For this case, the package
provides a mechanism to include parts
by |\input| which can also be processed individually.
However, by construction this mechanism
requires manual handling of the content to be output.

%%%%%%%%%%%%%%%%%%%%%%%%%%%%%%%%%%%%%%%%
\DescribeMacro{\ifchilddocmanual}
The main file should be prepared as usual, see \secref{sec:include}.
However, the document body must make a distinction
between processing of an individual part and of the main document, e.g.:
%
\begin{center}
\begin{tabular}{l}
|\ifchilddocmanual|\\
|\input{\childdocname}|\\
|\||else|\\
\textit{document body with }|\input{|\textit{part}|}|\\
|\||fi|
\end{tabular}
\end{center}
%
The conditional |\ifchilddocmanual| is true whenever
a part to be included by |\input| is being compiled,
and the name of the part is stored in |\childdocname|.

%%%%%%%%%%%%%%%%%%%%%%%%%%%%%%%%%%%%%%%%
\DescribeMacro{\childdocby}
Each part to be included by |\input| should start with:
%
\begin{center}
\begin{tabular}{l}
|\input{childdoc.def}|\\
|\childdocby{|\textit{main}|}|\\
\end{tabular}
\end{center}
%
The directive |\childdocby| is similar to |\childdocof|
described in \secref{sec:include},
but the subsequent selection of content must be done manually.
To that end, both |\ifchilddoc| and |\ifchilddocmanual|
will be true upon processing of a part,
and the name of the part is stored in |\childdocname|.
Note that |\jobname| will be set to the filename of the current part
so that each part receives an individual |.aux| file
that does not interfere with the |.aux| file(s) of the main document.
This behaviour can be altered by the alternative form
|\childdocby[*]{|\textit{main}|}| (with a non-empty optional argument)
which uses the |.aux| file of the main document
by setting |\jobname| to \textit{main}.

%%%%%%%%%%%%%%%%%%%%%%%%%%%%%%%%%%%%%%%%%%%%%%%%%%%%%%%%%%%%%%%%%%%%%%%%%%%%%%%%
\subsection{Driver Development}
\label{sec:driver}

The \textsf{childdoc} mechanism can also be use for the development
of definition files such as \LaTeX{} styles or classes.
This case differs from the above setup with multiple parts
included by |\include| in that no |\includeonly| should be invoked.
This can be achieved by starting the include file
(before |\ProvidesPackage|) with:
%
\begin{center}
\begin{tabular}{l}
|\input{childdoc.def}|\\
|\childdocforward{|\textit{main}|}|\\
\end{tabular}
\end{center}
%
or alternatively with:
%
\begin{center}
\begin{tabular}{l}
|\input{childdoc.def}|\\
|\childdocby{|\textit{main}|}|\\
\end{tabular}
\end{center}
%
Both forms have slightly different effects as described above.
The main file is prepared as usual, see \secref{sec:include}.

%%%%%%%%%%%%%%%%%%%%%%%%%%%%%%%%%%%%%%%%%%%%%%%%%%%%%%%%%%%%%%%%%%%%%%%%%%%%%%%%
\subsection{Legacy Detection}
\label{sec:detection}

The directive |\childdocmain| in the main file can detect
whether the complete document or merely a child is to be compiled
even without using the directive |\childdocof|.
This method is deprecated because it is less robust
and there is no compelling reason to use it;
it is merely provided for backward compatibility
and it may be removed in future versions.

If the detection mechanism is to be used,
it is mandatory to correctly specify
the filename of the main file as the argument of |\childdocmain|:
%
\begin{center}
\begin{tabular}{l}
|\input{childdoc.def}|\\
|\childdocmain{|\textit{main}|}|\\
\end{tabular}
\end{center}
%
If |\jobname| does not match the argument \textit{main} of |\childdocmain|,
it is assumed that |\jobname| points to the child file to be compiled.
When using |\childdocmain| with the main file specified as argument,
it suffices to start a child file
with just |\input{|\textit{main}|}|
without loading of the package and using |\childdocof|.
If instead all processing is done
with the appropriate \textsf{childdoc} directives,
the argument of \textit{main} of |\childdocmain| can be empty.

An alternative version of the command line processing described
in \secref{sec:commandline} using the detection mechanism reads:
%
\begin{center}
|... -jobname "|\textit{target}|" "|[\textit{flags}]%
[|\def\jobname{|\textit{dest}|}|]|\input{|\textit{main}|}"|
\end{center}

%%%%%%%%%%%%%%%%%%%%%%%%%%%%%%%%%%%%%%%%%%%%%%%%%%%%%%%%%%%%%%%%%%%%%%%%%%%%%%%%
\subsection{Manual Code}
\label{sec:manual}

In case one cannot be certain whether the definitions file |childdoc.def|
is installed on the target \TeX{} distribution
and one prefers not to ship it,
it is conceivable to paste a few relevant commands into the sources.

To that end, drop all statements |\input{childdoc.def}|
and perform the replacements as outlined below.
Instead of |\childdocmain{|\textit{main}|}| add the following code
to the top of the main file:
%
\begin{center}
\begin{tabular}{l}
|\||ifdefined\childdocname\endinput\||fi\newif\ifchilddoc|\\
|\edef\childdocname{\scantokens\expandafter{\jobname\noexpand}}|\\
|\def\childdocmain{|\textit{main}|}\||ifx\childdocmain\childdocname\||else|\\
|\childdoctrue\includeonly{\childdocname}\let\jobname\childdocmain\||fi|\\
\end{tabular}
\end{center}
%
Instead of |\childdocof{|\textit{main}|}| just include the main file
at the top of each child file:
%
\begin{center}
|\input{|\textit{main}|}|
\end{center}
%
A simple redirection |\childdocforward{|\textit{dest}|}| is achieved by:
%
\begin{center}
|\def\jobname{|\textit{dest}|}\input{\jobname}|
\end{center}
%
The redirection with prefix
|\childdocforwardprefix[|\textit{prefix}|]{|\textit{dest}|}|
is accomplished by:
%
\begin{center}
\begin{tabular}{l}
|{\edef\jobname{\scantokens\expandafter{\jobname\noexpand}}|\\
|\def\redirectjob |\textit{prefix}|#1~~~{\gdef\jobname{|\textit{dest}|#1}}|\\
|\expandafter\redirectjob\jobname~~~}\input{\jobname}|
\end{tabular}
\end{center}

In an alternative approach,
child documents can be compiled by a specific command line
without additional code or specific definitions:
%
\begin{center}
|... -jobname "|\textit{target}|" "|[\textit{flags}]%
|\includeonly{|\textit{dest}|}\input{|\textit{main}|}"|
\end{center}
%

%%%%%%%%%%%%%%%%%%%%%%%%%%%%%%%%%%%%%%%%%%%%%%%%%%%%%%%%%%%%%%%%%%%%%%%%%%%%%%%%
%%%%%%%%%%%%%%%%%%%%%%%%%%%%%%%%%%%%%%%%%%%%%%%%%%%%%%%%%%%%%%%%%%%%%%%%%%%%%%%%
\section{Information}

%%%%%%%%%%%%%%%%%%%%%%%%%%%%%%%%%%%%%%%%%%%%%%%%%%%%%%%%%%%%%%%%%%%%%%%%%%%%%%%%
\subsection{Copyright}

Copyright \copyright{} 2017--2018 Niklas Beisert

This work may be distributed and/or modified under the
conditions of the \LaTeX{} Project Public License, either version 1.3
of this license or (at your option) any later version.
The latest version of this license is in
  \url{http://www.latex-project.org/lppl.txt}
and version 1.3 or later is part of all distributions of \LaTeX{}
version 2005/12/01 or later.

This work has the LPPL maintenance status `maintained'.

The Current Maintainer of this work is Niklas Beisert.

This work consists of the files |README.txt|, |childdoc.ins| and |childdoc.dtx|
as well as the derived files |childdoc.def|, |cdocsamp.tex|
with |cdocsch1.tex|, |cdocsch2.tex|, |cdocspt3.tex|, |cdocspt4.tex|,
|cdocsdrf.tex|, |cdocsfn1.tex|, |cdocsfn2.tex|
as well as |childdoc.pdf|.

%%%%%%%%%%%%%%%%%%%%%%%%%%%%%%%%%%%%%%%%%%%%%%%%%%%%%%%%%%%%%%%%%%%%%%%%%%%%%%%%
\subsection{Files and Installation}

The package consists of the files:
%
\begin{center}
\begin{tabular}{ll}
    |README.txt|   & readme file \\
    |childdoc.ins| & installation file \\
    |childdoc.dtx| & source file \\
    |childdoc.def| & definition file \\
    |cdocsamp.tex| & sample main file \\
    |cdocsch1.tex| & sample include file \\
    |cdocsch2.tex| & sample include file \\
    |cdocspt3.tex| & sample part file \\
    |cdocspt4.tex| & sample part file \\
    |cdocsdrf.tex| & sample redirection file \\
    |cdocsfn1.tex| & sample redirection file \\
    |cdocsfn2.tex| & sample redirection file \\
    |childdoc.pdf| & manual
\end{tabular}
\end{center}
%
The distribution consists of the files
|README.txt|, |childdoc.ins| and |childdoc.dtx|.
%
\begin{itemize}
\item
Run (pdf)\LaTeX{} on |childdoc.dtx|
to compile the manual |childdoc.pdf| (this file).
\item
Run \LaTeX{} on |childdoc.ins| to create the definitions file |childdoc.def|
and the sample |cdocsamp.tex| with include files
|cdocsch1.tex|, |cdocsch2.tex|, |cdocspt3.tex|, |cdocspt4.tex|,
|cdocsdrf.tex|, |cdocsfn1.tex|, |cdocsfn2.tex|.
Then copy the file |childdoc.def| to an appropriate directory of your \LaTeX{}
distribution, e.g.\ \textit{texmf-root}|/tex/latex/childdoc|.
\end{itemize}

%%%%%%%%%%%%%%%%%%%%%%%%%%%%%%%%%%%%%%%%%%%%%%%%%%%%%%%%%%%%%%%%%%%%%%%%%%%%%%%%
\subsection{Related CTAN Packages}

There are several other packages which offer a similar functionality:
%
\begin{itemize}
\item
The packages
\href{http://ctan.org/pkg/docmute}{\textsf{docmute}},
\href{http://ctan.org/pkg/includex}{\textsf{includex}} and
\href{http://ctan.org/pkg/standalone}{\textsf{standalone}}
provide commands to include only the document body of
a child file thus allowing both files to be compiled individually.
\item
The packages \href{http://ctan.org/pkg/subdocs}{\textsf{subdocs}}
and \href{http://ctan.org/pkg/subfiles}{\textsf{subfiles}}
provide structures in which the main and child documents can be
encapsulated and allowing them to be compiled individually.
The inclusion mechanism is different from the conventional |\include|.
\item
The package \href{http://ctan.org/pkg/combine}{\textsf{combine}}
is an elaborate solution to combine several documents into one.
\end{itemize}
%
See also the CTAN topic \href{http://ctan.org/topic/subdocs}{\textsf{subdocs}}
for further related packages.
The present package differs from the above solutions in that
a document structure constructed with the conventional |\include| mechanism
just needs two extra commands at the top of every file
such that all constituent files can be compiled individually.

%%%%%%%%%%%%%%%%%%%%%%%%%%%%%%%%%%%%%%%%%%%%%%%%%%%%%%%%%%%%%%%%%%%%%%%%%%%%%%%%
%\subsection{Feature Suggestions}
%
%The following is a list of features which may be useful for future
%versions of this package:
%%
%\begin{itemize}
%\item
%\ldots
%\end{itemize}

%%%%%%%%%%%%%%%%%%%%%%%%%%%%%%%%%%%%%%%%%%%%%%%%%%%%%%%%%%%%%%%%%%%%%%%%%%%%%%%%
\subsection{Revision History}

%%%%%%%%%%%%%%%%%%%%%%%%%%%%%%%%%%%%%%%%
\paragraph{v2.0:} 2018/12/30

\begin{itemize}
\item
immediate forward processing
\item
added |\childdocby| mechanism
\item
manual restructured
\end{itemize}

%%%%%%%%%%%%%%%%%%%%%%%%%%%%%%%%%%%%%%%%
\paragraph{v1.6:} 2018/01/17

\begin{itemize}
\item
application for development of include files
\item
corrections to manual
\end{itemize}

%%%%%%%%%%%%%%%%%%%%%%%%%%%%%%%%%%%%%%%%
\paragraph{v1.5:} 2017/05/21

\begin{itemize}
\item
more complete structuring introduced
\item
|\childdocof| introduced
\item
|\childdoc| renamed to |\childdocmain|
\item
|\childredirect| renamed to |\childdocforward| and |\childdocforwardprefix|
and functionality expanded
\end{itemize}

%%%%%%%%%%%%%%%%%%%%%%%%%%%%%%%%%%%%%%%%
\paragraph{v1.0:} 2017/04/27

\begin{itemize}
\item
manual and install package
\item
first version published on CTAN
\end{itemize}

%%%%%%%%%%%%%%%%%%%%%%%%%%%%%%%%%%%%%%%%
\paragraph{v0.6:} 2017/04/26

\begin{itemize}
\item
redirection mechanism added
\end{itemize}

%%%%%%%%%%%%%%%%%%%%%%%%%%%%%%%%%%%%%%%%
\paragraph{v0.5:} 2017/04/26

\begin{itemize}
\item
functionality in definition file
\end{itemize}


%%%%%%%%%%%%%%%%%%%%%%%%%%%%%%%%%%%%%%%%%%%%%%%%%%%%%%%%%%%%%%%%%%%%%%%%%%%%%%%%
%%%%%%%%%%%%%%%%%%%%%%%%%%%%%%%%%%%%%%%%%%%%%%%%%%%%%%%%%%%%%%%%%%%%%%%%%%%%%%%%
%%%%%%%%%%%%%%%%%%%%%%%%%%%%%%%%%%%%%%%%%%%%%%%%%%%%%%%%%%%%%%%%%%%%%%%%%%%%%%%%
\appendix

\settowidth\MacroIndent{\rmfamily\scriptsize 000\ }

 \DocInput{childdoc.dtx}

\end{document}
%</driver>
% \fi
%
% %%%%%%%%%%%%%%%%%%%%%%%%%%%%%%%%%%%%%%%%%%%%%%%%%%%%%%%%%%%%%%%%%%%%%%%%%%%%%%
% %%%%%%%%%%%%%%%%%%%%%%%%%%%%%%%%%%%%%%%%%%%%%%%%%%%%%%%%%%%%%%%%%%%%%%%%%%%%%%
% \section{Sample}
%\iffalse
%<*samplemain>
%\fi
%
% The following presents a sample document
% with two chapters, two parts, a title page,
% a compile flag as well as three forwarding files to set the flag.
% It consists of eight |.tex| files:
% \begin{center}
% \begin{tabular}{ll}
% |cdocsamp.tex|&main file\\
% |cdocsch1.tex|&include file for chapter 1\\
% |cdocsch2.tex|&include file for chapter 2\\
% |cdocspt3.tex|&include file for part 3\\
% |cdocspt4.tex|&include file for part 4\\
% |cdocsdrf.tex|&forwarding file for main file in draft mode\\
% |cdocsfi1.tex|&forwarding file for final version of chapter 1\\
% |cdocsfi2.tex|&forwarding file for final version of chapter 2\\
% \end{tabular}
% \end{center}
% Each of the eight files can be compiled directly by the \LaTeX{} compiler.
%
% %%%%%%%%%%%%%%%%%%%%%%%%%%%%%%%%%%%%%%
% \paragraph{Main File.}
%
% The main file is called |cdocsamp.tex|.
%
% Load the \textsf{childdoc} definitions and
% declare the filename for the main document:
%    \begin{macrocode}
\input{childdoc.def}
\childdocmain{}
%    \end{macrocode}

% Optional override for |\version| flag:
%    \begin{macrocode}
%%\ifchilddoc\else\providecommand{\version}{draft}\fi
%    \end{macrocode}

% Define the default values for the |\version| flag
% (|final| for the main file and |draft| for childs):
%    \begin{macrocode}
\ifchilddoc
\providecommand{\version}{draft}
\else
\providecommand{\version}{final}
\fi
%    \end{macrocode}

% Load the standard document class:
%    \begin{macrocode}
\documentclass[12pt]{article}
%    \end{macrocode}

% Start the document body:
%    \begin{macrocode}
\begin{document}
%    \end{macrocode}

% Declare a title page.
% Print title, part of document being processed and version flag:
%    \begin{macrocode}
\addtocounter{page}{-1}
\begin{center}
{\LARGE\bfseries{}childdoc example\par}
\vspace{1cm}
\ifchilddoc
\ifchilddocmanual part\else chapter\fi:
`\childdocname' of `\childdocjob'\par
\else
main document: `\childdocjob'\par
\fi
version: \version\par
\end{center}
\newpage
%    \end{macrocode}

% Manually include selected file,
% otherwise process as usual:
%    \begin{macrocode}
\ifchilddocmanual
\section*{part `\childdocname'}
\input{\childdocname}
\else
%    \end{macrocode}

% Include the two chapters:
%    \begin{macrocode}
\include{cdocsch1}
\include{cdocsch2}
%    \end{macrocode}

% Include the two parts unless only chapters should be displayed:
%    \begin{macrocode}
\ifchilddoc\else
\section{part three}
\input{cdocspt3}
\section{part four}
\input{cdocspt4}
\fi
%    \end{macrocode}

% Process as usual until here:
%    \begin{macrocode}
\fi
%    \end{macrocode}

% End of document body:
%    \begin{macrocode}
\end{document}
%    \end{macrocode}
%\iffalse
%</samplemain>
%\fi
%
% %%%%%%%%%%%%%%%%%%%%%%%%%%%%%%%%%%%%%%
% \paragraph{Chapter Include Files.}
%
% The include files are called |cdocsch1.tex| and |cdocsch2.tex|.
%
%\iffalse
%<*samplechap1|samplechap2>
%\fi

% Optional override for |\version| flag:
%    \begin{macrocode}
%%\providecommand{\version}{final}
%    \end{macrocode}

% Include the main document:
%    \begin{macrocode}
\input{childdoc.def}
\childdocof{cdocsamp}
%    \end{macrocode}

%\iffalse
%</samplechap1|samplechap2>
%\fi
%
%\iffalse
%<*samplechap1>
%\fi
% Some text for chapter 1:
%    \begin{macrocode}
\section{one}
some text in chapter one
%    \end{macrocode}

%\iffalse
%</samplechap1>
%\fi
% Some text for chapter 2:
%\iffalse
%<*samplechap2>
%\fi
%    \begin{macrocode}
\section{two}
more text in chapter two
%    \end{macrocode}

%\iffalse
%</samplechap2>
%\fi
%
% %%%%%%%%%%%%%%%%%%%%%%%%%%%%%%%%%%%%%%
% \paragraph{Part Include Files.}
%
% The include files are called |cdocspt3.tex| and |cdocspt4.tex|.
%
%\iffalse
%<*samplepart3|samplepart4>
%\fi

% Optional override for |\version| flag:
%    \begin{macrocode}
%%\providecommand{\version}{final}
%    \end{macrocode}

% Include the main document:
%    \begin{macrocode}
\input{childdoc.def}
\childdocby{cdocsamp}
%    \end{macrocode}

%\iffalse
%</samplepart3|samplepart4>
%\fi
%
%\iffalse
%<*samplepart3>
%\fi
% Some text for part 3:
%    \begin{macrocode}
some text in part three
%    \end{macrocode}

%\iffalse
%</samplepart3>
%\fi
% Some text for part 4:
%\iffalse
%<*samplepart4>
%\fi
%    \begin{macrocode}
more text in part four
%    \end{macrocode}

%\iffalse
%</samplepart4>
%\fi
%
% %%%%%%%%%%%%%%%%%%%%%%%%%%%%%%%%%%%%%%
% \paragraph{Forwarding for a Complete Draft.}
%
% The following forwarding file |cdocsdrf.tex|
% compiles the main document in draft mode:
%\iffalse
%<*sampledraft>
%\fi
%    \begin{macrocode}
\def\version{draft}
\input{childdoc.def}
\childdocforward{cdocsamp}
%    \end{macrocode}

%\iffalse
%</sampledraft>
%\fi
%
% %%%%%%%%%%%%%%%%%%%%%%%%%%%%%%%%%%%%%%
% \paragraph{Forwarding for Final Version of the Chapters.}
%
% The following forwarding files |cdocsfn1.tex| and |cdocsfn2.tex|
% (with identical content)
% compile the final versions of the child documents
% |cdocsch1.tex| and |cdocsch2.tex|, respectively:
%\iffalse
%<*samplefinal>
%\fi
%    \begin{macrocode}
\def\version{final}
\input{childdoc.def}
\childdocforwardprefix[cdocsamp]{cdocsfn}{cdocsch}
%    \end{macrocode}

%\iffalse
%</samplefinal>
%\fi
%
% %%%%%%%%%%%%%%%%%%%%%%%%%%%%%%%%%%%%%%
% \paragraph{Command Line Processing.}
%
% The following three command lines generate the output files
% |cdocscld|, |cdocscl1| and |cdocscl2|
% which should be identical to
% |cdocsdrf|, |cdocsch1| and |cdocsfn2|, respectively:
% \begin{center}
% \begin{tabular}{l}
% |latex -jobname cdocscld \|\\
% |  "\def\version{draft}\input{childdoc.def}\childdocforward{cdocsamp}"|\\
% |latex -jobname cdocscl1 \|\\
% |  "\input{childdoc.def}\childdocforward[cdocsamp]{cdocsch1}"|\\
% |latex -jobname cdocscl2 \|\\
% |  "\def\version{final}\input{childdoc.def}\childdocforward{cdocsch2}"|
% \end{tabular}
% \end{center}
% Note that the trailing backslash on each first line
% merely continues the input to the second line
% (for convenient cut ant paste).
% Furthermore, the command |latex| can be replaced by any
% of its alternative versions such as |pdflatex|.
%
% %%%%%%%%%%%%%%%%%%%%%%%%%%%%%%%%%%%%%%%%%%%%%%%%%%%%%%%%%%%%%%%%%%%%%%%%%%%%%%
% %%%%%%%%%%%%%%%%%%%%%%%%%%%%%%%%%%%%%%%%%%%%%%%%%%%%%%%%%%%%%%%%%%%%%%%%%%%%%%
% \section{Implementation}
%\iffalse
%<*package>
%\fi
%
% This section describes the definitions file |childdoc.def|.

% The definitions cannot be loaded using |\usepackage| or |\RequirePackage|
% which has a mechanism to prevent loading a style file more than once.
% When loading the definitions by means of |\input|
% multiple instances have to be prevented manually:
%\iffalse
%This code needs to be before the `\ProvidesFile' directive
%which is defined at the beginning of this file.
%Therefore it is also placed there and commented out here.
%</package>
%<*discard>
%\fi
%    \begin{macrocode}
\ifdefined\childdocmain\endinput\fi
%    \end{macrocode}
%\iffalse
%</discard>
%<*package>
%\fi
%
% \macro{\ifchilddoc}
% \macro{\ifchilddocmanual}
% The conditional |\ifchilddoc| tells whether a
% child (true) or main (false) document is being compiled.
% The conditional |\ifchilddocmanual| tells whether
% the |\includeonly| mechanism is used (false) or
% the selection of child files must be performed manually (true).
% The definitions initialise to false:
%    \begin{macrocode}
\newif\ifchilddoc
\newif\ifchilddocmanual
%    \end{macrocode}

% \macro{\childdocname}
% \macro{\childdocjob}
% The macro |\childdocname| stores the name of the main document
% to be compiled. The macro |\childdocjob| stores the name of
% the document on which the \LaTeX{} compiler was originally invoked.
% The content of |\jobname| cannot be compared
% to filenames specified in the source due to different catcodes.
% The following code rescans |\jobname|, stores the result
% in |\childdocname| and saves a copy in |\childdocjob|:
%    \begin{macrocode}
\edef\childdocname{\scantokens\expandafter{\jobname\noexpand}}
\let\childdocjob\childdocname
%    \end{macrocode}

% \macro{\childdocdisable}
% The macro |\childdocdisable| prevents the main file
% from being processed more than once.
% At this stage, the main document command |\childdocmain|
% is assumed to be called once again where it should do nothing.
% Any subsequent call to it should prevent
% a secondary processing of the main document
% It overwrites the forwarding commands
% |\childdocof| and |\childdocforward|
% with empty macros to prevent further inclusions of the main document:
%    \begin{macrocode}
\newcommand{\childdocdisable}
{
  \renewcommand{\childdocmain}[1]{\renewcommand{\childdocmain}[1]{\endinput}}
  \renewcommand{\childdocof}[1]{}
  \renewcommand{\childdocby}[2][]{}
  \renewcommand{\childdocforward}[2][]{}
  \renewcommand{\childdocdisable}{}
}
%    \end{macrocode}

% \macro{\childdocmain}
% The macro |\childdocmain| is to be called at the top of the main file
% with nothing or the main filename (without extension) as argument.
% First, it breaks loops.
% If the argument is not empty and does not match |\childdocname|
% (which is set by the first inclusion of |childdoc.def|),
% |\ifchilddoc| is set to true, |\includeonly| is applied to the child file
% and |\jobname| is set to the main file
% (for proper handling of |.aux| files):
%    \begin{macrocode}
\newcommand{\childdocmain}[1]
{
  \childdocdisable\childdocmain{}
  \if?#1?\else
    \begingroup
      \def\childdoctmp{#1}
      \ifx\childdoctmp\childdocname
        \def\childdoctmp{}
      \else
        \def\childdoctmp
        {
          \childdoctrue
          \includeonly{\childdocname}
          \def\childdocjob{#1}
          \def\jobname{#1}
        }
      \fi
      \expandafter
    \endgroup
    \childdoctmp
  \fi
}
%    \end{macrocode}

% \macro{\childdocof}
% The command |\childdocof| redirects
% compilation to the main file |#1|.
%    \begin{macrocode}
\newcommand{\childdocof}[1]
{
  \childdocdisable
  \childdoctrue
  \includeonly{\childdocname}
  \def\jobname{#1}
  \def\childdocjob{#1}
  \input{#1}
}
%    \end{macrocode}

% \macro{\childdocby}
% The command |\childdocby| ....
%    \begin{macrocode}
\newcommand{\childdocby}[2][]
{
  \childdocdisable
  \childdoctrue
  \childdocmanualtrue
  \if?#1?\else
    \def\jobname{#2}
  \fi
  \def\childdocjob{#2}
  \input{#2}
  \endinput
}
%    \end{macrocode}

% \macro{\childdocforward}
% The command |\childdocforward| redirects
% compilation to the main file or
% (if the optional argument is given) a child file.
% Parameters are set as if the main file
% or a child file starting with |\childdocof| was compiled.
% Then compilation is handed over to the main file:
%    \begin{macrocode}
\newcommand{\childdocforward}[2][]
{
  \begingroup
    \if?#1?
      \def\childdoctmp
      {
        \def\childdocname{#2}
        \def\childdocjob{#2}
        \def\jobname{#2}
        \input{#2}
        \endinput
      }
    \else
      \def\childdoctmp
      {
        \childdocdisable
        \def\childdocname{#2}
        \childdoctrue
        \includeonly{#2}
        \def\childdocjob{#1}
        \def\jobname{#1}
        \input{#1}
        \endinput
      }
    \fi
    \expandafter
  \endgroup
  \childdoctmp
}
%    \end{macrocode}

% \macro{\childdocforwardprefix}
% The command |\childdocforwardprefix| redirects
% compilation to the main or a child file by means of a pattern.
% The prefix |#1| in the current filename is replaced by |#2|
% and the suffix of the current filename is kept
% (it is assumed that the filename does not contain the substring `|~~~|'
% which is used as a delimiter).
% Compilation is handed over to the new file by |\childdocforward|:
%    \begin{macrocode}
\newcommand{\childdocforwardprefix}[3][]
{
  \begingroup
    \def\childdocextract #2##1~~~{\def\childdoctmp{\childdocforward[#1]{#3##1}}}
    \expandafter\childdocextract\childdocname~~~
    \expandafter
  \endgroup
  \childdoctmp
}
%    \end{macrocode}

% \macro{\childdoc}
% The deprecated macro |\childdoc| is a legacy version of |\childdocmain|:
%    \begin{macrocode}
\newcommand{\childdoc}{\childdocmain}
%    \end{macrocode}

% \macro{\childdocredirect}
% The deprecated macro |\childdocredirect| is a legacy version
% of |\childdocforward| and |\childdocforwardprefix|:
%    \begin{macrocode}
\newcommand{\childdocredirect}[2][]
{
  \begingroup
    \if?#1?
      \def\childdoctmp{\childdocforward{#2}}
    \else
      \def\childdoctmp{\childdocforwardprefix{#1}{#2}}
    \fi
    \expandafter
  \endgroup
  \childdoctmp
}
%    \end{macrocode}

%\iffalse
%</package>
%\fi
%
\endinput

\childdocby{cdocsamp}
%    \end{macrocode}

%\iffalse
%</samplepart3|samplepart4>
%\fi
%
%\iffalse
%<*samplepart3>
%\fi
% Some text for part 3:
%    \begin{macrocode}
some text in part three
%    \end{macrocode}

%\iffalse
%</samplepart3>
%\fi
% Some text for part 4:
%\iffalse
%<*samplepart4>
%\fi
%    \begin{macrocode}
more text in part four
%    \end{macrocode}

%\iffalse
%</samplepart4>
%\fi
%
% %%%%%%%%%%%%%%%%%%%%%%%%%%%%%%%%%%%%%%
% \paragraph{Forwarding for a Complete Draft.}
%
% The following forwarding file |cdocsdrf.tex|
% compiles the main document in draft mode:
%\iffalse
%<*sampledraft>
%\fi
%    \begin{macrocode}
\def\version{draft}
% \iffalse
%
% childdoc.dtx Copyright (C) 2017-2018 Niklas Beisert
%
% This work may be distributed and/or modified under the
% conditions of the LaTeX Project Public License, either version 1.3
% of this license or (at your option) any later version.
% The latest version of this license is in
%   http://www.latex-project.org/lppl.txt
% and version 1.3 or later is part of all distributions of LaTeX
% version 2005/12/01 or later.
%
% This work has the LPPL maintenance status `maintained'.
%
% The Current Maintainer of this work is Niklas Beisert.
%
% This work consists of the files childdoc.dtx and childdoc.ins
% and the derived files childdoc.def and cdocsamp.tex with
% cdocsch1.tex, cdocsch2.tex, cdocsdrf.tex, cdocsfn1.tex, cdocsfn2.tex.
%
%<package>\ifdefined\childdocmain\endinput\fi
%<package>\ProvidesFile{childdoc.def}[2018/12/30 v2.0 child document driver]
%<samplemain>\ProvidesFile{cdocsamp.tex}[2018/12/30 v2.0 sample for childdoc]
%<*driver>
%\ProvidesFile{childdoc.drv}[2018/12/30 v2.0 childdoc reference manual file]
\PassOptionsToClass{10pt,a4paper}{article}
\documentclass{ltxdoc}

\usepackage[margin=35mm]{geometry}
\usepackage{hyperref}
\usepackage{hyperxmp}
\usepackage[usenames]{color}

\hypersetup{colorlinks=true}
\hypersetup{pdfstartview=FitH}
\hypersetup{pdfpagemode=UseNone}
\hypersetup{pdfsource={}}
\hypersetup{pdflang={en-UK}}
\hypersetup{pdfcopyright={Copyright 2017-2018 Niklas Beisert.
  This work may be distributed and/or modified under the
  conditions of the LaTeX Project Public License, either version 1.3
  of this license or (at your option) any later version.}}
\hypersetup{pdflicenseurl={http://www.latex-project.org/lppl.txt}}
\hypersetup{pdfcontactaddress={ETH Zurich, ITP, HIT K,
  Wolfgang-Pauli-Strasse 27}}
\hypersetup{pdfcontactpostcode={8093}}
\hypersetup{pdfcontactcity={Zurich}}
\hypersetup{pdfcontactcountry={Switzerland}}
\hypersetup{pdfcontactemail={nbeisert@itp.phys.ethz.ch}}
\hypersetup{pdfcontacturl={http://people.phys.ethz.ch/\xmptilde nbeisert/}}

\newcommand{\secref}[1]{\hyperref[#1]{section \ref*{#1}}}

\parskip1ex
\parindent0pt
\let\olditemize\itemize
\def\itemize{\olditemize\parskip0pt}

\begin{document}

\title{The \textsf{childdoc} Package}
\hypersetup{pdftitle={The childdoc Package}}
\author{Niklas Beisert\\[2ex]
  Institut f\"ur Theoretische Physik\\
  Eidgen\"ossische Technische Hochschule Z\"urich\\
  Wolfgang-Pauli-Strasse 27, 8093 Z\"urich, Switzerland\\[1ex]
  \href{mailto:nbeisert@itp.phys.ethz.ch}
  {\texttt{nbeisert@itp.phys.ethz.ch}}}
\hypersetup{pdfauthor={Niklas Beisert}}
\hypersetup{pdfsubject={Manual for the LaTeX2e Package childdoc}}
\date{30 December 2018, \textsf{v2.0}}
\maketitle

\begin{abstract}\noindent
\textsf{childdoc} is a \LaTeXe{} package
that enables the direct compilation
of document sections included by |\include|
to individual files.
\end{abstract}

\begingroup
\parskip0ex
\tableofcontents
\endgroup

%%%%%%%%%%%%%%%%%%%%%%%%%%%%%%%%%%%%%%%%%%%%%%%%%%%%%%%%%%%%%%%%%%%%%%%%%%%%%%%%
%%%%%%%%%%%%%%%%%%%%%%%%%%%%%%%%%%%%%%%%%%%%%%%%%%%%%%%%%%%%%%%%%%%%%%%%%%%%%%%%
\section{Introduction}

\LaTeX{} provides a mechanism to structure a large document (such as a book)
into a main file and several child files (containing the chapters)
using the |\include| command.
This mechanism is beneficial for documents
which span hundreds of pages in order to
make the source file(s) more manageable.
Moreover, compilation can be restricted to
selected child files by means of the |\includeonly| command.
The latter feature can be used to reduce the compilation time while editing
(this was significantly more useful in the earlier days of \LaTeX{})
or to generate a smaller document which is easier to navigate.
Another application of |\includeonly| is to generate
documents consisting of selected parts of the complete document.

However, there are a few drawbacks of the plain |\include| mechanism:
\begin{itemize}
\item
The child files cannot be compiled on their own,
they can only be compiled via the main file.
A naive editing environment
(such as a text editor with an option
to have the current file processed by \LaTeX)
may require one to switch to the main file before compiling;
attempting to compile the child file produces errors.
\item
The main file must be modified (each time)
to adjust the |\includeonly| command
to the present needs. This easily leaves the main file in a messy state.
\item
The generated document will always carry the filename
of the main document. This is inconvenient if
several child files are to be compiled and
to be kept for distribution.
\end{itemize}

The present package provides a simple interface
to make child files individually compilable by \LaTeX{}.
Compiling a child file then has the same effect as compiling
the main file with an |\includeonly| command
to select the appropriate child.
Moreover the generated document will carry the name of the child
rather than the main file.
This resolves all three above issues.

This feature is meant to make the editing of books,
thesis documents and lecture notes somewhat more convenient.
However, the package can also be used efficiently for
composing a series of documents (such as exercise sheets)
which are typically distributed individually.
It then assists the author in generating the individual documents
(potentially in different versions)
as well as a document containing the collected series.
Another application is in developing style files
or other kinds of included material
where compilation of the style file could redirect
to a sample or test file.

%%%%%%%%%%%%%%%%%%%%%%%%%%%%%%%%%%%%%%%%%%%%%%%%%%%%%%%%%%%%%%%%%%%%%%%%%%%%%%%%
%%%%%%%%%%%%%%%%%%%%%%%%%%%%%%%%%%%%%%%%%%%%%%%%%%%%%%%%%%%%%%%%%%%%%%%%%%%%%%%%
\section{Usage}

First of all, the package \textsf{childdoc} is \emph{not} a standard
\LaTeXe{} |.sty| style file! Therefore it needs to be invoked in
a non-standard way.

%%%%%%%%%%%%%%%%%%%%%%%%%%%%%%%%%%%%%%%%%%%%%%%%%%%%%%%%%%%%%%%%%%%%%%%%%%%%%%%%
\subsection{Included Files}
\label{sec:include}

%%%%%%%%%%%%%%%%%%%%%%%%%%%%%%%%%%%%%%%%
\DescribeMacro{\childdocmain}
To use the package, add the commands
\begin{center}
\begin{tabular}{l}
|\input{childdoc.def}|\\
|\childdocmain{}|\\
\end{tabular}
\end{center}
at the very top of the main \LaTeX{} file,
in particular \emph{before} the |\documentclass| statement!
The argument of |\childdocmain| should be left empty
(but it must be present).

%%%%%%%%%%%%%%%%%%%%%%%%%%%%%%%%%%%%%%%%
\DescribeMacro{\childdocof}
Furthermore, add the commands
\begin{center}
\begin{tabular}{l}
|\input{childdoc.def}|\\
|\childdocof{|\textit{main}|}|\\
\end{tabular}
\end{center}
at the top of every child file \textit{child}
which is included by |\include{|\textit{child}|}|
from within the main file
(or at least for those files to be compiled individually).
The argument \textit{main} must be the filename of the main file.

There are a couple of
considerations in setting up the main and child documents:

%%%%%%%%%%%%%%%%%%%%%%%%%%%%%%%%%%%%%%%%
\paragraph{Restrictions.}

Please note the following restrictions:
\begin{itemize}
\item
|\childdocmain| must be called with one argument \textit{main}
to ensure compatibility with earlier version of the package.
It must either be empty (|\childdocmain{}|)
or precisely match the filename of the main file in which it is specified.
See \secref{sec:detection} for further information.
\item
The filename \textit{main} must be specified without the |.tex| extension.
\item
The filename \textit{main} is case sensitive
(even in case-insensitive file systems)
due to internal string comparison.
\item
The argument \textit{main} should be fully expanded, it cannot be a macro.
\item
Subdirectories and special characters should be avoided in filenames.
\item
The command |\childdocmain{|\textit{main}|}| must be followed by a whitespace.
It should not be followed immediately by another command
or by a comment mark `|%|'.
This is because the \TeX{} parser reads the token immediately following
the argument of |\childdocmain| and puts it
at the beginning of every child section;
however, a white\-space is ignored.
\end{itemize}

%%%%%%%%%%%%%%%%%%%%%%%%%%%%%%%%%%%%%%%%
\paragraph{Content of Main File.}

It is advisable to place all content in the child files included by |\include|.
Any output contained in the main file will appear in all child documents
unless suppressed manually;
it cannot be suppressed automatically by the |\includeonly| directive
and thus should normally be avoided.
A method to include some content in the main file
by means of conditional processing is described in \secref{sec:conditional}.

%%%%%%%%%%%%%%%%%%%%%%%%%%%%%%%%%%%%%%%%
\paragraph{Page Numbering.}

When only a part of the document is compiled,
the appropriate numbering of pages
(as well as other status parameters)
is determined from the |.aux| files.
The latter contain information from previous passes.
However this information needs to propagate through
all intermediate child documents.
Therefore the page numbering in child documents may well
be inconsistent until the complete document is compiled at least once.

A useful (if unconventional) way to always ensure a consistent
page numbering is to restart the numbering in each child document
and denote the pages by `\textit{child}|.|\textit{page}'
where \textit{child} represents the chapter/section number of the child file.
This can be achieved by the command
|\numberwithin{page}{|\textit{child}|}|
of the \textsf{amsmath} package
where \textit{child} can be |chapter| or |section|
depending on the chosen structuring.
Alternatively, one can modify the macro |\thepage| appropriately
and reset the counter |page| at the start of each child file.

%%%%%%%%%%%%%%%%%%%%%%%%%%%%%%%%%%%%%%%%%%%%%%%%%%%%%%%%%%%%%%%%%%%%%%%%%%%%%%%%
\subsection{Conditional Processing}
\label{sec:conditional}

The package provides a mechanism to compile different versions
of a document. To customise the versions further some conditional processing
can come in handy to distinguish which version is being compiled.
The package provides two macros to describe the compilation context:

%%%%%%%%%%%%%%%%%%%%%%%%%%%%%%%%%%%%%%%%
\DescribeMacro{\ifchilddoc}
The conditional |\ifchilddoc| distinguishes between the compilation of
child documents and the main document:
%
\begin{center}
|\ifchilddoc |\textit{child-code}| |[|\||else |\textit{main-code}]| \||fi|
\end{center}

%%%%%%%%%%%%%%%%%%%%%%%%%%%%%%%%%%%%%%%%
\DescribeMacro{\childdocname}
\DescribeMacro{\childdocjob}
The macro |\childdocname| contains the filename (without extension)
of the main or child file being processed.
Note that |\childdocjob| will always contain the name of the main file.

%%%%%%%%%%%%%%%%%%%%%%%%%%%%%%%%%%%%%%%%
\paragraph{Title Page.}

Conditional processing can be used to include a title or banner page
in the main document when proper precautions are taken.
Importantly, the code in the main file should ensure that the page counter
(as well as other status parameters which are stored in the |.aux| files)
takes the same value after the conditional processing.
Otherwise the page numbers may take divergent values
depending on which part is compiled.

For example, a title page could be declared by:
%
\begin{center}
\begin{tabular}{l}
|\ifchilddoc\||else|\\
|\addtocounter{page}{-1}|\\
\textit{code for title page}\\
|\newpage|\\
|\||fi|
\end{tabular}
\end{center}
%
A banner page for the child documents can be generated by:
%
\begin{center}
\begin{tabular}{l}
|\ifchilddoc|\\
|\addtocounter{page}{-1}|\\
\textit{code for banner page}\\
|\newpage|\\
|\||fi|
\end{tabular}
\end{center}
%
Here one could write a message such as:
\begin{center}
|This is the part \childdocname{} of \childdocjob{}.|
\end{center}

%%%%%%%%%%%%%%%%%%%%%%%%%%%%%%%%%%%%%%%%%%%%%%%%%%%%%%%%%%%%%%%%%%%%%%%%%%%%%%%%
\subsection{Flags}
\label{sec:flags}

The package makes it easy to generate different versions
of the main or child documents.
To this end compilation flags can be defined
and assigned different default values.
They will be particularly useful in conjunction
with the forwarding mechanism described in \secref{sec:forward}.

For example, it may be useful to have a flag |\version|
which can be set to |draft| or |final|.
The document source will contain some conditional code
depending on the value of |\version|.
Suppose further, the flag should default to |final| for the main file
and to |draft| for child files
which is a natural assignment for editing the document.
This is achieved by placing the following code
in the preamble of the main document
(below the |\childdocmain| directive):
%
\begin{center}
\begin{tabular}{l}
|\ifchilddoc|\\
|\providecommand{\version}{draft}|\\
|\||else|\\
|\providecommand{\version}{final}|\\
|\||fi|
\end{tabular}
\end{center}
%
The definition by |\providecommand| makes sure
that previous definitions are not overwritten.
Further statements |\providecommand{\version}{...}|
can thus be added before the above code to override it.

For the main file, one might add a line
(between |\childdocmain| and the above block)
%
\begin{center}
|%\ifchilddoc\||else\providecommand{\version}{draft}\||fi|
\end{center}
%
which can be uncommented to produce a draft version.
Likewise one can add a line to the very top of a child file
(above the |\childdocof{|\textit{main}|}| directive)
%
\begin{center}
|%\providecommand{\version}{final}|
\end{center}
%
which can be uncommented to produce the final version of this child document.

%%%%%%%%%%%%%%%%%%%%%%%%%%%%%%%%%%%%%%%%%%%%%%%%%%%%%%%%%%%%%%%%%%%%%%%%%%%%%%%%
\subsection{Forwarding}
\label{sec:forward}

Different versions of the main or child documents
using compilation flags as described in \secref{sec:flags}
can be (permanently) stored in different files
for convenient compilation, viewing and distribution.
To this end, the package defines a command
to pass on compilation to a different file:

%%%%%%%%%%%%%%%%%%%%%%%%%%%%%%%%%%%%%%%%
\DescribeMacro{\childdocforward}
The command |\childdocforward| redirects processing to
another source file:
%
\begin{center}
\begin{tabular}{l}
|\input{childdoc.def}|\\
|\childdocforward[|\textit{main}|]{|\textit{dest}|}|\\
\end{tabular}
\end{center}
%
The argument \textit{dest} is the destination file
(without extension).
It should be the main file or one of the child files.
Note that further \textsf{childdoc} directives
such as |\childdocof| and |\childdocforward|
in the indicated file will be processed in this form.
The optional argument \textit{main}
passes on directly to the main file \textit{main}
while pretending to compile the child \textit{dest}.
This form behaves as if \textit{dest}
issues |\childdocof{|\textit{main}|}| right away,
and no further \textsf{childdoc} directives will be processed.

%%%%%%%%%%%%%%%%%%%%%%%%%%%%%%%%%%%%%%%%
\DescribeMacro{\...prefix}
In the alternative form |\childdocforwardprefix|,
%
\begin{center}
\begin{tabular}{l}
|\input{childdoc.def}|\\
|\childdocforwardprefix[|\textit{main}|]{|\textit{prefix}|}{|\textit{dest}|}|
\end{tabular}
\end{center}
%
the destination file is determined by a pattern
depending on the current file:
To make this work, the current file must be called
`{\textit{prefix}\hspace{0.2em}\textit{suffix}}'
with \textit{prefix} matching precisely the argument.
Processing is then passed on to the file
`{\textit{dest}\hspace{0.2em}\textit{suffix}}'.
Surely, the same effect is achieved by
directly specifying the
argument `{\textit{dest}\hspace{0.2em}\textit{suffix}}'
in the first form.
However, that requires to set up a different file
for each child. With the alternative form of the command
all these files can have exactly the same content
which simplifies setting them up and maintaining them.

For example, the following file |draft.tex|
with a compilation flag |\version| as described in \secref{sec:flags}
compiles the main document as a draft:
%
\begin{center}
\begin{tabular}{l}
|\def\version{draft}|\\
|\input{childdoc.def}|\\
|\childdocforward{|\textit{main}|}|
\end{tabular}
\end{center}
%
Likewise, the following files |final|\textit{nn}|.tex|
compile the final version of the child document
|child|\textit{nn}|.tex|:
%
\begin{center}
\begin{tabular}{l}
|\def\version{final}|\\
|\input{childdoc.def}|\\
|\childdocforwardprefix{final}{child}|
\end{tabular}
\end{center}
%

Note that when several versions of a main file and/or of each child file
are to be generated, it may be convenient to set up a |Makefile| or
shell script to automatise the process.

%%%%%%%%%%%%%%%%%%%%%%%%%%%%%%%%%%%%%%%%%%%%%%%%%%%%%%%%%%%%%%%%%%%%%%%%%%%%%%%%
\subsection{Command Line Processing}
\label{sec:commandline}

The effect of redirection files can also be achieved by invoking
the \LaTeX{} compiler with a more elaborate command line.
Most conveniently this should be done as part
of a shell script or a |Makefile|.

When using \textsf{childdoc} in the main file, the following
command lines effectively perform a redirection
(note that depending on the shell being used,
backslashes may have to be doubled: `|\|' $\to$ `|\\|'):
%
\begin{center}
|... -jobname "|\textit{target}|" |\\|"|[\textit{flags}]%
|\input{childdoc.def}\childdocforward[|\textit{main}|]{|\textit{dest}|}"|
\end{center}
%
Here \textit{target} is the name of the output file,
\textit{main} is the name of the main file
and \textit{dest} is the name of the main or child file to be processed
(all filenames without extensions).
The optional argument \textit{main} can be omitted
if \textit{main} matches \textit{dest}.
Optionally, compilation \textit{flags} can be defined via |\def| commands.
This command line makes the \TeX{} engine believe
it is compiling the file \textit{target}
whose content is specified as the latter parameter.
The provided code then forwards the processing to
\textit{main} or \textit{dest} as described in \secref{sec:forward}.

%%%%%%%%%%%%%%%%%%%%%%%%%%%%%%%%%%%%%%%%%%%%%%%%%%%%%%%%%%%%%%%%%%%%%%%%%%%%%%%%
\subsection{Include by Input}
\label{sec:input}

Including child documents by |\include| has some restrictions by design.
Most notably, the content of a child document always occupies
its own set of pages; pages cannot be shared between child documents.
Usually, this behaviour makes perfect sense
because each child document contain an essential part of the document.
However, in some situations it may be desirable to compose
a document from a collection of parts
without having mandatory page breaks between then.
For this case, the package
provides a mechanism to include parts
by |\input| which can also be processed individually.
However, by construction this mechanism
requires manual handling of the content to be output.

%%%%%%%%%%%%%%%%%%%%%%%%%%%%%%%%%%%%%%%%
\DescribeMacro{\ifchilddocmanual}
The main file should be prepared as usual, see \secref{sec:include}.
However, the document body must make a distinction
between processing of an individual part and of the main document, e.g.:
%
\begin{center}
\begin{tabular}{l}
|\ifchilddocmanual|\\
|\input{\childdocname}|\\
|\||else|\\
\textit{document body with }|\input{|\textit{part}|}|\\
|\||fi|
\end{tabular}
\end{center}
%
The conditional |\ifchilddocmanual| is true whenever
a part to be included by |\input| is being compiled,
and the name of the part is stored in |\childdocname|.

%%%%%%%%%%%%%%%%%%%%%%%%%%%%%%%%%%%%%%%%
\DescribeMacro{\childdocby}
Each part to be included by |\input| should start with:
%
\begin{center}
\begin{tabular}{l}
|\input{childdoc.def}|\\
|\childdocby{|\textit{main}|}|\\
\end{tabular}
\end{center}
%
The directive |\childdocby| is similar to |\childdocof|
described in \secref{sec:include},
but the subsequent selection of content must be done manually.
To that end, both |\ifchilddoc| and |\ifchilddocmanual|
will be true upon processing of a part,
and the name of the part is stored in |\childdocname|.
Note that |\jobname| will be set to the filename of the current part
so that each part receives an individual |.aux| file
that does not interfere with the |.aux| file(s) of the main document.
This behaviour can be altered by the alternative form
|\childdocby[*]{|\textit{main}|}| (with a non-empty optional argument)
which uses the |.aux| file of the main document
by setting |\jobname| to \textit{main}.

%%%%%%%%%%%%%%%%%%%%%%%%%%%%%%%%%%%%%%%%%%%%%%%%%%%%%%%%%%%%%%%%%%%%%%%%%%%%%%%%
\subsection{Driver Development}
\label{sec:driver}

The \textsf{childdoc} mechanism can also be use for the development
of definition files such as \LaTeX{} styles or classes.
This case differs from the above setup with multiple parts
included by |\include| in that no |\includeonly| should be invoked.
This can be achieved by starting the include file
(before |\ProvidesPackage|) with:
%
\begin{center}
\begin{tabular}{l}
|\input{childdoc.def}|\\
|\childdocforward{|\textit{main}|}|\\
\end{tabular}
\end{center}
%
or alternatively with:
%
\begin{center}
\begin{tabular}{l}
|\input{childdoc.def}|\\
|\childdocby{|\textit{main}|}|\\
\end{tabular}
\end{center}
%
Both forms have slightly different effects as described above.
The main file is prepared as usual, see \secref{sec:include}.

%%%%%%%%%%%%%%%%%%%%%%%%%%%%%%%%%%%%%%%%%%%%%%%%%%%%%%%%%%%%%%%%%%%%%%%%%%%%%%%%
\subsection{Legacy Detection}
\label{sec:detection}

The directive |\childdocmain| in the main file can detect
whether the complete document or merely a child is to be compiled
even without using the directive |\childdocof|.
This method is deprecated because it is less robust
and there is no compelling reason to use it;
it is merely provided for backward compatibility
and it may be removed in future versions.

If the detection mechanism is to be used,
it is mandatory to correctly specify
the filename of the main file as the argument of |\childdocmain|:
%
\begin{center}
\begin{tabular}{l}
|\input{childdoc.def}|\\
|\childdocmain{|\textit{main}|}|\\
\end{tabular}
\end{center}
%
If |\jobname| does not match the argument \textit{main} of |\childdocmain|,
it is assumed that |\jobname| points to the child file to be compiled.
When using |\childdocmain| with the main file specified as argument,
it suffices to start a child file
with just |\input{|\textit{main}|}|
without loading of the package and using |\childdocof|.
If instead all processing is done
with the appropriate \textsf{childdoc} directives,
the argument of \textit{main} of |\childdocmain| can be empty.

An alternative version of the command line processing described
in \secref{sec:commandline} using the detection mechanism reads:
%
\begin{center}
|... -jobname "|\textit{target}|" "|[\textit{flags}]%
[|\def\jobname{|\textit{dest}|}|]|\input{|\textit{main}|}"|
\end{center}

%%%%%%%%%%%%%%%%%%%%%%%%%%%%%%%%%%%%%%%%%%%%%%%%%%%%%%%%%%%%%%%%%%%%%%%%%%%%%%%%
\subsection{Manual Code}
\label{sec:manual}

In case one cannot be certain whether the definitions file |childdoc.def|
is installed on the target \TeX{} distribution
and one prefers not to ship it,
it is conceivable to paste a few relevant commands into the sources.

To that end, drop all statements |\input{childdoc.def}|
and perform the replacements as outlined below.
Instead of |\childdocmain{|\textit{main}|}| add the following code
to the top of the main file:
%
\begin{center}
\begin{tabular}{l}
|\||ifdefined\childdocname\endinput\||fi\newif\ifchilddoc|\\
|\edef\childdocname{\scantokens\expandafter{\jobname\noexpand}}|\\
|\def\childdocmain{|\textit{main}|}\||ifx\childdocmain\childdocname\||else|\\
|\childdoctrue\includeonly{\childdocname}\let\jobname\childdocmain\||fi|\\
\end{tabular}
\end{center}
%
Instead of |\childdocof{|\textit{main}|}| just include the main file
at the top of each child file:
%
\begin{center}
|\input{|\textit{main}|}|
\end{center}
%
A simple redirection |\childdocforward{|\textit{dest}|}| is achieved by:
%
\begin{center}
|\def\jobname{|\textit{dest}|}\input{\jobname}|
\end{center}
%
The redirection with prefix
|\childdocforwardprefix[|\textit{prefix}|]{|\textit{dest}|}|
is accomplished by:
%
\begin{center}
\begin{tabular}{l}
|{\edef\jobname{\scantokens\expandafter{\jobname\noexpand}}|\\
|\def\redirectjob |\textit{prefix}|#1~~~{\gdef\jobname{|\textit{dest}|#1}}|\\
|\expandafter\redirectjob\jobname~~~}\input{\jobname}|
\end{tabular}
\end{center}

In an alternative approach,
child documents can be compiled by a specific command line
without additional code or specific definitions:
%
\begin{center}
|... -jobname "|\textit{target}|" "|[\textit{flags}]%
|\includeonly{|\textit{dest}|}\input{|\textit{main}|}"|
\end{center}
%

%%%%%%%%%%%%%%%%%%%%%%%%%%%%%%%%%%%%%%%%%%%%%%%%%%%%%%%%%%%%%%%%%%%%%%%%%%%%%%%%
%%%%%%%%%%%%%%%%%%%%%%%%%%%%%%%%%%%%%%%%%%%%%%%%%%%%%%%%%%%%%%%%%%%%%%%%%%%%%%%%
\section{Information}

%%%%%%%%%%%%%%%%%%%%%%%%%%%%%%%%%%%%%%%%%%%%%%%%%%%%%%%%%%%%%%%%%%%%%%%%%%%%%%%%
\subsection{Copyright}

Copyright \copyright{} 2017--2018 Niklas Beisert

This work may be distributed and/or modified under the
conditions of the \LaTeX{} Project Public License, either version 1.3
of this license or (at your option) any later version.
The latest version of this license is in
  \url{http://www.latex-project.org/lppl.txt}
and version 1.3 or later is part of all distributions of \LaTeX{}
version 2005/12/01 or later.

This work has the LPPL maintenance status `maintained'.

The Current Maintainer of this work is Niklas Beisert.

This work consists of the files |README.txt|, |childdoc.ins| and |childdoc.dtx|
as well as the derived files |childdoc.def|, |cdocsamp.tex|
with |cdocsch1.tex|, |cdocsch2.tex|, |cdocspt3.tex|, |cdocspt4.tex|,
|cdocsdrf.tex|, |cdocsfn1.tex|, |cdocsfn2.tex|
as well as |childdoc.pdf|.

%%%%%%%%%%%%%%%%%%%%%%%%%%%%%%%%%%%%%%%%%%%%%%%%%%%%%%%%%%%%%%%%%%%%%%%%%%%%%%%%
\subsection{Files and Installation}

The package consists of the files:
%
\begin{center}
\begin{tabular}{ll}
    |README.txt|   & readme file \\
    |childdoc.ins| & installation file \\
    |childdoc.dtx| & source file \\
    |childdoc.def| & definition file \\
    |cdocsamp.tex| & sample main file \\
    |cdocsch1.tex| & sample include file \\
    |cdocsch2.tex| & sample include file \\
    |cdocspt3.tex| & sample part file \\
    |cdocspt4.tex| & sample part file \\
    |cdocsdrf.tex| & sample redirection file \\
    |cdocsfn1.tex| & sample redirection file \\
    |cdocsfn2.tex| & sample redirection file \\
    |childdoc.pdf| & manual
\end{tabular}
\end{center}
%
The distribution consists of the files
|README.txt|, |childdoc.ins| and |childdoc.dtx|.
%
\begin{itemize}
\item
Run (pdf)\LaTeX{} on |childdoc.dtx|
to compile the manual |childdoc.pdf| (this file).
\item
Run \LaTeX{} on |childdoc.ins| to create the definitions file |childdoc.def|
and the sample |cdocsamp.tex| with include files
|cdocsch1.tex|, |cdocsch2.tex|, |cdocspt3.tex|, |cdocspt4.tex|,
|cdocsdrf.tex|, |cdocsfn1.tex|, |cdocsfn2.tex|.
Then copy the file |childdoc.def| to an appropriate directory of your \LaTeX{}
distribution, e.g.\ \textit{texmf-root}|/tex/latex/childdoc|.
\end{itemize}

%%%%%%%%%%%%%%%%%%%%%%%%%%%%%%%%%%%%%%%%%%%%%%%%%%%%%%%%%%%%%%%%%%%%%%%%%%%%%%%%
\subsection{Related CTAN Packages}

There are several other packages which offer a similar functionality:
%
\begin{itemize}
\item
The packages
\href{http://ctan.org/pkg/docmute}{\textsf{docmute}},
\href{http://ctan.org/pkg/includex}{\textsf{includex}} and
\href{http://ctan.org/pkg/standalone}{\textsf{standalone}}
provide commands to include only the document body of
a child file thus allowing both files to be compiled individually.
\item
The packages \href{http://ctan.org/pkg/subdocs}{\textsf{subdocs}}
and \href{http://ctan.org/pkg/subfiles}{\textsf{subfiles}}
provide structures in which the main and child documents can be
encapsulated and allowing them to be compiled individually.
The inclusion mechanism is different from the conventional |\include|.
\item
The package \href{http://ctan.org/pkg/combine}{\textsf{combine}}
is an elaborate solution to combine several documents into one.
\end{itemize}
%
See also the CTAN topic \href{http://ctan.org/topic/subdocs}{\textsf{subdocs}}
for further related packages.
The present package differs from the above solutions in that
a document structure constructed with the conventional |\include| mechanism
just needs two extra commands at the top of every file
such that all constituent files can be compiled individually.

%%%%%%%%%%%%%%%%%%%%%%%%%%%%%%%%%%%%%%%%%%%%%%%%%%%%%%%%%%%%%%%%%%%%%%%%%%%%%%%%
%\subsection{Feature Suggestions}
%
%The following is a list of features which may be useful for future
%versions of this package:
%%
%\begin{itemize}
%\item
%\ldots
%\end{itemize}

%%%%%%%%%%%%%%%%%%%%%%%%%%%%%%%%%%%%%%%%%%%%%%%%%%%%%%%%%%%%%%%%%%%%%%%%%%%%%%%%
\subsection{Revision History}

%%%%%%%%%%%%%%%%%%%%%%%%%%%%%%%%%%%%%%%%
\paragraph{v2.0:} 2018/12/30

\begin{itemize}
\item
immediate forward processing
\item
added |\childdocby| mechanism
\item
manual restructured
\end{itemize}

%%%%%%%%%%%%%%%%%%%%%%%%%%%%%%%%%%%%%%%%
\paragraph{v1.6:} 2018/01/17

\begin{itemize}
\item
application for development of include files
\item
corrections to manual
\end{itemize}

%%%%%%%%%%%%%%%%%%%%%%%%%%%%%%%%%%%%%%%%
\paragraph{v1.5:} 2017/05/21

\begin{itemize}
\item
more complete structuring introduced
\item
|\childdocof| introduced
\item
|\childdoc| renamed to |\childdocmain|
\item
|\childredirect| renamed to |\childdocforward| and |\childdocforwardprefix|
and functionality expanded
\end{itemize}

%%%%%%%%%%%%%%%%%%%%%%%%%%%%%%%%%%%%%%%%
\paragraph{v1.0:} 2017/04/27

\begin{itemize}
\item
manual and install package
\item
first version published on CTAN
\end{itemize}

%%%%%%%%%%%%%%%%%%%%%%%%%%%%%%%%%%%%%%%%
\paragraph{v0.6:} 2017/04/26

\begin{itemize}
\item
redirection mechanism added
\end{itemize}

%%%%%%%%%%%%%%%%%%%%%%%%%%%%%%%%%%%%%%%%
\paragraph{v0.5:} 2017/04/26

\begin{itemize}
\item
functionality in definition file
\end{itemize}


%%%%%%%%%%%%%%%%%%%%%%%%%%%%%%%%%%%%%%%%%%%%%%%%%%%%%%%%%%%%%%%%%%%%%%%%%%%%%%%%
%%%%%%%%%%%%%%%%%%%%%%%%%%%%%%%%%%%%%%%%%%%%%%%%%%%%%%%%%%%%%%%%%%%%%%%%%%%%%%%%
%%%%%%%%%%%%%%%%%%%%%%%%%%%%%%%%%%%%%%%%%%%%%%%%%%%%%%%%%%%%%%%%%%%%%%%%%%%%%%%%
\appendix

\settowidth\MacroIndent{\rmfamily\scriptsize 000\ }

 \DocInput{childdoc.dtx}

\end{document}
%</driver>
% \fi
%
% %%%%%%%%%%%%%%%%%%%%%%%%%%%%%%%%%%%%%%%%%%%%%%%%%%%%%%%%%%%%%%%%%%%%%%%%%%%%%%
% %%%%%%%%%%%%%%%%%%%%%%%%%%%%%%%%%%%%%%%%%%%%%%%%%%%%%%%%%%%%%%%%%%%%%%%%%%%%%%
% \section{Sample}
%\iffalse
%<*samplemain>
%\fi
%
% The following presents a sample document
% with two chapters, two parts, a title page,
% a compile flag as well as three forwarding files to set the flag.
% It consists of eight |.tex| files:
% \begin{center}
% \begin{tabular}{ll}
% |cdocsamp.tex|&main file\\
% |cdocsch1.tex|&include file for chapter 1\\
% |cdocsch2.tex|&include file for chapter 2\\
% |cdocspt3.tex|&include file for part 3\\
% |cdocspt4.tex|&include file for part 4\\
% |cdocsdrf.tex|&forwarding file for main file in draft mode\\
% |cdocsfi1.tex|&forwarding file for final version of chapter 1\\
% |cdocsfi2.tex|&forwarding file for final version of chapter 2\\
% \end{tabular}
% \end{center}
% Each of the eight files can be compiled directly by the \LaTeX{} compiler.
%
% %%%%%%%%%%%%%%%%%%%%%%%%%%%%%%%%%%%%%%
% \paragraph{Main File.}
%
% The main file is called |cdocsamp.tex|.
%
% Load the \textsf{childdoc} definitions and
% declare the filename for the main document:
%    \begin{macrocode}
\input{childdoc.def}
\childdocmain{}
%    \end{macrocode}

% Optional override for |\version| flag:
%    \begin{macrocode}
%%\ifchilddoc\else\providecommand{\version}{draft}\fi
%    \end{macrocode}

% Define the default values for the |\version| flag
% (|final| for the main file and |draft| for childs):
%    \begin{macrocode}
\ifchilddoc
\providecommand{\version}{draft}
\else
\providecommand{\version}{final}
\fi
%    \end{macrocode}

% Load the standard document class:
%    \begin{macrocode}
\documentclass[12pt]{article}
%    \end{macrocode}

% Start the document body:
%    \begin{macrocode}
\begin{document}
%    \end{macrocode}

% Declare a title page.
% Print title, part of document being processed and version flag:
%    \begin{macrocode}
\addtocounter{page}{-1}
\begin{center}
{\LARGE\bfseries{}childdoc example\par}
\vspace{1cm}
\ifchilddoc
\ifchilddocmanual part\else chapter\fi:
`\childdocname' of `\childdocjob'\par
\else
main document: `\childdocjob'\par
\fi
version: \version\par
\end{center}
\newpage
%    \end{macrocode}

% Manually include selected file,
% otherwise process as usual:
%    \begin{macrocode}
\ifchilddocmanual
\section*{part `\childdocname'}
\input{\childdocname}
\else
%    \end{macrocode}

% Include the two chapters:
%    \begin{macrocode}
\include{cdocsch1}
\include{cdocsch2}
%    \end{macrocode}

% Include the two parts unless only chapters should be displayed:
%    \begin{macrocode}
\ifchilddoc\else
\section{part three}
\input{cdocspt3}
\section{part four}
\input{cdocspt4}
\fi
%    \end{macrocode}

% Process as usual until here:
%    \begin{macrocode}
\fi
%    \end{macrocode}

% End of document body:
%    \begin{macrocode}
\end{document}
%    \end{macrocode}
%\iffalse
%</samplemain>
%\fi
%
% %%%%%%%%%%%%%%%%%%%%%%%%%%%%%%%%%%%%%%
% \paragraph{Chapter Include Files.}
%
% The include files are called |cdocsch1.tex| and |cdocsch2.tex|.
%
%\iffalse
%<*samplechap1|samplechap2>
%\fi

% Optional override for |\version| flag:
%    \begin{macrocode}
%%\providecommand{\version}{final}
%    \end{macrocode}

% Include the main document:
%    \begin{macrocode}
\input{childdoc.def}
\childdocof{cdocsamp}
%    \end{macrocode}

%\iffalse
%</samplechap1|samplechap2>
%\fi
%
%\iffalse
%<*samplechap1>
%\fi
% Some text for chapter 1:
%    \begin{macrocode}
\section{one}
some text in chapter one
%    \end{macrocode}

%\iffalse
%</samplechap1>
%\fi
% Some text for chapter 2:
%\iffalse
%<*samplechap2>
%\fi
%    \begin{macrocode}
\section{two}
more text in chapter two
%    \end{macrocode}

%\iffalse
%</samplechap2>
%\fi
%
% %%%%%%%%%%%%%%%%%%%%%%%%%%%%%%%%%%%%%%
% \paragraph{Part Include Files.}
%
% The include files are called |cdocspt3.tex| and |cdocspt4.tex|.
%
%\iffalse
%<*samplepart3|samplepart4>
%\fi

% Optional override for |\version| flag:
%    \begin{macrocode}
%%\providecommand{\version}{final}
%    \end{macrocode}

% Include the main document:
%    \begin{macrocode}
\input{childdoc.def}
\childdocby{cdocsamp}
%    \end{macrocode}

%\iffalse
%</samplepart3|samplepart4>
%\fi
%
%\iffalse
%<*samplepart3>
%\fi
% Some text for part 3:
%    \begin{macrocode}
some text in part three
%    \end{macrocode}

%\iffalse
%</samplepart3>
%\fi
% Some text for part 4:
%\iffalse
%<*samplepart4>
%\fi
%    \begin{macrocode}
more text in part four
%    \end{macrocode}

%\iffalse
%</samplepart4>
%\fi
%
% %%%%%%%%%%%%%%%%%%%%%%%%%%%%%%%%%%%%%%
% \paragraph{Forwarding for a Complete Draft.}
%
% The following forwarding file |cdocsdrf.tex|
% compiles the main document in draft mode:
%\iffalse
%<*sampledraft>
%\fi
%    \begin{macrocode}
\def\version{draft}
\input{childdoc.def}
\childdocforward{cdocsamp}
%    \end{macrocode}

%\iffalse
%</sampledraft>
%\fi
%
% %%%%%%%%%%%%%%%%%%%%%%%%%%%%%%%%%%%%%%
% \paragraph{Forwarding for Final Version of the Chapters.}
%
% The following forwarding files |cdocsfn1.tex| and |cdocsfn2.tex|
% (with identical content)
% compile the final versions of the child documents
% |cdocsch1.tex| and |cdocsch2.tex|, respectively:
%\iffalse
%<*samplefinal>
%\fi
%    \begin{macrocode}
\def\version{final}
\input{childdoc.def}
\childdocforwardprefix[cdocsamp]{cdocsfn}{cdocsch}
%    \end{macrocode}

%\iffalse
%</samplefinal>
%\fi
%
% %%%%%%%%%%%%%%%%%%%%%%%%%%%%%%%%%%%%%%
% \paragraph{Command Line Processing.}
%
% The following three command lines generate the output files
% |cdocscld|, |cdocscl1| and |cdocscl2|
% which should be identical to
% |cdocsdrf|, |cdocsch1| and |cdocsfn2|, respectively:
% \begin{center}
% \begin{tabular}{l}
% |latex -jobname cdocscld \|\\
% |  "\def\version{draft}\input{childdoc.def}\childdocforward{cdocsamp}"|\\
% |latex -jobname cdocscl1 \|\\
% |  "\input{childdoc.def}\childdocforward[cdocsamp]{cdocsch1}"|\\
% |latex -jobname cdocscl2 \|\\
% |  "\def\version{final}\input{childdoc.def}\childdocforward{cdocsch2}"|
% \end{tabular}
% \end{center}
% Note that the trailing backslash on each first line
% merely continues the input to the second line
% (for convenient cut ant paste).
% Furthermore, the command |latex| can be replaced by any
% of its alternative versions such as |pdflatex|.
%
% %%%%%%%%%%%%%%%%%%%%%%%%%%%%%%%%%%%%%%%%%%%%%%%%%%%%%%%%%%%%%%%%%%%%%%%%%%%%%%
% %%%%%%%%%%%%%%%%%%%%%%%%%%%%%%%%%%%%%%%%%%%%%%%%%%%%%%%%%%%%%%%%%%%%%%%%%%%%%%
% \section{Implementation}
%\iffalse
%<*package>
%\fi
%
% This section describes the definitions file |childdoc.def|.

% The definitions cannot be loaded using |\usepackage| or |\RequirePackage|
% which has a mechanism to prevent loading a style file more than once.
% When loading the definitions by means of |\input|
% multiple instances have to be prevented manually:
%\iffalse
%This code needs to be before the `\ProvidesFile' directive
%which is defined at the beginning of this file.
%Therefore it is also placed there and commented out here.
%</package>
%<*discard>
%\fi
%    \begin{macrocode}
\ifdefined\childdocmain\endinput\fi
%    \end{macrocode}
%\iffalse
%</discard>
%<*package>
%\fi
%
% \macro{\ifchilddoc}
% \macro{\ifchilddocmanual}
% The conditional |\ifchilddoc| tells whether a
% child (true) or main (false) document is being compiled.
% The conditional |\ifchilddocmanual| tells whether
% the |\includeonly| mechanism is used (false) or
% the selection of child files must be performed manually (true).
% The definitions initialise to false:
%    \begin{macrocode}
\newif\ifchilddoc
\newif\ifchilddocmanual
%    \end{macrocode}

% \macro{\childdocname}
% \macro{\childdocjob}
% The macro |\childdocname| stores the name of the main document
% to be compiled. The macro |\childdocjob| stores the name of
% the document on which the \LaTeX{} compiler was originally invoked.
% The content of |\jobname| cannot be compared
% to filenames specified in the source due to different catcodes.
% The following code rescans |\jobname|, stores the result
% in |\childdocname| and saves a copy in |\childdocjob|:
%    \begin{macrocode}
\edef\childdocname{\scantokens\expandafter{\jobname\noexpand}}
\let\childdocjob\childdocname
%    \end{macrocode}

% \macro{\childdocdisable}
% The macro |\childdocdisable| prevents the main file
% from being processed more than once.
% At this stage, the main document command |\childdocmain|
% is assumed to be called once again where it should do nothing.
% Any subsequent call to it should prevent
% a secondary processing of the main document
% It overwrites the forwarding commands
% |\childdocof| and |\childdocforward|
% with empty macros to prevent further inclusions of the main document:
%    \begin{macrocode}
\newcommand{\childdocdisable}
{
  \renewcommand{\childdocmain}[1]{\renewcommand{\childdocmain}[1]{\endinput}}
  \renewcommand{\childdocof}[1]{}
  \renewcommand{\childdocby}[2][]{}
  \renewcommand{\childdocforward}[2][]{}
  \renewcommand{\childdocdisable}{}
}
%    \end{macrocode}

% \macro{\childdocmain}
% The macro |\childdocmain| is to be called at the top of the main file
% with nothing or the main filename (without extension) as argument.
% First, it breaks loops.
% If the argument is not empty and does not match |\childdocname|
% (which is set by the first inclusion of |childdoc.def|),
% |\ifchilddoc| is set to true, |\includeonly| is applied to the child file
% and |\jobname| is set to the main file
% (for proper handling of |.aux| files):
%    \begin{macrocode}
\newcommand{\childdocmain}[1]
{
  \childdocdisable\childdocmain{}
  \if?#1?\else
    \begingroup
      \def\childdoctmp{#1}
      \ifx\childdoctmp\childdocname
        \def\childdoctmp{}
      \else
        \def\childdoctmp
        {
          \childdoctrue
          \includeonly{\childdocname}
          \def\childdocjob{#1}
          \def\jobname{#1}
        }
      \fi
      \expandafter
    \endgroup
    \childdoctmp
  \fi
}
%    \end{macrocode}

% \macro{\childdocof}
% The command |\childdocof| redirects
% compilation to the main file |#1|.
%    \begin{macrocode}
\newcommand{\childdocof}[1]
{
  \childdocdisable
  \childdoctrue
  \includeonly{\childdocname}
  \def\jobname{#1}
  \def\childdocjob{#1}
  \input{#1}
}
%    \end{macrocode}

% \macro{\childdocby}
% The command |\childdocby| ....
%    \begin{macrocode}
\newcommand{\childdocby}[2][]
{
  \childdocdisable
  \childdoctrue
  \childdocmanualtrue
  \if?#1?\else
    \def\jobname{#2}
  \fi
  \def\childdocjob{#2}
  \input{#2}
  \endinput
}
%    \end{macrocode}

% \macro{\childdocforward}
% The command |\childdocforward| redirects
% compilation to the main file or
% (if the optional argument is given) a child file.
% Parameters are set as if the main file
% or a child file starting with |\childdocof| was compiled.
% Then compilation is handed over to the main file:
%    \begin{macrocode}
\newcommand{\childdocforward}[2][]
{
  \begingroup
    \if?#1?
      \def\childdoctmp
      {
        \def\childdocname{#2}
        \def\childdocjob{#2}
        \def\jobname{#2}
        \input{#2}
        \endinput
      }
    \else
      \def\childdoctmp
      {
        \childdocdisable
        \def\childdocname{#2}
        \childdoctrue
        \includeonly{#2}
        \def\childdocjob{#1}
        \def\jobname{#1}
        \input{#1}
        \endinput
      }
    \fi
    \expandafter
  \endgroup
  \childdoctmp
}
%    \end{macrocode}

% \macro{\childdocforwardprefix}
% The command |\childdocforwardprefix| redirects
% compilation to the main or a child file by means of a pattern.
% The prefix |#1| in the current filename is replaced by |#2|
% and the suffix of the current filename is kept
% (it is assumed that the filename does not contain the substring `|~~~|'
% which is used as a delimiter).
% Compilation is handed over to the new file by |\childdocforward|:
%    \begin{macrocode}
\newcommand{\childdocforwardprefix}[3][]
{
  \begingroup
    \def\childdocextract #2##1~~~{\def\childdoctmp{\childdocforward[#1]{#3##1}}}
    \expandafter\childdocextract\childdocname~~~
    \expandafter
  \endgroup
  \childdoctmp
}
%    \end{macrocode}

% \macro{\childdoc}
% The deprecated macro |\childdoc| is a legacy version of |\childdocmain|:
%    \begin{macrocode}
\newcommand{\childdoc}{\childdocmain}
%    \end{macrocode}

% \macro{\childdocredirect}
% The deprecated macro |\childdocredirect| is a legacy version
% of |\childdocforward| and |\childdocforwardprefix|:
%    \begin{macrocode}
\newcommand{\childdocredirect}[2][]
{
  \begingroup
    \if?#1?
      \def\childdoctmp{\childdocforward{#2}}
    \else
      \def\childdoctmp{\childdocforwardprefix{#1}{#2}}
    \fi
    \expandafter
  \endgroup
  \childdoctmp
}
%    \end{macrocode}

%\iffalse
%</package>
%\fi
%
\endinput

\childdocforward{cdocsamp}
%    \end{macrocode}

%\iffalse
%</sampledraft>
%\fi
%
% %%%%%%%%%%%%%%%%%%%%%%%%%%%%%%%%%%%%%%
% \paragraph{Forwarding for Final Version of the Chapters.}
%
% The following forwarding files |cdocsfn1.tex| and |cdocsfn2.tex|
% (with identical content)
% compile the final versions of the child documents
% |cdocsch1.tex| and |cdocsch2.tex|, respectively:
%\iffalse
%<*samplefinal>
%\fi
%    \begin{macrocode}
\def\version{final}
% \iffalse
%
% childdoc.dtx Copyright (C) 2017-2018 Niklas Beisert
%
% This work may be distributed and/or modified under the
% conditions of the LaTeX Project Public License, either version 1.3
% of this license or (at your option) any later version.
% The latest version of this license is in
%   http://www.latex-project.org/lppl.txt
% and version 1.3 or later is part of all distributions of LaTeX
% version 2005/12/01 or later.
%
% This work has the LPPL maintenance status `maintained'.
%
% The Current Maintainer of this work is Niklas Beisert.
%
% This work consists of the files childdoc.dtx and childdoc.ins
% and the derived files childdoc.def and cdocsamp.tex with
% cdocsch1.tex, cdocsch2.tex, cdocsdrf.tex, cdocsfn1.tex, cdocsfn2.tex.
%
%<package>\ifdefined\childdocmain\endinput\fi
%<package>\ProvidesFile{childdoc.def}[2018/12/30 v2.0 child document driver]
%<samplemain>\ProvidesFile{cdocsamp.tex}[2018/12/30 v2.0 sample for childdoc]
%<*driver>
%\ProvidesFile{childdoc.drv}[2018/12/30 v2.0 childdoc reference manual file]
\PassOptionsToClass{10pt,a4paper}{article}
\documentclass{ltxdoc}

\usepackage[margin=35mm]{geometry}
\usepackage{hyperref}
\usepackage{hyperxmp}
\usepackage[usenames]{color}

\hypersetup{colorlinks=true}
\hypersetup{pdfstartview=FitH}
\hypersetup{pdfpagemode=UseNone}
\hypersetup{pdfsource={}}
\hypersetup{pdflang={en-UK}}
\hypersetup{pdfcopyright={Copyright 2017-2018 Niklas Beisert.
  This work may be distributed and/or modified under the
  conditions of the LaTeX Project Public License, either version 1.3
  of this license or (at your option) any later version.}}
\hypersetup{pdflicenseurl={http://www.latex-project.org/lppl.txt}}
\hypersetup{pdfcontactaddress={ETH Zurich, ITP, HIT K,
  Wolfgang-Pauli-Strasse 27}}
\hypersetup{pdfcontactpostcode={8093}}
\hypersetup{pdfcontactcity={Zurich}}
\hypersetup{pdfcontactcountry={Switzerland}}
\hypersetup{pdfcontactemail={nbeisert@itp.phys.ethz.ch}}
\hypersetup{pdfcontacturl={http://people.phys.ethz.ch/\xmptilde nbeisert/}}

\newcommand{\secref}[1]{\hyperref[#1]{section \ref*{#1}}}

\parskip1ex
\parindent0pt
\let\olditemize\itemize
\def\itemize{\olditemize\parskip0pt}

\begin{document}

\title{The \textsf{childdoc} Package}
\hypersetup{pdftitle={The childdoc Package}}
\author{Niklas Beisert\\[2ex]
  Institut f\"ur Theoretische Physik\\
  Eidgen\"ossische Technische Hochschule Z\"urich\\
  Wolfgang-Pauli-Strasse 27, 8093 Z\"urich, Switzerland\\[1ex]
  \href{mailto:nbeisert@itp.phys.ethz.ch}
  {\texttt{nbeisert@itp.phys.ethz.ch}}}
\hypersetup{pdfauthor={Niklas Beisert}}
\hypersetup{pdfsubject={Manual for the LaTeX2e Package childdoc}}
\date{30 December 2018, \textsf{v2.0}}
\maketitle

\begin{abstract}\noindent
\textsf{childdoc} is a \LaTeXe{} package
that enables the direct compilation
of document sections included by |\include|
to individual files.
\end{abstract}

\begingroup
\parskip0ex
\tableofcontents
\endgroup

%%%%%%%%%%%%%%%%%%%%%%%%%%%%%%%%%%%%%%%%%%%%%%%%%%%%%%%%%%%%%%%%%%%%%%%%%%%%%%%%
%%%%%%%%%%%%%%%%%%%%%%%%%%%%%%%%%%%%%%%%%%%%%%%%%%%%%%%%%%%%%%%%%%%%%%%%%%%%%%%%
\section{Introduction}

\LaTeX{} provides a mechanism to structure a large document (such as a book)
into a main file and several child files (containing the chapters)
using the |\include| command.
This mechanism is beneficial for documents
which span hundreds of pages in order to
make the source file(s) more manageable.
Moreover, compilation can be restricted to
selected child files by means of the |\includeonly| command.
The latter feature can be used to reduce the compilation time while editing
(this was significantly more useful in the earlier days of \LaTeX{})
or to generate a smaller document which is easier to navigate.
Another application of |\includeonly| is to generate
documents consisting of selected parts of the complete document.

However, there are a few drawbacks of the plain |\include| mechanism:
\begin{itemize}
\item
The child files cannot be compiled on their own,
they can only be compiled via the main file.
A naive editing environment
(such as a text editor with an option
to have the current file processed by \LaTeX)
may require one to switch to the main file before compiling;
attempting to compile the child file produces errors.
\item
The main file must be modified (each time)
to adjust the |\includeonly| command
to the present needs. This easily leaves the main file in a messy state.
\item
The generated document will always carry the filename
of the main document. This is inconvenient if
several child files are to be compiled and
to be kept for distribution.
\end{itemize}

The present package provides a simple interface
to make child files individually compilable by \LaTeX{}.
Compiling a child file then has the same effect as compiling
the main file with an |\includeonly| command
to select the appropriate child.
Moreover the generated document will carry the name of the child
rather than the main file.
This resolves all three above issues.

This feature is meant to make the editing of books,
thesis documents and lecture notes somewhat more convenient.
However, the package can also be used efficiently for
composing a series of documents (such as exercise sheets)
which are typically distributed individually.
It then assists the author in generating the individual documents
(potentially in different versions)
as well as a document containing the collected series.
Another application is in developing style files
or other kinds of included material
where compilation of the style file could redirect
to a sample or test file.

%%%%%%%%%%%%%%%%%%%%%%%%%%%%%%%%%%%%%%%%%%%%%%%%%%%%%%%%%%%%%%%%%%%%%%%%%%%%%%%%
%%%%%%%%%%%%%%%%%%%%%%%%%%%%%%%%%%%%%%%%%%%%%%%%%%%%%%%%%%%%%%%%%%%%%%%%%%%%%%%%
\section{Usage}

First of all, the package \textsf{childdoc} is \emph{not} a standard
\LaTeXe{} |.sty| style file! Therefore it needs to be invoked in
a non-standard way.

%%%%%%%%%%%%%%%%%%%%%%%%%%%%%%%%%%%%%%%%%%%%%%%%%%%%%%%%%%%%%%%%%%%%%%%%%%%%%%%%
\subsection{Included Files}
\label{sec:include}

%%%%%%%%%%%%%%%%%%%%%%%%%%%%%%%%%%%%%%%%
\DescribeMacro{\childdocmain}
To use the package, add the commands
\begin{center}
\begin{tabular}{l}
|\input{childdoc.def}|\\
|\childdocmain{}|\\
\end{tabular}
\end{center}
at the very top of the main \LaTeX{} file,
in particular \emph{before} the |\documentclass| statement!
The argument of |\childdocmain| should be left empty
(but it must be present).

%%%%%%%%%%%%%%%%%%%%%%%%%%%%%%%%%%%%%%%%
\DescribeMacro{\childdocof}
Furthermore, add the commands
\begin{center}
\begin{tabular}{l}
|\input{childdoc.def}|\\
|\childdocof{|\textit{main}|}|\\
\end{tabular}
\end{center}
at the top of every child file \textit{child}
which is included by |\include{|\textit{child}|}|
from within the main file
(or at least for those files to be compiled individually).
The argument \textit{main} must be the filename of the main file.

There are a couple of
considerations in setting up the main and child documents:

%%%%%%%%%%%%%%%%%%%%%%%%%%%%%%%%%%%%%%%%
\paragraph{Restrictions.}

Please note the following restrictions:
\begin{itemize}
\item
|\childdocmain| must be called with one argument \textit{main}
to ensure compatibility with earlier version of the package.
It must either be empty (|\childdocmain{}|)
or precisely match the filename of the main file in which it is specified.
See \secref{sec:detection} for further information.
\item
The filename \textit{main} must be specified without the |.tex| extension.
\item
The filename \textit{main} is case sensitive
(even in case-insensitive file systems)
due to internal string comparison.
\item
The argument \textit{main} should be fully expanded, it cannot be a macro.
\item
Subdirectories and special characters should be avoided in filenames.
\item
The command |\childdocmain{|\textit{main}|}| must be followed by a whitespace.
It should not be followed immediately by another command
or by a comment mark `|%|'.
This is because the \TeX{} parser reads the token immediately following
the argument of |\childdocmain| and puts it
at the beginning of every child section;
however, a white\-space is ignored.
\end{itemize}

%%%%%%%%%%%%%%%%%%%%%%%%%%%%%%%%%%%%%%%%
\paragraph{Content of Main File.}

It is advisable to place all content in the child files included by |\include|.
Any output contained in the main file will appear in all child documents
unless suppressed manually;
it cannot be suppressed automatically by the |\includeonly| directive
and thus should normally be avoided.
A method to include some content in the main file
by means of conditional processing is described in \secref{sec:conditional}.

%%%%%%%%%%%%%%%%%%%%%%%%%%%%%%%%%%%%%%%%
\paragraph{Page Numbering.}

When only a part of the document is compiled,
the appropriate numbering of pages
(as well as other status parameters)
is determined from the |.aux| files.
The latter contain information from previous passes.
However this information needs to propagate through
all intermediate child documents.
Therefore the page numbering in child documents may well
be inconsistent until the complete document is compiled at least once.

A useful (if unconventional) way to always ensure a consistent
page numbering is to restart the numbering in each child document
and denote the pages by `\textit{child}|.|\textit{page}'
where \textit{child} represents the chapter/section number of the child file.
This can be achieved by the command
|\numberwithin{page}{|\textit{child}|}|
of the \textsf{amsmath} package
where \textit{child} can be |chapter| or |section|
depending on the chosen structuring.
Alternatively, one can modify the macro |\thepage| appropriately
and reset the counter |page| at the start of each child file.

%%%%%%%%%%%%%%%%%%%%%%%%%%%%%%%%%%%%%%%%%%%%%%%%%%%%%%%%%%%%%%%%%%%%%%%%%%%%%%%%
\subsection{Conditional Processing}
\label{sec:conditional}

The package provides a mechanism to compile different versions
of a document. To customise the versions further some conditional processing
can come in handy to distinguish which version is being compiled.
The package provides two macros to describe the compilation context:

%%%%%%%%%%%%%%%%%%%%%%%%%%%%%%%%%%%%%%%%
\DescribeMacro{\ifchilddoc}
The conditional |\ifchilddoc| distinguishes between the compilation of
child documents and the main document:
%
\begin{center}
|\ifchilddoc |\textit{child-code}| |[|\||else |\textit{main-code}]| \||fi|
\end{center}

%%%%%%%%%%%%%%%%%%%%%%%%%%%%%%%%%%%%%%%%
\DescribeMacro{\childdocname}
\DescribeMacro{\childdocjob}
The macro |\childdocname| contains the filename (without extension)
of the main or child file being processed.
Note that |\childdocjob| will always contain the name of the main file.

%%%%%%%%%%%%%%%%%%%%%%%%%%%%%%%%%%%%%%%%
\paragraph{Title Page.}

Conditional processing can be used to include a title or banner page
in the main document when proper precautions are taken.
Importantly, the code in the main file should ensure that the page counter
(as well as other status parameters which are stored in the |.aux| files)
takes the same value after the conditional processing.
Otherwise the page numbers may take divergent values
depending on which part is compiled.

For example, a title page could be declared by:
%
\begin{center}
\begin{tabular}{l}
|\ifchilddoc\||else|\\
|\addtocounter{page}{-1}|\\
\textit{code for title page}\\
|\newpage|\\
|\||fi|
\end{tabular}
\end{center}
%
A banner page for the child documents can be generated by:
%
\begin{center}
\begin{tabular}{l}
|\ifchilddoc|\\
|\addtocounter{page}{-1}|\\
\textit{code for banner page}\\
|\newpage|\\
|\||fi|
\end{tabular}
\end{center}
%
Here one could write a message such as:
\begin{center}
|This is the part \childdocname{} of \childdocjob{}.|
\end{center}

%%%%%%%%%%%%%%%%%%%%%%%%%%%%%%%%%%%%%%%%%%%%%%%%%%%%%%%%%%%%%%%%%%%%%%%%%%%%%%%%
\subsection{Flags}
\label{sec:flags}

The package makes it easy to generate different versions
of the main or child documents.
To this end compilation flags can be defined
and assigned different default values.
They will be particularly useful in conjunction
with the forwarding mechanism described in \secref{sec:forward}.

For example, it may be useful to have a flag |\version|
which can be set to |draft| or |final|.
The document source will contain some conditional code
depending on the value of |\version|.
Suppose further, the flag should default to |final| for the main file
and to |draft| for child files
which is a natural assignment for editing the document.
This is achieved by placing the following code
in the preamble of the main document
(below the |\childdocmain| directive):
%
\begin{center}
\begin{tabular}{l}
|\ifchilddoc|\\
|\providecommand{\version}{draft}|\\
|\||else|\\
|\providecommand{\version}{final}|\\
|\||fi|
\end{tabular}
\end{center}
%
The definition by |\providecommand| makes sure
that previous definitions are not overwritten.
Further statements |\providecommand{\version}{...}|
can thus be added before the above code to override it.

For the main file, one might add a line
(between |\childdocmain| and the above block)
%
\begin{center}
|%\ifchilddoc\||else\providecommand{\version}{draft}\||fi|
\end{center}
%
which can be uncommented to produce a draft version.
Likewise one can add a line to the very top of a child file
(above the |\childdocof{|\textit{main}|}| directive)
%
\begin{center}
|%\providecommand{\version}{final}|
\end{center}
%
which can be uncommented to produce the final version of this child document.

%%%%%%%%%%%%%%%%%%%%%%%%%%%%%%%%%%%%%%%%%%%%%%%%%%%%%%%%%%%%%%%%%%%%%%%%%%%%%%%%
\subsection{Forwarding}
\label{sec:forward}

Different versions of the main or child documents
using compilation flags as described in \secref{sec:flags}
can be (permanently) stored in different files
for convenient compilation, viewing and distribution.
To this end, the package defines a command
to pass on compilation to a different file:

%%%%%%%%%%%%%%%%%%%%%%%%%%%%%%%%%%%%%%%%
\DescribeMacro{\childdocforward}
The command |\childdocforward| redirects processing to
another source file:
%
\begin{center}
\begin{tabular}{l}
|\input{childdoc.def}|\\
|\childdocforward[|\textit{main}|]{|\textit{dest}|}|\\
\end{tabular}
\end{center}
%
The argument \textit{dest} is the destination file
(without extension).
It should be the main file or one of the child files.
Note that further \textsf{childdoc} directives
such as |\childdocof| and |\childdocforward|
in the indicated file will be processed in this form.
The optional argument \textit{main}
passes on directly to the main file \textit{main}
while pretending to compile the child \textit{dest}.
This form behaves as if \textit{dest}
issues |\childdocof{|\textit{main}|}| right away,
and no further \textsf{childdoc} directives will be processed.

%%%%%%%%%%%%%%%%%%%%%%%%%%%%%%%%%%%%%%%%
\DescribeMacro{\...prefix}
In the alternative form |\childdocforwardprefix|,
%
\begin{center}
\begin{tabular}{l}
|\input{childdoc.def}|\\
|\childdocforwardprefix[|\textit{main}|]{|\textit{prefix}|}{|\textit{dest}|}|
\end{tabular}
\end{center}
%
the destination file is determined by a pattern
depending on the current file:
To make this work, the current file must be called
`{\textit{prefix}\hspace{0.2em}\textit{suffix}}'
with \textit{prefix} matching precisely the argument.
Processing is then passed on to the file
`{\textit{dest}\hspace{0.2em}\textit{suffix}}'.
Surely, the same effect is achieved by
directly specifying the
argument `{\textit{dest}\hspace{0.2em}\textit{suffix}}'
in the first form.
However, that requires to set up a different file
for each child. With the alternative form of the command
all these files can have exactly the same content
which simplifies setting them up and maintaining them.

For example, the following file |draft.tex|
with a compilation flag |\version| as described in \secref{sec:flags}
compiles the main document as a draft:
%
\begin{center}
\begin{tabular}{l}
|\def\version{draft}|\\
|\input{childdoc.def}|\\
|\childdocforward{|\textit{main}|}|
\end{tabular}
\end{center}
%
Likewise, the following files |final|\textit{nn}|.tex|
compile the final version of the child document
|child|\textit{nn}|.tex|:
%
\begin{center}
\begin{tabular}{l}
|\def\version{final}|\\
|\input{childdoc.def}|\\
|\childdocforwardprefix{final}{child}|
\end{tabular}
\end{center}
%

Note that when several versions of a main file and/or of each child file
are to be generated, it may be convenient to set up a |Makefile| or
shell script to automatise the process.

%%%%%%%%%%%%%%%%%%%%%%%%%%%%%%%%%%%%%%%%%%%%%%%%%%%%%%%%%%%%%%%%%%%%%%%%%%%%%%%%
\subsection{Command Line Processing}
\label{sec:commandline}

The effect of redirection files can also be achieved by invoking
the \LaTeX{} compiler with a more elaborate command line.
Most conveniently this should be done as part
of a shell script or a |Makefile|.

When using \textsf{childdoc} in the main file, the following
command lines effectively perform a redirection
(note that depending on the shell being used,
backslashes may have to be doubled: `|\|' $\to$ `|\\|'):
%
\begin{center}
|... -jobname "|\textit{target}|" |\\|"|[\textit{flags}]%
|\input{childdoc.def}\childdocforward[|\textit{main}|]{|\textit{dest}|}"|
\end{center}
%
Here \textit{target} is the name of the output file,
\textit{main} is the name of the main file
and \textit{dest} is the name of the main or child file to be processed
(all filenames without extensions).
The optional argument \textit{main} can be omitted
if \textit{main} matches \textit{dest}.
Optionally, compilation \textit{flags} can be defined via |\def| commands.
This command line makes the \TeX{} engine believe
it is compiling the file \textit{target}
whose content is specified as the latter parameter.
The provided code then forwards the processing to
\textit{main} or \textit{dest} as described in \secref{sec:forward}.

%%%%%%%%%%%%%%%%%%%%%%%%%%%%%%%%%%%%%%%%%%%%%%%%%%%%%%%%%%%%%%%%%%%%%%%%%%%%%%%%
\subsection{Include by Input}
\label{sec:input}

Including child documents by |\include| has some restrictions by design.
Most notably, the content of a child document always occupies
its own set of pages; pages cannot be shared between child documents.
Usually, this behaviour makes perfect sense
because each child document contain an essential part of the document.
However, in some situations it may be desirable to compose
a document from a collection of parts
without having mandatory page breaks between then.
For this case, the package
provides a mechanism to include parts
by |\input| which can also be processed individually.
However, by construction this mechanism
requires manual handling of the content to be output.

%%%%%%%%%%%%%%%%%%%%%%%%%%%%%%%%%%%%%%%%
\DescribeMacro{\ifchilddocmanual}
The main file should be prepared as usual, see \secref{sec:include}.
However, the document body must make a distinction
between processing of an individual part and of the main document, e.g.:
%
\begin{center}
\begin{tabular}{l}
|\ifchilddocmanual|\\
|\input{\childdocname}|\\
|\||else|\\
\textit{document body with }|\input{|\textit{part}|}|\\
|\||fi|
\end{tabular}
\end{center}
%
The conditional |\ifchilddocmanual| is true whenever
a part to be included by |\input| is being compiled,
and the name of the part is stored in |\childdocname|.

%%%%%%%%%%%%%%%%%%%%%%%%%%%%%%%%%%%%%%%%
\DescribeMacro{\childdocby}
Each part to be included by |\input| should start with:
%
\begin{center}
\begin{tabular}{l}
|\input{childdoc.def}|\\
|\childdocby{|\textit{main}|}|\\
\end{tabular}
\end{center}
%
The directive |\childdocby| is similar to |\childdocof|
described in \secref{sec:include},
but the subsequent selection of content must be done manually.
To that end, both |\ifchilddoc| and |\ifchilddocmanual|
will be true upon processing of a part,
and the name of the part is stored in |\childdocname|.
Note that |\jobname| will be set to the filename of the current part
so that each part receives an individual |.aux| file
that does not interfere with the |.aux| file(s) of the main document.
This behaviour can be altered by the alternative form
|\childdocby[*]{|\textit{main}|}| (with a non-empty optional argument)
which uses the |.aux| file of the main document
by setting |\jobname| to \textit{main}.

%%%%%%%%%%%%%%%%%%%%%%%%%%%%%%%%%%%%%%%%%%%%%%%%%%%%%%%%%%%%%%%%%%%%%%%%%%%%%%%%
\subsection{Driver Development}
\label{sec:driver}

The \textsf{childdoc} mechanism can also be use for the development
of definition files such as \LaTeX{} styles or classes.
This case differs from the above setup with multiple parts
included by |\include| in that no |\includeonly| should be invoked.
This can be achieved by starting the include file
(before |\ProvidesPackage|) with:
%
\begin{center}
\begin{tabular}{l}
|\input{childdoc.def}|\\
|\childdocforward{|\textit{main}|}|\\
\end{tabular}
\end{center}
%
or alternatively with:
%
\begin{center}
\begin{tabular}{l}
|\input{childdoc.def}|\\
|\childdocby{|\textit{main}|}|\\
\end{tabular}
\end{center}
%
Both forms have slightly different effects as described above.
The main file is prepared as usual, see \secref{sec:include}.

%%%%%%%%%%%%%%%%%%%%%%%%%%%%%%%%%%%%%%%%%%%%%%%%%%%%%%%%%%%%%%%%%%%%%%%%%%%%%%%%
\subsection{Legacy Detection}
\label{sec:detection}

The directive |\childdocmain| in the main file can detect
whether the complete document or merely a child is to be compiled
even without using the directive |\childdocof|.
This method is deprecated because it is less robust
and there is no compelling reason to use it;
it is merely provided for backward compatibility
and it may be removed in future versions.

If the detection mechanism is to be used,
it is mandatory to correctly specify
the filename of the main file as the argument of |\childdocmain|:
%
\begin{center}
\begin{tabular}{l}
|\input{childdoc.def}|\\
|\childdocmain{|\textit{main}|}|\\
\end{tabular}
\end{center}
%
If |\jobname| does not match the argument \textit{main} of |\childdocmain|,
it is assumed that |\jobname| points to the child file to be compiled.
When using |\childdocmain| with the main file specified as argument,
it suffices to start a child file
with just |\input{|\textit{main}|}|
without loading of the package and using |\childdocof|.
If instead all processing is done
with the appropriate \textsf{childdoc} directives,
the argument of \textit{main} of |\childdocmain| can be empty.

An alternative version of the command line processing described
in \secref{sec:commandline} using the detection mechanism reads:
%
\begin{center}
|... -jobname "|\textit{target}|" "|[\textit{flags}]%
[|\def\jobname{|\textit{dest}|}|]|\input{|\textit{main}|}"|
\end{center}

%%%%%%%%%%%%%%%%%%%%%%%%%%%%%%%%%%%%%%%%%%%%%%%%%%%%%%%%%%%%%%%%%%%%%%%%%%%%%%%%
\subsection{Manual Code}
\label{sec:manual}

In case one cannot be certain whether the definitions file |childdoc.def|
is installed on the target \TeX{} distribution
and one prefers not to ship it,
it is conceivable to paste a few relevant commands into the sources.

To that end, drop all statements |\input{childdoc.def}|
and perform the replacements as outlined below.
Instead of |\childdocmain{|\textit{main}|}| add the following code
to the top of the main file:
%
\begin{center}
\begin{tabular}{l}
|\||ifdefined\childdocname\endinput\||fi\newif\ifchilddoc|\\
|\edef\childdocname{\scantokens\expandafter{\jobname\noexpand}}|\\
|\def\childdocmain{|\textit{main}|}\||ifx\childdocmain\childdocname\||else|\\
|\childdoctrue\includeonly{\childdocname}\let\jobname\childdocmain\||fi|\\
\end{tabular}
\end{center}
%
Instead of |\childdocof{|\textit{main}|}| just include the main file
at the top of each child file:
%
\begin{center}
|\input{|\textit{main}|}|
\end{center}
%
A simple redirection |\childdocforward{|\textit{dest}|}| is achieved by:
%
\begin{center}
|\def\jobname{|\textit{dest}|}\input{\jobname}|
\end{center}
%
The redirection with prefix
|\childdocforwardprefix[|\textit{prefix}|]{|\textit{dest}|}|
is accomplished by:
%
\begin{center}
\begin{tabular}{l}
|{\edef\jobname{\scantokens\expandafter{\jobname\noexpand}}|\\
|\def\redirectjob |\textit{prefix}|#1~~~{\gdef\jobname{|\textit{dest}|#1}}|\\
|\expandafter\redirectjob\jobname~~~}\input{\jobname}|
\end{tabular}
\end{center}

In an alternative approach,
child documents can be compiled by a specific command line
without additional code or specific definitions:
%
\begin{center}
|... -jobname "|\textit{target}|" "|[\textit{flags}]%
|\includeonly{|\textit{dest}|}\input{|\textit{main}|}"|
\end{center}
%

%%%%%%%%%%%%%%%%%%%%%%%%%%%%%%%%%%%%%%%%%%%%%%%%%%%%%%%%%%%%%%%%%%%%%%%%%%%%%%%%
%%%%%%%%%%%%%%%%%%%%%%%%%%%%%%%%%%%%%%%%%%%%%%%%%%%%%%%%%%%%%%%%%%%%%%%%%%%%%%%%
\section{Information}

%%%%%%%%%%%%%%%%%%%%%%%%%%%%%%%%%%%%%%%%%%%%%%%%%%%%%%%%%%%%%%%%%%%%%%%%%%%%%%%%
\subsection{Copyright}

Copyright \copyright{} 2017--2018 Niklas Beisert

This work may be distributed and/or modified under the
conditions of the \LaTeX{} Project Public License, either version 1.3
of this license or (at your option) any later version.
The latest version of this license is in
  \url{http://www.latex-project.org/lppl.txt}
and version 1.3 or later is part of all distributions of \LaTeX{}
version 2005/12/01 or later.

This work has the LPPL maintenance status `maintained'.

The Current Maintainer of this work is Niklas Beisert.

This work consists of the files |README.txt|, |childdoc.ins| and |childdoc.dtx|
as well as the derived files |childdoc.def|, |cdocsamp.tex|
with |cdocsch1.tex|, |cdocsch2.tex|, |cdocspt3.tex|, |cdocspt4.tex|,
|cdocsdrf.tex|, |cdocsfn1.tex|, |cdocsfn2.tex|
as well as |childdoc.pdf|.

%%%%%%%%%%%%%%%%%%%%%%%%%%%%%%%%%%%%%%%%%%%%%%%%%%%%%%%%%%%%%%%%%%%%%%%%%%%%%%%%
\subsection{Files and Installation}

The package consists of the files:
%
\begin{center}
\begin{tabular}{ll}
    |README.txt|   & readme file \\
    |childdoc.ins| & installation file \\
    |childdoc.dtx| & source file \\
    |childdoc.def| & definition file \\
    |cdocsamp.tex| & sample main file \\
    |cdocsch1.tex| & sample include file \\
    |cdocsch2.tex| & sample include file \\
    |cdocspt3.tex| & sample part file \\
    |cdocspt4.tex| & sample part file \\
    |cdocsdrf.tex| & sample redirection file \\
    |cdocsfn1.tex| & sample redirection file \\
    |cdocsfn2.tex| & sample redirection file \\
    |childdoc.pdf| & manual
\end{tabular}
\end{center}
%
The distribution consists of the files
|README.txt|, |childdoc.ins| and |childdoc.dtx|.
%
\begin{itemize}
\item
Run (pdf)\LaTeX{} on |childdoc.dtx|
to compile the manual |childdoc.pdf| (this file).
\item
Run \LaTeX{} on |childdoc.ins| to create the definitions file |childdoc.def|
and the sample |cdocsamp.tex| with include files
|cdocsch1.tex|, |cdocsch2.tex|, |cdocspt3.tex|, |cdocspt4.tex|,
|cdocsdrf.tex|, |cdocsfn1.tex|, |cdocsfn2.tex|.
Then copy the file |childdoc.def| to an appropriate directory of your \LaTeX{}
distribution, e.g.\ \textit{texmf-root}|/tex/latex/childdoc|.
\end{itemize}

%%%%%%%%%%%%%%%%%%%%%%%%%%%%%%%%%%%%%%%%%%%%%%%%%%%%%%%%%%%%%%%%%%%%%%%%%%%%%%%%
\subsection{Related CTAN Packages}

There are several other packages which offer a similar functionality:
%
\begin{itemize}
\item
The packages
\href{http://ctan.org/pkg/docmute}{\textsf{docmute}},
\href{http://ctan.org/pkg/includex}{\textsf{includex}} and
\href{http://ctan.org/pkg/standalone}{\textsf{standalone}}
provide commands to include only the document body of
a child file thus allowing both files to be compiled individually.
\item
The packages \href{http://ctan.org/pkg/subdocs}{\textsf{subdocs}}
and \href{http://ctan.org/pkg/subfiles}{\textsf{subfiles}}
provide structures in which the main and child documents can be
encapsulated and allowing them to be compiled individually.
The inclusion mechanism is different from the conventional |\include|.
\item
The package \href{http://ctan.org/pkg/combine}{\textsf{combine}}
is an elaborate solution to combine several documents into one.
\end{itemize}
%
See also the CTAN topic \href{http://ctan.org/topic/subdocs}{\textsf{subdocs}}
for further related packages.
The present package differs from the above solutions in that
a document structure constructed with the conventional |\include| mechanism
just needs two extra commands at the top of every file
such that all constituent files can be compiled individually.

%%%%%%%%%%%%%%%%%%%%%%%%%%%%%%%%%%%%%%%%%%%%%%%%%%%%%%%%%%%%%%%%%%%%%%%%%%%%%%%%
%\subsection{Feature Suggestions}
%
%The following is a list of features which may be useful for future
%versions of this package:
%%
%\begin{itemize}
%\item
%\ldots
%\end{itemize}

%%%%%%%%%%%%%%%%%%%%%%%%%%%%%%%%%%%%%%%%%%%%%%%%%%%%%%%%%%%%%%%%%%%%%%%%%%%%%%%%
\subsection{Revision History}

%%%%%%%%%%%%%%%%%%%%%%%%%%%%%%%%%%%%%%%%
\paragraph{v2.0:} 2018/12/30

\begin{itemize}
\item
immediate forward processing
\item
added |\childdocby| mechanism
\item
manual restructured
\end{itemize}

%%%%%%%%%%%%%%%%%%%%%%%%%%%%%%%%%%%%%%%%
\paragraph{v1.6:} 2018/01/17

\begin{itemize}
\item
application for development of include files
\item
corrections to manual
\end{itemize}

%%%%%%%%%%%%%%%%%%%%%%%%%%%%%%%%%%%%%%%%
\paragraph{v1.5:} 2017/05/21

\begin{itemize}
\item
more complete structuring introduced
\item
|\childdocof| introduced
\item
|\childdoc| renamed to |\childdocmain|
\item
|\childredirect| renamed to |\childdocforward| and |\childdocforwardprefix|
and functionality expanded
\end{itemize}

%%%%%%%%%%%%%%%%%%%%%%%%%%%%%%%%%%%%%%%%
\paragraph{v1.0:} 2017/04/27

\begin{itemize}
\item
manual and install package
\item
first version published on CTAN
\end{itemize}

%%%%%%%%%%%%%%%%%%%%%%%%%%%%%%%%%%%%%%%%
\paragraph{v0.6:} 2017/04/26

\begin{itemize}
\item
redirection mechanism added
\end{itemize}

%%%%%%%%%%%%%%%%%%%%%%%%%%%%%%%%%%%%%%%%
\paragraph{v0.5:} 2017/04/26

\begin{itemize}
\item
functionality in definition file
\end{itemize}


%%%%%%%%%%%%%%%%%%%%%%%%%%%%%%%%%%%%%%%%%%%%%%%%%%%%%%%%%%%%%%%%%%%%%%%%%%%%%%%%
%%%%%%%%%%%%%%%%%%%%%%%%%%%%%%%%%%%%%%%%%%%%%%%%%%%%%%%%%%%%%%%%%%%%%%%%%%%%%%%%
%%%%%%%%%%%%%%%%%%%%%%%%%%%%%%%%%%%%%%%%%%%%%%%%%%%%%%%%%%%%%%%%%%%%%%%%%%%%%%%%
\appendix

\settowidth\MacroIndent{\rmfamily\scriptsize 000\ }

 \DocInput{childdoc.dtx}

\end{document}
%</driver>
% \fi
%
% %%%%%%%%%%%%%%%%%%%%%%%%%%%%%%%%%%%%%%%%%%%%%%%%%%%%%%%%%%%%%%%%%%%%%%%%%%%%%%
% %%%%%%%%%%%%%%%%%%%%%%%%%%%%%%%%%%%%%%%%%%%%%%%%%%%%%%%%%%%%%%%%%%%%%%%%%%%%%%
% \section{Sample}
%\iffalse
%<*samplemain>
%\fi
%
% The following presents a sample document
% with two chapters, two parts, a title page,
% a compile flag as well as three forwarding files to set the flag.
% It consists of eight |.tex| files:
% \begin{center}
% \begin{tabular}{ll}
% |cdocsamp.tex|&main file\\
% |cdocsch1.tex|&include file for chapter 1\\
% |cdocsch2.tex|&include file for chapter 2\\
% |cdocspt3.tex|&include file for part 3\\
% |cdocspt4.tex|&include file for part 4\\
% |cdocsdrf.tex|&forwarding file for main file in draft mode\\
% |cdocsfi1.tex|&forwarding file for final version of chapter 1\\
% |cdocsfi2.tex|&forwarding file for final version of chapter 2\\
% \end{tabular}
% \end{center}
% Each of the eight files can be compiled directly by the \LaTeX{} compiler.
%
% %%%%%%%%%%%%%%%%%%%%%%%%%%%%%%%%%%%%%%
% \paragraph{Main File.}
%
% The main file is called |cdocsamp.tex|.
%
% Load the \textsf{childdoc} definitions and
% declare the filename for the main document:
%    \begin{macrocode}
\input{childdoc.def}
\childdocmain{}
%    \end{macrocode}

% Optional override for |\version| flag:
%    \begin{macrocode}
%%\ifchilddoc\else\providecommand{\version}{draft}\fi
%    \end{macrocode}

% Define the default values for the |\version| flag
% (|final| for the main file and |draft| for childs):
%    \begin{macrocode}
\ifchilddoc
\providecommand{\version}{draft}
\else
\providecommand{\version}{final}
\fi
%    \end{macrocode}

% Load the standard document class:
%    \begin{macrocode}
\documentclass[12pt]{article}
%    \end{macrocode}

% Start the document body:
%    \begin{macrocode}
\begin{document}
%    \end{macrocode}

% Declare a title page.
% Print title, part of document being processed and version flag:
%    \begin{macrocode}
\addtocounter{page}{-1}
\begin{center}
{\LARGE\bfseries{}childdoc example\par}
\vspace{1cm}
\ifchilddoc
\ifchilddocmanual part\else chapter\fi:
`\childdocname' of `\childdocjob'\par
\else
main document: `\childdocjob'\par
\fi
version: \version\par
\end{center}
\newpage
%    \end{macrocode}

% Manually include selected file,
% otherwise process as usual:
%    \begin{macrocode}
\ifchilddocmanual
\section*{part `\childdocname'}
\input{\childdocname}
\else
%    \end{macrocode}

% Include the two chapters:
%    \begin{macrocode}
\include{cdocsch1}
\include{cdocsch2}
%    \end{macrocode}

% Include the two parts unless only chapters should be displayed:
%    \begin{macrocode}
\ifchilddoc\else
\section{part three}
\input{cdocspt3}
\section{part four}
\input{cdocspt4}
\fi
%    \end{macrocode}

% Process as usual until here:
%    \begin{macrocode}
\fi
%    \end{macrocode}

% End of document body:
%    \begin{macrocode}
\end{document}
%    \end{macrocode}
%\iffalse
%</samplemain>
%\fi
%
% %%%%%%%%%%%%%%%%%%%%%%%%%%%%%%%%%%%%%%
% \paragraph{Chapter Include Files.}
%
% The include files are called |cdocsch1.tex| and |cdocsch2.tex|.
%
%\iffalse
%<*samplechap1|samplechap2>
%\fi

% Optional override for |\version| flag:
%    \begin{macrocode}
%%\providecommand{\version}{final}
%    \end{macrocode}

% Include the main document:
%    \begin{macrocode}
\input{childdoc.def}
\childdocof{cdocsamp}
%    \end{macrocode}

%\iffalse
%</samplechap1|samplechap2>
%\fi
%
%\iffalse
%<*samplechap1>
%\fi
% Some text for chapter 1:
%    \begin{macrocode}
\section{one}
some text in chapter one
%    \end{macrocode}

%\iffalse
%</samplechap1>
%\fi
% Some text for chapter 2:
%\iffalse
%<*samplechap2>
%\fi
%    \begin{macrocode}
\section{two}
more text in chapter two
%    \end{macrocode}

%\iffalse
%</samplechap2>
%\fi
%
% %%%%%%%%%%%%%%%%%%%%%%%%%%%%%%%%%%%%%%
% \paragraph{Part Include Files.}
%
% The include files are called |cdocspt3.tex| and |cdocspt4.tex|.
%
%\iffalse
%<*samplepart3|samplepart4>
%\fi

% Optional override for |\version| flag:
%    \begin{macrocode}
%%\providecommand{\version}{final}
%    \end{macrocode}

% Include the main document:
%    \begin{macrocode}
\input{childdoc.def}
\childdocby{cdocsamp}
%    \end{macrocode}

%\iffalse
%</samplepart3|samplepart4>
%\fi
%
%\iffalse
%<*samplepart3>
%\fi
% Some text for part 3:
%    \begin{macrocode}
some text in part three
%    \end{macrocode}

%\iffalse
%</samplepart3>
%\fi
% Some text for part 4:
%\iffalse
%<*samplepart4>
%\fi
%    \begin{macrocode}
more text in part four
%    \end{macrocode}

%\iffalse
%</samplepart4>
%\fi
%
% %%%%%%%%%%%%%%%%%%%%%%%%%%%%%%%%%%%%%%
% \paragraph{Forwarding for a Complete Draft.}
%
% The following forwarding file |cdocsdrf.tex|
% compiles the main document in draft mode:
%\iffalse
%<*sampledraft>
%\fi
%    \begin{macrocode}
\def\version{draft}
\input{childdoc.def}
\childdocforward{cdocsamp}
%    \end{macrocode}

%\iffalse
%</sampledraft>
%\fi
%
% %%%%%%%%%%%%%%%%%%%%%%%%%%%%%%%%%%%%%%
% \paragraph{Forwarding for Final Version of the Chapters.}
%
% The following forwarding files |cdocsfn1.tex| and |cdocsfn2.tex|
% (with identical content)
% compile the final versions of the child documents
% |cdocsch1.tex| and |cdocsch2.tex|, respectively:
%\iffalse
%<*samplefinal>
%\fi
%    \begin{macrocode}
\def\version{final}
\input{childdoc.def}
\childdocforwardprefix[cdocsamp]{cdocsfn}{cdocsch}
%    \end{macrocode}

%\iffalse
%</samplefinal>
%\fi
%
% %%%%%%%%%%%%%%%%%%%%%%%%%%%%%%%%%%%%%%
% \paragraph{Command Line Processing.}
%
% The following three command lines generate the output files
% |cdocscld|, |cdocscl1| and |cdocscl2|
% which should be identical to
% |cdocsdrf|, |cdocsch1| and |cdocsfn2|, respectively:
% \begin{center}
% \begin{tabular}{l}
% |latex -jobname cdocscld \|\\
% |  "\def\version{draft}\input{childdoc.def}\childdocforward{cdocsamp}"|\\
% |latex -jobname cdocscl1 \|\\
% |  "\input{childdoc.def}\childdocforward[cdocsamp]{cdocsch1}"|\\
% |latex -jobname cdocscl2 \|\\
% |  "\def\version{final}\input{childdoc.def}\childdocforward{cdocsch2}"|
% \end{tabular}
% \end{center}
% Note that the trailing backslash on each first line
% merely continues the input to the second line
% (for convenient cut ant paste).
% Furthermore, the command |latex| can be replaced by any
% of its alternative versions such as |pdflatex|.
%
% %%%%%%%%%%%%%%%%%%%%%%%%%%%%%%%%%%%%%%%%%%%%%%%%%%%%%%%%%%%%%%%%%%%%%%%%%%%%%%
% %%%%%%%%%%%%%%%%%%%%%%%%%%%%%%%%%%%%%%%%%%%%%%%%%%%%%%%%%%%%%%%%%%%%%%%%%%%%%%
% \section{Implementation}
%\iffalse
%<*package>
%\fi
%
% This section describes the definitions file |childdoc.def|.

% The definitions cannot be loaded using |\usepackage| or |\RequirePackage|
% which has a mechanism to prevent loading a style file more than once.
% When loading the definitions by means of |\input|
% multiple instances have to be prevented manually:
%\iffalse
%This code needs to be before the `\ProvidesFile' directive
%which is defined at the beginning of this file.
%Therefore it is also placed there and commented out here.
%</package>
%<*discard>
%\fi
%    \begin{macrocode}
\ifdefined\childdocmain\endinput\fi
%    \end{macrocode}
%\iffalse
%</discard>
%<*package>
%\fi
%
% \macro{\ifchilddoc}
% \macro{\ifchilddocmanual}
% The conditional |\ifchilddoc| tells whether a
% child (true) or main (false) document is being compiled.
% The conditional |\ifchilddocmanual| tells whether
% the |\includeonly| mechanism is used (false) or
% the selection of child files must be performed manually (true).
% The definitions initialise to false:
%    \begin{macrocode}
\newif\ifchilddoc
\newif\ifchilddocmanual
%    \end{macrocode}

% \macro{\childdocname}
% \macro{\childdocjob}
% The macro |\childdocname| stores the name of the main document
% to be compiled. The macro |\childdocjob| stores the name of
% the document on which the \LaTeX{} compiler was originally invoked.
% The content of |\jobname| cannot be compared
% to filenames specified in the source due to different catcodes.
% The following code rescans |\jobname|, stores the result
% in |\childdocname| and saves a copy in |\childdocjob|:
%    \begin{macrocode}
\edef\childdocname{\scantokens\expandafter{\jobname\noexpand}}
\let\childdocjob\childdocname
%    \end{macrocode}

% \macro{\childdocdisable}
% The macro |\childdocdisable| prevents the main file
% from being processed more than once.
% At this stage, the main document command |\childdocmain|
% is assumed to be called once again where it should do nothing.
% Any subsequent call to it should prevent
% a secondary processing of the main document
% It overwrites the forwarding commands
% |\childdocof| and |\childdocforward|
% with empty macros to prevent further inclusions of the main document:
%    \begin{macrocode}
\newcommand{\childdocdisable}
{
  \renewcommand{\childdocmain}[1]{\renewcommand{\childdocmain}[1]{\endinput}}
  \renewcommand{\childdocof}[1]{}
  \renewcommand{\childdocby}[2][]{}
  \renewcommand{\childdocforward}[2][]{}
  \renewcommand{\childdocdisable}{}
}
%    \end{macrocode}

% \macro{\childdocmain}
% The macro |\childdocmain| is to be called at the top of the main file
% with nothing or the main filename (without extension) as argument.
% First, it breaks loops.
% If the argument is not empty and does not match |\childdocname|
% (which is set by the first inclusion of |childdoc.def|),
% |\ifchilddoc| is set to true, |\includeonly| is applied to the child file
% and |\jobname| is set to the main file
% (for proper handling of |.aux| files):
%    \begin{macrocode}
\newcommand{\childdocmain}[1]
{
  \childdocdisable\childdocmain{}
  \if?#1?\else
    \begingroup
      \def\childdoctmp{#1}
      \ifx\childdoctmp\childdocname
        \def\childdoctmp{}
      \else
        \def\childdoctmp
        {
          \childdoctrue
          \includeonly{\childdocname}
          \def\childdocjob{#1}
          \def\jobname{#1}
        }
      \fi
      \expandafter
    \endgroup
    \childdoctmp
  \fi
}
%    \end{macrocode}

% \macro{\childdocof}
% The command |\childdocof| redirects
% compilation to the main file |#1|.
%    \begin{macrocode}
\newcommand{\childdocof}[1]
{
  \childdocdisable
  \childdoctrue
  \includeonly{\childdocname}
  \def\jobname{#1}
  \def\childdocjob{#1}
  \input{#1}
}
%    \end{macrocode}

% \macro{\childdocby}
% The command |\childdocby| ....
%    \begin{macrocode}
\newcommand{\childdocby}[2][]
{
  \childdocdisable
  \childdoctrue
  \childdocmanualtrue
  \if?#1?\else
    \def\jobname{#2}
  \fi
  \def\childdocjob{#2}
  \input{#2}
  \endinput
}
%    \end{macrocode}

% \macro{\childdocforward}
% The command |\childdocforward| redirects
% compilation to the main file or
% (if the optional argument is given) a child file.
% Parameters are set as if the main file
% or a child file starting with |\childdocof| was compiled.
% Then compilation is handed over to the main file:
%    \begin{macrocode}
\newcommand{\childdocforward}[2][]
{
  \begingroup
    \if?#1?
      \def\childdoctmp
      {
        \def\childdocname{#2}
        \def\childdocjob{#2}
        \def\jobname{#2}
        \input{#2}
        \endinput
      }
    \else
      \def\childdoctmp
      {
        \childdocdisable
        \def\childdocname{#2}
        \childdoctrue
        \includeonly{#2}
        \def\childdocjob{#1}
        \def\jobname{#1}
        \input{#1}
        \endinput
      }
    \fi
    \expandafter
  \endgroup
  \childdoctmp
}
%    \end{macrocode}

% \macro{\childdocforwardprefix}
% The command |\childdocforwardprefix| redirects
% compilation to the main or a child file by means of a pattern.
% The prefix |#1| in the current filename is replaced by |#2|
% and the suffix of the current filename is kept
% (it is assumed that the filename does not contain the substring `|~~~|'
% which is used as a delimiter).
% Compilation is handed over to the new file by |\childdocforward|:
%    \begin{macrocode}
\newcommand{\childdocforwardprefix}[3][]
{
  \begingroup
    \def\childdocextract #2##1~~~{\def\childdoctmp{\childdocforward[#1]{#3##1}}}
    \expandafter\childdocextract\childdocname~~~
    \expandafter
  \endgroup
  \childdoctmp
}
%    \end{macrocode}

% \macro{\childdoc}
% The deprecated macro |\childdoc| is a legacy version of |\childdocmain|:
%    \begin{macrocode}
\newcommand{\childdoc}{\childdocmain}
%    \end{macrocode}

% \macro{\childdocredirect}
% The deprecated macro |\childdocredirect| is a legacy version
% of |\childdocforward| and |\childdocforwardprefix|:
%    \begin{macrocode}
\newcommand{\childdocredirect}[2][]
{
  \begingroup
    \if?#1?
      \def\childdoctmp{\childdocforward{#2}}
    \else
      \def\childdoctmp{\childdocforwardprefix{#1}{#2}}
    \fi
    \expandafter
  \endgroup
  \childdoctmp
}
%    \end{macrocode}

%\iffalse
%</package>
%\fi
%
\endinput

\childdocforwardprefix[cdocsamp]{cdocsfn}{cdocsch}
%    \end{macrocode}

%\iffalse
%</samplefinal>
%\fi
%
% %%%%%%%%%%%%%%%%%%%%%%%%%%%%%%%%%%%%%%
% \paragraph{Command Line Processing.}
%
% The following three command lines generate the output files
% |cdocscld|, |cdocscl1| and |cdocscl2|
% which should be identical to
% |cdocsdrf|, |cdocsch1| and |cdocsfn2|, respectively:
% \begin{center}
% \begin{tabular}{l}
% |latex -jobname cdocscld \|\\
% |  "\def\version{draft}% \iffalse
%
% childdoc.dtx Copyright (C) 2017-2018 Niklas Beisert
%
% This work may be distributed and/or modified under the
% conditions of the LaTeX Project Public License, either version 1.3
% of this license or (at your option) any later version.
% The latest version of this license is in
%   http://www.latex-project.org/lppl.txt
% and version 1.3 or later is part of all distributions of LaTeX
% version 2005/12/01 or later.
%
% This work has the LPPL maintenance status `maintained'.
%
% The Current Maintainer of this work is Niklas Beisert.
%
% This work consists of the files childdoc.dtx and childdoc.ins
% and the derived files childdoc.def and cdocsamp.tex with
% cdocsch1.tex, cdocsch2.tex, cdocsdrf.tex, cdocsfn1.tex, cdocsfn2.tex.
%
%<package>\ifdefined\childdocmain\endinput\fi
%<package>\ProvidesFile{childdoc.def}[2018/12/30 v2.0 child document driver]
%<samplemain>\ProvidesFile{cdocsamp.tex}[2018/12/30 v2.0 sample for childdoc]
%<*driver>
%\ProvidesFile{childdoc.drv}[2018/12/30 v2.0 childdoc reference manual file]
\PassOptionsToClass{10pt,a4paper}{article}
\documentclass{ltxdoc}

\usepackage[margin=35mm]{geometry}
\usepackage{hyperref}
\usepackage{hyperxmp}
\usepackage[usenames]{color}

\hypersetup{colorlinks=true}
\hypersetup{pdfstartview=FitH}
\hypersetup{pdfpagemode=UseNone}
\hypersetup{pdfsource={}}
\hypersetup{pdflang={en-UK}}
\hypersetup{pdfcopyright={Copyright 2017-2018 Niklas Beisert.
  This work may be distributed and/or modified under the
  conditions of the LaTeX Project Public License, either version 1.3
  of this license or (at your option) any later version.}}
\hypersetup{pdflicenseurl={http://www.latex-project.org/lppl.txt}}
\hypersetup{pdfcontactaddress={ETH Zurich, ITP, HIT K,
  Wolfgang-Pauli-Strasse 27}}
\hypersetup{pdfcontactpostcode={8093}}
\hypersetup{pdfcontactcity={Zurich}}
\hypersetup{pdfcontactcountry={Switzerland}}
\hypersetup{pdfcontactemail={nbeisert@itp.phys.ethz.ch}}
\hypersetup{pdfcontacturl={http://people.phys.ethz.ch/\xmptilde nbeisert/}}

\newcommand{\secref}[1]{\hyperref[#1]{section \ref*{#1}}}

\parskip1ex
\parindent0pt
\let\olditemize\itemize
\def\itemize{\olditemize\parskip0pt}

\begin{document}

\title{The \textsf{childdoc} Package}
\hypersetup{pdftitle={The childdoc Package}}
\author{Niklas Beisert\\[2ex]
  Institut f\"ur Theoretische Physik\\
  Eidgen\"ossische Technische Hochschule Z\"urich\\
  Wolfgang-Pauli-Strasse 27, 8093 Z\"urich, Switzerland\\[1ex]
  \href{mailto:nbeisert@itp.phys.ethz.ch}
  {\texttt{nbeisert@itp.phys.ethz.ch}}}
\hypersetup{pdfauthor={Niklas Beisert}}
\hypersetup{pdfsubject={Manual for the LaTeX2e Package childdoc}}
\date{30 December 2018, \textsf{v2.0}}
\maketitle

\begin{abstract}\noindent
\textsf{childdoc} is a \LaTeXe{} package
that enables the direct compilation
of document sections included by |\include|
to individual files.
\end{abstract}

\begingroup
\parskip0ex
\tableofcontents
\endgroup

%%%%%%%%%%%%%%%%%%%%%%%%%%%%%%%%%%%%%%%%%%%%%%%%%%%%%%%%%%%%%%%%%%%%%%%%%%%%%%%%
%%%%%%%%%%%%%%%%%%%%%%%%%%%%%%%%%%%%%%%%%%%%%%%%%%%%%%%%%%%%%%%%%%%%%%%%%%%%%%%%
\section{Introduction}

\LaTeX{} provides a mechanism to structure a large document (such as a book)
into a main file and several child files (containing the chapters)
using the |\include| command.
This mechanism is beneficial for documents
which span hundreds of pages in order to
make the source file(s) more manageable.
Moreover, compilation can be restricted to
selected child files by means of the |\includeonly| command.
The latter feature can be used to reduce the compilation time while editing
(this was significantly more useful in the earlier days of \LaTeX{})
or to generate a smaller document which is easier to navigate.
Another application of |\includeonly| is to generate
documents consisting of selected parts of the complete document.

However, there are a few drawbacks of the plain |\include| mechanism:
\begin{itemize}
\item
The child files cannot be compiled on their own,
they can only be compiled via the main file.
A naive editing environment
(such as a text editor with an option
to have the current file processed by \LaTeX)
may require one to switch to the main file before compiling;
attempting to compile the child file produces errors.
\item
The main file must be modified (each time)
to adjust the |\includeonly| command
to the present needs. This easily leaves the main file in a messy state.
\item
The generated document will always carry the filename
of the main document. This is inconvenient if
several child files are to be compiled and
to be kept for distribution.
\end{itemize}

The present package provides a simple interface
to make child files individually compilable by \LaTeX{}.
Compiling a child file then has the same effect as compiling
the main file with an |\includeonly| command
to select the appropriate child.
Moreover the generated document will carry the name of the child
rather than the main file.
This resolves all three above issues.

This feature is meant to make the editing of books,
thesis documents and lecture notes somewhat more convenient.
However, the package can also be used efficiently for
composing a series of documents (such as exercise sheets)
which are typically distributed individually.
It then assists the author in generating the individual documents
(potentially in different versions)
as well as a document containing the collected series.
Another application is in developing style files
or other kinds of included material
where compilation of the style file could redirect
to a sample or test file.

%%%%%%%%%%%%%%%%%%%%%%%%%%%%%%%%%%%%%%%%%%%%%%%%%%%%%%%%%%%%%%%%%%%%%%%%%%%%%%%%
%%%%%%%%%%%%%%%%%%%%%%%%%%%%%%%%%%%%%%%%%%%%%%%%%%%%%%%%%%%%%%%%%%%%%%%%%%%%%%%%
\section{Usage}

First of all, the package \textsf{childdoc} is \emph{not} a standard
\LaTeXe{} |.sty| style file! Therefore it needs to be invoked in
a non-standard way.

%%%%%%%%%%%%%%%%%%%%%%%%%%%%%%%%%%%%%%%%%%%%%%%%%%%%%%%%%%%%%%%%%%%%%%%%%%%%%%%%
\subsection{Included Files}
\label{sec:include}

%%%%%%%%%%%%%%%%%%%%%%%%%%%%%%%%%%%%%%%%
\DescribeMacro{\childdocmain}
To use the package, add the commands
\begin{center}
\begin{tabular}{l}
|\input{childdoc.def}|\\
|\childdocmain{}|\\
\end{tabular}
\end{center}
at the very top of the main \LaTeX{} file,
in particular \emph{before} the |\documentclass| statement!
The argument of |\childdocmain| should be left empty
(but it must be present).

%%%%%%%%%%%%%%%%%%%%%%%%%%%%%%%%%%%%%%%%
\DescribeMacro{\childdocof}
Furthermore, add the commands
\begin{center}
\begin{tabular}{l}
|\input{childdoc.def}|\\
|\childdocof{|\textit{main}|}|\\
\end{tabular}
\end{center}
at the top of every child file \textit{child}
which is included by |\include{|\textit{child}|}|
from within the main file
(or at least for those files to be compiled individually).
The argument \textit{main} must be the filename of the main file.

There are a couple of
considerations in setting up the main and child documents:

%%%%%%%%%%%%%%%%%%%%%%%%%%%%%%%%%%%%%%%%
\paragraph{Restrictions.}

Please note the following restrictions:
\begin{itemize}
\item
|\childdocmain| must be called with one argument \textit{main}
to ensure compatibility with earlier version of the package.
It must either be empty (|\childdocmain{}|)
or precisely match the filename of the main file in which it is specified.
See \secref{sec:detection} for further information.
\item
The filename \textit{main} must be specified without the |.tex| extension.
\item
The filename \textit{main} is case sensitive
(even in case-insensitive file systems)
due to internal string comparison.
\item
The argument \textit{main} should be fully expanded, it cannot be a macro.
\item
Subdirectories and special characters should be avoided in filenames.
\item
The command |\childdocmain{|\textit{main}|}| must be followed by a whitespace.
It should not be followed immediately by another command
or by a comment mark `|%|'.
This is because the \TeX{} parser reads the token immediately following
the argument of |\childdocmain| and puts it
at the beginning of every child section;
however, a white\-space is ignored.
\end{itemize}

%%%%%%%%%%%%%%%%%%%%%%%%%%%%%%%%%%%%%%%%
\paragraph{Content of Main File.}

It is advisable to place all content in the child files included by |\include|.
Any output contained in the main file will appear in all child documents
unless suppressed manually;
it cannot be suppressed automatically by the |\includeonly| directive
and thus should normally be avoided.
A method to include some content in the main file
by means of conditional processing is described in \secref{sec:conditional}.

%%%%%%%%%%%%%%%%%%%%%%%%%%%%%%%%%%%%%%%%
\paragraph{Page Numbering.}

When only a part of the document is compiled,
the appropriate numbering of pages
(as well as other status parameters)
is determined from the |.aux| files.
The latter contain information from previous passes.
However this information needs to propagate through
all intermediate child documents.
Therefore the page numbering in child documents may well
be inconsistent until the complete document is compiled at least once.

A useful (if unconventional) way to always ensure a consistent
page numbering is to restart the numbering in each child document
and denote the pages by `\textit{child}|.|\textit{page}'
where \textit{child} represents the chapter/section number of the child file.
This can be achieved by the command
|\numberwithin{page}{|\textit{child}|}|
of the \textsf{amsmath} package
where \textit{child} can be |chapter| or |section|
depending on the chosen structuring.
Alternatively, one can modify the macro |\thepage| appropriately
and reset the counter |page| at the start of each child file.

%%%%%%%%%%%%%%%%%%%%%%%%%%%%%%%%%%%%%%%%%%%%%%%%%%%%%%%%%%%%%%%%%%%%%%%%%%%%%%%%
\subsection{Conditional Processing}
\label{sec:conditional}

The package provides a mechanism to compile different versions
of a document. To customise the versions further some conditional processing
can come in handy to distinguish which version is being compiled.
The package provides two macros to describe the compilation context:

%%%%%%%%%%%%%%%%%%%%%%%%%%%%%%%%%%%%%%%%
\DescribeMacro{\ifchilddoc}
The conditional |\ifchilddoc| distinguishes between the compilation of
child documents and the main document:
%
\begin{center}
|\ifchilddoc |\textit{child-code}| |[|\||else |\textit{main-code}]| \||fi|
\end{center}

%%%%%%%%%%%%%%%%%%%%%%%%%%%%%%%%%%%%%%%%
\DescribeMacro{\childdocname}
\DescribeMacro{\childdocjob}
The macro |\childdocname| contains the filename (without extension)
of the main or child file being processed.
Note that |\childdocjob| will always contain the name of the main file.

%%%%%%%%%%%%%%%%%%%%%%%%%%%%%%%%%%%%%%%%
\paragraph{Title Page.}

Conditional processing can be used to include a title or banner page
in the main document when proper precautions are taken.
Importantly, the code in the main file should ensure that the page counter
(as well as other status parameters which are stored in the |.aux| files)
takes the same value after the conditional processing.
Otherwise the page numbers may take divergent values
depending on which part is compiled.

For example, a title page could be declared by:
%
\begin{center}
\begin{tabular}{l}
|\ifchilddoc\||else|\\
|\addtocounter{page}{-1}|\\
\textit{code for title page}\\
|\newpage|\\
|\||fi|
\end{tabular}
\end{center}
%
A banner page for the child documents can be generated by:
%
\begin{center}
\begin{tabular}{l}
|\ifchilddoc|\\
|\addtocounter{page}{-1}|\\
\textit{code for banner page}\\
|\newpage|\\
|\||fi|
\end{tabular}
\end{center}
%
Here one could write a message such as:
\begin{center}
|This is the part \childdocname{} of \childdocjob{}.|
\end{center}

%%%%%%%%%%%%%%%%%%%%%%%%%%%%%%%%%%%%%%%%%%%%%%%%%%%%%%%%%%%%%%%%%%%%%%%%%%%%%%%%
\subsection{Flags}
\label{sec:flags}

The package makes it easy to generate different versions
of the main or child documents.
To this end compilation flags can be defined
and assigned different default values.
They will be particularly useful in conjunction
with the forwarding mechanism described in \secref{sec:forward}.

For example, it may be useful to have a flag |\version|
which can be set to |draft| or |final|.
The document source will contain some conditional code
depending on the value of |\version|.
Suppose further, the flag should default to |final| for the main file
and to |draft| for child files
which is a natural assignment for editing the document.
This is achieved by placing the following code
in the preamble of the main document
(below the |\childdocmain| directive):
%
\begin{center}
\begin{tabular}{l}
|\ifchilddoc|\\
|\providecommand{\version}{draft}|\\
|\||else|\\
|\providecommand{\version}{final}|\\
|\||fi|
\end{tabular}
\end{center}
%
The definition by |\providecommand| makes sure
that previous definitions are not overwritten.
Further statements |\providecommand{\version}{...}|
can thus be added before the above code to override it.

For the main file, one might add a line
(between |\childdocmain| and the above block)
%
\begin{center}
|%\ifchilddoc\||else\providecommand{\version}{draft}\||fi|
\end{center}
%
which can be uncommented to produce a draft version.
Likewise one can add a line to the very top of a child file
(above the |\childdocof{|\textit{main}|}| directive)
%
\begin{center}
|%\providecommand{\version}{final}|
\end{center}
%
which can be uncommented to produce the final version of this child document.

%%%%%%%%%%%%%%%%%%%%%%%%%%%%%%%%%%%%%%%%%%%%%%%%%%%%%%%%%%%%%%%%%%%%%%%%%%%%%%%%
\subsection{Forwarding}
\label{sec:forward}

Different versions of the main or child documents
using compilation flags as described in \secref{sec:flags}
can be (permanently) stored in different files
for convenient compilation, viewing and distribution.
To this end, the package defines a command
to pass on compilation to a different file:

%%%%%%%%%%%%%%%%%%%%%%%%%%%%%%%%%%%%%%%%
\DescribeMacro{\childdocforward}
The command |\childdocforward| redirects processing to
another source file:
%
\begin{center}
\begin{tabular}{l}
|\input{childdoc.def}|\\
|\childdocforward[|\textit{main}|]{|\textit{dest}|}|\\
\end{tabular}
\end{center}
%
The argument \textit{dest} is the destination file
(without extension).
It should be the main file or one of the child files.
Note that further \textsf{childdoc} directives
such as |\childdocof| and |\childdocforward|
in the indicated file will be processed in this form.
The optional argument \textit{main}
passes on directly to the main file \textit{main}
while pretending to compile the child \textit{dest}.
This form behaves as if \textit{dest}
issues |\childdocof{|\textit{main}|}| right away,
and no further \textsf{childdoc} directives will be processed.

%%%%%%%%%%%%%%%%%%%%%%%%%%%%%%%%%%%%%%%%
\DescribeMacro{\...prefix}
In the alternative form |\childdocforwardprefix|,
%
\begin{center}
\begin{tabular}{l}
|\input{childdoc.def}|\\
|\childdocforwardprefix[|\textit{main}|]{|\textit{prefix}|}{|\textit{dest}|}|
\end{tabular}
\end{center}
%
the destination file is determined by a pattern
depending on the current file:
To make this work, the current file must be called
`{\textit{prefix}\hspace{0.2em}\textit{suffix}}'
with \textit{prefix} matching precisely the argument.
Processing is then passed on to the file
`{\textit{dest}\hspace{0.2em}\textit{suffix}}'.
Surely, the same effect is achieved by
directly specifying the
argument `{\textit{dest}\hspace{0.2em}\textit{suffix}}'
in the first form.
However, that requires to set up a different file
for each child. With the alternative form of the command
all these files can have exactly the same content
which simplifies setting them up and maintaining them.

For example, the following file |draft.tex|
with a compilation flag |\version| as described in \secref{sec:flags}
compiles the main document as a draft:
%
\begin{center}
\begin{tabular}{l}
|\def\version{draft}|\\
|\input{childdoc.def}|\\
|\childdocforward{|\textit{main}|}|
\end{tabular}
\end{center}
%
Likewise, the following files |final|\textit{nn}|.tex|
compile the final version of the child document
|child|\textit{nn}|.tex|:
%
\begin{center}
\begin{tabular}{l}
|\def\version{final}|\\
|\input{childdoc.def}|\\
|\childdocforwardprefix{final}{child}|
\end{tabular}
\end{center}
%

Note that when several versions of a main file and/or of each child file
are to be generated, it may be convenient to set up a |Makefile| or
shell script to automatise the process.

%%%%%%%%%%%%%%%%%%%%%%%%%%%%%%%%%%%%%%%%%%%%%%%%%%%%%%%%%%%%%%%%%%%%%%%%%%%%%%%%
\subsection{Command Line Processing}
\label{sec:commandline}

The effect of redirection files can also be achieved by invoking
the \LaTeX{} compiler with a more elaborate command line.
Most conveniently this should be done as part
of a shell script or a |Makefile|.

When using \textsf{childdoc} in the main file, the following
command lines effectively perform a redirection
(note that depending on the shell being used,
backslashes may have to be doubled: `|\|' $\to$ `|\\|'):
%
\begin{center}
|... -jobname "|\textit{target}|" |\\|"|[\textit{flags}]%
|\input{childdoc.def}\childdocforward[|\textit{main}|]{|\textit{dest}|}"|
\end{center}
%
Here \textit{target} is the name of the output file,
\textit{main} is the name of the main file
and \textit{dest} is the name of the main or child file to be processed
(all filenames without extensions).
The optional argument \textit{main} can be omitted
if \textit{main} matches \textit{dest}.
Optionally, compilation \textit{flags} can be defined via |\def| commands.
This command line makes the \TeX{} engine believe
it is compiling the file \textit{target}
whose content is specified as the latter parameter.
The provided code then forwards the processing to
\textit{main} or \textit{dest} as described in \secref{sec:forward}.

%%%%%%%%%%%%%%%%%%%%%%%%%%%%%%%%%%%%%%%%%%%%%%%%%%%%%%%%%%%%%%%%%%%%%%%%%%%%%%%%
\subsection{Include by Input}
\label{sec:input}

Including child documents by |\include| has some restrictions by design.
Most notably, the content of a child document always occupies
its own set of pages; pages cannot be shared between child documents.
Usually, this behaviour makes perfect sense
because each child document contain an essential part of the document.
However, in some situations it may be desirable to compose
a document from a collection of parts
without having mandatory page breaks between then.
For this case, the package
provides a mechanism to include parts
by |\input| which can also be processed individually.
However, by construction this mechanism
requires manual handling of the content to be output.

%%%%%%%%%%%%%%%%%%%%%%%%%%%%%%%%%%%%%%%%
\DescribeMacro{\ifchilddocmanual}
The main file should be prepared as usual, see \secref{sec:include}.
However, the document body must make a distinction
between processing of an individual part and of the main document, e.g.:
%
\begin{center}
\begin{tabular}{l}
|\ifchilddocmanual|\\
|\input{\childdocname}|\\
|\||else|\\
\textit{document body with }|\input{|\textit{part}|}|\\
|\||fi|
\end{tabular}
\end{center}
%
The conditional |\ifchilddocmanual| is true whenever
a part to be included by |\input| is being compiled,
and the name of the part is stored in |\childdocname|.

%%%%%%%%%%%%%%%%%%%%%%%%%%%%%%%%%%%%%%%%
\DescribeMacro{\childdocby}
Each part to be included by |\input| should start with:
%
\begin{center}
\begin{tabular}{l}
|\input{childdoc.def}|\\
|\childdocby{|\textit{main}|}|\\
\end{tabular}
\end{center}
%
The directive |\childdocby| is similar to |\childdocof|
described in \secref{sec:include},
but the subsequent selection of content must be done manually.
To that end, both |\ifchilddoc| and |\ifchilddocmanual|
will be true upon processing of a part,
and the name of the part is stored in |\childdocname|.
Note that |\jobname| will be set to the filename of the current part
so that each part receives an individual |.aux| file
that does not interfere with the |.aux| file(s) of the main document.
This behaviour can be altered by the alternative form
|\childdocby[*]{|\textit{main}|}| (with a non-empty optional argument)
which uses the |.aux| file of the main document
by setting |\jobname| to \textit{main}.

%%%%%%%%%%%%%%%%%%%%%%%%%%%%%%%%%%%%%%%%%%%%%%%%%%%%%%%%%%%%%%%%%%%%%%%%%%%%%%%%
\subsection{Driver Development}
\label{sec:driver}

The \textsf{childdoc} mechanism can also be use for the development
of definition files such as \LaTeX{} styles or classes.
This case differs from the above setup with multiple parts
included by |\include| in that no |\includeonly| should be invoked.
This can be achieved by starting the include file
(before |\ProvidesPackage|) with:
%
\begin{center}
\begin{tabular}{l}
|\input{childdoc.def}|\\
|\childdocforward{|\textit{main}|}|\\
\end{tabular}
\end{center}
%
or alternatively with:
%
\begin{center}
\begin{tabular}{l}
|\input{childdoc.def}|\\
|\childdocby{|\textit{main}|}|\\
\end{tabular}
\end{center}
%
Both forms have slightly different effects as described above.
The main file is prepared as usual, see \secref{sec:include}.

%%%%%%%%%%%%%%%%%%%%%%%%%%%%%%%%%%%%%%%%%%%%%%%%%%%%%%%%%%%%%%%%%%%%%%%%%%%%%%%%
\subsection{Legacy Detection}
\label{sec:detection}

The directive |\childdocmain| in the main file can detect
whether the complete document or merely a child is to be compiled
even without using the directive |\childdocof|.
This method is deprecated because it is less robust
and there is no compelling reason to use it;
it is merely provided for backward compatibility
and it may be removed in future versions.

If the detection mechanism is to be used,
it is mandatory to correctly specify
the filename of the main file as the argument of |\childdocmain|:
%
\begin{center}
\begin{tabular}{l}
|\input{childdoc.def}|\\
|\childdocmain{|\textit{main}|}|\\
\end{tabular}
\end{center}
%
If |\jobname| does not match the argument \textit{main} of |\childdocmain|,
it is assumed that |\jobname| points to the child file to be compiled.
When using |\childdocmain| with the main file specified as argument,
it suffices to start a child file
with just |\input{|\textit{main}|}|
without loading of the package and using |\childdocof|.
If instead all processing is done
with the appropriate \textsf{childdoc} directives,
the argument of \textit{main} of |\childdocmain| can be empty.

An alternative version of the command line processing described
in \secref{sec:commandline} using the detection mechanism reads:
%
\begin{center}
|... -jobname "|\textit{target}|" "|[\textit{flags}]%
[|\def\jobname{|\textit{dest}|}|]|\input{|\textit{main}|}"|
\end{center}

%%%%%%%%%%%%%%%%%%%%%%%%%%%%%%%%%%%%%%%%%%%%%%%%%%%%%%%%%%%%%%%%%%%%%%%%%%%%%%%%
\subsection{Manual Code}
\label{sec:manual}

In case one cannot be certain whether the definitions file |childdoc.def|
is installed on the target \TeX{} distribution
and one prefers not to ship it,
it is conceivable to paste a few relevant commands into the sources.

To that end, drop all statements |\input{childdoc.def}|
and perform the replacements as outlined below.
Instead of |\childdocmain{|\textit{main}|}| add the following code
to the top of the main file:
%
\begin{center}
\begin{tabular}{l}
|\||ifdefined\childdocname\endinput\||fi\newif\ifchilddoc|\\
|\edef\childdocname{\scantokens\expandafter{\jobname\noexpand}}|\\
|\def\childdocmain{|\textit{main}|}\||ifx\childdocmain\childdocname\||else|\\
|\childdoctrue\includeonly{\childdocname}\let\jobname\childdocmain\||fi|\\
\end{tabular}
\end{center}
%
Instead of |\childdocof{|\textit{main}|}| just include the main file
at the top of each child file:
%
\begin{center}
|\input{|\textit{main}|}|
\end{center}
%
A simple redirection |\childdocforward{|\textit{dest}|}| is achieved by:
%
\begin{center}
|\def\jobname{|\textit{dest}|}\input{\jobname}|
\end{center}
%
The redirection with prefix
|\childdocforwardprefix[|\textit{prefix}|]{|\textit{dest}|}|
is accomplished by:
%
\begin{center}
\begin{tabular}{l}
|{\edef\jobname{\scantokens\expandafter{\jobname\noexpand}}|\\
|\def\redirectjob |\textit{prefix}|#1~~~{\gdef\jobname{|\textit{dest}|#1}}|\\
|\expandafter\redirectjob\jobname~~~}\input{\jobname}|
\end{tabular}
\end{center}

In an alternative approach,
child documents can be compiled by a specific command line
without additional code or specific definitions:
%
\begin{center}
|... -jobname "|\textit{target}|" "|[\textit{flags}]%
|\includeonly{|\textit{dest}|}\input{|\textit{main}|}"|
\end{center}
%

%%%%%%%%%%%%%%%%%%%%%%%%%%%%%%%%%%%%%%%%%%%%%%%%%%%%%%%%%%%%%%%%%%%%%%%%%%%%%%%%
%%%%%%%%%%%%%%%%%%%%%%%%%%%%%%%%%%%%%%%%%%%%%%%%%%%%%%%%%%%%%%%%%%%%%%%%%%%%%%%%
\section{Information}

%%%%%%%%%%%%%%%%%%%%%%%%%%%%%%%%%%%%%%%%%%%%%%%%%%%%%%%%%%%%%%%%%%%%%%%%%%%%%%%%
\subsection{Copyright}

Copyright \copyright{} 2017--2018 Niklas Beisert

This work may be distributed and/or modified under the
conditions of the \LaTeX{} Project Public License, either version 1.3
of this license or (at your option) any later version.
The latest version of this license is in
  \url{http://www.latex-project.org/lppl.txt}
and version 1.3 or later is part of all distributions of \LaTeX{}
version 2005/12/01 or later.

This work has the LPPL maintenance status `maintained'.

The Current Maintainer of this work is Niklas Beisert.

This work consists of the files |README.txt|, |childdoc.ins| and |childdoc.dtx|
as well as the derived files |childdoc.def|, |cdocsamp.tex|
with |cdocsch1.tex|, |cdocsch2.tex|, |cdocspt3.tex|, |cdocspt4.tex|,
|cdocsdrf.tex|, |cdocsfn1.tex|, |cdocsfn2.tex|
as well as |childdoc.pdf|.

%%%%%%%%%%%%%%%%%%%%%%%%%%%%%%%%%%%%%%%%%%%%%%%%%%%%%%%%%%%%%%%%%%%%%%%%%%%%%%%%
\subsection{Files and Installation}

The package consists of the files:
%
\begin{center}
\begin{tabular}{ll}
    |README.txt|   & readme file \\
    |childdoc.ins| & installation file \\
    |childdoc.dtx| & source file \\
    |childdoc.def| & definition file \\
    |cdocsamp.tex| & sample main file \\
    |cdocsch1.tex| & sample include file \\
    |cdocsch2.tex| & sample include file \\
    |cdocspt3.tex| & sample part file \\
    |cdocspt4.tex| & sample part file \\
    |cdocsdrf.tex| & sample redirection file \\
    |cdocsfn1.tex| & sample redirection file \\
    |cdocsfn2.tex| & sample redirection file \\
    |childdoc.pdf| & manual
\end{tabular}
\end{center}
%
The distribution consists of the files
|README.txt|, |childdoc.ins| and |childdoc.dtx|.
%
\begin{itemize}
\item
Run (pdf)\LaTeX{} on |childdoc.dtx|
to compile the manual |childdoc.pdf| (this file).
\item
Run \LaTeX{} on |childdoc.ins| to create the definitions file |childdoc.def|
and the sample |cdocsamp.tex| with include files
|cdocsch1.tex|, |cdocsch2.tex|, |cdocspt3.tex|, |cdocspt4.tex|,
|cdocsdrf.tex|, |cdocsfn1.tex|, |cdocsfn2.tex|.
Then copy the file |childdoc.def| to an appropriate directory of your \LaTeX{}
distribution, e.g.\ \textit{texmf-root}|/tex/latex/childdoc|.
\end{itemize}

%%%%%%%%%%%%%%%%%%%%%%%%%%%%%%%%%%%%%%%%%%%%%%%%%%%%%%%%%%%%%%%%%%%%%%%%%%%%%%%%
\subsection{Related CTAN Packages}

There are several other packages which offer a similar functionality:
%
\begin{itemize}
\item
The packages
\href{http://ctan.org/pkg/docmute}{\textsf{docmute}},
\href{http://ctan.org/pkg/includex}{\textsf{includex}} and
\href{http://ctan.org/pkg/standalone}{\textsf{standalone}}
provide commands to include only the document body of
a child file thus allowing both files to be compiled individually.
\item
The packages \href{http://ctan.org/pkg/subdocs}{\textsf{subdocs}}
and \href{http://ctan.org/pkg/subfiles}{\textsf{subfiles}}
provide structures in which the main and child documents can be
encapsulated and allowing them to be compiled individually.
The inclusion mechanism is different from the conventional |\include|.
\item
The package \href{http://ctan.org/pkg/combine}{\textsf{combine}}
is an elaborate solution to combine several documents into one.
\end{itemize}
%
See also the CTAN topic \href{http://ctan.org/topic/subdocs}{\textsf{subdocs}}
for further related packages.
The present package differs from the above solutions in that
a document structure constructed with the conventional |\include| mechanism
just needs two extra commands at the top of every file
such that all constituent files can be compiled individually.

%%%%%%%%%%%%%%%%%%%%%%%%%%%%%%%%%%%%%%%%%%%%%%%%%%%%%%%%%%%%%%%%%%%%%%%%%%%%%%%%
%\subsection{Feature Suggestions}
%
%The following is a list of features which may be useful for future
%versions of this package:
%%
%\begin{itemize}
%\item
%\ldots
%\end{itemize}

%%%%%%%%%%%%%%%%%%%%%%%%%%%%%%%%%%%%%%%%%%%%%%%%%%%%%%%%%%%%%%%%%%%%%%%%%%%%%%%%
\subsection{Revision History}

%%%%%%%%%%%%%%%%%%%%%%%%%%%%%%%%%%%%%%%%
\paragraph{v2.0:} 2018/12/30

\begin{itemize}
\item
immediate forward processing
\item
added |\childdocby| mechanism
\item
manual restructured
\end{itemize}

%%%%%%%%%%%%%%%%%%%%%%%%%%%%%%%%%%%%%%%%
\paragraph{v1.6:} 2018/01/17

\begin{itemize}
\item
application for development of include files
\item
corrections to manual
\end{itemize}

%%%%%%%%%%%%%%%%%%%%%%%%%%%%%%%%%%%%%%%%
\paragraph{v1.5:} 2017/05/21

\begin{itemize}
\item
more complete structuring introduced
\item
|\childdocof| introduced
\item
|\childdoc| renamed to |\childdocmain|
\item
|\childredirect| renamed to |\childdocforward| and |\childdocforwardprefix|
and functionality expanded
\end{itemize}

%%%%%%%%%%%%%%%%%%%%%%%%%%%%%%%%%%%%%%%%
\paragraph{v1.0:} 2017/04/27

\begin{itemize}
\item
manual and install package
\item
first version published on CTAN
\end{itemize}

%%%%%%%%%%%%%%%%%%%%%%%%%%%%%%%%%%%%%%%%
\paragraph{v0.6:} 2017/04/26

\begin{itemize}
\item
redirection mechanism added
\end{itemize}

%%%%%%%%%%%%%%%%%%%%%%%%%%%%%%%%%%%%%%%%
\paragraph{v0.5:} 2017/04/26

\begin{itemize}
\item
functionality in definition file
\end{itemize}


%%%%%%%%%%%%%%%%%%%%%%%%%%%%%%%%%%%%%%%%%%%%%%%%%%%%%%%%%%%%%%%%%%%%%%%%%%%%%%%%
%%%%%%%%%%%%%%%%%%%%%%%%%%%%%%%%%%%%%%%%%%%%%%%%%%%%%%%%%%%%%%%%%%%%%%%%%%%%%%%%
%%%%%%%%%%%%%%%%%%%%%%%%%%%%%%%%%%%%%%%%%%%%%%%%%%%%%%%%%%%%%%%%%%%%%%%%%%%%%%%%
\appendix

\settowidth\MacroIndent{\rmfamily\scriptsize 000\ }

 \DocInput{childdoc.dtx}

\end{document}
%</driver>
% \fi
%
% %%%%%%%%%%%%%%%%%%%%%%%%%%%%%%%%%%%%%%%%%%%%%%%%%%%%%%%%%%%%%%%%%%%%%%%%%%%%%%
% %%%%%%%%%%%%%%%%%%%%%%%%%%%%%%%%%%%%%%%%%%%%%%%%%%%%%%%%%%%%%%%%%%%%%%%%%%%%%%
% \section{Sample}
%\iffalse
%<*samplemain>
%\fi
%
% The following presents a sample document
% with two chapters, two parts, a title page,
% a compile flag as well as three forwarding files to set the flag.
% It consists of eight |.tex| files:
% \begin{center}
% \begin{tabular}{ll}
% |cdocsamp.tex|&main file\\
% |cdocsch1.tex|&include file for chapter 1\\
% |cdocsch2.tex|&include file for chapter 2\\
% |cdocspt3.tex|&include file for part 3\\
% |cdocspt4.tex|&include file for part 4\\
% |cdocsdrf.tex|&forwarding file for main file in draft mode\\
% |cdocsfi1.tex|&forwarding file for final version of chapter 1\\
% |cdocsfi2.tex|&forwarding file for final version of chapter 2\\
% \end{tabular}
% \end{center}
% Each of the eight files can be compiled directly by the \LaTeX{} compiler.
%
% %%%%%%%%%%%%%%%%%%%%%%%%%%%%%%%%%%%%%%
% \paragraph{Main File.}
%
% The main file is called |cdocsamp.tex|.
%
% Load the \textsf{childdoc} definitions and
% declare the filename for the main document:
%    \begin{macrocode}
\input{childdoc.def}
\childdocmain{}
%    \end{macrocode}

% Optional override for |\version| flag:
%    \begin{macrocode}
%%\ifchilddoc\else\providecommand{\version}{draft}\fi
%    \end{macrocode}

% Define the default values for the |\version| flag
% (|final| for the main file and |draft| for childs):
%    \begin{macrocode}
\ifchilddoc
\providecommand{\version}{draft}
\else
\providecommand{\version}{final}
\fi
%    \end{macrocode}

% Load the standard document class:
%    \begin{macrocode}
\documentclass[12pt]{article}
%    \end{macrocode}

% Start the document body:
%    \begin{macrocode}
\begin{document}
%    \end{macrocode}

% Declare a title page.
% Print title, part of document being processed and version flag:
%    \begin{macrocode}
\addtocounter{page}{-1}
\begin{center}
{\LARGE\bfseries{}childdoc example\par}
\vspace{1cm}
\ifchilddoc
\ifchilddocmanual part\else chapter\fi:
`\childdocname' of `\childdocjob'\par
\else
main document: `\childdocjob'\par
\fi
version: \version\par
\end{center}
\newpage
%    \end{macrocode}

% Manually include selected file,
% otherwise process as usual:
%    \begin{macrocode}
\ifchilddocmanual
\section*{part `\childdocname'}
\input{\childdocname}
\else
%    \end{macrocode}

% Include the two chapters:
%    \begin{macrocode}
\include{cdocsch1}
\include{cdocsch2}
%    \end{macrocode}

% Include the two parts unless only chapters should be displayed:
%    \begin{macrocode}
\ifchilddoc\else
\section{part three}
\input{cdocspt3}
\section{part four}
\input{cdocspt4}
\fi
%    \end{macrocode}

% Process as usual until here:
%    \begin{macrocode}
\fi
%    \end{macrocode}

% End of document body:
%    \begin{macrocode}
\end{document}
%    \end{macrocode}
%\iffalse
%</samplemain>
%\fi
%
% %%%%%%%%%%%%%%%%%%%%%%%%%%%%%%%%%%%%%%
% \paragraph{Chapter Include Files.}
%
% The include files are called |cdocsch1.tex| and |cdocsch2.tex|.
%
%\iffalse
%<*samplechap1|samplechap2>
%\fi

% Optional override for |\version| flag:
%    \begin{macrocode}
%%\providecommand{\version}{final}
%    \end{macrocode}

% Include the main document:
%    \begin{macrocode}
\input{childdoc.def}
\childdocof{cdocsamp}
%    \end{macrocode}

%\iffalse
%</samplechap1|samplechap2>
%\fi
%
%\iffalse
%<*samplechap1>
%\fi
% Some text for chapter 1:
%    \begin{macrocode}
\section{one}
some text in chapter one
%    \end{macrocode}

%\iffalse
%</samplechap1>
%\fi
% Some text for chapter 2:
%\iffalse
%<*samplechap2>
%\fi
%    \begin{macrocode}
\section{two}
more text in chapter two
%    \end{macrocode}

%\iffalse
%</samplechap2>
%\fi
%
% %%%%%%%%%%%%%%%%%%%%%%%%%%%%%%%%%%%%%%
% \paragraph{Part Include Files.}
%
% The include files are called |cdocspt3.tex| and |cdocspt4.tex|.
%
%\iffalse
%<*samplepart3|samplepart4>
%\fi

% Optional override for |\version| flag:
%    \begin{macrocode}
%%\providecommand{\version}{final}
%    \end{macrocode}

% Include the main document:
%    \begin{macrocode}
\input{childdoc.def}
\childdocby{cdocsamp}
%    \end{macrocode}

%\iffalse
%</samplepart3|samplepart4>
%\fi
%
%\iffalse
%<*samplepart3>
%\fi
% Some text for part 3:
%    \begin{macrocode}
some text in part three
%    \end{macrocode}

%\iffalse
%</samplepart3>
%\fi
% Some text for part 4:
%\iffalse
%<*samplepart4>
%\fi
%    \begin{macrocode}
more text in part four
%    \end{macrocode}

%\iffalse
%</samplepart4>
%\fi
%
% %%%%%%%%%%%%%%%%%%%%%%%%%%%%%%%%%%%%%%
% \paragraph{Forwarding for a Complete Draft.}
%
% The following forwarding file |cdocsdrf.tex|
% compiles the main document in draft mode:
%\iffalse
%<*sampledraft>
%\fi
%    \begin{macrocode}
\def\version{draft}
\input{childdoc.def}
\childdocforward{cdocsamp}
%    \end{macrocode}

%\iffalse
%</sampledraft>
%\fi
%
% %%%%%%%%%%%%%%%%%%%%%%%%%%%%%%%%%%%%%%
% \paragraph{Forwarding for Final Version of the Chapters.}
%
% The following forwarding files |cdocsfn1.tex| and |cdocsfn2.tex|
% (with identical content)
% compile the final versions of the child documents
% |cdocsch1.tex| and |cdocsch2.tex|, respectively:
%\iffalse
%<*samplefinal>
%\fi
%    \begin{macrocode}
\def\version{final}
\input{childdoc.def}
\childdocforwardprefix[cdocsamp]{cdocsfn}{cdocsch}
%    \end{macrocode}

%\iffalse
%</samplefinal>
%\fi
%
% %%%%%%%%%%%%%%%%%%%%%%%%%%%%%%%%%%%%%%
% \paragraph{Command Line Processing.}
%
% The following three command lines generate the output files
% |cdocscld|, |cdocscl1| and |cdocscl2|
% which should be identical to
% |cdocsdrf|, |cdocsch1| and |cdocsfn2|, respectively:
% \begin{center}
% \begin{tabular}{l}
% |latex -jobname cdocscld \|\\
% |  "\def\version{draft}\input{childdoc.def}\childdocforward{cdocsamp}"|\\
% |latex -jobname cdocscl1 \|\\
% |  "\input{childdoc.def}\childdocforward[cdocsamp]{cdocsch1}"|\\
% |latex -jobname cdocscl2 \|\\
% |  "\def\version{final}\input{childdoc.def}\childdocforward{cdocsch2}"|
% \end{tabular}
% \end{center}
% Note that the trailing backslash on each first line
% merely continues the input to the second line
% (for convenient cut ant paste).
% Furthermore, the command |latex| can be replaced by any
% of its alternative versions such as |pdflatex|.
%
% %%%%%%%%%%%%%%%%%%%%%%%%%%%%%%%%%%%%%%%%%%%%%%%%%%%%%%%%%%%%%%%%%%%%%%%%%%%%%%
% %%%%%%%%%%%%%%%%%%%%%%%%%%%%%%%%%%%%%%%%%%%%%%%%%%%%%%%%%%%%%%%%%%%%%%%%%%%%%%
% \section{Implementation}
%\iffalse
%<*package>
%\fi
%
% This section describes the definitions file |childdoc.def|.

% The definitions cannot be loaded using |\usepackage| or |\RequirePackage|
% which has a mechanism to prevent loading a style file more than once.
% When loading the definitions by means of |\input|
% multiple instances have to be prevented manually:
%\iffalse
%This code needs to be before the `\ProvidesFile' directive
%which is defined at the beginning of this file.
%Therefore it is also placed there and commented out here.
%</package>
%<*discard>
%\fi
%    \begin{macrocode}
\ifdefined\childdocmain\endinput\fi
%    \end{macrocode}
%\iffalse
%</discard>
%<*package>
%\fi
%
% \macro{\ifchilddoc}
% \macro{\ifchilddocmanual}
% The conditional |\ifchilddoc| tells whether a
% child (true) or main (false) document is being compiled.
% The conditional |\ifchilddocmanual| tells whether
% the |\includeonly| mechanism is used (false) or
% the selection of child files must be performed manually (true).
% The definitions initialise to false:
%    \begin{macrocode}
\newif\ifchilddoc
\newif\ifchilddocmanual
%    \end{macrocode}

% \macro{\childdocname}
% \macro{\childdocjob}
% The macro |\childdocname| stores the name of the main document
% to be compiled. The macro |\childdocjob| stores the name of
% the document on which the \LaTeX{} compiler was originally invoked.
% The content of |\jobname| cannot be compared
% to filenames specified in the source due to different catcodes.
% The following code rescans |\jobname|, stores the result
% in |\childdocname| and saves a copy in |\childdocjob|:
%    \begin{macrocode}
\edef\childdocname{\scantokens\expandafter{\jobname\noexpand}}
\let\childdocjob\childdocname
%    \end{macrocode}

% \macro{\childdocdisable}
% The macro |\childdocdisable| prevents the main file
% from being processed more than once.
% At this stage, the main document command |\childdocmain|
% is assumed to be called once again where it should do nothing.
% Any subsequent call to it should prevent
% a secondary processing of the main document
% It overwrites the forwarding commands
% |\childdocof| and |\childdocforward|
% with empty macros to prevent further inclusions of the main document:
%    \begin{macrocode}
\newcommand{\childdocdisable}
{
  \renewcommand{\childdocmain}[1]{\renewcommand{\childdocmain}[1]{\endinput}}
  \renewcommand{\childdocof}[1]{}
  \renewcommand{\childdocby}[2][]{}
  \renewcommand{\childdocforward}[2][]{}
  \renewcommand{\childdocdisable}{}
}
%    \end{macrocode}

% \macro{\childdocmain}
% The macro |\childdocmain| is to be called at the top of the main file
% with nothing or the main filename (without extension) as argument.
% First, it breaks loops.
% If the argument is not empty and does not match |\childdocname|
% (which is set by the first inclusion of |childdoc.def|),
% |\ifchilddoc| is set to true, |\includeonly| is applied to the child file
% and |\jobname| is set to the main file
% (for proper handling of |.aux| files):
%    \begin{macrocode}
\newcommand{\childdocmain}[1]
{
  \childdocdisable\childdocmain{}
  \if?#1?\else
    \begingroup
      \def\childdoctmp{#1}
      \ifx\childdoctmp\childdocname
        \def\childdoctmp{}
      \else
        \def\childdoctmp
        {
          \childdoctrue
          \includeonly{\childdocname}
          \def\childdocjob{#1}
          \def\jobname{#1}
        }
      \fi
      \expandafter
    \endgroup
    \childdoctmp
  \fi
}
%    \end{macrocode}

% \macro{\childdocof}
% The command |\childdocof| redirects
% compilation to the main file |#1|.
%    \begin{macrocode}
\newcommand{\childdocof}[1]
{
  \childdocdisable
  \childdoctrue
  \includeonly{\childdocname}
  \def\jobname{#1}
  \def\childdocjob{#1}
  \input{#1}
}
%    \end{macrocode}

% \macro{\childdocby}
% The command |\childdocby| ....
%    \begin{macrocode}
\newcommand{\childdocby}[2][]
{
  \childdocdisable
  \childdoctrue
  \childdocmanualtrue
  \if?#1?\else
    \def\jobname{#2}
  \fi
  \def\childdocjob{#2}
  \input{#2}
  \endinput
}
%    \end{macrocode}

% \macro{\childdocforward}
% The command |\childdocforward| redirects
% compilation to the main file or
% (if the optional argument is given) a child file.
% Parameters are set as if the main file
% or a child file starting with |\childdocof| was compiled.
% Then compilation is handed over to the main file:
%    \begin{macrocode}
\newcommand{\childdocforward}[2][]
{
  \begingroup
    \if?#1?
      \def\childdoctmp
      {
        \def\childdocname{#2}
        \def\childdocjob{#2}
        \def\jobname{#2}
        \input{#2}
        \endinput
      }
    \else
      \def\childdoctmp
      {
        \childdocdisable
        \def\childdocname{#2}
        \childdoctrue
        \includeonly{#2}
        \def\childdocjob{#1}
        \def\jobname{#1}
        \input{#1}
        \endinput
      }
    \fi
    \expandafter
  \endgroup
  \childdoctmp
}
%    \end{macrocode}

% \macro{\childdocforwardprefix}
% The command |\childdocforwardprefix| redirects
% compilation to the main or a child file by means of a pattern.
% The prefix |#1| in the current filename is replaced by |#2|
% and the suffix of the current filename is kept
% (it is assumed that the filename does not contain the substring `|~~~|'
% which is used as a delimiter).
% Compilation is handed over to the new file by |\childdocforward|:
%    \begin{macrocode}
\newcommand{\childdocforwardprefix}[3][]
{
  \begingroup
    \def\childdocextract #2##1~~~{\def\childdoctmp{\childdocforward[#1]{#3##1}}}
    \expandafter\childdocextract\childdocname~~~
    \expandafter
  \endgroup
  \childdoctmp
}
%    \end{macrocode}

% \macro{\childdoc}
% The deprecated macro |\childdoc| is a legacy version of |\childdocmain|:
%    \begin{macrocode}
\newcommand{\childdoc}{\childdocmain}
%    \end{macrocode}

% \macro{\childdocredirect}
% The deprecated macro |\childdocredirect| is a legacy version
% of |\childdocforward| and |\childdocforwardprefix|:
%    \begin{macrocode}
\newcommand{\childdocredirect}[2][]
{
  \begingroup
    \if?#1?
      \def\childdoctmp{\childdocforward{#2}}
    \else
      \def\childdoctmp{\childdocforwardprefix{#1}{#2}}
    \fi
    \expandafter
  \endgroup
  \childdoctmp
}
%    \end{macrocode}

%\iffalse
%</package>
%\fi
%
\endinput
\childdocforward{cdocsamp}"|\\
% |latex -jobname cdocscl1 \|\\
% |  "% \iffalse
%
% childdoc.dtx Copyright (C) 2017-2018 Niklas Beisert
%
% This work may be distributed and/or modified under the
% conditions of the LaTeX Project Public License, either version 1.3
% of this license or (at your option) any later version.
% The latest version of this license is in
%   http://www.latex-project.org/lppl.txt
% and version 1.3 or later is part of all distributions of LaTeX
% version 2005/12/01 or later.
%
% This work has the LPPL maintenance status `maintained'.
%
% The Current Maintainer of this work is Niklas Beisert.
%
% This work consists of the files childdoc.dtx and childdoc.ins
% and the derived files childdoc.def and cdocsamp.tex with
% cdocsch1.tex, cdocsch2.tex, cdocsdrf.tex, cdocsfn1.tex, cdocsfn2.tex.
%
%<package>\ifdefined\childdocmain\endinput\fi
%<package>\ProvidesFile{childdoc.def}[2018/12/30 v2.0 child document driver]
%<samplemain>\ProvidesFile{cdocsamp.tex}[2018/12/30 v2.0 sample for childdoc]
%<*driver>
%\ProvidesFile{childdoc.drv}[2018/12/30 v2.0 childdoc reference manual file]
\PassOptionsToClass{10pt,a4paper}{article}
\documentclass{ltxdoc}

\usepackage[margin=35mm]{geometry}
\usepackage{hyperref}
\usepackage{hyperxmp}
\usepackage[usenames]{color}

\hypersetup{colorlinks=true}
\hypersetup{pdfstartview=FitH}
\hypersetup{pdfpagemode=UseNone}
\hypersetup{pdfsource={}}
\hypersetup{pdflang={en-UK}}
\hypersetup{pdfcopyright={Copyright 2017-2018 Niklas Beisert.
  This work may be distributed and/or modified under the
  conditions of the LaTeX Project Public License, either version 1.3
  of this license or (at your option) any later version.}}
\hypersetup{pdflicenseurl={http://www.latex-project.org/lppl.txt}}
\hypersetup{pdfcontactaddress={ETH Zurich, ITP, HIT K,
  Wolfgang-Pauli-Strasse 27}}
\hypersetup{pdfcontactpostcode={8093}}
\hypersetup{pdfcontactcity={Zurich}}
\hypersetup{pdfcontactcountry={Switzerland}}
\hypersetup{pdfcontactemail={nbeisert@itp.phys.ethz.ch}}
\hypersetup{pdfcontacturl={http://people.phys.ethz.ch/\xmptilde nbeisert/}}

\newcommand{\secref}[1]{\hyperref[#1]{section \ref*{#1}}}

\parskip1ex
\parindent0pt
\let\olditemize\itemize
\def\itemize{\olditemize\parskip0pt}

\begin{document}

\title{The \textsf{childdoc} Package}
\hypersetup{pdftitle={The childdoc Package}}
\author{Niklas Beisert\\[2ex]
  Institut f\"ur Theoretische Physik\\
  Eidgen\"ossische Technische Hochschule Z\"urich\\
  Wolfgang-Pauli-Strasse 27, 8093 Z\"urich, Switzerland\\[1ex]
  \href{mailto:nbeisert@itp.phys.ethz.ch}
  {\texttt{nbeisert@itp.phys.ethz.ch}}}
\hypersetup{pdfauthor={Niklas Beisert}}
\hypersetup{pdfsubject={Manual for the LaTeX2e Package childdoc}}
\date{30 December 2018, \textsf{v2.0}}
\maketitle

\begin{abstract}\noindent
\textsf{childdoc} is a \LaTeXe{} package
that enables the direct compilation
of document sections included by |\include|
to individual files.
\end{abstract}

\begingroup
\parskip0ex
\tableofcontents
\endgroup

%%%%%%%%%%%%%%%%%%%%%%%%%%%%%%%%%%%%%%%%%%%%%%%%%%%%%%%%%%%%%%%%%%%%%%%%%%%%%%%%
%%%%%%%%%%%%%%%%%%%%%%%%%%%%%%%%%%%%%%%%%%%%%%%%%%%%%%%%%%%%%%%%%%%%%%%%%%%%%%%%
\section{Introduction}

\LaTeX{} provides a mechanism to structure a large document (such as a book)
into a main file and several child files (containing the chapters)
using the |\include| command.
This mechanism is beneficial for documents
which span hundreds of pages in order to
make the source file(s) more manageable.
Moreover, compilation can be restricted to
selected child files by means of the |\includeonly| command.
The latter feature can be used to reduce the compilation time while editing
(this was significantly more useful in the earlier days of \LaTeX{})
or to generate a smaller document which is easier to navigate.
Another application of |\includeonly| is to generate
documents consisting of selected parts of the complete document.

However, there are a few drawbacks of the plain |\include| mechanism:
\begin{itemize}
\item
The child files cannot be compiled on their own,
they can only be compiled via the main file.
A naive editing environment
(such as a text editor with an option
to have the current file processed by \LaTeX)
may require one to switch to the main file before compiling;
attempting to compile the child file produces errors.
\item
The main file must be modified (each time)
to adjust the |\includeonly| command
to the present needs. This easily leaves the main file in a messy state.
\item
The generated document will always carry the filename
of the main document. This is inconvenient if
several child files are to be compiled and
to be kept for distribution.
\end{itemize}

The present package provides a simple interface
to make child files individually compilable by \LaTeX{}.
Compiling a child file then has the same effect as compiling
the main file with an |\includeonly| command
to select the appropriate child.
Moreover the generated document will carry the name of the child
rather than the main file.
This resolves all three above issues.

This feature is meant to make the editing of books,
thesis documents and lecture notes somewhat more convenient.
However, the package can also be used efficiently for
composing a series of documents (such as exercise sheets)
which are typically distributed individually.
It then assists the author in generating the individual documents
(potentially in different versions)
as well as a document containing the collected series.
Another application is in developing style files
or other kinds of included material
where compilation of the style file could redirect
to a sample or test file.

%%%%%%%%%%%%%%%%%%%%%%%%%%%%%%%%%%%%%%%%%%%%%%%%%%%%%%%%%%%%%%%%%%%%%%%%%%%%%%%%
%%%%%%%%%%%%%%%%%%%%%%%%%%%%%%%%%%%%%%%%%%%%%%%%%%%%%%%%%%%%%%%%%%%%%%%%%%%%%%%%
\section{Usage}

First of all, the package \textsf{childdoc} is \emph{not} a standard
\LaTeXe{} |.sty| style file! Therefore it needs to be invoked in
a non-standard way.

%%%%%%%%%%%%%%%%%%%%%%%%%%%%%%%%%%%%%%%%%%%%%%%%%%%%%%%%%%%%%%%%%%%%%%%%%%%%%%%%
\subsection{Included Files}
\label{sec:include}

%%%%%%%%%%%%%%%%%%%%%%%%%%%%%%%%%%%%%%%%
\DescribeMacro{\childdocmain}
To use the package, add the commands
\begin{center}
\begin{tabular}{l}
|\input{childdoc.def}|\\
|\childdocmain{}|\\
\end{tabular}
\end{center}
at the very top of the main \LaTeX{} file,
in particular \emph{before} the |\documentclass| statement!
The argument of |\childdocmain| should be left empty
(but it must be present).

%%%%%%%%%%%%%%%%%%%%%%%%%%%%%%%%%%%%%%%%
\DescribeMacro{\childdocof}
Furthermore, add the commands
\begin{center}
\begin{tabular}{l}
|\input{childdoc.def}|\\
|\childdocof{|\textit{main}|}|\\
\end{tabular}
\end{center}
at the top of every child file \textit{child}
which is included by |\include{|\textit{child}|}|
from within the main file
(or at least for those files to be compiled individually).
The argument \textit{main} must be the filename of the main file.

There are a couple of
considerations in setting up the main and child documents:

%%%%%%%%%%%%%%%%%%%%%%%%%%%%%%%%%%%%%%%%
\paragraph{Restrictions.}

Please note the following restrictions:
\begin{itemize}
\item
|\childdocmain| must be called with one argument \textit{main}
to ensure compatibility with earlier version of the package.
It must either be empty (|\childdocmain{}|)
or precisely match the filename of the main file in which it is specified.
See \secref{sec:detection} for further information.
\item
The filename \textit{main} must be specified without the |.tex| extension.
\item
The filename \textit{main} is case sensitive
(even in case-insensitive file systems)
due to internal string comparison.
\item
The argument \textit{main} should be fully expanded, it cannot be a macro.
\item
Subdirectories and special characters should be avoided in filenames.
\item
The command |\childdocmain{|\textit{main}|}| must be followed by a whitespace.
It should not be followed immediately by another command
or by a comment mark `|%|'.
This is because the \TeX{} parser reads the token immediately following
the argument of |\childdocmain| and puts it
at the beginning of every child section;
however, a white\-space is ignored.
\end{itemize}

%%%%%%%%%%%%%%%%%%%%%%%%%%%%%%%%%%%%%%%%
\paragraph{Content of Main File.}

It is advisable to place all content in the child files included by |\include|.
Any output contained in the main file will appear in all child documents
unless suppressed manually;
it cannot be suppressed automatically by the |\includeonly| directive
and thus should normally be avoided.
A method to include some content in the main file
by means of conditional processing is described in \secref{sec:conditional}.

%%%%%%%%%%%%%%%%%%%%%%%%%%%%%%%%%%%%%%%%
\paragraph{Page Numbering.}

When only a part of the document is compiled,
the appropriate numbering of pages
(as well as other status parameters)
is determined from the |.aux| files.
The latter contain information from previous passes.
However this information needs to propagate through
all intermediate child documents.
Therefore the page numbering in child documents may well
be inconsistent until the complete document is compiled at least once.

A useful (if unconventional) way to always ensure a consistent
page numbering is to restart the numbering in each child document
and denote the pages by `\textit{child}|.|\textit{page}'
where \textit{child} represents the chapter/section number of the child file.
This can be achieved by the command
|\numberwithin{page}{|\textit{child}|}|
of the \textsf{amsmath} package
where \textit{child} can be |chapter| or |section|
depending on the chosen structuring.
Alternatively, one can modify the macro |\thepage| appropriately
and reset the counter |page| at the start of each child file.

%%%%%%%%%%%%%%%%%%%%%%%%%%%%%%%%%%%%%%%%%%%%%%%%%%%%%%%%%%%%%%%%%%%%%%%%%%%%%%%%
\subsection{Conditional Processing}
\label{sec:conditional}

The package provides a mechanism to compile different versions
of a document. To customise the versions further some conditional processing
can come in handy to distinguish which version is being compiled.
The package provides two macros to describe the compilation context:

%%%%%%%%%%%%%%%%%%%%%%%%%%%%%%%%%%%%%%%%
\DescribeMacro{\ifchilddoc}
The conditional |\ifchilddoc| distinguishes between the compilation of
child documents and the main document:
%
\begin{center}
|\ifchilddoc |\textit{child-code}| |[|\||else |\textit{main-code}]| \||fi|
\end{center}

%%%%%%%%%%%%%%%%%%%%%%%%%%%%%%%%%%%%%%%%
\DescribeMacro{\childdocname}
\DescribeMacro{\childdocjob}
The macro |\childdocname| contains the filename (without extension)
of the main or child file being processed.
Note that |\childdocjob| will always contain the name of the main file.

%%%%%%%%%%%%%%%%%%%%%%%%%%%%%%%%%%%%%%%%
\paragraph{Title Page.}

Conditional processing can be used to include a title or banner page
in the main document when proper precautions are taken.
Importantly, the code in the main file should ensure that the page counter
(as well as other status parameters which are stored in the |.aux| files)
takes the same value after the conditional processing.
Otherwise the page numbers may take divergent values
depending on which part is compiled.

For example, a title page could be declared by:
%
\begin{center}
\begin{tabular}{l}
|\ifchilddoc\||else|\\
|\addtocounter{page}{-1}|\\
\textit{code for title page}\\
|\newpage|\\
|\||fi|
\end{tabular}
\end{center}
%
A banner page for the child documents can be generated by:
%
\begin{center}
\begin{tabular}{l}
|\ifchilddoc|\\
|\addtocounter{page}{-1}|\\
\textit{code for banner page}\\
|\newpage|\\
|\||fi|
\end{tabular}
\end{center}
%
Here one could write a message such as:
\begin{center}
|This is the part \childdocname{} of \childdocjob{}.|
\end{center}

%%%%%%%%%%%%%%%%%%%%%%%%%%%%%%%%%%%%%%%%%%%%%%%%%%%%%%%%%%%%%%%%%%%%%%%%%%%%%%%%
\subsection{Flags}
\label{sec:flags}

The package makes it easy to generate different versions
of the main or child documents.
To this end compilation flags can be defined
and assigned different default values.
They will be particularly useful in conjunction
with the forwarding mechanism described in \secref{sec:forward}.

For example, it may be useful to have a flag |\version|
which can be set to |draft| or |final|.
The document source will contain some conditional code
depending on the value of |\version|.
Suppose further, the flag should default to |final| for the main file
and to |draft| for child files
which is a natural assignment for editing the document.
This is achieved by placing the following code
in the preamble of the main document
(below the |\childdocmain| directive):
%
\begin{center}
\begin{tabular}{l}
|\ifchilddoc|\\
|\providecommand{\version}{draft}|\\
|\||else|\\
|\providecommand{\version}{final}|\\
|\||fi|
\end{tabular}
\end{center}
%
The definition by |\providecommand| makes sure
that previous definitions are not overwritten.
Further statements |\providecommand{\version}{...}|
can thus be added before the above code to override it.

For the main file, one might add a line
(between |\childdocmain| and the above block)
%
\begin{center}
|%\ifchilddoc\||else\providecommand{\version}{draft}\||fi|
\end{center}
%
which can be uncommented to produce a draft version.
Likewise one can add a line to the very top of a child file
(above the |\childdocof{|\textit{main}|}| directive)
%
\begin{center}
|%\providecommand{\version}{final}|
\end{center}
%
which can be uncommented to produce the final version of this child document.

%%%%%%%%%%%%%%%%%%%%%%%%%%%%%%%%%%%%%%%%%%%%%%%%%%%%%%%%%%%%%%%%%%%%%%%%%%%%%%%%
\subsection{Forwarding}
\label{sec:forward}

Different versions of the main or child documents
using compilation flags as described in \secref{sec:flags}
can be (permanently) stored in different files
for convenient compilation, viewing and distribution.
To this end, the package defines a command
to pass on compilation to a different file:

%%%%%%%%%%%%%%%%%%%%%%%%%%%%%%%%%%%%%%%%
\DescribeMacro{\childdocforward}
The command |\childdocforward| redirects processing to
another source file:
%
\begin{center}
\begin{tabular}{l}
|\input{childdoc.def}|\\
|\childdocforward[|\textit{main}|]{|\textit{dest}|}|\\
\end{tabular}
\end{center}
%
The argument \textit{dest} is the destination file
(without extension).
It should be the main file or one of the child files.
Note that further \textsf{childdoc} directives
such as |\childdocof| and |\childdocforward|
in the indicated file will be processed in this form.
The optional argument \textit{main}
passes on directly to the main file \textit{main}
while pretending to compile the child \textit{dest}.
This form behaves as if \textit{dest}
issues |\childdocof{|\textit{main}|}| right away,
and no further \textsf{childdoc} directives will be processed.

%%%%%%%%%%%%%%%%%%%%%%%%%%%%%%%%%%%%%%%%
\DescribeMacro{\...prefix}
In the alternative form |\childdocforwardprefix|,
%
\begin{center}
\begin{tabular}{l}
|\input{childdoc.def}|\\
|\childdocforwardprefix[|\textit{main}|]{|\textit{prefix}|}{|\textit{dest}|}|
\end{tabular}
\end{center}
%
the destination file is determined by a pattern
depending on the current file:
To make this work, the current file must be called
`{\textit{prefix}\hspace{0.2em}\textit{suffix}}'
with \textit{prefix} matching precisely the argument.
Processing is then passed on to the file
`{\textit{dest}\hspace{0.2em}\textit{suffix}}'.
Surely, the same effect is achieved by
directly specifying the
argument `{\textit{dest}\hspace{0.2em}\textit{suffix}}'
in the first form.
However, that requires to set up a different file
for each child. With the alternative form of the command
all these files can have exactly the same content
which simplifies setting them up and maintaining them.

For example, the following file |draft.tex|
with a compilation flag |\version| as described in \secref{sec:flags}
compiles the main document as a draft:
%
\begin{center}
\begin{tabular}{l}
|\def\version{draft}|\\
|\input{childdoc.def}|\\
|\childdocforward{|\textit{main}|}|
\end{tabular}
\end{center}
%
Likewise, the following files |final|\textit{nn}|.tex|
compile the final version of the child document
|child|\textit{nn}|.tex|:
%
\begin{center}
\begin{tabular}{l}
|\def\version{final}|\\
|\input{childdoc.def}|\\
|\childdocforwardprefix{final}{child}|
\end{tabular}
\end{center}
%

Note that when several versions of a main file and/or of each child file
are to be generated, it may be convenient to set up a |Makefile| or
shell script to automatise the process.

%%%%%%%%%%%%%%%%%%%%%%%%%%%%%%%%%%%%%%%%%%%%%%%%%%%%%%%%%%%%%%%%%%%%%%%%%%%%%%%%
\subsection{Command Line Processing}
\label{sec:commandline}

The effect of redirection files can also be achieved by invoking
the \LaTeX{} compiler with a more elaborate command line.
Most conveniently this should be done as part
of a shell script or a |Makefile|.

When using \textsf{childdoc} in the main file, the following
command lines effectively perform a redirection
(note that depending on the shell being used,
backslashes may have to be doubled: `|\|' $\to$ `|\\|'):
%
\begin{center}
|... -jobname "|\textit{target}|" |\\|"|[\textit{flags}]%
|\input{childdoc.def}\childdocforward[|\textit{main}|]{|\textit{dest}|}"|
\end{center}
%
Here \textit{target} is the name of the output file,
\textit{main} is the name of the main file
and \textit{dest} is the name of the main or child file to be processed
(all filenames without extensions).
The optional argument \textit{main} can be omitted
if \textit{main} matches \textit{dest}.
Optionally, compilation \textit{flags} can be defined via |\def| commands.
This command line makes the \TeX{} engine believe
it is compiling the file \textit{target}
whose content is specified as the latter parameter.
The provided code then forwards the processing to
\textit{main} or \textit{dest} as described in \secref{sec:forward}.

%%%%%%%%%%%%%%%%%%%%%%%%%%%%%%%%%%%%%%%%%%%%%%%%%%%%%%%%%%%%%%%%%%%%%%%%%%%%%%%%
\subsection{Include by Input}
\label{sec:input}

Including child documents by |\include| has some restrictions by design.
Most notably, the content of a child document always occupies
its own set of pages; pages cannot be shared between child documents.
Usually, this behaviour makes perfect sense
because each child document contain an essential part of the document.
However, in some situations it may be desirable to compose
a document from a collection of parts
without having mandatory page breaks between then.
For this case, the package
provides a mechanism to include parts
by |\input| which can also be processed individually.
However, by construction this mechanism
requires manual handling of the content to be output.

%%%%%%%%%%%%%%%%%%%%%%%%%%%%%%%%%%%%%%%%
\DescribeMacro{\ifchilddocmanual}
The main file should be prepared as usual, see \secref{sec:include}.
However, the document body must make a distinction
between processing of an individual part and of the main document, e.g.:
%
\begin{center}
\begin{tabular}{l}
|\ifchilddocmanual|\\
|\input{\childdocname}|\\
|\||else|\\
\textit{document body with }|\input{|\textit{part}|}|\\
|\||fi|
\end{tabular}
\end{center}
%
The conditional |\ifchilddocmanual| is true whenever
a part to be included by |\input| is being compiled,
and the name of the part is stored in |\childdocname|.

%%%%%%%%%%%%%%%%%%%%%%%%%%%%%%%%%%%%%%%%
\DescribeMacro{\childdocby}
Each part to be included by |\input| should start with:
%
\begin{center}
\begin{tabular}{l}
|\input{childdoc.def}|\\
|\childdocby{|\textit{main}|}|\\
\end{tabular}
\end{center}
%
The directive |\childdocby| is similar to |\childdocof|
described in \secref{sec:include},
but the subsequent selection of content must be done manually.
To that end, both |\ifchilddoc| and |\ifchilddocmanual|
will be true upon processing of a part,
and the name of the part is stored in |\childdocname|.
Note that |\jobname| will be set to the filename of the current part
so that each part receives an individual |.aux| file
that does not interfere with the |.aux| file(s) of the main document.
This behaviour can be altered by the alternative form
|\childdocby[*]{|\textit{main}|}| (with a non-empty optional argument)
which uses the |.aux| file of the main document
by setting |\jobname| to \textit{main}.

%%%%%%%%%%%%%%%%%%%%%%%%%%%%%%%%%%%%%%%%%%%%%%%%%%%%%%%%%%%%%%%%%%%%%%%%%%%%%%%%
\subsection{Driver Development}
\label{sec:driver}

The \textsf{childdoc} mechanism can also be use for the development
of definition files such as \LaTeX{} styles or classes.
This case differs from the above setup with multiple parts
included by |\include| in that no |\includeonly| should be invoked.
This can be achieved by starting the include file
(before |\ProvidesPackage|) with:
%
\begin{center}
\begin{tabular}{l}
|\input{childdoc.def}|\\
|\childdocforward{|\textit{main}|}|\\
\end{tabular}
\end{center}
%
or alternatively with:
%
\begin{center}
\begin{tabular}{l}
|\input{childdoc.def}|\\
|\childdocby{|\textit{main}|}|\\
\end{tabular}
\end{center}
%
Both forms have slightly different effects as described above.
The main file is prepared as usual, see \secref{sec:include}.

%%%%%%%%%%%%%%%%%%%%%%%%%%%%%%%%%%%%%%%%%%%%%%%%%%%%%%%%%%%%%%%%%%%%%%%%%%%%%%%%
\subsection{Legacy Detection}
\label{sec:detection}

The directive |\childdocmain| in the main file can detect
whether the complete document or merely a child is to be compiled
even without using the directive |\childdocof|.
This method is deprecated because it is less robust
and there is no compelling reason to use it;
it is merely provided for backward compatibility
and it may be removed in future versions.

If the detection mechanism is to be used,
it is mandatory to correctly specify
the filename of the main file as the argument of |\childdocmain|:
%
\begin{center}
\begin{tabular}{l}
|\input{childdoc.def}|\\
|\childdocmain{|\textit{main}|}|\\
\end{tabular}
\end{center}
%
If |\jobname| does not match the argument \textit{main} of |\childdocmain|,
it is assumed that |\jobname| points to the child file to be compiled.
When using |\childdocmain| with the main file specified as argument,
it suffices to start a child file
with just |\input{|\textit{main}|}|
without loading of the package and using |\childdocof|.
If instead all processing is done
with the appropriate \textsf{childdoc} directives,
the argument of \textit{main} of |\childdocmain| can be empty.

An alternative version of the command line processing described
in \secref{sec:commandline} using the detection mechanism reads:
%
\begin{center}
|... -jobname "|\textit{target}|" "|[\textit{flags}]%
[|\def\jobname{|\textit{dest}|}|]|\input{|\textit{main}|}"|
\end{center}

%%%%%%%%%%%%%%%%%%%%%%%%%%%%%%%%%%%%%%%%%%%%%%%%%%%%%%%%%%%%%%%%%%%%%%%%%%%%%%%%
\subsection{Manual Code}
\label{sec:manual}

In case one cannot be certain whether the definitions file |childdoc.def|
is installed on the target \TeX{} distribution
and one prefers not to ship it,
it is conceivable to paste a few relevant commands into the sources.

To that end, drop all statements |\input{childdoc.def}|
and perform the replacements as outlined below.
Instead of |\childdocmain{|\textit{main}|}| add the following code
to the top of the main file:
%
\begin{center}
\begin{tabular}{l}
|\||ifdefined\childdocname\endinput\||fi\newif\ifchilddoc|\\
|\edef\childdocname{\scantokens\expandafter{\jobname\noexpand}}|\\
|\def\childdocmain{|\textit{main}|}\||ifx\childdocmain\childdocname\||else|\\
|\childdoctrue\includeonly{\childdocname}\let\jobname\childdocmain\||fi|\\
\end{tabular}
\end{center}
%
Instead of |\childdocof{|\textit{main}|}| just include the main file
at the top of each child file:
%
\begin{center}
|\input{|\textit{main}|}|
\end{center}
%
A simple redirection |\childdocforward{|\textit{dest}|}| is achieved by:
%
\begin{center}
|\def\jobname{|\textit{dest}|}\input{\jobname}|
\end{center}
%
The redirection with prefix
|\childdocforwardprefix[|\textit{prefix}|]{|\textit{dest}|}|
is accomplished by:
%
\begin{center}
\begin{tabular}{l}
|{\edef\jobname{\scantokens\expandafter{\jobname\noexpand}}|\\
|\def\redirectjob |\textit{prefix}|#1~~~{\gdef\jobname{|\textit{dest}|#1}}|\\
|\expandafter\redirectjob\jobname~~~}\input{\jobname}|
\end{tabular}
\end{center}

In an alternative approach,
child documents can be compiled by a specific command line
without additional code or specific definitions:
%
\begin{center}
|... -jobname "|\textit{target}|" "|[\textit{flags}]%
|\includeonly{|\textit{dest}|}\input{|\textit{main}|}"|
\end{center}
%

%%%%%%%%%%%%%%%%%%%%%%%%%%%%%%%%%%%%%%%%%%%%%%%%%%%%%%%%%%%%%%%%%%%%%%%%%%%%%%%%
%%%%%%%%%%%%%%%%%%%%%%%%%%%%%%%%%%%%%%%%%%%%%%%%%%%%%%%%%%%%%%%%%%%%%%%%%%%%%%%%
\section{Information}

%%%%%%%%%%%%%%%%%%%%%%%%%%%%%%%%%%%%%%%%%%%%%%%%%%%%%%%%%%%%%%%%%%%%%%%%%%%%%%%%
\subsection{Copyright}

Copyright \copyright{} 2017--2018 Niklas Beisert

This work may be distributed and/or modified under the
conditions of the \LaTeX{} Project Public License, either version 1.3
of this license or (at your option) any later version.
The latest version of this license is in
  \url{http://www.latex-project.org/lppl.txt}
and version 1.3 or later is part of all distributions of \LaTeX{}
version 2005/12/01 or later.

This work has the LPPL maintenance status `maintained'.

The Current Maintainer of this work is Niklas Beisert.

This work consists of the files |README.txt|, |childdoc.ins| and |childdoc.dtx|
as well as the derived files |childdoc.def|, |cdocsamp.tex|
with |cdocsch1.tex|, |cdocsch2.tex|, |cdocspt3.tex|, |cdocspt4.tex|,
|cdocsdrf.tex|, |cdocsfn1.tex|, |cdocsfn2.tex|
as well as |childdoc.pdf|.

%%%%%%%%%%%%%%%%%%%%%%%%%%%%%%%%%%%%%%%%%%%%%%%%%%%%%%%%%%%%%%%%%%%%%%%%%%%%%%%%
\subsection{Files and Installation}

The package consists of the files:
%
\begin{center}
\begin{tabular}{ll}
    |README.txt|   & readme file \\
    |childdoc.ins| & installation file \\
    |childdoc.dtx| & source file \\
    |childdoc.def| & definition file \\
    |cdocsamp.tex| & sample main file \\
    |cdocsch1.tex| & sample include file \\
    |cdocsch2.tex| & sample include file \\
    |cdocspt3.tex| & sample part file \\
    |cdocspt4.tex| & sample part file \\
    |cdocsdrf.tex| & sample redirection file \\
    |cdocsfn1.tex| & sample redirection file \\
    |cdocsfn2.tex| & sample redirection file \\
    |childdoc.pdf| & manual
\end{tabular}
\end{center}
%
The distribution consists of the files
|README.txt|, |childdoc.ins| and |childdoc.dtx|.
%
\begin{itemize}
\item
Run (pdf)\LaTeX{} on |childdoc.dtx|
to compile the manual |childdoc.pdf| (this file).
\item
Run \LaTeX{} on |childdoc.ins| to create the definitions file |childdoc.def|
and the sample |cdocsamp.tex| with include files
|cdocsch1.tex|, |cdocsch2.tex|, |cdocspt3.tex|, |cdocspt4.tex|,
|cdocsdrf.tex|, |cdocsfn1.tex|, |cdocsfn2.tex|.
Then copy the file |childdoc.def| to an appropriate directory of your \LaTeX{}
distribution, e.g.\ \textit{texmf-root}|/tex/latex/childdoc|.
\end{itemize}

%%%%%%%%%%%%%%%%%%%%%%%%%%%%%%%%%%%%%%%%%%%%%%%%%%%%%%%%%%%%%%%%%%%%%%%%%%%%%%%%
\subsection{Related CTAN Packages}

There are several other packages which offer a similar functionality:
%
\begin{itemize}
\item
The packages
\href{http://ctan.org/pkg/docmute}{\textsf{docmute}},
\href{http://ctan.org/pkg/includex}{\textsf{includex}} and
\href{http://ctan.org/pkg/standalone}{\textsf{standalone}}
provide commands to include only the document body of
a child file thus allowing both files to be compiled individually.
\item
The packages \href{http://ctan.org/pkg/subdocs}{\textsf{subdocs}}
and \href{http://ctan.org/pkg/subfiles}{\textsf{subfiles}}
provide structures in which the main and child documents can be
encapsulated and allowing them to be compiled individually.
The inclusion mechanism is different from the conventional |\include|.
\item
The package \href{http://ctan.org/pkg/combine}{\textsf{combine}}
is an elaborate solution to combine several documents into one.
\end{itemize}
%
See also the CTAN topic \href{http://ctan.org/topic/subdocs}{\textsf{subdocs}}
for further related packages.
The present package differs from the above solutions in that
a document structure constructed with the conventional |\include| mechanism
just needs two extra commands at the top of every file
such that all constituent files can be compiled individually.

%%%%%%%%%%%%%%%%%%%%%%%%%%%%%%%%%%%%%%%%%%%%%%%%%%%%%%%%%%%%%%%%%%%%%%%%%%%%%%%%
%\subsection{Feature Suggestions}
%
%The following is a list of features which may be useful for future
%versions of this package:
%%
%\begin{itemize}
%\item
%\ldots
%\end{itemize}

%%%%%%%%%%%%%%%%%%%%%%%%%%%%%%%%%%%%%%%%%%%%%%%%%%%%%%%%%%%%%%%%%%%%%%%%%%%%%%%%
\subsection{Revision History}

%%%%%%%%%%%%%%%%%%%%%%%%%%%%%%%%%%%%%%%%
\paragraph{v2.0:} 2018/12/30

\begin{itemize}
\item
immediate forward processing
\item
added |\childdocby| mechanism
\item
manual restructured
\end{itemize}

%%%%%%%%%%%%%%%%%%%%%%%%%%%%%%%%%%%%%%%%
\paragraph{v1.6:} 2018/01/17

\begin{itemize}
\item
application for development of include files
\item
corrections to manual
\end{itemize}

%%%%%%%%%%%%%%%%%%%%%%%%%%%%%%%%%%%%%%%%
\paragraph{v1.5:} 2017/05/21

\begin{itemize}
\item
more complete structuring introduced
\item
|\childdocof| introduced
\item
|\childdoc| renamed to |\childdocmain|
\item
|\childredirect| renamed to |\childdocforward| and |\childdocforwardprefix|
and functionality expanded
\end{itemize}

%%%%%%%%%%%%%%%%%%%%%%%%%%%%%%%%%%%%%%%%
\paragraph{v1.0:} 2017/04/27

\begin{itemize}
\item
manual and install package
\item
first version published on CTAN
\end{itemize}

%%%%%%%%%%%%%%%%%%%%%%%%%%%%%%%%%%%%%%%%
\paragraph{v0.6:} 2017/04/26

\begin{itemize}
\item
redirection mechanism added
\end{itemize}

%%%%%%%%%%%%%%%%%%%%%%%%%%%%%%%%%%%%%%%%
\paragraph{v0.5:} 2017/04/26

\begin{itemize}
\item
functionality in definition file
\end{itemize}


%%%%%%%%%%%%%%%%%%%%%%%%%%%%%%%%%%%%%%%%%%%%%%%%%%%%%%%%%%%%%%%%%%%%%%%%%%%%%%%%
%%%%%%%%%%%%%%%%%%%%%%%%%%%%%%%%%%%%%%%%%%%%%%%%%%%%%%%%%%%%%%%%%%%%%%%%%%%%%%%%
%%%%%%%%%%%%%%%%%%%%%%%%%%%%%%%%%%%%%%%%%%%%%%%%%%%%%%%%%%%%%%%%%%%%%%%%%%%%%%%%
\appendix

\settowidth\MacroIndent{\rmfamily\scriptsize 000\ }

 \DocInput{childdoc.dtx}

\end{document}
%</driver>
% \fi
%
% %%%%%%%%%%%%%%%%%%%%%%%%%%%%%%%%%%%%%%%%%%%%%%%%%%%%%%%%%%%%%%%%%%%%%%%%%%%%%%
% %%%%%%%%%%%%%%%%%%%%%%%%%%%%%%%%%%%%%%%%%%%%%%%%%%%%%%%%%%%%%%%%%%%%%%%%%%%%%%
% \section{Sample}
%\iffalse
%<*samplemain>
%\fi
%
% The following presents a sample document
% with two chapters, two parts, a title page,
% a compile flag as well as three forwarding files to set the flag.
% It consists of eight |.tex| files:
% \begin{center}
% \begin{tabular}{ll}
% |cdocsamp.tex|&main file\\
% |cdocsch1.tex|&include file for chapter 1\\
% |cdocsch2.tex|&include file for chapter 2\\
% |cdocspt3.tex|&include file for part 3\\
% |cdocspt4.tex|&include file for part 4\\
% |cdocsdrf.tex|&forwarding file for main file in draft mode\\
% |cdocsfi1.tex|&forwarding file for final version of chapter 1\\
% |cdocsfi2.tex|&forwarding file for final version of chapter 2\\
% \end{tabular}
% \end{center}
% Each of the eight files can be compiled directly by the \LaTeX{} compiler.
%
% %%%%%%%%%%%%%%%%%%%%%%%%%%%%%%%%%%%%%%
% \paragraph{Main File.}
%
% The main file is called |cdocsamp.tex|.
%
% Load the \textsf{childdoc} definitions and
% declare the filename for the main document:
%    \begin{macrocode}
\input{childdoc.def}
\childdocmain{}
%    \end{macrocode}

% Optional override for |\version| flag:
%    \begin{macrocode}
%%\ifchilddoc\else\providecommand{\version}{draft}\fi
%    \end{macrocode}

% Define the default values for the |\version| flag
% (|final| for the main file and |draft| for childs):
%    \begin{macrocode}
\ifchilddoc
\providecommand{\version}{draft}
\else
\providecommand{\version}{final}
\fi
%    \end{macrocode}

% Load the standard document class:
%    \begin{macrocode}
\documentclass[12pt]{article}
%    \end{macrocode}

% Start the document body:
%    \begin{macrocode}
\begin{document}
%    \end{macrocode}

% Declare a title page.
% Print title, part of document being processed and version flag:
%    \begin{macrocode}
\addtocounter{page}{-1}
\begin{center}
{\LARGE\bfseries{}childdoc example\par}
\vspace{1cm}
\ifchilddoc
\ifchilddocmanual part\else chapter\fi:
`\childdocname' of `\childdocjob'\par
\else
main document: `\childdocjob'\par
\fi
version: \version\par
\end{center}
\newpage
%    \end{macrocode}

% Manually include selected file,
% otherwise process as usual:
%    \begin{macrocode}
\ifchilddocmanual
\section*{part `\childdocname'}
\input{\childdocname}
\else
%    \end{macrocode}

% Include the two chapters:
%    \begin{macrocode}
\include{cdocsch1}
\include{cdocsch2}
%    \end{macrocode}

% Include the two parts unless only chapters should be displayed:
%    \begin{macrocode}
\ifchilddoc\else
\section{part three}
\input{cdocspt3}
\section{part four}
\input{cdocspt4}
\fi
%    \end{macrocode}

% Process as usual until here:
%    \begin{macrocode}
\fi
%    \end{macrocode}

% End of document body:
%    \begin{macrocode}
\end{document}
%    \end{macrocode}
%\iffalse
%</samplemain>
%\fi
%
% %%%%%%%%%%%%%%%%%%%%%%%%%%%%%%%%%%%%%%
% \paragraph{Chapter Include Files.}
%
% The include files are called |cdocsch1.tex| and |cdocsch2.tex|.
%
%\iffalse
%<*samplechap1|samplechap2>
%\fi

% Optional override for |\version| flag:
%    \begin{macrocode}
%%\providecommand{\version}{final}
%    \end{macrocode}

% Include the main document:
%    \begin{macrocode}
\input{childdoc.def}
\childdocof{cdocsamp}
%    \end{macrocode}

%\iffalse
%</samplechap1|samplechap2>
%\fi
%
%\iffalse
%<*samplechap1>
%\fi
% Some text for chapter 1:
%    \begin{macrocode}
\section{one}
some text in chapter one
%    \end{macrocode}

%\iffalse
%</samplechap1>
%\fi
% Some text for chapter 2:
%\iffalse
%<*samplechap2>
%\fi
%    \begin{macrocode}
\section{two}
more text in chapter two
%    \end{macrocode}

%\iffalse
%</samplechap2>
%\fi
%
% %%%%%%%%%%%%%%%%%%%%%%%%%%%%%%%%%%%%%%
% \paragraph{Part Include Files.}
%
% The include files are called |cdocspt3.tex| and |cdocspt4.tex|.
%
%\iffalse
%<*samplepart3|samplepart4>
%\fi

% Optional override for |\version| flag:
%    \begin{macrocode}
%%\providecommand{\version}{final}
%    \end{macrocode}

% Include the main document:
%    \begin{macrocode}
\input{childdoc.def}
\childdocby{cdocsamp}
%    \end{macrocode}

%\iffalse
%</samplepart3|samplepart4>
%\fi
%
%\iffalse
%<*samplepart3>
%\fi
% Some text for part 3:
%    \begin{macrocode}
some text in part three
%    \end{macrocode}

%\iffalse
%</samplepart3>
%\fi
% Some text for part 4:
%\iffalse
%<*samplepart4>
%\fi
%    \begin{macrocode}
more text in part four
%    \end{macrocode}

%\iffalse
%</samplepart4>
%\fi
%
% %%%%%%%%%%%%%%%%%%%%%%%%%%%%%%%%%%%%%%
% \paragraph{Forwarding for a Complete Draft.}
%
% The following forwarding file |cdocsdrf.tex|
% compiles the main document in draft mode:
%\iffalse
%<*sampledraft>
%\fi
%    \begin{macrocode}
\def\version{draft}
\input{childdoc.def}
\childdocforward{cdocsamp}
%    \end{macrocode}

%\iffalse
%</sampledraft>
%\fi
%
% %%%%%%%%%%%%%%%%%%%%%%%%%%%%%%%%%%%%%%
% \paragraph{Forwarding for Final Version of the Chapters.}
%
% The following forwarding files |cdocsfn1.tex| and |cdocsfn2.tex|
% (with identical content)
% compile the final versions of the child documents
% |cdocsch1.tex| and |cdocsch2.tex|, respectively:
%\iffalse
%<*samplefinal>
%\fi
%    \begin{macrocode}
\def\version{final}
\input{childdoc.def}
\childdocforwardprefix[cdocsamp]{cdocsfn}{cdocsch}
%    \end{macrocode}

%\iffalse
%</samplefinal>
%\fi
%
% %%%%%%%%%%%%%%%%%%%%%%%%%%%%%%%%%%%%%%
% \paragraph{Command Line Processing.}
%
% The following three command lines generate the output files
% |cdocscld|, |cdocscl1| and |cdocscl2|
% which should be identical to
% |cdocsdrf|, |cdocsch1| and |cdocsfn2|, respectively:
% \begin{center}
% \begin{tabular}{l}
% |latex -jobname cdocscld \|\\
% |  "\def\version{draft}\input{childdoc.def}\childdocforward{cdocsamp}"|\\
% |latex -jobname cdocscl1 \|\\
% |  "\input{childdoc.def}\childdocforward[cdocsamp]{cdocsch1}"|\\
% |latex -jobname cdocscl2 \|\\
% |  "\def\version{final}\input{childdoc.def}\childdocforward{cdocsch2}"|
% \end{tabular}
% \end{center}
% Note that the trailing backslash on each first line
% merely continues the input to the second line
% (for convenient cut ant paste).
% Furthermore, the command |latex| can be replaced by any
% of its alternative versions such as |pdflatex|.
%
% %%%%%%%%%%%%%%%%%%%%%%%%%%%%%%%%%%%%%%%%%%%%%%%%%%%%%%%%%%%%%%%%%%%%%%%%%%%%%%
% %%%%%%%%%%%%%%%%%%%%%%%%%%%%%%%%%%%%%%%%%%%%%%%%%%%%%%%%%%%%%%%%%%%%%%%%%%%%%%
% \section{Implementation}
%\iffalse
%<*package>
%\fi
%
% This section describes the definitions file |childdoc.def|.

% The definitions cannot be loaded using |\usepackage| or |\RequirePackage|
% which has a mechanism to prevent loading a style file more than once.
% When loading the definitions by means of |\input|
% multiple instances have to be prevented manually:
%\iffalse
%This code needs to be before the `\ProvidesFile' directive
%which is defined at the beginning of this file.
%Therefore it is also placed there and commented out here.
%</package>
%<*discard>
%\fi
%    \begin{macrocode}
\ifdefined\childdocmain\endinput\fi
%    \end{macrocode}
%\iffalse
%</discard>
%<*package>
%\fi
%
% \macro{\ifchilddoc}
% \macro{\ifchilddocmanual}
% The conditional |\ifchilddoc| tells whether a
% child (true) or main (false) document is being compiled.
% The conditional |\ifchilddocmanual| tells whether
% the |\includeonly| mechanism is used (false) or
% the selection of child files must be performed manually (true).
% The definitions initialise to false:
%    \begin{macrocode}
\newif\ifchilddoc
\newif\ifchilddocmanual
%    \end{macrocode}

% \macro{\childdocname}
% \macro{\childdocjob}
% The macro |\childdocname| stores the name of the main document
% to be compiled. The macro |\childdocjob| stores the name of
% the document on which the \LaTeX{} compiler was originally invoked.
% The content of |\jobname| cannot be compared
% to filenames specified in the source due to different catcodes.
% The following code rescans |\jobname|, stores the result
% in |\childdocname| and saves a copy in |\childdocjob|:
%    \begin{macrocode}
\edef\childdocname{\scantokens\expandafter{\jobname\noexpand}}
\let\childdocjob\childdocname
%    \end{macrocode}

% \macro{\childdocdisable}
% The macro |\childdocdisable| prevents the main file
% from being processed more than once.
% At this stage, the main document command |\childdocmain|
% is assumed to be called once again where it should do nothing.
% Any subsequent call to it should prevent
% a secondary processing of the main document
% It overwrites the forwarding commands
% |\childdocof| and |\childdocforward|
% with empty macros to prevent further inclusions of the main document:
%    \begin{macrocode}
\newcommand{\childdocdisable}
{
  \renewcommand{\childdocmain}[1]{\renewcommand{\childdocmain}[1]{\endinput}}
  \renewcommand{\childdocof}[1]{}
  \renewcommand{\childdocby}[2][]{}
  \renewcommand{\childdocforward}[2][]{}
  \renewcommand{\childdocdisable}{}
}
%    \end{macrocode}

% \macro{\childdocmain}
% The macro |\childdocmain| is to be called at the top of the main file
% with nothing or the main filename (without extension) as argument.
% First, it breaks loops.
% If the argument is not empty and does not match |\childdocname|
% (which is set by the first inclusion of |childdoc.def|),
% |\ifchilddoc| is set to true, |\includeonly| is applied to the child file
% and |\jobname| is set to the main file
% (for proper handling of |.aux| files):
%    \begin{macrocode}
\newcommand{\childdocmain}[1]
{
  \childdocdisable\childdocmain{}
  \if?#1?\else
    \begingroup
      \def\childdoctmp{#1}
      \ifx\childdoctmp\childdocname
        \def\childdoctmp{}
      \else
        \def\childdoctmp
        {
          \childdoctrue
          \includeonly{\childdocname}
          \def\childdocjob{#1}
          \def\jobname{#1}
        }
      \fi
      \expandafter
    \endgroup
    \childdoctmp
  \fi
}
%    \end{macrocode}

% \macro{\childdocof}
% The command |\childdocof| redirects
% compilation to the main file |#1|.
%    \begin{macrocode}
\newcommand{\childdocof}[1]
{
  \childdocdisable
  \childdoctrue
  \includeonly{\childdocname}
  \def\jobname{#1}
  \def\childdocjob{#1}
  \input{#1}
}
%    \end{macrocode}

% \macro{\childdocby}
% The command |\childdocby| ....
%    \begin{macrocode}
\newcommand{\childdocby}[2][]
{
  \childdocdisable
  \childdoctrue
  \childdocmanualtrue
  \if?#1?\else
    \def\jobname{#2}
  \fi
  \def\childdocjob{#2}
  \input{#2}
  \endinput
}
%    \end{macrocode}

% \macro{\childdocforward}
% The command |\childdocforward| redirects
% compilation to the main file or
% (if the optional argument is given) a child file.
% Parameters are set as if the main file
% or a child file starting with |\childdocof| was compiled.
% Then compilation is handed over to the main file:
%    \begin{macrocode}
\newcommand{\childdocforward}[2][]
{
  \begingroup
    \if?#1?
      \def\childdoctmp
      {
        \def\childdocname{#2}
        \def\childdocjob{#2}
        \def\jobname{#2}
        \input{#2}
        \endinput
      }
    \else
      \def\childdoctmp
      {
        \childdocdisable
        \def\childdocname{#2}
        \childdoctrue
        \includeonly{#2}
        \def\childdocjob{#1}
        \def\jobname{#1}
        \input{#1}
        \endinput
      }
    \fi
    \expandafter
  \endgroup
  \childdoctmp
}
%    \end{macrocode}

% \macro{\childdocforwardprefix}
% The command |\childdocforwardprefix| redirects
% compilation to the main or a child file by means of a pattern.
% The prefix |#1| in the current filename is replaced by |#2|
% and the suffix of the current filename is kept
% (it is assumed that the filename does not contain the substring `|~~~|'
% which is used as a delimiter).
% Compilation is handed over to the new file by |\childdocforward|:
%    \begin{macrocode}
\newcommand{\childdocforwardprefix}[3][]
{
  \begingroup
    \def\childdocextract #2##1~~~{\def\childdoctmp{\childdocforward[#1]{#3##1}}}
    \expandafter\childdocextract\childdocname~~~
    \expandafter
  \endgroup
  \childdoctmp
}
%    \end{macrocode}

% \macro{\childdoc}
% The deprecated macro |\childdoc| is a legacy version of |\childdocmain|:
%    \begin{macrocode}
\newcommand{\childdoc}{\childdocmain}
%    \end{macrocode}

% \macro{\childdocredirect}
% The deprecated macro |\childdocredirect| is a legacy version
% of |\childdocforward| and |\childdocforwardprefix|:
%    \begin{macrocode}
\newcommand{\childdocredirect}[2][]
{
  \begingroup
    \if?#1?
      \def\childdoctmp{\childdocforward{#2}}
    \else
      \def\childdoctmp{\childdocforwardprefix{#1}{#2}}
    \fi
    \expandafter
  \endgroup
  \childdoctmp
}
%    \end{macrocode}

%\iffalse
%</package>
%\fi
%
\endinput
\childdocforward[cdocsamp]{cdocsch1}"|\\
% |latex -jobname cdocscl2 \|\\
% |  "\def\version{final}% \iffalse
%
% childdoc.dtx Copyright (C) 2017-2018 Niklas Beisert
%
% This work may be distributed and/or modified under the
% conditions of the LaTeX Project Public License, either version 1.3
% of this license or (at your option) any later version.
% The latest version of this license is in
%   http://www.latex-project.org/lppl.txt
% and version 1.3 or later is part of all distributions of LaTeX
% version 2005/12/01 or later.
%
% This work has the LPPL maintenance status `maintained'.
%
% The Current Maintainer of this work is Niklas Beisert.
%
% This work consists of the files childdoc.dtx and childdoc.ins
% and the derived files childdoc.def and cdocsamp.tex with
% cdocsch1.tex, cdocsch2.tex, cdocsdrf.tex, cdocsfn1.tex, cdocsfn2.tex.
%
%<package>\ifdefined\childdocmain\endinput\fi
%<package>\ProvidesFile{childdoc.def}[2018/12/30 v2.0 child document driver]
%<samplemain>\ProvidesFile{cdocsamp.tex}[2018/12/30 v2.0 sample for childdoc]
%<*driver>
%\ProvidesFile{childdoc.drv}[2018/12/30 v2.0 childdoc reference manual file]
\PassOptionsToClass{10pt,a4paper}{article}
\documentclass{ltxdoc}

\usepackage[margin=35mm]{geometry}
\usepackage{hyperref}
\usepackage{hyperxmp}
\usepackage[usenames]{color}

\hypersetup{colorlinks=true}
\hypersetup{pdfstartview=FitH}
\hypersetup{pdfpagemode=UseNone}
\hypersetup{pdfsource={}}
\hypersetup{pdflang={en-UK}}
\hypersetup{pdfcopyright={Copyright 2017-2018 Niklas Beisert.
  This work may be distributed and/or modified under the
  conditions of the LaTeX Project Public License, either version 1.3
  of this license or (at your option) any later version.}}
\hypersetup{pdflicenseurl={http://www.latex-project.org/lppl.txt}}
\hypersetup{pdfcontactaddress={ETH Zurich, ITP, HIT K,
  Wolfgang-Pauli-Strasse 27}}
\hypersetup{pdfcontactpostcode={8093}}
\hypersetup{pdfcontactcity={Zurich}}
\hypersetup{pdfcontactcountry={Switzerland}}
\hypersetup{pdfcontactemail={nbeisert@itp.phys.ethz.ch}}
\hypersetup{pdfcontacturl={http://people.phys.ethz.ch/\xmptilde nbeisert/}}

\newcommand{\secref}[1]{\hyperref[#1]{section \ref*{#1}}}

\parskip1ex
\parindent0pt
\let\olditemize\itemize
\def\itemize{\olditemize\parskip0pt}

\begin{document}

\title{The \textsf{childdoc} Package}
\hypersetup{pdftitle={The childdoc Package}}
\author{Niklas Beisert\\[2ex]
  Institut f\"ur Theoretische Physik\\
  Eidgen\"ossische Technische Hochschule Z\"urich\\
  Wolfgang-Pauli-Strasse 27, 8093 Z\"urich, Switzerland\\[1ex]
  \href{mailto:nbeisert@itp.phys.ethz.ch}
  {\texttt{nbeisert@itp.phys.ethz.ch}}}
\hypersetup{pdfauthor={Niklas Beisert}}
\hypersetup{pdfsubject={Manual for the LaTeX2e Package childdoc}}
\date{30 December 2018, \textsf{v2.0}}
\maketitle

\begin{abstract}\noindent
\textsf{childdoc} is a \LaTeXe{} package
that enables the direct compilation
of document sections included by |\include|
to individual files.
\end{abstract}

\begingroup
\parskip0ex
\tableofcontents
\endgroup

%%%%%%%%%%%%%%%%%%%%%%%%%%%%%%%%%%%%%%%%%%%%%%%%%%%%%%%%%%%%%%%%%%%%%%%%%%%%%%%%
%%%%%%%%%%%%%%%%%%%%%%%%%%%%%%%%%%%%%%%%%%%%%%%%%%%%%%%%%%%%%%%%%%%%%%%%%%%%%%%%
\section{Introduction}

\LaTeX{} provides a mechanism to structure a large document (such as a book)
into a main file and several child files (containing the chapters)
using the |\include| command.
This mechanism is beneficial for documents
which span hundreds of pages in order to
make the source file(s) more manageable.
Moreover, compilation can be restricted to
selected child files by means of the |\includeonly| command.
The latter feature can be used to reduce the compilation time while editing
(this was significantly more useful in the earlier days of \LaTeX{})
or to generate a smaller document which is easier to navigate.
Another application of |\includeonly| is to generate
documents consisting of selected parts of the complete document.

However, there are a few drawbacks of the plain |\include| mechanism:
\begin{itemize}
\item
The child files cannot be compiled on their own,
they can only be compiled via the main file.
A naive editing environment
(such as a text editor with an option
to have the current file processed by \LaTeX)
may require one to switch to the main file before compiling;
attempting to compile the child file produces errors.
\item
The main file must be modified (each time)
to adjust the |\includeonly| command
to the present needs. This easily leaves the main file in a messy state.
\item
The generated document will always carry the filename
of the main document. This is inconvenient if
several child files are to be compiled and
to be kept for distribution.
\end{itemize}

The present package provides a simple interface
to make child files individually compilable by \LaTeX{}.
Compiling a child file then has the same effect as compiling
the main file with an |\includeonly| command
to select the appropriate child.
Moreover the generated document will carry the name of the child
rather than the main file.
This resolves all three above issues.

This feature is meant to make the editing of books,
thesis documents and lecture notes somewhat more convenient.
However, the package can also be used efficiently for
composing a series of documents (such as exercise sheets)
which are typically distributed individually.
It then assists the author in generating the individual documents
(potentially in different versions)
as well as a document containing the collected series.
Another application is in developing style files
or other kinds of included material
where compilation of the style file could redirect
to a sample or test file.

%%%%%%%%%%%%%%%%%%%%%%%%%%%%%%%%%%%%%%%%%%%%%%%%%%%%%%%%%%%%%%%%%%%%%%%%%%%%%%%%
%%%%%%%%%%%%%%%%%%%%%%%%%%%%%%%%%%%%%%%%%%%%%%%%%%%%%%%%%%%%%%%%%%%%%%%%%%%%%%%%
\section{Usage}

First of all, the package \textsf{childdoc} is \emph{not} a standard
\LaTeXe{} |.sty| style file! Therefore it needs to be invoked in
a non-standard way.

%%%%%%%%%%%%%%%%%%%%%%%%%%%%%%%%%%%%%%%%%%%%%%%%%%%%%%%%%%%%%%%%%%%%%%%%%%%%%%%%
\subsection{Included Files}
\label{sec:include}

%%%%%%%%%%%%%%%%%%%%%%%%%%%%%%%%%%%%%%%%
\DescribeMacro{\childdocmain}
To use the package, add the commands
\begin{center}
\begin{tabular}{l}
|\input{childdoc.def}|\\
|\childdocmain{}|\\
\end{tabular}
\end{center}
at the very top of the main \LaTeX{} file,
in particular \emph{before} the |\documentclass| statement!
The argument of |\childdocmain| should be left empty
(but it must be present).

%%%%%%%%%%%%%%%%%%%%%%%%%%%%%%%%%%%%%%%%
\DescribeMacro{\childdocof}
Furthermore, add the commands
\begin{center}
\begin{tabular}{l}
|\input{childdoc.def}|\\
|\childdocof{|\textit{main}|}|\\
\end{tabular}
\end{center}
at the top of every child file \textit{child}
which is included by |\include{|\textit{child}|}|
from within the main file
(or at least for those files to be compiled individually).
The argument \textit{main} must be the filename of the main file.

There are a couple of
considerations in setting up the main and child documents:

%%%%%%%%%%%%%%%%%%%%%%%%%%%%%%%%%%%%%%%%
\paragraph{Restrictions.}

Please note the following restrictions:
\begin{itemize}
\item
|\childdocmain| must be called with one argument \textit{main}
to ensure compatibility with earlier version of the package.
It must either be empty (|\childdocmain{}|)
or precisely match the filename of the main file in which it is specified.
See \secref{sec:detection} for further information.
\item
The filename \textit{main} must be specified without the |.tex| extension.
\item
The filename \textit{main} is case sensitive
(even in case-insensitive file systems)
due to internal string comparison.
\item
The argument \textit{main} should be fully expanded, it cannot be a macro.
\item
Subdirectories and special characters should be avoided in filenames.
\item
The command |\childdocmain{|\textit{main}|}| must be followed by a whitespace.
It should not be followed immediately by another command
or by a comment mark `|%|'.
This is because the \TeX{} parser reads the token immediately following
the argument of |\childdocmain| and puts it
at the beginning of every child section;
however, a white\-space is ignored.
\end{itemize}

%%%%%%%%%%%%%%%%%%%%%%%%%%%%%%%%%%%%%%%%
\paragraph{Content of Main File.}

It is advisable to place all content in the child files included by |\include|.
Any output contained in the main file will appear in all child documents
unless suppressed manually;
it cannot be suppressed automatically by the |\includeonly| directive
and thus should normally be avoided.
A method to include some content in the main file
by means of conditional processing is described in \secref{sec:conditional}.

%%%%%%%%%%%%%%%%%%%%%%%%%%%%%%%%%%%%%%%%
\paragraph{Page Numbering.}

When only a part of the document is compiled,
the appropriate numbering of pages
(as well as other status parameters)
is determined from the |.aux| files.
The latter contain information from previous passes.
However this information needs to propagate through
all intermediate child documents.
Therefore the page numbering in child documents may well
be inconsistent until the complete document is compiled at least once.

A useful (if unconventional) way to always ensure a consistent
page numbering is to restart the numbering in each child document
and denote the pages by `\textit{child}|.|\textit{page}'
where \textit{child} represents the chapter/section number of the child file.
This can be achieved by the command
|\numberwithin{page}{|\textit{child}|}|
of the \textsf{amsmath} package
where \textit{child} can be |chapter| or |section|
depending on the chosen structuring.
Alternatively, one can modify the macro |\thepage| appropriately
and reset the counter |page| at the start of each child file.

%%%%%%%%%%%%%%%%%%%%%%%%%%%%%%%%%%%%%%%%%%%%%%%%%%%%%%%%%%%%%%%%%%%%%%%%%%%%%%%%
\subsection{Conditional Processing}
\label{sec:conditional}

The package provides a mechanism to compile different versions
of a document. To customise the versions further some conditional processing
can come in handy to distinguish which version is being compiled.
The package provides two macros to describe the compilation context:

%%%%%%%%%%%%%%%%%%%%%%%%%%%%%%%%%%%%%%%%
\DescribeMacro{\ifchilddoc}
The conditional |\ifchilddoc| distinguishes between the compilation of
child documents and the main document:
%
\begin{center}
|\ifchilddoc |\textit{child-code}| |[|\||else |\textit{main-code}]| \||fi|
\end{center}

%%%%%%%%%%%%%%%%%%%%%%%%%%%%%%%%%%%%%%%%
\DescribeMacro{\childdocname}
\DescribeMacro{\childdocjob}
The macro |\childdocname| contains the filename (without extension)
of the main or child file being processed.
Note that |\childdocjob| will always contain the name of the main file.

%%%%%%%%%%%%%%%%%%%%%%%%%%%%%%%%%%%%%%%%
\paragraph{Title Page.}

Conditional processing can be used to include a title or banner page
in the main document when proper precautions are taken.
Importantly, the code in the main file should ensure that the page counter
(as well as other status parameters which are stored in the |.aux| files)
takes the same value after the conditional processing.
Otherwise the page numbers may take divergent values
depending on which part is compiled.

For example, a title page could be declared by:
%
\begin{center}
\begin{tabular}{l}
|\ifchilddoc\||else|\\
|\addtocounter{page}{-1}|\\
\textit{code for title page}\\
|\newpage|\\
|\||fi|
\end{tabular}
\end{center}
%
A banner page for the child documents can be generated by:
%
\begin{center}
\begin{tabular}{l}
|\ifchilddoc|\\
|\addtocounter{page}{-1}|\\
\textit{code for banner page}\\
|\newpage|\\
|\||fi|
\end{tabular}
\end{center}
%
Here one could write a message such as:
\begin{center}
|This is the part \childdocname{} of \childdocjob{}.|
\end{center}

%%%%%%%%%%%%%%%%%%%%%%%%%%%%%%%%%%%%%%%%%%%%%%%%%%%%%%%%%%%%%%%%%%%%%%%%%%%%%%%%
\subsection{Flags}
\label{sec:flags}

The package makes it easy to generate different versions
of the main or child documents.
To this end compilation flags can be defined
and assigned different default values.
They will be particularly useful in conjunction
with the forwarding mechanism described in \secref{sec:forward}.

For example, it may be useful to have a flag |\version|
which can be set to |draft| or |final|.
The document source will contain some conditional code
depending on the value of |\version|.
Suppose further, the flag should default to |final| for the main file
and to |draft| for child files
which is a natural assignment for editing the document.
This is achieved by placing the following code
in the preamble of the main document
(below the |\childdocmain| directive):
%
\begin{center}
\begin{tabular}{l}
|\ifchilddoc|\\
|\providecommand{\version}{draft}|\\
|\||else|\\
|\providecommand{\version}{final}|\\
|\||fi|
\end{tabular}
\end{center}
%
The definition by |\providecommand| makes sure
that previous definitions are not overwritten.
Further statements |\providecommand{\version}{...}|
can thus be added before the above code to override it.

For the main file, one might add a line
(between |\childdocmain| and the above block)
%
\begin{center}
|%\ifchilddoc\||else\providecommand{\version}{draft}\||fi|
\end{center}
%
which can be uncommented to produce a draft version.
Likewise one can add a line to the very top of a child file
(above the |\childdocof{|\textit{main}|}| directive)
%
\begin{center}
|%\providecommand{\version}{final}|
\end{center}
%
which can be uncommented to produce the final version of this child document.

%%%%%%%%%%%%%%%%%%%%%%%%%%%%%%%%%%%%%%%%%%%%%%%%%%%%%%%%%%%%%%%%%%%%%%%%%%%%%%%%
\subsection{Forwarding}
\label{sec:forward}

Different versions of the main or child documents
using compilation flags as described in \secref{sec:flags}
can be (permanently) stored in different files
for convenient compilation, viewing and distribution.
To this end, the package defines a command
to pass on compilation to a different file:

%%%%%%%%%%%%%%%%%%%%%%%%%%%%%%%%%%%%%%%%
\DescribeMacro{\childdocforward}
The command |\childdocforward| redirects processing to
another source file:
%
\begin{center}
\begin{tabular}{l}
|\input{childdoc.def}|\\
|\childdocforward[|\textit{main}|]{|\textit{dest}|}|\\
\end{tabular}
\end{center}
%
The argument \textit{dest} is the destination file
(without extension).
It should be the main file or one of the child files.
Note that further \textsf{childdoc} directives
such as |\childdocof| and |\childdocforward|
in the indicated file will be processed in this form.
The optional argument \textit{main}
passes on directly to the main file \textit{main}
while pretending to compile the child \textit{dest}.
This form behaves as if \textit{dest}
issues |\childdocof{|\textit{main}|}| right away,
and no further \textsf{childdoc} directives will be processed.

%%%%%%%%%%%%%%%%%%%%%%%%%%%%%%%%%%%%%%%%
\DescribeMacro{\...prefix}
In the alternative form |\childdocforwardprefix|,
%
\begin{center}
\begin{tabular}{l}
|\input{childdoc.def}|\\
|\childdocforwardprefix[|\textit{main}|]{|\textit{prefix}|}{|\textit{dest}|}|
\end{tabular}
\end{center}
%
the destination file is determined by a pattern
depending on the current file:
To make this work, the current file must be called
`{\textit{prefix}\hspace{0.2em}\textit{suffix}}'
with \textit{prefix} matching precisely the argument.
Processing is then passed on to the file
`{\textit{dest}\hspace{0.2em}\textit{suffix}}'.
Surely, the same effect is achieved by
directly specifying the
argument `{\textit{dest}\hspace{0.2em}\textit{suffix}}'
in the first form.
However, that requires to set up a different file
for each child. With the alternative form of the command
all these files can have exactly the same content
which simplifies setting them up and maintaining them.

For example, the following file |draft.tex|
with a compilation flag |\version| as described in \secref{sec:flags}
compiles the main document as a draft:
%
\begin{center}
\begin{tabular}{l}
|\def\version{draft}|\\
|\input{childdoc.def}|\\
|\childdocforward{|\textit{main}|}|
\end{tabular}
\end{center}
%
Likewise, the following files |final|\textit{nn}|.tex|
compile the final version of the child document
|child|\textit{nn}|.tex|:
%
\begin{center}
\begin{tabular}{l}
|\def\version{final}|\\
|\input{childdoc.def}|\\
|\childdocforwardprefix{final}{child}|
\end{tabular}
\end{center}
%

Note that when several versions of a main file and/or of each child file
are to be generated, it may be convenient to set up a |Makefile| or
shell script to automatise the process.

%%%%%%%%%%%%%%%%%%%%%%%%%%%%%%%%%%%%%%%%%%%%%%%%%%%%%%%%%%%%%%%%%%%%%%%%%%%%%%%%
\subsection{Command Line Processing}
\label{sec:commandline}

The effect of redirection files can also be achieved by invoking
the \LaTeX{} compiler with a more elaborate command line.
Most conveniently this should be done as part
of a shell script or a |Makefile|.

When using \textsf{childdoc} in the main file, the following
command lines effectively perform a redirection
(note that depending on the shell being used,
backslashes may have to be doubled: `|\|' $\to$ `|\\|'):
%
\begin{center}
|... -jobname "|\textit{target}|" |\\|"|[\textit{flags}]%
|\input{childdoc.def}\childdocforward[|\textit{main}|]{|\textit{dest}|}"|
\end{center}
%
Here \textit{target} is the name of the output file,
\textit{main} is the name of the main file
and \textit{dest} is the name of the main or child file to be processed
(all filenames without extensions).
The optional argument \textit{main} can be omitted
if \textit{main} matches \textit{dest}.
Optionally, compilation \textit{flags} can be defined via |\def| commands.
This command line makes the \TeX{} engine believe
it is compiling the file \textit{target}
whose content is specified as the latter parameter.
The provided code then forwards the processing to
\textit{main} or \textit{dest} as described in \secref{sec:forward}.

%%%%%%%%%%%%%%%%%%%%%%%%%%%%%%%%%%%%%%%%%%%%%%%%%%%%%%%%%%%%%%%%%%%%%%%%%%%%%%%%
\subsection{Include by Input}
\label{sec:input}

Including child documents by |\include| has some restrictions by design.
Most notably, the content of a child document always occupies
its own set of pages; pages cannot be shared between child documents.
Usually, this behaviour makes perfect sense
because each child document contain an essential part of the document.
However, in some situations it may be desirable to compose
a document from a collection of parts
without having mandatory page breaks between then.
For this case, the package
provides a mechanism to include parts
by |\input| which can also be processed individually.
However, by construction this mechanism
requires manual handling of the content to be output.

%%%%%%%%%%%%%%%%%%%%%%%%%%%%%%%%%%%%%%%%
\DescribeMacro{\ifchilddocmanual}
The main file should be prepared as usual, see \secref{sec:include}.
However, the document body must make a distinction
between processing of an individual part and of the main document, e.g.:
%
\begin{center}
\begin{tabular}{l}
|\ifchilddocmanual|\\
|\input{\childdocname}|\\
|\||else|\\
\textit{document body with }|\input{|\textit{part}|}|\\
|\||fi|
\end{tabular}
\end{center}
%
The conditional |\ifchilddocmanual| is true whenever
a part to be included by |\input| is being compiled,
and the name of the part is stored in |\childdocname|.

%%%%%%%%%%%%%%%%%%%%%%%%%%%%%%%%%%%%%%%%
\DescribeMacro{\childdocby}
Each part to be included by |\input| should start with:
%
\begin{center}
\begin{tabular}{l}
|\input{childdoc.def}|\\
|\childdocby{|\textit{main}|}|\\
\end{tabular}
\end{center}
%
The directive |\childdocby| is similar to |\childdocof|
described in \secref{sec:include},
but the subsequent selection of content must be done manually.
To that end, both |\ifchilddoc| and |\ifchilddocmanual|
will be true upon processing of a part,
and the name of the part is stored in |\childdocname|.
Note that |\jobname| will be set to the filename of the current part
so that each part receives an individual |.aux| file
that does not interfere with the |.aux| file(s) of the main document.
This behaviour can be altered by the alternative form
|\childdocby[*]{|\textit{main}|}| (with a non-empty optional argument)
which uses the |.aux| file of the main document
by setting |\jobname| to \textit{main}.

%%%%%%%%%%%%%%%%%%%%%%%%%%%%%%%%%%%%%%%%%%%%%%%%%%%%%%%%%%%%%%%%%%%%%%%%%%%%%%%%
\subsection{Driver Development}
\label{sec:driver}

The \textsf{childdoc} mechanism can also be use for the development
of definition files such as \LaTeX{} styles or classes.
This case differs from the above setup with multiple parts
included by |\include| in that no |\includeonly| should be invoked.
This can be achieved by starting the include file
(before |\ProvidesPackage|) with:
%
\begin{center}
\begin{tabular}{l}
|\input{childdoc.def}|\\
|\childdocforward{|\textit{main}|}|\\
\end{tabular}
\end{center}
%
or alternatively with:
%
\begin{center}
\begin{tabular}{l}
|\input{childdoc.def}|\\
|\childdocby{|\textit{main}|}|\\
\end{tabular}
\end{center}
%
Both forms have slightly different effects as described above.
The main file is prepared as usual, see \secref{sec:include}.

%%%%%%%%%%%%%%%%%%%%%%%%%%%%%%%%%%%%%%%%%%%%%%%%%%%%%%%%%%%%%%%%%%%%%%%%%%%%%%%%
\subsection{Legacy Detection}
\label{sec:detection}

The directive |\childdocmain| in the main file can detect
whether the complete document or merely a child is to be compiled
even without using the directive |\childdocof|.
This method is deprecated because it is less robust
and there is no compelling reason to use it;
it is merely provided for backward compatibility
and it may be removed in future versions.

If the detection mechanism is to be used,
it is mandatory to correctly specify
the filename of the main file as the argument of |\childdocmain|:
%
\begin{center}
\begin{tabular}{l}
|\input{childdoc.def}|\\
|\childdocmain{|\textit{main}|}|\\
\end{tabular}
\end{center}
%
If |\jobname| does not match the argument \textit{main} of |\childdocmain|,
it is assumed that |\jobname| points to the child file to be compiled.
When using |\childdocmain| with the main file specified as argument,
it suffices to start a child file
with just |\input{|\textit{main}|}|
without loading of the package and using |\childdocof|.
If instead all processing is done
with the appropriate \textsf{childdoc} directives,
the argument of \textit{main} of |\childdocmain| can be empty.

An alternative version of the command line processing described
in \secref{sec:commandline} using the detection mechanism reads:
%
\begin{center}
|... -jobname "|\textit{target}|" "|[\textit{flags}]%
[|\def\jobname{|\textit{dest}|}|]|\input{|\textit{main}|}"|
\end{center}

%%%%%%%%%%%%%%%%%%%%%%%%%%%%%%%%%%%%%%%%%%%%%%%%%%%%%%%%%%%%%%%%%%%%%%%%%%%%%%%%
\subsection{Manual Code}
\label{sec:manual}

In case one cannot be certain whether the definitions file |childdoc.def|
is installed on the target \TeX{} distribution
and one prefers not to ship it,
it is conceivable to paste a few relevant commands into the sources.

To that end, drop all statements |\input{childdoc.def}|
and perform the replacements as outlined below.
Instead of |\childdocmain{|\textit{main}|}| add the following code
to the top of the main file:
%
\begin{center}
\begin{tabular}{l}
|\||ifdefined\childdocname\endinput\||fi\newif\ifchilddoc|\\
|\edef\childdocname{\scantokens\expandafter{\jobname\noexpand}}|\\
|\def\childdocmain{|\textit{main}|}\||ifx\childdocmain\childdocname\||else|\\
|\childdoctrue\includeonly{\childdocname}\let\jobname\childdocmain\||fi|\\
\end{tabular}
\end{center}
%
Instead of |\childdocof{|\textit{main}|}| just include the main file
at the top of each child file:
%
\begin{center}
|\input{|\textit{main}|}|
\end{center}
%
A simple redirection |\childdocforward{|\textit{dest}|}| is achieved by:
%
\begin{center}
|\def\jobname{|\textit{dest}|}\input{\jobname}|
\end{center}
%
The redirection with prefix
|\childdocforwardprefix[|\textit{prefix}|]{|\textit{dest}|}|
is accomplished by:
%
\begin{center}
\begin{tabular}{l}
|{\edef\jobname{\scantokens\expandafter{\jobname\noexpand}}|\\
|\def\redirectjob |\textit{prefix}|#1~~~{\gdef\jobname{|\textit{dest}|#1}}|\\
|\expandafter\redirectjob\jobname~~~}\input{\jobname}|
\end{tabular}
\end{center}

In an alternative approach,
child documents can be compiled by a specific command line
without additional code or specific definitions:
%
\begin{center}
|... -jobname "|\textit{target}|" "|[\textit{flags}]%
|\includeonly{|\textit{dest}|}\input{|\textit{main}|}"|
\end{center}
%

%%%%%%%%%%%%%%%%%%%%%%%%%%%%%%%%%%%%%%%%%%%%%%%%%%%%%%%%%%%%%%%%%%%%%%%%%%%%%%%%
%%%%%%%%%%%%%%%%%%%%%%%%%%%%%%%%%%%%%%%%%%%%%%%%%%%%%%%%%%%%%%%%%%%%%%%%%%%%%%%%
\section{Information}

%%%%%%%%%%%%%%%%%%%%%%%%%%%%%%%%%%%%%%%%%%%%%%%%%%%%%%%%%%%%%%%%%%%%%%%%%%%%%%%%
\subsection{Copyright}

Copyright \copyright{} 2017--2018 Niklas Beisert

This work may be distributed and/or modified under the
conditions of the \LaTeX{} Project Public License, either version 1.3
of this license or (at your option) any later version.
The latest version of this license is in
  \url{http://www.latex-project.org/lppl.txt}
and version 1.3 or later is part of all distributions of \LaTeX{}
version 2005/12/01 or later.

This work has the LPPL maintenance status `maintained'.

The Current Maintainer of this work is Niklas Beisert.

This work consists of the files |README.txt|, |childdoc.ins| and |childdoc.dtx|
as well as the derived files |childdoc.def|, |cdocsamp.tex|
with |cdocsch1.tex|, |cdocsch2.tex|, |cdocspt3.tex|, |cdocspt4.tex|,
|cdocsdrf.tex|, |cdocsfn1.tex|, |cdocsfn2.tex|
as well as |childdoc.pdf|.

%%%%%%%%%%%%%%%%%%%%%%%%%%%%%%%%%%%%%%%%%%%%%%%%%%%%%%%%%%%%%%%%%%%%%%%%%%%%%%%%
\subsection{Files and Installation}

The package consists of the files:
%
\begin{center}
\begin{tabular}{ll}
    |README.txt|   & readme file \\
    |childdoc.ins| & installation file \\
    |childdoc.dtx| & source file \\
    |childdoc.def| & definition file \\
    |cdocsamp.tex| & sample main file \\
    |cdocsch1.tex| & sample include file \\
    |cdocsch2.tex| & sample include file \\
    |cdocspt3.tex| & sample part file \\
    |cdocspt4.tex| & sample part file \\
    |cdocsdrf.tex| & sample redirection file \\
    |cdocsfn1.tex| & sample redirection file \\
    |cdocsfn2.tex| & sample redirection file \\
    |childdoc.pdf| & manual
\end{tabular}
\end{center}
%
The distribution consists of the files
|README.txt|, |childdoc.ins| and |childdoc.dtx|.
%
\begin{itemize}
\item
Run (pdf)\LaTeX{} on |childdoc.dtx|
to compile the manual |childdoc.pdf| (this file).
\item
Run \LaTeX{} on |childdoc.ins| to create the definitions file |childdoc.def|
and the sample |cdocsamp.tex| with include files
|cdocsch1.tex|, |cdocsch2.tex|, |cdocspt3.tex|, |cdocspt4.tex|,
|cdocsdrf.tex|, |cdocsfn1.tex|, |cdocsfn2.tex|.
Then copy the file |childdoc.def| to an appropriate directory of your \LaTeX{}
distribution, e.g.\ \textit{texmf-root}|/tex/latex/childdoc|.
\end{itemize}

%%%%%%%%%%%%%%%%%%%%%%%%%%%%%%%%%%%%%%%%%%%%%%%%%%%%%%%%%%%%%%%%%%%%%%%%%%%%%%%%
\subsection{Related CTAN Packages}

There are several other packages which offer a similar functionality:
%
\begin{itemize}
\item
The packages
\href{http://ctan.org/pkg/docmute}{\textsf{docmute}},
\href{http://ctan.org/pkg/includex}{\textsf{includex}} and
\href{http://ctan.org/pkg/standalone}{\textsf{standalone}}
provide commands to include only the document body of
a child file thus allowing both files to be compiled individually.
\item
The packages \href{http://ctan.org/pkg/subdocs}{\textsf{subdocs}}
and \href{http://ctan.org/pkg/subfiles}{\textsf{subfiles}}
provide structures in which the main and child documents can be
encapsulated and allowing them to be compiled individually.
The inclusion mechanism is different from the conventional |\include|.
\item
The package \href{http://ctan.org/pkg/combine}{\textsf{combine}}
is an elaborate solution to combine several documents into one.
\end{itemize}
%
See also the CTAN topic \href{http://ctan.org/topic/subdocs}{\textsf{subdocs}}
for further related packages.
The present package differs from the above solutions in that
a document structure constructed with the conventional |\include| mechanism
just needs two extra commands at the top of every file
such that all constituent files can be compiled individually.

%%%%%%%%%%%%%%%%%%%%%%%%%%%%%%%%%%%%%%%%%%%%%%%%%%%%%%%%%%%%%%%%%%%%%%%%%%%%%%%%
%\subsection{Feature Suggestions}
%
%The following is a list of features which may be useful for future
%versions of this package:
%%
%\begin{itemize}
%\item
%\ldots
%\end{itemize}

%%%%%%%%%%%%%%%%%%%%%%%%%%%%%%%%%%%%%%%%%%%%%%%%%%%%%%%%%%%%%%%%%%%%%%%%%%%%%%%%
\subsection{Revision History}

%%%%%%%%%%%%%%%%%%%%%%%%%%%%%%%%%%%%%%%%
\paragraph{v2.0:} 2018/12/30

\begin{itemize}
\item
immediate forward processing
\item
added |\childdocby| mechanism
\item
manual restructured
\end{itemize}

%%%%%%%%%%%%%%%%%%%%%%%%%%%%%%%%%%%%%%%%
\paragraph{v1.6:} 2018/01/17

\begin{itemize}
\item
application for development of include files
\item
corrections to manual
\end{itemize}

%%%%%%%%%%%%%%%%%%%%%%%%%%%%%%%%%%%%%%%%
\paragraph{v1.5:} 2017/05/21

\begin{itemize}
\item
more complete structuring introduced
\item
|\childdocof| introduced
\item
|\childdoc| renamed to |\childdocmain|
\item
|\childredirect| renamed to |\childdocforward| and |\childdocforwardprefix|
and functionality expanded
\end{itemize}

%%%%%%%%%%%%%%%%%%%%%%%%%%%%%%%%%%%%%%%%
\paragraph{v1.0:} 2017/04/27

\begin{itemize}
\item
manual and install package
\item
first version published on CTAN
\end{itemize}

%%%%%%%%%%%%%%%%%%%%%%%%%%%%%%%%%%%%%%%%
\paragraph{v0.6:} 2017/04/26

\begin{itemize}
\item
redirection mechanism added
\end{itemize}

%%%%%%%%%%%%%%%%%%%%%%%%%%%%%%%%%%%%%%%%
\paragraph{v0.5:} 2017/04/26

\begin{itemize}
\item
functionality in definition file
\end{itemize}


%%%%%%%%%%%%%%%%%%%%%%%%%%%%%%%%%%%%%%%%%%%%%%%%%%%%%%%%%%%%%%%%%%%%%%%%%%%%%%%%
%%%%%%%%%%%%%%%%%%%%%%%%%%%%%%%%%%%%%%%%%%%%%%%%%%%%%%%%%%%%%%%%%%%%%%%%%%%%%%%%
%%%%%%%%%%%%%%%%%%%%%%%%%%%%%%%%%%%%%%%%%%%%%%%%%%%%%%%%%%%%%%%%%%%%%%%%%%%%%%%%
\appendix

\settowidth\MacroIndent{\rmfamily\scriptsize 000\ }

 \DocInput{childdoc.dtx}

\end{document}
%</driver>
% \fi
%
% %%%%%%%%%%%%%%%%%%%%%%%%%%%%%%%%%%%%%%%%%%%%%%%%%%%%%%%%%%%%%%%%%%%%%%%%%%%%%%
% %%%%%%%%%%%%%%%%%%%%%%%%%%%%%%%%%%%%%%%%%%%%%%%%%%%%%%%%%%%%%%%%%%%%%%%%%%%%%%
% \section{Sample}
%\iffalse
%<*samplemain>
%\fi
%
% The following presents a sample document
% with two chapters, two parts, a title page,
% a compile flag as well as three forwarding files to set the flag.
% It consists of eight |.tex| files:
% \begin{center}
% \begin{tabular}{ll}
% |cdocsamp.tex|&main file\\
% |cdocsch1.tex|&include file for chapter 1\\
% |cdocsch2.tex|&include file for chapter 2\\
% |cdocspt3.tex|&include file for part 3\\
% |cdocspt4.tex|&include file for part 4\\
% |cdocsdrf.tex|&forwarding file for main file in draft mode\\
% |cdocsfi1.tex|&forwarding file for final version of chapter 1\\
% |cdocsfi2.tex|&forwarding file for final version of chapter 2\\
% \end{tabular}
% \end{center}
% Each of the eight files can be compiled directly by the \LaTeX{} compiler.
%
% %%%%%%%%%%%%%%%%%%%%%%%%%%%%%%%%%%%%%%
% \paragraph{Main File.}
%
% The main file is called |cdocsamp.tex|.
%
% Load the \textsf{childdoc} definitions and
% declare the filename for the main document:
%    \begin{macrocode}
\input{childdoc.def}
\childdocmain{}
%    \end{macrocode}

% Optional override for |\version| flag:
%    \begin{macrocode}
%%\ifchilddoc\else\providecommand{\version}{draft}\fi
%    \end{macrocode}

% Define the default values for the |\version| flag
% (|final| for the main file and |draft| for childs):
%    \begin{macrocode}
\ifchilddoc
\providecommand{\version}{draft}
\else
\providecommand{\version}{final}
\fi
%    \end{macrocode}

% Load the standard document class:
%    \begin{macrocode}
\documentclass[12pt]{article}
%    \end{macrocode}

% Start the document body:
%    \begin{macrocode}
\begin{document}
%    \end{macrocode}

% Declare a title page.
% Print title, part of document being processed and version flag:
%    \begin{macrocode}
\addtocounter{page}{-1}
\begin{center}
{\LARGE\bfseries{}childdoc example\par}
\vspace{1cm}
\ifchilddoc
\ifchilddocmanual part\else chapter\fi:
`\childdocname' of `\childdocjob'\par
\else
main document: `\childdocjob'\par
\fi
version: \version\par
\end{center}
\newpage
%    \end{macrocode}

% Manually include selected file,
% otherwise process as usual:
%    \begin{macrocode}
\ifchilddocmanual
\section*{part `\childdocname'}
\input{\childdocname}
\else
%    \end{macrocode}

% Include the two chapters:
%    \begin{macrocode}
\include{cdocsch1}
\include{cdocsch2}
%    \end{macrocode}

% Include the two parts unless only chapters should be displayed:
%    \begin{macrocode}
\ifchilddoc\else
\section{part three}
\input{cdocspt3}
\section{part four}
\input{cdocspt4}
\fi
%    \end{macrocode}

% Process as usual until here:
%    \begin{macrocode}
\fi
%    \end{macrocode}

% End of document body:
%    \begin{macrocode}
\end{document}
%    \end{macrocode}
%\iffalse
%</samplemain>
%\fi
%
% %%%%%%%%%%%%%%%%%%%%%%%%%%%%%%%%%%%%%%
% \paragraph{Chapter Include Files.}
%
% The include files are called |cdocsch1.tex| and |cdocsch2.tex|.
%
%\iffalse
%<*samplechap1|samplechap2>
%\fi

% Optional override for |\version| flag:
%    \begin{macrocode}
%%\providecommand{\version}{final}
%    \end{macrocode}

% Include the main document:
%    \begin{macrocode}
\input{childdoc.def}
\childdocof{cdocsamp}
%    \end{macrocode}

%\iffalse
%</samplechap1|samplechap2>
%\fi
%
%\iffalse
%<*samplechap1>
%\fi
% Some text for chapter 1:
%    \begin{macrocode}
\section{one}
some text in chapter one
%    \end{macrocode}

%\iffalse
%</samplechap1>
%\fi
% Some text for chapter 2:
%\iffalse
%<*samplechap2>
%\fi
%    \begin{macrocode}
\section{two}
more text in chapter two
%    \end{macrocode}

%\iffalse
%</samplechap2>
%\fi
%
% %%%%%%%%%%%%%%%%%%%%%%%%%%%%%%%%%%%%%%
% \paragraph{Part Include Files.}
%
% The include files are called |cdocspt3.tex| and |cdocspt4.tex|.
%
%\iffalse
%<*samplepart3|samplepart4>
%\fi

% Optional override for |\version| flag:
%    \begin{macrocode}
%%\providecommand{\version}{final}
%    \end{macrocode}

% Include the main document:
%    \begin{macrocode}
\input{childdoc.def}
\childdocby{cdocsamp}
%    \end{macrocode}

%\iffalse
%</samplepart3|samplepart4>
%\fi
%
%\iffalse
%<*samplepart3>
%\fi
% Some text for part 3:
%    \begin{macrocode}
some text in part three
%    \end{macrocode}

%\iffalse
%</samplepart3>
%\fi
% Some text for part 4:
%\iffalse
%<*samplepart4>
%\fi
%    \begin{macrocode}
more text in part four
%    \end{macrocode}

%\iffalse
%</samplepart4>
%\fi
%
% %%%%%%%%%%%%%%%%%%%%%%%%%%%%%%%%%%%%%%
% \paragraph{Forwarding for a Complete Draft.}
%
% The following forwarding file |cdocsdrf.tex|
% compiles the main document in draft mode:
%\iffalse
%<*sampledraft>
%\fi
%    \begin{macrocode}
\def\version{draft}
\input{childdoc.def}
\childdocforward{cdocsamp}
%    \end{macrocode}

%\iffalse
%</sampledraft>
%\fi
%
% %%%%%%%%%%%%%%%%%%%%%%%%%%%%%%%%%%%%%%
% \paragraph{Forwarding for Final Version of the Chapters.}
%
% The following forwarding files |cdocsfn1.tex| and |cdocsfn2.tex|
% (with identical content)
% compile the final versions of the child documents
% |cdocsch1.tex| and |cdocsch2.tex|, respectively:
%\iffalse
%<*samplefinal>
%\fi
%    \begin{macrocode}
\def\version{final}
\input{childdoc.def}
\childdocforwardprefix[cdocsamp]{cdocsfn}{cdocsch}
%    \end{macrocode}

%\iffalse
%</samplefinal>
%\fi
%
% %%%%%%%%%%%%%%%%%%%%%%%%%%%%%%%%%%%%%%
% \paragraph{Command Line Processing.}
%
% The following three command lines generate the output files
% |cdocscld|, |cdocscl1| and |cdocscl2|
% which should be identical to
% |cdocsdrf|, |cdocsch1| and |cdocsfn2|, respectively:
% \begin{center}
% \begin{tabular}{l}
% |latex -jobname cdocscld \|\\
% |  "\def\version{draft}\input{childdoc.def}\childdocforward{cdocsamp}"|\\
% |latex -jobname cdocscl1 \|\\
% |  "\input{childdoc.def}\childdocforward[cdocsamp]{cdocsch1}"|\\
% |latex -jobname cdocscl2 \|\\
% |  "\def\version{final}\input{childdoc.def}\childdocforward{cdocsch2}"|
% \end{tabular}
% \end{center}
% Note that the trailing backslash on each first line
% merely continues the input to the second line
% (for convenient cut ant paste).
% Furthermore, the command |latex| can be replaced by any
% of its alternative versions such as |pdflatex|.
%
% %%%%%%%%%%%%%%%%%%%%%%%%%%%%%%%%%%%%%%%%%%%%%%%%%%%%%%%%%%%%%%%%%%%%%%%%%%%%%%
% %%%%%%%%%%%%%%%%%%%%%%%%%%%%%%%%%%%%%%%%%%%%%%%%%%%%%%%%%%%%%%%%%%%%%%%%%%%%%%
% \section{Implementation}
%\iffalse
%<*package>
%\fi
%
% This section describes the definitions file |childdoc.def|.

% The definitions cannot be loaded using |\usepackage| or |\RequirePackage|
% which has a mechanism to prevent loading a style file more than once.
% When loading the definitions by means of |\input|
% multiple instances have to be prevented manually:
%\iffalse
%This code needs to be before the `\ProvidesFile' directive
%which is defined at the beginning of this file.
%Therefore it is also placed there and commented out here.
%</package>
%<*discard>
%\fi
%    \begin{macrocode}
\ifdefined\childdocmain\endinput\fi
%    \end{macrocode}
%\iffalse
%</discard>
%<*package>
%\fi
%
% \macro{\ifchilddoc}
% \macro{\ifchilddocmanual}
% The conditional |\ifchilddoc| tells whether a
% child (true) or main (false) document is being compiled.
% The conditional |\ifchilddocmanual| tells whether
% the |\includeonly| mechanism is used (false) or
% the selection of child files must be performed manually (true).
% The definitions initialise to false:
%    \begin{macrocode}
\newif\ifchilddoc
\newif\ifchilddocmanual
%    \end{macrocode}

% \macro{\childdocname}
% \macro{\childdocjob}
% The macro |\childdocname| stores the name of the main document
% to be compiled. The macro |\childdocjob| stores the name of
% the document on which the \LaTeX{} compiler was originally invoked.
% The content of |\jobname| cannot be compared
% to filenames specified in the source due to different catcodes.
% The following code rescans |\jobname|, stores the result
% in |\childdocname| and saves a copy in |\childdocjob|:
%    \begin{macrocode}
\edef\childdocname{\scantokens\expandafter{\jobname\noexpand}}
\let\childdocjob\childdocname
%    \end{macrocode}

% \macro{\childdocdisable}
% The macro |\childdocdisable| prevents the main file
% from being processed more than once.
% At this stage, the main document command |\childdocmain|
% is assumed to be called once again where it should do nothing.
% Any subsequent call to it should prevent
% a secondary processing of the main document
% It overwrites the forwarding commands
% |\childdocof| and |\childdocforward|
% with empty macros to prevent further inclusions of the main document:
%    \begin{macrocode}
\newcommand{\childdocdisable}
{
  \renewcommand{\childdocmain}[1]{\renewcommand{\childdocmain}[1]{\endinput}}
  \renewcommand{\childdocof}[1]{}
  \renewcommand{\childdocby}[2][]{}
  \renewcommand{\childdocforward}[2][]{}
  \renewcommand{\childdocdisable}{}
}
%    \end{macrocode}

% \macro{\childdocmain}
% The macro |\childdocmain| is to be called at the top of the main file
% with nothing or the main filename (without extension) as argument.
% First, it breaks loops.
% If the argument is not empty and does not match |\childdocname|
% (which is set by the first inclusion of |childdoc.def|),
% |\ifchilddoc| is set to true, |\includeonly| is applied to the child file
% and |\jobname| is set to the main file
% (for proper handling of |.aux| files):
%    \begin{macrocode}
\newcommand{\childdocmain}[1]
{
  \childdocdisable\childdocmain{}
  \if?#1?\else
    \begingroup
      \def\childdoctmp{#1}
      \ifx\childdoctmp\childdocname
        \def\childdoctmp{}
      \else
        \def\childdoctmp
        {
          \childdoctrue
          \includeonly{\childdocname}
          \def\childdocjob{#1}
          \def\jobname{#1}
        }
      \fi
      \expandafter
    \endgroup
    \childdoctmp
  \fi
}
%    \end{macrocode}

% \macro{\childdocof}
% The command |\childdocof| redirects
% compilation to the main file |#1|.
%    \begin{macrocode}
\newcommand{\childdocof}[1]
{
  \childdocdisable
  \childdoctrue
  \includeonly{\childdocname}
  \def\jobname{#1}
  \def\childdocjob{#1}
  \input{#1}
}
%    \end{macrocode}

% \macro{\childdocby}
% The command |\childdocby| ....
%    \begin{macrocode}
\newcommand{\childdocby}[2][]
{
  \childdocdisable
  \childdoctrue
  \childdocmanualtrue
  \if?#1?\else
    \def\jobname{#2}
  \fi
  \def\childdocjob{#2}
  \input{#2}
  \endinput
}
%    \end{macrocode}

% \macro{\childdocforward}
% The command |\childdocforward| redirects
% compilation to the main file or
% (if the optional argument is given) a child file.
% Parameters are set as if the main file
% or a child file starting with |\childdocof| was compiled.
% Then compilation is handed over to the main file:
%    \begin{macrocode}
\newcommand{\childdocforward}[2][]
{
  \begingroup
    \if?#1?
      \def\childdoctmp
      {
        \def\childdocname{#2}
        \def\childdocjob{#2}
        \def\jobname{#2}
        \input{#2}
        \endinput
      }
    \else
      \def\childdoctmp
      {
        \childdocdisable
        \def\childdocname{#2}
        \childdoctrue
        \includeonly{#2}
        \def\childdocjob{#1}
        \def\jobname{#1}
        \input{#1}
        \endinput
      }
    \fi
    \expandafter
  \endgroup
  \childdoctmp
}
%    \end{macrocode}

% \macro{\childdocforwardprefix}
% The command |\childdocforwardprefix| redirects
% compilation to the main or a child file by means of a pattern.
% The prefix |#1| in the current filename is replaced by |#2|
% and the suffix of the current filename is kept
% (it is assumed that the filename does not contain the substring `|~~~|'
% which is used as a delimiter).
% Compilation is handed over to the new file by |\childdocforward|:
%    \begin{macrocode}
\newcommand{\childdocforwardprefix}[3][]
{
  \begingroup
    \def\childdocextract #2##1~~~{\def\childdoctmp{\childdocforward[#1]{#3##1}}}
    \expandafter\childdocextract\childdocname~~~
    \expandafter
  \endgroup
  \childdoctmp
}
%    \end{macrocode}

% \macro{\childdoc}
% The deprecated macro |\childdoc| is a legacy version of |\childdocmain|:
%    \begin{macrocode}
\newcommand{\childdoc}{\childdocmain}
%    \end{macrocode}

% \macro{\childdocredirect}
% The deprecated macro |\childdocredirect| is a legacy version
% of |\childdocforward| and |\childdocforwardprefix|:
%    \begin{macrocode}
\newcommand{\childdocredirect}[2][]
{
  \begingroup
    \if?#1?
      \def\childdoctmp{\childdocforward{#2}}
    \else
      \def\childdoctmp{\childdocforwardprefix{#1}{#2}}
    \fi
    \expandafter
  \endgroup
  \childdoctmp
}
%    \end{macrocode}

%\iffalse
%</package>
%\fi
%
\endinput
\childdocforward{cdocsch2}"|
% \end{tabular}
% \end{center}
% Note that the trailing backslash on each first line
% merely continues the input to the second line
% (for convenient cut ant paste).
% Furthermore, the command |latex| can be replaced by any
% of its alternative versions such as |pdflatex|.
%
% %%%%%%%%%%%%%%%%%%%%%%%%%%%%%%%%%%%%%%%%%%%%%%%%%%%%%%%%%%%%%%%%%%%%%%%%%%%%%%
% %%%%%%%%%%%%%%%%%%%%%%%%%%%%%%%%%%%%%%%%%%%%%%%%%%%%%%%%%%%%%%%%%%%%%%%%%%%%%%
% \section{Implementation}
%\iffalse
%<*package>
%\fi
%
% This section describes the definitions file |childdoc.def|.

% The definitions cannot be loaded using |\usepackage| or |\RequirePackage|
% which has a mechanism to prevent loading a style file more than once.
% When loading the definitions by means of |\input|
% multiple instances have to be prevented manually:
%\iffalse
%This code needs to be before the `\ProvidesFile' directive
%which is defined at the beginning of this file.
%Therefore it is also placed there and commented out here.
%</package>
%<*discard>
%\fi
%    \begin{macrocode}
\ifdefined\childdocmain\endinput\fi
%    \end{macrocode}
%\iffalse
%</discard>
%<*package>
%\fi
%
% \macro{\ifchilddoc}
% \macro{\ifchilddocmanual}
% The conditional |\ifchilddoc| tells whether a
% child (true) or main (false) document is being compiled.
% The conditional |\ifchilddocmanual| tells whether
% the |\includeonly| mechanism is used (false) or
% the selection of child files must be performed manually (true).
% The definitions initialise to false:
%    \begin{macrocode}
\newif\ifchilddoc
\newif\ifchilddocmanual
%    \end{macrocode}

% \macro{\childdocname}
% \macro{\childdocjob}
% The macro |\childdocname| stores the name of the main document
% to be compiled. The macro |\childdocjob| stores the name of
% the document on which the \LaTeX{} compiler was originally invoked.
% The content of |\jobname| cannot be compared
% to filenames specified in the source due to different catcodes.
% The following code rescans |\jobname|, stores the result
% in |\childdocname| and saves a copy in |\childdocjob|:
%    \begin{macrocode}
\edef\childdocname{\scantokens\expandafter{\jobname\noexpand}}
\let\childdocjob\childdocname
%    \end{macrocode}

% \macro{\childdocdisable}
% The macro |\childdocdisable| prevents the main file
% from being processed more than once.
% At this stage, the main document command |\childdocmain|
% is assumed to be called once again where it should do nothing.
% Any subsequent call to it should prevent
% a secondary processing of the main document
% It overwrites the forwarding commands
% |\childdocof| and |\childdocforward|
% with empty macros to prevent further inclusions of the main document:
%    \begin{macrocode}
\newcommand{\childdocdisable}
{
  \renewcommand{\childdocmain}[1]{\renewcommand{\childdocmain}[1]{\endinput}}
  \renewcommand{\childdocof}[1]{}
  \renewcommand{\childdocby}[2][]{}
  \renewcommand{\childdocforward}[2][]{}
  \renewcommand{\childdocdisable}{}
}
%    \end{macrocode}

% \macro{\childdocmain}
% The macro |\childdocmain| is to be called at the top of the main file
% with nothing or the main filename (without extension) as argument.
% First, it breaks loops.
% If the argument is not empty and does not match |\childdocname|
% (which is set by the first inclusion of |childdoc.def|),
% |\ifchilddoc| is set to true, |\includeonly| is applied to the child file
% and |\jobname| is set to the main file
% (for proper handling of |.aux| files):
%    \begin{macrocode}
\newcommand{\childdocmain}[1]
{
  \childdocdisable\childdocmain{}
  \if?#1?\else
    \begingroup
      \def\childdoctmp{#1}
      \ifx\childdoctmp\childdocname
        \def\childdoctmp{}
      \else
        \def\childdoctmp
        {
          \childdoctrue
          \includeonly{\childdocname}
          \def\childdocjob{#1}
          \def\jobname{#1}
        }
      \fi
      \expandafter
    \endgroup
    \childdoctmp
  \fi
}
%    \end{macrocode}

% \macro{\childdocof}
% The command |\childdocof| redirects
% compilation to the main file |#1|.
%    \begin{macrocode}
\newcommand{\childdocof}[1]
{
  \childdocdisable
  \childdoctrue
  \includeonly{\childdocname}
  \def\jobname{#1}
  \def\childdocjob{#1}
  \input{#1}
}
%    \end{macrocode}

% \macro{\childdocby}
% The command |\childdocby| ....
%    \begin{macrocode}
\newcommand{\childdocby}[2][]
{
  \childdocdisable
  \childdoctrue
  \childdocmanualtrue
  \if?#1?\else
    \def\jobname{#2}
  \fi
  \def\childdocjob{#2}
  \input{#2}
  \endinput
}
%    \end{macrocode}

% \macro{\childdocforward}
% The command |\childdocforward| redirects
% compilation to the main file or
% (if the optional argument is given) a child file.
% Parameters are set as if the main file
% or a child file starting with |\childdocof| was compiled.
% Then compilation is handed over to the main file:
%    \begin{macrocode}
\newcommand{\childdocforward}[2][]
{
  \begingroup
    \if?#1?
      \def\childdoctmp
      {
        \def\childdocname{#2}
        \def\childdocjob{#2}
        \def\jobname{#2}
        \input{#2}
        \endinput
      }
    \else
      \def\childdoctmp
      {
        \childdocdisable
        \def\childdocname{#2}
        \childdoctrue
        \includeonly{#2}
        \def\childdocjob{#1}
        \def\jobname{#1}
        \input{#1}
        \endinput
      }
    \fi
    \expandafter
  \endgroup
  \childdoctmp
}
%    \end{macrocode}

% \macro{\childdocforwardprefix}
% The command |\childdocforwardprefix| redirects
% compilation to the main or a child file by means of a pattern.
% The prefix |#1| in the current filename is replaced by |#2|
% and the suffix of the current filename is kept
% (it is assumed that the filename does not contain the substring `|~~~|'
% which is used as a delimiter).
% Compilation is handed over to the new file by |\childdocforward|:
%    \begin{macrocode}
\newcommand{\childdocforwardprefix}[3][]
{
  \begingroup
    \def\childdocextract #2##1~~~{\def\childdoctmp{\childdocforward[#1]{#3##1}}}
    \expandafter\childdocextract\childdocname~~~
    \expandafter
  \endgroup
  \childdoctmp
}
%    \end{macrocode}

% \macro{\childdoc}
% The deprecated macro |\childdoc| is a legacy version of |\childdocmain|:
%    \begin{macrocode}
\newcommand{\childdoc}{\childdocmain}
%    \end{macrocode}

% \macro{\childdocredirect}
% The deprecated macro |\childdocredirect| is a legacy version
% of |\childdocforward| and |\childdocforwardprefix|:
%    \begin{macrocode}
\newcommand{\childdocredirect}[2][]
{
  \begingroup
    \if?#1?
      \def\childdoctmp{\childdocforward{#2}}
    \else
      \def\childdoctmp{\childdocforwardprefix{#1}{#2}}
    \fi
    \expandafter
  \endgroup
  \childdoctmp
}
%    \end{macrocode}

%\iffalse
%</package>
%\fi
%
\endinput
|\\
|\childdocmain{}|\\
\end{tabular}
\end{center}
at the very top of the main \LaTeX{} file,
in particular \emph{before} the |\documentclass| statement!
The argument of |\childdocmain| should be left empty
(but it must be present).

%%%%%%%%%%%%%%%%%%%%%%%%%%%%%%%%%%%%%%%%
\DescribeMacro{\childdocof}
Furthermore, add the commands
\begin{center}
\begin{tabular}{l}
|% \iffalse
%
% childdoc.dtx Copyright (C) 2017-2018 Niklas Beisert
%
% This work may be distributed and/or modified under the
% conditions of the LaTeX Project Public License, either version 1.3
% of this license or (at your option) any later version.
% The latest version of this license is in
%   http://www.latex-project.org/lppl.txt
% and version 1.3 or later is part of all distributions of LaTeX
% version 2005/12/01 or later.
%
% This work has the LPPL maintenance status `maintained'.
%
% The Current Maintainer of this work is Niklas Beisert.
%
% This work consists of the files childdoc.dtx and childdoc.ins
% and the derived files childdoc.def and cdocsamp.tex with
% cdocsch1.tex, cdocsch2.tex, cdocsdrf.tex, cdocsfn1.tex, cdocsfn2.tex.
%
%<package>\ifdefined\childdocmain\endinput\fi
%<package>\ProvidesFile{childdoc.def}[2018/12/30 v2.0 child document driver]
%<samplemain>\ProvidesFile{cdocsamp.tex}[2018/12/30 v2.0 sample for childdoc]
%<*driver>
%\ProvidesFile{childdoc.drv}[2018/12/30 v2.0 childdoc reference manual file]
\PassOptionsToClass{10pt,a4paper}{article}
\documentclass{ltxdoc}

\usepackage[margin=35mm]{geometry}
\usepackage{hyperref}
\usepackage{hyperxmp}
\usepackage[usenames]{color}

\hypersetup{colorlinks=true}
\hypersetup{pdfstartview=FitH}
\hypersetup{pdfpagemode=UseNone}
\hypersetup{pdfsource={}}
\hypersetup{pdflang={en-UK}}
\hypersetup{pdfcopyright={Copyright 2017-2018 Niklas Beisert.
  This work may be distributed and/or modified under the
  conditions of the LaTeX Project Public License, either version 1.3
  of this license or (at your option) any later version.}}
\hypersetup{pdflicenseurl={http://www.latex-project.org/lppl.txt}}
\hypersetup{pdfcontactaddress={ETH Zurich, ITP, HIT K,
  Wolfgang-Pauli-Strasse 27}}
\hypersetup{pdfcontactpostcode={8093}}
\hypersetup{pdfcontactcity={Zurich}}
\hypersetup{pdfcontactcountry={Switzerland}}
\hypersetup{pdfcontactemail={nbeisert@itp.phys.ethz.ch}}
\hypersetup{pdfcontacturl={http://people.phys.ethz.ch/\xmptilde nbeisert/}}

\newcommand{\secref}[1]{\hyperref[#1]{section \ref*{#1}}}

\parskip1ex
\parindent0pt
\let\olditemize\itemize
\def\itemize{\olditemize\parskip0pt}

\begin{document}

\title{The \textsf{childdoc} Package}
\hypersetup{pdftitle={The childdoc Package}}
\author{Niklas Beisert\\[2ex]
  Institut f\"ur Theoretische Physik\\
  Eidgen\"ossische Technische Hochschule Z\"urich\\
  Wolfgang-Pauli-Strasse 27, 8093 Z\"urich, Switzerland\\[1ex]
  \href{mailto:nbeisert@itp.phys.ethz.ch}
  {\texttt{nbeisert@itp.phys.ethz.ch}}}
\hypersetup{pdfauthor={Niklas Beisert}}
\hypersetup{pdfsubject={Manual for the LaTeX2e Package childdoc}}
\date{30 December 2018, \textsf{v2.0}}
\maketitle

\begin{abstract}\noindent
\textsf{childdoc} is a \LaTeXe{} package
that enables the direct compilation
of document sections included by |\include|
to individual files.
\end{abstract}

\begingroup
\parskip0ex
\tableofcontents
\endgroup

%%%%%%%%%%%%%%%%%%%%%%%%%%%%%%%%%%%%%%%%%%%%%%%%%%%%%%%%%%%%%%%%%%%%%%%%%%%%%%%%
%%%%%%%%%%%%%%%%%%%%%%%%%%%%%%%%%%%%%%%%%%%%%%%%%%%%%%%%%%%%%%%%%%%%%%%%%%%%%%%%
\section{Introduction}

\LaTeX{} provides a mechanism to structure a large document (such as a book)
into a main file and several child files (containing the chapters)
using the |\include| command.
This mechanism is beneficial for documents
which span hundreds of pages in order to
make the source file(s) more manageable.
Moreover, compilation can be restricted to
selected child files by means of the |\includeonly| command.
The latter feature can be used to reduce the compilation time while editing
(this was significantly more useful in the earlier days of \LaTeX{})
or to generate a smaller document which is easier to navigate.
Another application of |\includeonly| is to generate
documents consisting of selected parts of the complete document.

However, there are a few drawbacks of the plain |\include| mechanism:
\begin{itemize}
\item
The child files cannot be compiled on their own,
they can only be compiled via the main file.
A naive editing environment
(such as a text editor with an option
to have the current file processed by \LaTeX)
may require one to switch to the main file before compiling;
attempting to compile the child file produces errors.
\item
The main file must be modified (each time)
to adjust the |\includeonly| command
to the present needs. This easily leaves the main file in a messy state.
\item
The generated document will always carry the filename
of the main document. This is inconvenient if
several child files are to be compiled and
to be kept for distribution.
\end{itemize}

The present package provides a simple interface
to make child files individually compilable by \LaTeX{}.
Compiling a child file then has the same effect as compiling
the main file with an |\includeonly| command
to select the appropriate child.
Moreover the generated document will carry the name of the child
rather than the main file.
This resolves all three above issues.

This feature is meant to make the editing of books,
thesis documents and lecture notes somewhat more convenient.
However, the package can also be used efficiently for
composing a series of documents (such as exercise sheets)
which are typically distributed individually.
It then assists the author in generating the individual documents
(potentially in different versions)
as well as a document containing the collected series.
Another application is in developing style files
or other kinds of included material
where compilation of the style file could redirect
to a sample or test file.

%%%%%%%%%%%%%%%%%%%%%%%%%%%%%%%%%%%%%%%%%%%%%%%%%%%%%%%%%%%%%%%%%%%%%%%%%%%%%%%%
%%%%%%%%%%%%%%%%%%%%%%%%%%%%%%%%%%%%%%%%%%%%%%%%%%%%%%%%%%%%%%%%%%%%%%%%%%%%%%%%
\section{Usage}

First of all, the package \textsf{childdoc} is \emph{not} a standard
\LaTeXe{} |.sty| style file! Therefore it needs to be invoked in
a non-standard way.

%%%%%%%%%%%%%%%%%%%%%%%%%%%%%%%%%%%%%%%%%%%%%%%%%%%%%%%%%%%%%%%%%%%%%%%%%%%%%%%%
\subsection{Included Files}
\label{sec:include}

%%%%%%%%%%%%%%%%%%%%%%%%%%%%%%%%%%%%%%%%
\DescribeMacro{\childdocmain}
To use the package, add the commands
\begin{center}
\begin{tabular}{l}
|% \iffalse
%
% childdoc.dtx Copyright (C) 2017-2018 Niklas Beisert
%
% This work may be distributed and/or modified under the
% conditions of the LaTeX Project Public License, either version 1.3
% of this license or (at your option) any later version.
% The latest version of this license is in
%   http://www.latex-project.org/lppl.txt
% and version 1.3 or later is part of all distributions of LaTeX
% version 2005/12/01 or later.
%
% This work has the LPPL maintenance status `maintained'.
%
% The Current Maintainer of this work is Niklas Beisert.
%
% This work consists of the files childdoc.dtx and childdoc.ins
% and the derived files childdoc.def and cdocsamp.tex with
% cdocsch1.tex, cdocsch2.tex, cdocsdrf.tex, cdocsfn1.tex, cdocsfn2.tex.
%
%<package>\ifdefined\childdocmain\endinput\fi
%<package>\ProvidesFile{childdoc.def}[2018/12/30 v2.0 child document driver]
%<samplemain>\ProvidesFile{cdocsamp.tex}[2018/12/30 v2.0 sample for childdoc]
%<*driver>
%\ProvidesFile{childdoc.drv}[2018/12/30 v2.0 childdoc reference manual file]
\PassOptionsToClass{10pt,a4paper}{article}
\documentclass{ltxdoc}

\usepackage[margin=35mm]{geometry}
\usepackage{hyperref}
\usepackage{hyperxmp}
\usepackage[usenames]{color}

\hypersetup{colorlinks=true}
\hypersetup{pdfstartview=FitH}
\hypersetup{pdfpagemode=UseNone}
\hypersetup{pdfsource={}}
\hypersetup{pdflang={en-UK}}
\hypersetup{pdfcopyright={Copyright 2017-2018 Niklas Beisert.
  This work may be distributed and/or modified under the
  conditions of the LaTeX Project Public License, either version 1.3
  of this license or (at your option) any later version.}}
\hypersetup{pdflicenseurl={http://www.latex-project.org/lppl.txt}}
\hypersetup{pdfcontactaddress={ETH Zurich, ITP, HIT K,
  Wolfgang-Pauli-Strasse 27}}
\hypersetup{pdfcontactpostcode={8093}}
\hypersetup{pdfcontactcity={Zurich}}
\hypersetup{pdfcontactcountry={Switzerland}}
\hypersetup{pdfcontactemail={nbeisert@itp.phys.ethz.ch}}
\hypersetup{pdfcontacturl={http://people.phys.ethz.ch/\xmptilde nbeisert/}}

\newcommand{\secref}[1]{\hyperref[#1]{section \ref*{#1}}}

\parskip1ex
\parindent0pt
\let\olditemize\itemize
\def\itemize{\olditemize\parskip0pt}

\begin{document}

\title{The \textsf{childdoc} Package}
\hypersetup{pdftitle={The childdoc Package}}
\author{Niklas Beisert\\[2ex]
  Institut f\"ur Theoretische Physik\\
  Eidgen\"ossische Technische Hochschule Z\"urich\\
  Wolfgang-Pauli-Strasse 27, 8093 Z\"urich, Switzerland\\[1ex]
  \href{mailto:nbeisert@itp.phys.ethz.ch}
  {\texttt{nbeisert@itp.phys.ethz.ch}}}
\hypersetup{pdfauthor={Niklas Beisert}}
\hypersetup{pdfsubject={Manual for the LaTeX2e Package childdoc}}
\date{30 December 2018, \textsf{v2.0}}
\maketitle

\begin{abstract}\noindent
\textsf{childdoc} is a \LaTeXe{} package
that enables the direct compilation
of document sections included by |\include|
to individual files.
\end{abstract}

\begingroup
\parskip0ex
\tableofcontents
\endgroup

%%%%%%%%%%%%%%%%%%%%%%%%%%%%%%%%%%%%%%%%%%%%%%%%%%%%%%%%%%%%%%%%%%%%%%%%%%%%%%%%
%%%%%%%%%%%%%%%%%%%%%%%%%%%%%%%%%%%%%%%%%%%%%%%%%%%%%%%%%%%%%%%%%%%%%%%%%%%%%%%%
\section{Introduction}

\LaTeX{} provides a mechanism to structure a large document (such as a book)
into a main file and several child files (containing the chapters)
using the |\include| command.
This mechanism is beneficial for documents
which span hundreds of pages in order to
make the source file(s) more manageable.
Moreover, compilation can be restricted to
selected child files by means of the |\includeonly| command.
The latter feature can be used to reduce the compilation time while editing
(this was significantly more useful in the earlier days of \LaTeX{})
or to generate a smaller document which is easier to navigate.
Another application of |\includeonly| is to generate
documents consisting of selected parts of the complete document.

However, there are a few drawbacks of the plain |\include| mechanism:
\begin{itemize}
\item
The child files cannot be compiled on their own,
they can only be compiled via the main file.
A naive editing environment
(such as a text editor with an option
to have the current file processed by \LaTeX)
may require one to switch to the main file before compiling;
attempting to compile the child file produces errors.
\item
The main file must be modified (each time)
to adjust the |\includeonly| command
to the present needs. This easily leaves the main file in a messy state.
\item
The generated document will always carry the filename
of the main document. This is inconvenient if
several child files are to be compiled and
to be kept for distribution.
\end{itemize}

The present package provides a simple interface
to make child files individually compilable by \LaTeX{}.
Compiling a child file then has the same effect as compiling
the main file with an |\includeonly| command
to select the appropriate child.
Moreover the generated document will carry the name of the child
rather than the main file.
This resolves all three above issues.

This feature is meant to make the editing of books,
thesis documents and lecture notes somewhat more convenient.
However, the package can also be used efficiently for
composing a series of documents (such as exercise sheets)
which are typically distributed individually.
It then assists the author in generating the individual documents
(potentially in different versions)
as well as a document containing the collected series.
Another application is in developing style files
or other kinds of included material
where compilation of the style file could redirect
to a sample or test file.

%%%%%%%%%%%%%%%%%%%%%%%%%%%%%%%%%%%%%%%%%%%%%%%%%%%%%%%%%%%%%%%%%%%%%%%%%%%%%%%%
%%%%%%%%%%%%%%%%%%%%%%%%%%%%%%%%%%%%%%%%%%%%%%%%%%%%%%%%%%%%%%%%%%%%%%%%%%%%%%%%
\section{Usage}

First of all, the package \textsf{childdoc} is \emph{not} a standard
\LaTeXe{} |.sty| style file! Therefore it needs to be invoked in
a non-standard way.

%%%%%%%%%%%%%%%%%%%%%%%%%%%%%%%%%%%%%%%%%%%%%%%%%%%%%%%%%%%%%%%%%%%%%%%%%%%%%%%%
\subsection{Included Files}
\label{sec:include}

%%%%%%%%%%%%%%%%%%%%%%%%%%%%%%%%%%%%%%%%
\DescribeMacro{\childdocmain}
To use the package, add the commands
\begin{center}
\begin{tabular}{l}
|\input{childdoc.def}|\\
|\childdocmain{}|\\
\end{tabular}
\end{center}
at the very top of the main \LaTeX{} file,
in particular \emph{before} the |\documentclass| statement!
The argument of |\childdocmain| should be left empty
(but it must be present).

%%%%%%%%%%%%%%%%%%%%%%%%%%%%%%%%%%%%%%%%
\DescribeMacro{\childdocof}
Furthermore, add the commands
\begin{center}
\begin{tabular}{l}
|\input{childdoc.def}|\\
|\childdocof{|\textit{main}|}|\\
\end{tabular}
\end{center}
at the top of every child file \textit{child}
which is included by |\include{|\textit{child}|}|
from within the main file
(or at least for those files to be compiled individually).
The argument \textit{main} must be the filename of the main file.

There are a couple of
considerations in setting up the main and child documents:

%%%%%%%%%%%%%%%%%%%%%%%%%%%%%%%%%%%%%%%%
\paragraph{Restrictions.}

Please note the following restrictions:
\begin{itemize}
\item
|\childdocmain| must be called with one argument \textit{main}
to ensure compatibility with earlier version of the package.
It must either be empty (|\childdocmain{}|)
or precisely match the filename of the main file in which it is specified.
See \secref{sec:detection} for further information.
\item
The filename \textit{main} must be specified without the |.tex| extension.
\item
The filename \textit{main} is case sensitive
(even in case-insensitive file systems)
due to internal string comparison.
\item
The argument \textit{main} should be fully expanded, it cannot be a macro.
\item
Subdirectories and special characters should be avoided in filenames.
\item
The command |\childdocmain{|\textit{main}|}| must be followed by a whitespace.
It should not be followed immediately by another command
or by a comment mark `|%|'.
This is because the \TeX{} parser reads the token immediately following
the argument of |\childdocmain| and puts it
at the beginning of every child section;
however, a white\-space is ignored.
\end{itemize}

%%%%%%%%%%%%%%%%%%%%%%%%%%%%%%%%%%%%%%%%
\paragraph{Content of Main File.}

It is advisable to place all content in the child files included by |\include|.
Any output contained in the main file will appear in all child documents
unless suppressed manually;
it cannot be suppressed automatically by the |\includeonly| directive
and thus should normally be avoided.
A method to include some content in the main file
by means of conditional processing is described in \secref{sec:conditional}.

%%%%%%%%%%%%%%%%%%%%%%%%%%%%%%%%%%%%%%%%
\paragraph{Page Numbering.}

When only a part of the document is compiled,
the appropriate numbering of pages
(as well as other status parameters)
is determined from the |.aux| files.
The latter contain information from previous passes.
However this information needs to propagate through
all intermediate child documents.
Therefore the page numbering in child documents may well
be inconsistent until the complete document is compiled at least once.

A useful (if unconventional) way to always ensure a consistent
page numbering is to restart the numbering in each child document
and denote the pages by `\textit{child}|.|\textit{page}'
where \textit{child} represents the chapter/section number of the child file.
This can be achieved by the command
|\numberwithin{page}{|\textit{child}|}|
of the \textsf{amsmath} package
where \textit{child} can be |chapter| or |section|
depending on the chosen structuring.
Alternatively, one can modify the macro |\thepage| appropriately
and reset the counter |page| at the start of each child file.

%%%%%%%%%%%%%%%%%%%%%%%%%%%%%%%%%%%%%%%%%%%%%%%%%%%%%%%%%%%%%%%%%%%%%%%%%%%%%%%%
\subsection{Conditional Processing}
\label{sec:conditional}

The package provides a mechanism to compile different versions
of a document. To customise the versions further some conditional processing
can come in handy to distinguish which version is being compiled.
The package provides two macros to describe the compilation context:

%%%%%%%%%%%%%%%%%%%%%%%%%%%%%%%%%%%%%%%%
\DescribeMacro{\ifchilddoc}
The conditional |\ifchilddoc| distinguishes between the compilation of
child documents and the main document:
%
\begin{center}
|\ifchilddoc |\textit{child-code}| |[|\||else |\textit{main-code}]| \||fi|
\end{center}

%%%%%%%%%%%%%%%%%%%%%%%%%%%%%%%%%%%%%%%%
\DescribeMacro{\childdocname}
\DescribeMacro{\childdocjob}
The macro |\childdocname| contains the filename (without extension)
of the main or child file being processed.
Note that |\childdocjob| will always contain the name of the main file.

%%%%%%%%%%%%%%%%%%%%%%%%%%%%%%%%%%%%%%%%
\paragraph{Title Page.}

Conditional processing can be used to include a title or banner page
in the main document when proper precautions are taken.
Importantly, the code in the main file should ensure that the page counter
(as well as other status parameters which are stored in the |.aux| files)
takes the same value after the conditional processing.
Otherwise the page numbers may take divergent values
depending on which part is compiled.

For example, a title page could be declared by:
%
\begin{center}
\begin{tabular}{l}
|\ifchilddoc\||else|\\
|\addtocounter{page}{-1}|\\
\textit{code for title page}\\
|\newpage|\\
|\||fi|
\end{tabular}
\end{center}
%
A banner page for the child documents can be generated by:
%
\begin{center}
\begin{tabular}{l}
|\ifchilddoc|\\
|\addtocounter{page}{-1}|\\
\textit{code for banner page}\\
|\newpage|\\
|\||fi|
\end{tabular}
\end{center}
%
Here one could write a message such as:
\begin{center}
|This is the part \childdocname{} of \childdocjob{}.|
\end{center}

%%%%%%%%%%%%%%%%%%%%%%%%%%%%%%%%%%%%%%%%%%%%%%%%%%%%%%%%%%%%%%%%%%%%%%%%%%%%%%%%
\subsection{Flags}
\label{sec:flags}

The package makes it easy to generate different versions
of the main or child documents.
To this end compilation flags can be defined
and assigned different default values.
They will be particularly useful in conjunction
with the forwarding mechanism described in \secref{sec:forward}.

For example, it may be useful to have a flag |\version|
which can be set to |draft| or |final|.
The document source will contain some conditional code
depending on the value of |\version|.
Suppose further, the flag should default to |final| for the main file
and to |draft| for child files
which is a natural assignment for editing the document.
This is achieved by placing the following code
in the preamble of the main document
(below the |\childdocmain| directive):
%
\begin{center}
\begin{tabular}{l}
|\ifchilddoc|\\
|\providecommand{\version}{draft}|\\
|\||else|\\
|\providecommand{\version}{final}|\\
|\||fi|
\end{tabular}
\end{center}
%
The definition by |\providecommand| makes sure
that previous definitions are not overwritten.
Further statements |\providecommand{\version}{...}|
can thus be added before the above code to override it.

For the main file, one might add a line
(between |\childdocmain| and the above block)
%
\begin{center}
|%\ifchilddoc\||else\providecommand{\version}{draft}\||fi|
\end{center}
%
which can be uncommented to produce a draft version.
Likewise one can add a line to the very top of a child file
(above the |\childdocof{|\textit{main}|}| directive)
%
\begin{center}
|%\providecommand{\version}{final}|
\end{center}
%
which can be uncommented to produce the final version of this child document.

%%%%%%%%%%%%%%%%%%%%%%%%%%%%%%%%%%%%%%%%%%%%%%%%%%%%%%%%%%%%%%%%%%%%%%%%%%%%%%%%
\subsection{Forwarding}
\label{sec:forward}

Different versions of the main or child documents
using compilation flags as described in \secref{sec:flags}
can be (permanently) stored in different files
for convenient compilation, viewing and distribution.
To this end, the package defines a command
to pass on compilation to a different file:

%%%%%%%%%%%%%%%%%%%%%%%%%%%%%%%%%%%%%%%%
\DescribeMacro{\childdocforward}
The command |\childdocforward| redirects processing to
another source file:
%
\begin{center}
\begin{tabular}{l}
|\input{childdoc.def}|\\
|\childdocforward[|\textit{main}|]{|\textit{dest}|}|\\
\end{tabular}
\end{center}
%
The argument \textit{dest} is the destination file
(without extension).
It should be the main file or one of the child files.
Note that further \textsf{childdoc} directives
such as |\childdocof| and |\childdocforward|
in the indicated file will be processed in this form.
The optional argument \textit{main}
passes on directly to the main file \textit{main}
while pretending to compile the child \textit{dest}.
This form behaves as if \textit{dest}
issues |\childdocof{|\textit{main}|}| right away,
and no further \textsf{childdoc} directives will be processed.

%%%%%%%%%%%%%%%%%%%%%%%%%%%%%%%%%%%%%%%%
\DescribeMacro{\...prefix}
In the alternative form |\childdocforwardprefix|,
%
\begin{center}
\begin{tabular}{l}
|\input{childdoc.def}|\\
|\childdocforwardprefix[|\textit{main}|]{|\textit{prefix}|}{|\textit{dest}|}|
\end{tabular}
\end{center}
%
the destination file is determined by a pattern
depending on the current file:
To make this work, the current file must be called
`{\textit{prefix}\hspace{0.2em}\textit{suffix}}'
with \textit{prefix} matching precisely the argument.
Processing is then passed on to the file
`{\textit{dest}\hspace{0.2em}\textit{suffix}}'.
Surely, the same effect is achieved by
directly specifying the
argument `{\textit{dest}\hspace{0.2em}\textit{suffix}}'
in the first form.
However, that requires to set up a different file
for each child. With the alternative form of the command
all these files can have exactly the same content
which simplifies setting them up and maintaining them.

For example, the following file |draft.tex|
with a compilation flag |\version| as described in \secref{sec:flags}
compiles the main document as a draft:
%
\begin{center}
\begin{tabular}{l}
|\def\version{draft}|\\
|\input{childdoc.def}|\\
|\childdocforward{|\textit{main}|}|
\end{tabular}
\end{center}
%
Likewise, the following files |final|\textit{nn}|.tex|
compile the final version of the child document
|child|\textit{nn}|.tex|:
%
\begin{center}
\begin{tabular}{l}
|\def\version{final}|\\
|\input{childdoc.def}|\\
|\childdocforwardprefix{final}{child}|
\end{tabular}
\end{center}
%

Note that when several versions of a main file and/or of each child file
are to be generated, it may be convenient to set up a |Makefile| or
shell script to automatise the process.

%%%%%%%%%%%%%%%%%%%%%%%%%%%%%%%%%%%%%%%%%%%%%%%%%%%%%%%%%%%%%%%%%%%%%%%%%%%%%%%%
\subsection{Command Line Processing}
\label{sec:commandline}

The effect of redirection files can also be achieved by invoking
the \LaTeX{} compiler with a more elaborate command line.
Most conveniently this should be done as part
of a shell script or a |Makefile|.

When using \textsf{childdoc} in the main file, the following
command lines effectively perform a redirection
(note that depending on the shell being used,
backslashes may have to be doubled: `|\|' $\to$ `|\\|'):
%
\begin{center}
|... -jobname "|\textit{target}|" |\\|"|[\textit{flags}]%
|\input{childdoc.def}\childdocforward[|\textit{main}|]{|\textit{dest}|}"|
\end{center}
%
Here \textit{target} is the name of the output file,
\textit{main} is the name of the main file
and \textit{dest} is the name of the main or child file to be processed
(all filenames without extensions).
The optional argument \textit{main} can be omitted
if \textit{main} matches \textit{dest}.
Optionally, compilation \textit{flags} can be defined via |\def| commands.
This command line makes the \TeX{} engine believe
it is compiling the file \textit{target}
whose content is specified as the latter parameter.
The provided code then forwards the processing to
\textit{main} or \textit{dest} as described in \secref{sec:forward}.

%%%%%%%%%%%%%%%%%%%%%%%%%%%%%%%%%%%%%%%%%%%%%%%%%%%%%%%%%%%%%%%%%%%%%%%%%%%%%%%%
\subsection{Include by Input}
\label{sec:input}

Including child documents by |\include| has some restrictions by design.
Most notably, the content of a child document always occupies
its own set of pages; pages cannot be shared between child documents.
Usually, this behaviour makes perfect sense
because each child document contain an essential part of the document.
However, in some situations it may be desirable to compose
a document from a collection of parts
without having mandatory page breaks between then.
For this case, the package
provides a mechanism to include parts
by |\input| which can also be processed individually.
However, by construction this mechanism
requires manual handling of the content to be output.

%%%%%%%%%%%%%%%%%%%%%%%%%%%%%%%%%%%%%%%%
\DescribeMacro{\ifchilddocmanual}
The main file should be prepared as usual, see \secref{sec:include}.
However, the document body must make a distinction
between processing of an individual part and of the main document, e.g.:
%
\begin{center}
\begin{tabular}{l}
|\ifchilddocmanual|\\
|\input{\childdocname}|\\
|\||else|\\
\textit{document body with }|\input{|\textit{part}|}|\\
|\||fi|
\end{tabular}
\end{center}
%
The conditional |\ifchilddocmanual| is true whenever
a part to be included by |\input| is being compiled,
and the name of the part is stored in |\childdocname|.

%%%%%%%%%%%%%%%%%%%%%%%%%%%%%%%%%%%%%%%%
\DescribeMacro{\childdocby}
Each part to be included by |\input| should start with:
%
\begin{center}
\begin{tabular}{l}
|\input{childdoc.def}|\\
|\childdocby{|\textit{main}|}|\\
\end{tabular}
\end{center}
%
The directive |\childdocby| is similar to |\childdocof|
described in \secref{sec:include},
but the subsequent selection of content must be done manually.
To that end, both |\ifchilddoc| and |\ifchilddocmanual|
will be true upon processing of a part,
and the name of the part is stored in |\childdocname|.
Note that |\jobname| will be set to the filename of the current part
so that each part receives an individual |.aux| file
that does not interfere with the |.aux| file(s) of the main document.
This behaviour can be altered by the alternative form
|\childdocby[*]{|\textit{main}|}| (with a non-empty optional argument)
which uses the |.aux| file of the main document
by setting |\jobname| to \textit{main}.

%%%%%%%%%%%%%%%%%%%%%%%%%%%%%%%%%%%%%%%%%%%%%%%%%%%%%%%%%%%%%%%%%%%%%%%%%%%%%%%%
\subsection{Driver Development}
\label{sec:driver}

The \textsf{childdoc} mechanism can also be use for the development
of definition files such as \LaTeX{} styles or classes.
This case differs from the above setup with multiple parts
included by |\include| in that no |\includeonly| should be invoked.
This can be achieved by starting the include file
(before |\ProvidesPackage|) with:
%
\begin{center}
\begin{tabular}{l}
|\input{childdoc.def}|\\
|\childdocforward{|\textit{main}|}|\\
\end{tabular}
\end{center}
%
or alternatively with:
%
\begin{center}
\begin{tabular}{l}
|\input{childdoc.def}|\\
|\childdocby{|\textit{main}|}|\\
\end{tabular}
\end{center}
%
Both forms have slightly different effects as described above.
The main file is prepared as usual, see \secref{sec:include}.

%%%%%%%%%%%%%%%%%%%%%%%%%%%%%%%%%%%%%%%%%%%%%%%%%%%%%%%%%%%%%%%%%%%%%%%%%%%%%%%%
\subsection{Legacy Detection}
\label{sec:detection}

The directive |\childdocmain| in the main file can detect
whether the complete document or merely a child is to be compiled
even without using the directive |\childdocof|.
This method is deprecated because it is less robust
and there is no compelling reason to use it;
it is merely provided for backward compatibility
and it may be removed in future versions.

If the detection mechanism is to be used,
it is mandatory to correctly specify
the filename of the main file as the argument of |\childdocmain|:
%
\begin{center}
\begin{tabular}{l}
|\input{childdoc.def}|\\
|\childdocmain{|\textit{main}|}|\\
\end{tabular}
\end{center}
%
If |\jobname| does not match the argument \textit{main} of |\childdocmain|,
it is assumed that |\jobname| points to the child file to be compiled.
When using |\childdocmain| with the main file specified as argument,
it suffices to start a child file
with just |\input{|\textit{main}|}|
without loading of the package and using |\childdocof|.
If instead all processing is done
with the appropriate \textsf{childdoc} directives,
the argument of \textit{main} of |\childdocmain| can be empty.

An alternative version of the command line processing described
in \secref{sec:commandline} using the detection mechanism reads:
%
\begin{center}
|... -jobname "|\textit{target}|" "|[\textit{flags}]%
[|\def\jobname{|\textit{dest}|}|]|\input{|\textit{main}|}"|
\end{center}

%%%%%%%%%%%%%%%%%%%%%%%%%%%%%%%%%%%%%%%%%%%%%%%%%%%%%%%%%%%%%%%%%%%%%%%%%%%%%%%%
\subsection{Manual Code}
\label{sec:manual}

In case one cannot be certain whether the definitions file |childdoc.def|
is installed on the target \TeX{} distribution
and one prefers not to ship it,
it is conceivable to paste a few relevant commands into the sources.

To that end, drop all statements |\input{childdoc.def}|
and perform the replacements as outlined below.
Instead of |\childdocmain{|\textit{main}|}| add the following code
to the top of the main file:
%
\begin{center}
\begin{tabular}{l}
|\||ifdefined\childdocname\endinput\||fi\newif\ifchilddoc|\\
|\edef\childdocname{\scantokens\expandafter{\jobname\noexpand}}|\\
|\def\childdocmain{|\textit{main}|}\||ifx\childdocmain\childdocname\||else|\\
|\childdoctrue\includeonly{\childdocname}\let\jobname\childdocmain\||fi|\\
\end{tabular}
\end{center}
%
Instead of |\childdocof{|\textit{main}|}| just include the main file
at the top of each child file:
%
\begin{center}
|\input{|\textit{main}|}|
\end{center}
%
A simple redirection |\childdocforward{|\textit{dest}|}| is achieved by:
%
\begin{center}
|\def\jobname{|\textit{dest}|}\input{\jobname}|
\end{center}
%
The redirection with prefix
|\childdocforwardprefix[|\textit{prefix}|]{|\textit{dest}|}|
is accomplished by:
%
\begin{center}
\begin{tabular}{l}
|{\edef\jobname{\scantokens\expandafter{\jobname\noexpand}}|\\
|\def\redirectjob |\textit{prefix}|#1~~~{\gdef\jobname{|\textit{dest}|#1}}|\\
|\expandafter\redirectjob\jobname~~~}\input{\jobname}|
\end{tabular}
\end{center}

In an alternative approach,
child documents can be compiled by a specific command line
without additional code or specific definitions:
%
\begin{center}
|... -jobname "|\textit{target}|" "|[\textit{flags}]%
|\includeonly{|\textit{dest}|}\input{|\textit{main}|}"|
\end{center}
%

%%%%%%%%%%%%%%%%%%%%%%%%%%%%%%%%%%%%%%%%%%%%%%%%%%%%%%%%%%%%%%%%%%%%%%%%%%%%%%%%
%%%%%%%%%%%%%%%%%%%%%%%%%%%%%%%%%%%%%%%%%%%%%%%%%%%%%%%%%%%%%%%%%%%%%%%%%%%%%%%%
\section{Information}

%%%%%%%%%%%%%%%%%%%%%%%%%%%%%%%%%%%%%%%%%%%%%%%%%%%%%%%%%%%%%%%%%%%%%%%%%%%%%%%%
\subsection{Copyright}

Copyright \copyright{} 2017--2018 Niklas Beisert

This work may be distributed and/or modified under the
conditions of the \LaTeX{} Project Public License, either version 1.3
of this license or (at your option) any later version.
The latest version of this license is in
  \url{http://www.latex-project.org/lppl.txt}
and version 1.3 or later is part of all distributions of \LaTeX{}
version 2005/12/01 or later.

This work has the LPPL maintenance status `maintained'.

The Current Maintainer of this work is Niklas Beisert.

This work consists of the files |README.txt|, |childdoc.ins| and |childdoc.dtx|
as well as the derived files |childdoc.def|, |cdocsamp.tex|
with |cdocsch1.tex|, |cdocsch2.tex|, |cdocspt3.tex|, |cdocspt4.tex|,
|cdocsdrf.tex|, |cdocsfn1.tex|, |cdocsfn2.tex|
as well as |childdoc.pdf|.

%%%%%%%%%%%%%%%%%%%%%%%%%%%%%%%%%%%%%%%%%%%%%%%%%%%%%%%%%%%%%%%%%%%%%%%%%%%%%%%%
\subsection{Files and Installation}

The package consists of the files:
%
\begin{center}
\begin{tabular}{ll}
    |README.txt|   & readme file \\
    |childdoc.ins| & installation file \\
    |childdoc.dtx| & source file \\
    |childdoc.def| & definition file \\
    |cdocsamp.tex| & sample main file \\
    |cdocsch1.tex| & sample include file \\
    |cdocsch2.tex| & sample include file \\
    |cdocspt3.tex| & sample part file \\
    |cdocspt4.tex| & sample part file \\
    |cdocsdrf.tex| & sample redirection file \\
    |cdocsfn1.tex| & sample redirection file \\
    |cdocsfn2.tex| & sample redirection file \\
    |childdoc.pdf| & manual
\end{tabular}
\end{center}
%
The distribution consists of the files
|README.txt|, |childdoc.ins| and |childdoc.dtx|.
%
\begin{itemize}
\item
Run (pdf)\LaTeX{} on |childdoc.dtx|
to compile the manual |childdoc.pdf| (this file).
\item
Run \LaTeX{} on |childdoc.ins| to create the definitions file |childdoc.def|
and the sample |cdocsamp.tex| with include files
|cdocsch1.tex|, |cdocsch2.tex|, |cdocspt3.tex|, |cdocspt4.tex|,
|cdocsdrf.tex|, |cdocsfn1.tex|, |cdocsfn2.tex|.
Then copy the file |childdoc.def| to an appropriate directory of your \LaTeX{}
distribution, e.g.\ \textit{texmf-root}|/tex/latex/childdoc|.
\end{itemize}

%%%%%%%%%%%%%%%%%%%%%%%%%%%%%%%%%%%%%%%%%%%%%%%%%%%%%%%%%%%%%%%%%%%%%%%%%%%%%%%%
\subsection{Related CTAN Packages}

There are several other packages which offer a similar functionality:
%
\begin{itemize}
\item
The packages
\href{http://ctan.org/pkg/docmute}{\textsf{docmute}},
\href{http://ctan.org/pkg/includex}{\textsf{includex}} and
\href{http://ctan.org/pkg/standalone}{\textsf{standalone}}
provide commands to include only the document body of
a child file thus allowing both files to be compiled individually.
\item
The packages \href{http://ctan.org/pkg/subdocs}{\textsf{subdocs}}
and \href{http://ctan.org/pkg/subfiles}{\textsf{subfiles}}
provide structures in which the main and child documents can be
encapsulated and allowing them to be compiled individually.
The inclusion mechanism is different from the conventional |\include|.
\item
The package \href{http://ctan.org/pkg/combine}{\textsf{combine}}
is an elaborate solution to combine several documents into one.
\end{itemize}
%
See also the CTAN topic \href{http://ctan.org/topic/subdocs}{\textsf{subdocs}}
for further related packages.
The present package differs from the above solutions in that
a document structure constructed with the conventional |\include| mechanism
just needs two extra commands at the top of every file
such that all constituent files can be compiled individually.

%%%%%%%%%%%%%%%%%%%%%%%%%%%%%%%%%%%%%%%%%%%%%%%%%%%%%%%%%%%%%%%%%%%%%%%%%%%%%%%%
%\subsection{Feature Suggestions}
%
%The following is a list of features which may be useful for future
%versions of this package:
%%
%\begin{itemize}
%\item
%\ldots
%\end{itemize}

%%%%%%%%%%%%%%%%%%%%%%%%%%%%%%%%%%%%%%%%%%%%%%%%%%%%%%%%%%%%%%%%%%%%%%%%%%%%%%%%
\subsection{Revision History}

%%%%%%%%%%%%%%%%%%%%%%%%%%%%%%%%%%%%%%%%
\paragraph{v2.0:} 2018/12/30

\begin{itemize}
\item
immediate forward processing
\item
added |\childdocby| mechanism
\item
manual restructured
\end{itemize}

%%%%%%%%%%%%%%%%%%%%%%%%%%%%%%%%%%%%%%%%
\paragraph{v1.6:} 2018/01/17

\begin{itemize}
\item
application for development of include files
\item
corrections to manual
\end{itemize}

%%%%%%%%%%%%%%%%%%%%%%%%%%%%%%%%%%%%%%%%
\paragraph{v1.5:} 2017/05/21

\begin{itemize}
\item
more complete structuring introduced
\item
|\childdocof| introduced
\item
|\childdoc| renamed to |\childdocmain|
\item
|\childredirect| renamed to |\childdocforward| and |\childdocforwardprefix|
and functionality expanded
\end{itemize}

%%%%%%%%%%%%%%%%%%%%%%%%%%%%%%%%%%%%%%%%
\paragraph{v1.0:} 2017/04/27

\begin{itemize}
\item
manual and install package
\item
first version published on CTAN
\end{itemize}

%%%%%%%%%%%%%%%%%%%%%%%%%%%%%%%%%%%%%%%%
\paragraph{v0.6:} 2017/04/26

\begin{itemize}
\item
redirection mechanism added
\end{itemize}

%%%%%%%%%%%%%%%%%%%%%%%%%%%%%%%%%%%%%%%%
\paragraph{v0.5:} 2017/04/26

\begin{itemize}
\item
functionality in definition file
\end{itemize}


%%%%%%%%%%%%%%%%%%%%%%%%%%%%%%%%%%%%%%%%%%%%%%%%%%%%%%%%%%%%%%%%%%%%%%%%%%%%%%%%
%%%%%%%%%%%%%%%%%%%%%%%%%%%%%%%%%%%%%%%%%%%%%%%%%%%%%%%%%%%%%%%%%%%%%%%%%%%%%%%%
%%%%%%%%%%%%%%%%%%%%%%%%%%%%%%%%%%%%%%%%%%%%%%%%%%%%%%%%%%%%%%%%%%%%%%%%%%%%%%%%
\appendix

\settowidth\MacroIndent{\rmfamily\scriptsize 000\ }

 \DocInput{childdoc.dtx}

\end{document}
%</driver>
% \fi
%
% %%%%%%%%%%%%%%%%%%%%%%%%%%%%%%%%%%%%%%%%%%%%%%%%%%%%%%%%%%%%%%%%%%%%%%%%%%%%%%
% %%%%%%%%%%%%%%%%%%%%%%%%%%%%%%%%%%%%%%%%%%%%%%%%%%%%%%%%%%%%%%%%%%%%%%%%%%%%%%
% \section{Sample}
%\iffalse
%<*samplemain>
%\fi
%
% The following presents a sample document
% with two chapters, two parts, a title page,
% a compile flag as well as three forwarding files to set the flag.
% It consists of eight |.tex| files:
% \begin{center}
% \begin{tabular}{ll}
% |cdocsamp.tex|&main file\\
% |cdocsch1.tex|&include file for chapter 1\\
% |cdocsch2.tex|&include file for chapter 2\\
% |cdocspt3.tex|&include file for part 3\\
% |cdocspt4.tex|&include file for part 4\\
% |cdocsdrf.tex|&forwarding file for main file in draft mode\\
% |cdocsfi1.tex|&forwarding file for final version of chapter 1\\
% |cdocsfi2.tex|&forwarding file for final version of chapter 2\\
% \end{tabular}
% \end{center}
% Each of the eight files can be compiled directly by the \LaTeX{} compiler.
%
% %%%%%%%%%%%%%%%%%%%%%%%%%%%%%%%%%%%%%%
% \paragraph{Main File.}
%
% The main file is called |cdocsamp.tex|.
%
% Load the \textsf{childdoc} definitions and
% declare the filename for the main document:
%    \begin{macrocode}
\input{childdoc.def}
\childdocmain{}
%    \end{macrocode}

% Optional override for |\version| flag:
%    \begin{macrocode}
%%\ifchilddoc\else\providecommand{\version}{draft}\fi
%    \end{macrocode}

% Define the default values for the |\version| flag
% (|final| for the main file and |draft| for childs):
%    \begin{macrocode}
\ifchilddoc
\providecommand{\version}{draft}
\else
\providecommand{\version}{final}
\fi
%    \end{macrocode}

% Load the standard document class:
%    \begin{macrocode}
\documentclass[12pt]{article}
%    \end{macrocode}

% Start the document body:
%    \begin{macrocode}
\begin{document}
%    \end{macrocode}

% Declare a title page.
% Print title, part of document being processed and version flag:
%    \begin{macrocode}
\addtocounter{page}{-1}
\begin{center}
{\LARGE\bfseries{}childdoc example\par}
\vspace{1cm}
\ifchilddoc
\ifchilddocmanual part\else chapter\fi:
`\childdocname' of `\childdocjob'\par
\else
main document: `\childdocjob'\par
\fi
version: \version\par
\end{center}
\newpage
%    \end{macrocode}

% Manually include selected file,
% otherwise process as usual:
%    \begin{macrocode}
\ifchilddocmanual
\section*{part `\childdocname'}
\input{\childdocname}
\else
%    \end{macrocode}

% Include the two chapters:
%    \begin{macrocode}
\include{cdocsch1}
\include{cdocsch2}
%    \end{macrocode}

% Include the two parts unless only chapters should be displayed:
%    \begin{macrocode}
\ifchilddoc\else
\section{part three}
\input{cdocspt3}
\section{part four}
\input{cdocspt4}
\fi
%    \end{macrocode}

% Process as usual until here:
%    \begin{macrocode}
\fi
%    \end{macrocode}

% End of document body:
%    \begin{macrocode}
\end{document}
%    \end{macrocode}
%\iffalse
%</samplemain>
%\fi
%
% %%%%%%%%%%%%%%%%%%%%%%%%%%%%%%%%%%%%%%
% \paragraph{Chapter Include Files.}
%
% The include files are called |cdocsch1.tex| and |cdocsch2.tex|.
%
%\iffalse
%<*samplechap1|samplechap2>
%\fi

% Optional override for |\version| flag:
%    \begin{macrocode}
%%\providecommand{\version}{final}
%    \end{macrocode}

% Include the main document:
%    \begin{macrocode}
\input{childdoc.def}
\childdocof{cdocsamp}
%    \end{macrocode}

%\iffalse
%</samplechap1|samplechap2>
%\fi
%
%\iffalse
%<*samplechap1>
%\fi
% Some text for chapter 1:
%    \begin{macrocode}
\section{one}
some text in chapter one
%    \end{macrocode}

%\iffalse
%</samplechap1>
%\fi
% Some text for chapter 2:
%\iffalse
%<*samplechap2>
%\fi
%    \begin{macrocode}
\section{two}
more text in chapter two
%    \end{macrocode}

%\iffalse
%</samplechap2>
%\fi
%
% %%%%%%%%%%%%%%%%%%%%%%%%%%%%%%%%%%%%%%
% \paragraph{Part Include Files.}
%
% The include files are called |cdocspt3.tex| and |cdocspt4.tex|.
%
%\iffalse
%<*samplepart3|samplepart4>
%\fi

% Optional override for |\version| flag:
%    \begin{macrocode}
%%\providecommand{\version}{final}
%    \end{macrocode}

% Include the main document:
%    \begin{macrocode}
\input{childdoc.def}
\childdocby{cdocsamp}
%    \end{macrocode}

%\iffalse
%</samplepart3|samplepart4>
%\fi
%
%\iffalse
%<*samplepart3>
%\fi
% Some text for part 3:
%    \begin{macrocode}
some text in part three
%    \end{macrocode}

%\iffalse
%</samplepart3>
%\fi
% Some text for part 4:
%\iffalse
%<*samplepart4>
%\fi
%    \begin{macrocode}
more text in part four
%    \end{macrocode}

%\iffalse
%</samplepart4>
%\fi
%
% %%%%%%%%%%%%%%%%%%%%%%%%%%%%%%%%%%%%%%
% \paragraph{Forwarding for a Complete Draft.}
%
% The following forwarding file |cdocsdrf.tex|
% compiles the main document in draft mode:
%\iffalse
%<*sampledraft>
%\fi
%    \begin{macrocode}
\def\version{draft}
\input{childdoc.def}
\childdocforward{cdocsamp}
%    \end{macrocode}

%\iffalse
%</sampledraft>
%\fi
%
% %%%%%%%%%%%%%%%%%%%%%%%%%%%%%%%%%%%%%%
% \paragraph{Forwarding for Final Version of the Chapters.}
%
% The following forwarding files |cdocsfn1.tex| and |cdocsfn2.tex|
% (with identical content)
% compile the final versions of the child documents
% |cdocsch1.tex| and |cdocsch2.tex|, respectively:
%\iffalse
%<*samplefinal>
%\fi
%    \begin{macrocode}
\def\version{final}
\input{childdoc.def}
\childdocforwardprefix[cdocsamp]{cdocsfn}{cdocsch}
%    \end{macrocode}

%\iffalse
%</samplefinal>
%\fi
%
% %%%%%%%%%%%%%%%%%%%%%%%%%%%%%%%%%%%%%%
% \paragraph{Command Line Processing.}
%
% The following three command lines generate the output files
% |cdocscld|, |cdocscl1| and |cdocscl2|
% which should be identical to
% |cdocsdrf|, |cdocsch1| and |cdocsfn2|, respectively:
% \begin{center}
% \begin{tabular}{l}
% |latex -jobname cdocscld \|\\
% |  "\def\version{draft}\input{childdoc.def}\childdocforward{cdocsamp}"|\\
% |latex -jobname cdocscl1 \|\\
% |  "\input{childdoc.def}\childdocforward[cdocsamp]{cdocsch1}"|\\
% |latex -jobname cdocscl2 \|\\
% |  "\def\version{final}\input{childdoc.def}\childdocforward{cdocsch2}"|
% \end{tabular}
% \end{center}
% Note that the trailing backslash on each first line
% merely continues the input to the second line
% (for convenient cut ant paste).
% Furthermore, the command |latex| can be replaced by any
% of its alternative versions such as |pdflatex|.
%
% %%%%%%%%%%%%%%%%%%%%%%%%%%%%%%%%%%%%%%%%%%%%%%%%%%%%%%%%%%%%%%%%%%%%%%%%%%%%%%
% %%%%%%%%%%%%%%%%%%%%%%%%%%%%%%%%%%%%%%%%%%%%%%%%%%%%%%%%%%%%%%%%%%%%%%%%%%%%%%
% \section{Implementation}
%\iffalse
%<*package>
%\fi
%
% This section describes the definitions file |childdoc.def|.

% The definitions cannot be loaded using |\usepackage| or |\RequirePackage|
% which has a mechanism to prevent loading a style file more than once.
% When loading the definitions by means of |\input|
% multiple instances have to be prevented manually:
%\iffalse
%This code needs to be before the `\ProvidesFile' directive
%which is defined at the beginning of this file.
%Therefore it is also placed there and commented out here.
%</package>
%<*discard>
%\fi
%    \begin{macrocode}
\ifdefined\childdocmain\endinput\fi
%    \end{macrocode}
%\iffalse
%</discard>
%<*package>
%\fi
%
% \macro{\ifchilddoc}
% \macro{\ifchilddocmanual}
% The conditional |\ifchilddoc| tells whether a
% child (true) or main (false) document is being compiled.
% The conditional |\ifchilddocmanual| tells whether
% the |\includeonly| mechanism is used (false) or
% the selection of child files must be performed manually (true).
% The definitions initialise to false:
%    \begin{macrocode}
\newif\ifchilddoc
\newif\ifchilddocmanual
%    \end{macrocode}

% \macro{\childdocname}
% \macro{\childdocjob}
% The macro |\childdocname| stores the name of the main document
% to be compiled. The macro |\childdocjob| stores the name of
% the document on which the \LaTeX{} compiler was originally invoked.
% The content of |\jobname| cannot be compared
% to filenames specified in the source due to different catcodes.
% The following code rescans |\jobname|, stores the result
% in |\childdocname| and saves a copy in |\childdocjob|:
%    \begin{macrocode}
\edef\childdocname{\scantokens\expandafter{\jobname\noexpand}}
\let\childdocjob\childdocname
%    \end{macrocode}

% \macro{\childdocdisable}
% The macro |\childdocdisable| prevents the main file
% from being processed more than once.
% At this stage, the main document command |\childdocmain|
% is assumed to be called once again where it should do nothing.
% Any subsequent call to it should prevent
% a secondary processing of the main document
% It overwrites the forwarding commands
% |\childdocof| and |\childdocforward|
% with empty macros to prevent further inclusions of the main document:
%    \begin{macrocode}
\newcommand{\childdocdisable}
{
  \renewcommand{\childdocmain}[1]{\renewcommand{\childdocmain}[1]{\endinput}}
  \renewcommand{\childdocof}[1]{}
  \renewcommand{\childdocby}[2][]{}
  \renewcommand{\childdocforward}[2][]{}
  \renewcommand{\childdocdisable}{}
}
%    \end{macrocode}

% \macro{\childdocmain}
% The macro |\childdocmain| is to be called at the top of the main file
% with nothing or the main filename (without extension) as argument.
% First, it breaks loops.
% If the argument is not empty and does not match |\childdocname|
% (which is set by the first inclusion of |childdoc.def|),
% |\ifchilddoc| is set to true, |\includeonly| is applied to the child file
% and |\jobname| is set to the main file
% (for proper handling of |.aux| files):
%    \begin{macrocode}
\newcommand{\childdocmain}[1]
{
  \childdocdisable\childdocmain{}
  \if?#1?\else
    \begingroup
      \def\childdoctmp{#1}
      \ifx\childdoctmp\childdocname
        \def\childdoctmp{}
      \else
        \def\childdoctmp
        {
          \childdoctrue
          \includeonly{\childdocname}
          \def\childdocjob{#1}
          \def\jobname{#1}
        }
      \fi
      \expandafter
    \endgroup
    \childdoctmp
  \fi
}
%    \end{macrocode}

% \macro{\childdocof}
% The command |\childdocof| redirects
% compilation to the main file |#1|.
%    \begin{macrocode}
\newcommand{\childdocof}[1]
{
  \childdocdisable
  \childdoctrue
  \includeonly{\childdocname}
  \def\jobname{#1}
  \def\childdocjob{#1}
  \input{#1}
}
%    \end{macrocode}

% \macro{\childdocby}
% The command |\childdocby| ....
%    \begin{macrocode}
\newcommand{\childdocby}[2][]
{
  \childdocdisable
  \childdoctrue
  \childdocmanualtrue
  \if?#1?\else
    \def\jobname{#2}
  \fi
  \def\childdocjob{#2}
  \input{#2}
  \endinput
}
%    \end{macrocode}

% \macro{\childdocforward}
% The command |\childdocforward| redirects
% compilation to the main file or
% (if the optional argument is given) a child file.
% Parameters are set as if the main file
% or a child file starting with |\childdocof| was compiled.
% Then compilation is handed over to the main file:
%    \begin{macrocode}
\newcommand{\childdocforward}[2][]
{
  \begingroup
    \if?#1?
      \def\childdoctmp
      {
        \def\childdocname{#2}
        \def\childdocjob{#2}
        \def\jobname{#2}
        \input{#2}
        \endinput
      }
    \else
      \def\childdoctmp
      {
        \childdocdisable
        \def\childdocname{#2}
        \childdoctrue
        \includeonly{#2}
        \def\childdocjob{#1}
        \def\jobname{#1}
        \input{#1}
        \endinput
      }
    \fi
    \expandafter
  \endgroup
  \childdoctmp
}
%    \end{macrocode}

% \macro{\childdocforwardprefix}
% The command |\childdocforwardprefix| redirects
% compilation to the main or a child file by means of a pattern.
% The prefix |#1| in the current filename is replaced by |#2|
% and the suffix of the current filename is kept
% (it is assumed that the filename does not contain the substring `|~~~|'
% which is used as a delimiter).
% Compilation is handed over to the new file by |\childdocforward|:
%    \begin{macrocode}
\newcommand{\childdocforwardprefix}[3][]
{
  \begingroup
    \def\childdocextract #2##1~~~{\def\childdoctmp{\childdocforward[#1]{#3##1}}}
    \expandafter\childdocextract\childdocname~~~
    \expandafter
  \endgroup
  \childdoctmp
}
%    \end{macrocode}

% \macro{\childdoc}
% The deprecated macro |\childdoc| is a legacy version of |\childdocmain|:
%    \begin{macrocode}
\newcommand{\childdoc}{\childdocmain}
%    \end{macrocode}

% \macro{\childdocredirect}
% The deprecated macro |\childdocredirect| is a legacy version
% of |\childdocforward| and |\childdocforwardprefix|:
%    \begin{macrocode}
\newcommand{\childdocredirect}[2][]
{
  \begingroup
    \if?#1?
      \def\childdoctmp{\childdocforward{#2}}
    \else
      \def\childdoctmp{\childdocforwardprefix{#1}{#2}}
    \fi
    \expandafter
  \endgroup
  \childdoctmp
}
%    \end{macrocode}

%\iffalse
%</package>
%\fi
%
\endinput
|\\
|\childdocmain{}|\\
\end{tabular}
\end{center}
at the very top of the main \LaTeX{} file,
in particular \emph{before} the |\documentclass| statement!
The argument of |\childdocmain| should be left empty
(but it must be present).

%%%%%%%%%%%%%%%%%%%%%%%%%%%%%%%%%%%%%%%%
\DescribeMacro{\childdocof}
Furthermore, add the commands
\begin{center}
\begin{tabular}{l}
|% \iffalse
%
% childdoc.dtx Copyright (C) 2017-2018 Niklas Beisert
%
% This work may be distributed and/or modified under the
% conditions of the LaTeX Project Public License, either version 1.3
% of this license or (at your option) any later version.
% The latest version of this license is in
%   http://www.latex-project.org/lppl.txt
% and version 1.3 or later is part of all distributions of LaTeX
% version 2005/12/01 or later.
%
% This work has the LPPL maintenance status `maintained'.
%
% The Current Maintainer of this work is Niklas Beisert.
%
% This work consists of the files childdoc.dtx and childdoc.ins
% and the derived files childdoc.def and cdocsamp.tex with
% cdocsch1.tex, cdocsch2.tex, cdocsdrf.tex, cdocsfn1.tex, cdocsfn2.tex.
%
%<package>\ifdefined\childdocmain\endinput\fi
%<package>\ProvidesFile{childdoc.def}[2018/12/30 v2.0 child document driver]
%<samplemain>\ProvidesFile{cdocsamp.tex}[2018/12/30 v2.0 sample for childdoc]
%<*driver>
%\ProvidesFile{childdoc.drv}[2018/12/30 v2.0 childdoc reference manual file]
\PassOptionsToClass{10pt,a4paper}{article}
\documentclass{ltxdoc}

\usepackage[margin=35mm]{geometry}
\usepackage{hyperref}
\usepackage{hyperxmp}
\usepackage[usenames]{color}

\hypersetup{colorlinks=true}
\hypersetup{pdfstartview=FitH}
\hypersetup{pdfpagemode=UseNone}
\hypersetup{pdfsource={}}
\hypersetup{pdflang={en-UK}}
\hypersetup{pdfcopyright={Copyright 2017-2018 Niklas Beisert.
  This work may be distributed and/or modified under the
  conditions of the LaTeX Project Public License, either version 1.3
  of this license or (at your option) any later version.}}
\hypersetup{pdflicenseurl={http://www.latex-project.org/lppl.txt}}
\hypersetup{pdfcontactaddress={ETH Zurich, ITP, HIT K,
  Wolfgang-Pauli-Strasse 27}}
\hypersetup{pdfcontactpostcode={8093}}
\hypersetup{pdfcontactcity={Zurich}}
\hypersetup{pdfcontactcountry={Switzerland}}
\hypersetup{pdfcontactemail={nbeisert@itp.phys.ethz.ch}}
\hypersetup{pdfcontacturl={http://people.phys.ethz.ch/\xmptilde nbeisert/}}

\newcommand{\secref}[1]{\hyperref[#1]{section \ref*{#1}}}

\parskip1ex
\parindent0pt
\let\olditemize\itemize
\def\itemize{\olditemize\parskip0pt}

\begin{document}

\title{The \textsf{childdoc} Package}
\hypersetup{pdftitle={The childdoc Package}}
\author{Niklas Beisert\\[2ex]
  Institut f\"ur Theoretische Physik\\
  Eidgen\"ossische Technische Hochschule Z\"urich\\
  Wolfgang-Pauli-Strasse 27, 8093 Z\"urich, Switzerland\\[1ex]
  \href{mailto:nbeisert@itp.phys.ethz.ch}
  {\texttt{nbeisert@itp.phys.ethz.ch}}}
\hypersetup{pdfauthor={Niklas Beisert}}
\hypersetup{pdfsubject={Manual for the LaTeX2e Package childdoc}}
\date{30 December 2018, \textsf{v2.0}}
\maketitle

\begin{abstract}\noindent
\textsf{childdoc} is a \LaTeXe{} package
that enables the direct compilation
of document sections included by |\include|
to individual files.
\end{abstract}

\begingroup
\parskip0ex
\tableofcontents
\endgroup

%%%%%%%%%%%%%%%%%%%%%%%%%%%%%%%%%%%%%%%%%%%%%%%%%%%%%%%%%%%%%%%%%%%%%%%%%%%%%%%%
%%%%%%%%%%%%%%%%%%%%%%%%%%%%%%%%%%%%%%%%%%%%%%%%%%%%%%%%%%%%%%%%%%%%%%%%%%%%%%%%
\section{Introduction}

\LaTeX{} provides a mechanism to structure a large document (such as a book)
into a main file and several child files (containing the chapters)
using the |\include| command.
This mechanism is beneficial for documents
which span hundreds of pages in order to
make the source file(s) more manageable.
Moreover, compilation can be restricted to
selected child files by means of the |\includeonly| command.
The latter feature can be used to reduce the compilation time while editing
(this was significantly more useful in the earlier days of \LaTeX{})
or to generate a smaller document which is easier to navigate.
Another application of |\includeonly| is to generate
documents consisting of selected parts of the complete document.

However, there are a few drawbacks of the plain |\include| mechanism:
\begin{itemize}
\item
The child files cannot be compiled on their own,
they can only be compiled via the main file.
A naive editing environment
(such as a text editor with an option
to have the current file processed by \LaTeX)
may require one to switch to the main file before compiling;
attempting to compile the child file produces errors.
\item
The main file must be modified (each time)
to adjust the |\includeonly| command
to the present needs. This easily leaves the main file in a messy state.
\item
The generated document will always carry the filename
of the main document. This is inconvenient if
several child files are to be compiled and
to be kept for distribution.
\end{itemize}

The present package provides a simple interface
to make child files individually compilable by \LaTeX{}.
Compiling a child file then has the same effect as compiling
the main file with an |\includeonly| command
to select the appropriate child.
Moreover the generated document will carry the name of the child
rather than the main file.
This resolves all three above issues.

This feature is meant to make the editing of books,
thesis documents and lecture notes somewhat more convenient.
However, the package can also be used efficiently for
composing a series of documents (such as exercise sheets)
which are typically distributed individually.
It then assists the author in generating the individual documents
(potentially in different versions)
as well as a document containing the collected series.
Another application is in developing style files
or other kinds of included material
where compilation of the style file could redirect
to a sample or test file.

%%%%%%%%%%%%%%%%%%%%%%%%%%%%%%%%%%%%%%%%%%%%%%%%%%%%%%%%%%%%%%%%%%%%%%%%%%%%%%%%
%%%%%%%%%%%%%%%%%%%%%%%%%%%%%%%%%%%%%%%%%%%%%%%%%%%%%%%%%%%%%%%%%%%%%%%%%%%%%%%%
\section{Usage}

First of all, the package \textsf{childdoc} is \emph{not} a standard
\LaTeXe{} |.sty| style file! Therefore it needs to be invoked in
a non-standard way.

%%%%%%%%%%%%%%%%%%%%%%%%%%%%%%%%%%%%%%%%%%%%%%%%%%%%%%%%%%%%%%%%%%%%%%%%%%%%%%%%
\subsection{Included Files}
\label{sec:include}

%%%%%%%%%%%%%%%%%%%%%%%%%%%%%%%%%%%%%%%%
\DescribeMacro{\childdocmain}
To use the package, add the commands
\begin{center}
\begin{tabular}{l}
|\input{childdoc.def}|\\
|\childdocmain{}|\\
\end{tabular}
\end{center}
at the very top of the main \LaTeX{} file,
in particular \emph{before} the |\documentclass| statement!
The argument of |\childdocmain| should be left empty
(but it must be present).

%%%%%%%%%%%%%%%%%%%%%%%%%%%%%%%%%%%%%%%%
\DescribeMacro{\childdocof}
Furthermore, add the commands
\begin{center}
\begin{tabular}{l}
|\input{childdoc.def}|\\
|\childdocof{|\textit{main}|}|\\
\end{tabular}
\end{center}
at the top of every child file \textit{child}
which is included by |\include{|\textit{child}|}|
from within the main file
(or at least for those files to be compiled individually).
The argument \textit{main} must be the filename of the main file.

There are a couple of
considerations in setting up the main and child documents:

%%%%%%%%%%%%%%%%%%%%%%%%%%%%%%%%%%%%%%%%
\paragraph{Restrictions.}

Please note the following restrictions:
\begin{itemize}
\item
|\childdocmain| must be called with one argument \textit{main}
to ensure compatibility with earlier version of the package.
It must either be empty (|\childdocmain{}|)
or precisely match the filename of the main file in which it is specified.
See \secref{sec:detection} for further information.
\item
The filename \textit{main} must be specified without the |.tex| extension.
\item
The filename \textit{main} is case sensitive
(even in case-insensitive file systems)
due to internal string comparison.
\item
The argument \textit{main} should be fully expanded, it cannot be a macro.
\item
Subdirectories and special characters should be avoided in filenames.
\item
The command |\childdocmain{|\textit{main}|}| must be followed by a whitespace.
It should not be followed immediately by another command
or by a comment mark `|%|'.
This is because the \TeX{} parser reads the token immediately following
the argument of |\childdocmain| and puts it
at the beginning of every child section;
however, a white\-space is ignored.
\end{itemize}

%%%%%%%%%%%%%%%%%%%%%%%%%%%%%%%%%%%%%%%%
\paragraph{Content of Main File.}

It is advisable to place all content in the child files included by |\include|.
Any output contained in the main file will appear in all child documents
unless suppressed manually;
it cannot be suppressed automatically by the |\includeonly| directive
and thus should normally be avoided.
A method to include some content in the main file
by means of conditional processing is described in \secref{sec:conditional}.

%%%%%%%%%%%%%%%%%%%%%%%%%%%%%%%%%%%%%%%%
\paragraph{Page Numbering.}

When only a part of the document is compiled,
the appropriate numbering of pages
(as well as other status parameters)
is determined from the |.aux| files.
The latter contain information from previous passes.
However this information needs to propagate through
all intermediate child documents.
Therefore the page numbering in child documents may well
be inconsistent until the complete document is compiled at least once.

A useful (if unconventional) way to always ensure a consistent
page numbering is to restart the numbering in each child document
and denote the pages by `\textit{child}|.|\textit{page}'
where \textit{child} represents the chapter/section number of the child file.
This can be achieved by the command
|\numberwithin{page}{|\textit{child}|}|
of the \textsf{amsmath} package
where \textit{child} can be |chapter| or |section|
depending on the chosen structuring.
Alternatively, one can modify the macro |\thepage| appropriately
and reset the counter |page| at the start of each child file.

%%%%%%%%%%%%%%%%%%%%%%%%%%%%%%%%%%%%%%%%%%%%%%%%%%%%%%%%%%%%%%%%%%%%%%%%%%%%%%%%
\subsection{Conditional Processing}
\label{sec:conditional}

The package provides a mechanism to compile different versions
of a document. To customise the versions further some conditional processing
can come in handy to distinguish which version is being compiled.
The package provides two macros to describe the compilation context:

%%%%%%%%%%%%%%%%%%%%%%%%%%%%%%%%%%%%%%%%
\DescribeMacro{\ifchilddoc}
The conditional |\ifchilddoc| distinguishes between the compilation of
child documents and the main document:
%
\begin{center}
|\ifchilddoc |\textit{child-code}| |[|\||else |\textit{main-code}]| \||fi|
\end{center}

%%%%%%%%%%%%%%%%%%%%%%%%%%%%%%%%%%%%%%%%
\DescribeMacro{\childdocname}
\DescribeMacro{\childdocjob}
The macro |\childdocname| contains the filename (without extension)
of the main or child file being processed.
Note that |\childdocjob| will always contain the name of the main file.

%%%%%%%%%%%%%%%%%%%%%%%%%%%%%%%%%%%%%%%%
\paragraph{Title Page.}

Conditional processing can be used to include a title or banner page
in the main document when proper precautions are taken.
Importantly, the code in the main file should ensure that the page counter
(as well as other status parameters which are stored in the |.aux| files)
takes the same value after the conditional processing.
Otherwise the page numbers may take divergent values
depending on which part is compiled.

For example, a title page could be declared by:
%
\begin{center}
\begin{tabular}{l}
|\ifchilddoc\||else|\\
|\addtocounter{page}{-1}|\\
\textit{code for title page}\\
|\newpage|\\
|\||fi|
\end{tabular}
\end{center}
%
A banner page for the child documents can be generated by:
%
\begin{center}
\begin{tabular}{l}
|\ifchilddoc|\\
|\addtocounter{page}{-1}|\\
\textit{code for banner page}\\
|\newpage|\\
|\||fi|
\end{tabular}
\end{center}
%
Here one could write a message such as:
\begin{center}
|This is the part \childdocname{} of \childdocjob{}.|
\end{center}

%%%%%%%%%%%%%%%%%%%%%%%%%%%%%%%%%%%%%%%%%%%%%%%%%%%%%%%%%%%%%%%%%%%%%%%%%%%%%%%%
\subsection{Flags}
\label{sec:flags}

The package makes it easy to generate different versions
of the main or child documents.
To this end compilation flags can be defined
and assigned different default values.
They will be particularly useful in conjunction
with the forwarding mechanism described in \secref{sec:forward}.

For example, it may be useful to have a flag |\version|
which can be set to |draft| or |final|.
The document source will contain some conditional code
depending on the value of |\version|.
Suppose further, the flag should default to |final| for the main file
and to |draft| for child files
which is a natural assignment for editing the document.
This is achieved by placing the following code
in the preamble of the main document
(below the |\childdocmain| directive):
%
\begin{center}
\begin{tabular}{l}
|\ifchilddoc|\\
|\providecommand{\version}{draft}|\\
|\||else|\\
|\providecommand{\version}{final}|\\
|\||fi|
\end{tabular}
\end{center}
%
The definition by |\providecommand| makes sure
that previous definitions are not overwritten.
Further statements |\providecommand{\version}{...}|
can thus be added before the above code to override it.

For the main file, one might add a line
(between |\childdocmain| and the above block)
%
\begin{center}
|%\ifchilddoc\||else\providecommand{\version}{draft}\||fi|
\end{center}
%
which can be uncommented to produce a draft version.
Likewise one can add a line to the very top of a child file
(above the |\childdocof{|\textit{main}|}| directive)
%
\begin{center}
|%\providecommand{\version}{final}|
\end{center}
%
which can be uncommented to produce the final version of this child document.

%%%%%%%%%%%%%%%%%%%%%%%%%%%%%%%%%%%%%%%%%%%%%%%%%%%%%%%%%%%%%%%%%%%%%%%%%%%%%%%%
\subsection{Forwarding}
\label{sec:forward}

Different versions of the main or child documents
using compilation flags as described in \secref{sec:flags}
can be (permanently) stored in different files
for convenient compilation, viewing and distribution.
To this end, the package defines a command
to pass on compilation to a different file:

%%%%%%%%%%%%%%%%%%%%%%%%%%%%%%%%%%%%%%%%
\DescribeMacro{\childdocforward}
The command |\childdocforward| redirects processing to
another source file:
%
\begin{center}
\begin{tabular}{l}
|\input{childdoc.def}|\\
|\childdocforward[|\textit{main}|]{|\textit{dest}|}|\\
\end{tabular}
\end{center}
%
The argument \textit{dest} is the destination file
(without extension).
It should be the main file or one of the child files.
Note that further \textsf{childdoc} directives
such as |\childdocof| and |\childdocforward|
in the indicated file will be processed in this form.
The optional argument \textit{main}
passes on directly to the main file \textit{main}
while pretending to compile the child \textit{dest}.
This form behaves as if \textit{dest}
issues |\childdocof{|\textit{main}|}| right away,
and no further \textsf{childdoc} directives will be processed.

%%%%%%%%%%%%%%%%%%%%%%%%%%%%%%%%%%%%%%%%
\DescribeMacro{\...prefix}
In the alternative form |\childdocforwardprefix|,
%
\begin{center}
\begin{tabular}{l}
|\input{childdoc.def}|\\
|\childdocforwardprefix[|\textit{main}|]{|\textit{prefix}|}{|\textit{dest}|}|
\end{tabular}
\end{center}
%
the destination file is determined by a pattern
depending on the current file:
To make this work, the current file must be called
`{\textit{prefix}\hspace{0.2em}\textit{suffix}}'
with \textit{prefix} matching precisely the argument.
Processing is then passed on to the file
`{\textit{dest}\hspace{0.2em}\textit{suffix}}'.
Surely, the same effect is achieved by
directly specifying the
argument `{\textit{dest}\hspace{0.2em}\textit{suffix}}'
in the first form.
However, that requires to set up a different file
for each child. With the alternative form of the command
all these files can have exactly the same content
which simplifies setting them up and maintaining them.

For example, the following file |draft.tex|
with a compilation flag |\version| as described in \secref{sec:flags}
compiles the main document as a draft:
%
\begin{center}
\begin{tabular}{l}
|\def\version{draft}|\\
|\input{childdoc.def}|\\
|\childdocforward{|\textit{main}|}|
\end{tabular}
\end{center}
%
Likewise, the following files |final|\textit{nn}|.tex|
compile the final version of the child document
|child|\textit{nn}|.tex|:
%
\begin{center}
\begin{tabular}{l}
|\def\version{final}|\\
|\input{childdoc.def}|\\
|\childdocforwardprefix{final}{child}|
\end{tabular}
\end{center}
%

Note that when several versions of a main file and/or of each child file
are to be generated, it may be convenient to set up a |Makefile| or
shell script to automatise the process.

%%%%%%%%%%%%%%%%%%%%%%%%%%%%%%%%%%%%%%%%%%%%%%%%%%%%%%%%%%%%%%%%%%%%%%%%%%%%%%%%
\subsection{Command Line Processing}
\label{sec:commandline}

The effect of redirection files can also be achieved by invoking
the \LaTeX{} compiler with a more elaborate command line.
Most conveniently this should be done as part
of a shell script or a |Makefile|.

When using \textsf{childdoc} in the main file, the following
command lines effectively perform a redirection
(note that depending on the shell being used,
backslashes may have to be doubled: `|\|' $\to$ `|\\|'):
%
\begin{center}
|... -jobname "|\textit{target}|" |\\|"|[\textit{flags}]%
|\input{childdoc.def}\childdocforward[|\textit{main}|]{|\textit{dest}|}"|
\end{center}
%
Here \textit{target} is the name of the output file,
\textit{main} is the name of the main file
and \textit{dest} is the name of the main or child file to be processed
(all filenames without extensions).
The optional argument \textit{main} can be omitted
if \textit{main} matches \textit{dest}.
Optionally, compilation \textit{flags} can be defined via |\def| commands.
This command line makes the \TeX{} engine believe
it is compiling the file \textit{target}
whose content is specified as the latter parameter.
The provided code then forwards the processing to
\textit{main} or \textit{dest} as described in \secref{sec:forward}.

%%%%%%%%%%%%%%%%%%%%%%%%%%%%%%%%%%%%%%%%%%%%%%%%%%%%%%%%%%%%%%%%%%%%%%%%%%%%%%%%
\subsection{Include by Input}
\label{sec:input}

Including child documents by |\include| has some restrictions by design.
Most notably, the content of a child document always occupies
its own set of pages; pages cannot be shared between child documents.
Usually, this behaviour makes perfect sense
because each child document contain an essential part of the document.
However, in some situations it may be desirable to compose
a document from a collection of parts
without having mandatory page breaks between then.
For this case, the package
provides a mechanism to include parts
by |\input| which can also be processed individually.
However, by construction this mechanism
requires manual handling of the content to be output.

%%%%%%%%%%%%%%%%%%%%%%%%%%%%%%%%%%%%%%%%
\DescribeMacro{\ifchilddocmanual}
The main file should be prepared as usual, see \secref{sec:include}.
However, the document body must make a distinction
between processing of an individual part and of the main document, e.g.:
%
\begin{center}
\begin{tabular}{l}
|\ifchilddocmanual|\\
|\input{\childdocname}|\\
|\||else|\\
\textit{document body with }|\input{|\textit{part}|}|\\
|\||fi|
\end{tabular}
\end{center}
%
The conditional |\ifchilddocmanual| is true whenever
a part to be included by |\input| is being compiled,
and the name of the part is stored in |\childdocname|.

%%%%%%%%%%%%%%%%%%%%%%%%%%%%%%%%%%%%%%%%
\DescribeMacro{\childdocby}
Each part to be included by |\input| should start with:
%
\begin{center}
\begin{tabular}{l}
|\input{childdoc.def}|\\
|\childdocby{|\textit{main}|}|\\
\end{tabular}
\end{center}
%
The directive |\childdocby| is similar to |\childdocof|
described in \secref{sec:include},
but the subsequent selection of content must be done manually.
To that end, both |\ifchilddoc| and |\ifchilddocmanual|
will be true upon processing of a part,
and the name of the part is stored in |\childdocname|.
Note that |\jobname| will be set to the filename of the current part
so that each part receives an individual |.aux| file
that does not interfere with the |.aux| file(s) of the main document.
This behaviour can be altered by the alternative form
|\childdocby[*]{|\textit{main}|}| (with a non-empty optional argument)
which uses the |.aux| file of the main document
by setting |\jobname| to \textit{main}.

%%%%%%%%%%%%%%%%%%%%%%%%%%%%%%%%%%%%%%%%%%%%%%%%%%%%%%%%%%%%%%%%%%%%%%%%%%%%%%%%
\subsection{Driver Development}
\label{sec:driver}

The \textsf{childdoc} mechanism can also be use for the development
of definition files such as \LaTeX{} styles or classes.
This case differs from the above setup with multiple parts
included by |\include| in that no |\includeonly| should be invoked.
This can be achieved by starting the include file
(before |\ProvidesPackage|) with:
%
\begin{center}
\begin{tabular}{l}
|\input{childdoc.def}|\\
|\childdocforward{|\textit{main}|}|\\
\end{tabular}
\end{center}
%
or alternatively with:
%
\begin{center}
\begin{tabular}{l}
|\input{childdoc.def}|\\
|\childdocby{|\textit{main}|}|\\
\end{tabular}
\end{center}
%
Both forms have slightly different effects as described above.
The main file is prepared as usual, see \secref{sec:include}.

%%%%%%%%%%%%%%%%%%%%%%%%%%%%%%%%%%%%%%%%%%%%%%%%%%%%%%%%%%%%%%%%%%%%%%%%%%%%%%%%
\subsection{Legacy Detection}
\label{sec:detection}

The directive |\childdocmain| in the main file can detect
whether the complete document or merely a child is to be compiled
even without using the directive |\childdocof|.
This method is deprecated because it is less robust
and there is no compelling reason to use it;
it is merely provided for backward compatibility
and it may be removed in future versions.

If the detection mechanism is to be used,
it is mandatory to correctly specify
the filename of the main file as the argument of |\childdocmain|:
%
\begin{center}
\begin{tabular}{l}
|\input{childdoc.def}|\\
|\childdocmain{|\textit{main}|}|\\
\end{tabular}
\end{center}
%
If |\jobname| does not match the argument \textit{main} of |\childdocmain|,
it is assumed that |\jobname| points to the child file to be compiled.
When using |\childdocmain| with the main file specified as argument,
it suffices to start a child file
with just |\input{|\textit{main}|}|
without loading of the package and using |\childdocof|.
If instead all processing is done
with the appropriate \textsf{childdoc} directives,
the argument of \textit{main} of |\childdocmain| can be empty.

An alternative version of the command line processing described
in \secref{sec:commandline} using the detection mechanism reads:
%
\begin{center}
|... -jobname "|\textit{target}|" "|[\textit{flags}]%
[|\def\jobname{|\textit{dest}|}|]|\input{|\textit{main}|}"|
\end{center}

%%%%%%%%%%%%%%%%%%%%%%%%%%%%%%%%%%%%%%%%%%%%%%%%%%%%%%%%%%%%%%%%%%%%%%%%%%%%%%%%
\subsection{Manual Code}
\label{sec:manual}

In case one cannot be certain whether the definitions file |childdoc.def|
is installed on the target \TeX{} distribution
and one prefers not to ship it,
it is conceivable to paste a few relevant commands into the sources.

To that end, drop all statements |\input{childdoc.def}|
and perform the replacements as outlined below.
Instead of |\childdocmain{|\textit{main}|}| add the following code
to the top of the main file:
%
\begin{center}
\begin{tabular}{l}
|\||ifdefined\childdocname\endinput\||fi\newif\ifchilddoc|\\
|\edef\childdocname{\scantokens\expandafter{\jobname\noexpand}}|\\
|\def\childdocmain{|\textit{main}|}\||ifx\childdocmain\childdocname\||else|\\
|\childdoctrue\includeonly{\childdocname}\let\jobname\childdocmain\||fi|\\
\end{tabular}
\end{center}
%
Instead of |\childdocof{|\textit{main}|}| just include the main file
at the top of each child file:
%
\begin{center}
|\input{|\textit{main}|}|
\end{center}
%
A simple redirection |\childdocforward{|\textit{dest}|}| is achieved by:
%
\begin{center}
|\def\jobname{|\textit{dest}|}\input{\jobname}|
\end{center}
%
The redirection with prefix
|\childdocforwardprefix[|\textit{prefix}|]{|\textit{dest}|}|
is accomplished by:
%
\begin{center}
\begin{tabular}{l}
|{\edef\jobname{\scantokens\expandafter{\jobname\noexpand}}|\\
|\def\redirectjob |\textit{prefix}|#1~~~{\gdef\jobname{|\textit{dest}|#1}}|\\
|\expandafter\redirectjob\jobname~~~}\input{\jobname}|
\end{tabular}
\end{center}

In an alternative approach,
child documents can be compiled by a specific command line
without additional code or specific definitions:
%
\begin{center}
|... -jobname "|\textit{target}|" "|[\textit{flags}]%
|\includeonly{|\textit{dest}|}\input{|\textit{main}|}"|
\end{center}
%

%%%%%%%%%%%%%%%%%%%%%%%%%%%%%%%%%%%%%%%%%%%%%%%%%%%%%%%%%%%%%%%%%%%%%%%%%%%%%%%%
%%%%%%%%%%%%%%%%%%%%%%%%%%%%%%%%%%%%%%%%%%%%%%%%%%%%%%%%%%%%%%%%%%%%%%%%%%%%%%%%
\section{Information}

%%%%%%%%%%%%%%%%%%%%%%%%%%%%%%%%%%%%%%%%%%%%%%%%%%%%%%%%%%%%%%%%%%%%%%%%%%%%%%%%
\subsection{Copyright}

Copyright \copyright{} 2017--2018 Niklas Beisert

This work may be distributed and/or modified under the
conditions of the \LaTeX{} Project Public License, either version 1.3
of this license or (at your option) any later version.
The latest version of this license is in
  \url{http://www.latex-project.org/lppl.txt}
and version 1.3 or later is part of all distributions of \LaTeX{}
version 2005/12/01 or later.

This work has the LPPL maintenance status `maintained'.

The Current Maintainer of this work is Niklas Beisert.

This work consists of the files |README.txt|, |childdoc.ins| and |childdoc.dtx|
as well as the derived files |childdoc.def|, |cdocsamp.tex|
with |cdocsch1.tex|, |cdocsch2.tex|, |cdocspt3.tex|, |cdocspt4.tex|,
|cdocsdrf.tex|, |cdocsfn1.tex|, |cdocsfn2.tex|
as well as |childdoc.pdf|.

%%%%%%%%%%%%%%%%%%%%%%%%%%%%%%%%%%%%%%%%%%%%%%%%%%%%%%%%%%%%%%%%%%%%%%%%%%%%%%%%
\subsection{Files and Installation}

The package consists of the files:
%
\begin{center}
\begin{tabular}{ll}
    |README.txt|   & readme file \\
    |childdoc.ins| & installation file \\
    |childdoc.dtx| & source file \\
    |childdoc.def| & definition file \\
    |cdocsamp.tex| & sample main file \\
    |cdocsch1.tex| & sample include file \\
    |cdocsch2.tex| & sample include file \\
    |cdocspt3.tex| & sample part file \\
    |cdocspt4.tex| & sample part file \\
    |cdocsdrf.tex| & sample redirection file \\
    |cdocsfn1.tex| & sample redirection file \\
    |cdocsfn2.tex| & sample redirection file \\
    |childdoc.pdf| & manual
\end{tabular}
\end{center}
%
The distribution consists of the files
|README.txt|, |childdoc.ins| and |childdoc.dtx|.
%
\begin{itemize}
\item
Run (pdf)\LaTeX{} on |childdoc.dtx|
to compile the manual |childdoc.pdf| (this file).
\item
Run \LaTeX{} on |childdoc.ins| to create the definitions file |childdoc.def|
and the sample |cdocsamp.tex| with include files
|cdocsch1.tex|, |cdocsch2.tex|, |cdocspt3.tex|, |cdocspt4.tex|,
|cdocsdrf.tex|, |cdocsfn1.tex|, |cdocsfn2.tex|.
Then copy the file |childdoc.def| to an appropriate directory of your \LaTeX{}
distribution, e.g.\ \textit{texmf-root}|/tex/latex/childdoc|.
\end{itemize}

%%%%%%%%%%%%%%%%%%%%%%%%%%%%%%%%%%%%%%%%%%%%%%%%%%%%%%%%%%%%%%%%%%%%%%%%%%%%%%%%
\subsection{Related CTAN Packages}

There are several other packages which offer a similar functionality:
%
\begin{itemize}
\item
The packages
\href{http://ctan.org/pkg/docmute}{\textsf{docmute}},
\href{http://ctan.org/pkg/includex}{\textsf{includex}} and
\href{http://ctan.org/pkg/standalone}{\textsf{standalone}}
provide commands to include only the document body of
a child file thus allowing both files to be compiled individually.
\item
The packages \href{http://ctan.org/pkg/subdocs}{\textsf{subdocs}}
and \href{http://ctan.org/pkg/subfiles}{\textsf{subfiles}}
provide structures in which the main and child documents can be
encapsulated and allowing them to be compiled individually.
The inclusion mechanism is different from the conventional |\include|.
\item
The package \href{http://ctan.org/pkg/combine}{\textsf{combine}}
is an elaborate solution to combine several documents into one.
\end{itemize}
%
See also the CTAN topic \href{http://ctan.org/topic/subdocs}{\textsf{subdocs}}
for further related packages.
The present package differs from the above solutions in that
a document structure constructed with the conventional |\include| mechanism
just needs two extra commands at the top of every file
such that all constituent files can be compiled individually.

%%%%%%%%%%%%%%%%%%%%%%%%%%%%%%%%%%%%%%%%%%%%%%%%%%%%%%%%%%%%%%%%%%%%%%%%%%%%%%%%
%\subsection{Feature Suggestions}
%
%The following is a list of features which may be useful for future
%versions of this package:
%%
%\begin{itemize}
%\item
%\ldots
%\end{itemize}

%%%%%%%%%%%%%%%%%%%%%%%%%%%%%%%%%%%%%%%%%%%%%%%%%%%%%%%%%%%%%%%%%%%%%%%%%%%%%%%%
\subsection{Revision History}

%%%%%%%%%%%%%%%%%%%%%%%%%%%%%%%%%%%%%%%%
\paragraph{v2.0:} 2018/12/30

\begin{itemize}
\item
immediate forward processing
\item
added |\childdocby| mechanism
\item
manual restructured
\end{itemize}

%%%%%%%%%%%%%%%%%%%%%%%%%%%%%%%%%%%%%%%%
\paragraph{v1.6:} 2018/01/17

\begin{itemize}
\item
application for development of include files
\item
corrections to manual
\end{itemize}

%%%%%%%%%%%%%%%%%%%%%%%%%%%%%%%%%%%%%%%%
\paragraph{v1.5:} 2017/05/21

\begin{itemize}
\item
more complete structuring introduced
\item
|\childdocof| introduced
\item
|\childdoc| renamed to |\childdocmain|
\item
|\childredirect| renamed to |\childdocforward| and |\childdocforwardprefix|
and functionality expanded
\end{itemize}

%%%%%%%%%%%%%%%%%%%%%%%%%%%%%%%%%%%%%%%%
\paragraph{v1.0:} 2017/04/27

\begin{itemize}
\item
manual and install package
\item
first version published on CTAN
\end{itemize}

%%%%%%%%%%%%%%%%%%%%%%%%%%%%%%%%%%%%%%%%
\paragraph{v0.6:} 2017/04/26

\begin{itemize}
\item
redirection mechanism added
\end{itemize}

%%%%%%%%%%%%%%%%%%%%%%%%%%%%%%%%%%%%%%%%
\paragraph{v0.5:} 2017/04/26

\begin{itemize}
\item
functionality in definition file
\end{itemize}


%%%%%%%%%%%%%%%%%%%%%%%%%%%%%%%%%%%%%%%%%%%%%%%%%%%%%%%%%%%%%%%%%%%%%%%%%%%%%%%%
%%%%%%%%%%%%%%%%%%%%%%%%%%%%%%%%%%%%%%%%%%%%%%%%%%%%%%%%%%%%%%%%%%%%%%%%%%%%%%%%
%%%%%%%%%%%%%%%%%%%%%%%%%%%%%%%%%%%%%%%%%%%%%%%%%%%%%%%%%%%%%%%%%%%%%%%%%%%%%%%%
\appendix

\settowidth\MacroIndent{\rmfamily\scriptsize 000\ }

 \DocInput{childdoc.dtx}

\end{document}
%</driver>
% \fi
%
% %%%%%%%%%%%%%%%%%%%%%%%%%%%%%%%%%%%%%%%%%%%%%%%%%%%%%%%%%%%%%%%%%%%%%%%%%%%%%%
% %%%%%%%%%%%%%%%%%%%%%%%%%%%%%%%%%%%%%%%%%%%%%%%%%%%%%%%%%%%%%%%%%%%%%%%%%%%%%%
% \section{Sample}
%\iffalse
%<*samplemain>
%\fi
%
% The following presents a sample document
% with two chapters, two parts, a title page,
% a compile flag as well as three forwarding files to set the flag.
% It consists of eight |.tex| files:
% \begin{center}
% \begin{tabular}{ll}
% |cdocsamp.tex|&main file\\
% |cdocsch1.tex|&include file for chapter 1\\
% |cdocsch2.tex|&include file for chapter 2\\
% |cdocspt3.tex|&include file for part 3\\
% |cdocspt4.tex|&include file for part 4\\
% |cdocsdrf.tex|&forwarding file for main file in draft mode\\
% |cdocsfi1.tex|&forwarding file for final version of chapter 1\\
% |cdocsfi2.tex|&forwarding file for final version of chapter 2\\
% \end{tabular}
% \end{center}
% Each of the eight files can be compiled directly by the \LaTeX{} compiler.
%
% %%%%%%%%%%%%%%%%%%%%%%%%%%%%%%%%%%%%%%
% \paragraph{Main File.}
%
% The main file is called |cdocsamp.tex|.
%
% Load the \textsf{childdoc} definitions and
% declare the filename for the main document:
%    \begin{macrocode}
\input{childdoc.def}
\childdocmain{}
%    \end{macrocode}

% Optional override for |\version| flag:
%    \begin{macrocode}
%%\ifchilddoc\else\providecommand{\version}{draft}\fi
%    \end{macrocode}

% Define the default values for the |\version| flag
% (|final| for the main file and |draft| for childs):
%    \begin{macrocode}
\ifchilddoc
\providecommand{\version}{draft}
\else
\providecommand{\version}{final}
\fi
%    \end{macrocode}

% Load the standard document class:
%    \begin{macrocode}
\documentclass[12pt]{article}
%    \end{macrocode}

% Start the document body:
%    \begin{macrocode}
\begin{document}
%    \end{macrocode}

% Declare a title page.
% Print title, part of document being processed and version flag:
%    \begin{macrocode}
\addtocounter{page}{-1}
\begin{center}
{\LARGE\bfseries{}childdoc example\par}
\vspace{1cm}
\ifchilddoc
\ifchilddocmanual part\else chapter\fi:
`\childdocname' of `\childdocjob'\par
\else
main document: `\childdocjob'\par
\fi
version: \version\par
\end{center}
\newpage
%    \end{macrocode}

% Manually include selected file,
% otherwise process as usual:
%    \begin{macrocode}
\ifchilddocmanual
\section*{part `\childdocname'}
\input{\childdocname}
\else
%    \end{macrocode}

% Include the two chapters:
%    \begin{macrocode}
\include{cdocsch1}
\include{cdocsch2}
%    \end{macrocode}

% Include the two parts unless only chapters should be displayed:
%    \begin{macrocode}
\ifchilddoc\else
\section{part three}
\input{cdocspt3}
\section{part four}
\input{cdocspt4}
\fi
%    \end{macrocode}

% Process as usual until here:
%    \begin{macrocode}
\fi
%    \end{macrocode}

% End of document body:
%    \begin{macrocode}
\end{document}
%    \end{macrocode}
%\iffalse
%</samplemain>
%\fi
%
% %%%%%%%%%%%%%%%%%%%%%%%%%%%%%%%%%%%%%%
% \paragraph{Chapter Include Files.}
%
% The include files are called |cdocsch1.tex| and |cdocsch2.tex|.
%
%\iffalse
%<*samplechap1|samplechap2>
%\fi

% Optional override for |\version| flag:
%    \begin{macrocode}
%%\providecommand{\version}{final}
%    \end{macrocode}

% Include the main document:
%    \begin{macrocode}
\input{childdoc.def}
\childdocof{cdocsamp}
%    \end{macrocode}

%\iffalse
%</samplechap1|samplechap2>
%\fi
%
%\iffalse
%<*samplechap1>
%\fi
% Some text for chapter 1:
%    \begin{macrocode}
\section{one}
some text in chapter one
%    \end{macrocode}

%\iffalse
%</samplechap1>
%\fi
% Some text for chapter 2:
%\iffalse
%<*samplechap2>
%\fi
%    \begin{macrocode}
\section{two}
more text in chapter two
%    \end{macrocode}

%\iffalse
%</samplechap2>
%\fi
%
% %%%%%%%%%%%%%%%%%%%%%%%%%%%%%%%%%%%%%%
% \paragraph{Part Include Files.}
%
% The include files are called |cdocspt3.tex| and |cdocspt4.tex|.
%
%\iffalse
%<*samplepart3|samplepart4>
%\fi

% Optional override for |\version| flag:
%    \begin{macrocode}
%%\providecommand{\version}{final}
%    \end{macrocode}

% Include the main document:
%    \begin{macrocode}
\input{childdoc.def}
\childdocby{cdocsamp}
%    \end{macrocode}

%\iffalse
%</samplepart3|samplepart4>
%\fi
%
%\iffalse
%<*samplepart3>
%\fi
% Some text for part 3:
%    \begin{macrocode}
some text in part three
%    \end{macrocode}

%\iffalse
%</samplepart3>
%\fi
% Some text for part 4:
%\iffalse
%<*samplepart4>
%\fi
%    \begin{macrocode}
more text in part four
%    \end{macrocode}

%\iffalse
%</samplepart4>
%\fi
%
% %%%%%%%%%%%%%%%%%%%%%%%%%%%%%%%%%%%%%%
% \paragraph{Forwarding for a Complete Draft.}
%
% The following forwarding file |cdocsdrf.tex|
% compiles the main document in draft mode:
%\iffalse
%<*sampledraft>
%\fi
%    \begin{macrocode}
\def\version{draft}
\input{childdoc.def}
\childdocforward{cdocsamp}
%    \end{macrocode}

%\iffalse
%</sampledraft>
%\fi
%
% %%%%%%%%%%%%%%%%%%%%%%%%%%%%%%%%%%%%%%
% \paragraph{Forwarding for Final Version of the Chapters.}
%
% The following forwarding files |cdocsfn1.tex| and |cdocsfn2.tex|
% (with identical content)
% compile the final versions of the child documents
% |cdocsch1.tex| and |cdocsch2.tex|, respectively:
%\iffalse
%<*samplefinal>
%\fi
%    \begin{macrocode}
\def\version{final}
\input{childdoc.def}
\childdocforwardprefix[cdocsamp]{cdocsfn}{cdocsch}
%    \end{macrocode}

%\iffalse
%</samplefinal>
%\fi
%
% %%%%%%%%%%%%%%%%%%%%%%%%%%%%%%%%%%%%%%
% \paragraph{Command Line Processing.}
%
% The following three command lines generate the output files
% |cdocscld|, |cdocscl1| and |cdocscl2|
% which should be identical to
% |cdocsdrf|, |cdocsch1| and |cdocsfn2|, respectively:
% \begin{center}
% \begin{tabular}{l}
% |latex -jobname cdocscld \|\\
% |  "\def\version{draft}\input{childdoc.def}\childdocforward{cdocsamp}"|\\
% |latex -jobname cdocscl1 \|\\
% |  "\input{childdoc.def}\childdocforward[cdocsamp]{cdocsch1}"|\\
% |latex -jobname cdocscl2 \|\\
% |  "\def\version{final}\input{childdoc.def}\childdocforward{cdocsch2}"|
% \end{tabular}
% \end{center}
% Note that the trailing backslash on each first line
% merely continues the input to the second line
% (for convenient cut ant paste).
% Furthermore, the command |latex| can be replaced by any
% of its alternative versions such as |pdflatex|.
%
% %%%%%%%%%%%%%%%%%%%%%%%%%%%%%%%%%%%%%%%%%%%%%%%%%%%%%%%%%%%%%%%%%%%%%%%%%%%%%%
% %%%%%%%%%%%%%%%%%%%%%%%%%%%%%%%%%%%%%%%%%%%%%%%%%%%%%%%%%%%%%%%%%%%%%%%%%%%%%%
% \section{Implementation}
%\iffalse
%<*package>
%\fi
%
% This section describes the definitions file |childdoc.def|.

% The definitions cannot be loaded using |\usepackage| or |\RequirePackage|
% which has a mechanism to prevent loading a style file more than once.
% When loading the definitions by means of |\input|
% multiple instances have to be prevented manually:
%\iffalse
%This code needs to be before the `\ProvidesFile' directive
%which is defined at the beginning of this file.
%Therefore it is also placed there and commented out here.
%</package>
%<*discard>
%\fi
%    \begin{macrocode}
\ifdefined\childdocmain\endinput\fi
%    \end{macrocode}
%\iffalse
%</discard>
%<*package>
%\fi
%
% \macro{\ifchilddoc}
% \macro{\ifchilddocmanual}
% The conditional |\ifchilddoc| tells whether a
% child (true) or main (false) document is being compiled.
% The conditional |\ifchilddocmanual| tells whether
% the |\includeonly| mechanism is used (false) or
% the selection of child files must be performed manually (true).
% The definitions initialise to false:
%    \begin{macrocode}
\newif\ifchilddoc
\newif\ifchilddocmanual
%    \end{macrocode}

% \macro{\childdocname}
% \macro{\childdocjob}
% The macro |\childdocname| stores the name of the main document
% to be compiled. The macro |\childdocjob| stores the name of
% the document on which the \LaTeX{} compiler was originally invoked.
% The content of |\jobname| cannot be compared
% to filenames specified in the source due to different catcodes.
% The following code rescans |\jobname|, stores the result
% in |\childdocname| and saves a copy in |\childdocjob|:
%    \begin{macrocode}
\edef\childdocname{\scantokens\expandafter{\jobname\noexpand}}
\let\childdocjob\childdocname
%    \end{macrocode}

% \macro{\childdocdisable}
% The macro |\childdocdisable| prevents the main file
% from being processed more than once.
% At this stage, the main document command |\childdocmain|
% is assumed to be called once again where it should do nothing.
% Any subsequent call to it should prevent
% a secondary processing of the main document
% It overwrites the forwarding commands
% |\childdocof| and |\childdocforward|
% with empty macros to prevent further inclusions of the main document:
%    \begin{macrocode}
\newcommand{\childdocdisable}
{
  \renewcommand{\childdocmain}[1]{\renewcommand{\childdocmain}[1]{\endinput}}
  \renewcommand{\childdocof}[1]{}
  \renewcommand{\childdocby}[2][]{}
  \renewcommand{\childdocforward}[2][]{}
  \renewcommand{\childdocdisable}{}
}
%    \end{macrocode}

% \macro{\childdocmain}
% The macro |\childdocmain| is to be called at the top of the main file
% with nothing or the main filename (without extension) as argument.
% First, it breaks loops.
% If the argument is not empty and does not match |\childdocname|
% (which is set by the first inclusion of |childdoc.def|),
% |\ifchilddoc| is set to true, |\includeonly| is applied to the child file
% and |\jobname| is set to the main file
% (for proper handling of |.aux| files):
%    \begin{macrocode}
\newcommand{\childdocmain}[1]
{
  \childdocdisable\childdocmain{}
  \if?#1?\else
    \begingroup
      \def\childdoctmp{#1}
      \ifx\childdoctmp\childdocname
        \def\childdoctmp{}
      \else
        \def\childdoctmp
        {
          \childdoctrue
          \includeonly{\childdocname}
          \def\childdocjob{#1}
          \def\jobname{#1}
        }
      \fi
      \expandafter
    \endgroup
    \childdoctmp
  \fi
}
%    \end{macrocode}

% \macro{\childdocof}
% The command |\childdocof| redirects
% compilation to the main file |#1|.
%    \begin{macrocode}
\newcommand{\childdocof}[1]
{
  \childdocdisable
  \childdoctrue
  \includeonly{\childdocname}
  \def\jobname{#1}
  \def\childdocjob{#1}
  \input{#1}
}
%    \end{macrocode}

% \macro{\childdocby}
% The command |\childdocby| ....
%    \begin{macrocode}
\newcommand{\childdocby}[2][]
{
  \childdocdisable
  \childdoctrue
  \childdocmanualtrue
  \if?#1?\else
    \def\jobname{#2}
  \fi
  \def\childdocjob{#2}
  \input{#2}
  \endinput
}
%    \end{macrocode}

% \macro{\childdocforward}
% The command |\childdocforward| redirects
% compilation to the main file or
% (if the optional argument is given) a child file.
% Parameters are set as if the main file
% or a child file starting with |\childdocof| was compiled.
% Then compilation is handed over to the main file:
%    \begin{macrocode}
\newcommand{\childdocforward}[2][]
{
  \begingroup
    \if?#1?
      \def\childdoctmp
      {
        \def\childdocname{#2}
        \def\childdocjob{#2}
        \def\jobname{#2}
        \input{#2}
        \endinput
      }
    \else
      \def\childdoctmp
      {
        \childdocdisable
        \def\childdocname{#2}
        \childdoctrue
        \includeonly{#2}
        \def\childdocjob{#1}
        \def\jobname{#1}
        \input{#1}
        \endinput
      }
    \fi
    \expandafter
  \endgroup
  \childdoctmp
}
%    \end{macrocode}

% \macro{\childdocforwardprefix}
% The command |\childdocforwardprefix| redirects
% compilation to the main or a child file by means of a pattern.
% The prefix |#1| in the current filename is replaced by |#2|
% and the suffix of the current filename is kept
% (it is assumed that the filename does not contain the substring `|~~~|'
% which is used as a delimiter).
% Compilation is handed over to the new file by |\childdocforward|:
%    \begin{macrocode}
\newcommand{\childdocforwardprefix}[3][]
{
  \begingroup
    \def\childdocextract #2##1~~~{\def\childdoctmp{\childdocforward[#1]{#3##1}}}
    \expandafter\childdocextract\childdocname~~~
    \expandafter
  \endgroup
  \childdoctmp
}
%    \end{macrocode}

% \macro{\childdoc}
% The deprecated macro |\childdoc| is a legacy version of |\childdocmain|:
%    \begin{macrocode}
\newcommand{\childdoc}{\childdocmain}
%    \end{macrocode}

% \macro{\childdocredirect}
% The deprecated macro |\childdocredirect| is a legacy version
% of |\childdocforward| and |\childdocforwardprefix|:
%    \begin{macrocode}
\newcommand{\childdocredirect}[2][]
{
  \begingroup
    \if?#1?
      \def\childdoctmp{\childdocforward{#2}}
    \else
      \def\childdoctmp{\childdocforwardprefix{#1}{#2}}
    \fi
    \expandafter
  \endgroup
  \childdoctmp
}
%    \end{macrocode}

%\iffalse
%</package>
%\fi
%
\endinput
|\\
|\childdocof{|\textit{main}|}|\\
\end{tabular}
\end{center}
at the top of every child file \textit{child}
which is included by |\include{|\textit{child}|}|
from within the main file
(or at least for those files to be compiled individually).
The argument \textit{main} must be the filename of the main file.

There are a couple of
considerations in setting up the main and child documents:

%%%%%%%%%%%%%%%%%%%%%%%%%%%%%%%%%%%%%%%%
\paragraph{Restrictions.}

Please note the following restrictions:
\begin{itemize}
\item
|\childdocmain| must be called with one argument \textit{main}
to ensure compatibility with earlier version of the package.
It must either be empty (|\childdocmain{}|)
or precisely match the filename of the main file in which it is specified.
See \secref{sec:detection} for further information.
\item
The filename \textit{main} must be specified without the |.tex| extension.
\item
The filename \textit{main} is case sensitive
(even in case-insensitive file systems)
due to internal string comparison.
\item
The argument \textit{main} should be fully expanded, it cannot be a macro.
\item
Subdirectories and special characters should be avoided in filenames.
\item
The command |\childdocmain{|\textit{main}|}| must be followed by a whitespace.
It should not be followed immediately by another command
or by a comment mark `|%|'.
This is because the \TeX{} parser reads the token immediately following
the argument of |\childdocmain| and puts it
at the beginning of every child section;
however, a white\-space is ignored.
\end{itemize}

%%%%%%%%%%%%%%%%%%%%%%%%%%%%%%%%%%%%%%%%
\paragraph{Content of Main File.}

It is advisable to place all content in the child files included by |\include|.
Any output contained in the main file will appear in all child documents
unless suppressed manually;
it cannot be suppressed automatically by the |\includeonly| directive
and thus should normally be avoided.
A method to include some content in the main file
by means of conditional processing is described in \secref{sec:conditional}.

%%%%%%%%%%%%%%%%%%%%%%%%%%%%%%%%%%%%%%%%
\paragraph{Page Numbering.}

When only a part of the document is compiled,
the appropriate numbering of pages
(as well as other status parameters)
is determined from the |.aux| files.
The latter contain information from previous passes.
However this information needs to propagate through
all intermediate child documents.
Therefore the page numbering in child documents may well
be inconsistent until the complete document is compiled at least once.

A useful (if unconventional) way to always ensure a consistent
page numbering is to restart the numbering in each child document
and denote the pages by `\textit{child}|.|\textit{page}'
where \textit{child} represents the chapter/section number of the child file.
This can be achieved by the command
|\numberwithin{page}{|\textit{child}|}|
of the \textsf{amsmath} package
where \textit{child} can be |chapter| or |section|
depending on the chosen structuring.
Alternatively, one can modify the macro |\thepage| appropriately
and reset the counter |page| at the start of each child file.

%%%%%%%%%%%%%%%%%%%%%%%%%%%%%%%%%%%%%%%%%%%%%%%%%%%%%%%%%%%%%%%%%%%%%%%%%%%%%%%%
\subsection{Conditional Processing}
\label{sec:conditional}

The package provides a mechanism to compile different versions
of a document. To customise the versions further some conditional processing
can come in handy to distinguish which version is being compiled.
The package provides two macros to describe the compilation context:

%%%%%%%%%%%%%%%%%%%%%%%%%%%%%%%%%%%%%%%%
\DescribeMacro{\ifchilddoc}
The conditional |\ifchilddoc| distinguishes between the compilation of
child documents and the main document:
%
\begin{center}
|\ifchilddoc |\textit{child-code}| |[|\||else |\textit{main-code}]| \||fi|
\end{center}

%%%%%%%%%%%%%%%%%%%%%%%%%%%%%%%%%%%%%%%%
\DescribeMacro{\childdocname}
\DescribeMacro{\childdocjob}
The macro |\childdocname| contains the filename (without extension)
of the main or child file being processed.
Note that |\childdocjob| will always contain the name of the main file.

%%%%%%%%%%%%%%%%%%%%%%%%%%%%%%%%%%%%%%%%
\paragraph{Title Page.}

Conditional processing can be used to include a title or banner page
in the main document when proper precautions are taken.
Importantly, the code in the main file should ensure that the page counter
(as well as other status parameters which are stored in the |.aux| files)
takes the same value after the conditional processing.
Otherwise the page numbers may take divergent values
depending on which part is compiled.

For example, a title page could be declared by:
%
\begin{center}
\begin{tabular}{l}
|\ifchilddoc\||else|\\
|\addtocounter{page}{-1}|\\
\textit{code for title page}\\
|\newpage|\\
|\||fi|
\end{tabular}
\end{center}
%
A banner page for the child documents can be generated by:
%
\begin{center}
\begin{tabular}{l}
|\ifchilddoc|\\
|\addtocounter{page}{-1}|\\
\textit{code for banner page}\\
|\newpage|\\
|\||fi|
\end{tabular}
\end{center}
%
Here one could write a message such as:
\begin{center}
|This is the part \childdocname{} of \childdocjob{}.|
\end{center}

%%%%%%%%%%%%%%%%%%%%%%%%%%%%%%%%%%%%%%%%%%%%%%%%%%%%%%%%%%%%%%%%%%%%%%%%%%%%%%%%
\subsection{Flags}
\label{sec:flags}

The package makes it easy to generate different versions
of the main or child documents.
To this end compilation flags can be defined
and assigned different default values.
They will be particularly useful in conjunction
with the forwarding mechanism described in \secref{sec:forward}.

For example, it may be useful to have a flag |\version|
which can be set to |draft| or |final|.
The document source will contain some conditional code
depending on the value of |\version|.
Suppose further, the flag should default to |final| for the main file
and to |draft| for child files
which is a natural assignment for editing the document.
This is achieved by placing the following code
in the preamble of the main document
(below the |\childdocmain| directive):
%
\begin{center}
\begin{tabular}{l}
|\ifchilddoc|\\
|\providecommand{\version}{draft}|\\
|\||else|\\
|\providecommand{\version}{final}|\\
|\||fi|
\end{tabular}
\end{center}
%
The definition by |\providecommand| makes sure
that previous definitions are not overwritten.
Further statements |\providecommand{\version}{...}|
can thus be added before the above code to override it.

For the main file, one might add a line
(between |\childdocmain| and the above block)
%
\begin{center}
|%\ifchilddoc\||else\providecommand{\version}{draft}\||fi|
\end{center}
%
which can be uncommented to produce a draft version.
Likewise one can add a line to the very top of a child file
(above the |\childdocof{|\textit{main}|}| directive)
%
\begin{center}
|%\providecommand{\version}{final}|
\end{center}
%
which can be uncommented to produce the final version of this child document.

%%%%%%%%%%%%%%%%%%%%%%%%%%%%%%%%%%%%%%%%%%%%%%%%%%%%%%%%%%%%%%%%%%%%%%%%%%%%%%%%
\subsection{Forwarding}
\label{sec:forward}

Different versions of the main or child documents
using compilation flags as described in \secref{sec:flags}
can be (permanently) stored in different files
for convenient compilation, viewing and distribution.
To this end, the package defines a command
to pass on compilation to a different file:

%%%%%%%%%%%%%%%%%%%%%%%%%%%%%%%%%%%%%%%%
\DescribeMacro{\childdocforward}
The command |\childdocforward| redirects processing to
another source file:
%
\begin{center}
\begin{tabular}{l}
|% \iffalse
%
% childdoc.dtx Copyright (C) 2017-2018 Niklas Beisert
%
% This work may be distributed and/or modified under the
% conditions of the LaTeX Project Public License, either version 1.3
% of this license or (at your option) any later version.
% The latest version of this license is in
%   http://www.latex-project.org/lppl.txt
% and version 1.3 or later is part of all distributions of LaTeX
% version 2005/12/01 or later.
%
% This work has the LPPL maintenance status `maintained'.
%
% The Current Maintainer of this work is Niklas Beisert.
%
% This work consists of the files childdoc.dtx and childdoc.ins
% and the derived files childdoc.def and cdocsamp.tex with
% cdocsch1.tex, cdocsch2.tex, cdocsdrf.tex, cdocsfn1.tex, cdocsfn2.tex.
%
%<package>\ifdefined\childdocmain\endinput\fi
%<package>\ProvidesFile{childdoc.def}[2018/12/30 v2.0 child document driver]
%<samplemain>\ProvidesFile{cdocsamp.tex}[2018/12/30 v2.0 sample for childdoc]
%<*driver>
%\ProvidesFile{childdoc.drv}[2018/12/30 v2.0 childdoc reference manual file]
\PassOptionsToClass{10pt,a4paper}{article}
\documentclass{ltxdoc}

\usepackage[margin=35mm]{geometry}
\usepackage{hyperref}
\usepackage{hyperxmp}
\usepackage[usenames]{color}

\hypersetup{colorlinks=true}
\hypersetup{pdfstartview=FitH}
\hypersetup{pdfpagemode=UseNone}
\hypersetup{pdfsource={}}
\hypersetup{pdflang={en-UK}}
\hypersetup{pdfcopyright={Copyright 2017-2018 Niklas Beisert.
  This work may be distributed and/or modified under the
  conditions of the LaTeX Project Public License, either version 1.3
  of this license or (at your option) any later version.}}
\hypersetup{pdflicenseurl={http://www.latex-project.org/lppl.txt}}
\hypersetup{pdfcontactaddress={ETH Zurich, ITP, HIT K,
  Wolfgang-Pauli-Strasse 27}}
\hypersetup{pdfcontactpostcode={8093}}
\hypersetup{pdfcontactcity={Zurich}}
\hypersetup{pdfcontactcountry={Switzerland}}
\hypersetup{pdfcontactemail={nbeisert@itp.phys.ethz.ch}}
\hypersetup{pdfcontacturl={http://people.phys.ethz.ch/\xmptilde nbeisert/}}

\newcommand{\secref}[1]{\hyperref[#1]{section \ref*{#1}}}

\parskip1ex
\parindent0pt
\let\olditemize\itemize
\def\itemize{\olditemize\parskip0pt}

\begin{document}

\title{The \textsf{childdoc} Package}
\hypersetup{pdftitle={The childdoc Package}}
\author{Niklas Beisert\\[2ex]
  Institut f\"ur Theoretische Physik\\
  Eidgen\"ossische Technische Hochschule Z\"urich\\
  Wolfgang-Pauli-Strasse 27, 8093 Z\"urich, Switzerland\\[1ex]
  \href{mailto:nbeisert@itp.phys.ethz.ch}
  {\texttt{nbeisert@itp.phys.ethz.ch}}}
\hypersetup{pdfauthor={Niklas Beisert}}
\hypersetup{pdfsubject={Manual for the LaTeX2e Package childdoc}}
\date{30 December 2018, \textsf{v2.0}}
\maketitle

\begin{abstract}\noindent
\textsf{childdoc} is a \LaTeXe{} package
that enables the direct compilation
of document sections included by |\include|
to individual files.
\end{abstract}

\begingroup
\parskip0ex
\tableofcontents
\endgroup

%%%%%%%%%%%%%%%%%%%%%%%%%%%%%%%%%%%%%%%%%%%%%%%%%%%%%%%%%%%%%%%%%%%%%%%%%%%%%%%%
%%%%%%%%%%%%%%%%%%%%%%%%%%%%%%%%%%%%%%%%%%%%%%%%%%%%%%%%%%%%%%%%%%%%%%%%%%%%%%%%
\section{Introduction}

\LaTeX{} provides a mechanism to structure a large document (such as a book)
into a main file and several child files (containing the chapters)
using the |\include| command.
This mechanism is beneficial for documents
which span hundreds of pages in order to
make the source file(s) more manageable.
Moreover, compilation can be restricted to
selected child files by means of the |\includeonly| command.
The latter feature can be used to reduce the compilation time while editing
(this was significantly more useful in the earlier days of \LaTeX{})
or to generate a smaller document which is easier to navigate.
Another application of |\includeonly| is to generate
documents consisting of selected parts of the complete document.

However, there are a few drawbacks of the plain |\include| mechanism:
\begin{itemize}
\item
The child files cannot be compiled on their own,
they can only be compiled via the main file.
A naive editing environment
(such as a text editor with an option
to have the current file processed by \LaTeX)
may require one to switch to the main file before compiling;
attempting to compile the child file produces errors.
\item
The main file must be modified (each time)
to adjust the |\includeonly| command
to the present needs. This easily leaves the main file in a messy state.
\item
The generated document will always carry the filename
of the main document. This is inconvenient if
several child files are to be compiled and
to be kept for distribution.
\end{itemize}

The present package provides a simple interface
to make child files individually compilable by \LaTeX{}.
Compiling a child file then has the same effect as compiling
the main file with an |\includeonly| command
to select the appropriate child.
Moreover the generated document will carry the name of the child
rather than the main file.
This resolves all three above issues.

This feature is meant to make the editing of books,
thesis documents and lecture notes somewhat more convenient.
However, the package can also be used efficiently for
composing a series of documents (such as exercise sheets)
which are typically distributed individually.
It then assists the author in generating the individual documents
(potentially in different versions)
as well as a document containing the collected series.
Another application is in developing style files
or other kinds of included material
where compilation of the style file could redirect
to a sample or test file.

%%%%%%%%%%%%%%%%%%%%%%%%%%%%%%%%%%%%%%%%%%%%%%%%%%%%%%%%%%%%%%%%%%%%%%%%%%%%%%%%
%%%%%%%%%%%%%%%%%%%%%%%%%%%%%%%%%%%%%%%%%%%%%%%%%%%%%%%%%%%%%%%%%%%%%%%%%%%%%%%%
\section{Usage}

First of all, the package \textsf{childdoc} is \emph{not} a standard
\LaTeXe{} |.sty| style file! Therefore it needs to be invoked in
a non-standard way.

%%%%%%%%%%%%%%%%%%%%%%%%%%%%%%%%%%%%%%%%%%%%%%%%%%%%%%%%%%%%%%%%%%%%%%%%%%%%%%%%
\subsection{Included Files}
\label{sec:include}

%%%%%%%%%%%%%%%%%%%%%%%%%%%%%%%%%%%%%%%%
\DescribeMacro{\childdocmain}
To use the package, add the commands
\begin{center}
\begin{tabular}{l}
|\input{childdoc.def}|\\
|\childdocmain{}|\\
\end{tabular}
\end{center}
at the very top of the main \LaTeX{} file,
in particular \emph{before} the |\documentclass| statement!
The argument of |\childdocmain| should be left empty
(but it must be present).

%%%%%%%%%%%%%%%%%%%%%%%%%%%%%%%%%%%%%%%%
\DescribeMacro{\childdocof}
Furthermore, add the commands
\begin{center}
\begin{tabular}{l}
|\input{childdoc.def}|\\
|\childdocof{|\textit{main}|}|\\
\end{tabular}
\end{center}
at the top of every child file \textit{child}
which is included by |\include{|\textit{child}|}|
from within the main file
(or at least for those files to be compiled individually).
The argument \textit{main} must be the filename of the main file.

There are a couple of
considerations in setting up the main and child documents:

%%%%%%%%%%%%%%%%%%%%%%%%%%%%%%%%%%%%%%%%
\paragraph{Restrictions.}

Please note the following restrictions:
\begin{itemize}
\item
|\childdocmain| must be called with one argument \textit{main}
to ensure compatibility with earlier version of the package.
It must either be empty (|\childdocmain{}|)
or precisely match the filename of the main file in which it is specified.
See \secref{sec:detection} for further information.
\item
The filename \textit{main} must be specified without the |.tex| extension.
\item
The filename \textit{main} is case sensitive
(even in case-insensitive file systems)
due to internal string comparison.
\item
The argument \textit{main} should be fully expanded, it cannot be a macro.
\item
Subdirectories and special characters should be avoided in filenames.
\item
The command |\childdocmain{|\textit{main}|}| must be followed by a whitespace.
It should not be followed immediately by another command
or by a comment mark `|%|'.
This is because the \TeX{} parser reads the token immediately following
the argument of |\childdocmain| and puts it
at the beginning of every child section;
however, a white\-space is ignored.
\end{itemize}

%%%%%%%%%%%%%%%%%%%%%%%%%%%%%%%%%%%%%%%%
\paragraph{Content of Main File.}

It is advisable to place all content in the child files included by |\include|.
Any output contained in the main file will appear in all child documents
unless suppressed manually;
it cannot be suppressed automatically by the |\includeonly| directive
and thus should normally be avoided.
A method to include some content in the main file
by means of conditional processing is described in \secref{sec:conditional}.

%%%%%%%%%%%%%%%%%%%%%%%%%%%%%%%%%%%%%%%%
\paragraph{Page Numbering.}

When only a part of the document is compiled,
the appropriate numbering of pages
(as well as other status parameters)
is determined from the |.aux| files.
The latter contain information from previous passes.
However this information needs to propagate through
all intermediate child documents.
Therefore the page numbering in child documents may well
be inconsistent until the complete document is compiled at least once.

A useful (if unconventional) way to always ensure a consistent
page numbering is to restart the numbering in each child document
and denote the pages by `\textit{child}|.|\textit{page}'
where \textit{child} represents the chapter/section number of the child file.
This can be achieved by the command
|\numberwithin{page}{|\textit{child}|}|
of the \textsf{amsmath} package
where \textit{child} can be |chapter| or |section|
depending on the chosen structuring.
Alternatively, one can modify the macro |\thepage| appropriately
and reset the counter |page| at the start of each child file.

%%%%%%%%%%%%%%%%%%%%%%%%%%%%%%%%%%%%%%%%%%%%%%%%%%%%%%%%%%%%%%%%%%%%%%%%%%%%%%%%
\subsection{Conditional Processing}
\label{sec:conditional}

The package provides a mechanism to compile different versions
of a document. To customise the versions further some conditional processing
can come in handy to distinguish which version is being compiled.
The package provides two macros to describe the compilation context:

%%%%%%%%%%%%%%%%%%%%%%%%%%%%%%%%%%%%%%%%
\DescribeMacro{\ifchilddoc}
The conditional |\ifchilddoc| distinguishes between the compilation of
child documents and the main document:
%
\begin{center}
|\ifchilddoc |\textit{child-code}| |[|\||else |\textit{main-code}]| \||fi|
\end{center}

%%%%%%%%%%%%%%%%%%%%%%%%%%%%%%%%%%%%%%%%
\DescribeMacro{\childdocname}
\DescribeMacro{\childdocjob}
The macro |\childdocname| contains the filename (without extension)
of the main or child file being processed.
Note that |\childdocjob| will always contain the name of the main file.

%%%%%%%%%%%%%%%%%%%%%%%%%%%%%%%%%%%%%%%%
\paragraph{Title Page.}

Conditional processing can be used to include a title or banner page
in the main document when proper precautions are taken.
Importantly, the code in the main file should ensure that the page counter
(as well as other status parameters which are stored in the |.aux| files)
takes the same value after the conditional processing.
Otherwise the page numbers may take divergent values
depending on which part is compiled.

For example, a title page could be declared by:
%
\begin{center}
\begin{tabular}{l}
|\ifchilddoc\||else|\\
|\addtocounter{page}{-1}|\\
\textit{code for title page}\\
|\newpage|\\
|\||fi|
\end{tabular}
\end{center}
%
A banner page for the child documents can be generated by:
%
\begin{center}
\begin{tabular}{l}
|\ifchilddoc|\\
|\addtocounter{page}{-1}|\\
\textit{code for banner page}\\
|\newpage|\\
|\||fi|
\end{tabular}
\end{center}
%
Here one could write a message such as:
\begin{center}
|This is the part \childdocname{} of \childdocjob{}.|
\end{center}

%%%%%%%%%%%%%%%%%%%%%%%%%%%%%%%%%%%%%%%%%%%%%%%%%%%%%%%%%%%%%%%%%%%%%%%%%%%%%%%%
\subsection{Flags}
\label{sec:flags}

The package makes it easy to generate different versions
of the main or child documents.
To this end compilation flags can be defined
and assigned different default values.
They will be particularly useful in conjunction
with the forwarding mechanism described in \secref{sec:forward}.

For example, it may be useful to have a flag |\version|
which can be set to |draft| or |final|.
The document source will contain some conditional code
depending on the value of |\version|.
Suppose further, the flag should default to |final| for the main file
and to |draft| for child files
which is a natural assignment for editing the document.
This is achieved by placing the following code
in the preamble of the main document
(below the |\childdocmain| directive):
%
\begin{center}
\begin{tabular}{l}
|\ifchilddoc|\\
|\providecommand{\version}{draft}|\\
|\||else|\\
|\providecommand{\version}{final}|\\
|\||fi|
\end{tabular}
\end{center}
%
The definition by |\providecommand| makes sure
that previous definitions are not overwritten.
Further statements |\providecommand{\version}{...}|
can thus be added before the above code to override it.

For the main file, one might add a line
(between |\childdocmain| and the above block)
%
\begin{center}
|%\ifchilddoc\||else\providecommand{\version}{draft}\||fi|
\end{center}
%
which can be uncommented to produce a draft version.
Likewise one can add a line to the very top of a child file
(above the |\childdocof{|\textit{main}|}| directive)
%
\begin{center}
|%\providecommand{\version}{final}|
\end{center}
%
which can be uncommented to produce the final version of this child document.

%%%%%%%%%%%%%%%%%%%%%%%%%%%%%%%%%%%%%%%%%%%%%%%%%%%%%%%%%%%%%%%%%%%%%%%%%%%%%%%%
\subsection{Forwarding}
\label{sec:forward}

Different versions of the main or child documents
using compilation flags as described in \secref{sec:flags}
can be (permanently) stored in different files
for convenient compilation, viewing and distribution.
To this end, the package defines a command
to pass on compilation to a different file:

%%%%%%%%%%%%%%%%%%%%%%%%%%%%%%%%%%%%%%%%
\DescribeMacro{\childdocforward}
The command |\childdocforward| redirects processing to
another source file:
%
\begin{center}
\begin{tabular}{l}
|\input{childdoc.def}|\\
|\childdocforward[|\textit{main}|]{|\textit{dest}|}|\\
\end{tabular}
\end{center}
%
The argument \textit{dest} is the destination file
(without extension).
It should be the main file or one of the child files.
Note that further \textsf{childdoc} directives
such as |\childdocof| and |\childdocforward|
in the indicated file will be processed in this form.
The optional argument \textit{main}
passes on directly to the main file \textit{main}
while pretending to compile the child \textit{dest}.
This form behaves as if \textit{dest}
issues |\childdocof{|\textit{main}|}| right away,
and no further \textsf{childdoc} directives will be processed.

%%%%%%%%%%%%%%%%%%%%%%%%%%%%%%%%%%%%%%%%
\DescribeMacro{\...prefix}
In the alternative form |\childdocforwardprefix|,
%
\begin{center}
\begin{tabular}{l}
|\input{childdoc.def}|\\
|\childdocforwardprefix[|\textit{main}|]{|\textit{prefix}|}{|\textit{dest}|}|
\end{tabular}
\end{center}
%
the destination file is determined by a pattern
depending on the current file:
To make this work, the current file must be called
`{\textit{prefix}\hspace{0.2em}\textit{suffix}}'
with \textit{prefix} matching precisely the argument.
Processing is then passed on to the file
`{\textit{dest}\hspace{0.2em}\textit{suffix}}'.
Surely, the same effect is achieved by
directly specifying the
argument `{\textit{dest}\hspace{0.2em}\textit{suffix}}'
in the first form.
However, that requires to set up a different file
for each child. With the alternative form of the command
all these files can have exactly the same content
which simplifies setting them up and maintaining them.

For example, the following file |draft.tex|
with a compilation flag |\version| as described in \secref{sec:flags}
compiles the main document as a draft:
%
\begin{center}
\begin{tabular}{l}
|\def\version{draft}|\\
|\input{childdoc.def}|\\
|\childdocforward{|\textit{main}|}|
\end{tabular}
\end{center}
%
Likewise, the following files |final|\textit{nn}|.tex|
compile the final version of the child document
|child|\textit{nn}|.tex|:
%
\begin{center}
\begin{tabular}{l}
|\def\version{final}|\\
|\input{childdoc.def}|\\
|\childdocforwardprefix{final}{child}|
\end{tabular}
\end{center}
%

Note that when several versions of a main file and/or of each child file
are to be generated, it may be convenient to set up a |Makefile| or
shell script to automatise the process.

%%%%%%%%%%%%%%%%%%%%%%%%%%%%%%%%%%%%%%%%%%%%%%%%%%%%%%%%%%%%%%%%%%%%%%%%%%%%%%%%
\subsection{Command Line Processing}
\label{sec:commandline}

The effect of redirection files can also be achieved by invoking
the \LaTeX{} compiler with a more elaborate command line.
Most conveniently this should be done as part
of a shell script or a |Makefile|.

When using \textsf{childdoc} in the main file, the following
command lines effectively perform a redirection
(note that depending on the shell being used,
backslashes may have to be doubled: `|\|' $\to$ `|\\|'):
%
\begin{center}
|... -jobname "|\textit{target}|" |\\|"|[\textit{flags}]%
|\input{childdoc.def}\childdocforward[|\textit{main}|]{|\textit{dest}|}"|
\end{center}
%
Here \textit{target} is the name of the output file,
\textit{main} is the name of the main file
and \textit{dest} is the name of the main or child file to be processed
(all filenames without extensions).
The optional argument \textit{main} can be omitted
if \textit{main} matches \textit{dest}.
Optionally, compilation \textit{flags} can be defined via |\def| commands.
This command line makes the \TeX{} engine believe
it is compiling the file \textit{target}
whose content is specified as the latter parameter.
The provided code then forwards the processing to
\textit{main} or \textit{dest} as described in \secref{sec:forward}.

%%%%%%%%%%%%%%%%%%%%%%%%%%%%%%%%%%%%%%%%%%%%%%%%%%%%%%%%%%%%%%%%%%%%%%%%%%%%%%%%
\subsection{Include by Input}
\label{sec:input}

Including child documents by |\include| has some restrictions by design.
Most notably, the content of a child document always occupies
its own set of pages; pages cannot be shared between child documents.
Usually, this behaviour makes perfect sense
because each child document contain an essential part of the document.
However, in some situations it may be desirable to compose
a document from a collection of parts
without having mandatory page breaks between then.
For this case, the package
provides a mechanism to include parts
by |\input| which can also be processed individually.
However, by construction this mechanism
requires manual handling of the content to be output.

%%%%%%%%%%%%%%%%%%%%%%%%%%%%%%%%%%%%%%%%
\DescribeMacro{\ifchilddocmanual}
The main file should be prepared as usual, see \secref{sec:include}.
However, the document body must make a distinction
between processing of an individual part and of the main document, e.g.:
%
\begin{center}
\begin{tabular}{l}
|\ifchilddocmanual|\\
|\input{\childdocname}|\\
|\||else|\\
\textit{document body with }|\input{|\textit{part}|}|\\
|\||fi|
\end{tabular}
\end{center}
%
The conditional |\ifchilddocmanual| is true whenever
a part to be included by |\input| is being compiled,
and the name of the part is stored in |\childdocname|.

%%%%%%%%%%%%%%%%%%%%%%%%%%%%%%%%%%%%%%%%
\DescribeMacro{\childdocby}
Each part to be included by |\input| should start with:
%
\begin{center}
\begin{tabular}{l}
|\input{childdoc.def}|\\
|\childdocby{|\textit{main}|}|\\
\end{tabular}
\end{center}
%
The directive |\childdocby| is similar to |\childdocof|
described in \secref{sec:include},
but the subsequent selection of content must be done manually.
To that end, both |\ifchilddoc| and |\ifchilddocmanual|
will be true upon processing of a part,
and the name of the part is stored in |\childdocname|.
Note that |\jobname| will be set to the filename of the current part
so that each part receives an individual |.aux| file
that does not interfere with the |.aux| file(s) of the main document.
This behaviour can be altered by the alternative form
|\childdocby[*]{|\textit{main}|}| (with a non-empty optional argument)
which uses the |.aux| file of the main document
by setting |\jobname| to \textit{main}.

%%%%%%%%%%%%%%%%%%%%%%%%%%%%%%%%%%%%%%%%%%%%%%%%%%%%%%%%%%%%%%%%%%%%%%%%%%%%%%%%
\subsection{Driver Development}
\label{sec:driver}

The \textsf{childdoc} mechanism can also be use for the development
of definition files such as \LaTeX{} styles or classes.
This case differs from the above setup with multiple parts
included by |\include| in that no |\includeonly| should be invoked.
This can be achieved by starting the include file
(before |\ProvidesPackage|) with:
%
\begin{center}
\begin{tabular}{l}
|\input{childdoc.def}|\\
|\childdocforward{|\textit{main}|}|\\
\end{tabular}
\end{center}
%
or alternatively with:
%
\begin{center}
\begin{tabular}{l}
|\input{childdoc.def}|\\
|\childdocby{|\textit{main}|}|\\
\end{tabular}
\end{center}
%
Both forms have slightly different effects as described above.
The main file is prepared as usual, see \secref{sec:include}.

%%%%%%%%%%%%%%%%%%%%%%%%%%%%%%%%%%%%%%%%%%%%%%%%%%%%%%%%%%%%%%%%%%%%%%%%%%%%%%%%
\subsection{Legacy Detection}
\label{sec:detection}

The directive |\childdocmain| in the main file can detect
whether the complete document or merely a child is to be compiled
even without using the directive |\childdocof|.
This method is deprecated because it is less robust
and there is no compelling reason to use it;
it is merely provided for backward compatibility
and it may be removed in future versions.

If the detection mechanism is to be used,
it is mandatory to correctly specify
the filename of the main file as the argument of |\childdocmain|:
%
\begin{center}
\begin{tabular}{l}
|\input{childdoc.def}|\\
|\childdocmain{|\textit{main}|}|\\
\end{tabular}
\end{center}
%
If |\jobname| does not match the argument \textit{main} of |\childdocmain|,
it is assumed that |\jobname| points to the child file to be compiled.
When using |\childdocmain| with the main file specified as argument,
it suffices to start a child file
with just |\input{|\textit{main}|}|
without loading of the package and using |\childdocof|.
If instead all processing is done
with the appropriate \textsf{childdoc} directives,
the argument of \textit{main} of |\childdocmain| can be empty.

An alternative version of the command line processing described
in \secref{sec:commandline} using the detection mechanism reads:
%
\begin{center}
|... -jobname "|\textit{target}|" "|[\textit{flags}]%
[|\def\jobname{|\textit{dest}|}|]|\input{|\textit{main}|}"|
\end{center}

%%%%%%%%%%%%%%%%%%%%%%%%%%%%%%%%%%%%%%%%%%%%%%%%%%%%%%%%%%%%%%%%%%%%%%%%%%%%%%%%
\subsection{Manual Code}
\label{sec:manual}

In case one cannot be certain whether the definitions file |childdoc.def|
is installed on the target \TeX{} distribution
and one prefers not to ship it,
it is conceivable to paste a few relevant commands into the sources.

To that end, drop all statements |\input{childdoc.def}|
and perform the replacements as outlined below.
Instead of |\childdocmain{|\textit{main}|}| add the following code
to the top of the main file:
%
\begin{center}
\begin{tabular}{l}
|\||ifdefined\childdocname\endinput\||fi\newif\ifchilddoc|\\
|\edef\childdocname{\scantokens\expandafter{\jobname\noexpand}}|\\
|\def\childdocmain{|\textit{main}|}\||ifx\childdocmain\childdocname\||else|\\
|\childdoctrue\includeonly{\childdocname}\let\jobname\childdocmain\||fi|\\
\end{tabular}
\end{center}
%
Instead of |\childdocof{|\textit{main}|}| just include the main file
at the top of each child file:
%
\begin{center}
|\input{|\textit{main}|}|
\end{center}
%
A simple redirection |\childdocforward{|\textit{dest}|}| is achieved by:
%
\begin{center}
|\def\jobname{|\textit{dest}|}\input{\jobname}|
\end{center}
%
The redirection with prefix
|\childdocforwardprefix[|\textit{prefix}|]{|\textit{dest}|}|
is accomplished by:
%
\begin{center}
\begin{tabular}{l}
|{\edef\jobname{\scantokens\expandafter{\jobname\noexpand}}|\\
|\def\redirectjob |\textit{prefix}|#1~~~{\gdef\jobname{|\textit{dest}|#1}}|\\
|\expandafter\redirectjob\jobname~~~}\input{\jobname}|
\end{tabular}
\end{center}

In an alternative approach,
child documents can be compiled by a specific command line
without additional code or specific definitions:
%
\begin{center}
|... -jobname "|\textit{target}|" "|[\textit{flags}]%
|\includeonly{|\textit{dest}|}\input{|\textit{main}|}"|
\end{center}
%

%%%%%%%%%%%%%%%%%%%%%%%%%%%%%%%%%%%%%%%%%%%%%%%%%%%%%%%%%%%%%%%%%%%%%%%%%%%%%%%%
%%%%%%%%%%%%%%%%%%%%%%%%%%%%%%%%%%%%%%%%%%%%%%%%%%%%%%%%%%%%%%%%%%%%%%%%%%%%%%%%
\section{Information}

%%%%%%%%%%%%%%%%%%%%%%%%%%%%%%%%%%%%%%%%%%%%%%%%%%%%%%%%%%%%%%%%%%%%%%%%%%%%%%%%
\subsection{Copyright}

Copyright \copyright{} 2017--2018 Niklas Beisert

This work may be distributed and/or modified under the
conditions of the \LaTeX{} Project Public License, either version 1.3
of this license or (at your option) any later version.
The latest version of this license is in
  \url{http://www.latex-project.org/lppl.txt}
and version 1.3 or later is part of all distributions of \LaTeX{}
version 2005/12/01 or later.

This work has the LPPL maintenance status `maintained'.

The Current Maintainer of this work is Niklas Beisert.

This work consists of the files |README.txt|, |childdoc.ins| and |childdoc.dtx|
as well as the derived files |childdoc.def|, |cdocsamp.tex|
with |cdocsch1.tex|, |cdocsch2.tex|, |cdocspt3.tex|, |cdocspt4.tex|,
|cdocsdrf.tex|, |cdocsfn1.tex|, |cdocsfn2.tex|
as well as |childdoc.pdf|.

%%%%%%%%%%%%%%%%%%%%%%%%%%%%%%%%%%%%%%%%%%%%%%%%%%%%%%%%%%%%%%%%%%%%%%%%%%%%%%%%
\subsection{Files and Installation}

The package consists of the files:
%
\begin{center}
\begin{tabular}{ll}
    |README.txt|   & readme file \\
    |childdoc.ins| & installation file \\
    |childdoc.dtx| & source file \\
    |childdoc.def| & definition file \\
    |cdocsamp.tex| & sample main file \\
    |cdocsch1.tex| & sample include file \\
    |cdocsch2.tex| & sample include file \\
    |cdocspt3.tex| & sample part file \\
    |cdocspt4.tex| & sample part file \\
    |cdocsdrf.tex| & sample redirection file \\
    |cdocsfn1.tex| & sample redirection file \\
    |cdocsfn2.tex| & sample redirection file \\
    |childdoc.pdf| & manual
\end{tabular}
\end{center}
%
The distribution consists of the files
|README.txt|, |childdoc.ins| and |childdoc.dtx|.
%
\begin{itemize}
\item
Run (pdf)\LaTeX{} on |childdoc.dtx|
to compile the manual |childdoc.pdf| (this file).
\item
Run \LaTeX{} on |childdoc.ins| to create the definitions file |childdoc.def|
and the sample |cdocsamp.tex| with include files
|cdocsch1.tex|, |cdocsch2.tex|, |cdocspt3.tex|, |cdocspt4.tex|,
|cdocsdrf.tex|, |cdocsfn1.tex|, |cdocsfn2.tex|.
Then copy the file |childdoc.def| to an appropriate directory of your \LaTeX{}
distribution, e.g.\ \textit{texmf-root}|/tex/latex/childdoc|.
\end{itemize}

%%%%%%%%%%%%%%%%%%%%%%%%%%%%%%%%%%%%%%%%%%%%%%%%%%%%%%%%%%%%%%%%%%%%%%%%%%%%%%%%
\subsection{Related CTAN Packages}

There are several other packages which offer a similar functionality:
%
\begin{itemize}
\item
The packages
\href{http://ctan.org/pkg/docmute}{\textsf{docmute}},
\href{http://ctan.org/pkg/includex}{\textsf{includex}} and
\href{http://ctan.org/pkg/standalone}{\textsf{standalone}}
provide commands to include only the document body of
a child file thus allowing both files to be compiled individually.
\item
The packages \href{http://ctan.org/pkg/subdocs}{\textsf{subdocs}}
and \href{http://ctan.org/pkg/subfiles}{\textsf{subfiles}}
provide structures in which the main and child documents can be
encapsulated and allowing them to be compiled individually.
The inclusion mechanism is different from the conventional |\include|.
\item
The package \href{http://ctan.org/pkg/combine}{\textsf{combine}}
is an elaborate solution to combine several documents into one.
\end{itemize}
%
See also the CTAN topic \href{http://ctan.org/topic/subdocs}{\textsf{subdocs}}
for further related packages.
The present package differs from the above solutions in that
a document structure constructed with the conventional |\include| mechanism
just needs two extra commands at the top of every file
such that all constituent files can be compiled individually.

%%%%%%%%%%%%%%%%%%%%%%%%%%%%%%%%%%%%%%%%%%%%%%%%%%%%%%%%%%%%%%%%%%%%%%%%%%%%%%%%
%\subsection{Feature Suggestions}
%
%The following is a list of features which may be useful for future
%versions of this package:
%%
%\begin{itemize}
%\item
%\ldots
%\end{itemize}

%%%%%%%%%%%%%%%%%%%%%%%%%%%%%%%%%%%%%%%%%%%%%%%%%%%%%%%%%%%%%%%%%%%%%%%%%%%%%%%%
\subsection{Revision History}

%%%%%%%%%%%%%%%%%%%%%%%%%%%%%%%%%%%%%%%%
\paragraph{v2.0:} 2018/12/30

\begin{itemize}
\item
immediate forward processing
\item
added |\childdocby| mechanism
\item
manual restructured
\end{itemize}

%%%%%%%%%%%%%%%%%%%%%%%%%%%%%%%%%%%%%%%%
\paragraph{v1.6:} 2018/01/17

\begin{itemize}
\item
application for development of include files
\item
corrections to manual
\end{itemize}

%%%%%%%%%%%%%%%%%%%%%%%%%%%%%%%%%%%%%%%%
\paragraph{v1.5:} 2017/05/21

\begin{itemize}
\item
more complete structuring introduced
\item
|\childdocof| introduced
\item
|\childdoc| renamed to |\childdocmain|
\item
|\childredirect| renamed to |\childdocforward| and |\childdocforwardprefix|
and functionality expanded
\end{itemize}

%%%%%%%%%%%%%%%%%%%%%%%%%%%%%%%%%%%%%%%%
\paragraph{v1.0:} 2017/04/27

\begin{itemize}
\item
manual and install package
\item
first version published on CTAN
\end{itemize}

%%%%%%%%%%%%%%%%%%%%%%%%%%%%%%%%%%%%%%%%
\paragraph{v0.6:} 2017/04/26

\begin{itemize}
\item
redirection mechanism added
\end{itemize}

%%%%%%%%%%%%%%%%%%%%%%%%%%%%%%%%%%%%%%%%
\paragraph{v0.5:} 2017/04/26

\begin{itemize}
\item
functionality in definition file
\end{itemize}


%%%%%%%%%%%%%%%%%%%%%%%%%%%%%%%%%%%%%%%%%%%%%%%%%%%%%%%%%%%%%%%%%%%%%%%%%%%%%%%%
%%%%%%%%%%%%%%%%%%%%%%%%%%%%%%%%%%%%%%%%%%%%%%%%%%%%%%%%%%%%%%%%%%%%%%%%%%%%%%%%
%%%%%%%%%%%%%%%%%%%%%%%%%%%%%%%%%%%%%%%%%%%%%%%%%%%%%%%%%%%%%%%%%%%%%%%%%%%%%%%%
\appendix

\settowidth\MacroIndent{\rmfamily\scriptsize 000\ }

 \DocInput{childdoc.dtx}

\end{document}
%</driver>
% \fi
%
% %%%%%%%%%%%%%%%%%%%%%%%%%%%%%%%%%%%%%%%%%%%%%%%%%%%%%%%%%%%%%%%%%%%%%%%%%%%%%%
% %%%%%%%%%%%%%%%%%%%%%%%%%%%%%%%%%%%%%%%%%%%%%%%%%%%%%%%%%%%%%%%%%%%%%%%%%%%%%%
% \section{Sample}
%\iffalse
%<*samplemain>
%\fi
%
% The following presents a sample document
% with two chapters, two parts, a title page,
% a compile flag as well as three forwarding files to set the flag.
% It consists of eight |.tex| files:
% \begin{center}
% \begin{tabular}{ll}
% |cdocsamp.tex|&main file\\
% |cdocsch1.tex|&include file for chapter 1\\
% |cdocsch2.tex|&include file for chapter 2\\
% |cdocspt3.tex|&include file for part 3\\
% |cdocspt4.tex|&include file for part 4\\
% |cdocsdrf.tex|&forwarding file for main file in draft mode\\
% |cdocsfi1.tex|&forwarding file for final version of chapter 1\\
% |cdocsfi2.tex|&forwarding file for final version of chapter 2\\
% \end{tabular}
% \end{center}
% Each of the eight files can be compiled directly by the \LaTeX{} compiler.
%
% %%%%%%%%%%%%%%%%%%%%%%%%%%%%%%%%%%%%%%
% \paragraph{Main File.}
%
% The main file is called |cdocsamp.tex|.
%
% Load the \textsf{childdoc} definitions and
% declare the filename for the main document:
%    \begin{macrocode}
\input{childdoc.def}
\childdocmain{}
%    \end{macrocode}

% Optional override for |\version| flag:
%    \begin{macrocode}
%%\ifchilddoc\else\providecommand{\version}{draft}\fi
%    \end{macrocode}

% Define the default values for the |\version| flag
% (|final| for the main file and |draft| for childs):
%    \begin{macrocode}
\ifchilddoc
\providecommand{\version}{draft}
\else
\providecommand{\version}{final}
\fi
%    \end{macrocode}

% Load the standard document class:
%    \begin{macrocode}
\documentclass[12pt]{article}
%    \end{macrocode}

% Start the document body:
%    \begin{macrocode}
\begin{document}
%    \end{macrocode}

% Declare a title page.
% Print title, part of document being processed and version flag:
%    \begin{macrocode}
\addtocounter{page}{-1}
\begin{center}
{\LARGE\bfseries{}childdoc example\par}
\vspace{1cm}
\ifchilddoc
\ifchilddocmanual part\else chapter\fi:
`\childdocname' of `\childdocjob'\par
\else
main document: `\childdocjob'\par
\fi
version: \version\par
\end{center}
\newpage
%    \end{macrocode}

% Manually include selected file,
% otherwise process as usual:
%    \begin{macrocode}
\ifchilddocmanual
\section*{part `\childdocname'}
\input{\childdocname}
\else
%    \end{macrocode}

% Include the two chapters:
%    \begin{macrocode}
\include{cdocsch1}
\include{cdocsch2}
%    \end{macrocode}

% Include the two parts unless only chapters should be displayed:
%    \begin{macrocode}
\ifchilddoc\else
\section{part three}
\input{cdocspt3}
\section{part four}
\input{cdocspt4}
\fi
%    \end{macrocode}

% Process as usual until here:
%    \begin{macrocode}
\fi
%    \end{macrocode}

% End of document body:
%    \begin{macrocode}
\end{document}
%    \end{macrocode}
%\iffalse
%</samplemain>
%\fi
%
% %%%%%%%%%%%%%%%%%%%%%%%%%%%%%%%%%%%%%%
% \paragraph{Chapter Include Files.}
%
% The include files are called |cdocsch1.tex| and |cdocsch2.tex|.
%
%\iffalse
%<*samplechap1|samplechap2>
%\fi

% Optional override for |\version| flag:
%    \begin{macrocode}
%%\providecommand{\version}{final}
%    \end{macrocode}

% Include the main document:
%    \begin{macrocode}
\input{childdoc.def}
\childdocof{cdocsamp}
%    \end{macrocode}

%\iffalse
%</samplechap1|samplechap2>
%\fi
%
%\iffalse
%<*samplechap1>
%\fi
% Some text for chapter 1:
%    \begin{macrocode}
\section{one}
some text in chapter one
%    \end{macrocode}

%\iffalse
%</samplechap1>
%\fi
% Some text for chapter 2:
%\iffalse
%<*samplechap2>
%\fi
%    \begin{macrocode}
\section{two}
more text in chapter two
%    \end{macrocode}

%\iffalse
%</samplechap2>
%\fi
%
% %%%%%%%%%%%%%%%%%%%%%%%%%%%%%%%%%%%%%%
% \paragraph{Part Include Files.}
%
% The include files are called |cdocspt3.tex| and |cdocspt4.tex|.
%
%\iffalse
%<*samplepart3|samplepart4>
%\fi

% Optional override for |\version| flag:
%    \begin{macrocode}
%%\providecommand{\version}{final}
%    \end{macrocode}

% Include the main document:
%    \begin{macrocode}
\input{childdoc.def}
\childdocby{cdocsamp}
%    \end{macrocode}

%\iffalse
%</samplepart3|samplepart4>
%\fi
%
%\iffalse
%<*samplepart3>
%\fi
% Some text for part 3:
%    \begin{macrocode}
some text in part three
%    \end{macrocode}

%\iffalse
%</samplepart3>
%\fi
% Some text for part 4:
%\iffalse
%<*samplepart4>
%\fi
%    \begin{macrocode}
more text in part four
%    \end{macrocode}

%\iffalse
%</samplepart4>
%\fi
%
% %%%%%%%%%%%%%%%%%%%%%%%%%%%%%%%%%%%%%%
% \paragraph{Forwarding for a Complete Draft.}
%
% The following forwarding file |cdocsdrf.tex|
% compiles the main document in draft mode:
%\iffalse
%<*sampledraft>
%\fi
%    \begin{macrocode}
\def\version{draft}
\input{childdoc.def}
\childdocforward{cdocsamp}
%    \end{macrocode}

%\iffalse
%</sampledraft>
%\fi
%
% %%%%%%%%%%%%%%%%%%%%%%%%%%%%%%%%%%%%%%
% \paragraph{Forwarding for Final Version of the Chapters.}
%
% The following forwarding files |cdocsfn1.tex| and |cdocsfn2.tex|
% (with identical content)
% compile the final versions of the child documents
% |cdocsch1.tex| and |cdocsch2.tex|, respectively:
%\iffalse
%<*samplefinal>
%\fi
%    \begin{macrocode}
\def\version{final}
\input{childdoc.def}
\childdocforwardprefix[cdocsamp]{cdocsfn}{cdocsch}
%    \end{macrocode}

%\iffalse
%</samplefinal>
%\fi
%
% %%%%%%%%%%%%%%%%%%%%%%%%%%%%%%%%%%%%%%
% \paragraph{Command Line Processing.}
%
% The following three command lines generate the output files
% |cdocscld|, |cdocscl1| and |cdocscl2|
% which should be identical to
% |cdocsdrf|, |cdocsch1| and |cdocsfn2|, respectively:
% \begin{center}
% \begin{tabular}{l}
% |latex -jobname cdocscld \|\\
% |  "\def\version{draft}\input{childdoc.def}\childdocforward{cdocsamp}"|\\
% |latex -jobname cdocscl1 \|\\
% |  "\input{childdoc.def}\childdocforward[cdocsamp]{cdocsch1}"|\\
% |latex -jobname cdocscl2 \|\\
% |  "\def\version{final}\input{childdoc.def}\childdocforward{cdocsch2}"|
% \end{tabular}
% \end{center}
% Note that the trailing backslash on each first line
% merely continues the input to the second line
% (for convenient cut ant paste).
% Furthermore, the command |latex| can be replaced by any
% of its alternative versions such as |pdflatex|.
%
% %%%%%%%%%%%%%%%%%%%%%%%%%%%%%%%%%%%%%%%%%%%%%%%%%%%%%%%%%%%%%%%%%%%%%%%%%%%%%%
% %%%%%%%%%%%%%%%%%%%%%%%%%%%%%%%%%%%%%%%%%%%%%%%%%%%%%%%%%%%%%%%%%%%%%%%%%%%%%%
% \section{Implementation}
%\iffalse
%<*package>
%\fi
%
% This section describes the definitions file |childdoc.def|.

% The definitions cannot be loaded using |\usepackage| or |\RequirePackage|
% which has a mechanism to prevent loading a style file more than once.
% When loading the definitions by means of |\input|
% multiple instances have to be prevented manually:
%\iffalse
%This code needs to be before the `\ProvidesFile' directive
%which is defined at the beginning of this file.
%Therefore it is also placed there and commented out here.
%</package>
%<*discard>
%\fi
%    \begin{macrocode}
\ifdefined\childdocmain\endinput\fi
%    \end{macrocode}
%\iffalse
%</discard>
%<*package>
%\fi
%
% \macro{\ifchilddoc}
% \macro{\ifchilddocmanual}
% The conditional |\ifchilddoc| tells whether a
% child (true) or main (false) document is being compiled.
% The conditional |\ifchilddocmanual| tells whether
% the |\includeonly| mechanism is used (false) or
% the selection of child files must be performed manually (true).
% The definitions initialise to false:
%    \begin{macrocode}
\newif\ifchilddoc
\newif\ifchilddocmanual
%    \end{macrocode}

% \macro{\childdocname}
% \macro{\childdocjob}
% The macro |\childdocname| stores the name of the main document
% to be compiled. The macro |\childdocjob| stores the name of
% the document on which the \LaTeX{} compiler was originally invoked.
% The content of |\jobname| cannot be compared
% to filenames specified in the source due to different catcodes.
% The following code rescans |\jobname|, stores the result
% in |\childdocname| and saves a copy in |\childdocjob|:
%    \begin{macrocode}
\edef\childdocname{\scantokens\expandafter{\jobname\noexpand}}
\let\childdocjob\childdocname
%    \end{macrocode}

% \macro{\childdocdisable}
% The macro |\childdocdisable| prevents the main file
% from being processed more than once.
% At this stage, the main document command |\childdocmain|
% is assumed to be called once again where it should do nothing.
% Any subsequent call to it should prevent
% a secondary processing of the main document
% It overwrites the forwarding commands
% |\childdocof| and |\childdocforward|
% with empty macros to prevent further inclusions of the main document:
%    \begin{macrocode}
\newcommand{\childdocdisable}
{
  \renewcommand{\childdocmain}[1]{\renewcommand{\childdocmain}[1]{\endinput}}
  \renewcommand{\childdocof}[1]{}
  \renewcommand{\childdocby}[2][]{}
  \renewcommand{\childdocforward}[2][]{}
  \renewcommand{\childdocdisable}{}
}
%    \end{macrocode}

% \macro{\childdocmain}
% The macro |\childdocmain| is to be called at the top of the main file
% with nothing or the main filename (without extension) as argument.
% First, it breaks loops.
% If the argument is not empty and does not match |\childdocname|
% (which is set by the first inclusion of |childdoc.def|),
% |\ifchilddoc| is set to true, |\includeonly| is applied to the child file
% and |\jobname| is set to the main file
% (for proper handling of |.aux| files):
%    \begin{macrocode}
\newcommand{\childdocmain}[1]
{
  \childdocdisable\childdocmain{}
  \if?#1?\else
    \begingroup
      \def\childdoctmp{#1}
      \ifx\childdoctmp\childdocname
        \def\childdoctmp{}
      \else
        \def\childdoctmp
        {
          \childdoctrue
          \includeonly{\childdocname}
          \def\childdocjob{#1}
          \def\jobname{#1}
        }
      \fi
      \expandafter
    \endgroup
    \childdoctmp
  \fi
}
%    \end{macrocode}

% \macro{\childdocof}
% The command |\childdocof| redirects
% compilation to the main file |#1|.
%    \begin{macrocode}
\newcommand{\childdocof}[1]
{
  \childdocdisable
  \childdoctrue
  \includeonly{\childdocname}
  \def\jobname{#1}
  \def\childdocjob{#1}
  \input{#1}
}
%    \end{macrocode}

% \macro{\childdocby}
% The command |\childdocby| ....
%    \begin{macrocode}
\newcommand{\childdocby}[2][]
{
  \childdocdisable
  \childdoctrue
  \childdocmanualtrue
  \if?#1?\else
    \def\jobname{#2}
  \fi
  \def\childdocjob{#2}
  \input{#2}
  \endinput
}
%    \end{macrocode}

% \macro{\childdocforward}
% The command |\childdocforward| redirects
% compilation to the main file or
% (if the optional argument is given) a child file.
% Parameters are set as if the main file
% or a child file starting with |\childdocof| was compiled.
% Then compilation is handed over to the main file:
%    \begin{macrocode}
\newcommand{\childdocforward}[2][]
{
  \begingroup
    \if?#1?
      \def\childdoctmp
      {
        \def\childdocname{#2}
        \def\childdocjob{#2}
        \def\jobname{#2}
        \input{#2}
        \endinput
      }
    \else
      \def\childdoctmp
      {
        \childdocdisable
        \def\childdocname{#2}
        \childdoctrue
        \includeonly{#2}
        \def\childdocjob{#1}
        \def\jobname{#1}
        \input{#1}
        \endinput
      }
    \fi
    \expandafter
  \endgroup
  \childdoctmp
}
%    \end{macrocode}

% \macro{\childdocforwardprefix}
% The command |\childdocforwardprefix| redirects
% compilation to the main or a child file by means of a pattern.
% The prefix |#1| in the current filename is replaced by |#2|
% and the suffix of the current filename is kept
% (it is assumed that the filename does not contain the substring `|~~~|'
% which is used as a delimiter).
% Compilation is handed over to the new file by |\childdocforward|:
%    \begin{macrocode}
\newcommand{\childdocforwardprefix}[3][]
{
  \begingroup
    \def\childdocextract #2##1~~~{\def\childdoctmp{\childdocforward[#1]{#3##1}}}
    \expandafter\childdocextract\childdocname~~~
    \expandafter
  \endgroup
  \childdoctmp
}
%    \end{macrocode}

% \macro{\childdoc}
% The deprecated macro |\childdoc| is a legacy version of |\childdocmain|:
%    \begin{macrocode}
\newcommand{\childdoc}{\childdocmain}
%    \end{macrocode}

% \macro{\childdocredirect}
% The deprecated macro |\childdocredirect| is a legacy version
% of |\childdocforward| and |\childdocforwardprefix|:
%    \begin{macrocode}
\newcommand{\childdocredirect}[2][]
{
  \begingroup
    \if?#1?
      \def\childdoctmp{\childdocforward{#2}}
    \else
      \def\childdoctmp{\childdocforwardprefix{#1}{#2}}
    \fi
    \expandafter
  \endgroup
  \childdoctmp
}
%    \end{macrocode}

%\iffalse
%</package>
%\fi
%
\endinput
|\\
|\childdocforward[|\textit{main}|]{|\textit{dest}|}|\\
\end{tabular}
\end{center}
%
The argument \textit{dest} is the destination file
(without extension).
It should be the main file or one of the child files.
Note that further \textsf{childdoc} directives
such as |\childdocof| and |\childdocforward|
in the indicated file will be processed in this form.
The optional argument \textit{main}
passes on directly to the main file \textit{main}
while pretending to compile the child \textit{dest}.
This form behaves as if \textit{dest}
issues |\childdocof{|\textit{main}|}| right away,
and no further \textsf{childdoc} directives will be processed.

%%%%%%%%%%%%%%%%%%%%%%%%%%%%%%%%%%%%%%%%
\DescribeMacro{\...prefix}
In the alternative form |\childdocforwardprefix|,
%
\begin{center}
\begin{tabular}{l}
|% \iffalse
%
% childdoc.dtx Copyright (C) 2017-2018 Niklas Beisert
%
% This work may be distributed and/or modified under the
% conditions of the LaTeX Project Public License, either version 1.3
% of this license or (at your option) any later version.
% The latest version of this license is in
%   http://www.latex-project.org/lppl.txt
% and version 1.3 or later is part of all distributions of LaTeX
% version 2005/12/01 or later.
%
% This work has the LPPL maintenance status `maintained'.
%
% The Current Maintainer of this work is Niklas Beisert.
%
% This work consists of the files childdoc.dtx and childdoc.ins
% and the derived files childdoc.def and cdocsamp.tex with
% cdocsch1.tex, cdocsch2.tex, cdocsdrf.tex, cdocsfn1.tex, cdocsfn2.tex.
%
%<package>\ifdefined\childdocmain\endinput\fi
%<package>\ProvidesFile{childdoc.def}[2018/12/30 v2.0 child document driver]
%<samplemain>\ProvidesFile{cdocsamp.tex}[2018/12/30 v2.0 sample for childdoc]
%<*driver>
%\ProvidesFile{childdoc.drv}[2018/12/30 v2.0 childdoc reference manual file]
\PassOptionsToClass{10pt,a4paper}{article}
\documentclass{ltxdoc}

\usepackage[margin=35mm]{geometry}
\usepackage{hyperref}
\usepackage{hyperxmp}
\usepackage[usenames]{color}

\hypersetup{colorlinks=true}
\hypersetup{pdfstartview=FitH}
\hypersetup{pdfpagemode=UseNone}
\hypersetup{pdfsource={}}
\hypersetup{pdflang={en-UK}}
\hypersetup{pdfcopyright={Copyright 2017-2018 Niklas Beisert.
  This work may be distributed and/or modified under the
  conditions of the LaTeX Project Public License, either version 1.3
  of this license or (at your option) any later version.}}
\hypersetup{pdflicenseurl={http://www.latex-project.org/lppl.txt}}
\hypersetup{pdfcontactaddress={ETH Zurich, ITP, HIT K,
  Wolfgang-Pauli-Strasse 27}}
\hypersetup{pdfcontactpostcode={8093}}
\hypersetup{pdfcontactcity={Zurich}}
\hypersetup{pdfcontactcountry={Switzerland}}
\hypersetup{pdfcontactemail={nbeisert@itp.phys.ethz.ch}}
\hypersetup{pdfcontacturl={http://people.phys.ethz.ch/\xmptilde nbeisert/}}

\newcommand{\secref}[1]{\hyperref[#1]{section \ref*{#1}}}

\parskip1ex
\parindent0pt
\let\olditemize\itemize
\def\itemize{\olditemize\parskip0pt}

\begin{document}

\title{The \textsf{childdoc} Package}
\hypersetup{pdftitle={The childdoc Package}}
\author{Niklas Beisert\\[2ex]
  Institut f\"ur Theoretische Physik\\
  Eidgen\"ossische Technische Hochschule Z\"urich\\
  Wolfgang-Pauli-Strasse 27, 8093 Z\"urich, Switzerland\\[1ex]
  \href{mailto:nbeisert@itp.phys.ethz.ch}
  {\texttt{nbeisert@itp.phys.ethz.ch}}}
\hypersetup{pdfauthor={Niklas Beisert}}
\hypersetup{pdfsubject={Manual for the LaTeX2e Package childdoc}}
\date{30 December 2018, \textsf{v2.0}}
\maketitle

\begin{abstract}\noindent
\textsf{childdoc} is a \LaTeXe{} package
that enables the direct compilation
of document sections included by |\include|
to individual files.
\end{abstract}

\begingroup
\parskip0ex
\tableofcontents
\endgroup

%%%%%%%%%%%%%%%%%%%%%%%%%%%%%%%%%%%%%%%%%%%%%%%%%%%%%%%%%%%%%%%%%%%%%%%%%%%%%%%%
%%%%%%%%%%%%%%%%%%%%%%%%%%%%%%%%%%%%%%%%%%%%%%%%%%%%%%%%%%%%%%%%%%%%%%%%%%%%%%%%
\section{Introduction}

\LaTeX{} provides a mechanism to structure a large document (such as a book)
into a main file and several child files (containing the chapters)
using the |\include| command.
This mechanism is beneficial for documents
which span hundreds of pages in order to
make the source file(s) more manageable.
Moreover, compilation can be restricted to
selected child files by means of the |\includeonly| command.
The latter feature can be used to reduce the compilation time while editing
(this was significantly more useful in the earlier days of \LaTeX{})
or to generate a smaller document which is easier to navigate.
Another application of |\includeonly| is to generate
documents consisting of selected parts of the complete document.

However, there are a few drawbacks of the plain |\include| mechanism:
\begin{itemize}
\item
The child files cannot be compiled on their own,
they can only be compiled via the main file.
A naive editing environment
(such as a text editor with an option
to have the current file processed by \LaTeX)
may require one to switch to the main file before compiling;
attempting to compile the child file produces errors.
\item
The main file must be modified (each time)
to adjust the |\includeonly| command
to the present needs. This easily leaves the main file in a messy state.
\item
The generated document will always carry the filename
of the main document. This is inconvenient if
several child files are to be compiled and
to be kept for distribution.
\end{itemize}

The present package provides a simple interface
to make child files individually compilable by \LaTeX{}.
Compiling a child file then has the same effect as compiling
the main file with an |\includeonly| command
to select the appropriate child.
Moreover the generated document will carry the name of the child
rather than the main file.
This resolves all three above issues.

This feature is meant to make the editing of books,
thesis documents and lecture notes somewhat more convenient.
However, the package can also be used efficiently for
composing a series of documents (such as exercise sheets)
which are typically distributed individually.
It then assists the author in generating the individual documents
(potentially in different versions)
as well as a document containing the collected series.
Another application is in developing style files
or other kinds of included material
where compilation of the style file could redirect
to a sample or test file.

%%%%%%%%%%%%%%%%%%%%%%%%%%%%%%%%%%%%%%%%%%%%%%%%%%%%%%%%%%%%%%%%%%%%%%%%%%%%%%%%
%%%%%%%%%%%%%%%%%%%%%%%%%%%%%%%%%%%%%%%%%%%%%%%%%%%%%%%%%%%%%%%%%%%%%%%%%%%%%%%%
\section{Usage}

First of all, the package \textsf{childdoc} is \emph{not} a standard
\LaTeXe{} |.sty| style file! Therefore it needs to be invoked in
a non-standard way.

%%%%%%%%%%%%%%%%%%%%%%%%%%%%%%%%%%%%%%%%%%%%%%%%%%%%%%%%%%%%%%%%%%%%%%%%%%%%%%%%
\subsection{Included Files}
\label{sec:include}

%%%%%%%%%%%%%%%%%%%%%%%%%%%%%%%%%%%%%%%%
\DescribeMacro{\childdocmain}
To use the package, add the commands
\begin{center}
\begin{tabular}{l}
|\input{childdoc.def}|\\
|\childdocmain{}|\\
\end{tabular}
\end{center}
at the very top of the main \LaTeX{} file,
in particular \emph{before} the |\documentclass| statement!
The argument of |\childdocmain| should be left empty
(but it must be present).

%%%%%%%%%%%%%%%%%%%%%%%%%%%%%%%%%%%%%%%%
\DescribeMacro{\childdocof}
Furthermore, add the commands
\begin{center}
\begin{tabular}{l}
|\input{childdoc.def}|\\
|\childdocof{|\textit{main}|}|\\
\end{tabular}
\end{center}
at the top of every child file \textit{child}
which is included by |\include{|\textit{child}|}|
from within the main file
(or at least for those files to be compiled individually).
The argument \textit{main} must be the filename of the main file.

There are a couple of
considerations in setting up the main and child documents:

%%%%%%%%%%%%%%%%%%%%%%%%%%%%%%%%%%%%%%%%
\paragraph{Restrictions.}

Please note the following restrictions:
\begin{itemize}
\item
|\childdocmain| must be called with one argument \textit{main}
to ensure compatibility with earlier version of the package.
It must either be empty (|\childdocmain{}|)
or precisely match the filename of the main file in which it is specified.
See \secref{sec:detection} for further information.
\item
The filename \textit{main} must be specified without the |.tex| extension.
\item
The filename \textit{main} is case sensitive
(even in case-insensitive file systems)
due to internal string comparison.
\item
The argument \textit{main} should be fully expanded, it cannot be a macro.
\item
Subdirectories and special characters should be avoided in filenames.
\item
The command |\childdocmain{|\textit{main}|}| must be followed by a whitespace.
It should not be followed immediately by another command
or by a comment mark `|%|'.
This is because the \TeX{} parser reads the token immediately following
the argument of |\childdocmain| and puts it
at the beginning of every child section;
however, a white\-space is ignored.
\end{itemize}

%%%%%%%%%%%%%%%%%%%%%%%%%%%%%%%%%%%%%%%%
\paragraph{Content of Main File.}

It is advisable to place all content in the child files included by |\include|.
Any output contained in the main file will appear in all child documents
unless suppressed manually;
it cannot be suppressed automatically by the |\includeonly| directive
and thus should normally be avoided.
A method to include some content in the main file
by means of conditional processing is described in \secref{sec:conditional}.

%%%%%%%%%%%%%%%%%%%%%%%%%%%%%%%%%%%%%%%%
\paragraph{Page Numbering.}

When only a part of the document is compiled,
the appropriate numbering of pages
(as well as other status parameters)
is determined from the |.aux| files.
The latter contain information from previous passes.
However this information needs to propagate through
all intermediate child documents.
Therefore the page numbering in child documents may well
be inconsistent until the complete document is compiled at least once.

A useful (if unconventional) way to always ensure a consistent
page numbering is to restart the numbering in each child document
and denote the pages by `\textit{child}|.|\textit{page}'
where \textit{child} represents the chapter/section number of the child file.
This can be achieved by the command
|\numberwithin{page}{|\textit{child}|}|
of the \textsf{amsmath} package
where \textit{child} can be |chapter| or |section|
depending on the chosen structuring.
Alternatively, one can modify the macro |\thepage| appropriately
and reset the counter |page| at the start of each child file.

%%%%%%%%%%%%%%%%%%%%%%%%%%%%%%%%%%%%%%%%%%%%%%%%%%%%%%%%%%%%%%%%%%%%%%%%%%%%%%%%
\subsection{Conditional Processing}
\label{sec:conditional}

The package provides a mechanism to compile different versions
of a document. To customise the versions further some conditional processing
can come in handy to distinguish which version is being compiled.
The package provides two macros to describe the compilation context:

%%%%%%%%%%%%%%%%%%%%%%%%%%%%%%%%%%%%%%%%
\DescribeMacro{\ifchilddoc}
The conditional |\ifchilddoc| distinguishes between the compilation of
child documents and the main document:
%
\begin{center}
|\ifchilddoc |\textit{child-code}| |[|\||else |\textit{main-code}]| \||fi|
\end{center}

%%%%%%%%%%%%%%%%%%%%%%%%%%%%%%%%%%%%%%%%
\DescribeMacro{\childdocname}
\DescribeMacro{\childdocjob}
The macro |\childdocname| contains the filename (without extension)
of the main or child file being processed.
Note that |\childdocjob| will always contain the name of the main file.

%%%%%%%%%%%%%%%%%%%%%%%%%%%%%%%%%%%%%%%%
\paragraph{Title Page.}

Conditional processing can be used to include a title or banner page
in the main document when proper precautions are taken.
Importantly, the code in the main file should ensure that the page counter
(as well as other status parameters which are stored in the |.aux| files)
takes the same value after the conditional processing.
Otherwise the page numbers may take divergent values
depending on which part is compiled.

For example, a title page could be declared by:
%
\begin{center}
\begin{tabular}{l}
|\ifchilddoc\||else|\\
|\addtocounter{page}{-1}|\\
\textit{code for title page}\\
|\newpage|\\
|\||fi|
\end{tabular}
\end{center}
%
A banner page for the child documents can be generated by:
%
\begin{center}
\begin{tabular}{l}
|\ifchilddoc|\\
|\addtocounter{page}{-1}|\\
\textit{code for banner page}\\
|\newpage|\\
|\||fi|
\end{tabular}
\end{center}
%
Here one could write a message such as:
\begin{center}
|This is the part \childdocname{} of \childdocjob{}.|
\end{center}

%%%%%%%%%%%%%%%%%%%%%%%%%%%%%%%%%%%%%%%%%%%%%%%%%%%%%%%%%%%%%%%%%%%%%%%%%%%%%%%%
\subsection{Flags}
\label{sec:flags}

The package makes it easy to generate different versions
of the main or child documents.
To this end compilation flags can be defined
and assigned different default values.
They will be particularly useful in conjunction
with the forwarding mechanism described in \secref{sec:forward}.

For example, it may be useful to have a flag |\version|
which can be set to |draft| or |final|.
The document source will contain some conditional code
depending on the value of |\version|.
Suppose further, the flag should default to |final| for the main file
and to |draft| for child files
which is a natural assignment for editing the document.
This is achieved by placing the following code
in the preamble of the main document
(below the |\childdocmain| directive):
%
\begin{center}
\begin{tabular}{l}
|\ifchilddoc|\\
|\providecommand{\version}{draft}|\\
|\||else|\\
|\providecommand{\version}{final}|\\
|\||fi|
\end{tabular}
\end{center}
%
The definition by |\providecommand| makes sure
that previous definitions are not overwritten.
Further statements |\providecommand{\version}{...}|
can thus be added before the above code to override it.

For the main file, one might add a line
(between |\childdocmain| and the above block)
%
\begin{center}
|%\ifchilddoc\||else\providecommand{\version}{draft}\||fi|
\end{center}
%
which can be uncommented to produce a draft version.
Likewise one can add a line to the very top of a child file
(above the |\childdocof{|\textit{main}|}| directive)
%
\begin{center}
|%\providecommand{\version}{final}|
\end{center}
%
which can be uncommented to produce the final version of this child document.

%%%%%%%%%%%%%%%%%%%%%%%%%%%%%%%%%%%%%%%%%%%%%%%%%%%%%%%%%%%%%%%%%%%%%%%%%%%%%%%%
\subsection{Forwarding}
\label{sec:forward}

Different versions of the main or child documents
using compilation flags as described in \secref{sec:flags}
can be (permanently) stored in different files
for convenient compilation, viewing and distribution.
To this end, the package defines a command
to pass on compilation to a different file:

%%%%%%%%%%%%%%%%%%%%%%%%%%%%%%%%%%%%%%%%
\DescribeMacro{\childdocforward}
The command |\childdocforward| redirects processing to
another source file:
%
\begin{center}
\begin{tabular}{l}
|\input{childdoc.def}|\\
|\childdocforward[|\textit{main}|]{|\textit{dest}|}|\\
\end{tabular}
\end{center}
%
The argument \textit{dest} is the destination file
(without extension).
It should be the main file or one of the child files.
Note that further \textsf{childdoc} directives
such as |\childdocof| and |\childdocforward|
in the indicated file will be processed in this form.
The optional argument \textit{main}
passes on directly to the main file \textit{main}
while pretending to compile the child \textit{dest}.
This form behaves as if \textit{dest}
issues |\childdocof{|\textit{main}|}| right away,
and no further \textsf{childdoc} directives will be processed.

%%%%%%%%%%%%%%%%%%%%%%%%%%%%%%%%%%%%%%%%
\DescribeMacro{\...prefix}
In the alternative form |\childdocforwardprefix|,
%
\begin{center}
\begin{tabular}{l}
|\input{childdoc.def}|\\
|\childdocforwardprefix[|\textit{main}|]{|\textit{prefix}|}{|\textit{dest}|}|
\end{tabular}
\end{center}
%
the destination file is determined by a pattern
depending on the current file:
To make this work, the current file must be called
`{\textit{prefix}\hspace{0.2em}\textit{suffix}}'
with \textit{prefix} matching precisely the argument.
Processing is then passed on to the file
`{\textit{dest}\hspace{0.2em}\textit{suffix}}'.
Surely, the same effect is achieved by
directly specifying the
argument `{\textit{dest}\hspace{0.2em}\textit{suffix}}'
in the first form.
However, that requires to set up a different file
for each child. With the alternative form of the command
all these files can have exactly the same content
which simplifies setting them up and maintaining them.

For example, the following file |draft.tex|
with a compilation flag |\version| as described in \secref{sec:flags}
compiles the main document as a draft:
%
\begin{center}
\begin{tabular}{l}
|\def\version{draft}|\\
|\input{childdoc.def}|\\
|\childdocforward{|\textit{main}|}|
\end{tabular}
\end{center}
%
Likewise, the following files |final|\textit{nn}|.tex|
compile the final version of the child document
|child|\textit{nn}|.tex|:
%
\begin{center}
\begin{tabular}{l}
|\def\version{final}|\\
|\input{childdoc.def}|\\
|\childdocforwardprefix{final}{child}|
\end{tabular}
\end{center}
%

Note that when several versions of a main file and/or of each child file
are to be generated, it may be convenient to set up a |Makefile| or
shell script to automatise the process.

%%%%%%%%%%%%%%%%%%%%%%%%%%%%%%%%%%%%%%%%%%%%%%%%%%%%%%%%%%%%%%%%%%%%%%%%%%%%%%%%
\subsection{Command Line Processing}
\label{sec:commandline}

The effect of redirection files can also be achieved by invoking
the \LaTeX{} compiler with a more elaborate command line.
Most conveniently this should be done as part
of a shell script or a |Makefile|.

When using \textsf{childdoc} in the main file, the following
command lines effectively perform a redirection
(note that depending on the shell being used,
backslashes may have to be doubled: `|\|' $\to$ `|\\|'):
%
\begin{center}
|... -jobname "|\textit{target}|" |\\|"|[\textit{flags}]%
|\input{childdoc.def}\childdocforward[|\textit{main}|]{|\textit{dest}|}"|
\end{center}
%
Here \textit{target} is the name of the output file,
\textit{main} is the name of the main file
and \textit{dest} is the name of the main or child file to be processed
(all filenames without extensions).
The optional argument \textit{main} can be omitted
if \textit{main} matches \textit{dest}.
Optionally, compilation \textit{flags} can be defined via |\def| commands.
This command line makes the \TeX{} engine believe
it is compiling the file \textit{target}
whose content is specified as the latter parameter.
The provided code then forwards the processing to
\textit{main} or \textit{dest} as described in \secref{sec:forward}.

%%%%%%%%%%%%%%%%%%%%%%%%%%%%%%%%%%%%%%%%%%%%%%%%%%%%%%%%%%%%%%%%%%%%%%%%%%%%%%%%
\subsection{Include by Input}
\label{sec:input}

Including child documents by |\include| has some restrictions by design.
Most notably, the content of a child document always occupies
its own set of pages; pages cannot be shared between child documents.
Usually, this behaviour makes perfect sense
because each child document contain an essential part of the document.
However, in some situations it may be desirable to compose
a document from a collection of parts
without having mandatory page breaks between then.
For this case, the package
provides a mechanism to include parts
by |\input| which can also be processed individually.
However, by construction this mechanism
requires manual handling of the content to be output.

%%%%%%%%%%%%%%%%%%%%%%%%%%%%%%%%%%%%%%%%
\DescribeMacro{\ifchilddocmanual}
The main file should be prepared as usual, see \secref{sec:include}.
However, the document body must make a distinction
between processing of an individual part and of the main document, e.g.:
%
\begin{center}
\begin{tabular}{l}
|\ifchilddocmanual|\\
|\input{\childdocname}|\\
|\||else|\\
\textit{document body with }|\input{|\textit{part}|}|\\
|\||fi|
\end{tabular}
\end{center}
%
The conditional |\ifchilddocmanual| is true whenever
a part to be included by |\input| is being compiled,
and the name of the part is stored in |\childdocname|.

%%%%%%%%%%%%%%%%%%%%%%%%%%%%%%%%%%%%%%%%
\DescribeMacro{\childdocby}
Each part to be included by |\input| should start with:
%
\begin{center}
\begin{tabular}{l}
|\input{childdoc.def}|\\
|\childdocby{|\textit{main}|}|\\
\end{tabular}
\end{center}
%
The directive |\childdocby| is similar to |\childdocof|
described in \secref{sec:include},
but the subsequent selection of content must be done manually.
To that end, both |\ifchilddoc| and |\ifchilddocmanual|
will be true upon processing of a part,
and the name of the part is stored in |\childdocname|.
Note that |\jobname| will be set to the filename of the current part
so that each part receives an individual |.aux| file
that does not interfere with the |.aux| file(s) of the main document.
This behaviour can be altered by the alternative form
|\childdocby[*]{|\textit{main}|}| (with a non-empty optional argument)
which uses the |.aux| file of the main document
by setting |\jobname| to \textit{main}.

%%%%%%%%%%%%%%%%%%%%%%%%%%%%%%%%%%%%%%%%%%%%%%%%%%%%%%%%%%%%%%%%%%%%%%%%%%%%%%%%
\subsection{Driver Development}
\label{sec:driver}

The \textsf{childdoc} mechanism can also be use for the development
of definition files such as \LaTeX{} styles or classes.
This case differs from the above setup with multiple parts
included by |\include| in that no |\includeonly| should be invoked.
This can be achieved by starting the include file
(before |\ProvidesPackage|) with:
%
\begin{center}
\begin{tabular}{l}
|\input{childdoc.def}|\\
|\childdocforward{|\textit{main}|}|\\
\end{tabular}
\end{center}
%
or alternatively with:
%
\begin{center}
\begin{tabular}{l}
|\input{childdoc.def}|\\
|\childdocby{|\textit{main}|}|\\
\end{tabular}
\end{center}
%
Both forms have slightly different effects as described above.
The main file is prepared as usual, see \secref{sec:include}.

%%%%%%%%%%%%%%%%%%%%%%%%%%%%%%%%%%%%%%%%%%%%%%%%%%%%%%%%%%%%%%%%%%%%%%%%%%%%%%%%
\subsection{Legacy Detection}
\label{sec:detection}

The directive |\childdocmain| in the main file can detect
whether the complete document or merely a child is to be compiled
even without using the directive |\childdocof|.
This method is deprecated because it is less robust
and there is no compelling reason to use it;
it is merely provided for backward compatibility
and it may be removed in future versions.

If the detection mechanism is to be used,
it is mandatory to correctly specify
the filename of the main file as the argument of |\childdocmain|:
%
\begin{center}
\begin{tabular}{l}
|\input{childdoc.def}|\\
|\childdocmain{|\textit{main}|}|\\
\end{tabular}
\end{center}
%
If |\jobname| does not match the argument \textit{main} of |\childdocmain|,
it is assumed that |\jobname| points to the child file to be compiled.
When using |\childdocmain| with the main file specified as argument,
it suffices to start a child file
with just |\input{|\textit{main}|}|
without loading of the package and using |\childdocof|.
If instead all processing is done
with the appropriate \textsf{childdoc} directives,
the argument of \textit{main} of |\childdocmain| can be empty.

An alternative version of the command line processing described
in \secref{sec:commandline} using the detection mechanism reads:
%
\begin{center}
|... -jobname "|\textit{target}|" "|[\textit{flags}]%
[|\def\jobname{|\textit{dest}|}|]|\input{|\textit{main}|}"|
\end{center}

%%%%%%%%%%%%%%%%%%%%%%%%%%%%%%%%%%%%%%%%%%%%%%%%%%%%%%%%%%%%%%%%%%%%%%%%%%%%%%%%
\subsection{Manual Code}
\label{sec:manual}

In case one cannot be certain whether the definitions file |childdoc.def|
is installed on the target \TeX{} distribution
and one prefers not to ship it,
it is conceivable to paste a few relevant commands into the sources.

To that end, drop all statements |\input{childdoc.def}|
and perform the replacements as outlined below.
Instead of |\childdocmain{|\textit{main}|}| add the following code
to the top of the main file:
%
\begin{center}
\begin{tabular}{l}
|\||ifdefined\childdocname\endinput\||fi\newif\ifchilddoc|\\
|\edef\childdocname{\scantokens\expandafter{\jobname\noexpand}}|\\
|\def\childdocmain{|\textit{main}|}\||ifx\childdocmain\childdocname\||else|\\
|\childdoctrue\includeonly{\childdocname}\let\jobname\childdocmain\||fi|\\
\end{tabular}
\end{center}
%
Instead of |\childdocof{|\textit{main}|}| just include the main file
at the top of each child file:
%
\begin{center}
|\input{|\textit{main}|}|
\end{center}
%
A simple redirection |\childdocforward{|\textit{dest}|}| is achieved by:
%
\begin{center}
|\def\jobname{|\textit{dest}|}\input{\jobname}|
\end{center}
%
The redirection with prefix
|\childdocforwardprefix[|\textit{prefix}|]{|\textit{dest}|}|
is accomplished by:
%
\begin{center}
\begin{tabular}{l}
|{\edef\jobname{\scantokens\expandafter{\jobname\noexpand}}|\\
|\def\redirectjob |\textit{prefix}|#1~~~{\gdef\jobname{|\textit{dest}|#1}}|\\
|\expandafter\redirectjob\jobname~~~}\input{\jobname}|
\end{tabular}
\end{center}

In an alternative approach,
child documents can be compiled by a specific command line
without additional code or specific definitions:
%
\begin{center}
|... -jobname "|\textit{target}|" "|[\textit{flags}]%
|\includeonly{|\textit{dest}|}\input{|\textit{main}|}"|
\end{center}
%

%%%%%%%%%%%%%%%%%%%%%%%%%%%%%%%%%%%%%%%%%%%%%%%%%%%%%%%%%%%%%%%%%%%%%%%%%%%%%%%%
%%%%%%%%%%%%%%%%%%%%%%%%%%%%%%%%%%%%%%%%%%%%%%%%%%%%%%%%%%%%%%%%%%%%%%%%%%%%%%%%
\section{Information}

%%%%%%%%%%%%%%%%%%%%%%%%%%%%%%%%%%%%%%%%%%%%%%%%%%%%%%%%%%%%%%%%%%%%%%%%%%%%%%%%
\subsection{Copyright}

Copyright \copyright{} 2017--2018 Niklas Beisert

This work may be distributed and/or modified under the
conditions of the \LaTeX{} Project Public License, either version 1.3
of this license or (at your option) any later version.
The latest version of this license is in
  \url{http://www.latex-project.org/lppl.txt}
and version 1.3 or later is part of all distributions of \LaTeX{}
version 2005/12/01 or later.

This work has the LPPL maintenance status `maintained'.

The Current Maintainer of this work is Niklas Beisert.

This work consists of the files |README.txt|, |childdoc.ins| and |childdoc.dtx|
as well as the derived files |childdoc.def|, |cdocsamp.tex|
with |cdocsch1.tex|, |cdocsch2.tex|, |cdocspt3.tex|, |cdocspt4.tex|,
|cdocsdrf.tex|, |cdocsfn1.tex|, |cdocsfn2.tex|
as well as |childdoc.pdf|.

%%%%%%%%%%%%%%%%%%%%%%%%%%%%%%%%%%%%%%%%%%%%%%%%%%%%%%%%%%%%%%%%%%%%%%%%%%%%%%%%
\subsection{Files and Installation}

The package consists of the files:
%
\begin{center}
\begin{tabular}{ll}
    |README.txt|   & readme file \\
    |childdoc.ins| & installation file \\
    |childdoc.dtx| & source file \\
    |childdoc.def| & definition file \\
    |cdocsamp.tex| & sample main file \\
    |cdocsch1.tex| & sample include file \\
    |cdocsch2.tex| & sample include file \\
    |cdocspt3.tex| & sample part file \\
    |cdocspt4.tex| & sample part file \\
    |cdocsdrf.tex| & sample redirection file \\
    |cdocsfn1.tex| & sample redirection file \\
    |cdocsfn2.tex| & sample redirection file \\
    |childdoc.pdf| & manual
\end{tabular}
\end{center}
%
The distribution consists of the files
|README.txt|, |childdoc.ins| and |childdoc.dtx|.
%
\begin{itemize}
\item
Run (pdf)\LaTeX{} on |childdoc.dtx|
to compile the manual |childdoc.pdf| (this file).
\item
Run \LaTeX{} on |childdoc.ins| to create the definitions file |childdoc.def|
and the sample |cdocsamp.tex| with include files
|cdocsch1.tex|, |cdocsch2.tex|, |cdocspt3.tex|, |cdocspt4.tex|,
|cdocsdrf.tex|, |cdocsfn1.tex|, |cdocsfn2.tex|.
Then copy the file |childdoc.def| to an appropriate directory of your \LaTeX{}
distribution, e.g.\ \textit{texmf-root}|/tex/latex/childdoc|.
\end{itemize}

%%%%%%%%%%%%%%%%%%%%%%%%%%%%%%%%%%%%%%%%%%%%%%%%%%%%%%%%%%%%%%%%%%%%%%%%%%%%%%%%
\subsection{Related CTAN Packages}

There are several other packages which offer a similar functionality:
%
\begin{itemize}
\item
The packages
\href{http://ctan.org/pkg/docmute}{\textsf{docmute}},
\href{http://ctan.org/pkg/includex}{\textsf{includex}} and
\href{http://ctan.org/pkg/standalone}{\textsf{standalone}}
provide commands to include only the document body of
a child file thus allowing both files to be compiled individually.
\item
The packages \href{http://ctan.org/pkg/subdocs}{\textsf{subdocs}}
and \href{http://ctan.org/pkg/subfiles}{\textsf{subfiles}}
provide structures in which the main and child documents can be
encapsulated and allowing them to be compiled individually.
The inclusion mechanism is different from the conventional |\include|.
\item
The package \href{http://ctan.org/pkg/combine}{\textsf{combine}}
is an elaborate solution to combine several documents into one.
\end{itemize}
%
See also the CTAN topic \href{http://ctan.org/topic/subdocs}{\textsf{subdocs}}
for further related packages.
The present package differs from the above solutions in that
a document structure constructed with the conventional |\include| mechanism
just needs two extra commands at the top of every file
such that all constituent files can be compiled individually.

%%%%%%%%%%%%%%%%%%%%%%%%%%%%%%%%%%%%%%%%%%%%%%%%%%%%%%%%%%%%%%%%%%%%%%%%%%%%%%%%
%\subsection{Feature Suggestions}
%
%The following is a list of features which may be useful for future
%versions of this package:
%%
%\begin{itemize}
%\item
%\ldots
%\end{itemize}

%%%%%%%%%%%%%%%%%%%%%%%%%%%%%%%%%%%%%%%%%%%%%%%%%%%%%%%%%%%%%%%%%%%%%%%%%%%%%%%%
\subsection{Revision History}

%%%%%%%%%%%%%%%%%%%%%%%%%%%%%%%%%%%%%%%%
\paragraph{v2.0:} 2018/12/30

\begin{itemize}
\item
immediate forward processing
\item
added |\childdocby| mechanism
\item
manual restructured
\end{itemize}

%%%%%%%%%%%%%%%%%%%%%%%%%%%%%%%%%%%%%%%%
\paragraph{v1.6:} 2018/01/17

\begin{itemize}
\item
application for development of include files
\item
corrections to manual
\end{itemize}

%%%%%%%%%%%%%%%%%%%%%%%%%%%%%%%%%%%%%%%%
\paragraph{v1.5:} 2017/05/21

\begin{itemize}
\item
more complete structuring introduced
\item
|\childdocof| introduced
\item
|\childdoc| renamed to |\childdocmain|
\item
|\childredirect| renamed to |\childdocforward| and |\childdocforwardprefix|
and functionality expanded
\end{itemize}

%%%%%%%%%%%%%%%%%%%%%%%%%%%%%%%%%%%%%%%%
\paragraph{v1.0:} 2017/04/27

\begin{itemize}
\item
manual and install package
\item
first version published on CTAN
\end{itemize}

%%%%%%%%%%%%%%%%%%%%%%%%%%%%%%%%%%%%%%%%
\paragraph{v0.6:} 2017/04/26

\begin{itemize}
\item
redirection mechanism added
\end{itemize}

%%%%%%%%%%%%%%%%%%%%%%%%%%%%%%%%%%%%%%%%
\paragraph{v0.5:} 2017/04/26

\begin{itemize}
\item
functionality in definition file
\end{itemize}


%%%%%%%%%%%%%%%%%%%%%%%%%%%%%%%%%%%%%%%%%%%%%%%%%%%%%%%%%%%%%%%%%%%%%%%%%%%%%%%%
%%%%%%%%%%%%%%%%%%%%%%%%%%%%%%%%%%%%%%%%%%%%%%%%%%%%%%%%%%%%%%%%%%%%%%%%%%%%%%%%
%%%%%%%%%%%%%%%%%%%%%%%%%%%%%%%%%%%%%%%%%%%%%%%%%%%%%%%%%%%%%%%%%%%%%%%%%%%%%%%%
\appendix

\settowidth\MacroIndent{\rmfamily\scriptsize 000\ }

 \DocInput{childdoc.dtx}

\end{document}
%</driver>
% \fi
%
% %%%%%%%%%%%%%%%%%%%%%%%%%%%%%%%%%%%%%%%%%%%%%%%%%%%%%%%%%%%%%%%%%%%%%%%%%%%%%%
% %%%%%%%%%%%%%%%%%%%%%%%%%%%%%%%%%%%%%%%%%%%%%%%%%%%%%%%%%%%%%%%%%%%%%%%%%%%%%%
% \section{Sample}
%\iffalse
%<*samplemain>
%\fi
%
% The following presents a sample document
% with two chapters, two parts, a title page,
% a compile flag as well as three forwarding files to set the flag.
% It consists of eight |.tex| files:
% \begin{center}
% \begin{tabular}{ll}
% |cdocsamp.tex|&main file\\
% |cdocsch1.tex|&include file for chapter 1\\
% |cdocsch2.tex|&include file for chapter 2\\
% |cdocspt3.tex|&include file for part 3\\
% |cdocspt4.tex|&include file for part 4\\
% |cdocsdrf.tex|&forwarding file for main file in draft mode\\
% |cdocsfi1.tex|&forwarding file for final version of chapter 1\\
% |cdocsfi2.tex|&forwarding file for final version of chapter 2\\
% \end{tabular}
% \end{center}
% Each of the eight files can be compiled directly by the \LaTeX{} compiler.
%
% %%%%%%%%%%%%%%%%%%%%%%%%%%%%%%%%%%%%%%
% \paragraph{Main File.}
%
% The main file is called |cdocsamp.tex|.
%
% Load the \textsf{childdoc} definitions and
% declare the filename for the main document:
%    \begin{macrocode}
\input{childdoc.def}
\childdocmain{}
%    \end{macrocode}

% Optional override for |\version| flag:
%    \begin{macrocode}
%%\ifchilddoc\else\providecommand{\version}{draft}\fi
%    \end{macrocode}

% Define the default values for the |\version| flag
% (|final| for the main file and |draft| for childs):
%    \begin{macrocode}
\ifchilddoc
\providecommand{\version}{draft}
\else
\providecommand{\version}{final}
\fi
%    \end{macrocode}

% Load the standard document class:
%    \begin{macrocode}
\documentclass[12pt]{article}
%    \end{macrocode}

% Start the document body:
%    \begin{macrocode}
\begin{document}
%    \end{macrocode}

% Declare a title page.
% Print title, part of document being processed and version flag:
%    \begin{macrocode}
\addtocounter{page}{-1}
\begin{center}
{\LARGE\bfseries{}childdoc example\par}
\vspace{1cm}
\ifchilddoc
\ifchilddocmanual part\else chapter\fi:
`\childdocname' of `\childdocjob'\par
\else
main document: `\childdocjob'\par
\fi
version: \version\par
\end{center}
\newpage
%    \end{macrocode}

% Manually include selected file,
% otherwise process as usual:
%    \begin{macrocode}
\ifchilddocmanual
\section*{part `\childdocname'}
\input{\childdocname}
\else
%    \end{macrocode}

% Include the two chapters:
%    \begin{macrocode}
\include{cdocsch1}
\include{cdocsch2}
%    \end{macrocode}

% Include the two parts unless only chapters should be displayed:
%    \begin{macrocode}
\ifchilddoc\else
\section{part three}
\input{cdocspt3}
\section{part four}
\input{cdocspt4}
\fi
%    \end{macrocode}

% Process as usual until here:
%    \begin{macrocode}
\fi
%    \end{macrocode}

% End of document body:
%    \begin{macrocode}
\end{document}
%    \end{macrocode}
%\iffalse
%</samplemain>
%\fi
%
% %%%%%%%%%%%%%%%%%%%%%%%%%%%%%%%%%%%%%%
% \paragraph{Chapter Include Files.}
%
% The include files are called |cdocsch1.tex| and |cdocsch2.tex|.
%
%\iffalse
%<*samplechap1|samplechap2>
%\fi

% Optional override for |\version| flag:
%    \begin{macrocode}
%%\providecommand{\version}{final}
%    \end{macrocode}

% Include the main document:
%    \begin{macrocode}
\input{childdoc.def}
\childdocof{cdocsamp}
%    \end{macrocode}

%\iffalse
%</samplechap1|samplechap2>
%\fi
%
%\iffalse
%<*samplechap1>
%\fi
% Some text for chapter 1:
%    \begin{macrocode}
\section{one}
some text in chapter one
%    \end{macrocode}

%\iffalse
%</samplechap1>
%\fi
% Some text for chapter 2:
%\iffalse
%<*samplechap2>
%\fi
%    \begin{macrocode}
\section{two}
more text in chapter two
%    \end{macrocode}

%\iffalse
%</samplechap2>
%\fi
%
% %%%%%%%%%%%%%%%%%%%%%%%%%%%%%%%%%%%%%%
% \paragraph{Part Include Files.}
%
% The include files are called |cdocspt3.tex| and |cdocspt4.tex|.
%
%\iffalse
%<*samplepart3|samplepart4>
%\fi

% Optional override for |\version| flag:
%    \begin{macrocode}
%%\providecommand{\version}{final}
%    \end{macrocode}

% Include the main document:
%    \begin{macrocode}
\input{childdoc.def}
\childdocby{cdocsamp}
%    \end{macrocode}

%\iffalse
%</samplepart3|samplepart4>
%\fi
%
%\iffalse
%<*samplepart3>
%\fi
% Some text for part 3:
%    \begin{macrocode}
some text in part three
%    \end{macrocode}

%\iffalse
%</samplepart3>
%\fi
% Some text for part 4:
%\iffalse
%<*samplepart4>
%\fi
%    \begin{macrocode}
more text in part four
%    \end{macrocode}

%\iffalse
%</samplepart4>
%\fi
%
% %%%%%%%%%%%%%%%%%%%%%%%%%%%%%%%%%%%%%%
% \paragraph{Forwarding for a Complete Draft.}
%
% The following forwarding file |cdocsdrf.tex|
% compiles the main document in draft mode:
%\iffalse
%<*sampledraft>
%\fi
%    \begin{macrocode}
\def\version{draft}
\input{childdoc.def}
\childdocforward{cdocsamp}
%    \end{macrocode}

%\iffalse
%</sampledraft>
%\fi
%
% %%%%%%%%%%%%%%%%%%%%%%%%%%%%%%%%%%%%%%
% \paragraph{Forwarding for Final Version of the Chapters.}
%
% The following forwarding files |cdocsfn1.tex| and |cdocsfn2.tex|
% (with identical content)
% compile the final versions of the child documents
% |cdocsch1.tex| and |cdocsch2.tex|, respectively:
%\iffalse
%<*samplefinal>
%\fi
%    \begin{macrocode}
\def\version{final}
\input{childdoc.def}
\childdocforwardprefix[cdocsamp]{cdocsfn}{cdocsch}
%    \end{macrocode}

%\iffalse
%</samplefinal>
%\fi
%
% %%%%%%%%%%%%%%%%%%%%%%%%%%%%%%%%%%%%%%
% \paragraph{Command Line Processing.}
%
% The following three command lines generate the output files
% |cdocscld|, |cdocscl1| and |cdocscl2|
% which should be identical to
% |cdocsdrf|, |cdocsch1| and |cdocsfn2|, respectively:
% \begin{center}
% \begin{tabular}{l}
% |latex -jobname cdocscld \|\\
% |  "\def\version{draft}\input{childdoc.def}\childdocforward{cdocsamp}"|\\
% |latex -jobname cdocscl1 \|\\
% |  "\input{childdoc.def}\childdocforward[cdocsamp]{cdocsch1}"|\\
% |latex -jobname cdocscl2 \|\\
% |  "\def\version{final}\input{childdoc.def}\childdocforward{cdocsch2}"|
% \end{tabular}
% \end{center}
% Note that the trailing backslash on each first line
% merely continues the input to the second line
% (for convenient cut ant paste).
% Furthermore, the command |latex| can be replaced by any
% of its alternative versions such as |pdflatex|.
%
% %%%%%%%%%%%%%%%%%%%%%%%%%%%%%%%%%%%%%%%%%%%%%%%%%%%%%%%%%%%%%%%%%%%%%%%%%%%%%%
% %%%%%%%%%%%%%%%%%%%%%%%%%%%%%%%%%%%%%%%%%%%%%%%%%%%%%%%%%%%%%%%%%%%%%%%%%%%%%%
% \section{Implementation}
%\iffalse
%<*package>
%\fi
%
% This section describes the definitions file |childdoc.def|.

% The definitions cannot be loaded using |\usepackage| or |\RequirePackage|
% which has a mechanism to prevent loading a style file more than once.
% When loading the definitions by means of |\input|
% multiple instances have to be prevented manually:
%\iffalse
%This code needs to be before the `\ProvidesFile' directive
%which is defined at the beginning of this file.
%Therefore it is also placed there and commented out here.
%</package>
%<*discard>
%\fi
%    \begin{macrocode}
\ifdefined\childdocmain\endinput\fi
%    \end{macrocode}
%\iffalse
%</discard>
%<*package>
%\fi
%
% \macro{\ifchilddoc}
% \macro{\ifchilddocmanual}
% The conditional |\ifchilddoc| tells whether a
% child (true) or main (false) document is being compiled.
% The conditional |\ifchilddocmanual| tells whether
% the |\includeonly| mechanism is used (false) or
% the selection of child files must be performed manually (true).
% The definitions initialise to false:
%    \begin{macrocode}
\newif\ifchilddoc
\newif\ifchilddocmanual
%    \end{macrocode}

% \macro{\childdocname}
% \macro{\childdocjob}
% The macro |\childdocname| stores the name of the main document
% to be compiled. The macro |\childdocjob| stores the name of
% the document on which the \LaTeX{} compiler was originally invoked.
% The content of |\jobname| cannot be compared
% to filenames specified in the source due to different catcodes.
% The following code rescans |\jobname|, stores the result
% in |\childdocname| and saves a copy in |\childdocjob|:
%    \begin{macrocode}
\edef\childdocname{\scantokens\expandafter{\jobname\noexpand}}
\let\childdocjob\childdocname
%    \end{macrocode}

% \macro{\childdocdisable}
% The macro |\childdocdisable| prevents the main file
% from being processed more than once.
% At this stage, the main document command |\childdocmain|
% is assumed to be called once again where it should do nothing.
% Any subsequent call to it should prevent
% a secondary processing of the main document
% It overwrites the forwarding commands
% |\childdocof| and |\childdocforward|
% with empty macros to prevent further inclusions of the main document:
%    \begin{macrocode}
\newcommand{\childdocdisable}
{
  \renewcommand{\childdocmain}[1]{\renewcommand{\childdocmain}[1]{\endinput}}
  \renewcommand{\childdocof}[1]{}
  \renewcommand{\childdocby}[2][]{}
  \renewcommand{\childdocforward}[2][]{}
  \renewcommand{\childdocdisable}{}
}
%    \end{macrocode}

% \macro{\childdocmain}
% The macro |\childdocmain| is to be called at the top of the main file
% with nothing or the main filename (without extension) as argument.
% First, it breaks loops.
% If the argument is not empty and does not match |\childdocname|
% (which is set by the first inclusion of |childdoc.def|),
% |\ifchilddoc| is set to true, |\includeonly| is applied to the child file
% and |\jobname| is set to the main file
% (for proper handling of |.aux| files):
%    \begin{macrocode}
\newcommand{\childdocmain}[1]
{
  \childdocdisable\childdocmain{}
  \if?#1?\else
    \begingroup
      \def\childdoctmp{#1}
      \ifx\childdoctmp\childdocname
        \def\childdoctmp{}
      \else
        \def\childdoctmp
        {
          \childdoctrue
          \includeonly{\childdocname}
          \def\childdocjob{#1}
          \def\jobname{#1}
        }
      \fi
      \expandafter
    \endgroup
    \childdoctmp
  \fi
}
%    \end{macrocode}

% \macro{\childdocof}
% The command |\childdocof| redirects
% compilation to the main file |#1|.
%    \begin{macrocode}
\newcommand{\childdocof}[1]
{
  \childdocdisable
  \childdoctrue
  \includeonly{\childdocname}
  \def\jobname{#1}
  \def\childdocjob{#1}
  \input{#1}
}
%    \end{macrocode}

% \macro{\childdocby}
% The command |\childdocby| ....
%    \begin{macrocode}
\newcommand{\childdocby}[2][]
{
  \childdocdisable
  \childdoctrue
  \childdocmanualtrue
  \if?#1?\else
    \def\jobname{#2}
  \fi
  \def\childdocjob{#2}
  \input{#2}
  \endinput
}
%    \end{macrocode}

% \macro{\childdocforward}
% The command |\childdocforward| redirects
% compilation to the main file or
% (if the optional argument is given) a child file.
% Parameters are set as if the main file
% or a child file starting with |\childdocof| was compiled.
% Then compilation is handed over to the main file:
%    \begin{macrocode}
\newcommand{\childdocforward}[2][]
{
  \begingroup
    \if?#1?
      \def\childdoctmp
      {
        \def\childdocname{#2}
        \def\childdocjob{#2}
        \def\jobname{#2}
        \input{#2}
        \endinput
      }
    \else
      \def\childdoctmp
      {
        \childdocdisable
        \def\childdocname{#2}
        \childdoctrue
        \includeonly{#2}
        \def\childdocjob{#1}
        \def\jobname{#1}
        \input{#1}
        \endinput
      }
    \fi
    \expandafter
  \endgroup
  \childdoctmp
}
%    \end{macrocode}

% \macro{\childdocforwardprefix}
% The command |\childdocforwardprefix| redirects
% compilation to the main or a child file by means of a pattern.
% The prefix |#1| in the current filename is replaced by |#2|
% and the suffix of the current filename is kept
% (it is assumed that the filename does not contain the substring `|~~~|'
% which is used as a delimiter).
% Compilation is handed over to the new file by |\childdocforward|:
%    \begin{macrocode}
\newcommand{\childdocforwardprefix}[3][]
{
  \begingroup
    \def\childdocextract #2##1~~~{\def\childdoctmp{\childdocforward[#1]{#3##1}}}
    \expandafter\childdocextract\childdocname~~~
    \expandafter
  \endgroup
  \childdoctmp
}
%    \end{macrocode}

% \macro{\childdoc}
% The deprecated macro |\childdoc| is a legacy version of |\childdocmain|:
%    \begin{macrocode}
\newcommand{\childdoc}{\childdocmain}
%    \end{macrocode}

% \macro{\childdocredirect}
% The deprecated macro |\childdocredirect| is a legacy version
% of |\childdocforward| and |\childdocforwardprefix|:
%    \begin{macrocode}
\newcommand{\childdocredirect}[2][]
{
  \begingroup
    \if?#1?
      \def\childdoctmp{\childdocforward{#2}}
    \else
      \def\childdoctmp{\childdocforwardprefix{#1}{#2}}
    \fi
    \expandafter
  \endgroup
  \childdoctmp
}
%    \end{macrocode}

%\iffalse
%</package>
%\fi
%
\endinput
|\\
|\childdocforwardprefix[|\textit{main}|]{|\textit{prefix}|}{|\textit{dest}|}|
\end{tabular}
\end{center}
%
the destination file is determined by a pattern
depending on the current file:
To make this work, the current file must be called
`{\textit{prefix}\hspace{0.2em}\textit{suffix}}'
with \textit{prefix} matching precisely the argument.
Processing is then passed on to the file
`{\textit{dest}\hspace{0.2em}\textit{suffix}}'.
Surely, the same effect is achieved by
directly specifying the
argument `{\textit{dest}\hspace{0.2em}\textit{suffix}}'
in the first form.
However, that requires to set up a different file
for each child. With the alternative form of the command
all these files can have exactly the same content
which simplifies setting them up and maintaining them.

For example, the following file |draft.tex|
with a compilation flag |\version| as described in \secref{sec:flags}
compiles the main document as a draft:
%
\begin{center}
\begin{tabular}{l}
|\def\version{draft}|\\
|% \iffalse
%
% childdoc.dtx Copyright (C) 2017-2018 Niklas Beisert
%
% This work may be distributed and/or modified under the
% conditions of the LaTeX Project Public License, either version 1.3
% of this license or (at your option) any later version.
% The latest version of this license is in
%   http://www.latex-project.org/lppl.txt
% and version 1.3 or later is part of all distributions of LaTeX
% version 2005/12/01 or later.
%
% This work has the LPPL maintenance status `maintained'.
%
% The Current Maintainer of this work is Niklas Beisert.
%
% This work consists of the files childdoc.dtx and childdoc.ins
% and the derived files childdoc.def and cdocsamp.tex with
% cdocsch1.tex, cdocsch2.tex, cdocsdrf.tex, cdocsfn1.tex, cdocsfn2.tex.
%
%<package>\ifdefined\childdocmain\endinput\fi
%<package>\ProvidesFile{childdoc.def}[2018/12/30 v2.0 child document driver]
%<samplemain>\ProvidesFile{cdocsamp.tex}[2018/12/30 v2.0 sample for childdoc]
%<*driver>
%\ProvidesFile{childdoc.drv}[2018/12/30 v2.0 childdoc reference manual file]
\PassOptionsToClass{10pt,a4paper}{article}
\documentclass{ltxdoc}

\usepackage[margin=35mm]{geometry}
\usepackage{hyperref}
\usepackage{hyperxmp}
\usepackage[usenames]{color}

\hypersetup{colorlinks=true}
\hypersetup{pdfstartview=FitH}
\hypersetup{pdfpagemode=UseNone}
\hypersetup{pdfsource={}}
\hypersetup{pdflang={en-UK}}
\hypersetup{pdfcopyright={Copyright 2017-2018 Niklas Beisert.
  This work may be distributed and/or modified under the
  conditions of the LaTeX Project Public License, either version 1.3
  of this license or (at your option) any later version.}}
\hypersetup{pdflicenseurl={http://www.latex-project.org/lppl.txt}}
\hypersetup{pdfcontactaddress={ETH Zurich, ITP, HIT K,
  Wolfgang-Pauli-Strasse 27}}
\hypersetup{pdfcontactpostcode={8093}}
\hypersetup{pdfcontactcity={Zurich}}
\hypersetup{pdfcontactcountry={Switzerland}}
\hypersetup{pdfcontactemail={nbeisert@itp.phys.ethz.ch}}
\hypersetup{pdfcontacturl={http://people.phys.ethz.ch/\xmptilde nbeisert/}}

\newcommand{\secref}[1]{\hyperref[#1]{section \ref*{#1}}}

\parskip1ex
\parindent0pt
\let\olditemize\itemize
\def\itemize{\olditemize\parskip0pt}

\begin{document}

\title{The \textsf{childdoc} Package}
\hypersetup{pdftitle={The childdoc Package}}
\author{Niklas Beisert\\[2ex]
  Institut f\"ur Theoretische Physik\\
  Eidgen\"ossische Technische Hochschule Z\"urich\\
  Wolfgang-Pauli-Strasse 27, 8093 Z\"urich, Switzerland\\[1ex]
  \href{mailto:nbeisert@itp.phys.ethz.ch}
  {\texttt{nbeisert@itp.phys.ethz.ch}}}
\hypersetup{pdfauthor={Niklas Beisert}}
\hypersetup{pdfsubject={Manual for the LaTeX2e Package childdoc}}
\date{30 December 2018, \textsf{v2.0}}
\maketitle

\begin{abstract}\noindent
\textsf{childdoc} is a \LaTeXe{} package
that enables the direct compilation
of document sections included by |\include|
to individual files.
\end{abstract}

\begingroup
\parskip0ex
\tableofcontents
\endgroup

%%%%%%%%%%%%%%%%%%%%%%%%%%%%%%%%%%%%%%%%%%%%%%%%%%%%%%%%%%%%%%%%%%%%%%%%%%%%%%%%
%%%%%%%%%%%%%%%%%%%%%%%%%%%%%%%%%%%%%%%%%%%%%%%%%%%%%%%%%%%%%%%%%%%%%%%%%%%%%%%%
\section{Introduction}

\LaTeX{} provides a mechanism to structure a large document (such as a book)
into a main file and several child files (containing the chapters)
using the |\include| command.
This mechanism is beneficial for documents
which span hundreds of pages in order to
make the source file(s) more manageable.
Moreover, compilation can be restricted to
selected child files by means of the |\includeonly| command.
The latter feature can be used to reduce the compilation time while editing
(this was significantly more useful in the earlier days of \LaTeX{})
or to generate a smaller document which is easier to navigate.
Another application of |\includeonly| is to generate
documents consisting of selected parts of the complete document.

However, there are a few drawbacks of the plain |\include| mechanism:
\begin{itemize}
\item
The child files cannot be compiled on their own,
they can only be compiled via the main file.
A naive editing environment
(such as a text editor with an option
to have the current file processed by \LaTeX)
may require one to switch to the main file before compiling;
attempting to compile the child file produces errors.
\item
The main file must be modified (each time)
to adjust the |\includeonly| command
to the present needs. This easily leaves the main file in a messy state.
\item
The generated document will always carry the filename
of the main document. This is inconvenient if
several child files are to be compiled and
to be kept for distribution.
\end{itemize}

The present package provides a simple interface
to make child files individually compilable by \LaTeX{}.
Compiling a child file then has the same effect as compiling
the main file with an |\includeonly| command
to select the appropriate child.
Moreover the generated document will carry the name of the child
rather than the main file.
This resolves all three above issues.

This feature is meant to make the editing of books,
thesis documents and lecture notes somewhat more convenient.
However, the package can also be used efficiently for
composing a series of documents (such as exercise sheets)
which are typically distributed individually.
It then assists the author in generating the individual documents
(potentially in different versions)
as well as a document containing the collected series.
Another application is in developing style files
or other kinds of included material
where compilation of the style file could redirect
to a sample or test file.

%%%%%%%%%%%%%%%%%%%%%%%%%%%%%%%%%%%%%%%%%%%%%%%%%%%%%%%%%%%%%%%%%%%%%%%%%%%%%%%%
%%%%%%%%%%%%%%%%%%%%%%%%%%%%%%%%%%%%%%%%%%%%%%%%%%%%%%%%%%%%%%%%%%%%%%%%%%%%%%%%
\section{Usage}

First of all, the package \textsf{childdoc} is \emph{not} a standard
\LaTeXe{} |.sty| style file! Therefore it needs to be invoked in
a non-standard way.

%%%%%%%%%%%%%%%%%%%%%%%%%%%%%%%%%%%%%%%%%%%%%%%%%%%%%%%%%%%%%%%%%%%%%%%%%%%%%%%%
\subsection{Included Files}
\label{sec:include}

%%%%%%%%%%%%%%%%%%%%%%%%%%%%%%%%%%%%%%%%
\DescribeMacro{\childdocmain}
To use the package, add the commands
\begin{center}
\begin{tabular}{l}
|\input{childdoc.def}|\\
|\childdocmain{}|\\
\end{tabular}
\end{center}
at the very top of the main \LaTeX{} file,
in particular \emph{before} the |\documentclass| statement!
The argument of |\childdocmain| should be left empty
(but it must be present).

%%%%%%%%%%%%%%%%%%%%%%%%%%%%%%%%%%%%%%%%
\DescribeMacro{\childdocof}
Furthermore, add the commands
\begin{center}
\begin{tabular}{l}
|\input{childdoc.def}|\\
|\childdocof{|\textit{main}|}|\\
\end{tabular}
\end{center}
at the top of every child file \textit{child}
which is included by |\include{|\textit{child}|}|
from within the main file
(or at least for those files to be compiled individually).
The argument \textit{main} must be the filename of the main file.

There are a couple of
considerations in setting up the main and child documents:

%%%%%%%%%%%%%%%%%%%%%%%%%%%%%%%%%%%%%%%%
\paragraph{Restrictions.}

Please note the following restrictions:
\begin{itemize}
\item
|\childdocmain| must be called with one argument \textit{main}
to ensure compatibility with earlier version of the package.
It must either be empty (|\childdocmain{}|)
or precisely match the filename of the main file in which it is specified.
See \secref{sec:detection} for further information.
\item
The filename \textit{main} must be specified without the |.tex| extension.
\item
The filename \textit{main} is case sensitive
(even in case-insensitive file systems)
due to internal string comparison.
\item
The argument \textit{main} should be fully expanded, it cannot be a macro.
\item
Subdirectories and special characters should be avoided in filenames.
\item
The command |\childdocmain{|\textit{main}|}| must be followed by a whitespace.
It should not be followed immediately by another command
or by a comment mark `|%|'.
This is because the \TeX{} parser reads the token immediately following
the argument of |\childdocmain| and puts it
at the beginning of every child section;
however, a white\-space is ignored.
\end{itemize}

%%%%%%%%%%%%%%%%%%%%%%%%%%%%%%%%%%%%%%%%
\paragraph{Content of Main File.}

It is advisable to place all content in the child files included by |\include|.
Any output contained in the main file will appear in all child documents
unless suppressed manually;
it cannot be suppressed automatically by the |\includeonly| directive
and thus should normally be avoided.
A method to include some content in the main file
by means of conditional processing is described in \secref{sec:conditional}.

%%%%%%%%%%%%%%%%%%%%%%%%%%%%%%%%%%%%%%%%
\paragraph{Page Numbering.}

When only a part of the document is compiled,
the appropriate numbering of pages
(as well as other status parameters)
is determined from the |.aux| files.
The latter contain information from previous passes.
However this information needs to propagate through
all intermediate child documents.
Therefore the page numbering in child documents may well
be inconsistent until the complete document is compiled at least once.

A useful (if unconventional) way to always ensure a consistent
page numbering is to restart the numbering in each child document
and denote the pages by `\textit{child}|.|\textit{page}'
where \textit{child} represents the chapter/section number of the child file.
This can be achieved by the command
|\numberwithin{page}{|\textit{child}|}|
of the \textsf{amsmath} package
where \textit{child} can be |chapter| or |section|
depending on the chosen structuring.
Alternatively, one can modify the macro |\thepage| appropriately
and reset the counter |page| at the start of each child file.

%%%%%%%%%%%%%%%%%%%%%%%%%%%%%%%%%%%%%%%%%%%%%%%%%%%%%%%%%%%%%%%%%%%%%%%%%%%%%%%%
\subsection{Conditional Processing}
\label{sec:conditional}

The package provides a mechanism to compile different versions
of a document. To customise the versions further some conditional processing
can come in handy to distinguish which version is being compiled.
The package provides two macros to describe the compilation context:

%%%%%%%%%%%%%%%%%%%%%%%%%%%%%%%%%%%%%%%%
\DescribeMacro{\ifchilddoc}
The conditional |\ifchilddoc| distinguishes between the compilation of
child documents and the main document:
%
\begin{center}
|\ifchilddoc |\textit{child-code}| |[|\||else |\textit{main-code}]| \||fi|
\end{center}

%%%%%%%%%%%%%%%%%%%%%%%%%%%%%%%%%%%%%%%%
\DescribeMacro{\childdocname}
\DescribeMacro{\childdocjob}
The macro |\childdocname| contains the filename (without extension)
of the main or child file being processed.
Note that |\childdocjob| will always contain the name of the main file.

%%%%%%%%%%%%%%%%%%%%%%%%%%%%%%%%%%%%%%%%
\paragraph{Title Page.}

Conditional processing can be used to include a title or banner page
in the main document when proper precautions are taken.
Importantly, the code in the main file should ensure that the page counter
(as well as other status parameters which are stored in the |.aux| files)
takes the same value after the conditional processing.
Otherwise the page numbers may take divergent values
depending on which part is compiled.

For example, a title page could be declared by:
%
\begin{center}
\begin{tabular}{l}
|\ifchilddoc\||else|\\
|\addtocounter{page}{-1}|\\
\textit{code for title page}\\
|\newpage|\\
|\||fi|
\end{tabular}
\end{center}
%
A banner page for the child documents can be generated by:
%
\begin{center}
\begin{tabular}{l}
|\ifchilddoc|\\
|\addtocounter{page}{-1}|\\
\textit{code for banner page}\\
|\newpage|\\
|\||fi|
\end{tabular}
\end{center}
%
Here one could write a message such as:
\begin{center}
|This is the part \childdocname{} of \childdocjob{}.|
\end{center}

%%%%%%%%%%%%%%%%%%%%%%%%%%%%%%%%%%%%%%%%%%%%%%%%%%%%%%%%%%%%%%%%%%%%%%%%%%%%%%%%
\subsection{Flags}
\label{sec:flags}

The package makes it easy to generate different versions
of the main or child documents.
To this end compilation flags can be defined
and assigned different default values.
They will be particularly useful in conjunction
with the forwarding mechanism described in \secref{sec:forward}.

For example, it may be useful to have a flag |\version|
which can be set to |draft| or |final|.
The document source will contain some conditional code
depending on the value of |\version|.
Suppose further, the flag should default to |final| for the main file
and to |draft| for child files
which is a natural assignment for editing the document.
This is achieved by placing the following code
in the preamble of the main document
(below the |\childdocmain| directive):
%
\begin{center}
\begin{tabular}{l}
|\ifchilddoc|\\
|\providecommand{\version}{draft}|\\
|\||else|\\
|\providecommand{\version}{final}|\\
|\||fi|
\end{tabular}
\end{center}
%
The definition by |\providecommand| makes sure
that previous definitions are not overwritten.
Further statements |\providecommand{\version}{...}|
can thus be added before the above code to override it.

For the main file, one might add a line
(between |\childdocmain| and the above block)
%
\begin{center}
|%\ifchilddoc\||else\providecommand{\version}{draft}\||fi|
\end{center}
%
which can be uncommented to produce a draft version.
Likewise one can add a line to the very top of a child file
(above the |\childdocof{|\textit{main}|}| directive)
%
\begin{center}
|%\providecommand{\version}{final}|
\end{center}
%
which can be uncommented to produce the final version of this child document.

%%%%%%%%%%%%%%%%%%%%%%%%%%%%%%%%%%%%%%%%%%%%%%%%%%%%%%%%%%%%%%%%%%%%%%%%%%%%%%%%
\subsection{Forwarding}
\label{sec:forward}

Different versions of the main or child documents
using compilation flags as described in \secref{sec:flags}
can be (permanently) stored in different files
for convenient compilation, viewing and distribution.
To this end, the package defines a command
to pass on compilation to a different file:

%%%%%%%%%%%%%%%%%%%%%%%%%%%%%%%%%%%%%%%%
\DescribeMacro{\childdocforward}
The command |\childdocforward| redirects processing to
another source file:
%
\begin{center}
\begin{tabular}{l}
|\input{childdoc.def}|\\
|\childdocforward[|\textit{main}|]{|\textit{dest}|}|\\
\end{tabular}
\end{center}
%
The argument \textit{dest} is the destination file
(without extension).
It should be the main file or one of the child files.
Note that further \textsf{childdoc} directives
such as |\childdocof| and |\childdocforward|
in the indicated file will be processed in this form.
The optional argument \textit{main}
passes on directly to the main file \textit{main}
while pretending to compile the child \textit{dest}.
This form behaves as if \textit{dest}
issues |\childdocof{|\textit{main}|}| right away,
and no further \textsf{childdoc} directives will be processed.

%%%%%%%%%%%%%%%%%%%%%%%%%%%%%%%%%%%%%%%%
\DescribeMacro{\...prefix}
In the alternative form |\childdocforwardprefix|,
%
\begin{center}
\begin{tabular}{l}
|\input{childdoc.def}|\\
|\childdocforwardprefix[|\textit{main}|]{|\textit{prefix}|}{|\textit{dest}|}|
\end{tabular}
\end{center}
%
the destination file is determined by a pattern
depending on the current file:
To make this work, the current file must be called
`{\textit{prefix}\hspace{0.2em}\textit{suffix}}'
with \textit{prefix} matching precisely the argument.
Processing is then passed on to the file
`{\textit{dest}\hspace{0.2em}\textit{suffix}}'.
Surely, the same effect is achieved by
directly specifying the
argument `{\textit{dest}\hspace{0.2em}\textit{suffix}}'
in the first form.
However, that requires to set up a different file
for each child. With the alternative form of the command
all these files can have exactly the same content
which simplifies setting them up and maintaining them.

For example, the following file |draft.tex|
with a compilation flag |\version| as described in \secref{sec:flags}
compiles the main document as a draft:
%
\begin{center}
\begin{tabular}{l}
|\def\version{draft}|\\
|\input{childdoc.def}|\\
|\childdocforward{|\textit{main}|}|
\end{tabular}
\end{center}
%
Likewise, the following files |final|\textit{nn}|.tex|
compile the final version of the child document
|child|\textit{nn}|.tex|:
%
\begin{center}
\begin{tabular}{l}
|\def\version{final}|\\
|\input{childdoc.def}|\\
|\childdocforwardprefix{final}{child}|
\end{tabular}
\end{center}
%

Note that when several versions of a main file and/or of each child file
are to be generated, it may be convenient to set up a |Makefile| or
shell script to automatise the process.

%%%%%%%%%%%%%%%%%%%%%%%%%%%%%%%%%%%%%%%%%%%%%%%%%%%%%%%%%%%%%%%%%%%%%%%%%%%%%%%%
\subsection{Command Line Processing}
\label{sec:commandline}

The effect of redirection files can also be achieved by invoking
the \LaTeX{} compiler with a more elaborate command line.
Most conveniently this should be done as part
of a shell script or a |Makefile|.

When using \textsf{childdoc} in the main file, the following
command lines effectively perform a redirection
(note that depending on the shell being used,
backslashes may have to be doubled: `|\|' $\to$ `|\\|'):
%
\begin{center}
|... -jobname "|\textit{target}|" |\\|"|[\textit{flags}]%
|\input{childdoc.def}\childdocforward[|\textit{main}|]{|\textit{dest}|}"|
\end{center}
%
Here \textit{target} is the name of the output file,
\textit{main} is the name of the main file
and \textit{dest} is the name of the main or child file to be processed
(all filenames without extensions).
The optional argument \textit{main} can be omitted
if \textit{main} matches \textit{dest}.
Optionally, compilation \textit{flags} can be defined via |\def| commands.
This command line makes the \TeX{} engine believe
it is compiling the file \textit{target}
whose content is specified as the latter parameter.
The provided code then forwards the processing to
\textit{main} or \textit{dest} as described in \secref{sec:forward}.

%%%%%%%%%%%%%%%%%%%%%%%%%%%%%%%%%%%%%%%%%%%%%%%%%%%%%%%%%%%%%%%%%%%%%%%%%%%%%%%%
\subsection{Include by Input}
\label{sec:input}

Including child documents by |\include| has some restrictions by design.
Most notably, the content of a child document always occupies
its own set of pages; pages cannot be shared between child documents.
Usually, this behaviour makes perfect sense
because each child document contain an essential part of the document.
However, in some situations it may be desirable to compose
a document from a collection of parts
without having mandatory page breaks between then.
For this case, the package
provides a mechanism to include parts
by |\input| which can also be processed individually.
However, by construction this mechanism
requires manual handling of the content to be output.

%%%%%%%%%%%%%%%%%%%%%%%%%%%%%%%%%%%%%%%%
\DescribeMacro{\ifchilddocmanual}
The main file should be prepared as usual, see \secref{sec:include}.
However, the document body must make a distinction
between processing of an individual part and of the main document, e.g.:
%
\begin{center}
\begin{tabular}{l}
|\ifchilddocmanual|\\
|\input{\childdocname}|\\
|\||else|\\
\textit{document body with }|\input{|\textit{part}|}|\\
|\||fi|
\end{tabular}
\end{center}
%
The conditional |\ifchilddocmanual| is true whenever
a part to be included by |\input| is being compiled,
and the name of the part is stored in |\childdocname|.

%%%%%%%%%%%%%%%%%%%%%%%%%%%%%%%%%%%%%%%%
\DescribeMacro{\childdocby}
Each part to be included by |\input| should start with:
%
\begin{center}
\begin{tabular}{l}
|\input{childdoc.def}|\\
|\childdocby{|\textit{main}|}|\\
\end{tabular}
\end{center}
%
The directive |\childdocby| is similar to |\childdocof|
described in \secref{sec:include},
but the subsequent selection of content must be done manually.
To that end, both |\ifchilddoc| and |\ifchilddocmanual|
will be true upon processing of a part,
and the name of the part is stored in |\childdocname|.
Note that |\jobname| will be set to the filename of the current part
so that each part receives an individual |.aux| file
that does not interfere with the |.aux| file(s) of the main document.
This behaviour can be altered by the alternative form
|\childdocby[*]{|\textit{main}|}| (with a non-empty optional argument)
which uses the |.aux| file of the main document
by setting |\jobname| to \textit{main}.

%%%%%%%%%%%%%%%%%%%%%%%%%%%%%%%%%%%%%%%%%%%%%%%%%%%%%%%%%%%%%%%%%%%%%%%%%%%%%%%%
\subsection{Driver Development}
\label{sec:driver}

The \textsf{childdoc} mechanism can also be use for the development
of definition files such as \LaTeX{} styles or classes.
This case differs from the above setup with multiple parts
included by |\include| in that no |\includeonly| should be invoked.
This can be achieved by starting the include file
(before |\ProvidesPackage|) with:
%
\begin{center}
\begin{tabular}{l}
|\input{childdoc.def}|\\
|\childdocforward{|\textit{main}|}|\\
\end{tabular}
\end{center}
%
or alternatively with:
%
\begin{center}
\begin{tabular}{l}
|\input{childdoc.def}|\\
|\childdocby{|\textit{main}|}|\\
\end{tabular}
\end{center}
%
Both forms have slightly different effects as described above.
The main file is prepared as usual, see \secref{sec:include}.

%%%%%%%%%%%%%%%%%%%%%%%%%%%%%%%%%%%%%%%%%%%%%%%%%%%%%%%%%%%%%%%%%%%%%%%%%%%%%%%%
\subsection{Legacy Detection}
\label{sec:detection}

The directive |\childdocmain| in the main file can detect
whether the complete document or merely a child is to be compiled
even without using the directive |\childdocof|.
This method is deprecated because it is less robust
and there is no compelling reason to use it;
it is merely provided for backward compatibility
and it may be removed in future versions.

If the detection mechanism is to be used,
it is mandatory to correctly specify
the filename of the main file as the argument of |\childdocmain|:
%
\begin{center}
\begin{tabular}{l}
|\input{childdoc.def}|\\
|\childdocmain{|\textit{main}|}|\\
\end{tabular}
\end{center}
%
If |\jobname| does not match the argument \textit{main} of |\childdocmain|,
it is assumed that |\jobname| points to the child file to be compiled.
When using |\childdocmain| with the main file specified as argument,
it suffices to start a child file
with just |\input{|\textit{main}|}|
without loading of the package and using |\childdocof|.
If instead all processing is done
with the appropriate \textsf{childdoc} directives,
the argument of \textit{main} of |\childdocmain| can be empty.

An alternative version of the command line processing described
in \secref{sec:commandline} using the detection mechanism reads:
%
\begin{center}
|... -jobname "|\textit{target}|" "|[\textit{flags}]%
[|\def\jobname{|\textit{dest}|}|]|\input{|\textit{main}|}"|
\end{center}

%%%%%%%%%%%%%%%%%%%%%%%%%%%%%%%%%%%%%%%%%%%%%%%%%%%%%%%%%%%%%%%%%%%%%%%%%%%%%%%%
\subsection{Manual Code}
\label{sec:manual}

In case one cannot be certain whether the definitions file |childdoc.def|
is installed on the target \TeX{} distribution
and one prefers not to ship it,
it is conceivable to paste a few relevant commands into the sources.

To that end, drop all statements |\input{childdoc.def}|
and perform the replacements as outlined below.
Instead of |\childdocmain{|\textit{main}|}| add the following code
to the top of the main file:
%
\begin{center}
\begin{tabular}{l}
|\||ifdefined\childdocname\endinput\||fi\newif\ifchilddoc|\\
|\edef\childdocname{\scantokens\expandafter{\jobname\noexpand}}|\\
|\def\childdocmain{|\textit{main}|}\||ifx\childdocmain\childdocname\||else|\\
|\childdoctrue\includeonly{\childdocname}\let\jobname\childdocmain\||fi|\\
\end{tabular}
\end{center}
%
Instead of |\childdocof{|\textit{main}|}| just include the main file
at the top of each child file:
%
\begin{center}
|\input{|\textit{main}|}|
\end{center}
%
A simple redirection |\childdocforward{|\textit{dest}|}| is achieved by:
%
\begin{center}
|\def\jobname{|\textit{dest}|}\input{\jobname}|
\end{center}
%
The redirection with prefix
|\childdocforwardprefix[|\textit{prefix}|]{|\textit{dest}|}|
is accomplished by:
%
\begin{center}
\begin{tabular}{l}
|{\edef\jobname{\scantokens\expandafter{\jobname\noexpand}}|\\
|\def\redirectjob |\textit{prefix}|#1~~~{\gdef\jobname{|\textit{dest}|#1}}|\\
|\expandafter\redirectjob\jobname~~~}\input{\jobname}|
\end{tabular}
\end{center}

In an alternative approach,
child documents can be compiled by a specific command line
without additional code or specific definitions:
%
\begin{center}
|... -jobname "|\textit{target}|" "|[\textit{flags}]%
|\includeonly{|\textit{dest}|}\input{|\textit{main}|}"|
\end{center}
%

%%%%%%%%%%%%%%%%%%%%%%%%%%%%%%%%%%%%%%%%%%%%%%%%%%%%%%%%%%%%%%%%%%%%%%%%%%%%%%%%
%%%%%%%%%%%%%%%%%%%%%%%%%%%%%%%%%%%%%%%%%%%%%%%%%%%%%%%%%%%%%%%%%%%%%%%%%%%%%%%%
\section{Information}

%%%%%%%%%%%%%%%%%%%%%%%%%%%%%%%%%%%%%%%%%%%%%%%%%%%%%%%%%%%%%%%%%%%%%%%%%%%%%%%%
\subsection{Copyright}

Copyright \copyright{} 2017--2018 Niklas Beisert

This work may be distributed and/or modified under the
conditions of the \LaTeX{} Project Public License, either version 1.3
of this license or (at your option) any later version.
The latest version of this license is in
  \url{http://www.latex-project.org/lppl.txt}
and version 1.3 or later is part of all distributions of \LaTeX{}
version 2005/12/01 or later.

This work has the LPPL maintenance status `maintained'.

The Current Maintainer of this work is Niklas Beisert.

This work consists of the files |README.txt|, |childdoc.ins| and |childdoc.dtx|
as well as the derived files |childdoc.def|, |cdocsamp.tex|
with |cdocsch1.tex|, |cdocsch2.tex|, |cdocspt3.tex|, |cdocspt4.tex|,
|cdocsdrf.tex|, |cdocsfn1.tex|, |cdocsfn2.tex|
as well as |childdoc.pdf|.

%%%%%%%%%%%%%%%%%%%%%%%%%%%%%%%%%%%%%%%%%%%%%%%%%%%%%%%%%%%%%%%%%%%%%%%%%%%%%%%%
\subsection{Files and Installation}

The package consists of the files:
%
\begin{center}
\begin{tabular}{ll}
    |README.txt|   & readme file \\
    |childdoc.ins| & installation file \\
    |childdoc.dtx| & source file \\
    |childdoc.def| & definition file \\
    |cdocsamp.tex| & sample main file \\
    |cdocsch1.tex| & sample include file \\
    |cdocsch2.tex| & sample include file \\
    |cdocspt3.tex| & sample part file \\
    |cdocspt4.tex| & sample part file \\
    |cdocsdrf.tex| & sample redirection file \\
    |cdocsfn1.tex| & sample redirection file \\
    |cdocsfn2.tex| & sample redirection file \\
    |childdoc.pdf| & manual
\end{tabular}
\end{center}
%
The distribution consists of the files
|README.txt|, |childdoc.ins| and |childdoc.dtx|.
%
\begin{itemize}
\item
Run (pdf)\LaTeX{} on |childdoc.dtx|
to compile the manual |childdoc.pdf| (this file).
\item
Run \LaTeX{} on |childdoc.ins| to create the definitions file |childdoc.def|
and the sample |cdocsamp.tex| with include files
|cdocsch1.tex|, |cdocsch2.tex|, |cdocspt3.tex|, |cdocspt4.tex|,
|cdocsdrf.tex|, |cdocsfn1.tex|, |cdocsfn2.tex|.
Then copy the file |childdoc.def| to an appropriate directory of your \LaTeX{}
distribution, e.g.\ \textit{texmf-root}|/tex/latex/childdoc|.
\end{itemize}

%%%%%%%%%%%%%%%%%%%%%%%%%%%%%%%%%%%%%%%%%%%%%%%%%%%%%%%%%%%%%%%%%%%%%%%%%%%%%%%%
\subsection{Related CTAN Packages}

There are several other packages which offer a similar functionality:
%
\begin{itemize}
\item
The packages
\href{http://ctan.org/pkg/docmute}{\textsf{docmute}},
\href{http://ctan.org/pkg/includex}{\textsf{includex}} and
\href{http://ctan.org/pkg/standalone}{\textsf{standalone}}
provide commands to include only the document body of
a child file thus allowing both files to be compiled individually.
\item
The packages \href{http://ctan.org/pkg/subdocs}{\textsf{subdocs}}
and \href{http://ctan.org/pkg/subfiles}{\textsf{subfiles}}
provide structures in which the main and child documents can be
encapsulated and allowing them to be compiled individually.
The inclusion mechanism is different from the conventional |\include|.
\item
The package \href{http://ctan.org/pkg/combine}{\textsf{combine}}
is an elaborate solution to combine several documents into one.
\end{itemize}
%
See also the CTAN topic \href{http://ctan.org/topic/subdocs}{\textsf{subdocs}}
for further related packages.
The present package differs from the above solutions in that
a document structure constructed with the conventional |\include| mechanism
just needs two extra commands at the top of every file
such that all constituent files can be compiled individually.

%%%%%%%%%%%%%%%%%%%%%%%%%%%%%%%%%%%%%%%%%%%%%%%%%%%%%%%%%%%%%%%%%%%%%%%%%%%%%%%%
%\subsection{Feature Suggestions}
%
%The following is a list of features which may be useful for future
%versions of this package:
%%
%\begin{itemize}
%\item
%\ldots
%\end{itemize}

%%%%%%%%%%%%%%%%%%%%%%%%%%%%%%%%%%%%%%%%%%%%%%%%%%%%%%%%%%%%%%%%%%%%%%%%%%%%%%%%
\subsection{Revision History}

%%%%%%%%%%%%%%%%%%%%%%%%%%%%%%%%%%%%%%%%
\paragraph{v2.0:} 2018/12/30

\begin{itemize}
\item
immediate forward processing
\item
added |\childdocby| mechanism
\item
manual restructured
\end{itemize}

%%%%%%%%%%%%%%%%%%%%%%%%%%%%%%%%%%%%%%%%
\paragraph{v1.6:} 2018/01/17

\begin{itemize}
\item
application for development of include files
\item
corrections to manual
\end{itemize}

%%%%%%%%%%%%%%%%%%%%%%%%%%%%%%%%%%%%%%%%
\paragraph{v1.5:} 2017/05/21

\begin{itemize}
\item
more complete structuring introduced
\item
|\childdocof| introduced
\item
|\childdoc| renamed to |\childdocmain|
\item
|\childredirect| renamed to |\childdocforward| and |\childdocforwardprefix|
and functionality expanded
\end{itemize}

%%%%%%%%%%%%%%%%%%%%%%%%%%%%%%%%%%%%%%%%
\paragraph{v1.0:} 2017/04/27

\begin{itemize}
\item
manual and install package
\item
first version published on CTAN
\end{itemize}

%%%%%%%%%%%%%%%%%%%%%%%%%%%%%%%%%%%%%%%%
\paragraph{v0.6:} 2017/04/26

\begin{itemize}
\item
redirection mechanism added
\end{itemize}

%%%%%%%%%%%%%%%%%%%%%%%%%%%%%%%%%%%%%%%%
\paragraph{v0.5:} 2017/04/26

\begin{itemize}
\item
functionality in definition file
\end{itemize}


%%%%%%%%%%%%%%%%%%%%%%%%%%%%%%%%%%%%%%%%%%%%%%%%%%%%%%%%%%%%%%%%%%%%%%%%%%%%%%%%
%%%%%%%%%%%%%%%%%%%%%%%%%%%%%%%%%%%%%%%%%%%%%%%%%%%%%%%%%%%%%%%%%%%%%%%%%%%%%%%%
%%%%%%%%%%%%%%%%%%%%%%%%%%%%%%%%%%%%%%%%%%%%%%%%%%%%%%%%%%%%%%%%%%%%%%%%%%%%%%%%
\appendix

\settowidth\MacroIndent{\rmfamily\scriptsize 000\ }

 \DocInput{childdoc.dtx}

\end{document}
%</driver>
% \fi
%
% %%%%%%%%%%%%%%%%%%%%%%%%%%%%%%%%%%%%%%%%%%%%%%%%%%%%%%%%%%%%%%%%%%%%%%%%%%%%%%
% %%%%%%%%%%%%%%%%%%%%%%%%%%%%%%%%%%%%%%%%%%%%%%%%%%%%%%%%%%%%%%%%%%%%%%%%%%%%%%
% \section{Sample}
%\iffalse
%<*samplemain>
%\fi
%
% The following presents a sample document
% with two chapters, two parts, a title page,
% a compile flag as well as three forwarding files to set the flag.
% It consists of eight |.tex| files:
% \begin{center}
% \begin{tabular}{ll}
% |cdocsamp.tex|&main file\\
% |cdocsch1.tex|&include file for chapter 1\\
% |cdocsch2.tex|&include file for chapter 2\\
% |cdocspt3.tex|&include file for part 3\\
% |cdocspt4.tex|&include file for part 4\\
% |cdocsdrf.tex|&forwarding file for main file in draft mode\\
% |cdocsfi1.tex|&forwarding file for final version of chapter 1\\
% |cdocsfi2.tex|&forwarding file for final version of chapter 2\\
% \end{tabular}
% \end{center}
% Each of the eight files can be compiled directly by the \LaTeX{} compiler.
%
% %%%%%%%%%%%%%%%%%%%%%%%%%%%%%%%%%%%%%%
% \paragraph{Main File.}
%
% The main file is called |cdocsamp.tex|.
%
% Load the \textsf{childdoc} definitions and
% declare the filename for the main document:
%    \begin{macrocode}
\input{childdoc.def}
\childdocmain{}
%    \end{macrocode}

% Optional override for |\version| flag:
%    \begin{macrocode}
%%\ifchilddoc\else\providecommand{\version}{draft}\fi
%    \end{macrocode}

% Define the default values for the |\version| flag
% (|final| for the main file and |draft| for childs):
%    \begin{macrocode}
\ifchilddoc
\providecommand{\version}{draft}
\else
\providecommand{\version}{final}
\fi
%    \end{macrocode}

% Load the standard document class:
%    \begin{macrocode}
\documentclass[12pt]{article}
%    \end{macrocode}

% Start the document body:
%    \begin{macrocode}
\begin{document}
%    \end{macrocode}

% Declare a title page.
% Print title, part of document being processed and version flag:
%    \begin{macrocode}
\addtocounter{page}{-1}
\begin{center}
{\LARGE\bfseries{}childdoc example\par}
\vspace{1cm}
\ifchilddoc
\ifchilddocmanual part\else chapter\fi:
`\childdocname' of `\childdocjob'\par
\else
main document: `\childdocjob'\par
\fi
version: \version\par
\end{center}
\newpage
%    \end{macrocode}

% Manually include selected file,
% otherwise process as usual:
%    \begin{macrocode}
\ifchilddocmanual
\section*{part `\childdocname'}
\input{\childdocname}
\else
%    \end{macrocode}

% Include the two chapters:
%    \begin{macrocode}
\include{cdocsch1}
\include{cdocsch2}
%    \end{macrocode}

% Include the two parts unless only chapters should be displayed:
%    \begin{macrocode}
\ifchilddoc\else
\section{part three}
\input{cdocspt3}
\section{part four}
\input{cdocspt4}
\fi
%    \end{macrocode}

% Process as usual until here:
%    \begin{macrocode}
\fi
%    \end{macrocode}

% End of document body:
%    \begin{macrocode}
\end{document}
%    \end{macrocode}
%\iffalse
%</samplemain>
%\fi
%
% %%%%%%%%%%%%%%%%%%%%%%%%%%%%%%%%%%%%%%
% \paragraph{Chapter Include Files.}
%
% The include files are called |cdocsch1.tex| and |cdocsch2.tex|.
%
%\iffalse
%<*samplechap1|samplechap2>
%\fi

% Optional override for |\version| flag:
%    \begin{macrocode}
%%\providecommand{\version}{final}
%    \end{macrocode}

% Include the main document:
%    \begin{macrocode}
\input{childdoc.def}
\childdocof{cdocsamp}
%    \end{macrocode}

%\iffalse
%</samplechap1|samplechap2>
%\fi
%
%\iffalse
%<*samplechap1>
%\fi
% Some text for chapter 1:
%    \begin{macrocode}
\section{one}
some text in chapter one
%    \end{macrocode}

%\iffalse
%</samplechap1>
%\fi
% Some text for chapter 2:
%\iffalse
%<*samplechap2>
%\fi
%    \begin{macrocode}
\section{two}
more text in chapter two
%    \end{macrocode}

%\iffalse
%</samplechap2>
%\fi
%
% %%%%%%%%%%%%%%%%%%%%%%%%%%%%%%%%%%%%%%
% \paragraph{Part Include Files.}
%
% The include files are called |cdocspt3.tex| and |cdocspt4.tex|.
%
%\iffalse
%<*samplepart3|samplepart4>
%\fi

% Optional override for |\version| flag:
%    \begin{macrocode}
%%\providecommand{\version}{final}
%    \end{macrocode}

% Include the main document:
%    \begin{macrocode}
\input{childdoc.def}
\childdocby{cdocsamp}
%    \end{macrocode}

%\iffalse
%</samplepart3|samplepart4>
%\fi
%
%\iffalse
%<*samplepart3>
%\fi
% Some text for part 3:
%    \begin{macrocode}
some text in part three
%    \end{macrocode}

%\iffalse
%</samplepart3>
%\fi
% Some text for part 4:
%\iffalse
%<*samplepart4>
%\fi
%    \begin{macrocode}
more text in part four
%    \end{macrocode}

%\iffalse
%</samplepart4>
%\fi
%
% %%%%%%%%%%%%%%%%%%%%%%%%%%%%%%%%%%%%%%
% \paragraph{Forwarding for a Complete Draft.}
%
% The following forwarding file |cdocsdrf.tex|
% compiles the main document in draft mode:
%\iffalse
%<*sampledraft>
%\fi
%    \begin{macrocode}
\def\version{draft}
\input{childdoc.def}
\childdocforward{cdocsamp}
%    \end{macrocode}

%\iffalse
%</sampledraft>
%\fi
%
% %%%%%%%%%%%%%%%%%%%%%%%%%%%%%%%%%%%%%%
% \paragraph{Forwarding for Final Version of the Chapters.}
%
% The following forwarding files |cdocsfn1.tex| and |cdocsfn2.tex|
% (with identical content)
% compile the final versions of the child documents
% |cdocsch1.tex| and |cdocsch2.tex|, respectively:
%\iffalse
%<*samplefinal>
%\fi
%    \begin{macrocode}
\def\version{final}
\input{childdoc.def}
\childdocforwardprefix[cdocsamp]{cdocsfn}{cdocsch}
%    \end{macrocode}

%\iffalse
%</samplefinal>
%\fi
%
% %%%%%%%%%%%%%%%%%%%%%%%%%%%%%%%%%%%%%%
% \paragraph{Command Line Processing.}
%
% The following three command lines generate the output files
% |cdocscld|, |cdocscl1| and |cdocscl2|
% which should be identical to
% |cdocsdrf|, |cdocsch1| and |cdocsfn2|, respectively:
% \begin{center}
% \begin{tabular}{l}
% |latex -jobname cdocscld \|\\
% |  "\def\version{draft}\input{childdoc.def}\childdocforward{cdocsamp}"|\\
% |latex -jobname cdocscl1 \|\\
% |  "\input{childdoc.def}\childdocforward[cdocsamp]{cdocsch1}"|\\
% |latex -jobname cdocscl2 \|\\
% |  "\def\version{final}\input{childdoc.def}\childdocforward{cdocsch2}"|
% \end{tabular}
% \end{center}
% Note that the trailing backslash on each first line
% merely continues the input to the second line
% (for convenient cut ant paste).
% Furthermore, the command |latex| can be replaced by any
% of its alternative versions such as |pdflatex|.
%
% %%%%%%%%%%%%%%%%%%%%%%%%%%%%%%%%%%%%%%%%%%%%%%%%%%%%%%%%%%%%%%%%%%%%%%%%%%%%%%
% %%%%%%%%%%%%%%%%%%%%%%%%%%%%%%%%%%%%%%%%%%%%%%%%%%%%%%%%%%%%%%%%%%%%%%%%%%%%%%
% \section{Implementation}
%\iffalse
%<*package>
%\fi
%
% This section describes the definitions file |childdoc.def|.

% The definitions cannot be loaded using |\usepackage| or |\RequirePackage|
% which has a mechanism to prevent loading a style file more than once.
% When loading the definitions by means of |\input|
% multiple instances have to be prevented manually:
%\iffalse
%This code needs to be before the `\ProvidesFile' directive
%which is defined at the beginning of this file.
%Therefore it is also placed there and commented out here.
%</package>
%<*discard>
%\fi
%    \begin{macrocode}
\ifdefined\childdocmain\endinput\fi
%    \end{macrocode}
%\iffalse
%</discard>
%<*package>
%\fi
%
% \macro{\ifchilddoc}
% \macro{\ifchilddocmanual}
% The conditional |\ifchilddoc| tells whether a
% child (true) or main (false) document is being compiled.
% The conditional |\ifchilddocmanual| tells whether
% the |\includeonly| mechanism is used (false) or
% the selection of child files must be performed manually (true).
% The definitions initialise to false:
%    \begin{macrocode}
\newif\ifchilddoc
\newif\ifchilddocmanual
%    \end{macrocode}

% \macro{\childdocname}
% \macro{\childdocjob}
% The macro |\childdocname| stores the name of the main document
% to be compiled. The macro |\childdocjob| stores the name of
% the document on which the \LaTeX{} compiler was originally invoked.
% The content of |\jobname| cannot be compared
% to filenames specified in the source due to different catcodes.
% The following code rescans |\jobname|, stores the result
% in |\childdocname| and saves a copy in |\childdocjob|:
%    \begin{macrocode}
\edef\childdocname{\scantokens\expandafter{\jobname\noexpand}}
\let\childdocjob\childdocname
%    \end{macrocode}

% \macro{\childdocdisable}
% The macro |\childdocdisable| prevents the main file
% from being processed more than once.
% At this stage, the main document command |\childdocmain|
% is assumed to be called once again where it should do nothing.
% Any subsequent call to it should prevent
% a secondary processing of the main document
% It overwrites the forwarding commands
% |\childdocof| and |\childdocforward|
% with empty macros to prevent further inclusions of the main document:
%    \begin{macrocode}
\newcommand{\childdocdisable}
{
  \renewcommand{\childdocmain}[1]{\renewcommand{\childdocmain}[1]{\endinput}}
  \renewcommand{\childdocof}[1]{}
  \renewcommand{\childdocby}[2][]{}
  \renewcommand{\childdocforward}[2][]{}
  \renewcommand{\childdocdisable}{}
}
%    \end{macrocode}

% \macro{\childdocmain}
% The macro |\childdocmain| is to be called at the top of the main file
% with nothing or the main filename (without extension) as argument.
% First, it breaks loops.
% If the argument is not empty and does not match |\childdocname|
% (which is set by the first inclusion of |childdoc.def|),
% |\ifchilddoc| is set to true, |\includeonly| is applied to the child file
% and |\jobname| is set to the main file
% (for proper handling of |.aux| files):
%    \begin{macrocode}
\newcommand{\childdocmain}[1]
{
  \childdocdisable\childdocmain{}
  \if?#1?\else
    \begingroup
      \def\childdoctmp{#1}
      \ifx\childdoctmp\childdocname
        \def\childdoctmp{}
      \else
        \def\childdoctmp
        {
          \childdoctrue
          \includeonly{\childdocname}
          \def\childdocjob{#1}
          \def\jobname{#1}
        }
      \fi
      \expandafter
    \endgroup
    \childdoctmp
  \fi
}
%    \end{macrocode}

% \macro{\childdocof}
% The command |\childdocof| redirects
% compilation to the main file |#1|.
%    \begin{macrocode}
\newcommand{\childdocof}[1]
{
  \childdocdisable
  \childdoctrue
  \includeonly{\childdocname}
  \def\jobname{#1}
  \def\childdocjob{#1}
  \input{#1}
}
%    \end{macrocode}

% \macro{\childdocby}
% The command |\childdocby| ....
%    \begin{macrocode}
\newcommand{\childdocby}[2][]
{
  \childdocdisable
  \childdoctrue
  \childdocmanualtrue
  \if?#1?\else
    \def\jobname{#2}
  \fi
  \def\childdocjob{#2}
  \input{#2}
  \endinput
}
%    \end{macrocode}

% \macro{\childdocforward}
% The command |\childdocforward| redirects
% compilation to the main file or
% (if the optional argument is given) a child file.
% Parameters are set as if the main file
% or a child file starting with |\childdocof| was compiled.
% Then compilation is handed over to the main file:
%    \begin{macrocode}
\newcommand{\childdocforward}[2][]
{
  \begingroup
    \if?#1?
      \def\childdoctmp
      {
        \def\childdocname{#2}
        \def\childdocjob{#2}
        \def\jobname{#2}
        \input{#2}
        \endinput
      }
    \else
      \def\childdoctmp
      {
        \childdocdisable
        \def\childdocname{#2}
        \childdoctrue
        \includeonly{#2}
        \def\childdocjob{#1}
        \def\jobname{#1}
        \input{#1}
        \endinput
      }
    \fi
    \expandafter
  \endgroup
  \childdoctmp
}
%    \end{macrocode}

% \macro{\childdocforwardprefix}
% The command |\childdocforwardprefix| redirects
% compilation to the main or a child file by means of a pattern.
% The prefix |#1| in the current filename is replaced by |#2|
% and the suffix of the current filename is kept
% (it is assumed that the filename does not contain the substring `|~~~|'
% which is used as a delimiter).
% Compilation is handed over to the new file by |\childdocforward|:
%    \begin{macrocode}
\newcommand{\childdocforwardprefix}[3][]
{
  \begingroup
    \def\childdocextract #2##1~~~{\def\childdoctmp{\childdocforward[#1]{#3##1}}}
    \expandafter\childdocextract\childdocname~~~
    \expandafter
  \endgroup
  \childdoctmp
}
%    \end{macrocode}

% \macro{\childdoc}
% The deprecated macro |\childdoc| is a legacy version of |\childdocmain|:
%    \begin{macrocode}
\newcommand{\childdoc}{\childdocmain}
%    \end{macrocode}

% \macro{\childdocredirect}
% The deprecated macro |\childdocredirect| is a legacy version
% of |\childdocforward| and |\childdocforwardprefix|:
%    \begin{macrocode}
\newcommand{\childdocredirect}[2][]
{
  \begingroup
    \if?#1?
      \def\childdoctmp{\childdocforward{#2}}
    \else
      \def\childdoctmp{\childdocforwardprefix{#1}{#2}}
    \fi
    \expandafter
  \endgroup
  \childdoctmp
}
%    \end{macrocode}

%\iffalse
%</package>
%\fi
%
\endinput
|\\
|\childdocforward{|\textit{main}|}|
\end{tabular}
\end{center}
%
Likewise, the following files |final|\textit{nn}|.tex|
compile the final version of the child document
|child|\textit{nn}|.tex|:
%
\begin{center}
\begin{tabular}{l}
|\def\version{final}|\\
|% \iffalse
%
% childdoc.dtx Copyright (C) 2017-2018 Niklas Beisert
%
% This work may be distributed and/or modified under the
% conditions of the LaTeX Project Public License, either version 1.3
% of this license or (at your option) any later version.
% The latest version of this license is in
%   http://www.latex-project.org/lppl.txt
% and version 1.3 or later is part of all distributions of LaTeX
% version 2005/12/01 or later.
%
% This work has the LPPL maintenance status `maintained'.
%
% The Current Maintainer of this work is Niklas Beisert.
%
% This work consists of the files childdoc.dtx and childdoc.ins
% and the derived files childdoc.def and cdocsamp.tex with
% cdocsch1.tex, cdocsch2.tex, cdocsdrf.tex, cdocsfn1.tex, cdocsfn2.tex.
%
%<package>\ifdefined\childdocmain\endinput\fi
%<package>\ProvidesFile{childdoc.def}[2018/12/30 v2.0 child document driver]
%<samplemain>\ProvidesFile{cdocsamp.tex}[2018/12/30 v2.0 sample for childdoc]
%<*driver>
%\ProvidesFile{childdoc.drv}[2018/12/30 v2.0 childdoc reference manual file]
\PassOptionsToClass{10pt,a4paper}{article}
\documentclass{ltxdoc}

\usepackage[margin=35mm]{geometry}
\usepackage{hyperref}
\usepackage{hyperxmp}
\usepackage[usenames]{color}

\hypersetup{colorlinks=true}
\hypersetup{pdfstartview=FitH}
\hypersetup{pdfpagemode=UseNone}
\hypersetup{pdfsource={}}
\hypersetup{pdflang={en-UK}}
\hypersetup{pdfcopyright={Copyright 2017-2018 Niklas Beisert.
  This work may be distributed and/or modified under the
  conditions of the LaTeX Project Public License, either version 1.3
  of this license or (at your option) any later version.}}
\hypersetup{pdflicenseurl={http://www.latex-project.org/lppl.txt}}
\hypersetup{pdfcontactaddress={ETH Zurich, ITP, HIT K,
  Wolfgang-Pauli-Strasse 27}}
\hypersetup{pdfcontactpostcode={8093}}
\hypersetup{pdfcontactcity={Zurich}}
\hypersetup{pdfcontactcountry={Switzerland}}
\hypersetup{pdfcontactemail={nbeisert@itp.phys.ethz.ch}}
\hypersetup{pdfcontacturl={http://people.phys.ethz.ch/\xmptilde nbeisert/}}

\newcommand{\secref}[1]{\hyperref[#1]{section \ref*{#1}}}

\parskip1ex
\parindent0pt
\let\olditemize\itemize
\def\itemize{\olditemize\parskip0pt}

\begin{document}

\title{The \textsf{childdoc} Package}
\hypersetup{pdftitle={The childdoc Package}}
\author{Niklas Beisert\\[2ex]
  Institut f\"ur Theoretische Physik\\
  Eidgen\"ossische Technische Hochschule Z\"urich\\
  Wolfgang-Pauli-Strasse 27, 8093 Z\"urich, Switzerland\\[1ex]
  \href{mailto:nbeisert@itp.phys.ethz.ch}
  {\texttt{nbeisert@itp.phys.ethz.ch}}}
\hypersetup{pdfauthor={Niklas Beisert}}
\hypersetup{pdfsubject={Manual for the LaTeX2e Package childdoc}}
\date{30 December 2018, \textsf{v2.0}}
\maketitle

\begin{abstract}\noindent
\textsf{childdoc} is a \LaTeXe{} package
that enables the direct compilation
of document sections included by |\include|
to individual files.
\end{abstract}

\begingroup
\parskip0ex
\tableofcontents
\endgroup

%%%%%%%%%%%%%%%%%%%%%%%%%%%%%%%%%%%%%%%%%%%%%%%%%%%%%%%%%%%%%%%%%%%%%%%%%%%%%%%%
%%%%%%%%%%%%%%%%%%%%%%%%%%%%%%%%%%%%%%%%%%%%%%%%%%%%%%%%%%%%%%%%%%%%%%%%%%%%%%%%
\section{Introduction}

\LaTeX{} provides a mechanism to structure a large document (such as a book)
into a main file and several child files (containing the chapters)
using the |\include| command.
This mechanism is beneficial for documents
which span hundreds of pages in order to
make the source file(s) more manageable.
Moreover, compilation can be restricted to
selected child files by means of the |\includeonly| command.
The latter feature can be used to reduce the compilation time while editing
(this was significantly more useful in the earlier days of \LaTeX{})
or to generate a smaller document which is easier to navigate.
Another application of |\includeonly| is to generate
documents consisting of selected parts of the complete document.

However, there are a few drawbacks of the plain |\include| mechanism:
\begin{itemize}
\item
The child files cannot be compiled on their own,
they can only be compiled via the main file.
A naive editing environment
(such as a text editor with an option
to have the current file processed by \LaTeX)
may require one to switch to the main file before compiling;
attempting to compile the child file produces errors.
\item
The main file must be modified (each time)
to adjust the |\includeonly| command
to the present needs. This easily leaves the main file in a messy state.
\item
The generated document will always carry the filename
of the main document. This is inconvenient if
several child files are to be compiled and
to be kept for distribution.
\end{itemize}

The present package provides a simple interface
to make child files individually compilable by \LaTeX{}.
Compiling a child file then has the same effect as compiling
the main file with an |\includeonly| command
to select the appropriate child.
Moreover the generated document will carry the name of the child
rather than the main file.
This resolves all three above issues.

This feature is meant to make the editing of books,
thesis documents and lecture notes somewhat more convenient.
However, the package can also be used efficiently for
composing a series of documents (such as exercise sheets)
which are typically distributed individually.
It then assists the author in generating the individual documents
(potentially in different versions)
as well as a document containing the collected series.
Another application is in developing style files
or other kinds of included material
where compilation of the style file could redirect
to a sample or test file.

%%%%%%%%%%%%%%%%%%%%%%%%%%%%%%%%%%%%%%%%%%%%%%%%%%%%%%%%%%%%%%%%%%%%%%%%%%%%%%%%
%%%%%%%%%%%%%%%%%%%%%%%%%%%%%%%%%%%%%%%%%%%%%%%%%%%%%%%%%%%%%%%%%%%%%%%%%%%%%%%%
\section{Usage}

First of all, the package \textsf{childdoc} is \emph{not} a standard
\LaTeXe{} |.sty| style file! Therefore it needs to be invoked in
a non-standard way.

%%%%%%%%%%%%%%%%%%%%%%%%%%%%%%%%%%%%%%%%%%%%%%%%%%%%%%%%%%%%%%%%%%%%%%%%%%%%%%%%
\subsection{Included Files}
\label{sec:include}

%%%%%%%%%%%%%%%%%%%%%%%%%%%%%%%%%%%%%%%%
\DescribeMacro{\childdocmain}
To use the package, add the commands
\begin{center}
\begin{tabular}{l}
|\input{childdoc.def}|\\
|\childdocmain{}|\\
\end{tabular}
\end{center}
at the very top of the main \LaTeX{} file,
in particular \emph{before} the |\documentclass| statement!
The argument of |\childdocmain| should be left empty
(but it must be present).

%%%%%%%%%%%%%%%%%%%%%%%%%%%%%%%%%%%%%%%%
\DescribeMacro{\childdocof}
Furthermore, add the commands
\begin{center}
\begin{tabular}{l}
|\input{childdoc.def}|\\
|\childdocof{|\textit{main}|}|\\
\end{tabular}
\end{center}
at the top of every child file \textit{child}
which is included by |\include{|\textit{child}|}|
from within the main file
(or at least for those files to be compiled individually).
The argument \textit{main} must be the filename of the main file.

There are a couple of
considerations in setting up the main and child documents:

%%%%%%%%%%%%%%%%%%%%%%%%%%%%%%%%%%%%%%%%
\paragraph{Restrictions.}

Please note the following restrictions:
\begin{itemize}
\item
|\childdocmain| must be called with one argument \textit{main}
to ensure compatibility with earlier version of the package.
It must either be empty (|\childdocmain{}|)
or precisely match the filename of the main file in which it is specified.
See \secref{sec:detection} for further information.
\item
The filename \textit{main} must be specified without the |.tex| extension.
\item
The filename \textit{main} is case sensitive
(even in case-insensitive file systems)
due to internal string comparison.
\item
The argument \textit{main} should be fully expanded, it cannot be a macro.
\item
Subdirectories and special characters should be avoided in filenames.
\item
The command |\childdocmain{|\textit{main}|}| must be followed by a whitespace.
It should not be followed immediately by another command
or by a comment mark `|%|'.
This is because the \TeX{} parser reads the token immediately following
the argument of |\childdocmain| and puts it
at the beginning of every child section;
however, a white\-space is ignored.
\end{itemize}

%%%%%%%%%%%%%%%%%%%%%%%%%%%%%%%%%%%%%%%%
\paragraph{Content of Main File.}

It is advisable to place all content in the child files included by |\include|.
Any output contained in the main file will appear in all child documents
unless suppressed manually;
it cannot be suppressed automatically by the |\includeonly| directive
and thus should normally be avoided.
A method to include some content in the main file
by means of conditional processing is described in \secref{sec:conditional}.

%%%%%%%%%%%%%%%%%%%%%%%%%%%%%%%%%%%%%%%%
\paragraph{Page Numbering.}

When only a part of the document is compiled,
the appropriate numbering of pages
(as well as other status parameters)
is determined from the |.aux| files.
The latter contain information from previous passes.
However this information needs to propagate through
all intermediate child documents.
Therefore the page numbering in child documents may well
be inconsistent until the complete document is compiled at least once.

A useful (if unconventional) way to always ensure a consistent
page numbering is to restart the numbering in each child document
and denote the pages by `\textit{child}|.|\textit{page}'
where \textit{child} represents the chapter/section number of the child file.
This can be achieved by the command
|\numberwithin{page}{|\textit{child}|}|
of the \textsf{amsmath} package
where \textit{child} can be |chapter| or |section|
depending on the chosen structuring.
Alternatively, one can modify the macro |\thepage| appropriately
and reset the counter |page| at the start of each child file.

%%%%%%%%%%%%%%%%%%%%%%%%%%%%%%%%%%%%%%%%%%%%%%%%%%%%%%%%%%%%%%%%%%%%%%%%%%%%%%%%
\subsection{Conditional Processing}
\label{sec:conditional}

The package provides a mechanism to compile different versions
of a document. To customise the versions further some conditional processing
can come in handy to distinguish which version is being compiled.
The package provides two macros to describe the compilation context:

%%%%%%%%%%%%%%%%%%%%%%%%%%%%%%%%%%%%%%%%
\DescribeMacro{\ifchilddoc}
The conditional |\ifchilddoc| distinguishes between the compilation of
child documents and the main document:
%
\begin{center}
|\ifchilddoc |\textit{child-code}| |[|\||else |\textit{main-code}]| \||fi|
\end{center}

%%%%%%%%%%%%%%%%%%%%%%%%%%%%%%%%%%%%%%%%
\DescribeMacro{\childdocname}
\DescribeMacro{\childdocjob}
The macro |\childdocname| contains the filename (without extension)
of the main or child file being processed.
Note that |\childdocjob| will always contain the name of the main file.

%%%%%%%%%%%%%%%%%%%%%%%%%%%%%%%%%%%%%%%%
\paragraph{Title Page.}

Conditional processing can be used to include a title or banner page
in the main document when proper precautions are taken.
Importantly, the code in the main file should ensure that the page counter
(as well as other status parameters which are stored in the |.aux| files)
takes the same value after the conditional processing.
Otherwise the page numbers may take divergent values
depending on which part is compiled.

For example, a title page could be declared by:
%
\begin{center}
\begin{tabular}{l}
|\ifchilddoc\||else|\\
|\addtocounter{page}{-1}|\\
\textit{code for title page}\\
|\newpage|\\
|\||fi|
\end{tabular}
\end{center}
%
A banner page for the child documents can be generated by:
%
\begin{center}
\begin{tabular}{l}
|\ifchilddoc|\\
|\addtocounter{page}{-1}|\\
\textit{code for banner page}\\
|\newpage|\\
|\||fi|
\end{tabular}
\end{center}
%
Here one could write a message such as:
\begin{center}
|This is the part \childdocname{} of \childdocjob{}.|
\end{center}

%%%%%%%%%%%%%%%%%%%%%%%%%%%%%%%%%%%%%%%%%%%%%%%%%%%%%%%%%%%%%%%%%%%%%%%%%%%%%%%%
\subsection{Flags}
\label{sec:flags}

The package makes it easy to generate different versions
of the main or child documents.
To this end compilation flags can be defined
and assigned different default values.
They will be particularly useful in conjunction
with the forwarding mechanism described in \secref{sec:forward}.

For example, it may be useful to have a flag |\version|
which can be set to |draft| or |final|.
The document source will contain some conditional code
depending on the value of |\version|.
Suppose further, the flag should default to |final| for the main file
and to |draft| for child files
which is a natural assignment for editing the document.
This is achieved by placing the following code
in the preamble of the main document
(below the |\childdocmain| directive):
%
\begin{center}
\begin{tabular}{l}
|\ifchilddoc|\\
|\providecommand{\version}{draft}|\\
|\||else|\\
|\providecommand{\version}{final}|\\
|\||fi|
\end{tabular}
\end{center}
%
The definition by |\providecommand| makes sure
that previous definitions are not overwritten.
Further statements |\providecommand{\version}{...}|
can thus be added before the above code to override it.

For the main file, one might add a line
(between |\childdocmain| and the above block)
%
\begin{center}
|%\ifchilddoc\||else\providecommand{\version}{draft}\||fi|
\end{center}
%
which can be uncommented to produce a draft version.
Likewise one can add a line to the very top of a child file
(above the |\childdocof{|\textit{main}|}| directive)
%
\begin{center}
|%\providecommand{\version}{final}|
\end{center}
%
which can be uncommented to produce the final version of this child document.

%%%%%%%%%%%%%%%%%%%%%%%%%%%%%%%%%%%%%%%%%%%%%%%%%%%%%%%%%%%%%%%%%%%%%%%%%%%%%%%%
\subsection{Forwarding}
\label{sec:forward}

Different versions of the main or child documents
using compilation flags as described in \secref{sec:flags}
can be (permanently) stored in different files
for convenient compilation, viewing and distribution.
To this end, the package defines a command
to pass on compilation to a different file:

%%%%%%%%%%%%%%%%%%%%%%%%%%%%%%%%%%%%%%%%
\DescribeMacro{\childdocforward}
The command |\childdocforward| redirects processing to
another source file:
%
\begin{center}
\begin{tabular}{l}
|\input{childdoc.def}|\\
|\childdocforward[|\textit{main}|]{|\textit{dest}|}|\\
\end{tabular}
\end{center}
%
The argument \textit{dest} is the destination file
(without extension).
It should be the main file or one of the child files.
Note that further \textsf{childdoc} directives
such as |\childdocof| and |\childdocforward|
in the indicated file will be processed in this form.
The optional argument \textit{main}
passes on directly to the main file \textit{main}
while pretending to compile the child \textit{dest}.
This form behaves as if \textit{dest}
issues |\childdocof{|\textit{main}|}| right away,
and no further \textsf{childdoc} directives will be processed.

%%%%%%%%%%%%%%%%%%%%%%%%%%%%%%%%%%%%%%%%
\DescribeMacro{\...prefix}
In the alternative form |\childdocforwardprefix|,
%
\begin{center}
\begin{tabular}{l}
|\input{childdoc.def}|\\
|\childdocforwardprefix[|\textit{main}|]{|\textit{prefix}|}{|\textit{dest}|}|
\end{tabular}
\end{center}
%
the destination file is determined by a pattern
depending on the current file:
To make this work, the current file must be called
`{\textit{prefix}\hspace{0.2em}\textit{suffix}}'
with \textit{prefix} matching precisely the argument.
Processing is then passed on to the file
`{\textit{dest}\hspace{0.2em}\textit{suffix}}'.
Surely, the same effect is achieved by
directly specifying the
argument `{\textit{dest}\hspace{0.2em}\textit{suffix}}'
in the first form.
However, that requires to set up a different file
for each child. With the alternative form of the command
all these files can have exactly the same content
which simplifies setting them up and maintaining them.

For example, the following file |draft.tex|
with a compilation flag |\version| as described in \secref{sec:flags}
compiles the main document as a draft:
%
\begin{center}
\begin{tabular}{l}
|\def\version{draft}|\\
|\input{childdoc.def}|\\
|\childdocforward{|\textit{main}|}|
\end{tabular}
\end{center}
%
Likewise, the following files |final|\textit{nn}|.tex|
compile the final version of the child document
|child|\textit{nn}|.tex|:
%
\begin{center}
\begin{tabular}{l}
|\def\version{final}|\\
|\input{childdoc.def}|\\
|\childdocforwardprefix{final}{child}|
\end{tabular}
\end{center}
%

Note that when several versions of a main file and/or of each child file
are to be generated, it may be convenient to set up a |Makefile| or
shell script to automatise the process.

%%%%%%%%%%%%%%%%%%%%%%%%%%%%%%%%%%%%%%%%%%%%%%%%%%%%%%%%%%%%%%%%%%%%%%%%%%%%%%%%
\subsection{Command Line Processing}
\label{sec:commandline}

The effect of redirection files can also be achieved by invoking
the \LaTeX{} compiler with a more elaborate command line.
Most conveniently this should be done as part
of a shell script or a |Makefile|.

When using \textsf{childdoc} in the main file, the following
command lines effectively perform a redirection
(note that depending on the shell being used,
backslashes may have to be doubled: `|\|' $\to$ `|\\|'):
%
\begin{center}
|... -jobname "|\textit{target}|" |\\|"|[\textit{flags}]%
|\input{childdoc.def}\childdocforward[|\textit{main}|]{|\textit{dest}|}"|
\end{center}
%
Here \textit{target} is the name of the output file,
\textit{main} is the name of the main file
and \textit{dest} is the name of the main or child file to be processed
(all filenames without extensions).
The optional argument \textit{main} can be omitted
if \textit{main} matches \textit{dest}.
Optionally, compilation \textit{flags} can be defined via |\def| commands.
This command line makes the \TeX{} engine believe
it is compiling the file \textit{target}
whose content is specified as the latter parameter.
The provided code then forwards the processing to
\textit{main} or \textit{dest} as described in \secref{sec:forward}.

%%%%%%%%%%%%%%%%%%%%%%%%%%%%%%%%%%%%%%%%%%%%%%%%%%%%%%%%%%%%%%%%%%%%%%%%%%%%%%%%
\subsection{Include by Input}
\label{sec:input}

Including child documents by |\include| has some restrictions by design.
Most notably, the content of a child document always occupies
its own set of pages; pages cannot be shared between child documents.
Usually, this behaviour makes perfect sense
because each child document contain an essential part of the document.
However, in some situations it may be desirable to compose
a document from a collection of parts
without having mandatory page breaks between then.
For this case, the package
provides a mechanism to include parts
by |\input| which can also be processed individually.
However, by construction this mechanism
requires manual handling of the content to be output.

%%%%%%%%%%%%%%%%%%%%%%%%%%%%%%%%%%%%%%%%
\DescribeMacro{\ifchilddocmanual}
The main file should be prepared as usual, see \secref{sec:include}.
However, the document body must make a distinction
between processing of an individual part and of the main document, e.g.:
%
\begin{center}
\begin{tabular}{l}
|\ifchilddocmanual|\\
|\input{\childdocname}|\\
|\||else|\\
\textit{document body with }|\input{|\textit{part}|}|\\
|\||fi|
\end{tabular}
\end{center}
%
The conditional |\ifchilddocmanual| is true whenever
a part to be included by |\input| is being compiled,
and the name of the part is stored in |\childdocname|.

%%%%%%%%%%%%%%%%%%%%%%%%%%%%%%%%%%%%%%%%
\DescribeMacro{\childdocby}
Each part to be included by |\input| should start with:
%
\begin{center}
\begin{tabular}{l}
|\input{childdoc.def}|\\
|\childdocby{|\textit{main}|}|\\
\end{tabular}
\end{center}
%
The directive |\childdocby| is similar to |\childdocof|
described in \secref{sec:include},
but the subsequent selection of content must be done manually.
To that end, both |\ifchilddoc| and |\ifchilddocmanual|
will be true upon processing of a part,
and the name of the part is stored in |\childdocname|.
Note that |\jobname| will be set to the filename of the current part
so that each part receives an individual |.aux| file
that does not interfere with the |.aux| file(s) of the main document.
This behaviour can be altered by the alternative form
|\childdocby[*]{|\textit{main}|}| (with a non-empty optional argument)
which uses the |.aux| file of the main document
by setting |\jobname| to \textit{main}.

%%%%%%%%%%%%%%%%%%%%%%%%%%%%%%%%%%%%%%%%%%%%%%%%%%%%%%%%%%%%%%%%%%%%%%%%%%%%%%%%
\subsection{Driver Development}
\label{sec:driver}

The \textsf{childdoc} mechanism can also be use for the development
of definition files such as \LaTeX{} styles or classes.
This case differs from the above setup with multiple parts
included by |\include| in that no |\includeonly| should be invoked.
This can be achieved by starting the include file
(before |\ProvidesPackage|) with:
%
\begin{center}
\begin{tabular}{l}
|\input{childdoc.def}|\\
|\childdocforward{|\textit{main}|}|\\
\end{tabular}
\end{center}
%
or alternatively with:
%
\begin{center}
\begin{tabular}{l}
|\input{childdoc.def}|\\
|\childdocby{|\textit{main}|}|\\
\end{tabular}
\end{center}
%
Both forms have slightly different effects as described above.
The main file is prepared as usual, see \secref{sec:include}.

%%%%%%%%%%%%%%%%%%%%%%%%%%%%%%%%%%%%%%%%%%%%%%%%%%%%%%%%%%%%%%%%%%%%%%%%%%%%%%%%
\subsection{Legacy Detection}
\label{sec:detection}

The directive |\childdocmain| in the main file can detect
whether the complete document or merely a child is to be compiled
even without using the directive |\childdocof|.
This method is deprecated because it is less robust
and there is no compelling reason to use it;
it is merely provided for backward compatibility
and it may be removed in future versions.

If the detection mechanism is to be used,
it is mandatory to correctly specify
the filename of the main file as the argument of |\childdocmain|:
%
\begin{center}
\begin{tabular}{l}
|\input{childdoc.def}|\\
|\childdocmain{|\textit{main}|}|\\
\end{tabular}
\end{center}
%
If |\jobname| does not match the argument \textit{main} of |\childdocmain|,
it is assumed that |\jobname| points to the child file to be compiled.
When using |\childdocmain| with the main file specified as argument,
it suffices to start a child file
with just |\input{|\textit{main}|}|
without loading of the package and using |\childdocof|.
If instead all processing is done
with the appropriate \textsf{childdoc} directives,
the argument of \textit{main} of |\childdocmain| can be empty.

An alternative version of the command line processing described
in \secref{sec:commandline} using the detection mechanism reads:
%
\begin{center}
|... -jobname "|\textit{target}|" "|[\textit{flags}]%
[|\def\jobname{|\textit{dest}|}|]|\input{|\textit{main}|}"|
\end{center}

%%%%%%%%%%%%%%%%%%%%%%%%%%%%%%%%%%%%%%%%%%%%%%%%%%%%%%%%%%%%%%%%%%%%%%%%%%%%%%%%
\subsection{Manual Code}
\label{sec:manual}

In case one cannot be certain whether the definitions file |childdoc.def|
is installed on the target \TeX{} distribution
and one prefers not to ship it,
it is conceivable to paste a few relevant commands into the sources.

To that end, drop all statements |\input{childdoc.def}|
and perform the replacements as outlined below.
Instead of |\childdocmain{|\textit{main}|}| add the following code
to the top of the main file:
%
\begin{center}
\begin{tabular}{l}
|\||ifdefined\childdocname\endinput\||fi\newif\ifchilddoc|\\
|\edef\childdocname{\scantokens\expandafter{\jobname\noexpand}}|\\
|\def\childdocmain{|\textit{main}|}\||ifx\childdocmain\childdocname\||else|\\
|\childdoctrue\includeonly{\childdocname}\let\jobname\childdocmain\||fi|\\
\end{tabular}
\end{center}
%
Instead of |\childdocof{|\textit{main}|}| just include the main file
at the top of each child file:
%
\begin{center}
|\input{|\textit{main}|}|
\end{center}
%
A simple redirection |\childdocforward{|\textit{dest}|}| is achieved by:
%
\begin{center}
|\def\jobname{|\textit{dest}|}\input{\jobname}|
\end{center}
%
The redirection with prefix
|\childdocforwardprefix[|\textit{prefix}|]{|\textit{dest}|}|
is accomplished by:
%
\begin{center}
\begin{tabular}{l}
|{\edef\jobname{\scantokens\expandafter{\jobname\noexpand}}|\\
|\def\redirectjob |\textit{prefix}|#1~~~{\gdef\jobname{|\textit{dest}|#1}}|\\
|\expandafter\redirectjob\jobname~~~}\input{\jobname}|
\end{tabular}
\end{center}

In an alternative approach,
child documents can be compiled by a specific command line
without additional code or specific definitions:
%
\begin{center}
|... -jobname "|\textit{target}|" "|[\textit{flags}]%
|\includeonly{|\textit{dest}|}\input{|\textit{main}|}"|
\end{center}
%

%%%%%%%%%%%%%%%%%%%%%%%%%%%%%%%%%%%%%%%%%%%%%%%%%%%%%%%%%%%%%%%%%%%%%%%%%%%%%%%%
%%%%%%%%%%%%%%%%%%%%%%%%%%%%%%%%%%%%%%%%%%%%%%%%%%%%%%%%%%%%%%%%%%%%%%%%%%%%%%%%
\section{Information}

%%%%%%%%%%%%%%%%%%%%%%%%%%%%%%%%%%%%%%%%%%%%%%%%%%%%%%%%%%%%%%%%%%%%%%%%%%%%%%%%
\subsection{Copyright}

Copyright \copyright{} 2017--2018 Niklas Beisert

This work may be distributed and/or modified under the
conditions of the \LaTeX{} Project Public License, either version 1.3
of this license or (at your option) any later version.
The latest version of this license is in
  \url{http://www.latex-project.org/lppl.txt}
and version 1.3 or later is part of all distributions of \LaTeX{}
version 2005/12/01 or later.

This work has the LPPL maintenance status `maintained'.

The Current Maintainer of this work is Niklas Beisert.

This work consists of the files |README.txt|, |childdoc.ins| and |childdoc.dtx|
as well as the derived files |childdoc.def|, |cdocsamp.tex|
with |cdocsch1.tex|, |cdocsch2.tex|, |cdocspt3.tex|, |cdocspt4.tex|,
|cdocsdrf.tex|, |cdocsfn1.tex|, |cdocsfn2.tex|
as well as |childdoc.pdf|.

%%%%%%%%%%%%%%%%%%%%%%%%%%%%%%%%%%%%%%%%%%%%%%%%%%%%%%%%%%%%%%%%%%%%%%%%%%%%%%%%
\subsection{Files and Installation}

The package consists of the files:
%
\begin{center}
\begin{tabular}{ll}
    |README.txt|   & readme file \\
    |childdoc.ins| & installation file \\
    |childdoc.dtx| & source file \\
    |childdoc.def| & definition file \\
    |cdocsamp.tex| & sample main file \\
    |cdocsch1.tex| & sample include file \\
    |cdocsch2.tex| & sample include file \\
    |cdocspt3.tex| & sample part file \\
    |cdocspt4.tex| & sample part file \\
    |cdocsdrf.tex| & sample redirection file \\
    |cdocsfn1.tex| & sample redirection file \\
    |cdocsfn2.tex| & sample redirection file \\
    |childdoc.pdf| & manual
\end{tabular}
\end{center}
%
The distribution consists of the files
|README.txt|, |childdoc.ins| and |childdoc.dtx|.
%
\begin{itemize}
\item
Run (pdf)\LaTeX{} on |childdoc.dtx|
to compile the manual |childdoc.pdf| (this file).
\item
Run \LaTeX{} on |childdoc.ins| to create the definitions file |childdoc.def|
and the sample |cdocsamp.tex| with include files
|cdocsch1.tex|, |cdocsch2.tex|, |cdocspt3.tex|, |cdocspt4.tex|,
|cdocsdrf.tex|, |cdocsfn1.tex|, |cdocsfn2.tex|.
Then copy the file |childdoc.def| to an appropriate directory of your \LaTeX{}
distribution, e.g.\ \textit{texmf-root}|/tex/latex/childdoc|.
\end{itemize}

%%%%%%%%%%%%%%%%%%%%%%%%%%%%%%%%%%%%%%%%%%%%%%%%%%%%%%%%%%%%%%%%%%%%%%%%%%%%%%%%
\subsection{Related CTAN Packages}

There are several other packages which offer a similar functionality:
%
\begin{itemize}
\item
The packages
\href{http://ctan.org/pkg/docmute}{\textsf{docmute}},
\href{http://ctan.org/pkg/includex}{\textsf{includex}} and
\href{http://ctan.org/pkg/standalone}{\textsf{standalone}}
provide commands to include only the document body of
a child file thus allowing both files to be compiled individually.
\item
The packages \href{http://ctan.org/pkg/subdocs}{\textsf{subdocs}}
and \href{http://ctan.org/pkg/subfiles}{\textsf{subfiles}}
provide structures in which the main and child documents can be
encapsulated and allowing them to be compiled individually.
The inclusion mechanism is different from the conventional |\include|.
\item
The package \href{http://ctan.org/pkg/combine}{\textsf{combine}}
is an elaborate solution to combine several documents into one.
\end{itemize}
%
See also the CTAN topic \href{http://ctan.org/topic/subdocs}{\textsf{subdocs}}
for further related packages.
The present package differs from the above solutions in that
a document structure constructed with the conventional |\include| mechanism
just needs two extra commands at the top of every file
such that all constituent files can be compiled individually.

%%%%%%%%%%%%%%%%%%%%%%%%%%%%%%%%%%%%%%%%%%%%%%%%%%%%%%%%%%%%%%%%%%%%%%%%%%%%%%%%
%\subsection{Feature Suggestions}
%
%The following is a list of features which may be useful for future
%versions of this package:
%%
%\begin{itemize}
%\item
%\ldots
%\end{itemize}

%%%%%%%%%%%%%%%%%%%%%%%%%%%%%%%%%%%%%%%%%%%%%%%%%%%%%%%%%%%%%%%%%%%%%%%%%%%%%%%%
\subsection{Revision History}

%%%%%%%%%%%%%%%%%%%%%%%%%%%%%%%%%%%%%%%%
\paragraph{v2.0:} 2018/12/30

\begin{itemize}
\item
immediate forward processing
\item
added |\childdocby| mechanism
\item
manual restructured
\end{itemize}

%%%%%%%%%%%%%%%%%%%%%%%%%%%%%%%%%%%%%%%%
\paragraph{v1.6:} 2018/01/17

\begin{itemize}
\item
application for development of include files
\item
corrections to manual
\end{itemize}

%%%%%%%%%%%%%%%%%%%%%%%%%%%%%%%%%%%%%%%%
\paragraph{v1.5:} 2017/05/21

\begin{itemize}
\item
more complete structuring introduced
\item
|\childdocof| introduced
\item
|\childdoc| renamed to |\childdocmain|
\item
|\childredirect| renamed to |\childdocforward| and |\childdocforwardprefix|
and functionality expanded
\end{itemize}

%%%%%%%%%%%%%%%%%%%%%%%%%%%%%%%%%%%%%%%%
\paragraph{v1.0:} 2017/04/27

\begin{itemize}
\item
manual and install package
\item
first version published on CTAN
\end{itemize}

%%%%%%%%%%%%%%%%%%%%%%%%%%%%%%%%%%%%%%%%
\paragraph{v0.6:} 2017/04/26

\begin{itemize}
\item
redirection mechanism added
\end{itemize}

%%%%%%%%%%%%%%%%%%%%%%%%%%%%%%%%%%%%%%%%
\paragraph{v0.5:} 2017/04/26

\begin{itemize}
\item
functionality in definition file
\end{itemize}


%%%%%%%%%%%%%%%%%%%%%%%%%%%%%%%%%%%%%%%%%%%%%%%%%%%%%%%%%%%%%%%%%%%%%%%%%%%%%%%%
%%%%%%%%%%%%%%%%%%%%%%%%%%%%%%%%%%%%%%%%%%%%%%%%%%%%%%%%%%%%%%%%%%%%%%%%%%%%%%%%
%%%%%%%%%%%%%%%%%%%%%%%%%%%%%%%%%%%%%%%%%%%%%%%%%%%%%%%%%%%%%%%%%%%%%%%%%%%%%%%%
\appendix

\settowidth\MacroIndent{\rmfamily\scriptsize 000\ }

 \DocInput{childdoc.dtx}

\end{document}
%</driver>
% \fi
%
% %%%%%%%%%%%%%%%%%%%%%%%%%%%%%%%%%%%%%%%%%%%%%%%%%%%%%%%%%%%%%%%%%%%%%%%%%%%%%%
% %%%%%%%%%%%%%%%%%%%%%%%%%%%%%%%%%%%%%%%%%%%%%%%%%%%%%%%%%%%%%%%%%%%%%%%%%%%%%%
% \section{Sample}
%\iffalse
%<*samplemain>
%\fi
%
% The following presents a sample document
% with two chapters, two parts, a title page,
% a compile flag as well as three forwarding files to set the flag.
% It consists of eight |.tex| files:
% \begin{center}
% \begin{tabular}{ll}
% |cdocsamp.tex|&main file\\
% |cdocsch1.tex|&include file for chapter 1\\
% |cdocsch2.tex|&include file for chapter 2\\
% |cdocspt3.tex|&include file for part 3\\
% |cdocspt4.tex|&include file for part 4\\
% |cdocsdrf.tex|&forwarding file for main file in draft mode\\
% |cdocsfi1.tex|&forwarding file for final version of chapter 1\\
% |cdocsfi2.tex|&forwarding file for final version of chapter 2\\
% \end{tabular}
% \end{center}
% Each of the eight files can be compiled directly by the \LaTeX{} compiler.
%
% %%%%%%%%%%%%%%%%%%%%%%%%%%%%%%%%%%%%%%
% \paragraph{Main File.}
%
% The main file is called |cdocsamp.tex|.
%
% Load the \textsf{childdoc} definitions and
% declare the filename for the main document:
%    \begin{macrocode}
\input{childdoc.def}
\childdocmain{}
%    \end{macrocode}

% Optional override for |\version| flag:
%    \begin{macrocode}
%%\ifchilddoc\else\providecommand{\version}{draft}\fi
%    \end{macrocode}

% Define the default values for the |\version| flag
% (|final| for the main file and |draft| for childs):
%    \begin{macrocode}
\ifchilddoc
\providecommand{\version}{draft}
\else
\providecommand{\version}{final}
\fi
%    \end{macrocode}

% Load the standard document class:
%    \begin{macrocode}
\documentclass[12pt]{article}
%    \end{macrocode}

% Start the document body:
%    \begin{macrocode}
\begin{document}
%    \end{macrocode}

% Declare a title page.
% Print title, part of document being processed and version flag:
%    \begin{macrocode}
\addtocounter{page}{-1}
\begin{center}
{\LARGE\bfseries{}childdoc example\par}
\vspace{1cm}
\ifchilddoc
\ifchilddocmanual part\else chapter\fi:
`\childdocname' of `\childdocjob'\par
\else
main document: `\childdocjob'\par
\fi
version: \version\par
\end{center}
\newpage
%    \end{macrocode}

% Manually include selected file,
% otherwise process as usual:
%    \begin{macrocode}
\ifchilddocmanual
\section*{part `\childdocname'}
\input{\childdocname}
\else
%    \end{macrocode}

% Include the two chapters:
%    \begin{macrocode}
\include{cdocsch1}
\include{cdocsch2}
%    \end{macrocode}

% Include the two parts unless only chapters should be displayed:
%    \begin{macrocode}
\ifchilddoc\else
\section{part three}
\input{cdocspt3}
\section{part four}
\input{cdocspt4}
\fi
%    \end{macrocode}

% Process as usual until here:
%    \begin{macrocode}
\fi
%    \end{macrocode}

% End of document body:
%    \begin{macrocode}
\end{document}
%    \end{macrocode}
%\iffalse
%</samplemain>
%\fi
%
% %%%%%%%%%%%%%%%%%%%%%%%%%%%%%%%%%%%%%%
% \paragraph{Chapter Include Files.}
%
% The include files are called |cdocsch1.tex| and |cdocsch2.tex|.
%
%\iffalse
%<*samplechap1|samplechap2>
%\fi

% Optional override for |\version| flag:
%    \begin{macrocode}
%%\providecommand{\version}{final}
%    \end{macrocode}

% Include the main document:
%    \begin{macrocode}
\input{childdoc.def}
\childdocof{cdocsamp}
%    \end{macrocode}

%\iffalse
%</samplechap1|samplechap2>
%\fi
%
%\iffalse
%<*samplechap1>
%\fi
% Some text for chapter 1:
%    \begin{macrocode}
\section{one}
some text in chapter one
%    \end{macrocode}

%\iffalse
%</samplechap1>
%\fi
% Some text for chapter 2:
%\iffalse
%<*samplechap2>
%\fi
%    \begin{macrocode}
\section{two}
more text in chapter two
%    \end{macrocode}

%\iffalse
%</samplechap2>
%\fi
%
% %%%%%%%%%%%%%%%%%%%%%%%%%%%%%%%%%%%%%%
% \paragraph{Part Include Files.}
%
% The include files are called |cdocspt3.tex| and |cdocspt4.tex|.
%
%\iffalse
%<*samplepart3|samplepart4>
%\fi

% Optional override for |\version| flag:
%    \begin{macrocode}
%%\providecommand{\version}{final}
%    \end{macrocode}

% Include the main document:
%    \begin{macrocode}
\input{childdoc.def}
\childdocby{cdocsamp}
%    \end{macrocode}

%\iffalse
%</samplepart3|samplepart4>
%\fi
%
%\iffalse
%<*samplepart3>
%\fi
% Some text for part 3:
%    \begin{macrocode}
some text in part three
%    \end{macrocode}

%\iffalse
%</samplepart3>
%\fi
% Some text for part 4:
%\iffalse
%<*samplepart4>
%\fi
%    \begin{macrocode}
more text in part four
%    \end{macrocode}

%\iffalse
%</samplepart4>
%\fi
%
% %%%%%%%%%%%%%%%%%%%%%%%%%%%%%%%%%%%%%%
% \paragraph{Forwarding for a Complete Draft.}
%
% The following forwarding file |cdocsdrf.tex|
% compiles the main document in draft mode:
%\iffalse
%<*sampledraft>
%\fi
%    \begin{macrocode}
\def\version{draft}
\input{childdoc.def}
\childdocforward{cdocsamp}
%    \end{macrocode}

%\iffalse
%</sampledraft>
%\fi
%
% %%%%%%%%%%%%%%%%%%%%%%%%%%%%%%%%%%%%%%
% \paragraph{Forwarding for Final Version of the Chapters.}
%
% The following forwarding files |cdocsfn1.tex| and |cdocsfn2.tex|
% (with identical content)
% compile the final versions of the child documents
% |cdocsch1.tex| and |cdocsch2.tex|, respectively:
%\iffalse
%<*samplefinal>
%\fi
%    \begin{macrocode}
\def\version{final}
\input{childdoc.def}
\childdocforwardprefix[cdocsamp]{cdocsfn}{cdocsch}
%    \end{macrocode}

%\iffalse
%</samplefinal>
%\fi
%
% %%%%%%%%%%%%%%%%%%%%%%%%%%%%%%%%%%%%%%
% \paragraph{Command Line Processing.}
%
% The following three command lines generate the output files
% |cdocscld|, |cdocscl1| and |cdocscl2|
% which should be identical to
% |cdocsdrf|, |cdocsch1| and |cdocsfn2|, respectively:
% \begin{center}
% \begin{tabular}{l}
% |latex -jobname cdocscld \|\\
% |  "\def\version{draft}\input{childdoc.def}\childdocforward{cdocsamp}"|\\
% |latex -jobname cdocscl1 \|\\
% |  "\input{childdoc.def}\childdocforward[cdocsamp]{cdocsch1}"|\\
% |latex -jobname cdocscl2 \|\\
% |  "\def\version{final}\input{childdoc.def}\childdocforward{cdocsch2}"|
% \end{tabular}
% \end{center}
% Note that the trailing backslash on each first line
% merely continues the input to the second line
% (for convenient cut ant paste).
% Furthermore, the command |latex| can be replaced by any
% of its alternative versions such as |pdflatex|.
%
% %%%%%%%%%%%%%%%%%%%%%%%%%%%%%%%%%%%%%%%%%%%%%%%%%%%%%%%%%%%%%%%%%%%%%%%%%%%%%%
% %%%%%%%%%%%%%%%%%%%%%%%%%%%%%%%%%%%%%%%%%%%%%%%%%%%%%%%%%%%%%%%%%%%%%%%%%%%%%%
% \section{Implementation}
%\iffalse
%<*package>
%\fi
%
% This section describes the definitions file |childdoc.def|.

% The definitions cannot be loaded using |\usepackage| or |\RequirePackage|
% which has a mechanism to prevent loading a style file more than once.
% When loading the definitions by means of |\input|
% multiple instances have to be prevented manually:
%\iffalse
%This code needs to be before the `\ProvidesFile' directive
%which is defined at the beginning of this file.
%Therefore it is also placed there and commented out here.
%</package>
%<*discard>
%\fi
%    \begin{macrocode}
\ifdefined\childdocmain\endinput\fi
%    \end{macrocode}
%\iffalse
%</discard>
%<*package>
%\fi
%
% \macro{\ifchilddoc}
% \macro{\ifchilddocmanual}
% The conditional |\ifchilddoc| tells whether a
% child (true) or main (false) document is being compiled.
% The conditional |\ifchilddocmanual| tells whether
% the |\includeonly| mechanism is used (false) or
% the selection of child files must be performed manually (true).
% The definitions initialise to false:
%    \begin{macrocode}
\newif\ifchilddoc
\newif\ifchilddocmanual
%    \end{macrocode}

% \macro{\childdocname}
% \macro{\childdocjob}
% The macro |\childdocname| stores the name of the main document
% to be compiled. The macro |\childdocjob| stores the name of
% the document on which the \LaTeX{} compiler was originally invoked.
% The content of |\jobname| cannot be compared
% to filenames specified in the source due to different catcodes.
% The following code rescans |\jobname|, stores the result
% in |\childdocname| and saves a copy in |\childdocjob|:
%    \begin{macrocode}
\edef\childdocname{\scantokens\expandafter{\jobname\noexpand}}
\let\childdocjob\childdocname
%    \end{macrocode}

% \macro{\childdocdisable}
% The macro |\childdocdisable| prevents the main file
% from being processed more than once.
% At this stage, the main document command |\childdocmain|
% is assumed to be called once again where it should do nothing.
% Any subsequent call to it should prevent
% a secondary processing of the main document
% It overwrites the forwarding commands
% |\childdocof| and |\childdocforward|
% with empty macros to prevent further inclusions of the main document:
%    \begin{macrocode}
\newcommand{\childdocdisable}
{
  \renewcommand{\childdocmain}[1]{\renewcommand{\childdocmain}[1]{\endinput}}
  \renewcommand{\childdocof}[1]{}
  \renewcommand{\childdocby}[2][]{}
  \renewcommand{\childdocforward}[2][]{}
  \renewcommand{\childdocdisable}{}
}
%    \end{macrocode}

% \macro{\childdocmain}
% The macro |\childdocmain| is to be called at the top of the main file
% with nothing or the main filename (without extension) as argument.
% First, it breaks loops.
% If the argument is not empty and does not match |\childdocname|
% (which is set by the first inclusion of |childdoc.def|),
% |\ifchilddoc| is set to true, |\includeonly| is applied to the child file
% and |\jobname| is set to the main file
% (for proper handling of |.aux| files):
%    \begin{macrocode}
\newcommand{\childdocmain}[1]
{
  \childdocdisable\childdocmain{}
  \if?#1?\else
    \begingroup
      \def\childdoctmp{#1}
      \ifx\childdoctmp\childdocname
        \def\childdoctmp{}
      \else
        \def\childdoctmp
        {
          \childdoctrue
          \includeonly{\childdocname}
          \def\childdocjob{#1}
          \def\jobname{#1}
        }
      \fi
      \expandafter
    \endgroup
    \childdoctmp
  \fi
}
%    \end{macrocode}

% \macro{\childdocof}
% The command |\childdocof| redirects
% compilation to the main file |#1|.
%    \begin{macrocode}
\newcommand{\childdocof}[1]
{
  \childdocdisable
  \childdoctrue
  \includeonly{\childdocname}
  \def\jobname{#1}
  \def\childdocjob{#1}
  \input{#1}
}
%    \end{macrocode}

% \macro{\childdocby}
% The command |\childdocby| ....
%    \begin{macrocode}
\newcommand{\childdocby}[2][]
{
  \childdocdisable
  \childdoctrue
  \childdocmanualtrue
  \if?#1?\else
    \def\jobname{#2}
  \fi
  \def\childdocjob{#2}
  \input{#2}
  \endinput
}
%    \end{macrocode}

% \macro{\childdocforward}
% The command |\childdocforward| redirects
% compilation to the main file or
% (if the optional argument is given) a child file.
% Parameters are set as if the main file
% or a child file starting with |\childdocof| was compiled.
% Then compilation is handed over to the main file:
%    \begin{macrocode}
\newcommand{\childdocforward}[2][]
{
  \begingroup
    \if?#1?
      \def\childdoctmp
      {
        \def\childdocname{#2}
        \def\childdocjob{#2}
        \def\jobname{#2}
        \input{#2}
        \endinput
      }
    \else
      \def\childdoctmp
      {
        \childdocdisable
        \def\childdocname{#2}
        \childdoctrue
        \includeonly{#2}
        \def\childdocjob{#1}
        \def\jobname{#1}
        \input{#1}
        \endinput
      }
    \fi
    \expandafter
  \endgroup
  \childdoctmp
}
%    \end{macrocode}

% \macro{\childdocforwardprefix}
% The command |\childdocforwardprefix| redirects
% compilation to the main or a child file by means of a pattern.
% The prefix |#1| in the current filename is replaced by |#2|
% and the suffix of the current filename is kept
% (it is assumed that the filename does not contain the substring `|~~~|'
% which is used as a delimiter).
% Compilation is handed over to the new file by |\childdocforward|:
%    \begin{macrocode}
\newcommand{\childdocforwardprefix}[3][]
{
  \begingroup
    \def\childdocextract #2##1~~~{\def\childdoctmp{\childdocforward[#1]{#3##1}}}
    \expandafter\childdocextract\childdocname~~~
    \expandafter
  \endgroup
  \childdoctmp
}
%    \end{macrocode}

% \macro{\childdoc}
% The deprecated macro |\childdoc| is a legacy version of |\childdocmain|:
%    \begin{macrocode}
\newcommand{\childdoc}{\childdocmain}
%    \end{macrocode}

% \macro{\childdocredirect}
% The deprecated macro |\childdocredirect| is a legacy version
% of |\childdocforward| and |\childdocforwardprefix|:
%    \begin{macrocode}
\newcommand{\childdocredirect}[2][]
{
  \begingroup
    \if?#1?
      \def\childdoctmp{\childdocforward{#2}}
    \else
      \def\childdoctmp{\childdocforwardprefix{#1}{#2}}
    \fi
    \expandafter
  \endgroup
  \childdoctmp
}
%    \end{macrocode}

%\iffalse
%</package>
%\fi
%
\endinput
|\\
|\childdocforwardprefix{final}{child}|
\end{tabular}
\end{center}
%

Note that when several versions of a main file and/or of each child file
are to be generated, it may be convenient to set up a |Makefile| or
shell script to automatise the process.

%%%%%%%%%%%%%%%%%%%%%%%%%%%%%%%%%%%%%%%%%%%%%%%%%%%%%%%%%%%%%%%%%%%%%%%%%%%%%%%%
\subsection{Command Line Processing}
\label{sec:commandline}

The effect of redirection files can also be achieved by invoking
the \LaTeX{} compiler with a more elaborate command line.
Most conveniently this should be done as part
of a shell script or a |Makefile|.

When using \textsf{childdoc} in the main file, the following
command lines effectively perform a redirection
(note that depending on the shell being used,
backslashes may have to be doubled: `|\|' $\to$ `|\\|'):
%
\begin{center}
|... -jobname "|\textit{target}|" |\\|"|[\textit{flags}]%
|% \iffalse
%
% childdoc.dtx Copyright (C) 2017-2018 Niklas Beisert
%
% This work may be distributed and/or modified under the
% conditions of the LaTeX Project Public License, either version 1.3
% of this license or (at your option) any later version.
% The latest version of this license is in
%   http://www.latex-project.org/lppl.txt
% and version 1.3 or later is part of all distributions of LaTeX
% version 2005/12/01 or later.
%
% This work has the LPPL maintenance status `maintained'.
%
% The Current Maintainer of this work is Niklas Beisert.
%
% This work consists of the files childdoc.dtx and childdoc.ins
% and the derived files childdoc.def and cdocsamp.tex with
% cdocsch1.tex, cdocsch2.tex, cdocsdrf.tex, cdocsfn1.tex, cdocsfn2.tex.
%
%<package>\ifdefined\childdocmain\endinput\fi
%<package>\ProvidesFile{childdoc.def}[2018/12/30 v2.0 child document driver]
%<samplemain>\ProvidesFile{cdocsamp.tex}[2018/12/30 v2.0 sample for childdoc]
%<*driver>
%\ProvidesFile{childdoc.drv}[2018/12/30 v2.0 childdoc reference manual file]
\PassOptionsToClass{10pt,a4paper}{article}
\documentclass{ltxdoc}

\usepackage[margin=35mm]{geometry}
\usepackage{hyperref}
\usepackage{hyperxmp}
\usepackage[usenames]{color}

\hypersetup{colorlinks=true}
\hypersetup{pdfstartview=FitH}
\hypersetup{pdfpagemode=UseNone}
\hypersetup{pdfsource={}}
\hypersetup{pdflang={en-UK}}
\hypersetup{pdfcopyright={Copyright 2017-2018 Niklas Beisert.
  This work may be distributed and/or modified under the
  conditions of the LaTeX Project Public License, either version 1.3
  of this license or (at your option) any later version.}}
\hypersetup{pdflicenseurl={http://www.latex-project.org/lppl.txt}}
\hypersetup{pdfcontactaddress={ETH Zurich, ITP, HIT K,
  Wolfgang-Pauli-Strasse 27}}
\hypersetup{pdfcontactpostcode={8093}}
\hypersetup{pdfcontactcity={Zurich}}
\hypersetup{pdfcontactcountry={Switzerland}}
\hypersetup{pdfcontactemail={nbeisert@itp.phys.ethz.ch}}
\hypersetup{pdfcontacturl={http://people.phys.ethz.ch/\xmptilde nbeisert/}}

\newcommand{\secref}[1]{\hyperref[#1]{section \ref*{#1}}}

\parskip1ex
\parindent0pt
\let\olditemize\itemize
\def\itemize{\olditemize\parskip0pt}

\begin{document}

\title{The \textsf{childdoc} Package}
\hypersetup{pdftitle={The childdoc Package}}
\author{Niklas Beisert\\[2ex]
  Institut f\"ur Theoretische Physik\\
  Eidgen\"ossische Technische Hochschule Z\"urich\\
  Wolfgang-Pauli-Strasse 27, 8093 Z\"urich, Switzerland\\[1ex]
  \href{mailto:nbeisert@itp.phys.ethz.ch}
  {\texttt{nbeisert@itp.phys.ethz.ch}}}
\hypersetup{pdfauthor={Niklas Beisert}}
\hypersetup{pdfsubject={Manual for the LaTeX2e Package childdoc}}
\date{30 December 2018, \textsf{v2.0}}
\maketitle

\begin{abstract}\noindent
\textsf{childdoc} is a \LaTeXe{} package
that enables the direct compilation
of document sections included by |\include|
to individual files.
\end{abstract}

\begingroup
\parskip0ex
\tableofcontents
\endgroup

%%%%%%%%%%%%%%%%%%%%%%%%%%%%%%%%%%%%%%%%%%%%%%%%%%%%%%%%%%%%%%%%%%%%%%%%%%%%%%%%
%%%%%%%%%%%%%%%%%%%%%%%%%%%%%%%%%%%%%%%%%%%%%%%%%%%%%%%%%%%%%%%%%%%%%%%%%%%%%%%%
\section{Introduction}

\LaTeX{} provides a mechanism to structure a large document (such as a book)
into a main file and several child files (containing the chapters)
using the |\include| command.
This mechanism is beneficial for documents
which span hundreds of pages in order to
make the source file(s) more manageable.
Moreover, compilation can be restricted to
selected child files by means of the |\includeonly| command.
The latter feature can be used to reduce the compilation time while editing
(this was significantly more useful in the earlier days of \LaTeX{})
or to generate a smaller document which is easier to navigate.
Another application of |\includeonly| is to generate
documents consisting of selected parts of the complete document.

However, there are a few drawbacks of the plain |\include| mechanism:
\begin{itemize}
\item
The child files cannot be compiled on their own,
they can only be compiled via the main file.
A naive editing environment
(such as a text editor with an option
to have the current file processed by \LaTeX)
may require one to switch to the main file before compiling;
attempting to compile the child file produces errors.
\item
The main file must be modified (each time)
to adjust the |\includeonly| command
to the present needs. This easily leaves the main file in a messy state.
\item
The generated document will always carry the filename
of the main document. This is inconvenient if
several child files are to be compiled and
to be kept for distribution.
\end{itemize}

The present package provides a simple interface
to make child files individually compilable by \LaTeX{}.
Compiling a child file then has the same effect as compiling
the main file with an |\includeonly| command
to select the appropriate child.
Moreover the generated document will carry the name of the child
rather than the main file.
This resolves all three above issues.

This feature is meant to make the editing of books,
thesis documents and lecture notes somewhat more convenient.
However, the package can also be used efficiently for
composing a series of documents (such as exercise sheets)
which are typically distributed individually.
It then assists the author in generating the individual documents
(potentially in different versions)
as well as a document containing the collected series.
Another application is in developing style files
or other kinds of included material
where compilation of the style file could redirect
to a sample or test file.

%%%%%%%%%%%%%%%%%%%%%%%%%%%%%%%%%%%%%%%%%%%%%%%%%%%%%%%%%%%%%%%%%%%%%%%%%%%%%%%%
%%%%%%%%%%%%%%%%%%%%%%%%%%%%%%%%%%%%%%%%%%%%%%%%%%%%%%%%%%%%%%%%%%%%%%%%%%%%%%%%
\section{Usage}

First of all, the package \textsf{childdoc} is \emph{not} a standard
\LaTeXe{} |.sty| style file! Therefore it needs to be invoked in
a non-standard way.

%%%%%%%%%%%%%%%%%%%%%%%%%%%%%%%%%%%%%%%%%%%%%%%%%%%%%%%%%%%%%%%%%%%%%%%%%%%%%%%%
\subsection{Included Files}
\label{sec:include}

%%%%%%%%%%%%%%%%%%%%%%%%%%%%%%%%%%%%%%%%
\DescribeMacro{\childdocmain}
To use the package, add the commands
\begin{center}
\begin{tabular}{l}
|\input{childdoc.def}|\\
|\childdocmain{}|\\
\end{tabular}
\end{center}
at the very top of the main \LaTeX{} file,
in particular \emph{before} the |\documentclass| statement!
The argument of |\childdocmain| should be left empty
(but it must be present).

%%%%%%%%%%%%%%%%%%%%%%%%%%%%%%%%%%%%%%%%
\DescribeMacro{\childdocof}
Furthermore, add the commands
\begin{center}
\begin{tabular}{l}
|\input{childdoc.def}|\\
|\childdocof{|\textit{main}|}|\\
\end{tabular}
\end{center}
at the top of every child file \textit{child}
which is included by |\include{|\textit{child}|}|
from within the main file
(or at least for those files to be compiled individually).
The argument \textit{main} must be the filename of the main file.

There are a couple of
considerations in setting up the main and child documents:

%%%%%%%%%%%%%%%%%%%%%%%%%%%%%%%%%%%%%%%%
\paragraph{Restrictions.}

Please note the following restrictions:
\begin{itemize}
\item
|\childdocmain| must be called with one argument \textit{main}
to ensure compatibility with earlier version of the package.
It must either be empty (|\childdocmain{}|)
or precisely match the filename of the main file in which it is specified.
See \secref{sec:detection} for further information.
\item
The filename \textit{main} must be specified without the |.tex| extension.
\item
The filename \textit{main} is case sensitive
(even in case-insensitive file systems)
due to internal string comparison.
\item
The argument \textit{main} should be fully expanded, it cannot be a macro.
\item
Subdirectories and special characters should be avoided in filenames.
\item
The command |\childdocmain{|\textit{main}|}| must be followed by a whitespace.
It should not be followed immediately by another command
or by a comment mark `|%|'.
This is because the \TeX{} parser reads the token immediately following
the argument of |\childdocmain| and puts it
at the beginning of every child section;
however, a white\-space is ignored.
\end{itemize}

%%%%%%%%%%%%%%%%%%%%%%%%%%%%%%%%%%%%%%%%
\paragraph{Content of Main File.}

It is advisable to place all content in the child files included by |\include|.
Any output contained in the main file will appear in all child documents
unless suppressed manually;
it cannot be suppressed automatically by the |\includeonly| directive
and thus should normally be avoided.
A method to include some content in the main file
by means of conditional processing is described in \secref{sec:conditional}.

%%%%%%%%%%%%%%%%%%%%%%%%%%%%%%%%%%%%%%%%
\paragraph{Page Numbering.}

When only a part of the document is compiled,
the appropriate numbering of pages
(as well as other status parameters)
is determined from the |.aux| files.
The latter contain information from previous passes.
However this information needs to propagate through
all intermediate child documents.
Therefore the page numbering in child documents may well
be inconsistent until the complete document is compiled at least once.

A useful (if unconventional) way to always ensure a consistent
page numbering is to restart the numbering in each child document
and denote the pages by `\textit{child}|.|\textit{page}'
where \textit{child} represents the chapter/section number of the child file.
This can be achieved by the command
|\numberwithin{page}{|\textit{child}|}|
of the \textsf{amsmath} package
where \textit{child} can be |chapter| or |section|
depending on the chosen structuring.
Alternatively, one can modify the macro |\thepage| appropriately
and reset the counter |page| at the start of each child file.

%%%%%%%%%%%%%%%%%%%%%%%%%%%%%%%%%%%%%%%%%%%%%%%%%%%%%%%%%%%%%%%%%%%%%%%%%%%%%%%%
\subsection{Conditional Processing}
\label{sec:conditional}

The package provides a mechanism to compile different versions
of a document. To customise the versions further some conditional processing
can come in handy to distinguish which version is being compiled.
The package provides two macros to describe the compilation context:

%%%%%%%%%%%%%%%%%%%%%%%%%%%%%%%%%%%%%%%%
\DescribeMacro{\ifchilddoc}
The conditional |\ifchilddoc| distinguishes between the compilation of
child documents and the main document:
%
\begin{center}
|\ifchilddoc |\textit{child-code}| |[|\||else |\textit{main-code}]| \||fi|
\end{center}

%%%%%%%%%%%%%%%%%%%%%%%%%%%%%%%%%%%%%%%%
\DescribeMacro{\childdocname}
\DescribeMacro{\childdocjob}
The macro |\childdocname| contains the filename (without extension)
of the main or child file being processed.
Note that |\childdocjob| will always contain the name of the main file.

%%%%%%%%%%%%%%%%%%%%%%%%%%%%%%%%%%%%%%%%
\paragraph{Title Page.}

Conditional processing can be used to include a title or banner page
in the main document when proper precautions are taken.
Importantly, the code in the main file should ensure that the page counter
(as well as other status parameters which are stored in the |.aux| files)
takes the same value after the conditional processing.
Otherwise the page numbers may take divergent values
depending on which part is compiled.

For example, a title page could be declared by:
%
\begin{center}
\begin{tabular}{l}
|\ifchilddoc\||else|\\
|\addtocounter{page}{-1}|\\
\textit{code for title page}\\
|\newpage|\\
|\||fi|
\end{tabular}
\end{center}
%
A banner page for the child documents can be generated by:
%
\begin{center}
\begin{tabular}{l}
|\ifchilddoc|\\
|\addtocounter{page}{-1}|\\
\textit{code for banner page}\\
|\newpage|\\
|\||fi|
\end{tabular}
\end{center}
%
Here one could write a message such as:
\begin{center}
|This is the part \childdocname{} of \childdocjob{}.|
\end{center}

%%%%%%%%%%%%%%%%%%%%%%%%%%%%%%%%%%%%%%%%%%%%%%%%%%%%%%%%%%%%%%%%%%%%%%%%%%%%%%%%
\subsection{Flags}
\label{sec:flags}

The package makes it easy to generate different versions
of the main or child documents.
To this end compilation flags can be defined
and assigned different default values.
They will be particularly useful in conjunction
with the forwarding mechanism described in \secref{sec:forward}.

For example, it may be useful to have a flag |\version|
which can be set to |draft| or |final|.
The document source will contain some conditional code
depending on the value of |\version|.
Suppose further, the flag should default to |final| for the main file
and to |draft| for child files
which is a natural assignment for editing the document.
This is achieved by placing the following code
in the preamble of the main document
(below the |\childdocmain| directive):
%
\begin{center}
\begin{tabular}{l}
|\ifchilddoc|\\
|\providecommand{\version}{draft}|\\
|\||else|\\
|\providecommand{\version}{final}|\\
|\||fi|
\end{tabular}
\end{center}
%
The definition by |\providecommand| makes sure
that previous definitions are not overwritten.
Further statements |\providecommand{\version}{...}|
can thus be added before the above code to override it.

For the main file, one might add a line
(between |\childdocmain| and the above block)
%
\begin{center}
|%\ifchilddoc\||else\providecommand{\version}{draft}\||fi|
\end{center}
%
which can be uncommented to produce a draft version.
Likewise one can add a line to the very top of a child file
(above the |\childdocof{|\textit{main}|}| directive)
%
\begin{center}
|%\providecommand{\version}{final}|
\end{center}
%
which can be uncommented to produce the final version of this child document.

%%%%%%%%%%%%%%%%%%%%%%%%%%%%%%%%%%%%%%%%%%%%%%%%%%%%%%%%%%%%%%%%%%%%%%%%%%%%%%%%
\subsection{Forwarding}
\label{sec:forward}

Different versions of the main or child documents
using compilation flags as described in \secref{sec:flags}
can be (permanently) stored in different files
for convenient compilation, viewing and distribution.
To this end, the package defines a command
to pass on compilation to a different file:

%%%%%%%%%%%%%%%%%%%%%%%%%%%%%%%%%%%%%%%%
\DescribeMacro{\childdocforward}
The command |\childdocforward| redirects processing to
another source file:
%
\begin{center}
\begin{tabular}{l}
|\input{childdoc.def}|\\
|\childdocforward[|\textit{main}|]{|\textit{dest}|}|\\
\end{tabular}
\end{center}
%
The argument \textit{dest} is the destination file
(without extension).
It should be the main file or one of the child files.
Note that further \textsf{childdoc} directives
such as |\childdocof| and |\childdocforward|
in the indicated file will be processed in this form.
The optional argument \textit{main}
passes on directly to the main file \textit{main}
while pretending to compile the child \textit{dest}.
This form behaves as if \textit{dest}
issues |\childdocof{|\textit{main}|}| right away,
and no further \textsf{childdoc} directives will be processed.

%%%%%%%%%%%%%%%%%%%%%%%%%%%%%%%%%%%%%%%%
\DescribeMacro{\...prefix}
In the alternative form |\childdocforwardprefix|,
%
\begin{center}
\begin{tabular}{l}
|\input{childdoc.def}|\\
|\childdocforwardprefix[|\textit{main}|]{|\textit{prefix}|}{|\textit{dest}|}|
\end{tabular}
\end{center}
%
the destination file is determined by a pattern
depending on the current file:
To make this work, the current file must be called
`{\textit{prefix}\hspace{0.2em}\textit{suffix}}'
with \textit{prefix} matching precisely the argument.
Processing is then passed on to the file
`{\textit{dest}\hspace{0.2em}\textit{suffix}}'.
Surely, the same effect is achieved by
directly specifying the
argument `{\textit{dest}\hspace{0.2em}\textit{suffix}}'
in the first form.
However, that requires to set up a different file
for each child. With the alternative form of the command
all these files can have exactly the same content
which simplifies setting them up and maintaining them.

For example, the following file |draft.tex|
with a compilation flag |\version| as described in \secref{sec:flags}
compiles the main document as a draft:
%
\begin{center}
\begin{tabular}{l}
|\def\version{draft}|\\
|\input{childdoc.def}|\\
|\childdocforward{|\textit{main}|}|
\end{tabular}
\end{center}
%
Likewise, the following files |final|\textit{nn}|.tex|
compile the final version of the child document
|child|\textit{nn}|.tex|:
%
\begin{center}
\begin{tabular}{l}
|\def\version{final}|\\
|\input{childdoc.def}|\\
|\childdocforwardprefix{final}{child}|
\end{tabular}
\end{center}
%

Note that when several versions of a main file and/or of each child file
are to be generated, it may be convenient to set up a |Makefile| or
shell script to automatise the process.

%%%%%%%%%%%%%%%%%%%%%%%%%%%%%%%%%%%%%%%%%%%%%%%%%%%%%%%%%%%%%%%%%%%%%%%%%%%%%%%%
\subsection{Command Line Processing}
\label{sec:commandline}

The effect of redirection files can also be achieved by invoking
the \LaTeX{} compiler with a more elaborate command line.
Most conveniently this should be done as part
of a shell script or a |Makefile|.

When using \textsf{childdoc} in the main file, the following
command lines effectively perform a redirection
(note that depending on the shell being used,
backslashes may have to be doubled: `|\|' $\to$ `|\\|'):
%
\begin{center}
|... -jobname "|\textit{target}|" |\\|"|[\textit{flags}]%
|\input{childdoc.def}\childdocforward[|\textit{main}|]{|\textit{dest}|}"|
\end{center}
%
Here \textit{target} is the name of the output file,
\textit{main} is the name of the main file
and \textit{dest} is the name of the main or child file to be processed
(all filenames without extensions).
The optional argument \textit{main} can be omitted
if \textit{main} matches \textit{dest}.
Optionally, compilation \textit{flags} can be defined via |\def| commands.
This command line makes the \TeX{} engine believe
it is compiling the file \textit{target}
whose content is specified as the latter parameter.
The provided code then forwards the processing to
\textit{main} or \textit{dest} as described in \secref{sec:forward}.

%%%%%%%%%%%%%%%%%%%%%%%%%%%%%%%%%%%%%%%%%%%%%%%%%%%%%%%%%%%%%%%%%%%%%%%%%%%%%%%%
\subsection{Include by Input}
\label{sec:input}

Including child documents by |\include| has some restrictions by design.
Most notably, the content of a child document always occupies
its own set of pages; pages cannot be shared between child documents.
Usually, this behaviour makes perfect sense
because each child document contain an essential part of the document.
However, in some situations it may be desirable to compose
a document from a collection of parts
without having mandatory page breaks between then.
For this case, the package
provides a mechanism to include parts
by |\input| which can also be processed individually.
However, by construction this mechanism
requires manual handling of the content to be output.

%%%%%%%%%%%%%%%%%%%%%%%%%%%%%%%%%%%%%%%%
\DescribeMacro{\ifchilddocmanual}
The main file should be prepared as usual, see \secref{sec:include}.
However, the document body must make a distinction
between processing of an individual part and of the main document, e.g.:
%
\begin{center}
\begin{tabular}{l}
|\ifchilddocmanual|\\
|\input{\childdocname}|\\
|\||else|\\
\textit{document body with }|\input{|\textit{part}|}|\\
|\||fi|
\end{tabular}
\end{center}
%
The conditional |\ifchilddocmanual| is true whenever
a part to be included by |\input| is being compiled,
and the name of the part is stored in |\childdocname|.

%%%%%%%%%%%%%%%%%%%%%%%%%%%%%%%%%%%%%%%%
\DescribeMacro{\childdocby}
Each part to be included by |\input| should start with:
%
\begin{center}
\begin{tabular}{l}
|\input{childdoc.def}|\\
|\childdocby{|\textit{main}|}|\\
\end{tabular}
\end{center}
%
The directive |\childdocby| is similar to |\childdocof|
described in \secref{sec:include},
but the subsequent selection of content must be done manually.
To that end, both |\ifchilddoc| and |\ifchilddocmanual|
will be true upon processing of a part,
and the name of the part is stored in |\childdocname|.
Note that |\jobname| will be set to the filename of the current part
so that each part receives an individual |.aux| file
that does not interfere with the |.aux| file(s) of the main document.
This behaviour can be altered by the alternative form
|\childdocby[*]{|\textit{main}|}| (with a non-empty optional argument)
which uses the |.aux| file of the main document
by setting |\jobname| to \textit{main}.

%%%%%%%%%%%%%%%%%%%%%%%%%%%%%%%%%%%%%%%%%%%%%%%%%%%%%%%%%%%%%%%%%%%%%%%%%%%%%%%%
\subsection{Driver Development}
\label{sec:driver}

The \textsf{childdoc} mechanism can also be use for the development
of definition files such as \LaTeX{} styles or classes.
This case differs from the above setup with multiple parts
included by |\include| in that no |\includeonly| should be invoked.
This can be achieved by starting the include file
(before |\ProvidesPackage|) with:
%
\begin{center}
\begin{tabular}{l}
|\input{childdoc.def}|\\
|\childdocforward{|\textit{main}|}|\\
\end{tabular}
\end{center}
%
or alternatively with:
%
\begin{center}
\begin{tabular}{l}
|\input{childdoc.def}|\\
|\childdocby{|\textit{main}|}|\\
\end{tabular}
\end{center}
%
Both forms have slightly different effects as described above.
The main file is prepared as usual, see \secref{sec:include}.

%%%%%%%%%%%%%%%%%%%%%%%%%%%%%%%%%%%%%%%%%%%%%%%%%%%%%%%%%%%%%%%%%%%%%%%%%%%%%%%%
\subsection{Legacy Detection}
\label{sec:detection}

The directive |\childdocmain| in the main file can detect
whether the complete document or merely a child is to be compiled
even without using the directive |\childdocof|.
This method is deprecated because it is less robust
and there is no compelling reason to use it;
it is merely provided for backward compatibility
and it may be removed in future versions.

If the detection mechanism is to be used,
it is mandatory to correctly specify
the filename of the main file as the argument of |\childdocmain|:
%
\begin{center}
\begin{tabular}{l}
|\input{childdoc.def}|\\
|\childdocmain{|\textit{main}|}|\\
\end{tabular}
\end{center}
%
If |\jobname| does not match the argument \textit{main} of |\childdocmain|,
it is assumed that |\jobname| points to the child file to be compiled.
When using |\childdocmain| with the main file specified as argument,
it suffices to start a child file
with just |\input{|\textit{main}|}|
without loading of the package and using |\childdocof|.
If instead all processing is done
with the appropriate \textsf{childdoc} directives,
the argument of \textit{main} of |\childdocmain| can be empty.

An alternative version of the command line processing described
in \secref{sec:commandline} using the detection mechanism reads:
%
\begin{center}
|... -jobname "|\textit{target}|" "|[\textit{flags}]%
[|\def\jobname{|\textit{dest}|}|]|\input{|\textit{main}|}"|
\end{center}

%%%%%%%%%%%%%%%%%%%%%%%%%%%%%%%%%%%%%%%%%%%%%%%%%%%%%%%%%%%%%%%%%%%%%%%%%%%%%%%%
\subsection{Manual Code}
\label{sec:manual}

In case one cannot be certain whether the definitions file |childdoc.def|
is installed on the target \TeX{} distribution
and one prefers not to ship it,
it is conceivable to paste a few relevant commands into the sources.

To that end, drop all statements |\input{childdoc.def}|
and perform the replacements as outlined below.
Instead of |\childdocmain{|\textit{main}|}| add the following code
to the top of the main file:
%
\begin{center}
\begin{tabular}{l}
|\||ifdefined\childdocname\endinput\||fi\newif\ifchilddoc|\\
|\edef\childdocname{\scantokens\expandafter{\jobname\noexpand}}|\\
|\def\childdocmain{|\textit{main}|}\||ifx\childdocmain\childdocname\||else|\\
|\childdoctrue\includeonly{\childdocname}\let\jobname\childdocmain\||fi|\\
\end{tabular}
\end{center}
%
Instead of |\childdocof{|\textit{main}|}| just include the main file
at the top of each child file:
%
\begin{center}
|\input{|\textit{main}|}|
\end{center}
%
A simple redirection |\childdocforward{|\textit{dest}|}| is achieved by:
%
\begin{center}
|\def\jobname{|\textit{dest}|}\input{\jobname}|
\end{center}
%
The redirection with prefix
|\childdocforwardprefix[|\textit{prefix}|]{|\textit{dest}|}|
is accomplished by:
%
\begin{center}
\begin{tabular}{l}
|{\edef\jobname{\scantokens\expandafter{\jobname\noexpand}}|\\
|\def\redirectjob |\textit{prefix}|#1~~~{\gdef\jobname{|\textit{dest}|#1}}|\\
|\expandafter\redirectjob\jobname~~~}\input{\jobname}|
\end{tabular}
\end{center}

In an alternative approach,
child documents can be compiled by a specific command line
without additional code or specific definitions:
%
\begin{center}
|... -jobname "|\textit{target}|" "|[\textit{flags}]%
|\includeonly{|\textit{dest}|}\input{|\textit{main}|}"|
\end{center}
%

%%%%%%%%%%%%%%%%%%%%%%%%%%%%%%%%%%%%%%%%%%%%%%%%%%%%%%%%%%%%%%%%%%%%%%%%%%%%%%%%
%%%%%%%%%%%%%%%%%%%%%%%%%%%%%%%%%%%%%%%%%%%%%%%%%%%%%%%%%%%%%%%%%%%%%%%%%%%%%%%%
\section{Information}

%%%%%%%%%%%%%%%%%%%%%%%%%%%%%%%%%%%%%%%%%%%%%%%%%%%%%%%%%%%%%%%%%%%%%%%%%%%%%%%%
\subsection{Copyright}

Copyright \copyright{} 2017--2018 Niklas Beisert

This work may be distributed and/or modified under the
conditions of the \LaTeX{} Project Public License, either version 1.3
of this license or (at your option) any later version.
The latest version of this license is in
  \url{http://www.latex-project.org/lppl.txt}
and version 1.3 or later is part of all distributions of \LaTeX{}
version 2005/12/01 or later.

This work has the LPPL maintenance status `maintained'.

The Current Maintainer of this work is Niklas Beisert.

This work consists of the files |README.txt|, |childdoc.ins| and |childdoc.dtx|
as well as the derived files |childdoc.def|, |cdocsamp.tex|
with |cdocsch1.tex|, |cdocsch2.tex|, |cdocspt3.tex|, |cdocspt4.tex|,
|cdocsdrf.tex|, |cdocsfn1.tex|, |cdocsfn2.tex|
as well as |childdoc.pdf|.

%%%%%%%%%%%%%%%%%%%%%%%%%%%%%%%%%%%%%%%%%%%%%%%%%%%%%%%%%%%%%%%%%%%%%%%%%%%%%%%%
\subsection{Files and Installation}

The package consists of the files:
%
\begin{center}
\begin{tabular}{ll}
    |README.txt|   & readme file \\
    |childdoc.ins| & installation file \\
    |childdoc.dtx| & source file \\
    |childdoc.def| & definition file \\
    |cdocsamp.tex| & sample main file \\
    |cdocsch1.tex| & sample include file \\
    |cdocsch2.tex| & sample include file \\
    |cdocspt3.tex| & sample part file \\
    |cdocspt4.tex| & sample part file \\
    |cdocsdrf.tex| & sample redirection file \\
    |cdocsfn1.tex| & sample redirection file \\
    |cdocsfn2.tex| & sample redirection file \\
    |childdoc.pdf| & manual
\end{tabular}
\end{center}
%
The distribution consists of the files
|README.txt|, |childdoc.ins| and |childdoc.dtx|.
%
\begin{itemize}
\item
Run (pdf)\LaTeX{} on |childdoc.dtx|
to compile the manual |childdoc.pdf| (this file).
\item
Run \LaTeX{} on |childdoc.ins| to create the definitions file |childdoc.def|
and the sample |cdocsamp.tex| with include files
|cdocsch1.tex|, |cdocsch2.tex|, |cdocspt3.tex|, |cdocspt4.tex|,
|cdocsdrf.tex|, |cdocsfn1.tex|, |cdocsfn2.tex|.
Then copy the file |childdoc.def| to an appropriate directory of your \LaTeX{}
distribution, e.g.\ \textit{texmf-root}|/tex/latex/childdoc|.
\end{itemize}

%%%%%%%%%%%%%%%%%%%%%%%%%%%%%%%%%%%%%%%%%%%%%%%%%%%%%%%%%%%%%%%%%%%%%%%%%%%%%%%%
\subsection{Related CTAN Packages}

There are several other packages which offer a similar functionality:
%
\begin{itemize}
\item
The packages
\href{http://ctan.org/pkg/docmute}{\textsf{docmute}},
\href{http://ctan.org/pkg/includex}{\textsf{includex}} and
\href{http://ctan.org/pkg/standalone}{\textsf{standalone}}
provide commands to include only the document body of
a child file thus allowing both files to be compiled individually.
\item
The packages \href{http://ctan.org/pkg/subdocs}{\textsf{subdocs}}
and \href{http://ctan.org/pkg/subfiles}{\textsf{subfiles}}
provide structures in which the main and child documents can be
encapsulated and allowing them to be compiled individually.
The inclusion mechanism is different from the conventional |\include|.
\item
The package \href{http://ctan.org/pkg/combine}{\textsf{combine}}
is an elaborate solution to combine several documents into one.
\end{itemize}
%
See also the CTAN topic \href{http://ctan.org/topic/subdocs}{\textsf{subdocs}}
for further related packages.
The present package differs from the above solutions in that
a document structure constructed with the conventional |\include| mechanism
just needs two extra commands at the top of every file
such that all constituent files can be compiled individually.

%%%%%%%%%%%%%%%%%%%%%%%%%%%%%%%%%%%%%%%%%%%%%%%%%%%%%%%%%%%%%%%%%%%%%%%%%%%%%%%%
%\subsection{Feature Suggestions}
%
%The following is a list of features which may be useful for future
%versions of this package:
%%
%\begin{itemize}
%\item
%\ldots
%\end{itemize}

%%%%%%%%%%%%%%%%%%%%%%%%%%%%%%%%%%%%%%%%%%%%%%%%%%%%%%%%%%%%%%%%%%%%%%%%%%%%%%%%
\subsection{Revision History}

%%%%%%%%%%%%%%%%%%%%%%%%%%%%%%%%%%%%%%%%
\paragraph{v2.0:} 2018/12/30

\begin{itemize}
\item
immediate forward processing
\item
added |\childdocby| mechanism
\item
manual restructured
\end{itemize}

%%%%%%%%%%%%%%%%%%%%%%%%%%%%%%%%%%%%%%%%
\paragraph{v1.6:} 2018/01/17

\begin{itemize}
\item
application for development of include files
\item
corrections to manual
\end{itemize}

%%%%%%%%%%%%%%%%%%%%%%%%%%%%%%%%%%%%%%%%
\paragraph{v1.5:} 2017/05/21

\begin{itemize}
\item
more complete structuring introduced
\item
|\childdocof| introduced
\item
|\childdoc| renamed to |\childdocmain|
\item
|\childredirect| renamed to |\childdocforward| and |\childdocforwardprefix|
and functionality expanded
\end{itemize}

%%%%%%%%%%%%%%%%%%%%%%%%%%%%%%%%%%%%%%%%
\paragraph{v1.0:} 2017/04/27

\begin{itemize}
\item
manual and install package
\item
first version published on CTAN
\end{itemize}

%%%%%%%%%%%%%%%%%%%%%%%%%%%%%%%%%%%%%%%%
\paragraph{v0.6:} 2017/04/26

\begin{itemize}
\item
redirection mechanism added
\end{itemize}

%%%%%%%%%%%%%%%%%%%%%%%%%%%%%%%%%%%%%%%%
\paragraph{v0.5:} 2017/04/26

\begin{itemize}
\item
functionality in definition file
\end{itemize}


%%%%%%%%%%%%%%%%%%%%%%%%%%%%%%%%%%%%%%%%%%%%%%%%%%%%%%%%%%%%%%%%%%%%%%%%%%%%%%%%
%%%%%%%%%%%%%%%%%%%%%%%%%%%%%%%%%%%%%%%%%%%%%%%%%%%%%%%%%%%%%%%%%%%%%%%%%%%%%%%%
%%%%%%%%%%%%%%%%%%%%%%%%%%%%%%%%%%%%%%%%%%%%%%%%%%%%%%%%%%%%%%%%%%%%%%%%%%%%%%%%
\appendix

\settowidth\MacroIndent{\rmfamily\scriptsize 000\ }

 \DocInput{childdoc.dtx}

\end{document}
%</driver>
% \fi
%
% %%%%%%%%%%%%%%%%%%%%%%%%%%%%%%%%%%%%%%%%%%%%%%%%%%%%%%%%%%%%%%%%%%%%%%%%%%%%%%
% %%%%%%%%%%%%%%%%%%%%%%%%%%%%%%%%%%%%%%%%%%%%%%%%%%%%%%%%%%%%%%%%%%%%%%%%%%%%%%
% \section{Sample}
%\iffalse
%<*samplemain>
%\fi
%
% The following presents a sample document
% with two chapters, two parts, a title page,
% a compile flag as well as three forwarding files to set the flag.
% It consists of eight |.tex| files:
% \begin{center}
% \begin{tabular}{ll}
% |cdocsamp.tex|&main file\\
% |cdocsch1.tex|&include file for chapter 1\\
% |cdocsch2.tex|&include file for chapter 2\\
% |cdocspt3.tex|&include file for part 3\\
% |cdocspt4.tex|&include file for part 4\\
% |cdocsdrf.tex|&forwarding file for main file in draft mode\\
% |cdocsfi1.tex|&forwarding file for final version of chapter 1\\
% |cdocsfi2.tex|&forwarding file for final version of chapter 2\\
% \end{tabular}
% \end{center}
% Each of the eight files can be compiled directly by the \LaTeX{} compiler.
%
% %%%%%%%%%%%%%%%%%%%%%%%%%%%%%%%%%%%%%%
% \paragraph{Main File.}
%
% The main file is called |cdocsamp.tex|.
%
% Load the \textsf{childdoc} definitions and
% declare the filename for the main document:
%    \begin{macrocode}
\input{childdoc.def}
\childdocmain{}
%    \end{macrocode}

% Optional override for |\version| flag:
%    \begin{macrocode}
%%\ifchilddoc\else\providecommand{\version}{draft}\fi
%    \end{macrocode}

% Define the default values for the |\version| flag
% (|final| for the main file and |draft| for childs):
%    \begin{macrocode}
\ifchilddoc
\providecommand{\version}{draft}
\else
\providecommand{\version}{final}
\fi
%    \end{macrocode}

% Load the standard document class:
%    \begin{macrocode}
\documentclass[12pt]{article}
%    \end{macrocode}

% Start the document body:
%    \begin{macrocode}
\begin{document}
%    \end{macrocode}

% Declare a title page.
% Print title, part of document being processed and version flag:
%    \begin{macrocode}
\addtocounter{page}{-1}
\begin{center}
{\LARGE\bfseries{}childdoc example\par}
\vspace{1cm}
\ifchilddoc
\ifchilddocmanual part\else chapter\fi:
`\childdocname' of `\childdocjob'\par
\else
main document: `\childdocjob'\par
\fi
version: \version\par
\end{center}
\newpage
%    \end{macrocode}

% Manually include selected file,
% otherwise process as usual:
%    \begin{macrocode}
\ifchilddocmanual
\section*{part `\childdocname'}
\input{\childdocname}
\else
%    \end{macrocode}

% Include the two chapters:
%    \begin{macrocode}
\include{cdocsch1}
\include{cdocsch2}
%    \end{macrocode}

% Include the two parts unless only chapters should be displayed:
%    \begin{macrocode}
\ifchilddoc\else
\section{part three}
\input{cdocspt3}
\section{part four}
\input{cdocspt4}
\fi
%    \end{macrocode}

% Process as usual until here:
%    \begin{macrocode}
\fi
%    \end{macrocode}

% End of document body:
%    \begin{macrocode}
\end{document}
%    \end{macrocode}
%\iffalse
%</samplemain>
%\fi
%
% %%%%%%%%%%%%%%%%%%%%%%%%%%%%%%%%%%%%%%
% \paragraph{Chapter Include Files.}
%
% The include files are called |cdocsch1.tex| and |cdocsch2.tex|.
%
%\iffalse
%<*samplechap1|samplechap2>
%\fi

% Optional override for |\version| flag:
%    \begin{macrocode}
%%\providecommand{\version}{final}
%    \end{macrocode}

% Include the main document:
%    \begin{macrocode}
\input{childdoc.def}
\childdocof{cdocsamp}
%    \end{macrocode}

%\iffalse
%</samplechap1|samplechap2>
%\fi
%
%\iffalse
%<*samplechap1>
%\fi
% Some text for chapter 1:
%    \begin{macrocode}
\section{one}
some text in chapter one
%    \end{macrocode}

%\iffalse
%</samplechap1>
%\fi
% Some text for chapter 2:
%\iffalse
%<*samplechap2>
%\fi
%    \begin{macrocode}
\section{two}
more text in chapter two
%    \end{macrocode}

%\iffalse
%</samplechap2>
%\fi
%
% %%%%%%%%%%%%%%%%%%%%%%%%%%%%%%%%%%%%%%
% \paragraph{Part Include Files.}
%
% The include files are called |cdocspt3.tex| and |cdocspt4.tex|.
%
%\iffalse
%<*samplepart3|samplepart4>
%\fi

% Optional override for |\version| flag:
%    \begin{macrocode}
%%\providecommand{\version}{final}
%    \end{macrocode}

% Include the main document:
%    \begin{macrocode}
\input{childdoc.def}
\childdocby{cdocsamp}
%    \end{macrocode}

%\iffalse
%</samplepart3|samplepart4>
%\fi
%
%\iffalse
%<*samplepart3>
%\fi
% Some text for part 3:
%    \begin{macrocode}
some text in part three
%    \end{macrocode}

%\iffalse
%</samplepart3>
%\fi
% Some text for part 4:
%\iffalse
%<*samplepart4>
%\fi
%    \begin{macrocode}
more text in part four
%    \end{macrocode}

%\iffalse
%</samplepart4>
%\fi
%
% %%%%%%%%%%%%%%%%%%%%%%%%%%%%%%%%%%%%%%
% \paragraph{Forwarding for a Complete Draft.}
%
% The following forwarding file |cdocsdrf.tex|
% compiles the main document in draft mode:
%\iffalse
%<*sampledraft>
%\fi
%    \begin{macrocode}
\def\version{draft}
\input{childdoc.def}
\childdocforward{cdocsamp}
%    \end{macrocode}

%\iffalse
%</sampledraft>
%\fi
%
% %%%%%%%%%%%%%%%%%%%%%%%%%%%%%%%%%%%%%%
% \paragraph{Forwarding for Final Version of the Chapters.}
%
% The following forwarding files |cdocsfn1.tex| and |cdocsfn2.tex|
% (with identical content)
% compile the final versions of the child documents
% |cdocsch1.tex| and |cdocsch2.tex|, respectively:
%\iffalse
%<*samplefinal>
%\fi
%    \begin{macrocode}
\def\version{final}
\input{childdoc.def}
\childdocforwardprefix[cdocsamp]{cdocsfn}{cdocsch}
%    \end{macrocode}

%\iffalse
%</samplefinal>
%\fi
%
% %%%%%%%%%%%%%%%%%%%%%%%%%%%%%%%%%%%%%%
% \paragraph{Command Line Processing.}
%
% The following three command lines generate the output files
% |cdocscld|, |cdocscl1| and |cdocscl2|
% which should be identical to
% |cdocsdrf|, |cdocsch1| and |cdocsfn2|, respectively:
% \begin{center}
% \begin{tabular}{l}
% |latex -jobname cdocscld \|\\
% |  "\def\version{draft}\input{childdoc.def}\childdocforward{cdocsamp}"|\\
% |latex -jobname cdocscl1 \|\\
% |  "\input{childdoc.def}\childdocforward[cdocsamp]{cdocsch1}"|\\
% |latex -jobname cdocscl2 \|\\
% |  "\def\version{final}\input{childdoc.def}\childdocforward{cdocsch2}"|
% \end{tabular}
% \end{center}
% Note that the trailing backslash on each first line
% merely continues the input to the second line
% (for convenient cut ant paste).
% Furthermore, the command |latex| can be replaced by any
% of its alternative versions such as |pdflatex|.
%
% %%%%%%%%%%%%%%%%%%%%%%%%%%%%%%%%%%%%%%%%%%%%%%%%%%%%%%%%%%%%%%%%%%%%%%%%%%%%%%
% %%%%%%%%%%%%%%%%%%%%%%%%%%%%%%%%%%%%%%%%%%%%%%%%%%%%%%%%%%%%%%%%%%%%%%%%%%%%%%
% \section{Implementation}
%\iffalse
%<*package>
%\fi
%
% This section describes the definitions file |childdoc.def|.

% The definitions cannot be loaded using |\usepackage| or |\RequirePackage|
% which has a mechanism to prevent loading a style file more than once.
% When loading the definitions by means of |\input|
% multiple instances have to be prevented manually:
%\iffalse
%This code needs to be before the `\ProvidesFile' directive
%which is defined at the beginning of this file.
%Therefore it is also placed there and commented out here.
%</package>
%<*discard>
%\fi
%    \begin{macrocode}
\ifdefined\childdocmain\endinput\fi
%    \end{macrocode}
%\iffalse
%</discard>
%<*package>
%\fi
%
% \macro{\ifchilddoc}
% \macro{\ifchilddocmanual}
% The conditional |\ifchilddoc| tells whether a
% child (true) or main (false) document is being compiled.
% The conditional |\ifchilddocmanual| tells whether
% the |\includeonly| mechanism is used (false) or
% the selection of child files must be performed manually (true).
% The definitions initialise to false:
%    \begin{macrocode}
\newif\ifchilddoc
\newif\ifchilddocmanual
%    \end{macrocode}

% \macro{\childdocname}
% \macro{\childdocjob}
% The macro |\childdocname| stores the name of the main document
% to be compiled. The macro |\childdocjob| stores the name of
% the document on which the \LaTeX{} compiler was originally invoked.
% The content of |\jobname| cannot be compared
% to filenames specified in the source due to different catcodes.
% The following code rescans |\jobname|, stores the result
% in |\childdocname| and saves a copy in |\childdocjob|:
%    \begin{macrocode}
\edef\childdocname{\scantokens\expandafter{\jobname\noexpand}}
\let\childdocjob\childdocname
%    \end{macrocode}

% \macro{\childdocdisable}
% The macro |\childdocdisable| prevents the main file
% from being processed more than once.
% At this stage, the main document command |\childdocmain|
% is assumed to be called once again where it should do nothing.
% Any subsequent call to it should prevent
% a secondary processing of the main document
% It overwrites the forwarding commands
% |\childdocof| and |\childdocforward|
% with empty macros to prevent further inclusions of the main document:
%    \begin{macrocode}
\newcommand{\childdocdisable}
{
  \renewcommand{\childdocmain}[1]{\renewcommand{\childdocmain}[1]{\endinput}}
  \renewcommand{\childdocof}[1]{}
  \renewcommand{\childdocby}[2][]{}
  \renewcommand{\childdocforward}[2][]{}
  \renewcommand{\childdocdisable}{}
}
%    \end{macrocode}

% \macro{\childdocmain}
% The macro |\childdocmain| is to be called at the top of the main file
% with nothing or the main filename (without extension) as argument.
% First, it breaks loops.
% If the argument is not empty and does not match |\childdocname|
% (which is set by the first inclusion of |childdoc.def|),
% |\ifchilddoc| is set to true, |\includeonly| is applied to the child file
% and |\jobname| is set to the main file
% (for proper handling of |.aux| files):
%    \begin{macrocode}
\newcommand{\childdocmain}[1]
{
  \childdocdisable\childdocmain{}
  \if?#1?\else
    \begingroup
      \def\childdoctmp{#1}
      \ifx\childdoctmp\childdocname
        \def\childdoctmp{}
      \else
        \def\childdoctmp
        {
          \childdoctrue
          \includeonly{\childdocname}
          \def\childdocjob{#1}
          \def\jobname{#1}
        }
      \fi
      \expandafter
    \endgroup
    \childdoctmp
  \fi
}
%    \end{macrocode}

% \macro{\childdocof}
% The command |\childdocof| redirects
% compilation to the main file |#1|.
%    \begin{macrocode}
\newcommand{\childdocof}[1]
{
  \childdocdisable
  \childdoctrue
  \includeonly{\childdocname}
  \def\jobname{#1}
  \def\childdocjob{#1}
  \input{#1}
}
%    \end{macrocode}

% \macro{\childdocby}
% The command |\childdocby| ....
%    \begin{macrocode}
\newcommand{\childdocby}[2][]
{
  \childdocdisable
  \childdoctrue
  \childdocmanualtrue
  \if?#1?\else
    \def\jobname{#2}
  \fi
  \def\childdocjob{#2}
  \input{#2}
  \endinput
}
%    \end{macrocode}

% \macro{\childdocforward}
% The command |\childdocforward| redirects
% compilation to the main file or
% (if the optional argument is given) a child file.
% Parameters are set as if the main file
% or a child file starting with |\childdocof| was compiled.
% Then compilation is handed over to the main file:
%    \begin{macrocode}
\newcommand{\childdocforward}[2][]
{
  \begingroup
    \if?#1?
      \def\childdoctmp
      {
        \def\childdocname{#2}
        \def\childdocjob{#2}
        \def\jobname{#2}
        \input{#2}
        \endinput
      }
    \else
      \def\childdoctmp
      {
        \childdocdisable
        \def\childdocname{#2}
        \childdoctrue
        \includeonly{#2}
        \def\childdocjob{#1}
        \def\jobname{#1}
        \input{#1}
        \endinput
      }
    \fi
    \expandafter
  \endgroup
  \childdoctmp
}
%    \end{macrocode}

% \macro{\childdocforwardprefix}
% The command |\childdocforwardprefix| redirects
% compilation to the main or a child file by means of a pattern.
% The prefix |#1| in the current filename is replaced by |#2|
% and the suffix of the current filename is kept
% (it is assumed that the filename does not contain the substring `|~~~|'
% which is used as a delimiter).
% Compilation is handed over to the new file by |\childdocforward|:
%    \begin{macrocode}
\newcommand{\childdocforwardprefix}[3][]
{
  \begingroup
    \def\childdocextract #2##1~~~{\def\childdoctmp{\childdocforward[#1]{#3##1}}}
    \expandafter\childdocextract\childdocname~~~
    \expandafter
  \endgroup
  \childdoctmp
}
%    \end{macrocode}

% \macro{\childdoc}
% The deprecated macro |\childdoc| is a legacy version of |\childdocmain|:
%    \begin{macrocode}
\newcommand{\childdoc}{\childdocmain}
%    \end{macrocode}

% \macro{\childdocredirect}
% The deprecated macro |\childdocredirect| is a legacy version
% of |\childdocforward| and |\childdocforwardprefix|:
%    \begin{macrocode}
\newcommand{\childdocredirect}[2][]
{
  \begingroup
    \if?#1?
      \def\childdoctmp{\childdocforward{#2}}
    \else
      \def\childdoctmp{\childdocforwardprefix{#1}{#2}}
    \fi
    \expandafter
  \endgroup
  \childdoctmp
}
%    \end{macrocode}

%\iffalse
%</package>
%\fi
%
\endinput
\childdocforward[|\textit{main}|]{|\textit{dest}|}"|
\end{center}
%
Here \textit{target} is the name of the output file,
\textit{main} is the name of the main file
and \textit{dest} is the name of the main or child file to be processed
(all filenames without extensions).
The optional argument \textit{main} can be omitted
if \textit{main} matches \textit{dest}.
Optionally, compilation \textit{flags} can be defined via |\def| commands.
This command line makes the \TeX{} engine believe
it is compiling the file \textit{target}
whose content is specified as the latter parameter.
The provided code then forwards the processing to
\textit{main} or \textit{dest} as described in \secref{sec:forward}.

%%%%%%%%%%%%%%%%%%%%%%%%%%%%%%%%%%%%%%%%%%%%%%%%%%%%%%%%%%%%%%%%%%%%%%%%%%%%%%%%
\subsection{Include by Input}
\label{sec:input}

Including child documents by |\include| has some restrictions by design.
Most notably, the content of a child document always occupies
its own set of pages; pages cannot be shared between child documents.
Usually, this behaviour makes perfect sense
because each child document contain an essential part of the document.
However, in some situations it may be desirable to compose
a document from a collection of parts
without having mandatory page breaks between then.
For this case, the package
provides a mechanism to include parts
by |\input| which can also be processed individually.
However, by construction this mechanism
requires manual handling of the content to be output.

%%%%%%%%%%%%%%%%%%%%%%%%%%%%%%%%%%%%%%%%
\DescribeMacro{\ifchilddocmanual}
The main file should be prepared as usual, see \secref{sec:include}.
However, the document body must make a distinction
between processing of an individual part and of the main document, e.g.:
%
\begin{center}
\begin{tabular}{l}
|\ifchilddocmanual|\\
|\input{\childdocname}|\\
|\||else|\\
\textit{document body with }|\input{|\textit{part}|}|\\
|\||fi|
\end{tabular}
\end{center}
%
The conditional |\ifchilddocmanual| is true whenever
a part to be included by |\input| is being compiled,
and the name of the part is stored in |\childdocname|.

%%%%%%%%%%%%%%%%%%%%%%%%%%%%%%%%%%%%%%%%
\DescribeMacro{\childdocby}
Each part to be included by |\input| should start with:
%
\begin{center}
\begin{tabular}{l}
|% \iffalse
%
% childdoc.dtx Copyright (C) 2017-2018 Niklas Beisert
%
% This work may be distributed and/or modified under the
% conditions of the LaTeX Project Public License, either version 1.3
% of this license or (at your option) any later version.
% The latest version of this license is in
%   http://www.latex-project.org/lppl.txt
% and version 1.3 or later is part of all distributions of LaTeX
% version 2005/12/01 or later.
%
% This work has the LPPL maintenance status `maintained'.
%
% The Current Maintainer of this work is Niklas Beisert.
%
% This work consists of the files childdoc.dtx and childdoc.ins
% and the derived files childdoc.def and cdocsamp.tex with
% cdocsch1.tex, cdocsch2.tex, cdocsdrf.tex, cdocsfn1.tex, cdocsfn2.tex.
%
%<package>\ifdefined\childdocmain\endinput\fi
%<package>\ProvidesFile{childdoc.def}[2018/12/30 v2.0 child document driver]
%<samplemain>\ProvidesFile{cdocsamp.tex}[2018/12/30 v2.0 sample for childdoc]
%<*driver>
%\ProvidesFile{childdoc.drv}[2018/12/30 v2.0 childdoc reference manual file]
\PassOptionsToClass{10pt,a4paper}{article}
\documentclass{ltxdoc}

\usepackage[margin=35mm]{geometry}
\usepackage{hyperref}
\usepackage{hyperxmp}
\usepackage[usenames]{color}

\hypersetup{colorlinks=true}
\hypersetup{pdfstartview=FitH}
\hypersetup{pdfpagemode=UseNone}
\hypersetup{pdfsource={}}
\hypersetup{pdflang={en-UK}}
\hypersetup{pdfcopyright={Copyright 2017-2018 Niklas Beisert.
  This work may be distributed and/or modified under the
  conditions of the LaTeX Project Public License, either version 1.3
  of this license or (at your option) any later version.}}
\hypersetup{pdflicenseurl={http://www.latex-project.org/lppl.txt}}
\hypersetup{pdfcontactaddress={ETH Zurich, ITP, HIT K,
  Wolfgang-Pauli-Strasse 27}}
\hypersetup{pdfcontactpostcode={8093}}
\hypersetup{pdfcontactcity={Zurich}}
\hypersetup{pdfcontactcountry={Switzerland}}
\hypersetup{pdfcontactemail={nbeisert@itp.phys.ethz.ch}}
\hypersetup{pdfcontacturl={http://people.phys.ethz.ch/\xmptilde nbeisert/}}

\newcommand{\secref}[1]{\hyperref[#1]{section \ref*{#1}}}

\parskip1ex
\parindent0pt
\let\olditemize\itemize
\def\itemize{\olditemize\parskip0pt}

\begin{document}

\title{The \textsf{childdoc} Package}
\hypersetup{pdftitle={The childdoc Package}}
\author{Niklas Beisert\\[2ex]
  Institut f\"ur Theoretische Physik\\
  Eidgen\"ossische Technische Hochschule Z\"urich\\
  Wolfgang-Pauli-Strasse 27, 8093 Z\"urich, Switzerland\\[1ex]
  \href{mailto:nbeisert@itp.phys.ethz.ch}
  {\texttt{nbeisert@itp.phys.ethz.ch}}}
\hypersetup{pdfauthor={Niklas Beisert}}
\hypersetup{pdfsubject={Manual for the LaTeX2e Package childdoc}}
\date{30 December 2018, \textsf{v2.0}}
\maketitle

\begin{abstract}\noindent
\textsf{childdoc} is a \LaTeXe{} package
that enables the direct compilation
of document sections included by |\include|
to individual files.
\end{abstract}

\begingroup
\parskip0ex
\tableofcontents
\endgroup

%%%%%%%%%%%%%%%%%%%%%%%%%%%%%%%%%%%%%%%%%%%%%%%%%%%%%%%%%%%%%%%%%%%%%%%%%%%%%%%%
%%%%%%%%%%%%%%%%%%%%%%%%%%%%%%%%%%%%%%%%%%%%%%%%%%%%%%%%%%%%%%%%%%%%%%%%%%%%%%%%
\section{Introduction}

\LaTeX{} provides a mechanism to structure a large document (such as a book)
into a main file and several child files (containing the chapters)
using the |\include| command.
This mechanism is beneficial for documents
which span hundreds of pages in order to
make the source file(s) more manageable.
Moreover, compilation can be restricted to
selected child files by means of the |\includeonly| command.
The latter feature can be used to reduce the compilation time while editing
(this was significantly more useful in the earlier days of \LaTeX{})
or to generate a smaller document which is easier to navigate.
Another application of |\includeonly| is to generate
documents consisting of selected parts of the complete document.

However, there are a few drawbacks of the plain |\include| mechanism:
\begin{itemize}
\item
The child files cannot be compiled on their own,
they can only be compiled via the main file.
A naive editing environment
(such as a text editor with an option
to have the current file processed by \LaTeX)
may require one to switch to the main file before compiling;
attempting to compile the child file produces errors.
\item
The main file must be modified (each time)
to adjust the |\includeonly| command
to the present needs. This easily leaves the main file in a messy state.
\item
The generated document will always carry the filename
of the main document. This is inconvenient if
several child files are to be compiled and
to be kept for distribution.
\end{itemize}

The present package provides a simple interface
to make child files individually compilable by \LaTeX{}.
Compiling a child file then has the same effect as compiling
the main file with an |\includeonly| command
to select the appropriate child.
Moreover the generated document will carry the name of the child
rather than the main file.
This resolves all three above issues.

This feature is meant to make the editing of books,
thesis documents and lecture notes somewhat more convenient.
However, the package can also be used efficiently for
composing a series of documents (such as exercise sheets)
which are typically distributed individually.
It then assists the author in generating the individual documents
(potentially in different versions)
as well as a document containing the collected series.
Another application is in developing style files
or other kinds of included material
where compilation of the style file could redirect
to a sample or test file.

%%%%%%%%%%%%%%%%%%%%%%%%%%%%%%%%%%%%%%%%%%%%%%%%%%%%%%%%%%%%%%%%%%%%%%%%%%%%%%%%
%%%%%%%%%%%%%%%%%%%%%%%%%%%%%%%%%%%%%%%%%%%%%%%%%%%%%%%%%%%%%%%%%%%%%%%%%%%%%%%%
\section{Usage}

First of all, the package \textsf{childdoc} is \emph{not} a standard
\LaTeXe{} |.sty| style file! Therefore it needs to be invoked in
a non-standard way.

%%%%%%%%%%%%%%%%%%%%%%%%%%%%%%%%%%%%%%%%%%%%%%%%%%%%%%%%%%%%%%%%%%%%%%%%%%%%%%%%
\subsection{Included Files}
\label{sec:include}

%%%%%%%%%%%%%%%%%%%%%%%%%%%%%%%%%%%%%%%%
\DescribeMacro{\childdocmain}
To use the package, add the commands
\begin{center}
\begin{tabular}{l}
|\input{childdoc.def}|\\
|\childdocmain{}|\\
\end{tabular}
\end{center}
at the very top of the main \LaTeX{} file,
in particular \emph{before} the |\documentclass| statement!
The argument of |\childdocmain| should be left empty
(but it must be present).

%%%%%%%%%%%%%%%%%%%%%%%%%%%%%%%%%%%%%%%%
\DescribeMacro{\childdocof}
Furthermore, add the commands
\begin{center}
\begin{tabular}{l}
|\input{childdoc.def}|\\
|\childdocof{|\textit{main}|}|\\
\end{tabular}
\end{center}
at the top of every child file \textit{child}
which is included by |\include{|\textit{child}|}|
from within the main file
(or at least for those files to be compiled individually).
The argument \textit{main} must be the filename of the main file.

There are a couple of
considerations in setting up the main and child documents:

%%%%%%%%%%%%%%%%%%%%%%%%%%%%%%%%%%%%%%%%
\paragraph{Restrictions.}

Please note the following restrictions:
\begin{itemize}
\item
|\childdocmain| must be called with one argument \textit{main}
to ensure compatibility with earlier version of the package.
It must either be empty (|\childdocmain{}|)
or precisely match the filename of the main file in which it is specified.
See \secref{sec:detection} for further information.
\item
The filename \textit{main} must be specified without the |.tex| extension.
\item
The filename \textit{main} is case sensitive
(even in case-insensitive file systems)
due to internal string comparison.
\item
The argument \textit{main} should be fully expanded, it cannot be a macro.
\item
Subdirectories and special characters should be avoided in filenames.
\item
The command |\childdocmain{|\textit{main}|}| must be followed by a whitespace.
It should not be followed immediately by another command
or by a comment mark `|%|'.
This is because the \TeX{} parser reads the token immediately following
the argument of |\childdocmain| and puts it
at the beginning of every child section;
however, a white\-space is ignored.
\end{itemize}

%%%%%%%%%%%%%%%%%%%%%%%%%%%%%%%%%%%%%%%%
\paragraph{Content of Main File.}

It is advisable to place all content in the child files included by |\include|.
Any output contained in the main file will appear in all child documents
unless suppressed manually;
it cannot be suppressed automatically by the |\includeonly| directive
and thus should normally be avoided.
A method to include some content in the main file
by means of conditional processing is described in \secref{sec:conditional}.

%%%%%%%%%%%%%%%%%%%%%%%%%%%%%%%%%%%%%%%%
\paragraph{Page Numbering.}

When only a part of the document is compiled,
the appropriate numbering of pages
(as well as other status parameters)
is determined from the |.aux| files.
The latter contain information from previous passes.
However this information needs to propagate through
all intermediate child documents.
Therefore the page numbering in child documents may well
be inconsistent until the complete document is compiled at least once.

A useful (if unconventional) way to always ensure a consistent
page numbering is to restart the numbering in each child document
and denote the pages by `\textit{child}|.|\textit{page}'
where \textit{child} represents the chapter/section number of the child file.
This can be achieved by the command
|\numberwithin{page}{|\textit{child}|}|
of the \textsf{amsmath} package
where \textit{child} can be |chapter| or |section|
depending on the chosen structuring.
Alternatively, one can modify the macro |\thepage| appropriately
and reset the counter |page| at the start of each child file.

%%%%%%%%%%%%%%%%%%%%%%%%%%%%%%%%%%%%%%%%%%%%%%%%%%%%%%%%%%%%%%%%%%%%%%%%%%%%%%%%
\subsection{Conditional Processing}
\label{sec:conditional}

The package provides a mechanism to compile different versions
of a document. To customise the versions further some conditional processing
can come in handy to distinguish which version is being compiled.
The package provides two macros to describe the compilation context:

%%%%%%%%%%%%%%%%%%%%%%%%%%%%%%%%%%%%%%%%
\DescribeMacro{\ifchilddoc}
The conditional |\ifchilddoc| distinguishes between the compilation of
child documents and the main document:
%
\begin{center}
|\ifchilddoc |\textit{child-code}| |[|\||else |\textit{main-code}]| \||fi|
\end{center}

%%%%%%%%%%%%%%%%%%%%%%%%%%%%%%%%%%%%%%%%
\DescribeMacro{\childdocname}
\DescribeMacro{\childdocjob}
The macro |\childdocname| contains the filename (without extension)
of the main or child file being processed.
Note that |\childdocjob| will always contain the name of the main file.

%%%%%%%%%%%%%%%%%%%%%%%%%%%%%%%%%%%%%%%%
\paragraph{Title Page.}

Conditional processing can be used to include a title or banner page
in the main document when proper precautions are taken.
Importantly, the code in the main file should ensure that the page counter
(as well as other status parameters which are stored in the |.aux| files)
takes the same value after the conditional processing.
Otherwise the page numbers may take divergent values
depending on which part is compiled.

For example, a title page could be declared by:
%
\begin{center}
\begin{tabular}{l}
|\ifchilddoc\||else|\\
|\addtocounter{page}{-1}|\\
\textit{code for title page}\\
|\newpage|\\
|\||fi|
\end{tabular}
\end{center}
%
A banner page for the child documents can be generated by:
%
\begin{center}
\begin{tabular}{l}
|\ifchilddoc|\\
|\addtocounter{page}{-1}|\\
\textit{code for banner page}\\
|\newpage|\\
|\||fi|
\end{tabular}
\end{center}
%
Here one could write a message such as:
\begin{center}
|This is the part \childdocname{} of \childdocjob{}.|
\end{center}

%%%%%%%%%%%%%%%%%%%%%%%%%%%%%%%%%%%%%%%%%%%%%%%%%%%%%%%%%%%%%%%%%%%%%%%%%%%%%%%%
\subsection{Flags}
\label{sec:flags}

The package makes it easy to generate different versions
of the main or child documents.
To this end compilation flags can be defined
and assigned different default values.
They will be particularly useful in conjunction
with the forwarding mechanism described in \secref{sec:forward}.

For example, it may be useful to have a flag |\version|
which can be set to |draft| or |final|.
The document source will contain some conditional code
depending on the value of |\version|.
Suppose further, the flag should default to |final| for the main file
and to |draft| for child files
which is a natural assignment for editing the document.
This is achieved by placing the following code
in the preamble of the main document
(below the |\childdocmain| directive):
%
\begin{center}
\begin{tabular}{l}
|\ifchilddoc|\\
|\providecommand{\version}{draft}|\\
|\||else|\\
|\providecommand{\version}{final}|\\
|\||fi|
\end{tabular}
\end{center}
%
The definition by |\providecommand| makes sure
that previous definitions are not overwritten.
Further statements |\providecommand{\version}{...}|
can thus be added before the above code to override it.

For the main file, one might add a line
(between |\childdocmain| and the above block)
%
\begin{center}
|%\ifchilddoc\||else\providecommand{\version}{draft}\||fi|
\end{center}
%
which can be uncommented to produce a draft version.
Likewise one can add a line to the very top of a child file
(above the |\childdocof{|\textit{main}|}| directive)
%
\begin{center}
|%\providecommand{\version}{final}|
\end{center}
%
which can be uncommented to produce the final version of this child document.

%%%%%%%%%%%%%%%%%%%%%%%%%%%%%%%%%%%%%%%%%%%%%%%%%%%%%%%%%%%%%%%%%%%%%%%%%%%%%%%%
\subsection{Forwarding}
\label{sec:forward}

Different versions of the main or child documents
using compilation flags as described in \secref{sec:flags}
can be (permanently) stored in different files
for convenient compilation, viewing and distribution.
To this end, the package defines a command
to pass on compilation to a different file:

%%%%%%%%%%%%%%%%%%%%%%%%%%%%%%%%%%%%%%%%
\DescribeMacro{\childdocforward}
The command |\childdocforward| redirects processing to
another source file:
%
\begin{center}
\begin{tabular}{l}
|\input{childdoc.def}|\\
|\childdocforward[|\textit{main}|]{|\textit{dest}|}|\\
\end{tabular}
\end{center}
%
The argument \textit{dest} is the destination file
(without extension).
It should be the main file or one of the child files.
Note that further \textsf{childdoc} directives
such as |\childdocof| and |\childdocforward|
in the indicated file will be processed in this form.
The optional argument \textit{main}
passes on directly to the main file \textit{main}
while pretending to compile the child \textit{dest}.
This form behaves as if \textit{dest}
issues |\childdocof{|\textit{main}|}| right away,
and no further \textsf{childdoc} directives will be processed.

%%%%%%%%%%%%%%%%%%%%%%%%%%%%%%%%%%%%%%%%
\DescribeMacro{\...prefix}
In the alternative form |\childdocforwardprefix|,
%
\begin{center}
\begin{tabular}{l}
|\input{childdoc.def}|\\
|\childdocforwardprefix[|\textit{main}|]{|\textit{prefix}|}{|\textit{dest}|}|
\end{tabular}
\end{center}
%
the destination file is determined by a pattern
depending on the current file:
To make this work, the current file must be called
`{\textit{prefix}\hspace{0.2em}\textit{suffix}}'
with \textit{prefix} matching precisely the argument.
Processing is then passed on to the file
`{\textit{dest}\hspace{0.2em}\textit{suffix}}'.
Surely, the same effect is achieved by
directly specifying the
argument `{\textit{dest}\hspace{0.2em}\textit{suffix}}'
in the first form.
However, that requires to set up a different file
for each child. With the alternative form of the command
all these files can have exactly the same content
which simplifies setting them up and maintaining them.

For example, the following file |draft.tex|
with a compilation flag |\version| as described in \secref{sec:flags}
compiles the main document as a draft:
%
\begin{center}
\begin{tabular}{l}
|\def\version{draft}|\\
|\input{childdoc.def}|\\
|\childdocforward{|\textit{main}|}|
\end{tabular}
\end{center}
%
Likewise, the following files |final|\textit{nn}|.tex|
compile the final version of the child document
|child|\textit{nn}|.tex|:
%
\begin{center}
\begin{tabular}{l}
|\def\version{final}|\\
|\input{childdoc.def}|\\
|\childdocforwardprefix{final}{child}|
\end{tabular}
\end{center}
%

Note that when several versions of a main file and/or of each child file
are to be generated, it may be convenient to set up a |Makefile| or
shell script to automatise the process.

%%%%%%%%%%%%%%%%%%%%%%%%%%%%%%%%%%%%%%%%%%%%%%%%%%%%%%%%%%%%%%%%%%%%%%%%%%%%%%%%
\subsection{Command Line Processing}
\label{sec:commandline}

The effect of redirection files can also be achieved by invoking
the \LaTeX{} compiler with a more elaborate command line.
Most conveniently this should be done as part
of a shell script or a |Makefile|.

When using \textsf{childdoc} in the main file, the following
command lines effectively perform a redirection
(note that depending on the shell being used,
backslashes may have to be doubled: `|\|' $\to$ `|\\|'):
%
\begin{center}
|... -jobname "|\textit{target}|" |\\|"|[\textit{flags}]%
|\input{childdoc.def}\childdocforward[|\textit{main}|]{|\textit{dest}|}"|
\end{center}
%
Here \textit{target} is the name of the output file,
\textit{main} is the name of the main file
and \textit{dest} is the name of the main or child file to be processed
(all filenames without extensions).
The optional argument \textit{main} can be omitted
if \textit{main} matches \textit{dest}.
Optionally, compilation \textit{flags} can be defined via |\def| commands.
This command line makes the \TeX{} engine believe
it is compiling the file \textit{target}
whose content is specified as the latter parameter.
The provided code then forwards the processing to
\textit{main} or \textit{dest} as described in \secref{sec:forward}.

%%%%%%%%%%%%%%%%%%%%%%%%%%%%%%%%%%%%%%%%%%%%%%%%%%%%%%%%%%%%%%%%%%%%%%%%%%%%%%%%
\subsection{Include by Input}
\label{sec:input}

Including child documents by |\include| has some restrictions by design.
Most notably, the content of a child document always occupies
its own set of pages; pages cannot be shared between child documents.
Usually, this behaviour makes perfect sense
because each child document contain an essential part of the document.
However, in some situations it may be desirable to compose
a document from a collection of parts
without having mandatory page breaks between then.
For this case, the package
provides a mechanism to include parts
by |\input| which can also be processed individually.
However, by construction this mechanism
requires manual handling of the content to be output.

%%%%%%%%%%%%%%%%%%%%%%%%%%%%%%%%%%%%%%%%
\DescribeMacro{\ifchilddocmanual}
The main file should be prepared as usual, see \secref{sec:include}.
However, the document body must make a distinction
between processing of an individual part and of the main document, e.g.:
%
\begin{center}
\begin{tabular}{l}
|\ifchilddocmanual|\\
|\input{\childdocname}|\\
|\||else|\\
\textit{document body with }|\input{|\textit{part}|}|\\
|\||fi|
\end{tabular}
\end{center}
%
The conditional |\ifchilddocmanual| is true whenever
a part to be included by |\input| is being compiled,
and the name of the part is stored in |\childdocname|.

%%%%%%%%%%%%%%%%%%%%%%%%%%%%%%%%%%%%%%%%
\DescribeMacro{\childdocby}
Each part to be included by |\input| should start with:
%
\begin{center}
\begin{tabular}{l}
|\input{childdoc.def}|\\
|\childdocby{|\textit{main}|}|\\
\end{tabular}
\end{center}
%
The directive |\childdocby| is similar to |\childdocof|
described in \secref{sec:include},
but the subsequent selection of content must be done manually.
To that end, both |\ifchilddoc| and |\ifchilddocmanual|
will be true upon processing of a part,
and the name of the part is stored in |\childdocname|.
Note that |\jobname| will be set to the filename of the current part
so that each part receives an individual |.aux| file
that does not interfere with the |.aux| file(s) of the main document.
This behaviour can be altered by the alternative form
|\childdocby[*]{|\textit{main}|}| (with a non-empty optional argument)
which uses the |.aux| file of the main document
by setting |\jobname| to \textit{main}.

%%%%%%%%%%%%%%%%%%%%%%%%%%%%%%%%%%%%%%%%%%%%%%%%%%%%%%%%%%%%%%%%%%%%%%%%%%%%%%%%
\subsection{Driver Development}
\label{sec:driver}

The \textsf{childdoc} mechanism can also be use for the development
of definition files such as \LaTeX{} styles or classes.
This case differs from the above setup with multiple parts
included by |\include| in that no |\includeonly| should be invoked.
This can be achieved by starting the include file
(before |\ProvidesPackage|) with:
%
\begin{center}
\begin{tabular}{l}
|\input{childdoc.def}|\\
|\childdocforward{|\textit{main}|}|\\
\end{tabular}
\end{center}
%
or alternatively with:
%
\begin{center}
\begin{tabular}{l}
|\input{childdoc.def}|\\
|\childdocby{|\textit{main}|}|\\
\end{tabular}
\end{center}
%
Both forms have slightly different effects as described above.
The main file is prepared as usual, see \secref{sec:include}.

%%%%%%%%%%%%%%%%%%%%%%%%%%%%%%%%%%%%%%%%%%%%%%%%%%%%%%%%%%%%%%%%%%%%%%%%%%%%%%%%
\subsection{Legacy Detection}
\label{sec:detection}

The directive |\childdocmain| in the main file can detect
whether the complete document or merely a child is to be compiled
even without using the directive |\childdocof|.
This method is deprecated because it is less robust
and there is no compelling reason to use it;
it is merely provided for backward compatibility
and it may be removed in future versions.

If the detection mechanism is to be used,
it is mandatory to correctly specify
the filename of the main file as the argument of |\childdocmain|:
%
\begin{center}
\begin{tabular}{l}
|\input{childdoc.def}|\\
|\childdocmain{|\textit{main}|}|\\
\end{tabular}
\end{center}
%
If |\jobname| does not match the argument \textit{main} of |\childdocmain|,
it is assumed that |\jobname| points to the child file to be compiled.
When using |\childdocmain| with the main file specified as argument,
it suffices to start a child file
with just |\input{|\textit{main}|}|
without loading of the package and using |\childdocof|.
If instead all processing is done
with the appropriate \textsf{childdoc} directives,
the argument of \textit{main} of |\childdocmain| can be empty.

An alternative version of the command line processing described
in \secref{sec:commandline} using the detection mechanism reads:
%
\begin{center}
|... -jobname "|\textit{target}|" "|[\textit{flags}]%
[|\def\jobname{|\textit{dest}|}|]|\input{|\textit{main}|}"|
\end{center}

%%%%%%%%%%%%%%%%%%%%%%%%%%%%%%%%%%%%%%%%%%%%%%%%%%%%%%%%%%%%%%%%%%%%%%%%%%%%%%%%
\subsection{Manual Code}
\label{sec:manual}

In case one cannot be certain whether the definitions file |childdoc.def|
is installed on the target \TeX{} distribution
and one prefers not to ship it,
it is conceivable to paste a few relevant commands into the sources.

To that end, drop all statements |\input{childdoc.def}|
and perform the replacements as outlined below.
Instead of |\childdocmain{|\textit{main}|}| add the following code
to the top of the main file:
%
\begin{center}
\begin{tabular}{l}
|\||ifdefined\childdocname\endinput\||fi\newif\ifchilddoc|\\
|\edef\childdocname{\scantokens\expandafter{\jobname\noexpand}}|\\
|\def\childdocmain{|\textit{main}|}\||ifx\childdocmain\childdocname\||else|\\
|\childdoctrue\includeonly{\childdocname}\let\jobname\childdocmain\||fi|\\
\end{tabular}
\end{center}
%
Instead of |\childdocof{|\textit{main}|}| just include the main file
at the top of each child file:
%
\begin{center}
|\input{|\textit{main}|}|
\end{center}
%
A simple redirection |\childdocforward{|\textit{dest}|}| is achieved by:
%
\begin{center}
|\def\jobname{|\textit{dest}|}\input{\jobname}|
\end{center}
%
The redirection with prefix
|\childdocforwardprefix[|\textit{prefix}|]{|\textit{dest}|}|
is accomplished by:
%
\begin{center}
\begin{tabular}{l}
|{\edef\jobname{\scantokens\expandafter{\jobname\noexpand}}|\\
|\def\redirectjob |\textit{prefix}|#1~~~{\gdef\jobname{|\textit{dest}|#1}}|\\
|\expandafter\redirectjob\jobname~~~}\input{\jobname}|
\end{tabular}
\end{center}

In an alternative approach,
child documents can be compiled by a specific command line
without additional code or specific definitions:
%
\begin{center}
|... -jobname "|\textit{target}|" "|[\textit{flags}]%
|\includeonly{|\textit{dest}|}\input{|\textit{main}|}"|
\end{center}
%

%%%%%%%%%%%%%%%%%%%%%%%%%%%%%%%%%%%%%%%%%%%%%%%%%%%%%%%%%%%%%%%%%%%%%%%%%%%%%%%%
%%%%%%%%%%%%%%%%%%%%%%%%%%%%%%%%%%%%%%%%%%%%%%%%%%%%%%%%%%%%%%%%%%%%%%%%%%%%%%%%
\section{Information}

%%%%%%%%%%%%%%%%%%%%%%%%%%%%%%%%%%%%%%%%%%%%%%%%%%%%%%%%%%%%%%%%%%%%%%%%%%%%%%%%
\subsection{Copyright}

Copyright \copyright{} 2017--2018 Niklas Beisert

This work may be distributed and/or modified under the
conditions of the \LaTeX{} Project Public License, either version 1.3
of this license or (at your option) any later version.
The latest version of this license is in
  \url{http://www.latex-project.org/lppl.txt}
and version 1.3 or later is part of all distributions of \LaTeX{}
version 2005/12/01 or later.

This work has the LPPL maintenance status `maintained'.

The Current Maintainer of this work is Niklas Beisert.

This work consists of the files |README.txt|, |childdoc.ins| and |childdoc.dtx|
as well as the derived files |childdoc.def|, |cdocsamp.tex|
with |cdocsch1.tex|, |cdocsch2.tex|, |cdocspt3.tex|, |cdocspt4.tex|,
|cdocsdrf.tex|, |cdocsfn1.tex|, |cdocsfn2.tex|
as well as |childdoc.pdf|.

%%%%%%%%%%%%%%%%%%%%%%%%%%%%%%%%%%%%%%%%%%%%%%%%%%%%%%%%%%%%%%%%%%%%%%%%%%%%%%%%
\subsection{Files and Installation}

The package consists of the files:
%
\begin{center}
\begin{tabular}{ll}
    |README.txt|   & readme file \\
    |childdoc.ins| & installation file \\
    |childdoc.dtx| & source file \\
    |childdoc.def| & definition file \\
    |cdocsamp.tex| & sample main file \\
    |cdocsch1.tex| & sample include file \\
    |cdocsch2.tex| & sample include file \\
    |cdocspt3.tex| & sample part file \\
    |cdocspt4.tex| & sample part file \\
    |cdocsdrf.tex| & sample redirection file \\
    |cdocsfn1.tex| & sample redirection file \\
    |cdocsfn2.tex| & sample redirection file \\
    |childdoc.pdf| & manual
\end{tabular}
\end{center}
%
The distribution consists of the files
|README.txt|, |childdoc.ins| and |childdoc.dtx|.
%
\begin{itemize}
\item
Run (pdf)\LaTeX{} on |childdoc.dtx|
to compile the manual |childdoc.pdf| (this file).
\item
Run \LaTeX{} on |childdoc.ins| to create the definitions file |childdoc.def|
and the sample |cdocsamp.tex| with include files
|cdocsch1.tex|, |cdocsch2.tex|, |cdocspt3.tex|, |cdocspt4.tex|,
|cdocsdrf.tex|, |cdocsfn1.tex|, |cdocsfn2.tex|.
Then copy the file |childdoc.def| to an appropriate directory of your \LaTeX{}
distribution, e.g.\ \textit{texmf-root}|/tex/latex/childdoc|.
\end{itemize}

%%%%%%%%%%%%%%%%%%%%%%%%%%%%%%%%%%%%%%%%%%%%%%%%%%%%%%%%%%%%%%%%%%%%%%%%%%%%%%%%
\subsection{Related CTAN Packages}

There are several other packages which offer a similar functionality:
%
\begin{itemize}
\item
The packages
\href{http://ctan.org/pkg/docmute}{\textsf{docmute}},
\href{http://ctan.org/pkg/includex}{\textsf{includex}} and
\href{http://ctan.org/pkg/standalone}{\textsf{standalone}}
provide commands to include only the document body of
a child file thus allowing both files to be compiled individually.
\item
The packages \href{http://ctan.org/pkg/subdocs}{\textsf{subdocs}}
and \href{http://ctan.org/pkg/subfiles}{\textsf{subfiles}}
provide structures in which the main and child documents can be
encapsulated and allowing them to be compiled individually.
The inclusion mechanism is different from the conventional |\include|.
\item
The package \href{http://ctan.org/pkg/combine}{\textsf{combine}}
is an elaborate solution to combine several documents into one.
\end{itemize}
%
See also the CTAN topic \href{http://ctan.org/topic/subdocs}{\textsf{subdocs}}
for further related packages.
The present package differs from the above solutions in that
a document structure constructed with the conventional |\include| mechanism
just needs two extra commands at the top of every file
such that all constituent files can be compiled individually.

%%%%%%%%%%%%%%%%%%%%%%%%%%%%%%%%%%%%%%%%%%%%%%%%%%%%%%%%%%%%%%%%%%%%%%%%%%%%%%%%
%\subsection{Feature Suggestions}
%
%The following is a list of features which may be useful for future
%versions of this package:
%%
%\begin{itemize}
%\item
%\ldots
%\end{itemize}

%%%%%%%%%%%%%%%%%%%%%%%%%%%%%%%%%%%%%%%%%%%%%%%%%%%%%%%%%%%%%%%%%%%%%%%%%%%%%%%%
\subsection{Revision History}

%%%%%%%%%%%%%%%%%%%%%%%%%%%%%%%%%%%%%%%%
\paragraph{v2.0:} 2018/12/30

\begin{itemize}
\item
immediate forward processing
\item
added |\childdocby| mechanism
\item
manual restructured
\end{itemize}

%%%%%%%%%%%%%%%%%%%%%%%%%%%%%%%%%%%%%%%%
\paragraph{v1.6:} 2018/01/17

\begin{itemize}
\item
application for development of include files
\item
corrections to manual
\end{itemize}

%%%%%%%%%%%%%%%%%%%%%%%%%%%%%%%%%%%%%%%%
\paragraph{v1.5:} 2017/05/21

\begin{itemize}
\item
more complete structuring introduced
\item
|\childdocof| introduced
\item
|\childdoc| renamed to |\childdocmain|
\item
|\childredirect| renamed to |\childdocforward| and |\childdocforwardprefix|
and functionality expanded
\end{itemize}

%%%%%%%%%%%%%%%%%%%%%%%%%%%%%%%%%%%%%%%%
\paragraph{v1.0:} 2017/04/27

\begin{itemize}
\item
manual and install package
\item
first version published on CTAN
\end{itemize}

%%%%%%%%%%%%%%%%%%%%%%%%%%%%%%%%%%%%%%%%
\paragraph{v0.6:} 2017/04/26

\begin{itemize}
\item
redirection mechanism added
\end{itemize}

%%%%%%%%%%%%%%%%%%%%%%%%%%%%%%%%%%%%%%%%
\paragraph{v0.5:} 2017/04/26

\begin{itemize}
\item
functionality in definition file
\end{itemize}


%%%%%%%%%%%%%%%%%%%%%%%%%%%%%%%%%%%%%%%%%%%%%%%%%%%%%%%%%%%%%%%%%%%%%%%%%%%%%%%%
%%%%%%%%%%%%%%%%%%%%%%%%%%%%%%%%%%%%%%%%%%%%%%%%%%%%%%%%%%%%%%%%%%%%%%%%%%%%%%%%
%%%%%%%%%%%%%%%%%%%%%%%%%%%%%%%%%%%%%%%%%%%%%%%%%%%%%%%%%%%%%%%%%%%%%%%%%%%%%%%%
\appendix

\settowidth\MacroIndent{\rmfamily\scriptsize 000\ }

 \DocInput{childdoc.dtx}

\end{document}
%</driver>
% \fi
%
% %%%%%%%%%%%%%%%%%%%%%%%%%%%%%%%%%%%%%%%%%%%%%%%%%%%%%%%%%%%%%%%%%%%%%%%%%%%%%%
% %%%%%%%%%%%%%%%%%%%%%%%%%%%%%%%%%%%%%%%%%%%%%%%%%%%%%%%%%%%%%%%%%%%%%%%%%%%%%%
% \section{Sample}
%\iffalse
%<*samplemain>
%\fi
%
% The following presents a sample document
% with two chapters, two parts, a title page,
% a compile flag as well as three forwarding files to set the flag.
% It consists of eight |.tex| files:
% \begin{center}
% \begin{tabular}{ll}
% |cdocsamp.tex|&main file\\
% |cdocsch1.tex|&include file for chapter 1\\
% |cdocsch2.tex|&include file for chapter 2\\
% |cdocspt3.tex|&include file for part 3\\
% |cdocspt4.tex|&include file for part 4\\
% |cdocsdrf.tex|&forwarding file for main file in draft mode\\
% |cdocsfi1.tex|&forwarding file for final version of chapter 1\\
% |cdocsfi2.tex|&forwarding file for final version of chapter 2\\
% \end{tabular}
% \end{center}
% Each of the eight files can be compiled directly by the \LaTeX{} compiler.
%
% %%%%%%%%%%%%%%%%%%%%%%%%%%%%%%%%%%%%%%
% \paragraph{Main File.}
%
% The main file is called |cdocsamp.tex|.
%
% Load the \textsf{childdoc} definitions and
% declare the filename for the main document:
%    \begin{macrocode}
\input{childdoc.def}
\childdocmain{}
%    \end{macrocode}

% Optional override for |\version| flag:
%    \begin{macrocode}
%%\ifchilddoc\else\providecommand{\version}{draft}\fi
%    \end{macrocode}

% Define the default values for the |\version| flag
% (|final| for the main file and |draft| for childs):
%    \begin{macrocode}
\ifchilddoc
\providecommand{\version}{draft}
\else
\providecommand{\version}{final}
\fi
%    \end{macrocode}

% Load the standard document class:
%    \begin{macrocode}
\documentclass[12pt]{article}
%    \end{macrocode}

% Start the document body:
%    \begin{macrocode}
\begin{document}
%    \end{macrocode}

% Declare a title page.
% Print title, part of document being processed and version flag:
%    \begin{macrocode}
\addtocounter{page}{-1}
\begin{center}
{\LARGE\bfseries{}childdoc example\par}
\vspace{1cm}
\ifchilddoc
\ifchilddocmanual part\else chapter\fi:
`\childdocname' of `\childdocjob'\par
\else
main document: `\childdocjob'\par
\fi
version: \version\par
\end{center}
\newpage
%    \end{macrocode}

% Manually include selected file,
% otherwise process as usual:
%    \begin{macrocode}
\ifchilddocmanual
\section*{part `\childdocname'}
\input{\childdocname}
\else
%    \end{macrocode}

% Include the two chapters:
%    \begin{macrocode}
\include{cdocsch1}
\include{cdocsch2}
%    \end{macrocode}

% Include the two parts unless only chapters should be displayed:
%    \begin{macrocode}
\ifchilddoc\else
\section{part three}
\input{cdocspt3}
\section{part four}
\input{cdocspt4}
\fi
%    \end{macrocode}

% Process as usual until here:
%    \begin{macrocode}
\fi
%    \end{macrocode}

% End of document body:
%    \begin{macrocode}
\end{document}
%    \end{macrocode}
%\iffalse
%</samplemain>
%\fi
%
% %%%%%%%%%%%%%%%%%%%%%%%%%%%%%%%%%%%%%%
% \paragraph{Chapter Include Files.}
%
% The include files are called |cdocsch1.tex| and |cdocsch2.tex|.
%
%\iffalse
%<*samplechap1|samplechap2>
%\fi

% Optional override for |\version| flag:
%    \begin{macrocode}
%%\providecommand{\version}{final}
%    \end{macrocode}

% Include the main document:
%    \begin{macrocode}
\input{childdoc.def}
\childdocof{cdocsamp}
%    \end{macrocode}

%\iffalse
%</samplechap1|samplechap2>
%\fi
%
%\iffalse
%<*samplechap1>
%\fi
% Some text for chapter 1:
%    \begin{macrocode}
\section{one}
some text in chapter one
%    \end{macrocode}

%\iffalse
%</samplechap1>
%\fi
% Some text for chapter 2:
%\iffalse
%<*samplechap2>
%\fi
%    \begin{macrocode}
\section{two}
more text in chapter two
%    \end{macrocode}

%\iffalse
%</samplechap2>
%\fi
%
% %%%%%%%%%%%%%%%%%%%%%%%%%%%%%%%%%%%%%%
% \paragraph{Part Include Files.}
%
% The include files are called |cdocspt3.tex| and |cdocspt4.tex|.
%
%\iffalse
%<*samplepart3|samplepart4>
%\fi

% Optional override for |\version| flag:
%    \begin{macrocode}
%%\providecommand{\version}{final}
%    \end{macrocode}

% Include the main document:
%    \begin{macrocode}
\input{childdoc.def}
\childdocby{cdocsamp}
%    \end{macrocode}

%\iffalse
%</samplepart3|samplepart4>
%\fi
%
%\iffalse
%<*samplepart3>
%\fi
% Some text for part 3:
%    \begin{macrocode}
some text in part three
%    \end{macrocode}

%\iffalse
%</samplepart3>
%\fi
% Some text for part 4:
%\iffalse
%<*samplepart4>
%\fi
%    \begin{macrocode}
more text in part four
%    \end{macrocode}

%\iffalse
%</samplepart4>
%\fi
%
% %%%%%%%%%%%%%%%%%%%%%%%%%%%%%%%%%%%%%%
% \paragraph{Forwarding for a Complete Draft.}
%
% The following forwarding file |cdocsdrf.tex|
% compiles the main document in draft mode:
%\iffalse
%<*sampledraft>
%\fi
%    \begin{macrocode}
\def\version{draft}
\input{childdoc.def}
\childdocforward{cdocsamp}
%    \end{macrocode}

%\iffalse
%</sampledraft>
%\fi
%
% %%%%%%%%%%%%%%%%%%%%%%%%%%%%%%%%%%%%%%
% \paragraph{Forwarding for Final Version of the Chapters.}
%
% The following forwarding files |cdocsfn1.tex| and |cdocsfn2.tex|
% (with identical content)
% compile the final versions of the child documents
% |cdocsch1.tex| and |cdocsch2.tex|, respectively:
%\iffalse
%<*samplefinal>
%\fi
%    \begin{macrocode}
\def\version{final}
\input{childdoc.def}
\childdocforwardprefix[cdocsamp]{cdocsfn}{cdocsch}
%    \end{macrocode}

%\iffalse
%</samplefinal>
%\fi
%
% %%%%%%%%%%%%%%%%%%%%%%%%%%%%%%%%%%%%%%
% \paragraph{Command Line Processing.}
%
% The following three command lines generate the output files
% |cdocscld|, |cdocscl1| and |cdocscl2|
% which should be identical to
% |cdocsdrf|, |cdocsch1| and |cdocsfn2|, respectively:
% \begin{center}
% \begin{tabular}{l}
% |latex -jobname cdocscld \|\\
% |  "\def\version{draft}\input{childdoc.def}\childdocforward{cdocsamp}"|\\
% |latex -jobname cdocscl1 \|\\
% |  "\input{childdoc.def}\childdocforward[cdocsamp]{cdocsch1}"|\\
% |latex -jobname cdocscl2 \|\\
% |  "\def\version{final}\input{childdoc.def}\childdocforward{cdocsch2}"|
% \end{tabular}
% \end{center}
% Note that the trailing backslash on each first line
% merely continues the input to the second line
% (for convenient cut ant paste).
% Furthermore, the command |latex| can be replaced by any
% of its alternative versions such as |pdflatex|.
%
% %%%%%%%%%%%%%%%%%%%%%%%%%%%%%%%%%%%%%%%%%%%%%%%%%%%%%%%%%%%%%%%%%%%%%%%%%%%%%%
% %%%%%%%%%%%%%%%%%%%%%%%%%%%%%%%%%%%%%%%%%%%%%%%%%%%%%%%%%%%%%%%%%%%%%%%%%%%%%%
% \section{Implementation}
%\iffalse
%<*package>
%\fi
%
% This section describes the definitions file |childdoc.def|.

% The definitions cannot be loaded using |\usepackage| or |\RequirePackage|
% which has a mechanism to prevent loading a style file more than once.
% When loading the definitions by means of |\input|
% multiple instances have to be prevented manually:
%\iffalse
%This code needs to be before the `\ProvidesFile' directive
%which is defined at the beginning of this file.
%Therefore it is also placed there and commented out here.
%</package>
%<*discard>
%\fi
%    \begin{macrocode}
\ifdefined\childdocmain\endinput\fi
%    \end{macrocode}
%\iffalse
%</discard>
%<*package>
%\fi
%
% \macro{\ifchilddoc}
% \macro{\ifchilddocmanual}
% The conditional |\ifchilddoc| tells whether a
% child (true) or main (false) document is being compiled.
% The conditional |\ifchilddocmanual| tells whether
% the |\includeonly| mechanism is used (false) or
% the selection of child files must be performed manually (true).
% The definitions initialise to false:
%    \begin{macrocode}
\newif\ifchilddoc
\newif\ifchilddocmanual
%    \end{macrocode}

% \macro{\childdocname}
% \macro{\childdocjob}
% The macro |\childdocname| stores the name of the main document
% to be compiled. The macro |\childdocjob| stores the name of
% the document on which the \LaTeX{} compiler was originally invoked.
% The content of |\jobname| cannot be compared
% to filenames specified in the source due to different catcodes.
% The following code rescans |\jobname|, stores the result
% in |\childdocname| and saves a copy in |\childdocjob|:
%    \begin{macrocode}
\edef\childdocname{\scantokens\expandafter{\jobname\noexpand}}
\let\childdocjob\childdocname
%    \end{macrocode}

% \macro{\childdocdisable}
% The macro |\childdocdisable| prevents the main file
% from being processed more than once.
% At this stage, the main document command |\childdocmain|
% is assumed to be called once again where it should do nothing.
% Any subsequent call to it should prevent
% a secondary processing of the main document
% It overwrites the forwarding commands
% |\childdocof| and |\childdocforward|
% with empty macros to prevent further inclusions of the main document:
%    \begin{macrocode}
\newcommand{\childdocdisable}
{
  \renewcommand{\childdocmain}[1]{\renewcommand{\childdocmain}[1]{\endinput}}
  \renewcommand{\childdocof}[1]{}
  \renewcommand{\childdocby}[2][]{}
  \renewcommand{\childdocforward}[2][]{}
  \renewcommand{\childdocdisable}{}
}
%    \end{macrocode}

% \macro{\childdocmain}
% The macro |\childdocmain| is to be called at the top of the main file
% with nothing or the main filename (without extension) as argument.
% First, it breaks loops.
% If the argument is not empty and does not match |\childdocname|
% (which is set by the first inclusion of |childdoc.def|),
% |\ifchilddoc| is set to true, |\includeonly| is applied to the child file
% and |\jobname| is set to the main file
% (for proper handling of |.aux| files):
%    \begin{macrocode}
\newcommand{\childdocmain}[1]
{
  \childdocdisable\childdocmain{}
  \if?#1?\else
    \begingroup
      \def\childdoctmp{#1}
      \ifx\childdoctmp\childdocname
        \def\childdoctmp{}
      \else
        \def\childdoctmp
        {
          \childdoctrue
          \includeonly{\childdocname}
          \def\childdocjob{#1}
          \def\jobname{#1}
        }
      \fi
      \expandafter
    \endgroup
    \childdoctmp
  \fi
}
%    \end{macrocode}

% \macro{\childdocof}
% The command |\childdocof| redirects
% compilation to the main file |#1|.
%    \begin{macrocode}
\newcommand{\childdocof}[1]
{
  \childdocdisable
  \childdoctrue
  \includeonly{\childdocname}
  \def\jobname{#1}
  \def\childdocjob{#1}
  \input{#1}
}
%    \end{macrocode}

% \macro{\childdocby}
% The command |\childdocby| ....
%    \begin{macrocode}
\newcommand{\childdocby}[2][]
{
  \childdocdisable
  \childdoctrue
  \childdocmanualtrue
  \if?#1?\else
    \def\jobname{#2}
  \fi
  \def\childdocjob{#2}
  \input{#2}
  \endinput
}
%    \end{macrocode}

% \macro{\childdocforward}
% The command |\childdocforward| redirects
% compilation to the main file or
% (if the optional argument is given) a child file.
% Parameters are set as if the main file
% or a child file starting with |\childdocof| was compiled.
% Then compilation is handed over to the main file:
%    \begin{macrocode}
\newcommand{\childdocforward}[2][]
{
  \begingroup
    \if?#1?
      \def\childdoctmp
      {
        \def\childdocname{#2}
        \def\childdocjob{#2}
        \def\jobname{#2}
        \input{#2}
        \endinput
      }
    \else
      \def\childdoctmp
      {
        \childdocdisable
        \def\childdocname{#2}
        \childdoctrue
        \includeonly{#2}
        \def\childdocjob{#1}
        \def\jobname{#1}
        \input{#1}
        \endinput
      }
    \fi
    \expandafter
  \endgroup
  \childdoctmp
}
%    \end{macrocode}

% \macro{\childdocforwardprefix}
% The command |\childdocforwardprefix| redirects
% compilation to the main or a child file by means of a pattern.
% The prefix |#1| in the current filename is replaced by |#2|
% and the suffix of the current filename is kept
% (it is assumed that the filename does not contain the substring `|~~~|'
% which is used as a delimiter).
% Compilation is handed over to the new file by |\childdocforward|:
%    \begin{macrocode}
\newcommand{\childdocforwardprefix}[3][]
{
  \begingroup
    \def\childdocextract #2##1~~~{\def\childdoctmp{\childdocforward[#1]{#3##1}}}
    \expandafter\childdocextract\childdocname~~~
    \expandafter
  \endgroup
  \childdoctmp
}
%    \end{macrocode}

% \macro{\childdoc}
% The deprecated macro |\childdoc| is a legacy version of |\childdocmain|:
%    \begin{macrocode}
\newcommand{\childdoc}{\childdocmain}
%    \end{macrocode}

% \macro{\childdocredirect}
% The deprecated macro |\childdocredirect| is a legacy version
% of |\childdocforward| and |\childdocforwardprefix|:
%    \begin{macrocode}
\newcommand{\childdocredirect}[2][]
{
  \begingroup
    \if?#1?
      \def\childdoctmp{\childdocforward{#2}}
    \else
      \def\childdoctmp{\childdocforwardprefix{#1}{#2}}
    \fi
    \expandafter
  \endgroup
  \childdoctmp
}
%    \end{macrocode}

%\iffalse
%</package>
%\fi
%
\endinput
|\\
|\childdocby{|\textit{main}|}|\\
\end{tabular}
\end{center}
%
The directive |\childdocby| is similar to |\childdocof|
described in \secref{sec:include},
but the subsequent selection of content must be done manually.
To that end, both |\ifchilddoc| and |\ifchilddocmanual|
will be true upon processing of a part,
and the name of the part is stored in |\childdocname|.
Note that |\jobname| will be set to the filename of the current part
so that each part receives an individual |.aux| file
that does not interfere with the |.aux| file(s) of the main document.
This behaviour can be altered by the alternative form
|\childdocby[*]{|\textit{main}|}| (with a non-empty optional argument)
which uses the |.aux| file of the main document
by setting |\jobname| to \textit{main}.

%%%%%%%%%%%%%%%%%%%%%%%%%%%%%%%%%%%%%%%%%%%%%%%%%%%%%%%%%%%%%%%%%%%%%%%%%%%%%%%%
\subsection{Driver Development}
\label{sec:driver}

The \textsf{childdoc} mechanism can also be use for the development
of definition files such as \LaTeX{} styles or classes.
This case differs from the above setup with multiple parts
included by |\include| in that no |\includeonly| should be invoked.
This can be achieved by starting the include file
(before |\ProvidesPackage|) with:
%
\begin{center}
\begin{tabular}{l}
|% \iffalse
%
% childdoc.dtx Copyright (C) 2017-2018 Niklas Beisert
%
% This work may be distributed and/or modified under the
% conditions of the LaTeX Project Public License, either version 1.3
% of this license or (at your option) any later version.
% The latest version of this license is in
%   http://www.latex-project.org/lppl.txt
% and version 1.3 or later is part of all distributions of LaTeX
% version 2005/12/01 or later.
%
% This work has the LPPL maintenance status `maintained'.
%
% The Current Maintainer of this work is Niklas Beisert.
%
% This work consists of the files childdoc.dtx and childdoc.ins
% and the derived files childdoc.def and cdocsamp.tex with
% cdocsch1.tex, cdocsch2.tex, cdocsdrf.tex, cdocsfn1.tex, cdocsfn2.tex.
%
%<package>\ifdefined\childdocmain\endinput\fi
%<package>\ProvidesFile{childdoc.def}[2018/12/30 v2.0 child document driver]
%<samplemain>\ProvidesFile{cdocsamp.tex}[2018/12/30 v2.0 sample for childdoc]
%<*driver>
%\ProvidesFile{childdoc.drv}[2018/12/30 v2.0 childdoc reference manual file]
\PassOptionsToClass{10pt,a4paper}{article}
\documentclass{ltxdoc}

\usepackage[margin=35mm]{geometry}
\usepackage{hyperref}
\usepackage{hyperxmp}
\usepackage[usenames]{color}

\hypersetup{colorlinks=true}
\hypersetup{pdfstartview=FitH}
\hypersetup{pdfpagemode=UseNone}
\hypersetup{pdfsource={}}
\hypersetup{pdflang={en-UK}}
\hypersetup{pdfcopyright={Copyright 2017-2018 Niklas Beisert.
  This work may be distributed and/or modified under the
  conditions of the LaTeX Project Public License, either version 1.3
  of this license or (at your option) any later version.}}
\hypersetup{pdflicenseurl={http://www.latex-project.org/lppl.txt}}
\hypersetup{pdfcontactaddress={ETH Zurich, ITP, HIT K,
  Wolfgang-Pauli-Strasse 27}}
\hypersetup{pdfcontactpostcode={8093}}
\hypersetup{pdfcontactcity={Zurich}}
\hypersetup{pdfcontactcountry={Switzerland}}
\hypersetup{pdfcontactemail={nbeisert@itp.phys.ethz.ch}}
\hypersetup{pdfcontacturl={http://people.phys.ethz.ch/\xmptilde nbeisert/}}

\newcommand{\secref}[1]{\hyperref[#1]{section \ref*{#1}}}

\parskip1ex
\parindent0pt
\let\olditemize\itemize
\def\itemize{\olditemize\parskip0pt}

\begin{document}

\title{The \textsf{childdoc} Package}
\hypersetup{pdftitle={The childdoc Package}}
\author{Niklas Beisert\\[2ex]
  Institut f\"ur Theoretische Physik\\
  Eidgen\"ossische Technische Hochschule Z\"urich\\
  Wolfgang-Pauli-Strasse 27, 8093 Z\"urich, Switzerland\\[1ex]
  \href{mailto:nbeisert@itp.phys.ethz.ch}
  {\texttt{nbeisert@itp.phys.ethz.ch}}}
\hypersetup{pdfauthor={Niklas Beisert}}
\hypersetup{pdfsubject={Manual for the LaTeX2e Package childdoc}}
\date{30 December 2018, \textsf{v2.0}}
\maketitle

\begin{abstract}\noindent
\textsf{childdoc} is a \LaTeXe{} package
that enables the direct compilation
of document sections included by |\include|
to individual files.
\end{abstract}

\begingroup
\parskip0ex
\tableofcontents
\endgroup

%%%%%%%%%%%%%%%%%%%%%%%%%%%%%%%%%%%%%%%%%%%%%%%%%%%%%%%%%%%%%%%%%%%%%%%%%%%%%%%%
%%%%%%%%%%%%%%%%%%%%%%%%%%%%%%%%%%%%%%%%%%%%%%%%%%%%%%%%%%%%%%%%%%%%%%%%%%%%%%%%
\section{Introduction}

\LaTeX{} provides a mechanism to structure a large document (such as a book)
into a main file and several child files (containing the chapters)
using the |\include| command.
This mechanism is beneficial for documents
which span hundreds of pages in order to
make the source file(s) more manageable.
Moreover, compilation can be restricted to
selected child files by means of the |\includeonly| command.
The latter feature can be used to reduce the compilation time while editing
(this was significantly more useful in the earlier days of \LaTeX{})
or to generate a smaller document which is easier to navigate.
Another application of |\includeonly| is to generate
documents consisting of selected parts of the complete document.

However, there are a few drawbacks of the plain |\include| mechanism:
\begin{itemize}
\item
The child files cannot be compiled on their own,
they can only be compiled via the main file.
A naive editing environment
(such as a text editor with an option
to have the current file processed by \LaTeX)
may require one to switch to the main file before compiling;
attempting to compile the child file produces errors.
\item
The main file must be modified (each time)
to adjust the |\includeonly| command
to the present needs. This easily leaves the main file in a messy state.
\item
The generated document will always carry the filename
of the main document. This is inconvenient if
several child files are to be compiled and
to be kept for distribution.
\end{itemize}

The present package provides a simple interface
to make child files individually compilable by \LaTeX{}.
Compiling a child file then has the same effect as compiling
the main file with an |\includeonly| command
to select the appropriate child.
Moreover the generated document will carry the name of the child
rather than the main file.
This resolves all three above issues.

This feature is meant to make the editing of books,
thesis documents and lecture notes somewhat more convenient.
However, the package can also be used efficiently for
composing a series of documents (such as exercise sheets)
which are typically distributed individually.
It then assists the author in generating the individual documents
(potentially in different versions)
as well as a document containing the collected series.
Another application is in developing style files
or other kinds of included material
where compilation of the style file could redirect
to a sample or test file.

%%%%%%%%%%%%%%%%%%%%%%%%%%%%%%%%%%%%%%%%%%%%%%%%%%%%%%%%%%%%%%%%%%%%%%%%%%%%%%%%
%%%%%%%%%%%%%%%%%%%%%%%%%%%%%%%%%%%%%%%%%%%%%%%%%%%%%%%%%%%%%%%%%%%%%%%%%%%%%%%%
\section{Usage}

First of all, the package \textsf{childdoc} is \emph{not} a standard
\LaTeXe{} |.sty| style file! Therefore it needs to be invoked in
a non-standard way.

%%%%%%%%%%%%%%%%%%%%%%%%%%%%%%%%%%%%%%%%%%%%%%%%%%%%%%%%%%%%%%%%%%%%%%%%%%%%%%%%
\subsection{Included Files}
\label{sec:include}

%%%%%%%%%%%%%%%%%%%%%%%%%%%%%%%%%%%%%%%%
\DescribeMacro{\childdocmain}
To use the package, add the commands
\begin{center}
\begin{tabular}{l}
|\input{childdoc.def}|\\
|\childdocmain{}|\\
\end{tabular}
\end{center}
at the very top of the main \LaTeX{} file,
in particular \emph{before} the |\documentclass| statement!
The argument of |\childdocmain| should be left empty
(but it must be present).

%%%%%%%%%%%%%%%%%%%%%%%%%%%%%%%%%%%%%%%%
\DescribeMacro{\childdocof}
Furthermore, add the commands
\begin{center}
\begin{tabular}{l}
|\input{childdoc.def}|\\
|\childdocof{|\textit{main}|}|\\
\end{tabular}
\end{center}
at the top of every child file \textit{child}
which is included by |\include{|\textit{child}|}|
from within the main file
(or at least for those files to be compiled individually).
The argument \textit{main} must be the filename of the main file.

There are a couple of
considerations in setting up the main and child documents:

%%%%%%%%%%%%%%%%%%%%%%%%%%%%%%%%%%%%%%%%
\paragraph{Restrictions.}

Please note the following restrictions:
\begin{itemize}
\item
|\childdocmain| must be called with one argument \textit{main}
to ensure compatibility with earlier version of the package.
It must either be empty (|\childdocmain{}|)
or precisely match the filename of the main file in which it is specified.
See \secref{sec:detection} for further information.
\item
The filename \textit{main} must be specified without the |.tex| extension.
\item
The filename \textit{main} is case sensitive
(even in case-insensitive file systems)
due to internal string comparison.
\item
The argument \textit{main} should be fully expanded, it cannot be a macro.
\item
Subdirectories and special characters should be avoided in filenames.
\item
The command |\childdocmain{|\textit{main}|}| must be followed by a whitespace.
It should not be followed immediately by another command
or by a comment mark `|%|'.
This is because the \TeX{} parser reads the token immediately following
the argument of |\childdocmain| and puts it
at the beginning of every child section;
however, a white\-space is ignored.
\end{itemize}

%%%%%%%%%%%%%%%%%%%%%%%%%%%%%%%%%%%%%%%%
\paragraph{Content of Main File.}

It is advisable to place all content in the child files included by |\include|.
Any output contained in the main file will appear in all child documents
unless suppressed manually;
it cannot be suppressed automatically by the |\includeonly| directive
and thus should normally be avoided.
A method to include some content in the main file
by means of conditional processing is described in \secref{sec:conditional}.

%%%%%%%%%%%%%%%%%%%%%%%%%%%%%%%%%%%%%%%%
\paragraph{Page Numbering.}

When only a part of the document is compiled,
the appropriate numbering of pages
(as well as other status parameters)
is determined from the |.aux| files.
The latter contain information from previous passes.
However this information needs to propagate through
all intermediate child documents.
Therefore the page numbering in child documents may well
be inconsistent until the complete document is compiled at least once.

A useful (if unconventional) way to always ensure a consistent
page numbering is to restart the numbering in each child document
and denote the pages by `\textit{child}|.|\textit{page}'
where \textit{child} represents the chapter/section number of the child file.
This can be achieved by the command
|\numberwithin{page}{|\textit{child}|}|
of the \textsf{amsmath} package
where \textit{child} can be |chapter| or |section|
depending on the chosen structuring.
Alternatively, one can modify the macro |\thepage| appropriately
and reset the counter |page| at the start of each child file.

%%%%%%%%%%%%%%%%%%%%%%%%%%%%%%%%%%%%%%%%%%%%%%%%%%%%%%%%%%%%%%%%%%%%%%%%%%%%%%%%
\subsection{Conditional Processing}
\label{sec:conditional}

The package provides a mechanism to compile different versions
of a document. To customise the versions further some conditional processing
can come in handy to distinguish which version is being compiled.
The package provides two macros to describe the compilation context:

%%%%%%%%%%%%%%%%%%%%%%%%%%%%%%%%%%%%%%%%
\DescribeMacro{\ifchilddoc}
The conditional |\ifchilddoc| distinguishes between the compilation of
child documents and the main document:
%
\begin{center}
|\ifchilddoc |\textit{child-code}| |[|\||else |\textit{main-code}]| \||fi|
\end{center}

%%%%%%%%%%%%%%%%%%%%%%%%%%%%%%%%%%%%%%%%
\DescribeMacro{\childdocname}
\DescribeMacro{\childdocjob}
The macro |\childdocname| contains the filename (without extension)
of the main or child file being processed.
Note that |\childdocjob| will always contain the name of the main file.

%%%%%%%%%%%%%%%%%%%%%%%%%%%%%%%%%%%%%%%%
\paragraph{Title Page.}

Conditional processing can be used to include a title or banner page
in the main document when proper precautions are taken.
Importantly, the code in the main file should ensure that the page counter
(as well as other status parameters which are stored in the |.aux| files)
takes the same value after the conditional processing.
Otherwise the page numbers may take divergent values
depending on which part is compiled.

For example, a title page could be declared by:
%
\begin{center}
\begin{tabular}{l}
|\ifchilddoc\||else|\\
|\addtocounter{page}{-1}|\\
\textit{code for title page}\\
|\newpage|\\
|\||fi|
\end{tabular}
\end{center}
%
A banner page for the child documents can be generated by:
%
\begin{center}
\begin{tabular}{l}
|\ifchilddoc|\\
|\addtocounter{page}{-1}|\\
\textit{code for banner page}\\
|\newpage|\\
|\||fi|
\end{tabular}
\end{center}
%
Here one could write a message such as:
\begin{center}
|This is the part \childdocname{} of \childdocjob{}.|
\end{center}

%%%%%%%%%%%%%%%%%%%%%%%%%%%%%%%%%%%%%%%%%%%%%%%%%%%%%%%%%%%%%%%%%%%%%%%%%%%%%%%%
\subsection{Flags}
\label{sec:flags}

The package makes it easy to generate different versions
of the main or child documents.
To this end compilation flags can be defined
and assigned different default values.
They will be particularly useful in conjunction
with the forwarding mechanism described in \secref{sec:forward}.

For example, it may be useful to have a flag |\version|
which can be set to |draft| or |final|.
The document source will contain some conditional code
depending on the value of |\version|.
Suppose further, the flag should default to |final| for the main file
and to |draft| for child files
which is a natural assignment for editing the document.
This is achieved by placing the following code
in the preamble of the main document
(below the |\childdocmain| directive):
%
\begin{center}
\begin{tabular}{l}
|\ifchilddoc|\\
|\providecommand{\version}{draft}|\\
|\||else|\\
|\providecommand{\version}{final}|\\
|\||fi|
\end{tabular}
\end{center}
%
The definition by |\providecommand| makes sure
that previous definitions are not overwritten.
Further statements |\providecommand{\version}{...}|
can thus be added before the above code to override it.

For the main file, one might add a line
(between |\childdocmain| and the above block)
%
\begin{center}
|%\ifchilddoc\||else\providecommand{\version}{draft}\||fi|
\end{center}
%
which can be uncommented to produce a draft version.
Likewise one can add a line to the very top of a child file
(above the |\childdocof{|\textit{main}|}| directive)
%
\begin{center}
|%\providecommand{\version}{final}|
\end{center}
%
which can be uncommented to produce the final version of this child document.

%%%%%%%%%%%%%%%%%%%%%%%%%%%%%%%%%%%%%%%%%%%%%%%%%%%%%%%%%%%%%%%%%%%%%%%%%%%%%%%%
\subsection{Forwarding}
\label{sec:forward}

Different versions of the main or child documents
using compilation flags as described in \secref{sec:flags}
can be (permanently) stored in different files
for convenient compilation, viewing and distribution.
To this end, the package defines a command
to pass on compilation to a different file:

%%%%%%%%%%%%%%%%%%%%%%%%%%%%%%%%%%%%%%%%
\DescribeMacro{\childdocforward}
The command |\childdocforward| redirects processing to
another source file:
%
\begin{center}
\begin{tabular}{l}
|\input{childdoc.def}|\\
|\childdocforward[|\textit{main}|]{|\textit{dest}|}|\\
\end{tabular}
\end{center}
%
The argument \textit{dest} is the destination file
(without extension).
It should be the main file or one of the child files.
Note that further \textsf{childdoc} directives
such as |\childdocof| and |\childdocforward|
in the indicated file will be processed in this form.
The optional argument \textit{main}
passes on directly to the main file \textit{main}
while pretending to compile the child \textit{dest}.
This form behaves as if \textit{dest}
issues |\childdocof{|\textit{main}|}| right away,
and no further \textsf{childdoc} directives will be processed.

%%%%%%%%%%%%%%%%%%%%%%%%%%%%%%%%%%%%%%%%
\DescribeMacro{\...prefix}
In the alternative form |\childdocforwardprefix|,
%
\begin{center}
\begin{tabular}{l}
|\input{childdoc.def}|\\
|\childdocforwardprefix[|\textit{main}|]{|\textit{prefix}|}{|\textit{dest}|}|
\end{tabular}
\end{center}
%
the destination file is determined by a pattern
depending on the current file:
To make this work, the current file must be called
`{\textit{prefix}\hspace{0.2em}\textit{suffix}}'
with \textit{prefix} matching precisely the argument.
Processing is then passed on to the file
`{\textit{dest}\hspace{0.2em}\textit{suffix}}'.
Surely, the same effect is achieved by
directly specifying the
argument `{\textit{dest}\hspace{0.2em}\textit{suffix}}'
in the first form.
However, that requires to set up a different file
for each child. With the alternative form of the command
all these files can have exactly the same content
which simplifies setting them up and maintaining them.

For example, the following file |draft.tex|
with a compilation flag |\version| as described in \secref{sec:flags}
compiles the main document as a draft:
%
\begin{center}
\begin{tabular}{l}
|\def\version{draft}|\\
|\input{childdoc.def}|\\
|\childdocforward{|\textit{main}|}|
\end{tabular}
\end{center}
%
Likewise, the following files |final|\textit{nn}|.tex|
compile the final version of the child document
|child|\textit{nn}|.tex|:
%
\begin{center}
\begin{tabular}{l}
|\def\version{final}|\\
|\input{childdoc.def}|\\
|\childdocforwardprefix{final}{child}|
\end{tabular}
\end{center}
%

Note that when several versions of a main file and/or of each child file
are to be generated, it may be convenient to set up a |Makefile| or
shell script to automatise the process.

%%%%%%%%%%%%%%%%%%%%%%%%%%%%%%%%%%%%%%%%%%%%%%%%%%%%%%%%%%%%%%%%%%%%%%%%%%%%%%%%
\subsection{Command Line Processing}
\label{sec:commandline}

The effect of redirection files can also be achieved by invoking
the \LaTeX{} compiler with a more elaborate command line.
Most conveniently this should be done as part
of a shell script or a |Makefile|.

When using \textsf{childdoc} in the main file, the following
command lines effectively perform a redirection
(note that depending on the shell being used,
backslashes may have to be doubled: `|\|' $\to$ `|\\|'):
%
\begin{center}
|... -jobname "|\textit{target}|" |\\|"|[\textit{flags}]%
|\input{childdoc.def}\childdocforward[|\textit{main}|]{|\textit{dest}|}"|
\end{center}
%
Here \textit{target} is the name of the output file,
\textit{main} is the name of the main file
and \textit{dest} is the name of the main or child file to be processed
(all filenames without extensions).
The optional argument \textit{main} can be omitted
if \textit{main} matches \textit{dest}.
Optionally, compilation \textit{flags} can be defined via |\def| commands.
This command line makes the \TeX{} engine believe
it is compiling the file \textit{target}
whose content is specified as the latter parameter.
The provided code then forwards the processing to
\textit{main} or \textit{dest} as described in \secref{sec:forward}.

%%%%%%%%%%%%%%%%%%%%%%%%%%%%%%%%%%%%%%%%%%%%%%%%%%%%%%%%%%%%%%%%%%%%%%%%%%%%%%%%
\subsection{Include by Input}
\label{sec:input}

Including child documents by |\include| has some restrictions by design.
Most notably, the content of a child document always occupies
its own set of pages; pages cannot be shared between child documents.
Usually, this behaviour makes perfect sense
because each child document contain an essential part of the document.
However, in some situations it may be desirable to compose
a document from a collection of parts
without having mandatory page breaks between then.
For this case, the package
provides a mechanism to include parts
by |\input| which can also be processed individually.
However, by construction this mechanism
requires manual handling of the content to be output.

%%%%%%%%%%%%%%%%%%%%%%%%%%%%%%%%%%%%%%%%
\DescribeMacro{\ifchilddocmanual}
The main file should be prepared as usual, see \secref{sec:include}.
However, the document body must make a distinction
between processing of an individual part and of the main document, e.g.:
%
\begin{center}
\begin{tabular}{l}
|\ifchilddocmanual|\\
|\input{\childdocname}|\\
|\||else|\\
\textit{document body with }|\input{|\textit{part}|}|\\
|\||fi|
\end{tabular}
\end{center}
%
The conditional |\ifchilddocmanual| is true whenever
a part to be included by |\input| is being compiled,
and the name of the part is stored in |\childdocname|.

%%%%%%%%%%%%%%%%%%%%%%%%%%%%%%%%%%%%%%%%
\DescribeMacro{\childdocby}
Each part to be included by |\input| should start with:
%
\begin{center}
\begin{tabular}{l}
|\input{childdoc.def}|\\
|\childdocby{|\textit{main}|}|\\
\end{tabular}
\end{center}
%
The directive |\childdocby| is similar to |\childdocof|
described in \secref{sec:include},
but the subsequent selection of content must be done manually.
To that end, both |\ifchilddoc| and |\ifchilddocmanual|
will be true upon processing of a part,
and the name of the part is stored in |\childdocname|.
Note that |\jobname| will be set to the filename of the current part
so that each part receives an individual |.aux| file
that does not interfere with the |.aux| file(s) of the main document.
This behaviour can be altered by the alternative form
|\childdocby[*]{|\textit{main}|}| (with a non-empty optional argument)
which uses the |.aux| file of the main document
by setting |\jobname| to \textit{main}.

%%%%%%%%%%%%%%%%%%%%%%%%%%%%%%%%%%%%%%%%%%%%%%%%%%%%%%%%%%%%%%%%%%%%%%%%%%%%%%%%
\subsection{Driver Development}
\label{sec:driver}

The \textsf{childdoc} mechanism can also be use for the development
of definition files such as \LaTeX{} styles or classes.
This case differs from the above setup with multiple parts
included by |\include| in that no |\includeonly| should be invoked.
This can be achieved by starting the include file
(before |\ProvidesPackage|) with:
%
\begin{center}
\begin{tabular}{l}
|\input{childdoc.def}|\\
|\childdocforward{|\textit{main}|}|\\
\end{tabular}
\end{center}
%
or alternatively with:
%
\begin{center}
\begin{tabular}{l}
|\input{childdoc.def}|\\
|\childdocby{|\textit{main}|}|\\
\end{tabular}
\end{center}
%
Both forms have slightly different effects as described above.
The main file is prepared as usual, see \secref{sec:include}.

%%%%%%%%%%%%%%%%%%%%%%%%%%%%%%%%%%%%%%%%%%%%%%%%%%%%%%%%%%%%%%%%%%%%%%%%%%%%%%%%
\subsection{Legacy Detection}
\label{sec:detection}

The directive |\childdocmain| in the main file can detect
whether the complete document or merely a child is to be compiled
even without using the directive |\childdocof|.
This method is deprecated because it is less robust
and there is no compelling reason to use it;
it is merely provided for backward compatibility
and it may be removed in future versions.

If the detection mechanism is to be used,
it is mandatory to correctly specify
the filename of the main file as the argument of |\childdocmain|:
%
\begin{center}
\begin{tabular}{l}
|\input{childdoc.def}|\\
|\childdocmain{|\textit{main}|}|\\
\end{tabular}
\end{center}
%
If |\jobname| does not match the argument \textit{main} of |\childdocmain|,
it is assumed that |\jobname| points to the child file to be compiled.
When using |\childdocmain| with the main file specified as argument,
it suffices to start a child file
with just |\input{|\textit{main}|}|
without loading of the package and using |\childdocof|.
If instead all processing is done
with the appropriate \textsf{childdoc} directives,
the argument of \textit{main} of |\childdocmain| can be empty.

An alternative version of the command line processing described
in \secref{sec:commandline} using the detection mechanism reads:
%
\begin{center}
|... -jobname "|\textit{target}|" "|[\textit{flags}]%
[|\def\jobname{|\textit{dest}|}|]|\input{|\textit{main}|}"|
\end{center}

%%%%%%%%%%%%%%%%%%%%%%%%%%%%%%%%%%%%%%%%%%%%%%%%%%%%%%%%%%%%%%%%%%%%%%%%%%%%%%%%
\subsection{Manual Code}
\label{sec:manual}

In case one cannot be certain whether the definitions file |childdoc.def|
is installed on the target \TeX{} distribution
and one prefers not to ship it,
it is conceivable to paste a few relevant commands into the sources.

To that end, drop all statements |\input{childdoc.def}|
and perform the replacements as outlined below.
Instead of |\childdocmain{|\textit{main}|}| add the following code
to the top of the main file:
%
\begin{center}
\begin{tabular}{l}
|\||ifdefined\childdocname\endinput\||fi\newif\ifchilddoc|\\
|\edef\childdocname{\scantokens\expandafter{\jobname\noexpand}}|\\
|\def\childdocmain{|\textit{main}|}\||ifx\childdocmain\childdocname\||else|\\
|\childdoctrue\includeonly{\childdocname}\let\jobname\childdocmain\||fi|\\
\end{tabular}
\end{center}
%
Instead of |\childdocof{|\textit{main}|}| just include the main file
at the top of each child file:
%
\begin{center}
|\input{|\textit{main}|}|
\end{center}
%
A simple redirection |\childdocforward{|\textit{dest}|}| is achieved by:
%
\begin{center}
|\def\jobname{|\textit{dest}|}\input{\jobname}|
\end{center}
%
The redirection with prefix
|\childdocforwardprefix[|\textit{prefix}|]{|\textit{dest}|}|
is accomplished by:
%
\begin{center}
\begin{tabular}{l}
|{\edef\jobname{\scantokens\expandafter{\jobname\noexpand}}|\\
|\def\redirectjob |\textit{prefix}|#1~~~{\gdef\jobname{|\textit{dest}|#1}}|\\
|\expandafter\redirectjob\jobname~~~}\input{\jobname}|
\end{tabular}
\end{center}

In an alternative approach,
child documents can be compiled by a specific command line
without additional code or specific definitions:
%
\begin{center}
|... -jobname "|\textit{target}|" "|[\textit{flags}]%
|\includeonly{|\textit{dest}|}\input{|\textit{main}|}"|
\end{center}
%

%%%%%%%%%%%%%%%%%%%%%%%%%%%%%%%%%%%%%%%%%%%%%%%%%%%%%%%%%%%%%%%%%%%%%%%%%%%%%%%%
%%%%%%%%%%%%%%%%%%%%%%%%%%%%%%%%%%%%%%%%%%%%%%%%%%%%%%%%%%%%%%%%%%%%%%%%%%%%%%%%
\section{Information}

%%%%%%%%%%%%%%%%%%%%%%%%%%%%%%%%%%%%%%%%%%%%%%%%%%%%%%%%%%%%%%%%%%%%%%%%%%%%%%%%
\subsection{Copyright}

Copyright \copyright{} 2017--2018 Niklas Beisert

This work may be distributed and/or modified under the
conditions of the \LaTeX{} Project Public License, either version 1.3
of this license or (at your option) any later version.
The latest version of this license is in
  \url{http://www.latex-project.org/lppl.txt}
and version 1.3 or later is part of all distributions of \LaTeX{}
version 2005/12/01 or later.

This work has the LPPL maintenance status `maintained'.

The Current Maintainer of this work is Niklas Beisert.

This work consists of the files |README.txt|, |childdoc.ins| and |childdoc.dtx|
as well as the derived files |childdoc.def|, |cdocsamp.tex|
with |cdocsch1.tex|, |cdocsch2.tex|, |cdocspt3.tex|, |cdocspt4.tex|,
|cdocsdrf.tex|, |cdocsfn1.tex|, |cdocsfn2.tex|
as well as |childdoc.pdf|.

%%%%%%%%%%%%%%%%%%%%%%%%%%%%%%%%%%%%%%%%%%%%%%%%%%%%%%%%%%%%%%%%%%%%%%%%%%%%%%%%
\subsection{Files and Installation}

The package consists of the files:
%
\begin{center}
\begin{tabular}{ll}
    |README.txt|   & readme file \\
    |childdoc.ins| & installation file \\
    |childdoc.dtx| & source file \\
    |childdoc.def| & definition file \\
    |cdocsamp.tex| & sample main file \\
    |cdocsch1.tex| & sample include file \\
    |cdocsch2.tex| & sample include file \\
    |cdocspt3.tex| & sample part file \\
    |cdocspt4.tex| & sample part file \\
    |cdocsdrf.tex| & sample redirection file \\
    |cdocsfn1.tex| & sample redirection file \\
    |cdocsfn2.tex| & sample redirection file \\
    |childdoc.pdf| & manual
\end{tabular}
\end{center}
%
The distribution consists of the files
|README.txt|, |childdoc.ins| and |childdoc.dtx|.
%
\begin{itemize}
\item
Run (pdf)\LaTeX{} on |childdoc.dtx|
to compile the manual |childdoc.pdf| (this file).
\item
Run \LaTeX{} on |childdoc.ins| to create the definitions file |childdoc.def|
and the sample |cdocsamp.tex| with include files
|cdocsch1.tex|, |cdocsch2.tex|, |cdocspt3.tex|, |cdocspt4.tex|,
|cdocsdrf.tex|, |cdocsfn1.tex|, |cdocsfn2.tex|.
Then copy the file |childdoc.def| to an appropriate directory of your \LaTeX{}
distribution, e.g.\ \textit{texmf-root}|/tex/latex/childdoc|.
\end{itemize}

%%%%%%%%%%%%%%%%%%%%%%%%%%%%%%%%%%%%%%%%%%%%%%%%%%%%%%%%%%%%%%%%%%%%%%%%%%%%%%%%
\subsection{Related CTAN Packages}

There are several other packages which offer a similar functionality:
%
\begin{itemize}
\item
The packages
\href{http://ctan.org/pkg/docmute}{\textsf{docmute}},
\href{http://ctan.org/pkg/includex}{\textsf{includex}} and
\href{http://ctan.org/pkg/standalone}{\textsf{standalone}}
provide commands to include only the document body of
a child file thus allowing both files to be compiled individually.
\item
The packages \href{http://ctan.org/pkg/subdocs}{\textsf{subdocs}}
and \href{http://ctan.org/pkg/subfiles}{\textsf{subfiles}}
provide structures in which the main and child documents can be
encapsulated and allowing them to be compiled individually.
The inclusion mechanism is different from the conventional |\include|.
\item
The package \href{http://ctan.org/pkg/combine}{\textsf{combine}}
is an elaborate solution to combine several documents into one.
\end{itemize}
%
See also the CTAN topic \href{http://ctan.org/topic/subdocs}{\textsf{subdocs}}
for further related packages.
The present package differs from the above solutions in that
a document structure constructed with the conventional |\include| mechanism
just needs two extra commands at the top of every file
such that all constituent files can be compiled individually.

%%%%%%%%%%%%%%%%%%%%%%%%%%%%%%%%%%%%%%%%%%%%%%%%%%%%%%%%%%%%%%%%%%%%%%%%%%%%%%%%
%\subsection{Feature Suggestions}
%
%The following is a list of features which may be useful for future
%versions of this package:
%%
%\begin{itemize}
%\item
%\ldots
%\end{itemize}

%%%%%%%%%%%%%%%%%%%%%%%%%%%%%%%%%%%%%%%%%%%%%%%%%%%%%%%%%%%%%%%%%%%%%%%%%%%%%%%%
\subsection{Revision History}

%%%%%%%%%%%%%%%%%%%%%%%%%%%%%%%%%%%%%%%%
\paragraph{v2.0:} 2018/12/30

\begin{itemize}
\item
immediate forward processing
\item
added |\childdocby| mechanism
\item
manual restructured
\end{itemize}

%%%%%%%%%%%%%%%%%%%%%%%%%%%%%%%%%%%%%%%%
\paragraph{v1.6:} 2018/01/17

\begin{itemize}
\item
application for development of include files
\item
corrections to manual
\end{itemize}

%%%%%%%%%%%%%%%%%%%%%%%%%%%%%%%%%%%%%%%%
\paragraph{v1.5:} 2017/05/21

\begin{itemize}
\item
more complete structuring introduced
\item
|\childdocof| introduced
\item
|\childdoc| renamed to |\childdocmain|
\item
|\childredirect| renamed to |\childdocforward| and |\childdocforwardprefix|
and functionality expanded
\end{itemize}

%%%%%%%%%%%%%%%%%%%%%%%%%%%%%%%%%%%%%%%%
\paragraph{v1.0:} 2017/04/27

\begin{itemize}
\item
manual and install package
\item
first version published on CTAN
\end{itemize}

%%%%%%%%%%%%%%%%%%%%%%%%%%%%%%%%%%%%%%%%
\paragraph{v0.6:} 2017/04/26

\begin{itemize}
\item
redirection mechanism added
\end{itemize}

%%%%%%%%%%%%%%%%%%%%%%%%%%%%%%%%%%%%%%%%
\paragraph{v0.5:} 2017/04/26

\begin{itemize}
\item
functionality in definition file
\end{itemize}


%%%%%%%%%%%%%%%%%%%%%%%%%%%%%%%%%%%%%%%%%%%%%%%%%%%%%%%%%%%%%%%%%%%%%%%%%%%%%%%%
%%%%%%%%%%%%%%%%%%%%%%%%%%%%%%%%%%%%%%%%%%%%%%%%%%%%%%%%%%%%%%%%%%%%%%%%%%%%%%%%
%%%%%%%%%%%%%%%%%%%%%%%%%%%%%%%%%%%%%%%%%%%%%%%%%%%%%%%%%%%%%%%%%%%%%%%%%%%%%%%%
\appendix

\settowidth\MacroIndent{\rmfamily\scriptsize 000\ }

 \DocInput{childdoc.dtx}

\end{document}
%</driver>
% \fi
%
% %%%%%%%%%%%%%%%%%%%%%%%%%%%%%%%%%%%%%%%%%%%%%%%%%%%%%%%%%%%%%%%%%%%%%%%%%%%%%%
% %%%%%%%%%%%%%%%%%%%%%%%%%%%%%%%%%%%%%%%%%%%%%%%%%%%%%%%%%%%%%%%%%%%%%%%%%%%%%%
% \section{Sample}
%\iffalse
%<*samplemain>
%\fi
%
% The following presents a sample document
% with two chapters, two parts, a title page,
% a compile flag as well as three forwarding files to set the flag.
% It consists of eight |.tex| files:
% \begin{center}
% \begin{tabular}{ll}
% |cdocsamp.tex|&main file\\
% |cdocsch1.tex|&include file for chapter 1\\
% |cdocsch2.tex|&include file for chapter 2\\
% |cdocspt3.tex|&include file for part 3\\
% |cdocspt4.tex|&include file for part 4\\
% |cdocsdrf.tex|&forwarding file for main file in draft mode\\
% |cdocsfi1.tex|&forwarding file for final version of chapter 1\\
% |cdocsfi2.tex|&forwarding file for final version of chapter 2\\
% \end{tabular}
% \end{center}
% Each of the eight files can be compiled directly by the \LaTeX{} compiler.
%
% %%%%%%%%%%%%%%%%%%%%%%%%%%%%%%%%%%%%%%
% \paragraph{Main File.}
%
% The main file is called |cdocsamp.tex|.
%
% Load the \textsf{childdoc} definitions and
% declare the filename for the main document:
%    \begin{macrocode}
\input{childdoc.def}
\childdocmain{}
%    \end{macrocode}

% Optional override for |\version| flag:
%    \begin{macrocode}
%%\ifchilddoc\else\providecommand{\version}{draft}\fi
%    \end{macrocode}

% Define the default values for the |\version| flag
% (|final| for the main file and |draft| for childs):
%    \begin{macrocode}
\ifchilddoc
\providecommand{\version}{draft}
\else
\providecommand{\version}{final}
\fi
%    \end{macrocode}

% Load the standard document class:
%    \begin{macrocode}
\documentclass[12pt]{article}
%    \end{macrocode}

% Start the document body:
%    \begin{macrocode}
\begin{document}
%    \end{macrocode}

% Declare a title page.
% Print title, part of document being processed and version flag:
%    \begin{macrocode}
\addtocounter{page}{-1}
\begin{center}
{\LARGE\bfseries{}childdoc example\par}
\vspace{1cm}
\ifchilddoc
\ifchilddocmanual part\else chapter\fi:
`\childdocname' of `\childdocjob'\par
\else
main document: `\childdocjob'\par
\fi
version: \version\par
\end{center}
\newpage
%    \end{macrocode}

% Manually include selected file,
% otherwise process as usual:
%    \begin{macrocode}
\ifchilddocmanual
\section*{part `\childdocname'}
\input{\childdocname}
\else
%    \end{macrocode}

% Include the two chapters:
%    \begin{macrocode}
\include{cdocsch1}
\include{cdocsch2}
%    \end{macrocode}

% Include the two parts unless only chapters should be displayed:
%    \begin{macrocode}
\ifchilddoc\else
\section{part three}
\input{cdocspt3}
\section{part four}
\input{cdocspt4}
\fi
%    \end{macrocode}

% Process as usual until here:
%    \begin{macrocode}
\fi
%    \end{macrocode}

% End of document body:
%    \begin{macrocode}
\end{document}
%    \end{macrocode}
%\iffalse
%</samplemain>
%\fi
%
% %%%%%%%%%%%%%%%%%%%%%%%%%%%%%%%%%%%%%%
% \paragraph{Chapter Include Files.}
%
% The include files are called |cdocsch1.tex| and |cdocsch2.tex|.
%
%\iffalse
%<*samplechap1|samplechap2>
%\fi

% Optional override for |\version| flag:
%    \begin{macrocode}
%%\providecommand{\version}{final}
%    \end{macrocode}

% Include the main document:
%    \begin{macrocode}
\input{childdoc.def}
\childdocof{cdocsamp}
%    \end{macrocode}

%\iffalse
%</samplechap1|samplechap2>
%\fi
%
%\iffalse
%<*samplechap1>
%\fi
% Some text for chapter 1:
%    \begin{macrocode}
\section{one}
some text in chapter one
%    \end{macrocode}

%\iffalse
%</samplechap1>
%\fi
% Some text for chapter 2:
%\iffalse
%<*samplechap2>
%\fi
%    \begin{macrocode}
\section{two}
more text in chapter two
%    \end{macrocode}

%\iffalse
%</samplechap2>
%\fi
%
% %%%%%%%%%%%%%%%%%%%%%%%%%%%%%%%%%%%%%%
% \paragraph{Part Include Files.}
%
% The include files are called |cdocspt3.tex| and |cdocspt4.tex|.
%
%\iffalse
%<*samplepart3|samplepart4>
%\fi

% Optional override for |\version| flag:
%    \begin{macrocode}
%%\providecommand{\version}{final}
%    \end{macrocode}

% Include the main document:
%    \begin{macrocode}
\input{childdoc.def}
\childdocby{cdocsamp}
%    \end{macrocode}

%\iffalse
%</samplepart3|samplepart4>
%\fi
%
%\iffalse
%<*samplepart3>
%\fi
% Some text for part 3:
%    \begin{macrocode}
some text in part three
%    \end{macrocode}

%\iffalse
%</samplepart3>
%\fi
% Some text for part 4:
%\iffalse
%<*samplepart4>
%\fi
%    \begin{macrocode}
more text in part four
%    \end{macrocode}

%\iffalse
%</samplepart4>
%\fi
%
% %%%%%%%%%%%%%%%%%%%%%%%%%%%%%%%%%%%%%%
% \paragraph{Forwarding for a Complete Draft.}
%
% The following forwarding file |cdocsdrf.tex|
% compiles the main document in draft mode:
%\iffalse
%<*sampledraft>
%\fi
%    \begin{macrocode}
\def\version{draft}
\input{childdoc.def}
\childdocforward{cdocsamp}
%    \end{macrocode}

%\iffalse
%</sampledraft>
%\fi
%
% %%%%%%%%%%%%%%%%%%%%%%%%%%%%%%%%%%%%%%
% \paragraph{Forwarding for Final Version of the Chapters.}
%
% The following forwarding files |cdocsfn1.tex| and |cdocsfn2.tex|
% (with identical content)
% compile the final versions of the child documents
% |cdocsch1.tex| and |cdocsch2.tex|, respectively:
%\iffalse
%<*samplefinal>
%\fi
%    \begin{macrocode}
\def\version{final}
\input{childdoc.def}
\childdocforwardprefix[cdocsamp]{cdocsfn}{cdocsch}
%    \end{macrocode}

%\iffalse
%</samplefinal>
%\fi
%
% %%%%%%%%%%%%%%%%%%%%%%%%%%%%%%%%%%%%%%
% \paragraph{Command Line Processing.}
%
% The following three command lines generate the output files
% |cdocscld|, |cdocscl1| and |cdocscl2|
% which should be identical to
% |cdocsdrf|, |cdocsch1| and |cdocsfn2|, respectively:
% \begin{center}
% \begin{tabular}{l}
% |latex -jobname cdocscld \|\\
% |  "\def\version{draft}\input{childdoc.def}\childdocforward{cdocsamp}"|\\
% |latex -jobname cdocscl1 \|\\
% |  "\input{childdoc.def}\childdocforward[cdocsamp]{cdocsch1}"|\\
% |latex -jobname cdocscl2 \|\\
% |  "\def\version{final}\input{childdoc.def}\childdocforward{cdocsch2}"|
% \end{tabular}
% \end{center}
% Note that the trailing backslash on each first line
% merely continues the input to the second line
% (for convenient cut ant paste).
% Furthermore, the command |latex| can be replaced by any
% of its alternative versions such as |pdflatex|.
%
% %%%%%%%%%%%%%%%%%%%%%%%%%%%%%%%%%%%%%%%%%%%%%%%%%%%%%%%%%%%%%%%%%%%%%%%%%%%%%%
% %%%%%%%%%%%%%%%%%%%%%%%%%%%%%%%%%%%%%%%%%%%%%%%%%%%%%%%%%%%%%%%%%%%%%%%%%%%%%%
% \section{Implementation}
%\iffalse
%<*package>
%\fi
%
% This section describes the definitions file |childdoc.def|.

% The definitions cannot be loaded using |\usepackage| or |\RequirePackage|
% which has a mechanism to prevent loading a style file more than once.
% When loading the definitions by means of |\input|
% multiple instances have to be prevented manually:
%\iffalse
%This code needs to be before the `\ProvidesFile' directive
%which is defined at the beginning of this file.
%Therefore it is also placed there and commented out here.
%</package>
%<*discard>
%\fi
%    \begin{macrocode}
\ifdefined\childdocmain\endinput\fi
%    \end{macrocode}
%\iffalse
%</discard>
%<*package>
%\fi
%
% \macro{\ifchilddoc}
% \macro{\ifchilddocmanual}
% The conditional |\ifchilddoc| tells whether a
% child (true) or main (false) document is being compiled.
% The conditional |\ifchilddocmanual| tells whether
% the |\includeonly| mechanism is used (false) or
% the selection of child files must be performed manually (true).
% The definitions initialise to false:
%    \begin{macrocode}
\newif\ifchilddoc
\newif\ifchilddocmanual
%    \end{macrocode}

% \macro{\childdocname}
% \macro{\childdocjob}
% The macro |\childdocname| stores the name of the main document
% to be compiled. The macro |\childdocjob| stores the name of
% the document on which the \LaTeX{} compiler was originally invoked.
% The content of |\jobname| cannot be compared
% to filenames specified in the source due to different catcodes.
% The following code rescans |\jobname|, stores the result
% in |\childdocname| and saves a copy in |\childdocjob|:
%    \begin{macrocode}
\edef\childdocname{\scantokens\expandafter{\jobname\noexpand}}
\let\childdocjob\childdocname
%    \end{macrocode}

% \macro{\childdocdisable}
% The macro |\childdocdisable| prevents the main file
% from being processed more than once.
% At this stage, the main document command |\childdocmain|
% is assumed to be called once again where it should do nothing.
% Any subsequent call to it should prevent
% a secondary processing of the main document
% It overwrites the forwarding commands
% |\childdocof| and |\childdocforward|
% with empty macros to prevent further inclusions of the main document:
%    \begin{macrocode}
\newcommand{\childdocdisable}
{
  \renewcommand{\childdocmain}[1]{\renewcommand{\childdocmain}[1]{\endinput}}
  \renewcommand{\childdocof}[1]{}
  \renewcommand{\childdocby}[2][]{}
  \renewcommand{\childdocforward}[2][]{}
  \renewcommand{\childdocdisable}{}
}
%    \end{macrocode}

% \macro{\childdocmain}
% The macro |\childdocmain| is to be called at the top of the main file
% with nothing or the main filename (without extension) as argument.
% First, it breaks loops.
% If the argument is not empty and does not match |\childdocname|
% (which is set by the first inclusion of |childdoc.def|),
% |\ifchilddoc| is set to true, |\includeonly| is applied to the child file
% and |\jobname| is set to the main file
% (for proper handling of |.aux| files):
%    \begin{macrocode}
\newcommand{\childdocmain}[1]
{
  \childdocdisable\childdocmain{}
  \if?#1?\else
    \begingroup
      \def\childdoctmp{#1}
      \ifx\childdoctmp\childdocname
        \def\childdoctmp{}
      \else
        \def\childdoctmp
        {
          \childdoctrue
          \includeonly{\childdocname}
          \def\childdocjob{#1}
          \def\jobname{#1}
        }
      \fi
      \expandafter
    \endgroup
    \childdoctmp
  \fi
}
%    \end{macrocode}

% \macro{\childdocof}
% The command |\childdocof| redirects
% compilation to the main file |#1|.
%    \begin{macrocode}
\newcommand{\childdocof}[1]
{
  \childdocdisable
  \childdoctrue
  \includeonly{\childdocname}
  \def\jobname{#1}
  \def\childdocjob{#1}
  \input{#1}
}
%    \end{macrocode}

% \macro{\childdocby}
% The command |\childdocby| ....
%    \begin{macrocode}
\newcommand{\childdocby}[2][]
{
  \childdocdisable
  \childdoctrue
  \childdocmanualtrue
  \if?#1?\else
    \def\jobname{#2}
  \fi
  \def\childdocjob{#2}
  \input{#2}
  \endinput
}
%    \end{macrocode}

% \macro{\childdocforward}
% The command |\childdocforward| redirects
% compilation to the main file or
% (if the optional argument is given) a child file.
% Parameters are set as if the main file
% or a child file starting with |\childdocof| was compiled.
% Then compilation is handed over to the main file:
%    \begin{macrocode}
\newcommand{\childdocforward}[2][]
{
  \begingroup
    \if?#1?
      \def\childdoctmp
      {
        \def\childdocname{#2}
        \def\childdocjob{#2}
        \def\jobname{#2}
        \input{#2}
        \endinput
      }
    \else
      \def\childdoctmp
      {
        \childdocdisable
        \def\childdocname{#2}
        \childdoctrue
        \includeonly{#2}
        \def\childdocjob{#1}
        \def\jobname{#1}
        \input{#1}
        \endinput
      }
    \fi
    \expandafter
  \endgroup
  \childdoctmp
}
%    \end{macrocode}

% \macro{\childdocforwardprefix}
% The command |\childdocforwardprefix| redirects
% compilation to the main or a child file by means of a pattern.
% The prefix |#1| in the current filename is replaced by |#2|
% and the suffix of the current filename is kept
% (it is assumed that the filename does not contain the substring `|~~~|'
% which is used as a delimiter).
% Compilation is handed over to the new file by |\childdocforward|:
%    \begin{macrocode}
\newcommand{\childdocforwardprefix}[3][]
{
  \begingroup
    \def\childdocextract #2##1~~~{\def\childdoctmp{\childdocforward[#1]{#3##1}}}
    \expandafter\childdocextract\childdocname~~~
    \expandafter
  \endgroup
  \childdoctmp
}
%    \end{macrocode}

% \macro{\childdoc}
% The deprecated macro |\childdoc| is a legacy version of |\childdocmain|:
%    \begin{macrocode}
\newcommand{\childdoc}{\childdocmain}
%    \end{macrocode}

% \macro{\childdocredirect}
% The deprecated macro |\childdocredirect| is a legacy version
% of |\childdocforward| and |\childdocforwardprefix|:
%    \begin{macrocode}
\newcommand{\childdocredirect}[2][]
{
  \begingroup
    \if?#1?
      \def\childdoctmp{\childdocforward{#2}}
    \else
      \def\childdoctmp{\childdocforwardprefix{#1}{#2}}
    \fi
    \expandafter
  \endgroup
  \childdoctmp
}
%    \end{macrocode}

%\iffalse
%</package>
%\fi
%
\endinput
|\\
|\childdocforward{|\textit{main}|}|\\
\end{tabular}
\end{center}
%
or alternatively with:
%
\begin{center}
\begin{tabular}{l}
|% \iffalse
%
% childdoc.dtx Copyright (C) 2017-2018 Niklas Beisert
%
% This work may be distributed and/or modified under the
% conditions of the LaTeX Project Public License, either version 1.3
% of this license or (at your option) any later version.
% The latest version of this license is in
%   http://www.latex-project.org/lppl.txt
% and version 1.3 or later is part of all distributions of LaTeX
% version 2005/12/01 or later.
%
% This work has the LPPL maintenance status `maintained'.
%
% The Current Maintainer of this work is Niklas Beisert.
%
% This work consists of the files childdoc.dtx and childdoc.ins
% and the derived files childdoc.def and cdocsamp.tex with
% cdocsch1.tex, cdocsch2.tex, cdocsdrf.tex, cdocsfn1.tex, cdocsfn2.tex.
%
%<package>\ifdefined\childdocmain\endinput\fi
%<package>\ProvidesFile{childdoc.def}[2018/12/30 v2.0 child document driver]
%<samplemain>\ProvidesFile{cdocsamp.tex}[2018/12/30 v2.0 sample for childdoc]
%<*driver>
%\ProvidesFile{childdoc.drv}[2018/12/30 v2.0 childdoc reference manual file]
\PassOptionsToClass{10pt,a4paper}{article}
\documentclass{ltxdoc}

\usepackage[margin=35mm]{geometry}
\usepackage{hyperref}
\usepackage{hyperxmp}
\usepackage[usenames]{color}

\hypersetup{colorlinks=true}
\hypersetup{pdfstartview=FitH}
\hypersetup{pdfpagemode=UseNone}
\hypersetup{pdfsource={}}
\hypersetup{pdflang={en-UK}}
\hypersetup{pdfcopyright={Copyright 2017-2018 Niklas Beisert.
  This work may be distributed and/or modified under the
  conditions of the LaTeX Project Public License, either version 1.3
  of this license or (at your option) any later version.}}
\hypersetup{pdflicenseurl={http://www.latex-project.org/lppl.txt}}
\hypersetup{pdfcontactaddress={ETH Zurich, ITP, HIT K,
  Wolfgang-Pauli-Strasse 27}}
\hypersetup{pdfcontactpostcode={8093}}
\hypersetup{pdfcontactcity={Zurich}}
\hypersetup{pdfcontactcountry={Switzerland}}
\hypersetup{pdfcontactemail={nbeisert@itp.phys.ethz.ch}}
\hypersetup{pdfcontacturl={http://people.phys.ethz.ch/\xmptilde nbeisert/}}

\newcommand{\secref}[1]{\hyperref[#1]{section \ref*{#1}}}

\parskip1ex
\parindent0pt
\let\olditemize\itemize
\def\itemize{\olditemize\parskip0pt}

\begin{document}

\title{The \textsf{childdoc} Package}
\hypersetup{pdftitle={The childdoc Package}}
\author{Niklas Beisert\\[2ex]
  Institut f\"ur Theoretische Physik\\
  Eidgen\"ossische Technische Hochschule Z\"urich\\
  Wolfgang-Pauli-Strasse 27, 8093 Z\"urich, Switzerland\\[1ex]
  \href{mailto:nbeisert@itp.phys.ethz.ch}
  {\texttt{nbeisert@itp.phys.ethz.ch}}}
\hypersetup{pdfauthor={Niklas Beisert}}
\hypersetup{pdfsubject={Manual for the LaTeX2e Package childdoc}}
\date{30 December 2018, \textsf{v2.0}}
\maketitle

\begin{abstract}\noindent
\textsf{childdoc} is a \LaTeXe{} package
that enables the direct compilation
of document sections included by |\include|
to individual files.
\end{abstract}

\begingroup
\parskip0ex
\tableofcontents
\endgroup

%%%%%%%%%%%%%%%%%%%%%%%%%%%%%%%%%%%%%%%%%%%%%%%%%%%%%%%%%%%%%%%%%%%%%%%%%%%%%%%%
%%%%%%%%%%%%%%%%%%%%%%%%%%%%%%%%%%%%%%%%%%%%%%%%%%%%%%%%%%%%%%%%%%%%%%%%%%%%%%%%
\section{Introduction}

\LaTeX{} provides a mechanism to structure a large document (such as a book)
into a main file and several child files (containing the chapters)
using the |\include| command.
This mechanism is beneficial for documents
which span hundreds of pages in order to
make the source file(s) more manageable.
Moreover, compilation can be restricted to
selected child files by means of the |\includeonly| command.
The latter feature can be used to reduce the compilation time while editing
(this was significantly more useful in the earlier days of \LaTeX{})
or to generate a smaller document which is easier to navigate.
Another application of |\includeonly| is to generate
documents consisting of selected parts of the complete document.

However, there are a few drawbacks of the plain |\include| mechanism:
\begin{itemize}
\item
The child files cannot be compiled on their own,
they can only be compiled via the main file.
A naive editing environment
(such as a text editor with an option
to have the current file processed by \LaTeX)
may require one to switch to the main file before compiling;
attempting to compile the child file produces errors.
\item
The main file must be modified (each time)
to adjust the |\includeonly| command
to the present needs. This easily leaves the main file in a messy state.
\item
The generated document will always carry the filename
of the main document. This is inconvenient if
several child files are to be compiled and
to be kept for distribution.
\end{itemize}

The present package provides a simple interface
to make child files individually compilable by \LaTeX{}.
Compiling a child file then has the same effect as compiling
the main file with an |\includeonly| command
to select the appropriate child.
Moreover the generated document will carry the name of the child
rather than the main file.
This resolves all three above issues.

This feature is meant to make the editing of books,
thesis documents and lecture notes somewhat more convenient.
However, the package can also be used efficiently for
composing a series of documents (such as exercise sheets)
which are typically distributed individually.
It then assists the author in generating the individual documents
(potentially in different versions)
as well as a document containing the collected series.
Another application is in developing style files
or other kinds of included material
where compilation of the style file could redirect
to a sample or test file.

%%%%%%%%%%%%%%%%%%%%%%%%%%%%%%%%%%%%%%%%%%%%%%%%%%%%%%%%%%%%%%%%%%%%%%%%%%%%%%%%
%%%%%%%%%%%%%%%%%%%%%%%%%%%%%%%%%%%%%%%%%%%%%%%%%%%%%%%%%%%%%%%%%%%%%%%%%%%%%%%%
\section{Usage}

First of all, the package \textsf{childdoc} is \emph{not} a standard
\LaTeXe{} |.sty| style file! Therefore it needs to be invoked in
a non-standard way.

%%%%%%%%%%%%%%%%%%%%%%%%%%%%%%%%%%%%%%%%%%%%%%%%%%%%%%%%%%%%%%%%%%%%%%%%%%%%%%%%
\subsection{Included Files}
\label{sec:include}

%%%%%%%%%%%%%%%%%%%%%%%%%%%%%%%%%%%%%%%%
\DescribeMacro{\childdocmain}
To use the package, add the commands
\begin{center}
\begin{tabular}{l}
|\input{childdoc.def}|\\
|\childdocmain{}|\\
\end{tabular}
\end{center}
at the very top of the main \LaTeX{} file,
in particular \emph{before} the |\documentclass| statement!
The argument of |\childdocmain| should be left empty
(but it must be present).

%%%%%%%%%%%%%%%%%%%%%%%%%%%%%%%%%%%%%%%%
\DescribeMacro{\childdocof}
Furthermore, add the commands
\begin{center}
\begin{tabular}{l}
|\input{childdoc.def}|\\
|\childdocof{|\textit{main}|}|\\
\end{tabular}
\end{center}
at the top of every child file \textit{child}
which is included by |\include{|\textit{child}|}|
from within the main file
(or at least for those files to be compiled individually).
The argument \textit{main} must be the filename of the main file.

There are a couple of
considerations in setting up the main and child documents:

%%%%%%%%%%%%%%%%%%%%%%%%%%%%%%%%%%%%%%%%
\paragraph{Restrictions.}

Please note the following restrictions:
\begin{itemize}
\item
|\childdocmain| must be called with one argument \textit{main}
to ensure compatibility with earlier version of the package.
It must either be empty (|\childdocmain{}|)
or precisely match the filename of the main file in which it is specified.
See \secref{sec:detection} for further information.
\item
The filename \textit{main} must be specified without the |.tex| extension.
\item
The filename \textit{main} is case sensitive
(even in case-insensitive file systems)
due to internal string comparison.
\item
The argument \textit{main} should be fully expanded, it cannot be a macro.
\item
Subdirectories and special characters should be avoided in filenames.
\item
The command |\childdocmain{|\textit{main}|}| must be followed by a whitespace.
It should not be followed immediately by another command
or by a comment mark `|%|'.
This is because the \TeX{} parser reads the token immediately following
the argument of |\childdocmain| and puts it
at the beginning of every child section;
however, a white\-space is ignored.
\end{itemize}

%%%%%%%%%%%%%%%%%%%%%%%%%%%%%%%%%%%%%%%%
\paragraph{Content of Main File.}

It is advisable to place all content in the child files included by |\include|.
Any output contained in the main file will appear in all child documents
unless suppressed manually;
it cannot be suppressed automatically by the |\includeonly| directive
and thus should normally be avoided.
A method to include some content in the main file
by means of conditional processing is described in \secref{sec:conditional}.

%%%%%%%%%%%%%%%%%%%%%%%%%%%%%%%%%%%%%%%%
\paragraph{Page Numbering.}

When only a part of the document is compiled,
the appropriate numbering of pages
(as well as other status parameters)
is determined from the |.aux| files.
The latter contain information from previous passes.
However this information needs to propagate through
all intermediate child documents.
Therefore the page numbering in child documents may well
be inconsistent until the complete document is compiled at least once.

A useful (if unconventional) way to always ensure a consistent
page numbering is to restart the numbering in each child document
and denote the pages by `\textit{child}|.|\textit{page}'
where \textit{child} represents the chapter/section number of the child file.
This can be achieved by the command
|\numberwithin{page}{|\textit{child}|}|
of the \textsf{amsmath} package
where \textit{child} can be |chapter| or |section|
depending on the chosen structuring.
Alternatively, one can modify the macro |\thepage| appropriately
and reset the counter |page| at the start of each child file.

%%%%%%%%%%%%%%%%%%%%%%%%%%%%%%%%%%%%%%%%%%%%%%%%%%%%%%%%%%%%%%%%%%%%%%%%%%%%%%%%
\subsection{Conditional Processing}
\label{sec:conditional}

The package provides a mechanism to compile different versions
of a document. To customise the versions further some conditional processing
can come in handy to distinguish which version is being compiled.
The package provides two macros to describe the compilation context:

%%%%%%%%%%%%%%%%%%%%%%%%%%%%%%%%%%%%%%%%
\DescribeMacro{\ifchilddoc}
The conditional |\ifchilddoc| distinguishes between the compilation of
child documents and the main document:
%
\begin{center}
|\ifchilddoc |\textit{child-code}| |[|\||else |\textit{main-code}]| \||fi|
\end{center}

%%%%%%%%%%%%%%%%%%%%%%%%%%%%%%%%%%%%%%%%
\DescribeMacro{\childdocname}
\DescribeMacro{\childdocjob}
The macro |\childdocname| contains the filename (without extension)
of the main or child file being processed.
Note that |\childdocjob| will always contain the name of the main file.

%%%%%%%%%%%%%%%%%%%%%%%%%%%%%%%%%%%%%%%%
\paragraph{Title Page.}

Conditional processing can be used to include a title or banner page
in the main document when proper precautions are taken.
Importantly, the code in the main file should ensure that the page counter
(as well as other status parameters which are stored in the |.aux| files)
takes the same value after the conditional processing.
Otherwise the page numbers may take divergent values
depending on which part is compiled.

For example, a title page could be declared by:
%
\begin{center}
\begin{tabular}{l}
|\ifchilddoc\||else|\\
|\addtocounter{page}{-1}|\\
\textit{code for title page}\\
|\newpage|\\
|\||fi|
\end{tabular}
\end{center}
%
A banner page for the child documents can be generated by:
%
\begin{center}
\begin{tabular}{l}
|\ifchilddoc|\\
|\addtocounter{page}{-1}|\\
\textit{code for banner page}\\
|\newpage|\\
|\||fi|
\end{tabular}
\end{center}
%
Here one could write a message such as:
\begin{center}
|This is the part \childdocname{} of \childdocjob{}.|
\end{center}

%%%%%%%%%%%%%%%%%%%%%%%%%%%%%%%%%%%%%%%%%%%%%%%%%%%%%%%%%%%%%%%%%%%%%%%%%%%%%%%%
\subsection{Flags}
\label{sec:flags}

The package makes it easy to generate different versions
of the main or child documents.
To this end compilation flags can be defined
and assigned different default values.
They will be particularly useful in conjunction
with the forwarding mechanism described in \secref{sec:forward}.

For example, it may be useful to have a flag |\version|
which can be set to |draft| or |final|.
The document source will contain some conditional code
depending on the value of |\version|.
Suppose further, the flag should default to |final| for the main file
and to |draft| for child files
which is a natural assignment for editing the document.
This is achieved by placing the following code
in the preamble of the main document
(below the |\childdocmain| directive):
%
\begin{center}
\begin{tabular}{l}
|\ifchilddoc|\\
|\providecommand{\version}{draft}|\\
|\||else|\\
|\providecommand{\version}{final}|\\
|\||fi|
\end{tabular}
\end{center}
%
The definition by |\providecommand| makes sure
that previous definitions are not overwritten.
Further statements |\providecommand{\version}{...}|
can thus be added before the above code to override it.

For the main file, one might add a line
(between |\childdocmain| and the above block)
%
\begin{center}
|%\ifchilddoc\||else\providecommand{\version}{draft}\||fi|
\end{center}
%
which can be uncommented to produce a draft version.
Likewise one can add a line to the very top of a child file
(above the |\childdocof{|\textit{main}|}| directive)
%
\begin{center}
|%\providecommand{\version}{final}|
\end{center}
%
which can be uncommented to produce the final version of this child document.

%%%%%%%%%%%%%%%%%%%%%%%%%%%%%%%%%%%%%%%%%%%%%%%%%%%%%%%%%%%%%%%%%%%%%%%%%%%%%%%%
\subsection{Forwarding}
\label{sec:forward}

Different versions of the main or child documents
using compilation flags as described in \secref{sec:flags}
can be (permanently) stored in different files
for convenient compilation, viewing and distribution.
To this end, the package defines a command
to pass on compilation to a different file:

%%%%%%%%%%%%%%%%%%%%%%%%%%%%%%%%%%%%%%%%
\DescribeMacro{\childdocforward}
The command |\childdocforward| redirects processing to
another source file:
%
\begin{center}
\begin{tabular}{l}
|\input{childdoc.def}|\\
|\childdocforward[|\textit{main}|]{|\textit{dest}|}|\\
\end{tabular}
\end{center}
%
The argument \textit{dest} is the destination file
(without extension).
It should be the main file or one of the child files.
Note that further \textsf{childdoc} directives
such as |\childdocof| and |\childdocforward|
in the indicated file will be processed in this form.
The optional argument \textit{main}
passes on directly to the main file \textit{main}
while pretending to compile the child \textit{dest}.
This form behaves as if \textit{dest}
issues |\childdocof{|\textit{main}|}| right away,
and no further \textsf{childdoc} directives will be processed.

%%%%%%%%%%%%%%%%%%%%%%%%%%%%%%%%%%%%%%%%
\DescribeMacro{\...prefix}
In the alternative form |\childdocforwardprefix|,
%
\begin{center}
\begin{tabular}{l}
|\input{childdoc.def}|\\
|\childdocforwardprefix[|\textit{main}|]{|\textit{prefix}|}{|\textit{dest}|}|
\end{tabular}
\end{center}
%
the destination file is determined by a pattern
depending on the current file:
To make this work, the current file must be called
`{\textit{prefix}\hspace{0.2em}\textit{suffix}}'
with \textit{prefix} matching precisely the argument.
Processing is then passed on to the file
`{\textit{dest}\hspace{0.2em}\textit{suffix}}'.
Surely, the same effect is achieved by
directly specifying the
argument `{\textit{dest}\hspace{0.2em}\textit{suffix}}'
in the first form.
However, that requires to set up a different file
for each child. With the alternative form of the command
all these files can have exactly the same content
which simplifies setting them up and maintaining them.

For example, the following file |draft.tex|
with a compilation flag |\version| as described in \secref{sec:flags}
compiles the main document as a draft:
%
\begin{center}
\begin{tabular}{l}
|\def\version{draft}|\\
|\input{childdoc.def}|\\
|\childdocforward{|\textit{main}|}|
\end{tabular}
\end{center}
%
Likewise, the following files |final|\textit{nn}|.tex|
compile the final version of the child document
|child|\textit{nn}|.tex|:
%
\begin{center}
\begin{tabular}{l}
|\def\version{final}|\\
|\input{childdoc.def}|\\
|\childdocforwardprefix{final}{child}|
\end{tabular}
\end{center}
%

Note that when several versions of a main file and/or of each child file
are to be generated, it may be convenient to set up a |Makefile| or
shell script to automatise the process.

%%%%%%%%%%%%%%%%%%%%%%%%%%%%%%%%%%%%%%%%%%%%%%%%%%%%%%%%%%%%%%%%%%%%%%%%%%%%%%%%
\subsection{Command Line Processing}
\label{sec:commandline}

The effect of redirection files can also be achieved by invoking
the \LaTeX{} compiler with a more elaborate command line.
Most conveniently this should be done as part
of a shell script or a |Makefile|.

When using \textsf{childdoc} in the main file, the following
command lines effectively perform a redirection
(note that depending on the shell being used,
backslashes may have to be doubled: `|\|' $\to$ `|\\|'):
%
\begin{center}
|... -jobname "|\textit{target}|" |\\|"|[\textit{flags}]%
|\input{childdoc.def}\childdocforward[|\textit{main}|]{|\textit{dest}|}"|
\end{center}
%
Here \textit{target} is the name of the output file,
\textit{main} is the name of the main file
and \textit{dest} is the name of the main or child file to be processed
(all filenames without extensions).
The optional argument \textit{main} can be omitted
if \textit{main} matches \textit{dest}.
Optionally, compilation \textit{flags} can be defined via |\def| commands.
This command line makes the \TeX{} engine believe
it is compiling the file \textit{target}
whose content is specified as the latter parameter.
The provided code then forwards the processing to
\textit{main} or \textit{dest} as described in \secref{sec:forward}.

%%%%%%%%%%%%%%%%%%%%%%%%%%%%%%%%%%%%%%%%%%%%%%%%%%%%%%%%%%%%%%%%%%%%%%%%%%%%%%%%
\subsection{Include by Input}
\label{sec:input}

Including child documents by |\include| has some restrictions by design.
Most notably, the content of a child document always occupies
its own set of pages; pages cannot be shared between child documents.
Usually, this behaviour makes perfect sense
because each child document contain an essential part of the document.
However, in some situations it may be desirable to compose
a document from a collection of parts
without having mandatory page breaks between then.
For this case, the package
provides a mechanism to include parts
by |\input| which can also be processed individually.
However, by construction this mechanism
requires manual handling of the content to be output.

%%%%%%%%%%%%%%%%%%%%%%%%%%%%%%%%%%%%%%%%
\DescribeMacro{\ifchilddocmanual}
The main file should be prepared as usual, see \secref{sec:include}.
However, the document body must make a distinction
between processing of an individual part and of the main document, e.g.:
%
\begin{center}
\begin{tabular}{l}
|\ifchilddocmanual|\\
|\input{\childdocname}|\\
|\||else|\\
\textit{document body with }|\input{|\textit{part}|}|\\
|\||fi|
\end{tabular}
\end{center}
%
The conditional |\ifchilddocmanual| is true whenever
a part to be included by |\input| is being compiled,
and the name of the part is stored in |\childdocname|.

%%%%%%%%%%%%%%%%%%%%%%%%%%%%%%%%%%%%%%%%
\DescribeMacro{\childdocby}
Each part to be included by |\input| should start with:
%
\begin{center}
\begin{tabular}{l}
|\input{childdoc.def}|\\
|\childdocby{|\textit{main}|}|\\
\end{tabular}
\end{center}
%
The directive |\childdocby| is similar to |\childdocof|
described in \secref{sec:include},
but the subsequent selection of content must be done manually.
To that end, both |\ifchilddoc| and |\ifchilddocmanual|
will be true upon processing of a part,
and the name of the part is stored in |\childdocname|.
Note that |\jobname| will be set to the filename of the current part
so that each part receives an individual |.aux| file
that does not interfere with the |.aux| file(s) of the main document.
This behaviour can be altered by the alternative form
|\childdocby[*]{|\textit{main}|}| (with a non-empty optional argument)
which uses the |.aux| file of the main document
by setting |\jobname| to \textit{main}.

%%%%%%%%%%%%%%%%%%%%%%%%%%%%%%%%%%%%%%%%%%%%%%%%%%%%%%%%%%%%%%%%%%%%%%%%%%%%%%%%
\subsection{Driver Development}
\label{sec:driver}

The \textsf{childdoc} mechanism can also be use for the development
of definition files such as \LaTeX{} styles or classes.
This case differs from the above setup with multiple parts
included by |\include| in that no |\includeonly| should be invoked.
This can be achieved by starting the include file
(before |\ProvidesPackage|) with:
%
\begin{center}
\begin{tabular}{l}
|\input{childdoc.def}|\\
|\childdocforward{|\textit{main}|}|\\
\end{tabular}
\end{center}
%
or alternatively with:
%
\begin{center}
\begin{tabular}{l}
|\input{childdoc.def}|\\
|\childdocby{|\textit{main}|}|\\
\end{tabular}
\end{center}
%
Both forms have slightly different effects as described above.
The main file is prepared as usual, see \secref{sec:include}.

%%%%%%%%%%%%%%%%%%%%%%%%%%%%%%%%%%%%%%%%%%%%%%%%%%%%%%%%%%%%%%%%%%%%%%%%%%%%%%%%
\subsection{Legacy Detection}
\label{sec:detection}

The directive |\childdocmain| in the main file can detect
whether the complete document or merely a child is to be compiled
even without using the directive |\childdocof|.
This method is deprecated because it is less robust
and there is no compelling reason to use it;
it is merely provided for backward compatibility
and it may be removed in future versions.

If the detection mechanism is to be used,
it is mandatory to correctly specify
the filename of the main file as the argument of |\childdocmain|:
%
\begin{center}
\begin{tabular}{l}
|\input{childdoc.def}|\\
|\childdocmain{|\textit{main}|}|\\
\end{tabular}
\end{center}
%
If |\jobname| does not match the argument \textit{main} of |\childdocmain|,
it is assumed that |\jobname| points to the child file to be compiled.
When using |\childdocmain| with the main file specified as argument,
it suffices to start a child file
with just |\input{|\textit{main}|}|
without loading of the package and using |\childdocof|.
If instead all processing is done
with the appropriate \textsf{childdoc} directives,
the argument of \textit{main} of |\childdocmain| can be empty.

An alternative version of the command line processing described
in \secref{sec:commandline} using the detection mechanism reads:
%
\begin{center}
|... -jobname "|\textit{target}|" "|[\textit{flags}]%
[|\def\jobname{|\textit{dest}|}|]|\input{|\textit{main}|}"|
\end{center}

%%%%%%%%%%%%%%%%%%%%%%%%%%%%%%%%%%%%%%%%%%%%%%%%%%%%%%%%%%%%%%%%%%%%%%%%%%%%%%%%
\subsection{Manual Code}
\label{sec:manual}

In case one cannot be certain whether the definitions file |childdoc.def|
is installed on the target \TeX{} distribution
and one prefers not to ship it,
it is conceivable to paste a few relevant commands into the sources.

To that end, drop all statements |\input{childdoc.def}|
and perform the replacements as outlined below.
Instead of |\childdocmain{|\textit{main}|}| add the following code
to the top of the main file:
%
\begin{center}
\begin{tabular}{l}
|\||ifdefined\childdocname\endinput\||fi\newif\ifchilddoc|\\
|\edef\childdocname{\scantokens\expandafter{\jobname\noexpand}}|\\
|\def\childdocmain{|\textit{main}|}\||ifx\childdocmain\childdocname\||else|\\
|\childdoctrue\includeonly{\childdocname}\let\jobname\childdocmain\||fi|\\
\end{tabular}
\end{center}
%
Instead of |\childdocof{|\textit{main}|}| just include the main file
at the top of each child file:
%
\begin{center}
|\input{|\textit{main}|}|
\end{center}
%
A simple redirection |\childdocforward{|\textit{dest}|}| is achieved by:
%
\begin{center}
|\def\jobname{|\textit{dest}|}\input{\jobname}|
\end{center}
%
The redirection with prefix
|\childdocforwardprefix[|\textit{prefix}|]{|\textit{dest}|}|
is accomplished by:
%
\begin{center}
\begin{tabular}{l}
|{\edef\jobname{\scantokens\expandafter{\jobname\noexpand}}|\\
|\def\redirectjob |\textit{prefix}|#1~~~{\gdef\jobname{|\textit{dest}|#1}}|\\
|\expandafter\redirectjob\jobname~~~}\input{\jobname}|
\end{tabular}
\end{center}

In an alternative approach,
child documents can be compiled by a specific command line
without additional code or specific definitions:
%
\begin{center}
|... -jobname "|\textit{target}|" "|[\textit{flags}]%
|\includeonly{|\textit{dest}|}\input{|\textit{main}|}"|
\end{center}
%

%%%%%%%%%%%%%%%%%%%%%%%%%%%%%%%%%%%%%%%%%%%%%%%%%%%%%%%%%%%%%%%%%%%%%%%%%%%%%%%%
%%%%%%%%%%%%%%%%%%%%%%%%%%%%%%%%%%%%%%%%%%%%%%%%%%%%%%%%%%%%%%%%%%%%%%%%%%%%%%%%
\section{Information}

%%%%%%%%%%%%%%%%%%%%%%%%%%%%%%%%%%%%%%%%%%%%%%%%%%%%%%%%%%%%%%%%%%%%%%%%%%%%%%%%
\subsection{Copyright}

Copyright \copyright{} 2017--2018 Niklas Beisert

This work may be distributed and/or modified under the
conditions of the \LaTeX{} Project Public License, either version 1.3
of this license or (at your option) any later version.
The latest version of this license is in
  \url{http://www.latex-project.org/lppl.txt}
and version 1.3 or later is part of all distributions of \LaTeX{}
version 2005/12/01 or later.

This work has the LPPL maintenance status `maintained'.

The Current Maintainer of this work is Niklas Beisert.

This work consists of the files |README.txt|, |childdoc.ins| and |childdoc.dtx|
as well as the derived files |childdoc.def|, |cdocsamp.tex|
with |cdocsch1.tex|, |cdocsch2.tex|, |cdocspt3.tex|, |cdocspt4.tex|,
|cdocsdrf.tex|, |cdocsfn1.tex|, |cdocsfn2.tex|
as well as |childdoc.pdf|.

%%%%%%%%%%%%%%%%%%%%%%%%%%%%%%%%%%%%%%%%%%%%%%%%%%%%%%%%%%%%%%%%%%%%%%%%%%%%%%%%
\subsection{Files and Installation}

The package consists of the files:
%
\begin{center}
\begin{tabular}{ll}
    |README.txt|   & readme file \\
    |childdoc.ins| & installation file \\
    |childdoc.dtx| & source file \\
    |childdoc.def| & definition file \\
    |cdocsamp.tex| & sample main file \\
    |cdocsch1.tex| & sample include file \\
    |cdocsch2.tex| & sample include file \\
    |cdocspt3.tex| & sample part file \\
    |cdocspt4.tex| & sample part file \\
    |cdocsdrf.tex| & sample redirection file \\
    |cdocsfn1.tex| & sample redirection file \\
    |cdocsfn2.tex| & sample redirection file \\
    |childdoc.pdf| & manual
\end{tabular}
\end{center}
%
The distribution consists of the files
|README.txt|, |childdoc.ins| and |childdoc.dtx|.
%
\begin{itemize}
\item
Run (pdf)\LaTeX{} on |childdoc.dtx|
to compile the manual |childdoc.pdf| (this file).
\item
Run \LaTeX{} on |childdoc.ins| to create the definitions file |childdoc.def|
and the sample |cdocsamp.tex| with include files
|cdocsch1.tex|, |cdocsch2.tex|, |cdocspt3.tex|, |cdocspt4.tex|,
|cdocsdrf.tex|, |cdocsfn1.tex|, |cdocsfn2.tex|.
Then copy the file |childdoc.def| to an appropriate directory of your \LaTeX{}
distribution, e.g.\ \textit{texmf-root}|/tex/latex/childdoc|.
\end{itemize}

%%%%%%%%%%%%%%%%%%%%%%%%%%%%%%%%%%%%%%%%%%%%%%%%%%%%%%%%%%%%%%%%%%%%%%%%%%%%%%%%
\subsection{Related CTAN Packages}

There are several other packages which offer a similar functionality:
%
\begin{itemize}
\item
The packages
\href{http://ctan.org/pkg/docmute}{\textsf{docmute}},
\href{http://ctan.org/pkg/includex}{\textsf{includex}} and
\href{http://ctan.org/pkg/standalone}{\textsf{standalone}}
provide commands to include only the document body of
a child file thus allowing both files to be compiled individually.
\item
The packages \href{http://ctan.org/pkg/subdocs}{\textsf{subdocs}}
and \href{http://ctan.org/pkg/subfiles}{\textsf{subfiles}}
provide structures in which the main and child documents can be
encapsulated and allowing them to be compiled individually.
The inclusion mechanism is different from the conventional |\include|.
\item
The package \href{http://ctan.org/pkg/combine}{\textsf{combine}}
is an elaborate solution to combine several documents into one.
\end{itemize}
%
See also the CTAN topic \href{http://ctan.org/topic/subdocs}{\textsf{subdocs}}
for further related packages.
The present package differs from the above solutions in that
a document structure constructed with the conventional |\include| mechanism
just needs two extra commands at the top of every file
such that all constituent files can be compiled individually.

%%%%%%%%%%%%%%%%%%%%%%%%%%%%%%%%%%%%%%%%%%%%%%%%%%%%%%%%%%%%%%%%%%%%%%%%%%%%%%%%
%\subsection{Feature Suggestions}
%
%The following is a list of features which may be useful for future
%versions of this package:
%%
%\begin{itemize}
%\item
%\ldots
%\end{itemize}

%%%%%%%%%%%%%%%%%%%%%%%%%%%%%%%%%%%%%%%%%%%%%%%%%%%%%%%%%%%%%%%%%%%%%%%%%%%%%%%%
\subsection{Revision History}

%%%%%%%%%%%%%%%%%%%%%%%%%%%%%%%%%%%%%%%%
\paragraph{v2.0:} 2018/12/30

\begin{itemize}
\item
immediate forward processing
\item
added |\childdocby| mechanism
\item
manual restructured
\end{itemize}

%%%%%%%%%%%%%%%%%%%%%%%%%%%%%%%%%%%%%%%%
\paragraph{v1.6:} 2018/01/17

\begin{itemize}
\item
application for development of include files
\item
corrections to manual
\end{itemize}

%%%%%%%%%%%%%%%%%%%%%%%%%%%%%%%%%%%%%%%%
\paragraph{v1.5:} 2017/05/21

\begin{itemize}
\item
more complete structuring introduced
\item
|\childdocof| introduced
\item
|\childdoc| renamed to |\childdocmain|
\item
|\childredirect| renamed to |\childdocforward| and |\childdocforwardprefix|
and functionality expanded
\end{itemize}

%%%%%%%%%%%%%%%%%%%%%%%%%%%%%%%%%%%%%%%%
\paragraph{v1.0:} 2017/04/27

\begin{itemize}
\item
manual and install package
\item
first version published on CTAN
\end{itemize}

%%%%%%%%%%%%%%%%%%%%%%%%%%%%%%%%%%%%%%%%
\paragraph{v0.6:} 2017/04/26

\begin{itemize}
\item
redirection mechanism added
\end{itemize}

%%%%%%%%%%%%%%%%%%%%%%%%%%%%%%%%%%%%%%%%
\paragraph{v0.5:} 2017/04/26

\begin{itemize}
\item
functionality in definition file
\end{itemize}


%%%%%%%%%%%%%%%%%%%%%%%%%%%%%%%%%%%%%%%%%%%%%%%%%%%%%%%%%%%%%%%%%%%%%%%%%%%%%%%%
%%%%%%%%%%%%%%%%%%%%%%%%%%%%%%%%%%%%%%%%%%%%%%%%%%%%%%%%%%%%%%%%%%%%%%%%%%%%%%%%
%%%%%%%%%%%%%%%%%%%%%%%%%%%%%%%%%%%%%%%%%%%%%%%%%%%%%%%%%%%%%%%%%%%%%%%%%%%%%%%%
\appendix

\settowidth\MacroIndent{\rmfamily\scriptsize 000\ }

 \DocInput{childdoc.dtx}

\end{document}
%</driver>
% \fi
%
% %%%%%%%%%%%%%%%%%%%%%%%%%%%%%%%%%%%%%%%%%%%%%%%%%%%%%%%%%%%%%%%%%%%%%%%%%%%%%%
% %%%%%%%%%%%%%%%%%%%%%%%%%%%%%%%%%%%%%%%%%%%%%%%%%%%%%%%%%%%%%%%%%%%%%%%%%%%%%%
% \section{Sample}
%\iffalse
%<*samplemain>
%\fi
%
% The following presents a sample document
% with two chapters, two parts, a title page,
% a compile flag as well as three forwarding files to set the flag.
% It consists of eight |.tex| files:
% \begin{center}
% \begin{tabular}{ll}
% |cdocsamp.tex|&main file\\
% |cdocsch1.tex|&include file for chapter 1\\
% |cdocsch2.tex|&include file for chapter 2\\
% |cdocspt3.tex|&include file for part 3\\
% |cdocspt4.tex|&include file for part 4\\
% |cdocsdrf.tex|&forwarding file for main file in draft mode\\
% |cdocsfi1.tex|&forwarding file for final version of chapter 1\\
% |cdocsfi2.tex|&forwarding file for final version of chapter 2\\
% \end{tabular}
% \end{center}
% Each of the eight files can be compiled directly by the \LaTeX{} compiler.
%
% %%%%%%%%%%%%%%%%%%%%%%%%%%%%%%%%%%%%%%
% \paragraph{Main File.}
%
% The main file is called |cdocsamp.tex|.
%
% Load the \textsf{childdoc} definitions and
% declare the filename for the main document:
%    \begin{macrocode}
\input{childdoc.def}
\childdocmain{}
%    \end{macrocode}

% Optional override for |\version| flag:
%    \begin{macrocode}
%%\ifchilddoc\else\providecommand{\version}{draft}\fi
%    \end{macrocode}

% Define the default values for the |\version| flag
% (|final| for the main file and |draft| for childs):
%    \begin{macrocode}
\ifchilddoc
\providecommand{\version}{draft}
\else
\providecommand{\version}{final}
\fi
%    \end{macrocode}

% Load the standard document class:
%    \begin{macrocode}
\documentclass[12pt]{article}
%    \end{macrocode}

% Start the document body:
%    \begin{macrocode}
\begin{document}
%    \end{macrocode}

% Declare a title page.
% Print title, part of document being processed and version flag:
%    \begin{macrocode}
\addtocounter{page}{-1}
\begin{center}
{\LARGE\bfseries{}childdoc example\par}
\vspace{1cm}
\ifchilddoc
\ifchilddocmanual part\else chapter\fi:
`\childdocname' of `\childdocjob'\par
\else
main document: `\childdocjob'\par
\fi
version: \version\par
\end{center}
\newpage
%    \end{macrocode}

% Manually include selected file,
% otherwise process as usual:
%    \begin{macrocode}
\ifchilddocmanual
\section*{part `\childdocname'}
\input{\childdocname}
\else
%    \end{macrocode}

% Include the two chapters:
%    \begin{macrocode}
\include{cdocsch1}
\include{cdocsch2}
%    \end{macrocode}

% Include the two parts unless only chapters should be displayed:
%    \begin{macrocode}
\ifchilddoc\else
\section{part three}
\input{cdocspt3}
\section{part four}
\input{cdocspt4}
\fi
%    \end{macrocode}

% Process as usual until here:
%    \begin{macrocode}
\fi
%    \end{macrocode}

% End of document body:
%    \begin{macrocode}
\end{document}
%    \end{macrocode}
%\iffalse
%</samplemain>
%\fi
%
% %%%%%%%%%%%%%%%%%%%%%%%%%%%%%%%%%%%%%%
% \paragraph{Chapter Include Files.}
%
% The include files are called |cdocsch1.tex| and |cdocsch2.tex|.
%
%\iffalse
%<*samplechap1|samplechap2>
%\fi

% Optional override for |\version| flag:
%    \begin{macrocode}
%%\providecommand{\version}{final}
%    \end{macrocode}

% Include the main document:
%    \begin{macrocode}
\input{childdoc.def}
\childdocof{cdocsamp}
%    \end{macrocode}

%\iffalse
%</samplechap1|samplechap2>
%\fi
%
%\iffalse
%<*samplechap1>
%\fi
% Some text for chapter 1:
%    \begin{macrocode}
\section{one}
some text in chapter one
%    \end{macrocode}

%\iffalse
%</samplechap1>
%\fi
% Some text for chapter 2:
%\iffalse
%<*samplechap2>
%\fi
%    \begin{macrocode}
\section{two}
more text in chapter two
%    \end{macrocode}

%\iffalse
%</samplechap2>
%\fi
%
% %%%%%%%%%%%%%%%%%%%%%%%%%%%%%%%%%%%%%%
% \paragraph{Part Include Files.}
%
% The include files are called |cdocspt3.tex| and |cdocspt4.tex|.
%
%\iffalse
%<*samplepart3|samplepart4>
%\fi

% Optional override for |\version| flag:
%    \begin{macrocode}
%%\providecommand{\version}{final}
%    \end{macrocode}

% Include the main document:
%    \begin{macrocode}
\input{childdoc.def}
\childdocby{cdocsamp}
%    \end{macrocode}

%\iffalse
%</samplepart3|samplepart4>
%\fi
%
%\iffalse
%<*samplepart3>
%\fi
% Some text for part 3:
%    \begin{macrocode}
some text in part three
%    \end{macrocode}

%\iffalse
%</samplepart3>
%\fi
% Some text for part 4:
%\iffalse
%<*samplepart4>
%\fi
%    \begin{macrocode}
more text in part four
%    \end{macrocode}

%\iffalse
%</samplepart4>
%\fi
%
% %%%%%%%%%%%%%%%%%%%%%%%%%%%%%%%%%%%%%%
% \paragraph{Forwarding for a Complete Draft.}
%
% The following forwarding file |cdocsdrf.tex|
% compiles the main document in draft mode:
%\iffalse
%<*sampledraft>
%\fi
%    \begin{macrocode}
\def\version{draft}
\input{childdoc.def}
\childdocforward{cdocsamp}
%    \end{macrocode}

%\iffalse
%</sampledraft>
%\fi
%
% %%%%%%%%%%%%%%%%%%%%%%%%%%%%%%%%%%%%%%
% \paragraph{Forwarding for Final Version of the Chapters.}
%
% The following forwarding files |cdocsfn1.tex| and |cdocsfn2.tex|
% (with identical content)
% compile the final versions of the child documents
% |cdocsch1.tex| and |cdocsch2.tex|, respectively:
%\iffalse
%<*samplefinal>
%\fi
%    \begin{macrocode}
\def\version{final}
\input{childdoc.def}
\childdocforwardprefix[cdocsamp]{cdocsfn}{cdocsch}
%    \end{macrocode}

%\iffalse
%</samplefinal>
%\fi
%
% %%%%%%%%%%%%%%%%%%%%%%%%%%%%%%%%%%%%%%
% \paragraph{Command Line Processing.}
%
% The following three command lines generate the output files
% |cdocscld|, |cdocscl1| and |cdocscl2|
% which should be identical to
% |cdocsdrf|, |cdocsch1| and |cdocsfn2|, respectively:
% \begin{center}
% \begin{tabular}{l}
% |latex -jobname cdocscld \|\\
% |  "\def\version{draft}\input{childdoc.def}\childdocforward{cdocsamp}"|\\
% |latex -jobname cdocscl1 \|\\
% |  "\input{childdoc.def}\childdocforward[cdocsamp]{cdocsch1}"|\\
% |latex -jobname cdocscl2 \|\\
% |  "\def\version{final}\input{childdoc.def}\childdocforward{cdocsch2}"|
% \end{tabular}
% \end{center}
% Note that the trailing backslash on each first line
% merely continues the input to the second line
% (for convenient cut ant paste).
% Furthermore, the command |latex| can be replaced by any
% of its alternative versions such as |pdflatex|.
%
% %%%%%%%%%%%%%%%%%%%%%%%%%%%%%%%%%%%%%%%%%%%%%%%%%%%%%%%%%%%%%%%%%%%%%%%%%%%%%%
% %%%%%%%%%%%%%%%%%%%%%%%%%%%%%%%%%%%%%%%%%%%%%%%%%%%%%%%%%%%%%%%%%%%%%%%%%%%%%%
% \section{Implementation}
%\iffalse
%<*package>
%\fi
%
% This section describes the definitions file |childdoc.def|.

% The definitions cannot be loaded using |\usepackage| or |\RequirePackage|
% which has a mechanism to prevent loading a style file more than once.
% When loading the definitions by means of |\input|
% multiple instances have to be prevented manually:
%\iffalse
%This code needs to be before the `\ProvidesFile' directive
%which is defined at the beginning of this file.
%Therefore it is also placed there and commented out here.
%</package>
%<*discard>
%\fi
%    \begin{macrocode}
\ifdefined\childdocmain\endinput\fi
%    \end{macrocode}
%\iffalse
%</discard>
%<*package>
%\fi
%
% \macro{\ifchilddoc}
% \macro{\ifchilddocmanual}
% The conditional |\ifchilddoc| tells whether a
% child (true) or main (false) document is being compiled.
% The conditional |\ifchilddocmanual| tells whether
% the |\includeonly| mechanism is used (false) or
% the selection of child files must be performed manually (true).
% The definitions initialise to false:
%    \begin{macrocode}
\newif\ifchilddoc
\newif\ifchilddocmanual
%    \end{macrocode}

% \macro{\childdocname}
% \macro{\childdocjob}
% The macro |\childdocname| stores the name of the main document
% to be compiled. The macro |\childdocjob| stores the name of
% the document on which the \LaTeX{} compiler was originally invoked.
% The content of |\jobname| cannot be compared
% to filenames specified in the source due to different catcodes.
% The following code rescans |\jobname|, stores the result
% in |\childdocname| and saves a copy in |\childdocjob|:
%    \begin{macrocode}
\edef\childdocname{\scantokens\expandafter{\jobname\noexpand}}
\let\childdocjob\childdocname
%    \end{macrocode}

% \macro{\childdocdisable}
% The macro |\childdocdisable| prevents the main file
% from being processed more than once.
% At this stage, the main document command |\childdocmain|
% is assumed to be called once again where it should do nothing.
% Any subsequent call to it should prevent
% a secondary processing of the main document
% It overwrites the forwarding commands
% |\childdocof| and |\childdocforward|
% with empty macros to prevent further inclusions of the main document:
%    \begin{macrocode}
\newcommand{\childdocdisable}
{
  \renewcommand{\childdocmain}[1]{\renewcommand{\childdocmain}[1]{\endinput}}
  \renewcommand{\childdocof}[1]{}
  \renewcommand{\childdocby}[2][]{}
  \renewcommand{\childdocforward}[2][]{}
  \renewcommand{\childdocdisable}{}
}
%    \end{macrocode}

% \macro{\childdocmain}
% The macro |\childdocmain| is to be called at the top of the main file
% with nothing or the main filename (without extension) as argument.
% First, it breaks loops.
% If the argument is not empty and does not match |\childdocname|
% (which is set by the first inclusion of |childdoc.def|),
% |\ifchilddoc| is set to true, |\includeonly| is applied to the child file
% and |\jobname| is set to the main file
% (for proper handling of |.aux| files):
%    \begin{macrocode}
\newcommand{\childdocmain}[1]
{
  \childdocdisable\childdocmain{}
  \if?#1?\else
    \begingroup
      \def\childdoctmp{#1}
      \ifx\childdoctmp\childdocname
        \def\childdoctmp{}
      \else
        \def\childdoctmp
        {
          \childdoctrue
          \includeonly{\childdocname}
          \def\childdocjob{#1}
          \def\jobname{#1}
        }
      \fi
      \expandafter
    \endgroup
    \childdoctmp
  \fi
}
%    \end{macrocode}

% \macro{\childdocof}
% The command |\childdocof| redirects
% compilation to the main file |#1|.
%    \begin{macrocode}
\newcommand{\childdocof}[1]
{
  \childdocdisable
  \childdoctrue
  \includeonly{\childdocname}
  \def\jobname{#1}
  \def\childdocjob{#1}
  \input{#1}
}
%    \end{macrocode}

% \macro{\childdocby}
% The command |\childdocby| ....
%    \begin{macrocode}
\newcommand{\childdocby}[2][]
{
  \childdocdisable
  \childdoctrue
  \childdocmanualtrue
  \if?#1?\else
    \def\jobname{#2}
  \fi
  \def\childdocjob{#2}
  \input{#2}
  \endinput
}
%    \end{macrocode}

% \macro{\childdocforward}
% The command |\childdocforward| redirects
% compilation to the main file or
% (if the optional argument is given) a child file.
% Parameters are set as if the main file
% or a child file starting with |\childdocof| was compiled.
% Then compilation is handed over to the main file:
%    \begin{macrocode}
\newcommand{\childdocforward}[2][]
{
  \begingroup
    \if?#1?
      \def\childdoctmp
      {
        \def\childdocname{#2}
        \def\childdocjob{#2}
        \def\jobname{#2}
        \input{#2}
        \endinput
      }
    \else
      \def\childdoctmp
      {
        \childdocdisable
        \def\childdocname{#2}
        \childdoctrue
        \includeonly{#2}
        \def\childdocjob{#1}
        \def\jobname{#1}
        \input{#1}
        \endinput
      }
    \fi
    \expandafter
  \endgroup
  \childdoctmp
}
%    \end{macrocode}

% \macro{\childdocforwardprefix}
% The command |\childdocforwardprefix| redirects
% compilation to the main or a child file by means of a pattern.
% The prefix |#1| in the current filename is replaced by |#2|
% and the suffix of the current filename is kept
% (it is assumed that the filename does not contain the substring `|~~~|'
% which is used as a delimiter).
% Compilation is handed over to the new file by |\childdocforward|:
%    \begin{macrocode}
\newcommand{\childdocforwardprefix}[3][]
{
  \begingroup
    \def\childdocextract #2##1~~~{\def\childdoctmp{\childdocforward[#1]{#3##1}}}
    \expandafter\childdocextract\childdocname~~~
    \expandafter
  \endgroup
  \childdoctmp
}
%    \end{macrocode}

% \macro{\childdoc}
% The deprecated macro |\childdoc| is a legacy version of |\childdocmain|:
%    \begin{macrocode}
\newcommand{\childdoc}{\childdocmain}
%    \end{macrocode}

% \macro{\childdocredirect}
% The deprecated macro |\childdocredirect| is a legacy version
% of |\childdocforward| and |\childdocforwardprefix|:
%    \begin{macrocode}
\newcommand{\childdocredirect}[2][]
{
  \begingroup
    \if?#1?
      \def\childdoctmp{\childdocforward{#2}}
    \else
      \def\childdoctmp{\childdocforwardprefix{#1}{#2}}
    \fi
    \expandafter
  \endgroup
  \childdoctmp
}
%    \end{macrocode}

%\iffalse
%</package>
%\fi
%
\endinput
|\\
|\childdocby{|\textit{main}|}|\\
\end{tabular}
\end{center}
%
Both forms have slightly different effects as described above.
The main file is prepared as usual, see \secref{sec:include}.

%%%%%%%%%%%%%%%%%%%%%%%%%%%%%%%%%%%%%%%%%%%%%%%%%%%%%%%%%%%%%%%%%%%%%%%%%%%%%%%%
\subsection{Legacy Detection}
\label{sec:detection}

The directive |\childdocmain| in the main file can detect
whether the complete document or merely a child is to be compiled
even without using the directive |\childdocof|.
This method is deprecated because it is less robust
and there is no compelling reason to use it;
it is merely provided for backward compatibility
and it may be removed in future versions.

If the detection mechanism is to be used,
it is mandatory to correctly specify
the filename of the main file as the argument of |\childdocmain|:
%
\begin{center}
\begin{tabular}{l}
|% \iffalse
%
% childdoc.dtx Copyright (C) 2017-2018 Niklas Beisert
%
% This work may be distributed and/or modified under the
% conditions of the LaTeX Project Public License, either version 1.3
% of this license or (at your option) any later version.
% The latest version of this license is in
%   http://www.latex-project.org/lppl.txt
% and version 1.3 or later is part of all distributions of LaTeX
% version 2005/12/01 or later.
%
% This work has the LPPL maintenance status `maintained'.
%
% The Current Maintainer of this work is Niklas Beisert.
%
% This work consists of the files childdoc.dtx and childdoc.ins
% and the derived files childdoc.def and cdocsamp.tex with
% cdocsch1.tex, cdocsch2.tex, cdocsdrf.tex, cdocsfn1.tex, cdocsfn2.tex.
%
%<package>\ifdefined\childdocmain\endinput\fi
%<package>\ProvidesFile{childdoc.def}[2018/12/30 v2.0 child document driver]
%<samplemain>\ProvidesFile{cdocsamp.tex}[2018/12/30 v2.0 sample for childdoc]
%<*driver>
%\ProvidesFile{childdoc.drv}[2018/12/30 v2.0 childdoc reference manual file]
\PassOptionsToClass{10pt,a4paper}{article}
\documentclass{ltxdoc}

\usepackage[margin=35mm]{geometry}
\usepackage{hyperref}
\usepackage{hyperxmp}
\usepackage[usenames]{color}

\hypersetup{colorlinks=true}
\hypersetup{pdfstartview=FitH}
\hypersetup{pdfpagemode=UseNone}
\hypersetup{pdfsource={}}
\hypersetup{pdflang={en-UK}}
\hypersetup{pdfcopyright={Copyright 2017-2018 Niklas Beisert.
  This work may be distributed and/or modified under the
  conditions of the LaTeX Project Public License, either version 1.3
  of this license or (at your option) any later version.}}
\hypersetup{pdflicenseurl={http://www.latex-project.org/lppl.txt}}
\hypersetup{pdfcontactaddress={ETH Zurich, ITP, HIT K,
  Wolfgang-Pauli-Strasse 27}}
\hypersetup{pdfcontactpostcode={8093}}
\hypersetup{pdfcontactcity={Zurich}}
\hypersetup{pdfcontactcountry={Switzerland}}
\hypersetup{pdfcontactemail={nbeisert@itp.phys.ethz.ch}}
\hypersetup{pdfcontacturl={http://people.phys.ethz.ch/\xmptilde nbeisert/}}

\newcommand{\secref}[1]{\hyperref[#1]{section \ref*{#1}}}

\parskip1ex
\parindent0pt
\let\olditemize\itemize
\def\itemize{\olditemize\parskip0pt}

\begin{document}

\title{The \textsf{childdoc} Package}
\hypersetup{pdftitle={The childdoc Package}}
\author{Niklas Beisert\\[2ex]
  Institut f\"ur Theoretische Physik\\
  Eidgen\"ossische Technische Hochschule Z\"urich\\
  Wolfgang-Pauli-Strasse 27, 8093 Z\"urich, Switzerland\\[1ex]
  \href{mailto:nbeisert@itp.phys.ethz.ch}
  {\texttt{nbeisert@itp.phys.ethz.ch}}}
\hypersetup{pdfauthor={Niklas Beisert}}
\hypersetup{pdfsubject={Manual for the LaTeX2e Package childdoc}}
\date{30 December 2018, \textsf{v2.0}}
\maketitle

\begin{abstract}\noindent
\textsf{childdoc} is a \LaTeXe{} package
that enables the direct compilation
of document sections included by |\include|
to individual files.
\end{abstract}

\begingroup
\parskip0ex
\tableofcontents
\endgroup

%%%%%%%%%%%%%%%%%%%%%%%%%%%%%%%%%%%%%%%%%%%%%%%%%%%%%%%%%%%%%%%%%%%%%%%%%%%%%%%%
%%%%%%%%%%%%%%%%%%%%%%%%%%%%%%%%%%%%%%%%%%%%%%%%%%%%%%%%%%%%%%%%%%%%%%%%%%%%%%%%
\section{Introduction}

\LaTeX{} provides a mechanism to structure a large document (such as a book)
into a main file and several child files (containing the chapters)
using the |\include| command.
This mechanism is beneficial for documents
which span hundreds of pages in order to
make the source file(s) more manageable.
Moreover, compilation can be restricted to
selected child files by means of the |\includeonly| command.
The latter feature can be used to reduce the compilation time while editing
(this was significantly more useful in the earlier days of \LaTeX{})
or to generate a smaller document which is easier to navigate.
Another application of |\includeonly| is to generate
documents consisting of selected parts of the complete document.

However, there are a few drawbacks of the plain |\include| mechanism:
\begin{itemize}
\item
The child files cannot be compiled on their own,
they can only be compiled via the main file.
A naive editing environment
(such as a text editor with an option
to have the current file processed by \LaTeX)
may require one to switch to the main file before compiling;
attempting to compile the child file produces errors.
\item
The main file must be modified (each time)
to adjust the |\includeonly| command
to the present needs. This easily leaves the main file in a messy state.
\item
The generated document will always carry the filename
of the main document. This is inconvenient if
several child files are to be compiled and
to be kept for distribution.
\end{itemize}

The present package provides a simple interface
to make child files individually compilable by \LaTeX{}.
Compiling a child file then has the same effect as compiling
the main file with an |\includeonly| command
to select the appropriate child.
Moreover the generated document will carry the name of the child
rather than the main file.
This resolves all three above issues.

This feature is meant to make the editing of books,
thesis documents and lecture notes somewhat more convenient.
However, the package can also be used efficiently for
composing a series of documents (such as exercise sheets)
which are typically distributed individually.
It then assists the author in generating the individual documents
(potentially in different versions)
as well as a document containing the collected series.
Another application is in developing style files
or other kinds of included material
where compilation of the style file could redirect
to a sample or test file.

%%%%%%%%%%%%%%%%%%%%%%%%%%%%%%%%%%%%%%%%%%%%%%%%%%%%%%%%%%%%%%%%%%%%%%%%%%%%%%%%
%%%%%%%%%%%%%%%%%%%%%%%%%%%%%%%%%%%%%%%%%%%%%%%%%%%%%%%%%%%%%%%%%%%%%%%%%%%%%%%%
\section{Usage}

First of all, the package \textsf{childdoc} is \emph{not} a standard
\LaTeXe{} |.sty| style file! Therefore it needs to be invoked in
a non-standard way.

%%%%%%%%%%%%%%%%%%%%%%%%%%%%%%%%%%%%%%%%%%%%%%%%%%%%%%%%%%%%%%%%%%%%%%%%%%%%%%%%
\subsection{Included Files}
\label{sec:include}

%%%%%%%%%%%%%%%%%%%%%%%%%%%%%%%%%%%%%%%%
\DescribeMacro{\childdocmain}
To use the package, add the commands
\begin{center}
\begin{tabular}{l}
|\input{childdoc.def}|\\
|\childdocmain{}|\\
\end{tabular}
\end{center}
at the very top of the main \LaTeX{} file,
in particular \emph{before} the |\documentclass| statement!
The argument of |\childdocmain| should be left empty
(but it must be present).

%%%%%%%%%%%%%%%%%%%%%%%%%%%%%%%%%%%%%%%%
\DescribeMacro{\childdocof}
Furthermore, add the commands
\begin{center}
\begin{tabular}{l}
|\input{childdoc.def}|\\
|\childdocof{|\textit{main}|}|\\
\end{tabular}
\end{center}
at the top of every child file \textit{child}
which is included by |\include{|\textit{child}|}|
from within the main file
(or at least for those files to be compiled individually).
The argument \textit{main} must be the filename of the main file.

There are a couple of
considerations in setting up the main and child documents:

%%%%%%%%%%%%%%%%%%%%%%%%%%%%%%%%%%%%%%%%
\paragraph{Restrictions.}

Please note the following restrictions:
\begin{itemize}
\item
|\childdocmain| must be called with one argument \textit{main}
to ensure compatibility with earlier version of the package.
It must either be empty (|\childdocmain{}|)
or precisely match the filename of the main file in which it is specified.
See \secref{sec:detection} for further information.
\item
The filename \textit{main} must be specified without the |.tex| extension.
\item
The filename \textit{main} is case sensitive
(even in case-insensitive file systems)
due to internal string comparison.
\item
The argument \textit{main} should be fully expanded, it cannot be a macro.
\item
Subdirectories and special characters should be avoided in filenames.
\item
The command |\childdocmain{|\textit{main}|}| must be followed by a whitespace.
It should not be followed immediately by another command
or by a comment mark `|%|'.
This is because the \TeX{} parser reads the token immediately following
the argument of |\childdocmain| and puts it
at the beginning of every child section;
however, a white\-space is ignored.
\end{itemize}

%%%%%%%%%%%%%%%%%%%%%%%%%%%%%%%%%%%%%%%%
\paragraph{Content of Main File.}

It is advisable to place all content in the child files included by |\include|.
Any output contained in the main file will appear in all child documents
unless suppressed manually;
it cannot be suppressed automatically by the |\includeonly| directive
and thus should normally be avoided.
A method to include some content in the main file
by means of conditional processing is described in \secref{sec:conditional}.

%%%%%%%%%%%%%%%%%%%%%%%%%%%%%%%%%%%%%%%%
\paragraph{Page Numbering.}

When only a part of the document is compiled,
the appropriate numbering of pages
(as well as other status parameters)
is determined from the |.aux| files.
The latter contain information from previous passes.
However this information needs to propagate through
all intermediate child documents.
Therefore the page numbering in child documents may well
be inconsistent until the complete document is compiled at least once.

A useful (if unconventional) way to always ensure a consistent
page numbering is to restart the numbering in each child document
and denote the pages by `\textit{child}|.|\textit{page}'
where \textit{child} represents the chapter/section number of the child file.
This can be achieved by the command
|\numberwithin{page}{|\textit{child}|}|
of the \textsf{amsmath} package
where \textit{child} can be |chapter| or |section|
depending on the chosen structuring.
Alternatively, one can modify the macro |\thepage| appropriately
and reset the counter |page| at the start of each child file.

%%%%%%%%%%%%%%%%%%%%%%%%%%%%%%%%%%%%%%%%%%%%%%%%%%%%%%%%%%%%%%%%%%%%%%%%%%%%%%%%
\subsection{Conditional Processing}
\label{sec:conditional}

The package provides a mechanism to compile different versions
of a document. To customise the versions further some conditional processing
can come in handy to distinguish which version is being compiled.
The package provides two macros to describe the compilation context:

%%%%%%%%%%%%%%%%%%%%%%%%%%%%%%%%%%%%%%%%
\DescribeMacro{\ifchilddoc}
The conditional |\ifchilddoc| distinguishes between the compilation of
child documents and the main document:
%
\begin{center}
|\ifchilddoc |\textit{child-code}| |[|\||else |\textit{main-code}]| \||fi|
\end{center}

%%%%%%%%%%%%%%%%%%%%%%%%%%%%%%%%%%%%%%%%
\DescribeMacro{\childdocname}
\DescribeMacro{\childdocjob}
The macro |\childdocname| contains the filename (without extension)
of the main or child file being processed.
Note that |\childdocjob| will always contain the name of the main file.

%%%%%%%%%%%%%%%%%%%%%%%%%%%%%%%%%%%%%%%%
\paragraph{Title Page.}

Conditional processing can be used to include a title or banner page
in the main document when proper precautions are taken.
Importantly, the code in the main file should ensure that the page counter
(as well as other status parameters which are stored in the |.aux| files)
takes the same value after the conditional processing.
Otherwise the page numbers may take divergent values
depending on which part is compiled.

For example, a title page could be declared by:
%
\begin{center}
\begin{tabular}{l}
|\ifchilddoc\||else|\\
|\addtocounter{page}{-1}|\\
\textit{code for title page}\\
|\newpage|\\
|\||fi|
\end{tabular}
\end{center}
%
A banner page for the child documents can be generated by:
%
\begin{center}
\begin{tabular}{l}
|\ifchilddoc|\\
|\addtocounter{page}{-1}|\\
\textit{code for banner page}\\
|\newpage|\\
|\||fi|
\end{tabular}
\end{center}
%
Here one could write a message such as:
\begin{center}
|This is the part \childdocname{} of \childdocjob{}.|
\end{center}

%%%%%%%%%%%%%%%%%%%%%%%%%%%%%%%%%%%%%%%%%%%%%%%%%%%%%%%%%%%%%%%%%%%%%%%%%%%%%%%%
\subsection{Flags}
\label{sec:flags}

The package makes it easy to generate different versions
of the main or child documents.
To this end compilation flags can be defined
and assigned different default values.
They will be particularly useful in conjunction
with the forwarding mechanism described in \secref{sec:forward}.

For example, it may be useful to have a flag |\version|
which can be set to |draft| or |final|.
The document source will contain some conditional code
depending on the value of |\version|.
Suppose further, the flag should default to |final| for the main file
and to |draft| for child files
which is a natural assignment for editing the document.
This is achieved by placing the following code
in the preamble of the main document
(below the |\childdocmain| directive):
%
\begin{center}
\begin{tabular}{l}
|\ifchilddoc|\\
|\providecommand{\version}{draft}|\\
|\||else|\\
|\providecommand{\version}{final}|\\
|\||fi|
\end{tabular}
\end{center}
%
The definition by |\providecommand| makes sure
that previous definitions are not overwritten.
Further statements |\providecommand{\version}{...}|
can thus be added before the above code to override it.

For the main file, one might add a line
(between |\childdocmain| and the above block)
%
\begin{center}
|%\ifchilddoc\||else\providecommand{\version}{draft}\||fi|
\end{center}
%
which can be uncommented to produce a draft version.
Likewise one can add a line to the very top of a child file
(above the |\childdocof{|\textit{main}|}| directive)
%
\begin{center}
|%\providecommand{\version}{final}|
\end{center}
%
which can be uncommented to produce the final version of this child document.

%%%%%%%%%%%%%%%%%%%%%%%%%%%%%%%%%%%%%%%%%%%%%%%%%%%%%%%%%%%%%%%%%%%%%%%%%%%%%%%%
\subsection{Forwarding}
\label{sec:forward}

Different versions of the main or child documents
using compilation flags as described in \secref{sec:flags}
can be (permanently) stored in different files
for convenient compilation, viewing and distribution.
To this end, the package defines a command
to pass on compilation to a different file:

%%%%%%%%%%%%%%%%%%%%%%%%%%%%%%%%%%%%%%%%
\DescribeMacro{\childdocforward}
The command |\childdocforward| redirects processing to
another source file:
%
\begin{center}
\begin{tabular}{l}
|\input{childdoc.def}|\\
|\childdocforward[|\textit{main}|]{|\textit{dest}|}|\\
\end{tabular}
\end{center}
%
The argument \textit{dest} is the destination file
(without extension).
It should be the main file or one of the child files.
Note that further \textsf{childdoc} directives
such as |\childdocof| and |\childdocforward|
in the indicated file will be processed in this form.
The optional argument \textit{main}
passes on directly to the main file \textit{main}
while pretending to compile the child \textit{dest}.
This form behaves as if \textit{dest}
issues |\childdocof{|\textit{main}|}| right away,
and no further \textsf{childdoc} directives will be processed.

%%%%%%%%%%%%%%%%%%%%%%%%%%%%%%%%%%%%%%%%
\DescribeMacro{\...prefix}
In the alternative form |\childdocforwardprefix|,
%
\begin{center}
\begin{tabular}{l}
|\input{childdoc.def}|\\
|\childdocforwardprefix[|\textit{main}|]{|\textit{prefix}|}{|\textit{dest}|}|
\end{tabular}
\end{center}
%
the destination file is determined by a pattern
depending on the current file:
To make this work, the current file must be called
`{\textit{prefix}\hspace{0.2em}\textit{suffix}}'
with \textit{prefix} matching precisely the argument.
Processing is then passed on to the file
`{\textit{dest}\hspace{0.2em}\textit{suffix}}'.
Surely, the same effect is achieved by
directly specifying the
argument `{\textit{dest}\hspace{0.2em}\textit{suffix}}'
in the first form.
However, that requires to set up a different file
for each child. With the alternative form of the command
all these files can have exactly the same content
which simplifies setting them up and maintaining them.

For example, the following file |draft.tex|
with a compilation flag |\version| as described in \secref{sec:flags}
compiles the main document as a draft:
%
\begin{center}
\begin{tabular}{l}
|\def\version{draft}|\\
|\input{childdoc.def}|\\
|\childdocforward{|\textit{main}|}|
\end{tabular}
\end{center}
%
Likewise, the following files |final|\textit{nn}|.tex|
compile the final version of the child document
|child|\textit{nn}|.tex|:
%
\begin{center}
\begin{tabular}{l}
|\def\version{final}|\\
|\input{childdoc.def}|\\
|\childdocforwardprefix{final}{child}|
\end{tabular}
\end{center}
%

Note that when several versions of a main file and/or of each child file
are to be generated, it may be convenient to set up a |Makefile| or
shell script to automatise the process.

%%%%%%%%%%%%%%%%%%%%%%%%%%%%%%%%%%%%%%%%%%%%%%%%%%%%%%%%%%%%%%%%%%%%%%%%%%%%%%%%
\subsection{Command Line Processing}
\label{sec:commandline}

The effect of redirection files can also be achieved by invoking
the \LaTeX{} compiler with a more elaborate command line.
Most conveniently this should be done as part
of a shell script or a |Makefile|.

When using \textsf{childdoc} in the main file, the following
command lines effectively perform a redirection
(note that depending on the shell being used,
backslashes may have to be doubled: `|\|' $\to$ `|\\|'):
%
\begin{center}
|... -jobname "|\textit{target}|" |\\|"|[\textit{flags}]%
|\input{childdoc.def}\childdocforward[|\textit{main}|]{|\textit{dest}|}"|
\end{center}
%
Here \textit{target} is the name of the output file,
\textit{main} is the name of the main file
and \textit{dest} is the name of the main or child file to be processed
(all filenames without extensions).
The optional argument \textit{main} can be omitted
if \textit{main} matches \textit{dest}.
Optionally, compilation \textit{flags} can be defined via |\def| commands.
This command line makes the \TeX{} engine believe
it is compiling the file \textit{target}
whose content is specified as the latter parameter.
The provided code then forwards the processing to
\textit{main} or \textit{dest} as described in \secref{sec:forward}.

%%%%%%%%%%%%%%%%%%%%%%%%%%%%%%%%%%%%%%%%%%%%%%%%%%%%%%%%%%%%%%%%%%%%%%%%%%%%%%%%
\subsection{Include by Input}
\label{sec:input}

Including child documents by |\include| has some restrictions by design.
Most notably, the content of a child document always occupies
its own set of pages; pages cannot be shared between child documents.
Usually, this behaviour makes perfect sense
because each child document contain an essential part of the document.
However, in some situations it may be desirable to compose
a document from a collection of parts
without having mandatory page breaks between then.
For this case, the package
provides a mechanism to include parts
by |\input| which can also be processed individually.
However, by construction this mechanism
requires manual handling of the content to be output.

%%%%%%%%%%%%%%%%%%%%%%%%%%%%%%%%%%%%%%%%
\DescribeMacro{\ifchilddocmanual}
The main file should be prepared as usual, see \secref{sec:include}.
However, the document body must make a distinction
between processing of an individual part and of the main document, e.g.:
%
\begin{center}
\begin{tabular}{l}
|\ifchilddocmanual|\\
|\input{\childdocname}|\\
|\||else|\\
\textit{document body with }|\input{|\textit{part}|}|\\
|\||fi|
\end{tabular}
\end{center}
%
The conditional |\ifchilddocmanual| is true whenever
a part to be included by |\input| is being compiled,
and the name of the part is stored in |\childdocname|.

%%%%%%%%%%%%%%%%%%%%%%%%%%%%%%%%%%%%%%%%
\DescribeMacro{\childdocby}
Each part to be included by |\input| should start with:
%
\begin{center}
\begin{tabular}{l}
|\input{childdoc.def}|\\
|\childdocby{|\textit{main}|}|\\
\end{tabular}
\end{center}
%
The directive |\childdocby| is similar to |\childdocof|
described in \secref{sec:include},
but the subsequent selection of content must be done manually.
To that end, both |\ifchilddoc| and |\ifchilddocmanual|
will be true upon processing of a part,
and the name of the part is stored in |\childdocname|.
Note that |\jobname| will be set to the filename of the current part
so that each part receives an individual |.aux| file
that does not interfere with the |.aux| file(s) of the main document.
This behaviour can be altered by the alternative form
|\childdocby[*]{|\textit{main}|}| (with a non-empty optional argument)
which uses the |.aux| file of the main document
by setting |\jobname| to \textit{main}.

%%%%%%%%%%%%%%%%%%%%%%%%%%%%%%%%%%%%%%%%%%%%%%%%%%%%%%%%%%%%%%%%%%%%%%%%%%%%%%%%
\subsection{Driver Development}
\label{sec:driver}

The \textsf{childdoc} mechanism can also be use for the development
of definition files such as \LaTeX{} styles or classes.
This case differs from the above setup with multiple parts
included by |\include| in that no |\includeonly| should be invoked.
This can be achieved by starting the include file
(before |\ProvidesPackage|) with:
%
\begin{center}
\begin{tabular}{l}
|\input{childdoc.def}|\\
|\childdocforward{|\textit{main}|}|\\
\end{tabular}
\end{center}
%
or alternatively with:
%
\begin{center}
\begin{tabular}{l}
|\input{childdoc.def}|\\
|\childdocby{|\textit{main}|}|\\
\end{tabular}
\end{center}
%
Both forms have slightly different effects as described above.
The main file is prepared as usual, see \secref{sec:include}.

%%%%%%%%%%%%%%%%%%%%%%%%%%%%%%%%%%%%%%%%%%%%%%%%%%%%%%%%%%%%%%%%%%%%%%%%%%%%%%%%
\subsection{Legacy Detection}
\label{sec:detection}

The directive |\childdocmain| in the main file can detect
whether the complete document or merely a child is to be compiled
even without using the directive |\childdocof|.
This method is deprecated because it is less robust
and there is no compelling reason to use it;
it is merely provided for backward compatibility
and it may be removed in future versions.

If the detection mechanism is to be used,
it is mandatory to correctly specify
the filename of the main file as the argument of |\childdocmain|:
%
\begin{center}
\begin{tabular}{l}
|\input{childdoc.def}|\\
|\childdocmain{|\textit{main}|}|\\
\end{tabular}
\end{center}
%
If |\jobname| does not match the argument \textit{main} of |\childdocmain|,
it is assumed that |\jobname| points to the child file to be compiled.
When using |\childdocmain| with the main file specified as argument,
it suffices to start a child file
with just |\input{|\textit{main}|}|
without loading of the package and using |\childdocof|.
If instead all processing is done
with the appropriate \textsf{childdoc} directives,
the argument of \textit{main} of |\childdocmain| can be empty.

An alternative version of the command line processing described
in \secref{sec:commandline} using the detection mechanism reads:
%
\begin{center}
|... -jobname "|\textit{target}|" "|[\textit{flags}]%
[|\def\jobname{|\textit{dest}|}|]|\input{|\textit{main}|}"|
\end{center}

%%%%%%%%%%%%%%%%%%%%%%%%%%%%%%%%%%%%%%%%%%%%%%%%%%%%%%%%%%%%%%%%%%%%%%%%%%%%%%%%
\subsection{Manual Code}
\label{sec:manual}

In case one cannot be certain whether the definitions file |childdoc.def|
is installed on the target \TeX{} distribution
and one prefers not to ship it,
it is conceivable to paste a few relevant commands into the sources.

To that end, drop all statements |\input{childdoc.def}|
and perform the replacements as outlined below.
Instead of |\childdocmain{|\textit{main}|}| add the following code
to the top of the main file:
%
\begin{center}
\begin{tabular}{l}
|\||ifdefined\childdocname\endinput\||fi\newif\ifchilddoc|\\
|\edef\childdocname{\scantokens\expandafter{\jobname\noexpand}}|\\
|\def\childdocmain{|\textit{main}|}\||ifx\childdocmain\childdocname\||else|\\
|\childdoctrue\includeonly{\childdocname}\let\jobname\childdocmain\||fi|\\
\end{tabular}
\end{center}
%
Instead of |\childdocof{|\textit{main}|}| just include the main file
at the top of each child file:
%
\begin{center}
|\input{|\textit{main}|}|
\end{center}
%
A simple redirection |\childdocforward{|\textit{dest}|}| is achieved by:
%
\begin{center}
|\def\jobname{|\textit{dest}|}\input{\jobname}|
\end{center}
%
The redirection with prefix
|\childdocforwardprefix[|\textit{prefix}|]{|\textit{dest}|}|
is accomplished by:
%
\begin{center}
\begin{tabular}{l}
|{\edef\jobname{\scantokens\expandafter{\jobname\noexpand}}|\\
|\def\redirectjob |\textit{prefix}|#1~~~{\gdef\jobname{|\textit{dest}|#1}}|\\
|\expandafter\redirectjob\jobname~~~}\input{\jobname}|
\end{tabular}
\end{center}

In an alternative approach,
child documents can be compiled by a specific command line
without additional code or specific definitions:
%
\begin{center}
|... -jobname "|\textit{target}|" "|[\textit{flags}]%
|\includeonly{|\textit{dest}|}\input{|\textit{main}|}"|
\end{center}
%

%%%%%%%%%%%%%%%%%%%%%%%%%%%%%%%%%%%%%%%%%%%%%%%%%%%%%%%%%%%%%%%%%%%%%%%%%%%%%%%%
%%%%%%%%%%%%%%%%%%%%%%%%%%%%%%%%%%%%%%%%%%%%%%%%%%%%%%%%%%%%%%%%%%%%%%%%%%%%%%%%
\section{Information}

%%%%%%%%%%%%%%%%%%%%%%%%%%%%%%%%%%%%%%%%%%%%%%%%%%%%%%%%%%%%%%%%%%%%%%%%%%%%%%%%
\subsection{Copyright}

Copyright \copyright{} 2017--2018 Niklas Beisert

This work may be distributed and/or modified under the
conditions of the \LaTeX{} Project Public License, either version 1.3
of this license or (at your option) any later version.
The latest version of this license is in
  \url{http://www.latex-project.org/lppl.txt}
and version 1.3 or later is part of all distributions of \LaTeX{}
version 2005/12/01 or later.

This work has the LPPL maintenance status `maintained'.

The Current Maintainer of this work is Niklas Beisert.

This work consists of the files |README.txt|, |childdoc.ins| and |childdoc.dtx|
as well as the derived files |childdoc.def|, |cdocsamp.tex|
with |cdocsch1.tex|, |cdocsch2.tex|, |cdocspt3.tex|, |cdocspt4.tex|,
|cdocsdrf.tex|, |cdocsfn1.tex|, |cdocsfn2.tex|
as well as |childdoc.pdf|.

%%%%%%%%%%%%%%%%%%%%%%%%%%%%%%%%%%%%%%%%%%%%%%%%%%%%%%%%%%%%%%%%%%%%%%%%%%%%%%%%
\subsection{Files and Installation}

The package consists of the files:
%
\begin{center}
\begin{tabular}{ll}
    |README.txt|   & readme file \\
    |childdoc.ins| & installation file \\
    |childdoc.dtx| & source file \\
    |childdoc.def| & definition file \\
    |cdocsamp.tex| & sample main file \\
    |cdocsch1.tex| & sample include file \\
    |cdocsch2.tex| & sample include file \\
    |cdocspt3.tex| & sample part file \\
    |cdocspt4.tex| & sample part file \\
    |cdocsdrf.tex| & sample redirection file \\
    |cdocsfn1.tex| & sample redirection file \\
    |cdocsfn2.tex| & sample redirection file \\
    |childdoc.pdf| & manual
\end{tabular}
\end{center}
%
The distribution consists of the files
|README.txt|, |childdoc.ins| and |childdoc.dtx|.
%
\begin{itemize}
\item
Run (pdf)\LaTeX{} on |childdoc.dtx|
to compile the manual |childdoc.pdf| (this file).
\item
Run \LaTeX{} on |childdoc.ins| to create the definitions file |childdoc.def|
and the sample |cdocsamp.tex| with include files
|cdocsch1.tex|, |cdocsch2.tex|, |cdocspt3.tex|, |cdocspt4.tex|,
|cdocsdrf.tex|, |cdocsfn1.tex|, |cdocsfn2.tex|.
Then copy the file |childdoc.def| to an appropriate directory of your \LaTeX{}
distribution, e.g.\ \textit{texmf-root}|/tex/latex/childdoc|.
\end{itemize}

%%%%%%%%%%%%%%%%%%%%%%%%%%%%%%%%%%%%%%%%%%%%%%%%%%%%%%%%%%%%%%%%%%%%%%%%%%%%%%%%
\subsection{Related CTAN Packages}

There are several other packages which offer a similar functionality:
%
\begin{itemize}
\item
The packages
\href{http://ctan.org/pkg/docmute}{\textsf{docmute}},
\href{http://ctan.org/pkg/includex}{\textsf{includex}} and
\href{http://ctan.org/pkg/standalone}{\textsf{standalone}}
provide commands to include only the document body of
a child file thus allowing both files to be compiled individually.
\item
The packages \href{http://ctan.org/pkg/subdocs}{\textsf{subdocs}}
and \href{http://ctan.org/pkg/subfiles}{\textsf{subfiles}}
provide structures in which the main and child documents can be
encapsulated and allowing them to be compiled individually.
The inclusion mechanism is different from the conventional |\include|.
\item
The package \href{http://ctan.org/pkg/combine}{\textsf{combine}}
is an elaborate solution to combine several documents into one.
\end{itemize}
%
See also the CTAN topic \href{http://ctan.org/topic/subdocs}{\textsf{subdocs}}
for further related packages.
The present package differs from the above solutions in that
a document structure constructed with the conventional |\include| mechanism
just needs two extra commands at the top of every file
such that all constituent files can be compiled individually.

%%%%%%%%%%%%%%%%%%%%%%%%%%%%%%%%%%%%%%%%%%%%%%%%%%%%%%%%%%%%%%%%%%%%%%%%%%%%%%%%
%\subsection{Feature Suggestions}
%
%The following is a list of features which may be useful for future
%versions of this package:
%%
%\begin{itemize}
%\item
%\ldots
%\end{itemize}

%%%%%%%%%%%%%%%%%%%%%%%%%%%%%%%%%%%%%%%%%%%%%%%%%%%%%%%%%%%%%%%%%%%%%%%%%%%%%%%%
\subsection{Revision History}

%%%%%%%%%%%%%%%%%%%%%%%%%%%%%%%%%%%%%%%%
\paragraph{v2.0:} 2018/12/30

\begin{itemize}
\item
immediate forward processing
\item
added |\childdocby| mechanism
\item
manual restructured
\end{itemize}

%%%%%%%%%%%%%%%%%%%%%%%%%%%%%%%%%%%%%%%%
\paragraph{v1.6:} 2018/01/17

\begin{itemize}
\item
application for development of include files
\item
corrections to manual
\end{itemize}

%%%%%%%%%%%%%%%%%%%%%%%%%%%%%%%%%%%%%%%%
\paragraph{v1.5:} 2017/05/21

\begin{itemize}
\item
more complete structuring introduced
\item
|\childdocof| introduced
\item
|\childdoc| renamed to |\childdocmain|
\item
|\childredirect| renamed to |\childdocforward| and |\childdocforwardprefix|
and functionality expanded
\end{itemize}

%%%%%%%%%%%%%%%%%%%%%%%%%%%%%%%%%%%%%%%%
\paragraph{v1.0:} 2017/04/27

\begin{itemize}
\item
manual and install package
\item
first version published on CTAN
\end{itemize}

%%%%%%%%%%%%%%%%%%%%%%%%%%%%%%%%%%%%%%%%
\paragraph{v0.6:} 2017/04/26

\begin{itemize}
\item
redirection mechanism added
\end{itemize}

%%%%%%%%%%%%%%%%%%%%%%%%%%%%%%%%%%%%%%%%
\paragraph{v0.5:} 2017/04/26

\begin{itemize}
\item
functionality in definition file
\end{itemize}


%%%%%%%%%%%%%%%%%%%%%%%%%%%%%%%%%%%%%%%%%%%%%%%%%%%%%%%%%%%%%%%%%%%%%%%%%%%%%%%%
%%%%%%%%%%%%%%%%%%%%%%%%%%%%%%%%%%%%%%%%%%%%%%%%%%%%%%%%%%%%%%%%%%%%%%%%%%%%%%%%
%%%%%%%%%%%%%%%%%%%%%%%%%%%%%%%%%%%%%%%%%%%%%%%%%%%%%%%%%%%%%%%%%%%%%%%%%%%%%%%%
\appendix

\settowidth\MacroIndent{\rmfamily\scriptsize 000\ }

 \DocInput{childdoc.dtx}

\end{document}
%</driver>
% \fi
%
% %%%%%%%%%%%%%%%%%%%%%%%%%%%%%%%%%%%%%%%%%%%%%%%%%%%%%%%%%%%%%%%%%%%%%%%%%%%%%%
% %%%%%%%%%%%%%%%%%%%%%%%%%%%%%%%%%%%%%%%%%%%%%%%%%%%%%%%%%%%%%%%%%%%%%%%%%%%%%%
% \section{Sample}
%\iffalse
%<*samplemain>
%\fi
%
% The following presents a sample document
% with two chapters, two parts, a title page,
% a compile flag as well as three forwarding files to set the flag.
% It consists of eight |.tex| files:
% \begin{center}
% \begin{tabular}{ll}
% |cdocsamp.tex|&main file\\
% |cdocsch1.tex|&include file for chapter 1\\
% |cdocsch2.tex|&include file for chapter 2\\
% |cdocspt3.tex|&include file for part 3\\
% |cdocspt4.tex|&include file for part 4\\
% |cdocsdrf.tex|&forwarding file for main file in draft mode\\
% |cdocsfi1.tex|&forwarding file for final version of chapter 1\\
% |cdocsfi2.tex|&forwarding file for final version of chapter 2\\
% \end{tabular}
% \end{center}
% Each of the eight files can be compiled directly by the \LaTeX{} compiler.
%
% %%%%%%%%%%%%%%%%%%%%%%%%%%%%%%%%%%%%%%
% \paragraph{Main File.}
%
% The main file is called |cdocsamp.tex|.
%
% Load the \textsf{childdoc} definitions and
% declare the filename for the main document:
%    \begin{macrocode}
\input{childdoc.def}
\childdocmain{}
%    \end{macrocode}

% Optional override for |\version| flag:
%    \begin{macrocode}
%%\ifchilddoc\else\providecommand{\version}{draft}\fi
%    \end{macrocode}

% Define the default values for the |\version| flag
% (|final| for the main file and |draft| for childs):
%    \begin{macrocode}
\ifchilddoc
\providecommand{\version}{draft}
\else
\providecommand{\version}{final}
\fi
%    \end{macrocode}

% Load the standard document class:
%    \begin{macrocode}
\documentclass[12pt]{article}
%    \end{macrocode}

% Start the document body:
%    \begin{macrocode}
\begin{document}
%    \end{macrocode}

% Declare a title page.
% Print title, part of document being processed and version flag:
%    \begin{macrocode}
\addtocounter{page}{-1}
\begin{center}
{\LARGE\bfseries{}childdoc example\par}
\vspace{1cm}
\ifchilddoc
\ifchilddocmanual part\else chapter\fi:
`\childdocname' of `\childdocjob'\par
\else
main document: `\childdocjob'\par
\fi
version: \version\par
\end{center}
\newpage
%    \end{macrocode}

% Manually include selected file,
% otherwise process as usual:
%    \begin{macrocode}
\ifchilddocmanual
\section*{part `\childdocname'}
\input{\childdocname}
\else
%    \end{macrocode}

% Include the two chapters:
%    \begin{macrocode}
\include{cdocsch1}
\include{cdocsch2}
%    \end{macrocode}

% Include the two parts unless only chapters should be displayed:
%    \begin{macrocode}
\ifchilddoc\else
\section{part three}
\input{cdocspt3}
\section{part four}
\input{cdocspt4}
\fi
%    \end{macrocode}

% Process as usual until here:
%    \begin{macrocode}
\fi
%    \end{macrocode}

% End of document body:
%    \begin{macrocode}
\end{document}
%    \end{macrocode}
%\iffalse
%</samplemain>
%\fi
%
% %%%%%%%%%%%%%%%%%%%%%%%%%%%%%%%%%%%%%%
% \paragraph{Chapter Include Files.}
%
% The include files are called |cdocsch1.tex| and |cdocsch2.tex|.
%
%\iffalse
%<*samplechap1|samplechap2>
%\fi

% Optional override for |\version| flag:
%    \begin{macrocode}
%%\providecommand{\version}{final}
%    \end{macrocode}

% Include the main document:
%    \begin{macrocode}
\input{childdoc.def}
\childdocof{cdocsamp}
%    \end{macrocode}

%\iffalse
%</samplechap1|samplechap2>
%\fi
%
%\iffalse
%<*samplechap1>
%\fi
% Some text for chapter 1:
%    \begin{macrocode}
\section{one}
some text in chapter one
%    \end{macrocode}

%\iffalse
%</samplechap1>
%\fi
% Some text for chapter 2:
%\iffalse
%<*samplechap2>
%\fi
%    \begin{macrocode}
\section{two}
more text in chapter two
%    \end{macrocode}

%\iffalse
%</samplechap2>
%\fi
%
% %%%%%%%%%%%%%%%%%%%%%%%%%%%%%%%%%%%%%%
% \paragraph{Part Include Files.}
%
% The include files are called |cdocspt3.tex| and |cdocspt4.tex|.
%
%\iffalse
%<*samplepart3|samplepart4>
%\fi

% Optional override for |\version| flag:
%    \begin{macrocode}
%%\providecommand{\version}{final}
%    \end{macrocode}

% Include the main document:
%    \begin{macrocode}
\input{childdoc.def}
\childdocby{cdocsamp}
%    \end{macrocode}

%\iffalse
%</samplepart3|samplepart4>
%\fi
%
%\iffalse
%<*samplepart3>
%\fi
% Some text for part 3:
%    \begin{macrocode}
some text in part three
%    \end{macrocode}

%\iffalse
%</samplepart3>
%\fi
% Some text for part 4:
%\iffalse
%<*samplepart4>
%\fi
%    \begin{macrocode}
more text in part four
%    \end{macrocode}

%\iffalse
%</samplepart4>
%\fi
%
% %%%%%%%%%%%%%%%%%%%%%%%%%%%%%%%%%%%%%%
% \paragraph{Forwarding for a Complete Draft.}
%
% The following forwarding file |cdocsdrf.tex|
% compiles the main document in draft mode:
%\iffalse
%<*sampledraft>
%\fi
%    \begin{macrocode}
\def\version{draft}
\input{childdoc.def}
\childdocforward{cdocsamp}
%    \end{macrocode}

%\iffalse
%</sampledraft>
%\fi
%
% %%%%%%%%%%%%%%%%%%%%%%%%%%%%%%%%%%%%%%
% \paragraph{Forwarding for Final Version of the Chapters.}
%
% The following forwarding files |cdocsfn1.tex| and |cdocsfn2.tex|
% (with identical content)
% compile the final versions of the child documents
% |cdocsch1.tex| and |cdocsch2.tex|, respectively:
%\iffalse
%<*samplefinal>
%\fi
%    \begin{macrocode}
\def\version{final}
\input{childdoc.def}
\childdocforwardprefix[cdocsamp]{cdocsfn}{cdocsch}
%    \end{macrocode}

%\iffalse
%</samplefinal>
%\fi
%
% %%%%%%%%%%%%%%%%%%%%%%%%%%%%%%%%%%%%%%
% \paragraph{Command Line Processing.}
%
% The following three command lines generate the output files
% |cdocscld|, |cdocscl1| and |cdocscl2|
% which should be identical to
% |cdocsdrf|, |cdocsch1| and |cdocsfn2|, respectively:
% \begin{center}
% \begin{tabular}{l}
% |latex -jobname cdocscld \|\\
% |  "\def\version{draft}\input{childdoc.def}\childdocforward{cdocsamp}"|\\
% |latex -jobname cdocscl1 \|\\
% |  "\input{childdoc.def}\childdocforward[cdocsamp]{cdocsch1}"|\\
% |latex -jobname cdocscl2 \|\\
% |  "\def\version{final}\input{childdoc.def}\childdocforward{cdocsch2}"|
% \end{tabular}
% \end{center}
% Note that the trailing backslash on each first line
% merely continues the input to the second line
% (for convenient cut ant paste).
% Furthermore, the command |latex| can be replaced by any
% of its alternative versions such as |pdflatex|.
%
% %%%%%%%%%%%%%%%%%%%%%%%%%%%%%%%%%%%%%%%%%%%%%%%%%%%%%%%%%%%%%%%%%%%%%%%%%%%%%%
% %%%%%%%%%%%%%%%%%%%%%%%%%%%%%%%%%%%%%%%%%%%%%%%%%%%%%%%%%%%%%%%%%%%%%%%%%%%%%%
% \section{Implementation}
%\iffalse
%<*package>
%\fi
%
% This section describes the definitions file |childdoc.def|.

% The definitions cannot be loaded using |\usepackage| or |\RequirePackage|
% which has a mechanism to prevent loading a style file more than once.
% When loading the definitions by means of |\input|
% multiple instances have to be prevented manually:
%\iffalse
%This code needs to be before the `\ProvidesFile' directive
%which is defined at the beginning of this file.
%Therefore it is also placed there and commented out here.
%</package>
%<*discard>
%\fi
%    \begin{macrocode}
\ifdefined\childdocmain\endinput\fi
%    \end{macrocode}
%\iffalse
%</discard>
%<*package>
%\fi
%
% \macro{\ifchilddoc}
% \macro{\ifchilddocmanual}
% The conditional |\ifchilddoc| tells whether a
% child (true) or main (false) document is being compiled.
% The conditional |\ifchilddocmanual| tells whether
% the |\includeonly| mechanism is used (false) or
% the selection of child files must be performed manually (true).
% The definitions initialise to false:
%    \begin{macrocode}
\newif\ifchilddoc
\newif\ifchilddocmanual
%    \end{macrocode}

% \macro{\childdocname}
% \macro{\childdocjob}
% The macro |\childdocname| stores the name of the main document
% to be compiled. The macro |\childdocjob| stores the name of
% the document on which the \LaTeX{} compiler was originally invoked.
% The content of |\jobname| cannot be compared
% to filenames specified in the source due to different catcodes.
% The following code rescans |\jobname|, stores the result
% in |\childdocname| and saves a copy in |\childdocjob|:
%    \begin{macrocode}
\edef\childdocname{\scantokens\expandafter{\jobname\noexpand}}
\let\childdocjob\childdocname
%    \end{macrocode}

% \macro{\childdocdisable}
% The macro |\childdocdisable| prevents the main file
% from being processed more than once.
% At this stage, the main document command |\childdocmain|
% is assumed to be called once again where it should do nothing.
% Any subsequent call to it should prevent
% a secondary processing of the main document
% It overwrites the forwarding commands
% |\childdocof| and |\childdocforward|
% with empty macros to prevent further inclusions of the main document:
%    \begin{macrocode}
\newcommand{\childdocdisable}
{
  \renewcommand{\childdocmain}[1]{\renewcommand{\childdocmain}[1]{\endinput}}
  \renewcommand{\childdocof}[1]{}
  \renewcommand{\childdocby}[2][]{}
  \renewcommand{\childdocforward}[2][]{}
  \renewcommand{\childdocdisable}{}
}
%    \end{macrocode}

% \macro{\childdocmain}
% The macro |\childdocmain| is to be called at the top of the main file
% with nothing or the main filename (without extension) as argument.
% First, it breaks loops.
% If the argument is not empty and does not match |\childdocname|
% (which is set by the first inclusion of |childdoc.def|),
% |\ifchilddoc| is set to true, |\includeonly| is applied to the child file
% and |\jobname| is set to the main file
% (for proper handling of |.aux| files):
%    \begin{macrocode}
\newcommand{\childdocmain}[1]
{
  \childdocdisable\childdocmain{}
  \if?#1?\else
    \begingroup
      \def\childdoctmp{#1}
      \ifx\childdoctmp\childdocname
        \def\childdoctmp{}
      \else
        \def\childdoctmp
        {
          \childdoctrue
          \includeonly{\childdocname}
          \def\childdocjob{#1}
          \def\jobname{#1}
        }
      \fi
      \expandafter
    \endgroup
    \childdoctmp
  \fi
}
%    \end{macrocode}

% \macro{\childdocof}
% The command |\childdocof| redirects
% compilation to the main file |#1|.
%    \begin{macrocode}
\newcommand{\childdocof}[1]
{
  \childdocdisable
  \childdoctrue
  \includeonly{\childdocname}
  \def\jobname{#1}
  \def\childdocjob{#1}
  \input{#1}
}
%    \end{macrocode}

% \macro{\childdocby}
% The command |\childdocby| ....
%    \begin{macrocode}
\newcommand{\childdocby}[2][]
{
  \childdocdisable
  \childdoctrue
  \childdocmanualtrue
  \if?#1?\else
    \def\jobname{#2}
  \fi
  \def\childdocjob{#2}
  \input{#2}
  \endinput
}
%    \end{macrocode}

% \macro{\childdocforward}
% The command |\childdocforward| redirects
% compilation to the main file or
% (if the optional argument is given) a child file.
% Parameters are set as if the main file
% or a child file starting with |\childdocof| was compiled.
% Then compilation is handed over to the main file:
%    \begin{macrocode}
\newcommand{\childdocforward}[2][]
{
  \begingroup
    \if?#1?
      \def\childdoctmp
      {
        \def\childdocname{#2}
        \def\childdocjob{#2}
        \def\jobname{#2}
        \input{#2}
        \endinput
      }
    \else
      \def\childdoctmp
      {
        \childdocdisable
        \def\childdocname{#2}
        \childdoctrue
        \includeonly{#2}
        \def\childdocjob{#1}
        \def\jobname{#1}
        \input{#1}
        \endinput
      }
    \fi
    \expandafter
  \endgroup
  \childdoctmp
}
%    \end{macrocode}

% \macro{\childdocforwardprefix}
% The command |\childdocforwardprefix| redirects
% compilation to the main or a child file by means of a pattern.
% The prefix |#1| in the current filename is replaced by |#2|
% and the suffix of the current filename is kept
% (it is assumed that the filename does not contain the substring `|~~~|'
% which is used as a delimiter).
% Compilation is handed over to the new file by |\childdocforward|:
%    \begin{macrocode}
\newcommand{\childdocforwardprefix}[3][]
{
  \begingroup
    \def\childdocextract #2##1~~~{\def\childdoctmp{\childdocforward[#1]{#3##1}}}
    \expandafter\childdocextract\childdocname~~~
    \expandafter
  \endgroup
  \childdoctmp
}
%    \end{macrocode}

% \macro{\childdoc}
% The deprecated macro |\childdoc| is a legacy version of |\childdocmain|:
%    \begin{macrocode}
\newcommand{\childdoc}{\childdocmain}
%    \end{macrocode}

% \macro{\childdocredirect}
% The deprecated macro |\childdocredirect| is a legacy version
% of |\childdocforward| and |\childdocforwardprefix|:
%    \begin{macrocode}
\newcommand{\childdocredirect}[2][]
{
  \begingroup
    \if?#1?
      \def\childdoctmp{\childdocforward{#2}}
    \else
      \def\childdoctmp{\childdocforwardprefix{#1}{#2}}
    \fi
    \expandafter
  \endgroup
  \childdoctmp
}
%    \end{macrocode}

%\iffalse
%</package>
%\fi
%
\endinput
|\\
|\childdocmain{|\textit{main}|}|\\
\end{tabular}
\end{center}
%
If |\jobname| does not match the argument \textit{main} of |\childdocmain|,
it is assumed that |\jobname| points to the child file to be compiled.
When using |\childdocmain| with the main file specified as argument,
it suffices to start a child file
with just |\input{|\textit{main}|}|
without loading of the package and using |\childdocof|.
If instead all processing is done
with the appropriate \textsf{childdoc} directives,
the argument of \textit{main} of |\childdocmain| can be empty.

An alternative version of the command line processing described
in \secref{sec:commandline} using the detection mechanism reads:
%
\begin{center}
|... -jobname "|\textit{target}|" "|[\textit{flags}]%
[|\def\jobname{|\textit{dest}|}|]|\input{|\textit{main}|}"|
\end{center}

%%%%%%%%%%%%%%%%%%%%%%%%%%%%%%%%%%%%%%%%%%%%%%%%%%%%%%%%%%%%%%%%%%%%%%%%%%%%%%%%
\subsection{Manual Code}
\label{sec:manual}

In case one cannot be certain whether the definitions file |childdoc.def|
is installed on the target \TeX{} distribution
and one prefers not to ship it,
it is conceivable to paste a few relevant commands into the sources.

To that end, drop all statements |% \iffalse
%
% childdoc.dtx Copyright (C) 2017-2018 Niklas Beisert
%
% This work may be distributed and/or modified under the
% conditions of the LaTeX Project Public License, either version 1.3
% of this license or (at your option) any later version.
% The latest version of this license is in
%   http://www.latex-project.org/lppl.txt
% and version 1.3 or later is part of all distributions of LaTeX
% version 2005/12/01 or later.
%
% This work has the LPPL maintenance status `maintained'.
%
% The Current Maintainer of this work is Niklas Beisert.
%
% This work consists of the files childdoc.dtx and childdoc.ins
% and the derived files childdoc.def and cdocsamp.tex with
% cdocsch1.tex, cdocsch2.tex, cdocsdrf.tex, cdocsfn1.tex, cdocsfn2.tex.
%
%<package>\ifdefined\childdocmain\endinput\fi
%<package>\ProvidesFile{childdoc.def}[2018/12/30 v2.0 child document driver]
%<samplemain>\ProvidesFile{cdocsamp.tex}[2018/12/30 v2.0 sample for childdoc]
%<*driver>
%\ProvidesFile{childdoc.drv}[2018/12/30 v2.0 childdoc reference manual file]
\PassOptionsToClass{10pt,a4paper}{article}
\documentclass{ltxdoc}

\usepackage[margin=35mm]{geometry}
\usepackage{hyperref}
\usepackage{hyperxmp}
\usepackage[usenames]{color}

\hypersetup{colorlinks=true}
\hypersetup{pdfstartview=FitH}
\hypersetup{pdfpagemode=UseNone}
\hypersetup{pdfsource={}}
\hypersetup{pdflang={en-UK}}
\hypersetup{pdfcopyright={Copyright 2017-2018 Niklas Beisert.
  This work may be distributed and/or modified under the
  conditions of the LaTeX Project Public License, either version 1.3
  of this license or (at your option) any later version.}}
\hypersetup{pdflicenseurl={http://www.latex-project.org/lppl.txt}}
\hypersetup{pdfcontactaddress={ETH Zurich, ITP, HIT K,
  Wolfgang-Pauli-Strasse 27}}
\hypersetup{pdfcontactpostcode={8093}}
\hypersetup{pdfcontactcity={Zurich}}
\hypersetup{pdfcontactcountry={Switzerland}}
\hypersetup{pdfcontactemail={nbeisert@itp.phys.ethz.ch}}
\hypersetup{pdfcontacturl={http://people.phys.ethz.ch/\xmptilde nbeisert/}}

\newcommand{\secref}[1]{\hyperref[#1]{section \ref*{#1}}}

\parskip1ex
\parindent0pt
\let\olditemize\itemize
\def\itemize{\olditemize\parskip0pt}

\begin{document}

\title{The \textsf{childdoc} Package}
\hypersetup{pdftitle={The childdoc Package}}
\author{Niklas Beisert\\[2ex]
  Institut f\"ur Theoretische Physik\\
  Eidgen\"ossische Technische Hochschule Z\"urich\\
  Wolfgang-Pauli-Strasse 27, 8093 Z\"urich, Switzerland\\[1ex]
  \href{mailto:nbeisert@itp.phys.ethz.ch}
  {\texttt{nbeisert@itp.phys.ethz.ch}}}
\hypersetup{pdfauthor={Niklas Beisert}}
\hypersetup{pdfsubject={Manual for the LaTeX2e Package childdoc}}
\date{30 December 2018, \textsf{v2.0}}
\maketitle

\begin{abstract}\noindent
\textsf{childdoc} is a \LaTeXe{} package
that enables the direct compilation
of document sections included by |\include|
to individual files.
\end{abstract}

\begingroup
\parskip0ex
\tableofcontents
\endgroup

%%%%%%%%%%%%%%%%%%%%%%%%%%%%%%%%%%%%%%%%%%%%%%%%%%%%%%%%%%%%%%%%%%%%%%%%%%%%%%%%
%%%%%%%%%%%%%%%%%%%%%%%%%%%%%%%%%%%%%%%%%%%%%%%%%%%%%%%%%%%%%%%%%%%%%%%%%%%%%%%%
\section{Introduction}

\LaTeX{} provides a mechanism to structure a large document (such as a book)
into a main file and several child files (containing the chapters)
using the |\include| command.
This mechanism is beneficial for documents
which span hundreds of pages in order to
make the source file(s) more manageable.
Moreover, compilation can be restricted to
selected child files by means of the |\includeonly| command.
The latter feature can be used to reduce the compilation time while editing
(this was significantly more useful in the earlier days of \LaTeX{})
or to generate a smaller document which is easier to navigate.
Another application of |\includeonly| is to generate
documents consisting of selected parts of the complete document.

However, there are a few drawbacks of the plain |\include| mechanism:
\begin{itemize}
\item
The child files cannot be compiled on their own,
they can only be compiled via the main file.
A naive editing environment
(such as a text editor with an option
to have the current file processed by \LaTeX)
may require one to switch to the main file before compiling;
attempting to compile the child file produces errors.
\item
The main file must be modified (each time)
to adjust the |\includeonly| command
to the present needs. This easily leaves the main file in a messy state.
\item
The generated document will always carry the filename
of the main document. This is inconvenient if
several child files are to be compiled and
to be kept for distribution.
\end{itemize}

The present package provides a simple interface
to make child files individually compilable by \LaTeX{}.
Compiling a child file then has the same effect as compiling
the main file with an |\includeonly| command
to select the appropriate child.
Moreover the generated document will carry the name of the child
rather than the main file.
This resolves all three above issues.

This feature is meant to make the editing of books,
thesis documents and lecture notes somewhat more convenient.
However, the package can also be used efficiently for
composing a series of documents (such as exercise sheets)
which are typically distributed individually.
It then assists the author in generating the individual documents
(potentially in different versions)
as well as a document containing the collected series.
Another application is in developing style files
or other kinds of included material
where compilation of the style file could redirect
to a sample or test file.

%%%%%%%%%%%%%%%%%%%%%%%%%%%%%%%%%%%%%%%%%%%%%%%%%%%%%%%%%%%%%%%%%%%%%%%%%%%%%%%%
%%%%%%%%%%%%%%%%%%%%%%%%%%%%%%%%%%%%%%%%%%%%%%%%%%%%%%%%%%%%%%%%%%%%%%%%%%%%%%%%
\section{Usage}

First of all, the package \textsf{childdoc} is \emph{not} a standard
\LaTeXe{} |.sty| style file! Therefore it needs to be invoked in
a non-standard way.

%%%%%%%%%%%%%%%%%%%%%%%%%%%%%%%%%%%%%%%%%%%%%%%%%%%%%%%%%%%%%%%%%%%%%%%%%%%%%%%%
\subsection{Included Files}
\label{sec:include}

%%%%%%%%%%%%%%%%%%%%%%%%%%%%%%%%%%%%%%%%
\DescribeMacro{\childdocmain}
To use the package, add the commands
\begin{center}
\begin{tabular}{l}
|\input{childdoc.def}|\\
|\childdocmain{}|\\
\end{tabular}
\end{center}
at the very top of the main \LaTeX{} file,
in particular \emph{before} the |\documentclass| statement!
The argument of |\childdocmain| should be left empty
(but it must be present).

%%%%%%%%%%%%%%%%%%%%%%%%%%%%%%%%%%%%%%%%
\DescribeMacro{\childdocof}
Furthermore, add the commands
\begin{center}
\begin{tabular}{l}
|\input{childdoc.def}|\\
|\childdocof{|\textit{main}|}|\\
\end{tabular}
\end{center}
at the top of every child file \textit{child}
which is included by |\include{|\textit{child}|}|
from within the main file
(or at least for those files to be compiled individually).
The argument \textit{main} must be the filename of the main file.

There are a couple of
considerations in setting up the main and child documents:

%%%%%%%%%%%%%%%%%%%%%%%%%%%%%%%%%%%%%%%%
\paragraph{Restrictions.}

Please note the following restrictions:
\begin{itemize}
\item
|\childdocmain| must be called with one argument \textit{main}
to ensure compatibility with earlier version of the package.
It must either be empty (|\childdocmain{}|)
or precisely match the filename of the main file in which it is specified.
See \secref{sec:detection} for further information.
\item
The filename \textit{main} must be specified without the |.tex| extension.
\item
The filename \textit{main} is case sensitive
(even in case-insensitive file systems)
due to internal string comparison.
\item
The argument \textit{main} should be fully expanded, it cannot be a macro.
\item
Subdirectories and special characters should be avoided in filenames.
\item
The command |\childdocmain{|\textit{main}|}| must be followed by a whitespace.
It should not be followed immediately by another command
or by a comment mark `|%|'.
This is because the \TeX{} parser reads the token immediately following
the argument of |\childdocmain| and puts it
at the beginning of every child section;
however, a white\-space is ignored.
\end{itemize}

%%%%%%%%%%%%%%%%%%%%%%%%%%%%%%%%%%%%%%%%
\paragraph{Content of Main File.}

It is advisable to place all content in the child files included by |\include|.
Any output contained in the main file will appear in all child documents
unless suppressed manually;
it cannot be suppressed automatically by the |\includeonly| directive
and thus should normally be avoided.
A method to include some content in the main file
by means of conditional processing is described in \secref{sec:conditional}.

%%%%%%%%%%%%%%%%%%%%%%%%%%%%%%%%%%%%%%%%
\paragraph{Page Numbering.}

When only a part of the document is compiled,
the appropriate numbering of pages
(as well as other status parameters)
is determined from the |.aux| files.
The latter contain information from previous passes.
However this information needs to propagate through
all intermediate child documents.
Therefore the page numbering in child documents may well
be inconsistent until the complete document is compiled at least once.

A useful (if unconventional) way to always ensure a consistent
page numbering is to restart the numbering in each child document
and denote the pages by `\textit{child}|.|\textit{page}'
where \textit{child} represents the chapter/section number of the child file.
This can be achieved by the command
|\numberwithin{page}{|\textit{child}|}|
of the \textsf{amsmath} package
where \textit{child} can be |chapter| or |section|
depending on the chosen structuring.
Alternatively, one can modify the macro |\thepage| appropriately
and reset the counter |page| at the start of each child file.

%%%%%%%%%%%%%%%%%%%%%%%%%%%%%%%%%%%%%%%%%%%%%%%%%%%%%%%%%%%%%%%%%%%%%%%%%%%%%%%%
\subsection{Conditional Processing}
\label{sec:conditional}

The package provides a mechanism to compile different versions
of a document. To customise the versions further some conditional processing
can come in handy to distinguish which version is being compiled.
The package provides two macros to describe the compilation context:

%%%%%%%%%%%%%%%%%%%%%%%%%%%%%%%%%%%%%%%%
\DescribeMacro{\ifchilddoc}
The conditional |\ifchilddoc| distinguishes between the compilation of
child documents and the main document:
%
\begin{center}
|\ifchilddoc |\textit{child-code}| |[|\||else |\textit{main-code}]| \||fi|
\end{center}

%%%%%%%%%%%%%%%%%%%%%%%%%%%%%%%%%%%%%%%%
\DescribeMacro{\childdocname}
\DescribeMacro{\childdocjob}
The macro |\childdocname| contains the filename (without extension)
of the main or child file being processed.
Note that |\childdocjob| will always contain the name of the main file.

%%%%%%%%%%%%%%%%%%%%%%%%%%%%%%%%%%%%%%%%
\paragraph{Title Page.}

Conditional processing can be used to include a title or banner page
in the main document when proper precautions are taken.
Importantly, the code in the main file should ensure that the page counter
(as well as other status parameters which are stored in the |.aux| files)
takes the same value after the conditional processing.
Otherwise the page numbers may take divergent values
depending on which part is compiled.

For example, a title page could be declared by:
%
\begin{center}
\begin{tabular}{l}
|\ifchilddoc\||else|\\
|\addtocounter{page}{-1}|\\
\textit{code for title page}\\
|\newpage|\\
|\||fi|
\end{tabular}
\end{center}
%
A banner page for the child documents can be generated by:
%
\begin{center}
\begin{tabular}{l}
|\ifchilddoc|\\
|\addtocounter{page}{-1}|\\
\textit{code for banner page}\\
|\newpage|\\
|\||fi|
\end{tabular}
\end{center}
%
Here one could write a message such as:
\begin{center}
|This is the part \childdocname{} of \childdocjob{}.|
\end{center}

%%%%%%%%%%%%%%%%%%%%%%%%%%%%%%%%%%%%%%%%%%%%%%%%%%%%%%%%%%%%%%%%%%%%%%%%%%%%%%%%
\subsection{Flags}
\label{sec:flags}

The package makes it easy to generate different versions
of the main or child documents.
To this end compilation flags can be defined
and assigned different default values.
They will be particularly useful in conjunction
with the forwarding mechanism described in \secref{sec:forward}.

For example, it may be useful to have a flag |\version|
which can be set to |draft| or |final|.
The document source will contain some conditional code
depending on the value of |\version|.
Suppose further, the flag should default to |final| for the main file
and to |draft| for child files
which is a natural assignment for editing the document.
This is achieved by placing the following code
in the preamble of the main document
(below the |\childdocmain| directive):
%
\begin{center}
\begin{tabular}{l}
|\ifchilddoc|\\
|\providecommand{\version}{draft}|\\
|\||else|\\
|\providecommand{\version}{final}|\\
|\||fi|
\end{tabular}
\end{center}
%
The definition by |\providecommand| makes sure
that previous definitions are not overwritten.
Further statements |\providecommand{\version}{...}|
can thus be added before the above code to override it.

For the main file, one might add a line
(between |\childdocmain| and the above block)
%
\begin{center}
|%\ifchilddoc\||else\providecommand{\version}{draft}\||fi|
\end{center}
%
which can be uncommented to produce a draft version.
Likewise one can add a line to the very top of a child file
(above the |\childdocof{|\textit{main}|}| directive)
%
\begin{center}
|%\providecommand{\version}{final}|
\end{center}
%
which can be uncommented to produce the final version of this child document.

%%%%%%%%%%%%%%%%%%%%%%%%%%%%%%%%%%%%%%%%%%%%%%%%%%%%%%%%%%%%%%%%%%%%%%%%%%%%%%%%
\subsection{Forwarding}
\label{sec:forward}

Different versions of the main or child documents
using compilation flags as described in \secref{sec:flags}
can be (permanently) stored in different files
for convenient compilation, viewing and distribution.
To this end, the package defines a command
to pass on compilation to a different file:

%%%%%%%%%%%%%%%%%%%%%%%%%%%%%%%%%%%%%%%%
\DescribeMacro{\childdocforward}
The command |\childdocforward| redirects processing to
another source file:
%
\begin{center}
\begin{tabular}{l}
|\input{childdoc.def}|\\
|\childdocforward[|\textit{main}|]{|\textit{dest}|}|\\
\end{tabular}
\end{center}
%
The argument \textit{dest} is the destination file
(without extension).
It should be the main file or one of the child files.
Note that further \textsf{childdoc} directives
such as |\childdocof| and |\childdocforward|
in the indicated file will be processed in this form.
The optional argument \textit{main}
passes on directly to the main file \textit{main}
while pretending to compile the child \textit{dest}.
This form behaves as if \textit{dest}
issues |\childdocof{|\textit{main}|}| right away,
and no further \textsf{childdoc} directives will be processed.

%%%%%%%%%%%%%%%%%%%%%%%%%%%%%%%%%%%%%%%%
\DescribeMacro{\...prefix}
In the alternative form |\childdocforwardprefix|,
%
\begin{center}
\begin{tabular}{l}
|\input{childdoc.def}|\\
|\childdocforwardprefix[|\textit{main}|]{|\textit{prefix}|}{|\textit{dest}|}|
\end{tabular}
\end{center}
%
the destination file is determined by a pattern
depending on the current file:
To make this work, the current file must be called
`{\textit{prefix}\hspace{0.2em}\textit{suffix}}'
with \textit{prefix} matching precisely the argument.
Processing is then passed on to the file
`{\textit{dest}\hspace{0.2em}\textit{suffix}}'.
Surely, the same effect is achieved by
directly specifying the
argument `{\textit{dest}\hspace{0.2em}\textit{suffix}}'
in the first form.
However, that requires to set up a different file
for each child. With the alternative form of the command
all these files can have exactly the same content
which simplifies setting them up and maintaining them.

For example, the following file |draft.tex|
with a compilation flag |\version| as described in \secref{sec:flags}
compiles the main document as a draft:
%
\begin{center}
\begin{tabular}{l}
|\def\version{draft}|\\
|\input{childdoc.def}|\\
|\childdocforward{|\textit{main}|}|
\end{tabular}
\end{center}
%
Likewise, the following files |final|\textit{nn}|.tex|
compile the final version of the child document
|child|\textit{nn}|.tex|:
%
\begin{center}
\begin{tabular}{l}
|\def\version{final}|\\
|\input{childdoc.def}|\\
|\childdocforwardprefix{final}{child}|
\end{tabular}
\end{center}
%

Note that when several versions of a main file and/or of each child file
are to be generated, it may be convenient to set up a |Makefile| or
shell script to automatise the process.

%%%%%%%%%%%%%%%%%%%%%%%%%%%%%%%%%%%%%%%%%%%%%%%%%%%%%%%%%%%%%%%%%%%%%%%%%%%%%%%%
\subsection{Command Line Processing}
\label{sec:commandline}

The effect of redirection files can also be achieved by invoking
the \LaTeX{} compiler with a more elaborate command line.
Most conveniently this should be done as part
of a shell script or a |Makefile|.

When using \textsf{childdoc} in the main file, the following
command lines effectively perform a redirection
(note that depending on the shell being used,
backslashes may have to be doubled: `|\|' $\to$ `|\\|'):
%
\begin{center}
|... -jobname "|\textit{target}|" |\\|"|[\textit{flags}]%
|\input{childdoc.def}\childdocforward[|\textit{main}|]{|\textit{dest}|}"|
\end{center}
%
Here \textit{target} is the name of the output file,
\textit{main} is the name of the main file
and \textit{dest} is the name of the main or child file to be processed
(all filenames without extensions).
The optional argument \textit{main} can be omitted
if \textit{main} matches \textit{dest}.
Optionally, compilation \textit{flags} can be defined via |\def| commands.
This command line makes the \TeX{} engine believe
it is compiling the file \textit{target}
whose content is specified as the latter parameter.
The provided code then forwards the processing to
\textit{main} or \textit{dest} as described in \secref{sec:forward}.

%%%%%%%%%%%%%%%%%%%%%%%%%%%%%%%%%%%%%%%%%%%%%%%%%%%%%%%%%%%%%%%%%%%%%%%%%%%%%%%%
\subsection{Include by Input}
\label{sec:input}

Including child documents by |\include| has some restrictions by design.
Most notably, the content of a child document always occupies
its own set of pages; pages cannot be shared between child documents.
Usually, this behaviour makes perfect sense
because each child document contain an essential part of the document.
However, in some situations it may be desirable to compose
a document from a collection of parts
without having mandatory page breaks between then.
For this case, the package
provides a mechanism to include parts
by |\input| which can also be processed individually.
However, by construction this mechanism
requires manual handling of the content to be output.

%%%%%%%%%%%%%%%%%%%%%%%%%%%%%%%%%%%%%%%%
\DescribeMacro{\ifchilddocmanual}
The main file should be prepared as usual, see \secref{sec:include}.
However, the document body must make a distinction
between processing of an individual part and of the main document, e.g.:
%
\begin{center}
\begin{tabular}{l}
|\ifchilddocmanual|\\
|\input{\childdocname}|\\
|\||else|\\
\textit{document body with }|\input{|\textit{part}|}|\\
|\||fi|
\end{tabular}
\end{center}
%
The conditional |\ifchilddocmanual| is true whenever
a part to be included by |\input| is being compiled,
and the name of the part is stored in |\childdocname|.

%%%%%%%%%%%%%%%%%%%%%%%%%%%%%%%%%%%%%%%%
\DescribeMacro{\childdocby}
Each part to be included by |\input| should start with:
%
\begin{center}
\begin{tabular}{l}
|\input{childdoc.def}|\\
|\childdocby{|\textit{main}|}|\\
\end{tabular}
\end{center}
%
The directive |\childdocby| is similar to |\childdocof|
described in \secref{sec:include},
but the subsequent selection of content must be done manually.
To that end, both |\ifchilddoc| and |\ifchilddocmanual|
will be true upon processing of a part,
and the name of the part is stored in |\childdocname|.
Note that |\jobname| will be set to the filename of the current part
so that each part receives an individual |.aux| file
that does not interfere with the |.aux| file(s) of the main document.
This behaviour can be altered by the alternative form
|\childdocby[*]{|\textit{main}|}| (with a non-empty optional argument)
which uses the |.aux| file of the main document
by setting |\jobname| to \textit{main}.

%%%%%%%%%%%%%%%%%%%%%%%%%%%%%%%%%%%%%%%%%%%%%%%%%%%%%%%%%%%%%%%%%%%%%%%%%%%%%%%%
\subsection{Driver Development}
\label{sec:driver}

The \textsf{childdoc} mechanism can also be use for the development
of definition files such as \LaTeX{} styles or classes.
This case differs from the above setup with multiple parts
included by |\include| in that no |\includeonly| should be invoked.
This can be achieved by starting the include file
(before |\ProvidesPackage|) with:
%
\begin{center}
\begin{tabular}{l}
|\input{childdoc.def}|\\
|\childdocforward{|\textit{main}|}|\\
\end{tabular}
\end{center}
%
or alternatively with:
%
\begin{center}
\begin{tabular}{l}
|\input{childdoc.def}|\\
|\childdocby{|\textit{main}|}|\\
\end{tabular}
\end{center}
%
Both forms have slightly different effects as described above.
The main file is prepared as usual, see \secref{sec:include}.

%%%%%%%%%%%%%%%%%%%%%%%%%%%%%%%%%%%%%%%%%%%%%%%%%%%%%%%%%%%%%%%%%%%%%%%%%%%%%%%%
\subsection{Legacy Detection}
\label{sec:detection}

The directive |\childdocmain| in the main file can detect
whether the complete document or merely a child is to be compiled
even without using the directive |\childdocof|.
This method is deprecated because it is less robust
and there is no compelling reason to use it;
it is merely provided for backward compatibility
and it may be removed in future versions.

If the detection mechanism is to be used,
it is mandatory to correctly specify
the filename of the main file as the argument of |\childdocmain|:
%
\begin{center}
\begin{tabular}{l}
|\input{childdoc.def}|\\
|\childdocmain{|\textit{main}|}|\\
\end{tabular}
\end{center}
%
If |\jobname| does not match the argument \textit{main} of |\childdocmain|,
it is assumed that |\jobname| points to the child file to be compiled.
When using |\childdocmain| with the main file specified as argument,
it suffices to start a child file
with just |\input{|\textit{main}|}|
without loading of the package and using |\childdocof|.
If instead all processing is done
with the appropriate \textsf{childdoc} directives,
the argument of \textit{main} of |\childdocmain| can be empty.

An alternative version of the command line processing described
in \secref{sec:commandline} using the detection mechanism reads:
%
\begin{center}
|... -jobname "|\textit{target}|" "|[\textit{flags}]%
[|\def\jobname{|\textit{dest}|}|]|\input{|\textit{main}|}"|
\end{center}

%%%%%%%%%%%%%%%%%%%%%%%%%%%%%%%%%%%%%%%%%%%%%%%%%%%%%%%%%%%%%%%%%%%%%%%%%%%%%%%%
\subsection{Manual Code}
\label{sec:manual}

In case one cannot be certain whether the definitions file |childdoc.def|
is installed on the target \TeX{} distribution
and one prefers not to ship it,
it is conceivable to paste a few relevant commands into the sources.

To that end, drop all statements |\input{childdoc.def}|
and perform the replacements as outlined below.
Instead of |\childdocmain{|\textit{main}|}| add the following code
to the top of the main file:
%
\begin{center}
\begin{tabular}{l}
|\||ifdefined\childdocname\endinput\||fi\newif\ifchilddoc|\\
|\edef\childdocname{\scantokens\expandafter{\jobname\noexpand}}|\\
|\def\childdocmain{|\textit{main}|}\||ifx\childdocmain\childdocname\||else|\\
|\childdoctrue\includeonly{\childdocname}\let\jobname\childdocmain\||fi|\\
\end{tabular}
\end{center}
%
Instead of |\childdocof{|\textit{main}|}| just include the main file
at the top of each child file:
%
\begin{center}
|\input{|\textit{main}|}|
\end{center}
%
A simple redirection |\childdocforward{|\textit{dest}|}| is achieved by:
%
\begin{center}
|\def\jobname{|\textit{dest}|}\input{\jobname}|
\end{center}
%
The redirection with prefix
|\childdocforwardprefix[|\textit{prefix}|]{|\textit{dest}|}|
is accomplished by:
%
\begin{center}
\begin{tabular}{l}
|{\edef\jobname{\scantokens\expandafter{\jobname\noexpand}}|\\
|\def\redirectjob |\textit{prefix}|#1~~~{\gdef\jobname{|\textit{dest}|#1}}|\\
|\expandafter\redirectjob\jobname~~~}\input{\jobname}|
\end{tabular}
\end{center}

In an alternative approach,
child documents can be compiled by a specific command line
without additional code or specific definitions:
%
\begin{center}
|... -jobname "|\textit{target}|" "|[\textit{flags}]%
|\includeonly{|\textit{dest}|}\input{|\textit{main}|}"|
\end{center}
%

%%%%%%%%%%%%%%%%%%%%%%%%%%%%%%%%%%%%%%%%%%%%%%%%%%%%%%%%%%%%%%%%%%%%%%%%%%%%%%%%
%%%%%%%%%%%%%%%%%%%%%%%%%%%%%%%%%%%%%%%%%%%%%%%%%%%%%%%%%%%%%%%%%%%%%%%%%%%%%%%%
\section{Information}

%%%%%%%%%%%%%%%%%%%%%%%%%%%%%%%%%%%%%%%%%%%%%%%%%%%%%%%%%%%%%%%%%%%%%%%%%%%%%%%%
\subsection{Copyright}

Copyright \copyright{} 2017--2018 Niklas Beisert

This work may be distributed and/or modified under the
conditions of the \LaTeX{} Project Public License, either version 1.3
of this license or (at your option) any later version.
The latest version of this license is in
  \url{http://www.latex-project.org/lppl.txt}
and version 1.3 or later is part of all distributions of \LaTeX{}
version 2005/12/01 or later.

This work has the LPPL maintenance status `maintained'.

The Current Maintainer of this work is Niklas Beisert.

This work consists of the files |README.txt|, |childdoc.ins| and |childdoc.dtx|
as well as the derived files |childdoc.def|, |cdocsamp.tex|
with |cdocsch1.tex|, |cdocsch2.tex|, |cdocspt3.tex|, |cdocspt4.tex|,
|cdocsdrf.tex|, |cdocsfn1.tex|, |cdocsfn2.tex|
as well as |childdoc.pdf|.

%%%%%%%%%%%%%%%%%%%%%%%%%%%%%%%%%%%%%%%%%%%%%%%%%%%%%%%%%%%%%%%%%%%%%%%%%%%%%%%%
\subsection{Files and Installation}

The package consists of the files:
%
\begin{center}
\begin{tabular}{ll}
    |README.txt|   & readme file \\
    |childdoc.ins| & installation file \\
    |childdoc.dtx| & source file \\
    |childdoc.def| & definition file \\
    |cdocsamp.tex| & sample main file \\
    |cdocsch1.tex| & sample include file \\
    |cdocsch2.tex| & sample include file \\
    |cdocspt3.tex| & sample part file \\
    |cdocspt4.tex| & sample part file \\
    |cdocsdrf.tex| & sample redirection file \\
    |cdocsfn1.tex| & sample redirection file \\
    |cdocsfn2.tex| & sample redirection file \\
    |childdoc.pdf| & manual
\end{tabular}
\end{center}
%
The distribution consists of the files
|README.txt|, |childdoc.ins| and |childdoc.dtx|.
%
\begin{itemize}
\item
Run (pdf)\LaTeX{} on |childdoc.dtx|
to compile the manual |childdoc.pdf| (this file).
\item
Run \LaTeX{} on |childdoc.ins| to create the definitions file |childdoc.def|
and the sample |cdocsamp.tex| with include files
|cdocsch1.tex|, |cdocsch2.tex|, |cdocspt3.tex|, |cdocspt4.tex|,
|cdocsdrf.tex|, |cdocsfn1.tex|, |cdocsfn2.tex|.
Then copy the file |childdoc.def| to an appropriate directory of your \LaTeX{}
distribution, e.g.\ \textit{texmf-root}|/tex/latex/childdoc|.
\end{itemize}

%%%%%%%%%%%%%%%%%%%%%%%%%%%%%%%%%%%%%%%%%%%%%%%%%%%%%%%%%%%%%%%%%%%%%%%%%%%%%%%%
\subsection{Related CTAN Packages}

There are several other packages which offer a similar functionality:
%
\begin{itemize}
\item
The packages
\href{http://ctan.org/pkg/docmute}{\textsf{docmute}},
\href{http://ctan.org/pkg/includex}{\textsf{includex}} and
\href{http://ctan.org/pkg/standalone}{\textsf{standalone}}
provide commands to include only the document body of
a child file thus allowing both files to be compiled individually.
\item
The packages \href{http://ctan.org/pkg/subdocs}{\textsf{subdocs}}
and \href{http://ctan.org/pkg/subfiles}{\textsf{subfiles}}
provide structures in which the main and child documents can be
encapsulated and allowing them to be compiled individually.
The inclusion mechanism is different from the conventional |\include|.
\item
The package \href{http://ctan.org/pkg/combine}{\textsf{combine}}
is an elaborate solution to combine several documents into one.
\end{itemize}
%
See also the CTAN topic \href{http://ctan.org/topic/subdocs}{\textsf{subdocs}}
for further related packages.
The present package differs from the above solutions in that
a document structure constructed with the conventional |\include| mechanism
just needs two extra commands at the top of every file
such that all constituent files can be compiled individually.

%%%%%%%%%%%%%%%%%%%%%%%%%%%%%%%%%%%%%%%%%%%%%%%%%%%%%%%%%%%%%%%%%%%%%%%%%%%%%%%%
%\subsection{Feature Suggestions}
%
%The following is a list of features which may be useful for future
%versions of this package:
%%
%\begin{itemize}
%\item
%\ldots
%\end{itemize}

%%%%%%%%%%%%%%%%%%%%%%%%%%%%%%%%%%%%%%%%%%%%%%%%%%%%%%%%%%%%%%%%%%%%%%%%%%%%%%%%
\subsection{Revision History}

%%%%%%%%%%%%%%%%%%%%%%%%%%%%%%%%%%%%%%%%
\paragraph{v2.0:} 2018/12/30

\begin{itemize}
\item
immediate forward processing
\item
added |\childdocby| mechanism
\item
manual restructured
\end{itemize}

%%%%%%%%%%%%%%%%%%%%%%%%%%%%%%%%%%%%%%%%
\paragraph{v1.6:} 2018/01/17

\begin{itemize}
\item
application for development of include files
\item
corrections to manual
\end{itemize}

%%%%%%%%%%%%%%%%%%%%%%%%%%%%%%%%%%%%%%%%
\paragraph{v1.5:} 2017/05/21

\begin{itemize}
\item
more complete structuring introduced
\item
|\childdocof| introduced
\item
|\childdoc| renamed to |\childdocmain|
\item
|\childredirect| renamed to |\childdocforward| and |\childdocforwardprefix|
and functionality expanded
\end{itemize}

%%%%%%%%%%%%%%%%%%%%%%%%%%%%%%%%%%%%%%%%
\paragraph{v1.0:} 2017/04/27

\begin{itemize}
\item
manual and install package
\item
first version published on CTAN
\end{itemize}

%%%%%%%%%%%%%%%%%%%%%%%%%%%%%%%%%%%%%%%%
\paragraph{v0.6:} 2017/04/26

\begin{itemize}
\item
redirection mechanism added
\end{itemize}

%%%%%%%%%%%%%%%%%%%%%%%%%%%%%%%%%%%%%%%%
\paragraph{v0.5:} 2017/04/26

\begin{itemize}
\item
functionality in definition file
\end{itemize}


%%%%%%%%%%%%%%%%%%%%%%%%%%%%%%%%%%%%%%%%%%%%%%%%%%%%%%%%%%%%%%%%%%%%%%%%%%%%%%%%
%%%%%%%%%%%%%%%%%%%%%%%%%%%%%%%%%%%%%%%%%%%%%%%%%%%%%%%%%%%%%%%%%%%%%%%%%%%%%%%%
%%%%%%%%%%%%%%%%%%%%%%%%%%%%%%%%%%%%%%%%%%%%%%%%%%%%%%%%%%%%%%%%%%%%%%%%%%%%%%%%
\appendix

\settowidth\MacroIndent{\rmfamily\scriptsize 000\ }

 \DocInput{childdoc.dtx}

\end{document}
%</driver>
% \fi
%
% %%%%%%%%%%%%%%%%%%%%%%%%%%%%%%%%%%%%%%%%%%%%%%%%%%%%%%%%%%%%%%%%%%%%%%%%%%%%%%
% %%%%%%%%%%%%%%%%%%%%%%%%%%%%%%%%%%%%%%%%%%%%%%%%%%%%%%%%%%%%%%%%%%%%%%%%%%%%%%
% \section{Sample}
%\iffalse
%<*samplemain>
%\fi
%
% The following presents a sample document
% with two chapters, two parts, a title page,
% a compile flag as well as three forwarding files to set the flag.
% It consists of eight |.tex| files:
% \begin{center}
% \begin{tabular}{ll}
% |cdocsamp.tex|&main file\\
% |cdocsch1.tex|&include file for chapter 1\\
% |cdocsch2.tex|&include file for chapter 2\\
% |cdocspt3.tex|&include file for part 3\\
% |cdocspt4.tex|&include file for part 4\\
% |cdocsdrf.tex|&forwarding file for main file in draft mode\\
% |cdocsfi1.tex|&forwarding file for final version of chapter 1\\
% |cdocsfi2.tex|&forwarding file for final version of chapter 2\\
% \end{tabular}
% \end{center}
% Each of the eight files can be compiled directly by the \LaTeX{} compiler.
%
% %%%%%%%%%%%%%%%%%%%%%%%%%%%%%%%%%%%%%%
% \paragraph{Main File.}
%
% The main file is called |cdocsamp.tex|.
%
% Load the \textsf{childdoc} definitions and
% declare the filename for the main document:
%    \begin{macrocode}
\input{childdoc.def}
\childdocmain{}
%    \end{macrocode}

% Optional override for |\version| flag:
%    \begin{macrocode}
%%\ifchilddoc\else\providecommand{\version}{draft}\fi
%    \end{macrocode}

% Define the default values for the |\version| flag
% (|final| for the main file and |draft| for childs):
%    \begin{macrocode}
\ifchilddoc
\providecommand{\version}{draft}
\else
\providecommand{\version}{final}
\fi
%    \end{macrocode}

% Load the standard document class:
%    \begin{macrocode}
\documentclass[12pt]{article}
%    \end{macrocode}

% Start the document body:
%    \begin{macrocode}
\begin{document}
%    \end{macrocode}

% Declare a title page.
% Print title, part of document being processed and version flag:
%    \begin{macrocode}
\addtocounter{page}{-1}
\begin{center}
{\LARGE\bfseries{}childdoc example\par}
\vspace{1cm}
\ifchilddoc
\ifchilddocmanual part\else chapter\fi:
`\childdocname' of `\childdocjob'\par
\else
main document: `\childdocjob'\par
\fi
version: \version\par
\end{center}
\newpage
%    \end{macrocode}

% Manually include selected file,
% otherwise process as usual:
%    \begin{macrocode}
\ifchilddocmanual
\section*{part `\childdocname'}
\input{\childdocname}
\else
%    \end{macrocode}

% Include the two chapters:
%    \begin{macrocode}
\include{cdocsch1}
\include{cdocsch2}
%    \end{macrocode}

% Include the two parts unless only chapters should be displayed:
%    \begin{macrocode}
\ifchilddoc\else
\section{part three}
\input{cdocspt3}
\section{part four}
\input{cdocspt4}
\fi
%    \end{macrocode}

% Process as usual until here:
%    \begin{macrocode}
\fi
%    \end{macrocode}

% End of document body:
%    \begin{macrocode}
\end{document}
%    \end{macrocode}
%\iffalse
%</samplemain>
%\fi
%
% %%%%%%%%%%%%%%%%%%%%%%%%%%%%%%%%%%%%%%
% \paragraph{Chapter Include Files.}
%
% The include files are called |cdocsch1.tex| and |cdocsch2.tex|.
%
%\iffalse
%<*samplechap1|samplechap2>
%\fi

% Optional override for |\version| flag:
%    \begin{macrocode}
%%\providecommand{\version}{final}
%    \end{macrocode}

% Include the main document:
%    \begin{macrocode}
\input{childdoc.def}
\childdocof{cdocsamp}
%    \end{macrocode}

%\iffalse
%</samplechap1|samplechap2>
%\fi
%
%\iffalse
%<*samplechap1>
%\fi
% Some text for chapter 1:
%    \begin{macrocode}
\section{one}
some text in chapter one
%    \end{macrocode}

%\iffalse
%</samplechap1>
%\fi
% Some text for chapter 2:
%\iffalse
%<*samplechap2>
%\fi
%    \begin{macrocode}
\section{two}
more text in chapter two
%    \end{macrocode}

%\iffalse
%</samplechap2>
%\fi
%
% %%%%%%%%%%%%%%%%%%%%%%%%%%%%%%%%%%%%%%
% \paragraph{Part Include Files.}
%
% The include files are called |cdocspt3.tex| and |cdocspt4.tex|.
%
%\iffalse
%<*samplepart3|samplepart4>
%\fi

% Optional override for |\version| flag:
%    \begin{macrocode}
%%\providecommand{\version}{final}
%    \end{macrocode}

% Include the main document:
%    \begin{macrocode}
\input{childdoc.def}
\childdocby{cdocsamp}
%    \end{macrocode}

%\iffalse
%</samplepart3|samplepart4>
%\fi
%
%\iffalse
%<*samplepart3>
%\fi
% Some text for part 3:
%    \begin{macrocode}
some text in part three
%    \end{macrocode}

%\iffalse
%</samplepart3>
%\fi
% Some text for part 4:
%\iffalse
%<*samplepart4>
%\fi
%    \begin{macrocode}
more text in part four
%    \end{macrocode}

%\iffalse
%</samplepart4>
%\fi
%
% %%%%%%%%%%%%%%%%%%%%%%%%%%%%%%%%%%%%%%
% \paragraph{Forwarding for a Complete Draft.}
%
% The following forwarding file |cdocsdrf.tex|
% compiles the main document in draft mode:
%\iffalse
%<*sampledraft>
%\fi
%    \begin{macrocode}
\def\version{draft}
\input{childdoc.def}
\childdocforward{cdocsamp}
%    \end{macrocode}

%\iffalse
%</sampledraft>
%\fi
%
% %%%%%%%%%%%%%%%%%%%%%%%%%%%%%%%%%%%%%%
% \paragraph{Forwarding for Final Version of the Chapters.}
%
% The following forwarding files |cdocsfn1.tex| and |cdocsfn2.tex|
% (with identical content)
% compile the final versions of the child documents
% |cdocsch1.tex| and |cdocsch2.tex|, respectively:
%\iffalse
%<*samplefinal>
%\fi
%    \begin{macrocode}
\def\version{final}
\input{childdoc.def}
\childdocforwardprefix[cdocsamp]{cdocsfn}{cdocsch}
%    \end{macrocode}

%\iffalse
%</samplefinal>
%\fi
%
% %%%%%%%%%%%%%%%%%%%%%%%%%%%%%%%%%%%%%%
% \paragraph{Command Line Processing.}
%
% The following three command lines generate the output files
% |cdocscld|, |cdocscl1| and |cdocscl2|
% which should be identical to
% |cdocsdrf|, |cdocsch1| and |cdocsfn2|, respectively:
% \begin{center}
% \begin{tabular}{l}
% |latex -jobname cdocscld \|\\
% |  "\def\version{draft}\input{childdoc.def}\childdocforward{cdocsamp}"|\\
% |latex -jobname cdocscl1 \|\\
% |  "\input{childdoc.def}\childdocforward[cdocsamp]{cdocsch1}"|\\
% |latex -jobname cdocscl2 \|\\
% |  "\def\version{final}\input{childdoc.def}\childdocforward{cdocsch2}"|
% \end{tabular}
% \end{center}
% Note that the trailing backslash on each first line
% merely continues the input to the second line
% (for convenient cut ant paste).
% Furthermore, the command |latex| can be replaced by any
% of its alternative versions such as |pdflatex|.
%
% %%%%%%%%%%%%%%%%%%%%%%%%%%%%%%%%%%%%%%%%%%%%%%%%%%%%%%%%%%%%%%%%%%%%%%%%%%%%%%
% %%%%%%%%%%%%%%%%%%%%%%%%%%%%%%%%%%%%%%%%%%%%%%%%%%%%%%%%%%%%%%%%%%%%%%%%%%%%%%
% \section{Implementation}
%\iffalse
%<*package>
%\fi
%
% This section describes the definitions file |childdoc.def|.

% The definitions cannot be loaded using |\usepackage| or |\RequirePackage|
% which has a mechanism to prevent loading a style file more than once.
% When loading the definitions by means of |\input|
% multiple instances have to be prevented manually:
%\iffalse
%This code needs to be before the `\ProvidesFile' directive
%which is defined at the beginning of this file.
%Therefore it is also placed there and commented out here.
%</package>
%<*discard>
%\fi
%    \begin{macrocode}
\ifdefined\childdocmain\endinput\fi
%    \end{macrocode}
%\iffalse
%</discard>
%<*package>
%\fi
%
% \macro{\ifchilddoc}
% \macro{\ifchilddocmanual}
% The conditional |\ifchilddoc| tells whether a
% child (true) or main (false) document is being compiled.
% The conditional |\ifchilddocmanual| tells whether
% the |\includeonly| mechanism is used (false) or
% the selection of child files must be performed manually (true).
% The definitions initialise to false:
%    \begin{macrocode}
\newif\ifchilddoc
\newif\ifchilddocmanual
%    \end{macrocode}

% \macro{\childdocname}
% \macro{\childdocjob}
% The macro |\childdocname| stores the name of the main document
% to be compiled. The macro |\childdocjob| stores the name of
% the document on which the \LaTeX{} compiler was originally invoked.
% The content of |\jobname| cannot be compared
% to filenames specified in the source due to different catcodes.
% The following code rescans |\jobname|, stores the result
% in |\childdocname| and saves a copy in |\childdocjob|:
%    \begin{macrocode}
\edef\childdocname{\scantokens\expandafter{\jobname\noexpand}}
\let\childdocjob\childdocname
%    \end{macrocode}

% \macro{\childdocdisable}
% The macro |\childdocdisable| prevents the main file
% from being processed more than once.
% At this stage, the main document command |\childdocmain|
% is assumed to be called once again where it should do nothing.
% Any subsequent call to it should prevent
% a secondary processing of the main document
% It overwrites the forwarding commands
% |\childdocof| and |\childdocforward|
% with empty macros to prevent further inclusions of the main document:
%    \begin{macrocode}
\newcommand{\childdocdisable}
{
  \renewcommand{\childdocmain}[1]{\renewcommand{\childdocmain}[1]{\endinput}}
  \renewcommand{\childdocof}[1]{}
  \renewcommand{\childdocby}[2][]{}
  \renewcommand{\childdocforward}[2][]{}
  \renewcommand{\childdocdisable}{}
}
%    \end{macrocode}

% \macro{\childdocmain}
% The macro |\childdocmain| is to be called at the top of the main file
% with nothing or the main filename (without extension) as argument.
% First, it breaks loops.
% If the argument is not empty and does not match |\childdocname|
% (which is set by the first inclusion of |childdoc.def|),
% |\ifchilddoc| is set to true, |\includeonly| is applied to the child file
% and |\jobname| is set to the main file
% (for proper handling of |.aux| files):
%    \begin{macrocode}
\newcommand{\childdocmain}[1]
{
  \childdocdisable\childdocmain{}
  \if?#1?\else
    \begingroup
      \def\childdoctmp{#1}
      \ifx\childdoctmp\childdocname
        \def\childdoctmp{}
      \else
        \def\childdoctmp
        {
          \childdoctrue
          \includeonly{\childdocname}
          \def\childdocjob{#1}
          \def\jobname{#1}
        }
      \fi
      \expandafter
    \endgroup
    \childdoctmp
  \fi
}
%    \end{macrocode}

% \macro{\childdocof}
% The command |\childdocof| redirects
% compilation to the main file |#1|.
%    \begin{macrocode}
\newcommand{\childdocof}[1]
{
  \childdocdisable
  \childdoctrue
  \includeonly{\childdocname}
  \def\jobname{#1}
  \def\childdocjob{#1}
  \input{#1}
}
%    \end{macrocode}

% \macro{\childdocby}
% The command |\childdocby| ....
%    \begin{macrocode}
\newcommand{\childdocby}[2][]
{
  \childdocdisable
  \childdoctrue
  \childdocmanualtrue
  \if?#1?\else
    \def\jobname{#2}
  \fi
  \def\childdocjob{#2}
  \input{#2}
  \endinput
}
%    \end{macrocode}

% \macro{\childdocforward}
% The command |\childdocforward| redirects
% compilation to the main file or
% (if the optional argument is given) a child file.
% Parameters are set as if the main file
% or a child file starting with |\childdocof| was compiled.
% Then compilation is handed over to the main file:
%    \begin{macrocode}
\newcommand{\childdocforward}[2][]
{
  \begingroup
    \if?#1?
      \def\childdoctmp
      {
        \def\childdocname{#2}
        \def\childdocjob{#2}
        \def\jobname{#2}
        \input{#2}
        \endinput
      }
    \else
      \def\childdoctmp
      {
        \childdocdisable
        \def\childdocname{#2}
        \childdoctrue
        \includeonly{#2}
        \def\childdocjob{#1}
        \def\jobname{#1}
        \input{#1}
        \endinput
      }
    \fi
    \expandafter
  \endgroup
  \childdoctmp
}
%    \end{macrocode}

% \macro{\childdocforwardprefix}
% The command |\childdocforwardprefix| redirects
% compilation to the main or a child file by means of a pattern.
% The prefix |#1| in the current filename is replaced by |#2|
% and the suffix of the current filename is kept
% (it is assumed that the filename does not contain the substring `|~~~|'
% which is used as a delimiter).
% Compilation is handed over to the new file by |\childdocforward|:
%    \begin{macrocode}
\newcommand{\childdocforwardprefix}[3][]
{
  \begingroup
    \def\childdocextract #2##1~~~{\def\childdoctmp{\childdocforward[#1]{#3##1}}}
    \expandafter\childdocextract\childdocname~~~
    \expandafter
  \endgroup
  \childdoctmp
}
%    \end{macrocode}

% \macro{\childdoc}
% The deprecated macro |\childdoc| is a legacy version of |\childdocmain|:
%    \begin{macrocode}
\newcommand{\childdoc}{\childdocmain}
%    \end{macrocode}

% \macro{\childdocredirect}
% The deprecated macro |\childdocredirect| is a legacy version
% of |\childdocforward| and |\childdocforwardprefix|:
%    \begin{macrocode}
\newcommand{\childdocredirect}[2][]
{
  \begingroup
    \if?#1?
      \def\childdoctmp{\childdocforward{#2}}
    \else
      \def\childdoctmp{\childdocforwardprefix{#1}{#2}}
    \fi
    \expandafter
  \endgroup
  \childdoctmp
}
%    \end{macrocode}

%\iffalse
%</package>
%\fi
%
\endinput
|
and perform the replacements as outlined below.
Instead of |\childdocmain{|\textit{main}|}| add the following code
to the top of the main file:
%
\begin{center}
\begin{tabular}{l}
|\||ifdefined\childdocname\endinput\||fi\newif\ifchilddoc|\\
|\edef\childdocname{\scantokens\expandafter{\jobname\noexpand}}|\\
|\def\childdocmain{|\textit{main}|}\||ifx\childdocmain\childdocname\||else|\\
|\childdoctrue\includeonly{\childdocname}\let\jobname\childdocmain\||fi|\\
\end{tabular}
\end{center}
%
Instead of |\childdocof{|\textit{main}|}| just include the main file
at the top of each child file:
%
\begin{center}
|\input{|\textit{main}|}|
\end{center}
%
A simple redirection |\childdocforward{|\textit{dest}|}| is achieved by:
%
\begin{center}
|\def\jobname{|\textit{dest}|}\input{\jobname}|
\end{center}
%
The redirection with prefix
|\childdocforwardprefix[|\textit{prefix}|]{|\textit{dest}|}|
is accomplished by:
%
\begin{center}
\begin{tabular}{l}
|{\edef\jobname{\scantokens\expandafter{\jobname\noexpand}}|\\
|\def\redirectjob |\textit{prefix}|#1~~~{\gdef\jobname{|\textit{dest}|#1}}|\\
|\expandafter\redirectjob\jobname~~~}\input{\jobname}|
\end{tabular}
\end{center}

In an alternative approach,
child documents can be compiled by a specific command line
without additional code or specific definitions:
%
\begin{center}
|... -jobname "|\textit{target}|" "|[\textit{flags}]%
|\includeonly{|\textit{dest}|}\input{|\textit{main}|}"|
\end{center}
%

%%%%%%%%%%%%%%%%%%%%%%%%%%%%%%%%%%%%%%%%%%%%%%%%%%%%%%%%%%%%%%%%%%%%%%%%%%%%%%%%
%%%%%%%%%%%%%%%%%%%%%%%%%%%%%%%%%%%%%%%%%%%%%%%%%%%%%%%%%%%%%%%%%%%%%%%%%%%%%%%%
\section{Information}

%%%%%%%%%%%%%%%%%%%%%%%%%%%%%%%%%%%%%%%%%%%%%%%%%%%%%%%%%%%%%%%%%%%%%%%%%%%%%%%%
\subsection{Copyright}

Copyright \copyright{} 2017--2018 Niklas Beisert

This work may be distributed and/or modified under the
conditions of the \LaTeX{} Project Public License, either version 1.3
of this license or (at your option) any later version.
The latest version of this license is in
  \url{http://www.latex-project.org/lppl.txt}
and version 1.3 or later is part of all distributions of \LaTeX{}
version 2005/12/01 or later.

This work has the LPPL maintenance status `maintained'.

The Current Maintainer of this work is Niklas Beisert.

This work consists of the files |README.txt|, |childdoc.ins| and |childdoc.dtx|
as well as the derived files |childdoc.def|, |cdocsamp.tex|
with |cdocsch1.tex|, |cdocsch2.tex|, |cdocspt3.tex|, |cdocspt4.tex|,
|cdocsdrf.tex|, |cdocsfn1.tex|, |cdocsfn2.tex|
as well as |childdoc.pdf|.

%%%%%%%%%%%%%%%%%%%%%%%%%%%%%%%%%%%%%%%%%%%%%%%%%%%%%%%%%%%%%%%%%%%%%%%%%%%%%%%%
\subsection{Files and Installation}

The package consists of the files:
%
\begin{center}
\begin{tabular}{ll}
    |README.txt|   & readme file \\
    |childdoc.ins| & installation file \\
    |childdoc.dtx| & source file \\
    |childdoc.def| & definition file \\
    |cdocsamp.tex| & sample main file \\
    |cdocsch1.tex| & sample include file \\
    |cdocsch2.tex| & sample include file \\
    |cdocspt3.tex| & sample part file \\
    |cdocspt4.tex| & sample part file \\
    |cdocsdrf.tex| & sample redirection file \\
    |cdocsfn1.tex| & sample redirection file \\
    |cdocsfn2.tex| & sample redirection file \\
    |childdoc.pdf| & manual
\end{tabular}
\end{center}
%
The distribution consists of the files
|README.txt|, |childdoc.ins| and |childdoc.dtx|.
%
\begin{itemize}
\item
Run (pdf)\LaTeX{} on |childdoc.dtx|
to compile the manual |childdoc.pdf| (this file).
\item
Run \LaTeX{} on |childdoc.ins| to create the definitions file |childdoc.def|
and the sample |cdocsamp.tex| with include files
|cdocsch1.tex|, |cdocsch2.tex|, |cdocspt3.tex|, |cdocspt4.tex|,
|cdocsdrf.tex|, |cdocsfn1.tex|, |cdocsfn2.tex|.
Then copy the file |childdoc.def| to an appropriate directory of your \LaTeX{}
distribution, e.g.\ \textit{texmf-root}|/tex/latex/childdoc|.
\end{itemize}

%%%%%%%%%%%%%%%%%%%%%%%%%%%%%%%%%%%%%%%%%%%%%%%%%%%%%%%%%%%%%%%%%%%%%%%%%%%%%%%%
\subsection{Related CTAN Packages}

There are several other packages which offer a similar functionality:
%
\begin{itemize}
\item
The packages
\href{http://ctan.org/pkg/docmute}{\textsf{docmute}},
\href{http://ctan.org/pkg/includex}{\textsf{includex}} and
\href{http://ctan.org/pkg/standalone}{\textsf{standalone}}
provide commands to include only the document body of
a child file thus allowing both files to be compiled individually.
\item
The packages \href{http://ctan.org/pkg/subdocs}{\textsf{subdocs}}
and \href{http://ctan.org/pkg/subfiles}{\textsf{subfiles}}
provide structures in which the main and child documents can be
encapsulated and allowing them to be compiled individually.
The inclusion mechanism is different from the conventional |\include|.
\item
The package \href{http://ctan.org/pkg/combine}{\textsf{combine}}
is an elaborate solution to combine several documents into one.
\end{itemize}
%
See also the CTAN topic \href{http://ctan.org/topic/subdocs}{\textsf{subdocs}}
for further related packages.
The present package differs from the above solutions in that
a document structure constructed with the conventional |\include| mechanism
just needs two extra commands at the top of every file
such that all constituent files can be compiled individually.

%%%%%%%%%%%%%%%%%%%%%%%%%%%%%%%%%%%%%%%%%%%%%%%%%%%%%%%%%%%%%%%%%%%%%%%%%%%%%%%%
%\subsection{Feature Suggestions}
%
%The following is a list of features which may be useful for future
%versions of this package:
%%
%\begin{itemize}
%\item
%\ldots
%\end{itemize}

%%%%%%%%%%%%%%%%%%%%%%%%%%%%%%%%%%%%%%%%%%%%%%%%%%%%%%%%%%%%%%%%%%%%%%%%%%%%%%%%
\subsection{Revision History}

%%%%%%%%%%%%%%%%%%%%%%%%%%%%%%%%%%%%%%%%
\paragraph{v2.0:} 2018/12/30

\begin{itemize}
\item
immediate forward processing
\item
added |\childdocby| mechanism
\item
manual restructured
\end{itemize}

%%%%%%%%%%%%%%%%%%%%%%%%%%%%%%%%%%%%%%%%
\paragraph{v1.6:} 2018/01/17

\begin{itemize}
\item
application for development of include files
\item
corrections to manual
\end{itemize}

%%%%%%%%%%%%%%%%%%%%%%%%%%%%%%%%%%%%%%%%
\paragraph{v1.5:} 2017/05/21

\begin{itemize}
\item
more complete structuring introduced
\item
|\childdocof| introduced
\item
|\childdoc| renamed to |\childdocmain|
\item
|\childredirect| renamed to |\childdocforward| and |\childdocforwardprefix|
and functionality expanded
\end{itemize}

%%%%%%%%%%%%%%%%%%%%%%%%%%%%%%%%%%%%%%%%
\paragraph{v1.0:} 2017/04/27

\begin{itemize}
\item
manual and install package
\item
first version published on CTAN
\end{itemize}

%%%%%%%%%%%%%%%%%%%%%%%%%%%%%%%%%%%%%%%%
\paragraph{v0.6:} 2017/04/26

\begin{itemize}
\item
redirection mechanism added
\end{itemize}

%%%%%%%%%%%%%%%%%%%%%%%%%%%%%%%%%%%%%%%%
\paragraph{v0.5:} 2017/04/26

\begin{itemize}
\item
functionality in definition file
\end{itemize}


%%%%%%%%%%%%%%%%%%%%%%%%%%%%%%%%%%%%%%%%%%%%%%%%%%%%%%%%%%%%%%%%%%%%%%%%%%%%%%%%
%%%%%%%%%%%%%%%%%%%%%%%%%%%%%%%%%%%%%%%%%%%%%%%%%%%%%%%%%%%%%%%%%%%%%%%%%%%%%%%%
%%%%%%%%%%%%%%%%%%%%%%%%%%%%%%%%%%%%%%%%%%%%%%%%%%%%%%%%%%%%%%%%%%%%%%%%%%%%%%%%
\appendix

\settowidth\MacroIndent{\rmfamily\scriptsize 000\ }

 \DocInput{childdoc.dtx}

\end{document}
%</driver>
% \fi
%
% %%%%%%%%%%%%%%%%%%%%%%%%%%%%%%%%%%%%%%%%%%%%%%%%%%%%%%%%%%%%%%%%%%%%%%%%%%%%%%
% %%%%%%%%%%%%%%%%%%%%%%%%%%%%%%%%%%%%%%%%%%%%%%%%%%%%%%%%%%%%%%%%%%%%%%%%%%%%%%
% \section{Sample}
%\iffalse
%<*samplemain>
%\fi
%
% The following presents a sample document
% with two chapters, two parts, a title page,
% a compile flag as well as three forwarding files to set the flag.
% It consists of eight |.tex| files:
% \begin{center}
% \begin{tabular}{ll}
% |cdocsamp.tex|&main file\\
% |cdocsch1.tex|&include file for chapter 1\\
% |cdocsch2.tex|&include file for chapter 2\\
% |cdocspt3.tex|&include file for part 3\\
% |cdocspt4.tex|&include file for part 4\\
% |cdocsdrf.tex|&forwarding file for main file in draft mode\\
% |cdocsfi1.tex|&forwarding file for final version of chapter 1\\
% |cdocsfi2.tex|&forwarding file for final version of chapter 2\\
% \end{tabular}
% \end{center}
% Each of the eight files can be compiled directly by the \LaTeX{} compiler.
%
% %%%%%%%%%%%%%%%%%%%%%%%%%%%%%%%%%%%%%%
% \paragraph{Main File.}
%
% The main file is called |cdocsamp.tex|.
%
% Load the \textsf{childdoc} definitions and
% declare the filename for the main document:
%    \begin{macrocode}
% \iffalse
%
% childdoc.dtx Copyright (C) 2017-2018 Niklas Beisert
%
% This work may be distributed and/or modified under the
% conditions of the LaTeX Project Public License, either version 1.3
% of this license or (at your option) any later version.
% The latest version of this license is in
%   http://www.latex-project.org/lppl.txt
% and version 1.3 or later is part of all distributions of LaTeX
% version 2005/12/01 or later.
%
% This work has the LPPL maintenance status `maintained'.
%
% The Current Maintainer of this work is Niklas Beisert.
%
% This work consists of the files childdoc.dtx and childdoc.ins
% and the derived files childdoc.def and cdocsamp.tex with
% cdocsch1.tex, cdocsch2.tex, cdocsdrf.tex, cdocsfn1.tex, cdocsfn2.tex.
%
%<package>\ifdefined\childdocmain\endinput\fi
%<package>\ProvidesFile{childdoc.def}[2018/12/30 v2.0 child document driver]
%<samplemain>\ProvidesFile{cdocsamp.tex}[2018/12/30 v2.0 sample for childdoc]
%<*driver>
%\ProvidesFile{childdoc.drv}[2018/12/30 v2.0 childdoc reference manual file]
\PassOptionsToClass{10pt,a4paper}{article}
\documentclass{ltxdoc}

\usepackage[margin=35mm]{geometry}
\usepackage{hyperref}
\usepackage{hyperxmp}
\usepackage[usenames]{color}

\hypersetup{colorlinks=true}
\hypersetup{pdfstartview=FitH}
\hypersetup{pdfpagemode=UseNone}
\hypersetup{pdfsource={}}
\hypersetup{pdflang={en-UK}}
\hypersetup{pdfcopyright={Copyright 2017-2018 Niklas Beisert.
  This work may be distributed and/or modified under the
  conditions of the LaTeX Project Public License, either version 1.3
  of this license or (at your option) any later version.}}
\hypersetup{pdflicenseurl={http://www.latex-project.org/lppl.txt}}
\hypersetup{pdfcontactaddress={ETH Zurich, ITP, HIT K,
  Wolfgang-Pauli-Strasse 27}}
\hypersetup{pdfcontactpostcode={8093}}
\hypersetup{pdfcontactcity={Zurich}}
\hypersetup{pdfcontactcountry={Switzerland}}
\hypersetup{pdfcontactemail={nbeisert@itp.phys.ethz.ch}}
\hypersetup{pdfcontacturl={http://people.phys.ethz.ch/\xmptilde nbeisert/}}

\newcommand{\secref}[1]{\hyperref[#1]{section \ref*{#1}}}

\parskip1ex
\parindent0pt
\let\olditemize\itemize
\def\itemize{\olditemize\parskip0pt}

\begin{document}

\title{The \textsf{childdoc} Package}
\hypersetup{pdftitle={The childdoc Package}}
\author{Niklas Beisert\\[2ex]
  Institut f\"ur Theoretische Physik\\
  Eidgen\"ossische Technische Hochschule Z\"urich\\
  Wolfgang-Pauli-Strasse 27, 8093 Z\"urich, Switzerland\\[1ex]
  \href{mailto:nbeisert@itp.phys.ethz.ch}
  {\texttt{nbeisert@itp.phys.ethz.ch}}}
\hypersetup{pdfauthor={Niklas Beisert}}
\hypersetup{pdfsubject={Manual for the LaTeX2e Package childdoc}}
\date{30 December 2018, \textsf{v2.0}}
\maketitle

\begin{abstract}\noindent
\textsf{childdoc} is a \LaTeXe{} package
that enables the direct compilation
of document sections included by |\include|
to individual files.
\end{abstract}

\begingroup
\parskip0ex
\tableofcontents
\endgroup

%%%%%%%%%%%%%%%%%%%%%%%%%%%%%%%%%%%%%%%%%%%%%%%%%%%%%%%%%%%%%%%%%%%%%%%%%%%%%%%%
%%%%%%%%%%%%%%%%%%%%%%%%%%%%%%%%%%%%%%%%%%%%%%%%%%%%%%%%%%%%%%%%%%%%%%%%%%%%%%%%
\section{Introduction}

\LaTeX{} provides a mechanism to structure a large document (such as a book)
into a main file and several child files (containing the chapters)
using the |\include| command.
This mechanism is beneficial for documents
which span hundreds of pages in order to
make the source file(s) more manageable.
Moreover, compilation can be restricted to
selected child files by means of the |\includeonly| command.
The latter feature can be used to reduce the compilation time while editing
(this was significantly more useful in the earlier days of \LaTeX{})
or to generate a smaller document which is easier to navigate.
Another application of |\includeonly| is to generate
documents consisting of selected parts of the complete document.

However, there are a few drawbacks of the plain |\include| mechanism:
\begin{itemize}
\item
The child files cannot be compiled on their own,
they can only be compiled via the main file.
A naive editing environment
(such as a text editor with an option
to have the current file processed by \LaTeX)
may require one to switch to the main file before compiling;
attempting to compile the child file produces errors.
\item
The main file must be modified (each time)
to adjust the |\includeonly| command
to the present needs. This easily leaves the main file in a messy state.
\item
The generated document will always carry the filename
of the main document. This is inconvenient if
several child files are to be compiled and
to be kept for distribution.
\end{itemize}

The present package provides a simple interface
to make child files individually compilable by \LaTeX{}.
Compiling a child file then has the same effect as compiling
the main file with an |\includeonly| command
to select the appropriate child.
Moreover the generated document will carry the name of the child
rather than the main file.
This resolves all three above issues.

This feature is meant to make the editing of books,
thesis documents and lecture notes somewhat more convenient.
However, the package can also be used efficiently for
composing a series of documents (such as exercise sheets)
which are typically distributed individually.
It then assists the author in generating the individual documents
(potentially in different versions)
as well as a document containing the collected series.
Another application is in developing style files
or other kinds of included material
where compilation of the style file could redirect
to a sample or test file.

%%%%%%%%%%%%%%%%%%%%%%%%%%%%%%%%%%%%%%%%%%%%%%%%%%%%%%%%%%%%%%%%%%%%%%%%%%%%%%%%
%%%%%%%%%%%%%%%%%%%%%%%%%%%%%%%%%%%%%%%%%%%%%%%%%%%%%%%%%%%%%%%%%%%%%%%%%%%%%%%%
\section{Usage}

First of all, the package \textsf{childdoc} is \emph{not} a standard
\LaTeXe{} |.sty| style file! Therefore it needs to be invoked in
a non-standard way.

%%%%%%%%%%%%%%%%%%%%%%%%%%%%%%%%%%%%%%%%%%%%%%%%%%%%%%%%%%%%%%%%%%%%%%%%%%%%%%%%
\subsection{Included Files}
\label{sec:include}

%%%%%%%%%%%%%%%%%%%%%%%%%%%%%%%%%%%%%%%%
\DescribeMacro{\childdocmain}
To use the package, add the commands
\begin{center}
\begin{tabular}{l}
|\input{childdoc.def}|\\
|\childdocmain{}|\\
\end{tabular}
\end{center}
at the very top of the main \LaTeX{} file,
in particular \emph{before} the |\documentclass| statement!
The argument of |\childdocmain| should be left empty
(but it must be present).

%%%%%%%%%%%%%%%%%%%%%%%%%%%%%%%%%%%%%%%%
\DescribeMacro{\childdocof}
Furthermore, add the commands
\begin{center}
\begin{tabular}{l}
|\input{childdoc.def}|\\
|\childdocof{|\textit{main}|}|\\
\end{tabular}
\end{center}
at the top of every child file \textit{child}
which is included by |\include{|\textit{child}|}|
from within the main file
(or at least for those files to be compiled individually).
The argument \textit{main} must be the filename of the main file.

There are a couple of
considerations in setting up the main and child documents:

%%%%%%%%%%%%%%%%%%%%%%%%%%%%%%%%%%%%%%%%
\paragraph{Restrictions.}

Please note the following restrictions:
\begin{itemize}
\item
|\childdocmain| must be called with one argument \textit{main}
to ensure compatibility with earlier version of the package.
It must either be empty (|\childdocmain{}|)
or precisely match the filename of the main file in which it is specified.
See \secref{sec:detection} for further information.
\item
The filename \textit{main} must be specified without the |.tex| extension.
\item
The filename \textit{main} is case sensitive
(even in case-insensitive file systems)
due to internal string comparison.
\item
The argument \textit{main} should be fully expanded, it cannot be a macro.
\item
Subdirectories and special characters should be avoided in filenames.
\item
The command |\childdocmain{|\textit{main}|}| must be followed by a whitespace.
It should not be followed immediately by another command
or by a comment mark `|%|'.
This is because the \TeX{} parser reads the token immediately following
the argument of |\childdocmain| and puts it
at the beginning of every child section;
however, a white\-space is ignored.
\end{itemize}

%%%%%%%%%%%%%%%%%%%%%%%%%%%%%%%%%%%%%%%%
\paragraph{Content of Main File.}

It is advisable to place all content in the child files included by |\include|.
Any output contained in the main file will appear in all child documents
unless suppressed manually;
it cannot be suppressed automatically by the |\includeonly| directive
and thus should normally be avoided.
A method to include some content in the main file
by means of conditional processing is described in \secref{sec:conditional}.

%%%%%%%%%%%%%%%%%%%%%%%%%%%%%%%%%%%%%%%%
\paragraph{Page Numbering.}

When only a part of the document is compiled,
the appropriate numbering of pages
(as well as other status parameters)
is determined from the |.aux| files.
The latter contain information from previous passes.
However this information needs to propagate through
all intermediate child documents.
Therefore the page numbering in child documents may well
be inconsistent until the complete document is compiled at least once.

A useful (if unconventional) way to always ensure a consistent
page numbering is to restart the numbering in each child document
and denote the pages by `\textit{child}|.|\textit{page}'
where \textit{child} represents the chapter/section number of the child file.
This can be achieved by the command
|\numberwithin{page}{|\textit{child}|}|
of the \textsf{amsmath} package
where \textit{child} can be |chapter| or |section|
depending on the chosen structuring.
Alternatively, one can modify the macro |\thepage| appropriately
and reset the counter |page| at the start of each child file.

%%%%%%%%%%%%%%%%%%%%%%%%%%%%%%%%%%%%%%%%%%%%%%%%%%%%%%%%%%%%%%%%%%%%%%%%%%%%%%%%
\subsection{Conditional Processing}
\label{sec:conditional}

The package provides a mechanism to compile different versions
of a document. To customise the versions further some conditional processing
can come in handy to distinguish which version is being compiled.
The package provides two macros to describe the compilation context:

%%%%%%%%%%%%%%%%%%%%%%%%%%%%%%%%%%%%%%%%
\DescribeMacro{\ifchilddoc}
The conditional |\ifchilddoc| distinguishes between the compilation of
child documents and the main document:
%
\begin{center}
|\ifchilddoc |\textit{child-code}| |[|\||else |\textit{main-code}]| \||fi|
\end{center}

%%%%%%%%%%%%%%%%%%%%%%%%%%%%%%%%%%%%%%%%
\DescribeMacro{\childdocname}
\DescribeMacro{\childdocjob}
The macro |\childdocname| contains the filename (without extension)
of the main or child file being processed.
Note that |\childdocjob| will always contain the name of the main file.

%%%%%%%%%%%%%%%%%%%%%%%%%%%%%%%%%%%%%%%%
\paragraph{Title Page.}

Conditional processing can be used to include a title or banner page
in the main document when proper precautions are taken.
Importantly, the code in the main file should ensure that the page counter
(as well as other status parameters which are stored in the |.aux| files)
takes the same value after the conditional processing.
Otherwise the page numbers may take divergent values
depending on which part is compiled.

For example, a title page could be declared by:
%
\begin{center}
\begin{tabular}{l}
|\ifchilddoc\||else|\\
|\addtocounter{page}{-1}|\\
\textit{code for title page}\\
|\newpage|\\
|\||fi|
\end{tabular}
\end{center}
%
A banner page for the child documents can be generated by:
%
\begin{center}
\begin{tabular}{l}
|\ifchilddoc|\\
|\addtocounter{page}{-1}|\\
\textit{code for banner page}\\
|\newpage|\\
|\||fi|
\end{tabular}
\end{center}
%
Here one could write a message such as:
\begin{center}
|This is the part \childdocname{} of \childdocjob{}.|
\end{center}

%%%%%%%%%%%%%%%%%%%%%%%%%%%%%%%%%%%%%%%%%%%%%%%%%%%%%%%%%%%%%%%%%%%%%%%%%%%%%%%%
\subsection{Flags}
\label{sec:flags}

The package makes it easy to generate different versions
of the main or child documents.
To this end compilation flags can be defined
and assigned different default values.
They will be particularly useful in conjunction
with the forwarding mechanism described in \secref{sec:forward}.

For example, it may be useful to have a flag |\version|
which can be set to |draft| or |final|.
The document source will contain some conditional code
depending on the value of |\version|.
Suppose further, the flag should default to |final| for the main file
and to |draft| for child files
which is a natural assignment for editing the document.
This is achieved by placing the following code
in the preamble of the main document
(below the |\childdocmain| directive):
%
\begin{center}
\begin{tabular}{l}
|\ifchilddoc|\\
|\providecommand{\version}{draft}|\\
|\||else|\\
|\providecommand{\version}{final}|\\
|\||fi|
\end{tabular}
\end{center}
%
The definition by |\providecommand| makes sure
that previous definitions are not overwritten.
Further statements |\providecommand{\version}{...}|
can thus be added before the above code to override it.

For the main file, one might add a line
(between |\childdocmain| and the above block)
%
\begin{center}
|%\ifchilddoc\||else\providecommand{\version}{draft}\||fi|
\end{center}
%
which can be uncommented to produce a draft version.
Likewise one can add a line to the very top of a child file
(above the |\childdocof{|\textit{main}|}| directive)
%
\begin{center}
|%\providecommand{\version}{final}|
\end{center}
%
which can be uncommented to produce the final version of this child document.

%%%%%%%%%%%%%%%%%%%%%%%%%%%%%%%%%%%%%%%%%%%%%%%%%%%%%%%%%%%%%%%%%%%%%%%%%%%%%%%%
\subsection{Forwarding}
\label{sec:forward}

Different versions of the main or child documents
using compilation flags as described in \secref{sec:flags}
can be (permanently) stored in different files
for convenient compilation, viewing and distribution.
To this end, the package defines a command
to pass on compilation to a different file:

%%%%%%%%%%%%%%%%%%%%%%%%%%%%%%%%%%%%%%%%
\DescribeMacro{\childdocforward}
The command |\childdocforward| redirects processing to
another source file:
%
\begin{center}
\begin{tabular}{l}
|\input{childdoc.def}|\\
|\childdocforward[|\textit{main}|]{|\textit{dest}|}|\\
\end{tabular}
\end{center}
%
The argument \textit{dest} is the destination file
(without extension).
It should be the main file or one of the child files.
Note that further \textsf{childdoc} directives
such as |\childdocof| and |\childdocforward|
in the indicated file will be processed in this form.
The optional argument \textit{main}
passes on directly to the main file \textit{main}
while pretending to compile the child \textit{dest}.
This form behaves as if \textit{dest}
issues |\childdocof{|\textit{main}|}| right away,
and no further \textsf{childdoc} directives will be processed.

%%%%%%%%%%%%%%%%%%%%%%%%%%%%%%%%%%%%%%%%
\DescribeMacro{\...prefix}
In the alternative form |\childdocforwardprefix|,
%
\begin{center}
\begin{tabular}{l}
|\input{childdoc.def}|\\
|\childdocforwardprefix[|\textit{main}|]{|\textit{prefix}|}{|\textit{dest}|}|
\end{tabular}
\end{center}
%
the destination file is determined by a pattern
depending on the current file:
To make this work, the current file must be called
`{\textit{prefix}\hspace{0.2em}\textit{suffix}}'
with \textit{prefix} matching precisely the argument.
Processing is then passed on to the file
`{\textit{dest}\hspace{0.2em}\textit{suffix}}'.
Surely, the same effect is achieved by
directly specifying the
argument `{\textit{dest}\hspace{0.2em}\textit{suffix}}'
in the first form.
However, that requires to set up a different file
for each child. With the alternative form of the command
all these files can have exactly the same content
which simplifies setting them up and maintaining them.

For example, the following file |draft.tex|
with a compilation flag |\version| as described in \secref{sec:flags}
compiles the main document as a draft:
%
\begin{center}
\begin{tabular}{l}
|\def\version{draft}|\\
|\input{childdoc.def}|\\
|\childdocforward{|\textit{main}|}|
\end{tabular}
\end{center}
%
Likewise, the following files |final|\textit{nn}|.tex|
compile the final version of the child document
|child|\textit{nn}|.tex|:
%
\begin{center}
\begin{tabular}{l}
|\def\version{final}|\\
|\input{childdoc.def}|\\
|\childdocforwardprefix{final}{child}|
\end{tabular}
\end{center}
%

Note that when several versions of a main file and/or of each child file
are to be generated, it may be convenient to set up a |Makefile| or
shell script to automatise the process.

%%%%%%%%%%%%%%%%%%%%%%%%%%%%%%%%%%%%%%%%%%%%%%%%%%%%%%%%%%%%%%%%%%%%%%%%%%%%%%%%
\subsection{Command Line Processing}
\label{sec:commandline}

The effect of redirection files can also be achieved by invoking
the \LaTeX{} compiler with a more elaborate command line.
Most conveniently this should be done as part
of a shell script or a |Makefile|.

When using \textsf{childdoc} in the main file, the following
command lines effectively perform a redirection
(note that depending on the shell being used,
backslashes may have to be doubled: `|\|' $\to$ `|\\|'):
%
\begin{center}
|... -jobname "|\textit{target}|" |\\|"|[\textit{flags}]%
|\input{childdoc.def}\childdocforward[|\textit{main}|]{|\textit{dest}|}"|
\end{center}
%
Here \textit{target} is the name of the output file,
\textit{main} is the name of the main file
and \textit{dest} is the name of the main or child file to be processed
(all filenames without extensions).
The optional argument \textit{main} can be omitted
if \textit{main} matches \textit{dest}.
Optionally, compilation \textit{flags} can be defined via |\def| commands.
This command line makes the \TeX{} engine believe
it is compiling the file \textit{target}
whose content is specified as the latter parameter.
The provided code then forwards the processing to
\textit{main} or \textit{dest} as described in \secref{sec:forward}.

%%%%%%%%%%%%%%%%%%%%%%%%%%%%%%%%%%%%%%%%%%%%%%%%%%%%%%%%%%%%%%%%%%%%%%%%%%%%%%%%
\subsection{Include by Input}
\label{sec:input}

Including child documents by |\include| has some restrictions by design.
Most notably, the content of a child document always occupies
its own set of pages; pages cannot be shared between child documents.
Usually, this behaviour makes perfect sense
because each child document contain an essential part of the document.
However, in some situations it may be desirable to compose
a document from a collection of parts
without having mandatory page breaks between then.
For this case, the package
provides a mechanism to include parts
by |\input| which can also be processed individually.
However, by construction this mechanism
requires manual handling of the content to be output.

%%%%%%%%%%%%%%%%%%%%%%%%%%%%%%%%%%%%%%%%
\DescribeMacro{\ifchilddocmanual}
The main file should be prepared as usual, see \secref{sec:include}.
However, the document body must make a distinction
between processing of an individual part and of the main document, e.g.:
%
\begin{center}
\begin{tabular}{l}
|\ifchilddocmanual|\\
|\input{\childdocname}|\\
|\||else|\\
\textit{document body with }|\input{|\textit{part}|}|\\
|\||fi|
\end{tabular}
\end{center}
%
The conditional |\ifchilddocmanual| is true whenever
a part to be included by |\input| is being compiled,
and the name of the part is stored in |\childdocname|.

%%%%%%%%%%%%%%%%%%%%%%%%%%%%%%%%%%%%%%%%
\DescribeMacro{\childdocby}
Each part to be included by |\input| should start with:
%
\begin{center}
\begin{tabular}{l}
|\input{childdoc.def}|\\
|\childdocby{|\textit{main}|}|\\
\end{tabular}
\end{center}
%
The directive |\childdocby| is similar to |\childdocof|
described in \secref{sec:include},
but the subsequent selection of content must be done manually.
To that end, both |\ifchilddoc| and |\ifchilddocmanual|
will be true upon processing of a part,
and the name of the part is stored in |\childdocname|.
Note that |\jobname| will be set to the filename of the current part
so that each part receives an individual |.aux| file
that does not interfere with the |.aux| file(s) of the main document.
This behaviour can be altered by the alternative form
|\childdocby[*]{|\textit{main}|}| (with a non-empty optional argument)
which uses the |.aux| file of the main document
by setting |\jobname| to \textit{main}.

%%%%%%%%%%%%%%%%%%%%%%%%%%%%%%%%%%%%%%%%%%%%%%%%%%%%%%%%%%%%%%%%%%%%%%%%%%%%%%%%
\subsection{Driver Development}
\label{sec:driver}

The \textsf{childdoc} mechanism can also be use for the development
of definition files such as \LaTeX{} styles or classes.
This case differs from the above setup with multiple parts
included by |\include| in that no |\includeonly| should be invoked.
This can be achieved by starting the include file
(before |\ProvidesPackage|) with:
%
\begin{center}
\begin{tabular}{l}
|\input{childdoc.def}|\\
|\childdocforward{|\textit{main}|}|\\
\end{tabular}
\end{center}
%
or alternatively with:
%
\begin{center}
\begin{tabular}{l}
|\input{childdoc.def}|\\
|\childdocby{|\textit{main}|}|\\
\end{tabular}
\end{center}
%
Both forms have slightly different effects as described above.
The main file is prepared as usual, see \secref{sec:include}.

%%%%%%%%%%%%%%%%%%%%%%%%%%%%%%%%%%%%%%%%%%%%%%%%%%%%%%%%%%%%%%%%%%%%%%%%%%%%%%%%
\subsection{Legacy Detection}
\label{sec:detection}

The directive |\childdocmain| in the main file can detect
whether the complete document or merely a child is to be compiled
even without using the directive |\childdocof|.
This method is deprecated because it is less robust
and there is no compelling reason to use it;
it is merely provided for backward compatibility
and it may be removed in future versions.

If the detection mechanism is to be used,
it is mandatory to correctly specify
the filename of the main file as the argument of |\childdocmain|:
%
\begin{center}
\begin{tabular}{l}
|\input{childdoc.def}|\\
|\childdocmain{|\textit{main}|}|\\
\end{tabular}
\end{center}
%
If |\jobname| does not match the argument \textit{main} of |\childdocmain|,
it is assumed that |\jobname| points to the child file to be compiled.
When using |\childdocmain| with the main file specified as argument,
it suffices to start a child file
with just |\input{|\textit{main}|}|
without loading of the package and using |\childdocof|.
If instead all processing is done
with the appropriate \textsf{childdoc} directives,
the argument of \textit{main} of |\childdocmain| can be empty.

An alternative version of the command line processing described
in \secref{sec:commandline} using the detection mechanism reads:
%
\begin{center}
|... -jobname "|\textit{target}|" "|[\textit{flags}]%
[|\def\jobname{|\textit{dest}|}|]|\input{|\textit{main}|}"|
\end{center}

%%%%%%%%%%%%%%%%%%%%%%%%%%%%%%%%%%%%%%%%%%%%%%%%%%%%%%%%%%%%%%%%%%%%%%%%%%%%%%%%
\subsection{Manual Code}
\label{sec:manual}

In case one cannot be certain whether the definitions file |childdoc.def|
is installed on the target \TeX{} distribution
and one prefers not to ship it,
it is conceivable to paste a few relevant commands into the sources.

To that end, drop all statements |\input{childdoc.def}|
and perform the replacements as outlined below.
Instead of |\childdocmain{|\textit{main}|}| add the following code
to the top of the main file:
%
\begin{center}
\begin{tabular}{l}
|\||ifdefined\childdocname\endinput\||fi\newif\ifchilddoc|\\
|\edef\childdocname{\scantokens\expandafter{\jobname\noexpand}}|\\
|\def\childdocmain{|\textit{main}|}\||ifx\childdocmain\childdocname\||else|\\
|\childdoctrue\includeonly{\childdocname}\let\jobname\childdocmain\||fi|\\
\end{tabular}
\end{center}
%
Instead of |\childdocof{|\textit{main}|}| just include the main file
at the top of each child file:
%
\begin{center}
|\input{|\textit{main}|}|
\end{center}
%
A simple redirection |\childdocforward{|\textit{dest}|}| is achieved by:
%
\begin{center}
|\def\jobname{|\textit{dest}|}\input{\jobname}|
\end{center}
%
The redirection with prefix
|\childdocforwardprefix[|\textit{prefix}|]{|\textit{dest}|}|
is accomplished by:
%
\begin{center}
\begin{tabular}{l}
|{\edef\jobname{\scantokens\expandafter{\jobname\noexpand}}|\\
|\def\redirectjob |\textit{prefix}|#1~~~{\gdef\jobname{|\textit{dest}|#1}}|\\
|\expandafter\redirectjob\jobname~~~}\input{\jobname}|
\end{tabular}
\end{center}

In an alternative approach,
child documents can be compiled by a specific command line
without additional code or specific definitions:
%
\begin{center}
|... -jobname "|\textit{target}|" "|[\textit{flags}]%
|\includeonly{|\textit{dest}|}\input{|\textit{main}|}"|
\end{center}
%

%%%%%%%%%%%%%%%%%%%%%%%%%%%%%%%%%%%%%%%%%%%%%%%%%%%%%%%%%%%%%%%%%%%%%%%%%%%%%%%%
%%%%%%%%%%%%%%%%%%%%%%%%%%%%%%%%%%%%%%%%%%%%%%%%%%%%%%%%%%%%%%%%%%%%%%%%%%%%%%%%
\section{Information}

%%%%%%%%%%%%%%%%%%%%%%%%%%%%%%%%%%%%%%%%%%%%%%%%%%%%%%%%%%%%%%%%%%%%%%%%%%%%%%%%
\subsection{Copyright}

Copyright \copyright{} 2017--2018 Niklas Beisert

This work may be distributed and/or modified under the
conditions of the \LaTeX{} Project Public License, either version 1.3
of this license or (at your option) any later version.
The latest version of this license is in
  \url{http://www.latex-project.org/lppl.txt}
and version 1.3 or later is part of all distributions of \LaTeX{}
version 2005/12/01 or later.

This work has the LPPL maintenance status `maintained'.

The Current Maintainer of this work is Niklas Beisert.

This work consists of the files |README.txt|, |childdoc.ins| and |childdoc.dtx|
as well as the derived files |childdoc.def|, |cdocsamp.tex|
with |cdocsch1.tex|, |cdocsch2.tex|, |cdocspt3.tex|, |cdocspt4.tex|,
|cdocsdrf.tex|, |cdocsfn1.tex|, |cdocsfn2.tex|
as well as |childdoc.pdf|.

%%%%%%%%%%%%%%%%%%%%%%%%%%%%%%%%%%%%%%%%%%%%%%%%%%%%%%%%%%%%%%%%%%%%%%%%%%%%%%%%
\subsection{Files and Installation}

The package consists of the files:
%
\begin{center}
\begin{tabular}{ll}
    |README.txt|   & readme file \\
    |childdoc.ins| & installation file \\
    |childdoc.dtx| & source file \\
    |childdoc.def| & definition file \\
    |cdocsamp.tex| & sample main file \\
    |cdocsch1.tex| & sample include file \\
    |cdocsch2.tex| & sample include file \\
    |cdocspt3.tex| & sample part file \\
    |cdocspt4.tex| & sample part file \\
    |cdocsdrf.tex| & sample redirection file \\
    |cdocsfn1.tex| & sample redirection file \\
    |cdocsfn2.tex| & sample redirection file \\
    |childdoc.pdf| & manual
\end{tabular}
\end{center}
%
The distribution consists of the files
|README.txt|, |childdoc.ins| and |childdoc.dtx|.
%
\begin{itemize}
\item
Run (pdf)\LaTeX{} on |childdoc.dtx|
to compile the manual |childdoc.pdf| (this file).
\item
Run \LaTeX{} on |childdoc.ins| to create the definitions file |childdoc.def|
and the sample |cdocsamp.tex| with include files
|cdocsch1.tex|, |cdocsch2.tex|, |cdocspt3.tex|, |cdocspt4.tex|,
|cdocsdrf.tex|, |cdocsfn1.tex|, |cdocsfn2.tex|.
Then copy the file |childdoc.def| to an appropriate directory of your \LaTeX{}
distribution, e.g.\ \textit{texmf-root}|/tex/latex/childdoc|.
\end{itemize}

%%%%%%%%%%%%%%%%%%%%%%%%%%%%%%%%%%%%%%%%%%%%%%%%%%%%%%%%%%%%%%%%%%%%%%%%%%%%%%%%
\subsection{Related CTAN Packages}

There are several other packages which offer a similar functionality:
%
\begin{itemize}
\item
The packages
\href{http://ctan.org/pkg/docmute}{\textsf{docmute}},
\href{http://ctan.org/pkg/includex}{\textsf{includex}} and
\href{http://ctan.org/pkg/standalone}{\textsf{standalone}}
provide commands to include only the document body of
a child file thus allowing both files to be compiled individually.
\item
The packages \href{http://ctan.org/pkg/subdocs}{\textsf{subdocs}}
and \href{http://ctan.org/pkg/subfiles}{\textsf{subfiles}}
provide structures in which the main and child documents can be
encapsulated and allowing them to be compiled individually.
The inclusion mechanism is different from the conventional |\include|.
\item
The package \href{http://ctan.org/pkg/combine}{\textsf{combine}}
is an elaborate solution to combine several documents into one.
\end{itemize}
%
See also the CTAN topic \href{http://ctan.org/topic/subdocs}{\textsf{subdocs}}
for further related packages.
The present package differs from the above solutions in that
a document structure constructed with the conventional |\include| mechanism
just needs two extra commands at the top of every file
such that all constituent files can be compiled individually.

%%%%%%%%%%%%%%%%%%%%%%%%%%%%%%%%%%%%%%%%%%%%%%%%%%%%%%%%%%%%%%%%%%%%%%%%%%%%%%%%
%\subsection{Feature Suggestions}
%
%The following is a list of features which may be useful for future
%versions of this package:
%%
%\begin{itemize}
%\item
%\ldots
%\end{itemize}

%%%%%%%%%%%%%%%%%%%%%%%%%%%%%%%%%%%%%%%%%%%%%%%%%%%%%%%%%%%%%%%%%%%%%%%%%%%%%%%%
\subsection{Revision History}

%%%%%%%%%%%%%%%%%%%%%%%%%%%%%%%%%%%%%%%%
\paragraph{v2.0:} 2018/12/30

\begin{itemize}
\item
immediate forward processing
\item
added |\childdocby| mechanism
\item
manual restructured
\end{itemize}

%%%%%%%%%%%%%%%%%%%%%%%%%%%%%%%%%%%%%%%%
\paragraph{v1.6:} 2018/01/17

\begin{itemize}
\item
application for development of include files
\item
corrections to manual
\end{itemize}

%%%%%%%%%%%%%%%%%%%%%%%%%%%%%%%%%%%%%%%%
\paragraph{v1.5:} 2017/05/21

\begin{itemize}
\item
more complete structuring introduced
\item
|\childdocof| introduced
\item
|\childdoc| renamed to |\childdocmain|
\item
|\childredirect| renamed to |\childdocforward| and |\childdocforwardprefix|
and functionality expanded
\end{itemize}

%%%%%%%%%%%%%%%%%%%%%%%%%%%%%%%%%%%%%%%%
\paragraph{v1.0:} 2017/04/27

\begin{itemize}
\item
manual and install package
\item
first version published on CTAN
\end{itemize}

%%%%%%%%%%%%%%%%%%%%%%%%%%%%%%%%%%%%%%%%
\paragraph{v0.6:} 2017/04/26

\begin{itemize}
\item
redirection mechanism added
\end{itemize}

%%%%%%%%%%%%%%%%%%%%%%%%%%%%%%%%%%%%%%%%
\paragraph{v0.5:} 2017/04/26

\begin{itemize}
\item
functionality in definition file
\end{itemize}


%%%%%%%%%%%%%%%%%%%%%%%%%%%%%%%%%%%%%%%%%%%%%%%%%%%%%%%%%%%%%%%%%%%%%%%%%%%%%%%%
%%%%%%%%%%%%%%%%%%%%%%%%%%%%%%%%%%%%%%%%%%%%%%%%%%%%%%%%%%%%%%%%%%%%%%%%%%%%%%%%
%%%%%%%%%%%%%%%%%%%%%%%%%%%%%%%%%%%%%%%%%%%%%%%%%%%%%%%%%%%%%%%%%%%%%%%%%%%%%%%%
\appendix

\settowidth\MacroIndent{\rmfamily\scriptsize 000\ }

 \DocInput{childdoc.dtx}

\end{document}
%</driver>
% \fi
%
% %%%%%%%%%%%%%%%%%%%%%%%%%%%%%%%%%%%%%%%%%%%%%%%%%%%%%%%%%%%%%%%%%%%%%%%%%%%%%%
% %%%%%%%%%%%%%%%%%%%%%%%%%%%%%%%%%%%%%%%%%%%%%%%%%%%%%%%%%%%%%%%%%%%%%%%%%%%%%%
% \section{Sample}
%\iffalse
%<*samplemain>
%\fi
%
% The following presents a sample document
% with two chapters, two parts, a title page,
% a compile flag as well as three forwarding files to set the flag.
% It consists of eight |.tex| files:
% \begin{center}
% \begin{tabular}{ll}
% |cdocsamp.tex|&main file\\
% |cdocsch1.tex|&include file for chapter 1\\
% |cdocsch2.tex|&include file for chapter 2\\
% |cdocspt3.tex|&include file for part 3\\
% |cdocspt4.tex|&include file for part 4\\
% |cdocsdrf.tex|&forwarding file for main file in draft mode\\
% |cdocsfi1.tex|&forwarding file for final version of chapter 1\\
% |cdocsfi2.tex|&forwarding file for final version of chapter 2\\
% \end{tabular}
% \end{center}
% Each of the eight files can be compiled directly by the \LaTeX{} compiler.
%
% %%%%%%%%%%%%%%%%%%%%%%%%%%%%%%%%%%%%%%
% \paragraph{Main File.}
%
% The main file is called |cdocsamp.tex|.
%
% Load the \textsf{childdoc} definitions and
% declare the filename for the main document:
%    \begin{macrocode}
\input{childdoc.def}
\childdocmain{}
%    \end{macrocode}

% Optional override for |\version| flag:
%    \begin{macrocode}
%%\ifchilddoc\else\providecommand{\version}{draft}\fi
%    \end{macrocode}

% Define the default values for the |\version| flag
% (|final| for the main file and |draft| for childs):
%    \begin{macrocode}
\ifchilddoc
\providecommand{\version}{draft}
\else
\providecommand{\version}{final}
\fi
%    \end{macrocode}

% Load the standard document class:
%    \begin{macrocode}
\documentclass[12pt]{article}
%    \end{macrocode}

% Start the document body:
%    \begin{macrocode}
\begin{document}
%    \end{macrocode}

% Declare a title page.
% Print title, part of document being processed and version flag:
%    \begin{macrocode}
\addtocounter{page}{-1}
\begin{center}
{\LARGE\bfseries{}childdoc example\par}
\vspace{1cm}
\ifchilddoc
\ifchilddocmanual part\else chapter\fi:
`\childdocname' of `\childdocjob'\par
\else
main document: `\childdocjob'\par
\fi
version: \version\par
\end{center}
\newpage
%    \end{macrocode}

% Manually include selected file,
% otherwise process as usual:
%    \begin{macrocode}
\ifchilddocmanual
\section*{part `\childdocname'}
\input{\childdocname}
\else
%    \end{macrocode}

% Include the two chapters:
%    \begin{macrocode}
\include{cdocsch1}
\include{cdocsch2}
%    \end{macrocode}

% Include the two parts unless only chapters should be displayed:
%    \begin{macrocode}
\ifchilddoc\else
\section{part three}
\input{cdocspt3}
\section{part four}
\input{cdocspt4}
\fi
%    \end{macrocode}

% Process as usual until here:
%    \begin{macrocode}
\fi
%    \end{macrocode}

% End of document body:
%    \begin{macrocode}
\end{document}
%    \end{macrocode}
%\iffalse
%</samplemain>
%\fi
%
% %%%%%%%%%%%%%%%%%%%%%%%%%%%%%%%%%%%%%%
% \paragraph{Chapter Include Files.}
%
% The include files are called |cdocsch1.tex| and |cdocsch2.tex|.
%
%\iffalse
%<*samplechap1|samplechap2>
%\fi

% Optional override for |\version| flag:
%    \begin{macrocode}
%%\providecommand{\version}{final}
%    \end{macrocode}

% Include the main document:
%    \begin{macrocode}
\input{childdoc.def}
\childdocof{cdocsamp}
%    \end{macrocode}

%\iffalse
%</samplechap1|samplechap2>
%\fi
%
%\iffalse
%<*samplechap1>
%\fi
% Some text for chapter 1:
%    \begin{macrocode}
\section{one}
some text in chapter one
%    \end{macrocode}

%\iffalse
%</samplechap1>
%\fi
% Some text for chapter 2:
%\iffalse
%<*samplechap2>
%\fi
%    \begin{macrocode}
\section{two}
more text in chapter two
%    \end{macrocode}

%\iffalse
%</samplechap2>
%\fi
%
% %%%%%%%%%%%%%%%%%%%%%%%%%%%%%%%%%%%%%%
% \paragraph{Part Include Files.}
%
% The include files are called |cdocspt3.tex| and |cdocspt4.tex|.
%
%\iffalse
%<*samplepart3|samplepart4>
%\fi

% Optional override for |\version| flag:
%    \begin{macrocode}
%%\providecommand{\version}{final}
%    \end{macrocode}

% Include the main document:
%    \begin{macrocode}
\input{childdoc.def}
\childdocby{cdocsamp}
%    \end{macrocode}

%\iffalse
%</samplepart3|samplepart4>
%\fi
%
%\iffalse
%<*samplepart3>
%\fi
% Some text for part 3:
%    \begin{macrocode}
some text in part three
%    \end{macrocode}

%\iffalse
%</samplepart3>
%\fi
% Some text for part 4:
%\iffalse
%<*samplepart4>
%\fi
%    \begin{macrocode}
more text in part four
%    \end{macrocode}

%\iffalse
%</samplepart4>
%\fi
%
% %%%%%%%%%%%%%%%%%%%%%%%%%%%%%%%%%%%%%%
% \paragraph{Forwarding for a Complete Draft.}
%
% The following forwarding file |cdocsdrf.tex|
% compiles the main document in draft mode:
%\iffalse
%<*sampledraft>
%\fi
%    \begin{macrocode}
\def\version{draft}
\input{childdoc.def}
\childdocforward{cdocsamp}
%    \end{macrocode}

%\iffalse
%</sampledraft>
%\fi
%
% %%%%%%%%%%%%%%%%%%%%%%%%%%%%%%%%%%%%%%
% \paragraph{Forwarding for Final Version of the Chapters.}
%
% The following forwarding files |cdocsfn1.tex| and |cdocsfn2.tex|
% (with identical content)
% compile the final versions of the child documents
% |cdocsch1.tex| and |cdocsch2.tex|, respectively:
%\iffalse
%<*samplefinal>
%\fi
%    \begin{macrocode}
\def\version{final}
\input{childdoc.def}
\childdocforwardprefix[cdocsamp]{cdocsfn}{cdocsch}
%    \end{macrocode}

%\iffalse
%</samplefinal>
%\fi
%
% %%%%%%%%%%%%%%%%%%%%%%%%%%%%%%%%%%%%%%
% \paragraph{Command Line Processing.}
%
% The following three command lines generate the output files
% |cdocscld|, |cdocscl1| and |cdocscl2|
% which should be identical to
% |cdocsdrf|, |cdocsch1| and |cdocsfn2|, respectively:
% \begin{center}
% \begin{tabular}{l}
% |latex -jobname cdocscld \|\\
% |  "\def\version{draft}\input{childdoc.def}\childdocforward{cdocsamp}"|\\
% |latex -jobname cdocscl1 \|\\
% |  "\input{childdoc.def}\childdocforward[cdocsamp]{cdocsch1}"|\\
% |latex -jobname cdocscl2 \|\\
% |  "\def\version{final}\input{childdoc.def}\childdocforward{cdocsch2}"|
% \end{tabular}
% \end{center}
% Note that the trailing backslash on each first line
% merely continues the input to the second line
% (for convenient cut ant paste).
% Furthermore, the command |latex| can be replaced by any
% of its alternative versions such as |pdflatex|.
%
% %%%%%%%%%%%%%%%%%%%%%%%%%%%%%%%%%%%%%%%%%%%%%%%%%%%%%%%%%%%%%%%%%%%%%%%%%%%%%%
% %%%%%%%%%%%%%%%%%%%%%%%%%%%%%%%%%%%%%%%%%%%%%%%%%%%%%%%%%%%%%%%%%%%%%%%%%%%%%%
% \section{Implementation}
%\iffalse
%<*package>
%\fi
%
% This section describes the definitions file |childdoc.def|.

% The definitions cannot be loaded using |\usepackage| or |\RequirePackage|
% which has a mechanism to prevent loading a style file more than once.
% When loading the definitions by means of |\input|
% multiple instances have to be prevented manually:
%\iffalse
%This code needs to be before the `\ProvidesFile' directive
%which is defined at the beginning of this file.
%Therefore it is also placed there and commented out here.
%</package>
%<*discard>
%\fi
%    \begin{macrocode}
\ifdefined\childdocmain\endinput\fi
%    \end{macrocode}
%\iffalse
%</discard>
%<*package>
%\fi
%
% \macro{\ifchilddoc}
% \macro{\ifchilddocmanual}
% The conditional |\ifchilddoc| tells whether a
% child (true) or main (false) document is being compiled.
% The conditional |\ifchilddocmanual| tells whether
% the |\includeonly| mechanism is used (false) or
% the selection of child files must be performed manually (true).
% The definitions initialise to false:
%    \begin{macrocode}
\newif\ifchilddoc
\newif\ifchilddocmanual
%    \end{macrocode}

% \macro{\childdocname}
% \macro{\childdocjob}
% The macro |\childdocname| stores the name of the main document
% to be compiled. The macro |\childdocjob| stores the name of
% the document on which the \LaTeX{} compiler was originally invoked.
% The content of |\jobname| cannot be compared
% to filenames specified in the source due to different catcodes.
% The following code rescans |\jobname|, stores the result
% in |\childdocname| and saves a copy in |\childdocjob|:
%    \begin{macrocode}
\edef\childdocname{\scantokens\expandafter{\jobname\noexpand}}
\let\childdocjob\childdocname
%    \end{macrocode}

% \macro{\childdocdisable}
% The macro |\childdocdisable| prevents the main file
% from being processed more than once.
% At this stage, the main document command |\childdocmain|
% is assumed to be called once again where it should do nothing.
% Any subsequent call to it should prevent
% a secondary processing of the main document
% It overwrites the forwarding commands
% |\childdocof| and |\childdocforward|
% with empty macros to prevent further inclusions of the main document:
%    \begin{macrocode}
\newcommand{\childdocdisable}
{
  \renewcommand{\childdocmain}[1]{\renewcommand{\childdocmain}[1]{\endinput}}
  \renewcommand{\childdocof}[1]{}
  \renewcommand{\childdocby}[2][]{}
  \renewcommand{\childdocforward}[2][]{}
  \renewcommand{\childdocdisable}{}
}
%    \end{macrocode}

% \macro{\childdocmain}
% The macro |\childdocmain| is to be called at the top of the main file
% with nothing or the main filename (without extension) as argument.
% First, it breaks loops.
% If the argument is not empty and does not match |\childdocname|
% (which is set by the first inclusion of |childdoc.def|),
% |\ifchilddoc| is set to true, |\includeonly| is applied to the child file
% and |\jobname| is set to the main file
% (for proper handling of |.aux| files):
%    \begin{macrocode}
\newcommand{\childdocmain}[1]
{
  \childdocdisable\childdocmain{}
  \if?#1?\else
    \begingroup
      \def\childdoctmp{#1}
      \ifx\childdoctmp\childdocname
        \def\childdoctmp{}
      \else
        \def\childdoctmp
        {
          \childdoctrue
          \includeonly{\childdocname}
          \def\childdocjob{#1}
          \def\jobname{#1}
        }
      \fi
      \expandafter
    \endgroup
    \childdoctmp
  \fi
}
%    \end{macrocode}

% \macro{\childdocof}
% The command |\childdocof| redirects
% compilation to the main file |#1|.
%    \begin{macrocode}
\newcommand{\childdocof}[1]
{
  \childdocdisable
  \childdoctrue
  \includeonly{\childdocname}
  \def\jobname{#1}
  \def\childdocjob{#1}
  \input{#1}
}
%    \end{macrocode}

% \macro{\childdocby}
% The command |\childdocby| ....
%    \begin{macrocode}
\newcommand{\childdocby}[2][]
{
  \childdocdisable
  \childdoctrue
  \childdocmanualtrue
  \if?#1?\else
    \def\jobname{#2}
  \fi
  \def\childdocjob{#2}
  \input{#2}
  \endinput
}
%    \end{macrocode}

% \macro{\childdocforward}
% The command |\childdocforward| redirects
% compilation to the main file or
% (if the optional argument is given) a child file.
% Parameters are set as if the main file
% or a child file starting with |\childdocof| was compiled.
% Then compilation is handed over to the main file:
%    \begin{macrocode}
\newcommand{\childdocforward}[2][]
{
  \begingroup
    \if?#1?
      \def\childdoctmp
      {
        \def\childdocname{#2}
        \def\childdocjob{#2}
        \def\jobname{#2}
        \input{#2}
        \endinput
      }
    \else
      \def\childdoctmp
      {
        \childdocdisable
        \def\childdocname{#2}
        \childdoctrue
        \includeonly{#2}
        \def\childdocjob{#1}
        \def\jobname{#1}
        \input{#1}
        \endinput
      }
    \fi
    \expandafter
  \endgroup
  \childdoctmp
}
%    \end{macrocode}

% \macro{\childdocforwardprefix}
% The command |\childdocforwardprefix| redirects
% compilation to the main or a child file by means of a pattern.
% The prefix |#1| in the current filename is replaced by |#2|
% and the suffix of the current filename is kept
% (it is assumed that the filename does not contain the substring `|~~~|'
% which is used as a delimiter).
% Compilation is handed over to the new file by |\childdocforward|:
%    \begin{macrocode}
\newcommand{\childdocforwardprefix}[3][]
{
  \begingroup
    \def\childdocextract #2##1~~~{\def\childdoctmp{\childdocforward[#1]{#3##1}}}
    \expandafter\childdocextract\childdocname~~~
    \expandafter
  \endgroup
  \childdoctmp
}
%    \end{macrocode}

% \macro{\childdoc}
% The deprecated macro |\childdoc| is a legacy version of |\childdocmain|:
%    \begin{macrocode}
\newcommand{\childdoc}{\childdocmain}
%    \end{macrocode}

% \macro{\childdocredirect}
% The deprecated macro |\childdocredirect| is a legacy version
% of |\childdocforward| and |\childdocforwardprefix|:
%    \begin{macrocode}
\newcommand{\childdocredirect}[2][]
{
  \begingroup
    \if?#1?
      \def\childdoctmp{\childdocforward{#2}}
    \else
      \def\childdoctmp{\childdocforwardprefix{#1}{#2}}
    \fi
    \expandafter
  \endgroup
  \childdoctmp
}
%    \end{macrocode}

%\iffalse
%</package>
%\fi
%
\endinput

\childdocmain{}
%    \end{macrocode}

% Optional override for |\version| flag:
%    \begin{macrocode}
%%\ifchilddoc\else\providecommand{\version}{draft}\fi
%    \end{macrocode}

% Define the default values for the |\version| flag
% (|final| for the main file and |draft| for childs):
%    \begin{macrocode}
\ifchilddoc
\providecommand{\version}{draft}
\else
\providecommand{\version}{final}
\fi
%    \end{macrocode}

% Load the standard document class:
%    \begin{macrocode}
\documentclass[12pt]{article}
%    \end{macrocode}

% Start the document body:
%    \begin{macrocode}
\begin{document}
%    \end{macrocode}

% Declare a title page.
% Print title, part of document being processed and version flag:
%    \begin{macrocode}
\addtocounter{page}{-1}
\begin{center}
{\LARGE\bfseries{}childdoc example\par}
\vspace{1cm}
\ifchilddoc
\ifchilddocmanual part\else chapter\fi:
`\childdocname' of `\childdocjob'\par
\else
main document: `\childdocjob'\par
\fi
version: \version\par
\end{center}
\newpage
%    \end{macrocode}

% Manually include selected file,
% otherwise process as usual:
%    \begin{macrocode}
\ifchilddocmanual
\section*{part `\childdocname'}
\input{\childdocname}
\else
%    \end{macrocode}

% Include the two chapters:
%    \begin{macrocode}
\include{cdocsch1}
\include{cdocsch2}
%    \end{macrocode}

% Include the two parts unless only chapters should be displayed:
%    \begin{macrocode}
\ifchilddoc\else
\section{part three}
\input{cdocspt3}
\section{part four}
\input{cdocspt4}
\fi
%    \end{macrocode}

% Process as usual until here:
%    \begin{macrocode}
\fi
%    \end{macrocode}

% End of document body:
%    \begin{macrocode}
\end{document}
%    \end{macrocode}
%\iffalse
%</samplemain>
%\fi
%
% %%%%%%%%%%%%%%%%%%%%%%%%%%%%%%%%%%%%%%
% \paragraph{Chapter Include Files.}
%
% The include files are called |cdocsch1.tex| and |cdocsch2.tex|.
%
%\iffalse
%<*samplechap1|samplechap2>
%\fi

% Optional override for |\version| flag:
%    \begin{macrocode}
%%\providecommand{\version}{final}
%    \end{macrocode}

% Include the main document:
%    \begin{macrocode}
% \iffalse
%
% childdoc.dtx Copyright (C) 2017-2018 Niklas Beisert
%
% This work may be distributed and/or modified under the
% conditions of the LaTeX Project Public License, either version 1.3
% of this license or (at your option) any later version.
% The latest version of this license is in
%   http://www.latex-project.org/lppl.txt
% and version 1.3 or later is part of all distributions of LaTeX
% version 2005/12/01 or later.
%
% This work has the LPPL maintenance status `maintained'.
%
% The Current Maintainer of this work is Niklas Beisert.
%
% This work consists of the files childdoc.dtx and childdoc.ins
% and the derived files childdoc.def and cdocsamp.tex with
% cdocsch1.tex, cdocsch2.tex, cdocsdrf.tex, cdocsfn1.tex, cdocsfn2.tex.
%
%<package>\ifdefined\childdocmain\endinput\fi
%<package>\ProvidesFile{childdoc.def}[2018/12/30 v2.0 child document driver]
%<samplemain>\ProvidesFile{cdocsamp.tex}[2018/12/30 v2.0 sample for childdoc]
%<*driver>
%\ProvidesFile{childdoc.drv}[2018/12/30 v2.0 childdoc reference manual file]
\PassOptionsToClass{10pt,a4paper}{article}
\documentclass{ltxdoc}

\usepackage[margin=35mm]{geometry}
\usepackage{hyperref}
\usepackage{hyperxmp}
\usepackage[usenames]{color}

\hypersetup{colorlinks=true}
\hypersetup{pdfstartview=FitH}
\hypersetup{pdfpagemode=UseNone}
\hypersetup{pdfsource={}}
\hypersetup{pdflang={en-UK}}
\hypersetup{pdfcopyright={Copyright 2017-2018 Niklas Beisert.
  This work may be distributed and/or modified under the
  conditions of the LaTeX Project Public License, either version 1.3
  of this license or (at your option) any later version.}}
\hypersetup{pdflicenseurl={http://www.latex-project.org/lppl.txt}}
\hypersetup{pdfcontactaddress={ETH Zurich, ITP, HIT K,
  Wolfgang-Pauli-Strasse 27}}
\hypersetup{pdfcontactpostcode={8093}}
\hypersetup{pdfcontactcity={Zurich}}
\hypersetup{pdfcontactcountry={Switzerland}}
\hypersetup{pdfcontactemail={nbeisert@itp.phys.ethz.ch}}
\hypersetup{pdfcontacturl={http://people.phys.ethz.ch/\xmptilde nbeisert/}}

\newcommand{\secref}[1]{\hyperref[#1]{section \ref*{#1}}}

\parskip1ex
\parindent0pt
\let\olditemize\itemize
\def\itemize{\olditemize\parskip0pt}

\begin{document}

\title{The \textsf{childdoc} Package}
\hypersetup{pdftitle={The childdoc Package}}
\author{Niklas Beisert\\[2ex]
  Institut f\"ur Theoretische Physik\\
  Eidgen\"ossische Technische Hochschule Z\"urich\\
  Wolfgang-Pauli-Strasse 27, 8093 Z\"urich, Switzerland\\[1ex]
  \href{mailto:nbeisert@itp.phys.ethz.ch}
  {\texttt{nbeisert@itp.phys.ethz.ch}}}
\hypersetup{pdfauthor={Niklas Beisert}}
\hypersetup{pdfsubject={Manual for the LaTeX2e Package childdoc}}
\date{30 December 2018, \textsf{v2.0}}
\maketitle

\begin{abstract}\noindent
\textsf{childdoc} is a \LaTeXe{} package
that enables the direct compilation
of document sections included by |\include|
to individual files.
\end{abstract}

\begingroup
\parskip0ex
\tableofcontents
\endgroup

%%%%%%%%%%%%%%%%%%%%%%%%%%%%%%%%%%%%%%%%%%%%%%%%%%%%%%%%%%%%%%%%%%%%%%%%%%%%%%%%
%%%%%%%%%%%%%%%%%%%%%%%%%%%%%%%%%%%%%%%%%%%%%%%%%%%%%%%%%%%%%%%%%%%%%%%%%%%%%%%%
\section{Introduction}

\LaTeX{} provides a mechanism to structure a large document (such as a book)
into a main file and several child files (containing the chapters)
using the |\include| command.
This mechanism is beneficial for documents
which span hundreds of pages in order to
make the source file(s) more manageable.
Moreover, compilation can be restricted to
selected child files by means of the |\includeonly| command.
The latter feature can be used to reduce the compilation time while editing
(this was significantly more useful in the earlier days of \LaTeX{})
or to generate a smaller document which is easier to navigate.
Another application of |\includeonly| is to generate
documents consisting of selected parts of the complete document.

However, there are a few drawbacks of the plain |\include| mechanism:
\begin{itemize}
\item
The child files cannot be compiled on their own,
they can only be compiled via the main file.
A naive editing environment
(such as a text editor with an option
to have the current file processed by \LaTeX)
may require one to switch to the main file before compiling;
attempting to compile the child file produces errors.
\item
The main file must be modified (each time)
to adjust the |\includeonly| command
to the present needs. This easily leaves the main file in a messy state.
\item
The generated document will always carry the filename
of the main document. This is inconvenient if
several child files are to be compiled and
to be kept for distribution.
\end{itemize}

The present package provides a simple interface
to make child files individually compilable by \LaTeX{}.
Compiling a child file then has the same effect as compiling
the main file with an |\includeonly| command
to select the appropriate child.
Moreover the generated document will carry the name of the child
rather than the main file.
This resolves all three above issues.

This feature is meant to make the editing of books,
thesis documents and lecture notes somewhat more convenient.
However, the package can also be used efficiently for
composing a series of documents (such as exercise sheets)
which are typically distributed individually.
It then assists the author in generating the individual documents
(potentially in different versions)
as well as a document containing the collected series.
Another application is in developing style files
or other kinds of included material
where compilation of the style file could redirect
to a sample or test file.

%%%%%%%%%%%%%%%%%%%%%%%%%%%%%%%%%%%%%%%%%%%%%%%%%%%%%%%%%%%%%%%%%%%%%%%%%%%%%%%%
%%%%%%%%%%%%%%%%%%%%%%%%%%%%%%%%%%%%%%%%%%%%%%%%%%%%%%%%%%%%%%%%%%%%%%%%%%%%%%%%
\section{Usage}

First of all, the package \textsf{childdoc} is \emph{not} a standard
\LaTeXe{} |.sty| style file! Therefore it needs to be invoked in
a non-standard way.

%%%%%%%%%%%%%%%%%%%%%%%%%%%%%%%%%%%%%%%%%%%%%%%%%%%%%%%%%%%%%%%%%%%%%%%%%%%%%%%%
\subsection{Included Files}
\label{sec:include}

%%%%%%%%%%%%%%%%%%%%%%%%%%%%%%%%%%%%%%%%
\DescribeMacro{\childdocmain}
To use the package, add the commands
\begin{center}
\begin{tabular}{l}
|\input{childdoc.def}|\\
|\childdocmain{}|\\
\end{tabular}
\end{center}
at the very top of the main \LaTeX{} file,
in particular \emph{before} the |\documentclass| statement!
The argument of |\childdocmain| should be left empty
(but it must be present).

%%%%%%%%%%%%%%%%%%%%%%%%%%%%%%%%%%%%%%%%
\DescribeMacro{\childdocof}
Furthermore, add the commands
\begin{center}
\begin{tabular}{l}
|\input{childdoc.def}|\\
|\childdocof{|\textit{main}|}|\\
\end{tabular}
\end{center}
at the top of every child file \textit{child}
which is included by |\include{|\textit{child}|}|
from within the main file
(or at least for those files to be compiled individually).
The argument \textit{main} must be the filename of the main file.

There are a couple of
considerations in setting up the main and child documents:

%%%%%%%%%%%%%%%%%%%%%%%%%%%%%%%%%%%%%%%%
\paragraph{Restrictions.}

Please note the following restrictions:
\begin{itemize}
\item
|\childdocmain| must be called with one argument \textit{main}
to ensure compatibility with earlier version of the package.
It must either be empty (|\childdocmain{}|)
or precisely match the filename of the main file in which it is specified.
See \secref{sec:detection} for further information.
\item
The filename \textit{main} must be specified without the |.tex| extension.
\item
The filename \textit{main} is case sensitive
(even in case-insensitive file systems)
due to internal string comparison.
\item
The argument \textit{main} should be fully expanded, it cannot be a macro.
\item
Subdirectories and special characters should be avoided in filenames.
\item
The command |\childdocmain{|\textit{main}|}| must be followed by a whitespace.
It should not be followed immediately by another command
or by a comment mark `|%|'.
This is because the \TeX{} parser reads the token immediately following
the argument of |\childdocmain| and puts it
at the beginning of every child section;
however, a white\-space is ignored.
\end{itemize}

%%%%%%%%%%%%%%%%%%%%%%%%%%%%%%%%%%%%%%%%
\paragraph{Content of Main File.}

It is advisable to place all content in the child files included by |\include|.
Any output contained in the main file will appear in all child documents
unless suppressed manually;
it cannot be suppressed automatically by the |\includeonly| directive
and thus should normally be avoided.
A method to include some content in the main file
by means of conditional processing is described in \secref{sec:conditional}.

%%%%%%%%%%%%%%%%%%%%%%%%%%%%%%%%%%%%%%%%
\paragraph{Page Numbering.}

When only a part of the document is compiled,
the appropriate numbering of pages
(as well as other status parameters)
is determined from the |.aux| files.
The latter contain information from previous passes.
However this information needs to propagate through
all intermediate child documents.
Therefore the page numbering in child documents may well
be inconsistent until the complete document is compiled at least once.

A useful (if unconventional) way to always ensure a consistent
page numbering is to restart the numbering in each child document
and denote the pages by `\textit{child}|.|\textit{page}'
where \textit{child} represents the chapter/section number of the child file.
This can be achieved by the command
|\numberwithin{page}{|\textit{child}|}|
of the \textsf{amsmath} package
where \textit{child} can be |chapter| or |section|
depending on the chosen structuring.
Alternatively, one can modify the macro |\thepage| appropriately
and reset the counter |page| at the start of each child file.

%%%%%%%%%%%%%%%%%%%%%%%%%%%%%%%%%%%%%%%%%%%%%%%%%%%%%%%%%%%%%%%%%%%%%%%%%%%%%%%%
\subsection{Conditional Processing}
\label{sec:conditional}

The package provides a mechanism to compile different versions
of a document. To customise the versions further some conditional processing
can come in handy to distinguish which version is being compiled.
The package provides two macros to describe the compilation context:

%%%%%%%%%%%%%%%%%%%%%%%%%%%%%%%%%%%%%%%%
\DescribeMacro{\ifchilddoc}
The conditional |\ifchilddoc| distinguishes between the compilation of
child documents and the main document:
%
\begin{center}
|\ifchilddoc |\textit{child-code}| |[|\||else |\textit{main-code}]| \||fi|
\end{center}

%%%%%%%%%%%%%%%%%%%%%%%%%%%%%%%%%%%%%%%%
\DescribeMacro{\childdocname}
\DescribeMacro{\childdocjob}
The macro |\childdocname| contains the filename (without extension)
of the main or child file being processed.
Note that |\childdocjob| will always contain the name of the main file.

%%%%%%%%%%%%%%%%%%%%%%%%%%%%%%%%%%%%%%%%
\paragraph{Title Page.}

Conditional processing can be used to include a title or banner page
in the main document when proper precautions are taken.
Importantly, the code in the main file should ensure that the page counter
(as well as other status parameters which are stored in the |.aux| files)
takes the same value after the conditional processing.
Otherwise the page numbers may take divergent values
depending on which part is compiled.

For example, a title page could be declared by:
%
\begin{center}
\begin{tabular}{l}
|\ifchilddoc\||else|\\
|\addtocounter{page}{-1}|\\
\textit{code for title page}\\
|\newpage|\\
|\||fi|
\end{tabular}
\end{center}
%
A banner page for the child documents can be generated by:
%
\begin{center}
\begin{tabular}{l}
|\ifchilddoc|\\
|\addtocounter{page}{-1}|\\
\textit{code for banner page}\\
|\newpage|\\
|\||fi|
\end{tabular}
\end{center}
%
Here one could write a message such as:
\begin{center}
|This is the part \childdocname{} of \childdocjob{}.|
\end{center}

%%%%%%%%%%%%%%%%%%%%%%%%%%%%%%%%%%%%%%%%%%%%%%%%%%%%%%%%%%%%%%%%%%%%%%%%%%%%%%%%
\subsection{Flags}
\label{sec:flags}

The package makes it easy to generate different versions
of the main or child documents.
To this end compilation flags can be defined
and assigned different default values.
They will be particularly useful in conjunction
with the forwarding mechanism described in \secref{sec:forward}.

For example, it may be useful to have a flag |\version|
which can be set to |draft| or |final|.
The document source will contain some conditional code
depending on the value of |\version|.
Suppose further, the flag should default to |final| for the main file
and to |draft| for child files
which is a natural assignment for editing the document.
This is achieved by placing the following code
in the preamble of the main document
(below the |\childdocmain| directive):
%
\begin{center}
\begin{tabular}{l}
|\ifchilddoc|\\
|\providecommand{\version}{draft}|\\
|\||else|\\
|\providecommand{\version}{final}|\\
|\||fi|
\end{tabular}
\end{center}
%
The definition by |\providecommand| makes sure
that previous definitions are not overwritten.
Further statements |\providecommand{\version}{...}|
can thus be added before the above code to override it.

For the main file, one might add a line
(between |\childdocmain| and the above block)
%
\begin{center}
|%\ifchilddoc\||else\providecommand{\version}{draft}\||fi|
\end{center}
%
which can be uncommented to produce a draft version.
Likewise one can add a line to the very top of a child file
(above the |\childdocof{|\textit{main}|}| directive)
%
\begin{center}
|%\providecommand{\version}{final}|
\end{center}
%
which can be uncommented to produce the final version of this child document.

%%%%%%%%%%%%%%%%%%%%%%%%%%%%%%%%%%%%%%%%%%%%%%%%%%%%%%%%%%%%%%%%%%%%%%%%%%%%%%%%
\subsection{Forwarding}
\label{sec:forward}

Different versions of the main or child documents
using compilation flags as described in \secref{sec:flags}
can be (permanently) stored in different files
for convenient compilation, viewing and distribution.
To this end, the package defines a command
to pass on compilation to a different file:

%%%%%%%%%%%%%%%%%%%%%%%%%%%%%%%%%%%%%%%%
\DescribeMacro{\childdocforward}
The command |\childdocforward| redirects processing to
another source file:
%
\begin{center}
\begin{tabular}{l}
|\input{childdoc.def}|\\
|\childdocforward[|\textit{main}|]{|\textit{dest}|}|\\
\end{tabular}
\end{center}
%
The argument \textit{dest} is the destination file
(without extension).
It should be the main file or one of the child files.
Note that further \textsf{childdoc} directives
such as |\childdocof| and |\childdocforward|
in the indicated file will be processed in this form.
The optional argument \textit{main}
passes on directly to the main file \textit{main}
while pretending to compile the child \textit{dest}.
This form behaves as if \textit{dest}
issues |\childdocof{|\textit{main}|}| right away,
and no further \textsf{childdoc} directives will be processed.

%%%%%%%%%%%%%%%%%%%%%%%%%%%%%%%%%%%%%%%%
\DescribeMacro{\...prefix}
In the alternative form |\childdocforwardprefix|,
%
\begin{center}
\begin{tabular}{l}
|\input{childdoc.def}|\\
|\childdocforwardprefix[|\textit{main}|]{|\textit{prefix}|}{|\textit{dest}|}|
\end{tabular}
\end{center}
%
the destination file is determined by a pattern
depending on the current file:
To make this work, the current file must be called
`{\textit{prefix}\hspace{0.2em}\textit{suffix}}'
with \textit{prefix} matching precisely the argument.
Processing is then passed on to the file
`{\textit{dest}\hspace{0.2em}\textit{suffix}}'.
Surely, the same effect is achieved by
directly specifying the
argument `{\textit{dest}\hspace{0.2em}\textit{suffix}}'
in the first form.
However, that requires to set up a different file
for each child. With the alternative form of the command
all these files can have exactly the same content
which simplifies setting them up and maintaining them.

For example, the following file |draft.tex|
with a compilation flag |\version| as described in \secref{sec:flags}
compiles the main document as a draft:
%
\begin{center}
\begin{tabular}{l}
|\def\version{draft}|\\
|\input{childdoc.def}|\\
|\childdocforward{|\textit{main}|}|
\end{tabular}
\end{center}
%
Likewise, the following files |final|\textit{nn}|.tex|
compile the final version of the child document
|child|\textit{nn}|.tex|:
%
\begin{center}
\begin{tabular}{l}
|\def\version{final}|\\
|\input{childdoc.def}|\\
|\childdocforwardprefix{final}{child}|
\end{tabular}
\end{center}
%

Note that when several versions of a main file and/or of each child file
are to be generated, it may be convenient to set up a |Makefile| or
shell script to automatise the process.

%%%%%%%%%%%%%%%%%%%%%%%%%%%%%%%%%%%%%%%%%%%%%%%%%%%%%%%%%%%%%%%%%%%%%%%%%%%%%%%%
\subsection{Command Line Processing}
\label{sec:commandline}

The effect of redirection files can also be achieved by invoking
the \LaTeX{} compiler with a more elaborate command line.
Most conveniently this should be done as part
of a shell script or a |Makefile|.

When using \textsf{childdoc} in the main file, the following
command lines effectively perform a redirection
(note that depending on the shell being used,
backslashes may have to be doubled: `|\|' $\to$ `|\\|'):
%
\begin{center}
|... -jobname "|\textit{target}|" |\\|"|[\textit{flags}]%
|\input{childdoc.def}\childdocforward[|\textit{main}|]{|\textit{dest}|}"|
\end{center}
%
Here \textit{target} is the name of the output file,
\textit{main} is the name of the main file
and \textit{dest} is the name of the main or child file to be processed
(all filenames without extensions).
The optional argument \textit{main} can be omitted
if \textit{main} matches \textit{dest}.
Optionally, compilation \textit{flags} can be defined via |\def| commands.
This command line makes the \TeX{} engine believe
it is compiling the file \textit{target}
whose content is specified as the latter parameter.
The provided code then forwards the processing to
\textit{main} or \textit{dest} as described in \secref{sec:forward}.

%%%%%%%%%%%%%%%%%%%%%%%%%%%%%%%%%%%%%%%%%%%%%%%%%%%%%%%%%%%%%%%%%%%%%%%%%%%%%%%%
\subsection{Include by Input}
\label{sec:input}

Including child documents by |\include| has some restrictions by design.
Most notably, the content of a child document always occupies
its own set of pages; pages cannot be shared between child documents.
Usually, this behaviour makes perfect sense
because each child document contain an essential part of the document.
However, in some situations it may be desirable to compose
a document from a collection of parts
without having mandatory page breaks between then.
For this case, the package
provides a mechanism to include parts
by |\input| which can also be processed individually.
However, by construction this mechanism
requires manual handling of the content to be output.

%%%%%%%%%%%%%%%%%%%%%%%%%%%%%%%%%%%%%%%%
\DescribeMacro{\ifchilddocmanual}
The main file should be prepared as usual, see \secref{sec:include}.
However, the document body must make a distinction
between processing of an individual part and of the main document, e.g.:
%
\begin{center}
\begin{tabular}{l}
|\ifchilddocmanual|\\
|\input{\childdocname}|\\
|\||else|\\
\textit{document body with }|\input{|\textit{part}|}|\\
|\||fi|
\end{tabular}
\end{center}
%
The conditional |\ifchilddocmanual| is true whenever
a part to be included by |\input| is being compiled,
and the name of the part is stored in |\childdocname|.

%%%%%%%%%%%%%%%%%%%%%%%%%%%%%%%%%%%%%%%%
\DescribeMacro{\childdocby}
Each part to be included by |\input| should start with:
%
\begin{center}
\begin{tabular}{l}
|\input{childdoc.def}|\\
|\childdocby{|\textit{main}|}|\\
\end{tabular}
\end{center}
%
The directive |\childdocby| is similar to |\childdocof|
described in \secref{sec:include},
but the subsequent selection of content must be done manually.
To that end, both |\ifchilddoc| and |\ifchilddocmanual|
will be true upon processing of a part,
and the name of the part is stored in |\childdocname|.
Note that |\jobname| will be set to the filename of the current part
so that each part receives an individual |.aux| file
that does not interfere with the |.aux| file(s) of the main document.
This behaviour can be altered by the alternative form
|\childdocby[*]{|\textit{main}|}| (with a non-empty optional argument)
which uses the |.aux| file of the main document
by setting |\jobname| to \textit{main}.

%%%%%%%%%%%%%%%%%%%%%%%%%%%%%%%%%%%%%%%%%%%%%%%%%%%%%%%%%%%%%%%%%%%%%%%%%%%%%%%%
\subsection{Driver Development}
\label{sec:driver}

The \textsf{childdoc} mechanism can also be use for the development
of definition files such as \LaTeX{} styles or classes.
This case differs from the above setup with multiple parts
included by |\include| in that no |\includeonly| should be invoked.
This can be achieved by starting the include file
(before |\ProvidesPackage|) with:
%
\begin{center}
\begin{tabular}{l}
|\input{childdoc.def}|\\
|\childdocforward{|\textit{main}|}|\\
\end{tabular}
\end{center}
%
or alternatively with:
%
\begin{center}
\begin{tabular}{l}
|\input{childdoc.def}|\\
|\childdocby{|\textit{main}|}|\\
\end{tabular}
\end{center}
%
Both forms have slightly different effects as described above.
The main file is prepared as usual, see \secref{sec:include}.

%%%%%%%%%%%%%%%%%%%%%%%%%%%%%%%%%%%%%%%%%%%%%%%%%%%%%%%%%%%%%%%%%%%%%%%%%%%%%%%%
\subsection{Legacy Detection}
\label{sec:detection}

The directive |\childdocmain| in the main file can detect
whether the complete document or merely a child is to be compiled
even without using the directive |\childdocof|.
This method is deprecated because it is less robust
and there is no compelling reason to use it;
it is merely provided for backward compatibility
and it may be removed in future versions.

If the detection mechanism is to be used,
it is mandatory to correctly specify
the filename of the main file as the argument of |\childdocmain|:
%
\begin{center}
\begin{tabular}{l}
|\input{childdoc.def}|\\
|\childdocmain{|\textit{main}|}|\\
\end{tabular}
\end{center}
%
If |\jobname| does not match the argument \textit{main} of |\childdocmain|,
it is assumed that |\jobname| points to the child file to be compiled.
When using |\childdocmain| with the main file specified as argument,
it suffices to start a child file
with just |\input{|\textit{main}|}|
without loading of the package and using |\childdocof|.
If instead all processing is done
with the appropriate \textsf{childdoc} directives,
the argument of \textit{main} of |\childdocmain| can be empty.

An alternative version of the command line processing described
in \secref{sec:commandline} using the detection mechanism reads:
%
\begin{center}
|... -jobname "|\textit{target}|" "|[\textit{flags}]%
[|\def\jobname{|\textit{dest}|}|]|\input{|\textit{main}|}"|
\end{center}

%%%%%%%%%%%%%%%%%%%%%%%%%%%%%%%%%%%%%%%%%%%%%%%%%%%%%%%%%%%%%%%%%%%%%%%%%%%%%%%%
\subsection{Manual Code}
\label{sec:manual}

In case one cannot be certain whether the definitions file |childdoc.def|
is installed on the target \TeX{} distribution
and one prefers not to ship it,
it is conceivable to paste a few relevant commands into the sources.

To that end, drop all statements |\input{childdoc.def}|
and perform the replacements as outlined below.
Instead of |\childdocmain{|\textit{main}|}| add the following code
to the top of the main file:
%
\begin{center}
\begin{tabular}{l}
|\||ifdefined\childdocname\endinput\||fi\newif\ifchilddoc|\\
|\edef\childdocname{\scantokens\expandafter{\jobname\noexpand}}|\\
|\def\childdocmain{|\textit{main}|}\||ifx\childdocmain\childdocname\||else|\\
|\childdoctrue\includeonly{\childdocname}\let\jobname\childdocmain\||fi|\\
\end{tabular}
\end{center}
%
Instead of |\childdocof{|\textit{main}|}| just include the main file
at the top of each child file:
%
\begin{center}
|\input{|\textit{main}|}|
\end{center}
%
A simple redirection |\childdocforward{|\textit{dest}|}| is achieved by:
%
\begin{center}
|\def\jobname{|\textit{dest}|}\input{\jobname}|
\end{center}
%
The redirection with prefix
|\childdocforwardprefix[|\textit{prefix}|]{|\textit{dest}|}|
is accomplished by:
%
\begin{center}
\begin{tabular}{l}
|{\edef\jobname{\scantokens\expandafter{\jobname\noexpand}}|\\
|\def\redirectjob |\textit{prefix}|#1~~~{\gdef\jobname{|\textit{dest}|#1}}|\\
|\expandafter\redirectjob\jobname~~~}\input{\jobname}|
\end{tabular}
\end{center}

In an alternative approach,
child documents can be compiled by a specific command line
without additional code or specific definitions:
%
\begin{center}
|... -jobname "|\textit{target}|" "|[\textit{flags}]%
|\includeonly{|\textit{dest}|}\input{|\textit{main}|}"|
\end{center}
%

%%%%%%%%%%%%%%%%%%%%%%%%%%%%%%%%%%%%%%%%%%%%%%%%%%%%%%%%%%%%%%%%%%%%%%%%%%%%%%%%
%%%%%%%%%%%%%%%%%%%%%%%%%%%%%%%%%%%%%%%%%%%%%%%%%%%%%%%%%%%%%%%%%%%%%%%%%%%%%%%%
\section{Information}

%%%%%%%%%%%%%%%%%%%%%%%%%%%%%%%%%%%%%%%%%%%%%%%%%%%%%%%%%%%%%%%%%%%%%%%%%%%%%%%%
\subsection{Copyright}

Copyright \copyright{} 2017--2018 Niklas Beisert

This work may be distributed and/or modified under the
conditions of the \LaTeX{} Project Public License, either version 1.3
of this license or (at your option) any later version.
The latest version of this license is in
  \url{http://www.latex-project.org/lppl.txt}
and version 1.3 or later is part of all distributions of \LaTeX{}
version 2005/12/01 or later.

This work has the LPPL maintenance status `maintained'.

The Current Maintainer of this work is Niklas Beisert.

This work consists of the files |README.txt|, |childdoc.ins| and |childdoc.dtx|
as well as the derived files |childdoc.def|, |cdocsamp.tex|
with |cdocsch1.tex|, |cdocsch2.tex|, |cdocspt3.tex|, |cdocspt4.tex|,
|cdocsdrf.tex|, |cdocsfn1.tex|, |cdocsfn2.tex|
as well as |childdoc.pdf|.

%%%%%%%%%%%%%%%%%%%%%%%%%%%%%%%%%%%%%%%%%%%%%%%%%%%%%%%%%%%%%%%%%%%%%%%%%%%%%%%%
\subsection{Files and Installation}

The package consists of the files:
%
\begin{center}
\begin{tabular}{ll}
    |README.txt|   & readme file \\
    |childdoc.ins| & installation file \\
    |childdoc.dtx| & source file \\
    |childdoc.def| & definition file \\
    |cdocsamp.tex| & sample main file \\
    |cdocsch1.tex| & sample include file \\
    |cdocsch2.tex| & sample include file \\
    |cdocspt3.tex| & sample part file \\
    |cdocspt4.tex| & sample part file \\
    |cdocsdrf.tex| & sample redirection file \\
    |cdocsfn1.tex| & sample redirection file \\
    |cdocsfn2.tex| & sample redirection file \\
    |childdoc.pdf| & manual
\end{tabular}
\end{center}
%
The distribution consists of the files
|README.txt|, |childdoc.ins| and |childdoc.dtx|.
%
\begin{itemize}
\item
Run (pdf)\LaTeX{} on |childdoc.dtx|
to compile the manual |childdoc.pdf| (this file).
\item
Run \LaTeX{} on |childdoc.ins| to create the definitions file |childdoc.def|
and the sample |cdocsamp.tex| with include files
|cdocsch1.tex|, |cdocsch2.tex|, |cdocspt3.tex|, |cdocspt4.tex|,
|cdocsdrf.tex|, |cdocsfn1.tex|, |cdocsfn2.tex|.
Then copy the file |childdoc.def| to an appropriate directory of your \LaTeX{}
distribution, e.g.\ \textit{texmf-root}|/tex/latex/childdoc|.
\end{itemize}

%%%%%%%%%%%%%%%%%%%%%%%%%%%%%%%%%%%%%%%%%%%%%%%%%%%%%%%%%%%%%%%%%%%%%%%%%%%%%%%%
\subsection{Related CTAN Packages}

There are several other packages which offer a similar functionality:
%
\begin{itemize}
\item
The packages
\href{http://ctan.org/pkg/docmute}{\textsf{docmute}},
\href{http://ctan.org/pkg/includex}{\textsf{includex}} and
\href{http://ctan.org/pkg/standalone}{\textsf{standalone}}
provide commands to include only the document body of
a child file thus allowing both files to be compiled individually.
\item
The packages \href{http://ctan.org/pkg/subdocs}{\textsf{subdocs}}
and \href{http://ctan.org/pkg/subfiles}{\textsf{subfiles}}
provide structures in which the main and child documents can be
encapsulated and allowing them to be compiled individually.
The inclusion mechanism is different from the conventional |\include|.
\item
The package \href{http://ctan.org/pkg/combine}{\textsf{combine}}
is an elaborate solution to combine several documents into one.
\end{itemize}
%
See also the CTAN topic \href{http://ctan.org/topic/subdocs}{\textsf{subdocs}}
for further related packages.
The present package differs from the above solutions in that
a document structure constructed with the conventional |\include| mechanism
just needs two extra commands at the top of every file
such that all constituent files can be compiled individually.

%%%%%%%%%%%%%%%%%%%%%%%%%%%%%%%%%%%%%%%%%%%%%%%%%%%%%%%%%%%%%%%%%%%%%%%%%%%%%%%%
%\subsection{Feature Suggestions}
%
%The following is a list of features which may be useful for future
%versions of this package:
%%
%\begin{itemize}
%\item
%\ldots
%\end{itemize}

%%%%%%%%%%%%%%%%%%%%%%%%%%%%%%%%%%%%%%%%%%%%%%%%%%%%%%%%%%%%%%%%%%%%%%%%%%%%%%%%
\subsection{Revision History}

%%%%%%%%%%%%%%%%%%%%%%%%%%%%%%%%%%%%%%%%
\paragraph{v2.0:} 2018/12/30

\begin{itemize}
\item
immediate forward processing
\item
added |\childdocby| mechanism
\item
manual restructured
\end{itemize}

%%%%%%%%%%%%%%%%%%%%%%%%%%%%%%%%%%%%%%%%
\paragraph{v1.6:} 2018/01/17

\begin{itemize}
\item
application for development of include files
\item
corrections to manual
\end{itemize}

%%%%%%%%%%%%%%%%%%%%%%%%%%%%%%%%%%%%%%%%
\paragraph{v1.5:} 2017/05/21

\begin{itemize}
\item
more complete structuring introduced
\item
|\childdocof| introduced
\item
|\childdoc| renamed to |\childdocmain|
\item
|\childredirect| renamed to |\childdocforward| and |\childdocforwardprefix|
and functionality expanded
\end{itemize}

%%%%%%%%%%%%%%%%%%%%%%%%%%%%%%%%%%%%%%%%
\paragraph{v1.0:} 2017/04/27

\begin{itemize}
\item
manual and install package
\item
first version published on CTAN
\end{itemize}

%%%%%%%%%%%%%%%%%%%%%%%%%%%%%%%%%%%%%%%%
\paragraph{v0.6:} 2017/04/26

\begin{itemize}
\item
redirection mechanism added
\end{itemize}

%%%%%%%%%%%%%%%%%%%%%%%%%%%%%%%%%%%%%%%%
\paragraph{v0.5:} 2017/04/26

\begin{itemize}
\item
functionality in definition file
\end{itemize}


%%%%%%%%%%%%%%%%%%%%%%%%%%%%%%%%%%%%%%%%%%%%%%%%%%%%%%%%%%%%%%%%%%%%%%%%%%%%%%%%
%%%%%%%%%%%%%%%%%%%%%%%%%%%%%%%%%%%%%%%%%%%%%%%%%%%%%%%%%%%%%%%%%%%%%%%%%%%%%%%%
%%%%%%%%%%%%%%%%%%%%%%%%%%%%%%%%%%%%%%%%%%%%%%%%%%%%%%%%%%%%%%%%%%%%%%%%%%%%%%%%
\appendix

\settowidth\MacroIndent{\rmfamily\scriptsize 000\ }

 \DocInput{childdoc.dtx}

\end{document}
%</driver>
% \fi
%
% %%%%%%%%%%%%%%%%%%%%%%%%%%%%%%%%%%%%%%%%%%%%%%%%%%%%%%%%%%%%%%%%%%%%%%%%%%%%%%
% %%%%%%%%%%%%%%%%%%%%%%%%%%%%%%%%%%%%%%%%%%%%%%%%%%%%%%%%%%%%%%%%%%%%%%%%%%%%%%
% \section{Sample}
%\iffalse
%<*samplemain>
%\fi
%
% The following presents a sample document
% with two chapters, two parts, a title page,
% a compile flag as well as three forwarding files to set the flag.
% It consists of eight |.tex| files:
% \begin{center}
% \begin{tabular}{ll}
% |cdocsamp.tex|&main file\\
% |cdocsch1.tex|&include file for chapter 1\\
% |cdocsch2.tex|&include file for chapter 2\\
% |cdocspt3.tex|&include file for part 3\\
% |cdocspt4.tex|&include file for part 4\\
% |cdocsdrf.tex|&forwarding file for main file in draft mode\\
% |cdocsfi1.tex|&forwarding file for final version of chapter 1\\
% |cdocsfi2.tex|&forwarding file for final version of chapter 2\\
% \end{tabular}
% \end{center}
% Each of the eight files can be compiled directly by the \LaTeX{} compiler.
%
% %%%%%%%%%%%%%%%%%%%%%%%%%%%%%%%%%%%%%%
% \paragraph{Main File.}
%
% The main file is called |cdocsamp.tex|.
%
% Load the \textsf{childdoc} definitions and
% declare the filename for the main document:
%    \begin{macrocode}
\input{childdoc.def}
\childdocmain{}
%    \end{macrocode}

% Optional override for |\version| flag:
%    \begin{macrocode}
%%\ifchilddoc\else\providecommand{\version}{draft}\fi
%    \end{macrocode}

% Define the default values for the |\version| flag
% (|final| for the main file and |draft| for childs):
%    \begin{macrocode}
\ifchilddoc
\providecommand{\version}{draft}
\else
\providecommand{\version}{final}
\fi
%    \end{macrocode}

% Load the standard document class:
%    \begin{macrocode}
\documentclass[12pt]{article}
%    \end{macrocode}

% Start the document body:
%    \begin{macrocode}
\begin{document}
%    \end{macrocode}

% Declare a title page.
% Print title, part of document being processed and version flag:
%    \begin{macrocode}
\addtocounter{page}{-1}
\begin{center}
{\LARGE\bfseries{}childdoc example\par}
\vspace{1cm}
\ifchilddoc
\ifchilddocmanual part\else chapter\fi:
`\childdocname' of `\childdocjob'\par
\else
main document: `\childdocjob'\par
\fi
version: \version\par
\end{center}
\newpage
%    \end{macrocode}

% Manually include selected file,
% otherwise process as usual:
%    \begin{macrocode}
\ifchilddocmanual
\section*{part `\childdocname'}
\input{\childdocname}
\else
%    \end{macrocode}

% Include the two chapters:
%    \begin{macrocode}
\include{cdocsch1}
\include{cdocsch2}
%    \end{macrocode}

% Include the two parts unless only chapters should be displayed:
%    \begin{macrocode}
\ifchilddoc\else
\section{part three}
\input{cdocspt3}
\section{part four}
\input{cdocspt4}
\fi
%    \end{macrocode}

% Process as usual until here:
%    \begin{macrocode}
\fi
%    \end{macrocode}

% End of document body:
%    \begin{macrocode}
\end{document}
%    \end{macrocode}
%\iffalse
%</samplemain>
%\fi
%
% %%%%%%%%%%%%%%%%%%%%%%%%%%%%%%%%%%%%%%
% \paragraph{Chapter Include Files.}
%
% The include files are called |cdocsch1.tex| and |cdocsch2.tex|.
%
%\iffalse
%<*samplechap1|samplechap2>
%\fi

% Optional override for |\version| flag:
%    \begin{macrocode}
%%\providecommand{\version}{final}
%    \end{macrocode}

% Include the main document:
%    \begin{macrocode}
\input{childdoc.def}
\childdocof{cdocsamp}
%    \end{macrocode}

%\iffalse
%</samplechap1|samplechap2>
%\fi
%
%\iffalse
%<*samplechap1>
%\fi
% Some text for chapter 1:
%    \begin{macrocode}
\section{one}
some text in chapter one
%    \end{macrocode}

%\iffalse
%</samplechap1>
%\fi
% Some text for chapter 2:
%\iffalse
%<*samplechap2>
%\fi
%    \begin{macrocode}
\section{two}
more text in chapter two
%    \end{macrocode}

%\iffalse
%</samplechap2>
%\fi
%
% %%%%%%%%%%%%%%%%%%%%%%%%%%%%%%%%%%%%%%
% \paragraph{Part Include Files.}
%
% The include files are called |cdocspt3.tex| and |cdocspt4.tex|.
%
%\iffalse
%<*samplepart3|samplepart4>
%\fi

% Optional override for |\version| flag:
%    \begin{macrocode}
%%\providecommand{\version}{final}
%    \end{macrocode}

% Include the main document:
%    \begin{macrocode}
\input{childdoc.def}
\childdocby{cdocsamp}
%    \end{macrocode}

%\iffalse
%</samplepart3|samplepart4>
%\fi
%
%\iffalse
%<*samplepart3>
%\fi
% Some text for part 3:
%    \begin{macrocode}
some text in part three
%    \end{macrocode}

%\iffalse
%</samplepart3>
%\fi
% Some text for part 4:
%\iffalse
%<*samplepart4>
%\fi
%    \begin{macrocode}
more text in part four
%    \end{macrocode}

%\iffalse
%</samplepart4>
%\fi
%
% %%%%%%%%%%%%%%%%%%%%%%%%%%%%%%%%%%%%%%
% \paragraph{Forwarding for a Complete Draft.}
%
% The following forwarding file |cdocsdrf.tex|
% compiles the main document in draft mode:
%\iffalse
%<*sampledraft>
%\fi
%    \begin{macrocode}
\def\version{draft}
\input{childdoc.def}
\childdocforward{cdocsamp}
%    \end{macrocode}

%\iffalse
%</sampledraft>
%\fi
%
% %%%%%%%%%%%%%%%%%%%%%%%%%%%%%%%%%%%%%%
% \paragraph{Forwarding for Final Version of the Chapters.}
%
% The following forwarding files |cdocsfn1.tex| and |cdocsfn2.tex|
% (with identical content)
% compile the final versions of the child documents
% |cdocsch1.tex| and |cdocsch2.tex|, respectively:
%\iffalse
%<*samplefinal>
%\fi
%    \begin{macrocode}
\def\version{final}
\input{childdoc.def}
\childdocforwardprefix[cdocsamp]{cdocsfn}{cdocsch}
%    \end{macrocode}

%\iffalse
%</samplefinal>
%\fi
%
% %%%%%%%%%%%%%%%%%%%%%%%%%%%%%%%%%%%%%%
% \paragraph{Command Line Processing.}
%
% The following three command lines generate the output files
% |cdocscld|, |cdocscl1| and |cdocscl2|
% which should be identical to
% |cdocsdrf|, |cdocsch1| and |cdocsfn2|, respectively:
% \begin{center}
% \begin{tabular}{l}
% |latex -jobname cdocscld \|\\
% |  "\def\version{draft}\input{childdoc.def}\childdocforward{cdocsamp}"|\\
% |latex -jobname cdocscl1 \|\\
% |  "\input{childdoc.def}\childdocforward[cdocsamp]{cdocsch1}"|\\
% |latex -jobname cdocscl2 \|\\
% |  "\def\version{final}\input{childdoc.def}\childdocforward{cdocsch2}"|
% \end{tabular}
% \end{center}
% Note that the trailing backslash on each first line
% merely continues the input to the second line
% (for convenient cut ant paste).
% Furthermore, the command |latex| can be replaced by any
% of its alternative versions such as |pdflatex|.
%
% %%%%%%%%%%%%%%%%%%%%%%%%%%%%%%%%%%%%%%%%%%%%%%%%%%%%%%%%%%%%%%%%%%%%%%%%%%%%%%
% %%%%%%%%%%%%%%%%%%%%%%%%%%%%%%%%%%%%%%%%%%%%%%%%%%%%%%%%%%%%%%%%%%%%%%%%%%%%%%
% \section{Implementation}
%\iffalse
%<*package>
%\fi
%
% This section describes the definitions file |childdoc.def|.

% The definitions cannot be loaded using |\usepackage| or |\RequirePackage|
% which has a mechanism to prevent loading a style file more than once.
% When loading the definitions by means of |\input|
% multiple instances have to be prevented manually:
%\iffalse
%This code needs to be before the `\ProvidesFile' directive
%which is defined at the beginning of this file.
%Therefore it is also placed there and commented out here.
%</package>
%<*discard>
%\fi
%    \begin{macrocode}
\ifdefined\childdocmain\endinput\fi
%    \end{macrocode}
%\iffalse
%</discard>
%<*package>
%\fi
%
% \macro{\ifchilddoc}
% \macro{\ifchilddocmanual}
% The conditional |\ifchilddoc| tells whether a
% child (true) or main (false) document is being compiled.
% The conditional |\ifchilddocmanual| tells whether
% the |\includeonly| mechanism is used (false) or
% the selection of child files must be performed manually (true).
% The definitions initialise to false:
%    \begin{macrocode}
\newif\ifchilddoc
\newif\ifchilddocmanual
%    \end{macrocode}

% \macro{\childdocname}
% \macro{\childdocjob}
% The macro |\childdocname| stores the name of the main document
% to be compiled. The macro |\childdocjob| stores the name of
% the document on which the \LaTeX{} compiler was originally invoked.
% The content of |\jobname| cannot be compared
% to filenames specified in the source due to different catcodes.
% The following code rescans |\jobname|, stores the result
% in |\childdocname| and saves a copy in |\childdocjob|:
%    \begin{macrocode}
\edef\childdocname{\scantokens\expandafter{\jobname\noexpand}}
\let\childdocjob\childdocname
%    \end{macrocode}

% \macro{\childdocdisable}
% The macro |\childdocdisable| prevents the main file
% from being processed more than once.
% At this stage, the main document command |\childdocmain|
% is assumed to be called once again where it should do nothing.
% Any subsequent call to it should prevent
% a secondary processing of the main document
% It overwrites the forwarding commands
% |\childdocof| and |\childdocforward|
% with empty macros to prevent further inclusions of the main document:
%    \begin{macrocode}
\newcommand{\childdocdisable}
{
  \renewcommand{\childdocmain}[1]{\renewcommand{\childdocmain}[1]{\endinput}}
  \renewcommand{\childdocof}[1]{}
  \renewcommand{\childdocby}[2][]{}
  \renewcommand{\childdocforward}[2][]{}
  \renewcommand{\childdocdisable}{}
}
%    \end{macrocode}

% \macro{\childdocmain}
% The macro |\childdocmain| is to be called at the top of the main file
% with nothing or the main filename (without extension) as argument.
% First, it breaks loops.
% If the argument is not empty and does not match |\childdocname|
% (which is set by the first inclusion of |childdoc.def|),
% |\ifchilddoc| is set to true, |\includeonly| is applied to the child file
% and |\jobname| is set to the main file
% (for proper handling of |.aux| files):
%    \begin{macrocode}
\newcommand{\childdocmain}[1]
{
  \childdocdisable\childdocmain{}
  \if?#1?\else
    \begingroup
      \def\childdoctmp{#1}
      \ifx\childdoctmp\childdocname
        \def\childdoctmp{}
      \else
        \def\childdoctmp
        {
          \childdoctrue
          \includeonly{\childdocname}
          \def\childdocjob{#1}
          \def\jobname{#1}
        }
      \fi
      \expandafter
    \endgroup
    \childdoctmp
  \fi
}
%    \end{macrocode}

% \macro{\childdocof}
% The command |\childdocof| redirects
% compilation to the main file |#1|.
%    \begin{macrocode}
\newcommand{\childdocof}[1]
{
  \childdocdisable
  \childdoctrue
  \includeonly{\childdocname}
  \def\jobname{#1}
  \def\childdocjob{#1}
  \input{#1}
}
%    \end{macrocode}

% \macro{\childdocby}
% The command |\childdocby| ....
%    \begin{macrocode}
\newcommand{\childdocby}[2][]
{
  \childdocdisable
  \childdoctrue
  \childdocmanualtrue
  \if?#1?\else
    \def\jobname{#2}
  \fi
  \def\childdocjob{#2}
  \input{#2}
  \endinput
}
%    \end{macrocode}

% \macro{\childdocforward}
% The command |\childdocforward| redirects
% compilation to the main file or
% (if the optional argument is given) a child file.
% Parameters are set as if the main file
% or a child file starting with |\childdocof| was compiled.
% Then compilation is handed over to the main file:
%    \begin{macrocode}
\newcommand{\childdocforward}[2][]
{
  \begingroup
    \if?#1?
      \def\childdoctmp
      {
        \def\childdocname{#2}
        \def\childdocjob{#2}
        \def\jobname{#2}
        \input{#2}
        \endinput
      }
    \else
      \def\childdoctmp
      {
        \childdocdisable
        \def\childdocname{#2}
        \childdoctrue
        \includeonly{#2}
        \def\childdocjob{#1}
        \def\jobname{#1}
        \input{#1}
        \endinput
      }
    \fi
    \expandafter
  \endgroup
  \childdoctmp
}
%    \end{macrocode}

% \macro{\childdocforwardprefix}
% The command |\childdocforwardprefix| redirects
% compilation to the main or a child file by means of a pattern.
% The prefix |#1| in the current filename is replaced by |#2|
% and the suffix of the current filename is kept
% (it is assumed that the filename does not contain the substring `|~~~|'
% which is used as a delimiter).
% Compilation is handed over to the new file by |\childdocforward|:
%    \begin{macrocode}
\newcommand{\childdocforwardprefix}[3][]
{
  \begingroup
    \def\childdocextract #2##1~~~{\def\childdoctmp{\childdocforward[#1]{#3##1}}}
    \expandafter\childdocextract\childdocname~~~
    \expandafter
  \endgroup
  \childdoctmp
}
%    \end{macrocode}

% \macro{\childdoc}
% The deprecated macro |\childdoc| is a legacy version of |\childdocmain|:
%    \begin{macrocode}
\newcommand{\childdoc}{\childdocmain}
%    \end{macrocode}

% \macro{\childdocredirect}
% The deprecated macro |\childdocredirect| is a legacy version
% of |\childdocforward| and |\childdocforwardprefix|:
%    \begin{macrocode}
\newcommand{\childdocredirect}[2][]
{
  \begingroup
    \if?#1?
      \def\childdoctmp{\childdocforward{#2}}
    \else
      \def\childdoctmp{\childdocforwardprefix{#1}{#2}}
    \fi
    \expandafter
  \endgroup
  \childdoctmp
}
%    \end{macrocode}

%\iffalse
%</package>
%\fi
%
\endinput

\childdocof{cdocsamp}
%    \end{macrocode}

%\iffalse
%</samplechap1|samplechap2>
%\fi
%
%\iffalse
%<*samplechap1>
%\fi
% Some text for chapter 1:
%    \begin{macrocode}
\section{one}
some text in chapter one
%    \end{macrocode}

%\iffalse
%</samplechap1>
%\fi
% Some text for chapter 2:
%\iffalse
%<*samplechap2>
%\fi
%    \begin{macrocode}
\section{two}
more text in chapter two
%    \end{macrocode}

%\iffalse
%</samplechap2>
%\fi
%
% %%%%%%%%%%%%%%%%%%%%%%%%%%%%%%%%%%%%%%
% \paragraph{Part Include Files.}
%
% The include files are called |cdocspt3.tex| and |cdocspt4.tex|.
%
%\iffalse
%<*samplepart3|samplepart4>
%\fi

% Optional override for |\version| flag:
%    \begin{macrocode}
%%\providecommand{\version}{final}
%    \end{macrocode}

% Include the main document:
%    \begin{macrocode}
% \iffalse
%
% childdoc.dtx Copyright (C) 2017-2018 Niklas Beisert
%
% This work may be distributed and/or modified under the
% conditions of the LaTeX Project Public License, either version 1.3
% of this license or (at your option) any later version.
% The latest version of this license is in
%   http://www.latex-project.org/lppl.txt
% and version 1.3 or later is part of all distributions of LaTeX
% version 2005/12/01 or later.
%
% This work has the LPPL maintenance status `maintained'.
%
% The Current Maintainer of this work is Niklas Beisert.
%
% This work consists of the files childdoc.dtx and childdoc.ins
% and the derived files childdoc.def and cdocsamp.tex with
% cdocsch1.tex, cdocsch2.tex, cdocsdrf.tex, cdocsfn1.tex, cdocsfn2.tex.
%
%<package>\ifdefined\childdocmain\endinput\fi
%<package>\ProvidesFile{childdoc.def}[2018/12/30 v2.0 child document driver]
%<samplemain>\ProvidesFile{cdocsamp.tex}[2018/12/30 v2.0 sample for childdoc]
%<*driver>
%\ProvidesFile{childdoc.drv}[2018/12/30 v2.0 childdoc reference manual file]
\PassOptionsToClass{10pt,a4paper}{article}
\documentclass{ltxdoc}

\usepackage[margin=35mm]{geometry}
\usepackage{hyperref}
\usepackage{hyperxmp}
\usepackage[usenames]{color}

\hypersetup{colorlinks=true}
\hypersetup{pdfstartview=FitH}
\hypersetup{pdfpagemode=UseNone}
\hypersetup{pdfsource={}}
\hypersetup{pdflang={en-UK}}
\hypersetup{pdfcopyright={Copyright 2017-2018 Niklas Beisert.
  This work may be distributed and/or modified under the
  conditions of the LaTeX Project Public License, either version 1.3
  of this license or (at your option) any later version.}}
\hypersetup{pdflicenseurl={http://www.latex-project.org/lppl.txt}}
\hypersetup{pdfcontactaddress={ETH Zurich, ITP, HIT K,
  Wolfgang-Pauli-Strasse 27}}
\hypersetup{pdfcontactpostcode={8093}}
\hypersetup{pdfcontactcity={Zurich}}
\hypersetup{pdfcontactcountry={Switzerland}}
\hypersetup{pdfcontactemail={nbeisert@itp.phys.ethz.ch}}
\hypersetup{pdfcontacturl={http://people.phys.ethz.ch/\xmptilde nbeisert/}}

\newcommand{\secref}[1]{\hyperref[#1]{section \ref*{#1}}}

\parskip1ex
\parindent0pt
\let\olditemize\itemize
\def\itemize{\olditemize\parskip0pt}

\begin{document}

\title{The \textsf{childdoc} Package}
\hypersetup{pdftitle={The childdoc Package}}
\author{Niklas Beisert\\[2ex]
  Institut f\"ur Theoretische Physik\\
  Eidgen\"ossische Technische Hochschule Z\"urich\\
  Wolfgang-Pauli-Strasse 27, 8093 Z\"urich, Switzerland\\[1ex]
  \href{mailto:nbeisert@itp.phys.ethz.ch}
  {\texttt{nbeisert@itp.phys.ethz.ch}}}
\hypersetup{pdfauthor={Niklas Beisert}}
\hypersetup{pdfsubject={Manual for the LaTeX2e Package childdoc}}
\date{30 December 2018, \textsf{v2.0}}
\maketitle

\begin{abstract}\noindent
\textsf{childdoc} is a \LaTeXe{} package
that enables the direct compilation
of document sections included by |\include|
to individual files.
\end{abstract}

\begingroup
\parskip0ex
\tableofcontents
\endgroup

%%%%%%%%%%%%%%%%%%%%%%%%%%%%%%%%%%%%%%%%%%%%%%%%%%%%%%%%%%%%%%%%%%%%%%%%%%%%%%%%
%%%%%%%%%%%%%%%%%%%%%%%%%%%%%%%%%%%%%%%%%%%%%%%%%%%%%%%%%%%%%%%%%%%%%%%%%%%%%%%%
\section{Introduction}

\LaTeX{} provides a mechanism to structure a large document (such as a book)
into a main file and several child files (containing the chapters)
using the |\include| command.
This mechanism is beneficial for documents
which span hundreds of pages in order to
make the source file(s) more manageable.
Moreover, compilation can be restricted to
selected child files by means of the |\includeonly| command.
The latter feature can be used to reduce the compilation time while editing
(this was significantly more useful in the earlier days of \LaTeX{})
or to generate a smaller document which is easier to navigate.
Another application of |\includeonly| is to generate
documents consisting of selected parts of the complete document.

However, there are a few drawbacks of the plain |\include| mechanism:
\begin{itemize}
\item
The child files cannot be compiled on their own,
they can only be compiled via the main file.
A naive editing environment
(such as a text editor with an option
to have the current file processed by \LaTeX)
may require one to switch to the main file before compiling;
attempting to compile the child file produces errors.
\item
The main file must be modified (each time)
to adjust the |\includeonly| command
to the present needs. This easily leaves the main file in a messy state.
\item
The generated document will always carry the filename
of the main document. This is inconvenient if
several child files are to be compiled and
to be kept for distribution.
\end{itemize}

The present package provides a simple interface
to make child files individually compilable by \LaTeX{}.
Compiling a child file then has the same effect as compiling
the main file with an |\includeonly| command
to select the appropriate child.
Moreover the generated document will carry the name of the child
rather than the main file.
This resolves all three above issues.

This feature is meant to make the editing of books,
thesis documents and lecture notes somewhat more convenient.
However, the package can also be used efficiently for
composing a series of documents (such as exercise sheets)
which are typically distributed individually.
It then assists the author in generating the individual documents
(potentially in different versions)
as well as a document containing the collected series.
Another application is in developing style files
or other kinds of included material
where compilation of the style file could redirect
to a sample or test file.

%%%%%%%%%%%%%%%%%%%%%%%%%%%%%%%%%%%%%%%%%%%%%%%%%%%%%%%%%%%%%%%%%%%%%%%%%%%%%%%%
%%%%%%%%%%%%%%%%%%%%%%%%%%%%%%%%%%%%%%%%%%%%%%%%%%%%%%%%%%%%%%%%%%%%%%%%%%%%%%%%
\section{Usage}

First of all, the package \textsf{childdoc} is \emph{not} a standard
\LaTeXe{} |.sty| style file! Therefore it needs to be invoked in
a non-standard way.

%%%%%%%%%%%%%%%%%%%%%%%%%%%%%%%%%%%%%%%%%%%%%%%%%%%%%%%%%%%%%%%%%%%%%%%%%%%%%%%%
\subsection{Included Files}
\label{sec:include}

%%%%%%%%%%%%%%%%%%%%%%%%%%%%%%%%%%%%%%%%
\DescribeMacro{\childdocmain}
To use the package, add the commands
\begin{center}
\begin{tabular}{l}
|\input{childdoc.def}|\\
|\childdocmain{}|\\
\end{tabular}
\end{center}
at the very top of the main \LaTeX{} file,
in particular \emph{before} the |\documentclass| statement!
The argument of |\childdocmain| should be left empty
(but it must be present).

%%%%%%%%%%%%%%%%%%%%%%%%%%%%%%%%%%%%%%%%
\DescribeMacro{\childdocof}
Furthermore, add the commands
\begin{center}
\begin{tabular}{l}
|\input{childdoc.def}|\\
|\childdocof{|\textit{main}|}|\\
\end{tabular}
\end{center}
at the top of every child file \textit{child}
which is included by |\include{|\textit{child}|}|
from within the main file
(or at least for those files to be compiled individually).
The argument \textit{main} must be the filename of the main file.

There are a couple of
considerations in setting up the main and child documents:

%%%%%%%%%%%%%%%%%%%%%%%%%%%%%%%%%%%%%%%%
\paragraph{Restrictions.}

Please note the following restrictions:
\begin{itemize}
\item
|\childdocmain| must be called with one argument \textit{main}
to ensure compatibility with earlier version of the package.
It must either be empty (|\childdocmain{}|)
or precisely match the filename of the main file in which it is specified.
See \secref{sec:detection} for further information.
\item
The filename \textit{main} must be specified without the |.tex| extension.
\item
The filename \textit{main} is case sensitive
(even in case-insensitive file systems)
due to internal string comparison.
\item
The argument \textit{main} should be fully expanded, it cannot be a macro.
\item
Subdirectories and special characters should be avoided in filenames.
\item
The command |\childdocmain{|\textit{main}|}| must be followed by a whitespace.
It should not be followed immediately by another command
or by a comment mark `|%|'.
This is because the \TeX{} parser reads the token immediately following
the argument of |\childdocmain| and puts it
at the beginning of every child section;
however, a white\-space is ignored.
\end{itemize}

%%%%%%%%%%%%%%%%%%%%%%%%%%%%%%%%%%%%%%%%
\paragraph{Content of Main File.}

It is advisable to place all content in the child files included by |\include|.
Any output contained in the main file will appear in all child documents
unless suppressed manually;
it cannot be suppressed automatically by the |\includeonly| directive
and thus should normally be avoided.
A method to include some content in the main file
by means of conditional processing is described in \secref{sec:conditional}.

%%%%%%%%%%%%%%%%%%%%%%%%%%%%%%%%%%%%%%%%
\paragraph{Page Numbering.}

When only a part of the document is compiled,
the appropriate numbering of pages
(as well as other status parameters)
is determined from the |.aux| files.
The latter contain information from previous passes.
However this information needs to propagate through
all intermediate child documents.
Therefore the page numbering in child documents may well
be inconsistent until the complete document is compiled at least once.

A useful (if unconventional) way to always ensure a consistent
page numbering is to restart the numbering in each child document
and denote the pages by `\textit{child}|.|\textit{page}'
where \textit{child} represents the chapter/section number of the child file.
This can be achieved by the command
|\numberwithin{page}{|\textit{child}|}|
of the \textsf{amsmath} package
where \textit{child} can be |chapter| or |section|
depending on the chosen structuring.
Alternatively, one can modify the macro |\thepage| appropriately
and reset the counter |page| at the start of each child file.

%%%%%%%%%%%%%%%%%%%%%%%%%%%%%%%%%%%%%%%%%%%%%%%%%%%%%%%%%%%%%%%%%%%%%%%%%%%%%%%%
\subsection{Conditional Processing}
\label{sec:conditional}

The package provides a mechanism to compile different versions
of a document. To customise the versions further some conditional processing
can come in handy to distinguish which version is being compiled.
The package provides two macros to describe the compilation context:

%%%%%%%%%%%%%%%%%%%%%%%%%%%%%%%%%%%%%%%%
\DescribeMacro{\ifchilddoc}
The conditional |\ifchilddoc| distinguishes between the compilation of
child documents and the main document:
%
\begin{center}
|\ifchilddoc |\textit{child-code}| |[|\||else |\textit{main-code}]| \||fi|
\end{center}

%%%%%%%%%%%%%%%%%%%%%%%%%%%%%%%%%%%%%%%%
\DescribeMacro{\childdocname}
\DescribeMacro{\childdocjob}
The macro |\childdocname| contains the filename (without extension)
of the main or child file being processed.
Note that |\childdocjob| will always contain the name of the main file.

%%%%%%%%%%%%%%%%%%%%%%%%%%%%%%%%%%%%%%%%
\paragraph{Title Page.}

Conditional processing can be used to include a title or banner page
in the main document when proper precautions are taken.
Importantly, the code in the main file should ensure that the page counter
(as well as other status parameters which are stored in the |.aux| files)
takes the same value after the conditional processing.
Otherwise the page numbers may take divergent values
depending on which part is compiled.

For example, a title page could be declared by:
%
\begin{center}
\begin{tabular}{l}
|\ifchilddoc\||else|\\
|\addtocounter{page}{-1}|\\
\textit{code for title page}\\
|\newpage|\\
|\||fi|
\end{tabular}
\end{center}
%
A banner page for the child documents can be generated by:
%
\begin{center}
\begin{tabular}{l}
|\ifchilddoc|\\
|\addtocounter{page}{-1}|\\
\textit{code for banner page}\\
|\newpage|\\
|\||fi|
\end{tabular}
\end{center}
%
Here one could write a message such as:
\begin{center}
|This is the part \childdocname{} of \childdocjob{}.|
\end{center}

%%%%%%%%%%%%%%%%%%%%%%%%%%%%%%%%%%%%%%%%%%%%%%%%%%%%%%%%%%%%%%%%%%%%%%%%%%%%%%%%
\subsection{Flags}
\label{sec:flags}

The package makes it easy to generate different versions
of the main or child documents.
To this end compilation flags can be defined
and assigned different default values.
They will be particularly useful in conjunction
with the forwarding mechanism described in \secref{sec:forward}.

For example, it may be useful to have a flag |\version|
which can be set to |draft| or |final|.
The document source will contain some conditional code
depending on the value of |\version|.
Suppose further, the flag should default to |final| for the main file
and to |draft| for child files
which is a natural assignment for editing the document.
This is achieved by placing the following code
in the preamble of the main document
(below the |\childdocmain| directive):
%
\begin{center}
\begin{tabular}{l}
|\ifchilddoc|\\
|\providecommand{\version}{draft}|\\
|\||else|\\
|\providecommand{\version}{final}|\\
|\||fi|
\end{tabular}
\end{center}
%
The definition by |\providecommand| makes sure
that previous definitions are not overwritten.
Further statements |\providecommand{\version}{...}|
can thus be added before the above code to override it.

For the main file, one might add a line
(between |\childdocmain| and the above block)
%
\begin{center}
|%\ifchilddoc\||else\providecommand{\version}{draft}\||fi|
\end{center}
%
which can be uncommented to produce a draft version.
Likewise one can add a line to the very top of a child file
(above the |\childdocof{|\textit{main}|}| directive)
%
\begin{center}
|%\providecommand{\version}{final}|
\end{center}
%
which can be uncommented to produce the final version of this child document.

%%%%%%%%%%%%%%%%%%%%%%%%%%%%%%%%%%%%%%%%%%%%%%%%%%%%%%%%%%%%%%%%%%%%%%%%%%%%%%%%
\subsection{Forwarding}
\label{sec:forward}

Different versions of the main or child documents
using compilation flags as described in \secref{sec:flags}
can be (permanently) stored in different files
for convenient compilation, viewing and distribution.
To this end, the package defines a command
to pass on compilation to a different file:

%%%%%%%%%%%%%%%%%%%%%%%%%%%%%%%%%%%%%%%%
\DescribeMacro{\childdocforward}
The command |\childdocforward| redirects processing to
another source file:
%
\begin{center}
\begin{tabular}{l}
|\input{childdoc.def}|\\
|\childdocforward[|\textit{main}|]{|\textit{dest}|}|\\
\end{tabular}
\end{center}
%
The argument \textit{dest} is the destination file
(without extension).
It should be the main file or one of the child files.
Note that further \textsf{childdoc} directives
such as |\childdocof| and |\childdocforward|
in the indicated file will be processed in this form.
The optional argument \textit{main}
passes on directly to the main file \textit{main}
while pretending to compile the child \textit{dest}.
This form behaves as if \textit{dest}
issues |\childdocof{|\textit{main}|}| right away,
and no further \textsf{childdoc} directives will be processed.

%%%%%%%%%%%%%%%%%%%%%%%%%%%%%%%%%%%%%%%%
\DescribeMacro{\...prefix}
In the alternative form |\childdocforwardprefix|,
%
\begin{center}
\begin{tabular}{l}
|\input{childdoc.def}|\\
|\childdocforwardprefix[|\textit{main}|]{|\textit{prefix}|}{|\textit{dest}|}|
\end{tabular}
\end{center}
%
the destination file is determined by a pattern
depending on the current file:
To make this work, the current file must be called
`{\textit{prefix}\hspace{0.2em}\textit{suffix}}'
with \textit{prefix} matching precisely the argument.
Processing is then passed on to the file
`{\textit{dest}\hspace{0.2em}\textit{suffix}}'.
Surely, the same effect is achieved by
directly specifying the
argument `{\textit{dest}\hspace{0.2em}\textit{suffix}}'
in the first form.
However, that requires to set up a different file
for each child. With the alternative form of the command
all these files can have exactly the same content
which simplifies setting them up and maintaining them.

For example, the following file |draft.tex|
with a compilation flag |\version| as described in \secref{sec:flags}
compiles the main document as a draft:
%
\begin{center}
\begin{tabular}{l}
|\def\version{draft}|\\
|\input{childdoc.def}|\\
|\childdocforward{|\textit{main}|}|
\end{tabular}
\end{center}
%
Likewise, the following files |final|\textit{nn}|.tex|
compile the final version of the child document
|child|\textit{nn}|.tex|:
%
\begin{center}
\begin{tabular}{l}
|\def\version{final}|\\
|\input{childdoc.def}|\\
|\childdocforwardprefix{final}{child}|
\end{tabular}
\end{center}
%

Note that when several versions of a main file and/or of each child file
are to be generated, it may be convenient to set up a |Makefile| or
shell script to automatise the process.

%%%%%%%%%%%%%%%%%%%%%%%%%%%%%%%%%%%%%%%%%%%%%%%%%%%%%%%%%%%%%%%%%%%%%%%%%%%%%%%%
\subsection{Command Line Processing}
\label{sec:commandline}

The effect of redirection files can also be achieved by invoking
the \LaTeX{} compiler with a more elaborate command line.
Most conveniently this should be done as part
of a shell script or a |Makefile|.

When using \textsf{childdoc} in the main file, the following
command lines effectively perform a redirection
(note that depending on the shell being used,
backslashes may have to be doubled: `|\|' $\to$ `|\\|'):
%
\begin{center}
|... -jobname "|\textit{target}|" |\\|"|[\textit{flags}]%
|\input{childdoc.def}\childdocforward[|\textit{main}|]{|\textit{dest}|}"|
\end{center}
%
Here \textit{target} is the name of the output file,
\textit{main} is the name of the main file
and \textit{dest} is the name of the main or child file to be processed
(all filenames without extensions).
The optional argument \textit{main} can be omitted
if \textit{main} matches \textit{dest}.
Optionally, compilation \textit{flags} can be defined via |\def| commands.
This command line makes the \TeX{} engine believe
it is compiling the file \textit{target}
whose content is specified as the latter parameter.
The provided code then forwards the processing to
\textit{main} or \textit{dest} as described in \secref{sec:forward}.

%%%%%%%%%%%%%%%%%%%%%%%%%%%%%%%%%%%%%%%%%%%%%%%%%%%%%%%%%%%%%%%%%%%%%%%%%%%%%%%%
\subsection{Include by Input}
\label{sec:input}

Including child documents by |\include| has some restrictions by design.
Most notably, the content of a child document always occupies
its own set of pages; pages cannot be shared between child documents.
Usually, this behaviour makes perfect sense
because each child document contain an essential part of the document.
However, in some situations it may be desirable to compose
a document from a collection of parts
without having mandatory page breaks between then.
For this case, the package
provides a mechanism to include parts
by |\input| which can also be processed individually.
However, by construction this mechanism
requires manual handling of the content to be output.

%%%%%%%%%%%%%%%%%%%%%%%%%%%%%%%%%%%%%%%%
\DescribeMacro{\ifchilddocmanual}
The main file should be prepared as usual, see \secref{sec:include}.
However, the document body must make a distinction
between processing of an individual part and of the main document, e.g.:
%
\begin{center}
\begin{tabular}{l}
|\ifchilddocmanual|\\
|\input{\childdocname}|\\
|\||else|\\
\textit{document body with }|\input{|\textit{part}|}|\\
|\||fi|
\end{tabular}
\end{center}
%
The conditional |\ifchilddocmanual| is true whenever
a part to be included by |\input| is being compiled,
and the name of the part is stored in |\childdocname|.

%%%%%%%%%%%%%%%%%%%%%%%%%%%%%%%%%%%%%%%%
\DescribeMacro{\childdocby}
Each part to be included by |\input| should start with:
%
\begin{center}
\begin{tabular}{l}
|\input{childdoc.def}|\\
|\childdocby{|\textit{main}|}|\\
\end{tabular}
\end{center}
%
The directive |\childdocby| is similar to |\childdocof|
described in \secref{sec:include},
but the subsequent selection of content must be done manually.
To that end, both |\ifchilddoc| and |\ifchilddocmanual|
will be true upon processing of a part,
and the name of the part is stored in |\childdocname|.
Note that |\jobname| will be set to the filename of the current part
so that each part receives an individual |.aux| file
that does not interfere with the |.aux| file(s) of the main document.
This behaviour can be altered by the alternative form
|\childdocby[*]{|\textit{main}|}| (with a non-empty optional argument)
which uses the |.aux| file of the main document
by setting |\jobname| to \textit{main}.

%%%%%%%%%%%%%%%%%%%%%%%%%%%%%%%%%%%%%%%%%%%%%%%%%%%%%%%%%%%%%%%%%%%%%%%%%%%%%%%%
\subsection{Driver Development}
\label{sec:driver}

The \textsf{childdoc} mechanism can also be use for the development
of definition files such as \LaTeX{} styles or classes.
This case differs from the above setup with multiple parts
included by |\include| in that no |\includeonly| should be invoked.
This can be achieved by starting the include file
(before |\ProvidesPackage|) with:
%
\begin{center}
\begin{tabular}{l}
|\input{childdoc.def}|\\
|\childdocforward{|\textit{main}|}|\\
\end{tabular}
\end{center}
%
or alternatively with:
%
\begin{center}
\begin{tabular}{l}
|\input{childdoc.def}|\\
|\childdocby{|\textit{main}|}|\\
\end{tabular}
\end{center}
%
Both forms have slightly different effects as described above.
The main file is prepared as usual, see \secref{sec:include}.

%%%%%%%%%%%%%%%%%%%%%%%%%%%%%%%%%%%%%%%%%%%%%%%%%%%%%%%%%%%%%%%%%%%%%%%%%%%%%%%%
\subsection{Legacy Detection}
\label{sec:detection}

The directive |\childdocmain| in the main file can detect
whether the complete document or merely a child is to be compiled
even without using the directive |\childdocof|.
This method is deprecated because it is less robust
and there is no compelling reason to use it;
it is merely provided for backward compatibility
and it may be removed in future versions.

If the detection mechanism is to be used,
it is mandatory to correctly specify
the filename of the main file as the argument of |\childdocmain|:
%
\begin{center}
\begin{tabular}{l}
|\input{childdoc.def}|\\
|\childdocmain{|\textit{main}|}|\\
\end{tabular}
\end{center}
%
If |\jobname| does not match the argument \textit{main} of |\childdocmain|,
it is assumed that |\jobname| points to the child file to be compiled.
When using |\childdocmain| with the main file specified as argument,
it suffices to start a child file
with just |\input{|\textit{main}|}|
without loading of the package and using |\childdocof|.
If instead all processing is done
with the appropriate \textsf{childdoc} directives,
the argument of \textit{main} of |\childdocmain| can be empty.

An alternative version of the command line processing described
in \secref{sec:commandline} using the detection mechanism reads:
%
\begin{center}
|... -jobname "|\textit{target}|" "|[\textit{flags}]%
[|\def\jobname{|\textit{dest}|}|]|\input{|\textit{main}|}"|
\end{center}

%%%%%%%%%%%%%%%%%%%%%%%%%%%%%%%%%%%%%%%%%%%%%%%%%%%%%%%%%%%%%%%%%%%%%%%%%%%%%%%%
\subsection{Manual Code}
\label{sec:manual}

In case one cannot be certain whether the definitions file |childdoc.def|
is installed on the target \TeX{} distribution
and one prefers not to ship it,
it is conceivable to paste a few relevant commands into the sources.

To that end, drop all statements |\input{childdoc.def}|
and perform the replacements as outlined below.
Instead of |\childdocmain{|\textit{main}|}| add the following code
to the top of the main file:
%
\begin{center}
\begin{tabular}{l}
|\||ifdefined\childdocname\endinput\||fi\newif\ifchilddoc|\\
|\edef\childdocname{\scantokens\expandafter{\jobname\noexpand}}|\\
|\def\childdocmain{|\textit{main}|}\||ifx\childdocmain\childdocname\||else|\\
|\childdoctrue\includeonly{\childdocname}\let\jobname\childdocmain\||fi|\\
\end{tabular}
\end{center}
%
Instead of |\childdocof{|\textit{main}|}| just include the main file
at the top of each child file:
%
\begin{center}
|\input{|\textit{main}|}|
\end{center}
%
A simple redirection |\childdocforward{|\textit{dest}|}| is achieved by:
%
\begin{center}
|\def\jobname{|\textit{dest}|}\input{\jobname}|
\end{center}
%
The redirection with prefix
|\childdocforwardprefix[|\textit{prefix}|]{|\textit{dest}|}|
is accomplished by:
%
\begin{center}
\begin{tabular}{l}
|{\edef\jobname{\scantokens\expandafter{\jobname\noexpand}}|\\
|\def\redirectjob |\textit{prefix}|#1~~~{\gdef\jobname{|\textit{dest}|#1}}|\\
|\expandafter\redirectjob\jobname~~~}\input{\jobname}|
\end{tabular}
\end{center}

In an alternative approach,
child documents can be compiled by a specific command line
without additional code or specific definitions:
%
\begin{center}
|... -jobname "|\textit{target}|" "|[\textit{flags}]%
|\includeonly{|\textit{dest}|}\input{|\textit{main}|}"|
\end{center}
%

%%%%%%%%%%%%%%%%%%%%%%%%%%%%%%%%%%%%%%%%%%%%%%%%%%%%%%%%%%%%%%%%%%%%%%%%%%%%%%%%
%%%%%%%%%%%%%%%%%%%%%%%%%%%%%%%%%%%%%%%%%%%%%%%%%%%%%%%%%%%%%%%%%%%%%%%%%%%%%%%%
\section{Information}

%%%%%%%%%%%%%%%%%%%%%%%%%%%%%%%%%%%%%%%%%%%%%%%%%%%%%%%%%%%%%%%%%%%%%%%%%%%%%%%%
\subsection{Copyright}

Copyright \copyright{} 2017--2018 Niklas Beisert

This work may be distributed and/or modified under the
conditions of the \LaTeX{} Project Public License, either version 1.3
of this license or (at your option) any later version.
The latest version of this license is in
  \url{http://www.latex-project.org/lppl.txt}
and version 1.3 or later is part of all distributions of \LaTeX{}
version 2005/12/01 or later.

This work has the LPPL maintenance status `maintained'.

The Current Maintainer of this work is Niklas Beisert.

This work consists of the files |README.txt|, |childdoc.ins| and |childdoc.dtx|
as well as the derived files |childdoc.def|, |cdocsamp.tex|
with |cdocsch1.tex|, |cdocsch2.tex|, |cdocspt3.tex|, |cdocspt4.tex|,
|cdocsdrf.tex|, |cdocsfn1.tex|, |cdocsfn2.tex|
as well as |childdoc.pdf|.

%%%%%%%%%%%%%%%%%%%%%%%%%%%%%%%%%%%%%%%%%%%%%%%%%%%%%%%%%%%%%%%%%%%%%%%%%%%%%%%%
\subsection{Files and Installation}

The package consists of the files:
%
\begin{center}
\begin{tabular}{ll}
    |README.txt|   & readme file \\
    |childdoc.ins| & installation file \\
    |childdoc.dtx| & source file \\
    |childdoc.def| & definition file \\
    |cdocsamp.tex| & sample main file \\
    |cdocsch1.tex| & sample include file \\
    |cdocsch2.tex| & sample include file \\
    |cdocspt3.tex| & sample part file \\
    |cdocspt4.tex| & sample part file \\
    |cdocsdrf.tex| & sample redirection file \\
    |cdocsfn1.tex| & sample redirection file \\
    |cdocsfn2.tex| & sample redirection file \\
    |childdoc.pdf| & manual
\end{tabular}
\end{center}
%
The distribution consists of the files
|README.txt|, |childdoc.ins| and |childdoc.dtx|.
%
\begin{itemize}
\item
Run (pdf)\LaTeX{} on |childdoc.dtx|
to compile the manual |childdoc.pdf| (this file).
\item
Run \LaTeX{} on |childdoc.ins| to create the definitions file |childdoc.def|
and the sample |cdocsamp.tex| with include files
|cdocsch1.tex|, |cdocsch2.tex|, |cdocspt3.tex|, |cdocspt4.tex|,
|cdocsdrf.tex|, |cdocsfn1.tex|, |cdocsfn2.tex|.
Then copy the file |childdoc.def| to an appropriate directory of your \LaTeX{}
distribution, e.g.\ \textit{texmf-root}|/tex/latex/childdoc|.
\end{itemize}

%%%%%%%%%%%%%%%%%%%%%%%%%%%%%%%%%%%%%%%%%%%%%%%%%%%%%%%%%%%%%%%%%%%%%%%%%%%%%%%%
\subsection{Related CTAN Packages}

There are several other packages which offer a similar functionality:
%
\begin{itemize}
\item
The packages
\href{http://ctan.org/pkg/docmute}{\textsf{docmute}},
\href{http://ctan.org/pkg/includex}{\textsf{includex}} and
\href{http://ctan.org/pkg/standalone}{\textsf{standalone}}
provide commands to include only the document body of
a child file thus allowing both files to be compiled individually.
\item
The packages \href{http://ctan.org/pkg/subdocs}{\textsf{subdocs}}
and \href{http://ctan.org/pkg/subfiles}{\textsf{subfiles}}
provide structures in which the main and child documents can be
encapsulated and allowing them to be compiled individually.
The inclusion mechanism is different from the conventional |\include|.
\item
The package \href{http://ctan.org/pkg/combine}{\textsf{combine}}
is an elaborate solution to combine several documents into one.
\end{itemize}
%
See also the CTAN topic \href{http://ctan.org/topic/subdocs}{\textsf{subdocs}}
for further related packages.
The present package differs from the above solutions in that
a document structure constructed with the conventional |\include| mechanism
just needs two extra commands at the top of every file
such that all constituent files can be compiled individually.

%%%%%%%%%%%%%%%%%%%%%%%%%%%%%%%%%%%%%%%%%%%%%%%%%%%%%%%%%%%%%%%%%%%%%%%%%%%%%%%%
%\subsection{Feature Suggestions}
%
%The following is a list of features which may be useful for future
%versions of this package:
%%
%\begin{itemize}
%\item
%\ldots
%\end{itemize}

%%%%%%%%%%%%%%%%%%%%%%%%%%%%%%%%%%%%%%%%%%%%%%%%%%%%%%%%%%%%%%%%%%%%%%%%%%%%%%%%
\subsection{Revision History}

%%%%%%%%%%%%%%%%%%%%%%%%%%%%%%%%%%%%%%%%
\paragraph{v2.0:} 2018/12/30

\begin{itemize}
\item
immediate forward processing
\item
added |\childdocby| mechanism
\item
manual restructured
\end{itemize}

%%%%%%%%%%%%%%%%%%%%%%%%%%%%%%%%%%%%%%%%
\paragraph{v1.6:} 2018/01/17

\begin{itemize}
\item
application for development of include files
\item
corrections to manual
\end{itemize}

%%%%%%%%%%%%%%%%%%%%%%%%%%%%%%%%%%%%%%%%
\paragraph{v1.5:} 2017/05/21

\begin{itemize}
\item
more complete structuring introduced
\item
|\childdocof| introduced
\item
|\childdoc| renamed to |\childdocmain|
\item
|\childredirect| renamed to |\childdocforward| and |\childdocforwardprefix|
and functionality expanded
\end{itemize}

%%%%%%%%%%%%%%%%%%%%%%%%%%%%%%%%%%%%%%%%
\paragraph{v1.0:} 2017/04/27

\begin{itemize}
\item
manual and install package
\item
first version published on CTAN
\end{itemize}

%%%%%%%%%%%%%%%%%%%%%%%%%%%%%%%%%%%%%%%%
\paragraph{v0.6:} 2017/04/26

\begin{itemize}
\item
redirection mechanism added
\end{itemize}

%%%%%%%%%%%%%%%%%%%%%%%%%%%%%%%%%%%%%%%%
\paragraph{v0.5:} 2017/04/26

\begin{itemize}
\item
functionality in definition file
\end{itemize}


%%%%%%%%%%%%%%%%%%%%%%%%%%%%%%%%%%%%%%%%%%%%%%%%%%%%%%%%%%%%%%%%%%%%%%%%%%%%%%%%
%%%%%%%%%%%%%%%%%%%%%%%%%%%%%%%%%%%%%%%%%%%%%%%%%%%%%%%%%%%%%%%%%%%%%%%%%%%%%%%%
%%%%%%%%%%%%%%%%%%%%%%%%%%%%%%%%%%%%%%%%%%%%%%%%%%%%%%%%%%%%%%%%%%%%%%%%%%%%%%%%
\appendix

\settowidth\MacroIndent{\rmfamily\scriptsize 000\ }

 \DocInput{childdoc.dtx}

\end{document}
%</driver>
% \fi
%
% %%%%%%%%%%%%%%%%%%%%%%%%%%%%%%%%%%%%%%%%%%%%%%%%%%%%%%%%%%%%%%%%%%%%%%%%%%%%%%
% %%%%%%%%%%%%%%%%%%%%%%%%%%%%%%%%%%%%%%%%%%%%%%%%%%%%%%%%%%%%%%%%%%%%%%%%%%%%%%
% \section{Sample}
%\iffalse
%<*samplemain>
%\fi
%
% The following presents a sample document
% with two chapters, two parts, a title page,
% a compile flag as well as three forwarding files to set the flag.
% It consists of eight |.tex| files:
% \begin{center}
% \begin{tabular}{ll}
% |cdocsamp.tex|&main file\\
% |cdocsch1.tex|&include file for chapter 1\\
% |cdocsch2.tex|&include file for chapter 2\\
% |cdocspt3.tex|&include file for part 3\\
% |cdocspt4.tex|&include file for part 4\\
% |cdocsdrf.tex|&forwarding file for main file in draft mode\\
% |cdocsfi1.tex|&forwarding file for final version of chapter 1\\
% |cdocsfi2.tex|&forwarding file for final version of chapter 2\\
% \end{tabular}
% \end{center}
% Each of the eight files can be compiled directly by the \LaTeX{} compiler.
%
% %%%%%%%%%%%%%%%%%%%%%%%%%%%%%%%%%%%%%%
% \paragraph{Main File.}
%
% The main file is called |cdocsamp.tex|.
%
% Load the \textsf{childdoc} definitions and
% declare the filename for the main document:
%    \begin{macrocode}
\input{childdoc.def}
\childdocmain{}
%    \end{macrocode}

% Optional override for |\version| flag:
%    \begin{macrocode}
%%\ifchilddoc\else\providecommand{\version}{draft}\fi
%    \end{macrocode}

% Define the default values for the |\version| flag
% (|final| for the main file and |draft| for childs):
%    \begin{macrocode}
\ifchilddoc
\providecommand{\version}{draft}
\else
\providecommand{\version}{final}
\fi
%    \end{macrocode}

% Load the standard document class:
%    \begin{macrocode}
\documentclass[12pt]{article}
%    \end{macrocode}

% Start the document body:
%    \begin{macrocode}
\begin{document}
%    \end{macrocode}

% Declare a title page.
% Print title, part of document being processed and version flag:
%    \begin{macrocode}
\addtocounter{page}{-1}
\begin{center}
{\LARGE\bfseries{}childdoc example\par}
\vspace{1cm}
\ifchilddoc
\ifchilddocmanual part\else chapter\fi:
`\childdocname' of `\childdocjob'\par
\else
main document: `\childdocjob'\par
\fi
version: \version\par
\end{center}
\newpage
%    \end{macrocode}

% Manually include selected file,
% otherwise process as usual:
%    \begin{macrocode}
\ifchilddocmanual
\section*{part `\childdocname'}
\input{\childdocname}
\else
%    \end{macrocode}

% Include the two chapters:
%    \begin{macrocode}
\include{cdocsch1}
\include{cdocsch2}
%    \end{macrocode}

% Include the two parts unless only chapters should be displayed:
%    \begin{macrocode}
\ifchilddoc\else
\section{part three}
\input{cdocspt3}
\section{part four}
\input{cdocspt4}
\fi
%    \end{macrocode}

% Process as usual until here:
%    \begin{macrocode}
\fi
%    \end{macrocode}

% End of document body:
%    \begin{macrocode}
\end{document}
%    \end{macrocode}
%\iffalse
%</samplemain>
%\fi
%
% %%%%%%%%%%%%%%%%%%%%%%%%%%%%%%%%%%%%%%
% \paragraph{Chapter Include Files.}
%
% The include files are called |cdocsch1.tex| and |cdocsch2.tex|.
%
%\iffalse
%<*samplechap1|samplechap2>
%\fi

% Optional override for |\version| flag:
%    \begin{macrocode}
%%\providecommand{\version}{final}
%    \end{macrocode}

% Include the main document:
%    \begin{macrocode}
\input{childdoc.def}
\childdocof{cdocsamp}
%    \end{macrocode}

%\iffalse
%</samplechap1|samplechap2>
%\fi
%
%\iffalse
%<*samplechap1>
%\fi
% Some text for chapter 1:
%    \begin{macrocode}
\section{one}
some text in chapter one
%    \end{macrocode}

%\iffalse
%</samplechap1>
%\fi
% Some text for chapter 2:
%\iffalse
%<*samplechap2>
%\fi
%    \begin{macrocode}
\section{two}
more text in chapter two
%    \end{macrocode}

%\iffalse
%</samplechap2>
%\fi
%
% %%%%%%%%%%%%%%%%%%%%%%%%%%%%%%%%%%%%%%
% \paragraph{Part Include Files.}
%
% The include files are called |cdocspt3.tex| and |cdocspt4.tex|.
%
%\iffalse
%<*samplepart3|samplepart4>
%\fi

% Optional override for |\version| flag:
%    \begin{macrocode}
%%\providecommand{\version}{final}
%    \end{macrocode}

% Include the main document:
%    \begin{macrocode}
\input{childdoc.def}
\childdocby{cdocsamp}
%    \end{macrocode}

%\iffalse
%</samplepart3|samplepart4>
%\fi
%
%\iffalse
%<*samplepart3>
%\fi
% Some text for part 3:
%    \begin{macrocode}
some text in part three
%    \end{macrocode}

%\iffalse
%</samplepart3>
%\fi
% Some text for part 4:
%\iffalse
%<*samplepart4>
%\fi
%    \begin{macrocode}
more text in part four
%    \end{macrocode}

%\iffalse
%</samplepart4>
%\fi
%
% %%%%%%%%%%%%%%%%%%%%%%%%%%%%%%%%%%%%%%
% \paragraph{Forwarding for a Complete Draft.}
%
% The following forwarding file |cdocsdrf.tex|
% compiles the main document in draft mode:
%\iffalse
%<*sampledraft>
%\fi
%    \begin{macrocode}
\def\version{draft}
\input{childdoc.def}
\childdocforward{cdocsamp}
%    \end{macrocode}

%\iffalse
%</sampledraft>
%\fi
%
% %%%%%%%%%%%%%%%%%%%%%%%%%%%%%%%%%%%%%%
% \paragraph{Forwarding for Final Version of the Chapters.}
%
% The following forwarding files |cdocsfn1.tex| and |cdocsfn2.tex|
% (with identical content)
% compile the final versions of the child documents
% |cdocsch1.tex| and |cdocsch2.tex|, respectively:
%\iffalse
%<*samplefinal>
%\fi
%    \begin{macrocode}
\def\version{final}
\input{childdoc.def}
\childdocforwardprefix[cdocsamp]{cdocsfn}{cdocsch}
%    \end{macrocode}

%\iffalse
%</samplefinal>
%\fi
%
% %%%%%%%%%%%%%%%%%%%%%%%%%%%%%%%%%%%%%%
% \paragraph{Command Line Processing.}
%
% The following three command lines generate the output files
% |cdocscld|, |cdocscl1| and |cdocscl2|
% which should be identical to
% |cdocsdrf|, |cdocsch1| and |cdocsfn2|, respectively:
% \begin{center}
% \begin{tabular}{l}
% |latex -jobname cdocscld \|\\
% |  "\def\version{draft}\input{childdoc.def}\childdocforward{cdocsamp}"|\\
% |latex -jobname cdocscl1 \|\\
% |  "\input{childdoc.def}\childdocforward[cdocsamp]{cdocsch1}"|\\
% |latex -jobname cdocscl2 \|\\
% |  "\def\version{final}\input{childdoc.def}\childdocforward{cdocsch2}"|
% \end{tabular}
% \end{center}
% Note that the trailing backslash on each first line
% merely continues the input to the second line
% (for convenient cut ant paste).
% Furthermore, the command |latex| can be replaced by any
% of its alternative versions such as |pdflatex|.
%
% %%%%%%%%%%%%%%%%%%%%%%%%%%%%%%%%%%%%%%%%%%%%%%%%%%%%%%%%%%%%%%%%%%%%%%%%%%%%%%
% %%%%%%%%%%%%%%%%%%%%%%%%%%%%%%%%%%%%%%%%%%%%%%%%%%%%%%%%%%%%%%%%%%%%%%%%%%%%%%
% \section{Implementation}
%\iffalse
%<*package>
%\fi
%
% This section describes the definitions file |childdoc.def|.

% The definitions cannot be loaded using |\usepackage| or |\RequirePackage|
% which has a mechanism to prevent loading a style file more than once.
% When loading the definitions by means of |\input|
% multiple instances have to be prevented manually:
%\iffalse
%This code needs to be before the `\ProvidesFile' directive
%which is defined at the beginning of this file.
%Therefore it is also placed there and commented out here.
%</package>
%<*discard>
%\fi
%    \begin{macrocode}
\ifdefined\childdocmain\endinput\fi
%    \end{macrocode}
%\iffalse
%</discard>
%<*package>
%\fi
%
% \macro{\ifchilddoc}
% \macro{\ifchilddocmanual}
% The conditional |\ifchilddoc| tells whether a
% child (true) or main (false) document is being compiled.
% The conditional |\ifchilddocmanual| tells whether
% the |\includeonly| mechanism is used (false) or
% the selection of child files must be performed manually (true).
% The definitions initialise to false:
%    \begin{macrocode}
\newif\ifchilddoc
\newif\ifchilddocmanual
%    \end{macrocode}

% \macro{\childdocname}
% \macro{\childdocjob}
% The macro |\childdocname| stores the name of the main document
% to be compiled. The macro |\childdocjob| stores the name of
% the document on which the \LaTeX{} compiler was originally invoked.
% The content of |\jobname| cannot be compared
% to filenames specified in the source due to different catcodes.
% The following code rescans |\jobname|, stores the result
% in |\childdocname| and saves a copy in |\childdocjob|:
%    \begin{macrocode}
\edef\childdocname{\scantokens\expandafter{\jobname\noexpand}}
\let\childdocjob\childdocname
%    \end{macrocode}

% \macro{\childdocdisable}
% The macro |\childdocdisable| prevents the main file
% from being processed more than once.
% At this stage, the main document command |\childdocmain|
% is assumed to be called once again where it should do nothing.
% Any subsequent call to it should prevent
% a secondary processing of the main document
% It overwrites the forwarding commands
% |\childdocof| and |\childdocforward|
% with empty macros to prevent further inclusions of the main document:
%    \begin{macrocode}
\newcommand{\childdocdisable}
{
  \renewcommand{\childdocmain}[1]{\renewcommand{\childdocmain}[1]{\endinput}}
  \renewcommand{\childdocof}[1]{}
  \renewcommand{\childdocby}[2][]{}
  \renewcommand{\childdocforward}[2][]{}
  \renewcommand{\childdocdisable}{}
}
%    \end{macrocode}

% \macro{\childdocmain}
% The macro |\childdocmain| is to be called at the top of the main file
% with nothing or the main filename (without extension) as argument.
% First, it breaks loops.
% If the argument is not empty and does not match |\childdocname|
% (which is set by the first inclusion of |childdoc.def|),
% |\ifchilddoc| is set to true, |\includeonly| is applied to the child file
% and |\jobname| is set to the main file
% (for proper handling of |.aux| files):
%    \begin{macrocode}
\newcommand{\childdocmain}[1]
{
  \childdocdisable\childdocmain{}
  \if?#1?\else
    \begingroup
      \def\childdoctmp{#1}
      \ifx\childdoctmp\childdocname
        \def\childdoctmp{}
      \else
        \def\childdoctmp
        {
          \childdoctrue
          \includeonly{\childdocname}
          \def\childdocjob{#1}
          \def\jobname{#1}
        }
      \fi
      \expandafter
    \endgroup
    \childdoctmp
  \fi
}
%    \end{macrocode}

% \macro{\childdocof}
% The command |\childdocof| redirects
% compilation to the main file |#1|.
%    \begin{macrocode}
\newcommand{\childdocof}[1]
{
  \childdocdisable
  \childdoctrue
  \includeonly{\childdocname}
  \def\jobname{#1}
  \def\childdocjob{#1}
  \input{#1}
}
%    \end{macrocode}

% \macro{\childdocby}
% The command |\childdocby| ....
%    \begin{macrocode}
\newcommand{\childdocby}[2][]
{
  \childdocdisable
  \childdoctrue
  \childdocmanualtrue
  \if?#1?\else
    \def\jobname{#2}
  \fi
  \def\childdocjob{#2}
  \input{#2}
  \endinput
}
%    \end{macrocode}

% \macro{\childdocforward}
% The command |\childdocforward| redirects
% compilation to the main file or
% (if the optional argument is given) a child file.
% Parameters are set as if the main file
% or a child file starting with |\childdocof| was compiled.
% Then compilation is handed over to the main file:
%    \begin{macrocode}
\newcommand{\childdocforward}[2][]
{
  \begingroup
    \if?#1?
      \def\childdoctmp
      {
        \def\childdocname{#2}
        \def\childdocjob{#2}
        \def\jobname{#2}
        \input{#2}
        \endinput
      }
    \else
      \def\childdoctmp
      {
        \childdocdisable
        \def\childdocname{#2}
        \childdoctrue
        \includeonly{#2}
        \def\childdocjob{#1}
        \def\jobname{#1}
        \input{#1}
        \endinput
      }
    \fi
    \expandafter
  \endgroup
  \childdoctmp
}
%    \end{macrocode}

% \macro{\childdocforwardprefix}
% The command |\childdocforwardprefix| redirects
% compilation to the main or a child file by means of a pattern.
% The prefix |#1| in the current filename is replaced by |#2|
% and the suffix of the current filename is kept
% (it is assumed that the filename does not contain the substring `|~~~|'
% which is used as a delimiter).
% Compilation is handed over to the new file by |\childdocforward|:
%    \begin{macrocode}
\newcommand{\childdocforwardprefix}[3][]
{
  \begingroup
    \def\childdocextract #2##1~~~{\def\childdoctmp{\childdocforward[#1]{#3##1}}}
    \expandafter\childdocextract\childdocname~~~
    \expandafter
  \endgroup
  \childdoctmp
}
%    \end{macrocode}

% \macro{\childdoc}
% The deprecated macro |\childdoc| is a legacy version of |\childdocmain|:
%    \begin{macrocode}
\newcommand{\childdoc}{\childdocmain}
%    \end{macrocode}

% \macro{\childdocredirect}
% The deprecated macro |\childdocredirect| is a legacy version
% of |\childdocforward| and |\childdocforwardprefix|:
%    \begin{macrocode}
\newcommand{\childdocredirect}[2][]
{
  \begingroup
    \if?#1?
      \def\childdoctmp{\childdocforward{#2}}
    \else
      \def\childdoctmp{\childdocforwardprefix{#1}{#2}}
    \fi
    \expandafter
  \endgroup
  \childdoctmp
}
%    \end{macrocode}

%\iffalse
%</package>
%\fi
%
\endinput

\childdocby{cdocsamp}
%    \end{macrocode}

%\iffalse
%</samplepart3|samplepart4>
%\fi
%
%\iffalse
%<*samplepart3>
%\fi
% Some text for part 3:
%    \begin{macrocode}
some text in part three
%    \end{macrocode}

%\iffalse
%</samplepart3>
%\fi
% Some text for part 4:
%\iffalse
%<*samplepart4>
%\fi
%    \begin{macrocode}
more text in part four
%    \end{macrocode}

%\iffalse
%</samplepart4>
%\fi
%
% %%%%%%%%%%%%%%%%%%%%%%%%%%%%%%%%%%%%%%
% \paragraph{Forwarding for a Complete Draft.}
%
% The following forwarding file |cdocsdrf.tex|
% compiles the main document in draft mode:
%\iffalse
%<*sampledraft>
%\fi
%    \begin{macrocode}
\def\version{draft}
% \iffalse
%
% childdoc.dtx Copyright (C) 2017-2018 Niklas Beisert
%
% This work may be distributed and/or modified under the
% conditions of the LaTeX Project Public License, either version 1.3
% of this license or (at your option) any later version.
% The latest version of this license is in
%   http://www.latex-project.org/lppl.txt
% and version 1.3 or later is part of all distributions of LaTeX
% version 2005/12/01 or later.
%
% This work has the LPPL maintenance status `maintained'.
%
% The Current Maintainer of this work is Niklas Beisert.
%
% This work consists of the files childdoc.dtx and childdoc.ins
% and the derived files childdoc.def and cdocsamp.tex with
% cdocsch1.tex, cdocsch2.tex, cdocsdrf.tex, cdocsfn1.tex, cdocsfn2.tex.
%
%<package>\ifdefined\childdocmain\endinput\fi
%<package>\ProvidesFile{childdoc.def}[2018/12/30 v2.0 child document driver]
%<samplemain>\ProvidesFile{cdocsamp.tex}[2018/12/30 v2.0 sample for childdoc]
%<*driver>
%\ProvidesFile{childdoc.drv}[2018/12/30 v2.0 childdoc reference manual file]
\PassOptionsToClass{10pt,a4paper}{article}
\documentclass{ltxdoc}

\usepackage[margin=35mm]{geometry}
\usepackage{hyperref}
\usepackage{hyperxmp}
\usepackage[usenames]{color}

\hypersetup{colorlinks=true}
\hypersetup{pdfstartview=FitH}
\hypersetup{pdfpagemode=UseNone}
\hypersetup{pdfsource={}}
\hypersetup{pdflang={en-UK}}
\hypersetup{pdfcopyright={Copyright 2017-2018 Niklas Beisert.
  This work may be distributed and/or modified under the
  conditions of the LaTeX Project Public License, either version 1.3
  of this license or (at your option) any later version.}}
\hypersetup{pdflicenseurl={http://www.latex-project.org/lppl.txt}}
\hypersetup{pdfcontactaddress={ETH Zurich, ITP, HIT K,
  Wolfgang-Pauli-Strasse 27}}
\hypersetup{pdfcontactpostcode={8093}}
\hypersetup{pdfcontactcity={Zurich}}
\hypersetup{pdfcontactcountry={Switzerland}}
\hypersetup{pdfcontactemail={nbeisert@itp.phys.ethz.ch}}
\hypersetup{pdfcontacturl={http://people.phys.ethz.ch/\xmptilde nbeisert/}}

\newcommand{\secref}[1]{\hyperref[#1]{section \ref*{#1}}}

\parskip1ex
\parindent0pt
\let\olditemize\itemize
\def\itemize{\olditemize\parskip0pt}

\begin{document}

\title{The \textsf{childdoc} Package}
\hypersetup{pdftitle={The childdoc Package}}
\author{Niklas Beisert\\[2ex]
  Institut f\"ur Theoretische Physik\\
  Eidgen\"ossische Technische Hochschule Z\"urich\\
  Wolfgang-Pauli-Strasse 27, 8093 Z\"urich, Switzerland\\[1ex]
  \href{mailto:nbeisert@itp.phys.ethz.ch}
  {\texttt{nbeisert@itp.phys.ethz.ch}}}
\hypersetup{pdfauthor={Niklas Beisert}}
\hypersetup{pdfsubject={Manual for the LaTeX2e Package childdoc}}
\date{30 December 2018, \textsf{v2.0}}
\maketitle

\begin{abstract}\noindent
\textsf{childdoc} is a \LaTeXe{} package
that enables the direct compilation
of document sections included by |\include|
to individual files.
\end{abstract}

\begingroup
\parskip0ex
\tableofcontents
\endgroup

%%%%%%%%%%%%%%%%%%%%%%%%%%%%%%%%%%%%%%%%%%%%%%%%%%%%%%%%%%%%%%%%%%%%%%%%%%%%%%%%
%%%%%%%%%%%%%%%%%%%%%%%%%%%%%%%%%%%%%%%%%%%%%%%%%%%%%%%%%%%%%%%%%%%%%%%%%%%%%%%%
\section{Introduction}

\LaTeX{} provides a mechanism to structure a large document (such as a book)
into a main file and several child files (containing the chapters)
using the |\include| command.
This mechanism is beneficial for documents
which span hundreds of pages in order to
make the source file(s) more manageable.
Moreover, compilation can be restricted to
selected child files by means of the |\includeonly| command.
The latter feature can be used to reduce the compilation time while editing
(this was significantly more useful in the earlier days of \LaTeX{})
or to generate a smaller document which is easier to navigate.
Another application of |\includeonly| is to generate
documents consisting of selected parts of the complete document.

However, there are a few drawbacks of the plain |\include| mechanism:
\begin{itemize}
\item
The child files cannot be compiled on their own,
they can only be compiled via the main file.
A naive editing environment
(such as a text editor with an option
to have the current file processed by \LaTeX)
may require one to switch to the main file before compiling;
attempting to compile the child file produces errors.
\item
The main file must be modified (each time)
to adjust the |\includeonly| command
to the present needs. This easily leaves the main file in a messy state.
\item
The generated document will always carry the filename
of the main document. This is inconvenient if
several child files are to be compiled and
to be kept for distribution.
\end{itemize}

The present package provides a simple interface
to make child files individually compilable by \LaTeX{}.
Compiling a child file then has the same effect as compiling
the main file with an |\includeonly| command
to select the appropriate child.
Moreover the generated document will carry the name of the child
rather than the main file.
This resolves all three above issues.

This feature is meant to make the editing of books,
thesis documents and lecture notes somewhat more convenient.
However, the package can also be used efficiently for
composing a series of documents (such as exercise sheets)
which are typically distributed individually.
It then assists the author in generating the individual documents
(potentially in different versions)
as well as a document containing the collected series.
Another application is in developing style files
or other kinds of included material
where compilation of the style file could redirect
to a sample or test file.

%%%%%%%%%%%%%%%%%%%%%%%%%%%%%%%%%%%%%%%%%%%%%%%%%%%%%%%%%%%%%%%%%%%%%%%%%%%%%%%%
%%%%%%%%%%%%%%%%%%%%%%%%%%%%%%%%%%%%%%%%%%%%%%%%%%%%%%%%%%%%%%%%%%%%%%%%%%%%%%%%
\section{Usage}

First of all, the package \textsf{childdoc} is \emph{not} a standard
\LaTeXe{} |.sty| style file! Therefore it needs to be invoked in
a non-standard way.

%%%%%%%%%%%%%%%%%%%%%%%%%%%%%%%%%%%%%%%%%%%%%%%%%%%%%%%%%%%%%%%%%%%%%%%%%%%%%%%%
\subsection{Included Files}
\label{sec:include}

%%%%%%%%%%%%%%%%%%%%%%%%%%%%%%%%%%%%%%%%
\DescribeMacro{\childdocmain}
To use the package, add the commands
\begin{center}
\begin{tabular}{l}
|\input{childdoc.def}|\\
|\childdocmain{}|\\
\end{tabular}
\end{center}
at the very top of the main \LaTeX{} file,
in particular \emph{before} the |\documentclass| statement!
The argument of |\childdocmain| should be left empty
(but it must be present).

%%%%%%%%%%%%%%%%%%%%%%%%%%%%%%%%%%%%%%%%
\DescribeMacro{\childdocof}
Furthermore, add the commands
\begin{center}
\begin{tabular}{l}
|\input{childdoc.def}|\\
|\childdocof{|\textit{main}|}|\\
\end{tabular}
\end{center}
at the top of every child file \textit{child}
which is included by |\include{|\textit{child}|}|
from within the main file
(or at least for those files to be compiled individually).
The argument \textit{main} must be the filename of the main file.

There are a couple of
considerations in setting up the main and child documents:

%%%%%%%%%%%%%%%%%%%%%%%%%%%%%%%%%%%%%%%%
\paragraph{Restrictions.}

Please note the following restrictions:
\begin{itemize}
\item
|\childdocmain| must be called with one argument \textit{main}
to ensure compatibility with earlier version of the package.
It must either be empty (|\childdocmain{}|)
or precisely match the filename of the main file in which it is specified.
See \secref{sec:detection} for further information.
\item
The filename \textit{main} must be specified without the |.tex| extension.
\item
The filename \textit{main} is case sensitive
(even in case-insensitive file systems)
due to internal string comparison.
\item
The argument \textit{main} should be fully expanded, it cannot be a macro.
\item
Subdirectories and special characters should be avoided in filenames.
\item
The command |\childdocmain{|\textit{main}|}| must be followed by a whitespace.
It should not be followed immediately by another command
or by a comment mark `|%|'.
This is because the \TeX{} parser reads the token immediately following
the argument of |\childdocmain| and puts it
at the beginning of every child section;
however, a white\-space is ignored.
\end{itemize}

%%%%%%%%%%%%%%%%%%%%%%%%%%%%%%%%%%%%%%%%
\paragraph{Content of Main File.}

It is advisable to place all content in the child files included by |\include|.
Any output contained in the main file will appear in all child documents
unless suppressed manually;
it cannot be suppressed automatically by the |\includeonly| directive
and thus should normally be avoided.
A method to include some content in the main file
by means of conditional processing is described in \secref{sec:conditional}.

%%%%%%%%%%%%%%%%%%%%%%%%%%%%%%%%%%%%%%%%
\paragraph{Page Numbering.}

When only a part of the document is compiled,
the appropriate numbering of pages
(as well as other status parameters)
is determined from the |.aux| files.
The latter contain information from previous passes.
However this information needs to propagate through
all intermediate child documents.
Therefore the page numbering in child documents may well
be inconsistent until the complete document is compiled at least once.

A useful (if unconventional) way to always ensure a consistent
page numbering is to restart the numbering in each child document
and denote the pages by `\textit{child}|.|\textit{page}'
where \textit{child} represents the chapter/section number of the child file.
This can be achieved by the command
|\numberwithin{page}{|\textit{child}|}|
of the \textsf{amsmath} package
where \textit{child} can be |chapter| or |section|
depending on the chosen structuring.
Alternatively, one can modify the macro |\thepage| appropriately
and reset the counter |page| at the start of each child file.

%%%%%%%%%%%%%%%%%%%%%%%%%%%%%%%%%%%%%%%%%%%%%%%%%%%%%%%%%%%%%%%%%%%%%%%%%%%%%%%%
\subsection{Conditional Processing}
\label{sec:conditional}

The package provides a mechanism to compile different versions
of a document. To customise the versions further some conditional processing
can come in handy to distinguish which version is being compiled.
The package provides two macros to describe the compilation context:

%%%%%%%%%%%%%%%%%%%%%%%%%%%%%%%%%%%%%%%%
\DescribeMacro{\ifchilddoc}
The conditional |\ifchilddoc| distinguishes between the compilation of
child documents and the main document:
%
\begin{center}
|\ifchilddoc |\textit{child-code}| |[|\||else |\textit{main-code}]| \||fi|
\end{center}

%%%%%%%%%%%%%%%%%%%%%%%%%%%%%%%%%%%%%%%%
\DescribeMacro{\childdocname}
\DescribeMacro{\childdocjob}
The macro |\childdocname| contains the filename (without extension)
of the main or child file being processed.
Note that |\childdocjob| will always contain the name of the main file.

%%%%%%%%%%%%%%%%%%%%%%%%%%%%%%%%%%%%%%%%
\paragraph{Title Page.}

Conditional processing can be used to include a title or banner page
in the main document when proper precautions are taken.
Importantly, the code in the main file should ensure that the page counter
(as well as other status parameters which are stored in the |.aux| files)
takes the same value after the conditional processing.
Otherwise the page numbers may take divergent values
depending on which part is compiled.

For example, a title page could be declared by:
%
\begin{center}
\begin{tabular}{l}
|\ifchilddoc\||else|\\
|\addtocounter{page}{-1}|\\
\textit{code for title page}\\
|\newpage|\\
|\||fi|
\end{tabular}
\end{center}
%
A banner page for the child documents can be generated by:
%
\begin{center}
\begin{tabular}{l}
|\ifchilddoc|\\
|\addtocounter{page}{-1}|\\
\textit{code for banner page}\\
|\newpage|\\
|\||fi|
\end{tabular}
\end{center}
%
Here one could write a message such as:
\begin{center}
|This is the part \childdocname{} of \childdocjob{}.|
\end{center}

%%%%%%%%%%%%%%%%%%%%%%%%%%%%%%%%%%%%%%%%%%%%%%%%%%%%%%%%%%%%%%%%%%%%%%%%%%%%%%%%
\subsection{Flags}
\label{sec:flags}

The package makes it easy to generate different versions
of the main or child documents.
To this end compilation flags can be defined
and assigned different default values.
They will be particularly useful in conjunction
with the forwarding mechanism described in \secref{sec:forward}.

For example, it may be useful to have a flag |\version|
which can be set to |draft| or |final|.
The document source will contain some conditional code
depending on the value of |\version|.
Suppose further, the flag should default to |final| for the main file
and to |draft| for child files
which is a natural assignment for editing the document.
This is achieved by placing the following code
in the preamble of the main document
(below the |\childdocmain| directive):
%
\begin{center}
\begin{tabular}{l}
|\ifchilddoc|\\
|\providecommand{\version}{draft}|\\
|\||else|\\
|\providecommand{\version}{final}|\\
|\||fi|
\end{tabular}
\end{center}
%
The definition by |\providecommand| makes sure
that previous definitions are not overwritten.
Further statements |\providecommand{\version}{...}|
can thus be added before the above code to override it.

For the main file, one might add a line
(between |\childdocmain| and the above block)
%
\begin{center}
|%\ifchilddoc\||else\providecommand{\version}{draft}\||fi|
\end{center}
%
which can be uncommented to produce a draft version.
Likewise one can add a line to the very top of a child file
(above the |\childdocof{|\textit{main}|}| directive)
%
\begin{center}
|%\providecommand{\version}{final}|
\end{center}
%
which can be uncommented to produce the final version of this child document.

%%%%%%%%%%%%%%%%%%%%%%%%%%%%%%%%%%%%%%%%%%%%%%%%%%%%%%%%%%%%%%%%%%%%%%%%%%%%%%%%
\subsection{Forwarding}
\label{sec:forward}

Different versions of the main or child documents
using compilation flags as described in \secref{sec:flags}
can be (permanently) stored in different files
for convenient compilation, viewing and distribution.
To this end, the package defines a command
to pass on compilation to a different file:

%%%%%%%%%%%%%%%%%%%%%%%%%%%%%%%%%%%%%%%%
\DescribeMacro{\childdocforward}
The command |\childdocforward| redirects processing to
another source file:
%
\begin{center}
\begin{tabular}{l}
|\input{childdoc.def}|\\
|\childdocforward[|\textit{main}|]{|\textit{dest}|}|\\
\end{tabular}
\end{center}
%
The argument \textit{dest} is the destination file
(without extension).
It should be the main file or one of the child files.
Note that further \textsf{childdoc} directives
such as |\childdocof| and |\childdocforward|
in the indicated file will be processed in this form.
The optional argument \textit{main}
passes on directly to the main file \textit{main}
while pretending to compile the child \textit{dest}.
This form behaves as if \textit{dest}
issues |\childdocof{|\textit{main}|}| right away,
and no further \textsf{childdoc} directives will be processed.

%%%%%%%%%%%%%%%%%%%%%%%%%%%%%%%%%%%%%%%%
\DescribeMacro{\...prefix}
In the alternative form |\childdocforwardprefix|,
%
\begin{center}
\begin{tabular}{l}
|\input{childdoc.def}|\\
|\childdocforwardprefix[|\textit{main}|]{|\textit{prefix}|}{|\textit{dest}|}|
\end{tabular}
\end{center}
%
the destination file is determined by a pattern
depending on the current file:
To make this work, the current file must be called
`{\textit{prefix}\hspace{0.2em}\textit{suffix}}'
with \textit{prefix} matching precisely the argument.
Processing is then passed on to the file
`{\textit{dest}\hspace{0.2em}\textit{suffix}}'.
Surely, the same effect is achieved by
directly specifying the
argument `{\textit{dest}\hspace{0.2em}\textit{suffix}}'
in the first form.
However, that requires to set up a different file
for each child. With the alternative form of the command
all these files can have exactly the same content
which simplifies setting them up and maintaining them.

For example, the following file |draft.tex|
with a compilation flag |\version| as described in \secref{sec:flags}
compiles the main document as a draft:
%
\begin{center}
\begin{tabular}{l}
|\def\version{draft}|\\
|\input{childdoc.def}|\\
|\childdocforward{|\textit{main}|}|
\end{tabular}
\end{center}
%
Likewise, the following files |final|\textit{nn}|.tex|
compile the final version of the child document
|child|\textit{nn}|.tex|:
%
\begin{center}
\begin{tabular}{l}
|\def\version{final}|\\
|\input{childdoc.def}|\\
|\childdocforwardprefix{final}{child}|
\end{tabular}
\end{center}
%

Note that when several versions of a main file and/or of each child file
are to be generated, it may be convenient to set up a |Makefile| or
shell script to automatise the process.

%%%%%%%%%%%%%%%%%%%%%%%%%%%%%%%%%%%%%%%%%%%%%%%%%%%%%%%%%%%%%%%%%%%%%%%%%%%%%%%%
\subsection{Command Line Processing}
\label{sec:commandline}

The effect of redirection files can also be achieved by invoking
the \LaTeX{} compiler with a more elaborate command line.
Most conveniently this should be done as part
of a shell script or a |Makefile|.

When using \textsf{childdoc} in the main file, the following
command lines effectively perform a redirection
(note that depending on the shell being used,
backslashes may have to be doubled: `|\|' $\to$ `|\\|'):
%
\begin{center}
|... -jobname "|\textit{target}|" |\\|"|[\textit{flags}]%
|\input{childdoc.def}\childdocforward[|\textit{main}|]{|\textit{dest}|}"|
\end{center}
%
Here \textit{target} is the name of the output file,
\textit{main} is the name of the main file
and \textit{dest} is the name of the main or child file to be processed
(all filenames without extensions).
The optional argument \textit{main} can be omitted
if \textit{main} matches \textit{dest}.
Optionally, compilation \textit{flags} can be defined via |\def| commands.
This command line makes the \TeX{} engine believe
it is compiling the file \textit{target}
whose content is specified as the latter parameter.
The provided code then forwards the processing to
\textit{main} or \textit{dest} as described in \secref{sec:forward}.

%%%%%%%%%%%%%%%%%%%%%%%%%%%%%%%%%%%%%%%%%%%%%%%%%%%%%%%%%%%%%%%%%%%%%%%%%%%%%%%%
\subsection{Include by Input}
\label{sec:input}

Including child documents by |\include| has some restrictions by design.
Most notably, the content of a child document always occupies
its own set of pages; pages cannot be shared between child documents.
Usually, this behaviour makes perfect sense
because each child document contain an essential part of the document.
However, in some situations it may be desirable to compose
a document from a collection of parts
without having mandatory page breaks between then.
For this case, the package
provides a mechanism to include parts
by |\input| which can also be processed individually.
However, by construction this mechanism
requires manual handling of the content to be output.

%%%%%%%%%%%%%%%%%%%%%%%%%%%%%%%%%%%%%%%%
\DescribeMacro{\ifchilddocmanual}
The main file should be prepared as usual, see \secref{sec:include}.
However, the document body must make a distinction
between processing of an individual part and of the main document, e.g.:
%
\begin{center}
\begin{tabular}{l}
|\ifchilddocmanual|\\
|\input{\childdocname}|\\
|\||else|\\
\textit{document body with }|\input{|\textit{part}|}|\\
|\||fi|
\end{tabular}
\end{center}
%
The conditional |\ifchilddocmanual| is true whenever
a part to be included by |\input| is being compiled,
and the name of the part is stored in |\childdocname|.

%%%%%%%%%%%%%%%%%%%%%%%%%%%%%%%%%%%%%%%%
\DescribeMacro{\childdocby}
Each part to be included by |\input| should start with:
%
\begin{center}
\begin{tabular}{l}
|\input{childdoc.def}|\\
|\childdocby{|\textit{main}|}|\\
\end{tabular}
\end{center}
%
The directive |\childdocby| is similar to |\childdocof|
described in \secref{sec:include},
but the subsequent selection of content must be done manually.
To that end, both |\ifchilddoc| and |\ifchilddocmanual|
will be true upon processing of a part,
and the name of the part is stored in |\childdocname|.
Note that |\jobname| will be set to the filename of the current part
so that each part receives an individual |.aux| file
that does not interfere with the |.aux| file(s) of the main document.
This behaviour can be altered by the alternative form
|\childdocby[*]{|\textit{main}|}| (with a non-empty optional argument)
which uses the |.aux| file of the main document
by setting |\jobname| to \textit{main}.

%%%%%%%%%%%%%%%%%%%%%%%%%%%%%%%%%%%%%%%%%%%%%%%%%%%%%%%%%%%%%%%%%%%%%%%%%%%%%%%%
\subsection{Driver Development}
\label{sec:driver}

The \textsf{childdoc} mechanism can also be use for the development
of definition files such as \LaTeX{} styles or classes.
This case differs from the above setup with multiple parts
included by |\include| in that no |\includeonly| should be invoked.
This can be achieved by starting the include file
(before |\ProvidesPackage|) with:
%
\begin{center}
\begin{tabular}{l}
|\input{childdoc.def}|\\
|\childdocforward{|\textit{main}|}|\\
\end{tabular}
\end{center}
%
or alternatively with:
%
\begin{center}
\begin{tabular}{l}
|\input{childdoc.def}|\\
|\childdocby{|\textit{main}|}|\\
\end{tabular}
\end{center}
%
Both forms have slightly different effects as described above.
The main file is prepared as usual, see \secref{sec:include}.

%%%%%%%%%%%%%%%%%%%%%%%%%%%%%%%%%%%%%%%%%%%%%%%%%%%%%%%%%%%%%%%%%%%%%%%%%%%%%%%%
\subsection{Legacy Detection}
\label{sec:detection}

The directive |\childdocmain| in the main file can detect
whether the complete document or merely a child is to be compiled
even without using the directive |\childdocof|.
This method is deprecated because it is less robust
and there is no compelling reason to use it;
it is merely provided for backward compatibility
and it may be removed in future versions.

If the detection mechanism is to be used,
it is mandatory to correctly specify
the filename of the main file as the argument of |\childdocmain|:
%
\begin{center}
\begin{tabular}{l}
|\input{childdoc.def}|\\
|\childdocmain{|\textit{main}|}|\\
\end{tabular}
\end{center}
%
If |\jobname| does not match the argument \textit{main} of |\childdocmain|,
it is assumed that |\jobname| points to the child file to be compiled.
When using |\childdocmain| with the main file specified as argument,
it suffices to start a child file
with just |\input{|\textit{main}|}|
without loading of the package and using |\childdocof|.
If instead all processing is done
with the appropriate \textsf{childdoc} directives,
the argument of \textit{main} of |\childdocmain| can be empty.

An alternative version of the command line processing described
in \secref{sec:commandline} using the detection mechanism reads:
%
\begin{center}
|... -jobname "|\textit{target}|" "|[\textit{flags}]%
[|\def\jobname{|\textit{dest}|}|]|\input{|\textit{main}|}"|
\end{center}

%%%%%%%%%%%%%%%%%%%%%%%%%%%%%%%%%%%%%%%%%%%%%%%%%%%%%%%%%%%%%%%%%%%%%%%%%%%%%%%%
\subsection{Manual Code}
\label{sec:manual}

In case one cannot be certain whether the definitions file |childdoc.def|
is installed on the target \TeX{} distribution
and one prefers not to ship it,
it is conceivable to paste a few relevant commands into the sources.

To that end, drop all statements |\input{childdoc.def}|
and perform the replacements as outlined below.
Instead of |\childdocmain{|\textit{main}|}| add the following code
to the top of the main file:
%
\begin{center}
\begin{tabular}{l}
|\||ifdefined\childdocname\endinput\||fi\newif\ifchilddoc|\\
|\edef\childdocname{\scantokens\expandafter{\jobname\noexpand}}|\\
|\def\childdocmain{|\textit{main}|}\||ifx\childdocmain\childdocname\||else|\\
|\childdoctrue\includeonly{\childdocname}\let\jobname\childdocmain\||fi|\\
\end{tabular}
\end{center}
%
Instead of |\childdocof{|\textit{main}|}| just include the main file
at the top of each child file:
%
\begin{center}
|\input{|\textit{main}|}|
\end{center}
%
A simple redirection |\childdocforward{|\textit{dest}|}| is achieved by:
%
\begin{center}
|\def\jobname{|\textit{dest}|}\input{\jobname}|
\end{center}
%
The redirection with prefix
|\childdocforwardprefix[|\textit{prefix}|]{|\textit{dest}|}|
is accomplished by:
%
\begin{center}
\begin{tabular}{l}
|{\edef\jobname{\scantokens\expandafter{\jobname\noexpand}}|\\
|\def\redirectjob |\textit{prefix}|#1~~~{\gdef\jobname{|\textit{dest}|#1}}|\\
|\expandafter\redirectjob\jobname~~~}\input{\jobname}|
\end{tabular}
\end{center}

In an alternative approach,
child documents can be compiled by a specific command line
without additional code or specific definitions:
%
\begin{center}
|... -jobname "|\textit{target}|" "|[\textit{flags}]%
|\includeonly{|\textit{dest}|}\input{|\textit{main}|}"|
\end{center}
%

%%%%%%%%%%%%%%%%%%%%%%%%%%%%%%%%%%%%%%%%%%%%%%%%%%%%%%%%%%%%%%%%%%%%%%%%%%%%%%%%
%%%%%%%%%%%%%%%%%%%%%%%%%%%%%%%%%%%%%%%%%%%%%%%%%%%%%%%%%%%%%%%%%%%%%%%%%%%%%%%%
\section{Information}

%%%%%%%%%%%%%%%%%%%%%%%%%%%%%%%%%%%%%%%%%%%%%%%%%%%%%%%%%%%%%%%%%%%%%%%%%%%%%%%%
\subsection{Copyright}

Copyright \copyright{} 2017--2018 Niklas Beisert

This work may be distributed and/or modified under the
conditions of the \LaTeX{} Project Public License, either version 1.3
of this license or (at your option) any later version.
The latest version of this license is in
  \url{http://www.latex-project.org/lppl.txt}
and version 1.3 or later is part of all distributions of \LaTeX{}
version 2005/12/01 or later.

This work has the LPPL maintenance status `maintained'.

The Current Maintainer of this work is Niklas Beisert.

This work consists of the files |README.txt|, |childdoc.ins| and |childdoc.dtx|
as well as the derived files |childdoc.def|, |cdocsamp.tex|
with |cdocsch1.tex|, |cdocsch2.tex|, |cdocspt3.tex|, |cdocspt4.tex|,
|cdocsdrf.tex|, |cdocsfn1.tex|, |cdocsfn2.tex|
as well as |childdoc.pdf|.

%%%%%%%%%%%%%%%%%%%%%%%%%%%%%%%%%%%%%%%%%%%%%%%%%%%%%%%%%%%%%%%%%%%%%%%%%%%%%%%%
\subsection{Files and Installation}

The package consists of the files:
%
\begin{center}
\begin{tabular}{ll}
    |README.txt|   & readme file \\
    |childdoc.ins| & installation file \\
    |childdoc.dtx| & source file \\
    |childdoc.def| & definition file \\
    |cdocsamp.tex| & sample main file \\
    |cdocsch1.tex| & sample include file \\
    |cdocsch2.tex| & sample include file \\
    |cdocspt3.tex| & sample part file \\
    |cdocspt4.tex| & sample part file \\
    |cdocsdrf.tex| & sample redirection file \\
    |cdocsfn1.tex| & sample redirection file \\
    |cdocsfn2.tex| & sample redirection file \\
    |childdoc.pdf| & manual
\end{tabular}
\end{center}
%
The distribution consists of the files
|README.txt|, |childdoc.ins| and |childdoc.dtx|.
%
\begin{itemize}
\item
Run (pdf)\LaTeX{} on |childdoc.dtx|
to compile the manual |childdoc.pdf| (this file).
\item
Run \LaTeX{} on |childdoc.ins| to create the definitions file |childdoc.def|
and the sample |cdocsamp.tex| with include files
|cdocsch1.tex|, |cdocsch2.tex|, |cdocspt3.tex|, |cdocspt4.tex|,
|cdocsdrf.tex|, |cdocsfn1.tex|, |cdocsfn2.tex|.
Then copy the file |childdoc.def| to an appropriate directory of your \LaTeX{}
distribution, e.g.\ \textit{texmf-root}|/tex/latex/childdoc|.
\end{itemize}

%%%%%%%%%%%%%%%%%%%%%%%%%%%%%%%%%%%%%%%%%%%%%%%%%%%%%%%%%%%%%%%%%%%%%%%%%%%%%%%%
\subsection{Related CTAN Packages}

There are several other packages which offer a similar functionality:
%
\begin{itemize}
\item
The packages
\href{http://ctan.org/pkg/docmute}{\textsf{docmute}},
\href{http://ctan.org/pkg/includex}{\textsf{includex}} and
\href{http://ctan.org/pkg/standalone}{\textsf{standalone}}
provide commands to include only the document body of
a child file thus allowing both files to be compiled individually.
\item
The packages \href{http://ctan.org/pkg/subdocs}{\textsf{subdocs}}
and \href{http://ctan.org/pkg/subfiles}{\textsf{subfiles}}
provide structures in which the main and child documents can be
encapsulated and allowing them to be compiled individually.
The inclusion mechanism is different from the conventional |\include|.
\item
The package \href{http://ctan.org/pkg/combine}{\textsf{combine}}
is an elaborate solution to combine several documents into one.
\end{itemize}
%
See also the CTAN topic \href{http://ctan.org/topic/subdocs}{\textsf{subdocs}}
for further related packages.
The present package differs from the above solutions in that
a document structure constructed with the conventional |\include| mechanism
just needs two extra commands at the top of every file
such that all constituent files can be compiled individually.

%%%%%%%%%%%%%%%%%%%%%%%%%%%%%%%%%%%%%%%%%%%%%%%%%%%%%%%%%%%%%%%%%%%%%%%%%%%%%%%%
%\subsection{Feature Suggestions}
%
%The following is a list of features which may be useful for future
%versions of this package:
%%
%\begin{itemize}
%\item
%\ldots
%\end{itemize}

%%%%%%%%%%%%%%%%%%%%%%%%%%%%%%%%%%%%%%%%%%%%%%%%%%%%%%%%%%%%%%%%%%%%%%%%%%%%%%%%
\subsection{Revision History}

%%%%%%%%%%%%%%%%%%%%%%%%%%%%%%%%%%%%%%%%
\paragraph{v2.0:} 2018/12/30

\begin{itemize}
\item
immediate forward processing
\item
added |\childdocby| mechanism
\item
manual restructured
\end{itemize}

%%%%%%%%%%%%%%%%%%%%%%%%%%%%%%%%%%%%%%%%
\paragraph{v1.6:} 2018/01/17

\begin{itemize}
\item
application for development of include files
\item
corrections to manual
\end{itemize}

%%%%%%%%%%%%%%%%%%%%%%%%%%%%%%%%%%%%%%%%
\paragraph{v1.5:} 2017/05/21

\begin{itemize}
\item
more complete structuring introduced
\item
|\childdocof| introduced
\item
|\childdoc| renamed to |\childdocmain|
\item
|\childredirect| renamed to |\childdocforward| and |\childdocforwardprefix|
and functionality expanded
\end{itemize}

%%%%%%%%%%%%%%%%%%%%%%%%%%%%%%%%%%%%%%%%
\paragraph{v1.0:} 2017/04/27

\begin{itemize}
\item
manual and install package
\item
first version published on CTAN
\end{itemize}

%%%%%%%%%%%%%%%%%%%%%%%%%%%%%%%%%%%%%%%%
\paragraph{v0.6:} 2017/04/26

\begin{itemize}
\item
redirection mechanism added
\end{itemize}

%%%%%%%%%%%%%%%%%%%%%%%%%%%%%%%%%%%%%%%%
\paragraph{v0.5:} 2017/04/26

\begin{itemize}
\item
functionality in definition file
\end{itemize}


%%%%%%%%%%%%%%%%%%%%%%%%%%%%%%%%%%%%%%%%%%%%%%%%%%%%%%%%%%%%%%%%%%%%%%%%%%%%%%%%
%%%%%%%%%%%%%%%%%%%%%%%%%%%%%%%%%%%%%%%%%%%%%%%%%%%%%%%%%%%%%%%%%%%%%%%%%%%%%%%%
%%%%%%%%%%%%%%%%%%%%%%%%%%%%%%%%%%%%%%%%%%%%%%%%%%%%%%%%%%%%%%%%%%%%%%%%%%%%%%%%
\appendix

\settowidth\MacroIndent{\rmfamily\scriptsize 000\ }

 \DocInput{childdoc.dtx}

\end{document}
%</driver>
% \fi
%
% %%%%%%%%%%%%%%%%%%%%%%%%%%%%%%%%%%%%%%%%%%%%%%%%%%%%%%%%%%%%%%%%%%%%%%%%%%%%%%
% %%%%%%%%%%%%%%%%%%%%%%%%%%%%%%%%%%%%%%%%%%%%%%%%%%%%%%%%%%%%%%%%%%%%%%%%%%%%%%
% \section{Sample}
%\iffalse
%<*samplemain>
%\fi
%
% The following presents a sample document
% with two chapters, two parts, a title page,
% a compile flag as well as three forwarding files to set the flag.
% It consists of eight |.tex| files:
% \begin{center}
% \begin{tabular}{ll}
% |cdocsamp.tex|&main file\\
% |cdocsch1.tex|&include file for chapter 1\\
% |cdocsch2.tex|&include file for chapter 2\\
% |cdocspt3.tex|&include file for part 3\\
% |cdocspt4.tex|&include file for part 4\\
% |cdocsdrf.tex|&forwarding file for main file in draft mode\\
% |cdocsfi1.tex|&forwarding file for final version of chapter 1\\
% |cdocsfi2.tex|&forwarding file for final version of chapter 2\\
% \end{tabular}
% \end{center}
% Each of the eight files can be compiled directly by the \LaTeX{} compiler.
%
% %%%%%%%%%%%%%%%%%%%%%%%%%%%%%%%%%%%%%%
% \paragraph{Main File.}
%
% The main file is called |cdocsamp.tex|.
%
% Load the \textsf{childdoc} definitions and
% declare the filename for the main document:
%    \begin{macrocode}
\input{childdoc.def}
\childdocmain{}
%    \end{macrocode}

% Optional override for |\version| flag:
%    \begin{macrocode}
%%\ifchilddoc\else\providecommand{\version}{draft}\fi
%    \end{macrocode}

% Define the default values for the |\version| flag
% (|final| for the main file and |draft| for childs):
%    \begin{macrocode}
\ifchilddoc
\providecommand{\version}{draft}
\else
\providecommand{\version}{final}
\fi
%    \end{macrocode}

% Load the standard document class:
%    \begin{macrocode}
\documentclass[12pt]{article}
%    \end{macrocode}

% Start the document body:
%    \begin{macrocode}
\begin{document}
%    \end{macrocode}

% Declare a title page.
% Print title, part of document being processed and version flag:
%    \begin{macrocode}
\addtocounter{page}{-1}
\begin{center}
{\LARGE\bfseries{}childdoc example\par}
\vspace{1cm}
\ifchilddoc
\ifchilddocmanual part\else chapter\fi:
`\childdocname' of `\childdocjob'\par
\else
main document: `\childdocjob'\par
\fi
version: \version\par
\end{center}
\newpage
%    \end{macrocode}

% Manually include selected file,
% otherwise process as usual:
%    \begin{macrocode}
\ifchilddocmanual
\section*{part `\childdocname'}
\input{\childdocname}
\else
%    \end{macrocode}

% Include the two chapters:
%    \begin{macrocode}
\include{cdocsch1}
\include{cdocsch2}
%    \end{macrocode}

% Include the two parts unless only chapters should be displayed:
%    \begin{macrocode}
\ifchilddoc\else
\section{part three}
\input{cdocspt3}
\section{part four}
\input{cdocspt4}
\fi
%    \end{macrocode}

% Process as usual until here:
%    \begin{macrocode}
\fi
%    \end{macrocode}

% End of document body:
%    \begin{macrocode}
\end{document}
%    \end{macrocode}
%\iffalse
%</samplemain>
%\fi
%
% %%%%%%%%%%%%%%%%%%%%%%%%%%%%%%%%%%%%%%
% \paragraph{Chapter Include Files.}
%
% The include files are called |cdocsch1.tex| and |cdocsch2.tex|.
%
%\iffalse
%<*samplechap1|samplechap2>
%\fi

% Optional override for |\version| flag:
%    \begin{macrocode}
%%\providecommand{\version}{final}
%    \end{macrocode}

% Include the main document:
%    \begin{macrocode}
\input{childdoc.def}
\childdocof{cdocsamp}
%    \end{macrocode}

%\iffalse
%</samplechap1|samplechap2>
%\fi
%
%\iffalse
%<*samplechap1>
%\fi
% Some text for chapter 1:
%    \begin{macrocode}
\section{one}
some text in chapter one
%    \end{macrocode}

%\iffalse
%</samplechap1>
%\fi
% Some text for chapter 2:
%\iffalse
%<*samplechap2>
%\fi
%    \begin{macrocode}
\section{two}
more text in chapter two
%    \end{macrocode}

%\iffalse
%</samplechap2>
%\fi
%
% %%%%%%%%%%%%%%%%%%%%%%%%%%%%%%%%%%%%%%
% \paragraph{Part Include Files.}
%
% The include files are called |cdocspt3.tex| and |cdocspt4.tex|.
%
%\iffalse
%<*samplepart3|samplepart4>
%\fi

% Optional override for |\version| flag:
%    \begin{macrocode}
%%\providecommand{\version}{final}
%    \end{macrocode}

% Include the main document:
%    \begin{macrocode}
\input{childdoc.def}
\childdocby{cdocsamp}
%    \end{macrocode}

%\iffalse
%</samplepart3|samplepart4>
%\fi
%
%\iffalse
%<*samplepart3>
%\fi
% Some text for part 3:
%    \begin{macrocode}
some text in part three
%    \end{macrocode}

%\iffalse
%</samplepart3>
%\fi
% Some text for part 4:
%\iffalse
%<*samplepart4>
%\fi
%    \begin{macrocode}
more text in part four
%    \end{macrocode}

%\iffalse
%</samplepart4>
%\fi
%
% %%%%%%%%%%%%%%%%%%%%%%%%%%%%%%%%%%%%%%
% \paragraph{Forwarding for a Complete Draft.}
%
% The following forwarding file |cdocsdrf.tex|
% compiles the main document in draft mode:
%\iffalse
%<*sampledraft>
%\fi
%    \begin{macrocode}
\def\version{draft}
\input{childdoc.def}
\childdocforward{cdocsamp}
%    \end{macrocode}

%\iffalse
%</sampledraft>
%\fi
%
% %%%%%%%%%%%%%%%%%%%%%%%%%%%%%%%%%%%%%%
% \paragraph{Forwarding for Final Version of the Chapters.}
%
% The following forwarding files |cdocsfn1.tex| and |cdocsfn2.tex|
% (with identical content)
% compile the final versions of the child documents
% |cdocsch1.tex| and |cdocsch2.tex|, respectively:
%\iffalse
%<*samplefinal>
%\fi
%    \begin{macrocode}
\def\version{final}
\input{childdoc.def}
\childdocforwardprefix[cdocsamp]{cdocsfn}{cdocsch}
%    \end{macrocode}

%\iffalse
%</samplefinal>
%\fi
%
% %%%%%%%%%%%%%%%%%%%%%%%%%%%%%%%%%%%%%%
% \paragraph{Command Line Processing.}
%
% The following three command lines generate the output files
% |cdocscld|, |cdocscl1| and |cdocscl2|
% which should be identical to
% |cdocsdrf|, |cdocsch1| and |cdocsfn2|, respectively:
% \begin{center}
% \begin{tabular}{l}
% |latex -jobname cdocscld \|\\
% |  "\def\version{draft}\input{childdoc.def}\childdocforward{cdocsamp}"|\\
% |latex -jobname cdocscl1 \|\\
% |  "\input{childdoc.def}\childdocforward[cdocsamp]{cdocsch1}"|\\
% |latex -jobname cdocscl2 \|\\
% |  "\def\version{final}\input{childdoc.def}\childdocforward{cdocsch2}"|
% \end{tabular}
% \end{center}
% Note that the trailing backslash on each first line
% merely continues the input to the second line
% (for convenient cut ant paste).
% Furthermore, the command |latex| can be replaced by any
% of its alternative versions such as |pdflatex|.
%
% %%%%%%%%%%%%%%%%%%%%%%%%%%%%%%%%%%%%%%%%%%%%%%%%%%%%%%%%%%%%%%%%%%%%%%%%%%%%%%
% %%%%%%%%%%%%%%%%%%%%%%%%%%%%%%%%%%%%%%%%%%%%%%%%%%%%%%%%%%%%%%%%%%%%%%%%%%%%%%
% \section{Implementation}
%\iffalse
%<*package>
%\fi
%
% This section describes the definitions file |childdoc.def|.

% The definitions cannot be loaded using |\usepackage| or |\RequirePackage|
% which has a mechanism to prevent loading a style file more than once.
% When loading the definitions by means of |\input|
% multiple instances have to be prevented manually:
%\iffalse
%This code needs to be before the `\ProvidesFile' directive
%which is defined at the beginning of this file.
%Therefore it is also placed there and commented out here.
%</package>
%<*discard>
%\fi
%    \begin{macrocode}
\ifdefined\childdocmain\endinput\fi
%    \end{macrocode}
%\iffalse
%</discard>
%<*package>
%\fi
%
% \macro{\ifchilddoc}
% \macro{\ifchilddocmanual}
% The conditional |\ifchilddoc| tells whether a
% child (true) or main (false) document is being compiled.
% The conditional |\ifchilddocmanual| tells whether
% the |\includeonly| mechanism is used (false) or
% the selection of child files must be performed manually (true).
% The definitions initialise to false:
%    \begin{macrocode}
\newif\ifchilddoc
\newif\ifchilddocmanual
%    \end{macrocode}

% \macro{\childdocname}
% \macro{\childdocjob}
% The macro |\childdocname| stores the name of the main document
% to be compiled. The macro |\childdocjob| stores the name of
% the document on which the \LaTeX{} compiler was originally invoked.
% The content of |\jobname| cannot be compared
% to filenames specified in the source due to different catcodes.
% The following code rescans |\jobname|, stores the result
% in |\childdocname| and saves a copy in |\childdocjob|:
%    \begin{macrocode}
\edef\childdocname{\scantokens\expandafter{\jobname\noexpand}}
\let\childdocjob\childdocname
%    \end{macrocode}

% \macro{\childdocdisable}
% The macro |\childdocdisable| prevents the main file
% from being processed more than once.
% At this stage, the main document command |\childdocmain|
% is assumed to be called once again where it should do nothing.
% Any subsequent call to it should prevent
% a secondary processing of the main document
% It overwrites the forwarding commands
% |\childdocof| and |\childdocforward|
% with empty macros to prevent further inclusions of the main document:
%    \begin{macrocode}
\newcommand{\childdocdisable}
{
  \renewcommand{\childdocmain}[1]{\renewcommand{\childdocmain}[1]{\endinput}}
  \renewcommand{\childdocof}[1]{}
  \renewcommand{\childdocby}[2][]{}
  \renewcommand{\childdocforward}[2][]{}
  \renewcommand{\childdocdisable}{}
}
%    \end{macrocode}

% \macro{\childdocmain}
% The macro |\childdocmain| is to be called at the top of the main file
% with nothing or the main filename (without extension) as argument.
% First, it breaks loops.
% If the argument is not empty and does not match |\childdocname|
% (which is set by the first inclusion of |childdoc.def|),
% |\ifchilddoc| is set to true, |\includeonly| is applied to the child file
% and |\jobname| is set to the main file
% (for proper handling of |.aux| files):
%    \begin{macrocode}
\newcommand{\childdocmain}[1]
{
  \childdocdisable\childdocmain{}
  \if?#1?\else
    \begingroup
      \def\childdoctmp{#1}
      \ifx\childdoctmp\childdocname
        \def\childdoctmp{}
      \else
        \def\childdoctmp
        {
          \childdoctrue
          \includeonly{\childdocname}
          \def\childdocjob{#1}
          \def\jobname{#1}
        }
      \fi
      \expandafter
    \endgroup
    \childdoctmp
  \fi
}
%    \end{macrocode}

% \macro{\childdocof}
% The command |\childdocof| redirects
% compilation to the main file |#1|.
%    \begin{macrocode}
\newcommand{\childdocof}[1]
{
  \childdocdisable
  \childdoctrue
  \includeonly{\childdocname}
  \def\jobname{#1}
  \def\childdocjob{#1}
  \input{#1}
}
%    \end{macrocode}

% \macro{\childdocby}
% The command |\childdocby| ....
%    \begin{macrocode}
\newcommand{\childdocby}[2][]
{
  \childdocdisable
  \childdoctrue
  \childdocmanualtrue
  \if?#1?\else
    \def\jobname{#2}
  \fi
  \def\childdocjob{#2}
  \input{#2}
  \endinput
}
%    \end{macrocode}

% \macro{\childdocforward}
% The command |\childdocforward| redirects
% compilation to the main file or
% (if the optional argument is given) a child file.
% Parameters are set as if the main file
% or a child file starting with |\childdocof| was compiled.
% Then compilation is handed over to the main file:
%    \begin{macrocode}
\newcommand{\childdocforward}[2][]
{
  \begingroup
    \if?#1?
      \def\childdoctmp
      {
        \def\childdocname{#2}
        \def\childdocjob{#2}
        \def\jobname{#2}
        \input{#2}
        \endinput
      }
    \else
      \def\childdoctmp
      {
        \childdocdisable
        \def\childdocname{#2}
        \childdoctrue
        \includeonly{#2}
        \def\childdocjob{#1}
        \def\jobname{#1}
        \input{#1}
        \endinput
      }
    \fi
    \expandafter
  \endgroup
  \childdoctmp
}
%    \end{macrocode}

% \macro{\childdocforwardprefix}
% The command |\childdocforwardprefix| redirects
% compilation to the main or a child file by means of a pattern.
% The prefix |#1| in the current filename is replaced by |#2|
% and the suffix of the current filename is kept
% (it is assumed that the filename does not contain the substring `|~~~|'
% which is used as a delimiter).
% Compilation is handed over to the new file by |\childdocforward|:
%    \begin{macrocode}
\newcommand{\childdocforwardprefix}[3][]
{
  \begingroup
    \def\childdocextract #2##1~~~{\def\childdoctmp{\childdocforward[#1]{#3##1}}}
    \expandafter\childdocextract\childdocname~~~
    \expandafter
  \endgroup
  \childdoctmp
}
%    \end{macrocode}

% \macro{\childdoc}
% The deprecated macro |\childdoc| is a legacy version of |\childdocmain|:
%    \begin{macrocode}
\newcommand{\childdoc}{\childdocmain}
%    \end{macrocode}

% \macro{\childdocredirect}
% The deprecated macro |\childdocredirect| is a legacy version
% of |\childdocforward| and |\childdocforwardprefix|:
%    \begin{macrocode}
\newcommand{\childdocredirect}[2][]
{
  \begingroup
    \if?#1?
      \def\childdoctmp{\childdocforward{#2}}
    \else
      \def\childdoctmp{\childdocforwardprefix{#1}{#2}}
    \fi
    \expandafter
  \endgroup
  \childdoctmp
}
%    \end{macrocode}

%\iffalse
%</package>
%\fi
%
\endinput

\childdocforward{cdocsamp}
%    \end{macrocode}

%\iffalse
%</sampledraft>
%\fi
%
% %%%%%%%%%%%%%%%%%%%%%%%%%%%%%%%%%%%%%%
% \paragraph{Forwarding for Final Version of the Chapters.}
%
% The following forwarding files |cdocsfn1.tex| and |cdocsfn2.tex|
% (with identical content)
% compile the final versions of the child documents
% |cdocsch1.tex| and |cdocsch2.tex|, respectively:
%\iffalse
%<*samplefinal>
%\fi
%    \begin{macrocode}
\def\version{final}
% \iffalse
%
% childdoc.dtx Copyright (C) 2017-2018 Niklas Beisert
%
% This work may be distributed and/or modified under the
% conditions of the LaTeX Project Public License, either version 1.3
% of this license or (at your option) any later version.
% The latest version of this license is in
%   http://www.latex-project.org/lppl.txt
% and version 1.3 or later is part of all distributions of LaTeX
% version 2005/12/01 or later.
%
% This work has the LPPL maintenance status `maintained'.
%
% The Current Maintainer of this work is Niklas Beisert.
%
% This work consists of the files childdoc.dtx and childdoc.ins
% and the derived files childdoc.def and cdocsamp.tex with
% cdocsch1.tex, cdocsch2.tex, cdocsdrf.tex, cdocsfn1.tex, cdocsfn2.tex.
%
%<package>\ifdefined\childdocmain\endinput\fi
%<package>\ProvidesFile{childdoc.def}[2018/12/30 v2.0 child document driver]
%<samplemain>\ProvidesFile{cdocsamp.tex}[2018/12/30 v2.0 sample for childdoc]
%<*driver>
%\ProvidesFile{childdoc.drv}[2018/12/30 v2.0 childdoc reference manual file]
\PassOptionsToClass{10pt,a4paper}{article}
\documentclass{ltxdoc}

\usepackage[margin=35mm]{geometry}
\usepackage{hyperref}
\usepackage{hyperxmp}
\usepackage[usenames]{color}

\hypersetup{colorlinks=true}
\hypersetup{pdfstartview=FitH}
\hypersetup{pdfpagemode=UseNone}
\hypersetup{pdfsource={}}
\hypersetup{pdflang={en-UK}}
\hypersetup{pdfcopyright={Copyright 2017-2018 Niklas Beisert.
  This work may be distributed and/or modified under the
  conditions of the LaTeX Project Public License, either version 1.3
  of this license or (at your option) any later version.}}
\hypersetup{pdflicenseurl={http://www.latex-project.org/lppl.txt}}
\hypersetup{pdfcontactaddress={ETH Zurich, ITP, HIT K,
  Wolfgang-Pauli-Strasse 27}}
\hypersetup{pdfcontactpostcode={8093}}
\hypersetup{pdfcontactcity={Zurich}}
\hypersetup{pdfcontactcountry={Switzerland}}
\hypersetup{pdfcontactemail={nbeisert@itp.phys.ethz.ch}}
\hypersetup{pdfcontacturl={http://people.phys.ethz.ch/\xmptilde nbeisert/}}

\newcommand{\secref}[1]{\hyperref[#1]{section \ref*{#1}}}

\parskip1ex
\parindent0pt
\let\olditemize\itemize
\def\itemize{\olditemize\parskip0pt}

\begin{document}

\title{The \textsf{childdoc} Package}
\hypersetup{pdftitle={The childdoc Package}}
\author{Niklas Beisert\\[2ex]
  Institut f\"ur Theoretische Physik\\
  Eidgen\"ossische Technische Hochschule Z\"urich\\
  Wolfgang-Pauli-Strasse 27, 8093 Z\"urich, Switzerland\\[1ex]
  \href{mailto:nbeisert@itp.phys.ethz.ch}
  {\texttt{nbeisert@itp.phys.ethz.ch}}}
\hypersetup{pdfauthor={Niklas Beisert}}
\hypersetup{pdfsubject={Manual for the LaTeX2e Package childdoc}}
\date{30 December 2018, \textsf{v2.0}}
\maketitle

\begin{abstract}\noindent
\textsf{childdoc} is a \LaTeXe{} package
that enables the direct compilation
of document sections included by |\include|
to individual files.
\end{abstract}

\begingroup
\parskip0ex
\tableofcontents
\endgroup

%%%%%%%%%%%%%%%%%%%%%%%%%%%%%%%%%%%%%%%%%%%%%%%%%%%%%%%%%%%%%%%%%%%%%%%%%%%%%%%%
%%%%%%%%%%%%%%%%%%%%%%%%%%%%%%%%%%%%%%%%%%%%%%%%%%%%%%%%%%%%%%%%%%%%%%%%%%%%%%%%
\section{Introduction}

\LaTeX{} provides a mechanism to structure a large document (such as a book)
into a main file and several child files (containing the chapters)
using the |\include| command.
This mechanism is beneficial for documents
which span hundreds of pages in order to
make the source file(s) more manageable.
Moreover, compilation can be restricted to
selected child files by means of the |\includeonly| command.
The latter feature can be used to reduce the compilation time while editing
(this was significantly more useful in the earlier days of \LaTeX{})
or to generate a smaller document which is easier to navigate.
Another application of |\includeonly| is to generate
documents consisting of selected parts of the complete document.

However, there are a few drawbacks of the plain |\include| mechanism:
\begin{itemize}
\item
The child files cannot be compiled on their own,
they can only be compiled via the main file.
A naive editing environment
(such as a text editor with an option
to have the current file processed by \LaTeX)
may require one to switch to the main file before compiling;
attempting to compile the child file produces errors.
\item
The main file must be modified (each time)
to adjust the |\includeonly| command
to the present needs. This easily leaves the main file in a messy state.
\item
The generated document will always carry the filename
of the main document. This is inconvenient if
several child files are to be compiled and
to be kept for distribution.
\end{itemize}

The present package provides a simple interface
to make child files individually compilable by \LaTeX{}.
Compiling a child file then has the same effect as compiling
the main file with an |\includeonly| command
to select the appropriate child.
Moreover the generated document will carry the name of the child
rather than the main file.
This resolves all three above issues.

This feature is meant to make the editing of books,
thesis documents and lecture notes somewhat more convenient.
However, the package can also be used efficiently for
composing a series of documents (such as exercise sheets)
which are typically distributed individually.
It then assists the author in generating the individual documents
(potentially in different versions)
as well as a document containing the collected series.
Another application is in developing style files
or other kinds of included material
where compilation of the style file could redirect
to a sample or test file.

%%%%%%%%%%%%%%%%%%%%%%%%%%%%%%%%%%%%%%%%%%%%%%%%%%%%%%%%%%%%%%%%%%%%%%%%%%%%%%%%
%%%%%%%%%%%%%%%%%%%%%%%%%%%%%%%%%%%%%%%%%%%%%%%%%%%%%%%%%%%%%%%%%%%%%%%%%%%%%%%%
\section{Usage}

First of all, the package \textsf{childdoc} is \emph{not} a standard
\LaTeXe{} |.sty| style file! Therefore it needs to be invoked in
a non-standard way.

%%%%%%%%%%%%%%%%%%%%%%%%%%%%%%%%%%%%%%%%%%%%%%%%%%%%%%%%%%%%%%%%%%%%%%%%%%%%%%%%
\subsection{Included Files}
\label{sec:include}

%%%%%%%%%%%%%%%%%%%%%%%%%%%%%%%%%%%%%%%%
\DescribeMacro{\childdocmain}
To use the package, add the commands
\begin{center}
\begin{tabular}{l}
|\input{childdoc.def}|\\
|\childdocmain{}|\\
\end{tabular}
\end{center}
at the very top of the main \LaTeX{} file,
in particular \emph{before} the |\documentclass| statement!
The argument of |\childdocmain| should be left empty
(but it must be present).

%%%%%%%%%%%%%%%%%%%%%%%%%%%%%%%%%%%%%%%%
\DescribeMacro{\childdocof}
Furthermore, add the commands
\begin{center}
\begin{tabular}{l}
|\input{childdoc.def}|\\
|\childdocof{|\textit{main}|}|\\
\end{tabular}
\end{center}
at the top of every child file \textit{child}
which is included by |\include{|\textit{child}|}|
from within the main file
(or at least for those files to be compiled individually).
The argument \textit{main} must be the filename of the main file.

There are a couple of
considerations in setting up the main and child documents:

%%%%%%%%%%%%%%%%%%%%%%%%%%%%%%%%%%%%%%%%
\paragraph{Restrictions.}

Please note the following restrictions:
\begin{itemize}
\item
|\childdocmain| must be called with one argument \textit{main}
to ensure compatibility with earlier version of the package.
It must either be empty (|\childdocmain{}|)
or precisely match the filename of the main file in which it is specified.
See \secref{sec:detection} for further information.
\item
The filename \textit{main} must be specified without the |.tex| extension.
\item
The filename \textit{main} is case sensitive
(even in case-insensitive file systems)
due to internal string comparison.
\item
The argument \textit{main} should be fully expanded, it cannot be a macro.
\item
Subdirectories and special characters should be avoided in filenames.
\item
The command |\childdocmain{|\textit{main}|}| must be followed by a whitespace.
It should not be followed immediately by another command
or by a comment mark `|%|'.
This is because the \TeX{} parser reads the token immediately following
the argument of |\childdocmain| and puts it
at the beginning of every child section;
however, a white\-space is ignored.
\end{itemize}

%%%%%%%%%%%%%%%%%%%%%%%%%%%%%%%%%%%%%%%%
\paragraph{Content of Main File.}

It is advisable to place all content in the child files included by |\include|.
Any output contained in the main file will appear in all child documents
unless suppressed manually;
it cannot be suppressed automatically by the |\includeonly| directive
and thus should normally be avoided.
A method to include some content in the main file
by means of conditional processing is described in \secref{sec:conditional}.

%%%%%%%%%%%%%%%%%%%%%%%%%%%%%%%%%%%%%%%%
\paragraph{Page Numbering.}

When only a part of the document is compiled,
the appropriate numbering of pages
(as well as other status parameters)
is determined from the |.aux| files.
The latter contain information from previous passes.
However this information needs to propagate through
all intermediate child documents.
Therefore the page numbering in child documents may well
be inconsistent until the complete document is compiled at least once.

A useful (if unconventional) way to always ensure a consistent
page numbering is to restart the numbering in each child document
and denote the pages by `\textit{child}|.|\textit{page}'
where \textit{child} represents the chapter/section number of the child file.
This can be achieved by the command
|\numberwithin{page}{|\textit{child}|}|
of the \textsf{amsmath} package
where \textit{child} can be |chapter| or |section|
depending on the chosen structuring.
Alternatively, one can modify the macro |\thepage| appropriately
and reset the counter |page| at the start of each child file.

%%%%%%%%%%%%%%%%%%%%%%%%%%%%%%%%%%%%%%%%%%%%%%%%%%%%%%%%%%%%%%%%%%%%%%%%%%%%%%%%
\subsection{Conditional Processing}
\label{sec:conditional}

The package provides a mechanism to compile different versions
of a document. To customise the versions further some conditional processing
can come in handy to distinguish which version is being compiled.
The package provides two macros to describe the compilation context:

%%%%%%%%%%%%%%%%%%%%%%%%%%%%%%%%%%%%%%%%
\DescribeMacro{\ifchilddoc}
The conditional |\ifchilddoc| distinguishes between the compilation of
child documents and the main document:
%
\begin{center}
|\ifchilddoc |\textit{child-code}| |[|\||else |\textit{main-code}]| \||fi|
\end{center}

%%%%%%%%%%%%%%%%%%%%%%%%%%%%%%%%%%%%%%%%
\DescribeMacro{\childdocname}
\DescribeMacro{\childdocjob}
The macro |\childdocname| contains the filename (without extension)
of the main or child file being processed.
Note that |\childdocjob| will always contain the name of the main file.

%%%%%%%%%%%%%%%%%%%%%%%%%%%%%%%%%%%%%%%%
\paragraph{Title Page.}

Conditional processing can be used to include a title or banner page
in the main document when proper precautions are taken.
Importantly, the code in the main file should ensure that the page counter
(as well as other status parameters which are stored in the |.aux| files)
takes the same value after the conditional processing.
Otherwise the page numbers may take divergent values
depending on which part is compiled.

For example, a title page could be declared by:
%
\begin{center}
\begin{tabular}{l}
|\ifchilddoc\||else|\\
|\addtocounter{page}{-1}|\\
\textit{code for title page}\\
|\newpage|\\
|\||fi|
\end{tabular}
\end{center}
%
A banner page for the child documents can be generated by:
%
\begin{center}
\begin{tabular}{l}
|\ifchilddoc|\\
|\addtocounter{page}{-1}|\\
\textit{code for banner page}\\
|\newpage|\\
|\||fi|
\end{tabular}
\end{center}
%
Here one could write a message such as:
\begin{center}
|This is the part \childdocname{} of \childdocjob{}.|
\end{center}

%%%%%%%%%%%%%%%%%%%%%%%%%%%%%%%%%%%%%%%%%%%%%%%%%%%%%%%%%%%%%%%%%%%%%%%%%%%%%%%%
\subsection{Flags}
\label{sec:flags}

The package makes it easy to generate different versions
of the main or child documents.
To this end compilation flags can be defined
and assigned different default values.
They will be particularly useful in conjunction
with the forwarding mechanism described in \secref{sec:forward}.

For example, it may be useful to have a flag |\version|
which can be set to |draft| or |final|.
The document source will contain some conditional code
depending on the value of |\version|.
Suppose further, the flag should default to |final| for the main file
and to |draft| for child files
which is a natural assignment for editing the document.
This is achieved by placing the following code
in the preamble of the main document
(below the |\childdocmain| directive):
%
\begin{center}
\begin{tabular}{l}
|\ifchilddoc|\\
|\providecommand{\version}{draft}|\\
|\||else|\\
|\providecommand{\version}{final}|\\
|\||fi|
\end{tabular}
\end{center}
%
The definition by |\providecommand| makes sure
that previous definitions are not overwritten.
Further statements |\providecommand{\version}{...}|
can thus be added before the above code to override it.

For the main file, one might add a line
(between |\childdocmain| and the above block)
%
\begin{center}
|%\ifchilddoc\||else\providecommand{\version}{draft}\||fi|
\end{center}
%
which can be uncommented to produce a draft version.
Likewise one can add a line to the very top of a child file
(above the |\childdocof{|\textit{main}|}| directive)
%
\begin{center}
|%\providecommand{\version}{final}|
\end{center}
%
which can be uncommented to produce the final version of this child document.

%%%%%%%%%%%%%%%%%%%%%%%%%%%%%%%%%%%%%%%%%%%%%%%%%%%%%%%%%%%%%%%%%%%%%%%%%%%%%%%%
\subsection{Forwarding}
\label{sec:forward}

Different versions of the main or child documents
using compilation flags as described in \secref{sec:flags}
can be (permanently) stored in different files
for convenient compilation, viewing and distribution.
To this end, the package defines a command
to pass on compilation to a different file:

%%%%%%%%%%%%%%%%%%%%%%%%%%%%%%%%%%%%%%%%
\DescribeMacro{\childdocforward}
The command |\childdocforward| redirects processing to
another source file:
%
\begin{center}
\begin{tabular}{l}
|\input{childdoc.def}|\\
|\childdocforward[|\textit{main}|]{|\textit{dest}|}|\\
\end{tabular}
\end{center}
%
The argument \textit{dest} is the destination file
(without extension).
It should be the main file or one of the child files.
Note that further \textsf{childdoc} directives
such as |\childdocof| and |\childdocforward|
in the indicated file will be processed in this form.
The optional argument \textit{main}
passes on directly to the main file \textit{main}
while pretending to compile the child \textit{dest}.
This form behaves as if \textit{dest}
issues |\childdocof{|\textit{main}|}| right away,
and no further \textsf{childdoc} directives will be processed.

%%%%%%%%%%%%%%%%%%%%%%%%%%%%%%%%%%%%%%%%
\DescribeMacro{\...prefix}
In the alternative form |\childdocforwardprefix|,
%
\begin{center}
\begin{tabular}{l}
|\input{childdoc.def}|\\
|\childdocforwardprefix[|\textit{main}|]{|\textit{prefix}|}{|\textit{dest}|}|
\end{tabular}
\end{center}
%
the destination file is determined by a pattern
depending on the current file:
To make this work, the current file must be called
`{\textit{prefix}\hspace{0.2em}\textit{suffix}}'
with \textit{prefix} matching precisely the argument.
Processing is then passed on to the file
`{\textit{dest}\hspace{0.2em}\textit{suffix}}'.
Surely, the same effect is achieved by
directly specifying the
argument `{\textit{dest}\hspace{0.2em}\textit{suffix}}'
in the first form.
However, that requires to set up a different file
for each child. With the alternative form of the command
all these files can have exactly the same content
which simplifies setting them up and maintaining them.

For example, the following file |draft.tex|
with a compilation flag |\version| as described in \secref{sec:flags}
compiles the main document as a draft:
%
\begin{center}
\begin{tabular}{l}
|\def\version{draft}|\\
|\input{childdoc.def}|\\
|\childdocforward{|\textit{main}|}|
\end{tabular}
\end{center}
%
Likewise, the following files |final|\textit{nn}|.tex|
compile the final version of the child document
|child|\textit{nn}|.tex|:
%
\begin{center}
\begin{tabular}{l}
|\def\version{final}|\\
|\input{childdoc.def}|\\
|\childdocforwardprefix{final}{child}|
\end{tabular}
\end{center}
%

Note that when several versions of a main file and/or of each child file
are to be generated, it may be convenient to set up a |Makefile| or
shell script to automatise the process.

%%%%%%%%%%%%%%%%%%%%%%%%%%%%%%%%%%%%%%%%%%%%%%%%%%%%%%%%%%%%%%%%%%%%%%%%%%%%%%%%
\subsection{Command Line Processing}
\label{sec:commandline}

The effect of redirection files can also be achieved by invoking
the \LaTeX{} compiler with a more elaborate command line.
Most conveniently this should be done as part
of a shell script or a |Makefile|.

When using \textsf{childdoc} in the main file, the following
command lines effectively perform a redirection
(note that depending on the shell being used,
backslashes may have to be doubled: `|\|' $\to$ `|\\|'):
%
\begin{center}
|... -jobname "|\textit{target}|" |\\|"|[\textit{flags}]%
|\input{childdoc.def}\childdocforward[|\textit{main}|]{|\textit{dest}|}"|
\end{center}
%
Here \textit{target} is the name of the output file,
\textit{main} is the name of the main file
and \textit{dest} is the name of the main or child file to be processed
(all filenames without extensions).
The optional argument \textit{main} can be omitted
if \textit{main} matches \textit{dest}.
Optionally, compilation \textit{flags} can be defined via |\def| commands.
This command line makes the \TeX{} engine believe
it is compiling the file \textit{target}
whose content is specified as the latter parameter.
The provided code then forwards the processing to
\textit{main} or \textit{dest} as described in \secref{sec:forward}.

%%%%%%%%%%%%%%%%%%%%%%%%%%%%%%%%%%%%%%%%%%%%%%%%%%%%%%%%%%%%%%%%%%%%%%%%%%%%%%%%
\subsection{Include by Input}
\label{sec:input}

Including child documents by |\include| has some restrictions by design.
Most notably, the content of a child document always occupies
its own set of pages; pages cannot be shared between child documents.
Usually, this behaviour makes perfect sense
because each child document contain an essential part of the document.
However, in some situations it may be desirable to compose
a document from a collection of parts
without having mandatory page breaks between then.
For this case, the package
provides a mechanism to include parts
by |\input| which can also be processed individually.
However, by construction this mechanism
requires manual handling of the content to be output.

%%%%%%%%%%%%%%%%%%%%%%%%%%%%%%%%%%%%%%%%
\DescribeMacro{\ifchilddocmanual}
The main file should be prepared as usual, see \secref{sec:include}.
However, the document body must make a distinction
between processing of an individual part and of the main document, e.g.:
%
\begin{center}
\begin{tabular}{l}
|\ifchilddocmanual|\\
|\input{\childdocname}|\\
|\||else|\\
\textit{document body with }|\input{|\textit{part}|}|\\
|\||fi|
\end{tabular}
\end{center}
%
The conditional |\ifchilddocmanual| is true whenever
a part to be included by |\input| is being compiled,
and the name of the part is stored in |\childdocname|.

%%%%%%%%%%%%%%%%%%%%%%%%%%%%%%%%%%%%%%%%
\DescribeMacro{\childdocby}
Each part to be included by |\input| should start with:
%
\begin{center}
\begin{tabular}{l}
|\input{childdoc.def}|\\
|\childdocby{|\textit{main}|}|\\
\end{tabular}
\end{center}
%
The directive |\childdocby| is similar to |\childdocof|
described in \secref{sec:include},
but the subsequent selection of content must be done manually.
To that end, both |\ifchilddoc| and |\ifchilddocmanual|
will be true upon processing of a part,
and the name of the part is stored in |\childdocname|.
Note that |\jobname| will be set to the filename of the current part
so that each part receives an individual |.aux| file
that does not interfere with the |.aux| file(s) of the main document.
This behaviour can be altered by the alternative form
|\childdocby[*]{|\textit{main}|}| (with a non-empty optional argument)
which uses the |.aux| file of the main document
by setting |\jobname| to \textit{main}.

%%%%%%%%%%%%%%%%%%%%%%%%%%%%%%%%%%%%%%%%%%%%%%%%%%%%%%%%%%%%%%%%%%%%%%%%%%%%%%%%
\subsection{Driver Development}
\label{sec:driver}

The \textsf{childdoc} mechanism can also be use for the development
of definition files such as \LaTeX{} styles or classes.
This case differs from the above setup with multiple parts
included by |\include| in that no |\includeonly| should be invoked.
This can be achieved by starting the include file
(before |\ProvidesPackage|) with:
%
\begin{center}
\begin{tabular}{l}
|\input{childdoc.def}|\\
|\childdocforward{|\textit{main}|}|\\
\end{tabular}
\end{center}
%
or alternatively with:
%
\begin{center}
\begin{tabular}{l}
|\input{childdoc.def}|\\
|\childdocby{|\textit{main}|}|\\
\end{tabular}
\end{center}
%
Both forms have slightly different effects as described above.
The main file is prepared as usual, see \secref{sec:include}.

%%%%%%%%%%%%%%%%%%%%%%%%%%%%%%%%%%%%%%%%%%%%%%%%%%%%%%%%%%%%%%%%%%%%%%%%%%%%%%%%
\subsection{Legacy Detection}
\label{sec:detection}

The directive |\childdocmain| in the main file can detect
whether the complete document or merely a child is to be compiled
even without using the directive |\childdocof|.
This method is deprecated because it is less robust
and there is no compelling reason to use it;
it is merely provided for backward compatibility
and it may be removed in future versions.

If the detection mechanism is to be used,
it is mandatory to correctly specify
the filename of the main file as the argument of |\childdocmain|:
%
\begin{center}
\begin{tabular}{l}
|\input{childdoc.def}|\\
|\childdocmain{|\textit{main}|}|\\
\end{tabular}
\end{center}
%
If |\jobname| does not match the argument \textit{main} of |\childdocmain|,
it is assumed that |\jobname| points to the child file to be compiled.
When using |\childdocmain| with the main file specified as argument,
it suffices to start a child file
with just |\input{|\textit{main}|}|
without loading of the package and using |\childdocof|.
If instead all processing is done
with the appropriate \textsf{childdoc} directives,
the argument of \textit{main} of |\childdocmain| can be empty.

An alternative version of the command line processing described
in \secref{sec:commandline} using the detection mechanism reads:
%
\begin{center}
|... -jobname "|\textit{target}|" "|[\textit{flags}]%
[|\def\jobname{|\textit{dest}|}|]|\input{|\textit{main}|}"|
\end{center}

%%%%%%%%%%%%%%%%%%%%%%%%%%%%%%%%%%%%%%%%%%%%%%%%%%%%%%%%%%%%%%%%%%%%%%%%%%%%%%%%
\subsection{Manual Code}
\label{sec:manual}

In case one cannot be certain whether the definitions file |childdoc.def|
is installed on the target \TeX{} distribution
and one prefers not to ship it,
it is conceivable to paste a few relevant commands into the sources.

To that end, drop all statements |\input{childdoc.def}|
and perform the replacements as outlined below.
Instead of |\childdocmain{|\textit{main}|}| add the following code
to the top of the main file:
%
\begin{center}
\begin{tabular}{l}
|\||ifdefined\childdocname\endinput\||fi\newif\ifchilddoc|\\
|\edef\childdocname{\scantokens\expandafter{\jobname\noexpand}}|\\
|\def\childdocmain{|\textit{main}|}\||ifx\childdocmain\childdocname\||else|\\
|\childdoctrue\includeonly{\childdocname}\let\jobname\childdocmain\||fi|\\
\end{tabular}
\end{center}
%
Instead of |\childdocof{|\textit{main}|}| just include the main file
at the top of each child file:
%
\begin{center}
|\input{|\textit{main}|}|
\end{center}
%
A simple redirection |\childdocforward{|\textit{dest}|}| is achieved by:
%
\begin{center}
|\def\jobname{|\textit{dest}|}\input{\jobname}|
\end{center}
%
The redirection with prefix
|\childdocforwardprefix[|\textit{prefix}|]{|\textit{dest}|}|
is accomplished by:
%
\begin{center}
\begin{tabular}{l}
|{\edef\jobname{\scantokens\expandafter{\jobname\noexpand}}|\\
|\def\redirectjob |\textit{prefix}|#1~~~{\gdef\jobname{|\textit{dest}|#1}}|\\
|\expandafter\redirectjob\jobname~~~}\input{\jobname}|
\end{tabular}
\end{center}

In an alternative approach,
child documents can be compiled by a specific command line
without additional code or specific definitions:
%
\begin{center}
|... -jobname "|\textit{target}|" "|[\textit{flags}]%
|\includeonly{|\textit{dest}|}\input{|\textit{main}|}"|
\end{center}
%

%%%%%%%%%%%%%%%%%%%%%%%%%%%%%%%%%%%%%%%%%%%%%%%%%%%%%%%%%%%%%%%%%%%%%%%%%%%%%%%%
%%%%%%%%%%%%%%%%%%%%%%%%%%%%%%%%%%%%%%%%%%%%%%%%%%%%%%%%%%%%%%%%%%%%%%%%%%%%%%%%
\section{Information}

%%%%%%%%%%%%%%%%%%%%%%%%%%%%%%%%%%%%%%%%%%%%%%%%%%%%%%%%%%%%%%%%%%%%%%%%%%%%%%%%
\subsection{Copyright}

Copyright \copyright{} 2017--2018 Niklas Beisert

This work may be distributed and/or modified under the
conditions of the \LaTeX{} Project Public License, either version 1.3
of this license or (at your option) any later version.
The latest version of this license is in
  \url{http://www.latex-project.org/lppl.txt}
and version 1.3 or later is part of all distributions of \LaTeX{}
version 2005/12/01 or later.

This work has the LPPL maintenance status `maintained'.

The Current Maintainer of this work is Niklas Beisert.

This work consists of the files |README.txt|, |childdoc.ins| and |childdoc.dtx|
as well as the derived files |childdoc.def|, |cdocsamp.tex|
with |cdocsch1.tex|, |cdocsch2.tex|, |cdocspt3.tex|, |cdocspt4.tex|,
|cdocsdrf.tex|, |cdocsfn1.tex|, |cdocsfn2.tex|
as well as |childdoc.pdf|.

%%%%%%%%%%%%%%%%%%%%%%%%%%%%%%%%%%%%%%%%%%%%%%%%%%%%%%%%%%%%%%%%%%%%%%%%%%%%%%%%
\subsection{Files and Installation}

The package consists of the files:
%
\begin{center}
\begin{tabular}{ll}
    |README.txt|   & readme file \\
    |childdoc.ins| & installation file \\
    |childdoc.dtx| & source file \\
    |childdoc.def| & definition file \\
    |cdocsamp.tex| & sample main file \\
    |cdocsch1.tex| & sample include file \\
    |cdocsch2.tex| & sample include file \\
    |cdocspt3.tex| & sample part file \\
    |cdocspt4.tex| & sample part file \\
    |cdocsdrf.tex| & sample redirection file \\
    |cdocsfn1.tex| & sample redirection file \\
    |cdocsfn2.tex| & sample redirection file \\
    |childdoc.pdf| & manual
\end{tabular}
\end{center}
%
The distribution consists of the files
|README.txt|, |childdoc.ins| and |childdoc.dtx|.
%
\begin{itemize}
\item
Run (pdf)\LaTeX{} on |childdoc.dtx|
to compile the manual |childdoc.pdf| (this file).
\item
Run \LaTeX{} on |childdoc.ins| to create the definitions file |childdoc.def|
and the sample |cdocsamp.tex| with include files
|cdocsch1.tex|, |cdocsch2.tex|, |cdocspt3.tex|, |cdocspt4.tex|,
|cdocsdrf.tex|, |cdocsfn1.tex|, |cdocsfn2.tex|.
Then copy the file |childdoc.def| to an appropriate directory of your \LaTeX{}
distribution, e.g.\ \textit{texmf-root}|/tex/latex/childdoc|.
\end{itemize}

%%%%%%%%%%%%%%%%%%%%%%%%%%%%%%%%%%%%%%%%%%%%%%%%%%%%%%%%%%%%%%%%%%%%%%%%%%%%%%%%
\subsection{Related CTAN Packages}

There are several other packages which offer a similar functionality:
%
\begin{itemize}
\item
The packages
\href{http://ctan.org/pkg/docmute}{\textsf{docmute}},
\href{http://ctan.org/pkg/includex}{\textsf{includex}} and
\href{http://ctan.org/pkg/standalone}{\textsf{standalone}}
provide commands to include only the document body of
a child file thus allowing both files to be compiled individually.
\item
The packages \href{http://ctan.org/pkg/subdocs}{\textsf{subdocs}}
and \href{http://ctan.org/pkg/subfiles}{\textsf{subfiles}}
provide structures in which the main and child documents can be
encapsulated and allowing them to be compiled individually.
The inclusion mechanism is different from the conventional |\include|.
\item
The package \href{http://ctan.org/pkg/combine}{\textsf{combine}}
is an elaborate solution to combine several documents into one.
\end{itemize}
%
See also the CTAN topic \href{http://ctan.org/topic/subdocs}{\textsf{subdocs}}
for further related packages.
The present package differs from the above solutions in that
a document structure constructed with the conventional |\include| mechanism
just needs two extra commands at the top of every file
such that all constituent files can be compiled individually.

%%%%%%%%%%%%%%%%%%%%%%%%%%%%%%%%%%%%%%%%%%%%%%%%%%%%%%%%%%%%%%%%%%%%%%%%%%%%%%%%
%\subsection{Feature Suggestions}
%
%The following is a list of features which may be useful for future
%versions of this package:
%%
%\begin{itemize}
%\item
%\ldots
%\end{itemize}

%%%%%%%%%%%%%%%%%%%%%%%%%%%%%%%%%%%%%%%%%%%%%%%%%%%%%%%%%%%%%%%%%%%%%%%%%%%%%%%%
\subsection{Revision History}

%%%%%%%%%%%%%%%%%%%%%%%%%%%%%%%%%%%%%%%%
\paragraph{v2.0:} 2018/12/30

\begin{itemize}
\item
immediate forward processing
\item
added |\childdocby| mechanism
\item
manual restructured
\end{itemize}

%%%%%%%%%%%%%%%%%%%%%%%%%%%%%%%%%%%%%%%%
\paragraph{v1.6:} 2018/01/17

\begin{itemize}
\item
application for development of include files
\item
corrections to manual
\end{itemize}

%%%%%%%%%%%%%%%%%%%%%%%%%%%%%%%%%%%%%%%%
\paragraph{v1.5:} 2017/05/21

\begin{itemize}
\item
more complete structuring introduced
\item
|\childdocof| introduced
\item
|\childdoc| renamed to |\childdocmain|
\item
|\childredirect| renamed to |\childdocforward| and |\childdocforwardprefix|
and functionality expanded
\end{itemize}

%%%%%%%%%%%%%%%%%%%%%%%%%%%%%%%%%%%%%%%%
\paragraph{v1.0:} 2017/04/27

\begin{itemize}
\item
manual and install package
\item
first version published on CTAN
\end{itemize}

%%%%%%%%%%%%%%%%%%%%%%%%%%%%%%%%%%%%%%%%
\paragraph{v0.6:} 2017/04/26

\begin{itemize}
\item
redirection mechanism added
\end{itemize}

%%%%%%%%%%%%%%%%%%%%%%%%%%%%%%%%%%%%%%%%
\paragraph{v0.5:} 2017/04/26

\begin{itemize}
\item
functionality in definition file
\end{itemize}


%%%%%%%%%%%%%%%%%%%%%%%%%%%%%%%%%%%%%%%%%%%%%%%%%%%%%%%%%%%%%%%%%%%%%%%%%%%%%%%%
%%%%%%%%%%%%%%%%%%%%%%%%%%%%%%%%%%%%%%%%%%%%%%%%%%%%%%%%%%%%%%%%%%%%%%%%%%%%%%%%
%%%%%%%%%%%%%%%%%%%%%%%%%%%%%%%%%%%%%%%%%%%%%%%%%%%%%%%%%%%%%%%%%%%%%%%%%%%%%%%%
\appendix

\settowidth\MacroIndent{\rmfamily\scriptsize 000\ }

 \DocInput{childdoc.dtx}

\end{document}
%</driver>
% \fi
%
% %%%%%%%%%%%%%%%%%%%%%%%%%%%%%%%%%%%%%%%%%%%%%%%%%%%%%%%%%%%%%%%%%%%%%%%%%%%%%%
% %%%%%%%%%%%%%%%%%%%%%%%%%%%%%%%%%%%%%%%%%%%%%%%%%%%%%%%%%%%%%%%%%%%%%%%%%%%%%%
% \section{Sample}
%\iffalse
%<*samplemain>
%\fi
%
% The following presents a sample document
% with two chapters, two parts, a title page,
% a compile flag as well as three forwarding files to set the flag.
% It consists of eight |.tex| files:
% \begin{center}
% \begin{tabular}{ll}
% |cdocsamp.tex|&main file\\
% |cdocsch1.tex|&include file for chapter 1\\
% |cdocsch2.tex|&include file for chapter 2\\
% |cdocspt3.tex|&include file for part 3\\
% |cdocspt4.tex|&include file for part 4\\
% |cdocsdrf.tex|&forwarding file for main file in draft mode\\
% |cdocsfi1.tex|&forwarding file for final version of chapter 1\\
% |cdocsfi2.tex|&forwarding file for final version of chapter 2\\
% \end{tabular}
% \end{center}
% Each of the eight files can be compiled directly by the \LaTeX{} compiler.
%
% %%%%%%%%%%%%%%%%%%%%%%%%%%%%%%%%%%%%%%
% \paragraph{Main File.}
%
% The main file is called |cdocsamp.tex|.
%
% Load the \textsf{childdoc} definitions and
% declare the filename for the main document:
%    \begin{macrocode}
\input{childdoc.def}
\childdocmain{}
%    \end{macrocode}

% Optional override for |\version| flag:
%    \begin{macrocode}
%%\ifchilddoc\else\providecommand{\version}{draft}\fi
%    \end{macrocode}

% Define the default values for the |\version| flag
% (|final| for the main file and |draft| for childs):
%    \begin{macrocode}
\ifchilddoc
\providecommand{\version}{draft}
\else
\providecommand{\version}{final}
\fi
%    \end{macrocode}

% Load the standard document class:
%    \begin{macrocode}
\documentclass[12pt]{article}
%    \end{macrocode}

% Start the document body:
%    \begin{macrocode}
\begin{document}
%    \end{macrocode}

% Declare a title page.
% Print title, part of document being processed and version flag:
%    \begin{macrocode}
\addtocounter{page}{-1}
\begin{center}
{\LARGE\bfseries{}childdoc example\par}
\vspace{1cm}
\ifchilddoc
\ifchilddocmanual part\else chapter\fi:
`\childdocname' of `\childdocjob'\par
\else
main document: `\childdocjob'\par
\fi
version: \version\par
\end{center}
\newpage
%    \end{macrocode}

% Manually include selected file,
% otherwise process as usual:
%    \begin{macrocode}
\ifchilddocmanual
\section*{part `\childdocname'}
\input{\childdocname}
\else
%    \end{macrocode}

% Include the two chapters:
%    \begin{macrocode}
\include{cdocsch1}
\include{cdocsch2}
%    \end{macrocode}

% Include the two parts unless only chapters should be displayed:
%    \begin{macrocode}
\ifchilddoc\else
\section{part three}
\input{cdocspt3}
\section{part four}
\input{cdocspt4}
\fi
%    \end{macrocode}

% Process as usual until here:
%    \begin{macrocode}
\fi
%    \end{macrocode}

% End of document body:
%    \begin{macrocode}
\end{document}
%    \end{macrocode}
%\iffalse
%</samplemain>
%\fi
%
% %%%%%%%%%%%%%%%%%%%%%%%%%%%%%%%%%%%%%%
% \paragraph{Chapter Include Files.}
%
% The include files are called |cdocsch1.tex| and |cdocsch2.tex|.
%
%\iffalse
%<*samplechap1|samplechap2>
%\fi

% Optional override for |\version| flag:
%    \begin{macrocode}
%%\providecommand{\version}{final}
%    \end{macrocode}

% Include the main document:
%    \begin{macrocode}
\input{childdoc.def}
\childdocof{cdocsamp}
%    \end{macrocode}

%\iffalse
%</samplechap1|samplechap2>
%\fi
%
%\iffalse
%<*samplechap1>
%\fi
% Some text for chapter 1:
%    \begin{macrocode}
\section{one}
some text in chapter one
%    \end{macrocode}

%\iffalse
%</samplechap1>
%\fi
% Some text for chapter 2:
%\iffalse
%<*samplechap2>
%\fi
%    \begin{macrocode}
\section{two}
more text in chapter two
%    \end{macrocode}

%\iffalse
%</samplechap2>
%\fi
%
% %%%%%%%%%%%%%%%%%%%%%%%%%%%%%%%%%%%%%%
% \paragraph{Part Include Files.}
%
% The include files are called |cdocspt3.tex| and |cdocspt4.tex|.
%
%\iffalse
%<*samplepart3|samplepart4>
%\fi

% Optional override for |\version| flag:
%    \begin{macrocode}
%%\providecommand{\version}{final}
%    \end{macrocode}

% Include the main document:
%    \begin{macrocode}
\input{childdoc.def}
\childdocby{cdocsamp}
%    \end{macrocode}

%\iffalse
%</samplepart3|samplepart4>
%\fi
%
%\iffalse
%<*samplepart3>
%\fi
% Some text for part 3:
%    \begin{macrocode}
some text in part three
%    \end{macrocode}

%\iffalse
%</samplepart3>
%\fi
% Some text for part 4:
%\iffalse
%<*samplepart4>
%\fi
%    \begin{macrocode}
more text in part four
%    \end{macrocode}

%\iffalse
%</samplepart4>
%\fi
%
% %%%%%%%%%%%%%%%%%%%%%%%%%%%%%%%%%%%%%%
% \paragraph{Forwarding for a Complete Draft.}
%
% The following forwarding file |cdocsdrf.tex|
% compiles the main document in draft mode:
%\iffalse
%<*sampledraft>
%\fi
%    \begin{macrocode}
\def\version{draft}
\input{childdoc.def}
\childdocforward{cdocsamp}
%    \end{macrocode}

%\iffalse
%</sampledraft>
%\fi
%
% %%%%%%%%%%%%%%%%%%%%%%%%%%%%%%%%%%%%%%
% \paragraph{Forwarding for Final Version of the Chapters.}
%
% The following forwarding files |cdocsfn1.tex| and |cdocsfn2.tex|
% (with identical content)
% compile the final versions of the child documents
% |cdocsch1.tex| and |cdocsch2.tex|, respectively:
%\iffalse
%<*samplefinal>
%\fi
%    \begin{macrocode}
\def\version{final}
\input{childdoc.def}
\childdocforwardprefix[cdocsamp]{cdocsfn}{cdocsch}
%    \end{macrocode}

%\iffalse
%</samplefinal>
%\fi
%
% %%%%%%%%%%%%%%%%%%%%%%%%%%%%%%%%%%%%%%
% \paragraph{Command Line Processing.}
%
% The following three command lines generate the output files
% |cdocscld|, |cdocscl1| and |cdocscl2|
% which should be identical to
% |cdocsdrf|, |cdocsch1| and |cdocsfn2|, respectively:
% \begin{center}
% \begin{tabular}{l}
% |latex -jobname cdocscld \|\\
% |  "\def\version{draft}\input{childdoc.def}\childdocforward{cdocsamp}"|\\
% |latex -jobname cdocscl1 \|\\
% |  "\input{childdoc.def}\childdocforward[cdocsamp]{cdocsch1}"|\\
% |latex -jobname cdocscl2 \|\\
% |  "\def\version{final}\input{childdoc.def}\childdocforward{cdocsch2}"|
% \end{tabular}
% \end{center}
% Note that the trailing backslash on each first line
% merely continues the input to the second line
% (for convenient cut ant paste).
% Furthermore, the command |latex| can be replaced by any
% of its alternative versions such as |pdflatex|.
%
% %%%%%%%%%%%%%%%%%%%%%%%%%%%%%%%%%%%%%%%%%%%%%%%%%%%%%%%%%%%%%%%%%%%%%%%%%%%%%%
% %%%%%%%%%%%%%%%%%%%%%%%%%%%%%%%%%%%%%%%%%%%%%%%%%%%%%%%%%%%%%%%%%%%%%%%%%%%%%%
% \section{Implementation}
%\iffalse
%<*package>
%\fi
%
% This section describes the definitions file |childdoc.def|.

% The definitions cannot be loaded using |\usepackage| or |\RequirePackage|
% which has a mechanism to prevent loading a style file more than once.
% When loading the definitions by means of |\input|
% multiple instances have to be prevented manually:
%\iffalse
%This code needs to be before the `\ProvidesFile' directive
%which is defined at the beginning of this file.
%Therefore it is also placed there and commented out here.
%</package>
%<*discard>
%\fi
%    \begin{macrocode}
\ifdefined\childdocmain\endinput\fi
%    \end{macrocode}
%\iffalse
%</discard>
%<*package>
%\fi
%
% \macro{\ifchilddoc}
% \macro{\ifchilddocmanual}
% The conditional |\ifchilddoc| tells whether a
% child (true) or main (false) document is being compiled.
% The conditional |\ifchilddocmanual| tells whether
% the |\includeonly| mechanism is used (false) or
% the selection of child files must be performed manually (true).
% The definitions initialise to false:
%    \begin{macrocode}
\newif\ifchilddoc
\newif\ifchilddocmanual
%    \end{macrocode}

% \macro{\childdocname}
% \macro{\childdocjob}
% The macro |\childdocname| stores the name of the main document
% to be compiled. The macro |\childdocjob| stores the name of
% the document on which the \LaTeX{} compiler was originally invoked.
% The content of |\jobname| cannot be compared
% to filenames specified in the source due to different catcodes.
% The following code rescans |\jobname|, stores the result
% in |\childdocname| and saves a copy in |\childdocjob|:
%    \begin{macrocode}
\edef\childdocname{\scantokens\expandafter{\jobname\noexpand}}
\let\childdocjob\childdocname
%    \end{macrocode}

% \macro{\childdocdisable}
% The macro |\childdocdisable| prevents the main file
% from being processed more than once.
% At this stage, the main document command |\childdocmain|
% is assumed to be called once again where it should do nothing.
% Any subsequent call to it should prevent
% a secondary processing of the main document
% It overwrites the forwarding commands
% |\childdocof| and |\childdocforward|
% with empty macros to prevent further inclusions of the main document:
%    \begin{macrocode}
\newcommand{\childdocdisable}
{
  \renewcommand{\childdocmain}[1]{\renewcommand{\childdocmain}[1]{\endinput}}
  \renewcommand{\childdocof}[1]{}
  \renewcommand{\childdocby}[2][]{}
  \renewcommand{\childdocforward}[2][]{}
  \renewcommand{\childdocdisable}{}
}
%    \end{macrocode}

% \macro{\childdocmain}
% The macro |\childdocmain| is to be called at the top of the main file
% with nothing or the main filename (without extension) as argument.
% First, it breaks loops.
% If the argument is not empty and does not match |\childdocname|
% (which is set by the first inclusion of |childdoc.def|),
% |\ifchilddoc| is set to true, |\includeonly| is applied to the child file
% and |\jobname| is set to the main file
% (for proper handling of |.aux| files):
%    \begin{macrocode}
\newcommand{\childdocmain}[1]
{
  \childdocdisable\childdocmain{}
  \if?#1?\else
    \begingroup
      \def\childdoctmp{#1}
      \ifx\childdoctmp\childdocname
        \def\childdoctmp{}
      \else
        \def\childdoctmp
        {
          \childdoctrue
          \includeonly{\childdocname}
          \def\childdocjob{#1}
          \def\jobname{#1}
        }
      \fi
      \expandafter
    \endgroup
    \childdoctmp
  \fi
}
%    \end{macrocode}

% \macro{\childdocof}
% The command |\childdocof| redirects
% compilation to the main file |#1|.
%    \begin{macrocode}
\newcommand{\childdocof}[1]
{
  \childdocdisable
  \childdoctrue
  \includeonly{\childdocname}
  \def\jobname{#1}
  \def\childdocjob{#1}
  \input{#1}
}
%    \end{macrocode}

% \macro{\childdocby}
% The command |\childdocby| ....
%    \begin{macrocode}
\newcommand{\childdocby}[2][]
{
  \childdocdisable
  \childdoctrue
  \childdocmanualtrue
  \if?#1?\else
    \def\jobname{#2}
  \fi
  \def\childdocjob{#2}
  \input{#2}
  \endinput
}
%    \end{macrocode}

% \macro{\childdocforward}
% The command |\childdocforward| redirects
% compilation to the main file or
% (if the optional argument is given) a child file.
% Parameters are set as if the main file
% or a child file starting with |\childdocof| was compiled.
% Then compilation is handed over to the main file:
%    \begin{macrocode}
\newcommand{\childdocforward}[2][]
{
  \begingroup
    \if?#1?
      \def\childdoctmp
      {
        \def\childdocname{#2}
        \def\childdocjob{#2}
        \def\jobname{#2}
        \input{#2}
        \endinput
      }
    \else
      \def\childdoctmp
      {
        \childdocdisable
        \def\childdocname{#2}
        \childdoctrue
        \includeonly{#2}
        \def\childdocjob{#1}
        \def\jobname{#1}
        \input{#1}
        \endinput
      }
    \fi
    \expandafter
  \endgroup
  \childdoctmp
}
%    \end{macrocode}

% \macro{\childdocforwardprefix}
% The command |\childdocforwardprefix| redirects
% compilation to the main or a child file by means of a pattern.
% The prefix |#1| in the current filename is replaced by |#2|
% and the suffix of the current filename is kept
% (it is assumed that the filename does not contain the substring `|~~~|'
% which is used as a delimiter).
% Compilation is handed over to the new file by |\childdocforward|:
%    \begin{macrocode}
\newcommand{\childdocforwardprefix}[3][]
{
  \begingroup
    \def\childdocextract #2##1~~~{\def\childdoctmp{\childdocforward[#1]{#3##1}}}
    \expandafter\childdocextract\childdocname~~~
    \expandafter
  \endgroup
  \childdoctmp
}
%    \end{macrocode}

% \macro{\childdoc}
% The deprecated macro |\childdoc| is a legacy version of |\childdocmain|:
%    \begin{macrocode}
\newcommand{\childdoc}{\childdocmain}
%    \end{macrocode}

% \macro{\childdocredirect}
% The deprecated macro |\childdocredirect| is a legacy version
% of |\childdocforward| and |\childdocforwardprefix|:
%    \begin{macrocode}
\newcommand{\childdocredirect}[2][]
{
  \begingroup
    \if?#1?
      \def\childdoctmp{\childdocforward{#2}}
    \else
      \def\childdoctmp{\childdocforwardprefix{#1}{#2}}
    \fi
    \expandafter
  \endgroup
  \childdoctmp
}
%    \end{macrocode}

%\iffalse
%</package>
%\fi
%
\endinput

\childdocforwardprefix[cdocsamp]{cdocsfn}{cdocsch}
%    \end{macrocode}

%\iffalse
%</samplefinal>
%\fi
%
% %%%%%%%%%%%%%%%%%%%%%%%%%%%%%%%%%%%%%%
% \paragraph{Command Line Processing.}
%
% The following three command lines generate the output files
% |cdocscld|, |cdocscl1| and |cdocscl2|
% which should be identical to
% |cdocsdrf|, |cdocsch1| and |cdocsfn2|, respectively:
% \begin{center}
% \begin{tabular}{l}
% |latex -jobname cdocscld \|\\
% |  "\def\version{draft}% \iffalse
%
% childdoc.dtx Copyright (C) 2017-2018 Niklas Beisert
%
% This work may be distributed and/or modified under the
% conditions of the LaTeX Project Public License, either version 1.3
% of this license or (at your option) any later version.
% The latest version of this license is in
%   http://www.latex-project.org/lppl.txt
% and version 1.3 or later is part of all distributions of LaTeX
% version 2005/12/01 or later.
%
% This work has the LPPL maintenance status `maintained'.
%
% The Current Maintainer of this work is Niklas Beisert.
%
% This work consists of the files childdoc.dtx and childdoc.ins
% and the derived files childdoc.def and cdocsamp.tex with
% cdocsch1.tex, cdocsch2.tex, cdocsdrf.tex, cdocsfn1.tex, cdocsfn2.tex.
%
%<package>\ifdefined\childdocmain\endinput\fi
%<package>\ProvidesFile{childdoc.def}[2018/12/30 v2.0 child document driver]
%<samplemain>\ProvidesFile{cdocsamp.tex}[2018/12/30 v2.0 sample for childdoc]
%<*driver>
%\ProvidesFile{childdoc.drv}[2018/12/30 v2.0 childdoc reference manual file]
\PassOptionsToClass{10pt,a4paper}{article}
\documentclass{ltxdoc}

\usepackage[margin=35mm]{geometry}
\usepackage{hyperref}
\usepackage{hyperxmp}
\usepackage[usenames]{color}

\hypersetup{colorlinks=true}
\hypersetup{pdfstartview=FitH}
\hypersetup{pdfpagemode=UseNone}
\hypersetup{pdfsource={}}
\hypersetup{pdflang={en-UK}}
\hypersetup{pdfcopyright={Copyright 2017-2018 Niklas Beisert.
  This work may be distributed and/or modified under the
  conditions of the LaTeX Project Public License, either version 1.3
  of this license or (at your option) any later version.}}
\hypersetup{pdflicenseurl={http://www.latex-project.org/lppl.txt}}
\hypersetup{pdfcontactaddress={ETH Zurich, ITP, HIT K,
  Wolfgang-Pauli-Strasse 27}}
\hypersetup{pdfcontactpostcode={8093}}
\hypersetup{pdfcontactcity={Zurich}}
\hypersetup{pdfcontactcountry={Switzerland}}
\hypersetup{pdfcontactemail={nbeisert@itp.phys.ethz.ch}}
\hypersetup{pdfcontacturl={http://people.phys.ethz.ch/\xmptilde nbeisert/}}

\newcommand{\secref}[1]{\hyperref[#1]{section \ref*{#1}}}

\parskip1ex
\parindent0pt
\let\olditemize\itemize
\def\itemize{\olditemize\parskip0pt}

\begin{document}

\title{The \textsf{childdoc} Package}
\hypersetup{pdftitle={The childdoc Package}}
\author{Niklas Beisert\\[2ex]
  Institut f\"ur Theoretische Physik\\
  Eidgen\"ossische Technische Hochschule Z\"urich\\
  Wolfgang-Pauli-Strasse 27, 8093 Z\"urich, Switzerland\\[1ex]
  \href{mailto:nbeisert@itp.phys.ethz.ch}
  {\texttt{nbeisert@itp.phys.ethz.ch}}}
\hypersetup{pdfauthor={Niklas Beisert}}
\hypersetup{pdfsubject={Manual for the LaTeX2e Package childdoc}}
\date{30 December 2018, \textsf{v2.0}}
\maketitle

\begin{abstract}\noindent
\textsf{childdoc} is a \LaTeXe{} package
that enables the direct compilation
of document sections included by |\include|
to individual files.
\end{abstract}

\begingroup
\parskip0ex
\tableofcontents
\endgroup

%%%%%%%%%%%%%%%%%%%%%%%%%%%%%%%%%%%%%%%%%%%%%%%%%%%%%%%%%%%%%%%%%%%%%%%%%%%%%%%%
%%%%%%%%%%%%%%%%%%%%%%%%%%%%%%%%%%%%%%%%%%%%%%%%%%%%%%%%%%%%%%%%%%%%%%%%%%%%%%%%
\section{Introduction}

\LaTeX{} provides a mechanism to structure a large document (such as a book)
into a main file and several child files (containing the chapters)
using the |\include| command.
This mechanism is beneficial for documents
which span hundreds of pages in order to
make the source file(s) more manageable.
Moreover, compilation can be restricted to
selected child files by means of the |\includeonly| command.
The latter feature can be used to reduce the compilation time while editing
(this was significantly more useful in the earlier days of \LaTeX{})
or to generate a smaller document which is easier to navigate.
Another application of |\includeonly| is to generate
documents consisting of selected parts of the complete document.

However, there are a few drawbacks of the plain |\include| mechanism:
\begin{itemize}
\item
The child files cannot be compiled on their own,
they can only be compiled via the main file.
A naive editing environment
(such as a text editor with an option
to have the current file processed by \LaTeX)
may require one to switch to the main file before compiling;
attempting to compile the child file produces errors.
\item
The main file must be modified (each time)
to adjust the |\includeonly| command
to the present needs. This easily leaves the main file in a messy state.
\item
The generated document will always carry the filename
of the main document. This is inconvenient if
several child files are to be compiled and
to be kept for distribution.
\end{itemize}

The present package provides a simple interface
to make child files individually compilable by \LaTeX{}.
Compiling a child file then has the same effect as compiling
the main file with an |\includeonly| command
to select the appropriate child.
Moreover the generated document will carry the name of the child
rather than the main file.
This resolves all three above issues.

This feature is meant to make the editing of books,
thesis documents and lecture notes somewhat more convenient.
However, the package can also be used efficiently for
composing a series of documents (such as exercise sheets)
which are typically distributed individually.
It then assists the author in generating the individual documents
(potentially in different versions)
as well as a document containing the collected series.
Another application is in developing style files
or other kinds of included material
where compilation of the style file could redirect
to a sample or test file.

%%%%%%%%%%%%%%%%%%%%%%%%%%%%%%%%%%%%%%%%%%%%%%%%%%%%%%%%%%%%%%%%%%%%%%%%%%%%%%%%
%%%%%%%%%%%%%%%%%%%%%%%%%%%%%%%%%%%%%%%%%%%%%%%%%%%%%%%%%%%%%%%%%%%%%%%%%%%%%%%%
\section{Usage}

First of all, the package \textsf{childdoc} is \emph{not} a standard
\LaTeXe{} |.sty| style file! Therefore it needs to be invoked in
a non-standard way.

%%%%%%%%%%%%%%%%%%%%%%%%%%%%%%%%%%%%%%%%%%%%%%%%%%%%%%%%%%%%%%%%%%%%%%%%%%%%%%%%
\subsection{Included Files}
\label{sec:include}

%%%%%%%%%%%%%%%%%%%%%%%%%%%%%%%%%%%%%%%%
\DescribeMacro{\childdocmain}
To use the package, add the commands
\begin{center}
\begin{tabular}{l}
|\input{childdoc.def}|\\
|\childdocmain{}|\\
\end{tabular}
\end{center}
at the very top of the main \LaTeX{} file,
in particular \emph{before} the |\documentclass| statement!
The argument of |\childdocmain| should be left empty
(but it must be present).

%%%%%%%%%%%%%%%%%%%%%%%%%%%%%%%%%%%%%%%%
\DescribeMacro{\childdocof}
Furthermore, add the commands
\begin{center}
\begin{tabular}{l}
|\input{childdoc.def}|\\
|\childdocof{|\textit{main}|}|\\
\end{tabular}
\end{center}
at the top of every child file \textit{child}
which is included by |\include{|\textit{child}|}|
from within the main file
(or at least for those files to be compiled individually).
The argument \textit{main} must be the filename of the main file.

There are a couple of
considerations in setting up the main and child documents:

%%%%%%%%%%%%%%%%%%%%%%%%%%%%%%%%%%%%%%%%
\paragraph{Restrictions.}

Please note the following restrictions:
\begin{itemize}
\item
|\childdocmain| must be called with one argument \textit{main}
to ensure compatibility with earlier version of the package.
It must either be empty (|\childdocmain{}|)
or precisely match the filename of the main file in which it is specified.
See \secref{sec:detection} for further information.
\item
The filename \textit{main} must be specified without the |.tex| extension.
\item
The filename \textit{main} is case sensitive
(even in case-insensitive file systems)
due to internal string comparison.
\item
The argument \textit{main} should be fully expanded, it cannot be a macro.
\item
Subdirectories and special characters should be avoided in filenames.
\item
The command |\childdocmain{|\textit{main}|}| must be followed by a whitespace.
It should not be followed immediately by another command
or by a comment mark `|%|'.
This is because the \TeX{} parser reads the token immediately following
the argument of |\childdocmain| and puts it
at the beginning of every child section;
however, a white\-space is ignored.
\end{itemize}

%%%%%%%%%%%%%%%%%%%%%%%%%%%%%%%%%%%%%%%%
\paragraph{Content of Main File.}

It is advisable to place all content in the child files included by |\include|.
Any output contained in the main file will appear in all child documents
unless suppressed manually;
it cannot be suppressed automatically by the |\includeonly| directive
and thus should normally be avoided.
A method to include some content in the main file
by means of conditional processing is described in \secref{sec:conditional}.

%%%%%%%%%%%%%%%%%%%%%%%%%%%%%%%%%%%%%%%%
\paragraph{Page Numbering.}

When only a part of the document is compiled,
the appropriate numbering of pages
(as well as other status parameters)
is determined from the |.aux| files.
The latter contain information from previous passes.
However this information needs to propagate through
all intermediate child documents.
Therefore the page numbering in child documents may well
be inconsistent until the complete document is compiled at least once.

A useful (if unconventional) way to always ensure a consistent
page numbering is to restart the numbering in each child document
and denote the pages by `\textit{child}|.|\textit{page}'
where \textit{child} represents the chapter/section number of the child file.
This can be achieved by the command
|\numberwithin{page}{|\textit{child}|}|
of the \textsf{amsmath} package
where \textit{child} can be |chapter| or |section|
depending on the chosen structuring.
Alternatively, one can modify the macro |\thepage| appropriately
and reset the counter |page| at the start of each child file.

%%%%%%%%%%%%%%%%%%%%%%%%%%%%%%%%%%%%%%%%%%%%%%%%%%%%%%%%%%%%%%%%%%%%%%%%%%%%%%%%
\subsection{Conditional Processing}
\label{sec:conditional}

The package provides a mechanism to compile different versions
of a document. To customise the versions further some conditional processing
can come in handy to distinguish which version is being compiled.
The package provides two macros to describe the compilation context:

%%%%%%%%%%%%%%%%%%%%%%%%%%%%%%%%%%%%%%%%
\DescribeMacro{\ifchilddoc}
The conditional |\ifchilddoc| distinguishes between the compilation of
child documents and the main document:
%
\begin{center}
|\ifchilddoc |\textit{child-code}| |[|\||else |\textit{main-code}]| \||fi|
\end{center}

%%%%%%%%%%%%%%%%%%%%%%%%%%%%%%%%%%%%%%%%
\DescribeMacro{\childdocname}
\DescribeMacro{\childdocjob}
The macro |\childdocname| contains the filename (without extension)
of the main or child file being processed.
Note that |\childdocjob| will always contain the name of the main file.

%%%%%%%%%%%%%%%%%%%%%%%%%%%%%%%%%%%%%%%%
\paragraph{Title Page.}

Conditional processing can be used to include a title or banner page
in the main document when proper precautions are taken.
Importantly, the code in the main file should ensure that the page counter
(as well as other status parameters which are stored in the |.aux| files)
takes the same value after the conditional processing.
Otherwise the page numbers may take divergent values
depending on which part is compiled.

For example, a title page could be declared by:
%
\begin{center}
\begin{tabular}{l}
|\ifchilddoc\||else|\\
|\addtocounter{page}{-1}|\\
\textit{code for title page}\\
|\newpage|\\
|\||fi|
\end{tabular}
\end{center}
%
A banner page for the child documents can be generated by:
%
\begin{center}
\begin{tabular}{l}
|\ifchilddoc|\\
|\addtocounter{page}{-1}|\\
\textit{code for banner page}\\
|\newpage|\\
|\||fi|
\end{tabular}
\end{center}
%
Here one could write a message such as:
\begin{center}
|This is the part \childdocname{} of \childdocjob{}.|
\end{center}

%%%%%%%%%%%%%%%%%%%%%%%%%%%%%%%%%%%%%%%%%%%%%%%%%%%%%%%%%%%%%%%%%%%%%%%%%%%%%%%%
\subsection{Flags}
\label{sec:flags}

The package makes it easy to generate different versions
of the main or child documents.
To this end compilation flags can be defined
and assigned different default values.
They will be particularly useful in conjunction
with the forwarding mechanism described in \secref{sec:forward}.

For example, it may be useful to have a flag |\version|
which can be set to |draft| or |final|.
The document source will contain some conditional code
depending on the value of |\version|.
Suppose further, the flag should default to |final| for the main file
and to |draft| for child files
which is a natural assignment for editing the document.
This is achieved by placing the following code
in the preamble of the main document
(below the |\childdocmain| directive):
%
\begin{center}
\begin{tabular}{l}
|\ifchilddoc|\\
|\providecommand{\version}{draft}|\\
|\||else|\\
|\providecommand{\version}{final}|\\
|\||fi|
\end{tabular}
\end{center}
%
The definition by |\providecommand| makes sure
that previous definitions are not overwritten.
Further statements |\providecommand{\version}{...}|
can thus be added before the above code to override it.

For the main file, one might add a line
(between |\childdocmain| and the above block)
%
\begin{center}
|%\ifchilddoc\||else\providecommand{\version}{draft}\||fi|
\end{center}
%
which can be uncommented to produce a draft version.
Likewise one can add a line to the very top of a child file
(above the |\childdocof{|\textit{main}|}| directive)
%
\begin{center}
|%\providecommand{\version}{final}|
\end{center}
%
which can be uncommented to produce the final version of this child document.

%%%%%%%%%%%%%%%%%%%%%%%%%%%%%%%%%%%%%%%%%%%%%%%%%%%%%%%%%%%%%%%%%%%%%%%%%%%%%%%%
\subsection{Forwarding}
\label{sec:forward}

Different versions of the main or child documents
using compilation flags as described in \secref{sec:flags}
can be (permanently) stored in different files
for convenient compilation, viewing and distribution.
To this end, the package defines a command
to pass on compilation to a different file:

%%%%%%%%%%%%%%%%%%%%%%%%%%%%%%%%%%%%%%%%
\DescribeMacro{\childdocforward}
The command |\childdocforward| redirects processing to
another source file:
%
\begin{center}
\begin{tabular}{l}
|\input{childdoc.def}|\\
|\childdocforward[|\textit{main}|]{|\textit{dest}|}|\\
\end{tabular}
\end{center}
%
The argument \textit{dest} is the destination file
(without extension).
It should be the main file or one of the child files.
Note that further \textsf{childdoc} directives
such as |\childdocof| and |\childdocforward|
in the indicated file will be processed in this form.
The optional argument \textit{main}
passes on directly to the main file \textit{main}
while pretending to compile the child \textit{dest}.
This form behaves as if \textit{dest}
issues |\childdocof{|\textit{main}|}| right away,
and no further \textsf{childdoc} directives will be processed.

%%%%%%%%%%%%%%%%%%%%%%%%%%%%%%%%%%%%%%%%
\DescribeMacro{\...prefix}
In the alternative form |\childdocforwardprefix|,
%
\begin{center}
\begin{tabular}{l}
|\input{childdoc.def}|\\
|\childdocforwardprefix[|\textit{main}|]{|\textit{prefix}|}{|\textit{dest}|}|
\end{tabular}
\end{center}
%
the destination file is determined by a pattern
depending on the current file:
To make this work, the current file must be called
`{\textit{prefix}\hspace{0.2em}\textit{suffix}}'
with \textit{prefix} matching precisely the argument.
Processing is then passed on to the file
`{\textit{dest}\hspace{0.2em}\textit{suffix}}'.
Surely, the same effect is achieved by
directly specifying the
argument `{\textit{dest}\hspace{0.2em}\textit{suffix}}'
in the first form.
However, that requires to set up a different file
for each child. With the alternative form of the command
all these files can have exactly the same content
which simplifies setting them up and maintaining them.

For example, the following file |draft.tex|
with a compilation flag |\version| as described in \secref{sec:flags}
compiles the main document as a draft:
%
\begin{center}
\begin{tabular}{l}
|\def\version{draft}|\\
|\input{childdoc.def}|\\
|\childdocforward{|\textit{main}|}|
\end{tabular}
\end{center}
%
Likewise, the following files |final|\textit{nn}|.tex|
compile the final version of the child document
|child|\textit{nn}|.tex|:
%
\begin{center}
\begin{tabular}{l}
|\def\version{final}|\\
|\input{childdoc.def}|\\
|\childdocforwardprefix{final}{child}|
\end{tabular}
\end{center}
%

Note that when several versions of a main file and/or of each child file
are to be generated, it may be convenient to set up a |Makefile| or
shell script to automatise the process.

%%%%%%%%%%%%%%%%%%%%%%%%%%%%%%%%%%%%%%%%%%%%%%%%%%%%%%%%%%%%%%%%%%%%%%%%%%%%%%%%
\subsection{Command Line Processing}
\label{sec:commandline}

The effect of redirection files can also be achieved by invoking
the \LaTeX{} compiler with a more elaborate command line.
Most conveniently this should be done as part
of a shell script or a |Makefile|.

When using \textsf{childdoc} in the main file, the following
command lines effectively perform a redirection
(note that depending on the shell being used,
backslashes may have to be doubled: `|\|' $\to$ `|\\|'):
%
\begin{center}
|... -jobname "|\textit{target}|" |\\|"|[\textit{flags}]%
|\input{childdoc.def}\childdocforward[|\textit{main}|]{|\textit{dest}|}"|
\end{center}
%
Here \textit{target} is the name of the output file,
\textit{main} is the name of the main file
and \textit{dest} is the name of the main or child file to be processed
(all filenames without extensions).
The optional argument \textit{main} can be omitted
if \textit{main} matches \textit{dest}.
Optionally, compilation \textit{flags} can be defined via |\def| commands.
This command line makes the \TeX{} engine believe
it is compiling the file \textit{target}
whose content is specified as the latter parameter.
The provided code then forwards the processing to
\textit{main} or \textit{dest} as described in \secref{sec:forward}.

%%%%%%%%%%%%%%%%%%%%%%%%%%%%%%%%%%%%%%%%%%%%%%%%%%%%%%%%%%%%%%%%%%%%%%%%%%%%%%%%
\subsection{Include by Input}
\label{sec:input}

Including child documents by |\include| has some restrictions by design.
Most notably, the content of a child document always occupies
its own set of pages; pages cannot be shared between child documents.
Usually, this behaviour makes perfect sense
because each child document contain an essential part of the document.
However, in some situations it may be desirable to compose
a document from a collection of parts
without having mandatory page breaks between then.
For this case, the package
provides a mechanism to include parts
by |\input| which can also be processed individually.
However, by construction this mechanism
requires manual handling of the content to be output.

%%%%%%%%%%%%%%%%%%%%%%%%%%%%%%%%%%%%%%%%
\DescribeMacro{\ifchilddocmanual}
The main file should be prepared as usual, see \secref{sec:include}.
However, the document body must make a distinction
between processing of an individual part and of the main document, e.g.:
%
\begin{center}
\begin{tabular}{l}
|\ifchilddocmanual|\\
|\input{\childdocname}|\\
|\||else|\\
\textit{document body with }|\input{|\textit{part}|}|\\
|\||fi|
\end{tabular}
\end{center}
%
The conditional |\ifchilddocmanual| is true whenever
a part to be included by |\input| is being compiled,
and the name of the part is stored in |\childdocname|.

%%%%%%%%%%%%%%%%%%%%%%%%%%%%%%%%%%%%%%%%
\DescribeMacro{\childdocby}
Each part to be included by |\input| should start with:
%
\begin{center}
\begin{tabular}{l}
|\input{childdoc.def}|\\
|\childdocby{|\textit{main}|}|\\
\end{tabular}
\end{center}
%
The directive |\childdocby| is similar to |\childdocof|
described in \secref{sec:include},
but the subsequent selection of content must be done manually.
To that end, both |\ifchilddoc| and |\ifchilddocmanual|
will be true upon processing of a part,
and the name of the part is stored in |\childdocname|.
Note that |\jobname| will be set to the filename of the current part
so that each part receives an individual |.aux| file
that does not interfere with the |.aux| file(s) of the main document.
This behaviour can be altered by the alternative form
|\childdocby[*]{|\textit{main}|}| (with a non-empty optional argument)
which uses the |.aux| file of the main document
by setting |\jobname| to \textit{main}.

%%%%%%%%%%%%%%%%%%%%%%%%%%%%%%%%%%%%%%%%%%%%%%%%%%%%%%%%%%%%%%%%%%%%%%%%%%%%%%%%
\subsection{Driver Development}
\label{sec:driver}

The \textsf{childdoc} mechanism can also be use for the development
of definition files such as \LaTeX{} styles or classes.
This case differs from the above setup with multiple parts
included by |\include| in that no |\includeonly| should be invoked.
This can be achieved by starting the include file
(before |\ProvidesPackage|) with:
%
\begin{center}
\begin{tabular}{l}
|\input{childdoc.def}|\\
|\childdocforward{|\textit{main}|}|\\
\end{tabular}
\end{center}
%
or alternatively with:
%
\begin{center}
\begin{tabular}{l}
|\input{childdoc.def}|\\
|\childdocby{|\textit{main}|}|\\
\end{tabular}
\end{center}
%
Both forms have slightly different effects as described above.
The main file is prepared as usual, see \secref{sec:include}.

%%%%%%%%%%%%%%%%%%%%%%%%%%%%%%%%%%%%%%%%%%%%%%%%%%%%%%%%%%%%%%%%%%%%%%%%%%%%%%%%
\subsection{Legacy Detection}
\label{sec:detection}

The directive |\childdocmain| in the main file can detect
whether the complete document or merely a child is to be compiled
even without using the directive |\childdocof|.
This method is deprecated because it is less robust
and there is no compelling reason to use it;
it is merely provided for backward compatibility
and it may be removed in future versions.

If the detection mechanism is to be used,
it is mandatory to correctly specify
the filename of the main file as the argument of |\childdocmain|:
%
\begin{center}
\begin{tabular}{l}
|\input{childdoc.def}|\\
|\childdocmain{|\textit{main}|}|\\
\end{tabular}
\end{center}
%
If |\jobname| does not match the argument \textit{main} of |\childdocmain|,
it is assumed that |\jobname| points to the child file to be compiled.
When using |\childdocmain| with the main file specified as argument,
it suffices to start a child file
with just |\input{|\textit{main}|}|
without loading of the package and using |\childdocof|.
If instead all processing is done
with the appropriate \textsf{childdoc} directives,
the argument of \textit{main} of |\childdocmain| can be empty.

An alternative version of the command line processing described
in \secref{sec:commandline} using the detection mechanism reads:
%
\begin{center}
|... -jobname "|\textit{target}|" "|[\textit{flags}]%
[|\def\jobname{|\textit{dest}|}|]|\input{|\textit{main}|}"|
\end{center}

%%%%%%%%%%%%%%%%%%%%%%%%%%%%%%%%%%%%%%%%%%%%%%%%%%%%%%%%%%%%%%%%%%%%%%%%%%%%%%%%
\subsection{Manual Code}
\label{sec:manual}

In case one cannot be certain whether the definitions file |childdoc.def|
is installed on the target \TeX{} distribution
and one prefers not to ship it,
it is conceivable to paste a few relevant commands into the sources.

To that end, drop all statements |\input{childdoc.def}|
and perform the replacements as outlined below.
Instead of |\childdocmain{|\textit{main}|}| add the following code
to the top of the main file:
%
\begin{center}
\begin{tabular}{l}
|\||ifdefined\childdocname\endinput\||fi\newif\ifchilddoc|\\
|\edef\childdocname{\scantokens\expandafter{\jobname\noexpand}}|\\
|\def\childdocmain{|\textit{main}|}\||ifx\childdocmain\childdocname\||else|\\
|\childdoctrue\includeonly{\childdocname}\let\jobname\childdocmain\||fi|\\
\end{tabular}
\end{center}
%
Instead of |\childdocof{|\textit{main}|}| just include the main file
at the top of each child file:
%
\begin{center}
|\input{|\textit{main}|}|
\end{center}
%
A simple redirection |\childdocforward{|\textit{dest}|}| is achieved by:
%
\begin{center}
|\def\jobname{|\textit{dest}|}\input{\jobname}|
\end{center}
%
The redirection with prefix
|\childdocforwardprefix[|\textit{prefix}|]{|\textit{dest}|}|
is accomplished by:
%
\begin{center}
\begin{tabular}{l}
|{\edef\jobname{\scantokens\expandafter{\jobname\noexpand}}|\\
|\def\redirectjob |\textit{prefix}|#1~~~{\gdef\jobname{|\textit{dest}|#1}}|\\
|\expandafter\redirectjob\jobname~~~}\input{\jobname}|
\end{tabular}
\end{center}

In an alternative approach,
child documents can be compiled by a specific command line
without additional code or specific definitions:
%
\begin{center}
|... -jobname "|\textit{target}|" "|[\textit{flags}]%
|\includeonly{|\textit{dest}|}\input{|\textit{main}|}"|
\end{center}
%

%%%%%%%%%%%%%%%%%%%%%%%%%%%%%%%%%%%%%%%%%%%%%%%%%%%%%%%%%%%%%%%%%%%%%%%%%%%%%%%%
%%%%%%%%%%%%%%%%%%%%%%%%%%%%%%%%%%%%%%%%%%%%%%%%%%%%%%%%%%%%%%%%%%%%%%%%%%%%%%%%
\section{Information}

%%%%%%%%%%%%%%%%%%%%%%%%%%%%%%%%%%%%%%%%%%%%%%%%%%%%%%%%%%%%%%%%%%%%%%%%%%%%%%%%
\subsection{Copyright}

Copyright \copyright{} 2017--2018 Niklas Beisert

This work may be distributed and/or modified under the
conditions of the \LaTeX{} Project Public License, either version 1.3
of this license or (at your option) any later version.
The latest version of this license is in
  \url{http://www.latex-project.org/lppl.txt}
and version 1.3 or later is part of all distributions of \LaTeX{}
version 2005/12/01 or later.

This work has the LPPL maintenance status `maintained'.

The Current Maintainer of this work is Niklas Beisert.

This work consists of the files |README.txt|, |childdoc.ins| and |childdoc.dtx|
as well as the derived files |childdoc.def|, |cdocsamp.tex|
with |cdocsch1.tex|, |cdocsch2.tex|, |cdocspt3.tex|, |cdocspt4.tex|,
|cdocsdrf.tex|, |cdocsfn1.tex|, |cdocsfn2.tex|
as well as |childdoc.pdf|.

%%%%%%%%%%%%%%%%%%%%%%%%%%%%%%%%%%%%%%%%%%%%%%%%%%%%%%%%%%%%%%%%%%%%%%%%%%%%%%%%
\subsection{Files and Installation}

The package consists of the files:
%
\begin{center}
\begin{tabular}{ll}
    |README.txt|   & readme file \\
    |childdoc.ins| & installation file \\
    |childdoc.dtx| & source file \\
    |childdoc.def| & definition file \\
    |cdocsamp.tex| & sample main file \\
    |cdocsch1.tex| & sample include file \\
    |cdocsch2.tex| & sample include file \\
    |cdocspt3.tex| & sample part file \\
    |cdocspt4.tex| & sample part file \\
    |cdocsdrf.tex| & sample redirection file \\
    |cdocsfn1.tex| & sample redirection file \\
    |cdocsfn2.tex| & sample redirection file \\
    |childdoc.pdf| & manual
\end{tabular}
\end{center}
%
The distribution consists of the files
|README.txt|, |childdoc.ins| and |childdoc.dtx|.
%
\begin{itemize}
\item
Run (pdf)\LaTeX{} on |childdoc.dtx|
to compile the manual |childdoc.pdf| (this file).
\item
Run \LaTeX{} on |childdoc.ins| to create the definitions file |childdoc.def|
and the sample |cdocsamp.tex| with include files
|cdocsch1.tex|, |cdocsch2.tex|, |cdocspt3.tex|, |cdocspt4.tex|,
|cdocsdrf.tex|, |cdocsfn1.tex|, |cdocsfn2.tex|.
Then copy the file |childdoc.def| to an appropriate directory of your \LaTeX{}
distribution, e.g.\ \textit{texmf-root}|/tex/latex/childdoc|.
\end{itemize}

%%%%%%%%%%%%%%%%%%%%%%%%%%%%%%%%%%%%%%%%%%%%%%%%%%%%%%%%%%%%%%%%%%%%%%%%%%%%%%%%
\subsection{Related CTAN Packages}

There are several other packages which offer a similar functionality:
%
\begin{itemize}
\item
The packages
\href{http://ctan.org/pkg/docmute}{\textsf{docmute}},
\href{http://ctan.org/pkg/includex}{\textsf{includex}} and
\href{http://ctan.org/pkg/standalone}{\textsf{standalone}}
provide commands to include only the document body of
a child file thus allowing both files to be compiled individually.
\item
The packages \href{http://ctan.org/pkg/subdocs}{\textsf{subdocs}}
and \href{http://ctan.org/pkg/subfiles}{\textsf{subfiles}}
provide structures in which the main and child documents can be
encapsulated and allowing them to be compiled individually.
The inclusion mechanism is different from the conventional |\include|.
\item
The package \href{http://ctan.org/pkg/combine}{\textsf{combine}}
is an elaborate solution to combine several documents into one.
\end{itemize}
%
See also the CTAN topic \href{http://ctan.org/topic/subdocs}{\textsf{subdocs}}
for further related packages.
The present package differs from the above solutions in that
a document structure constructed with the conventional |\include| mechanism
just needs two extra commands at the top of every file
such that all constituent files can be compiled individually.

%%%%%%%%%%%%%%%%%%%%%%%%%%%%%%%%%%%%%%%%%%%%%%%%%%%%%%%%%%%%%%%%%%%%%%%%%%%%%%%%
%\subsection{Feature Suggestions}
%
%The following is a list of features which may be useful for future
%versions of this package:
%%
%\begin{itemize}
%\item
%\ldots
%\end{itemize}

%%%%%%%%%%%%%%%%%%%%%%%%%%%%%%%%%%%%%%%%%%%%%%%%%%%%%%%%%%%%%%%%%%%%%%%%%%%%%%%%
\subsection{Revision History}

%%%%%%%%%%%%%%%%%%%%%%%%%%%%%%%%%%%%%%%%
\paragraph{v2.0:} 2018/12/30

\begin{itemize}
\item
immediate forward processing
\item
added |\childdocby| mechanism
\item
manual restructured
\end{itemize}

%%%%%%%%%%%%%%%%%%%%%%%%%%%%%%%%%%%%%%%%
\paragraph{v1.6:} 2018/01/17

\begin{itemize}
\item
application for development of include files
\item
corrections to manual
\end{itemize}

%%%%%%%%%%%%%%%%%%%%%%%%%%%%%%%%%%%%%%%%
\paragraph{v1.5:} 2017/05/21

\begin{itemize}
\item
more complete structuring introduced
\item
|\childdocof| introduced
\item
|\childdoc| renamed to |\childdocmain|
\item
|\childredirect| renamed to |\childdocforward| and |\childdocforwardprefix|
and functionality expanded
\end{itemize}

%%%%%%%%%%%%%%%%%%%%%%%%%%%%%%%%%%%%%%%%
\paragraph{v1.0:} 2017/04/27

\begin{itemize}
\item
manual and install package
\item
first version published on CTAN
\end{itemize}

%%%%%%%%%%%%%%%%%%%%%%%%%%%%%%%%%%%%%%%%
\paragraph{v0.6:} 2017/04/26

\begin{itemize}
\item
redirection mechanism added
\end{itemize}

%%%%%%%%%%%%%%%%%%%%%%%%%%%%%%%%%%%%%%%%
\paragraph{v0.5:} 2017/04/26

\begin{itemize}
\item
functionality in definition file
\end{itemize}


%%%%%%%%%%%%%%%%%%%%%%%%%%%%%%%%%%%%%%%%%%%%%%%%%%%%%%%%%%%%%%%%%%%%%%%%%%%%%%%%
%%%%%%%%%%%%%%%%%%%%%%%%%%%%%%%%%%%%%%%%%%%%%%%%%%%%%%%%%%%%%%%%%%%%%%%%%%%%%%%%
%%%%%%%%%%%%%%%%%%%%%%%%%%%%%%%%%%%%%%%%%%%%%%%%%%%%%%%%%%%%%%%%%%%%%%%%%%%%%%%%
\appendix

\settowidth\MacroIndent{\rmfamily\scriptsize 000\ }

 \DocInput{childdoc.dtx}

\end{document}
%</driver>
% \fi
%
% %%%%%%%%%%%%%%%%%%%%%%%%%%%%%%%%%%%%%%%%%%%%%%%%%%%%%%%%%%%%%%%%%%%%%%%%%%%%%%
% %%%%%%%%%%%%%%%%%%%%%%%%%%%%%%%%%%%%%%%%%%%%%%%%%%%%%%%%%%%%%%%%%%%%%%%%%%%%%%
% \section{Sample}
%\iffalse
%<*samplemain>
%\fi
%
% The following presents a sample document
% with two chapters, two parts, a title page,
% a compile flag as well as three forwarding files to set the flag.
% It consists of eight |.tex| files:
% \begin{center}
% \begin{tabular}{ll}
% |cdocsamp.tex|&main file\\
% |cdocsch1.tex|&include file for chapter 1\\
% |cdocsch2.tex|&include file for chapter 2\\
% |cdocspt3.tex|&include file for part 3\\
% |cdocspt4.tex|&include file for part 4\\
% |cdocsdrf.tex|&forwarding file for main file in draft mode\\
% |cdocsfi1.tex|&forwarding file for final version of chapter 1\\
% |cdocsfi2.tex|&forwarding file for final version of chapter 2\\
% \end{tabular}
% \end{center}
% Each of the eight files can be compiled directly by the \LaTeX{} compiler.
%
% %%%%%%%%%%%%%%%%%%%%%%%%%%%%%%%%%%%%%%
% \paragraph{Main File.}
%
% The main file is called |cdocsamp.tex|.
%
% Load the \textsf{childdoc} definitions and
% declare the filename for the main document:
%    \begin{macrocode}
\input{childdoc.def}
\childdocmain{}
%    \end{macrocode}

% Optional override for |\version| flag:
%    \begin{macrocode}
%%\ifchilddoc\else\providecommand{\version}{draft}\fi
%    \end{macrocode}

% Define the default values for the |\version| flag
% (|final| for the main file and |draft| for childs):
%    \begin{macrocode}
\ifchilddoc
\providecommand{\version}{draft}
\else
\providecommand{\version}{final}
\fi
%    \end{macrocode}

% Load the standard document class:
%    \begin{macrocode}
\documentclass[12pt]{article}
%    \end{macrocode}

% Start the document body:
%    \begin{macrocode}
\begin{document}
%    \end{macrocode}

% Declare a title page.
% Print title, part of document being processed and version flag:
%    \begin{macrocode}
\addtocounter{page}{-1}
\begin{center}
{\LARGE\bfseries{}childdoc example\par}
\vspace{1cm}
\ifchilddoc
\ifchilddocmanual part\else chapter\fi:
`\childdocname' of `\childdocjob'\par
\else
main document: `\childdocjob'\par
\fi
version: \version\par
\end{center}
\newpage
%    \end{macrocode}

% Manually include selected file,
% otherwise process as usual:
%    \begin{macrocode}
\ifchilddocmanual
\section*{part `\childdocname'}
\input{\childdocname}
\else
%    \end{macrocode}

% Include the two chapters:
%    \begin{macrocode}
\include{cdocsch1}
\include{cdocsch2}
%    \end{macrocode}

% Include the two parts unless only chapters should be displayed:
%    \begin{macrocode}
\ifchilddoc\else
\section{part three}
\input{cdocspt3}
\section{part four}
\input{cdocspt4}
\fi
%    \end{macrocode}

% Process as usual until here:
%    \begin{macrocode}
\fi
%    \end{macrocode}

% End of document body:
%    \begin{macrocode}
\end{document}
%    \end{macrocode}
%\iffalse
%</samplemain>
%\fi
%
% %%%%%%%%%%%%%%%%%%%%%%%%%%%%%%%%%%%%%%
% \paragraph{Chapter Include Files.}
%
% The include files are called |cdocsch1.tex| and |cdocsch2.tex|.
%
%\iffalse
%<*samplechap1|samplechap2>
%\fi

% Optional override for |\version| flag:
%    \begin{macrocode}
%%\providecommand{\version}{final}
%    \end{macrocode}

% Include the main document:
%    \begin{macrocode}
\input{childdoc.def}
\childdocof{cdocsamp}
%    \end{macrocode}

%\iffalse
%</samplechap1|samplechap2>
%\fi
%
%\iffalse
%<*samplechap1>
%\fi
% Some text for chapter 1:
%    \begin{macrocode}
\section{one}
some text in chapter one
%    \end{macrocode}

%\iffalse
%</samplechap1>
%\fi
% Some text for chapter 2:
%\iffalse
%<*samplechap2>
%\fi
%    \begin{macrocode}
\section{two}
more text in chapter two
%    \end{macrocode}

%\iffalse
%</samplechap2>
%\fi
%
% %%%%%%%%%%%%%%%%%%%%%%%%%%%%%%%%%%%%%%
% \paragraph{Part Include Files.}
%
% The include files are called |cdocspt3.tex| and |cdocspt4.tex|.
%
%\iffalse
%<*samplepart3|samplepart4>
%\fi

% Optional override for |\version| flag:
%    \begin{macrocode}
%%\providecommand{\version}{final}
%    \end{macrocode}

% Include the main document:
%    \begin{macrocode}
\input{childdoc.def}
\childdocby{cdocsamp}
%    \end{macrocode}

%\iffalse
%</samplepart3|samplepart4>
%\fi
%
%\iffalse
%<*samplepart3>
%\fi
% Some text for part 3:
%    \begin{macrocode}
some text in part three
%    \end{macrocode}

%\iffalse
%</samplepart3>
%\fi
% Some text for part 4:
%\iffalse
%<*samplepart4>
%\fi
%    \begin{macrocode}
more text in part four
%    \end{macrocode}

%\iffalse
%</samplepart4>
%\fi
%
% %%%%%%%%%%%%%%%%%%%%%%%%%%%%%%%%%%%%%%
% \paragraph{Forwarding for a Complete Draft.}
%
% The following forwarding file |cdocsdrf.tex|
% compiles the main document in draft mode:
%\iffalse
%<*sampledraft>
%\fi
%    \begin{macrocode}
\def\version{draft}
\input{childdoc.def}
\childdocforward{cdocsamp}
%    \end{macrocode}

%\iffalse
%</sampledraft>
%\fi
%
% %%%%%%%%%%%%%%%%%%%%%%%%%%%%%%%%%%%%%%
% \paragraph{Forwarding for Final Version of the Chapters.}
%
% The following forwarding files |cdocsfn1.tex| and |cdocsfn2.tex|
% (with identical content)
% compile the final versions of the child documents
% |cdocsch1.tex| and |cdocsch2.tex|, respectively:
%\iffalse
%<*samplefinal>
%\fi
%    \begin{macrocode}
\def\version{final}
\input{childdoc.def}
\childdocforwardprefix[cdocsamp]{cdocsfn}{cdocsch}
%    \end{macrocode}

%\iffalse
%</samplefinal>
%\fi
%
% %%%%%%%%%%%%%%%%%%%%%%%%%%%%%%%%%%%%%%
% \paragraph{Command Line Processing.}
%
% The following three command lines generate the output files
% |cdocscld|, |cdocscl1| and |cdocscl2|
% which should be identical to
% |cdocsdrf|, |cdocsch1| and |cdocsfn2|, respectively:
% \begin{center}
% \begin{tabular}{l}
% |latex -jobname cdocscld \|\\
% |  "\def\version{draft}\input{childdoc.def}\childdocforward{cdocsamp}"|\\
% |latex -jobname cdocscl1 \|\\
% |  "\input{childdoc.def}\childdocforward[cdocsamp]{cdocsch1}"|\\
% |latex -jobname cdocscl2 \|\\
% |  "\def\version{final}\input{childdoc.def}\childdocforward{cdocsch2}"|
% \end{tabular}
% \end{center}
% Note that the trailing backslash on each first line
% merely continues the input to the second line
% (for convenient cut ant paste).
% Furthermore, the command |latex| can be replaced by any
% of its alternative versions such as |pdflatex|.
%
% %%%%%%%%%%%%%%%%%%%%%%%%%%%%%%%%%%%%%%%%%%%%%%%%%%%%%%%%%%%%%%%%%%%%%%%%%%%%%%
% %%%%%%%%%%%%%%%%%%%%%%%%%%%%%%%%%%%%%%%%%%%%%%%%%%%%%%%%%%%%%%%%%%%%%%%%%%%%%%
% \section{Implementation}
%\iffalse
%<*package>
%\fi
%
% This section describes the definitions file |childdoc.def|.

% The definitions cannot be loaded using |\usepackage| or |\RequirePackage|
% which has a mechanism to prevent loading a style file more than once.
% When loading the definitions by means of |\input|
% multiple instances have to be prevented manually:
%\iffalse
%This code needs to be before the `\ProvidesFile' directive
%which is defined at the beginning of this file.
%Therefore it is also placed there and commented out here.
%</package>
%<*discard>
%\fi
%    \begin{macrocode}
\ifdefined\childdocmain\endinput\fi
%    \end{macrocode}
%\iffalse
%</discard>
%<*package>
%\fi
%
% \macro{\ifchilddoc}
% \macro{\ifchilddocmanual}
% The conditional |\ifchilddoc| tells whether a
% child (true) or main (false) document is being compiled.
% The conditional |\ifchilddocmanual| tells whether
% the |\includeonly| mechanism is used (false) or
% the selection of child files must be performed manually (true).
% The definitions initialise to false:
%    \begin{macrocode}
\newif\ifchilddoc
\newif\ifchilddocmanual
%    \end{macrocode}

% \macro{\childdocname}
% \macro{\childdocjob}
% The macro |\childdocname| stores the name of the main document
% to be compiled. The macro |\childdocjob| stores the name of
% the document on which the \LaTeX{} compiler was originally invoked.
% The content of |\jobname| cannot be compared
% to filenames specified in the source due to different catcodes.
% The following code rescans |\jobname|, stores the result
% in |\childdocname| and saves a copy in |\childdocjob|:
%    \begin{macrocode}
\edef\childdocname{\scantokens\expandafter{\jobname\noexpand}}
\let\childdocjob\childdocname
%    \end{macrocode}

% \macro{\childdocdisable}
% The macro |\childdocdisable| prevents the main file
% from being processed more than once.
% At this stage, the main document command |\childdocmain|
% is assumed to be called once again where it should do nothing.
% Any subsequent call to it should prevent
% a secondary processing of the main document
% It overwrites the forwarding commands
% |\childdocof| and |\childdocforward|
% with empty macros to prevent further inclusions of the main document:
%    \begin{macrocode}
\newcommand{\childdocdisable}
{
  \renewcommand{\childdocmain}[1]{\renewcommand{\childdocmain}[1]{\endinput}}
  \renewcommand{\childdocof}[1]{}
  \renewcommand{\childdocby}[2][]{}
  \renewcommand{\childdocforward}[2][]{}
  \renewcommand{\childdocdisable}{}
}
%    \end{macrocode}

% \macro{\childdocmain}
% The macro |\childdocmain| is to be called at the top of the main file
% with nothing or the main filename (without extension) as argument.
% First, it breaks loops.
% If the argument is not empty and does not match |\childdocname|
% (which is set by the first inclusion of |childdoc.def|),
% |\ifchilddoc| is set to true, |\includeonly| is applied to the child file
% and |\jobname| is set to the main file
% (for proper handling of |.aux| files):
%    \begin{macrocode}
\newcommand{\childdocmain}[1]
{
  \childdocdisable\childdocmain{}
  \if?#1?\else
    \begingroup
      \def\childdoctmp{#1}
      \ifx\childdoctmp\childdocname
        \def\childdoctmp{}
      \else
        \def\childdoctmp
        {
          \childdoctrue
          \includeonly{\childdocname}
          \def\childdocjob{#1}
          \def\jobname{#1}
        }
      \fi
      \expandafter
    \endgroup
    \childdoctmp
  \fi
}
%    \end{macrocode}

% \macro{\childdocof}
% The command |\childdocof| redirects
% compilation to the main file |#1|.
%    \begin{macrocode}
\newcommand{\childdocof}[1]
{
  \childdocdisable
  \childdoctrue
  \includeonly{\childdocname}
  \def\jobname{#1}
  \def\childdocjob{#1}
  \input{#1}
}
%    \end{macrocode}

% \macro{\childdocby}
% The command |\childdocby| ....
%    \begin{macrocode}
\newcommand{\childdocby}[2][]
{
  \childdocdisable
  \childdoctrue
  \childdocmanualtrue
  \if?#1?\else
    \def\jobname{#2}
  \fi
  \def\childdocjob{#2}
  \input{#2}
  \endinput
}
%    \end{macrocode}

% \macro{\childdocforward}
% The command |\childdocforward| redirects
% compilation to the main file or
% (if the optional argument is given) a child file.
% Parameters are set as if the main file
% or a child file starting with |\childdocof| was compiled.
% Then compilation is handed over to the main file:
%    \begin{macrocode}
\newcommand{\childdocforward}[2][]
{
  \begingroup
    \if?#1?
      \def\childdoctmp
      {
        \def\childdocname{#2}
        \def\childdocjob{#2}
        \def\jobname{#2}
        \input{#2}
        \endinput
      }
    \else
      \def\childdoctmp
      {
        \childdocdisable
        \def\childdocname{#2}
        \childdoctrue
        \includeonly{#2}
        \def\childdocjob{#1}
        \def\jobname{#1}
        \input{#1}
        \endinput
      }
    \fi
    \expandafter
  \endgroup
  \childdoctmp
}
%    \end{macrocode}

% \macro{\childdocforwardprefix}
% The command |\childdocforwardprefix| redirects
% compilation to the main or a child file by means of a pattern.
% The prefix |#1| in the current filename is replaced by |#2|
% and the suffix of the current filename is kept
% (it is assumed that the filename does not contain the substring `|~~~|'
% which is used as a delimiter).
% Compilation is handed over to the new file by |\childdocforward|:
%    \begin{macrocode}
\newcommand{\childdocforwardprefix}[3][]
{
  \begingroup
    \def\childdocextract #2##1~~~{\def\childdoctmp{\childdocforward[#1]{#3##1}}}
    \expandafter\childdocextract\childdocname~~~
    \expandafter
  \endgroup
  \childdoctmp
}
%    \end{macrocode}

% \macro{\childdoc}
% The deprecated macro |\childdoc| is a legacy version of |\childdocmain|:
%    \begin{macrocode}
\newcommand{\childdoc}{\childdocmain}
%    \end{macrocode}

% \macro{\childdocredirect}
% The deprecated macro |\childdocredirect| is a legacy version
% of |\childdocforward| and |\childdocforwardprefix|:
%    \begin{macrocode}
\newcommand{\childdocredirect}[2][]
{
  \begingroup
    \if?#1?
      \def\childdoctmp{\childdocforward{#2}}
    \else
      \def\childdoctmp{\childdocforwardprefix{#1}{#2}}
    \fi
    \expandafter
  \endgroup
  \childdoctmp
}
%    \end{macrocode}

%\iffalse
%</package>
%\fi
%
\endinput
\childdocforward{cdocsamp}"|\\
% |latex -jobname cdocscl1 \|\\
% |  "% \iffalse
%
% childdoc.dtx Copyright (C) 2017-2018 Niklas Beisert
%
% This work may be distributed and/or modified under the
% conditions of the LaTeX Project Public License, either version 1.3
% of this license or (at your option) any later version.
% The latest version of this license is in
%   http://www.latex-project.org/lppl.txt
% and version 1.3 or later is part of all distributions of LaTeX
% version 2005/12/01 or later.
%
% This work has the LPPL maintenance status `maintained'.
%
% The Current Maintainer of this work is Niklas Beisert.
%
% This work consists of the files childdoc.dtx and childdoc.ins
% and the derived files childdoc.def and cdocsamp.tex with
% cdocsch1.tex, cdocsch2.tex, cdocsdrf.tex, cdocsfn1.tex, cdocsfn2.tex.
%
%<package>\ifdefined\childdocmain\endinput\fi
%<package>\ProvidesFile{childdoc.def}[2018/12/30 v2.0 child document driver]
%<samplemain>\ProvidesFile{cdocsamp.tex}[2018/12/30 v2.0 sample for childdoc]
%<*driver>
%\ProvidesFile{childdoc.drv}[2018/12/30 v2.0 childdoc reference manual file]
\PassOptionsToClass{10pt,a4paper}{article}
\documentclass{ltxdoc}

\usepackage[margin=35mm]{geometry}
\usepackage{hyperref}
\usepackage{hyperxmp}
\usepackage[usenames]{color}

\hypersetup{colorlinks=true}
\hypersetup{pdfstartview=FitH}
\hypersetup{pdfpagemode=UseNone}
\hypersetup{pdfsource={}}
\hypersetup{pdflang={en-UK}}
\hypersetup{pdfcopyright={Copyright 2017-2018 Niklas Beisert.
  This work may be distributed and/or modified under the
  conditions of the LaTeX Project Public License, either version 1.3
  of this license or (at your option) any later version.}}
\hypersetup{pdflicenseurl={http://www.latex-project.org/lppl.txt}}
\hypersetup{pdfcontactaddress={ETH Zurich, ITP, HIT K,
  Wolfgang-Pauli-Strasse 27}}
\hypersetup{pdfcontactpostcode={8093}}
\hypersetup{pdfcontactcity={Zurich}}
\hypersetup{pdfcontactcountry={Switzerland}}
\hypersetup{pdfcontactemail={nbeisert@itp.phys.ethz.ch}}
\hypersetup{pdfcontacturl={http://people.phys.ethz.ch/\xmptilde nbeisert/}}

\newcommand{\secref}[1]{\hyperref[#1]{section \ref*{#1}}}

\parskip1ex
\parindent0pt
\let\olditemize\itemize
\def\itemize{\olditemize\parskip0pt}

\begin{document}

\title{The \textsf{childdoc} Package}
\hypersetup{pdftitle={The childdoc Package}}
\author{Niklas Beisert\\[2ex]
  Institut f\"ur Theoretische Physik\\
  Eidgen\"ossische Technische Hochschule Z\"urich\\
  Wolfgang-Pauli-Strasse 27, 8093 Z\"urich, Switzerland\\[1ex]
  \href{mailto:nbeisert@itp.phys.ethz.ch}
  {\texttt{nbeisert@itp.phys.ethz.ch}}}
\hypersetup{pdfauthor={Niklas Beisert}}
\hypersetup{pdfsubject={Manual for the LaTeX2e Package childdoc}}
\date{30 December 2018, \textsf{v2.0}}
\maketitle

\begin{abstract}\noindent
\textsf{childdoc} is a \LaTeXe{} package
that enables the direct compilation
of document sections included by |\include|
to individual files.
\end{abstract}

\begingroup
\parskip0ex
\tableofcontents
\endgroup

%%%%%%%%%%%%%%%%%%%%%%%%%%%%%%%%%%%%%%%%%%%%%%%%%%%%%%%%%%%%%%%%%%%%%%%%%%%%%%%%
%%%%%%%%%%%%%%%%%%%%%%%%%%%%%%%%%%%%%%%%%%%%%%%%%%%%%%%%%%%%%%%%%%%%%%%%%%%%%%%%
\section{Introduction}

\LaTeX{} provides a mechanism to structure a large document (such as a book)
into a main file and several child files (containing the chapters)
using the |\include| command.
This mechanism is beneficial for documents
which span hundreds of pages in order to
make the source file(s) more manageable.
Moreover, compilation can be restricted to
selected child files by means of the |\includeonly| command.
The latter feature can be used to reduce the compilation time while editing
(this was significantly more useful in the earlier days of \LaTeX{})
or to generate a smaller document which is easier to navigate.
Another application of |\includeonly| is to generate
documents consisting of selected parts of the complete document.

However, there are a few drawbacks of the plain |\include| mechanism:
\begin{itemize}
\item
The child files cannot be compiled on their own,
they can only be compiled via the main file.
A naive editing environment
(such as a text editor with an option
to have the current file processed by \LaTeX)
may require one to switch to the main file before compiling;
attempting to compile the child file produces errors.
\item
The main file must be modified (each time)
to adjust the |\includeonly| command
to the present needs. This easily leaves the main file in a messy state.
\item
The generated document will always carry the filename
of the main document. This is inconvenient if
several child files are to be compiled and
to be kept for distribution.
\end{itemize}

The present package provides a simple interface
to make child files individually compilable by \LaTeX{}.
Compiling a child file then has the same effect as compiling
the main file with an |\includeonly| command
to select the appropriate child.
Moreover the generated document will carry the name of the child
rather than the main file.
This resolves all three above issues.

This feature is meant to make the editing of books,
thesis documents and lecture notes somewhat more convenient.
However, the package can also be used efficiently for
composing a series of documents (such as exercise sheets)
which are typically distributed individually.
It then assists the author in generating the individual documents
(potentially in different versions)
as well as a document containing the collected series.
Another application is in developing style files
or other kinds of included material
where compilation of the style file could redirect
to a sample or test file.

%%%%%%%%%%%%%%%%%%%%%%%%%%%%%%%%%%%%%%%%%%%%%%%%%%%%%%%%%%%%%%%%%%%%%%%%%%%%%%%%
%%%%%%%%%%%%%%%%%%%%%%%%%%%%%%%%%%%%%%%%%%%%%%%%%%%%%%%%%%%%%%%%%%%%%%%%%%%%%%%%
\section{Usage}

First of all, the package \textsf{childdoc} is \emph{not} a standard
\LaTeXe{} |.sty| style file! Therefore it needs to be invoked in
a non-standard way.

%%%%%%%%%%%%%%%%%%%%%%%%%%%%%%%%%%%%%%%%%%%%%%%%%%%%%%%%%%%%%%%%%%%%%%%%%%%%%%%%
\subsection{Included Files}
\label{sec:include}

%%%%%%%%%%%%%%%%%%%%%%%%%%%%%%%%%%%%%%%%
\DescribeMacro{\childdocmain}
To use the package, add the commands
\begin{center}
\begin{tabular}{l}
|\input{childdoc.def}|\\
|\childdocmain{}|\\
\end{tabular}
\end{center}
at the very top of the main \LaTeX{} file,
in particular \emph{before} the |\documentclass| statement!
The argument of |\childdocmain| should be left empty
(but it must be present).

%%%%%%%%%%%%%%%%%%%%%%%%%%%%%%%%%%%%%%%%
\DescribeMacro{\childdocof}
Furthermore, add the commands
\begin{center}
\begin{tabular}{l}
|\input{childdoc.def}|\\
|\childdocof{|\textit{main}|}|\\
\end{tabular}
\end{center}
at the top of every child file \textit{child}
which is included by |\include{|\textit{child}|}|
from within the main file
(or at least for those files to be compiled individually).
The argument \textit{main} must be the filename of the main file.

There are a couple of
considerations in setting up the main and child documents:

%%%%%%%%%%%%%%%%%%%%%%%%%%%%%%%%%%%%%%%%
\paragraph{Restrictions.}

Please note the following restrictions:
\begin{itemize}
\item
|\childdocmain| must be called with one argument \textit{main}
to ensure compatibility with earlier version of the package.
It must either be empty (|\childdocmain{}|)
or precisely match the filename of the main file in which it is specified.
See \secref{sec:detection} for further information.
\item
The filename \textit{main} must be specified without the |.tex| extension.
\item
The filename \textit{main} is case sensitive
(even in case-insensitive file systems)
due to internal string comparison.
\item
The argument \textit{main} should be fully expanded, it cannot be a macro.
\item
Subdirectories and special characters should be avoided in filenames.
\item
The command |\childdocmain{|\textit{main}|}| must be followed by a whitespace.
It should not be followed immediately by another command
or by a comment mark `|%|'.
This is because the \TeX{} parser reads the token immediately following
the argument of |\childdocmain| and puts it
at the beginning of every child section;
however, a white\-space is ignored.
\end{itemize}

%%%%%%%%%%%%%%%%%%%%%%%%%%%%%%%%%%%%%%%%
\paragraph{Content of Main File.}

It is advisable to place all content in the child files included by |\include|.
Any output contained in the main file will appear in all child documents
unless suppressed manually;
it cannot be suppressed automatically by the |\includeonly| directive
and thus should normally be avoided.
A method to include some content in the main file
by means of conditional processing is described in \secref{sec:conditional}.

%%%%%%%%%%%%%%%%%%%%%%%%%%%%%%%%%%%%%%%%
\paragraph{Page Numbering.}

When only a part of the document is compiled,
the appropriate numbering of pages
(as well as other status parameters)
is determined from the |.aux| files.
The latter contain information from previous passes.
However this information needs to propagate through
all intermediate child documents.
Therefore the page numbering in child documents may well
be inconsistent until the complete document is compiled at least once.

A useful (if unconventional) way to always ensure a consistent
page numbering is to restart the numbering in each child document
and denote the pages by `\textit{child}|.|\textit{page}'
where \textit{child} represents the chapter/section number of the child file.
This can be achieved by the command
|\numberwithin{page}{|\textit{child}|}|
of the \textsf{amsmath} package
where \textit{child} can be |chapter| or |section|
depending on the chosen structuring.
Alternatively, one can modify the macro |\thepage| appropriately
and reset the counter |page| at the start of each child file.

%%%%%%%%%%%%%%%%%%%%%%%%%%%%%%%%%%%%%%%%%%%%%%%%%%%%%%%%%%%%%%%%%%%%%%%%%%%%%%%%
\subsection{Conditional Processing}
\label{sec:conditional}

The package provides a mechanism to compile different versions
of a document. To customise the versions further some conditional processing
can come in handy to distinguish which version is being compiled.
The package provides two macros to describe the compilation context:

%%%%%%%%%%%%%%%%%%%%%%%%%%%%%%%%%%%%%%%%
\DescribeMacro{\ifchilddoc}
The conditional |\ifchilddoc| distinguishes between the compilation of
child documents and the main document:
%
\begin{center}
|\ifchilddoc |\textit{child-code}| |[|\||else |\textit{main-code}]| \||fi|
\end{center}

%%%%%%%%%%%%%%%%%%%%%%%%%%%%%%%%%%%%%%%%
\DescribeMacro{\childdocname}
\DescribeMacro{\childdocjob}
The macro |\childdocname| contains the filename (without extension)
of the main or child file being processed.
Note that |\childdocjob| will always contain the name of the main file.

%%%%%%%%%%%%%%%%%%%%%%%%%%%%%%%%%%%%%%%%
\paragraph{Title Page.}

Conditional processing can be used to include a title or banner page
in the main document when proper precautions are taken.
Importantly, the code in the main file should ensure that the page counter
(as well as other status parameters which are stored in the |.aux| files)
takes the same value after the conditional processing.
Otherwise the page numbers may take divergent values
depending on which part is compiled.

For example, a title page could be declared by:
%
\begin{center}
\begin{tabular}{l}
|\ifchilddoc\||else|\\
|\addtocounter{page}{-1}|\\
\textit{code for title page}\\
|\newpage|\\
|\||fi|
\end{tabular}
\end{center}
%
A banner page for the child documents can be generated by:
%
\begin{center}
\begin{tabular}{l}
|\ifchilddoc|\\
|\addtocounter{page}{-1}|\\
\textit{code for banner page}\\
|\newpage|\\
|\||fi|
\end{tabular}
\end{center}
%
Here one could write a message such as:
\begin{center}
|This is the part \childdocname{} of \childdocjob{}.|
\end{center}

%%%%%%%%%%%%%%%%%%%%%%%%%%%%%%%%%%%%%%%%%%%%%%%%%%%%%%%%%%%%%%%%%%%%%%%%%%%%%%%%
\subsection{Flags}
\label{sec:flags}

The package makes it easy to generate different versions
of the main or child documents.
To this end compilation flags can be defined
and assigned different default values.
They will be particularly useful in conjunction
with the forwarding mechanism described in \secref{sec:forward}.

For example, it may be useful to have a flag |\version|
which can be set to |draft| or |final|.
The document source will contain some conditional code
depending on the value of |\version|.
Suppose further, the flag should default to |final| for the main file
and to |draft| for child files
which is a natural assignment for editing the document.
This is achieved by placing the following code
in the preamble of the main document
(below the |\childdocmain| directive):
%
\begin{center}
\begin{tabular}{l}
|\ifchilddoc|\\
|\providecommand{\version}{draft}|\\
|\||else|\\
|\providecommand{\version}{final}|\\
|\||fi|
\end{tabular}
\end{center}
%
The definition by |\providecommand| makes sure
that previous definitions are not overwritten.
Further statements |\providecommand{\version}{...}|
can thus be added before the above code to override it.

For the main file, one might add a line
(between |\childdocmain| and the above block)
%
\begin{center}
|%\ifchilddoc\||else\providecommand{\version}{draft}\||fi|
\end{center}
%
which can be uncommented to produce a draft version.
Likewise one can add a line to the very top of a child file
(above the |\childdocof{|\textit{main}|}| directive)
%
\begin{center}
|%\providecommand{\version}{final}|
\end{center}
%
which can be uncommented to produce the final version of this child document.

%%%%%%%%%%%%%%%%%%%%%%%%%%%%%%%%%%%%%%%%%%%%%%%%%%%%%%%%%%%%%%%%%%%%%%%%%%%%%%%%
\subsection{Forwarding}
\label{sec:forward}

Different versions of the main or child documents
using compilation flags as described in \secref{sec:flags}
can be (permanently) stored in different files
for convenient compilation, viewing and distribution.
To this end, the package defines a command
to pass on compilation to a different file:

%%%%%%%%%%%%%%%%%%%%%%%%%%%%%%%%%%%%%%%%
\DescribeMacro{\childdocforward}
The command |\childdocforward| redirects processing to
another source file:
%
\begin{center}
\begin{tabular}{l}
|\input{childdoc.def}|\\
|\childdocforward[|\textit{main}|]{|\textit{dest}|}|\\
\end{tabular}
\end{center}
%
The argument \textit{dest} is the destination file
(without extension).
It should be the main file or one of the child files.
Note that further \textsf{childdoc} directives
such as |\childdocof| and |\childdocforward|
in the indicated file will be processed in this form.
The optional argument \textit{main}
passes on directly to the main file \textit{main}
while pretending to compile the child \textit{dest}.
This form behaves as if \textit{dest}
issues |\childdocof{|\textit{main}|}| right away,
and no further \textsf{childdoc} directives will be processed.

%%%%%%%%%%%%%%%%%%%%%%%%%%%%%%%%%%%%%%%%
\DescribeMacro{\...prefix}
In the alternative form |\childdocforwardprefix|,
%
\begin{center}
\begin{tabular}{l}
|\input{childdoc.def}|\\
|\childdocforwardprefix[|\textit{main}|]{|\textit{prefix}|}{|\textit{dest}|}|
\end{tabular}
\end{center}
%
the destination file is determined by a pattern
depending on the current file:
To make this work, the current file must be called
`{\textit{prefix}\hspace{0.2em}\textit{suffix}}'
with \textit{prefix} matching precisely the argument.
Processing is then passed on to the file
`{\textit{dest}\hspace{0.2em}\textit{suffix}}'.
Surely, the same effect is achieved by
directly specifying the
argument `{\textit{dest}\hspace{0.2em}\textit{suffix}}'
in the first form.
However, that requires to set up a different file
for each child. With the alternative form of the command
all these files can have exactly the same content
which simplifies setting them up and maintaining them.

For example, the following file |draft.tex|
with a compilation flag |\version| as described in \secref{sec:flags}
compiles the main document as a draft:
%
\begin{center}
\begin{tabular}{l}
|\def\version{draft}|\\
|\input{childdoc.def}|\\
|\childdocforward{|\textit{main}|}|
\end{tabular}
\end{center}
%
Likewise, the following files |final|\textit{nn}|.tex|
compile the final version of the child document
|child|\textit{nn}|.tex|:
%
\begin{center}
\begin{tabular}{l}
|\def\version{final}|\\
|\input{childdoc.def}|\\
|\childdocforwardprefix{final}{child}|
\end{tabular}
\end{center}
%

Note that when several versions of a main file and/or of each child file
are to be generated, it may be convenient to set up a |Makefile| or
shell script to automatise the process.

%%%%%%%%%%%%%%%%%%%%%%%%%%%%%%%%%%%%%%%%%%%%%%%%%%%%%%%%%%%%%%%%%%%%%%%%%%%%%%%%
\subsection{Command Line Processing}
\label{sec:commandline}

The effect of redirection files can also be achieved by invoking
the \LaTeX{} compiler with a more elaborate command line.
Most conveniently this should be done as part
of a shell script or a |Makefile|.

When using \textsf{childdoc} in the main file, the following
command lines effectively perform a redirection
(note that depending on the shell being used,
backslashes may have to be doubled: `|\|' $\to$ `|\\|'):
%
\begin{center}
|... -jobname "|\textit{target}|" |\\|"|[\textit{flags}]%
|\input{childdoc.def}\childdocforward[|\textit{main}|]{|\textit{dest}|}"|
\end{center}
%
Here \textit{target} is the name of the output file,
\textit{main} is the name of the main file
and \textit{dest} is the name of the main or child file to be processed
(all filenames without extensions).
The optional argument \textit{main} can be omitted
if \textit{main} matches \textit{dest}.
Optionally, compilation \textit{flags} can be defined via |\def| commands.
This command line makes the \TeX{} engine believe
it is compiling the file \textit{target}
whose content is specified as the latter parameter.
The provided code then forwards the processing to
\textit{main} or \textit{dest} as described in \secref{sec:forward}.

%%%%%%%%%%%%%%%%%%%%%%%%%%%%%%%%%%%%%%%%%%%%%%%%%%%%%%%%%%%%%%%%%%%%%%%%%%%%%%%%
\subsection{Include by Input}
\label{sec:input}

Including child documents by |\include| has some restrictions by design.
Most notably, the content of a child document always occupies
its own set of pages; pages cannot be shared between child documents.
Usually, this behaviour makes perfect sense
because each child document contain an essential part of the document.
However, in some situations it may be desirable to compose
a document from a collection of parts
without having mandatory page breaks between then.
For this case, the package
provides a mechanism to include parts
by |\input| which can also be processed individually.
However, by construction this mechanism
requires manual handling of the content to be output.

%%%%%%%%%%%%%%%%%%%%%%%%%%%%%%%%%%%%%%%%
\DescribeMacro{\ifchilddocmanual}
The main file should be prepared as usual, see \secref{sec:include}.
However, the document body must make a distinction
between processing of an individual part and of the main document, e.g.:
%
\begin{center}
\begin{tabular}{l}
|\ifchilddocmanual|\\
|\input{\childdocname}|\\
|\||else|\\
\textit{document body with }|\input{|\textit{part}|}|\\
|\||fi|
\end{tabular}
\end{center}
%
The conditional |\ifchilddocmanual| is true whenever
a part to be included by |\input| is being compiled,
and the name of the part is stored in |\childdocname|.

%%%%%%%%%%%%%%%%%%%%%%%%%%%%%%%%%%%%%%%%
\DescribeMacro{\childdocby}
Each part to be included by |\input| should start with:
%
\begin{center}
\begin{tabular}{l}
|\input{childdoc.def}|\\
|\childdocby{|\textit{main}|}|\\
\end{tabular}
\end{center}
%
The directive |\childdocby| is similar to |\childdocof|
described in \secref{sec:include},
but the subsequent selection of content must be done manually.
To that end, both |\ifchilddoc| and |\ifchilddocmanual|
will be true upon processing of a part,
and the name of the part is stored in |\childdocname|.
Note that |\jobname| will be set to the filename of the current part
so that each part receives an individual |.aux| file
that does not interfere with the |.aux| file(s) of the main document.
This behaviour can be altered by the alternative form
|\childdocby[*]{|\textit{main}|}| (with a non-empty optional argument)
which uses the |.aux| file of the main document
by setting |\jobname| to \textit{main}.

%%%%%%%%%%%%%%%%%%%%%%%%%%%%%%%%%%%%%%%%%%%%%%%%%%%%%%%%%%%%%%%%%%%%%%%%%%%%%%%%
\subsection{Driver Development}
\label{sec:driver}

The \textsf{childdoc} mechanism can also be use for the development
of definition files such as \LaTeX{} styles or classes.
This case differs from the above setup with multiple parts
included by |\include| in that no |\includeonly| should be invoked.
This can be achieved by starting the include file
(before |\ProvidesPackage|) with:
%
\begin{center}
\begin{tabular}{l}
|\input{childdoc.def}|\\
|\childdocforward{|\textit{main}|}|\\
\end{tabular}
\end{center}
%
or alternatively with:
%
\begin{center}
\begin{tabular}{l}
|\input{childdoc.def}|\\
|\childdocby{|\textit{main}|}|\\
\end{tabular}
\end{center}
%
Both forms have slightly different effects as described above.
The main file is prepared as usual, see \secref{sec:include}.

%%%%%%%%%%%%%%%%%%%%%%%%%%%%%%%%%%%%%%%%%%%%%%%%%%%%%%%%%%%%%%%%%%%%%%%%%%%%%%%%
\subsection{Legacy Detection}
\label{sec:detection}

The directive |\childdocmain| in the main file can detect
whether the complete document or merely a child is to be compiled
even without using the directive |\childdocof|.
This method is deprecated because it is less robust
and there is no compelling reason to use it;
it is merely provided for backward compatibility
and it may be removed in future versions.

If the detection mechanism is to be used,
it is mandatory to correctly specify
the filename of the main file as the argument of |\childdocmain|:
%
\begin{center}
\begin{tabular}{l}
|\input{childdoc.def}|\\
|\childdocmain{|\textit{main}|}|\\
\end{tabular}
\end{center}
%
If |\jobname| does not match the argument \textit{main} of |\childdocmain|,
it is assumed that |\jobname| points to the child file to be compiled.
When using |\childdocmain| with the main file specified as argument,
it suffices to start a child file
with just |\input{|\textit{main}|}|
without loading of the package and using |\childdocof|.
If instead all processing is done
with the appropriate \textsf{childdoc} directives,
the argument of \textit{main} of |\childdocmain| can be empty.

An alternative version of the command line processing described
in \secref{sec:commandline} using the detection mechanism reads:
%
\begin{center}
|... -jobname "|\textit{target}|" "|[\textit{flags}]%
[|\def\jobname{|\textit{dest}|}|]|\input{|\textit{main}|}"|
\end{center}

%%%%%%%%%%%%%%%%%%%%%%%%%%%%%%%%%%%%%%%%%%%%%%%%%%%%%%%%%%%%%%%%%%%%%%%%%%%%%%%%
\subsection{Manual Code}
\label{sec:manual}

In case one cannot be certain whether the definitions file |childdoc.def|
is installed on the target \TeX{} distribution
and one prefers not to ship it,
it is conceivable to paste a few relevant commands into the sources.

To that end, drop all statements |\input{childdoc.def}|
and perform the replacements as outlined below.
Instead of |\childdocmain{|\textit{main}|}| add the following code
to the top of the main file:
%
\begin{center}
\begin{tabular}{l}
|\||ifdefined\childdocname\endinput\||fi\newif\ifchilddoc|\\
|\edef\childdocname{\scantokens\expandafter{\jobname\noexpand}}|\\
|\def\childdocmain{|\textit{main}|}\||ifx\childdocmain\childdocname\||else|\\
|\childdoctrue\includeonly{\childdocname}\let\jobname\childdocmain\||fi|\\
\end{tabular}
\end{center}
%
Instead of |\childdocof{|\textit{main}|}| just include the main file
at the top of each child file:
%
\begin{center}
|\input{|\textit{main}|}|
\end{center}
%
A simple redirection |\childdocforward{|\textit{dest}|}| is achieved by:
%
\begin{center}
|\def\jobname{|\textit{dest}|}\input{\jobname}|
\end{center}
%
The redirection with prefix
|\childdocforwardprefix[|\textit{prefix}|]{|\textit{dest}|}|
is accomplished by:
%
\begin{center}
\begin{tabular}{l}
|{\edef\jobname{\scantokens\expandafter{\jobname\noexpand}}|\\
|\def\redirectjob |\textit{prefix}|#1~~~{\gdef\jobname{|\textit{dest}|#1}}|\\
|\expandafter\redirectjob\jobname~~~}\input{\jobname}|
\end{tabular}
\end{center}

In an alternative approach,
child documents can be compiled by a specific command line
without additional code or specific definitions:
%
\begin{center}
|... -jobname "|\textit{target}|" "|[\textit{flags}]%
|\includeonly{|\textit{dest}|}\input{|\textit{main}|}"|
\end{center}
%

%%%%%%%%%%%%%%%%%%%%%%%%%%%%%%%%%%%%%%%%%%%%%%%%%%%%%%%%%%%%%%%%%%%%%%%%%%%%%%%%
%%%%%%%%%%%%%%%%%%%%%%%%%%%%%%%%%%%%%%%%%%%%%%%%%%%%%%%%%%%%%%%%%%%%%%%%%%%%%%%%
\section{Information}

%%%%%%%%%%%%%%%%%%%%%%%%%%%%%%%%%%%%%%%%%%%%%%%%%%%%%%%%%%%%%%%%%%%%%%%%%%%%%%%%
\subsection{Copyright}

Copyright \copyright{} 2017--2018 Niklas Beisert

This work may be distributed and/or modified under the
conditions of the \LaTeX{} Project Public License, either version 1.3
of this license or (at your option) any later version.
The latest version of this license is in
  \url{http://www.latex-project.org/lppl.txt}
and version 1.3 or later is part of all distributions of \LaTeX{}
version 2005/12/01 or later.

This work has the LPPL maintenance status `maintained'.

The Current Maintainer of this work is Niklas Beisert.

This work consists of the files |README.txt|, |childdoc.ins| and |childdoc.dtx|
as well as the derived files |childdoc.def|, |cdocsamp.tex|
with |cdocsch1.tex|, |cdocsch2.tex|, |cdocspt3.tex|, |cdocspt4.tex|,
|cdocsdrf.tex|, |cdocsfn1.tex|, |cdocsfn2.tex|
as well as |childdoc.pdf|.

%%%%%%%%%%%%%%%%%%%%%%%%%%%%%%%%%%%%%%%%%%%%%%%%%%%%%%%%%%%%%%%%%%%%%%%%%%%%%%%%
\subsection{Files and Installation}

The package consists of the files:
%
\begin{center}
\begin{tabular}{ll}
    |README.txt|   & readme file \\
    |childdoc.ins| & installation file \\
    |childdoc.dtx| & source file \\
    |childdoc.def| & definition file \\
    |cdocsamp.tex| & sample main file \\
    |cdocsch1.tex| & sample include file \\
    |cdocsch2.tex| & sample include file \\
    |cdocspt3.tex| & sample part file \\
    |cdocspt4.tex| & sample part file \\
    |cdocsdrf.tex| & sample redirection file \\
    |cdocsfn1.tex| & sample redirection file \\
    |cdocsfn2.tex| & sample redirection file \\
    |childdoc.pdf| & manual
\end{tabular}
\end{center}
%
The distribution consists of the files
|README.txt|, |childdoc.ins| and |childdoc.dtx|.
%
\begin{itemize}
\item
Run (pdf)\LaTeX{} on |childdoc.dtx|
to compile the manual |childdoc.pdf| (this file).
\item
Run \LaTeX{} on |childdoc.ins| to create the definitions file |childdoc.def|
and the sample |cdocsamp.tex| with include files
|cdocsch1.tex|, |cdocsch2.tex|, |cdocspt3.tex|, |cdocspt4.tex|,
|cdocsdrf.tex|, |cdocsfn1.tex|, |cdocsfn2.tex|.
Then copy the file |childdoc.def| to an appropriate directory of your \LaTeX{}
distribution, e.g.\ \textit{texmf-root}|/tex/latex/childdoc|.
\end{itemize}

%%%%%%%%%%%%%%%%%%%%%%%%%%%%%%%%%%%%%%%%%%%%%%%%%%%%%%%%%%%%%%%%%%%%%%%%%%%%%%%%
\subsection{Related CTAN Packages}

There are several other packages which offer a similar functionality:
%
\begin{itemize}
\item
The packages
\href{http://ctan.org/pkg/docmute}{\textsf{docmute}},
\href{http://ctan.org/pkg/includex}{\textsf{includex}} and
\href{http://ctan.org/pkg/standalone}{\textsf{standalone}}
provide commands to include only the document body of
a child file thus allowing both files to be compiled individually.
\item
The packages \href{http://ctan.org/pkg/subdocs}{\textsf{subdocs}}
and \href{http://ctan.org/pkg/subfiles}{\textsf{subfiles}}
provide structures in which the main and child documents can be
encapsulated and allowing them to be compiled individually.
The inclusion mechanism is different from the conventional |\include|.
\item
The package \href{http://ctan.org/pkg/combine}{\textsf{combine}}
is an elaborate solution to combine several documents into one.
\end{itemize}
%
See also the CTAN topic \href{http://ctan.org/topic/subdocs}{\textsf{subdocs}}
for further related packages.
The present package differs from the above solutions in that
a document structure constructed with the conventional |\include| mechanism
just needs two extra commands at the top of every file
such that all constituent files can be compiled individually.

%%%%%%%%%%%%%%%%%%%%%%%%%%%%%%%%%%%%%%%%%%%%%%%%%%%%%%%%%%%%%%%%%%%%%%%%%%%%%%%%
%\subsection{Feature Suggestions}
%
%The following is a list of features which may be useful for future
%versions of this package:
%%
%\begin{itemize}
%\item
%\ldots
%\end{itemize}

%%%%%%%%%%%%%%%%%%%%%%%%%%%%%%%%%%%%%%%%%%%%%%%%%%%%%%%%%%%%%%%%%%%%%%%%%%%%%%%%
\subsection{Revision History}

%%%%%%%%%%%%%%%%%%%%%%%%%%%%%%%%%%%%%%%%
\paragraph{v2.0:} 2018/12/30

\begin{itemize}
\item
immediate forward processing
\item
added |\childdocby| mechanism
\item
manual restructured
\end{itemize}

%%%%%%%%%%%%%%%%%%%%%%%%%%%%%%%%%%%%%%%%
\paragraph{v1.6:} 2018/01/17

\begin{itemize}
\item
application for development of include files
\item
corrections to manual
\end{itemize}

%%%%%%%%%%%%%%%%%%%%%%%%%%%%%%%%%%%%%%%%
\paragraph{v1.5:} 2017/05/21

\begin{itemize}
\item
more complete structuring introduced
\item
|\childdocof| introduced
\item
|\childdoc| renamed to |\childdocmain|
\item
|\childredirect| renamed to |\childdocforward| and |\childdocforwardprefix|
and functionality expanded
\end{itemize}

%%%%%%%%%%%%%%%%%%%%%%%%%%%%%%%%%%%%%%%%
\paragraph{v1.0:} 2017/04/27

\begin{itemize}
\item
manual and install package
\item
first version published on CTAN
\end{itemize}

%%%%%%%%%%%%%%%%%%%%%%%%%%%%%%%%%%%%%%%%
\paragraph{v0.6:} 2017/04/26

\begin{itemize}
\item
redirection mechanism added
\end{itemize}

%%%%%%%%%%%%%%%%%%%%%%%%%%%%%%%%%%%%%%%%
\paragraph{v0.5:} 2017/04/26

\begin{itemize}
\item
functionality in definition file
\end{itemize}


%%%%%%%%%%%%%%%%%%%%%%%%%%%%%%%%%%%%%%%%%%%%%%%%%%%%%%%%%%%%%%%%%%%%%%%%%%%%%%%%
%%%%%%%%%%%%%%%%%%%%%%%%%%%%%%%%%%%%%%%%%%%%%%%%%%%%%%%%%%%%%%%%%%%%%%%%%%%%%%%%
%%%%%%%%%%%%%%%%%%%%%%%%%%%%%%%%%%%%%%%%%%%%%%%%%%%%%%%%%%%%%%%%%%%%%%%%%%%%%%%%
\appendix

\settowidth\MacroIndent{\rmfamily\scriptsize 000\ }

 \DocInput{childdoc.dtx}

\end{document}
%</driver>
% \fi
%
% %%%%%%%%%%%%%%%%%%%%%%%%%%%%%%%%%%%%%%%%%%%%%%%%%%%%%%%%%%%%%%%%%%%%%%%%%%%%%%
% %%%%%%%%%%%%%%%%%%%%%%%%%%%%%%%%%%%%%%%%%%%%%%%%%%%%%%%%%%%%%%%%%%%%%%%%%%%%%%
% \section{Sample}
%\iffalse
%<*samplemain>
%\fi
%
% The following presents a sample document
% with two chapters, two parts, a title page,
% a compile flag as well as three forwarding files to set the flag.
% It consists of eight |.tex| files:
% \begin{center}
% \begin{tabular}{ll}
% |cdocsamp.tex|&main file\\
% |cdocsch1.tex|&include file for chapter 1\\
% |cdocsch2.tex|&include file for chapter 2\\
% |cdocspt3.tex|&include file for part 3\\
% |cdocspt4.tex|&include file for part 4\\
% |cdocsdrf.tex|&forwarding file for main file in draft mode\\
% |cdocsfi1.tex|&forwarding file for final version of chapter 1\\
% |cdocsfi2.tex|&forwarding file for final version of chapter 2\\
% \end{tabular}
% \end{center}
% Each of the eight files can be compiled directly by the \LaTeX{} compiler.
%
% %%%%%%%%%%%%%%%%%%%%%%%%%%%%%%%%%%%%%%
% \paragraph{Main File.}
%
% The main file is called |cdocsamp.tex|.
%
% Load the \textsf{childdoc} definitions and
% declare the filename for the main document:
%    \begin{macrocode}
\input{childdoc.def}
\childdocmain{}
%    \end{macrocode}

% Optional override for |\version| flag:
%    \begin{macrocode}
%%\ifchilddoc\else\providecommand{\version}{draft}\fi
%    \end{macrocode}

% Define the default values for the |\version| flag
% (|final| for the main file and |draft| for childs):
%    \begin{macrocode}
\ifchilddoc
\providecommand{\version}{draft}
\else
\providecommand{\version}{final}
\fi
%    \end{macrocode}

% Load the standard document class:
%    \begin{macrocode}
\documentclass[12pt]{article}
%    \end{macrocode}

% Start the document body:
%    \begin{macrocode}
\begin{document}
%    \end{macrocode}

% Declare a title page.
% Print title, part of document being processed and version flag:
%    \begin{macrocode}
\addtocounter{page}{-1}
\begin{center}
{\LARGE\bfseries{}childdoc example\par}
\vspace{1cm}
\ifchilddoc
\ifchilddocmanual part\else chapter\fi:
`\childdocname' of `\childdocjob'\par
\else
main document: `\childdocjob'\par
\fi
version: \version\par
\end{center}
\newpage
%    \end{macrocode}

% Manually include selected file,
% otherwise process as usual:
%    \begin{macrocode}
\ifchilddocmanual
\section*{part `\childdocname'}
\input{\childdocname}
\else
%    \end{macrocode}

% Include the two chapters:
%    \begin{macrocode}
\include{cdocsch1}
\include{cdocsch2}
%    \end{macrocode}

% Include the two parts unless only chapters should be displayed:
%    \begin{macrocode}
\ifchilddoc\else
\section{part three}
\input{cdocspt3}
\section{part four}
\input{cdocspt4}
\fi
%    \end{macrocode}

% Process as usual until here:
%    \begin{macrocode}
\fi
%    \end{macrocode}

% End of document body:
%    \begin{macrocode}
\end{document}
%    \end{macrocode}
%\iffalse
%</samplemain>
%\fi
%
% %%%%%%%%%%%%%%%%%%%%%%%%%%%%%%%%%%%%%%
% \paragraph{Chapter Include Files.}
%
% The include files are called |cdocsch1.tex| and |cdocsch2.tex|.
%
%\iffalse
%<*samplechap1|samplechap2>
%\fi

% Optional override for |\version| flag:
%    \begin{macrocode}
%%\providecommand{\version}{final}
%    \end{macrocode}

% Include the main document:
%    \begin{macrocode}
\input{childdoc.def}
\childdocof{cdocsamp}
%    \end{macrocode}

%\iffalse
%</samplechap1|samplechap2>
%\fi
%
%\iffalse
%<*samplechap1>
%\fi
% Some text for chapter 1:
%    \begin{macrocode}
\section{one}
some text in chapter one
%    \end{macrocode}

%\iffalse
%</samplechap1>
%\fi
% Some text for chapter 2:
%\iffalse
%<*samplechap2>
%\fi
%    \begin{macrocode}
\section{two}
more text in chapter two
%    \end{macrocode}

%\iffalse
%</samplechap2>
%\fi
%
% %%%%%%%%%%%%%%%%%%%%%%%%%%%%%%%%%%%%%%
% \paragraph{Part Include Files.}
%
% The include files are called |cdocspt3.tex| and |cdocspt4.tex|.
%
%\iffalse
%<*samplepart3|samplepart4>
%\fi

% Optional override for |\version| flag:
%    \begin{macrocode}
%%\providecommand{\version}{final}
%    \end{macrocode}

% Include the main document:
%    \begin{macrocode}
\input{childdoc.def}
\childdocby{cdocsamp}
%    \end{macrocode}

%\iffalse
%</samplepart3|samplepart4>
%\fi
%
%\iffalse
%<*samplepart3>
%\fi
% Some text for part 3:
%    \begin{macrocode}
some text in part three
%    \end{macrocode}

%\iffalse
%</samplepart3>
%\fi
% Some text for part 4:
%\iffalse
%<*samplepart4>
%\fi
%    \begin{macrocode}
more text in part four
%    \end{macrocode}

%\iffalse
%</samplepart4>
%\fi
%
% %%%%%%%%%%%%%%%%%%%%%%%%%%%%%%%%%%%%%%
% \paragraph{Forwarding for a Complete Draft.}
%
% The following forwarding file |cdocsdrf.tex|
% compiles the main document in draft mode:
%\iffalse
%<*sampledraft>
%\fi
%    \begin{macrocode}
\def\version{draft}
\input{childdoc.def}
\childdocforward{cdocsamp}
%    \end{macrocode}

%\iffalse
%</sampledraft>
%\fi
%
% %%%%%%%%%%%%%%%%%%%%%%%%%%%%%%%%%%%%%%
% \paragraph{Forwarding for Final Version of the Chapters.}
%
% The following forwarding files |cdocsfn1.tex| and |cdocsfn2.tex|
% (with identical content)
% compile the final versions of the child documents
% |cdocsch1.tex| and |cdocsch2.tex|, respectively:
%\iffalse
%<*samplefinal>
%\fi
%    \begin{macrocode}
\def\version{final}
\input{childdoc.def}
\childdocforwardprefix[cdocsamp]{cdocsfn}{cdocsch}
%    \end{macrocode}

%\iffalse
%</samplefinal>
%\fi
%
% %%%%%%%%%%%%%%%%%%%%%%%%%%%%%%%%%%%%%%
% \paragraph{Command Line Processing.}
%
% The following three command lines generate the output files
% |cdocscld|, |cdocscl1| and |cdocscl2|
% which should be identical to
% |cdocsdrf|, |cdocsch1| and |cdocsfn2|, respectively:
% \begin{center}
% \begin{tabular}{l}
% |latex -jobname cdocscld \|\\
% |  "\def\version{draft}\input{childdoc.def}\childdocforward{cdocsamp}"|\\
% |latex -jobname cdocscl1 \|\\
% |  "\input{childdoc.def}\childdocforward[cdocsamp]{cdocsch1}"|\\
% |latex -jobname cdocscl2 \|\\
% |  "\def\version{final}\input{childdoc.def}\childdocforward{cdocsch2}"|
% \end{tabular}
% \end{center}
% Note that the trailing backslash on each first line
% merely continues the input to the second line
% (for convenient cut ant paste).
% Furthermore, the command |latex| can be replaced by any
% of its alternative versions such as |pdflatex|.
%
% %%%%%%%%%%%%%%%%%%%%%%%%%%%%%%%%%%%%%%%%%%%%%%%%%%%%%%%%%%%%%%%%%%%%%%%%%%%%%%
% %%%%%%%%%%%%%%%%%%%%%%%%%%%%%%%%%%%%%%%%%%%%%%%%%%%%%%%%%%%%%%%%%%%%%%%%%%%%%%
% \section{Implementation}
%\iffalse
%<*package>
%\fi
%
% This section describes the definitions file |childdoc.def|.

% The definitions cannot be loaded using |\usepackage| or |\RequirePackage|
% which has a mechanism to prevent loading a style file more than once.
% When loading the definitions by means of |\input|
% multiple instances have to be prevented manually:
%\iffalse
%This code needs to be before the `\ProvidesFile' directive
%which is defined at the beginning of this file.
%Therefore it is also placed there and commented out here.
%</package>
%<*discard>
%\fi
%    \begin{macrocode}
\ifdefined\childdocmain\endinput\fi
%    \end{macrocode}
%\iffalse
%</discard>
%<*package>
%\fi
%
% \macro{\ifchilddoc}
% \macro{\ifchilddocmanual}
% The conditional |\ifchilddoc| tells whether a
% child (true) or main (false) document is being compiled.
% The conditional |\ifchilddocmanual| tells whether
% the |\includeonly| mechanism is used (false) or
% the selection of child files must be performed manually (true).
% The definitions initialise to false:
%    \begin{macrocode}
\newif\ifchilddoc
\newif\ifchilddocmanual
%    \end{macrocode}

% \macro{\childdocname}
% \macro{\childdocjob}
% The macro |\childdocname| stores the name of the main document
% to be compiled. The macro |\childdocjob| stores the name of
% the document on which the \LaTeX{} compiler was originally invoked.
% The content of |\jobname| cannot be compared
% to filenames specified in the source due to different catcodes.
% The following code rescans |\jobname|, stores the result
% in |\childdocname| and saves a copy in |\childdocjob|:
%    \begin{macrocode}
\edef\childdocname{\scantokens\expandafter{\jobname\noexpand}}
\let\childdocjob\childdocname
%    \end{macrocode}

% \macro{\childdocdisable}
% The macro |\childdocdisable| prevents the main file
% from being processed more than once.
% At this stage, the main document command |\childdocmain|
% is assumed to be called once again where it should do nothing.
% Any subsequent call to it should prevent
% a secondary processing of the main document
% It overwrites the forwarding commands
% |\childdocof| and |\childdocforward|
% with empty macros to prevent further inclusions of the main document:
%    \begin{macrocode}
\newcommand{\childdocdisable}
{
  \renewcommand{\childdocmain}[1]{\renewcommand{\childdocmain}[1]{\endinput}}
  \renewcommand{\childdocof}[1]{}
  \renewcommand{\childdocby}[2][]{}
  \renewcommand{\childdocforward}[2][]{}
  \renewcommand{\childdocdisable}{}
}
%    \end{macrocode}

% \macro{\childdocmain}
% The macro |\childdocmain| is to be called at the top of the main file
% with nothing or the main filename (without extension) as argument.
% First, it breaks loops.
% If the argument is not empty and does not match |\childdocname|
% (which is set by the first inclusion of |childdoc.def|),
% |\ifchilddoc| is set to true, |\includeonly| is applied to the child file
% and |\jobname| is set to the main file
% (for proper handling of |.aux| files):
%    \begin{macrocode}
\newcommand{\childdocmain}[1]
{
  \childdocdisable\childdocmain{}
  \if?#1?\else
    \begingroup
      \def\childdoctmp{#1}
      \ifx\childdoctmp\childdocname
        \def\childdoctmp{}
      \else
        \def\childdoctmp
        {
          \childdoctrue
          \includeonly{\childdocname}
          \def\childdocjob{#1}
          \def\jobname{#1}
        }
      \fi
      \expandafter
    \endgroup
    \childdoctmp
  \fi
}
%    \end{macrocode}

% \macro{\childdocof}
% The command |\childdocof| redirects
% compilation to the main file |#1|.
%    \begin{macrocode}
\newcommand{\childdocof}[1]
{
  \childdocdisable
  \childdoctrue
  \includeonly{\childdocname}
  \def\jobname{#1}
  \def\childdocjob{#1}
  \input{#1}
}
%    \end{macrocode}

% \macro{\childdocby}
% The command |\childdocby| ....
%    \begin{macrocode}
\newcommand{\childdocby}[2][]
{
  \childdocdisable
  \childdoctrue
  \childdocmanualtrue
  \if?#1?\else
    \def\jobname{#2}
  \fi
  \def\childdocjob{#2}
  \input{#2}
  \endinput
}
%    \end{macrocode}

% \macro{\childdocforward}
% The command |\childdocforward| redirects
% compilation to the main file or
% (if the optional argument is given) a child file.
% Parameters are set as if the main file
% or a child file starting with |\childdocof| was compiled.
% Then compilation is handed over to the main file:
%    \begin{macrocode}
\newcommand{\childdocforward}[2][]
{
  \begingroup
    \if?#1?
      \def\childdoctmp
      {
        \def\childdocname{#2}
        \def\childdocjob{#2}
        \def\jobname{#2}
        \input{#2}
        \endinput
      }
    \else
      \def\childdoctmp
      {
        \childdocdisable
        \def\childdocname{#2}
        \childdoctrue
        \includeonly{#2}
        \def\childdocjob{#1}
        \def\jobname{#1}
        \input{#1}
        \endinput
      }
    \fi
    \expandafter
  \endgroup
  \childdoctmp
}
%    \end{macrocode}

% \macro{\childdocforwardprefix}
% The command |\childdocforwardprefix| redirects
% compilation to the main or a child file by means of a pattern.
% The prefix |#1| in the current filename is replaced by |#2|
% and the suffix of the current filename is kept
% (it is assumed that the filename does not contain the substring `|~~~|'
% which is used as a delimiter).
% Compilation is handed over to the new file by |\childdocforward|:
%    \begin{macrocode}
\newcommand{\childdocforwardprefix}[3][]
{
  \begingroup
    \def\childdocextract #2##1~~~{\def\childdoctmp{\childdocforward[#1]{#3##1}}}
    \expandafter\childdocextract\childdocname~~~
    \expandafter
  \endgroup
  \childdoctmp
}
%    \end{macrocode}

% \macro{\childdoc}
% The deprecated macro |\childdoc| is a legacy version of |\childdocmain|:
%    \begin{macrocode}
\newcommand{\childdoc}{\childdocmain}
%    \end{macrocode}

% \macro{\childdocredirect}
% The deprecated macro |\childdocredirect| is a legacy version
% of |\childdocforward| and |\childdocforwardprefix|:
%    \begin{macrocode}
\newcommand{\childdocredirect}[2][]
{
  \begingroup
    \if?#1?
      \def\childdoctmp{\childdocforward{#2}}
    \else
      \def\childdoctmp{\childdocforwardprefix{#1}{#2}}
    \fi
    \expandafter
  \endgroup
  \childdoctmp
}
%    \end{macrocode}

%\iffalse
%</package>
%\fi
%
\endinput
\childdocforward[cdocsamp]{cdocsch1}"|\\
% |latex -jobname cdocscl2 \|\\
% |  "\def\version{final}% \iffalse
%
% childdoc.dtx Copyright (C) 2017-2018 Niklas Beisert
%
% This work may be distributed and/or modified under the
% conditions of the LaTeX Project Public License, either version 1.3
% of this license or (at your option) any later version.
% The latest version of this license is in
%   http://www.latex-project.org/lppl.txt
% and version 1.3 or later is part of all distributions of LaTeX
% version 2005/12/01 or later.
%
% This work has the LPPL maintenance status `maintained'.
%
% The Current Maintainer of this work is Niklas Beisert.
%
% This work consists of the files childdoc.dtx and childdoc.ins
% and the derived files childdoc.def and cdocsamp.tex with
% cdocsch1.tex, cdocsch2.tex, cdocsdrf.tex, cdocsfn1.tex, cdocsfn2.tex.
%
%<package>\ifdefined\childdocmain\endinput\fi
%<package>\ProvidesFile{childdoc.def}[2018/12/30 v2.0 child document driver]
%<samplemain>\ProvidesFile{cdocsamp.tex}[2018/12/30 v2.0 sample for childdoc]
%<*driver>
%\ProvidesFile{childdoc.drv}[2018/12/30 v2.0 childdoc reference manual file]
\PassOptionsToClass{10pt,a4paper}{article}
\documentclass{ltxdoc}

\usepackage[margin=35mm]{geometry}
\usepackage{hyperref}
\usepackage{hyperxmp}
\usepackage[usenames]{color}

\hypersetup{colorlinks=true}
\hypersetup{pdfstartview=FitH}
\hypersetup{pdfpagemode=UseNone}
\hypersetup{pdfsource={}}
\hypersetup{pdflang={en-UK}}
\hypersetup{pdfcopyright={Copyright 2017-2018 Niklas Beisert.
  This work may be distributed and/or modified under the
  conditions of the LaTeX Project Public License, either version 1.3
  of this license or (at your option) any later version.}}
\hypersetup{pdflicenseurl={http://www.latex-project.org/lppl.txt}}
\hypersetup{pdfcontactaddress={ETH Zurich, ITP, HIT K,
  Wolfgang-Pauli-Strasse 27}}
\hypersetup{pdfcontactpostcode={8093}}
\hypersetup{pdfcontactcity={Zurich}}
\hypersetup{pdfcontactcountry={Switzerland}}
\hypersetup{pdfcontactemail={nbeisert@itp.phys.ethz.ch}}
\hypersetup{pdfcontacturl={http://people.phys.ethz.ch/\xmptilde nbeisert/}}

\newcommand{\secref}[1]{\hyperref[#1]{section \ref*{#1}}}

\parskip1ex
\parindent0pt
\let\olditemize\itemize
\def\itemize{\olditemize\parskip0pt}

\begin{document}

\title{The \textsf{childdoc} Package}
\hypersetup{pdftitle={The childdoc Package}}
\author{Niklas Beisert\\[2ex]
  Institut f\"ur Theoretische Physik\\
  Eidgen\"ossische Technische Hochschule Z\"urich\\
  Wolfgang-Pauli-Strasse 27, 8093 Z\"urich, Switzerland\\[1ex]
  \href{mailto:nbeisert@itp.phys.ethz.ch}
  {\texttt{nbeisert@itp.phys.ethz.ch}}}
\hypersetup{pdfauthor={Niklas Beisert}}
\hypersetup{pdfsubject={Manual for the LaTeX2e Package childdoc}}
\date{30 December 2018, \textsf{v2.0}}
\maketitle

\begin{abstract}\noindent
\textsf{childdoc} is a \LaTeXe{} package
that enables the direct compilation
of document sections included by |\include|
to individual files.
\end{abstract}

\begingroup
\parskip0ex
\tableofcontents
\endgroup

%%%%%%%%%%%%%%%%%%%%%%%%%%%%%%%%%%%%%%%%%%%%%%%%%%%%%%%%%%%%%%%%%%%%%%%%%%%%%%%%
%%%%%%%%%%%%%%%%%%%%%%%%%%%%%%%%%%%%%%%%%%%%%%%%%%%%%%%%%%%%%%%%%%%%%%%%%%%%%%%%
\section{Introduction}

\LaTeX{} provides a mechanism to structure a large document (such as a book)
into a main file and several child files (containing the chapters)
using the |\include| command.
This mechanism is beneficial for documents
which span hundreds of pages in order to
make the source file(s) more manageable.
Moreover, compilation can be restricted to
selected child files by means of the |\includeonly| command.
The latter feature can be used to reduce the compilation time while editing
(this was significantly more useful in the earlier days of \LaTeX{})
or to generate a smaller document which is easier to navigate.
Another application of |\includeonly| is to generate
documents consisting of selected parts of the complete document.

However, there are a few drawbacks of the plain |\include| mechanism:
\begin{itemize}
\item
The child files cannot be compiled on their own,
they can only be compiled via the main file.
A naive editing environment
(such as a text editor with an option
to have the current file processed by \LaTeX)
may require one to switch to the main file before compiling;
attempting to compile the child file produces errors.
\item
The main file must be modified (each time)
to adjust the |\includeonly| command
to the present needs. This easily leaves the main file in a messy state.
\item
The generated document will always carry the filename
of the main document. This is inconvenient if
several child files are to be compiled and
to be kept for distribution.
\end{itemize}

The present package provides a simple interface
to make child files individually compilable by \LaTeX{}.
Compiling a child file then has the same effect as compiling
the main file with an |\includeonly| command
to select the appropriate child.
Moreover the generated document will carry the name of the child
rather than the main file.
This resolves all three above issues.

This feature is meant to make the editing of books,
thesis documents and lecture notes somewhat more convenient.
However, the package can also be used efficiently for
composing a series of documents (such as exercise sheets)
which are typically distributed individually.
It then assists the author in generating the individual documents
(potentially in different versions)
as well as a document containing the collected series.
Another application is in developing style files
or other kinds of included material
where compilation of the style file could redirect
to a sample or test file.

%%%%%%%%%%%%%%%%%%%%%%%%%%%%%%%%%%%%%%%%%%%%%%%%%%%%%%%%%%%%%%%%%%%%%%%%%%%%%%%%
%%%%%%%%%%%%%%%%%%%%%%%%%%%%%%%%%%%%%%%%%%%%%%%%%%%%%%%%%%%%%%%%%%%%%%%%%%%%%%%%
\section{Usage}

First of all, the package \textsf{childdoc} is \emph{not} a standard
\LaTeXe{} |.sty| style file! Therefore it needs to be invoked in
a non-standard way.

%%%%%%%%%%%%%%%%%%%%%%%%%%%%%%%%%%%%%%%%%%%%%%%%%%%%%%%%%%%%%%%%%%%%%%%%%%%%%%%%
\subsection{Included Files}
\label{sec:include}

%%%%%%%%%%%%%%%%%%%%%%%%%%%%%%%%%%%%%%%%
\DescribeMacro{\childdocmain}
To use the package, add the commands
\begin{center}
\begin{tabular}{l}
|\input{childdoc.def}|\\
|\childdocmain{}|\\
\end{tabular}
\end{center}
at the very top of the main \LaTeX{} file,
in particular \emph{before} the |\documentclass| statement!
The argument of |\childdocmain| should be left empty
(but it must be present).

%%%%%%%%%%%%%%%%%%%%%%%%%%%%%%%%%%%%%%%%
\DescribeMacro{\childdocof}
Furthermore, add the commands
\begin{center}
\begin{tabular}{l}
|\input{childdoc.def}|\\
|\childdocof{|\textit{main}|}|\\
\end{tabular}
\end{center}
at the top of every child file \textit{child}
which is included by |\include{|\textit{child}|}|
from within the main file
(or at least for those files to be compiled individually).
The argument \textit{main} must be the filename of the main file.

There are a couple of
considerations in setting up the main and child documents:

%%%%%%%%%%%%%%%%%%%%%%%%%%%%%%%%%%%%%%%%
\paragraph{Restrictions.}

Please note the following restrictions:
\begin{itemize}
\item
|\childdocmain| must be called with one argument \textit{main}
to ensure compatibility with earlier version of the package.
It must either be empty (|\childdocmain{}|)
or precisely match the filename of the main file in which it is specified.
See \secref{sec:detection} for further information.
\item
The filename \textit{main} must be specified without the |.tex| extension.
\item
The filename \textit{main} is case sensitive
(even in case-insensitive file systems)
due to internal string comparison.
\item
The argument \textit{main} should be fully expanded, it cannot be a macro.
\item
Subdirectories and special characters should be avoided in filenames.
\item
The command |\childdocmain{|\textit{main}|}| must be followed by a whitespace.
It should not be followed immediately by another command
or by a comment mark `|%|'.
This is because the \TeX{} parser reads the token immediately following
the argument of |\childdocmain| and puts it
at the beginning of every child section;
however, a white\-space is ignored.
\end{itemize}

%%%%%%%%%%%%%%%%%%%%%%%%%%%%%%%%%%%%%%%%
\paragraph{Content of Main File.}

It is advisable to place all content in the child files included by |\include|.
Any output contained in the main file will appear in all child documents
unless suppressed manually;
it cannot be suppressed automatically by the |\includeonly| directive
and thus should normally be avoided.
A method to include some content in the main file
by means of conditional processing is described in \secref{sec:conditional}.

%%%%%%%%%%%%%%%%%%%%%%%%%%%%%%%%%%%%%%%%
\paragraph{Page Numbering.}

When only a part of the document is compiled,
the appropriate numbering of pages
(as well as other status parameters)
is determined from the |.aux| files.
The latter contain information from previous passes.
However this information needs to propagate through
all intermediate child documents.
Therefore the page numbering in child documents may well
be inconsistent until the complete document is compiled at least once.

A useful (if unconventional) way to always ensure a consistent
page numbering is to restart the numbering in each child document
and denote the pages by `\textit{child}|.|\textit{page}'
where \textit{child} represents the chapter/section number of the child file.
This can be achieved by the command
|\numberwithin{page}{|\textit{child}|}|
of the \textsf{amsmath} package
where \textit{child} can be |chapter| or |section|
depending on the chosen structuring.
Alternatively, one can modify the macro |\thepage| appropriately
and reset the counter |page| at the start of each child file.

%%%%%%%%%%%%%%%%%%%%%%%%%%%%%%%%%%%%%%%%%%%%%%%%%%%%%%%%%%%%%%%%%%%%%%%%%%%%%%%%
\subsection{Conditional Processing}
\label{sec:conditional}

The package provides a mechanism to compile different versions
of a document. To customise the versions further some conditional processing
can come in handy to distinguish which version is being compiled.
The package provides two macros to describe the compilation context:

%%%%%%%%%%%%%%%%%%%%%%%%%%%%%%%%%%%%%%%%
\DescribeMacro{\ifchilddoc}
The conditional |\ifchilddoc| distinguishes between the compilation of
child documents and the main document:
%
\begin{center}
|\ifchilddoc |\textit{child-code}| |[|\||else |\textit{main-code}]| \||fi|
\end{center}

%%%%%%%%%%%%%%%%%%%%%%%%%%%%%%%%%%%%%%%%
\DescribeMacro{\childdocname}
\DescribeMacro{\childdocjob}
The macro |\childdocname| contains the filename (without extension)
of the main or child file being processed.
Note that |\childdocjob| will always contain the name of the main file.

%%%%%%%%%%%%%%%%%%%%%%%%%%%%%%%%%%%%%%%%
\paragraph{Title Page.}

Conditional processing can be used to include a title or banner page
in the main document when proper precautions are taken.
Importantly, the code in the main file should ensure that the page counter
(as well as other status parameters which are stored in the |.aux| files)
takes the same value after the conditional processing.
Otherwise the page numbers may take divergent values
depending on which part is compiled.

For example, a title page could be declared by:
%
\begin{center}
\begin{tabular}{l}
|\ifchilddoc\||else|\\
|\addtocounter{page}{-1}|\\
\textit{code for title page}\\
|\newpage|\\
|\||fi|
\end{tabular}
\end{center}
%
A banner page for the child documents can be generated by:
%
\begin{center}
\begin{tabular}{l}
|\ifchilddoc|\\
|\addtocounter{page}{-1}|\\
\textit{code for banner page}\\
|\newpage|\\
|\||fi|
\end{tabular}
\end{center}
%
Here one could write a message such as:
\begin{center}
|This is the part \childdocname{} of \childdocjob{}.|
\end{center}

%%%%%%%%%%%%%%%%%%%%%%%%%%%%%%%%%%%%%%%%%%%%%%%%%%%%%%%%%%%%%%%%%%%%%%%%%%%%%%%%
\subsection{Flags}
\label{sec:flags}

The package makes it easy to generate different versions
of the main or child documents.
To this end compilation flags can be defined
and assigned different default values.
They will be particularly useful in conjunction
with the forwarding mechanism described in \secref{sec:forward}.

For example, it may be useful to have a flag |\version|
which can be set to |draft| or |final|.
The document source will contain some conditional code
depending on the value of |\version|.
Suppose further, the flag should default to |final| for the main file
and to |draft| for child files
which is a natural assignment for editing the document.
This is achieved by placing the following code
in the preamble of the main document
(below the |\childdocmain| directive):
%
\begin{center}
\begin{tabular}{l}
|\ifchilddoc|\\
|\providecommand{\version}{draft}|\\
|\||else|\\
|\providecommand{\version}{final}|\\
|\||fi|
\end{tabular}
\end{center}
%
The definition by |\providecommand| makes sure
that previous definitions are not overwritten.
Further statements |\providecommand{\version}{...}|
can thus be added before the above code to override it.

For the main file, one might add a line
(between |\childdocmain| and the above block)
%
\begin{center}
|%\ifchilddoc\||else\providecommand{\version}{draft}\||fi|
\end{center}
%
which can be uncommented to produce a draft version.
Likewise one can add a line to the very top of a child file
(above the |\childdocof{|\textit{main}|}| directive)
%
\begin{center}
|%\providecommand{\version}{final}|
\end{center}
%
which can be uncommented to produce the final version of this child document.

%%%%%%%%%%%%%%%%%%%%%%%%%%%%%%%%%%%%%%%%%%%%%%%%%%%%%%%%%%%%%%%%%%%%%%%%%%%%%%%%
\subsection{Forwarding}
\label{sec:forward}

Different versions of the main or child documents
using compilation flags as described in \secref{sec:flags}
can be (permanently) stored in different files
for convenient compilation, viewing and distribution.
To this end, the package defines a command
to pass on compilation to a different file:

%%%%%%%%%%%%%%%%%%%%%%%%%%%%%%%%%%%%%%%%
\DescribeMacro{\childdocforward}
The command |\childdocforward| redirects processing to
another source file:
%
\begin{center}
\begin{tabular}{l}
|\input{childdoc.def}|\\
|\childdocforward[|\textit{main}|]{|\textit{dest}|}|\\
\end{tabular}
\end{center}
%
The argument \textit{dest} is the destination file
(without extension).
It should be the main file or one of the child files.
Note that further \textsf{childdoc} directives
such as |\childdocof| and |\childdocforward|
in the indicated file will be processed in this form.
The optional argument \textit{main}
passes on directly to the main file \textit{main}
while pretending to compile the child \textit{dest}.
This form behaves as if \textit{dest}
issues |\childdocof{|\textit{main}|}| right away,
and no further \textsf{childdoc} directives will be processed.

%%%%%%%%%%%%%%%%%%%%%%%%%%%%%%%%%%%%%%%%
\DescribeMacro{\...prefix}
In the alternative form |\childdocforwardprefix|,
%
\begin{center}
\begin{tabular}{l}
|\input{childdoc.def}|\\
|\childdocforwardprefix[|\textit{main}|]{|\textit{prefix}|}{|\textit{dest}|}|
\end{tabular}
\end{center}
%
the destination file is determined by a pattern
depending on the current file:
To make this work, the current file must be called
`{\textit{prefix}\hspace{0.2em}\textit{suffix}}'
with \textit{prefix} matching precisely the argument.
Processing is then passed on to the file
`{\textit{dest}\hspace{0.2em}\textit{suffix}}'.
Surely, the same effect is achieved by
directly specifying the
argument `{\textit{dest}\hspace{0.2em}\textit{suffix}}'
in the first form.
However, that requires to set up a different file
for each child. With the alternative form of the command
all these files can have exactly the same content
which simplifies setting them up and maintaining them.

For example, the following file |draft.tex|
with a compilation flag |\version| as described in \secref{sec:flags}
compiles the main document as a draft:
%
\begin{center}
\begin{tabular}{l}
|\def\version{draft}|\\
|\input{childdoc.def}|\\
|\childdocforward{|\textit{main}|}|
\end{tabular}
\end{center}
%
Likewise, the following files |final|\textit{nn}|.tex|
compile the final version of the child document
|child|\textit{nn}|.tex|:
%
\begin{center}
\begin{tabular}{l}
|\def\version{final}|\\
|\input{childdoc.def}|\\
|\childdocforwardprefix{final}{child}|
\end{tabular}
\end{center}
%

Note that when several versions of a main file and/or of each child file
are to be generated, it may be convenient to set up a |Makefile| or
shell script to automatise the process.

%%%%%%%%%%%%%%%%%%%%%%%%%%%%%%%%%%%%%%%%%%%%%%%%%%%%%%%%%%%%%%%%%%%%%%%%%%%%%%%%
\subsection{Command Line Processing}
\label{sec:commandline}

The effect of redirection files can also be achieved by invoking
the \LaTeX{} compiler with a more elaborate command line.
Most conveniently this should be done as part
of a shell script or a |Makefile|.

When using \textsf{childdoc} in the main file, the following
command lines effectively perform a redirection
(note that depending on the shell being used,
backslashes may have to be doubled: `|\|' $\to$ `|\\|'):
%
\begin{center}
|... -jobname "|\textit{target}|" |\\|"|[\textit{flags}]%
|\input{childdoc.def}\childdocforward[|\textit{main}|]{|\textit{dest}|}"|
\end{center}
%
Here \textit{target} is the name of the output file,
\textit{main} is the name of the main file
and \textit{dest} is the name of the main or child file to be processed
(all filenames without extensions).
The optional argument \textit{main} can be omitted
if \textit{main} matches \textit{dest}.
Optionally, compilation \textit{flags} can be defined via |\def| commands.
This command line makes the \TeX{} engine believe
it is compiling the file \textit{target}
whose content is specified as the latter parameter.
The provided code then forwards the processing to
\textit{main} or \textit{dest} as described in \secref{sec:forward}.

%%%%%%%%%%%%%%%%%%%%%%%%%%%%%%%%%%%%%%%%%%%%%%%%%%%%%%%%%%%%%%%%%%%%%%%%%%%%%%%%
\subsection{Include by Input}
\label{sec:input}

Including child documents by |\include| has some restrictions by design.
Most notably, the content of a child document always occupies
its own set of pages; pages cannot be shared between child documents.
Usually, this behaviour makes perfect sense
because each child document contain an essential part of the document.
However, in some situations it may be desirable to compose
a document from a collection of parts
without having mandatory page breaks between then.
For this case, the package
provides a mechanism to include parts
by |\input| which can also be processed individually.
However, by construction this mechanism
requires manual handling of the content to be output.

%%%%%%%%%%%%%%%%%%%%%%%%%%%%%%%%%%%%%%%%
\DescribeMacro{\ifchilddocmanual}
The main file should be prepared as usual, see \secref{sec:include}.
However, the document body must make a distinction
between processing of an individual part and of the main document, e.g.:
%
\begin{center}
\begin{tabular}{l}
|\ifchilddocmanual|\\
|\input{\childdocname}|\\
|\||else|\\
\textit{document body with }|\input{|\textit{part}|}|\\
|\||fi|
\end{tabular}
\end{center}
%
The conditional |\ifchilddocmanual| is true whenever
a part to be included by |\input| is being compiled,
and the name of the part is stored in |\childdocname|.

%%%%%%%%%%%%%%%%%%%%%%%%%%%%%%%%%%%%%%%%
\DescribeMacro{\childdocby}
Each part to be included by |\input| should start with:
%
\begin{center}
\begin{tabular}{l}
|\input{childdoc.def}|\\
|\childdocby{|\textit{main}|}|\\
\end{tabular}
\end{center}
%
The directive |\childdocby| is similar to |\childdocof|
described in \secref{sec:include},
but the subsequent selection of content must be done manually.
To that end, both |\ifchilddoc| and |\ifchilddocmanual|
will be true upon processing of a part,
and the name of the part is stored in |\childdocname|.
Note that |\jobname| will be set to the filename of the current part
so that each part receives an individual |.aux| file
that does not interfere with the |.aux| file(s) of the main document.
This behaviour can be altered by the alternative form
|\childdocby[*]{|\textit{main}|}| (with a non-empty optional argument)
which uses the |.aux| file of the main document
by setting |\jobname| to \textit{main}.

%%%%%%%%%%%%%%%%%%%%%%%%%%%%%%%%%%%%%%%%%%%%%%%%%%%%%%%%%%%%%%%%%%%%%%%%%%%%%%%%
\subsection{Driver Development}
\label{sec:driver}

The \textsf{childdoc} mechanism can also be use for the development
of definition files such as \LaTeX{} styles or classes.
This case differs from the above setup with multiple parts
included by |\include| in that no |\includeonly| should be invoked.
This can be achieved by starting the include file
(before |\ProvidesPackage|) with:
%
\begin{center}
\begin{tabular}{l}
|\input{childdoc.def}|\\
|\childdocforward{|\textit{main}|}|\\
\end{tabular}
\end{center}
%
or alternatively with:
%
\begin{center}
\begin{tabular}{l}
|\input{childdoc.def}|\\
|\childdocby{|\textit{main}|}|\\
\end{tabular}
\end{center}
%
Both forms have slightly different effects as described above.
The main file is prepared as usual, see \secref{sec:include}.

%%%%%%%%%%%%%%%%%%%%%%%%%%%%%%%%%%%%%%%%%%%%%%%%%%%%%%%%%%%%%%%%%%%%%%%%%%%%%%%%
\subsection{Legacy Detection}
\label{sec:detection}

The directive |\childdocmain| in the main file can detect
whether the complete document or merely a child is to be compiled
even without using the directive |\childdocof|.
This method is deprecated because it is less robust
and there is no compelling reason to use it;
it is merely provided for backward compatibility
and it may be removed in future versions.

If the detection mechanism is to be used,
it is mandatory to correctly specify
the filename of the main file as the argument of |\childdocmain|:
%
\begin{center}
\begin{tabular}{l}
|\input{childdoc.def}|\\
|\childdocmain{|\textit{main}|}|\\
\end{tabular}
\end{center}
%
If |\jobname| does not match the argument \textit{main} of |\childdocmain|,
it is assumed that |\jobname| points to the child file to be compiled.
When using |\childdocmain| with the main file specified as argument,
it suffices to start a child file
with just |\input{|\textit{main}|}|
without loading of the package and using |\childdocof|.
If instead all processing is done
with the appropriate \textsf{childdoc} directives,
the argument of \textit{main} of |\childdocmain| can be empty.

An alternative version of the command line processing described
in \secref{sec:commandline} using the detection mechanism reads:
%
\begin{center}
|... -jobname "|\textit{target}|" "|[\textit{flags}]%
[|\def\jobname{|\textit{dest}|}|]|\input{|\textit{main}|}"|
\end{center}

%%%%%%%%%%%%%%%%%%%%%%%%%%%%%%%%%%%%%%%%%%%%%%%%%%%%%%%%%%%%%%%%%%%%%%%%%%%%%%%%
\subsection{Manual Code}
\label{sec:manual}

In case one cannot be certain whether the definitions file |childdoc.def|
is installed on the target \TeX{} distribution
and one prefers not to ship it,
it is conceivable to paste a few relevant commands into the sources.

To that end, drop all statements |\input{childdoc.def}|
and perform the replacements as outlined below.
Instead of |\childdocmain{|\textit{main}|}| add the following code
to the top of the main file:
%
\begin{center}
\begin{tabular}{l}
|\||ifdefined\childdocname\endinput\||fi\newif\ifchilddoc|\\
|\edef\childdocname{\scantokens\expandafter{\jobname\noexpand}}|\\
|\def\childdocmain{|\textit{main}|}\||ifx\childdocmain\childdocname\||else|\\
|\childdoctrue\includeonly{\childdocname}\let\jobname\childdocmain\||fi|\\
\end{tabular}
\end{center}
%
Instead of |\childdocof{|\textit{main}|}| just include the main file
at the top of each child file:
%
\begin{center}
|\input{|\textit{main}|}|
\end{center}
%
A simple redirection |\childdocforward{|\textit{dest}|}| is achieved by:
%
\begin{center}
|\def\jobname{|\textit{dest}|}\input{\jobname}|
\end{center}
%
The redirection with prefix
|\childdocforwardprefix[|\textit{prefix}|]{|\textit{dest}|}|
is accomplished by:
%
\begin{center}
\begin{tabular}{l}
|{\edef\jobname{\scantokens\expandafter{\jobname\noexpand}}|\\
|\def\redirectjob |\textit{prefix}|#1~~~{\gdef\jobname{|\textit{dest}|#1}}|\\
|\expandafter\redirectjob\jobname~~~}\input{\jobname}|
\end{tabular}
\end{center}

In an alternative approach,
child documents can be compiled by a specific command line
without additional code or specific definitions:
%
\begin{center}
|... -jobname "|\textit{target}|" "|[\textit{flags}]%
|\includeonly{|\textit{dest}|}\input{|\textit{main}|}"|
\end{center}
%

%%%%%%%%%%%%%%%%%%%%%%%%%%%%%%%%%%%%%%%%%%%%%%%%%%%%%%%%%%%%%%%%%%%%%%%%%%%%%%%%
%%%%%%%%%%%%%%%%%%%%%%%%%%%%%%%%%%%%%%%%%%%%%%%%%%%%%%%%%%%%%%%%%%%%%%%%%%%%%%%%
\section{Information}

%%%%%%%%%%%%%%%%%%%%%%%%%%%%%%%%%%%%%%%%%%%%%%%%%%%%%%%%%%%%%%%%%%%%%%%%%%%%%%%%
\subsection{Copyright}

Copyright \copyright{} 2017--2018 Niklas Beisert

This work may be distributed and/or modified under the
conditions of the \LaTeX{} Project Public License, either version 1.3
of this license or (at your option) any later version.
The latest version of this license is in
  \url{http://www.latex-project.org/lppl.txt}
and version 1.3 or later is part of all distributions of \LaTeX{}
version 2005/12/01 or later.

This work has the LPPL maintenance status `maintained'.

The Current Maintainer of this work is Niklas Beisert.

This work consists of the files |README.txt|, |childdoc.ins| and |childdoc.dtx|
as well as the derived files |childdoc.def|, |cdocsamp.tex|
with |cdocsch1.tex|, |cdocsch2.tex|, |cdocspt3.tex|, |cdocspt4.tex|,
|cdocsdrf.tex|, |cdocsfn1.tex|, |cdocsfn2.tex|
as well as |childdoc.pdf|.

%%%%%%%%%%%%%%%%%%%%%%%%%%%%%%%%%%%%%%%%%%%%%%%%%%%%%%%%%%%%%%%%%%%%%%%%%%%%%%%%
\subsection{Files and Installation}

The package consists of the files:
%
\begin{center}
\begin{tabular}{ll}
    |README.txt|   & readme file \\
    |childdoc.ins| & installation file \\
    |childdoc.dtx| & source file \\
    |childdoc.def| & definition file \\
    |cdocsamp.tex| & sample main file \\
    |cdocsch1.tex| & sample include file \\
    |cdocsch2.tex| & sample include file \\
    |cdocspt3.tex| & sample part file \\
    |cdocspt4.tex| & sample part file \\
    |cdocsdrf.tex| & sample redirection file \\
    |cdocsfn1.tex| & sample redirection file \\
    |cdocsfn2.tex| & sample redirection file \\
    |childdoc.pdf| & manual
\end{tabular}
\end{center}
%
The distribution consists of the files
|README.txt|, |childdoc.ins| and |childdoc.dtx|.
%
\begin{itemize}
\item
Run (pdf)\LaTeX{} on |childdoc.dtx|
to compile the manual |childdoc.pdf| (this file).
\item
Run \LaTeX{} on |childdoc.ins| to create the definitions file |childdoc.def|
and the sample |cdocsamp.tex| with include files
|cdocsch1.tex|, |cdocsch2.tex|, |cdocspt3.tex|, |cdocspt4.tex|,
|cdocsdrf.tex|, |cdocsfn1.tex|, |cdocsfn2.tex|.
Then copy the file |childdoc.def| to an appropriate directory of your \LaTeX{}
distribution, e.g.\ \textit{texmf-root}|/tex/latex/childdoc|.
\end{itemize}

%%%%%%%%%%%%%%%%%%%%%%%%%%%%%%%%%%%%%%%%%%%%%%%%%%%%%%%%%%%%%%%%%%%%%%%%%%%%%%%%
\subsection{Related CTAN Packages}

There are several other packages which offer a similar functionality:
%
\begin{itemize}
\item
The packages
\href{http://ctan.org/pkg/docmute}{\textsf{docmute}},
\href{http://ctan.org/pkg/includex}{\textsf{includex}} and
\href{http://ctan.org/pkg/standalone}{\textsf{standalone}}
provide commands to include only the document body of
a child file thus allowing both files to be compiled individually.
\item
The packages \href{http://ctan.org/pkg/subdocs}{\textsf{subdocs}}
and \href{http://ctan.org/pkg/subfiles}{\textsf{subfiles}}
provide structures in which the main and child documents can be
encapsulated and allowing them to be compiled individually.
The inclusion mechanism is different from the conventional |\include|.
\item
The package \href{http://ctan.org/pkg/combine}{\textsf{combine}}
is an elaborate solution to combine several documents into one.
\end{itemize}
%
See also the CTAN topic \href{http://ctan.org/topic/subdocs}{\textsf{subdocs}}
for further related packages.
The present package differs from the above solutions in that
a document structure constructed with the conventional |\include| mechanism
just needs two extra commands at the top of every file
such that all constituent files can be compiled individually.

%%%%%%%%%%%%%%%%%%%%%%%%%%%%%%%%%%%%%%%%%%%%%%%%%%%%%%%%%%%%%%%%%%%%%%%%%%%%%%%%
%\subsection{Feature Suggestions}
%
%The following is a list of features which may be useful for future
%versions of this package:
%%
%\begin{itemize}
%\item
%\ldots
%\end{itemize}

%%%%%%%%%%%%%%%%%%%%%%%%%%%%%%%%%%%%%%%%%%%%%%%%%%%%%%%%%%%%%%%%%%%%%%%%%%%%%%%%
\subsection{Revision History}

%%%%%%%%%%%%%%%%%%%%%%%%%%%%%%%%%%%%%%%%
\paragraph{v2.0:} 2018/12/30

\begin{itemize}
\item
immediate forward processing
\item
added |\childdocby| mechanism
\item
manual restructured
\end{itemize}

%%%%%%%%%%%%%%%%%%%%%%%%%%%%%%%%%%%%%%%%
\paragraph{v1.6:} 2018/01/17

\begin{itemize}
\item
application for development of include files
\item
corrections to manual
\end{itemize}

%%%%%%%%%%%%%%%%%%%%%%%%%%%%%%%%%%%%%%%%
\paragraph{v1.5:} 2017/05/21

\begin{itemize}
\item
more complete structuring introduced
\item
|\childdocof| introduced
\item
|\childdoc| renamed to |\childdocmain|
\item
|\childredirect| renamed to |\childdocforward| and |\childdocforwardprefix|
and functionality expanded
\end{itemize}

%%%%%%%%%%%%%%%%%%%%%%%%%%%%%%%%%%%%%%%%
\paragraph{v1.0:} 2017/04/27

\begin{itemize}
\item
manual and install package
\item
first version published on CTAN
\end{itemize}

%%%%%%%%%%%%%%%%%%%%%%%%%%%%%%%%%%%%%%%%
\paragraph{v0.6:} 2017/04/26

\begin{itemize}
\item
redirection mechanism added
\end{itemize}

%%%%%%%%%%%%%%%%%%%%%%%%%%%%%%%%%%%%%%%%
\paragraph{v0.5:} 2017/04/26

\begin{itemize}
\item
functionality in definition file
\end{itemize}


%%%%%%%%%%%%%%%%%%%%%%%%%%%%%%%%%%%%%%%%%%%%%%%%%%%%%%%%%%%%%%%%%%%%%%%%%%%%%%%%
%%%%%%%%%%%%%%%%%%%%%%%%%%%%%%%%%%%%%%%%%%%%%%%%%%%%%%%%%%%%%%%%%%%%%%%%%%%%%%%%
%%%%%%%%%%%%%%%%%%%%%%%%%%%%%%%%%%%%%%%%%%%%%%%%%%%%%%%%%%%%%%%%%%%%%%%%%%%%%%%%
\appendix

\settowidth\MacroIndent{\rmfamily\scriptsize 000\ }

 \DocInput{childdoc.dtx}

\end{document}
%</driver>
% \fi
%
% %%%%%%%%%%%%%%%%%%%%%%%%%%%%%%%%%%%%%%%%%%%%%%%%%%%%%%%%%%%%%%%%%%%%%%%%%%%%%%
% %%%%%%%%%%%%%%%%%%%%%%%%%%%%%%%%%%%%%%%%%%%%%%%%%%%%%%%%%%%%%%%%%%%%%%%%%%%%%%
% \section{Sample}
%\iffalse
%<*samplemain>
%\fi
%
% The following presents a sample document
% with two chapters, two parts, a title page,
% a compile flag as well as three forwarding files to set the flag.
% It consists of eight |.tex| files:
% \begin{center}
% \begin{tabular}{ll}
% |cdocsamp.tex|&main file\\
% |cdocsch1.tex|&include file for chapter 1\\
% |cdocsch2.tex|&include file for chapter 2\\
% |cdocspt3.tex|&include file for part 3\\
% |cdocspt4.tex|&include file for part 4\\
% |cdocsdrf.tex|&forwarding file for main file in draft mode\\
% |cdocsfi1.tex|&forwarding file for final version of chapter 1\\
% |cdocsfi2.tex|&forwarding file for final version of chapter 2\\
% \end{tabular}
% \end{center}
% Each of the eight files can be compiled directly by the \LaTeX{} compiler.
%
% %%%%%%%%%%%%%%%%%%%%%%%%%%%%%%%%%%%%%%
% \paragraph{Main File.}
%
% The main file is called |cdocsamp.tex|.
%
% Load the \textsf{childdoc} definitions and
% declare the filename for the main document:
%    \begin{macrocode}
\input{childdoc.def}
\childdocmain{}
%    \end{macrocode}

% Optional override for |\version| flag:
%    \begin{macrocode}
%%\ifchilddoc\else\providecommand{\version}{draft}\fi
%    \end{macrocode}

% Define the default values for the |\version| flag
% (|final| for the main file and |draft| for childs):
%    \begin{macrocode}
\ifchilddoc
\providecommand{\version}{draft}
\else
\providecommand{\version}{final}
\fi
%    \end{macrocode}

% Load the standard document class:
%    \begin{macrocode}
\documentclass[12pt]{article}
%    \end{macrocode}

% Start the document body:
%    \begin{macrocode}
\begin{document}
%    \end{macrocode}

% Declare a title page.
% Print title, part of document being processed and version flag:
%    \begin{macrocode}
\addtocounter{page}{-1}
\begin{center}
{\LARGE\bfseries{}childdoc example\par}
\vspace{1cm}
\ifchilddoc
\ifchilddocmanual part\else chapter\fi:
`\childdocname' of `\childdocjob'\par
\else
main document: `\childdocjob'\par
\fi
version: \version\par
\end{center}
\newpage
%    \end{macrocode}

% Manually include selected file,
% otherwise process as usual:
%    \begin{macrocode}
\ifchilddocmanual
\section*{part `\childdocname'}
\input{\childdocname}
\else
%    \end{macrocode}

% Include the two chapters:
%    \begin{macrocode}
\include{cdocsch1}
\include{cdocsch2}
%    \end{macrocode}

% Include the two parts unless only chapters should be displayed:
%    \begin{macrocode}
\ifchilddoc\else
\section{part three}
\input{cdocspt3}
\section{part four}
\input{cdocspt4}
\fi
%    \end{macrocode}

% Process as usual until here:
%    \begin{macrocode}
\fi
%    \end{macrocode}

% End of document body:
%    \begin{macrocode}
\end{document}
%    \end{macrocode}
%\iffalse
%</samplemain>
%\fi
%
% %%%%%%%%%%%%%%%%%%%%%%%%%%%%%%%%%%%%%%
% \paragraph{Chapter Include Files.}
%
% The include files are called |cdocsch1.tex| and |cdocsch2.tex|.
%
%\iffalse
%<*samplechap1|samplechap2>
%\fi

% Optional override for |\version| flag:
%    \begin{macrocode}
%%\providecommand{\version}{final}
%    \end{macrocode}

% Include the main document:
%    \begin{macrocode}
\input{childdoc.def}
\childdocof{cdocsamp}
%    \end{macrocode}

%\iffalse
%</samplechap1|samplechap2>
%\fi
%
%\iffalse
%<*samplechap1>
%\fi
% Some text for chapter 1:
%    \begin{macrocode}
\section{one}
some text in chapter one
%    \end{macrocode}

%\iffalse
%</samplechap1>
%\fi
% Some text for chapter 2:
%\iffalse
%<*samplechap2>
%\fi
%    \begin{macrocode}
\section{two}
more text in chapter two
%    \end{macrocode}

%\iffalse
%</samplechap2>
%\fi
%
% %%%%%%%%%%%%%%%%%%%%%%%%%%%%%%%%%%%%%%
% \paragraph{Part Include Files.}
%
% The include files are called |cdocspt3.tex| and |cdocspt4.tex|.
%
%\iffalse
%<*samplepart3|samplepart4>
%\fi

% Optional override for |\version| flag:
%    \begin{macrocode}
%%\providecommand{\version}{final}
%    \end{macrocode}

% Include the main document:
%    \begin{macrocode}
\input{childdoc.def}
\childdocby{cdocsamp}
%    \end{macrocode}

%\iffalse
%</samplepart3|samplepart4>
%\fi
%
%\iffalse
%<*samplepart3>
%\fi
% Some text for part 3:
%    \begin{macrocode}
some text in part three
%    \end{macrocode}

%\iffalse
%</samplepart3>
%\fi
% Some text for part 4:
%\iffalse
%<*samplepart4>
%\fi
%    \begin{macrocode}
more text in part four
%    \end{macrocode}

%\iffalse
%</samplepart4>
%\fi
%
% %%%%%%%%%%%%%%%%%%%%%%%%%%%%%%%%%%%%%%
% \paragraph{Forwarding for a Complete Draft.}
%
% The following forwarding file |cdocsdrf.tex|
% compiles the main document in draft mode:
%\iffalse
%<*sampledraft>
%\fi
%    \begin{macrocode}
\def\version{draft}
\input{childdoc.def}
\childdocforward{cdocsamp}
%    \end{macrocode}

%\iffalse
%</sampledraft>
%\fi
%
% %%%%%%%%%%%%%%%%%%%%%%%%%%%%%%%%%%%%%%
% \paragraph{Forwarding for Final Version of the Chapters.}
%
% The following forwarding files |cdocsfn1.tex| and |cdocsfn2.tex|
% (with identical content)
% compile the final versions of the child documents
% |cdocsch1.tex| and |cdocsch2.tex|, respectively:
%\iffalse
%<*samplefinal>
%\fi
%    \begin{macrocode}
\def\version{final}
\input{childdoc.def}
\childdocforwardprefix[cdocsamp]{cdocsfn}{cdocsch}
%    \end{macrocode}

%\iffalse
%</samplefinal>
%\fi
%
% %%%%%%%%%%%%%%%%%%%%%%%%%%%%%%%%%%%%%%
% \paragraph{Command Line Processing.}
%
% The following three command lines generate the output files
% |cdocscld|, |cdocscl1| and |cdocscl2|
% which should be identical to
% |cdocsdrf|, |cdocsch1| and |cdocsfn2|, respectively:
% \begin{center}
% \begin{tabular}{l}
% |latex -jobname cdocscld \|\\
% |  "\def\version{draft}\input{childdoc.def}\childdocforward{cdocsamp}"|\\
% |latex -jobname cdocscl1 \|\\
% |  "\input{childdoc.def}\childdocforward[cdocsamp]{cdocsch1}"|\\
% |latex -jobname cdocscl2 \|\\
% |  "\def\version{final}\input{childdoc.def}\childdocforward{cdocsch2}"|
% \end{tabular}
% \end{center}
% Note that the trailing backslash on each first line
% merely continues the input to the second line
% (for convenient cut ant paste).
% Furthermore, the command |latex| can be replaced by any
% of its alternative versions such as |pdflatex|.
%
% %%%%%%%%%%%%%%%%%%%%%%%%%%%%%%%%%%%%%%%%%%%%%%%%%%%%%%%%%%%%%%%%%%%%%%%%%%%%%%
% %%%%%%%%%%%%%%%%%%%%%%%%%%%%%%%%%%%%%%%%%%%%%%%%%%%%%%%%%%%%%%%%%%%%%%%%%%%%%%
% \section{Implementation}
%\iffalse
%<*package>
%\fi
%
% This section describes the definitions file |childdoc.def|.

% The definitions cannot be loaded using |\usepackage| or |\RequirePackage|
% which has a mechanism to prevent loading a style file more than once.
% When loading the definitions by means of |\input|
% multiple instances have to be prevented manually:
%\iffalse
%This code needs to be before the `\ProvidesFile' directive
%which is defined at the beginning of this file.
%Therefore it is also placed there and commented out here.
%</package>
%<*discard>
%\fi
%    \begin{macrocode}
\ifdefined\childdocmain\endinput\fi
%    \end{macrocode}
%\iffalse
%</discard>
%<*package>
%\fi
%
% \macro{\ifchilddoc}
% \macro{\ifchilddocmanual}
% The conditional |\ifchilddoc| tells whether a
% child (true) or main (false) document is being compiled.
% The conditional |\ifchilddocmanual| tells whether
% the |\includeonly| mechanism is used (false) or
% the selection of child files must be performed manually (true).
% The definitions initialise to false:
%    \begin{macrocode}
\newif\ifchilddoc
\newif\ifchilddocmanual
%    \end{macrocode}

% \macro{\childdocname}
% \macro{\childdocjob}
% The macro |\childdocname| stores the name of the main document
% to be compiled. The macro |\childdocjob| stores the name of
% the document on which the \LaTeX{} compiler was originally invoked.
% The content of |\jobname| cannot be compared
% to filenames specified in the source due to different catcodes.
% The following code rescans |\jobname|, stores the result
% in |\childdocname| and saves a copy in |\childdocjob|:
%    \begin{macrocode}
\edef\childdocname{\scantokens\expandafter{\jobname\noexpand}}
\let\childdocjob\childdocname
%    \end{macrocode}

% \macro{\childdocdisable}
% The macro |\childdocdisable| prevents the main file
% from being processed more than once.
% At this stage, the main document command |\childdocmain|
% is assumed to be called once again where it should do nothing.
% Any subsequent call to it should prevent
% a secondary processing of the main document
% It overwrites the forwarding commands
% |\childdocof| and |\childdocforward|
% with empty macros to prevent further inclusions of the main document:
%    \begin{macrocode}
\newcommand{\childdocdisable}
{
  \renewcommand{\childdocmain}[1]{\renewcommand{\childdocmain}[1]{\endinput}}
  \renewcommand{\childdocof}[1]{}
  \renewcommand{\childdocby}[2][]{}
  \renewcommand{\childdocforward}[2][]{}
  \renewcommand{\childdocdisable}{}
}
%    \end{macrocode}

% \macro{\childdocmain}
% The macro |\childdocmain| is to be called at the top of the main file
% with nothing or the main filename (without extension) as argument.
% First, it breaks loops.
% If the argument is not empty and does not match |\childdocname|
% (which is set by the first inclusion of |childdoc.def|),
% |\ifchilddoc| is set to true, |\includeonly| is applied to the child file
% and |\jobname| is set to the main file
% (for proper handling of |.aux| files):
%    \begin{macrocode}
\newcommand{\childdocmain}[1]
{
  \childdocdisable\childdocmain{}
  \if?#1?\else
    \begingroup
      \def\childdoctmp{#1}
      \ifx\childdoctmp\childdocname
        \def\childdoctmp{}
      \else
        \def\childdoctmp
        {
          \childdoctrue
          \includeonly{\childdocname}
          \def\childdocjob{#1}
          \def\jobname{#1}
        }
      \fi
      \expandafter
    \endgroup
    \childdoctmp
  \fi
}
%    \end{macrocode}

% \macro{\childdocof}
% The command |\childdocof| redirects
% compilation to the main file |#1|.
%    \begin{macrocode}
\newcommand{\childdocof}[1]
{
  \childdocdisable
  \childdoctrue
  \includeonly{\childdocname}
  \def\jobname{#1}
  \def\childdocjob{#1}
  \input{#1}
}
%    \end{macrocode}

% \macro{\childdocby}
% The command |\childdocby| ....
%    \begin{macrocode}
\newcommand{\childdocby}[2][]
{
  \childdocdisable
  \childdoctrue
  \childdocmanualtrue
  \if?#1?\else
    \def\jobname{#2}
  \fi
  \def\childdocjob{#2}
  \input{#2}
  \endinput
}
%    \end{macrocode}

% \macro{\childdocforward}
% The command |\childdocforward| redirects
% compilation to the main file or
% (if the optional argument is given) a child file.
% Parameters are set as if the main file
% or a child file starting with |\childdocof| was compiled.
% Then compilation is handed over to the main file:
%    \begin{macrocode}
\newcommand{\childdocforward}[2][]
{
  \begingroup
    \if?#1?
      \def\childdoctmp
      {
        \def\childdocname{#2}
        \def\childdocjob{#2}
        \def\jobname{#2}
        \input{#2}
        \endinput
      }
    \else
      \def\childdoctmp
      {
        \childdocdisable
        \def\childdocname{#2}
        \childdoctrue
        \includeonly{#2}
        \def\childdocjob{#1}
        \def\jobname{#1}
        \input{#1}
        \endinput
      }
    \fi
    \expandafter
  \endgroup
  \childdoctmp
}
%    \end{macrocode}

% \macro{\childdocforwardprefix}
% The command |\childdocforwardprefix| redirects
% compilation to the main or a child file by means of a pattern.
% The prefix |#1| in the current filename is replaced by |#2|
% and the suffix of the current filename is kept
% (it is assumed that the filename does not contain the substring `|~~~|'
% which is used as a delimiter).
% Compilation is handed over to the new file by |\childdocforward|:
%    \begin{macrocode}
\newcommand{\childdocforwardprefix}[3][]
{
  \begingroup
    \def\childdocextract #2##1~~~{\def\childdoctmp{\childdocforward[#1]{#3##1}}}
    \expandafter\childdocextract\childdocname~~~
    \expandafter
  \endgroup
  \childdoctmp
}
%    \end{macrocode}

% \macro{\childdoc}
% The deprecated macro |\childdoc| is a legacy version of |\childdocmain|:
%    \begin{macrocode}
\newcommand{\childdoc}{\childdocmain}
%    \end{macrocode}

% \macro{\childdocredirect}
% The deprecated macro |\childdocredirect| is a legacy version
% of |\childdocforward| and |\childdocforwardprefix|:
%    \begin{macrocode}
\newcommand{\childdocredirect}[2][]
{
  \begingroup
    \if?#1?
      \def\childdoctmp{\childdocforward{#2}}
    \else
      \def\childdoctmp{\childdocforwardprefix{#1}{#2}}
    \fi
    \expandafter
  \endgroup
  \childdoctmp
}
%    \end{macrocode}

%\iffalse
%</package>
%\fi
%
\endinput
\childdocforward{cdocsch2}"|
% \end{tabular}
% \end{center}
% Note that the trailing backslash on each first line
% merely continues the input to the second line
% (for convenient cut ant paste).
% Furthermore, the command |latex| can be replaced by any
% of its alternative versions such as |pdflatex|.
%
% %%%%%%%%%%%%%%%%%%%%%%%%%%%%%%%%%%%%%%%%%%%%%%%%%%%%%%%%%%%%%%%%%%%%%%%%%%%%%%
% %%%%%%%%%%%%%%%%%%%%%%%%%%%%%%%%%%%%%%%%%%%%%%%%%%%%%%%%%%%%%%%%%%%%%%%%%%%%%%
% \section{Implementation}
%\iffalse
%<*package>
%\fi
%
% This section describes the definitions file |childdoc.def|.

% The definitions cannot be loaded using |\usepackage| or |\RequirePackage|
% which has a mechanism to prevent loading a style file more than once.
% When loading the definitions by means of |\input|
% multiple instances have to be prevented manually:
%\iffalse
%This code needs to be before the `\ProvidesFile' directive
%which is defined at the beginning of this file.
%Therefore it is also placed there and commented out here.
%</package>
%<*discard>
%\fi
%    \begin{macrocode}
\ifdefined\childdocmain\endinput\fi
%    \end{macrocode}
%\iffalse
%</discard>
%<*package>
%\fi
%
% \macro{\ifchilddoc}
% \macro{\ifchilddocmanual}
% The conditional |\ifchilddoc| tells whether a
% child (true) or main (false) document is being compiled.
% The conditional |\ifchilddocmanual| tells whether
% the |\includeonly| mechanism is used (false) or
% the selection of child files must be performed manually (true).
% The definitions initialise to false:
%    \begin{macrocode}
\newif\ifchilddoc
\newif\ifchilddocmanual
%    \end{macrocode}

% \macro{\childdocname}
% \macro{\childdocjob}
% The macro |\childdocname| stores the name of the main document
% to be compiled. The macro |\childdocjob| stores the name of
% the document on which the \LaTeX{} compiler was originally invoked.
% The content of |\jobname| cannot be compared
% to filenames specified in the source due to different catcodes.
% The following code rescans |\jobname|, stores the result
% in |\childdocname| and saves a copy in |\childdocjob|:
%    \begin{macrocode}
\edef\childdocname{\scantokens\expandafter{\jobname\noexpand}}
\let\childdocjob\childdocname
%    \end{macrocode}

% \macro{\childdocdisable}
% The macro |\childdocdisable| prevents the main file
% from being processed more than once.
% At this stage, the main document command |\childdocmain|
% is assumed to be called once again where it should do nothing.
% Any subsequent call to it should prevent
% a secondary processing of the main document
% It overwrites the forwarding commands
% |\childdocof| and |\childdocforward|
% with empty macros to prevent further inclusions of the main document:
%    \begin{macrocode}
\newcommand{\childdocdisable}
{
  \renewcommand{\childdocmain}[1]{\renewcommand{\childdocmain}[1]{\endinput}}
  \renewcommand{\childdocof}[1]{}
  \renewcommand{\childdocby}[2][]{}
  \renewcommand{\childdocforward}[2][]{}
  \renewcommand{\childdocdisable}{}
}
%    \end{macrocode}

% \macro{\childdocmain}
% The macro |\childdocmain| is to be called at the top of the main file
% with nothing or the main filename (without extension) as argument.
% First, it breaks loops.
% If the argument is not empty and does not match |\childdocname|
% (which is set by the first inclusion of |childdoc.def|),
% |\ifchilddoc| is set to true, |\includeonly| is applied to the child file
% and |\jobname| is set to the main file
% (for proper handling of |.aux| files):
%    \begin{macrocode}
\newcommand{\childdocmain}[1]
{
  \childdocdisable\childdocmain{}
  \if?#1?\else
    \begingroup
      \def\childdoctmp{#1}
      \ifx\childdoctmp\childdocname
        \def\childdoctmp{}
      \else
        \def\childdoctmp
        {
          \childdoctrue
          \includeonly{\childdocname}
          \def\childdocjob{#1}
          \def\jobname{#1}
        }
      \fi
      \expandafter
    \endgroup
    \childdoctmp
  \fi
}
%    \end{macrocode}

% \macro{\childdocof}
% The command |\childdocof| redirects
% compilation to the main file |#1|.
%    \begin{macrocode}
\newcommand{\childdocof}[1]
{
  \childdocdisable
  \childdoctrue
  \includeonly{\childdocname}
  \def\jobname{#1}
  \def\childdocjob{#1}
  \input{#1}
}
%    \end{macrocode}

% \macro{\childdocby}
% The command |\childdocby| ....
%    \begin{macrocode}
\newcommand{\childdocby}[2][]
{
  \childdocdisable
  \childdoctrue
  \childdocmanualtrue
  \if?#1?\else
    \def\jobname{#2}
  \fi
  \def\childdocjob{#2}
  \input{#2}
  \endinput
}
%    \end{macrocode}

% \macro{\childdocforward}
% The command |\childdocforward| redirects
% compilation to the main file or
% (if the optional argument is given) a child file.
% Parameters are set as if the main file
% or a child file starting with |\childdocof| was compiled.
% Then compilation is handed over to the main file:
%    \begin{macrocode}
\newcommand{\childdocforward}[2][]
{
  \begingroup
    \if?#1?
      \def\childdoctmp
      {
        \def\childdocname{#2}
        \def\childdocjob{#2}
        \def\jobname{#2}
        \input{#2}
        \endinput
      }
    \else
      \def\childdoctmp
      {
        \childdocdisable
        \def\childdocname{#2}
        \childdoctrue
        \includeonly{#2}
        \def\childdocjob{#1}
        \def\jobname{#1}
        \input{#1}
        \endinput
      }
    \fi
    \expandafter
  \endgroup
  \childdoctmp
}
%    \end{macrocode}

% \macro{\childdocforwardprefix}
% The command |\childdocforwardprefix| redirects
% compilation to the main or a child file by means of a pattern.
% The prefix |#1| in the current filename is replaced by |#2|
% and the suffix of the current filename is kept
% (it is assumed that the filename does not contain the substring `|~~~|'
% which is used as a delimiter).
% Compilation is handed over to the new file by |\childdocforward|:
%    \begin{macrocode}
\newcommand{\childdocforwardprefix}[3][]
{
  \begingroup
    \def\childdocextract #2##1~~~{\def\childdoctmp{\childdocforward[#1]{#3##1}}}
    \expandafter\childdocextract\childdocname~~~
    \expandafter
  \endgroup
  \childdoctmp
}
%    \end{macrocode}

% \macro{\childdoc}
% The deprecated macro |\childdoc| is a legacy version of |\childdocmain|:
%    \begin{macrocode}
\newcommand{\childdoc}{\childdocmain}
%    \end{macrocode}

% \macro{\childdocredirect}
% The deprecated macro |\childdocredirect| is a legacy version
% of |\childdocforward| and |\childdocforwardprefix|:
%    \begin{macrocode}
\newcommand{\childdocredirect}[2][]
{
  \begingroup
    \if?#1?
      \def\childdoctmp{\childdocforward{#2}}
    \else
      \def\childdoctmp{\childdocforwardprefix{#1}{#2}}
    \fi
    \expandafter
  \endgroup
  \childdoctmp
}
%    \end{macrocode}

%\iffalse
%</package>
%\fi
%
\endinput
|\\
|\childdocof{|\textit{main}|}|\\
\end{tabular}
\end{center}
at the top of every child file \textit{child}
which is included by |\include{|\textit{child}|}|
from within the main file
(or at least for those files to be compiled individually).
The argument \textit{main} must be the filename of the main file.

There are a couple of
considerations in setting up the main and child documents:

%%%%%%%%%%%%%%%%%%%%%%%%%%%%%%%%%%%%%%%%
\paragraph{Restrictions.}

Please note the following restrictions:
\begin{itemize}
\item
|\childdocmain| must be called with one argument \textit{main}
to ensure compatibility with earlier version of the package.
It must either be empty (|\childdocmain{}|)
or precisely match the filename of the main file in which it is specified.
See \secref{sec:detection} for further information.
\item
The filename \textit{main} must be specified without the |.tex| extension.
\item
The filename \textit{main} is case sensitive
(even in case-insensitive file systems)
due to internal string comparison.
\item
The argument \textit{main} should be fully expanded, it cannot be a macro.
\item
Subdirectories and special characters should be avoided in filenames.
\item
The command |\childdocmain{|\textit{main}|}| must be followed by a whitespace.
It should not be followed immediately by another command
or by a comment mark `|%|'.
This is because the \TeX{} parser reads the token immediately following
the argument of |\childdocmain| and puts it
at the beginning of every child section;
however, a white\-space is ignored.
\end{itemize}

%%%%%%%%%%%%%%%%%%%%%%%%%%%%%%%%%%%%%%%%
\paragraph{Content of Main File.}

It is advisable to place all content in the child files included by |\include|.
Any output contained in the main file will appear in all child documents
unless suppressed manually;
it cannot be suppressed automatically by the |\includeonly| directive
and thus should normally be avoided.
A method to include some content in the main file
by means of conditional processing is described in \secref{sec:conditional}.

%%%%%%%%%%%%%%%%%%%%%%%%%%%%%%%%%%%%%%%%
\paragraph{Page Numbering.}

When only a part of the document is compiled,
the appropriate numbering of pages
(as well as other status parameters)
is determined from the |.aux| files.
The latter contain information from previous passes.
However this information needs to propagate through
all intermediate child documents.
Therefore the page numbering in child documents may well
be inconsistent until the complete document is compiled at least once.

A useful (if unconventional) way to always ensure a consistent
page numbering is to restart the numbering in each child document
and denote the pages by `\textit{child}|.|\textit{page}'
where \textit{child} represents the chapter/section number of the child file.
This can be achieved by the command
|\numberwithin{page}{|\textit{child}|}|
of the \textsf{amsmath} package
where \textit{child} can be |chapter| or |section|
depending on the chosen structuring.
Alternatively, one can modify the macro |\thepage| appropriately
and reset the counter |page| at the start of each child file.

%%%%%%%%%%%%%%%%%%%%%%%%%%%%%%%%%%%%%%%%%%%%%%%%%%%%%%%%%%%%%%%%%%%%%%%%%%%%%%%%
\subsection{Conditional Processing}
\label{sec:conditional}

The package provides a mechanism to compile different versions
of a document. To customise the versions further some conditional processing
can come in handy to distinguish which version is being compiled.
The package provides two macros to describe the compilation context:

%%%%%%%%%%%%%%%%%%%%%%%%%%%%%%%%%%%%%%%%
\DescribeMacro{\ifchilddoc}
The conditional |\ifchilddoc| distinguishes between the compilation of
child documents and the main document:
%
\begin{center}
|\ifchilddoc |\textit{child-code}| |[|\||else |\textit{main-code}]| \||fi|
\end{center}

%%%%%%%%%%%%%%%%%%%%%%%%%%%%%%%%%%%%%%%%
\DescribeMacro{\childdocname}
\DescribeMacro{\childdocjob}
The macro |\childdocname| contains the filename (without extension)
of the main or child file being processed.
Note that |\childdocjob| will always contain the name of the main file.

%%%%%%%%%%%%%%%%%%%%%%%%%%%%%%%%%%%%%%%%
\paragraph{Title Page.}

Conditional processing can be used to include a title or banner page
in the main document when proper precautions are taken.
Importantly, the code in the main file should ensure that the page counter
(as well as other status parameters which are stored in the |.aux| files)
takes the same value after the conditional processing.
Otherwise the page numbers may take divergent values
depending on which part is compiled.

For example, a title page could be declared by:
%
\begin{center}
\begin{tabular}{l}
|\ifchilddoc\||else|\\
|\addtocounter{page}{-1}|\\
\textit{code for title page}\\
|\newpage|\\
|\||fi|
\end{tabular}
\end{center}
%
A banner page for the child documents can be generated by:
%
\begin{center}
\begin{tabular}{l}
|\ifchilddoc|\\
|\addtocounter{page}{-1}|\\
\textit{code for banner page}\\
|\newpage|\\
|\||fi|
\end{tabular}
\end{center}
%
Here one could write a message such as:
\begin{center}
|This is the part \childdocname{} of \childdocjob{}.|
\end{center}

%%%%%%%%%%%%%%%%%%%%%%%%%%%%%%%%%%%%%%%%%%%%%%%%%%%%%%%%%%%%%%%%%%%%%%%%%%%%%%%%
\subsection{Flags}
\label{sec:flags}

The package makes it easy to generate different versions
of the main or child documents.
To this end compilation flags can be defined
and assigned different default values.
They will be particularly useful in conjunction
with the forwarding mechanism described in \secref{sec:forward}.

For example, it may be useful to have a flag |\version|
which can be set to |draft| or |final|.
The document source will contain some conditional code
depending on the value of |\version|.
Suppose further, the flag should default to |final| for the main file
and to |draft| for child files
which is a natural assignment for editing the document.
This is achieved by placing the following code
in the preamble of the main document
(below the |\childdocmain| directive):
%
\begin{center}
\begin{tabular}{l}
|\ifchilddoc|\\
|\providecommand{\version}{draft}|\\
|\||else|\\
|\providecommand{\version}{final}|\\
|\||fi|
\end{tabular}
\end{center}
%
The definition by |\providecommand| makes sure
that previous definitions are not overwritten.
Further statements |\providecommand{\version}{...}|
can thus be added before the above code to override it.

For the main file, one might add a line
(between |\childdocmain| and the above block)
%
\begin{center}
|%\ifchilddoc\||else\providecommand{\version}{draft}\||fi|
\end{center}
%
which can be uncommented to produce a draft version.
Likewise one can add a line to the very top of a child file
(above the |\childdocof{|\textit{main}|}| directive)
%
\begin{center}
|%\providecommand{\version}{final}|
\end{center}
%
which can be uncommented to produce the final version of this child document.

%%%%%%%%%%%%%%%%%%%%%%%%%%%%%%%%%%%%%%%%%%%%%%%%%%%%%%%%%%%%%%%%%%%%%%%%%%%%%%%%
\subsection{Forwarding}
\label{sec:forward}

Different versions of the main or child documents
using compilation flags as described in \secref{sec:flags}
can be (permanently) stored in different files
for convenient compilation, viewing and distribution.
To this end, the package defines a command
to pass on compilation to a different file:

%%%%%%%%%%%%%%%%%%%%%%%%%%%%%%%%%%%%%%%%
\DescribeMacro{\childdocforward}
The command |\childdocforward| redirects processing to
another source file:
%
\begin{center}
\begin{tabular}{l}
|% \iffalse
%
% childdoc.dtx Copyright (C) 2017-2018 Niklas Beisert
%
% This work may be distributed and/or modified under the
% conditions of the LaTeX Project Public License, either version 1.3
% of this license or (at your option) any later version.
% The latest version of this license is in
%   http://www.latex-project.org/lppl.txt
% and version 1.3 or later is part of all distributions of LaTeX
% version 2005/12/01 or later.
%
% This work has the LPPL maintenance status `maintained'.
%
% The Current Maintainer of this work is Niklas Beisert.
%
% This work consists of the files childdoc.dtx and childdoc.ins
% and the derived files childdoc.def and cdocsamp.tex with
% cdocsch1.tex, cdocsch2.tex, cdocsdrf.tex, cdocsfn1.tex, cdocsfn2.tex.
%
%<package>\ifdefined\childdocmain\endinput\fi
%<package>\ProvidesFile{childdoc.def}[2018/12/30 v2.0 child document driver]
%<samplemain>\ProvidesFile{cdocsamp.tex}[2018/12/30 v2.0 sample for childdoc]
%<*driver>
%\ProvidesFile{childdoc.drv}[2018/12/30 v2.0 childdoc reference manual file]
\PassOptionsToClass{10pt,a4paper}{article}
\documentclass{ltxdoc}

\usepackage[margin=35mm]{geometry}
\usepackage{hyperref}
\usepackage{hyperxmp}
\usepackage[usenames]{color}

\hypersetup{colorlinks=true}
\hypersetup{pdfstartview=FitH}
\hypersetup{pdfpagemode=UseNone}
\hypersetup{pdfsource={}}
\hypersetup{pdflang={en-UK}}
\hypersetup{pdfcopyright={Copyright 2017-2018 Niklas Beisert.
  This work may be distributed and/or modified under the
  conditions of the LaTeX Project Public License, either version 1.3
  of this license or (at your option) any later version.}}
\hypersetup{pdflicenseurl={http://www.latex-project.org/lppl.txt}}
\hypersetup{pdfcontactaddress={ETH Zurich, ITP, HIT K,
  Wolfgang-Pauli-Strasse 27}}
\hypersetup{pdfcontactpostcode={8093}}
\hypersetup{pdfcontactcity={Zurich}}
\hypersetup{pdfcontactcountry={Switzerland}}
\hypersetup{pdfcontactemail={nbeisert@itp.phys.ethz.ch}}
\hypersetup{pdfcontacturl={http://people.phys.ethz.ch/\xmptilde nbeisert/}}

\newcommand{\secref}[1]{\hyperref[#1]{section \ref*{#1}}}

\parskip1ex
\parindent0pt
\let\olditemize\itemize
\def\itemize{\olditemize\parskip0pt}

\begin{document}

\title{The \textsf{childdoc} Package}
\hypersetup{pdftitle={The childdoc Package}}
\author{Niklas Beisert\\[2ex]
  Institut f\"ur Theoretische Physik\\
  Eidgen\"ossische Technische Hochschule Z\"urich\\
  Wolfgang-Pauli-Strasse 27, 8093 Z\"urich, Switzerland\\[1ex]
  \href{mailto:nbeisert@itp.phys.ethz.ch}
  {\texttt{nbeisert@itp.phys.ethz.ch}}}
\hypersetup{pdfauthor={Niklas Beisert}}
\hypersetup{pdfsubject={Manual for the LaTeX2e Package childdoc}}
\date{30 December 2018, \textsf{v2.0}}
\maketitle

\begin{abstract}\noindent
\textsf{childdoc} is a \LaTeXe{} package
that enables the direct compilation
of document sections included by |\include|
to individual files.
\end{abstract}

\begingroup
\parskip0ex
\tableofcontents
\endgroup

%%%%%%%%%%%%%%%%%%%%%%%%%%%%%%%%%%%%%%%%%%%%%%%%%%%%%%%%%%%%%%%%%%%%%%%%%%%%%%%%
%%%%%%%%%%%%%%%%%%%%%%%%%%%%%%%%%%%%%%%%%%%%%%%%%%%%%%%%%%%%%%%%%%%%%%%%%%%%%%%%
\section{Introduction}

\LaTeX{} provides a mechanism to structure a large document (such as a book)
into a main file and several child files (containing the chapters)
using the |\include| command.
This mechanism is beneficial for documents
which span hundreds of pages in order to
make the source file(s) more manageable.
Moreover, compilation can be restricted to
selected child files by means of the |\includeonly| command.
The latter feature can be used to reduce the compilation time while editing
(this was significantly more useful in the earlier days of \LaTeX{})
or to generate a smaller document which is easier to navigate.
Another application of |\includeonly| is to generate
documents consisting of selected parts of the complete document.

However, there are a few drawbacks of the plain |\include| mechanism:
\begin{itemize}
\item
The child files cannot be compiled on their own,
they can only be compiled via the main file.
A naive editing environment
(such as a text editor with an option
to have the current file processed by \LaTeX)
may require one to switch to the main file before compiling;
attempting to compile the child file produces errors.
\item
The main file must be modified (each time)
to adjust the |\includeonly| command
to the present needs. This easily leaves the main file in a messy state.
\item
The generated document will always carry the filename
of the main document. This is inconvenient if
several child files are to be compiled and
to be kept for distribution.
\end{itemize}

The present package provides a simple interface
to make child files individually compilable by \LaTeX{}.
Compiling a child file then has the same effect as compiling
the main file with an |\includeonly| command
to select the appropriate child.
Moreover the generated document will carry the name of the child
rather than the main file.
This resolves all three above issues.

This feature is meant to make the editing of books,
thesis documents and lecture notes somewhat more convenient.
However, the package can also be used efficiently for
composing a series of documents (such as exercise sheets)
which are typically distributed individually.
It then assists the author in generating the individual documents
(potentially in different versions)
as well as a document containing the collected series.
Another application is in developing style files
or other kinds of included material
where compilation of the style file could redirect
to a sample or test file.

%%%%%%%%%%%%%%%%%%%%%%%%%%%%%%%%%%%%%%%%%%%%%%%%%%%%%%%%%%%%%%%%%%%%%%%%%%%%%%%%
%%%%%%%%%%%%%%%%%%%%%%%%%%%%%%%%%%%%%%%%%%%%%%%%%%%%%%%%%%%%%%%%%%%%%%%%%%%%%%%%
\section{Usage}

First of all, the package \textsf{childdoc} is \emph{not} a standard
\LaTeXe{} |.sty| style file! Therefore it needs to be invoked in
a non-standard way.

%%%%%%%%%%%%%%%%%%%%%%%%%%%%%%%%%%%%%%%%%%%%%%%%%%%%%%%%%%%%%%%%%%%%%%%%%%%%%%%%
\subsection{Included Files}
\label{sec:include}

%%%%%%%%%%%%%%%%%%%%%%%%%%%%%%%%%%%%%%%%
\DescribeMacro{\childdocmain}
To use the package, add the commands
\begin{center}
\begin{tabular}{l}
|% \iffalse
%
% childdoc.dtx Copyright (C) 2017-2018 Niklas Beisert
%
% This work may be distributed and/or modified under the
% conditions of the LaTeX Project Public License, either version 1.3
% of this license or (at your option) any later version.
% The latest version of this license is in
%   http://www.latex-project.org/lppl.txt
% and version 1.3 or later is part of all distributions of LaTeX
% version 2005/12/01 or later.
%
% This work has the LPPL maintenance status `maintained'.
%
% The Current Maintainer of this work is Niklas Beisert.
%
% This work consists of the files childdoc.dtx and childdoc.ins
% and the derived files childdoc.def and cdocsamp.tex with
% cdocsch1.tex, cdocsch2.tex, cdocsdrf.tex, cdocsfn1.tex, cdocsfn2.tex.
%
%<package>\ifdefined\childdocmain\endinput\fi
%<package>\ProvidesFile{childdoc.def}[2018/12/30 v2.0 child document driver]
%<samplemain>\ProvidesFile{cdocsamp.tex}[2018/12/30 v2.0 sample for childdoc]
%<*driver>
%\ProvidesFile{childdoc.drv}[2018/12/30 v2.0 childdoc reference manual file]
\PassOptionsToClass{10pt,a4paper}{article}
\documentclass{ltxdoc}

\usepackage[margin=35mm]{geometry}
\usepackage{hyperref}
\usepackage{hyperxmp}
\usepackage[usenames]{color}

\hypersetup{colorlinks=true}
\hypersetup{pdfstartview=FitH}
\hypersetup{pdfpagemode=UseNone}
\hypersetup{pdfsource={}}
\hypersetup{pdflang={en-UK}}
\hypersetup{pdfcopyright={Copyright 2017-2018 Niklas Beisert.
  This work may be distributed and/or modified under the
  conditions of the LaTeX Project Public License, either version 1.3
  of this license or (at your option) any later version.}}
\hypersetup{pdflicenseurl={http://www.latex-project.org/lppl.txt}}
\hypersetup{pdfcontactaddress={ETH Zurich, ITP, HIT K,
  Wolfgang-Pauli-Strasse 27}}
\hypersetup{pdfcontactpostcode={8093}}
\hypersetup{pdfcontactcity={Zurich}}
\hypersetup{pdfcontactcountry={Switzerland}}
\hypersetup{pdfcontactemail={nbeisert@itp.phys.ethz.ch}}
\hypersetup{pdfcontacturl={http://people.phys.ethz.ch/\xmptilde nbeisert/}}

\newcommand{\secref}[1]{\hyperref[#1]{section \ref*{#1}}}

\parskip1ex
\parindent0pt
\let\olditemize\itemize
\def\itemize{\olditemize\parskip0pt}

\begin{document}

\title{The \textsf{childdoc} Package}
\hypersetup{pdftitle={The childdoc Package}}
\author{Niklas Beisert\\[2ex]
  Institut f\"ur Theoretische Physik\\
  Eidgen\"ossische Technische Hochschule Z\"urich\\
  Wolfgang-Pauli-Strasse 27, 8093 Z\"urich, Switzerland\\[1ex]
  \href{mailto:nbeisert@itp.phys.ethz.ch}
  {\texttt{nbeisert@itp.phys.ethz.ch}}}
\hypersetup{pdfauthor={Niklas Beisert}}
\hypersetup{pdfsubject={Manual for the LaTeX2e Package childdoc}}
\date{30 December 2018, \textsf{v2.0}}
\maketitle

\begin{abstract}\noindent
\textsf{childdoc} is a \LaTeXe{} package
that enables the direct compilation
of document sections included by |\include|
to individual files.
\end{abstract}

\begingroup
\parskip0ex
\tableofcontents
\endgroup

%%%%%%%%%%%%%%%%%%%%%%%%%%%%%%%%%%%%%%%%%%%%%%%%%%%%%%%%%%%%%%%%%%%%%%%%%%%%%%%%
%%%%%%%%%%%%%%%%%%%%%%%%%%%%%%%%%%%%%%%%%%%%%%%%%%%%%%%%%%%%%%%%%%%%%%%%%%%%%%%%
\section{Introduction}

\LaTeX{} provides a mechanism to structure a large document (such as a book)
into a main file and several child files (containing the chapters)
using the |\include| command.
This mechanism is beneficial for documents
which span hundreds of pages in order to
make the source file(s) more manageable.
Moreover, compilation can be restricted to
selected child files by means of the |\includeonly| command.
The latter feature can be used to reduce the compilation time while editing
(this was significantly more useful in the earlier days of \LaTeX{})
or to generate a smaller document which is easier to navigate.
Another application of |\includeonly| is to generate
documents consisting of selected parts of the complete document.

However, there are a few drawbacks of the plain |\include| mechanism:
\begin{itemize}
\item
The child files cannot be compiled on their own,
they can only be compiled via the main file.
A naive editing environment
(such as a text editor with an option
to have the current file processed by \LaTeX)
may require one to switch to the main file before compiling;
attempting to compile the child file produces errors.
\item
The main file must be modified (each time)
to adjust the |\includeonly| command
to the present needs. This easily leaves the main file in a messy state.
\item
The generated document will always carry the filename
of the main document. This is inconvenient if
several child files are to be compiled and
to be kept for distribution.
\end{itemize}

The present package provides a simple interface
to make child files individually compilable by \LaTeX{}.
Compiling a child file then has the same effect as compiling
the main file with an |\includeonly| command
to select the appropriate child.
Moreover the generated document will carry the name of the child
rather than the main file.
This resolves all three above issues.

This feature is meant to make the editing of books,
thesis documents and lecture notes somewhat more convenient.
However, the package can also be used efficiently for
composing a series of documents (such as exercise sheets)
which are typically distributed individually.
It then assists the author in generating the individual documents
(potentially in different versions)
as well as a document containing the collected series.
Another application is in developing style files
or other kinds of included material
where compilation of the style file could redirect
to a sample or test file.

%%%%%%%%%%%%%%%%%%%%%%%%%%%%%%%%%%%%%%%%%%%%%%%%%%%%%%%%%%%%%%%%%%%%%%%%%%%%%%%%
%%%%%%%%%%%%%%%%%%%%%%%%%%%%%%%%%%%%%%%%%%%%%%%%%%%%%%%%%%%%%%%%%%%%%%%%%%%%%%%%
\section{Usage}

First of all, the package \textsf{childdoc} is \emph{not} a standard
\LaTeXe{} |.sty| style file! Therefore it needs to be invoked in
a non-standard way.

%%%%%%%%%%%%%%%%%%%%%%%%%%%%%%%%%%%%%%%%%%%%%%%%%%%%%%%%%%%%%%%%%%%%%%%%%%%%%%%%
\subsection{Included Files}
\label{sec:include}

%%%%%%%%%%%%%%%%%%%%%%%%%%%%%%%%%%%%%%%%
\DescribeMacro{\childdocmain}
To use the package, add the commands
\begin{center}
\begin{tabular}{l}
|\input{childdoc.def}|\\
|\childdocmain{}|\\
\end{tabular}
\end{center}
at the very top of the main \LaTeX{} file,
in particular \emph{before} the |\documentclass| statement!
The argument of |\childdocmain| should be left empty
(but it must be present).

%%%%%%%%%%%%%%%%%%%%%%%%%%%%%%%%%%%%%%%%
\DescribeMacro{\childdocof}
Furthermore, add the commands
\begin{center}
\begin{tabular}{l}
|\input{childdoc.def}|\\
|\childdocof{|\textit{main}|}|\\
\end{tabular}
\end{center}
at the top of every child file \textit{child}
which is included by |\include{|\textit{child}|}|
from within the main file
(or at least for those files to be compiled individually).
The argument \textit{main} must be the filename of the main file.

There are a couple of
considerations in setting up the main and child documents:

%%%%%%%%%%%%%%%%%%%%%%%%%%%%%%%%%%%%%%%%
\paragraph{Restrictions.}

Please note the following restrictions:
\begin{itemize}
\item
|\childdocmain| must be called with one argument \textit{main}
to ensure compatibility with earlier version of the package.
It must either be empty (|\childdocmain{}|)
or precisely match the filename of the main file in which it is specified.
See \secref{sec:detection} for further information.
\item
The filename \textit{main} must be specified without the |.tex| extension.
\item
The filename \textit{main} is case sensitive
(even in case-insensitive file systems)
due to internal string comparison.
\item
The argument \textit{main} should be fully expanded, it cannot be a macro.
\item
Subdirectories and special characters should be avoided in filenames.
\item
The command |\childdocmain{|\textit{main}|}| must be followed by a whitespace.
It should not be followed immediately by another command
or by a comment mark `|%|'.
This is because the \TeX{} parser reads the token immediately following
the argument of |\childdocmain| and puts it
at the beginning of every child section;
however, a white\-space is ignored.
\end{itemize}

%%%%%%%%%%%%%%%%%%%%%%%%%%%%%%%%%%%%%%%%
\paragraph{Content of Main File.}

It is advisable to place all content in the child files included by |\include|.
Any output contained in the main file will appear in all child documents
unless suppressed manually;
it cannot be suppressed automatically by the |\includeonly| directive
and thus should normally be avoided.
A method to include some content in the main file
by means of conditional processing is described in \secref{sec:conditional}.

%%%%%%%%%%%%%%%%%%%%%%%%%%%%%%%%%%%%%%%%
\paragraph{Page Numbering.}

When only a part of the document is compiled,
the appropriate numbering of pages
(as well as other status parameters)
is determined from the |.aux| files.
The latter contain information from previous passes.
However this information needs to propagate through
all intermediate child documents.
Therefore the page numbering in child documents may well
be inconsistent until the complete document is compiled at least once.

A useful (if unconventional) way to always ensure a consistent
page numbering is to restart the numbering in each child document
and denote the pages by `\textit{child}|.|\textit{page}'
where \textit{child} represents the chapter/section number of the child file.
This can be achieved by the command
|\numberwithin{page}{|\textit{child}|}|
of the \textsf{amsmath} package
where \textit{child} can be |chapter| or |section|
depending on the chosen structuring.
Alternatively, one can modify the macro |\thepage| appropriately
and reset the counter |page| at the start of each child file.

%%%%%%%%%%%%%%%%%%%%%%%%%%%%%%%%%%%%%%%%%%%%%%%%%%%%%%%%%%%%%%%%%%%%%%%%%%%%%%%%
\subsection{Conditional Processing}
\label{sec:conditional}

The package provides a mechanism to compile different versions
of a document. To customise the versions further some conditional processing
can come in handy to distinguish which version is being compiled.
The package provides two macros to describe the compilation context:

%%%%%%%%%%%%%%%%%%%%%%%%%%%%%%%%%%%%%%%%
\DescribeMacro{\ifchilddoc}
The conditional |\ifchilddoc| distinguishes between the compilation of
child documents and the main document:
%
\begin{center}
|\ifchilddoc |\textit{child-code}| |[|\||else |\textit{main-code}]| \||fi|
\end{center}

%%%%%%%%%%%%%%%%%%%%%%%%%%%%%%%%%%%%%%%%
\DescribeMacro{\childdocname}
\DescribeMacro{\childdocjob}
The macro |\childdocname| contains the filename (without extension)
of the main or child file being processed.
Note that |\childdocjob| will always contain the name of the main file.

%%%%%%%%%%%%%%%%%%%%%%%%%%%%%%%%%%%%%%%%
\paragraph{Title Page.}

Conditional processing can be used to include a title or banner page
in the main document when proper precautions are taken.
Importantly, the code in the main file should ensure that the page counter
(as well as other status parameters which are stored in the |.aux| files)
takes the same value after the conditional processing.
Otherwise the page numbers may take divergent values
depending on which part is compiled.

For example, a title page could be declared by:
%
\begin{center}
\begin{tabular}{l}
|\ifchilddoc\||else|\\
|\addtocounter{page}{-1}|\\
\textit{code for title page}\\
|\newpage|\\
|\||fi|
\end{tabular}
\end{center}
%
A banner page for the child documents can be generated by:
%
\begin{center}
\begin{tabular}{l}
|\ifchilddoc|\\
|\addtocounter{page}{-1}|\\
\textit{code for banner page}\\
|\newpage|\\
|\||fi|
\end{tabular}
\end{center}
%
Here one could write a message such as:
\begin{center}
|This is the part \childdocname{} of \childdocjob{}.|
\end{center}

%%%%%%%%%%%%%%%%%%%%%%%%%%%%%%%%%%%%%%%%%%%%%%%%%%%%%%%%%%%%%%%%%%%%%%%%%%%%%%%%
\subsection{Flags}
\label{sec:flags}

The package makes it easy to generate different versions
of the main or child documents.
To this end compilation flags can be defined
and assigned different default values.
They will be particularly useful in conjunction
with the forwarding mechanism described in \secref{sec:forward}.

For example, it may be useful to have a flag |\version|
which can be set to |draft| or |final|.
The document source will contain some conditional code
depending on the value of |\version|.
Suppose further, the flag should default to |final| for the main file
and to |draft| for child files
which is a natural assignment for editing the document.
This is achieved by placing the following code
in the preamble of the main document
(below the |\childdocmain| directive):
%
\begin{center}
\begin{tabular}{l}
|\ifchilddoc|\\
|\providecommand{\version}{draft}|\\
|\||else|\\
|\providecommand{\version}{final}|\\
|\||fi|
\end{tabular}
\end{center}
%
The definition by |\providecommand| makes sure
that previous definitions are not overwritten.
Further statements |\providecommand{\version}{...}|
can thus be added before the above code to override it.

For the main file, one might add a line
(between |\childdocmain| and the above block)
%
\begin{center}
|%\ifchilddoc\||else\providecommand{\version}{draft}\||fi|
\end{center}
%
which can be uncommented to produce a draft version.
Likewise one can add a line to the very top of a child file
(above the |\childdocof{|\textit{main}|}| directive)
%
\begin{center}
|%\providecommand{\version}{final}|
\end{center}
%
which can be uncommented to produce the final version of this child document.

%%%%%%%%%%%%%%%%%%%%%%%%%%%%%%%%%%%%%%%%%%%%%%%%%%%%%%%%%%%%%%%%%%%%%%%%%%%%%%%%
\subsection{Forwarding}
\label{sec:forward}

Different versions of the main or child documents
using compilation flags as described in \secref{sec:flags}
can be (permanently) stored in different files
for convenient compilation, viewing and distribution.
To this end, the package defines a command
to pass on compilation to a different file:

%%%%%%%%%%%%%%%%%%%%%%%%%%%%%%%%%%%%%%%%
\DescribeMacro{\childdocforward}
The command |\childdocforward| redirects processing to
another source file:
%
\begin{center}
\begin{tabular}{l}
|\input{childdoc.def}|\\
|\childdocforward[|\textit{main}|]{|\textit{dest}|}|\\
\end{tabular}
\end{center}
%
The argument \textit{dest} is the destination file
(without extension).
It should be the main file or one of the child files.
Note that further \textsf{childdoc} directives
such as |\childdocof| and |\childdocforward|
in the indicated file will be processed in this form.
The optional argument \textit{main}
passes on directly to the main file \textit{main}
while pretending to compile the child \textit{dest}.
This form behaves as if \textit{dest}
issues |\childdocof{|\textit{main}|}| right away,
and no further \textsf{childdoc} directives will be processed.

%%%%%%%%%%%%%%%%%%%%%%%%%%%%%%%%%%%%%%%%
\DescribeMacro{\...prefix}
In the alternative form |\childdocforwardprefix|,
%
\begin{center}
\begin{tabular}{l}
|\input{childdoc.def}|\\
|\childdocforwardprefix[|\textit{main}|]{|\textit{prefix}|}{|\textit{dest}|}|
\end{tabular}
\end{center}
%
the destination file is determined by a pattern
depending on the current file:
To make this work, the current file must be called
`{\textit{prefix}\hspace{0.2em}\textit{suffix}}'
with \textit{prefix} matching precisely the argument.
Processing is then passed on to the file
`{\textit{dest}\hspace{0.2em}\textit{suffix}}'.
Surely, the same effect is achieved by
directly specifying the
argument `{\textit{dest}\hspace{0.2em}\textit{suffix}}'
in the first form.
However, that requires to set up a different file
for each child. With the alternative form of the command
all these files can have exactly the same content
which simplifies setting them up and maintaining them.

For example, the following file |draft.tex|
with a compilation flag |\version| as described in \secref{sec:flags}
compiles the main document as a draft:
%
\begin{center}
\begin{tabular}{l}
|\def\version{draft}|\\
|\input{childdoc.def}|\\
|\childdocforward{|\textit{main}|}|
\end{tabular}
\end{center}
%
Likewise, the following files |final|\textit{nn}|.tex|
compile the final version of the child document
|child|\textit{nn}|.tex|:
%
\begin{center}
\begin{tabular}{l}
|\def\version{final}|\\
|\input{childdoc.def}|\\
|\childdocforwardprefix{final}{child}|
\end{tabular}
\end{center}
%

Note that when several versions of a main file and/or of each child file
are to be generated, it may be convenient to set up a |Makefile| or
shell script to automatise the process.

%%%%%%%%%%%%%%%%%%%%%%%%%%%%%%%%%%%%%%%%%%%%%%%%%%%%%%%%%%%%%%%%%%%%%%%%%%%%%%%%
\subsection{Command Line Processing}
\label{sec:commandline}

The effect of redirection files can also be achieved by invoking
the \LaTeX{} compiler with a more elaborate command line.
Most conveniently this should be done as part
of a shell script or a |Makefile|.

When using \textsf{childdoc} in the main file, the following
command lines effectively perform a redirection
(note that depending on the shell being used,
backslashes may have to be doubled: `|\|' $\to$ `|\\|'):
%
\begin{center}
|... -jobname "|\textit{target}|" |\\|"|[\textit{flags}]%
|\input{childdoc.def}\childdocforward[|\textit{main}|]{|\textit{dest}|}"|
\end{center}
%
Here \textit{target} is the name of the output file,
\textit{main} is the name of the main file
and \textit{dest} is the name of the main or child file to be processed
(all filenames without extensions).
The optional argument \textit{main} can be omitted
if \textit{main} matches \textit{dest}.
Optionally, compilation \textit{flags} can be defined via |\def| commands.
This command line makes the \TeX{} engine believe
it is compiling the file \textit{target}
whose content is specified as the latter parameter.
The provided code then forwards the processing to
\textit{main} or \textit{dest} as described in \secref{sec:forward}.

%%%%%%%%%%%%%%%%%%%%%%%%%%%%%%%%%%%%%%%%%%%%%%%%%%%%%%%%%%%%%%%%%%%%%%%%%%%%%%%%
\subsection{Include by Input}
\label{sec:input}

Including child documents by |\include| has some restrictions by design.
Most notably, the content of a child document always occupies
its own set of pages; pages cannot be shared between child documents.
Usually, this behaviour makes perfect sense
because each child document contain an essential part of the document.
However, in some situations it may be desirable to compose
a document from a collection of parts
without having mandatory page breaks between then.
For this case, the package
provides a mechanism to include parts
by |\input| which can also be processed individually.
However, by construction this mechanism
requires manual handling of the content to be output.

%%%%%%%%%%%%%%%%%%%%%%%%%%%%%%%%%%%%%%%%
\DescribeMacro{\ifchilddocmanual}
The main file should be prepared as usual, see \secref{sec:include}.
However, the document body must make a distinction
between processing of an individual part and of the main document, e.g.:
%
\begin{center}
\begin{tabular}{l}
|\ifchilddocmanual|\\
|\input{\childdocname}|\\
|\||else|\\
\textit{document body with }|\input{|\textit{part}|}|\\
|\||fi|
\end{tabular}
\end{center}
%
The conditional |\ifchilddocmanual| is true whenever
a part to be included by |\input| is being compiled,
and the name of the part is stored in |\childdocname|.

%%%%%%%%%%%%%%%%%%%%%%%%%%%%%%%%%%%%%%%%
\DescribeMacro{\childdocby}
Each part to be included by |\input| should start with:
%
\begin{center}
\begin{tabular}{l}
|\input{childdoc.def}|\\
|\childdocby{|\textit{main}|}|\\
\end{tabular}
\end{center}
%
The directive |\childdocby| is similar to |\childdocof|
described in \secref{sec:include},
but the subsequent selection of content must be done manually.
To that end, both |\ifchilddoc| and |\ifchilddocmanual|
will be true upon processing of a part,
and the name of the part is stored in |\childdocname|.
Note that |\jobname| will be set to the filename of the current part
so that each part receives an individual |.aux| file
that does not interfere with the |.aux| file(s) of the main document.
This behaviour can be altered by the alternative form
|\childdocby[*]{|\textit{main}|}| (with a non-empty optional argument)
which uses the |.aux| file of the main document
by setting |\jobname| to \textit{main}.

%%%%%%%%%%%%%%%%%%%%%%%%%%%%%%%%%%%%%%%%%%%%%%%%%%%%%%%%%%%%%%%%%%%%%%%%%%%%%%%%
\subsection{Driver Development}
\label{sec:driver}

The \textsf{childdoc} mechanism can also be use for the development
of definition files such as \LaTeX{} styles or classes.
This case differs from the above setup with multiple parts
included by |\include| in that no |\includeonly| should be invoked.
This can be achieved by starting the include file
(before |\ProvidesPackage|) with:
%
\begin{center}
\begin{tabular}{l}
|\input{childdoc.def}|\\
|\childdocforward{|\textit{main}|}|\\
\end{tabular}
\end{center}
%
or alternatively with:
%
\begin{center}
\begin{tabular}{l}
|\input{childdoc.def}|\\
|\childdocby{|\textit{main}|}|\\
\end{tabular}
\end{center}
%
Both forms have slightly different effects as described above.
The main file is prepared as usual, see \secref{sec:include}.

%%%%%%%%%%%%%%%%%%%%%%%%%%%%%%%%%%%%%%%%%%%%%%%%%%%%%%%%%%%%%%%%%%%%%%%%%%%%%%%%
\subsection{Legacy Detection}
\label{sec:detection}

The directive |\childdocmain| in the main file can detect
whether the complete document or merely a child is to be compiled
even without using the directive |\childdocof|.
This method is deprecated because it is less robust
and there is no compelling reason to use it;
it is merely provided for backward compatibility
and it may be removed in future versions.

If the detection mechanism is to be used,
it is mandatory to correctly specify
the filename of the main file as the argument of |\childdocmain|:
%
\begin{center}
\begin{tabular}{l}
|\input{childdoc.def}|\\
|\childdocmain{|\textit{main}|}|\\
\end{tabular}
\end{center}
%
If |\jobname| does not match the argument \textit{main} of |\childdocmain|,
it is assumed that |\jobname| points to the child file to be compiled.
When using |\childdocmain| with the main file specified as argument,
it suffices to start a child file
with just |\input{|\textit{main}|}|
without loading of the package and using |\childdocof|.
If instead all processing is done
with the appropriate \textsf{childdoc} directives,
the argument of \textit{main} of |\childdocmain| can be empty.

An alternative version of the command line processing described
in \secref{sec:commandline} using the detection mechanism reads:
%
\begin{center}
|... -jobname "|\textit{target}|" "|[\textit{flags}]%
[|\def\jobname{|\textit{dest}|}|]|\input{|\textit{main}|}"|
\end{center}

%%%%%%%%%%%%%%%%%%%%%%%%%%%%%%%%%%%%%%%%%%%%%%%%%%%%%%%%%%%%%%%%%%%%%%%%%%%%%%%%
\subsection{Manual Code}
\label{sec:manual}

In case one cannot be certain whether the definitions file |childdoc.def|
is installed on the target \TeX{} distribution
and one prefers not to ship it,
it is conceivable to paste a few relevant commands into the sources.

To that end, drop all statements |\input{childdoc.def}|
and perform the replacements as outlined below.
Instead of |\childdocmain{|\textit{main}|}| add the following code
to the top of the main file:
%
\begin{center}
\begin{tabular}{l}
|\||ifdefined\childdocname\endinput\||fi\newif\ifchilddoc|\\
|\edef\childdocname{\scantokens\expandafter{\jobname\noexpand}}|\\
|\def\childdocmain{|\textit{main}|}\||ifx\childdocmain\childdocname\||else|\\
|\childdoctrue\includeonly{\childdocname}\let\jobname\childdocmain\||fi|\\
\end{tabular}
\end{center}
%
Instead of |\childdocof{|\textit{main}|}| just include the main file
at the top of each child file:
%
\begin{center}
|\input{|\textit{main}|}|
\end{center}
%
A simple redirection |\childdocforward{|\textit{dest}|}| is achieved by:
%
\begin{center}
|\def\jobname{|\textit{dest}|}\input{\jobname}|
\end{center}
%
The redirection with prefix
|\childdocforwardprefix[|\textit{prefix}|]{|\textit{dest}|}|
is accomplished by:
%
\begin{center}
\begin{tabular}{l}
|{\edef\jobname{\scantokens\expandafter{\jobname\noexpand}}|\\
|\def\redirectjob |\textit{prefix}|#1~~~{\gdef\jobname{|\textit{dest}|#1}}|\\
|\expandafter\redirectjob\jobname~~~}\input{\jobname}|
\end{tabular}
\end{center}

In an alternative approach,
child documents can be compiled by a specific command line
without additional code or specific definitions:
%
\begin{center}
|... -jobname "|\textit{target}|" "|[\textit{flags}]%
|\includeonly{|\textit{dest}|}\input{|\textit{main}|}"|
\end{center}
%

%%%%%%%%%%%%%%%%%%%%%%%%%%%%%%%%%%%%%%%%%%%%%%%%%%%%%%%%%%%%%%%%%%%%%%%%%%%%%%%%
%%%%%%%%%%%%%%%%%%%%%%%%%%%%%%%%%%%%%%%%%%%%%%%%%%%%%%%%%%%%%%%%%%%%%%%%%%%%%%%%
\section{Information}

%%%%%%%%%%%%%%%%%%%%%%%%%%%%%%%%%%%%%%%%%%%%%%%%%%%%%%%%%%%%%%%%%%%%%%%%%%%%%%%%
\subsection{Copyright}

Copyright \copyright{} 2017--2018 Niklas Beisert

This work may be distributed and/or modified under the
conditions of the \LaTeX{} Project Public License, either version 1.3
of this license or (at your option) any later version.
The latest version of this license is in
  \url{http://www.latex-project.org/lppl.txt}
and version 1.3 or later is part of all distributions of \LaTeX{}
version 2005/12/01 or later.

This work has the LPPL maintenance status `maintained'.

The Current Maintainer of this work is Niklas Beisert.

This work consists of the files |README.txt|, |childdoc.ins| and |childdoc.dtx|
as well as the derived files |childdoc.def|, |cdocsamp.tex|
with |cdocsch1.tex|, |cdocsch2.tex|, |cdocspt3.tex|, |cdocspt4.tex|,
|cdocsdrf.tex|, |cdocsfn1.tex|, |cdocsfn2.tex|
as well as |childdoc.pdf|.

%%%%%%%%%%%%%%%%%%%%%%%%%%%%%%%%%%%%%%%%%%%%%%%%%%%%%%%%%%%%%%%%%%%%%%%%%%%%%%%%
\subsection{Files and Installation}

The package consists of the files:
%
\begin{center}
\begin{tabular}{ll}
    |README.txt|   & readme file \\
    |childdoc.ins| & installation file \\
    |childdoc.dtx| & source file \\
    |childdoc.def| & definition file \\
    |cdocsamp.tex| & sample main file \\
    |cdocsch1.tex| & sample include file \\
    |cdocsch2.tex| & sample include file \\
    |cdocspt3.tex| & sample part file \\
    |cdocspt4.tex| & sample part file \\
    |cdocsdrf.tex| & sample redirection file \\
    |cdocsfn1.tex| & sample redirection file \\
    |cdocsfn2.tex| & sample redirection file \\
    |childdoc.pdf| & manual
\end{tabular}
\end{center}
%
The distribution consists of the files
|README.txt|, |childdoc.ins| and |childdoc.dtx|.
%
\begin{itemize}
\item
Run (pdf)\LaTeX{} on |childdoc.dtx|
to compile the manual |childdoc.pdf| (this file).
\item
Run \LaTeX{} on |childdoc.ins| to create the definitions file |childdoc.def|
and the sample |cdocsamp.tex| with include files
|cdocsch1.tex|, |cdocsch2.tex|, |cdocspt3.tex|, |cdocspt4.tex|,
|cdocsdrf.tex|, |cdocsfn1.tex|, |cdocsfn2.tex|.
Then copy the file |childdoc.def| to an appropriate directory of your \LaTeX{}
distribution, e.g.\ \textit{texmf-root}|/tex/latex/childdoc|.
\end{itemize}

%%%%%%%%%%%%%%%%%%%%%%%%%%%%%%%%%%%%%%%%%%%%%%%%%%%%%%%%%%%%%%%%%%%%%%%%%%%%%%%%
\subsection{Related CTAN Packages}

There are several other packages which offer a similar functionality:
%
\begin{itemize}
\item
The packages
\href{http://ctan.org/pkg/docmute}{\textsf{docmute}},
\href{http://ctan.org/pkg/includex}{\textsf{includex}} and
\href{http://ctan.org/pkg/standalone}{\textsf{standalone}}
provide commands to include only the document body of
a child file thus allowing both files to be compiled individually.
\item
The packages \href{http://ctan.org/pkg/subdocs}{\textsf{subdocs}}
and \href{http://ctan.org/pkg/subfiles}{\textsf{subfiles}}
provide structures in which the main and child documents can be
encapsulated and allowing them to be compiled individually.
The inclusion mechanism is different from the conventional |\include|.
\item
The package \href{http://ctan.org/pkg/combine}{\textsf{combine}}
is an elaborate solution to combine several documents into one.
\end{itemize}
%
See also the CTAN topic \href{http://ctan.org/topic/subdocs}{\textsf{subdocs}}
for further related packages.
The present package differs from the above solutions in that
a document structure constructed with the conventional |\include| mechanism
just needs two extra commands at the top of every file
such that all constituent files can be compiled individually.

%%%%%%%%%%%%%%%%%%%%%%%%%%%%%%%%%%%%%%%%%%%%%%%%%%%%%%%%%%%%%%%%%%%%%%%%%%%%%%%%
%\subsection{Feature Suggestions}
%
%The following is a list of features which may be useful for future
%versions of this package:
%%
%\begin{itemize}
%\item
%\ldots
%\end{itemize}

%%%%%%%%%%%%%%%%%%%%%%%%%%%%%%%%%%%%%%%%%%%%%%%%%%%%%%%%%%%%%%%%%%%%%%%%%%%%%%%%
\subsection{Revision History}

%%%%%%%%%%%%%%%%%%%%%%%%%%%%%%%%%%%%%%%%
\paragraph{v2.0:} 2018/12/30

\begin{itemize}
\item
immediate forward processing
\item
added |\childdocby| mechanism
\item
manual restructured
\end{itemize}

%%%%%%%%%%%%%%%%%%%%%%%%%%%%%%%%%%%%%%%%
\paragraph{v1.6:} 2018/01/17

\begin{itemize}
\item
application for development of include files
\item
corrections to manual
\end{itemize}

%%%%%%%%%%%%%%%%%%%%%%%%%%%%%%%%%%%%%%%%
\paragraph{v1.5:} 2017/05/21

\begin{itemize}
\item
more complete structuring introduced
\item
|\childdocof| introduced
\item
|\childdoc| renamed to |\childdocmain|
\item
|\childredirect| renamed to |\childdocforward| and |\childdocforwardprefix|
and functionality expanded
\end{itemize}

%%%%%%%%%%%%%%%%%%%%%%%%%%%%%%%%%%%%%%%%
\paragraph{v1.0:} 2017/04/27

\begin{itemize}
\item
manual and install package
\item
first version published on CTAN
\end{itemize}

%%%%%%%%%%%%%%%%%%%%%%%%%%%%%%%%%%%%%%%%
\paragraph{v0.6:} 2017/04/26

\begin{itemize}
\item
redirection mechanism added
\end{itemize}

%%%%%%%%%%%%%%%%%%%%%%%%%%%%%%%%%%%%%%%%
\paragraph{v0.5:} 2017/04/26

\begin{itemize}
\item
functionality in definition file
\end{itemize}


%%%%%%%%%%%%%%%%%%%%%%%%%%%%%%%%%%%%%%%%%%%%%%%%%%%%%%%%%%%%%%%%%%%%%%%%%%%%%%%%
%%%%%%%%%%%%%%%%%%%%%%%%%%%%%%%%%%%%%%%%%%%%%%%%%%%%%%%%%%%%%%%%%%%%%%%%%%%%%%%%
%%%%%%%%%%%%%%%%%%%%%%%%%%%%%%%%%%%%%%%%%%%%%%%%%%%%%%%%%%%%%%%%%%%%%%%%%%%%%%%%
\appendix

\settowidth\MacroIndent{\rmfamily\scriptsize 000\ }

 \DocInput{childdoc.dtx}

\end{document}
%</driver>
% \fi
%
% %%%%%%%%%%%%%%%%%%%%%%%%%%%%%%%%%%%%%%%%%%%%%%%%%%%%%%%%%%%%%%%%%%%%%%%%%%%%%%
% %%%%%%%%%%%%%%%%%%%%%%%%%%%%%%%%%%%%%%%%%%%%%%%%%%%%%%%%%%%%%%%%%%%%%%%%%%%%%%
% \section{Sample}
%\iffalse
%<*samplemain>
%\fi
%
% The following presents a sample document
% with two chapters, two parts, a title page,
% a compile flag as well as three forwarding files to set the flag.
% It consists of eight |.tex| files:
% \begin{center}
% \begin{tabular}{ll}
% |cdocsamp.tex|&main file\\
% |cdocsch1.tex|&include file for chapter 1\\
% |cdocsch2.tex|&include file for chapter 2\\
% |cdocspt3.tex|&include file for part 3\\
% |cdocspt4.tex|&include file for part 4\\
% |cdocsdrf.tex|&forwarding file for main file in draft mode\\
% |cdocsfi1.tex|&forwarding file for final version of chapter 1\\
% |cdocsfi2.tex|&forwarding file for final version of chapter 2\\
% \end{tabular}
% \end{center}
% Each of the eight files can be compiled directly by the \LaTeX{} compiler.
%
% %%%%%%%%%%%%%%%%%%%%%%%%%%%%%%%%%%%%%%
% \paragraph{Main File.}
%
% The main file is called |cdocsamp.tex|.
%
% Load the \textsf{childdoc} definitions and
% declare the filename for the main document:
%    \begin{macrocode}
\input{childdoc.def}
\childdocmain{}
%    \end{macrocode}

% Optional override for |\version| flag:
%    \begin{macrocode}
%%\ifchilddoc\else\providecommand{\version}{draft}\fi
%    \end{macrocode}

% Define the default values for the |\version| flag
% (|final| for the main file and |draft| for childs):
%    \begin{macrocode}
\ifchilddoc
\providecommand{\version}{draft}
\else
\providecommand{\version}{final}
\fi
%    \end{macrocode}

% Load the standard document class:
%    \begin{macrocode}
\documentclass[12pt]{article}
%    \end{macrocode}

% Start the document body:
%    \begin{macrocode}
\begin{document}
%    \end{macrocode}

% Declare a title page.
% Print title, part of document being processed and version flag:
%    \begin{macrocode}
\addtocounter{page}{-1}
\begin{center}
{\LARGE\bfseries{}childdoc example\par}
\vspace{1cm}
\ifchilddoc
\ifchilddocmanual part\else chapter\fi:
`\childdocname' of `\childdocjob'\par
\else
main document: `\childdocjob'\par
\fi
version: \version\par
\end{center}
\newpage
%    \end{macrocode}

% Manually include selected file,
% otherwise process as usual:
%    \begin{macrocode}
\ifchilddocmanual
\section*{part `\childdocname'}
\input{\childdocname}
\else
%    \end{macrocode}

% Include the two chapters:
%    \begin{macrocode}
\include{cdocsch1}
\include{cdocsch2}
%    \end{macrocode}

% Include the two parts unless only chapters should be displayed:
%    \begin{macrocode}
\ifchilddoc\else
\section{part three}
\input{cdocspt3}
\section{part four}
\input{cdocspt4}
\fi
%    \end{macrocode}

% Process as usual until here:
%    \begin{macrocode}
\fi
%    \end{macrocode}

% End of document body:
%    \begin{macrocode}
\end{document}
%    \end{macrocode}
%\iffalse
%</samplemain>
%\fi
%
% %%%%%%%%%%%%%%%%%%%%%%%%%%%%%%%%%%%%%%
% \paragraph{Chapter Include Files.}
%
% The include files are called |cdocsch1.tex| and |cdocsch2.tex|.
%
%\iffalse
%<*samplechap1|samplechap2>
%\fi

% Optional override for |\version| flag:
%    \begin{macrocode}
%%\providecommand{\version}{final}
%    \end{macrocode}

% Include the main document:
%    \begin{macrocode}
\input{childdoc.def}
\childdocof{cdocsamp}
%    \end{macrocode}

%\iffalse
%</samplechap1|samplechap2>
%\fi
%
%\iffalse
%<*samplechap1>
%\fi
% Some text for chapter 1:
%    \begin{macrocode}
\section{one}
some text in chapter one
%    \end{macrocode}

%\iffalse
%</samplechap1>
%\fi
% Some text for chapter 2:
%\iffalse
%<*samplechap2>
%\fi
%    \begin{macrocode}
\section{two}
more text in chapter two
%    \end{macrocode}

%\iffalse
%</samplechap2>
%\fi
%
% %%%%%%%%%%%%%%%%%%%%%%%%%%%%%%%%%%%%%%
% \paragraph{Part Include Files.}
%
% The include files are called |cdocspt3.tex| and |cdocspt4.tex|.
%
%\iffalse
%<*samplepart3|samplepart4>
%\fi

% Optional override for |\version| flag:
%    \begin{macrocode}
%%\providecommand{\version}{final}
%    \end{macrocode}

% Include the main document:
%    \begin{macrocode}
\input{childdoc.def}
\childdocby{cdocsamp}
%    \end{macrocode}

%\iffalse
%</samplepart3|samplepart4>
%\fi
%
%\iffalse
%<*samplepart3>
%\fi
% Some text for part 3:
%    \begin{macrocode}
some text in part three
%    \end{macrocode}

%\iffalse
%</samplepart3>
%\fi
% Some text for part 4:
%\iffalse
%<*samplepart4>
%\fi
%    \begin{macrocode}
more text in part four
%    \end{macrocode}

%\iffalse
%</samplepart4>
%\fi
%
% %%%%%%%%%%%%%%%%%%%%%%%%%%%%%%%%%%%%%%
% \paragraph{Forwarding for a Complete Draft.}
%
% The following forwarding file |cdocsdrf.tex|
% compiles the main document in draft mode:
%\iffalse
%<*sampledraft>
%\fi
%    \begin{macrocode}
\def\version{draft}
\input{childdoc.def}
\childdocforward{cdocsamp}
%    \end{macrocode}

%\iffalse
%</sampledraft>
%\fi
%
% %%%%%%%%%%%%%%%%%%%%%%%%%%%%%%%%%%%%%%
% \paragraph{Forwarding for Final Version of the Chapters.}
%
% The following forwarding files |cdocsfn1.tex| and |cdocsfn2.tex|
% (with identical content)
% compile the final versions of the child documents
% |cdocsch1.tex| and |cdocsch2.tex|, respectively:
%\iffalse
%<*samplefinal>
%\fi
%    \begin{macrocode}
\def\version{final}
\input{childdoc.def}
\childdocforwardprefix[cdocsamp]{cdocsfn}{cdocsch}
%    \end{macrocode}

%\iffalse
%</samplefinal>
%\fi
%
% %%%%%%%%%%%%%%%%%%%%%%%%%%%%%%%%%%%%%%
% \paragraph{Command Line Processing.}
%
% The following three command lines generate the output files
% |cdocscld|, |cdocscl1| and |cdocscl2|
% which should be identical to
% |cdocsdrf|, |cdocsch1| and |cdocsfn2|, respectively:
% \begin{center}
% \begin{tabular}{l}
% |latex -jobname cdocscld \|\\
% |  "\def\version{draft}\input{childdoc.def}\childdocforward{cdocsamp}"|\\
% |latex -jobname cdocscl1 \|\\
% |  "\input{childdoc.def}\childdocforward[cdocsamp]{cdocsch1}"|\\
% |latex -jobname cdocscl2 \|\\
% |  "\def\version{final}\input{childdoc.def}\childdocforward{cdocsch2}"|
% \end{tabular}
% \end{center}
% Note that the trailing backslash on each first line
% merely continues the input to the second line
% (for convenient cut ant paste).
% Furthermore, the command |latex| can be replaced by any
% of its alternative versions such as |pdflatex|.
%
% %%%%%%%%%%%%%%%%%%%%%%%%%%%%%%%%%%%%%%%%%%%%%%%%%%%%%%%%%%%%%%%%%%%%%%%%%%%%%%
% %%%%%%%%%%%%%%%%%%%%%%%%%%%%%%%%%%%%%%%%%%%%%%%%%%%%%%%%%%%%%%%%%%%%%%%%%%%%%%
% \section{Implementation}
%\iffalse
%<*package>
%\fi
%
% This section describes the definitions file |childdoc.def|.

% The definitions cannot be loaded using |\usepackage| or |\RequirePackage|
% which has a mechanism to prevent loading a style file more than once.
% When loading the definitions by means of |\input|
% multiple instances have to be prevented manually:
%\iffalse
%This code needs to be before the `\ProvidesFile' directive
%which is defined at the beginning of this file.
%Therefore it is also placed there and commented out here.
%</package>
%<*discard>
%\fi
%    \begin{macrocode}
\ifdefined\childdocmain\endinput\fi
%    \end{macrocode}
%\iffalse
%</discard>
%<*package>
%\fi
%
% \macro{\ifchilddoc}
% \macro{\ifchilddocmanual}
% The conditional |\ifchilddoc| tells whether a
% child (true) or main (false) document is being compiled.
% The conditional |\ifchilddocmanual| tells whether
% the |\includeonly| mechanism is used (false) or
% the selection of child files must be performed manually (true).
% The definitions initialise to false:
%    \begin{macrocode}
\newif\ifchilddoc
\newif\ifchilddocmanual
%    \end{macrocode}

% \macro{\childdocname}
% \macro{\childdocjob}
% The macro |\childdocname| stores the name of the main document
% to be compiled. The macro |\childdocjob| stores the name of
% the document on which the \LaTeX{} compiler was originally invoked.
% The content of |\jobname| cannot be compared
% to filenames specified in the source due to different catcodes.
% The following code rescans |\jobname|, stores the result
% in |\childdocname| and saves a copy in |\childdocjob|:
%    \begin{macrocode}
\edef\childdocname{\scantokens\expandafter{\jobname\noexpand}}
\let\childdocjob\childdocname
%    \end{macrocode}

% \macro{\childdocdisable}
% The macro |\childdocdisable| prevents the main file
% from being processed more than once.
% At this stage, the main document command |\childdocmain|
% is assumed to be called once again where it should do nothing.
% Any subsequent call to it should prevent
% a secondary processing of the main document
% It overwrites the forwarding commands
% |\childdocof| and |\childdocforward|
% with empty macros to prevent further inclusions of the main document:
%    \begin{macrocode}
\newcommand{\childdocdisable}
{
  \renewcommand{\childdocmain}[1]{\renewcommand{\childdocmain}[1]{\endinput}}
  \renewcommand{\childdocof}[1]{}
  \renewcommand{\childdocby}[2][]{}
  \renewcommand{\childdocforward}[2][]{}
  \renewcommand{\childdocdisable}{}
}
%    \end{macrocode}

% \macro{\childdocmain}
% The macro |\childdocmain| is to be called at the top of the main file
% with nothing or the main filename (without extension) as argument.
% First, it breaks loops.
% If the argument is not empty and does not match |\childdocname|
% (which is set by the first inclusion of |childdoc.def|),
% |\ifchilddoc| is set to true, |\includeonly| is applied to the child file
% and |\jobname| is set to the main file
% (for proper handling of |.aux| files):
%    \begin{macrocode}
\newcommand{\childdocmain}[1]
{
  \childdocdisable\childdocmain{}
  \if?#1?\else
    \begingroup
      \def\childdoctmp{#1}
      \ifx\childdoctmp\childdocname
        \def\childdoctmp{}
      \else
        \def\childdoctmp
        {
          \childdoctrue
          \includeonly{\childdocname}
          \def\childdocjob{#1}
          \def\jobname{#1}
        }
      \fi
      \expandafter
    \endgroup
    \childdoctmp
  \fi
}
%    \end{macrocode}

% \macro{\childdocof}
% The command |\childdocof| redirects
% compilation to the main file |#1|.
%    \begin{macrocode}
\newcommand{\childdocof}[1]
{
  \childdocdisable
  \childdoctrue
  \includeonly{\childdocname}
  \def\jobname{#1}
  \def\childdocjob{#1}
  \input{#1}
}
%    \end{macrocode}

% \macro{\childdocby}
% The command |\childdocby| ....
%    \begin{macrocode}
\newcommand{\childdocby}[2][]
{
  \childdocdisable
  \childdoctrue
  \childdocmanualtrue
  \if?#1?\else
    \def\jobname{#2}
  \fi
  \def\childdocjob{#2}
  \input{#2}
  \endinput
}
%    \end{macrocode}

% \macro{\childdocforward}
% The command |\childdocforward| redirects
% compilation to the main file or
% (if the optional argument is given) a child file.
% Parameters are set as if the main file
% or a child file starting with |\childdocof| was compiled.
% Then compilation is handed over to the main file:
%    \begin{macrocode}
\newcommand{\childdocforward}[2][]
{
  \begingroup
    \if?#1?
      \def\childdoctmp
      {
        \def\childdocname{#2}
        \def\childdocjob{#2}
        \def\jobname{#2}
        \input{#2}
        \endinput
      }
    \else
      \def\childdoctmp
      {
        \childdocdisable
        \def\childdocname{#2}
        \childdoctrue
        \includeonly{#2}
        \def\childdocjob{#1}
        \def\jobname{#1}
        \input{#1}
        \endinput
      }
    \fi
    \expandafter
  \endgroup
  \childdoctmp
}
%    \end{macrocode}

% \macro{\childdocforwardprefix}
% The command |\childdocforwardprefix| redirects
% compilation to the main or a child file by means of a pattern.
% The prefix |#1| in the current filename is replaced by |#2|
% and the suffix of the current filename is kept
% (it is assumed that the filename does not contain the substring `|~~~|'
% which is used as a delimiter).
% Compilation is handed over to the new file by |\childdocforward|:
%    \begin{macrocode}
\newcommand{\childdocforwardprefix}[3][]
{
  \begingroup
    \def\childdocextract #2##1~~~{\def\childdoctmp{\childdocforward[#1]{#3##1}}}
    \expandafter\childdocextract\childdocname~~~
    \expandafter
  \endgroup
  \childdoctmp
}
%    \end{macrocode}

% \macro{\childdoc}
% The deprecated macro |\childdoc| is a legacy version of |\childdocmain|:
%    \begin{macrocode}
\newcommand{\childdoc}{\childdocmain}
%    \end{macrocode}

% \macro{\childdocredirect}
% The deprecated macro |\childdocredirect| is a legacy version
% of |\childdocforward| and |\childdocforwardprefix|:
%    \begin{macrocode}
\newcommand{\childdocredirect}[2][]
{
  \begingroup
    \if?#1?
      \def\childdoctmp{\childdocforward{#2}}
    \else
      \def\childdoctmp{\childdocforwardprefix{#1}{#2}}
    \fi
    \expandafter
  \endgroup
  \childdoctmp
}
%    \end{macrocode}

%\iffalse
%</package>
%\fi
%
\endinput
|\\
|\childdocmain{}|\\
\end{tabular}
\end{center}
at the very top of the main \LaTeX{} file,
in particular \emph{before} the |\documentclass| statement!
The argument of |\childdocmain| should be left empty
(but it must be present).

%%%%%%%%%%%%%%%%%%%%%%%%%%%%%%%%%%%%%%%%
\DescribeMacro{\childdocof}
Furthermore, add the commands
\begin{center}
\begin{tabular}{l}
|% \iffalse
%
% childdoc.dtx Copyright (C) 2017-2018 Niklas Beisert
%
% This work may be distributed and/or modified under the
% conditions of the LaTeX Project Public License, either version 1.3
% of this license or (at your option) any later version.
% The latest version of this license is in
%   http://www.latex-project.org/lppl.txt
% and version 1.3 or later is part of all distributions of LaTeX
% version 2005/12/01 or later.
%
% This work has the LPPL maintenance status `maintained'.
%
% The Current Maintainer of this work is Niklas Beisert.
%
% This work consists of the files childdoc.dtx and childdoc.ins
% and the derived files childdoc.def and cdocsamp.tex with
% cdocsch1.tex, cdocsch2.tex, cdocsdrf.tex, cdocsfn1.tex, cdocsfn2.tex.
%
%<package>\ifdefined\childdocmain\endinput\fi
%<package>\ProvidesFile{childdoc.def}[2018/12/30 v2.0 child document driver]
%<samplemain>\ProvidesFile{cdocsamp.tex}[2018/12/30 v2.0 sample for childdoc]
%<*driver>
%\ProvidesFile{childdoc.drv}[2018/12/30 v2.0 childdoc reference manual file]
\PassOptionsToClass{10pt,a4paper}{article}
\documentclass{ltxdoc}

\usepackage[margin=35mm]{geometry}
\usepackage{hyperref}
\usepackage{hyperxmp}
\usepackage[usenames]{color}

\hypersetup{colorlinks=true}
\hypersetup{pdfstartview=FitH}
\hypersetup{pdfpagemode=UseNone}
\hypersetup{pdfsource={}}
\hypersetup{pdflang={en-UK}}
\hypersetup{pdfcopyright={Copyright 2017-2018 Niklas Beisert.
  This work may be distributed and/or modified under the
  conditions of the LaTeX Project Public License, either version 1.3
  of this license or (at your option) any later version.}}
\hypersetup{pdflicenseurl={http://www.latex-project.org/lppl.txt}}
\hypersetup{pdfcontactaddress={ETH Zurich, ITP, HIT K,
  Wolfgang-Pauli-Strasse 27}}
\hypersetup{pdfcontactpostcode={8093}}
\hypersetup{pdfcontactcity={Zurich}}
\hypersetup{pdfcontactcountry={Switzerland}}
\hypersetup{pdfcontactemail={nbeisert@itp.phys.ethz.ch}}
\hypersetup{pdfcontacturl={http://people.phys.ethz.ch/\xmptilde nbeisert/}}

\newcommand{\secref}[1]{\hyperref[#1]{section \ref*{#1}}}

\parskip1ex
\parindent0pt
\let\olditemize\itemize
\def\itemize{\olditemize\parskip0pt}

\begin{document}

\title{The \textsf{childdoc} Package}
\hypersetup{pdftitle={The childdoc Package}}
\author{Niklas Beisert\\[2ex]
  Institut f\"ur Theoretische Physik\\
  Eidgen\"ossische Technische Hochschule Z\"urich\\
  Wolfgang-Pauli-Strasse 27, 8093 Z\"urich, Switzerland\\[1ex]
  \href{mailto:nbeisert@itp.phys.ethz.ch}
  {\texttt{nbeisert@itp.phys.ethz.ch}}}
\hypersetup{pdfauthor={Niklas Beisert}}
\hypersetup{pdfsubject={Manual for the LaTeX2e Package childdoc}}
\date{30 December 2018, \textsf{v2.0}}
\maketitle

\begin{abstract}\noindent
\textsf{childdoc} is a \LaTeXe{} package
that enables the direct compilation
of document sections included by |\include|
to individual files.
\end{abstract}

\begingroup
\parskip0ex
\tableofcontents
\endgroup

%%%%%%%%%%%%%%%%%%%%%%%%%%%%%%%%%%%%%%%%%%%%%%%%%%%%%%%%%%%%%%%%%%%%%%%%%%%%%%%%
%%%%%%%%%%%%%%%%%%%%%%%%%%%%%%%%%%%%%%%%%%%%%%%%%%%%%%%%%%%%%%%%%%%%%%%%%%%%%%%%
\section{Introduction}

\LaTeX{} provides a mechanism to structure a large document (such as a book)
into a main file and several child files (containing the chapters)
using the |\include| command.
This mechanism is beneficial for documents
which span hundreds of pages in order to
make the source file(s) more manageable.
Moreover, compilation can be restricted to
selected child files by means of the |\includeonly| command.
The latter feature can be used to reduce the compilation time while editing
(this was significantly more useful in the earlier days of \LaTeX{})
or to generate a smaller document which is easier to navigate.
Another application of |\includeonly| is to generate
documents consisting of selected parts of the complete document.

However, there are a few drawbacks of the plain |\include| mechanism:
\begin{itemize}
\item
The child files cannot be compiled on their own,
they can only be compiled via the main file.
A naive editing environment
(such as a text editor with an option
to have the current file processed by \LaTeX)
may require one to switch to the main file before compiling;
attempting to compile the child file produces errors.
\item
The main file must be modified (each time)
to adjust the |\includeonly| command
to the present needs. This easily leaves the main file in a messy state.
\item
The generated document will always carry the filename
of the main document. This is inconvenient if
several child files are to be compiled and
to be kept for distribution.
\end{itemize}

The present package provides a simple interface
to make child files individually compilable by \LaTeX{}.
Compiling a child file then has the same effect as compiling
the main file with an |\includeonly| command
to select the appropriate child.
Moreover the generated document will carry the name of the child
rather than the main file.
This resolves all three above issues.

This feature is meant to make the editing of books,
thesis documents and lecture notes somewhat more convenient.
However, the package can also be used efficiently for
composing a series of documents (such as exercise sheets)
which are typically distributed individually.
It then assists the author in generating the individual documents
(potentially in different versions)
as well as a document containing the collected series.
Another application is in developing style files
or other kinds of included material
where compilation of the style file could redirect
to a sample or test file.

%%%%%%%%%%%%%%%%%%%%%%%%%%%%%%%%%%%%%%%%%%%%%%%%%%%%%%%%%%%%%%%%%%%%%%%%%%%%%%%%
%%%%%%%%%%%%%%%%%%%%%%%%%%%%%%%%%%%%%%%%%%%%%%%%%%%%%%%%%%%%%%%%%%%%%%%%%%%%%%%%
\section{Usage}

First of all, the package \textsf{childdoc} is \emph{not} a standard
\LaTeXe{} |.sty| style file! Therefore it needs to be invoked in
a non-standard way.

%%%%%%%%%%%%%%%%%%%%%%%%%%%%%%%%%%%%%%%%%%%%%%%%%%%%%%%%%%%%%%%%%%%%%%%%%%%%%%%%
\subsection{Included Files}
\label{sec:include}

%%%%%%%%%%%%%%%%%%%%%%%%%%%%%%%%%%%%%%%%
\DescribeMacro{\childdocmain}
To use the package, add the commands
\begin{center}
\begin{tabular}{l}
|\input{childdoc.def}|\\
|\childdocmain{}|\\
\end{tabular}
\end{center}
at the very top of the main \LaTeX{} file,
in particular \emph{before} the |\documentclass| statement!
The argument of |\childdocmain| should be left empty
(but it must be present).

%%%%%%%%%%%%%%%%%%%%%%%%%%%%%%%%%%%%%%%%
\DescribeMacro{\childdocof}
Furthermore, add the commands
\begin{center}
\begin{tabular}{l}
|\input{childdoc.def}|\\
|\childdocof{|\textit{main}|}|\\
\end{tabular}
\end{center}
at the top of every child file \textit{child}
which is included by |\include{|\textit{child}|}|
from within the main file
(or at least for those files to be compiled individually).
The argument \textit{main} must be the filename of the main file.

There are a couple of
considerations in setting up the main and child documents:

%%%%%%%%%%%%%%%%%%%%%%%%%%%%%%%%%%%%%%%%
\paragraph{Restrictions.}

Please note the following restrictions:
\begin{itemize}
\item
|\childdocmain| must be called with one argument \textit{main}
to ensure compatibility with earlier version of the package.
It must either be empty (|\childdocmain{}|)
or precisely match the filename of the main file in which it is specified.
See \secref{sec:detection} for further information.
\item
The filename \textit{main} must be specified without the |.tex| extension.
\item
The filename \textit{main} is case sensitive
(even in case-insensitive file systems)
due to internal string comparison.
\item
The argument \textit{main} should be fully expanded, it cannot be a macro.
\item
Subdirectories and special characters should be avoided in filenames.
\item
The command |\childdocmain{|\textit{main}|}| must be followed by a whitespace.
It should not be followed immediately by another command
or by a comment mark `|%|'.
This is because the \TeX{} parser reads the token immediately following
the argument of |\childdocmain| and puts it
at the beginning of every child section;
however, a white\-space is ignored.
\end{itemize}

%%%%%%%%%%%%%%%%%%%%%%%%%%%%%%%%%%%%%%%%
\paragraph{Content of Main File.}

It is advisable to place all content in the child files included by |\include|.
Any output contained in the main file will appear in all child documents
unless suppressed manually;
it cannot be suppressed automatically by the |\includeonly| directive
and thus should normally be avoided.
A method to include some content in the main file
by means of conditional processing is described in \secref{sec:conditional}.

%%%%%%%%%%%%%%%%%%%%%%%%%%%%%%%%%%%%%%%%
\paragraph{Page Numbering.}

When only a part of the document is compiled,
the appropriate numbering of pages
(as well as other status parameters)
is determined from the |.aux| files.
The latter contain information from previous passes.
However this information needs to propagate through
all intermediate child documents.
Therefore the page numbering in child documents may well
be inconsistent until the complete document is compiled at least once.

A useful (if unconventional) way to always ensure a consistent
page numbering is to restart the numbering in each child document
and denote the pages by `\textit{child}|.|\textit{page}'
where \textit{child} represents the chapter/section number of the child file.
This can be achieved by the command
|\numberwithin{page}{|\textit{child}|}|
of the \textsf{amsmath} package
where \textit{child} can be |chapter| or |section|
depending on the chosen structuring.
Alternatively, one can modify the macro |\thepage| appropriately
and reset the counter |page| at the start of each child file.

%%%%%%%%%%%%%%%%%%%%%%%%%%%%%%%%%%%%%%%%%%%%%%%%%%%%%%%%%%%%%%%%%%%%%%%%%%%%%%%%
\subsection{Conditional Processing}
\label{sec:conditional}

The package provides a mechanism to compile different versions
of a document. To customise the versions further some conditional processing
can come in handy to distinguish which version is being compiled.
The package provides two macros to describe the compilation context:

%%%%%%%%%%%%%%%%%%%%%%%%%%%%%%%%%%%%%%%%
\DescribeMacro{\ifchilddoc}
The conditional |\ifchilddoc| distinguishes between the compilation of
child documents and the main document:
%
\begin{center}
|\ifchilddoc |\textit{child-code}| |[|\||else |\textit{main-code}]| \||fi|
\end{center}

%%%%%%%%%%%%%%%%%%%%%%%%%%%%%%%%%%%%%%%%
\DescribeMacro{\childdocname}
\DescribeMacro{\childdocjob}
The macro |\childdocname| contains the filename (without extension)
of the main or child file being processed.
Note that |\childdocjob| will always contain the name of the main file.

%%%%%%%%%%%%%%%%%%%%%%%%%%%%%%%%%%%%%%%%
\paragraph{Title Page.}

Conditional processing can be used to include a title or banner page
in the main document when proper precautions are taken.
Importantly, the code in the main file should ensure that the page counter
(as well as other status parameters which are stored in the |.aux| files)
takes the same value after the conditional processing.
Otherwise the page numbers may take divergent values
depending on which part is compiled.

For example, a title page could be declared by:
%
\begin{center}
\begin{tabular}{l}
|\ifchilddoc\||else|\\
|\addtocounter{page}{-1}|\\
\textit{code for title page}\\
|\newpage|\\
|\||fi|
\end{tabular}
\end{center}
%
A banner page for the child documents can be generated by:
%
\begin{center}
\begin{tabular}{l}
|\ifchilddoc|\\
|\addtocounter{page}{-1}|\\
\textit{code for banner page}\\
|\newpage|\\
|\||fi|
\end{tabular}
\end{center}
%
Here one could write a message such as:
\begin{center}
|This is the part \childdocname{} of \childdocjob{}.|
\end{center}

%%%%%%%%%%%%%%%%%%%%%%%%%%%%%%%%%%%%%%%%%%%%%%%%%%%%%%%%%%%%%%%%%%%%%%%%%%%%%%%%
\subsection{Flags}
\label{sec:flags}

The package makes it easy to generate different versions
of the main or child documents.
To this end compilation flags can be defined
and assigned different default values.
They will be particularly useful in conjunction
with the forwarding mechanism described in \secref{sec:forward}.

For example, it may be useful to have a flag |\version|
which can be set to |draft| or |final|.
The document source will contain some conditional code
depending on the value of |\version|.
Suppose further, the flag should default to |final| for the main file
and to |draft| for child files
which is a natural assignment for editing the document.
This is achieved by placing the following code
in the preamble of the main document
(below the |\childdocmain| directive):
%
\begin{center}
\begin{tabular}{l}
|\ifchilddoc|\\
|\providecommand{\version}{draft}|\\
|\||else|\\
|\providecommand{\version}{final}|\\
|\||fi|
\end{tabular}
\end{center}
%
The definition by |\providecommand| makes sure
that previous definitions are not overwritten.
Further statements |\providecommand{\version}{...}|
can thus be added before the above code to override it.

For the main file, one might add a line
(between |\childdocmain| and the above block)
%
\begin{center}
|%\ifchilddoc\||else\providecommand{\version}{draft}\||fi|
\end{center}
%
which can be uncommented to produce a draft version.
Likewise one can add a line to the very top of a child file
(above the |\childdocof{|\textit{main}|}| directive)
%
\begin{center}
|%\providecommand{\version}{final}|
\end{center}
%
which can be uncommented to produce the final version of this child document.

%%%%%%%%%%%%%%%%%%%%%%%%%%%%%%%%%%%%%%%%%%%%%%%%%%%%%%%%%%%%%%%%%%%%%%%%%%%%%%%%
\subsection{Forwarding}
\label{sec:forward}

Different versions of the main or child documents
using compilation flags as described in \secref{sec:flags}
can be (permanently) stored in different files
for convenient compilation, viewing and distribution.
To this end, the package defines a command
to pass on compilation to a different file:

%%%%%%%%%%%%%%%%%%%%%%%%%%%%%%%%%%%%%%%%
\DescribeMacro{\childdocforward}
The command |\childdocforward| redirects processing to
another source file:
%
\begin{center}
\begin{tabular}{l}
|\input{childdoc.def}|\\
|\childdocforward[|\textit{main}|]{|\textit{dest}|}|\\
\end{tabular}
\end{center}
%
The argument \textit{dest} is the destination file
(without extension).
It should be the main file or one of the child files.
Note that further \textsf{childdoc} directives
such as |\childdocof| and |\childdocforward|
in the indicated file will be processed in this form.
The optional argument \textit{main}
passes on directly to the main file \textit{main}
while pretending to compile the child \textit{dest}.
This form behaves as if \textit{dest}
issues |\childdocof{|\textit{main}|}| right away,
and no further \textsf{childdoc} directives will be processed.

%%%%%%%%%%%%%%%%%%%%%%%%%%%%%%%%%%%%%%%%
\DescribeMacro{\...prefix}
In the alternative form |\childdocforwardprefix|,
%
\begin{center}
\begin{tabular}{l}
|\input{childdoc.def}|\\
|\childdocforwardprefix[|\textit{main}|]{|\textit{prefix}|}{|\textit{dest}|}|
\end{tabular}
\end{center}
%
the destination file is determined by a pattern
depending on the current file:
To make this work, the current file must be called
`{\textit{prefix}\hspace{0.2em}\textit{suffix}}'
with \textit{prefix} matching precisely the argument.
Processing is then passed on to the file
`{\textit{dest}\hspace{0.2em}\textit{suffix}}'.
Surely, the same effect is achieved by
directly specifying the
argument `{\textit{dest}\hspace{0.2em}\textit{suffix}}'
in the first form.
However, that requires to set up a different file
for each child. With the alternative form of the command
all these files can have exactly the same content
which simplifies setting them up and maintaining them.

For example, the following file |draft.tex|
with a compilation flag |\version| as described in \secref{sec:flags}
compiles the main document as a draft:
%
\begin{center}
\begin{tabular}{l}
|\def\version{draft}|\\
|\input{childdoc.def}|\\
|\childdocforward{|\textit{main}|}|
\end{tabular}
\end{center}
%
Likewise, the following files |final|\textit{nn}|.tex|
compile the final version of the child document
|child|\textit{nn}|.tex|:
%
\begin{center}
\begin{tabular}{l}
|\def\version{final}|\\
|\input{childdoc.def}|\\
|\childdocforwardprefix{final}{child}|
\end{tabular}
\end{center}
%

Note that when several versions of a main file and/or of each child file
are to be generated, it may be convenient to set up a |Makefile| or
shell script to automatise the process.

%%%%%%%%%%%%%%%%%%%%%%%%%%%%%%%%%%%%%%%%%%%%%%%%%%%%%%%%%%%%%%%%%%%%%%%%%%%%%%%%
\subsection{Command Line Processing}
\label{sec:commandline}

The effect of redirection files can also be achieved by invoking
the \LaTeX{} compiler with a more elaborate command line.
Most conveniently this should be done as part
of a shell script or a |Makefile|.

When using \textsf{childdoc} in the main file, the following
command lines effectively perform a redirection
(note that depending on the shell being used,
backslashes may have to be doubled: `|\|' $\to$ `|\\|'):
%
\begin{center}
|... -jobname "|\textit{target}|" |\\|"|[\textit{flags}]%
|\input{childdoc.def}\childdocforward[|\textit{main}|]{|\textit{dest}|}"|
\end{center}
%
Here \textit{target} is the name of the output file,
\textit{main} is the name of the main file
and \textit{dest} is the name of the main or child file to be processed
(all filenames without extensions).
The optional argument \textit{main} can be omitted
if \textit{main} matches \textit{dest}.
Optionally, compilation \textit{flags} can be defined via |\def| commands.
This command line makes the \TeX{} engine believe
it is compiling the file \textit{target}
whose content is specified as the latter parameter.
The provided code then forwards the processing to
\textit{main} or \textit{dest} as described in \secref{sec:forward}.

%%%%%%%%%%%%%%%%%%%%%%%%%%%%%%%%%%%%%%%%%%%%%%%%%%%%%%%%%%%%%%%%%%%%%%%%%%%%%%%%
\subsection{Include by Input}
\label{sec:input}

Including child documents by |\include| has some restrictions by design.
Most notably, the content of a child document always occupies
its own set of pages; pages cannot be shared between child documents.
Usually, this behaviour makes perfect sense
because each child document contain an essential part of the document.
However, in some situations it may be desirable to compose
a document from a collection of parts
without having mandatory page breaks between then.
For this case, the package
provides a mechanism to include parts
by |\input| which can also be processed individually.
However, by construction this mechanism
requires manual handling of the content to be output.

%%%%%%%%%%%%%%%%%%%%%%%%%%%%%%%%%%%%%%%%
\DescribeMacro{\ifchilddocmanual}
The main file should be prepared as usual, see \secref{sec:include}.
However, the document body must make a distinction
between processing of an individual part and of the main document, e.g.:
%
\begin{center}
\begin{tabular}{l}
|\ifchilddocmanual|\\
|\input{\childdocname}|\\
|\||else|\\
\textit{document body with }|\input{|\textit{part}|}|\\
|\||fi|
\end{tabular}
\end{center}
%
The conditional |\ifchilddocmanual| is true whenever
a part to be included by |\input| is being compiled,
and the name of the part is stored in |\childdocname|.

%%%%%%%%%%%%%%%%%%%%%%%%%%%%%%%%%%%%%%%%
\DescribeMacro{\childdocby}
Each part to be included by |\input| should start with:
%
\begin{center}
\begin{tabular}{l}
|\input{childdoc.def}|\\
|\childdocby{|\textit{main}|}|\\
\end{tabular}
\end{center}
%
The directive |\childdocby| is similar to |\childdocof|
described in \secref{sec:include},
but the subsequent selection of content must be done manually.
To that end, both |\ifchilddoc| and |\ifchilddocmanual|
will be true upon processing of a part,
and the name of the part is stored in |\childdocname|.
Note that |\jobname| will be set to the filename of the current part
so that each part receives an individual |.aux| file
that does not interfere with the |.aux| file(s) of the main document.
This behaviour can be altered by the alternative form
|\childdocby[*]{|\textit{main}|}| (with a non-empty optional argument)
which uses the |.aux| file of the main document
by setting |\jobname| to \textit{main}.

%%%%%%%%%%%%%%%%%%%%%%%%%%%%%%%%%%%%%%%%%%%%%%%%%%%%%%%%%%%%%%%%%%%%%%%%%%%%%%%%
\subsection{Driver Development}
\label{sec:driver}

The \textsf{childdoc} mechanism can also be use for the development
of definition files such as \LaTeX{} styles or classes.
This case differs from the above setup with multiple parts
included by |\include| in that no |\includeonly| should be invoked.
This can be achieved by starting the include file
(before |\ProvidesPackage|) with:
%
\begin{center}
\begin{tabular}{l}
|\input{childdoc.def}|\\
|\childdocforward{|\textit{main}|}|\\
\end{tabular}
\end{center}
%
or alternatively with:
%
\begin{center}
\begin{tabular}{l}
|\input{childdoc.def}|\\
|\childdocby{|\textit{main}|}|\\
\end{tabular}
\end{center}
%
Both forms have slightly different effects as described above.
The main file is prepared as usual, see \secref{sec:include}.

%%%%%%%%%%%%%%%%%%%%%%%%%%%%%%%%%%%%%%%%%%%%%%%%%%%%%%%%%%%%%%%%%%%%%%%%%%%%%%%%
\subsection{Legacy Detection}
\label{sec:detection}

The directive |\childdocmain| in the main file can detect
whether the complete document or merely a child is to be compiled
even without using the directive |\childdocof|.
This method is deprecated because it is less robust
and there is no compelling reason to use it;
it is merely provided for backward compatibility
and it may be removed in future versions.

If the detection mechanism is to be used,
it is mandatory to correctly specify
the filename of the main file as the argument of |\childdocmain|:
%
\begin{center}
\begin{tabular}{l}
|\input{childdoc.def}|\\
|\childdocmain{|\textit{main}|}|\\
\end{tabular}
\end{center}
%
If |\jobname| does not match the argument \textit{main} of |\childdocmain|,
it is assumed that |\jobname| points to the child file to be compiled.
When using |\childdocmain| with the main file specified as argument,
it suffices to start a child file
with just |\input{|\textit{main}|}|
without loading of the package and using |\childdocof|.
If instead all processing is done
with the appropriate \textsf{childdoc} directives,
the argument of \textit{main} of |\childdocmain| can be empty.

An alternative version of the command line processing described
in \secref{sec:commandline} using the detection mechanism reads:
%
\begin{center}
|... -jobname "|\textit{target}|" "|[\textit{flags}]%
[|\def\jobname{|\textit{dest}|}|]|\input{|\textit{main}|}"|
\end{center}

%%%%%%%%%%%%%%%%%%%%%%%%%%%%%%%%%%%%%%%%%%%%%%%%%%%%%%%%%%%%%%%%%%%%%%%%%%%%%%%%
\subsection{Manual Code}
\label{sec:manual}

In case one cannot be certain whether the definitions file |childdoc.def|
is installed on the target \TeX{} distribution
and one prefers not to ship it,
it is conceivable to paste a few relevant commands into the sources.

To that end, drop all statements |\input{childdoc.def}|
and perform the replacements as outlined below.
Instead of |\childdocmain{|\textit{main}|}| add the following code
to the top of the main file:
%
\begin{center}
\begin{tabular}{l}
|\||ifdefined\childdocname\endinput\||fi\newif\ifchilddoc|\\
|\edef\childdocname{\scantokens\expandafter{\jobname\noexpand}}|\\
|\def\childdocmain{|\textit{main}|}\||ifx\childdocmain\childdocname\||else|\\
|\childdoctrue\includeonly{\childdocname}\let\jobname\childdocmain\||fi|\\
\end{tabular}
\end{center}
%
Instead of |\childdocof{|\textit{main}|}| just include the main file
at the top of each child file:
%
\begin{center}
|\input{|\textit{main}|}|
\end{center}
%
A simple redirection |\childdocforward{|\textit{dest}|}| is achieved by:
%
\begin{center}
|\def\jobname{|\textit{dest}|}\input{\jobname}|
\end{center}
%
The redirection with prefix
|\childdocforwardprefix[|\textit{prefix}|]{|\textit{dest}|}|
is accomplished by:
%
\begin{center}
\begin{tabular}{l}
|{\edef\jobname{\scantokens\expandafter{\jobname\noexpand}}|\\
|\def\redirectjob |\textit{prefix}|#1~~~{\gdef\jobname{|\textit{dest}|#1}}|\\
|\expandafter\redirectjob\jobname~~~}\input{\jobname}|
\end{tabular}
\end{center}

In an alternative approach,
child documents can be compiled by a specific command line
without additional code or specific definitions:
%
\begin{center}
|... -jobname "|\textit{target}|" "|[\textit{flags}]%
|\includeonly{|\textit{dest}|}\input{|\textit{main}|}"|
\end{center}
%

%%%%%%%%%%%%%%%%%%%%%%%%%%%%%%%%%%%%%%%%%%%%%%%%%%%%%%%%%%%%%%%%%%%%%%%%%%%%%%%%
%%%%%%%%%%%%%%%%%%%%%%%%%%%%%%%%%%%%%%%%%%%%%%%%%%%%%%%%%%%%%%%%%%%%%%%%%%%%%%%%
\section{Information}

%%%%%%%%%%%%%%%%%%%%%%%%%%%%%%%%%%%%%%%%%%%%%%%%%%%%%%%%%%%%%%%%%%%%%%%%%%%%%%%%
\subsection{Copyright}

Copyright \copyright{} 2017--2018 Niklas Beisert

This work may be distributed and/or modified under the
conditions of the \LaTeX{} Project Public License, either version 1.3
of this license or (at your option) any later version.
The latest version of this license is in
  \url{http://www.latex-project.org/lppl.txt}
and version 1.3 or later is part of all distributions of \LaTeX{}
version 2005/12/01 or later.

This work has the LPPL maintenance status `maintained'.

The Current Maintainer of this work is Niklas Beisert.

This work consists of the files |README.txt|, |childdoc.ins| and |childdoc.dtx|
as well as the derived files |childdoc.def|, |cdocsamp.tex|
with |cdocsch1.tex|, |cdocsch2.tex|, |cdocspt3.tex|, |cdocspt4.tex|,
|cdocsdrf.tex|, |cdocsfn1.tex|, |cdocsfn2.tex|
as well as |childdoc.pdf|.

%%%%%%%%%%%%%%%%%%%%%%%%%%%%%%%%%%%%%%%%%%%%%%%%%%%%%%%%%%%%%%%%%%%%%%%%%%%%%%%%
\subsection{Files and Installation}

The package consists of the files:
%
\begin{center}
\begin{tabular}{ll}
    |README.txt|   & readme file \\
    |childdoc.ins| & installation file \\
    |childdoc.dtx| & source file \\
    |childdoc.def| & definition file \\
    |cdocsamp.tex| & sample main file \\
    |cdocsch1.tex| & sample include file \\
    |cdocsch2.tex| & sample include file \\
    |cdocspt3.tex| & sample part file \\
    |cdocspt4.tex| & sample part file \\
    |cdocsdrf.tex| & sample redirection file \\
    |cdocsfn1.tex| & sample redirection file \\
    |cdocsfn2.tex| & sample redirection file \\
    |childdoc.pdf| & manual
\end{tabular}
\end{center}
%
The distribution consists of the files
|README.txt|, |childdoc.ins| and |childdoc.dtx|.
%
\begin{itemize}
\item
Run (pdf)\LaTeX{} on |childdoc.dtx|
to compile the manual |childdoc.pdf| (this file).
\item
Run \LaTeX{} on |childdoc.ins| to create the definitions file |childdoc.def|
and the sample |cdocsamp.tex| with include files
|cdocsch1.tex|, |cdocsch2.tex|, |cdocspt3.tex|, |cdocspt4.tex|,
|cdocsdrf.tex|, |cdocsfn1.tex|, |cdocsfn2.tex|.
Then copy the file |childdoc.def| to an appropriate directory of your \LaTeX{}
distribution, e.g.\ \textit{texmf-root}|/tex/latex/childdoc|.
\end{itemize}

%%%%%%%%%%%%%%%%%%%%%%%%%%%%%%%%%%%%%%%%%%%%%%%%%%%%%%%%%%%%%%%%%%%%%%%%%%%%%%%%
\subsection{Related CTAN Packages}

There are several other packages which offer a similar functionality:
%
\begin{itemize}
\item
The packages
\href{http://ctan.org/pkg/docmute}{\textsf{docmute}},
\href{http://ctan.org/pkg/includex}{\textsf{includex}} and
\href{http://ctan.org/pkg/standalone}{\textsf{standalone}}
provide commands to include only the document body of
a child file thus allowing both files to be compiled individually.
\item
The packages \href{http://ctan.org/pkg/subdocs}{\textsf{subdocs}}
and \href{http://ctan.org/pkg/subfiles}{\textsf{subfiles}}
provide structures in which the main and child documents can be
encapsulated and allowing them to be compiled individually.
The inclusion mechanism is different from the conventional |\include|.
\item
The package \href{http://ctan.org/pkg/combine}{\textsf{combine}}
is an elaborate solution to combine several documents into one.
\end{itemize}
%
See also the CTAN topic \href{http://ctan.org/topic/subdocs}{\textsf{subdocs}}
for further related packages.
The present package differs from the above solutions in that
a document structure constructed with the conventional |\include| mechanism
just needs two extra commands at the top of every file
such that all constituent files can be compiled individually.

%%%%%%%%%%%%%%%%%%%%%%%%%%%%%%%%%%%%%%%%%%%%%%%%%%%%%%%%%%%%%%%%%%%%%%%%%%%%%%%%
%\subsection{Feature Suggestions}
%
%The following is a list of features which may be useful for future
%versions of this package:
%%
%\begin{itemize}
%\item
%\ldots
%\end{itemize}

%%%%%%%%%%%%%%%%%%%%%%%%%%%%%%%%%%%%%%%%%%%%%%%%%%%%%%%%%%%%%%%%%%%%%%%%%%%%%%%%
\subsection{Revision History}

%%%%%%%%%%%%%%%%%%%%%%%%%%%%%%%%%%%%%%%%
\paragraph{v2.0:} 2018/12/30

\begin{itemize}
\item
immediate forward processing
\item
added |\childdocby| mechanism
\item
manual restructured
\end{itemize}

%%%%%%%%%%%%%%%%%%%%%%%%%%%%%%%%%%%%%%%%
\paragraph{v1.6:} 2018/01/17

\begin{itemize}
\item
application for development of include files
\item
corrections to manual
\end{itemize}

%%%%%%%%%%%%%%%%%%%%%%%%%%%%%%%%%%%%%%%%
\paragraph{v1.5:} 2017/05/21

\begin{itemize}
\item
more complete structuring introduced
\item
|\childdocof| introduced
\item
|\childdoc| renamed to |\childdocmain|
\item
|\childredirect| renamed to |\childdocforward| and |\childdocforwardprefix|
and functionality expanded
\end{itemize}

%%%%%%%%%%%%%%%%%%%%%%%%%%%%%%%%%%%%%%%%
\paragraph{v1.0:} 2017/04/27

\begin{itemize}
\item
manual and install package
\item
first version published on CTAN
\end{itemize}

%%%%%%%%%%%%%%%%%%%%%%%%%%%%%%%%%%%%%%%%
\paragraph{v0.6:} 2017/04/26

\begin{itemize}
\item
redirection mechanism added
\end{itemize}

%%%%%%%%%%%%%%%%%%%%%%%%%%%%%%%%%%%%%%%%
\paragraph{v0.5:} 2017/04/26

\begin{itemize}
\item
functionality in definition file
\end{itemize}


%%%%%%%%%%%%%%%%%%%%%%%%%%%%%%%%%%%%%%%%%%%%%%%%%%%%%%%%%%%%%%%%%%%%%%%%%%%%%%%%
%%%%%%%%%%%%%%%%%%%%%%%%%%%%%%%%%%%%%%%%%%%%%%%%%%%%%%%%%%%%%%%%%%%%%%%%%%%%%%%%
%%%%%%%%%%%%%%%%%%%%%%%%%%%%%%%%%%%%%%%%%%%%%%%%%%%%%%%%%%%%%%%%%%%%%%%%%%%%%%%%
\appendix

\settowidth\MacroIndent{\rmfamily\scriptsize 000\ }

 \DocInput{childdoc.dtx}

\end{document}
%</driver>
% \fi
%
% %%%%%%%%%%%%%%%%%%%%%%%%%%%%%%%%%%%%%%%%%%%%%%%%%%%%%%%%%%%%%%%%%%%%%%%%%%%%%%
% %%%%%%%%%%%%%%%%%%%%%%%%%%%%%%%%%%%%%%%%%%%%%%%%%%%%%%%%%%%%%%%%%%%%%%%%%%%%%%
% \section{Sample}
%\iffalse
%<*samplemain>
%\fi
%
% The following presents a sample document
% with two chapters, two parts, a title page,
% a compile flag as well as three forwarding files to set the flag.
% It consists of eight |.tex| files:
% \begin{center}
% \begin{tabular}{ll}
% |cdocsamp.tex|&main file\\
% |cdocsch1.tex|&include file for chapter 1\\
% |cdocsch2.tex|&include file for chapter 2\\
% |cdocspt3.tex|&include file for part 3\\
% |cdocspt4.tex|&include file for part 4\\
% |cdocsdrf.tex|&forwarding file for main file in draft mode\\
% |cdocsfi1.tex|&forwarding file for final version of chapter 1\\
% |cdocsfi2.tex|&forwarding file for final version of chapter 2\\
% \end{tabular}
% \end{center}
% Each of the eight files can be compiled directly by the \LaTeX{} compiler.
%
% %%%%%%%%%%%%%%%%%%%%%%%%%%%%%%%%%%%%%%
% \paragraph{Main File.}
%
% The main file is called |cdocsamp.tex|.
%
% Load the \textsf{childdoc} definitions and
% declare the filename for the main document:
%    \begin{macrocode}
\input{childdoc.def}
\childdocmain{}
%    \end{macrocode}

% Optional override for |\version| flag:
%    \begin{macrocode}
%%\ifchilddoc\else\providecommand{\version}{draft}\fi
%    \end{macrocode}

% Define the default values for the |\version| flag
% (|final| for the main file and |draft| for childs):
%    \begin{macrocode}
\ifchilddoc
\providecommand{\version}{draft}
\else
\providecommand{\version}{final}
\fi
%    \end{macrocode}

% Load the standard document class:
%    \begin{macrocode}
\documentclass[12pt]{article}
%    \end{macrocode}

% Start the document body:
%    \begin{macrocode}
\begin{document}
%    \end{macrocode}

% Declare a title page.
% Print title, part of document being processed and version flag:
%    \begin{macrocode}
\addtocounter{page}{-1}
\begin{center}
{\LARGE\bfseries{}childdoc example\par}
\vspace{1cm}
\ifchilddoc
\ifchilddocmanual part\else chapter\fi:
`\childdocname' of `\childdocjob'\par
\else
main document: `\childdocjob'\par
\fi
version: \version\par
\end{center}
\newpage
%    \end{macrocode}

% Manually include selected file,
% otherwise process as usual:
%    \begin{macrocode}
\ifchilddocmanual
\section*{part `\childdocname'}
\input{\childdocname}
\else
%    \end{macrocode}

% Include the two chapters:
%    \begin{macrocode}
\include{cdocsch1}
\include{cdocsch2}
%    \end{macrocode}

% Include the two parts unless only chapters should be displayed:
%    \begin{macrocode}
\ifchilddoc\else
\section{part three}
\input{cdocspt3}
\section{part four}
\input{cdocspt4}
\fi
%    \end{macrocode}

% Process as usual until here:
%    \begin{macrocode}
\fi
%    \end{macrocode}

% End of document body:
%    \begin{macrocode}
\end{document}
%    \end{macrocode}
%\iffalse
%</samplemain>
%\fi
%
% %%%%%%%%%%%%%%%%%%%%%%%%%%%%%%%%%%%%%%
% \paragraph{Chapter Include Files.}
%
% The include files are called |cdocsch1.tex| and |cdocsch2.tex|.
%
%\iffalse
%<*samplechap1|samplechap2>
%\fi

% Optional override for |\version| flag:
%    \begin{macrocode}
%%\providecommand{\version}{final}
%    \end{macrocode}

% Include the main document:
%    \begin{macrocode}
\input{childdoc.def}
\childdocof{cdocsamp}
%    \end{macrocode}

%\iffalse
%</samplechap1|samplechap2>
%\fi
%
%\iffalse
%<*samplechap1>
%\fi
% Some text for chapter 1:
%    \begin{macrocode}
\section{one}
some text in chapter one
%    \end{macrocode}

%\iffalse
%</samplechap1>
%\fi
% Some text for chapter 2:
%\iffalse
%<*samplechap2>
%\fi
%    \begin{macrocode}
\section{two}
more text in chapter two
%    \end{macrocode}

%\iffalse
%</samplechap2>
%\fi
%
% %%%%%%%%%%%%%%%%%%%%%%%%%%%%%%%%%%%%%%
% \paragraph{Part Include Files.}
%
% The include files are called |cdocspt3.tex| and |cdocspt4.tex|.
%
%\iffalse
%<*samplepart3|samplepart4>
%\fi

% Optional override for |\version| flag:
%    \begin{macrocode}
%%\providecommand{\version}{final}
%    \end{macrocode}

% Include the main document:
%    \begin{macrocode}
\input{childdoc.def}
\childdocby{cdocsamp}
%    \end{macrocode}

%\iffalse
%</samplepart3|samplepart4>
%\fi
%
%\iffalse
%<*samplepart3>
%\fi
% Some text for part 3:
%    \begin{macrocode}
some text in part three
%    \end{macrocode}

%\iffalse
%</samplepart3>
%\fi
% Some text for part 4:
%\iffalse
%<*samplepart4>
%\fi
%    \begin{macrocode}
more text in part four
%    \end{macrocode}

%\iffalse
%</samplepart4>
%\fi
%
% %%%%%%%%%%%%%%%%%%%%%%%%%%%%%%%%%%%%%%
% \paragraph{Forwarding for a Complete Draft.}
%
% The following forwarding file |cdocsdrf.tex|
% compiles the main document in draft mode:
%\iffalse
%<*sampledraft>
%\fi
%    \begin{macrocode}
\def\version{draft}
\input{childdoc.def}
\childdocforward{cdocsamp}
%    \end{macrocode}

%\iffalse
%</sampledraft>
%\fi
%
% %%%%%%%%%%%%%%%%%%%%%%%%%%%%%%%%%%%%%%
% \paragraph{Forwarding for Final Version of the Chapters.}
%
% The following forwarding files |cdocsfn1.tex| and |cdocsfn2.tex|
% (with identical content)
% compile the final versions of the child documents
% |cdocsch1.tex| and |cdocsch2.tex|, respectively:
%\iffalse
%<*samplefinal>
%\fi
%    \begin{macrocode}
\def\version{final}
\input{childdoc.def}
\childdocforwardprefix[cdocsamp]{cdocsfn}{cdocsch}
%    \end{macrocode}

%\iffalse
%</samplefinal>
%\fi
%
% %%%%%%%%%%%%%%%%%%%%%%%%%%%%%%%%%%%%%%
% \paragraph{Command Line Processing.}
%
% The following three command lines generate the output files
% |cdocscld|, |cdocscl1| and |cdocscl2|
% which should be identical to
% |cdocsdrf|, |cdocsch1| and |cdocsfn2|, respectively:
% \begin{center}
% \begin{tabular}{l}
% |latex -jobname cdocscld \|\\
% |  "\def\version{draft}\input{childdoc.def}\childdocforward{cdocsamp}"|\\
% |latex -jobname cdocscl1 \|\\
% |  "\input{childdoc.def}\childdocforward[cdocsamp]{cdocsch1}"|\\
% |latex -jobname cdocscl2 \|\\
% |  "\def\version{final}\input{childdoc.def}\childdocforward{cdocsch2}"|
% \end{tabular}
% \end{center}
% Note that the trailing backslash on each first line
% merely continues the input to the second line
% (for convenient cut ant paste).
% Furthermore, the command |latex| can be replaced by any
% of its alternative versions such as |pdflatex|.
%
% %%%%%%%%%%%%%%%%%%%%%%%%%%%%%%%%%%%%%%%%%%%%%%%%%%%%%%%%%%%%%%%%%%%%%%%%%%%%%%
% %%%%%%%%%%%%%%%%%%%%%%%%%%%%%%%%%%%%%%%%%%%%%%%%%%%%%%%%%%%%%%%%%%%%%%%%%%%%%%
% \section{Implementation}
%\iffalse
%<*package>
%\fi
%
% This section describes the definitions file |childdoc.def|.

% The definitions cannot be loaded using |\usepackage| or |\RequirePackage|
% which has a mechanism to prevent loading a style file more than once.
% When loading the definitions by means of |\input|
% multiple instances have to be prevented manually:
%\iffalse
%This code needs to be before the `\ProvidesFile' directive
%which is defined at the beginning of this file.
%Therefore it is also placed there and commented out here.
%</package>
%<*discard>
%\fi
%    \begin{macrocode}
\ifdefined\childdocmain\endinput\fi
%    \end{macrocode}
%\iffalse
%</discard>
%<*package>
%\fi
%
% \macro{\ifchilddoc}
% \macro{\ifchilddocmanual}
% The conditional |\ifchilddoc| tells whether a
% child (true) or main (false) document is being compiled.
% The conditional |\ifchilddocmanual| tells whether
% the |\includeonly| mechanism is used (false) or
% the selection of child files must be performed manually (true).
% The definitions initialise to false:
%    \begin{macrocode}
\newif\ifchilddoc
\newif\ifchilddocmanual
%    \end{macrocode}

% \macro{\childdocname}
% \macro{\childdocjob}
% The macro |\childdocname| stores the name of the main document
% to be compiled. The macro |\childdocjob| stores the name of
% the document on which the \LaTeX{} compiler was originally invoked.
% The content of |\jobname| cannot be compared
% to filenames specified in the source due to different catcodes.
% The following code rescans |\jobname|, stores the result
% in |\childdocname| and saves a copy in |\childdocjob|:
%    \begin{macrocode}
\edef\childdocname{\scantokens\expandafter{\jobname\noexpand}}
\let\childdocjob\childdocname
%    \end{macrocode}

% \macro{\childdocdisable}
% The macro |\childdocdisable| prevents the main file
% from being processed more than once.
% At this stage, the main document command |\childdocmain|
% is assumed to be called once again where it should do nothing.
% Any subsequent call to it should prevent
% a secondary processing of the main document
% It overwrites the forwarding commands
% |\childdocof| and |\childdocforward|
% with empty macros to prevent further inclusions of the main document:
%    \begin{macrocode}
\newcommand{\childdocdisable}
{
  \renewcommand{\childdocmain}[1]{\renewcommand{\childdocmain}[1]{\endinput}}
  \renewcommand{\childdocof}[1]{}
  \renewcommand{\childdocby}[2][]{}
  \renewcommand{\childdocforward}[2][]{}
  \renewcommand{\childdocdisable}{}
}
%    \end{macrocode}

% \macro{\childdocmain}
% The macro |\childdocmain| is to be called at the top of the main file
% with nothing or the main filename (without extension) as argument.
% First, it breaks loops.
% If the argument is not empty and does not match |\childdocname|
% (which is set by the first inclusion of |childdoc.def|),
% |\ifchilddoc| is set to true, |\includeonly| is applied to the child file
% and |\jobname| is set to the main file
% (for proper handling of |.aux| files):
%    \begin{macrocode}
\newcommand{\childdocmain}[1]
{
  \childdocdisable\childdocmain{}
  \if?#1?\else
    \begingroup
      \def\childdoctmp{#1}
      \ifx\childdoctmp\childdocname
        \def\childdoctmp{}
      \else
        \def\childdoctmp
        {
          \childdoctrue
          \includeonly{\childdocname}
          \def\childdocjob{#1}
          \def\jobname{#1}
        }
      \fi
      \expandafter
    \endgroup
    \childdoctmp
  \fi
}
%    \end{macrocode}

% \macro{\childdocof}
% The command |\childdocof| redirects
% compilation to the main file |#1|.
%    \begin{macrocode}
\newcommand{\childdocof}[1]
{
  \childdocdisable
  \childdoctrue
  \includeonly{\childdocname}
  \def\jobname{#1}
  \def\childdocjob{#1}
  \input{#1}
}
%    \end{macrocode}

% \macro{\childdocby}
% The command |\childdocby| ....
%    \begin{macrocode}
\newcommand{\childdocby}[2][]
{
  \childdocdisable
  \childdoctrue
  \childdocmanualtrue
  \if?#1?\else
    \def\jobname{#2}
  \fi
  \def\childdocjob{#2}
  \input{#2}
  \endinput
}
%    \end{macrocode}

% \macro{\childdocforward}
% The command |\childdocforward| redirects
% compilation to the main file or
% (if the optional argument is given) a child file.
% Parameters are set as if the main file
% or a child file starting with |\childdocof| was compiled.
% Then compilation is handed over to the main file:
%    \begin{macrocode}
\newcommand{\childdocforward}[2][]
{
  \begingroup
    \if?#1?
      \def\childdoctmp
      {
        \def\childdocname{#2}
        \def\childdocjob{#2}
        \def\jobname{#2}
        \input{#2}
        \endinput
      }
    \else
      \def\childdoctmp
      {
        \childdocdisable
        \def\childdocname{#2}
        \childdoctrue
        \includeonly{#2}
        \def\childdocjob{#1}
        \def\jobname{#1}
        \input{#1}
        \endinput
      }
    \fi
    \expandafter
  \endgroup
  \childdoctmp
}
%    \end{macrocode}

% \macro{\childdocforwardprefix}
% The command |\childdocforwardprefix| redirects
% compilation to the main or a child file by means of a pattern.
% The prefix |#1| in the current filename is replaced by |#2|
% and the suffix of the current filename is kept
% (it is assumed that the filename does not contain the substring `|~~~|'
% which is used as a delimiter).
% Compilation is handed over to the new file by |\childdocforward|:
%    \begin{macrocode}
\newcommand{\childdocforwardprefix}[3][]
{
  \begingroup
    \def\childdocextract #2##1~~~{\def\childdoctmp{\childdocforward[#1]{#3##1}}}
    \expandafter\childdocextract\childdocname~~~
    \expandafter
  \endgroup
  \childdoctmp
}
%    \end{macrocode}

% \macro{\childdoc}
% The deprecated macro |\childdoc| is a legacy version of |\childdocmain|:
%    \begin{macrocode}
\newcommand{\childdoc}{\childdocmain}
%    \end{macrocode}

% \macro{\childdocredirect}
% The deprecated macro |\childdocredirect| is a legacy version
% of |\childdocforward| and |\childdocforwardprefix|:
%    \begin{macrocode}
\newcommand{\childdocredirect}[2][]
{
  \begingroup
    \if?#1?
      \def\childdoctmp{\childdocforward{#2}}
    \else
      \def\childdoctmp{\childdocforwardprefix{#1}{#2}}
    \fi
    \expandafter
  \endgroup
  \childdoctmp
}
%    \end{macrocode}

%\iffalse
%</package>
%\fi
%
\endinput
|\\
|\childdocof{|\textit{main}|}|\\
\end{tabular}
\end{center}
at the top of every child file \textit{child}
which is included by |\include{|\textit{child}|}|
from within the main file
(or at least for those files to be compiled individually).
The argument \textit{main} must be the filename of the main file.

There are a couple of
considerations in setting up the main and child documents:

%%%%%%%%%%%%%%%%%%%%%%%%%%%%%%%%%%%%%%%%
\paragraph{Restrictions.}

Please note the following restrictions:
\begin{itemize}
\item
|\childdocmain| must be called with one argument \textit{main}
to ensure compatibility with earlier version of the package.
It must either be empty (|\childdocmain{}|)
or precisely match the filename of the main file in which it is specified.
See \secref{sec:detection} for further information.
\item
The filename \textit{main} must be specified without the |.tex| extension.
\item
The filename \textit{main} is case sensitive
(even in case-insensitive file systems)
due to internal string comparison.
\item
The argument \textit{main} should be fully expanded, it cannot be a macro.
\item
Subdirectories and special characters should be avoided in filenames.
\item
The command |\childdocmain{|\textit{main}|}| must be followed by a whitespace.
It should not be followed immediately by another command
or by a comment mark `|%|'.
This is because the \TeX{} parser reads the token immediately following
the argument of |\childdocmain| and puts it
at the beginning of every child section;
however, a white\-space is ignored.
\end{itemize}

%%%%%%%%%%%%%%%%%%%%%%%%%%%%%%%%%%%%%%%%
\paragraph{Content of Main File.}

It is advisable to place all content in the child files included by |\include|.
Any output contained in the main file will appear in all child documents
unless suppressed manually;
it cannot be suppressed automatically by the |\includeonly| directive
and thus should normally be avoided.
A method to include some content in the main file
by means of conditional processing is described in \secref{sec:conditional}.

%%%%%%%%%%%%%%%%%%%%%%%%%%%%%%%%%%%%%%%%
\paragraph{Page Numbering.}

When only a part of the document is compiled,
the appropriate numbering of pages
(as well as other status parameters)
is determined from the |.aux| files.
The latter contain information from previous passes.
However this information needs to propagate through
all intermediate child documents.
Therefore the page numbering in child documents may well
be inconsistent until the complete document is compiled at least once.

A useful (if unconventional) way to always ensure a consistent
page numbering is to restart the numbering in each child document
and denote the pages by `\textit{child}|.|\textit{page}'
where \textit{child} represents the chapter/section number of the child file.
This can be achieved by the command
|\numberwithin{page}{|\textit{child}|}|
of the \textsf{amsmath} package
where \textit{child} can be |chapter| or |section|
depending on the chosen structuring.
Alternatively, one can modify the macro |\thepage| appropriately
and reset the counter |page| at the start of each child file.

%%%%%%%%%%%%%%%%%%%%%%%%%%%%%%%%%%%%%%%%%%%%%%%%%%%%%%%%%%%%%%%%%%%%%%%%%%%%%%%%
\subsection{Conditional Processing}
\label{sec:conditional}

The package provides a mechanism to compile different versions
of a document. To customise the versions further some conditional processing
can come in handy to distinguish which version is being compiled.
The package provides two macros to describe the compilation context:

%%%%%%%%%%%%%%%%%%%%%%%%%%%%%%%%%%%%%%%%
\DescribeMacro{\ifchilddoc}
The conditional |\ifchilddoc| distinguishes between the compilation of
child documents and the main document:
%
\begin{center}
|\ifchilddoc |\textit{child-code}| |[|\||else |\textit{main-code}]| \||fi|
\end{center}

%%%%%%%%%%%%%%%%%%%%%%%%%%%%%%%%%%%%%%%%
\DescribeMacro{\childdocname}
\DescribeMacro{\childdocjob}
The macro |\childdocname| contains the filename (without extension)
of the main or child file being processed.
Note that |\childdocjob| will always contain the name of the main file.

%%%%%%%%%%%%%%%%%%%%%%%%%%%%%%%%%%%%%%%%
\paragraph{Title Page.}

Conditional processing can be used to include a title or banner page
in the main document when proper precautions are taken.
Importantly, the code in the main file should ensure that the page counter
(as well as other status parameters which are stored in the |.aux| files)
takes the same value after the conditional processing.
Otherwise the page numbers may take divergent values
depending on which part is compiled.

For example, a title page could be declared by:
%
\begin{center}
\begin{tabular}{l}
|\ifchilddoc\||else|\\
|\addtocounter{page}{-1}|\\
\textit{code for title page}\\
|\newpage|\\
|\||fi|
\end{tabular}
\end{center}
%
A banner page for the child documents can be generated by:
%
\begin{center}
\begin{tabular}{l}
|\ifchilddoc|\\
|\addtocounter{page}{-1}|\\
\textit{code for banner page}\\
|\newpage|\\
|\||fi|
\end{tabular}
\end{center}
%
Here one could write a message such as:
\begin{center}
|This is the part \childdocname{} of \childdocjob{}.|
\end{center}

%%%%%%%%%%%%%%%%%%%%%%%%%%%%%%%%%%%%%%%%%%%%%%%%%%%%%%%%%%%%%%%%%%%%%%%%%%%%%%%%
\subsection{Flags}
\label{sec:flags}

The package makes it easy to generate different versions
of the main or child documents.
To this end compilation flags can be defined
and assigned different default values.
They will be particularly useful in conjunction
with the forwarding mechanism described in \secref{sec:forward}.

For example, it may be useful to have a flag |\version|
which can be set to |draft| or |final|.
The document source will contain some conditional code
depending on the value of |\version|.
Suppose further, the flag should default to |final| for the main file
and to |draft| for child files
which is a natural assignment for editing the document.
This is achieved by placing the following code
in the preamble of the main document
(below the |\childdocmain| directive):
%
\begin{center}
\begin{tabular}{l}
|\ifchilddoc|\\
|\providecommand{\version}{draft}|\\
|\||else|\\
|\providecommand{\version}{final}|\\
|\||fi|
\end{tabular}
\end{center}
%
The definition by |\providecommand| makes sure
that previous definitions are not overwritten.
Further statements |\providecommand{\version}{...}|
can thus be added before the above code to override it.

For the main file, one might add a line
(between |\childdocmain| and the above block)
%
\begin{center}
|%\ifchilddoc\||else\providecommand{\version}{draft}\||fi|
\end{center}
%
which can be uncommented to produce a draft version.
Likewise one can add a line to the very top of a child file
(above the |\childdocof{|\textit{main}|}| directive)
%
\begin{center}
|%\providecommand{\version}{final}|
\end{center}
%
which can be uncommented to produce the final version of this child document.

%%%%%%%%%%%%%%%%%%%%%%%%%%%%%%%%%%%%%%%%%%%%%%%%%%%%%%%%%%%%%%%%%%%%%%%%%%%%%%%%
\subsection{Forwarding}
\label{sec:forward}

Different versions of the main or child documents
using compilation flags as described in \secref{sec:flags}
can be (permanently) stored in different files
for convenient compilation, viewing and distribution.
To this end, the package defines a command
to pass on compilation to a different file:

%%%%%%%%%%%%%%%%%%%%%%%%%%%%%%%%%%%%%%%%
\DescribeMacro{\childdocforward}
The command |\childdocforward| redirects processing to
another source file:
%
\begin{center}
\begin{tabular}{l}
|% \iffalse
%
% childdoc.dtx Copyright (C) 2017-2018 Niklas Beisert
%
% This work may be distributed and/or modified under the
% conditions of the LaTeX Project Public License, either version 1.3
% of this license or (at your option) any later version.
% The latest version of this license is in
%   http://www.latex-project.org/lppl.txt
% and version 1.3 or later is part of all distributions of LaTeX
% version 2005/12/01 or later.
%
% This work has the LPPL maintenance status `maintained'.
%
% The Current Maintainer of this work is Niklas Beisert.
%
% This work consists of the files childdoc.dtx and childdoc.ins
% and the derived files childdoc.def and cdocsamp.tex with
% cdocsch1.tex, cdocsch2.tex, cdocsdrf.tex, cdocsfn1.tex, cdocsfn2.tex.
%
%<package>\ifdefined\childdocmain\endinput\fi
%<package>\ProvidesFile{childdoc.def}[2018/12/30 v2.0 child document driver]
%<samplemain>\ProvidesFile{cdocsamp.tex}[2018/12/30 v2.0 sample for childdoc]
%<*driver>
%\ProvidesFile{childdoc.drv}[2018/12/30 v2.0 childdoc reference manual file]
\PassOptionsToClass{10pt,a4paper}{article}
\documentclass{ltxdoc}

\usepackage[margin=35mm]{geometry}
\usepackage{hyperref}
\usepackage{hyperxmp}
\usepackage[usenames]{color}

\hypersetup{colorlinks=true}
\hypersetup{pdfstartview=FitH}
\hypersetup{pdfpagemode=UseNone}
\hypersetup{pdfsource={}}
\hypersetup{pdflang={en-UK}}
\hypersetup{pdfcopyright={Copyright 2017-2018 Niklas Beisert.
  This work may be distributed and/or modified under the
  conditions of the LaTeX Project Public License, either version 1.3
  of this license or (at your option) any later version.}}
\hypersetup{pdflicenseurl={http://www.latex-project.org/lppl.txt}}
\hypersetup{pdfcontactaddress={ETH Zurich, ITP, HIT K,
  Wolfgang-Pauli-Strasse 27}}
\hypersetup{pdfcontactpostcode={8093}}
\hypersetup{pdfcontactcity={Zurich}}
\hypersetup{pdfcontactcountry={Switzerland}}
\hypersetup{pdfcontactemail={nbeisert@itp.phys.ethz.ch}}
\hypersetup{pdfcontacturl={http://people.phys.ethz.ch/\xmptilde nbeisert/}}

\newcommand{\secref}[1]{\hyperref[#1]{section \ref*{#1}}}

\parskip1ex
\parindent0pt
\let\olditemize\itemize
\def\itemize{\olditemize\parskip0pt}

\begin{document}

\title{The \textsf{childdoc} Package}
\hypersetup{pdftitle={The childdoc Package}}
\author{Niklas Beisert\\[2ex]
  Institut f\"ur Theoretische Physik\\
  Eidgen\"ossische Technische Hochschule Z\"urich\\
  Wolfgang-Pauli-Strasse 27, 8093 Z\"urich, Switzerland\\[1ex]
  \href{mailto:nbeisert@itp.phys.ethz.ch}
  {\texttt{nbeisert@itp.phys.ethz.ch}}}
\hypersetup{pdfauthor={Niklas Beisert}}
\hypersetup{pdfsubject={Manual for the LaTeX2e Package childdoc}}
\date{30 December 2018, \textsf{v2.0}}
\maketitle

\begin{abstract}\noindent
\textsf{childdoc} is a \LaTeXe{} package
that enables the direct compilation
of document sections included by |\include|
to individual files.
\end{abstract}

\begingroup
\parskip0ex
\tableofcontents
\endgroup

%%%%%%%%%%%%%%%%%%%%%%%%%%%%%%%%%%%%%%%%%%%%%%%%%%%%%%%%%%%%%%%%%%%%%%%%%%%%%%%%
%%%%%%%%%%%%%%%%%%%%%%%%%%%%%%%%%%%%%%%%%%%%%%%%%%%%%%%%%%%%%%%%%%%%%%%%%%%%%%%%
\section{Introduction}

\LaTeX{} provides a mechanism to structure a large document (such as a book)
into a main file and several child files (containing the chapters)
using the |\include| command.
This mechanism is beneficial for documents
which span hundreds of pages in order to
make the source file(s) more manageable.
Moreover, compilation can be restricted to
selected child files by means of the |\includeonly| command.
The latter feature can be used to reduce the compilation time while editing
(this was significantly more useful in the earlier days of \LaTeX{})
or to generate a smaller document which is easier to navigate.
Another application of |\includeonly| is to generate
documents consisting of selected parts of the complete document.

However, there are a few drawbacks of the plain |\include| mechanism:
\begin{itemize}
\item
The child files cannot be compiled on their own,
they can only be compiled via the main file.
A naive editing environment
(such as a text editor with an option
to have the current file processed by \LaTeX)
may require one to switch to the main file before compiling;
attempting to compile the child file produces errors.
\item
The main file must be modified (each time)
to adjust the |\includeonly| command
to the present needs. This easily leaves the main file in a messy state.
\item
The generated document will always carry the filename
of the main document. This is inconvenient if
several child files are to be compiled and
to be kept for distribution.
\end{itemize}

The present package provides a simple interface
to make child files individually compilable by \LaTeX{}.
Compiling a child file then has the same effect as compiling
the main file with an |\includeonly| command
to select the appropriate child.
Moreover the generated document will carry the name of the child
rather than the main file.
This resolves all three above issues.

This feature is meant to make the editing of books,
thesis documents and lecture notes somewhat more convenient.
However, the package can also be used efficiently for
composing a series of documents (such as exercise sheets)
which are typically distributed individually.
It then assists the author in generating the individual documents
(potentially in different versions)
as well as a document containing the collected series.
Another application is in developing style files
or other kinds of included material
where compilation of the style file could redirect
to a sample or test file.

%%%%%%%%%%%%%%%%%%%%%%%%%%%%%%%%%%%%%%%%%%%%%%%%%%%%%%%%%%%%%%%%%%%%%%%%%%%%%%%%
%%%%%%%%%%%%%%%%%%%%%%%%%%%%%%%%%%%%%%%%%%%%%%%%%%%%%%%%%%%%%%%%%%%%%%%%%%%%%%%%
\section{Usage}

First of all, the package \textsf{childdoc} is \emph{not} a standard
\LaTeXe{} |.sty| style file! Therefore it needs to be invoked in
a non-standard way.

%%%%%%%%%%%%%%%%%%%%%%%%%%%%%%%%%%%%%%%%%%%%%%%%%%%%%%%%%%%%%%%%%%%%%%%%%%%%%%%%
\subsection{Included Files}
\label{sec:include}

%%%%%%%%%%%%%%%%%%%%%%%%%%%%%%%%%%%%%%%%
\DescribeMacro{\childdocmain}
To use the package, add the commands
\begin{center}
\begin{tabular}{l}
|\input{childdoc.def}|\\
|\childdocmain{}|\\
\end{tabular}
\end{center}
at the very top of the main \LaTeX{} file,
in particular \emph{before} the |\documentclass| statement!
The argument of |\childdocmain| should be left empty
(but it must be present).

%%%%%%%%%%%%%%%%%%%%%%%%%%%%%%%%%%%%%%%%
\DescribeMacro{\childdocof}
Furthermore, add the commands
\begin{center}
\begin{tabular}{l}
|\input{childdoc.def}|\\
|\childdocof{|\textit{main}|}|\\
\end{tabular}
\end{center}
at the top of every child file \textit{child}
which is included by |\include{|\textit{child}|}|
from within the main file
(or at least for those files to be compiled individually).
The argument \textit{main} must be the filename of the main file.

There are a couple of
considerations in setting up the main and child documents:

%%%%%%%%%%%%%%%%%%%%%%%%%%%%%%%%%%%%%%%%
\paragraph{Restrictions.}

Please note the following restrictions:
\begin{itemize}
\item
|\childdocmain| must be called with one argument \textit{main}
to ensure compatibility with earlier version of the package.
It must either be empty (|\childdocmain{}|)
or precisely match the filename of the main file in which it is specified.
See \secref{sec:detection} for further information.
\item
The filename \textit{main} must be specified without the |.tex| extension.
\item
The filename \textit{main} is case sensitive
(even in case-insensitive file systems)
due to internal string comparison.
\item
The argument \textit{main} should be fully expanded, it cannot be a macro.
\item
Subdirectories and special characters should be avoided in filenames.
\item
The command |\childdocmain{|\textit{main}|}| must be followed by a whitespace.
It should not be followed immediately by another command
or by a comment mark `|%|'.
This is because the \TeX{} parser reads the token immediately following
the argument of |\childdocmain| and puts it
at the beginning of every child section;
however, a white\-space is ignored.
\end{itemize}

%%%%%%%%%%%%%%%%%%%%%%%%%%%%%%%%%%%%%%%%
\paragraph{Content of Main File.}

It is advisable to place all content in the child files included by |\include|.
Any output contained in the main file will appear in all child documents
unless suppressed manually;
it cannot be suppressed automatically by the |\includeonly| directive
and thus should normally be avoided.
A method to include some content in the main file
by means of conditional processing is described in \secref{sec:conditional}.

%%%%%%%%%%%%%%%%%%%%%%%%%%%%%%%%%%%%%%%%
\paragraph{Page Numbering.}

When only a part of the document is compiled,
the appropriate numbering of pages
(as well as other status parameters)
is determined from the |.aux| files.
The latter contain information from previous passes.
However this information needs to propagate through
all intermediate child documents.
Therefore the page numbering in child documents may well
be inconsistent until the complete document is compiled at least once.

A useful (if unconventional) way to always ensure a consistent
page numbering is to restart the numbering in each child document
and denote the pages by `\textit{child}|.|\textit{page}'
where \textit{child} represents the chapter/section number of the child file.
This can be achieved by the command
|\numberwithin{page}{|\textit{child}|}|
of the \textsf{amsmath} package
where \textit{child} can be |chapter| or |section|
depending on the chosen structuring.
Alternatively, one can modify the macro |\thepage| appropriately
and reset the counter |page| at the start of each child file.

%%%%%%%%%%%%%%%%%%%%%%%%%%%%%%%%%%%%%%%%%%%%%%%%%%%%%%%%%%%%%%%%%%%%%%%%%%%%%%%%
\subsection{Conditional Processing}
\label{sec:conditional}

The package provides a mechanism to compile different versions
of a document. To customise the versions further some conditional processing
can come in handy to distinguish which version is being compiled.
The package provides two macros to describe the compilation context:

%%%%%%%%%%%%%%%%%%%%%%%%%%%%%%%%%%%%%%%%
\DescribeMacro{\ifchilddoc}
The conditional |\ifchilddoc| distinguishes between the compilation of
child documents and the main document:
%
\begin{center}
|\ifchilddoc |\textit{child-code}| |[|\||else |\textit{main-code}]| \||fi|
\end{center}

%%%%%%%%%%%%%%%%%%%%%%%%%%%%%%%%%%%%%%%%
\DescribeMacro{\childdocname}
\DescribeMacro{\childdocjob}
The macro |\childdocname| contains the filename (without extension)
of the main or child file being processed.
Note that |\childdocjob| will always contain the name of the main file.

%%%%%%%%%%%%%%%%%%%%%%%%%%%%%%%%%%%%%%%%
\paragraph{Title Page.}

Conditional processing can be used to include a title or banner page
in the main document when proper precautions are taken.
Importantly, the code in the main file should ensure that the page counter
(as well as other status parameters which are stored in the |.aux| files)
takes the same value after the conditional processing.
Otherwise the page numbers may take divergent values
depending on which part is compiled.

For example, a title page could be declared by:
%
\begin{center}
\begin{tabular}{l}
|\ifchilddoc\||else|\\
|\addtocounter{page}{-1}|\\
\textit{code for title page}\\
|\newpage|\\
|\||fi|
\end{tabular}
\end{center}
%
A banner page for the child documents can be generated by:
%
\begin{center}
\begin{tabular}{l}
|\ifchilddoc|\\
|\addtocounter{page}{-1}|\\
\textit{code for banner page}\\
|\newpage|\\
|\||fi|
\end{tabular}
\end{center}
%
Here one could write a message such as:
\begin{center}
|This is the part \childdocname{} of \childdocjob{}.|
\end{center}

%%%%%%%%%%%%%%%%%%%%%%%%%%%%%%%%%%%%%%%%%%%%%%%%%%%%%%%%%%%%%%%%%%%%%%%%%%%%%%%%
\subsection{Flags}
\label{sec:flags}

The package makes it easy to generate different versions
of the main or child documents.
To this end compilation flags can be defined
and assigned different default values.
They will be particularly useful in conjunction
with the forwarding mechanism described in \secref{sec:forward}.

For example, it may be useful to have a flag |\version|
which can be set to |draft| or |final|.
The document source will contain some conditional code
depending on the value of |\version|.
Suppose further, the flag should default to |final| for the main file
and to |draft| for child files
which is a natural assignment for editing the document.
This is achieved by placing the following code
in the preamble of the main document
(below the |\childdocmain| directive):
%
\begin{center}
\begin{tabular}{l}
|\ifchilddoc|\\
|\providecommand{\version}{draft}|\\
|\||else|\\
|\providecommand{\version}{final}|\\
|\||fi|
\end{tabular}
\end{center}
%
The definition by |\providecommand| makes sure
that previous definitions are not overwritten.
Further statements |\providecommand{\version}{...}|
can thus be added before the above code to override it.

For the main file, one might add a line
(between |\childdocmain| and the above block)
%
\begin{center}
|%\ifchilddoc\||else\providecommand{\version}{draft}\||fi|
\end{center}
%
which can be uncommented to produce a draft version.
Likewise one can add a line to the very top of a child file
(above the |\childdocof{|\textit{main}|}| directive)
%
\begin{center}
|%\providecommand{\version}{final}|
\end{center}
%
which can be uncommented to produce the final version of this child document.

%%%%%%%%%%%%%%%%%%%%%%%%%%%%%%%%%%%%%%%%%%%%%%%%%%%%%%%%%%%%%%%%%%%%%%%%%%%%%%%%
\subsection{Forwarding}
\label{sec:forward}

Different versions of the main or child documents
using compilation flags as described in \secref{sec:flags}
can be (permanently) stored in different files
for convenient compilation, viewing and distribution.
To this end, the package defines a command
to pass on compilation to a different file:

%%%%%%%%%%%%%%%%%%%%%%%%%%%%%%%%%%%%%%%%
\DescribeMacro{\childdocforward}
The command |\childdocforward| redirects processing to
another source file:
%
\begin{center}
\begin{tabular}{l}
|\input{childdoc.def}|\\
|\childdocforward[|\textit{main}|]{|\textit{dest}|}|\\
\end{tabular}
\end{center}
%
The argument \textit{dest} is the destination file
(without extension).
It should be the main file or one of the child files.
Note that further \textsf{childdoc} directives
such as |\childdocof| and |\childdocforward|
in the indicated file will be processed in this form.
The optional argument \textit{main}
passes on directly to the main file \textit{main}
while pretending to compile the child \textit{dest}.
This form behaves as if \textit{dest}
issues |\childdocof{|\textit{main}|}| right away,
and no further \textsf{childdoc} directives will be processed.

%%%%%%%%%%%%%%%%%%%%%%%%%%%%%%%%%%%%%%%%
\DescribeMacro{\...prefix}
In the alternative form |\childdocforwardprefix|,
%
\begin{center}
\begin{tabular}{l}
|\input{childdoc.def}|\\
|\childdocforwardprefix[|\textit{main}|]{|\textit{prefix}|}{|\textit{dest}|}|
\end{tabular}
\end{center}
%
the destination file is determined by a pattern
depending on the current file:
To make this work, the current file must be called
`{\textit{prefix}\hspace{0.2em}\textit{suffix}}'
with \textit{prefix} matching precisely the argument.
Processing is then passed on to the file
`{\textit{dest}\hspace{0.2em}\textit{suffix}}'.
Surely, the same effect is achieved by
directly specifying the
argument `{\textit{dest}\hspace{0.2em}\textit{suffix}}'
in the first form.
However, that requires to set up a different file
for each child. With the alternative form of the command
all these files can have exactly the same content
which simplifies setting them up and maintaining them.

For example, the following file |draft.tex|
with a compilation flag |\version| as described in \secref{sec:flags}
compiles the main document as a draft:
%
\begin{center}
\begin{tabular}{l}
|\def\version{draft}|\\
|\input{childdoc.def}|\\
|\childdocforward{|\textit{main}|}|
\end{tabular}
\end{center}
%
Likewise, the following files |final|\textit{nn}|.tex|
compile the final version of the child document
|child|\textit{nn}|.tex|:
%
\begin{center}
\begin{tabular}{l}
|\def\version{final}|\\
|\input{childdoc.def}|\\
|\childdocforwardprefix{final}{child}|
\end{tabular}
\end{center}
%

Note that when several versions of a main file and/or of each child file
are to be generated, it may be convenient to set up a |Makefile| or
shell script to automatise the process.

%%%%%%%%%%%%%%%%%%%%%%%%%%%%%%%%%%%%%%%%%%%%%%%%%%%%%%%%%%%%%%%%%%%%%%%%%%%%%%%%
\subsection{Command Line Processing}
\label{sec:commandline}

The effect of redirection files can also be achieved by invoking
the \LaTeX{} compiler with a more elaborate command line.
Most conveniently this should be done as part
of a shell script or a |Makefile|.

When using \textsf{childdoc} in the main file, the following
command lines effectively perform a redirection
(note that depending on the shell being used,
backslashes may have to be doubled: `|\|' $\to$ `|\\|'):
%
\begin{center}
|... -jobname "|\textit{target}|" |\\|"|[\textit{flags}]%
|\input{childdoc.def}\childdocforward[|\textit{main}|]{|\textit{dest}|}"|
\end{center}
%
Here \textit{target} is the name of the output file,
\textit{main} is the name of the main file
and \textit{dest} is the name of the main or child file to be processed
(all filenames without extensions).
The optional argument \textit{main} can be omitted
if \textit{main} matches \textit{dest}.
Optionally, compilation \textit{flags} can be defined via |\def| commands.
This command line makes the \TeX{} engine believe
it is compiling the file \textit{target}
whose content is specified as the latter parameter.
The provided code then forwards the processing to
\textit{main} or \textit{dest} as described in \secref{sec:forward}.

%%%%%%%%%%%%%%%%%%%%%%%%%%%%%%%%%%%%%%%%%%%%%%%%%%%%%%%%%%%%%%%%%%%%%%%%%%%%%%%%
\subsection{Include by Input}
\label{sec:input}

Including child documents by |\include| has some restrictions by design.
Most notably, the content of a child document always occupies
its own set of pages; pages cannot be shared between child documents.
Usually, this behaviour makes perfect sense
because each child document contain an essential part of the document.
However, in some situations it may be desirable to compose
a document from a collection of parts
without having mandatory page breaks between then.
For this case, the package
provides a mechanism to include parts
by |\input| which can also be processed individually.
However, by construction this mechanism
requires manual handling of the content to be output.

%%%%%%%%%%%%%%%%%%%%%%%%%%%%%%%%%%%%%%%%
\DescribeMacro{\ifchilddocmanual}
The main file should be prepared as usual, see \secref{sec:include}.
However, the document body must make a distinction
between processing of an individual part and of the main document, e.g.:
%
\begin{center}
\begin{tabular}{l}
|\ifchilddocmanual|\\
|\input{\childdocname}|\\
|\||else|\\
\textit{document body with }|\input{|\textit{part}|}|\\
|\||fi|
\end{tabular}
\end{center}
%
The conditional |\ifchilddocmanual| is true whenever
a part to be included by |\input| is being compiled,
and the name of the part is stored in |\childdocname|.

%%%%%%%%%%%%%%%%%%%%%%%%%%%%%%%%%%%%%%%%
\DescribeMacro{\childdocby}
Each part to be included by |\input| should start with:
%
\begin{center}
\begin{tabular}{l}
|\input{childdoc.def}|\\
|\childdocby{|\textit{main}|}|\\
\end{tabular}
\end{center}
%
The directive |\childdocby| is similar to |\childdocof|
described in \secref{sec:include},
but the subsequent selection of content must be done manually.
To that end, both |\ifchilddoc| and |\ifchilddocmanual|
will be true upon processing of a part,
and the name of the part is stored in |\childdocname|.
Note that |\jobname| will be set to the filename of the current part
so that each part receives an individual |.aux| file
that does not interfere with the |.aux| file(s) of the main document.
This behaviour can be altered by the alternative form
|\childdocby[*]{|\textit{main}|}| (with a non-empty optional argument)
which uses the |.aux| file of the main document
by setting |\jobname| to \textit{main}.

%%%%%%%%%%%%%%%%%%%%%%%%%%%%%%%%%%%%%%%%%%%%%%%%%%%%%%%%%%%%%%%%%%%%%%%%%%%%%%%%
\subsection{Driver Development}
\label{sec:driver}

The \textsf{childdoc} mechanism can also be use for the development
of definition files such as \LaTeX{} styles or classes.
This case differs from the above setup with multiple parts
included by |\include| in that no |\includeonly| should be invoked.
This can be achieved by starting the include file
(before |\ProvidesPackage|) with:
%
\begin{center}
\begin{tabular}{l}
|\input{childdoc.def}|\\
|\childdocforward{|\textit{main}|}|\\
\end{tabular}
\end{center}
%
or alternatively with:
%
\begin{center}
\begin{tabular}{l}
|\input{childdoc.def}|\\
|\childdocby{|\textit{main}|}|\\
\end{tabular}
\end{center}
%
Both forms have slightly different effects as described above.
The main file is prepared as usual, see \secref{sec:include}.

%%%%%%%%%%%%%%%%%%%%%%%%%%%%%%%%%%%%%%%%%%%%%%%%%%%%%%%%%%%%%%%%%%%%%%%%%%%%%%%%
\subsection{Legacy Detection}
\label{sec:detection}

The directive |\childdocmain| in the main file can detect
whether the complete document or merely a child is to be compiled
even without using the directive |\childdocof|.
This method is deprecated because it is less robust
and there is no compelling reason to use it;
it is merely provided for backward compatibility
and it may be removed in future versions.

If the detection mechanism is to be used,
it is mandatory to correctly specify
the filename of the main file as the argument of |\childdocmain|:
%
\begin{center}
\begin{tabular}{l}
|\input{childdoc.def}|\\
|\childdocmain{|\textit{main}|}|\\
\end{tabular}
\end{center}
%
If |\jobname| does not match the argument \textit{main} of |\childdocmain|,
it is assumed that |\jobname| points to the child file to be compiled.
When using |\childdocmain| with the main file specified as argument,
it suffices to start a child file
with just |\input{|\textit{main}|}|
without loading of the package and using |\childdocof|.
If instead all processing is done
with the appropriate \textsf{childdoc} directives,
the argument of \textit{main} of |\childdocmain| can be empty.

An alternative version of the command line processing described
in \secref{sec:commandline} using the detection mechanism reads:
%
\begin{center}
|... -jobname "|\textit{target}|" "|[\textit{flags}]%
[|\def\jobname{|\textit{dest}|}|]|\input{|\textit{main}|}"|
\end{center}

%%%%%%%%%%%%%%%%%%%%%%%%%%%%%%%%%%%%%%%%%%%%%%%%%%%%%%%%%%%%%%%%%%%%%%%%%%%%%%%%
\subsection{Manual Code}
\label{sec:manual}

In case one cannot be certain whether the definitions file |childdoc.def|
is installed on the target \TeX{} distribution
and one prefers not to ship it,
it is conceivable to paste a few relevant commands into the sources.

To that end, drop all statements |\input{childdoc.def}|
and perform the replacements as outlined below.
Instead of |\childdocmain{|\textit{main}|}| add the following code
to the top of the main file:
%
\begin{center}
\begin{tabular}{l}
|\||ifdefined\childdocname\endinput\||fi\newif\ifchilddoc|\\
|\edef\childdocname{\scantokens\expandafter{\jobname\noexpand}}|\\
|\def\childdocmain{|\textit{main}|}\||ifx\childdocmain\childdocname\||else|\\
|\childdoctrue\includeonly{\childdocname}\let\jobname\childdocmain\||fi|\\
\end{tabular}
\end{center}
%
Instead of |\childdocof{|\textit{main}|}| just include the main file
at the top of each child file:
%
\begin{center}
|\input{|\textit{main}|}|
\end{center}
%
A simple redirection |\childdocforward{|\textit{dest}|}| is achieved by:
%
\begin{center}
|\def\jobname{|\textit{dest}|}\input{\jobname}|
\end{center}
%
The redirection with prefix
|\childdocforwardprefix[|\textit{prefix}|]{|\textit{dest}|}|
is accomplished by:
%
\begin{center}
\begin{tabular}{l}
|{\edef\jobname{\scantokens\expandafter{\jobname\noexpand}}|\\
|\def\redirectjob |\textit{prefix}|#1~~~{\gdef\jobname{|\textit{dest}|#1}}|\\
|\expandafter\redirectjob\jobname~~~}\input{\jobname}|
\end{tabular}
\end{center}

In an alternative approach,
child documents can be compiled by a specific command line
without additional code or specific definitions:
%
\begin{center}
|... -jobname "|\textit{target}|" "|[\textit{flags}]%
|\includeonly{|\textit{dest}|}\input{|\textit{main}|}"|
\end{center}
%

%%%%%%%%%%%%%%%%%%%%%%%%%%%%%%%%%%%%%%%%%%%%%%%%%%%%%%%%%%%%%%%%%%%%%%%%%%%%%%%%
%%%%%%%%%%%%%%%%%%%%%%%%%%%%%%%%%%%%%%%%%%%%%%%%%%%%%%%%%%%%%%%%%%%%%%%%%%%%%%%%
\section{Information}

%%%%%%%%%%%%%%%%%%%%%%%%%%%%%%%%%%%%%%%%%%%%%%%%%%%%%%%%%%%%%%%%%%%%%%%%%%%%%%%%
\subsection{Copyright}

Copyright \copyright{} 2017--2018 Niklas Beisert

This work may be distributed and/or modified under the
conditions of the \LaTeX{} Project Public License, either version 1.3
of this license or (at your option) any later version.
The latest version of this license is in
  \url{http://www.latex-project.org/lppl.txt}
and version 1.3 or later is part of all distributions of \LaTeX{}
version 2005/12/01 or later.

This work has the LPPL maintenance status `maintained'.

The Current Maintainer of this work is Niklas Beisert.

This work consists of the files |README.txt|, |childdoc.ins| and |childdoc.dtx|
as well as the derived files |childdoc.def|, |cdocsamp.tex|
with |cdocsch1.tex|, |cdocsch2.tex|, |cdocspt3.tex|, |cdocspt4.tex|,
|cdocsdrf.tex|, |cdocsfn1.tex|, |cdocsfn2.tex|
as well as |childdoc.pdf|.

%%%%%%%%%%%%%%%%%%%%%%%%%%%%%%%%%%%%%%%%%%%%%%%%%%%%%%%%%%%%%%%%%%%%%%%%%%%%%%%%
\subsection{Files and Installation}

The package consists of the files:
%
\begin{center}
\begin{tabular}{ll}
    |README.txt|   & readme file \\
    |childdoc.ins| & installation file \\
    |childdoc.dtx| & source file \\
    |childdoc.def| & definition file \\
    |cdocsamp.tex| & sample main file \\
    |cdocsch1.tex| & sample include file \\
    |cdocsch2.tex| & sample include file \\
    |cdocspt3.tex| & sample part file \\
    |cdocspt4.tex| & sample part file \\
    |cdocsdrf.tex| & sample redirection file \\
    |cdocsfn1.tex| & sample redirection file \\
    |cdocsfn2.tex| & sample redirection file \\
    |childdoc.pdf| & manual
\end{tabular}
\end{center}
%
The distribution consists of the files
|README.txt|, |childdoc.ins| and |childdoc.dtx|.
%
\begin{itemize}
\item
Run (pdf)\LaTeX{} on |childdoc.dtx|
to compile the manual |childdoc.pdf| (this file).
\item
Run \LaTeX{} on |childdoc.ins| to create the definitions file |childdoc.def|
and the sample |cdocsamp.tex| with include files
|cdocsch1.tex|, |cdocsch2.tex|, |cdocspt3.tex|, |cdocspt4.tex|,
|cdocsdrf.tex|, |cdocsfn1.tex|, |cdocsfn2.tex|.
Then copy the file |childdoc.def| to an appropriate directory of your \LaTeX{}
distribution, e.g.\ \textit{texmf-root}|/tex/latex/childdoc|.
\end{itemize}

%%%%%%%%%%%%%%%%%%%%%%%%%%%%%%%%%%%%%%%%%%%%%%%%%%%%%%%%%%%%%%%%%%%%%%%%%%%%%%%%
\subsection{Related CTAN Packages}

There are several other packages which offer a similar functionality:
%
\begin{itemize}
\item
The packages
\href{http://ctan.org/pkg/docmute}{\textsf{docmute}},
\href{http://ctan.org/pkg/includex}{\textsf{includex}} and
\href{http://ctan.org/pkg/standalone}{\textsf{standalone}}
provide commands to include only the document body of
a child file thus allowing both files to be compiled individually.
\item
The packages \href{http://ctan.org/pkg/subdocs}{\textsf{subdocs}}
and \href{http://ctan.org/pkg/subfiles}{\textsf{subfiles}}
provide structures in which the main and child documents can be
encapsulated and allowing them to be compiled individually.
The inclusion mechanism is different from the conventional |\include|.
\item
The package \href{http://ctan.org/pkg/combine}{\textsf{combine}}
is an elaborate solution to combine several documents into one.
\end{itemize}
%
See also the CTAN topic \href{http://ctan.org/topic/subdocs}{\textsf{subdocs}}
for further related packages.
The present package differs from the above solutions in that
a document structure constructed with the conventional |\include| mechanism
just needs two extra commands at the top of every file
such that all constituent files can be compiled individually.

%%%%%%%%%%%%%%%%%%%%%%%%%%%%%%%%%%%%%%%%%%%%%%%%%%%%%%%%%%%%%%%%%%%%%%%%%%%%%%%%
%\subsection{Feature Suggestions}
%
%The following is a list of features which may be useful for future
%versions of this package:
%%
%\begin{itemize}
%\item
%\ldots
%\end{itemize}

%%%%%%%%%%%%%%%%%%%%%%%%%%%%%%%%%%%%%%%%%%%%%%%%%%%%%%%%%%%%%%%%%%%%%%%%%%%%%%%%
\subsection{Revision History}

%%%%%%%%%%%%%%%%%%%%%%%%%%%%%%%%%%%%%%%%
\paragraph{v2.0:} 2018/12/30

\begin{itemize}
\item
immediate forward processing
\item
added |\childdocby| mechanism
\item
manual restructured
\end{itemize}

%%%%%%%%%%%%%%%%%%%%%%%%%%%%%%%%%%%%%%%%
\paragraph{v1.6:} 2018/01/17

\begin{itemize}
\item
application for development of include files
\item
corrections to manual
\end{itemize}

%%%%%%%%%%%%%%%%%%%%%%%%%%%%%%%%%%%%%%%%
\paragraph{v1.5:} 2017/05/21

\begin{itemize}
\item
more complete structuring introduced
\item
|\childdocof| introduced
\item
|\childdoc| renamed to |\childdocmain|
\item
|\childredirect| renamed to |\childdocforward| and |\childdocforwardprefix|
and functionality expanded
\end{itemize}

%%%%%%%%%%%%%%%%%%%%%%%%%%%%%%%%%%%%%%%%
\paragraph{v1.0:} 2017/04/27

\begin{itemize}
\item
manual and install package
\item
first version published on CTAN
\end{itemize}

%%%%%%%%%%%%%%%%%%%%%%%%%%%%%%%%%%%%%%%%
\paragraph{v0.6:} 2017/04/26

\begin{itemize}
\item
redirection mechanism added
\end{itemize}

%%%%%%%%%%%%%%%%%%%%%%%%%%%%%%%%%%%%%%%%
\paragraph{v0.5:} 2017/04/26

\begin{itemize}
\item
functionality in definition file
\end{itemize}


%%%%%%%%%%%%%%%%%%%%%%%%%%%%%%%%%%%%%%%%%%%%%%%%%%%%%%%%%%%%%%%%%%%%%%%%%%%%%%%%
%%%%%%%%%%%%%%%%%%%%%%%%%%%%%%%%%%%%%%%%%%%%%%%%%%%%%%%%%%%%%%%%%%%%%%%%%%%%%%%%
%%%%%%%%%%%%%%%%%%%%%%%%%%%%%%%%%%%%%%%%%%%%%%%%%%%%%%%%%%%%%%%%%%%%%%%%%%%%%%%%
\appendix

\settowidth\MacroIndent{\rmfamily\scriptsize 000\ }

 \DocInput{childdoc.dtx}

\end{document}
%</driver>
% \fi
%
% %%%%%%%%%%%%%%%%%%%%%%%%%%%%%%%%%%%%%%%%%%%%%%%%%%%%%%%%%%%%%%%%%%%%%%%%%%%%%%
% %%%%%%%%%%%%%%%%%%%%%%%%%%%%%%%%%%%%%%%%%%%%%%%%%%%%%%%%%%%%%%%%%%%%%%%%%%%%%%
% \section{Sample}
%\iffalse
%<*samplemain>
%\fi
%
% The following presents a sample document
% with two chapters, two parts, a title page,
% a compile flag as well as three forwarding files to set the flag.
% It consists of eight |.tex| files:
% \begin{center}
% \begin{tabular}{ll}
% |cdocsamp.tex|&main file\\
% |cdocsch1.tex|&include file for chapter 1\\
% |cdocsch2.tex|&include file for chapter 2\\
% |cdocspt3.tex|&include file for part 3\\
% |cdocspt4.tex|&include file for part 4\\
% |cdocsdrf.tex|&forwarding file for main file in draft mode\\
% |cdocsfi1.tex|&forwarding file for final version of chapter 1\\
% |cdocsfi2.tex|&forwarding file for final version of chapter 2\\
% \end{tabular}
% \end{center}
% Each of the eight files can be compiled directly by the \LaTeX{} compiler.
%
% %%%%%%%%%%%%%%%%%%%%%%%%%%%%%%%%%%%%%%
% \paragraph{Main File.}
%
% The main file is called |cdocsamp.tex|.
%
% Load the \textsf{childdoc} definitions and
% declare the filename for the main document:
%    \begin{macrocode}
\input{childdoc.def}
\childdocmain{}
%    \end{macrocode}

% Optional override for |\version| flag:
%    \begin{macrocode}
%%\ifchilddoc\else\providecommand{\version}{draft}\fi
%    \end{macrocode}

% Define the default values for the |\version| flag
% (|final| for the main file and |draft| for childs):
%    \begin{macrocode}
\ifchilddoc
\providecommand{\version}{draft}
\else
\providecommand{\version}{final}
\fi
%    \end{macrocode}

% Load the standard document class:
%    \begin{macrocode}
\documentclass[12pt]{article}
%    \end{macrocode}

% Start the document body:
%    \begin{macrocode}
\begin{document}
%    \end{macrocode}

% Declare a title page.
% Print title, part of document being processed and version flag:
%    \begin{macrocode}
\addtocounter{page}{-1}
\begin{center}
{\LARGE\bfseries{}childdoc example\par}
\vspace{1cm}
\ifchilddoc
\ifchilddocmanual part\else chapter\fi:
`\childdocname' of `\childdocjob'\par
\else
main document: `\childdocjob'\par
\fi
version: \version\par
\end{center}
\newpage
%    \end{macrocode}

% Manually include selected file,
% otherwise process as usual:
%    \begin{macrocode}
\ifchilddocmanual
\section*{part `\childdocname'}
\input{\childdocname}
\else
%    \end{macrocode}

% Include the two chapters:
%    \begin{macrocode}
\include{cdocsch1}
\include{cdocsch2}
%    \end{macrocode}

% Include the two parts unless only chapters should be displayed:
%    \begin{macrocode}
\ifchilddoc\else
\section{part three}
\input{cdocspt3}
\section{part four}
\input{cdocspt4}
\fi
%    \end{macrocode}

% Process as usual until here:
%    \begin{macrocode}
\fi
%    \end{macrocode}

% End of document body:
%    \begin{macrocode}
\end{document}
%    \end{macrocode}
%\iffalse
%</samplemain>
%\fi
%
% %%%%%%%%%%%%%%%%%%%%%%%%%%%%%%%%%%%%%%
% \paragraph{Chapter Include Files.}
%
% The include files are called |cdocsch1.tex| and |cdocsch2.tex|.
%
%\iffalse
%<*samplechap1|samplechap2>
%\fi

% Optional override for |\version| flag:
%    \begin{macrocode}
%%\providecommand{\version}{final}
%    \end{macrocode}

% Include the main document:
%    \begin{macrocode}
\input{childdoc.def}
\childdocof{cdocsamp}
%    \end{macrocode}

%\iffalse
%</samplechap1|samplechap2>
%\fi
%
%\iffalse
%<*samplechap1>
%\fi
% Some text for chapter 1:
%    \begin{macrocode}
\section{one}
some text in chapter one
%    \end{macrocode}

%\iffalse
%</samplechap1>
%\fi
% Some text for chapter 2:
%\iffalse
%<*samplechap2>
%\fi
%    \begin{macrocode}
\section{two}
more text in chapter two
%    \end{macrocode}

%\iffalse
%</samplechap2>
%\fi
%
% %%%%%%%%%%%%%%%%%%%%%%%%%%%%%%%%%%%%%%
% \paragraph{Part Include Files.}
%
% The include files are called |cdocspt3.tex| and |cdocspt4.tex|.
%
%\iffalse
%<*samplepart3|samplepart4>
%\fi

% Optional override for |\version| flag:
%    \begin{macrocode}
%%\providecommand{\version}{final}
%    \end{macrocode}

% Include the main document:
%    \begin{macrocode}
\input{childdoc.def}
\childdocby{cdocsamp}
%    \end{macrocode}

%\iffalse
%</samplepart3|samplepart4>
%\fi
%
%\iffalse
%<*samplepart3>
%\fi
% Some text for part 3:
%    \begin{macrocode}
some text in part three
%    \end{macrocode}

%\iffalse
%</samplepart3>
%\fi
% Some text for part 4:
%\iffalse
%<*samplepart4>
%\fi
%    \begin{macrocode}
more text in part four
%    \end{macrocode}

%\iffalse
%</samplepart4>
%\fi
%
% %%%%%%%%%%%%%%%%%%%%%%%%%%%%%%%%%%%%%%
% \paragraph{Forwarding for a Complete Draft.}
%
% The following forwarding file |cdocsdrf.tex|
% compiles the main document in draft mode:
%\iffalse
%<*sampledraft>
%\fi
%    \begin{macrocode}
\def\version{draft}
\input{childdoc.def}
\childdocforward{cdocsamp}
%    \end{macrocode}

%\iffalse
%</sampledraft>
%\fi
%
% %%%%%%%%%%%%%%%%%%%%%%%%%%%%%%%%%%%%%%
% \paragraph{Forwarding for Final Version of the Chapters.}
%
% The following forwarding files |cdocsfn1.tex| and |cdocsfn2.tex|
% (with identical content)
% compile the final versions of the child documents
% |cdocsch1.tex| and |cdocsch2.tex|, respectively:
%\iffalse
%<*samplefinal>
%\fi
%    \begin{macrocode}
\def\version{final}
\input{childdoc.def}
\childdocforwardprefix[cdocsamp]{cdocsfn}{cdocsch}
%    \end{macrocode}

%\iffalse
%</samplefinal>
%\fi
%
% %%%%%%%%%%%%%%%%%%%%%%%%%%%%%%%%%%%%%%
% \paragraph{Command Line Processing.}
%
% The following three command lines generate the output files
% |cdocscld|, |cdocscl1| and |cdocscl2|
% which should be identical to
% |cdocsdrf|, |cdocsch1| and |cdocsfn2|, respectively:
% \begin{center}
% \begin{tabular}{l}
% |latex -jobname cdocscld \|\\
% |  "\def\version{draft}\input{childdoc.def}\childdocforward{cdocsamp}"|\\
% |latex -jobname cdocscl1 \|\\
% |  "\input{childdoc.def}\childdocforward[cdocsamp]{cdocsch1}"|\\
% |latex -jobname cdocscl2 \|\\
% |  "\def\version{final}\input{childdoc.def}\childdocforward{cdocsch2}"|
% \end{tabular}
% \end{center}
% Note that the trailing backslash on each first line
% merely continues the input to the second line
% (for convenient cut ant paste).
% Furthermore, the command |latex| can be replaced by any
% of its alternative versions such as |pdflatex|.
%
% %%%%%%%%%%%%%%%%%%%%%%%%%%%%%%%%%%%%%%%%%%%%%%%%%%%%%%%%%%%%%%%%%%%%%%%%%%%%%%
% %%%%%%%%%%%%%%%%%%%%%%%%%%%%%%%%%%%%%%%%%%%%%%%%%%%%%%%%%%%%%%%%%%%%%%%%%%%%%%
% \section{Implementation}
%\iffalse
%<*package>
%\fi
%
% This section describes the definitions file |childdoc.def|.

% The definitions cannot be loaded using |\usepackage| or |\RequirePackage|
% which has a mechanism to prevent loading a style file more than once.
% When loading the definitions by means of |\input|
% multiple instances have to be prevented manually:
%\iffalse
%This code needs to be before the `\ProvidesFile' directive
%which is defined at the beginning of this file.
%Therefore it is also placed there and commented out here.
%</package>
%<*discard>
%\fi
%    \begin{macrocode}
\ifdefined\childdocmain\endinput\fi
%    \end{macrocode}
%\iffalse
%</discard>
%<*package>
%\fi
%
% \macro{\ifchilddoc}
% \macro{\ifchilddocmanual}
% The conditional |\ifchilddoc| tells whether a
% child (true) or main (false) document is being compiled.
% The conditional |\ifchilddocmanual| tells whether
% the |\includeonly| mechanism is used (false) or
% the selection of child files must be performed manually (true).
% The definitions initialise to false:
%    \begin{macrocode}
\newif\ifchilddoc
\newif\ifchilddocmanual
%    \end{macrocode}

% \macro{\childdocname}
% \macro{\childdocjob}
% The macro |\childdocname| stores the name of the main document
% to be compiled. The macro |\childdocjob| stores the name of
% the document on which the \LaTeX{} compiler was originally invoked.
% The content of |\jobname| cannot be compared
% to filenames specified in the source due to different catcodes.
% The following code rescans |\jobname|, stores the result
% in |\childdocname| and saves a copy in |\childdocjob|:
%    \begin{macrocode}
\edef\childdocname{\scantokens\expandafter{\jobname\noexpand}}
\let\childdocjob\childdocname
%    \end{macrocode}

% \macro{\childdocdisable}
% The macro |\childdocdisable| prevents the main file
% from being processed more than once.
% At this stage, the main document command |\childdocmain|
% is assumed to be called once again where it should do nothing.
% Any subsequent call to it should prevent
% a secondary processing of the main document
% It overwrites the forwarding commands
% |\childdocof| and |\childdocforward|
% with empty macros to prevent further inclusions of the main document:
%    \begin{macrocode}
\newcommand{\childdocdisable}
{
  \renewcommand{\childdocmain}[1]{\renewcommand{\childdocmain}[1]{\endinput}}
  \renewcommand{\childdocof}[1]{}
  \renewcommand{\childdocby}[2][]{}
  \renewcommand{\childdocforward}[2][]{}
  \renewcommand{\childdocdisable}{}
}
%    \end{macrocode}

% \macro{\childdocmain}
% The macro |\childdocmain| is to be called at the top of the main file
% with nothing or the main filename (without extension) as argument.
% First, it breaks loops.
% If the argument is not empty and does not match |\childdocname|
% (which is set by the first inclusion of |childdoc.def|),
% |\ifchilddoc| is set to true, |\includeonly| is applied to the child file
% and |\jobname| is set to the main file
% (for proper handling of |.aux| files):
%    \begin{macrocode}
\newcommand{\childdocmain}[1]
{
  \childdocdisable\childdocmain{}
  \if?#1?\else
    \begingroup
      \def\childdoctmp{#1}
      \ifx\childdoctmp\childdocname
        \def\childdoctmp{}
      \else
        \def\childdoctmp
        {
          \childdoctrue
          \includeonly{\childdocname}
          \def\childdocjob{#1}
          \def\jobname{#1}
        }
      \fi
      \expandafter
    \endgroup
    \childdoctmp
  \fi
}
%    \end{macrocode}

% \macro{\childdocof}
% The command |\childdocof| redirects
% compilation to the main file |#1|.
%    \begin{macrocode}
\newcommand{\childdocof}[1]
{
  \childdocdisable
  \childdoctrue
  \includeonly{\childdocname}
  \def\jobname{#1}
  \def\childdocjob{#1}
  \input{#1}
}
%    \end{macrocode}

% \macro{\childdocby}
% The command |\childdocby| ....
%    \begin{macrocode}
\newcommand{\childdocby}[2][]
{
  \childdocdisable
  \childdoctrue
  \childdocmanualtrue
  \if?#1?\else
    \def\jobname{#2}
  \fi
  \def\childdocjob{#2}
  \input{#2}
  \endinput
}
%    \end{macrocode}

% \macro{\childdocforward}
% The command |\childdocforward| redirects
% compilation to the main file or
% (if the optional argument is given) a child file.
% Parameters are set as if the main file
% or a child file starting with |\childdocof| was compiled.
% Then compilation is handed over to the main file:
%    \begin{macrocode}
\newcommand{\childdocforward}[2][]
{
  \begingroup
    \if?#1?
      \def\childdoctmp
      {
        \def\childdocname{#2}
        \def\childdocjob{#2}
        \def\jobname{#2}
        \input{#2}
        \endinput
      }
    \else
      \def\childdoctmp
      {
        \childdocdisable
        \def\childdocname{#2}
        \childdoctrue
        \includeonly{#2}
        \def\childdocjob{#1}
        \def\jobname{#1}
        \input{#1}
        \endinput
      }
    \fi
    \expandafter
  \endgroup
  \childdoctmp
}
%    \end{macrocode}

% \macro{\childdocforwardprefix}
% The command |\childdocforwardprefix| redirects
% compilation to the main or a child file by means of a pattern.
% The prefix |#1| in the current filename is replaced by |#2|
% and the suffix of the current filename is kept
% (it is assumed that the filename does not contain the substring `|~~~|'
% which is used as a delimiter).
% Compilation is handed over to the new file by |\childdocforward|:
%    \begin{macrocode}
\newcommand{\childdocforwardprefix}[3][]
{
  \begingroup
    \def\childdocextract #2##1~~~{\def\childdoctmp{\childdocforward[#1]{#3##1}}}
    \expandafter\childdocextract\childdocname~~~
    \expandafter
  \endgroup
  \childdoctmp
}
%    \end{macrocode}

% \macro{\childdoc}
% The deprecated macro |\childdoc| is a legacy version of |\childdocmain|:
%    \begin{macrocode}
\newcommand{\childdoc}{\childdocmain}
%    \end{macrocode}

% \macro{\childdocredirect}
% The deprecated macro |\childdocredirect| is a legacy version
% of |\childdocforward| and |\childdocforwardprefix|:
%    \begin{macrocode}
\newcommand{\childdocredirect}[2][]
{
  \begingroup
    \if?#1?
      \def\childdoctmp{\childdocforward{#2}}
    \else
      \def\childdoctmp{\childdocforwardprefix{#1}{#2}}
    \fi
    \expandafter
  \endgroup
  \childdoctmp
}
%    \end{macrocode}

%\iffalse
%</package>
%\fi
%
\endinput
|\\
|\childdocforward[|\textit{main}|]{|\textit{dest}|}|\\
\end{tabular}
\end{center}
%
The argument \textit{dest} is the destination file
(without extension).
It should be the main file or one of the child files.
Note that further \textsf{childdoc} directives
such as |\childdocof| and |\childdocforward|
in the indicated file will be processed in this form.
The optional argument \textit{main}
passes on directly to the main file \textit{main}
while pretending to compile the child \textit{dest}.
This form behaves as if \textit{dest}
issues |\childdocof{|\textit{main}|}| right away,
and no further \textsf{childdoc} directives will be processed.

%%%%%%%%%%%%%%%%%%%%%%%%%%%%%%%%%%%%%%%%
\DescribeMacro{\...prefix}
In the alternative form |\childdocforwardprefix|,
%
\begin{center}
\begin{tabular}{l}
|% \iffalse
%
% childdoc.dtx Copyright (C) 2017-2018 Niklas Beisert
%
% This work may be distributed and/or modified under the
% conditions of the LaTeX Project Public License, either version 1.3
% of this license or (at your option) any later version.
% The latest version of this license is in
%   http://www.latex-project.org/lppl.txt
% and version 1.3 or later is part of all distributions of LaTeX
% version 2005/12/01 or later.
%
% This work has the LPPL maintenance status `maintained'.
%
% The Current Maintainer of this work is Niklas Beisert.
%
% This work consists of the files childdoc.dtx and childdoc.ins
% and the derived files childdoc.def and cdocsamp.tex with
% cdocsch1.tex, cdocsch2.tex, cdocsdrf.tex, cdocsfn1.tex, cdocsfn2.tex.
%
%<package>\ifdefined\childdocmain\endinput\fi
%<package>\ProvidesFile{childdoc.def}[2018/12/30 v2.0 child document driver]
%<samplemain>\ProvidesFile{cdocsamp.tex}[2018/12/30 v2.0 sample for childdoc]
%<*driver>
%\ProvidesFile{childdoc.drv}[2018/12/30 v2.0 childdoc reference manual file]
\PassOptionsToClass{10pt,a4paper}{article}
\documentclass{ltxdoc}

\usepackage[margin=35mm]{geometry}
\usepackage{hyperref}
\usepackage{hyperxmp}
\usepackage[usenames]{color}

\hypersetup{colorlinks=true}
\hypersetup{pdfstartview=FitH}
\hypersetup{pdfpagemode=UseNone}
\hypersetup{pdfsource={}}
\hypersetup{pdflang={en-UK}}
\hypersetup{pdfcopyright={Copyright 2017-2018 Niklas Beisert.
  This work may be distributed and/or modified under the
  conditions of the LaTeX Project Public License, either version 1.3
  of this license or (at your option) any later version.}}
\hypersetup{pdflicenseurl={http://www.latex-project.org/lppl.txt}}
\hypersetup{pdfcontactaddress={ETH Zurich, ITP, HIT K,
  Wolfgang-Pauli-Strasse 27}}
\hypersetup{pdfcontactpostcode={8093}}
\hypersetup{pdfcontactcity={Zurich}}
\hypersetup{pdfcontactcountry={Switzerland}}
\hypersetup{pdfcontactemail={nbeisert@itp.phys.ethz.ch}}
\hypersetup{pdfcontacturl={http://people.phys.ethz.ch/\xmptilde nbeisert/}}

\newcommand{\secref}[1]{\hyperref[#1]{section \ref*{#1}}}

\parskip1ex
\parindent0pt
\let\olditemize\itemize
\def\itemize{\olditemize\parskip0pt}

\begin{document}

\title{The \textsf{childdoc} Package}
\hypersetup{pdftitle={The childdoc Package}}
\author{Niklas Beisert\\[2ex]
  Institut f\"ur Theoretische Physik\\
  Eidgen\"ossische Technische Hochschule Z\"urich\\
  Wolfgang-Pauli-Strasse 27, 8093 Z\"urich, Switzerland\\[1ex]
  \href{mailto:nbeisert@itp.phys.ethz.ch}
  {\texttt{nbeisert@itp.phys.ethz.ch}}}
\hypersetup{pdfauthor={Niklas Beisert}}
\hypersetup{pdfsubject={Manual for the LaTeX2e Package childdoc}}
\date{30 December 2018, \textsf{v2.0}}
\maketitle

\begin{abstract}\noindent
\textsf{childdoc} is a \LaTeXe{} package
that enables the direct compilation
of document sections included by |\include|
to individual files.
\end{abstract}

\begingroup
\parskip0ex
\tableofcontents
\endgroup

%%%%%%%%%%%%%%%%%%%%%%%%%%%%%%%%%%%%%%%%%%%%%%%%%%%%%%%%%%%%%%%%%%%%%%%%%%%%%%%%
%%%%%%%%%%%%%%%%%%%%%%%%%%%%%%%%%%%%%%%%%%%%%%%%%%%%%%%%%%%%%%%%%%%%%%%%%%%%%%%%
\section{Introduction}

\LaTeX{} provides a mechanism to structure a large document (such as a book)
into a main file and several child files (containing the chapters)
using the |\include| command.
This mechanism is beneficial for documents
which span hundreds of pages in order to
make the source file(s) more manageable.
Moreover, compilation can be restricted to
selected child files by means of the |\includeonly| command.
The latter feature can be used to reduce the compilation time while editing
(this was significantly more useful in the earlier days of \LaTeX{})
or to generate a smaller document which is easier to navigate.
Another application of |\includeonly| is to generate
documents consisting of selected parts of the complete document.

However, there are a few drawbacks of the plain |\include| mechanism:
\begin{itemize}
\item
The child files cannot be compiled on their own,
they can only be compiled via the main file.
A naive editing environment
(such as a text editor with an option
to have the current file processed by \LaTeX)
may require one to switch to the main file before compiling;
attempting to compile the child file produces errors.
\item
The main file must be modified (each time)
to adjust the |\includeonly| command
to the present needs. This easily leaves the main file in a messy state.
\item
The generated document will always carry the filename
of the main document. This is inconvenient if
several child files are to be compiled and
to be kept for distribution.
\end{itemize}

The present package provides a simple interface
to make child files individually compilable by \LaTeX{}.
Compiling a child file then has the same effect as compiling
the main file with an |\includeonly| command
to select the appropriate child.
Moreover the generated document will carry the name of the child
rather than the main file.
This resolves all three above issues.

This feature is meant to make the editing of books,
thesis documents and lecture notes somewhat more convenient.
However, the package can also be used efficiently for
composing a series of documents (such as exercise sheets)
which are typically distributed individually.
It then assists the author in generating the individual documents
(potentially in different versions)
as well as a document containing the collected series.
Another application is in developing style files
or other kinds of included material
where compilation of the style file could redirect
to a sample or test file.

%%%%%%%%%%%%%%%%%%%%%%%%%%%%%%%%%%%%%%%%%%%%%%%%%%%%%%%%%%%%%%%%%%%%%%%%%%%%%%%%
%%%%%%%%%%%%%%%%%%%%%%%%%%%%%%%%%%%%%%%%%%%%%%%%%%%%%%%%%%%%%%%%%%%%%%%%%%%%%%%%
\section{Usage}

First of all, the package \textsf{childdoc} is \emph{not} a standard
\LaTeXe{} |.sty| style file! Therefore it needs to be invoked in
a non-standard way.

%%%%%%%%%%%%%%%%%%%%%%%%%%%%%%%%%%%%%%%%%%%%%%%%%%%%%%%%%%%%%%%%%%%%%%%%%%%%%%%%
\subsection{Included Files}
\label{sec:include}

%%%%%%%%%%%%%%%%%%%%%%%%%%%%%%%%%%%%%%%%
\DescribeMacro{\childdocmain}
To use the package, add the commands
\begin{center}
\begin{tabular}{l}
|\input{childdoc.def}|\\
|\childdocmain{}|\\
\end{tabular}
\end{center}
at the very top of the main \LaTeX{} file,
in particular \emph{before} the |\documentclass| statement!
The argument of |\childdocmain| should be left empty
(but it must be present).

%%%%%%%%%%%%%%%%%%%%%%%%%%%%%%%%%%%%%%%%
\DescribeMacro{\childdocof}
Furthermore, add the commands
\begin{center}
\begin{tabular}{l}
|\input{childdoc.def}|\\
|\childdocof{|\textit{main}|}|\\
\end{tabular}
\end{center}
at the top of every child file \textit{child}
which is included by |\include{|\textit{child}|}|
from within the main file
(or at least for those files to be compiled individually).
The argument \textit{main} must be the filename of the main file.

There are a couple of
considerations in setting up the main and child documents:

%%%%%%%%%%%%%%%%%%%%%%%%%%%%%%%%%%%%%%%%
\paragraph{Restrictions.}

Please note the following restrictions:
\begin{itemize}
\item
|\childdocmain| must be called with one argument \textit{main}
to ensure compatibility with earlier version of the package.
It must either be empty (|\childdocmain{}|)
or precisely match the filename of the main file in which it is specified.
See \secref{sec:detection} for further information.
\item
The filename \textit{main} must be specified without the |.tex| extension.
\item
The filename \textit{main} is case sensitive
(even in case-insensitive file systems)
due to internal string comparison.
\item
The argument \textit{main} should be fully expanded, it cannot be a macro.
\item
Subdirectories and special characters should be avoided in filenames.
\item
The command |\childdocmain{|\textit{main}|}| must be followed by a whitespace.
It should not be followed immediately by another command
or by a comment mark `|%|'.
This is because the \TeX{} parser reads the token immediately following
the argument of |\childdocmain| and puts it
at the beginning of every child section;
however, a white\-space is ignored.
\end{itemize}

%%%%%%%%%%%%%%%%%%%%%%%%%%%%%%%%%%%%%%%%
\paragraph{Content of Main File.}

It is advisable to place all content in the child files included by |\include|.
Any output contained in the main file will appear in all child documents
unless suppressed manually;
it cannot be suppressed automatically by the |\includeonly| directive
and thus should normally be avoided.
A method to include some content in the main file
by means of conditional processing is described in \secref{sec:conditional}.

%%%%%%%%%%%%%%%%%%%%%%%%%%%%%%%%%%%%%%%%
\paragraph{Page Numbering.}

When only a part of the document is compiled,
the appropriate numbering of pages
(as well as other status parameters)
is determined from the |.aux| files.
The latter contain information from previous passes.
However this information needs to propagate through
all intermediate child documents.
Therefore the page numbering in child documents may well
be inconsistent until the complete document is compiled at least once.

A useful (if unconventional) way to always ensure a consistent
page numbering is to restart the numbering in each child document
and denote the pages by `\textit{child}|.|\textit{page}'
where \textit{child} represents the chapter/section number of the child file.
This can be achieved by the command
|\numberwithin{page}{|\textit{child}|}|
of the \textsf{amsmath} package
where \textit{child} can be |chapter| or |section|
depending on the chosen structuring.
Alternatively, one can modify the macro |\thepage| appropriately
and reset the counter |page| at the start of each child file.

%%%%%%%%%%%%%%%%%%%%%%%%%%%%%%%%%%%%%%%%%%%%%%%%%%%%%%%%%%%%%%%%%%%%%%%%%%%%%%%%
\subsection{Conditional Processing}
\label{sec:conditional}

The package provides a mechanism to compile different versions
of a document. To customise the versions further some conditional processing
can come in handy to distinguish which version is being compiled.
The package provides two macros to describe the compilation context:

%%%%%%%%%%%%%%%%%%%%%%%%%%%%%%%%%%%%%%%%
\DescribeMacro{\ifchilddoc}
The conditional |\ifchilddoc| distinguishes between the compilation of
child documents and the main document:
%
\begin{center}
|\ifchilddoc |\textit{child-code}| |[|\||else |\textit{main-code}]| \||fi|
\end{center}

%%%%%%%%%%%%%%%%%%%%%%%%%%%%%%%%%%%%%%%%
\DescribeMacro{\childdocname}
\DescribeMacro{\childdocjob}
The macro |\childdocname| contains the filename (without extension)
of the main or child file being processed.
Note that |\childdocjob| will always contain the name of the main file.

%%%%%%%%%%%%%%%%%%%%%%%%%%%%%%%%%%%%%%%%
\paragraph{Title Page.}

Conditional processing can be used to include a title or banner page
in the main document when proper precautions are taken.
Importantly, the code in the main file should ensure that the page counter
(as well as other status parameters which are stored in the |.aux| files)
takes the same value after the conditional processing.
Otherwise the page numbers may take divergent values
depending on which part is compiled.

For example, a title page could be declared by:
%
\begin{center}
\begin{tabular}{l}
|\ifchilddoc\||else|\\
|\addtocounter{page}{-1}|\\
\textit{code for title page}\\
|\newpage|\\
|\||fi|
\end{tabular}
\end{center}
%
A banner page for the child documents can be generated by:
%
\begin{center}
\begin{tabular}{l}
|\ifchilddoc|\\
|\addtocounter{page}{-1}|\\
\textit{code for banner page}\\
|\newpage|\\
|\||fi|
\end{tabular}
\end{center}
%
Here one could write a message such as:
\begin{center}
|This is the part \childdocname{} of \childdocjob{}.|
\end{center}

%%%%%%%%%%%%%%%%%%%%%%%%%%%%%%%%%%%%%%%%%%%%%%%%%%%%%%%%%%%%%%%%%%%%%%%%%%%%%%%%
\subsection{Flags}
\label{sec:flags}

The package makes it easy to generate different versions
of the main or child documents.
To this end compilation flags can be defined
and assigned different default values.
They will be particularly useful in conjunction
with the forwarding mechanism described in \secref{sec:forward}.

For example, it may be useful to have a flag |\version|
which can be set to |draft| or |final|.
The document source will contain some conditional code
depending on the value of |\version|.
Suppose further, the flag should default to |final| for the main file
and to |draft| for child files
which is a natural assignment for editing the document.
This is achieved by placing the following code
in the preamble of the main document
(below the |\childdocmain| directive):
%
\begin{center}
\begin{tabular}{l}
|\ifchilddoc|\\
|\providecommand{\version}{draft}|\\
|\||else|\\
|\providecommand{\version}{final}|\\
|\||fi|
\end{tabular}
\end{center}
%
The definition by |\providecommand| makes sure
that previous definitions are not overwritten.
Further statements |\providecommand{\version}{...}|
can thus be added before the above code to override it.

For the main file, one might add a line
(between |\childdocmain| and the above block)
%
\begin{center}
|%\ifchilddoc\||else\providecommand{\version}{draft}\||fi|
\end{center}
%
which can be uncommented to produce a draft version.
Likewise one can add a line to the very top of a child file
(above the |\childdocof{|\textit{main}|}| directive)
%
\begin{center}
|%\providecommand{\version}{final}|
\end{center}
%
which can be uncommented to produce the final version of this child document.

%%%%%%%%%%%%%%%%%%%%%%%%%%%%%%%%%%%%%%%%%%%%%%%%%%%%%%%%%%%%%%%%%%%%%%%%%%%%%%%%
\subsection{Forwarding}
\label{sec:forward}

Different versions of the main or child documents
using compilation flags as described in \secref{sec:flags}
can be (permanently) stored in different files
for convenient compilation, viewing and distribution.
To this end, the package defines a command
to pass on compilation to a different file:

%%%%%%%%%%%%%%%%%%%%%%%%%%%%%%%%%%%%%%%%
\DescribeMacro{\childdocforward}
The command |\childdocforward| redirects processing to
another source file:
%
\begin{center}
\begin{tabular}{l}
|\input{childdoc.def}|\\
|\childdocforward[|\textit{main}|]{|\textit{dest}|}|\\
\end{tabular}
\end{center}
%
The argument \textit{dest} is the destination file
(without extension).
It should be the main file or one of the child files.
Note that further \textsf{childdoc} directives
such as |\childdocof| and |\childdocforward|
in the indicated file will be processed in this form.
The optional argument \textit{main}
passes on directly to the main file \textit{main}
while pretending to compile the child \textit{dest}.
This form behaves as if \textit{dest}
issues |\childdocof{|\textit{main}|}| right away,
and no further \textsf{childdoc} directives will be processed.

%%%%%%%%%%%%%%%%%%%%%%%%%%%%%%%%%%%%%%%%
\DescribeMacro{\...prefix}
In the alternative form |\childdocforwardprefix|,
%
\begin{center}
\begin{tabular}{l}
|\input{childdoc.def}|\\
|\childdocforwardprefix[|\textit{main}|]{|\textit{prefix}|}{|\textit{dest}|}|
\end{tabular}
\end{center}
%
the destination file is determined by a pattern
depending on the current file:
To make this work, the current file must be called
`{\textit{prefix}\hspace{0.2em}\textit{suffix}}'
with \textit{prefix} matching precisely the argument.
Processing is then passed on to the file
`{\textit{dest}\hspace{0.2em}\textit{suffix}}'.
Surely, the same effect is achieved by
directly specifying the
argument `{\textit{dest}\hspace{0.2em}\textit{suffix}}'
in the first form.
However, that requires to set up a different file
for each child. With the alternative form of the command
all these files can have exactly the same content
which simplifies setting them up and maintaining them.

For example, the following file |draft.tex|
with a compilation flag |\version| as described in \secref{sec:flags}
compiles the main document as a draft:
%
\begin{center}
\begin{tabular}{l}
|\def\version{draft}|\\
|\input{childdoc.def}|\\
|\childdocforward{|\textit{main}|}|
\end{tabular}
\end{center}
%
Likewise, the following files |final|\textit{nn}|.tex|
compile the final version of the child document
|child|\textit{nn}|.tex|:
%
\begin{center}
\begin{tabular}{l}
|\def\version{final}|\\
|\input{childdoc.def}|\\
|\childdocforwardprefix{final}{child}|
\end{tabular}
\end{center}
%

Note that when several versions of a main file and/or of each child file
are to be generated, it may be convenient to set up a |Makefile| or
shell script to automatise the process.

%%%%%%%%%%%%%%%%%%%%%%%%%%%%%%%%%%%%%%%%%%%%%%%%%%%%%%%%%%%%%%%%%%%%%%%%%%%%%%%%
\subsection{Command Line Processing}
\label{sec:commandline}

The effect of redirection files can also be achieved by invoking
the \LaTeX{} compiler with a more elaborate command line.
Most conveniently this should be done as part
of a shell script or a |Makefile|.

When using \textsf{childdoc} in the main file, the following
command lines effectively perform a redirection
(note that depending on the shell being used,
backslashes may have to be doubled: `|\|' $\to$ `|\\|'):
%
\begin{center}
|... -jobname "|\textit{target}|" |\\|"|[\textit{flags}]%
|\input{childdoc.def}\childdocforward[|\textit{main}|]{|\textit{dest}|}"|
\end{center}
%
Here \textit{target} is the name of the output file,
\textit{main} is the name of the main file
and \textit{dest} is the name of the main or child file to be processed
(all filenames without extensions).
The optional argument \textit{main} can be omitted
if \textit{main} matches \textit{dest}.
Optionally, compilation \textit{flags} can be defined via |\def| commands.
This command line makes the \TeX{} engine believe
it is compiling the file \textit{target}
whose content is specified as the latter parameter.
The provided code then forwards the processing to
\textit{main} or \textit{dest} as described in \secref{sec:forward}.

%%%%%%%%%%%%%%%%%%%%%%%%%%%%%%%%%%%%%%%%%%%%%%%%%%%%%%%%%%%%%%%%%%%%%%%%%%%%%%%%
\subsection{Include by Input}
\label{sec:input}

Including child documents by |\include| has some restrictions by design.
Most notably, the content of a child document always occupies
its own set of pages; pages cannot be shared between child documents.
Usually, this behaviour makes perfect sense
because each child document contain an essential part of the document.
However, in some situations it may be desirable to compose
a document from a collection of parts
without having mandatory page breaks between then.
For this case, the package
provides a mechanism to include parts
by |\input| which can also be processed individually.
However, by construction this mechanism
requires manual handling of the content to be output.

%%%%%%%%%%%%%%%%%%%%%%%%%%%%%%%%%%%%%%%%
\DescribeMacro{\ifchilddocmanual}
The main file should be prepared as usual, see \secref{sec:include}.
However, the document body must make a distinction
between processing of an individual part and of the main document, e.g.:
%
\begin{center}
\begin{tabular}{l}
|\ifchilddocmanual|\\
|\input{\childdocname}|\\
|\||else|\\
\textit{document body with }|\input{|\textit{part}|}|\\
|\||fi|
\end{tabular}
\end{center}
%
The conditional |\ifchilddocmanual| is true whenever
a part to be included by |\input| is being compiled,
and the name of the part is stored in |\childdocname|.

%%%%%%%%%%%%%%%%%%%%%%%%%%%%%%%%%%%%%%%%
\DescribeMacro{\childdocby}
Each part to be included by |\input| should start with:
%
\begin{center}
\begin{tabular}{l}
|\input{childdoc.def}|\\
|\childdocby{|\textit{main}|}|\\
\end{tabular}
\end{center}
%
The directive |\childdocby| is similar to |\childdocof|
described in \secref{sec:include},
but the subsequent selection of content must be done manually.
To that end, both |\ifchilddoc| and |\ifchilddocmanual|
will be true upon processing of a part,
and the name of the part is stored in |\childdocname|.
Note that |\jobname| will be set to the filename of the current part
so that each part receives an individual |.aux| file
that does not interfere with the |.aux| file(s) of the main document.
This behaviour can be altered by the alternative form
|\childdocby[*]{|\textit{main}|}| (with a non-empty optional argument)
which uses the |.aux| file of the main document
by setting |\jobname| to \textit{main}.

%%%%%%%%%%%%%%%%%%%%%%%%%%%%%%%%%%%%%%%%%%%%%%%%%%%%%%%%%%%%%%%%%%%%%%%%%%%%%%%%
\subsection{Driver Development}
\label{sec:driver}

The \textsf{childdoc} mechanism can also be use for the development
of definition files such as \LaTeX{} styles or classes.
This case differs from the above setup with multiple parts
included by |\include| in that no |\includeonly| should be invoked.
This can be achieved by starting the include file
(before |\ProvidesPackage|) with:
%
\begin{center}
\begin{tabular}{l}
|\input{childdoc.def}|\\
|\childdocforward{|\textit{main}|}|\\
\end{tabular}
\end{center}
%
or alternatively with:
%
\begin{center}
\begin{tabular}{l}
|\input{childdoc.def}|\\
|\childdocby{|\textit{main}|}|\\
\end{tabular}
\end{center}
%
Both forms have slightly different effects as described above.
The main file is prepared as usual, see \secref{sec:include}.

%%%%%%%%%%%%%%%%%%%%%%%%%%%%%%%%%%%%%%%%%%%%%%%%%%%%%%%%%%%%%%%%%%%%%%%%%%%%%%%%
\subsection{Legacy Detection}
\label{sec:detection}

The directive |\childdocmain| in the main file can detect
whether the complete document or merely a child is to be compiled
even without using the directive |\childdocof|.
This method is deprecated because it is less robust
and there is no compelling reason to use it;
it is merely provided for backward compatibility
and it may be removed in future versions.

If the detection mechanism is to be used,
it is mandatory to correctly specify
the filename of the main file as the argument of |\childdocmain|:
%
\begin{center}
\begin{tabular}{l}
|\input{childdoc.def}|\\
|\childdocmain{|\textit{main}|}|\\
\end{tabular}
\end{center}
%
If |\jobname| does not match the argument \textit{main} of |\childdocmain|,
it is assumed that |\jobname| points to the child file to be compiled.
When using |\childdocmain| with the main file specified as argument,
it suffices to start a child file
with just |\input{|\textit{main}|}|
without loading of the package and using |\childdocof|.
If instead all processing is done
with the appropriate \textsf{childdoc} directives,
the argument of \textit{main} of |\childdocmain| can be empty.

An alternative version of the command line processing described
in \secref{sec:commandline} using the detection mechanism reads:
%
\begin{center}
|... -jobname "|\textit{target}|" "|[\textit{flags}]%
[|\def\jobname{|\textit{dest}|}|]|\input{|\textit{main}|}"|
\end{center}

%%%%%%%%%%%%%%%%%%%%%%%%%%%%%%%%%%%%%%%%%%%%%%%%%%%%%%%%%%%%%%%%%%%%%%%%%%%%%%%%
\subsection{Manual Code}
\label{sec:manual}

In case one cannot be certain whether the definitions file |childdoc.def|
is installed on the target \TeX{} distribution
and one prefers not to ship it,
it is conceivable to paste a few relevant commands into the sources.

To that end, drop all statements |\input{childdoc.def}|
and perform the replacements as outlined below.
Instead of |\childdocmain{|\textit{main}|}| add the following code
to the top of the main file:
%
\begin{center}
\begin{tabular}{l}
|\||ifdefined\childdocname\endinput\||fi\newif\ifchilddoc|\\
|\edef\childdocname{\scantokens\expandafter{\jobname\noexpand}}|\\
|\def\childdocmain{|\textit{main}|}\||ifx\childdocmain\childdocname\||else|\\
|\childdoctrue\includeonly{\childdocname}\let\jobname\childdocmain\||fi|\\
\end{tabular}
\end{center}
%
Instead of |\childdocof{|\textit{main}|}| just include the main file
at the top of each child file:
%
\begin{center}
|\input{|\textit{main}|}|
\end{center}
%
A simple redirection |\childdocforward{|\textit{dest}|}| is achieved by:
%
\begin{center}
|\def\jobname{|\textit{dest}|}\input{\jobname}|
\end{center}
%
The redirection with prefix
|\childdocforwardprefix[|\textit{prefix}|]{|\textit{dest}|}|
is accomplished by:
%
\begin{center}
\begin{tabular}{l}
|{\edef\jobname{\scantokens\expandafter{\jobname\noexpand}}|\\
|\def\redirectjob |\textit{prefix}|#1~~~{\gdef\jobname{|\textit{dest}|#1}}|\\
|\expandafter\redirectjob\jobname~~~}\input{\jobname}|
\end{tabular}
\end{center}

In an alternative approach,
child documents can be compiled by a specific command line
without additional code or specific definitions:
%
\begin{center}
|... -jobname "|\textit{target}|" "|[\textit{flags}]%
|\includeonly{|\textit{dest}|}\input{|\textit{main}|}"|
\end{center}
%

%%%%%%%%%%%%%%%%%%%%%%%%%%%%%%%%%%%%%%%%%%%%%%%%%%%%%%%%%%%%%%%%%%%%%%%%%%%%%%%%
%%%%%%%%%%%%%%%%%%%%%%%%%%%%%%%%%%%%%%%%%%%%%%%%%%%%%%%%%%%%%%%%%%%%%%%%%%%%%%%%
\section{Information}

%%%%%%%%%%%%%%%%%%%%%%%%%%%%%%%%%%%%%%%%%%%%%%%%%%%%%%%%%%%%%%%%%%%%%%%%%%%%%%%%
\subsection{Copyright}

Copyright \copyright{} 2017--2018 Niklas Beisert

This work may be distributed and/or modified under the
conditions of the \LaTeX{} Project Public License, either version 1.3
of this license or (at your option) any later version.
The latest version of this license is in
  \url{http://www.latex-project.org/lppl.txt}
and version 1.3 or later is part of all distributions of \LaTeX{}
version 2005/12/01 or later.

This work has the LPPL maintenance status `maintained'.

The Current Maintainer of this work is Niklas Beisert.

This work consists of the files |README.txt|, |childdoc.ins| and |childdoc.dtx|
as well as the derived files |childdoc.def|, |cdocsamp.tex|
with |cdocsch1.tex|, |cdocsch2.tex|, |cdocspt3.tex|, |cdocspt4.tex|,
|cdocsdrf.tex|, |cdocsfn1.tex|, |cdocsfn2.tex|
as well as |childdoc.pdf|.

%%%%%%%%%%%%%%%%%%%%%%%%%%%%%%%%%%%%%%%%%%%%%%%%%%%%%%%%%%%%%%%%%%%%%%%%%%%%%%%%
\subsection{Files and Installation}

The package consists of the files:
%
\begin{center}
\begin{tabular}{ll}
    |README.txt|   & readme file \\
    |childdoc.ins| & installation file \\
    |childdoc.dtx| & source file \\
    |childdoc.def| & definition file \\
    |cdocsamp.tex| & sample main file \\
    |cdocsch1.tex| & sample include file \\
    |cdocsch2.tex| & sample include file \\
    |cdocspt3.tex| & sample part file \\
    |cdocspt4.tex| & sample part file \\
    |cdocsdrf.tex| & sample redirection file \\
    |cdocsfn1.tex| & sample redirection file \\
    |cdocsfn2.tex| & sample redirection file \\
    |childdoc.pdf| & manual
\end{tabular}
\end{center}
%
The distribution consists of the files
|README.txt|, |childdoc.ins| and |childdoc.dtx|.
%
\begin{itemize}
\item
Run (pdf)\LaTeX{} on |childdoc.dtx|
to compile the manual |childdoc.pdf| (this file).
\item
Run \LaTeX{} on |childdoc.ins| to create the definitions file |childdoc.def|
and the sample |cdocsamp.tex| with include files
|cdocsch1.tex|, |cdocsch2.tex|, |cdocspt3.tex|, |cdocspt4.tex|,
|cdocsdrf.tex|, |cdocsfn1.tex|, |cdocsfn2.tex|.
Then copy the file |childdoc.def| to an appropriate directory of your \LaTeX{}
distribution, e.g.\ \textit{texmf-root}|/tex/latex/childdoc|.
\end{itemize}

%%%%%%%%%%%%%%%%%%%%%%%%%%%%%%%%%%%%%%%%%%%%%%%%%%%%%%%%%%%%%%%%%%%%%%%%%%%%%%%%
\subsection{Related CTAN Packages}

There are several other packages which offer a similar functionality:
%
\begin{itemize}
\item
The packages
\href{http://ctan.org/pkg/docmute}{\textsf{docmute}},
\href{http://ctan.org/pkg/includex}{\textsf{includex}} and
\href{http://ctan.org/pkg/standalone}{\textsf{standalone}}
provide commands to include only the document body of
a child file thus allowing both files to be compiled individually.
\item
The packages \href{http://ctan.org/pkg/subdocs}{\textsf{subdocs}}
and \href{http://ctan.org/pkg/subfiles}{\textsf{subfiles}}
provide structures in which the main and child documents can be
encapsulated and allowing them to be compiled individually.
The inclusion mechanism is different from the conventional |\include|.
\item
The package \href{http://ctan.org/pkg/combine}{\textsf{combine}}
is an elaborate solution to combine several documents into one.
\end{itemize}
%
See also the CTAN topic \href{http://ctan.org/topic/subdocs}{\textsf{subdocs}}
for further related packages.
The present package differs from the above solutions in that
a document structure constructed with the conventional |\include| mechanism
just needs two extra commands at the top of every file
such that all constituent files can be compiled individually.

%%%%%%%%%%%%%%%%%%%%%%%%%%%%%%%%%%%%%%%%%%%%%%%%%%%%%%%%%%%%%%%%%%%%%%%%%%%%%%%%
%\subsection{Feature Suggestions}
%
%The following is a list of features which may be useful for future
%versions of this package:
%%
%\begin{itemize}
%\item
%\ldots
%\end{itemize}

%%%%%%%%%%%%%%%%%%%%%%%%%%%%%%%%%%%%%%%%%%%%%%%%%%%%%%%%%%%%%%%%%%%%%%%%%%%%%%%%
\subsection{Revision History}

%%%%%%%%%%%%%%%%%%%%%%%%%%%%%%%%%%%%%%%%
\paragraph{v2.0:} 2018/12/30

\begin{itemize}
\item
immediate forward processing
\item
added |\childdocby| mechanism
\item
manual restructured
\end{itemize}

%%%%%%%%%%%%%%%%%%%%%%%%%%%%%%%%%%%%%%%%
\paragraph{v1.6:} 2018/01/17

\begin{itemize}
\item
application for development of include files
\item
corrections to manual
\end{itemize}

%%%%%%%%%%%%%%%%%%%%%%%%%%%%%%%%%%%%%%%%
\paragraph{v1.5:} 2017/05/21

\begin{itemize}
\item
more complete structuring introduced
\item
|\childdocof| introduced
\item
|\childdoc| renamed to |\childdocmain|
\item
|\childredirect| renamed to |\childdocforward| and |\childdocforwardprefix|
and functionality expanded
\end{itemize}

%%%%%%%%%%%%%%%%%%%%%%%%%%%%%%%%%%%%%%%%
\paragraph{v1.0:} 2017/04/27

\begin{itemize}
\item
manual and install package
\item
first version published on CTAN
\end{itemize}

%%%%%%%%%%%%%%%%%%%%%%%%%%%%%%%%%%%%%%%%
\paragraph{v0.6:} 2017/04/26

\begin{itemize}
\item
redirection mechanism added
\end{itemize}

%%%%%%%%%%%%%%%%%%%%%%%%%%%%%%%%%%%%%%%%
\paragraph{v0.5:} 2017/04/26

\begin{itemize}
\item
functionality in definition file
\end{itemize}


%%%%%%%%%%%%%%%%%%%%%%%%%%%%%%%%%%%%%%%%%%%%%%%%%%%%%%%%%%%%%%%%%%%%%%%%%%%%%%%%
%%%%%%%%%%%%%%%%%%%%%%%%%%%%%%%%%%%%%%%%%%%%%%%%%%%%%%%%%%%%%%%%%%%%%%%%%%%%%%%%
%%%%%%%%%%%%%%%%%%%%%%%%%%%%%%%%%%%%%%%%%%%%%%%%%%%%%%%%%%%%%%%%%%%%%%%%%%%%%%%%
\appendix

\settowidth\MacroIndent{\rmfamily\scriptsize 000\ }

 \DocInput{childdoc.dtx}

\end{document}
%</driver>
% \fi
%
% %%%%%%%%%%%%%%%%%%%%%%%%%%%%%%%%%%%%%%%%%%%%%%%%%%%%%%%%%%%%%%%%%%%%%%%%%%%%%%
% %%%%%%%%%%%%%%%%%%%%%%%%%%%%%%%%%%%%%%%%%%%%%%%%%%%%%%%%%%%%%%%%%%%%%%%%%%%%%%
% \section{Sample}
%\iffalse
%<*samplemain>
%\fi
%
% The following presents a sample document
% with two chapters, two parts, a title page,
% a compile flag as well as three forwarding files to set the flag.
% It consists of eight |.tex| files:
% \begin{center}
% \begin{tabular}{ll}
% |cdocsamp.tex|&main file\\
% |cdocsch1.tex|&include file for chapter 1\\
% |cdocsch2.tex|&include file for chapter 2\\
% |cdocspt3.tex|&include file for part 3\\
% |cdocspt4.tex|&include file for part 4\\
% |cdocsdrf.tex|&forwarding file for main file in draft mode\\
% |cdocsfi1.tex|&forwarding file for final version of chapter 1\\
% |cdocsfi2.tex|&forwarding file for final version of chapter 2\\
% \end{tabular}
% \end{center}
% Each of the eight files can be compiled directly by the \LaTeX{} compiler.
%
% %%%%%%%%%%%%%%%%%%%%%%%%%%%%%%%%%%%%%%
% \paragraph{Main File.}
%
% The main file is called |cdocsamp.tex|.
%
% Load the \textsf{childdoc} definitions and
% declare the filename for the main document:
%    \begin{macrocode}
\input{childdoc.def}
\childdocmain{}
%    \end{macrocode}

% Optional override for |\version| flag:
%    \begin{macrocode}
%%\ifchilddoc\else\providecommand{\version}{draft}\fi
%    \end{macrocode}

% Define the default values for the |\version| flag
% (|final| for the main file and |draft| for childs):
%    \begin{macrocode}
\ifchilddoc
\providecommand{\version}{draft}
\else
\providecommand{\version}{final}
\fi
%    \end{macrocode}

% Load the standard document class:
%    \begin{macrocode}
\documentclass[12pt]{article}
%    \end{macrocode}

% Start the document body:
%    \begin{macrocode}
\begin{document}
%    \end{macrocode}

% Declare a title page.
% Print title, part of document being processed and version flag:
%    \begin{macrocode}
\addtocounter{page}{-1}
\begin{center}
{\LARGE\bfseries{}childdoc example\par}
\vspace{1cm}
\ifchilddoc
\ifchilddocmanual part\else chapter\fi:
`\childdocname' of `\childdocjob'\par
\else
main document: `\childdocjob'\par
\fi
version: \version\par
\end{center}
\newpage
%    \end{macrocode}

% Manually include selected file,
% otherwise process as usual:
%    \begin{macrocode}
\ifchilddocmanual
\section*{part `\childdocname'}
\input{\childdocname}
\else
%    \end{macrocode}

% Include the two chapters:
%    \begin{macrocode}
\include{cdocsch1}
\include{cdocsch2}
%    \end{macrocode}

% Include the two parts unless only chapters should be displayed:
%    \begin{macrocode}
\ifchilddoc\else
\section{part three}
\input{cdocspt3}
\section{part four}
\input{cdocspt4}
\fi
%    \end{macrocode}

% Process as usual until here:
%    \begin{macrocode}
\fi
%    \end{macrocode}

% End of document body:
%    \begin{macrocode}
\end{document}
%    \end{macrocode}
%\iffalse
%</samplemain>
%\fi
%
% %%%%%%%%%%%%%%%%%%%%%%%%%%%%%%%%%%%%%%
% \paragraph{Chapter Include Files.}
%
% The include files are called |cdocsch1.tex| and |cdocsch2.tex|.
%
%\iffalse
%<*samplechap1|samplechap2>
%\fi

% Optional override for |\version| flag:
%    \begin{macrocode}
%%\providecommand{\version}{final}
%    \end{macrocode}

% Include the main document:
%    \begin{macrocode}
\input{childdoc.def}
\childdocof{cdocsamp}
%    \end{macrocode}

%\iffalse
%</samplechap1|samplechap2>
%\fi
%
%\iffalse
%<*samplechap1>
%\fi
% Some text for chapter 1:
%    \begin{macrocode}
\section{one}
some text in chapter one
%    \end{macrocode}

%\iffalse
%</samplechap1>
%\fi
% Some text for chapter 2:
%\iffalse
%<*samplechap2>
%\fi
%    \begin{macrocode}
\section{two}
more text in chapter two
%    \end{macrocode}

%\iffalse
%</samplechap2>
%\fi
%
% %%%%%%%%%%%%%%%%%%%%%%%%%%%%%%%%%%%%%%
% \paragraph{Part Include Files.}
%
% The include files are called |cdocspt3.tex| and |cdocspt4.tex|.
%
%\iffalse
%<*samplepart3|samplepart4>
%\fi

% Optional override for |\version| flag:
%    \begin{macrocode}
%%\providecommand{\version}{final}
%    \end{macrocode}

% Include the main document:
%    \begin{macrocode}
\input{childdoc.def}
\childdocby{cdocsamp}
%    \end{macrocode}

%\iffalse
%</samplepart3|samplepart4>
%\fi
%
%\iffalse
%<*samplepart3>
%\fi
% Some text for part 3:
%    \begin{macrocode}
some text in part three
%    \end{macrocode}

%\iffalse
%</samplepart3>
%\fi
% Some text for part 4:
%\iffalse
%<*samplepart4>
%\fi
%    \begin{macrocode}
more text in part four
%    \end{macrocode}

%\iffalse
%</samplepart4>
%\fi
%
% %%%%%%%%%%%%%%%%%%%%%%%%%%%%%%%%%%%%%%
% \paragraph{Forwarding for a Complete Draft.}
%
% The following forwarding file |cdocsdrf.tex|
% compiles the main document in draft mode:
%\iffalse
%<*sampledraft>
%\fi
%    \begin{macrocode}
\def\version{draft}
\input{childdoc.def}
\childdocforward{cdocsamp}
%    \end{macrocode}

%\iffalse
%</sampledraft>
%\fi
%
% %%%%%%%%%%%%%%%%%%%%%%%%%%%%%%%%%%%%%%
% \paragraph{Forwarding for Final Version of the Chapters.}
%
% The following forwarding files |cdocsfn1.tex| and |cdocsfn2.tex|
% (with identical content)
% compile the final versions of the child documents
% |cdocsch1.tex| and |cdocsch2.tex|, respectively:
%\iffalse
%<*samplefinal>
%\fi
%    \begin{macrocode}
\def\version{final}
\input{childdoc.def}
\childdocforwardprefix[cdocsamp]{cdocsfn}{cdocsch}
%    \end{macrocode}

%\iffalse
%</samplefinal>
%\fi
%
% %%%%%%%%%%%%%%%%%%%%%%%%%%%%%%%%%%%%%%
% \paragraph{Command Line Processing.}
%
% The following three command lines generate the output files
% |cdocscld|, |cdocscl1| and |cdocscl2|
% which should be identical to
% |cdocsdrf|, |cdocsch1| and |cdocsfn2|, respectively:
% \begin{center}
% \begin{tabular}{l}
% |latex -jobname cdocscld \|\\
% |  "\def\version{draft}\input{childdoc.def}\childdocforward{cdocsamp}"|\\
% |latex -jobname cdocscl1 \|\\
% |  "\input{childdoc.def}\childdocforward[cdocsamp]{cdocsch1}"|\\
% |latex -jobname cdocscl2 \|\\
% |  "\def\version{final}\input{childdoc.def}\childdocforward{cdocsch2}"|
% \end{tabular}
% \end{center}
% Note that the trailing backslash on each first line
% merely continues the input to the second line
% (for convenient cut ant paste).
% Furthermore, the command |latex| can be replaced by any
% of its alternative versions such as |pdflatex|.
%
% %%%%%%%%%%%%%%%%%%%%%%%%%%%%%%%%%%%%%%%%%%%%%%%%%%%%%%%%%%%%%%%%%%%%%%%%%%%%%%
% %%%%%%%%%%%%%%%%%%%%%%%%%%%%%%%%%%%%%%%%%%%%%%%%%%%%%%%%%%%%%%%%%%%%%%%%%%%%%%
% \section{Implementation}
%\iffalse
%<*package>
%\fi
%
% This section describes the definitions file |childdoc.def|.

% The definitions cannot be loaded using |\usepackage| or |\RequirePackage|
% which has a mechanism to prevent loading a style file more than once.
% When loading the definitions by means of |\input|
% multiple instances have to be prevented manually:
%\iffalse
%This code needs to be before the `\ProvidesFile' directive
%which is defined at the beginning of this file.
%Therefore it is also placed there and commented out here.
%</package>
%<*discard>
%\fi
%    \begin{macrocode}
\ifdefined\childdocmain\endinput\fi
%    \end{macrocode}
%\iffalse
%</discard>
%<*package>
%\fi
%
% \macro{\ifchilddoc}
% \macro{\ifchilddocmanual}
% The conditional |\ifchilddoc| tells whether a
% child (true) or main (false) document is being compiled.
% The conditional |\ifchilddocmanual| tells whether
% the |\includeonly| mechanism is used (false) or
% the selection of child files must be performed manually (true).
% The definitions initialise to false:
%    \begin{macrocode}
\newif\ifchilddoc
\newif\ifchilddocmanual
%    \end{macrocode}

% \macro{\childdocname}
% \macro{\childdocjob}
% The macro |\childdocname| stores the name of the main document
% to be compiled. The macro |\childdocjob| stores the name of
% the document on which the \LaTeX{} compiler was originally invoked.
% The content of |\jobname| cannot be compared
% to filenames specified in the source due to different catcodes.
% The following code rescans |\jobname|, stores the result
% in |\childdocname| and saves a copy in |\childdocjob|:
%    \begin{macrocode}
\edef\childdocname{\scantokens\expandafter{\jobname\noexpand}}
\let\childdocjob\childdocname
%    \end{macrocode}

% \macro{\childdocdisable}
% The macro |\childdocdisable| prevents the main file
% from being processed more than once.
% At this stage, the main document command |\childdocmain|
% is assumed to be called once again where it should do nothing.
% Any subsequent call to it should prevent
% a secondary processing of the main document
% It overwrites the forwarding commands
% |\childdocof| and |\childdocforward|
% with empty macros to prevent further inclusions of the main document:
%    \begin{macrocode}
\newcommand{\childdocdisable}
{
  \renewcommand{\childdocmain}[1]{\renewcommand{\childdocmain}[1]{\endinput}}
  \renewcommand{\childdocof}[1]{}
  \renewcommand{\childdocby}[2][]{}
  \renewcommand{\childdocforward}[2][]{}
  \renewcommand{\childdocdisable}{}
}
%    \end{macrocode}

% \macro{\childdocmain}
% The macro |\childdocmain| is to be called at the top of the main file
% with nothing or the main filename (without extension) as argument.
% First, it breaks loops.
% If the argument is not empty and does not match |\childdocname|
% (which is set by the first inclusion of |childdoc.def|),
% |\ifchilddoc| is set to true, |\includeonly| is applied to the child file
% and |\jobname| is set to the main file
% (for proper handling of |.aux| files):
%    \begin{macrocode}
\newcommand{\childdocmain}[1]
{
  \childdocdisable\childdocmain{}
  \if?#1?\else
    \begingroup
      \def\childdoctmp{#1}
      \ifx\childdoctmp\childdocname
        \def\childdoctmp{}
      \else
        \def\childdoctmp
        {
          \childdoctrue
          \includeonly{\childdocname}
          \def\childdocjob{#1}
          \def\jobname{#1}
        }
      \fi
      \expandafter
    \endgroup
    \childdoctmp
  \fi
}
%    \end{macrocode}

% \macro{\childdocof}
% The command |\childdocof| redirects
% compilation to the main file |#1|.
%    \begin{macrocode}
\newcommand{\childdocof}[1]
{
  \childdocdisable
  \childdoctrue
  \includeonly{\childdocname}
  \def\jobname{#1}
  \def\childdocjob{#1}
  \input{#1}
}
%    \end{macrocode}

% \macro{\childdocby}
% The command |\childdocby| ....
%    \begin{macrocode}
\newcommand{\childdocby}[2][]
{
  \childdocdisable
  \childdoctrue
  \childdocmanualtrue
  \if?#1?\else
    \def\jobname{#2}
  \fi
  \def\childdocjob{#2}
  \input{#2}
  \endinput
}
%    \end{macrocode}

% \macro{\childdocforward}
% The command |\childdocforward| redirects
% compilation to the main file or
% (if the optional argument is given) a child file.
% Parameters are set as if the main file
% or a child file starting with |\childdocof| was compiled.
% Then compilation is handed over to the main file:
%    \begin{macrocode}
\newcommand{\childdocforward}[2][]
{
  \begingroup
    \if?#1?
      \def\childdoctmp
      {
        \def\childdocname{#2}
        \def\childdocjob{#2}
        \def\jobname{#2}
        \input{#2}
        \endinput
      }
    \else
      \def\childdoctmp
      {
        \childdocdisable
        \def\childdocname{#2}
        \childdoctrue
        \includeonly{#2}
        \def\childdocjob{#1}
        \def\jobname{#1}
        \input{#1}
        \endinput
      }
    \fi
    \expandafter
  \endgroup
  \childdoctmp
}
%    \end{macrocode}

% \macro{\childdocforwardprefix}
% The command |\childdocforwardprefix| redirects
% compilation to the main or a child file by means of a pattern.
% The prefix |#1| in the current filename is replaced by |#2|
% and the suffix of the current filename is kept
% (it is assumed that the filename does not contain the substring `|~~~|'
% which is used as a delimiter).
% Compilation is handed over to the new file by |\childdocforward|:
%    \begin{macrocode}
\newcommand{\childdocforwardprefix}[3][]
{
  \begingroup
    \def\childdocextract #2##1~~~{\def\childdoctmp{\childdocforward[#1]{#3##1}}}
    \expandafter\childdocextract\childdocname~~~
    \expandafter
  \endgroup
  \childdoctmp
}
%    \end{macrocode}

% \macro{\childdoc}
% The deprecated macro |\childdoc| is a legacy version of |\childdocmain|:
%    \begin{macrocode}
\newcommand{\childdoc}{\childdocmain}
%    \end{macrocode}

% \macro{\childdocredirect}
% The deprecated macro |\childdocredirect| is a legacy version
% of |\childdocforward| and |\childdocforwardprefix|:
%    \begin{macrocode}
\newcommand{\childdocredirect}[2][]
{
  \begingroup
    \if?#1?
      \def\childdoctmp{\childdocforward{#2}}
    \else
      \def\childdoctmp{\childdocforwardprefix{#1}{#2}}
    \fi
    \expandafter
  \endgroup
  \childdoctmp
}
%    \end{macrocode}

%\iffalse
%</package>
%\fi
%
\endinput
|\\
|\childdocforwardprefix[|\textit{main}|]{|\textit{prefix}|}{|\textit{dest}|}|
\end{tabular}
\end{center}
%
the destination file is determined by a pattern
depending on the current file:
To make this work, the current file must be called
`{\textit{prefix}\hspace{0.2em}\textit{suffix}}'
with \textit{prefix} matching precisely the argument.
Processing is then passed on to the file
`{\textit{dest}\hspace{0.2em}\textit{suffix}}'.
Surely, the same effect is achieved by
directly specifying the
argument `{\textit{dest}\hspace{0.2em}\textit{suffix}}'
in the first form.
However, that requires to set up a different file
for each child. With the alternative form of the command
all these files can have exactly the same content
which simplifies setting them up and maintaining them.

For example, the following file |draft.tex|
with a compilation flag |\version| as described in \secref{sec:flags}
compiles the main document as a draft:
%
\begin{center}
\begin{tabular}{l}
|\def\version{draft}|\\
|% \iffalse
%
% childdoc.dtx Copyright (C) 2017-2018 Niklas Beisert
%
% This work may be distributed and/or modified under the
% conditions of the LaTeX Project Public License, either version 1.3
% of this license or (at your option) any later version.
% The latest version of this license is in
%   http://www.latex-project.org/lppl.txt
% and version 1.3 or later is part of all distributions of LaTeX
% version 2005/12/01 or later.
%
% This work has the LPPL maintenance status `maintained'.
%
% The Current Maintainer of this work is Niklas Beisert.
%
% This work consists of the files childdoc.dtx and childdoc.ins
% and the derived files childdoc.def and cdocsamp.tex with
% cdocsch1.tex, cdocsch2.tex, cdocsdrf.tex, cdocsfn1.tex, cdocsfn2.tex.
%
%<package>\ifdefined\childdocmain\endinput\fi
%<package>\ProvidesFile{childdoc.def}[2018/12/30 v2.0 child document driver]
%<samplemain>\ProvidesFile{cdocsamp.tex}[2018/12/30 v2.0 sample for childdoc]
%<*driver>
%\ProvidesFile{childdoc.drv}[2018/12/30 v2.0 childdoc reference manual file]
\PassOptionsToClass{10pt,a4paper}{article}
\documentclass{ltxdoc}

\usepackage[margin=35mm]{geometry}
\usepackage{hyperref}
\usepackage{hyperxmp}
\usepackage[usenames]{color}

\hypersetup{colorlinks=true}
\hypersetup{pdfstartview=FitH}
\hypersetup{pdfpagemode=UseNone}
\hypersetup{pdfsource={}}
\hypersetup{pdflang={en-UK}}
\hypersetup{pdfcopyright={Copyright 2017-2018 Niklas Beisert.
  This work may be distributed and/or modified under the
  conditions of the LaTeX Project Public License, either version 1.3
  of this license or (at your option) any later version.}}
\hypersetup{pdflicenseurl={http://www.latex-project.org/lppl.txt}}
\hypersetup{pdfcontactaddress={ETH Zurich, ITP, HIT K,
  Wolfgang-Pauli-Strasse 27}}
\hypersetup{pdfcontactpostcode={8093}}
\hypersetup{pdfcontactcity={Zurich}}
\hypersetup{pdfcontactcountry={Switzerland}}
\hypersetup{pdfcontactemail={nbeisert@itp.phys.ethz.ch}}
\hypersetup{pdfcontacturl={http://people.phys.ethz.ch/\xmptilde nbeisert/}}

\newcommand{\secref}[1]{\hyperref[#1]{section \ref*{#1}}}

\parskip1ex
\parindent0pt
\let\olditemize\itemize
\def\itemize{\olditemize\parskip0pt}

\begin{document}

\title{The \textsf{childdoc} Package}
\hypersetup{pdftitle={The childdoc Package}}
\author{Niklas Beisert\\[2ex]
  Institut f\"ur Theoretische Physik\\
  Eidgen\"ossische Technische Hochschule Z\"urich\\
  Wolfgang-Pauli-Strasse 27, 8093 Z\"urich, Switzerland\\[1ex]
  \href{mailto:nbeisert@itp.phys.ethz.ch}
  {\texttt{nbeisert@itp.phys.ethz.ch}}}
\hypersetup{pdfauthor={Niklas Beisert}}
\hypersetup{pdfsubject={Manual for the LaTeX2e Package childdoc}}
\date{30 December 2018, \textsf{v2.0}}
\maketitle

\begin{abstract}\noindent
\textsf{childdoc} is a \LaTeXe{} package
that enables the direct compilation
of document sections included by |\include|
to individual files.
\end{abstract}

\begingroup
\parskip0ex
\tableofcontents
\endgroup

%%%%%%%%%%%%%%%%%%%%%%%%%%%%%%%%%%%%%%%%%%%%%%%%%%%%%%%%%%%%%%%%%%%%%%%%%%%%%%%%
%%%%%%%%%%%%%%%%%%%%%%%%%%%%%%%%%%%%%%%%%%%%%%%%%%%%%%%%%%%%%%%%%%%%%%%%%%%%%%%%
\section{Introduction}

\LaTeX{} provides a mechanism to structure a large document (such as a book)
into a main file and several child files (containing the chapters)
using the |\include| command.
This mechanism is beneficial for documents
which span hundreds of pages in order to
make the source file(s) more manageable.
Moreover, compilation can be restricted to
selected child files by means of the |\includeonly| command.
The latter feature can be used to reduce the compilation time while editing
(this was significantly more useful in the earlier days of \LaTeX{})
or to generate a smaller document which is easier to navigate.
Another application of |\includeonly| is to generate
documents consisting of selected parts of the complete document.

However, there are a few drawbacks of the plain |\include| mechanism:
\begin{itemize}
\item
The child files cannot be compiled on their own,
they can only be compiled via the main file.
A naive editing environment
(such as a text editor with an option
to have the current file processed by \LaTeX)
may require one to switch to the main file before compiling;
attempting to compile the child file produces errors.
\item
The main file must be modified (each time)
to adjust the |\includeonly| command
to the present needs. This easily leaves the main file in a messy state.
\item
The generated document will always carry the filename
of the main document. This is inconvenient if
several child files are to be compiled and
to be kept for distribution.
\end{itemize}

The present package provides a simple interface
to make child files individually compilable by \LaTeX{}.
Compiling a child file then has the same effect as compiling
the main file with an |\includeonly| command
to select the appropriate child.
Moreover the generated document will carry the name of the child
rather than the main file.
This resolves all three above issues.

This feature is meant to make the editing of books,
thesis documents and lecture notes somewhat more convenient.
However, the package can also be used efficiently for
composing a series of documents (such as exercise sheets)
which are typically distributed individually.
It then assists the author in generating the individual documents
(potentially in different versions)
as well as a document containing the collected series.
Another application is in developing style files
or other kinds of included material
where compilation of the style file could redirect
to a sample or test file.

%%%%%%%%%%%%%%%%%%%%%%%%%%%%%%%%%%%%%%%%%%%%%%%%%%%%%%%%%%%%%%%%%%%%%%%%%%%%%%%%
%%%%%%%%%%%%%%%%%%%%%%%%%%%%%%%%%%%%%%%%%%%%%%%%%%%%%%%%%%%%%%%%%%%%%%%%%%%%%%%%
\section{Usage}

First of all, the package \textsf{childdoc} is \emph{not} a standard
\LaTeXe{} |.sty| style file! Therefore it needs to be invoked in
a non-standard way.

%%%%%%%%%%%%%%%%%%%%%%%%%%%%%%%%%%%%%%%%%%%%%%%%%%%%%%%%%%%%%%%%%%%%%%%%%%%%%%%%
\subsection{Included Files}
\label{sec:include}

%%%%%%%%%%%%%%%%%%%%%%%%%%%%%%%%%%%%%%%%
\DescribeMacro{\childdocmain}
To use the package, add the commands
\begin{center}
\begin{tabular}{l}
|\input{childdoc.def}|\\
|\childdocmain{}|\\
\end{tabular}
\end{center}
at the very top of the main \LaTeX{} file,
in particular \emph{before} the |\documentclass| statement!
The argument of |\childdocmain| should be left empty
(but it must be present).

%%%%%%%%%%%%%%%%%%%%%%%%%%%%%%%%%%%%%%%%
\DescribeMacro{\childdocof}
Furthermore, add the commands
\begin{center}
\begin{tabular}{l}
|\input{childdoc.def}|\\
|\childdocof{|\textit{main}|}|\\
\end{tabular}
\end{center}
at the top of every child file \textit{child}
which is included by |\include{|\textit{child}|}|
from within the main file
(or at least for those files to be compiled individually).
The argument \textit{main} must be the filename of the main file.

There are a couple of
considerations in setting up the main and child documents:

%%%%%%%%%%%%%%%%%%%%%%%%%%%%%%%%%%%%%%%%
\paragraph{Restrictions.}

Please note the following restrictions:
\begin{itemize}
\item
|\childdocmain| must be called with one argument \textit{main}
to ensure compatibility with earlier version of the package.
It must either be empty (|\childdocmain{}|)
or precisely match the filename of the main file in which it is specified.
See \secref{sec:detection} for further information.
\item
The filename \textit{main} must be specified without the |.tex| extension.
\item
The filename \textit{main} is case sensitive
(even in case-insensitive file systems)
due to internal string comparison.
\item
The argument \textit{main} should be fully expanded, it cannot be a macro.
\item
Subdirectories and special characters should be avoided in filenames.
\item
The command |\childdocmain{|\textit{main}|}| must be followed by a whitespace.
It should not be followed immediately by another command
or by a comment mark `|%|'.
This is because the \TeX{} parser reads the token immediately following
the argument of |\childdocmain| and puts it
at the beginning of every child section;
however, a white\-space is ignored.
\end{itemize}

%%%%%%%%%%%%%%%%%%%%%%%%%%%%%%%%%%%%%%%%
\paragraph{Content of Main File.}

It is advisable to place all content in the child files included by |\include|.
Any output contained in the main file will appear in all child documents
unless suppressed manually;
it cannot be suppressed automatically by the |\includeonly| directive
and thus should normally be avoided.
A method to include some content in the main file
by means of conditional processing is described in \secref{sec:conditional}.

%%%%%%%%%%%%%%%%%%%%%%%%%%%%%%%%%%%%%%%%
\paragraph{Page Numbering.}

When only a part of the document is compiled,
the appropriate numbering of pages
(as well as other status parameters)
is determined from the |.aux| files.
The latter contain information from previous passes.
However this information needs to propagate through
all intermediate child documents.
Therefore the page numbering in child documents may well
be inconsistent until the complete document is compiled at least once.

A useful (if unconventional) way to always ensure a consistent
page numbering is to restart the numbering in each child document
and denote the pages by `\textit{child}|.|\textit{page}'
where \textit{child} represents the chapter/section number of the child file.
This can be achieved by the command
|\numberwithin{page}{|\textit{child}|}|
of the \textsf{amsmath} package
where \textit{child} can be |chapter| or |section|
depending on the chosen structuring.
Alternatively, one can modify the macro |\thepage| appropriately
and reset the counter |page| at the start of each child file.

%%%%%%%%%%%%%%%%%%%%%%%%%%%%%%%%%%%%%%%%%%%%%%%%%%%%%%%%%%%%%%%%%%%%%%%%%%%%%%%%
\subsection{Conditional Processing}
\label{sec:conditional}

The package provides a mechanism to compile different versions
of a document. To customise the versions further some conditional processing
can come in handy to distinguish which version is being compiled.
The package provides two macros to describe the compilation context:

%%%%%%%%%%%%%%%%%%%%%%%%%%%%%%%%%%%%%%%%
\DescribeMacro{\ifchilddoc}
The conditional |\ifchilddoc| distinguishes between the compilation of
child documents and the main document:
%
\begin{center}
|\ifchilddoc |\textit{child-code}| |[|\||else |\textit{main-code}]| \||fi|
\end{center}

%%%%%%%%%%%%%%%%%%%%%%%%%%%%%%%%%%%%%%%%
\DescribeMacro{\childdocname}
\DescribeMacro{\childdocjob}
The macro |\childdocname| contains the filename (without extension)
of the main or child file being processed.
Note that |\childdocjob| will always contain the name of the main file.

%%%%%%%%%%%%%%%%%%%%%%%%%%%%%%%%%%%%%%%%
\paragraph{Title Page.}

Conditional processing can be used to include a title or banner page
in the main document when proper precautions are taken.
Importantly, the code in the main file should ensure that the page counter
(as well as other status parameters which are stored in the |.aux| files)
takes the same value after the conditional processing.
Otherwise the page numbers may take divergent values
depending on which part is compiled.

For example, a title page could be declared by:
%
\begin{center}
\begin{tabular}{l}
|\ifchilddoc\||else|\\
|\addtocounter{page}{-1}|\\
\textit{code for title page}\\
|\newpage|\\
|\||fi|
\end{tabular}
\end{center}
%
A banner page for the child documents can be generated by:
%
\begin{center}
\begin{tabular}{l}
|\ifchilddoc|\\
|\addtocounter{page}{-1}|\\
\textit{code for banner page}\\
|\newpage|\\
|\||fi|
\end{tabular}
\end{center}
%
Here one could write a message such as:
\begin{center}
|This is the part \childdocname{} of \childdocjob{}.|
\end{center}

%%%%%%%%%%%%%%%%%%%%%%%%%%%%%%%%%%%%%%%%%%%%%%%%%%%%%%%%%%%%%%%%%%%%%%%%%%%%%%%%
\subsection{Flags}
\label{sec:flags}

The package makes it easy to generate different versions
of the main or child documents.
To this end compilation flags can be defined
and assigned different default values.
They will be particularly useful in conjunction
with the forwarding mechanism described in \secref{sec:forward}.

For example, it may be useful to have a flag |\version|
which can be set to |draft| or |final|.
The document source will contain some conditional code
depending on the value of |\version|.
Suppose further, the flag should default to |final| for the main file
and to |draft| for child files
which is a natural assignment for editing the document.
This is achieved by placing the following code
in the preamble of the main document
(below the |\childdocmain| directive):
%
\begin{center}
\begin{tabular}{l}
|\ifchilddoc|\\
|\providecommand{\version}{draft}|\\
|\||else|\\
|\providecommand{\version}{final}|\\
|\||fi|
\end{tabular}
\end{center}
%
The definition by |\providecommand| makes sure
that previous definitions are not overwritten.
Further statements |\providecommand{\version}{...}|
can thus be added before the above code to override it.

For the main file, one might add a line
(between |\childdocmain| and the above block)
%
\begin{center}
|%\ifchilddoc\||else\providecommand{\version}{draft}\||fi|
\end{center}
%
which can be uncommented to produce a draft version.
Likewise one can add a line to the very top of a child file
(above the |\childdocof{|\textit{main}|}| directive)
%
\begin{center}
|%\providecommand{\version}{final}|
\end{center}
%
which can be uncommented to produce the final version of this child document.

%%%%%%%%%%%%%%%%%%%%%%%%%%%%%%%%%%%%%%%%%%%%%%%%%%%%%%%%%%%%%%%%%%%%%%%%%%%%%%%%
\subsection{Forwarding}
\label{sec:forward}

Different versions of the main or child documents
using compilation flags as described in \secref{sec:flags}
can be (permanently) stored in different files
for convenient compilation, viewing and distribution.
To this end, the package defines a command
to pass on compilation to a different file:

%%%%%%%%%%%%%%%%%%%%%%%%%%%%%%%%%%%%%%%%
\DescribeMacro{\childdocforward}
The command |\childdocforward| redirects processing to
another source file:
%
\begin{center}
\begin{tabular}{l}
|\input{childdoc.def}|\\
|\childdocforward[|\textit{main}|]{|\textit{dest}|}|\\
\end{tabular}
\end{center}
%
The argument \textit{dest} is the destination file
(without extension).
It should be the main file or one of the child files.
Note that further \textsf{childdoc} directives
such as |\childdocof| and |\childdocforward|
in the indicated file will be processed in this form.
The optional argument \textit{main}
passes on directly to the main file \textit{main}
while pretending to compile the child \textit{dest}.
This form behaves as if \textit{dest}
issues |\childdocof{|\textit{main}|}| right away,
and no further \textsf{childdoc} directives will be processed.

%%%%%%%%%%%%%%%%%%%%%%%%%%%%%%%%%%%%%%%%
\DescribeMacro{\...prefix}
In the alternative form |\childdocforwardprefix|,
%
\begin{center}
\begin{tabular}{l}
|\input{childdoc.def}|\\
|\childdocforwardprefix[|\textit{main}|]{|\textit{prefix}|}{|\textit{dest}|}|
\end{tabular}
\end{center}
%
the destination file is determined by a pattern
depending on the current file:
To make this work, the current file must be called
`{\textit{prefix}\hspace{0.2em}\textit{suffix}}'
with \textit{prefix} matching precisely the argument.
Processing is then passed on to the file
`{\textit{dest}\hspace{0.2em}\textit{suffix}}'.
Surely, the same effect is achieved by
directly specifying the
argument `{\textit{dest}\hspace{0.2em}\textit{suffix}}'
in the first form.
However, that requires to set up a different file
for each child. With the alternative form of the command
all these files can have exactly the same content
which simplifies setting them up and maintaining them.

For example, the following file |draft.tex|
with a compilation flag |\version| as described in \secref{sec:flags}
compiles the main document as a draft:
%
\begin{center}
\begin{tabular}{l}
|\def\version{draft}|\\
|\input{childdoc.def}|\\
|\childdocforward{|\textit{main}|}|
\end{tabular}
\end{center}
%
Likewise, the following files |final|\textit{nn}|.tex|
compile the final version of the child document
|child|\textit{nn}|.tex|:
%
\begin{center}
\begin{tabular}{l}
|\def\version{final}|\\
|\input{childdoc.def}|\\
|\childdocforwardprefix{final}{child}|
\end{tabular}
\end{center}
%

Note that when several versions of a main file and/or of each child file
are to be generated, it may be convenient to set up a |Makefile| or
shell script to automatise the process.

%%%%%%%%%%%%%%%%%%%%%%%%%%%%%%%%%%%%%%%%%%%%%%%%%%%%%%%%%%%%%%%%%%%%%%%%%%%%%%%%
\subsection{Command Line Processing}
\label{sec:commandline}

The effect of redirection files can also be achieved by invoking
the \LaTeX{} compiler with a more elaborate command line.
Most conveniently this should be done as part
of a shell script or a |Makefile|.

When using \textsf{childdoc} in the main file, the following
command lines effectively perform a redirection
(note that depending on the shell being used,
backslashes may have to be doubled: `|\|' $\to$ `|\\|'):
%
\begin{center}
|... -jobname "|\textit{target}|" |\\|"|[\textit{flags}]%
|\input{childdoc.def}\childdocforward[|\textit{main}|]{|\textit{dest}|}"|
\end{center}
%
Here \textit{target} is the name of the output file,
\textit{main} is the name of the main file
and \textit{dest} is the name of the main or child file to be processed
(all filenames without extensions).
The optional argument \textit{main} can be omitted
if \textit{main} matches \textit{dest}.
Optionally, compilation \textit{flags} can be defined via |\def| commands.
This command line makes the \TeX{} engine believe
it is compiling the file \textit{target}
whose content is specified as the latter parameter.
The provided code then forwards the processing to
\textit{main} or \textit{dest} as described in \secref{sec:forward}.

%%%%%%%%%%%%%%%%%%%%%%%%%%%%%%%%%%%%%%%%%%%%%%%%%%%%%%%%%%%%%%%%%%%%%%%%%%%%%%%%
\subsection{Include by Input}
\label{sec:input}

Including child documents by |\include| has some restrictions by design.
Most notably, the content of a child document always occupies
its own set of pages; pages cannot be shared between child documents.
Usually, this behaviour makes perfect sense
because each child document contain an essential part of the document.
However, in some situations it may be desirable to compose
a document from a collection of parts
without having mandatory page breaks between then.
For this case, the package
provides a mechanism to include parts
by |\input| which can also be processed individually.
However, by construction this mechanism
requires manual handling of the content to be output.

%%%%%%%%%%%%%%%%%%%%%%%%%%%%%%%%%%%%%%%%
\DescribeMacro{\ifchilddocmanual}
The main file should be prepared as usual, see \secref{sec:include}.
However, the document body must make a distinction
between processing of an individual part and of the main document, e.g.:
%
\begin{center}
\begin{tabular}{l}
|\ifchilddocmanual|\\
|\input{\childdocname}|\\
|\||else|\\
\textit{document body with }|\input{|\textit{part}|}|\\
|\||fi|
\end{tabular}
\end{center}
%
The conditional |\ifchilddocmanual| is true whenever
a part to be included by |\input| is being compiled,
and the name of the part is stored in |\childdocname|.

%%%%%%%%%%%%%%%%%%%%%%%%%%%%%%%%%%%%%%%%
\DescribeMacro{\childdocby}
Each part to be included by |\input| should start with:
%
\begin{center}
\begin{tabular}{l}
|\input{childdoc.def}|\\
|\childdocby{|\textit{main}|}|\\
\end{tabular}
\end{center}
%
The directive |\childdocby| is similar to |\childdocof|
described in \secref{sec:include},
but the subsequent selection of content must be done manually.
To that end, both |\ifchilddoc| and |\ifchilddocmanual|
will be true upon processing of a part,
and the name of the part is stored in |\childdocname|.
Note that |\jobname| will be set to the filename of the current part
so that each part receives an individual |.aux| file
that does not interfere with the |.aux| file(s) of the main document.
This behaviour can be altered by the alternative form
|\childdocby[*]{|\textit{main}|}| (with a non-empty optional argument)
which uses the |.aux| file of the main document
by setting |\jobname| to \textit{main}.

%%%%%%%%%%%%%%%%%%%%%%%%%%%%%%%%%%%%%%%%%%%%%%%%%%%%%%%%%%%%%%%%%%%%%%%%%%%%%%%%
\subsection{Driver Development}
\label{sec:driver}

The \textsf{childdoc} mechanism can also be use for the development
of definition files such as \LaTeX{} styles or classes.
This case differs from the above setup with multiple parts
included by |\include| in that no |\includeonly| should be invoked.
This can be achieved by starting the include file
(before |\ProvidesPackage|) with:
%
\begin{center}
\begin{tabular}{l}
|\input{childdoc.def}|\\
|\childdocforward{|\textit{main}|}|\\
\end{tabular}
\end{center}
%
or alternatively with:
%
\begin{center}
\begin{tabular}{l}
|\input{childdoc.def}|\\
|\childdocby{|\textit{main}|}|\\
\end{tabular}
\end{center}
%
Both forms have slightly different effects as described above.
The main file is prepared as usual, see \secref{sec:include}.

%%%%%%%%%%%%%%%%%%%%%%%%%%%%%%%%%%%%%%%%%%%%%%%%%%%%%%%%%%%%%%%%%%%%%%%%%%%%%%%%
\subsection{Legacy Detection}
\label{sec:detection}

The directive |\childdocmain| in the main file can detect
whether the complete document or merely a child is to be compiled
even without using the directive |\childdocof|.
This method is deprecated because it is less robust
and there is no compelling reason to use it;
it is merely provided for backward compatibility
and it may be removed in future versions.

If the detection mechanism is to be used,
it is mandatory to correctly specify
the filename of the main file as the argument of |\childdocmain|:
%
\begin{center}
\begin{tabular}{l}
|\input{childdoc.def}|\\
|\childdocmain{|\textit{main}|}|\\
\end{tabular}
\end{center}
%
If |\jobname| does not match the argument \textit{main} of |\childdocmain|,
it is assumed that |\jobname| points to the child file to be compiled.
When using |\childdocmain| with the main file specified as argument,
it suffices to start a child file
with just |\input{|\textit{main}|}|
without loading of the package and using |\childdocof|.
If instead all processing is done
with the appropriate \textsf{childdoc} directives,
the argument of \textit{main} of |\childdocmain| can be empty.

An alternative version of the command line processing described
in \secref{sec:commandline} using the detection mechanism reads:
%
\begin{center}
|... -jobname "|\textit{target}|" "|[\textit{flags}]%
[|\def\jobname{|\textit{dest}|}|]|\input{|\textit{main}|}"|
\end{center}

%%%%%%%%%%%%%%%%%%%%%%%%%%%%%%%%%%%%%%%%%%%%%%%%%%%%%%%%%%%%%%%%%%%%%%%%%%%%%%%%
\subsection{Manual Code}
\label{sec:manual}

In case one cannot be certain whether the definitions file |childdoc.def|
is installed on the target \TeX{} distribution
and one prefers not to ship it,
it is conceivable to paste a few relevant commands into the sources.

To that end, drop all statements |\input{childdoc.def}|
and perform the replacements as outlined below.
Instead of |\childdocmain{|\textit{main}|}| add the following code
to the top of the main file:
%
\begin{center}
\begin{tabular}{l}
|\||ifdefined\childdocname\endinput\||fi\newif\ifchilddoc|\\
|\edef\childdocname{\scantokens\expandafter{\jobname\noexpand}}|\\
|\def\childdocmain{|\textit{main}|}\||ifx\childdocmain\childdocname\||else|\\
|\childdoctrue\includeonly{\childdocname}\let\jobname\childdocmain\||fi|\\
\end{tabular}
\end{center}
%
Instead of |\childdocof{|\textit{main}|}| just include the main file
at the top of each child file:
%
\begin{center}
|\input{|\textit{main}|}|
\end{center}
%
A simple redirection |\childdocforward{|\textit{dest}|}| is achieved by:
%
\begin{center}
|\def\jobname{|\textit{dest}|}\input{\jobname}|
\end{center}
%
The redirection with prefix
|\childdocforwardprefix[|\textit{prefix}|]{|\textit{dest}|}|
is accomplished by:
%
\begin{center}
\begin{tabular}{l}
|{\edef\jobname{\scantokens\expandafter{\jobname\noexpand}}|\\
|\def\redirectjob |\textit{prefix}|#1~~~{\gdef\jobname{|\textit{dest}|#1}}|\\
|\expandafter\redirectjob\jobname~~~}\input{\jobname}|
\end{tabular}
\end{center}

In an alternative approach,
child documents can be compiled by a specific command line
without additional code or specific definitions:
%
\begin{center}
|... -jobname "|\textit{target}|" "|[\textit{flags}]%
|\includeonly{|\textit{dest}|}\input{|\textit{main}|}"|
\end{center}
%

%%%%%%%%%%%%%%%%%%%%%%%%%%%%%%%%%%%%%%%%%%%%%%%%%%%%%%%%%%%%%%%%%%%%%%%%%%%%%%%%
%%%%%%%%%%%%%%%%%%%%%%%%%%%%%%%%%%%%%%%%%%%%%%%%%%%%%%%%%%%%%%%%%%%%%%%%%%%%%%%%
\section{Information}

%%%%%%%%%%%%%%%%%%%%%%%%%%%%%%%%%%%%%%%%%%%%%%%%%%%%%%%%%%%%%%%%%%%%%%%%%%%%%%%%
\subsection{Copyright}

Copyright \copyright{} 2017--2018 Niklas Beisert

This work may be distributed and/or modified under the
conditions of the \LaTeX{} Project Public License, either version 1.3
of this license or (at your option) any later version.
The latest version of this license is in
  \url{http://www.latex-project.org/lppl.txt}
and version 1.3 or later is part of all distributions of \LaTeX{}
version 2005/12/01 or later.

This work has the LPPL maintenance status `maintained'.

The Current Maintainer of this work is Niklas Beisert.

This work consists of the files |README.txt|, |childdoc.ins| and |childdoc.dtx|
as well as the derived files |childdoc.def|, |cdocsamp.tex|
with |cdocsch1.tex|, |cdocsch2.tex|, |cdocspt3.tex|, |cdocspt4.tex|,
|cdocsdrf.tex|, |cdocsfn1.tex|, |cdocsfn2.tex|
as well as |childdoc.pdf|.

%%%%%%%%%%%%%%%%%%%%%%%%%%%%%%%%%%%%%%%%%%%%%%%%%%%%%%%%%%%%%%%%%%%%%%%%%%%%%%%%
\subsection{Files and Installation}

The package consists of the files:
%
\begin{center}
\begin{tabular}{ll}
    |README.txt|   & readme file \\
    |childdoc.ins| & installation file \\
    |childdoc.dtx| & source file \\
    |childdoc.def| & definition file \\
    |cdocsamp.tex| & sample main file \\
    |cdocsch1.tex| & sample include file \\
    |cdocsch2.tex| & sample include file \\
    |cdocspt3.tex| & sample part file \\
    |cdocspt4.tex| & sample part file \\
    |cdocsdrf.tex| & sample redirection file \\
    |cdocsfn1.tex| & sample redirection file \\
    |cdocsfn2.tex| & sample redirection file \\
    |childdoc.pdf| & manual
\end{tabular}
\end{center}
%
The distribution consists of the files
|README.txt|, |childdoc.ins| and |childdoc.dtx|.
%
\begin{itemize}
\item
Run (pdf)\LaTeX{} on |childdoc.dtx|
to compile the manual |childdoc.pdf| (this file).
\item
Run \LaTeX{} on |childdoc.ins| to create the definitions file |childdoc.def|
and the sample |cdocsamp.tex| with include files
|cdocsch1.tex|, |cdocsch2.tex|, |cdocspt3.tex|, |cdocspt4.tex|,
|cdocsdrf.tex|, |cdocsfn1.tex|, |cdocsfn2.tex|.
Then copy the file |childdoc.def| to an appropriate directory of your \LaTeX{}
distribution, e.g.\ \textit{texmf-root}|/tex/latex/childdoc|.
\end{itemize}

%%%%%%%%%%%%%%%%%%%%%%%%%%%%%%%%%%%%%%%%%%%%%%%%%%%%%%%%%%%%%%%%%%%%%%%%%%%%%%%%
\subsection{Related CTAN Packages}

There are several other packages which offer a similar functionality:
%
\begin{itemize}
\item
The packages
\href{http://ctan.org/pkg/docmute}{\textsf{docmute}},
\href{http://ctan.org/pkg/includex}{\textsf{includex}} and
\href{http://ctan.org/pkg/standalone}{\textsf{standalone}}
provide commands to include only the document body of
a child file thus allowing both files to be compiled individually.
\item
The packages \href{http://ctan.org/pkg/subdocs}{\textsf{subdocs}}
and \href{http://ctan.org/pkg/subfiles}{\textsf{subfiles}}
provide structures in which the main and child documents can be
encapsulated and allowing them to be compiled individually.
The inclusion mechanism is different from the conventional |\include|.
\item
The package \href{http://ctan.org/pkg/combine}{\textsf{combine}}
is an elaborate solution to combine several documents into one.
\end{itemize}
%
See also the CTAN topic \href{http://ctan.org/topic/subdocs}{\textsf{subdocs}}
for further related packages.
The present package differs from the above solutions in that
a document structure constructed with the conventional |\include| mechanism
just needs two extra commands at the top of every file
such that all constituent files can be compiled individually.

%%%%%%%%%%%%%%%%%%%%%%%%%%%%%%%%%%%%%%%%%%%%%%%%%%%%%%%%%%%%%%%%%%%%%%%%%%%%%%%%
%\subsection{Feature Suggestions}
%
%The following is a list of features which may be useful for future
%versions of this package:
%%
%\begin{itemize}
%\item
%\ldots
%\end{itemize}

%%%%%%%%%%%%%%%%%%%%%%%%%%%%%%%%%%%%%%%%%%%%%%%%%%%%%%%%%%%%%%%%%%%%%%%%%%%%%%%%
\subsection{Revision History}

%%%%%%%%%%%%%%%%%%%%%%%%%%%%%%%%%%%%%%%%
\paragraph{v2.0:} 2018/12/30

\begin{itemize}
\item
immediate forward processing
\item
added |\childdocby| mechanism
\item
manual restructured
\end{itemize}

%%%%%%%%%%%%%%%%%%%%%%%%%%%%%%%%%%%%%%%%
\paragraph{v1.6:} 2018/01/17

\begin{itemize}
\item
application for development of include files
\item
corrections to manual
\end{itemize}

%%%%%%%%%%%%%%%%%%%%%%%%%%%%%%%%%%%%%%%%
\paragraph{v1.5:} 2017/05/21

\begin{itemize}
\item
more complete structuring introduced
\item
|\childdocof| introduced
\item
|\childdoc| renamed to |\childdocmain|
\item
|\childredirect| renamed to |\childdocforward| and |\childdocforwardprefix|
and functionality expanded
\end{itemize}

%%%%%%%%%%%%%%%%%%%%%%%%%%%%%%%%%%%%%%%%
\paragraph{v1.0:} 2017/04/27

\begin{itemize}
\item
manual and install package
\item
first version published on CTAN
\end{itemize}

%%%%%%%%%%%%%%%%%%%%%%%%%%%%%%%%%%%%%%%%
\paragraph{v0.6:} 2017/04/26

\begin{itemize}
\item
redirection mechanism added
\end{itemize}

%%%%%%%%%%%%%%%%%%%%%%%%%%%%%%%%%%%%%%%%
\paragraph{v0.5:} 2017/04/26

\begin{itemize}
\item
functionality in definition file
\end{itemize}


%%%%%%%%%%%%%%%%%%%%%%%%%%%%%%%%%%%%%%%%%%%%%%%%%%%%%%%%%%%%%%%%%%%%%%%%%%%%%%%%
%%%%%%%%%%%%%%%%%%%%%%%%%%%%%%%%%%%%%%%%%%%%%%%%%%%%%%%%%%%%%%%%%%%%%%%%%%%%%%%%
%%%%%%%%%%%%%%%%%%%%%%%%%%%%%%%%%%%%%%%%%%%%%%%%%%%%%%%%%%%%%%%%%%%%%%%%%%%%%%%%
\appendix

\settowidth\MacroIndent{\rmfamily\scriptsize 000\ }

 \DocInput{childdoc.dtx}

\end{document}
%</driver>
% \fi
%
% %%%%%%%%%%%%%%%%%%%%%%%%%%%%%%%%%%%%%%%%%%%%%%%%%%%%%%%%%%%%%%%%%%%%%%%%%%%%%%
% %%%%%%%%%%%%%%%%%%%%%%%%%%%%%%%%%%%%%%%%%%%%%%%%%%%%%%%%%%%%%%%%%%%%%%%%%%%%%%
% \section{Sample}
%\iffalse
%<*samplemain>
%\fi
%
% The following presents a sample document
% with two chapters, two parts, a title page,
% a compile flag as well as three forwarding files to set the flag.
% It consists of eight |.tex| files:
% \begin{center}
% \begin{tabular}{ll}
% |cdocsamp.tex|&main file\\
% |cdocsch1.tex|&include file for chapter 1\\
% |cdocsch2.tex|&include file for chapter 2\\
% |cdocspt3.tex|&include file for part 3\\
% |cdocspt4.tex|&include file for part 4\\
% |cdocsdrf.tex|&forwarding file for main file in draft mode\\
% |cdocsfi1.tex|&forwarding file for final version of chapter 1\\
% |cdocsfi2.tex|&forwarding file for final version of chapter 2\\
% \end{tabular}
% \end{center}
% Each of the eight files can be compiled directly by the \LaTeX{} compiler.
%
% %%%%%%%%%%%%%%%%%%%%%%%%%%%%%%%%%%%%%%
% \paragraph{Main File.}
%
% The main file is called |cdocsamp.tex|.
%
% Load the \textsf{childdoc} definitions and
% declare the filename for the main document:
%    \begin{macrocode}
\input{childdoc.def}
\childdocmain{}
%    \end{macrocode}

% Optional override for |\version| flag:
%    \begin{macrocode}
%%\ifchilddoc\else\providecommand{\version}{draft}\fi
%    \end{macrocode}

% Define the default values for the |\version| flag
% (|final| for the main file and |draft| for childs):
%    \begin{macrocode}
\ifchilddoc
\providecommand{\version}{draft}
\else
\providecommand{\version}{final}
\fi
%    \end{macrocode}

% Load the standard document class:
%    \begin{macrocode}
\documentclass[12pt]{article}
%    \end{macrocode}

% Start the document body:
%    \begin{macrocode}
\begin{document}
%    \end{macrocode}

% Declare a title page.
% Print title, part of document being processed and version flag:
%    \begin{macrocode}
\addtocounter{page}{-1}
\begin{center}
{\LARGE\bfseries{}childdoc example\par}
\vspace{1cm}
\ifchilddoc
\ifchilddocmanual part\else chapter\fi:
`\childdocname' of `\childdocjob'\par
\else
main document: `\childdocjob'\par
\fi
version: \version\par
\end{center}
\newpage
%    \end{macrocode}

% Manually include selected file,
% otherwise process as usual:
%    \begin{macrocode}
\ifchilddocmanual
\section*{part `\childdocname'}
\input{\childdocname}
\else
%    \end{macrocode}

% Include the two chapters:
%    \begin{macrocode}
\include{cdocsch1}
\include{cdocsch2}
%    \end{macrocode}

% Include the two parts unless only chapters should be displayed:
%    \begin{macrocode}
\ifchilddoc\else
\section{part three}
\input{cdocspt3}
\section{part four}
\input{cdocspt4}
\fi
%    \end{macrocode}

% Process as usual until here:
%    \begin{macrocode}
\fi
%    \end{macrocode}

% End of document body:
%    \begin{macrocode}
\end{document}
%    \end{macrocode}
%\iffalse
%</samplemain>
%\fi
%
% %%%%%%%%%%%%%%%%%%%%%%%%%%%%%%%%%%%%%%
% \paragraph{Chapter Include Files.}
%
% The include files are called |cdocsch1.tex| and |cdocsch2.tex|.
%
%\iffalse
%<*samplechap1|samplechap2>
%\fi

% Optional override for |\version| flag:
%    \begin{macrocode}
%%\providecommand{\version}{final}
%    \end{macrocode}

% Include the main document:
%    \begin{macrocode}
\input{childdoc.def}
\childdocof{cdocsamp}
%    \end{macrocode}

%\iffalse
%</samplechap1|samplechap2>
%\fi
%
%\iffalse
%<*samplechap1>
%\fi
% Some text for chapter 1:
%    \begin{macrocode}
\section{one}
some text in chapter one
%    \end{macrocode}

%\iffalse
%</samplechap1>
%\fi
% Some text for chapter 2:
%\iffalse
%<*samplechap2>
%\fi
%    \begin{macrocode}
\section{two}
more text in chapter two
%    \end{macrocode}

%\iffalse
%</samplechap2>
%\fi
%
% %%%%%%%%%%%%%%%%%%%%%%%%%%%%%%%%%%%%%%
% \paragraph{Part Include Files.}
%
% The include files are called |cdocspt3.tex| and |cdocspt4.tex|.
%
%\iffalse
%<*samplepart3|samplepart4>
%\fi

% Optional override for |\version| flag:
%    \begin{macrocode}
%%\providecommand{\version}{final}
%    \end{macrocode}

% Include the main document:
%    \begin{macrocode}
\input{childdoc.def}
\childdocby{cdocsamp}
%    \end{macrocode}

%\iffalse
%</samplepart3|samplepart4>
%\fi
%
%\iffalse
%<*samplepart3>
%\fi
% Some text for part 3:
%    \begin{macrocode}
some text in part three
%    \end{macrocode}

%\iffalse
%</samplepart3>
%\fi
% Some text for part 4:
%\iffalse
%<*samplepart4>
%\fi
%    \begin{macrocode}
more text in part four
%    \end{macrocode}

%\iffalse
%</samplepart4>
%\fi
%
% %%%%%%%%%%%%%%%%%%%%%%%%%%%%%%%%%%%%%%
% \paragraph{Forwarding for a Complete Draft.}
%
% The following forwarding file |cdocsdrf.tex|
% compiles the main document in draft mode:
%\iffalse
%<*sampledraft>
%\fi
%    \begin{macrocode}
\def\version{draft}
\input{childdoc.def}
\childdocforward{cdocsamp}
%    \end{macrocode}

%\iffalse
%</sampledraft>
%\fi
%
% %%%%%%%%%%%%%%%%%%%%%%%%%%%%%%%%%%%%%%
% \paragraph{Forwarding for Final Version of the Chapters.}
%
% The following forwarding files |cdocsfn1.tex| and |cdocsfn2.tex|
% (with identical content)
% compile the final versions of the child documents
% |cdocsch1.tex| and |cdocsch2.tex|, respectively:
%\iffalse
%<*samplefinal>
%\fi
%    \begin{macrocode}
\def\version{final}
\input{childdoc.def}
\childdocforwardprefix[cdocsamp]{cdocsfn}{cdocsch}
%    \end{macrocode}

%\iffalse
%</samplefinal>
%\fi
%
% %%%%%%%%%%%%%%%%%%%%%%%%%%%%%%%%%%%%%%
% \paragraph{Command Line Processing.}
%
% The following three command lines generate the output files
% |cdocscld|, |cdocscl1| and |cdocscl2|
% which should be identical to
% |cdocsdrf|, |cdocsch1| and |cdocsfn2|, respectively:
% \begin{center}
% \begin{tabular}{l}
% |latex -jobname cdocscld \|\\
% |  "\def\version{draft}\input{childdoc.def}\childdocforward{cdocsamp}"|\\
% |latex -jobname cdocscl1 \|\\
% |  "\input{childdoc.def}\childdocforward[cdocsamp]{cdocsch1}"|\\
% |latex -jobname cdocscl2 \|\\
% |  "\def\version{final}\input{childdoc.def}\childdocforward{cdocsch2}"|
% \end{tabular}
% \end{center}
% Note that the trailing backslash on each first line
% merely continues the input to the second line
% (for convenient cut ant paste).
% Furthermore, the command |latex| can be replaced by any
% of its alternative versions such as |pdflatex|.
%
% %%%%%%%%%%%%%%%%%%%%%%%%%%%%%%%%%%%%%%%%%%%%%%%%%%%%%%%%%%%%%%%%%%%%%%%%%%%%%%
% %%%%%%%%%%%%%%%%%%%%%%%%%%%%%%%%%%%%%%%%%%%%%%%%%%%%%%%%%%%%%%%%%%%%%%%%%%%%%%
% \section{Implementation}
%\iffalse
%<*package>
%\fi
%
% This section describes the definitions file |childdoc.def|.

% The definitions cannot be loaded using |\usepackage| or |\RequirePackage|
% which has a mechanism to prevent loading a style file more than once.
% When loading the definitions by means of |\input|
% multiple instances have to be prevented manually:
%\iffalse
%This code needs to be before the `\ProvidesFile' directive
%which is defined at the beginning of this file.
%Therefore it is also placed there and commented out here.
%</package>
%<*discard>
%\fi
%    \begin{macrocode}
\ifdefined\childdocmain\endinput\fi
%    \end{macrocode}
%\iffalse
%</discard>
%<*package>
%\fi
%
% \macro{\ifchilddoc}
% \macro{\ifchilddocmanual}
% The conditional |\ifchilddoc| tells whether a
% child (true) or main (false) document is being compiled.
% The conditional |\ifchilddocmanual| tells whether
% the |\includeonly| mechanism is used (false) or
% the selection of child files must be performed manually (true).
% The definitions initialise to false:
%    \begin{macrocode}
\newif\ifchilddoc
\newif\ifchilddocmanual
%    \end{macrocode}

% \macro{\childdocname}
% \macro{\childdocjob}
% The macro |\childdocname| stores the name of the main document
% to be compiled. The macro |\childdocjob| stores the name of
% the document on which the \LaTeX{} compiler was originally invoked.
% The content of |\jobname| cannot be compared
% to filenames specified in the source due to different catcodes.
% The following code rescans |\jobname|, stores the result
% in |\childdocname| and saves a copy in |\childdocjob|:
%    \begin{macrocode}
\edef\childdocname{\scantokens\expandafter{\jobname\noexpand}}
\let\childdocjob\childdocname
%    \end{macrocode}

% \macro{\childdocdisable}
% The macro |\childdocdisable| prevents the main file
% from being processed more than once.
% At this stage, the main document command |\childdocmain|
% is assumed to be called once again where it should do nothing.
% Any subsequent call to it should prevent
% a secondary processing of the main document
% It overwrites the forwarding commands
% |\childdocof| and |\childdocforward|
% with empty macros to prevent further inclusions of the main document:
%    \begin{macrocode}
\newcommand{\childdocdisable}
{
  \renewcommand{\childdocmain}[1]{\renewcommand{\childdocmain}[1]{\endinput}}
  \renewcommand{\childdocof}[1]{}
  \renewcommand{\childdocby}[2][]{}
  \renewcommand{\childdocforward}[2][]{}
  \renewcommand{\childdocdisable}{}
}
%    \end{macrocode}

% \macro{\childdocmain}
% The macro |\childdocmain| is to be called at the top of the main file
% with nothing or the main filename (without extension) as argument.
% First, it breaks loops.
% If the argument is not empty and does not match |\childdocname|
% (which is set by the first inclusion of |childdoc.def|),
% |\ifchilddoc| is set to true, |\includeonly| is applied to the child file
% and |\jobname| is set to the main file
% (for proper handling of |.aux| files):
%    \begin{macrocode}
\newcommand{\childdocmain}[1]
{
  \childdocdisable\childdocmain{}
  \if?#1?\else
    \begingroup
      \def\childdoctmp{#1}
      \ifx\childdoctmp\childdocname
        \def\childdoctmp{}
      \else
        \def\childdoctmp
        {
          \childdoctrue
          \includeonly{\childdocname}
          \def\childdocjob{#1}
          \def\jobname{#1}
        }
      \fi
      \expandafter
    \endgroup
    \childdoctmp
  \fi
}
%    \end{macrocode}

% \macro{\childdocof}
% The command |\childdocof| redirects
% compilation to the main file |#1|.
%    \begin{macrocode}
\newcommand{\childdocof}[1]
{
  \childdocdisable
  \childdoctrue
  \includeonly{\childdocname}
  \def\jobname{#1}
  \def\childdocjob{#1}
  \input{#1}
}
%    \end{macrocode}

% \macro{\childdocby}
% The command |\childdocby| ....
%    \begin{macrocode}
\newcommand{\childdocby}[2][]
{
  \childdocdisable
  \childdoctrue
  \childdocmanualtrue
  \if?#1?\else
    \def\jobname{#2}
  \fi
  \def\childdocjob{#2}
  \input{#2}
  \endinput
}
%    \end{macrocode}

% \macro{\childdocforward}
% The command |\childdocforward| redirects
% compilation to the main file or
% (if the optional argument is given) a child file.
% Parameters are set as if the main file
% or a child file starting with |\childdocof| was compiled.
% Then compilation is handed over to the main file:
%    \begin{macrocode}
\newcommand{\childdocforward}[2][]
{
  \begingroup
    \if?#1?
      \def\childdoctmp
      {
        \def\childdocname{#2}
        \def\childdocjob{#2}
        \def\jobname{#2}
        \input{#2}
        \endinput
      }
    \else
      \def\childdoctmp
      {
        \childdocdisable
        \def\childdocname{#2}
        \childdoctrue
        \includeonly{#2}
        \def\childdocjob{#1}
        \def\jobname{#1}
        \input{#1}
        \endinput
      }
    \fi
    \expandafter
  \endgroup
  \childdoctmp
}
%    \end{macrocode}

% \macro{\childdocforwardprefix}
% The command |\childdocforwardprefix| redirects
% compilation to the main or a child file by means of a pattern.
% The prefix |#1| in the current filename is replaced by |#2|
% and the suffix of the current filename is kept
% (it is assumed that the filename does not contain the substring `|~~~|'
% which is used as a delimiter).
% Compilation is handed over to the new file by |\childdocforward|:
%    \begin{macrocode}
\newcommand{\childdocforwardprefix}[3][]
{
  \begingroup
    \def\childdocextract #2##1~~~{\def\childdoctmp{\childdocforward[#1]{#3##1}}}
    \expandafter\childdocextract\childdocname~~~
    \expandafter
  \endgroup
  \childdoctmp
}
%    \end{macrocode}

% \macro{\childdoc}
% The deprecated macro |\childdoc| is a legacy version of |\childdocmain|:
%    \begin{macrocode}
\newcommand{\childdoc}{\childdocmain}
%    \end{macrocode}

% \macro{\childdocredirect}
% The deprecated macro |\childdocredirect| is a legacy version
% of |\childdocforward| and |\childdocforwardprefix|:
%    \begin{macrocode}
\newcommand{\childdocredirect}[2][]
{
  \begingroup
    \if?#1?
      \def\childdoctmp{\childdocforward{#2}}
    \else
      \def\childdoctmp{\childdocforwardprefix{#1}{#2}}
    \fi
    \expandafter
  \endgroup
  \childdoctmp
}
%    \end{macrocode}

%\iffalse
%</package>
%\fi
%
\endinput
|\\
|\childdocforward{|\textit{main}|}|
\end{tabular}
\end{center}
%
Likewise, the following files |final|\textit{nn}|.tex|
compile the final version of the child document
|child|\textit{nn}|.tex|:
%
\begin{center}
\begin{tabular}{l}
|\def\version{final}|\\
|% \iffalse
%
% childdoc.dtx Copyright (C) 2017-2018 Niklas Beisert
%
% This work may be distributed and/or modified under the
% conditions of the LaTeX Project Public License, either version 1.3
% of this license or (at your option) any later version.
% The latest version of this license is in
%   http://www.latex-project.org/lppl.txt
% and version 1.3 or later is part of all distributions of LaTeX
% version 2005/12/01 or later.
%
% This work has the LPPL maintenance status `maintained'.
%
% The Current Maintainer of this work is Niklas Beisert.
%
% This work consists of the files childdoc.dtx and childdoc.ins
% and the derived files childdoc.def and cdocsamp.tex with
% cdocsch1.tex, cdocsch2.tex, cdocsdrf.tex, cdocsfn1.tex, cdocsfn2.tex.
%
%<package>\ifdefined\childdocmain\endinput\fi
%<package>\ProvidesFile{childdoc.def}[2018/12/30 v2.0 child document driver]
%<samplemain>\ProvidesFile{cdocsamp.tex}[2018/12/30 v2.0 sample for childdoc]
%<*driver>
%\ProvidesFile{childdoc.drv}[2018/12/30 v2.0 childdoc reference manual file]
\PassOptionsToClass{10pt,a4paper}{article}
\documentclass{ltxdoc}

\usepackage[margin=35mm]{geometry}
\usepackage{hyperref}
\usepackage{hyperxmp}
\usepackage[usenames]{color}

\hypersetup{colorlinks=true}
\hypersetup{pdfstartview=FitH}
\hypersetup{pdfpagemode=UseNone}
\hypersetup{pdfsource={}}
\hypersetup{pdflang={en-UK}}
\hypersetup{pdfcopyright={Copyright 2017-2018 Niklas Beisert.
  This work may be distributed and/or modified under the
  conditions of the LaTeX Project Public License, either version 1.3
  of this license or (at your option) any later version.}}
\hypersetup{pdflicenseurl={http://www.latex-project.org/lppl.txt}}
\hypersetup{pdfcontactaddress={ETH Zurich, ITP, HIT K,
  Wolfgang-Pauli-Strasse 27}}
\hypersetup{pdfcontactpostcode={8093}}
\hypersetup{pdfcontactcity={Zurich}}
\hypersetup{pdfcontactcountry={Switzerland}}
\hypersetup{pdfcontactemail={nbeisert@itp.phys.ethz.ch}}
\hypersetup{pdfcontacturl={http://people.phys.ethz.ch/\xmptilde nbeisert/}}

\newcommand{\secref}[1]{\hyperref[#1]{section \ref*{#1}}}

\parskip1ex
\parindent0pt
\let\olditemize\itemize
\def\itemize{\olditemize\parskip0pt}

\begin{document}

\title{The \textsf{childdoc} Package}
\hypersetup{pdftitle={The childdoc Package}}
\author{Niklas Beisert\\[2ex]
  Institut f\"ur Theoretische Physik\\
  Eidgen\"ossische Technische Hochschule Z\"urich\\
  Wolfgang-Pauli-Strasse 27, 8093 Z\"urich, Switzerland\\[1ex]
  \href{mailto:nbeisert@itp.phys.ethz.ch}
  {\texttt{nbeisert@itp.phys.ethz.ch}}}
\hypersetup{pdfauthor={Niklas Beisert}}
\hypersetup{pdfsubject={Manual for the LaTeX2e Package childdoc}}
\date{30 December 2018, \textsf{v2.0}}
\maketitle

\begin{abstract}\noindent
\textsf{childdoc} is a \LaTeXe{} package
that enables the direct compilation
of document sections included by |\include|
to individual files.
\end{abstract}

\begingroup
\parskip0ex
\tableofcontents
\endgroup

%%%%%%%%%%%%%%%%%%%%%%%%%%%%%%%%%%%%%%%%%%%%%%%%%%%%%%%%%%%%%%%%%%%%%%%%%%%%%%%%
%%%%%%%%%%%%%%%%%%%%%%%%%%%%%%%%%%%%%%%%%%%%%%%%%%%%%%%%%%%%%%%%%%%%%%%%%%%%%%%%
\section{Introduction}

\LaTeX{} provides a mechanism to structure a large document (such as a book)
into a main file and several child files (containing the chapters)
using the |\include| command.
This mechanism is beneficial for documents
which span hundreds of pages in order to
make the source file(s) more manageable.
Moreover, compilation can be restricted to
selected child files by means of the |\includeonly| command.
The latter feature can be used to reduce the compilation time while editing
(this was significantly more useful in the earlier days of \LaTeX{})
or to generate a smaller document which is easier to navigate.
Another application of |\includeonly| is to generate
documents consisting of selected parts of the complete document.

However, there are a few drawbacks of the plain |\include| mechanism:
\begin{itemize}
\item
The child files cannot be compiled on their own,
they can only be compiled via the main file.
A naive editing environment
(such as a text editor with an option
to have the current file processed by \LaTeX)
may require one to switch to the main file before compiling;
attempting to compile the child file produces errors.
\item
The main file must be modified (each time)
to adjust the |\includeonly| command
to the present needs. This easily leaves the main file in a messy state.
\item
The generated document will always carry the filename
of the main document. This is inconvenient if
several child files are to be compiled and
to be kept for distribution.
\end{itemize}

The present package provides a simple interface
to make child files individually compilable by \LaTeX{}.
Compiling a child file then has the same effect as compiling
the main file with an |\includeonly| command
to select the appropriate child.
Moreover the generated document will carry the name of the child
rather than the main file.
This resolves all three above issues.

This feature is meant to make the editing of books,
thesis documents and lecture notes somewhat more convenient.
However, the package can also be used efficiently for
composing a series of documents (such as exercise sheets)
which are typically distributed individually.
It then assists the author in generating the individual documents
(potentially in different versions)
as well as a document containing the collected series.
Another application is in developing style files
or other kinds of included material
where compilation of the style file could redirect
to a sample or test file.

%%%%%%%%%%%%%%%%%%%%%%%%%%%%%%%%%%%%%%%%%%%%%%%%%%%%%%%%%%%%%%%%%%%%%%%%%%%%%%%%
%%%%%%%%%%%%%%%%%%%%%%%%%%%%%%%%%%%%%%%%%%%%%%%%%%%%%%%%%%%%%%%%%%%%%%%%%%%%%%%%
\section{Usage}

First of all, the package \textsf{childdoc} is \emph{not} a standard
\LaTeXe{} |.sty| style file! Therefore it needs to be invoked in
a non-standard way.

%%%%%%%%%%%%%%%%%%%%%%%%%%%%%%%%%%%%%%%%%%%%%%%%%%%%%%%%%%%%%%%%%%%%%%%%%%%%%%%%
\subsection{Included Files}
\label{sec:include}

%%%%%%%%%%%%%%%%%%%%%%%%%%%%%%%%%%%%%%%%
\DescribeMacro{\childdocmain}
To use the package, add the commands
\begin{center}
\begin{tabular}{l}
|\input{childdoc.def}|\\
|\childdocmain{}|\\
\end{tabular}
\end{center}
at the very top of the main \LaTeX{} file,
in particular \emph{before} the |\documentclass| statement!
The argument of |\childdocmain| should be left empty
(but it must be present).

%%%%%%%%%%%%%%%%%%%%%%%%%%%%%%%%%%%%%%%%
\DescribeMacro{\childdocof}
Furthermore, add the commands
\begin{center}
\begin{tabular}{l}
|\input{childdoc.def}|\\
|\childdocof{|\textit{main}|}|\\
\end{tabular}
\end{center}
at the top of every child file \textit{child}
which is included by |\include{|\textit{child}|}|
from within the main file
(or at least for those files to be compiled individually).
The argument \textit{main} must be the filename of the main file.

There are a couple of
considerations in setting up the main and child documents:

%%%%%%%%%%%%%%%%%%%%%%%%%%%%%%%%%%%%%%%%
\paragraph{Restrictions.}

Please note the following restrictions:
\begin{itemize}
\item
|\childdocmain| must be called with one argument \textit{main}
to ensure compatibility with earlier version of the package.
It must either be empty (|\childdocmain{}|)
or precisely match the filename of the main file in which it is specified.
See \secref{sec:detection} for further information.
\item
The filename \textit{main} must be specified without the |.tex| extension.
\item
The filename \textit{main} is case sensitive
(even in case-insensitive file systems)
due to internal string comparison.
\item
The argument \textit{main} should be fully expanded, it cannot be a macro.
\item
Subdirectories and special characters should be avoided in filenames.
\item
The command |\childdocmain{|\textit{main}|}| must be followed by a whitespace.
It should not be followed immediately by another command
or by a comment mark `|%|'.
This is because the \TeX{} parser reads the token immediately following
the argument of |\childdocmain| and puts it
at the beginning of every child section;
however, a white\-space is ignored.
\end{itemize}

%%%%%%%%%%%%%%%%%%%%%%%%%%%%%%%%%%%%%%%%
\paragraph{Content of Main File.}

It is advisable to place all content in the child files included by |\include|.
Any output contained in the main file will appear in all child documents
unless suppressed manually;
it cannot be suppressed automatically by the |\includeonly| directive
and thus should normally be avoided.
A method to include some content in the main file
by means of conditional processing is described in \secref{sec:conditional}.

%%%%%%%%%%%%%%%%%%%%%%%%%%%%%%%%%%%%%%%%
\paragraph{Page Numbering.}

When only a part of the document is compiled,
the appropriate numbering of pages
(as well as other status parameters)
is determined from the |.aux| files.
The latter contain information from previous passes.
However this information needs to propagate through
all intermediate child documents.
Therefore the page numbering in child documents may well
be inconsistent until the complete document is compiled at least once.

A useful (if unconventional) way to always ensure a consistent
page numbering is to restart the numbering in each child document
and denote the pages by `\textit{child}|.|\textit{page}'
where \textit{child} represents the chapter/section number of the child file.
This can be achieved by the command
|\numberwithin{page}{|\textit{child}|}|
of the \textsf{amsmath} package
where \textit{child} can be |chapter| or |section|
depending on the chosen structuring.
Alternatively, one can modify the macro |\thepage| appropriately
and reset the counter |page| at the start of each child file.

%%%%%%%%%%%%%%%%%%%%%%%%%%%%%%%%%%%%%%%%%%%%%%%%%%%%%%%%%%%%%%%%%%%%%%%%%%%%%%%%
\subsection{Conditional Processing}
\label{sec:conditional}

The package provides a mechanism to compile different versions
of a document. To customise the versions further some conditional processing
can come in handy to distinguish which version is being compiled.
The package provides two macros to describe the compilation context:

%%%%%%%%%%%%%%%%%%%%%%%%%%%%%%%%%%%%%%%%
\DescribeMacro{\ifchilddoc}
The conditional |\ifchilddoc| distinguishes between the compilation of
child documents and the main document:
%
\begin{center}
|\ifchilddoc |\textit{child-code}| |[|\||else |\textit{main-code}]| \||fi|
\end{center}

%%%%%%%%%%%%%%%%%%%%%%%%%%%%%%%%%%%%%%%%
\DescribeMacro{\childdocname}
\DescribeMacro{\childdocjob}
The macro |\childdocname| contains the filename (without extension)
of the main or child file being processed.
Note that |\childdocjob| will always contain the name of the main file.

%%%%%%%%%%%%%%%%%%%%%%%%%%%%%%%%%%%%%%%%
\paragraph{Title Page.}

Conditional processing can be used to include a title or banner page
in the main document when proper precautions are taken.
Importantly, the code in the main file should ensure that the page counter
(as well as other status parameters which are stored in the |.aux| files)
takes the same value after the conditional processing.
Otherwise the page numbers may take divergent values
depending on which part is compiled.

For example, a title page could be declared by:
%
\begin{center}
\begin{tabular}{l}
|\ifchilddoc\||else|\\
|\addtocounter{page}{-1}|\\
\textit{code for title page}\\
|\newpage|\\
|\||fi|
\end{tabular}
\end{center}
%
A banner page for the child documents can be generated by:
%
\begin{center}
\begin{tabular}{l}
|\ifchilddoc|\\
|\addtocounter{page}{-1}|\\
\textit{code for banner page}\\
|\newpage|\\
|\||fi|
\end{tabular}
\end{center}
%
Here one could write a message such as:
\begin{center}
|This is the part \childdocname{} of \childdocjob{}.|
\end{center}

%%%%%%%%%%%%%%%%%%%%%%%%%%%%%%%%%%%%%%%%%%%%%%%%%%%%%%%%%%%%%%%%%%%%%%%%%%%%%%%%
\subsection{Flags}
\label{sec:flags}

The package makes it easy to generate different versions
of the main or child documents.
To this end compilation flags can be defined
and assigned different default values.
They will be particularly useful in conjunction
with the forwarding mechanism described in \secref{sec:forward}.

For example, it may be useful to have a flag |\version|
which can be set to |draft| or |final|.
The document source will contain some conditional code
depending on the value of |\version|.
Suppose further, the flag should default to |final| for the main file
and to |draft| for child files
which is a natural assignment for editing the document.
This is achieved by placing the following code
in the preamble of the main document
(below the |\childdocmain| directive):
%
\begin{center}
\begin{tabular}{l}
|\ifchilddoc|\\
|\providecommand{\version}{draft}|\\
|\||else|\\
|\providecommand{\version}{final}|\\
|\||fi|
\end{tabular}
\end{center}
%
The definition by |\providecommand| makes sure
that previous definitions are not overwritten.
Further statements |\providecommand{\version}{...}|
can thus be added before the above code to override it.

For the main file, one might add a line
(between |\childdocmain| and the above block)
%
\begin{center}
|%\ifchilddoc\||else\providecommand{\version}{draft}\||fi|
\end{center}
%
which can be uncommented to produce a draft version.
Likewise one can add a line to the very top of a child file
(above the |\childdocof{|\textit{main}|}| directive)
%
\begin{center}
|%\providecommand{\version}{final}|
\end{center}
%
which can be uncommented to produce the final version of this child document.

%%%%%%%%%%%%%%%%%%%%%%%%%%%%%%%%%%%%%%%%%%%%%%%%%%%%%%%%%%%%%%%%%%%%%%%%%%%%%%%%
\subsection{Forwarding}
\label{sec:forward}

Different versions of the main or child documents
using compilation flags as described in \secref{sec:flags}
can be (permanently) stored in different files
for convenient compilation, viewing and distribution.
To this end, the package defines a command
to pass on compilation to a different file:

%%%%%%%%%%%%%%%%%%%%%%%%%%%%%%%%%%%%%%%%
\DescribeMacro{\childdocforward}
The command |\childdocforward| redirects processing to
another source file:
%
\begin{center}
\begin{tabular}{l}
|\input{childdoc.def}|\\
|\childdocforward[|\textit{main}|]{|\textit{dest}|}|\\
\end{tabular}
\end{center}
%
The argument \textit{dest} is the destination file
(without extension).
It should be the main file or one of the child files.
Note that further \textsf{childdoc} directives
such as |\childdocof| and |\childdocforward|
in the indicated file will be processed in this form.
The optional argument \textit{main}
passes on directly to the main file \textit{main}
while pretending to compile the child \textit{dest}.
This form behaves as if \textit{dest}
issues |\childdocof{|\textit{main}|}| right away,
and no further \textsf{childdoc} directives will be processed.

%%%%%%%%%%%%%%%%%%%%%%%%%%%%%%%%%%%%%%%%
\DescribeMacro{\...prefix}
In the alternative form |\childdocforwardprefix|,
%
\begin{center}
\begin{tabular}{l}
|\input{childdoc.def}|\\
|\childdocforwardprefix[|\textit{main}|]{|\textit{prefix}|}{|\textit{dest}|}|
\end{tabular}
\end{center}
%
the destination file is determined by a pattern
depending on the current file:
To make this work, the current file must be called
`{\textit{prefix}\hspace{0.2em}\textit{suffix}}'
with \textit{prefix} matching precisely the argument.
Processing is then passed on to the file
`{\textit{dest}\hspace{0.2em}\textit{suffix}}'.
Surely, the same effect is achieved by
directly specifying the
argument `{\textit{dest}\hspace{0.2em}\textit{suffix}}'
in the first form.
However, that requires to set up a different file
for each child. With the alternative form of the command
all these files can have exactly the same content
which simplifies setting them up and maintaining them.

For example, the following file |draft.tex|
with a compilation flag |\version| as described in \secref{sec:flags}
compiles the main document as a draft:
%
\begin{center}
\begin{tabular}{l}
|\def\version{draft}|\\
|\input{childdoc.def}|\\
|\childdocforward{|\textit{main}|}|
\end{tabular}
\end{center}
%
Likewise, the following files |final|\textit{nn}|.tex|
compile the final version of the child document
|child|\textit{nn}|.tex|:
%
\begin{center}
\begin{tabular}{l}
|\def\version{final}|\\
|\input{childdoc.def}|\\
|\childdocforwardprefix{final}{child}|
\end{tabular}
\end{center}
%

Note that when several versions of a main file and/or of each child file
are to be generated, it may be convenient to set up a |Makefile| or
shell script to automatise the process.

%%%%%%%%%%%%%%%%%%%%%%%%%%%%%%%%%%%%%%%%%%%%%%%%%%%%%%%%%%%%%%%%%%%%%%%%%%%%%%%%
\subsection{Command Line Processing}
\label{sec:commandline}

The effect of redirection files can also be achieved by invoking
the \LaTeX{} compiler with a more elaborate command line.
Most conveniently this should be done as part
of a shell script or a |Makefile|.

When using \textsf{childdoc} in the main file, the following
command lines effectively perform a redirection
(note that depending on the shell being used,
backslashes may have to be doubled: `|\|' $\to$ `|\\|'):
%
\begin{center}
|... -jobname "|\textit{target}|" |\\|"|[\textit{flags}]%
|\input{childdoc.def}\childdocforward[|\textit{main}|]{|\textit{dest}|}"|
\end{center}
%
Here \textit{target} is the name of the output file,
\textit{main} is the name of the main file
and \textit{dest} is the name of the main or child file to be processed
(all filenames without extensions).
The optional argument \textit{main} can be omitted
if \textit{main} matches \textit{dest}.
Optionally, compilation \textit{flags} can be defined via |\def| commands.
This command line makes the \TeX{} engine believe
it is compiling the file \textit{target}
whose content is specified as the latter parameter.
The provided code then forwards the processing to
\textit{main} or \textit{dest} as described in \secref{sec:forward}.

%%%%%%%%%%%%%%%%%%%%%%%%%%%%%%%%%%%%%%%%%%%%%%%%%%%%%%%%%%%%%%%%%%%%%%%%%%%%%%%%
\subsection{Include by Input}
\label{sec:input}

Including child documents by |\include| has some restrictions by design.
Most notably, the content of a child document always occupies
its own set of pages; pages cannot be shared between child documents.
Usually, this behaviour makes perfect sense
because each child document contain an essential part of the document.
However, in some situations it may be desirable to compose
a document from a collection of parts
without having mandatory page breaks between then.
For this case, the package
provides a mechanism to include parts
by |\input| which can also be processed individually.
However, by construction this mechanism
requires manual handling of the content to be output.

%%%%%%%%%%%%%%%%%%%%%%%%%%%%%%%%%%%%%%%%
\DescribeMacro{\ifchilddocmanual}
The main file should be prepared as usual, see \secref{sec:include}.
However, the document body must make a distinction
between processing of an individual part and of the main document, e.g.:
%
\begin{center}
\begin{tabular}{l}
|\ifchilddocmanual|\\
|\input{\childdocname}|\\
|\||else|\\
\textit{document body with }|\input{|\textit{part}|}|\\
|\||fi|
\end{tabular}
\end{center}
%
The conditional |\ifchilddocmanual| is true whenever
a part to be included by |\input| is being compiled,
and the name of the part is stored in |\childdocname|.

%%%%%%%%%%%%%%%%%%%%%%%%%%%%%%%%%%%%%%%%
\DescribeMacro{\childdocby}
Each part to be included by |\input| should start with:
%
\begin{center}
\begin{tabular}{l}
|\input{childdoc.def}|\\
|\childdocby{|\textit{main}|}|\\
\end{tabular}
\end{center}
%
The directive |\childdocby| is similar to |\childdocof|
described in \secref{sec:include},
but the subsequent selection of content must be done manually.
To that end, both |\ifchilddoc| and |\ifchilddocmanual|
will be true upon processing of a part,
and the name of the part is stored in |\childdocname|.
Note that |\jobname| will be set to the filename of the current part
so that each part receives an individual |.aux| file
that does not interfere with the |.aux| file(s) of the main document.
This behaviour can be altered by the alternative form
|\childdocby[*]{|\textit{main}|}| (with a non-empty optional argument)
which uses the |.aux| file of the main document
by setting |\jobname| to \textit{main}.

%%%%%%%%%%%%%%%%%%%%%%%%%%%%%%%%%%%%%%%%%%%%%%%%%%%%%%%%%%%%%%%%%%%%%%%%%%%%%%%%
\subsection{Driver Development}
\label{sec:driver}

The \textsf{childdoc} mechanism can also be use for the development
of definition files such as \LaTeX{} styles or classes.
This case differs from the above setup with multiple parts
included by |\include| in that no |\includeonly| should be invoked.
This can be achieved by starting the include file
(before |\ProvidesPackage|) with:
%
\begin{center}
\begin{tabular}{l}
|\input{childdoc.def}|\\
|\childdocforward{|\textit{main}|}|\\
\end{tabular}
\end{center}
%
or alternatively with:
%
\begin{center}
\begin{tabular}{l}
|\input{childdoc.def}|\\
|\childdocby{|\textit{main}|}|\\
\end{tabular}
\end{center}
%
Both forms have slightly different effects as described above.
The main file is prepared as usual, see \secref{sec:include}.

%%%%%%%%%%%%%%%%%%%%%%%%%%%%%%%%%%%%%%%%%%%%%%%%%%%%%%%%%%%%%%%%%%%%%%%%%%%%%%%%
\subsection{Legacy Detection}
\label{sec:detection}

The directive |\childdocmain| in the main file can detect
whether the complete document or merely a child is to be compiled
even without using the directive |\childdocof|.
This method is deprecated because it is less robust
and there is no compelling reason to use it;
it is merely provided for backward compatibility
and it may be removed in future versions.

If the detection mechanism is to be used,
it is mandatory to correctly specify
the filename of the main file as the argument of |\childdocmain|:
%
\begin{center}
\begin{tabular}{l}
|\input{childdoc.def}|\\
|\childdocmain{|\textit{main}|}|\\
\end{tabular}
\end{center}
%
If |\jobname| does not match the argument \textit{main} of |\childdocmain|,
it is assumed that |\jobname| points to the child file to be compiled.
When using |\childdocmain| with the main file specified as argument,
it suffices to start a child file
with just |\input{|\textit{main}|}|
without loading of the package and using |\childdocof|.
If instead all processing is done
with the appropriate \textsf{childdoc} directives,
the argument of \textit{main} of |\childdocmain| can be empty.

An alternative version of the command line processing described
in \secref{sec:commandline} using the detection mechanism reads:
%
\begin{center}
|... -jobname "|\textit{target}|" "|[\textit{flags}]%
[|\def\jobname{|\textit{dest}|}|]|\input{|\textit{main}|}"|
\end{center}

%%%%%%%%%%%%%%%%%%%%%%%%%%%%%%%%%%%%%%%%%%%%%%%%%%%%%%%%%%%%%%%%%%%%%%%%%%%%%%%%
\subsection{Manual Code}
\label{sec:manual}

In case one cannot be certain whether the definitions file |childdoc.def|
is installed on the target \TeX{} distribution
and one prefers not to ship it,
it is conceivable to paste a few relevant commands into the sources.

To that end, drop all statements |\input{childdoc.def}|
and perform the replacements as outlined below.
Instead of |\childdocmain{|\textit{main}|}| add the following code
to the top of the main file:
%
\begin{center}
\begin{tabular}{l}
|\||ifdefined\childdocname\endinput\||fi\newif\ifchilddoc|\\
|\edef\childdocname{\scantokens\expandafter{\jobname\noexpand}}|\\
|\def\childdocmain{|\textit{main}|}\||ifx\childdocmain\childdocname\||else|\\
|\childdoctrue\includeonly{\childdocname}\let\jobname\childdocmain\||fi|\\
\end{tabular}
\end{center}
%
Instead of |\childdocof{|\textit{main}|}| just include the main file
at the top of each child file:
%
\begin{center}
|\input{|\textit{main}|}|
\end{center}
%
A simple redirection |\childdocforward{|\textit{dest}|}| is achieved by:
%
\begin{center}
|\def\jobname{|\textit{dest}|}\input{\jobname}|
\end{center}
%
The redirection with prefix
|\childdocforwardprefix[|\textit{prefix}|]{|\textit{dest}|}|
is accomplished by:
%
\begin{center}
\begin{tabular}{l}
|{\edef\jobname{\scantokens\expandafter{\jobname\noexpand}}|\\
|\def\redirectjob |\textit{prefix}|#1~~~{\gdef\jobname{|\textit{dest}|#1}}|\\
|\expandafter\redirectjob\jobname~~~}\input{\jobname}|
\end{tabular}
\end{center}

In an alternative approach,
child documents can be compiled by a specific command line
without additional code or specific definitions:
%
\begin{center}
|... -jobname "|\textit{target}|" "|[\textit{flags}]%
|\includeonly{|\textit{dest}|}\input{|\textit{main}|}"|
\end{center}
%

%%%%%%%%%%%%%%%%%%%%%%%%%%%%%%%%%%%%%%%%%%%%%%%%%%%%%%%%%%%%%%%%%%%%%%%%%%%%%%%%
%%%%%%%%%%%%%%%%%%%%%%%%%%%%%%%%%%%%%%%%%%%%%%%%%%%%%%%%%%%%%%%%%%%%%%%%%%%%%%%%
\section{Information}

%%%%%%%%%%%%%%%%%%%%%%%%%%%%%%%%%%%%%%%%%%%%%%%%%%%%%%%%%%%%%%%%%%%%%%%%%%%%%%%%
\subsection{Copyright}

Copyright \copyright{} 2017--2018 Niklas Beisert

This work may be distributed and/or modified under the
conditions of the \LaTeX{} Project Public License, either version 1.3
of this license or (at your option) any later version.
The latest version of this license is in
  \url{http://www.latex-project.org/lppl.txt}
and version 1.3 or later is part of all distributions of \LaTeX{}
version 2005/12/01 or later.

This work has the LPPL maintenance status `maintained'.

The Current Maintainer of this work is Niklas Beisert.

This work consists of the files |README.txt|, |childdoc.ins| and |childdoc.dtx|
as well as the derived files |childdoc.def|, |cdocsamp.tex|
with |cdocsch1.tex|, |cdocsch2.tex|, |cdocspt3.tex|, |cdocspt4.tex|,
|cdocsdrf.tex|, |cdocsfn1.tex|, |cdocsfn2.tex|
as well as |childdoc.pdf|.

%%%%%%%%%%%%%%%%%%%%%%%%%%%%%%%%%%%%%%%%%%%%%%%%%%%%%%%%%%%%%%%%%%%%%%%%%%%%%%%%
\subsection{Files and Installation}

The package consists of the files:
%
\begin{center}
\begin{tabular}{ll}
    |README.txt|   & readme file \\
    |childdoc.ins| & installation file \\
    |childdoc.dtx| & source file \\
    |childdoc.def| & definition file \\
    |cdocsamp.tex| & sample main file \\
    |cdocsch1.tex| & sample include file \\
    |cdocsch2.tex| & sample include file \\
    |cdocspt3.tex| & sample part file \\
    |cdocspt4.tex| & sample part file \\
    |cdocsdrf.tex| & sample redirection file \\
    |cdocsfn1.tex| & sample redirection file \\
    |cdocsfn2.tex| & sample redirection file \\
    |childdoc.pdf| & manual
\end{tabular}
\end{center}
%
The distribution consists of the files
|README.txt|, |childdoc.ins| and |childdoc.dtx|.
%
\begin{itemize}
\item
Run (pdf)\LaTeX{} on |childdoc.dtx|
to compile the manual |childdoc.pdf| (this file).
\item
Run \LaTeX{} on |childdoc.ins| to create the definitions file |childdoc.def|
and the sample |cdocsamp.tex| with include files
|cdocsch1.tex|, |cdocsch2.tex|, |cdocspt3.tex|, |cdocspt4.tex|,
|cdocsdrf.tex|, |cdocsfn1.tex|, |cdocsfn2.tex|.
Then copy the file |childdoc.def| to an appropriate directory of your \LaTeX{}
distribution, e.g.\ \textit{texmf-root}|/tex/latex/childdoc|.
\end{itemize}

%%%%%%%%%%%%%%%%%%%%%%%%%%%%%%%%%%%%%%%%%%%%%%%%%%%%%%%%%%%%%%%%%%%%%%%%%%%%%%%%
\subsection{Related CTAN Packages}

There are several other packages which offer a similar functionality:
%
\begin{itemize}
\item
The packages
\href{http://ctan.org/pkg/docmute}{\textsf{docmute}},
\href{http://ctan.org/pkg/includex}{\textsf{includex}} and
\href{http://ctan.org/pkg/standalone}{\textsf{standalone}}
provide commands to include only the document body of
a child file thus allowing both files to be compiled individually.
\item
The packages \href{http://ctan.org/pkg/subdocs}{\textsf{subdocs}}
and \href{http://ctan.org/pkg/subfiles}{\textsf{subfiles}}
provide structures in which the main and child documents can be
encapsulated and allowing them to be compiled individually.
The inclusion mechanism is different from the conventional |\include|.
\item
The package \href{http://ctan.org/pkg/combine}{\textsf{combine}}
is an elaborate solution to combine several documents into one.
\end{itemize}
%
See also the CTAN topic \href{http://ctan.org/topic/subdocs}{\textsf{subdocs}}
for further related packages.
The present package differs from the above solutions in that
a document structure constructed with the conventional |\include| mechanism
just needs two extra commands at the top of every file
such that all constituent files can be compiled individually.

%%%%%%%%%%%%%%%%%%%%%%%%%%%%%%%%%%%%%%%%%%%%%%%%%%%%%%%%%%%%%%%%%%%%%%%%%%%%%%%%
%\subsection{Feature Suggestions}
%
%The following is a list of features which may be useful for future
%versions of this package:
%%
%\begin{itemize}
%\item
%\ldots
%\end{itemize}

%%%%%%%%%%%%%%%%%%%%%%%%%%%%%%%%%%%%%%%%%%%%%%%%%%%%%%%%%%%%%%%%%%%%%%%%%%%%%%%%
\subsection{Revision History}

%%%%%%%%%%%%%%%%%%%%%%%%%%%%%%%%%%%%%%%%
\paragraph{v2.0:} 2018/12/30

\begin{itemize}
\item
immediate forward processing
\item
added |\childdocby| mechanism
\item
manual restructured
\end{itemize}

%%%%%%%%%%%%%%%%%%%%%%%%%%%%%%%%%%%%%%%%
\paragraph{v1.6:} 2018/01/17

\begin{itemize}
\item
application for development of include files
\item
corrections to manual
\end{itemize}

%%%%%%%%%%%%%%%%%%%%%%%%%%%%%%%%%%%%%%%%
\paragraph{v1.5:} 2017/05/21

\begin{itemize}
\item
more complete structuring introduced
\item
|\childdocof| introduced
\item
|\childdoc| renamed to |\childdocmain|
\item
|\childredirect| renamed to |\childdocforward| and |\childdocforwardprefix|
and functionality expanded
\end{itemize}

%%%%%%%%%%%%%%%%%%%%%%%%%%%%%%%%%%%%%%%%
\paragraph{v1.0:} 2017/04/27

\begin{itemize}
\item
manual and install package
\item
first version published on CTAN
\end{itemize}

%%%%%%%%%%%%%%%%%%%%%%%%%%%%%%%%%%%%%%%%
\paragraph{v0.6:} 2017/04/26

\begin{itemize}
\item
redirection mechanism added
\end{itemize}

%%%%%%%%%%%%%%%%%%%%%%%%%%%%%%%%%%%%%%%%
\paragraph{v0.5:} 2017/04/26

\begin{itemize}
\item
functionality in definition file
\end{itemize}


%%%%%%%%%%%%%%%%%%%%%%%%%%%%%%%%%%%%%%%%%%%%%%%%%%%%%%%%%%%%%%%%%%%%%%%%%%%%%%%%
%%%%%%%%%%%%%%%%%%%%%%%%%%%%%%%%%%%%%%%%%%%%%%%%%%%%%%%%%%%%%%%%%%%%%%%%%%%%%%%%
%%%%%%%%%%%%%%%%%%%%%%%%%%%%%%%%%%%%%%%%%%%%%%%%%%%%%%%%%%%%%%%%%%%%%%%%%%%%%%%%
\appendix

\settowidth\MacroIndent{\rmfamily\scriptsize 000\ }

 \DocInput{childdoc.dtx}

\end{document}
%</driver>
% \fi
%
% %%%%%%%%%%%%%%%%%%%%%%%%%%%%%%%%%%%%%%%%%%%%%%%%%%%%%%%%%%%%%%%%%%%%%%%%%%%%%%
% %%%%%%%%%%%%%%%%%%%%%%%%%%%%%%%%%%%%%%%%%%%%%%%%%%%%%%%%%%%%%%%%%%%%%%%%%%%%%%
% \section{Sample}
%\iffalse
%<*samplemain>
%\fi
%
% The following presents a sample document
% with two chapters, two parts, a title page,
% a compile flag as well as three forwarding files to set the flag.
% It consists of eight |.tex| files:
% \begin{center}
% \begin{tabular}{ll}
% |cdocsamp.tex|&main file\\
% |cdocsch1.tex|&include file for chapter 1\\
% |cdocsch2.tex|&include file for chapter 2\\
% |cdocspt3.tex|&include file for part 3\\
% |cdocspt4.tex|&include file for part 4\\
% |cdocsdrf.tex|&forwarding file for main file in draft mode\\
% |cdocsfi1.tex|&forwarding file for final version of chapter 1\\
% |cdocsfi2.tex|&forwarding file for final version of chapter 2\\
% \end{tabular}
% \end{center}
% Each of the eight files can be compiled directly by the \LaTeX{} compiler.
%
% %%%%%%%%%%%%%%%%%%%%%%%%%%%%%%%%%%%%%%
% \paragraph{Main File.}
%
% The main file is called |cdocsamp.tex|.
%
% Load the \textsf{childdoc} definitions and
% declare the filename for the main document:
%    \begin{macrocode}
\input{childdoc.def}
\childdocmain{}
%    \end{macrocode}

% Optional override for |\version| flag:
%    \begin{macrocode}
%%\ifchilddoc\else\providecommand{\version}{draft}\fi
%    \end{macrocode}

% Define the default values for the |\version| flag
% (|final| for the main file and |draft| for childs):
%    \begin{macrocode}
\ifchilddoc
\providecommand{\version}{draft}
\else
\providecommand{\version}{final}
\fi
%    \end{macrocode}

% Load the standard document class:
%    \begin{macrocode}
\documentclass[12pt]{article}
%    \end{macrocode}

% Start the document body:
%    \begin{macrocode}
\begin{document}
%    \end{macrocode}

% Declare a title page.
% Print title, part of document being processed and version flag:
%    \begin{macrocode}
\addtocounter{page}{-1}
\begin{center}
{\LARGE\bfseries{}childdoc example\par}
\vspace{1cm}
\ifchilddoc
\ifchilddocmanual part\else chapter\fi:
`\childdocname' of `\childdocjob'\par
\else
main document: `\childdocjob'\par
\fi
version: \version\par
\end{center}
\newpage
%    \end{macrocode}

% Manually include selected file,
% otherwise process as usual:
%    \begin{macrocode}
\ifchilddocmanual
\section*{part `\childdocname'}
\input{\childdocname}
\else
%    \end{macrocode}

% Include the two chapters:
%    \begin{macrocode}
\include{cdocsch1}
\include{cdocsch2}
%    \end{macrocode}

% Include the two parts unless only chapters should be displayed:
%    \begin{macrocode}
\ifchilddoc\else
\section{part three}
\input{cdocspt3}
\section{part four}
\input{cdocspt4}
\fi
%    \end{macrocode}

% Process as usual until here:
%    \begin{macrocode}
\fi
%    \end{macrocode}

% End of document body:
%    \begin{macrocode}
\end{document}
%    \end{macrocode}
%\iffalse
%</samplemain>
%\fi
%
% %%%%%%%%%%%%%%%%%%%%%%%%%%%%%%%%%%%%%%
% \paragraph{Chapter Include Files.}
%
% The include files are called |cdocsch1.tex| and |cdocsch2.tex|.
%
%\iffalse
%<*samplechap1|samplechap2>
%\fi

% Optional override for |\version| flag:
%    \begin{macrocode}
%%\providecommand{\version}{final}
%    \end{macrocode}

% Include the main document:
%    \begin{macrocode}
\input{childdoc.def}
\childdocof{cdocsamp}
%    \end{macrocode}

%\iffalse
%</samplechap1|samplechap2>
%\fi
%
%\iffalse
%<*samplechap1>
%\fi
% Some text for chapter 1:
%    \begin{macrocode}
\section{one}
some text in chapter one
%    \end{macrocode}

%\iffalse
%</samplechap1>
%\fi
% Some text for chapter 2:
%\iffalse
%<*samplechap2>
%\fi
%    \begin{macrocode}
\section{two}
more text in chapter two
%    \end{macrocode}

%\iffalse
%</samplechap2>
%\fi
%
% %%%%%%%%%%%%%%%%%%%%%%%%%%%%%%%%%%%%%%
% \paragraph{Part Include Files.}
%
% The include files are called |cdocspt3.tex| and |cdocspt4.tex|.
%
%\iffalse
%<*samplepart3|samplepart4>
%\fi

% Optional override for |\version| flag:
%    \begin{macrocode}
%%\providecommand{\version}{final}
%    \end{macrocode}

% Include the main document:
%    \begin{macrocode}
\input{childdoc.def}
\childdocby{cdocsamp}
%    \end{macrocode}

%\iffalse
%</samplepart3|samplepart4>
%\fi
%
%\iffalse
%<*samplepart3>
%\fi
% Some text for part 3:
%    \begin{macrocode}
some text in part three
%    \end{macrocode}

%\iffalse
%</samplepart3>
%\fi
% Some text for part 4:
%\iffalse
%<*samplepart4>
%\fi
%    \begin{macrocode}
more text in part four
%    \end{macrocode}

%\iffalse
%</samplepart4>
%\fi
%
% %%%%%%%%%%%%%%%%%%%%%%%%%%%%%%%%%%%%%%
% \paragraph{Forwarding for a Complete Draft.}
%
% The following forwarding file |cdocsdrf.tex|
% compiles the main document in draft mode:
%\iffalse
%<*sampledraft>
%\fi
%    \begin{macrocode}
\def\version{draft}
\input{childdoc.def}
\childdocforward{cdocsamp}
%    \end{macrocode}

%\iffalse
%</sampledraft>
%\fi
%
% %%%%%%%%%%%%%%%%%%%%%%%%%%%%%%%%%%%%%%
% \paragraph{Forwarding for Final Version of the Chapters.}
%
% The following forwarding files |cdocsfn1.tex| and |cdocsfn2.tex|
% (with identical content)
% compile the final versions of the child documents
% |cdocsch1.tex| and |cdocsch2.tex|, respectively:
%\iffalse
%<*samplefinal>
%\fi
%    \begin{macrocode}
\def\version{final}
\input{childdoc.def}
\childdocforwardprefix[cdocsamp]{cdocsfn}{cdocsch}
%    \end{macrocode}

%\iffalse
%</samplefinal>
%\fi
%
% %%%%%%%%%%%%%%%%%%%%%%%%%%%%%%%%%%%%%%
% \paragraph{Command Line Processing.}
%
% The following three command lines generate the output files
% |cdocscld|, |cdocscl1| and |cdocscl2|
% which should be identical to
% |cdocsdrf|, |cdocsch1| and |cdocsfn2|, respectively:
% \begin{center}
% \begin{tabular}{l}
% |latex -jobname cdocscld \|\\
% |  "\def\version{draft}\input{childdoc.def}\childdocforward{cdocsamp}"|\\
% |latex -jobname cdocscl1 \|\\
% |  "\input{childdoc.def}\childdocforward[cdocsamp]{cdocsch1}"|\\
% |latex -jobname cdocscl2 \|\\
% |  "\def\version{final}\input{childdoc.def}\childdocforward{cdocsch2}"|
% \end{tabular}
% \end{center}
% Note that the trailing backslash on each first line
% merely continues the input to the second line
% (for convenient cut ant paste).
% Furthermore, the command |latex| can be replaced by any
% of its alternative versions such as |pdflatex|.
%
% %%%%%%%%%%%%%%%%%%%%%%%%%%%%%%%%%%%%%%%%%%%%%%%%%%%%%%%%%%%%%%%%%%%%%%%%%%%%%%
% %%%%%%%%%%%%%%%%%%%%%%%%%%%%%%%%%%%%%%%%%%%%%%%%%%%%%%%%%%%%%%%%%%%%%%%%%%%%%%
% \section{Implementation}
%\iffalse
%<*package>
%\fi
%
% This section describes the definitions file |childdoc.def|.

% The definitions cannot be loaded using |\usepackage| or |\RequirePackage|
% which has a mechanism to prevent loading a style file more than once.
% When loading the definitions by means of |\input|
% multiple instances have to be prevented manually:
%\iffalse
%This code needs to be before the `\ProvidesFile' directive
%which is defined at the beginning of this file.
%Therefore it is also placed there and commented out here.
%</package>
%<*discard>
%\fi
%    \begin{macrocode}
\ifdefined\childdocmain\endinput\fi
%    \end{macrocode}
%\iffalse
%</discard>
%<*package>
%\fi
%
% \macro{\ifchilddoc}
% \macro{\ifchilddocmanual}
% The conditional |\ifchilddoc| tells whether a
% child (true) or main (false) document is being compiled.
% The conditional |\ifchilddocmanual| tells whether
% the |\includeonly| mechanism is used (false) or
% the selection of child files must be performed manually (true).
% The definitions initialise to false:
%    \begin{macrocode}
\newif\ifchilddoc
\newif\ifchilddocmanual
%    \end{macrocode}

% \macro{\childdocname}
% \macro{\childdocjob}
% The macro |\childdocname| stores the name of the main document
% to be compiled. The macro |\childdocjob| stores the name of
% the document on which the \LaTeX{} compiler was originally invoked.
% The content of |\jobname| cannot be compared
% to filenames specified in the source due to different catcodes.
% The following code rescans |\jobname|, stores the result
% in |\childdocname| and saves a copy in |\childdocjob|:
%    \begin{macrocode}
\edef\childdocname{\scantokens\expandafter{\jobname\noexpand}}
\let\childdocjob\childdocname
%    \end{macrocode}

% \macro{\childdocdisable}
% The macro |\childdocdisable| prevents the main file
% from being processed more than once.
% At this stage, the main document command |\childdocmain|
% is assumed to be called once again where it should do nothing.
% Any subsequent call to it should prevent
% a secondary processing of the main document
% It overwrites the forwarding commands
% |\childdocof| and |\childdocforward|
% with empty macros to prevent further inclusions of the main document:
%    \begin{macrocode}
\newcommand{\childdocdisable}
{
  \renewcommand{\childdocmain}[1]{\renewcommand{\childdocmain}[1]{\endinput}}
  \renewcommand{\childdocof}[1]{}
  \renewcommand{\childdocby}[2][]{}
  \renewcommand{\childdocforward}[2][]{}
  \renewcommand{\childdocdisable}{}
}
%    \end{macrocode}

% \macro{\childdocmain}
% The macro |\childdocmain| is to be called at the top of the main file
% with nothing or the main filename (without extension) as argument.
% First, it breaks loops.
% If the argument is not empty and does not match |\childdocname|
% (which is set by the first inclusion of |childdoc.def|),
% |\ifchilddoc| is set to true, |\includeonly| is applied to the child file
% and |\jobname| is set to the main file
% (for proper handling of |.aux| files):
%    \begin{macrocode}
\newcommand{\childdocmain}[1]
{
  \childdocdisable\childdocmain{}
  \if?#1?\else
    \begingroup
      \def\childdoctmp{#1}
      \ifx\childdoctmp\childdocname
        \def\childdoctmp{}
      \else
        \def\childdoctmp
        {
          \childdoctrue
          \includeonly{\childdocname}
          \def\childdocjob{#1}
          \def\jobname{#1}
        }
      \fi
      \expandafter
    \endgroup
    \childdoctmp
  \fi
}
%    \end{macrocode}

% \macro{\childdocof}
% The command |\childdocof| redirects
% compilation to the main file |#1|.
%    \begin{macrocode}
\newcommand{\childdocof}[1]
{
  \childdocdisable
  \childdoctrue
  \includeonly{\childdocname}
  \def\jobname{#1}
  \def\childdocjob{#1}
  \input{#1}
}
%    \end{macrocode}

% \macro{\childdocby}
% The command |\childdocby| ....
%    \begin{macrocode}
\newcommand{\childdocby}[2][]
{
  \childdocdisable
  \childdoctrue
  \childdocmanualtrue
  \if?#1?\else
    \def\jobname{#2}
  \fi
  \def\childdocjob{#2}
  \input{#2}
  \endinput
}
%    \end{macrocode}

% \macro{\childdocforward}
% The command |\childdocforward| redirects
% compilation to the main file or
% (if the optional argument is given) a child file.
% Parameters are set as if the main file
% or a child file starting with |\childdocof| was compiled.
% Then compilation is handed over to the main file:
%    \begin{macrocode}
\newcommand{\childdocforward}[2][]
{
  \begingroup
    \if?#1?
      \def\childdoctmp
      {
        \def\childdocname{#2}
        \def\childdocjob{#2}
        \def\jobname{#2}
        \input{#2}
        \endinput
      }
    \else
      \def\childdoctmp
      {
        \childdocdisable
        \def\childdocname{#2}
        \childdoctrue
        \includeonly{#2}
        \def\childdocjob{#1}
        \def\jobname{#1}
        \input{#1}
        \endinput
      }
    \fi
    \expandafter
  \endgroup
  \childdoctmp
}
%    \end{macrocode}

% \macro{\childdocforwardprefix}
% The command |\childdocforwardprefix| redirects
% compilation to the main or a child file by means of a pattern.
% The prefix |#1| in the current filename is replaced by |#2|
% and the suffix of the current filename is kept
% (it is assumed that the filename does not contain the substring `|~~~|'
% which is used as a delimiter).
% Compilation is handed over to the new file by |\childdocforward|:
%    \begin{macrocode}
\newcommand{\childdocforwardprefix}[3][]
{
  \begingroup
    \def\childdocextract #2##1~~~{\def\childdoctmp{\childdocforward[#1]{#3##1}}}
    \expandafter\childdocextract\childdocname~~~
    \expandafter
  \endgroup
  \childdoctmp
}
%    \end{macrocode}

% \macro{\childdoc}
% The deprecated macro |\childdoc| is a legacy version of |\childdocmain|:
%    \begin{macrocode}
\newcommand{\childdoc}{\childdocmain}
%    \end{macrocode}

% \macro{\childdocredirect}
% The deprecated macro |\childdocredirect| is a legacy version
% of |\childdocforward| and |\childdocforwardprefix|:
%    \begin{macrocode}
\newcommand{\childdocredirect}[2][]
{
  \begingroup
    \if?#1?
      \def\childdoctmp{\childdocforward{#2}}
    \else
      \def\childdoctmp{\childdocforwardprefix{#1}{#2}}
    \fi
    \expandafter
  \endgroup
  \childdoctmp
}
%    \end{macrocode}

%\iffalse
%</package>
%\fi
%
\endinput
|\\
|\childdocforwardprefix{final}{child}|
\end{tabular}
\end{center}
%

Note that when several versions of a main file and/or of each child file
are to be generated, it may be convenient to set up a |Makefile| or
shell script to automatise the process.

%%%%%%%%%%%%%%%%%%%%%%%%%%%%%%%%%%%%%%%%%%%%%%%%%%%%%%%%%%%%%%%%%%%%%%%%%%%%%%%%
\subsection{Command Line Processing}
\label{sec:commandline}

The effect of redirection files can also be achieved by invoking
the \LaTeX{} compiler with a more elaborate command line.
Most conveniently this should be done as part
of a shell script or a |Makefile|.

When using \textsf{childdoc} in the main file, the following
command lines effectively perform a redirection
(note that depending on the shell being used,
backslashes may have to be doubled: `|\|' $\to$ `|\\|'):
%
\begin{center}
|... -jobname "|\textit{target}|" |\\|"|[\textit{flags}]%
|% \iffalse
%
% childdoc.dtx Copyright (C) 2017-2018 Niklas Beisert
%
% This work may be distributed and/or modified under the
% conditions of the LaTeX Project Public License, either version 1.3
% of this license or (at your option) any later version.
% The latest version of this license is in
%   http://www.latex-project.org/lppl.txt
% and version 1.3 or later is part of all distributions of LaTeX
% version 2005/12/01 or later.
%
% This work has the LPPL maintenance status `maintained'.
%
% The Current Maintainer of this work is Niklas Beisert.
%
% This work consists of the files childdoc.dtx and childdoc.ins
% and the derived files childdoc.def and cdocsamp.tex with
% cdocsch1.tex, cdocsch2.tex, cdocsdrf.tex, cdocsfn1.tex, cdocsfn2.tex.
%
%<package>\ifdefined\childdocmain\endinput\fi
%<package>\ProvidesFile{childdoc.def}[2018/12/30 v2.0 child document driver]
%<samplemain>\ProvidesFile{cdocsamp.tex}[2018/12/30 v2.0 sample for childdoc]
%<*driver>
%\ProvidesFile{childdoc.drv}[2018/12/30 v2.0 childdoc reference manual file]
\PassOptionsToClass{10pt,a4paper}{article}
\documentclass{ltxdoc}

\usepackage[margin=35mm]{geometry}
\usepackage{hyperref}
\usepackage{hyperxmp}
\usepackage[usenames]{color}

\hypersetup{colorlinks=true}
\hypersetup{pdfstartview=FitH}
\hypersetup{pdfpagemode=UseNone}
\hypersetup{pdfsource={}}
\hypersetup{pdflang={en-UK}}
\hypersetup{pdfcopyright={Copyright 2017-2018 Niklas Beisert.
  This work may be distributed and/or modified under the
  conditions of the LaTeX Project Public License, either version 1.3
  of this license or (at your option) any later version.}}
\hypersetup{pdflicenseurl={http://www.latex-project.org/lppl.txt}}
\hypersetup{pdfcontactaddress={ETH Zurich, ITP, HIT K,
  Wolfgang-Pauli-Strasse 27}}
\hypersetup{pdfcontactpostcode={8093}}
\hypersetup{pdfcontactcity={Zurich}}
\hypersetup{pdfcontactcountry={Switzerland}}
\hypersetup{pdfcontactemail={nbeisert@itp.phys.ethz.ch}}
\hypersetup{pdfcontacturl={http://people.phys.ethz.ch/\xmptilde nbeisert/}}

\newcommand{\secref}[1]{\hyperref[#1]{section \ref*{#1}}}

\parskip1ex
\parindent0pt
\let\olditemize\itemize
\def\itemize{\olditemize\parskip0pt}

\begin{document}

\title{The \textsf{childdoc} Package}
\hypersetup{pdftitle={The childdoc Package}}
\author{Niklas Beisert\\[2ex]
  Institut f\"ur Theoretische Physik\\
  Eidgen\"ossische Technische Hochschule Z\"urich\\
  Wolfgang-Pauli-Strasse 27, 8093 Z\"urich, Switzerland\\[1ex]
  \href{mailto:nbeisert@itp.phys.ethz.ch}
  {\texttt{nbeisert@itp.phys.ethz.ch}}}
\hypersetup{pdfauthor={Niklas Beisert}}
\hypersetup{pdfsubject={Manual for the LaTeX2e Package childdoc}}
\date{30 December 2018, \textsf{v2.0}}
\maketitle

\begin{abstract}\noindent
\textsf{childdoc} is a \LaTeXe{} package
that enables the direct compilation
of document sections included by |\include|
to individual files.
\end{abstract}

\begingroup
\parskip0ex
\tableofcontents
\endgroup

%%%%%%%%%%%%%%%%%%%%%%%%%%%%%%%%%%%%%%%%%%%%%%%%%%%%%%%%%%%%%%%%%%%%%%%%%%%%%%%%
%%%%%%%%%%%%%%%%%%%%%%%%%%%%%%%%%%%%%%%%%%%%%%%%%%%%%%%%%%%%%%%%%%%%%%%%%%%%%%%%
\section{Introduction}

\LaTeX{} provides a mechanism to structure a large document (such as a book)
into a main file and several child files (containing the chapters)
using the |\include| command.
This mechanism is beneficial for documents
which span hundreds of pages in order to
make the source file(s) more manageable.
Moreover, compilation can be restricted to
selected child files by means of the |\includeonly| command.
The latter feature can be used to reduce the compilation time while editing
(this was significantly more useful in the earlier days of \LaTeX{})
or to generate a smaller document which is easier to navigate.
Another application of |\includeonly| is to generate
documents consisting of selected parts of the complete document.

However, there are a few drawbacks of the plain |\include| mechanism:
\begin{itemize}
\item
The child files cannot be compiled on their own,
they can only be compiled via the main file.
A naive editing environment
(such as a text editor with an option
to have the current file processed by \LaTeX)
may require one to switch to the main file before compiling;
attempting to compile the child file produces errors.
\item
The main file must be modified (each time)
to adjust the |\includeonly| command
to the present needs. This easily leaves the main file in a messy state.
\item
The generated document will always carry the filename
of the main document. This is inconvenient if
several child files are to be compiled and
to be kept for distribution.
\end{itemize}

The present package provides a simple interface
to make child files individually compilable by \LaTeX{}.
Compiling a child file then has the same effect as compiling
the main file with an |\includeonly| command
to select the appropriate child.
Moreover the generated document will carry the name of the child
rather than the main file.
This resolves all three above issues.

This feature is meant to make the editing of books,
thesis documents and lecture notes somewhat more convenient.
However, the package can also be used efficiently for
composing a series of documents (such as exercise sheets)
which are typically distributed individually.
It then assists the author in generating the individual documents
(potentially in different versions)
as well as a document containing the collected series.
Another application is in developing style files
or other kinds of included material
where compilation of the style file could redirect
to a sample or test file.

%%%%%%%%%%%%%%%%%%%%%%%%%%%%%%%%%%%%%%%%%%%%%%%%%%%%%%%%%%%%%%%%%%%%%%%%%%%%%%%%
%%%%%%%%%%%%%%%%%%%%%%%%%%%%%%%%%%%%%%%%%%%%%%%%%%%%%%%%%%%%%%%%%%%%%%%%%%%%%%%%
\section{Usage}

First of all, the package \textsf{childdoc} is \emph{not} a standard
\LaTeXe{} |.sty| style file! Therefore it needs to be invoked in
a non-standard way.

%%%%%%%%%%%%%%%%%%%%%%%%%%%%%%%%%%%%%%%%%%%%%%%%%%%%%%%%%%%%%%%%%%%%%%%%%%%%%%%%
\subsection{Included Files}
\label{sec:include}

%%%%%%%%%%%%%%%%%%%%%%%%%%%%%%%%%%%%%%%%
\DescribeMacro{\childdocmain}
To use the package, add the commands
\begin{center}
\begin{tabular}{l}
|\input{childdoc.def}|\\
|\childdocmain{}|\\
\end{tabular}
\end{center}
at the very top of the main \LaTeX{} file,
in particular \emph{before} the |\documentclass| statement!
The argument of |\childdocmain| should be left empty
(but it must be present).

%%%%%%%%%%%%%%%%%%%%%%%%%%%%%%%%%%%%%%%%
\DescribeMacro{\childdocof}
Furthermore, add the commands
\begin{center}
\begin{tabular}{l}
|\input{childdoc.def}|\\
|\childdocof{|\textit{main}|}|\\
\end{tabular}
\end{center}
at the top of every child file \textit{child}
which is included by |\include{|\textit{child}|}|
from within the main file
(or at least for those files to be compiled individually).
The argument \textit{main} must be the filename of the main file.

There are a couple of
considerations in setting up the main and child documents:

%%%%%%%%%%%%%%%%%%%%%%%%%%%%%%%%%%%%%%%%
\paragraph{Restrictions.}

Please note the following restrictions:
\begin{itemize}
\item
|\childdocmain| must be called with one argument \textit{main}
to ensure compatibility with earlier version of the package.
It must either be empty (|\childdocmain{}|)
or precisely match the filename of the main file in which it is specified.
See \secref{sec:detection} for further information.
\item
The filename \textit{main} must be specified without the |.tex| extension.
\item
The filename \textit{main} is case sensitive
(even in case-insensitive file systems)
due to internal string comparison.
\item
The argument \textit{main} should be fully expanded, it cannot be a macro.
\item
Subdirectories and special characters should be avoided in filenames.
\item
The command |\childdocmain{|\textit{main}|}| must be followed by a whitespace.
It should not be followed immediately by another command
or by a comment mark `|%|'.
This is because the \TeX{} parser reads the token immediately following
the argument of |\childdocmain| and puts it
at the beginning of every child section;
however, a white\-space is ignored.
\end{itemize}

%%%%%%%%%%%%%%%%%%%%%%%%%%%%%%%%%%%%%%%%
\paragraph{Content of Main File.}

It is advisable to place all content in the child files included by |\include|.
Any output contained in the main file will appear in all child documents
unless suppressed manually;
it cannot be suppressed automatically by the |\includeonly| directive
and thus should normally be avoided.
A method to include some content in the main file
by means of conditional processing is described in \secref{sec:conditional}.

%%%%%%%%%%%%%%%%%%%%%%%%%%%%%%%%%%%%%%%%
\paragraph{Page Numbering.}

When only a part of the document is compiled,
the appropriate numbering of pages
(as well as other status parameters)
is determined from the |.aux| files.
The latter contain information from previous passes.
However this information needs to propagate through
all intermediate child documents.
Therefore the page numbering in child documents may well
be inconsistent until the complete document is compiled at least once.

A useful (if unconventional) way to always ensure a consistent
page numbering is to restart the numbering in each child document
and denote the pages by `\textit{child}|.|\textit{page}'
where \textit{child} represents the chapter/section number of the child file.
This can be achieved by the command
|\numberwithin{page}{|\textit{child}|}|
of the \textsf{amsmath} package
where \textit{child} can be |chapter| or |section|
depending on the chosen structuring.
Alternatively, one can modify the macro |\thepage| appropriately
and reset the counter |page| at the start of each child file.

%%%%%%%%%%%%%%%%%%%%%%%%%%%%%%%%%%%%%%%%%%%%%%%%%%%%%%%%%%%%%%%%%%%%%%%%%%%%%%%%
\subsection{Conditional Processing}
\label{sec:conditional}

The package provides a mechanism to compile different versions
of a document. To customise the versions further some conditional processing
can come in handy to distinguish which version is being compiled.
The package provides two macros to describe the compilation context:

%%%%%%%%%%%%%%%%%%%%%%%%%%%%%%%%%%%%%%%%
\DescribeMacro{\ifchilddoc}
The conditional |\ifchilddoc| distinguishes between the compilation of
child documents and the main document:
%
\begin{center}
|\ifchilddoc |\textit{child-code}| |[|\||else |\textit{main-code}]| \||fi|
\end{center}

%%%%%%%%%%%%%%%%%%%%%%%%%%%%%%%%%%%%%%%%
\DescribeMacro{\childdocname}
\DescribeMacro{\childdocjob}
The macro |\childdocname| contains the filename (without extension)
of the main or child file being processed.
Note that |\childdocjob| will always contain the name of the main file.

%%%%%%%%%%%%%%%%%%%%%%%%%%%%%%%%%%%%%%%%
\paragraph{Title Page.}

Conditional processing can be used to include a title or banner page
in the main document when proper precautions are taken.
Importantly, the code in the main file should ensure that the page counter
(as well as other status parameters which are stored in the |.aux| files)
takes the same value after the conditional processing.
Otherwise the page numbers may take divergent values
depending on which part is compiled.

For example, a title page could be declared by:
%
\begin{center}
\begin{tabular}{l}
|\ifchilddoc\||else|\\
|\addtocounter{page}{-1}|\\
\textit{code for title page}\\
|\newpage|\\
|\||fi|
\end{tabular}
\end{center}
%
A banner page for the child documents can be generated by:
%
\begin{center}
\begin{tabular}{l}
|\ifchilddoc|\\
|\addtocounter{page}{-1}|\\
\textit{code for banner page}\\
|\newpage|\\
|\||fi|
\end{tabular}
\end{center}
%
Here one could write a message such as:
\begin{center}
|This is the part \childdocname{} of \childdocjob{}.|
\end{center}

%%%%%%%%%%%%%%%%%%%%%%%%%%%%%%%%%%%%%%%%%%%%%%%%%%%%%%%%%%%%%%%%%%%%%%%%%%%%%%%%
\subsection{Flags}
\label{sec:flags}

The package makes it easy to generate different versions
of the main or child documents.
To this end compilation flags can be defined
and assigned different default values.
They will be particularly useful in conjunction
with the forwarding mechanism described in \secref{sec:forward}.

For example, it may be useful to have a flag |\version|
which can be set to |draft| or |final|.
The document source will contain some conditional code
depending on the value of |\version|.
Suppose further, the flag should default to |final| for the main file
and to |draft| for child files
which is a natural assignment for editing the document.
This is achieved by placing the following code
in the preamble of the main document
(below the |\childdocmain| directive):
%
\begin{center}
\begin{tabular}{l}
|\ifchilddoc|\\
|\providecommand{\version}{draft}|\\
|\||else|\\
|\providecommand{\version}{final}|\\
|\||fi|
\end{tabular}
\end{center}
%
The definition by |\providecommand| makes sure
that previous definitions are not overwritten.
Further statements |\providecommand{\version}{...}|
can thus be added before the above code to override it.

For the main file, one might add a line
(between |\childdocmain| and the above block)
%
\begin{center}
|%\ifchilddoc\||else\providecommand{\version}{draft}\||fi|
\end{center}
%
which can be uncommented to produce a draft version.
Likewise one can add a line to the very top of a child file
(above the |\childdocof{|\textit{main}|}| directive)
%
\begin{center}
|%\providecommand{\version}{final}|
\end{center}
%
which can be uncommented to produce the final version of this child document.

%%%%%%%%%%%%%%%%%%%%%%%%%%%%%%%%%%%%%%%%%%%%%%%%%%%%%%%%%%%%%%%%%%%%%%%%%%%%%%%%
\subsection{Forwarding}
\label{sec:forward}

Different versions of the main or child documents
using compilation flags as described in \secref{sec:flags}
can be (permanently) stored in different files
for convenient compilation, viewing and distribution.
To this end, the package defines a command
to pass on compilation to a different file:

%%%%%%%%%%%%%%%%%%%%%%%%%%%%%%%%%%%%%%%%
\DescribeMacro{\childdocforward}
The command |\childdocforward| redirects processing to
another source file:
%
\begin{center}
\begin{tabular}{l}
|\input{childdoc.def}|\\
|\childdocforward[|\textit{main}|]{|\textit{dest}|}|\\
\end{tabular}
\end{center}
%
The argument \textit{dest} is the destination file
(without extension).
It should be the main file or one of the child files.
Note that further \textsf{childdoc} directives
such as |\childdocof| and |\childdocforward|
in the indicated file will be processed in this form.
The optional argument \textit{main}
passes on directly to the main file \textit{main}
while pretending to compile the child \textit{dest}.
This form behaves as if \textit{dest}
issues |\childdocof{|\textit{main}|}| right away,
and no further \textsf{childdoc} directives will be processed.

%%%%%%%%%%%%%%%%%%%%%%%%%%%%%%%%%%%%%%%%
\DescribeMacro{\...prefix}
In the alternative form |\childdocforwardprefix|,
%
\begin{center}
\begin{tabular}{l}
|\input{childdoc.def}|\\
|\childdocforwardprefix[|\textit{main}|]{|\textit{prefix}|}{|\textit{dest}|}|
\end{tabular}
\end{center}
%
the destination file is determined by a pattern
depending on the current file:
To make this work, the current file must be called
`{\textit{prefix}\hspace{0.2em}\textit{suffix}}'
with \textit{prefix} matching precisely the argument.
Processing is then passed on to the file
`{\textit{dest}\hspace{0.2em}\textit{suffix}}'.
Surely, the same effect is achieved by
directly specifying the
argument `{\textit{dest}\hspace{0.2em}\textit{suffix}}'
in the first form.
However, that requires to set up a different file
for each child. With the alternative form of the command
all these files can have exactly the same content
which simplifies setting them up and maintaining them.

For example, the following file |draft.tex|
with a compilation flag |\version| as described in \secref{sec:flags}
compiles the main document as a draft:
%
\begin{center}
\begin{tabular}{l}
|\def\version{draft}|\\
|\input{childdoc.def}|\\
|\childdocforward{|\textit{main}|}|
\end{tabular}
\end{center}
%
Likewise, the following files |final|\textit{nn}|.tex|
compile the final version of the child document
|child|\textit{nn}|.tex|:
%
\begin{center}
\begin{tabular}{l}
|\def\version{final}|\\
|\input{childdoc.def}|\\
|\childdocforwardprefix{final}{child}|
\end{tabular}
\end{center}
%

Note that when several versions of a main file and/or of each child file
are to be generated, it may be convenient to set up a |Makefile| or
shell script to automatise the process.

%%%%%%%%%%%%%%%%%%%%%%%%%%%%%%%%%%%%%%%%%%%%%%%%%%%%%%%%%%%%%%%%%%%%%%%%%%%%%%%%
\subsection{Command Line Processing}
\label{sec:commandline}

The effect of redirection files can also be achieved by invoking
the \LaTeX{} compiler with a more elaborate command line.
Most conveniently this should be done as part
of a shell script or a |Makefile|.

When using \textsf{childdoc} in the main file, the following
command lines effectively perform a redirection
(note that depending on the shell being used,
backslashes may have to be doubled: `|\|' $\to$ `|\\|'):
%
\begin{center}
|... -jobname "|\textit{target}|" |\\|"|[\textit{flags}]%
|\input{childdoc.def}\childdocforward[|\textit{main}|]{|\textit{dest}|}"|
\end{center}
%
Here \textit{target} is the name of the output file,
\textit{main} is the name of the main file
and \textit{dest} is the name of the main or child file to be processed
(all filenames without extensions).
The optional argument \textit{main} can be omitted
if \textit{main} matches \textit{dest}.
Optionally, compilation \textit{flags} can be defined via |\def| commands.
This command line makes the \TeX{} engine believe
it is compiling the file \textit{target}
whose content is specified as the latter parameter.
The provided code then forwards the processing to
\textit{main} or \textit{dest} as described in \secref{sec:forward}.

%%%%%%%%%%%%%%%%%%%%%%%%%%%%%%%%%%%%%%%%%%%%%%%%%%%%%%%%%%%%%%%%%%%%%%%%%%%%%%%%
\subsection{Include by Input}
\label{sec:input}

Including child documents by |\include| has some restrictions by design.
Most notably, the content of a child document always occupies
its own set of pages; pages cannot be shared between child documents.
Usually, this behaviour makes perfect sense
because each child document contain an essential part of the document.
However, in some situations it may be desirable to compose
a document from a collection of parts
without having mandatory page breaks between then.
For this case, the package
provides a mechanism to include parts
by |\input| which can also be processed individually.
However, by construction this mechanism
requires manual handling of the content to be output.

%%%%%%%%%%%%%%%%%%%%%%%%%%%%%%%%%%%%%%%%
\DescribeMacro{\ifchilddocmanual}
The main file should be prepared as usual, see \secref{sec:include}.
However, the document body must make a distinction
between processing of an individual part and of the main document, e.g.:
%
\begin{center}
\begin{tabular}{l}
|\ifchilddocmanual|\\
|\input{\childdocname}|\\
|\||else|\\
\textit{document body with }|\input{|\textit{part}|}|\\
|\||fi|
\end{tabular}
\end{center}
%
The conditional |\ifchilddocmanual| is true whenever
a part to be included by |\input| is being compiled,
and the name of the part is stored in |\childdocname|.

%%%%%%%%%%%%%%%%%%%%%%%%%%%%%%%%%%%%%%%%
\DescribeMacro{\childdocby}
Each part to be included by |\input| should start with:
%
\begin{center}
\begin{tabular}{l}
|\input{childdoc.def}|\\
|\childdocby{|\textit{main}|}|\\
\end{tabular}
\end{center}
%
The directive |\childdocby| is similar to |\childdocof|
described in \secref{sec:include},
but the subsequent selection of content must be done manually.
To that end, both |\ifchilddoc| and |\ifchilddocmanual|
will be true upon processing of a part,
and the name of the part is stored in |\childdocname|.
Note that |\jobname| will be set to the filename of the current part
so that each part receives an individual |.aux| file
that does not interfere with the |.aux| file(s) of the main document.
This behaviour can be altered by the alternative form
|\childdocby[*]{|\textit{main}|}| (with a non-empty optional argument)
which uses the |.aux| file of the main document
by setting |\jobname| to \textit{main}.

%%%%%%%%%%%%%%%%%%%%%%%%%%%%%%%%%%%%%%%%%%%%%%%%%%%%%%%%%%%%%%%%%%%%%%%%%%%%%%%%
\subsection{Driver Development}
\label{sec:driver}

The \textsf{childdoc} mechanism can also be use for the development
of definition files such as \LaTeX{} styles or classes.
This case differs from the above setup with multiple parts
included by |\include| in that no |\includeonly| should be invoked.
This can be achieved by starting the include file
(before |\ProvidesPackage|) with:
%
\begin{center}
\begin{tabular}{l}
|\input{childdoc.def}|\\
|\childdocforward{|\textit{main}|}|\\
\end{tabular}
\end{center}
%
or alternatively with:
%
\begin{center}
\begin{tabular}{l}
|\input{childdoc.def}|\\
|\childdocby{|\textit{main}|}|\\
\end{tabular}
\end{center}
%
Both forms have slightly different effects as described above.
The main file is prepared as usual, see \secref{sec:include}.

%%%%%%%%%%%%%%%%%%%%%%%%%%%%%%%%%%%%%%%%%%%%%%%%%%%%%%%%%%%%%%%%%%%%%%%%%%%%%%%%
\subsection{Legacy Detection}
\label{sec:detection}

The directive |\childdocmain| in the main file can detect
whether the complete document or merely a child is to be compiled
even without using the directive |\childdocof|.
This method is deprecated because it is less robust
and there is no compelling reason to use it;
it is merely provided for backward compatibility
and it may be removed in future versions.

If the detection mechanism is to be used,
it is mandatory to correctly specify
the filename of the main file as the argument of |\childdocmain|:
%
\begin{center}
\begin{tabular}{l}
|\input{childdoc.def}|\\
|\childdocmain{|\textit{main}|}|\\
\end{tabular}
\end{center}
%
If |\jobname| does not match the argument \textit{main} of |\childdocmain|,
it is assumed that |\jobname| points to the child file to be compiled.
When using |\childdocmain| with the main file specified as argument,
it suffices to start a child file
with just |\input{|\textit{main}|}|
without loading of the package and using |\childdocof|.
If instead all processing is done
with the appropriate \textsf{childdoc} directives,
the argument of \textit{main} of |\childdocmain| can be empty.

An alternative version of the command line processing described
in \secref{sec:commandline} using the detection mechanism reads:
%
\begin{center}
|... -jobname "|\textit{target}|" "|[\textit{flags}]%
[|\def\jobname{|\textit{dest}|}|]|\input{|\textit{main}|}"|
\end{center}

%%%%%%%%%%%%%%%%%%%%%%%%%%%%%%%%%%%%%%%%%%%%%%%%%%%%%%%%%%%%%%%%%%%%%%%%%%%%%%%%
\subsection{Manual Code}
\label{sec:manual}

In case one cannot be certain whether the definitions file |childdoc.def|
is installed on the target \TeX{} distribution
and one prefers not to ship it,
it is conceivable to paste a few relevant commands into the sources.

To that end, drop all statements |\input{childdoc.def}|
and perform the replacements as outlined below.
Instead of |\childdocmain{|\textit{main}|}| add the following code
to the top of the main file:
%
\begin{center}
\begin{tabular}{l}
|\||ifdefined\childdocname\endinput\||fi\newif\ifchilddoc|\\
|\edef\childdocname{\scantokens\expandafter{\jobname\noexpand}}|\\
|\def\childdocmain{|\textit{main}|}\||ifx\childdocmain\childdocname\||else|\\
|\childdoctrue\includeonly{\childdocname}\let\jobname\childdocmain\||fi|\\
\end{tabular}
\end{center}
%
Instead of |\childdocof{|\textit{main}|}| just include the main file
at the top of each child file:
%
\begin{center}
|\input{|\textit{main}|}|
\end{center}
%
A simple redirection |\childdocforward{|\textit{dest}|}| is achieved by:
%
\begin{center}
|\def\jobname{|\textit{dest}|}\input{\jobname}|
\end{center}
%
The redirection with prefix
|\childdocforwardprefix[|\textit{prefix}|]{|\textit{dest}|}|
is accomplished by:
%
\begin{center}
\begin{tabular}{l}
|{\edef\jobname{\scantokens\expandafter{\jobname\noexpand}}|\\
|\def\redirectjob |\textit{prefix}|#1~~~{\gdef\jobname{|\textit{dest}|#1}}|\\
|\expandafter\redirectjob\jobname~~~}\input{\jobname}|
\end{tabular}
\end{center}

In an alternative approach,
child documents can be compiled by a specific command line
without additional code or specific definitions:
%
\begin{center}
|... -jobname "|\textit{target}|" "|[\textit{flags}]%
|\includeonly{|\textit{dest}|}\input{|\textit{main}|}"|
\end{center}
%

%%%%%%%%%%%%%%%%%%%%%%%%%%%%%%%%%%%%%%%%%%%%%%%%%%%%%%%%%%%%%%%%%%%%%%%%%%%%%%%%
%%%%%%%%%%%%%%%%%%%%%%%%%%%%%%%%%%%%%%%%%%%%%%%%%%%%%%%%%%%%%%%%%%%%%%%%%%%%%%%%
\section{Information}

%%%%%%%%%%%%%%%%%%%%%%%%%%%%%%%%%%%%%%%%%%%%%%%%%%%%%%%%%%%%%%%%%%%%%%%%%%%%%%%%
\subsection{Copyright}

Copyright \copyright{} 2017--2018 Niklas Beisert

This work may be distributed and/or modified under the
conditions of the \LaTeX{} Project Public License, either version 1.3
of this license or (at your option) any later version.
The latest version of this license is in
  \url{http://www.latex-project.org/lppl.txt}
and version 1.3 or later is part of all distributions of \LaTeX{}
version 2005/12/01 or later.

This work has the LPPL maintenance status `maintained'.

The Current Maintainer of this work is Niklas Beisert.

This work consists of the files |README.txt|, |childdoc.ins| and |childdoc.dtx|
as well as the derived files |childdoc.def|, |cdocsamp.tex|
with |cdocsch1.tex|, |cdocsch2.tex|, |cdocspt3.tex|, |cdocspt4.tex|,
|cdocsdrf.tex|, |cdocsfn1.tex|, |cdocsfn2.tex|
as well as |childdoc.pdf|.

%%%%%%%%%%%%%%%%%%%%%%%%%%%%%%%%%%%%%%%%%%%%%%%%%%%%%%%%%%%%%%%%%%%%%%%%%%%%%%%%
\subsection{Files and Installation}

The package consists of the files:
%
\begin{center}
\begin{tabular}{ll}
    |README.txt|   & readme file \\
    |childdoc.ins| & installation file \\
    |childdoc.dtx| & source file \\
    |childdoc.def| & definition file \\
    |cdocsamp.tex| & sample main file \\
    |cdocsch1.tex| & sample include file \\
    |cdocsch2.tex| & sample include file \\
    |cdocspt3.tex| & sample part file \\
    |cdocspt4.tex| & sample part file \\
    |cdocsdrf.tex| & sample redirection file \\
    |cdocsfn1.tex| & sample redirection file \\
    |cdocsfn2.tex| & sample redirection file \\
    |childdoc.pdf| & manual
\end{tabular}
\end{center}
%
The distribution consists of the files
|README.txt|, |childdoc.ins| and |childdoc.dtx|.
%
\begin{itemize}
\item
Run (pdf)\LaTeX{} on |childdoc.dtx|
to compile the manual |childdoc.pdf| (this file).
\item
Run \LaTeX{} on |childdoc.ins| to create the definitions file |childdoc.def|
and the sample |cdocsamp.tex| with include files
|cdocsch1.tex|, |cdocsch2.tex|, |cdocspt3.tex|, |cdocspt4.tex|,
|cdocsdrf.tex|, |cdocsfn1.tex|, |cdocsfn2.tex|.
Then copy the file |childdoc.def| to an appropriate directory of your \LaTeX{}
distribution, e.g.\ \textit{texmf-root}|/tex/latex/childdoc|.
\end{itemize}

%%%%%%%%%%%%%%%%%%%%%%%%%%%%%%%%%%%%%%%%%%%%%%%%%%%%%%%%%%%%%%%%%%%%%%%%%%%%%%%%
\subsection{Related CTAN Packages}

There are several other packages which offer a similar functionality:
%
\begin{itemize}
\item
The packages
\href{http://ctan.org/pkg/docmute}{\textsf{docmute}},
\href{http://ctan.org/pkg/includex}{\textsf{includex}} and
\href{http://ctan.org/pkg/standalone}{\textsf{standalone}}
provide commands to include only the document body of
a child file thus allowing both files to be compiled individually.
\item
The packages \href{http://ctan.org/pkg/subdocs}{\textsf{subdocs}}
and \href{http://ctan.org/pkg/subfiles}{\textsf{subfiles}}
provide structures in which the main and child documents can be
encapsulated and allowing them to be compiled individually.
The inclusion mechanism is different from the conventional |\include|.
\item
The package \href{http://ctan.org/pkg/combine}{\textsf{combine}}
is an elaborate solution to combine several documents into one.
\end{itemize}
%
See also the CTAN topic \href{http://ctan.org/topic/subdocs}{\textsf{subdocs}}
for further related packages.
The present package differs from the above solutions in that
a document structure constructed with the conventional |\include| mechanism
just needs two extra commands at the top of every file
such that all constituent files can be compiled individually.

%%%%%%%%%%%%%%%%%%%%%%%%%%%%%%%%%%%%%%%%%%%%%%%%%%%%%%%%%%%%%%%%%%%%%%%%%%%%%%%%
%\subsection{Feature Suggestions}
%
%The following is a list of features which may be useful for future
%versions of this package:
%%
%\begin{itemize}
%\item
%\ldots
%\end{itemize}

%%%%%%%%%%%%%%%%%%%%%%%%%%%%%%%%%%%%%%%%%%%%%%%%%%%%%%%%%%%%%%%%%%%%%%%%%%%%%%%%
\subsection{Revision History}

%%%%%%%%%%%%%%%%%%%%%%%%%%%%%%%%%%%%%%%%
\paragraph{v2.0:} 2018/12/30

\begin{itemize}
\item
immediate forward processing
\item
added |\childdocby| mechanism
\item
manual restructured
\end{itemize}

%%%%%%%%%%%%%%%%%%%%%%%%%%%%%%%%%%%%%%%%
\paragraph{v1.6:} 2018/01/17

\begin{itemize}
\item
application for development of include files
\item
corrections to manual
\end{itemize}

%%%%%%%%%%%%%%%%%%%%%%%%%%%%%%%%%%%%%%%%
\paragraph{v1.5:} 2017/05/21

\begin{itemize}
\item
more complete structuring introduced
\item
|\childdocof| introduced
\item
|\childdoc| renamed to |\childdocmain|
\item
|\childredirect| renamed to |\childdocforward| and |\childdocforwardprefix|
and functionality expanded
\end{itemize}

%%%%%%%%%%%%%%%%%%%%%%%%%%%%%%%%%%%%%%%%
\paragraph{v1.0:} 2017/04/27

\begin{itemize}
\item
manual and install package
\item
first version published on CTAN
\end{itemize}

%%%%%%%%%%%%%%%%%%%%%%%%%%%%%%%%%%%%%%%%
\paragraph{v0.6:} 2017/04/26

\begin{itemize}
\item
redirection mechanism added
\end{itemize}

%%%%%%%%%%%%%%%%%%%%%%%%%%%%%%%%%%%%%%%%
\paragraph{v0.5:} 2017/04/26

\begin{itemize}
\item
functionality in definition file
\end{itemize}


%%%%%%%%%%%%%%%%%%%%%%%%%%%%%%%%%%%%%%%%%%%%%%%%%%%%%%%%%%%%%%%%%%%%%%%%%%%%%%%%
%%%%%%%%%%%%%%%%%%%%%%%%%%%%%%%%%%%%%%%%%%%%%%%%%%%%%%%%%%%%%%%%%%%%%%%%%%%%%%%%
%%%%%%%%%%%%%%%%%%%%%%%%%%%%%%%%%%%%%%%%%%%%%%%%%%%%%%%%%%%%%%%%%%%%%%%%%%%%%%%%
\appendix

\settowidth\MacroIndent{\rmfamily\scriptsize 000\ }

 \DocInput{childdoc.dtx}

\end{document}
%</driver>
% \fi
%
% %%%%%%%%%%%%%%%%%%%%%%%%%%%%%%%%%%%%%%%%%%%%%%%%%%%%%%%%%%%%%%%%%%%%%%%%%%%%%%
% %%%%%%%%%%%%%%%%%%%%%%%%%%%%%%%%%%%%%%%%%%%%%%%%%%%%%%%%%%%%%%%%%%%%%%%%%%%%%%
% \section{Sample}
%\iffalse
%<*samplemain>
%\fi
%
% The following presents a sample document
% with two chapters, two parts, a title page,
% a compile flag as well as three forwarding files to set the flag.
% It consists of eight |.tex| files:
% \begin{center}
% \begin{tabular}{ll}
% |cdocsamp.tex|&main file\\
% |cdocsch1.tex|&include file for chapter 1\\
% |cdocsch2.tex|&include file for chapter 2\\
% |cdocspt3.tex|&include file for part 3\\
% |cdocspt4.tex|&include file for part 4\\
% |cdocsdrf.tex|&forwarding file for main file in draft mode\\
% |cdocsfi1.tex|&forwarding file for final version of chapter 1\\
% |cdocsfi2.tex|&forwarding file for final version of chapter 2\\
% \end{tabular}
% \end{center}
% Each of the eight files can be compiled directly by the \LaTeX{} compiler.
%
% %%%%%%%%%%%%%%%%%%%%%%%%%%%%%%%%%%%%%%
% \paragraph{Main File.}
%
% The main file is called |cdocsamp.tex|.
%
% Load the \textsf{childdoc} definitions and
% declare the filename for the main document:
%    \begin{macrocode}
\input{childdoc.def}
\childdocmain{}
%    \end{macrocode}

% Optional override for |\version| flag:
%    \begin{macrocode}
%%\ifchilddoc\else\providecommand{\version}{draft}\fi
%    \end{macrocode}

% Define the default values for the |\version| flag
% (|final| for the main file and |draft| for childs):
%    \begin{macrocode}
\ifchilddoc
\providecommand{\version}{draft}
\else
\providecommand{\version}{final}
\fi
%    \end{macrocode}

% Load the standard document class:
%    \begin{macrocode}
\documentclass[12pt]{article}
%    \end{macrocode}

% Start the document body:
%    \begin{macrocode}
\begin{document}
%    \end{macrocode}

% Declare a title page.
% Print title, part of document being processed and version flag:
%    \begin{macrocode}
\addtocounter{page}{-1}
\begin{center}
{\LARGE\bfseries{}childdoc example\par}
\vspace{1cm}
\ifchilddoc
\ifchilddocmanual part\else chapter\fi:
`\childdocname' of `\childdocjob'\par
\else
main document: `\childdocjob'\par
\fi
version: \version\par
\end{center}
\newpage
%    \end{macrocode}

% Manually include selected file,
% otherwise process as usual:
%    \begin{macrocode}
\ifchilddocmanual
\section*{part `\childdocname'}
\input{\childdocname}
\else
%    \end{macrocode}

% Include the two chapters:
%    \begin{macrocode}
\include{cdocsch1}
\include{cdocsch2}
%    \end{macrocode}

% Include the two parts unless only chapters should be displayed:
%    \begin{macrocode}
\ifchilddoc\else
\section{part three}
\input{cdocspt3}
\section{part four}
\input{cdocspt4}
\fi
%    \end{macrocode}

% Process as usual until here:
%    \begin{macrocode}
\fi
%    \end{macrocode}

% End of document body:
%    \begin{macrocode}
\end{document}
%    \end{macrocode}
%\iffalse
%</samplemain>
%\fi
%
% %%%%%%%%%%%%%%%%%%%%%%%%%%%%%%%%%%%%%%
% \paragraph{Chapter Include Files.}
%
% The include files are called |cdocsch1.tex| and |cdocsch2.tex|.
%
%\iffalse
%<*samplechap1|samplechap2>
%\fi

% Optional override for |\version| flag:
%    \begin{macrocode}
%%\providecommand{\version}{final}
%    \end{macrocode}

% Include the main document:
%    \begin{macrocode}
\input{childdoc.def}
\childdocof{cdocsamp}
%    \end{macrocode}

%\iffalse
%</samplechap1|samplechap2>
%\fi
%
%\iffalse
%<*samplechap1>
%\fi
% Some text for chapter 1:
%    \begin{macrocode}
\section{one}
some text in chapter one
%    \end{macrocode}

%\iffalse
%</samplechap1>
%\fi
% Some text for chapter 2:
%\iffalse
%<*samplechap2>
%\fi
%    \begin{macrocode}
\section{two}
more text in chapter two
%    \end{macrocode}

%\iffalse
%</samplechap2>
%\fi
%
% %%%%%%%%%%%%%%%%%%%%%%%%%%%%%%%%%%%%%%
% \paragraph{Part Include Files.}
%
% The include files are called |cdocspt3.tex| and |cdocspt4.tex|.
%
%\iffalse
%<*samplepart3|samplepart4>
%\fi

% Optional override for |\version| flag:
%    \begin{macrocode}
%%\providecommand{\version}{final}
%    \end{macrocode}

% Include the main document:
%    \begin{macrocode}
\input{childdoc.def}
\childdocby{cdocsamp}
%    \end{macrocode}

%\iffalse
%</samplepart3|samplepart4>
%\fi
%
%\iffalse
%<*samplepart3>
%\fi
% Some text for part 3:
%    \begin{macrocode}
some text in part three
%    \end{macrocode}

%\iffalse
%</samplepart3>
%\fi
% Some text for part 4:
%\iffalse
%<*samplepart4>
%\fi
%    \begin{macrocode}
more text in part four
%    \end{macrocode}

%\iffalse
%</samplepart4>
%\fi
%
% %%%%%%%%%%%%%%%%%%%%%%%%%%%%%%%%%%%%%%
% \paragraph{Forwarding for a Complete Draft.}
%
% The following forwarding file |cdocsdrf.tex|
% compiles the main document in draft mode:
%\iffalse
%<*sampledraft>
%\fi
%    \begin{macrocode}
\def\version{draft}
\input{childdoc.def}
\childdocforward{cdocsamp}
%    \end{macrocode}

%\iffalse
%</sampledraft>
%\fi
%
% %%%%%%%%%%%%%%%%%%%%%%%%%%%%%%%%%%%%%%
% \paragraph{Forwarding for Final Version of the Chapters.}
%
% The following forwarding files |cdocsfn1.tex| and |cdocsfn2.tex|
% (with identical content)
% compile the final versions of the child documents
% |cdocsch1.tex| and |cdocsch2.tex|, respectively:
%\iffalse
%<*samplefinal>
%\fi
%    \begin{macrocode}
\def\version{final}
\input{childdoc.def}
\childdocforwardprefix[cdocsamp]{cdocsfn}{cdocsch}
%    \end{macrocode}

%\iffalse
%</samplefinal>
%\fi
%
% %%%%%%%%%%%%%%%%%%%%%%%%%%%%%%%%%%%%%%
% \paragraph{Command Line Processing.}
%
% The following three command lines generate the output files
% |cdocscld|, |cdocscl1| and |cdocscl2|
% which should be identical to
% |cdocsdrf|, |cdocsch1| and |cdocsfn2|, respectively:
% \begin{center}
% \begin{tabular}{l}
% |latex -jobname cdocscld \|\\
% |  "\def\version{draft}\input{childdoc.def}\childdocforward{cdocsamp}"|\\
% |latex -jobname cdocscl1 \|\\
% |  "\input{childdoc.def}\childdocforward[cdocsamp]{cdocsch1}"|\\
% |latex -jobname cdocscl2 \|\\
% |  "\def\version{final}\input{childdoc.def}\childdocforward{cdocsch2}"|
% \end{tabular}
% \end{center}
% Note that the trailing backslash on each first line
% merely continues the input to the second line
% (for convenient cut ant paste).
% Furthermore, the command |latex| can be replaced by any
% of its alternative versions such as |pdflatex|.
%
% %%%%%%%%%%%%%%%%%%%%%%%%%%%%%%%%%%%%%%%%%%%%%%%%%%%%%%%%%%%%%%%%%%%%%%%%%%%%%%
% %%%%%%%%%%%%%%%%%%%%%%%%%%%%%%%%%%%%%%%%%%%%%%%%%%%%%%%%%%%%%%%%%%%%%%%%%%%%%%
% \section{Implementation}
%\iffalse
%<*package>
%\fi
%
% This section describes the definitions file |childdoc.def|.

% The definitions cannot be loaded using |\usepackage| or |\RequirePackage|
% which has a mechanism to prevent loading a style file more than once.
% When loading the definitions by means of |\input|
% multiple instances have to be prevented manually:
%\iffalse
%This code needs to be before the `\ProvidesFile' directive
%which is defined at the beginning of this file.
%Therefore it is also placed there and commented out here.
%</package>
%<*discard>
%\fi
%    \begin{macrocode}
\ifdefined\childdocmain\endinput\fi
%    \end{macrocode}
%\iffalse
%</discard>
%<*package>
%\fi
%
% \macro{\ifchilddoc}
% \macro{\ifchilddocmanual}
% The conditional |\ifchilddoc| tells whether a
% child (true) or main (false) document is being compiled.
% The conditional |\ifchilddocmanual| tells whether
% the |\includeonly| mechanism is used (false) or
% the selection of child files must be performed manually (true).
% The definitions initialise to false:
%    \begin{macrocode}
\newif\ifchilddoc
\newif\ifchilddocmanual
%    \end{macrocode}

% \macro{\childdocname}
% \macro{\childdocjob}
% The macro |\childdocname| stores the name of the main document
% to be compiled. The macro |\childdocjob| stores the name of
% the document on which the \LaTeX{} compiler was originally invoked.
% The content of |\jobname| cannot be compared
% to filenames specified in the source due to different catcodes.
% The following code rescans |\jobname|, stores the result
% in |\childdocname| and saves a copy in |\childdocjob|:
%    \begin{macrocode}
\edef\childdocname{\scantokens\expandafter{\jobname\noexpand}}
\let\childdocjob\childdocname
%    \end{macrocode}

% \macro{\childdocdisable}
% The macro |\childdocdisable| prevents the main file
% from being processed more than once.
% At this stage, the main document command |\childdocmain|
% is assumed to be called once again where it should do nothing.
% Any subsequent call to it should prevent
% a secondary processing of the main document
% It overwrites the forwarding commands
% |\childdocof| and |\childdocforward|
% with empty macros to prevent further inclusions of the main document:
%    \begin{macrocode}
\newcommand{\childdocdisable}
{
  \renewcommand{\childdocmain}[1]{\renewcommand{\childdocmain}[1]{\endinput}}
  \renewcommand{\childdocof}[1]{}
  \renewcommand{\childdocby}[2][]{}
  \renewcommand{\childdocforward}[2][]{}
  \renewcommand{\childdocdisable}{}
}
%    \end{macrocode}

% \macro{\childdocmain}
% The macro |\childdocmain| is to be called at the top of the main file
% with nothing or the main filename (without extension) as argument.
% First, it breaks loops.
% If the argument is not empty and does not match |\childdocname|
% (which is set by the first inclusion of |childdoc.def|),
% |\ifchilddoc| is set to true, |\includeonly| is applied to the child file
% and |\jobname| is set to the main file
% (for proper handling of |.aux| files):
%    \begin{macrocode}
\newcommand{\childdocmain}[1]
{
  \childdocdisable\childdocmain{}
  \if?#1?\else
    \begingroup
      \def\childdoctmp{#1}
      \ifx\childdoctmp\childdocname
        \def\childdoctmp{}
      \else
        \def\childdoctmp
        {
          \childdoctrue
          \includeonly{\childdocname}
          \def\childdocjob{#1}
          \def\jobname{#1}
        }
      \fi
      \expandafter
    \endgroup
    \childdoctmp
  \fi
}
%    \end{macrocode}

% \macro{\childdocof}
% The command |\childdocof| redirects
% compilation to the main file |#1|.
%    \begin{macrocode}
\newcommand{\childdocof}[1]
{
  \childdocdisable
  \childdoctrue
  \includeonly{\childdocname}
  \def\jobname{#1}
  \def\childdocjob{#1}
  \input{#1}
}
%    \end{macrocode}

% \macro{\childdocby}
% The command |\childdocby| ....
%    \begin{macrocode}
\newcommand{\childdocby}[2][]
{
  \childdocdisable
  \childdoctrue
  \childdocmanualtrue
  \if?#1?\else
    \def\jobname{#2}
  \fi
  \def\childdocjob{#2}
  \input{#2}
  \endinput
}
%    \end{macrocode}

% \macro{\childdocforward}
% The command |\childdocforward| redirects
% compilation to the main file or
% (if the optional argument is given) a child file.
% Parameters are set as if the main file
% or a child file starting with |\childdocof| was compiled.
% Then compilation is handed over to the main file:
%    \begin{macrocode}
\newcommand{\childdocforward}[2][]
{
  \begingroup
    \if?#1?
      \def\childdoctmp
      {
        \def\childdocname{#2}
        \def\childdocjob{#2}
        \def\jobname{#2}
        \input{#2}
        \endinput
      }
    \else
      \def\childdoctmp
      {
        \childdocdisable
        \def\childdocname{#2}
        \childdoctrue
        \includeonly{#2}
        \def\childdocjob{#1}
        \def\jobname{#1}
        \input{#1}
        \endinput
      }
    \fi
    \expandafter
  \endgroup
  \childdoctmp
}
%    \end{macrocode}

% \macro{\childdocforwardprefix}
% The command |\childdocforwardprefix| redirects
% compilation to the main or a child file by means of a pattern.
% The prefix |#1| in the current filename is replaced by |#2|
% and the suffix of the current filename is kept
% (it is assumed that the filename does not contain the substring `|~~~|'
% which is used as a delimiter).
% Compilation is handed over to the new file by |\childdocforward|:
%    \begin{macrocode}
\newcommand{\childdocforwardprefix}[3][]
{
  \begingroup
    \def\childdocextract #2##1~~~{\def\childdoctmp{\childdocforward[#1]{#3##1}}}
    \expandafter\childdocextract\childdocname~~~
    \expandafter
  \endgroup
  \childdoctmp
}
%    \end{macrocode}

% \macro{\childdoc}
% The deprecated macro |\childdoc| is a legacy version of |\childdocmain|:
%    \begin{macrocode}
\newcommand{\childdoc}{\childdocmain}
%    \end{macrocode}

% \macro{\childdocredirect}
% The deprecated macro |\childdocredirect| is a legacy version
% of |\childdocforward| and |\childdocforwardprefix|:
%    \begin{macrocode}
\newcommand{\childdocredirect}[2][]
{
  \begingroup
    \if?#1?
      \def\childdoctmp{\childdocforward{#2}}
    \else
      \def\childdoctmp{\childdocforwardprefix{#1}{#2}}
    \fi
    \expandafter
  \endgroup
  \childdoctmp
}
%    \end{macrocode}

%\iffalse
%</package>
%\fi
%
\endinput
\childdocforward[|\textit{main}|]{|\textit{dest}|}"|
\end{center}
%
Here \textit{target} is the name of the output file,
\textit{main} is the name of the main file
and \textit{dest} is the name of the main or child file to be processed
(all filenames without extensions).
The optional argument \textit{main} can be omitted
if \textit{main} matches \textit{dest}.
Optionally, compilation \textit{flags} can be defined via |\def| commands.
This command line makes the \TeX{} engine believe
it is compiling the file \textit{target}
whose content is specified as the latter parameter.
The provided code then forwards the processing to
\textit{main} or \textit{dest} as described in \secref{sec:forward}.

%%%%%%%%%%%%%%%%%%%%%%%%%%%%%%%%%%%%%%%%%%%%%%%%%%%%%%%%%%%%%%%%%%%%%%%%%%%%%%%%
\subsection{Include by Input}
\label{sec:input}

Including child documents by |\include| has some restrictions by design.
Most notably, the content of a child document always occupies
its own set of pages; pages cannot be shared between child documents.
Usually, this behaviour makes perfect sense
because each child document contain an essential part of the document.
However, in some situations it may be desirable to compose
a document from a collection of parts
without having mandatory page breaks between then.
For this case, the package
provides a mechanism to include parts
by |\input| which can also be processed individually.
However, by construction this mechanism
requires manual handling of the content to be output.

%%%%%%%%%%%%%%%%%%%%%%%%%%%%%%%%%%%%%%%%
\DescribeMacro{\ifchilddocmanual}
The main file should be prepared as usual, see \secref{sec:include}.
However, the document body must make a distinction
between processing of an individual part and of the main document, e.g.:
%
\begin{center}
\begin{tabular}{l}
|\ifchilddocmanual|\\
|\input{\childdocname}|\\
|\||else|\\
\textit{document body with }|\input{|\textit{part}|}|\\
|\||fi|
\end{tabular}
\end{center}
%
The conditional |\ifchilddocmanual| is true whenever
a part to be included by |\input| is being compiled,
and the name of the part is stored in |\childdocname|.

%%%%%%%%%%%%%%%%%%%%%%%%%%%%%%%%%%%%%%%%
\DescribeMacro{\childdocby}
Each part to be included by |\input| should start with:
%
\begin{center}
\begin{tabular}{l}
|% \iffalse
%
% childdoc.dtx Copyright (C) 2017-2018 Niklas Beisert
%
% This work may be distributed and/or modified under the
% conditions of the LaTeX Project Public License, either version 1.3
% of this license or (at your option) any later version.
% The latest version of this license is in
%   http://www.latex-project.org/lppl.txt
% and version 1.3 or later is part of all distributions of LaTeX
% version 2005/12/01 or later.
%
% This work has the LPPL maintenance status `maintained'.
%
% The Current Maintainer of this work is Niklas Beisert.
%
% This work consists of the files childdoc.dtx and childdoc.ins
% and the derived files childdoc.def and cdocsamp.tex with
% cdocsch1.tex, cdocsch2.tex, cdocsdrf.tex, cdocsfn1.tex, cdocsfn2.tex.
%
%<package>\ifdefined\childdocmain\endinput\fi
%<package>\ProvidesFile{childdoc.def}[2018/12/30 v2.0 child document driver]
%<samplemain>\ProvidesFile{cdocsamp.tex}[2018/12/30 v2.0 sample for childdoc]
%<*driver>
%\ProvidesFile{childdoc.drv}[2018/12/30 v2.0 childdoc reference manual file]
\PassOptionsToClass{10pt,a4paper}{article}
\documentclass{ltxdoc}

\usepackage[margin=35mm]{geometry}
\usepackage{hyperref}
\usepackage{hyperxmp}
\usepackage[usenames]{color}

\hypersetup{colorlinks=true}
\hypersetup{pdfstartview=FitH}
\hypersetup{pdfpagemode=UseNone}
\hypersetup{pdfsource={}}
\hypersetup{pdflang={en-UK}}
\hypersetup{pdfcopyright={Copyright 2017-2018 Niklas Beisert.
  This work may be distributed and/or modified under the
  conditions of the LaTeX Project Public License, either version 1.3
  of this license or (at your option) any later version.}}
\hypersetup{pdflicenseurl={http://www.latex-project.org/lppl.txt}}
\hypersetup{pdfcontactaddress={ETH Zurich, ITP, HIT K,
  Wolfgang-Pauli-Strasse 27}}
\hypersetup{pdfcontactpostcode={8093}}
\hypersetup{pdfcontactcity={Zurich}}
\hypersetup{pdfcontactcountry={Switzerland}}
\hypersetup{pdfcontactemail={nbeisert@itp.phys.ethz.ch}}
\hypersetup{pdfcontacturl={http://people.phys.ethz.ch/\xmptilde nbeisert/}}

\newcommand{\secref}[1]{\hyperref[#1]{section \ref*{#1}}}

\parskip1ex
\parindent0pt
\let\olditemize\itemize
\def\itemize{\olditemize\parskip0pt}

\begin{document}

\title{The \textsf{childdoc} Package}
\hypersetup{pdftitle={The childdoc Package}}
\author{Niklas Beisert\\[2ex]
  Institut f\"ur Theoretische Physik\\
  Eidgen\"ossische Technische Hochschule Z\"urich\\
  Wolfgang-Pauli-Strasse 27, 8093 Z\"urich, Switzerland\\[1ex]
  \href{mailto:nbeisert@itp.phys.ethz.ch}
  {\texttt{nbeisert@itp.phys.ethz.ch}}}
\hypersetup{pdfauthor={Niklas Beisert}}
\hypersetup{pdfsubject={Manual for the LaTeX2e Package childdoc}}
\date{30 December 2018, \textsf{v2.0}}
\maketitle

\begin{abstract}\noindent
\textsf{childdoc} is a \LaTeXe{} package
that enables the direct compilation
of document sections included by |\include|
to individual files.
\end{abstract}

\begingroup
\parskip0ex
\tableofcontents
\endgroup

%%%%%%%%%%%%%%%%%%%%%%%%%%%%%%%%%%%%%%%%%%%%%%%%%%%%%%%%%%%%%%%%%%%%%%%%%%%%%%%%
%%%%%%%%%%%%%%%%%%%%%%%%%%%%%%%%%%%%%%%%%%%%%%%%%%%%%%%%%%%%%%%%%%%%%%%%%%%%%%%%
\section{Introduction}

\LaTeX{} provides a mechanism to structure a large document (such as a book)
into a main file and several child files (containing the chapters)
using the |\include| command.
This mechanism is beneficial for documents
which span hundreds of pages in order to
make the source file(s) more manageable.
Moreover, compilation can be restricted to
selected child files by means of the |\includeonly| command.
The latter feature can be used to reduce the compilation time while editing
(this was significantly more useful in the earlier days of \LaTeX{})
or to generate a smaller document which is easier to navigate.
Another application of |\includeonly| is to generate
documents consisting of selected parts of the complete document.

However, there are a few drawbacks of the plain |\include| mechanism:
\begin{itemize}
\item
The child files cannot be compiled on their own,
they can only be compiled via the main file.
A naive editing environment
(such as a text editor with an option
to have the current file processed by \LaTeX)
may require one to switch to the main file before compiling;
attempting to compile the child file produces errors.
\item
The main file must be modified (each time)
to adjust the |\includeonly| command
to the present needs. This easily leaves the main file in a messy state.
\item
The generated document will always carry the filename
of the main document. This is inconvenient if
several child files are to be compiled and
to be kept for distribution.
\end{itemize}

The present package provides a simple interface
to make child files individually compilable by \LaTeX{}.
Compiling a child file then has the same effect as compiling
the main file with an |\includeonly| command
to select the appropriate child.
Moreover the generated document will carry the name of the child
rather than the main file.
This resolves all three above issues.

This feature is meant to make the editing of books,
thesis documents and lecture notes somewhat more convenient.
However, the package can also be used efficiently for
composing a series of documents (such as exercise sheets)
which are typically distributed individually.
It then assists the author in generating the individual documents
(potentially in different versions)
as well as a document containing the collected series.
Another application is in developing style files
or other kinds of included material
where compilation of the style file could redirect
to a sample or test file.

%%%%%%%%%%%%%%%%%%%%%%%%%%%%%%%%%%%%%%%%%%%%%%%%%%%%%%%%%%%%%%%%%%%%%%%%%%%%%%%%
%%%%%%%%%%%%%%%%%%%%%%%%%%%%%%%%%%%%%%%%%%%%%%%%%%%%%%%%%%%%%%%%%%%%%%%%%%%%%%%%
\section{Usage}

First of all, the package \textsf{childdoc} is \emph{not} a standard
\LaTeXe{} |.sty| style file! Therefore it needs to be invoked in
a non-standard way.

%%%%%%%%%%%%%%%%%%%%%%%%%%%%%%%%%%%%%%%%%%%%%%%%%%%%%%%%%%%%%%%%%%%%%%%%%%%%%%%%
\subsection{Included Files}
\label{sec:include}

%%%%%%%%%%%%%%%%%%%%%%%%%%%%%%%%%%%%%%%%
\DescribeMacro{\childdocmain}
To use the package, add the commands
\begin{center}
\begin{tabular}{l}
|\input{childdoc.def}|\\
|\childdocmain{}|\\
\end{tabular}
\end{center}
at the very top of the main \LaTeX{} file,
in particular \emph{before} the |\documentclass| statement!
The argument of |\childdocmain| should be left empty
(but it must be present).

%%%%%%%%%%%%%%%%%%%%%%%%%%%%%%%%%%%%%%%%
\DescribeMacro{\childdocof}
Furthermore, add the commands
\begin{center}
\begin{tabular}{l}
|\input{childdoc.def}|\\
|\childdocof{|\textit{main}|}|\\
\end{tabular}
\end{center}
at the top of every child file \textit{child}
which is included by |\include{|\textit{child}|}|
from within the main file
(or at least for those files to be compiled individually).
The argument \textit{main} must be the filename of the main file.

There are a couple of
considerations in setting up the main and child documents:

%%%%%%%%%%%%%%%%%%%%%%%%%%%%%%%%%%%%%%%%
\paragraph{Restrictions.}

Please note the following restrictions:
\begin{itemize}
\item
|\childdocmain| must be called with one argument \textit{main}
to ensure compatibility with earlier version of the package.
It must either be empty (|\childdocmain{}|)
or precisely match the filename of the main file in which it is specified.
See \secref{sec:detection} for further information.
\item
The filename \textit{main} must be specified without the |.tex| extension.
\item
The filename \textit{main} is case sensitive
(even in case-insensitive file systems)
due to internal string comparison.
\item
The argument \textit{main} should be fully expanded, it cannot be a macro.
\item
Subdirectories and special characters should be avoided in filenames.
\item
The command |\childdocmain{|\textit{main}|}| must be followed by a whitespace.
It should not be followed immediately by another command
or by a comment mark `|%|'.
This is because the \TeX{} parser reads the token immediately following
the argument of |\childdocmain| and puts it
at the beginning of every child section;
however, a white\-space is ignored.
\end{itemize}

%%%%%%%%%%%%%%%%%%%%%%%%%%%%%%%%%%%%%%%%
\paragraph{Content of Main File.}

It is advisable to place all content in the child files included by |\include|.
Any output contained in the main file will appear in all child documents
unless suppressed manually;
it cannot be suppressed automatically by the |\includeonly| directive
and thus should normally be avoided.
A method to include some content in the main file
by means of conditional processing is described in \secref{sec:conditional}.

%%%%%%%%%%%%%%%%%%%%%%%%%%%%%%%%%%%%%%%%
\paragraph{Page Numbering.}

When only a part of the document is compiled,
the appropriate numbering of pages
(as well as other status parameters)
is determined from the |.aux| files.
The latter contain information from previous passes.
However this information needs to propagate through
all intermediate child documents.
Therefore the page numbering in child documents may well
be inconsistent until the complete document is compiled at least once.

A useful (if unconventional) way to always ensure a consistent
page numbering is to restart the numbering in each child document
and denote the pages by `\textit{child}|.|\textit{page}'
where \textit{child} represents the chapter/section number of the child file.
This can be achieved by the command
|\numberwithin{page}{|\textit{child}|}|
of the \textsf{amsmath} package
where \textit{child} can be |chapter| or |section|
depending on the chosen structuring.
Alternatively, one can modify the macro |\thepage| appropriately
and reset the counter |page| at the start of each child file.

%%%%%%%%%%%%%%%%%%%%%%%%%%%%%%%%%%%%%%%%%%%%%%%%%%%%%%%%%%%%%%%%%%%%%%%%%%%%%%%%
\subsection{Conditional Processing}
\label{sec:conditional}

The package provides a mechanism to compile different versions
of a document. To customise the versions further some conditional processing
can come in handy to distinguish which version is being compiled.
The package provides two macros to describe the compilation context:

%%%%%%%%%%%%%%%%%%%%%%%%%%%%%%%%%%%%%%%%
\DescribeMacro{\ifchilddoc}
The conditional |\ifchilddoc| distinguishes between the compilation of
child documents and the main document:
%
\begin{center}
|\ifchilddoc |\textit{child-code}| |[|\||else |\textit{main-code}]| \||fi|
\end{center}

%%%%%%%%%%%%%%%%%%%%%%%%%%%%%%%%%%%%%%%%
\DescribeMacro{\childdocname}
\DescribeMacro{\childdocjob}
The macro |\childdocname| contains the filename (without extension)
of the main or child file being processed.
Note that |\childdocjob| will always contain the name of the main file.

%%%%%%%%%%%%%%%%%%%%%%%%%%%%%%%%%%%%%%%%
\paragraph{Title Page.}

Conditional processing can be used to include a title or banner page
in the main document when proper precautions are taken.
Importantly, the code in the main file should ensure that the page counter
(as well as other status parameters which are stored in the |.aux| files)
takes the same value after the conditional processing.
Otherwise the page numbers may take divergent values
depending on which part is compiled.

For example, a title page could be declared by:
%
\begin{center}
\begin{tabular}{l}
|\ifchilddoc\||else|\\
|\addtocounter{page}{-1}|\\
\textit{code for title page}\\
|\newpage|\\
|\||fi|
\end{tabular}
\end{center}
%
A banner page for the child documents can be generated by:
%
\begin{center}
\begin{tabular}{l}
|\ifchilddoc|\\
|\addtocounter{page}{-1}|\\
\textit{code for banner page}\\
|\newpage|\\
|\||fi|
\end{tabular}
\end{center}
%
Here one could write a message such as:
\begin{center}
|This is the part \childdocname{} of \childdocjob{}.|
\end{center}

%%%%%%%%%%%%%%%%%%%%%%%%%%%%%%%%%%%%%%%%%%%%%%%%%%%%%%%%%%%%%%%%%%%%%%%%%%%%%%%%
\subsection{Flags}
\label{sec:flags}

The package makes it easy to generate different versions
of the main or child documents.
To this end compilation flags can be defined
and assigned different default values.
They will be particularly useful in conjunction
with the forwarding mechanism described in \secref{sec:forward}.

For example, it may be useful to have a flag |\version|
which can be set to |draft| or |final|.
The document source will contain some conditional code
depending on the value of |\version|.
Suppose further, the flag should default to |final| for the main file
and to |draft| for child files
which is a natural assignment for editing the document.
This is achieved by placing the following code
in the preamble of the main document
(below the |\childdocmain| directive):
%
\begin{center}
\begin{tabular}{l}
|\ifchilddoc|\\
|\providecommand{\version}{draft}|\\
|\||else|\\
|\providecommand{\version}{final}|\\
|\||fi|
\end{tabular}
\end{center}
%
The definition by |\providecommand| makes sure
that previous definitions are not overwritten.
Further statements |\providecommand{\version}{...}|
can thus be added before the above code to override it.

For the main file, one might add a line
(between |\childdocmain| and the above block)
%
\begin{center}
|%\ifchilddoc\||else\providecommand{\version}{draft}\||fi|
\end{center}
%
which can be uncommented to produce a draft version.
Likewise one can add a line to the very top of a child file
(above the |\childdocof{|\textit{main}|}| directive)
%
\begin{center}
|%\providecommand{\version}{final}|
\end{center}
%
which can be uncommented to produce the final version of this child document.

%%%%%%%%%%%%%%%%%%%%%%%%%%%%%%%%%%%%%%%%%%%%%%%%%%%%%%%%%%%%%%%%%%%%%%%%%%%%%%%%
\subsection{Forwarding}
\label{sec:forward}

Different versions of the main or child documents
using compilation flags as described in \secref{sec:flags}
can be (permanently) stored in different files
for convenient compilation, viewing and distribution.
To this end, the package defines a command
to pass on compilation to a different file:

%%%%%%%%%%%%%%%%%%%%%%%%%%%%%%%%%%%%%%%%
\DescribeMacro{\childdocforward}
The command |\childdocforward| redirects processing to
another source file:
%
\begin{center}
\begin{tabular}{l}
|\input{childdoc.def}|\\
|\childdocforward[|\textit{main}|]{|\textit{dest}|}|\\
\end{tabular}
\end{center}
%
The argument \textit{dest} is the destination file
(without extension).
It should be the main file or one of the child files.
Note that further \textsf{childdoc} directives
such as |\childdocof| and |\childdocforward|
in the indicated file will be processed in this form.
The optional argument \textit{main}
passes on directly to the main file \textit{main}
while pretending to compile the child \textit{dest}.
This form behaves as if \textit{dest}
issues |\childdocof{|\textit{main}|}| right away,
and no further \textsf{childdoc} directives will be processed.

%%%%%%%%%%%%%%%%%%%%%%%%%%%%%%%%%%%%%%%%
\DescribeMacro{\...prefix}
In the alternative form |\childdocforwardprefix|,
%
\begin{center}
\begin{tabular}{l}
|\input{childdoc.def}|\\
|\childdocforwardprefix[|\textit{main}|]{|\textit{prefix}|}{|\textit{dest}|}|
\end{tabular}
\end{center}
%
the destination file is determined by a pattern
depending on the current file:
To make this work, the current file must be called
`{\textit{prefix}\hspace{0.2em}\textit{suffix}}'
with \textit{prefix} matching precisely the argument.
Processing is then passed on to the file
`{\textit{dest}\hspace{0.2em}\textit{suffix}}'.
Surely, the same effect is achieved by
directly specifying the
argument `{\textit{dest}\hspace{0.2em}\textit{suffix}}'
in the first form.
However, that requires to set up a different file
for each child. With the alternative form of the command
all these files can have exactly the same content
which simplifies setting them up and maintaining them.

For example, the following file |draft.tex|
with a compilation flag |\version| as described in \secref{sec:flags}
compiles the main document as a draft:
%
\begin{center}
\begin{tabular}{l}
|\def\version{draft}|\\
|\input{childdoc.def}|\\
|\childdocforward{|\textit{main}|}|
\end{tabular}
\end{center}
%
Likewise, the following files |final|\textit{nn}|.tex|
compile the final version of the child document
|child|\textit{nn}|.tex|:
%
\begin{center}
\begin{tabular}{l}
|\def\version{final}|\\
|\input{childdoc.def}|\\
|\childdocforwardprefix{final}{child}|
\end{tabular}
\end{center}
%

Note that when several versions of a main file and/or of each child file
are to be generated, it may be convenient to set up a |Makefile| or
shell script to automatise the process.

%%%%%%%%%%%%%%%%%%%%%%%%%%%%%%%%%%%%%%%%%%%%%%%%%%%%%%%%%%%%%%%%%%%%%%%%%%%%%%%%
\subsection{Command Line Processing}
\label{sec:commandline}

The effect of redirection files can also be achieved by invoking
the \LaTeX{} compiler with a more elaborate command line.
Most conveniently this should be done as part
of a shell script or a |Makefile|.

When using \textsf{childdoc} in the main file, the following
command lines effectively perform a redirection
(note that depending on the shell being used,
backslashes may have to be doubled: `|\|' $\to$ `|\\|'):
%
\begin{center}
|... -jobname "|\textit{target}|" |\\|"|[\textit{flags}]%
|\input{childdoc.def}\childdocforward[|\textit{main}|]{|\textit{dest}|}"|
\end{center}
%
Here \textit{target} is the name of the output file,
\textit{main} is the name of the main file
and \textit{dest} is the name of the main or child file to be processed
(all filenames without extensions).
The optional argument \textit{main} can be omitted
if \textit{main} matches \textit{dest}.
Optionally, compilation \textit{flags} can be defined via |\def| commands.
This command line makes the \TeX{} engine believe
it is compiling the file \textit{target}
whose content is specified as the latter parameter.
The provided code then forwards the processing to
\textit{main} or \textit{dest} as described in \secref{sec:forward}.

%%%%%%%%%%%%%%%%%%%%%%%%%%%%%%%%%%%%%%%%%%%%%%%%%%%%%%%%%%%%%%%%%%%%%%%%%%%%%%%%
\subsection{Include by Input}
\label{sec:input}

Including child documents by |\include| has some restrictions by design.
Most notably, the content of a child document always occupies
its own set of pages; pages cannot be shared between child documents.
Usually, this behaviour makes perfect sense
because each child document contain an essential part of the document.
However, in some situations it may be desirable to compose
a document from a collection of parts
without having mandatory page breaks between then.
For this case, the package
provides a mechanism to include parts
by |\input| which can also be processed individually.
However, by construction this mechanism
requires manual handling of the content to be output.

%%%%%%%%%%%%%%%%%%%%%%%%%%%%%%%%%%%%%%%%
\DescribeMacro{\ifchilddocmanual}
The main file should be prepared as usual, see \secref{sec:include}.
However, the document body must make a distinction
between processing of an individual part and of the main document, e.g.:
%
\begin{center}
\begin{tabular}{l}
|\ifchilddocmanual|\\
|\input{\childdocname}|\\
|\||else|\\
\textit{document body with }|\input{|\textit{part}|}|\\
|\||fi|
\end{tabular}
\end{center}
%
The conditional |\ifchilddocmanual| is true whenever
a part to be included by |\input| is being compiled,
and the name of the part is stored in |\childdocname|.

%%%%%%%%%%%%%%%%%%%%%%%%%%%%%%%%%%%%%%%%
\DescribeMacro{\childdocby}
Each part to be included by |\input| should start with:
%
\begin{center}
\begin{tabular}{l}
|\input{childdoc.def}|\\
|\childdocby{|\textit{main}|}|\\
\end{tabular}
\end{center}
%
The directive |\childdocby| is similar to |\childdocof|
described in \secref{sec:include},
but the subsequent selection of content must be done manually.
To that end, both |\ifchilddoc| and |\ifchilddocmanual|
will be true upon processing of a part,
and the name of the part is stored in |\childdocname|.
Note that |\jobname| will be set to the filename of the current part
so that each part receives an individual |.aux| file
that does not interfere with the |.aux| file(s) of the main document.
This behaviour can be altered by the alternative form
|\childdocby[*]{|\textit{main}|}| (with a non-empty optional argument)
which uses the |.aux| file of the main document
by setting |\jobname| to \textit{main}.

%%%%%%%%%%%%%%%%%%%%%%%%%%%%%%%%%%%%%%%%%%%%%%%%%%%%%%%%%%%%%%%%%%%%%%%%%%%%%%%%
\subsection{Driver Development}
\label{sec:driver}

The \textsf{childdoc} mechanism can also be use for the development
of definition files such as \LaTeX{} styles or classes.
This case differs from the above setup with multiple parts
included by |\include| in that no |\includeonly| should be invoked.
This can be achieved by starting the include file
(before |\ProvidesPackage|) with:
%
\begin{center}
\begin{tabular}{l}
|\input{childdoc.def}|\\
|\childdocforward{|\textit{main}|}|\\
\end{tabular}
\end{center}
%
or alternatively with:
%
\begin{center}
\begin{tabular}{l}
|\input{childdoc.def}|\\
|\childdocby{|\textit{main}|}|\\
\end{tabular}
\end{center}
%
Both forms have slightly different effects as described above.
The main file is prepared as usual, see \secref{sec:include}.

%%%%%%%%%%%%%%%%%%%%%%%%%%%%%%%%%%%%%%%%%%%%%%%%%%%%%%%%%%%%%%%%%%%%%%%%%%%%%%%%
\subsection{Legacy Detection}
\label{sec:detection}

The directive |\childdocmain| in the main file can detect
whether the complete document or merely a child is to be compiled
even without using the directive |\childdocof|.
This method is deprecated because it is less robust
and there is no compelling reason to use it;
it is merely provided for backward compatibility
and it may be removed in future versions.

If the detection mechanism is to be used,
it is mandatory to correctly specify
the filename of the main file as the argument of |\childdocmain|:
%
\begin{center}
\begin{tabular}{l}
|\input{childdoc.def}|\\
|\childdocmain{|\textit{main}|}|\\
\end{tabular}
\end{center}
%
If |\jobname| does not match the argument \textit{main} of |\childdocmain|,
it is assumed that |\jobname| points to the child file to be compiled.
When using |\childdocmain| with the main file specified as argument,
it suffices to start a child file
with just |\input{|\textit{main}|}|
without loading of the package and using |\childdocof|.
If instead all processing is done
with the appropriate \textsf{childdoc} directives,
the argument of \textit{main} of |\childdocmain| can be empty.

An alternative version of the command line processing described
in \secref{sec:commandline} using the detection mechanism reads:
%
\begin{center}
|... -jobname "|\textit{target}|" "|[\textit{flags}]%
[|\def\jobname{|\textit{dest}|}|]|\input{|\textit{main}|}"|
\end{center}

%%%%%%%%%%%%%%%%%%%%%%%%%%%%%%%%%%%%%%%%%%%%%%%%%%%%%%%%%%%%%%%%%%%%%%%%%%%%%%%%
\subsection{Manual Code}
\label{sec:manual}

In case one cannot be certain whether the definitions file |childdoc.def|
is installed on the target \TeX{} distribution
and one prefers not to ship it,
it is conceivable to paste a few relevant commands into the sources.

To that end, drop all statements |\input{childdoc.def}|
and perform the replacements as outlined below.
Instead of |\childdocmain{|\textit{main}|}| add the following code
to the top of the main file:
%
\begin{center}
\begin{tabular}{l}
|\||ifdefined\childdocname\endinput\||fi\newif\ifchilddoc|\\
|\edef\childdocname{\scantokens\expandafter{\jobname\noexpand}}|\\
|\def\childdocmain{|\textit{main}|}\||ifx\childdocmain\childdocname\||else|\\
|\childdoctrue\includeonly{\childdocname}\let\jobname\childdocmain\||fi|\\
\end{tabular}
\end{center}
%
Instead of |\childdocof{|\textit{main}|}| just include the main file
at the top of each child file:
%
\begin{center}
|\input{|\textit{main}|}|
\end{center}
%
A simple redirection |\childdocforward{|\textit{dest}|}| is achieved by:
%
\begin{center}
|\def\jobname{|\textit{dest}|}\input{\jobname}|
\end{center}
%
The redirection with prefix
|\childdocforwardprefix[|\textit{prefix}|]{|\textit{dest}|}|
is accomplished by:
%
\begin{center}
\begin{tabular}{l}
|{\edef\jobname{\scantokens\expandafter{\jobname\noexpand}}|\\
|\def\redirectjob |\textit{prefix}|#1~~~{\gdef\jobname{|\textit{dest}|#1}}|\\
|\expandafter\redirectjob\jobname~~~}\input{\jobname}|
\end{tabular}
\end{center}

In an alternative approach,
child documents can be compiled by a specific command line
without additional code or specific definitions:
%
\begin{center}
|... -jobname "|\textit{target}|" "|[\textit{flags}]%
|\includeonly{|\textit{dest}|}\input{|\textit{main}|}"|
\end{center}
%

%%%%%%%%%%%%%%%%%%%%%%%%%%%%%%%%%%%%%%%%%%%%%%%%%%%%%%%%%%%%%%%%%%%%%%%%%%%%%%%%
%%%%%%%%%%%%%%%%%%%%%%%%%%%%%%%%%%%%%%%%%%%%%%%%%%%%%%%%%%%%%%%%%%%%%%%%%%%%%%%%
\section{Information}

%%%%%%%%%%%%%%%%%%%%%%%%%%%%%%%%%%%%%%%%%%%%%%%%%%%%%%%%%%%%%%%%%%%%%%%%%%%%%%%%
\subsection{Copyright}

Copyright \copyright{} 2017--2018 Niklas Beisert

This work may be distributed and/or modified under the
conditions of the \LaTeX{} Project Public License, either version 1.3
of this license or (at your option) any later version.
The latest version of this license is in
  \url{http://www.latex-project.org/lppl.txt}
and version 1.3 or later is part of all distributions of \LaTeX{}
version 2005/12/01 or later.

This work has the LPPL maintenance status `maintained'.

The Current Maintainer of this work is Niklas Beisert.

This work consists of the files |README.txt|, |childdoc.ins| and |childdoc.dtx|
as well as the derived files |childdoc.def|, |cdocsamp.tex|
with |cdocsch1.tex|, |cdocsch2.tex|, |cdocspt3.tex|, |cdocspt4.tex|,
|cdocsdrf.tex|, |cdocsfn1.tex|, |cdocsfn2.tex|
as well as |childdoc.pdf|.

%%%%%%%%%%%%%%%%%%%%%%%%%%%%%%%%%%%%%%%%%%%%%%%%%%%%%%%%%%%%%%%%%%%%%%%%%%%%%%%%
\subsection{Files and Installation}

The package consists of the files:
%
\begin{center}
\begin{tabular}{ll}
    |README.txt|   & readme file \\
    |childdoc.ins| & installation file \\
    |childdoc.dtx| & source file \\
    |childdoc.def| & definition file \\
    |cdocsamp.tex| & sample main file \\
    |cdocsch1.tex| & sample include file \\
    |cdocsch2.tex| & sample include file \\
    |cdocspt3.tex| & sample part file \\
    |cdocspt4.tex| & sample part file \\
    |cdocsdrf.tex| & sample redirection file \\
    |cdocsfn1.tex| & sample redirection file \\
    |cdocsfn2.tex| & sample redirection file \\
    |childdoc.pdf| & manual
\end{tabular}
\end{center}
%
The distribution consists of the files
|README.txt|, |childdoc.ins| and |childdoc.dtx|.
%
\begin{itemize}
\item
Run (pdf)\LaTeX{} on |childdoc.dtx|
to compile the manual |childdoc.pdf| (this file).
\item
Run \LaTeX{} on |childdoc.ins| to create the definitions file |childdoc.def|
and the sample |cdocsamp.tex| with include files
|cdocsch1.tex|, |cdocsch2.tex|, |cdocspt3.tex|, |cdocspt4.tex|,
|cdocsdrf.tex|, |cdocsfn1.tex|, |cdocsfn2.tex|.
Then copy the file |childdoc.def| to an appropriate directory of your \LaTeX{}
distribution, e.g.\ \textit{texmf-root}|/tex/latex/childdoc|.
\end{itemize}

%%%%%%%%%%%%%%%%%%%%%%%%%%%%%%%%%%%%%%%%%%%%%%%%%%%%%%%%%%%%%%%%%%%%%%%%%%%%%%%%
\subsection{Related CTAN Packages}

There are several other packages which offer a similar functionality:
%
\begin{itemize}
\item
The packages
\href{http://ctan.org/pkg/docmute}{\textsf{docmute}},
\href{http://ctan.org/pkg/includex}{\textsf{includex}} and
\href{http://ctan.org/pkg/standalone}{\textsf{standalone}}
provide commands to include only the document body of
a child file thus allowing both files to be compiled individually.
\item
The packages \href{http://ctan.org/pkg/subdocs}{\textsf{subdocs}}
and \href{http://ctan.org/pkg/subfiles}{\textsf{subfiles}}
provide structures in which the main and child documents can be
encapsulated and allowing them to be compiled individually.
The inclusion mechanism is different from the conventional |\include|.
\item
The package \href{http://ctan.org/pkg/combine}{\textsf{combine}}
is an elaborate solution to combine several documents into one.
\end{itemize}
%
See also the CTAN topic \href{http://ctan.org/topic/subdocs}{\textsf{subdocs}}
for further related packages.
The present package differs from the above solutions in that
a document structure constructed with the conventional |\include| mechanism
just needs two extra commands at the top of every file
such that all constituent files can be compiled individually.

%%%%%%%%%%%%%%%%%%%%%%%%%%%%%%%%%%%%%%%%%%%%%%%%%%%%%%%%%%%%%%%%%%%%%%%%%%%%%%%%
%\subsection{Feature Suggestions}
%
%The following is a list of features which may be useful for future
%versions of this package:
%%
%\begin{itemize}
%\item
%\ldots
%\end{itemize}

%%%%%%%%%%%%%%%%%%%%%%%%%%%%%%%%%%%%%%%%%%%%%%%%%%%%%%%%%%%%%%%%%%%%%%%%%%%%%%%%
\subsection{Revision History}

%%%%%%%%%%%%%%%%%%%%%%%%%%%%%%%%%%%%%%%%
\paragraph{v2.0:} 2018/12/30

\begin{itemize}
\item
immediate forward processing
\item
added |\childdocby| mechanism
\item
manual restructured
\end{itemize}

%%%%%%%%%%%%%%%%%%%%%%%%%%%%%%%%%%%%%%%%
\paragraph{v1.6:} 2018/01/17

\begin{itemize}
\item
application for development of include files
\item
corrections to manual
\end{itemize}

%%%%%%%%%%%%%%%%%%%%%%%%%%%%%%%%%%%%%%%%
\paragraph{v1.5:} 2017/05/21

\begin{itemize}
\item
more complete structuring introduced
\item
|\childdocof| introduced
\item
|\childdoc| renamed to |\childdocmain|
\item
|\childredirect| renamed to |\childdocforward| and |\childdocforwardprefix|
and functionality expanded
\end{itemize}

%%%%%%%%%%%%%%%%%%%%%%%%%%%%%%%%%%%%%%%%
\paragraph{v1.0:} 2017/04/27

\begin{itemize}
\item
manual and install package
\item
first version published on CTAN
\end{itemize}

%%%%%%%%%%%%%%%%%%%%%%%%%%%%%%%%%%%%%%%%
\paragraph{v0.6:} 2017/04/26

\begin{itemize}
\item
redirection mechanism added
\end{itemize}

%%%%%%%%%%%%%%%%%%%%%%%%%%%%%%%%%%%%%%%%
\paragraph{v0.5:} 2017/04/26

\begin{itemize}
\item
functionality in definition file
\end{itemize}


%%%%%%%%%%%%%%%%%%%%%%%%%%%%%%%%%%%%%%%%%%%%%%%%%%%%%%%%%%%%%%%%%%%%%%%%%%%%%%%%
%%%%%%%%%%%%%%%%%%%%%%%%%%%%%%%%%%%%%%%%%%%%%%%%%%%%%%%%%%%%%%%%%%%%%%%%%%%%%%%%
%%%%%%%%%%%%%%%%%%%%%%%%%%%%%%%%%%%%%%%%%%%%%%%%%%%%%%%%%%%%%%%%%%%%%%%%%%%%%%%%
\appendix

\settowidth\MacroIndent{\rmfamily\scriptsize 000\ }

 \DocInput{childdoc.dtx}

\end{document}
%</driver>
% \fi
%
% %%%%%%%%%%%%%%%%%%%%%%%%%%%%%%%%%%%%%%%%%%%%%%%%%%%%%%%%%%%%%%%%%%%%%%%%%%%%%%
% %%%%%%%%%%%%%%%%%%%%%%%%%%%%%%%%%%%%%%%%%%%%%%%%%%%%%%%%%%%%%%%%%%%%%%%%%%%%%%
% \section{Sample}
%\iffalse
%<*samplemain>
%\fi
%
% The following presents a sample document
% with two chapters, two parts, a title page,
% a compile flag as well as three forwarding files to set the flag.
% It consists of eight |.tex| files:
% \begin{center}
% \begin{tabular}{ll}
% |cdocsamp.tex|&main file\\
% |cdocsch1.tex|&include file for chapter 1\\
% |cdocsch2.tex|&include file for chapter 2\\
% |cdocspt3.tex|&include file for part 3\\
% |cdocspt4.tex|&include file for part 4\\
% |cdocsdrf.tex|&forwarding file for main file in draft mode\\
% |cdocsfi1.tex|&forwarding file for final version of chapter 1\\
% |cdocsfi2.tex|&forwarding file for final version of chapter 2\\
% \end{tabular}
% \end{center}
% Each of the eight files can be compiled directly by the \LaTeX{} compiler.
%
% %%%%%%%%%%%%%%%%%%%%%%%%%%%%%%%%%%%%%%
% \paragraph{Main File.}
%
% The main file is called |cdocsamp.tex|.
%
% Load the \textsf{childdoc} definitions and
% declare the filename for the main document:
%    \begin{macrocode}
\input{childdoc.def}
\childdocmain{}
%    \end{macrocode}

% Optional override for |\version| flag:
%    \begin{macrocode}
%%\ifchilddoc\else\providecommand{\version}{draft}\fi
%    \end{macrocode}

% Define the default values for the |\version| flag
% (|final| for the main file and |draft| for childs):
%    \begin{macrocode}
\ifchilddoc
\providecommand{\version}{draft}
\else
\providecommand{\version}{final}
\fi
%    \end{macrocode}

% Load the standard document class:
%    \begin{macrocode}
\documentclass[12pt]{article}
%    \end{macrocode}

% Start the document body:
%    \begin{macrocode}
\begin{document}
%    \end{macrocode}

% Declare a title page.
% Print title, part of document being processed and version flag:
%    \begin{macrocode}
\addtocounter{page}{-1}
\begin{center}
{\LARGE\bfseries{}childdoc example\par}
\vspace{1cm}
\ifchilddoc
\ifchilddocmanual part\else chapter\fi:
`\childdocname' of `\childdocjob'\par
\else
main document: `\childdocjob'\par
\fi
version: \version\par
\end{center}
\newpage
%    \end{macrocode}

% Manually include selected file,
% otherwise process as usual:
%    \begin{macrocode}
\ifchilddocmanual
\section*{part `\childdocname'}
\input{\childdocname}
\else
%    \end{macrocode}

% Include the two chapters:
%    \begin{macrocode}
\include{cdocsch1}
\include{cdocsch2}
%    \end{macrocode}

% Include the two parts unless only chapters should be displayed:
%    \begin{macrocode}
\ifchilddoc\else
\section{part three}
\input{cdocspt3}
\section{part four}
\input{cdocspt4}
\fi
%    \end{macrocode}

% Process as usual until here:
%    \begin{macrocode}
\fi
%    \end{macrocode}

% End of document body:
%    \begin{macrocode}
\end{document}
%    \end{macrocode}
%\iffalse
%</samplemain>
%\fi
%
% %%%%%%%%%%%%%%%%%%%%%%%%%%%%%%%%%%%%%%
% \paragraph{Chapter Include Files.}
%
% The include files are called |cdocsch1.tex| and |cdocsch2.tex|.
%
%\iffalse
%<*samplechap1|samplechap2>
%\fi

% Optional override for |\version| flag:
%    \begin{macrocode}
%%\providecommand{\version}{final}
%    \end{macrocode}

% Include the main document:
%    \begin{macrocode}
\input{childdoc.def}
\childdocof{cdocsamp}
%    \end{macrocode}

%\iffalse
%</samplechap1|samplechap2>
%\fi
%
%\iffalse
%<*samplechap1>
%\fi
% Some text for chapter 1:
%    \begin{macrocode}
\section{one}
some text in chapter one
%    \end{macrocode}

%\iffalse
%</samplechap1>
%\fi
% Some text for chapter 2:
%\iffalse
%<*samplechap2>
%\fi
%    \begin{macrocode}
\section{two}
more text in chapter two
%    \end{macrocode}

%\iffalse
%</samplechap2>
%\fi
%
% %%%%%%%%%%%%%%%%%%%%%%%%%%%%%%%%%%%%%%
% \paragraph{Part Include Files.}
%
% The include files are called |cdocspt3.tex| and |cdocspt4.tex|.
%
%\iffalse
%<*samplepart3|samplepart4>
%\fi

% Optional override for |\version| flag:
%    \begin{macrocode}
%%\providecommand{\version}{final}
%    \end{macrocode}

% Include the main document:
%    \begin{macrocode}
\input{childdoc.def}
\childdocby{cdocsamp}
%    \end{macrocode}

%\iffalse
%</samplepart3|samplepart4>
%\fi
%
%\iffalse
%<*samplepart3>
%\fi
% Some text for part 3:
%    \begin{macrocode}
some text in part three
%    \end{macrocode}

%\iffalse
%</samplepart3>
%\fi
% Some text for part 4:
%\iffalse
%<*samplepart4>
%\fi
%    \begin{macrocode}
more text in part four
%    \end{macrocode}

%\iffalse
%</samplepart4>
%\fi
%
% %%%%%%%%%%%%%%%%%%%%%%%%%%%%%%%%%%%%%%
% \paragraph{Forwarding for a Complete Draft.}
%
% The following forwarding file |cdocsdrf.tex|
% compiles the main document in draft mode:
%\iffalse
%<*sampledraft>
%\fi
%    \begin{macrocode}
\def\version{draft}
\input{childdoc.def}
\childdocforward{cdocsamp}
%    \end{macrocode}

%\iffalse
%</sampledraft>
%\fi
%
% %%%%%%%%%%%%%%%%%%%%%%%%%%%%%%%%%%%%%%
% \paragraph{Forwarding for Final Version of the Chapters.}
%
% The following forwarding files |cdocsfn1.tex| and |cdocsfn2.tex|
% (with identical content)
% compile the final versions of the child documents
% |cdocsch1.tex| and |cdocsch2.tex|, respectively:
%\iffalse
%<*samplefinal>
%\fi
%    \begin{macrocode}
\def\version{final}
\input{childdoc.def}
\childdocforwardprefix[cdocsamp]{cdocsfn}{cdocsch}
%    \end{macrocode}

%\iffalse
%</samplefinal>
%\fi
%
% %%%%%%%%%%%%%%%%%%%%%%%%%%%%%%%%%%%%%%
% \paragraph{Command Line Processing.}
%
% The following three command lines generate the output files
% |cdocscld|, |cdocscl1| and |cdocscl2|
% which should be identical to
% |cdocsdrf|, |cdocsch1| and |cdocsfn2|, respectively:
% \begin{center}
% \begin{tabular}{l}
% |latex -jobname cdocscld \|\\
% |  "\def\version{draft}\input{childdoc.def}\childdocforward{cdocsamp}"|\\
% |latex -jobname cdocscl1 \|\\
% |  "\input{childdoc.def}\childdocforward[cdocsamp]{cdocsch1}"|\\
% |latex -jobname cdocscl2 \|\\
% |  "\def\version{final}\input{childdoc.def}\childdocforward{cdocsch2}"|
% \end{tabular}
% \end{center}
% Note that the trailing backslash on each first line
% merely continues the input to the second line
% (for convenient cut ant paste).
% Furthermore, the command |latex| can be replaced by any
% of its alternative versions such as |pdflatex|.
%
% %%%%%%%%%%%%%%%%%%%%%%%%%%%%%%%%%%%%%%%%%%%%%%%%%%%%%%%%%%%%%%%%%%%%%%%%%%%%%%
% %%%%%%%%%%%%%%%%%%%%%%%%%%%%%%%%%%%%%%%%%%%%%%%%%%%%%%%%%%%%%%%%%%%%%%%%%%%%%%
% \section{Implementation}
%\iffalse
%<*package>
%\fi
%
% This section describes the definitions file |childdoc.def|.

% The definitions cannot be loaded using |\usepackage| or |\RequirePackage|
% which has a mechanism to prevent loading a style file more than once.
% When loading the definitions by means of |\input|
% multiple instances have to be prevented manually:
%\iffalse
%This code needs to be before the `\ProvidesFile' directive
%which is defined at the beginning of this file.
%Therefore it is also placed there and commented out here.
%</package>
%<*discard>
%\fi
%    \begin{macrocode}
\ifdefined\childdocmain\endinput\fi
%    \end{macrocode}
%\iffalse
%</discard>
%<*package>
%\fi
%
% \macro{\ifchilddoc}
% \macro{\ifchilddocmanual}
% The conditional |\ifchilddoc| tells whether a
% child (true) or main (false) document is being compiled.
% The conditional |\ifchilddocmanual| tells whether
% the |\includeonly| mechanism is used (false) or
% the selection of child files must be performed manually (true).
% The definitions initialise to false:
%    \begin{macrocode}
\newif\ifchilddoc
\newif\ifchilddocmanual
%    \end{macrocode}

% \macro{\childdocname}
% \macro{\childdocjob}
% The macro |\childdocname| stores the name of the main document
% to be compiled. The macro |\childdocjob| stores the name of
% the document on which the \LaTeX{} compiler was originally invoked.
% The content of |\jobname| cannot be compared
% to filenames specified in the source due to different catcodes.
% The following code rescans |\jobname|, stores the result
% in |\childdocname| and saves a copy in |\childdocjob|:
%    \begin{macrocode}
\edef\childdocname{\scantokens\expandafter{\jobname\noexpand}}
\let\childdocjob\childdocname
%    \end{macrocode}

% \macro{\childdocdisable}
% The macro |\childdocdisable| prevents the main file
% from being processed more than once.
% At this stage, the main document command |\childdocmain|
% is assumed to be called once again where it should do nothing.
% Any subsequent call to it should prevent
% a secondary processing of the main document
% It overwrites the forwarding commands
% |\childdocof| and |\childdocforward|
% with empty macros to prevent further inclusions of the main document:
%    \begin{macrocode}
\newcommand{\childdocdisable}
{
  \renewcommand{\childdocmain}[1]{\renewcommand{\childdocmain}[1]{\endinput}}
  \renewcommand{\childdocof}[1]{}
  \renewcommand{\childdocby}[2][]{}
  \renewcommand{\childdocforward}[2][]{}
  \renewcommand{\childdocdisable}{}
}
%    \end{macrocode}

% \macro{\childdocmain}
% The macro |\childdocmain| is to be called at the top of the main file
% with nothing or the main filename (without extension) as argument.
% First, it breaks loops.
% If the argument is not empty and does not match |\childdocname|
% (which is set by the first inclusion of |childdoc.def|),
% |\ifchilddoc| is set to true, |\includeonly| is applied to the child file
% and |\jobname| is set to the main file
% (for proper handling of |.aux| files):
%    \begin{macrocode}
\newcommand{\childdocmain}[1]
{
  \childdocdisable\childdocmain{}
  \if?#1?\else
    \begingroup
      \def\childdoctmp{#1}
      \ifx\childdoctmp\childdocname
        \def\childdoctmp{}
      \else
        \def\childdoctmp
        {
          \childdoctrue
          \includeonly{\childdocname}
          \def\childdocjob{#1}
          \def\jobname{#1}
        }
      \fi
      \expandafter
    \endgroup
    \childdoctmp
  \fi
}
%    \end{macrocode}

% \macro{\childdocof}
% The command |\childdocof| redirects
% compilation to the main file |#1|.
%    \begin{macrocode}
\newcommand{\childdocof}[1]
{
  \childdocdisable
  \childdoctrue
  \includeonly{\childdocname}
  \def\jobname{#1}
  \def\childdocjob{#1}
  \input{#1}
}
%    \end{macrocode}

% \macro{\childdocby}
% The command |\childdocby| ....
%    \begin{macrocode}
\newcommand{\childdocby}[2][]
{
  \childdocdisable
  \childdoctrue
  \childdocmanualtrue
  \if?#1?\else
    \def\jobname{#2}
  \fi
  \def\childdocjob{#2}
  \input{#2}
  \endinput
}
%    \end{macrocode}

% \macro{\childdocforward}
% The command |\childdocforward| redirects
% compilation to the main file or
% (if the optional argument is given) a child file.
% Parameters are set as if the main file
% or a child file starting with |\childdocof| was compiled.
% Then compilation is handed over to the main file:
%    \begin{macrocode}
\newcommand{\childdocforward}[2][]
{
  \begingroup
    \if?#1?
      \def\childdoctmp
      {
        \def\childdocname{#2}
        \def\childdocjob{#2}
        \def\jobname{#2}
        \input{#2}
        \endinput
      }
    \else
      \def\childdoctmp
      {
        \childdocdisable
        \def\childdocname{#2}
        \childdoctrue
        \includeonly{#2}
        \def\childdocjob{#1}
        \def\jobname{#1}
        \input{#1}
        \endinput
      }
    \fi
    \expandafter
  \endgroup
  \childdoctmp
}
%    \end{macrocode}

% \macro{\childdocforwardprefix}
% The command |\childdocforwardprefix| redirects
% compilation to the main or a child file by means of a pattern.
% The prefix |#1| in the current filename is replaced by |#2|
% and the suffix of the current filename is kept
% (it is assumed that the filename does not contain the substring `|~~~|'
% which is used as a delimiter).
% Compilation is handed over to the new file by |\childdocforward|:
%    \begin{macrocode}
\newcommand{\childdocforwardprefix}[3][]
{
  \begingroup
    \def\childdocextract #2##1~~~{\def\childdoctmp{\childdocforward[#1]{#3##1}}}
    \expandafter\childdocextract\childdocname~~~
    \expandafter
  \endgroup
  \childdoctmp
}
%    \end{macrocode}

% \macro{\childdoc}
% The deprecated macro |\childdoc| is a legacy version of |\childdocmain|:
%    \begin{macrocode}
\newcommand{\childdoc}{\childdocmain}
%    \end{macrocode}

% \macro{\childdocredirect}
% The deprecated macro |\childdocredirect| is a legacy version
% of |\childdocforward| and |\childdocforwardprefix|:
%    \begin{macrocode}
\newcommand{\childdocredirect}[2][]
{
  \begingroup
    \if?#1?
      \def\childdoctmp{\childdocforward{#2}}
    \else
      \def\childdoctmp{\childdocforwardprefix{#1}{#2}}
    \fi
    \expandafter
  \endgroup
  \childdoctmp
}
%    \end{macrocode}

%\iffalse
%</package>
%\fi
%
\endinput
|\\
|\childdocby{|\textit{main}|}|\\
\end{tabular}
\end{center}
%
The directive |\childdocby| is similar to |\childdocof|
described in \secref{sec:include},
but the subsequent selection of content must be done manually.
To that end, both |\ifchilddoc| and |\ifchilddocmanual|
will be true upon processing of a part,
and the name of the part is stored in |\childdocname|.
Note that |\jobname| will be set to the filename of the current part
so that each part receives an individual |.aux| file
that does not interfere with the |.aux| file(s) of the main document.
This behaviour can be altered by the alternative form
|\childdocby[*]{|\textit{main}|}| (with a non-empty optional argument)
which uses the |.aux| file of the main document
by setting |\jobname| to \textit{main}.

%%%%%%%%%%%%%%%%%%%%%%%%%%%%%%%%%%%%%%%%%%%%%%%%%%%%%%%%%%%%%%%%%%%%%%%%%%%%%%%%
\subsection{Driver Development}
\label{sec:driver}

The \textsf{childdoc} mechanism can also be use for the development
of definition files such as \LaTeX{} styles or classes.
This case differs from the above setup with multiple parts
included by |\include| in that no |\includeonly| should be invoked.
This can be achieved by starting the include file
(before |\ProvidesPackage|) with:
%
\begin{center}
\begin{tabular}{l}
|% \iffalse
%
% childdoc.dtx Copyright (C) 2017-2018 Niklas Beisert
%
% This work may be distributed and/or modified under the
% conditions of the LaTeX Project Public License, either version 1.3
% of this license or (at your option) any later version.
% The latest version of this license is in
%   http://www.latex-project.org/lppl.txt
% and version 1.3 or later is part of all distributions of LaTeX
% version 2005/12/01 or later.
%
% This work has the LPPL maintenance status `maintained'.
%
% The Current Maintainer of this work is Niklas Beisert.
%
% This work consists of the files childdoc.dtx and childdoc.ins
% and the derived files childdoc.def and cdocsamp.tex with
% cdocsch1.tex, cdocsch2.tex, cdocsdrf.tex, cdocsfn1.tex, cdocsfn2.tex.
%
%<package>\ifdefined\childdocmain\endinput\fi
%<package>\ProvidesFile{childdoc.def}[2018/12/30 v2.0 child document driver]
%<samplemain>\ProvidesFile{cdocsamp.tex}[2018/12/30 v2.0 sample for childdoc]
%<*driver>
%\ProvidesFile{childdoc.drv}[2018/12/30 v2.0 childdoc reference manual file]
\PassOptionsToClass{10pt,a4paper}{article}
\documentclass{ltxdoc}

\usepackage[margin=35mm]{geometry}
\usepackage{hyperref}
\usepackage{hyperxmp}
\usepackage[usenames]{color}

\hypersetup{colorlinks=true}
\hypersetup{pdfstartview=FitH}
\hypersetup{pdfpagemode=UseNone}
\hypersetup{pdfsource={}}
\hypersetup{pdflang={en-UK}}
\hypersetup{pdfcopyright={Copyright 2017-2018 Niklas Beisert.
  This work may be distributed and/or modified under the
  conditions of the LaTeX Project Public License, either version 1.3
  of this license or (at your option) any later version.}}
\hypersetup{pdflicenseurl={http://www.latex-project.org/lppl.txt}}
\hypersetup{pdfcontactaddress={ETH Zurich, ITP, HIT K,
  Wolfgang-Pauli-Strasse 27}}
\hypersetup{pdfcontactpostcode={8093}}
\hypersetup{pdfcontactcity={Zurich}}
\hypersetup{pdfcontactcountry={Switzerland}}
\hypersetup{pdfcontactemail={nbeisert@itp.phys.ethz.ch}}
\hypersetup{pdfcontacturl={http://people.phys.ethz.ch/\xmptilde nbeisert/}}

\newcommand{\secref}[1]{\hyperref[#1]{section \ref*{#1}}}

\parskip1ex
\parindent0pt
\let\olditemize\itemize
\def\itemize{\olditemize\parskip0pt}

\begin{document}

\title{The \textsf{childdoc} Package}
\hypersetup{pdftitle={The childdoc Package}}
\author{Niklas Beisert\\[2ex]
  Institut f\"ur Theoretische Physik\\
  Eidgen\"ossische Technische Hochschule Z\"urich\\
  Wolfgang-Pauli-Strasse 27, 8093 Z\"urich, Switzerland\\[1ex]
  \href{mailto:nbeisert@itp.phys.ethz.ch}
  {\texttt{nbeisert@itp.phys.ethz.ch}}}
\hypersetup{pdfauthor={Niklas Beisert}}
\hypersetup{pdfsubject={Manual for the LaTeX2e Package childdoc}}
\date{30 December 2018, \textsf{v2.0}}
\maketitle

\begin{abstract}\noindent
\textsf{childdoc} is a \LaTeXe{} package
that enables the direct compilation
of document sections included by |\include|
to individual files.
\end{abstract}

\begingroup
\parskip0ex
\tableofcontents
\endgroup

%%%%%%%%%%%%%%%%%%%%%%%%%%%%%%%%%%%%%%%%%%%%%%%%%%%%%%%%%%%%%%%%%%%%%%%%%%%%%%%%
%%%%%%%%%%%%%%%%%%%%%%%%%%%%%%%%%%%%%%%%%%%%%%%%%%%%%%%%%%%%%%%%%%%%%%%%%%%%%%%%
\section{Introduction}

\LaTeX{} provides a mechanism to structure a large document (such as a book)
into a main file and several child files (containing the chapters)
using the |\include| command.
This mechanism is beneficial for documents
which span hundreds of pages in order to
make the source file(s) more manageable.
Moreover, compilation can be restricted to
selected child files by means of the |\includeonly| command.
The latter feature can be used to reduce the compilation time while editing
(this was significantly more useful in the earlier days of \LaTeX{})
or to generate a smaller document which is easier to navigate.
Another application of |\includeonly| is to generate
documents consisting of selected parts of the complete document.

However, there are a few drawbacks of the plain |\include| mechanism:
\begin{itemize}
\item
The child files cannot be compiled on their own,
they can only be compiled via the main file.
A naive editing environment
(such as a text editor with an option
to have the current file processed by \LaTeX)
may require one to switch to the main file before compiling;
attempting to compile the child file produces errors.
\item
The main file must be modified (each time)
to adjust the |\includeonly| command
to the present needs. This easily leaves the main file in a messy state.
\item
The generated document will always carry the filename
of the main document. This is inconvenient if
several child files are to be compiled and
to be kept for distribution.
\end{itemize}

The present package provides a simple interface
to make child files individually compilable by \LaTeX{}.
Compiling a child file then has the same effect as compiling
the main file with an |\includeonly| command
to select the appropriate child.
Moreover the generated document will carry the name of the child
rather than the main file.
This resolves all three above issues.

This feature is meant to make the editing of books,
thesis documents and lecture notes somewhat more convenient.
However, the package can also be used efficiently for
composing a series of documents (such as exercise sheets)
which are typically distributed individually.
It then assists the author in generating the individual documents
(potentially in different versions)
as well as a document containing the collected series.
Another application is in developing style files
or other kinds of included material
where compilation of the style file could redirect
to a sample or test file.

%%%%%%%%%%%%%%%%%%%%%%%%%%%%%%%%%%%%%%%%%%%%%%%%%%%%%%%%%%%%%%%%%%%%%%%%%%%%%%%%
%%%%%%%%%%%%%%%%%%%%%%%%%%%%%%%%%%%%%%%%%%%%%%%%%%%%%%%%%%%%%%%%%%%%%%%%%%%%%%%%
\section{Usage}

First of all, the package \textsf{childdoc} is \emph{not} a standard
\LaTeXe{} |.sty| style file! Therefore it needs to be invoked in
a non-standard way.

%%%%%%%%%%%%%%%%%%%%%%%%%%%%%%%%%%%%%%%%%%%%%%%%%%%%%%%%%%%%%%%%%%%%%%%%%%%%%%%%
\subsection{Included Files}
\label{sec:include}

%%%%%%%%%%%%%%%%%%%%%%%%%%%%%%%%%%%%%%%%
\DescribeMacro{\childdocmain}
To use the package, add the commands
\begin{center}
\begin{tabular}{l}
|\input{childdoc.def}|\\
|\childdocmain{}|\\
\end{tabular}
\end{center}
at the very top of the main \LaTeX{} file,
in particular \emph{before} the |\documentclass| statement!
The argument of |\childdocmain| should be left empty
(but it must be present).

%%%%%%%%%%%%%%%%%%%%%%%%%%%%%%%%%%%%%%%%
\DescribeMacro{\childdocof}
Furthermore, add the commands
\begin{center}
\begin{tabular}{l}
|\input{childdoc.def}|\\
|\childdocof{|\textit{main}|}|\\
\end{tabular}
\end{center}
at the top of every child file \textit{child}
which is included by |\include{|\textit{child}|}|
from within the main file
(or at least for those files to be compiled individually).
The argument \textit{main} must be the filename of the main file.

There are a couple of
considerations in setting up the main and child documents:

%%%%%%%%%%%%%%%%%%%%%%%%%%%%%%%%%%%%%%%%
\paragraph{Restrictions.}

Please note the following restrictions:
\begin{itemize}
\item
|\childdocmain| must be called with one argument \textit{main}
to ensure compatibility with earlier version of the package.
It must either be empty (|\childdocmain{}|)
or precisely match the filename of the main file in which it is specified.
See \secref{sec:detection} for further information.
\item
The filename \textit{main} must be specified without the |.tex| extension.
\item
The filename \textit{main} is case sensitive
(even in case-insensitive file systems)
due to internal string comparison.
\item
The argument \textit{main} should be fully expanded, it cannot be a macro.
\item
Subdirectories and special characters should be avoided in filenames.
\item
The command |\childdocmain{|\textit{main}|}| must be followed by a whitespace.
It should not be followed immediately by another command
or by a comment mark `|%|'.
This is because the \TeX{} parser reads the token immediately following
the argument of |\childdocmain| and puts it
at the beginning of every child section;
however, a white\-space is ignored.
\end{itemize}

%%%%%%%%%%%%%%%%%%%%%%%%%%%%%%%%%%%%%%%%
\paragraph{Content of Main File.}

It is advisable to place all content in the child files included by |\include|.
Any output contained in the main file will appear in all child documents
unless suppressed manually;
it cannot be suppressed automatically by the |\includeonly| directive
and thus should normally be avoided.
A method to include some content in the main file
by means of conditional processing is described in \secref{sec:conditional}.

%%%%%%%%%%%%%%%%%%%%%%%%%%%%%%%%%%%%%%%%
\paragraph{Page Numbering.}

When only a part of the document is compiled,
the appropriate numbering of pages
(as well as other status parameters)
is determined from the |.aux| files.
The latter contain information from previous passes.
However this information needs to propagate through
all intermediate child documents.
Therefore the page numbering in child documents may well
be inconsistent until the complete document is compiled at least once.

A useful (if unconventional) way to always ensure a consistent
page numbering is to restart the numbering in each child document
and denote the pages by `\textit{child}|.|\textit{page}'
where \textit{child} represents the chapter/section number of the child file.
This can be achieved by the command
|\numberwithin{page}{|\textit{child}|}|
of the \textsf{amsmath} package
where \textit{child} can be |chapter| or |section|
depending on the chosen structuring.
Alternatively, one can modify the macro |\thepage| appropriately
and reset the counter |page| at the start of each child file.

%%%%%%%%%%%%%%%%%%%%%%%%%%%%%%%%%%%%%%%%%%%%%%%%%%%%%%%%%%%%%%%%%%%%%%%%%%%%%%%%
\subsection{Conditional Processing}
\label{sec:conditional}

The package provides a mechanism to compile different versions
of a document. To customise the versions further some conditional processing
can come in handy to distinguish which version is being compiled.
The package provides two macros to describe the compilation context:

%%%%%%%%%%%%%%%%%%%%%%%%%%%%%%%%%%%%%%%%
\DescribeMacro{\ifchilddoc}
The conditional |\ifchilddoc| distinguishes between the compilation of
child documents and the main document:
%
\begin{center}
|\ifchilddoc |\textit{child-code}| |[|\||else |\textit{main-code}]| \||fi|
\end{center}

%%%%%%%%%%%%%%%%%%%%%%%%%%%%%%%%%%%%%%%%
\DescribeMacro{\childdocname}
\DescribeMacro{\childdocjob}
The macro |\childdocname| contains the filename (without extension)
of the main or child file being processed.
Note that |\childdocjob| will always contain the name of the main file.

%%%%%%%%%%%%%%%%%%%%%%%%%%%%%%%%%%%%%%%%
\paragraph{Title Page.}

Conditional processing can be used to include a title or banner page
in the main document when proper precautions are taken.
Importantly, the code in the main file should ensure that the page counter
(as well as other status parameters which are stored in the |.aux| files)
takes the same value after the conditional processing.
Otherwise the page numbers may take divergent values
depending on which part is compiled.

For example, a title page could be declared by:
%
\begin{center}
\begin{tabular}{l}
|\ifchilddoc\||else|\\
|\addtocounter{page}{-1}|\\
\textit{code for title page}\\
|\newpage|\\
|\||fi|
\end{tabular}
\end{center}
%
A banner page for the child documents can be generated by:
%
\begin{center}
\begin{tabular}{l}
|\ifchilddoc|\\
|\addtocounter{page}{-1}|\\
\textit{code for banner page}\\
|\newpage|\\
|\||fi|
\end{tabular}
\end{center}
%
Here one could write a message such as:
\begin{center}
|This is the part \childdocname{} of \childdocjob{}.|
\end{center}

%%%%%%%%%%%%%%%%%%%%%%%%%%%%%%%%%%%%%%%%%%%%%%%%%%%%%%%%%%%%%%%%%%%%%%%%%%%%%%%%
\subsection{Flags}
\label{sec:flags}

The package makes it easy to generate different versions
of the main or child documents.
To this end compilation flags can be defined
and assigned different default values.
They will be particularly useful in conjunction
with the forwarding mechanism described in \secref{sec:forward}.

For example, it may be useful to have a flag |\version|
which can be set to |draft| or |final|.
The document source will contain some conditional code
depending on the value of |\version|.
Suppose further, the flag should default to |final| for the main file
and to |draft| for child files
which is a natural assignment for editing the document.
This is achieved by placing the following code
in the preamble of the main document
(below the |\childdocmain| directive):
%
\begin{center}
\begin{tabular}{l}
|\ifchilddoc|\\
|\providecommand{\version}{draft}|\\
|\||else|\\
|\providecommand{\version}{final}|\\
|\||fi|
\end{tabular}
\end{center}
%
The definition by |\providecommand| makes sure
that previous definitions are not overwritten.
Further statements |\providecommand{\version}{...}|
can thus be added before the above code to override it.

For the main file, one might add a line
(between |\childdocmain| and the above block)
%
\begin{center}
|%\ifchilddoc\||else\providecommand{\version}{draft}\||fi|
\end{center}
%
which can be uncommented to produce a draft version.
Likewise one can add a line to the very top of a child file
(above the |\childdocof{|\textit{main}|}| directive)
%
\begin{center}
|%\providecommand{\version}{final}|
\end{center}
%
which can be uncommented to produce the final version of this child document.

%%%%%%%%%%%%%%%%%%%%%%%%%%%%%%%%%%%%%%%%%%%%%%%%%%%%%%%%%%%%%%%%%%%%%%%%%%%%%%%%
\subsection{Forwarding}
\label{sec:forward}

Different versions of the main or child documents
using compilation flags as described in \secref{sec:flags}
can be (permanently) stored in different files
for convenient compilation, viewing and distribution.
To this end, the package defines a command
to pass on compilation to a different file:

%%%%%%%%%%%%%%%%%%%%%%%%%%%%%%%%%%%%%%%%
\DescribeMacro{\childdocforward}
The command |\childdocforward| redirects processing to
another source file:
%
\begin{center}
\begin{tabular}{l}
|\input{childdoc.def}|\\
|\childdocforward[|\textit{main}|]{|\textit{dest}|}|\\
\end{tabular}
\end{center}
%
The argument \textit{dest} is the destination file
(without extension).
It should be the main file or one of the child files.
Note that further \textsf{childdoc} directives
such as |\childdocof| and |\childdocforward|
in the indicated file will be processed in this form.
The optional argument \textit{main}
passes on directly to the main file \textit{main}
while pretending to compile the child \textit{dest}.
This form behaves as if \textit{dest}
issues |\childdocof{|\textit{main}|}| right away,
and no further \textsf{childdoc} directives will be processed.

%%%%%%%%%%%%%%%%%%%%%%%%%%%%%%%%%%%%%%%%
\DescribeMacro{\...prefix}
In the alternative form |\childdocforwardprefix|,
%
\begin{center}
\begin{tabular}{l}
|\input{childdoc.def}|\\
|\childdocforwardprefix[|\textit{main}|]{|\textit{prefix}|}{|\textit{dest}|}|
\end{tabular}
\end{center}
%
the destination file is determined by a pattern
depending on the current file:
To make this work, the current file must be called
`{\textit{prefix}\hspace{0.2em}\textit{suffix}}'
with \textit{prefix} matching precisely the argument.
Processing is then passed on to the file
`{\textit{dest}\hspace{0.2em}\textit{suffix}}'.
Surely, the same effect is achieved by
directly specifying the
argument `{\textit{dest}\hspace{0.2em}\textit{suffix}}'
in the first form.
However, that requires to set up a different file
for each child. With the alternative form of the command
all these files can have exactly the same content
which simplifies setting them up and maintaining them.

For example, the following file |draft.tex|
with a compilation flag |\version| as described in \secref{sec:flags}
compiles the main document as a draft:
%
\begin{center}
\begin{tabular}{l}
|\def\version{draft}|\\
|\input{childdoc.def}|\\
|\childdocforward{|\textit{main}|}|
\end{tabular}
\end{center}
%
Likewise, the following files |final|\textit{nn}|.tex|
compile the final version of the child document
|child|\textit{nn}|.tex|:
%
\begin{center}
\begin{tabular}{l}
|\def\version{final}|\\
|\input{childdoc.def}|\\
|\childdocforwardprefix{final}{child}|
\end{tabular}
\end{center}
%

Note that when several versions of a main file and/or of each child file
are to be generated, it may be convenient to set up a |Makefile| or
shell script to automatise the process.

%%%%%%%%%%%%%%%%%%%%%%%%%%%%%%%%%%%%%%%%%%%%%%%%%%%%%%%%%%%%%%%%%%%%%%%%%%%%%%%%
\subsection{Command Line Processing}
\label{sec:commandline}

The effect of redirection files can also be achieved by invoking
the \LaTeX{} compiler with a more elaborate command line.
Most conveniently this should be done as part
of a shell script or a |Makefile|.

When using \textsf{childdoc} in the main file, the following
command lines effectively perform a redirection
(note that depending on the shell being used,
backslashes may have to be doubled: `|\|' $\to$ `|\\|'):
%
\begin{center}
|... -jobname "|\textit{target}|" |\\|"|[\textit{flags}]%
|\input{childdoc.def}\childdocforward[|\textit{main}|]{|\textit{dest}|}"|
\end{center}
%
Here \textit{target} is the name of the output file,
\textit{main} is the name of the main file
and \textit{dest} is the name of the main or child file to be processed
(all filenames without extensions).
The optional argument \textit{main} can be omitted
if \textit{main} matches \textit{dest}.
Optionally, compilation \textit{flags} can be defined via |\def| commands.
This command line makes the \TeX{} engine believe
it is compiling the file \textit{target}
whose content is specified as the latter parameter.
The provided code then forwards the processing to
\textit{main} or \textit{dest} as described in \secref{sec:forward}.

%%%%%%%%%%%%%%%%%%%%%%%%%%%%%%%%%%%%%%%%%%%%%%%%%%%%%%%%%%%%%%%%%%%%%%%%%%%%%%%%
\subsection{Include by Input}
\label{sec:input}

Including child documents by |\include| has some restrictions by design.
Most notably, the content of a child document always occupies
its own set of pages; pages cannot be shared between child documents.
Usually, this behaviour makes perfect sense
because each child document contain an essential part of the document.
However, in some situations it may be desirable to compose
a document from a collection of parts
without having mandatory page breaks between then.
For this case, the package
provides a mechanism to include parts
by |\input| which can also be processed individually.
However, by construction this mechanism
requires manual handling of the content to be output.

%%%%%%%%%%%%%%%%%%%%%%%%%%%%%%%%%%%%%%%%
\DescribeMacro{\ifchilddocmanual}
The main file should be prepared as usual, see \secref{sec:include}.
However, the document body must make a distinction
between processing of an individual part and of the main document, e.g.:
%
\begin{center}
\begin{tabular}{l}
|\ifchilddocmanual|\\
|\input{\childdocname}|\\
|\||else|\\
\textit{document body with }|\input{|\textit{part}|}|\\
|\||fi|
\end{tabular}
\end{center}
%
The conditional |\ifchilddocmanual| is true whenever
a part to be included by |\input| is being compiled,
and the name of the part is stored in |\childdocname|.

%%%%%%%%%%%%%%%%%%%%%%%%%%%%%%%%%%%%%%%%
\DescribeMacro{\childdocby}
Each part to be included by |\input| should start with:
%
\begin{center}
\begin{tabular}{l}
|\input{childdoc.def}|\\
|\childdocby{|\textit{main}|}|\\
\end{tabular}
\end{center}
%
The directive |\childdocby| is similar to |\childdocof|
described in \secref{sec:include},
but the subsequent selection of content must be done manually.
To that end, both |\ifchilddoc| and |\ifchilddocmanual|
will be true upon processing of a part,
and the name of the part is stored in |\childdocname|.
Note that |\jobname| will be set to the filename of the current part
so that each part receives an individual |.aux| file
that does not interfere with the |.aux| file(s) of the main document.
This behaviour can be altered by the alternative form
|\childdocby[*]{|\textit{main}|}| (with a non-empty optional argument)
which uses the |.aux| file of the main document
by setting |\jobname| to \textit{main}.

%%%%%%%%%%%%%%%%%%%%%%%%%%%%%%%%%%%%%%%%%%%%%%%%%%%%%%%%%%%%%%%%%%%%%%%%%%%%%%%%
\subsection{Driver Development}
\label{sec:driver}

The \textsf{childdoc} mechanism can also be use for the development
of definition files such as \LaTeX{} styles or classes.
This case differs from the above setup with multiple parts
included by |\include| in that no |\includeonly| should be invoked.
This can be achieved by starting the include file
(before |\ProvidesPackage|) with:
%
\begin{center}
\begin{tabular}{l}
|\input{childdoc.def}|\\
|\childdocforward{|\textit{main}|}|\\
\end{tabular}
\end{center}
%
or alternatively with:
%
\begin{center}
\begin{tabular}{l}
|\input{childdoc.def}|\\
|\childdocby{|\textit{main}|}|\\
\end{tabular}
\end{center}
%
Both forms have slightly different effects as described above.
The main file is prepared as usual, see \secref{sec:include}.

%%%%%%%%%%%%%%%%%%%%%%%%%%%%%%%%%%%%%%%%%%%%%%%%%%%%%%%%%%%%%%%%%%%%%%%%%%%%%%%%
\subsection{Legacy Detection}
\label{sec:detection}

The directive |\childdocmain| in the main file can detect
whether the complete document or merely a child is to be compiled
even without using the directive |\childdocof|.
This method is deprecated because it is less robust
and there is no compelling reason to use it;
it is merely provided for backward compatibility
and it may be removed in future versions.

If the detection mechanism is to be used,
it is mandatory to correctly specify
the filename of the main file as the argument of |\childdocmain|:
%
\begin{center}
\begin{tabular}{l}
|\input{childdoc.def}|\\
|\childdocmain{|\textit{main}|}|\\
\end{tabular}
\end{center}
%
If |\jobname| does not match the argument \textit{main} of |\childdocmain|,
it is assumed that |\jobname| points to the child file to be compiled.
When using |\childdocmain| with the main file specified as argument,
it suffices to start a child file
with just |\input{|\textit{main}|}|
without loading of the package and using |\childdocof|.
If instead all processing is done
with the appropriate \textsf{childdoc} directives,
the argument of \textit{main} of |\childdocmain| can be empty.

An alternative version of the command line processing described
in \secref{sec:commandline} using the detection mechanism reads:
%
\begin{center}
|... -jobname "|\textit{target}|" "|[\textit{flags}]%
[|\def\jobname{|\textit{dest}|}|]|\input{|\textit{main}|}"|
\end{center}

%%%%%%%%%%%%%%%%%%%%%%%%%%%%%%%%%%%%%%%%%%%%%%%%%%%%%%%%%%%%%%%%%%%%%%%%%%%%%%%%
\subsection{Manual Code}
\label{sec:manual}

In case one cannot be certain whether the definitions file |childdoc.def|
is installed on the target \TeX{} distribution
and one prefers not to ship it,
it is conceivable to paste a few relevant commands into the sources.

To that end, drop all statements |\input{childdoc.def}|
and perform the replacements as outlined below.
Instead of |\childdocmain{|\textit{main}|}| add the following code
to the top of the main file:
%
\begin{center}
\begin{tabular}{l}
|\||ifdefined\childdocname\endinput\||fi\newif\ifchilddoc|\\
|\edef\childdocname{\scantokens\expandafter{\jobname\noexpand}}|\\
|\def\childdocmain{|\textit{main}|}\||ifx\childdocmain\childdocname\||else|\\
|\childdoctrue\includeonly{\childdocname}\let\jobname\childdocmain\||fi|\\
\end{tabular}
\end{center}
%
Instead of |\childdocof{|\textit{main}|}| just include the main file
at the top of each child file:
%
\begin{center}
|\input{|\textit{main}|}|
\end{center}
%
A simple redirection |\childdocforward{|\textit{dest}|}| is achieved by:
%
\begin{center}
|\def\jobname{|\textit{dest}|}\input{\jobname}|
\end{center}
%
The redirection with prefix
|\childdocforwardprefix[|\textit{prefix}|]{|\textit{dest}|}|
is accomplished by:
%
\begin{center}
\begin{tabular}{l}
|{\edef\jobname{\scantokens\expandafter{\jobname\noexpand}}|\\
|\def\redirectjob |\textit{prefix}|#1~~~{\gdef\jobname{|\textit{dest}|#1}}|\\
|\expandafter\redirectjob\jobname~~~}\input{\jobname}|
\end{tabular}
\end{center}

In an alternative approach,
child documents can be compiled by a specific command line
without additional code or specific definitions:
%
\begin{center}
|... -jobname "|\textit{target}|" "|[\textit{flags}]%
|\includeonly{|\textit{dest}|}\input{|\textit{main}|}"|
\end{center}
%

%%%%%%%%%%%%%%%%%%%%%%%%%%%%%%%%%%%%%%%%%%%%%%%%%%%%%%%%%%%%%%%%%%%%%%%%%%%%%%%%
%%%%%%%%%%%%%%%%%%%%%%%%%%%%%%%%%%%%%%%%%%%%%%%%%%%%%%%%%%%%%%%%%%%%%%%%%%%%%%%%
\section{Information}

%%%%%%%%%%%%%%%%%%%%%%%%%%%%%%%%%%%%%%%%%%%%%%%%%%%%%%%%%%%%%%%%%%%%%%%%%%%%%%%%
\subsection{Copyright}

Copyright \copyright{} 2017--2018 Niklas Beisert

This work may be distributed and/or modified under the
conditions of the \LaTeX{} Project Public License, either version 1.3
of this license or (at your option) any later version.
The latest version of this license is in
  \url{http://www.latex-project.org/lppl.txt}
and version 1.3 or later is part of all distributions of \LaTeX{}
version 2005/12/01 or later.

This work has the LPPL maintenance status `maintained'.

The Current Maintainer of this work is Niklas Beisert.

This work consists of the files |README.txt|, |childdoc.ins| and |childdoc.dtx|
as well as the derived files |childdoc.def|, |cdocsamp.tex|
with |cdocsch1.tex|, |cdocsch2.tex|, |cdocspt3.tex|, |cdocspt4.tex|,
|cdocsdrf.tex|, |cdocsfn1.tex|, |cdocsfn2.tex|
as well as |childdoc.pdf|.

%%%%%%%%%%%%%%%%%%%%%%%%%%%%%%%%%%%%%%%%%%%%%%%%%%%%%%%%%%%%%%%%%%%%%%%%%%%%%%%%
\subsection{Files and Installation}

The package consists of the files:
%
\begin{center}
\begin{tabular}{ll}
    |README.txt|   & readme file \\
    |childdoc.ins| & installation file \\
    |childdoc.dtx| & source file \\
    |childdoc.def| & definition file \\
    |cdocsamp.tex| & sample main file \\
    |cdocsch1.tex| & sample include file \\
    |cdocsch2.tex| & sample include file \\
    |cdocspt3.tex| & sample part file \\
    |cdocspt4.tex| & sample part file \\
    |cdocsdrf.tex| & sample redirection file \\
    |cdocsfn1.tex| & sample redirection file \\
    |cdocsfn2.tex| & sample redirection file \\
    |childdoc.pdf| & manual
\end{tabular}
\end{center}
%
The distribution consists of the files
|README.txt|, |childdoc.ins| and |childdoc.dtx|.
%
\begin{itemize}
\item
Run (pdf)\LaTeX{} on |childdoc.dtx|
to compile the manual |childdoc.pdf| (this file).
\item
Run \LaTeX{} on |childdoc.ins| to create the definitions file |childdoc.def|
and the sample |cdocsamp.tex| with include files
|cdocsch1.tex|, |cdocsch2.tex|, |cdocspt3.tex|, |cdocspt4.tex|,
|cdocsdrf.tex|, |cdocsfn1.tex|, |cdocsfn2.tex|.
Then copy the file |childdoc.def| to an appropriate directory of your \LaTeX{}
distribution, e.g.\ \textit{texmf-root}|/tex/latex/childdoc|.
\end{itemize}

%%%%%%%%%%%%%%%%%%%%%%%%%%%%%%%%%%%%%%%%%%%%%%%%%%%%%%%%%%%%%%%%%%%%%%%%%%%%%%%%
\subsection{Related CTAN Packages}

There are several other packages which offer a similar functionality:
%
\begin{itemize}
\item
The packages
\href{http://ctan.org/pkg/docmute}{\textsf{docmute}},
\href{http://ctan.org/pkg/includex}{\textsf{includex}} and
\href{http://ctan.org/pkg/standalone}{\textsf{standalone}}
provide commands to include only the document body of
a child file thus allowing both files to be compiled individually.
\item
The packages \href{http://ctan.org/pkg/subdocs}{\textsf{subdocs}}
and \href{http://ctan.org/pkg/subfiles}{\textsf{subfiles}}
provide structures in which the main and child documents can be
encapsulated and allowing them to be compiled individually.
The inclusion mechanism is different from the conventional |\include|.
\item
The package \href{http://ctan.org/pkg/combine}{\textsf{combine}}
is an elaborate solution to combine several documents into one.
\end{itemize}
%
See also the CTAN topic \href{http://ctan.org/topic/subdocs}{\textsf{subdocs}}
for further related packages.
The present package differs from the above solutions in that
a document structure constructed with the conventional |\include| mechanism
just needs two extra commands at the top of every file
such that all constituent files can be compiled individually.

%%%%%%%%%%%%%%%%%%%%%%%%%%%%%%%%%%%%%%%%%%%%%%%%%%%%%%%%%%%%%%%%%%%%%%%%%%%%%%%%
%\subsection{Feature Suggestions}
%
%The following is a list of features which may be useful for future
%versions of this package:
%%
%\begin{itemize}
%\item
%\ldots
%\end{itemize}

%%%%%%%%%%%%%%%%%%%%%%%%%%%%%%%%%%%%%%%%%%%%%%%%%%%%%%%%%%%%%%%%%%%%%%%%%%%%%%%%
\subsection{Revision History}

%%%%%%%%%%%%%%%%%%%%%%%%%%%%%%%%%%%%%%%%
\paragraph{v2.0:} 2018/12/30

\begin{itemize}
\item
immediate forward processing
\item
added |\childdocby| mechanism
\item
manual restructured
\end{itemize}

%%%%%%%%%%%%%%%%%%%%%%%%%%%%%%%%%%%%%%%%
\paragraph{v1.6:} 2018/01/17

\begin{itemize}
\item
application for development of include files
\item
corrections to manual
\end{itemize}

%%%%%%%%%%%%%%%%%%%%%%%%%%%%%%%%%%%%%%%%
\paragraph{v1.5:} 2017/05/21

\begin{itemize}
\item
more complete structuring introduced
\item
|\childdocof| introduced
\item
|\childdoc| renamed to |\childdocmain|
\item
|\childredirect| renamed to |\childdocforward| and |\childdocforwardprefix|
and functionality expanded
\end{itemize}

%%%%%%%%%%%%%%%%%%%%%%%%%%%%%%%%%%%%%%%%
\paragraph{v1.0:} 2017/04/27

\begin{itemize}
\item
manual and install package
\item
first version published on CTAN
\end{itemize}

%%%%%%%%%%%%%%%%%%%%%%%%%%%%%%%%%%%%%%%%
\paragraph{v0.6:} 2017/04/26

\begin{itemize}
\item
redirection mechanism added
\end{itemize}

%%%%%%%%%%%%%%%%%%%%%%%%%%%%%%%%%%%%%%%%
\paragraph{v0.5:} 2017/04/26

\begin{itemize}
\item
functionality in definition file
\end{itemize}


%%%%%%%%%%%%%%%%%%%%%%%%%%%%%%%%%%%%%%%%%%%%%%%%%%%%%%%%%%%%%%%%%%%%%%%%%%%%%%%%
%%%%%%%%%%%%%%%%%%%%%%%%%%%%%%%%%%%%%%%%%%%%%%%%%%%%%%%%%%%%%%%%%%%%%%%%%%%%%%%%
%%%%%%%%%%%%%%%%%%%%%%%%%%%%%%%%%%%%%%%%%%%%%%%%%%%%%%%%%%%%%%%%%%%%%%%%%%%%%%%%
\appendix

\settowidth\MacroIndent{\rmfamily\scriptsize 000\ }

 \DocInput{childdoc.dtx}

\end{document}
%</driver>
% \fi
%
% %%%%%%%%%%%%%%%%%%%%%%%%%%%%%%%%%%%%%%%%%%%%%%%%%%%%%%%%%%%%%%%%%%%%%%%%%%%%%%
% %%%%%%%%%%%%%%%%%%%%%%%%%%%%%%%%%%%%%%%%%%%%%%%%%%%%%%%%%%%%%%%%%%%%%%%%%%%%%%
% \section{Sample}
%\iffalse
%<*samplemain>
%\fi
%
% The following presents a sample document
% with two chapters, two parts, a title page,
% a compile flag as well as three forwarding files to set the flag.
% It consists of eight |.tex| files:
% \begin{center}
% \begin{tabular}{ll}
% |cdocsamp.tex|&main file\\
% |cdocsch1.tex|&include file for chapter 1\\
% |cdocsch2.tex|&include file for chapter 2\\
% |cdocspt3.tex|&include file for part 3\\
% |cdocspt4.tex|&include file for part 4\\
% |cdocsdrf.tex|&forwarding file for main file in draft mode\\
% |cdocsfi1.tex|&forwarding file for final version of chapter 1\\
% |cdocsfi2.tex|&forwarding file for final version of chapter 2\\
% \end{tabular}
% \end{center}
% Each of the eight files can be compiled directly by the \LaTeX{} compiler.
%
% %%%%%%%%%%%%%%%%%%%%%%%%%%%%%%%%%%%%%%
% \paragraph{Main File.}
%
% The main file is called |cdocsamp.tex|.
%
% Load the \textsf{childdoc} definitions and
% declare the filename for the main document:
%    \begin{macrocode}
\input{childdoc.def}
\childdocmain{}
%    \end{macrocode}

% Optional override for |\version| flag:
%    \begin{macrocode}
%%\ifchilddoc\else\providecommand{\version}{draft}\fi
%    \end{macrocode}

% Define the default values for the |\version| flag
% (|final| for the main file and |draft| for childs):
%    \begin{macrocode}
\ifchilddoc
\providecommand{\version}{draft}
\else
\providecommand{\version}{final}
\fi
%    \end{macrocode}

% Load the standard document class:
%    \begin{macrocode}
\documentclass[12pt]{article}
%    \end{macrocode}

% Start the document body:
%    \begin{macrocode}
\begin{document}
%    \end{macrocode}

% Declare a title page.
% Print title, part of document being processed and version flag:
%    \begin{macrocode}
\addtocounter{page}{-1}
\begin{center}
{\LARGE\bfseries{}childdoc example\par}
\vspace{1cm}
\ifchilddoc
\ifchilddocmanual part\else chapter\fi:
`\childdocname' of `\childdocjob'\par
\else
main document: `\childdocjob'\par
\fi
version: \version\par
\end{center}
\newpage
%    \end{macrocode}

% Manually include selected file,
% otherwise process as usual:
%    \begin{macrocode}
\ifchilddocmanual
\section*{part `\childdocname'}
\input{\childdocname}
\else
%    \end{macrocode}

% Include the two chapters:
%    \begin{macrocode}
\include{cdocsch1}
\include{cdocsch2}
%    \end{macrocode}

% Include the two parts unless only chapters should be displayed:
%    \begin{macrocode}
\ifchilddoc\else
\section{part three}
\input{cdocspt3}
\section{part four}
\input{cdocspt4}
\fi
%    \end{macrocode}

% Process as usual until here:
%    \begin{macrocode}
\fi
%    \end{macrocode}

% End of document body:
%    \begin{macrocode}
\end{document}
%    \end{macrocode}
%\iffalse
%</samplemain>
%\fi
%
% %%%%%%%%%%%%%%%%%%%%%%%%%%%%%%%%%%%%%%
% \paragraph{Chapter Include Files.}
%
% The include files are called |cdocsch1.tex| and |cdocsch2.tex|.
%
%\iffalse
%<*samplechap1|samplechap2>
%\fi

% Optional override for |\version| flag:
%    \begin{macrocode}
%%\providecommand{\version}{final}
%    \end{macrocode}

% Include the main document:
%    \begin{macrocode}
\input{childdoc.def}
\childdocof{cdocsamp}
%    \end{macrocode}

%\iffalse
%</samplechap1|samplechap2>
%\fi
%
%\iffalse
%<*samplechap1>
%\fi
% Some text for chapter 1:
%    \begin{macrocode}
\section{one}
some text in chapter one
%    \end{macrocode}

%\iffalse
%</samplechap1>
%\fi
% Some text for chapter 2:
%\iffalse
%<*samplechap2>
%\fi
%    \begin{macrocode}
\section{two}
more text in chapter two
%    \end{macrocode}

%\iffalse
%</samplechap2>
%\fi
%
% %%%%%%%%%%%%%%%%%%%%%%%%%%%%%%%%%%%%%%
% \paragraph{Part Include Files.}
%
% The include files are called |cdocspt3.tex| and |cdocspt4.tex|.
%
%\iffalse
%<*samplepart3|samplepart4>
%\fi

% Optional override for |\version| flag:
%    \begin{macrocode}
%%\providecommand{\version}{final}
%    \end{macrocode}

% Include the main document:
%    \begin{macrocode}
\input{childdoc.def}
\childdocby{cdocsamp}
%    \end{macrocode}

%\iffalse
%</samplepart3|samplepart4>
%\fi
%
%\iffalse
%<*samplepart3>
%\fi
% Some text for part 3:
%    \begin{macrocode}
some text in part three
%    \end{macrocode}

%\iffalse
%</samplepart3>
%\fi
% Some text for part 4:
%\iffalse
%<*samplepart4>
%\fi
%    \begin{macrocode}
more text in part four
%    \end{macrocode}

%\iffalse
%</samplepart4>
%\fi
%
% %%%%%%%%%%%%%%%%%%%%%%%%%%%%%%%%%%%%%%
% \paragraph{Forwarding for a Complete Draft.}
%
% The following forwarding file |cdocsdrf.tex|
% compiles the main document in draft mode:
%\iffalse
%<*sampledraft>
%\fi
%    \begin{macrocode}
\def\version{draft}
\input{childdoc.def}
\childdocforward{cdocsamp}
%    \end{macrocode}

%\iffalse
%</sampledraft>
%\fi
%
% %%%%%%%%%%%%%%%%%%%%%%%%%%%%%%%%%%%%%%
% \paragraph{Forwarding for Final Version of the Chapters.}
%
% The following forwarding files |cdocsfn1.tex| and |cdocsfn2.tex|
% (with identical content)
% compile the final versions of the child documents
% |cdocsch1.tex| and |cdocsch2.tex|, respectively:
%\iffalse
%<*samplefinal>
%\fi
%    \begin{macrocode}
\def\version{final}
\input{childdoc.def}
\childdocforwardprefix[cdocsamp]{cdocsfn}{cdocsch}
%    \end{macrocode}

%\iffalse
%</samplefinal>
%\fi
%
% %%%%%%%%%%%%%%%%%%%%%%%%%%%%%%%%%%%%%%
% \paragraph{Command Line Processing.}
%
% The following three command lines generate the output files
% |cdocscld|, |cdocscl1| and |cdocscl2|
% which should be identical to
% |cdocsdrf|, |cdocsch1| and |cdocsfn2|, respectively:
% \begin{center}
% \begin{tabular}{l}
% |latex -jobname cdocscld \|\\
% |  "\def\version{draft}\input{childdoc.def}\childdocforward{cdocsamp}"|\\
% |latex -jobname cdocscl1 \|\\
% |  "\input{childdoc.def}\childdocforward[cdocsamp]{cdocsch1}"|\\
% |latex -jobname cdocscl2 \|\\
% |  "\def\version{final}\input{childdoc.def}\childdocforward{cdocsch2}"|
% \end{tabular}
% \end{center}
% Note that the trailing backslash on each first line
% merely continues the input to the second line
% (for convenient cut ant paste).
% Furthermore, the command |latex| can be replaced by any
% of its alternative versions such as |pdflatex|.
%
% %%%%%%%%%%%%%%%%%%%%%%%%%%%%%%%%%%%%%%%%%%%%%%%%%%%%%%%%%%%%%%%%%%%%%%%%%%%%%%
% %%%%%%%%%%%%%%%%%%%%%%%%%%%%%%%%%%%%%%%%%%%%%%%%%%%%%%%%%%%%%%%%%%%%%%%%%%%%%%
% \section{Implementation}
%\iffalse
%<*package>
%\fi
%
% This section describes the definitions file |childdoc.def|.

% The definitions cannot be loaded using |\usepackage| or |\RequirePackage|
% which has a mechanism to prevent loading a style file more than once.
% When loading the definitions by means of |\input|
% multiple instances have to be prevented manually:
%\iffalse
%This code needs to be before the `\ProvidesFile' directive
%which is defined at the beginning of this file.
%Therefore it is also placed there and commented out here.
%</package>
%<*discard>
%\fi
%    \begin{macrocode}
\ifdefined\childdocmain\endinput\fi
%    \end{macrocode}
%\iffalse
%</discard>
%<*package>
%\fi
%
% \macro{\ifchilddoc}
% \macro{\ifchilddocmanual}
% The conditional |\ifchilddoc| tells whether a
% child (true) or main (false) document is being compiled.
% The conditional |\ifchilddocmanual| tells whether
% the |\includeonly| mechanism is used (false) or
% the selection of child files must be performed manually (true).
% The definitions initialise to false:
%    \begin{macrocode}
\newif\ifchilddoc
\newif\ifchilddocmanual
%    \end{macrocode}

% \macro{\childdocname}
% \macro{\childdocjob}
% The macro |\childdocname| stores the name of the main document
% to be compiled. The macro |\childdocjob| stores the name of
% the document on which the \LaTeX{} compiler was originally invoked.
% The content of |\jobname| cannot be compared
% to filenames specified in the source due to different catcodes.
% The following code rescans |\jobname|, stores the result
% in |\childdocname| and saves a copy in |\childdocjob|:
%    \begin{macrocode}
\edef\childdocname{\scantokens\expandafter{\jobname\noexpand}}
\let\childdocjob\childdocname
%    \end{macrocode}

% \macro{\childdocdisable}
% The macro |\childdocdisable| prevents the main file
% from being processed more than once.
% At this stage, the main document command |\childdocmain|
% is assumed to be called once again where it should do nothing.
% Any subsequent call to it should prevent
% a secondary processing of the main document
% It overwrites the forwarding commands
% |\childdocof| and |\childdocforward|
% with empty macros to prevent further inclusions of the main document:
%    \begin{macrocode}
\newcommand{\childdocdisable}
{
  \renewcommand{\childdocmain}[1]{\renewcommand{\childdocmain}[1]{\endinput}}
  \renewcommand{\childdocof}[1]{}
  \renewcommand{\childdocby}[2][]{}
  \renewcommand{\childdocforward}[2][]{}
  \renewcommand{\childdocdisable}{}
}
%    \end{macrocode}

% \macro{\childdocmain}
% The macro |\childdocmain| is to be called at the top of the main file
% with nothing or the main filename (without extension) as argument.
% First, it breaks loops.
% If the argument is not empty and does not match |\childdocname|
% (which is set by the first inclusion of |childdoc.def|),
% |\ifchilddoc| is set to true, |\includeonly| is applied to the child file
% and |\jobname| is set to the main file
% (for proper handling of |.aux| files):
%    \begin{macrocode}
\newcommand{\childdocmain}[1]
{
  \childdocdisable\childdocmain{}
  \if?#1?\else
    \begingroup
      \def\childdoctmp{#1}
      \ifx\childdoctmp\childdocname
        \def\childdoctmp{}
      \else
        \def\childdoctmp
        {
          \childdoctrue
          \includeonly{\childdocname}
          \def\childdocjob{#1}
          \def\jobname{#1}
        }
      \fi
      \expandafter
    \endgroup
    \childdoctmp
  \fi
}
%    \end{macrocode}

% \macro{\childdocof}
% The command |\childdocof| redirects
% compilation to the main file |#1|.
%    \begin{macrocode}
\newcommand{\childdocof}[1]
{
  \childdocdisable
  \childdoctrue
  \includeonly{\childdocname}
  \def\jobname{#1}
  \def\childdocjob{#1}
  \input{#1}
}
%    \end{macrocode}

% \macro{\childdocby}
% The command |\childdocby| ....
%    \begin{macrocode}
\newcommand{\childdocby}[2][]
{
  \childdocdisable
  \childdoctrue
  \childdocmanualtrue
  \if?#1?\else
    \def\jobname{#2}
  \fi
  \def\childdocjob{#2}
  \input{#2}
  \endinput
}
%    \end{macrocode}

% \macro{\childdocforward}
% The command |\childdocforward| redirects
% compilation to the main file or
% (if the optional argument is given) a child file.
% Parameters are set as if the main file
% or a child file starting with |\childdocof| was compiled.
% Then compilation is handed over to the main file:
%    \begin{macrocode}
\newcommand{\childdocforward}[2][]
{
  \begingroup
    \if?#1?
      \def\childdoctmp
      {
        \def\childdocname{#2}
        \def\childdocjob{#2}
        \def\jobname{#2}
        \input{#2}
        \endinput
      }
    \else
      \def\childdoctmp
      {
        \childdocdisable
        \def\childdocname{#2}
        \childdoctrue
        \includeonly{#2}
        \def\childdocjob{#1}
        \def\jobname{#1}
        \input{#1}
        \endinput
      }
    \fi
    \expandafter
  \endgroup
  \childdoctmp
}
%    \end{macrocode}

% \macro{\childdocforwardprefix}
% The command |\childdocforwardprefix| redirects
% compilation to the main or a child file by means of a pattern.
% The prefix |#1| in the current filename is replaced by |#2|
% and the suffix of the current filename is kept
% (it is assumed that the filename does not contain the substring `|~~~|'
% which is used as a delimiter).
% Compilation is handed over to the new file by |\childdocforward|:
%    \begin{macrocode}
\newcommand{\childdocforwardprefix}[3][]
{
  \begingroup
    \def\childdocextract #2##1~~~{\def\childdoctmp{\childdocforward[#1]{#3##1}}}
    \expandafter\childdocextract\childdocname~~~
    \expandafter
  \endgroup
  \childdoctmp
}
%    \end{macrocode}

% \macro{\childdoc}
% The deprecated macro |\childdoc| is a legacy version of |\childdocmain|:
%    \begin{macrocode}
\newcommand{\childdoc}{\childdocmain}
%    \end{macrocode}

% \macro{\childdocredirect}
% The deprecated macro |\childdocredirect| is a legacy version
% of |\childdocforward| and |\childdocforwardprefix|:
%    \begin{macrocode}
\newcommand{\childdocredirect}[2][]
{
  \begingroup
    \if?#1?
      \def\childdoctmp{\childdocforward{#2}}
    \else
      \def\childdoctmp{\childdocforwardprefix{#1}{#2}}
    \fi
    \expandafter
  \endgroup
  \childdoctmp
}
%    \end{macrocode}

%\iffalse
%</package>
%\fi
%
\endinput
|\\
|\childdocforward{|\textit{main}|}|\\
\end{tabular}
\end{center}
%
or alternatively with:
%
\begin{center}
\begin{tabular}{l}
|% \iffalse
%
% childdoc.dtx Copyright (C) 2017-2018 Niklas Beisert
%
% This work may be distributed and/or modified under the
% conditions of the LaTeX Project Public License, either version 1.3
% of this license or (at your option) any later version.
% The latest version of this license is in
%   http://www.latex-project.org/lppl.txt
% and version 1.3 or later is part of all distributions of LaTeX
% version 2005/12/01 or later.
%
% This work has the LPPL maintenance status `maintained'.
%
% The Current Maintainer of this work is Niklas Beisert.
%
% This work consists of the files childdoc.dtx and childdoc.ins
% and the derived files childdoc.def and cdocsamp.tex with
% cdocsch1.tex, cdocsch2.tex, cdocsdrf.tex, cdocsfn1.tex, cdocsfn2.tex.
%
%<package>\ifdefined\childdocmain\endinput\fi
%<package>\ProvidesFile{childdoc.def}[2018/12/30 v2.0 child document driver]
%<samplemain>\ProvidesFile{cdocsamp.tex}[2018/12/30 v2.0 sample for childdoc]
%<*driver>
%\ProvidesFile{childdoc.drv}[2018/12/30 v2.0 childdoc reference manual file]
\PassOptionsToClass{10pt,a4paper}{article}
\documentclass{ltxdoc}

\usepackage[margin=35mm]{geometry}
\usepackage{hyperref}
\usepackage{hyperxmp}
\usepackage[usenames]{color}

\hypersetup{colorlinks=true}
\hypersetup{pdfstartview=FitH}
\hypersetup{pdfpagemode=UseNone}
\hypersetup{pdfsource={}}
\hypersetup{pdflang={en-UK}}
\hypersetup{pdfcopyright={Copyright 2017-2018 Niklas Beisert.
  This work may be distributed and/or modified under the
  conditions of the LaTeX Project Public License, either version 1.3
  of this license or (at your option) any later version.}}
\hypersetup{pdflicenseurl={http://www.latex-project.org/lppl.txt}}
\hypersetup{pdfcontactaddress={ETH Zurich, ITP, HIT K,
  Wolfgang-Pauli-Strasse 27}}
\hypersetup{pdfcontactpostcode={8093}}
\hypersetup{pdfcontactcity={Zurich}}
\hypersetup{pdfcontactcountry={Switzerland}}
\hypersetup{pdfcontactemail={nbeisert@itp.phys.ethz.ch}}
\hypersetup{pdfcontacturl={http://people.phys.ethz.ch/\xmptilde nbeisert/}}

\newcommand{\secref}[1]{\hyperref[#1]{section \ref*{#1}}}

\parskip1ex
\parindent0pt
\let\olditemize\itemize
\def\itemize{\olditemize\parskip0pt}

\begin{document}

\title{The \textsf{childdoc} Package}
\hypersetup{pdftitle={The childdoc Package}}
\author{Niklas Beisert\\[2ex]
  Institut f\"ur Theoretische Physik\\
  Eidgen\"ossische Technische Hochschule Z\"urich\\
  Wolfgang-Pauli-Strasse 27, 8093 Z\"urich, Switzerland\\[1ex]
  \href{mailto:nbeisert@itp.phys.ethz.ch}
  {\texttt{nbeisert@itp.phys.ethz.ch}}}
\hypersetup{pdfauthor={Niklas Beisert}}
\hypersetup{pdfsubject={Manual for the LaTeX2e Package childdoc}}
\date{30 December 2018, \textsf{v2.0}}
\maketitle

\begin{abstract}\noindent
\textsf{childdoc} is a \LaTeXe{} package
that enables the direct compilation
of document sections included by |\include|
to individual files.
\end{abstract}

\begingroup
\parskip0ex
\tableofcontents
\endgroup

%%%%%%%%%%%%%%%%%%%%%%%%%%%%%%%%%%%%%%%%%%%%%%%%%%%%%%%%%%%%%%%%%%%%%%%%%%%%%%%%
%%%%%%%%%%%%%%%%%%%%%%%%%%%%%%%%%%%%%%%%%%%%%%%%%%%%%%%%%%%%%%%%%%%%%%%%%%%%%%%%
\section{Introduction}

\LaTeX{} provides a mechanism to structure a large document (such as a book)
into a main file and several child files (containing the chapters)
using the |\include| command.
This mechanism is beneficial for documents
which span hundreds of pages in order to
make the source file(s) more manageable.
Moreover, compilation can be restricted to
selected child files by means of the |\includeonly| command.
The latter feature can be used to reduce the compilation time while editing
(this was significantly more useful in the earlier days of \LaTeX{})
or to generate a smaller document which is easier to navigate.
Another application of |\includeonly| is to generate
documents consisting of selected parts of the complete document.

However, there are a few drawbacks of the plain |\include| mechanism:
\begin{itemize}
\item
The child files cannot be compiled on their own,
they can only be compiled via the main file.
A naive editing environment
(such as a text editor with an option
to have the current file processed by \LaTeX)
may require one to switch to the main file before compiling;
attempting to compile the child file produces errors.
\item
The main file must be modified (each time)
to adjust the |\includeonly| command
to the present needs. This easily leaves the main file in a messy state.
\item
The generated document will always carry the filename
of the main document. This is inconvenient if
several child files are to be compiled and
to be kept for distribution.
\end{itemize}

The present package provides a simple interface
to make child files individually compilable by \LaTeX{}.
Compiling a child file then has the same effect as compiling
the main file with an |\includeonly| command
to select the appropriate child.
Moreover the generated document will carry the name of the child
rather than the main file.
This resolves all three above issues.

This feature is meant to make the editing of books,
thesis documents and lecture notes somewhat more convenient.
However, the package can also be used efficiently for
composing a series of documents (such as exercise sheets)
which are typically distributed individually.
It then assists the author in generating the individual documents
(potentially in different versions)
as well as a document containing the collected series.
Another application is in developing style files
or other kinds of included material
where compilation of the style file could redirect
to a sample or test file.

%%%%%%%%%%%%%%%%%%%%%%%%%%%%%%%%%%%%%%%%%%%%%%%%%%%%%%%%%%%%%%%%%%%%%%%%%%%%%%%%
%%%%%%%%%%%%%%%%%%%%%%%%%%%%%%%%%%%%%%%%%%%%%%%%%%%%%%%%%%%%%%%%%%%%%%%%%%%%%%%%
\section{Usage}

First of all, the package \textsf{childdoc} is \emph{not} a standard
\LaTeXe{} |.sty| style file! Therefore it needs to be invoked in
a non-standard way.

%%%%%%%%%%%%%%%%%%%%%%%%%%%%%%%%%%%%%%%%%%%%%%%%%%%%%%%%%%%%%%%%%%%%%%%%%%%%%%%%
\subsection{Included Files}
\label{sec:include}

%%%%%%%%%%%%%%%%%%%%%%%%%%%%%%%%%%%%%%%%
\DescribeMacro{\childdocmain}
To use the package, add the commands
\begin{center}
\begin{tabular}{l}
|\input{childdoc.def}|\\
|\childdocmain{}|\\
\end{tabular}
\end{center}
at the very top of the main \LaTeX{} file,
in particular \emph{before} the |\documentclass| statement!
The argument of |\childdocmain| should be left empty
(but it must be present).

%%%%%%%%%%%%%%%%%%%%%%%%%%%%%%%%%%%%%%%%
\DescribeMacro{\childdocof}
Furthermore, add the commands
\begin{center}
\begin{tabular}{l}
|\input{childdoc.def}|\\
|\childdocof{|\textit{main}|}|\\
\end{tabular}
\end{center}
at the top of every child file \textit{child}
which is included by |\include{|\textit{child}|}|
from within the main file
(or at least for those files to be compiled individually).
The argument \textit{main} must be the filename of the main file.

There are a couple of
considerations in setting up the main and child documents:

%%%%%%%%%%%%%%%%%%%%%%%%%%%%%%%%%%%%%%%%
\paragraph{Restrictions.}

Please note the following restrictions:
\begin{itemize}
\item
|\childdocmain| must be called with one argument \textit{main}
to ensure compatibility with earlier version of the package.
It must either be empty (|\childdocmain{}|)
or precisely match the filename of the main file in which it is specified.
See \secref{sec:detection} for further information.
\item
The filename \textit{main} must be specified without the |.tex| extension.
\item
The filename \textit{main} is case sensitive
(even in case-insensitive file systems)
due to internal string comparison.
\item
The argument \textit{main} should be fully expanded, it cannot be a macro.
\item
Subdirectories and special characters should be avoided in filenames.
\item
The command |\childdocmain{|\textit{main}|}| must be followed by a whitespace.
It should not be followed immediately by another command
or by a comment mark `|%|'.
This is because the \TeX{} parser reads the token immediately following
the argument of |\childdocmain| and puts it
at the beginning of every child section;
however, a white\-space is ignored.
\end{itemize}

%%%%%%%%%%%%%%%%%%%%%%%%%%%%%%%%%%%%%%%%
\paragraph{Content of Main File.}

It is advisable to place all content in the child files included by |\include|.
Any output contained in the main file will appear in all child documents
unless suppressed manually;
it cannot be suppressed automatically by the |\includeonly| directive
and thus should normally be avoided.
A method to include some content in the main file
by means of conditional processing is described in \secref{sec:conditional}.

%%%%%%%%%%%%%%%%%%%%%%%%%%%%%%%%%%%%%%%%
\paragraph{Page Numbering.}

When only a part of the document is compiled,
the appropriate numbering of pages
(as well as other status parameters)
is determined from the |.aux| files.
The latter contain information from previous passes.
However this information needs to propagate through
all intermediate child documents.
Therefore the page numbering in child documents may well
be inconsistent until the complete document is compiled at least once.

A useful (if unconventional) way to always ensure a consistent
page numbering is to restart the numbering in each child document
and denote the pages by `\textit{child}|.|\textit{page}'
where \textit{child} represents the chapter/section number of the child file.
This can be achieved by the command
|\numberwithin{page}{|\textit{child}|}|
of the \textsf{amsmath} package
where \textit{child} can be |chapter| or |section|
depending on the chosen structuring.
Alternatively, one can modify the macro |\thepage| appropriately
and reset the counter |page| at the start of each child file.

%%%%%%%%%%%%%%%%%%%%%%%%%%%%%%%%%%%%%%%%%%%%%%%%%%%%%%%%%%%%%%%%%%%%%%%%%%%%%%%%
\subsection{Conditional Processing}
\label{sec:conditional}

The package provides a mechanism to compile different versions
of a document. To customise the versions further some conditional processing
can come in handy to distinguish which version is being compiled.
The package provides two macros to describe the compilation context:

%%%%%%%%%%%%%%%%%%%%%%%%%%%%%%%%%%%%%%%%
\DescribeMacro{\ifchilddoc}
The conditional |\ifchilddoc| distinguishes between the compilation of
child documents and the main document:
%
\begin{center}
|\ifchilddoc |\textit{child-code}| |[|\||else |\textit{main-code}]| \||fi|
\end{center}

%%%%%%%%%%%%%%%%%%%%%%%%%%%%%%%%%%%%%%%%
\DescribeMacro{\childdocname}
\DescribeMacro{\childdocjob}
The macro |\childdocname| contains the filename (without extension)
of the main or child file being processed.
Note that |\childdocjob| will always contain the name of the main file.

%%%%%%%%%%%%%%%%%%%%%%%%%%%%%%%%%%%%%%%%
\paragraph{Title Page.}

Conditional processing can be used to include a title or banner page
in the main document when proper precautions are taken.
Importantly, the code in the main file should ensure that the page counter
(as well as other status parameters which are stored in the |.aux| files)
takes the same value after the conditional processing.
Otherwise the page numbers may take divergent values
depending on which part is compiled.

For example, a title page could be declared by:
%
\begin{center}
\begin{tabular}{l}
|\ifchilddoc\||else|\\
|\addtocounter{page}{-1}|\\
\textit{code for title page}\\
|\newpage|\\
|\||fi|
\end{tabular}
\end{center}
%
A banner page for the child documents can be generated by:
%
\begin{center}
\begin{tabular}{l}
|\ifchilddoc|\\
|\addtocounter{page}{-1}|\\
\textit{code for banner page}\\
|\newpage|\\
|\||fi|
\end{tabular}
\end{center}
%
Here one could write a message such as:
\begin{center}
|This is the part \childdocname{} of \childdocjob{}.|
\end{center}

%%%%%%%%%%%%%%%%%%%%%%%%%%%%%%%%%%%%%%%%%%%%%%%%%%%%%%%%%%%%%%%%%%%%%%%%%%%%%%%%
\subsection{Flags}
\label{sec:flags}

The package makes it easy to generate different versions
of the main or child documents.
To this end compilation flags can be defined
and assigned different default values.
They will be particularly useful in conjunction
with the forwarding mechanism described in \secref{sec:forward}.

For example, it may be useful to have a flag |\version|
which can be set to |draft| or |final|.
The document source will contain some conditional code
depending on the value of |\version|.
Suppose further, the flag should default to |final| for the main file
and to |draft| for child files
which is a natural assignment for editing the document.
This is achieved by placing the following code
in the preamble of the main document
(below the |\childdocmain| directive):
%
\begin{center}
\begin{tabular}{l}
|\ifchilddoc|\\
|\providecommand{\version}{draft}|\\
|\||else|\\
|\providecommand{\version}{final}|\\
|\||fi|
\end{tabular}
\end{center}
%
The definition by |\providecommand| makes sure
that previous definitions are not overwritten.
Further statements |\providecommand{\version}{...}|
can thus be added before the above code to override it.

For the main file, one might add a line
(between |\childdocmain| and the above block)
%
\begin{center}
|%\ifchilddoc\||else\providecommand{\version}{draft}\||fi|
\end{center}
%
which can be uncommented to produce a draft version.
Likewise one can add a line to the very top of a child file
(above the |\childdocof{|\textit{main}|}| directive)
%
\begin{center}
|%\providecommand{\version}{final}|
\end{center}
%
which can be uncommented to produce the final version of this child document.

%%%%%%%%%%%%%%%%%%%%%%%%%%%%%%%%%%%%%%%%%%%%%%%%%%%%%%%%%%%%%%%%%%%%%%%%%%%%%%%%
\subsection{Forwarding}
\label{sec:forward}

Different versions of the main or child documents
using compilation flags as described in \secref{sec:flags}
can be (permanently) stored in different files
for convenient compilation, viewing and distribution.
To this end, the package defines a command
to pass on compilation to a different file:

%%%%%%%%%%%%%%%%%%%%%%%%%%%%%%%%%%%%%%%%
\DescribeMacro{\childdocforward}
The command |\childdocforward| redirects processing to
another source file:
%
\begin{center}
\begin{tabular}{l}
|\input{childdoc.def}|\\
|\childdocforward[|\textit{main}|]{|\textit{dest}|}|\\
\end{tabular}
\end{center}
%
The argument \textit{dest} is the destination file
(without extension).
It should be the main file or one of the child files.
Note that further \textsf{childdoc} directives
such as |\childdocof| and |\childdocforward|
in the indicated file will be processed in this form.
The optional argument \textit{main}
passes on directly to the main file \textit{main}
while pretending to compile the child \textit{dest}.
This form behaves as if \textit{dest}
issues |\childdocof{|\textit{main}|}| right away,
and no further \textsf{childdoc} directives will be processed.

%%%%%%%%%%%%%%%%%%%%%%%%%%%%%%%%%%%%%%%%
\DescribeMacro{\...prefix}
In the alternative form |\childdocforwardprefix|,
%
\begin{center}
\begin{tabular}{l}
|\input{childdoc.def}|\\
|\childdocforwardprefix[|\textit{main}|]{|\textit{prefix}|}{|\textit{dest}|}|
\end{tabular}
\end{center}
%
the destination file is determined by a pattern
depending on the current file:
To make this work, the current file must be called
`{\textit{prefix}\hspace{0.2em}\textit{suffix}}'
with \textit{prefix} matching precisely the argument.
Processing is then passed on to the file
`{\textit{dest}\hspace{0.2em}\textit{suffix}}'.
Surely, the same effect is achieved by
directly specifying the
argument `{\textit{dest}\hspace{0.2em}\textit{suffix}}'
in the first form.
However, that requires to set up a different file
for each child. With the alternative form of the command
all these files can have exactly the same content
which simplifies setting them up and maintaining them.

For example, the following file |draft.tex|
with a compilation flag |\version| as described in \secref{sec:flags}
compiles the main document as a draft:
%
\begin{center}
\begin{tabular}{l}
|\def\version{draft}|\\
|\input{childdoc.def}|\\
|\childdocforward{|\textit{main}|}|
\end{tabular}
\end{center}
%
Likewise, the following files |final|\textit{nn}|.tex|
compile the final version of the child document
|child|\textit{nn}|.tex|:
%
\begin{center}
\begin{tabular}{l}
|\def\version{final}|\\
|\input{childdoc.def}|\\
|\childdocforwardprefix{final}{child}|
\end{tabular}
\end{center}
%

Note that when several versions of a main file and/or of each child file
are to be generated, it may be convenient to set up a |Makefile| or
shell script to automatise the process.

%%%%%%%%%%%%%%%%%%%%%%%%%%%%%%%%%%%%%%%%%%%%%%%%%%%%%%%%%%%%%%%%%%%%%%%%%%%%%%%%
\subsection{Command Line Processing}
\label{sec:commandline}

The effect of redirection files can also be achieved by invoking
the \LaTeX{} compiler with a more elaborate command line.
Most conveniently this should be done as part
of a shell script or a |Makefile|.

When using \textsf{childdoc} in the main file, the following
command lines effectively perform a redirection
(note that depending on the shell being used,
backslashes may have to be doubled: `|\|' $\to$ `|\\|'):
%
\begin{center}
|... -jobname "|\textit{target}|" |\\|"|[\textit{flags}]%
|\input{childdoc.def}\childdocforward[|\textit{main}|]{|\textit{dest}|}"|
\end{center}
%
Here \textit{target} is the name of the output file,
\textit{main} is the name of the main file
and \textit{dest} is the name of the main or child file to be processed
(all filenames without extensions).
The optional argument \textit{main} can be omitted
if \textit{main} matches \textit{dest}.
Optionally, compilation \textit{flags} can be defined via |\def| commands.
This command line makes the \TeX{} engine believe
it is compiling the file \textit{target}
whose content is specified as the latter parameter.
The provided code then forwards the processing to
\textit{main} or \textit{dest} as described in \secref{sec:forward}.

%%%%%%%%%%%%%%%%%%%%%%%%%%%%%%%%%%%%%%%%%%%%%%%%%%%%%%%%%%%%%%%%%%%%%%%%%%%%%%%%
\subsection{Include by Input}
\label{sec:input}

Including child documents by |\include| has some restrictions by design.
Most notably, the content of a child document always occupies
its own set of pages; pages cannot be shared between child documents.
Usually, this behaviour makes perfect sense
because each child document contain an essential part of the document.
However, in some situations it may be desirable to compose
a document from a collection of parts
without having mandatory page breaks between then.
For this case, the package
provides a mechanism to include parts
by |\input| which can also be processed individually.
However, by construction this mechanism
requires manual handling of the content to be output.

%%%%%%%%%%%%%%%%%%%%%%%%%%%%%%%%%%%%%%%%
\DescribeMacro{\ifchilddocmanual}
The main file should be prepared as usual, see \secref{sec:include}.
However, the document body must make a distinction
between processing of an individual part and of the main document, e.g.:
%
\begin{center}
\begin{tabular}{l}
|\ifchilddocmanual|\\
|\input{\childdocname}|\\
|\||else|\\
\textit{document body with }|\input{|\textit{part}|}|\\
|\||fi|
\end{tabular}
\end{center}
%
The conditional |\ifchilddocmanual| is true whenever
a part to be included by |\input| is being compiled,
and the name of the part is stored in |\childdocname|.

%%%%%%%%%%%%%%%%%%%%%%%%%%%%%%%%%%%%%%%%
\DescribeMacro{\childdocby}
Each part to be included by |\input| should start with:
%
\begin{center}
\begin{tabular}{l}
|\input{childdoc.def}|\\
|\childdocby{|\textit{main}|}|\\
\end{tabular}
\end{center}
%
The directive |\childdocby| is similar to |\childdocof|
described in \secref{sec:include},
but the subsequent selection of content must be done manually.
To that end, both |\ifchilddoc| and |\ifchilddocmanual|
will be true upon processing of a part,
and the name of the part is stored in |\childdocname|.
Note that |\jobname| will be set to the filename of the current part
so that each part receives an individual |.aux| file
that does not interfere with the |.aux| file(s) of the main document.
This behaviour can be altered by the alternative form
|\childdocby[*]{|\textit{main}|}| (with a non-empty optional argument)
which uses the |.aux| file of the main document
by setting |\jobname| to \textit{main}.

%%%%%%%%%%%%%%%%%%%%%%%%%%%%%%%%%%%%%%%%%%%%%%%%%%%%%%%%%%%%%%%%%%%%%%%%%%%%%%%%
\subsection{Driver Development}
\label{sec:driver}

The \textsf{childdoc} mechanism can also be use for the development
of definition files such as \LaTeX{} styles or classes.
This case differs from the above setup with multiple parts
included by |\include| in that no |\includeonly| should be invoked.
This can be achieved by starting the include file
(before |\ProvidesPackage|) with:
%
\begin{center}
\begin{tabular}{l}
|\input{childdoc.def}|\\
|\childdocforward{|\textit{main}|}|\\
\end{tabular}
\end{center}
%
or alternatively with:
%
\begin{center}
\begin{tabular}{l}
|\input{childdoc.def}|\\
|\childdocby{|\textit{main}|}|\\
\end{tabular}
\end{center}
%
Both forms have slightly different effects as described above.
The main file is prepared as usual, see \secref{sec:include}.

%%%%%%%%%%%%%%%%%%%%%%%%%%%%%%%%%%%%%%%%%%%%%%%%%%%%%%%%%%%%%%%%%%%%%%%%%%%%%%%%
\subsection{Legacy Detection}
\label{sec:detection}

The directive |\childdocmain| in the main file can detect
whether the complete document or merely a child is to be compiled
even without using the directive |\childdocof|.
This method is deprecated because it is less robust
and there is no compelling reason to use it;
it is merely provided for backward compatibility
and it may be removed in future versions.

If the detection mechanism is to be used,
it is mandatory to correctly specify
the filename of the main file as the argument of |\childdocmain|:
%
\begin{center}
\begin{tabular}{l}
|\input{childdoc.def}|\\
|\childdocmain{|\textit{main}|}|\\
\end{tabular}
\end{center}
%
If |\jobname| does not match the argument \textit{main} of |\childdocmain|,
it is assumed that |\jobname| points to the child file to be compiled.
When using |\childdocmain| with the main file specified as argument,
it suffices to start a child file
with just |\input{|\textit{main}|}|
without loading of the package and using |\childdocof|.
If instead all processing is done
with the appropriate \textsf{childdoc} directives,
the argument of \textit{main} of |\childdocmain| can be empty.

An alternative version of the command line processing described
in \secref{sec:commandline} using the detection mechanism reads:
%
\begin{center}
|... -jobname "|\textit{target}|" "|[\textit{flags}]%
[|\def\jobname{|\textit{dest}|}|]|\input{|\textit{main}|}"|
\end{center}

%%%%%%%%%%%%%%%%%%%%%%%%%%%%%%%%%%%%%%%%%%%%%%%%%%%%%%%%%%%%%%%%%%%%%%%%%%%%%%%%
\subsection{Manual Code}
\label{sec:manual}

In case one cannot be certain whether the definitions file |childdoc.def|
is installed on the target \TeX{} distribution
and one prefers not to ship it,
it is conceivable to paste a few relevant commands into the sources.

To that end, drop all statements |\input{childdoc.def}|
and perform the replacements as outlined below.
Instead of |\childdocmain{|\textit{main}|}| add the following code
to the top of the main file:
%
\begin{center}
\begin{tabular}{l}
|\||ifdefined\childdocname\endinput\||fi\newif\ifchilddoc|\\
|\edef\childdocname{\scantokens\expandafter{\jobname\noexpand}}|\\
|\def\childdocmain{|\textit{main}|}\||ifx\childdocmain\childdocname\||else|\\
|\childdoctrue\includeonly{\childdocname}\let\jobname\childdocmain\||fi|\\
\end{tabular}
\end{center}
%
Instead of |\childdocof{|\textit{main}|}| just include the main file
at the top of each child file:
%
\begin{center}
|\input{|\textit{main}|}|
\end{center}
%
A simple redirection |\childdocforward{|\textit{dest}|}| is achieved by:
%
\begin{center}
|\def\jobname{|\textit{dest}|}\input{\jobname}|
\end{center}
%
The redirection with prefix
|\childdocforwardprefix[|\textit{prefix}|]{|\textit{dest}|}|
is accomplished by:
%
\begin{center}
\begin{tabular}{l}
|{\edef\jobname{\scantokens\expandafter{\jobname\noexpand}}|\\
|\def\redirectjob |\textit{prefix}|#1~~~{\gdef\jobname{|\textit{dest}|#1}}|\\
|\expandafter\redirectjob\jobname~~~}\input{\jobname}|
\end{tabular}
\end{center}

In an alternative approach,
child documents can be compiled by a specific command line
without additional code or specific definitions:
%
\begin{center}
|... -jobname "|\textit{target}|" "|[\textit{flags}]%
|\includeonly{|\textit{dest}|}\input{|\textit{main}|}"|
\end{center}
%

%%%%%%%%%%%%%%%%%%%%%%%%%%%%%%%%%%%%%%%%%%%%%%%%%%%%%%%%%%%%%%%%%%%%%%%%%%%%%%%%
%%%%%%%%%%%%%%%%%%%%%%%%%%%%%%%%%%%%%%%%%%%%%%%%%%%%%%%%%%%%%%%%%%%%%%%%%%%%%%%%
\section{Information}

%%%%%%%%%%%%%%%%%%%%%%%%%%%%%%%%%%%%%%%%%%%%%%%%%%%%%%%%%%%%%%%%%%%%%%%%%%%%%%%%
\subsection{Copyright}

Copyright \copyright{} 2017--2018 Niklas Beisert

This work may be distributed and/or modified under the
conditions of the \LaTeX{} Project Public License, either version 1.3
of this license or (at your option) any later version.
The latest version of this license is in
  \url{http://www.latex-project.org/lppl.txt}
and version 1.3 or later is part of all distributions of \LaTeX{}
version 2005/12/01 or later.

This work has the LPPL maintenance status `maintained'.

The Current Maintainer of this work is Niklas Beisert.

This work consists of the files |README.txt|, |childdoc.ins| and |childdoc.dtx|
as well as the derived files |childdoc.def|, |cdocsamp.tex|
with |cdocsch1.tex|, |cdocsch2.tex|, |cdocspt3.tex|, |cdocspt4.tex|,
|cdocsdrf.tex|, |cdocsfn1.tex|, |cdocsfn2.tex|
as well as |childdoc.pdf|.

%%%%%%%%%%%%%%%%%%%%%%%%%%%%%%%%%%%%%%%%%%%%%%%%%%%%%%%%%%%%%%%%%%%%%%%%%%%%%%%%
\subsection{Files and Installation}

The package consists of the files:
%
\begin{center}
\begin{tabular}{ll}
    |README.txt|   & readme file \\
    |childdoc.ins| & installation file \\
    |childdoc.dtx| & source file \\
    |childdoc.def| & definition file \\
    |cdocsamp.tex| & sample main file \\
    |cdocsch1.tex| & sample include file \\
    |cdocsch2.tex| & sample include file \\
    |cdocspt3.tex| & sample part file \\
    |cdocspt4.tex| & sample part file \\
    |cdocsdrf.tex| & sample redirection file \\
    |cdocsfn1.tex| & sample redirection file \\
    |cdocsfn2.tex| & sample redirection file \\
    |childdoc.pdf| & manual
\end{tabular}
\end{center}
%
The distribution consists of the files
|README.txt|, |childdoc.ins| and |childdoc.dtx|.
%
\begin{itemize}
\item
Run (pdf)\LaTeX{} on |childdoc.dtx|
to compile the manual |childdoc.pdf| (this file).
\item
Run \LaTeX{} on |childdoc.ins| to create the definitions file |childdoc.def|
and the sample |cdocsamp.tex| with include files
|cdocsch1.tex|, |cdocsch2.tex|, |cdocspt3.tex|, |cdocspt4.tex|,
|cdocsdrf.tex|, |cdocsfn1.tex|, |cdocsfn2.tex|.
Then copy the file |childdoc.def| to an appropriate directory of your \LaTeX{}
distribution, e.g.\ \textit{texmf-root}|/tex/latex/childdoc|.
\end{itemize}

%%%%%%%%%%%%%%%%%%%%%%%%%%%%%%%%%%%%%%%%%%%%%%%%%%%%%%%%%%%%%%%%%%%%%%%%%%%%%%%%
\subsection{Related CTAN Packages}

There are several other packages which offer a similar functionality:
%
\begin{itemize}
\item
The packages
\href{http://ctan.org/pkg/docmute}{\textsf{docmute}},
\href{http://ctan.org/pkg/includex}{\textsf{includex}} and
\href{http://ctan.org/pkg/standalone}{\textsf{standalone}}
provide commands to include only the document body of
a child file thus allowing both files to be compiled individually.
\item
The packages \href{http://ctan.org/pkg/subdocs}{\textsf{subdocs}}
and \href{http://ctan.org/pkg/subfiles}{\textsf{subfiles}}
provide structures in which the main and child documents can be
encapsulated and allowing them to be compiled individually.
The inclusion mechanism is different from the conventional |\include|.
\item
The package \href{http://ctan.org/pkg/combine}{\textsf{combine}}
is an elaborate solution to combine several documents into one.
\end{itemize}
%
See also the CTAN topic \href{http://ctan.org/topic/subdocs}{\textsf{subdocs}}
for further related packages.
The present package differs from the above solutions in that
a document structure constructed with the conventional |\include| mechanism
just needs two extra commands at the top of every file
such that all constituent files can be compiled individually.

%%%%%%%%%%%%%%%%%%%%%%%%%%%%%%%%%%%%%%%%%%%%%%%%%%%%%%%%%%%%%%%%%%%%%%%%%%%%%%%%
%\subsection{Feature Suggestions}
%
%The following is a list of features which may be useful for future
%versions of this package:
%%
%\begin{itemize}
%\item
%\ldots
%\end{itemize}

%%%%%%%%%%%%%%%%%%%%%%%%%%%%%%%%%%%%%%%%%%%%%%%%%%%%%%%%%%%%%%%%%%%%%%%%%%%%%%%%
\subsection{Revision History}

%%%%%%%%%%%%%%%%%%%%%%%%%%%%%%%%%%%%%%%%
\paragraph{v2.0:} 2018/12/30

\begin{itemize}
\item
immediate forward processing
\item
added |\childdocby| mechanism
\item
manual restructured
\end{itemize}

%%%%%%%%%%%%%%%%%%%%%%%%%%%%%%%%%%%%%%%%
\paragraph{v1.6:} 2018/01/17

\begin{itemize}
\item
application for development of include files
\item
corrections to manual
\end{itemize}

%%%%%%%%%%%%%%%%%%%%%%%%%%%%%%%%%%%%%%%%
\paragraph{v1.5:} 2017/05/21

\begin{itemize}
\item
more complete structuring introduced
\item
|\childdocof| introduced
\item
|\childdoc| renamed to |\childdocmain|
\item
|\childredirect| renamed to |\childdocforward| and |\childdocforwardprefix|
and functionality expanded
\end{itemize}

%%%%%%%%%%%%%%%%%%%%%%%%%%%%%%%%%%%%%%%%
\paragraph{v1.0:} 2017/04/27

\begin{itemize}
\item
manual and install package
\item
first version published on CTAN
\end{itemize}

%%%%%%%%%%%%%%%%%%%%%%%%%%%%%%%%%%%%%%%%
\paragraph{v0.6:} 2017/04/26

\begin{itemize}
\item
redirection mechanism added
\end{itemize}

%%%%%%%%%%%%%%%%%%%%%%%%%%%%%%%%%%%%%%%%
\paragraph{v0.5:} 2017/04/26

\begin{itemize}
\item
functionality in definition file
\end{itemize}


%%%%%%%%%%%%%%%%%%%%%%%%%%%%%%%%%%%%%%%%%%%%%%%%%%%%%%%%%%%%%%%%%%%%%%%%%%%%%%%%
%%%%%%%%%%%%%%%%%%%%%%%%%%%%%%%%%%%%%%%%%%%%%%%%%%%%%%%%%%%%%%%%%%%%%%%%%%%%%%%%
%%%%%%%%%%%%%%%%%%%%%%%%%%%%%%%%%%%%%%%%%%%%%%%%%%%%%%%%%%%%%%%%%%%%%%%%%%%%%%%%
\appendix

\settowidth\MacroIndent{\rmfamily\scriptsize 000\ }

 \DocInput{childdoc.dtx}

\end{document}
%</driver>
% \fi
%
% %%%%%%%%%%%%%%%%%%%%%%%%%%%%%%%%%%%%%%%%%%%%%%%%%%%%%%%%%%%%%%%%%%%%%%%%%%%%%%
% %%%%%%%%%%%%%%%%%%%%%%%%%%%%%%%%%%%%%%%%%%%%%%%%%%%%%%%%%%%%%%%%%%%%%%%%%%%%%%
% \section{Sample}
%\iffalse
%<*samplemain>
%\fi
%
% The following presents a sample document
% with two chapters, two parts, a title page,
% a compile flag as well as three forwarding files to set the flag.
% It consists of eight |.tex| files:
% \begin{center}
% \begin{tabular}{ll}
% |cdocsamp.tex|&main file\\
% |cdocsch1.tex|&include file for chapter 1\\
% |cdocsch2.tex|&include file for chapter 2\\
% |cdocspt3.tex|&include file for part 3\\
% |cdocspt4.tex|&include file for part 4\\
% |cdocsdrf.tex|&forwarding file for main file in draft mode\\
% |cdocsfi1.tex|&forwarding file for final version of chapter 1\\
% |cdocsfi2.tex|&forwarding file for final version of chapter 2\\
% \end{tabular}
% \end{center}
% Each of the eight files can be compiled directly by the \LaTeX{} compiler.
%
% %%%%%%%%%%%%%%%%%%%%%%%%%%%%%%%%%%%%%%
% \paragraph{Main File.}
%
% The main file is called |cdocsamp.tex|.
%
% Load the \textsf{childdoc} definitions and
% declare the filename for the main document:
%    \begin{macrocode}
\input{childdoc.def}
\childdocmain{}
%    \end{macrocode}

% Optional override for |\version| flag:
%    \begin{macrocode}
%%\ifchilddoc\else\providecommand{\version}{draft}\fi
%    \end{macrocode}

% Define the default values for the |\version| flag
% (|final| for the main file and |draft| for childs):
%    \begin{macrocode}
\ifchilddoc
\providecommand{\version}{draft}
\else
\providecommand{\version}{final}
\fi
%    \end{macrocode}

% Load the standard document class:
%    \begin{macrocode}
\documentclass[12pt]{article}
%    \end{macrocode}

% Start the document body:
%    \begin{macrocode}
\begin{document}
%    \end{macrocode}

% Declare a title page.
% Print title, part of document being processed and version flag:
%    \begin{macrocode}
\addtocounter{page}{-1}
\begin{center}
{\LARGE\bfseries{}childdoc example\par}
\vspace{1cm}
\ifchilddoc
\ifchilddocmanual part\else chapter\fi:
`\childdocname' of `\childdocjob'\par
\else
main document: `\childdocjob'\par
\fi
version: \version\par
\end{center}
\newpage
%    \end{macrocode}

% Manually include selected file,
% otherwise process as usual:
%    \begin{macrocode}
\ifchilddocmanual
\section*{part `\childdocname'}
\input{\childdocname}
\else
%    \end{macrocode}

% Include the two chapters:
%    \begin{macrocode}
\include{cdocsch1}
\include{cdocsch2}
%    \end{macrocode}

% Include the two parts unless only chapters should be displayed:
%    \begin{macrocode}
\ifchilddoc\else
\section{part three}
\input{cdocspt3}
\section{part four}
\input{cdocspt4}
\fi
%    \end{macrocode}

% Process as usual until here:
%    \begin{macrocode}
\fi
%    \end{macrocode}

% End of document body:
%    \begin{macrocode}
\end{document}
%    \end{macrocode}
%\iffalse
%</samplemain>
%\fi
%
% %%%%%%%%%%%%%%%%%%%%%%%%%%%%%%%%%%%%%%
% \paragraph{Chapter Include Files.}
%
% The include files are called |cdocsch1.tex| and |cdocsch2.tex|.
%
%\iffalse
%<*samplechap1|samplechap2>
%\fi

% Optional override for |\version| flag:
%    \begin{macrocode}
%%\providecommand{\version}{final}
%    \end{macrocode}

% Include the main document:
%    \begin{macrocode}
\input{childdoc.def}
\childdocof{cdocsamp}
%    \end{macrocode}

%\iffalse
%</samplechap1|samplechap2>
%\fi
%
%\iffalse
%<*samplechap1>
%\fi
% Some text for chapter 1:
%    \begin{macrocode}
\section{one}
some text in chapter one
%    \end{macrocode}

%\iffalse
%</samplechap1>
%\fi
% Some text for chapter 2:
%\iffalse
%<*samplechap2>
%\fi
%    \begin{macrocode}
\section{two}
more text in chapter two
%    \end{macrocode}

%\iffalse
%</samplechap2>
%\fi
%
% %%%%%%%%%%%%%%%%%%%%%%%%%%%%%%%%%%%%%%
% \paragraph{Part Include Files.}
%
% The include files are called |cdocspt3.tex| and |cdocspt4.tex|.
%
%\iffalse
%<*samplepart3|samplepart4>
%\fi

% Optional override for |\version| flag:
%    \begin{macrocode}
%%\providecommand{\version}{final}
%    \end{macrocode}

% Include the main document:
%    \begin{macrocode}
\input{childdoc.def}
\childdocby{cdocsamp}
%    \end{macrocode}

%\iffalse
%</samplepart3|samplepart4>
%\fi
%
%\iffalse
%<*samplepart3>
%\fi
% Some text for part 3:
%    \begin{macrocode}
some text in part three
%    \end{macrocode}

%\iffalse
%</samplepart3>
%\fi
% Some text for part 4:
%\iffalse
%<*samplepart4>
%\fi
%    \begin{macrocode}
more text in part four
%    \end{macrocode}

%\iffalse
%</samplepart4>
%\fi
%
% %%%%%%%%%%%%%%%%%%%%%%%%%%%%%%%%%%%%%%
% \paragraph{Forwarding for a Complete Draft.}
%
% The following forwarding file |cdocsdrf.tex|
% compiles the main document in draft mode:
%\iffalse
%<*sampledraft>
%\fi
%    \begin{macrocode}
\def\version{draft}
\input{childdoc.def}
\childdocforward{cdocsamp}
%    \end{macrocode}

%\iffalse
%</sampledraft>
%\fi
%
% %%%%%%%%%%%%%%%%%%%%%%%%%%%%%%%%%%%%%%
% \paragraph{Forwarding for Final Version of the Chapters.}
%
% The following forwarding files |cdocsfn1.tex| and |cdocsfn2.tex|
% (with identical content)
% compile the final versions of the child documents
% |cdocsch1.tex| and |cdocsch2.tex|, respectively:
%\iffalse
%<*samplefinal>
%\fi
%    \begin{macrocode}
\def\version{final}
\input{childdoc.def}
\childdocforwardprefix[cdocsamp]{cdocsfn}{cdocsch}
%    \end{macrocode}

%\iffalse
%</samplefinal>
%\fi
%
% %%%%%%%%%%%%%%%%%%%%%%%%%%%%%%%%%%%%%%
% \paragraph{Command Line Processing.}
%
% The following three command lines generate the output files
% |cdocscld|, |cdocscl1| and |cdocscl2|
% which should be identical to
% |cdocsdrf|, |cdocsch1| and |cdocsfn2|, respectively:
% \begin{center}
% \begin{tabular}{l}
% |latex -jobname cdocscld \|\\
% |  "\def\version{draft}\input{childdoc.def}\childdocforward{cdocsamp}"|\\
% |latex -jobname cdocscl1 \|\\
% |  "\input{childdoc.def}\childdocforward[cdocsamp]{cdocsch1}"|\\
% |latex -jobname cdocscl2 \|\\
% |  "\def\version{final}\input{childdoc.def}\childdocforward{cdocsch2}"|
% \end{tabular}
% \end{center}
% Note that the trailing backslash on each first line
% merely continues the input to the second line
% (for convenient cut ant paste).
% Furthermore, the command |latex| can be replaced by any
% of its alternative versions such as |pdflatex|.
%
% %%%%%%%%%%%%%%%%%%%%%%%%%%%%%%%%%%%%%%%%%%%%%%%%%%%%%%%%%%%%%%%%%%%%%%%%%%%%%%
% %%%%%%%%%%%%%%%%%%%%%%%%%%%%%%%%%%%%%%%%%%%%%%%%%%%%%%%%%%%%%%%%%%%%%%%%%%%%%%
% \section{Implementation}
%\iffalse
%<*package>
%\fi
%
% This section describes the definitions file |childdoc.def|.

% The definitions cannot be loaded using |\usepackage| or |\RequirePackage|
% which has a mechanism to prevent loading a style file more than once.
% When loading the definitions by means of |\input|
% multiple instances have to be prevented manually:
%\iffalse
%This code needs to be before the `\ProvidesFile' directive
%which is defined at the beginning of this file.
%Therefore it is also placed there and commented out here.
%</package>
%<*discard>
%\fi
%    \begin{macrocode}
\ifdefined\childdocmain\endinput\fi
%    \end{macrocode}
%\iffalse
%</discard>
%<*package>
%\fi
%
% \macro{\ifchilddoc}
% \macro{\ifchilddocmanual}
% The conditional |\ifchilddoc| tells whether a
% child (true) or main (false) document is being compiled.
% The conditional |\ifchilddocmanual| tells whether
% the |\includeonly| mechanism is used (false) or
% the selection of child files must be performed manually (true).
% The definitions initialise to false:
%    \begin{macrocode}
\newif\ifchilddoc
\newif\ifchilddocmanual
%    \end{macrocode}

% \macro{\childdocname}
% \macro{\childdocjob}
% The macro |\childdocname| stores the name of the main document
% to be compiled. The macro |\childdocjob| stores the name of
% the document on which the \LaTeX{} compiler was originally invoked.
% The content of |\jobname| cannot be compared
% to filenames specified in the source due to different catcodes.
% The following code rescans |\jobname|, stores the result
% in |\childdocname| and saves a copy in |\childdocjob|:
%    \begin{macrocode}
\edef\childdocname{\scantokens\expandafter{\jobname\noexpand}}
\let\childdocjob\childdocname
%    \end{macrocode}

% \macro{\childdocdisable}
% The macro |\childdocdisable| prevents the main file
% from being processed more than once.
% At this stage, the main document command |\childdocmain|
% is assumed to be called once again where it should do nothing.
% Any subsequent call to it should prevent
% a secondary processing of the main document
% It overwrites the forwarding commands
% |\childdocof| and |\childdocforward|
% with empty macros to prevent further inclusions of the main document:
%    \begin{macrocode}
\newcommand{\childdocdisable}
{
  \renewcommand{\childdocmain}[1]{\renewcommand{\childdocmain}[1]{\endinput}}
  \renewcommand{\childdocof}[1]{}
  \renewcommand{\childdocby}[2][]{}
  \renewcommand{\childdocforward}[2][]{}
  \renewcommand{\childdocdisable}{}
}
%    \end{macrocode}

% \macro{\childdocmain}
% The macro |\childdocmain| is to be called at the top of the main file
% with nothing or the main filename (without extension) as argument.
% First, it breaks loops.
% If the argument is not empty and does not match |\childdocname|
% (which is set by the first inclusion of |childdoc.def|),
% |\ifchilddoc| is set to true, |\includeonly| is applied to the child file
% and |\jobname| is set to the main file
% (for proper handling of |.aux| files):
%    \begin{macrocode}
\newcommand{\childdocmain}[1]
{
  \childdocdisable\childdocmain{}
  \if?#1?\else
    \begingroup
      \def\childdoctmp{#1}
      \ifx\childdoctmp\childdocname
        \def\childdoctmp{}
      \else
        \def\childdoctmp
        {
          \childdoctrue
          \includeonly{\childdocname}
          \def\childdocjob{#1}
          \def\jobname{#1}
        }
      \fi
      \expandafter
    \endgroup
    \childdoctmp
  \fi
}
%    \end{macrocode}

% \macro{\childdocof}
% The command |\childdocof| redirects
% compilation to the main file |#1|.
%    \begin{macrocode}
\newcommand{\childdocof}[1]
{
  \childdocdisable
  \childdoctrue
  \includeonly{\childdocname}
  \def\jobname{#1}
  \def\childdocjob{#1}
  \input{#1}
}
%    \end{macrocode}

% \macro{\childdocby}
% The command |\childdocby| ....
%    \begin{macrocode}
\newcommand{\childdocby}[2][]
{
  \childdocdisable
  \childdoctrue
  \childdocmanualtrue
  \if?#1?\else
    \def\jobname{#2}
  \fi
  \def\childdocjob{#2}
  \input{#2}
  \endinput
}
%    \end{macrocode}

% \macro{\childdocforward}
% The command |\childdocforward| redirects
% compilation to the main file or
% (if the optional argument is given) a child file.
% Parameters are set as if the main file
% or a child file starting with |\childdocof| was compiled.
% Then compilation is handed over to the main file:
%    \begin{macrocode}
\newcommand{\childdocforward}[2][]
{
  \begingroup
    \if?#1?
      \def\childdoctmp
      {
        \def\childdocname{#2}
        \def\childdocjob{#2}
        \def\jobname{#2}
        \input{#2}
        \endinput
      }
    \else
      \def\childdoctmp
      {
        \childdocdisable
        \def\childdocname{#2}
        \childdoctrue
        \includeonly{#2}
        \def\childdocjob{#1}
        \def\jobname{#1}
        \input{#1}
        \endinput
      }
    \fi
    \expandafter
  \endgroup
  \childdoctmp
}
%    \end{macrocode}

% \macro{\childdocforwardprefix}
% The command |\childdocforwardprefix| redirects
% compilation to the main or a child file by means of a pattern.
% The prefix |#1| in the current filename is replaced by |#2|
% and the suffix of the current filename is kept
% (it is assumed that the filename does not contain the substring `|~~~|'
% which is used as a delimiter).
% Compilation is handed over to the new file by |\childdocforward|:
%    \begin{macrocode}
\newcommand{\childdocforwardprefix}[3][]
{
  \begingroup
    \def\childdocextract #2##1~~~{\def\childdoctmp{\childdocforward[#1]{#3##1}}}
    \expandafter\childdocextract\childdocname~~~
    \expandafter
  \endgroup
  \childdoctmp
}
%    \end{macrocode}

% \macro{\childdoc}
% The deprecated macro |\childdoc| is a legacy version of |\childdocmain|:
%    \begin{macrocode}
\newcommand{\childdoc}{\childdocmain}
%    \end{macrocode}

% \macro{\childdocredirect}
% The deprecated macro |\childdocredirect| is a legacy version
% of |\childdocforward| and |\childdocforwardprefix|:
%    \begin{macrocode}
\newcommand{\childdocredirect}[2][]
{
  \begingroup
    \if?#1?
      \def\childdoctmp{\childdocforward{#2}}
    \else
      \def\childdoctmp{\childdocforwardprefix{#1}{#2}}
    \fi
    \expandafter
  \endgroup
  \childdoctmp
}
%    \end{macrocode}

%\iffalse
%</package>
%\fi
%
\endinput
|\\
|\childdocby{|\textit{main}|}|\\
\end{tabular}
\end{center}
%
Both forms have slightly different effects as described above.
The main file is prepared as usual, see \secref{sec:include}.

%%%%%%%%%%%%%%%%%%%%%%%%%%%%%%%%%%%%%%%%%%%%%%%%%%%%%%%%%%%%%%%%%%%%%%%%%%%%%%%%
\subsection{Legacy Detection}
\label{sec:detection}

The directive |\childdocmain| in the main file can detect
whether the complete document or merely a child is to be compiled
even without using the directive |\childdocof|.
This method is deprecated because it is less robust
and there is no compelling reason to use it;
it is merely provided for backward compatibility
and it may be removed in future versions.

If the detection mechanism is to be used,
it is mandatory to correctly specify
the filename of the main file as the argument of |\childdocmain|:
%
\begin{center}
\begin{tabular}{l}
|% \iffalse
%
% childdoc.dtx Copyright (C) 2017-2018 Niklas Beisert
%
% This work may be distributed and/or modified under the
% conditions of the LaTeX Project Public License, either version 1.3
% of this license or (at your option) any later version.
% The latest version of this license is in
%   http://www.latex-project.org/lppl.txt
% and version 1.3 or later is part of all distributions of LaTeX
% version 2005/12/01 or later.
%
% This work has the LPPL maintenance status `maintained'.
%
% The Current Maintainer of this work is Niklas Beisert.
%
% This work consists of the files childdoc.dtx and childdoc.ins
% and the derived files childdoc.def and cdocsamp.tex with
% cdocsch1.tex, cdocsch2.tex, cdocsdrf.tex, cdocsfn1.tex, cdocsfn2.tex.
%
%<package>\ifdefined\childdocmain\endinput\fi
%<package>\ProvidesFile{childdoc.def}[2018/12/30 v2.0 child document driver]
%<samplemain>\ProvidesFile{cdocsamp.tex}[2018/12/30 v2.0 sample for childdoc]
%<*driver>
%\ProvidesFile{childdoc.drv}[2018/12/30 v2.0 childdoc reference manual file]
\PassOptionsToClass{10pt,a4paper}{article}
\documentclass{ltxdoc}

\usepackage[margin=35mm]{geometry}
\usepackage{hyperref}
\usepackage{hyperxmp}
\usepackage[usenames]{color}

\hypersetup{colorlinks=true}
\hypersetup{pdfstartview=FitH}
\hypersetup{pdfpagemode=UseNone}
\hypersetup{pdfsource={}}
\hypersetup{pdflang={en-UK}}
\hypersetup{pdfcopyright={Copyright 2017-2018 Niklas Beisert.
  This work may be distributed and/or modified under the
  conditions of the LaTeX Project Public License, either version 1.3
  of this license or (at your option) any later version.}}
\hypersetup{pdflicenseurl={http://www.latex-project.org/lppl.txt}}
\hypersetup{pdfcontactaddress={ETH Zurich, ITP, HIT K,
  Wolfgang-Pauli-Strasse 27}}
\hypersetup{pdfcontactpostcode={8093}}
\hypersetup{pdfcontactcity={Zurich}}
\hypersetup{pdfcontactcountry={Switzerland}}
\hypersetup{pdfcontactemail={nbeisert@itp.phys.ethz.ch}}
\hypersetup{pdfcontacturl={http://people.phys.ethz.ch/\xmptilde nbeisert/}}

\newcommand{\secref}[1]{\hyperref[#1]{section \ref*{#1}}}

\parskip1ex
\parindent0pt
\let\olditemize\itemize
\def\itemize{\olditemize\parskip0pt}

\begin{document}

\title{The \textsf{childdoc} Package}
\hypersetup{pdftitle={The childdoc Package}}
\author{Niklas Beisert\\[2ex]
  Institut f\"ur Theoretische Physik\\
  Eidgen\"ossische Technische Hochschule Z\"urich\\
  Wolfgang-Pauli-Strasse 27, 8093 Z\"urich, Switzerland\\[1ex]
  \href{mailto:nbeisert@itp.phys.ethz.ch}
  {\texttt{nbeisert@itp.phys.ethz.ch}}}
\hypersetup{pdfauthor={Niklas Beisert}}
\hypersetup{pdfsubject={Manual for the LaTeX2e Package childdoc}}
\date{30 December 2018, \textsf{v2.0}}
\maketitle

\begin{abstract}\noindent
\textsf{childdoc} is a \LaTeXe{} package
that enables the direct compilation
of document sections included by |\include|
to individual files.
\end{abstract}

\begingroup
\parskip0ex
\tableofcontents
\endgroup

%%%%%%%%%%%%%%%%%%%%%%%%%%%%%%%%%%%%%%%%%%%%%%%%%%%%%%%%%%%%%%%%%%%%%%%%%%%%%%%%
%%%%%%%%%%%%%%%%%%%%%%%%%%%%%%%%%%%%%%%%%%%%%%%%%%%%%%%%%%%%%%%%%%%%%%%%%%%%%%%%
\section{Introduction}

\LaTeX{} provides a mechanism to structure a large document (such as a book)
into a main file and several child files (containing the chapters)
using the |\include| command.
This mechanism is beneficial for documents
which span hundreds of pages in order to
make the source file(s) more manageable.
Moreover, compilation can be restricted to
selected child files by means of the |\includeonly| command.
The latter feature can be used to reduce the compilation time while editing
(this was significantly more useful in the earlier days of \LaTeX{})
or to generate a smaller document which is easier to navigate.
Another application of |\includeonly| is to generate
documents consisting of selected parts of the complete document.

However, there are a few drawbacks of the plain |\include| mechanism:
\begin{itemize}
\item
The child files cannot be compiled on their own,
they can only be compiled via the main file.
A naive editing environment
(such as a text editor with an option
to have the current file processed by \LaTeX)
may require one to switch to the main file before compiling;
attempting to compile the child file produces errors.
\item
The main file must be modified (each time)
to adjust the |\includeonly| command
to the present needs. This easily leaves the main file in a messy state.
\item
The generated document will always carry the filename
of the main document. This is inconvenient if
several child files are to be compiled and
to be kept for distribution.
\end{itemize}

The present package provides a simple interface
to make child files individually compilable by \LaTeX{}.
Compiling a child file then has the same effect as compiling
the main file with an |\includeonly| command
to select the appropriate child.
Moreover the generated document will carry the name of the child
rather than the main file.
This resolves all three above issues.

This feature is meant to make the editing of books,
thesis documents and lecture notes somewhat more convenient.
However, the package can also be used efficiently for
composing a series of documents (such as exercise sheets)
which are typically distributed individually.
It then assists the author in generating the individual documents
(potentially in different versions)
as well as a document containing the collected series.
Another application is in developing style files
or other kinds of included material
where compilation of the style file could redirect
to a sample or test file.

%%%%%%%%%%%%%%%%%%%%%%%%%%%%%%%%%%%%%%%%%%%%%%%%%%%%%%%%%%%%%%%%%%%%%%%%%%%%%%%%
%%%%%%%%%%%%%%%%%%%%%%%%%%%%%%%%%%%%%%%%%%%%%%%%%%%%%%%%%%%%%%%%%%%%%%%%%%%%%%%%
\section{Usage}

First of all, the package \textsf{childdoc} is \emph{not} a standard
\LaTeXe{} |.sty| style file! Therefore it needs to be invoked in
a non-standard way.

%%%%%%%%%%%%%%%%%%%%%%%%%%%%%%%%%%%%%%%%%%%%%%%%%%%%%%%%%%%%%%%%%%%%%%%%%%%%%%%%
\subsection{Included Files}
\label{sec:include}

%%%%%%%%%%%%%%%%%%%%%%%%%%%%%%%%%%%%%%%%
\DescribeMacro{\childdocmain}
To use the package, add the commands
\begin{center}
\begin{tabular}{l}
|\input{childdoc.def}|\\
|\childdocmain{}|\\
\end{tabular}
\end{center}
at the very top of the main \LaTeX{} file,
in particular \emph{before} the |\documentclass| statement!
The argument of |\childdocmain| should be left empty
(but it must be present).

%%%%%%%%%%%%%%%%%%%%%%%%%%%%%%%%%%%%%%%%
\DescribeMacro{\childdocof}
Furthermore, add the commands
\begin{center}
\begin{tabular}{l}
|\input{childdoc.def}|\\
|\childdocof{|\textit{main}|}|\\
\end{tabular}
\end{center}
at the top of every child file \textit{child}
which is included by |\include{|\textit{child}|}|
from within the main file
(or at least for those files to be compiled individually).
The argument \textit{main} must be the filename of the main file.

There are a couple of
considerations in setting up the main and child documents:

%%%%%%%%%%%%%%%%%%%%%%%%%%%%%%%%%%%%%%%%
\paragraph{Restrictions.}

Please note the following restrictions:
\begin{itemize}
\item
|\childdocmain| must be called with one argument \textit{main}
to ensure compatibility with earlier version of the package.
It must either be empty (|\childdocmain{}|)
or precisely match the filename of the main file in which it is specified.
See \secref{sec:detection} for further information.
\item
The filename \textit{main} must be specified without the |.tex| extension.
\item
The filename \textit{main} is case sensitive
(even in case-insensitive file systems)
due to internal string comparison.
\item
The argument \textit{main} should be fully expanded, it cannot be a macro.
\item
Subdirectories and special characters should be avoided in filenames.
\item
The command |\childdocmain{|\textit{main}|}| must be followed by a whitespace.
It should not be followed immediately by another command
or by a comment mark `|%|'.
This is because the \TeX{} parser reads the token immediately following
the argument of |\childdocmain| and puts it
at the beginning of every child section;
however, a white\-space is ignored.
\end{itemize}

%%%%%%%%%%%%%%%%%%%%%%%%%%%%%%%%%%%%%%%%
\paragraph{Content of Main File.}

It is advisable to place all content in the child files included by |\include|.
Any output contained in the main file will appear in all child documents
unless suppressed manually;
it cannot be suppressed automatically by the |\includeonly| directive
and thus should normally be avoided.
A method to include some content in the main file
by means of conditional processing is described in \secref{sec:conditional}.

%%%%%%%%%%%%%%%%%%%%%%%%%%%%%%%%%%%%%%%%
\paragraph{Page Numbering.}

When only a part of the document is compiled,
the appropriate numbering of pages
(as well as other status parameters)
is determined from the |.aux| files.
The latter contain information from previous passes.
However this information needs to propagate through
all intermediate child documents.
Therefore the page numbering in child documents may well
be inconsistent until the complete document is compiled at least once.

A useful (if unconventional) way to always ensure a consistent
page numbering is to restart the numbering in each child document
and denote the pages by `\textit{child}|.|\textit{page}'
where \textit{child} represents the chapter/section number of the child file.
This can be achieved by the command
|\numberwithin{page}{|\textit{child}|}|
of the \textsf{amsmath} package
where \textit{child} can be |chapter| or |section|
depending on the chosen structuring.
Alternatively, one can modify the macro |\thepage| appropriately
and reset the counter |page| at the start of each child file.

%%%%%%%%%%%%%%%%%%%%%%%%%%%%%%%%%%%%%%%%%%%%%%%%%%%%%%%%%%%%%%%%%%%%%%%%%%%%%%%%
\subsection{Conditional Processing}
\label{sec:conditional}

The package provides a mechanism to compile different versions
of a document. To customise the versions further some conditional processing
can come in handy to distinguish which version is being compiled.
The package provides two macros to describe the compilation context:

%%%%%%%%%%%%%%%%%%%%%%%%%%%%%%%%%%%%%%%%
\DescribeMacro{\ifchilddoc}
The conditional |\ifchilddoc| distinguishes between the compilation of
child documents and the main document:
%
\begin{center}
|\ifchilddoc |\textit{child-code}| |[|\||else |\textit{main-code}]| \||fi|
\end{center}

%%%%%%%%%%%%%%%%%%%%%%%%%%%%%%%%%%%%%%%%
\DescribeMacro{\childdocname}
\DescribeMacro{\childdocjob}
The macro |\childdocname| contains the filename (without extension)
of the main or child file being processed.
Note that |\childdocjob| will always contain the name of the main file.

%%%%%%%%%%%%%%%%%%%%%%%%%%%%%%%%%%%%%%%%
\paragraph{Title Page.}

Conditional processing can be used to include a title or banner page
in the main document when proper precautions are taken.
Importantly, the code in the main file should ensure that the page counter
(as well as other status parameters which are stored in the |.aux| files)
takes the same value after the conditional processing.
Otherwise the page numbers may take divergent values
depending on which part is compiled.

For example, a title page could be declared by:
%
\begin{center}
\begin{tabular}{l}
|\ifchilddoc\||else|\\
|\addtocounter{page}{-1}|\\
\textit{code for title page}\\
|\newpage|\\
|\||fi|
\end{tabular}
\end{center}
%
A banner page for the child documents can be generated by:
%
\begin{center}
\begin{tabular}{l}
|\ifchilddoc|\\
|\addtocounter{page}{-1}|\\
\textit{code for banner page}\\
|\newpage|\\
|\||fi|
\end{tabular}
\end{center}
%
Here one could write a message such as:
\begin{center}
|This is the part \childdocname{} of \childdocjob{}.|
\end{center}

%%%%%%%%%%%%%%%%%%%%%%%%%%%%%%%%%%%%%%%%%%%%%%%%%%%%%%%%%%%%%%%%%%%%%%%%%%%%%%%%
\subsection{Flags}
\label{sec:flags}

The package makes it easy to generate different versions
of the main or child documents.
To this end compilation flags can be defined
and assigned different default values.
They will be particularly useful in conjunction
with the forwarding mechanism described in \secref{sec:forward}.

For example, it may be useful to have a flag |\version|
which can be set to |draft| or |final|.
The document source will contain some conditional code
depending on the value of |\version|.
Suppose further, the flag should default to |final| for the main file
and to |draft| for child files
which is a natural assignment for editing the document.
This is achieved by placing the following code
in the preamble of the main document
(below the |\childdocmain| directive):
%
\begin{center}
\begin{tabular}{l}
|\ifchilddoc|\\
|\providecommand{\version}{draft}|\\
|\||else|\\
|\providecommand{\version}{final}|\\
|\||fi|
\end{tabular}
\end{center}
%
The definition by |\providecommand| makes sure
that previous definitions are not overwritten.
Further statements |\providecommand{\version}{...}|
can thus be added before the above code to override it.

For the main file, one might add a line
(between |\childdocmain| and the above block)
%
\begin{center}
|%\ifchilddoc\||else\providecommand{\version}{draft}\||fi|
\end{center}
%
which can be uncommented to produce a draft version.
Likewise one can add a line to the very top of a child file
(above the |\childdocof{|\textit{main}|}| directive)
%
\begin{center}
|%\providecommand{\version}{final}|
\end{center}
%
which can be uncommented to produce the final version of this child document.

%%%%%%%%%%%%%%%%%%%%%%%%%%%%%%%%%%%%%%%%%%%%%%%%%%%%%%%%%%%%%%%%%%%%%%%%%%%%%%%%
\subsection{Forwarding}
\label{sec:forward}

Different versions of the main or child documents
using compilation flags as described in \secref{sec:flags}
can be (permanently) stored in different files
for convenient compilation, viewing and distribution.
To this end, the package defines a command
to pass on compilation to a different file:

%%%%%%%%%%%%%%%%%%%%%%%%%%%%%%%%%%%%%%%%
\DescribeMacro{\childdocforward}
The command |\childdocforward| redirects processing to
another source file:
%
\begin{center}
\begin{tabular}{l}
|\input{childdoc.def}|\\
|\childdocforward[|\textit{main}|]{|\textit{dest}|}|\\
\end{tabular}
\end{center}
%
The argument \textit{dest} is the destination file
(without extension).
It should be the main file or one of the child files.
Note that further \textsf{childdoc} directives
such as |\childdocof| and |\childdocforward|
in the indicated file will be processed in this form.
The optional argument \textit{main}
passes on directly to the main file \textit{main}
while pretending to compile the child \textit{dest}.
This form behaves as if \textit{dest}
issues |\childdocof{|\textit{main}|}| right away,
and no further \textsf{childdoc} directives will be processed.

%%%%%%%%%%%%%%%%%%%%%%%%%%%%%%%%%%%%%%%%
\DescribeMacro{\...prefix}
In the alternative form |\childdocforwardprefix|,
%
\begin{center}
\begin{tabular}{l}
|\input{childdoc.def}|\\
|\childdocforwardprefix[|\textit{main}|]{|\textit{prefix}|}{|\textit{dest}|}|
\end{tabular}
\end{center}
%
the destination file is determined by a pattern
depending on the current file:
To make this work, the current file must be called
`{\textit{prefix}\hspace{0.2em}\textit{suffix}}'
with \textit{prefix} matching precisely the argument.
Processing is then passed on to the file
`{\textit{dest}\hspace{0.2em}\textit{suffix}}'.
Surely, the same effect is achieved by
directly specifying the
argument `{\textit{dest}\hspace{0.2em}\textit{suffix}}'
in the first form.
However, that requires to set up a different file
for each child. With the alternative form of the command
all these files can have exactly the same content
which simplifies setting them up and maintaining them.

For example, the following file |draft.tex|
with a compilation flag |\version| as described in \secref{sec:flags}
compiles the main document as a draft:
%
\begin{center}
\begin{tabular}{l}
|\def\version{draft}|\\
|\input{childdoc.def}|\\
|\childdocforward{|\textit{main}|}|
\end{tabular}
\end{center}
%
Likewise, the following files |final|\textit{nn}|.tex|
compile the final version of the child document
|child|\textit{nn}|.tex|:
%
\begin{center}
\begin{tabular}{l}
|\def\version{final}|\\
|\input{childdoc.def}|\\
|\childdocforwardprefix{final}{child}|
\end{tabular}
\end{center}
%

Note that when several versions of a main file and/or of each child file
are to be generated, it may be convenient to set up a |Makefile| or
shell script to automatise the process.

%%%%%%%%%%%%%%%%%%%%%%%%%%%%%%%%%%%%%%%%%%%%%%%%%%%%%%%%%%%%%%%%%%%%%%%%%%%%%%%%
\subsection{Command Line Processing}
\label{sec:commandline}

The effect of redirection files can also be achieved by invoking
the \LaTeX{} compiler with a more elaborate command line.
Most conveniently this should be done as part
of a shell script or a |Makefile|.

When using \textsf{childdoc} in the main file, the following
command lines effectively perform a redirection
(note that depending on the shell being used,
backslashes may have to be doubled: `|\|' $\to$ `|\\|'):
%
\begin{center}
|... -jobname "|\textit{target}|" |\\|"|[\textit{flags}]%
|\input{childdoc.def}\childdocforward[|\textit{main}|]{|\textit{dest}|}"|
\end{center}
%
Here \textit{target} is the name of the output file,
\textit{main} is the name of the main file
and \textit{dest} is the name of the main or child file to be processed
(all filenames without extensions).
The optional argument \textit{main} can be omitted
if \textit{main} matches \textit{dest}.
Optionally, compilation \textit{flags} can be defined via |\def| commands.
This command line makes the \TeX{} engine believe
it is compiling the file \textit{target}
whose content is specified as the latter parameter.
The provided code then forwards the processing to
\textit{main} or \textit{dest} as described in \secref{sec:forward}.

%%%%%%%%%%%%%%%%%%%%%%%%%%%%%%%%%%%%%%%%%%%%%%%%%%%%%%%%%%%%%%%%%%%%%%%%%%%%%%%%
\subsection{Include by Input}
\label{sec:input}

Including child documents by |\include| has some restrictions by design.
Most notably, the content of a child document always occupies
its own set of pages; pages cannot be shared between child documents.
Usually, this behaviour makes perfect sense
because each child document contain an essential part of the document.
However, in some situations it may be desirable to compose
a document from a collection of parts
without having mandatory page breaks between then.
For this case, the package
provides a mechanism to include parts
by |\input| which can also be processed individually.
However, by construction this mechanism
requires manual handling of the content to be output.

%%%%%%%%%%%%%%%%%%%%%%%%%%%%%%%%%%%%%%%%
\DescribeMacro{\ifchilddocmanual}
The main file should be prepared as usual, see \secref{sec:include}.
However, the document body must make a distinction
between processing of an individual part and of the main document, e.g.:
%
\begin{center}
\begin{tabular}{l}
|\ifchilddocmanual|\\
|\input{\childdocname}|\\
|\||else|\\
\textit{document body with }|\input{|\textit{part}|}|\\
|\||fi|
\end{tabular}
\end{center}
%
The conditional |\ifchilddocmanual| is true whenever
a part to be included by |\input| is being compiled,
and the name of the part is stored in |\childdocname|.

%%%%%%%%%%%%%%%%%%%%%%%%%%%%%%%%%%%%%%%%
\DescribeMacro{\childdocby}
Each part to be included by |\input| should start with:
%
\begin{center}
\begin{tabular}{l}
|\input{childdoc.def}|\\
|\childdocby{|\textit{main}|}|\\
\end{tabular}
\end{center}
%
The directive |\childdocby| is similar to |\childdocof|
described in \secref{sec:include},
but the subsequent selection of content must be done manually.
To that end, both |\ifchilddoc| and |\ifchilddocmanual|
will be true upon processing of a part,
and the name of the part is stored in |\childdocname|.
Note that |\jobname| will be set to the filename of the current part
so that each part receives an individual |.aux| file
that does not interfere with the |.aux| file(s) of the main document.
This behaviour can be altered by the alternative form
|\childdocby[*]{|\textit{main}|}| (with a non-empty optional argument)
which uses the |.aux| file of the main document
by setting |\jobname| to \textit{main}.

%%%%%%%%%%%%%%%%%%%%%%%%%%%%%%%%%%%%%%%%%%%%%%%%%%%%%%%%%%%%%%%%%%%%%%%%%%%%%%%%
\subsection{Driver Development}
\label{sec:driver}

The \textsf{childdoc} mechanism can also be use for the development
of definition files such as \LaTeX{} styles or classes.
This case differs from the above setup with multiple parts
included by |\include| in that no |\includeonly| should be invoked.
This can be achieved by starting the include file
(before |\ProvidesPackage|) with:
%
\begin{center}
\begin{tabular}{l}
|\input{childdoc.def}|\\
|\childdocforward{|\textit{main}|}|\\
\end{tabular}
\end{center}
%
or alternatively with:
%
\begin{center}
\begin{tabular}{l}
|\input{childdoc.def}|\\
|\childdocby{|\textit{main}|}|\\
\end{tabular}
\end{center}
%
Both forms have slightly different effects as described above.
The main file is prepared as usual, see \secref{sec:include}.

%%%%%%%%%%%%%%%%%%%%%%%%%%%%%%%%%%%%%%%%%%%%%%%%%%%%%%%%%%%%%%%%%%%%%%%%%%%%%%%%
\subsection{Legacy Detection}
\label{sec:detection}

The directive |\childdocmain| in the main file can detect
whether the complete document or merely a child is to be compiled
even without using the directive |\childdocof|.
This method is deprecated because it is less robust
and there is no compelling reason to use it;
it is merely provided for backward compatibility
and it may be removed in future versions.

If the detection mechanism is to be used,
it is mandatory to correctly specify
the filename of the main file as the argument of |\childdocmain|:
%
\begin{center}
\begin{tabular}{l}
|\input{childdoc.def}|\\
|\childdocmain{|\textit{main}|}|\\
\end{tabular}
\end{center}
%
If |\jobname| does not match the argument \textit{main} of |\childdocmain|,
it is assumed that |\jobname| points to the child file to be compiled.
When using |\childdocmain| with the main file specified as argument,
it suffices to start a child file
with just |\input{|\textit{main}|}|
without loading of the package and using |\childdocof|.
If instead all processing is done
with the appropriate \textsf{childdoc} directives,
the argument of \textit{main} of |\childdocmain| can be empty.

An alternative version of the command line processing described
in \secref{sec:commandline} using the detection mechanism reads:
%
\begin{center}
|... -jobname "|\textit{target}|" "|[\textit{flags}]%
[|\def\jobname{|\textit{dest}|}|]|\input{|\textit{main}|}"|
\end{center}

%%%%%%%%%%%%%%%%%%%%%%%%%%%%%%%%%%%%%%%%%%%%%%%%%%%%%%%%%%%%%%%%%%%%%%%%%%%%%%%%
\subsection{Manual Code}
\label{sec:manual}

In case one cannot be certain whether the definitions file |childdoc.def|
is installed on the target \TeX{} distribution
and one prefers not to ship it,
it is conceivable to paste a few relevant commands into the sources.

To that end, drop all statements |\input{childdoc.def}|
and perform the replacements as outlined below.
Instead of |\childdocmain{|\textit{main}|}| add the following code
to the top of the main file:
%
\begin{center}
\begin{tabular}{l}
|\||ifdefined\childdocname\endinput\||fi\newif\ifchilddoc|\\
|\edef\childdocname{\scantokens\expandafter{\jobname\noexpand}}|\\
|\def\childdocmain{|\textit{main}|}\||ifx\childdocmain\childdocname\||else|\\
|\childdoctrue\includeonly{\childdocname}\let\jobname\childdocmain\||fi|\\
\end{tabular}
\end{center}
%
Instead of |\childdocof{|\textit{main}|}| just include the main file
at the top of each child file:
%
\begin{center}
|\input{|\textit{main}|}|
\end{center}
%
A simple redirection |\childdocforward{|\textit{dest}|}| is achieved by:
%
\begin{center}
|\def\jobname{|\textit{dest}|}\input{\jobname}|
\end{center}
%
The redirection with prefix
|\childdocforwardprefix[|\textit{prefix}|]{|\textit{dest}|}|
is accomplished by:
%
\begin{center}
\begin{tabular}{l}
|{\edef\jobname{\scantokens\expandafter{\jobname\noexpand}}|\\
|\def\redirectjob |\textit{prefix}|#1~~~{\gdef\jobname{|\textit{dest}|#1}}|\\
|\expandafter\redirectjob\jobname~~~}\input{\jobname}|
\end{tabular}
\end{center}

In an alternative approach,
child documents can be compiled by a specific command line
without additional code or specific definitions:
%
\begin{center}
|... -jobname "|\textit{target}|" "|[\textit{flags}]%
|\includeonly{|\textit{dest}|}\input{|\textit{main}|}"|
\end{center}
%

%%%%%%%%%%%%%%%%%%%%%%%%%%%%%%%%%%%%%%%%%%%%%%%%%%%%%%%%%%%%%%%%%%%%%%%%%%%%%%%%
%%%%%%%%%%%%%%%%%%%%%%%%%%%%%%%%%%%%%%%%%%%%%%%%%%%%%%%%%%%%%%%%%%%%%%%%%%%%%%%%
\section{Information}

%%%%%%%%%%%%%%%%%%%%%%%%%%%%%%%%%%%%%%%%%%%%%%%%%%%%%%%%%%%%%%%%%%%%%%%%%%%%%%%%
\subsection{Copyright}

Copyright \copyright{} 2017--2018 Niklas Beisert

This work may be distributed and/or modified under the
conditions of the \LaTeX{} Project Public License, either version 1.3
of this license or (at your option) any later version.
The latest version of this license is in
  \url{http://www.latex-project.org/lppl.txt}
and version 1.3 or later is part of all distributions of \LaTeX{}
version 2005/12/01 or later.

This work has the LPPL maintenance status `maintained'.

The Current Maintainer of this work is Niklas Beisert.

This work consists of the files |README.txt|, |childdoc.ins| and |childdoc.dtx|
as well as the derived files |childdoc.def|, |cdocsamp.tex|
with |cdocsch1.tex|, |cdocsch2.tex|, |cdocspt3.tex|, |cdocspt4.tex|,
|cdocsdrf.tex|, |cdocsfn1.tex|, |cdocsfn2.tex|
as well as |childdoc.pdf|.

%%%%%%%%%%%%%%%%%%%%%%%%%%%%%%%%%%%%%%%%%%%%%%%%%%%%%%%%%%%%%%%%%%%%%%%%%%%%%%%%
\subsection{Files and Installation}

The package consists of the files:
%
\begin{center}
\begin{tabular}{ll}
    |README.txt|   & readme file \\
    |childdoc.ins| & installation file \\
    |childdoc.dtx| & source file \\
    |childdoc.def| & definition file \\
    |cdocsamp.tex| & sample main file \\
    |cdocsch1.tex| & sample include file \\
    |cdocsch2.tex| & sample include file \\
    |cdocspt3.tex| & sample part file \\
    |cdocspt4.tex| & sample part file \\
    |cdocsdrf.tex| & sample redirection file \\
    |cdocsfn1.tex| & sample redirection file \\
    |cdocsfn2.tex| & sample redirection file \\
    |childdoc.pdf| & manual
\end{tabular}
\end{center}
%
The distribution consists of the files
|README.txt|, |childdoc.ins| and |childdoc.dtx|.
%
\begin{itemize}
\item
Run (pdf)\LaTeX{} on |childdoc.dtx|
to compile the manual |childdoc.pdf| (this file).
\item
Run \LaTeX{} on |childdoc.ins| to create the definitions file |childdoc.def|
and the sample |cdocsamp.tex| with include files
|cdocsch1.tex|, |cdocsch2.tex|, |cdocspt3.tex|, |cdocspt4.tex|,
|cdocsdrf.tex|, |cdocsfn1.tex|, |cdocsfn2.tex|.
Then copy the file |childdoc.def| to an appropriate directory of your \LaTeX{}
distribution, e.g.\ \textit{texmf-root}|/tex/latex/childdoc|.
\end{itemize}

%%%%%%%%%%%%%%%%%%%%%%%%%%%%%%%%%%%%%%%%%%%%%%%%%%%%%%%%%%%%%%%%%%%%%%%%%%%%%%%%
\subsection{Related CTAN Packages}

There are several other packages which offer a similar functionality:
%
\begin{itemize}
\item
The packages
\href{http://ctan.org/pkg/docmute}{\textsf{docmute}},
\href{http://ctan.org/pkg/includex}{\textsf{includex}} and
\href{http://ctan.org/pkg/standalone}{\textsf{standalone}}
provide commands to include only the document body of
a child file thus allowing both files to be compiled individually.
\item
The packages \href{http://ctan.org/pkg/subdocs}{\textsf{subdocs}}
and \href{http://ctan.org/pkg/subfiles}{\textsf{subfiles}}
provide structures in which the main and child documents can be
encapsulated and allowing them to be compiled individually.
The inclusion mechanism is different from the conventional |\include|.
\item
The package \href{http://ctan.org/pkg/combine}{\textsf{combine}}
is an elaborate solution to combine several documents into one.
\end{itemize}
%
See also the CTAN topic \href{http://ctan.org/topic/subdocs}{\textsf{subdocs}}
for further related packages.
The present package differs from the above solutions in that
a document structure constructed with the conventional |\include| mechanism
just needs two extra commands at the top of every file
such that all constituent files can be compiled individually.

%%%%%%%%%%%%%%%%%%%%%%%%%%%%%%%%%%%%%%%%%%%%%%%%%%%%%%%%%%%%%%%%%%%%%%%%%%%%%%%%
%\subsection{Feature Suggestions}
%
%The following is a list of features which may be useful for future
%versions of this package:
%%
%\begin{itemize}
%\item
%\ldots
%\end{itemize}

%%%%%%%%%%%%%%%%%%%%%%%%%%%%%%%%%%%%%%%%%%%%%%%%%%%%%%%%%%%%%%%%%%%%%%%%%%%%%%%%
\subsection{Revision History}

%%%%%%%%%%%%%%%%%%%%%%%%%%%%%%%%%%%%%%%%
\paragraph{v2.0:} 2018/12/30

\begin{itemize}
\item
immediate forward processing
\item
added |\childdocby| mechanism
\item
manual restructured
\end{itemize}

%%%%%%%%%%%%%%%%%%%%%%%%%%%%%%%%%%%%%%%%
\paragraph{v1.6:} 2018/01/17

\begin{itemize}
\item
application for development of include files
\item
corrections to manual
\end{itemize}

%%%%%%%%%%%%%%%%%%%%%%%%%%%%%%%%%%%%%%%%
\paragraph{v1.5:} 2017/05/21

\begin{itemize}
\item
more complete structuring introduced
\item
|\childdocof| introduced
\item
|\childdoc| renamed to |\childdocmain|
\item
|\childredirect| renamed to |\childdocforward| and |\childdocforwardprefix|
and functionality expanded
\end{itemize}

%%%%%%%%%%%%%%%%%%%%%%%%%%%%%%%%%%%%%%%%
\paragraph{v1.0:} 2017/04/27

\begin{itemize}
\item
manual and install package
\item
first version published on CTAN
\end{itemize}

%%%%%%%%%%%%%%%%%%%%%%%%%%%%%%%%%%%%%%%%
\paragraph{v0.6:} 2017/04/26

\begin{itemize}
\item
redirection mechanism added
\end{itemize}

%%%%%%%%%%%%%%%%%%%%%%%%%%%%%%%%%%%%%%%%
\paragraph{v0.5:} 2017/04/26

\begin{itemize}
\item
functionality in definition file
\end{itemize}


%%%%%%%%%%%%%%%%%%%%%%%%%%%%%%%%%%%%%%%%%%%%%%%%%%%%%%%%%%%%%%%%%%%%%%%%%%%%%%%%
%%%%%%%%%%%%%%%%%%%%%%%%%%%%%%%%%%%%%%%%%%%%%%%%%%%%%%%%%%%%%%%%%%%%%%%%%%%%%%%%
%%%%%%%%%%%%%%%%%%%%%%%%%%%%%%%%%%%%%%%%%%%%%%%%%%%%%%%%%%%%%%%%%%%%%%%%%%%%%%%%
\appendix

\settowidth\MacroIndent{\rmfamily\scriptsize 000\ }

 \DocInput{childdoc.dtx}

\end{document}
%</driver>
% \fi
%
% %%%%%%%%%%%%%%%%%%%%%%%%%%%%%%%%%%%%%%%%%%%%%%%%%%%%%%%%%%%%%%%%%%%%%%%%%%%%%%
% %%%%%%%%%%%%%%%%%%%%%%%%%%%%%%%%%%%%%%%%%%%%%%%%%%%%%%%%%%%%%%%%%%%%%%%%%%%%%%
% \section{Sample}
%\iffalse
%<*samplemain>
%\fi
%
% The following presents a sample document
% with two chapters, two parts, a title page,
% a compile flag as well as three forwarding files to set the flag.
% It consists of eight |.tex| files:
% \begin{center}
% \begin{tabular}{ll}
% |cdocsamp.tex|&main file\\
% |cdocsch1.tex|&include file for chapter 1\\
% |cdocsch2.tex|&include file for chapter 2\\
% |cdocspt3.tex|&include file for part 3\\
% |cdocspt4.tex|&include file for part 4\\
% |cdocsdrf.tex|&forwarding file for main file in draft mode\\
% |cdocsfi1.tex|&forwarding file for final version of chapter 1\\
% |cdocsfi2.tex|&forwarding file for final version of chapter 2\\
% \end{tabular}
% \end{center}
% Each of the eight files can be compiled directly by the \LaTeX{} compiler.
%
% %%%%%%%%%%%%%%%%%%%%%%%%%%%%%%%%%%%%%%
% \paragraph{Main File.}
%
% The main file is called |cdocsamp.tex|.
%
% Load the \textsf{childdoc} definitions and
% declare the filename for the main document:
%    \begin{macrocode}
\input{childdoc.def}
\childdocmain{}
%    \end{macrocode}

% Optional override for |\version| flag:
%    \begin{macrocode}
%%\ifchilddoc\else\providecommand{\version}{draft}\fi
%    \end{macrocode}

% Define the default values for the |\version| flag
% (|final| for the main file and |draft| for childs):
%    \begin{macrocode}
\ifchilddoc
\providecommand{\version}{draft}
\else
\providecommand{\version}{final}
\fi
%    \end{macrocode}

% Load the standard document class:
%    \begin{macrocode}
\documentclass[12pt]{article}
%    \end{macrocode}

% Start the document body:
%    \begin{macrocode}
\begin{document}
%    \end{macrocode}

% Declare a title page.
% Print title, part of document being processed and version flag:
%    \begin{macrocode}
\addtocounter{page}{-1}
\begin{center}
{\LARGE\bfseries{}childdoc example\par}
\vspace{1cm}
\ifchilddoc
\ifchilddocmanual part\else chapter\fi:
`\childdocname' of `\childdocjob'\par
\else
main document: `\childdocjob'\par
\fi
version: \version\par
\end{center}
\newpage
%    \end{macrocode}

% Manually include selected file,
% otherwise process as usual:
%    \begin{macrocode}
\ifchilddocmanual
\section*{part `\childdocname'}
\input{\childdocname}
\else
%    \end{macrocode}

% Include the two chapters:
%    \begin{macrocode}
\include{cdocsch1}
\include{cdocsch2}
%    \end{macrocode}

% Include the two parts unless only chapters should be displayed:
%    \begin{macrocode}
\ifchilddoc\else
\section{part three}
\input{cdocspt3}
\section{part four}
\input{cdocspt4}
\fi
%    \end{macrocode}

% Process as usual until here:
%    \begin{macrocode}
\fi
%    \end{macrocode}

% End of document body:
%    \begin{macrocode}
\end{document}
%    \end{macrocode}
%\iffalse
%</samplemain>
%\fi
%
% %%%%%%%%%%%%%%%%%%%%%%%%%%%%%%%%%%%%%%
% \paragraph{Chapter Include Files.}
%
% The include files are called |cdocsch1.tex| and |cdocsch2.tex|.
%
%\iffalse
%<*samplechap1|samplechap2>
%\fi

% Optional override for |\version| flag:
%    \begin{macrocode}
%%\providecommand{\version}{final}
%    \end{macrocode}

% Include the main document:
%    \begin{macrocode}
\input{childdoc.def}
\childdocof{cdocsamp}
%    \end{macrocode}

%\iffalse
%</samplechap1|samplechap2>
%\fi
%
%\iffalse
%<*samplechap1>
%\fi
% Some text for chapter 1:
%    \begin{macrocode}
\section{one}
some text in chapter one
%    \end{macrocode}

%\iffalse
%</samplechap1>
%\fi
% Some text for chapter 2:
%\iffalse
%<*samplechap2>
%\fi
%    \begin{macrocode}
\section{two}
more text in chapter two
%    \end{macrocode}

%\iffalse
%</samplechap2>
%\fi
%
% %%%%%%%%%%%%%%%%%%%%%%%%%%%%%%%%%%%%%%
% \paragraph{Part Include Files.}
%
% The include files are called |cdocspt3.tex| and |cdocspt4.tex|.
%
%\iffalse
%<*samplepart3|samplepart4>
%\fi

% Optional override for |\version| flag:
%    \begin{macrocode}
%%\providecommand{\version}{final}
%    \end{macrocode}

% Include the main document:
%    \begin{macrocode}
\input{childdoc.def}
\childdocby{cdocsamp}
%    \end{macrocode}

%\iffalse
%</samplepart3|samplepart4>
%\fi
%
%\iffalse
%<*samplepart3>
%\fi
% Some text for part 3:
%    \begin{macrocode}
some text in part three
%    \end{macrocode}

%\iffalse
%</samplepart3>
%\fi
% Some text for part 4:
%\iffalse
%<*samplepart4>
%\fi
%    \begin{macrocode}
more text in part four
%    \end{macrocode}

%\iffalse
%</samplepart4>
%\fi
%
% %%%%%%%%%%%%%%%%%%%%%%%%%%%%%%%%%%%%%%
% \paragraph{Forwarding for a Complete Draft.}
%
% The following forwarding file |cdocsdrf.tex|
% compiles the main document in draft mode:
%\iffalse
%<*sampledraft>
%\fi
%    \begin{macrocode}
\def\version{draft}
\input{childdoc.def}
\childdocforward{cdocsamp}
%    \end{macrocode}

%\iffalse
%</sampledraft>
%\fi
%
% %%%%%%%%%%%%%%%%%%%%%%%%%%%%%%%%%%%%%%
% \paragraph{Forwarding for Final Version of the Chapters.}
%
% The following forwarding files |cdocsfn1.tex| and |cdocsfn2.tex|
% (with identical content)
% compile the final versions of the child documents
% |cdocsch1.tex| and |cdocsch2.tex|, respectively:
%\iffalse
%<*samplefinal>
%\fi
%    \begin{macrocode}
\def\version{final}
\input{childdoc.def}
\childdocforwardprefix[cdocsamp]{cdocsfn}{cdocsch}
%    \end{macrocode}

%\iffalse
%</samplefinal>
%\fi
%
% %%%%%%%%%%%%%%%%%%%%%%%%%%%%%%%%%%%%%%
% \paragraph{Command Line Processing.}
%
% The following three command lines generate the output files
% |cdocscld|, |cdocscl1| and |cdocscl2|
% which should be identical to
% |cdocsdrf|, |cdocsch1| and |cdocsfn2|, respectively:
% \begin{center}
% \begin{tabular}{l}
% |latex -jobname cdocscld \|\\
% |  "\def\version{draft}\input{childdoc.def}\childdocforward{cdocsamp}"|\\
% |latex -jobname cdocscl1 \|\\
% |  "\input{childdoc.def}\childdocforward[cdocsamp]{cdocsch1}"|\\
% |latex -jobname cdocscl2 \|\\
% |  "\def\version{final}\input{childdoc.def}\childdocforward{cdocsch2}"|
% \end{tabular}
% \end{center}
% Note that the trailing backslash on each first line
% merely continues the input to the second line
% (for convenient cut ant paste).
% Furthermore, the command |latex| can be replaced by any
% of its alternative versions such as |pdflatex|.
%
% %%%%%%%%%%%%%%%%%%%%%%%%%%%%%%%%%%%%%%%%%%%%%%%%%%%%%%%%%%%%%%%%%%%%%%%%%%%%%%
% %%%%%%%%%%%%%%%%%%%%%%%%%%%%%%%%%%%%%%%%%%%%%%%%%%%%%%%%%%%%%%%%%%%%%%%%%%%%%%
% \section{Implementation}
%\iffalse
%<*package>
%\fi
%
% This section describes the definitions file |childdoc.def|.

% The definitions cannot be loaded using |\usepackage| or |\RequirePackage|
% which has a mechanism to prevent loading a style file more than once.
% When loading the definitions by means of |\input|
% multiple instances have to be prevented manually:
%\iffalse
%This code needs to be before the `\ProvidesFile' directive
%which is defined at the beginning of this file.
%Therefore it is also placed there and commented out here.
%</package>
%<*discard>
%\fi
%    \begin{macrocode}
\ifdefined\childdocmain\endinput\fi
%    \end{macrocode}
%\iffalse
%</discard>
%<*package>
%\fi
%
% \macro{\ifchilddoc}
% \macro{\ifchilddocmanual}
% The conditional |\ifchilddoc| tells whether a
% child (true) or main (false) document is being compiled.
% The conditional |\ifchilddocmanual| tells whether
% the |\includeonly| mechanism is used (false) or
% the selection of child files must be performed manually (true).
% The definitions initialise to false:
%    \begin{macrocode}
\newif\ifchilddoc
\newif\ifchilddocmanual
%    \end{macrocode}

% \macro{\childdocname}
% \macro{\childdocjob}
% The macro |\childdocname| stores the name of the main document
% to be compiled. The macro |\childdocjob| stores the name of
% the document on which the \LaTeX{} compiler was originally invoked.
% The content of |\jobname| cannot be compared
% to filenames specified in the source due to different catcodes.
% The following code rescans |\jobname|, stores the result
% in |\childdocname| and saves a copy in |\childdocjob|:
%    \begin{macrocode}
\edef\childdocname{\scantokens\expandafter{\jobname\noexpand}}
\let\childdocjob\childdocname
%    \end{macrocode}

% \macro{\childdocdisable}
% The macro |\childdocdisable| prevents the main file
% from being processed more than once.
% At this stage, the main document command |\childdocmain|
% is assumed to be called once again where it should do nothing.
% Any subsequent call to it should prevent
% a secondary processing of the main document
% It overwrites the forwarding commands
% |\childdocof| and |\childdocforward|
% with empty macros to prevent further inclusions of the main document:
%    \begin{macrocode}
\newcommand{\childdocdisable}
{
  \renewcommand{\childdocmain}[1]{\renewcommand{\childdocmain}[1]{\endinput}}
  \renewcommand{\childdocof}[1]{}
  \renewcommand{\childdocby}[2][]{}
  \renewcommand{\childdocforward}[2][]{}
  \renewcommand{\childdocdisable}{}
}
%    \end{macrocode}

% \macro{\childdocmain}
% The macro |\childdocmain| is to be called at the top of the main file
% with nothing or the main filename (without extension) as argument.
% First, it breaks loops.
% If the argument is not empty and does not match |\childdocname|
% (which is set by the first inclusion of |childdoc.def|),
% |\ifchilddoc| is set to true, |\includeonly| is applied to the child file
% and |\jobname| is set to the main file
% (for proper handling of |.aux| files):
%    \begin{macrocode}
\newcommand{\childdocmain}[1]
{
  \childdocdisable\childdocmain{}
  \if?#1?\else
    \begingroup
      \def\childdoctmp{#1}
      \ifx\childdoctmp\childdocname
        \def\childdoctmp{}
      \else
        \def\childdoctmp
        {
          \childdoctrue
          \includeonly{\childdocname}
          \def\childdocjob{#1}
          \def\jobname{#1}
        }
      \fi
      \expandafter
    \endgroup
    \childdoctmp
  \fi
}
%    \end{macrocode}

% \macro{\childdocof}
% The command |\childdocof| redirects
% compilation to the main file |#1|.
%    \begin{macrocode}
\newcommand{\childdocof}[1]
{
  \childdocdisable
  \childdoctrue
  \includeonly{\childdocname}
  \def\jobname{#1}
  \def\childdocjob{#1}
  \input{#1}
}
%    \end{macrocode}

% \macro{\childdocby}
% The command |\childdocby| ....
%    \begin{macrocode}
\newcommand{\childdocby}[2][]
{
  \childdocdisable
  \childdoctrue
  \childdocmanualtrue
  \if?#1?\else
    \def\jobname{#2}
  \fi
  \def\childdocjob{#2}
  \input{#2}
  \endinput
}
%    \end{macrocode}

% \macro{\childdocforward}
% The command |\childdocforward| redirects
% compilation to the main file or
% (if the optional argument is given) a child file.
% Parameters are set as if the main file
% or a child file starting with |\childdocof| was compiled.
% Then compilation is handed over to the main file:
%    \begin{macrocode}
\newcommand{\childdocforward}[2][]
{
  \begingroup
    \if?#1?
      \def\childdoctmp
      {
        \def\childdocname{#2}
        \def\childdocjob{#2}
        \def\jobname{#2}
        \input{#2}
        \endinput
      }
    \else
      \def\childdoctmp
      {
        \childdocdisable
        \def\childdocname{#2}
        \childdoctrue
        \includeonly{#2}
        \def\childdocjob{#1}
        \def\jobname{#1}
        \input{#1}
        \endinput
      }
    \fi
    \expandafter
  \endgroup
  \childdoctmp
}
%    \end{macrocode}

% \macro{\childdocforwardprefix}
% The command |\childdocforwardprefix| redirects
% compilation to the main or a child file by means of a pattern.
% The prefix |#1| in the current filename is replaced by |#2|
% and the suffix of the current filename is kept
% (it is assumed that the filename does not contain the substring `|~~~|'
% which is used as a delimiter).
% Compilation is handed over to the new file by |\childdocforward|:
%    \begin{macrocode}
\newcommand{\childdocforwardprefix}[3][]
{
  \begingroup
    \def\childdocextract #2##1~~~{\def\childdoctmp{\childdocforward[#1]{#3##1}}}
    \expandafter\childdocextract\childdocname~~~
    \expandafter
  \endgroup
  \childdoctmp
}
%    \end{macrocode}

% \macro{\childdoc}
% The deprecated macro |\childdoc| is a legacy version of |\childdocmain|:
%    \begin{macrocode}
\newcommand{\childdoc}{\childdocmain}
%    \end{macrocode}

% \macro{\childdocredirect}
% The deprecated macro |\childdocredirect| is a legacy version
% of |\childdocforward| and |\childdocforwardprefix|:
%    \begin{macrocode}
\newcommand{\childdocredirect}[2][]
{
  \begingroup
    \if?#1?
      \def\childdoctmp{\childdocforward{#2}}
    \else
      \def\childdoctmp{\childdocforwardprefix{#1}{#2}}
    \fi
    \expandafter
  \endgroup
  \childdoctmp
}
%    \end{macrocode}

%\iffalse
%</package>
%\fi
%
\endinput
|\\
|\childdocmain{|\textit{main}|}|\\
\end{tabular}
\end{center}
%
If |\jobname| does not match the argument \textit{main} of |\childdocmain|,
it is assumed that |\jobname| points to the child file to be compiled.
When using |\childdocmain| with the main file specified as argument,
it suffices to start a child file
with just |\input{|\textit{main}|}|
without loading of the package and using |\childdocof|.
If instead all processing is done
with the appropriate \textsf{childdoc} directives,
the argument of \textit{main} of |\childdocmain| can be empty.

An alternative version of the command line processing described
in \secref{sec:commandline} using the detection mechanism reads:
%
\begin{center}
|... -jobname "|\textit{target}|" "|[\textit{flags}]%
[|\def\jobname{|\textit{dest}|}|]|\input{|\textit{main}|}"|
\end{center}

%%%%%%%%%%%%%%%%%%%%%%%%%%%%%%%%%%%%%%%%%%%%%%%%%%%%%%%%%%%%%%%%%%%%%%%%%%%%%%%%
\subsection{Manual Code}
\label{sec:manual}

In case one cannot be certain whether the definitions file |childdoc.def|
is installed on the target \TeX{} distribution
and one prefers not to ship it,
it is conceivable to paste a few relevant commands into the sources.

To that end, drop all statements |% \iffalse
%
% childdoc.dtx Copyright (C) 2017-2018 Niklas Beisert
%
% This work may be distributed and/or modified under the
% conditions of the LaTeX Project Public License, either version 1.3
% of this license or (at your option) any later version.
% The latest version of this license is in
%   http://www.latex-project.org/lppl.txt
% and version 1.3 or later is part of all distributions of LaTeX
% version 2005/12/01 or later.
%
% This work has the LPPL maintenance status `maintained'.
%
% The Current Maintainer of this work is Niklas Beisert.
%
% This work consists of the files childdoc.dtx and childdoc.ins
% and the derived files childdoc.def and cdocsamp.tex with
% cdocsch1.tex, cdocsch2.tex, cdocsdrf.tex, cdocsfn1.tex, cdocsfn2.tex.
%
%<package>\ifdefined\childdocmain\endinput\fi
%<package>\ProvidesFile{childdoc.def}[2018/12/30 v2.0 child document driver]
%<samplemain>\ProvidesFile{cdocsamp.tex}[2018/12/30 v2.0 sample for childdoc]
%<*driver>
%\ProvidesFile{childdoc.drv}[2018/12/30 v2.0 childdoc reference manual file]
\PassOptionsToClass{10pt,a4paper}{article}
\documentclass{ltxdoc}

\usepackage[margin=35mm]{geometry}
\usepackage{hyperref}
\usepackage{hyperxmp}
\usepackage[usenames]{color}

\hypersetup{colorlinks=true}
\hypersetup{pdfstartview=FitH}
\hypersetup{pdfpagemode=UseNone}
\hypersetup{pdfsource={}}
\hypersetup{pdflang={en-UK}}
\hypersetup{pdfcopyright={Copyright 2017-2018 Niklas Beisert.
  This work may be distributed and/or modified under the
  conditions of the LaTeX Project Public License, either version 1.3
  of this license or (at your option) any later version.}}
\hypersetup{pdflicenseurl={http://www.latex-project.org/lppl.txt}}
\hypersetup{pdfcontactaddress={ETH Zurich, ITP, HIT K,
  Wolfgang-Pauli-Strasse 27}}
\hypersetup{pdfcontactpostcode={8093}}
\hypersetup{pdfcontactcity={Zurich}}
\hypersetup{pdfcontactcountry={Switzerland}}
\hypersetup{pdfcontactemail={nbeisert@itp.phys.ethz.ch}}
\hypersetup{pdfcontacturl={http://people.phys.ethz.ch/\xmptilde nbeisert/}}

\newcommand{\secref}[1]{\hyperref[#1]{section \ref*{#1}}}

\parskip1ex
\parindent0pt
\let\olditemize\itemize
\def\itemize{\olditemize\parskip0pt}

\begin{document}

\title{The \textsf{childdoc} Package}
\hypersetup{pdftitle={The childdoc Package}}
\author{Niklas Beisert\\[2ex]
  Institut f\"ur Theoretische Physik\\
  Eidgen\"ossische Technische Hochschule Z\"urich\\
  Wolfgang-Pauli-Strasse 27, 8093 Z\"urich, Switzerland\\[1ex]
  \href{mailto:nbeisert@itp.phys.ethz.ch}
  {\texttt{nbeisert@itp.phys.ethz.ch}}}
\hypersetup{pdfauthor={Niklas Beisert}}
\hypersetup{pdfsubject={Manual for the LaTeX2e Package childdoc}}
\date{30 December 2018, \textsf{v2.0}}
\maketitle

\begin{abstract}\noindent
\textsf{childdoc} is a \LaTeXe{} package
that enables the direct compilation
of document sections included by |\include|
to individual files.
\end{abstract}

\begingroup
\parskip0ex
\tableofcontents
\endgroup

%%%%%%%%%%%%%%%%%%%%%%%%%%%%%%%%%%%%%%%%%%%%%%%%%%%%%%%%%%%%%%%%%%%%%%%%%%%%%%%%
%%%%%%%%%%%%%%%%%%%%%%%%%%%%%%%%%%%%%%%%%%%%%%%%%%%%%%%%%%%%%%%%%%%%%%%%%%%%%%%%
\section{Introduction}

\LaTeX{} provides a mechanism to structure a large document (such as a book)
into a main file and several child files (containing the chapters)
using the |\include| command.
This mechanism is beneficial for documents
which span hundreds of pages in order to
make the source file(s) more manageable.
Moreover, compilation can be restricted to
selected child files by means of the |\includeonly| command.
The latter feature can be used to reduce the compilation time while editing
(this was significantly more useful in the earlier days of \LaTeX{})
or to generate a smaller document which is easier to navigate.
Another application of |\includeonly| is to generate
documents consisting of selected parts of the complete document.

However, there are a few drawbacks of the plain |\include| mechanism:
\begin{itemize}
\item
The child files cannot be compiled on their own,
they can only be compiled via the main file.
A naive editing environment
(such as a text editor with an option
to have the current file processed by \LaTeX)
may require one to switch to the main file before compiling;
attempting to compile the child file produces errors.
\item
The main file must be modified (each time)
to adjust the |\includeonly| command
to the present needs. This easily leaves the main file in a messy state.
\item
The generated document will always carry the filename
of the main document. This is inconvenient if
several child files are to be compiled and
to be kept for distribution.
\end{itemize}

The present package provides a simple interface
to make child files individually compilable by \LaTeX{}.
Compiling a child file then has the same effect as compiling
the main file with an |\includeonly| command
to select the appropriate child.
Moreover the generated document will carry the name of the child
rather than the main file.
This resolves all three above issues.

This feature is meant to make the editing of books,
thesis documents and lecture notes somewhat more convenient.
However, the package can also be used efficiently for
composing a series of documents (such as exercise sheets)
which are typically distributed individually.
It then assists the author in generating the individual documents
(potentially in different versions)
as well as a document containing the collected series.
Another application is in developing style files
or other kinds of included material
where compilation of the style file could redirect
to a sample or test file.

%%%%%%%%%%%%%%%%%%%%%%%%%%%%%%%%%%%%%%%%%%%%%%%%%%%%%%%%%%%%%%%%%%%%%%%%%%%%%%%%
%%%%%%%%%%%%%%%%%%%%%%%%%%%%%%%%%%%%%%%%%%%%%%%%%%%%%%%%%%%%%%%%%%%%%%%%%%%%%%%%
\section{Usage}

First of all, the package \textsf{childdoc} is \emph{not} a standard
\LaTeXe{} |.sty| style file! Therefore it needs to be invoked in
a non-standard way.

%%%%%%%%%%%%%%%%%%%%%%%%%%%%%%%%%%%%%%%%%%%%%%%%%%%%%%%%%%%%%%%%%%%%%%%%%%%%%%%%
\subsection{Included Files}
\label{sec:include}

%%%%%%%%%%%%%%%%%%%%%%%%%%%%%%%%%%%%%%%%
\DescribeMacro{\childdocmain}
To use the package, add the commands
\begin{center}
\begin{tabular}{l}
|\input{childdoc.def}|\\
|\childdocmain{}|\\
\end{tabular}
\end{center}
at the very top of the main \LaTeX{} file,
in particular \emph{before} the |\documentclass| statement!
The argument of |\childdocmain| should be left empty
(but it must be present).

%%%%%%%%%%%%%%%%%%%%%%%%%%%%%%%%%%%%%%%%
\DescribeMacro{\childdocof}
Furthermore, add the commands
\begin{center}
\begin{tabular}{l}
|\input{childdoc.def}|\\
|\childdocof{|\textit{main}|}|\\
\end{tabular}
\end{center}
at the top of every child file \textit{child}
which is included by |\include{|\textit{child}|}|
from within the main file
(or at least for those files to be compiled individually).
The argument \textit{main} must be the filename of the main file.

There are a couple of
considerations in setting up the main and child documents:

%%%%%%%%%%%%%%%%%%%%%%%%%%%%%%%%%%%%%%%%
\paragraph{Restrictions.}

Please note the following restrictions:
\begin{itemize}
\item
|\childdocmain| must be called with one argument \textit{main}
to ensure compatibility with earlier version of the package.
It must either be empty (|\childdocmain{}|)
or precisely match the filename of the main file in which it is specified.
See \secref{sec:detection} for further information.
\item
The filename \textit{main} must be specified without the |.tex| extension.
\item
The filename \textit{main} is case sensitive
(even in case-insensitive file systems)
due to internal string comparison.
\item
The argument \textit{main} should be fully expanded, it cannot be a macro.
\item
Subdirectories and special characters should be avoided in filenames.
\item
The command |\childdocmain{|\textit{main}|}| must be followed by a whitespace.
It should not be followed immediately by another command
or by a comment mark `|%|'.
This is because the \TeX{} parser reads the token immediately following
the argument of |\childdocmain| and puts it
at the beginning of every child section;
however, a white\-space is ignored.
\end{itemize}

%%%%%%%%%%%%%%%%%%%%%%%%%%%%%%%%%%%%%%%%
\paragraph{Content of Main File.}

It is advisable to place all content in the child files included by |\include|.
Any output contained in the main file will appear in all child documents
unless suppressed manually;
it cannot be suppressed automatically by the |\includeonly| directive
and thus should normally be avoided.
A method to include some content in the main file
by means of conditional processing is described in \secref{sec:conditional}.

%%%%%%%%%%%%%%%%%%%%%%%%%%%%%%%%%%%%%%%%
\paragraph{Page Numbering.}

When only a part of the document is compiled,
the appropriate numbering of pages
(as well as other status parameters)
is determined from the |.aux| files.
The latter contain information from previous passes.
However this information needs to propagate through
all intermediate child documents.
Therefore the page numbering in child documents may well
be inconsistent until the complete document is compiled at least once.

A useful (if unconventional) way to always ensure a consistent
page numbering is to restart the numbering in each child document
and denote the pages by `\textit{child}|.|\textit{page}'
where \textit{child} represents the chapter/section number of the child file.
This can be achieved by the command
|\numberwithin{page}{|\textit{child}|}|
of the \textsf{amsmath} package
where \textit{child} can be |chapter| or |section|
depending on the chosen structuring.
Alternatively, one can modify the macro |\thepage| appropriately
and reset the counter |page| at the start of each child file.

%%%%%%%%%%%%%%%%%%%%%%%%%%%%%%%%%%%%%%%%%%%%%%%%%%%%%%%%%%%%%%%%%%%%%%%%%%%%%%%%
\subsection{Conditional Processing}
\label{sec:conditional}

The package provides a mechanism to compile different versions
of a document. To customise the versions further some conditional processing
can come in handy to distinguish which version is being compiled.
The package provides two macros to describe the compilation context:

%%%%%%%%%%%%%%%%%%%%%%%%%%%%%%%%%%%%%%%%
\DescribeMacro{\ifchilddoc}
The conditional |\ifchilddoc| distinguishes between the compilation of
child documents and the main document:
%
\begin{center}
|\ifchilddoc |\textit{child-code}| |[|\||else |\textit{main-code}]| \||fi|
\end{center}

%%%%%%%%%%%%%%%%%%%%%%%%%%%%%%%%%%%%%%%%
\DescribeMacro{\childdocname}
\DescribeMacro{\childdocjob}
The macro |\childdocname| contains the filename (without extension)
of the main or child file being processed.
Note that |\childdocjob| will always contain the name of the main file.

%%%%%%%%%%%%%%%%%%%%%%%%%%%%%%%%%%%%%%%%
\paragraph{Title Page.}

Conditional processing can be used to include a title or banner page
in the main document when proper precautions are taken.
Importantly, the code in the main file should ensure that the page counter
(as well as other status parameters which are stored in the |.aux| files)
takes the same value after the conditional processing.
Otherwise the page numbers may take divergent values
depending on which part is compiled.

For example, a title page could be declared by:
%
\begin{center}
\begin{tabular}{l}
|\ifchilddoc\||else|\\
|\addtocounter{page}{-1}|\\
\textit{code for title page}\\
|\newpage|\\
|\||fi|
\end{tabular}
\end{center}
%
A banner page for the child documents can be generated by:
%
\begin{center}
\begin{tabular}{l}
|\ifchilddoc|\\
|\addtocounter{page}{-1}|\\
\textit{code for banner page}\\
|\newpage|\\
|\||fi|
\end{tabular}
\end{center}
%
Here one could write a message such as:
\begin{center}
|This is the part \childdocname{} of \childdocjob{}.|
\end{center}

%%%%%%%%%%%%%%%%%%%%%%%%%%%%%%%%%%%%%%%%%%%%%%%%%%%%%%%%%%%%%%%%%%%%%%%%%%%%%%%%
\subsection{Flags}
\label{sec:flags}

The package makes it easy to generate different versions
of the main or child documents.
To this end compilation flags can be defined
and assigned different default values.
They will be particularly useful in conjunction
with the forwarding mechanism described in \secref{sec:forward}.

For example, it may be useful to have a flag |\version|
which can be set to |draft| or |final|.
The document source will contain some conditional code
depending on the value of |\version|.
Suppose further, the flag should default to |final| for the main file
and to |draft| for child files
which is a natural assignment for editing the document.
This is achieved by placing the following code
in the preamble of the main document
(below the |\childdocmain| directive):
%
\begin{center}
\begin{tabular}{l}
|\ifchilddoc|\\
|\providecommand{\version}{draft}|\\
|\||else|\\
|\providecommand{\version}{final}|\\
|\||fi|
\end{tabular}
\end{center}
%
The definition by |\providecommand| makes sure
that previous definitions are not overwritten.
Further statements |\providecommand{\version}{...}|
can thus be added before the above code to override it.

For the main file, one might add a line
(between |\childdocmain| and the above block)
%
\begin{center}
|%\ifchilddoc\||else\providecommand{\version}{draft}\||fi|
\end{center}
%
which can be uncommented to produce a draft version.
Likewise one can add a line to the very top of a child file
(above the |\childdocof{|\textit{main}|}| directive)
%
\begin{center}
|%\providecommand{\version}{final}|
\end{center}
%
which can be uncommented to produce the final version of this child document.

%%%%%%%%%%%%%%%%%%%%%%%%%%%%%%%%%%%%%%%%%%%%%%%%%%%%%%%%%%%%%%%%%%%%%%%%%%%%%%%%
\subsection{Forwarding}
\label{sec:forward}

Different versions of the main or child documents
using compilation flags as described in \secref{sec:flags}
can be (permanently) stored in different files
for convenient compilation, viewing and distribution.
To this end, the package defines a command
to pass on compilation to a different file:

%%%%%%%%%%%%%%%%%%%%%%%%%%%%%%%%%%%%%%%%
\DescribeMacro{\childdocforward}
The command |\childdocforward| redirects processing to
another source file:
%
\begin{center}
\begin{tabular}{l}
|\input{childdoc.def}|\\
|\childdocforward[|\textit{main}|]{|\textit{dest}|}|\\
\end{tabular}
\end{center}
%
The argument \textit{dest} is the destination file
(without extension).
It should be the main file or one of the child files.
Note that further \textsf{childdoc} directives
such as |\childdocof| and |\childdocforward|
in the indicated file will be processed in this form.
The optional argument \textit{main}
passes on directly to the main file \textit{main}
while pretending to compile the child \textit{dest}.
This form behaves as if \textit{dest}
issues |\childdocof{|\textit{main}|}| right away,
and no further \textsf{childdoc} directives will be processed.

%%%%%%%%%%%%%%%%%%%%%%%%%%%%%%%%%%%%%%%%
\DescribeMacro{\...prefix}
In the alternative form |\childdocforwardprefix|,
%
\begin{center}
\begin{tabular}{l}
|\input{childdoc.def}|\\
|\childdocforwardprefix[|\textit{main}|]{|\textit{prefix}|}{|\textit{dest}|}|
\end{tabular}
\end{center}
%
the destination file is determined by a pattern
depending on the current file:
To make this work, the current file must be called
`{\textit{prefix}\hspace{0.2em}\textit{suffix}}'
with \textit{prefix} matching precisely the argument.
Processing is then passed on to the file
`{\textit{dest}\hspace{0.2em}\textit{suffix}}'.
Surely, the same effect is achieved by
directly specifying the
argument `{\textit{dest}\hspace{0.2em}\textit{suffix}}'
in the first form.
However, that requires to set up a different file
for each child. With the alternative form of the command
all these files can have exactly the same content
which simplifies setting them up and maintaining them.

For example, the following file |draft.tex|
with a compilation flag |\version| as described in \secref{sec:flags}
compiles the main document as a draft:
%
\begin{center}
\begin{tabular}{l}
|\def\version{draft}|\\
|\input{childdoc.def}|\\
|\childdocforward{|\textit{main}|}|
\end{tabular}
\end{center}
%
Likewise, the following files |final|\textit{nn}|.tex|
compile the final version of the child document
|child|\textit{nn}|.tex|:
%
\begin{center}
\begin{tabular}{l}
|\def\version{final}|\\
|\input{childdoc.def}|\\
|\childdocforwardprefix{final}{child}|
\end{tabular}
\end{center}
%

Note that when several versions of a main file and/or of each child file
are to be generated, it may be convenient to set up a |Makefile| or
shell script to automatise the process.

%%%%%%%%%%%%%%%%%%%%%%%%%%%%%%%%%%%%%%%%%%%%%%%%%%%%%%%%%%%%%%%%%%%%%%%%%%%%%%%%
\subsection{Command Line Processing}
\label{sec:commandline}

The effect of redirection files can also be achieved by invoking
the \LaTeX{} compiler with a more elaborate command line.
Most conveniently this should be done as part
of a shell script or a |Makefile|.

When using \textsf{childdoc} in the main file, the following
command lines effectively perform a redirection
(note that depending on the shell being used,
backslashes may have to be doubled: `|\|' $\to$ `|\\|'):
%
\begin{center}
|... -jobname "|\textit{target}|" |\\|"|[\textit{flags}]%
|\input{childdoc.def}\childdocforward[|\textit{main}|]{|\textit{dest}|}"|
\end{center}
%
Here \textit{target} is the name of the output file,
\textit{main} is the name of the main file
and \textit{dest} is the name of the main or child file to be processed
(all filenames without extensions).
The optional argument \textit{main} can be omitted
if \textit{main} matches \textit{dest}.
Optionally, compilation \textit{flags} can be defined via |\def| commands.
This command line makes the \TeX{} engine believe
it is compiling the file \textit{target}
whose content is specified as the latter parameter.
The provided code then forwards the processing to
\textit{main} or \textit{dest} as described in \secref{sec:forward}.

%%%%%%%%%%%%%%%%%%%%%%%%%%%%%%%%%%%%%%%%%%%%%%%%%%%%%%%%%%%%%%%%%%%%%%%%%%%%%%%%
\subsection{Include by Input}
\label{sec:input}

Including child documents by |\include| has some restrictions by design.
Most notably, the content of a child document always occupies
its own set of pages; pages cannot be shared between child documents.
Usually, this behaviour makes perfect sense
because each child document contain an essential part of the document.
However, in some situations it may be desirable to compose
a document from a collection of parts
without having mandatory page breaks between then.
For this case, the package
provides a mechanism to include parts
by |\input| which can also be processed individually.
However, by construction this mechanism
requires manual handling of the content to be output.

%%%%%%%%%%%%%%%%%%%%%%%%%%%%%%%%%%%%%%%%
\DescribeMacro{\ifchilddocmanual}
The main file should be prepared as usual, see \secref{sec:include}.
However, the document body must make a distinction
between processing of an individual part and of the main document, e.g.:
%
\begin{center}
\begin{tabular}{l}
|\ifchilddocmanual|\\
|\input{\childdocname}|\\
|\||else|\\
\textit{document body with }|\input{|\textit{part}|}|\\
|\||fi|
\end{tabular}
\end{center}
%
The conditional |\ifchilddocmanual| is true whenever
a part to be included by |\input| is being compiled,
and the name of the part is stored in |\childdocname|.

%%%%%%%%%%%%%%%%%%%%%%%%%%%%%%%%%%%%%%%%
\DescribeMacro{\childdocby}
Each part to be included by |\input| should start with:
%
\begin{center}
\begin{tabular}{l}
|\input{childdoc.def}|\\
|\childdocby{|\textit{main}|}|\\
\end{tabular}
\end{center}
%
The directive |\childdocby| is similar to |\childdocof|
described in \secref{sec:include},
but the subsequent selection of content must be done manually.
To that end, both |\ifchilddoc| and |\ifchilddocmanual|
will be true upon processing of a part,
and the name of the part is stored in |\childdocname|.
Note that |\jobname| will be set to the filename of the current part
so that each part receives an individual |.aux| file
that does not interfere with the |.aux| file(s) of the main document.
This behaviour can be altered by the alternative form
|\childdocby[*]{|\textit{main}|}| (with a non-empty optional argument)
which uses the |.aux| file of the main document
by setting |\jobname| to \textit{main}.

%%%%%%%%%%%%%%%%%%%%%%%%%%%%%%%%%%%%%%%%%%%%%%%%%%%%%%%%%%%%%%%%%%%%%%%%%%%%%%%%
\subsection{Driver Development}
\label{sec:driver}

The \textsf{childdoc} mechanism can also be use for the development
of definition files such as \LaTeX{} styles or classes.
This case differs from the above setup with multiple parts
included by |\include| in that no |\includeonly| should be invoked.
This can be achieved by starting the include file
(before |\ProvidesPackage|) with:
%
\begin{center}
\begin{tabular}{l}
|\input{childdoc.def}|\\
|\childdocforward{|\textit{main}|}|\\
\end{tabular}
\end{center}
%
or alternatively with:
%
\begin{center}
\begin{tabular}{l}
|\input{childdoc.def}|\\
|\childdocby{|\textit{main}|}|\\
\end{tabular}
\end{center}
%
Both forms have slightly different effects as described above.
The main file is prepared as usual, see \secref{sec:include}.

%%%%%%%%%%%%%%%%%%%%%%%%%%%%%%%%%%%%%%%%%%%%%%%%%%%%%%%%%%%%%%%%%%%%%%%%%%%%%%%%
\subsection{Legacy Detection}
\label{sec:detection}

The directive |\childdocmain| in the main file can detect
whether the complete document or merely a child is to be compiled
even without using the directive |\childdocof|.
This method is deprecated because it is less robust
and there is no compelling reason to use it;
it is merely provided for backward compatibility
and it may be removed in future versions.

If the detection mechanism is to be used,
it is mandatory to correctly specify
the filename of the main file as the argument of |\childdocmain|:
%
\begin{center}
\begin{tabular}{l}
|\input{childdoc.def}|\\
|\childdocmain{|\textit{main}|}|\\
\end{tabular}
\end{center}
%
If |\jobname| does not match the argument \textit{main} of |\childdocmain|,
it is assumed that |\jobname| points to the child file to be compiled.
When using |\childdocmain| with the main file specified as argument,
it suffices to start a child file
with just |\input{|\textit{main}|}|
without loading of the package and using |\childdocof|.
If instead all processing is done
with the appropriate \textsf{childdoc} directives,
the argument of \textit{main} of |\childdocmain| can be empty.

An alternative version of the command line processing described
in \secref{sec:commandline} using the detection mechanism reads:
%
\begin{center}
|... -jobname "|\textit{target}|" "|[\textit{flags}]%
[|\def\jobname{|\textit{dest}|}|]|\input{|\textit{main}|}"|
\end{center}

%%%%%%%%%%%%%%%%%%%%%%%%%%%%%%%%%%%%%%%%%%%%%%%%%%%%%%%%%%%%%%%%%%%%%%%%%%%%%%%%
\subsection{Manual Code}
\label{sec:manual}

In case one cannot be certain whether the definitions file |childdoc.def|
is installed on the target \TeX{} distribution
and one prefers not to ship it,
it is conceivable to paste a few relevant commands into the sources.

To that end, drop all statements |\input{childdoc.def}|
and perform the replacements as outlined below.
Instead of |\childdocmain{|\textit{main}|}| add the following code
to the top of the main file:
%
\begin{center}
\begin{tabular}{l}
|\||ifdefined\childdocname\endinput\||fi\newif\ifchilddoc|\\
|\edef\childdocname{\scantokens\expandafter{\jobname\noexpand}}|\\
|\def\childdocmain{|\textit{main}|}\||ifx\childdocmain\childdocname\||else|\\
|\childdoctrue\includeonly{\childdocname}\let\jobname\childdocmain\||fi|\\
\end{tabular}
\end{center}
%
Instead of |\childdocof{|\textit{main}|}| just include the main file
at the top of each child file:
%
\begin{center}
|\input{|\textit{main}|}|
\end{center}
%
A simple redirection |\childdocforward{|\textit{dest}|}| is achieved by:
%
\begin{center}
|\def\jobname{|\textit{dest}|}\input{\jobname}|
\end{center}
%
The redirection with prefix
|\childdocforwardprefix[|\textit{prefix}|]{|\textit{dest}|}|
is accomplished by:
%
\begin{center}
\begin{tabular}{l}
|{\edef\jobname{\scantokens\expandafter{\jobname\noexpand}}|\\
|\def\redirectjob |\textit{prefix}|#1~~~{\gdef\jobname{|\textit{dest}|#1}}|\\
|\expandafter\redirectjob\jobname~~~}\input{\jobname}|
\end{tabular}
\end{center}

In an alternative approach,
child documents can be compiled by a specific command line
without additional code or specific definitions:
%
\begin{center}
|... -jobname "|\textit{target}|" "|[\textit{flags}]%
|\includeonly{|\textit{dest}|}\input{|\textit{main}|}"|
\end{center}
%

%%%%%%%%%%%%%%%%%%%%%%%%%%%%%%%%%%%%%%%%%%%%%%%%%%%%%%%%%%%%%%%%%%%%%%%%%%%%%%%%
%%%%%%%%%%%%%%%%%%%%%%%%%%%%%%%%%%%%%%%%%%%%%%%%%%%%%%%%%%%%%%%%%%%%%%%%%%%%%%%%
\section{Information}

%%%%%%%%%%%%%%%%%%%%%%%%%%%%%%%%%%%%%%%%%%%%%%%%%%%%%%%%%%%%%%%%%%%%%%%%%%%%%%%%
\subsection{Copyright}

Copyright \copyright{} 2017--2018 Niklas Beisert

This work may be distributed and/or modified under the
conditions of the \LaTeX{} Project Public License, either version 1.3
of this license or (at your option) any later version.
The latest version of this license is in
  \url{http://www.latex-project.org/lppl.txt}
and version 1.3 or later is part of all distributions of \LaTeX{}
version 2005/12/01 or later.

This work has the LPPL maintenance status `maintained'.

The Current Maintainer of this work is Niklas Beisert.

This work consists of the files |README.txt|, |childdoc.ins| and |childdoc.dtx|
as well as the derived files |childdoc.def|, |cdocsamp.tex|
with |cdocsch1.tex|, |cdocsch2.tex|, |cdocspt3.tex|, |cdocspt4.tex|,
|cdocsdrf.tex|, |cdocsfn1.tex|, |cdocsfn2.tex|
as well as |childdoc.pdf|.

%%%%%%%%%%%%%%%%%%%%%%%%%%%%%%%%%%%%%%%%%%%%%%%%%%%%%%%%%%%%%%%%%%%%%%%%%%%%%%%%
\subsection{Files and Installation}

The package consists of the files:
%
\begin{center}
\begin{tabular}{ll}
    |README.txt|   & readme file \\
    |childdoc.ins| & installation file \\
    |childdoc.dtx| & source file \\
    |childdoc.def| & definition file \\
    |cdocsamp.tex| & sample main file \\
    |cdocsch1.tex| & sample include file \\
    |cdocsch2.tex| & sample include file \\
    |cdocspt3.tex| & sample part file \\
    |cdocspt4.tex| & sample part file \\
    |cdocsdrf.tex| & sample redirection file \\
    |cdocsfn1.tex| & sample redirection file \\
    |cdocsfn2.tex| & sample redirection file \\
    |childdoc.pdf| & manual
\end{tabular}
\end{center}
%
The distribution consists of the files
|README.txt|, |childdoc.ins| and |childdoc.dtx|.
%
\begin{itemize}
\item
Run (pdf)\LaTeX{} on |childdoc.dtx|
to compile the manual |childdoc.pdf| (this file).
\item
Run \LaTeX{} on |childdoc.ins| to create the definitions file |childdoc.def|
and the sample |cdocsamp.tex| with include files
|cdocsch1.tex|, |cdocsch2.tex|, |cdocspt3.tex|, |cdocspt4.tex|,
|cdocsdrf.tex|, |cdocsfn1.tex|, |cdocsfn2.tex|.
Then copy the file |childdoc.def| to an appropriate directory of your \LaTeX{}
distribution, e.g.\ \textit{texmf-root}|/tex/latex/childdoc|.
\end{itemize}

%%%%%%%%%%%%%%%%%%%%%%%%%%%%%%%%%%%%%%%%%%%%%%%%%%%%%%%%%%%%%%%%%%%%%%%%%%%%%%%%
\subsection{Related CTAN Packages}

There are several other packages which offer a similar functionality:
%
\begin{itemize}
\item
The packages
\href{http://ctan.org/pkg/docmute}{\textsf{docmute}},
\href{http://ctan.org/pkg/includex}{\textsf{includex}} and
\href{http://ctan.org/pkg/standalone}{\textsf{standalone}}
provide commands to include only the document body of
a child file thus allowing both files to be compiled individually.
\item
The packages \href{http://ctan.org/pkg/subdocs}{\textsf{subdocs}}
and \href{http://ctan.org/pkg/subfiles}{\textsf{subfiles}}
provide structures in which the main and child documents can be
encapsulated and allowing them to be compiled individually.
The inclusion mechanism is different from the conventional |\include|.
\item
The package \href{http://ctan.org/pkg/combine}{\textsf{combine}}
is an elaborate solution to combine several documents into one.
\end{itemize}
%
See also the CTAN topic \href{http://ctan.org/topic/subdocs}{\textsf{subdocs}}
for further related packages.
The present package differs from the above solutions in that
a document structure constructed with the conventional |\include| mechanism
just needs two extra commands at the top of every file
such that all constituent files can be compiled individually.

%%%%%%%%%%%%%%%%%%%%%%%%%%%%%%%%%%%%%%%%%%%%%%%%%%%%%%%%%%%%%%%%%%%%%%%%%%%%%%%%
%\subsection{Feature Suggestions}
%
%The following is a list of features which may be useful for future
%versions of this package:
%%
%\begin{itemize}
%\item
%\ldots
%\end{itemize}

%%%%%%%%%%%%%%%%%%%%%%%%%%%%%%%%%%%%%%%%%%%%%%%%%%%%%%%%%%%%%%%%%%%%%%%%%%%%%%%%
\subsection{Revision History}

%%%%%%%%%%%%%%%%%%%%%%%%%%%%%%%%%%%%%%%%
\paragraph{v2.0:} 2018/12/30

\begin{itemize}
\item
immediate forward processing
\item
added |\childdocby| mechanism
\item
manual restructured
\end{itemize}

%%%%%%%%%%%%%%%%%%%%%%%%%%%%%%%%%%%%%%%%
\paragraph{v1.6:} 2018/01/17

\begin{itemize}
\item
application for development of include files
\item
corrections to manual
\end{itemize}

%%%%%%%%%%%%%%%%%%%%%%%%%%%%%%%%%%%%%%%%
\paragraph{v1.5:} 2017/05/21

\begin{itemize}
\item
more complete structuring introduced
\item
|\childdocof| introduced
\item
|\childdoc| renamed to |\childdocmain|
\item
|\childredirect| renamed to |\childdocforward| and |\childdocforwardprefix|
and functionality expanded
\end{itemize}

%%%%%%%%%%%%%%%%%%%%%%%%%%%%%%%%%%%%%%%%
\paragraph{v1.0:} 2017/04/27

\begin{itemize}
\item
manual and install package
\item
first version published on CTAN
\end{itemize}

%%%%%%%%%%%%%%%%%%%%%%%%%%%%%%%%%%%%%%%%
\paragraph{v0.6:} 2017/04/26

\begin{itemize}
\item
redirection mechanism added
\end{itemize}

%%%%%%%%%%%%%%%%%%%%%%%%%%%%%%%%%%%%%%%%
\paragraph{v0.5:} 2017/04/26

\begin{itemize}
\item
functionality in definition file
\end{itemize}


%%%%%%%%%%%%%%%%%%%%%%%%%%%%%%%%%%%%%%%%%%%%%%%%%%%%%%%%%%%%%%%%%%%%%%%%%%%%%%%%
%%%%%%%%%%%%%%%%%%%%%%%%%%%%%%%%%%%%%%%%%%%%%%%%%%%%%%%%%%%%%%%%%%%%%%%%%%%%%%%%
%%%%%%%%%%%%%%%%%%%%%%%%%%%%%%%%%%%%%%%%%%%%%%%%%%%%%%%%%%%%%%%%%%%%%%%%%%%%%%%%
\appendix

\settowidth\MacroIndent{\rmfamily\scriptsize 000\ }

 \DocInput{childdoc.dtx}

\end{document}
%</driver>
% \fi
%
% %%%%%%%%%%%%%%%%%%%%%%%%%%%%%%%%%%%%%%%%%%%%%%%%%%%%%%%%%%%%%%%%%%%%%%%%%%%%%%
% %%%%%%%%%%%%%%%%%%%%%%%%%%%%%%%%%%%%%%%%%%%%%%%%%%%%%%%%%%%%%%%%%%%%%%%%%%%%%%
% \section{Sample}
%\iffalse
%<*samplemain>
%\fi
%
% The following presents a sample document
% with two chapters, two parts, a title page,
% a compile flag as well as three forwarding files to set the flag.
% It consists of eight |.tex| files:
% \begin{center}
% \begin{tabular}{ll}
% |cdocsamp.tex|&main file\\
% |cdocsch1.tex|&include file for chapter 1\\
% |cdocsch2.tex|&include file for chapter 2\\
% |cdocspt3.tex|&include file for part 3\\
% |cdocspt4.tex|&include file for part 4\\
% |cdocsdrf.tex|&forwarding file for main file in draft mode\\
% |cdocsfi1.tex|&forwarding file for final version of chapter 1\\
% |cdocsfi2.tex|&forwarding file for final version of chapter 2\\
% \end{tabular}
% \end{center}
% Each of the eight files can be compiled directly by the \LaTeX{} compiler.
%
% %%%%%%%%%%%%%%%%%%%%%%%%%%%%%%%%%%%%%%
% \paragraph{Main File.}
%
% The main file is called |cdocsamp.tex|.
%
% Load the \textsf{childdoc} definitions and
% declare the filename for the main document:
%    \begin{macrocode}
\input{childdoc.def}
\childdocmain{}
%    \end{macrocode}

% Optional override for |\version| flag:
%    \begin{macrocode}
%%\ifchilddoc\else\providecommand{\version}{draft}\fi
%    \end{macrocode}

% Define the default values for the |\version| flag
% (|final| for the main file and |draft| for childs):
%    \begin{macrocode}
\ifchilddoc
\providecommand{\version}{draft}
\else
\providecommand{\version}{final}
\fi
%    \end{macrocode}

% Load the standard document class:
%    \begin{macrocode}
\documentclass[12pt]{article}
%    \end{macrocode}

% Start the document body:
%    \begin{macrocode}
\begin{document}
%    \end{macrocode}

% Declare a title page.
% Print title, part of document being processed and version flag:
%    \begin{macrocode}
\addtocounter{page}{-1}
\begin{center}
{\LARGE\bfseries{}childdoc example\par}
\vspace{1cm}
\ifchilddoc
\ifchilddocmanual part\else chapter\fi:
`\childdocname' of `\childdocjob'\par
\else
main document: `\childdocjob'\par
\fi
version: \version\par
\end{center}
\newpage
%    \end{macrocode}

% Manually include selected file,
% otherwise process as usual:
%    \begin{macrocode}
\ifchilddocmanual
\section*{part `\childdocname'}
\input{\childdocname}
\else
%    \end{macrocode}

% Include the two chapters:
%    \begin{macrocode}
\include{cdocsch1}
\include{cdocsch2}
%    \end{macrocode}

% Include the two parts unless only chapters should be displayed:
%    \begin{macrocode}
\ifchilddoc\else
\section{part three}
\input{cdocspt3}
\section{part four}
\input{cdocspt4}
\fi
%    \end{macrocode}

% Process as usual until here:
%    \begin{macrocode}
\fi
%    \end{macrocode}

% End of document body:
%    \begin{macrocode}
\end{document}
%    \end{macrocode}
%\iffalse
%</samplemain>
%\fi
%
% %%%%%%%%%%%%%%%%%%%%%%%%%%%%%%%%%%%%%%
% \paragraph{Chapter Include Files.}
%
% The include files are called |cdocsch1.tex| and |cdocsch2.tex|.
%
%\iffalse
%<*samplechap1|samplechap2>
%\fi

% Optional override for |\version| flag:
%    \begin{macrocode}
%%\providecommand{\version}{final}
%    \end{macrocode}

% Include the main document:
%    \begin{macrocode}
\input{childdoc.def}
\childdocof{cdocsamp}
%    \end{macrocode}

%\iffalse
%</samplechap1|samplechap2>
%\fi
%
%\iffalse
%<*samplechap1>
%\fi
% Some text for chapter 1:
%    \begin{macrocode}
\section{one}
some text in chapter one
%    \end{macrocode}

%\iffalse
%</samplechap1>
%\fi
% Some text for chapter 2:
%\iffalse
%<*samplechap2>
%\fi
%    \begin{macrocode}
\section{two}
more text in chapter two
%    \end{macrocode}

%\iffalse
%</samplechap2>
%\fi
%
% %%%%%%%%%%%%%%%%%%%%%%%%%%%%%%%%%%%%%%
% \paragraph{Part Include Files.}
%
% The include files are called |cdocspt3.tex| and |cdocspt4.tex|.
%
%\iffalse
%<*samplepart3|samplepart4>
%\fi

% Optional override for |\version| flag:
%    \begin{macrocode}
%%\providecommand{\version}{final}
%    \end{macrocode}

% Include the main document:
%    \begin{macrocode}
\input{childdoc.def}
\childdocby{cdocsamp}
%    \end{macrocode}

%\iffalse
%</samplepart3|samplepart4>
%\fi
%
%\iffalse
%<*samplepart3>
%\fi
% Some text for part 3:
%    \begin{macrocode}
some text in part three
%    \end{macrocode}

%\iffalse
%</samplepart3>
%\fi
% Some text for part 4:
%\iffalse
%<*samplepart4>
%\fi
%    \begin{macrocode}
more text in part four
%    \end{macrocode}

%\iffalse
%</samplepart4>
%\fi
%
% %%%%%%%%%%%%%%%%%%%%%%%%%%%%%%%%%%%%%%
% \paragraph{Forwarding for a Complete Draft.}
%
% The following forwarding file |cdocsdrf.tex|
% compiles the main document in draft mode:
%\iffalse
%<*sampledraft>
%\fi
%    \begin{macrocode}
\def\version{draft}
\input{childdoc.def}
\childdocforward{cdocsamp}
%    \end{macrocode}

%\iffalse
%</sampledraft>
%\fi
%
% %%%%%%%%%%%%%%%%%%%%%%%%%%%%%%%%%%%%%%
% \paragraph{Forwarding for Final Version of the Chapters.}
%
% The following forwarding files |cdocsfn1.tex| and |cdocsfn2.tex|
% (with identical content)
% compile the final versions of the child documents
% |cdocsch1.tex| and |cdocsch2.tex|, respectively:
%\iffalse
%<*samplefinal>
%\fi
%    \begin{macrocode}
\def\version{final}
\input{childdoc.def}
\childdocforwardprefix[cdocsamp]{cdocsfn}{cdocsch}
%    \end{macrocode}

%\iffalse
%</samplefinal>
%\fi
%
% %%%%%%%%%%%%%%%%%%%%%%%%%%%%%%%%%%%%%%
% \paragraph{Command Line Processing.}
%
% The following three command lines generate the output files
% |cdocscld|, |cdocscl1| and |cdocscl2|
% which should be identical to
% |cdocsdrf|, |cdocsch1| and |cdocsfn2|, respectively:
% \begin{center}
% \begin{tabular}{l}
% |latex -jobname cdocscld \|\\
% |  "\def\version{draft}\input{childdoc.def}\childdocforward{cdocsamp}"|\\
% |latex -jobname cdocscl1 \|\\
% |  "\input{childdoc.def}\childdocforward[cdocsamp]{cdocsch1}"|\\
% |latex -jobname cdocscl2 \|\\
% |  "\def\version{final}\input{childdoc.def}\childdocforward{cdocsch2}"|
% \end{tabular}
% \end{center}
% Note that the trailing backslash on each first line
% merely continues the input to the second line
% (for convenient cut ant paste).
% Furthermore, the command |latex| can be replaced by any
% of its alternative versions such as |pdflatex|.
%
% %%%%%%%%%%%%%%%%%%%%%%%%%%%%%%%%%%%%%%%%%%%%%%%%%%%%%%%%%%%%%%%%%%%%%%%%%%%%%%
% %%%%%%%%%%%%%%%%%%%%%%%%%%%%%%%%%%%%%%%%%%%%%%%%%%%%%%%%%%%%%%%%%%%%%%%%%%%%%%
% \section{Implementation}
%\iffalse
%<*package>
%\fi
%
% This section describes the definitions file |childdoc.def|.

% The definitions cannot be loaded using |\usepackage| or |\RequirePackage|
% which has a mechanism to prevent loading a style file more than once.
% When loading the definitions by means of |\input|
% multiple instances have to be prevented manually:
%\iffalse
%This code needs to be before the `\ProvidesFile' directive
%which is defined at the beginning of this file.
%Therefore it is also placed there and commented out here.
%</package>
%<*discard>
%\fi
%    \begin{macrocode}
\ifdefined\childdocmain\endinput\fi
%    \end{macrocode}
%\iffalse
%</discard>
%<*package>
%\fi
%
% \macro{\ifchilddoc}
% \macro{\ifchilddocmanual}
% The conditional |\ifchilddoc| tells whether a
% child (true) or main (false) document is being compiled.
% The conditional |\ifchilddocmanual| tells whether
% the |\includeonly| mechanism is used (false) or
% the selection of child files must be performed manually (true).
% The definitions initialise to false:
%    \begin{macrocode}
\newif\ifchilddoc
\newif\ifchilddocmanual
%    \end{macrocode}

% \macro{\childdocname}
% \macro{\childdocjob}
% The macro |\childdocname| stores the name of the main document
% to be compiled. The macro |\childdocjob| stores the name of
% the document on which the \LaTeX{} compiler was originally invoked.
% The content of |\jobname| cannot be compared
% to filenames specified in the source due to different catcodes.
% The following code rescans |\jobname|, stores the result
% in |\childdocname| and saves a copy in |\childdocjob|:
%    \begin{macrocode}
\edef\childdocname{\scantokens\expandafter{\jobname\noexpand}}
\let\childdocjob\childdocname
%    \end{macrocode}

% \macro{\childdocdisable}
% The macro |\childdocdisable| prevents the main file
% from being processed more than once.
% At this stage, the main document command |\childdocmain|
% is assumed to be called once again where it should do nothing.
% Any subsequent call to it should prevent
% a secondary processing of the main document
% It overwrites the forwarding commands
% |\childdocof| and |\childdocforward|
% with empty macros to prevent further inclusions of the main document:
%    \begin{macrocode}
\newcommand{\childdocdisable}
{
  \renewcommand{\childdocmain}[1]{\renewcommand{\childdocmain}[1]{\endinput}}
  \renewcommand{\childdocof}[1]{}
  \renewcommand{\childdocby}[2][]{}
  \renewcommand{\childdocforward}[2][]{}
  \renewcommand{\childdocdisable}{}
}
%    \end{macrocode}

% \macro{\childdocmain}
% The macro |\childdocmain| is to be called at the top of the main file
% with nothing or the main filename (without extension) as argument.
% First, it breaks loops.
% If the argument is not empty and does not match |\childdocname|
% (which is set by the first inclusion of |childdoc.def|),
% |\ifchilddoc| is set to true, |\includeonly| is applied to the child file
% and |\jobname| is set to the main file
% (for proper handling of |.aux| files):
%    \begin{macrocode}
\newcommand{\childdocmain}[1]
{
  \childdocdisable\childdocmain{}
  \if?#1?\else
    \begingroup
      \def\childdoctmp{#1}
      \ifx\childdoctmp\childdocname
        \def\childdoctmp{}
      \else
        \def\childdoctmp
        {
          \childdoctrue
          \includeonly{\childdocname}
          \def\childdocjob{#1}
          \def\jobname{#1}
        }
      \fi
      \expandafter
    \endgroup
    \childdoctmp
  \fi
}
%    \end{macrocode}

% \macro{\childdocof}
% The command |\childdocof| redirects
% compilation to the main file |#1|.
%    \begin{macrocode}
\newcommand{\childdocof}[1]
{
  \childdocdisable
  \childdoctrue
  \includeonly{\childdocname}
  \def\jobname{#1}
  \def\childdocjob{#1}
  \input{#1}
}
%    \end{macrocode}

% \macro{\childdocby}
% The command |\childdocby| ....
%    \begin{macrocode}
\newcommand{\childdocby}[2][]
{
  \childdocdisable
  \childdoctrue
  \childdocmanualtrue
  \if?#1?\else
    \def\jobname{#2}
  \fi
  \def\childdocjob{#2}
  \input{#2}
  \endinput
}
%    \end{macrocode}

% \macro{\childdocforward}
% The command |\childdocforward| redirects
% compilation to the main file or
% (if the optional argument is given) a child file.
% Parameters are set as if the main file
% or a child file starting with |\childdocof| was compiled.
% Then compilation is handed over to the main file:
%    \begin{macrocode}
\newcommand{\childdocforward}[2][]
{
  \begingroup
    \if?#1?
      \def\childdoctmp
      {
        \def\childdocname{#2}
        \def\childdocjob{#2}
        \def\jobname{#2}
        \input{#2}
        \endinput
      }
    \else
      \def\childdoctmp
      {
        \childdocdisable
        \def\childdocname{#2}
        \childdoctrue
        \includeonly{#2}
        \def\childdocjob{#1}
        \def\jobname{#1}
        \input{#1}
        \endinput
      }
    \fi
    \expandafter
  \endgroup
  \childdoctmp
}
%    \end{macrocode}

% \macro{\childdocforwardprefix}
% The command |\childdocforwardprefix| redirects
% compilation to the main or a child file by means of a pattern.
% The prefix |#1| in the current filename is replaced by |#2|
% and the suffix of the current filename is kept
% (it is assumed that the filename does not contain the substring `|~~~|'
% which is used as a delimiter).
% Compilation is handed over to the new file by |\childdocforward|:
%    \begin{macrocode}
\newcommand{\childdocforwardprefix}[3][]
{
  \begingroup
    \def\childdocextract #2##1~~~{\def\childdoctmp{\childdocforward[#1]{#3##1}}}
    \expandafter\childdocextract\childdocname~~~
    \expandafter
  \endgroup
  \childdoctmp
}
%    \end{macrocode}

% \macro{\childdoc}
% The deprecated macro |\childdoc| is a legacy version of |\childdocmain|:
%    \begin{macrocode}
\newcommand{\childdoc}{\childdocmain}
%    \end{macrocode}

% \macro{\childdocredirect}
% The deprecated macro |\childdocredirect| is a legacy version
% of |\childdocforward| and |\childdocforwardprefix|:
%    \begin{macrocode}
\newcommand{\childdocredirect}[2][]
{
  \begingroup
    \if?#1?
      \def\childdoctmp{\childdocforward{#2}}
    \else
      \def\childdoctmp{\childdocforwardprefix{#1}{#2}}
    \fi
    \expandafter
  \endgroup
  \childdoctmp
}
%    \end{macrocode}

%\iffalse
%</package>
%\fi
%
\endinput
|
and perform the replacements as outlined below.
Instead of |\childdocmain{|\textit{main}|}| add the following code
to the top of the main file:
%
\begin{center}
\begin{tabular}{l}
|\||ifdefined\childdocname\endinput\||fi\newif\ifchilddoc|\\
|\edef\childdocname{\scantokens\expandafter{\jobname\noexpand}}|\\
|\def\childdocmain{|\textit{main}|}\||ifx\childdocmain\childdocname\||else|\\
|\childdoctrue\includeonly{\childdocname}\let\jobname\childdocmain\||fi|\\
\end{tabular}
\end{center}
%
Instead of |\childdocof{|\textit{main}|}| just include the main file
at the top of each child file:
%
\begin{center}
|\input{|\textit{main}|}|
\end{center}
%
A simple redirection |\childdocforward{|\textit{dest}|}| is achieved by:
%
\begin{center}
|\def\jobname{|\textit{dest}|}\input{\jobname}|
\end{center}
%
The redirection with prefix
|\childdocforwardprefix[|\textit{prefix}|]{|\textit{dest}|}|
is accomplished by:
%
\begin{center}
\begin{tabular}{l}
|{\edef\jobname{\scantokens\expandafter{\jobname\noexpand}}|\\
|\def\redirectjob |\textit{prefix}|#1~~~{\gdef\jobname{|\textit{dest}|#1}}|\\
|\expandafter\redirectjob\jobname~~~}\input{\jobname}|
\end{tabular}
\end{center}

In an alternative approach,
child documents can be compiled by a specific command line
without additional code or specific definitions:
%
\begin{center}
|... -jobname "|\textit{target}|" "|[\textit{flags}]%
|\includeonly{|\textit{dest}|}\input{|\textit{main}|}"|
\end{center}
%

%%%%%%%%%%%%%%%%%%%%%%%%%%%%%%%%%%%%%%%%%%%%%%%%%%%%%%%%%%%%%%%%%%%%%%%%%%%%%%%%
%%%%%%%%%%%%%%%%%%%%%%%%%%%%%%%%%%%%%%%%%%%%%%%%%%%%%%%%%%%%%%%%%%%%%%%%%%%%%%%%
\section{Information}

%%%%%%%%%%%%%%%%%%%%%%%%%%%%%%%%%%%%%%%%%%%%%%%%%%%%%%%%%%%%%%%%%%%%%%%%%%%%%%%%
\subsection{Copyright}

Copyright \copyright{} 2017--2018 Niklas Beisert

This work may be distributed and/or modified under the
conditions of the \LaTeX{} Project Public License, either version 1.3
of this license or (at your option) any later version.
The latest version of this license is in
  \url{http://www.latex-project.org/lppl.txt}
and version 1.3 or later is part of all distributions of \LaTeX{}
version 2005/12/01 or later.

This work has the LPPL maintenance status `maintained'.

The Current Maintainer of this work is Niklas Beisert.

This work consists of the files |README.txt|, |childdoc.ins| and |childdoc.dtx|
as well as the derived files |childdoc.def|, |cdocsamp.tex|
with |cdocsch1.tex|, |cdocsch2.tex|, |cdocspt3.tex|, |cdocspt4.tex|,
|cdocsdrf.tex|, |cdocsfn1.tex|, |cdocsfn2.tex|
as well as |childdoc.pdf|.

%%%%%%%%%%%%%%%%%%%%%%%%%%%%%%%%%%%%%%%%%%%%%%%%%%%%%%%%%%%%%%%%%%%%%%%%%%%%%%%%
\subsection{Files and Installation}

The package consists of the files:
%
\begin{center}
\begin{tabular}{ll}
    |README.txt|   & readme file \\
    |childdoc.ins| & installation file \\
    |childdoc.dtx| & source file \\
    |childdoc.def| & definition file \\
    |cdocsamp.tex| & sample main file \\
    |cdocsch1.tex| & sample include file \\
    |cdocsch2.tex| & sample include file \\
    |cdocspt3.tex| & sample part file \\
    |cdocspt4.tex| & sample part file \\
    |cdocsdrf.tex| & sample redirection file \\
    |cdocsfn1.tex| & sample redirection file \\
    |cdocsfn2.tex| & sample redirection file \\
    |childdoc.pdf| & manual
\end{tabular}
\end{center}
%
The distribution consists of the files
|README.txt|, |childdoc.ins| and |childdoc.dtx|.
%
\begin{itemize}
\item
Run (pdf)\LaTeX{} on |childdoc.dtx|
to compile the manual |childdoc.pdf| (this file).
\item
Run \LaTeX{} on |childdoc.ins| to create the definitions file |childdoc.def|
and the sample |cdocsamp.tex| with include files
|cdocsch1.tex|, |cdocsch2.tex|, |cdocspt3.tex|, |cdocspt4.tex|,
|cdocsdrf.tex|, |cdocsfn1.tex|, |cdocsfn2.tex|.
Then copy the file |childdoc.def| to an appropriate directory of your \LaTeX{}
distribution, e.g.\ \textit{texmf-root}|/tex/latex/childdoc|.
\end{itemize}

%%%%%%%%%%%%%%%%%%%%%%%%%%%%%%%%%%%%%%%%%%%%%%%%%%%%%%%%%%%%%%%%%%%%%%%%%%%%%%%%
\subsection{Related CTAN Packages}

There are several other packages which offer a similar functionality:
%
\begin{itemize}
\item
The packages
\href{http://ctan.org/pkg/docmute}{\textsf{docmute}},
\href{http://ctan.org/pkg/includex}{\textsf{includex}} and
\href{http://ctan.org/pkg/standalone}{\textsf{standalone}}
provide commands to include only the document body of
a child file thus allowing both files to be compiled individually.
\item
The packages \href{http://ctan.org/pkg/subdocs}{\textsf{subdocs}}
and \href{http://ctan.org/pkg/subfiles}{\textsf{subfiles}}
provide structures in which the main and child documents can be
encapsulated and allowing them to be compiled individually.
The inclusion mechanism is different from the conventional |\include|.
\item
The package \href{http://ctan.org/pkg/combine}{\textsf{combine}}
is an elaborate solution to combine several documents into one.
\end{itemize}
%
See also the CTAN topic \href{http://ctan.org/topic/subdocs}{\textsf{subdocs}}
for further related packages.
The present package differs from the above solutions in that
a document structure constructed with the conventional |\include| mechanism
just needs two extra commands at the top of every file
such that all constituent files can be compiled individually.

%%%%%%%%%%%%%%%%%%%%%%%%%%%%%%%%%%%%%%%%%%%%%%%%%%%%%%%%%%%%%%%%%%%%%%%%%%%%%%%%
%\subsection{Feature Suggestions}
%
%The following is a list of features which may be useful for future
%versions of this package:
%%
%\begin{itemize}
%\item
%\ldots
%\end{itemize}

%%%%%%%%%%%%%%%%%%%%%%%%%%%%%%%%%%%%%%%%%%%%%%%%%%%%%%%%%%%%%%%%%%%%%%%%%%%%%%%%
\subsection{Revision History}

%%%%%%%%%%%%%%%%%%%%%%%%%%%%%%%%%%%%%%%%
\paragraph{v2.0:} 2018/12/30

\begin{itemize}
\item
immediate forward processing
\item
added |\childdocby| mechanism
\item
manual restructured
\end{itemize}

%%%%%%%%%%%%%%%%%%%%%%%%%%%%%%%%%%%%%%%%
\paragraph{v1.6:} 2018/01/17

\begin{itemize}
\item
application for development of include files
\item
corrections to manual
\end{itemize}

%%%%%%%%%%%%%%%%%%%%%%%%%%%%%%%%%%%%%%%%
\paragraph{v1.5:} 2017/05/21

\begin{itemize}
\item
more complete structuring introduced
\item
|\childdocof| introduced
\item
|\childdoc| renamed to |\childdocmain|
\item
|\childredirect| renamed to |\childdocforward| and |\childdocforwardprefix|
and functionality expanded
\end{itemize}

%%%%%%%%%%%%%%%%%%%%%%%%%%%%%%%%%%%%%%%%
\paragraph{v1.0:} 2017/04/27

\begin{itemize}
\item
manual and install package
\item
first version published on CTAN
\end{itemize}

%%%%%%%%%%%%%%%%%%%%%%%%%%%%%%%%%%%%%%%%
\paragraph{v0.6:} 2017/04/26

\begin{itemize}
\item
redirection mechanism added
\end{itemize}

%%%%%%%%%%%%%%%%%%%%%%%%%%%%%%%%%%%%%%%%
\paragraph{v0.5:} 2017/04/26

\begin{itemize}
\item
functionality in definition file
\end{itemize}


%%%%%%%%%%%%%%%%%%%%%%%%%%%%%%%%%%%%%%%%%%%%%%%%%%%%%%%%%%%%%%%%%%%%%%%%%%%%%%%%
%%%%%%%%%%%%%%%%%%%%%%%%%%%%%%%%%%%%%%%%%%%%%%%%%%%%%%%%%%%%%%%%%%%%%%%%%%%%%%%%
%%%%%%%%%%%%%%%%%%%%%%%%%%%%%%%%%%%%%%%%%%%%%%%%%%%%%%%%%%%%%%%%%%%%%%%%%%%%%%%%
\appendix

\settowidth\MacroIndent{\rmfamily\scriptsize 000\ }

 \DocInput{childdoc.dtx}

\end{document}
%</driver>
% \fi
%
% %%%%%%%%%%%%%%%%%%%%%%%%%%%%%%%%%%%%%%%%%%%%%%%%%%%%%%%%%%%%%%%%%%%%%%%%%%%%%%
% %%%%%%%%%%%%%%%%%%%%%%%%%%%%%%%%%%%%%%%%%%%%%%%%%%%%%%%%%%%%%%%%%%%%%%%%%%%%%%
% \section{Sample}
%\iffalse
%<*samplemain>
%\fi
%
% The following presents a sample document
% with two chapters, two parts, a title page,
% a compile flag as well as three forwarding files to set the flag.
% It consists of eight |.tex| files:
% \begin{center}
% \begin{tabular}{ll}
% |cdocsamp.tex|&main file\\
% |cdocsch1.tex|&include file for chapter 1\\
% |cdocsch2.tex|&include file for chapter 2\\
% |cdocspt3.tex|&include file for part 3\\
% |cdocspt4.tex|&include file for part 4\\
% |cdocsdrf.tex|&forwarding file for main file in draft mode\\
% |cdocsfi1.tex|&forwarding file for final version of chapter 1\\
% |cdocsfi2.tex|&forwarding file for final version of chapter 2\\
% \end{tabular}
% \end{center}
% Each of the eight files can be compiled directly by the \LaTeX{} compiler.
%
% %%%%%%%%%%%%%%%%%%%%%%%%%%%%%%%%%%%%%%
% \paragraph{Main File.}
%
% The main file is called |cdocsamp.tex|.
%
% Load the \textsf{childdoc} definitions and
% declare the filename for the main document:
%    \begin{macrocode}
% \iffalse
%
% childdoc.dtx Copyright (C) 2017-2018 Niklas Beisert
%
% This work may be distributed and/or modified under the
% conditions of the LaTeX Project Public License, either version 1.3
% of this license or (at your option) any later version.
% The latest version of this license is in
%   http://www.latex-project.org/lppl.txt
% and version 1.3 or later is part of all distributions of LaTeX
% version 2005/12/01 or later.
%
% This work has the LPPL maintenance status `maintained'.
%
% The Current Maintainer of this work is Niklas Beisert.
%
% This work consists of the files childdoc.dtx and childdoc.ins
% and the derived files childdoc.def and cdocsamp.tex with
% cdocsch1.tex, cdocsch2.tex, cdocsdrf.tex, cdocsfn1.tex, cdocsfn2.tex.
%
%<package>\ifdefined\childdocmain\endinput\fi
%<package>\ProvidesFile{childdoc.def}[2018/12/30 v2.0 child document driver]
%<samplemain>\ProvidesFile{cdocsamp.tex}[2018/12/30 v2.0 sample for childdoc]
%<*driver>
%\ProvidesFile{childdoc.drv}[2018/12/30 v2.0 childdoc reference manual file]
\PassOptionsToClass{10pt,a4paper}{article}
\documentclass{ltxdoc}

\usepackage[margin=35mm]{geometry}
\usepackage{hyperref}
\usepackage{hyperxmp}
\usepackage[usenames]{color}

\hypersetup{colorlinks=true}
\hypersetup{pdfstartview=FitH}
\hypersetup{pdfpagemode=UseNone}
\hypersetup{pdfsource={}}
\hypersetup{pdflang={en-UK}}
\hypersetup{pdfcopyright={Copyright 2017-2018 Niklas Beisert.
  This work may be distributed and/or modified under the
  conditions of the LaTeX Project Public License, either version 1.3
  of this license or (at your option) any later version.}}
\hypersetup{pdflicenseurl={http://www.latex-project.org/lppl.txt}}
\hypersetup{pdfcontactaddress={ETH Zurich, ITP, HIT K,
  Wolfgang-Pauli-Strasse 27}}
\hypersetup{pdfcontactpostcode={8093}}
\hypersetup{pdfcontactcity={Zurich}}
\hypersetup{pdfcontactcountry={Switzerland}}
\hypersetup{pdfcontactemail={nbeisert@itp.phys.ethz.ch}}
\hypersetup{pdfcontacturl={http://people.phys.ethz.ch/\xmptilde nbeisert/}}

\newcommand{\secref}[1]{\hyperref[#1]{section \ref*{#1}}}

\parskip1ex
\parindent0pt
\let\olditemize\itemize
\def\itemize{\olditemize\parskip0pt}

\begin{document}

\title{The \textsf{childdoc} Package}
\hypersetup{pdftitle={The childdoc Package}}
\author{Niklas Beisert\\[2ex]
  Institut f\"ur Theoretische Physik\\
  Eidgen\"ossische Technische Hochschule Z\"urich\\
  Wolfgang-Pauli-Strasse 27, 8093 Z\"urich, Switzerland\\[1ex]
  \href{mailto:nbeisert@itp.phys.ethz.ch}
  {\texttt{nbeisert@itp.phys.ethz.ch}}}
\hypersetup{pdfauthor={Niklas Beisert}}
\hypersetup{pdfsubject={Manual for the LaTeX2e Package childdoc}}
\date{30 December 2018, \textsf{v2.0}}
\maketitle

\begin{abstract}\noindent
\textsf{childdoc} is a \LaTeXe{} package
that enables the direct compilation
of document sections included by |\include|
to individual files.
\end{abstract}

\begingroup
\parskip0ex
\tableofcontents
\endgroup

%%%%%%%%%%%%%%%%%%%%%%%%%%%%%%%%%%%%%%%%%%%%%%%%%%%%%%%%%%%%%%%%%%%%%%%%%%%%%%%%
%%%%%%%%%%%%%%%%%%%%%%%%%%%%%%%%%%%%%%%%%%%%%%%%%%%%%%%%%%%%%%%%%%%%%%%%%%%%%%%%
\section{Introduction}

\LaTeX{} provides a mechanism to structure a large document (such as a book)
into a main file and several child files (containing the chapters)
using the |\include| command.
This mechanism is beneficial for documents
which span hundreds of pages in order to
make the source file(s) more manageable.
Moreover, compilation can be restricted to
selected child files by means of the |\includeonly| command.
The latter feature can be used to reduce the compilation time while editing
(this was significantly more useful in the earlier days of \LaTeX{})
or to generate a smaller document which is easier to navigate.
Another application of |\includeonly| is to generate
documents consisting of selected parts of the complete document.

However, there are a few drawbacks of the plain |\include| mechanism:
\begin{itemize}
\item
The child files cannot be compiled on their own,
they can only be compiled via the main file.
A naive editing environment
(such as a text editor with an option
to have the current file processed by \LaTeX)
may require one to switch to the main file before compiling;
attempting to compile the child file produces errors.
\item
The main file must be modified (each time)
to adjust the |\includeonly| command
to the present needs. This easily leaves the main file in a messy state.
\item
The generated document will always carry the filename
of the main document. This is inconvenient if
several child files are to be compiled and
to be kept for distribution.
\end{itemize}

The present package provides a simple interface
to make child files individually compilable by \LaTeX{}.
Compiling a child file then has the same effect as compiling
the main file with an |\includeonly| command
to select the appropriate child.
Moreover the generated document will carry the name of the child
rather than the main file.
This resolves all three above issues.

This feature is meant to make the editing of books,
thesis documents and lecture notes somewhat more convenient.
However, the package can also be used efficiently for
composing a series of documents (such as exercise sheets)
which are typically distributed individually.
It then assists the author in generating the individual documents
(potentially in different versions)
as well as a document containing the collected series.
Another application is in developing style files
or other kinds of included material
where compilation of the style file could redirect
to a sample or test file.

%%%%%%%%%%%%%%%%%%%%%%%%%%%%%%%%%%%%%%%%%%%%%%%%%%%%%%%%%%%%%%%%%%%%%%%%%%%%%%%%
%%%%%%%%%%%%%%%%%%%%%%%%%%%%%%%%%%%%%%%%%%%%%%%%%%%%%%%%%%%%%%%%%%%%%%%%%%%%%%%%
\section{Usage}

First of all, the package \textsf{childdoc} is \emph{not} a standard
\LaTeXe{} |.sty| style file! Therefore it needs to be invoked in
a non-standard way.

%%%%%%%%%%%%%%%%%%%%%%%%%%%%%%%%%%%%%%%%%%%%%%%%%%%%%%%%%%%%%%%%%%%%%%%%%%%%%%%%
\subsection{Included Files}
\label{sec:include}

%%%%%%%%%%%%%%%%%%%%%%%%%%%%%%%%%%%%%%%%
\DescribeMacro{\childdocmain}
To use the package, add the commands
\begin{center}
\begin{tabular}{l}
|\input{childdoc.def}|\\
|\childdocmain{}|\\
\end{tabular}
\end{center}
at the very top of the main \LaTeX{} file,
in particular \emph{before} the |\documentclass| statement!
The argument of |\childdocmain| should be left empty
(but it must be present).

%%%%%%%%%%%%%%%%%%%%%%%%%%%%%%%%%%%%%%%%
\DescribeMacro{\childdocof}
Furthermore, add the commands
\begin{center}
\begin{tabular}{l}
|\input{childdoc.def}|\\
|\childdocof{|\textit{main}|}|\\
\end{tabular}
\end{center}
at the top of every child file \textit{child}
which is included by |\include{|\textit{child}|}|
from within the main file
(or at least for those files to be compiled individually).
The argument \textit{main} must be the filename of the main file.

There are a couple of
considerations in setting up the main and child documents:

%%%%%%%%%%%%%%%%%%%%%%%%%%%%%%%%%%%%%%%%
\paragraph{Restrictions.}

Please note the following restrictions:
\begin{itemize}
\item
|\childdocmain| must be called with one argument \textit{main}
to ensure compatibility with earlier version of the package.
It must either be empty (|\childdocmain{}|)
or precisely match the filename of the main file in which it is specified.
See \secref{sec:detection} for further information.
\item
The filename \textit{main} must be specified without the |.tex| extension.
\item
The filename \textit{main} is case sensitive
(even in case-insensitive file systems)
due to internal string comparison.
\item
The argument \textit{main} should be fully expanded, it cannot be a macro.
\item
Subdirectories and special characters should be avoided in filenames.
\item
The command |\childdocmain{|\textit{main}|}| must be followed by a whitespace.
It should not be followed immediately by another command
or by a comment mark `|%|'.
This is because the \TeX{} parser reads the token immediately following
the argument of |\childdocmain| and puts it
at the beginning of every child section;
however, a white\-space is ignored.
\end{itemize}

%%%%%%%%%%%%%%%%%%%%%%%%%%%%%%%%%%%%%%%%
\paragraph{Content of Main File.}

It is advisable to place all content in the child files included by |\include|.
Any output contained in the main file will appear in all child documents
unless suppressed manually;
it cannot be suppressed automatically by the |\includeonly| directive
and thus should normally be avoided.
A method to include some content in the main file
by means of conditional processing is described in \secref{sec:conditional}.

%%%%%%%%%%%%%%%%%%%%%%%%%%%%%%%%%%%%%%%%
\paragraph{Page Numbering.}

When only a part of the document is compiled,
the appropriate numbering of pages
(as well as other status parameters)
is determined from the |.aux| files.
The latter contain information from previous passes.
However this information needs to propagate through
all intermediate child documents.
Therefore the page numbering in child documents may well
be inconsistent until the complete document is compiled at least once.

A useful (if unconventional) way to always ensure a consistent
page numbering is to restart the numbering in each child document
and denote the pages by `\textit{child}|.|\textit{page}'
where \textit{child} represents the chapter/section number of the child file.
This can be achieved by the command
|\numberwithin{page}{|\textit{child}|}|
of the \textsf{amsmath} package
where \textit{child} can be |chapter| or |section|
depending on the chosen structuring.
Alternatively, one can modify the macro |\thepage| appropriately
and reset the counter |page| at the start of each child file.

%%%%%%%%%%%%%%%%%%%%%%%%%%%%%%%%%%%%%%%%%%%%%%%%%%%%%%%%%%%%%%%%%%%%%%%%%%%%%%%%
\subsection{Conditional Processing}
\label{sec:conditional}

The package provides a mechanism to compile different versions
of a document. To customise the versions further some conditional processing
can come in handy to distinguish which version is being compiled.
The package provides two macros to describe the compilation context:

%%%%%%%%%%%%%%%%%%%%%%%%%%%%%%%%%%%%%%%%
\DescribeMacro{\ifchilddoc}
The conditional |\ifchilddoc| distinguishes between the compilation of
child documents and the main document:
%
\begin{center}
|\ifchilddoc |\textit{child-code}| |[|\||else |\textit{main-code}]| \||fi|
\end{center}

%%%%%%%%%%%%%%%%%%%%%%%%%%%%%%%%%%%%%%%%
\DescribeMacro{\childdocname}
\DescribeMacro{\childdocjob}
The macro |\childdocname| contains the filename (without extension)
of the main or child file being processed.
Note that |\childdocjob| will always contain the name of the main file.

%%%%%%%%%%%%%%%%%%%%%%%%%%%%%%%%%%%%%%%%
\paragraph{Title Page.}

Conditional processing can be used to include a title or banner page
in the main document when proper precautions are taken.
Importantly, the code in the main file should ensure that the page counter
(as well as other status parameters which are stored in the |.aux| files)
takes the same value after the conditional processing.
Otherwise the page numbers may take divergent values
depending on which part is compiled.

For example, a title page could be declared by:
%
\begin{center}
\begin{tabular}{l}
|\ifchilddoc\||else|\\
|\addtocounter{page}{-1}|\\
\textit{code for title page}\\
|\newpage|\\
|\||fi|
\end{tabular}
\end{center}
%
A banner page for the child documents can be generated by:
%
\begin{center}
\begin{tabular}{l}
|\ifchilddoc|\\
|\addtocounter{page}{-1}|\\
\textit{code for banner page}\\
|\newpage|\\
|\||fi|
\end{tabular}
\end{center}
%
Here one could write a message such as:
\begin{center}
|This is the part \childdocname{} of \childdocjob{}.|
\end{center}

%%%%%%%%%%%%%%%%%%%%%%%%%%%%%%%%%%%%%%%%%%%%%%%%%%%%%%%%%%%%%%%%%%%%%%%%%%%%%%%%
\subsection{Flags}
\label{sec:flags}

The package makes it easy to generate different versions
of the main or child documents.
To this end compilation flags can be defined
and assigned different default values.
They will be particularly useful in conjunction
with the forwarding mechanism described in \secref{sec:forward}.

For example, it may be useful to have a flag |\version|
which can be set to |draft| or |final|.
The document source will contain some conditional code
depending on the value of |\version|.
Suppose further, the flag should default to |final| for the main file
and to |draft| for child files
which is a natural assignment for editing the document.
This is achieved by placing the following code
in the preamble of the main document
(below the |\childdocmain| directive):
%
\begin{center}
\begin{tabular}{l}
|\ifchilddoc|\\
|\providecommand{\version}{draft}|\\
|\||else|\\
|\providecommand{\version}{final}|\\
|\||fi|
\end{tabular}
\end{center}
%
The definition by |\providecommand| makes sure
that previous definitions are not overwritten.
Further statements |\providecommand{\version}{...}|
can thus be added before the above code to override it.

For the main file, one might add a line
(between |\childdocmain| and the above block)
%
\begin{center}
|%\ifchilddoc\||else\providecommand{\version}{draft}\||fi|
\end{center}
%
which can be uncommented to produce a draft version.
Likewise one can add a line to the very top of a child file
(above the |\childdocof{|\textit{main}|}| directive)
%
\begin{center}
|%\providecommand{\version}{final}|
\end{center}
%
which can be uncommented to produce the final version of this child document.

%%%%%%%%%%%%%%%%%%%%%%%%%%%%%%%%%%%%%%%%%%%%%%%%%%%%%%%%%%%%%%%%%%%%%%%%%%%%%%%%
\subsection{Forwarding}
\label{sec:forward}

Different versions of the main or child documents
using compilation flags as described in \secref{sec:flags}
can be (permanently) stored in different files
for convenient compilation, viewing and distribution.
To this end, the package defines a command
to pass on compilation to a different file:

%%%%%%%%%%%%%%%%%%%%%%%%%%%%%%%%%%%%%%%%
\DescribeMacro{\childdocforward}
The command |\childdocforward| redirects processing to
another source file:
%
\begin{center}
\begin{tabular}{l}
|\input{childdoc.def}|\\
|\childdocforward[|\textit{main}|]{|\textit{dest}|}|\\
\end{tabular}
\end{center}
%
The argument \textit{dest} is the destination file
(without extension).
It should be the main file or one of the child files.
Note that further \textsf{childdoc} directives
such as |\childdocof| and |\childdocforward|
in the indicated file will be processed in this form.
The optional argument \textit{main}
passes on directly to the main file \textit{main}
while pretending to compile the child \textit{dest}.
This form behaves as if \textit{dest}
issues |\childdocof{|\textit{main}|}| right away,
and no further \textsf{childdoc} directives will be processed.

%%%%%%%%%%%%%%%%%%%%%%%%%%%%%%%%%%%%%%%%
\DescribeMacro{\...prefix}
In the alternative form |\childdocforwardprefix|,
%
\begin{center}
\begin{tabular}{l}
|\input{childdoc.def}|\\
|\childdocforwardprefix[|\textit{main}|]{|\textit{prefix}|}{|\textit{dest}|}|
\end{tabular}
\end{center}
%
the destination file is determined by a pattern
depending on the current file:
To make this work, the current file must be called
`{\textit{prefix}\hspace{0.2em}\textit{suffix}}'
with \textit{prefix} matching precisely the argument.
Processing is then passed on to the file
`{\textit{dest}\hspace{0.2em}\textit{suffix}}'.
Surely, the same effect is achieved by
directly specifying the
argument `{\textit{dest}\hspace{0.2em}\textit{suffix}}'
in the first form.
However, that requires to set up a different file
for each child. With the alternative form of the command
all these files can have exactly the same content
which simplifies setting them up and maintaining them.

For example, the following file |draft.tex|
with a compilation flag |\version| as described in \secref{sec:flags}
compiles the main document as a draft:
%
\begin{center}
\begin{tabular}{l}
|\def\version{draft}|\\
|\input{childdoc.def}|\\
|\childdocforward{|\textit{main}|}|
\end{tabular}
\end{center}
%
Likewise, the following files |final|\textit{nn}|.tex|
compile the final version of the child document
|child|\textit{nn}|.tex|:
%
\begin{center}
\begin{tabular}{l}
|\def\version{final}|\\
|\input{childdoc.def}|\\
|\childdocforwardprefix{final}{child}|
\end{tabular}
\end{center}
%

Note that when several versions of a main file and/or of each child file
are to be generated, it may be convenient to set up a |Makefile| or
shell script to automatise the process.

%%%%%%%%%%%%%%%%%%%%%%%%%%%%%%%%%%%%%%%%%%%%%%%%%%%%%%%%%%%%%%%%%%%%%%%%%%%%%%%%
\subsection{Command Line Processing}
\label{sec:commandline}

The effect of redirection files can also be achieved by invoking
the \LaTeX{} compiler with a more elaborate command line.
Most conveniently this should be done as part
of a shell script or a |Makefile|.

When using \textsf{childdoc} in the main file, the following
command lines effectively perform a redirection
(note that depending on the shell being used,
backslashes may have to be doubled: `|\|' $\to$ `|\\|'):
%
\begin{center}
|... -jobname "|\textit{target}|" |\\|"|[\textit{flags}]%
|\input{childdoc.def}\childdocforward[|\textit{main}|]{|\textit{dest}|}"|
\end{center}
%
Here \textit{target} is the name of the output file,
\textit{main} is the name of the main file
and \textit{dest} is the name of the main or child file to be processed
(all filenames without extensions).
The optional argument \textit{main} can be omitted
if \textit{main} matches \textit{dest}.
Optionally, compilation \textit{flags} can be defined via |\def| commands.
This command line makes the \TeX{} engine believe
it is compiling the file \textit{target}
whose content is specified as the latter parameter.
The provided code then forwards the processing to
\textit{main} or \textit{dest} as described in \secref{sec:forward}.

%%%%%%%%%%%%%%%%%%%%%%%%%%%%%%%%%%%%%%%%%%%%%%%%%%%%%%%%%%%%%%%%%%%%%%%%%%%%%%%%
\subsection{Include by Input}
\label{sec:input}

Including child documents by |\include| has some restrictions by design.
Most notably, the content of a child document always occupies
its own set of pages; pages cannot be shared between child documents.
Usually, this behaviour makes perfect sense
because each child document contain an essential part of the document.
However, in some situations it may be desirable to compose
a document from a collection of parts
without having mandatory page breaks between then.
For this case, the package
provides a mechanism to include parts
by |\input| which can also be processed individually.
However, by construction this mechanism
requires manual handling of the content to be output.

%%%%%%%%%%%%%%%%%%%%%%%%%%%%%%%%%%%%%%%%
\DescribeMacro{\ifchilddocmanual}
The main file should be prepared as usual, see \secref{sec:include}.
However, the document body must make a distinction
between processing of an individual part and of the main document, e.g.:
%
\begin{center}
\begin{tabular}{l}
|\ifchilddocmanual|\\
|\input{\childdocname}|\\
|\||else|\\
\textit{document body with }|\input{|\textit{part}|}|\\
|\||fi|
\end{tabular}
\end{center}
%
The conditional |\ifchilddocmanual| is true whenever
a part to be included by |\input| is being compiled,
and the name of the part is stored in |\childdocname|.

%%%%%%%%%%%%%%%%%%%%%%%%%%%%%%%%%%%%%%%%
\DescribeMacro{\childdocby}
Each part to be included by |\input| should start with:
%
\begin{center}
\begin{tabular}{l}
|\input{childdoc.def}|\\
|\childdocby{|\textit{main}|}|\\
\end{tabular}
\end{center}
%
The directive |\childdocby| is similar to |\childdocof|
described in \secref{sec:include},
but the subsequent selection of content must be done manually.
To that end, both |\ifchilddoc| and |\ifchilddocmanual|
will be true upon processing of a part,
and the name of the part is stored in |\childdocname|.
Note that |\jobname| will be set to the filename of the current part
so that each part receives an individual |.aux| file
that does not interfere with the |.aux| file(s) of the main document.
This behaviour can be altered by the alternative form
|\childdocby[*]{|\textit{main}|}| (with a non-empty optional argument)
which uses the |.aux| file of the main document
by setting |\jobname| to \textit{main}.

%%%%%%%%%%%%%%%%%%%%%%%%%%%%%%%%%%%%%%%%%%%%%%%%%%%%%%%%%%%%%%%%%%%%%%%%%%%%%%%%
\subsection{Driver Development}
\label{sec:driver}

The \textsf{childdoc} mechanism can also be use for the development
of definition files such as \LaTeX{} styles or classes.
This case differs from the above setup with multiple parts
included by |\include| in that no |\includeonly| should be invoked.
This can be achieved by starting the include file
(before |\ProvidesPackage|) with:
%
\begin{center}
\begin{tabular}{l}
|\input{childdoc.def}|\\
|\childdocforward{|\textit{main}|}|\\
\end{tabular}
\end{center}
%
or alternatively with:
%
\begin{center}
\begin{tabular}{l}
|\input{childdoc.def}|\\
|\childdocby{|\textit{main}|}|\\
\end{tabular}
\end{center}
%
Both forms have slightly different effects as described above.
The main file is prepared as usual, see \secref{sec:include}.

%%%%%%%%%%%%%%%%%%%%%%%%%%%%%%%%%%%%%%%%%%%%%%%%%%%%%%%%%%%%%%%%%%%%%%%%%%%%%%%%
\subsection{Legacy Detection}
\label{sec:detection}

The directive |\childdocmain| in the main file can detect
whether the complete document or merely a child is to be compiled
even without using the directive |\childdocof|.
This method is deprecated because it is less robust
and there is no compelling reason to use it;
it is merely provided for backward compatibility
and it may be removed in future versions.

If the detection mechanism is to be used,
it is mandatory to correctly specify
the filename of the main file as the argument of |\childdocmain|:
%
\begin{center}
\begin{tabular}{l}
|\input{childdoc.def}|\\
|\childdocmain{|\textit{main}|}|\\
\end{tabular}
\end{center}
%
If |\jobname| does not match the argument \textit{main} of |\childdocmain|,
it is assumed that |\jobname| points to the child file to be compiled.
When using |\childdocmain| with the main file specified as argument,
it suffices to start a child file
with just |\input{|\textit{main}|}|
without loading of the package and using |\childdocof|.
If instead all processing is done
with the appropriate \textsf{childdoc} directives,
the argument of \textit{main} of |\childdocmain| can be empty.

An alternative version of the command line processing described
in \secref{sec:commandline} using the detection mechanism reads:
%
\begin{center}
|... -jobname "|\textit{target}|" "|[\textit{flags}]%
[|\def\jobname{|\textit{dest}|}|]|\input{|\textit{main}|}"|
\end{center}

%%%%%%%%%%%%%%%%%%%%%%%%%%%%%%%%%%%%%%%%%%%%%%%%%%%%%%%%%%%%%%%%%%%%%%%%%%%%%%%%
\subsection{Manual Code}
\label{sec:manual}

In case one cannot be certain whether the definitions file |childdoc.def|
is installed on the target \TeX{} distribution
and one prefers not to ship it,
it is conceivable to paste a few relevant commands into the sources.

To that end, drop all statements |\input{childdoc.def}|
and perform the replacements as outlined below.
Instead of |\childdocmain{|\textit{main}|}| add the following code
to the top of the main file:
%
\begin{center}
\begin{tabular}{l}
|\||ifdefined\childdocname\endinput\||fi\newif\ifchilddoc|\\
|\edef\childdocname{\scantokens\expandafter{\jobname\noexpand}}|\\
|\def\childdocmain{|\textit{main}|}\||ifx\childdocmain\childdocname\||else|\\
|\childdoctrue\includeonly{\childdocname}\let\jobname\childdocmain\||fi|\\
\end{tabular}
\end{center}
%
Instead of |\childdocof{|\textit{main}|}| just include the main file
at the top of each child file:
%
\begin{center}
|\input{|\textit{main}|}|
\end{center}
%
A simple redirection |\childdocforward{|\textit{dest}|}| is achieved by:
%
\begin{center}
|\def\jobname{|\textit{dest}|}\input{\jobname}|
\end{center}
%
The redirection with prefix
|\childdocforwardprefix[|\textit{prefix}|]{|\textit{dest}|}|
is accomplished by:
%
\begin{center}
\begin{tabular}{l}
|{\edef\jobname{\scantokens\expandafter{\jobname\noexpand}}|\\
|\def\redirectjob |\textit{prefix}|#1~~~{\gdef\jobname{|\textit{dest}|#1}}|\\
|\expandafter\redirectjob\jobname~~~}\input{\jobname}|
\end{tabular}
\end{center}

In an alternative approach,
child documents can be compiled by a specific command line
without additional code or specific definitions:
%
\begin{center}
|... -jobname "|\textit{target}|" "|[\textit{flags}]%
|\includeonly{|\textit{dest}|}\input{|\textit{main}|}"|
\end{center}
%

%%%%%%%%%%%%%%%%%%%%%%%%%%%%%%%%%%%%%%%%%%%%%%%%%%%%%%%%%%%%%%%%%%%%%%%%%%%%%%%%
%%%%%%%%%%%%%%%%%%%%%%%%%%%%%%%%%%%%%%%%%%%%%%%%%%%%%%%%%%%%%%%%%%%%%%%%%%%%%%%%
\section{Information}

%%%%%%%%%%%%%%%%%%%%%%%%%%%%%%%%%%%%%%%%%%%%%%%%%%%%%%%%%%%%%%%%%%%%%%%%%%%%%%%%
\subsection{Copyright}

Copyright \copyright{} 2017--2018 Niklas Beisert

This work may be distributed and/or modified under the
conditions of the \LaTeX{} Project Public License, either version 1.3
of this license or (at your option) any later version.
The latest version of this license is in
  \url{http://www.latex-project.org/lppl.txt}
and version 1.3 or later is part of all distributions of \LaTeX{}
version 2005/12/01 or later.

This work has the LPPL maintenance status `maintained'.

The Current Maintainer of this work is Niklas Beisert.

This work consists of the files |README.txt|, |childdoc.ins| and |childdoc.dtx|
as well as the derived files |childdoc.def|, |cdocsamp.tex|
with |cdocsch1.tex|, |cdocsch2.tex|, |cdocspt3.tex|, |cdocspt4.tex|,
|cdocsdrf.tex|, |cdocsfn1.tex|, |cdocsfn2.tex|
as well as |childdoc.pdf|.

%%%%%%%%%%%%%%%%%%%%%%%%%%%%%%%%%%%%%%%%%%%%%%%%%%%%%%%%%%%%%%%%%%%%%%%%%%%%%%%%
\subsection{Files and Installation}

The package consists of the files:
%
\begin{center}
\begin{tabular}{ll}
    |README.txt|   & readme file \\
    |childdoc.ins| & installation file \\
    |childdoc.dtx| & source file \\
    |childdoc.def| & definition file \\
    |cdocsamp.tex| & sample main file \\
    |cdocsch1.tex| & sample include file \\
    |cdocsch2.tex| & sample include file \\
    |cdocspt3.tex| & sample part file \\
    |cdocspt4.tex| & sample part file \\
    |cdocsdrf.tex| & sample redirection file \\
    |cdocsfn1.tex| & sample redirection file \\
    |cdocsfn2.tex| & sample redirection file \\
    |childdoc.pdf| & manual
\end{tabular}
\end{center}
%
The distribution consists of the files
|README.txt|, |childdoc.ins| and |childdoc.dtx|.
%
\begin{itemize}
\item
Run (pdf)\LaTeX{} on |childdoc.dtx|
to compile the manual |childdoc.pdf| (this file).
\item
Run \LaTeX{} on |childdoc.ins| to create the definitions file |childdoc.def|
and the sample |cdocsamp.tex| with include files
|cdocsch1.tex|, |cdocsch2.tex|, |cdocspt3.tex|, |cdocspt4.tex|,
|cdocsdrf.tex|, |cdocsfn1.tex|, |cdocsfn2.tex|.
Then copy the file |childdoc.def| to an appropriate directory of your \LaTeX{}
distribution, e.g.\ \textit{texmf-root}|/tex/latex/childdoc|.
\end{itemize}

%%%%%%%%%%%%%%%%%%%%%%%%%%%%%%%%%%%%%%%%%%%%%%%%%%%%%%%%%%%%%%%%%%%%%%%%%%%%%%%%
\subsection{Related CTAN Packages}

There are several other packages which offer a similar functionality:
%
\begin{itemize}
\item
The packages
\href{http://ctan.org/pkg/docmute}{\textsf{docmute}},
\href{http://ctan.org/pkg/includex}{\textsf{includex}} and
\href{http://ctan.org/pkg/standalone}{\textsf{standalone}}
provide commands to include only the document body of
a child file thus allowing both files to be compiled individually.
\item
The packages \href{http://ctan.org/pkg/subdocs}{\textsf{subdocs}}
and \href{http://ctan.org/pkg/subfiles}{\textsf{subfiles}}
provide structures in which the main and child documents can be
encapsulated and allowing them to be compiled individually.
The inclusion mechanism is different from the conventional |\include|.
\item
The package \href{http://ctan.org/pkg/combine}{\textsf{combine}}
is an elaborate solution to combine several documents into one.
\end{itemize}
%
See also the CTAN topic \href{http://ctan.org/topic/subdocs}{\textsf{subdocs}}
for further related packages.
The present package differs from the above solutions in that
a document structure constructed with the conventional |\include| mechanism
just needs two extra commands at the top of every file
such that all constituent files can be compiled individually.

%%%%%%%%%%%%%%%%%%%%%%%%%%%%%%%%%%%%%%%%%%%%%%%%%%%%%%%%%%%%%%%%%%%%%%%%%%%%%%%%
%\subsection{Feature Suggestions}
%
%The following is a list of features which may be useful for future
%versions of this package:
%%
%\begin{itemize}
%\item
%\ldots
%\end{itemize}

%%%%%%%%%%%%%%%%%%%%%%%%%%%%%%%%%%%%%%%%%%%%%%%%%%%%%%%%%%%%%%%%%%%%%%%%%%%%%%%%
\subsection{Revision History}

%%%%%%%%%%%%%%%%%%%%%%%%%%%%%%%%%%%%%%%%
\paragraph{v2.0:} 2018/12/30

\begin{itemize}
\item
immediate forward processing
\item
added |\childdocby| mechanism
\item
manual restructured
\end{itemize}

%%%%%%%%%%%%%%%%%%%%%%%%%%%%%%%%%%%%%%%%
\paragraph{v1.6:} 2018/01/17

\begin{itemize}
\item
application for development of include files
\item
corrections to manual
\end{itemize}

%%%%%%%%%%%%%%%%%%%%%%%%%%%%%%%%%%%%%%%%
\paragraph{v1.5:} 2017/05/21

\begin{itemize}
\item
more complete structuring introduced
\item
|\childdocof| introduced
\item
|\childdoc| renamed to |\childdocmain|
\item
|\childredirect| renamed to |\childdocforward| and |\childdocforwardprefix|
and functionality expanded
\end{itemize}

%%%%%%%%%%%%%%%%%%%%%%%%%%%%%%%%%%%%%%%%
\paragraph{v1.0:} 2017/04/27

\begin{itemize}
\item
manual and install package
\item
first version published on CTAN
\end{itemize}

%%%%%%%%%%%%%%%%%%%%%%%%%%%%%%%%%%%%%%%%
\paragraph{v0.6:} 2017/04/26

\begin{itemize}
\item
redirection mechanism added
\end{itemize}

%%%%%%%%%%%%%%%%%%%%%%%%%%%%%%%%%%%%%%%%
\paragraph{v0.5:} 2017/04/26

\begin{itemize}
\item
functionality in definition file
\end{itemize}


%%%%%%%%%%%%%%%%%%%%%%%%%%%%%%%%%%%%%%%%%%%%%%%%%%%%%%%%%%%%%%%%%%%%%%%%%%%%%%%%
%%%%%%%%%%%%%%%%%%%%%%%%%%%%%%%%%%%%%%%%%%%%%%%%%%%%%%%%%%%%%%%%%%%%%%%%%%%%%%%%
%%%%%%%%%%%%%%%%%%%%%%%%%%%%%%%%%%%%%%%%%%%%%%%%%%%%%%%%%%%%%%%%%%%%%%%%%%%%%%%%
\appendix

\settowidth\MacroIndent{\rmfamily\scriptsize 000\ }

 \DocInput{childdoc.dtx}

\end{document}
%</driver>
% \fi
%
% %%%%%%%%%%%%%%%%%%%%%%%%%%%%%%%%%%%%%%%%%%%%%%%%%%%%%%%%%%%%%%%%%%%%%%%%%%%%%%
% %%%%%%%%%%%%%%%%%%%%%%%%%%%%%%%%%%%%%%%%%%%%%%%%%%%%%%%%%%%%%%%%%%%%%%%%%%%%%%
% \section{Sample}
%\iffalse
%<*samplemain>
%\fi
%
% The following presents a sample document
% with two chapters, two parts, a title page,
% a compile flag as well as three forwarding files to set the flag.
% It consists of eight |.tex| files:
% \begin{center}
% \begin{tabular}{ll}
% |cdocsamp.tex|&main file\\
% |cdocsch1.tex|&include file for chapter 1\\
% |cdocsch2.tex|&include file for chapter 2\\
% |cdocspt3.tex|&include file for part 3\\
% |cdocspt4.tex|&include file for part 4\\
% |cdocsdrf.tex|&forwarding file for main file in draft mode\\
% |cdocsfi1.tex|&forwarding file for final version of chapter 1\\
% |cdocsfi2.tex|&forwarding file for final version of chapter 2\\
% \end{tabular}
% \end{center}
% Each of the eight files can be compiled directly by the \LaTeX{} compiler.
%
% %%%%%%%%%%%%%%%%%%%%%%%%%%%%%%%%%%%%%%
% \paragraph{Main File.}
%
% The main file is called |cdocsamp.tex|.
%
% Load the \textsf{childdoc} definitions and
% declare the filename for the main document:
%    \begin{macrocode}
\input{childdoc.def}
\childdocmain{}
%    \end{macrocode}

% Optional override for |\version| flag:
%    \begin{macrocode}
%%\ifchilddoc\else\providecommand{\version}{draft}\fi
%    \end{macrocode}

% Define the default values for the |\version| flag
% (|final| for the main file and |draft| for childs):
%    \begin{macrocode}
\ifchilddoc
\providecommand{\version}{draft}
\else
\providecommand{\version}{final}
\fi
%    \end{macrocode}

% Load the standard document class:
%    \begin{macrocode}
\documentclass[12pt]{article}
%    \end{macrocode}

% Start the document body:
%    \begin{macrocode}
\begin{document}
%    \end{macrocode}

% Declare a title page.
% Print title, part of document being processed and version flag:
%    \begin{macrocode}
\addtocounter{page}{-1}
\begin{center}
{\LARGE\bfseries{}childdoc example\par}
\vspace{1cm}
\ifchilddoc
\ifchilddocmanual part\else chapter\fi:
`\childdocname' of `\childdocjob'\par
\else
main document: `\childdocjob'\par
\fi
version: \version\par
\end{center}
\newpage
%    \end{macrocode}

% Manually include selected file,
% otherwise process as usual:
%    \begin{macrocode}
\ifchilddocmanual
\section*{part `\childdocname'}
\input{\childdocname}
\else
%    \end{macrocode}

% Include the two chapters:
%    \begin{macrocode}
\include{cdocsch1}
\include{cdocsch2}
%    \end{macrocode}

% Include the two parts unless only chapters should be displayed:
%    \begin{macrocode}
\ifchilddoc\else
\section{part three}
\input{cdocspt3}
\section{part four}
\input{cdocspt4}
\fi
%    \end{macrocode}

% Process as usual until here:
%    \begin{macrocode}
\fi
%    \end{macrocode}

% End of document body:
%    \begin{macrocode}
\end{document}
%    \end{macrocode}
%\iffalse
%</samplemain>
%\fi
%
% %%%%%%%%%%%%%%%%%%%%%%%%%%%%%%%%%%%%%%
% \paragraph{Chapter Include Files.}
%
% The include files are called |cdocsch1.tex| and |cdocsch2.tex|.
%
%\iffalse
%<*samplechap1|samplechap2>
%\fi

% Optional override for |\version| flag:
%    \begin{macrocode}
%%\providecommand{\version}{final}
%    \end{macrocode}

% Include the main document:
%    \begin{macrocode}
\input{childdoc.def}
\childdocof{cdocsamp}
%    \end{macrocode}

%\iffalse
%</samplechap1|samplechap2>
%\fi
%
%\iffalse
%<*samplechap1>
%\fi
% Some text for chapter 1:
%    \begin{macrocode}
\section{one}
some text in chapter one
%    \end{macrocode}

%\iffalse
%</samplechap1>
%\fi
% Some text for chapter 2:
%\iffalse
%<*samplechap2>
%\fi
%    \begin{macrocode}
\section{two}
more text in chapter two
%    \end{macrocode}

%\iffalse
%</samplechap2>
%\fi
%
% %%%%%%%%%%%%%%%%%%%%%%%%%%%%%%%%%%%%%%
% \paragraph{Part Include Files.}
%
% The include files are called |cdocspt3.tex| and |cdocspt4.tex|.
%
%\iffalse
%<*samplepart3|samplepart4>
%\fi

% Optional override for |\version| flag:
%    \begin{macrocode}
%%\providecommand{\version}{final}
%    \end{macrocode}

% Include the main document:
%    \begin{macrocode}
\input{childdoc.def}
\childdocby{cdocsamp}
%    \end{macrocode}

%\iffalse
%</samplepart3|samplepart4>
%\fi
%
%\iffalse
%<*samplepart3>
%\fi
% Some text for part 3:
%    \begin{macrocode}
some text in part three
%    \end{macrocode}

%\iffalse
%</samplepart3>
%\fi
% Some text for part 4:
%\iffalse
%<*samplepart4>
%\fi
%    \begin{macrocode}
more text in part four
%    \end{macrocode}

%\iffalse
%</samplepart4>
%\fi
%
% %%%%%%%%%%%%%%%%%%%%%%%%%%%%%%%%%%%%%%
% \paragraph{Forwarding for a Complete Draft.}
%
% The following forwarding file |cdocsdrf.tex|
% compiles the main document in draft mode:
%\iffalse
%<*sampledraft>
%\fi
%    \begin{macrocode}
\def\version{draft}
\input{childdoc.def}
\childdocforward{cdocsamp}
%    \end{macrocode}

%\iffalse
%</sampledraft>
%\fi
%
% %%%%%%%%%%%%%%%%%%%%%%%%%%%%%%%%%%%%%%
% \paragraph{Forwarding for Final Version of the Chapters.}
%
% The following forwarding files |cdocsfn1.tex| and |cdocsfn2.tex|
% (with identical content)
% compile the final versions of the child documents
% |cdocsch1.tex| and |cdocsch2.tex|, respectively:
%\iffalse
%<*samplefinal>
%\fi
%    \begin{macrocode}
\def\version{final}
\input{childdoc.def}
\childdocforwardprefix[cdocsamp]{cdocsfn}{cdocsch}
%    \end{macrocode}

%\iffalse
%</samplefinal>
%\fi
%
% %%%%%%%%%%%%%%%%%%%%%%%%%%%%%%%%%%%%%%
% \paragraph{Command Line Processing.}
%
% The following three command lines generate the output files
% |cdocscld|, |cdocscl1| and |cdocscl2|
% which should be identical to
% |cdocsdrf|, |cdocsch1| and |cdocsfn2|, respectively:
% \begin{center}
% \begin{tabular}{l}
% |latex -jobname cdocscld \|\\
% |  "\def\version{draft}\input{childdoc.def}\childdocforward{cdocsamp}"|\\
% |latex -jobname cdocscl1 \|\\
% |  "\input{childdoc.def}\childdocforward[cdocsamp]{cdocsch1}"|\\
% |latex -jobname cdocscl2 \|\\
% |  "\def\version{final}\input{childdoc.def}\childdocforward{cdocsch2}"|
% \end{tabular}
% \end{center}
% Note that the trailing backslash on each first line
% merely continues the input to the second line
% (for convenient cut ant paste).
% Furthermore, the command |latex| can be replaced by any
% of its alternative versions such as |pdflatex|.
%
% %%%%%%%%%%%%%%%%%%%%%%%%%%%%%%%%%%%%%%%%%%%%%%%%%%%%%%%%%%%%%%%%%%%%%%%%%%%%%%
% %%%%%%%%%%%%%%%%%%%%%%%%%%%%%%%%%%%%%%%%%%%%%%%%%%%%%%%%%%%%%%%%%%%%%%%%%%%%%%
% \section{Implementation}
%\iffalse
%<*package>
%\fi
%
% This section describes the definitions file |childdoc.def|.

% The definitions cannot be loaded using |\usepackage| or |\RequirePackage|
% which has a mechanism to prevent loading a style file more than once.
% When loading the definitions by means of |\input|
% multiple instances have to be prevented manually:
%\iffalse
%This code needs to be before the `\ProvidesFile' directive
%which is defined at the beginning of this file.
%Therefore it is also placed there and commented out here.
%</package>
%<*discard>
%\fi
%    \begin{macrocode}
\ifdefined\childdocmain\endinput\fi
%    \end{macrocode}
%\iffalse
%</discard>
%<*package>
%\fi
%
% \macro{\ifchilddoc}
% \macro{\ifchilddocmanual}
% The conditional |\ifchilddoc| tells whether a
% child (true) or main (false) document is being compiled.
% The conditional |\ifchilddocmanual| tells whether
% the |\includeonly| mechanism is used (false) or
% the selection of child files must be performed manually (true).
% The definitions initialise to false:
%    \begin{macrocode}
\newif\ifchilddoc
\newif\ifchilddocmanual
%    \end{macrocode}

% \macro{\childdocname}
% \macro{\childdocjob}
% The macro |\childdocname| stores the name of the main document
% to be compiled. The macro |\childdocjob| stores the name of
% the document on which the \LaTeX{} compiler was originally invoked.
% The content of |\jobname| cannot be compared
% to filenames specified in the source due to different catcodes.
% The following code rescans |\jobname|, stores the result
% in |\childdocname| and saves a copy in |\childdocjob|:
%    \begin{macrocode}
\edef\childdocname{\scantokens\expandafter{\jobname\noexpand}}
\let\childdocjob\childdocname
%    \end{macrocode}

% \macro{\childdocdisable}
% The macro |\childdocdisable| prevents the main file
% from being processed more than once.
% At this stage, the main document command |\childdocmain|
% is assumed to be called once again where it should do nothing.
% Any subsequent call to it should prevent
% a secondary processing of the main document
% It overwrites the forwarding commands
% |\childdocof| and |\childdocforward|
% with empty macros to prevent further inclusions of the main document:
%    \begin{macrocode}
\newcommand{\childdocdisable}
{
  \renewcommand{\childdocmain}[1]{\renewcommand{\childdocmain}[1]{\endinput}}
  \renewcommand{\childdocof}[1]{}
  \renewcommand{\childdocby}[2][]{}
  \renewcommand{\childdocforward}[2][]{}
  \renewcommand{\childdocdisable}{}
}
%    \end{macrocode}

% \macro{\childdocmain}
% The macro |\childdocmain| is to be called at the top of the main file
% with nothing or the main filename (without extension) as argument.
% First, it breaks loops.
% If the argument is not empty and does not match |\childdocname|
% (which is set by the first inclusion of |childdoc.def|),
% |\ifchilddoc| is set to true, |\includeonly| is applied to the child file
% and |\jobname| is set to the main file
% (for proper handling of |.aux| files):
%    \begin{macrocode}
\newcommand{\childdocmain}[1]
{
  \childdocdisable\childdocmain{}
  \if?#1?\else
    \begingroup
      \def\childdoctmp{#1}
      \ifx\childdoctmp\childdocname
        \def\childdoctmp{}
      \else
        \def\childdoctmp
        {
          \childdoctrue
          \includeonly{\childdocname}
          \def\childdocjob{#1}
          \def\jobname{#1}
        }
      \fi
      \expandafter
    \endgroup
    \childdoctmp
  \fi
}
%    \end{macrocode}

% \macro{\childdocof}
% The command |\childdocof| redirects
% compilation to the main file |#1|.
%    \begin{macrocode}
\newcommand{\childdocof}[1]
{
  \childdocdisable
  \childdoctrue
  \includeonly{\childdocname}
  \def\jobname{#1}
  \def\childdocjob{#1}
  \input{#1}
}
%    \end{macrocode}

% \macro{\childdocby}
% The command |\childdocby| ....
%    \begin{macrocode}
\newcommand{\childdocby}[2][]
{
  \childdocdisable
  \childdoctrue
  \childdocmanualtrue
  \if?#1?\else
    \def\jobname{#2}
  \fi
  \def\childdocjob{#2}
  \input{#2}
  \endinput
}
%    \end{macrocode}

% \macro{\childdocforward}
% The command |\childdocforward| redirects
% compilation to the main file or
% (if the optional argument is given) a child file.
% Parameters are set as if the main file
% or a child file starting with |\childdocof| was compiled.
% Then compilation is handed over to the main file:
%    \begin{macrocode}
\newcommand{\childdocforward}[2][]
{
  \begingroup
    \if?#1?
      \def\childdoctmp
      {
        \def\childdocname{#2}
        \def\childdocjob{#2}
        \def\jobname{#2}
        \input{#2}
        \endinput
      }
    \else
      \def\childdoctmp
      {
        \childdocdisable
        \def\childdocname{#2}
        \childdoctrue
        \includeonly{#2}
        \def\childdocjob{#1}
        \def\jobname{#1}
        \input{#1}
        \endinput
      }
    \fi
    \expandafter
  \endgroup
  \childdoctmp
}
%    \end{macrocode}

% \macro{\childdocforwardprefix}
% The command |\childdocforwardprefix| redirects
% compilation to the main or a child file by means of a pattern.
% The prefix |#1| in the current filename is replaced by |#2|
% and the suffix of the current filename is kept
% (it is assumed that the filename does not contain the substring `|~~~|'
% which is used as a delimiter).
% Compilation is handed over to the new file by |\childdocforward|:
%    \begin{macrocode}
\newcommand{\childdocforwardprefix}[3][]
{
  \begingroup
    \def\childdocextract #2##1~~~{\def\childdoctmp{\childdocforward[#1]{#3##1}}}
    \expandafter\childdocextract\childdocname~~~
    \expandafter
  \endgroup
  \childdoctmp
}
%    \end{macrocode}

% \macro{\childdoc}
% The deprecated macro |\childdoc| is a legacy version of |\childdocmain|:
%    \begin{macrocode}
\newcommand{\childdoc}{\childdocmain}
%    \end{macrocode}

% \macro{\childdocredirect}
% The deprecated macro |\childdocredirect| is a legacy version
% of |\childdocforward| and |\childdocforwardprefix|:
%    \begin{macrocode}
\newcommand{\childdocredirect}[2][]
{
  \begingroup
    \if?#1?
      \def\childdoctmp{\childdocforward{#2}}
    \else
      \def\childdoctmp{\childdocforwardprefix{#1}{#2}}
    \fi
    \expandafter
  \endgroup
  \childdoctmp
}
%    \end{macrocode}

%\iffalse
%</package>
%\fi
%
\endinput

\childdocmain{}
%    \end{macrocode}

% Optional override for |\version| flag:
%    \begin{macrocode}
%%\ifchilddoc\else\providecommand{\version}{draft}\fi
%    \end{macrocode}

% Define the default values for the |\version| flag
% (|final| for the main file and |draft| for childs):
%    \begin{macrocode}
\ifchilddoc
\providecommand{\version}{draft}
\else
\providecommand{\version}{final}
\fi
%    \end{macrocode}

% Load the standard document class:
%    \begin{macrocode}
\documentclass[12pt]{article}
%    \end{macrocode}

% Start the document body:
%    \begin{macrocode}
\begin{document}
%    \end{macrocode}

% Declare a title page.
% Print title, part of document being processed and version flag:
%    \begin{macrocode}
\addtocounter{page}{-1}
\begin{center}
{\LARGE\bfseries{}childdoc example\par}
\vspace{1cm}
\ifchilddoc
\ifchilddocmanual part\else chapter\fi:
`\childdocname' of `\childdocjob'\par
\else
main document: `\childdocjob'\par
\fi
version: \version\par
\end{center}
\newpage
%    \end{macrocode}

% Manually include selected file,
% otherwise process as usual:
%    \begin{macrocode}
\ifchilddocmanual
\section*{part `\childdocname'}
\input{\childdocname}
\else
%    \end{macrocode}

% Include the two chapters:
%    \begin{macrocode}
\include{cdocsch1}
\include{cdocsch2}
%    \end{macrocode}

% Include the two parts unless only chapters should be displayed:
%    \begin{macrocode}
\ifchilddoc\else
\section{part three}
\input{cdocspt3}
\section{part four}
\input{cdocspt4}
\fi
%    \end{macrocode}

% Process as usual until here:
%    \begin{macrocode}
\fi
%    \end{macrocode}

% End of document body:
%    \begin{macrocode}
\end{document}
%    \end{macrocode}
%\iffalse
%</samplemain>
%\fi
%
% %%%%%%%%%%%%%%%%%%%%%%%%%%%%%%%%%%%%%%
% \paragraph{Chapter Include Files.}
%
% The include files are called |cdocsch1.tex| and |cdocsch2.tex|.
%
%\iffalse
%<*samplechap1|samplechap2>
%\fi

% Optional override for |\version| flag:
%    \begin{macrocode}
%%\providecommand{\version}{final}
%    \end{macrocode}

% Include the main document:
%    \begin{macrocode}
% \iffalse
%
% childdoc.dtx Copyright (C) 2017-2018 Niklas Beisert
%
% This work may be distributed and/or modified under the
% conditions of the LaTeX Project Public License, either version 1.3
% of this license or (at your option) any later version.
% The latest version of this license is in
%   http://www.latex-project.org/lppl.txt
% and version 1.3 or later is part of all distributions of LaTeX
% version 2005/12/01 or later.
%
% This work has the LPPL maintenance status `maintained'.
%
% The Current Maintainer of this work is Niklas Beisert.
%
% This work consists of the files childdoc.dtx and childdoc.ins
% and the derived files childdoc.def and cdocsamp.tex with
% cdocsch1.tex, cdocsch2.tex, cdocsdrf.tex, cdocsfn1.tex, cdocsfn2.tex.
%
%<package>\ifdefined\childdocmain\endinput\fi
%<package>\ProvidesFile{childdoc.def}[2018/12/30 v2.0 child document driver]
%<samplemain>\ProvidesFile{cdocsamp.tex}[2018/12/30 v2.0 sample for childdoc]
%<*driver>
%\ProvidesFile{childdoc.drv}[2018/12/30 v2.0 childdoc reference manual file]
\PassOptionsToClass{10pt,a4paper}{article}
\documentclass{ltxdoc}

\usepackage[margin=35mm]{geometry}
\usepackage{hyperref}
\usepackage{hyperxmp}
\usepackage[usenames]{color}

\hypersetup{colorlinks=true}
\hypersetup{pdfstartview=FitH}
\hypersetup{pdfpagemode=UseNone}
\hypersetup{pdfsource={}}
\hypersetup{pdflang={en-UK}}
\hypersetup{pdfcopyright={Copyright 2017-2018 Niklas Beisert.
  This work may be distributed and/or modified under the
  conditions of the LaTeX Project Public License, either version 1.3
  of this license or (at your option) any later version.}}
\hypersetup{pdflicenseurl={http://www.latex-project.org/lppl.txt}}
\hypersetup{pdfcontactaddress={ETH Zurich, ITP, HIT K,
  Wolfgang-Pauli-Strasse 27}}
\hypersetup{pdfcontactpostcode={8093}}
\hypersetup{pdfcontactcity={Zurich}}
\hypersetup{pdfcontactcountry={Switzerland}}
\hypersetup{pdfcontactemail={nbeisert@itp.phys.ethz.ch}}
\hypersetup{pdfcontacturl={http://people.phys.ethz.ch/\xmptilde nbeisert/}}

\newcommand{\secref}[1]{\hyperref[#1]{section \ref*{#1}}}

\parskip1ex
\parindent0pt
\let\olditemize\itemize
\def\itemize{\olditemize\parskip0pt}

\begin{document}

\title{The \textsf{childdoc} Package}
\hypersetup{pdftitle={The childdoc Package}}
\author{Niklas Beisert\\[2ex]
  Institut f\"ur Theoretische Physik\\
  Eidgen\"ossische Technische Hochschule Z\"urich\\
  Wolfgang-Pauli-Strasse 27, 8093 Z\"urich, Switzerland\\[1ex]
  \href{mailto:nbeisert@itp.phys.ethz.ch}
  {\texttt{nbeisert@itp.phys.ethz.ch}}}
\hypersetup{pdfauthor={Niklas Beisert}}
\hypersetup{pdfsubject={Manual for the LaTeX2e Package childdoc}}
\date{30 December 2018, \textsf{v2.0}}
\maketitle

\begin{abstract}\noindent
\textsf{childdoc} is a \LaTeXe{} package
that enables the direct compilation
of document sections included by |\include|
to individual files.
\end{abstract}

\begingroup
\parskip0ex
\tableofcontents
\endgroup

%%%%%%%%%%%%%%%%%%%%%%%%%%%%%%%%%%%%%%%%%%%%%%%%%%%%%%%%%%%%%%%%%%%%%%%%%%%%%%%%
%%%%%%%%%%%%%%%%%%%%%%%%%%%%%%%%%%%%%%%%%%%%%%%%%%%%%%%%%%%%%%%%%%%%%%%%%%%%%%%%
\section{Introduction}

\LaTeX{} provides a mechanism to structure a large document (such as a book)
into a main file and several child files (containing the chapters)
using the |\include| command.
This mechanism is beneficial for documents
which span hundreds of pages in order to
make the source file(s) more manageable.
Moreover, compilation can be restricted to
selected child files by means of the |\includeonly| command.
The latter feature can be used to reduce the compilation time while editing
(this was significantly more useful in the earlier days of \LaTeX{})
or to generate a smaller document which is easier to navigate.
Another application of |\includeonly| is to generate
documents consisting of selected parts of the complete document.

However, there are a few drawbacks of the plain |\include| mechanism:
\begin{itemize}
\item
The child files cannot be compiled on their own,
they can only be compiled via the main file.
A naive editing environment
(such as a text editor with an option
to have the current file processed by \LaTeX)
may require one to switch to the main file before compiling;
attempting to compile the child file produces errors.
\item
The main file must be modified (each time)
to adjust the |\includeonly| command
to the present needs. This easily leaves the main file in a messy state.
\item
The generated document will always carry the filename
of the main document. This is inconvenient if
several child files are to be compiled and
to be kept for distribution.
\end{itemize}

The present package provides a simple interface
to make child files individually compilable by \LaTeX{}.
Compiling a child file then has the same effect as compiling
the main file with an |\includeonly| command
to select the appropriate child.
Moreover the generated document will carry the name of the child
rather than the main file.
This resolves all three above issues.

This feature is meant to make the editing of books,
thesis documents and lecture notes somewhat more convenient.
However, the package can also be used efficiently for
composing a series of documents (such as exercise sheets)
which are typically distributed individually.
It then assists the author in generating the individual documents
(potentially in different versions)
as well as a document containing the collected series.
Another application is in developing style files
or other kinds of included material
where compilation of the style file could redirect
to a sample or test file.

%%%%%%%%%%%%%%%%%%%%%%%%%%%%%%%%%%%%%%%%%%%%%%%%%%%%%%%%%%%%%%%%%%%%%%%%%%%%%%%%
%%%%%%%%%%%%%%%%%%%%%%%%%%%%%%%%%%%%%%%%%%%%%%%%%%%%%%%%%%%%%%%%%%%%%%%%%%%%%%%%
\section{Usage}

First of all, the package \textsf{childdoc} is \emph{not} a standard
\LaTeXe{} |.sty| style file! Therefore it needs to be invoked in
a non-standard way.

%%%%%%%%%%%%%%%%%%%%%%%%%%%%%%%%%%%%%%%%%%%%%%%%%%%%%%%%%%%%%%%%%%%%%%%%%%%%%%%%
\subsection{Included Files}
\label{sec:include}

%%%%%%%%%%%%%%%%%%%%%%%%%%%%%%%%%%%%%%%%
\DescribeMacro{\childdocmain}
To use the package, add the commands
\begin{center}
\begin{tabular}{l}
|\input{childdoc.def}|\\
|\childdocmain{}|\\
\end{tabular}
\end{center}
at the very top of the main \LaTeX{} file,
in particular \emph{before} the |\documentclass| statement!
The argument of |\childdocmain| should be left empty
(but it must be present).

%%%%%%%%%%%%%%%%%%%%%%%%%%%%%%%%%%%%%%%%
\DescribeMacro{\childdocof}
Furthermore, add the commands
\begin{center}
\begin{tabular}{l}
|\input{childdoc.def}|\\
|\childdocof{|\textit{main}|}|\\
\end{tabular}
\end{center}
at the top of every child file \textit{child}
which is included by |\include{|\textit{child}|}|
from within the main file
(or at least for those files to be compiled individually).
The argument \textit{main} must be the filename of the main file.

There are a couple of
considerations in setting up the main and child documents:

%%%%%%%%%%%%%%%%%%%%%%%%%%%%%%%%%%%%%%%%
\paragraph{Restrictions.}

Please note the following restrictions:
\begin{itemize}
\item
|\childdocmain| must be called with one argument \textit{main}
to ensure compatibility with earlier version of the package.
It must either be empty (|\childdocmain{}|)
or precisely match the filename of the main file in which it is specified.
See \secref{sec:detection} for further information.
\item
The filename \textit{main} must be specified without the |.tex| extension.
\item
The filename \textit{main} is case sensitive
(even in case-insensitive file systems)
due to internal string comparison.
\item
The argument \textit{main} should be fully expanded, it cannot be a macro.
\item
Subdirectories and special characters should be avoided in filenames.
\item
The command |\childdocmain{|\textit{main}|}| must be followed by a whitespace.
It should not be followed immediately by another command
or by a comment mark `|%|'.
This is because the \TeX{} parser reads the token immediately following
the argument of |\childdocmain| and puts it
at the beginning of every child section;
however, a white\-space is ignored.
\end{itemize}

%%%%%%%%%%%%%%%%%%%%%%%%%%%%%%%%%%%%%%%%
\paragraph{Content of Main File.}

It is advisable to place all content in the child files included by |\include|.
Any output contained in the main file will appear in all child documents
unless suppressed manually;
it cannot be suppressed automatically by the |\includeonly| directive
and thus should normally be avoided.
A method to include some content in the main file
by means of conditional processing is described in \secref{sec:conditional}.

%%%%%%%%%%%%%%%%%%%%%%%%%%%%%%%%%%%%%%%%
\paragraph{Page Numbering.}

When only a part of the document is compiled,
the appropriate numbering of pages
(as well as other status parameters)
is determined from the |.aux| files.
The latter contain information from previous passes.
However this information needs to propagate through
all intermediate child documents.
Therefore the page numbering in child documents may well
be inconsistent until the complete document is compiled at least once.

A useful (if unconventional) way to always ensure a consistent
page numbering is to restart the numbering in each child document
and denote the pages by `\textit{child}|.|\textit{page}'
where \textit{child} represents the chapter/section number of the child file.
This can be achieved by the command
|\numberwithin{page}{|\textit{child}|}|
of the \textsf{amsmath} package
where \textit{child} can be |chapter| or |section|
depending on the chosen structuring.
Alternatively, one can modify the macro |\thepage| appropriately
and reset the counter |page| at the start of each child file.

%%%%%%%%%%%%%%%%%%%%%%%%%%%%%%%%%%%%%%%%%%%%%%%%%%%%%%%%%%%%%%%%%%%%%%%%%%%%%%%%
\subsection{Conditional Processing}
\label{sec:conditional}

The package provides a mechanism to compile different versions
of a document. To customise the versions further some conditional processing
can come in handy to distinguish which version is being compiled.
The package provides two macros to describe the compilation context:

%%%%%%%%%%%%%%%%%%%%%%%%%%%%%%%%%%%%%%%%
\DescribeMacro{\ifchilddoc}
The conditional |\ifchilddoc| distinguishes between the compilation of
child documents and the main document:
%
\begin{center}
|\ifchilddoc |\textit{child-code}| |[|\||else |\textit{main-code}]| \||fi|
\end{center}

%%%%%%%%%%%%%%%%%%%%%%%%%%%%%%%%%%%%%%%%
\DescribeMacro{\childdocname}
\DescribeMacro{\childdocjob}
The macro |\childdocname| contains the filename (without extension)
of the main or child file being processed.
Note that |\childdocjob| will always contain the name of the main file.

%%%%%%%%%%%%%%%%%%%%%%%%%%%%%%%%%%%%%%%%
\paragraph{Title Page.}

Conditional processing can be used to include a title or banner page
in the main document when proper precautions are taken.
Importantly, the code in the main file should ensure that the page counter
(as well as other status parameters which are stored in the |.aux| files)
takes the same value after the conditional processing.
Otherwise the page numbers may take divergent values
depending on which part is compiled.

For example, a title page could be declared by:
%
\begin{center}
\begin{tabular}{l}
|\ifchilddoc\||else|\\
|\addtocounter{page}{-1}|\\
\textit{code for title page}\\
|\newpage|\\
|\||fi|
\end{tabular}
\end{center}
%
A banner page for the child documents can be generated by:
%
\begin{center}
\begin{tabular}{l}
|\ifchilddoc|\\
|\addtocounter{page}{-1}|\\
\textit{code for banner page}\\
|\newpage|\\
|\||fi|
\end{tabular}
\end{center}
%
Here one could write a message such as:
\begin{center}
|This is the part \childdocname{} of \childdocjob{}.|
\end{center}

%%%%%%%%%%%%%%%%%%%%%%%%%%%%%%%%%%%%%%%%%%%%%%%%%%%%%%%%%%%%%%%%%%%%%%%%%%%%%%%%
\subsection{Flags}
\label{sec:flags}

The package makes it easy to generate different versions
of the main or child documents.
To this end compilation flags can be defined
and assigned different default values.
They will be particularly useful in conjunction
with the forwarding mechanism described in \secref{sec:forward}.

For example, it may be useful to have a flag |\version|
which can be set to |draft| or |final|.
The document source will contain some conditional code
depending on the value of |\version|.
Suppose further, the flag should default to |final| for the main file
and to |draft| for child files
which is a natural assignment for editing the document.
This is achieved by placing the following code
in the preamble of the main document
(below the |\childdocmain| directive):
%
\begin{center}
\begin{tabular}{l}
|\ifchilddoc|\\
|\providecommand{\version}{draft}|\\
|\||else|\\
|\providecommand{\version}{final}|\\
|\||fi|
\end{tabular}
\end{center}
%
The definition by |\providecommand| makes sure
that previous definitions are not overwritten.
Further statements |\providecommand{\version}{...}|
can thus be added before the above code to override it.

For the main file, one might add a line
(between |\childdocmain| and the above block)
%
\begin{center}
|%\ifchilddoc\||else\providecommand{\version}{draft}\||fi|
\end{center}
%
which can be uncommented to produce a draft version.
Likewise one can add a line to the very top of a child file
(above the |\childdocof{|\textit{main}|}| directive)
%
\begin{center}
|%\providecommand{\version}{final}|
\end{center}
%
which can be uncommented to produce the final version of this child document.

%%%%%%%%%%%%%%%%%%%%%%%%%%%%%%%%%%%%%%%%%%%%%%%%%%%%%%%%%%%%%%%%%%%%%%%%%%%%%%%%
\subsection{Forwarding}
\label{sec:forward}

Different versions of the main or child documents
using compilation flags as described in \secref{sec:flags}
can be (permanently) stored in different files
for convenient compilation, viewing and distribution.
To this end, the package defines a command
to pass on compilation to a different file:

%%%%%%%%%%%%%%%%%%%%%%%%%%%%%%%%%%%%%%%%
\DescribeMacro{\childdocforward}
The command |\childdocforward| redirects processing to
another source file:
%
\begin{center}
\begin{tabular}{l}
|\input{childdoc.def}|\\
|\childdocforward[|\textit{main}|]{|\textit{dest}|}|\\
\end{tabular}
\end{center}
%
The argument \textit{dest} is the destination file
(without extension).
It should be the main file or one of the child files.
Note that further \textsf{childdoc} directives
such as |\childdocof| and |\childdocforward|
in the indicated file will be processed in this form.
The optional argument \textit{main}
passes on directly to the main file \textit{main}
while pretending to compile the child \textit{dest}.
This form behaves as if \textit{dest}
issues |\childdocof{|\textit{main}|}| right away,
and no further \textsf{childdoc} directives will be processed.

%%%%%%%%%%%%%%%%%%%%%%%%%%%%%%%%%%%%%%%%
\DescribeMacro{\...prefix}
In the alternative form |\childdocforwardprefix|,
%
\begin{center}
\begin{tabular}{l}
|\input{childdoc.def}|\\
|\childdocforwardprefix[|\textit{main}|]{|\textit{prefix}|}{|\textit{dest}|}|
\end{tabular}
\end{center}
%
the destination file is determined by a pattern
depending on the current file:
To make this work, the current file must be called
`{\textit{prefix}\hspace{0.2em}\textit{suffix}}'
with \textit{prefix} matching precisely the argument.
Processing is then passed on to the file
`{\textit{dest}\hspace{0.2em}\textit{suffix}}'.
Surely, the same effect is achieved by
directly specifying the
argument `{\textit{dest}\hspace{0.2em}\textit{suffix}}'
in the first form.
However, that requires to set up a different file
for each child. With the alternative form of the command
all these files can have exactly the same content
which simplifies setting them up and maintaining them.

For example, the following file |draft.tex|
with a compilation flag |\version| as described in \secref{sec:flags}
compiles the main document as a draft:
%
\begin{center}
\begin{tabular}{l}
|\def\version{draft}|\\
|\input{childdoc.def}|\\
|\childdocforward{|\textit{main}|}|
\end{tabular}
\end{center}
%
Likewise, the following files |final|\textit{nn}|.tex|
compile the final version of the child document
|child|\textit{nn}|.tex|:
%
\begin{center}
\begin{tabular}{l}
|\def\version{final}|\\
|\input{childdoc.def}|\\
|\childdocforwardprefix{final}{child}|
\end{tabular}
\end{center}
%

Note that when several versions of a main file and/or of each child file
are to be generated, it may be convenient to set up a |Makefile| or
shell script to automatise the process.

%%%%%%%%%%%%%%%%%%%%%%%%%%%%%%%%%%%%%%%%%%%%%%%%%%%%%%%%%%%%%%%%%%%%%%%%%%%%%%%%
\subsection{Command Line Processing}
\label{sec:commandline}

The effect of redirection files can also be achieved by invoking
the \LaTeX{} compiler with a more elaborate command line.
Most conveniently this should be done as part
of a shell script or a |Makefile|.

When using \textsf{childdoc} in the main file, the following
command lines effectively perform a redirection
(note that depending on the shell being used,
backslashes may have to be doubled: `|\|' $\to$ `|\\|'):
%
\begin{center}
|... -jobname "|\textit{target}|" |\\|"|[\textit{flags}]%
|\input{childdoc.def}\childdocforward[|\textit{main}|]{|\textit{dest}|}"|
\end{center}
%
Here \textit{target} is the name of the output file,
\textit{main} is the name of the main file
and \textit{dest} is the name of the main or child file to be processed
(all filenames without extensions).
The optional argument \textit{main} can be omitted
if \textit{main} matches \textit{dest}.
Optionally, compilation \textit{flags} can be defined via |\def| commands.
This command line makes the \TeX{} engine believe
it is compiling the file \textit{target}
whose content is specified as the latter parameter.
The provided code then forwards the processing to
\textit{main} or \textit{dest} as described in \secref{sec:forward}.

%%%%%%%%%%%%%%%%%%%%%%%%%%%%%%%%%%%%%%%%%%%%%%%%%%%%%%%%%%%%%%%%%%%%%%%%%%%%%%%%
\subsection{Include by Input}
\label{sec:input}

Including child documents by |\include| has some restrictions by design.
Most notably, the content of a child document always occupies
its own set of pages; pages cannot be shared between child documents.
Usually, this behaviour makes perfect sense
because each child document contain an essential part of the document.
However, in some situations it may be desirable to compose
a document from a collection of parts
without having mandatory page breaks between then.
For this case, the package
provides a mechanism to include parts
by |\input| which can also be processed individually.
However, by construction this mechanism
requires manual handling of the content to be output.

%%%%%%%%%%%%%%%%%%%%%%%%%%%%%%%%%%%%%%%%
\DescribeMacro{\ifchilddocmanual}
The main file should be prepared as usual, see \secref{sec:include}.
However, the document body must make a distinction
between processing of an individual part and of the main document, e.g.:
%
\begin{center}
\begin{tabular}{l}
|\ifchilddocmanual|\\
|\input{\childdocname}|\\
|\||else|\\
\textit{document body with }|\input{|\textit{part}|}|\\
|\||fi|
\end{tabular}
\end{center}
%
The conditional |\ifchilddocmanual| is true whenever
a part to be included by |\input| is being compiled,
and the name of the part is stored in |\childdocname|.

%%%%%%%%%%%%%%%%%%%%%%%%%%%%%%%%%%%%%%%%
\DescribeMacro{\childdocby}
Each part to be included by |\input| should start with:
%
\begin{center}
\begin{tabular}{l}
|\input{childdoc.def}|\\
|\childdocby{|\textit{main}|}|\\
\end{tabular}
\end{center}
%
The directive |\childdocby| is similar to |\childdocof|
described in \secref{sec:include},
but the subsequent selection of content must be done manually.
To that end, both |\ifchilddoc| and |\ifchilddocmanual|
will be true upon processing of a part,
and the name of the part is stored in |\childdocname|.
Note that |\jobname| will be set to the filename of the current part
so that each part receives an individual |.aux| file
that does not interfere with the |.aux| file(s) of the main document.
This behaviour can be altered by the alternative form
|\childdocby[*]{|\textit{main}|}| (with a non-empty optional argument)
which uses the |.aux| file of the main document
by setting |\jobname| to \textit{main}.

%%%%%%%%%%%%%%%%%%%%%%%%%%%%%%%%%%%%%%%%%%%%%%%%%%%%%%%%%%%%%%%%%%%%%%%%%%%%%%%%
\subsection{Driver Development}
\label{sec:driver}

The \textsf{childdoc} mechanism can also be use for the development
of definition files such as \LaTeX{} styles or classes.
This case differs from the above setup with multiple parts
included by |\include| in that no |\includeonly| should be invoked.
This can be achieved by starting the include file
(before |\ProvidesPackage|) with:
%
\begin{center}
\begin{tabular}{l}
|\input{childdoc.def}|\\
|\childdocforward{|\textit{main}|}|\\
\end{tabular}
\end{center}
%
or alternatively with:
%
\begin{center}
\begin{tabular}{l}
|\input{childdoc.def}|\\
|\childdocby{|\textit{main}|}|\\
\end{tabular}
\end{center}
%
Both forms have slightly different effects as described above.
The main file is prepared as usual, see \secref{sec:include}.

%%%%%%%%%%%%%%%%%%%%%%%%%%%%%%%%%%%%%%%%%%%%%%%%%%%%%%%%%%%%%%%%%%%%%%%%%%%%%%%%
\subsection{Legacy Detection}
\label{sec:detection}

The directive |\childdocmain| in the main file can detect
whether the complete document or merely a child is to be compiled
even without using the directive |\childdocof|.
This method is deprecated because it is less robust
and there is no compelling reason to use it;
it is merely provided for backward compatibility
and it may be removed in future versions.

If the detection mechanism is to be used,
it is mandatory to correctly specify
the filename of the main file as the argument of |\childdocmain|:
%
\begin{center}
\begin{tabular}{l}
|\input{childdoc.def}|\\
|\childdocmain{|\textit{main}|}|\\
\end{tabular}
\end{center}
%
If |\jobname| does not match the argument \textit{main} of |\childdocmain|,
it is assumed that |\jobname| points to the child file to be compiled.
When using |\childdocmain| with the main file specified as argument,
it suffices to start a child file
with just |\input{|\textit{main}|}|
without loading of the package and using |\childdocof|.
If instead all processing is done
with the appropriate \textsf{childdoc} directives,
the argument of \textit{main} of |\childdocmain| can be empty.

An alternative version of the command line processing described
in \secref{sec:commandline} using the detection mechanism reads:
%
\begin{center}
|... -jobname "|\textit{target}|" "|[\textit{flags}]%
[|\def\jobname{|\textit{dest}|}|]|\input{|\textit{main}|}"|
\end{center}

%%%%%%%%%%%%%%%%%%%%%%%%%%%%%%%%%%%%%%%%%%%%%%%%%%%%%%%%%%%%%%%%%%%%%%%%%%%%%%%%
\subsection{Manual Code}
\label{sec:manual}

In case one cannot be certain whether the definitions file |childdoc.def|
is installed on the target \TeX{} distribution
and one prefers not to ship it,
it is conceivable to paste a few relevant commands into the sources.

To that end, drop all statements |\input{childdoc.def}|
and perform the replacements as outlined below.
Instead of |\childdocmain{|\textit{main}|}| add the following code
to the top of the main file:
%
\begin{center}
\begin{tabular}{l}
|\||ifdefined\childdocname\endinput\||fi\newif\ifchilddoc|\\
|\edef\childdocname{\scantokens\expandafter{\jobname\noexpand}}|\\
|\def\childdocmain{|\textit{main}|}\||ifx\childdocmain\childdocname\||else|\\
|\childdoctrue\includeonly{\childdocname}\let\jobname\childdocmain\||fi|\\
\end{tabular}
\end{center}
%
Instead of |\childdocof{|\textit{main}|}| just include the main file
at the top of each child file:
%
\begin{center}
|\input{|\textit{main}|}|
\end{center}
%
A simple redirection |\childdocforward{|\textit{dest}|}| is achieved by:
%
\begin{center}
|\def\jobname{|\textit{dest}|}\input{\jobname}|
\end{center}
%
The redirection with prefix
|\childdocforwardprefix[|\textit{prefix}|]{|\textit{dest}|}|
is accomplished by:
%
\begin{center}
\begin{tabular}{l}
|{\edef\jobname{\scantokens\expandafter{\jobname\noexpand}}|\\
|\def\redirectjob |\textit{prefix}|#1~~~{\gdef\jobname{|\textit{dest}|#1}}|\\
|\expandafter\redirectjob\jobname~~~}\input{\jobname}|
\end{tabular}
\end{center}

In an alternative approach,
child documents can be compiled by a specific command line
without additional code or specific definitions:
%
\begin{center}
|... -jobname "|\textit{target}|" "|[\textit{flags}]%
|\includeonly{|\textit{dest}|}\input{|\textit{main}|}"|
\end{center}
%

%%%%%%%%%%%%%%%%%%%%%%%%%%%%%%%%%%%%%%%%%%%%%%%%%%%%%%%%%%%%%%%%%%%%%%%%%%%%%%%%
%%%%%%%%%%%%%%%%%%%%%%%%%%%%%%%%%%%%%%%%%%%%%%%%%%%%%%%%%%%%%%%%%%%%%%%%%%%%%%%%
\section{Information}

%%%%%%%%%%%%%%%%%%%%%%%%%%%%%%%%%%%%%%%%%%%%%%%%%%%%%%%%%%%%%%%%%%%%%%%%%%%%%%%%
\subsection{Copyright}

Copyright \copyright{} 2017--2018 Niklas Beisert

This work may be distributed and/or modified under the
conditions of the \LaTeX{} Project Public License, either version 1.3
of this license or (at your option) any later version.
The latest version of this license is in
  \url{http://www.latex-project.org/lppl.txt}
and version 1.3 or later is part of all distributions of \LaTeX{}
version 2005/12/01 or later.

This work has the LPPL maintenance status `maintained'.

The Current Maintainer of this work is Niklas Beisert.

This work consists of the files |README.txt|, |childdoc.ins| and |childdoc.dtx|
as well as the derived files |childdoc.def|, |cdocsamp.tex|
with |cdocsch1.tex|, |cdocsch2.tex|, |cdocspt3.tex|, |cdocspt4.tex|,
|cdocsdrf.tex|, |cdocsfn1.tex|, |cdocsfn2.tex|
as well as |childdoc.pdf|.

%%%%%%%%%%%%%%%%%%%%%%%%%%%%%%%%%%%%%%%%%%%%%%%%%%%%%%%%%%%%%%%%%%%%%%%%%%%%%%%%
\subsection{Files and Installation}

The package consists of the files:
%
\begin{center}
\begin{tabular}{ll}
    |README.txt|   & readme file \\
    |childdoc.ins| & installation file \\
    |childdoc.dtx| & source file \\
    |childdoc.def| & definition file \\
    |cdocsamp.tex| & sample main file \\
    |cdocsch1.tex| & sample include file \\
    |cdocsch2.tex| & sample include file \\
    |cdocspt3.tex| & sample part file \\
    |cdocspt4.tex| & sample part file \\
    |cdocsdrf.tex| & sample redirection file \\
    |cdocsfn1.tex| & sample redirection file \\
    |cdocsfn2.tex| & sample redirection file \\
    |childdoc.pdf| & manual
\end{tabular}
\end{center}
%
The distribution consists of the files
|README.txt|, |childdoc.ins| and |childdoc.dtx|.
%
\begin{itemize}
\item
Run (pdf)\LaTeX{} on |childdoc.dtx|
to compile the manual |childdoc.pdf| (this file).
\item
Run \LaTeX{} on |childdoc.ins| to create the definitions file |childdoc.def|
and the sample |cdocsamp.tex| with include files
|cdocsch1.tex|, |cdocsch2.tex|, |cdocspt3.tex|, |cdocspt4.tex|,
|cdocsdrf.tex|, |cdocsfn1.tex|, |cdocsfn2.tex|.
Then copy the file |childdoc.def| to an appropriate directory of your \LaTeX{}
distribution, e.g.\ \textit{texmf-root}|/tex/latex/childdoc|.
\end{itemize}

%%%%%%%%%%%%%%%%%%%%%%%%%%%%%%%%%%%%%%%%%%%%%%%%%%%%%%%%%%%%%%%%%%%%%%%%%%%%%%%%
\subsection{Related CTAN Packages}

There are several other packages which offer a similar functionality:
%
\begin{itemize}
\item
The packages
\href{http://ctan.org/pkg/docmute}{\textsf{docmute}},
\href{http://ctan.org/pkg/includex}{\textsf{includex}} and
\href{http://ctan.org/pkg/standalone}{\textsf{standalone}}
provide commands to include only the document body of
a child file thus allowing both files to be compiled individually.
\item
The packages \href{http://ctan.org/pkg/subdocs}{\textsf{subdocs}}
and \href{http://ctan.org/pkg/subfiles}{\textsf{subfiles}}
provide structures in which the main and child documents can be
encapsulated and allowing them to be compiled individually.
The inclusion mechanism is different from the conventional |\include|.
\item
The package \href{http://ctan.org/pkg/combine}{\textsf{combine}}
is an elaborate solution to combine several documents into one.
\end{itemize}
%
See also the CTAN topic \href{http://ctan.org/topic/subdocs}{\textsf{subdocs}}
for further related packages.
The present package differs from the above solutions in that
a document structure constructed with the conventional |\include| mechanism
just needs two extra commands at the top of every file
such that all constituent files can be compiled individually.

%%%%%%%%%%%%%%%%%%%%%%%%%%%%%%%%%%%%%%%%%%%%%%%%%%%%%%%%%%%%%%%%%%%%%%%%%%%%%%%%
%\subsection{Feature Suggestions}
%
%The following is a list of features which may be useful for future
%versions of this package:
%%
%\begin{itemize}
%\item
%\ldots
%\end{itemize}

%%%%%%%%%%%%%%%%%%%%%%%%%%%%%%%%%%%%%%%%%%%%%%%%%%%%%%%%%%%%%%%%%%%%%%%%%%%%%%%%
\subsection{Revision History}

%%%%%%%%%%%%%%%%%%%%%%%%%%%%%%%%%%%%%%%%
\paragraph{v2.0:} 2018/12/30

\begin{itemize}
\item
immediate forward processing
\item
added |\childdocby| mechanism
\item
manual restructured
\end{itemize}

%%%%%%%%%%%%%%%%%%%%%%%%%%%%%%%%%%%%%%%%
\paragraph{v1.6:} 2018/01/17

\begin{itemize}
\item
application for development of include files
\item
corrections to manual
\end{itemize}

%%%%%%%%%%%%%%%%%%%%%%%%%%%%%%%%%%%%%%%%
\paragraph{v1.5:} 2017/05/21

\begin{itemize}
\item
more complete structuring introduced
\item
|\childdocof| introduced
\item
|\childdoc| renamed to |\childdocmain|
\item
|\childredirect| renamed to |\childdocforward| and |\childdocforwardprefix|
and functionality expanded
\end{itemize}

%%%%%%%%%%%%%%%%%%%%%%%%%%%%%%%%%%%%%%%%
\paragraph{v1.0:} 2017/04/27

\begin{itemize}
\item
manual and install package
\item
first version published on CTAN
\end{itemize}

%%%%%%%%%%%%%%%%%%%%%%%%%%%%%%%%%%%%%%%%
\paragraph{v0.6:} 2017/04/26

\begin{itemize}
\item
redirection mechanism added
\end{itemize}

%%%%%%%%%%%%%%%%%%%%%%%%%%%%%%%%%%%%%%%%
\paragraph{v0.5:} 2017/04/26

\begin{itemize}
\item
functionality in definition file
\end{itemize}


%%%%%%%%%%%%%%%%%%%%%%%%%%%%%%%%%%%%%%%%%%%%%%%%%%%%%%%%%%%%%%%%%%%%%%%%%%%%%%%%
%%%%%%%%%%%%%%%%%%%%%%%%%%%%%%%%%%%%%%%%%%%%%%%%%%%%%%%%%%%%%%%%%%%%%%%%%%%%%%%%
%%%%%%%%%%%%%%%%%%%%%%%%%%%%%%%%%%%%%%%%%%%%%%%%%%%%%%%%%%%%%%%%%%%%%%%%%%%%%%%%
\appendix

\settowidth\MacroIndent{\rmfamily\scriptsize 000\ }

 \DocInput{childdoc.dtx}

\end{document}
%</driver>
% \fi
%
% %%%%%%%%%%%%%%%%%%%%%%%%%%%%%%%%%%%%%%%%%%%%%%%%%%%%%%%%%%%%%%%%%%%%%%%%%%%%%%
% %%%%%%%%%%%%%%%%%%%%%%%%%%%%%%%%%%%%%%%%%%%%%%%%%%%%%%%%%%%%%%%%%%%%%%%%%%%%%%
% \section{Sample}
%\iffalse
%<*samplemain>
%\fi
%
% The following presents a sample document
% with two chapters, two parts, a title page,
% a compile flag as well as three forwarding files to set the flag.
% It consists of eight |.tex| files:
% \begin{center}
% \begin{tabular}{ll}
% |cdocsamp.tex|&main file\\
% |cdocsch1.tex|&include file for chapter 1\\
% |cdocsch2.tex|&include file for chapter 2\\
% |cdocspt3.tex|&include file for part 3\\
% |cdocspt4.tex|&include file for part 4\\
% |cdocsdrf.tex|&forwarding file for main file in draft mode\\
% |cdocsfi1.tex|&forwarding file for final version of chapter 1\\
% |cdocsfi2.tex|&forwarding file for final version of chapter 2\\
% \end{tabular}
% \end{center}
% Each of the eight files can be compiled directly by the \LaTeX{} compiler.
%
% %%%%%%%%%%%%%%%%%%%%%%%%%%%%%%%%%%%%%%
% \paragraph{Main File.}
%
% The main file is called |cdocsamp.tex|.
%
% Load the \textsf{childdoc} definitions and
% declare the filename for the main document:
%    \begin{macrocode}
\input{childdoc.def}
\childdocmain{}
%    \end{macrocode}

% Optional override for |\version| flag:
%    \begin{macrocode}
%%\ifchilddoc\else\providecommand{\version}{draft}\fi
%    \end{macrocode}

% Define the default values for the |\version| flag
% (|final| for the main file and |draft| for childs):
%    \begin{macrocode}
\ifchilddoc
\providecommand{\version}{draft}
\else
\providecommand{\version}{final}
\fi
%    \end{macrocode}

% Load the standard document class:
%    \begin{macrocode}
\documentclass[12pt]{article}
%    \end{macrocode}

% Start the document body:
%    \begin{macrocode}
\begin{document}
%    \end{macrocode}

% Declare a title page.
% Print title, part of document being processed and version flag:
%    \begin{macrocode}
\addtocounter{page}{-1}
\begin{center}
{\LARGE\bfseries{}childdoc example\par}
\vspace{1cm}
\ifchilddoc
\ifchilddocmanual part\else chapter\fi:
`\childdocname' of `\childdocjob'\par
\else
main document: `\childdocjob'\par
\fi
version: \version\par
\end{center}
\newpage
%    \end{macrocode}

% Manually include selected file,
% otherwise process as usual:
%    \begin{macrocode}
\ifchilddocmanual
\section*{part `\childdocname'}
\input{\childdocname}
\else
%    \end{macrocode}

% Include the two chapters:
%    \begin{macrocode}
\include{cdocsch1}
\include{cdocsch2}
%    \end{macrocode}

% Include the two parts unless only chapters should be displayed:
%    \begin{macrocode}
\ifchilddoc\else
\section{part three}
\input{cdocspt3}
\section{part four}
\input{cdocspt4}
\fi
%    \end{macrocode}

% Process as usual until here:
%    \begin{macrocode}
\fi
%    \end{macrocode}

% End of document body:
%    \begin{macrocode}
\end{document}
%    \end{macrocode}
%\iffalse
%</samplemain>
%\fi
%
% %%%%%%%%%%%%%%%%%%%%%%%%%%%%%%%%%%%%%%
% \paragraph{Chapter Include Files.}
%
% The include files are called |cdocsch1.tex| and |cdocsch2.tex|.
%
%\iffalse
%<*samplechap1|samplechap2>
%\fi

% Optional override for |\version| flag:
%    \begin{macrocode}
%%\providecommand{\version}{final}
%    \end{macrocode}

% Include the main document:
%    \begin{macrocode}
\input{childdoc.def}
\childdocof{cdocsamp}
%    \end{macrocode}

%\iffalse
%</samplechap1|samplechap2>
%\fi
%
%\iffalse
%<*samplechap1>
%\fi
% Some text for chapter 1:
%    \begin{macrocode}
\section{one}
some text in chapter one
%    \end{macrocode}

%\iffalse
%</samplechap1>
%\fi
% Some text for chapter 2:
%\iffalse
%<*samplechap2>
%\fi
%    \begin{macrocode}
\section{two}
more text in chapter two
%    \end{macrocode}

%\iffalse
%</samplechap2>
%\fi
%
% %%%%%%%%%%%%%%%%%%%%%%%%%%%%%%%%%%%%%%
% \paragraph{Part Include Files.}
%
% The include files are called |cdocspt3.tex| and |cdocspt4.tex|.
%
%\iffalse
%<*samplepart3|samplepart4>
%\fi

% Optional override for |\version| flag:
%    \begin{macrocode}
%%\providecommand{\version}{final}
%    \end{macrocode}

% Include the main document:
%    \begin{macrocode}
\input{childdoc.def}
\childdocby{cdocsamp}
%    \end{macrocode}

%\iffalse
%</samplepart3|samplepart4>
%\fi
%
%\iffalse
%<*samplepart3>
%\fi
% Some text for part 3:
%    \begin{macrocode}
some text in part three
%    \end{macrocode}

%\iffalse
%</samplepart3>
%\fi
% Some text for part 4:
%\iffalse
%<*samplepart4>
%\fi
%    \begin{macrocode}
more text in part four
%    \end{macrocode}

%\iffalse
%</samplepart4>
%\fi
%
% %%%%%%%%%%%%%%%%%%%%%%%%%%%%%%%%%%%%%%
% \paragraph{Forwarding for a Complete Draft.}
%
% The following forwarding file |cdocsdrf.tex|
% compiles the main document in draft mode:
%\iffalse
%<*sampledraft>
%\fi
%    \begin{macrocode}
\def\version{draft}
\input{childdoc.def}
\childdocforward{cdocsamp}
%    \end{macrocode}

%\iffalse
%</sampledraft>
%\fi
%
% %%%%%%%%%%%%%%%%%%%%%%%%%%%%%%%%%%%%%%
% \paragraph{Forwarding for Final Version of the Chapters.}
%
% The following forwarding files |cdocsfn1.tex| and |cdocsfn2.tex|
% (with identical content)
% compile the final versions of the child documents
% |cdocsch1.tex| and |cdocsch2.tex|, respectively:
%\iffalse
%<*samplefinal>
%\fi
%    \begin{macrocode}
\def\version{final}
\input{childdoc.def}
\childdocforwardprefix[cdocsamp]{cdocsfn}{cdocsch}
%    \end{macrocode}

%\iffalse
%</samplefinal>
%\fi
%
% %%%%%%%%%%%%%%%%%%%%%%%%%%%%%%%%%%%%%%
% \paragraph{Command Line Processing.}
%
% The following three command lines generate the output files
% |cdocscld|, |cdocscl1| and |cdocscl2|
% which should be identical to
% |cdocsdrf|, |cdocsch1| and |cdocsfn2|, respectively:
% \begin{center}
% \begin{tabular}{l}
% |latex -jobname cdocscld \|\\
% |  "\def\version{draft}\input{childdoc.def}\childdocforward{cdocsamp}"|\\
% |latex -jobname cdocscl1 \|\\
% |  "\input{childdoc.def}\childdocforward[cdocsamp]{cdocsch1}"|\\
% |latex -jobname cdocscl2 \|\\
% |  "\def\version{final}\input{childdoc.def}\childdocforward{cdocsch2}"|
% \end{tabular}
% \end{center}
% Note that the trailing backslash on each first line
% merely continues the input to the second line
% (for convenient cut ant paste).
% Furthermore, the command |latex| can be replaced by any
% of its alternative versions such as |pdflatex|.
%
% %%%%%%%%%%%%%%%%%%%%%%%%%%%%%%%%%%%%%%%%%%%%%%%%%%%%%%%%%%%%%%%%%%%%%%%%%%%%%%
% %%%%%%%%%%%%%%%%%%%%%%%%%%%%%%%%%%%%%%%%%%%%%%%%%%%%%%%%%%%%%%%%%%%%%%%%%%%%%%
% \section{Implementation}
%\iffalse
%<*package>
%\fi
%
% This section describes the definitions file |childdoc.def|.

% The definitions cannot be loaded using |\usepackage| or |\RequirePackage|
% which has a mechanism to prevent loading a style file more than once.
% When loading the definitions by means of |\input|
% multiple instances have to be prevented manually:
%\iffalse
%This code needs to be before the `\ProvidesFile' directive
%which is defined at the beginning of this file.
%Therefore it is also placed there and commented out here.
%</package>
%<*discard>
%\fi
%    \begin{macrocode}
\ifdefined\childdocmain\endinput\fi
%    \end{macrocode}
%\iffalse
%</discard>
%<*package>
%\fi
%
% \macro{\ifchilddoc}
% \macro{\ifchilddocmanual}
% The conditional |\ifchilddoc| tells whether a
% child (true) or main (false) document is being compiled.
% The conditional |\ifchilddocmanual| tells whether
% the |\includeonly| mechanism is used (false) or
% the selection of child files must be performed manually (true).
% The definitions initialise to false:
%    \begin{macrocode}
\newif\ifchilddoc
\newif\ifchilddocmanual
%    \end{macrocode}

% \macro{\childdocname}
% \macro{\childdocjob}
% The macro |\childdocname| stores the name of the main document
% to be compiled. The macro |\childdocjob| stores the name of
% the document on which the \LaTeX{} compiler was originally invoked.
% The content of |\jobname| cannot be compared
% to filenames specified in the source due to different catcodes.
% The following code rescans |\jobname|, stores the result
% in |\childdocname| and saves a copy in |\childdocjob|:
%    \begin{macrocode}
\edef\childdocname{\scantokens\expandafter{\jobname\noexpand}}
\let\childdocjob\childdocname
%    \end{macrocode}

% \macro{\childdocdisable}
% The macro |\childdocdisable| prevents the main file
% from being processed more than once.
% At this stage, the main document command |\childdocmain|
% is assumed to be called once again where it should do nothing.
% Any subsequent call to it should prevent
% a secondary processing of the main document
% It overwrites the forwarding commands
% |\childdocof| and |\childdocforward|
% with empty macros to prevent further inclusions of the main document:
%    \begin{macrocode}
\newcommand{\childdocdisable}
{
  \renewcommand{\childdocmain}[1]{\renewcommand{\childdocmain}[1]{\endinput}}
  \renewcommand{\childdocof}[1]{}
  \renewcommand{\childdocby}[2][]{}
  \renewcommand{\childdocforward}[2][]{}
  \renewcommand{\childdocdisable}{}
}
%    \end{macrocode}

% \macro{\childdocmain}
% The macro |\childdocmain| is to be called at the top of the main file
% with nothing or the main filename (without extension) as argument.
% First, it breaks loops.
% If the argument is not empty and does not match |\childdocname|
% (which is set by the first inclusion of |childdoc.def|),
% |\ifchilddoc| is set to true, |\includeonly| is applied to the child file
% and |\jobname| is set to the main file
% (for proper handling of |.aux| files):
%    \begin{macrocode}
\newcommand{\childdocmain}[1]
{
  \childdocdisable\childdocmain{}
  \if?#1?\else
    \begingroup
      \def\childdoctmp{#1}
      \ifx\childdoctmp\childdocname
        \def\childdoctmp{}
      \else
        \def\childdoctmp
        {
          \childdoctrue
          \includeonly{\childdocname}
          \def\childdocjob{#1}
          \def\jobname{#1}
        }
      \fi
      \expandafter
    \endgroup
    \childdoctmp
  \fi
}
%    \end{macrocode}

% \macro{\childdocof}
% The command |\childdocof| redirects
% compilation to the main file |#1|.
%    \begin{macrocode}
\newcommand{\childdocof}[1]
{
  \childdocdisable
  \childdoctrue
  \includeonly{\childdocname}
  \def\jobname{#1}
  \def\childdocjob{#1}
  \input{#1}
}
%    \end{macrocode}

% \macro{\childdocby}
% The command |\childdocby| ....
%    \begin{macrocode}
\newcommand{\childdocby}[2][]
{
  \childdocdisable
  \childdoctrue
  \childdocmanualtrue
  \if?#1?\else
    \def\jobname{#2}
  \fi
  \def\childdocjob{#2}
  \input{#2}
  \endinput
}
%    \end{macrocode}

% \macro{\childdocforward}
% The command |\childdocforward| redirects
% compilation to the main file or
% (if the optional argument is given) a child file.
% Parameters are set as if the main file
% or a child file starting with |\childdocof| was compiled.
% Then compilation is handed over to the main file:
%    \begin{macrocode}
\newcommand{\childdocforward}[2][]
{
  \begingroup
    \if?#1?
      \def\childdoctmp
      {
        \def\childdocname{#2}
        \def\childdocjob{#2}
        \def\jobname{#2}
        \input{#2}
        \endinput
      }
    \else
      \def\childdoctmp
      {
        \childdocdisable
        \def\childdocname{#2}
        \childdoctrue
        \includeonly{#2}
        \def\childdocjob{#1}
        \def\jobname{#1}
        \input{#1}
        \endinput
      }
    \fi
    \expandafter
  \endgroup
  \childdoctmp
}
%    \end{macrocode}

% \macro{\childdocforwardprefix}
% The command |\childdocforwardprefix| redirects
% compilation to the main or a child file by means of a pattern.
% The prefix |#1| in the current filename is replaced by |#2|
% and the suffix of the current filename is kept
% (it is assumed that the filename does not contain the substring `|~~~|'
% which is used as a delimiter).
% Compilation is handed over to the new file by |\childdocforward|:
%    \begin{macrocode}
\newcommand{\childdocforwardprefix}[3][]
{
  \begingroup
    \def\childdocextract #2##1~~~{\def\childdoctmp{\childdocforward[#1]{#3##1}}}
    \expandafter\childdocextract\childdocname~~~
    \expandafter
  \endgroup
  \childdoctmp
}
%    \end{macrocode}

% \macro{\childdoc}
% The deprecated macro |\childdoc| is a legacy version of |\childdocmain|:
%    \begin{macrocode}
\newcommand{\childdoc}{\childdocmain}
%    \end{macrocode}

% \macro{\childdocredirect}
% The deprecated macro |\childdocredirect| is a legacy version
% of |\childdocforward| and |\childdocforwardprefix|:
%    \begin{macrocode}
\newcommand{\childdocredirect}[2][]
{
  \begingroup
    \if?#1?
      \def\childdoctmp{\childdocforward{#2}}
    \else
      \def\childdoctmp{\childdocforwardprefix{#1}{#2}}
    \fi
    \expandafter
  \endgroup
  \childdoctmp
}
%    \end{macrocode}

%\iffalse
%</package>
%\fi
%
\endinput

\childdocof{cdocsamp}
%    \end{macrocode}

%\iffalse
%</samplechap1|samplechap2>
%\fi
%
%\iffalse
%<*samplechap1>
%\fi
% Some text for chapter 1:
%    \begin{macrocode}
\section{one}
some text in chapter one
%    \end{macrocode}

%\iffalse
%</samplechap1>
%\fi
% Some text for chapter 2:
%\iffalse
%<*samplechap2>
%\fi
%    \begin{macrocode}
\section{two}
more text in chapter two
%    \end{macrocode}

%\iffalse
%</samplechap2>
%\fi
%
% %%%%%%%%%%%%%%%%%%%%%%%%%%%%%%%%%%%%%%
% \paragraph{Part Include Files.}
%
% The include files are called |cdocspt3.tex| and |cdocspt4.tex|.
%
%\iffalse
%<*samplepart3|samplepart4>
%\fi

% Optional override for |\version| flag:
%    \begin{macrocode}
%%\providecommand{\version}{final}
%    \end{macrocode}

% Include the main document:
%    \begin{macrocode}
% \iffalse
%
% childdoc.dtx Copyright (C) 2017-2018 Niklas Beisert
%
% This work may be distributed and/or modified under the
% conditions of the LaTeX Project Public License, either version 1.3
% of this license or (at your option) any later version.
% The latest version of this license is in
%   http://www.latex-project.org/lppl.txt
% and version 1.3 or later is part of all distributions of LaTeX
% version 2005/12/01 or later.
%
% This work has the LPPL maintenance status `maintained'.
%
% The Current Maintainer of this work is Niklas Beisert.
%
% This work consists of the files childdoc.dtx and childdoc.ins
% and the derived files childdoc.def and cdocsamp.tex with
% cdocsch1.tex, cdocsch2.tex, cdocsdrf.tex, cdocsfn1.tex, cdocsfn2.tex.
%
%<package>\ifdefined\childdocmain\endinput\fi
%<package>\ProvidesFile{childdoc.def}[2018/12/30 v2.0 child document driver]
%<samplemain>\ProvidesFile{cdocsamp.tex}[2018/12/30 v2.0 sample for childdoc]
%<*driver>
%\ProvidesFile{childdoc.drv}[2018/12/30 v2.0 childdoc reference manual file]
\PassOptionsToClass{10pt,a4paper}{article}
\documentclass{ltxdoc}

\usepackage[margin=35mm]{geometry}
\usepackage{hyperref}
\usepackage{hyperxmp}
\usepackage[usenames]{color}

\hypersetup{colorlinks=true}
\hypersetup{pdfstartview=FitH}
\hypersetup{pdfpagemode=UseNone}
\hypersetup{pdfsource={}}
\hypersetup{pdflang={en-UK}}
\hypersetup{pdfcopyright={Copyright 2017-2018 Niklas Beisert.
  This work may be distributed and/or modified under the
  conditions of the LaTeX Project Public License, either version 1.3
  of this license or (at your option) any later version.}}
\hypersetup{pdflicenseurl={http://www.latex-project.org/lppl.txt}}
\hypersetup{pdfcontactaddress={ETH Zurich, ITP, HIT K,
  Wolfgang-Pauli-Strasse 27}}
\hypersetup{pdfcontactpostcode={8093}}
\hypersetup{pdfcontactcity={Zurich}}
\hypersetup{pdfcontactcountry={Switzerland}}
\hypersetup{pdfcontactemail={nbeisert@itp.phys.ethz.ch}}
\hypersetup{pdfcontacturl={http://people.phys.ethz.ch/\xmptilde nbeisert/}}

\newcommand{\secref}[1]{\hyperref[#1]{section \ref*{#1}}}

\parskip1ex
\parindent0pt
\let\olditemize\itemize
\def\itemize{\olditemize\parskip0pt}

\begin{document}

\title{The \textsf{childdoc} Package}
\hypersetup{pdftitle={The childdoc Package}}
\author{Niklas Beisert\\[2ex]
  Institut f\"ur Theoretische Physik\\
  Eidgen\"ossische Technische Hochschule Z\"urich\\
  Wolfgang-Pauli-Strasse 27, 8093 Z\"urich, Switzerland\\[1ex]
  \href{mailto:nbeisert@itp.phys.ethz.ch}
  {\texttt{nbeisert@itp.phys.ethz.ch}}}
\hypersetup{pdfauthor={Niklas Beisert}}
\hypersetup{pdfsubject={Manual for the LaTeX2e Package childdoc}}
\date{30 December 2018, \textsf{v2.0}}
\maketitle

\begin{abstract}\noindent
\textsf{childdoc} is a \LaTeXe{} package
that enables the direct compilation
of document sections included by |\include|
to individual files.
\end{abstract}

\begingroup
\parskip0ex
\tableofcontents
\endgroup

%%%%%%%%%%%%%%%%%%%%%%%%%%%%%%%%%%%%%%%%%%%%%%%%%%%%%%%%%%%%%%%%%%%%%%%%%%%%%%%%
%%%%%%%%%%%%%%%%%%%%%%%%%%%%%%%%%%%%%%%%%%%%%%%%%%%%%%%%%%%%%%%%%%%%%%%%%%%%%%%%
\section{Introduction}

\LaTeX{} provides a mechanism to structure a large document (such as a book)
into a main file and several child files (containing the chapters)
using the |\include| command.
This mechanism is beneficial for documents
which span hundreds of pages in order to
make the source file(s) more manageable.
Moreover, compilation can be restricted to
selected child files by means of the |\includeonly| command.
The latter feature can be used to reduce the compilation time while editing
(this was significantly more useful in the earlier days of \LaTeX{})
or to generate a smaller document which is easier to navigate.
Another application of |\includeonly| is to generate
documents consisting of selected parts of the complete document.

However, there are a few drawbacks of the plain |\include| mechanism:
\begin{itemize}
\item
The child files cannot be compiled on their own,
they can only be compiled via the main file.
A naive editing environment
(such as a text editor with an option
to have the current file processed by \LaTeX)
may require one to switch to the main file before compiling;
attempting to compile the child file produces errors.
\item
The main file must be modified (each time)
to adjust the |\includeonly| command
to the present needs. This easily leaves the main file in a messy state.
\item
The generated document will always carry the filename
of the main document. This is inconvenient if
several child files are to be compiled and
to be kept for distribution.
\end{itemize}

The present package provides a simple interface
to make child files individually compilable by \LaTeX{}.
Compiling a child file then has the same effect as compiling
the main file with an |\includeonly| command
to select the appropriate child.
Moreover the generated document will carry the name of the child
rather than the main file.
This resolves all three above issues.

This feature is meant to make the editing of books,
thesis documents and lecture notes somewhat more convenient.
However, the package can also be used efficiently for
composing a series of documents (such as exercise sheets)
which are typically distributed individually.
It then assists the author in generating the individual documents
(potentially in different versions)
as well as a document containing the collected series.
Another application is in developing style files
or other kinds of included material
where compilation of the style file could redirect
to a sample or test file.

%%%%%%%%%%%%%%%%%%%%%%%%%%%%%%%%%%%%%%%%%%%%%%%%%%%%%%%%%%%%%%%%%%%%%%%%%%%%%%%%
%%%%%%%%%%%%%%%%%%%%%%%%%%%%%%%%%%%%%%%%%%%%%%%%%%%%%%%%%%%%%%%%%%%%%%%%%%%%%%%%
\section{Usage}

First of all, the package \textsf{childdoc} is \emph{not} a standard
\LaTeXe{} |.sty| style file! Therefore it needs to be invoked in
a non-standard way.

%%%%%%%%%%%%%%%%%%%%%%%%%%%%%%%%%%%%%%%%%%%%%%%%%%%%%%%%%%%%%%%%%%%%%%%%%%%%%%%%
\subsection{Included Files}
\label{sec:include}

%%%%%%%%%%%%%%%%%%%%%%%%%%%%%%%%%%%%%%%%
\DescribeMacro{\childdocmain}
To use the package, add the commands
\begin{center}
\begin{tabular}{l}
|\input{childdoc.def}|\\
|\childdocmain{}|\\
\end{tabular}
\end{center}
at the very top of the main \LaTeX{} file,
in particular \emph{before} the |\documentclass| statement!
The argument of |\childdocmain| should be left empty
(but it must be present).

%%%%%%%%%%%%%%%%%%%%%%%%%%%%%%%%%%%%%%%%
\DescribeMacro{\childdocof}
Furthermore, add the commands
\begin{center}
\begin{tabular}{l}
|\input{childdoc.def}|\\
|\childdocof{|\textit{main}|}|\\
\end{tabular}
\end{center}
at the top of every child file \textit{child}
which is included by |\include{|\textit{child}|}|
from within the main file
(or at least for those files to be compiled individually).
The argument \textit{main} must be the filename of the main file.

There are a couple of
considerations in setting up the main and child documents:

%%%%%%%%%%%%%%%%%%%%%%%%%%%%%%%%%%%%%%%%
\paragraph{Restrictions.}

Please note the following restrictions:
\begin{itemize}
\item
|\childdocmain| must be called with one argument \textit{main}
to ensure compatibility with earlier version of the package.
It must either be empty (|\childdocmain{}|)
or precisely match the filename of the main file in which it is specified.
See \secref{sec:detection} for further information.
\item
The filename \textit{main} must be specified without the |.tex| extension.
\item
The filename \textit{main} is case sensitive
(even in case-insensitive file systems)
due to internal string comparison.
\item
The argument \textit{main} should be fully expanded, it cannot be a macro.
\item
Subdirectories and special characters should be avoided in filenames.
\item
The command |\childdocmain{|\textit{main}|}| must be followed by a whitespace.
It should not be followed immediately by another command
or by a comment mark `|%|'.
This is because the \TeX{} parser reads the token immediately following
the argument of |\childdocmain| and puts it
at the beginning of every child section;
however, a white\-space is ignored.
\end{itemize}

%%%%%%%%%%%%%%%%%%%%%%%%%%%%%%%%%%%%%%%%
\paragraph{Content of Main File.}

It is advisable to place all content in the child files included by |\include|.
Any output contained in the main file will appear in all child documents
unless suppressed manually;
it cannot be suppressed automatically by the |\includeonly| directive
and thus should normally be avoided.
A method to include some content in the main file
by means of conditional processing is described in \secref{sec:conditional}.

%%%%%%%%%%%%%%%%%%%%%%%%%%%%%%%%%%%%%%%%
\paragraph{Page Numbering.}

When only a part of the document is compiled,
the appropriate numbering of pages
(as well as other status parameters)
is determined from the |.aux| files.
The latter contain information from previous passes.
However this information needs to propagate through
all intermediate child documents.
Therefore the page numbering in child documents may well
be inconsistent until the complete document is compiled at least once.

A useful (if unconventional) way to always ensure a consistent
page numbering is to restart the numbering in each child document
and denote the pages by `\textit{child}|.|\textit{page}'
where \textit{child} represents the chapter/section number of the child file.
This can be achieved by the command
|\numberwithin{page}{|\textit{child}|}|
of the \textsf{amsmath} package
where \textit{child} can be |chapter| or |section|
depending on the chosen structuring.
Alternatively, one can modify the macro |\thepage| appropriately
and reset the counter |page| at the start of each child file.

%%%%%%%%%%%%%%%%%%%%%%%%%%%%%%%%%%%%%%%%%%%%%%%%%%%%%%%%%%%%%%%%%%%%%%%%%%%%%%%%
\subsection{Conditional Processing}
\label{sec:conditional}

The package provides a mechanism to compile different versions
of a document. To customise the versions further some conditional processing
can come in handy to distinguish which version is being compiled.
The package provides two macros to describe the compilation context:

%%%%%%%%%%%%%%%%%%%%%%%%%%%%%%%%%%%%%%%%
\DescribeMacro{\ifchilddoc}
The conditional |\ifchilddoc| distinguishes between the compilation of
child documents and the main document:
%
\begin{center}
|\ifchilddoc |\textit{child-code}| |[|\||else |\textit{main-code}]| \||fi|
\end{center}

%%%%%%%%%%%%%%%%%%%%%%%%%%%%%%%%%%%%%%%%
\DescribeMacro{\childdocname}
\DescribeMacro{\childdocjob}
The macro |\childdocname| contains the filename (without extension)
of the main or child file being processed.
Note that |\childdocjob| will always contain the name of the main file.

%%%%%%%%%%%%%%%%%%%%%%%%%%%%%%%%%%%%%%%%
\paragraph{Title Page.}

Conditional processing can be used to include a title or banner page
in the main document when proper precautions are taken.
Importantly, the code in the main file should ensure that the page counter
(as well as other status parameters which are stored in the |.aux| files)
takes the same value after the conditional processing.
Otherwise the page numbers may take divergent values
depending on which part is compiled.

For example, a title page could be declared by:
%
\begin{center}
\begin{tabular}{l}
|\ifchilddoc\||else|\\
|\addtocounter{page}{-1}|\\
\textit{code for title page}\\
|\newpage|\\
|\||fi|
\end{tabular}
\end{center}
%
A banner page for the child documents can be generated by:
%
\begin{center}
\begin{tabular}{l}
|\ifchilddoc|\\
|\addtocounter{page}{-1}|\\
\textit{code for banner page}\\
|\newpage|\\
|\||fi|
\end{tabular}
\end{center}
%
Here one could write a message such as:
\begin{center}
|This is the part \childdocname{} of \childdocjob{}.|
\end{center}

%%%%%%%%%%%%%%%%%%%%%%%%%%%%%%%%%%%%%%%%%%%%%%%%%%%%%%%%%%%%%%%%%%%%%%%%%%%%%%%%
\subsection{Flags}
\label{sec:flags}

The package makes it easy to generate different versions
of the main or child documents.
To this end compilation flags can be defined
and assigned different default values.
They will be particularly useful in conjunction
with the forwarding mechanism described in \secref{sec:forward}.

For example, it may be useful to have a flag |\version|
which can be set to |draft| or |final|.
The document source will contain some conditional code
depending on the value of |\version|.
Suppose further, the flag should default to |final| for the main file
and to |draft| for child files
which is a natural assignment for editing the document.
This is achieved by placing the following code
in the preamble of the main document
(below the |\childdocmain| directive):
%
\begin{center}
\begin{tabular}{l}
|\ifchilddoc|\\
|\providecommand{\version}{draft}|\\
|\||else|\\
|\providecommand{\version}{final}|\\
|\||fi|
\end{tabular}
\end{center}
%
The definition by |\providecommand| makes sure
that previous definitions are not overwritten.
Further statements |\providecommand{\version}{...}|
can thus be added before the above code to override it.

For the main file, one might add a line
(between |\childdocmain| and the above block)
%
\begin{center}
|%\ifchilddoc\||else\providecommand{\version}{draft}\||fi|
\end{center}
%
which can be uncommented to produce a draft version.
Likewise one can add a line to the very top of a child file
(above the |\childdocof{|\textit{main}|}| directive)
%
\begin{center}
|%\providecommand{\version}{final}|
\end{center}
%
which can be uncommented to produce the final version of this child document.

%%%%%%%%%%%%%%%%%%%%%%%%%%%%%%%%%%%%%%%%%%%%%%%%%%%%%%%%%%%%%%%%%%%%%%%%%%%%%%%%
\subsection{Forwarding}
\label{sec:forward}

Different versions of the main or child documents
using compilation flags as described in \secref{sec:flags}
can be (permanently) stored in different files
for convenient compilation, viewing and distribution.
To this end, the package defines a command
to pass on compilation to a different file:

%%%%%%%%%%%%%%%%%%%%%%%%%%%%%%%%%%%%%%%%
\DescribeMacro{\childdocforward}
The command |\childdocforward| redirects processing to
another source file:
%
\begin{center}
\begin{tabular}{l}
|\input{childdoc.def}|\\
|\childdocforward[|\textit{main}|]{|\textit{dest}|}|\\
\end{tabular}
\end{center}
%
The argument \textit{dest} is the destination file
(without extension).
It should be the main file or one of the child files.
Note that further \textsf{childdoc} directives
such as |\childdocof| and |\childdocforward|
in the indicated file will be processed in this form.
The optional argument \textit{main}
passes on directly to the main file \textit{main}
while pretending to compile the child \textit{dest}.
This form behaves as if \textit{dest}
issues |\childdocof{|\textit{main}|}| right away,
and no further \textsf{childdoc} directives will be processed.

%%%%%%%%%%%%%%%%%%%%%%%%%%%%%%%%%%%%%%%%
\DescribeMacro{\...prefix}
In the alternative form |\childdocforwardprefix|,
%
\begin{center}
\begin{tabular}{l}
|\input{childdoc.def}|\\
|\childdocforwardprefix[|\textit{main}|]{|\textit{prefix}|}{|\textit{dest}|}|
\end{tabular}
\end{center}
%
the destination file is determined by a pattern
depending on the current file:
To make this work, the current file must be called
`{\textit{prefix}\hspace{0.2em}\textit{suffix}}'
with \textit{prefix} matching precisely the argument.
Processing is then passed on to the file
`{\textit{dest}\hspace{0.2em}\textit{suffix}}'.
Surely, the same effect is achieved by
directly specifying the
argument `{\textit{dest}\hspace{0.2em}\textit{suffix}}'
in the first form.
However, that requires to set up a different file
for each child. With the alternative form of the command
all these files can have exactly the same content
which simplifies setting them up and maintaining them.

For example, the following file |draft.tex|
with a compilation flag |\version| as described in \secref{sec:flags}
compiles the main document as a draft:
%
\begin{center}
\begin{tabular}{l}
|\def\version{draft}|\\
|\input{childdoc.def}|\\
|\childdocforward{|\textit{main}|}|
\end{tabular}
\end{center}
%
Likewise, the following files |final|\textit{nn}|.tex|
compile the final version of the child document
|child|\textit{nn}|.tex|:
%
\begin{center}
\begin{tabular}{l}
|\def\version{final}|\\
|\input{childdoc.def}|\\
|\childdocforwardprefix{final}{child}|
\end{tabular}
\end{center}
%

Note that when several versions of a main file and/or of each child file
are to be generated, it may be convenient to set up a |Makefile| or
shell script to automatise the process.

%%%%%%%%%%%%%%%%%%%%%%%%%%%%%%%%%%%%%%%%%%%%%%%%%%%%%%%%%%%%%%%%%%%%%%%%%%%%%%%%
\subsection{Command Line Processing}
\label{sec:commandline}

The effect of redirection files can also be achieved by invoking
the \LaTeX{} compiler with a more elaborate command line.
Most conveniently this should be done as part
of a shell script or a |Makefile|.

When using \textsf{childdoc} in the main file, the following
command lines effectively perform a redirection
(note that depending on the shell being used,
backslashes may have to be doubled: `|\|' $\to$ `|\\|'):
%
\begin{center}
|... -jobname "|\textit{target}|" |\\|"|[\textit{flags}]%
|\input{childdoc.def}\childdocforward[|\textit{main}|]{|\textit{dest}|}"|
\end{center}
%
Here \textit{target} is the name of the output file,
\textit{main} is the name of the main file
and \textit{dest} is the name of the main or child file to be processed
(all filenames without extensions).
The optional argument \textit{main} can be omitted
if \textit{main} matches \textit{dest}.
Optionally, compilation \textit{flags} can be defined via |\def| commands.
This command line makes the \TeX{} engine believe
it is compiling the file \textit{target}
whose content is specified as the latter parameter.
The provided code then forwards the processing to
\textit{main} or \textit{dest} as described in \secref{sec:forward}.

%%%%%%%%%%%%%%%%%%%%%%%%%%%%%%%%%%%%%%%%%%%%%%%%%%%%%%%%%%%%%%%%%%%%%%%%%%%%%%%%
\subsection{Include by Input}
\label{sec:input}

Including child documents by |\include| has some restrictions by design.
Most notably, the content of a child document always occupies
its own set of pages; pages cannot be shared between child documents.
Usually, this behaviour makes perfect sense
because each child document contain an essential part of the document.
However, in some situations it may be desirable to compose
a document from a collection of parts
without having mandatory page breaks between then.
For this case, the package
provides a mechanism to include parts
by |\input| which can also be processed individually.
However, by construction this mechanism
requires manual handling of the content to be output.

%%%%%%%%%%%%%%%%%%%%%%%%%%%%%%%%%%%%%%%%
\DescribeMacro{\ifchilddocmanual}
The main file should be prepared as usual, see \secref{sec:include}.
However, the document body must make a distinction
between processing of an individual part and of the main document, e.g.:
%
\begin{center}
\begin{tabular}{l}
|\ifchilddocmanual|\\
|\input{\childdocname}|\\
|\||else|\\
\textit{document body with }|\input{|\textit{part}|}|\\
|\||fi|
\end{tabular}
\end{center}
%
The conditional |\ifchilddocmanual| is true whenever
a part to be included by |\input| is being compiled,
and the name of the part is stored in |\childdocname|.

%%%%%%%%%%%%%%%%%%%%%%%%%%%%%%%%%%%%%%%%
\DescribeMacro{\childdocby}
Each part to be included by |\input| should start with:
%
\begin{center}
\begin{tabular}{l}
|\input{childdoc.def}|\\
|\childdocby{|\textit{main}|}|\\
\end{tabular}
\end{center}
%
The directive |\childdocby| is similar to |\childdocof|
described in \secref{sec:include},
but the subsequent selection of content must be done manually.
To that end, both |\ifchilddoc| and |\ifchilddocmanual|
will be true upon processing of a part,
and the name of the part is stored in |\childdocname|.
Note that |\jobname| will be set to the filename of the current part
so that each part receives an individual |.aux| file
that does not interfere with the |.aux| file(s) of the main document.
This behaviour can be altered by the alternative form
|\childdocby[*]{|\textit{main}|}| (with a non-empty optional argument)
which uses the |.aux| file of the main document
by setting |\jobname| to \textit{main}.

%%%%%%%%%%%%%%%%%%%%%%%%%%%%%%%%%%%%%%%%%%%%%%%%%%%%%%%%%%%%%%%%%%%%%%%%%%%%%%%%
\subsection{Driver Development}
\label{sec:driver}

The \textsf{childdoc} mechanism can also be use for the development
of definition files such as \LaTeX{} styles or classes.
This case differs from the above setup with multiple parts
included by |\include| in that no |\includeonly| should be invoked.
This can be achieved by starting the include file
(before |\ProvidesPackage|) with:
%
\begin{center}
\begin{tabular}{l}
|\input{childdoc.def}|\\
|\childdocforward{|\textit{main}|}|\\
\end{tabular}
\end{center}
%
or alternatively with:
%
\begin{center}
\begin{tabular}{l}
|\input{childdoc.def}|\\
|\childdocby{|\textit{main}|}|\\
\end{tabular}
\end{center}
%
Both forms have slightly different effects as described above.
The main file is prepared as usual, see \secref{sec:include}.

%%%%%%%%%%%%%%%%%%%%%%%%%%%%%%%%%%%%%%%%%%%%%%%%%%%%%%%%%%%%%%%%%%%%%%%%%%%%%%%%
\subsection{Legacy Detection}
\label{sec:detection}

The directive |\childdocmain| in the main file can detect
whether the complete document or merely a child is to be compiled
even without using the directive |\childdocof|.
This method is deprecated because it is less robust
and there is no compelling reason to use it;
it is merely provided for backward compatibility
and it may be removed in future versions.

If the detection mechanism is to be used,
it is mandatory to correctly specify
the filename of the main file as the argument of |\childdocmain|:
%
\begin{center}
\begin{tabular}{l}
|\input{childdoc.def}|\\
|\childdocmain{|\textit{main}|}|\\
\end{tabular}
\end{center}
%
If |\jobname| does not match the argument \textit{main} of |\childdocmain|,
it is assumed that |\jobname| points to the child file to be compiled.
When using |\childdocmain| with the main file specified as argument,
it suffices to start a child file
with just |\input{|\textit{main}|}|
without loading of the package and using |\childdocof|.
If instead all processing is done
with the appropriate \textsf{childdoc} directives,
the argument of \textit{main} of |\childdocmain| can be empty.

An alternative version of the command line processing described
in \secref{sec:commandline} using the detection mechanism reads:
%
\begin{center}
|... -jobname "|\textit{target}|" "|[\textit{flags}]%
[|\def\jobname{|\textit{dest}|}|]|\input{|\textit{main}|}"|
\end{center}

%%%%%%%%%%%%%%%%%%%%%%%%%%%%%%%%%%%%%%%%%%%%%%%%%%%%%%%%%%%%%%%%%%%%%%%%%%%%%%%%
\subsection{Manual Code}
\label{sec:manual}

In case one cannot be certain whether the definitions file |childdoc.def|
is installed on the target \TeX{} distribution
and one prefers not to ship it,
it is conceivable to paste a few relevant commands into the sources.

To that end, drop all statements |\input{childdoc.def}|
and perform the replacements as outlined below.
Instead of |\childdocmain{|\textit{main}|}| add the following code
to the top of the main file:
%
\begin{center}
\begin{tabular}{l}
|\||ifdefined\childdocname\endinput\||fi\newif\ifchilddoc|\\
|\edef\childdocname{\scantokens\expandafter{\jobname\noexpand}}|\\
|\def\childdocmain{|\textit{main}|}\||ifx\childdocmain\childdocname\||else|\\
|\childdoctrue\includeonly{\childdocname}\let\jobname\childdocmain\||fi|\\
\end{tabular}
\end{center}
%
Instead of |\childdocof{|\textit{main}|}| just include the main file
at the top of each child file:
%
\begin{center}
|\input{|\textit{main}|}|
\end{center}
%
A simple redirection |\childdocforward{|\textit{dest}|}| is achieved by:
%
\begin{center}
|\def\jobname{|\textit{dest}|}\input{\jobname}|
\end{center}
%
The redirection with prefix
|\childdocforwardprefix[|\textit{prefix}|]{|\textit{dest}|}|
is accomplished by:
%
\begin{center}
\begin{tabular}{l}
|{\edef\jobname{\scantokens\expandafter{\jobname\noexpand}}|\\
|\def\redirectjob |\textit{prefix}|#1~~~{\gdef\jobname{|\textit{dest}|#1}}|\\
|\expandafter\redirectjob\jobname~~~}\input{\jobname}|
\end{tabular}
\end{center}

In an alternative approach,
child documents can be compiled by a specific command line
without additional code or specific definitions:
%
\begin{center}
|... -jobname "|\textit{target}|" "|[\textit{flags}]%
|\includeonly{|\textit{dest}|}\input{|\textit{main}|}"|
\end{center}
%

%%%%%%%%%%%%%%%%%%%%%%%%%%%%%%%%%%%%%%%%%%%%%%%%%%%%%%%%%%%%%%%%%%%%%%%%%%%%%%%%
%%%%%%%%%%%%%%%%%%%%%%%%%%%%%%%%%%%%%%%%%%%%%%%%%%%%%%%%%%%%%%%%%%%%%%%%%%%%%%%%
\section{Information}

%%%%%%%%%%%%%%%%%%%%%%%%%%%%%%%%%%%%%%%%%%%%%%%%%%%%%%%%%%%%%%%%%%%%%%%%%%%%%%%%
\subsection{Copyright}

Copyright \copyright{} 2017--2018 Niklas Beisert

This work may be distributed and/or modified under the
conditions of the \LaTeX{} Project Public License, either version 1.3
of this license or (at your option) any later version.
The latest version of this license is in
  \url{http://www.latex-project.org/lppl.txt}
and version 1.3 or later is part of all distributions of \LaTeX{}
version 2005/12/01 or later.

This work has the LPPL maintenance status `maintained'.

The Current Maintainer of this work is Niklas Beisert.

This work consists of the files |README.txt|, |childdoc.ins| and |childdoc.dtx|
as well as the derived files |childdoc.def|, |cdocsamp.tex|
with |cdocsch1.tex|, |cdocsch2.tex|, |cdocspt3.tex|, |cdocspt4.tex|,
|cdocsdrf.tex|, |cdocsfn1.tex|, |cdocsfn2.tex|
as well as |childdoc.pdf|.

%%%%%%%%%%%%%%%%%%%%%%%%%%%%%%%%%%%%%%%%%%%%%%%%%%%%%%%%%%%%%%%%%%%%%%%%%%%%%%%%
\subsection{Files and Installation}

The package consists of the files:
%
\begin{center}
\begin{tabular}{ll}
    |README.txt|   & readme file \\
    |childdoc.ins| & installation file \\
    |childdoc.dtx| & source file \\
    |childdoc.def| & definition file \\
    |cdocsamp.tex| & sample main file \\
    |cdocsch1.tex| & sample include file \\
    |cdocsch2.tex| & sample include file \\
    |cdocspt3.tex| & sample part file \\
    |cdocspt4.tex| & sample part file \\
    |cdocsdrf.tex| & sample redirection file \\
    |cdocsfn1.tex| & sample redirection file \\
    |cdocsfn2.tex| & sample redirection file \\
    |childdoc.pdf| & manual
\end{tabular}
\end{center}
%
The distribution consists of the files
|README.txt|, |childdoc.ins| and |childdoc.dtx|.
%
\begin{itemize}
\item
Run (pdf)\LaTeX{} on |childdoc.dtx|
to compile the manual |childdoc.pdf| (this file).
\item
Run \LaTeX{} on |childdoc.ins| to create the definitions file |childdoc.def|
and the sample |cdocsamp.tex| with include files
|cdocsch1.tex|, |cdocsch2.tex|, |cdocspt3.tex|, |cdocspt4.tex|,
|cdocsdrf.tex|, |cdocsfn1.tex|, |cdocsfn2.tex|.
Then copy the file |childdoc.def| to an appropriate directory of your \LaTeX{}
distribution, e.g.\ \textit{texmf-root}|/tex/latex/childdoc|.
\end{itemize}

%%%%%%%%%%%%%%%%%%%%%%%%%%%%%%%%%%%%%%%%%%%%%%%%%%%%%%%%%%%%%%%%%%%%%%%%%%%%%%%%
\subsection{Related CTAN Packages}

There are several other packages which offer a similar functionality:
%
\begin{itemize}
\item
The packages
\href{http://ctan.org/pkg/docmute}{\textsf{docmute}},
\href{http://ctan.org/pkg/includex}{\textsf{includex}} and
\href{http://ctan.org/pkg/standalone}{\textsf{standalone}}
provide commands to include only the document body of
a child file thus allowing both files to be compiled individually.
\item
The packages \href{http://ctan.org/pkg/subdocs}{\textsf{subdocs}}
and \href{http://ctan.org/pkg/subfiles}{\textsf{subfiles}}
provide structures in which the main and child documents can be
encapsulated and allowing them to be compiled individually.
The inclusion mechanism is different from the conventional |\include|.
\item
The package \href{http://ctan.org/pkg/combine}{\textsf{combine}}
is an elaborate solution to combine several documents into one.
\end{itemize}
%
See also the CTAN topic \href{http://ctan.org/topic/subdocs}{\textsf{subdocs}}
for further related packages.
The present package differs from the above solutions in that
a document structure constructed with the conventional |\include| mechanism
just needs two extra commands at the top of every file
such that all constituent files can be compiled individually.

%%%%%%%%%%%%%%%%%%%%%%%%%%%%%%%%%%%%%%%%%%%%%%%%%%%%%%%%%%%%%%%%%%%%%%%%%%%%%%%%
%\subsection{Feature Suggestions}
%
%The following is a list of features which may be useful for future
%versions of this package:
%%
%\begin{itemize}
%\item
%\ldots
%\end{itemize}

%%%%%%%%%%%%%%%%%%%%%%%%%%%%%%%%%%%%%%%%%%%%%%%%%%%%%%%%%%%%%%%%%%%%%%%%%%%%%%%%
\subsection{Revision History}

%%%%%%%%%%%%%%%%%%%%%%%%%%%%%%%%%%%%%%%%
\paragraph{v2.0:} 2018/12/30

\begin{itemize}
\item
immediate forward processing
\item
added |\childdocby| mechanism
\item
manual restructured
\end{itemize}

%%%%%%%%%%%%%%%%%%%%%%%%%%%%%%%%%%%%%%%%
\paragraph{v1.6:} 2018/01/17

\begin{itemize}
\item
application for development of include files
\item
corrections to manual
\end{itemize}

%%%%%%%%%%%%%%%%%%%%%%%%%%%%%%%%%%%%%%%%
\paragraph{v1.5:} 2017/05/21

\begin{itemize}
\item
more complete structuring introduced
\item
|\childdocof| introduced
\item
|\childdoc| renamed to |\childdocmain|
\item
|\childredirect| renamed to |\childdocforward| and |\childdocforwardprefix|
and functionality expanded
\end{itemize}

%%%%%%%%%%%%%%%%%%%%%%%%%%%%%%%%%%%%%%%%
\paragraph{v1.0:} 2017/04/27

\begin{itemize}
\item
manual and install package
\item
first version published on CTAN
\end{itemize}

%%%%%%%%%%%%%%%%%%%%%%%%%%%%%%%%%%%%%%%%
\paragraph{v0.6:} 2017/04/26

\begin{itemize}
\item
redirection mechanism added
\end{itemize}

%%%%%%%%%%%%%%%%%%%%%%%%%%%%%%%%%%%%%%%%
\paragraph{v0.5:} 2017/04/26

\begin{itemize}
\item
functionality in definition file
\end{itemize}


%%%%%%%%%%%%%%%%%%%%%%%%%%%%%%%%%%%%%%%%%%%%%%%%%%%%%%%%%%%%%%%%%%%%%%%%%%%%%%%%
%%%%%%%%%%%%%%%%%%%%%%%%%%%%%%%%%%%%%%%%%%%%%%%%%%%%%%%%%%%%%%%%%%%%%%%%%%%%%%%%
%%%%%%%%%%%%%%%%%%%%%%%%%%%%%%%%%%%%%%%%%%%%%%%%%%%%%%%%%%%%%%%%%%%%%%%%%%%%%%%%
\appendix

\settowidth\MacroIndent{\rmfamily\scriptsize 000\ }

 \DocInput{childdoc.dtx}

\end{document}
%</driver>
% \fi
%
% %%%%%%%%%%%%%%%%%%%%%%%%%%%%%%%%%%%%%%%%%%%%%%%%%%%%%%%%%%%%%%%%%%%%%%%%%%%%%%
% %%%%%%%%%%%%%%%%%%%%%%%%%%%%%%%%%%%%%%%%%%%%%%%%%%%%%%%%%%%%%%%%%%%%%%%%%%%%%%
% \section{Sample}
%\iffalse
%<*samplemain>
%\fi
%
% The following presents a sample document
% with two chapters, two parts, a title page,
% a compile flag as well as three forwarding files to set the flag.
% It consists of eight |.tex| files:
% \begin{center}
% \begin{tabular}{ll}
% |cdocsamp.tex|&main file\\
% |cdocsch1.tex|&include file for chapter 1\\
% |cdocsch2.tex|&include file for chapter 2\\
% |cdocspt3.tex|&include file for part 3\\
% |cdocspt4.tex|&include file for part 4\\
% |cdocsdrf.tex|&forwarding file for main file in draft mode\\
% |cdocsfi1.tex|&forwarding file for final version of chapter 1\\
% |cdocsfi2.tex|&forwarding file for final version of chapter 2\\
% \end{tabular}
% \end{center}
% Each of the eight files can be compiled directly by the \LaTeX{} compiler.
%
% %%%%%%%%%%%%%%%%%%%%%%%%%%%%%%%%%%%%%%
% \paragraph{Main File.}
%
% The main file is called |cdocsamp.tex|.
%
% Load the \textsf{childdoc} definitions and
% declare the filename for the main document:
%    \begin{macrocode}
\input{childdoc.def}
\childdocmain{}
%    \end{macrocode}

% Optional override for |\version| flag:
%    \begin{macrocode}
%%\ifchilddoc\else\providecommand{\version}{draft}\fi
%    \end{macrocode}

% Define the default values for the |\version| flag
% (|final| for the main file and |draft| for childs):
%    \begin{macrocode}
\ifchilddoc
\providecommand{\version}{draft}
\else
\providecommand{\version}{final}
\fi
%    \end{macrocode}

% Load the standard document class:
%    \begin{macrocode}
\documentclass[12pt]{article}
%    \end{macrocode}

% Start the document body:
%    \begin{macrocode}
\begin{document}
%    \end{macrocode}

% Declare a title page.
% Print title, part of document being processed and version flag:
%    \begin{macrocode}
\addtocounter{page}{-1}
\begin{center}
{\LARGE\bfseries{}childdoc example\par}
\vspace{1cm}
\ifchilddoc
\ifchilddocmanual part\else chapter\fi:
`\childdocname' of `\childdocjob'\par
\else
main document: `\childdocjob'\par
\fi
version: \version\par
\end{center}
\newpage
%    \end{macrocode}

% Manually include selected file,
% otherwise process as usual:
%    \begin{macrocode}
\ifchilddocmanual
\section*{part `\childdocname'}
\input{\childdocname}
\else
%    \end{macrocode}

% Include the two chapters:
%    \begin{macrocode}
\include{cdocsch1}
\include{cdocsch2}
%    \end{macrocode}

% Include the two parts unless only chapters should be displayed:
%    \begin{macrocode}
\ifchilddoc\else
\section{part three}
\input{cdocspt3}
\section{part four}
\input{cdocspt4}
\fi
%    \end{macrocode}

% Process as usual until here:
%    \begin{macrocode}
\fi
%    \end{macrocode}

% End of document body:
%    \begin{macrocode}
\end{document}
%    \end{macrocode}
%\iffalse
%</samplemain>
%\fi
%
% %%%%%%%%%%%%%%%%%%%%%%%%%%%%%%%%%%%%%%
% \paragraph{Chapter Include Files.}
%
% The include files are called |cdocsch1.tex| and |cdocsch2.tex|.
%
%\iffalse
%<*samplechap1|samplechap2>
%\fi

% Optional override for |\version| flag:
%    \begin{macrocode}
%%\providecommand{\version}{final}
%    \end{macrocode}

% Include the main document:
%    \begin{macrocode}
\input{childdoc.def}
\childdocof{cdocsamp}
%    \end{macrocode}

%\iffalse
%</samplechap1|samplechap2>
%\fi
%
%\iffalse
%<*samplechap1>
%\fi
% Some text for chapter 1:
%    \begin{macrocode}
\section{one}
some text in chapter one
%    \end{macrocode}

%\iffalse
%</samplechap1>
%\fi
% Some text for chapter 2:
%\iffalse
%<*samplechap2>
%\fi
%    \begin{macrocode}
\section{two}
more text in chapter two
%    \end{macrocode}

%\iffalse
%</samplechap2>
%\fi
%
% %%%%%%%%%%%%%%%%%%%%%%%%%%%%%%%%%%%%%%
% \paragraph{Part Include Files.}
%
% The include files are called |cdocspt3.tex| and |cdocspt4.tex|.
%
%\iffalse
%<*samplepart3|samplepart4>
%\fi

% Optional override for |\version| flag:
%    \begin{macrocode}
%%\providecommand{\version}{final}
%    \end{macrocode}

% Include the main document:
%    \begin{macrocode}
\input{childdoc.def}
\childdocby{cdocsamp}
%    \end{macrocode}

%\iffalse
%</samplepart3|samplepart4>
%\fi
%
%\iffalse
%<*samplepart3>
%\fi
% Some text for part 3:
%    \begin{macrocode}
some text in part three
%    \end{macrocode}

%\iffalse
%</samplepart3>
%\fi
% Some text for part 4:
%\iffalse
%<*samplepart4>
%\fi
%    \begin{macrocode}
more text in part four
%    \end{macrocode}

%\iffalse
%</samplepart4>
%\fi
%
% %%%%%%%%%%%%%%%%%%%%%%%%%%%%%%%%%%%%%%
% \paragraph{Forwarding for a Complete Draft.}
%
% The following forwarding file |cdocsdrf.tex|
% compiles the main document in draft mode:
%\iffalse
%<*sampledraft>
%\fi
%    \begin{macrocode}
\def\version{draft}
\input{childdoc.def}
\childdocforward{cdocsamp}
%    \end{macrocode}

%\iffalse
%</sampledraft>
%\fi
%
% %%%%%%%%%%%%%%%%%%%%%%%%%%%%%%%%%%%%%%
% \paragraph{Forwarding for Final Version of the Chapters.}
%
% The following forwarding files |cdocsfn1.tex| and |cdocsfn2.tex|
% (with identical content)
% compile the final versions of the child documents
% |cdocsch1.tex| and |cdocsch2.tex|, respectively:
%\iffalse
%<*samplefinal>
%\fi
%    \begin{macrocode}
\def\version{final}
\input{childdoc.def}
\childdocforwardprefix[cdocsamp]{cdocsfn}{cdocsch}
%    \end{macrocode}

%\iffalse
%</samplefinal>
%\fi
%
% %%%%%%%%%%%%%%%%%%%%%%%%%%%%%%%%%%%%%%
% \paragraph{Command Line Processing.}
%
% The following three command lines generate the output files
% |cdocscld|, |cdocscl1| and |cdocscl2|
% which should be identical to
% |cdocsdrf|, |cdocsch1| and |cdocsfn2|, respectively:
% \begin{center}
% \begin{tabular}{l}
% |latex -jobname cdocscld \|\\
% |  "\def\version{draft}\input{childdoc.def}\childdocforward{cdocsamp}"|\\
% |latex -jobname cdocscl1 \|\\
% |  "\input{childdoc.def}\childdocforward[cdocsamp]{cdocsch1}"|\\
% |latex -jobname cdocscl2 \|\\
% |  "\def\version{final}\input{childdoc.def}\childdocforward{cdocsch2}"|
% \end{tabular}
% \end{center}
% Note that the trailing backslash on each first line
% merely continues the input to the second line
% (for convenient cut ant paste).
% Furthermore, the command |latex| can be replaced by any
% of its alternative versions such as |pdflatex|.
%
% %%%%%%%%%%%%%%%%%%%%%%%%%%%%%%%%%%%%%%%%%%%%%%%%%%%%%%%%%%%%%%%%%%%%%%%%%%%%%%
% %%%%%%%%%%%%%%%%%%%%%%%%%%%%%%%%%%%%%%%%%%%%%%%%%%%%%%%%%%%%%%%%%%%%%%%%%%%%%%
% \section{Implementation}
%\iffalse
%<*package>
%\fi
%
% This section describes the definitions file |childdoc.def|.

% The definitions cannot be loaded using |\usepackage| or |\RequirePackage|
% which has a mechanism to prevent loading a style file more than once.
% When loading the definitions by means of |\input|
% multiple instances have to be prevented manually:
%\iffalse
%This code needs to be before the `\ProvidesFile' directive
%which is defined at the beginning of this file.
%Therefore it is also placed there and commented out here.
%</package>
%<*discard>
%\fi
%    \begin{macrocode}
\ifdefined\childdocmain\endinput\fi
%    \end{macrocode}
%\iffalse
%</discard>
%<*package>
%\fi
%
% \macro{\ifchilddoc}
% \macro{\ifchilddocmanual}
% The conditional |\ifchilddoc| tells whether a
% child (true) or main (false) document is being compiled.
% The conditional |\ifchilddocmanual| tells whether
% the |\includeonly| mechanism is used (false) or
% the selection of child files must be performed manually (true).
% The definitions initialise to false:
%    \begin{macrocode}
\newif\ifchilddoc
\newif\ifchilddocmanual
%    \end{macrocode}

% \macro{\childdocname}
% \macro{\childdocjob}
% The macro |\childdocname| stores the name of the main document
% to be compiled. The macro |\childdocjob| stores the name of
% the document on which the \LaTeX{} compiler was originally invoked.
% The content of |\jobname| cannot be compared
% to filenames specified in the source due to different catcodes.
% The following code rescans |\jobname|, stores the result
% in |\childdocname| and saves a copy in |\childdocjob|:
%    \begin{macrocode}
\edef\childdocname{\scantokens\expandafter{\jobname\noexpand}}
\let\childdocjob\childdocname
%    \end{macrocode}

% \macro{\childdocdisable}
% The macro |\childdocdisable| prevents the main file
% from being processed more than once.
% At this stage, the main document command |\childdocmain|
% is assumed to be called once again where it should do nothing.
% Any subsequent call to it should prevent
% a secondary processing of the main document
% It overwrites the forwarding commands
% |\childdocof| and |\childdocforward|
% with empty macros to prevent further inclusions of the main document:
%    \begin{macrocode}
\newcommand{\childdocdisable}
{
  \renewcommand{\childdocmain}[1]{\renewcommand{\childdocmain}[1]{\endinput}}
  \renewcommand{\childdocof}[1]{}
  \renewcommand{\childdocby}[2][]{}
  \renewcommand{\childdocforward}[2][]{}
  \renewcommand{\childdocdisable}{}
}
%    \end{macrocode}

% \macro{\childdocmain}
% The macro |\childdocmain| is to be called at the top of the main file
% with nothing or the main filename (without extension) as argument.
% First, it breaks loops.
% If the argument is not empty and does not match |\childdocname|
% (which is set by the first inclusion of |childdoc.def|),
% |\ifchilddoc| is set to true, |\includeonly| is applied to the child file
% and |\jobname| is set to the main file
% (for proper handling of |.aux| files):
%    \begin{macrocode}
\newcommand{\childdocmain}[1]
{
  \childdocdisable\childdocmain{}
  \if?#1?\else
    \begingroup
      \def\childdoctmp{#1}
      \ifx\childdoctmp\childdocname
        \def\childdoctmp{}
      \else
        \def\childdoctmp
        {
          \childdoctrue
          \includeonly{\childdocname}
          \def\childdocjob{#1}
          \def\jobname{#1}
        }
      \fi
      \expandafter
    \endgroup
    \childdoctmp
  \fi
}
%    \end{macrocode}

% \macro{\childdocof}
% The command |\childdocof| redirects
% compilation to the main file |#1|.
%    \begin{macrocode}
\newcommand{\childdocof}[1]
{
  \childdocdisable
  \childdoctrue
  \includeonly{\childdocname}
  \def\jobname{#1}
  \def\childdocjob{#1}
  \input{#1}
}
%    \end{macrocode}

% \macro{\childdocby}
% The command |\childdocby| ....
%    \begin{macrocode}
\newcommand{\childdocby}[2][]
{
  \childdocdisable
  \childdoctrue
  \childdocmanualtrue
  \if?#1?\else
    \def\jobname{#2}
  \fi
  \def\childdocjob{#2}
  \input{#2}
  \endinput
}
%    \end{macrocode}

% \macro{\childdocforward}
% The command |\childdocforward| redirects
% compilation to the main file or
% (if the optional argument is given) a child file.
% Parameters are set as if the main file
% or a child file starting with |\childdocof| was compiled.
% Then compilation is handed over to the main file:
%    \begin{macrocode}
\newcommand{\childdocforward}[2][]
{
  \begingroup
    \if?#1?
      \def\childdoctmp
      {
        \def\childdocname{#2}
        \def\childdocjob{#2}
        \def\jobname{#2}
        \input{#2}
        \endinput
      }
    \else
      \def\childdoctmp
      {
        \childdocdisable
        \def\childdocname{#2}
        \childdoctrue
        \includeonly{#2}
        \def\childdocjob{#1}
        \def\jobname{#1}
        \input{#1}
        \endinput
      }
    \fi
    \expandafter
  \endgroup
  \childdoctmp
}
%    \end{macrocode}

% \macro{\childdocforwardprefix}
% The command |\childdocforwardprefix| redirects
% compilation to the main or a child file by means of a pattern.
% The prefix |#1| in the current filename is replaced by |#2|
% and the suffix of the current filename is kept
% (it is assumed that the filename does not contain the substring `|~~~|'
% which is used as a delimiter).
% Compilation is handed over to the new file by |\childdocforward|:
%    \begin{macrocode}
\newcommand{\childdocforwardprefix}[3][]
{
  \begingroup
    \def\childdocextract #2##1~~~{\def\childdoctmp{\childdocforward[#1]{#3##1}}}
    \expandafter\childdocextract\childdocname~~~
    \expandafter
  \endgroup
  \childdoctmp
}
%    \end{macrocode}

% \macro{\childdoc}
% The deprecated macro |\childdoc| is a legacy version of |\childdocmain|:
%    \begin{macrocode}
\newcommand{\childdoc}{\childdocmain}
%    \end{macrocode}

% \macro{\childdocredirect}
% The deprecated macro |\childdocredirect| is a legacy version
% of |\childdocforward| and |\childdocforwardprefix|:
%    \begin{macrocode}
\newcommand{\childdocredirect}[2][]
{
  \begingroup
    \if?#1?
      \def\childdoctmp{\childdocforward{#2}}
    \else
      \def\childdoctmp{\childdocforwardprefix{#1}{#2}}
    \fi
    \expandafter
  \endgroup
  \childdoctmp
}
%    \end{macrocode}

%\iffalse
%</package>
%\fi
%
\endinput

\childdocby{cdocsamp}
%    \end{macrocode}

%\iffalse
%</samplepart3|samplepart4>
%\fi
%
%\iffalse
%<*samplepart3>
%\fi
% Some text for part 3:
%    \begin{macrocode}
some text in part three
%    \end{macrocode}

%\iffalse
%</samplepart3>
%\fi
% Some text for part 4:
%\iffalse
%<*samplepart4>
%\fi
%    \begin{macrocode}
more text in part four
%    \end{macrocode}

%\iffalse
%</samplepart4>
%\fi
%
% %%%%%%%%%%%%%%%%%%%%%%%%%%%%%%%%%%%%%%
% \paragraph{Forwarding for a Complete Draft.}
%
% The following forwarding file |cdocsdrf.tex|
% compiles the main document in draft mode:
%\iffalse
%<*sampledraft>
%\fi
%    \begin{macrocode}
\def\version{draft}
% \iffalse
%
% childdoc.dtx Copyright (C) 2017-2018 Niklas Beisert
%
% This work may be distributed and/or modified under the
% conditions of the LaTeX Project Public License, either version 1.3
% of this license or (at your option) any later version.
% The latest version of this license is in
%   http://www.latex-project.org/lppl.txt
% and version 1.3 or later is part of all distributions of LaTeX
% version 2005/12/01 or later.
%
% This work has the LPPL maintenance status `maintained'.
%
% The Current Maintainer of this work is Niklas Beisert.
%
% This work consists of the files childdoc.dtx and childdoc.ins
% and the derived files childdoc.def and cdocsamp.tex with
% cdocsch1.tex, cdocsch2.tex, cdocsdrf.tex, cdocsfn1.tex, cdocsfn2.tex.
%
%<package>\ifdefined\childdocmain\endinput\fi
%<package>\ProvidesFile{childdoc.def}[2018/12/30 v2.0 child document driver]
%<samplemain>\ProvidesFile{cdocsamp.tex}[2018/12/30 v2.0 sample for childdoc]
%<*driver>
%\ProvidesFile{childdoc.drv}[2018/12/30 v2.0 childdoc reference manual file]
\PassOptionsToClass{10pt,a4paper}{article}
\documentclass{ltxdoc}

\usepackage[margin=35mm]{geometry}
\usepackage{hyperref}
\usepackage{hyperxmp}
\usepackage[usenames]{color}

\hypersetup{colorlinks=true}
\hypersetup{pdfstartview=FitH}
\hypersetup{pdfpagemode=UseNone}
\hypersetup{pdfsource={}}
\hypersetup{pdflang={en-UK}}
\hypersetup{pdfcopyright={Copyright 2017-2018 Niklas Beisert.
  This work may be distributed and/or modified under the
  conditions of the LaTeX Project Public License, either version 1.3
  of this license or (at your option) any later version.}}
\hypersetup{pdflicenseurl={http://www.latex-project.org/lppl.txt}}
\hypersetup{pdfcontactaddress={ETH Zurich, ITP, HIT K,
  Wolfgang-Pauli-Strasse 27}}
\hypersetup{pdfcontactpostcode={8093}}
\hypersetup{pdfcontactcity={Zurich}}
\hypersetup{pdfcontactcountry={Switzerland}}
\hypersetup{pdfcontactemail={nbeisert@itp.phys.ethz.ch}}
\hypersetup{pdfcontacturl={http://people.phys.ethz.ch/\xmptilde nbeisert/}}

\newcommand{\secref}[1]{\hyperref[#1]{section \ref*{#1}}}

\parskip1ex
\parindent0pt
\let\olditemize\itemize
\def\itemize{\olditemize\parskip0pt}

\begin{document}

\title{The \textsf{childdoc} Package}
\hypersetup{pdftitle={The childdoc Package}}
\author{Niklas Beisert\\[2ex]
  Institut f\"ur Theoretische Physik\\
  Eidgen\"ossische Technische Hochschule Z\"urich\\
  Wolfgang-Pauli-Strasse 27, 8093 Z\"urich, Switzerland\\[1ex]
  \href{mailto:nbeisert@itp.phys.ethz.ch}
  {\texttt{nbeisert@itp.phys.ethz.ch}}}
\hypersetup{pdfauthor={Niklas Beisert}}
\hypersetup{pdfsubject={Manual for the LaTeX2e Package childdoc}}
\date{30 December 2018, \textsf{v2.0}}
\maketitle

\begin{abstract}\noindent
\textsf{childdoc} is a \LaTeXe{} package
that enables the direct compilation
of document sections included by |\include|
to individual files.
\end{abstract}

\begingroup
\parskip0ex
\tableofcontents
\endgroup

%%%%%%%%%%%%%%%%%%%%%%%%%%%%%%%%%%%%%%%%%%%%%%%%%%%%%%%%%%%%%%%%%%%%%%%%%%%%%%%%
%%%%%%%%%%%%%%%%%%%%%%%%%%%%%%%%%%%%%%%%%%%%%%%%%%%%%%%%%%%%%%%%%%%%%%%%%%%%%%%%
\section{Introduction}

\LaTeX{} provides a mechanism to structure a large document (such as a book)
into a main file and several child files (containing the chapters)
using the |\include| command.
This mechanism is beneficial for documents
which span hundreds of pages in order to
make the source file(s) more manageable.
Moreover, compilation can be restricted to
selected child files by means of the |\includeonly| command.
The latter feature can be used to reduce the compilation time while editing
(this was significantly more useful in the earlier days of \LaTeX{})
or to generate a smaller document which is easier to navigate.
Another application of |\includeonly| is to generate
documents consisting of selected parts of the complete document.

However, there are a few drawbacks of the plain |\include| mechanism:
\begin{itemize}
\item
The child files cannot be compiled on their own,
they can only be compiled via the main file.
A naive editing environment
(such as a text editor with an option
to have the current file processed by \LaTeX)
may require one to switch to the main file before compiling;
attempting to compile the child file produces errors.
\item
The main file must be modified (each time)
to adjust the |\includeonly| command
to the present needs. This easily leaves the main file in a messy state.
\item
The generated document will always carry the filename
of the main document. This is inconvenient if
several child files are to be compiled and
to be kept for distribution.
\end{itemize}

The present package provides a simple interface
to make child files individually compilable by \LaTeX{}.
Compiling a child file then has the same effect as compiling
the main file with an |\includeonly| command
to select the appropriate child.
Moreover the generated document will carry the name of the child
rather than the main file.
This resolves all three above issues.

This feature is meant to make the editing of books,
thesis documents and lecture notes somewhat more convenient.
However, the package can also be used efficiently for
composing a series of documents (such as exercise sheets)
which are typically distributed individually.
It then assists the author in generating the individual documents
(potentially in different versions)
as well as a document containing the collected series.
Another application is in developing style files
or other kinds of included material
where compilation of the style file could redirect
to a sample or test file.

%%%%%%%%%%%%%%%%%%%%%%%%%%%%%%%%%%%%%%%%%%%%%%%%%%%%%%%%%%%%%%%%%%%%%%%%%%%%%%%%
%%%%%%%%%%%%%%%%%%%%%%%%%%%%%%%%%%%%%%%%%%%%%%%%%%%%%%%%%%%%%%%%%%%%%%%%%%%%%%%%
\section{Usage}

First of all, the package \textsf{childdoc} is \emph{not} a standard
\LaTeXe{} |.sty| style file! Therefore it needs to be invoked in
a non-standard way.

%%%%%%%%%%%%%%%%%%%%%%%%%%%%%%%%%%%%%%%%%%%%%%%%%%%%%%%%%%%%%%%%%%%%%%%%%%%%%%%%
\subsection{Included Files}
\label{sec:include}

%%%%%%%%%%%%%%%%%%%%%%%%%%%%%%%%%%%%%%%%
\DescribeMacro{\childdocmain}
To use the package, add the commands
\begin{center}
\begin{tabular}{l}
|\input{childdoc.def}|\\
|\childdocmain{}|\\
\end{tabular}
\end{center}
at the very top of the main \LaTeX{} file,
in particular \emph{before} the |\documentclass| statement!
The argument of |\childdocmain| should be left empty
(but it must be present).

%%%%%%%%%%%%%%%%%%%%%%%%%%%%%%%%%%%%%%%%
\DescribeMacro{\childdocof}
Furthermore, add the commands
\begin{center}
\begin{tabular}{l}
|\input{childdoc.def}|\\
|\childdocof{|\textit{main}|}|\\
\end{tabular}
\end{center}
at the top of every child file \textit{child}
which is included by |\include{|\textit{child}|}|
from within the main file
(or at least for those files to be compiled individually).
The argument \textit{main} must be the filename of the main file.

There are a couple of
considerations in setting up the main and child documents:

%%%%%%%%%%%%%%%%%%%%%%%%%%%%%%%%%%%%%%%%
\paragraph{Restrictions.}

Please note the following restrictions:
\begin{itemize}
\item
|\childdocmain| must be called with one argument \textit{main}
to ensure compatibility with earlier version of the package.
It must either be empty (|\childdocmain{}|)
or precisely match the filename of the main file in which it is specified.
See \secref{sec:detection} for further information.
\item
The filename \textit{main} must be specified without the |.tex| extension.
\item
The filename \textit{main} is case sensitive
(even in case-insensitive file systems)
due to internal string comparison.
\item
The argument \textit{main} should be fully expanded, it cannot be a macro.
\item
Subdirectories and special characters should be avoided in filenames.
\item
The command |\childdocmain{|\textit{main}|}| must be followed by a whitespace.
It should not be followed immediately by another command
or by a comment mark `|%|'.
This is because the \TeX{} parser reads the token immediately following
the argument of |\childdocmain| and puts it
at the beginning of every child section;
however, a white\-space is ignored.
\end{itemize}

%%%%%%%%%%%%%%%%%%%%%%%%%%%%%%%%%%%%%%%%
\paragraph{Content of Main File.}

It is advisable to place all content in the child files included by |\include|.
Any output contained in the main file will appear in all child documents
unless suppressed manually;
it cannot be suppressed automatically by the |\includeonly| directive
and thus should normally be avoided.
A method to include some content in the main file
by means of conditional processing is described in \secref{sec:conditional}.

%%%%%%%%%%%%%%%%%%%%%%%%%%%%%%%%%%%%%%%%
\paragraph{Page Numbering.}

When only a part of the document is compiled,
the appropriate numbering of pages
(as well as other status parameters)
is determined from the |.aux| files.
The latter contain information from previous passes.
However this information needs to propagate through
all intermediate child documents.
Therefore the page numbering in child documents may well
be inconsistent until the complete document is compiled at least once.

A useful (if unconventional) way to always ensure a consistent
page numbering is to restart the numbering in each child document
and denote the pages by `\textit{child}|.|\textit{page}'
where \textit{child} represents the chapter/section number of the child file.
This can be achieved by the command
|\numberwithin{page}{|\textit{child}|}|
of the \textsf{amsmath} package
where \textit{child} can be |chapter| or |section|
depending on the chosen structuring.
Alternatively, one can modify the macro |\thepage| appropriately
and reset the counter |page| at the start of each child file.

%%%%%%%%%%%%%%%%%%%%%%%%%%%%%%%%%%%%%%%%%%%%%%%%%%%%%%%%%%%%%%%%%%%%%%%%%%%%%%%%
\subsection{Conditional Processing}
\label{sec:conditional}

The package provides a mechanism to compile different versions
of a document. To customise the versions further some conditional processing
can come in handy to distinguish which version is being compiled.
The package provides two macros to describe the compilation context:

%%%%%%%%%%%%%%%%%%%%%%%%%%%%%%%%%%%%%%%%
\DescribeMacro{\ifchilddoc}
The conditional |\ifchilddoc| distinguishes between the compilation of
child documents and the main document:
%
\begin{center}
|\ifchilddoc |\textit{child-code}| |[|\||else |\textit{main-code}]| \||fi|
\end{center}

%%%%%%%%%%%%%%%%%%%%%%%%%%%%%%%%%%%%%%%%
\DescribeMacro{\childdocname}
\DescribeMacro{\childdocjob}
The macro |\childdocname| contains the filename (without extension)
of the main or child file being processed.
Note that |\childdocjob| will always contain the name of the main file.

%%%%%%%%%%%%%%%%%%%%%%%%%%%%%%%%%%%%%%%%
\paragraph{Title Page.}

Conditional processing can be used to include a title or banner page
in the main document when proper precautions are taken.
Importantly, the code in the main file should ensure that the page counter
(as well as other status parameters which are stored in the |.aux| files)
takes the same value after the conditional processing.
Otherwise the page numbers may take divergent values
depending on which part is compiled.

For example, a title page could be declared by:
%
\begin{center}
\begin{tabular}{l}
|\ifchilddoc\||else|\\
|\addtocounter{page}{-1}|\\
\textit{code for title page}\\
|\newpage|\\
|\||fi|
\end{tabular}
\end{center}
%
A banner page for the child documents can be generated by:
%
\begin{center}
\begin{tabular}{l}
|\ifchilddoc|\\
|\addtocounter{page}{-1}|\\
\textit{code for banner page}\\
|\newpage|\\
|\||fi|
\end{tabular}
\end{center}
%
Here one could write a message such as:
\begin{center}
|This is the part \childdocname{} of \childdocjob{}.|
\end{center}

%%%%%%%%%%%%%%%%%%%%%%%%%%%%%%%%%%%%%%%%%%%%%%%%%%%%%%%%%%%%%%%%%%%%%%%%%%%%%%%%
\subsection{Flags}
\label{sec:flags}

The package makes it easy to generate different versions
of the main or child documents.
To this end compilation flags can be defined
and assigned different default values.
They will be particularly useful in conjunction
with the forwarding mechanism described in \secref{sec:forward}.

For example, it may be useful to have a flag |\version|
which can be set to |draft| or |final|.
The document source will contain some conditional code
depending on the value of |\version|.
Suppose further, the flag should default to |final| for the main file
and to |draft| for child files
which is a natural assignment for editing the document.
This is achieved by placing the following code
in the preamble of the main document
(below the |\childdocmain| directive):
%
\begin{center}
\begin{tabular}{l}
|\ifchilddoc|\\
|\providecommand{\version}{draft}|\\
|\||else|\\
|\providecommand{\version}{final}|\\
|\||fi|
\end{tabular}
\end{center}
%
The definition by |\providecommand| makes sure
that previous definitions are not overwritten.
Further statements |\providecommand{\version}{...}|
can thus be added before the above code to override it.

For the main file, one might add a line
(between |\childdocmain| and the above block)
%
\begin{center}
|%\ifchilddoc\||else\providecommand{\version}{draft}\||fi|
\end{center}
%
which can be uncommented to produce a draft version.
Likewise one can add a line to the very top of a child file
(above the |\childdocof{|\textit{main}|}| directive)
%
\begin{center}
|%\providecommand{\version}{final}|
\end{center}
%
which can be uncommented to produce the final version of this child document.

%%%%%%%%%%%%%%%%%%%%%%%%%%%%%%%%%%%%%%%%%%%%%%%%%%%%%%%%%%%%%%%%%%%%%%%%%%%%%%%%
\subsection{Forwarding}
\label{sec:forward}

Different versions of the main or child documents
using compilation flags as described in \secref{sec:flags}
can be (permanently) stored in different files
for convenient compilation, viewing and distribution.
To this end, the package defines a command
to pass on compilation to a different file:

%%%%%%%%%%%%%%%%%%%%%%%%%%%%%%%%%%%%%%%%
\DescribeMacro{\childdocforward}
The command |\childdocforward| redirects processing to
another source file:
%
\begin{center}
\begin{tabular}{l}
|\input{childdoc.def}|\\
|\childdocforward[|\textit{main}|]{|\textit{dest}|}|\\
\end{tabular}
\end{center}
%
The argument \textit{dest} is the destination file
(without extension).
It should be the main file or one of the child files.
Note that further \textsf{childdoc} directives
such as |\childdocof| and |\childdocforward|
in the indicated file will be processed in this form.
The optional argument \textit{main}
passes on directly to the main file \textit{main}
while pretending to compile the child \textit{dest}.
This form behaves as if \textit{dest}
issues |\childdocof{|\textit{main}|}| right away,
and no further \textsf{childdoc} directives will be processed.

%%%%%%%%%%%%%%%%%%%%%%%%%%%%%%%%%%%%%%%%
\DescribeMacro{\...prefix}
In the alternative form |\childdocforwardprefix|,
%
\begin{center}
\begin{tabular}{l}
|\input{childdoc.def}|\\
|\childdocforwardprefix[|\textit{main}|]{|\textit{prefix}|}{|\textit{dest}|}|
\end{tabular}
\end{center}
%
the destination file is determined by a pattern
depending on the current file:
To make this work, the current file must be called
`{\textit{prefix}\hspace{0.2em}\textit{suffix}}'
with \textit{prefix} matching precisely the argument.
Processing is then passed on to the file
`{\textit{dest}\hspace{0.2em}\textit{suffix}}'.
Surely, the same effect is achieved by
directly specifying the
argument `{\textit{dest}\hspace{0.2em}\textit{suffix}}'
in the first form.
However, that requires to set up a different file
for each child. With the alternative form of the command
all these files can have exactly the same content
which simplifies setting them up and maintaining them.

For example, the following file |draft.tex|
with a compilation flag |\version| as described in \secref{sec:flags}
compiles the main document as a draft:
%
\begin{center}
\begin{tabular}{l}
|\def\version{draft}|\\
|\input{childdoc.def}|\\
|\childdocforward{|\textit{main}|}|
\end{tabular}
\end{center}
%
Likewise, the following files |final|\textit{nn}|.tex|
compile the final version of the child document
|child|\textit{nn}|.tex|:
%
\begin{center}
\begin{tabular}{l}
|\def\version{final}|\\
|\input{childdoc.def}|\\
|\childdocforwardprefix{final}{child}|
\end{tabular}
\end{center}
%

Note that when several versions of a main file and/or of each child file
are to be generated, it may be convenient to set up a |Makefile| or
shell script to automatise the process.

%%%%%%%%%%%%%%%%%%%%%%%%%%%%%%%%%%%%%%%%%%%%%%%%%%%%%%%%%%%%%%%%%%%%%%%%%%%%%%%%
\subsection{Command Line Processing}
\label{sec:commandline}

The effect of redirection files can also be achieved by invoking
the \LaTeX{} compiler with a more elaborate command line.
Most conveniently this should be done as part
of a shell script or a |Makefile|.

When using \textsf{childdoc} in the main file, the following
command lines effectively perform a redirection
(note that depending on the shell being used,
backslashes may have to be doubled: `|\|' $\to$ `|\\|'):
%
\begin{center}
|... -jobname "|\textit{target}|" |\\|"|[\textit{flags}]%
|\input{childdoc.def}\childdocforward[|\textit{main}|]{|\textit{dest}|}"|
\end{center}
%
Here \textit{target} is the name of the output file,
\textit{main} is the name of the main file
and \textit{dest} is the name of the main or child file to be processed
(all filenames without extensions).
The optional argument \textit{main} can be omitted
if \textit{main} matches \textit{dest}.
Optionally, compilation \textit{flags} can be defined via |\def| commands.
This command line makes the \TeX{} engine believe
it is compiling the file \textit{target}
whose content is specified as the latter parameter.
The provided code then forwards the processing to
\textit{main} or \textit{dest} as described in \secref{sec:forward}.

%%%%%%%%%%%%%%%%%%%%%%%%%%%%%%%%%%%%%%%%%%%%%%%%%%%%%%%%%%%%%%%%%%%%%%%%%%%%%%%%
\subsection{Include by Input}
\label{sec:input}

Including child documents by |\include| has some restrictions by design.
Most notably, the content of a child document always occupies
its own set of pages; pages cannot be shared between child documents.
Usually, this behaviour makes perfect sense
because each child document contain an essential part of the document.
However, in some situations it may be desirable to compose
a document from a collection of parts
without having mandatory page breaks between then.
For this case, the package
provides a mechanism to include parts
by |\input| which can also be processed individually.
However, by construction this mechanism
requires manual handling of the content to be output.

%%%%%%%%%%%%%%%%%%%%%%%%%%%%%%%%%%%%%%%%
\DescribeMacro{\ifchilddocmanual}
The main file should be prepared as usual, see \secref{sec:include}.
However, the document body must make a distinction
between processing of an individual part and of the main document, e.g.:
%
\begin{center}
\begin{tabular}{l}
|\ifchilddocmanual|\\
|\input{\childdocname}|\\
|\||else|\\
\textit{document body with }|\input{|\textit{part}|}|\\
|\||fi|
\end{tabular}
\end{center}
%
The conditional |\ifchilddocmanual| is true whenever
a part to be included by |\input| is being compiled,
and the name of the part is stored in |\childdocname|.

%%%%%%%%%%%%%%%%%%%%%%%%%%%%%%%%%%%%%%%%
\DescribeMacro{\childdocby}
Each part to be included by |\input| should start with:
%
\begin{center}
\begin{tabular}{l}
|\input{childdoc.def}|\\
|\childdocby{|\textit{main}|}|\\
\end{tabular}
\end{center}
%
The directive |\childdocby| is similar to |\childdocof|
described in \secref{sec:include},
but the subsequent selection of content must be done manually.
To that end, both |\ifchilddoc| and |\ifchilddocmanual|
will be true upon processing of a part,
and the name of the part is stored in |\childdocname|.
Note that |\jobname| will be set to the filename of the current part
so that each part receives an individual |.aux| file
that does not interfere with the |.aux| file(s) of the main document.
This behaviour can be altered by the alternative form
|\childdocby[*]{|\textit{main}|}| (with a non-empty optional argument)
which uses the |.aux| file of the main document
by setting |\jobname| to \textit{main}.

%%%%%%%%%%%%%%%%%%%%%%%%%%%%%%%%%%%%%%%%%%%%%%%%%%%%%%%%%%%%%%%%%%%%%%%%%%%%%%%%
\subsection{Driver Development}
\label{sec:driver}

The \textsf{childdoc} mechanism can also be use for the development
of definition files such as \LaTeX{} styles or classes.
This case differs from the above setup with multiple parts
included by |\include| in that no |\includeonly| should be invoked.
This can be achieved by starting the include file
(before |\ProvidesPackage|) with:
%
\begin{center}
\begin{tabular}{l}
|\input{childdoc.def}|\\
|\childdocforward{|\textit{main}|}|\\
\end{tabular}
\end{center}
%
or alternatively with:
%
\begin{center}
\begin{tabular}{l}
|\input{childdoc.def}|\\
|\childdocby{|\textit{main}|}|\\
\end{tabular}
\end{center}
%
Both forms have slightly different effects as described above.
The main file is prepared as usual, see \secref{sec:include}.

%%%%%%%%%%%%%%%%%%%%%%%%%%%%%%%%%%%%%%%%%%%%%%%%%%%%%%%%%%%%%%%%%%%%%%%%%%%%%%%%
\subsection{Legacy Detection}
\label{sec:detection}

The directive |\childdocmain| in the main file can detect
whether the complete document or merely a child is to be compiled
even without using the directive |\childdocof|.
This method is deprecated because it is less robust
and there is no compelling reason to use it;
it is merely provided for backward compatibility
and it may be removed in future versions.

If the detection mechanism is to be used,
it is mandatory to correctly specify
the filename of the main file as the argument of |\childdocmain|:
%
\begin{center}
\begin{tabular}{l}
|\input{childdoc.def}|\\
|\childdocmain{|\textit{main}|}|\\
\end{tabular}
\end{center}
%
If |\jobname| does not match the argument \textit{main} of |\childdocmain|,
it is assumed that |\jobname| points to the child file to be compiled.
When using |\childdocmain| with the main file specified as argument,
it suffices to start a child file
with just |\input{|\textit{main}|}|
without loading of the package and using |\childdocof|.
If instead all processing is done
with the appropriate \textsf{childdoc} directives,
the argument of \textit{main} of |\childdocmain| can be empty.

An alternative version of the command line processing described
in \secref{sec:commandline} using the detection mechanism reads:
%
\begin{center}
|... -jobname "|\textit{target}|" "|[\textit{flags}]%
[|\def\jobname{|\textit{dest}|}|]|\input{|\textit{main}|}"|
\end{center}

%%%%%%%%%%%%%%%%%%%%%%%%%%%%%%%%%%%%%%%%%%%%%%%%%%%%%%%%%%%%%%%%%%%%%%%%%%%%%%%%
\subsection{Manual Code}
\label{sec:manual}

In case one cannot be certain whether the definitions file |childdoc.def|
is installed on the target \TeX{} distribution
and one prefers not to ship it,
it is conceivable to paste a few relevant commands into the sources.

To that end, drop all statements |\input{childdoc.def}|
and perform the replacements as outlined below.
Instead of |\childdocmain{|\textit{main}|}| add the following code
to the top of the main file:
%
\begin{center}
\begin{tabular}{l}
|\||ifdefined\childdocname\endinput\||fi\newif\ifchilddoc|\\
|\edef\childdocname{\scantokens\expandafter{\jobname\noexpand}}|\\
|\def\childdocmain{|\textit{main}|}\||ifx\childdocmain\childdocname\||else|\\
|\childdoctrue\includeonly{\childdocname}\let\jobname\childdocmain\||fi|\\
\end{tabular}
\end{center}
%
Instead of |\childdocof{|\textit{main}|}| just include the main file
at the top of each child file:
%
\begin{center}
|\input{|\textit{main}|}|
\end{center}
%
A simple redirection |\childdocforward{|\textit{dest}|}| is achieved by:
%
\begin{center}
|\def\jobname{|\textit{dest}|}\input{\jobname}|
\end{center}
%
The redirection with prefix
|\childdocforwardprefix[|\textit{prefix}|]{|\textit{dest}|}|
is accomplished by:
%
\begin{center}
\begin{tabular}{l}
|{\edef\jobname{\scantokens\expandafter{\jobname\noexpand}}|\\
|\def\redirectjob |\textit{prefix}|#1~~~{\gdef\jobname{|\textit{dest}|#1}}|\\
|\expandafter\redirectjob\jobname~~~}\input{\jobname}|
\end{tabular}
\end{center}

In an alternative approach,
child documents can be compiled by a specific command line
without additional code or specific definitions:
%
\begin{center}
|... -jobname "|\textit{target}|" "|[\textit{flags}]%
|\includeonly{|\textit{dest}|}\input{|\textit{main}|}"|
\end{center}
%

%%%%%%%%%%%%%%%%%%%%%%%%%%%%%%%%%%%%%%%%%%%%%%%%%%%%%%%%%%%%%%%%%%%%%%%%%%%%%%%%
%%%%%%%%%%%%%%%%%%%%%%%%%%%%%%%%%%%%%%%%%%%%%%%%%%%%%%%%%%%%%%%%%%%%%%%%%%%%%%%%
\section{Information}

%%%%%%%%%%%%%%%%%%%%%%%%%%%%%%%%%%%%%%%%%%%%%%%%%%%%%%%%%%%%%%%%%%%%%%%%%%%%%%%%
\subsection{Copyright}

Copyright \copyright{} 2017--2018 Niklas Beisert

This work may be distributed and/or modified under the
conditions of the \LaTeX{} Project Public License, either version 1.3
of this license or (at your option) any later version.
The latest version of this license is in
  \url{http://www.latex-project.org/lppl.txt}
and version 1.3 or later is part of all distributions of \LaTeX{}
version 2005/12/01 or later.

This work has the LPPL maintenance status `maintained'.

The Current Maintainer of this work is Niklas Beisert.

This work consists of the files |README.txt|, |childdoc.ins| and |childdoc.dtx|
as well as the derived files |childdoc.def|, |cdocsamp.tex|
with |cdocsch1.tex|, |cdocsch2.tex|, |cdocspt3.tex|, |cdocspt4.tex|,
|cdocsdrf.tex|, |cdocsfn1.tex|, |cdocsfn2.tex|
as well as |childdoc.pdf|.

%%%%%%%%%%%%%%%%%%%%%%%%%%%%%%%%%%%%%%%%%%%%%%%%%%%%%%%%%%%%%%%%%%%%%%%%%%%%%%%%
\subsection{Files and Installation}

The package consists of the files:
%
\begin{center}
\begin{tabular}{ll}
    |README.txt|   & readme file \\
    |childdoc.ins| & installation file \\
    |childdoc.dtx| & source file \\
    |childdoc.def| & definition file \\
    |cdocsamp.tex| & sample main file \\
    |cdocsch1.tex| & sample include file \\
    |cdocsch2.tex| & sample include file \\
    |cdocspt3.tex| & sample part file \\
    |cdocspt4.tex| & sample part file \\
    |cdocsdrf.tex| & sample redirection file \\
    |cdocsfn1.tex| & sample redirection file \\
    |cdocsfn2.tex| & sample redirection file \\
    |childdoc.pdf| & manual
\end{tabular}
\end{center}
%
The distribution consists of the files
|README.txt|, |childdoc.ins| and |childdoc.dtx|.
%
\begin{itemize}
\item
Run (pdf)\LaTeX{} on |childdoc.dtx|
to compile the manual |childdoc.pdf| (this file).
\item
Run \LaTeX{} on |childdoc.ins| to create the definitions file |childdoc.def|
and the sample |cdocsamp.tex| with include files
|cdocsch1.tex|, |cdocsch2.tex|, |cdocspt3.tex|, |cdocspt4.tex|,
|cdocsdrf.tex|, |cdocsfn1.tex|, |cdocsfn2.tex|.
Then copy the file |childdoc.def| to an appropriate directory of your \LaTeX{}
distribution, e.g.\ \textit{texmf-root}|/tex/latex/childdoc|.
\end{itemize}

%%%%%%%%%%%%%%%%%%%%%%%%%%%%%%%%%%%%%%%%%%%%%%%%%%%%%%%%%%%%%%%%%%%%%%%%%%%%%%%%
\subsection{Related CTAN Packages}

There are several other packages which offer a similar functionality:
%
\begin{itemize}
\item
The packages
\href{http://ctan.org/pkg/docmute}{\textsf{docmute}},
\href{http://ctan.org/pkg/includex}{\textsf{includex}} and
\href{http://ctan.org/pkg/standalone}{\textsf{standalone}}
provide commands to include only the document body of
a child file thus allowing both files to be compiled individually.
\item
The packages \href{http://ctan.org/pkg/subdocs}{\textsf{subdocs}}
and \href{http://ctan.org/pkg/subfiles}{\textsf{subfiles}}
provide structures in which the main and child documents can be
encapsulated and allowing them to be compiled individually.
The inclusion mechanism is different from the conventional |\include|.
\item
The package \href{http://ctan.org/pkg/combine}{\textsf{combine}}
is an elaborate solution to combine several documents into one.
\end{itemize}
%
See also the CTAN topic \href{http://ctan.org/topic/subdocs}{\textsf{subdocs}}
for further related packages.
The present package differs from the above solutions in that
a document structure constructed with the conventional |\include| mechanism
just needs two extra commands at the top of every file
such that all constituent files can be compiled individually.

%%%%%%%%%%%%%%%%%%%%%%%%%%%%%%%%%%%%%%%%%%%%%%%%%%%%%%%%%%%%%%%%%%%%%%%%%%%%%%%%
%\subsection{Feature Suggestions}
%
%The following is a list of features which may be useful for future
%versions of this package:
%%
%\begin{itemize}
%\item
%\ldots
%\end{itemize}

%%%%%%%%%%%%%%%%%%%%%%%%%%%%%%%%%%%%%%%%%%%%%%%%%%%%%%%%%%%%%%%%%%%%%%%%%%%%%%%%
\subsection{Revision History}

%%%%%%%%%%%%%%%%%%%%%%%%%%%%%%%%%%%%%%%%
\paragraph{v2.0:} 2018/12/30

\begin{itemize}
\item
immediate forward processing
\item
added |\childdocby| mechanism
\item
manual restructured
\end{itemize}

%%%%%%%%%%%%%%%%%%%%%%%%%%%%%%%%%%%%%%%%
\paragraph{v1.6:} 2018/01/17

\begin{itemize}
\item
application for development of include files
\item
corrections to manual
\end{itemize}

%%%%%%%%%%%%%%%%%%%%%%%%%%%%%%%%%%%%%%%%
\paragraph{v1.5:} 2017/05/21

\begin{itemize}
\item
more complete structuring introduced
\item
|\childdocof| introduced
\item
|\childdoc| renamed to |\childdocmain|
\item
|\childredirect| renamed to |\childdocforward| and |\childdocforwardprefix|
and functionality expanded
\end{itemize}

%%%%%%%%%%%%%%%%%%%%%%%%%%%%%%%%%%%%%%%%
\paragraph{v1.0:} 2017/04/27

\begin{itemize}
\item
manual and install package
\item
first version published on CTAN
\end{itemize}

%%%%%%%%%%%%%%%%%%%%%%%%%%%%%%%%%%%%%%%%
\paragraph{v0.6:} 2017/04/26

\begin{itemize}
\item
redirection mechanism added
\end{itemize}

%%%%%%%%%%%%%%%%%%%%%%%%%%%%%%%%%%%%%%%%
\paragraph{v0.5:} 2017/04/26

\begin{itemize}
\item
functionality in definition file
\end{itemize}


%%%%%%%%%%%%%%%%%%%%%%%%%%%%%%%%%%%%%%%%%%%%%%%%%%%%%%%%%%%%%%%%%%%%%%%%%%%%%%%%
%%%%%%%%%%%%%%%%%%%%%%%%%%%%%%%%%%%%%%%%%%%%%%%%%%%%%%%%%%%%%%%%%%%%%%%%%%%%%%%%
%%%%%%%%%%%%%%%%%%%%%%%%%%%%%%%%%%%%%%%%%%%%%%%%%%%%%%%%%%%%%%%%%%%%%%%%%%%%%%%%
\appendix

\settowidth\MacroIndent{\rmfamily\scriptsize 000\ }

 \DocInput{childdoc.dtx}

\end{document}
%</driver>
% \fi
%
% %%%%%%%%%%%%%%%%%%%%%%%%%%%%%%%%%%%%%%%%%%%%%%%%%%%%%%%%%%%%%%%%%%%%%%%%%%%%%%
% %%%%%%%%%%%%%%%%%%%%%%%%%%%%%%%%%%%%%%%%%%%%%%%%%%%%%%%%%%%%%%%%%%%%%%%%%%%%%%
% \section{Sample}
%\iffalse
%<*samplemain>
%\fi
%
% The following presents a sample document
% with two chapters, two parts, a title page,
% a compile flag as well as three forwarding files to set the flag.
% It consists of eight |.tex| files:
% \begin{center}
% \begin{tabular}{ll}
% |cdocsamp.tex|&main file\\
% |cdocsch1.tex|&include file for chapter 1\\
% |cdocsch2.tex|&include file for chapter 2\\
% |cdocspt3.tex|&include file for part 3\\
% |cdocspt4.tex|&include file for part 4\\
% |cdocsdrf.tex|&forwarding file for main file in draft mode\\
% |cdocsfi1.tex|&forwarding file for final version of chapter 1\\
% |cdocsfi2.tex|&forwarding file for final version of chapter 2\\
% \end{tabular}
% \end{center}
% Each of the eight files can be compiled directly by the \LaTeX{} compiler.
%
% %%%%%%%%%%%%%%%%%%%%%%%%%%%%%%%%%%%%%%
% \paragraph{Main File.}
%
% The main file is called |cdocsamp.tex|.
%
% Load the \textsf{childdoc} definitions and
% declare the filename for the main document:
%    \begin{macrocode}
\input{childdoc.def}
\childdocmain{}
%    \end{macrocode}

% Optional override for |\version| flag:
%    \begin{macrocode}
%%\ifchilddoc\else\providecommand{\version}{draft}\fi
%    \end{macrocode}

% Define the default values for the |\version| flag
% (|final| for the main file and |draft| for childs):
%    \begin{macrocode}
\ifchilddoc
\providecommand{\version}{draft}
\else
\providecommand{\version}{final}
\fi
%    \end{macrocode}

% Load the standard document class:
%    \begin{macrocode}
\documentclass[12pt]{article}
%    \end{macrocode}

% Start the document body:
%    \begin{macrocode}
\begin{document}
%    \end{macrocode}

% Declare a title page.
% Print title, part of document being processed and version flag:
%    \begin{macrocode}
\addtocounter{page}{-1}
\begin{center}
{\LARGE\bfseries{}childdoc example\par}
\vspace{1cm}
\ifchilddoc
\ifchilddocmanual part\else chapter\fi:
`\childdocname' of `\childdocjob'\par
\else
main document: `\childdocjob'\par
\fi
version: \version\par
\end{center}
\newpage
%    \end{macrocode}

% Manually include selected file,
% otherwise process as usual:
%    \begin{macrocode}
\ifchilddocmanual
\section*{part `\childdocname'}
\input{\childdocname}
\else
%    \end{macrocode}

% Include the two chapters:
%    \begin{macrocode}
\include{cdocsch1}
\include{cdocsch2}
%    \end{macrocode}

% Include the two parts unless only chapters should be displayed:
%    \begin{macrocode}
\ifchilddoc\else
\section{part three}
\input{cdocspt3}
\section{part four}
\input{cdocspt4}
\fi
%    \end{macrocode}

% Process as usual until here:
%    \begin{macrocode}
\fi
%    \end{macrocode}

% End of document body:
%    \begin{macrocode}
\end{document}
%    \end{macrocode}
%\iffalse
%</samplemain>
%\fi
%
% %%%%%%%%%%%%%%%%%%%%%%%%%%%%%%%%%%%%%%
% \paragraph{Chapter Include Files.}
%
% The include files are called |cdocsch1.tex| and |cdocsch2.tex|.
%
%\iffalse
%<*samplechap1|samplechap2>
%\fi

% Optional override for |\version| flag:
%    \begin{macrocode}
%%\providecommand{\version}{final}
%    \end{macrocode}

% Include the main document:
%    \begin{macrocode}
\input{childdoc.def}
\childdocof{cdocsamp}
%    \end{macrocode}

%\iffalse
%</samplechap1|samplechap2>
%\fi
%
%\iffalse
%<*samplechap1>
%\fi
% Some text for chapter 1:
%    \begin{macrocode}
\section{one}
some text in chapter one
%    \end{macrocode}

%\iffalse
%</samplechap1>
%\fi
% Some text for chapter 2:
%\iffalse
%<*samplechap2>
%\fi
%    \begin{macrocode}
\section{two}
more text in chapter two
%    \end{macrocode}

%\iffalse
%</samplechap2>
%\fi
%
% %%%%%%%%%%%%%%%%%%%%%%%%%%%%%%%%%%%%%%
% \paragraph{Part Include Files.}
%
% The include files are called |cdocspt3.tex| and |cdocspt4.tex|.
%
%\iffalse
%<*samplepart3|samplepart4>
%\fi

% Optional override for |\version| flag:
%    \begin{macrocode}
%%\providecommand{\version}{final}
%    \end{macrocode}

% Include the main document:
%    \begin{macrocode}
\input{childdoc.def}
\childdocby{cdocsamp}
%    \end{macrocode}

%\iffalse
%</samplepart3|samplepart4>
%\fi
%
%\iffalse
%<*samplepart3>
%\fi
% Some text for part 3:
%    \begin{macrocode}
some text in part three
%    \end{macrocode}

%\iffalse
%</samplepart3>
%\fi
% Some text for part 4:
%\iffalse
%<*samplepart4>
%\fi
%    \begin{macrocode}
more text in part four
%    \end{macrocode}

%\iffalse
%</samplepart4>
%\fi
%
% %%%%%%%%%%%%%%%%%%%%%%%%%%%%%%%%%%%%%%
% \paragraph{Forwarding for a Complete Draft.}
%
% The following forwarding file |cdocsdrf.tex|
% compiles the main document in draft mode:
%\iffalse
%<*sampledraft>
%\fi
%    \begin{macrocode}
\def\version{draft}
\input{childdoc.def}
\childdocforward{cdocsamp}
%    \end{macrocode}

%\iffalse
%</sampledraft>
%\fi
%
% %%%%%%%%%%%%%%%%%%%%%%%%%%%%%%%%%%%%%%
% \paragraph{Forwarding for Final Version of the Chapters.}
%
% The following forwarding files |cdocsfn1.tex| and |cdocsfn2.tex|
% (with identical content)
% compile the final versions of the child documents
% |cdocsch1.tex| and |cdocsch2.tex|, respectively:
%\iffalse
%<*samplefinal>
%\fi
%    \begin{macrocode}
\def\version{final}
\input{childdoc.def}
\childdocforwardprefix[cdocsamp]{cdocsfn}{cdocsch}
%    \end{macrocode}

%\iffalse
%</samplefinal>
%\fi
%
% %%%%%%%%%%%%%%%%%%%%%%%%%%%%%%%%%%%%%%
% \paragraph{Command Line Processing.}
%
% The following three command lines generate the output files
% |cdocscld|, |cdocscl1| and |cdocscl2|
% which should be identical to
% |cdocsdrf|, |cdocsch1| and |cdocsfn2|, respectively:
% \begin{center}
% \begin{tabular}{l}
% |latex -jobname cdocscld \|\\
% |  "\def\version{draft}\input{childdoc.def}\childdocforward{cdocsamp}"|\\
% |latex -jobname cdocscl1 \|\\
% |  "\input{childdoc.def}\childdocforward[cdocsamp]{cdocsch1}"|\\
% |latex -jobname cdocscl2 \|\\
% |  "\def\version{final}\input{childdoc.def}\childdocforward{cdocsch2}"|
% \end{tabular}
% \end{center}
% Note that the trailing backslash on each first line
% merely continues the input to the second line
% (for convenient cut ant paste).
% Furthermore, the command |latex| can be replaced by any
% of its alternative versions such as |pdflatex|.
%
% %%%%%%%%%%%%%%%%%%%%%%%%%%%%%%%%%%%%%%%%%%%%%%%%%%%%%%%%%%%%%%%%%%%%%%%%%%%%%%
% %%%%%%%%%%%%%%%%%%%%%%%%%%%%%%%%%%%%%%%%%%%%%%%%%%%%%%%%%%%%%%%%%%%%%%%%%%%%%%
% \section{Implementation}
%\iffalse
%<*package>
%\fi
%
% This section describes the definitions file |childdoc.def|.

% The definitions cannot be loaded using |\usepackage| or |\RequirePackage|
% which has a mechanism to prevent loading a style file more than once.
% When loading the definitions by means of |\input|
% multiple instances have to be prevented manually:
%\iffalse
%This code needs to be before the `\ProvidesFile' directive
%which is defined at the beginning of this file.
%Therefore it is also placed there and commented out here.
%</package>
%<*discard>
%\fi
%    \begin{macrocode}
\ifdefined\childdocmain\endinput\fi
%    \end{macrocode}
%\iffalse
%</discard>
%<*package>
%\fi
%
% \macro{\ifchilddoc}
% \macro{\ifchilddocmanual}
% The conditional |\ifchilddoc| tells whether a
% child (true) or main (false) document is being compiled.
% The conditional |\ifchilddocmanual| tells whether
% the |\includeonly| mechanism is used (false) or
% the selection of child files must be performed manually (true).
% The definitions initialise to false:
%    \begin{macrocode}
\newif\ifchilddoc
\newif\ifchilddocmanual
%    \end{macrocode}

% \macro{\childdocname}
% \macro{\childdocjob}
% The macro |\childdocname| stores the name of the main document
% to be compiled. The macro |\childdocjob| stores the name of
% the document on which the \LaTeX{} compiler was originally invoked.
% The content of |\jobname| cannot be compared
% to filenames specified in the source due to different catcodes.
% The following code rescans |\jobname|, stores the result
% in |\childdocname| and saves a copy in |\childdocjob|:
%    \begin{macrocode}
\edef\childdocname{\scantokens\expandafter{\jobname\noexpand}}
\let\childdocjob\childdocname
%    \end{macrocode}

% \macro{\childdocdisable}
% The macro |\childdocdisable| prevents the main file
% from being processed more than once.
% At this stage, the main document command |\childdocmain|
% is assumed to be called once again where it should do nothing.
% Any subsequent call to it should prevent
% a secondary processing of the main document
% It overwrites the forwarding commands
% |\childdocof| and |\childdocforward|
% with empty macros to prevent further inclusions of the main document:
%    \begin{macrocode}
\newcommand{\childdocdisable}
{
  \renewcommand{\childdocmain}[1]{\renewcommand{\childdocmain}[1]{\endinput}}
  \renewcommand{\childdocof}[1]{}
  \renewcommand{\childdocby}[2][]{}
  \renewcommand{\childdocforward}[2][]{}
  \renewcommand{\childdocdisable}{}
}
%    \end{macrocode}

% \macro{\childdocmain}
% The macro |\childdocmain| is to be called at the top of the main file
% with nothing or the main filename (without extension) as argument.
% First, it breaks loops.
% If the argument is not empty and does not match |\childdocname|
% (which is set by the first inclusion of |childdoc.def|),
% |\ifchilddoc| is set to true, |\includeonly| is applied to the child file
% and |\jobname| is set to the main file
% (for proper handling of |.aux| files):
%    \begin{macrocode}
\newcommand{\childdocmain}[1]
{
  \childdocdisable\childdocmain{}
  \if?#1?\else
    \begingroup
      \def\childdoctmp{#1}
      \ifx\childdoctmp\childdocname
        \def\childdoctmp{}
      \else
        \def\childdoctmp
        {
          \childdoctrue
          \includeonly{\childdocname}
          \def\childdocjob{#1}
          \def\jobname{#1}
        }
      \fi
      \expandafter
    \endgroup
    \childdoctmp
  \fi
}
%    \end{macrocode}

% \macro{\childdocof}
% The command |\childdocof| redirects
% compilation to the main file |#1|.
%    \begin{macrocode}
\newcommand{\childdocof}[1]
{
  \childdocdisable
  \childdoctrue
  \includeonly{\childdocname}
  \def\jobname{#1}
  \def\childdocjob{#1}
  \input{#1}
}
%    \end{macrocode}

% \macro{\childdocby}
% The command |\childdocby| ....
%    \begin{macrocode}
\newcommand{\childdocby}[2][]
{
  \childdocdisable
  \childdoctrue
  \childdocmanualtrue
  \if?#1?\else
    \def\jobname{#2}
  \fi
  \def\childdocjob{#2}
  \input{#2}
  \endinput
}
%    \end{macrocode}

% \macro{\childdocforward}
% The command |\childdocforward| redirects
% compilation to the main file or
% (if the optional argument is given) a child file.
% Parameters are set as if the main file
% or a child file starting with |\childdocof| was compiled.
% Then compilation is handed over to the main file:
%    \begin{macrocode}
\newcommand{\childdocforward}[2][]
{
  \begingroup
    \if?#1?
      \def\childdoctmp
      {
        \def\childdocname{#2}
        \def\childdocjob{#2}
        \def\jobname{#2}
        \input{#2}
        \endinput
      }
    \else
      \def\childdoctmp
      {
        \childdocdisable
        \def\childdocname{#2}
        \childdoctrue
        \includeonly{#2}
        \def\childdocjob{#1}
        \def\jobname{#1}
        \input{#1}
        \endinput
      }
    \fi
    \expandafter
  \endgroup
  \childdoctmp
}
%    \end{macrocode}

% \macro{\childdocforwardprefix}
% The command |\childdocforwardprefix| redirects
% compilation to the main or a child file by means of a pattern.
% The prefix |#1| in the current filename is replaced by |#2|
% and the suffix of the current filename is kept
% (it is assumed that the filename does not contain the substring `|~~~|'
% which is used as a delimiter).
% Compilation is handed over to the new file by |\childdocforward|:
%    \begin{macrocode}
\newcommand{\childdocforwardprefix}[3][]
{
  \begingroup
    \def\childdocextract #2##1~~~{\def\childdoctmp{\childdocforward[#1]{#3##1}}}
    \expandafter\childdocextract\childdocname~~~
    \expandafter
  \endgroup
  \childdoctmp
}
%    \end{macrocode}

% \macro{\childdoc}
% The deprecated macro |\childdoc| is a legacy version of |\childdocmain|:
%    \begin{macrocode}
\newcommand{\childdoc}{\childdocmain}
%    \end{macrocode}

% \macro{\childdocredirect}
% The deprecated macro |\childdocredirect| is a legacy version
% of |\childdocforward| and |\childdocforwardprefix|:
%    \begin{macrocode}
\newcommand{\childdocredirect}[2][]
{
  \begingroup
    \if?#1?
      \def\childdoctmp{\childdocforward{#2}}
    \else
      \def\childdoctmp{\childdocforwardprefix{#1}{#2}}
    \fi
    \expandafter
  \endgroup
  \childdoctmp
}
%    \end{macrocode}

%\iffalse
%</package>
%\fi
%
\endinput

\childdocforward{cdocsamp}
%    \end{macrocode}

%\iffalse
%</sampledraft>
%\fi
%
% %%%%%%%%%%%%%%%%%%%%%%%%%%%%%%%%%%%%%%
% \paragraph{Forwarding for Final Version of the Chapters.}
%
% The following forwarding files |cdocsfn1.tex| and |cdocsfn2.tex|
% (with identical content)
% compile the final versions of the child documents
% |cdocsch1.tex| and |cdocsch2.tex|, respectively:
%\iffalse
%<*samplefinal>
%\fi
%    \begin{macrocode}
\def\version{final}
% \iffalse
%
% childdoc.dtx Copyright (C) 2017-2018 Niklas Beisert
%
% This work may be distributed and/or modified under the
% conditions of the LaTeX Project Public License, either version 1.3
% of this license or (at your option) any later version.
% The latest version of this license is in
%   http://www.latex-project.org/lppl.txt
% and version 1.3 or later is part of all distributions of LaTeX
% version 2005/12/01 or later.
%
% This work has the LPPL maintenance status `maintained'.
%
% The Current Maintainer of this work is Niklas Beisert.
%
% This work consists of the files childdoc.dtx and childdoc.ins
% and the derived files childdoc.def and cdocsamp.tex with
% cdocsch1.tex, cdocsch2.tex, cdocsdrf.tex, cdocsfn1.tex, cdocsfn2.tex.
%
%<package>\ifdefined\childdocmain\endinput\fi
%<package>\ProvidesFile{childdoc.def}[2018/12/30 v2.0 child document driver]
%<samplemain>\ProvidesFile{cdocsamp.tex}[2018/12/30 v2.0 sample for childdoc]
%<*driver>
%\ProvidesFile{childdoc.drv}[2018/12/30 v2.0 childdoc reference manual file]
\PassOptionsToClass{10pt,a4paper}{article}
\documentclass{ltxdoc}

\usepackage[margin=35mm]{geometry}
\usepackage{hyperref}
\usepackage{hyperxmp}
\usepackage[usenames]{color}

\hypersetup{colorlinks=true}
\hypersetup{pdfstartview=FitH}
\hypersetup{pdfpagemode=UseNone}
\hypersetup{pdfsource={}}
\hypersetup{pdflang={en-UK}}
\hypersetup{pdfcopyright={Copyright 2017-2018 Niklas Beisert.
  This work may be distributed and/or modified under the
  conditions of the LaTeX Project Public License, either version 1.3
  of this license or (at your option) any later version.}}
\hypersetup{pdflicenseurl={http://www.latex-project.org/lppl.txt}}
\hypersetup{pdfcontactaddress={ETH Zurich, ITP, HIT K,
  Wolfgang-Pauli-Strasse 27}}
\hypersetup{pdfcontactpostcode={8093}}
\hypersetup{pdfcontactcity={Zurich}}
\hypersetup{pdfcontactcountry={Switzerland}}
\hypersetup{pdfcontactemail={nbeisert@itp.phys.ethz.ch}}
\hypersetup{pdfcontacturl={http://people.phys.ethz.ch/\xmptilde nbeisert/}}

\newcommand{\secref}[1]{\hyperref[#1]{section \ref*{#1}}}

\parskip1ex
\parindent0pt
\let\olditemize\itemize
\def\itemize{\olditemize\parskip0pt}

\begin{document}

\title{The \textsf{childdoc} Package}
\hypersetup{pdftitle={The childdoc Package}}
\author{Niklas Beisert\\[2ex]
  Institut f\"ur Theoretische Physik\\
  Eidgen\"ossische Technische Hochschule Z\"urich\\
  Wolfgang-Pauli-Strasse 27, 8093 Z\"urich, Switzerland\\[1ex]
  \href{mailto:nbeisert@itp.phys.ethz.ch}
  {\texttt{nbeisert@itp.phys.ethz.ch}}}
\hypersetup{pdfauthor={Niklas Beisert}}
\hypersetup{pdfsubject={Manual for the LaTeX2e Package childdoc}}
\date{30 December 2018, \textsf{v2.0}}
\maketitle

\begin{abstract}\noindent
\textsf{childdoc} is a \LaTeXe{} package
that enables the direct compilation
of document sections included by |\include|
to individual files.
\end{abstract}

\begingroup
\parskip0ex
\tableofcontents
\endgroup

%%%%%%%%%%%%%%%%%%%%%%%%%%%%%%%%%%%%%%%%%%%%%%%%%%%%%%%%%%%%%%%%%%%%%%%%%%%%%%%%
%%%%%%%%%%%%%%%%%%%%%%%%%%%%%%%%%%%%%%%%%%%%%%%%%%%%%%%%%%%%%%%%%%%%%%%%%%%%%%%%
\section{Introduction}

\LaTeX{} provides a mechanism to structure a large document (such as a book)
into a main file and several child files (containing the chapters)
using the |\include| command.
This mechanism is beneficial for documents
which span hundreds of pages in order to
make the source file(s) more manageable.
Moreover, compilation can be restricted to
selected child files by means of the |\includeonly| command.
The latter feature can be used to reduce the compilation time while editing
(this was significantly more useful in the earlier days of \LaTeX{})
or to generate a smaller document which is easier to navigate.
Another application of |\includeonly| is to generate
documents consisting of selected parts of the complete document.

However, there are a few drawbacks of the plain |\include| mechanism:
\begin{itemize}
\item
The child files cannot be compiled on their own,
they can only be compiled via the main file.
A naive editing environment
(such as a text editor with an option
to have the current file processed by \LaTeX)
may require one to switch to the main file before compiling;
attempting to compile the child file produces errors.
\item
The main file must be modified (each time)
to adjust the |\includeonly| command
to the present needs. This easily leaves the main file in a messy state.
\item
The generated document will always carry the filename
of the main document. This is inconvenient if
several child files are to be compiled and
to be kept for distribution.
\end{itemize}

The present package provides a simple interface
to make child files individually compilable by \LaTeX{}.
Compiling a child file then has the same effect as compiling
the main file with an |\includeonly| command
to select the appropriate child.
Moreover the generated document will carry the name of the child
rather than the main file.
This resolves all three above issues.

This feature is meant to make the editing of books,
thesis documents and lecture notes somewhat more convenient.
However, the package can also be used efficiently for
composing a series of documents (such as exercise sheets)
which are typically distributed individually.
It then assists the author in generating the individual documents
(potentially in different versions)
as well as a document containing the collected series.
Another application is in developing style files
or other kinds of included material
where compilation of the style file could redirect
to a sample or test file.

%%%%%%%%%%%%%%%%%%%%%%%%%%%%%%%%%%%%%%%%%%%%%%%%%%%%%%%%%%%%%%%%%%%%%%%%%%%%%%%%
%%%%%%%%%%%%%%%%%%%%%%%%%%%%%%%%%%%%%%%%%%%%%%%%%%%%%%%%%%%%%%%%%%%%%%%%%%%%%%%%
\section{Usage}

First of all, the package \textsf{childdoc} is \emph{not} a standard
\LaTeXe{} |.sty| style file! Therefore it needs to be invoked in
a non-standard way.

%%%%%%%%%%%%%%%%%%%%%%%%%%%%%%%%%%%%%%%%%%%%%%%%%%%%%%%%%%%%%%%%%%%%%%%%%%%%%%%%
\subsection{Included Files}
\label{sec:include}

%%%%%%%%%%%%%%%%%%%%%%%%%%%%%%%%%%%%%%%%
\DescribeMacro{\childdocmain}
To use the package, add the commands
\begin{center}
\begin{tabular}{l}
|\input{childdoc.def}|\\
|\childdocmain{}|\\
\end{tabular}
\end{center}
at the very top of the main \LaTeX{} file,
in particular \emph{before} the |\documentclass| statement!
The argument of |\childdocmain| should be left empty
(but it must be present).

%%%%%%%%%%%%%%%%%%%%%%%%%%%%%%%%%%%%%%%%
\DescribeMacro{\childdocof}
Furthermore, add the commands
\begin{center}
\begin{tabular}{l}
|\input{childdoc.def}|\\
|\childdocof{|\textit{main}|}|\\
\end{tabular}
\end{center}
at the top of every child file \textit{child}
which is included by |\include{|\textit{child}|}|
from within the main file
(or at least for those files to be compiled individually).
The argument \textit{main} must be the filename of the main file.

There are a couple of
considerations in setting up the main and child documents:

%%%%%%%%%%%%%%%%%%%%%%%%%%%%%%%%%%%%%%%%
\paragraph{Restrictions.}

Please note the following restrictions:
\begin{itemize}
\item
|\childdocmain| must be called with one argument \textit{main}
to ensure compatibility with earlier version of the package.
It must either be empty (|\childdocmain{}|)
or precisely match the filename of the main file in which it is specified.
See \secref{sec:detection} for further information.
\item
The filename \textit{main} must be specified without the |.tex| extension.
\item
The filename \textit{main} is case sensitive
(even in case-insensitive file systems)
due to internal string comparison.
\item
The argument \textit{main} should be fully expanded, it cannot be a macro.
\item
Subdirectories and special characters should be avoided in filenames.
\item
The command |\childdocmain{|\textit{main}|}| must be followed by a whitespace.
It should not be followed immediately by another command
or by a comment mark `|%|'.
This is because the \TeX{} parser reads the token immediately following
the argument of |\childdocmain| and puts it
at the beginning of every child section;
however, a white\-space is ignored.
\end{itemize}

%%%%%%%%%%%%%%%%%%%%%%%%%%%%%%%%%%%%%%%%
\paragraph{Content of Main File.}

It is advisable to place all content in the child files included by |\include|.
Any output contained in the main file will appear in all child documents
unless suppressed manually;
it cannot be suppressed automatically by the |\includeonly| directive
and thus should normally be avoided.
A method to include some content in the main file
by means of conditional processing is described in \secref{sec:conditional}.

%%%%%%%%%%%%%%%%%%%%%%%%%%%%%%%%%%%%%%%%
\paragraph{Page Numbering.}

When only a part of the document is compiled,
the appropriate numbering of pages
(as well as other status parameters)
is determined from the |.aux| files.
The latter contain information from previous passes.
However this information needs to propagate through
all intermediate child documents.
Therefore the page numbering in child documents may well
be inconsistent until the complete document is compiled at least once.

A useful (if unconventional) way to always ensure a consistent
page numbering is to restart the numbering in each child document
and denote the pages by `\textit{child}|.|\textit{page}'
where \textit{child} represents the chapter/section number of the child file.
This can be achieved by the command
|\numberwithin{page}{|\textit{child}|}|
of the \textsf{amsmath} package
where \textit{child} can be |chapter| or |section|
depending on the chosen structuring.
Alternatively, one can modify the macro |\thepage| appropriately
and reset the counter |page| at the start of each child file.

%%%%%%%%%%%%%%%%%%%%%%%%%%%%%%%%%%%%%%%%%%%%%%%%%%%%%%%%%%%%%%%%%%%%%%%%%%%%%%%%
\subsection{Conditional Processing}
\label{sec:conditional}

The package provides a mechanism to compile different versions
of a document. To customise the versions further some conditional processing
can come in handy to distinguish which version is being compiled.
The package provides two macros to describe the compilation context:

%%%%%%%%%%%%%%%%%%%%%%%%%%%%%%%%%%%%%%%%
\DescribeMacro{\ifchilddoc}
The conditional |\ifchilddoc| distinguishes between the compilation of
child documents and the main document:
%
\begin{center}
|\ifchilddoc |\textit{child-code}| |[|\||else |\textit{main-code}]| \||fi|
\end{center}

%%%%%%%%%%%%%%%%%%%%%%%%%%%%%%%%%%%%%%%%
\DescribeMacro{\childdocname}
\DescribeMacro{\childdocjob}
The macro |\childdocname| contains the filename (without extension)
of the main or child file being processed.
Note that |\childdocjob| will always contain the name of the main file.

%%%%%%%%%%%%%%%%%%%%%%%%%%%%%%%%%%%%%%%%
\paragraph{Title Page.}

Conditional processing can be used to include a title or banner page
in the main document when proper precautions are taken.
Importantly, the code in the main file should ensure that the page counter
(as well as other status parameters which are stored in the |.aux| files)
takes the same value after the conditional processing.
Otherwise the page numbers may take divergent values
depending on which part is compiled.

For example, a title page could be declared by:
%
\begin{center}
\begin{tabular}{l}
|\ifchilddoc\||else|\\
|\addtocounter{page}{-1}|\\
\textit{code for title page}\\
|\newpage|\\
|\||fi|
\end{tabular}
\end{center}
%
A banner page for the child documents can be generated by:
%
\begin{center}
\begin{tabular}{l}
|\ifchilddoc|\\
|\addtocounter{page}{-1}|\\
\textit{code for banner page}\\
|\newpage|\\
|\||fi|
\end{tabular}
\end{center}
%
Here one could write a message such as:
\begin{center}
|This is the part \childdocname{} of \childdocjob{}.|
\end{center}

%%%%%%%%%%%%%%%%%%%%%%%%%%%%%%%%%%%%%%%%%%%%%%%%%%%%%%%%%%%%%%%%%%%%%%%%%%%%%%%%
\subsection{Flags}
\label{sec:flags}

The package makes it easy to generate different versions
of the main or child documents.
To this end compilation flags can be defined
and assigned different default values.
They will be particularly useful in conjunction
with the forwarding mechanism described in \secref{sec:forward}.

For example, it may be useful to have a flag |\version|
which can be set to |draft| or |final|.
The document source will contain some conditional code
depending on the value of |\version|.
Suppose further, the flag should default to |final| for the main file
and to |draft| for child files
which is a natural assignment for editing the document.
This is achieved by placing the following code
in the preamble of the main document
(below the |\childdocmain| directive):
%
\begin{center}
\begin{tabular}{l}
|\ifchilddoc|\\
|\providecommand{\version}{draft}|\\
|\||else|\\
|\providecommand{\version}{final}|\\
|\||fi|
\end{tabular}
\end{center}
%
The definition by |\providecommand| makes sure
that previous definitions are not overwritten.
Further statements |\providecommand{\version}{...}|
can thus be added before the above code to override it.

For the main file, one might add a line
(between |\childdocmain| and the above block)
%
\begin{center}
|%\ifchilddoc\||else\providecommand{\version}{draft}\||fi|
\end{center}
%
which can be uncommented to produce a draft version.
Likewise one can add a line to the very top of a child file
(above the |\childdocof{|\textit{main}|}| directive)
%
\begin{center}
|%\providecommand{\version}{final}|
\end{center}
%
which can be uncommented to produce the final version of this child document.

%%%%%%%%%%%%%%%%%%%%%%%%%%%%%%%%%%%%%%%%%%%%%%%%%%%%%%%%%%%%%%%%%%%%%%%%%%%%%%%%
\subsection{Forwarding}
\label{sec:forward}

Different versions of the main or child documents
using compilation flags as described in \secref{sec:flags}
can be (permanently) stored in different files
for convenient compilation, viewing and distribution.
To this end, the package defines a command
to pass on compilation to a different file:

%%%%%%%%%%%%%%%%%%%%%%%%%%%%%%%%%%%%%%%%
\DescribeMacro{\childdocforward}
The command |\childdocforward| redirects processing to
another source file:
%
\begin{center}
\begin{tabular}{l}
|\input{childdoc.def}|\\
|\childdocforward[|\textit{main}|]{|\textit{dest}|}|\\
\end{tabular}
\end{center}
%
The argument \textit{dest} is the destination file
(without extension).
It should be the main file or one of the child files.
Note that further \textsf{childdoc} directives
such as |\childdocof| and |\childdocforward|
in the indicated file will be processed in this form.
The optional argument \textit{main}
passes on directly to the main file \textit{main}
while pretending to compile the child \textit{dest}.
This form behaves as if \textit{dest}
issues |\childdocof{|\textit{main}|}| right away,
and no further \textsf{childdoc} directives will be processed.

%%%%%%%%%%%%%%%%%%%%%%%%%%%%%%%%%%%%%%%%
\DescribeMacro{\...prefix}
In the alternative form |\childdocforwardprefix|,
%
\begin{center}
\begin{tabular}{l}
|\input{childdoc.def}|\\
|\childdocforwardprefix[|\textit{main}|]{|\textit{prefix}|}{|\textit{dest}|}|
\end{tabular}
\end{center}
%
the destination file is determined by a pattern
depending on the current file:
To make this work, the current file must be called
`{\textit{prefix}\hspace{0.2em}\textit{suffix}}'
with \textit{prefix} matching precisely the argument.
Processing is then passed on to the file
`{\textit{dest}\hspace{0.2em}\textit{suffix}}'.
Surely, the same effect is achieved by
directly specifying the
argument `{\textit{dest}\hspace{0.2em}\textit{suffix}}'
in the first form.
However, that requires to set up a different file
for each child. With the alternative form of the command
all these files can have exactly the same content
which simplifies setting them up and maintaining them.

For example, the following file |draft.tex|
with a compilation flag |\version| as described in \secref{sec:flags}
compiles the main document as a draft:
%
\begin{center}
\begin{tabular}{l}
|\def\version{draft}|\\
|\input{childdoc.def}|\\
|\childdocforward{|\textit{main}|}|
\end{tabular}
\end{center}
%
Likewise, the following files |final|\textit{nn}|.tex|
compile the final version of the child document
|child|\textit{nn}|.tex|:
%
\begin{center}
\begin{tabular}{l}
|\def\version{final}|\\
|\input{childdoc.def}|\\
|\childdocforwardprefix{final}{child}|
\end{tabular}
\end{center}
%

Note that when several versions of a main file and/or of each child file
are to be generated, it may be convenient to set up a |Makefile| or
shell script to automatise the process.

%%%%%%%%%%%%%%%%%%%%%%%%%%%%%%%%%%%%%%%%%%%%%%%%%%%%%%%%%%%%%%%%%%%%%%%%%%%%%%%%
\subsection{Command Line Processing}
\label{sec:commandline}

The effect of redirection files can also be achieved by invoking
the \LaTeX{} compiler with a more elaborate command line.
Most conveniently this should be done as part
of a shell script or a |Makefile|.

When using \textsf{childdoc} in the main file, the following
command lines effectively perform a redirection
(note that depending on the shell being used,
backslashes may have to be doubled: `|\|' $\to$ `|\\|'):
%
\begin{center}
|... -jobname "|\textit{target}|" |\\|"|[\textit{flags}]%
|\input{childdoc.def}\childdocforward[|\textit{main}|]{|\textit{dest}|}"|
\end{center}
%
Here \textit{target} is the name of the output file,
\textit{main} is the name of the main file
and \textit{dest} is the name of the main or child file to be processed
(all filenames without extensions).
The optional argument \textit{main} can be omitted
if \textit{main} matches \textit{dest}.
Optionally, compilation \textit{flags} can be defined via |\def| commands.
This command line makes the \TeX{} engine believe
it is compiling the file \textit{target}
whose content is specified as the latter parameter.
The provided code then forwards the processing to
\textit{main} or \textit{dest} as described in \secref{sec:forward}.

%%%%%%%%%%%%%%%%%%%%%%%%%%%%%%%%%%%%%%%%%%%%%%%%%%%%%%%%%%%%%%%%%%%%%%%%%%%%%%%%
\subsection{Include by Input}
\label{sec:input}

Including child documents by |\include| has some restrictions by design.
Most notably, the content of a child document always occupies
its own set of pages; pages cannot be shared between child documents.
Usually, this behaviour makes perfect sense
because each child document contain an essential part of the document.
However, in some situations it may be desirable to compose
a document from a collection of parts
without having mandatory page breaks between then.
For this case, the package
provides a mechanism to include parts
by |\input| which can also be processed individually.
However, by construction this mechanism
requires manual handling of the content to be output.

%%%%%%%%%%%%%%%%%%%%%%%%%%%%%%%%%%%%%%%%
\DescribeMacro{\ifchilddocmanual}
The main file should be prepared as usual, see \secref{sec:include}.
However, the document body must make a distinction
between processing of an individual part and of the main document, e.g.:
%
\begin{center}
\begin{tabular}{l}
|\ifchilddocmanual|\\
|\input{\childdocname}|\\
|\||else|\\
\textit{document body with }|\input{|\textit{part}|}|\\
|\||fi|
\end{tabular}
\end{center}
%
The conditional |\ifchilddocmanual| is true whenever
a part to be included by |\input| is being compiled,
and the name of the part is stored in |\childdocname|.

%%%%%%%%%%%%%%%%%%%%%%%%%%%%%%%%%%%%%%%%
\DescribeMacro{\childdocby}
Each part to be included by |\input| should start with:
%
\begin{center}
\begin{tabular}{l}
|\input{childdoc.def}|\\
|\childdocby{|\textit{main}|}|\\
\end{tabular}
\end{center}
%
The directive |\childdocby| is similar to |\childdocof|
described in \secref{sec:include},
but the subsequent selection of content must be done manually.
To that end, both |\ifchilddoc| and |\ifchilddocmanual|
will be true upon processing of a part,
and the name of the part is stored in |\childdocname|.
Note that |\jobname| will be set to the filename of the current part
so that each part receives an individual |.aux| file
that does not interfere with the |.aux| file(s) of the main document.
This behaviour can be altered by the alternative form
|\childdocby[*]{|\textit{main}|}| (with a non-empty optional argument)
which uses the |.aux| file of the main document
by setting |\jobname| to \textit{main}.

%%%%%%%%%%%%%%%%%%%%%%%%%%%%%%%%%%%%%%%%%%%%%%%%%%%%%%%%%%%%%%%%%%%%%%%%%%%%%%%%
\subsection{Driver Development}
\label{sec:driver}

The \textsf{childdoc} mechanism can also be use for the development
of definition files such as \LaTeX{} styles or classes.
This case differs from the above setup with multiple parts
included by |\include| in that no |\includeonly| should be invoked.
This can be achieved by starting the include file
(before |\ProvidesPackage|) with:
%
\begin{center}
\begin{tabular}{l}
|\input{childdoc.def}|\\
|\childdocforward{|\textit{main}|}|\\
\end{tabular}
\end{center}
%
or alternatively with:
%
\begin{center}
\begin{tabular}{l}
|\input{childdoc.def}|\\
|\childdocby{|\textit{main}|}|\\
\end{tabular}
\end{center}
%
Both forms have slightly different effects as described above.
The main file is prepared as usual, see \secref{sec:include}.

%%%%%%%%%%%%%%%%%%%%%%%%%%%%%%%%%%%%%%%%%%%%%%%%%%%%%%%%%%%%%%%%%%%%%%%%%%%%%%%%
\subsection{Legacy Detection}
\label{sec:detection}

The directive |\childdocmain| in the main file can detect
whether the complete document or merely a child is to be compiled
even without using the directive |\childdocof|.
This method is deprecated because it is less robust
and there is no compelling reason to use it;
it is merely provided for backward compatibility
and it may be removed in future versions.

If the detection mechanism is to be used,
it is mandatory to correctly specify
the filename of the main file as the argument of |\childdocmain|:
%
\begin{center}
\begin{tabular}{l}
|\input{childdoc.def}|\\
|\childdocmain{|\textit{main}|}|\\
\end{tabular}
\end{center}
%
If |\jobname| does not match the argument \textit{main} of |\childdocmain|,
it is assumed that |\jobname| points to the child file to be compiled.
When using |\childdocmain| with the main file specified as argument,
it suffices to start a child file
with just |\input{|\textit{main}|}|
without loading of the package and using |\childdocof|.
If instead all processing is done
with the appropriate \textsf{childdoc} directives,
the argument of \textit{main} of |\childdocmain| can be empty.

An alternative version of the command line processing described
in \secref{sec:commandline} using the detection mechanism reads:
%
\begin{center}
|... -jobname "|\textit{target}|" "|[\textit{flags}]%
[|\def\jobname{|\textit{dest}|}|]|\input{|\textit{main}|}"|
\end{center}

%%%%%%%%%%%%%%%%%%%%%%%%%%%%%%%%%%%%%%%%%%%%%%%%%%%%%%%%%%%%%%%%%%%%%%%%%%%%%%%%
\subsection{Manual Code}
\label{sec:manual}

In case one cannot be certain whether the definitions file |childdoc.def|
is installed on the target \TeX{} distribution
and one prefers not to ship it,
it is conceivable to paste a few relevant commands into the sources.

To that end, drop all statements |\input{childdoc.def}|
and perform the replacements as outlined below.
Instead of |\childdocmain{|\textit{main}|}| add the following code
to the top of the main file:
%
\begin{center}
\begin{tabular}{l}
|\||ifdefined\childdocname\endinput\||fi\newif\ifchilddoc|\\
|\edef\childdocname{\scantokens\expandafter{\jobname\noexpand}}|\\
|\def\childdocmain{|\textit{main}|}\||ifx\childdocmain\childdocname\||else|\\
|\childdoctrue\includeonly{\childdocname}\let\jobname\childdocmain\||fi|\\
\end{tabular}
\end{center}
%
Instead of |\childdocof{|\textit{main}|}| just include the main file
at the top of each child file:
%
\begin{center}
|\input{|\textit{main}|}|
\end{center}
%
A simple redirection |\childdocforward{|\textit{dest}|}| is achieved by:
%
\begin{center}
|\def\jobname{|\textit{dest}|}\input{\jobname}|
\end{center}
%
The redirection with prefix
|\childdocforwardprefix[|\textit{prefix}|]{|\textit{dest}|}|
is accomplished by:
%
\begin{center}
\begin{tabular}{l}
|{\edef\jobname{\scantokens\expandafter{\jobname\noexpand}}|\\
|\def\redirectjob |\textit{prefix}|#1~~~{\gdef\jobname{|\textit{dest}|#1}}|\\
|\expandafter\redirectjob\jobname~~~}\input{\jobname}|
\end{tabular}
\end{center}

In an alternative approach,
child documents can be compiled by a specific command line
without additional code or specific definitions:
%
\begin{center}
|... -jobname "|\textit{target}|" "|[\textit{flags}]%
|\includeonly{|\textit{dest}|}\input{|\textit{main}|}"|
\end{center}
%

%%%%%%%%%%%%%%%%%%%%%%%%%%%%%%%%%%%%%%%%%%%%%%%%%%%%%%%%%%%%%%%%%%%%%%%%%%%%%%%%
%%%%%%%%%%%%%%%%%%%%%%%%%%%%%%%%%%%%%%%%%%%%%%%%%%%%%%%%%%%%%%%%%%%%%%%%%%%%%%%%
\section{Information}

%%%%%%%%%%%%%%%%%%%%%%%%%%%%%%%%%%%%%%%%%%%%%%%%%%%%%%%%%%%%%%%%%%%%%%%%%%%%%%%%
\subsection{Copyright}

Copyright \copyright{} 2017--2018 Niklas Beisert

This work may be distributed and/or modified under the
conditions of the \LaTeX{} Project Public License, either version 1.3
of this license or (at your option) any later version.
The latest version of this license is in
  \url{http://www.latex-project.org/lppl.txt}
and version 1.3 or later is part of all distributions of \LaTeX{}
version 2005/12/01 or later.

This work has the LPPL maintenance status `maintained'.

The Current Maintainer of this work is Niklas Beisert.

This work consists of the files |README.txt|, |childdoc.ins| and |childdoc.dtx|
as well as the derived files |childdoc.def|, |cdocsamp.tex|
with |cdocsch1.tex|, |cdocsch2.tex|, |cdocspt3.tex|, |cdocspt4.tex|,
|cdocsdrf.tex|, |cdocsfn1.tex|, |cdocsfn2.tex|
as well as |childdoc.pdf|.

%%%%%%%%%%%%%%%%%%%%%%%%%%%%%%%%%%%%%%%%%%%%%%%%%%%%%%%%%%%%%%%%%%%%%%%%%%%%%%%%
\subsection{Files and Installation}

The package consists of the files:
%
\begin{center}
\begin{tabular}{ll}
    |README.txt|   & readme file \\
    |childdoc.ins| & installation file \\
    |childdoc.dtx| & source file \\
    |childdoc.def| & definition file \\
    |cdocsamp.tex| & sample main file \\
    |cdocsch1.tex| & sample include file \\
    |cdocsch2.tex| & sample include file \\
    |cdocspt3.tex| & sample part file \\
    |cdocspt4.tex| & sample part file \\
    |cdocsdrf.tex| & sample redirection file \\
    |cdocsfn1.tex| & sample redirection file \\
    |cdocsfn2.tex| & sample redirection file \\
    |childdoc.pdf| & manual
\end{tabular}
\end{center}
%
The distribution consists of the files
|README.txt|, |childdoc.ins| and |childdoc.dtx|.
%
\begin{itemize}
\item
Run (pdf)\LaTeX{} on |childdoc.dtx|
to compile the manual |childdoc.pdf| (this file).
\item
Run \LaTeX{} on |childdoc.ins| to create the definitions file |childdoc.def|
and the sample |cdocsamp.tex| with include files
|cdocsch1.tex|, |cdocsch2.tex|, |cdocspt3.tex|, |cdocspt4.tex|,
|cdocsdrf.tex|, |cdocsfn1.tex|, |cdocsfn2.tex|.
Then copy the file |childdoc.def| to an appropriate directory of your \LaTeX{}
distribution, e.g.\ \textit{texmf-root}|/tex/latex/childdoc|.
\end{itemize}

%%%%%%%%%%%%%%%%%%%%%%%%%%%%%%%%%%%%%%%%%%%%%%%%%%%%%%%%%%%%%%%%%%%%%%%%%%%%%%%%
\subsection{Related CTAN Packages}

There are several other packages which offer a similar functionality:
%
\begin{itemize}
\item
The packages
\href{http://ctan.org/pkg/docmute}{\textsf{docmute}},
\href{http://ctan.org/pkg/includex}{\textsf{includex}} and
\href{http://ctan.org/pkg/standalone}{\textsf{standalone}}
provide commands to include only the document body of
a child file thus allowing both files to be compiled individually.
\item
The packages \href{http://ctan.org/pkg/subdocs}{\textsf{subdocs}}
and \href{http://ctan.org/pkg/subfiles}{\textsf{subfiles}}
provide structures in which the main and child documents can be
encapsulated and allowing them to be compiled individually.
The inclusion mechanism is different from the conventional |\include|.
\item
The package \href{http://ctan.org/pkg/combine}{\textsf{combine}}
is an elaborate solution to combine several documents into one.
\end{itemize}
%
See also the CTAN topic \href{http://ctan.org/topic/subdocs}{\textsf{subdocs}}
for further related packages.
The present package differs from the above solutions in that
a document structure constructed with the conventional |\include| mechanism
just needs two extra commands at the top of every file
such that all constituent files can be compiled individually.

%%%%%%%%%%%%%%%%%%%%%%%%%%%%%%%%%%%%%%%%%%%%%%%%%%%%%%%%%%%%%%%%%%%%%%%%%%%%%%%%
%\subsection{Feature Suggestions}
%
%The following is a list of features which may be useful for future
%versions of this package:
%%
%\begin{itemize}
%\item
%\ldots
%\end{itemize}

%%%%%%%%%%%%%%%%%%%%%%%%%%%%%%%%%%%%%%%%%%%%%%%%%%%%%%%%%%%%%%%%%%%%%%%%%%%%%%%%
\subsection{Revision History}

%%%%%%%%%%%%%%%%%%%%%%%%%%%%%%%%%%%%%%%%
\paragraph{v2.0:} 2018/12/30

\begin{itemize}
\item
immediate forward processing
\item
added |\childdocby| mechanism
\item
manual restructured
\end{itemize}

%%%%%%%%%%%%%%%%%%%%%%%%%%%%%%%%%%%%%%%%
\paragraph{v1.6:} 2018/01/17

\begin{itemize}
\item
application for development of include files
\item
corrections to manual
\end{itemize}

%%%%%%%%%%%%%%%%%%%%%%%%%%%%%%%%%%%%%%%%
\paragraph{v1.5:} 2017/05/21

\begin{itemize}
\item
more complete structuring introduced
\item
|\childdocof| introduced
\item
|\childdoc| renamed to |\childdocmain|
\item
|\childredirect| renamed to |\childdocforward| and |\childdocforwardprefix|
and functionality expanded
\end{itemize}

%%%%%%%%%%%%%%%%%%%%%%%%%%%%%%%%%%%%%%%%
\paragraph{v1.0:} 2017/04/27

\begin{itemize}
\item
manual and install package
\item
first version published on CTAN
\end{itemize}

%%%%%%%%%%%%%%%%%%%%%%%%%%%%%%%%%%%%%%%%
\paragraph{v0.6:} 2017/04/26

\begin{itemize}
\item
redirection mechanism added
\end{itemize}

%%%%%%%%%%%%%%%%%%%%%%%%%%%%%%%%%%%%%%%%
\paragraph{v0.5:} 2017/04/26

\begin{itemize}
\item
functionality in definition file
\end{itemize}


%%%%%%%%%%%%%%%%%%%%%%%%%%%%%%%%%%%%%%%%%%%%%%%%%%%%%%%%%%%%%%%%%%%%%%%%%%%%%%%%
%%%%%%%%%%%%%%%%%%%%%%%%%%%%%%%%%%%%%%%%%%%%%%%%%%%%%%%%%%%%%%%%%%%%%%%%%%%%%%%%
%%%%%%%%%%%%%%%%%%%%%%%%%%%%%%%%%%%%%%%%%%%%%%%%%%%%%%%%%%%%%%%%%%%%%%%%%%%%%%%%
\appendix

\settowidth\MacroIndent{\rmfamily\scriptsize 000\ }

 \DocInput{childdoc.dtx}

\end{document}
%</driver>
% \fi
%
% %%%%%%%%%%%%%%%%%%%%%%%%%%%%%%%%%%%%%%%%%%%%%%%%%%%%%%%%%%%%%%%%%%%%%%%%%%%%%%
% %%%%%%%%%%%%%%%%%%%%%%%%%%%%%%%%%%%%%%%%%%%%%%%%%%%%%%%%%%%%%%%%%%%%%%%%%%%%%%
% \section{Sample}
%\iffalse
%<*samplemain>
%\fi
%
% The following presents a sample document
% with two chapters, two parts, a title page,
% a compile flag as well as three forwarding files to set the flag.
% It consists of eight |.tex| files:
% \begin{center}
% \begin{tabular}{ll}
% |cdocsamp.tex|&main file\\
% |cdocsch1.tex|&include file for chapter 1\\
% |cdocsch2.tex|&include file for chapter 2\\
% |cdocspt3.tex|&include file for part 3\\
% |cdocspt4.tex|&include file for part 4\\
% |cdocsdrf.tex|&forwarding file for main file in draft mode\\
% |cdocsfi1.tex|&forwarding file for final version of chapter 1\\
% |cdocsfi2.tex|&forwarding file for final version of chapter 2\\
% \end{tabular}
% \end{center}
% Each of the eight files can be compiled directly by the \LaTeX{} compiler.
%
% %%%%%%%%%%%%%%%%%%%%%%%%%%%%%%%%%%%%%%
% \paragraph{Main File.}
%
% The main file is called |cdocsamp.tex|.
%
% Load the \textsf{childdoc} definitions and
% declare the filename for the main document:
%    \begin{macrocode}
\input{childdoc.def}
\childdocmain{}
%    \end{macrocode}

% Optional override for |\version| flag:
%    \begin{macrocode}
%%\ifchilddoc\else\providecommand{\version}{draft}\fi
%    \end{macrocode}

% Define the default values for the |\version| flag
% (|final| for the main file and |draft| for childs):
%    \begin{macrocode}
\ifchilddoc
\providecommand{\version}{draft}
\else
\providecommand{\version}{final}
\fi
%    \end{macrocode}

% Load the standard document class:
%    \begin{macrocode}
\documentclass[12pt]{article}
%    \end{macrocode}

% Start the document body:
%    \begin{macrocode}
\begin{document}
%    \end{macrocode}

% Declare a title page.
% Print title, part of document being processed and version flag:
%    \begin{macrocode}
\addtocounter{page}{-1}
\begin{center}
{\LARGE\bfseries{}childdoc example\par}
\vspace{1cm}
\ifchilddoc
\ifchilddocmanual part\else chapter\fi:
`\childdocname' of `\childdocjob'\par
\else
main document: `\childdocjob'\par
\fi
version: \version\par
\end{center}
\newpage
%    \end{macrocode}

% Manually include selected file,
% otherwise process as usual:
%    \begin{macrocode}
\ifchilddocmanual
\section*{part `\childdocname'}
\input{\childdocname}
\else
%    \end{macrocode}

% Include the two chapters:
%    \begin{macrocode}
\include{cdocsch1}
\include{cdocsch2}
%    \end{macrocode}

% Include the two parts unless only chapters should be displayed:
%    \begin{macrocode}
\ifchilddoc\else
\section{part three}
\input{cdocspt3}
\section{part four}
\input{cdocspt4}
\fi
%    \end{macrocode}

% Process as usual until here:
%    \begin{macrocode}
\fi
%    \end{macrocode}

% End of document body:
%    \begin{macrocode}
\end{document}
%    \end{macrocode}
%\iffalse
%</samplemain>
%\fi
%
% %%%%%%%%%%%%%%%%%%%%%%%%%%%%%%%%%%%%%%
% \paragraph{Chapter Include Files.}
%
% The include files are called |cdocsch1.tex| and |cdocsch2.tex|.
%
%\iffalse
%<*samplechap1|samplechap2>
%\fi

% Optional override for |\version| flag:
%    \begin{macrocode}
%%\providecommand{\version}{final}
%    \end{macrocode}

% Include the main document:
%    \begin{macrocode}
\input{childdoc.def}
\childdocof{cdocsamp}
%    \end{macrocode}

%\iffalse
%</samplechap1|samplechap2>
%\fi
%
%\iffalse
%<*samplechap1>
%\fi
% Some text for chapter 1:
%    \begin{macrocode}
\section{one}
some text in chapter one
%    \end{macrocode}

%\iffalse
%</samplechap1>
%\fi
% Some text for chapter 2:
%\iffalse
%<*samplechap2>
%\fi
%    \begin{macrocode}
\section{two}
more text in chapter two
%    \end{macrocode}

%\iffalse
%</samplechap2>
%\fi
%
% %%%%%%%%%%%%%%%%%%%%%%%%%%%%%%%%%%%%%%
% \paragraph{Part Include Files.}
%
% The include files are called |cdocspt3.tex| and |cdocspt4.tex|.
%
%\iffalse
%<*samplepart3|samplepart4>
%\fi

% Optional override for |\version| flag:
%    \begin{macrocode}
%%\providecommand{\version}{final}
%    \end{macrocode}

% Include the main document:
%    \begin{macrocode}
\input{childdoc.def}
\childdocby{cdocsamp}
%    \end{macrocode}

%\iffalse
%</samplepart3|samplepart4>
%\fi
%
%\iffalse
%<*samplepart3>
%\fi
% Some text for part 3:
%    \begin{macrocode}
some text in part three
%    \end{macrocode}

%\iffalse
%</samplepart3>
%\fi
% Some text for part 4:
%\iffalse
%<*samplepart4>
%\fi
%    \begin{macrocode}
more text in part four
%    \end{macrocode}

%\iffalse
%</samplepart4>
%\fi
%
% %%%%%%%%%%%%%%%%%%%%%%%%%%%%%%%%%%%%%%
% \paragraph{Forwarding for a Complete Draft.}
%
% The following forwarding file |cdocsdrf.tex|
% compiles the main document in draft mode:
%\iffalse
%<*sampledraft>
%\fi
%    \begin{macrocode}
\def\version{draft}
\input{childdoc.def}
\childdocforward{cdocsamp}
%    \end{macrocode}

%\iffalse
%</sampledraft>
%\fi
%
% %%%%%%%%%%%%%%%%%%%%%%%%%%%%%%%%%%%%%%
% \paragraph{Forwarding for Final Version of the Chapters.}
%
% The following forwarding files |cdocsfn1.tex| and |cdocsfn2.tex|
% (with identical content)
% compile the final versions of the child documents
% |cdocsch1.tex| and |cdocsch2.tex|, respectively:
%\iffalse
%<*samplefinal>
%\fi
%    \begin{macrocode}
\def\version{final}
\input{childdoc.def}
\childdocforwardprefix[cdocsamp]{cdocsfn}{cdocsch}
%    \end{macrocode}

%\iffalse
%</samplefinal>
%\fi
%
% %%%%%%%%%%%%%%%%%%%%%%%%%%%%%%%%%%%%%%
% \paragraph{Command Line Processing.}
%
% The following three command lines generate the output files
% |cdocscld|, |cdocscl1| and |cdocscl2|
% which should be identical to
% |cdocsdrf|, |cdocsch1| and |cdocsfn2|, respectively:
% \begin{center}
% \begin{tabular}{l}
% |latex -jobname cdocscld \|\\
% |  "\def\version{draft}\input{childdoc.def}\childdocforward{cdocsamp}"|\\
% |latex -jobname cdocscl1 \|\\
% |  "\input{childdoc.def}\childdocforward[cdocsamp]{cdocsch1}"|\\
% |latex -jobname cdocscl2 \|\\
% |  "\def\version{final}\input{childdoc.def}\childdocforward{cdocsch2}"|
% \end{tabular}
% \end{center}
% Note that the trailing backslash on each first line
% merely continues the input to the second line
% (for convenient cut ant paste).
% Furthermore, the command |latex| can be replaced by any
% of its alternative versions such as |pdflatex|.
%
% %%%%%%%%%%%%%%%%%%%%%%%%%%%%%%%%%%%%%%%%%%%%%%%%%%%%%%%%%%%%%%%%%%%%%%%%%%%%%%
% %%%%%%%%%%%%%%%%%%%%%%%%%%%%%%%%%%%%%%%%%%%%%%%%%%%%%%%%%%%%%%%%%%%%%%%%%%%%%%
% \section{Implementation}
%\iffalse
%<*package>
%\fi
%
% This section describes the definitions file |childdoc.def|.

% The definitions cannot be loaded using |\usepackage| or |\RequirePackage|
% which has a mechanism to prevent loading a style file more than once.
% When loading the definitions by means of |\input|
% multiple instances have to be prevented manually:
%\iffalse
%This code needs to be before the `\ProvidesFile' directive
%which is defined at the beginning of this file.
%Therefore it is also placed there and commented out here.
%</package>
%<*discard>
%\fi
%    \begin{macrocode}
\ifdefined\childdocmain\endinput\fi
%    \end{macrocode}
%\iffalse
%</discard>
%<*package>
%\fi
%
% \macro{\ifchilddoc}
% \macro{\ifchilddocmanual}
% The conditional |\ifchilddoc| tells whether a
% child (true) or main (false) document is being compiled.
% The conditional |\ifchilddocmanual| tells whether
% the |\includeonly| mechanism is used (false) or
% the selection of child files must be performed manually (true).
% The definitions initialise to false:
%    \begin{macrocode}
\newif\ifchilddoc
\newif\ifchilddocmanual
%    \end{macrocode}

% \macro{\childdocname}
% \macro{\childdocjob}
% The macro |\childdocname| stores the name of the main document
% to be compiled. The macro |\childdocjob| stores the name of
% the document on which the \LaTeX{} compiler was originally invoked.
% The content of |\jobname| cannot be compared
% to filenames specified in the source due to different catcodes.
% The following code rescans |\jobname|, stores the result
% in |\childdocname| and saves a copy in |\childdocjob|:
%    \begin{macrocode}
\edef\childdocname{\scantokens\expandafter{\jobname\noexpand}}
\let\childdocjob\childdocname
%    \end{macrocode}

% \macro{\childdocdisable}
% The macro |\childdocdisable| prevents the main file
% from being processed more than once.
% At this stage, the main document command |\childdocmain|
% is assumed to be called once again where it should do nothing.
% Any subsequent call to it should prevent
% a secondary processing of the main document
% It overwrites the forwarding commands
% |\childdocof| and |\childdocforward|
% with empty macros to prevent further inclusions of the main document:
%    \begin{macrocode}
\newcommand{\childdocdisable}
{
  \renewcommand{\childdocmain}[1]{\renewcommand{\childdocmain}[1]{\endinput}}
  \renewcommand{\childdocof}[1]{}
  \renewcommand{\childdocby}[2][]{}
  \renewcommand{\childdocforward}[2][]{}
  \renewcommand{\childdocdisable}{}
}
%    \end{macrocode}

% \macro{\childdocmain}
% The macro |\childdocmain| is to be called at the top of the main file
% with nothing or the main filename (without extension) as argument.
% First, it breaks loops.
% If the argument is not empty and does not match |\childdocname|
% (which is set by the first inclusion of |childdoc.def|),
% |\ifchilddoc| is set to true, |\includeonly| is applied to the child file
% and |\jobname| is set to the main file
% (for proper handling of |.aux| files):
%    \begin{macrocode}
\newcommand{\childdocmain}[1]
{
  \childdocdisable\childdocmain{}
  \if?#1?\else
    \begingroup
      \def\childdoctmp{#1}
      \ifx\childdoctmp\childdocname
        \def\childdoctmp{}
      \else
        \def\childdoctmp
        {
          \childdoctrue
          \includeonly{\childdocname}
          \def\childdocjob{#1}
          \def\jobname{#1}
        }
      \fi
      \expandafter
    \endgroup
    \childdoctmp
  \fi
}
%    \end{macrocode}

% \macro{\childdocof}
% The command |\childdocof| redirects
% compilation to the main file |#1|.
%    \begin{macrocode}
\newcommand{\childdocof}[1]
{
  \childdocdisable
  \childdoctrue
  \includeonly{\childdocname}
  \def\jobname{#1}
  \def\childdocjob{#1}
  \input{#1}
}
%    \end{macrocode}

% \macro{\childdocby}
% The command |\childdocby| ....
%    \begin{macrocode}
\newcommand{\childdocby}[2][]
{
  \childdocdisable
  \childdoctrue
  \childdocmanualtrue
  \if?#1?\else
    \def\jobname{#2}
  \fi
  \def\childdocjob{#2}
  \input{#2}
  \endinput
}
%    \end{macrocode}

% \macro{\childdocforward}
% The command |\childdocforward| redirects
% compilation to the main file or
% (if the optional argument is given) a child file.
% Parameters are set as if the main file
% or a child file starting with |\childdocof| was compiled.
% Then compilation is handed over to the main file:
%    \begin{macrocode}
\newcommand{\childdocforward}[2][]
{
  \begingroup
    \if?#1?
      \def\childdoctmp
      {
        \def\childdocname{#2}
        \def\childdocjob{#2}
        \def\jobname{#2}
        \input{#2}
        \endinput
      }
    \else
      \def\childdoctmp
      {
        \childdocdisable
        \def\childdocname{#2}
        \childdoctrue
        \includeonly{#2}
        \def\childdocjob{#1}
        \def\jobname{#1}
        \input{#1}
        \endinput
      }
    \fi
    \expandafter
  \endgroup
  \childdoctmp
}
%    \end{macrocode}

% \macro{\childdocforwardprefix}
% The command |\childdocforwardprefix| redirects
% compilation to the main or a child file by means of a pattern.
% The prefix |#1| in the current filename is replaced by |#2|
% and the suffix of the current filename is kept
% (it is assumed that the filename does not contain the substring `|~~~|'
% which is used as a delimiter).
% Compilation is handed over to the new file by |\childdocforward|:
%    \begin{macrocode}
\newcommand{\childdocforwardprefix}[3][]
{
  \begingroup
    \def\childdocextract #2##1~~~{\def\childdoctmp{\childdocforward[#1]{#3##1}}}
    \expandafter\childdocextract\childdocname~~~
    \expandafter
  \endgroup
  \childdoctmp
}
%    \end{macrocode}

% \macro{\childdoc}
% The deprecated macro |\childdoc| is a legacy version of |\childdocmain|:
%    \begin{macrocode}
\newcommand{\childdoc}{\childdocmain}
%    \end{macrocode}

% \macro{\childdocredirect}
% The deprecated macro |\childdocredirect| is a legacy version
% of |\childdocforward| and |\childdocforwardprefix|:
%    \begin{macrocode}
\newcommand{\childdocredirect}[2][]
{
  \begingroup
    \if?#1?
      \def\childdoctmp{\childdocforward{#2}}
    \else
      \def\childdoctmp{\childdocforwardprefix{#1}{#2}}
    \fi
    \expandafter
  \endgroup
  \childdoctmp
}
%    \end{macrocode}

%\iffalse
%</package>
%\fi
%
\endinput

\childdocforwardprefix[cdocsamp]{cdocsfn}{cdocsch}
%    \end{macrocode}

%\iffalse
%</samplefinal>
%\fi
%
% %%%%%%%%%%%%%%%%%%%%%%%%%%%%%%%%%%%%%%
% \paragraph{Command Line Processing.}
%
% The following three command lines generate the output files
% |cdocscld|, |cdocscl1| and |cdocscl2|
% which should be identical to
% |cdocsdrf|, |cdocsch1| and |cdocsfn2|, respectively:
% \begin{center}
% \begin{tabular}{l}
% |latex -jobname cdocscld \|\\
% |  "\def\version{draft}% \iffalse
%
% childdoc.dtx Copyright (C) 2017-2018 Niklas Beisert
%
% This work may be distributed and/or modified under the
% conditions of the LaTeX Project Public License, either version 1.3
% of this license or (at your option) any later version.
% The latest version of this license is in
%   http://www.latex-project.org/lppl.txt
% and version 1.3 or later is part of all distributions of LaTeX
% version 2005/12/01 or later.
%
% This work has the LPPL maintenance status `maintained'.
%
% The Current Maintainer of this work is Niklas Beisert.
%
% This work consists of the files childdoc.dtx and childdoc.ins
% and the derived files childdoc.def and cdocsamp.tex with
% cdocsch1.tex, cdocsch2.tex, cdocsdrf.tex, cdocsfn1.tex, cdocsfn2.tex.
%
%<package>\ifdefined\childdocmain\endinput\fi
%<package>\ProvidesFile{childdoc.def}[2018/12/30 v2.0 child document driver]
%<samplemain>\ProvidesFile{cdocsamp.tex}[2018/12/30 v2.0 sample for childdoc]
%<*driver>
%\ProvidesFile{childdoc.drv}[2018/12/30 v2.0 childdoc reference manual file]
\PassOptionsToClass{10pt,a4paper}{article}
\documentclass{ltxdoc}

\usepackage[margin=35mm]{geometry}
\usepackage{hyperref}
\usepackage{hyperxmp}
\usepackage[usenames]{color}

\hypersetup{colorlinks=true}
\hypersetup{pdfstartview=FitH}
\hypersetup{pdfpagemode=UseNone}
\hypersetup{pdfsource={}}
\hypersetup{pdflang={en-UK}}
\hypersetup{pdfcopyright={Copyright 2017-2018 Niklas Beisert.
  This work may be distributed and/or modified under the
  conditions of the LaTeX Project Public License, either version 1.3
  of this license or (at your option) any later version.}}
\hypersetup{pdflicenseurl={http://www.latex-project.org/lppl.txt}}
\hypersetup{pdfcontactaddress={ETH Zurich, ITP, HIT K,
  Wolfgang-Pauli-Strasse 27}}
\hypersetup{pdfcontactpostcode={8093}}
\hypersetup{pdfcontactcity={Zurich}}
\hypersetup{pdfcontactcountry={Switzerland}}
\hypersetup{pdfcontactemail={nbeisert@itp.phys.ethz.ch}}
\hypersetup{pdfcontacturl={http://people.phys.ethz.ch/\xmptilde nbeisert/}}

\newcommand{\secref}[1]{\hyperref[#1]{section \ref*{#1}}}

\parskip1ex
\parindent0pt
\let\olditemize\itemize
\def\itemize{\olditemize\parskip0pt}

\begin{document}

\title{The \textsf{childdoc} Package}
\hypersetup{pdftitle={The childdoc Package}}
\author{Niklas Beisert\\[2ex]
  Institut f\"ur Theoretische Physik\\
  Eidgen\"ossische Technische Hochschule Z\"urich\\
  Wolfgang-Pauli-Strasse 27, 8093 Z\"urich, Switzerland\\[1ex]
  \href{mailto:nbeisert@itp.phys.ethz.ch}
  {\texttt{nbeisert@itp.phys.ethz.ch}}}
\hypersetup{pdfauthor={Niklas Beisert}}
\hypersetup{pdfsubject={Manual for the LaTeX2e Package childdoc}}
\date{30 December 2018, \textsf{v2.0}}
\maketitle

\begin{abstract}\noindent
\textsf{childdoc} is a \LaTeXe{} package
that enables the direct compilation
of document sections included by |\include|
to individual files.
\end{abstract}

\begingroup
\parskip0ex
\tableofcontents
\endgroup

%%%%%%%%%%%%%%%%%%%%%%%%%%%%%%%%%%%%%%%%%%%%%%%%%%%%%%%%%%%%%%%%%%%%%%%%%%%%%%%%
%%%%%%%%%%%%%%%%%%%%%%%%%%%%%%%%%%%%%%%%%%%%%%%%%%%%%%%%%%%%%%%%%%%%%%%%%%%%%%%%
\section{Introduction}

\LaTeX{} provides a mechanism to structure a large document (such as a book)
into a main file and several child files (containing the chapters)
using the |\include| command.
This mechanism is beneficial for documents
which span hundreds of pages in order to
make the source file(s) more manageable.
Moreover, compilation can be restricted to
selected child files by means of the |\includeonly| command.
The latter feature can be used to reduce the compilation time while editing
(this was significantly more useful in the earlier days of \LaTeX{})
or to generate a smaller document which is easier to navigate.
Another application of |\includeonly| is to generate
documents consisting of selected parts of the complete document.

However, there are a few drawbacks of the plain |\include| mechanism:
\begin{itemize}
\item
The child files cannot be compiled on their own,
they can only be compiled via the main file.
A naive editing environment
(such as a text editor with an option
to have the current file processed by \LaTeX)
may require one to switch to the main file before compiling;
attempting to compile the child file produces errors.
\item
The main file must be modified (each time)
to adjust the |\includeonly| command
to the present needs. This easily leaves the main file in a messy state.
\item
The generated document will always carry the filename
of the main document. This is inconvenient if
several child files are to be compiled and
to be kept for distribution.
\end{itemize}

The present package provides a simple interface
to make child files individually compilable by \LaTeX{}.
Compiling a child file then has the same effect as compiling
the main file with an |\includeonly| command
to select the appropriate child.
Moreover the generated document will carry the name of the child
rather than the main file.
This resolves all three above issues.

This feature is meant to make the editing of books,
thesis documents and lecture notes somewhat more convenient.
However, the package can also be used efficiently for
composing a series of documents (such as exercise sheets)
which are typically distributed individually.
It then assists the author in generating the individual documents
(potentially in different versions)
as well as a document containing the collected series.
Another application is in developing style files
or other kinds of included material
where compilation of the style file could redirect
to a sample or test file.

%%%%%%%%%%%%%%%%%%%%%%%%%%%%%%%%%%%%%%%%%%%%%%%%%%%%%%%%%%%%%%%%%%%%%%%%%%%%%%%%
%%%%%%%%%%%%%%%%%%%%%%%%%%%%%%%%%%%%%%%%%%%%%%%%%%%%%%%%%%%%%%%%%%%%%%%%%%%%%%%%
\section{Usage}

First of all, the package \textsf{childdoc} is \emph{not} a standard
\LaTeXe{} |.sty| style file! Therefore it needs to be invoked in
a non-standard way.

%%%%%%%%%%%%%%%%%%%%%%%%%%%%%%%%%%%%%%%%%%%%%%%%%%%%%%%%%%%%%%%%%%%%%%%%%%%%%%%%
\subsection{Included Files}
\label{sec:include}

%%%%%%%%%%%%%%%%%%%%%%%%%%%%%%%%%%%%%%%%
\DescribeMacro{\childdocmain}
To use the package, add the commands
\begin{center}
\begin{tabular}{l}
|\input{childdoc.def}|\\
|\childdocmain{}|\\
\end{tabular}
\end{center}
at the very top of the main \LaTeX{} file,
in particular \emph{before} the |\documentclass| statement!
The argument of |\childdocmain| should be left empty
(but it must be present).

%%%%%%%%%%%%%%%%%%%%%%%%%%%%%%%%%%%%%%%%
\DescribeMacro{\childdocof}
Furthermore, add the commands
\begin{center}
\begin{tabular}{l}
|\input{childdoc.def}|\\
|\childdocof{|\textit{main}|}|\\
\end{tabular}
\end{center}
at the top of every child file \textit{child}
which is included by |\include{|\textit{child}|}|
from within the main file
(or at least for those files to be compiled individually).
The argument \textit{main} must be the filename of the main file.

There are a couple of
considerations in setting up the main and child documents:

%%%%%%%%%%%%%%%%%%%%%%%%%%%%%%%%%%%%%%%%
\paragraph{Restrictions.}

Please note the following restrictions:
\begin{itemize}
\item
|\childdocmain| must be called with one argument \textit{main}
to ensure compatibility with earlier version of the package.
It must either be empty (|\childdocmain{}|)
or precisely match the filename of the main file in which it is specified.
See \secref{sec:detection} for further information.
\item
The filename \textit{main} must be specified without the |.tex| extension.
\item
The filename \textit{main} is case sensitive
(even in case-insensitive file systems)
due to internal string comparison.
\item
The argument \textit{main} should be fully expanded, it cannot be a macro.
\item
Subdirectories and special characters should be avoided in filenames.
\item
The command |\childdocmain{|\textit{main}|}| must be followed by a whitespace.
It should not be followed immediately by another command
or by a comment mark `|%|'.
This is because the \TeX{} parser reads the token immediately following
the argument of |\childdocmain| and puts it
at the beginning of every child section;
however, a white\-space is ignored.
\end{itemize}

%%%%%%%%%%%%%%%%%%%%%%%%%%%%%%%%%%%%%%%%
\paragraph{Content of Main File.}

It is advisable to place all content in the child files included by |\include|.
Any output contained in the main file will appear in all child documents
unless suppressed manually;
it cannot be suppressed automatically by the |\includeonly| directive
and thus should normally be avoided.
A method to include some content in the main file
by means of conditional processing is described in \secref{sec:conditional}.

%%%%%%%%%%%%%%%%%%%%%%%%%%%%%%%%%%%%%%%%
\paragraph{Page Numbering.}

When only a part of the document is compiled,
the appropriate numbering of pages
(as well as other status parameters)
is determined from the |.aux| files.
The latter contain information from previous passes.
However this information needs to propagate through
all intermediate child documents.
Therefore the page numbering in child documents may well
be inconsistent until the complete document is compiled at least once.

A useful (if unconventional) way to always ensure a consistent
page numbering is to restart the numbering in each child document
and denote the pages by `\textit{child}|.|\textit{page}'
where \textit{child} represents the chapter/section number of the child file.
This can be achieved by the command
|\numberwithin{page}{|\textit{child}|}|
of the \textsf{amsmath} package
where \textit{child} can be |chapter| or |section|
depending on the chosen structuring.
Alternatively, one can modify the macro |\thepage| appropriately
and reset the counter |page| at the start of each child file.

%%%%%%%%%%%%%%%%%%%%%%%%%%%%%%%%%%%%%%%%%%%%%%%%%%%%%%%%%%%%%%%%%%%%%%%%%%%%%%%%
\subsection{Conditional Processing}
\label{sec:conditional}

The package provides a mechanism to compile different versions
of a document. To customise the versions further some conditional processing
can come in handy to distinguish which version is being compiled.
The package provides two macros to describe the compilation context:

%%%%%%%%%%%%%%%%%%%%%%%%%%%%%%%%%%%%%%%%
\DescribeMacro{\ifchilddoc}
The conditional |\ifchilddoc| distinguishes between the compilation of
child documents and the main document:
%
\begin{center}
|\ifchilddoc |\textit{child-code}| |[|\||else |\textit{main-code}]| \||fi|
\end{center}

%%%%%%%%%%%%%%%%%%%%%%%%%%%%%%%%%%%%%%%%
\DescribeMacro{\childdocname}
\DescribeMacro{\childdocjob}
The macro |\childdocname| contains the filename (without extension)
of the main or child file being processed.
Note that |\childdocjob| will always contain the name of the main file.

%%%%%%%%%%%%%%%%%%%%%%%%%%%%%%%%%%%%%%%%
\paragraph{Title Page.}

Conditional processing can be used to include a title or banner page
in the main document when proper precautions are taken.
Importantly, the code in the main file should ensure that the page counter
(as well as other status parameters which are stored in the |.aux| files)
takes the same value after the conditional processing.
Otherwise the page numbers may take divergent values
depending on which part is compiled.

For example, a title page could be declared by:
%
\begin{center}
\begin{tabular}{l}
|\ifchilddoc\||else|\\
|\addtocounter{page}{-1}|\\
\textit{code for title page}\\
|\newpage|\\
|\||fi|
\end{tabular}
\end{center}
%
A banner page for the child documents can be generated by:
%
\begin{center}
\begin{tabular}{l}
|\ifchilddoc|\\
|\addtocounter{page}{-1}|\\
\textit{code for banner page}\\
|\newpage|\\
|\||fi|
\end{tabular}
\end{center}
%
Here one could write a message such as:
\begin{center}
|This is the part \childdocname{} of \childdocjob{}.|
\end{center}

%%%%%%%%%%%%%%%%%%%%%%%%%%%%%%%%%%%%%%%%%%%%%%%%%%%%%%%%%%%%%%%%%%%%%%%%%%%%%%%%
\subsection{Flags}
\label{sec:flags}

The package makes it easy to generate different versions
of the main or child documents.
To this end compilation flags can be defined
and assigned different default values.
They will be particularly useful in conjunction
with the forwarding mechanism described in \secref{sec:forward}.

For example, it may be useful to have a flag |\version|
which can be set to |draft| or |final|.
The document source will contain some conditional code
depending on the value of |\version|.
Suppose further, the flag should default to |final| for the main file
and to |draft| for child files
which is a natural assignment for editing the document.
This is achieved by placing the following code
in the preamble of the main document
(below the |\childdocmain| directive):
%
\begin{center}
\begin{tabular}{l}
|\ifchilddoc|\\
|\providecommand{\version}{draft}|\\
|\||else|\\
|\providecommand{\version}{final}|\\
|\||fi|
\end{tabular}
\end{center}
%
The definition by |\providecommand| makes sure
that previous definitions are not overwritten.
Further statements |\providecommand{\version}{...}|
can thus be added before the above code to override it.

For the main file, one might add a line
(between |\childdocmain| and the above block)
%
\begin{center}
|%\ifchilddoc\||else\providecommand{\version}{draft}\||fi|
\end{center}
%
which can be uncommented to produce a draft version.
Likewise one can add a line to the very top of a child file
(above the |\childdocof{|\textit{main}|}| directive)
%
\begin{center}
|%\providecommand{\version}{final}|
\end{center}
%
which can be uncommented to produce the final version of this child document.

%%%%%%%%%%%%%%%%%%%%%%%%%%%%%%%%%%%%%%%%%%%%%%%%%%%%%%%%%%%%%%%%%%%%%%%%%%%%%%%%
\subsection{Forwarding}
\label{sec:forward}

Different versions of the main or child documents
using compilation flags as described in \secref{sec:flags}
can be (permanently) stored in different files
for convenient compilation, viewing and distribution.
To this end, the package defines a command
to pass on compilation to a different file:

%%%%%%%%%%%%%%%%%%%%%%%%%%%%%%%%%%%%%%%%
\DescribeMacro{\childdocforward}
The command |\childdocforward| redirects processing to
another source file:
%
\begin{center}
\begin{tabular}{l}
|\input{childdoc.def}|\\
|\childdocforward[|\textit{main}|]{|\textit{dest}|}|\\
\end{tabular}
\end{center}
%
The argument \textit{dest} is the destination file
(without extension).
It should be the main file or one of the child files.
Note that further \textsf{childdoc} directives
such as |\childdocof| and |\childdocforward|
in the indicated file will be processed in this form.
The optional argument \textit{main}
passes on directly to the main file \textit{main}
while pretending to compile the child \textit{dest}.
This form behaves as if \textit{dest}
issues |\childdocof{|\textit{main}|}| right away,
and no further \textsf{childdoc} directives will be processed.

%%%%%%%%%%%%%%%%%%%%%%%%%%%%%%%%%%%%%%%%
\DescribeMacro{\...prefix}
In the alternative form |\childdocforwardprefix|,
%
\begin{center}
\begin{tabular}{l}
|\input{childdoc.def}|\\
|\childdocforwardprefix[|\textit{main}|]{|\textit{prefix}|}{|\textit{dest}|}|
\end{tabular}
\end{center}
%
the destination file is determined by a pattern
depending on the current file:
To make this work, the current file must be called
`{\textit{prefix}\hspace{0.2em}\textit{suffix}}'
with \textit{prefix} matching precisely the argument.
Processing is then passed on to the file
`{\textit{dest}\hspace{0.2em}\textit{suffix}}'.
Surely, the same effect is achieved by
directly specifying the
argument `{\textit{dest}\hspace{0.2em}\textit{suffix}}'
in the first form.
However, that requires to set up a different file
for each child. With the alternative form of the command
all these files can have exactly the same content
which simplifies setting them up and maintaining them.

For example, the following file |draft.tex|
with a compilation flag |\version| as described in \secref{sec:flags}
compiles the main document as a draft:
%
\begin{center}
\begin{tabular}{l}
|\def\version{draft}|\\
|\input{childdoc.def}|\\
|\childdocforward{|\textit{main}|}|
\end{tabular}
\end{center}
%
Likewise, the following files |final|\textit{nn}|.tex|
compile the final version of the child document
|child|\textit{nn}|.tex|:
%
\begin{center}
\begin{tabular}{l}
|\def\version{final}|\\
|\input{childdoc.def}|\\
|\childdocforwardprefix{final}{child}|
\end{tabular}
\end{center}
%

Note that when several versions of a main file and/or of each child file
are to be generated, it may be convenient to set up a |Makefile| or
shell script to automatise the process.

%%%%%%%%%%%%%%%%%%%%%%%%%%%%%%%%%%%%%%%%%%%%%%%%%%%%%%%%%%%%%%%%%%%%%%%%%%%%%%%%
\subsection{Command Line Processing}
\label{sec:commandline}

The effect of redirection files can also be achieved by invoking
the \LaTeX{} compiler with a more elaborate command line.
Most conveniently this should be done as part
of a shell script or a |Makefile|.

When using \textsf{childdoc} in the main file, the following
command lines effectively perform a redirection
(note that depending on the shell being used,
backslashes may have to be doubled: `|\|' $\to$ `|\\|'):
%
\begin{center}
|... -jobname "|\textit{target}|" |\\|"|[\textit{flags}]%
|\input{childdoc.def}\childdocforward[|\textit{main}|]{|\textit{dest}|}"|
\end{center}
%
Here \textit{target} is the name of the output file,
\textit{main} is the name of the main file
and \textit{dest} is the name of the main or child file to be processed
(all filenames without extensions).
The optional argument \textit{main} can be omitted
if \textit{main} matches \textit{dest}.
Optionally, compilation \textit{flags} can be defined via |\def| commands.
This command line makes the \TeX{} engine believe
it is compiling the file \textit{target}
whose content is specified as the latter parameter.
The provided code then forwards the processing to
\textit{main} or \textit{dest} as described in \secref{sec:forward}.

%%%%%%%%%%%%%%%%%%%%%%%%%%%%%%%%%%%%%%%%%%%%%%%%%%%%%%%%%%%%%%%%%%%%%%%%%%%%%%%%
\subsection{Include by Input}
\label{sec:input}

Including child documents by |\include| has some restrictions by design.
Most notably, the content of a child document always occupies
its own set of pages; pages cannot be shared between child documents.
Usually, this behaviour makes perfect sense
because each child document contain an essential part of the document.
However, in some situations it may be desirable to compose
a document from a collection of parts
without having mandatory page breaks between then.
For this case, the package
provides a mechanism to include parts
by |\input| which can also be processed individually.
However, by construction this mechanism
requires manual handling of the content to be output.

%%%%%%%%%%%%%%%%%%%%%%%%%%%%%%%%%%%%%%%%
\DescribeMacro{\ifchilddocmanual}
The main file should be prepared as usual, see \secref{sec:include}.
However, the document body must make a distinction
between processing of an individual part and of the main document, e.g.:
%
\begin{center}
\begin{tabular}{l}
|\ifchilddocmanual|\\
|\input{\childdocname}|\\
|\||else|\\
\textit{document body with }|\input{|\textit{part}|}|\\
|\||fi|
\end{tabular}
\end{center}
%
The conditional |\ifchilddocmanual| is true whenever
a part to be included by |\input| is being compiled,
and the name of the part is stored in |\childdocname|.

%%%%%%%%%%%%%%%%%%%%%%%%%%%%%%%%%%%%%%%%
\DescribeMacro{\childdocby}
Each part to be included by |\input| should start with:
%
\begin{center}
\begin{tabular}{l}
|\input{childdoc.def}|\\
|\childdocby{|\textit{main}|}|\\
\end{tabular}
\end{center}
%
The directive |\childdocby| is similar to |\childdocof|
described in \secref{sec:include},
but the subsequent selection of content must be done manually.
To that end, both |\ifchilddoc| and |\ifchilddocmanual|
will be true upon processing of a part,
and the name of the part is stored in |\childdocname|.
Note that |\jobname| will be set to the filename of the current part
so that each part receives an individual |.aux| file
that does not interfere with the |.aux| file(s) of the main document.
This behaviour can be altered by the alternative form
|\childdocby[*]{|\textit{main}|}| (with a non-empty optional argument)
which uses the |.aux| file of the main document
by setting |\jobname| to \textit{main}.

%%%%%%%%%%%%%%%%%%%%%%%%%%%%%%%%%%%%%%%%%%%%%%%%%%%%%%%%%%%%%%%%%%%%%%%%%%%%%%%%
\subsection{Driver Development}
\label{sec:driver}

The \textsf{childdoc} mechanism can also be use for the development
of definition files such as \LaTeX{} styles or classes.
This case differs from the above setup with multiple parts
included by |\include| in that no |\includeonly| should be invoked.
This can be achieved by starting the include file
(before |\ProvidesPackage|) with:
%
\begin{center}
\begin{tabular}{l}
|\input{childdoc.def}|\\
|\childdocforward{|\textit{main}|}|\\
\end{tabular}
\end{center}
%
or alternatively with:
%
\begin{center}
\begin{tabular}{l}
|\input{childdoc.def}|\\
|\childdocby{|\textit{main}|}|\\
\end{tabular}
\end{center}
%
Both forms have slightly different effects as described above.
The main file is prepared as usual, see \secref{sec:include}.

%%%%%%%%%%%%%%%%%%%%%%%%%%%%%%%%%%%%%%%%%%%%%%%%%%%%%%%%%%%%%%%%%%%%%%%%%%%%%%%%
\subsection{Legacy Detection}
\label{sec:detection}

The directive |\childdocmain| in the main file can detect
whether the complete document or merely a child is to be compiled
even without using the directive |\childdocof|.
This method is deprecated because it is less robust
and there is no compelling reason to use it;
it is merely provided for backward compatibility
and it may be removed in future versions.

If the detection mechanism is to be used,
it is mandatory to correctly specify
the filename of the main file as the argument of |\childdocmain|:
%
\begin{center}
\begin{tabular}{l}
|\input{childdoc.def}|\\
|\childdocmain{|\textit{main}|}|\\
\end{tabular}
\end{center}
%
If |\jobname| does not match the argument \textit{main} of |\childdocmain|,
it is assumed that |\jobname| points to the child file to be compiled.
When using |\childdocmain| with the main file specified as argument,
it suffices to start a child file
with just |\input{|\textit{main}|}|
without loading of the package and using |\childdocof|.
If instead all processing is done
with the appropriate \textsf{childdoc} directives,
the argument of \textit{main} of |\childdocmain| can be empty.

An alternative version of the command line processing described
in \secref{sec:commandline} using the detection mechanism reads:
%
\begin{center}
|... -jobname "|\textit{target}|" "|[\textit{flags}]%
[|\def\jobname{|\textit{dest}|}|]|\input{|\textit{main}|}"|
\end{center}

%%%%%%%%%%%%%%%%%%%%%%%%%%%%%%%%%%%%%%%%%%%%%%%%%%%%%%%%%%%%%%%%%%%%%%%%%%%%%%%%
\subsection{Manual Code}
\label{sec:manual}

In case one cannot be certain whether the definitions file |childdoc.def|
is installed on the target \TeX{} distribution
and one prefers not to ship it,
it is conceivable to paste a few relevant commands into the sources.

To that end, drop all statements |\input{childdoc.def}|
and perform the replacements as outlined below.
Instead of |\childdocmain{|\textit{main}|}| add the following code
to the top of the main file:
%
\begin{center}
\begin{tabular}{l}
|\||ifdefined\childdocname\endinput\||fi\newif\ifchilddoc|\\
|\edef\childdocname{\scantokens\expandafter{\jobname\noexpand}}|\\
|\def\childdocmain{|\textit{main}|}\||ifx\childdocmain\childdocname\||else|\\
|\childdoctrue\includeonly{\childdocname}\let\jobname\childdocmain\||fi|\\
\end{tabular}
\end{center}
%
Instead of |\childdocof{|\textit{main}|}| just include the main file
at the top of each child file:
%
\begin{center}
|\input{|\textit{main}|}|
\end{center}
%
A simple redirection |\childdocforward{|\textit{dest}|}| is achieved by:
%
\begin{center}
|\def\jobname{|\textit{dest}|}\input{\jobname}|
\end{center}
%
The redirection with prefix
|\childdocforwardprefix[|\textit{prefix}|]{|\textit{dest}|}|
is accomplished by:
%
\begin{center}
\begin{tabular}{l}
|{\edef\jobname{\scantokens\expandafter{\jobname\noexpand}}|\\
|\def\redirectjob |\textit{prefix}|#1~~~{\gdef\jobname{|\textit{dest}|#1}}|\\
|\expandafter\redirectjob\jobname~~~}\input{\jobname}|
\end{tabular}
\end{center}

In an alternative approach,
child documents can be compiled by a specific command line
without additional code or specific definitions:
%
\begin{center}
|... -jobname "|\textit{target}|" "|[\textit{flags}]%
|\includeonly{|\textit{dest}|}\input{|\textit{main}|}"|
\end{center}
%

%%%%%%%%%%%%%%%%%%%%%%%%%%%%%%%%%%%%%%%%%%%%%%%%%%%%%%%%%%%%%%%%%%%%%%%%%%%%%%%%
%%%%%%%%%%%%%%%%%%%%%%%%%%%%%%%%%%%%%%%%%%%%%%%%%%%%%%%%%%%%%%%%%%%%%%%%%%%%%%%%
\section{Information}

%%%%%%%%%%%%%%%%%%%%%%%%%%%%%%%%%%%%%%%%%%%%%%%%%%%%%%%%%%%%%%%%%%%%%%%%%%%%%%%%
\subsection{Copyright}

Copyright \copyright{} 2017--2018 Niklas Beisert

This work may be distributed and/or modified under the
conditions of the \LaTeX{} Project Public License, either version 1.3
of this license or (at your option) any later version.
The latest version of this license is in
  \url{http://www.latex-project.org/lppl.txt}
and version 1.3 or later is part of all distributions of \LaTeX{}
version 2005/12/01 or later.

This work has the LPPL maintenance status `maintained'.

The Current Maintainer of this work is Niklas Beisert.

This work consists of the files |README.txt|, |childdoc.ins| and |childdoc.dtx|
as well as the derived files |childdoc.def|, |cdocsamp.tex|
with |cdocsch1.tex|, |cdocsch2.tex|, |cdocspt3.tex|, |cdocspt4.tex|,
|cdocsdrf.tex|, |cdocsfn1.tex|, |cdocsfn2.tex|
as well as |childdoc.pdf|.

%%%%%%%%%%%%%%%%%%%%%%%%%%%%%%%%%%%%%%%%%%%%%%%%%%%%%%%%%%%%%%%%%%%%%%%%%%%%%%%%
\subsection{Files and Installation}

The package consists of the files:
%
\begin{center}
\begin{tabular}{ll}
    |README.txt|   & readme file \\
    |childdoc.ins| & installation file \\
    |childdoc.dtx| & source file \\
    |childdoc.def| & definition file \\
    |cdocsamp.tex| & sample main file \\
    |cdocsch1.tex| & sample include file \\
    |cdocsch2.tex| & sample include file \\
    |cdocspt3.tex| & sample part file \\
    |cdocspt4.tex| & sample part file \\
    |cdocsdrf.tex| & sample redirection file \\
    |cdocsfn1.tex| & sample redirection file \\
    |cdocsfn2.tex| & sample redirection file \\
    |childdoc.pdf| & manual
\end{tabular}
\end{center}
%
The distribution consists of the files
|README.txt|, |childdoc.ins| and |childdoc.dtx|.
%
\begin{itemize}
\item
Run (pdf)\LaTeX{} on |childdoc.dtx|
to compile the manual |childdoc.pdf| (this file).
\item
Run \LaTeX{} on |childdoc.ins| to create the definitions file |childdoc.def|
and the sample |cdocsamp.tex| with include files
|cdocsch1.tex|, |cdocsch2.tex|, |cdocspt3.tex|, |cdocspt4.tex|,
|cdocsdrf.tex|, |cdocsfn1.tex|, |cdocsfn2.tex|.
Then copy the file |childdoc.def| to an appropriate directory of your \LaTeX{}
distribution, e.g.\ \textit{texmf-root}|/tex/latex/childdoc|.
\end{itemize}

%%%%%%%%%%%%%%%%%%%%%%%%%%%%%%%%%%%%%%%%%%%%%%%%%%%%%%%%%%%%%%%%%%%%%%%%%%%%%%%%
\subsection{Related CTAN Packages}

There are several other packages which offer a similar functionality:
%
\begin{itemize}
\item
The packages
\href{http://ctan.org/pkg/docmute}{\textsf{docmute}},
\href{http://ctan.org/pkg/includex}{\textsf{includex}} and
\href{http://ctan.org/pkg/standalone}{\textsf{standalone}}
provide commands to include only the document body of
a child file thus allowing both files to be compiled individually.
\item
The packages \href{http://ctan.org/pkg/subdocs}{\textsf{subdocs}}
and \href{http://ctan.org/pkg/subfiles}{\textsf{subfiles}}
provide structures in which the main and child documents can be
encapsulated and allowing them to be compiled individually.
The inclusion mechanism is different from the conventional |\include|.
\item
The package \href{http://ctan.org/pkg/combine}{\textsf{combine}}
is an elaborate solution to combine several documents into one.
\end{itemize}
%
See also the CTAN topic \href{http://ctan.org/topic/subdocs}{\textsf{subdocs}}
for further related packages.
The present package differs from the above solutions in that
a document structure constructed with the conventional |\include| mechanism
just needs two extra commands at the top of every file
such that all constituent files can be compiled individually.

%%%%%%%%%%%%%%%%%%%%%%%%%%%%%%%%%%%%%%%%%%%%%%%%%%%%%%%%%%%%%%%%%%%%%%%%%%%%%%%%
%\subsection{Feature Suggestions}
%
%The following is a list of features which may be useful for future
%versions of this package:
%%
%\begin{itemize}
%\item
%\ldots
%\end{itemize}

%%%%%%%%%%%%%%%%%%%%%%%%%%%%%%%%%%%%%%%%%%%%%%%%%%%%%%%%%%%%%%%%%%%%%%%%%%%%%%%%
\subsection{Revision History}

%%%%%%%%%%%%%%%%%%%%%%%%%%%%%%%%%%%%%%%%
\paragraph{v2.0:} 2018/12/30

\begin{itemize}
\item
immediate forward processing
\item
added |\childdocby| mechanism
\item
manual restructured
\end{itemize}

%%%%%%%%%%%%%%%%%%%%%%%%%%%%%%%%%%%%%%%%
\paragraph{v1.6:} 2018/01/17

\begin{itemize}
\item
application for development of include files
\item
corrections to manual
\end{itemize}

%%%%%%%%%%%%%%%%%%%%%%%%%%%%%%%%%%%%%%%%
\paragraph{v1.5:} 2017/05/21

\begin{itemize}
\item
more complete structuring introduced
\item
|\childdocof| introduced
\item
|\childdoc| renamed to |\childdocmain|
\item
|\childredirect| renamed to |\childdocforward| and |\childdocforwardprefix|
and functionality expanded
\end{itemize}

%%%%%%%%%%%%%%%%%%%%%%%%%%%%%%%%%%%%%%%%
\paragraph{v1.0:} 2017/04/27

\begin{itemize}
\item
manual and install package
\item
first version published on CTAN
\end{itemize}

%%%%%%%%%%%%%%%%%%%%%%%%%%%%%%%%%%%%%%%%
\paragraph{v0.6:} 2017/04/26

\begin{itemize}
\item
redirection mechanism added
\end{itemize}

%%%%%%%%%%%%%%%%%%%%%%%%%%%%%%%%%%%%%%%%
\paragraph{v0.5:} 2017/04/26

\begin{itemize}
\item
functionality in definition file
\end{itemize}


%%%%%%%%%%%%%%%%%%%%%%%%%%%%%%%%%%%%%%%%%%%%%%%%%%%%%%%%%%%%%%%%%%%%%%%%%%%%%%%%
%%%%%%%%%%%%%%%%%%%%%%%%%%%%%%%%%%%%%%%%%%%%%%%%%%%%%%%%%%%%%%%%%%%%%%%%%%%%%%%%
%%%%%%%%%%%%%%%%%%%%%%%%%%%%%%%%%%%%%%%%%%%%%%%%%%%%%%%%%%%%%%%%%%%%%%%%%%%%%%%%
\appendix

\settowidth\MacroIndent{\rmfamily\scriptsize 000\ }

 \DocInput{childdoc.dtx}

\end{document}
%</driver>
% \fi
%
% %%%%%%%%%%%%%%%%%%%%%%%%%%%%%%%%%%%%%%%%%%%%%%%%%%%%%%%%%%%%%%%%%%%%%%%%%%%%%%
% %%%%%%%%%%%%%%%%%%%%%%%%%%%%%%%%%%%%%%%%%%%%%%%%%%%%%%%%%%%%%%%%%%%%%%%%%%%%%%
% \section{Sample}
%\iffalse
%<*samplemain>
%\fi
%
% The following presents a sample document
% with two chapters, two parts, a title page,
% a compile flag as well as three forwarding files to set the flag.
% It consists of eight |.tex| files:
% \begin{center}
% \begin{tabular}{ll}
% |cdocsamp.tex|&main file\\
% |cdocsch1.tex|&include file for chapter 1\\
% |cdocsch2.tex|&include file for chapter 2\\
% |cdocspt3.tex|&include file for part 3\\
% |cdocspt4.tex|&include file for part 4\\
% |cdocsdrf.tex|&forwarding file for main file in draft mode\\
% |cdocsfi1.tex|&forwarding file for final version of chapter 1\\
% |cdocsfi2.tex|&forwarding file for final version of chapter 2\\
% \end{tabular}
% \end{center}
% Each of the eight files can be compiled directly by the \LaTeX{} compiler.
%
% %%%%%%%%%%%%%%%%%%%%%%%%%%%%%%%%%%%%%%
% \paragraph{Main File.}
%
% The main file is called |cdocsamp.tex|.
%
% Load the \textsf{childdoc} definitions and
% declare the filename for the main document:
%    \begin{macrocode}
\input{childdoc.def}
\childdocmain{}
%    \end{macrocode}

% Optional override for |\version| flag:
%    \begin{macrocode}
%%\ifchilddoc\else\providecommand{\version}{draft}\fi
%    \end{macrocode}

% Define the default values for the |\version| flag
% (|final| for the main file and |draft| for childs):
%    \begin{macrocode}
\ifchilddoc
\providecommand{\version}{draft}
\else
\providecommand{\version}{final}
\fi
%    \end{macrocode}

% Load the standard document class:
%    \begin{macrocode}
\documentclass[12pt]{article}
%    \end{macrocode}

% Start the document body:
%    \begin{macrocode}
\begin{document}
%    \end{macrocode}

% Declare a title page.
% Print title, part of document being processed and version flag:
%    \begin{macrocode}
\addtocounter{page}{-1}
\begin{center}
{\LARGE\bfseries{}childdoc example\par}
\vspace{1cm}
\ifchilddoc
\ifchilddocmanual part\else chapter\fi:
`\childdocname' of `\childdocjob'\par
\else
main document: `\childdocjob'\par
\fi
version: \version\par
\end{center}
\newpage
%    \end{macrocode}

% Manually include selected file,
% otherwise process as usual:
%    \begin{macrocode}
\ifchilddocmanual
\section*{part `\childdocname'}
\input{\childdocname}
\else
%    \end{macrocode}

% Include the two chapters:
%    \begin{macrocode}
\include{cdocsch1}
\include{cdocsch2}
%    \end{macrocode}

% Include the two parts unless only chapters should be displayed:
%    \begin{macrocode}
\ifchilddoc\else
\section{part three}
\input{cdocspt3}
\section{part four}
\input{cdocspt4}
\fi
%    \end{macrocode}

% Process as usual until here:
%    \begin{macrocode}
\fi
%    \end{macrocode}

% End of document body:
%    \begin{macrocode}
\end{document}
%    \end{macrocode}
%\iffalse
%</samplemain>
%\fi
%
% %%%%%%%%%%%%%%%%%%%%%%%%%%%%%%%%%%%%%%
% \paragraph{Chapter Include Files.}
%
% The include files are called |cdocsch1.tex| and |cdocsch2.tex|.
%
%\iffalse
%<*samplechap1|samplechap2>
%\fi

% Optional override for |\version| flag:
%    \begin{macrocode}
%%\providecommand{\version}{final}
%    \end{macrocode}

% Include the main document:
%    \begin{macrocode}
\input{childdoc.def}
\childdocof{cdocsamp}
%    \end{macrocode}

%\iffalse
%</samplechap1|samplechap2>
%\fi
%
%\iffalse
%<*samplechap1>
%\fi
% Some text for chapter 1:
%    \begin{macrocode}
\section{one}
some text in chapter one
%    \end{macrocode}

%\iffalse
%</samplechap1>
%\fi
% Some text for chapter 2:
%\iffalse
%<*samplechap2>
%\fi
%    \begin{macrocode}
\section{two}
more text in chapter two
%    \end{macrocode}

%\iffalse
%</samplechap2>
%\fi
%
% %%%%%%%%%%%%%%%%%%%%%%%%%%%%%%%%%%%%%%
% \paragraph{Part Include Files.}
%
% The include files are called |cdocspt3.tex| and |cdocspt4.tex|.
%
%\iffalse
%<*samplepart3|samplepart4>
%\fi

% Optional override for |\version| flag:
%    \begin{macrocode}
%%\providecommand{\version}{final}
%    \end{macrocode}

% Include the main document:
%    \begin{macrocode}
\input{childdoc.def}
\childdocby{cdocsamp}
%    \end{macrocode}

%\iffalse
%</samplepart3|samplepart4>
%\fi
%
%\iffalse
%<*samplepart3>
%\fi
% Some text for part 3:
%    \begin{macrocode}
some text in part three
%    \end{macrocode}

%\iffalse
%</samplepart3>
%\fi
% Some text for part 4:
%\iffalse
%<*samplepart4>
%\fi
%    \begin{macrocode}
more text in part four
%    \end{macrocode}

%\iffalse
%</samplepart4>
%\fi
%
% %%%%%%%%%%%%%%%%%%%%%%%%%%%%%%%%%%%%%%
% \paragraph{Forwarding for a Complete Draft.}
%
% The following forwarding file |cdocsdrf.tex|
% compiles the main document in draft mode:
%\iffalse
%<*sampledraft>
%\fi
%    \begin{macrocode}
\def\version{draft}
\input{childdoc.def}
\childdocforward{cdocsamp}
%    \end{macrocode}

%\iffalse
%</sampledraft>
%\fi
%
% %%%%%%%%%%%%%%%%%%%%%%%%%%%%%%%%%%%%%%
% \paragraph{Forwarding for Final Version of the Chapters.}
%
% The following forwarding files |cdocsfn1.tex| and |cdocsfn2.tex|
% (with identical content)
% compile the final versions of the child documents
% |cdocsch1.tex| and |cdocsch2.tex|, respectively:
%\iffalse
%<*samplefinal>
%\fi
%    \begin{macrocode}
\def\version{final}
\input{childdoc.def}
\childdocforwardprefix[cdocsamp]{cdocsfn}{cdocsch}
%    \end{macrocode}

%\iffalse
%</samplefinal>
%\fi
%
% %%%%%%%%%%%%%%%%%%%%%%%%%%%%%%%%%%%%%%
% \paragraph{Command Line Processing.}
%
% The following three command lines generate the output files
% |cdocscld|, |cdocscl1| and |cdocscl2|
% which should be identical to
% |cdocsdrf|, |cdocsch1| and |cdocsfn2|, respectively:
% \begin{center}
% \begin{tabular}{l}
% |latex -jobname cdocscld \|\\
% |  "\def\version{draft}\input{childdoc.def}\childdocforward{cdocsamp}"|\\
% |latex -jobname cdocscl1 \|\\
% |  "\input{childdoc.def}\childdocforward[cdocsamp]{cdocsch1}"|\\
% |latex -jobname cdocscl2 \|\\
% |  "\def\version{final}\input{childdoc.def}\childdocforward{cdocsch2}"|
% \end{tabular}
% \end{center}
% Note that the trailing backslash on each first line
% merely continues the input to the second line
% (for convenient cut ant paste).
% Furthermore, the command |latex| can be replaced by any
% of its alternative versions such as |pdflatex|.
%
% %%%%%%%%%%%%%%%%%%%%%%%%%%%%%%%%%%%%%%%%%%%%%%%%%%%%%%%%%%%%%%%%%%%%%%%%%%%%%%
% %%%%%%%%%%%%%%%%%%%%%%%%%%%%%%%%%%%%%%%%%%%%%%%%%%%%%%%%%%%%%%%%%%%%%%%%%%%%%%
% \section{Implementation}
%\iffalse
%<*package>
%\fi
%
% This section describes the definitions file |childdoc.def|.

% The definitions cannot be loaded using |\usepackage| or |\RequirePackage|
% which has a mechanism to prevent loading a style file more than once.
% When loading the definitions by means of |\input|
% multiple instances have to be prevented manually:
%\iffalse
%This code needs to be before the `\ProvidesFile' directive
%which is defined at the beginning of this file.
%Therefore it is also placed there and commented out here.
%</package>
%<*discard>
%\fi
%    \begin{macrocode}
\ifdefined\childdocmain\endinput\fi
%    \end{macrocode}
%\iffalse
%</discard>
%<*package>
%\fi
%
% \macro{\ifchilddoc}
% \macro{\ifchilddocmanual}
% The conditional |\ifchilddoc| tells whether a
% child (true) or main (false) document is being compiled.
% The conditional |\ifchilddocmanual| tells whether
% the |\includeonly| mechanism is used (false) or
% the selection of child files must be performed manually (true).
% The definitions initialise to false:
%    \begin{macrocode}
\newif\ifchilddoc
\newif\ifchilddocmanual
%    \end{macrocode}

% \macro{\childdocname}
% \macro{\childdocjob}
% The macro |\childdocname| stores the name of the main document
% to be compiled. The macro |\childdocjob| stores the name of
% the document on which the \LaTeX{} compiler was originally invoked.
% The content of |\jobname| cannot be compared
% to filenames specified in the source due to different catcodes.
% The following code rescans |\jobname|, stores the result
% in |\childdocname| and saves a copy in |\childdocjob|:
%    \begin{macrocode}
\edef\childdocname{\scantokens\expandafter{\jobname\noexpand}}
\let\childdocjob\childdocname
%    \end{macrocode}

% \macro{\childdocdisable}
% The macro |\childdocdisable| prevents the main file
% from being processed more than once.
% At this stage, the main document command |\childdocmain|
% is assumed to be called once again where it should do nothing.
% Any subsequent call to it should prevent
% a secondary processing of the main document
% It overwrites the forwarding commands
% |\childdocof| and |\childdocforward|
% with empty macros to prevent further inclusions of the main document:
%    \begin{macrocode}
\newcommand{\childdocdisable}
{
  \renewcommand{\childdocmain}[1]{\renewcommand{\childdocmain}[1]{\endinput}}
  \renewcommand{\childdocof}[1]{}
  \renewcommand{\childdocby}[2][]{}
  \renewcommand{\childdocforward}[2][]{}
  \renewcommand{\childdocdisable}{}
}
%    \end{macrocode}

% \macro{\childdocmain}
% The macro |\childdocmain| is to be called at the top of the main file
% with nothing or the main filename (without extension) as argument.
% First, it breaks loops.
% If the argument is not empty and does not match |\childdocname|
% (which is set by the first inclusion of |childdoc.def|),
% |\ifchilddoc| is set to true, |\includeonly| is applied to the child file
% and |\jobname| is set to the main file
% (for proper handling of |.aux| files):
%    \begin{macrocode}
\newcommand{\childdocmain}[1]
{
  \childdocdisable\childdocmain{}
  \if?#1?\else
    \begingroup
      \def\childdoctmp{#1}
      \ifx\childdoctmp\childdocname
        \def\childdoctmp{}
      \else
        \def\childdoctmp
        {
          \childdoctrue
          \includeonly{\childdocname}
          \def\childdocjob{#1}
          \def\jobname{#1}
        }
      \fi
      \expandafter
    \endgroup
    \childdoctmp
  \fi
}
%    \end{macrocode}

% \macro{\childdocof}
% The command |\childdocof| redirects
% compilation to the main file |#1|.
%    \begin{macrocode}
\newcommand{\childdocof}[1]
{
  \childdocdisable
  \childdoctrue
  \includeonly{\childdocname}
  \def\jobname{#1}
  \def\childdocjob{#1}
  \input{#1}
}
%    \end{macrocode}

% \macro{\childdocby}
% The command |\childdocby| ....
%    \begin{macrocode}
\newcommand{\childdocby}[2][]
{
  \childdocdisable
  \childdoctrue
  \childdocmanualtrue
  \if?#1?\else
    \def\jobname{#2}
  \fi
  \def\childdocjob{#2}
  \input{#2}
  \endinput
}
%    \end{macrocode}

% \macro{\childdocforward}
% The command |\childdocforward| redirects
% compilation to the main file or
% (if the optional argument is given) a child file.
% Parameters are set as if the main file
% or a child file starting with |\childdocof| was compiled.
% Then compilation is handed over to the main file:
%    \begin{macrocode}
\newcommand{\childdocforward}[2][]
{
  \begingroup
    \if?#1?
      \def\childdoctmp
      {
        \def\childdocname{#2}
        \def\childdocjob{#2}
        \def\jobname{#2}
        \input{#2}
        \endinput
      }
    \else
      \def\childdoctmp
      {
        \childdocdisable
        \def\childdocname{#2}
        \childdoctrue
        \includeonly{#2}
        \def\childdocjob{#1}
        \def\jobname{#1}
        \input{#1}
        \endinput
      }
    \fi
    \expandafter
  \endgroup
  \childdoctmp
}
%    \end{macrocode}

% \macro{\childdocforwardprefix}
% The command |\childdocforwardprefix| redirects
% compilation to the main or a child file by means of a pattern.
% The prefix |#1| in the current filename is replaced by |#2|
% and the suffix of the current filename is kept
% (it is assumed that the filename does not contain the substring `|~~~|'
% which is used as a delimiter).
% Compilation is handed over to the new file by |\childdocforward|:
%    \begin{macrocode}
\newcommand{\childdocforwardprefix}[3][]
{
  \begingroup
    \def\childdocextract #2##1~~~{\def\childdoctmp{\childdocforward[#1]{#3##1}}}
    \expandafter\childdocextract\childdocname~~~
    \expandafter
  \endgroup
  \childdoctmp
}
%    \end{macrocode}

% \macro{\childdoc}
% The deprecated macro |\childdoc| is a legacy version of |\childdocmain|:
%    \begin{macrocode}
\newcommand{\childdoc}{\childdocmain}
%    \end{macrocode}

% \macro{\childdocredirect}
% The deprecated macro |\childdocredirect| is a legacy version
% of |\childdocforward| and |\childdocforwardprefix|:
%    \begin{macrocode}
\newcommand{\childdocredirect}[2][]
{
  \begingroup
    \if?#1?
      \def\childdoctmp{\childdocforward{#2}}
    \else
      \def\childdoctmp{\childdocforwardprefix{#1}{#2}}
    \fi
    \expandafter
  \endgroup
  \childdoctmp
}
%    \end{macrocode}

%\iffalse
%</package>
%\fi
%
\endinput
\childdocforward{cdocsamp}"|\\
% |latex -jobname cdocscl1 \|\\
% |  "% \iffalse
%
% childdoc.dtx Copyright (C) 2017-2018 Niklas Beisert
%
% This work may be distributed and/or modified under the
% conditions of the LaTeX Project Public License, either version 1.3
% of this license or (at your option) any later version.
% The latest version of this license is in
%   http://www.latex-project.org/lppl.txt
% and version 1.3 or later is part of all distributions of LaTeX
% version 2005/12/01 or later.
%
% This work has the LPPL maintenance status `maintained'.
%
% The Current Maintainer of this work is Niklas Beisert.
%
% This work consists of the files childdoc.dtx and childdoc.ins
% and the derived files childdoc.def and cdocsamp.tex with
% cdocsch1.tex, cdocsch2.tex, cdocsdrf.tex, cdocsfn1.tex, cdocsfn2.tex.
%
%<package>\ifdefined\childdocmain\endinput\fi
%<package>\ProvidesFile{childdoc.def}[2018/12/30 v2.0 child document driver]
%<samplemain>\ProvidesFile{cdocsamp.tex}[2018/12/30 v2.0 sample for childdoc]
%<*driver>
%\ProvidesFile{childdoc.drv}[2018/12/30 v2.0 childdoc reference manual file]
\PassOptionsToClass{10pt,a4paper}{article}
\documentclass{ltxdoc}

\usepackage[margin=35mm]{geometry}
\usepackage{hyperref}
\usepackage{hyperxmp}
\usepackage[usenames]{color}

\hypersetup{colorlinks=true}
\hypersetup{pdfstartview=FitH}
\hypersetup{pdfpagemode=UseNone}
\hypersetup{pdfsource={}}
\hypersetup{pdflang={en-UK}}
\hypersetup{pdfcopyright={Copyright 2017-2018 Niklas Beisert.
  This work may be distributed and/or modified under the
  conditions of the LaTeX Project Public License, either version 1.3
  of this license or (at your option) any later version.}}
\hypersetup{pdflicenseurl={http://www.latex-project.org/lppl.txt}}
\hypersetup{pdfcontactaddress={ETH Zurich, ITP, HIT K,
  Wolfgang-Pauli-Strasse 27}}
\hypersetup{pdfcontactpostcode={8093}}
\hypersetup{pdfcontactcity={Zurich}}
\hypersetup{pdfcontactcountry={Switzerland}}
\hypersetup{pdfcontactemail={nbeisert@itp.phys.ethz.ch}}
\hypersetup{pdfcontacturl={http://people.phys.ethz.ch/\xmptilde nbeisert/}}

\newcommand{\secref}[1]{\hyperref[#1]{section \ref*{#1}}}

\parskip1ex
\parindent0pt
\let\olditemize\itemize
\def\itemize{\olditemize\parskip0pt}

\begin{document}

\title{The \textsf{childdoc} Package}
\hypersetup{pdftitle={The childdoc Package}}
\author{Niklas Beisert\\[2ex]
  Institut f\"ur Theoretische Physik\\
  Eidgen\"ossische Technische Hochschule Z\"urich\\
  Wolfgang-Pauli-Strasse 27, 8093 Z\"urich, Switzerland\\[1ex]
  \href{mailto:nbeisert@itp.phys.ethz.ch}
  {\texttt{nbeisert@itp.phys.ethz.ch}}}
\hypersetup{pdfauthor={Niklas Beisert}}
\hypersetup{pdfsubject={Manual for the LaTeX2e Package childdoc}}
\date{30 December 2018, \textsf{v2.0}}
\maketitle

\begin{abstract}\noindent
\textsf{childdoc} is a \LaTeXe{} package
that enables the direct compilation
of document sections included by |\include|
to individual files.
\end{abstract}

\begingroup
\parskip0ex
\tableofcontents
\endgroup

%%%%%%%%%%%%%%%%%%%%%%%%%%%%%%%%%%%%%%%%%%%%%%%%%%%%%%%%%%%%%%%%%%%%%%%%%%%%%%%%
%%%%%%%%%%%%%%%%%%%%%%%%%%%%%%%%%%%%%%%%%%%%%%%%%%%%%%%%%%%%%%%%%%%%%%%%%%%%%%%%
\section{Introduction}

\LaTeX{} provides a mechanism to structure a large document (such as a book)
into a main file and several child files (containing the chapters)
using the |\include| command.
This mechanism is beneficial for documents
which span hundreds of pages in order to
make the source file(s) more manageable.
Moreover, compilation can be restricted to
selected child files by means of the |\includeonly| command.
The latter feature can be used to reduce the compilation time while editing
(this was significantly more useful in the earlier days of \LaTeX{})
or to generate a smaller document which is easier to navigate.
Another application of |\includeonly| is to generate
documents consisting of selected parts of the complete document.

However, there are a few drawbacks of the plain |\include| mechanism:
\begin{itemize}
\item
The child files cannot be compiled on their own,
they can only be compiled via the main file.
A naive editing environment
(such as a text editor with an option
to have the current file processed by \LaTeX)
may require one to switch to the main file before compiling;
attempting to compile the child file produces errors.
\item
The main file must be modified (each time)
to adjust the |\includeonly| command
to the present needs. This easily leaves the main file in a messy state.
\item
The generated document will always carry the filename
of the main document. This is inconvenient if
several child files are to be compiled and
to be kept for distribution.
\end{itemize}

The present package provides a simple interface
to make child files individually compilable by \LaTeX{}.
Compiling a child file then has the same effect as compiling
the main file with an |\includeonly| command
to select the appropriate child.
Moreover the generated document will carry the name of the child
rather than the main file.
This resolves all three above issues.

This feature is meant to make the editing of books,
thesis documents and lecture notes somewhat more convenient.
However, the package can also be used efficiently for
composing a series of documents (such as exercise sheets)
which are typically distributed individually.
It then assists the author in generating the individual documents
(potentially in different versions)
as well as a document containing the collected series.
Another application is in developing style files
or other kinds of included material
where compilation of the style file could redirect
to a sample or test file.

%%%%%%%%%%%%%%%%%%%%%%%%%%%%%%%%%%%%%%%%%%%%%%%%%%%%%%%%%%%%%%%%%%%%%%%%%%%%%%%%
%%%%%%%%%%%%%%%%%%%%%%%%%%%%%%%%%%%%%%%%%%%%%%%%%%%%%%%%%%%%%%%%%%%%%%%%%%%%%%%%
\section{Usage}

First of all, the package \textsf{childdoc} is \emph{not} a standard
\LaTeXe{} |.sty| style file! Therefore it needs to be invoked in
a non-standard way.

%%%%%%%%%%%%%%%%%%%%%%%%%%%%%%%%%%%%%%%%%%%%%%%%%%%%%%%%%%%%%%%%%%%%%%%%%%%%%%%%
\subsection{Included Files}
\label{sec:include}

%%%%%%%%%%%%%%%%%%%%%%%%%%%%%%%%%%%%%%%%
\DescribeMacro{\childdocmain}
To use the package, add the commands
\begin{center}
\begin{tabular}{l}
|\input{childdoc.def}|\\
|\childdocmain{}|\\
\end{tabular}
\end{center}
at the very top of the main \LaTeX{} file,
in particular \emph{before} the |\documentclass| statement!
The argument of |\childdocmain| should be left empty
(but it must be present).

%%%%%%%%%%%%%%%%%%%%%%%%%%%%%%%%%%%%%%%%
\DescribeMacro{\childdocof}
Furthermore, add the commands
\begin{center}
\begin{tabular}{l}
|\input{childdoc.def}|\\
|\childdocof{|\textit{main}|}|\\
\end{tabular}
\end{center}
at the top of every child file \textit{child}
which is included by |\include{|\textit{child}|}|
from within the main file
(or at least for those files to be compiled individually).
The argument \textit{main} must be the filename of the main file.

There are a couple of
considerations in setting up the main and child documents:

%%%%%%%%%%%%%%%%%%%%%%%%%%%%%%%%%%%%%%%%
\paragraph{Restrictions.}

Please note the following restrictions:
\begin{itemize}
\item
|\childdocmain| must be called with one argument \textit{main}
to ensure compatibility with earlier version of the package.
It must either be empty (|\childdocmain{}|)
or precisely match the filename of the main file in which it is specified.
See \secref{sec:detection} for further information.
\item
The filename \textit{main} must be specified without the |.tex| extension.
\item
The filename \textit{main} is case sensitive
(even in case-insensitive file systems)
due to internal string comparison.
\item
The argument \textit{main} should be fully expanded, it cannot be a macro.
\item
Subdirectories and special characters should be avoided in filenames.
\item
The command |\childdocmain{|\textit{main}|}| must be followed by a whitespace.
It should not be followed immediately by another command
or by a comment mark `|%|'.
This is because the \TeX{} parser reads the token immediately following
the argument of |\childdocmain| and puts it
at the beginning of every child section;
however, a white\-space is ignored.
\end{itemize}

%%%%%%%%%%%%%%%%%%%%%%%%%%%%%%%%%%%%%%%%
\paragraph{Content of Main File.}

It is advisable to place all content in the child files included by |\include|.
Any output contained in the main file will appear in all child documents
unless suppressed manually;
it cannot be suppressed automatically by the |\includeonly| directive
and thus should normally be avoided.
A method to include some content in the main file
by means of conditional processing is described in \secref{sec:conditional}.

%%%%%%%%%%%%%%%%%%%%%%%%%%%%%%%%%%%%%%%%
\paragraph{Page Numbering.}

When only a part of the document is compiled,
the appropriate numbering of pages
(as well as other status parameters)
is determined from the |.aux| files.
The latter contain information from previous passes.
However this information needs to propagate through
all intermediate child documents.
Therefore the page numbering in child documents may well
be inconsistent until the complete document is compiled at least once.

A useful (if unconventional) way to always ensure a consistent
page numbering is to restart the numbering in each child document
and denote the pages by `\textit{child}|.|\textit{page}'
where \textit{child} represents the chapter/section number of the child file.
This can be achieved by the command
|\numberwithin{page}{|\textit{child}|}|
of the \textsf{amsmath} package
where \textit{child} can be |chapter| or |section|
depending on the chosen structuring.
Alternatively, one can modify the macro |\thepage| appropriately
and reset the counter |page| at the start of each child file.

%%%%%%%%%%%%%%%%%%%%%%%%%%%%%%%%%%%%%%%%%%%%%%%%%%%%%%%%%%%%%%%%%%%%%%%%%%%%%%%%
\subsection{Conditional Processing}
\label{sec:conditional}

The package provides a mechanism to compile different versions
of a document. To customise the versions further some conditional processing
can come in handy to distinguish which version is being compiled.
The package provides two macros to describe the compilation context:

%%%%%%%%%%%%%%%%%%%%%%%%%%%%%%%%%%%%%%%%
\DescribeMacro{\ifchilddoc}
The conditional |\ifchilddoc| distinguishes between the compilation of
child documents and the main document:
%
\begin{center}
|\ifchilddoc |\textit{child-code}| |[|\||else |\textit{main-code}]| \||fi|
\end{center}

%%%%%%%%%%%%%%%%%%%%%%%%%%%%%%%%%%%%%%%%
\DescribeMacro{\childdocname}
\DescribeMacro{\childdocjob}
The macro |\childdocname| contains the filename (without extension)
of the main or child file being processed.
Note that |\childdocjob| will always contain the name of the main file.

%%%%%%%%%%%%%%%%%%%%%%%%%%%%%%%%%%%%%%%%
\paragraph{Title Page.}

Conditional processing can be used to include a title or banner page
in the main document when proper precautions are taken.
Importantly, the code in the main file should ensure that the page counter
(as well as other status parameters which are stored in the |.aux| files)
takes the same value after the conditional processing.
Otherwise the page numbers may take divergent values
depending on which part is compiled.

For example, a title page could be declared by:
%
\begin{center}
\begin{tabular}{l}
|\ifchilddoc\||else|\\
|\addtocounter{page}{-1}|\\
\textit{code for title page}\\
|\newpage|\\
|\||fi|
\end{tabular}
\end{center}
%
A banner page for the child documents can be generated by:
%
\begin{center}
\begin{tabular}{l}
|\ifchilddoc|\\
|\addtocounter{page}{-1}|\\
\textit{code for banner page}\\
|\newpage|\\
|\||fi|
\end{tabular}
\end{center}
%
Here one could write a message such as:
\begin{center}
|This is the part \childdocname{} of \childdocjob{}.|
\end{center}

%%%%%%%%%%%%%%%%%%%%%%%%%%%%%%%%%%%%%%%%%%%%%%%%%%%%%%%%%%%%%%%%%%%%%%%%%%%%%%%%
\subsection{Flags}
\label{sec:flags}

The package makes it easy to generate different versions
of the main or child documents.
To this end compilation flags can be defined
and assigned different default values.
They will be particularly useful in conjunction
with the forwarding mechanism described in \secref{sec:forward}.

For example, it may be useful to have a flag |\version|
which can be set to |draft| or |final|.
The document source will contain some conditional code
depending on the value of |\version|.
Suppose further, the flag should default to |final| for the main file
and to |draft| for child files
which is a natural assignment for editing the document.
This is achieved by placing the following code
in the preamble of the main document
(below the |\childdocmain| directive):
%
\begin{center}
\begin{tabular}{l}
|\ifchilddoc|\\
|\providecommand{\version}{draft}|\\
|\||else|\\
|\providecommand{\version}{final}|\\
|\||fi|
\end{tabular}
\end{center}
%
The definition by |\providecommand| makes sure
that previous definitions are not overwritten.
Further statements |\providecommand{\version}{...}|
can thus be added before the above code to override it.

For the main file, one might add a line
(between |\childdocmain| and the above block)
%
\begin{center}
|%\ifchilddoc\||else\providecommand{\version}{draft}\||fi|
\end{center}
%
which can be uncommented to produce a draft version.
Likewise one can add a line to the very top of a child file
(above the |\childdocof{|\textit{main}|}| directive)
%
\begin{center}
|%\providecommand{\version}{final}|
\end{center}
%
which can be uncommented to produce the final version of this child document.

%%%%%%%%%%%%%%%%%%%%%%%%%%%%%%%%%%%%%%%%%%%%%%%%%%%%%%%%%%%%%%%%%%%%%%%%%%%%%%%%
\subsection{Forwarding}
\label{sec:forward}

Different versions of the main or child documents
using compilation flags as described in \secref{sec:flags}
can be (permanently) stored in different files
for convenient compilation, viewing and distribution.
To this end, the package defines a command
to pass on compilation to a different file:

%%%%%%%%%%%%%%%%%%%%%%%%%%%%%%%%%%%%%%%%
\DescribeMacro{\childdocforward}
The command |\childdocforward| redirects processing to
another source file:
%
\begin{center}
\begin{tabular}{l}
|\input{childdoc.def}|\\
|\childdocforward[|\textit{main}|]{|\textit{dest}|}|\\
\end{tabular}
\end{center}
%
The argument \textit{dest} is the destination file
(without extension).
It should be the main file or one of the child files.
Note that further \textsf{childdoc} directives
such as |\childdocof| and |\childdocforward|
in the indicated file will be processed in this form.
The optional argument \textit{main}
passes on directly to the main file \textit{main}
while pretending to compile the child \textit{dest}.
This form behaves as if \textit{dest}
issues |\childdocof{|\textit{main}|}| right away,
and no further \textsf{childdoc} directives will be processed.

%%%%%%%%%%%%%%%%%%%%%%%%%%%%%%%%%%%%%%%%
\DescribeMacro{\...prefix}
In the alternative form |\childdocforwardprefix|,
%
\begin{center}
\begin{tabular}{l}
|\input{childdoc.def}|\\
|\childdocforwardprefix[|\textit{main}|]{|\textit{prefix}|}{|\textit{dest}|}|
\end{tabular}
\end{center}
%
the destination file is determined by a pattern
depending on the current file:
To make this work, the current file must be called
`{\textit{prefix}\hspace{0.2em}\textit{suffix}}'
with \textit{prefix} matching precisely the argument.
Processing is then passed on to the file
`{\textit{dest}\hspace{0.2em}\textit{suffix}}'.
Surely, the same effect is achieved by
directly specifying the
argument `{\textit{dest}\hspace{0.2em}\textit{suffix}}'
in the first form.
However, that requires to set up a different file
for each child. With the alternative form of the command
all these files can have exactly the same content
which simplifies setting them up and maintaining them.

For example, the following file |draft.tex|
with a compilation flag |\version| as described in \secref{sec:flags}
compiles the main document as a draft:
%
\begin{center}
\begin{tabular}{l}
|\def\version{draft}|\\
|\input{childdoc.def}|\\
|\childdocforward{|\textit{main}|}|
\end{tabular}
\end{center}
%
Likewise, the following files |final|\textit{nn}|.tex|
compile the final version of the child document
|child|\textit{nn}|.tex|:
%
\begin{center}
\begin{tabular}{l}
|\def\version{final}|\\
|\input{childdoc.def}|\\
|\childdocforwardprefix{final}{child}|
\end{tabular}
\end{center}
%

Note that when several versions of a main file and/or of each child file
are to be generated, it may be convenient to set up a |Makefile| or
shell script to automatise the process.

%%%%%%%%%%%%%%%%%%%%%%%%%%%%%%%%%%%%%%%%%%%%%%%%%%%%%%%%%%%%%%%%%%%%%%%%%%%%%%%%
\subsection{Command Line Processing}
\label{sec:commandline}

The effect of redirection files can also be achieved by invoking
the \LaTeX{} compiler with a more elaborate command line.
Most conveniently this should be done as part
of a shell script or a |Makefile|.

When using \textsf{childdoc} in the main file, the following
command lines effectively perform a redirection
(note that depending on the shell being used,
backslashes may have to be doubled: `|\|' $\to$ `|\\|'):
%
\begin{center}
|... -jobname "|\textit{target}|" |\\|"|[\textit{flags}]%
|\input{childdoc.def}\childdocforward[|\textit{main}|]{|\textit{dest}|}"|
\end{center}
%
Here \textit{target} is the name of the output file,
\textit{main} is the name of the main file
and \textit{dest} is the name of the main or child file to be processed
(all filenames without extensions).
The optional argument \textit{main} can be omitted
if \textit{main} matches \textit{dest}.
Optionally, compilation \textit{flags} can be defined via |\def| commands.
This command line makes the \TeX{} engine believe
it is compiling the file \textit{target}
whose content is specified as the latter parameter.
The provided code then forwards the processing to
\textit{main} or \textit{dest} as described in \secref{sec:forward}.

%%%%%%%%%%%%%%%%%%%%%%%%%%%%%%%%%%%%%%%%%%%%%%%%%%%%%%%%%%%%%%%%%%%%%%%%%%%%%%%%
\subsection{Include by Input}
\label{sec:input}

Including child documents by |\include| has some restrictions by design.
Most notably, the content of a child document always occupies
its own set of pages; pages cannot be shared between child documents.
Usually, this behaviour makes perfect sense
because each child document contain an essential part of the document.
However, in some situations it may be desirable to compose
a document from a collection of parts
without having mandatory page breaks between then.
For this case, the package
provides a mechanism to include parts
by |\input| which can also be processed individually.
However, by construction this mechanism
requires manual handling of the content to be output.

%%%%%%%%%%%%%%%%%%%%%%%%%%%%%%%%%%%%%%%%
\DescribeMacro{\ifchilddocmanual}
The main file should be prepared as usual, see \secref{sec:include}.
However, the document body must make a distinction
between processing of an individual part and of the main document, e.g.:
%
\begin{center}
\begin{tabular}{l}
|\ifchilddocmanual|\\
|\input{\childdocname}|\\
|\||else|\\
\textit{document body with }|\input{|\textit{part}|}|\\
|\||fi|
\end{tabular}
\end{center}
%
The conditional |\ifchilddocmanual| is true whenever
a part to be included by |\input| is being compiled,
and the name of the part is stored in |\childdocname|.

%%%%%%%%%%%%%%%%%%%%%%%%%%%%%%%%%%%%%%%%
\DescribeMacro{\childdocby}
Each part to be included by |\input| should start with:
%
\begin{center}
\begin{tabular}{l}
|\input{childdoc.def}|\\
|\childdocby{|\textit{main}|}|\\
\end{tabular}
\end{center}
%
The directive |\childdocby| is similar to |\childdocof|
described in \secref{sec:include},
but the subsequent selection of content must be done manually.
To that end, both |\ifchilddoc| and |\ifchilddocmanual|
will be true upon processing of a part,
and the name of the part is stored in |\childdocname|.
Note that |\jobname| will be set to the filename of the current part
so that each part receives an individual |.aux| file
that does not interfere with the |.aux| file(s) of the main document.
This behaviour can be altered by the alternative form
|\childdocby[*]{|\textit{main}|}| (with a non-empty optional argument)
which uses the |.aux| file of the main document
by setting |\jobname| to \textit{main}.

%%%%%%%%%%%%%%%%%%%%%%%%%%%%%%%%%%%%%%%%%%%%%%%%%%%%%%%%%%%%%%%%%%%%%%%%%%%%%%%%
\subsection{Driver Development}
\label{sec:driver}

The \textsf{childdoc} mechanism can also be use for the development
of definition files such as \LaTeX{} styles or classes.
This case differs from the above setup with multiple parts
included by |\include| in that no |\includeonly| should be invoked.
This can be achieved by starting the include file
(before |\ProvidesPackage|) with:
%
\begin{center}
\begin{tabular}{l}
|\input{childdoc.def}|\\
|\childdocforward{|\textit{main}|}|\\
\end{tabular}
\end{center}
%
or alternatively with:
%
\begin{center}
\begin{tabular}{l}
|\input{childdoc.def}|\\
|\childdocby{|\textit{main}|}|\\
\end{tabular}
\end{center}
%
Both forms have slightly different effects as described above.
The main file is prepared as usual, see \secref{sec:include}.

%%%%%%%%%%%%%%%%%%%%%%%%%%%%%%%%%%%%%%%%%%%%%%%%%%%%%%%%%%%%%%%%%%%%%%%%%%%%%%%%
\subsection{Legacy Detection}
\label{sec:detection}

The directive |\childdocmain| in the main file can detect
whether the complete document or merely a child is to be compiled
even without using the directive |\childdocof|.
This method is deprecated because it is less robust
and there is no compelling reason to use it;
it is merely provided for backward compatibility
and it may be removed in future versions.

If the detection mechanism is to be used,
it is mandatory to correctly specify
the filename of the main file as the argument of |\childdocmain|:
%
\begin{center}
\begin{tabular}{l}
|\input{childdoc.def}|\\
|\childdocmain{|\textit{main}|}|\\
\end{tabular}
\end{center}
%
If |\jobname| does not match the argument \textit{main} of |\childdocmain|,
it is assumed that |\jobname| points to the child file to be compiled.
When using |\childdocmain| with the main file specified as argument,
it suffices to start a child file
with just |\input{|\textit{main}|}|
without loading of the package and using |\childdocof|.
If instead all processing is done
with the appropriate \textsf{childdoc} directives,
the argument of \textit{main} of |\childdocmain| can be empty.

An alternative version of the command line processing described
in \secref{sec:commandline} using the detection mechanism reads:
%
\begin{center}
|... -jobname "|\textit{target}|" "|[\textit{flags}]%
[|\def\jobname{|\textit{dest}|}|]|\input{|\textit{main}|}"|
\end{center}

%%%%%%%%%%%%%%%%%%%%%%%%%%%%%%%%%%%%%%%%%%%%%%%%%%%%%%%%%%%%%%%%%%%%%%%%%%%%%%%%
\subsection{Manual Code}
\label{sec:manual}

In case one cannot be certain whether the definitions file |childdoc.def|
is installed on the target \TeX{} distribution
and one prefers not to ship it,
it is conceivable to paste a few relevant commands into the sources.

To that end, drop all statements |\input{childdoc.def}|
and perform the replacements as outlined below.
Instead of |\childdocmain{|\textit{main}|}| add the following code
to the top of the main file:
%
\begin{center}
\begin{tabular}{l}
|\||ifdefined\childdocname\endinput\||fi\newif\ifchilddoc|\\
|\edef\childdocname{\scantokens\expandafter{\jobname\noexpand}}|\\
|\def\childdocmain{|\textit{main}|}\||ifx\childdocmain\childdocname\||else|\\
|\childdoctrue\includeonly{\childdocname}\let\jobname\childdocmain\||fi|\\
\end{tabular}
\end{center}
%
Instead of |\childdocof{|\textit{main}|}| just include the main file
at the top of each child file:
%
\begin{center}
|\input{|\textit{main}|}|
\end{center}
%
A simple redirection |\childdocforward{|\textit{dest}|}| is achieved by:
%
\begin{center}
|\def\jobname{|\textit{dest}|}\input{\jobname}|
\end{center}
%
The redirection with prefix
|\childdocforwardprefix[|\textit{prefix}|]{|\textit{dest}|}|
is accomplished by:
%
\begin{center}
\begin{tabular}{l}
|{\edef\jobname{\scantokens\expandafter{\jobname\noexpand}}|\\
|\def\redirectjob |\textit{prefix}|#1~~~{\gdef\jobname{|\textit{dest}|#1}}|\\
|\expandafter\redirectjob\jobname~~~}\input{\jobname}|
\end{tabular}
\end{center}

In an alternative approach,
child documents can be compiled by a specific command line
without additional code or specific definitions:
%
\begin{center}
|... -jobname "|\textit{target}|" "|[\textit{flags}]%
|\includeonly{|\textit{dest}|}\input{|\textit{main}|}"|
\end{center}
%

%%%%%%%%%%%%%%%%%%%%%%%%%%%%%%%%%%%%%%%%%%%%%%%%%%%%%%%%%%%%%%%%%%%%%%%%%%%%%%%%
%%%%%%%%%%%%%%%%%%%%%%%%%%%%%%%%%%%%%%%%%%%%%%%%%%%%%%%%%%%%%%%%%%%%%%%%%%%%%%%%
\section{Information}

%%%%%%%%%%%%%%%%%%%%%%%%%%%%%%%%%%%%%%%%%%%%%%%%%%%%%%%%%%%%%%%%%%%%%%%%%%%%%%%%
\subsection{Copyright}

Copyright \copyright{} 2017--2018 Niklas Beisert

This work may be distributed and/or modified under the
conditions of the \LaTeX{} Project Public License, either version 1.3
of this license or (at your option) any later version.
The latest version of this license is in
  \url{http://www.latex-project.org/lppl.txt}
and version 1.3 or later is part of all distributions of \LaTeX{}
version 2005/12/01 or later.

This work has the LPPL maintenance status `maintained'.

The Current Maintainer of this work is Niklas Beisert.

This work consists of the files |README.txt|, |childdoc.ins| and |childdoc.dtx|
as well as the derived files |childdoc.def|, |cdocsamp.tex|
with |cdocsch1.tex|, |cdocsch2.tex|, |cdocspt3.tex|, |cdocspt4.tex|,
|cdocsdrf.tex|, |cdocsfn1.tex|, |cdocsfn2.tex|
as well as |childdoc.pdf|.

%%%%%%%%%%%%%%%%%%%%%%%%%%%%%%%%%%%%%%%%%%%%%%%%%%%%%%%%%%%%%%%%%%%%%%%%%%%%%%%%
\subsection{Files and Installation}

The package consists of the files:
%
\begin{center}
\begin{tabular}{ll}
    |README.txt|   & readme file \\
    |childdoc.ins| & installation file \\
    |childdoc.dtx| & source file \\
    |childdoc.def| & definition file \\
    |cdocsamp.tex| & sample main file \\
    |cdocsch1.tex| & sample include file \\
    |cdocsch2.tex| & sample include file \\
    |cdocspt3.tex| & sample part file \\
    |cdocspt4.tex| & sample part file \\
    |cdocsdrf.tex| & sample redirection file \\
    |cdocsfn1.tex| & sample redirection file \\
    |cdocsfn2.tex| & sample redirection file \\
    |childdoc.pdf| & manual
\end{tabular}
\end{center}
%
The distribution consists of the files
|README.txt|, |childdoc.ins| and |childdoc.dtx|.
%
\begin{itemize}
\item
Run (pdf)\LaTeX{} on |childdoc.dtx|
to compile the manual |childdoc.pdf| (this file).
\item
Run \LaTeX{} on |childdoc.ins| to create the definitions file |childdoc.def|
and the sample |cdocsamp.tex| with include files
|cdocsch1.tex|, |cdocsch2.tex|, |cdocspt3.tex|, |cdocspt4.tex|,
|cdocsdrf.tex|, |cdocsfn1.tex|, |cdocsfn2.tex|.
Then copy the file |childdoc.def| to an appropriate directory of your \LaTeX{}
distribution, e.g.\ \textit{texmf-root}|/tex/latex/childdoc|.
\end{itemize}

%%%%%%%%%%%%%%%%%%%%%%%%%%%%%%%%%%%%%%%%%%%%%%%%%%%%%%%%%%%%%%%%%%%%%%%%%%%%%%%%
\subsection{Related CTAN Packages}

There are several other packages which offer a similar functionality:
%
\begin{itemize}
\item
The packages
\href{http://ctan.org/pkg/docmute}{\textsf{docmute}},
\href{http://ctan.org/pkg/includex}{\textsf{includex}} and
\href{http://ctan.org/pkg/standalone}{\textsf{standalone}}
provide commands to include only the document body of
a child file thus allowing both files to be compiled individually.
\item
The packages \href{http://ctan.org/pkg/subdocs}{\textsf{subdocs}}
and \href{http://ctan.org/pkg/subfiles}{\textsf{subfiles}}
provide structures in which the main and child documents can be
encapsulated and allowing them to be compiled individually.
The inclusion mechanism is different from the conventional |\include|.
\item
The package \href{http://ctan.org/pkg/combine}{\textsf{combine}}
is an elaborate solution to combine several documents into one.
\end{itemize}
%
See also the CTAN topic \href{http://ctan.org/topic/subdocs}{\textsf{subdocs}}
for further related packages.
The present package differs from the above solutions in that
a document structure constructed with the conventional |\include| mechanism
just needs two extra commands at the top of every file
such that all constituent files can be compiled individually.

%%%%%%%%%%%%%%%%%%%%%%%%%%%%%%%%%%%%%%%%%%%%%%%%%%%%%%%%%%%%%%%%%%%%%%%%%%%%%%%%
%\subsection{Feature Suggestions}
%
%The following is a list of features which may be useful for future
%versions of this package:
%%
%\begin{itemize}
%\item
%\ldots
%\end{itemize}

%%%%%%%%%%%%%%%%%%%%%%%%%%%%%%%%%%%%%%%%%%%%%%%%%%%%%%%%%%%%%%%%%%%%%%%%%%%%%%%%
\subsection{Revision History}

%%%%%%%%%%%%%%%%%%%%%%%%%%%%%%%%%%%%%%%%
\paragraph{v2.0:} 2018/12/30

\begin{itemize}
\item
immediate forward processing
\item
added |\childdocby| mechanism
\item
manual restructured
\end{itemize}

%%%%%%%%%%%%%%%%%%%%%%%%%%%%%%%%%%%%%%%%
\paragraph{v1.6:} 2018/01/17

\begin{itemize}
\item
application for development of include files
\item
corrections to manual
\end{itemize}

%%%%%%%%%%%%%%%%%%%%%%%%%%%%%%%%%%%%%%%%
\paragraph{v1.5:} 2017/05/21

\begin{itemize}
\item
more complete structuring introduced
\item
|\childdocof| introduced
\item
|\childdoc| renamed to |\childdocmain|
\item
|\childredirect| renamed to |\childdocforward| and |\childdocforwardprefix|
and functionality expanded
\end{itemize}

%%%%%%%%%%%%%%%%%%%%%%%%%%%%%%%%%%%%%%%%
\paragraph{v1.0:} 2017/04/27

\begin{itemize}
\item
manual and install package
\item
first version published on CTAN
\end{itemize}

%%%%%%%%%%%%%%%%%%%%%%%%%%%%%%%%%%%%%%%%
\paragraph{v0.6:} 2017/04/26

\begin{itemize}
\item
redirection mechanism added
\end{itemize}

%%%%%%%%%%%%%%%%%%%%%%%%%%%%%%%%%%%%%%%%
\paragraph{v0.5:} 2017/04/26

\begin{itemize}
\item
functionality in definition file
\end{itemize}


%%%%%%%%%%%%%%%%%%%%%%%%%%%%%%%%%%%%%%%%%%%%%%%%%%%%%%%%%%%%%%%%%%%%%%%%%%%%%%%%
%%%%%%%%%%%%%%%%%%%%%%%%%%%%%%%%%%%%%%%%%%%%%%%%%%%%%%%%%%%%%%%%%%%%%%%%%%%%%%%%
%%%%%%%%%%%%%%%%%%%%%%%%%%%%%%%%%%%%%%%%%%%%%%%%%%%%%%%%%%%%%%%%%%%%%%%%%%%%%%%%
\appendix

\settowidth\MacroIndent{\rmfamily\scriptsize 000\ }

 \DocInput{childdoc.dtx}

\end{document}
%</driver>
% \fi
%
% %%%%%%%%%%%%%%%%%%%%%%%%%%%%%%%%%%%%%%%%%%%%%%%%%%%%%%%%%%%%%%%%%%%%%%%%%%%%%%
% %%%%%%%%%%%%%%%%%%%%%%%%%%%%%%%%%%%%%%%%%%%%%%%%%%%%%%%%%%%%%%%%%%%%%%%%%%%%%%
% \section{Sample}
%\iffalse
%<*samplemain>
%\fi
%
% The following presents a sample document
% with two chapters, two parts, a title page,
% a compile flag as well as three forwarding files to set the flag.
% It consists of eight |.tex| files:
% \begin{center}
% \begin{tabular}{ll}
% |cdocsamp.tex|&main file\\
% |cdocsch1.tex|&include file for chapter 1\\
% |cdocsch2.tex|&include file for chapter 2\\
% |cdocspt3.tex|&include file for part 3\\
% |cdocspt4.tex|&include file for part 4\\
% |cdocsdrf.tex|&forwarding file for main file in draft mode\\
% |cdocsfi1.tex|&forwarding file for final version of chapter 1\\
% |cdocsfi2.tex|&forwarding file for final version of chapter 2\\
% \end{tabular}
% \end{center}
% Each of the eight files can be compiled directly by the \LaTeX{} compiler.
%
% %%%%%%%%%%%%%%%%%%%%%%%%%%%%%%%%%%%%%%
% \paragraph{Main File.}
%
% The main file is called |cdocsamp.tex|.
%
% Load the \textsf{childdoc} definitions and
% declare the filename for the main document:
%    \begin{macrocode}
\input{childdoc.def}
\childdocmain{}
%    \end{macrocode}

% Optional override for |\version| flag:
%    \begin{macrocode}
%%\ifchilddoc\else\providecommand{\version}{draft}\fi
%    \end{macrocode}

% Define the default values for the |\version| flag
% (|final| for the main file and |draft| for childs):
%    \begin{macrocode}
\ifchilddoc
\providecommand{\version}{draft}
\else
\providecommand{\version}{final}
\fi
%    \end{macrocode}

% Load the standard document class:
%    \begin{macrocode}
\documentclass[12pt]{article}
%    \end{macrocode}

% Start the document body:
%    \begin{macrocode}
\begin{document}
%    \end{macrocode}

% Declare a title page.
% Print title, part of document being processed and version flag:
%    \begin{macrocode}
\addtocounter{page}{-1}
\begin{center}
{\LARGE\bfseries{}childdoc example\par}
\vspace{1cm}
\ifchilddoc
\ifchilddocmanual part\else chapter\fi:
`\childdocname' of `\childdocjob'\par
\else
main document: `\childdocjob'\par
\fi
version: \version\par
\end{center}
\newpage
%    \end{macrocode}

% Manually include selected file,
% otherwise process as usual:
%    \begin{macrocode}
\ifchilddocmanual
\section*{part `\childdocname'}
\input{\childdocname}
\else
%    \end{macrocode}

% Include the two chapters:
%    \begin{macrocode}
\include{cdocsch1}
\include{cdocsch2}
%    \end{macrocode}

% Include the two parts unless only chapters should be displayed:
%    \begin{macrocode}
\ifchilddoc\else
\section{part three}
\input{cdocspt3}
\section{part four}
\input{cdocspt4}
\fi
%    \end{macrocode}

% Process as usual until here:
%    \begin{macrocode}
\fi
%    \end{macrocode}

% End of document body:
%    \begin{macrocode}
\end{document}
%    \end{macrocode}
%\iffalse
%</samplemain>
%\fi
%
% %%%%%%%%%%%%%%%%%%%%%%%%%%%%%%%%%%%%%%
% \paragraph{Chapter Include Files.}
%
% The include files are called |cdocsch1.tex| and |cdocsch2.tex|.
%
%\iffalse
%<*samplechap1|samplechap2>
%\fi

% Optional override for |\version| flag:
%    \begin{macrocode}
%%\providecommand{\version}{final}
%    \end{macrocode}

% Include the main document:
%    \begin{macrocode}
\input{childdoc.def}
\childdocof{cdocsamp}
%    \end{macrocode}

%\iffalse
%</samplechap1|samplechap2>
%\fi
%
%\iffalse
%<*samplechap1>
%\fi
% Some text for chapter 1:
%    \begin{macrocode}
\section{one}
some text in chapter one
%    \end{macrocode}

%\iffalse
%</samplechap1>
%\fi
% Some text for chapter 2:
%\iffalse
%<*samplechap2>
%\fi
%    \begin{macrocode}
\section{two}
more text in chapter two
%    \end{macrocode}

%\iffalse
%</samplechap2>
%\fi
%
% %%%%%%%%%%%%%%%%%%%%%%%%%%%%%%%%%%%%%%
% \paragraph{Part Include Files.}
%
% The include files are called |cdocspt3.tex| and |cdocspt4.tex|.
%
%\iffalse
%<*samplepart3|samplepart4>
%\fi

% Optional override for |\version| flag:
%    \begin{macrocode}
%%\providecommand{\version}{final}
%    \end{macrocode}

% Include the main document:
%    \begin{macrocode}
\input{childdoc.def}
\childdocby{cdocsamp}
%    \end{macrocode}

%\iffalse
%</samplepart3|samplepart4>
%\fi
%
%\iffalse
%<*samplepart3>
%\fi
% Some text for part 3:
%    \begin{macrocode}
some text in part three
%    \end{macrocode}

%\iffalse
%</samplepart3>
%\fi
% Some text for part 4:
%\iffalse
%<*samplepart4>
%\fi
%    \begin{macrocode}
more text in part four
%    \end{macrocode}

%\iffalse
%</samplepart4>
%\fi
%
% %%%%%%%%%%%%%%%%%%%%%%%%%%%%%%%%%%%%%%
% \paragraph{Forwarding for a Complete Draft.}
%
% The following forwarding file |cdocsdrf.tex|
% compiles the main document in draft mode:
%\iffalse
%<*sampledraft>
%\fi
%    \begin{macrocode}
\def\version{draft}
\input{childdoc.def}
\childdocforward{cdocsamp}
%    \end{macrocode}

%\iffalse
%</sampledraft>
%\fi
%
% %%%%%%%%%%%%%%%%%%%%%%%%%%%%%%%%%%%%%%
% \paragraph{Forwarding for Final Version of the Chapters.}
%
% The following forwarding files |cdocsfn1.tex| and |cdocsfn2.tex|
% (with identical content)
% compile the final versions of the child documents
% |cdocsch1.tex| and |cdocsch2.tex|, respectively:
%\iffalse
%<*samplefinal>
%\fi
%    \begin{macrocode}
\def\version{final}
\input{childdoc.def}
\childdocforwardprefix[cdocsamp]{cdocsfn}{cdocsch}
%    \end{macrocode}

%\iffalse
%</samplefinal>
%\fi
%
% %%%%%%%%%%%%%%%%%%%%%%%%%%%%%%%%%%%%%%
% \paragraph{Command Line Processing.}
%
% The following three command lines generate the output files
% |cdocscld|, |cdocscl1| and |cdocscl2|
% which should be identical to
% |cdocsdrf|, |cdocsch1| and |cdocsfn2|, respectively:
% \begin{center}
% \begin{tabular}{l}
% |latex -jobname cdocscld \|\\
% |  "\def\version{draft}\input{childdoc.def}\childdocforward{cdocsamp}"|\\
% |latex -jobname cdocscl1 \|\\
% |  "\input{childdoc.def}\childdocforward[cdocsamp]{cdocsch1}"|\\
% |latex -jobname cdocscl2 \|\\
% |  "\def\version{final}\input{childdoc.def}\childdocforward{cdocsch2}"|
% \end{tabular}
% \end{center}
% Note that the trailing backslash on each first line
% merely continues the input to the second line
% (for convenient cut ant paste).
% Furthermore, the command |latex| can be replaced by any
% of its alternative versions such as |pdflatex|.
%
% %%%%%%%%%%%%%%%%%%%%%%%%%%%%%%%%%%%%%%%%%%%%%%%%%%%%%%%%%%%%%%%%%%%%%%%%%%%%%%
% %%%%%%%%%%%%%%%%%%%%%%%%%%%%%%%%%%%%%%%%%%%%%%%%%%%%%%%%%%%%%%%%%%%%%%%%%%%%%%
% \section{Implementation}
%\iffalse
%<*package>
%\fi
%
% This section describes the definitions file |childdoc.def|.

% The definitions cannot be loaded using |\usepackage| or |\RequirePackage|
% which has a mechanism to prevent loading a style file more than once.
% When loading the definitions by means of |\input|
% multiple instances have to be prevented manually:
%\iffalse
%This code needs to be before the `\ProvidesFile' directive
%which is defined at the beginning of this file.
%Therefore it is also placed there and commented out here.
%</package>
%<*discard>
%\fi
%    \begin{macrocode}
\ifdefined\childdocmain\endinput\fi
%    \end{macrocode}
%\iffalse
%</discard>
%<*package>
%\fi
%
% \macro{\ifchilddoc}
% \macro{\ifchilddocmanual}
% The conditional |\ifchilddoc| tells whether a
% child (true) or main (false) document is being compiled.
% The conditional |\ifchilddocmanual| tells whether
% the |\includeonly| mechanism is used (false) or
% the selection of child files must be performed manually (true).
% The definitions initialise to false:
%    \begin{macrocode}
\newif\ifchilddoc
\newif\ifchilddocmanual
%    \end{macrocode}

% \macro{\childdocname}
% \macro{\childdocjob}
% The macro |\childdocname| stores the name of the main document
% to be compiled. The macro |\childdocjob| stores the name of
% the document on which the \LaTeX{} compiler was originally invoked.
% The content of |\jobname| cannot be compared
% to filenames specified in the source due to different catcodes.
% The following code rescans |\jobname|, stores the result
% in |\childdocname| and saves a copy in |\childdocjob|:
%    \begin{macrocode}
\edef\childdocname{\scantokens\expandafter{\jobname\noexpand}}
\let\childdocjob\childdocname
%    \end{macrocode}

% \macro{\childdocdisable}
% The macro |\childdocdisable| prevents the main file
% from being processed more than once.
% At this stage, the main document command |\childdocmain|
% is assumed to be called once again where it should do nothing.
% Any subsequent call to it should prevent
% a secondary processing of the main document
% It overwrites the forwarding commands
% |\childdocof| and |\childdocforward|
% with empty macros to prevent further inclusions of the main document:
%    \begin{macrocode}
\newcommand{\childdocdisable}
{
  \renewcommand{\childdocmain}[1]{\renewcommand{\childdocmain}[1]{\endinput}}
  \renewcommand{\childdocof}[1]{}
  \renewcommand{\childdocby}[2][]{}
  \renewcommand{\childdocforward}[2][]{}
  \renewcommand{\childdocdisable}{}
}
%    \end{macrocode}

% \macro{\childdocmain}
% The macro |\childdocmain| is to be called at the top of the main file
% with nothing or the main filename (without extension) as argument.
% First, it breaks loops.
% If the argument is not empty and does not match |\childdocname|
% (which is set by the first inclusion of |childdoc.def|),
% |\ifchilddoc| is set to true, |\includeonly| is applied to the child file
% and |\jobname| is set to the main file
% (for proper handling of |.aux| files):
%    \begin{macrocode}
\newcommand{\childdocmain}[1]
{
  \childdocdisable\childdocmain{}
  \if?#1?\else
    \begingroup
      \def\childdoctmp{#1}
      \ifx\childdoctmp\childdocname
        \def\childdoctmp{}
      \else
        \def\childdoctmp
        {
          \childdoctrue
          \includeonly{\childdocname}
          \def\childdocjob{#1}
          \def\jobname{#1}
        }
      \fi
      \expandafter
    \endgroup
    \childdoctmp
  \fi
}
%    \end{macrocode}

% \macro{\childdocof}
% The command |\childdocof| redirects
% compilation to the main file |#1|.
%    \begin{macrocode}
\newcommand{\childdocof}[1]
{
  \childdocdisable
  \childdoctrue
  \includeonly{\childdocname}
  \def\jobname{#1}
  \def\childdocjob{#1}
  \input{#1}
}
%    \end{macrocode}

% \macro{\childdocby}
% The command |\childdocby| ....
%    \begin{macrocode}
\newcommand{\childdocby}[2][]
{
  \childdocdisable
  \childdoctrue
  \childdocmanualtrue
  \if?#1?\else
    \def\jobname{#2}
  \fi
  \def\childdocjob{#2}
  \input{#2}
  \endinput
}
%    \end{macrocode}

% \macro{\childdocforward}
% The command |\childdocforward| redirects
% compilation to the main file or
% (if the optional argument is given) a child file.
% Parameters are set as if the main file
% or a child file starting with |\childdocof| was compiled.
% Then compilation is handed over to the main file:
%    \begin{macrocode}
\newcommand{\childdocforward}[2][]
{
  \begingroup
    \if?#1?
      \def\childdoctmp
      {
        \def\childdocname{#2}
        \def\childdocjob{#2}
        \def\jobname{#2}
        \input{#2}
        \endinput
      }
    \else
      \def\childdoctmp
      {
        \childdocdisable
        \def\childdocname{#2}
        \childdoctrue
        \includeonly{#2}
        \def\childdocjob{#1}
        \def\jobname{#1}
        \input{#1}
        \endinput
      }
    \fi
    \expandafter
  \endgroup
  \childdoctmp
}
%    \end{macrocode}

% \macro{\childdocforwardprefix}
% The command |\childdocforwardprefix| redirects
% compilation to the main or a child file by means of a pattern.
% The prefix |#1| in the current filename is replaced by |#2|
% and the suffix of the current filename is kept
% (it is assumed that the filename does not contain the substring `|~~~|'
% which is used as a delimiter).
% Compilation is handed over to the new file by |\childdocforward|:
%    \begin{macrocode}
\newcommand{\childdocforwardprefix}[3][]
{
  \begingroup
    \def\childdocextract #2##1~~~{\def\childdoctmp{\childdocforward[#1]{#3##1}}}
    \expandafter\childdocextract\childdocname~~~
    \expandafter
  \endgroup
  \childdoctmp
}
%    \end{macrocode}

% \macro{\childdoc}
% The deprecated macro |\childdoc| is a legacy version of |\childdocmain|:
%    \begin{macrocode}
\newcommand{\childdoc}{\childdocmain}
%    \end{macrocode}

% \macro{\childdocredirect}
% The deprecated macro |\childdocredirect| is a legacy version
% of |\childdocforward| and |\childdocforwardprefix|:
%    \begin{macrocode}
\newcommand{\childdocredirect}[2][]
{
  \begingroup
    \if?#1?
      \def\childdoctmp{\childdocforward{#2}}
    \else
      \def\childdoctmp{\childdocforwardprefix{#1}{#2}}
    \fi
    \expandafter
  \endgroup
  \childdoctmp
}
%    \end{macrocode}

%\iffalse
%</package>
%\fi
%
\endinput
\childdocforward[cdocsamp]{cdocsch1}"|\\
% |latex -jobname cdocscl2 \|\\
% |  "\def\version{final}% \iffalse
%
% childdoc.dtx Copyright (C) 2017-2018 Niklas Beisert
%
% This work may be distributed and/or modified under the
% conditions of the LaTeX Project Public License, either version 1.3
% of this license or (at your option) any later version.
% The latest version of this license is in
%   http://www.latex-project.org/lppl.txt
% and version 1.3 or later is part of all distributions of LaTeX
% version 2005/12/01 or later.
%
% This work has the LPPL maintenance status `maintained'.
%
% The Current Maintainer of this work is Niklas Beisert.
%
% This work consists of the files childdoc.dtx and childdoc.ins
% and the derived files childdoc.def and cdocsamp.tex with
% cdocsch1.tex, cdocsch2.tex, cdocsdrf.tex, cdocsfn1.tex, cdocsfn2.tex.
%
%<package>\ifdefined\childdocmain\endinput\fi
%<package>\ProvidesFile{childdoc.def}[2018/12/30 v2.0 child document driver]
%<samplemain>\ProvidesFile{cdocsamp.tex}[2018/12/30 v2.0 sample for childdoc]
%<*driver>
%\ProvidesFile{childdoc.drv}[2018/12/30 v2.0 childdoc reference manual file]
\PassOptionsToClass{10pt,a4paper}{article}
\documentclass{ltxdoc}

\usepackage[margin=35mm]{geometry}
\usepackage{hyperref}
\usepackage{hyperxmp}
\usepackage[usenames]{color}

\hypersetup{colorlinks=true}
\hypersetup{pdfstartview=FitH}
\hypersetup{pdfpagemode=UseNone}
\hypersetup{pdfsource={}}
\hypersetup{pdflang={en-UK}}
\hypersetup{pdfcopyright={Copyright 2017-2018 Niklas Beisert.
  This work may be distributed and/or modified under the
  conditions of the LaTeX Project Public License, either version 1.3
  of this license or (at your option) any later version.}}
\hypersetup{pdflicenseurl={http://www.latex-project.org/lppl.txt}}
\hypersetup{pdfcontactaddress={ETH Zurich, ITP, HIT K,
  Wolfgang-Pauli-Strasse 27}}
\hypersetup{pdfcontactpostcode={8093}}
\hypersetup{pdfcontactcity={Zurich}}
\hypersetup{pdfcontactcountry={Switzerland}}
\hypersetup{pdfcontactemail={nbeisert@itp.phys.ethz.ch}}
\hypersetup{pdfcontacturl={http://people.phys.ethz.ch/\xmptilde nbeisert/}}

\newcommand{\secref}[1]{\hyperref[#1]{section \ref*{#1}}}

\parskip1ex
\parindent0pt
\let\olditemize\itemize
\def\itemize{\olditemize\parskip0pt}

\begin{document}

\title{The \textsf{childdoc} Package}
\hypersetup{pdftitle={The childdoc Package}}
\author{Niklas Beisert\\[2ex]
  Institut f\"ur Theoretische Physik\\
  Eidgen\"ossische Technische Hochschule Z\"urich\\
  Wolfgang-Pauli-Strasse 27, 8093 Z\"urich, Switzerland\\[1ex]
  \href{mailto:nbeisert@itp.phys.ethz.ch}
  {\texttt{nbeisert@itp.phys.ethz.ch}}}
\hypersetup{pdfauthor={Niklas Beisert}}
\hypersetup{pdfsubject={Manual for the LaTeX2e Package childdoc}}
\date{30 December 2018, \textsf{v2.0}}
\maketitle

\begin{abstract}\noindent
\textsf{childdoc} is a \LaTeXe{} package
that enables the direct compilation
of document sections included by |\include|
to individual files.
\end{abstract}

\begingroup
\parskip0ex
\tableofcontents
\endgroup

%%%%%%%%%%%%%%%%%%%%%%%%%%%%%%%%%%%%%%%%%%%%%%%%%%%%%%%%%%%%%%%%%%%%%%%%%%%%%%%%
%%%%%%%%%%%%%%%%%%%%%%%%%%%%%%%%%%%%%%%%%%%%%%%%%%%%%%%%%%%%%%%%%%%%%%%%%%%%%%%%
\section{Introduction}

\LaTeX{} provides a mechanism to structure a large document (such as a book)
into a main file and several child files (containing the chapters)
using the |\include| command.
This mechanism is beneficial for documents
which span hundreds of pages in order to
make the source file(s) more manageable.
Moreover, compilation can be restricted to
selected child files by means of the |\includeonly| command.
The latter feature can be used to reduce the compilation time while editing
(this was significantly more useful in the earlier days of \LaTeX{})
or to generate a smaller document which is easier to navigate.
Another application of |\includeonly| is to generate
documents consisting of selected parts of the complete document.

However, there are a few drawbacks of the plain |\include| mechanism:
\begin{itemize}
\item
The child files cannot be compiled on their own,
they can only be compiled via the main file.
A naive editing environment
(such as a text editor with an option
to have the current file processed by \LaTeX)
may require one to switch to the main file before compiling;
attempting to compile the child file produces errors.
\item
The main file must be modified (each time)
to adjust the |\includeonly| command
to the present needs. This easily leaves the main file in a messy state.
\item
The generated document will always carry the filename
of the main document. This is inconvenient if
several child files are to be compiled and
to be kept for distribution.
\end{itemize}

The present package provides a simple interface
to make child files individually compilable by \LaTeX{}.
Compiling a child file then has the same effect as compiling
the main file with an |\includeonly| command
to select the appropriate child.
Moreover the generated document will carry the name of the child
rather than the main file.
This resolves all three above issues.

This feature is meant to make the editing of books,
thesis documents and lecture notes somewhat more convenient.
However, the package can also be used efficiently for
composing a series of documents (such as exercise sheets)
which are typically distributed individually.
It then assists the author in generating the individual documents
(potentially in different versions)
as well as a document containing the collected series.
Another application is in developing style files
or other kinds of included material
where compilation of the style file could redirect
to a sample or test file.

%%%%%%%%%%%%%%%%%%%%%%%%%%%%%%%%%%%%%%%%%%%%%%%%%%%%%%%%%%%%%%%%%%%%%%%%%%%%%%%%
%%%%%%%%%%%%%%%%%%%%%%%%%%%%%%%%%%%%%%%%%%%%%%%%%%%%%%%%%%%%%%%%%%%%%%%%%%%%%%%%
\section{Usage}

First of all, the package \textsf{childdoc} is \emph{not} a standard
\LaTeXe{} |.sty| style file! Therefore it needs to be invoked in
a non-standard way.

%%%%%%%%%%%%%%%%%%%%%%%%%%%%%%%%%%%%%%%%%%%%%%%%%%%%%%%%%%%%%%%%%%%%%%%%%%%%%%%%
\subsection{Included Files}
\label{sec:include}

%%%%%%%%%%%%%%%%%%%%%%%%%%%%%%%%%%%%%%%%
\DescribeMacro{\childdocmain}
To use the package, add the commands
\begin{center}
\begin{tabular}{l}
|\input{childdoc.def}|\\
|\childdocmain{}|\\
\end{tabular}
\end{center}
at the very top of the main \LaTeX{} file,
in particular \emph{before} the |\documentclass| statement!
The argument of |\childdocmain| should be left empty
(but it must be present).

%%%%%%%%%%%%%%%%%%%%%%%%%%%%%%%%%%%%%%%%
\DescribeMacro{\childdocof}
Furthermore, add the commands
\begin{center}
\begin{tabular}{l}
|\input{childdoc.def}|\\
|\childdocof{|\textit{main}|}|\\
\end{tabular}
\end{center}
at the top of every child file \textit{child}
which is included by |\include{|\textit{child}|}|
from within the main file
(or at least for those files to be compiled individually).
The argument \textit{main} must be the filename of the main file.

There are a couple of
considerations in setting up the main and child documents:

%%%%%%%%%%%%%%%%%%%%%%%%%%%%%%%%%%%%%%%%
\paragraph{Restrictions.}

Please note the following restrictions:
\begin{itemize}
\item
|\childdocmain| must be called with one argument \textit{main}
to ensure compatibility with earlier version of the package.
It must either be empty (|\childdocmain{}|)
or precisely match the filename of the main file in which it is specified.
See \secref{sec:detection} for further information.
\item
The filename \textit{main} must be specified without the |.tex| extension.
\item
The filename \textit{main} is case sensitive
(even in case-insensitive file systems)
due to internal string comparison.
\item
The argument \textit{main} should be fully expanded, it cannot be a macro.
\item
Subdirectories and special characters should be avoided in filenames.
\item
The command |\childdocmain{|\textit{main}|}| must be followed by a whitespace.
It should not be followed immediately by another command
or by a comment mark `|%|'.
This is because the \TeX{} parser reads the token immediately following
the argument of |\childdocmain| and puts it
at the beginning of every child section;
however, a white\-space is ignored.
\end{itemize}

%%%%%%%%%%%%%%%%%%%%%%%%%%%%%%%%%%%%%%%%
\paragraph{Content of Main File.}

It is advisable to place all content in the child files included by |\include|.
Any output contained in the main file will appear in all child documents
unless suppressed manually;
it cannot be suppressed automatically by the |\includeonly| directive
and thus should normally be avoided.
A method to include some content in the main file
by means of conditional processing is described in \secref{sec:conditional}.

%%%%%%%%%%%%%%%%%%%%%%%%%%%%%%%%%%%%%%%%
\paragraph{Page Numbering.}

When only a part of the document is compiled,
the appropriate numbering of pages
(as well as other status parameters)
is determined from the |.aux| files.
The latter contain information from previous passes.
However this information needs to propagate through
all intermediate child documents.
Therefore the page numbering in child documents may well
be inconsistent until the complete document is compiled at least once.

A useful (if unconventional) way to always ensure a consistent
page numbering is to restart the numbering in each child document
and denote the pages by `\textit{child}|.|\textit{page}'
where \textit{child} represents the chapter/section number of the child file.
This can be achieved by the command
|\numberwithin{page}{|\textit{child}|}|
of the \textsf{amsmath} package
where \textit{child} can be |chapter| or |section|
depending on the chosen structuring.
Alternatively, one can modify the macro |\thepage| appropriately
and reset the counter |page| at the start of each child file.

%%%%%%%%%%%%%%%%%%%%%%%%%%%%%%%%%%%%%%%%%%%%%%%%%%%%%%%%%%%%%%%%%%%%%%%%%%%%%%%%
\subsection{Conditional Processing}
\label{sec:conditional}

The package provides a mechanism to compile different versions
of a document. To customise the versions further some conditional processing
can come in handy to distinguish which version is being compiled.
The package provides two macros to describe the compilation context:

%%%%%%%%%%%%%%%%%%%%%%%%%%%%%%%%%%%%%%%%
\DescribeMacro{\ifchilddoc}
The conditional |\ifchilddoc| distinguishes between the compilation of
child documents and the main document:
%
\begin{center}
|\ifchilddoc |\textit{child-code}| |[|\||else |\textit{main-code}]| \||fi|
\end{center}

%%%%%%%%%%%%%%%%%%%%%%%%%%%%%%%%%%%%%%%%
\DescribeMacro{\childdocname}
\DescribeMacro{\childdocjob}
The macro |\childdocname| contains the filename (without extension)
of the main or child file being processed.
Note that |\childdocjob| will always contain the name of the main file.

%%%%%%%%%%%%%%%%%%%%%%%%%%%%%%%%%%%%%%%%
\paragraph{Title Page.}

Conditional processing can be used to include a title or banner page
in the main document when proper precautions are taken.
Importantly, the code in the main file should ensure that the page counter
(as well as other status parameters which are stored in the |.aux| files)
takes the same value after the conditional processing.
Otherwise the page numbers may take divergent values
depending on which part is compiled.

For example, a title page could be declared by:
%
\begin{center}
\begin{tabular}{l}
|\ifchilddoc\||else|\\
|\addtocounter{page}{-1}|\\
\textit{code for title page}\\
|\newpage|\\
|\||fi|
\end{tabular}
\end{center}
%
A banner page for the child documents can be generated by:
%
\begin{center}
\begin{tabular}{l}
|\ifchilddoc|\\
|\addtocounter{page}{-1}|\\
\textit{code for banner page}\\
|\newpage|\\
|\||fi|
\end{tabular}
\end{center}
%
Here one could write a message such as:
\begin{center}
|This is the part \childdocname{} of \childdocjob{}.|
\end{center}

%%%%%%%%%%%%%%%%%%%%%%%%%%%%%%%%%%%%%%%%%%%%%%%%%%%%%%%%%%%%%%%%%%%%%%%%%%%%%%%%
\subsection{Flags}
\label{sec:flags}

The package makes it easy to generate different versions
of the main or child documents.
To this end compilation flags can be defined
and assigned different default values.
They will be particularly useful in conjunction
with the forwarding mechanism described in \secref{sec:forward}.

For example, it may be useful to have a flag |\version|
which can be set to |draft| or |final|.
The document source will contain some conditional code
depending on the value of |\version|.
Suppose further, the flag should default to |final| for the main file
and to |draft| for child files
which is a natural assignment for editing the document.
This is achieved by placing the following code
in the preamble of the main document
(below the |\childdocmain| directive):
%
\begin{center}
\begin{tabular}{l}
|\ifchilddoc|\\
|\providecommand{\version}{draft}|\\
|\||else|\\
|\providecommand{\version}{final}|\\
|\||fi|
\end{tabular}
\end{center}
%
The definition by |\providecommand| makes sure
that previous definitions are not overwritten.
Further statements |\providecommand{\version}{...}|
can thus be added before the above code to override it.

For the main file, one might add a line
(between |\childdocmain| and the above block)
%
\begin{center}
|%\ifchilddoc\||else\providecommand{\version}{draft}\||fi|
\end{center}
%
which can be uncommented to produce a draft version.
Likewise one can add a line to the very top of a child file
(above the |\childdocof{|\textit{main}|}| directive)
%
\begin{center}
|%\providecommand{\version}{final}|
\end{center}
%
which can be uncommented to produce the final version of this child document.

%%%%%%%%%%%%%%%%%%%%%%%%%%%%%%%%%%%%%%%%%%%%%%%%%%%%%%%%%%%%%%%%%%%%%%%%%%%%%%%%
\subsection{Forwarding}
\label{sec:forward}

Different versions of the main or child documents
using compilation flags as described in \secref{sec:flags}
can be (permanently) stored in different files
for convenient compilation, viewing and distribution.
To this end, the package defines a command
to pass on compilation to a different file:

%%%%%%%%%%%%%%%%%%%%%%%%%%%%%%%%%%%%%%%%
\DescribeMacro{\childdocforward}
The command |\childdocforward| redirects processing to
another source file:
%
\begin{center}
\begin{tabular}{l}
|\input{childdoc.def}|\\
|\childdocforward[|\textit{main}|]{|\textit{dest}|}|\\
\end{tabular}
\end{center}
%
The argument \textit{dest} is the destination file
(without extension).
It should be the main file or one of the child files.
Note that further \textsf{childdoc} directives
such as |\childdocof| and |\childdocforward|
in the indicated file will be processed in this form.
The optional argument \textit{main}
passes on directly to the main file \textit{main}
while pretending to compile the child \textit{dest}.
This form behaves as if \textit{dest}
issues |\childdocof{|\textit{main}|}| right away,
and no further \textsf{childdoc} directives will be processed.

%%%%%%%%%%%%%%%%%%%%%%%%%%%%%%%%%%%%%%%%
\DescribeMacro{\...prefix}
In the alternative form |\childdocforwardprefix|,
%
\begin{center}
\begin{tabular}{l}
|\input{childdoc.def}|\\
|\childdocforwardprefix[|\textit{main}|]{|\textit{prefix}|}{|\textit{dest}|}|
\end{tabular}
\end{center}
%
the destination file is determined by a pattern
depending on the current file:
To make this work, the current file must be called
`{\textit{prefix}\hspace{0.2em}\textit{suffix}}'
with \textit{prefix} matching precisely the argument.
Processing is then passed on to the file
`{\textit{dest}\hspace{0.2em}\textit{suffix}}'.
Surely, the same effect is achieved by
directly specifying the
argument `{\textit{dest}\hspace{0.2em}\textit{suffix}}'
in the first form.
However, that requires to set up a different file
for each child. With the alternative form of the command
all these files can have exactly the same content
which simplifies setting them up and maintaining them.

For example, the following file |draft.tex|
with a compilation flag |\version| as described in \secref{sec:flags}
compiles the main document as a draft:
%
\begin{center}
\begin{tabular}{l}
|\def\version{draft}|\\
|\input{childdoc.def}|\\
|\childdocforward{|\textit{main}|}|
\end{tabular}
\end{center}
%
Likewise, the following files |final|\textit{nn}|.tex|
compile the final version of the child document
|child|\textit{nn}|.tex|:
%
\begin{center}
\begin{tabular}{l}
|\def\version{final}|\\
|\input{childdoc.def}|\\
|\childdocforwardprefix{final}{child}|
\end{tabular}
\end{center}
%

Note that when several versions of a main file and/or of each child file
are to be generated, it may be convenient to set up a |Makefile| or
shell script to automatise the process.

%%%%%%%%%%%%%%%%%%%%%%%%%%%%%%%%%%%%%%%%%%%%%%%%%%%%%%%%%%%%%%%%%%%%%%%%%%%%%%%%
\subsection{Command Line Processing}
\label{sec:commandline}

The effect of redirection files can also be achieved by invoking
the \LaTeX{} compiler with a more elaborate command line.
Most conveniently this should be done as part
of a shell script or a |Makefile|.

When using \textsf{childdoc} in the main file, the following
command lines effectively perform a redirection
(note that depending on the shell being used,
backslashes may have to be doubled: `|\|' $\to$ `|\\|'):
%
\begin{center}
|... -jobname "|\textit{target}|" |\\|"|[\textit{flags}]%
|\input{childdoc.def}\childdocforward[|\textit{main}|]{|\textit{dest}|}"|
\end{center}
%
Here \textit{target} is the name of the output file,
\textit{main} is the name of the main file
and \textit{dest} is the name of the main or child file to be processed
(all filenames without extensions).
The optional argument \textit{main} can be omitted
if \textit{main} matches \textit{dest}.
Optionally, compilation \textit{flags} can be defined via |\def| commands.
This command line makes the \TeX{} engine believe
it is compiling the file \textit{target}
whose content is specified as the latter parameter.
The provided code then forwards the processing to
\textit{main} or \textit{dest} as described in \secref{sec:forward}.

%%%%%%%%%%%%%%%%%%%%%%%%%%%%%%%%%%%%%%%%%%%%%%%%%%%%%%%%%%%%%%%%%%%%%%%%%%%%%%%%
\subsection{Include by Input}
\label{sec:input}

Including child documents by |\include| has some restrictions by design.
Most notably, the content of a child document always occupies
its own set of pages; pages cannot be shared between child documents.
Usually, this behaviour makes perfect sense
because each child document contain an essential part of the document.
However, in some situations it may be desirable to compose
a document from a collection of parts
without having mandatory page breaks between then.
For this case, the package
provides a mechanism to include parts
by |\input| which can also be processed individually.
However, by construction this mechanism
requires manual handling of the content to be output.

%%%%%%%%%%%%%%%%%%%%%%%%%%%%%%%%%%%%%%%%
\DescribeMacro{\ifchilddocmanual}
The main file should be prepared as usual, see \secref{sec:include}.
However, the document body must make a distinction
between processing of an individual part and of the main document, e.g.:
%
\begin{center}
\begin{tabular}{l}
|\ifchilddocmanual|\\
|\input{\childdocname}|\\
|\||else|\\
\textit{document body with }|\input{|\textit{part}|}|\\
|\||fi|
\end{tabular}
\end{center}
%
The conditional |\ifchilddocmanual| is true whenever
a part to be included by |\input| is being compiled,
and the name of the part is stored in |\childdocname|.

%%%%%%%%%%%%%%%%%%%%%%%%%%%%%%%%%%%%%%%%
\DescribeMacro{\childdocby}
Each part to be included by |\input| should start with:
%
\begin{center}
\begin{tabular}{l}
|\input{childdoc.def}|\\
|\childdocby{|\textit{main}|}|\\
\end{tabular}
\end{center}
%
The directive |\childdocby| is similar to |\childdocof|
described in \secref{sec:include},
but the subsequent selection of content must be done manually.
To that end, both |\ifchilddoc| and |\ifchilddocmanual|
will be true upon processing of a part,
and the name of the part is stored in |\childdocname|.
Note that |\jobname| will be set to the filename of the current part
so that each part receives an individual |.aux| file
that does not interfere with the |.aux| file(s) of the main document.
This behaviour can be altered by the alternative form
|\childdocby[*]{|\textit{main}|}| (with a non-empty optional argument)
which uses the |.aux| file of the main document
by setting |\jobname| to \textit{main}.

%%%%%%%%%%%%%%%%%%%%%%%%%%%%%%%%%%%%%%%%%%%%%%%%%%%%%%%%%%%%%%%%%%%%%%%%%%%%%%%%
\subsection{Driver Development}
\label{sec:driver}

The \textsf{childdoc} mechanism can also be use for the development
of definition files such as \LaTeX{} styles or classes.
This case differs from the above setup with multiple parts
included by |\include| in that no |\includeonly| should be invoked.
This can be achieved by starting the include file
(before |\ProvidesPackage|) with:
%
\begin{center}
\begin{tabular}{l}
|\input{childdoc.def}|\\
|\childdocforward{|\textit{main}|}|\\
\end{tabular}
\end{center}
%
or alternatively with:
%
\begin{center}
\begin{tabular}{l}
|\input{childdoc.def}|\\
|\childdocby{|\textit{main}|}|\\
\end{tabular}
\end{center}
%
Both forms have slightly different effects as described above.
The main file is prepared as usual, see \secref{sec:include}.

%%%%%%%%%%%%%%%%%%%%%%%%%%%%%%%%%%%%%%%%%%%%%%%%%%%%%%%%%%%%%%%%%%%%%%%%%%%%%%%%
\subsection{Legacy Detection}
\label{sec:detection}

The directive |\childdocmain| in the main file can detect
whether the complete document or merely a child is to be compiled
even without using the directive |\childdocof|.
This method is deprecated because it is less robust
and there is no compelling reason to use it;
it is merely provided for backward compatibility
and it may be removed in future versions.

If the detection mechanism is to be used,
it is mandatory to correctly specify
the filename of the main file as the argument of |\childdocmain|:
%
\begin{center}
\begin{tabular}{l}
|\input{childdoc.def}|\\
|\childdocmain{|\textit{main}|}|\\
\end{tabular}
\end{center}
%
If |\jobname| does not match the argument \textit{main} of |\childdocmain|,
it is assumed that |\jobname| points to the child file to be compiled.
When using |\childdocmain| with the main file specified as argument,
it suffices to start a child file
with just |\input{|\textit{main}|}|
without loading of the package and using |\childdocof|.
If instead all processing is done
with the appropriate \textsf{childdoc} directives,
the argument of \textit{main} of |\childdocmain| can be empty.

An alternative version of the command line processing described
in \secref{sec:commandline} using the detection mechanism reads:
%
\begin{center}
|... -jobname "|\textit{target}|" "|[\textit{flags}]%
[|\def\jobname{|\textit{dest}|}|]|\input{|\textit{main}|}"|
\end{center}

%%%%%%%%%%%%%%%%%%%%%%%%%%%%%%%%%%%%%%%%%%%%%%%%%%%%%%%%%%%%%%%%%%%%%%%%%%%%%%%%
\subsection{Manual Code}
\label{sec:manual}

In case one cannot be certain whether the definitions file |childdoc.def|
is installed on the target \TeX{} distribution
and one prefers not to ship it,
it is conceivable to paste a few relevant commands into the sources.

To that end, drop all statements |\input{childdoc.def}|
and perform the replacements as outlined below.
Instead of |\childdocmain{|\textit{main}|}| add the following code
to the top of the main file:
%
\begin{center}
\begin{tabular}{l}
|\||ifdefined\childdocname\endinput\||fi\newif\ifchilddoc|\\
|\edef\childdocname{\scantokens\expandafter{\jobname\noexpand}}|\\
|\def\childdocmain{|\textit{main}|}\||ifx\childdocmain\childdocname\||else|\\
|\childdoctrue\includeonly{\childdocname}\let\jobname\childdocmain\||fi|\\
\end{tabular}
\end{center}
%
Instead of |\childdocof{|\textit{main}|}| just include the main file
at the top of each child file:
%
\begin{center}
|\input{|\textit{main}|}|
\end{center}
%
A simple redirection |\childdocforward{|\textit{dest}|}| is achieved by:
%
\begin{center}
|\def\jobname{|\textit{dest}|}\input{\jobname}|
\end{center}
%
The redirection with prefix
|\childdocforwardprefix[|\textit{prefix}|]{|\textit{dest}|}|
is accomplished by:
%
\begin{center}
\begin{tabular}{l}
|{\edef\jobname{\scantokens\expandafter{\jobname\noexpand}}|\\
|\def\redirectjob |\textit{prefix}|#1~~~{\gdef\jobname{|\textit{dest}|#1}}|\\
|\expandafter\redirectjob\jobname~~~}\input{\jobname}|
\end{tabular}
\end{center}

In an alternative approach,
child documents can be compiled by a specific command line
without additional code or specific definitions:
%
\begin{center}
|... -jobname "|\textit{target}|" "|[\textit{flags}]%
|\includeonly{|\textit{dest}|}\input{|\textit{main}|}"|
\end{center}
%

%%%%%%%%%%%%%%%%%%%%%%%%%%%%%%%%%%%%%%%%%%%%%%%%%%%%%%%%%%%%%%%%%%%%%%%%%%%%%%%%
%%%%%%%%%%%%%%%%%%%%%%%%%%%%%%%%%%%%%%%%%%%%%%%%%%%%%%%%%%%%%%%%%%%%%%%%%%%%%%%%
\section{Information}

%%%%%%%%%%%%%%%%%%%%%%%%%%%%%%%%%%%%%%%%%%%%%%%%%%%%%%%%%%%%%%%%%%%%%%%%%%%%%%%%
\subsection{Copyright}

Copyright \copyright{} 2017--2018 Niklas Beisert

This work may be distributed and/or modified under the
conditions of the \LaTeX{} Project Public License, either version 1.3
of this license or (at your option) any later version.
The latest version of this license is in
  \url{http://www.latex-project.org/lppl.txt}
and version 1.3 or later is part of all distributions of \LaTeX{}
version 2005/12/01 or later.

This work has the LPPL maintenance status `maintained'.

The Current Maintainer of this work is Niklas Beisert.

This work consists of the files |README.txt|, |childdoc.ins| and |childdoc.dtx|
as well as the derived files |childdoc.def|, |cdocsamp.tex|
with |cdocsch1.tex|, |cdocsch2.tex|, |cdocspt3.tex|, |cdocspt4.tex|,
|cdocsdrf.tex|, |cdocsfn1.tex|, |cdocsfn2.tex|
as well as |childdoc.pdf|.

%%%%%%%%%%%%%%%%%%%%%%%%%%%%%%%%%%%%%%%%%%%%%%%%%%%%%%%%%%%%%%%%%%%%%%%%%%%%%%%%
\subsection{Files and Installation}

The package consists of the files:
%
\begin{center}
\begin{tabular}{ll}
    |README.txt|   & readme file \\
    |childdoc.ins| & installation file \\
    |childdoc.dtx| & source file \\
    |childdoc.def| & definition file \\
    |cdocsamp.tex| & sample main file \\
    |cdocsch1.tex| & sample include file \\
    |cdocsch2.tex| & sample include file \\
    |cdocspt3.tex| & sample part file \\
    |cdocspt4.tex| & sample part file \\
    |cdocsdrf.tex| & sample redirection file \\
    |cdocsfn1.tex| & sample redirection file \\
    |cdocsfn2.tex| & sample redirection file \\
    |childdoc.pdf| & manual
\end{tabular}
\end{center}
%
The distribution consists of the files
|README.txt|, |childdoc.ins| and |childdoc.dtx|.
%
\begin{itemize}
\item
Run (pdf)\LaTeX{} on |childdoc.dtx|
to compile the manual |childdoc.pdf| (this file).
\item
Run \LaTeX{} on |childdoc.ins| to create the definitions file |childdoc.def|
and the sample |cdocsamp.tex| with include files
|cdocsch1.tex|, |cdocsch2.tex|, |cdocspt3.tex|, |cdocspt4.tex|,
|cdocsdrf.tex|, |cdocsfn1.tex|, |cdocsfn2.tex|.
Then copy the file |childdoc.def| to an appropriate directory of your \LaTeX{}
distribution, e.g.\ \textit{texmf-root}|/tex/latex/childdoc|.
\end{itemize}

%%%%%%%%%%%%%%%%%%%%%%%%%%%%%%%%%%%%%%%%%%%%%%%%%%%%%%%%%%%%%%%%%%%%%%%%%%%%%%%%
\subsection{Related CTAN Packages}

There are several other packages which offer a similar functionality:
%
\begin{itemize}
\item
The packages
\href{http://ctan.org/pkg/docmute}{\textsf{docmute}},
\href{http://ctan.org/pkg/includex}{\textsf{includex}} and
\href{http://ctan.org/pkg/standalone}{\textsf{standalone}}
provide commands to include only the document body of
a child file thus allowing both files to be compiled individually.
\item
The packages \href{http://ctan.org/pkg/subdocs}{\textsf{subdocs}}
and \href{http://ctan.org/pkg/subfiles}{\textsf{subfiles}}
provide structures in which the main and child documents can be
encapsulated and allowing them to be compiled individually.
The inclusion mechanism is different from the conventional |\include|.
\item
The package \href{http://ctan.org/pkg/combine}{\textsf{combine}}
is an elaborate solution to combine several documents into one.
\end{itemize}
%
See also the CTAN topic \href{http://ctan.org/topic/subdocs}{\textsf{subdocs}}
for further related packages.
The present package differs from the above solutions in that
a document structure constructed with the conventional |\include| mechanism
just needs two extra commands at the top of every file
such that all constituent files can be compiled individually.

%%%%%%%%%%%%%%%%%%%%%%%%%%%%%%%%%%%%%%%%%%%%%%%%%%%%%%%%%%%%%%%%%%%%%%%%%%%%%%%%
%\subsection{Feature Suggestions}
%
%The following is a list of features which may be useful for future
%versions of this package:
%%
%\begin{itemize}
%\item
%\ldots
%\end{itemize}

%%%%%%%%%%%%%%%%%%%%%%%%%%%%%%%%%%%%%%%%%%%%%%%%%%%%%%%%%%%%%%%%%%%%%%%%%%%%%%%%
\subsection{Revision History}

%%%%%%%%%%%%%%%%%%%%%%%%%%%%%%%%%%%%%%%%
\paragraph{v2.0:} 2018/12/30

\begin{itemize}
\item
immediate forward processing
\item
added |\childdocby| mechanism
\item
manual restructured
\end{itemize}

%%%%%%%%%%%%%%%%%%%%%%%%%%%%%%%%%%%%%%%%
\paragraph{v1.6:} 2018/01/17

\begin{itemize}
\item
application for development of include files
\item
corrections to manual
\end{itemize}

%%%%%%%%%%%%%%%%%%%%%%%%%%%%%%%%%%%%%%%%
\paragraph{v1.5:} 2017/05/21

\begin{itemize}
\item
more complete structuring introduced
\item
|\childdocof| introduced
\item
|\childdoc| renamed to |\childdocmain|
\item
|\childredirect| renamed to |\childdocforward| and |\childdocforwardprefix|
and functionality expanded
\end{itemize}

%%%%%%%%%%%%%%%%%%%%%%%%%%%%%%%%%%%%%%%%
\paragraph{v1.0:} 2017/04/27

\begin{itemize}
\item
manual and install package
\item
first version published on CTAN
\end{itemize}

%%%%%%%%%%%%%%%%%%%%%%%%%%%%%%%%%%%%%%%%
\paragraph{v0.6:} 2017/04/26

\begin{itemize}
\item
redirection mechanism added
\end{itemize}

%%%%%%%%%%%%%%%%%%%%%%%%%%%%%%%%%%%%%%%%
\paragraph{v0.5:} 2017/04/26

\begin{itemize}
\item
functionality in definition file
\end{itemize}


%%%%%%%%%%%%%%%%%%%%%%%%%%%%%%%%%%%%%%%%%%%%%%%%%%%%%%%%%%%%%%%%%%%%%%%%%%%%%%%%
%%%%%%%%%%%%%%%%%%%%%%%%%%%%%%%%%%%%%%%%%%%%%%%%%%%%%%%%%%%%%%%%%%%%%%%%%%%%%%%%
%%%%%%%%%%%%%%%%%%%%%%%%%%%%%%%%%%%%%%%%%%%%%%%%%%%%%%%%%%%%%%%%%%%%%%%%%%%%%%%%
\appendix

\settowidth\MacroIndent{\rmfamily\scriptsize 000\ }

 \DocInput{childdoc.dtx}

\end{document}
%</driver>
% \fi
%
% %%%%%%%%%%%%%%%%%%%%%%%%%%%%%%%%%%%%%%%%%%%%%%%%%%%%%%%%%%%%%%%%%%%%%%%%%%%%%%
% %%%%%%%%%%%%%%%%%%%%%%%%%%%%%%%%%%%%%%%%%%%%%%%%%%%%%%%%%%%%%%%%%%%%%%%%%%%%%%
% \section{Sample}
%\iffalse
%<*samplemain>
%\fi
%
% The following presents a sample document
% with two chapters, two parts, a title page,
% a compile flag as well as three forwarding files to set the flag.
% It consists of eight |.tex| files:
% \begin{center}
% \begin{tabular}{ll}
% |cdocsamp.tex|&main file\\
% |cdocsch1.tex|&include file for chapter 1\\
% |cdocsch2.tex|&include file for chapter 2\\
% |cdocspt3.tex|&include file for part 3\\
% |cdocspt4.tex|&include file for part 4\\
% |cdocsdrf.tex|&forwarding file for main file in draft mode\\
% |cdocsfi1.tex|&forwarding file for final version of chapter 1\\
% |cdocsfi2.tex|&forwarding file for final version of chapter 2\\
% \end{tabular}
% \end{center}
% Each of the eight files can be compiled directly by the \LaTeX{} compiler.
%
% %%%%%%%%%%%%%%%%%%%%%%%%%%%%%%%%%%%%%%
% \paragraph{Main File.}
%
% The main file is called |cdocsamp.tex|.
%
% Load the \textsf{childdoc} definitions and
% declare the filename for the main document:
%    \begin{macrocode}
\input{childdoc.def}
\childdocmain{}
%    \end{macrocode}

% Optional override for |\version| flag:
%    \begin{macrocode}
%%\ifchilddoc\else\providecommand{\version}{draft}\fi
%    \end{macrocode}

% Define the default values for the |\version| flag
% (|final| for the main file and |draft| for childs):
%    \begin{macrocode}
\ifchilddoc
\providecommand{\version}{draft}
\else
\providecommand{\version}{final}
\fi
%    \end{macrocode}

% Load the standard document class:
%    \begin{macrocode}
\documentclass[12pt]{article}
%    \end{macrocode}

% Start the document body:
%    \begin{macrocode}
\begin{document}
%    \end{macrocode}

% Declare a title page.
% Print title, part of document being processed and version flag:
%    \begin{macrocode}
\addtocounter{page}{-1}
\begin{center}
{\LARGE\bfseries{}childdoc example\par}
\vspace{1cm}
\ifchilddoc
\ifchilddocmanual part\else chapter\fi:
`\childdocname' of `\childdocjob'\par
\else
main document: `\childdocjob'\par
\fi
version: \version\par
\end{center}
\newpage
%    \end{macrocode}

% Manually include selected file,
% otherwise process as usual:
%    \begin{macrocode}
\ifchilddocmanual
\section*{part `\childdocname'}
\input{\childdocname}
\else
%    \end{macrocode}

% Include the two chapters:
%    \begin{macrocode}
\include{cdocsch1}
\include{cdocsch2}
%    \end{macrocode}

% Include the two parts unless only chapters should be displayed:
%    \begin{macrocode}
\ifchilddoc\else
\section{part three}
\input{cdocspt3}
\section{part four}
\input{cdocspt4}
\fi
%    \end{macrocode}

% Process as usual until here:
%    \begin{macrocode}
\fi
%    \end{macrocode}

% End of document body:
%    \begin{macrocode}
\end{document}
%    \end{macrocode}
%\iffalse
%</samplemain>
%\fi
%
% %%%%%%%%%%%%%%%%%%%%%%%%%%%%%%%%%%%%%%
% \paragraph{Chapter Include Files.}
%
% The include files are called |cdocsch1.tex| and |cdocsch2.tex|.
%
%\iffalse
%<*samplechap1|samplechap2>
%\fi

% Optional override for |\version| flag:
%    \begin{macrocode}
%%\providecommand{\version}{final}
%    \end{macrocode}

% Include the main document:
%    \begin{macrocode}
\input{childdoc.def}
\childdocof{cdocsamp}
%    \end{macrocode}

%\iffalse
%</samplechap1|samplechap2>
%\fi
%
%\iffalse
%<*samplechap1>
%\fi
% Some text for chapter 1:
%    \begin{macrocode}
\section{one}
some text in chapter one
%    \end{macrocode}

%\iffalse
%</samplechap1>
%\fi
% Some text for chapter 2:
%\iffalse
%<*samplechap2>
%\fi
%    \begin{macrocode}
\section{two}
more text in chapter two
%    \end{macrocode}

%\iffalse
%</samplechap2>
%\fi
%
% %%%%%%%%%%%%%%%%%%%%%%%%%%%%%%%%%%%%%%
% \paragraph{Part Include Files.}
%
% The include files are called |cdocspt3.tex| and |cdocspt4.tex|.
%
%\iffalse
%<*samplepart3|samplepart4>
%\fi

% Optional override for |\version| flag:
%    \begin{macrocode}
%%\providecommand{\version}{final}
%    \end{macrocode}

% Include the main document:
%    \begin{macrocode}
\input{childdoc.def}
\childdocby{cdocsamp}
%    \end{macrocode}

%\iffalse
%</samplepart3|samplepart4>
%\fi
%
%\iffalse
%<*samplepart3>
%\fi
% Some text for part 3:
%    \begin{macrocode}
some text in part three
%    \end{macrocode}

%\iffalse
%</samplepart3>
%\fi
% Some text for part 4:
%\iffalse
%<*samplepart4>
%\fi
%    \begin{macrocode}
more text in part four
%    \end{macrocode}

%\iffalse
%</samplepart4>
%\fi
%
% %%%%%%%%%%%%%%%%%%%%%%%%%%%%%%%%%%%%%%
% \paragraph{Forwarding for a Complete Draft.}
%
% The following forwarding file |cdocsdrf.tex|
% compiles the main document in draft mode:
%\iffalse
%<*sampledraft>
%\fi
%    \begin{macrocode}
\def\version{draft}
\input{childdoc.def}
\childdocforward{cdocsamp}
%    \end{macrocode}

%\iffalse
%</sampledraft>
%\fi
%
% %%%%%%%%%%%%%%%%%%%%%%%%%%%%%%%%%%%%%%
% \paragraph{Forwarding for Final Version of the Chapters.}
%
% The following forwarding files |cdocsfn1.tex| and |cdocsfn2.tex|
% (with identical content)
% compile the final versions of the child documents
% |cdocsch1.tex| and |cdocsch2.tex|, respectively:
%\iffalse
%<*samplefinal>
%\fi
%    \begin{macrocode}
\def\version{final}
\input{childdoc.def}
\childdocforwardprefix[cdocsamp]{cdocsfn}{cdocsch}
%    \end{macrocode}

%\iffalse
%</samplefinal>
%\fi
%
% %%%%%%%%%%%%%%%%%%%%%%%%%%%%%%%%%%%%%%
% \paragraph{Command Line Processing.}
%
% The following three command lines generate the output files
% |cdocscld|, |cdocscl1| and |cdocscl2|
% which should be identical to
% |cdocsdrf|, |cdocsch1| and |cdocsfn2|, respectively:
% \begin{center}
% \begin{tabular}{l}
% |latex -jobname cdocscld \|\\
% |  "\def\version{draft}\input{childdoc.def}\childdocforward{cdocsamp}"|\\
% |latex -jobname cdocscl1 \|\\
% |  "\input{childdoc.def}\childdocforward[cdocsamp]{cdocsch1}"|\\
% |latex -jobname cdocscl2 \|\\
% |  "\def\version{final}\input{childdoc.def}\childdocforward{cdocsch2}"|
% \end{tabular}
% \end{center}
% Note that the trailing backslash on each first line
% merely continues the input to the second line
% (for convenient cut ant paste).
% Furthermore, the command |latex| can be replaced by any
% of its alternative versions such as |pdflatex|.
%
% %%%%%%%%%%%%%%%%%%%%%%%%%%%%%%%%%%%%%%%%%%%%%%%%%%%%%%%%%%%%%%%%%%%%%%%%%%%%%%
% %%%%%%%%%%%%%%%%%%%%%%%%%%%%%%%%%%%%%%%%%%%%%%%%%%%%%%%%%%%%%%%%%%%%%%%%%%%%%%
% \section{Implementation}
%\iffalse
%<*package>
%\fi
%
% This section describes the definitions file |childdoc.def|.

% The definitions cannot be loaded using |\usepackage| or |\RequirePackage|
% which has a mechanism to prevent loading a style file more than once.
% When loading the definitions by means of |\input|
% multiple instances have to be prevented manually:
%\iffalse
%This code needs to be before the `\ProvidesFile' directive
%which is defined at the beginning of this file.
%Therefore it is also placed there and commented out here.
%</package>
%<*discard>
%\fi
%    \begin{macrocode}
\ifdefined\childdocmain\endinput\fi
%    \end{macrocode}
%\iffalse
%</discard>
%<*package>
%\fi
%
% \macro{\ifchilddoc}
% \macro{\ifchilddocmanual}
% The conditional |\ifchilddoc| tells whether a
% child (true) or main (false) document is being compiled.
% The conditional |\ifchilddocmanual| tells whether
% the |\includeonly| mechanism is used (false) or
% the selection of child files must be performed manually (true).
% The definitions initialise to false:
%    \begin{macrocode}
\newif\ifchilddoc
\newif\ifchilddocmanual
%    \end{macrocode}

% \macro{\childdocname}
% \macro{\childdocjob}
% The macro |\childdocname| stores the name of the main document
% to be compiled. The macro |\childdocjob| stores the name of
% the document on which the \LaTeX{} compiler was originally invoked.
% The content of |\jobname| cannot be compared
% to filenames specified in the source due to different catcodes.
% The following code rescans |\jobname|, stores the result
% in |\childdocname| and saves a copy in |\childdocjob|:
%    \begin{macrocode}
\edef\childdocname{\scantokens\expandafter{\jobname\noexpand}}
\let\childdocjob\childdocname
%    \end{macrocode}

% \macro{\childdocdisable}
% The macro |\childdocdisable| prevents the main file
% from being processed more than once.
% At this stage, the main document command |\childdocmain|
% is assumed to be called once again where it should do nothing.
% Any subsequent call to it should prevent
% a secondary processing of the main document
% It overwrites the forwarding commands
% |\childdocof| and |\childdocforward|
% with empty macros to prevent further inclusions of the main document:
%    \begin{macrocode}
\newcommand{\childdocdisable}
{
  \renewcommand{\childdocmain}[1]{\renewcommand{\childdocmain}[1]{\endinput}}
  \renewcommand{\childdocof}[1]{}
  \renewcommand{\childdocby}[2][]{}
  \renewcommand{\childdocforward}[2][]{}
  \renewcommand{\childdocdisable}{}
}
%    \end{macrocode}

% \macro{\childdocmain}
% The macro |\childdocmain| is to be called at the top of the main file
% with nothing or the main filename (without extension) as argument.
% First, it breaks loops.
% If the argument is not empty and does not match |\childdocname|
% (which is set by the first inclusion of |childdoc.def|),
% |\ifchilddoc| is set to true, |\includeonly| is applied to the child file
% and |\jobname| is set to the main file
% (for proper handling of |.aux| files):
%    \begin{macrocode}
\newcommand{\childdocmain}[1]
{
  \childdocdisable\childdocmain{}
  \if?#1?\else
    \begingroup
      \def\childdoctmp{#1}
      \ifx\childdoctmp\childdocname
        \def\childdoctmp{}
      \else
        \def\childdoctmp
        {
          \childdoctrue
          \includeonly{\childdocname}
          \def\childdocjob{#1}
          \def\jobname{#1}
        }
      \fi
      \expandafter
    \endgroup
    \childdoctmp
  \fi
}
%    \end{macrocode}

% \macro{\childdocof}
% The command |\childdocof| redirects
% compilation to the main file |#1|.
%    \begin{macrocode}
\newcommand{\childdocof}[1]
{
  \childdocdisable
  \childdoctrue
  \includeonly{\childdocname}
  \def\jobname{#1}
  \def\childdocjob{#1}
  \input{#1}
}
%    \end{macrocode}

% \macro{\childdocby}
% The command |\childdocby| ....
%    \begin{macrocode}
\newcommand{\childdocby}[2][]
{
  \childdocdisable
  \childdoctrue
  \childdocmanualtrue
  \if?#1?\else
    \def\jobname{#2}
  \fi
  \def\childdocjob{#2}
  \input{#2}
  \endinput
}
%    \end{macrocode}

% \macro{\childdocforward}
% The command |\childdocforward| redirects
% compilation to the main file or
% (if the optional argument is given) a child file.
% Parameters are set as if the main file
% or a child file starting with |\childdocof| was compiled.
% Then compilation is handed over to the main file:
%    \begin{macrocode}
\newcommand{\childdocforward}[2][]
{
  \begingroup
    \if?#1?
      \def\childdoctmp
      {
        \def\childdocname{#2}
        \def\childdocjob{#2}
        \def\jobname{#2}
        \input{#2}
        \endinput
      }
    \else
      \def\childdoctmp
      {
        \childdocdisable
        \def\childdocname{#2}
        \childdoctrue
        \includeonly{#2}
        \def\childdocjob{#1}
        \def\jobname{#1}
        \input{#1}
        \endinput
      }
    \fi
    \expandafter
  \endgroup
  \childdoctmp
}
%    \end{macrocode}

% \macro{\childdocforwardprefix}
% The command |\childdocforwardprefix| redirects
% compilation to the main or a child file by means of a pattern.
% The prefix |#1| in the current filename is replaced by |#2|
% and the suffix of the current filename is kept
% (it is assumed that the filename does not contain the substring `|~~~|'
% which is used as a delimiter).
% Compilation is handed over to the new file by |\childdocforward|:
%    \begin{macrocode}
\newcommand{\childdocforwardprefix}[3][]
{
  \begingroup
    \def\childdocextract #2##1~~~{\def\childdoctmp{\childdocforward[#1]{#3##1}}}
    \expandafter\childdocextract\childdocname~~~
    \expandafter
  \endgroup
  \childdoctmp
}
%    \end{macrocode}

% \macro{\childdoc}
% The deprecated macro |\childdoc| is a legacy version of |\childdocmain|:
%    \begin{macrocode}
\newcommand{\childdoc}{\childdocmain}
%    \end{macrocode}

% \macro{\childdocredirect}
% The deprecated macro |\childdocredirect| is a legacy version
% of |\childdocforward| and |\childdocforwardprefix|:
%    \begin{macrocode}
\newcommand{\childdocredirect}[2][]
{
  \begingroup
    \if?#1?
      \def\childdoctmp{\childdocforward{#2}}
    \else
      \def\childdoctmp{\childdocforwardprefix{#1}{#2}}
    \fi
    \expandafter
  \endgroup
  \childdoctmp
}
%    \end{macrocode}

%\iffalse
%</package>
%\fi
%
\endinput
\childdocforward{cdocsch2}"|
% \end{tabular}
% \end{center}
% Note that the trailing backslash on each first line
% merely continues the input to the second line
% (for convenient cut ant paste).
% Furthermore, the command |latex| can be replaced by any
% of its alternative versions such as |pdflatex|.
%
% %%%%%%%%%%%%%%%%%%%%%%%%%%%%%%%%%%%%%%%%%%%%%%%%%%%%%%%%%%%%%%%%%%%%%%%%%%%%%%
% %%%%%%%%%%%%%%%%%%%%%%%%%%%%%%%%%%%%%%%%%%%%%%%%%%%%%%%%%%%%%%%%%%%%%%%%%%%%%%
% \section{Implementation}
%\iffalse
%<*package>
%\fi
%
% This section describes the definitions file |childdoc.def|.

% The definitions cannot be loaded using |\usepackage| or |\RequirePackage|
% which has a mechanism to prevent loading a style file more than once.
% When loading the definitions by means of |\input|
% multiple instances have to be prevented manually:
%\iffalse
%This code needs to be before the `\ProvidesFile' directive
%which is defined at the beginning of this file.
%Therefore it is also placed there and commented out here.
%</package>
%<*discard>
%\fi
%    \begin{macrocode}
\ifdefined\childdocmain\endinput\fi
%    \end{macrocode}
%\iffalse
%</discard>
%<*package>
%\fi
%
% \macro{\ifchilddoc}
% \macro{\ifchilddocmanual}
% The conditional |\ifchilddoc| tells whether a
% child (true) or main (false) document is being compiled.
% The conditional |\ifchilddocmanual| tells whether
% the |\includeonly| mechanism is used (false) or
% the selection of child files must be performed manually (true).
% The definitions initialise to false:
%    \begin{macrocode}
\newif\ifchilddoc
\newif\ifchilddocmanual
%    \end{macrocode}

% \macro{\childdocname}
% \macro{\childdocjob}
% The macro |\childdocname| stores the name of the main document
% to be compiled. The macro |\childdocjob| stores the name of
% the document on which the \LaTeX{} compiler was originally invoked.
% The content of |\jobname| cannot be compared
% to filenames specified in the source due to different catcodes.
% The following code rescans |\jobname|, stores the result
% in |\childdocname| and saves a copy in |\childdocjob|:
%    \begin{macrocode}
\edef\childdocname{\scantokens\expandafter{\jobname\noexpand}}
\let\childdocjob\childdocname
%    \end{macrocode}

% \macro{\childdocdisable}
% The macro |\childdocdisable| prevents the main file
% from being processed more than once.
% At this stage, the main document command |\childdocmain|
% is assumed to be called once again where it should do nothing.
% Any subsequent call to it should prevent
% a secondary processing of the main document
% It overwrites the forwarding commands
% |\childdocof| and |\childdocforward|
% with empty macros to prevent further inclusions of the main document:
%    \begin{macrocode}
\newcommand{\childdocdisable}
{
  \renewcommand{\childdocmain}[1]{\renewcommand{\childdocmain}[1]{\endinput}}
  \renewcommand{\childdocof}[1]{}
  \renewcommand{\childdocby}[2][]{}
  \renewcommand{\childdocforward}[2][]{}
  \renewcommand{\childdocdisable}{}
}
%    \end{macrocode}

% \macro{\childdocmain}
% The macro |\childdocmain| is to be called at the top of the main file
% with nothing or the main filename (without extension) as argument.
% First, it breaks loops.
% If the argument is not empty and does not match |\childdocname|
% (which is set by the first inclusion of |childdoc.def|),
% |\ifchilddoc| is set to true, |\includeonly| is applied to the child file
% and |\jobname| is set to the main file
% (for proper handling of |.aux| files):
%    \begin{macrocode}
\newcommand{\childdocmain}[1]
{
  \childdocdisable\childdocmain{}
  \if?#1?\else
    \begingroup
      \def\childdoctmp{#1}
      \ifx\childdoctmp\childdocname
        \def\childdoctmp{}
      \else
        \def\childdoctmp
        {
          \childdoctrue
          \includeonly{\childdocname}
          \def\childdocjob{#1}
          \def\jobname{#1}
        }
      \fi
      \expandafter
    \endgroup
    \childdoctmp
  \fi
}
%    \end{macrocode}

% \macro{\childdocof}
% The command |\childdocof| redirects
% compilation to the main file |#1|.
%    \begin{macrocode}
\newcommand{\childdocof}[1]
{
  \childdocdisable
  \childdoctrue
  \includeonly{\childdocname}
  \def\jobname{#1}
  \def\childdocjob{#1}
  \input{#1}
}
%    \end{macrocode}

% \macro{\childdocby}
% The command |\childdocby| ....
%    \begin{macrocode}
\newcommand{\childdocby}[2][]
{
  \childdocdisable
  \childdoctrue
  \childdocmanualtrue
  \if?#1?\else
    \def\jobname{#2}
  \fi
  \def\childdocjob{#2}
  \input{#2}
  \endinput
}
%    \end{macrocode}

% \macro{\childdocforward}
% The command |\childdocforward| redirects
% compilation to the main file or
% (if the optional argument is given) a child file.
% Parameters are set as if the main file
% or a child file starting with |\childdocof| was compiled.
% Then compilation is handed over to the main file:
%    \begin{macrocode}
\newcommand{\childdocforward}[2][]
{
  \begingroup
    \if?#1?
      \def\childdoctmp
      {
        \def\childdocname{#2}
        \def\childdocjob{#2}
        \def\jobname{#2}
        \input{#2}
        \endinput
      }
    \else
      \def\childdoctmp
      {
        \childdocdisable
        \def\childdocname{#2}
        \childdoctrue
        \includeonly{#2}
        \def\childdocjob{#1}
        \def\jobname{#1}
        \input{#1}
        \endinput
      }
    \fi
    \expandafter
  \endgroup
  \childdoctmp
}
%    \end{macrocode}

% \macro{\childdocforwardprefix}
% The command |\childdocforwardprefix| redirects
% compilation to the main or a child file by means of a pattern.
% The prefix |#1| in the current filename is replaced by |#2|
% and the suffix of the current filename is kept
% (it is assumed that the filename does not contain the substring `|~~~|'
% which is used as a delimiter).
% Compilation is handed over to the new file by |\childdocforward|:
%    \begin{macrocode}
\newcommand{\childdocforwardprefix}[3][]
{
  \begingroup
    \def\childdocextract #2##1~~~{\def\childdoctmp{\childdocforward[#1]{#3##1}}}
    \expandafter\childdocextract\childdocname~~~
    \expandafter
  \endgroup
  \childdoctmp
}
%    \end{macrocode}

% \macro{\childdoc}
% The deprecated macro |\childdoc| is a legacy version of |\childdocmain|:
%    \begin{macrocode}
\newcommand{\childdoc}{\childdocmain}
%    \end{macrocode}

% \macro{\childdocredirect}
% The deprecated macro |\childdocredirect| is a legacy version
% of |\childdocforward| and |\childdocforwardprefix|:
%    \begin{macrocode}
\newcommand{\childdocredirect}[2][]
{
  \begingroup
    \if?#1?
      \def\childdoctmp{\childdocforward{#2}}
    \else
      \def\childdoctmp{\childdocforwardprefix{#1}{#2}}
    \fi
    \expandafter
  \endgroup
  \childdoctmp
}
%    \end{macrocode}

%\iffalse
%</package>
%\fi
%
\endinput
|\\
|\childdocforward[|\textit{main}|]{|\textit{dest}|}|\\
\end{tabular}
\end{center}
%
The argument \textit{dest} is the destination file
(without extension).
It should be the main file or one of the child files.
Note that further \textsf{childdoc} directives
such as |\childdocof| and |\childdocforward|
in the indicated file will be processed in this form.
The optional argument \textit{main}
passes on directly to the main file \textit{main}
while pretending to compile the child \textit{dest}.
This form behaves as if \textit{dest}
issues |\childdocof{|\textit{main}|}| right away,
and no further \textsf{childdoc} directives will be processed.

%%%%%%%%%%%%%%%%%%%%%%%%%%%%%%%%%%%%%%%%
\DescribeMacro{\...prefix}
In the alternative form |\childdocforwardprefix|,
%
\begin{center}
\begin{tabular}{l}
|% \iffalse
%
% childdoc.dtx Copyright (C) 2017-2018 Niklas Beisert
%
% This work may be distributed and/or modified under the
% conditions of the LaTeX Project Public License, either version 1.3
% of this license or (at your option) any later version.
% The latest version of this license is in
%   http://www.latex-project.org/lppl.txt
% and version 1.3 or later is part of all distributions of LaTeX
% version 2005/12/01 or later.
%
% This work has the LPPL maintenance status `maintained'.
%
% The Current Maintainer of this work is Niklas Beisert.
%
% This work consists of the files childdoc.dtx and childdoc.ins
% and the derived files childdoc.def and cdocsamp.tex with
% cdocsch1.tex, cdocsch2.tex, cdocsdrf.tex, cdocsfn1.tex, cdocsfn2.tex.
%
%<package>\ifdefined\childdocmain\endinput\fi
%<package>\ProvidesFile{childdoc.def}[2018/12/30 v2.0 child document driver]
%<samplemain>\ProvidesFile{cdocsamp.tex}[2018/12/30 v2.0 sample for childdoc]
%<*driver>
%\ProvidesFile{childdoc.drv}[2018/12/30 v2.0 childdoc reference manual file]
\PassOptionsToClass{10pt,a4paper}{article}
\documentclass{ltxdoc}

\usepackage[margin=35mm]{geometry}
\usepackage{hyperref}
\usepackage{hyperxmp}
\usepackage[usenames]{color}

\hypersetup{colorlinks=true}
\hypersetup{pdfstartview=FitH}
\hypersetup{pdfpagemode=UseNone}
\hypersetup{pdfsource={}}
\hypersetup{pdflang={en-UK}}
\hypersetup{pdfcopyright={Copyright 2017-2018 Niklas Beisert.
  This work may be distributed and/or modified under the
  conditions of the LaTeX Project Public License, either version 1.3
  of this license or (at your option) any later version.}}
\hypersetup{pdflicenseurl={http://www.latex-project.org/lppl.txt}}
\hypersetup{pdfcontactaddress={ETH Zurich, ITP, HIT K,
  Wolfgang-Pauli-Strasse 27}}
\hypersetup{pdfcontactpostcode={8093}}
\hypersetup{pdfcontactcity={Zurich}}
\hypersetup{pdfcontactcountry={Switzerland}}
\hypersetup{pdfcontactemail={nbeisert@itp.phys.ethz.ch}}
\hypersetup{pdfcontacturl={http://people.phys.ethz.ch/\xmptilde nbeisert/}}

\newcommand{\secref}[1]{\hyperref[#1]{section \ref*{#1}}}

\parskip1ex
\parindent0pt
\let\olditemize\itemize
\def\itemize{\olditemize\parskip0pt}

\begin{document}

\title{The \textsf{childdoc} Package}
\hypersetup{pdftitle={The childdoc Package}}
\author{Niklas Beisert\\[2ex]
  Institut f\"ur Theoretische Physik\\
  Eidgen\"ossische Technische Hochschule Z\"urich\\
  Wolfgang-Pauli-Strasse 27, 8093 Z\"urich, Switzerland\\[1ex]
  \href{mailto:nbeisert@itp.phys.ethz.ch}
  {\texttt{nbeisert@itp.phys.ethz.ch}}}
\hypersetup{pdfauthor={Niklas Beisert}}
\hypersetup{pdfsubject={Manual for the LaTeX2e Package childdoc}}
\date{30 December 2018, \textsf{v2.0}}
\maketitle

\begin{abstract}\noindent
\textsf{childdoc} is a \LaTeXe{} package
that enables the direct compilation
of document sections included by |\include|
to individual files.
\end{abstract}

\begingroup
\parskip0ex
\tableofcontents
\endgroup

%%%%%%%%%%%%%%%%%%%%%%%%%%%%%%%%%%%%%%%%%%%%%%%%%%%%%%%%%%%%%%%%%%%%%%%%%%%%%%%%
%%%%%%%%%%%%%%%%%%%%%%%%%%%%%%%%%%%%%%%%%%%%%%%%%%%%%%%%%%%%%%%%%%%%%%%%%%%%%%%%
\section{Introduction}

\LaTeX{} provides a mechanism to structure a large document (such as a book)
into a main file and several child files (containing the chapters)
using the |\include| command.
This mechanism is beneficial for documents
which span hundreds of pages in order to
make the source file(s) more manageable.
Moreover, compilation can be restricted to
selected child files by means of the |\includeonly| command.
The latter feature can be used to reduce the compilation time while editing
(this was significantly more useful in the earlier days of \LaTeX{})
or to generate a smaller document which is easier to navigate.
Another application of |\includeonly| is to generate
documents consisting of selected parts of the complete document.

However, there are a few drawbacks of the plain |\include| mechanism:
\begin{itemize}
\item
The child files cannot be compiled on their own,
they can only be compiled via the main file.
A naive editing environment
(such as a text editor with an option
to have the current file processed by \LaTeX)
may require one to switch to the main file before compiling;
attempting to compile the child file produces errors.
\item
The main file must be modified (each time)
to adjust the |\includeonly| command
to the present needs. This easily leaves the main file in a messy state.
\item
The generated document will always carry the filename
of the main document. This is inconvenient if
several child files are to be compiled and
to be kept for distribution.
\end{itemize}

The present package provides a simple interface
to make child files individually compilable by \LaTeX{}.
Compiling a child file then has the same effect as compiling
the main file with an |\includeonly| command
to select the appropriate child.
Moreover the generated document will carry the name of the child
rather than the main file.
This resolves all three above issues.

This feature is meant to make the editing of books,
thesis documents and lecture notes somewhat more convenient.
However, the package can also be used efficiently for
composing a series of documents (such as exercise sheets)
which are typically distributed individually.
It then assists the author in generating the individual documents
(potentially in different versions)
as well as a document containing the collected series.
Another application is in developing style files
or other kinds of included material
where compilation of the style file could redirect
to a sample or test file.

%%%%%%%%%%%%%%%%%%%%%%%%%%%%%%%%%%%%%%%%%%%%%%%%%%%%%%%%%%%%%%%%%%%%%%%%%%%%%%%%
%%%%%%%%%%%%%%%%%%%%%%%%%%%%%%%%%%%%%%%%%%%%%%%%%%%%%%%%%%%%%%%%%%%%%%%%%%%%%%%%
\section{Usage}

First of all, the package \textsf{childdoc} is \emph{not} a standard
\LaTeXe{} |.sty| style file! Therefore it needs to be invoked in
a non-standard way.

%%%%%%%%%%%%%%%%%%%%%%%%%%%%%%%%%%%%%%%%%%%%%%%%%%%%%%%%%%%%%%%%%%%%%%%%%%%%%%%%
\subsection{Included Files}
\label{sec:include}

%%%%%%%%%%%%%%%%%%%%%%%%%%%%%%%%%%%%%%%%
\DescribeMacro{\childdocmain}
To use the package, add the commands
\begin{center}
\begin{tabular}{l}
|% \iffalse
%
% childdoc.dtx Copyright (C) 2017-2018 Niklas Beisert
%
% This work may be distributed and/or modified under the
% conditions of the LaTeX Project Public License, either version 1.3
% of this license or (at your option) any later version.
% The latest version of this license is in
%   http://www.latex-project.org/lppl.txt
% and version 1.3 or later is part of all distributions of LaTeX
% version 2005/12/01 or later.
%
% This work has the LPPL maintenance status `maintained'.
%
% The Current Maintainer of this work is Niklas Beisert.
%
% This work consists of the files childdoc.dtx and childdoc.ins
% and the derived files childdoc.def and cdocsamp.tex with
% cdocsch1.tex, cdocsch2.tex, cdocsdrf.tex, cdocsfn1.tex, cdocsfn2.tex.
%
%<package>\ifdefined\childdocmain\endinput\fi
%<package>\ProvidesFile{childdoc.def}[2018/12/30 v2.0 child document driver]
%<samplemain>\ProvidesFile{cdocsamp.tex}[2018/12/30 v2.0 sample for childdoc]
%<*driver>
%\ProvidesFile{childdoc.drv}[2018/12/30 v2.0 childdoc reference manual file]
\PassOptionsToClass{10pt,a4paper}{article}
\documentclass{ltxdoc}

\usepackage[margin=35mm]{geometry}
\usepackage{hyperref}
\usepackage{hyperxmp}
\usepackage[usenames]{color}

\hypersetup{colorlinks=true}
\hypersetup{pdfstartview=FitH}
\hypersetup{pdfpagemode=UseNone}
\hypersetup{pdfsource={}}
\hypersetup{pdflang={en-UK}}
\hypersetup{pdfcopyright={Copyright 2017-2018 Niklas Beisert.
  This work may be distributed and/or modified under the
  conditions of the LaTeX Project Public License, either version 1.3
  of this license or (at your option) any later version.}}
\hypersetup{pdflicenseurl={http://www.latex-project.org/lppl.txt}}
\hypersetup{pdfcontactaddress={ETH Zurich, ITP, HIT K,
  Wolfgang-Pauli-Strasse 27}}
\hypersetup{pdfcontactpostcode={8093}}
\hypersetup{pdfcontactcity={Zurich}}
\hypersetup{pdfcontactcountry={Switzerland}}
\hypersetup{pdfcontactemail={nbeisert@itp.phys.ethz.ch}}
\hypersetup{pdfcontacturl={http://people.phys.ethz.ch/\xmptilde nbeisert/}}

\newcommand{\secref}[1]{\hyperref[#1]{section \ref*{#1}}}

\parskip1ex
\parindent0pt
\let\olditemize\itemize
\def\itemize{\olditemize\parskip0pt}

\begin{document}

\title{The \textsf{childdoc} Package}
\hypersetup{pdftitle={The childdoc Package}}
\author{Niklas Beisert\\[2ex]
  Institut f\"ur Theoretische Physik\\
  Eidgen\"ossische Technische Hochschule Z\"urich\\
  Wolfgang-Pauli-Strasse 27, 8093 Z\"urich, Switzerland\\[1ex]
  \href{mailto:nbeisert@itp.phys.ethz.ch}
  {\texttt{nbeisert@itp.phys.ethz.ch}}}
\hypersetup{pdfauthor={Niklas Beisert}}
\hypersetup{pdfsubject={Manual for the LaTeX2e Package childdoc}}
\date{30 December 2018, \textsf{v2.0}}
\maketitle

\begin{abstract}\noindent
\textsf{childdoc} is a \LaTeXe{} package
that enables the direct compilation
of document sections included by |\include|
to individual files.
\end{abstract}

\begingroup
\parskip0ex
\tableofcontents
\endgroup

%%%%%%%%%%%%%%%%%%%%%%%%%%%%%%%%%%%%%%%%%%%%%%%%%%%%%%%%%%%%%%%%%%%%%%%%%%%%%%%%
%%%%%%%%%%%%%%%%%%%%%%%%%%%%%%%%%%%%%%%%%%%%%%%%%%%%%%%%%%%%%%%%%%%%%%%%%%%%%%%%
\section{Introduction}

\LaTeX{} provides a mechanism to structure a large document (such as a book)
into a main file and several child files (containing the chapters)
using the |\include| command.
This mechanism is beneficial for documents
which span hundreds of pages in order to
make the source file(s) more manageable.
Moreover, compilation can be restricted to
selected child files by means of the |\includeonly| command.
The latter feature can be used to reduce the compilation time while editing
(this was significantly more useful in the earlier days of \LaTeX{})
or to generate a smaller document which is easier to navigate.
Another application of |\includeonly| is to generate
documents consisting of selected parts of the complete document.

However, there are a few drawbacks of the plain |\include| mechanism:
\begin{itemize}
\item
The child files cannot be compiled on their own,
they can only be compiled via the main file.
A naive editing environment
(such as a text editor with an option
to have the current file processed by \LaTeX)
may require one to switch to the main file before compiling;
attempting to compile the child file produces errors.
\item
The main file must be modified (each time)
to adjust the |\includeonly| command
to the present needs. This easily leaves the main file in a messy state.
\item
The generated document will always carry the filename
of the main document. This is inconvenient if
several child files are to be compiled and
to be kept for distribution.
\end{itemize}

The present package provides a simple interface
to make child files individually compilable by \LaTeX{}.
Compiling a child file then has the same effect as compiling
the main file with an |\includeonly| command
to select the appropriate child.
Moreover the generated document will carry the name of the child
rather than the main file.
This resolves all three above issues.

This feature is meant to make the editing of books,
thesis documents and lecture notes somewhat more convenient.
However, the package can also be used efficiently for
composing a series of documents (such as exercise sheets)
which are typically distributed individually.
It then assists the author in generating the individual documents
(potentially in different versions)
as well as a document containing the collected series.
Another application is in developing style files
or other kinds of included material
where compilation of the style file could redirect
to a sample or test file.

%%%%%%%%%%%%%%%%%%%%%%%%%%%%%%%%%%%%%%%%%%%%%%%%%%%%%%%%%%%%%%%%%%%%%%%%%%%%%%%%
%%%%%%%%%%%%%%%%%%%%%%%%%%%%%%%%%%%%%%%%%%%%%%%%%%%%%%%%%%%%%%%%%%%%%%%%%%%%%%%%
\section{Usage}

First of all, the package \textsf{childdoc} is \emph{not} a standard
\LaTeXe{} |.sty| style file! Therefore it needs to be invoked in
a non-standard way.

%%%%%%%%%%%%%%%%%%%%%%%%%%%%%%%%%%%%%%%%%%%%%%%%%%%%%%%%%%%%%%%%%%%%%%%%%%%%%%%%
\subsection{Included Files}
\label{sec:include}

%%%%%%%%%%%%%%%%%%%%%%%%%%%%%%%%%%%%%%%%
\DescribeMacro{\childdocmain}
To use the package, add the commands
\begin{center}
\begin{tabular}{l}
|\input{childdoc.def}|\\
|\childdocmain{}|\\
\end{tabular}
\end{center}
at the very top of the main \LaTeX{} file,
in particular \emph{before} the |\documentclass| statement!
The argument of |\childdocmain| should be left empty
(but it must be present).

%%%%%%%%%%%%%%%%%%%%%%%%%%%%%%%%%%%%%%%%
\DescribeMacro{\childdocof}
Furthermore, add the commands
\begin{center}
\begin{tabular}{l}
|\input{childdoc.def}|\\
|\childdocof{|\textit{main}|}|\\
\end{tabular}
\end{center}
at the top of every child file \textit{child}
which is included by |\include{|\textit{child}|}|
from within the main file
(or at least for those files to be compiled individually).
The argument \textit{main} must be the filename of the main file.

There are a couple of
considerations in setting up the main and child documents:

%%%%%%%%%%%%%%%%%%%%%%%%%%%%%%%%%%%%%%%%
\paragraph{Restrictions.}

Please note the following restrictions:
\begin{itemize}
\item
|\childdocmain| must be called with one argument \textit{main}
to ensure compatibility with earlier version of the package.
It must either be empty (|\childdocmain{}|)
or precisely match the filename of the main file in which it is specified.
See \secref{sec:detection} for further information.
\item
The filename \textit{main} must be specified without the |.tex| extension.
\item
The filename \textit{main} is case sensitive
(even in case-insensitive file systems)
due to internal string comparison.
\item
The argument \textit{main} should be fully expanded, it cannot be a macro.
\item
Subdirectories and special characters should be avoided in filenames.
\item
The command |\childdocmain{|\textit{main}|}| must be followed by a whitespace.
It should not be followed immediately by another command
or by a comment mark `|%|'.
This is because the \TeX{} parser reads the token immediately following
the argument of |\childdocmain| and puts it
at the beginning of every child section;
however, a white\-space is ignored.
\end{itemize}

%%%%%%%%%%%%%%%%%%%%%%%%%%%%%%%%%%%%%%%%
\paragraph{Content of Main File.}

It is advisable to place all content in the child files included by |\include|.
Any output contained in the main file will appear in all child documents
unless suppressed manually;
it cannot be suppressed automatically by the |\includeonly| directive
and thus should normally be avoided.
A method to include some content in the main file
by means of conditional processing is described in \secref{sec:conditional}.

%%%%%%%%%%%%%%%%%%%%%%%%%%%%%%%%%%%%%%%%
\paragraph{Page Numbering.}

When only a part of the document is compiled,
the appropriate numbering of pages
(as well as other status parameters)
is determined from the |.aux| files.
The latter contain information from previous passes.
However this information needs to propagate through
all intermediate child documents.
Therefore the page numbering in child documents may well
be inconsistent until the complete document is compiled at least once.

A useful (if unconventional) way to always ensure a consistent
page numbering is to restart the numbering in each child document
and denote the pages by `\textit{child}|.|\textit{page}'
where \textit{child} represents the chapter/section number of the child file.
This can be achieved by the command
|\numberwithin{page}{|\textit{child}|}|
of the \textsf{amsmath} package
where \textit{child} can be |chapter| or |section|
depending on the chosen structuring.
Alternatively, one can modify the macro |\thepage| appropriately
and reset the counter |page| at the start of each child file.

%%%%%%%%%%%%%%%%%%%%%%%%%%%%%%%%%%%%%%%%%%%%%%%%%%%%%%%%%%%%%%%%%%%%%%%%%%%%%%%%
\subsection{Conditional Processing}
\label{sec:conditional}

The package provides a mechanism to compile different versions
of a document. To customise the versions further some conditional processing
can come in handy to distinguish which version is being compiled.
The package provides two macros to describe the compilation context:

%%%%%%%%%%%%%%%%%%%%%%%%%%%%%%%%%%%%%%%%
\DescribeMacro{\ifchilddoc}
The conditional |\ifchilddoc| distinguishes between the compilation of
child documents and the main document:
%
\begin{center}
|\ifchilddoc |\textit{child-code}| |[|\||else |\textit{main-code}]| \||fi|
\end{center}

%%%%%%%%%%%%%%%%%%%%%%%%%%%%%%%%%%%%%%%%
\DescribeMacro{\childdocname}
\DescribeMacro{\childdocjob}
The macro |\childdocname| contains the filename (without extension)
of the main or child file being processed.
Note that |\childdocjob| will always contain the name of the main file.

%%%%%%%%%%%%%%%%%%%%%%%%%%%%%%%%%%%%%%%%
\paragraph{Title Page.}

Conditional processing can be used to include a title or banner page
in the main document when proper precautions are taken.
Importantly, the code in the main file should ensure that the page counter
(as well as other status parameters which are stored in the |.aux| files)
takes the same value after the conditional processing.
Otherwise the page numbers may take divergent values
depending on which part is compiled.

For example, a title page could be declared by:
%
\begin{center}
\begin{tabular}{l}
|\ifchilddoc\||else|\\
|\addtocounter{page}{-1}|\\
\textit{code for title page}\\
|\newpage|\\
|\||fi|
\end{tabular}
\end{center}
%
A banner page for the child documents can be generated by:
%
\begin{center}
\begin{tabular}{l}
|\ifchilddoc|\\
|\addtocounter{page}{-1}|\\
\textit{code for banner page}\\
|\newpage|\\
|\||fi|
\end{tabular}
\end{center}
%
Here one could write a message such as:
\begin{center}
|This is the part \childdocname{} of \childdocjob{}.|
\end{center}

%%%%%%%%%%%%%%%%%%%%%%%%%%%%%%%%%%%%%%%%%%%%%%%%%%%%%%%%%%%%%%%%%%%%%%%%%%%%%%%%
\subsection{Flags}
\label{sec:flags}

The package makes it easy to generate different versions
of the main or child documents.
To this end compilation flags can be defined
and assigned different default values.
They will be particularly useful in conjunction
with the forwarding mechanism described in \secref{sec:forward}.

For example, it may be useful to have a flag |\version|
which can be set to |draft| or |final|.
The document source will contain some conditional code
depending on the value of |\version|.
Suppose further, the flag should default to |final| for the main file
and to |draft| for child files
which is a natural assignment for editing the document.
This is achieved by placing the following code
in the preamble of the main document
(below the |\childdocmain| directive):
%
\begin{center}
\begin{tabular}{l}
|\ifchilddoc|\\
|\providecommand{\version}{draft}|\\
|\||else|\\
|\providecommand{\version}{final}|\\
|\||fi|
\end{tabular}
\end{center}
%
The definition by |\providecommand| makes sure
that previous definitions are not overwritten.
Further statements |\providecommand{\version}{...}|
can thus be added before the above code to override it.

For the main file, one might add a line
(between |\childdocmain| and the above block)
%
\begin{center}
|%\ifchilddoc\||else\providecommand{\version}{draft}\||fi|
\end{center}
%
which can be uncommented to produce a draft version.
Likewise one can add a line to the very top of a child file
(above the |\childdocof{|\textit{main}|}| directive)
%
\begin{center}
|%\providecommand{\version}{final}|
\end{center}
%
which can be uncommented to produce the final version of this child document.

%%%%%%%%%%%%%%%%%%%%%%%%%%%%%%%%%%%%%%%%%%%%%%%%%%%%%%%%%%%%%%%%%%%%%%%%%%%%%%%%
\subsection{Forwarding}
\label{sec:forward}

Different versions of the main or child documents
using compilation flags as described in \secref{sec:flags}
can be (permanently) stored in different files
for convenient compilation, viewing and distribution.
To this end, the package defines a command
to pass on compilation to a different file:

%%%%%%%%%%%%%%%%%%%%%%%%%%%%%%%%%%%%%%%%
\DescribeMacro{\childdocforward}
The command |\childdocforward| redirects processing to
another source file:
%
\begin{center}
\begin{tabular}{l}
|\input{childdoc.def}|\\
|\childdocforward[|\textit{main}|]{|\textit{dest}|}|\\
\end{tabular}
\end{center}
%
The argument \textit{dest} is the destination file
(without extension).
It should be the main file or one of the child files.
Note that further \textsf{childdoc} directives
such as |\childdocof| and |\childdocforward|
in the indicated file will be processed in this form.
The optional argument \textit{main}
passes on directly to the main file \textit{main}
while pretending to compile the child \textit{dest}.
This form behaves as if \textit{dest}
issues |\childdocof{|\textit{main}|}| right away,
and no further \textsf{childdoc} directives will be processed.

%%%%%%%%%%%%%%%%%%%%%%%%%%%%%%%%%%%%%%%%
\DescribeMacro{\...prefix}
In the alternative form |\childdocforwardprefix|,
%
\begin{center}
\begin{tabular}{l}
|\input{childdoc.def}|\\
|\childdocforwardprefix[|\textit{main}|]{|\textit{prefix}|}{|\textit{dest}|}|
\end{tabular}
\end{center}
%
the destination file is determined by a pattern
depending on the current file:
To make this work, the current file must be called
`{\textit{prefix}\hspace{0.2em}\textit{suffix}}'
with \textit{prefix} matching precisely the argument.
Processing is then passed on to the file
`{\textit{dest}\hspace{0.2em}\textit{suffix}}'.
Surely, the same effect is achieved by
directly specifying the
argument `{\textit{dest}\hspace{0.2em}\textit{suffix}}'
in the first form.
However, that requires to set up a different file
for each child. With the alternative form of the command
all these files can have exactly the same content
which simplifies setting them up and maintaining them.

For example, the following file |draft.tex|
with a compilation flag |\version| as described in \secref{sec:flags}
compiles the main document as a draft:
%
\begin{center}
\begin{tabular}{l}
|\def\version{draft}|\\
|\input{childdoc.def}|\\
|\childdocforward{|\textit{main}|}|
\end{tabular}
\end{center}
%
Likewise, the following files |final|\textit{nn}|.tex|
compile the final version of the child document
|child|\textit{nn}|.tex|:
%
\begin{center}
\begin{tabular}{l}
|\def\version{final}|\\
|\input{childdoc.def}|\\
|\childdocforwardprefix{final}{child}|
\end{tabular}
\end{center}
%

Note that when several versions of a main file and/or of each child file
are to be generated, it may be convenient to set up a |Makefile| or
shell script to automatise the process.

%%%%%%%%%%%%%%%%%%%%%%%%%%%%%%%%%%%%%%%%%%%%%%%%%%%%%%%%%%%%%%%%%%%%%%%%%%%%%%%%
\subsection{Command Line Processing}
\label{sec:commandline}

The effect of redirection files can also be achieved by invoking
the \LaTeX{} compiler with a more elaborate command line.
Most conveniently this should be done as part
of a shell script or a |Makefile|.

When using \textsf{childdoc} in the main file, the following
command lines effectively perform a redirection
(note that depending on the shell being used,
backslashes may have to be doubled: `|\|' $\to$ `|\\|'):
%
\begin{center}
|... -jobname "|\textit{target}|" |\\|"|[\textit{flags}]%
|\input{childdoc.def}\childdocforward[|\textit{main}|]{|\textit{dest}|}"|
\end{center}
%
Here \textit{target} is the name of the output file,
\textit{main} is the name of the main file
and \textit{dest} is the name of the main or child file to be processed
(all filenames without extensions).
The optional argument \textit{main} can be omitted
if \textit{main} matches \textit{dest}.
Optionally, compilation \textit{flags} can be defined via |\def| commands.
This command line makes the \TeX{} engine believe
it is compiling the file \textit{target}
whose content is specified as the latter parameter.
The provided code then forwards the processing to
\textit{main} or \textit{dest} as described in \secref{sec:forward}.

%%%%%%%%%%%%%%%%%%%%%%%%%%%%%%%%%%%%%%%%%%%%%%%%%%%%%%%%%%%%%%%%%%%%%%%%%%%%%%%%
\subsection{Include by Input}
\label{sec:input}

Including child documents by |\include| has some restrictions by design.
Most notably, the content of a child document always occupies
its own set of pages; pages cannot be shared between child documents.
Usually, this behaviour makes perfect sense
because each child document contain an essential part of the document.
However, in some situations it may be desirable to compose
a document from a collection of parts
without having mandatory page breaks between then.
For this case, the package
provides a mechanism to include parts
by |\input| which can also be processed individually.
However, by construction this mechanism
requires manual handling of the content to be output.

%%%%%%%%%%%%%%%%%%%%%%%%%%%%%%%%%%%%%%%%
\DescribeMacro{\ifchilddocmanual}
The main file should be prepared as usual, see \secref{sec:include}.
However, the document body must make a distinction
between processing of an individual part and of the main document, e.g.:
%
\begin{center}
\begin{tabular}{l}
|\ifchilddocmanual|\\
|\input{\childdocname}|\\
|\||else|\\
\textit{document body with }|\input{|\textit{part}|}|\\
|\||fi|
\end{tabular}
\end{center}
%
The conditional |\ifchilddocmanual| is true whenever
a part to be included by |\input| is being compiled,
and the name of the part is stored in |\childdocname|.

%%%%%%%%%%%%%%%%%%%%%%%%%%%%%%%%%%%%%%%%
\DescribeMacro{\childdocby}
Each part to be included by |\input| should start with:
%
\begin{center}
\begin{tabular}{l}
|\input{childdoc.def}|\\
|\childdocby{|\textit{main}|}|\\
\end{tabular}
\end{center}
%
The directive |\childdocby| is similar to |\childdocof|
described in \secref{sec:include},
but the subsequent selection of content must be done manually.
To that end, both |\ifchilddoc| and |\ifchilddocmanual|
will be true upon processing of a part,
and the name of the part is stored in |\childdocname|.
Note that |\jobname| will be set to the filename of the current part
so that each part receives an individual |.aux| file
that does not interfere with the |.aux| file(s) of the main document.
This behaviour can be altered by the alternative form
|\childdocby[*]{|\textit{main}|}| (with a non-empty optional argument)
which uses the |.aux| file of the main document
by setting |\jobname| to \textit{main}.

%%%%%%%%%%%%%%%%%%%%%%%%%%%%%%%%%%%%%%%%%%%%%%%%%%%%%%%%%%%%%%%%%%%%%%%%%%%%%%%%
\subsection{Driver Development}
\label{sec:driver}

The \textsf{childdoc} mechanism can also be use for the development
of definition files such as \LaTeX{} styles or classes.
This case differs from the above setup with multiple parts
included by |\include| in that no |\includeonly| should be invoked.
This can be achieved by starting the include file
(before |\ProvidesPackage|) with:
%
\begin{center}
\begin{tabular}{l}
|\input{childdoc.def}|\\
|\childdocforward{|\textit{main}|}|\\
\end{tabular}
\end{center}
%
or alternatively with:
%
\begin{center}
\begin{tabular}{l}
|\input{childdoc.def}|\\
|\childdocby{|\textit{main}|}|\\
\end{tabular}
\end{center}
%
Both forms have slightly different effects as described above.
The main file is prepared as usual, see \secref{sec:include}.

%%%%%%%%%%%%%%%%%%%%%%%%%%%%%%%%%%%%%%%%%%%%%%%%%%%%%%%%%%%%%%%%%%%%%%%%%%%%%%%%
\subsection{Legacy Detection}
\label{sec:detection}

The directive |\childdocmain| in the main file can detect
whether the complete document or merely a child is to be compiled
even without using the directive |\childdocof|.
This method is deprecated because it is less robust
and there is no compelling reason to use it;
it is merely provided for backward compatibility
and it may be removed in future versions.

If the detection mechanism is to be used,
it is mandatory to correctly specify
the filename of the main file as the argument of |\childdocmain|:
%
\begin{center}
\begin{tabular}{l}
|\input{childdoc.def}|\\
|\childdocmain{|\textit{main}|}|\\
\end{tabular}
\end{center}
%
If |\jobname| does not match the argument \textit{main} of |\childdocmain|,
it is assumed that |\jobname| points to the child file to be compiled.
When using |\childdocmain| with the main file specified as argument,
it suffices to start a child file
with just |\input{|\textit{main}|}|
without loading of the package and using |\childdocof|.
If instead all processing is done
with the appropriate \textsf{childdoc} directives,
the argument of \textit{main} of |\childdocmain| can be empty.

An alternative version of the command line processing described
in \secref{sec:commandline} using the detection mechanism reads:
%
\begin{center}
|... -jobname "|\textit{target}|" "|[\textit{flags}]%
[|\def\jobname{|\textit{dest}|}|]|\input{|\textit{main}|}"|
\end{center}

%%%%%%%%%%%%%%%%%%%%%%%%%%%%%%%%%%%%%%%%%%%%%%%%%%%%%%%%%%%%%%%%%%%%%%%%%%%%%%%%
\subsection{Manual Code}
\label{sec:manual}

In case one cannot be certain whether the definitions file |childdoc.def|
is installed on the target \TeX{} distribution
and one prefers not to ship it,
it is conceivable to paste a few relevant commands into the sources.

To that end, drop all statements |\input{childdoc.def}|
and perform the replacements as outlined below.
Instead of |\childdocmain{|\textit{main}|}| add the following code
to the top of the main file:
%
\begin{center}
\begin{tabular}{l}
|\||ifdefined\childdocname\endinput\||fi\newif\ifchilddoc|\\
|\edef\childdocname{\scantokens\expandafter{\jobname\noexpand}}|\\
|\def\childdocmain{|\textit{main}|}\||ifx\childdocmain\childdocname\||else|\\
|\childdoctrue\includeonly{\childdocname}\let\jobname\childdocmain\||fi|\\
\end{tabular}
\end{center}
%
Instead of |\childdocof{|\textit{main}|}| just include the main file
at the top of each child file:
%
\begin{center}
|\input{|\textit{main}|}|
\end{center}
%
A simple redirection |\childdocforward{|\textit{dest}|}| is achieved by:
%
\begin{center}
|\def\jobname{|\textit{dest}|}\input{\jobname}|
\end{center}
%
The redirection with prefix
|\childdocforwardprefix[|\textit{prefix}|]{|\textit{dest}|}|
is accomplished by:
%
\begin{center}
\begin{tabular}{l}
|{\edef\jobname{\scantokens\expandafter{\jobname\noexpand}}|\\
|\def\redirectjob |\textit{prefix}|#1~~~{\gdef\jobname{|\textit{dest}|#1}}|\\
|\expandafter\redirectjob\jobname~~~}\input{\jobname}|
\end{tabular}
\end{center}

In an alternative approach,
child documents can be compiled by a specific command line
without additional code or specific definitions:
%
\begin{center}
|... -jobname "|\textit{target}|" "|[\textit{flags}]%
|\includeonly{|\textit{dest}|}\input{|\textit{main}|}"|
\end{center}
%

%%%%%%%%%%%%%%%%%%%%%%%%%%%%%%%%%%%%%%%%%%%%%%%%%%%%%%%%%%%%%%%%%%%%%%%%%%%%%%%%
%%%%%%%%%%%%%%%%%%%%%%%%%%%%%%%%%%%%%%%%%%%%%%%%%%%%%%%%%%%%%%%%%%%%%%%%%%%%%%%%
\section{Information}

%%%%%%%%%%%%%%%%%%%%%%%%%%%%%%%%%%%%%%%%%%%%%%%%%%%%%%%%%%%%%%%%%%%%%%%%%%%%%%%%
\subsection{Copyright}

Copyright \copyright{} 2017--2018 Niklas Beisert

This work may be distributed and/or modified under the
conditions of the \LaTeX{} Project Public License, either version 1.3
of this license or (at your option) any later version.
The latest version of this license is in
  \url{http://www.latex-project.org/lppl.txt}
and version 1.3 or later is part of all distributions of \LaTeX{}
version 2005/12/01 or later.

This work has the LPPL maintenance status `maintained'.

The Current Maintainer of this work is Niklas Beisert.

This work consists of the files |README.txt|, |childdoc.ins| and |childdoc.dtx|
as well as the derived files |childdoc.def|, |cdocsamp.tex|
with |cdocsch1.tex|, |cdocsch2.tex|, |cdocspt3.tex|, |cdocspt4.tex|,
|cdocsdrf.tex|, |cdocsfn1.tex|, |cdocsfn2.tex|
as well as |childdoc.pdf|.

%%%%%%%%%%%%%%%%%%%%%%%%%%%%%%%%%%%%%%%%%%%%%%%%%%%%%%%%%%%%%%%%%%%%%%%%%%%%%%%%
\subsection{Files and Installation}

The package consists of the files:
%
\begin{center}
\begin{tabular}{ll}
    |README.txt|   & readme file \\
    |childdoc.ins| & installation file \\
    |childdoc.dtx| & source file \\
    |childdoc.def| & definition file \\
    |cdocsamp.tex| & sample main file \\
    |cdocsch1.tex| & sample include file \\
    |cdocsch2.tex| & sample include file \\
    |cdocspt3.tex| & sample part file \\
    |cdocspt4.tex| & sample part file \\
    |cdocsdrf.tex| & sample redirection file \\
    |cdocsfn1.tex| & sample redirection file \\
    |cdocsfn2.tex| & sample redirection file \\
    |childdoc.pdf| & manual
\end{tabular}
\end{center}
%
The distribution consists of the files
|README.txt|, |childdoc.ins| and |childdoc.dtx|.
%
\begin{itemize}
\item
Run (pdf)\LaTeX{} on |childdoc.dtx|
to compile the manual |childdoc.pdf| (this file).
\item
Run \LaTeX{} on |childdoc.ins| to create the definitions file |childdoc.def|
and the sample |cdocsamp.tex| with include files
|cdocsch1.tex|, |cdocsch2.tex|, |cdocspt3.tex|, |cdocspt4.tex|,
|cdocsdrf.tex|, |cdocsfn1.tex|, |cdocsfn2.tex|.
Then copy the file |childdoc.def| to an appropriate directory of your \LaTeX{}
distribution, e.g.\ \textit{texmf-root}|/tex/latex/childdoc|.
\end{itemize}

%%%%%%%%%%%%%%%%%%%%%%%%%%%%%%%%%%%%%%%%%%%%%%%%%%%%%%%%%%%%%%%%%%%%%%%%%%%%%%%%
\subsection{Related CTAN Packages}

There are several other packages which offer a similar functionality:
%
\begin{itemize}
\item
The packages
\href{http://ctan.org/pkg/docmute}{\textsf{docmute}},
\href{http://ctan.org/pkg/includex}{\textsf{includex}} and
\href{http://ctan.org/pkg/standalone}{\textsf{standalone}}
provide commands to include only the document body of
a child file thus allowing both files to be compiled individually.
\item
The packages \href{http://ctan.org/pkg/subdocs}{\textsf{subdocs}}
and \href{http://ctan.org/pkg/subfiles}{\textsf{subfiles}}
provide structures in which the main and child documents can be
encapsulated and allowing them to be compiled individually.
The inclusion mechanism is different from the conventional |\include|.
\item
The package \href{http://ctan.org/pkg/combine}{\textsf{combine}}
is an elaborate solution to combine several documents into one.
\end{itemize}
%
See also the CTAN topic \href{http://ctan.org/topic/subdocs}{\textsf{subdocs}}
for further related packages.
The present package differs from the above solutions in that
a document structure constructed with the conventional |\include| mechanism
just needs two extra commands at the top of every file
such that all constituent files can be compiled individually.

%%%%%%%%%%%%%%%%%%%%%%%%%%%%%%%%%%%%%%%%%%%%%%%%%%%%%%%%%%%%%%%%%%%%%%%%%%%%%%%%
%\subsection{Feature Suggestions}
%
%The following is a list of features which may be useful for future
%versions of this package:
%%
%\begin{itemize}
%\item
%\ldots
%\end{itemize}

%%%%%%%%%%%%%%%%%%%%%%%%%%%%%%%%%%%%%%%%%%%%%%%%%%%%%%%%%%%%%%%%%%%%%%%%%%%%%%%%
\subsection{Revision History}

%%%%%%%%%%%%%%%%%%%%%%%%%%%%%%%%%%%%%%%%
\paragraph{v2.0:} 2018/12/30

\begin{itemize}
\item
immediate forward processing
\item
added |\childdocby| mechanism
\item
manual restructured
\end{itemize}

%%%%%%%%%%%%%%%%%%%%%%%%%%%%%%%%%%%%%%%%
\paragraph{v1.6:} 2018/01/17

\begin{itemize}
\item
application for development of include files
\item
corrections to manual
\end{itemize}

%%%%%%%%%%%%%%%%%%%%%%%%%%%%%%%%%%%%%%%%
\paragraph{v1.5:} 2017/05/21

\begin{itemize}
\item
more complete structuring introduced
\item
|\childdocof| introduced
\item
|\childdoc| renamed to |\childdocmain|
\item
|\childredirect| renamed to |\childdocforward| and |\childdocforwardprefix|
and functionality expanded
\end{itemize}

%%%%%%%%%%%%%%%%%%%%%%%%%%%%%%%%%%%%%%%%
\paragraph{v1.0:} 2017/04/27

\begin{itemize}
\item
manual and install package
\item
first version published on CTAN
\end{itemize}

%%%%%%%%%%%%%%%%%%%%%%%%%%%%%%%%%%%%%%%%
\paragraph{v0.6:} 2017/04/26

\begin{itemize}
\item
redirection mechanism added
\end{itemize}

%%%%%%%%%%%%%%%%%%%%%%%%%%%%%%%%%%%%%%%%
\paragraph{v0.5:} 2017/04/26

\begin{itemize}
\item
functionality in definition file
\end{itemize}


%%%%%%%%%%%%%%%%%%%%%%%%%%%%%%%%%%%%%%%%%%%%%%%%%%%%%%%%%%%%%%%%%%%%%%%%%%%%%%%%
%%%%%%%%%%%%%%%%%%%%%%%%%%%%%%%%%%%%%%%%%%%%%%%%%%%%%%%%%%%%%%%%%%%%%%%%%%%%%%%%
%%%%%%%%%%%%%%%%%%%%%%%%%%%%%%%%%%%%%%%%%%%%%%%%%%%%%%%%%%%%%%%%%%%%%%%%%%%%%%%%
\appendix

\settowidth\MacroIndent{\rmfamily\scriptsize 000\ }

 \DocInput{childdoc.dtx}

\end{document}
%</driver>
% \fi
%
% %%%%%%%%%%%%%%%%%%%%%%%%%%%%%%%%%%%%%%%%%%%%%%%%%%%%%%%%%%%%%%%%%%%%%%%%%%%%%%
% %%%%%%%%%%%%%%%%%%%%%%%%%%%%%%%%%%%%%%%%%%%%%%%%%%%%%%%%%%%%%%%%%%%%%%%%%%%%%%
% \section{Sample}
%\iffalse
%<*samplemain>
%\fi
%
% The following presents a sample document
% with two chapters, two parts, a title page,
% a compile flag as well as three forwarding files to set the flag.
% It consists of eight |.tex| files:
% \begin{center}
% \begin{tabular}{ll}
% |cdocsamp.tex|&main file\\
% |cdocsch1.tex|&include file for chapter 1\\
% |cdocsch2.tex|&include file for chapter 2\\
% |cdocspt3.tex|&include file for part 3\\
% |cdocspt4.tex|&include file for part 4\\
% |cdocsdrf.tex|&forwarding file for main file in draft mode\\
% |cdocsfi1.tex|&forwarding file for final version of chapter 1\\
% |cdocsfi2.tex|&forwarding file for final version of chapter 2\\
% \end{tabular}
% \end{center}
% Each of the eight files can be compiled directly by the \LaTeX{} compiler.
%
% %%%%%%%%%%%%%%%%%%%%%%%%%%%%%%%%%%%%%%
% \paragraph{Main File.}
%
% The main file is called |cdocsamp.tex|.
%
% Load the \textsf{childdoc} definitions and
% declare the filename for the main document:
%    \begin{macrocode}
\input{childdoc.def}
\childdocmain{}
%    \end{macrocode}

% Optional override for |\version| flag:
%    \begin{macrocode}
%%\ifchilddoc\else\providecommand{\version}{draft}\fi
%    \end{macrocode}

% Define the default values for the |\version| flag
% (|final| for the main file and |draft| for childs):
%    \begin{macrocode}
\ifchilddoc
\providecommand{\version}{draft}
\else
\providecommand{\version}{final}
\fi
%    \end{macrocode}

% Load the standard document class:
%    \begin{macrocode}
\documentclass[12pt]{article}
%    \end{macrocode}

% Start the document body:
%    \begin{macrocode}
\begin{document}
%    \end{macrocode}

% Declare a title page.
% Print title, part of document being processed and version flag:
%    \begin{macrocode}
\addtocounter{page}{-1}
\begin{center}
{\LARGE\bfseries{}childdoc example\par}
\vspace{1cm}
\ifchilddoc
\ifchilddocmanual part\else chapter\fi:
`\childdocname' of `\childdocjob'\par
\else
main document: `\childdocjob'\par
\fi
version: \version\par
\end{center}
\newpage
%    \end{macrocode}

% Manually include selected file,
% otherwise process as usual:
%    \begin{macrocode}
\ifchilddocmanual
\section*{part `\childdocname'}
\input{\childdocname}
\else
%    \end{macrocode}

% Include the two chapters:
%    \begin{macrocode}
\include{cdocsch1}
\include{cdocsch2}
%    \end{macrocode}

% Include the two parts unless only chapters should be displayed:
%    \begin{macrocode}
\ifchilddoc\else
\section{part three}
\input{cdocspt3}
\section{part four}
\input{cdocspt4}
\fi
%    \end{macrocode}

% Process as usual until here:
%    \begin{macrocode}
\fi
%    \end{macrocode}

% End of document body:
%    \begin{macrocode}
\end{document}
%    \end{macrocode}
%\iffalse
%</samplemain>
%\fi
%
% %%%%%%%%%%%%%%%%%%%%%%%%%%%%%%%%%%%%%%
% \paragraph{Chapter Include Files.}
%
% The include files are called |cdocsch1.tex| and |cdocsch2.tex|.
%
%\iffalse
%<*samplechap1|samplechap2>
%\fi

% Optional override for |\version| flag:
%    \begin{macrocode}
%%\providecommand{\version}{final}
%    \end{macrocode}

% Include the main document:
%    \begin{macrocode}
\input{childdoc.def}
\childdocof{cdocsamp}
%    \end{macrocode}

%\iffalse
%</samplechap1|samplechap2>
%\fi
%
%\iffalse
%<*samplechap1>
%\fi
% Some text for chapter 1:
%    \begin{macrocode}
\section{one}
some text in chapter one
%    \end{macrocode}

%\iffalse
%</samplechap1>
%\fi
% Some text for chapter 2:
%\iffalse
%<*samplechap2>
%\fi
%    \begin{macrocode}
\section{two}
more text in chapter two
%    \end{macrocode}

%\iffalse
%</samplechap2>
%\fi
%
% %%%%%%%%%%%%%%%%%%%%%%%%%%%%%%%%%%%%%%
% \paragraph{Part Include Files.}
%
% The include files are called |cdocspt3.tex| and |cdocspt4.tex|.
%
%\iffalse
%<*samplepart3|samplepart4>
%\fi

% Optional override for |\version| flag:
%    \begin{macrocode}
%%\providecommand{\version}{final}
%    \end{macrocode}

% Include the main document:
%    \begin{macrocode}
\input{childdoc.def}
\childdocby{cdocsamp}
%    \end{macrocode}

%\iffalse
%</samplepart3|samplepart4>
%\fi
%
%\iffalse
%<*samplepart3>
%\fi
% Some text for part 3:
%    \begin{macrocode}
some text in part three
%    \end{macrocode}

%\iffalse
%</samplepart3>
%\fi
% Some text for part 4:
%\iffalse
%<*samplepart4>
%\fi
%    \begin{macrocode}
more text in part four
%    \end{macrocode}

%\iffalse
%</samplepart4>
%\fi
%
% %%%%%%%%%%%%%%%%%%%%%%%%%%%%%%%%%%%%%%
% \paragraph{Forwarding for a Complete Draft.}
%
% The following forwarding file |cdocsdrf.tex|
% compiles the main document in draft mode:
%\iffalse
%<*sampledraft>
%\fi
%    \begin{macrocode}
\def\version{draft}
\input{childdoc.def}
\childdocforward{cdocsamp}
%    \end{macrocode}

%\iffalse
%</sampledraft>
%\fi
%
% %%%%%%%%%%%%%%%%%%%%%%%%%%%%%%%%%%%%%%
% \paragraph{Forwarding for Final Version of the Chapters.}
%
% The following forwarding files |cdocsfn1.tex| and |cdocsfn2.tex|
% (with identical content)
% compile the final versions of the child documents
% |cdocsch1.tex| and |cdocsch2.tex|, respectively:
%\iffalse
%<*samplefinal>
%\fi
%    \begin{macrocode}
\def\version{final}
\input{childdoc.def}
\childdocforwardprefix[cdocsamp]{cdocsfn}{cdocsch}
%    \end{macrocode}

%\iffalse
%</samplefinal>
%\fi
%
% %%%%%%%%%%%%%%%%%%%%%%%%%%%%%%%%%%%%%%
% \paragraph{Command Line Processing.}
%
% The following three command lines generate the output files
% |cdocscld|, |cdocscl1| and |cdocscl2|
% which should be identical to
% |cdocsdrf|, |cdocsch1| and |cdocsfn2|, respectively:
% \begin{center}
% \begin{tabular}{l}
% |latex -jobname cdocscld \|\\
% |  "\def\version{draft}\input{childdoc.def}\childdocforward{cdocsamp}"|\\
% |latex -jobname cdocscl1 \|\\
% |  "\input{childdoc.def}\childdocforward[cdocsamp]{cdocsch1}"|\\
% |latex -jobname cdocscl2 \|\\
% |  "\def\version{final}\input{childdoc.def}\childdocforward{cdocsch2}"|
% \end{tabular}
% \end{center}
% Note that the trailing backslash on each first line
% merely continues the input to the second line
% (for convenient cut ant paste).
% Furthermore, the command |latex| can be replaced by any
% of its alternative versions such as |pdflatex|.
%
% %%%%%%%%%%%%%%%%%%%%%%%%%%%%%%%%%%%%%%%%%%%%%%%%%%%%%%%%%%%%%%%%%%%%%%%%%%%%%%
% %%%%%%%%%%%%%%%%%%%%%%%%%%%%%%%%%%%%%%%%%%%%%%%%%%%%%%%%%%%%%%%%%%%%%%%%%%%%%%
% \section{Implementation}
%\iffalse
%<*package>
%\fi
%
% This section describes the definitions file |childdoc.def|.

% The definitions cannot be loaded using |\usepackage| or |\RequirePackage|
% which has a mechanism to prevent loading a style file more than once.
% When loading the definitions by means of |\input|
% multiple instances have to be prevented manually:
%\iffalse
%This code needs to be before the `\ProvidesFile' directive
%which is defined at the beginning of this file.
%Therefore it is also placed there and commented out here.
%</package>
%<*discard>
%\fi
%    \begin{macrocode}
\ifdefined\childdocmain\endinput\fi
%    \end{macrocode}
%\iffalse
%</discard>
%<*package>
%\fi
%
% \macro{\ifchilddoc}
% \macro{\ifchilddocmanual}
% The conditional |\ifchilddoc| tells whether a
% child (true) or main (false) document is being compiled.
% The conditional |\ifchilddocmanual| tells whether
% the |\includeonly| mechanism is used (false) or
% the selection of child files must be performed manually (true).
% The definitions initialise to false:
%    \begin{macrocode}
\newif\ifchilddoc
\newif\ifchilddocmanual
%    \end{macrocode}

% \macro{\childdocname}
% \macro{\childdocjob}
% The macro |\childdocname| stores the name of the main document
% to be compiled. The macro |\childdocjob| stores the name of
% the document on which the \LaTeX{} compiler was originally invoked.
% The content of |\jobname| cannot be compared
% to filenames specified in the source due to different catcodes.
% The following code rescans |\jobname|, stores the result
% in |\childdocname| and saves a copy in |\childdocjob|:
%    \begin{macrocode}
\edef\childdocname{\scantokens\expandafter{\jobname\noexpand}}
\let\childdocjob\childdocname
%    \end{macrocode}

% \macro{\childdocdisable}
% The macro |\childdocdisable| prevents the main file
% from being processed more than once.
% At this stage, the main document command |\childdocmain|
% is assumed to be called once again where it should do nothing.
% Any subsequent call to it should prevent
% a secondary processing of the main document
% It overwrites the forwarding commands
% |\childdocof| and |\childdocforward|
% with empty macros to prevent further inclusions of the main document:
%    \begin{macrocode}
\newcommand{\childdocdisable}
{
  \renewcommand{\childdocmain}[1]{\renewcommand{\childdocmain}[1]{\endinput}}
  \renewcommand{\childdocof}[1]{}
  \renewcommand{\childdocby}[2][]{}
  \renewcommand{\childdocforward}[2][]{}
  \renewcommand{\childdocdisable}{}
}
%    \end{macrocode}

% \macro{\childdocmain}
% The macro |\childdocmain| is to be called at the top of the main file
% with nothing or the main filename (without extension) as argument.
% First, it breaks loops.
% If the argument is not empty and does not match |\childdocname|
% (which is set by the first inclusion of |childdoc.def|),
% |\ifchilddoc| is set to true, |\includeonly| is applied to the child file
% and |\jobname| is set to the main file
% (for proper handling of |.aux| files):
%    \begin{macrocode}
\newcommand{\childdocmain}[1]
{
  \childdocdisable\childdocmain{}
  \if?#1?\else
    \begingroup
      \def\childdoctmp{#1}
      \ifx\childdoctmp\childdocname
        \def\childdoctmp{}
      \else
        \def\childdoctmp
        {
          \childdoctrue
          \includeonly{\childdocname}
          \def\childdocjob{#1}
          \def\jobname{#1}
        }
      \fi
      \expandafter
    \endgroup
    \childdoctmp
  \fi
}
%    \end{macrocode}

% \macro{\childdocof}
% The command |\childdocof| redirects
% compilation to the main file |#1|.
%    \begin{macrocode}
\newcommand{\childdocof}[1]
{
  \childdocdisable
  \childdoctrue
  \includeonly{\childdocname}
  \def\jobname{#1}
  \def\childdocjob{#1}
  \input{#1}
}
%    \end{macrocode}

% \macro{\childdocby}
% The command |\childdocby| ....
%    \begin{macrocode}
\newcommand{\childdocby}[2][]
{
  \childdocdisable
  \childdoctrue
  \childdocmanualtrue
  \if?#1?\else
    \def\jobname{#2}
  \fi
  \def\childdocjob{#2}
  \input{#2}
  \endinput
}
%    \end{macrocode}

% \macro{\childdocforward}
% The command |\childdocforward| redirects
% compilation to the main file or
% (if the optional argument is given) a child file.
% Parameters are set as if the main file
% or a child file starting with |\childdocof| was compiled.
% Then compilation is handed over to the main file:
%    \begin{macrocode}
\newcommand{\childdocforward}[2][]
{
  \begingroup
    \if?#1?
      \def\childdoctmp
      {
        \def\childdocname{#2}
        \def\childdocjob{#2}
        \def\jobname{#2}
        \input{#2}
        \endinput
      }
    \else
      \def\childdoctmp
      {
        \childdocdisable
        \def\childdocname{#2}
        \childdoctrue
        \includeonly{#2}
        \def\childdocjob{#1}
        \def\jobname{#1}
        \input{#1}
        \endinput
      }
    \fi
    \expandafter
  \endgroup
  \childdoctmp
}
%    \end{macrocode}

% \macro{\childdocforwardprefix}
% The command |\childdocforwardprefix| redirects
% compilation to the main or a child file by means of a pattern.
% The prefix |#1| in the current filename is replaced by |#2|
% and the suffix of the current filename is kept
% (it is assumed that the filename does not contain the substring `|~~~|'
% which is used as a delimiter).
% Compilation is handed over to the new file by |\childdocforward|:
%    \begin{macrocode}
\newcommand{\childdocforwardprefix}[3][]
{
  \begingroup
    \def\childdocextract #2##1~~~{\def\childdoctmp{\childdocforward[#1]{#3##1}}}
    \expandafter\childdocextract\childdocname~~~
    \expandafter
  \endgroup
  \childdoctmp
}
%    \end{macrocode}

% \macro{\childdoc}
% The deprecated macro |\childdoc| is a legacy version of |\childdocmain|:
%    \begin{macrocode}
\newcommand{\childdoc}{\childdocmain}
%    \end{macrocode}

% \macro{\childdocredirect}
% The deprecated macro |\childdocredirect| is a legacy version
% of |\childdocforward| and |\childdocforwardprefix|:
%    \begin{macrocode}
\newcommand{\childdocredirect}[2][]
{
  \begingroup
    \if?#1?
      \def\childdoctmp{\childdocforward{#2}}
    \else
      \def\childdoctmp{\childdocforwardprefix{#1}{#2}}
    \fi
    \expandafter
  \endgroup
  \childdoctmp
}
%    \end{macrocode}

%\iffalse
%</package>
%\fi
%
\endinput
|\\
|\childdocmain{}|\\
\end{tabular}
\end{center}
at the very top of the main \LaTeX{} file,
in particular \emph{before} the |\documentclass| statement!
The argument of |\childdocmain| should be left empty
(but it must be present).

%%%%%%%%%%%%%%%%%%%%%%%%%%%%%%%%%%%%%%%%
\DescribeMacro{\childdocof}
Furthermore, add the commands
\begin{center}
\begin{tabular}{l}
|% \iffalse
%
% childdoc.dtx Copyright (C) 2017-2018 Niklas Beisert
%
% This work may be distributed and/or modified under the
% conditions of the LaTeX Project Public License, either version 1.3
% of this license or (at your option) any later version.
% The latest version of this license is in
%   http://www.latex-project.org/lppl.txt
% and version 1.3 or later is part of all distributions of LaTeX
% version 2005/12/01 or later.
%
% This work has the LPPL maintenance status `maintained'.
%
% The Current Maintainer of this work is Niklas Beisert.
%
% This work consists of the files childdoc.dtx and childdoc.ins
% and the derived files childdoc.def and cdocsamp.tex with
% cdocsch1.tex, cdocsch2.tex, cdocsdrf.tex, cdocsfn1.tex, cdocsfn2.tex.
%
%<package>\ifdefined\childdocmain\endinput\fi
%<package>\ProvidesFile{childdoc.def}[2018/12/30 v2.0 child document driver]
%<samplemain>\ProvidesFile{cdocsamp.tex}[2018/12/30 v2.0 sample for childdoc]
%<*driver>
%\ProvidesFile{childdoc.drv}[2018/12/30 v2.0 childdoc reference manual file]
\PassOptionsToClass{10pt,a4paper}{article}
\documentclass{ltxdoc}

\usepackage[margin=35mm]{geometry}
\usepackage{hyperref}
\usepackage{hyperxmp}
\usepackage[usenames]{color}

\hypersetup{colorlinks=true}
\hypersetup{pdfstartview=FitH}
\hypersetup{pdfpagemode=UseNone}
\hypersetup{pdfsource={}}
\hypersetup{pdflang={en-UK}}
\hypersetup{pdfcopyright={Copyright 2017-2018 Niklas Beisert.
  This work may be distributed and/or modified under the
  conditions of the LaTeX Project Public License, either version 1.3
  of this license or (at your option) any later version.}}
\hypersetup{pdflicenseurl={http://www.latex-project.org/lppl.txt}}
\hypersetup{pdfcontactaddress={ETH Zurich, ITP, HIT K,
  Wolfgang-Pauli-Strasse 27}}
\hypersetup{pdfcontactpostcode={8093}}
\hypersetup{pdfcontactcity={Zurich}}
\hypersetup{pdfcontactcountry={Switzerland}}
\hypersetup{pdfcontactemail={nbeisert@itp.phys.ethz.ch}}
\hypersetup{pdfcontacturl={http://people.phys.ethz.ch/\xmptilde nbeisert/}}

\newcommand{\secref}[1]{\hyperref[#1]{section \ref*{#1}}}

\parskip1ex
\parindent0pt
\let\olditemize\itemize
\def\itemize{\olditemize\parskip0pt}

\begin{document}

\title{The \textsf{childdoc} Package}
\hypersetup{pdftitle={The childdoc Package}}
\author{Niklas Beisert\\[2ex]
  Institut f\"ur Theoretische Physik\\
  Eidgen\"ossische Technische Hochschule Z\"urich\\
  Wolfgang-Pauli-Strasse 27, 8093 Z\"urich, Switzerland\\[1ex]
  \href{mailto:nbeisert@itp.phys.ethz.ch}
  {\texttt{nbeisert@itp.phys.ethz.ch}}}
\hypersetup{pdfauthor={Niklas Beisert}}
\hypersetup{pdfsubject={Manual for the LaTeX2e Package childdoc}}
\date{30 December 2018, \textsf{v2.0}}
\maketitle

\begin{abstract}\noindent
\textsf{childdoc} is a \LaTeXe{} package
that enables the direct compilation
of document sections included by |\include|
to individual files.
\end{abstract}

\begingroup
\parskip0ex
\tableofcontents
\endgroup

%%%%%%%%%%%%%%%%%%%%%%%%%%%%%%%%%%%%%%%%%%%%%%%%%%%%%%%%%%%%%%%%%%%%%%%%%%%%%%%%
%%%%%%%%%%%%%%%%%%%%%%%%%%%%%%%%%%%%%%%%%%%%%%%%%%%%%%%%%%%%%%%%%%%%%%%%%%%%%%%%
\section{Introduction}

\LaTeX{} provides a mechanism to structure a large document (such as a book)
into a main file and several child files (containing the chapters)
using the |\include| command.
This mechanism is beneficial for documents
which span hundreds of pages in order to
make the source file(s) more manageable.
Moreover, compilation can be restricted to
selected child files by means of the |\includeonly| command.
The latter feature can be used to reduce the compilation time while editing
(this was significantly more useful in the earlier days of \LaTeX{})
or to generate a smaller document which is easier to navigate.
Another application of |\includeonly| is to generate
documents consisting of selected parts of the complete document.

However, there are a few drawbacks of the plain |\include| mechanism:
\begin{itemize}
\item
The child files cannot be compiled on their own,
they can only be compiled via the main file.
A naive editing environment
(such as a text editor with an option
to have the current file processed by \LaTeX)
may require one to switch to the main file before compiling;
attempting to compile the child file produces errors.
\item
The main file must be modified (each time)
to adjust the |\includeonly| command
to the present needs. This easily leaves the main file in a messy state.
\item
The generated document will always carry the filename
of the main document. This is inconvenient if
several child files are to be compiled and
to be kept for distribution.
\end{itemize}

The present package provides a simple interface
to make child files individually compilable by \LaTeX{}.
Compiling a child file then has the same effect as compiling
the main file with an |\includeonly| command
to select the appropriate child.
Moreover the generated document will carry the name of the child
rather than the main file.
This resolves all three above issues.

This feature is meant to make the editing of books,
thesis documents and lecture notes somewhat more convenient.
However, the package can also be used efficiently for
composing a series of documents (such as exercise sheets)
which are typically distributed individually.
It then assists the author in generating the individual documents
(potentially in different versions)
as well as a document containing the collected series.
Another application is in developing style files
or other kinds of included material
where compilation of the style file could redirect
to a sample or test file.

%%%%%%%%%%%%%%%%%%%%%%%%%%%%%%%%%%%%%%%%%%%%%%%%%%%%%%%%%%%%%%%%%%%%%%%%%%%%%%%%
%%%%%%%%%%%%%%%%%%%%%%%%%%%%%%%%%%%%%%%%%%%%%%%%%%%%%%%%%%%%%%%%%%%%%%%%%%%%%%%%
\section{Usage}

First of all, the package \textsf{childdoc} is \emph{not} a standard
\LaTeXe{} |.sty| style file! Therefore it needs to be invoked in
a non-standard way.

%%%%%%%%%%%%%%%%%%%%%%%%%%%%%%%%%%%%%%%%%%%%%%%%%%%%%%%%%%%%%%%%%%%%%%%%%%%%%%%%
\subsection{Included Files}
\label{sec:include}

%%%%%%%%%%%%%%%%%%%%%%%%%%%%%%%%%%%%%%%%
\DescribeMacro{\childdocmain}
To use the package, add the commands
\begin{center}
\begin{tabular}{l}
|\input{childdoc.def}|\\
|\childdocmain{}|\\
\end{tabular}
\end{center}
at the very top of the main \LaTeX{} file,
in particular \emph{before} the |\documentclass| statement!
The argument of |\childdocmain| should be left empty
(but it must be present).

%%%%%%%%%%%%%%%%%%%%%%%%%%%%%%%%%%%%%%%%
\DescribeMacro{\childdocof}
Furthermore, add the commands
\begin{center}
\begin{tabular}{l}
|\input{childdoc.def}|\\
|\childdocof{|\textit{main}|}|\\
\end{tabular}
\end{center}
at the top of every child file \textit{child}
which is included by |\include{|\textit{child}|}|
from within the main file
(or at least for those files to be compiled individually).
The argument \textit{main} must be the filename of the main file.

There are a couple of
considerations in setting up the main and child documents:

%%%%%%%%%%%%%%%%%%%%%%%%%%%%%%%%%%%%%%%%
\paragraph{Restrictions.}

Please note the following restrictions:
\begin{itemize}
\item
|\childdocmain| must be called with one argument \textit{main}
to ensure compatibility with earlier version of the package.
It must either be empty (|\childdocmain{}|)
or precisely match the filename of the main file in which it is specified.
See \secref{sec:detection} for further information.
\item
The filename \textit{main} must be specified without the |.tex| extension.
\item
The filename \textit{main} is case sensitive
(even in case-insensitive file systems)
due to internal string comparison.
\item
The argument \textit{main} should be fully expanded, it cannot be a macro.
\item
Subdirectories and special characters should be avoided in filenames.
\item
The command |\childdocmain{|\textit{main}|}| must be followed by a whitespace.
It should not be followed immediately by another command
or by a comment mark `|%|'.
This is because the \TeX{} parser reads the token immediately following
the argument of |\childdocmain| and puts it
at the beginning of every child section;
however, a white\-space is ignored.
\end{itemize}

%%%%%%%%%%%%%%%%%%%%%%%%%%%%%%%%%%%%%%%%
\paragraph{Content of Main File.}

It is advisable to place all content in the child files included by |\include|.
Any output contained in the main file will appear in all child documents
unless suppressed manually;
it cannot be suppressed automatically by the |\includeonly| directive
and thus should normally be avoided.
A method to include some content in the main file
by means of conditional processing is described in \secref{sec:conditional}.

%%%%%%%%%%%%%%%%%%%%%%%%%%%%%%%%%%%%%%%%
\paragraph{Page Numbering.}

When only a part of the document is compiled,
the appropriate numbering of pages
(as well as other status parameters)
is determined from the |.aux| files.
The latter contain information from previous passes.
However this information needs to propagate through
all intermediate child documents.
Therefore the page numbering in child documents may well
be inconsistent until the complete document is compiled at least once.

A useful (if unconventional) way to always ensure a consistent
page numbering is to restart the numbering in each child document
and denote the pages by `\textit{child}|.|\textit{page}'
where \textit{child} represents the chapter/section number of the child file.
This can be achieved by the command
|\numberwithin{page}{|\textit{child}|}|
of the \textsf{amsmath} package
where \textit{child} can be |chapter| or |section|
depending on the chosen structuring.
Alternatively, one can modify the macro |\thepage| appropriately
and reset the counter |page| at the start of each child file.

%%%%%%%%%%%%%%%%%%%%%%%%%%%%%%%%%%%%%%%%%%%%%%%%%%%%%%%%%%%%%%%%%%%%%%%%%%%%%%%%
\subsection{Conditional Processing}
\label{sec:conditional}

The package provides a mechanism to compile different versions
of a document. To customise the versions further some conditional processing
can come in handy to distinguish which version is being compiled.
The package provides two macros to describe the compilation context:

%%%%%%%%%%%%%%%%%%%%%%%%%%%%%%%%%%%%%%%%
\DescribeMacro{\ifchilddoc}
The conditional |\ifchilddoc| distinguishes between the compilation of
child documents and the main document:
%
\begin{center}
|\ifchilddoc |\textit{child-code}| |[|\||else |\textit{main-code}]| \||fi|
\end{center}

%%%%%%%%%%%%%%%%%%%%%%%%%%%%%%%%%%%%%%%%
\DescribeMacro{\childdocname}
\DescribeMacro{\childdocjob}
The macro |\childdocname| contains the filename (without extension)
of the main or child file being processed.
Note that |\childdocjob| will always contain the name of the main file.

%%%%%%%%%%%%%%%%%%%%%%%%%%%%%%%%%%%%%%%%
\paragraph{Title Page.}

Conditional processing can be used to include a title or banner page
in the main document when proper precautions are taken.
Importantly, the code in the main file should ensure that the page counter
(as well as other status parameters which are stored in the |.aux| files)
takes the same value after the conditional processing.
Otherwise the page numbers may take divergent values
depending on which part is compiled.

For example, a title page could be declared by:
%
\begin{center}
\begin{tabular}{l}
|\ifchilddoc\||else|\\
|\addtocounter{page}{-1}|\\
\textit{code for title page}\\
|\newpage|\\
|\||fi|
\end{tabular}
\end{center}
%
A banner page for the child documents can be generated by:
%
\begin{center}
\begin{tabular}{l}
|\ifchilddoc|\\
|\addtocounter{page}{-1}|\\
\textit{code for banner page}\\
|\newpage|\\
|\||fi|
\end{tabular}
\end{center}
%
Here one could write a message such as:
\begin{center}
|This is the part \childdocname{} of \childdocjob{}.|
\end{center}

%%%%%%%%%%%%%%%%%%%%%%%%%%%%%%%%%%%%%%%%%%%%%%%%%%%%%%%%%%%%%%%%%%%%%%%%%%%%%%%%
\subsection{Flags}
\label{sec:flags}

The package makes it easy to generate different versions
of the main or child documents.
To this end compilation flags can be defined
and assigned different default values.
They will be particularly useful in conjunction
with the forwarding mechanism described in \secref{sec:forward}.

For example, it may be useful to have a flag |\version|
which can be set to |draft| or |final|.
The document source will contain some conditional code
depending on the value of |\version|.
Suppose further, the flag should default to |final| for the main file
and to |draft| for child files
which is a natural assignment for editing the document.
This is achieved by placing the following code
in the preamble of the main document
(below the |\childdocmain| directive):
%
\begin{center}
\begin{tabular}{l}
|\ifchilddoc|\\
|\providecommand{\version}{draft}|\\
|\||else|\\
|\providecommand{\version}{final}|\\
|\||fi|
\end{tabular}
\end{center}
%
The definition by |\providecommand| makes sure
that previous definitions are not overwritten.
Further statements |\providecommand{\version}{...}|
can thus be added before the above code to override it.

For the main file, one might add a line
(between |\childdocmain| and the above block)
%
\begin{center}
|%\ifchilddoc\||else\providecommand{\version}{draft}\||fi|
\end{center}
%
which can be uncommented to produce a draft version.
Likewise one can add a line to the very top of a child file
(above the |\childdocof{|\textit{main}|}| directive)
%
\begin{center}
|%\providecommand{\version}{final}|
\end{center}
%
which can be uncommented to produce the final version of this child document.

%%%%%%%%%%%%%%%%%%%%%%%%%%%%%%%%%%%%%%%%%%%%%%%%%%%%%%%%%%%%%%%%%%%%%%%%%%%%%%%%
\subsection{Forwarding}
\label{sec:forward}

Different versions of the main or child documents
using compilation flags as described in \secref{sec:flags}
can be (permanently) stored in different files
for convenient compilation, viewing and distribution.
To this end, the package defines a command
to pass on compilation to a different file:

%%%%%%%%%%%%%%%%%%%%%%%%%%%%%%%%%%%%%%%%
\DescribeMacro{\childdocforward}
The command |\childdocforward| redirects processing to
another source file:
%
\begin{center}
\begin{tabular}{l}
|\input{childdoc.def}|\\
|\childdocforward[|\textit{main}|]{|\textit{dest}|}|\\
\end{tabular}
\end{center}
%
The argument \textit{dest} is the destination file
(without extension).
It should be the main file or one of the child files.
Note that further \textsf{childdoc} directives
such as |\childdocof| and |\childdocforward|
in the indicated file will be processed in this form.
The optional argument \textit{main}
passes on directly to the main file \textit{main}
while pretending to compile the child \textit{dest}.
This form behaves as if \textit{dest}
issues |\childdocof{|\textit{main}|}| right away,
and no further \textsf{childdoc} directives will be processed.

%%%%%%%%%%%%%%%%%%%%%%%%%%%%%%%%%%%%%%%%
\DescribeMacro{\...prefix}
In the alternative form |\childdocforwardprefix|,
%
\begin{center}
\begin{tabular}{l}
|\input{childdoc.def}|\\
|\childdocforwardprefix[|\textit{main}|]{|\textit{prefix}|}{|\textit{dest}|}|
\end{tabular}
\end{center}
%
the destination file is determined by a pattern
depending on the current file:
To make this work, the current file must be called
`{\textit{prefix}\hspace{0.2em}\textit{suffix}}'
with \textit{prefix} matching precisely the argument.
Processing is then passed on to the file
`{\textit{dest}\hspace{0.2em}\textit{suffix}}'.
Surely, the same effect is achieved by
directly specifying the
argument `{\textit{dest}\hspace{0.2em}\textit{suffix}}'
in the first form.
However, that requires to set up a different file
for each child. With the alternative form of the command
all these files can have exactly the same content
which simplifies setting them up and maintaining them.

For example, the following file |draft.tex|
with a compilation flag |\version| as described in \secref{sec:flags}
compiles the main document as a draft:
%
\begin{center}
\begin{tabular}{l}
|\def\version{draft}|\\
|\input{childdoc.def}|\\
|\childdocforward{|\textit{main}|}|
\end{tabular}
\end{center}
%
Likewise, the following files |final|\textit{nn}|.tex|
compile the final version of the child document
|child|\textit{nn}|.tex|:
%
\begin{center}
\begin{tabular}{l}
|\def\version{final}|\\
|\input{childdoc.def}|\\
|\childdocforwardprefix{final}{child}|
\end{tabular}
\end{center}
%

Note that when several versions of a main file and/or of each child file
are to be generated, it may be convenient to set up a |Makefile| or
shell script to automatise the process.

%%%%%%%%%%%%%%%%%%%%%%%%%%%%%%%%%%%%%%%%%%%%%%%%%%%%%%%%%%%%%%%%%%%%%%%%%%%%%%%%
\subsection{Command Line Processing}
\label{sec:commandline}

The effect of redirection files can also be achieved by invoking
the \LaTeX{} compiler with a more elaborate command line.
Most conveniently this should be done as part
of a shell script or a |Makefile|.

When using \textsf{childdoc} in the main file, the following
command lines effectively perform a redirection
(note that depending on the shell being used,
backslashes may have to be doubled: `|\|' $\to$ `|\\|'):
%
\begin{center}
|... -jobname "|\textit{target}|" |\\|"|[\textit{flags}]%
|\input{childdoc.def}\childdocforward[|\textit{main}|]{|\textit{dest}|}"|
\end{center}
%
Here \textit{target} is the name of the output file,
\textit{main} is the name of the main file
and \textit{dest} is the name of the main or child file to be processed
(all filenames without extensions).
The optional argument \textit{main} can be omitted
if \textit{main} matches \textit{dest}.
Optionally, compilation \textit{flags} can be defined via |\def| commands.
This command line makes the \TeX{} engine believe
it is compiling the file \textit{target}
whose content is specified as the latter parameter.
The provided code then forwards the processing to
\textit{main} or \textit{dest} as described in \secref{sec:forward}.

%%%%%%%%%%%%%%%%%%%%%%%%%%%%%%%%%%%%%%%%%%%%%%%%%%%%%%%%%%%%%%%%%%%%%%%%%%%%%%%%
\subsection{Include by Input}
\label{sec:input}

Including child documents by |\include| has some restrictions by design.
Most notably, the content of a child document always occupies
its own set of pages; pages cannot be shared between child documents.
Usually, this behaviour makes perfect sense
because each child document contain an essential part of the document.
However, in some situations it may be desirable to compose
a document from a collection of parts
without having mandatory page breaks between then.
For this case, the package
provides a mechanism to include parts
by |\input| which can also be processed individually.
However, by construction this mechanism
requires manual handling of the content to be output.

%%%%%%%%%%%%%%%%%%%%%%%%%%%%%%%%%%%%%%%%
\DescribeMacro{\ifchilddocmanual}
The main file should be prepared as usual, see \secref{sec:include}.
However, the document body must make a distinction
between processing of an individual part and of the main document, e.g.:
%
\begin{center}
\begin{tabular}{l}
|\ifchilddocmanual|\\
|\input{\childdocname}|\\
|\||else|\\
\textit{document body with }|\input{|\textit{part}|}|\\
|\||fi|
\end{tabular}
\end{center}
%
The conditional |\ifchilddocmanual| is true whenever
a part to be included by |\input| is being compiled,
and the name of the part is stored in |\childdocname|.

%%%%%%%%%%%%%%%%%%%%%%%%%%%%%%%%%%%%%%%%
\DescribeMacro{\childdocby}
Each part to be included by |\input| should start with:
%
\begin{center}
\begin{tabular}{l}
|\input{childdoc.def}|\\
|\childdocby{|\textit{main}|}|\\
\end{tabular}
\end{center}
%
The directive |\childdocby| is similar to |\childdocof|
described in \secref{sec:include},
but the subsequent selection of content must be done manually.
To that end, both |\ifchilddoc| and |\ifchilddocmanual|
will be true upon processing of a part,
and the name of the part is stored in |\childdocname|.
Note that |\jobname| will be set to the filename of the current part
so that each part receives an individual |.aux| file
that does not interfere with the |.aux| file(s) of the main document.
This behaviour can be altered by the alternative form
|\childdocby[*]{|\textit{main}|}| (with a non-empty optional argument)
which uses the |.aux| file of the main document
by setting |\jobname| to \textit{main}.

%%%%%%%%%%%%%%%%%%%%%%%%%%%%%%%%%%%%%%%%%%%%%%%%%%%%%%%%%%%%%%%%%%%%%%%%%%%%%%%%
\subsection{Driver Development}
\label{sec:driver}

The \textsf{childdoc} mechanism can also be use for the development
of definition files such as \LaTeX{} styles or classes.
This case differs from the above setup with multiple parts
included by |\include| in that no |\includeonly| should be invoked.
This can be achieved by starting the include file
(before |\ProvidesPackage|) with:
%
\begin{center}
\begin{tabular}{l}
|\input{childdoc.def}|\\
|\childdocforward{|\textit{main}|}|\\
\end{tabular}
\end{center}
%
or alternatively with:
%
\begin{center}
\begin{tabular}{l}
|\input{childdoc.def}|\\
|\childdocby{|\textit{main}|}|\\
\end{tabular}
\end{center}
%
Both forms have slightly different effects as described above.
The main file is prepared as usual, see \secref{sec:include}.

%%%%%%%%%%%%%%%%%%%%%%%%%%%%%%%%%%%%%%%%%%%%%%%%%%%%%%%%%%%%%%%%%%%%%%%%%%%%%%%%
\subsection{Legacy Detection}
\label{sec:detection}

The directive |\childdocmain| in the main file can detect
whether the complete document or merely a child is to be compiled
even without using the directive |\childdocof|.
This method is deprecated because it is less robust
and there is no compelling reason to use it;
it is merely provided for backward compatibility
and it may be removed in future versions.

If the detection mechanism is to be used,
it is mandatory to correctly specify
the filename of the main file as the argument of |\childdocmain|:
%
\begin{center}
\begin{tabular}{l}
|\input{childdoc.def}|\\
|\childdocmain{|\textit{main}|}|\\
\end{tabular}
\end{center}
%
If |\jobname| does not match the argument \textit{main} of |\childdocmain|,
it is assumed that |\jobname| points to the child file to be compiled.
When using |\childdocmain| with the main file specified as argument,
it suffices to start a child file
with just |\input{|\textit{main}|}|
without loading of the package and using |\childdocof|.
If instead all processing is done
with the appropriate \textsf{childdoc} directives,
the argument of \textit{main} of |\childdocmain| can be empty.

An alternative version of the command line processing described
in \secref{sec:commandline} using the detection mechanism reads:
%
\begin{center}
|... -jobname "|\textit{target}|" "|[\textit{flags}]%
[|\def\jobname{|\textit{dest}|}|]|\input{|\textit{main}|}"|
\end{center}

%%%%%%%%%%%%%%%%%%%%%%%%%%%%%%%%%%%%%%%%%%%%%%%%%%%%%%%%%%%%%%%%%%%%%%%%%%%%%%%%
\subsection{Manual Code}
\label{sec:manual}

In case one cannot be certain whether the definitions file |childdoc.def|
is installed on the target \TeX{} distribution
and one prefers not to ship it,
it is conceivable to paste a few relevant commands into the sources.

To that end, drop all statements |\input{childdoc.def}|
and perform the replacements as outlined below.
Instead of |\childdocmain{|\textit{main}|}| add the following code
to the top of the main file:
%
\begin{center}
\begin{tabular}{l}
|\||ifdefined\childdocname\endinput\||fi\newif\ifchilddoc|\\
|\edef\childdocname{\scantokens\expandafter{\jobname\noexpand}}|\\
|\def\childdocmain{|\textit{main}|}\||ifx\childdocmain\childdocname\||else|\\
|\childdoctrue\includeonly{\childdocname}\let\jobname\childdocmain\||fi|\\
\end{tabular}
\end{center}
%
Instead of |\childdocof{|\textit{main}|}| just include the main file
at the top of each child file:
%
\begin{center}
|\input{|\textit{main}|}|
\end{center}
%
A simple redirection |\childdocforward{|\textit{dest}|}| is achieved by:
%
\begin{center}
|\def\jobname{|\textit{dest}|}\input{\jobname}|
\end{center}
%
The redirection with prefix
|\childdocforwardprefix[|\textit{prefix}|]{|\textit{dest}|}|
is accomplished by:
%
\begin{center}
\begin{tabular}{l}
|{\edef\jobname{\scantokens\expandafter{\jobname\noexpand}}|\\
|\def\redirectjob |\textit{prefix}|#1~~~{\gdef\jobname{|\textit{dest}|#1}}|\\
|\expandafter\redirectjob\jobname~~~}\input{\jobname}|
\end{tabular}
\end{center}

In an alternative approach,
child documents can be compiled by a specific command line
without additional code or specific definitions:
%
\begin{center}
|... -jobname "|\textit{target}|" "|[\textit{flags}]%
|\includeonly{|\textit{dest}|}\input{|\textit{main}|}"|
\end{center}
%

%%%%%%%%%%%%%%%%%%%%%%%%%%%%%%%%%%%%%%%%%%%%%%%%%%%%%%%%%%%%%%%%%%%%%%%%%%%%%%%%
%%%%%%%%%%%%%%%%%%%%%%%%%%%%%%%%%%%%%%%%%%%%%%%%%%%%%%%%%%%%%%%%%%%%%%%%%%%%%%%%
\section{Information}

%%%%%%%%%%%%%%%%%%%%%%%%%%%%%%%%%%%%%%%%%%%%%%%%%%%%%%%%%%%%%%%%%%%%%%%%%%%%%%%%
\subsection{Copyright}

Copyright \copyright{} 2017--2018 Niklas Beisert

This work may be distributed and/or modified under the
conditions of the \LaTeX{} Project Public License, either version 1.3
of this license or (at your option) any later version.
The latest version of this license is in
  \url{http://www.latex-project.org/lppl.txt}
and version 1.3 or later is part of all distributions of \LaTeX{}
version 2005/12/01 or later.

This work has the LPPL maintenance status `maintained'.

The Current Maintainer of this work is Niklas Beisert.

This work consists of the files |README.txt|, |childdoc.ins| and |childdoc.dtx|
as well as the derived files |childdoc.def|, |cdocsamp.tex|
with |cdocsch1.tex|, |cdocsch2.tex|, |cdocspt3.tex|, |cdocspt4.tex|,
|cdocsdrf.tex|, |cdocsfn1.tex|, |cdocsfn2.tex|
as well as |childdoc.pdf|.

%%%%%%%%%%%%%%%%%%%%%%%%%%%%%%%%%%%%%%%%%%%%%%%%%%%%%%%%%%%%%%%%%%%%%%%%%%%%%%%%
\subsection{Files and Installation}

The package consists of the files:
%
\begin{center}
\begin{tabular}{ll}
    |README.txt|   & readme file \\
    |childdoc.ins| & installation file \\
    |childdoc.dtx| & source file \\
    |childdoc.def| & definition file \\
    |cdocsamp.tex| & sample main file \\
    |cdocsch1.tex| & sample include file \\
    |cdocsch2.tex| & sample include file \\
    |cdocspt3.tex| & sample part file \\
    |cdocspt4.tex| & sample part file \\
    |cdocsdrf.tex| & sample redirection file \\
    |cdocsfn1.tex| & sample redirection file \\
    |cdocsfn2.tex| & sample redirection file \\
    |childdoc.pdf| & manual
\end{tabular}
\end{center}
%
The distribution consists of the files
|README.txt|, |childdoc.ins| and |childdoc.dtx|.
%
\begin{itemize}
\item
Run (pdf)\LaTeX{} on |childdoc.dtx|
to compile the manual |childdoc.pdf| (this file).
\item
Run \LaTeX{} on |childdoc.ins| to create the definitions file |childdoc.def|
and the sample |cdocsamp.tex| with include files
|cdocsch1.tex|, |cdocsch2.tex|, |cdocspt3.tex|, |cdocspt4.tex|,
|cdocsdrf.tex|, |cdocsfn1.tex|, |cdocsfn2.tex|.
Then copy the file |childdoc.def| to an appropriate directory of your \LaTeX{}
distribution, e.g.\ \textit{texmf-root}|/tex/latex/childdoc|.
\end{itemize}

%%%%%%%%%%%%%%%%%%%%%%%%%%%%%%%%%%%%%%%%%%%%%%%%%%%%%%%%%%%%%%%%%%%%%%%%%%%%%%%%
\subsection{Related CTAN Packages}

There are several other packages which offer a similar functionality:
%
\begin{itemize}
\item
The packages
\href{http://ctan.org/pkg/docmute}{\textsf{docmute}},
\href{http://ctan.org/pkg/includex}{\textsf{includex}} and
\href{http://ctan.org/pkg/standalone}{\textsf{standalone}}
provide commands to include only the document body of
a child file thus allowing both files to be compiled individually.
\item
The packages \href{http://ctan.org/pkg/subdocs}{\textsf{subdocs}}
and \href{http://ctan.org/pkg/subfiles}{\textsf{subfiles}}
provide structures in which the main and child documents can be
encapsulated and allowing them to be compiled individually.
The inclusion mechanism is different from the conventional |\include|.
\item
The package \href{http://ctan.org/pkg/combine}{\textsf{combine}}
is an elaborate solution to combine several documents into one.
\end{itemize}
%
See also the CTAN topic \href{http://ctan.org/topic/subdocs}{\textsf{subdocs}}
for further related packages.
The present package differs from the above solutions in that
a document structure constructed with the conventional |\include| mechanism
just needs two extra commands at the top of every file
such that all constituent files can be compiled individually.

%%%%%%%%%%%%%%%%%%%%%%%%%%%%%%%%%%%%%%%%%%%%%%%%%%%%%%%%%%%%%%%%%%%%%%%%%%%%%%%%
%\subsection{Feature Suggestions}
%
%The following is a list of features which may be useful for future
%versions of this package:
%%
%\begin{itemize}
%\item
%\ldots
%\end{itemize}

%%%%%%%%%%%%%%%%%%%%%%%%%%%%%%%%%%%%%%%%%%%%%%%%%%%%%%%%%%%%%%%%%%%%%%%%%%%%%%%%
\subsection{Revision History}

%%%%%%%%%%%%%%%%%%%%%%%%%%%%%%%%%%%%%%%%
\paragraph{v2.0:} 2018/12/30

\begin{itemize}
\item
immediate forward processing
\item
added |\childdocby| mechanism
\item
manual restructured
\end{itemize}

%%%%%%%%%%%%%%%%%%%%%%%%%%%%%%%%%%%%%%%%
\paragraph{v1.6:} 2018/01/17

\begin{itemize}
\item
application for development of include files
\item
corrections to manual
\end{itemize}

%%%%%%%%%%%%%%%%%%%%%%%%%%%%%%%%%%%%%%%%
\paragraph{v1.5:} 2017/05/21

\begin{itemize}
\item
more complete structuring introduced
\item
|\childdocof| introduced
\item
|\childdoc| renamed to |\childdocmain|
\item
|\childredirect| renamed to |\childdocforward| and |\childdocforwardprefix|
and functionality expanded
\end{itemize}

%%%%%%%%%%%%%%%%%%%%%%%%%%%%%%%%%%%%%%%%
\paragraph{v1.0:} 2017/04/27

\begin{itemize}
\item
manual and install package
\item
first version published on CTAN
\end{itemize}

%%%%%%%%%%%%%%%%%%%%%%%%%%%%%%%%%%%%%%%%
\paragraph{v0.6:} 2017/04/26

\begin{itemize}
\item
redirection mechanism added
\end{itemize}

%%%%%%%%%%%%%%%%%%%%%%%%%%%%%%%%%%%%%%%%
\paragraph{v0.5:} 2017/04/26

\begin{itemize}
\item
functionality in definition file
\end{itemize}


%%%%%%%%%%%%%%%%%%%%%%%%%%%%%%%%%%%%%%%%%%%%%%%%%%%%%%%%%%%%%%%%%%%%%%%%%%%%%%%%
%%%%%%%%%%%%%%%%%%%%%%%%%%%%%%%%%%%%%%%%%%%%%%%%%%%%%%%%%%%%%%%%%%%%%%%%%%%%%%%%
%%%%%%%%%%%%%%%%%%%%%%%%%%%%%%%%%%%%%%%%%%%%%%%%%%%%%%%%%%%%%%%%%%%%%%%%%%%%%%%%
\appendix

\settowidth\MacroIndent{\rmfamily\scriptsize 000\ }

 \DocInput{childdoc.dtx}

\end{document}
%</driver>
% \fi
%
% %%%%%%%%%%%%%%%%%%%%%%%%%%%%%%%%%%%%%%%%%%%%%%%%%%%%%%%%%%%%%%%%%%%%%%%%%%%%%%
% %%%%%%%%%%%%%%%%%%%%%%%%%%%%%%%%%%%%%%%%%%%%%%%%%%%%%%%%%%%%%%%%%%%%%%%%%%%%%%
% \section{Sample}
%\iffalse
%<*samplemain>
%\fi
%
% The following presents a sample document
% with two chapters, two parts, a title page,
% a compile flag as well as three forwarding files to set the flag.
% It consists of eight |.tex| files:
% \begin{center}
% \begin{tabular}{ll}
% |cdocsamp.tex|&main file\\
% |cdocsch1.tex|&include file for chapter 1\\
% |cdocsch2.tex|&include file for chapter 2\\
% |cdocspt3.tex|&include file for part 3\\
% |cdocspt4.tex|&include file for part 4\\
% |cdocsdrf.tex|&forwarding file for main file in draft mode\\
% |cdocsfi1.tex|&forwarding file for final version of chapter 1\\
% |cdocsfi2.tex|&forwarding file for final version of chapter 2\\
% \end{tabular}
% \end{center}
% Each of the eight files can be compiled directly by the \LaTeX{} compiler.
%
% %%%%%%%%%%%%%%%%%%%%%%%%%%%%%%%%%%%%%%
% \paragraph{Main File.}
%
% The main file is called |cdocsamp.tex|.
%
% Load the \textsf{childdoc} definitions and
% declare the filename for the main document:
%    \begin{macrocode}
\input{childdoc.def}
\childdocmain{}
%    \end{macrocode}

% Optional override for |\version| flag:
%    \begin{macrocode}
%%\ifchilddoc\else\providecommand{\version}{draft}\fi
%    \end{macrocode}

% Define the default values for the |\version| flag
% (|final| for the main file and |draft| for childs):
%    \begin{macrocode}
\ifchilddoc
\providecommand{\version}{draft}
\else
\providecommand{\version}{final}
\fi
%    \end{macrocode}

% Load the standard document class:
%    \begin{macrocode}
\documentclass[12pt]{article}
%    \end{macrocode}

% Start the document body:
%    \begin{macrocode}
\begin{document}
%    \end{macrocode}

% Declare a title page.
% Print title, part of document being processed and version flag:
%    \begin{macrocode}
\addtocounter{page}{-1}
\begin{center}
{\LARGE\bfseries{}childdoc example\par}
\vspace{1cm}
\ifchilddoc
\ifchilddocmanual part\else chapter\fi:
`\childdocname' of `\childdocjob'\par
\else
main document: `\childdocjob'\par
\fi
version: \version\par
\end{center}
\newpage
%    \end{macrocode}

% Manually include selected file,
% otherwise process as usual:
%    \begin{macrocode}
\ifchilddocmanual
\section*{part `\childdocname'}
\input{\childdocname}
\else
%    \end{macrocode}

% Include the two chapters:
%    \begin{macrocode}
\include{cdocsch1}
\include{cdocsch2}
%    \end{macrocode}

% Include the two parts unless only chapters should be displayed:
%    \begin{macrocode}
\ifchilddoc\else
\section{part three}
\input{cdocspt3}
\section{part four}
\input{cdocspt4}
\fi
%    \end{macrocode}

% Process as usual until here:
%    \begin{macrocode}
\fi
%    \end{macrocode}

% End of document body:
%    \begin{macrocode}
\end{document}
%    \end{macrocode}
%\iffalse
%</samplemain>
%\fi
%
% %%%%%%%%%%%%%%%%%%%%%%%%%%%%%%%%%%%%%%
% \paragraph{Chapter Include Files.}
%
% The include files are called |cdocsch1.tex| and |cdocsch2.tex|.
%
%\iffalse
%<*samplechap1|samplechap2>
%\fi

% Optional override for |\version| flag:
%    \begin{macrocode}
%%\providecommand{\version}{final}
%    \end{macrocode}

% Include the main document:
%    \begin{macrocode}
\input{childdoc.def}
\childdocof{cdocsamp}
%    \end{macrocode}

%\iffalse
%</samplechap1|samplechap2>
%\fi
%
%\iffalse
%<*samplechap1>
%\fi
% Some text for chapter 1:
%    \begin{macrocode}
\section{one}
some text in chapter one
%    \end{macrocode}

%\iffalse
%</samplechap1>
%\fi
% Some text for chapter 2:
%\iffalse
%<*samplechap2>
%\fi
%    \begin{macrocode}
\section{two}
more text in chapter two
%    \end{macrocode}

%\iffalse
%</samplechap2>
%\fi
%
% %%%%%%%%%%%%%%%%%%%%%%%%%%%%%%%%%%%%%%
% \paragraph{Part Include Files.}
%
% The include files are called |cdocspt3.tex| and |cdocspt4.tex|.
%
%\iffalse
%<*samplepart3|samplepart4>
%\fi

% Optional override for |\version| flag:
%    \begin{macrocode}
%%\providecommand{\version}{final}
%    \end{macrocode}

% Include the main document:
%    \begin{macrocode}
\input{childdoc.def}
\childdocby{cdocsamp}
%    \end{macrocode}

%\iffalse
%</samplepart3|samplepart4>
%\fi
%
%\iffalse
%<*samplepart3>
%\fi
% Some text for part 3:
%    \begin{macrocode}
some text in part three
%    \end{macrocode}

%\iffalse
%</samplepart3>
%\fi
% Some text for part 4:
%\iffalse
%<*samplepart4>
%\fi
%    \begin{macrocode}
more text in part four
%    \end{macrocode}

%\iffalse
%</samplepart4>
%\fi
%
% %%%%%%%%%%%%%%%%%%%%%%%%%%%%%%%%%%%%%%
% \paragraph{Forwarding for a Complete Draft.}
%
% The following forwarding file |cdocsdrf.tex|
% compiles the main document in draft mode:
%\iffalse
%<*sampledraft>
%\fi
%    \begin{macrocode}
\def\version{draft}
\input{childdoc.def}
\childdocforward{cdocsamp}
%    \end{macrocode}

%\iffalse
%</sampledraft>
%\fi
%
% %%%%%%%%%%%%%%%%%%%%%%%%%%%%%%%%%%%%%%
% \paragraph{Forwarding for Final Version of the Chapters.}
%
% The following forwarding files |cdocsfn1.tex| and |cdocsfn2.tex|
% (with identical content)
% compile the final versions of the child documents
% |cdocsch1.tex| and |cdocsch2.tex|, respectively:
%\iffalse
%<*samplefinal>
%\fi
%    \begin{macrocode}
\def\version{final}
\input{childdoc.def}
\childdocforwardprefix[cdocsamp]{cdocsfn}{cdocsch}
%    \end{macrocode}

%\iffalse
%</samplefinal>
%\fi
%
% %%%%%%%%%%%%%%%%%%%%%%%%%%%%%%%%%%%%%%
% \paragraph{Command Line Processing.}
%
% The following three command lines generate the output files
% |cdocscld|, |cdocscl1| and |cdocscl2|
% which should be identical to
% |cdocsdrf|, |cdocsch1| and |cdocsfn2|, respectively:
% \begin{center}
% \begin{tabular}{l}
% |latex -jobname cdocscld \|\\
% |  "\def\version{draft}\input{childdoc.def}\childdocforward{cdocsamp}"|\\
% |latex -jobname cdocscl1 \|\\
% |  "\input{childdoc.def}\childdocforward[cdocsamp]{cdocsch1}"|\\
% |latex -jobname cdocscl2 \|\\
% |  "\def\version{final}\input{childdoc.def}\childdocforward{cdocsch2}"|
% \end{tabular}
% \end{center}
% Note that the trailing backslash on each first line
% merely continues the input to the second line
% (for convenient cut ant paste).
% Furthermore, the command |latex| can be replaced by any
% of its alternative versions such as |pdflatex|.
%
% %%%%%%%%%%%%%%%%%%%%%%%%%%%%%%%%%%%%%%%%%%%%%%%%%%%%%%%%%%%%%%%%%%%%%%%%%%%%%%
% %%%%%%%%%%%%%%%%%%%%%%%%%%%%%%%%%%%%%%%%%%%%%%%%%%%%%%%%%%%%%%%%%%%%%%%%%%%%%%
% \section{Implementation}
%\iffalse
%<*package>
%\fi
%
% This section describes the definitions file |childdoc.def|.

% The definitions cannot be loaded using |\usepackage| or |\RequirePackage|
% which has a mechanism to prevent loading a style file more than once.
% When loading the definitions by means of |\input|
% multiple instances have to be prevented manually:
%\iffalse
%This code needs to be before the `\ProvidesFile' directive
%which is defined at the beginning of this file.
%Therefore it is also placed there and commented out here.
%</package>
%<*discard>
%\fi
%    \begin{macrocode}
\ifdefined\childdocmain\endinput\fi
%    \end{macrocode}
%\iffalse
%</discard>
%<*package>
%\fi
%
% \macro{\ifchilddoc}
% \macro{\ifchilddocmanual}
% The conditional |\ifchilddoc| tells whether a
% child (true) or main (false) document is being compiled.
% The conditional |\ifchilddocmanual| tells whether
% the |\includeonly| mechanism is used (false) or
% the selection of child files must be performed manually (true).
% The definitions initialise to false:
%    \begin{macrocode}
\newif\ifchilddoc
\newif\ifchilddocmanual
%    \end{macrocode}

% \macro{\childdocname}
% \macro{\childdocjob}
% The macro |\childdocname| stores the name of the main document
% to be compiled. The macro |\childdocjob| stores the name of
% the document on which the \LaTeX{} compiler was originally invoked.
% The content of |\jobname| cannot be compared
% to filenames specified in the source due to different catcodes.
% The following code rescans |\jobname|, stores the result
% in |\childdocname| and saves a copy in |\childdocjob|:
%    \begin{macrocode}
\edef\childdocname{\scantokens\expandafter{\jobname\noexpand}}
\let\childdocjob\childdocname
%    \end{macrocode}

% \macro{\childdocdisable}
% The macro |\childdocdisable| prevents the main file
% from being processed more than once.
% At this stage, the main document command |\childdocmain|
% is assumed to be called once again where it should do nothing.
% Any subsequent call to it should prevent
% a secondary processing of the main document
% It overwrites the forwarding commands
% |\childdocof| and |\childdocforward|
% with empty macros to prevent further inclusions of the main document:
%    \begin{macrocode}
\newcommand{\childdocdisable}
{
  \renewcommand{\childdocmain}[1]{\renewcommand{\childdocmain}[1]{\endinput}}
  \renewcommand{\childdocof}[1]{}
  \renewcommand{\childdocby}[2][]{}
  \renewcommand{\childdocforward}[2][]{}
  \renewcommand{\childdocdisable}{}
}
%    \end{macrocode}

% \macro{\childdocmain}
% The macro |\childdocmain| is to be called at the top of the main file
% with nothing or the main filename (without extension) as argument.
% First, it breaks loops.
% If the argument is not empty and does not match |\childdocname|
% (which is set by the first inclusion of |childdoc.def|),
% |\ifchilddoc| is set to true, |\includeonly| is applied to the child file
% and |\jobname| is set to the main file
% (for proper handling of |.aux| files):
%    \begin{macrocode}
\newcommand{\childdocmain}[1]
{
  \childdocdisable\childdocmain{}
  \if?#1?\else
    \begingroup
      \def\childdoctmp{#1}
      \ifx\childdoctmp\childdocname
        \def\childdoctmp{}
      \else
        \def\childdoctmp
        {
          \childdoctrue
          \includeonly{\childdocname}
          \def\childdocjob{#1}
          \def\jobname{#1}
        }
      \fi
      \expandafter
    \endgroup
    \childdoctmp
  \fi
}
%    \end{macrocode}

% \macro{\childdocof}
% The command |\childdocof| redirects
% compilation to the main file |#1|.
%    \begin{macrocode}
\newcommand{\childdocof}[1]
{
  \childdocdisable
  \childdoctrue
  \includeonly{\childdocname}
  \def\jobname{#1}
  \def\childdocjob{#1}
  \input{#1}
}
%    \end{macrocode}

% \macro{\childdocby}
% The command |\childdocby| ....
%    \begin{macrocode}
\newcommand{\childdocby}[2][]
{
  \childdocdisable
  \childdoctrue
  \childdocmanualtrue
  \if?#1?\else
    \def\jobname{#2}
  \fi
  \def\childdocjob{#2}
  \input{#2}
  \endinput
}
%    \end{macrocode}

% \macro{\childdocforward}
% The command |\childdocforward| redirects
% compilation to the main file or
% (if the optional argument is given) a child file.
% Parameters are set as if the main file
% or a child file starting with |\childdocof| was compiled.
% Then compilation is handed over to the main file:
%    \begin{macrocode}
\newcommand{\childdocforward}[2][]
{
  \begingroup
    \if?#1?
      \def\childdoctmp
      {
        \def\childdocname{#2}
        \def\childdocjob{#2}
        \def\jobname{#2}
        \input{#2}
        \endinput
      }
    \else
      \def\childdoctmp
      {
        \childdocdisable
        \def\childdocname{#2}
        \childdoctrue
        \includeonly{#2}
        \def\childdocjob{#1}
        \def\jobname{#1}
        \input{#1}
        \endinput
      }
    \fi
    \expandafter
  \endgroup
  \childdoctmp
}
%    \end{macrocode}

% \macro{\childdocforwardprefix}
% The command |\childdocforwardprefix| redirects
% compilation to the main or a child file by means of a pattern.
% The prefix |#1| in the current filename is replaced by |#2|
% and the suffix of the current filename is kept
% (it is assumed that the filename does not contain the substring `|~~~|'
% which is used as a delimiter).
% Compilation is handed over to the new file by |\childdocforward|:
%    \begin{macrocode}
\newcommand{\childdocforwardprefix}[3][]
{
  \begingroup
    \def\childdocextract #2##1~~~{\def\childdoctmp{\childdocforward[#1]{#3##1}}}
    \expandafter\childdocextract\childdocname~~~
    \expandafter
  \endgroup
  \childdoctmp
}
%    \end{macrocode}

% \macro{\childdoc}
% The deprecated macro |\childdoc| is a legacy version of |\childdocmain|:
%    \begin{macrocode}
\newcommand{\childdoc}{\childdocmain}
%    \end{macrocode}

% \macro{\childdocredirect}
% The deprecated macro |\childdocredirect| is a legacy version
% of |\childdocforward| and |\childdocforwardprefix|:
%    \begin{macrocode}
\newcommand{\childdocredirect}[2][]
{
  \begingroup
    \if?#1?
      \def\childdoctmp{\childdocforward{#2}}
    \else
      \def\childdoctmp{\childdocforwardprefix{#1}{#2}}
    \fi
    \expandafter
  \endgroup
  \childdoctmp
}
%    \end{macrocode}

%\iffalse
%</package>
%\fi
%
\endinput
|\\
|\childdocof{|\textit{main}|}|\\
\end{tabular}
\end{center}
at the top of every child file \textit{child}
which is included by |\include{|\textit{child}|}|
from within the main file
(or at least for those files to be compiled individually).
The argument \textit{main} must be the filename of the main file.

There are a couple of
considerations in setting up the main and child documents:

%%%%%%%%%%%%%%%%%%%%%%%%%%%%%%%%%%%%%%%%
\paragraph{Restrictions.}

Please note the following restrictions:
\begin{itemize}
\item
|\childdocmain| must be called with one argument \textit{main}
to ensure compatibility with earlier version of the package.
It must either be empty (|\childdocmain{}|)
or precisely match the filename of the main file in which it is specified.
See \secref{sec:detection} for further information.
\item
The filename \textit{main} must be specified without the |.tex| extension.
\item
The filename \textit{main} is case sensitive
(even in case-insensitive file systems)
due to internal string comparison.
\item
The argument \textit{main} should be fully expanded, it cannot be a macro.
\item
Subdirectories and special characters should be avoided in filenames.
\item
The command |\childdocmain{|\textit{main}|}| must be followed by a whitespace.
It should not be followed immediately by another command
or by a comment mark `|%|'.
This is because the \TeX{} parser reads the token immediately following
the argument of |\childdocmain| and puts it
at the beginning of every child section;
however, a white\-space is ignored.
\end{itemize}

%%%%%%%%%%%%%%%%%%%%%%%%%%%%%%%%%%%%%%%%
\paragraph{Content of Main File.}

It is advisable to place all content in the child files included by |\include|.
Any output contained in the main file will appear in all child documents
unless suppressed manually;
it cannot be suppressed automatically by the |\includeonly| directive
and thus should normally be avoided.
A method to include some content in the main file
by means of conditional processing is described in \secref{sec:conditional}.

%%%%%%%%%%%%%%%%%%%%%%%%%%%%%%%%%%%%%%%%
\paragraph{Page Numbering.}

When only a part of the document is compiled,
the appropriate numbering of pages
(as well as other status parameters)
is determined from the |.aux| files.
The latter contain information from previous passes.
However this information needs to propagate through
all intermediate child documents.
Therefore the page numbering in child documents may well
be inconsistent until the complete document is compiled at least once.

A useful (if unconventional) way to always ensure a consistent
page numbering is to restart the numbering in each child document
and denote the pages by `\textit{child}|.|\textit{page}'
where \textit{child} represents the chapter/section number of the child file.
This can be achieved by the command
|\numberwithin{page}{|\textit{child}|}|
of the \textsf{amsmath} package
where \textit{child} can be |chapter| or |section|
depending on the chosen structuring.
Alternatively, one can modify the macro |\thepage| appropriately
and reset the counter |page| at the start of each child file.

%%%%%%%%%%%%%%%%%%%%%%%%%%%%%%%%%%%%%%%%%%%%%%%%%%%%%%%%%%%%%%%%%%%%%%%%%%%%%%%%
\subsection{Conditional Processing}
\label{sec:conditional}

The package provides a mechanism to compile different versions
of a document. To customise the versions further some conditional processing
can come in handy to distinguish which version is being compiled.
The package provides two macros to describe the compilation context:

%%%%%%%%%%%%%%%%%%%%%%%%%%%%%%%%%%%%%%%%
\DescribeMacro{\ifchilddoc}
The conditional |\ifchilddoc| distinguishes between the compilation of
child documents and the main document:
%
\begin{center}
|\ifchilddoc |\textit{child-code}| |[|\||else |\textit{main-code}]| \||fi|
\end{center}

%%%%%%%%%%%%%%%%%%%%%%%%%%%%%%%%%%%%%%%%
\DescribeMacro{\childdocname}
\DescribeMacro{\childdocjob}
The macro |\childdocname| contains the filename (without extension)
of the main or child file being processed.
Note that |\childdocjob| will always contain the name of the main file.

%%%%%%%%%%%%%%%%%%%%%%%%%%%%%%%%%%%%%%%%
\paragraph{Title Page.}

Conditional processing can be used to include a title or banner page
in the main document when proper precautions are taken.
Importantly, the code in the main file should ensure that the page counter
(as well as other status parameters which are stored in the |.aux| files)
takes the same value after the conditional processing.
Otherwise the page numbers may take divergent values
depending on which part is compiled.

For example, a title page could be declared by:
%
\begin{center}
\begin{tabular}{l}
|\ifchilddoc\||else|\\
|\addtocounter{page}{-1}|\\
\textit{code for title page}\\
|\newpage|\\
|\||fi|
\end{tabular}
\end{center}
%
A banner page for the child documents can be generated by:
%
\begin{center}
\begin{tabular}{l}
|\ifchilddoc|\\
|\addtocounter{page}{-1}|\\
\textit{code for banner page}\\
|\newpage|\\
|\||fi|
\end{tabular}
\end{center}
%
Here one could write a message such as:
\begin{center}
|This is the part \childdocname{} of \childdocjob{}.|
\end{center}

%%%%%%%%%%%%%%%%%%%%%%%%%%%%%%%%%%%%%%%%%%%%%%%%%%%%%%%%%%%%%%%%%%%%%%%%%%%%%%%%
\subsection{Flags}
\label{sec:flags}

The package makes it easy to generate different versions
of the main or child documents.
To this end compilation flags can be defined
and assigned different default values.
They will be particularly useful in conjunction
with the forwarding mechanism described in \secref{sec:forward}.

For example, it may be useful to have a flag |\version|
which can be set to |draft| or |final|.
The document source will contain some conditional code
depending on the value of |\version|.
Suppose further, the flag should default to |final| for the main file
and to |draft| for child files
which is a natural assignment for editing the document.
This is achieved by placing the following code
in the preamble of the main document
(below the |\childdocmain| directive):
%
\begin{center}
\begin{tabular}{l}
|\ifchilddoc|\\
|\providecommand{\version}{draft}|\\
|\||else|\\
|\providecommand{\version}{final}|\\
|\||fi|
\end{tabular}
\end{center}
%
The definition by |\providecommand| makes sure
that previous definitions are not overwritten.
Further statements |\providecommand{\version}{...}|
can thus be added before the above code to override it.

For the main file, one might add a line
(between |\childdocmain| and the above block)
%
\begin{center}
|%\ifchilddoc\||else\providecommand{\version}{draft}\||fi|
\end{center}
%
which can be uncommented to produce a draft version.
Likewise one can add a line to the very top of a child file
(above the |\childdocof{|\textit{main}|}| directive)
%
\begin{center}
|%\providecommand{\version}{final}|
\end{center}
%
which can be uncommented to produce the final version of this child document.

%%%%%%%%%%%%%%%%%%%%%%%%%%%%%%%%%%%%%%%%%%%%%%%%%%%%%%%%%%%%%%%%%%%%%%%%%%%%%%%%
\subsection{Forwarding}
\label{sec:forward}

Different versions of the main or child documents
using compilation flags as described in \secref{sec:flags}
can be (permanently) stored in different files
for convenient compilation, viewing and distribution.
To this end, the package defines a command
to pass on compilation to a different file:

%%%%%%%%%%%%%%%%%%%%%%%%%%%%%%%%%%%%%%%%
\DescribeMacro{\childdocforward}
The command |\childdocforward| redirects processing to
another source file:
%
\begin{center}
\begin{tabular}{l}
|% \iffalse
%
% childdoc.dtx Copyright (C) 2017-2018 Niklas Beisert
%
% This work may be distributed and/or modified under the
% conditions of the LaTeX Project Public License, either version 1.3
% of this license or (at your option) any later version.
% The latest version of this license is in
%   http://www.latex-project.org/lppl.txt
% and version 1.3 or later is part of all distributions of LaTeX
% version 2005/12/01 or later.
%
% This work has the LPPL maintenance status `maintained'.
%
% The Current Maintainer of this work is Niklas Beisert.
%
% This work consists of the files childdoc.dtx and childdoc.ins
% and the derived files childdoc.def and cdocsamp.tex with
% cdocsch1.tex, cdocsch2.tex, cdocsdrf.tex, cdocsfn1.tex, cdocsfn2.tex.
%
%<package>\ifdefined\childdocmain\endinput\fi
%<package>\ProvidesFile{childdoc.def}[2018/12/30 v2.0 child document driver]
%<samplemain>\ProvidesFile{cdocsamp.tex}[2018/12/30 v2.0 sample for childdoc]
%<*driver>
%\ProvidesFile{childdoc.drv}[2018/12/30 v2.0 childdoc reference manual file]
\PassOptionsToClass{10pt,a4paper}{article}
\documentclass{ltxdoc}

\usepackage[margin=35mm]{geometry}
\usepackage{hyperref}
\usepackage{hyperxmp}
\usepackage[usenames]{color}

\hypersetup{colorlinks=true}
\hypersetup{pdfstartview=FitH}
\hypersetup{pdfpagemode=UseNone}
\hypersetup{pdfsource={}}
\hypersetup{pdflang={en-UK}}
\hypersetup{pdfcopyright={Copyright 2017-2018 Niklas Beisert.
  This work may be distributed and/or modified under the
  conditions of the LaTeX Project Public License, either version 1.3
  of this license or (at your option) any later version.}}
\hypersetup{pdflicenseurl={http://www.latex-project.org/lppl.txt}}
\hypersetup{pdfcontactaddress={ETH Zurich, ITP, HIT K,
  Wolfgang-Pauli-Strasse 27}}
\hypersetup{pdfcontactpostcode={8093}}
\hypersetup{pdfcontactcity={Zurich}}
\hypersetup{pdfcontactcountry={Switzerland}}
\hypersetup{pdfcontactemail={nbeisert@itp.phys.ethz.ch}}
\hypersetup{pdfcontacturl={http://people.phys.ethz.ch/\xmptilde nbeisert/}}

\newcommand{\secref}[1]{\hyperref[#1]{section \ref*{#1}}}

\parskip1ex
\parindent0pt
\let\olditemize\itemize
\def\itemize{\olditemize\parskip0pt}

\begin{document}

\title{The \textsf{childdoc} Package}
\hypersetup{pdftitle={The childdoc Package}}
\author{Niklas Beisert\\[2ex]
  Institut f\"ur Theoretische Physik\\
  Eidgen\"ossische Technische Hochschule Z\"urich\\
  Wolfgang-Pauli-Strasse 27, 8093 Z\"urich, Switzerland\\[1ex]
  \href{mailto:nbeisert@itp.phys.ethz.ch}
  {\texttt{nbeisert@itp.phys.ethz.ch}}}
\hypersetup{pdfauthor={Niklas Beisert}}
\hypersetup{pdfsubject={Manual for the LaTeX2e Package childdoc}}
\date{30 December 2018, \textsf{v2.0}}
\maketitle

\begin{abstract}\noindent
\textsf{childdoc} is a \LaTeXe{} package
that enables the direct compilation
of document sections included by |\include|
to individual files.
\end{abstract}

\begingroup
\parskip0ex
\tableofcontents
\endgroup

%%%%%%%%%%%%%%%%%%%%%%%%%%%%%%%%%%%%%%%%%%%%%%%%%%%%%%%%%%%%%%%%%%%%%%%%%%%%%%%%
%%%%%%%%%%%%%%%%%%%%%%%%%%%%%%%%%%%%%%%%%%%%%%%%%%%%%%%%%%%%%%%%%%%%%%%%%%%%%%%%
\section{Introduction}

\LaTeX{} provides a mechanism to structure a large document (such as a book)
into a main file and several child files (containing the chapters)
using the |\include| command.
This mechanism is beneficial for documents
which span hundreds of pages in order to
make the source file(s) more manageable.
Moreover, compilation can be restricted to
selected child files by means of the |\includeonly| command.
The latter feature can be used to reduce the compilation time while editing
(this was significantly more useful in the earlier days of \LaTeX{})
or to generate a smaller document which is easier to navigate.
Another application of |\includeonly| is to generate
documents consisting of selected parts of the complete document.

However, there are a few drawbacks of the plain |\include| mechanism:
\begin{itemize}
\item
The child files cannot be compiled on their own,
they can only be compiled via the main file.
A naive editing environment
(such as a text editor with an option
to have the current file processed by \LaTeX)
may require one to switch to the main file before compiling;
attempting to compile the child file produces errors.
\item
The main file must be modified (each time)
to adjust the |\includeonly| command
to the present needs. This easily leaves the main file in a messy state.
\item
The generated document will always carry the filename
of the main document. This is inconvenient if
several child files are to be compiled and
to be kept for distribution.
\end{itemize}

The present package provides a simple interface
to make child files individually compilable by \LaTeX{}.
Compiling a child file then has the same effect as compiling
the main file with an |\includeonly| command
to select the appropriate child.
Moreover the generated document will carry the name of the child
rather than the main file.
This resolves all three above issues.

This feature is meant to make the editing of books,
thesis documents and lecture notes somewhat more convenient.
However, the package can also be used efficiently for
composing a series of documents (such as exercise sheets)
which are typically distributed individually.
It then assists the author in generating the individual documents
(potentially in different versions)
as well as a document containing the collected series.
Another application is in developing style files
or other kinds of included material
where compilation of the style file could redirect
to a sample or test file.

%%%%%%%%%%%%%%%%%%%%%%%%%%%%%%%%%%%%%%%%%%%%%%%%%%%%%%%%%%%%%%%%%%%%%%%%%%%%%%%%
%%%%%%%%%%%%%%%%%%%%%%%%%%%%%%%%%%%%%%%%%%%%%%%%%%%%%%%%%%%%%%%%%%%%%%%%%%%%%%%%
\section{Usage}

First of all, the package \textsf{childdoc} is \emph{not} a standard
\LaTeXe{} |.sty| style file! Therefore it needs to be invoked in
a non-standard way.

%%%%%%%%%%%%%%%%%%%%%%%%%%%%%%%%%%%%%%%%%%%%%%%%%%%%%%%%%%%%%%%%%%%%%%%%%%%%%%%%
\subsection{Included Files}
\label{sec:include}

%%%%%%%%%%%%%%%%%%%%%%%%%%%%%%%%%%%%%%%%
\DescribeMacro{\childdocmain}
To use the package, add the commands
\begin{center}
\begin{tabular}{l}
|\input{childdoc.def}|\\
|\childdocmain{}|\\
\end{tabular}
\end{center}
at the very top of the main \LaTeX{} file,
in particular \emph{before} the |\documentclass| statement!
The argument of |\childdocmain| should be left empty
(but it must be present).

%%%%%%%%%%%%%%%%%%%%%%%%%%%%%%%%%%%%%%%%
\DescribeMacro{\childdocof}
Furthermore, add the commands
\begin{center}
\begin{tabular}{l}
|\input{childdoc.def}|\\
|\childdocof{|\textit{main}|}|\\
\end{tabular}
\end{center}
at the top of every child file \textit{child}
which is included by |\include{|\textit{child}|}|
from within the main file
(or at least for those files to be compiled individually).
The argument \textit{main} must be the filename of the main file.

There are a couple of
considerations in setting up the main and child documents:

%%%%%%%%%%%%%%%%%%%%%%%%%%%%%%%%%%%%%%%%
\paragraph{Restrictions.}

Please note the following restrictions:
\begin{itemize}
\item
|\childdocmain| must be called with one argument \textit{main}
to ensure compatibility with earlier version of the package.
It must either be empty (|\childdocmain{}|)
or precisely match the filename of the main file in which it is specified.
See \secref{sec:detection} for further information.
\item
The filename \textit{main} must be specified without the |.tex| extension.
\item
The filename \textit{main} is case sensitive
(even in case-insensitive file systems)
due to internal string comparison.
\item
The argument \textit{main} should be fully expanded, it cannot be a macro.
\item
Subdirectories and special characters should be avoided in filenames.
\item
The command |\childdocmain{|\textit{main}|}| must be followed by a whitespace.
It should not be followed immediately by another command
or by a comment mark `|%|'.
This is because the \TeX{} parser reads the token immediately following
the argument of |\childdocmain| and puts it
at the beginning of every child section;
however, a white\-space is ignored.
\end{itemize}

%%%%%%%%%%%%%%%%%%%%%%%%%%%%%%%%%%%%%%%%
\paragraph{Content of Main File.}

It is advisable to place all content in the child files included by |\include|.
Any output contained in the main file will appear in all child documents
unless suppressed manually;
it cannot be suppressed automatically by the |\includeonly| directive
and thus should normally be avoided.
A method to include some content in the main file
by means of conditional processing is described in \secref{sec:conditional}.

%%%%%%%%%%%%%%%%%%%%%%%%%%%%%%%%%%%%%%%%
\paragraph{Page Numbering.}

When only a part of the document is compiled,
the appropriate numbering of pages
(as well as other status parameters)
is determined from the |.aux| files.
The latter contain information from previous passes.
However this information needs to propagate through
all intermediate child documents.
Therefore the page numbering in child documents may well
be inconsistent until the complete document is compiled at least once.

A useful (if unconventional) way to always ensure a consistent
page numbering is to restart the numbering in each child document
and denote the pages by `\textit{child}|.|\textit{page}'
where \textit{child} represents the chapter/section number of the child file.
This can be achieved by the command
|\numberwithin{page}{|\textit{child}|}|
of the \textsf{amsmath} package
where \textit{child} can be |chapter| or |section|
depending on the chosen structuring.
Alternatively, one can modify the macro |\thepage| appropriately
and reset the counter |page| at the start of each child file.

%%%%%%%%%%%%%%%%%%%%%%%%%%%%%%%%%%%%%%%%%%%%%%%%%%%%%%%%%%%%%%%%%%%%%%%%%%%%%%%%
\subsection{Conditional Processing}
\label{sec:conditional}

The package provides a mechanism to compile different versions
of a document. To customise the versions further some conditional processing
can come in handy to distinguish which version is being compiled.
The package provides two macros to describe the compilation context:

%%%%%%%%%%%%%%%%%%%%%%%%%%%%%%%%%%%%%%%%
\DescribeMacro{\ifchilddoc}
The conditional |\ifchilddoc| distinguishes between the compilation of
child documents and the main document:
%
\begin{center}
|\ifchilddoc |\textit{child-code}| |[|\||else |\textit{main-code}]| \||fi|
\end{center}

%%%%%%%%%%%%%%%%%%%%%%%%%%%%%%%%%%%%%%%%
\DescribeMacro{\childdocname}
\DescribeMacro{\childdocjob}
The macro |\childdocname| contains the filename (without extension)
of the main or child file being processed.
Note that |\childdocjob| will always contain the name of the main file.

%%%%%%%%%%%%%%%%%%%%%%%%%%%%%%%%%%%%%%%%
\paragraph{Title Page.}

Conditional processing can be used to include a title or banner page
in the main document when proper precautions are taken.
Importantly, the code in the main file should ensure that the page counter
(as well as other status parameters which are stored in the |.aux| files)
takes the same value after the conditional processing.
Otherwise the page numbers may take divergent values
depending on which part is compiled.

For example, a title page could be declared by:
%
\begin{center}
\begin{tabular}{l}
|\ifchilddoc\||else|\\
|\addtocounter{page}{-1}|\\
\textit{code for title page}\\
|\newpage|\\
|\||fi|
\end{tabular}
\end{center}
%
A banner page for the child documents can be generated by:
%
\begin{center}
\begin{tabular}{l}
|\ifchilddoc|\\
|\addtocounter{page}{-1}|\\
\textit{code for banner page}\\
|\newpage|\\
|\||fi|
\end{tabular}
\end{center}
%
Here one could write a message such as:
\begin{center}
|This is the part \childdocname{} of \childdocjob{}.|
\end{center}

%%%%%%%%%%%%%%%%%%%%%%%%%%%%%%%%%%%%%%%%%%%%%%%%%%%%%%%%%%%%%%%%%%%%%%%%%%%%%%%%
\subsection{Flags}
\label{sec:flags}

The package makes it easy to generate different versions
of the main or child documents.
To this end compilation flags can be defined
and assigned different default values.
They will be particularly useful in conjunction
with the forwarding mechanism described in \secref{sec:forward}.

For example, it may be useful to have a flag |\version|
which can be set to |draft| or |final|.
The document source will contain some conditional code
depending on the value of |\version|.
Suppose further, the flag should default to |final| for the main file
and to |draft| for child files
which is a natural assignment for editing the document.
This is achieved by placing the following code
in the preamble of the main document
(below the |\childdocmain| directive):
%
\begin{center}
\begin{tabular}{l}
|\ifchilddoc|\\
|\providecommand{\version}{draft}|\\
|\||else|\\
|\providecommand{\version}{final}|\\
|\||fi|
\end{tabular}
\end{center}
%
The definition by |\providecommand| makes sure
that previous definitions are not overwritten.
Further statements |\providecommand{\version}{...}|
can thus be added before the above code to override it.

For the main file, one might add a line
(between |\childdocmain| and the above block)
%
\begin{center}
|%\ifchilddoc\||else\providecommand{\version}{draft}\||fi|
\end{center}
%
which can be uncommented to produce a draft version.
Likewise one can add a line to the very top of a child file
(above the |\childdocof{|\textit{main}|}| directive)
%
\begin{center}
|%\providecommand{\version}{final}|
\end{center}
%
which can be uncommented to produce the final version of this child document.

%%%%%%%%%%%%%%%%%%%%%%%%%%%%%%%%%%%%%%%%%%%%%%%%%%%%%%%%%%%%%%%%%%%%%%%%%%%%%%%%
\subsection{Forwarding}
\label{sec:forward}

Different versions of the main or child documents
using compilation flags as described in \secref{sec:flags}
can be (permanently) stored in different files
for convenient compilation, viewing and distribution.
To this end, the package defines a command
to pass on compilation to a different file:

%%%%%%%%%%%%%%%%%%%%%%%%%%%%%%%%%%%%%%%%
\DescribeMacro{\childdocforward}
The command |\childdocforward| redirects processing to
another source file:
%
\begin{center}
\begin{tabular}{l}
|\input{childdoc.def}|\\
|\childdocforward[|\textit{main}|]{|\textit{dest}|}|\\
\end{tabular}
\end{center}
%
The argument \textit{dest} is the destination file
(without extension).
It should be the main file or one of the child files.
Note that further \textsf{childdoc} directives
such as |\childdocof| and |\childdocforward|
in the indicated file will be processed in this form.
The optional argument \textit{main}
passes on directly to the main file \textit{main}
while pretending to compile the child \textit{dest}.
This form behaves as if \textit{dest}
issues |\childdocof{|\textit{main}|}| right away,
and no further \textsf{childdoc} directives will be processed.

%%%%%%%%%%%%%%%%%%%%%%%%%%%%%%%%%%%%%%%%
\DescribeMacro{\...prefix}
In the alternative form |\childdocforwardprefix|,
%
\begin{center}
\begin{tabular}{l}
|\input{childdoc.def}|\\
|\childdocforwardprefix[|\textit{main}|]{|\textit{prefix}|}{|\textit{dest}|}|
\end{tabular}
\end{center}
%
the destination file is determined by a pattern
depending on the current file:
To make this work, the current file must be called
`{\textit{prefix}\hspace{0.2em}\textit{suffix}}'
with \textit{prefix} matching precisely the argument.
Processing is then passed on to the file
`{\textit{dest}\hspace{0.2em}\textit{suffix}}'.
Surely, the same effect is achieved by
directly specifying the
argument `{\textit{dest}\hspace{0.2em}\textit{suffix}}'
in the first form.
However, that requires to set up a different file
for each child. With the alternative form of the command
all these files can have exactly the same content
which simplifies setting them up and maintaining them.

For example, the following file |draft.tex|
with a compilation flag |\version| as described in \secref{sec:flags}
compiles the main document as a draft:
%
\begin{center}
\begin{tabular}{l}
|\def\version{draft}|\\
|\input{childdoc.def}|\\
|\childdocforward{|\textit{main}|}|
\end{tabular}
\end{center}
%
Likewise, the following files |final|\textit{nn}|.tex|
compile the final version of the child document
|child|\textit{nn}|.tex|:
%
\begin{center}
\begin{tabular}{l}
|\def\version{final}|\\
|\input{childdoc.def}|\\
|\childdocforwardprefix{final}{child}|
\end{tabular}
\end{center}
%

Note that when several versions of a main file and/or of each child file
are to be generated, it may be convenient to set up a |Makefile| or
shell script to automatise the process.

%%%%%%%%%%%%%%%%%%%%%%%%%%%%%%%%%%%%%%%%%%%%%%%%%%%%%%%%%%%%%%%%%%%%%%%%%%%%%%%%
\subsection{Command Line Processing}
\label{sec:commandline}

The effect of redirection files can also be achieved by invoking
the \LaTeX{} compiler with a more elaborate command line.
Most conveniently this should be done as part
of a shell script or a |Makefile|.

When using \textsf{childdoc} in the main file, the following
command lines effectively perform a redirection
(note that depending on the shell being used,
backslashes may have to be doubled: `|\|' $\to$ `|\\|'):
%
\begin{center}
|... -jobname "|\textit{target}|" |\\|"|[\textit{flags}]%
|\input{childdoc.def}\childdocforward[|\textit{main}|]{|\textit{dest}|}"|
\end{center}
%
Here \textit{target} is the name of the output file,
\textit{main} is the name of the main file
and \textit{dest} is the name of the main or child file to be processed
(all filenames without extensions).
The optional argument \textit{main} can be omitted
if \textit{main} matches \textit{dest}.
Optionally, compilation \textit{flags} can be defined via |\def| commands.
This command line makes the \TeX{} engine believe
it is compiling the file \textit{target}
whose content is specified as the latter parameter.
The provided code then forwards the processing to
\textit{main} or \textit{dest} as described in \secref{sec:forward}.

%%%%%%%%%%%%%%%%%%%%%%%%%%%%%%%%%%%%%%%%%%%%%%%%%%%%%%%%%%%%%%%%%%%%%%%%%%%%%%%%
\subsection{Include by Input}
\label{sec:input}

Including child documents by |\include| has some restrictions by design.
Most notably, the content of a child document always occupies
its own set of pages; pages cannot be shared between child documents.
Usually, this behaviour makes perfect sense
because each child document contain an essential part of the document.
However, in some situations it may be desirable to compose
a document from a collection of parts
without having mandatory page breaks between then.
For this case, the package
provides a mechanism to include parts
by |\input| which can also be processed individually.
However, by construction this mechanism
requires manual handling of the content to be output.

%%%%%%%%%%%%%%%%%%%%%%%%%%%%%%%%%%%%%%%%
\DescribeMacro{\ifchilddocmanual}
The main file should be prepared as usual, see \secref{sec:include}.
However, the document body must make a distinction
between processing of an individual part and of the main document, e.g.:
%
\begin{center}
\begin{tabular}{l}
|\ifchilddocmanual|\\
|\input{\childdocname}|\\
|\||else|\\
\textit{document body with }|\input{|\textit{part}|}|\\
|\||fi|
\end{tabular}
\end{center}
%
The conditional |\ifchilddocmanual| is true whenever
a part to be included by |\input| is being compiled,
and the name of the part is stored in |\childdocname|.

%%%%%%%%%%%%%%%%%%%%%%%%%%%%%%%%%%%%%%%%
\DescribeMacro{\childdocby}
Each part to be included by |\input| should start with:
%
\begin{center}
\begin{tabular}{l}
|\input{childdoc.def}|\\
|\childdocby{|\textit{main}|}|\\
\end{tabular}
\end{center}
%
The directive |\childdocby| is similar to |\childdocof|
described in \secref{sec:include},
but the subsequent selection of content must be done manually.
To that end, both |\ifchilddoc| and |\ifchilddocmanual|
will be true upon processing of a part,
and the name of the part is stored in |\childdocname|.
Note that |\jobname| will be set to the filename of the current part
so that each part receives an individual |.aux| file
that does not interfere with the |.aux| file(s) of the main document.
This behaviour can be altered by the alternative form
|\childdocby[*]{|\textit{main}|}| (with a non-empty optional argument)
which uses the |.aux| file of the main document
by setting |\jobname| to \textit{main}.

%%%%%%%%%%%%%%%%%%%%%%%%%%%%%%%%%%%%%%%%%%%%%%%%%%%%%%%%%%%%%%%%%%%%%%%%%%%%%%%%
\subsection{Driver Development}
\label{sec:driver}

The \textsf{childdoc} mechanism can also be use for the development
of definition files such as \LaTeX{} styles or classes.
This case differs from the above setup with multiple parts
included by |\include| in that no |\includeonly| should be invoked.
This can be achieved by starting the include file
(before |\ProvidesPackage|) with:
%
\begin{center}
\begin{tabular}{l}
|\input{childdoc.def}|\\
|\childdocforward{|\textit{main}|}|\\
\end{tabular}
\end{center}
%
or alternatively with:
%
\begin{center}
\begin{tabular}{l}
|\input{childdoc.def}|\\
|\childdocby{|\textit{main}|}|\\
\end{tabular}
\end{center}
%
Both forms have slightly different effects as described above.
The main file is prepared as usual, see \secref{sec:include}.

%%%%%%%%%%%%%%%%%%%%%%%%%%%%%%%%%%%%%%%%%%%%%%%%%%%%%%%%%%%%%%%%%%%%%%%%%%%%%%%%
\subsection{Legacy Detection}
\label{sec:detection}

The directive |\childdocmain| in the main file can detect
whether the complete document or merely a child is to be compiled
even without using the directive |\childdocof|.
This method is deprecated because it is less robust
and there is no compelling reason to use it;
it is merely provided for backward compatibility
and it may be removed in future versions.

If the detection mechanism is to be used,
it is mandatory to correctly specify
the filename of the main file as the argument of |\childdocmain|:
%
\begin{center}
\begin{tabular}{l}
|\input{childdoc.def}|\\
|\childdocmain{|\textit{main}|}|\\
\end{tabular}
\end{center}
%
If |\jobname| does not match the argument \textit{main} of |\childdocmain|,
it is assumed that |\jobname| points to the child file to be compiled.
When using |\childdocmain| with the main file specified as argument,
it suffices to start a child file
with just |\input{|\textit{main}|}|
without loading of the package and using |\childdocof|.
If instead all processing is done
with the appropriate \textsf{childdoc} directives,
the argument of \textit{main} of |\childdocmain| can be empty.

An alternative version of the command line processing described
in \secref{sec:commandline} using the detection mechanism reads:
%
\begin{center}
|... -jobname "|\textit{target}|" "|[\textit{flags}]%
[|\def\jobname{|\textit{dest}|}|]|\input{|\textit{main}|}"|
\end{center}

%%%%%%%%%%%%%%%%%%%%%%%%%%%%%%%%%%%%%%%%%%%%%%%%%%%%%%%%%%%%%%%%%%%%%%%%%%%%%%%%
\subsection{Manual Code}
\label{sec:manual}

In case one cannot be certain whether the definitions file |childdoc.def|
is installed on the target \TeX{} distribution
and one prefers not to ship it,
it is conceivable to paste a few relevant commands into the sources.

To that end, drop all statements |\input{childdoc.def}|
and perform the replacements as outlined below.
Instead of |\childdocmain{|\textit{main}|}| add the following code
to the top of the main file:
%
\begin{center}
\begin{tabular}{l}
|\||ifdefined\childdocname\endinput\||fi\newif\ifchilddoc|\\
|\edef\childdocname{\scantokens\expandafter{\jobname\noexpand}}|\\
|\def\childdocmain{|\textit{main}|}\||ifx\childdocmain\childdocname\||else|\\
|\childdoctrue\includeonly{\childdocname}\let\jobname\childdocmain\||fi|\\
\end{tabular}
\end{center}
%
Instead of |\childdocof{|\textit{main}|}| just include the main file
at the top of each child file:
%
\begin{center}
|\input{|\textit{main}|}|
\end{center}
%
A simple redirection |\childdocforward{|\textit{dest}|}| is achieved by:
%
\begin{center}
|\def\jobname{|\textit{dest}|}\input{\jobname}|
\end{center}
%
The redirection with prefix
|\childdocforwardprefix[|\textit{prefix}|]{|\textit{dest}|}|
is accomplished by:
%
\begin{center}
\begin{tabular}{l}
|{\edef\jobname{\scantokens\expandafter{\jobname\noexpand}}|\\
|\def\redirectjob |\textit{prefix}|#1~~~{\gdef\jobname{|\textit{dest}|#1}}|\\
|\expandafter\redirectjob\jobname~~~}\input{\jobname}|
\end{tabular}
\end{center}

In an alternative approach,
child documents can be compiled by a specific command line
without additional code or specific definitions:
%
\begin{center}
|... -jobname "|\textit{target}|" "|[\textit{flags}]%
|\includeonly{|\textit{dest}|}\input{|\textit{main}|}"|
\end{center}
%

%%%%%%%%%%%%%%%%%%%%%%%%%%%%%%%%%%%%%%%%%%%%%%%%%%%%%%%%%%%%%%%%%%%%%%%%%%%%%%%%
%%%%%%%%%%%%%%%%%%%%%%%%%%%%%%%%%%%%%%%%%%%%%%%%%%%%%%%%%%%%%%%%%%%%%%%%%%%%%%%%
\section{Information}

%%%%%%%%%%%%%%%%%%%%%%%%%%%%%%%%%%%%%%%%%%%%%%%%%%%%%%%%%%%%%%%%%%%%%%%%%%%%%%%%
\subsection{Copyright}

Copyright \copyright{} 2017--2018 Niklas Beisert

This work may be distributed and/or modified under the
conditions of the \LaTeX{} Project Public License, either version 1.3
of this license or (at your option) any later version.
The latest version of this license is in
  \url{http://www.latex-project.org/lppl.txt}
and version 1.3 or later is part of all distributions of \LaTeX{}
version 2005/12/01 or later.

This work has the LPPL maintenance status `maintained'.

The Current Maintainer of this work is Niklas Beisert.

This work consists of the files |README.txt|, |childdoc.ins| and |childdoc.dtx|
as well as the derived files |childdoc.def|, |cdocsamp.tex|
with |cdocsch1.tex|, |cdocsch2.tex|, |cdocspt3.tex|, |cdocspt4.tex|,
|cdocsdrf.tex|, |cdocsfn1.tex|, |cdocsfn2.tex|
as well as |childdoc.pdf|.

%%%%%%%%%%%%%%%%%%%%%%%%%%%%%%%%%%%%%%%%%%%%%%%%%%%%%%%%%%%%%%%%%%%%%%%%%%%%%%%%
\subsection{Files and Installation}

The package consists of the files:
%
\begin{center}
\begin{tabular}{ll}
    |README.txt|   & readme file \\
    |childdoc.ins| & installation file \\
    |childdoc.dtx| & source file \\
    |childdoc.def| & definition file \\
    |cdocsamp.tex| & sample main file \\
    |cdocsch1.tex| & sample include file \\
    |cdocsch2.tex| & sample include file \\
    |cdocspt3.tex| & sample part file \\
    |cdocspt4.tex| & sample part file \\
    |cdocsdrf.tex| & sample redirection file \\
    |cdocsfn1.tex| & sample redirection file \\
    |cdocsfn2.tex| & sample redirection file \\
    |childdoc.pdf| & manual
\end{tabular}
\end{center}
%
The distribution consists of the files
|README.txt|, |childdoc.ins| and |childdoc.dtx|.
%
\begin{itemize}
\item
Run (pdf)\LaTeX{} on |childdoc.dtx|
to compile the manual |childdoc.pdf| (this file).
\item
Run \LaTeX{} on |childdoc.ins| to create the definitions file |childdoc.def|
and the sample |cdocsamp.tex| with include files
|cdocsch1.tex|, |cdocsch2.tex|, |cdocspt3.tex|, |cdocspt4.tex|,
|cdocsdrf.tex|, |cdocsfn1.tex|, |cdocsfn2.tex|.
Then copy the file |childdoc.def| to an appropriate directory of your \LaTeX{}
distribution, e.g.\ \textit{texmf-root}|/tex/latex/childdoc|.
\end{itemize}

%%%%%%%%%%%%%%%%%%%%%%%%%%%%%%%%%%%%%%%%%%%%%%%%%%%%%%%%%%%%%%%%%%%%%%%%%%%%%%%%
\subsection{Related CTAN Packages}

There are several other packages which offer a similar functionality:
%
\begin{itemize}
\item
The packages
\href{http://ctan.org/pkg/docmute}{\textsf{docmute}},
\href{http://ctan.org/pkg/includex}{\textsf{includex}} and
\href{http://ctan.org/pkg/standalone}{\textsf{standalone}}
provide commands to include only the document body of
a child file thus allowing both files to be compiled individually.
\item
The packages \href{http://ctan.org/pkg/subdocs}{\textsf{subdocs}}
and \href{http://ctan.org/pkg/subfiles}{\textsf{subfiles}}
provide structures in which the main and child documents can be
encapsulated and allowing them to be compiled individually.
The inclusion mechanism is different from the conventional |\include|.
\item
The package \href{http://ctan.org/pkg/combine}{\textsf{combine}}
is an elaborate solution to combine several documents into one.
\end{itemize}
%
See also the CTAN topic \href{http://ctan.org/topic/subdocs}{\textsf{subdocs}}
for further related packages.
The present package differs from the above solutions in that
a document structure constructed with the conventional |\include| mechanism
just needs two extra commands at the top of every file
such that all constituent files can be compiled individually.

%%%%%%%%%%%%%%%%%%%%%%%%%%%%%%%%%%%%%%%%%%%%%%%%%%%%%%%%%%%%%%%%%%%%%%%%%%%%%%%%
%\subsection{Feature Suggestions}
%
%The following is a list of features which may be useful for future
%versions of this package:
%%
%\begin{itemize}
%\item
%\ldots
%\end{itemize}

%%%%%%%%%%%%%%%%%%%%%%%%%%%%%%%%%%%%%%%%%%%%%%%%%%%%%%%%%%%%%%%%%%%%%%%%%%%%%%%%
\subsection{Revision History}

%%%%%%%%%%%%%%%%%%%%%%%%%%%%%%%%%%%%%%%%
\paragraph{v2.0:} 2018/12/30

\begin{itemize}
\item
immediate forward processing
\item
added |\childdocby| mechanism
\item
manual restructured
\end{itemize}

%%%%%%%%%%%%%%%%%%%%%%%%%%%%%%%%%%%%%%%%
\paragraph{v1.6:} 2018/01/17

\begin{itemize}
\item
application for development of include files
\item
corrections to manual
\end{itemize}

%%%%%%%%%%%%%%%%%%%%%%%%%%%%%%%%%%%%%%%%
\paragraph{v1.5:} 2017/05/21

\begin{itemize}
\item
more complete structuring introduced
\item
|\childdocof| introduced
\item
|\childdoc| renamed to |\childdocmain|
\item
|\childredirect| renamed to |\childdocforward| and |\childdocforwardprefix|
and functionality expanded
\end{itemize}

%%%%%%%%%%%%%%%%%%%%%%%%%%%%%%%%%%%%%%%%
\paragraph{v1.0:} 2017/04/27

\begin{itemize}
\item
manual and install package
\item
first version published on CTAN
\end{itemize}

%%%%%%%%%%%%%%%%%%%%%%%%%%%%%%%%%%%%%%%%
\paragraph{v0.6:} 2017/04/26

\begin{itemize}
\item
redirection mechanism added
\end{itemize}

%%%%%%%%%%%%%%%%%%%%%%%%%%%%%%%%%%%%%%%%
\paragraph{v0.5:} 2017/04/26

\begin{itemize}
\item
functionality in definition file
\end{itemize}


%%%%%%%%%%%%%%%%%%%%%%%%%%%%%%%%%%%%%%%%%%%%%%%%%%%%%%%%%%%%%%%%%%%%%%%%%%%%%%%%
%%%%%%%%%%%%%%%%%%%%%%%%%%%%%%%%%%%%%%%%%%%%%%%%%%%%%%%%%%%%%%%%%%%%%%%%%%%%%%%%
%%%%%%%%%%%%%%%%%%%%%%%%%%%%%%%%%%%%%%%%%%%%%%%%%%%%%%%%%%%%%%%%%%%%%%%%%%%%%%%%
\appendix

\settowidth\MacroIndent{\rmfamily\scriptsize 000\ }

 \DocInput{childdoc.dtx}

\end{document}
%</driver>
% \fi
%
% %%%%%%%%%%%%%%%%%%%%%%%%%%%%%%%%%%%%%%%%%%%%%%%%%%%%%%%%%%%%%%%%%%%%%%%%%%%%%%
% %%%%%%%%%%%%%%%%%%%%%%%%%%%%%%%%%%%%%%%%%%%%%%%%%%%%%%%%%%%%%%%%%%%%%%%%%%%%%%
% \section{Sample}
%\iffalse
%<*samplemain>
%\fi
%
% The following presents a sample document
% with two chapters, two parts, a title page,
% a compile flag as well as three forwarding files to set the flag.
% It consists of eight |.tex| files:
% \begin{center}
% \begin{tabular}{ll}
% |cdocsamp.tex|&main file\\
% |cdocsch1.tex|&include file for chapter 1\\
% |cdocsch2.tex|&include file for chapter 2\\
% |cdocspt3.tex|&include file for part 3\\
% |cdocspt4.tex|&include file for part 4\\
% |cdocsdrf.tex|&forwarding file for main file in draft mode\\
% |cdocsfi1.tex|&forwarding file for final version of chapter 1\\
% |cdocsfi2.tex|&forwarding file for final version of chapter 2\\
% \end{tabular}
% \end{center}
% Each of the eight files can be compiled directly by the \LaTeX{} compiler.
%
% %%%%%%%%%%%%%%%%%%%%%%%%%%%%%%%%%%%%%%
% \paragraph{Main File.}
%
% The main file is called |cdocsamp.tex|.
%
% Load the \textsf{childdoc} definitions and
% declare the filename for the main document:
%    \begin{macrocode}
\input{childdoc.def}
\childdocmain{}
%    \end{macrocode}

% Optional override for |\version| flag:
%    \begin{macrocode}
%%\ifchilddoc\else\providecommand{\version}{draft}\fi
%    \end{macrocode}

% Define the default values for the |\version| flag
% (|final| for the main file and |draft| for childs):
%    \begin{macrocode}
\ifchilddoc
\providecommand{\version}{draft}
\else
\providecommand{\version}{final}
\fi
%    \end{macrocode}

% Load the standard document class:
%    \begin{macrocode}
\documentclass[12pt]{article}
%    \end{macrocode}

% Start the document body:
%    \begin{macrocode}
\begin{document}
%    \end{macrocode}

% Declare a title page.
% Print title, part of document being processed and version flag:
%    \begin{macrocode}
\addtocounter{page}{-1}
\begin{center}
{\LARGE\bfseries{}childdoc example\par}
\vspace{1cm}
\ifchilddoc
\ifchilddocmanual part\else chapter\fi:
`\childdocname' of `\childdocjob'\par
\else
main document: `\childdocjob'\par
\fi
version: \version\par
\end{center}
\newpage
%    \end{macrocode}

% Manually include selected file,
% otherwise process as usual:
%    \begin{macrocode}
\ifchilddocmanual
\section*{part `\childdocname'}
\input{\childdocname}
\else
%    \end{macrocode}

% Include the two chapters:
%    \begin{macrocode}
\include{cdocsch1}
\include{cdocsch2}
%    \end{macrocode}

% Include the two parts unless only chapters should be displayed:
%    \begin{macrocode}
\ifchilddoc\else
\section{part three}
\input{cdocspt3}
\section{part four}
\input{cdocspt4}
\fi
%    \end{macrocode}

% Process as usual until here:
%    \begin{macrocode}
\fi
%    \end{macrocode}

% End of document body:
%    \begin{macrocode}
\end{document}
%    \end{macrocode}
%\iffalse
%</samplemain>
%\fi
%
% %%%%%%%%%%%%%%%%%%%%%%%%%%%%%%%%%%%%%%
% \paragraph{Chapter Include Files.}
%
% The include files are called |cdocsch1.tex| and |cdocsch2.tex|.
%
%\iffalse
%<*samplechap1|samplechap2>
%\fi

% Optional override for |\version| flag:
%    \begin{macrocode}
%%\providecommand{\version}{final}
%    \end{macrocode}

% Include the main document:
%    \begin{macrocode}
\input{childdoc.def}
\childdocof{cdocsamp}
%    \end{macrocode}

%\iffalse
%</samplechap1|samplechap2>
%\fi
%
%\iffalse
%<*samplechap1>
%\fi
% Some text for chapter 1:
%    \begin{macrocode}
\section{one}
some text in chapter one
%    \end{macrocode}

%\iffalse
%</samplechap1>
%\fi
% Some text for chapter 2:
%\iffalse
%<*samplechap2>
%\fi
%    \begin{macrocode}
\section{two}
more text in chapter two
%    \end{macrocode}

%\iffalse
%</samplechap2>
%\fi
%
% %%%%%%%%%%%%%%%%%%%%%%%%%%%%%%%%%%%%%%
% \paragraph{Part Include Files.}
%
% The include files are called |cdocspt3.tex| and |cdocspt4.tex|.
%
%\iffalse
%<*samplepart3|samplepart4>
%\fi

% Optional override for |\version| flag:
%    \begin{macrocode}
%%\providecommand{\version}{final}
%    \end{macrocode}

% Include the main document:
%    \begin{macrocode}
\input{childdoc.def}
\childdocby{cdocsamp}
%    \end{macrocode}

%\iffalse
%</samplepart3|samplepart4>
%\fi
%
%\iffalse
%<*samplepart3>
%\fi
% Some text for part 3:
%    \begin{macrocode}
some text in part three
%    \end{macrocode}

%\iffalse
%</samplepart3>
%\fi
% Some text for part 4:
%\iffalse
%<*samplepart4>
%\fi
%    \begin{macrocode}
more text in part four
%    \end{macrocode}

%\iffalse
%</samplepart4>
%\fi
%
% %%%%%%%%%%%%%%%%%%%%%%%%%%%%%%%%%%%%%%
% \paragraph{Forwarding for a Complete Draft.}
%
% The following forwarding file |cdocsdrf.tex|
% compiles the main document in draft mode:
%\iffalse
%<*sampledraft>
%\fi
%    \begin{macrocode}
\def\version{draft}
\input{childdoc.def}
\childdocforward{cdocsamp}
%    \end{macrocode}

%\iffalse
%</sampledraft>
%\fi
%
% %%%%%%%%%%%%%%%%%%%%%%%%%%%%%%%%%%%%%%
% \paragraph{Forwarding for Final Version of the Chapters.}
%
% The following forwarding files |cdocsfn1.tex| and |cdocsfn2.tex|
% (with identical content)
% compile the final versions of the child documents
% |cdocsch1.tex| and |cdocsch2.tex|, respectively:
%\iffalse
%<*samplefinal>
%\fi
%    \begin{macrocode}
\def\version{final}
\input{childdoc.def}
\childdocforwardprefix[cdocsamp]{cdocsfn}{cdocsch}
%    \end{macrocode}

%\iffalse
%</samplefinal>
%\fi
%
% %%%%%%%%%%%%%%%%%%%%%%%%%%%%%%%%%%%%%%
% \paragraph{Command Line Processing.}
%
% The following three command lines generate the output files
% |cdocscld|, |cdocscl1| and |cdocscl2|
% which should be identical to
% |cdocsdrf|, |cdocsch1| and |cdocsfn2|, respectively:
% \begin{center}
% \begin{tabular}{l}
% |latex -jobname cdocscld \|\\
% |  "\def\version{draft}\input{childdoc.def}\childdocforward{cdocsamp}"|\\
% |latex -jobname cdocscl1 \|\\
% |  "\input{childdoc.def}\childdocforward[cdocsamp]{cdocsch1}"|\\
% |latex -jobname cdocscl2 \|\\
% |  "\def\version{final}\input{childdoc.def}\childdocforward{cdocsch2}"|
% \end{tabular}
% \end{center}
% Note that the trailing backslash on each first line
% merely continues the input to the second line
% (for convenient cut ant paste).
% Furthermore, the command |latex| can be replaced by any
% of its alternative versions such as |pdflatex|.
%
% %%%%%%%%%%%%%%%%%%%%%%%%%%%%%%%%%%%%%%%%%%%%%%%%%%%%%%%%%%%%%%%%%%%%%%%%%%%%%%
% %%%%%%%%%%%%%%%%%%%%%%%%%%%%%%%%%%%%%%%%%%%%%%%%%%%%%%%%%%%%%%%%%%%%%%%%%%%%%%
% \section{Implementation}
%\iffalse
%<*package>
%\fi
%
% This section describes the definitions file |childdoc.def|.

% The definitions cannot be loaded using |\usepackage| or |\RequirePackage|
% which has a mechanism to prevent loading a style file more than once.
% When loading the definitions by means of |\input|
% multiple instances have to be prevented manually:
%\iffalse
%This code needs to be before the `\ProvidesFile' directive
%which is defined at the beginning of this file.
%Therefore it is also placed there and commented out here.
%</package>
%<*discard>
%\fi
%    \begin{macrocode}
\ifdefined\childdocmain\endinput\fi
%    \end{macrocode}
%\iffalse
%</discard>
%<*package>
%\fi
%
% \macro{\ifchilddoc}
% \macro{\ifchilddocmanual}
% The conditional |\ifchilddoc| tells whether a
% child (true) or main (false) document is being compiled.
% The conditional |\ifchilddocmanual| tells whether
% the |\includeonly| mechanism is used (false) or
% the selection of child files must be performed manually (true).
% The definitions initialise to false:
%    \begin{macrocode}
\newif\ifchilddoc
\newif\ifchilddocmanual
%    \end{macrocode}

% \macro{\childdocname}
% \macro{\childdocjob}
% The macro |\childdocname| stores the name of the main document
% to be compiled. The macro |\childdocjob| stores the name of
% the document on which the \LaTeX{} compiler was originally invoked.
% The content of |\jobname| cannot be compared
% to filenames specified in the source due to different catcodes.
% The following code rescans |\jobname|, stores the result
% in |\childdocname| and saves a copy in |\childdocjob|:
%    \begin{macrocode}
\edef\childdocname{\scantokens\expandafter{\jobname\noexpand}}
\let\childdocjob\childdocname
%    \end{macrocode}

% \macro{\childdocdisable}
% The macro |\childdocdisable| prevents the main file
% from being processed more than once.
% At this stage, the main document command |\childdocmain|
% is assumed to be called once again where it should do nothing.
% Any subsequent call to it should prevent
% a secondary processing of the main document
% It overwrites the forwarding commands
% |\childdocof| and |\childdocforward|
% with empty macros to prevent further inclusions of the main document:
%    \begin{macrocode}
\newcommand{\childdocdisable}
{
  \renewcommand{\childdocmain}[1]{\renewcommand{\childdocmain}[1]{\endinput}}
  \renewcommand{\childdocof}[1]{}
  \renewcommand{\childdocby}[2][]{}
  \renewcommand{\childdocforward}[2][]{}
  \renewcommand{\childdocdisable}{}
}
%    \end{macrocode}

% \macro{\childdocmain}
% The macro |\childdocmain| is to be called at the top of the main file
% with nothing or the main filename (without extension) as argument.
% First, it breaks loops.
% If the argument is not empty and does not match |\childdocname|
% (which is set by the first inclusion of |childdoc.def|),
% |\ifchilddoc| is set to true, |\includeonly| is applied to the child file
% and |\jobname| is set to the main file
% (for proper handling of |.aux| files):
%    \begin{macrocode}
\newcommand{\childdocmain}[1]
{
  \childdocdisable\childdocmain{}
  \if?#1?\else
    \begingroup
      \def\childdoctmp{#1}
      \ifx\childdoctmp\childdocname
        \def\childdoctmp{}
      \else
        \def\childdoctmp
        {
          \childdoctrue
          \includeonly{\childdocname}
          \def\childdocjob{#1}
          \def\jobname{#1}
        }
      \fi
      \expandafter
    \endgroup
    \childdoctmp
  \fi
}
%    \end{macrocode}

% \macro{\childdocof}
% The command |\childdocof| redirects
% compilation to the main file |#1|.
%    \begin{macrocode}
\newcommand{\childdocof}[1]
{
  \childdocdisable
  \childdoctrue
  \includeonly{\childdocname}
  \def\jobname{#1}
  \def\childdocjob{#1}
  \input{#1}
}
%    \end{macrocode}

% \macro{\childdocby}
% The command |\childdocby| ....
%    \begin{macrocode}
\newcommand{\childdocby}[2][]
{
  \childdocdisable
  \childdoctrue
  \childdocmanualtrue
  \if?#1?\else
    \def\jobname{#2}
  \fi
  \def\childdocjob{#2}
  \input{#2}
  \endinput
}
%    \end{macrocode}

% \macro{\childdocforward}
% The command |\childdocforward| redirects
% compilation to the main file or
% (if the optional argument is given) a child file.
% Parameters are set as if the main file
% or a child file starting with |\childdocof| was compiled.
% Then compilation is handed over to the main file:
%    \begin{macrocode}
\newcommand{\childdocforward}[2][]
{
  \begingroup
    \if?#1?
      \def\childdoctmp
      {
        \def\childdocname{#2}
        \def\childdocjob{#2}
        \def\jobname{#2}
        \input{#2}
        \endinput
      }
    \else
      \def\childdoctmp
      {
        \childdocdisable
        \def\childdocname{#2}
        \childdoctrue
        \includeonly{#2}
        \def\childdocjob{#1}
        \def\jobname{#1}
        \input{#1}
        \endinput
      }
    \fi
    \expandafter
  \endgroup
  \childdoctmp
}
%    \end{macrocode}

% \macro{\childdocforwardprefix}
% The command |\childdocforwardprefix| redirects
% compilation to the main or a child file by means of a pattern.
% The prefix |#1| in the current filename is replaced by |#2|
% and the suffix of the current filename is kept
% (it is assumed that the filename does not contain the substring `|~~~|'
% which is used as a delimiter).
% Compilation is handed over to the new file by |\childdocforward|:
%    \begin{macrocode}
\newcommand{\childdocforwardprefix}[3][]
{
  \begingroup
    \def\childdocextract #2##1~~~{\def\childdoctmp{\childdocforward[#1]{#3##1}}}
    \expandafter\childdocextract\childdocname~~~
    \expandafter
  \endgroup
  \childdoctmp
}
%    \end{macrocode}

% \macro{\childdoc}
% The deprecated macro |\childdoc| is a legacy version of |\childdocmain|:
%    \begin{macrocode}
\newcommand{\childdoc}{\childdocmain}
%    \end{macrocode}

% \macro{\childdocredirect}
% The deprecated macro |\childdocredirect| is a legacy version
% of |\childdocforward| and |\childdocforwardprefix|:
%    \begin{macrocode}
\newcommand{\childdocredirect}[2][]
{
  \begingroup
    \if?#1?
      \def\childdoctmp{\childdocforward{#2}}
    \else
      \def\childdoctmp{\childdocforwardprefix{#1}{#2}}
    \fi
    \expandafter
  \endgroup
  \childdoctmp
}
%    \end{macrocode}

%\iffalse
%</package>
%\fi
%
\endinput
|\\
|\childdocforward[|\textit{main}|]{|\textit{dest}|}|\\
\end{tabular}
\end{center}
%
The argument \textit{dest} is the destination file
(without extension).
It should be the main file or one of the child files.
Note that further \textsf{childdoc} directives
such as |\childdocof| and |\childdocforward|
in the indicated file will be processed in this form.
The optional argument \textit{main}
passes on directly to the main file \textit{main}
while pretending to compile the child \textit{dest}.
This form behaves as if \textit{dest}
issues |\childdocof{|\textit{main}|}| right away,
and no further \textsf{childdoc} directives will be processed.

%%%%%%%%%%%%%%%%%%%%%%%%%%%%%%%%%%%%%%%%
\DescribeMacro{\...prefix}
In the alternative form |\childdocforwardprefix|,
%
\begin{center}
\begin{tabular}{l}
|% \iffalse
%
% childdoc.dtx Copyright (C) 2017-2018 Niklas Beisert
%
% This work may be distributed and/or modified under the
% conditions of the LaTeX Project Public License, either version 1.3
% of this license or (at your option) any later version.
% The latest version of this license is in
%   http://www.latex-project.org/lppl.txt
% and version 1.3 or later is part of all distributions of LaTeX
% version 2005/12/01 or later.
%
% This work has the LPPL maintenance status `maintained'.
%
% The Current Maintainer of this work is Niklas Beisert.
%
% This work consists of the files childdoc.dtx and childdoc.ins
% and the derived files childdoc.def and cdocsamp.tex with
% cdocsch1.tex, cdocsch2.tex, cdocsdrf.tex, cdocsfn1.tex, cdocsfn2.tex.
%
%<package>\ifdefined\childdocmain\endinput\fi
%<package>\ProvidesFile{childdoc.def}[2018/12/30 v2.0 child document driver]
%<samplemain>\ProvidesFile{cdocsamp.tex}[2018/12/30 v2.0 sample for childdoc]
%<*driver>
%\ProvidesFile{childdoc.drv}[2018/12/30 v2.0 childdoc reference manual file]
\PassOptionsToClass{10pt,a4paper}{article}
\documentclass{ltxdoc}

\usepackage[margin=35mm]{geometry}
\usepackage{hyperref}
\usepackage{hyperxmp}
\usepackage[usenames]{color}

\hypersetup{colorlinks=true}
\hypersetup{pdfstartview=FitH}
\hypersetup{pdfpagemode=UseNone}
\hypersetup{pdfsource={}}
\hypersetup{pdflang={en-UK}}
\hypersetup{pdfcopyright={Copyright 2017-2018 Niklas Beisert.
  This work may be distributed and/or modified under the
  conditions of the LaTeX Project Public License, either version 1.3
  of this license or (at your option) any later version.}}
\hypersetup{pdflicenseurl={http://www.latex-project.org/lppl.txt}}
\hypersetup{pdfcontactaddress={ETH Zurich, ITP, HIT K,
  Wolfgang-Pauli-Strasse 27}}
\hypersetup{pdfcontactpostcode={8093}}
\hypersetup{pdfcontactcity={Zurich}}
\hypersetup{pdfcontactcountry={Switzerland}}
\hypersetup{pdfcontactemail={nbeisert@itp.phys.ethz.ch}}
\hypersetup{pdfcontacturl={http://people.phys.ethz.ch/\xmptilde nbeisert/}}

\newcommand{\secref}[1]{\hyperref[#1]{section \ref*{#1}}}

\parskip1ex
\parindent0pt
\let\olditemize\itemize
\def\itemize{\olditemize\parskip0pt}

\begin{document}

\title{The \textsf{childdoc} Package}
\hypersetup{pdftitle={The childdoc Package}}
\author{Niklas Beisert\\[2ex]
  Institut f\"ur Theoretische Physik\\
  Eidgen\"ossische Technische Hochschule Z\"urich\\
  Wolfgang-Pauli-Strasse 27, 8093 Z\"urich, Switzerland\\[1ex]
  \href{mailto:nbeisert@itp.phys.ethz.ch}
  {\texttt{nbeisert@itp.phys.ethz.ch}}}
\hypersetup{pdfauthor={Niklas Beisert}}
\hypersetup{pdfsubject={Manual for the LaTeX2e Package childdoc}}
\date{30 December 2018, \textsf{v2.0}}
\maketitle

\begin{abstract}\noindent
\textsf{childdoc} is a \LaTeXe{} package
that enables the direct compilation
of document sections included by |\include|
to individual files.
\end{abstract}

\begingroup
\parskip0ex
\tableofcontents
\endgroup

%%%%%%%%%%%%%%%%%%%%%%%%%%%%%%%%%%%%%%%%%%%%%%%%%%%%%%%%%%%%%%%%%%%%%%%%%%%%%%%%
%%%%%%%%%%%%%%%%%%%%%%%%%%%%%%%%%%%%%%%%%%%%%%%%%%%%%%%%%%%%%%%%%%%%%%%%%%%%%%%%
\section{Introduction}

\LaTeX{} provides a mechanism to structure a large document (such as a book)
into a main file and several child files (containing the chapters)
using the |\include| command.
This mechanism is beneficial for documents
which span hundreds of pages in order to
make the source file(s) more manageable.
Moreover, compilation can be restricted to
selected child files by means of the |\includeonly| command.
The latter feature can be used to reduce the compilation time while editing
(this was significantly more useful in the earlier days of \LaTeX{})
or to generate a smaller document which is easier to navigate.
Another application of |\includeonly| is to generate
documents consisting of selected parts of the complete document.

However, there are a few drawbacks of the plain |\include| mechanism:
\begin{itemize}
\item
The child files cannot be compiled on their own,
they can only be compiled via the main file.
A naive editing environment
(such as a text editor with an option
to have the current file processed by \LaTeX)
may require one to switch to the main file before compiling;
attempting to compile the child file produces errors.
\item
The main file must be modified (each time)
to adjust the |\includeonly| command
to the present needs. This easily leaves the main file in a messy state.
\item
The generated document will always carry the filename
of the main document. This is inconvenient if
several child files are to be compiled and
to be kept for distribution.
\end{itemize}

The present package provides a simple interface
to make child files individually compilable by \LaTeX{}.
Compiling a child file then has the same effect as compiling
the main file with an |\includeonly| command
to select the appropriate child.
Moreover the generated document will carry the name of the child
rather than the main file.
This resolves all three above issues.

This feature is meant to make the editing of books,
thesis documents and lecture notes somewhat more convenient.
However, the package can also be used efficiently for
composing a series of documents (such as exercise sheets)
which are typically distributed individually.
It then assists the author in generating the individual documents
(potentially in different versions)
as well as a document containing the collected series.
Another application is in developing style files
or other kinds of included material
where compilation of the style file could redirect
to a sample or test file.

%%%%%%%%%%%%%%%%%%%%%%%%%%%%%%%%%%%%%%%%%%%%%%%%%%%%%%%%%%%%%%%%%%%%%%%%%%%%%%%%
%%%%%%%%%%%%%%%%%%%%%%%%%%%%%%%%%%%%%%%%%%%%%%%%%%%%%%%%%%%%%%%%%%%%%%%%%%%%%%%%
\section{Usage}

First of all, the package \textsf{childdoc} is \emph{not} a standard
\LaTeXe{} |.sty| style file! Therefore it needs to be invoked in
a non-standard way.

%%%%%%%%%%%%%%%%%%%%%%%%%%%%%%%%%%%%%%%%%%%%%%%%%%%%%%%%%%%%%%%%%%%%%%%%%%%%%%%%
\subsection{Included Files}
\label{sec:include}

%%%%%%%%%%%%%%%%%%%%%%%%%%%%%%%%%%%%%%%%
\DescribeMacro{\childdocmain}
To use the package, add the commands
\begin{center}
\begin{tabular}{l}
|\input{childdoc.def}|\\
|\childdocmain{}|\\
\end{tabular}
\end{center}
at the very top of the main \LaTeX{} file,
in particular \emph{before} the |\documentclass| statement!
The argument of |\childdocmain| should be left empty
(but it must be present).

%%%%%%%%%%%%%%%%%%%%%%%%%%%%%%%%%%%%%%%%
\DescribeMacro{\childdocof}
Furthermore, add the commands
\begin{center}
\begin{tabular}{l}
|\input{childdoc.def}|\\
|\childdocof{|\textit{main}|}|\\
\end{tabular}
\end{center}
at the top of every child file \textit{child}
which is included by |\include{|\textit{child}|}|
from within the main file
(or at least for those files to be compiled individually).
The argument \textit{main} must be the filename of the main file.

There are a couple of
considerations in setting up the main and child documents:

%%%%%%%%%%%%%%%%%%%%%%%%%%%%%%%%%%%%%%%%
\paragraph{Restrictions.}

Please note the following restrictions:
\begin{itemize}
\item
|\childdocmain| must be called with one argument \textit{main}
to ensure compatibility with earlier version of the package.
It must either be empty (|\childdocmain{}|)
or precisely match the filename of the main file in which it is specified.
See \secref{sec:detection} for further information.
\item
The filename \textit{main} must be specified without the |.tex| extension.
\item
The filename \textit{main} is case sensitive
(even in case-insensitive file systems)
due to internal string comparison.
\item
The argument \textit{main} should be fully expanded, it cannot be a macro.
\item
Subdirectories and special characters should be avoided in filenames.
\item
The command |\childdocmain{|\textit{main}|}| must be followed by a whitespace.
It should not be followed immediately by another command
or by a comment mark `|%|'.
This is because the \TeX{} parser reads the token immediately following
the argument of |\childdocmain| and puts it
at the beginning of every child section;
however, a white\-space is ignored.
\end{itemize}

%%%%%%%%%%%%%%%%%%%%%%%%%%%%%%%%%%%%%%%%
\paragraph{Content of Main File.}

It is advisable to place all content in the child files included by |\include|.
Any output contained in the main file will appear in all child documents
unless suppressed manually;
it cannot be suppressed automatically by the |\includeonly| directive
and thus should normally be avoided.
A method to include some content in the main file
by means of conditional processing is described in \secref{sec:conditional}.

%%%%%%%%%%%%%%%%%%%%%%%%%%%%%%%%%%%%%%%%
\paragraph{Page Numbering.}

When only a part of the document is compiled,
the appropriate numbering of pages
(as well as other status parameters)
is determined from the |.aux| files.
The latter contain information from previous passes.
However this information needs to propagate through
all intermediate child documents.
Therefore the page numbering in child documents may well
be inconsistent until the complete document is compiled at least once.

A useful (if unconventional) way to always ensure a consistent
page numbering is to restart the numbering in each child document
and denote the pages by `\textit{child}|.|\textit{page}'
where \textit{child} represents the chapter/section number of the child file.
This can be achieved by the command
|\numberwithin{page}{|\textit{child}|}|
of the \textsf{amsmath} package
where \textit{child} can be |chapter| or |section|
depending on the chosen structuring.
Alternatively, one can modify the macro |\thepage| appropriately
and reset the counter |page| at the start of each child file.

%%%%%%%%%%%%%%%%%%%%%%%%%%%%%%%%%%%%%%%%%%%%%%%%%%%%%%%%%%%%%%%%%%%%%%%%%%%%%%%%
\subsection{Conditional Processing}
\label{sec:conditional}

The package provides a mechanism to compile different versions
of a document. To customise the versions further some conditional processing
can come in handy to distinguish which version is being compiled.
The package provides two macros to describe the compilation context:

%%%%%%%%%%%%%%%%%%%%%%%%%%%%%%%%%%%%%%%%
\DescribeMacro{\ifchilddoc}
The conditional |\ifchilddoc| distinguishes between the compilation of
child documents and the main document:
%
\begin{center}
|\ifchilddoc |\textit{child-code}| |[|\||else |\textit{main-code}]| \||fi|
\end{center}

%%%%%%%%%%%%%%%%%%%%%%%%%%%%%%%%%%%%%%%%
\DescribeMacro{\childdocname}
\DescribeMacro{\childdocjob}
The macro |\childdocname| contains the filename (without extension)
of the main or child file being processed.
Note that |\childdocjob| will always contain the name of the main file.

%%%%%%%%%%%%%%%%%%%%%%%%%%%%%%%%%%%%%%%%
\paragraph{Title Page.}

Conditional processing can be used to include a title or banner page
in the main document when proper precautions are taken.
Importantly, the code in the main file should ensure that the page counter
(as well as other status parameters which are stored in the |.aux| files)
takes the same value after the conditional processing.
Otherwise the page numbers may take divergent values
depending on which part is compiled.

For example, a title page could be declared by:
%
\begin{center}
\begin{tabular}{l}
|\ifchilddoc\||else|\\
|\addtocounter{page}{-1}|\\
\textit{code for title page}\\
|\newpage|\\
|\||fi|
\end{tabular}
\end{center}
%
A banner page for the child documents can be generated by:
%
\begin{center}
\begin{tabular}{l}
|\ifchilddoc|\\
|\addtocounter{page}{-1}|\\
\textit{code for banner page}\\
|\newpage|\\
|\||fi|
\end{tabular}
\end{center}
%
Here one could write a message such as:
\begin{center}
|This is the part \childdocname{} of \childdocjob{}.|
\end{center}

%%%%%%%%%%%%%%%%%%%%%%%%%%%%%%%%%%%%%%%%%%%%%%%%%%%%%%%%%%%%%%%%%%%%%%%%%%%%%%%%
\subsection{Flags}
\label{sec:flags}

The package makes it easy to generate different versions
of the main or child documents.
To this end compilation flags can be defined
and assigned different default values.
They will be particularly useful in conjunction
with the forwarding mechanism described in \secref{sec:forward}.

For example, it may be useful to have a flag |\version|
which can be set to |draft| or |final|.
The document source will contain some conditional code
depending on the value of |\version|.
Suppose further, the flag should default to |final| for the main file
and to |draft| for child files
which is a natural assignment for editing the document.
This is achieved by placing the following code
in the preamble of the main document
(below the |\childdocmain| directive):
%
\begin{center}
\begin{tabular}{l}
|\ifchilddoc|\\
|\providecommand{\version}{draft}|\\
|\||else|\\
|\providecommand{\version}{final}|\\
|\||fi|
\end{tabular}
\end{center}
%
The definition by |\providecommand| makes sure
that previous definitions are not overwritten.
Further statements |\providecommand{\version}{...}|
can thus be added before the above code to override it.

For the main file, one might add a line
(between |\childdocmain| and the above block)
%
\begin{center}
|%\ifchilddoc\||else\providecommand{\version}{draft}\||fi|
\end{center}
%
which can be uncommented to produce a draft version.
Likewise one can add a line to the very top of a child file
(above the |\childdocof{|\textit{main}|}| directive)
%
\begin{center}
|%\providecommand{\version}{final}|
\end{center}
%
which can be uncommented to produce the final version of this child document.

%%%%%%%%%%%%%%%%%%%%%%%%%%%%%%%%%%%%%%%%%%%%%%%%%%%%%%%%%%%%%%%%%%%%%%%%%%%%%%%%
\subsection{Forwarding}
\label{sec:forward}

Different versions of the main or child documents
using compilation flags as described in \secref{sec:flags}
can be (permanently) stored in different files
for convenient compilation, viewing and distribution.
To this end, the package defines a command
to pass on compilation to a different file:

%%%%%%%%%%%%%%%%%%%%%%%%%%%%%%%%%%%%%%%%
\DescribeMacro{\childdocforward}
The command |\childdocforward| redirects processing to
another source file:
%
\begin{center}
\begin{tabular}{l}
|\input{childdoc.def}|\\
|\childdocforward[|\textit{main}|]{|\textit{dest}|}|\\
\end{tabular}
\end{center}
%
The argument \textit{dest} is the destination file
(without extension).
It should be the main file or one of the child files.
Note that further \textsf{childdoc} directives
such as |\childdocof| and |\childdocforward|
in the indicated file will be processed in this form.
The optional argument \textit{main}
passes on directly to the main file \textit{main}
while pretending to compile the child \textit{dest}.
This form behaves as if \textit{dest}
issues |\childdocof{|\textit{main}|}| right away,
and no further \textsf{childdoc} directives will be processed.

%%%%%%%%%%%%%%%%%%%%%%%%%%%%%%%%%%%%%%%%
\DescribeMacro{\...prefix}
In the alternative form |\childdocforwardprefix|,
%
\begin{center}
\begin{tabular}{l}
|\input{childdoc.def}|\\
|\childdocforwardprefix[|\textit{main}|]{|\textit{prefix}|}{|\textit{dest}|}|
\end{tabular}
\end{center}
%
the destination file is determined by a pattern
depending on the current file:
To make this work, the current file must be called
`{\textit{prefix}\hspace{0.2em}\textit{suffix}}'
with \textit{prefix} matching precisely the argument.
Processing is then passed on to the file
`{\textit{dest}\hspace{0.2em}\textit{suffix}}'.
Surely, the same effect is achieved by
directly specifying the
argument `{\textit{dest}\hspace{0.2em}\textit{suffix}}'
in the first form.
However, that requires to set up a different file
for each child. With the alternative form of the command
all these files can have exactly the same content
which simplifies setting them up and maintaining them.

For example, the following file |draft.tex|
with a compilation flag |\version| as described in \secref{sec:flags}
compiles the main document as a draft:
%
\begin{center}
\begin{tabular}{l}
|\def\version{draft}|\\
|\input{childdoc.def}|\\
|\childdocforward{|\textit{main}|}|
\end{tabular}
\end{center}
%
Likewise, the following files |final|\textit{nn}|.tex|
compile the final version of the child document
|child|\textit{nn}|.tex|:
%
\begin{center}
\begin{tabular}{l}
|\def\version{final}|\\
|\input{childdoc.def}|\\
|\childdocforwardprefix{final}{child}|
\end{tabular}
\end{center}
%

Note that when several versions of a main file and/or of each child file
are to be generated, it may be convenient to set up a |Makefile| or
shell script to automatise the process.

%%%%%%%%%%%%%%%%%%%%%%%%%%%%%%%%%%%%%%%%%%%%%%%%%%%%%%%%%%%%%%%%%%%%%%%%%%%%%%%%
\subsection{Command Line Processing}
\label{sec:commandline}

The effect of redirection files can also be achieved by invoking
the \LaTeX{} compiler with a more elaborate command line.
Most conveniently this should be done as part
of a shell script or a |Makefile|.

When using \textsf{childdoc} in the main file, the following
command lines effectively perform a redirection
(note that depending on the shell being used,
backslashes may have to be doubled: `|\|' $\to$ `|\\|'):
%
\begin{center}
|... -jobname "|\textit{target}|" |\\|"|[\textit{flags}]%
|\input{childdoc.def}\childdocforward[|\textit{main}|]{|\textit{dest}|}"|
\end{center}
%
Here \textit{target} is the name of the output file,
\textit{main} is the name of the main file
and \textit{dest} is the name of the main or child file to be processed
(all filenames without extensions).
The optional argument \textit{main} can be omitted
if \textit{main} matches \textit{dest}.
Optionally, compilation \textit{flags} can be defined via |\def| commands.
This command line makes the \TeX{} engine believe
it is compiling the file \textit{target}
whose content is specified as the latter parameter.
The provided code then forwards the processing to
\textit{main} or \textit{dest} as described in \secref{sec:forward}.

%%%%%%%%%%%%%%%%%%%%%%%%%%%%%%%%%%%%%%%%%%%%%%%%%%%%%%%%%%%%%%%%%%%%%%%%%%%%%%%%
\subsection{Include by Input}
\label{sec:input}

Including child documents by |\include| has some restrictions by design.
Most notably, the content of a child document always occupies
its own set of pages; pages cannot be shared between child documents.
Usually, this behaviour makes perfect sense
because each child document contain an essential part of the document.
However, in some situations it may be desirable to compose
a document from a collection of parts
without having mandatory page breaks between then.
For this case, the package
provides a mechanism to include parts
by |\input| which can also be processed individually.
However, by construction this mechanism
requires manual handling of the content to be output.

%%%%%%%%%%%%%%%%%%%%%%%%%%%%%%%%%%%%%%%%
\DescribeMacro{\ifchilddocmanual}
The main file should be prepared as usual, see \secref{sec:include}.
However, the document body must make a distinction
between processing of an individual part and of the main document, e.g.:
%
\begin{center}
\begin{tabular}{l}
|\ifchilddocmanual|\\
|\input{\childdocname}|\\
|\||else|\\
\textit{document body with }|\input{|\textit{part}|}|\\
|\||fi|
\end{tabular}
\end{center}
%
The conditional |\ifchilddocmanual| is true whenever
a part to be included by |\input| is being compiled,
and the name of the part is stored in |\childdocname|.

%%%%%%%%%%%%%%%%%%%%%%%%%%%%%%%%%%%%%%%%
\DescribeMacro{\childdocby}
Each part to be included by |\input| should start with:
%
\begin{center}
\begin{tabular}{l}
|\input{childdoc.def}|\\
|\childdocby{|\textit{main}|}|\\
\end{tabular}
\end{center}
%
The directive |\childdocby| is similar to |\childdocof|
described in \secref{sec:include},
but the subsequent selection of content must be done manually.
To that end, both |\ifchilddoc| and |\ifchilddocmanual|
will be true upon processing of a part,
and the name of the part is stored in |\childdocname|.
Note that |\jobname| will be set to the filename of the current part
so that each part receives an individual |.aux| file
that does not interfere with the |.aux| file(s) of the main document.
This behaviour can be altered by the alternative form
|\childdocby[*]{|\textit{main}|}| (with a non-empty optional argument)
which uses the |.aux| file of the main document
by setting |\jobname| to \textit{main}.

%%%%%%%%%%%%%%%%%%%%%%%%%%%%%%%%%%%%%%%%%%%%%%%%%%%%%%%%%%%%%%%%%%%%%%%%%%%%%%%%
\subsection{Driver Development}
\label{sec:driver}

The \textsf{childdoc} mechanism can also be use for the development
of definition files such as \LaTeX{} styles or classes.
This case differs from the above setup with multiple parts
included by |\include| in that no |\includeonly| should be invoked.
This can be achieved by starting the include file
(before |\ProvidesPackage|) with:
%
\begin{center}
\begin{tabular}{l}
|\input{childdoc.def}|\\
|\childdocforward{|\textit{main}|}|\\
\end{tabular}
\end{center}
%
or alternatively with:
%
\begin{center}
\begin{tabular}{l}
|\input{childdoc.def}|\\
|\childdocby{|\textit{main}|}|\\
\end{tabular}
\end{center}
%
Both forms have slightly different effects as described above.
The main file is prepared as usual, see \secref{sec:include}.

%%%%%%%%%%%%%%%%%%%%%%%%%%%%%%%%%%%%%%%%%%%%%%%%%%%%%%%%%%%%%%%%%%%%%%%%%%%%%%%%
\subsection{Legacy Detection}
\label{sec:detection}

The directive |\childdocmain| in the main file can detect
whether the complete document or merely a child is to be compiled
even without using the directive |\childdocof|.
This method is deprecated because it is less robust
and there is no compelling reason to use it;
it is merely provided for backward compatibility
and it may be removed in future versions.

If the detection mechanism is to be used,
it is mandatory to correctly specify
the filename of the main file as the argument of |\childdocmain|:
%
\begin{center}
\begin{tabular}{l}
|\input{childdoc.def}|\\
|\childdocmain{|\textit{main}|}|\\
\end{tabular}
\end{center}
%
If |\jobname| does not match the argument \textit{main} of |\childdocmain|,
it is assumed that |\jobname| points to the child file to be compiled.
When using |\childdocmain| with the main file specified as argument,
it suffices to start a child file
with just |\input{|\textit{main}|}|
without loading of the package and using |\childdocof|.
If instead all processing is done
with the appropriate \textsf{childdoc} directives,
the argument of \textit{main} of |\childdocmain| can be empty.

An alternative version of the command line processing described
in \secref{sec:commandline} using the detection mechanism reads:
%
\begin{center}
|... -jobname "|\textit{target}|" "|[\textit{flags}]%
[|\def\jobname{|\textit{dest}|}|]|\input{|\textit{main}|}"|
\end{center}

%%%%%%%%%%%%%%%%%%%%%%%%%%%%%%%%%%%%%%%%%%%%%%%%%%%%%%%%%%%%%%%%%%%%%%%%%%%%%%%%
\subsection{Manual Code}
\label{sec:manual}

In case one cannot be certain whether the definitions file |childdoc.def|
is installed on the target \TeX{} distribution
and one prefers not to ship it,
it is conceivable to paste a few relevant commands into the sources.

To that end, drop all statements |\input{childdoc.def}|
and perform the replacements as outlined below.
Instead of |\childdocmain{|\textit{main}|}| add the following code
to the top of the main file:
%
\begin{center}
\begin{tabular}{l}
|\||ifdefined\childdocname\endinput\||fi\newif\ifchilddoc|\\
|\edef\childdocname{\scantokens\expandafter{\jobname\noexpand}}|\\
|\def\childdocmain{|\textit{main}|}\||ifx\childdocmain\childdocname\||else|\\
|\childdoctrue\includeonly{\childdocname}\let\jobname\childdocmain\||fi|\\
\end{tabular}
\end{center}
%
Instead of |\childdocof{|\textit{main}|}| just include the main file
at the top of each child file:
%
\begin{center}
|\input{|\textit{main}|}|
\end{center}
%
A simple redirection |\childdocforward{|\textit{dest}|}| is achieved by:
%
\begin{center}
|\def\jobname{|\textit{dest}|}\input{\jobname}|
\end{center}
%
The redirection with prefix
|\childdocforwardprefix[|\textit{prefix}|]{|\textit{dest}|}|
is accomplished by:
%
\begin{center}
\begin{tabular}{l}
|{\edef\jobname{\scantokens\expandafter{\jobname\noexpand}}|\\
|\def\redirectjob |\textit{prefix}|#1~~~{\gdef\jobname{|\textit{dest}|#1}}|\\
|\expandafter\redirectjob\jobname~~~}\input{\jobname}|
\end{tabular}
\end{center}

In an alternative approach,
child documents can be compiled by a specific command line
without additional code or specific definitions:
%
\begin{center}
|... -jobname "|\textit{target}|" "|[\textit{flags}]%
|\includeonly{|\textit{dest}|}\input{|\textit{main}|}"|
\end{center}
%

%%%%%%%%%%%%%%%%%%%%%%%%%%%%%%%%%%%%%%%%%%%%%%%%%%%%%%%%%%%%%%%%%%%%%%%%%%%%%%%%
%%%%%%%%%%%%%%%%%%%%%%%%%%%%%%%%%%%%%%%%%%%%%%%%%%%%%%%%%%%%%%%%%%%%%%%%%%%%%%%%
\section{Information}

%%%%%%%%%%%%%%%%%%%%%%%%%%%%%%%%%%%%%%%%%%%%%%%%%%%%%%%%%%%%%%%%%%%%%%%%%%%%%%%%
\subsection{Copyright}

Copyright \copyright{} 2017--2018 Niklas Beisert

This work may be distributed and/or modified under the
conditions of the \LaTeX{} Project Public License, either version 1.3
of this license or (at your option) any later version.
The latest version of this license is in
  \url{http://www.latex-project.org/lppl.txt}
and version 1.3 or later is part of all distributions of \LaTeX{}
version 2005/12/01 or later.

This work has the LPPL maintenance status `maintained'.

The Current Maintainer of this work is Niklas Beisert.

This work consists of the files |README.txt|, |childdoc.ins| and |childdoc.dtx|
as well as the derived files |childdoc.def|, |cdocsamp.tex|
with |cdocsch1.tex|, |cdocsch2.tex|, |cdocspt3.tex|, |cdocspt4.tex|,
|cdocsdrf.tex|, |cdocsfn1.tex|, |cdocsfn2.tex|
as well as |childdoc.pdf|.

%%%%%%%%%%%%%%%%%%%%%%%%%%%%%%%%%%%%%%%%%%%%%%%%%%%%%%%%%%%%%%%%%%%%%%%%%%%%%%%%
\subsection{Files and Installation}

The package consists of the files:
%
\begin{center}
\begin{tabular}{ll}
    |README.txt|   & readme file \\
    |childdoc.ins| & installation file \\
    |childdoc.dtx| & source file \\
    |childdoc.def| & definition file \\
    |cdocsamp.tex| & sample main file \\
    |cdocsch1.tex| & sample include file \\
    |cdocsch2.tex| & sample include file \\
    |cdocspt3.tex| & sample part file \\
    |cdocspt4.tex| & sample part file \\
    |cdocsdrf.tex| & sample redirection file \\
    |cdocsfn1.tex| & sample redirection file \\
    |cdocsfn2.tex| & sample redirection file \\
    |childdoc.pdf| & manual
\end{tabular}
\end{center}
%
The distribution consists of the files
|README.txt|, |childdoc.ins| and |childdoc.dtx|.
%
\begin{itemize}
\item
Run (pdf)\LaTeX{} on |childdoc.dtx|
to compile the manual |childdoc.pdf| (this file).
\item
Run \LaTeX{} on |childdoc.ins| to create the definitions file |childdoc.def|
and the sample |cdocsamp.tex| with include files
|cdocsch1.tex|, |cdocsch2.tex|, |cdocspt3.tex|, |cdocspt4.tex|,
|cdocsdrf.tex|, |cdocsfn1.tex|, |cdocsfn2.tex|.
Then copy the file |childdoc.def| to an appropriate directory of your \LaTeX{}
distribution, e.g.\ \textit{texmf-root}|/tex/latex/childdoc|.
\end{itemize}

%%%%%%%%%%%%%%%%%%%%%%%%%%%%%%%%%%%%%%%%%%%%%%%%%%%%%%%%%%%%%%%%%%%%%%%%%%%%%%%%
\subsection{Related CTAN Packages}

There are several other packages which offer a similar functionality:
%
\begin{itemize}
\item
The packages
\href{http://ctan.org/pkg/docmute}{\textsf{docmute}},
\href{http://ctan.org/pkg/includex}{\textsf{includex}} and
\href{http://ctan.org/pkg/standalone}{\textsf{standalone}}
provide commands to include only the document body of
a child file thus allowing both files to be compiled individually.
\item
The packages \href{http://ctan.org/pkg/subdocs}{\textsf{subdocs}}
and \href{http://ctan.org/pkg/subfiles}{\textsf{subfiles}}
provide structures in which the main and child documents can be
encapsulated and allowing them to be compiled individually.
The inclusion mechanism is different from the conventional |\include|.
\item
The package \href{http://ctan.org/pkg/combine}{\textsf{combine}}
is an elaborate solution to combine several documents into one.
\end{itemize}
%
See also the CTAN topic \href{http://ctan.org/topic/subdocs}{\textsf{subdocs}}
for further related packages.
The present package differs from the above solutions in that
a document structure constructed with the conventional |\include| mechanism
just needs two extra commands at the top of every file
such that all constituent files can be compiled individually.

%%%%%%%%%%%%%%%%%%%%%%%%%%%%%%%%%%%%%%%%%%%%%%%%%%%%%%%%%%%%%%%%%%%%%%%%%%%%%%%%
%\subsection{Feature Suggestions}
%
%The following is a list of features which may be useful for future
%versions of this package:
%%
%\begin{itemize}
%\item
%\ldots
%\end{itemize}

%%%%%%%%%%%%%%%%%%%%%%%%%%%%%%%%%%%%%%%%%%%%%%%%%%%%%%%%%%%%%%%%%%%%%%%%%%%%%%%%
\subsection{Revision History}

%%%%%%%%%%%%%%%%%%%%%%%%%%%%%%%%%%%%%%%%
\paragraph{v2.0:} 2018/12/30

\begin{itemize}
\item
immediate forward processing
\item
added |\childdocby| mechanism
\item
manual restructured
\end{itemize}

%%%%%%%%%%%%%%%%%%%%%%%%%%%%%%%%%%%%%%%%
\paragraph{v1.6:} 2018/01/17

\begin{itemize}
\item
application for development of include files
\item
corrections to manual
\end{itemize}

%%%%%%%%%%%%%%%%%%%%%%%%%%%%%%%%%%%%%%%%
\paragraph{v1.5:} 2017/05/21

\begin{itemize}
\item
more complete structuring introduced
\item
|\childdocof| introduced
\item
|\childdoc| renamed to |\childdocmain|
\item
|\childredirect| renamed to |\childdocforward| and |\childdocforwardprefix|
and functionality expanded
\end{itemize}

%%%%%%%%%%%%%%%%%%%%%%%%%%%%%%%%%%%%%%%%
\paragraph{v1.0:} 2017/04/27

\begin{itemize}
\item
manual and install package
\item
first version published on CTAN
\end{itemize}

%%%%%%%%%%%%%%%%%%%%%%%%%%%%%%%%%%%%%%%%
\paragraph{v0.6:} 2017/04/26

\begin{itemize}
\item
redirection mechanism added
\end{itemize}

%%%%%%%%%%%%%%%%%%%%%%%%%%%%%%%%%%%%%%%%
\paragraph{v0.5:} 2017/04/26

\begin{itemize}
\item
functionality in definition file
\end{itemize}


%%%%%%%%%%%%%%%%%%%%%%%%%%%%%%%%%%%%%%%%%%%%%%%%%%%%%%%%%%%%%%%%%%%%%%%%%%%%%%%%
%%%%%%%%%%%%%%%%%%%%%%%%%%%%%%%%%%%%%%%%%%%%%%%%%%%%%%%%%%%%%%%%%%%%%%%%%%%%%%%%
%%%%%%%%%%%%%%%%%%%%%%%%%%%%%%%%%%%%%%%%%%%%%%%%%%%%%%%%%%%%%%%%%%%%%%%%%%%%%%%%
\appendix

\settowidth\MacroIndent{\rmfamily\scriptsize 000\ }

 \DocInput{childdoc.dtx}

\end{document}
%</driver>
% \fi
%
% %%%%%%%%%%%%%%%%%%%%%%%%%%%%%%%%%%%%%%%%%%%%%%%%%%%%%%%%%%%%%%%%%%%%%%%%%%%%%%
% %%%%%%%%%%%%%%%%%%%%%%%%%%%%%%%%%%%%%%%%%%%%%%%%%%%%%%%%%%%%%%%%%%%%%%%%%%%%%%
% \section{Sample}
%\iffalse
%<*samplemain>
%\fi
%
% The following presents a sample document
% with two chapters, two parts, a title page,
% a compile flag as well as three forwarding files to set the flag.
% It consists of eight |.tex| files:
% \begin{center}
% \begin{tabular}{ll}
% |cdocsamp.tex|&main file\\
% |cdocsch1.tex|&include file for chapter 1\\
% |cdocsch2.tex|&include file for chapter 2\\
% |cdocspt3.tex|&include file for part 3\\
% |cdocspt4.tex|&include file for part 4\\
% |cdocsdrf.tex|&forwarding file for main file in draft mode\\
% |cdocsfi1.tex|&forwarding file for final version of chapter 1\\
% |cdocsfi2.tex|&forwarding file for final version of chapter 2\\
% \end{tabular}
% \end{center}
% Each of the eight files can be compiled directly by the \LaTeX{} compiler.
%
% %%%%%%%%%%%%%%%%%%%%%%%%%%%%%%%%%%%%%%
% \paragraph{Main File.}
%
% The main file is called |cdocsamp.tex|.
%
% Load the \textsf{childdoc} definitions and
% declare the filename for the main document:
%    \begin{macrocode}
\input{childdoc.def}
\childdocmain{}
%    \end{macrocode}

% Optional override for |\version| flag:
%    \begin{macrocode}
%%\ifchilddoc\else\providecommand{\version}{draft}\fi
%    \end{macrocode}

% Define the default values for the |\version| flag
% (|final| for the main file and |draft| for childs):
%    \begin{macrocode}
\ifchilddoc
\providecommand{\version}{draft}
\else
\providecommand{\version}{final}
\fi
%    \end{macrocode}

% Load the standard document class:
%    \begin{macrocode}
\documentclass[12pt]{article}
%    \end{macrocode}

% Start the document body:
%    \begin{macrocode}
\begin{document}
%    \end{macrocode}

% Declare a title page.
% Print title, part of document being processed and version flag:
%    \begin{macrocode}
\addtocounter{page}{-1}
\begin{center}
{\LARGE\bfseries{}childdoc example\par}
\vspace{1cm}
\ifchilddoc
\ifchilddocmanual part\else chapter\fi:
`\childdocname' of `\childdocjob'\par
\else
main document: `\childdocjob'\par
\fi
version: \version\par
\end{center}
\newpage
%    \end{macrocode}

% Manually include selected file,
% otherwise process as usual:
%    \begin{macrocode}
\ifchilddocmanual
\section*{part `\childdocname'}
\input{\childdocname}
\else
%    \end{macrocode}

% Include the two chapters:
%    \begin{macrocode}
\include{cdocsch1}
\include{cdocsch2}
%    \end{macrocode}

% Include the two parts unless only chapters should be displayed:
%    \begin{macrocode}
\ifchilddoc\else
\section{part three}
\input{cdocspt3}
\section{part four}
\input{cdocspt4}
\fi
%    \end{macrocode}

% Process as usual until here:
%    \begin{macrocode}
\fi
%    \end{macrocode}

% End of document body:
%    \begin{macrocode}
\end{document}
%    \end{macrocode}
%\iffalse
%</samplemain>
%\fi
%
% %%%%%%%%%%%%%%%%%%%%%%%%%%%%%%%%%%%%%%
% \paragraph{Chapter Include Files.}
%
% The include files are called |cdocsch1.tex| and |cdocsch2.tex|.
%
%\iffalse
%<*samplechap1|samplechap2>
%\fi

% Optional override for |\version| flag:
%    \begin{macrocode}
%%\providecommand{\version}{final}
%    \end{macrocode}

% Include the main document:
%    \begin{macrocode}
\input{childdoc.def}
\childdocof{cdocsamp}
%    \end{macrocode}

%\iffalse
%</samplechap1|samplechap2>
%\fi
%
%\iffalse
%<*samplechap1>
%\fi
% Some text for chapter 1:
%    \begin{macrocode}
\section{one}
some text in chapter one
%    \end{macrocode}

%\iffalse
%</samplechap1>
%\fi
% Some text for chapter 2:
%\iffalse
%<*samplechap2>
%\fi
%    \begin{macrocode}
\section{two}
more text in chapter two
%    \end{macrocode}

%\iffalse
%</samplechap2>
%\fi
%
% %%%%%%%%%%%%%%%%%%%%%%%%%%%%%%%%%%%%%%
% \paragraph{Part Include Files.}
%
% The include files are called |cdocspt3.tex| and |cdocspt4.tex|.
%
%\iffalse
%<*samplepart3|samplepart4>
%\fi

% Optional override for |\version| flag:
%    \begin{macrocode}
%%\providecommand{\version}{final}
%    \end{macrocode}

% Include the main document:
%    \begin{macrocode}
\input{childdoc.def}
\childdocby{cdocsamp}
%    \end{macrocode}

%\iffalse
%</samplepart3|samplepart4>
%\fi
%
%\iffalse
%<*samplepart3>
%\fi
% Some text for part 3:
%    \begin{macrocode}
some text in part three
%    \end{macrocode}

%\iffalse
%</samplepart3>
%\fi
% Some text for part 4:
%\iffalse
%<*samplepart4>
%\fi
%    \begin{macrocode}
more text in part four
%    \end{macrocode}

%\iffalse
%</samplepart4>
%\fi
%
% %%%%%%%%%%%%%%%%%%%%%%%%%%%%%%%%%%%%%%
% \paragraph{Forwarding for a Complete Draft.}
%
% The following forwarding file |cdocsdrf.tex|
% compiles the main document in draft mode:
%\iffalse
%<*sampledraft>
%\fi
%    \begin{macrocode}
\def\version{draft}
\input{childdoc.def}
\childdocforward{cdocsamp}
%    \end{macrocode}

%\iffalse
%</sampledraft>
%\fi
%
% %%%%%%%%%%%%%%%%%%%%%%%%%%%%%%%%%%%%%%
% \paragraph{Forwarding for Final Version of the Chapters.}
%
% The following forwarding files |cdocsfn1.tex| and |cdocsfn2.tex|
% (with identical content)
% compile the final versions of the child documents
% |cdocsch1.tex| and |cdocsch2.tex|, respectively:
%\iffalse
%<*samplefinal>
%\fi
%    \begin{macrocode}
\def\version{final}
\input{childdoc.def}
\childdocforwardprefix[cdocsamp]{cdocsfn}{cdocsch}
%    \end{macrocode}

%\iffalse
%</samplefinal>
%\fi
%
% %%%%%%%%%%%%%%%%%%%%%%%%%%%%%%%%%%%%%%
% \paragraph{Command Line Processing.}
%
% The following three command lines generate the output files
% |cdocscld|, |cdocscl1| and |cdocscl2|
% which should be identical to
% |cdocsdrf|, |cdocsch1| and |cdocsfn2|, respectively:
% \begin{center}
% \begin{tabular}{l}
% |latex -jobname cdocscld \|\\
% |  "\def\version{draft}\input{childdoc.def}\childdocforward{cdocsamp}"|\\
% |latex -jobname cdocscl1 \|\\
% |  "\input{childdoc.def}\childdocforward[cdocsamp]{cdocsch1}"|\\
% |latex -jobname cdocscl2 \|\\
% |  "\def\version{final}\input{childdoc.def}\childdocforward{cdocsch2}"|
% \end{tabular}
% \end{center}
% Note that the trailing backslash on each first line
% merely continues the input to the second line
% (for convenient cut ant paste).
% Furthermore, the command |latex| can be replaced by any
% of its alternative versions such as |pdflatex|.
%
% %%%%%%%%%%%%%%%%%%%%%%%%%%%%%%%%%%%%%%%%%%%%%%%%%%%%%%%%%%%%%%%%%%%%%%%%%%%%%%
% %%%%%%%%%%%%%%%%%%%%%%%%%%%%%%%%%%%%%%%%%%%%%%%%%%%%%%%%%%%%%%%%%%%%%%%%%%%%%%
% \section{Implementation}
%\iffalse
%<*package>
%\fi
%
% This section describes the definitions file |childdoc.def|.

% The definitions cannot be loaded using |\usepackage| or |\RequirePackage|
% which has a mechanism to prevent loading a style file more than once.
% When loading the definitions by means of |\input|
% multiple instances have to be prevented manually:
%\iffalse
%This code needs to be before the `\ProvidesFile' directive
%which is defined at the beginning of this file.
%Therefore it is also placed there and commented out here.
%</package>
%<*discard>
%\fi
%    \begin{macrocode}
\ifdefined\childdocmain\endinput\fi
%    \end{macrocode}
%\iffalse
%</discard>
%<*package>
%\fi
%
% \macro{\ifchilddoc}
% \macro{\ifchilddocmanual}
% The conditional |\ifchilddoc| tells whether a
% child (true) or main (false) document is being compiled.
% The conditional |\ifchilddocmanual| tells whether
% the |\includeonly| mechanism is used (false) or
% the selection of child files must be performed manually (true).
% The definitions initialise to false:
%    \begin{macrocode}
\newif\ifchilddoc
\newif\ifchilddocmanual
%    \end{macrocode}

% \macro{\childdocname}
% \macro{\childdocjob}
% The macro |\childdocname| stores the name of the main document
% to be compiled. The macro |\childdocjob| stores the name of
% the document on which the \LaTeX{} compiler was originally invoked.
% The content of |\jobname| cannot be compared
% to filenames specified in the source due to different catcodes.
% The following code rescans |\jobname|, stores the result
% in |\childdocname| and saves a copy in |\childdocjob|:
%    \begin{macrocode}
\edef\childdocname{\scantokens\expandafter{\jobname\noexpand}}
\let\childdocjob\childdocname
%    \end{macrocode}

% \macro{\childdocdisable}
% The macro |\childdocdisable| prevents the main file
% from being processed more than once.
% At this stage, the main document command |\childdocmain|
% is assumed to be called once again where it should do nothing.
% Any subsequent call to it should prevent
% a secondary processing of the main document
% It overwrites the forwarding commands
% |\childdocof| and |\childdocforward|
% with empty macros to prevent further inclusions of the main document:
%    \begin{macrocode}
\newcommand{\childdocdisable}
{
  \renewcommand{\childdocmain}[1]{\renewcommand{\childdocmain}[1]{\endinput}}
  \renewcommand{\childdocof}[1]{}
  \renewcommand{\childdocby}[2][]{}
  \renewcommand{\childdocforward}[2][]{}
  \renewcommand{\childdocdisable}{}
}
%    \end{macrocode}

% \macro{\childdocmain}
% The macro |\childdocmain| is to be called at the top of the main file
% with nothing or the main filename (without extension) as argument.
% First, it breaks loops.
% If the argument is not empty and does not match |\childdocname|
% (which is set by the first inclusion of |childdoc.def|),
% |\ifchilddoc| is set to true, |\includeonly| is applied to the child file
% and |\jobname| is set to the main file
% (for proper handling of |.aux| files):
%    \begin{macrocode}
\newcommand{\childdocmain}[1]
{
  \childdocdisable\childdocmain{}
  \if?#1?\else
    \begingroup
      \def\childdoctmp{#1}
      \ifx\childdoctmp\childdocname
        \def\childdoctmp{}
      \else
        \def\childdoctmp
        {
          \childdoctrue
          \includeonly{\childdocname}
          \def\childdocjob{#1}
          \def\jobname{#1}
        }
      \fi
      \expandafter
    \endgroup
    \childdoctmp
  \fi
}
%    \end{macrocode}

% \macro{\childdocof}
% The command |\childdocof| redirects
% compilation to the main file |#1|.
%    \begin{macrocode}
\newcommand{\childdocof}[1]
{
  \childdocdisable
  \childdoctrue
  \includeonly{\childdocname}
  \def\jobname{#1}
  \def\childdocjob{#1}
  \input{#1}
}
%    \end{macrocode}

% \macro{\childdocby}
% The command |\childdocby| ....
%    \begin{macrocode}
\newcommand{\childdocby}[2][]
{
  \childdocdisable
  \childdoctrue
  \childdocmanualtrue
  \if?#1?\else
    \def\jobname{#2}
  \fi
  \def\childdocjob{#2}
  \input{#2}
  \endinput
}
%    \end{macrocode}

% \macro{\childdocforward}
% The command |\childdocforward| redirects
% compilation to the main file or
% (if the optional argument is given) a child file.
% Parameters are set as if the main file
% or a child file starting with |\childdocof| was compiled.
% Then compilation is handed over to the main file:
%    \begin{macrocode}
\newcommand{\childdocforward}[2][]
{
  \begingroup
    \if?#1?
      \def\childdoctmp
      {
        \def\childdocname{#2}
        \def\childdocjob{#2}
        \def\jobname{#2}
        \input{#2}
        \endinput
      }
    \else
      \def\childdoctmp
      {
        \childdocdisable
        \def\childdocname{#2}
        \childdoctrue
        \includeonly{#2}
        \def\childdocjob{#1}
        \def\jobname{#1}
        \input{#1}
        \endinput
      }
    \fi
    \expandafter
  \endgroup
  \childdoctmp
}
%    \end{macrocode}

% \macro{\childdocforwardprefix}
% The command |\childdocforwardprefix| redirects
% compilation to the main or a child file by means of a pattern.
% The prefix |#1| in the current filename is replaced by |#2|
% and the suffix of the current filename is kept
% (it is assumed that the filename does not contain the substring `|~~~|'
% which is used as a delimiter).
% Compilation is handed over to the new file by |\childdocforward|:
%    \begin{macrocode}
\newcommand{\childdocforwardprefix}[3][]
{
  \begingroup
    \def\childdocextract #2##1~~~{\def\childdoctmp{\childdocforward[#1]{#3##1}}}
    \expandafter\childdocextract\childdocname~~~
    \expandafter
  \endgroup
  \childdoctmp
}
%    \end{macrocode}

% \macro{\childdoc}
% The deprecated macro |\childdoc| is a legacy version of |\childdocmain|:
%    \begin{macrocode}
\newcommand{\childdoc}{\childdocmain}
%    \end{macrocode}

% \macro{\childdocredirect}
% The deprecated macro |\childdocredirect| is a legacy version
% of |\childdocforward| and |\childdocforwardprefix|:
%    \begin{macrocode}
\newcommand{\childdocredirect}[2][]
{
  \begingroup
    \if?#1?
      \def\childdoctmp{\childdocforward{#2}}
    \else
      \def\childdoctmp{\childdocforwardprefix{#1}{#2}}
    \fi
    \expandafter
  \endgroup
  \childdoctmp
}
%    \end{macrocode}

%\iffalse
%</package>
%\fi
%
\endinput
|\\
|\childdocforwardprefix[|\textit{main}|]{|\textit{prefix}|}{|\textit{dest}|}|
\end{tabular}
\end{center}
%
the destination file is determined by a pattern
depending on the current file:
To make this work, the current file must be called
`{\textit{prefix}\hspace{0.2em}\textit{suffix}}'
with \textit{prefix} matching precisely the argument.
Processing is then passed on to the file
`{\textit{dest}\hspace{0.2em}\textit{suffix}}'.
Surely, the same effect is achieved by
directly specifying the
argument `{\textit{dest}\hspace{0.2em}\textit{suffix}}'
in the first form.
However, that requires to set up a different file
for each child. With the alternative form of the command
all these files can have exactly the same content
which simplifies setting them up and maintaining them.

For example, the following file |draft.tex|
with a compilation flag |\version| as described in \secref{sec:flags}
compiles the main document as a draft:
%
\begin{center}
\begin{tabular}{l}
|\def\version{draft}|\\
|% \iffalse
%
% childdoc.dtx Copyright (C) 2017-2018 Niklas Beisert
%
% This work may be distributed and/or modified under the
% conditions of the LaTeX Project Public License, either version 1.3
% of this license or (at your option) any later version.
% The latest version of this license is in
%   http://www.latex-project.org/lppl.txt
% and version 1.3 or later is part of all distributions of LaTeX
% version 2005/12/01 or later.
%
% This work has the LPPL maintenance status `maintained'.
%
% The Current Maintainer of this work is Niklas Beisert.
%
% This work consists of the files childdoc.dtx and childdoc.ins
% and the derived files childdoc.def and cdocsamp.tex with
% cdocsch1.tex, cdocsch2.tex, cdocsdrf.tex, cdocsfn1.tex, cdocsfn2.tex.
%
%<package>\ifdefined\childdocmain\endinput\fi
%<package>\ProvidesFile{childdoc.def}[2018/12/30 v2.0 child document driver]
%<samplemain>\ProvidesFile{cdocsamp.tex}[2018/12/30 v2.0 sample for childdoc]
%<*driver>
%\ProvidesFile{childdoc.drv}[2018/12/30 v2.0 childdoc reference manual file]
\PassOptionsToClass{10pt,a4paper}{article}
\documentclass{ltxdoc}

\usepackage[margin=35mm]{geometry}
\usepackage{hyperref}
\usepackage{hyperxmp}
\usepackage[usenames]{color}

\hypersetup{colorlinks=true}
\hypersetup{pdfstartview=FitH}
\hypersetup{pdfpagemode=UseNone}
\hypersetup{pdfsource={}}
\hypersetup{pdflang={en-UK}}
\hypersetup{pdfcopyright={Copyright 2017-2018 Niklas Beisert.
  This work may be distributed and/or modified under the
  conditions of the LaTeX Project Public License, either version 1.3
  of this license or (at your option) any later version.}}
\hypersetup{pdflicenseurl={http://www.latex-project.org/lppl.txt}}
\hypersetup{pdfcontactaddress={ETH Zurich, ITP, HIT K,
  Wolfgang-Pauli-Strasse 27}}
\hypersetup{pdfcontactpostcode={8093}}
\hypersetup{pdfcontactcity={Zurich}}
\hypersetup{pdfcontactcountry={Switzerland}}
\hypersetup{pdfcontactemail={nbeisert@itp.phys.ethz.ch}}
\hypersetup{pdfcontacturl={http://people.phys.ethz.ch/\xmptilde nbeisert/}}

\newcommand{\secref}[1]{\hyperref[#1]{section \ref*{#1}}}

\parskip1ex
\parindent0pt
\let\olditemize\itemize
\def\itemize{\olditemize\parskip0pt}

\begin{document}

\title{The \textsf{childdoc} Package}
\hypersetup{pdftitle={The childdoc Package}}
\author{Niklas Beisert\\[2ex]
  Institut f\"ur Theoretische Physik\\
  Eidgen\"ossische Technische Hochschule Z\"urich\\
  Wolfgang-Pauli-Strasse 27, 8093 Z\"urich, Switzerland\\[1ex]
  \href{mailto:nbeisert@itp.phys.ethz.ch}
  {\texttt{nbeisert@itp.phys.ethz.ch}}}
\hypersetup{pdfauthor={Niklas Beisert}}
\hypersetup{pdfsubject={Manual for the LaTeX2e Package childdoc}}
\date{30 December 2018, \textsf{v2.0}}
\maketitle

\begin{abstract}\noindent
\textsf{childdoc} is a \LaTeXe{} package
that enables the direct compilation
of document sections included by |\include|
to individual files.
\end{abstract}

\begingroup
\parskip0ex
\tableofcontents
\endgroup

%%%%%%%%%%%%%%%%%%%%%%%%%%%%%%%%%%%%%%%%%%%%%%%%%%%%%%%%%%%%%%%%%%%%%%%%%%%%%%%%
%%%%%%%%%%%%%%%%%%%%%%%%%%%%%%%%%%%%%%%%%%%%%%%%%%%%%%%%%%%%%%%%%%%%%%%%%%%%%%%%
\section{Introduction}

\LaTeX{} provides a mechanism to structure a large document (such as a book)
into a main file and several child files (containing the chapters)
using the |\include| command.
This mechanism is beneficial for documents
which span hundreds of pages in order to
make the source file(s) more manageable.
Moreover, compilation can be restricted to
selected child files by means of the |\includeonly| command.
The latter feature can be used to reduce the compilation time while editing
(this was significantly more useful in the earlier days of \LaTeX{})
or to generate a smaller document which is easier to navigate.
Another application of |\includeonly| is to generate
documents consisting of selected parts of the complete document.

However, there are a few drawbacks of the plain |\include| mechanism:
\begin{itemize}
\item
The child files cannot be compiled on their own,
they can only be compiled via the main file.
A naive editing environment
(such as a text editor with an option
to have the current file processed by \LaTeX)
may require one to switch to the main file before compiling;
attempting to compile the child file produces errors.
\item
The main file must be modified (each time)
to adjust the |\includeonly| command
to the present needs. This easily leaves the main file in a messy state.
\item
The generated document will always carry the filename
of the main document. This is inconvenient if
several child files are to be compiled and
to be kept for distribution.
\end{itemize}

The present package provides a simple interface
to make child files individually compilable by \LaTeX{}.
Compiling a child file then has the same effect as compiling
the main file with an |\includeonly| command
to select the appropriate child.
Moreover the generated document will carry the name of the child
rather than the main file.
This resolves all three above issues.

This feature is meant to make the editing of books,
thesis documents and lecture notes somewhat more convenient.
However, the package can also be used efficiently for
composing a series of documents (such as exercise sheets)
which are typically distributed individually.
It then assists the author in generating the individual documents
(potentially in different versions)
as well as a document containing the collected series.
Another application is in developing style files
or other kinds of included material
where compilation of the style file could redirect
to a sample or test file.

%%%%%%%%%%%%%%%%%%%%%%%%%%%%%%%%%%%%%%%%%%%%%%%%%%%%%%%%%%%%%%%%%%%%%%%%%%%%%%%%
%%%%%%%%%%%%%%%%%%%%%%%%%%%%%%%%%%%%%%%%%%%%%%%%%%%%%%%%%%%%%%%%%%%%%%%%%%%%%%%%
\section{Usage}

First of all, the package \textsf{childdoc} is \emph{not} a standard
\LaTeXe{} |.sty| style file! Therefore it needs to be invoked in
a non-standard way.

%%%%%%%%%%%%%%%%%%%%%%%%%%%%%%%%%%%%%%%%%%%%%%%%%%%%%%%%%%%%%%%%%%%%%%%%%%%%%%%%
\subsection{Included Files}
\label{sec:include}

%%%%%%%%%%%%%%%%%%%%%%%%%%%%%%%%%%%%%%%%
\DescribeMacro{\childdocmain}
To use the package, add the commands
\begin{center}
\begin{tabular}{l}
|\input{childdoc.def}|\\
|\childdocmain{}|\\
\end{tabular}
\end{center}
at the very top of the main \LaTeX{} file,
in particular \emph{before} the |\documentclass| statement!
The argument of |\childdocmain| should be left empty
(but it must be present).

%%%%%%%%%%%%%%%%%%%%%%%%%%%%%%%%%%%%%%%%
\DescribeMacro{\childdocof}
Furthermore, add the commands
\begin{center}
\begin{tabular}{l}
|\input{childdoc.def}|\\
|\childdocof{|\textit{main}|}|\\
\end{tabular}
\end{center}
at the top of every child file \textit{child}
which is included by |\include{|\textit{child}|}|
from within the main file
(or at least for those files to be compiled individually).
The argument \textit{main} must be the filename of the main file.

There are a couple of
considerations in setting up the main and child documents:

%%%%%%%%%%%%%%%%%%%%%%%%%%%%%%%%%%%%%%%%
\paragraph{Restrictions.}

Please note the following restrictions:
\begin{itemize}
\item
|\childdocmain| must be called with one argument \textit{main}
to ensure compatibility with earlier version of the package.
It must either be empty (|\childdocmain{}|)
or precisely match the filename of the main file in which it is specified.
See \secref{sec:detection} for further information.
\item
The filename \textit{main} must be specified without the |.tex| extension.
\item
The filename \textit{main} is case sensitive
(even in case-insensitive file systems)
due to internal string comparison.
\item
The argument \textit{main} should be fully expanded, it cannot be a macro.
\item
Subdirectories and special characters should be avoided in filenames.
\item
The command |\childdocmain{|\textit{main}|}| must be followed by a whitespace.
It should not be followed immediately by another command
or by a comment mark `|%|'.
This is because the \TeX{} parser reads the token immediately following
the argument of |\childdocmain| and puts it
at the beginning of every child section;
however, a white\-space is ignored.
\end{itemize}

%%%%%%%%%%%%%%%%%%%%%%%%%%%%%%%%%%%%%%%%
\paragraph{Content of Main File.}

It is advisable to place all content in the child files included by |\include|.
Any output contained in the main file will appear in all child documents
unless suppressed manually;
it cannot be suppressed automatically by the |\includeonly| directive
and thus should normally be avoided.
A method to include some content in the main file
by means of conditional processing is described in \secref{sec:conditional}.

%%%%%%%%%%%%%%%%%%%%%%%%%%%%%%%%%%%%%%%%
\paragraph{Page Numbering.}

When only a part of the document is compiled,
the appropriate numbering of pages
(as well as other status parameters)
is determined from the |.aux| files.
The latter contain information from previous passes.
However this information needs to propagate through
all intermediate child documents.
Therefore the page numbering in child documents may well
be inconsistent until the complete document is compiled at least once.

A useful (if unconventional) way to always ensure a consistent
page numbering is to restart the numbering in each child document
and denote the pages by `\textit{child}|.|\textit{page}'
where \textit{child} represents the chapter/section number of the child file.
This can be achieved by the command
|\numberwithin{page}{|\textit{child}|}|
of the \textsf{amsmath} package
where \textit{child} can be |chapter| or |section|
depending on the chosen structuring.
Alternatively, one can modify the macro |\thepage| appropriately
and reset the counter |page| at the start of each child file.

%%%%%%%%%%%%%%%%%%%%%%%%%%%%%%%%%%%%%%%%%%%%%%%%%%%%%%%%%%%%%%%%%%%%%%%%%%%%%%%%
\subsection{Conditional Processing}
\label{sec:conditional}

The package provides a mechanism to compile different versions
of a document. To customise the versions further some conditional processing
can come in handy to distinguish which version is being compiled.
The package provides two macros to describe the compilation context:

%%%%%%%%%%%%%%%%%%%%%%%%%%%%%%%%%%%%%%%%
\DescribeMacro{\ifchilddoc}
The conditional |\ifchilddoc| distinguishes between the compilation of
child documents and the main document:
%
\begin{center}
|\ifchilddoc |\textit{child-code}| |[|\||else |\textit{main-code}]| \||fi|
\end{center}

%%%%%%%%%%%%%%%%%%%%%%%%%%%%%%%%%%%%%%%%
\DescribeMacro{\childdocname}
\DescribeMacro{\childdocjob}
The macro |\childdocname| contains the filename (without extension)
of the main or child file being processed.
Note that |\childdocjob| will always contain the name of the main file.

%%%%%%%%%%%%%%%%%%%%%%%%%%%%%%%%%%%%%%%%
\paragraph{Title Page.}

Conditional processing can be used to include a title or banner page
in the main document when proper precautions are taken.
Importantly, the code in the main file should ensure that the page counter
(as well as other status parameters which are stored in the |.aux| files)
takes the same value after the conditional processing.
Otherwise the page numbers may take divergent values
depending on which part is compiled.

For example, a title page could be declared by:
%
\begin{center}
\begin{tabular}{l}
|\ifchilddoc\||else|\\
|\addtocounter{page}{-1}|\\
\textit{code for title page}\\
|\newpage|\\
|\||fi|
\end{tabular}
\end{center}
%
A banner page for the child documents can be generated by:
%
\begin{center}
\begin{tabular}{l}
|\ifchilddoc|\\
|\addtocounter{page}{-1}|\\
\textit{code for banner page}\\
|\newpage|\\
|\||fi|
\end{tabular}
\end{center}
%
Here one could write a message such as:
\begin{center}
|This is the part \childdocname{} of \childdocjob{}.|
\end{center}

%%%%%%%%%%%%%%%%%%%%%%%%%%%%%%%%%%%%%%%%%%%%%%%%%%%%%%%%%%%%%%%%%%%%%%%%%%%%%%%%
\subsection{Flags}
\label{sec:flags}

The package makes it easy to generate different versions
of the main or child documents.
To this end compilation flags can be defined
and assigned different default values.
They will be particularly useful in conjunction
with the forwarding mechanism described in \secref{sec:forward}.

For example, it may be useful to have a flag |\version|
which can be set to |draft| or |final|.
The document source will contain some conditional code
depending on the value of |\version|.
Suppose further, the flag should default to |final| for the main file
and to |draft| for child files
which is a natural assignment for editing the document.
This is achieved by placing the following code
in the preamble of the main document
(below the |\childdocmain| directive):
%
\begin{center}
\begin{tabular}{l}
|\ifchilddoc|\\
|\providecommand{\version}{draft}|\\
|\||else|\\
|\providecommand{\version}{final}|\\
|\||fi|
\end{tabular}
\end{center}
%
The definition by |\providecommand| makes sure
that previous definitions are not overwritten.
Further statements |\providecommand{\version}{...}|
can thus be added before the above code to override it.

For the main file, one might add a line
(between |\childdocmain| and the above block)
%
\begin{center}
|%\ifchilddoc\||else\providecommand{\version}{draft}\||fi|
\end{center}
%
which can be uncommented to produce a draft version.
Likewise one can add a line to the very top of a child file
(above the |\childdocof{|\textit{main}|}| directive)
%
\begin{center}
|%\providecommand{\version}{final}|
\end{center}
%
which can be uncommented to produce the final version of this child document.

%%%%%%%%%%%%%%%%%%%%%%%%%%%%%%%%%%%%%%%%%%%%%%%%%%%%%%%%%%%%%%%%%%%%%%%%%%%%%%%%
\subsection{Forwarding}
\label{sec:forward}

Different versions of the main or child documents
using compilation flags as described in \secref{sec:flags}
can be (permanently) stored in different files
for convenient compilation, viewing and distribution.
To this end, the package defines a command
to pass on compilation to a different file:

%%%%%%%%%%%%%%%%%%%%%%%%%%%%%%%%%%%%%%%%
\DescribeMacro{\childdocforward}
The command |\childdocforward| redirects processing to
another source file:
%
\begin{center}
\begin{tabular}{l}
|\input{childdoc.def}|\\
|\childdocforward[|\textit{main}|]{|\textit{dest}|}|\\
\end{tabular}
\end{center}
%
The argument \textit{dest} is the destination file
(without extension).
It should be the main file or one of the child files.
Note that further \textsf{childdoc} directives
such as |\childdocof| and |\childdocforward|
in the indicated file will be processed in this form.
The optional argument \textit{main}
passes on directly to the main file \textit{main}
while pretending to compile the child \textit{dest}.
This form behaves as if \textit{dest}
issues |\childdocof{|\textit{main}|}| right away,
and no further \textsf{childdoc} directives will be processed.

%%%%%%%%%%%%%%%%%%%%%%%%%%%%%%%%%%%%%%%%
\DescribeMacro{\...prefix}
In the alternative form |\childdocforwardprefix|,
%
\begin{center}
\begin{tabular}{l}
|\input{childdoc.def}|\\
|\childdocforwardprefix[|\textit{main}|]{|\textit{prefix}|}{|\textit{dest}|}|
\end{tabular}
\end{center}
%
the destination file is determined by a pattern
depending on the current file:
To make this work, the current file must be called
`{\textit{prefix}\hspace{0.2em}\textit{suffix}}'
with \textit{prefix} matching precisely the argument.
Processing is then passed on to the file
`{\textit{dest}\hspace{0.2em}\textit{suffix}}'.
Surely, the same effect is achieved by
directly specifying the
argument `{\textit{dest}\hspace{0.2em}\textit{suffix}}'
in the first form.
However, that requires to set up a different file
for each child. With the alternative form of the command
all these files can have exactly the same content
which simplifies setting them up and maintaining them.

For example, the following file |draft.tex|
with a compilation flag |\version| as described in \secref{sec:flags}
compiles the main document as a draft:
%
\begin{center}
\begin{tabular}{l}
|\def\version{draft}|\\
|\input{childdoc.def}|\\
|\childdocforward{|\textit{main}|}|
\end{tabular}
\end{center}
%
Likewise, the following files |final|\textit{nn}|.tex|
compile the final version of the child document
|child|\textit{nn}|.tex|:
%
\begin{center}
\begin{tabular}{l}
|\def\version{final}|\\
|\input{childdoc.def}|\\
|\childdocforwardprefix{final}{child}|
\end{tabular}
\end{center}
%

Note that when several versions of a main file and/or of each child file
are to be generated, it may be convenient to set up a |Makefile| or
shell script to automatise the process.

%%%%%%%%%%%%%%%%%%%%%%%%%%%%%%%%%%%%%%%%%%%%%%%%%%%%%%%%%%%%%%%%%%%%%%%%%%%%%%%%
\subsection{Command Line Processing}
\label{sec:commandline}

The effect of redirection files can also be achieved by invoking
the \LaTeX{} compiler with a more elaborate command line.
Most conveniently this should be done as part
of a shell script or a |Makefile|.

When using \textsf{childdoc} in the main file, the following
command lines effectively perform a redirection
(note that depending on the shell being used,
backslashes may have to be doubled: `|\|' $\to$ `|\\|'):
%
\begin{center}
|... -jobname "|\textit{target}|" |\\|"|[\textit{flags}]%
|\input{childdoc.def}\childdocforward[|\textit{main}|]{|\textit{dest}|}"|
\end{center}
%
Here \textit{target} is the name of the output file,
\textit{main} is the name of the main file
and \textit{dest} is the name of the main or child file to be processed
(all filenames without extensions).
The optional argument \textit{main} can be omitted
if \textit{main} matches \textit{dest}.
Optionally, compilation \textit{flags} can be defined via |\def| commands.
This command line makes the \TeX{} engine believe
it is compiling the file \textit{target}
whose content is specified as the latter parameter.
The provided code then forwards the processing to
\textit{main} or \textit{dest} as described in \secref{sec:forward}.

%%%%%%%%%%%%%%%%%%%%%%%%%%%%%%%%%%%%%%%%%%%%%%%%%%%%%%%%%%%%%%%%%%%%%%%%%%%%%%%%
\subsection{Include by Input}
\label{sec:input}

Including child documents by |\include| has some restrictions by design.
Most notably, the content of a child document always occupies
its own set of pages; pages cannot be shared between child documents.
Usually, this behaviour makes perfect sense
because each child document contain an essential part of the document.
However, in some situations it may be desirable to compose
a document from a collection of parts
without having mandatory page breaks between then.
For this case, the package
provides a mechanism to include parts
by |\input| which can also be processed individually.
However, by construction this mechanism
requires manual handling of the content to be output.

%%%%%%%%%%%%%%%%%%%%%%%%%%%%%%%%%%%%%%%%
\DescribeMacro{\ifchilddocmanual}
The main file should be prepared as usual, see \secref{sec:include}.
However, the document body must make a distinction
between processing of an individual part and of the main document, e.g.:
%
\begin{center}
\begin{tabular}{l}
|\ifchilddocmanual|\\
|\input{\childdocname}|\\
|\||else|\\
\textit{document body with }|\input{|\textit{part}|}|\\
|\||fi|
\end{tabular}
\end{center}
%
The conditional |\ifchilddocmanual| is true whenever
a part to be included by |\input| is being compiled,
and the name of the part is stored in |\childdocname|.

%%%%%%%%%%%%%%%%%%%%%%%%%%%%%%%%%%%%%%%%
\DescribeMacro{\childdocby}
Each part to be included by |\input| should start with:
%
\begin{center}
\begin{tabular}{l}
|\input{childdoc.def}|\\
|\childdocby{|\textit{main}|}|\\
\end{tabular}
\end{center}
%
The directive |\childdocby| is similar to |\childdocof|
described in \secref{sec:include},
but the subsequent selection of content must be done manually.
To that end, both |\ifchilddoc| and |\ifchilddocmanual|
will be true upon processing of a part,
and the name of the part is stored in |\childdocname|.
Note that |\jobname| will be set to the filename of the current part
so that each part receives an individual |.aux| file
that does not interfere with the |.aux| file(s) of the main document.
This behaviour can be altered by the alternative form
|\childdocby[*]{|\textit{main}|}| (with a non-empty optional argument)
which uses the |.aux| file of the main document
by setting |\jobname| to \textit{main}.

%%%%%%%%%%%%%%%%%%%%%%%%%%%%%%%%%%%%%%%%%%%%%%%%%%%%%%%%%%%%%%%%%%%%%%%%%%%%%%%%
\subsection{Driver Development}
\label{sec:driver}

The \textsf{childdoc} mechanism can also be use for the development
of definition files such as \LaTeX{} styles or classes.
This case differs from the above setup with multiple parts
included by |\include| in that no |\includeonly| should be invoked.
This can be achieved by starting the include file
(before |\ProvidesPackage|) with:
%
\begin{center}
\begin{tabular}{l}
|\input{childdoc.def}|\\
|\childdocforward{|\textit{main}|}|\\
\end{tabular}
\end{center}
%
or alternatively with:
%
\begin{center}
\begin{tabular}{l}
|\input{childdoc.def}|\\
|\childdocby{|\textit{main}|}|\\
\end{tabular}
\end{center}
%
Both forms have slightly different effects as described above.
The main file is prepared as usual, see \secref{sec:include}.

%%%%%%%%%%%%%%%%%%%%%%%%%%%%%%%%%%%%%%%%%%%%%%%%%%%%%%%%%%%%%%%%%%%%%%%%%%%%%%%%
\subsection{Legacy Detection}
\label{sec:detection}

The directive |\childdocmain| in the main file can detect
whether the complete document or merely a child is to be compiled
even without using the directive |\childdocof|.
This method is deprecated because it is less robust
and there is no compelling reason to use it;
it is merely provided for backward compatibility
and it may be removed in future versions.

If the detection mechanism is to be used,
it is mandatory to correctly specify
the filename of the main file as the argument of |\childdocmain|:
%
\begin{center}
\begin{tabular}{l}
|\input{childdoc.def}|\\
|\childdocmain{|\textit{main}|}|\\
\end{tabular}
\end{center}
%
If |\jobname| does not match the argument \textit{main} of |\childdocmain|,
it is assumed that |\jobname| points to the child file to be compiled.
When using |\childdocmain| with the main file specified as argument,
it suffices to start a child file
with just |\input{|\textit{main}|}|
without loading of the package and using |\childdocof|.
If instead all processing is done
with the appropriate \textsf{childdoc} directives,
the argument of \textit{main} of |\childdocmain| can be empty.

An alternative version of the command line processing described
in \secref{sec:commandline} using the detection mechanism reads:
%
\begin{center}
|... -jobname "|\textit{target}|" "|[\textit{flags}]%
[|\def\jobname{|\textit{dest}|}|]|\input{|\textit{main}|}"|
\end{center}

%%%%%%%%%%%%%%%%%%%%%%%%%%%%%%%%%%%%%%%%%%%%%%%%%%%%%%%%%%%%%%%%%%%%%%%%%%%%%%%%
\subsection{Manual Code}
\label{sec:manual}

In case one cannot be certain whether the definitions file |childdoc.def|
is installed on the target \TeX{} distribution
and one prefers not to ship it,
it is conceivable to paste a few relevant commands into the sources.

To that end, drop all statements |\input{childdoc.def}|
and perform the replacements as outlined below.
Instead of |\childdocmain{|\textit{main}|}| add the following code
to the top of the main file:
%
\begin{center}
\begin{tabular}{l}
|\||ifdefined\childdocname\endinput\||fi\newif\ifchilddoc|\\
|\edef\childdocname{\scantokens\expandafter{\jobname\noexpand}}|\\
|\def\childdocmain{|\textit{main}|}\||ifx\childdocmain\childdocname\||else|\\
|\childdoctrue\includeonly{\childdocname}\let\jobname\childdocmain\||fi|\\
\end{tabular}
\end{center}
%
Instead of |\childdocof{|\textit{main}|}| just include the main file
at the top of each child file:
%
\begin{center}
|\input{|\textit{main}|}|
\end{center}
%
A simple redirection |\childdocforward{|\textit{dest}|}| is achieved by:
%
\begin{center}
|\def\jobname{|\textit{dest}|}\input{\jobname}|
\end{center}
%
The redirection with prefix
|\childdocforwardprefix[|\textit{prefix}|]{|\textit{dest}|}|
is accomplished by:
%
\begin{center}
\begin{tabular}{l}
|{\edef\jobname{\scantokens\expandafter{\jobname\noexpand}}|\\
|\def\redirectjob |\textit{prefix}|#1~~~{\gdef\jobname{|\textit{dest}|#1}}|\\
|\expandafter\redirectjob\jobname~~~}\input{\jobname}|
\end{tabular}
\end{center}

In an alternative approach,
child documents can be compiled by a specific command line
without additional code or specific definitions:
%
\begin{center}
|... -jobname "|\textit{target}|" "|[\textit{flags}]%
|\includeonly{|\textit{dest}|}\input{|\textit{main}|}"|
\end{center}
%

%%%%%%%%%%%%%%%%%%%%%%%%%%%%%%%%%%%%%%%%%%%%%%%%%%%%%%%%%%%%%%%%%%%%%%%%%%%%%%%%
%%%%%%%%%%%%%%%%%%%%%%%%%%%%%%%%%%%%%%%%%%%%%%%%%%%%%%%%%%%%%%%%%%%%%%%%%%%%%%%%
\section{Information}

%%%%%%%%%%%%%%%%%%%%%%%%%%%%%%%%%%%%%%%%%%%%%%%%%%%%%%%%%%%%%%%%%%%%%%%%%%%%%%%%
\subsection{Copyright}

Copyright \copyright{} 2017--2018 Niklas Beisert

This work may be distributed and/or modified under the
conditions of the \LaTeX{} Project Public License, either version 1.3
of this license or (at your option) any later version.
The latest version of this license is in
  \url{http://www.latex-project.org/lppl.txt}
and version 1.3 or later is part of all distributions of \LaTeX{}
version 2005/12/01 or later.

This work has the LPPL maintenance status `maintained'.

The Current Maintainer of this work is Niklas Beisert.

This work consists of the files |README.txt|, |childdoc.ins| and |childdoc.dtx|
as well as the derived files |childdoc.def|, |cdocsamp.tex|
with |cdocsch1.tex|, |cdocsch2.tex|, |cdocspt3.tex|, |cdocspt4.tex|,
|cdocsdrf.tex|, |cdocsfn1.tex|, |cdocsfn2.tex|
as well as |childdoc.pdf|.

%%%%%%%%%%%%%%%%%%%%%%%%%%%%%%%%%%%%%%%%%%%%%%%%%%%%%%%%%%%%%%%%%%%%%%%%%%%%%%%%
\subsection{Files and Installation}

The package consists of the files:
%
\begin{center}
\begin{tabular}{ll}
    |README.txt|   & readme file \\
    |childdoc.ins| & installation file \\
    |childdoc.dtx| & source file \\
    |childdoc.def| & definition file \\
    |cdocsamp.tex| & sample main file \\
    |cdocsch1.tex| & sample include file \\
    |cdocsch2.tex| & sample include file \\
    |cdocspt3.tex| & sample part file \\
    |cdocspt4.tex| & sample part file \\
    |cdocsdrf.tex| & sample redirection file \\
    |cdocsfn1.tex| & sample redirection file \\
    |cdocsfn2.tex| & sample redirection file \\
    |childdoc.pdf| & manual
\end{tabular}
\end{center}
%
The distribution consists of the files
|README.txt|, |childdoc.ins| and |childdoc.dtx|.
%
\begin{itemize}
\item
Run (pdf)\LaTeX{} on |childdoc.dtx|
to compile the manual |childdoc.pdf| (this file).
\item
Run \LaTeX{} on |childdoc.ins| to create the definitions file |childdoc.def|
and the sample |cdocsamp.tex| with include files
|cdocsch1.tex|, |cdocsch2.tex|, |cdocspt3.tex|, |cdocspt4.tex|,
|cdocsdrf.tex|, |cdocsfn1.tex|, |cdocsfn2.tex|.
Then copy the file |childdoc.def| to an appropriate directory of your \LaTeX{}
distribution, e.g.\ \textit{texmf-root}|/tex/latex/childdoc|.
\end{itemize}

%%%%%%%%%%%%%%%%%%%%%%%%%%%%%%%%%%%%%%%%%%%%%%%%%%%%%%%%%%%%%%%%%%%%%%%%%%%%%%%%
\subsection{Related CTAN Packages}

There are several other packages which offer a similar functionality:
%
\begin{itemize}
\item
The packages
\href{http://ctan.org/pkg/docmute}{\textsf{docmute}},
\href{http://ctan.org/pkg/includex}{\textsf{includex}} and
\href{http://ctan.org/pkg/standalone}{\textsf{standalone}}
provide commands to include only the document body of
a child file thus allowing both files to be compiled individually.
\item
The packages \href{http://ctan.org/pkg/subdocs}{\textsf{subdocs}}
and \href{http://ctan.org/pkg/subfiles}{\textsf{subfiles}}
provide structures in which the main and child documents can be
encapsulated and allowing them to be compiled individually.
The inclusion mechanism is different from the conventional |\include|.
\item
The package \href{http://ctan.org/pkg/combine}{\textsf{combine}}
is an elaborate solution to combine several documents into one.
\end{itemize}
%
See also the CTAN topic \href{http://ctan.org/topic/subdocs}{\textsf{subdocs}}
for further related packages.
The present package differs from the above solutions in that
a document structure constructed with the conventional |\include| mechanism
just needs two extra commands at the top of every file
such that all constituent files can be compiled individually.

%%%%%%%%%%%%%%%%%%%%%%%%%%%%%%%%%%%%%%%%%%%%%%%%%%%%%%%%%%%%%%%%%%%%%%%%%%%%%%%%
%\subsection{Feature Suggestions}
%
%The following is a list of features which may be useful for future
%versions of this package:
%%
%\begin{itemize}
%\item
%\ldots
%\end{itemize}

%%%%%%%%%%%%%%%%%%%%%%%%%%%%%%%%%%%%%%%%%%%%%%%%%%%%%%%%%%%%%%%%%%%%%%%%%%%%%%%%
\subsection{Revision History}

%%%%%%%%%%%%%%%%%%%%%%%%%%%%%%%%%%%%%%%%
\paragraph{v2.0:} 2018/12/30

\begin{itemize}
\item
immediate forward processing
\item
added |\childdocby| mechanism
\item
manual restructured
\end{itemize}

%%%%%%%%%%%%%%%%%%%%%%%%%%%%%%%%%%%%%%%%
\paragraph{v1.6:} 2018/01/17

\begin{itemize}
\item
application for development of include files
\item
corrections to manual
\end{itemize}

%%%%%%%%%%%%%%%%%%%%%%%%%%%%%%%%%%%%%%%%
\paragraph{v1.5:} 2017/05/21

\begin{itemize}
\item
more complete structuring introduced
\item
|\childdocof| introduced
\item
|\childdoc| renamed to |\childdocmain|
\item
|\childredirect| renamed to |\childdocforward| and |\childdocforwardprefix|
and functionality expanded
\end{itemize}

%%%%%%%%%%%%%%%%%%%%%%%%%%%%%%%%%%%%%%%%
\paragraph{v1.0:} 2017/04/27

\begin{itemize}
\item
manual and install package
\item
first version published on CTAN
\end{itemize}

%%%%%%%%%%%%%%%%%%%%%%%%%%%%%%%%%%%%%%%%
\paragraph{v0.6:} 2017/04/26

\begin{itemize}
\item
redirection mechanism added
\end{itemize}

%%%%%%%%%%%%%%%%%%%%%%%%%%%%%%%%%%%%%%%%
\paragraph{v0.5:} 2017/04/26

\begin{itemize}
\item
functionality in definition file
\end{itemize}


%%%%%%%%%%%%%%%%%%%%%%%%%%%%%%%%%%%%%%%%%%%%%%%%%%%%%%%%%%%%%%%%%%%%%%%%%%%%%%%%
%%%%%%%%%%%%%%%%%%%%%%%%%%%%%%%%%%%%%%%%%%%%%%%%%%%%%%%%%%%%%%%%%%%%%%%%%%%%%%%%
%%%%%%%%%%%%%%%%%%%%%%%%%%%%%%%%%%%%%%%%%%%%%%%%%%%%%%%%%%%%%%%%%%%%%%%%%%%%%%%%
\appendix

\settowidth\MacroIndent{\rmfamily\scriptsize 000\ }

 \DocInput{childdoc.dtx}

\end{document}
%</driver>
% \fi
%
% %%%%%%%%%%%%%%%%%%%%%%%%%%%%%%%%%%%%%%%%%%%%%%%%%%%%%%%%%%%%%%%%%%%%%%%%%%%%%%
% %%%%%%%%%%%%%%%%%%%%%%%%%%%%%%%%%%%%%%%%%%%%%%%%%%%%%%%%%%%%%%%%%%%%%%%%%%%%%%
% \section{Sample}
%\iffalse
%<*samplemain>
%\fi
%
% The following presents a sample document
% with two chapters, two parts, a title page,
% a compile flag as well as three forwarding files to set the flag.
% It consists of eight |.tex| files:
% \begin{center}
% \begin{tabular}{ll}
% |cdocsamp.tex|&main file\\
% |cdocsch1.tex|&include file for chapter 1\\
% |cdocsch2.tex|&include file for chapter 2\\
% |cdocspt3.tex|&include file for part 3\\
% |cdocspt4.tex|&include file for part 4\\
% |cdocsdrf.tex|&forwarding file for main file in draft mode\\
% |cdocsfi1.tex|&forwarding file for final version of chapter 1\\
% |cdocsfi2.tex|&forwarding file for final version of chapter 2\\
% \end{tabular}
% \end{center}
% Each of the eight files can be compiled directly by the \LaTeX{} compiler.
%
% %%%%%%%%%%%%%%%%%%%%%%%%%%%%%%%%%%%%%%
% \paragraph{Main File.}
%
% The main file is called |cdocsamp.tex|.
%
% Load the \textsf{childdoc} definitions and
% declare the filename for the main document:
%    \begin{macrocode}
\input{childdoc.def}
\childdocmain{}
%    \end{macrocode}

% Optional override for |\version| flag:
%    \begin{macrocode}
%%\ifchilddoc\else\providecommand{\version}{draft}\fi
%    \end{macrocode}

% Define the default values for the |\version| flag
% (|final| for the main file and |draft| for childs):
%    \begin{macrocode}
\ifchilddoc
\providecommand{\version}{draft}
\else
\providecommand{\version}{final}
\fi
%    \end{macrocode}

% Load the standard document class:
%    \begin{macrocode}
\documentclass[12pt]{article}
%    \end{macrocode}

% Start the document body:
%    \begin{macrocode}
\begin{document}
%    \end{macrocode}

% Declare a title page.
% Print title, part of document being processed and version flag:
%    \begin{macrocode}
\addtocounter{page}{-1}
\begin{center}
{\LARGE\bfseries{}childdoc example\par}
\vspace{1cm}
\ifchilddoc
\ifchilddocmanual part\else chapter\fi:
`\childdocname' of `\childdocjob'\par
\else
main document: `\childdocjob'\par
\fi
version: \version\par
\end{center}
\newpage
%    \end{macrocode}

% Manually include selected file,
% otherwise process as usual:
%    \begin{macrocode}
\ifchilddocmanual
\section*{part `\childdocname'}
\input{\childdocname}
\else
%    \end{macrocode}

% Include the two chapters:
%    \begin{macrocode}
\include{cdocsch1}
\include{cdocsch2}
%    \end{macrocode}

% Include the two parts unless only chapters should be displayed:
%    \begin{macrocode}
\ifchilddoc\else
\section{part three}
\input{cdocspt3}
\section{part four}
\input{cdocspt4}
\fi
%    \end{macrocode}

% Process as usual until here:
%    \begin{macrocode}
\fi
%    \end{macrocode}

% End of document body:
%    \begin{macrocode}
\end{document}
%    \end{macrocode}
%\iffalse
%</samplemain>
%\fi
%
% %%%%%%%%%%%%%%%%%%%%%%%%%%%%%%%%%%%%%%
% \paragraph{Chapter Include Files.}
%
% The include files are called |cdocsch1.tex| and |cdocsch2.tex|.
%
%\iffalse
%<*samplechap1|samplechap2>
%\fi

% Optional override for |\version| flag:
%    \begin{macrocode}
%%\providecommand{\version}{final}
%    \end{macrocode}

% Include the main document:
%    \begin{macrocode}
\input{childdoc.def}
\childdocof{cdocsamp}
%    \end{macrocode}

%\iffalse
%</samplechap1|samplechap2>
%\fi
%
%\iffalse
%<*samplechap1>
%\fi
% Some text for chapter 1:
%    \begin{macrocode}
\section{one}
some text in chapter one
%    \end{macrocode}

%\iffalse
%</samplechap1>
%\fi
% Some text for chapter 2:
%\iffalse
%<*samplechap2>
%\fi
%    \begin{macrocode}
\section{two}
more text in chapter two
%    \end{macrocode}

%\iffalse
%</samplechap2>
%\fi
%
% %%%%%%%%%%%%%%%%%%%%%%%%%%%%%%%%%%%%%%
% \paragraph{Part Include Files.}
%
% The include files are called |cdocspt3.tex| and |cdocspt4.tex|.
%
%\iffalse
%<*samplepart3|samplepart4>
%\fi

% Optional override for |\version| flag:
%    \begin{macrocode}
%%\providecommand{\version}{final}
%    \end{macrocode}

% Include the main document:
%    \begin{macrocode}
\input{childdoc.def}
\childdocby{cdocsamp}
%    \end{macrocode}

%\iffalse
%</samplepart3|samplepart4>
%\fi
%
%\iffalse
%<*samplepart3>
%\fi
% Some text for part 3:
%    \begin{macrocode}
some text in part three
%    \end{macrocode}

%\iffalse
%</samplepart3>
%\fi
% Some text for part 4:
%\iffalse
%<*samplepart4>
%\fi
%    \begin{macrocode}
more text in part four
%    \end{macrocode}

%\iffalse
%</samplepart4>
%\fi
%
% %%%%%%%%%%%%%%%%%%%%%%%%%%%%%%%%%%%%%%
% \paragraph{Forwarding for a Complete Draft.}
%
% The following forwarding file |cdocsdrf.tex|
% compiles the main document in draft mode:
%\iffalse
%<*sampledraft>
%\fi
%    \begin{macrocode}
\def\version{draft}
\input{childdoc.def}
\childdocforward{cdocsamp}
%    \end{macrocode}

%\iffalse
%</sampledraft>
%\fi
%
% %%%%%%%%%%%%%%%%%%%%%%%%%%%%%%%%%%%%%%
% \paragraph{Forwarding for Final Version of the Chapters.}
%
% The following forwarding files |cdocsfn1.tex| and |cdocsfn2.tex|
% (with identical content)
% compile the final versions of the child documents
% |cdocsch1.tex| and |cdocsch2.tex|, respectively:
%\iffalse
%<*samplefinal>
%\fi
%    \begin{macrocode}
\def\version{final}
\input{childdoc.def}
\childdocforwardprefix[cdocsamp]{cdocsfn}{cdocsch}
%    \end{macrocode}

%\iffalse
%</samplefinal>
%\fi
%
% %%%%%%%%%%%%%%%%%%%%%%%%%%%%%%%%%%%%%%
% \paragraph{Command Line Processing.}
%
% The following three command lines generate the output files
% |cdocscld|, |cdocscl1| and |cdocscl2|
% which should be identical to
% |cdocsdrf|, |cdocsch1| and |cdocsfn2|, respectively:
% \begin{center}
% \begin{tabular}{l}
% |latex -jobname cdocscld \|\\
% |  "\def\version{draft}\input{childdoc.def}\childdocforward{cdocsamp}"|\\
% |latex -jobname cdocscl1 \|\\
% |  "\input{childdoc.def}\childdocforward[cdocsamp]{cdocsch1}"|\\
% |latex -jobname cdocscl2 \|\\
% |  "\def\version{final}\input{childdoc.def}\childdocforward{cdocsch2}"|
% \end{tabular}
% \end{center}
% Note that the trailing backslash on each first line
% merely continues the input to the second line
% (for convenient cut ant paste).
% Furthermore, the command |latex| can be replaced by any
% of its alternative versions such as |pdflatex|.
%
% %%%%%%%%%%%%%%%%%%%%%%%%%%%%%%%%%%%%%%%%%%%%%%%%%%%%%%%%%%%%%%%%%%%%%%%%%%%%%%
% %%%%%%%%%%%%%%%%%%%%%%%%%%%%%%%%%%%%%%%%%%%%%%%%%%%%%%%%%%%%%%%%%%%%%%%%%%%%%%
% \section{Implementation}
%\iffalse
%<*package>
%\fi
%
% This section describes the definitions file |childdoc.def|.

% The definitions cannot be loaded using |\usepackage| or |\RequirePackage|
% which has a mechanism to prevent loading a style file more than once.
% When loading the definitions by means of |\input|
% multiple instances have to be prevented manually:
%\iffalse
%This code needs to be before the `\ProvidesFile' directive
%which is defined at the beginning of this file.
%Therefore it is also placed there and commented out here.
%</package>
%<*discard>
%\fi
%    \begin{macrocode}
\ifdefined\childdocmain\endinput\fi
%    \end{macrocode}
%\iffalse
%</discard>
%<*package>
%\fi
%
% \macro{\ifchilddoc}
% \macro{\ifchilddocmanual}
% The conditional |\ifchilddoc| tells whether a
% child (true) or main (false) document is being compiled.
% The conditional |\ifchilddocmanual| tells whether
% the |\includeonly| mechanism is used (false) or
% the selection of child files must be performed manually (true).
% The definitions initialise to false:
%    \begin{macrocode}
\newif\ifchilddoc
\newif\ifchilddocmanual
%    \end{macrocode}

% \macro{\childdocname}
% \macro{\childdocjob}
% The macro |\childdocname| stores the name of the main document
% to be compiled. The macro |\childdocjob| stores the name of
% the document on which the \LaTeX{} compiler was originally invoked.
% The content of |\jobname| cannot be compared
% to filenames specified in the source due to different catcodes.
% The following code rescans |\jobname|, stores the result
% in |\childdocname| and saves a copy in |\childdocjob|:
%    \begin{macrocode}
\edef\childdocname{\scantokens\expandafter{\jobname\noexpand}}
\let\childdocjob\childdocname
%    \end{macrocode}

% \macro{\childdocdisable}
% The macro |\childdocdisable| prevents the main file
% from being processed more than once.
% At this stage, the main document command |\childdocmain|
% is assumed to be called once again where it should do nothing.
% Any subsequent call to it should prevent
% a secondary processing of the main document
% It overwrites the forwarding commands
% |\childdocof| and |\childdocforward|
% with empty macros to prevent further inclusions of the main document:
%    \begin{macrocode}
\newcommand{\childdocdisable}
{
  \renewcommand{\childdocmain}[1]{\renewcommand{\childdocmain}[1]{\endinput}}
  \renewcommand{\childdocof}[1]{}
  \renewcommand{\childdocby}[2][]{}
  \renewcommand{\childdocforward}[2][]{}
  \renewcommand{\childdocdisable}{}
}
%    \end{macrocode}

% \macro{\childdocmain}
% The macro |\childdocmain| is to be called at the top of the main file
% with nothing or the main filename (without extension) as argument.
% First, it breaks loops.
% If the argument is not empty and does not match |\childdocname|
% (which is set by the first inclusion of |childdoc.def|),
% |\ifchilddoc| is set to true, |\includeonly| is applied to the child file
% and |\jobname| is set to the main file
% (for proper handling of |.aux| files):
%    \begin{macrocode}
\newcommand{\childdocmain}[1]
{
  \childdocdisable\childdocmain{}
  \if?#1?\else
    \begingroup
      \def\childdoctmp{#1}
      \ifx\childdoctmp\childdocname
        \def\childdoctmp{}
      \else
        \def\childdoctmp
        {
          \childdoctrue
          \includeonly{\childdocname}
          \def\childdocjob{#1}
          \def\jobname{#1}
        }
      \fi
      \expandafter
    \endgroup
    \childdoctmp
  \fi
}
%    \end{macrocode}

% \macro{\childdocof}
% The command |\childdocof| redirects
% compilation to the main file |#1|.
%    \begin{macrocode}
\newcommand{\childdocof}[1]
{
  \childdocdisable
  \childdoctrue
  \includeonly{\childdocname}
  \def\jobname{#1}
  \def\childdocjob{#1}
  \input{#1}
}
%    \end{macrocode}

% \macro{\childdocby}
% The command |\childdocby| ....
%    \begin{macrocode}
\newcommand{\childdocby}[2][]
{
  \childdocdisable
  \childdoctrue
  \childdocmanualtrue
  \if?#1?\else
    \def\jobname{#2}
  \fi
  \def\childdocjob{#2}
  \input{#2}
  \endinput
}
%    \end{macrocode}

% \macro{\childdocforward}
% The command |\childdocforward| redirects
% compilation to the main file or
% (if the optional argument is given) a child file.
% Parameters are set as if the main file
% or a child file starting with |\childdocof| was compiled.
% Then compilation is handed over to the main file:
%    \begin{macrocode}
\newcommand{\childdocforward}[2][]
{
  \begingroup
    \if?#1?
      \def\childdoctmp
      {
        \def\childdocname{#2}
        \def\childdocjob{#2}
        \def\jobname{#2}
        \input{#2}
        \endinput
      }
    \else
      \def\childdoctmp
      {
        \childdocdisable
        \def\childdocname{#2}
        \childdoctrue
        \includeonly{#2}
        \def\childdocjob{#1}
        \def\jobname{#1}
        \input{#1}
        \endinput
      }
    \fi
    \expandafter
  \endgroup
  \childdoctmp
}
%    \end{macrocode}

% \macro{\childdocforwardprefix}
% The command |\childdocforwardprefix| redirects
% compilation to the main or a child file by means of a pattern.
% The prefix |#1| in the current filename is replaced by |#2|
% and the suffix of the current filename is kept
% (it is assumed that the filename does not contain the substring `|~~~|'
% which is used as a delimiter).
% Compilation is handed over to the new file by |\childdocforward|:
%    \begin{macrocode}
\newcommand{\childdocforwardprefix}[3][]
{
  \begingroup
    \def\childdocextract #2##1~~~{\def\childdoctmp{\childdocforward[#1]{#3##1}}}
    \expandafter\childdocextract\childdocname~~~
    \expandafter
  \endgroup
  \childdoctmp
}
%    \end{macrocode}

% \macro{\childdoc}
% The deprecated macro |\childdoc| is a legacy version of |\childdocmain|:
%    \begin{macrocode}
\newcommand{\childdoc}{\childdocmain}
%    \end{macrocode}

% \macro{\childdocredirect}
% The deprecated macro |\childdocredirect| is a legacy version
% of |\childdocforward| and |\childdocforwardprefix|:
%    \begin{macrocode}
\newcommand{\childdocredirect}[2][]
{
  \begingroup
    \if?#1?
      \def\childdoctmp{\childdocforward{#2}}
    \else
      \def\childdoctmp{\childdocforwardprefix{#1}{#2}}
    \fi
    \expandafter
  \endgroup
  \childdoctmp
}
%    \end{macrocode}

%\iffalse
%</package>
%\fi
%
\endinput
|\\
|\childdocforward{|\textit{main}|}|
\end{tabular}
\end{center}
%
Likewise, the following files |final|\textit{nn}|.tex|
compile the final version of the child document
|child|\textit{nn}|.tex|:
%
\begin{center}
\begin{tabular}{l}
|\def\version{final}|\\
|% \iffalse
%
% childdoc.dtx Copyright (C) 2017-2018 Niklas Beisert
%
% This work may be distributed and/or modified under the
% conditions of the LaTeX Project Public License, either version 1.3
% of this license or (at your option) any later version.
% The latest version of this license is in
%   http://www.latex-project.org/lppl.txt
% and version 1.3 or later is part of all distributions of LaTeX
% version 2005/12/01 or later.
%
% This work has the LPPL maintenance status `maintained'.
%
% The Current Maintainer of this work is Niklas Beisert.
%
% This work consists of the files childdoc.dtx and childdoc.ins
% and the derived files childdoc.def and cdocsamp.tex with
% cdocsch1.tex, cdocsch2.tex, cdocsdrf.tex, cdocsfn1.tex, cdocsfn2.tex.
%
%<package>\ifdefined\childdocmain\endinput\fi
%<package>\ProvidesFile{childdoc.def}[2018/12/30 v2.0 child document driver]
%<samplemain>\ProvidesFile{cdocsamp.tex}[2018/12/30 v2.0 sample for childdoc]
%<*driver>
%\ProvidesFile{childdoc.drv}[2018/12/30 v2.0 childdoc reference manual file]
\PassOptionsToClass{10pt,a4paper}{article}
\documentclass{ltxdoc}

\usepackage[margin=35mm]{geometry}
\usepackage{hyperref}
\usepackage{hyperxmp}
\usepackage[usenames]{color}

\hypersetup{colorlinks=true}
\hypersetup{pdfstartview=FitH}
\hypersetup{pdfpagemode=UseNone}
\hypersetup{pdfsource={}}
\hypersetup{pdflang={en-UK}}
\hypersetup{pdfcopyright={Copyright 2017-2018 Niklas Beisert.
  This work may be distributed and/or modified under the
  conditions of the LaTeX Project Public License, either version 1.3
  of this license or (at your option) any later version.}}
\hypersetup{pdflicenseurl={http://www.latex-project.org/lppl.txt}}
\hypersetup{pdfcontactaddress={ETH Zurich, ITP, HIT K,
  Wolfgang-Pauli-Strasse 27}}
\hypersetup{pdfcontactpostcode={8093}}
\hypersetup{pdfcontactcity={Zurich}}
\hypersetup{pdfcontactcountry={Switzerland}}
\hypersetup{pdfcontactemail={nbeisert@itp.phys.ethz.ch}}
\hypersetup{pdfcontacturl={http://people.phys.ethz.ch/\xmptilde nbeisert/}}

\newcommand{\secref}[1]{\hyperref[#1]{section \ref*{#1}}}

\parskip1ex
\parindent0pt
\let\olditemize\itemize
\def\itemize{\olditemize\parskip0pt}

\begin{document}

\title{The \textsf{childdoc} Package}
\hypersetup{pdftitle={The childdoc Package}}
\author{Niklas Beisert\\[2ex]
  Institut f\"ur Theoretische Physik\\
  Eidgen\"ossische Technische Hochschule Z\"urich\\
  Wolfgang-Pauli-Strasse 27, 8093 Z\"urich, Switzerland\\[1ex]
  \href{mailto:nbeisert@itp.phys.ethz.ch}
  {\texttt{nbeisert@itp.phys.ethz.ch}}}
\hypersetup{pdfauthor={Niklas Beisert}}
\hypersetup{pdfsubject={Manual for the LaTeX2e Package childdoc}}
\date{30 December 2018, \textsf{v2.0}}
\maketitle

\begin{abstract}\noindent
\textsf{childdoc} is a \LaTeXe{} package
that enables the direct compilation
of document sections included by |\include|
to individual files.
\end{abstract}

\begingroup
\parskip0ex
\tableofcontents
\endgroup

%%%%%%%%%%%%%%%%%%%%%%%%%%%%%%%%%%%%%%%%%%%%%%%%%%%%%%%%%%%%%%%%%%%%%%%%%%%%%%%%
%%%%%%%%%%%%%%%%%%%%%%%%%%%%%%%%%%%%%%%%%%%%%%%%%%%%%%%%%%%%%%%%%%%%%%%%%%%%%%%%
\section{Introduction}

\LaTeX{} provides a mechanism to structure a large document (such as a book)
into a main file and several child files (containing the chapters)
using the |\include| command.
This mechanism is beneficial for documents
which span hundreds of pages in order to
make the source file(s) more manageable.
Moreover, compilation can be restricted to
selected child files by means of the |\includeonly| command.
The latter feature can be used to reduce the compilation time while editing
(this was significantly more useful in the earlier days of \LaTeX{})
or to generate a smaller document which is easier to navigate.
Another application of |\includeonly| is to generate
documents consisting of selected parts of the complete document.

However, there are a few drawbacks of the plain |\include| mechanism:
\begin{itemize}
\item
The child files cannot be compiled on their own,
they can only be compiled via the main file.
A naive editing environment
(such as a text editor with an option
to have the current file processed by \LaTeX)
may require one to switch to the main file before compiling;
attempting to compile the child file produces errors.
\item
The main file must be modified (each time)
to adjust the |\includeonly| command
to the present needs. This easily leaves the main file in a messy state.
\item
The generated document will always carry the filename
of the main document. This is inconvenient if
several child files are to be compiled and
to be kept for distribution.
\end{itemize}

The present package provides a simple interface
to make child files individually compilable by \LaTeX{}.
Compiling a child file then has the same effect as compiling
the main file with an |\includeonly| command
to select the appropriate child.
Moreover the generated document will carry the name of the child
rather than the main file.
This resolves all three above issues.

This feature is meant to make the editing of books,
thesis documents and lecture notes somewhat more convenient.
However, the package can also be used efficiently for
composing a series of documents (such as exercise sheets)
which are typically distributed individually.
It then assists the author in generating the individual documents
(potentially in different versions)
as well as a document containing the collected series.
Another application is in developing style files
or other kinds of included material
where compilation of the style file could redirect
to a sample or test file.

%%%%%%%%%%%%%%%%%%%%%%%%%%%%%%%%%%%%%%%%%%%%%%%%%%%%%%%%%%%%%%%%%%%%%%%%%%%%%%%%
%%%%%%%%%%%%%%%%%%%%%%%%%%%%%%%%%%%%%%%%%%%%%%%%%%%%%%%%%%%%%%%%%%%%%%%%%%%%%%%%
\section{Usage}

First of all, the package \textsf{childdoc} is \emph{not} a standard
\LaTeXe{} |.sty| style file! Therefore it needs to be invoked in
a non-standard way.

%%%%%%%%%%%%%%%%%%%%%%%%%%%%%%%%%%%%%%%%%%%%%%%%%%%%%%%%%%%%%%%%%%%%%%%%%%%%%%%%
\subsection{Included Files}
\label{sec:include}

%%%%%%%%%%%%%%%%%%%%%%%%%%%%%%%%%%%%%%%%
\DescribeMacro{\childdocmain}
To use the package, add the commands
\begin{center}
\begin{tabular}{l}
|\input{childdoc.def}|\\
|\childdocmain{}|\\
\end{tabular}
\end{center}
at the very top of the main \LaTeX{} file,
in particular \emph{before} the |\documentclass| statement!
The argument of |\childdocmain| should be left empty
(but it must be present).

%%%%%%%%%%%%%%%%%%%%%%%%%%%%%%%%%%%%%%%%
\DescribeMacro{\childdocof}
Furthermore, add the commands
\begin{center}
\begin{tabular}{l}
|\input{childdoc.def}|\\
|\childdocof{|\textit{main}|}|\\
\end{tabular}
\end{center}
at the top of every child file \textit{child}
which is included by |\include{|\textit{child}|}|
from within the main file
(or at least for those files to be compiled individually).
The argument \textit{main} must be the filename of the main file.

There are a couple of
considerations in setting up the main and child documents:

%%%%%%%%%%%%%%%%%%%%%%%%%%%%%%%%%%%%%%%%
\paragraph{Restrictions.}

Please note the following restrictions:
\begin{itemize}
\item
|\childdocmain| must be called with one argument \textit{main}
to ensure compatibility with earlier version of the package.
It must either be empty (|\childdocmain{}|)
or precisely match the filename of the main file in which it is specified.
See \secref{sec:detection} for further information.
\item
The filename \textit{main} must be specified without the |.tex| extension.
\item
The filename \textit{main} is case sensitive
(even in case-insensitive file systems)
due to internal string comparison.
\item
The argument \textit{main} should be fully expanded, it cannot be a macro.
\item
Subdirectories and special characters should be avoided in filenames.
\item
The command |\childdocmain{|\textit{main}|}| must be followed by a whitespace.
It should not be followed immediately by another command
or by a comment mark `|%|'.
This is because the \TeX{} parser reads the token immediately following
the argument of |\childdocmain| and puts it
at the beginning of every child section;
however, a white\-space is ignored.
\end{itemize}

%%%%%%%%%%%%%%%%%%%%%%%%%%%%%%%%%%%%%%%%
\paragraph{Content of Main File.}

It is advisable to place all content in the child files included by |\include|.
Any output contained in the main file will appear in all child documents
unless suppressed manually;
it cannot be suppressed automatically by the |\includeonly| directive
and thus should normally be avoided.
A method to include some content in the main file
by means of conditional processing is described in \secref{sec:conditional}.

%%%%%%%%%%%%%%%%%%%%%%%%%%%%%%%%%%%%%%%%
\paragraph{Page Numbering.}

When only a part of the document is compiled,
the appropriate numbering of pages
(as well as other status parameters)
is determined from the |.aux| files.
The latter contain information from previous passes.
However this information needs to propagate through
all intermediate child documents.
Therefore the page numbering in child documents may well
be inconsistent until the complete document is compiled at least once.

A useful (if unconventional) way to always ensure a consistent
page numbering is to restart the numbering in each child document
and denote the pages by `\textit{child}|.|\textit{page}'
where \textit{child} represents the chapter/section number of the child file.
This can be achieved by the command
|\numberwithin{page}{|\textit{child}|}|
of the \textsf{amsmath} package
where \textit{child} can be |chapter| or |section|
depending on the chosen structuring.
Alternatively, one can modify the macro |\thepage| appropriately
and reset the counter |page| at the start of each child file.

%%%%%%%%%%%%%%%%%%%%%%%%%%%%%%%%%%%%%%%%%%%%%%%%%%%%%%%%%%%%%%%%%%%%%%%%%%%%%%%%
\subsection{Conditional Processing}
\label{sec:conditional}

The package provides a mechanism to compile different versions
of a document. To customise the versions further some conditional processing
can come in handy to distinguish which version is being compiled.
The package provides two macros to describe the compilation context:

%%%%%%%%%%%%%%%%%%%%%%%%%%%%%%%%%%%%%%%%
\DescribeMacro{\ifchilddoc}
The conditional |\ifchilddoc| distinguishes between the compilation of
child documents and the main document:
%
\begin{center}
|\ifchilddoc |\textit{child-code}| |[|\||else |\textit{main-code}]| \||fi|
\end{center}

%%%%%%%%%%%%%%%%%%%%%%%%%%%%%%%%%%%%%%%%
\DescribeMacro{\childdocname}
\DescribeMacro{\childdocjob}
The macro |\childdocname| contains the filename (without extension)
of the main or child file being processed.
Note that |\childdocjob| will always contain the name of the main file.

%%%%%%%%%%%%%%%%%%%%%%%%%%%%%%%%%%%%%%%%
\paragraph{Title Page.}

Conditional processing can be used to include a title or banner page
in the main document when proper precautions are taken.
Importantly, the code in the main file should ensure that the page counter
(as well as other status parameters which are stored in the |.aux| files)
takes the same value after the conditional processing.
Otherwise the page numbers may take divergent values
depending on which part is compiled.

For example, a title page could be declared by:
%
\begin{center}
\begin{tabular}{l}
|\ifchilddoc\||else|\\
|\addtocounter{page}{-1}|\\
\textit{code for title page}\\
|\newpage|\\
|\||fi|
\end{tabular}
\end{center}
%
A banner page for the child documents can be generated by:
%
\begin{center}
\begin{tabular}{l}
|\ifchilddoc|\\
|\addtocounter{page}{-1}|\\
\textit{code for banner page}\\
|\newpage|\\
|\||fi|
\end{tabular}
\end{center}
%
Here one could write a message such as:
\begin{center}
|This is the part \childdocname{} of \childdocjob{}.|
\end{center}

%%%%%%%%%%%%%%%%%%%%%%%%%%%%%%%%%%%%%%%%%%%%%%%%%%%%%%%%%%%%%%%%%%%%%%%%%%%%%%%%
\subsection{Flags}
\label{sec:flags}

The package makes it easy to generate different versions
of the main or child documents.
To this end compilation flags can be defined
and assigned different default values.
They will be particularly useful in conjunction
with the forwarding mechanism described in \secref{sec:forward}.

For example, it may be useful to have a flag |\version|
which can be set to |draft| or |final|.
The document source will contain some conditional code
depending on the value of |\version|.
Suppose further, the flag should default to |final| for the main file
and to |draft| for child files
which is a natural assignment for editing the document.
This is achieved by placing the following code
in the preamble of the main document
(below the |\childdocmain| directive):
%
\begin{center}
\begin{tabular}{l}
|\ifchilddoc|\\
|\providecommand{\version}{draft}|\\
|\||else|\\
|\providecommand{\version}{final}|\\
|\||fi|
\end{tabular}
\end{center}
%
The definition by |\providecommand| makes sure
that previous definitions are not overwritten.
Further statements |\providecommand{\version}{...}|
can thus be added before the above code to override it.

For the main file, one might add a line
(between |\childdocmain| and the above block)
%
\begin{center}
|%\ifchilddoc\||else\providecommand{\version}{draft}\||fi|
\end{center}
%
which can be uncommented to produce a draft version.
Likewise one can add a line to the very top of a child file
(above the |\childdocof{|\textit{main}|}| directive)
%
\begin{center}
|%\providecommand{\version}{final}|
\end{center}
%
which can be uncommented to produce the final version of this child document.

%%%%%%%%%%%%%%%%%%%%%%%%%%%%%%%%%%%%%%%%%%%%%%%%%%%%%%%%%%%%%%%%%%%%%%%%%%%%%%%%
\subsection{Forwarding}
\label{sec:forward}

Different versions of the main or child documents
using compilation flags as described in \secref{sec:flags}
can be (permanently) stored in different files
for convenient compilation, viewing and distribution.
To this end, the package defines a command
to pass on compilation to a different file:

%%%%%%%%%%%%%%%%%%%%%%%%%%%%%%%%%%%%%%%%
\DescribeMacro{\childdocforward}
The command |\childdocforward| redirects processing to
another source file:
%
\begin{center}
\begin{tabular}{l}
|\input{childdoc.def}|\\
|\childdocforward[|\textit{main}|]{|\textit{dest}|}|\\
\end{tabular}
\end{center}
%
The argument \textit{dest} is the destination file
(without extension).
It should be the main file or one of the child files.
Note that further \textsf{childdoc} directives
such as |\childdocof| and |\childdocforward|
in the indicated file will be processed in this form.
The optional argument \textit{main}
passes on directly to the main file \textit{main}
while pretending to compile the child \textit{dest}.
This form behaves as if \textit{dest}
issues |\childdocof{|\textit{main}|}| right away,
and no further \textsf{childdoc} directives will be processed.

%%%%%%%%%%%%%%%%%%%%%%%%%%%%%%%%%%%%%%%%
\DescribeMacro{\...prefix}
In the alternative form |\childdocforwardprefix|,
%
\begin{center}
\begin{tabular}{l}
|\input{childdoc.def}|\\
|\childdocforwardprefix[|\textit{main}|]{|\textit{prefix}|}{|\textit{dest}|}|
\end{tabular}
\end{center}
%
the destination file is determined by a pattern
depending on the current file:
To make this work, the current file must be called
`{\textit{prefix}\hspace{0.2em}\textit{suffix}}'
with \textit{prefix} matching precisely the argument.
Processing is then passed on to the file
`{\textit{dest}\hspace{0.2em}\textit{suffix}}'.
Surely, the same effect is achieved by
directly specifying the
argument `{\textit{dest}\hspace{0.2em}\textit{suffix}}'
in the first form.
However, that requires to set up a different file
for each child. With the alternative form of the command
all these files can have exactly the same content
which simplifies setting them up and maintaining them.

For example, the following file |draft.tex|
with a compilation flag |\version| as described in \secref{sec:flags}
compiles the main document as a draft:
%
\begin{center}
\begin{tabular}{l}
|\def\version{draft}|\\
|\input{childdoc.def}|\\
|\childdocforward{|\textit{main}|}|
\end{tabular}
\end{center}
%
Likewise, the following files |final|\textit{nn}|.tex|
compile the final version of the child document
|child|\textit{nn}|.tex|:
%
\begin{center}
\begin{tabular}{l}
|\def\version{final}|\\
|\input{childdoc.def}|\\
|\childdocforwardprefix{final}{child}|
\end{tabular}
\end{center}
%

Note that when several versions of a main file and/or of each child file
are to be generated, it may be convenient to set up a |Makefile| or
shell script to automatise the process.

%%%%%%%%%%%%%%%%%%%%%%%%%%%%%%%%%%%%%%%%%%%%%%%%%%%%%%%%%%%%%%%%%%%%%%%%%%%%%%%%
\subsection{Command Line Processing}
\label{sec:commandline}

The effect of redirection files can also be achieved by invoking
the \LaTeX{} compiler with a more elaborate command line.
Most conveniently this should be done as part
of a shell script or a |Makefile|.

When using \textsf{childdoc} in the main file, the following
command lines effectively perform a redirection
(note that depending on the shell being used,
backslashes may have to be doubled: `|\|' $\to$ `|\\|'):
%
\begin{center}
|... -jobname "|\textit{target}|" |\\|"|[\textit{flags}]%
|\input{childdoc.def}\childdocforward[|\textit{main}|]{|\textit{dest}|}"|
\end{center}
%
Here \textit{target} is the name of the output file,
\textit{main} is the name of the main file
and \textit{dest} is the name of the main or child file to be processed
(all filenames without extensions).
The optional argument \textit{main} can be omitted
if \textit{main} matches \textit{dest}.
Optionally, compilation \textit{flags} can be defined via |\def| commands.
This command line makes the \TeX{} engine believe
it is compiling the file \textit{target}
whose content is specified as the latter parameter.
The provided code then forwards the processing to
\textit{main} or \textit{dest} as described in \secref{sec:forward}.

%%%%%%%%%%%%%%%%%%%%%%%%%%%%%%%%%%%%%%%%%%%%%%%%%%%%%%%%%%%%%%%%%%%%%%%%%%%%%%%%
\subsection{Include by Input}
\label{sec:input}

Including child documents by |\include| has some restrictions by design.
Most notably, the content of a child document always occupies
its own set of pages; pages cannot be shared between child documents.
Usually, this behaviour makes perfect sense
because each child document contain an essential part of the document.
However, in some situations it may be desirable to compose
a document from a collection of parts
without having mandatory page breaks between then.
For this case, the package
provides a mechanism to include parts
by |\input| which can also be processed individually.
However, by construction this mechanism
requires manual handling of the content to be output.

%%%%%%%%%%%%%%%%%%%%%%%%%%%%%%%%%%%%%%%%
\DescribeMacro{\ifchilddocmanual}
The main file should be prepared as usual, see \secref{sec:include}.
However, the document body must make a distinction
between processing of an individual part and of the main document, e.g.:
%
\begin{center}
\begin{tabular}{l}
|\ifchilddocmanual|\\
|\input{\childdocname}|\\
|\||else|\\
\textit{document body with }|\input{|\textit{part}|}|\\
|\||fi|
\end{tabular}
\end{center}
%
The conditional |\ifchilddocmanual| is true whenever
a part to be included by |\input| is being compiled,
and the name of the part is stored in |\childdocname|.

%%%%%%%%%%%%%%%%%%%%%%%%%%%%%%%%%%%%%%%%
\DescribeMacro{\childdocby}
Each part to be included by |\input| should start with:
%
\begin{center}
\begin{tabular}{l}
|\input{childdoc.def}|\\
|\childdocby{|\textit{main}|}|\\
\end{tabular}
\end{center}
%
The directive |\childdocby| is similar to |\childdocof|
described in \secref{sec:include},
but the subsequent selection of content must be done manually.
To that end, both |\ifchilddoc| and |\ifchilddocmanual|
will be true upon processing of a part,
and the name of the part is stored in |\childdocname|.
Note that |\jobname| will be set to the filename of the current part
so that each part receives an individual |.aux| file
that does not interfere with the |.aux| file(s) of the main document.
This behaviour can be altered by the alternative form
|\childdocby[*]{|\textit{main}|}| (with a non-empty optional argument)
which uses the |.aux| file of the main document
by setting |\jobname| to \textit{main}.

%%%%%%%%%%%%%%%%%%%%%%%%%%%%%%%%%%%%%%%%%%%%%%%%%%%%%%%%%%%%%%%%%%%%%%%%%%%%%%%%
\subsection{Driver Development}
\label{sec:driver}

The \textsf{childdoc} mechanism can also be use for the development
of definition files such as \LaTeX{} styles or classes.
This case differs from the above setup with multiple parts
included by |\include| in that no |\includeonly| should be invoked.
This can be achieved by starting the include file
(before |\ProvidesPackage|) with:
%
\begin{center}
\begin{tabular}{l}
|\input{childdoc.def}|\\
|\childdocforward{|\textit{main}|}|\\
\end{tabular}
\end{center}
%
or alternatively with:
%
\begin{center}
\begin{tabular}{l}
|\input{childdoc.def}|\\
|\childdocby{|\textit{main}|}|\\
\end{tabular}
\end{center}
%
Both forms have slightly different effects as described above.
The main file is prepared as usual, see \secref{sec:include}.

%%%%%%%%%%%%%%%%%%%%%%%%%%%%%%%%%%%%%%%%%%%%%%%%%%%%%%%%%%%%%%%%%%%%%%%%%%%%%%%%
\subsection{Legacy Detection}
\label{sec:detection}

The directive |\childdocmain| in the main file can detect
whether the complete document or merely a child is to be compiled
even without using the directive |\childdocof|.
This method is deprecated because it is less robust
and there is no compelling reason to use it;
it is merely provided for backward compatibility
and it may be removed in future versions.

If the detection mechanism is to be used,
it is mandatory to correctly specify
the filename of the main file as the argument of |\childdocmain|:
%
\begin{center}
\begin{tabular}{l}
|\input{childdoc.def}|\\
|\childdocmain{|\textit{main}|}|\\
\end{tabular}
\end{center}
%
If |\jobname| does not match the argument \textit{main} of |\childdocmain|,
it is assumed that |\jobname| points to the child file to be compiled.
When using |\childdocmain| with the main file specified as argument,
it suffices to start a child file
with just |\input{|\textit{main}|}|
without loading of the package and using |\childdocof|.
If instead all processing is done
with the appropriate \textsf{childdoc} directives,
the argument of \textit{main} of |\childdocmain| can be empty.

An alternative version of the command line processing described
in \secref{sec:commandline} using the detection mechanism reads:
%
\begin{center}
|... -jobname "|\textit{target}|" "|[\textit{flags}]%
[|\def\jobname{|\textit{dest}|}|]|\input{|\textit{main}|}"|
\end{center}

%%%%%%%%%%%%%%%%%%%%%%%%%%%%%%%%%%%%%%%%%%%%%%%%%%%%%%%%%%%%%%%%%%%%%%%%%%%%%%%%
\subsection{Manual Code}
\label{sec:manual}

In case one cannot be certain whether the definitions file |childdoc.def|
is installed on the target \TeX{} distribution
and one prefers not to ship it,
it is conceivable to paste a few relevant commands into the sources.

To that end, drop all statements |\input{childdoc.def}|
and perform the replacements as outlined below.
Instead of |\childdocmain{|\textit{main}|}| add the following code
to the top of the main file:
%
\begin{center}
\begin{tabular}{l}
|\||ifdefined\childdocname\endinput\||fi\newif\ifchilddoc|\\
|\edef\childdocname{\scantokens\expandafter{\jobname\noexpand}}|\\
|\def\childdocmain{|\textit{main}|}\||ifx\childdocmain\childdocname\||else|\\
|\childdoctrue\includeonly{\childdocname}\let\jobname\childdocmain\||fi|\\
\end{tabular}
\end{center}
%
Instead of |\childdocof{|\textit{main}|}| just include the main file
at the top of each child file:
%
\begin{center}
|\input{|\textit{main}|}|
\end{center}
%
A simple redirection |\childdocforward{|\textit{dest}|}| is achieved by:
%
\begin{center}
|\def\jobname{|\textit{dest}|}\input{\jobname}|
\end{center}
%
The redirection with prefix
|\childdocforwardprefix[|\textit{prefix}|]{|\textit{dest}|}|
is accomplished by:
%
\begin{center}
\begin{tabular}{l}
|{\edef\jobname{\scantokens\expandafter{\jobname\noexpand}}|\\
|\def\redirectjob |\textit{prefix}|#1~~~{\gdef\jobname{|\textit{dest}|#1}}|\\
|\expandafter\redirectjob\jobname~~~}\input{\jobname}|
\end{tabular}
\end{center}

In an alternative approach,
child documents can be compiled by a specific command line
without additional code or specific definitions:
%
\begin{center}
|... -jobname "|\textit{target}|" "|[\textit{flags}]%
|\includeonly{|\textit{dest}|}\input{|\textit{main}|}"|
\end{center}
%

%%%%%%%%%%%%%%%%%%%%%%%%%%%%%%%%%%%%%%%%%%%%%%%%%%%%%%%%%%%%%%%%%%%%%%%%%%%%%%%%
%%%%%%%%%%%%%%%%%%%%%%%%%%%%%%%%%%%%%%%%%%%%%%%%%%%%%%%%%%%%%%%%%%%%%%%%%%%%%%%%
\section{Information}

%%%%%%%%%%%%%%%%%%%%%%%%%%%%%%%%%%%%%%%%%%%%%%%%%%%%%%%%%%%%%%%%%%%%%%%%%%%%%%%%
\subsection{Copyright}

Copyright \copyright{} 2017--2018 Niklas Beisert

This work may be distributed and/or modified under the
conditions of the \LaTeX{} Project Public License, either version 1.3
of this license or (at your option) any later version.
The latest version of this license is in
  \url{http://www.latex-project.org/lppl.txt}
and version 1.3 or later is part of all distributions of \LaTeX{}
version 2005/12/01 or later.

This work has the LPPL maintenance status `maintained'.

The Current Maintainer of this work is Niklas Beisert.

This work consists of the files |README.txt|, |childdoc.ins| and |childdoc.dtx|
as well as the derived files |childdoc.def|, |cdocsamp.tex|
with |cdocsch1.tex|, |cdocsch2.tex|, |cdocspt3.tex|, |cdocspt4.tex|,
|cdocsdrf.tex|, |cdocsfn1.tex|, |cdocsfn2.tex|
as well as |childdoc.pdf|.

%%%%%%%%%%%%%%%%%%%%%%%%%%%%%%%%%%%%%%%%%%%%%%%%%%%%%%%%%%%%%%%%%%%%%%%%%%%%%%%%
\subsection{Files and Installation}

The package consists of the files:
%
\begin{center}
\begin{tabular}{ll}
    |README.txt|   & readme file \\
    |childdoc.ins| & installation file \\
    |childdoc.dtx| & source file \\
    |childdoc.def| & definition file \\
    |cdocsamp.tex| & sample main file \\
    |cdocsch1.tex| & sample include file \\
    |cdocsch2.tex| & sample include file \\
    |cdocspt3.tex| & sample part file \\
    |cdocspt4.tex| & sample part file \\
    |cdocsdrf.tex| & sample redirection file \\
    |cdocsfn1.tex| & sample redirection file \\
    |cdocsfn2.tex| & sample redirection file \\
    |childdoc.pdf| & manual
\end{tabular}
\end{center}
%
The distribution consists of the files
|README.txt|, |childdoc.ins| and |childdoc.dtx|.
%
\begin{itemize}
\item
Run (pdf)\LaTeX{} on |childdoc.dtx|
to compile the manual |childdoc.pdf| (this file).
\item
Run \LaTeX{} on |childdoc.ins| to create the definitions file |childdoc.def|
and the sample |cdocsamp.tex| with include files
|cdocsch1.tex|, |cdocsch2.tex|, |cdocspt3.tex|, |cdocspt4.tex|,
|cdocsdrf.tex|, |cdocsfn1.tex|, |cdocsfn2.tex|.
Then copy the file |childdoc.def| to an appropriate directory of your \LaTeX{}
distribution, e.g.\ \textit{texmf-root}|/tex/latex/childdoc|.
\end{itemize}

%%%%%%%%%%%%%%%%%%%%%%%%%%%%%%%%%%%%%%%%%%%%%%%%%%%%%%%%%%%%%%%%%%%%%%%%%%%%%%%%
\subsection{Related CTAN Packages}

There are several other packages which offer a similar functionality:
%
\begin{itemize}
\item
The packages
\href{http://ctan.org/pkg/docmute}{\textsf{docmute}},
\href{http://ctan.org/pkg/includex}{\textsf{includex}} and
\href{http://ctan.org/pkg/standalone}{\textsf{standalone}}
provide commands to include only the document body of
a child file thus allowing both files to be compiled individually.
\item
The packages \href{http://ctan.org/pkg/subdocs}{\textsf{subdocs}}
and \href{http://ctan.org/pkg/subfiles}{\textsf{subfiles}}
provide structures in which the main and child documents can be
encapsulated and allowing them to be compiled individually.
The inclusion mechanism is different from the conventional |\include|.
\item
The package \href{http://ctan.org/pkg/combine}{\textsf{combine}}
is an elaborate solution to combine several documents into one.
\end{itemize}
%
See also the CTAN topic \href{http://ctan.org/topic/subdocs}{\textsf{subdocs}}
for further related packages.
The present package differs from the above solutions in that
a document structure constructed with the conventional |\include| mechanism
just needs two extra commands at the top of every file
such that all constituent files can be compiled individually.

%%%%%%%%%%%%%%%%%%%%%%%%%%%%%%%%%%%%%%%%%%%%%%%%%%%%%%%%%%%%%%%%%%%%%%%%%%%%%%%%
%\subsection{Feature Suggestions}
%
%The following is a list of features which may be useful for future
%versions of this package:
%%
%\begin{itemize}
%\item
%\ldots
%\end{itemize}

%%%%%%%%%%%%%%%%%%%%%%%%%%%%%%%%%%%%%%%%%%%%%%%%%%%%%%%%%%%%%%%%%%%%%%%%%%%%%%%%
\subsection{Revision History}

%%%%%%%%%%%%%%%%%%%%%%%%%%%%%%%%%%%%%%%%
\paragraph{v2.0:} 2018/12/30

\begin{itemize}
\item
immediate forward processing
\item
added |\childdocby| mechanism
\item
manual restructured
\end{itemize}

%%%%%%%%%%%%%%%%%%%%%%%%%%%%%%%%%%%%%%%%
\paragraph{v1.6:} 2018/01/17

\begin{itemize}
\item
application for development of include files
\item
corrections to manual
\end{itemize}

%%%%%%%%%%%%%%%%%%%%%%%%%%%%%%%%%%%%%%%%
\paragraph{v1.5:} 2017/05/21

\begin{itemize}
\item
more complete structuring introduced
\item
|\childdocof| introduced
\item
|\childdoc| renamed to |\childdocmain|
\item
|\childredirect| renamed to |\childdocforward| and |\childdocforwardprefix|
and functionality expanded
\end{itemize}

%%%%%%%%%%%%%%%%%%%%%%%%%%%%%%%%%%%%%%%%
\paragraph{v1.0:} 2017/04/27

\begin{itemize}
\item
manual and install package
\item
first version published on CTAN
\end{itemize}

%%%%%%%%%%%%%%%%%%%%%%%%%%%%%%%%%%%%%%%%
\paragraph{v0.6:} 2017/04/26

\begin{itemize}
\item
redirection mechanism added
\end{itemize}

%%%%%%%%%%%%%%%%%%%%%%%%%%%%%%%%%%%%%%%%
\paragraph{v0.5:} 2017/04/26

\begin{itemize}
\item
functionality in definition file
\end{itemize}


%%%%%%%%%%%%%%%%%%%%%%%%%%%%%%%%%%%%%%%%%%%%%%%%%%%%%%%%%%%%%%%%%%%%%%%%%%%%%%%%
%%%%%%%%%%%%%%%%%%%%%%%%%%%%%%%%%%%%%%%%%%%%%%%%%%%%%%%%%%%%%%%%%%%%%%%%%%%%%%%%
%%%%%%%%%%%%%%%%%%%%%%%%%%%%%%%%%%%%%%%%%%%%%%%%%%%%%%%%%%%%%%%%%%%%%%%%%%%%%%%%
\appendix

\settowidth\MacroIndent{\rmfamily\scriptsize 000\ }

 \DocInput{childdoc.dtx}

\end{document}
%</driver>
% \fi
%
% %%%%%%%%%%%%%%%%%%%%%%%%%%%%%%%%%%%%%%%%%%%%%%%%%%%%%%%%%%%%%%%%%%%%%%%%%%%%%%
% %%%%%%%%%%%%%%%%%%%%%%%%%%%%%%%%%%%%%%%%%%%%%%%%%%%%%%%%%%%%%%%%%%%%%%%%%%%%%%
% \section{Sample}
%\iffalse
%<*samplemain>
%\fi
%
% The following presents a sample document
% with two chapters, two parts, a title page,
% a compile flag as well as three forwarding files to set the flag.
% It consists of eight |.tex| files:
% \begin{center}
% \begin{tabular}{ll}
% |cdocsamp.tex|&main file\\
% |cdocsch1.tex|&include file for chapter 1\\
% |cdocsch2.tex|&include file for chapter 2\\
% |cdocspt3.tex|&include file for part 3\\
% |cdocspt4.tex|&include file for part 4\\
% |cdocsdrf.tex|&forwarding file for main file in draft mode\\
% |cdocsfi1.tex|&forwarding file for final version of chapter 1\\
% |cdocsfi2.tex|&forwarding file for final version of chapter 2\\
% \end{tabular}
% \end{center}
% Each of the eight files can be compiled directly by the \LaTeX{} compiler.
%
% %%%%%%%%%%%%%%%%%%%%%%%%%%%%%%%%%%%%%%
% \paragraph{Main File.}
%
% The main file is called |cdocsamp.tex|.
%
% Load the \textsf{childdoc} definitions and
% declare the filename for the main document:
%    \begin{macrocode}
\input{childdoc.def}
\childdocmain{}
%    \end{macrocode}

% Optional override for |\version| flag:
%    \begin{macrocode}
%%\ifchilddoc\else\providecommand{\version}{draft}\fi
%    \end{macrocode}

% Define the default values for the |\version| flag
% (|final| for the main file and |draft| for childs):
%    \begin{macrocode}
\ifchilddoc
\providecommand{\version}{draft}
\else
\providecommand{\version}{final}
\fi
%    \end{macrocode}

% Load the standard document class:
%    \begin{macrocode}
\documentclass[12pt]{article}
%    \end{macrocode}

% Start the document body:
%    \begin{macrocode}
\begin{document}
%    \end{macrocode}

% Declare a title page.
% Print title, part of document being processed and version flag:
%    \begin{macrocode}
\addtocounter{page}{-1}
\begin{center}
{\LARGE\bfseries{}childdoc example\par}
\vspace{1cm}
\ifchilddoc
\ifchilddocmanual part\else chapter\fi:
`\childdocname' of `\childdocjob'\par
\else
main document: `\childdocjob'\par
\fi
version: \version\par
\end{center}
\newpage
%    \end{macrocode}

% Manually include selected file,
% otherwise process as usual:
%    \begin{macrocode}
\ifchilddocmanual
\section*{part `\childdocname'}
\input{\childdocname}
\else
%    \end{macrocode}

% Include the two chapters:
%    \begin{macrocode}
\include{cdocsch1}
\include{cdocsch2}
%    \end{macrocode}

% Include the two parts unless only chapters should be displayed:
%    \begin{macrocode}
\ifchilddoc\else
\section{part three}
\input{cdocspt3}
\section{part four}
\input{cdocspt4}
\fi
%    \end{macrocode}

% Process as usual until here:
%    \begin{macrocode}
\fi
%    \end{macrocode}

% End of document body:
%    \begin{macrocode}
\end{document}
%    \end{macrocode}
%\iffalse
%</samplemain>
%\fi
%
% %%%%%%%%%%%%%%%%%%%%%%%%%%%%%%%%%%%%%%
% \paragraph{Chapter Include Files.}
%
% The include files are called |cdocsch1.tex| and |cdocsch2.tex|.
%
%\iffalse
%<*samplechap1|samplechap2>
%\fi

% Optional override for |\version| flag:
%    \begin{macrocode}
%%\providecommand{\version}{final}
%    \end{macrocode}

% Include the main document:
%    \begin{macrocode}
\input{childdoc.def}
\childdocof{cdocsamp}
%    \end{macrocode}

%\iffalse
%</samplechap1|samplechap2>
%\fi
%
%\iffalse
%<*samplechap1>
%\fi
% Some text for chapter 1:
%    \begin{macrocode}
\section{one}
some text in chapter one
%    \end{macrocode}

%\iffalse
%</samplechap1>
%\fi
% Some text for chapter 2:
%\iffalse
%<*samplechap2>
%\fi
%    \begin{macrocode}
\section{two}
more text in chapter two
%    \end{macrocode}

%\iffalse
%</samplechap2>
%\fi
%
% %%%%%%%%%%%%%%%%%%%%%%%%%%%%%%%%%%%%%%
% \paragraph{Part Include Files.}
%
% The include files are called |cdocspt3.tex| and |cdocspt4.tex|.
%
%\iffalse
%<*samplepart3|samplepart4>
%\fi

% Optional override for |\version| flag:
%    \begin{macrocode}
%%\providecommand{\version}{final}
%    \end{macrocode}

% Include the main document:
%    \begin{macrocode}
\input{childdoc.def}
\childdocby{cdocsamp}
%    \end{macrocode}

%\iffalse
%</samplepart3|samplepart4>
%\fi
%
%\iffalse
%<*samplepart3>
%\fi
% Some text for part 3:
%    \begin{macrocode}
some text in part three
%    \end{macrocode}

%\iffalse
%</samplepart3>
%\fi
% Some text for part 4:
%\iffalse
%<*samplepart4>
%\fi
%    \begin{macrocode}
more text in part four
%    \end{macrocode}

%\iffalse
%</samplepart4>
%\fi
%
% %%%%%%%%%%%%%%%%%%%%%%%%%%%%%%%%%%%%%%
% \paragraph{Forwarding for a Complete Draft.}
%
% The following forwarding file |cdocsdrf.tex|
% compiles the main document in draft mode:
%\iffalse
%<*sampledraft>
%\fi
%    \begin{macrocode}
\def\version{draft}
\input{childdoc.def}
\childdocforward{cdocsamp}
%    \end{macrocode}

%\iffalse
%</sampledraft>
%\fi
%
% %%%%%%%%%%%%%%%%%%%%%%%%%%%%%%%%%%%%%%
% \paragraph{Forwarding for Final Version of the Chapters.}
%
% The following forwarding files |cdocsfn1.tex| and |cdocsfn2.tex|
% (with identical content)
% compile the final versions of the child documents
% |cdocsch1.tex| and |cdocsch2.tex|, respectively:
%\iffalse
%<*samplefinal>
%\fi
%    \begin{macrocode}
\def\version{final}
\input{childdoc.def}
\childdocforwardprefix[cdocsamp]{cdocsfn}{cdocsch}
%    \end{macrocode}

%\iffalse
%</samplefinal>
%\fi
%
% %%%%%%%%%%%%%%%%%%%%%%%%%%%%%%%%%%%%%%
% \paragraph{Command Line Processing.}
%
% The following three command lines generate the output files
% |cdocscld|, |cdocscl1| and |cdocscl2|
% which should be identical to
% |cdocsdrf|, |cdocsch1| and |cdocsfn2|, respectively:
% \begin{center}
% \begin{tabular}{l}
% |latex -jobname cdocscld \|\\
% |  "\def\version{draft}\input{childdoc.def}\childdocforward{cdocsamp}"|\\
% |latex -jobname cdocscl1 \|\\
% |  "\input{childdoc.def}\childdocforward[cdocsamp]{cdocsch1}"|\\
% |latex -jobname cdocscl2 \|\\
% |  "\def\version{final}\input{childdoc.def}\childdocforward{cdocsch2}"|
% \end{tabular}
% \end{center}
% Note that the trailing backslash on each first line
% merely continues the input to the second line
% (for convenient cut ant paste).
% Furthermore, the command |latex| can be replaced by any
% of its alternative versions such as |pdflatex|.
%
% %%%%%%%%%%%%%%%%%%%%%%%%%%%%%%%%%%%%%%%%%%%%%%%%%%%%%%%%%%%%%%%%%%%%%%%%%%%%%%
% %%%%%%%%%%%%%%%%%%%%%%%%%%%%%%%%%%%%%%%%%%%%%%%%%%%%%%%%%%%%%%%%%%%%%%%%%%%%%%
% \section{Implementation}
%\iffalse
%<*package>
%\fi
%
% This section describes the definitions file |childdoc.def|.

% The definitions cannot be loaded using |\usepackage| or |\RequirePackage|
% which has a mechanism to prevent loading a style file more than once.
% When loading the definitions by means of |\input|
% multiple instances have to be prevented manually:
%\iffalse
%This code needs to be before the `\ProvidesFile' directive
%which is defined at the beginning of this file.
%Therefore it is also placed there and commented out here.
%</package>
%<*discard>
%\fi
%    \begin{macrocode}
\ifdefined\childdocmain\endinput\fi
%    \end{macrocode}
%\iffalse
%</discard>
%<*package>
%\fi
%
% \macro{\ifchilddoc}
% \macro{\ifchilddocmanual}
% The conditional |\ifchilddoc| tells whether a
% child (true) or main (false) document is being compiled.
% The conditional |\ifchilddocmanual| tells whether
% the |\includeonly| mechanism is used (false) or
% the selection of child files must be performed manually (true).
% The definitions initialise to false:
%    \begin{macrocode}
\newif\ifchilddoc
\newif\ifchilddocmanual
%    \end{macrocode}

% \macro{\childdocname}
% \macro{\childdocjob}
% The macro |\childdocname| stores the name of the main document
% to be compiled. The macro |\childdocjob| stores the name of
% the document on which the \LaTeX{} compiler was originally invoked.
% The content of |\jobname| cannot be compared
% to filenames specified in the source due to different catcodes.
% The following code rescans |\jobname|, stores the result
% in |\childdocname| and saves a copy in |\childdocjob|:
%    \begin{macrocode}
\edef\childdocname{\scantokens\expandafter{\jobname\noexpand}}
\let\childdocjob\childdocname
%    \end{macrocode}

% \macro{\childdocdisable}
% The macro |\childdocdisable| prevents the main file
% from being processed more than once.
% At this stage, the main document command |\childdocmain|
% is assumed to be called once again where it should do nothing.
% Any subsequent call to it should prevent
% a secondary processing of the main document
% It overwrites the forwarding commands
% |\childdocof| and |\childdocforward|
% with empty macros to prevent further inclusions of the main document:
%    \begin{macrocode}
\newcommand{\childdocdisable}
{
  \renewcommand{\childdocmain}[1]{\renewcommand{\childdocmain}[1]{\endinput}}
  \renewcommand{\childdocof}[1]{}
  \renewcommand{\childdocby}[2][]{}
  \renewcommand{\childdocforward}[2][]{}
  \renewcommand{\childdocdisable}{}
}
%    \end{macrocode}

% \macro{\childdocmain}
% The macro |\childdocmain| is to be called at the top of the main file
% with nothing or the main filename (without extension) as argument.
% First, it breaks loops.
% If the argument is not empty and does not match |\childdocname|
% (which is set by the first inclusion of |childdoc.def|),
% |\ifchilddoc| is set to true, |\includeonly| is applied to the child file
% and |\jobname| is set to the main file
% (for proper handling of |.aux| files):
%    \begin{macrocode}
\newcommand{\childdocmain}[1]
{
  \childdocdisable\childdocmain{}
  \if?#1?\else
    \begingroup
      \def\childdoctmp{#1}
      \ifx\childdoctmp\childdocname
        \def\childdoctmp{}
      \else
        \def\childdoctmp
        {
          \childdoctrue
          \includeonly{\childdocname}
          \def\childdocjob{#1}
          \def\jobname{#1}
        }
      \fi
      \expandafter
    \endgroup
    \childdoctmp
  \fi
}
%    \end{macrocode}

% \macro{\childdocof}
% The command |\childdocof| redirects
% compilation to the main file |#1|.
%    \begin{macrocode}
\newcommand{\childdocof}[1]
{
  \childdocdisable
  \childdoctrue
  \includeonly{\childdocname}
  \def\jobname{#1}
  \def\childdocjob{#1}
  \input{#1}
}
%    \end{macrocode}

% \macro{\childdocby}
% The command |\childdocby| ....
%    \begin{macrocode}
\newcommand{\childdocby}[2][]
{
  \childdocdisable
  \childdoctrue
  \childdocmanualtrue
  \if?#1?\else
    \def\jobname{#2}
  \fi
  \def\childdocjob{#2}
  \input{#2}
  \endinput
}
%    \end{macrocode}

% \macro{\childdocforward}
% The command |\childdocforward| redirects
% compilation to the main file or
% (if the optional argument is given) a child file.
% Parameters are set as if the main file
% or a child file starting with |\childdocof| was compiled.
% Then compilation is handed over to the main file:
%    \begin{macrocode}
\newcommand{\childdocforward}[2][]
{
  \begingroup
    \if?#1?
      \def\childdoctmp
      {
        \def\childdocname{#2}
        \def\childdocjob{#2}
        \def\jobname{#2}
        \input{#2}
        \endinput
      }
    \else
      \def\childdoctmp
      {
        \childdocdisable
        \def\childdocname{#2}
        \childdoctrue
        \includeonly{#2}
        \def\childdocjob{#1}
        \def\jobname{#1}
        \input{#1}
        \endinput
      }
    \fi
    \expandafter
  \endgroup
  \childdoctmp
}
%    \end{macrocode}

% \macro{\childdocforwardprefix}
% The command |\childdocforwardprefix| redirects
% compilation to the main or a child file by means of a pattern.
% The prefix |#1| in the current filename is replaced by |#2|
% and the suffix of the current filename is kept
% (it is assumed that the filename does not contain the substring `|~~~|'
% which is used as a delimiter).
% Compilation is handed over to the new file by |\childdocforward|:
%    \begin{macrocode}
\newcommand{\childdocforwardprefix}[3][]
{
  \begingroup
    \def\childdocextract #2##1~~~{\def\childdoctmp{\childdocforward[#1]{#3##1}}}
    \expandafter\childdocextract\childdocname~~~
    \expandafter
  \endgroup
  \childdoctmp
}
%    \end{macrocode}

% \macro{\childdoc}
% The deprecated macro |\childdoc| is a legacy version of |\childdocmain|:
%    \begin{macrocode}
\newcommand{\childdoc}{\childdocmain}
%    \end{macrocode}

% \macro{\childdocredirect}
% The deprecated macro |\childdocredirect| is a legacy version
% of |\childdocforward| and |\childdocforwardprefix|:
%    \begin{macrocode}
\newcommand{\childdocredirect}[2][]
{
  \begingroup
    \if?#1?
      \def\childdoctmp{\childdocforward{#2}}
    \else
      \def\childdoctmp{\childdocforwardprefix{#1}{#2}}
    \fi
    \expandafter
  \endgroup
  \childdoctmp
}
%    \end{macrocode}

%\iffalse
%</package>
%\fi
%
\endinput
|\\
|\childdocforwardprefix{final}{child}|
\end{tabular}
\end{center}
%

Note that when several versions of a main file and/or of each child file
are to be generated, it may be convenient to set up a |Makefile| or
shell script to automatise the process.

%%%%%%%%%%%%%%%%%%%%%%%%%%%%%%%%%%%%%%%%%%%%%%%%%%%%%%%%%%%%%%%%%%%%%%%%%%%%%%%%
\subsection{Command Line Processing}
\label{sec:commandline}

The effect of redirection files can also be achieved by invoking
the \LaTeX{} compiler with a more elaborate command line.
Most conveniently this should be done as part
of a shell script or a |Makefile|.

When using \textsf{childdoc} in the main file, the following
command lines effectively perform a redirection
(note that depending on the shell being used,
backslashes may have to be doubled: `|\|' $\to$ `|\\|'):
%
\begin{center}
|... -jobname "|\textit{target}|" |\\|"|[\textit{flags}]%
|% \iffalse
%
% childdoc.dtx Copyright (C) 2017-2018 Niklas Beisert
%
% This work may be distributed and/or modified under the
% conditions of the LaTeX Project Public License, either version 1.3
% of this license or (at your option) any later version.
% The latest version of this license is in
%   http://www.latex-project.org/lppl.txt
% and version 1.3 or later is part of all distributions of LaTeX
% version 2005/12/01 or later.
%
% This work has the LPPL maintenance status `maintained'.
%
% The Current Maintainer of this work is Niklas Beisert.
%
% This work consists of the files childdoc.dtx and childdoc.ins
% and the derived files childdoc.def and cdocsamp.tex with
% cdocsch1.tex, cdocsch2.tex, cdocsdrf.tex, cdocsfn1.tex, cdocsfn2.tex.
%
%<package>\ifdefined\childdocmain\endinput\fi
%<package>\ProvidesFile{childdoc.def}[2018/12/30 v2.0 child document driver]
%<samplemain>\ProvidesFile{cdocsamp.tex}[2018/12/30 v2.0 sample for childdoc]
%<*driver>
%\ProvidesFile{childdoc.drv}[2018/12/30 v2.0 childdoc reference manual file]
\PassOptionsToClass{10pt,a4paper}{article}
\documentclass{ltxdoc}

\usepackage[margin=35mm]{geometry}
\usepackage{hyperref}
\usepackage{hyperxmp}
\usepackage[usenames]{color}

\hypersetup{colorlinks=true}
\hypersetup{pdfstartview=FitH}
\hypersetup{pdfpagemode=UseNone}
\hypersetup{pdfsource={}}
\hypersetup{pdflang={en-UK}}
\hypersetup{pdfcopyright={Copyright 2017-2018 Niklas Beisert.
  This work may be distributed and/or modified under the
  conditions of the LaTeX Project Public License, either version 1.3
  of this license or (at your option) any later version.}}
\hypersetup{pdflicenseurl={http://www.latex-project.org/lppl.txt}}
\hypersetup{pdfcontactaddress={ETH Zurich, ITP, HIT K,
  Wolfgang-Pauli-Strasse 27}}
\hypersetup{pdfcontactpostcode={8093}}
\hypersetup{pdfcontactcity={Zurich}}
\hypersetup{pdfcontactcountry={Switzerland}}
\hypersetup{pdfcontactemail={nbeisert@itp.phys.ethz.ch}}
\hypersetup{pdfcontacturl={http://people.phys.ethz.ch/\xmptilde nbeisert/}}

\newcommand{\secref}[1]{\hyperref[#1]{section \ref*{#1}}}

\parskip1ex
\parindent0pt
\let\olditemize\itemize
\def\itemize{\olditemize\parskip0pt}

\begin{document}

\title{The \textsf{childdoc} Package}
\hypersetup{pdftitle={The childdoc Package}}
\author{Niklas Beisert\\[2ex]
  Institut f\"ur Theoretische Physik\\
  Eidgen\"ossische Technische Hochschule Z\"urich\\
  Wolfgang-Pauli-Strasse 27, 8093 Z\"urich, Switzerland\\[1ex]
  \href{mailto:nbeisert@itp.phys.ethz.ch}
  {\texttt{nbeisert@itp.phys.ethz.ch}}}
\hypersetup{pdfauthor={Niklas Beisert}}
\hypersetup{pdfsubject={Manual for the LaTeX2e Package childdoc}}
\date{30 December 2018, \textsf{v2.0}}
\maketitle

\begin{abstract}\noindent
\textsf{childdoc} is a \LaTeXe{} package
that enables the direct compilation
of document sections included by |\include|
to individual files.
\end{abstract}

\begingroup
\parskip0ex
\tableofcontents
\endgroup

%%%%%%%%%%%%%%%%%%%%%%%%%%%%%%%%%%%%%%%%%%%%%%%%%%%%%%%%%%%%%%%%%%%%%%%%%%%%%%%%
%%%%%%%%%%%%%%%%%%%%%%%%%%%%%%%%%%%%%%%%%%%%%%%%%%%%%%%%%%%%%%%%%%%%%%%%%%%%%%%%
\section{Introduction}

\LaTeX{} provides a mechanism to structure a large document (such as a book)
into a main file and several child files (containing the chapters)
using the |\include| command.
This mechanism is beneficial for documents
which span hundreds of pages in order to
make the source file(s) more manageable.
Moreover, compilation can be restricted to
selected child files by means of the |\includeonly| command.
The latter feature can be used to reduce the compilation time while editing
(this was significantly more useful in the earlier days of \LaTeX{})
or to generate a smaller document which is easier to navigate.
Another application of |\includeonly| is to generate
documents consisting of selected parts of the complete document.

However, there are a few drawbacks of the plain |\include| mechanism:
\begin{itemize}
\item
The child files cannot be compiled on their own,
they can only be compiled via the main file.
A naive editing environment
(such as a text editor with an option
to have the current file processed by \LaTeX)
may require one to switch to the main file before compiling;
attempting to compile the child file produces errors.
\item
The main file must be modified (each time)
to adjust the |\includeonly| command
to the present needs. This easily leaves the main file in a messy state.
\item
The generated document will always carry the filename
of the main document. This is inconvenient if
several child files are to be compiled and
to be kept for distribution.
\end{itemize}

The present package provides a simple interface
to make child files individually compilable by \LaTeX{}.
Compiling a child file then has the same effect as compiling
the main file with an |\includeonly| command
to select the appropriate child.
Moreover the generated document will carry the name of the child
rather than the main file.
This resolves all three above issues.

This feature is meant to make the editing of books,
thesis documents and lecture notes somewhat more convenient.
However, the package can also be used efficiently for
composing a series of documents (such as exercise sheets)
which are typically distributed individually.
It then assists the author in generating the individual documents
(potentially in different versions)
as well as a document containing the collected series.
Another application is in developing style files
or other kinds of included material
where compilation of the style file could redirect
to a sample or test file.

%%%%%%%%%%%%%%%%%%%%%%%%%%%%%%%%%%%%%%%%%%%%%%%%%%%%%%%%%%%%%%%%%%%%%%%%%%%%%%%%
%%%%%%%%%%%%%%%%%%%%%%%%%%%%%%%%%%%%%%%%%%%%%%%%%%%%%%%%%%%%%%%%%%%%%%%%%%%%%%%%
\section{Usage}

First of all, the package \textsf{childdoc} is \emph{not} a standard
\LaTeXe{} |.sty| style file! Therefore it needs to be invoked in
a non-standard way.

%%%%%%%%%%%%%%%%%%%%%%%%%%%%%%%%%%%%%%%%%%%%%%%%%%%%%%%%%%%%%%%%%%%%%%%%%%%%%%%%
\subsection{Included Files}
\label{sec:include}

%%%%%%%%%%%%%%%%%%%%%%%%%%%%%%%%%%%%%%%%
\DescribeMacro{\childdocmain}
To use the package, add the commands
\begin{center}
\begin{tabular}{l}
|\input{childdoc.def}|\\
|\childdocmain{}|\\
\end{tabular}
\end{center}
at the very top of the main \LaTeX{} file,
in particular \emph{before} the |\documentclass| statement!
The argument of |\childdocmain| should be left empty
(but it must be present).

%%%%%%%%%%%%%%%%%%%%%%%%%%%%%%%%%%%%%%%%
\DescribeMacro{\childdocof}
Furthermore, add the commands
\begin{center}
\begin{tabular}{l}
|\input{childdoc.def}|\\
|\childdocof{|\textit{main}|}|\\
\end{tabular}
\end{center}
at the top of every child file \textit{child}
which is included by |\include{|\textit{child}|}|
from within the main file
(or at least for those files to be compiled individually).
The argument \textit{main} must be the filename of the main file.

There are a couple of
considerations in setting up the main and child documents:

%%%%%%%%%%%%%%%%%%%%%%%%%%%%%%%%%%%%%%%%
\paragraph{Restrictions.}

Please note the following restrictions:
\begin{itemize}
\item
|\childdocmain| must be called with one argument \textit{main}
to ensure compatibility with earlier version of the package.
It must either be empty (|\childdocmain{}|)
or precisely match the filename of the main file in which it is specified.
See \secref{sec:detection} for further information.
\item
The filename \textit{main} must be specified without the |.tex| extension.
\item
The filename \textit{main} is case sensitive
(even in case-insensitive file systems)
due to internal string comparison.
\item
The argument \textit{main} should be fully expanded, it cannot be a macro.
\item
Subdirectories and special characters should be avoided in filenames.
\item
The command |\childdocmain{|\textit{main}|}| must be followed by a whitespace.
It should not be followed immediately by another command
or by a comment mark `|%|'.
This is because the \TeX{} parser reads the token immediately following
the argument of |\childdocmain| and puts it
at the beginning of every child section;
however, a white\-space is ignored.
\end{itemize}

%%%%%%%%%%%%%%%%%%%%%%%%%%%%%%%%%%%%%%%%
\paragraph{Content of Main File.}

It is advisable to place all content in the child files included by |\include|.
Any output contained in the main file will appear in all child documents
unless suppressed manually;
it cannot be suppressed automatically by the |\includeonly| directive
and thus should normally be avoided.
A method to include some content in the main file
by means of conditional processing is described in \secref{sec:conditional}.

%%%%%%%%%%%%%%%%%%%%%%%%%%%%%%%%%%%%%%%%
\paragraph{Page Numbering.}

When only a part of the document is compiled,
the appropriate numbering of pages
(as well as other status parameters)
is determined from the |.aux| files.
The latter contain information from previous passes.
However this information needs to propagate through
all intermediate child documents.
Therefore the page numbering in child documents may well
be inconsistent until the complete document is compiled at least once.

A useful (if unconventional) way to always ensure a consistent
page numbering is to restart the numbering in each child document
and denote the pages by `\textit{child}|.|\textit{page}'
where \textit{child} represents the chapter/section number of the child file.
This can be achieved by the command
|\numberwithin{page}{|\textit{child}|}|
of the \textsf{amsmath} package
where \textit{child} can be |chapter| or |section|
depending on the chosen structuring.
Alternatively, one can modify the macro |\thepage| appropriately
and reset the counter |page| at the start of each child file.

%%%%%%%%%%%%%%%%%%%%%%%%%%%%%%%%%%%%%%%%%%%%%%%%%%%%%%%%%%%%%%%%%%%%%%%%%%%%%%%%
\subsection{Conditional Processing}
\label{sec:conditional}

The package provides a mechanism to compile different versions
of a document. To customise the versions further some conditional processing
can come in handy to distinguish which version is being compiled.
The package provides two macros to describe the compilation context:

%%%%%%%%%%%%%%%%%%%%%%%%%%%%%%%%%%%%%%%%
\DescribeMacro{\ifchilddoc}
The conditional |\ifchilddoc| distinguishes between the compilation of
child documents and the main document:
%
\begin{center}
|\ifchilddoc |\textit{child-code}| |[|\||else |\textit{main-code}]| \||fi|
\end{center}

%%%%%%%%%%%%%%%%%%%%%%%%%%%%%%%%%%%%%%%%
\DescribeMacro{\childdocname}
\DescribeMacro{\childdocjob}
The macro |\childdocname| contains the filename (without extension)
of the main or child file being processed.
Note that |\childdocjob| will always contain the name of the main file.

%%%%%%%%%%%%%%%%%%%%%%%%%%%%%%%%%%%%%%%%
\paragraph{Title Page.}

Conditional processing can be used to include a title or banner page
in the main document when proper precautions are taken.
Importantly, the code in the main file should ensure that the page counter
(as well as other status parameters which are stored in the |.aux| files)
takes the same value after the conditional processing.
Otherwise the page numbers may take divergent values
depending on which part is compiled.

For example, a title page could be declared by:
%
\begin{center}
\begin{tabular}{l}
|\ifchilddoc\||else|\\
|\addtocounter{page}{-1}|\\
\textit{code for title page}\\
|\newpage|\\
|\||fi|
\end{tabular}
\end{center}
%
A banner page for the child documents can be generated by:
%
\begin{center}
\begin{tabular}{l}
|\ifchilddoc|\\
|\addtocounter{page}{-1}|\\
\textit{code for banner page}\\
|\newpage|\\
|\||fi|
\end{tabular}
\end{center}
%
Here one could write a message such as:
\begin{center}
|This is the part \childdocname{} of \childdocjob{}.|
\end{center}

%%%%%%%%%%%%%%%%%%%%%%%%%%%%%%%%%%%%%%%%%%%%%%%%%%%%%%%%%%%%%%%%%%%%%%%%%%%%%%%%
\subsection{Flags}
\label{sec:flags}

The package makes it easy to generate different versions
of the main or child documents.
To this end compilation flags can be defined
and assigned different default values.
They will be particularly useful in conjunction
with the forwarding mechanism described in \secref{sec:forward}.

For example, it may be useful to have a flag |\version|
which can be set to |draft| or |final|.
The document source will contain some conditional code
depending on the value of |\version|.
Suppose further, the flag should default to |final| for the main file
and to |draft| for child files
which is a natural assignment for editing the document.
This is achieved by placing the following code
in the preamble of the main document
(below the |\childdocmain| directive):
%
\begin{center}
\begin{tabular}{l}
|\ifchilddoc|\\
|\providecommand{\version}{draft}|\\
|\||else|\\
|\providecommand{\version}{final}|\\
|\||fi|
\end{tabular}
\end{center}
%
The definition by |\providecommand| makes sure
that previous definitions are not overwritten.
Further statements |\providecommand{\version}{...}|
can thus be added before the above code to override it.

For the main file, one might add a line
(between |\childdocmain| and the above block)
%
\begin{center}
|%\ifchilddoc\||else\providecommand{\version}{draft}\||fi|
\end{center}
%
which can be uncommented to produce a draft version.
Likewise one can add a line to the very top of a child file
(above the |\childdocof{|\textit{main}|}| directive)
%
\begin{center}
|%\providecommand{\version}{final}|
\end{center}
%
which can be uncommented to produce the final version of this child document.

%%%%%%%%%%%%%%%%%%%%%%%%%%%%%%%%%%%%%%%%%%%%%%%%%%%%%%%%%%%%%%%%%%%%%%%%%%%%%%%%
\subsection{Forwarding}
\label{sec:forward}

Different versions of the main or child documents
using compilation flags as described in \secref{sec:flags}
can be (permanently) stored in different files
for convenient compilation, viewing and distribution.
To this end, the package defines a command
to pass on compilation to a different file:

%%%%%%%%%%%%%%%%%%%%%%%%%%%%%%%%%%%%%%%%
\DescribeMacro{\childdocforward}
The command |\childdocforward| redirects processing to
another source file:
%
\begin{center}
\begin{tabular}{l}
|\input{childdoc.def}|\\
|\childdocforward[|\textit{main}|]{|\textit{dest}|}|\\
\end{tabular}
\end{center}
%
The argument \textit{dest} is the destination file
(without extension).
It should be the main file or one of the child files.
Note that further \textsf{childdoc} directives
such as |\childdocof| and |\childdocforward|
in the indicated file will be processed in this form.
The optional argument \textit{main}
passes on directly to the main file \textit{main}
while pretending to compile the child \textit{dest}.
This form behaves as if \textit{dest}
issues |\childdocof{|\textit{main}|}| right away,
and no further \textsf{childdoc} directives will be processed.

%%%%%%%%%%%%%%%%%%%%%%%%%%%%%%%%%%%%%%%%
\DescribeMacro{\...prefix}
In the alternative form |\childdocforwardprefix|,
%
\begin{center}
\begin{tabular}{l}
|\input{childdoc.def}|\\
|\childdocforwardprefix[|\textit{main}|]{|\textit{prefix}|}{|\textit{dest}|}|
\end{tabular}
\end{center}
%
the destination file is determined by a pattern
depending on the current file:
To make this work, the current file must be called
`{\textit{prefix}\hspace{0.2em}\textit{suffix}}'
with \textit{prefix} matching precisely the argument.
Processing is then passed on to the file
`{\textit{dest}\hspace{0.2em}\textit{suffix}}'.
Surely, the same effect is achieved by
directly specifying the
argument `{\textit{dest}\hspace{0.2em}\textit{suffix}}'
in the first form.
However, that requires to set up a different file
for each child. With the alternative form of the command
all these files can have exactly the same content
which simplifies setting them up and maintaining them.

For example, the following file |draft.tex|
with a compilation flag |\version| as described in \secref{sec:flags}
compiles the main document as a draft:
%
\begin{center}
\begin{tabular}{l}
|\def\version{draft}|\\
|\input{childdoc.def}|\\
|\childdocforward{|\textit{main}|}|
\end{tabular}
\end{center}
%
Likewise, the following files |final|\textit{nn}|.tex|
compile the final version of the child document
|child|\textit{nn}|.tex|:
%
\begin{center}
\begin{tabular}{l}
|\def\version{final}|\\
|\input{childdoc.def}|\\
|\childdocforwardprefix{final}{child}|
\end{tabular}
\end{center}
%

Note that when several versions of a main file and/or of each child file
are to be generated, it may be convenient to set up a |Makefile| or
shell script to automatise the process.

%%%%%%%%%%%%%%%%%%%%%%%%%%%%%%%%%%%%%%%%%%%%%%%%%%%%%%%%%%%%%%%%%%%%%%%%%%%%%%%%
\subsection{Command Line Processing}
\label{sec:commandline}

The effect of redirection files can also be achieved by invoking
the \LaTeX{} compiler with a more elaborate command line.
Most conveniently this should be done as part
of a shell script or a |Makefile|.

When using \textsf{childdoc} in the main file, the following
command lines effectively perform a redirection
(note that depending on the shell being used,
backslashes may have to be doubled: `|\|' $\to$ `|\\|'):
%
\begin{center}
|... -jobname "|\textit{target}|" |\\|"|[\textit{flags}]%
|\input{childdoc.def}\childdocforward[|\textit{main}|]{|\textit{dest}|}"|
\end{center}
%
Here \textit{target} is the name of the output file,
\textit{main} is the name of the main file
and \textit{dest} is the name of the main or child file to be processed
(all filenames without extensions).
The optional argument \textit{main} can be omitted
if \textit{main} matches \textit{dest}.
Optionally, compilation \textit{flags} can be defined via |\def| commands.
This command line makes the \TeX{} engine believe
it is compiling the file \textit{target}
whose content is specified as the latter parameter.
The provided code then forwards the processing to
\textit{main} or \textit{dest} as described in \secref{sec:forward}.

%%%%%%%%%%%%%%%%%%%%%%%%%%%%%%%%%%%%%%%%%%%%%%%%%%%%%%%%%%%%%%%%%%%%%%%%%%%%%%%%
\subsection{Include by Input}
\label{sec:input}

Including child documents by |\include| has some restrictions by design.
Most notably, the content of a child document always occupies
its own set of pages; pages cannot be shared between child documents.
Usually, this behaviour makes perfect sense
because each child document contain an essential part of the document.
However, in some situations it may be desirable to compose
a document from a collection of parts
without having mandatory page breaks between then.
For this case, the package
provides a mechanism to include parts
by |\input| which can also be processed individually.
However, by construction this mechanism
requires manual handling of the content to be output.

%%%%%%%%%%%%%%%%%%%%%%%%%%%%%%%%%%%%%%%%
\DescribeMacro{\ifchilddocmanual}
The main file should be prepared as usual, see \secref{sec:include}.
However, the document body must make a distinction
between processing of an individual part and of the main document, e.g.:
%
\begin{center}
\begin{tabular}{l}
|\ifchilddocmanual|\\
|\input{\childdocname}|\\
|\||else|\\
\textit{document body with }|\input{|\textit{part}|}|\\
|\||fi|
\end{tabular}
\end{center}
%
The conditional |\ifchilddocmanual| is true whenever
a part to be included by |\input| is being compiled,
and the name of the part is stored in |\childdocname|.

%%%%%%%%%%%%%%%%%%%%%%%%%%%%%%%%%%%%%%%%
\DescribeMacro{\childdocby}
Each part to be included by |\input| should start with:
%
\begin{center}
\begin{tabular}{l}
|\input{childdoc.def}|\\
|\childdocby{|\textit{main}|}|\\
\end{tabular}
\end{center}
%
The directive |\childdocby| is similar to |\childdocof|
described in \secref{sec:include},
but the subsequent selection of content must be done manually.
To that end, both |\ifchilddoc| and |\ifchilddocmanual|
will be true upon processing of a part,
and the name of the part is stored in |\childdocname|.
Note that |\jobname| will be set to the filename of the current part
so that each part receives an individual |.aux| file
that does not interfere with the |.aux| file(s) of the main document.
This behaviour can be altered by the alternative form
|\childdocby[*]{|\textit{main}|}| (with a non-empty optional argument)
which uses the |.aux| file of the main document
by setting |\jobname| to \textit{main}.

%%%%%%%%%%%%%%%%%%%%%%%%%%%%%%%%%%%%%%%%%%%%%%%%%%%%%%%%%%%%%%%%%%%%%%%%%%%%%%%%
\subsection{Driver Development}
\label{sec:driver}

The \textsf{childdoc} mechanism can also be use for the development
of definition files such as \LaTeX{} styles or classes.
This case differs from the above setup with multiple parts
included by |\include| in that no |\includeonly| should be invoked.
This can be achieved by starting the include file
(before |\ProvidesPackage|) with:
%
\begin{center}
\begin{tabular}{l}
|\input{childdoc.def}|\\
|\childdocforward{|\textit{main}|}|\\
\end{tabular}
\end{center}
%
or alternatively with:
%
\begin{center}
\begin{tabular}{l}
|\input{childdoc.def}|\\
|\childdocby{|\textit{main}|}|\\
\end{tabular}
\end{center}
%
Both forms have slightly different effects as described above.
The main file is prepared as usual, see \secref{sec:include}.

%%%%%%%%%%%%%%%%%%%%%%%%%%%%%%%%%%%%%%%%%%%%%%%%%%%%%%%%%%%%%%%%%%%%%%%%%%%%%%%%
\subsection{Legacy Detection}
\label{sec:detection}

The directive |\childdocmain| in the main file can detect
whether the complete document or merely a child is to be compiled
even without using the directive |\childdocof|.
This method is deprecated because it is less robust
and there is no compelling reason to use it;
it is merely provided for backward compatibility
and it may be removed in future versions.

If the detection mechanism is to be used,
it is mandatory to correctly specify
the filename of the main file as the argument of |\childdocmain|:
%
\begin{center}
\begin{tabular}{l}
|\input{childdoc.def}|\\
|\childdocmain{|\textit{main}|}|\\
\end{tabular}
\end{center}
%
If |\jobname| does not match the argument \textit{main} of |\childdocmain|,
it is assumed that |\jobname| points to the child file to be compiled.
When using |\childdocmain| with the main file specified as argument,
it suffices to start a child file
with just |\input{|\textit{main}|}|
without loading of the package and using |\childdocof|.
If instead all processing is done
with the appropriate \textsf{childdoc} directives,
the argument of \textit{main} of |\childdocmain| can be empty.

An alternative version of the command line processing described
in \secref{sec:commandline} using the detection mechanism reads:
%
\begin{center}
|... -jobname "|\textit{target}|" "|[\textit{flags}]%
[|\def\jobname{|\textit{dest}|}|]|\input{|\textit{main}|}"|
\end{center}

%%%%%%%%%%%%%%%%%%%%%%%%%%%%%%%%%%%%%%%%%%%%%%%%%%%%%%%%%%%%%%%%%%%%%%%%%%%%%%%%
\subsection{Manual Code}
\label{sec:manual}

In case one cannot be certain whether the definitions file |childdoc.def|
is installed on the target \TeX{} distribution
and one prefers not to ship it,
it is conceivable to paste a few relevant commands into the sources.

To that end, drop all statements |\input{childdoc.def}|
and perform the replacements as outlined below.
Instead of |\childdocmain{|\textit{main}|}| add the following code
to the top of the main file:
%
\begin{center}
\begin{tabular}{l}
|\||ifdefined\childdocname\endinput\||fi\newif\ifchilddoc|\\
|\edef\childdocname{\scantokens\expandafter{\jobname\noexpand}}|\\
|\def\childdocmain{|\textit{main}|}\||ifx\childdocmain\childdocname\||else|\\
|\childdoctrue\includeonly{\childdocname}\let\jobname\childdocmain\||fi|\\
\end{tabular}
\end{center}
%
Instead of |\childdocof{|\textit{main}|}| just include the main file
at the top of each child file:
%
\begin{center}
|\input{|\textit{main}|}|
\end{center}
%
A simple redirection |\childdocforward{|\textit{dest}|}| is achieved by:
%
\begin{center}
|\def\jobname{|\textit{dest}|}\input{\jobname}|
\end{center}
%
The redirection with prefix
|\childdocforwardprefix[|\textit{prefix}|]{|\textit{dest}|}|
is accomplished by:
%
\begin{center}
\begin{tabular}{l}
|{\edef\jobname{\scantokens\expandafter{\jobname\noexpand}}|\\
|\def\redirectjob |\textit{prefix}|#1~~~{\gdef\jobname{|\textit{dest}|#1}}|\\
|\expandafter\redirectjob\jobname~~~}\input{\jobname}|
\end{tabular}
\end{center}

In an alternative approach,
child documents can be compiled by a specific command line
without additional code or specific definitions:
%
\begin{center}
|... -jobname "|\textit{target}|" "|[\textit{flags}]%
|\includeonly{|\textit{dest}|}\input{|\textit{main}|}"|
\end{center}
%

%%%%%%%%%%%%%%%%%%%%%%%%%%%%%%%%%%%%%%%%%%%%%%%%%%%%%%%%%%%%%%%%%%%%%%%%%%%%%%%%
%%%%%%%%%%%%%%%%%%%%%%%%%%%%%%%%%%%%%%%%%%%%%%%%%%%%%%%%%%%%%%%%%%%%%%%%%%%%%%%%
\section{Information}

%%%%%%%%%%%%%%%%%%%%%%%%%%%%%%%%%%%%%%%%%%%%%%%%%%%%%%%%%%%%%%%%%%%%%%%%%%%%%%%%
\subsection{Copyright}

Copyright \copyright{} 2017--2018 Niklas Beisert

This work may be distributed and/or modified under the
conditions of the \LaTeX{} Project Public License, either version 1.3
of this license or (at your option) any later version.
The latest version of this license is in
  \url{http://www.latex-project.org/lppl.txt}
and version 1.3 or later is part of all distributions of \LaTeX{}
version 2005/12/01 or later.

This work has the LPPL maintenance status `maintained'.

The Current Maintainer of this work is Niklas Beisert.

This work consists of the files |README.txt|, |childdoc.ins| and |childdoc.dtx|
as well as the derived files |childdoc.def|, |cdocsamp.tex|
with |cdocsch1.tex|, |cdocsch2.tex|, |cdocspt3.tex|, |cdocspt4.tex|,
|cdocsdrf.tex|, |cdocsfn1.tex|, |cdocsfn2.tex|
as well as |childdoc.pdf|.

%%%%%%%%%%%%%%%%%%%%%%%%%%%%%%%%%%%%%%%%%%%%%%%%%%%%%%%%%%%%%%%%%%%%%%%%%%%%%%%%
\subsection{Files and Installation}

The package consists of the files:
%
\begin{center}
\begin{tabular}{ll}
    |README.txt|   & readme file \\
    |childdoc.ins| & installation file \\
    |childdoc.dtx| & source file \\
    |childdoc.def| & definition file \\
    |cdocsamp.tex| & sample main file \\
    |cdocsch1.tex| & sample include file \\
    |cdocsch2.tex| & sample include file \\
    |cdocspt3.tex| & sample part file \\
    |cdocspt4.tex| & sample part file \\
    |cdocsdrf.tex| & sample redirection file \\
    |cdocsfn1.tex| & sample redirection file \\
    |cdocsfn2.tex| & sample redirection file \\
    |childdoc.pdf| & manual
\end{tabular}
\end{center}
%
The distribution consists of the files
|README.txt|, |childdoc.ins| and |childdoc.dtx|.
%
\begin{itemize}
\item
Run (pdf)\LaTeX{} on |childdoc.dtx|
to compile the manual |childdoc.pdf| (this file).
\item
Run \LaTeX{} on |childdoc.ins| to create the definitions file |childdoc.def|
and the sample |cdocsamp.tex| with include files
|cdocsch1.tex|, |cdocsch2.tex|, |cdocspt3.tex|, |cdocspt4.tex|,
|cdocsdrf.tex|, |cdocsfn1.tex|, |cdocsfn2.tex|.
Then copy the file |childdoc.def| to an appropriate directory of your \LaTeX{}
distribution, e.g.\ \textit{texmf-root}|/tex/latex/childdoc|.
\end{itemize}

%%%%%%%%%%%%%%%%%%%%%%%%%%%%%%%%%%%%%%%%%%%%%%%%%%%%%%%%%%%%%%%%%%%%%%%%%%%%%%%%
\subsection{Related CTAN Packages}

There are several other packages which offer a similar functionality:
%
\begin{itemize}
\item
The packages
\href{http://ctan.org/pkg/docmute}{\textsf{docmute}},
\href{http://ctan.org/pkg/includex}{\textsf{includex}} and
\href{http://ctan.org/pkg/standalone}{\textsf{standalone}}
provide commands to include only the document body of
a child file thus allowing both files to be compiled individually.
\item
The packages \href{http://ctan.org/pkg/subdocs}{\textsf{subdocs}}
and \href{http://ctan.org/pkg/subfiles}{\textsf{subfiles}}
provide structures in which the main and child documents can be
encapsulated and allowing them to be compiled individually.
The inclusion mechanism is different from the conventional |\include|.
\item
The package \href{http://ctan.org/pkg/combine}{\textsf{combine}}
is an elaborate solution to combine several documents into one.
\end{itemize}
%
See also the CTAN topic \href{http://ctan.org/topic/subdocs}{\textsf{subdocs}}
for further related packages.
The present package differs from the above solutions in that
a document structure constructed with the conventional |\include| mechanism
just needs two extra commands at the top of every file
such that all constituent files can be compiled individually.

%%%%%%%%%%%%%%%%%%%%%%%%%%%%%%%%%%%%%%%%%%%%%%%%%%%%%%%%%%%%%%%%%%%%%%%%%%%%%%%%
%\subsection{Feature Suggestions}
%
%The following is a list of features which may be useful for future
%versions of this package:
%%
%\begin{itemize}
%\item
%\ldots
%\end{itemize}

%%%%%%%%%%%%%%%%%%%%%%%%%%%%%%%%%%%%%%%%%%%%%%%%%%%%%%%%%%%%%%%%%%%%%%%%%%%%%%%%
\subsection{Revision History}

%%%%%%%%%%%%%%%%%%%%%%%%%%%%%%%%%%%%%%%%
\paragraph{v2.0:} 2018/12/30

\begin{itemize}
\item
immediate forward processing
\item
added |\childdocby| mechanism
\item
manual restructured
\end{itemize}

%%%%%%%%%%%%%%%%%%%%%%%%%%%%%%%%%%%%%%%%
\paragraph{v1.6:} 2018/01/17

\begin{itemize}
\item
application for development of include files
\item
corrections to manual
\end{itemize}

%%%%%%%%%%%%%%%%%%%%%%%%%%%%%%%%%%%%%%%%
\paragraph{v1.5:} 2017/05/21

\begin{itemize}
\item
more complete structuring introduced
\item
|\childdocof| introduced
\item
|\childdoc| renamed to |\childdocmain|
\item
|\childredirect| renamed to |\childdocforward| and |\childdocforwardprefix|
and functionality expanded
\end{itemize}

%%%%%%%%%%%%%%%%%%%%%%%%%%%%%%%%%%%%%%%%
\paragraph{v1.0:} 2017/04/27

\begin{itemize}
\item
manual and install package
\item
first version published on CTAN
\end{itemize}

%%%%%%%%%%%%%%%%%%%%%%%%%%%%%%%%%%%%%%%%
\paragraph{v0.6:} 2017/04/26

\begin{itemize}
\item
redirection mechanism added
\end{itemize}

%%%%%%%%%%%%%%%%%%%%%%%%%%%%%%%%%%%%%%%%
\paragraph{v0.5:} 2017/04/26

\begin{itemize}
\item
functionality in definition file
\end{itemize}


%%%%%%%%%%%%%%%%%%%%%%%%%%%%%%%%%%%%%%%%%%%%%%%%%%%%%%%%%%%%%%%%%%%%%%%%%%%%%%%%
%%%%%%%%%%%%%%%%%%%%%%%%%%%%%%%%%%%%%%%%%%%%%%%%%%%%%%%%%%%%%%%%%%%%%%%%%%%%%%%%
%%%%%%%%%%%%%%%%%%%%%%%%%%%%%%%%%%%%%%%%%%%%%%%%%%%%%%%%%%%%%%%%%%%%%%%%%%%%%%%%
\appendix

\settowidth\MacroIndent{\rmfamily\scriptsize 000\ }

 \DocInput{childdoc.dtx}

\end{document}
%</driver>
% \fi
%
% %%%%%%%%%%%%%%%%%%%%%%%%%%%%%%%%%%%%%%%%%%%%%%%%%%%%%%%%%%%%%%%%%%%%%%%%%%%%%%
% %%%%%%%%%%%%%%%%%%%%%%%%%%%%%%%%%%%%%%%%%%%%%%%%%%%%%%%%%%%%%%%%%%%%%%%%%%%%%%
% \section{Sample}
%\iffalse
%<*samplemain>
%\fi
%
% The following presents a sample document
% with two chapters, two parts, a title page,
% a compile flag as well as three forwarding files to set the flag.
% It consists of eight |.tex| files:
% \begin{center}
% \begin{tabular}{ll}
% |cdocsamp.tex|&main file\\
% |cdocsch1.tex|&include file for chapter 1\\
% |cdocsch2.tex|&include file for chapter 2\\
% |cdocspt3.tex|&include file for part 3\\
% |cdocspt4.tex|&include file for part 4\\
% |cdocsdrf.tex|&forwarding file for main file in draft mode\\
% |cdocsfi1.tex|&forwarding file for final version of chapter 1\\
% |cdocsfi2.tex|&forwarding file for final version of chapter 2\\
% \end{tabular}
% \end{center}
% Each of the eight files can be compiled directly by the \LaTeX{} compiler.
%
% %%%%%%%%%%%%%%%%%%%%%%%%%%%%%%%%%%%%%%
% \paragraph{Main File.}
%
% The main file is called |cdocsamp.tex|.
%
% Load the \textsf{childdoc} definitions and
% declare the filename for the main document:
%    \begin{macrocode}
\input{childdoc.def}
\childdocmain{}
%    \end{macrocode}

% Optional override for |\version| flag:
%    \begin{macrocode}
%%\ifchilddoc\else\providecommand{\version}{draft}\fi
%    \end{macrocode}

% Define the default values for the |\version| flag
% (|final| for the main file and |draft| for childs):
%    \begin{macrocode}
\ifchilddoc
\providecommand{\version}{draft}
\else
\providecommand{\version}{final}
\fi
%    \end{macrocode}

% Load the standard document class:
%    \begin{macrocode}
\documentclass[12pt]{article}
%    \end{macrocode}

% Start the document body:
%    \begin{macrocode}
\begin{document}
%    \end{macrocode}

% Declare a title page.
% Print title, part of document being processed and version flag:
%    \begin{macrocode}
\addtocounter{page}{-1}
\begin{center}
{\LARGE\bfseries{}childdoc example\par}
\vspace{1cm}
\ifchilddoc
\ifchilddocmanual part\else chapter\fi:
`\childdocname' of `\childdocjob'\par
\else
main document: `\childdocjob'\par
\fi
version: \version\par
\end{center}
\newpage
%    \end{macrocode}

% Manually include selected file,
% otherwise process as usual:
%    \begin{macrocode}
\ifchilddocmanual
\section*{part `\childdocname'}
\input{\childdocname}
\else
%    \end{macrocode}

% Include the two chapters:
%    \begin{macrocode}
\include{cdocsch1}
\include{cdocsch2}
%    \end{macrocode}

% Include the two parts unless only chapters should be displayed:
%    \begin{macrocode}
\ifchilddoc\else
\section{part three}
\input{cdocspt3}
\section{part four}
\input{cdocspt4}
\fi
%    \end{macrocode}

% Process as usual until here:
%    \begin{macrocode}
\fi
%    \end{macrocode}

% End of document body:
%    \begin{macrocode}
\end{document}
%    \end{macrocode}
%\iffalse
%</samplemain>
%\fi
%
% %%%%%%%%%%%%%%%%%%%%%%%%%%%%%%%%%%%%%%
% \paragraph{Chapter Include Files.}
%
% The include files are called |cdocsch1.tex| and |cdocsch2.tex|.
%
%\iffalse
%<*samplechap1|samplechap2>
%\fi

% Optional override for |\version| flag:
%    \begin{macrocode}
%%\providecommand{\version}{final}
%    \end{macrocode}

% Include the main document:
%    \begin{macrocode}
\input{childdoc.def}
\childdocof{cdocsamp}
%    \end{macrocode}

%\iffalse
%</samplechap1|samplechap2>
%\fi
%
%\iffalse
%<*samplechap1>
%\fi
% Some text for chapter 1:
%    \begin{macrocode}
\section{one}
some text in chapter one
%    \end{macrocode}

%\iffalse
%</samplechap1>
%\fi
% Some text for chapter 2:
%\iffalse
%<*samplechap2>
%\fi
%    \begin{macrocode}
\section{two}
more text in chapter two
%    \end{macrocode}

%\iffalse
%</samplechap2>
%\fi
%
% %%%%%%%%%%%%%%%%%%%%%%%%%%%%%%%%%%%%%%
% \paragraph{Part Include Files.}
%
% The include files are called |cdocspt3.tex| and |cdocspt4.tex|.
%
%\iffalse
%<*samplepart3|samplepart4>
%\fi

% Optional override for |\version| flag:
%    \begin{macrocode}
%%\providecommand{\version}{final}
%    \end{macrocode}

% Include the main document:
%    \begin{macrocode}
\input{childdoc.def}
\childdocby{cdocsamp}
%    \end{macrocode}

%\iffalse
%</samplepart3|samplepart4>
%\fi
%
%\iffalse
%<*samplepart3>
%\fi
% Some text for part 3:
%    \begin{macrocode}
some text in part three
%    \end{macrocode}

%\iffalse
%</samplepart3>
%\fi
% Some text for part 4:
%\iffalse
%<*samplepart4>
%\fi
%    \begin{macrocode}
more text in part four
%    \end{macrocode}

%\iffalse
%</samplepart4>
%\fi
%
% %%%%%%%%%%%%%%%%%%%%%%%%%%%%%%%%%%%%%%
% \paragraph{Forwarding for a Complete Draft.}
%
% The following forwarding file |cdocsdrf.tex|
% compiles the main document in draft mode:
%\iffalse
%<*sampledraft>
%\fi
%    \begin{macrocode}
\def\version{draft}
\input{childdoc.def}
\childdocforward{cdocsamp}
%    \end{macrocode}

%\iffalse
%</sampledraft>
%\fi
%
% %%%%%%%%%%%%%%%%%%%%%%%%%%%%%%%%%%%%%%
% \paragraph{Forwarding for Final Version of the Chapters.}
%
% The following forwarding files |cdocsfn1.tex| and |cdocsfn2.tex|
% (with identical content)
% compile the final versions of the child documents
% |cdocsch1.tex| and |cdocsch2.tex|, respectively:
%\iffalse
%<*samplefinal>
%\fi
%    \begin{macrocode}
\def\version{final}
\input{childdoc.def}
\childdocforwardprefix[cdocsamp]{cdocsfn}{cdocsch}
%    \end{macrocode}

%\iffalse
%</samplefinal>
%\fi
%
% %%%%%%%%%%%%%%%%%%%%%%%%%%%%%%%%%%%%%%
% \paragraph{Command Line Processing.}
%
% The following three command lines generate the output files
% |cdocscld|, |cdocscl1| and |cdocscl2|
% which should be identical to
% |cdocsdrf|, |cdocsch1| and |cdocsfn2|, respectively:
% \begin{center}
% \begin{tabular}{l}
% |latex -jobname cdocscld \|\\
% |  "\def\version{draft}\input{childdoc.def}\childdocforward{cdocsamp}"|\\
% |latex -jobname cdocscl1 \|\\
% |  "\input{childdoc.def}\childdocforward[cdocsamp]{cdocsch1}"|\\
% |latex -jobname cdocscl2 \|\\
% |  "\def\version{final}\input{childdoc.def}\childdocforward{cdocsch2}"|
% \end{tabular}
% \end{center}
% Note that the trailing backslash on each first line
% merely continues the input to the second line
% (for convenient cut ant paste).
% Furthermore, the command |latex| can be replaced by any
% of its alternative versions such as |pdflatex|.
%
% %%%%%%%%%%%%%%%%%%%%%%%%%%%%%%%%%%%%%%%%%%%%%%%%%%%%%%%%%%%%%%%%%%%%%%%%%%%%%%
% %%%%%%%%%%%%%%%%%%%%%%%%%%%%%%%%%%%%%%%%%%%%%%%%%%%%%%%%%%%%%%%%%%%%%%%%%%%%%%
% \section{Implementation}
%\iffalse
%<*package>
%\fi
%
% This section describes the definitions file |childdoc.def|.

% The definitions cannot be loaded using |\usepackage| or |\RequirePackage|
% which has a mechanism to prevent loading a style file more than once.
% When loading the definitions by means of |\input|
% multiple instances have to be prevented manually:
%\iffalse
%This code needs to be before the `\ProvidesFile' directive
%which is defined at the beginning of this file.
%Therefore it is also placed there and commented out here.
%</package>
%<*discard>
%\fi
%    \begin{macrocode}
\ifdefined\childdocmain\endinput\fi
%    \end{macrocode}
%\iffalse
%</discard>
%<*package>
%\fi
%
% \macro{\ifchilddoc}
% \macro{\ifchilddocmanual}
% The conditional |\ifchilddoc| tells whether a
% child (true) or main (false) document is being compiled.
% The conditional |\ifchilddocmanual| tells whether
% the |\includeonly| mechanism is used (false) or
% the selection of child files must be performed manually (true).
% The definitions initialise to false:
%    \begin{macrocode}
\newif\ifchilddoc
\newif\ifchilddocmanual
%    \end{macrocode}

% \macro{\childdocname}
% \macro{\childdocjob}
% The macro |\childdocname| stores the name of the main document
% to be compiled. The macro |\childdocjob| stores the name of
% the document on which the \LaTeX{} compiler was originally invoked.
% The content of |\jobname| cannot be compared
% to filenames specified in the source due to different catcodes.
% The following code rescans |\jobname|, stores the result
% in |\childdocname| and saves a copy in |\childdocjob|:
%    \begin{macrocode}
\edef\childdocname{\scantokens\expandafter{\jobname\noexpand}}
\let\childdocjob\childdocname
%    \end{macrocode}

% \macro{\childdocdisable}
% The macro |\childdocdisable| prevents the main file
% from being processed more than once.
% At this stage, the main document command |\childdocmain|
% is assumed to be called once again where it should do nothing.
% Any subsequent call to it should prevent
% a secondary processing of the main document
% It overwrites the forwarding commands
% |\childdocof| and |\childdocforward|
% with empty macros to prevent further inclusions of the main document:
%    \begin{macrocode}
\newcommand{\childdocdisable}
{
  \renewcommand{\childdocmain}[1]{\renewcommand{\childdocmain}[1]{\endinput}}
  \renewcommand{\childdocof}[1]{}
  \renewcommand{\childdocby}[2][]{}
  \renewcommand{\childdocforward}[2][]{}
  \renewcommand{\childdocdisable}{}
}
%    \end{macrocode}

% \macro{\childdocmain}
% The macro |\childdocmain| is to be called at the top of the main file
% with nothing or the main filename (without extension) as argument.
% First, it breaks loops.
% If the argument is not empty and does not match |\childdocname|
% (which is set by the first inclusion of |childdoc.def|),
% |\ifchilddoc| is set to true, |\includeonly| is applied to the child file
% and |\jobname| is set to the main file
% (for proper handling of |.aux| files):
%    \begin{macrocode}
\newcommand{\childdocmain}[1]
{
  \childdocdisable\childdocmain{}
  \if?#1?\else
    \begingroup
      \def\childdoctmp{#1}
      \ifx\childdoctmp\childdocname
        \def\childdoctmp{}
      \else
        \def\childdoctmp
        {
          \childdoctrue
          \includeonly{\childdocname}
          \def\childdocjob{#1}
          \def\jobname{#1}
        }
      \fi
      \expandafter
    \endgroup
    \childdoctmp
  \fi
}
%    \end{macrocode}

% \macro{\childdocof}
% The command |\childdocof| redirects
% compilation to the main file |#1|.
%    \begin{macrocode}
\newcommand{\childdocof}[1]
{
  \childdocdisable
  \childdoctrue
  \includeonly{\childdocname}
  \def\jobname{#1}
  \def\childdocjob{#1}
  \input{#1}
}
%    \end{macrocode}

% \macro{\childdocby}
% The command |\childdocby| ....
%    \begin{macrocode}
\newcommand{\childdocby}[2][]
{
  \childdocdisable
  \childdoctrue
  \childdocmanualtrue
  \if?#1?\else
    \def\jobname{#2}
  \fi
  \def\childdocjob{#2}
  \input{#2}
  \endinput
}
%    \end{macrocode}

% \macro{\childdocforward}
% The command |\childdocforward| redirects
% compilation to the main file or
% (if the optional argument is given) a child file.
% Parameters are set as if the main file
% or a child file starting with |\childdocof| was compiled.
% Then compilation is handed over to the main file:
%    \begin{macrocode}
\newcommand{\childdocforward}[2][]
{
  \begingroup
    \if?#1?
      \def\childdoctmp
      {
        \def\childdocname{#2}
        \def\childdocjob{#2}
        \def\jobname{#2}
        \input{#2}
        \endinput
      }
    \else
      \def\childdoctmp
      {
        \childdocdisable
        \def\childdocname{#2}
        \childdoctrue
        \includeonly{#2}
        \def\childdocjob{#1}
        \def\jobname{#1}
        \input{#1}
        \endinput
      }
    \fi
    \expandafter
  \endgroup
  \childdoctmp
}
%    \end{macrocode}

% \macro{\childdocforwardprefix}
% The command |\childdocforwardprefix| redirects
% compilation to the main or a child file by means of a pattern.
% The prefix |#1| in the current filename is replaced by |#2|
% and the suffix of the current filename is kept
% (it is assumed that the filename does not contain the substring `|~~~|'
% which is used as a delimiter).
% Compilation is handed over to the new file by |\childdocforward|:
%    \begin{macrocode}
\newcommand{\childdocforwardprefix}[3][]
{
  \begingroup
    \def\childdocextract #2##1~~~{\def\childdoctmp{\childdocforward[#1]{#3##1}}}
    \expandafter\childdocextract\childdocname~~~
    \expandafter
  \endgroup
  \childdoctmp
}
%    \end{macrocode}

% \macro{\childdoc}
% The deprecated macro |\childdoc| is a legacy version of |\childdocmain|:
%    \begin{macrocode}
\newcommand{\childdoc}{\childdocmain}
%    \end{macrocode}

% \macro{\childdocredirect}
% The deprecated macro |\childdocredirect| is a legacy version
% of |\childdocforward| and |\childdocforwardprefix|:
%    \begin{macrocode}
\newcommand{\childdocredirect}[2][]
{
  \begingroup
    \if?#1?
      \def\childdoctmp{\childdocforward{#2}}
    \else
      \def\childdoctmp{\childdocforwardprefix{#1}{#2}}
    \fi
    \expandafter
  \endgroup
  \childdoctmp
}
%    \end{macrocode}

%\iffalse
%</package>
%\fi
%
\endinput
\childdocforward[|\textit{main}|]{|\textit{dest}|}"|
\end{center}
%
Here \textit{target} is the name of the output file,
\textit{main} is the name of the main file
and \textit{dest} is the name of the main or child file to be processed
(all filenames without extensions).
The optional argument \textit{main} can be omitted
if \textit{main} matches \textit{dest}.
Optionally, compilation \textit{flags} can be defined via |\def| commands.
This command line makes the \TeX{} engine believe
it is compiling the file \textit{target}
whose content is specified as the latter parameter.
The provided code then forwards the processing to
\textit{main} or \textit{dest} as described in \secref{sec:forward}.

%%%%%%%%%%%%%%%%%%%%%%%%%%%%%%%%%%%%%%%%%%%%%%%%%%%%%%%%%%%%%%%%%%%%%%%%%%%%%%%%
\subsection{Include by Input}
\label{sec:input}

Including child documents by |\include| has some restrictions by design.
Most notably, the content of a child document always occupies
its own set of pages; pages cannot be shared between child documents.
Usually, this behaviour makes perfect sense
because each child document contain an essential part of the document.
However, in some situations it may be desirable to compose
a document from a collection of parts
without having mandatory page breaks between then.
For this case, the package
provides a mechanism to include parts
by |\input| which can also be processed individually.
However, by construction this mechanism
requires manual handling of the content to be output.

%%%%%%%%%%%%%%%%%%%%%%%%%%%%%%%%%%%%%%%%
\DescribeMacro{\ifchilddocmanual}
The main file should be prepared as usual, see \secref{sec:include}.
However, the document body must make a distinction
between processing of an individual part and of the main document, e.g.:
%
\begin{center}
\begin{tabular}{l}
|\ifchilddocmanual|\\
|\input{\childdocname}|\\
|\||else|\\
\textit{document body with }|\input{|\textit{part}|}|\\
|\||fi|
\end{tabular}
\end{center}
%
The conditional |\ifchilddocmanual| is true whenever
a part to be included by |\input| is being compiled,
and the name of the part is stored in |\childdocname|.

%%%%%%%%%%%%%%%%%%%%%%%%%%%%%%%%%%%%%%%%
\DescribeMacro{\childdocby}
Each part to be included by |\input| should start with:
%
\begin{center}
\begin{tabular}{l}
|% \iffalse
%
% childdoc.dtx Copyright (C) 2017-2018 Niklas Beisert
%
% This work may be distributed and/or modified under the
% conditions of the LaTeX Project Public License, either version 1.3
% of this license or (at your option) any later version.
% The latest version of this license is in
%   http://www.latex-project.org/lppl.txt
% and version 1.3 or later is part of all distributions of LaTeX
% version 2005/12/01 or later.
%
% This work has the LPPL maintenance status `maintained'.
%
% The Current Maintainer of this work is Niklas Beisert.
%
% This work consists of the files childdoc.dtx and childdoc.ins
% and the derived files childdoc.def and cdocsamp.tex with
% cdocsch1.tex, cdocsch2.tex, cdocsdrf.tex, cdocsfn1.tex, cdocsfn2.tex.
%
%<package>\ifdefined\childdocmain\endinput\fi
%<package>\ProvidesFile{childdoc.def}[2018/12/30 v2.0 child document driver]
%<samplemain>\ProvidesFile{cdocsamp.tex}[2018/12/30 v2.0 sample for childdoc]
%<*driver>
%\ProvidesFile{childdoc.drv}[2018/12/30 v2.0 childdoc reference manual file]
\PassOptionsToClass{10pt,a4paper}{article}
\documentclass{ltxdoc}

\usepackage[margin=35mm]{geometry}
\usepackage{hyperref}
\usepackage{hyperxmp}
\usepackage[usenames]{color}

\hypersetup{colorlinks=true}
\hypersetup{pdfstartview=FitH}
\hypersetup{pdfpagemode=UseNone}
\hypersetup{pdfsource={}}
\hypersetup{pdflang={en-UK}}
\hypersetup{pdfcopyright={Copyright 2017-2018 Niklas Beisert.
  This work may be distributed and/or modified under the
  conditions of the LaTeX Project Public License, either version 1.3
  of this license or (at your option) any later version.}}
\hypersetup{pdflicenseurl={http://www.latex-project.org/lppl.txt}}
\hypersetup{pdfcontactaddress={ETH Zurich, ITP, HIT K,
  Wolfgang-Pauli-Strasse 27}}
\hypersetup{pdfcontactpostcode={8093}}
\hypersetup{pdfcontactcity={Zurich}}
\hypersetup{pdfcontactcountry={Switzerland}}
\hypersetup{pdfcontactemail={nbeisert@itp.phys.ethz.ch}}
\hypersetup{pdfcontacturl={http://people.phys.ethz.ch/\xmptilde nbeisert/}}

\newcommand{\secref}[1]{\hyperref[#1]{section \ref*{#1}}}

\parskip1ex
\parindent0pt
\let\olditemize\itemize
\def\itemize{\olditemize\parskip0pt}

\begin{document}

\title{The \textsf{childdoc} Package}
\hypersetup{pdftitle={The childdoc Package}}
\author{Niklas Beisert\\[2ex]
  Institut f\"ur Theoretische Physik\\
  Eidgen\"ossische Technische Hochschule Z\"urich\\
  Wolfgang-Pauli-Strasse 27, 8093 Z\"urich, Switzerland\\[1ex]
  \href{mailto:nbeisert@itp.phys.ethz.ch}
  {\texttt{nbeisert@itp.phys.ethz.ch}}}
\hypersetup{pdfauthor={Niklas Beisert}}
\hypersetup{pdfsubject={Manual for the LaTeX2e Package childdoc}}
\date{30 December 2018, \textsf{v2.0}}
\maketitle

\begin{abstract}\noindent
\textsf{childdoc} is a \LaTeXe{} package
that enables the direct compilation
of document sections included by |\include|
to individual files.
\end{abstract}

\begingroup
\parskip0ex
\tableofcontents
\endgroup

%%%%%%%%%%%%%%%%%%%%%%%%%%%%%%%%%%%%%%%%%%%%%%%%%%%%%%%%%%%%%%%%%%%%%%%%%%%%%%%%
%%%%%%%%%%%%%%%%%%%%%%%%%%%%%%%%%%%%%%%%%%%%%%%%%%%%%%%%%%%%%%%%%%%%%%%%%%%%%%%%
\section{Introduction}

\LaTeX{} provides a mechanism to structure a large document (such as a book)
into a main file and several child files (containing the chapters)
using the |\include| command.
This mechanism is beneficial for documents
which span hundreds of pages in order to
make the source file(s) more manageable.
Moreover, compilation can be restricted to
selected child files by means of the |\includeonly| command.
The latter feature can be used to reduce the compilation time while editing
(this was significantly more useful in the earlier days of \LaTeX{})
or to generate a smaller document which is easier to navigate.
Another application of |\includeonly| is to generate
documents consisting of selected parts of the complete document.

However, there are a few drawbacks of the plain |\include| mechanism:
\begin{itemize}
\item
The child files cannot be compiled on their own,
they can only be compiled via the main file.
A naive editing environment
(such as a text editor with an option
to have the current file processed by \LaTeX)
may require one to switch to the main file before compiling;
attempting to compile the child file produces errors.
\item
The main file must be modified (each time)
to adjust the |\includeonly| command
to the present needs. This easily leaves the main file in a messy state.
\item
The generated document will always carry the filename
of the main document. This is inconvenient if
several child files are to be compiled and
to be kept for distribution.
\end{itemize}

The present package provides a simple interface
to make child files individually compilable by \LaTeX{}.
Compiling a child file then has the same effect as compiling
the main file with an |\includeonly| command
to select the appropriate child.
Moreover the generated document will carry the name of the child
rather than the main file.
This resolves all three above issues.

This feature is meant to make the editing of books,
thesis documents and lecture notes somewhat more convenient.
However, the package can also be used efficiently for
composing a series of documents (such as exercise sheets)
which are typically distributed individually.
It then assists the author in generating the individual documents
(potentially in different versions)
as well as a document containing the collected series.
Another application is in developing style files
or other kinds of included material
where compilation of the style file could redirect
to a sample or test file.

%%%%%%%%%%%%%%%%%%%%%%%%%%%%%%%%%%%%%%%%%%%%%%%%%%%%%%%%%%%%%%%%%%%%%%%%%%%%%%%%
%%%%%%%%%%%%%%%%%%%%%%%%%%%%%%%%%%%%%%%%%%%%%%%%%%%%%%%%%%%%%%%%%%%%%%%%%%%%%%%%
\section{Usage}

First of all, the package \textsf{childdoc} is \emph{not} a standard
\LaTeXe{} |.sty| style file! Therefore it needs to be invoked in
a non-standard way.

%%%%%%%%%%%%%%%%%%%%%%%%%%%%%%%%%%%%%%%%%%%%%%%%%%%%%%%%%%%%%%%%%%%%%%%%%%%%%%%%
\subsection{Included Files}
\label{sec:include}

%%%%%%%%%%%%%%%%%%%%%%%%%%%%%%%%%%%%%%%%
\DescribeMacro{\childdocmain}
To use the package, add the commands
\begin{center}
\begin{tabular}{l}
|\input{childdoc.def}|\\
|\childdocmain{}|\\
\end{tabular}
\end{center}
at the very top of the main \LaTeX{} file,
in particular \emph{before} the |\documentclass| statement!
The argument of |\childdocmain| should be left empty
(but it must be present).

%%%%%%%%%%%%%%%%%%%%%%%%%%%%%%%%%%%%%%%%
\DescribeMacro{\childdocof}
Furthermore, add the commands
\begin{center}
\begin{tabular}{l}
|\input{childdoc.def}|\\
|\childdocof{|\textit{main}|}|\\
\end{tabular}
\end{center}
at the top of every child file \textit{child}
which is included by |\include{|\textit{child}|}|
from within the main file
(or at least for those files to be compiled individually).
The argument \textit{main} must be the filename of the main file.

There are a couple of
considerations in setting up the main and child documents:

%%%%%%%%%%%%%%%%%%%%%%%%%%%%%%%%%%%%%%%%
\paragraph{Restrictions.}

Please note the following restrictions:
\begin{itemize}
\item
|\childdocmain| must be called with one argument \textit{main}
to ensure compatibility with earlier version of the package.
It must either be empty (|\childdocmain{}|)
or precisely match the filename of the main file in which it is specified.
See \secref{sec:detection} for further information.
\item
The filename \textit{main} must be specified without the |.tex| extension.
\item
The filename \textit{main} is case sensitive
(even in case-insensitive file systems)
due to internal string comparison.
\item
The argument \textit{main} should be fully expanded, it cannot be a macro.
\item
Subdirectories and special characters should be avoided in filenames.
\item
The command |\childdocmain{|\textit{main}|}| must be followed by a whitespace.
It should not be followed immediately by another command
or by a comment mark `|%|'.
This is because the \TeX{} parser reads the token immediately following
the argument of |\childdocmain| and puts it
at the beginning of every child section;
however, a white\-space is ignored.
\end{itemize}

%%%%%%%%%%%%%%%%%%%%%%%%%%%%%%%%%%%%%%%%
\paragraph{Content of Main File.}

It is advisable to place all content in the child files included by |\include|.
Any output contained in the main file will appear in all child documents
unless suppressed manually;
it cannot be suppressed automatically by the |\includeonly| directive
and thus should normally be avoided.
A method to include some content in the main file
by means of conditional processing is described in \secref{sec:conditional}.

%%%%%%%%%%%%%%%%%%%%%%%%%%%%%%%%%%%%%%%%
\paragraph{Page Numbering.}

When only a part of the document is compiled,
the appropriate numbering of pages
(as well as other status parameters)
is determined from the |.aux| files.
The latter contain information from previous passes.
However this information needs to propagate through
all intermediate child documents.
Therefore the page numbering in child documents may well
be inconsistent until the complete document is compiled at least once.

A useful (if unconventional) way to always ensure a consistent
page numbering is to restart the numbering in each child document
and denote the pages by `\textit{child}|.|\textit{page}'
where \textit{child} represents the chapter/section number of the child file.
This can be achieved by the command
|\numberwithin{page}{|\textit{child}|}|
of the \textsf{amsmath} package
where \textit{child} can be |chapter| or |section|
depending on the chosen structuring.
Alternatively, one can modify the macro |\thepage| appropriately
and reset the counter |page| at the start of each child file.

%%%%%%%%%%%%%%%%%%%%%%%%%%%%%%%%%%%%%%%%%%%%%%%%%%%%%%%%%%%%%%%%%%%%%%%%%%%%%%%%
\subsection{Conditional Processing}
\label{sec:conditional}

The package provides a mechanism to compile different versions
of a document. To customise the versions further some conditional processing
can come in handy to distinguish which version is being compiled.
The package provides two macros to describe the compilation context:

%%%%%%%%%%%%%%%%%%%%%%%%%%%%%%%%%%%%%%%%
\DescribeMacro{\ifchilddoc}
The conditional |\ifchilddoc| distinguishes between the compilation of
child documents and the main document:
%
\begin{center}
|\ifchilddoc |\textit{child-code}| |[|\||else |\textit{main-code}]| \||fi|
\end{center}

%%%%%%%%%%%%%%%%%%%%%%%%%%%%%%%%%%%%%%%%
\DescribeMacro{\childdocname}
\DescribeMacro{\childdocjob}
The macro |\childdocname| contains the filename (without extension)
of the main or child file being processed.
Note that |\childdocjob| will always contain the name of the main file.

%%%%%%%%%%%%%%%%%%%%%%%%%%%%%%%%%%%%%%%%
\paragraph{Title Page.}

Conditional processing can be used to include a title or banner page
in the main document when proper precautions are taken.
Importantly, the code in the main file should ensure that the page counter
(as well as other status parameters which are stored in the |.aux| files)
takes the same value after the conditional processing.
Otherwise the page numbers may take divergent values
depending on which part is compiled.

For example, a title page could be declared by:
%
\begin{center}
\begin{tabular}{l}
|\ifchilddoc\||else|\\
|\addtocounter{page}{-1}|\\
\textit{code for title page}\\
|\newpage|\\
|\||fi|
\end{tabular}
\end{center}
%
A banner page for the child documents can be generated by:
%
\begin{center}
\begin{tabular}{l}
|\ifchilddoc|\\
|\addtocounter{page}{-1}|\\
\textit{code for banner page}\\
|\newpage|\\
|\||fi|
\end{tabular}
\end{center}
%
Here one could write a message such as:
\begin{center}
|This is the part \childdocname{} of \childdocjob{}.|
\end{center}

%%%%%%%%%%%%%%%%%%%%%%%%%%%%%%%%%%%%%%%%%%%%%%%%%%%%%%%%%%%%%%%%%%%%%%%%%%%%%%%%
\subsection{Flags}
\label{sec:flags}

The package makes it easy to generate different versions
of the main or child documents.
To this end compilation flags can be defined
and assigned different default values.
They will be particularly useful in conjunction
with the forwarding mechanism described in \secref{sec:forward}.

For example, it may be useful to have a flag |\version|
which can be set to |draft| or |final|.
The document source will contain some conditional code
depending on the value of |\version|.
Suppose further, the flag should default to |final| for the main file
and to |draft| for child files
which is a natural assignment for editing the document.
This is achieved by placing the following code
in the preamble of the main document
(below the |\childdocmain| directive):
%
\begin{center}
\begin{tabular}{l}
|\ifchilddoc|\\
|\providecommand{\version}{draft}|\\
|\||else|\\
|\providecommand{\version}{final}|\\
|\||fi|
\end{tabular}
\end{center}
%
The definition by |\providecommand| makes sure
that previous definitions are not overwritten.
Further statements |\providecommand{\version}{...}|
can thus be added before the above code to override it.

For the main file, one might add a line
(between |\childdocmain| and the above block)
%
\begin{center}
|%\ifchilddoc\||else\providecommand{\version}{draft}\||fi|
\end{center}
%
which can be uncommented to produce a draft version.
Likewise one can add a line to the very top of a child file
(above the |\childdocof{|\textit{main}|}| directive)
%
\begin{center}
|%\providecommand{\version}{final}|
\end{center}
%
which can be uncommented to produce the final version of this child document.

%%%%%%%%%%%%%%%%%%%%%%%%%%%%%%%%%%%%%%%%%%%%%%%%%%%%%%%%%%%%%%%%%%%%%%%%%%%%%%%%
\subsection{Forwarding}
\label{sec:forward}

Different versions of the main or child documents
using compilation flags as described in \secref{sec:flags}
can be (permanently) stored in different files
for convenient compilation, viewing and distribution.
To this end, the package defines a command
to pass on compilation to a different file:

%%%%%%%%%%%%%%%%%%%%%%%%%%%%%%%%%%%%%%%%
\DescribeMacro{\childdocforward}
The command |\childdocforward| redirects processing to
another source file:
%
\begin{center}
\begin{tabular}{l}
|\input{childdoc.def}|\\
|\childdocforward[|\textit{main}|]{|\textit{dest}|}|\\
\end{tabular}
\end{center}
%
The argument \textit{dest} is the destination file
(without extension).
It should be the main file or one of the child files.
Note that further \textsf{childdoc} directives
such as |\childdocof| and |\childdocforward|
in the indicated file will be processed in this form.
The optional argument \textit{main}
passes on directly to the main file \textit{main}
while pretending to compile the child \textit{dest}.
This form behaves as if \textit{dest}
issues |\childdocof{|\textit{main}|}| right away,
and no further \textsf{childdoc} directives will be processed.

%%%%%%%%%%%%%%%%%%%%%%%%%%%%%%%%%%%%%%%%
\DescribeMacro{\...prefix}
In the alternative form |\childdocforwardprefix|,
%
\begin{center}
\begin{tabular}{l}
|\input{childdoc.def}|\\
|\childdocforwardprefix[|\textit{main}|]{|\textit{prefix}|}{|\textit{dest}|}|
\end{tabular}
\end{center}
%
the destination file is determined by a pattern
depending on the current file:
To make this work, the current file must be called
`{\textit{prefix}\hspace{0.2em}\textit{suffix}}'
with \textit{prefix} matching precisely the argument.
Processing is then passed on to the file
`{\textit{dest}\hspace{0.2em}\textit{suffix}}'.
Surely, the same effect is achieved by
directly specifying the
argument `{\textit{dest}\hspace{0.2em}\textit{suffix}}'
in the first form.
However, that requires to set up a different file
for each child. With the alternative form of the command
all these files can have exactly the same content
which simplifies setting them up and maintaining them.

For example, the following file |draft.tex|
with a compilation flag |\version| as described in \secref{sec:flags}
compiles the main document as a draft:
%
\begin{center}
\begin{tabular}{l}
|\def\version{draft}|\\
|\input{childdoc.def}|\\
|\childdocforward{|\textit{main}|}|
\end{tabular}
\end{center}
%
Likewise, the following files |final|\textit{nn}|.tex|
compile the final version of the child document
|child|\textit{nn}|.tex|:
%
\begin{center}
\begin{tabular}{l}
|\def\version{final}|\\
|\input{childdoc.def}|\\
|\childdocforwardprefix{final}{child}|
\end{tabular}
\end{center}
%

Note that when several versions of a main file and/or of each child file
are to be generated, it may be convenient to set up a |Makefile| or
shell script to automatise the process.

%%%%%%%%%%%%%%%%%%%%%%%%%%%%%%%%%%%%%%%%%%%%%%%%%%%%%%%%%%%%%%%%%%%%%%%%%%%%%%%%
\subsection{Command Line Processing}
\label{sec:commandline}

The effect of redirection files can also be achieved by invoking
the \LaTeX{} compiler with a more elaborate command line.
Most conveniently this should be done as part
of a shell script or a |Makefile|.

When using \textsf{childdoc} in the main file, the following
command lines effectively perform a redirection
(note that depending on the shell being used,
backslashes may have to be doubled: `|\|' $\to$ `|\\|'):
%
\begin{center}
|... -jobname "|\textit{target}|" |\\|"|[\textit{flags}]%
|\input{childdoc.def}\childdocforward[|\textit{main}|]{|\textit{dest}|}"|
\end{center}
%
Here \textit{target} is the name of the output file,
\textit{main} is the name of the main file
and \textit{dest} is the name of the main or child file to be processed
(all filenames without extensions).
The optional argument \textit{main} can be omitted
if \textit{main} matches \textit{dest}.
Optionally, compilation \textit{flags} can be defined via |\def| commands.
This command line makes the \TeX{} engine believe
it is compiling the file \textit{target}
whose content is specified as the latter parameter.
The provided code then forwards the processing to
\textit{main} or \textit{dest} as described in \secref{sec:forward}.

%%%%%%%%%%%%%%%%%%%%%%%%%%%%%%%%%%%%%%%%%%%%%%%%%%%%%%%%%%%%%%%%%%%%%%%%%%%%%%%%
\subsection{Include by Input}
\label{sec:input}

Including child documents by |\include| has some restrictions by design.
Most notably, the content of a child document always occupies
its own set of pages; pages cannot be shared between child documents.
Usually, this behaviour makes perfect sense
because each child document contain an essential part of the document.
However, in some situations it may be desirable to compose
a document from a collection of parts
without having mandatory page breaks between then.
For this case, the package
provides a mechanism to include parts
by |\input| which can also be processed individually.
However, by construction this mechanism
requires manual handling of the content to be output.

%%%%%%%%%%%%%%%%%%%%%%%%%%%%%%%%%%%%%%%%
\DescribeMacro{\ifchilddocmanual}
The main file should be prepared as usual, see \secref{sec:include}.
However, the document body must make a distinction
between processing of an individual part and of the main document, e.g.:
%
\begin{center}
\begin{tabular}{l}
|\ifchilddocmanual|\\
|\input{\childdocname}|\\
|\||else|\\
\textit{document body with }|\input{|\textit{part}|}|\\
|\||fi|
\end{tabular}
\end{center}
%
The conditional |\ifchilddocmanual| is true whenever
a part to be included by |\input| is being compiled,
and the name of the part is stored in |\childdocname|.

%%%%%%%%%%%%%%%%%%%%%%%%%%%%%%%%%%%%%%%%
\DescribeMacro{\childdocby}
Each part to be included by |\input| should start with:
%
\begin{center}
\begin{tabular}{l}
|\input{childdoc.def}|\\
|\childdocby{|\textit{main}|}|\\
\end{tabular}
\end{center}
%
The directive |\childdocby| is similar to |\childdocof|
described in \secref{sec:include},
but the subsequent selection of content must be done manually.
To that end, both |\ifchilddoc| and |\ifchilddocmanual|
will be true upon processing of a part,
and the name of the part is stored in |\childdocname|.
Note that |\jobname| will be set to the filename of the current part
so that each part receives an individual |.aux| file
that does not interfere with the |.aux| file(s) of the main document.
This behaviour can be altered by the alternative form
|\childdocby[*]{|\textit{main}|}| (with a non-empty optional argument)
which uses the |.aux| file of the main document
by setting |\jobname| to \textit{main}.

%%%%%%%%%%%%%%%%%%%%%%%%%%%%%%%%%%%%%%%%%%%%%%%%%%%%%%%%%%%%%%%%%%%%%%%%%%%%%%%%
\subsection{Driver Development}
\label{sec:driver}

The \textsf{childdoc} mechanism can also be use for the development
of definition files such as \LaTeX{} styles or classes.
This case differs from the above setup with multiple parts
included by |\include| in that no |\includeonly| should be invoked.
This can be achieved by starting the include file
(before |\ProvidesPackage|) with:
%
\begin{center}
\begin{tabular}{l}
|\input{childdoc.def}|\\
|\childdocforward{|\textit{main}|}|\\
\end{tabular}
\end{center}
%
or alternatively with:
%
\begin{center}
\begin{tabular}{l}
|\input{childdoc.def}|\\
|\childdocby{|\textit{main}|}|\\
\end{tabular}
\end{center}
%
Both forms have slightly different effects as described above.
The main file is prepared as usual, see \secref{sec:include}.

%%%%%%%%%%%%%%%%%%%%%%%%%%%%%%%%%%%%%%%%%%%%%%%%%%%%%%%%%%%%%%%%%%%%%%%%%%%%%%%%
\subsection{Legacy Detection}
\label{sec:detection}

The directive |\childdocmain| in the main file can detect
whether the complete document or merely a child is to be compiled
even without using the directive |\childdocof|.
This method is deprecated because it is less robust
and there is no compelling reason to use it;
it is merely provided for backward compatibility
and it may be removed in future versions.

If the detection mechanism is to be used,
it is mandatory to correctly specify
the filename of the main file as the argument of |\childdocmain|:
%
\begin{center}
\begin{tabular}{l}
|\input{childdoc.def}|\\
|\childdocmain{|\textit{main}|}|\\
\end{tabular}
\end{center}
%
If |\jobname| does not match the argument \textit{main} of |\childdocmain|,
it is assumed that |\jobname| points to the child file to be compiled.
When using |\childdocmain| with the main file specified as argument,
it suffices to start a child file
with just |\input{|\textit{main}|}|
without loading of the package and using |\childdocof|.
If instead all processing is done
with the appropriate \textsf{childdoc} directives,
the argument of \textit{main} of |\childdocmain| can be empty.

An alternative version of the command line processing described
in \secref{sec:commandline} using the detection mechanism reads:
%
\begin{center}
|... -jobname "|\textit{target}|" "|[\textit{flags}]%
[|\def\jobname{|\textit{dest}|}|]|\input{|\textit{main}|}"|
\end{center}

%%%%%%%%%%%%%%%%%%%%%%%%%%%%%%%%%%%%%%%%%%%%%%%%%%%%%%%%%%%%%%%%%%%%%%%%%%%%%%%%
\subsection{Manual Code}
\label{sec:manual}

In case one cannot be certain whether the definitions file |childdoc.def|
is installed on the target \TeX{} distribution
and one prefers not to ship it,
it is conceivable to paste a few relevant commands into the sources.

To that end, drop all statements |\input{childdoc.def}|
and perform the replacements as outlined below.
Instead of |\childdocmain{|\textit{main}|}| add the following code
to the top of the main file:
%
\begin{center}
\begin{tabular}{l}
|\||ifdefined\childdocname\endinput\||fi\newif\ifchilddoc|\\
|\edef\childdocname{\scantokens\expandafter{\jobname\noexpand}}|\\
|\def\childdocmain{|\textit{main}|}\||ifx\childdocmain\childdocname\||else|\\
|\childdoctrue\includeonly{\childdocname}\let\jobname\childdocmain\||fi|\\
\end{tabular}
\end{center}
%
Instead of |\childdocof{|\textit{main}|}| just include the main file
at the top of each child file:
%
\begin{center}
|\input{|\textit{main}|}|
\end{center}
%
A simple redirection |\childdocforward{|\textit{dest}|}| is achieved by:
%
\begin{center}
|\def\jobname{|\textit{dest}|}\input{\jobname}|
\end{center}
%
The redirection with prefix
|\childdocforwardprefix[|\textit{prefix}|]{|\textit{dest}|}|
is accomplished by:
%
\begin{center}
\begin{tabular}{l}
|{\edef\jobname{\scantokens\expandafter{\jobname\noexpand}}|\\
|\def\redirectjob |\textit{prefix}|#1~~~{\gdef\jobname{|\textit{dest}|#1}}|\\
|\expandafter\redirectjob\jobname~~~}\input{\jobname}|
\end{tabular}
\end{center}

In an alternative approach,
child documents can be compiled by a specific command line
without additional code or specific definitions:
%
\begin{center}
|... -jobname "|\textit{target}|" "|[\textit{flags}]%
|\includeonly{|\textit{dest}|}\input{|\textit{main}|}"|
\end{center}
%

%%%%%%%%%%%%%%%%%%%%%%%%%%%%%%%%%%%%%%%%%%%%%%%%%%%%%%%%%%%%%%%%%%%%%%%%%%%%%%%%
%%%%%%%%%%%%%%%%%%%%%%%%%%%%%%%%%%%%%%%%%%%%%%%%%%%%%%%%%%%%%%%%%%%%%%%%%%%%%%%%
\section{Information}

%%%%%%%%%%%%%%%%%%%%%%%%%%%%%%%%%%%%%%%%%%%%%%%%%%%%%%%%%%%%%%%%%%%%%%%%%%%%%%%%
\subsection{Copyright}

Copyright \copyright{} 2017--2018 Niklas Beisert

This work may be distributed and/or modified under the
conditions of the \LaTeX{} Project Public License, either version 1.3
of this license or (at your option) any later version.
The latest version of this license is in
  \url{http://www.latex-project.org/lppl.txt}
and version 1.3 or later is part of all distributions of \LaTeX{}
version 2005/12/01 or later.

This work has the LPPL maintenance status `maintained'.

The Current Maintainer of this work is Niklas Beisert.

This work consists of the files |README.txt|, |childdoc.ins| and |childdoc.dtx|
as well as the derived files |childdoc.def|, |cdocsamp.tex|
with |cdocsch1.tex|, |cdocsch2.tex|, |cdocspt3.tex|, |cdocspt4.tex|,
|cdocsdrf.tex|, |cdocsfn1.tex|, |cdocsfn2.tex|
as well as |childdoc.pdf|.

%%%%%%%%%%%%%%%%%%%%%%%%%%%%%%%%%%%%%%%%%%%%%%%%%%%%%%%%%%%%%%%%%%%%%%%%%%%%%%%%
\subsection{Files and Installation}

The package consists of the files:
%
\begin{center}
\begin{tabular}{ll}
    |README.txt|   & readme file \\
    |childdoc.ins| & installation file \\
    |childdoc.dtx| & source file \\
    |childdoc.def| & definition file \\
    |cdocsamp.tex| & sample main file \\
    |cdocsch1.tex| & sample include file \\
    |cdocsch2.tex| & sample include file \\
    |cdocspt3.tex| & sample part file \\
    |cdocspt4.tex| & sample part file \\
    |cdocsdrf.tex| & sample redirection file \\
    |cdocsfn1.tex| & sample redirection file \\
    |cdocsfn2.tex| & sample redirection file \\
    |childdoc.pdf| & manual
\end{tabular}
\end{center}
%
The distribution consists of the files
|README.txt|, |childdoc.ins| and |childdoc.dtx|.
%
\begin{itemize}
\item
Run (pdf)\LaTeX{} on |childdoc.dtx|
to compile the manual |childdoc.pdf| (this file).
\item
Run \LaTeX{} on |childdoc.ins| to create the definitions file |childdoc.def|
and the sample |cdocsamp.tex| with include files
|cdocsch1.tex|, |cdocsch2.tex|, |cdocspt3.tex|, |cdocspt4.tex|,
|cdocsdrf.tex|, |cdocsfn1.tex|, |cdocsfn2.tex|.
Then copy the file |childdoc.def| to an appropriate directory of your \LaTeX{}
distribution, e.g.\ \textit{texmf-root}|/tex/latex/childdoc|.
\end{itemize}

%%%%%%%%%%%%%%%%%%%%%%%%%%%%%%%%%%%%%%%%%%%%%%%%%%%%%%%%%%%%%%%%%%%%%%%%%%%%%%%%
\subsection{Related CTAN Packages}

There are several other packages which offer a similar functionality:
%
\begin{itemize}
\item
The packages
\href{http://ctan.org/pkg/docmute}{\textsf{docmute}},
\href{http://ctan.org/pkg/includex}{\textsf{includex}} and
\href{http://ctan.org/pkg/standalone}{\textsf{standalone}}
provide commands to include only the document body of
a child file thus allowing both files to be compiled individually.
\item
The packages \href{http://ctan.org/pkg/subdocs}{\textsf{subdocs}}
and \href{http://ctan.org/pkg/subfiles}{\textsf{subfiles}}
provide structures in which the main and child documents can be
encapsulated and allowing them to be compiled individually.
The inclusion mechanism is different from the conventional |\include|.
\item
The package \href{http://ctan.org/pkg/combine}{\textsf{combine}}
is an elaborate solution to combine several documents into one.
\end{itemize}
%
See also the CTAN topic \href{http://ctan.org/topic/subdocs}{\textsf{subdocs}}
for further related packages.
The present package differs from the above solutions in that
a document structure constructed with the conventional |\include| mechanism
just needs two extra commands at the top of every file
such that all constituent files can be compiled individually.

%%%%%%%%%%%%%%%%%%%%%%%%%%%%%%%%%%%%%%%%%%%%%%%%%%%%%%%%%%%%%%%%%%%%%%%%%%%%%%%%
%\subsection{Feature Suggestions}
%
%The following is a list of features which may be useful for future
%versions of this package:
%%
%\begin{itemize}
%\item
%\ldots
%\end{itemize}

%%%%%%%%%%%%%%%%%%%%%%%%%%%%%%%%%%%%%%%%%%%%%%%%%%%%%%%%%%%%%%%%%%%%%%%%%%%%%%%%
\subsection{Revision History}

%%%%%%%%%%%%%%%%%%%%%%%%%%%%%%%%%%%%%%%%
\paragraph{v2.0:} 2018/12/30

\begin{itemize}
\item
immediate forward processing
\item
added |\childdocby| mechanism
\item
manual restructured
\end{itemize}

%%%%%%%%%%%%%%%%%%%%%%%%%%%%%%%%%%%%%%%%
\paragraph{v1.6:} 2018/01/17

\begin{itemize}
\item
application for development of include files
\item
corrections to manual
\end{itemize}

%%%%%%%%%%%%%%%%%%%%%%%%%%%%%%%%%%%%%%%%
\paragraph{v1.5:} 2017/05/21

\begin{itemize}
\item
more complete structuring introduced
\item
|\childdocof| introduced
\item
|\childdoc| renamed to |\childdocmain|
\item
|\childredirect| renamed to |\childdocforward| and |\childdocforwardprefix|
and functionality expanded
\end{itemize}

%%%%%%%%%%%%%%%%%%%%%%%%%%%%%%%%%%%%%%%%
\paragraph{v1.0:} 2017/04/27

\begin{itemize}
\item
manual and install package
\item
first version published on CTAN
\end{itemize}

%%%%%%%%%%%%%%%%%%%%%%%%%%%%%%%%%%%%%%%%
\paragraph{v0.6:} 2017/04/26

\begin{itemize}
\item
redirection mechanism added
\end{itemize}

%%%%%%%%%%%%%%%%%%%%%%%%%%%%%%%%%%%%%%%%
\paragraph{v0.5:} 2017/04/26

\begin{itemize}
\item
functionality in definition file
\end{itemize}


%%%%%%%%%%%%%%%%%%%%%%%%%%%%%%%%%%%%%%%%%%%%%%%%%%%%%%%%%%%%%%%%%%%%%%%%%%%%%%%%
%%%%%%%%%%%%%%%%%%%%%%%%%%%%%%%%%%%%%%%%%%%%%%%%%%%%%%%%%%%%%%%%%%%%%%%%%%%%%%%%
%%%%%%%%%%%%%%%%%%%%%%%%%%%%%%%%%%%%%%%%%%%%%%%%%%%%%%%%%%%%%%%%%%%%%%%%%%%%%%%%
\appendix

\settowidth\MacroIndent{\rmfamily\scriptsize 000\ }

 \DocInput{childdoc.dtx}

\end{document}
%</driver>
% \fi
%
% %%%%%%%%%%%%%%%%%%%%%%%%%%%%%%%%%%%%%%%%%%%%%%%%%%%%%%%%%%%%%%%%%%%%%%%%%%%%%%
% %%%%%%%%%%%%%%%%%%%%%%%%%%%%%%%%%%%%%%%%%%%%%%%%%%%%%%%%%%%%%%%%%%%%%%%%%%%%%%
% \section{Sample}
%\iffalse
%<*samplemain>
%\fi
%
% The following presents a sample document
% with two chapters, two parts, a title page,
% a compile flag as well as three forwarding files to set the flag.
% It consists of eight |.tex| files:
% \begin{center}
% \begin{tabular}{ll}
% |cdocsamp.tex|&main file\\
% |cdocsch1.tex|&include file for chapter 1\\
% |cdocsch2.tex|&include file for chapter 2\\
% |cdocspt3.tex|&include file for part 3\\
% |cdocspt4.tex|&include file for part 4\\
% |cdocsdrf.tex|&forwarding file for main file in draft mode\\
% |cdocsfi1.tex|&forwarding file for final version of chapter 1\\
% |cdocsfi2.tex|&forwarding file for final version of chapter 2\\
% \end{tabular}
% \end{center}
% Each of the eight files can be compiled directly by the \LaTeX{} compiler.
%
% %%%%%%%%%%%%%%%%%%%%%%%%%%%%%%%%%%%%%%
% \paragraph{Main File.}
%
% The main file is called |cdocsamp.tex|.
%
% Load the \textsf{childdoc} definitions and
% declare the filename for the main document:
%    \begin{macrocode}
\input{childdoc.def}
\childdocmain{}
%    \end{macrocode}

% Optional override for |\version| flag:
%    \begin{macrocode}
%%\ifchilddoc\else\providecommand{\version}{draft}\fi
%    \end{macrocode}

% Define the default values for the |\version| flag
% (|final| for the main file and |draft| for childs):
%    \begin{macrocode}
\ifchilddoc
\providecommand{\version}{draft}
\else
\providecommand{\version}{final}
\fi
%    \end{macrocode}

% Load the standard document class:
%    \begin{macrocode}
\documentclass[12pt]{article}
%    \end{macrocode}

% Start the document body:
%    \begin{macrocode}
\begin{document}
%    \end{macrocode}

% Declare a title page.
% Print title, part of document being processed and version flag:
%    \begin{macrocode}
\addtocounter{page}{-1}
\begin{center}
{\LARGE\bfseries{}childdoc example\par}
\vspace{1cm}
\ifchilddoc
\ifchilddocmanual part\else chapter\fi:
`\childdocname' of `\childdocjob'\par
\else
main document: `\childdocjob'\par
\fi
version: \version\par
\end{center}
\newpage
%    \end{macrocode}

% Manually include selected file,
% otherwise process as usual:
%    \begin{macrocode}
\ifchilddocmanual
\section*{part `\childdocname'}
\input{\childdocname}
\else
%    \end{macrocode}

% Include the two chapters:
%    \begin{macrocode}
\include{cdocsch1}
\include{cdocsch2}
%    \end{macrocode}

% Include the two parts unless only chapters should be displayed:
%    \begin{macrocode}
\ifchilddoc\else
\section{part three}
\input{cdocspt3}
\section{part four}
\input{cdocspt4}
\fi
%    \end{macrocode}

% Process as usual until here:
%    \begin{macrocode}
\fi
%    \end{macrocode}

% End of document body:
%    \begin{macrocode}
\end{document}
%    \end{macrocode}
%\iffalse
%</samplemain>
%\fi
%
% %%%%%%%%%%%%%%%%%%%%%%%%%%%%%%%%%%%%%%
% \paragraph{Chapter Include Files.}
%
% The include files are called |cdocsch1.tex| and |cdocsch2.tex|.
%
%\iffalse
%<*samplechap1|samplechap2>
%\fi

% Optional override for |\version| flag:
%    \begin{macrocode}
%%\providecommand{\version}{final}
%    \end{macrocode}

% Include the main document:
%    \begin{macrocode}
\input{childdoc.def}
\childdocof{cdocsamp}
%    \end{macrocode}

%\iffalse
%</samplechap1|samplechap2>
%\fi
%
%\iffalse
%<*samplechap1>
%\fi
% Some text for chapter 1:
%    \begin{macrocode}
\section{one}
some text in chapter one
%    \end{macrocode}

%\iffalse
%</samplechap1>
%\fi
% Some text for chapter 2:
%\iffalse
%<*samplechap2>
%\fi
%    \begin{macrocode}
\section{two}
more text in chapter two
%    \end{macrocode}

%\iffalse
%</samplechap2>
%\fi
%
% %%%%%%%%%%%%%%%%%%%%%%%%%%%%%%%%%%%%%%
% \paragraph{Part Include Files.}
%
% The include files are called |cdocspt3.tex| and |cdocspt4.tex|.
%
%\iffalse
%<*samplepart3|samplepart4>
%\fi

% Optional override for |\version| flag:
%    \begin{macrocode}
%%\providecommand{\version}{final}
%    \end{macrocode}

% Include the main document:
%    \begin{macrocode}
\input{childdoc.def}
\childdocby{cdocsamp}
%    \end{macrocode}

%\iffalse
%</samplepart3|samplepart4>
%\fi
%
%\iffalse
%<*samplepart3>
%\fi
% Some text for part 3:
%    \begin{macrocode}
some text in part three
%    \end{macrocode}

%\iffalse
%</samplepart3>
%\fi
% Some text for part 4:
%\iffalse
%<*samplepart4>
%\fi
%    \begin{macrocode}
more text in part four
%    \end{macrocode}

%\iffalse
%</samplepart4>
%\fi
%
% %%%%%%%%%%%%%%%%%%%%%%%%%%%%%%%%%%%%%%
% \paragraph{Forwarding for a Complete Draft.}
%
% The following forwarding file |cdocsdrf.tex|
% compiles the main document in draft mode:
%\iffalse
%<*sampledraft>
%\fi
%    \begin{macrocode}
\def\version{draft}
\input{childdoc.def}
\childdocforward{cdocsamp}
%    \end{macrocode}

%\iffalse
%</sampledraft>
%\fi
%
% %%%%%%%%%%%%%%%%%%%%%%%%%%%%%%%%%%%%%%
% \paragraph{Forwarding for Final Version of the Chapters.}
%
% The following forwarding files |cdocsfn1.tex| and |cdocsfn2.tex|
% (with identical content)
% compile the final versions of the child documents
% |cdocsch1.tex| and |cdocsch2.tex|, respectively:
%\iffalse
%<*samplefinal>
%\fi
%    \begin{macrocode}
\def\version{final}
\input{childdoc.def}
\childdocforwardprefix[cdocsamp]{cdocsfn}{cdocsch}
%    \end{macrocode}

%\iffalse
%</samplefinal>
%\fi
%
% %%%%%%%%%%%%%%%%%%%%%%%%%%%%%%%%%%%%%%
% \paragraph{Command Line Processing.}
%
% The following three command lines generate the output files
% |cdocscld|, |cdocscl1| and |cdocscl2|
% which should be identical to
% |cdocsdrf|, |cdocsch1| and |cdocsfn2|, respectively:
% \begin{center}
% \begin{tabular}{l}
% |latex -jobname cdocscld \|\\
% |  "\def\version{draft}\input{childdoc.def}\childdocforward{cdocsamp}"|\\
% |latex -jobname cdocscl1 \|\\
% |  "\input{childdoc.def}\childdocforward[cdocsamp]{cdocsch1}"|\\
% |latex -jobname cdocscl2 \|\\
% |  "\def\version{final}\input{childdoc.def}\childdocforward{cdocsch2}"|
% \end{tabular}
% \end{center}
% Note that the trailing backslash on each first line
% merely continues the input to the second line
% (for convenient cut ant paste).
% Furthermore, the command |latex| can be replaced by any
% of its alternative versions such as |pdflatex|.
%
% %%%%%%%%%%%%%%%%%%%%%%%%%%%%%%%%%%%%%%%%%%%%%%%%%%%%%%%%%%%%%%%%%%%%%%%%%%%%%%
% %%%%%%%%%%%%%%%%%%%%%%%%%%%%%%%%%%%%%%%%%%%%%%%%%%%%%%%%%%%%%%%%%%%%%%%%%%%%%%
% \section{Implementation}
%\iffalse
%<*package>
%\fi
%
% This section describes the definitions file |childdoc.def|.

% The definitions cannot be loaded using |\usepackage| or |\RequirePackage|
% which has a mechanism to prevent loading a style file more than once.
% When loading the definitions by means of |\input|
% multiple instances have to be prevented manually:
%\iffalse
%This code needs to be before the `\ProvidesFile' directive
%which is defined at the beginning of this file.
%Therefore it is also placed there and commented out here.
%</package>
%<*discard>
%\fi
%    \begin{macrocode}
\ifdefined\childdocmain\endinput\fi
%    \end{macrocode}
%\iffalse
%</discard>
%<*package>
%\fi
%
% \macro{\ifchilddoc}
% \macro{\ifchilddocmanual}
% The conditional |\ifchilddoc| tells whether a
% child (true) or main (false) document is being compiled.
% The conditional |\ifchilddocmanual| tells whether
% the |\includeonly| mechanism is used (false) or
% the selection of child files must be performed manually (true).
% The definitions initialise to false:
%    \begin{macrocode}
\newif\ifchilddoc
\newif\ifchilddocmanual
%    \end{macrocode}

% \macro{\childdocname}
% \macro{\childdocjob}
% The macro |\childdocname| stores the name of the main document
% to be compiled. The macro |\childdocjob| stores the name of
% the document on which the \LaTeX{} compiler was originally invoked.
% The content of |\jobname| cannot be compared
% to filenames specified in the source due to different catcodes.
% The following code rescans |\jobname|, stores the result
% in |\childdocname| and saves a copy in |\childdocjob|:
%    \begin{macrocode}
\edef\childdocname{\scantokens\expandafter{\jobname\noexpand}}
\let\childdocjob\childdocname
%    \end{macrocode}

% \macro{\childdocdisable}
% The macro |\childdocdisable| prevents the main file
% from being processed more than once.
% At this stage, the main document command |\childdocmain|
% is assumed to be called once again where it should do nothing.
% Any subsequent call to it should prevent
% a secondary processing of the main document
% It overwrites the forwarding commands
% |\childdocof| and |\childdocforward|
% with empty macros to prevent further inclusions of the main document:
%    \begin{macrocode}
\newcommand{\childdocdisable}
{
  \renewcommand{\childdocmain}[1]{\renewcommand{\childdocmain}[1]{\endinput}}
  \renewcommand{\childdocof}[1]{}
  \renewcommand{\childdocby}[2][]{}
  \renewcommand{\childdocforward}[2][]{}
  \renewcommand{\childdocdisable}{}
}
%    \end{macrocode}

% \macro{\childdocmain}
% The macro |\childdocmain| is to be called at the top of the main file
% with nothing or the main filename (without extension) as argument.
% First, it breaks loops.
% If the argument is not empty and does not match |\childdocname|
% (which is set by the first inclusion of |childdoc.def|),
% |\ifchilddoc| is set to true, |\includeonly| is applied to the child file
% and |\jobname| is set to the main file
% (for proper handling of |.aux| files):
%    \begin{macrocode}
\newcommand{\childdocmain}[1]
{
  \childdocdisable\childdocmain{}
  \if?#1?\else
    \begingroup
      \def\childdoctmp{#1}
      \ifx\childdoctmp\childdocname
        \def\childdoctmp{}
      \else
        \def\childdoctmp
        {
          \childdoctrue
          \includeonly{\childdocname}
          \def\childdocjob{#1}
          \def\jobname{#1}
        }
      \fi
      \expandafter
    \endgroup
    \childdoctmp
  \fi
}
%    \end{macrocode}

% \macro{\childdocof}
% The command |\childdocof| redirects
% compilation to the main file |#1|.
%    \begin{macrocode}
\newcommand{\childdocof}[1]
{
  \childdocdisable
  \childdoctrue
  \includeonly{\childdocname}
  \def\jobname{#1}
  \def\childdocjob{#1}
  \input{#1}
}
%    \end{macrocode}

% \macro{\childdocby}
% The command |\childdocby| ....
%    \begin{macrocode}
\newcommand{\childdocby}[2][]
{
  \childdocdisable
  \childdoctrue
  \childdocmanualtrue
  \if?#1?\else
    \def\jobname{#2}
  \fi
  \def\childdocjob{#2}
  \input{#2}
  \endinput
}
%    \end{macrocode}

% \macro{\childdocforward}
% The command |\childdocforward| redirects
% compilation to the main file or
% (if the optional argument is given) a child file.
% Parameters are set as if the main file
% or a child file starting with |\childdocof| was compiled.
% Then compilation is handed over to the main file:
%    \begin{macrocode}
\newcommand{\childdocforward}[2][]
{
  \begingroup
    \if?#1?
      \def\childdoctmp
      {
        \def\childdocname{#2}
        \def\childdocjob{#2}
        \def\jobname{#2}
        \input{#2}
        \endinput
      }
    \else
      \def\childdoctmp
      {
        \childdocdisable
        \def\childdocname{#2}
        \childdoctrue
        \includeonly{#2}
        \def\childdocjob{#1}
        \def\jobname{#1}
        \input{#1}
        \endinput
      }
    \fi
    \expandafter
  \endgroup
  \childdoctmp
}
%    \end{macrocode}

% \macro{\childdocforwardprefix}
% The command |\childdocforwardprefix| redirects
% compilation to the main or a child file by means of a pattern.
% The prefix |#1| in the current filename is replaced by |#2|
% and the suffix of the current filename is kept
% (it is assumed that the filename does not contain the substring `|~~~|'
% which is used as a delimiter).
% Compilation is handed over to the new file by |\childdocforward|:
%    \begin{macrocode}
\newcommand{\childdocforwardprefix}[3][]
{
  \begingroup
    \def\childdocextract #2##1~~~{\def\childdoctmp{\childdocforward[#1]{#3##1}}}
    \expandafter\childdocextract\childdocname~~~
    \expandafter
  \endgroup
  \childdoctmp
}
%    \end{macrocode}

% \macro{\childdoc}
% The deprecated macro |\childdoc| is a legacy version of |\childdocmain|:
%    \begin{macrocode}
\newcommand{\childdoc}{\childdocmain}
%    \end{macrocode}

% \macro{\childdocredirect}
% The deprecated macro |\childdocredirect| is a legacy version
% of |\childdocforward| and |\childdocforwardprefix|:
%    \begin{macrocode}
\newcommand{\childdocredirect}[2][]
{
  \begingroup
    \if?#1?
      \def\childdoctmp{\childdocforward{#2}}
    \else
      \def\childdoctmp{\childdocforwardprefix{#1}{#2}}
    \fi
    \expandafter
  \endgroup
  \childdoctmp
}
%    \end{macrocode}

%\iffalse
%</package>
%\fi
%
\endinput
|\\
|\childdocby{|\textit{main}|}|\\
\end{tabular}
\end{center}
%
The directive |\childdocby| is similar to |\childdocof|
described in \secref{sec:include},
but the subsequent selection of content must be done manually.
To that end, both |\ifchilddoc| and |\ifchilddocmanual|
will be true upon processing of a part,
and the name of the part is stored in |\childdocname|.
Note that |\jobname| will be set to the filename of the current part
so that each part receives an individual |.aux| file
that does not interfere with the |.aux| file(s) of the main document.
This behaviour can be altered by the alternative form
|\childdocby[*]{|\textit{main}|}| (with a non-empty optional argument)
which uses the |.aux| file of the main document
by setting |\jobname| to \textit{main}.

%%%%%%%%%%%%%%%%%%%%%%%%%%%%%%%%%%%%%%%%%%%%%%%%%%%%%%%%%%%%%%%%%%%%%%%%%%%%%%%%
\subsection{Driver Development}
\label{sec:driver}

The \textsf{childdoc} mechanism can also be use for the development
of definition files such as \LaTeX{} styles or classes.
This case differs from the above setup with multiple parts
included by |\include| in that no |\includeonly| should be invoked.
This can be achieved by starting the include file
(before |\ProvidesPackage|) with:
%
\begin{center}
\begin{tabular}{l}
|% \iffalse
%
% childdoc.dtx Copyright (C) 2017-2018 Niklas Beisert
%
% This work may be distributed and/or modified under the
% conditions of the LaTeX Project Public License, either version 1.3
% of this license or (at your option) any later version.
% The latest version of this license is in
%   http://www.latex-project.org/lppl.txt
% and version 1.3 or later is part of all distributions of LaTeX
% version 2005/12/01 or later.
%
% This work has the LPPL maintenance status `maintained'.
%
% The Current Maintainer of this work is Niklas Beisert.
%
% This work consists of the files childdoc.dtx and childdoc.ins
% and the derived files childdoc.def and cdocsamp.tex with
% cdocsch1.tex, cdocsch2.tex, cdocsdrf.tex, cdocsfn1.tex, cdocsfn2.tex.
%
%<package>\ifdefined\childdocmain\endinput\fi
%<package>\ProvidesFile{childdoc.def}[2018/12/30 v2.0 child document driver]
%<samplemain>\ProvidesFile{cdocsamp.tex}[2018/12/30 v2.0 sample for childdoc]
%<*driver>
%\ProvidesFile{childdoc.drv}[2018/12/30 v2.0 childdoc reference manual file]
\PassOptionsToClass{10pt,a4paper}{article}
\documentclass{ltxdoc}

\usepackage[margin=35mm]{geometry}
\usepackage{hyperref}
\usepackage{hyperxmp}
\usepackage[usenames]{color}

\hypersetup{colorlinks=true}
\hypersetup{pdfstartview=FitH}
\hypersetup{pdfpagemode=UseNone}
\hypersetup{pdfsource={}}
\hypersetup{pdflang={en-UK}}
\hypersetup{pdfcopyright={Copyright 2017-2018 Niklas Beisert.
  This work may be distributed and/or modified under the
  conditions of the LaTeX Project Public License, either version 1.3
  of this license or (at your option) any later version.}}
\hypersetup{pdflicenseurl={http://www.latex-project.org/lppl.txt}}
\hypersetup{pdfcontactaddress={ETH Zurich, ITP, HIT K,
  Wolfgang-Pauli-Strasse 27}}
\hypersetup{pdfcontactpostcode={8093}}
\hypersetup{pdfcontactcity={Zurich}}
\hypersetup{pdfcontactcountry={Switzerland}}
\hypersetup{pdfcontactemail={nbeisert@itp.phys.ethz.ch}}
\hypersetup{pdfcontacturl={http://people.phys.ethz.ch/\xmptilde nbeisert/}}

\newcommand{\secref}[1]{\hyperref[#1]{section \ref*{#1}}}

\parskip1ex
\parindent0pt
\let\olditemize\itemize
\def\itemize{\olditemize\parskip0pt}

\begin{document}

\title{The \textsf{childdoc} Package}
\hypersetup{pdftitle={The childdoc Package}}
\author{Niklas Beisert\\[2ex]
  Institut f\"ur Theoretische Physik\\
  Eidgen\"ossische Technische Hochschule Z\"urich\\
  Wolfgang-Pauli-Strasse 27, 8093 Z\"urich, Switzerland\\[1ex]
  \href{mailto:nbeisert@itp.phys.ethz.ch}
  {\texttt{nbeisert@itp.phys.ethz.ch}}}
\hypersetup{pdfauthor={Niklas Beisert}}
\hypersetup{pdfsubject={Manual for the LaTeX2e Package childdoc}}
\date{30 December 2018, \textsf{v2.0}}
\maketitle

\begin{abstract}\noindent
\textsf{childdoc} is a \LaTeXe{} package
that enables the direct compilation
of document sections included by |\include|
to individual files.
\end{abstract}

\begingroup
\parskip0ex
\tableofcontents
\endgroup

%%%%%%%%%%%%%%%%%%%%%%%%%%%%%%%%%%%%%%%%%%%%%%%%%%%%%%%%%%%%%%%%%%%%%%%%%%%%%%%%
%%%%%%%%%%%%%%%%%%%%%%%%%%%%%%%%%%%%%%%%%%%%%%%%%%%%%%%%%%%%%%%%%%%%%%%%%%%%%%%%
\section{Introduction}

\LaTeX{} provides a mechanism to structure a large document (such as a book)
into a main file and several child files (containing the chapters)
using the |\include| command.
This mechanism is beneficial for documents
which span hundreds of pages in order to
make the source file(s) more manageable.
Moreover, compilation can be restricted to
selected child files by means of the |\includeonly| command.
The latter feature can be used to reduce the compilation time while editing
(this was significantly more useful in the earlier days of \LaTeX{})
or to generate a smaller document which is easier to navigate.
Another application of |\includeonly| is to generate
documents consisting of selected parts of the complete document.

However, there are a few drawbacks of the plain |\include| mechanism:
\begin{itemize}
\item
The child files cannot be compiled on their own,
they can only be compiled via the main file.
A naive editing environment
(such as a text editor with an option
to have the current file processed by \LaTeX)
may require one to switch to the main file before compiling;
attempting to compile the child file produces errors.
\item
The main file must be modified (each time)
to adjust the |\includeonly| command
to the present needs. This easily leaves the main file in a messy state.
\item
The generated document will always carry the filename
of the main document. This is inconvenient if
several child files are to be compiled and
to be kept for distribution.
\end{itemize}

The present package provides a simple interface
to make child files individually compilable by \LaTeX{}.
Compiling a child file then has the same effect as compiling
the main file with an |\includeonly| command
to select the appropriate child.
Moreover the generated document will carry the name of the child
rather than the main file.
This resolves all three above issues.

This feature is meant to make the editing of books,
thesis documents and lecture notes somewhat more convenient.
However, the package can also be used efficiently for
composing a series of documents (such as exercise sheets)
which are typically distributed individually.
It then assists the author in generating the individual documents
(potentially in different versions)
as well as a document containing the collected series.
Another application is in developing style files
or other kinds of included material
where compilation of the style file could redirect
to a sample or test file.

%%%%%%%%%%%%%%%%%%%%%%%%%%%%%%%%%%%%%%%%%%%%%%%%%%%%%%%%%%%%%%%%%%%%%%%%%%%%%%%%
%%%%%%%%%%%%%%%%%%%%%%%%%%%%%%%%%%%%%%%%%%%%%%%%%%%%%%%%%%%%%%%%%%%%%%%%%%%%%%%%
\section{Usage}

First of all, the package \textsf{childdoc} is \emph{not} a standard
\LaTeXe{} |.sty| style file! Therefore it needs to be invoked in
a non-standard way.

%%%%%%%%%%%%%%%%%%%%%%%%%%%%%%%%%%%%%%%%%%%%%%%%%%%%%%%%%%%%%%%%%%%%%%%%%%%%%%%%
\subsection{Included Files}
\label{sec:include}

%%%%%%%%%%%%%%%%%%%%%%%%%%%%%%%%%%%%%%%%
\DescribeMacro{\childdocmain}
To use the package, add the commands
\begin{center}
\begin{tabular}{l}
|\input{childdoc.def}|\\
|\childdocmain{}|\\
\end{tabular}
\end{center}
at the very top of the main \LaTeX{} file,
in particular \emph{before} the |\documentclass| statement!
The argument of |\childdocmain| should be left empty
(but it must be present).

%%%%%%%%%%%%%%%%%%%%%%%%%%%%%%%%%%%%%%%%
\DescribeMacro{\childdocof}
Furthermore, add the commands
\begin{center}
\begin{tabular}{l}
|\input{childdoc.def}|\\
|\childdocof{|\textit{main}|}|\\
\end{tabular}
\end{center}
at the top of every child file \textit{child}
which is included by |\include{|\textit{child}|}|
from within the main file
(or at least for those files to be compiled individually).
The argument \textit{main} must be the filename of the main file.

There are a couple of
considerations in setting up the main and child documents:

%%%%%%%%%%%%%%%%%%%%%%%%%%%%%%%%%%%%%%%%
\paragraph{Restrictions.}

Please note the following restrictions:
\begin{itemize}
\item
|\childdocmain| must be called with one argument \textit{main}
to ensure compatibility with earlier version of the package.
It must either be empty (|\childdocmain{}|)
or precisely match the filename of the main file in which it is specified.
See \secref{sec:detection} for further information.
\item
The filename \textit{main} must be specified without the |.tex| extension.
\item
The filename \textit{main} is case sensitive
(even in case-insensitive file systems)
due to internal string comparison.
\item
The argument \textit{main} should be fully expanded, it cannot be a macro.
\item
Subdirectories and special characters should be avoided in filenames.
\item
The command |\childdocmain{|\textit{main}|}| must be followed by a whitespace.
It should not be followed immediately by another command
or by a comment mark `|%|'.
This is because the \TeX{} parser reads the token immediately following
the argument of |\childdocmain| and puts it
at the beginning of every child section;
however, a white\-space is ignored.
\end{itemize}

%%%%%%%%%%%%%%%%%%%%%%%%%%%%%%%%%%%%%%%%
\paragraph{Content of Main File.}

It is advisable to place all content in the child files included by |\include|.
Any output contained in the main file will appear in all child documents
unless suppressed manually;
it cannot be suppressed automatically by the |\includeonly| directive
and thus should normally be avoided.
A method to include some content in the main file
by means of conditional processing is described in \secref{sec:conditional}.

%%%%%%%%%%%%%%%%%%%%%%%%%%%%%%%%%%%%%%%%
\paragraph{Page Numbering.}

When only a part of the document is compiled,
the appropriate numbering of pages
(as well as other status parameters)
is determined from the |.aux| files.
The latter contain information from previous passes.
However this information needs to propagate through
all intermediate child documents.
Therefore the page numbering in child documents may well
be inconsistent until the complete document is compiled at least once.

A useful (if unconventional) way to always ensure a consistent
page numbering is to restart the numbering in each child document
and denote the pages by `\textit{child}|.|\textit{page}'
where \textit{child} represents the chapter/section number of the child file.
This can be achieved by the command
|\numberwithin{page}{|\textit{child}|}|
of the \textsf{amsmath} package
where \textit{child} can be |chapter| or |section|
depending on the chosen structuring.
Alternatively, one can modify the macro |\thepage| appropriately
and reset the counter |page| at the start of each child file.

%%%%%%%%%%%%%%%%%%%%%%%%%%%%%%%%%%%%%%%%%%%%%%%%%%%%%%%%%%%%%%%%%%%%%%%%%%%%%%%%
\subsection{Conditional Processing}
\label{sec:conditional}

The package provides a mechanism to compile different versions
of a document. To customise the versions further some conditional processing
can come in handy to distinguish which version is being compiled.
The package provides two macros to describe the compilation context:

%%%%%%%%%%%%%%%%%%%%%%%%%%%%%%%%%%%%%%%%
\DescribeMacro{\ifchilddoc}
The conditional |\ifchilddoc| distinguishes between the compilation of
child documents and the main document:
%
\begin{center}
|\ifchilddoc |\textit{child-code}| |[|\||else |\textit{main-code}]| \||fi|
\end{center}

%%%%%%%%%%%%%%%%%%%%%%%%%%%%%%%%%%%%%%%%
\DescribeMacro{\childdocname}
\DescribeMacro{\childdocjob}
The macro |\childdocname| contains the filename (without extension)
of the main or child file being processed.
Note that |\childdocjob| will always contain the name of the main file.

%%%%%%%%%%%%%%%%%%%%%%%%%%%%%%%%%%%%%%%%
\paragraph{Title Page.}

Conditional processing can be used to include a title or banner page
in the main document when proper precautions are taken.
Importantly, the code in the main file should ensure that the page counter
(as well as other status parameters which are stored in the |.aux| files)
takes the same value after the conditional processing.
Otherwise the page numbers may take divergent values
depending on which part is compiled.

For example, a title page could be declared by:
%
\begin{center}
\begin{tabular}{l}
|\ifchilddoc\||else|\\
|\addtocounter{page}{-1}|\\
\textit{code for title page}\\
|\newpage|\\
|\||fi|
\end{tabular}
\end{center}
%
A banner page for the child documents can be generated by:
%
\begin{center}
\begin{tabular}{l}
|\ifchilddoc|\\
|\addtocounter{page}{-1}|\\
\textit{code for banner page}\\
|\newpage|\\
|\||fi|
\end{tabular}
\end{center}
%
Here one could write a message such as:
\begin{center}
|This is the part \childdocname{} of \childdocjob{}.|
\end{center}

%%%%%%%%%%%%%%%%%%%%%%%%%%%%%%%%%%%%%%%%%%%%%%%%%%%%%%%%%%%%%%%%%%%%%%%%%%%%%%%%
\subsection{Flags}
\label{sec:flags}

The package makes it easy to generate different versions
of the main or child documents.
To this end compilation flags can be defined
and assigned different default values.
They will be particularly useful in conjunction
with the forwarding mechanism described in \secref{sec:forward}.

For example, it may be useful to have a flag |\version|
which can be set to |draft| or |final|.
The document source will contain some conditional code
depending on the value of |\version|.
Suppose further, the flag should default to |final| for the main file
and to |draft| for child files
which is a natural assignment for editing the document.
This is achieved by placing the following code
in the preamble of the main document
(below the |\childdocmain| directive):
%
\begin{center}
\begin{tabular}{l}
|\ifchilddoc|\\
|\providecommand{\version}{draft}|\\
|\||else|\\
|\providecommand{\version}{final}|\\
|\||fi|
\end{tabular}
\end{center}
%
The definition by |\providecommand| makes sure
that previous definitions are not overwritten.
Further statements |\providecommand{\version}{...}|
can thus be added before the above code to override it.

For the main file, one might add a line
(between |\childdocmain| and the above block)
%
\begin{center}
|%\ifchilddoc\||else\providecommand{\version}{draft}\||fi|
\end{center}
%
which can be uncommented to produce a draft version.
Likewise one can add a line to the very top of a child file
(above the |\childdocof{|\textit{main}|}| directive)
%
\begin{center}
|%\providecommand{\version}{final}|
\end{center}
%
which can be uncommented to produce the final version of this child document.

%%%%%%%%%%%%%%%%%%%%%%%%%%%%%%%%%%%%%%%%%%%%%%%%%%%%%%%%%%%%%%%%%%%%%%%%%%%%%%%%
\subsection{Forwarding}
\label{sec:forward}

Different versions of the main or child documents
using compilation flags as described in \secref{sec:flags}
can be (permanently) stored in different files
for convenient compilation, viewing and distribution.
To this end, the package defines a command
to pass on compilation to a different file:

%%%%%%%%%%%%%%%%%%%%%%%%%%%%%%%%%%%%%%%%
\DescribeMacro{\childdocforward}
The command |\childdocforward| redirects processing to
another source file:
%
\begin{center}
\begin{tabular}{l}
|\input{childdoc.def}|\\
|\childdocforward[|\textit{main}|]{|\textit{dest}|}|\\
\end{tabular}
\end{center}
%
The argument \textit{dest} is the destination file
(without extension).
It should be the main file or one of the child files.
Note that further \textsf{childdoc} directives
such as |\childdocof| and |\childdocforward|
in the indicated file will be processed in this form.
The optional argument \textit{main}
passes on directly to the main file \textit{main}
while pretending to compile the child \textit{dest}.
This form behaves as if \textit{dest}
issues |\childdocof{|\textit{main}|}| right away,
and no further \textsf{childdoc} directives will be processed.

%%%%%%%%%%%%%%%%%%%%%%%%%%%%%%%%%%%%%%%%
\DescribeMacro{\...prefix}
In the alternative form |\childdocforwardprefix|,
%
\begin{center}
\begin{tabular}{l}
|\input{childdoc.def}|\\
|\childdocforwardprefix[|\textit{main}|]{|\textit{prefix}|}{|\textit{dest}|}|
\end{tabular}
\end{center}
%
the destination file is determined by a pattern
depending on the current file:
To make this work, the current file must be called
`{\textit{prefix}\hspace{0.2em}\textit{suffix}}'
with \textit{prefix} matching precisely the argument.
Processing is then passed on to the file
`{\textit{dest}\hspace{0.2em}\textit{suffix}}'.
Surely, the same effect is achieved by
directly specifying the
argument `{\textit{dest}\hspace{0.2em}\textit{suffix}}'
in the first form.
However, that requires to set up a different file
for each child. With the alternative form of the command
all these files can have exactly the same content
which simplifies setting them up and maintaining them.

For example, the following file |draft.tex|
with a compilation flag |\version| as described in \secref{sec:flags}
compiles the main document as a draft:
%
\begin{center}
\begin{tabular}{l}
|\def\version{draft}|\\
|\input{childdoc.def}|\\
|\childdocforward{|\textit{main}|}|
\end{tabular}
\end{center}
%
Likewise, the following files |final|\textit{nn}|.tex|
compile the final version of the child document
|child|\textit{nn}|.tex|:
%
\begin{center}
\begin{tabular}{l}
|\def\version{final}|\\
|\input{childdoc.def}|\\
|\childdocforwardprefix{final}{child}|
\end{tabular}
\end{center}
%

Note that when several versions of a main file and/or of each child file
are to be generated, it may be convenient to set up a |Makefile| or
shell script to automatise the process.

%%%%%%%%%%%%%%%%%%%%%%%%%%%%%%%%%%%%%%%%%%%%%%%%%%%%%%%%%%%%%%%%%%%%%%%%%%%%%%%%
\subsection{Command Line Processing}
\label{sec:commandline}

The effect of redirection files can also be achieved by invoking
the \LaTeX{} compiler with a more elaborate command line.
Most conveniently this should be done as part
of a shell script or a |Makefile|.

When using \textsf{childdoc} in the main file, the following
command lines effectively perform a redirection
(note that depending on the shell being used,
backslashes may have to be doubled: `|\|' $\to$ `|\\|'):
%
\begin{center}
|... -jobname "|\textit{target}|" |\\|"|[\textit{flags}]%
|\input{childdoc.def}\childdocforward[|\textit{main}|]{|\textit{dest}|}"|
\end{center}
%
Here \textit{target} is the name of the output file,
\textit{main} is the name of the main file
and \textit{dest} is the name of the main or child file to be processed
(all filenames without extensions).
The optional argument \textit{main} can be omitted
if \textit{main} matches \textit{dest}.
Optionally, compilation \textit{flags} can be defined via |\def| commands.
This command line makes the \TeX{} engine believe
it is compiling the file \textit{target}
whose content is specified as the latter parameter.
The provided code then forwards the processing to
\textit{main} or \textit{dest} as described in \secref{sec:forward}.

%%%%%%%%%%%%%%%%%%%%%%%%%%%%%%%%%%%%%%%%%%%%%%%%%%%%%%%%%%%%%%%%%%%%%%%%%%%%%%%%
\subsection{Include by Input}
\label{sec:input}

Including child documents by |\include| has some restrictions by design.
Most notably, the content of a child document always occupies
its own set of pages; pages cannot be shared between child documents.
Usually, this behaviour makes perfect sense
because each child document contain an essential part of the document.
However, in some situations it may be desirable to compose
a document from a collection of parts
without having mandatory page breaks between then.
For this case, the package
provides a mechanism to include parts
by |\input| which can also be processed individually.
However, by construction this mechanism
requires manual handling of the content to be output.

%%%%%%%%%%%%%%%%%%%%%%%%%%%%%%%%%%%%%%%%
\DescribeMacro{\ifchilddocmanual}
The main file should be prepared as usual, see \secref{sec:include}.
However, the document body must make a distinction
between processing of an individual part and of the main document, e.g.:
%
\begin{center}
\begin{tabular}{l}
|\ifchilddocmanual|\\
|\input{\childdocname}|\\
|\||else|\\
\textit{document body with }|\input{|\textit{part}|}|\\
|\||fi|
\end{tabular}
\end{center}
%
The conditional |\ifchilddocmanual| is true whenever
a part to be included by |\input| is being compiled,
and the name of the part is stored in |\childdocname|.

%%%%%%%%%%%%%%%%%%%%%%%%%%%%%%%%%%%%%%%%
\DescribeMacro{\childdocby}
Each part to be included by |\input| should start with:
%
\begin{center}
\begin{tabular}{l}
|\input{childdoc.def}|\\
|\childdocby{|\textit{main}|}|\\
\end{tabular}
\end{center}
%
The directive |\childdocby| is similar to |\childdocof|
described in \secref{sec:include},
but the subsequent selection of content must be done manually.
To that end, both |\ifchilddoc| and |\ifchilddocmanual|
will be true upon processing of a part,
and the name of the part is stored in |\childdocname|.
Note that |\jobname| will be set to the filename of the current part
so that each part receives an individual |.aux| file
that does not interfere with the |.aux| file(s) of the main document.
This behaviour can be altered by the alternative form
|\childdocby[*]{|\textit{main}|}| (with a non-empty optional argument)
which uses the |.aux| file of the main document
by setting |\jobname| to \textit{main}.

%%%%%%%%%%%%%%%%%%%%%%%%%%%%%%%%%%%%%%%%%%%%%%%%%%%%%%%%%%%%%%%%%%%%%%%%%%%%%%%%
\subsection{Driver Development}
\label{sec:driver}

The \textsf{childdoc} mechanism can also be use for the development
of definition files such as \LaTeX{} styles or classes.
This case differs from the above setup with multiple parts
included by |\include| in that no |\includeonly| should be invoked.
This can be achieved by starting the include file
(before |\ProvidesPackage|) with:
%
\begin{center}
\begin{tabular}{l}
|\input{childdoc.def}|\\
|\childdocforward{|\textit{main}|}|\\
\end{tabular}
\end{center}
%
or alternatively with:
%
\begin{center}
\begin{tabular}{l}
|\input{childdoc.def}|\\
|\childdocby{|\textit{main}|}|\\
\end{tabular}
\end{center}
%
Both forms have slightly different effects as described above.
The main file is prepared as usual, see \secref{sec:include}.

%%%%%%%%%%%%%%%%%%%%%%%%%%%%%%%%%%%%%%%%%%%%%%%%%%%%%%%%%%%%%%%%%%%%%%%%%%%%%%%%
\subsection{Legacy Detection}
\label{sec:detection}

The directive |\childdocmain| in the main file can detect
whether the complete document or merely a child is to be compiled
even without using the directive |\childdocof|.
This method is deprecated because it is less robust
and there is no compelling reason to use it;
it is merely provided for backward compatibility
and it may be removed in future versions.

If the detection mechanism is to be used,
it is mandatory to correctly specify
the filename of the main file as the argument of |\childdocmain|:
%
\begin{center}
\begin{tabular}{l}
|\input{childdoc.def}|\\
|\childdocmain{|\textit{main}|}|\\
\end{tabular}
\end{center}
%
If |\jobname| does not match the argument \textit{main} of |\childdocmain|,
it is assumed that |\jobname| points to the child file to be compiled.
When using |\childdocmain| with the main file specified as argument,
it suffices to start a child file
with just |\input{|\textit{main}|}|
without loading of the package and using |\childdocof|.
If instead all processing is done
with the appropriate \textsf{childdoc} directives,
the argument of \textit{main} of |\childdocmain| can be empty.

An alternative version of the command line processing described
in \secref{sec:commandline} using the detection mechanism reads:
%
\begin{center}
|... -jobname "|\textit{target}|" "|[\textit{flags}]%
[|\def\jobname{|\textit{dest}|}|]|\input{|\textit{main}|}"|
\end{center}

%%%%%%%%%%%%%%%%%%%%%%%%%%%%%%%%%%%%%%%%%%%%%%%%%%%%%%%%%%%%%%%%%%%%%%%%%%%%%%%%
\subsection{Manual Code}
\label{sec:manual}

In case one cannot be certain whether the definitions file |childdoc.def|
is installed on the target \TeX{} distribution
and one prefers not to ship it,
it is conceivable to paste a few relevant commands into the sources.

To that end, drop all statements |\input{childdoc.def}|
and perform the replacements as outlined below.
Instead of |\childdocmain{|\textit{main}|}| add the following code
to the top of the main file:
%
\begin{center}
\begin{tabular}{l}
|\||ifdefined\childdocname\endinput\||fi\newif\ifchilddoc|\\
|\edef\childdocname{\scantokens\expandafter{\jobname\noexpand}}|\\
|\def\childdocmain{|\textit{main}|}\||ifx\childdocmain\childdocname\||else|\\
|\childdoctrue\includeonly{\childdocname}\let\jobname\childdocmain\||fi|\\
\end{tabular}
\end{center}
%
Instead of |\childdocof{|\textit{main}|}| just include the main file
at the top of each child file:
%
\begin{center}
|\input{|\textit{main}|}|
\end{center}
%
A simple redirection |\childdocforward{|\textit{dest}|}| is achieved by:
%
\begin{center}
|\def\jobname{|\textit{dest}|}\input{\jobname}|
\end{center}
%
The redirection with prefix
|\childdocforwardprefix[|\textit{prefix}|]{|\textit{dest}|}|
is accomplished by:
%
\begin{center}
\begin{tabular}{l}
|{\edef\jobname{\scantokens\expandafter{\jobname\noexpand}}|\\
|\def\redirectjob |\textit{prefix}|#1~~~{\gdef\jobname{|\textit{dest}|#1}}|\\
|\expandafter\redirectjob\jobname~~~}\input{\jobname}|
\end{tabular}
\end{center}

In an alternative approach,
child documents can be compiled by a specific command line
without additional code or specific definitions:
%
\begin{center}
|... -jobname "|\textit{target}|" "|[\textit{flags}]%
|\includeonly{|\textit{dest}|}\input{|\textit{main}|}"|
\end{center}
%

%%%%%%%%%%%%%%%%%%%%%%%%%%%%%%%%%%%%%%%%%%%%%%%%%%%%%%%%%%%%%%%%%%%%%%%%%%%%%%%%
%%%%%%%%%%%%%%%%%%%%%%%%%%%%%%%%%%%%%%%%%%%%%%%%%%%%%%%%%%%%%%%%%%%%%%%%%%%%%%%%
\section{Information}

%%%%%%%%%%%%%%%%%%%%%%%%%%%%%%%%%%%%%%%%%%%%%%%%%%%%%%%%%%%%%%%%%%%%%%%%%%%%%%%%
\subsection{Copyright}

Copyright \copyright{} 2017--2018 Niklas Beisert

This work may be distributed and/or modified under the
conditions of the \LaTeX{} Project Public License, either version 1.3
of this license or (at your option) any later version.
The latest version of this license is in
  \url{http://www.latex-project.org/lppl.txt}
and version 1.3 or later is part of all distributions of \LaTeX{}
version 2005/12/01 or later.

This work has the LPPL maintenance status `maintained'.

The Current Maintainer of this work is Niklas Beisert.

This work consists of the files |README.txt|, |childdoc.ins| and |childdoc.dtx|
as well as the derived files |childdoc.def|, |cdocsamp.tex|
with |cdocsch1.tex|, |cdocsch2.tex|, |cdocspt3.tex|, |cdocspt4.tex|,
|cdocsdrf.tex|, |cdocsfn1.tex|, |cdocsfn2.tex|
as well as |childdoc.pdf|.

%%%%%%%%%%%%%%%%%%%%%%%%%%%%%%%%%%%%%%%%%%%%%%%%%%%%%%%%%%%%%%%%%%%%%%%%%%%%%%%%
\subsection{Files and Installation}

The package consists of the files:
%
\begin{center}
\begin{tabular}{ll}
    |README.txt|   & readme file \\
    |childdoc.ins| & installation file \\
    |childdoc.dtx| & source file \\
    |childdoc.def| & definition file \\
    |cdocsamp.tex| & sample main file \\
    |cdocsch1.tex| & sample include file \\
    |cdocsch2.tex| & sample include file \\
    |cdocspt3.tex| & sample part file \\
    |cdocspt4.tex| & sample part file \\
    |cdocsdrf.tex| & sample redirection file \\
    |cdocsfn1.tex| & sample redirection file \\
    |cdocsfn2.tex| & sample redirection file \\
    |childdoc.pdf| & manual
\end{tabular}
\end{center}
%
The distribution consists of the files
|README.txt|, |childdoc.ins| and |childdoc.dtx|.
%
\begin{itemize}
\item
Run (pdf)\LaTeX{} on |childdoc.dtx|
to compile the manual |childdoc.pdf| (this file).
\item
Run \LaTeX{} on |childdoc.ins| to create the definitions file |childdoc.def|
and the sample |cdocsamp.tex| with include files
|cdocsch1.tex|, |cdocsch2.tex|, |cdocspt3.tex|, |cdocspt4.tex|,
|cdocsdrf.tex|, |cdocsfn1.tex|, |cdocsfn2.tex|.
Then copy the file |childdoc.def| to an appropriate directory of your \LaTeX{}
distribution, e.g.\ \textit{texmf-root}|/tex/latex/childdoc|.
\end{itemize}

%%%%%%%%%%%%%%%%%%%%%%%%%%%%%%%%%%%%%%%%%%%%%%%%%%%%%%%%%%%%%%%%%%%%%%%%%%%%%%%%
\subsection{Related CTAN Packages}

There are several other packages which offer a similar functionality:
%
\begin{itemize}
\item
The packages
\href{http://ctan.org/pkg/docmute}{\textsf{docmute}},
\href{http://ctan.org/pkg/includex}{\textsf{includex}} and
\href{http://ctan.org/pkg/standalone}{\textsf{standalone}}
provide commands to include only the document body of
a child file thus allowing both files to be compiled individually.
\item
The packages \href{http://ctan.org/pkg/subdocs}{\textsf{subdocs}}
and \href{http://ctan.org/pkg/subfiles}{\textsf{subfiles}}
provide structures in which the main and child documents can be
encapsulated and allowing them to be compiled individually.
The inclusion mechanism is different from the conventional |\include|.
\item
The package \href{http://ctan.org/pkg/combine}{\textsf{combine}}
is an elaborate solution to combine several documents into one.
\end{itemize}
%
See also the CTAN topic \href{http://ctan.org/topic/subdocs}{\textsf{subdocs}}
for further related packages.
The present package differs from the above solutions in that
a document structure constructed with the conventional |\include| mechanism
just needs two extra commands at the top of every file
such that all constituent files can be compiled individually.

%%%%%%%%%%%%%%%%%%%%%%%%%%%%%%%%%%%%%%%%%%%%%%%%%%%%%%%%%%%%%%%%%%%%%%%%%%%%%%%%
%\subsection{Feature Suggestions}
%
%The following is a list of features which may be useful for future
%versions of this package:
%%
%\begin{itemize}
%\item
%\ldots
%\end{itemize}

%%%%%%%%%%%%%%%%%%%%%%%%%%%%%%%%%%%%%%%%%%%%%%%%%%%%%%%%%%%%%%%%%%%%%%%%%%%%%%%%
\subsection{Revision History}

%%%%%%%%%%%%%%%%%%%%%%%%%%%%%%%%%%%%%%%%
\paragraph{v2.0:} 2018/12/30

\begin{itemize}
\item
immediate forward processing
\item
added |\childdocby| mechanism
\item
manual restructured
\end{itemize}

%%%%%%%%%%%%%%%%%%%%%%%%%%%%%%%%%%%%%%%%
\paragraph{v1.6:} 2018/01/17

\begin{itemize}
\item
application for development of include files
\item
corrections to manual
\end{itemize}

%%%%%%%%%%%%%%%%%%%%%%%%%%%%%%%%%%%%%%%%
\paragraph{v1.5:} 2017/05/21

\begin{itemize}
\item
more complete structuring introduced
\item
|\childdocof| introduced
\item
|\childdoc| renamed to |\childdocmain|
\item
|\childredirect| renamed to |\childdocforward| and |\childdocforwardprefix|
and functionality expanded
\end{itemize}

%%%%%%%%%%%%%%%%%%%%%%%%%%%%%%%%%%%%%%%%
\paragraph{v1.0:} 2017/04/27

\begin{itemize}
\item
manual and install package
\item
first version published on CTAN
\end{itemize}

%%%%%%%%%%%%%%%%%%%%%%%%%%%%%%%%%%%%%%%%
\paragraph{v0.6:} 2017/04/26

\begin{itemize}
\item
redirection mechanism added
\end{itemize}

%%%%%%%%%%%%%%%%%%%%%%%%%%%%%%%%%%%%%%%%
\paragraph{v0.5:} 2017/04/26

\begin{itemize}
\item
functionality in definition file
\end{itemize}


%%%%%%%%%%%%%%%%%%%%%%%%%%%%%%%%%%%%%%%%%%%%%%%%%%%%%%%%%%%%%%%%%%%%%%%%%%%%%%%%
%%%%%%%%%%%%%%%%%%%%%%%%%%%%%%%%%%%%%%%%%%%%%%%%%%%%%%%%%%%%%%%%%%%%%%%%%%%%%%%%
%%%%%%%%%%%%%%%%%%%%%%%%%%%%%%%%%%%%%%%%%%%%%%%%%%%%%%%%%%%%%%%%%%%%%%%%%%%%%%%%
\appendix

\settowidth\MacroIndent{\rmfamily\scriptsize 000\ }

 \DocInput{childdoc.dtx}

\end{document}
%</driver>
% \fi
%
% %%%%%%%%%%%%%%%%%%%%%%%%%%%%%%%%%%%%%%%%%%%%%%%%%%%%%%%%%%%%%%%%%%%%%%%%%%%%%%
% %%%%%%%%%%%%%%%%%%%%%%%%%%%%%%%%%%%%%%%%%%%%%%%%%%%%%%%%%%%%%%%%%%%%%%%%%%%%%%
% \section{Sample}
%\iffalse
%<*samplemain>
%\fi
%
% The following presents a sample document
% with two chapters, two parts, a title page,
% a compile flag as well as three forwarding files to set the flag.
% It consists of eight |.tex| files:
% \begin{center}
% \begin{tabular}{ll}
% |cdocsamp.tex|&main file\\
% |cdocsch1.tex|&include file for chapter 1\\
% |cdocsch2.tex|&include file for chapter 2\\
% |cdocspt3.tex|&include file for part 3\\
% |cdocspt4.tex|&include file for part 4\\
% |cdocsdrf.tex|&forwarding file for main file in draft mode\\
% |cdocsfi1.tex|&forwarding file for final version of chapter 1\\
% |cdocsfi2.tex|&forwarding file for final version of chapter 2\\
% \end{tabular}
% \end{center}
% Each of the eight files can be compiled directly by the \LaTeX{} compiler.
%
% %%%%%%%%%%%%%%%%%%%%%%%%%%%%%%%%%%%%%%
% \paragraph{Main File.}
%
% The main file is called |cdocsamp.tex|.
%
% Load the \textsf{childdoc} definitions and
% declare the filename for the main document:
%    \begin{macrocode}
\input{childdoc.def}
\childdocmain{}
%    \end{macrocode}

% Optional override for |\version| flag:
%    \begin{macrocode}
%%\ifchilddoc\else\providecommand{\version}{draft}\fi
%    \end{macrocode}

% Define the default values for the |\version| flag
% (|final| for the main file and |draft| for childs):
%    \begin{macrocode}
\ifchilddoc
\providecommand{\version}{draft}
\else
\providecommand{\version}{final}
\fi
%    \end{macrocode}

% Load the standard document class:
%    \begin{macrocode}
\documentclass[12pt]{article}
%    \end{macrocode}

% Start the document body:
%    \begin{macrocode}
\begin{document}
%    \end{macrocode}

% Declare a title page.
% Print title, part of document being processed and version flag:
%    \begin{macrocode}
\addtocounter{page}{-1}
\begin{center}
{\LARGE\bfseries{}childdoc example\par}
\vspace{1cm}
\ifchilddoc
\ifchilddocmanual part\else chapter\fi:
`\childdocname' of `\childdocjob'\par
\else
main document: `\childdocjob'\par
\fi
version: \version\par
\end{center}
\newpage
%    \end{macrocode}

% Manually include selected file,
% otherwise process as usual:
%    \begin{macrocode}
\ifchilddocmanual
\section*{part `\childdocname'}
\input{\childdocname}
\else
%    \end{macrocode}

% Include the two chapters:
%    \begin{macrocode}
\include{cdocsch1}
\include{cdocsch2}
%    \end{macrocode}

% Include the two parts unless only chapters should be displayed:
%    \begin{macrocode}
\ifchilddoc\else
\section{part three}
\input{cdocspt3}
\section{part four}
\input{cdocspt4}
\fi
%    \end{macrocode}

% Process as usual until here:
%    \begin{macrocode}
\fi
%    \end{macrocode}

% End of document body:
%    \begin{macrocode}
\end{document}
%    \end{macrocode}
%\iffalse
%</samplemain>
%\fi
%
% %%%%%%%%%%%%%%%%%%%%%%%%%%%%%%%%%%%%%%
% \paragraph{Chapter Include Files.}
%
% The include files are called |cdocsch1.tex| and |cdocsch2.tex|.
%
%\iffalse
%<*samplechap1|samplechap2>
%\fi

% Optional override for |\version| flag:
%    \begin{macrocode}
%%\providecommand{\version}{final}
%    \end{macrocode}

% Include the main document:
%    \begin{macrocode}
\input{childdoc.def}
\childdocof{cdocsamp}
%    \end{macrocode}

%\iffalse
%</samplechap1|samplechap2>
%\fi
%
%\iffalse
%<*samplechap1>
%\fi
% Some text for chapter 1:
%    \begin{macrocode}
\section{one}
some text in chapter one
%    \end{macrocode}

%\iffalse
%</samplechap1>
%\fi
% Some text for chapter 2:
%\iffalse
%<*samplechap2>
%\fi
%    \begin{macrocode}
\section{two}
more text in chapter two
%    \end{macrocode}

%\iffalse
%</samplechap2>
%\fi
%
% %%%%%%%%%%%%%%%%%%%%%%%%%%%%%%%%%%%%%%
% \paragraph{Part Include Files.}
%
% The include files are called |cdocspt3.tex| and |cdocspt4.tex|.
%
%\iffalse
%<*samplepart3|samplepart4>
%\fi

% Optional override for |\version| flag:
%    \begin{macrocode}
%%\providecommand{\version}{final}
%    \end{macrocode}

% Include the main document:
%    \begin{macrocode}
\input{childdoc.def}
\childdocby{cdocsamp}
%    \end{macrocode}

%\iffalse
%</samplepart3|samplepart4>
%\fi
%
%\iffalse
%<*samplepart3>
%\fi
% Some text for part 3:
%    \begin{macrocode}
some text in part three
%    \end{macrocode}

%\iffalse
%</samplepart3>
%\fi
% Some text for part 4:
%\iffalse
%<*samplepart4>
%\fi
%    \begin{macrocode}
more text in part four
%    \end{macrocode}

%\iffalse
%</samplepart4>
%\fi
%
% %%%%%%%%%%%%%%%%%%%%%%%%%%%%%%%%%%%%%%
% \paragraph{Forwarding for a Complete Draft.}
%
% The following forwarding file |cdocsdrf.tex|
% compiles the main document in draft mode:
%\iffalse
%<*sampledraft>
%\fi
%    \begin{macrocode}
\def\version{draft}
\input{childdoc.def}
\childdocforward{cdocsamp}
%    \end{macrocode}

%\iffalse
%</sampledraft>
%\fi
%
% %%%%%%%%%%%%%%%%%%%%%%%%%%%%%%%%%%%%%%
% \paragraph{Forwarding for Final Version of the Chapters.}
%
% The following forwarding files |cdocsfn1.tex| and |cdocsfn2.tex|
% (with identical content)
% compile the final versions of the child documents
% |cdocsch1.tex| and |cdocsch2.tex|, respectively:
%\iffalse
%<*samplefinal>
%\fi
%    \begin{macrocode}
\def\version{final}
\input{childdoc.def}
\childdocforwardprefix[cdocsamp]{cdocsfn}{cdocsch}
%    \end{macrocode}

%\iffalse
%</samplefinal>
%\fi
%
% %%%%%%%%%%%%%%%%%%%%%%%%%%%%%%%%%%%%%%
% \paragraph{Command Line Processing.}
%
% The following three command lines generate the output files
% |cdocscld|, |cdocscl1| and |cdocscl2|
% which should be identical to
% |cdocsdrf|, |cdocsch1| and |cdocsfn2|, respectively:
% \begin{center}
% \begin{tabular}{l}
% |latex -jobname cdocscld \|\\
% |  "\def\version{draft}\input{childdoc.def}\childdocforward{cdocsamp}"|\\
% |latex -jobname cdocscl1 \|\\
% |  "\input{childdoc.def}\childdocforward[cdocsamp]{cdocsch1}"|\\
% |latex -jobname cdocscl2 \|\\
% |  "\def\version{final}\input{childdoc.def}\childdocforward{cdocsch2}"|
% \end{tabular}
% \end{center}
% Note that the trailing backslash on each first line
% merely continues the input to the second line
% (for convenient cut ant paste).
% Furthermore, the command |latex| can be replaced by any
% of its alternative versions such as |pdflatex|.
%
% %%%%%%%%%%%%%%%%%%%%%%%%%%%%%%%%%%%%%%%%%%%%%%%%%%%%%%%%%%%%%%%%%%%%%%%%%%%%%%
% %%%%%%%%%%%%%%%%%%%%%%%%%%%%%%%%%%%%%%%%%%%%%%%%%%%%%%%%%%%%%%%%%%%%%%%%%%%%%%
% \section{Implementation}
%\iffalse
%<*package>
%\fi
%
% This section describes the definitions file |childdoc.def|.

% The definitions cannot be loaded using |\usepackage| or |\RequirePackage|
% which has a mechanism to prevent loading a style file more than once.
% When loading the definitions by means of |\input|
% multiple instances have to be prevented manually:
%\iffalse
%This code needs to be before the `\ProvidesFile' directive
%which is defined at the beginning of this file.
%Therefore it is also placed there and commented out here.
%</package>
%<*discard>
%\fi
%    \begin{macrocode}
\ifdefined\childdocmain\endinput\fi
%    \end{macrocode}
%\iffalse
%</discard>
%<*package>
%\fi
%
% \macro{\ifchilddoc}
% \macro{\ifchilddocmanual}
% The conditional |\ifchilddoc| tells whether a
% child (true) or main (false) document is being compiled.
% The conditional |\ifchilddocmanual| tells whether
% the |\includeonly| mechanism is used (false) or
% the selection of child files must be performed manually (true).
% The definitions initialise to false:
%    \begin{macrocode}
\newif\ifchilddoc
\newif\ifchilddocmanual
%    \end{macrocode}

% \macro{\childdocname}
% \macro{\childdocjob}
% The macro |\childdocname| stores the name of the main document
% to be compiled. The macro |\childdocjob| stores the name of
% the document on which the \LaTeX{} compiler was originally invoked.
% The content of |\jobname| cannot be compared
% to filenames specified in the source due to different catcodes.
% The following code rescans |\jobname|, stores the result
% in |\childdocname| and saves a copy in |\childdocjob|:
%    \begin{macrocode}
\edef\childdocname{\scantokens\expandafter{\jobname\noexpand}}
\let\childdocjob\childdocname
%    \end{macrocode}

% \macro{\childdocdisable}
% The macro |\childdocdisable| prevents the main file
% from being processed more than once.
% At this stage, the main document command |\childdocmain|
% is assumed to be called once again where it should do nothing.
% Any subsequent call to it should prevent
% a secondary processing of the main document
% It overwrites the forwarding commands
% |\childdocof| and |\childdocforward|
% with empty macros to prevent further inclusions of the main document:
%    \begin{macrocode}
\newcommand{\childdocdisable}
{
  \renewcommand{\childdocmain}[1]{\renewcommand{\childdocmain}[1]{\endinput}}
  \renewcommand{\childdocof}[1]{}
  \renewcommand{\childdocby}[2][]{}
  \renewcommand{\childdocforward}[2][]{}
  \renewcommand{\childdocdisable}{}
}
%    \end{macrocode}

% \macro{\childdocmain}
% The macro |\childdocmain| is to be called at the top of the main file
% with nothing or the main filename (without extension) as argument.
% First, it breaks loops.
% If the argument is not empty and does not match |\childdocname|
% (which is set by the first inclusion of |childdoc.def|),
% |\ifchilddoc| is set to true, |\includeonly| is applied to the child file
% and |\jobname| is set to the main file
% (for proper handling of |.aux| files):
%    \begin{macrocode}
\newcommand{\childdocmain}[1]
{
  \childdocdisable\childdocmain{}
  \if?#1?\else
    \begingroup
      \def\childdoctmp{#1}
      \ifx\childdoctmp\childdocname
        \def\childdoctmp{}
      \else
        \def\childdoctmp
        {
          \childdoctrue
          \includeonly{\childdocname}
          \def\childdocjob{#1}
          \def\jobname{#1}
        }
      \fi
      \expandafter
    \endgroup
    \childdoctmp
  \fi
}
%    \end{macrocode}

% \macro{\childdocof}
% The command |\childdocof| redirects
% compilation to the main file |#1|.
%    \begin{macrocode}
\newcommand{\childdocof}[1]
{
  \childdocdisable
  \childdoctrue
  \includeonly{\childdocname}
  \def\jobname{#1}
  \def\childdocjob{#1}
  \input{#1}
}
%    \end{macrocode}

% \macro{\childdocby}
% The command |\childdocby| ....
%    \begin{macrocode}
\newcommand{\childdocby}[2][]
{
  \childdocdisable
  \childdoctrue
  \childdocmanualtrue
  \if?#1?\else
    \def\jobname{#2}
  \fi
  \def\childdocjob{#2}
  \input{#2}
  \endinput
}
%    \end{macrocode}

% \macro{\childdocforward}
% The command |\childdocforward| redirects
% compilation to the main file or
% (if the optional argument is given) a child file.
% Parameters are set as if the main file
% or a child file starting with |\childdocof| was compiled.
% Then compilation is handed over to the main file:
%    \begin{macrocode}
\newcommand{\childdocforward}[2][]
{
  \begingroup
    \if?#1?
      \def\childdoctmp
      {
        \def\childdocname{#2}
        \def\childdocjob{#2}
        \def\jobname{#2}
        \input{#2}
        \endinput
      }
    \else
      \def\childdoctmp
      {
        \childdocdisable
        \def\childdocname{#2}
        \childdoctrue
        \includeonly{#2}
        \def\childdocjob{#1}
        \def\jobname{#1}
        \input{#1}
        \endinput
      }
    \fi
    \expandafter
  \endgroup
  \childdoctmp
}
%    \end{macrocode}

% \macro{\childdocforwardprefix}
% The command |\childdocforwardprefix| redirects
% compilation to the main or a child file by means of a pattern.
% The prefix |#1| in the current filename is replaced by |#2|
% and the suffix of the current filename is kept
% (it is assumed that the filename does not contain the substring `|~~~|'
% which is used as a delimiter).
% Compilation is handed over to the new file by |\childdocforward|:
%    \begin{macrocode}
\newcommand{\childdocforwardprefix}[3][]
{
  \begingroup
    \def\childdocextract #2##1~~~{\def\childdoctmp{\childdocforward[#1]{#3##1}}}
    \expandafter\childdocextract\childdocname~~~
    \expandafter
  \endgroup
  \childdoctmp
}
%    \end{macrocode}

% \macro{\childdoc}
% The deprecated macro |\childdoc| is a legacy version of |\childdocmain|:
%    \begin{macrocode}
\newcommand{\childdoc}{\childdocmain}
%    \end{macrocode}

% \macro{\childdocredirect}
% The deprecated macro |\childdocredirect| is a legacy version
% of |\childdocforward| and |\childdocforwardprefix|:
%    \begin{macrocode}
\newcommand{\childdocredirect}[2][]
{
  \begingroup
    \if?#1?
      \def\childdoctmp{\childdocforward{#2}}
    \else
      \def\childdoctmp{\childdocforwardprefix{#1}{#2}}
    \fi
    \expandafter
  \endgroup
  \childdoctmp
}
%    \end{macrocode}

%\iffalse
%</package>
%\fi
%
\endinput
|\\
|\childdocforward{|\textit{main}|}|\\
\end{tabular}
\end{center}
%
or alternatively with:
%
\begin{center}
\begin{tabular}{l}
|% \iffalse
%
% childdoc.dtx Copyright (C) 2017-2018 Niklas Beisert
%
% This work may be distributed and/or modified under the
% conditions of the LaTeX Project Public License, either version 1.3
% of this license or (at your option) any later version.
% The latest version of this license is in
%   http://www.latex-project.org/lppl.txt
% and version 1.3 or later is part of all distributions of LaTeX
% version 2005/12/01 or later.
%
% This work has the LPPL maintenance status `maintained'.
%
% The Current Maintainer of this work is Niklas Beisert.
%
% This work consists of the files childdoc.dtx and childdoc.ins
% and the derived files childdoc.def and cdocsamp.tex with
% cdocsch1.tex, cdocsch2.tex, cdocsdrf.tex, cdocsfn1.tex, cdocsfn2.tex.
%
%<package>\ifdefined\childdocmain\endinput\fi
%<package>\ProvidesFile{childdoc.def}[2018/12/30 v2.0 child document driver]
%<samplemain>\ProvidesFile{cdocsamp.tex}[2018/12/30 v2.0 sample for childdoc]
%<*driver>
%\ProvidesFile{childdoc.drv}[2018/12/30 v2.0 childdoc reference manual file]
\PassOptionsToClass{10pt,a4paper}{article}
\documentclass{ltxdoc}

\usepackage[margin=35mm]{geometry}
\usepackage{hyperref}
\usepackage{hyperxmp}
\usepackage[usenames]{color}

\hypersetup{colorlinks=true}
\hypersetup{pdfstartview=FitH}
\hypersetup{pdfpagemode=UseNone}
\hypersetup{pdfsource={}}
\hypersetup{pdflang={en-UK}}
\hypersetup{pdfcopyright={Copyright 2017-2018 Niklas Beisert.
  This work may be distributed and/or modified under the
  conditions of the LaTeX Project Public License, either version 1.3
  of this license or (at your option) any later version.}}
\hypersetup{pdflicenseurl={http://www.latex-project.org/lppl.txt}}
\hypersetup{pdfcontactaddress={ETH Zurich, ITP, HIT K,
  Wolfgang-Pauli-Strasse 27}}
\hypersetup{pdfcontactpostcode={8093}}
\hypersetup{pdfcontactcity={Zurich}}
\hypersetup{pdfcontactcountry={Switzerland}}
\hypersetup{pdfcontactemail={nbeisert@itp.phys.ethz.ch}}
\hypersetup{pdfcontacturl={http://people.phys.ethz.ch/\xmptilde nbeisert/}}

\newcommand{\secref}[1]{\hyperref[#1]{section \ref*{#1}}}

\parskip1ex
\parindent0pt
\let\olditemize\itemize
\def\itemize{\olditemize\parskip0pt}

\begin{document}

\title{The \textsf{childdoc} Package}
\hypersetup{pdftitle={The childdoc Package}}
\author{Niklas Beisert\\[2ex]
  Institut f\"ur Theoretische Physik\\
  Eidgen\"ossische Technische Hochschule Z\"urich\\
  Wolfgang-Pauli-Strasse 27, 8093 Z\"urich, Switzerland\\[1ex]
  \href{mailto:nbeisert@itp.phys.ethz.ch}
  {\texttt{nbeisert@itp.phys.ethz.ch}}}
\hypersetup{pdfauthor={Niklas Beisert}}
\hypersetup{pdfsubject={Manual for the LaTeX2e Package childdoc}}
\date{30 December 2018, \textsf{v2.0}}
\maketitle

\begin{abstract}\noindent
\textsf{childdoc} is a \LaTeXe{} package
that enables the direct compilation
of document sections included by |\include|
to individual files.
\end{abstract}

\begingroup
\parskip0ex
\tableofcontents
\endgroup

%%%%%%%%%%%%%%%%%%%%%%%%%%%%%%%%%%%%%%%%%%%%%%%%%%%%%%%%%%%%%%%%%%%%%%%%%%%%%%%%
%%%%%%%%%%%%%%%%%%%%%%%%%%%%%%%%%%%%%%%%%%%%%%%%%%%%%%%%%%%%%%%%%%%%%%%%%%%%%%%%
\section{Introduction}

\LaTeX{} provides a mechanism to structure a large document (such as a book)
into a main file and several child files (containing the chapters)
using the |\include| command.
This mechanism is beneficial for documents
which span hundreds of pages in order to
make the source file(s) more manageable.
Moreover, compilation can be restricted to
selected child files by means of the |\includeonly| command.
The latter feature can be used to reduce the compilation time while editing
(this was significantly more useful in the earlier days of \LaTeX{})
or to generate a smaller document which is easier to navigate.
Another application of |\includeonly| is to generate
documents consisting of selected parts of the complete document.

However, there are a few drawbacks of the plain |\include| mechanism:
\begin{itemize}
\item
The child files cannot be compiled on their own,
they can only be compiled via the main file.
A naive editing environment
(such as a text editor with an option
to have the current file processed by \LaTeX)
may require one to switch to the main file before compiling;
attempting to compile the child file produces errors.
\item
The main file must be modified (each time)
to adjust the |\includeonly| command
to the present needs. This easily leaves the main file in a messy state.
\item
The generated document will always carry the filename
of the main document. This is inconvenient if
several child files are to be compiled and
to be kept for distribution.
\end{itemize}

The present package provides a simple interface
to make child files individually compilable by \LaTeX{}.
Compiling a child file then has the same effect as compiling
the main file with an |\includeonly| command
to select the appropriate child.
Moreover the generated document will carry the name of the child
rather than the main file.
This resolves all three above issues.

This feature is meant to make the editing of books,
thesis documents and lecture notes somewhat more convenient.
However, the package can also be used efficiently for
composing a series of documents (such as exercise sheets)
which are typically distributed individually.
It then assists the author in generating the individual documents
(potentially in different versions)
as well as a document containing the collected series.
Another application is in developing style files
or other kinds of included material
where compilation of the style file could redirect
to a sample or test file.

%%%%%%%%%%%%%%%%%%%%%%%%%%%%%%%%%%%%%%%%%%%%%%%%%%%%%%%%%%%%%%%%%%%%%%%%%%%%%%%%
%%%%%%%%%%%%%%%%%%%%%%%%%%%%%%%%%%%%%%%%%%%%%%%%%%%%%%%%%%%%%%%%%%%%%%%%%%%%%%%%
\section{Usage}

First of all, the package \textsf{childdoc} is \emph{not} a standard
\LaTeXe{} |.sty| style file! Therefore it needs to be invoked in
a non-standard way.

%%%%%%%%%%%%%%%%%%%%%%%%%%%%%%%%%%%%%%%%%%%%%%%%%%%%%%%%%%%%%%%%%%%%%%%%%%%%%%%%
\subsection{Included Files}
\label{sec:include}

%%%%%%%%%%%%%%%%%%%%%%%%%%%%%%%%%%%%%%%%
\DescribeMacro{\childdocmain}
To use the package, add the commands
\begin{center}
\begin{tabular}{l}
|\input{childdoc.def}|\\
|\childdocmain{}|\\
\end{tabular}
\end{center}
at the very top of the main \LaTeX{} file,
in particular \emph{before} the |\documentclass| statement!
The argument of |\childdocmain| should be left empty
(but it must be present).

%%%%%%%%%%%%%%%%%%%%%%%%%%%%%%%%%%%%%%%%
\DescribeMacro{\childdocof}
Furthermore, add the commands
\begin{center}
\begin{tabular}{l}
|\input{childdoc.def}|\\
|\childdocof{|\textit{main}|}|\\
\end{tabular}
\end{center}
at the top of every child file \textit{child}
which is included by |\include{|\textit{child}|}|
from within the main file
(or at least for those files to be compiled individually).
The argument \textit{main} must be the filename of the main file.

There are a couple of
considerations in setting up the main and child documents:

%%%%%%%%%%%%%%%%%%%%%%%%%%%%%%%%%%%%%%%%
\paragraph{Restrictions.}

Please note the following restrictions:
\begin{itemize}
\item
|\childdocmain| must be called with one argument \textit{main}
to ensure compatibility with earlier version of the package.
It must either be empty (|\childdocmain{}|)
or precisely match the filename of the main file in which it is specified.
See \secref{sec:detection} for further information.
\item
The filename \textit{main} must be specified without the |.tex| extension.
\item
The filename \textit{main} is case sensitive
(even in case-insensitive file systems)
due to internal string comparison.
\item
The argument \textit{main} should be fully expanded, it cannot be a macro.
\item
Subdirectories and special characters should be avoided in filenames.
\item
The command |\childdocmain{|\textit{main}|}| must be followed by a whitespace.
It should not be followed immediately by another command
or by a comment mark `|%|'.
This is because the \TeX{} parser reads the token immediately following
the argument of |\childdocmain| and puts it
at the beginning of every child section;
however, a white\-space is ignored.
\end{itemize}

%%%%%%%%%%%%%%%%%%%%%%%%%%%%%%%%%%%%%%%%
\paragraph{Content of Main File.}

It is advisable to place all content in the child files included by |\include|.
Any output contained in the main file will appear in all child documents
unless suppressed manually;
it cannot be suppressed automatically by the |\includeonly| directive
and thus should normally be avoided.
A method to include some content in the main file
by means of conditional processing is described in \secref{sec:conditional}.

%%%%%%%%%%%%%%%%%%%%%%%%%%%%%%%%%%%%%%%%
\paragraph{Page Numbering.}

When only a part of the document is compiled,
the appropriate numbering of pages
(as well as other status parameters)
is determined from the |.aux| files.
The latter contain information from previous passes.
However this information needs to propagate through
all intermediate child documents.
Therefore the page numbering in child documents may well
be inconsistent until the complete document is compiled at least once.

A useful (if unconventional) way to always ensure a consistent
page numbering is to restart the numbering in each child document
and denote the pages by `\textit{child}|.|\textit{page}'
where \textit{child} represents the chapter/section number of the child file.
This can be achieved by the command
|\numberwithin{page}{|\textit{child}|}|
of the \textsf{amsmath} package
where \textit{child} can be |chapter| or |section|
depending on the chosen structuring.
Alternatively, one can modify the macro |\thepage| appropriately
and reset the counter |page| at the start of each child file.

%%%%%%%%%%%%%%%%%%%%%%%%%%%%%%%%%%%%%%%%%%%%%%%%%%%%%%%%%%%%%%%%%%%%%%%%%%%%%%%%
\subsection{Conditional Processing}
\label{sec:conditional}

The package provides a mechanism to compile different versions
of a document. To customise the versions further some conditional processing
can come in handy to distinguish which version is being compiled.
The package provides two macros to describe the compilation context:

%%%%%%%%%%%%%%%%%%%%%%%%%%%%%%%%%%%%%%%%
\DescribeMacro{\ifchilddoc}
The conditional |\ifchilddoc| distinguishes between the compilation of
child documents and the main document:
%
\begin{center}
|\ifchilddoc |\textit{child-code}| |[|\||else |\textit{main-code}]| \||fi|
\end{center}

%%%%%%%%%%%%%%%%%%%%%%%%%%%%%%%%%%%%%%%%
\DescribeMacro{\childdocname}
\DescribeMacro{\childdocjob}
The macro |\childdocname| contains the filename (without extension)
of the main or child file being processed.
Note that |\childdocjob| will always contain the name of the main file.

%%%%%%%%%%%%%%%%%%%%%%%%%%%%%%%%%%%%%%%%
\paragraph{Title Page.}

Conditional processing can be used to include a title or banner page
in the main document when proper precautions are taken.
Importantly, the code in the main file should ensure that the page counter
(as well as other status parameters which are stored in the |.aux| files)
takes the same value after the conditional processing.
Otherwise the page numbers may take divergent values
depending on which part is compiled.

For example, a title page could be declared by:
%
\begin{center}
\begin{tabular}{l}
|\ifchilddoc\||else|\\
|\addtocounter{page}{-1}|\\
\textit{code for title page}\\
|\newpage|\\
|\||fi|
\end{tabular}
\end{center}
%
A banner page for the child documents can be generated by:
%
\begin{center}
\begin{tabular}{l}
|\ifchilddoc|\\
|\addtocounter{page}{-1}|\\
\textit{code for banner page}\\
|\newpage|\\
|\||fi|
\end{tabular}
\end{center}
%
Here one could write a message such as:
\begin{center}
|This is the part \childdocname{} of \childdocjob{}.|
\end{center}

%%%%%%%%%%%%%%%%%%%%%%%%%%%%%%%%%%%%%%%%%%%%%%%%%%%%%%%%%%%%%%%%%%%%%%%%%%%%%%%%
\subsection{Flags}
\label{sec:flags}

The package makes it easy to generate different versions
of the main or child documents.
To this end compilation flags can be defined
and assigned different default values.
They will be particularly useful in conjunction
with the forwarding mechanism described in \secref{sec:forward}.

For example, it may be useful to have a flag |\version|
which can be set to |draft| or |final|.
The document source will contain some conditional code
depending on the value of |\version|.
Suppose further, the flag should default to |final| for the main file
and to |draft| for child files
which is a natural assignment for editing the document.
This is achieved by placing the following code
in the preamble of the main document
(below the |\childdocmain| directive):
%
\begin{center}
\begin{tabular}{l}
|\ifchilddoc|\\
|\providecommand{\version}{draft}|\\
|\||else|\\
|\providecommand{\version}{final}|\\
|\||fi|
\end{tabular}
\end{center}
%
The definition by |\providecommand| makes sure
that previous definitions are not overwritten.
Further statements |\providecommand{\version}{...}|
can thus be added before the above code to override it.

For the main file, one might add a line
(between |\childdocmain| and the above block)
%
\begin{center}
|%\ifchilddoc\||else\providecommand{\version}{draft}\||fi|
\end{center}
%
which can be uncommented to produce a draft version.
Likewise one can add a line to the very top of a child file
(above the |\childdocof{|\textit{main}|}| directive)
%
\begin{center}
|%\providecommand{\version}{final}|
\end{center}
%
which can be uncommented to produce the final version of this child document.

%%%%%%%%%%%%%%%%%%%%%%%%%%%%%%%%%%%%%%%%%%%%%%%%%%%%%%%%%%%%%%%%%%%%%%%%%%%%%%%%
\subsection{Forwarding}
\label{sec:forward}

Different versions of the main or child documents
using compilation flags as described in \secref{sec:flags}
can be (permanently) stored in different files
for convenient compilation, viewing and distribution.
To this end, the package defines a command
to pass on compilation to a different file:

%%%%%%%%%%%%%%%%%%%%%%%%%%%%%%%%%%%%%%%%
\DescribeMacro{\childdocforward}
The command |\childdocforward| redirects processing to
another source file:
%
\begin{center}
\begin{tabular}{l}
|\input{childdoc.def}|\\
|\childdocforward[|\textit{main}|]{|\textit{dest}|}|\\
\end{tabular}
\end{center}
%
The argument \textit{dest} is the destination file
(without extension).
It should be the main file or one of the child files.
Note that further \textsf{childdoc} directives
such as |\childdocof| and |\childdocforward|
in the indicated file will be processed in this form.
The optional argument \textit{main}
passes on directly to the main file \textit{main}
while pretending to compile the child \textit{dest}.
This form behaves as if \textit{dest}
issues |\childdocof{|\textit{main}|}| right away,
and no further \textsf{childdoc} directives will be processed.

%%%%%%%%%%%%%%%%%%%%%%%%%%%%%%%%%%%%%%%%
\DescribeMacro{\...prefix}
In the alternative form |\childdocforwardprefix|,
%
\begin{center}
\begin{tabular}{l}
|\input{childdoc.def}|\\
|\childdocforwardprefix[|\textit{main}|]{|\textit{prefix}|}{|\textit{dest}|}|
\end{tabular}
\end{center}
%
the destination file is determined by a pattern
depending on the current file:
To make this work, the current file must be called
`{\textit{prefix}\hspace{0.2em}\textit{suffix}}'
with \textit{prefix} matching precisely the argument.
Processing is then passed on to the file
`{\textit{dest}\hspace{0.2em}\textit{suffix}}'.
Surely, the same effect is achieved by
directly specifying the
argument `{\textit{dest}\hspace{0.2em}\textit{suffix}}'
in the first form.
However, that requires to set up a different file
for each child. With the alternative form of the command
all these files can have exactly the same content
which simplifies setting them up and maintaining them.

For example, the following file |draft.tex|
with a compilation flag |\version| as described in \secref{sec:flags}
compiles the main document as a draft:
%
\begin{center}
\begin{tabular}{l}
|\def\version{draft}|\\
|\input{childdoc.def}|\\
|\childdocforward{|\textit{main}|}|
\end{tabular}
\end{center}
%
Likewise, the following files |final|\textit{nn}|.tex|
compile the final version of the child document
|child|\textit{nn}|.tex|:
%
\begin{center}
\begin{tabular}{l}
|\def\version{final}|\\
|\input{childdoc.def}|\\
|\childdocforwardprefix{final}{child}|
\end{tabular}
\end{center}
%

Note that when several versions of a main file and/or of each child file
are to be generated, it may be convenient to set up a |Makefile| or
shell script to automatise the process.

%%%%%%%%%%%%%%%%%%%%%%%%%%%%%%%%%%%%%%%%%%%%%%%%%%%%%%%%%%%%%%%%%%%%%%%%%%%%%%%%
\subsection{Command Line Processing}
\label{sec:commandline}

The effect of redirection files can also be achieved by invoking
the \LaTeX{} compiler with a more elaborate command line.
Most conveniently this should be done as part
of a shell script or a |Makefile|.

When using \textsf{childdoc} in the main file, the following
command lines effectively perform a redirection
(note that depending on the shell being used,
backslashes may have to be doubled: `|\|' $\to$ `|\\|'):
%
\begin{center}
|... -jobname "|\textit{target}|" |\\|"|[\textit{flags}]%
|\input{childdoc.def}\childdocforward[|\textit{main}|]{|\textit{dest}|}"|
\end{center}
%
Here \textit{target} is the name of the output file,
\textit{main} is the name of the main file
and \textit{dest} is the name of the main or child file to be processed
(all filenames without extensions).
The optional argument \textit{main} can be omitted
if \textit{main} matches \textit{dest}.
Optionally, compilation \textit{flags} can be defined via |\def| commands.
This command line makes the \TeX{} engine believe
it is compiling the file \textit{target}
whose content is specified as the latter parameter.
The provided code then forwards the processing to
\textit{main} or \textit{dest} as described in \secref{sec:forward}.

%%%%%%%%%%%%%%%%%%%%%%%%%%%%%%%%%%%%%%%%%%%%%%%%%%%%%%%%%%%%%%%%%%%%%%%%%%%%%%%%
\subsection{Include by Input}
\label{sec:input}

Including child documents by |\include| has some restrictions by design.
Most notably, the content of a child document always occupies
its own set of pages; pages cannot be shared between child documents.
Usually, this behaviour makes perfect sense
because each child document contain an essential part of the document.
However, in some situations it may be desirable to compose
a document from a collection of parts
without having mandatory page breaks between then.
For this case, the package
provides a mechanism to include parts
by |\input| which can also be processed individually.
However, by construction this mechanism
requires manual handling of the content to be output.

%%%%%%%%%%%%%%%%%%%%%%%%%%%%%%%%%%%%%%%%
\DescribeMacro{\ifchilddocmanual}
The main file should be prepared as usual, see \secref{sec:include}.
However, the document body must make a distinction
between processing of an individual part and of the main document, e.g.:
%
\begin{center}
\begin{tabular}{l}
|\ifchilddocmanual|\\
|\input{\childdocname}|\\
|\||else|\\
\textit{document body with }|\input{|\textit{part}|}|\\
|\||fi|
\end{tabular}
\end{center}
%
The conditional |\ifchilddocmanual| is true whenever
a part to be included by |\input| is being compiled,
and the name of the part is stored in |\childdocname|.

%%%%%%%%%%%%%%%%%%%%%%%%%%%%%%%%%%%%%%%%
\DescribeMacro{\childdocby}
Each part to be included by |\input| should start with:
%
\begin{center}
\begin{tabular}{l}
|\input{childdoc.def}|\\
|\childdocby{|\textit{main}|}|\\
\end{tabular}
\end{center}
%
The directive |\childdocby| is similar to |\childdocof|
described in \secref{sec:include},
but the subsequent selection of content must be done manually.
To that end, both |\ifchilddoc| and |\ifchilddocmanual|
will be true upon processing of a part,
and the name of the part is stored in |\childdocname|.
Note that |\jobname| will be set to the filename of the current part
so that each part receives an individual |.aux| file
that does not interfere with the |.aux| file(s) of the main document.
This behaviour can be altered by the alternative form
|\childdocby[*]{|\textit{main}|}| (with a non-empty optional argument)
which uses the |.aux| file of the main document
by setting |\jobname| to \textit{main}.

%%%%%%%%%%%%%%%%%%%%%%%%%%%%%%%%%%%%%%%%%%%%%%%%%%%%%%%%%%%%%%%%%%%%%%%%%%%%%%%%
\subsection{Driver Development}
\label{sec:driver}

The \textsf{childdoc} mechanism can also be use for the development
of definition files such as \LaTeX{} styles or classes.
This case differs from the above setup with multiple parts
included by |\include| in that no |\includeonly| should be invoked.
This can be achieved by starting the include file
(before |\ProvidesPackage|) with:
%
\begin{center}
\begin{tabular}{l}
|\input{childdoc.def}|\\
|\childdocforward{|\textit{main}|}|\\
\end{tabular}
\end{center}
%
or alternatively with:
%
\begin{center}
\begin{tabular}{l}
|\input{childdoc.def}|\\
|\childdocby{|\textit{main}|}|\\
\end{tabular}
\end{center}
%
Both forms have slightly different effects as described above.
The main file is prepared as usual, see \secref{sec:include}.

%%%%%%%%%%%%%%%%%%%%%%%%%%%%%%%%%%%%%%%%%%%%%%%%%%%%%%%%%%%%%%%%%%%%%%%%%%%%%%%%
\subsection{Legacy Detection}
\label{sec:detection}

The directive |\childdocmain| in the main file can detect
whether the complete document or merely a child is to be compiled
even without using the directive |\childdocof|.
This method is deprecated because it is less robust
and there is no compelling reason to use it;
it is merely provided for backward compatibility
and it may be removed in future versions.

If the detection mechanism is to be used,
it is mandatory to correctly specify
the filename of the main file as the argument of |\childdocmain|:
%
\begin{center}
\begin{tabular}{l}
|\input{childdoc.def}|\\
|\childdocmain{|\textit{main}|}|\\
\end{tabular}
\end{center}
%
If |\jobname| does not match the argument \textit{main} of |\childdocmain|,
it is assumed that |\jobname| points to the child file to be compiled.
When using |\childdocmain| with the main file specified as argument,
it suffices to start a child file
with just |\input{|\textit{main}|}|
without loading of the package and using |\childdocof|.
If instead all processing is done
with the appropriate \textsf{childdoc} directives,
the argument of \textit{main} of |\childdocmain| can be empty.

An alternative version of the command line processing described
in \secref{sec:commandline} using the detection mechanism reads:
%
\begin{center}
|... -jobname "|\textit{target}|" "|[\textit{flags}]%
[|\def\jobname{|\textit{dest}|}|]|\input{|\textit{main}|}"|
\end{center}

%%%%%%%%%%%%%%%%%%%%%%%%%%%%%%%%%%%%%%%%%%%%%%%%%%%%%%%%%%%%%%%%%%%%%%%%%%%%%%%%
\subsection{Manual Code}
\label{sec:manual}

In case one cannot be certain whether the definitions file |childdoc.def|
is installed on the target \TeX{} distribution
and one prefers not to ship it,
it is conceivable to paste a few relevant commands into the sources.

To that end, drop all statements |\input{childdoc.def}|
and perform the replacements as outlined below.
Instead of |\childdocmain{|\textit{main}|}| add the following code
to the top of the main file:
%
\begin{center}
\begin{tabular}{l}
|\||ifdefined\childdocname\endinput\||fi\newif\ifchilddoc|\\
|\edef\childdocname{\scantokens\expandafter{\jobname\noexpand}}|\\
|\def\childdocmain{|\textit{main}|}\||ifx\childdocmain\childdocname\||else|\\
|\childdoctrue\includeonly{\childdocname}\let\jobname\childdocmain\||fi|\\
\end{tabular}
\end{center}
%
Instead of |\childdocof{|\textit{main}|}| just include the main file
at the top of each child file:
%
\begin{center}
|\input{|\textit{main}|}|
\end{center}
%
A simple redirection |\childdocforward{|\textit{dest}|}| is achieved by:
%
\begin{center}
|\def\jobname{|\textit{dest}|}\input{\jobname}|
\end{center}
%
The redirection with prefix
|\childdocforwardprefix[|\textit{prefix}|]{|\textit{dest}|}|
is accomplished by:
%
\begin{center}
\begin{tabular}{l}
|{\edef\jobname{\scantokens\expandafter{\jobname\noexpand}}|\\
|\def\redirectjob |\textit{prefix}|#1~~~{\gdef\jobname{|\textit{dest}|#1}}|\\
|\expandafter\redirectjob\jobname~~~}\input{\jobname}|
\end{tabular}
\end{center}

In an alternative approach,
child documents can be compiled by a specific command line
without additional code or specific definitions:
%
\begin{center}
|... -jobname "|\textit{target}|" "|[\textit{flags}]%
|\includeonly{|\textit{dest}|}\input{|\textit{main}|}"|
\end{center}
%

%%%%%%%%%%%%%%%%%%%%%%%%%%%%%%%%%%%%%%%%%%%%%%%%%%%%%%%%%%%%%%%%%%%%%%%%%%%%%%%%
%%%%%%%%%%%%%%%%%%%%%%%%%%%%%%%%%%%%%%%%%%%%%%%%%%%%%%%%%%%%%%%%%%%%%%%%%%%%%%%%
\section{Information}

%%%%%%%%%%%%%%%%%%%%%%%%%%%%%%%%%%%%%%%%%%%%%%%%%%%%%%%%%%%%%%%%%%%%%%%%%%%%%%%%
\subsection{Copyright}

Copyright \copyright{} 2017--2018 Niklas Beisert

This work may be distributed and/or modified under the
conditions of the \LaTeX{} Project Public License, either version 1.3
of this license or (at your option) any later version.
The latest version of this license is in
  \url{http://www.latex-project.org/lppl.txt}
and version 1.3 or later is part of all distributions of \LaTeX{}
version 2005/12/01 or later.

This work has the LPPL maintenance status `maintained'.

The Current Maintainer of this work is Niklas Beisert.

This work consists of the files |README.txt|, |childdoc.ins| and |childdoc.dtx|
as well as the derived files |childdoc.def|, |cdocsamp.tex|
with |cdocsch1.tex|, |cdocsch2.tex|, |cdocspt3.tex|, |cdocspt4.tex|,
|cdocsdrf.tex|, |cdocsfn1.tex|, |cdocsfn2.tex|
as well as |childdoc.pdf|.

%%%%%%%%%%%%%%%%%%%%%%%%%%%%%%%%%%%%%%%%%%%%%%%%%%%%%%%%%%%%%%%%%%%%%%%%%%%%%%%%
\subsection{Files and Installation}

The package consists of the files:
%
\begin{center}
\begin{tabular}{ll}
    |README.txt|   & readme file \\
    |childdoc.ins| & installation file \\
    |childdoc.dtx| & source file \\
    |childdoc.def| & definition file \\
    |cdocsamp.tex| & sample main file \\
    |cdocsch1.tex| & sample include file \\
    |cdocsch2.tex| & sample include file \\
    |cdocspt3.tex| & sample part file \\
    |cdocspt4.tex| & sample part file \\
    |cdocsdrf.tex| & sample redirection file \\
    |cdocsfn1.tex| & sample redirection file \\
    |cdocsfn2.tex| & sample redirection file \\
    |childdoc.pdf| & manual
\end{tabular}
\end{center}
%
The distribution consists of the files
|README.txt|, |childdoc.ins| and |childdoc.dtx|.
%
\begin{itemize}
\item
Run (pdf)\LaTeX{} on |childdoc.dtx|
to compile the manual |childdoc.pdf| (this file).
\item
Run \LaTeX{} on |childdoc.ins| to create the definitions file |childdoc.def|
and the sample |cdocsamp.tex| with include files
|cdocsch1.tex|, |cdocsch2.tex|, |cdocspt3.tex|, |cdocspt4.tex|,
|cdocsdrf.tex|, |cdocsfn1.tex|, |cdocsfn2.tex|.
Then copy the file |childdoc.def| to an appropriate directory of your \LaTeX{}
distribution, e.g.\ \textit{texmf-root}|/tex/latex/childdoc|.
\end{itemize}

%%%%%%%%%%%%%%%%%%%%%%%%%%%%%%%%%%%%%%%%%%%%%%%%%%%%%%%%%%%%%%%%%%%%%%%%%%%%%%%%
\subsection{Related CTAN Packages}

There are several other packages which offer a similar functionality:
%
\begin{itemize}
\item
The packages
\href{http://ctan.org/pkg/docmute}{\textsf{docmute}},
\href{http://ctan.org/pkg/includex}{\textsf{includex}} and
\href{http://ctan.org/pkg/standalone}{\textsf{standalone}}
provide commands to include only the document body of
a child file thus allowing both files to be compiled individually.
\item
The packages \href{http://ctan.org/pkg/subdocs}{\textsf{subdocs}}
and \href{http://ctan.org/pkg/subfiles}{\textsf{subfiles}}
provide structures in which the main and child documents can be
encapsulated and allowing them to be compiled individually.
The inclusion mechanism is different from the conventional |\include|.
\item
The package \href{http://ctan.org/pkg/combine}{\textsf{combine}}
is an elaborate solution to combine several documents into one.
\end{itemize}
%
See also the CTAN topic \href{http://ctan.org/topic/subdocs}{\textsf{subdocs}}
for further related packages.
The present package differs from the above solutions in that
a document structure constructed with the conventional |\include| mechanism
just needs two extra commands at the top of every file
such that all constituent files can be compiled individually.

%%%%%%%%%%%%%%%%%%%%%%%%%%%%%%%%%%%%%%%%%%%%%%%%%%%%%%%%%%%%%%%%%%%%%%%%%%%%%%%%
%\subsection{Feature Suggestions}
%
%The following is a list of features which may be useful for future
%versions of this package:
%%
%\begin{itemize}
%\item
%\ldots
%\end{itemize}

%%%%%%%%%%%%%%%%%%%%%%%%%%%%%%%%%%%%%%%%%%%%%%%%%%%%%%%%%%%%%%%%%%%%%%%%%%%%%%%%
\subsection{Revision History}

%%%%%%%%%%%%%%%%%%%%%%%%%%%%%%%%%%%%%%%%
\paragraph{v2.0:} 2018/12/30

\begin{itemize}
\item
immediate forward processing
\item
added |\childdocby| mechanism
\item
manual restructured
\end{itemize}

%%%%%%%%%%%%%%%%%%%%%%%%%%%%%%%%%%%%%%%%
\paragraph{v1.6:} 2018/01/17

\begin{itemize}
\item
application for development of include files
\item
corrections to manual
\end{itemize}

%%%%%%%%%%%%%%%%%%%%%%%%%%%%%%%%%%%%%%%%
\paragraph{v1.5:} 2017/05/21

\begin{itemize}
\item
more complete structuring introduced
\item
|\childdocof| introduced
\item
|\childdoc| renamed to |\childdocmain|
\item
|\childredirect| renamed to |\childdocforward| and |\childdocforwardprefix|
and functionality expanded
\end{itemize}

%%%%%%%%%%%%%%%%%%%%%%%%%%%%%%%%%%%%%%%%
\paragraph{v1.0:} 2017/04/27

\begin{itemize}
\item
manual and install package
\item
first version published on CTAN
\end{itemize}

%%%%%%%%%%%%%%%%%%%%%%%%%%%%%%%%%%%%%%%%
\paragraph{v0.6:} 2017/04/26

\begin{itemize}
\item
redirection mechanism added
\end{itemize}

%%%%%%%%%%%%%%%%%%%%%%%%%%%%%%%%%%%%%%%%
\paragraph{v0.5:} 2017/04/26

\begin{itemize}
\item
functionality in definition file
\end{itemize}


%%%%%%%%%%%%%%%%%%%%%%%%%%%%%%%%%%%%%%%%%%%%%%%%%%%%%%%%%%%%%%%%%%%%%%%%%%%%%%%%
%%%%%%%%%%%%%%%%%%%%%%%%%%%%%%%%%%%%%%%%%%%%%%%%%%%%%%%%%%%%%%%%%%%%%%%%%%%%%%%%
%%%%%%%%%%%%%%%%%%%%%%%%%%%%%%%%%%%%%%%%%%%%%%%%%%%%%%%%%%%%%%%%%%%%%%%%%%%%%%%%
\appendix

\settowidth\MacroIndent{\rmfamily\scriptsize 000\ }

 \DocInput{childdoc.dtx}

\end{document}
%</driver>
% \fi
%
% %%%%%%%%%%%%%%%%%%%%%%%%%%%%%%%%%%%%%%%%%%%%%%%%%%%%%%%%%%%%%%%%%%%%%%%%%%%%%%
% %%%%%%%%%%%%%%%%%%%%%%%%%%%%%%%%%%%%%%%%%%%%%%%%%%%%%%%%%%%%%%%%%%%%%%%%%%%%%%
% \section{Sample}
%\iffalse
%<*samplemain>
%\fi
%
% The following presents a sample document
% with two chapters, two parts, a title page,
% a compile flag as well as three forwarding files to set the flag.
% It consists of eight |.tex| files:
% \begin{center}
% \begin{tabular}{ll}
% |cdocsamp.tex|&main file\\
% |cdocsch1.tex|&include file for chapter 1\\
% |cdocsch2.tex|&include file for chapter 2\\
% |cdocspt3.tex|&include file for part 3\\
% |cdocspt4.tex|&include file for part 4\\
% |cdocsdrf.tex|&forwarding file for main file in draft mode\\
% |cdocsfi1.tex|&forwarding file for final version of chapter 1\\
% |cdocsfi2.tex|&forwarding file for final version of chapter 2\\
% \end{tabular}
% \end{center}
% Each of the eight files can be compiled directly by the \LaTeX{} compiler.
%
% %%%%%%%%%%%%%%%%%%%%%%%%%%%%%%%%%%%%%%
% \paragraph{Main File.}
%
% The main file is called |cdocsamp.tex|.
%
% Load the \textsf{childdoc} definitions and
% declare the filename for the main document:
%    \begin{macrocode}
\input{childdoc.def}
\childdocmain{}
%    \end{macrocode}

% Optional override for |\version| flag:
%    \begin{macrocode}
%%\ifchilddoc\else\providecommand{\version}{draft}\fi
%    \end{macrocode}

% Define the default values for the |\version| flag
% (|final| for the main file and |draft| for childs):
%    \begin{macrocode}
\ifchilddoc
\providecommand{\version}{draft}
\else
\providecommand{\version}{final}
\fi
%    \end{macrocode}

% Load the standard document class:
%    \begin{macrocode}
\documentclass[12pt]{article}
%    \end{macrocode}

% Start the document body:
%    \begin{macrocode}
\begin{document}
%    \end{macrocode}

% Declare a title page.
% Print title, part of document being processed and version flag:
%    \begin{macrocode}
\addtocounter{page}{-1}
\begin{center}
{\LARGE\bfseries{}childdoc example\par}
\vspace{1cm}
\ifchilddoc
\ifchilddocmanual part\else chapter\fi:
`\childdocname' of `\childdocjob'\par
\else
main document: `\childdocjob'\par
\fi
version: \version\par
\end{center}
\newpage
%    \end{macrocode}

% Manually include selected file,
% otherwise process as usual:
%    \begin{macrocode}
\ifchilddocmanual
\section*{part `\childdocname'}
\input{\childdocname}
\else
%    \end{macrocode}

% Include the two chapters:
%    \begin{macrocode}
\include{cdocsch1}
\include{cdocsch2}
%    \end{macrocode}

% Include the two parts unless only chapters should be displayed:
%    \begin{macrocode}
\ifchilddoc\else
\section{part three}
\input{cdocspt3}
\section{part four}
\input{cdocspt4}
\fi
%    \end{macrocode}

% Process as usual until here:
%    \begin{macrocode}
\fi
%    \end{macrocode}

% End of document body:
%    \begin{macrocode}
\end{document}
%    \end{macrocode}
%\iffalse
%</samplemain>
%\fi
%
% %%%%%%%%%%%%%%%%%%%%%%%%%%%%%%%%%%%%%%
% \paragraph{Chapter Include Files.}
%
% The include files are called |cdocsch1.tex| and |cdocsch2.tex|.
%
%\iffalse
%<*samplechap1|samplechap2>
%\fi

% Optional override for |\version| flag:
%    \begin{macrocode}
%%\providecommand{\version}{final}
%    \end{macrocode}

% Include the main document:
%    \begin{macrocode}
\input{childdoc.def}
\childdocof{cdocsamp}
%    \end{macrocode}

%\iffalse
%</samplechap1|samplechap2>
%\fi
%
%\iffalse
%<*samplechap1>
%\fi
% Some text for chapter 1:
%    \begin{macrocode}
\section{one}
some text in chapter one
%    \end{macrocode}

%\iffalse
%</samplechap1>
%\fi
% Some text for chapter 2:
%\iffalse
%<*samplechap2>
%\fi
%    \begin{macrocode}
\section{two}
more text in chapter two
%    \end{macrocode}

%\iffalse
%</samplechap2>
%\fi
%
% %%%%%%%%%%%%%%%%%%%%%%%%%%%%%%%%%%%%%%
% \paragraph{Part Include Files.}
%
% The include files are called |cdocspt3.tex| and |cdocspt4.tex|.
%
%\iffalse
%<*samplepart3|samplepart4>
%\fi

% Optional override for |\version| flag:
%    \begin{macrocode}
%%\providecommand{\version}{final}
%    \end{macrocode}

% Include the main document:
%    \begin{macrocode}
\input{childdoc.def}
\childdocby{cdocsamp}
%    \end{macrocode}

%\iffalse
%</samplepart3|samplepart4>
%\fi
%
%\iffalse
%<*samplepart3>
%\fi
% Some text for part 3:
%    \begin{macrocode}
some text in part three
%    \end{macrocode}

%\iffalse
%</samplepart3>
%\fi
% Some text for part 4:
%\iffalse
%<*samplepart4>
%\fi
%    \begin{macrocode}
more text in part four
%    \end{macrocode}

%\iffalse
%</samplepart4>
%\fi
%
% %%%%%%%%%%%%%%%%%%%%%%%%%%%%%%%%%%%%%%
% \paragraph{Forwarding for a Complete Draft.}
%
% The following forwarding file |cdocsdrf.tex|
% compiles the main document in draft mode:
%\iffalse
%<*sampledraft>
%\fi
%    \begin{macrocode}
\def\version{draft}
\input{childdoc.def}
\childdocforward{cdocsamp}
%    \end{macrocode}

%\iffalse
%</sampledraft>
%\fi
%
% %%%%%%%%%%%%%%%%%%%%%%%%%%%%%%%%%%%%%%
% \paragraph{Forwarding for Final Version of the Chapters.}
%
% The following forwarding files |cdocsfn1.tex| and |cdocsfn2.tex|
% (with identical content)
% compile the final versions of the child documents
% |cdocsch1.tex| and |cdocsch2.tex|, respectively:
%\iffalse
%<*samplefinal>
%\fi
%    \begin{macrocode}
\def\version{final}
\input{childdoc.def}
\childdocforwardprefix[cdocsamp]{cdocsfn}{cdocsch}
%    \end{macrocode}

%\iffalse
%</samplefinal>
%\fi
%
% %%%%%%%%%%%%%%%%%%%%%%%%%%%%%%%%%%%%%%
% \paragraph{Command Line Processing.}
%
% The following three command lines generate the output files
% |cdocscld|, |cdocscl1| and |cdocscl2|
% which should be identical to
% |cdocsdrf|, |cdocsch1| and |cdocsfn2|, respectively:
% \begin{center}
% \begin{tabular}{l}
% |latex -jobname cdocscld \|\\
% |  "\def\version{draft}\input{childdoc.def}\childdocforward{cdocsamp}"|\\
% |latex -jobname cdocscl1 \|\\
% |  "\input{childdoc.def}\childdocforward[cdocsamp]{cdocsch1}"|\\
% |latex -jobname cdocscl2 \|\\
% |  "\def\version{final}\input{childdoc.def}\childdocforward{cdocsch2}"|
% \end{tabular}
% \end{center}
% Note that the trailing backslash on each first line
% merely continues the input to the second line
% (for convenient cut ant paste).
% Furthermore, the command |latex| can be replaced by any
% of its alternative versions such as |pdflatex|.
%
% %%%%%%%%%%%%%%%%%%%%%%%%%%%%%%%%%%%%%%%%%%%%%%%%%%%%%%%%%%%%%%%%%%%%%%%%%%%%%%
% %%%%%%%%%%%%%%%%%%%%%%%%%%%%%%%%%%%%%%%%%%%%%%%%%%%%%%%%%%%%%%%%%%%%%%%%%%%%%%
% \section{Implementation}
%\iffalse
%<*package>
%\fi
%
% This section describes the definitions file |childdoc.def|.

% The definitions cannot be loaded using |\usepackage| or |\RequirePackage|
% which has a mechanism to prevent loading a style file more than once.
% When loading the definitions by means of |\input|
% multiple instances have to be prevented manually:
%\iffalse
%This code needs to be before the `\ProvidesFile' directive
%which is defined at the beginning of this file.
%Therefore it is also placed there and commented out here.
%</package>
%<*discard>
%\fi
%    \begin{macrocode}
\ifdefined\childdocmain\endinput\fi
%    \end{macrocode}
%\iffalse
%</discard>
%<*package>
%\fi
%
% \macro{\ifchilddoc}
% \macro{\ifchilddocmanual}
% The conditional |\ifchilddoc| tells whether a
% child (true) or main (false) document is being compiled.
% The conditional |\ifchilddocmanual| tells whether
% the |\includeonly| mechanism is used (false) or
% the selection of child files must be performed manually (true).
% The definitions initialise to false:
%    \begin{macrocode}
\newif\ifchilddoc
\newif\ifchilddocmanual
%    \end{macrocode}

% \macro{\childdocname}
% \macro{\childdocjob}
% The macro |\childdocname| stores the name of the main document
% to be compiled. The macro |\childdocjob| stores the name of
% the document on which the \LaTeX{} compiler was originally invoked.
% The content of |\jobname| cannot be compared
% to filenames specified in the source due to different catcodes.
% The following code rescans |\jobname|, stores the result
% in |\childdocname| and saves a copy in |\childdocjob|:
%    \begin{macrocode}
\edef\childdocname{\scantokens\expandafter{\jobname\noexpand}}
\let\childdocjob\childdocname
%    \end{macrocode}

% \macro{\childdocdisable}
% The macro |\childdocdisable| prevents the main file
% from being processed more than once.
% At this stage, the main document command |\childdocmain|
% is assumed to be called once again where it should do nothing.
% Any subsequent call to it should prevent
% a secondary processing of the main document
% It overwrites the forwarding commands
% |\childdocof| and |\childdocforward|
% with empty macros to prevent further inclusions of the main document:
%    \begin{macrocode}
\newcommand{\childdocdisable}
{
  \renewcommand{\childdocmain}[1]{\renewcommand{\childdocmain}[1]{\endinput}}
  \renewcommand{\childdocof}[1]{}
  \renewcommand{\childdocby}[2][]{}
  \renewcommand{\childdocforward}[2][]{}
  \renewcommand{\childdocdisable}{}
}
%    \end{macrocode}

% \macro{\childdocmain}
% The macro |\childdocmain| is to be called at the top of the main file
% with nothing or the main filename (without extension) as argument.
% First, it breaks loops.
% If the argument is not empty and does not match |\childdocname|
% (which is set by the first inclusion of |childdoc.def|),
% |\ifchilddoc| is set to true, |\includeonly| is applied to the child file
% and |\jobname| is set to the main file
% (for proper handling of |.aux| files):
%    \begin{macrocode}
\newcommand{\childdocmain}[1]
{
  \childdocdisable\childdocmain{}
  \if?#1?\else
    \begingroup
      \def\childdoctmp{#1}
      \ifx\childdoctmp\childdocname
        \def\childdoctmp{}
      \else
        \def\childdoctmp
        {
          \childdoctrue
          \includeonly{\childdocname}
          \def\childdocjob{#1}
          \def\jobname{#1}
        }
      \fi
      \expandafter
    \endgroup
    \childdoctmp
  \fi
}
%    \end{macrocode}

% \macro{\childdocof}
% The command |\childdocof| redirects
% compilation to the main file |#1|.
%    \begin{macrocode}
\newcommand{\childdocof}[1]
{
  \childdocdisable
  \childdoctrue
  \includeonly{\childdocname}
  \def\jobname{#1}
  \def\childdocjob{#1}
  \input{#1}
}
%    \end{macrocode}

% \macro{\childdocby}
% The command |\childdocby| ....
%    \begin{macrocode}
\newcommand{\childdocby}[2][]
{
  \childdocdisable
  \childdoctrue
  \childdocmanualtrue
  \if?#1?\else
    \def\jobname{#2}
  \fi
  \def\childdocjob{#2}
  \input{#2}
  \endinput
}
%    \end{macrocode}

% \macro{\childdocforward}
% The command |\childdocforward| redirects
% compilation to the main file or
% (if the optional argument is given) a child file.
% Parameters are set as if the main file
% or a child file starting with |\childdocof| was compiled.
% Then compilation is handed over to the main file:
%    \begin{macrocode}
\newcommand{\childdocforward}[2][]
{
  \begingroup
    \if?#1?
      \def\childdoctmp
      {
        \def\childdocname{#2}
        \def\childdocjob{#2}
        \def\jobname{#2}
        \input{#2}
        \endinput
      }
    \else
      \def\childdoctmp
      {
        \childdocdisable
        \def\childdocname{#2}
        \childdoctrue
        \includeonly{#2}
        \def\childdocjob{#1}
        \def\jobname{#1}
        \input{#1}
        \endinput
      }
    \fi
    \expandafter
  \endgroup
  \childdoctmp
}
%    \end{macrocode}

% \macro{\childdocforwardprefix}
% The command |\childdocforwardprefix| redirects
% compilation to the main or a child file by means of a pattern.
% The prefix |#1| in the current filename is replaced by |#2|
% and the suffix of the current filename is kept
% (it is assumed that the filename does not contain the substring `|~~~|'
% which is used as a delimiter).
% Compilation is handed over to the new file by |\childdocforward|:
%    \begin{macrocode}
\newcommand{\childdocforwardprefix}[3][]
{
  \begingroup
    \def\childdocextract #2##1~~~{\def\childdoctmp{\childdocforward[#1]{#3##1}}}
    \expandafter\childdocextract\childdocname~~~
    \expandafter
  \endgroup
  \childdoctmp
}
%    \end{macrocode}

% \macro{\childdoc}
% The deprecated macro |\childdoc| is a legacy version of |\childdocmain|:
%    \begin{macrocode}
\newcommand{\childdoc}{\childdocmain}
%    \end{macrocode}

% \macro{\childdocredirect}
% The deprecated macro |\childdocredirect| is a legacy version
% of |\childdocforward| and |\childdocforwardprefix|:
%    \begin{macrocode}
\newcommand{\childdocredirect}[2][]
{
  \begingroup
    \if?#1?
      \def\childdoctmp{\childdocforward{#2}}
    \else
      \def\childdoctmp{\childdocforwardprefix{#1}{#2}}
    \fi
    \expandafter
  \endgroup
  \childdoctmp
}
%    \end{macrocode}

%\iffalse
%</package>
%\fi
%
\endinput
|\\
|\childdocby{|\textit{main}|}|\\
\end{tabular}
\end{center}
%
Both forms have slightly different effects as described above.
The main file is prepared as usual, see \secref{sec:include}.

%%%%%%%%%%%%%%%%%%%%%%%%%%%%%%%%%%%%%%%%%%%%%%%%%%%%%%%%%%%%%%%%%%%%%%%%%%%%%%%%
\subsection{Legacy Detection}
\label{sec:detection}

The directive |\childdocmain| in the main file can detect
whether the complete document or merely a child is to be compiled
even without using the directive |\childdocof|.
This method is deprecated because it is less robust
and there is no compelling reason to use it;
it is merely provided for backward compatibility
and it may be removed in future versions.

If the detection mechanism is to be used,
it is mandatory to correctly specify
the filename of the main file as the argument of |\childdocmain|:
%
\begin{center}
\begin{tabular}{l}
|% \iffalse
%
% childdoc.dtx Copyright (C) 2017-2018 Niklas Beisert
%
% This work may be distributed and/or modified under the
% conditions of the LaTeX Project Public License, either version 1.3
% of this license or (at your option) any later version.
% The latest version of this license is in
%   http://www.latex-project.org/lppl.txt
% and version 1.3 or later is part of all distributions of LaTeX
% version 2005/12/01 or later.
%
% This work has the LPPL maintenance status `maintained'.
%
% The Current Maintainer of this work is Niklas Beisert.
%
% This work consists of the files childdoc.dtx and childdoc.ins
% and the derived files childdoc.def and cdocsamp.tex with
% cdocsch1.tex, cdocsch2.tex, cdocsdrf.tex, cdocsfn1.tex, cdocsfn2.tex.
%
%<package>\ifdefined\childdocmain\endinput\fi
%<package>\ProvidesFile{childdoc.def}[2018/12/30 v2.0 child document driver]
%<samplemain>\ProvidesFile{cdocsamp.tex}[2018/12/30 v2.0 sample for childdoc]
%<*driver>
%\ProvidesFile{childdoc.drv}[2018/12/30 v2.0 childdoc reference manual file]
\PassOptionsToClass{10pt,a4paper}{article}
\documentclass{ltxdoc}

\usepackage[margin=35mm]{geometry}
\usepackage{hyperref}
\usepackage{hyperxmp}
\usepackage[usenames]{color}

\hypersetup{colorlinks=true}
\hypersetup{pdfstartview=FitH}
\hypersetup{pdfpagemode=UseNone}
\hypersetup{pdfsource={}}
\hypersetup{pdflang={en-UK}}
\hypersetup{pdfcopyright={Copyright 2017-2018 Niklas Beisert.
  This work may be distributed and/or modified under the
  conditions of the LaTeX Project Public License, either version 1.3
  of this license or (at your option) any later version.}}
\hypersetup{pdflicenseurl={http://www.latex-project.org/lppl.txt}}
\hypersetup{pdfcontactaddress={ETH Zurich, ITP, HIT K,
  Wolfgang-Pauli-Strasse 27}}
\hypersetup{pdfcontactpostcode={8093}}
\hypersetup{pdfcontactcity={Zurich}}
\hypersetup{pdfcontactcountry={Switzerland}}
\hypersetup{pdfcontactemail={nbeisert@itp.phys.ethz.ch}}
\hypersetup{pdfcontacturl={http://people.phys.ethz.ch/\xmptilde nbeisert/}}

\newcommand{\secref}[1]{\hyperref[#1]{section \ref*{#1}}}

\parskip1ex
\parindent0pt
\let\olditemize\itemize
\def\itemize{\olditemize\parskip0pt}

\begin{document}

\title{The \textsf{childdoc} Package}
\hypersetup{pdftitle={The childdoc Package}}
\author{Niklas Beisert\\[2ex]
  Institut f\"ur Theoretische Physik\\
  Eidgen\"ossische Technische Hochschule Z\"urich\\
  Wolfgang-Pauli-Strasse 27, 8093 Z\"urich, Switzerland\\[1ex]
  \href{mailto:nbeisert@itp.phys.ethz.ch}
  {\texttt{nbeisert@itp.phys.ethz.ch}}}
\hypersetup{pdfauthor={Niklas Beisert}}
\hypersetup{pdfsubject={Manual for the LaTeX2e Package childdoc}}
\date{30 December 2018, \textsf{v2.0}}
\maketitle

\begin{abstract}\noindent
\textsf{childdoc} is a \LaTeXe{} package
that enables the direct compilation
of document sections included by |\include|
to individual files.
\end{abstract}

\begingroup
\parskip0ex
\tableofcontents
\endgroup

%%%%%%%%%%%%%%%%%%%%%%%%%%%%%%%%%%%%%%%%%%%%%%%%%%%%%%%%%%%%%%%%%%%%%%%%%%%%%%%%
%%%%%%%%%%%%%%%%%%%%%%%%%%%%%%%%%%%%%%%%%%%%%%%%%%%%%%%%%%%%%%%%%%%%%%%%%%%%%%%%
\section{Introduction}

\LaTeX{} provides a mechanism to structure a large document (such as a book)
into a main file and several child files (containing the chapters)
using the |\include| command.
This mechanism is beneficial for documents
which span hundreds of pages in order to
make the source file(s) more manageable.
Moreover, compilation can be restricted to
selected child files by means of the |\includeonly| command.
The latter feature can be used to reduce the compilation time while editing
(this was significantly more useful in the earlier days of \LaTeX{})
or to generate a smaller document which is easier to navigate.
Another application of |\includeonly| is to generate
documents consisting of selected parts of the complete document.

However, there are a few drawbacks of the plain |\include| mechanism:
\begin{itemize}
\item
The child files cannot be compiled on their own,
they can only be compiled via the main file.
A naive editing environment
(such as a text editor with an option
to have the current file processed by \LaTeX)
may require one to switch to the main file before compiling;
attempting to compile the child file produces errors.
\item
The main file must be modified (each time)
to adjust the |\includeonly| command
to the present needs. This easily leaves the main file in a messy state.
\item
The generated document will always carry the filename
of the main document. This is inconvenient if
several child files are to be compiled and
to be kept for distribution.
\end{itemize}

The present package provides a simple interface
to make child files individually compilable by \LaTeX{}.
Compiling a child file then has the same effect as compiling
the main file with an |\includeonly| command
to select the appropriate child.
Moreover the generated document will carry the name of the child
rather than the main file.
This resolves all three above issues.

This feature is meant to make the editing of books,
thesis documents and lecture notes somewhat more convenient.
However, the package can also be used efficiently for
composing a series of documents (such as exercise sheets)
which are typically distributed individually.
It then assists the author in generating the individual documents
(potentially in different versions)
as well as a document containing the collected series.
Another application is in developing style files
or other kinds of included material
where compilation of the style file could redirect
to a sample or test file.

%%%%%%%%%%%%%%%%%%%%%%%%%%%%%%%%%%%%%%%%%%%%%%%%%%%%%%%%%%%%%%%%%%%%%%%%%%%%%%%%
%%%%%%%%%%%%%%%%%%%%%%%%%%%%%%%%%%%%%%%%%%%%%%%%%%%%%%%%%%%%%%%%%%%%%%%%%%%%%%%%
\section{Usage}

First of all, the package \textsf{childdoc} is \emph{not} a standard
\LaTeXe{} |.sty| style file! Therefore it needs to be invoked in
a non-standard way.

%%%%%%%%%%%%%%%%%%%%%%%%%%%%%%%%%%%%%%%%%%%%%%%%%%%%%%%%%%%%%%%%%%%%%%%%%%%%%%%%
\subsection{Included Files}
\label{sec:include}

%%%%%%%%%%%%%%%%%%%%%%%%%%%%%%%%%%%%%%%%
\DescribeMacro{\childdocmain}
To use the package, add the commands
\begin{center}
\begin{tabular}{l}
|\input{childdoc.def}|\\
|\childdocmain{}|\\
\end{tabular}
\end{center}
at the very top of the main \LaTeX{} file,
in particular \emph{before} the |\documentclass| statement!
The argument of |\childdocmain| should be left empty
(but it must be present).

%%%%%%%%%%%%%%%%%%%%%%%%%%%%%%%%%%%%%%%%
\DescribeMacro{\childdocof}
Furthermore, add the commands
\begin{center}
\begin{tabular}{l}
|\input{childdoc.def}|\\
|\childdocof{|\textit{main}|}|\\
\end{tabular}
\end{center}
at the top of every child file \textit{child}
which is included by |\include{|\textit{child}|}|
from within the main file
(or at least for those files to be compiled individually).
The argument \textit{main} must be the filename of the main file.

There are a couple of
considerations in setting up the main and child documents:

%%%%%%%%%%%%%%%%%%%%%%%%%%%%%%%%%%%%%%%%
\paragraph{Restrictions.}

Please note the following restrictions:
\begin{itemize}
\item
|\childdocmain| must be called with one argument \textit{main}
to ensure compatibility with earlier version of the package.
It must either be empty (|\childdocmain{}|)
or precisely match the filename of the main file in which it is specified.
See \secref{sec:detection} for further information.
\item
The filename \textit{main} must be specified without the |.tex| extension.
\item
The filename \textit{main} is case sensitive
(even in case-insensitive file systems)
due to internal string comparison.
\item
The argument \textit{main} should be fully expanded, it cannot be a macro.
\item
Subdirectories and special characters should be avoided in filenames.
\item
The command |\childdocmain{|\textit{main}|}| must be followed by a whitespace.
It should not be followed immediately by another command
or by a comment mark `|%|'.
This is because the \TeX{} parser reads the token immediately following
the argument of |\childdocmain| and puts it
at the beginning of every child section;
however, a white\-space is ignored.
\end{itemize}

%%%%%%%%%%%%%%%%%%%%%%%%%%%%%%%%%%%%%%%%
\paragraph{Content of Main File.}

It is advisable to place all content in the child files included by |\include|.
Any output contained in the main file will appear in all child documents
unless suppressed manually;
it cannot be suppressed automatically by the |\includeonly| directive
and thus should normally be avoided.
A method to include some content in the main file
by means of conditional processing is described in \secref{sec:conditional}.

%%%%%%%%%%%%%%%%%%%%%%%%%%%%%%%%%%%%%%%%
\paragraph{Page Numbering.}

When only a part of the document is compiled,
the appropriate numbering of pages
(as well as other status parameters)
is determined from the |.aux| files.
The latter contain information from previous passes.
However this information needs to propagate through
all intermediate child documents.
Therefore the page numbering in child documents may well
be inconsistent until the complete document is compiled at least once.

A useful (if unconventional) way to always ensure a consistent
page numbering is to restart the numbering in each child document
and denote the pages by `\textit{child}|.|\textit{page}'
where \textit{child} represents the chapter/section number of the child file.
This can be achieved by the command
|\numberwithin{page}{|\textit{child}|}|
of the \textsf{amsmath} package
where \textit{child} can be |chapter| or |section|
depending on the chosen structuring.
Alternatively, one can modify the macro |\thepage| appropriately
and reset the counter |page| at the start of each child file.

%%%%%%%%%%%%%%%%%%%%%%%%%%%%%%%%%%%%%%%%%%%%%%%%%%%%%%%%%%%%%%%%%%%%%%%%%%%%%%%%
\subsection{Conditional Processing}
\label{sec:conditional}

The package provides a mechanism to compile different versions
of a document. To customise the versions further some conditional processing
can come in handy to distinguish which version is being compiled.
The package provides two macros to describe the compilation context:

%%%%%%%%%%%%%%%%%%%%%%%%%%%%%%%%%%%%%%%%
\DescribeMacro{\ifchilddoc}
The conditional |\ifchilddoc| distinguishes between the compilation of
child documents and the main document:
%
\begin{center}
|\ifchilddoc |\textit{child-code}| |[|\||else |\textit{main-code}]| \||fi|
\end{center}

%%%%%%%%%%%%%%%%%%%%%%%%%%%%%%%%%%%%%%%%
\DescribeMacro{\childdocname}
\DescribeMacro{\childdocjob}
The macro |\childdocname| contains the filename (without extension)
of the main or child file being processed.
Note that |\childdocjob| will always contain the name of the main file.

%%%%%%%%%%%%%%%%%%%%%%%%%%%%%%%%%%%%%%%%
\paragraph{Title Page.}

Conditional processing can be used to include a title or banner page
in the main document when proper precautions are taken.
Importantly, the code in the main file should ensure that the page counter
(as well as other status parameters which are stored in the |.aux| files)
takes the same value after the conditional processing.
Otherwise the page numbers may take divergent values
depending on which part is compiled.

For example, a title page could be declared by:
%
\begin{center}
\begin{tabular}{l}
|\ifchilddoc\||else|\\
|\addtocounter{page}{-1}|\\
\textit{code for title page}\\
|\newpage|\\
|\||fi|
\end{tabular}
\end{center}
%
A banner page for the child documents can be generated by:
%
\begin{center}
\begin{tabular}{l}
|\ifchilddoc|\\
|\addtocounter{page}{-1}|\\
\textit{code for banner page}\\
|\newpage|\\
|\||fi|
\end{tabular}
\end{center}
%
Here one could write a message such as:
\begin{center}
|This is the part \childdocname{} of \childdocjob{}.|
\end{center}

%%%%%%%%%%%%%%%%%%%%%%%%%%%%%%%%%%%%%%%%%%%%%%%%%%%%%%%%%%%%%%%%%%%%%%%%%%%%%%%%
\subsection{Flags}
\label{sec:flags}

The package makes it easy to generate different versions
of the main or child documents.
To this end compilation flags can be defined
and assigned different default values.
They will be particularly useful in conjunction
with the forwarding mechanism described in \secref{sec:forward}.

For example, it may be useful to have a flag |\version|
which can be set to |draft| or |final|.
The document source will contain some conditional code
depending on the value of |\version|.
Suppose further, the flag should default to |final| for the main file
and to |draft| for child files
which is a natural assignment for editing the document.
This is achieved by placing the following code
in the preamble of the main document
(below the |\childdocmain| directive):
%
\begin{center}
\begin{tabular}{l}
|\ifchilddoc|\\
|\providecommand{\version}{draft}|\\
|\||else|\\
|\providecommand{\version}{final}|\\
|\||fi|
\end{tabular}
\end{center}
%
The definition by |\providecommand| makes sure
that previous definitions are not overwritten.
Further statements |\providecommand{\version}{...}|
can thus be added before the above code to override it.

For the main file, one might add a line
(between |\childdocmain| and the above block)
%
\begin{center}
|%\ifchilddoc\||else\providecommand{\version}{draft}\||fi|
\end{center}
%
which can be uncommented to produce a draft version.
Likewise one can add a line to the very top of a child file
(above the |\childdocof{|\textit{main}|}| directive)
%
\begin{center}
|%\providecommand{\version}{final}|
\end{center}
%
which can be uncommented to produce the final version of this child document.

%%%%%%%%%%%%%%%%%%%%%%%%%%%%%%%%%%%%%%%%%%%%%%%%%%%%%%%%%%%%%%%%%%%%%%%%%%%%%%%%
\subsection{Forwarding}
\label{sec:forward}

Different versions of the main or child documents
using compilation flags as described in \secref{sec:flags}
can be (permanently) stored in different files
for convenient compilation, viewing and distribution.
To this end, the package defines a command
to pass on compilation to a different file:

%%%%%%%%%%%%%%%%%%%%%%%%%%%%%%%%%%%%%%%%
\DescribeMacro{\childdocforward}
The command |\childdocforward| redirects processing to
another source file:
%
\begin{center}
\begin{tabular}{l}
|\input{childdoc.def}|\\
|\childdocforward[|\textit{main}|]{|\textit{dest}|}|\\
\end{tabular}
\end{center}
%
The argument \textit{dest} is the destination file
(without extension).
It should be the main file or one of the child files.
Note that further \textsf{childdoc} directives
such as |\childdocof| and |\childdocforward|
in the indicated file will be processed in this form.
The optional argument \textit{main}
passes on directly to the main file \textit{main}
while pretending to compile the child \textit{dest}.
This form behaves as if \textit{dest}
issues |\childdocof{|\textit{main}|}| right away,
and no further \textsf{childdoc} directives will be processed.

%%%%%%%%%%%%%%%%%%%%%%%%%%%%%%%%%%%%%%%%
\DescribeMacro{\...prefix}
In the alternative form |\childdocforwardprefix|,
%
\begin{center}
\begin{tabular}{l}
|\input{childdoc.def}|\\
|\childdocforwardprefix[|\textit{main}|]{|\textit{prefix}|}{|\textit{dest}|}|
\end{tabular}
\end{center}
%
the destination file is determined by a pattern
depending on the current file:
To make this work, the current file must be called
`{\textit{prefix}\hspace{0.2em}\textit{suffix}}'
with \textit{prefix} matching precisely the argument.
Processing is then passed on to the file
`{\textit{dest}\hspace{0.2em}\textit{suffix}}'.
Surely, the same effect is achieved by
directly specifying the
argument `{\textit{dest}\hspace{0.2em}\textit{suffix}}'
in the first form.
However, that requires to set up a different file
for each child. With the alternative form of the command
all these files can have exactly the same content
which simplifies setting them up and maintaining them.

For example, the following file |draft.tex|
with a compilation flag |\version| as described in \secref{sec:flags}
compiles the main document as a draft:
%
\begin{center}
\begin{tabular}{l}
|\def\version{draft}|\\
|\input{childdoc.def}|\\
|\childdocforward{|\textit{main}|}|
\end{tabular}
\end{center}
%
Likewise, the following files |final|\textit{nn}|.tex|
compile the final version of the child document
|child|\textit{nn}|.tex|:
%
\begin{center}
\begin{tabular}{l}
|\def\version{final}|\\
|\input{childdoc.def}|\\
|\childdocforwardprefix{final}{child}|
\end{tabular}
\end{center}
%

Note that when several versions of a main file and/or of each child file
are to be generated, it may be convenient to set up a |Makefile| or
shell script to automatise the process.

%%%%%%%%%%%%%%%%%%%%%%%%%%%%%%%%%%%%%%%%%%%%%%%%%%%%%%%%%%%%%%%%%%%%%%%%%%%%%%%%
\subsection{Command Line Processing}
\label{sec:commandline}

The effect of redirection files can also be achieved by invoking
the \LaTeX{} compiler with a more elaborate command line.
Most conveniently this should be done as part
of a shell script or a |Makefile|.

When using \textsf{childdoc} in the main file, the following
command lines effectively perform a redirection
(note that depending on the shell being used,
backslashes may have to be doubled: `|\|' $\to$ `|\\|'):
%
\begin{center}
|... -jobname "|\textit{target}|" |\\|"|[\textit{flags}]%
|\input{childdoc.def}\childdocforward[|\textit{main}|]{|\textit{dest}|}"|
\end{center}
%
Here \textit{target} is the name of the output file,
\textit{main} is the name of the main file
and \textit{dest} is the name of the main or child file to be processed
(all filenames without extensions).
The optional argument \textit{main} can be omitted
if \textit{main} matches \textit{dest}.
Optionally, compilation \textit{flags} can be defined via |\def| commands.
This command line makes the \TeX{} engine believe
it is compiling the file \textit{target}
whose content is specified as the latter parameter.
The provided code then forwards the processing to
\textit{main} or \textit{dest} as described in \secref{sec:forward}.

%%%%%%%%%%%%%%%%%%%%%%%%%%%%%%%%%%%%%%%%%%%%%%%%%%%%%%%%%%%%%%%%%%%%%%%%%%%%%%%%
\subsection{Include by Input}
\label{sec:input}

Including child documents by |\include| has some restrictions by design.
Most notably, the content of a child document always occupies
its own set of pages; pages cannot be shared between child documents.
Usually, this behaviour makes perfect sense
because each child document contain an essential part of the document.
However, in some situations it may be desirable to compose
a document from a collection of parts
without having mandatory page breaks between then.
For this case, the package
provides a mechanism to include parts
by |\input| which can also be processed individually.
However, by construction this mechanism
requires manual handling of the content to be output.

%%%%%%%%%%%%%%%%%%%%%%%%%%%%%%%%%%%%%%%%
\DescribeMacro{\ifchilddocmanual}
The main file should be prepared as usual, see \secref{sec:include}.
However, the document body must make a distinction
between processing of an individual part and of the main document, e.g.:
%
\begin{center}
\begin{tabular}{l}
|\ifchilddocmanual|\\
|\input{\childdocname}|\\
|\||else|\\
\textit{document body with }|\input{|\textit{part}|}|\\
|\||fi|
\end{tabular}
\end{center}
%
The conditional |\ifchilddocmanual| is true whenever
a part to be included by |\input| is being compiled,
and the name of the part is stored in |\childdocname|.

%%%%%%%%%%%%%%%%%%%%%%%%%%%%%%%%%%%%%%%%
\DescribeMacro{\childdocby}
Each part to be included by |\input| should start with:
%
\begin{center}
\begin{tabular}{l}
|\input{childdoc.def}|\\
|\childdocby{|\textit{main}|}|\\
\end{tabular}
\end{center}
%
The directive |\childdocby| is similar to |\childdocof|
described in \secref{sec:include},
but the subsequent selection of content must be done manually.
To that end, both |\ifchilddoc| and |\ifchilddocmanual|
will be true upon processing of a part,
and the name of the part is stored in |\childdocname|.
Note that |\jobname| will be set to the filename of the current part
so that each part receives an individual |.aux| file
that does not interfere with the |.aux| file(s) of the main document.
This behaviour can be altered by the alternative form
|\childdocby[*]{|\textit{main}|}| (with a non-empty optional argument)
which uses the |.aux| file of the main document
by setting |\jobname| to \textit{main}.

%%%%%%%%%%%%%%%%%%%%%%%%%%%%%%%%%%%%%%%%%%%%%%%%%%%%%%%%%%%%%%%%%%%%%%%%%%%%%%%%
\subsection{Driver Development}
\label{sec:driver}

The \textsf{childdoc} mechanism can also be use for the development
of definition files such as \LaTeX{} styles or classes.
This case differs from the above setup with multiple parts
included by |\include| in that no |\includeonly| should be invoked.
This can be achieved by starting the include file
(before |\ProvidesPackage|) with:
%
\begin{center}
\begin{tabular}{l}
|\input{childdoc.def}|\\
|\childdocforward{|\textit{main}|}|\\
\end{tabular}
\end{center}
%
or alternatively with:
%
\begin{center}
\begin{tabular}{l}
|\input{childdoc.def}|\\
|\childdocby{|\textit{main}|}|\\
\end{tabular}
\end{center}
%
Both forms have slightly different effects as described above.
The main file is prepared as usual, see \secref{sec:include}.

%%%%%%%%%%%%%%%%%%%%%%%%%%%%%%%%%%%%%%%%%%%%%%%%%%%%%%%%%%%%%%%%%%%%%%%%%%%%%%%%
\subsection{Legacy Detection}
\label{sec:detection}

The directive |\childdocmain| in the main file can detect
whether the complete document or merely a child is to be compiled
even without using the directive |\childdocof|.
This method is deprecated because it is less robust
and there is no compelling reason to use it;
it is merely provided for backward compatibility
and it may be removed in future versions.

If the detection mechanism is to be used,
it is mandatory to correctly specify
the filename of the main file as the argument of |\childdocmain|:
%
\begin{center}
\begin{tabular}{l}
|\input{childdoc.def}|\\
|\childdocmain{|\textit{main}|}|\\
\end{tabular}
\end{center}
%
If |\jobname| does not match the argument \textit{main} of |\childdocmain|,
it is assumed that |\jobname| points to the child file to be compiled.
When using |\childdocmain| with the main file specified as argument,
it suffices to start a child file
with just |\input{|\textit{main}|}|
without loading of the package and using |\childdocof|.
If instead all processing is done
with the appropriate \textsf{childdoc} directives,
the argument of \textit{main} of |\childdocmain| can be empty.

An alternative version of the command line processing described
in \secref{sec:commandline} using the detection mechanism reads:
%
\begin{center}
|... -jobname "|\textit{target}|" "|[\textit{flags}]%
[|\def\jobname{|\textit{dest}|}|]|\input{|\textit{main}|}"|
\end{center}

%%%%%%%%%%%%%%%%%%%%%%%%%%%%%%%%%%%%%%%%%%%%%%%%%%%%%%%%%%%%%%%%%%%%%%%%%%%%%%%%
\subsection{Manual Code}
\label{sec:manual}

In case one cannot be certain whether the definitions file |childdoc.def|
is installed on the target \TeX{} distribution
and one prefers not to ship it,
it is conceivable to paste a few relevant commands into the sources.

To that end, drop all statements |\input{childdoc.def}|
and perform the replacements as outlined below.
Instead of |\childdocmain{|\textit{main}|}| add the following code
to the top of the main file:
%
\begin{center}
\begin{tabular}{l}
|\||ifdefined\childdocname\endinput\||fi\newif\ifchilddoc|\\
|\edef\childdocname{\scantokens\expandafter{\jobname\noexpand}}|\\
|\def\childdocmain{|\textit{main}|}\||ifx\childdocmain\childdocname\||else|\\
|\childdoctrue\includeonly{\childdocname}\let\jobname\childdocmain\||fi|\\
\end{tabular}
\end{center}
%
Instead of |\childdocof{|\textit{main}|}| just include the main file
at the top of each child file:
%
\begin{center}
|\input{|\textit{main}|}|
\end{center}
%
A simple redirection |\childdocforward{|\textit{dest}|}| is achieved by:
%
\begin{center}
|\def\jobname{|\textit{dest}|}\input{\jobname}|
\end{center}
%
The redirection with prefix
|\childdocforwardprefix[|\textit{prefix}|]{|\textit{dest}|}|
is accomplished by:
%
\begin{center}
\begin{tabular}{l}
|{\edef\jobname{\scantokens\expandafter{\jobname\noexpand}}|\\
|\def\redirectjob |\textit{prefix}|#1~~~{\gdef\jobname{|\textit{dest}|#1}}|\\
|\expandafter\redirectjob\jobname~~~}\input{\jobname}|
\end{tabular}
\end{center}

In an alternative approach,
child documents can be compiled by a specific command line
without additional code or specific definitions:
%
\begin{center}
|... -jobname "|\textit{target}|" "|[\textit{flags}]%
|\includeonly{|\textit{dest}|}\input{|\textit{main}|}"|
\end{center}
%

%%%%%%%%%%%%%%%%%%%%%%%%%%%%%%%%%%%%%%%%%%%%%%%%%%%%%%%%%%%%%%%%%%%%%%%%%%%%%%%%
%%%%%%%%%%%%%%%%%%%%%%%%%%%%%%%%%%%%%%%%%%%%%%%%%%%%%%%%%%%%%%%%%%%%%%%%%%%%%%%%
\section{Information}

%%%%%%%%%%%%%%%%%%%%%%%%%%%%%%%%%%%%%%%%%%%%%%%%%%%%%%%%%%%%%%%%%%%%%%%%%%%%%%%%
\subsection{Copyright}

Copyright \copyright{} 2017--2018 Niklas Beisert

This work may be distributed and/or modified under the
conditions of the \LaTeX{} Project Public License, either version 1.3
of this license or (at your option) any later version.
The latest version of this license is in
  \url{http://www.latex-project.org/lppl.txt}
and version 1.3 or later is part of all distributions of \LaTeX{}
version 2005/12/01 or later.

This work has the LPPL maintenance status `maintained'.

The Current Maintainer of this work is Niklas Beisert.

This work consists of the files |README.txt|, |childdoc.ins| and |childdoc.dtx|
as well as the derived files |childdoc.def|, |cdocsamp.tex|
with |cdocsch1.tex|, |cdocsch2.tex|, |cdocspt3.tex|, |cdocspt4.tex|,
|cdocsdrf.tex|, |cdocsfn1.tex|, |cdocsfn2.tex|
as well as |childdoc.pdf|.

%%%%%%%%%%%%%%%%%%%%%%%%%%%%%%%%%%%%%%%%%%%%%%%%%%%%%%%%%%%%%%%%%%%%%%%%%%%%%%%%
\subsection{Files and Installation}

The package consists of the files:
%
\begin{center}
\begin{tabular}{ll}
    |README.txt|   & readme file \\
    |childdoc.ins| & installation file \\
    |childdoc.dtx| & source file \\
    |childdoc.def| & definition file \\
    |cdocsamp.tex| & sample main file \\
    |cdocsch1.tex| & sample include file \\
    |cdocsch2.tex| & sample include file \\
    |cdocspt3.tex| & sample part file \\
    |cdocspt4.tex| & sample part file \\
    |cdocsdrf.tex| & sample redirection file \\
    |cdocsfn1.tex| & sample redirection file \\
    |cdocsfn2.tex| & sample redirection file \\
    |childdoc.pdf| & manual
\end{tabular}
\end{center}
%
The distribution consists of the files
|README.txt|, |childdoc.ins| and |childdoc.dtx|.
%
\begin{itemize}
\item
Run (pdf)\LaTeX{} on |childdoc.dtx|
to compile the manual |childdoc.pdf| (this file).
\item
Run \LaTeX{} on |childdoc.ins| to create the definitions file |childdoc.def|
and the sample |cdocsamp.tex| with include files
|cdocsch1.tex|, |cdocsch2.tex|, |cdocspt3.tex|, |cdocspt4.tex|,
|cdocsdrf.tex|, |cdocsfn1.tex|, |cdocsfn2.tex|.
Then copy the file |childdoc.def| to an appropriate directory of your \LaTeX{}
distribution, e.g.\ \textit{texmf-root}|/tex/latex/childdoc|.
\end{itemize}

%%%%%%%%%%%%%%%%%%%%%%%%%%%%%%%%%%%%%%%%%%%%%%%%%%%%%%%%%%%%%%%%%%%%%%%%%%%%%%%%
\subsection{Related CTAN Packages}

There are several other packages which offer a similar functionality:
%
\begin{itemize}
\item
The packages
\href{http://ctan.org/pkg/docmute}{\textsf{docmute}},
\href{http://ctan.org/pkg/includex}{\textsf{includex}} and
\href{http://ctan.org/pkg/standalone}{\textsf{standalone}}
provide commands to include only the document body of
a child file thus allowing both files to be compiled individually.
\item
The packages \href{http://ctan.org/pkg/subdocs}{\textsf{subdocs}}
and \href{http://ctan.org/pkg/subfiles}{\textsf{subfiles}}
provide structures in which the main and child documents can be
encapsulated and allowing them to be compiled individually.
The inclusion mechanism is different from the conventional |\include|.
\item
The package \href{http://ctan.org/pkg/combine}{\textsf{combine}}
is an elaborate solution to combine several documents into one.
\end{itemize}
%
See also the CTAN topic \href{http://ctan.org/topic/subdocs}{\textsf{subdocs}}
for further related packages.
The present package differs from the above solutions in that
a document structure constructed with the conventional |\include| mechanism
just needs two extra commands at the top of every file
such that all constituent files can be compiled individually.

%%%%%%%%%%%%%%%%%%%%%%%%%%%%%%%%%%%%%%%%%%%%%%%%%%%%%%%%%%%%%%%%%%%%%%%%%%%%%%%%
%\subsection{Feature Suggestions}
%
%The following is a list of features which may be useful for future
%versions of this package:
%%
%\begin{itemize}
%\item
%\ldots
%\end{itemize}

%%%%%%%%%%%%%%%%%%%%%%%%%%%%%%%%%%%%%%%%%%%%%%%%%%%%%%%%%%%%%%%%%%%%%%%%%%%%%%%%
\subsection{Revision History}

%%%%%%%%%%%%%%%%%%%%%%%%%%%%%%%%%%%%%%%%
\paragraph{v2.0:} 2018/12/30

\begin{itemize}
\item
immediate forward processing
\item
added |\childdocby| mechanism
\item
manual restructured
\end{itemize}

%%%%%%%%%%%%%%%%%%%%%%%%%%%%%%%%%%%%%%%%
\paragraph{v1.6:} 2018/01/17

\begin{itemize}
\item
application for development of include files
\item
corrections to manual
\end{itemize}

%%%%%%%%%%%%%%%%%%%%%%%%%%%%%%%%%%%%%%%%
\paragraph{v1.5:} 2017/05/21

\begin{itemize}
\item
more complete structuring introduced
\item
|\childdocof| introduced
\item
|\childdoc| renamed to |\childdocmain|
\item
|\childredirect| renamed to |\childdocforward| and |\childdocforwardprefix|
and functionality expanded
\end{itemize}

%%%%%%%%%%%%%%%%%%%%%%%%%%%%%%%%%%%%%%%%
\paragraph{v1.0:} 2017/04/27

\begin{itemize}
\item
manual and install package
\item
first version published on CTAN
\end{itemize}

%%%%%%%%%%%%%%%%%%%%%%%%%%%%%%%%%%%%%%%%
\paragraph{v0.6:} 2017/04/26

\begin{itemize}
\item
redirection mechanism added
\end{itemize}

%%%%%%%%%%%%%%%%%%%%%%%%%%%%%%%%%%%%%%%%
\paragraph{v0.5:} 2017/04/26

\begin{itemize}
\item
functionality in definition file
\end{itemize}


%%%%%%%%%%%%%%%%%%%%%%%%%%%%%%%%%%%%%%%%%%%%%%%%%%%%%%%%%%%%%%%%%%%%%%%%%%%%%%%%
%%%%%%%%%%%%%%%%%%%%%%%%%%%%%%%%%%%%%%%%%%%%%%%%%%%%%%%%%%%%%%%%%%%%%%%%%%%%%%%%
%%%%%%%%%%%%%%%%%%%%%%%%%%%%%%%%%%%%%%%%%%%%%%%%%%%%%%%%%%%%%%%%%%%%%%%%%%%%%%%%
\appendix

\settowidth\MacroIndent{\rmfamily\scriptsize 000\ }

 \DocInput{childdoc.dtx}

\end{document}
%</driver>
% \fi
%
% %%%%%%%%%%%%%%%%%%%%%%%%%%%%%%%%%%%%%%%%%%%%%%%%%%%%%%%%%%%%%%%%%%%%%%%%%%%%%%
% %%%%%%%%%%%%%%%%%%%%%%%%%%%%%%%%%%%%%%%%%%%%%%%%%%%%%%%%%%%%%%%%%%%%%%%%%%%%%%
% \section{Sample}
%\iffalse
%<*samplemain>
%\fi
%
% The following presents a sample document
% with two chapters, two parts, a title page,
% a compile flag as well as three forwarding files to set the flag.
% It consists of eight |.tex| files:
% \begin{center}
% \begin{tabular}{ll}
% |cdocsamp.tex|&main file\\
% |cdocsch1.tex|&include file for chapter 1\\
% |cdocsch2.tex|&include file for chapter 2\\
% |cdocspt3.tex|&include file for part 3\\
% |cdocspt4.tex|&include file for part 4\\
% |cdocsdrf.tex|&forwarding file for main file in draft mode\\
% |cdocsfi1.tex|&forwarding file for final version of chapter 1\\
% |cdocsfi2.tex|&forwarding file for final version of chapter 2\\
% \end{tabular}
% \end{center}
% Each of the eight files can be compiled directly by the \LaTeX{} compiler.
%
% %%%%%%%%%%%%%%%%%%%%%%%%%%%%%%%%%%%%%%
% \paragraph{Main File.}
%
% The main file is called |cdocsamp.tex|.
%
% Load the \textsf{childdoc} definitions and
% declare the filename for the main document:
%    \begin{macrocode}
\input{childdoc.def}
\childdocmain{}
%    \end{macrocode}

% Optional override for |\version| flag:
%    \begin{macrocode}
%%\ifchilddoc\else\providecommand{\version}{draft}\fi
%    \end{macrocode}

% Define the default values for the |\version| flag
% (|final| for the main file and |draft| for childs):
%    \begin{macrocode}
\ifchilddoc
\providecommand{\version}{draft}
\else
\providecommand{\version}{final}
\fi
%    \end{macrocode}

% Load the standard document class:
%    \begin{macrocode}
\documentclass[12pt]{article}
%    \end{macrocode}

% Start the document body:
%    \begin{macrocode}
\begin{document}
%    \end{macrocode}

% Declare a title page.
% Print title, part of document being processed and version flag:
%    \begin{macrocode}
\addtocounter{page}{-1}
\begin{center}
{\LARGE\bfseries{}childdoc example\par}
\vspace{1cm}
\ifchilddoc
\ifchilddocmanual part\else chapter\fi:
`\childdocname' of `\childdocjob'\par
\else
main document: `\childdocjob'\par
\fi
version: \version\par
\end{center}
\newpage
%    \end{macrocode}

% Manually include selected file,
% otherwise process as usual:
%    \begin{macrocode}
\ifchilddocmanual
\section*{part `\childdocname'}
\input{\childdocname}
\else
%    \end{macrocode}

% Include the two chapters:
%    \begin{macrocode}
\include{cdocsch1}
\include{cdocsch2}
%    \end{macrocode}

% Include the two parts unless only chapters should be displayed:
%    \begin{macrocode}
\ifchilddoc\else
\section{part three}
\input{cdocspt3}
\section{part four}
\input{cdocspt4}
\fi
%    \end{macrocode}

% Process as usual until here:
%    \begin{macrocode}
\fi
%    \end{macrocode}

% End of document body:
%    \begin{macrocode}
\end{document}
%    \end{macrocode}
%\iffalse
%</samplemain>
%\fi
%
% %%%%%%%%%%%%%%%%%%%%%%%%%%%%%%%%%%%%%%
% \paragraph{Chapter Include Files.}
%
% The include files are called |cdocsch1.tex| and |cdocsch2.tex|.
%
%\iffalse
%<*samplechap1|samplechap2>
%\fi

% Optional override for |\version| flag:
%    \begin{macrocode}
%%\providecommand{\version}{final}
%    \end{macrocode}

% Include the main document:
%    \begin{macrocode}
\input{childdoc.def}
\childdocof{cdocsamp}
%    \end{macrocode}

%\iffalse
%</samplechap1|samplechap2>
%\fi
%
%\iffalse
%<*samplechap1>
%\fi
% Some text for chapter 1:
%    \begin{macrocode}
\section{one}
some text in chapter one
%    \end{macrocode}

%\iffalse
%</samplechap1>
%\fi
% Some text for chapter 2:
%\iffalse
%<*samplechap2>
%\fi
%    \begin{macrocode}
\section{two}
more text in chapter two
%    \end{macrocode}

%\iffalse
%</samplechap2>
%\fi
%
% %%%%%%%%%%%%%%%%%%%%%%%%%%%%%%%%%%%%%%
% \paragraph{Part Include Files.}
%
% The include files are called |cdocspt3.tex| and |cdocspt4.tex|.
%
%\iffalse
%<*samplepart3|samplepart4>
%\fi

% Optional override for |\version| flag:
%    \begin{macrocode}
%%\providecommand{\version}{final}
%    \end{macrocode}

% Include the main document:
%    \begin{macrocode}
\input{childdoc.def}
\childdocby{cdocsamp}
%    \end{macrocode}

%\iffalse
%</samplepart3|samplepart4>
%\fi
%
%\iffalse
%<*samplepart3>
%\fi
% Some text for part 3:
%    \begin{macrocode}
some text in part three
%    \end{macrocode}

%\iffalse
%</samplepart3>
%\fi
% Some text for part 4:
%\iffalse
%<*samplepart4>
%\fi
%    \begin{macrocode}
more text in part four
%    \end{macrocode}

%\iffalse
%</samplepart4>
%\fi
%
% %%%%%%%%%%%%%%%%%%%%%%%%%%%%%%%%%%%%%%
% \paragraph{Forwarding for a Complete Draft.}
%
% The following forwarding file |cdocsdrf.tex|
% compiles the main document in draft mode:
%\iffalse
%<*sampledraft>
%\fi
%    \begin{macrocode}
\def\version{draft}
\input{childdoc.def}
\childdocforward{cdocsamp}
%    \end{macrocode}

%\iffalse
%</sampledraft>
%\fi
%
% %%%%%%%%%%%%%%%%%%%%%%%%%%%%%%%%%%%%%%
% \paragraph{Forwarding for Final Version of the Chapters.}
%
% The following forwarding files |cdocsfn1.tex| and |cdocsfn2.tex|
% (with identical content)
% compile the final versions of the child documents
% |cdocsch1.tex| and |cdocsch2.tex|, respectively:
%\iffalse
%<*samplefinal>
%\fi
%    \begin{macrocode}
\def\version{final}
\input{childdoc.def}
\childdocforwardprefix[cdocsamp]{cdocsfn}{cdocsch}
%    \end{macrocode}

%\iffalse
%</samplefinal>
%\fi
%
% %%%%%%%%%%%%%%%%%%%%%%%%%%%%%%%%%%%%%%
% \paragraph{Command Line Processing.}
%
% The following three command lines generate the output files
% |cdocscld|, |cdocscl1| and |cdocscl2|
% which should be identical to
% |cdocsdrf|, |cdocsch1| and |cdocsfn2|, respectively:
% \begin{center}
% \begin{tabular}{l}
% |latex -jobname cdocscld \|\\
% |  "\def\version{draft}\input{childdoc.def}\childdocforward{cdocsamp}"|\\
% |latex -jobname cdocscl1 \|\\
% |  "\input{childdoc.def}\childdocforward[cdocsamp]{cdocsch1}"|\\
% |latex -jobname cdocscl2 \|\\
% |  "\def\version{final}\input{childdoc.def}\childdocforward{cdocsch2}"|
% \end{tabular}
% \end{center}
% Note that the trailing backslash on each first line
% merely continues the input to the second line
% (for convenient cut ant paste).
% Furthermore, the command |latex| can be replaced by any
% of its alternative versions such as |pdflatex|.
%
% %%%%%%%%%%%%%%%%%%%%%%%%%%%%%%%%%%%%%%%%%%%%%%%%%%%%%%%%%%%%%%%%%%%%%%%%%%%%%%
% %%%%%%%%%%%%%%%%%%%%%%%%%%%%%%%%%%%%%%%%%%%%%%%%%%%%%%%%%%%%%%%%%%%%%%%%%%%%%%
% \section{Implementation}
%\iffalse
%<*package>
%\fi
%
% This section describes the definitions file |childdoc.def|.

% The definitions cannot be loaded using |\usepackage| or |\RequirePackage|
% which has a mechanism to prevent loading a style file more than once.
% When loading the definitions by means of |\input|
% multiple instances have to be prevented manually:
%\iffalse
%This code needs to be before the `\ProvidesFile' directive
%which is defined at the beginning of this file.
%Therefore it is also placed there and commented out here.
%</package>
%<*discard>
%\fi
%    \begin{macrocode}
\ifdefined\childdocmain\endinput\fi
%    \end{macrocode}
%\iffalse
%</discard>
%<*package>
%\fi
%
% \macro{\ifchilddoc}
% \macro{\ifchilddocmanual}
% The conditional |\ifchilddoc| tells whether a
% child (true) or main (false) document is being compiled.
% The conditional |\ifchilddocmanual| tells whether
% the |\includeonly| mechanism is used (false) or
% the selection of child files must be performed manually (true).
% The definitions initialise to false:
%    \begin{macrocode}
\newif\ifchilddoc
\newif\ifchilddocmanual
%    \end{macrocode}

% \macro{\childdocname}
% \macro{\childdocjob}
% The macro |\childdocname| stores the name of the main document
% to be compiled. The macro |\childdocjob| stores the name of
% the document on which the \LaTeX{} compiler was originally invoked.
% The content of |\jobname| cannot be compared
% to filenames specified in the source due to different catcodes.
% The following code rescans |\jobname|, stores the result
% in |\childdocname| and saves a copy in |\childdocjob|:
%    \begin{macrocode}
\edef\childdocname{\scantokens\expandafter{\jobname\noexpand}}
\let\childdocjob\childdocname
%    \end{macrocode}

% \macro{\childdocdisable}
% The macro |\childdocdisable| prevents the main file
% from being processed more than once.
% At this stage, the main document command |\childdocmain|
% is assumed to be called once again where it should do nothing.
% Any subsequent call to it should prevent
% a secondary processing of the main document
% It overwrites the forwarding commands
% |\childdocof| and |\childdocforward|
% with empty macros to prevent further inclusions of the main document:
%    \begin{macrocode}
\newcommand{\childdocdisable}
{
  \renewcommand{\childdocmain}[1]{\renewcommand{\childdocmain}[1]{\endinput}}
  \renewcommand{\childdocof}[1]{}
  \renewcommand{\childdocby}[2][]{}
  \renewcommand{\childdocforward}[2][]{}
  \renewcommand{\childdocdisable}{}
}
%    \end{macrocode}

% \macro{\childdocmain}
% The macro |\childdocmain| is to be called at the top of the main file
% with nothing or the main filename (without extension) as argument.
% First, it breaks loops.
% If the argument is not empty and does not match |\childdocname|
% (which is set by the first inclusion of |childdoc.def|),
% |\ifchilddoc| is set to true, |\includeonly| is applied to the child file
% and |\jobname| is set to the main file
% (for proper handling of |.aux| files):
%    \begin{macrocode}
\newcommand{\childdocmain}[1]
{
  \childdocdisable\childdocmain{}
  \if?#1?\else
    \begingroup
      \def\childdoctmp{#1}
      \ifx\childdoctmp\childdocname
        \def\childdoctmp{}
      \else
        \def\childdoctmp
        {
          \childdoctrue
          \includeonly{\childdocname}
          \def\childdocjob{#1}
          \def\jobname{#1}
        }
      \fi
      \expandafter
    \endgroup
    \childdoctmp
  \fi
}
%    \end{macrocode}

% \macro{\childdocof}
% The command |\childdocof| redirects
% compilation to the main file |#1|.
%    \begin{macrocode}
\newcommand{\childdocof}[1]
{
  \childdocdisable
  \childdoctrue
  \includeonly{\childdocname}
  \def\jobname{#1}
  \def\childdocjob{#1}
  \input{#1}
}
%    \end{macrocode}

% \macro{\childdocby}
% The command |\childdocby| ....
%    \begin{macrocode}
\newcommand{\childdocby}[2][]
{
  \childdocdisable
  \childdoctrue
  \childdocmanualtrue
  \if?#1?\else
    \def\jobname{#2}
  \fi
  \def\childdocjob{#2}
  \input{#2}
  \endinput
}
%    \end{macrocode}

% \macro{\childdocforward}
% The command |\childdocforward| redirects
% compilation to the main file or
% (if the optional argument is given) a child file.
% Parameters are set as if the main file
% or a child file starting with |\childdocof| was compiled.
% Then compilation is handed over to the main file:
%    \begin{macrocode}
\newcommand{\childdocforward}[2][]
{
  \begingroup
    \if?#1?
      \def\childdoctmp
      {
        \def\childdocname{#2}
        \def\childdocjob{#2}
        \def\jobname{#2}
        \input{#2}
        \endinput
      }
    \else
      \def\childdoctmp
      {
        \childdocdisable
        \def\childdocname{#2}
        \childdoctrue
        \includeonly{#2}
        \def\childdocjob{#1}
        \def\jobname{#1}
        \input{#1}
        \endinput
      }
    \fi
    \expandafter
  \endgroup
  \childdoctmp
}
%    \end{macrocode}

% \macro{\childdocforwardprefix}
% The command |\childdocforwardprefix| redirects
% compilation to the main or a child file by means of a pattern.
% The prefix |#1| in the current filename is replaced by |#2|
% and the suffix of the current filename is kept
% (it is assumed that the filename does not contain the substring `|~~~|'
% which is used as a delimiter).
% Compilation is handed over to the new file by |\childdocforward|:
%    \begin{macrocode}
\newcommand{\childdocforwardprefix}[3][]
{
  \begingroup
    \def\childdocextract #2##1~~~{\def\childdoctmp{\childdocforward[#1]{#3##1}}}
    \expandafter\childdocextract\childdocname~~~
    \expandafter
  \endgroup
  \childdoctmp
}
%    \end{macrocode}

% \macro{\childdoc}
% The deprecated macro |\childdoc| is a legacy version of |\childdocmain|:
%    \begin{macrocode}
\newcommand{\childdoc}{\childdocmain}
%    \end{macrocode}

% \macro{\childdocredirect}
% The deprecated macro |\childdocredirect| is a legacy version
% of |\childdocforward| and |\childdocforwardprefix|:
%    \begin{macrocode}
\newcommand{\childdocredirect}[2][]
{
  \begingroup
    \if?#1?
      \def\childdoctmp{\childdocforward{#2}}
    \else
      \def\childdoctmp{\childdocforwardprefix{#1}{#2}}
    \fi
    \expandafter
  \endgroup
  \childdoctmp
}
%    \end{macrocode}

%\iffalse
%</package>
%\fi
%
\endinput
|\\
|\childdocmain{|\textit{main}|}|\\
\end{tabular}
\end{center}
%
If |\jobname| does not match the argument \textit{main} of |\childdocmain|,
it is assumed that |\jobname| points to the child file to be compiled.
When using |\childdocmain| with the main file specified as argument,
it suffices to start a child file
with just |\input{|\textit{main}|}|
without loading of the package and using |\childdocof|.
If instead all processing is done
with the appropriate \textsf{childdoc} directives,
the argument of \textit{main} of |\childdocmain| can be empty.

An alternative version of the command line processing described
in \secref{sec:commandline} using the detection mechanism reads:
%
\begin{center}
|... -jobname "|\textit{target}|" "|[\textit{flags}]%
[|\def\jobname{|\textit{dest}|}|]|\input{|\textit{main}|}"|
\end{center}

%%%%%%%%%%%%%%%%%%%%%%%%%%%%%%%%%%%%%%%%%%%%%%%%%%%%%%%%%%%%%%%%%%%%%%%%%%%%%%%%
\subsection{Manual Code}
\label{sec:manual}

In case one cannot be certain whether the definitions file |childdoc.def|
is installed on the target \TeX{} distribution
and one prefers not to ship it,
it is conceivable to paste a few relevant commands into the sources.

To that end, drop all statements |% \iffalse
%
% childdoc.dtx Copyright (C) 2017-2018 Niklas Beisert
%
% This work may be distributed and/or modified under the
% conditions of the LaTeX Project Public License, either version 1.3
% of this license or (at your option) any later version.
% The latest version of this license is in
%   http://www.latex-project.org/lppl.txt
% and version 1.3 or later is part of all distributions of LaTeX
% version 2005/12/01 or later.
%
% This work has the LPPL maintenance status `maintained'.
%
% The Current Maintainer of this work is Niklas Beisert.
%
% This work consists of the files childdoc.dtx and childdoc.ins
% and the derived files childdoc.def and cdocsamp.tex with
% cdocsch1.tex, cdocsch2.tex, cdocsdrf.tex, cdocsfn1.tex, cdocsfn2.tex.
%
%<package>\ifdefined\childdocmain\endinput\fi
%<package>\ProvidesFile{childdoc.def}[2018/12/30 v2.0 child document driver]
%<samplemain>\ProvidesFile{cdocsamp.tex}[2018/12/30 v2.0 sample for childdoc]
%<*driver>
%\ProvidesFile{childdoc.drv}[2018/12/30 v2.0 childdoc reference manual file]
\PassOptionsToClass{10pt,a4paper}{article}
\documentclass{ltxdoc}

\usepackage[margin=35mm]{geometry}
\usepackage{hyperref}
\usepackage{hyperxmp}
\usepackage[usenames]{color}

\hypersetup{colorlinks=true}
\hypersetup{pdfstartview=FitH}
\hypersetup{pdfpagemode=UseNone}
\hypersetup{pdfsource={}}
\hypersetup{pdflang={en-UK}}
\hypersetup{pdfcopyright={Copyright 2017-2018 Niklas Beisert.
  This work may be distributed and/or modified under the
  conditions of the LaTeX Project Public License, either version 1.3
  of this license or (at your option) any later version.}}
\hypersetup{pdflicenseurl={http://www.latex-project.org/lppl.txt}}
\hypersetup{pdfcontactaddress={ETH Zurich, ITP, HIT K,
  Wolfgang-Pauli-Strasse 27}}
\hypersetup{pdfcontactpostcode={8093}}
\hypersetup{pdfcontactcity={Zurich}}
\hypersetup{pdfcontactcountry={Switzerland}}
\hypersetup{pdfcontactemail={nbeisert@itp.phys.ethz.ch}}
\hypersetup{pdfcontacturl={http://people.phys.ethz.ch/\xmptilde nbeisert/}}

\newcommand{\secref}[1]{\hyperref[#1]{section \ref*{#1}}}

\parskip1ex
\parindent0pt
\let\olditemize\itemize
\def\itemize{\olditemize\parskip0pt}

\begin{document}

\title{The \textsf{childdoc} Package}
\hypersetup{pdftitle={The childdoc Package}}
\author{Niklas Beisert\\[2ex]
  Institut f\"ur Theoretische Physik\\
  Eidgen\"ossische Technische Hochschule Z\"urich\\
  Wolfgang-Pauli-Strasse 27, 8093 Z\"urich, Switzerland\\[1ex]
  \href{mailto:nbeisert@itp.phys.ethz.ch}
  {\texttt{nbeisert@itp.phys.ethz.ch}}}
\hypersetup{pdfauthor={Niklas Beisert}}
\hypersetup{pdfsubject={Manual for the LaTeX2e Package childdoc}}
\date{30 December 2018, \textsf{v2.0}}
\maketitle

\begin{abstract}\noindent
\textsf{childdoc} is a \LaTeXe{} package
that enables the direct compilation
of document sections included by |\include|
to individual files.
\end{abstract}

\begingroup
\parskip0ex
\tableofcontents
\endgroup

%%%%%%%%%%%%%%%%%%%%%%%%%%%%%%%%%%%%%%%%%%%%%%%%%%%%%%%%%%%%%%%%%%%%%%%%%%%%%%%%
%%%%%%%%%%%%%%%%%%%%%%%%%%%%%%%%%%%%%%%%%%%%%%%%%%%%%%%%%%%%%%%%%%%%%%%%%%%%%%%%
\section{Introduction}

\LaTeX{} provides a mechanism to structure a large document (such as a book)
into a main file and several child files (containing the chapters)
using the |\include| command.
This mechanism is beneficial for documents
which span hundreds of pages in order to
make the source file(s) more manageable.
Moreover, compilation can be restricted to
selected child files by means of the |\includeonly| command.
The latter feature can be used to reduce the compilation time while editing
(this was significantly more useful in the earlier days of \LaTeX{})
or to generate a smaller document which is easier to navigate.
Another application of |\includeonly| is to generate
documents consisting of selected parts of the complete document.

However, there are a few drawbacks of the plain |\include| mechanism:
\begin{itemize}
\item
The child files cannot be compiled on their own,
they can only be compiled via the main file.
A naive editing environment
(such as a text editor with an option
to have the current file processed by \LaTeX)
may require one to switch to the main file before compiling;
attempting to compile the child file produces errors.
\item
The main file must be modified (each time)
to adjust the |\includeonly| command
to the present needs. This easily leaves the main file in a messy state.
\item
The generated document will always carry the filename
of the main document. This is inconvenient if
several child files are to be compiled and
to be kept for distribution.
\end{itemize}

The present package provides a simple interface
to make child files individually compilable by \LaTeX{}.
Compiling a child file then has the same effect as compiling
the main file with an |\includeonly| command
to select the appropriate child.
Moreover the generated document will carry the name of the child
rather than the main file.
This resolves all three above issues.

This feature is meant to make the editing of books,
thesis documents and lecture notes somewhat more convenient.
However, the package can also be used efficiently for
composing a series of documents (such as exercise sheets)
which are typically distributed individually.
It then assists the author in generating the individual documents
(potentially in different versions)
as well as a document containing the collected series.
Another application is in developing style files
or other kinds of included material
where compilation of the style file could redirect
to a sample or test file.

%%%%%%%%%%%%%%%%%%%%%%%%%%%%%%%%%%%%%%%%%%%%%%%%%%%%%%%%%%%%%%%%%%%%%%%%%%%%%%%%
%%%%%%%%%%%%%%%%%%%%%%%%%%%%%%%%%%%%%%%%%%%%%%%%%%%%%%%%%%%%%%%%%%%%%%%%%%%%%%%%
\section{Usage}

First of all, the package \textsf{childdoc} is \emph{not} a standard
\LaTeXe{} |.sty| style file! Therefore it needs to be invoked in
a non-standard way.

%%%%%%%%%%%%%%%%%%%%%%%%%%%%%%%%%%%%%%%%%%%%%%%%%%%%%%%%%%%%%%%%%%%%%%%%%%%%%%%%
\subsection{Included Files}
\label{sec:include}

%%%%%%%%%%%%%%%%%%%%%%%%%%%%%%%%%%%%%%%%
\DescribeMacro{\childdocmain}
To use the package, add the commands
\begin{center}
\begin{tabular}{l}
|\input{childdoc.def}|\\
|\childdocmain{}|\\
\end{tabular}
\end{center}
at the very top of the main \LaTeX{} file,
in particular \emph{before} the |\documentclass| statement!
The argument of |\childdocmain| should be left empty
(but it must be present).

%%%%%%%%%%%%%%%%%%%%%%%%%%%%%%%%%%%%%%%%
\DescribeMacro{\childdocof}
Furthermore, add the commands
\begin{center}
\begin{tabular}{l}
|\input{childdoc.def}|\\
|\childdocof{|\textit{main}|}|\\
\end{tabular}
\end{center}
at the top of every child file \textit{child}
which is included by |\include{|\textit{child}|}|
from within the main file
(or at least for those files to be compiled individually).
The argument \textit{main} must be the filename of the main file.

There are a couple of
considerations in setting up the main and child documents:

%%%%%%%%%%%%%%%%%%%%%%%%%%%%%%%%%%%%%%%%
\paragraph{Restrictions.}

Please note the following restrictions:
\begin{itemize}
\item
|\childdocmain| must be called with one argument \textit{main}
to ensure compatibility with earlier version of the package.
It must either be empty (|\childdocmain{}|)
or precisely match the filename of the main file in which it is specified.
See \secref{sec:detection} for further information.
\item
The filename \textit{main} must be specified without the |.tex| extension.
\item
The filename \textit{main} is case sensitive
(even in case-insensitive file systems)
due to internal string comparison.
\item
The argument \textit{main} should be fully expanded, it cannot be a macro.
\item
Subdirectories and special characters should be avoided in filenames.
\item
The command |\childdocmain{|\textit{main}|}| must be followed by a whitespace.
It should not be followed immediately by another command
or by a comment mark `|%|'.
This is because the \TeX{} parser reads the token immediately following
the argument of |\childdocmain| and puts it
at the beginning of every child section;
however, a white\-space is ignored.
\end{itemize}

%%%%%%%%%%%%%%%%%%%%%%%%%%%%%%%%%%%%%%%%
\paragraph{Content of Main File.}

It is advisable to place all content in the child files included by |\include|.
Any output contained in the main file will appear in all child documents
unless suppressed manually;
it cannot be suppressed automatically by the |\includeonly| directive
and thus should normally be avoided.
A method to include some content in the main file
by means of conditional processing is described in \secref{sec:conditional}.

%%%%%%%%%%%%%%%%%%%%%%%%%%%%%%%%%%%%%%%%
\paragraph{Page Numbering.}

When only a part of the document is compiled,
the appropriate numbering of pages
(as well as other status parameters)
is determined from the |.aux| files.
The latter contain information from previous passes.
However this information needs to propagate through
all intermediate child documents.
Therefore the page numbering in child documents may well
be inconsistent until the complete document is compiled at least once.

A useful (if unconventional) way to always ensure a consistent
page numbering is to restart the numbering in each child document
and denote the pages by `\textit{child}|.|\textit{page}'
where \textit{child} represents the chapter/section number of the child file.
This can be achieved by the command
|\numberwithin{page}{|\textit{child}|}|
of the \textsf{amsmath} package
where \textit{child} can be |chapter| or |section|
depending on the chosen structuring.
Alternatively, one can modify the macro |\thepage| appropriately
and reset the counter |page| at the start of each child file.

%%%%%%%%%%%%%%%%%%%%%%%%%%%%%%%%%%%%%%%%%%%%%%%%%%%%%%%%%%%%%%%%%%%%%%%%%%%%%%%%
\subsection{Conditional Processing}
\label{sec:conditional}

The package provides a mechanism to compile different versions
of a document. To customise the versions further some conditional processing
can come in handy to distinguish which version is being compiled.
The package provides two macros to describe the compilation context:

%%%%%%%%%%%%%%%%%%%%%%%%%%%%%%%%%%%%%%%%
\DescribeMacro{\ifchilddoc}
The conditional |\ifchilddoc| distinguishes between the compilation of
child documents and the main document:
%
\begin{center}
|\ifchilddoc |\textit{child-code}| |[|\||else |\textit{main-code}]| \||fi|
\end{center}

%%%%%%%%%%%%%%%%%%%%%%%%%%%%%%%%%%%%%%%%
\DescribeMacro{\childdocname}
\DescribeMacro{\childdocjob}
The macro |\childdocname| contains the filename (without extension)
of the main or child file being processed.
Note that |\childdocjob| will always contain the name of the main file.

%%%%%%%%%%%%%%%%%%%%%%%%%%%%%%%%%%%%%%%%
\paragraph{Title Page.}

Conditional processing can be used to include a title or banner page
in the main document when proper precautions are taken.
Importantly, the code in the main file should ensure that the page counter
(as well as other status parameters which are stored in the |.aux| files)
takes the same value after the conditional processing.
Otherwise the page numbers may take divergent values
depending on which part is compiled.

For example, a title page could be declared by:
%
\begin{center}
\begin{tabular}{l}
|\ifchilddoc\||else|\\
|\addtocounter{page}{-1}|\\
\textit{code for title page}\\
|\newpage|\\
|\||fi|
\end{tabular}
\end{center}
%
A banner page for the child documents can be generated by:
%
\begin{center}
\begin{tabular}{l}
|\ifchilddoc|\\
|\addtocounter{page}{-1}|\\
\textit{code for banner page}\\
|\newpage|\\
|\||fi|
\end{tabular}
\end{center}
%
Here one could write a message such as:
\begin{center}
|This is the part \childdocname{} of \childdocjob{}.|
\end{center}

%%%%%%%%%%%%%%%%%%%%%%%%%%%%%%%%%%%%%%%%%%%%%%%%%%%%%%%%%%%%%%%%%%%%%%%%%%%%%%%%
\subsection{Flags}
\label{sec:flags}

The package makes it easy to generate different versions
of the main or child documents.
To this end compilation flags can be defined
and assigned different default values.
They will be particularly useful in conjunction
with the forwarding mechanism described in \secref{sec:forward}.

For example, it may be useful to have a flag |\version|
which can be set to |draft| or |final|.
The document source will contain some conditional code
depending on the value of |\version|.
Suppose further, the flag should default to |final| for the main file
and to |draft| for child files
which is a natural assignment for editing the document.
This is achieved by placing the following code
in the preamble of the main document
(below the |\childdocmain| directive):
%
\begin{center}
\begin{tabular}{l}
|\ifchilddoc|\\
|\providecommand{\version}{draft}|\\
|\||else|\\
|\providecommand{\version}{final}|\\
|\||fi|
\end{tabular}
\end{center}
%
The definition by |\providecommand| makes sure
that previous definitions are not overwritten.
Further statements |\providecommand{\version}{...}|
can thus be added before the above code to override it.

For the main file, one might add a line
(between |\childdocmain| and the above block)
%
\begin{center}
|%\ifchilddoc\||else\providecommand{\version}{draft}\||fi|
\end{center}
%
which can be uncommented to produce a draft version.
Likewise one can add a line to the very top of a child file
(above the |\childdocof{|\textit{main}|}| directive)
%
\begin{center}
|%\providecommand{\version}{final}|
\end{center}
%
which can be uncommented to produce the final version of this child document.

%%%%%%%%%%%%%%%%%%%%%%%%%%%%%%%%%%%%%%%%%%%%%%%%%%%%%%%%%%%%%%%%%%%%%%%%%%%%%%%%
\subsection{Forwarding}
\label{sec:forward}

Different versions of the main or child documents
using compilation flags as described in \secref{sec:flags}
can be (permanently) stored in different files
for convenient compilation, viewing and distribution.
To this end, the package defines a command
to pass on compilation to a different file:

%%%%%%%%%%%%%%%%%%%%%%%%%%%%%%%%%%%%%%%%
\DescribeMacro{\childdocforward}
The command |\childdocforward| redirects processing to
another source file:
%
\begin{center}
\begin{tabular}{l}
|\input{childdoc.def}|\\
|\childdocforward[|\textit{main}|]{|\textit{dest}|}|\\
\end{tabular}
\end{center}
%
The argument \textit{dest} is the destination file
(without extension).
It should be the main file or one of the child files.
Note that further \textsf{childdoc} directives
such as |\childdocof| and |\childdocforward|
in the indicated file will be processed in this form.
The optional argument \textit{main}
passes on directly to the main file \textit{main}
while pretending to compile the child \textit{dest}.
This form behaves as if \textit{dest}
issues |\childdocof{|\textit{main}|}| right away,
and no further \textsf{childdoc} directives will be processed.

%%%%%%%%%%%%%%%%%%%%%%%%%%%%%%%%%%%%%%%%
\DescribeMacro{\...prefix}
In the alternative form |\childdocforwardprefix|,
%
\begin{center}
\begin{tabular}{l}
|\input{childdoc.def}|\\
|\childdocforwardprefix[|\textit{main}|]{|\textit{prefix}|}{|\textit{dest}|}|
\end{tabular}
\end{center}
%
the destination file is determined by a pattern
depending on the current file:
To make this work, the current file must be called
`{\textit{prefix}\hspace{0.2em}\textit{suffix}}'
with \textit{prefix} matching precisely the argument.
Processing is then passed on to the file
`{\textit{dest}\hspace{0.2em}\textit{suffix}}'.
Surely, the same effect is achieved by
directly specifying the
argument `{\textit{dest}\hspace{0.2em}\textit{suffix}}'
in the first form.
However, that requires to set up a different file
for each child. With the alternative form of the command
all these files can have exactly the same content
which simplifies setting them up and maintaining them.

For example, the following file |draft.tex|
with a compilation flag |\version| as described in \secref{sec:flags}
compiles the main document as a draft:
%
\begin{center}
\begin{tabular}{l}
|\def\version{draft}|\\
|\input{childdoc.def}|\\
|\childdocforward{|\textit{main}|}|
\end{tabular}
\end{center}
%
Likewise, the following files |final|\textit{nn}|.tex|
compile the final version of the child document
|child|\textit{nn}|.tex|:
%
\begin{center}
\begin{tabular}{l}
|\def\version{final}|\\
|\input{childdoc.def}|\\
|\childdocforwardprefix{final}{child}|
\end{tabular}
\end{center}
%

Note that when several versions of a main file and/or of each child file
are to be generated, it may be convenient to set up a |Makefile| or
shell script to automatise the process.

%%%%%%%%%%%%%%%%%%%%%%%%%%%%%%%%%%%%%%%%%%%%%%%%%%%%%%%%%%%%%%%%%%%%%%%%%%%%%%%%
\subsection{Command Line Processing}
\label{sec:commandline}

The effect of redirection files can also be achieved by invoking
the \LaTeX{} compiler with a more elaborate command line.
Most conveniently this should be done as part
of a shell script or a |Makefile|.

When using \textsf{childdoc} in the main file, the following
command lines effectively perform a redirection
(note that depending on the shell being used,
backslashes may have to be doubled: `|\|' $\to$ `|\\|'):
%
\begin{center}
|... -jobname "|\textit{target}|" |\\|"|[\textit{flags}]%
|\input{childdoc.def}\childdocforward[|\textit{main}|]{|\textit{dest}|}"|
\end{center}
%
Here \textit{target} is the name of the output file,
\textit{main} is the name of the main file
and \textit{dest} is the name of the main or child file to be processed
(all filenames without extensions).
The optional argument \textit{main} can be omitted
if \textit{main} matches \textit{dest}.
Optionally, compilation \textit{flags} can be defined via |\def| commands.
This command line makes the \TeX{} engine believe
it is compiling the file \textit{target}
whose content is specified as the latter parameter.
The provided code then forwards the processing to
\textit{main} or \textit{dest} as described in \secref{sec:forward}.

%%%%%%%%%%%%%%%%%%%%%%%%%%%%%%%%%%%%%%%%%%%%%%%%%%%%%%%%%%%%%%%%%%%%%%%%%%%%%%%%
\subsection{Include by Input}
\label{sec:input}

Including child documents by |\include| has some restrictions by design.
Most notably, the content of a child document always occupies
its own set of pages; pages cannot be shared between child documents.
Usually, this behaviour makes perfect sense
because each child document contain an essential part of the document.
However, in some situations it may be desirable to compose
a document from a collection of parts
without having mandatory page breaks between then.
For this case, the package
provides a mechanism to include parts
by |\input| which can also be processed individually.
However, by construction this mechanism
requires manual handling of the content to be output.

%%%%%%%%%%%%%%%%%%%%%%%%%%%%%%%%%%%%%%%%
\DescribeMacro{\ifchilddocmanual}
The main file should be prepared as usual, see \secref{sec:include}.
However, the document body must make a distinction
between processing of an individual part and of the main document, e.g.:
%
\begin{center}
\begin{tabular}{l}
|\ifchilddocmanual|\\
|\input{\childdocname}|\\
|\||else|\\
\textit{document body with }|\input{|\textit{part}|}|\\
|\||fi|
\end{tabular}
\end{center}
%
The conditional |\ifchilddocmanual| is true whenever
a part to be included by |\input| is being compiled,
and the name of the part is stored in |\childdocname|.

%%%%%%%%%%%%%%%%%%%%%%%%%%%%%%%%%%%%%%%%
\DescribeMacro{\childdocby}
Each part to be included by |\input| should start with:
%
\begin{center}
\begin{tabular}{l}
|\input{childdoc.def}|\\
|\childdocby{|\textit{main}|}|\\
\end{tabular}
\end{center}
%
The directive |\childdocby| is similar to |\childdocof|
described in \secref{sec:include},
but the subsequent selection of content must be done manually.
To that end, both |\ifchilddoc| and |\ifchilddocmanual|
will be true upon processing of a part,
and the name of the part is stored in |\childdocname|.
Note that |\jobname| will be set to the filename of the current part
so that each part receives an individual |.aux| file
that does not interfere with the |.aux| file(s) of the main document.
This behaviour can be altered by the alternative form
|\childdocby[*]{|\textit{main}|}| (with a non-empty optional argument)
which uses the |.aux| file of the main document
by setting |\jobname| to \textit{main}.

%%%%%%%%%%%%%%%%%%%%%%%%%%%%%%%%%%%%%%%%%%%%%%%%%%%%%%%%%%%%%%%%%%%%%%%%%%%%%%%%
\subsection{Driver Development}
\label{sec:driver}

The \textsf{childdoc} mechanism can also be use for the development
of definition files such as \LaTeX{} styles or classes.
This case differs from the above setup with multiple parts
included by |\include| in that no |\includeonly| should be invoked.
This can be achieved by starting the include file
(before |\ProvidesPackage|) with:
%
\begin{center}
\begin{tabular}{l}
|\input{childdoc.def}|\\
|\childdocforward{|\textit{main}|}|\\
\end{tabular}
\end{center}
%
or alternatively with:
%
\begin{center}
\begin{tabular}{l}
|\input{childdoc.def}|\\
|\childdocby{|\textit{main}|}|\\
\end{tabular}
\end{center}
%
Both forms have slightly different effects as described above.
The main file is prepared as usual, see \secref{sec:include}.

%%%%%%%%%%%%%%%%%%%%%%%%%%%%%%%%%%%%%%%%%%%%%%%%%%%%%%%%%%%%%%%%%%%%%%%%%%%%%%%%
\subsection{Legacy Detection}
\label{sec:detection}

The directive |\childdocmain| in the main file can detect
whether the complete document or merely a child is to be compiled
even without using the directive |\childdocof|.
This method is deprecated because it is less robust
and there is no compelling reason to use it;
it is merely provided for backward compatibility
and it may be removed in future versions.

If the detection mechanism is to be used,
it is mandatory to correctly specify
the filename of the main file as the argument of |\childdocmain|:
%
\begin{center}
\begin{tabular}{l}
|\input{childdoc.def}|\\
|\childdocmain{|\textit{main}|}|\\
\end{tabular}
\end{center}
%
If |\jobname| does not match the argument \textit{main} of |\childdocmain|,
it is assumed that |\jobname| points to the child file to be compiled.
When using |\childdocmain| with the main file specified as argument,
it suffices to start a child file
with just |\input{|\textit{main}|}|
without loading of the package and using |\childdocof|.
If instead all processing is done
with the appropriate \textsf{childdoc} directives,
the argument of \textit{main} of |\childdocmain| can be empty.

An alternative version of the command line processing described
in \secref{sec:commandline} using the detection mechanism reads:
%
\begin{center}
|... -jobname "|\textit{target}|" "|[\textit{flags}]%
[|\def\jobname{|\textit{dest}|}|]|\input{|\textit{main}|}"|
\end{center}

%%%%%%%%%%%%%%%%%%%%%%%%%%%%%%%%%%%%%%%%%%%%%%%%%%%%%%%%%%%%%%%%%%%%%%%%%%%%%%%%
\subsection{Manual Code}
\label{sec:manual}

In case one cannot be certain whether the definitions file |childdoc.def|
is installed on the target \TeX{} distribution
and one prefers not to ship it,
it is conceivable to paste a few relevant commands into the sources.

To that end, drop all statements |\input{childdoc.def}|
and perform the replacements as outlined below.
Instead of |\childdocmain{|\textit{main}|}| add the following code
to the top of the main file:
%
\begin{center}
\begin{tabular}{l}
|\||ifdefined\childdocname\endinput\||fi\newif\ifchilddoc|\\
|\edef\childdocname{\scantokens\expandafter{\jobname\noexpand}}|\\
|\def\childdocmain{|\textit{main}|}\||ifx\childdocmain\childdocname\||else|\\
|\childdoctrue\includeonly{\childdocname}\let\jobname\childdocmain\||fi|\\
\end{tabular}
\end{center}
%
Instead of |\childdocof{|\textit{main}|}| just include the main file
at the top of each child file:
%
\begin{center}
|\input{|\textit{main}|}|
\end{center}
%
A simple redirection |\childdocforward{|\textit{dest}|}| is achieved by:
%
\begin{center}
|\def\jobname{|\textit{dest}|}\input{\jobname}|
\end{center}
%
The redirection with prefix
|\childdocforwardprefix[|\textit{prefix}|]{|\textit{dest}|}|
is accomplished by:
%
\begin{center}
\begin{tabular}{l}
|{\edef\jobname{\scantokens\expandafter{\jobname\noexpand}}|\\
|\def\redirectjob |\textit{prefix}|#1~~~{\gdef\jobname{|\textit{dest}|#1}}|\\
|\expandafter\redirectjob\jobname~~~}\input{\jobname}|
\end{tabular}
\end{center}

In an alternative approach,
child documents can be compiled by a specific command line
without additional code or specific definitions:
%
\begin{center}
|... -jobname "|\textit{target}|" "|[\textit{flags}]%
|\includeonly{|\textit{dest}|}\input{|\textit{main}|}"|
\end{center}
%

%%%%%%%%%%%%%%%%%%%%%%%%%%%%%%%%%%%%%%%%%%%%%%%%%%%%%%%%%%%%%%%%%%%%%%%%%%%%%%%%
%%%%%%%%%%%%%%%%%%%%%%%%%%%%%%%%%%%%%%%%%%%%%%%%%%%%%%%%%%%%%%%%%%%%%%%%%%%%%%%%
\section{Information}

%%%%%%%%%%%%%%%%%%%%%%%%%%%%%%%%%%%%%%%%%%%%%%%%%%%%%%%%%%%%%%%%%%%%%%%%%%%%%%%%
\subsection{Copyright}

Copyright \copyright{} 2017--2018 Niklas Beisert

This work may be distributed and/or modified under the
conditions of the \LaTeX{} Project Public License, either version 1.3
of this license or (at your option) any later version.
The latest version of this license is in
  \url{http://www.latex-project.org/lppl.txt}
and version 1.3 or later is part of all distributions of \LaTeX{}
version 2005/12/01 or later.

This work has the LPPL maintenance status `maintained'.

The Current Maintainer of this work is Niklas Beisert.

This work consists of the files |README.txt|, |childdoc.ins| and |childdoc.dtx|
as well as the derived files |childdoc.def|, |cdocsamp.tex|
with |cdocsch1.tex|, |cdocsch2.tex|, |cdocspt3.tex|, |cdocspt4.tex|,
|cdocsdrf.tex|, |cdocsfn1.tex|, |cdocsfn2.tex|
as well as |childdoc.pdf|.

%%%%%%%%%%%%%%%%%%%%%%%%%%%%%%%%%%%%%%%%%%%%%%%%%%%%%%%%%%%%%%%%%%%%%%%%%%%%%%%%
\subsection{Files and Installation}

The package consists of the files:
%
\begin{center}
\begin{tabular}{ll}
    |README.txt|   & readme file \\
    |childdoc.ins| & installation file \\
    |childdoc.dtx| & source file \\
    |childdoc.def| & definition file \\
    |cdocsamp.tex| & sample main file \\
    |cdocsch1.tex| & sample include file \\
    |cdocsch2.tex| & sample include file \\
    |cdocspt3.tex| & sample part file \\
    |cdocspt4.tex| & sample part file \\
    |cdocsdrf.tex| & sample redirection file \\
    |cdocsfn1.tex| & sample redirection file \\
    |cdocsfn2.tex| & sample redirection file \\
    |childdoc.pdf| & manual
\end{tabular}
\end{center}
%
The distribution consists of the files
|README.txt|, |childdoc.ins| and |childdoc.dtx|.
%
\begin{itemize}
\item
Run (pdf)\LaTeX{} on |childdoc.dtx|
to compile the manual |childdoc.pdf| (this file).
\item
Run \LaTeX{} on |childdoc.ins| to create the definitions file |childdoc.def|
and the sample |cdocsamp.tex| with include files
|cdocsch1.tex|, |cdocsch2.tex|, |cdocspt3.tex|, |cdocspt4.tex|,
|cdocsdrf.tex|, |cdocsfn1.tex|, |cdocsfn2.tex|.
Then copy the file |childdoc.def| to an appropriate directory of your \LaTeX{}
distribution, e.g.\ \textit{texmf-root}|/tex/latex/childdoc|.
\end{itemize}

%%%%%%%%%%%%%%%%%%%%%%%%%%%%%%%%%%%%%%%%%%%%%%%%%%%%%%%%%%%%%%%%%%%%%%%%%%%%%%%%
\subsection{Related CTAN Packages}

There are several other packages which offer a similar functionality:
%
\begin{itemize}
\item
The packages
\href{http://ctan.org/pkg/docmute}{\textsf{docmute}},
\href{http://ctan.org/pkg/includex}{\textsf{includex}} and
\href{http://ctan.org/pkg/standalone}{\textsf{standalone}}
provide commands to include only the document body of
a child file thus allowing both files to be compiled individually.
\item
The packages \href{http://ctan.org/pkg/subdocs}{\textsf{subdocs}}
and \href{http://ctan.org/pkg/subfiles}{\textsf{subfiles}}
provide structures in which the main and child documents can be
encapsulated and allowing them to be compiled individually.
The inclusion mechanism is different from the conventional |\include|.
\item
The package \href{http://ctan.org/pkg/combine}{\textsf{combine}}
is an elaborate solution to combine several documents into one.
\end{itemize}
%
See also the CTAN topic \href{http://ctan.org/topic/subdocs}{\textsf{subdocs}}
for further related packages.
The present package differs from the above solutions in that
a document structure constructed with the conventional |\include| mechanism
just needs two extra commands at the top of every file
such that all constituent files can be compiled individually.

%%%%%%%%%%%%%%%%%%%%%%%%%%%%%%%%%%%%%%%%%%%%%%%%%%%%%%%%%%%%%%%%%%%%%%%%%%%%%%%%
%\subsection{Feature Suggestions}
%
%The following is a list of features which may be useful for future
%versions of this package:
%%
%\begin{itemize}
%\item
%\ldots
%\end{itemize}

%%%%%%%%%%%%%%%%%%%%%%%%%%%%%%%%%%%%%%%%%%%%%%%%%%%%%%%%%%%%%%%%%%%%%%%%%%%%%%%%
\subsection{Revision History}

%%%%%%%%%%%%%%%%%%%%%%%%%%%%%%%%%%%%%%%%
\paragraph{v2.0:} 2018/12/30

\begin{itemize}
\item
immediate forward processing
\item
added |\childdocby| mechanism
\item
manual restructured
\end{itemize}

%%%%%%%%%%%%%%%%%%%%%%%%%%%%%%%%%%%%%%%%
\paragraph{v1.6:} 2018/01/17

\begin{itemize}
\item
application for development of include files
\item
corrections to manual
\end{itemize}

%%%%%%%%%%%%%%%%%%%%%%%%%%%%%%%%%%%%%%%%
\paragraph{v1.5:} 2017/05/21

\begin{itemize}
\item
more complete structuring introduced
\item
|\childdocof| introduced
\item
|\childdoc| renamed to |\childdocmain|
\item
|\childredirect| renamed to |\childdocforward| and |\childdocforwardprefix|
and functionality expanded
\end{itemize}

%%%%%%%%%%%%%%%%%%%%%%%%%%%%%%%%%%%%%%%%
\paragraph{v1.0:} 2017/04/27

\begin{itemize}
\item
manual and install package
\item
first version published on CTAN
\end{itemize}

%%%%%%%%%%%%%%%%%%%%%%%%%%%%%%%%%%%%%%%%
\paragraph{v0.6:} 2017/04/26

\begin{itemize}
\item
redirection mechanism added
\end{itemize}

%%%%%%%%%%%%%%%%%%%%%%%%%%%%%%%%%%%%%%%%
\paragraph{v0.5:} 2017/04/26

\begin{itemize}
\item
functionality in definition file
\end{itemize}


%%%%%%%%%%%%%%%%%%%%%%%%%%%%%%%%%%%%%%%%%%%%%%%%%%%%%%%%%%%%%%%%%%%%%%%%%%%%%%%%
%%%%%%%%%%%%%%%%%%%%%%%%%%%%%%%%%%%%%%%%%%%%%%%%%%%%%%%%%%%%%%%%%%%%%%%%%%%%%%%%
%%%%%%%%%%%%%%%%%%%%%%%%%%%%%%%%%%%%%%%%%%%%%%%%%%%%%%%%%%%%%%%%%%%%%%%%%%%%%%%%
\appendix

\settowidth\MacroIndent{\rmfamily\scriptsize 000\ }

 \DocInput{childdoc.dtx}

\end{document}
%</driver>
% \fi
%
% %%%%%%%%%%%%%%%%%%%%%%%%%%%%%%%%%%%%%%%%%%%%%%%%%%%%%%%%%%%%%%%%%%%%%%%%%%%%%%
% %%%%%%%%%%%%%%%%%%%%%%%%%%%%%%%%%%%%%%%%%%%%%%%%%%%%%%%%%%%%%%%%%%%%%%%%%%%%%%
% \section{Sample}
%\iffalse
%<*samplemain>
%\fi
%
% The following presents a sample document
% with two chapters, two parts, a title page,
% a compile flag as well as three forwarding files to set the flag.
% It consists of eight |.tex| files:
% \begin{center}
% \begin{tabular}{ll}
% |cdocsamp.tex|&main file\\
% |cdocsch1.tex|&include file for chapter 1\\
% |cdocsch2.tex|&include file for chapter 2\\
% |cdocspt3.tex|&include file for part 3\\
% |cdocspt4.tex|&include file for part 4\\
% |cdocsdrf.tex|&forwarding file for main file in draft mode\\
% |cdocsfi1.tex|&forwarding file for final version of chapter 1\\
% |cdocsfi2.tex|&forwarding file for final version of chapter 2\\
% \end{tabular}
% \end{center}
% Each of the eight files can be compiled directly by the \LaTeX{} compiler.
%
% %%%%%%%%%%%%%%%%%%%%%%%%%%%%%%%%%%%%%%
% \paragraph{Main File.}
%
% The main file is called |cdocsamp.tex|.
%
% Load the \textsf{childdoc} definitions and
% declare the filename for the main document:
%    \begin{macrocode}
\input{childdoc.def}
\childdocmain{}
%    \end{macrocode}

% Optional override for |\version| flag:
%    \begin{macrocode}
%%\ifchilddoc\else\providecommand{\version}{draft}\fi
%    \end{macrocode}

% Define the default values for the |\version| flag
% (|final| for the main file and |draft| for childs):
%    \begin{macrocode}
\ifchilddoc
\providecommand{\version}{draft}
\else
\providecommand{\version}{final}
\fi
%    \end{macrocode}

% Load the standard document class:
%    \begin{macrocode}
\documentclass[12pt]{article}
%    \end{macrocode}

% Start the document body:
%    \begin{macrocode}
\begin{document}
%    \end{macrocode}

% Declare a title page.
% Print title, part of document being processed and version flag:
%    \begin{macrocode}
\addtocounter{page}{-1}
\begin{center}
{\LARGE\bfseries{}childdoc example\par}
\vspace{1cm}
\ifchilddoc
\ifchilddocmanual part\else chapter\fi:
`\childdocname' of `\childdocjob'\par
\else
main document: `\childdocjob'\par
\fi
version: \version\par
\end{center}
\newpage
%    \end{macrocode}

% Manually include selected file,
% otherwise process as usual:
%    \begin{macrocode}
\ifchilddocmanual
\section*{part `\childdocname'}
\input{\childdocname}
\else
%    \end{macrocode}

% Include the two chapters:
%    \begin{macrocode}
\include{cdocsch1}
\include{cdocsch2}
%    \end{macrocode}

% Include the two parts unless only chapters should be displayed:
%    \begin{macrocode}
\ifchilddoc\else
\section{part three}
\input{cdocspt3}
\section{part four}
\input{cdocspt4}
\fi
%    \end{macrocode}

% Process as usual until here:
%    \begin{macrocode}
\fi
%    \end{macrocode}

% End of document body:
%    \begin{macrocode}
\end{document}
%    \end{macrocode}
%\iffalse
%</samplemain>
%\fi
%
% %%%%%%%%%%%%%%%%%%%%%%%%%%%%%%%%%%%%%%
% \paragraph{Chapter Include Files.}
%
% The include files are called |cdocsch1.tex| and |cdocsch2.tex|.
%
%\iffalse
%<*samplechap1|samplechap2>
%\fi

% Optional override for |\version| flag:
%    \begin{macrocode}
%%\providecommand{\version}{final}
%    \end{macrocode}

% Include the main document:
%    \begin{macrocode}
\input{childdoc.def}
\childdocof{cdocsamp}
%    \end{macrocode}

%\iffalse
%</samplechap1|samplechap2>
%\fi
%
%\iffalse
%<*samplechap1>
%\fi
% Some text for chapter 1:
%    \begin{macrocode}
\section{one}
some text in chapter one
%    \end{macrocode}

%\iffalse
%</samplechap1>
%\fi
% Some text for chapter 2:
%\iffalse
%<*samplechap2>
%\fi
%    \begin{macrocode}
\section{two}
more text in chapter two
%    \end{macrocode}

%\iffalse
%</samplechap2>
%\fi
%
% %%%%%%%%%%%%%%%%%%%%%%%%%%%%%%%%%%%%%%
% \paragraph{Part Include Files.}
%
% The include files are called |cdocspt3.tex| and |cdocspt4.tex|.
%
%\iffalse
%<*samplepart3|samplepart4>
%\fi

% Optional override for |\version| flag:
%    \begin{macrocode}
%%\providecommand{\version}{final}
%    \end{macrocode}

% Include the main document:
%    \begin{macrocode}
\input{childdoc.def}
\childdocby{cdocsamp}
%    \end{macrocode}

%\iffalse
%</samplepart3|samplepart4>
%\fi
%
%\iffalse
%<*samplepart3>
%\fi
% Some text for part 3:
%    \begin{macrocode}
some text in part three
%    \end{macrocode}

%\iffalse
%</samplepart3>
%\fi
% Some text for part 4:
%\iffalse
%<*samplepart4>
%\fi
%    \begin{macrocode}
more text in part four
%    \end{macrocode}

%\iffalse
%</samplepart4>
%\fi
%
% %%%%%%%%%%%%%%%%%%%%%%%%%%%%%%%%%%%%%%
% \paragraph{Forwarding for a Complete Draft.}
%
% The following forwarding file |cdocsdrf.tex|
% compiles the main document in draft mode:
%\iffalse
%<*sampledraft>
%\fi
%    \begin{macrocode}
\def\version{draft}
\input{childdoc.def}
\childdocforward{cdocsamp}
%    \end{macrocode}

%\iffalse
%</sampledraft>
%\fi
%
% %%%%%%%%%%%%%%%%%%%%%%%%%%%%%%%%%%%%%%
% \paragraph{Forwarding for Final Version of the Chapters.}
%
% The following forwarding files |cdocsfn1.tex| and |cdocsfn2.tex|
% (with identical content)
% compile the final versions of the child documents
% |cdocsch1.tex| and |cdocsch2.tex|, respectively:
%\iffalse
%<*samplefinal>
%\fi
%    \begin{macrocode}
\def\version{final}
\input{childdoc.def}
\childdocforwardprefix[cdocsamp]{cdocsfn}{cdocsch}
%    \end{macrocode}

%\iffalse
%</samplefinal>
%\fi
%
% %%%%%%%%%%%%%%%%%%%%%%%%%%%%%%%%%%%%%%
% \paragraph{Command Line Processing.}
%
% The following three command lines generate the output files
% |cdocscld|, |cdocscl1| and |cdocscl2|
% which should be identical to
% |cdocsdrf|, |cdocsch1| and |cdocsfn2|, respectively:
% \begin{center}
% \begin{tabular}{l}
% |latex -jobname cdocscld \|\\
% |  "\def\version{draft}\input{childdoc.def}\childdocforward{cdocsamp}"|\\
% |latex -jobname cdocscl1 \|\\
% |  "\input{childdoc.def}\childdocforward[cdocsamp]{cdocsch1}"|\\
% |latex -jobname cdocscl2 \|\\
% |  "\def\version{final}\input{childdoc.def}\childdocforward{cdocsch2}"|
% \end{tabular}
% \end{center}
% Note that the trailing backslash on each first line
% merely continues the input to the second line
% (for convenient cut ant paste).
% Furthermore, the command |latex| can be replaced by any
% of its alternative versions such as |pdflatex|.
%
% %%%%%%%%%%%%%%%%%%%%%%%%%%%%%%%%%%%%%%%%%%%%%%%%%%%%%%%%%%%%%%%%%%%%%%%%%%%%%%
% %%%%%%%%%%%%%%%%%%%%%%%%%%%%%%%%%%%%%%%%%%%%%%%%%%%%%%%%%%%%%%%%%%%%%%%%%%%%%%
% \section{Implementation}
%\iffalse
%<*package>
%\fi
%
% This section describes the definitions file |childdoc.def|.

% The definitions cannot be loaded using |\usepackage| or |\RequirePackage|
% which has a mechanism to prevent loading a style file more than once.
% When loading the definitions by means of |\input|
% multiple instances have to be prevented manually:
%\iffalse
%This code needs to be before the `\ProvidesFile' directive
%which is defined at the beginning of this file.
%Therefore it is also placed there and commented out here.
%</package>
%<*discard>
%\fi
%    \begin{macrocode}
\ifdefined\childdocmain\endinput\fi
%    \end{macrocode}
%\iffalse
%</discard>
%<*package>
%\fi
%
% \macro{\ifchilddoc}
% \macro{\ifchilddocmanual}
% The conditional |\ifchilddoc| tells whether a
% child (true) or main (false) document is being compiled.
% The conditional |\ifchilddocmanual| tells whether
% the |\includeonly| mechanism is used (false) or
% the selection of child files must be performed manually (true).
% The definitions initialise to false:
%    \begin{macrocode}
\newif\ifchilddoc
\newif\ifchilddocmanual
%    \end{macrocode}

% \macro{\childdocname}
% \macro{\childdocjob}
% The macro |\childdocname| stores the name of the main document
% to be compiled. The macro |\childdocjob| stores the name of
% the document on which the \LaTeX{} compiler was originally invoked.
% The content of |\jobname| cannot be compared
% to filenames specified in the source due to different catcodes.
% The following code rescans |\jobname|, stores the result
% in |\childdocname| and saves a copy in |\childdocjob|:
%    \begin{macrocode}
\edef\childdocname{\scantokens\expandafter{\jobname\noexpand}}
\let\childdocjob\childdocname
%    \end{macrocode}

% \macro{\childdocdisable}
% The macro |\childdocdisable| prevents the main file
% from being processed more than once.
% At this stage, the main document command |\childdocmain|
% is assumed to be called once again where it should do nothing.
% Any subsequent call to it should prevent
% a secondary processing of the main document
% It overwrites the forwarding commands
% |\childdocof| and |\childdocforward|
% with empty macros to prevent further inclusions of the main document:
%    \begin{macrocode}
\newcommand{\childdocdisable}
{
  \renewcommand{\childdocmain}[1]{\renewcommand{\childdocmain}[1]{\endinput}}
  \renewcommand{\childdocof}[1]{}
  \renewcommand{\childdocby}[2][]{}
  \renewcommand{\childdocforward}[2][]{}
  \renewcommand{\childdocdisable}{}
}
%    \end{macrocode}

% \macro{\childdocmain}
% The macro |\childdocmain| is to be called at the top of the main file
% with nothing or the main filename (without extension) as argument.
% First, it breaks loops.
% If the argument is not empty and does not match |\childdocname|
% (which is set by the first inclusion of |childdoc.def|),
% |\ifchilddoc| is set to true, |\includeonly| is applied to the child file
% and |\jobname| is set to the main file
% (for proper handling of |.aux| files):
%    \begin{macrocode}
\newcommand{\childdocmain}[1]
{
  \childdocdisable\childdocmain{}
  \if?#1?\else
    \begingroup
      \def\childdoctmp{#1}
      \ifx\childdoctmp\childdocname
        \def\childdoctmp{}
      \else
        \def\childdoctmp
        {
          \childdoctrue
          \includeonly{\childdocname}
          \def\childdocjob{#1}
          \def\jobname{#1}
        }
      \fi
      \expandafter
    \endgroup
    \childdoctmp
  \fi
}
%    \end{macrocode}

% \macro{\childdocof}
% The command |\childdocof| redirects
% compilation to the main file |#1|.
%    \begin{macrocode}
\newcommand{\childdocof}[1]
{
  \childdocdisable
  \childdoctrue
  \includeonly{\childdocname}
  \def\jobname{#1}
  \def\childdocjob{#1}
  \input{#1}
}
%    \end{macrocode}

% \macro{\childdocby}
% The command |\childdocby| ....
%    \begin{macrocode}
\newcommand{\childdocby}[2][]
{
  \childdocdisable
  \childdoctrue
  \childdocmanualtrue
  \if?#1?\else
    \def\jobname{#2}
  \fi
  \def\childdocjob{#2}
  \input{#2}
  \endinput
}
%    \end{macrocode}

% \macro{\childdocforward}
% The command |\childdocforward| redirects
% compilation to the main file or
% (if the optional argument is given) a child file.
% Parameters are set as if the main file
% or a child file starting with |\childdocof| was compiled.
% Then compilation is handed over to the main file:
%    \begin{macrocode}
\newcommand{\childdocforward}[2][]
{
  \begingroup
    \if?#1?
      \def\childdoctmp
      {
        \def\childdocname{#2}
        \def\childdocjob{#2}
        \def\jobname{#2}
        \input{#2}
        \endinput
      }
    \else
      \def\childdoctmp
      {
        \childdocdisable
        \def\childdocname{#2}
        \childdoctrue
        \includeonly{#2}
        \def\childdocjob{#1}
        \def\jobname{#1}
        \input{#1}
        \endinput
      }
    \fi
    \expandafter
  \endgroup
  \childdoctmp
}
%    \end{macrocode}

% \macro{\childdocforwardprefix}
% The command |\childdocforwardprefix| redirects
% compilation to the main or a child file by means of a pattern.
% The prefix |#1| in the current filename is replaced by |#2|
% and the suffix of the current filename is kept
% (it is assumed that the filename does not contain the substring `|~~~|'
% which is used as a delimiter).
% Compilation is handed over to the new file by |\childdocforward|:
%    \begin{macrocode}
\newcommand{\childdocforwardprefix}[3][]
{
  \begingroup
    \def\childdocextract #2##1~~~{\def\childdoctmp{\childdocforward[#1]{#3##1}}}
    \expandafter\childdocextract\childdocname~~~
    \expandafter
  \endgroup
  \childdoctmp
}
%    \end{macrocode}

% \macro{\childdoc}
% The deprecated macro |\childdoc| is a legacy version of |\childdocmain|:
%    \begin{macrocode}
\newcommand{\childdoc}{\childdocmain}
%    \end{macrocode}

% \macro{\childdocredirect}
% The deprecated macro |\childdocredirect| is a legacy version
% of |\childdocforward| and |\childdocforwardprefix|:
%    \begin{macrocode}
\newcommand{\childdocredirect}[2][]
{
  \begingroup
    \if?#1?
      \def\childdoctmp{\childdocforward{#2}}
    \else
      \def\childdoctmp{\childdocforwardprefix{#1}{#2}}
    \fi
    \expandafter
  \endgroup
  \childdoctmp
}
%    \end{macrocode}

%\iffalse
%</package>
%\fi
%
\endinput
|
and perform the replacements as outlined below.
Instead of |\childdocmain{|\textit{main}|}| add the following code
to the top of the main file:
%
\begin{center}
\begin{tabular}{l}
|\||ifdefined\childdocname\endinput\||fi\newif\ifchilddoc|\\
|\edef\childdocname{\scantokens\expandafter{\jobname\noexpand}}|\\
|\def\childdocmain{|\textit{main}|}\||ifx\childdocmain\childdocname\||else|\\
|\childdoctrue\includeonly{\childdocname}\let\jobname\childdocmain\||fi|\\
\end{tabular}
\end{center}
%
Instead of |\childdocof{|\textit{main}|}| just include the main file
at the top of each child file:
%
\begin{center}
|\input{|\textit{main}|}|
\end{center}
%
A simple redirection |\childdocforward{|\textit{dest}|}| is achieved by:
%
\begin{center}
|\def\jobname{|\textit{dest}|}\input{\jobname}|
\end{center}
%
The redirection with prefix
|\childdocforwardprefix[|\textit{prefix}|]{|\textit{dest}|}|
is accomplished by:
%
\begin{center}
\begin{tabular}{l}
|{\edef\jobname{\scantokens\expandafter{\jobname\noexpand}}|\\
|\def\redirectjob |\textit{prefix}|#1~~~{\gdef\jobname{|\textit{dest}|#1}}|\\
|\expandafter\redirectjob\jobname~~~}\input{\jobname}|
\end{tabular}
\end{center}

In an alternative approach,
child documents can be compiled by a specific command line
without additional code or specific definitions:
%
\begin{center}
|... -jobname "|\textit{target}|" "|[\textit{flags}]%
|\includeonly{|\textit{dest}|}\input{|\textit{main}|}"|
\end{center}
%

%%%%%%%%%%%%%%%%%%%%%%%%%%%%%%%%%%%%%%%%%%%%%%%%%%%%%%%%%%%%%%%%%%%%%%%%%%%%%%%%
%%%%%%%%%%%%%%%%%%%%%%%%%%%%%%%%%%%%%%%%%%%%%%%%%%%%%%%%%%%%%%%%%%%%%%%%%%%%%%%%
\section{Information}

%%%%%%%%%%%%%%%%%%%%%%%%%%%%%%%%%%%%%%%%%%%%%%%%%%%%%%%%%%%%%%%%%%%%%%%%%%%%%%%%
\subsection{Copyright}

Copyright \copyright{} 2017--2018 Niklas Beisert

This work may be distributed and/or modified under the
conditions of the \LaTeX{} Project Public License, either version 1.3
of this license or (at your option) any later version.
The latest version of this license is in
  \url{http://www.latex-project.org/lppl.txt}
and version 1.3 or later is part of all distributions of \LaTeX{}
version 2005/12/01 or later.

This work has the LPPL maintenance status `maintained'.

The Current Maintainer of this work is Niklas Beisert.

This work consists of the files |README.txt|, |childdoc.ins| and |childdoc.dtx|
as well as the derived files |childdoc.def|, |cdocsamp.tex|
with |cdocsch1.tex|, |cdocsch2.tex|, |cdocspt3.tex|, |cdocspt4.tex|,
|cdocsdrf.tex|, |cdocsfn1.tex|, |cdocsfn2.tex|
as well as |childdoc.pdf|.

%%%%%%%%%%%%%%%%%%%%%%%%%%%%%%%%%%%%%%%%%%%%%%%%%%%%%%%%%%%%%%%%%%%%%%%%%%%%%%%%
\subsection{Files and Installation}

The package consists of the files:
%
\begin{center}
\begin{tabular}{ll}
    |README.txt|   & readme file \\
    |childdoc.ins| & installation file \\
    |childdoc.dtx| & source file \\
    |childdoc.def| & definition file \\
    |cdocsamp.tex| & sample main file \\
    |cdocsch1.tex| & sample include file \\
    |cdocsch2.tex| & sample include file \\
    |cdocspt3.tex| & sample part file \\
    |cdocspt4.tex| & sample part file \\
    |cdocsdrf.tex| & sample redirection file \\
    |cdocsfn1.tex| & sample redirection file \\
    |cdocsfn2.tex| & sample redirection file \\
    |childdoc.pdf| & manual
\end{tabular}
\end{center}
%
The distribution consists of the files
|README.txt|, |childdoc.ins| and |childdoc.dtx|.
%
\begin{itemize}
\item
Run (pdf)\LaTeX{} on |childdoc.dtx|
to compile the manual |childdoc.pdf| (this file).
\item
Run \LaTeX{} on |childdoc.ins| to create the definitions file |childdoc.def|
and the sample |cdocsamp.tex| with include files
|cdocsch1.tex|, |cdocsch2.tex|, |cdocspt3.tex|, |cdocspt4.tex|,
|cdocsdrf.tex|, |cdocsfn1.tex|, |cdocsfn2.tex|.
Then copy the file |childdoc.def| to an appropriate directory of your \LaTeX{}
distribution, e.g.\ \textit{texmf-root}|/tex/latex/childdoc|.
\end{itemize}

%%%%%%%%%%%%%%%%%%%%%%%%%%%%%%%%%%%%%%%%%%%%%%%%%%%%%%%%%%%%%%%%%%%%%%%%%%%%%%%%
\subsection{Related CTAN Packages}

There are several other packages which offer a similar functionality:
%
\begin{itemize}
\item
The packages
\href{http://ctan.org/pkg/docmute}{\textsf{docmute}},
\href{http://ctan.org/pkg/includex}{\textsf{includex}} and
\href{http://ctan.org/pkg/standalone}{\textsf{standalone}}
provide commands to include only the document body of
a child file thus allowing both files to be compiled individually.
\item
The packages \href{http://ctan.org/pkg/subdocs}{\textsf{subdocs}}
and \href{http://ctan.org/pkg/subfiles}{\textsf{subfiles}}
provide structures in which the main and child documents can be
encapsulated and allowing them to be compiled individually.
The inclusion mechanism is different from the conventional |\include|.
\item
The package \href{http://ctan.org/pkg/combine}{\textsf{combine}}
is an elaborate solution to combine several documents into one.
\end{itemize}
%
See also the CTAN topic \href{http://ctan.org/topic/subdocs}{\textsf{subdocs}}
for further related packages.
The present package differs from the above solutions in that
a document structure constructed with the conventional |\include| mechanism
just needs two extra commands at the top of every file
such that all constituent files can be compiled individually.

%%%%%%%%%%%%%%%%%%%%%%%%%%%%%%%%%%%%%%%%%%%%%%%%%%%%%%%%%%%%%%%%%%%%%%%%%%%%%%%%
%\subsection{Feature Suggestions}
%
%The following is a list of features which may be useful for future
%versions of this package:
%%
%\begin{itemize}
%\item
%\ldots
%\end{itemize}

%%%%%%%%%%%%%%%%%%%%%%%%%%%%%%%%%%%%%%%%%%%%%%%%%%%%%%%%%%%%%%%%%%%%%%%%%%%%%%%%
\subsection{Revision History}

%%%%%%%%%%%%%%%%%%%%%%%%%%%%%%%%%%%%%%%%
\paragraph{v2.0:} 2018/12/30

\begin{itemize}
\item
immediate forward processing
\item
added |\childdocby| mechanism
\item
manual restructured
\end{itemize}

%%%%%%%%%%%%%%%%%%%%%%%%%%%%%%%%%%%%%%%%
\paragraph{v1.6:} 2018/01/17

\begin{itemize}
\item
application for development of include files
\item
corrections to manual
\end{itemize}

%%%%%%%%%%%%%%%%%%%%%%%%%%%%%%%%%%%%%%%%
\paragraph{v1.5:} 2017/05/21

\begin{itemize}
\item
more complete structuring introduced
\item
|\childdocof| introduced
\item
|\childdoc| renamed to |\childdocmain|
\item
|\childredirect| renamed to |\childdocforward| and |\childdocforwardprefix|
and functionality expanded
\end{itemize}

%%%%%%%%%%%%%%%%%%%%%%%%%%%%%%%%%%%%%%%%
\paragraph{v1.0:} 2017/04/27

\begin{itemize}
\item
manual and install package
\item
first version published on CTAN
\end{itemize}

%%%%%%%%%%%%%%%%%%%%%%%%%%%%%%%%%%%%%%%%
\paragraph{v0.6:} 2017/04/26

\begin{itemize}
\item
redirection mechanism added
\end{itemize}

%%%%%%%%%%%%%%%%%%%%%%%%%%%%%%%%%%%%%%%%
\paragraph{v0.5:} 2017/04/26

\begin{itemize}
\item
functionality in definition file
\end{itemize}


%%%%%%%%%%%%%%%%%%%%%%%%%%%%%%%%%%%%%%%%%%%%%%%%%%%%%%%%%%%%%%%%%%%%%%%%%%%%%%%%
%%%%%%%%%%%%%%%%%%%%%%%%%%%%%%%%%%%%%%%%%%%%%%%%%%%%%%%%%%%%%%%%%%%%%%%%%%%%%%%%
%%%%%%%%%%%%%%%%%%%%%%%%%%%%%%%%%%%%%%%%%%%%%%%%%%%%%%%%%%%%%%%%%%%%%%%%%%%%%%%%
\appendix

\settowidth\MacroIndent{\rmfamily\scriptsize 000\ }

 \DocInput{childdoc.dtx}

\end{document}
%</driver>
% \fi
%
% %%%%%%%%%%%%%%%%%%%%%%%%%%%%%%%%%%%%%%%%%%%%%%%%%%%%%%%%%%%%%%%%%%%%%%%%%%%%%%
% %%%%%%%%%%%%%%%%%%%%%%%%%%%%%%%%%%%%%%%%%%%%%%%%%%%%%%%%%%%%%%%%%%%%%%%%%%%%%%
% \section{Sample}
%\iffalse
%<*samplemain>
%\fi
%
% The following presents a sample document
% with two chapters, two parts, a title page,
% a compile flag as well as three forwarding files to set the flag.
% It consists of eight |.tex| files:
% \begin{center}
% \begin{tabular}{ll}
% |cdocsamp.tex|&main file\\
% |cdocsch1.tex|&include file for chapter 1\\
% |cdocsch2.tex|&include file for chapter 2\\
% |cdocspt3.tex|&include file for part 3\\
% |cdocspt4.tex|&include file for part 4\\
% |cdocsdrf.tex|&forwarding file for main file in draft mode\\
% |cdocsfi1.tex|&forwarding file for final version of chapter 1\\
% |cdocsfi2.tex|&forwarding file for final version of chapter 2\\
% \end{tabular}
% \end{center}
% Each of the eight files can be compiled directly by the \LaTeX{} compiler.
%
% %%%%%%%%%%%%%%%%%%%%%%%%%%%%%%%%%%%%%%
% \paragraph{Main File.}
%
% The main file is called |cdocsamp.tex|.
%
% Load the \textsf{childdoc} definitions and
% declare the filename for the main document:
%    \begin{macrocode}
% \iffalse
%
% childdoc.dtx Copyright (C) 2017-2018 Niklas Beisert
%
% This work may be distributed and/or modified under the
% conditions of the LaTeX Project Public License, either version 1.3
% of this license or (at your option) any later version.
% The latest version of this license is in
%   http://www.latex-project.org/lppl.txt
% and version 1.3 or later is part of all distributions of LaTeX
% version 2005/12/01 or later.
%
% This work has the LPPL maintenance status `maintained'.
%
% The Current Maintainer of this work is Niklas Beisert.
%
% This work consists of the files childdoc.dtx and childdoc.ins
% and the derived files childdoc.def and cdocsamp.tex with
% cdocsch1.tex, cdocsch2.tex, cdocsdrf.tex, cdocsfn1.tex, cdocsfn2.tex.
%
%<package>\ifdefined\childdocmain\endinput\fi
%<package>\ProvidesFile{childdoc.def}[2018/12/30 v2.0 child document driver]
%<samplemain>\ProvidesFile{cdocsamp.tex}[2018/12/30 v2.0 sample for childdoc]
%<*driver>
%\ProvidesFile{childdoc.drv}[2018/12/30 v2.0 childdoc reference manual file]
\PassOptionsToClass{10pt,a4paper}{article}
\documentclass{ltxdoc}

\usepackage[margin=35mm]{geometry}
\usepackage{hyperref}
\usepackage{hyperxmp}
\usepackage[usenames]{color}

\hypersetup{colorlinks=true}
\hypersetup{pdfstartview=FitH}
\hypersetup{pdfpagemode=UseNone}
\hypersetup{pdfsource={}}
\hypersetup{pdflang={en-UK}}
\hypersetup{pdfcopyright={Copyright 2017-2018 Niklas Beisert.
  This work may be distributed and/or modified under the
  conditions of the LaTeX Project Public License, either version 1.3
  of this license or (at your option) any later version.}}
\hypersetup{pdflicenseurl={http://www.latex-project.org/lppl.txt}}
\hypersetup{pdfcontactaddress={ETH Zurich, ITP, HIT K,
  Wolfgang-Pauli-Strasse 27}}
\hypersetup{pdfcontactpostcode={8093}}
\hypersetup{pdfcontactcity={Zurich}}
\hypersetup{pdfcontactcountry={Switzerland}}
\hypersetup{pdfcontactemail={nbeisert@itp.phys.ethz.ch}}
\hypersetup{pdfcontacturl={http://people.phys.ethz.ch/\xmptilde nbeisert/}}

\newcommand{\secref}[1]{\hyperref[#1]{section \ref*{#1}}}

\parskip1ex
\parindent0pt
\let\olditemize\itemize
\def\itemize{\olditemize\parskip0pt}

\begin{document}

\title{The \textsf{childdoc} Package}
\hypersetup{pdftitle={The childdoc Package}}
\author{Niklas Beisert\\[2ex]
  Institut f\"ur Theoretische Physik\\
  Eidgen\"ossische Technische Hochschule Z\"urich\\
  Wolfgang-Pauli-Strasse 27, 8093 Z\"urich, Switzerland\\[1ex]
  \href{mailto:nbeisert@itp.phys.ethz.ch}
  {\texttt{nbeisert@itp.phys.ethz.ch}}}
\hypersetup{pdfauthor={Niklas Beisert}}
\hypersetup{pdfsubject={Manual for the LaTeX2e Package childdoc}}
\date{30 December 2018, \textsf{v2.0}}
\maketitle

\begin{abstract}\noindent
\textsf{childdoc} is a \LaTeXe{} package
that enables the direct compilation
of document sections included by |\include|
to individual files.
\end{abstract}

\begingroup
\parskip0ex
\tableofcontents
\endgroup

%%%%%%%%%%%%%%%%%%%%%%%%%%%%%%%%%%%%%%%%%%%%%%%%%%%%%%%%%%%%%%%%%%%%%%%%%%%%%%%%
%%%%%%%%%%%%%%%%%%%%%%%%%%%%%%%%%%%%%%%%%%%%%%%%%%%%%%%%%%%%%%%%%%%%%%%%%%%%%%%%
\section{Introduction}

\LaTeX{} provides a mechanism to structure a large document (such as a book)
into a main file and several child files (containing the chapters)
using the |\include| command.
This mechanism is beneficial for documents
which span hundreds of pages in order to
make the source file(s) more manageable.
Moreover, compilation can be restricted to
selected child files by means of the |\includeonly| command.
The latter feature can be used to reduce the compilation time while editing
(this was significantly more useful in the earlier days of \LaTeX{})
or to generate a smaller document which is easier to navigate.
Another application of |\includeonly| is to generate
documents consisting of selected parts of the complete document.

However, there are a few drawbacks of the plain |\include| mechanism:
\begin{itemize}
\item
The child files cannot be compiled on their own,
they can only be compiled via the main file.
A naive editing environment
(such as a text editor with an option
to have the current file processed by \LaTeX)
may require one to switch to the main file before compiling;
attempting to compile the child file produces errors.
\item
The main file must be modified (each time)
to adjust the |\includeonly| command
to the present needs. This easily leaves the main file in a messy state.
\item
The generated document will always carry the filename
of the main document. This is inconvenient if
several child files are to be compiled and
to be kept for distribution.
\end{itemize}

The present package provides a simple interface
to make child files individually compilable by \LaTeX{}.
Compiling a child file then has the same effect as compiling
the main file with an |\includeonly| command
to select the appropriate child.
Moreover the generated document will carry the name of the child
rather than the main file.
This resolves all three above issues.

This feature is meant to make the editing of books,
thesis documents and lecture notes somewhat more convenient.
However, the package can also be used efficiently for
composing a series of documents (such as exercise sheets)
which are typically distributed individually.
It then assists the author in generating the individual documents
(potentially in different versions)
as well as a document containing the collected series.
Another application is in developing style files
or other kinds of included material
where compilation of the style file could redirect
to a sample or test file.

%%%%%%%%%%%%%%%%%%%%%%%%%%%%%%%%%%%%%%%%%%%%%%%%%%%%%%%%%%%%%%%%%%%%%%%%%%%%%%%%
%%%%%%%%%%%%%%%%%%%%%%%%%%%%%%%%%%%%%%%%%%%%%%%%%%%%%%%%%%%%%%%%%%%%%%%%%%%%%%%%
\section{Usage}

First of all, the package \textsf{childdoc} is \emph{not} a standard
\LaTeXe{} |.sty| style file! Therefore it needs to be invoked in
a non-standard way.

%%%%%%%%%%%%%%%%%%%%%%%%%%%%%%%%%%%%%%%%%%%%%%%%%%%%%%%%%%%%%%%%%%%%%%%%%%%%%%%%
\subsection{Included Files}
\label{sec:include}

%%%%%%%%%%%%%%%%%%%%%%%%%%%%%%%%%%%%%%%%
\DescribeMacro{\childdocmain}
To use the package, add the commands
\begin{center}
\begin{tabular}{l}
|\input{childdoc.def}|\\
|\childdocmain{}|\\
\end{tabular}
\end{center}
at the very top of the main \LaTeX{} file,
in particular \emph{before} the |\documentclass| statement!
The argument of |\childdocmain| should be left empty
(but it must be present).

%%%%%%%%%%%%%%%%%%%%%%%%%%%%%%%%%%%%%%%%
\DescribeMacro{\childdocof}
Furthermore, add the commands
\begin{center}
\begin{tabular}{l}
|\input{childdoc.def}|\\
|\childdocof{|\textit{main}|}|\\
\end{tabular}
\end{center}
at the top of every child file \textit{child}
which is included by |\include{|\textit{child}|}|
from within the main file
(or at least for those files to be compiled individually).
The argument \textit{main} must be the filename of the main file.

There are a couple of
considerations in setting up the main and child documents:

%%%%%%%%%%%%%%%%%%%%%%%%%%%%%%%%%%%%%%%%
\paragraph{Restrictions.}

Please note the following restrictions:
\begin{itemize}
\item
|\childdocmain| must be called with one argument \textit{main}
to ensure compatibility with earlier version of the package.
It must either be empty (|\childdocmain{}|)
or precisely match the filename of the main file in which it is specified.
See \secref{sec:detection} for further information.
\item
The filename \textit{main} must be specified without the |.tex| extension.
\item
The filename \textit{main} is case sensitive
(even in case-insensitive file systems)
due to internal string comparison.
\item
The argument \textit{main} should be fully expanded, it cannot be a macro.
\item
Subdirectories and special characters should be avoided in filenames.
\item
The command |\childdocmain{|\textit{main}|}| must be followed by a whitespace.
It should not be followed immediately by another command
or by a comment mark `|%|'.
This is because the \TeX{} parser reads the token immediately following
the argument of |\childdocmain| and puts it
at the beginning of every child section;
however, a white\-space is ignored.
\end{itemize}

%%%%%%%%%%%%%%%%%%%%%%%%%%%%%%%%%%%%%%%%
\paragraph{Content of Main File.}

It is advisable to place all content in the child files included by |\include|.
Any output contained in the main file will appear in all child documents
unless suppressed manually;
it cannot be suppressed automatically by the |\includeonly| directive
and thus should normally be avoided.
A method to include some content in the main file
by means of conditional processing is described in \secref{sec:conditional}.

%%%%%%%%%%%%%%%%%%%%%%%%%%%%%%%%%%%%%%%%
\paragraph{Page Numbering.}

When only a part of the document is compiled,
the appropriate numbering of pages
(as well as other status parameters)
is determined from the |.aux| files.
The latter contain information from previous passes.
However this information needs to propagate through
all intermediate child documents.
Therefore the page numbering in child documents may well
be inconsistent until the complete document is compiled at least once.

A useful (if unconventional) way to always ensure a consistent
page numbering is to restart the numbering in each child document
and denote the pages by `\textit{child}|.|\textit{page}'
where \textit{child} represents the chapter/section number of the child file.
This can be achieved by the command
|\numberwithin{page}{|\textit{child}|}|
of the \textsf{amsmath} package
where \textit{child} can be |chapter| or |section|
depending on the chosen structuring.
Alternatively, one can modify the macro |\thepage| appropriately
and reset the counter |page| at the start of each child file.

%%%%%%%%%%%%%%%%%%%%%%%%%%%%%%%%%%%%%%%%%%%%%%%%%%%%%%%%%%%%%%%%%%%%%%%%%%%%%%%%
\subsection{Conditional Processing}
\label{sec:conditional}

The package provides a mechanism to compile different versions
of a document. To customise the versions further some conditional processing
can come in handy to distinguish which version is being compiled.
The package provides two macros to describe the compilation context:

%%%%%%%%%%%%%%%%%%%%%%%%%%%%%%%%%%%%%%%%
\DescribeMacro{\ifchilddoc}
The conditional |\ifchilddoc| distinguishes between the compilation of
child documents and the main document:
%
\begin{center}
|\ifchilddoc |\textit{child-code}| |[|\||else |\textit{main-code}]| \||fi|
\end{center}

%%%%%%%%%%%%%%%%%%%%%%%%%%%%%%%%%%%%%%%%
\DescribeMacro{\childdocname}
\DescribeMacro{\childdocjob}
The macro |\childdocname| contains the filename (without extension)
of the main or child file being processed.
Note that |\childdocjob| will always contain the name of the main file.

%%%%%%%%%%%%%%%%%%%%%%%%%%%%%%%%%%%%%%%%
\paragraph{Title Page.}

Conditional processing can be used to include a title or banner page
in the main document when proper precautions are taken.
Importantly, the code in the main file should ensure that the page counter
(as well as other status parameters which are stored in the |.aux| files)
takes the same value after the conditional processing.
Otherwise the page numbers may take divergent values
depending on which part is compiled.

For example, a title page could be declared by:
%
\begin{center}
\begin{tabular}{l}
|\ifchilddoc\||else|\\
|\addtocounter{page}{-1}|\\
\textit{code for title page}\\
|\newpage|\\
|\||fi|
\end{tabular}
\end{center}
%
A banner page for the child documents can be generated by:
%
\begin{center}
\begin{tabular}{l}
|\ifchilddoc|\\
|\addtocounter{page}{-1}|\\
\textit{code for banner page}\\
|\newpage|\\
|\||fi|
\end{tabular}
\end{center}
%
Here one could write a message such as:
\begin{center}
|This is the part \childdocname{} of \childdocjob{}.|
\end{center}

%%%%%%%%%%%%%%%%%%%%%%%%%%%%%%%%%%%%%%%%%%%%%%%%%%%%%%%%%%%%%%%%%%%%%%%%%%%%%%%%
\subsection{Flags}
\label{sec:flags}

The package makes it easy to generate different versions
of the main or child documents.
To this end compilation flags can be defined
and assigned different default values.
They will be particularly useful in conjunction
with the forwarding mechanism described in \secref{sec:forward}.

For example, it may be useful to have a flag |\version|
which can be set to |draft| or |final|.
The document source will contain some conditional code
depending on the value of |\version|.
Suppose further, the flag should default to |final| for the main file
and to |draft| for child files
which is a natural assignment for editing the document.
This is achieved by placing the following code
in the preamble of the main document
(below the |\childdocmain| directive):
%
\begin{center}
\begin{tabular}{l}
|\ifchilddoc|\\
|\providecommand{\version}{draft}|\\
|\||else|\\
|\providecommand{\version}{final}|\\
|\||fi|
\end{tabular}
\end{center}
%
The definition by |\providecommand| makes sure
that previous definitions are not overwritten.
Further statements |\providecommand{\version}{...}|
can thus be added before the above code to override it.

For the main file, one might add a line
(between |\childdocmain| and the above block)
%
\begin{center}
|%\ifchilddoc\||else\providecommand{\version}{draft}\||fi|
\end{center}
%
which can be uncommented to produce a draft version.
Likewise one can add a line to the very top of a child file
(above the |\childdocof{|\textit{main}|}| directive)
%
\begin{center}
|%\providecommand{\version}{final}|
\end{center}
%
which can be uncommented to produce the final version of this child document.

%%%%%%%%%%%%%%%%%%%%%%%%%%%%%%%%%%%%%%%%%%%%%%%%%%%%%%%%%%%%%%%%%%%%%%%%%%%%%%%%
\subsection{Forwarding}
\label{sec:forward}

Different versions of the main or child documents
using compilation flags as described in \secref{sec:flags}
can be (permanently) stored in different files
for convenient compilation, viewing and distribution.
To this end, the package defines a command
to pass on compilation to a different file:

%%%%%%%%%%%%%%%%%%%%%%%%%%%%%%%%%%%%%%%%
\DescribeMacro{\childdocforward}
The command |\childdocforward| redirects processing to
another source file:
%
\begin{center}
\begin{tabular}{l}
|\input{childdoc.def}|\\
|\childdocforward[|\textit{main}|]{|\textit{dest}|}|\\
\end{tabular}
\end{center}
%
The argument \textit{dest} is the destination file
(without extension).
It should be the main file or one of the child files.
Note that further \textsf{childdoc} directives
such as |\childdocof| and |\childdocforward|
in the indicated file will be processed in this form.
The optional argument \textit{main}
passes on directly to the main file \textit{main}
while pretending to compile the child \textit{dest}.
This form behaves as if \textit{dest}
issues |\childdocof{|\textit{main}|}| right away,
and no further \textsf{childdoc} directives will be processed.

%%%%%%%%%%%%%%%%%%%%%%%%%%%%%%%%%%%%%%%%
\DescribeMacro{\...prefix}
In the alternative form |\childdocforwardprefix|,
%
\begin{center}
\begin{tabular}{l}
|\input{childdoc.def}|\\
|\childdocforwardprefix[|\textit{main}|]{|\textit{prefix}|}{|\textit{dest}|}|
\end{tabular}
\end{center}
%
the destination file is determined by a pattern
depending on the current file:
To make this work, the current file must be called
`{\textit{prefix}\hspace{0.2em}\textit{suffix}}'
with \textit{prefix} matching precisely the argument.
Processing is then passed on to the file
`{\textit{dest}\hspace{0.2em}\textit{suffix}}'.
Surely, the same effect is achieved by
directly specifying the
argument `{\textit{dest}\hspace{0.2em}\textit{suffix}}'
in the first form.
However, that requires to set up a different file
for each child. With the alternative form of the command
all these files can have exactly the same content
which simplifies setting them up and maintaining them.

For example, the following file |draft.tex|
with a compilation flag |\version| as described in \secref{sec:flags}
compiles the main document as a draft:
%
\begin{center}
\begin{tabular}{l}
|\def\version{draft}|\\
|\input{childdoc.def}|\\
|\childdocforward{|\textit{main}|}|
\end{tabular}
\end{center}
%
Likewise, the following files |final|\textit{nn}|.tex|
compile the final version of the child document
|child|\textit{nn}|.tex|:
%
\begin{center}
\begin{tabular}{l}
|\def\version{final}|\\
|\input{childdoc.def}|\\
|\childdocforwardprefix{final}{child}|
\end{tabular}
\end{center}
%

Note that when several versions of a main file and/or of each child file
are to be generated, it may be convenient to set up a |Makefile| or
shell script to automatise the process.

%%%%%%%%%%%%%%%%%%%%%%%%%%%%%%%%%%%%%%%%%%%%%%%%%%%%%%%%%%%%%%%%%%%%%%%%%%%%%%%%
\subsection{Command Line Processing}
\label{sec:commandline}

The effect of redirection files can also be achieved by invoking
the \LaTeX{} compiler with a more elaborate command line.
Most conveniently this should be done as part
of a shell script or a |Makefile|.

When using \textsf{childdoc} in the main file, the following
command lines effectively perform a redirection
(note that depending on the shell being used,
backslashes may have to be doubled: `|\|' $\to$ `|\\|'):
%
\begin{center}
|... -jobname "|\textit{target}|" |\\|"|[\textit{flags}]%
|\input{childdoc.def}\childdocforward[|\textit{main}|]{|\textit{dest}|}"|
\end{center}
%
Here \textit{target} is the name of the output file,
\textit{main} is the name of the main file
and \textit{dest} is the name of the main or child file to be processed
(all filenames without extensions).
The optional argument \textit{main} can be omitted
if \textit{main} matches \textit{dest}.
Optionally, compilation \textit{flags} can be defined via |\def| commands.
This command line makes the \TeX{} engine believe
it is compiling the file \textit{target}
whose content is specified as the latter parameter.
The provided code then forwards the processing to
\textit{main} or \textit{dest} as described in \secref{sec:forward}.

%%%%%%%%%%%%%%%%%%%%%%%%%%%%%%%%%%%%%%%%%%%%%%%%%%%%%%%%%%%%%%%%%%%%%%%%%%%%%%%%
\subsection{Include by Input}
\label{sec:input}

Including child documents by |\include| has some restrictions by design.
Most notably, the content of a child document always occupies
its own set of pages; pages cannot be shared between child documents.
Usually, this behaviour makes perfect sense
because each child document contain an essential part of the document.
However, in some situations it may be desirable to compose
a document from a collection of parts
without having mandatory page breaks between then.
For this case, the package
provides a mechanism to include parts
by |\input| which can also be processed individually.
However, by construction this mechanism
requires manual handling of the content to be output.

%%%%%%%%%%%%%%%%%%%%%%%%%%%%%%%%%%%%%%%%
\DescribeMacro{\ifchilddocmanual}
The main file should be prepared as usual, see \secref{sec:include}.
However, the document body must make a distinction
between processing of an individual part and of the main document, e.g.:
%
\begin{center}
\begin{tabular}{l}
|\ifchilddocmanual|\\
|\input{\childdocname}|\\
|\||else|\\
\textit{document body with }|\input{|\textit{part}|}|\\
|\||fi|
\end{tabular}
\end{center}
%
The conditional |\ifchilddocmanual| is true whenever
a part to be included by |\input| is being compiled,
and the name of the part is stored in |\childdocname|.

%%%%%%%%%%%%%%%%%%%%%%%%%%%%%%%%%%%%%%%%
\DescribeMacro{\childdocby}
Each part to be included by |\input| should start with:
%
\begin{center}
\begin{tabular}{l}
|\input{childdoc.def}|\\
|\childdocby{|\textit{main}|}|\\
\end{tabular}
\end{center}
%
The directive |\childdocby| is similar to |\childdocof|
described in \secref{sec:include},
but the subsequent selection of content must be done manually.
To that end, both |\ifchilddoc| and |\ifchilddocmanual|
will be true upon processing of a part,
and the name of the part is stored in |\childdocname|.
Note that |\jobname| will be set to the filename of the current part
so that each part receives an individual |.aux| file
that does not interfere with the |.aux| file(s) of the main document.
This behaviour can be altered by the alternative form
|\childdocby[*]{|\textit{main}|}| (with a non-empty optional argument)
which uses the |.aux| file of the main document
by setting |\jobname| to \textit{main}.

%%%%%%%%%%%%%%%%%%%%%%%%%%%%%%%%%%%%%%%%%%%%%%%%%%%%%%%%%%%%%%%%%%%%%%%%%%%%%%%%
\subsection{Driver Development}
\label{sec:driver}

The \textsf{childdoc} mechanism can also be use for the development
of definition files such as \LaTeX{} styles or classes.
This case differs from the above setup with multiple parts
included by |\include| in that no |\includeonly| should be invoked.
This can be achieved by starting the include file
(before |\ProvidesPackage|) with:
%
\begin{center}
\begin{tabular}{l}
|\input{childdoc.def}|\\
|\childdocforward{|\textit{main}|}|\\
\end{tabular}
\end{center}
%
or alternatively with:
%
\begin{center}
\begin{tabular}{l}
|\input{childdoc.def}|\\
|\childdocby{|\textit{main}|}|\\
\end{tabular}
\end{center}
%
Both forms have slightly different effects as described above.
The main file is prepared as usual, see \secref{sec:include}.

%%%%%%%%%%%%%%%%%%%%%%%%%%%%%%%%%%%%%%%%%%%%%%%%%%%%%%%%%%%%%%%%%%%%%%%%%%%%%%%%
\subsection{Legacy Detection}
\label{sec:detection}

The directive |\childdocmain| in the main file can detect
whether the complete document or merely a child is to be compiled
even without using the directive |\childdocof|.
This method is deprecated because it is less robust
and there is no compelling reason to use it;
it is merely provided for backward compatibility
and it may be removed in future versions.

If the detection mechanism is to be used,
it is mandatory to correctly specify
the filename of the main file as the argument of |\childdocmain|:
%
\begin{center}
\begin{tabular}{l}
|\input{childdoc.def}|\\
|\childdocmain{|\textit{main}|}|\\
\end{tabular}
\end{center}
%
If |\jobname| does not match the argument \textit{main} of |\childdocmain|,
it is assumed that |\jobname| points to the child file to be compiled.
When using |\childdocmain| with the main file specified as argument,
it suffices to start a child file
with just |\input{|\textit{main}|}|
without loading of the package and using |\childdocof|.
If instead all processing is done
with the appropriate \textsf{childdoc} directives,
the argument of \textit{main} of |\childdocmain| can be empty.

An alternative version of the command line processing described
in \secref{sec:commandline} using the detection mechanism reads:
%
\begin{center}
|... -jobname "|\textit{target}|" "|[\textit{flags}]%
[|\def\jobname{|\textit{dest}|}|]|\input{|\textit{main}|}"|
\end{center}

%%%%%%%%%%%%%%%%%%%%%%%%%%%%%%%%%%%%%%%%%%%%%%%%%%%%%%%%%%%%%%%%%%%%%%%%%%%%%%%%
\subsection{Manual Code}
\label{sec:manual}

In case one cannot be certain whether the definitions file |childdoc.def|
is installed on the target \TeX{} distribution
and one prefers not to ship it,
it is conceivable to paste a few relevant commands into the sources.

To that end, drop all statements |\input{childdoc.def}|
and perform the replacements as outlined below.
Instead of |\childdocmain{|\textit{main}|}| add the following code
to the top of the main file:
%
\begin{center}
\begin{tabular}{l}
|\||ifdefined\childdocname\endinput\||fi\newif\ifchilddoc|\\
|\edef\childdocname{\scantokens\expandafter{\jobname\noexpand}}|\\
|\def\childdocmain{|\textit{main}|}\||ifx\childdocmain\childdocname\||else|\\
|\childdoctrue\includeonly{\childdocname}\let\jobname\childdocmain\||fi|\\
\end{tabular}
\end{center}
%
Instead of |\childdocof{|\textit{main}|}| just include the main file
at the top of each child file:
%
\begin{center}
|\input{|\textit{main}|}|
\end{center}
%
A simple redirection |\childdocforward{|\textit{dest}|}| is achieved by:
%
\begin{center}
|\def\jobname{|\textit{dest}|}\input{\jobname}|
\end{center}
%
The redirection with prefix
|\childdocforwardprefix[|\textit{prefix}|]{|\textit{dest}|}|
is accomplished by:
%
\begin{center}
\begin{tabular}{l}
|{\edef\jobname{\scantokens\expandafter{\jobname\noexpand}}|\\
|\def\redirectjob |\textit{prefix}|#1~~~{\gdef\jobname{|\textit{dest}|#1}}|\\
|\expandafter\redirectjob\jobname~~~}\input{\jobname}|
\end{tabular}
\end{center}

In an alternative approach,
child documents can be compiled by a specific command line
without additional code or specific definitions:
%
\begin{center}
|... -jobname "|\textit{target}|" "|[\textit{flags}]%
|\includeonly{|\textit{dest}|}\input{|\textit{main}|}"|
\end{center}
%

%%%%%%%%%%%%%%%%%%%%%%%%%%%%%%%%%%%%%%%%%%%%%%%%%%%%%%%%%%%%%%%%%%%%%%%%%%%%%%%%
%%%%%%%%%%%%%%%%%%%%%%%%%%%%%%%%%%%%%%%%%%%%%%%%%%%%%%%%%%%%%%%%%%%%%%%%%%%%%%%%
\section{Information}

%%%%%%%%%%%%%%%%%%%%%%%%%%%%%%%%%%%%%%%%%%%%%%%%%%%%%%%%%%%%%%%%%%%%%%%%%%%%%%%%
\subsection{Copyright}

Copyright \copyright{} 2017--2018 Niklas Beisert

This work may be distributed and/or modified under the
conditions of the \LaTeX{} Project Public License, either version 1.3
of this license or (at your option) any later version.
The latest version of this license is in
  \url{http://www.latex-project.org/lppl.txt}
and version 1.3 or later is part of all distributions of \LaTeX{}
version 2005/12/01 or later.

This work has the LPPL maintenance status `maintained'.

The Current Maintainer of this work is Niklas Beisert.

This work consists of the files |README.txt|, |childdoc.ins| and |childdoc.dtx|
as well as the derived files |childdoc.def|, |cdocsamp.tex|
with |cdocsch1.tex|, |cdocsch2.tex|, |cdocspt3.tex|, |cdocspt4.tex|,
|cdocsdrf.tex|, |cdocsfn1.tex|, |cdocsfn2.tex|
as well as |childdoc.pdf|.

%%%%%%%%%%%%%%%%%%%%%%%%%%%%%%%%%%%%%%%%%%%%%%%%%%%%%%%%%%%%%%%%%%%%%%%%%%%%%%%%
\subsection{Files and Installation}

The package consists of the files:
%
\begin{center}
\begin{tabular}{ll}
    |README.txt|   & readme file \\
    |childdoc.ins| & installation file \\
    |childdoc.dtx| & source file \\
    |childdoc.def| & definition file \\
    |cdocsamp.tex| & sample main file \\
    |cdocsch1.tex| & sample include file \\
    |cdocsch2.tex| & sample include file \\
    |cdocspt3.tex| & sample part file \\
    |cdocspt4.tex| & sample part file \\
    |cdocsdrf.tex| & sample redirection file \\
    |cdocsfn1.tex| & sample redirection file \\
    |cdocsfn2.tex| & sample redirection file \\
    |childdoc.pdf| & manual
\end{tabular}
\end{center}
%
The distribution consists of the files
|README.txt|, |childdoc.ins| and |childdoc.dtx|.
%
\begin{itemize}
\item
Run (pdf)\LaTeX{} on |childdoc.dtx|
to compile the manual |childdoc.pdf| (this file).
\item
Run \LaTeX{} on |childdoc.ins| to create the definitions file |childdoc.def|
and the sample |cdocsamp.tex| with include files
|cdocsch1.tex|, |cdocsch2.tex|, |cdocspt3.tex|, |cdocspt4.tex|,
|cdocsdrf.tex|, |cdocsfn1.tex|, |cdocsfn2.tex|.
Then copy the file |childdoc.def| to an appropriate directory of your \LaTeX{}
distribution, e.g.\ \textit{texmf-root}|/tex/latex/childdoc|.
\end{itemize}

%%%%%%%%%%%%%%%%%%%%%%%%%%%%%%%%%%%%%%%%%%%%%%%%%%%%%%%%%%%%%%%%%%%%%%%%%%%%%%%%
\subsection{Related CTAN Packages}

There are several other packages which offer a similar functionality:
%
\begin{itemize}
\item
The packages
\href{http://ctan.org/pkg/docmute}{\textsf{docmute}},
\href{http://ctan.org/pkg/includex}{\textsf{includex}} and
\href{http://ctan.org/pkg/standalone}{\textsf{standalone}}
provide commands to include only the document body of
a child file thus allowing both files to be compiled individually.
\item
The packages \href{http://ctan.org/pkg/subdocs}{\textsf{subdocs}}
and \href{http://ctan.org/pkg/subfiles}{\textsf{subfiles}}
provide structures in which the main and child documents can be
encapsulated and allowing them to be compiled individually.
The inclusion mechanism is different from the conventional |\include|.
\item
The package \href{http://ctan.org/pkg/combine}{\textsf{combine}}
is an elaborate solution to combine several documents into one.
\end{itemize}
%
See also the CTAN topic \href{http://ctan.org/topic/subdocs}{\textsf{subdocs}}
for further related packages.
The present package differs from the above solutions in that
a document structure constructed with the conventional |\include| mechanism
just needs two extra commands at the top of every file
such that all constituent files can be compiled individually.

%%%%%%%%%%%%%%%%%%%%%%%%%%%%%%%%%%%%%%%%%%%%%%%%%%%%%%%%%%%%%%%%%%%%%%%%%%%%%%%%
%\subsection{Feature Suggestions}
%
%The following is a list of features which may be useful for future
%versions of this package:
%%
%\begin{itemize}
%\item
%\ldots
%\end{itemize}

%%%%%%%%%%%%%%%%%%%%%%%%%%%%%%%%%%%%%%%%%%%%%%%%%%%%%%%%%%%%%%%%%%%%%%%%%%%%%%%%
\subsection{Revision History}

%%%%%%%%%%%%%%%%%%%%%%%%%%%%%%%%%%%%%%%%
\paragraph{v2.0:} 2018/12/30

\begin{itemize}
\item
immediate forward processing
\item
added |\childdocby| mechanism
\item
manual restructured
\end{itemize}

%%%%%%%%%%%%%%%%%%%%%%%%%%%%%%%%%%%%%%%%
\paragraph{v1.6:} 2018/01/17

\begin{itemize}
\item
application for development of include files
\item
corrections to manual
\end{itemize}

%%%%%%%%%%%%%%%%%%%%%%%%%%%%%%%%%%%%%%%%
\paragraph{v1.5:} 2017/05/21

\begin{itemize}
\item
more complete structuring introduced
\item
|\childdocof| introduced
\item
|\childdoc| renamed to |\childdocmain|
\item
|\childredirect| renamed to |\childdocforward| and |\childdocforwardprefix|
and functionality expanded
\end{itemize}

%%%%%%%%%%%%%%%%%%%%%%%%%%%%%%%%%%%%%%%%
\paragraph{v1.0:} 2017/04/27

\begin{itemize}
\item
manual and install package
\item
first version published on CTAN
\end{itemize}

%%%%%%%%%%%%%%%%%%%%%%%%%%%%%%%%%%%%%%%%
\paragraph{v0.6:} 2017/04/26

\begin{itemize}
\item
redirection mechanism added
\end{itemize}

%%%%%%%%%%%%%%%%%%%%%%%%%%%%%%%%%%%%%%%%
\paragraph{v0.5:} 2017/04/26

\begin{itemize}
\item
functionality in definition file
\end{itemize}


%%%%%%%%%%%%%%%%%%%%%%%%%%%%%%%%%%%%%%%%%%%%%%%%%%%%%%%%%%%%%%%%%%%%%%%%%%%%%%%%
%%%%%%%%%%%%%%%%%%%%%%%%%%%%%%%%%%%%%%%%%%%%%%%%%%%%%%%%%%%%%%%%%%%%%%%%%%%%%%%%
%%%%%%%%%%%%%%%%%%%%%%%%%%%%%%%%%%%%%%%%%%%%%%%%%%%%%%%%%%%%%%%%%%%%%%%%%%%%%%%%
\appendix

\settowidth\MacroIndent{\rmfamily\scriptsize 000\ }

 \DocInput{childdoc.dtx}

\end{document}
%</driver>
% \fi
%
% %%%%%%%%%%%%%%%%%%%%%%%%%%%%%%%%%%%%%%%%%%%%%%%%%%%%%%%%%%%%%%%%%%%%%%%%%%%%%%
% %%%%%%%%%%%%%%%%%%%%%%%%%%%%%%%%%%%%%%%%%%%%%%%%%%%%%%%%%%%%%%%%%%%%%%%%%%%%%%
% \section{Sample}
%\iffalse
%<*samplemain>
%\fi
%
% The following presents a sample document
% with two chapters, two parts, a title page,
% a compile flag as well as three forwarding files to set the flag.
% It consists of eight |.tex| files:
% \begin{center}
% \begin{tabular}{ll}
% |cdocsamp.tex|&main file\\
% |cdocsch1.tex|&include file for chapter 1\\
% |cdocsch2.tex|&include file for chapter 2\\
% |cdocspt3.tex|&include file for part 3\\
% |cdocspt4.tex|&include file for part 4\\
% |cdocsdrf.tex|&forwarding file for main file in draft mode\\
% |cdocsfi1.tex|&forwarding file for final version of chapter 1\\
% |cdocsfi2.tex|&forwarding file for final version of chapter 2\\
% \end{tabular}
% \end{center}
% Each of the eight files can be compiled directly by the \LaTeX{} compiler.
%
% %%%%%%%%%%%%%%%%%%%%%%%%%%%%%%%%%%%%%%
% \paragraph{Main File.}
%
% The main file is called |cdocsamp.tex|.
%
% Load the \textsf{childdoc} definitions and
% declare the filename for the main document:
%    \begin{macrocode}
\input{childdoc.def}
\childdocmain{}
%    \end{macrocode}

% Optional override for |\version| flag:
%    \begin{macrocode}
%%\ifchilddoc\else\providecommand{\version}{draft}\fi
%    \end{macrocode}

% Define the default values for the |\version| flag
% (|final| for the main file and |draft| for childs):
%    \begin{macrocode}
\ifchilddoc
\providecommand{\version}{draft}
\else
\providecommand{\version}{final}
\fi
%    \end{macrocode}

% Load the standard document class:
%    \begin{macrocode}
\documentclass[12pt]{article}
%    \end{macrocode}

% Start the document body:
%    \begin{macrocode}
\begin{document}
%    \end{macrocode}

% Declare a title page.
% Print title, part of document being processed and version flag:
%    \begin{macrocode}
\addtocounter{page}{-1}
\begin{center}
{\LARGE\bfseries{}childdoc example\par}
\vspace{1cm}
\ifchilddoc
\ifchilddocmanual part\else chapter\fi:
`\childdocname' of `\childdocjob'\par
\else
main document: `\childdocjob'\par
\fi
version: \version\par
\end{center}
\newpage
%    \end{macrocode}

% Manually include selected file,
% otherwise process as usual:
%    \begin{macrocode}
\ifchilddocmanual
\section*{part `\childdocname'}
\input{\childdocname}
\else
%    \end{macrocode}

% Include the two chapters:
%    \begin{macrocode}
\include{cdocsch1}
\include{cdocsch2}
%    \end{macrocode}

% Include the two parts unless only chapters should be displayed:
%    \begin{macrocode}
\ifchilddoc\else
\section{part three}
\input{cdocspt3}
\section{part four}
\input{cdocspt4}
\fi
%    \end{macrocode}

% Process as usual until here:
%    \begin{macrocode}
\fi
%    \end{macrocode}

% End of document body:
%    \begin{macrocode}
\end{document}
%    \end{macrocode}
%\iffalse
%</samplemain>
%\fi
%
% %%%%%%%%%%%%%%%%%%%%%%%%%%%%%%%%%%%%%%
% \paragraph{Chapter Include Files.}
%
% The include files are called |cdocsch1.tex| and |cdocsch2.tex|.
%
%\iffalse
%<*samplechap1|samplechap2>
%\fi

% Optional override for |\version| flag:
%    \begin{macrocode}
%%\providecommand{\version}{final}
%    \end{macrocode}

% Include the main document:
%    \begin{macrocode}
\input{childdoc.def}
\childdocof{cdocsamp}
%    \end{macrocode}

%\iffalse
%</samplechap1|samplechap2>
%\fi
%
%\iffalse
%<*samplechap1>
%\fi
% Some text for chapter 1:
%    \begin{macrocode}
\section{one}
some text in chapter one
%    \end{macrocode}

%\iffalse
%</samplechap1>
%\fi
% Some text for chapter 2:
%\iffalse
%<*samplechap2>
%\fi
%    \begin{macrocode}
\section{two}
more text in chapter two
%    \end{macrocode}

%\iffalse
%</samplechap2>
%\fi
%
% %%%%%%%%%%%%%%%%%%%%%%%%%%%%%%%%%%%%%%
% \paragraph{Part Include Files.}
%
% The include files are called |cdocspt3.tex| and |cdocspt4.tex|.
%
%\iffalse
%<*samplepart3|samplepart4>
%\fi

% Optional override for |\version| flag:
%    \begin{macrocode}
%%\providecommand{\version}{final}
%    \end{macrocode}

% Include the main document:
%    \begin{macrocode}
\input{childdoc.def}
\childdocby{cdocsamp}
%    \end{macrocode}

%\iffalse
%</samplepart3|samplepart4>
%\fi
%
%\iffalse
%<*samplepart3>
%\fi
% Some text for part 3:
%    \begin{macrocode}
some text in part three
%    \end{macrocode}

%\iffalse
%</samplepart3>
%\fi
% Some text for part 4:
%\iffalse
%<*samplepart4>
%\fi
%    \begin{macrocode}
more text in part four
%    \end{macrocode}

%\iffalse
%</samplepart4>
%\fi
%
% %%%%%%%%%%%%%%%%%%%%%%%%%%%%%%%%%%%%%%
% \paragraph{Forwarding for a Complete Draft.}
%
% The following forwarding file |cdocsdrf.tex|
% compiles the main document in draft mode:
%\iffalse
%<*sampledraft>
%\fi
%    \begin{macrocode}
\def\version{draft}
\input{childdoc.def}
\childdocforward{cdocsamp}
%    \end{macrocode}

%\iffalse
%</sampledraft>
%\fi
%
% %%%%%%%%%%%%%%%%%%%%%%%%%%%%%%%%%%%%%%
% \paragraph{Forwarding for Final Version of the Chapters.}
%
% The following forwarding files |cdocsfn1.tex| and |cdocsfn2.tex|
% (with identical content)
% compile the final versions of the child documents
% |cdocsch1.tex| and |cdocsch2.tex|, respectively:
%\iffalse
%<*samplefinal>
%\fi
%    \begin{macrocode}
\def\version{final}
\input{childdoc.def}
\childdocforwardprefix[cdocsamp]{cdocsfn}{cdocsch}
%    \end{macrocode}

%\iffalse
%</samplefinal>
%\fi
%
% %%%%%%%%%%%%%%%%%%%%%%%%%%%%%%%%%%%%%%
% \paragraph{Command Line Processing.}
%
% The following three command lines generate the output files
% |cdocscld|, |cdocscl1| and |cdocscl2|
% which should be identical to
% |cdocsdrf|, |cdocsch1| and |cdocsfn2|, respectively:
% \begin{center}
% \begin{tabular}{l}
% |latex -jobname cdocscld \|\\
% |  "\def\version{draft}\input{childdoc.def}\childdocforward{cdocsamp}"|\\
% |latex -jobname cdocscl1 \|\\
% |  "\input{childdoc.def}\childdocforward[cdocsamp]{cdocsch1}"|\\
% |latex -jobname cdocscl2 \|\\
% |  "\def\version{final}\input{childdoc.def}\childdocforward{cdocsch2}"|
% \end{tabular}
% \end{center}
% Note that the trailing backslash on each first line
% merely continues the input to the second line
% (for convenient cut ant paste).
% Furthermore, the command |latex| can be replaced by any
% of its alternative versions such as |pdflatex|.
%
% %%%%%%%%%%%%%%%%%%%%%%%%%%%%%%%%%%%%%%%%%%%%%%%%%%%%%%%%%%%%%%%%%%%%%%%%%%%%%%
% %%%%%%%%%%%%%%%%%%%%%%%%%%%%%%%%%%%%%%%%%%%%%%%%%%%%%%%%%%%%%%%%%%%%%%%%%%%%%%
% \section{Implementation}
%\iffalse
%<*package>
%\fi
%
% This section describes the definitions file |childdoc.def|.

% The definitions cannot be loaded using |\usepackage| or |\RequirePackage|
% which has a mechanism to prevent loading a style file more than once.
% When loading the definitions by means of |\input|
% multiple instances have to be prevented manually:
%\iffalse
%This code needs to be before the `\ProvidesFile' directive
%which is defined at the beginning of this file.
%Therefore it is also placed there and commented out here.
%</package>
%<*discard>
%\fi
%    \begin{macrocode}
\ifdefined\childdocmain\endinput\fi
%    \end{macrocode}
%\iffalse
%</discard>
%<*package>
%\fi
%
% \macro{\ifchilddoc}
% \macro{\ifchilddocmanual}
% The conditional |\ifchilddoc| tells whether a
% child (true) or main (false) document is being compiled.
% The conditional |\ifchilddocmanual| tells whether
% the |\includeonly| mechanism is used (false) or
% the selection of child files must be performed manually (true).
% The definitions initialise to false:
%    \begin{macrocode}
\newif\ifchilddoc
\newif\ifchilddocmanual
%    \end{macrocode}

% \macro{\childdocname}
% \macro{\childdocjob}
% The macro |\childdocname| stores the name of the main document
% to be compiled. The macro |\childdocjob| stores the name of
% the document on which the \LaTeX{} compiler was originally invoked.
% The content of |\jobname| cannot be compared
% to filenames specified in the source due to different catcodes.
% The following code rescans |\jobname|, stores the result
% in |\childdocname| and saves a copy in |\childdocjob|:
%    \begin{macrocode}
\edef\childdocname{\scantokens\expandafter{\jobname\noexpand}}
\let\childdocjob\childdocname
%    \end{macrocode}

% \macro{\childdocdisable}
% The macro |\childdocdisable| prevents the main file
% from being processed more than once.
% At this stage, the main document command |\childdocmain|
% is assumed to be called once again where it should do nothing.
% Any subsequent call to it should prevent
% a secondary processing of the main document
% It overwrites the forwarding commands
% |\childdocof| and |\childdocforward|
% with empty macros to prevent further inclusions of the main document:
%    \begin{macrocode}
\newcommand{\childdocdisable}
{
  \renewcommand{\childdocmain}[1]{\renewcommand{\childdocmain}[1]{\endinput}}
  \renewcommand{\childdocof}[1]{}
  \renewcommand{\childdocby}[2][]{}
  \renewcommand{\childdocforward}[2][]{}
  \renewcommand{\childdocdisable}{}
}
%    \end{macrocode}

% \macro{\childdocmain}
% The macro |\childdocmain| is to be called at the top of the main file
% with nothing or the main filename (without extension) as argument.
% First, it breaks loops.
% If the argument is not empty and does not match |\childdocname|
% (which is set by the first inclusion of |childdoc.def|),
% |\ifchilddoc| is set to true, |\includeonly| is applied to the child file
% and |\jobname| is set to the main file
% (for proper handling of |.aux| files):
%    \begin{macrocode}
\newcommand{\childdocmain}[1]
{
  \childdocdisable\childdocmain{}
  \if?#1?\else
    \begingroup
      \def\childdoctmp{#1}
      \ifx\childdoctmp\childdocname
        \def\childdoctmp{}
      \else
        \def\childdoctmp
        {
          \childdoctrue
          \includeonly{\childdocname}
          \def\childdocjob{#1}
          \def\jobname{#1}
        }
      \fi
      \expandafter
    \endgroup
    \childdoctmp
  \fi
}
%    \end{macrocode}

% \macro{\childdocof}
% The command |\childdocof| redirects
% compilation to the main file |#1|.
%    \begin{macrocode}
\newcommand{\childdocof}[1]
{
  \childdocdisable
  \childdoctrue
  \includeonly{\childdocname}
  \def\jobname{#1}
  \def\childdocjob{#1}
  \input{#1}
}
%    \end{macrocode}

% \macro{\childdocby}
% The command |\childdocby| ....
%    \begin{macrocode}
\newcommand{\childdocby}[2][]
{
  \childdocdisable
  \childdoctrue
  \childdocmanualtrue
  \if?#1?\else
    \def\jobname{#2}
  \fi
  \def\childdocjob{#2}
  \input{#2}
  \endinput
}
%    \end{macrocode}

% \macro{\childdocforward}
% The command |\childdocforward| redirects
% compilation to the main file or
% (if the optional argument is given) a child file.
% Parameters are set as if the main file
% or a child file starting with |\childdocof| was compiled.
% Then compilation is handed over to the main file:
%    \begin{macrocode}
\newcommand{\childdocforward}[2][]
{
  \begingroup
    \if?#1?
      \def\childdoctmp
      {
        \def\childdocname{#2}
        \def\childdocjob{#2}
        \def\jobname{#2}
        \input{#2}
        \endinput
      }
    \else
      \def\childdoctmp
      {
        \childdocdisable
        \def\childdocname{#2}
        \childdoctrue
        \includeonly{#2}
        \def\childdocjob{#1}
        \def\jobname{#1}
        \input{#1}
        \endinput
      }
    \fi
    \expandafter
  \endgroup
  \childdoctmp
}
%    \end{macrocode}

% \macro{\childdocforwardprefix}
% The command |\childdocforwardprefix| redirects
% compilation to the main or a child file by means of a pattern.
% The prefix |#1| in the current filename is replaced by |#2|
% and the suffix of the current filename is kept
% (it is assumed that the filename does not contain the substring `|~~~|'
% which is used as a delimiter).
% Compilation is handed over to the new file by |\childdocforward|:
%    \begin{macrocode}
\newcommand{\childdocforwardprefix}[3][]
{
  \begingroup
    \def\childdocextract #2##1~~~{\def\childdoctmp{\childdocforward[#1]{#3##1}}}
    \expandafter\childdocextract\childdocname~~~
    \expandafter
  \endgroup
  \childdoctmp
}
%    \end{macrocode}

% \macro{\childdoc}
% The deprecated macro |\childdoc| is a legacy version of |\childdocmain|:
%    \begin{macrocode}
\newcommand{\childdoc}{\childdocmain}
%    \end{macrocode}

% \macro{\childdocredirect}
% The deprecated macro |\childdocredirect| is a legacy version
% of |\childdocforward| and |\childdocforwardprefix|:
%    \begin{macrocode}
\newcommand{\childdocredirect}[2][]
{
  \begingroup
    \if?#1?
      \def\childdoctmp{\childdocforward{#2}}
    \else
      \def\childdoctmp{\childdocforwardprefix{#1}{#2}}
    \fi
    \expandafter
  \endgroup
  \childdoctmp
}
%    \end{macrocode}

%\iffalse
%</package>
%\fi
%
\endinput

\childdocmain{}
%    \end{macrocode}

% Optional override for |\version| flag:
%    \begin{macrocode}
%%\ifchilddoc\else\providecommand{\version}{draft}\fi
%    \end{macrocode}

% Define the default values for the |\version| flag
% (|final| for the main file and |draft| for childs):
%    \begin{macrocode}
\ifchilddoc
\providecommand{\version}{draft}
\else
\providecommand{\version}{final}
\fi
%    \end{macrocode}

% Load the standard document class:
%    \begin{macrocode}
\documentclass[12pt]{article}
%    \end{macrocode}

% Start the document body:
%    \begin{macrocode}
\begin{document}
%    \end{macrocode}

% Declare a title page.
% Print title, part of document being processed and version flag:
%    \begin{macrocode}
\addtocounter{page}{-1}
\begin{center}
{\LARGE\bfseries{}childdoc example\par}
\vspace{1cm}
\ifchilddoc
\ifchilddocmanual part\else chapter\fi:
`\childdocname' of `\childdocjob'\par
\else
main document: `\childdocjob'\par
\fi
version: \version\par
\end{center}
\newpage
%    \end{macrocode}

% Manually include selected file,
% otherwise process as usual:
%    \begin{macrocode}
\ifchilddocmanual
\section*{part `\childdocname'}
\input{\childdocname}
\else
%    \end{macrocode}

% Include the two chapters:
%    \begin{macrocode}
\include{cdocsch1}
\include{cdocsch2}
%    \end{macrocode}

% Include the two parts unless only chapters should be displayed:
%    \begin{macrocode}
\ifchilddoc\else
\section{part three}
\input{cdocspt3}
\section{part four}
\input{cdocspt4}
\fi
%    \end{macrocode}

% Process as usual until here:
%    \begin{macrocode}
\fi
%    \end{macrocode}

% End of document body:
%    \begin{macrocode}
\end{document}
%    \end{macrocode}
%\iffalse
%</samplemain>
%\fi
%
% %%%%%%%%%%%%%%%%%%%%%%%%%%%%%%%%%%%%%%
% \paragraph{Chapter Include Files.}
%
% The include files are called |cdocsch1.tex| and |cdocsch2.tex|.
%
%\iffalse
%<*samplechap1|samplechap2>
%\fi

% Optional override for |\version| flag:
%    \begin{macrocode}
%%\providecommand{\version}{final}
%    \end{macrocode}

% Include the main document:
%    \begin{macrocode}
% \iffalse
%
% childdoc.dtx Copyright (C) 2017-2018 Niklas Beisert
%
% This work may be distributed and/or modified under the
% conditions of the LaTeX Project Public License, either version 1.3
% of this license or (at your option) any later version.
% The latest version of this license is in
%   http://www.latex-project.org/lppl.txt
% and version 1.3 or later is part of all distributions of LaTeX
% version 2005/12/01 or later.
%
% This work has the LPPL maintenance status `maintained'.
%
% The Current Maintainer of this work is Niklas Beisert.
%
% This work consists of the files childdoc.dtx and childdoc.ins
% and the derived files childdoc.def and cdocsamp.tex with
% cdocsch1.tex, cdocsch2.tex, cdocsdrf.tex, cdocsfn1.tex, cdocsfn2.tex.
%
%<package>\ifdefined\childdocmain\endinput\fi
%<package>\ProvidesFile{childdoc.def}[2018/12/30 v2.0 child document driver]
%<samplemain>\ProvidesFile{cdocsamp.tex}[2018/12/30 v2.0 sample for childdoc]
%<*driver>
%\ProvidesFile{childdoc.drv}[2018/12/30 v2.0 childdoc reference manual file]
\PassOptionsToClass{10pt,a4paper}{article}
\documentclass{ltxdoc}

\usepackage[margin=35mm]{geometry}
\usepackage{hyperref}
\usepackage{hyperxmp}
\usepackage[usenames]{color}

\hypersetup{colorlinks=true}
\hypersetup{pdfstartview=FitH}
\hypersetup{pdfpagemode=UseNone}
\hypersetup{pdfsource={}}
\hypersetup{pdflang={en-UK}}
\hypersetup{pdfcopyright={Copyright 2017-2018 Niklas Beisert.
  This work may be distributed and/or modified under the
  conditions of the LaTeX Project Public License, either version 1.3
  of this license or (at your option) any later version.}}
\hypersetup{pdflicenseurl={http://www.latex-project.org/lppl.txt}}
\hypersetup{pdfcontactaddress={ETH Zurich, ITP, HIT K,
  Wolfgang-Pauli-Strasse 27}}
\hypersetup{pdfcontactpostcode={8093}}
\hypersetup{pdfcontactcity={Zurich}}
\hypersetup{pdfcontactcountry={Switzerland}}
\hypersetup{pdfcontactemail={nbeisert@itp.phys.ethz.ch}}
\hypersetup{pdfcontacturl={http://people.phys.ethz.ch/\xmptilde nbeisert/}}

\newcommand{\secref}[1]{\hyperref[#1]{section \ref*{#1}}}

\parskip1ex
\parindent0pt
\let\olditemize\itemize
\def\itemize{\olditemize\parskip0pt}

\begin{document}

\title{The \textsf{childdoc} Package}
\hypersetup{pdftitle={The childdoc Package}}
\author{Niklas Beisert\\[2ex]
  Institut f\"ur Theoretische Physik\\
  Eidgen\"ossische Technische Hochschule Z\"urich\\
  Wolfgang-Pauli-Strasse 27, 8093 Z\"urich, Switzerland\\[1ex]
  \href{mailto:nbeisert@itp.phys.ethz.ch}
  {\texttt{nbeisert@itp.phys.ethz.ch}}}
\hypersetup{pdfauthor={Niklas Beisert}}
\hypersetup{pdfsubject={Manual for the LaTeX2e Package childdoc}}
\date{30 December 2018, \textsf{v2.0}}
\maketitle

\begin{abstract}\noindent
\textsf{childdoc} is a \LaTeXe{} package
that enables the direct compilation
of document sections included by |\include|
to individual files.
\end{abstract}

\begingroup
\parskip0ex
\tableofcontents
\endgroup

%%%%%%%%%%%%%%%%%%%%%%%%%%%%%%%%%%%%%%%%%%%%%%%%%%%%%%%%%%%%%%%%%%%%%%%%%%%%%%%%
%%%%%%%%%%%%%%%%%%%%%%%%%%%%%%%%%%%%%%%%%%%%%%%%%%%%%%%%%%%%%%%%%%%%%%%%%%%%%%%%
\section{Introduction}

\LaTeX{} provides a mechanism to structure a large document (such as a book)
into a main file and several child files (containing the chapters)
using the |\include| command.
This mechanism is beneficial for documents
which span hundreds of pages in order to
make the source file(s) more manageable.
Moreover, compilation can be restricted to
selected child files by means of the |\includeonly| command.
The latter feature can be used to reduce the compilation time while editing
(this was significantly more useful in the earlier days of \LaTeX{})
or to generate a smaller document which is easier to navigate.
Another application of |\includeonly| is to generate
documents consisting of selected parts of the complete document.

However, there are a few drawbacks of the plain |\include| mechanism:
\begin{itemize}
\item
The child files cannot be compiled on their own,
they can only be compiled via the main file.
A naive editing environment
(such as a text editor with an option
to have the current file processed by \LaTeX)
may require one to switch to the main file before compiling;
attempting to compile the child file produces errors.
\item
The main file must be modified (each time)
to adjust the |\includeonly| command
to the present needs. This easily leaves the main file in a messy state.
\item
The generated document will always carry the filename
of the main document. This is inconvenient if
several child files are to be compiled and
to be kept for distribution.
\end{itemize}

The present package provides a simple interface
to make child files individually compilable by \LaTeX{}.
Compiling a child file then has the same effect as compiling
the main file with an |\includeonly| command
to select the appropriate child.
Moreover the generated document will carry the name of the child
rather than the main file.
This resolves all three above issues.

This feature is meant to make the editing of books,
thesis documents and lecture notes somewhat more convenient.
However, the package can also be used efficiently for
composing a series of documents (such as exercise sheets)
which are typically distributed individually.
It then assists the author in generating the individual documents
(potentially in different versions)
as well as a document containing the collected series.
Another application is in developing style files
or other kinds of included material
where compilation of the style file could redirect
to a sample or test file.

%%%%%%%%%%%%%%%%%%%%%%%%%%%%%%%%%%%%%%%%%%%%%%%%%%%%%%%%%%%%%%%%%%%%%%%%%%%%%%%%
%%%%%%%%%%%%%%%%%%%%%%%%%%%%%%%%%%%%%%%%%%%%%%%%%%%%%%%%%%%%%%%%%%%%%%%%%%%%%%%%
\section{Usage}

First of all, the package \textsf{childdoc} is \emph{not} a standard
\LaTeXe{} |.sty| style file! Therefore it needs to be invoked in
a non-standard way.

%%%%%%%%%%%%%%%%%%%%%%%%%%%%%%%%%%%%%%%%%%%%%%%%%%%%%%%%%%%%%%%%%%%%%%%%%%%%%%%%
\subsection{Included Files}
\label{sec:include}

%%%%%%%%%%%%%%%%%%%%%%%%%%%%%%%%%%%%%%%%
\DescribeMacro{\childdocmain}
To use the package, add the commands
\begin{center}
\begin{tabular}{l}
|\input{childdoc.def}|\\
|\childdocmain{}|\\
\end{tabular}
\end{center}
at the very top of the main \LaTeX{} file,
in particular \emph{before} the |\documentclass| statement!
The argument of |\childdocmain| should be left empty
(but it must be present).

%%%%%%%%%%%%%%%%%%%%%%%%%%%%%%%%%%%%%%%%
\DescribeMacro{\childdocof}
Furthermore, add the commands
\begin{center}
\begin{tabular}{l}
|\input{childdoc.def}|\\
|\childdocof{|\textit{main}|}|\\
\end{tabular}
\end{center}
at the top of every child file \textit{child}
which is included by |\include{|\textit{child}|}|
from within the main file
(or at least for those files to be compiled individually).
The argument \textit{main} must be the filename of the main file.

There are a couple of
considerations in setting up the main and child documents:

%%%%%%%%%%%%%%%%%%%%%%%%%%%%%%%%%%%%%%%%
\paragraph{Restrictions.}

Please note the following restrictions:
\begin{itemize}
\item
|\childdocmain| must be called with one argument \textit{main}
to ensure compatibility with earlier version of the package.
It must either be empty (|\childdocmain{}|)
or precisely match the filename of the main file in which it is specified.
See \secref{sec:detection} for further information.
\item
The filename \textit{main} must be specified without the |.tex| extension.
\item
The filename \textit{main} is case sensitive
(even in case-insensitive file systems)
due to internal string comparison.
\item
The argument \textit{main} should be fully expanded, it cannot be a macro.
\item
Subdirectories and special characters should be avoided in filenames.
\item
The command |\childdocmain{|\textit{main}|}| must be followed by a whitespace.
It should not be followed immediately by another command
or by a comment mark `|%|'.
This is because the \TeX{} parser reads the token immediately following
the argument of |\childdocmain| and puts it
at the beginning of every child section;
however, a white\-space is ignored.
\end{itemize}

%%%%%%%%%%%%%%%%%%%%%%%%%%%%%%%%%%%%%%%%
\paragraph{Content of Main File.}

It is advisable to place all content in the child files included by |\include|.
Any output contained in the main file will appear in all child documents
unless suppressed manually;
it cannot be suppressed automatically by the |\includeonly| directive
and thus should normally be avoided.
A method to include some content in the main file
by means of conditional processing is described in \secref{sec:conditional}.

%%%%%%%%%%%%%%%%%%%%%%%%%%%%%%%%%%%%%%%%
\paragraph{Page Numbering.}

When only a part of the document is compiled,
the appropriate numbering of pages
(as well as other status parameters)
is determined from the |.aux| files.
The latter contain information from previous passes.
However this information needs to propagate through
all intermediate child documents.
Therefore the page numbering in child documents may well
be inconsistent until the complete document is compiled at least once.

A useful (if unconventional) way to always ensure a consistent
page numbering is to restart the numbering in each child document
and denote the pages by `\textit{child}|.|\textit{page}'
where \textit{child} represents the chapter/section number of the child file.
This can be achieved by the command
|\numberwithin{page}{|\textit{child}|}|
of the \textsf{amsmath} package
where \textit{child} can be |chapter| or |section|
depending on the chosen structuring.
Alternatively, one can modify the macro |\thepage| appropriately
and reset the counter |page| at the start of each child file.

%%%%%%%%%%%%%%%%%%%%%%%%%%%%%%%%%%%%%%%%%%%%%%%%%%%%%%%%%%%%%%%%%%%%%%%%%%%%%%%%
\subsection{Conditional Processing}
\label{sec:conditional}

The package provides a mechanism to compile different versions
of a document. To customise the versions further some conditional processing
can come in handy to distinguish which version is being compiled.
The package provides two macros to describe the compilation context:

%%%%%%%%%%%%%%%%%%%%%%%%%%%%%%%%%%%%%%%%
\DescribeMacro{\ifchilddoc}
The conditional |\ifchilddoc| distinguishes between the compilation of
child documents and the main document:
%
\begin{center}
|\ifchilddoc |\textit{child-code}| |[|\||else |\textit{main-code}]| \||fi|
\end{center}

%%%%%%%%%%%%%%%%%%%%%%%%%%%%%%%%%%%%%%%%
\DescribeMacro{\childdocname}
\DescribeMacro{\childdocjob}
The macro |\childdocname| contains the filename (without extension)
of the main or child file being processed.
Note that |\childdocjob| will always contain the name of the main file.

%%%%%%%%%%%%%%%%%%%%%%%%%%%%%%%%%%%%%%%%
\paragraph{Title Page.}

Conditional processing can be used to include a title or banner page
in the main document when proper precautions are taken.
Importantly, the code in the main file should ensure that the page counter
(as well as other status parameters which are stored in the |.aux| files)
takes the same value after the conditional processing.
Otherwise the page numbers may take divergent values
depending on which part is compiled.

For example, a title page could be declared by:
%
\begin{center}
\begin{tabular}{l}
|\ifchilddoc\||else|\\
|\addtocounter{page}{-1}|\\
\textit{code for title page}\\
|\newpage|\\
|\||fi|
\end{tabular}
\end{center}
%
A banner page for the child documents can be generated by:
%
\begin{center}
\begin{tabular}{l}
|\ifchilddoc|\\
|\addtocounter{page}{-1}|\\
\textit{code for banner page}\\
|\newpage|\\
|\||fi|
\end{tabular}
\end{center}
%
Here one could write a message such as:
\begin{center}
|This is the part \childdocname{} of \childdocjob{}.|
\end{center}

%%%%%%%%%%%%%%%%%%%%%%%%%%%%%%%%%%%%%%%%%%%%%%%%%%%%%%%%%%%%%%%%%%%%%%%%%%%%%%%%
\subsection{Flags}
\label{sec:flags}

The package makes it easy to generate different versions
of the main or child documents.
To this end compilation flags can be defined
and assigned different default values.
They will be particularly useful in conjunction
with the forwarding mechanism described in \secref{sec:forward}.

For example, it may be useful to have a flag |\version|
which can be set to |draft| or |final|.
The document source will contain some conditional code
depending on the value of |\version|.
Suppose further, the flag should default to |final| for the main file
and to |draft| for child files
which is a natural assignment for editing the document.
This is achieved by placing the following code
in the preamble of the main document
(below the |\childdocmain| directive):
%
\begin{center}
\begin{tabular}{l}
|\ifchilddoc|\\
|\providecommand{\version}{draft}|\\
|\||else|\\
|\providecommand{\version}{final}|\\
|\||fi|
\end{tabular}
\end{center}
%
The definition by |\providecommand| makes sure
that previous definitions are not overwritten.
Further statements |\providecommand{\version}{...}|
can thus be added before the above code to override it.

For the main file, one might add a line
(between |\childdocmain| and the above block)
%
\begin{center}
|%\ifchilddoc\||else\providecommand{\version}{draft}\||fi|
\end{center}
%
which can be uncommented to produce a draft version.
Likewise one can add a line to the very top of a child file
(above the |\childdocof{|\textit{main}|}| directive)
%
\begin{center}
|%\providecommand{\version}{final}|
\end{center}
%
which can be uncommented to produce the final version of this child document.

%%%%%%%%%%%%%%%%%%%%%%%%%%%%%%%%%%%%%%%%%%%%%%%%%%%%%%%%%%%%%%%%%%%%%%%%%%%%%%%%
\subsection{Forwarding}
\label{sec:forward}

Different versions of the main or child documents
using compilation flags as described in \secref{sec:flags}
can be (permanently) stored in different files
for convenient compilation, viewing and distribution.
To this end, the package defines a command
to pass on compilation to a different file:

%%%%%%%%%%%%%%%%%%%%%%%%%%%%%%%%%%%%%%%%
\DescribeMacro{\childdocforward}
The command |\childdocforward| redirects processing to
another source file:
%
\begin{center}
\begin{tabular}{l}
|\input{childdoc.def}|\\
|\childdocforward[|\textit{main}|]{|\textit{dest}|}|\\
\end{tabular}
\end{center}
%
The argument \textit{dest} is the destination file
(without extension).
It should be the main file or one of the child files.
Note that further \textsf{childdoc} directives
such as |\childdocof| and |\childdocforward|
in the indicated file will be processed in this form.
The optional argument \textit{main}
passes on directly to the main file \textit{main}
while pretending to compile the child \textit{dest}.
This form behaves as if \textit{dest}
issues |\childdocof{|\textit{main}|}| right away,
and no further \textsf{childdoc} directives will be processed.

%%%%%%%%%%%%%%%%%%%%%%%%%%%%%%%%%%%%%%%%
\DescribeMacro{\...prefix}
In the alternative form |\childdocforwardprefix|,
%
\begin{center}
\begin{tabular}{l}
|\input{childdoc.def}|\\
|\childdocforwardprefix[|\textit{main}|]{|\textit{prefix}|}{|\textit{dest}|}|
\end{tabular}
\end{center}
%
the destination file is determined by a pattern
depending on the current file:
To make this work, the current file must be called
`{\textit{prefix}\hspace{0.2em}\textit{suffix}}'
with \textit{prefix} matching precisely the argument.
Processing is then passed on to the file
`{\textit{dest}\hspace{0.2em}\textit{suffix}}'.
Surely, the same effect is achieved by
directly specifying the
argument `{\textit{dest}\hspace{0.2em}\textit{suffix}}'
in the first form.
However, that requires to set up a different file
for each child. With the alternative form of the command
all these files can have exactly the same content
which simplifies setting them up and maintaining them.

For example, the following file |draft.tex|
with a compilation flag |\version| as described in \secref{sec:flags}
compiles the main document as a draft:
%
\begin{center}
\begin{tabular}{l}
|\def\version{draft}|\\
|\input{childdoc.def}|\\
|\childdocforward{|\textit{main}|}|
\end{tabular}
\end{center}
%
Likewise, the following files |final|\textit{nn}|.tex|
compile the final version of the child document
|child|\textit{nn}|.tex|:
%
\begin{center}
\begin{tabular}{l}
|\def\version{final}|\\
|\input{childdoc.def}|\\
|\childdocforwardprefix{final}{child}|
\end{tabular}
\end{center}
%

Note that when several versions of a main file and/or of each child file
are to be generated, it may be convenient to set up a |Makefile| or
shell script to automatise the process.

%%%%%%%%%%%%%%%%%%%%%%%%%%%%%%%%%%%%%%%%%%%%%%%%%%%%%%%%%%%%%%%%%%%%%%%%%%%%%%%%
\subsection{Command Line Processing}
\label{sec:commandline}

The effect of redirection files can also be achieved by invoking
the \LaTeX{} compiler with a more elaborate command line.
Most conveniently this should be done as part
of a shell script or a |Makefile|.

When using \textsf{childdoc} in the main file, the following
command lines effectively perform a redirection
(note that depending on the shell being used,
backslashes may have to be doubled: `|\|' $\to$ `|\\|'):
%
\begin{center}
|... -jobname "|\textit{target}|" |\\|"|[\textit{flags}]%
|\input{childdoc.def}\childdocforward[|\textit{main}|]{|\textit{dest}|}"|
\end{center}
%
Here \textit{target} is the name of the output file,
\textit{main} is the name of the main file
and \textit{dest} is the name of the main or child file to be processed
(all filenames without extensions).
The optional argument \textit{main} can be omitted
if \textit{main} matches \textit{dest}.
Optionally, compilation \textit{flags} can be defined via |\def| commands.
This command line makes the \TeX{} engine believe
it is compiling the file \textit{target}
whose content is specified as the latter parameter.
The provided code then forwards the processing to
\textit{main} or \textit{dest} as described in \secref{sec:forward}.

%%%%%%%%%%%%%%%%%%%%%%%%%%%%%%%%%%%%%%%%%%%%%%%%%%%%%%%%%%%%%%%%%%%%%%%%%%%%%%%%
\subsection{Include by Input}
\label{sec:input}

Including child documents by |\include| has some restrictions by design.
Most notably, the content of a child document always occupies
its own set of pages; pages cannot be shared between child documents.
Usually, this behaviour makes perfect sense
because each child document contain an essential part of the document.
However, in some situations it may be desirable to compose
a document from a collection of parts
without having mandatory page breaks between then.
For this case, the package
provides a mechanism to include parts
by |\input| which can also be processed individually.
However, by construction this mechanism
requires manual handling of the content to be output.

%%%%%%%%%%%%%%%%%%%%%%%%%%%%%%%%%%%%%%%%
\DescribeMacro{\ifchilddocmanual}
The main file should be prepared as usual, see \secref{sec:include}.
However, the document body must make a distinction
between processing of an individual part and of the main document, e.g.:
%
\begin{center}
\begin{tabular}{l}
|\ifchilddocmanual|\\
|\input{\childdocname}|\\
|\||else|\\
\textit{document body with }|\input{|\textit{part}|}|\\
|\||fi|
\end{tabular}
\end{center}
%
The conditional |\ifchilddocmanual| is true whenever
a part to be included by |\input| is being compiled,
and the name of the part is stored in |\childdocname|.

%%%%%%%%%%%%%%%%%%%%%%%%%%%%%%%%%%%%%%%%
\DescribeMacro{\childdocby}
Each part to be included by |\input| should start with:
%
\begin{center}
\begin{tabular}{l}
|\input{childdoc.def}|\\
|\childdocby{|\textit{main}|}|\\
\end{tabular}
\end{center}
%
The directive |\childdocby| is similar to |\childdocof|
described in \secref{sec:include},
but the subsequent selection of content must be done manually.
To that end, both |\ifchilddoc| and |\ifchilddocmanual|
will be true upon processing of a part,
and the name of the part is stored in |\childdocname|.
Note that |\jobname| will be set to the filename of the current part
so that each part receives an individual |.aux| file
that does not interfere with the |.aux| file(s) of the main document.
This behaviour can be altered by the alternative form
|\childdocby[*]{|\textit{main}|}| (with a non-empty optional argument)
which uses the |.aux| file of the main document
by setting |\jobname| to \textit{main}.

%%%%%%%%%%%%%%%%%%%%%%%%%%%%%%%%%%%%%%%%%%%%%%%%%%%%%%%%%%%%%%%%%%%%%%%%%%%%%%%%
\subsection{Driver Development}
\label{sec:driver}

The \textsf{childdoc} mechanism can also be use for the development
of definition files such as \LaTeX{} styles or classes.
This case differs from the above setup with multiple parts
included by |\include| in that no |\includeonly| should be invoked.
This can be achieved by starting the include file
(before |\ProvidesPackage|) with:
%
\begin{center}
\begin{tabular}{l}
|\input{childdoc.def}|\\
|\childdocforward{|\textit{main}|}|\\
\end{tabular}
\end{center}
%
or alternatively with:
%
\begin{center}
\begin{tabular}{l}
|\input{childdoc.def}|\\
|\childdocby{|\textit{main}|}|\\
\end{tabular}
\end{center}
%
Both forms have slightly different effects as described above.
The main file is prepared as usual, see \secref{sec:include}.

%%%%%%%%%%%%%%%%%%%%%%%%%%%%%%%%%%%%%%%%%%%%%%%%%%%%%%%%%%%%%%%%%%%%%%%%%%%%%%%%
\subsection{Legacy Detection}
\label{sec:detection}

The directive |\childdocmain| in the main file can detect
whether the complete document or merely a child is to be compiled
even without using the directive |\childdocof|.
This method is deprecated because it is less robust
and there is no compelling reason to use it;
it is merely provided for backward compatibility
and it may be removed in future versions.

If the detection mechanism is to be used,
it is mandatory to correctly specify
the filename of the main file as the argument of |\childdocmain|:
%
\begin{center}
\begin{tabular}{l}
|\input{childdoc.def}|\\
|\childdocmain{|\textit{main}|}|\\
\end{tabular}
\end{center}
%
If |\jobname| does not match the argument \textit{main} of |\childdocmain|,
it is assumed that |\jobname| points to the child file to be compiled.
When using |\childdocmain| with the main file specified as argument,
it suffices to start a child file
with just |\input{|\textit{main}|}|
without loading of the package and using |\childdocof|.
If instead all processing is done
with the appropriate \textsf{childdoc} directives,
the argument of \textit{main} of |\childdocmain| can be empty.

An alternative version of the command line processing described
in \secref{sec:commandline} using the detection mechanism reads:
%
\begin{center}
|... -jobname "|\textit{target}|" "|[\textit{flags}]%
[|\def\jobname{|\textit{dest}|}|]|\input{|\textit{main}|}"|
\end{center}

%%%%%%%%%%%%%%%%%%%%%%%%%%%%%%%%%%%%%%%%%%%%%%%%%%%%%%%%%%%%%%%%%%%%%%%%%%%%%%%%
\subsection{Manual Code}
\label{sec:manual}

In case one cannot be certain whether the definitions file |childdoc.def|
is installed on the target \TeX{} distribution
and one prefers not to ship it,
it is conceivable to paste a few relevant commands into the sources.

To that end, drop all statements |\input{childdoc.def}|
and perform the replacements as outlined below.
Instead of |\childdocmain{|\textit{main}|}| add the following code
to the top of the main file:
%
\begin{center}
\begin{tabular}{l}
|\||ifdefined\childdocname\endinput\||fi\newif\ifchilddoc|\\
|\edef\childdocname{\scantokens\expandafter{\jobname\noexpand}}|\\
|\def\childdocmain{|\textit{main}|}\||ifx\childdocmain\childdocname\||else|\\
|\childdoctrue\includeonly{\childdocname}\let\jobname\childdocmain\||fi|\\
\end{tabular}
\end{center}
%
Instead of |\childdocof{|\textit{main}|}| just include the main file
at the top of each child file:
%
\begin{center}
|\input{|\textit{main}|}|
\end{center}
%
A simple redirection |\childdocforward{|\textit{dest}|}| is achieved by:
%
\begin{center}
|\def\jobname{|\textit{dest}|}\input{\jobname}|
\end{center}
%
The redirection with prefix
|\childdocforwardprefix[|\textit{prefix}|]{|\textit{dest}|}|
is accomplished by:
%
\begin{center}
\begin{tabular}{l}
|{\edef\jobname{\scantokens\expandafter{\jobname\noexpand}}|\\
|\def\redirectjob |\textit{prefix}|#1~~~{\gdef\jobname{|\textit{dest}|#1}}|\\
|\expandafter\redirectjob\jobname~~~}\input{\jobname}|
\end{tabular}
\end{center}

In an alternative approach,
child documents can be compiled by a specific command line
without additional code or specific definitions:
%
\begin{center}
|... -jobname "|\textit{target}|" "|[\textit{flags}]%
|\includeonly{|\textit{dest}|}\input{|\textit{main}|}"|
\end{center}
%

%%%%%%%%%%%%%%%%%%%%%%%%%%%%%%%%%%%%%%%%%%%%%%%%%%%%%%%%%%%%%%%%%%%%%%%%%%%%%%%%
%%%%%%%%%%%%%%%%%%%%%%%%%%%%%%%%%%%%%%%%%%%%%%%%%%%%%%%%%%%%%%%%%%%%%%%%%%%%%%%%
\section{Information}

%%%%%%%%%%%%%%%%%%%%%%%%%%%%%%%%%%%%%%%%%%%%%%%%%%%%%%%%%%%%%%%%%%%%%%%%%%%%%%%%
\subsection{Copyright}

Copyright \copyright{} 2017--2018 Niklas Beisert

This work may be distributed and/or modified under the
conditions of the \LaTeX{} Project Public License, either version 1.3
of this license or (at your option) any later version.
The latest version of this license is in
  \url{http://www.latex-project.org/lppl.txt}
and version 1.3 or later is part of all distributions of \LaTeX{}
version 2005/12/01 or later.

This work has the LPPL maintenance status `maintained'.

The Current Maintainer of this work is Niklas Beisert.

This work consists of the files |README.txt|, |childdoc.ins| and |childdoc.dtx|
as well as the derived files |childdoc.def|, |cdocsamp.tex|
with |cdocsch1.tex|, |cdocsch2.tex|, |cdocspt3.tex|, |cdocspt4.tex|,
|cdocsdrf.tex|, |cdocsfn1.tex|, |cdocsfn2.tex|
as well as |childdoc.pdf|.

%%%%%%%%%%%%%%%%%%%%%%%%%%%%%%%%%%%%%%%%%%%%%%%%%%%%%%%%%%%%%%%%%%%%%%%%%%%%%%%%
\subsection{Files and Installation}

The package consists of the files:
%
\begin{center}
\begin{tabular}{ll}
    |README.txt|   & readme file \\
    |childdoc.ins| & installation file \\
    |childdoc.dtx| & source file \\
    |childdoc.def| & definition file \\
    |cdocsamp.tex| & sample main file \\
    |cdocsch1.tex| & sample include file \\
    |cdocsch2.tex| & sample include file \\
    |cdocspt3.tex| & sample part file \\
    |cdocspt4.tex| & sample part file \\
    |cdocsdrf.tex| & sample redirection file \\
    |cdocsfn1.tex| & sample redirection file \\
    |cdocsfn2.tex| & sample redirection file \\
    |childdoc.pdf| & manual
\end{tabular}
\end{center}
%
The distribution consists of the files
|README.txt|, |childdoc.ins| and |childdoc.dtx|.
%
\begin{itemize}
\item
Run (pdf)\LaTeX{} on |childdoc.dtx|
to compile the manual |childdoc.pdf| (this file).
\item
Run \LaTeX{} on |childdoc.ins| to create the definitions file |childdoc.def|
and the sample |cdocsamp.tex| with include files
|cdocsch1.tex|, |cdocsch2.tex|, |cdocspt3.tex|, |cdocspt4.tex|,
|cdocsdrf.tex|, |cdocsfn1.tex|, |cdocsfn2.tex|.
Then copy the file |childdoc.def| to an appropriate directory of your \LaTeX{}
distribution, e.g.\ \textit{texmf-root}|/tex/latex/childdoc|.
\end{itemize}

%%%%%%%%%%%%%%%%%%%%%%%%%%%%%%%%%%%%%%%%%%%%%%%%%%%%%%%%%%%%%%%%%%%%%%%%%%%%%%%%
\subsection{Related CTAN Packages}

There are several other packages which offer a similar functionality:
%
\begin{itemize}
\item
The packages
\href{http://ctan.org/pkg/docmute}{\textsf{docmute}},
\href{http://ctan.org/pkg/includex}{\textsf{includex}} and
\href{http://ctan.org/pkg/standalone}{\textsf{standalone}}
provide commands to include only the document body of
a child file thus allowing both files to be compiled individually.
\item
The packages \href{http://ctan.org/pkg/subdocs}{\textsf{subdocs}}
and \href{http://ctan.org/pkg/subfiles}{\textsf{subfiles}}
provide structures in which the main and child documents can be
encapsulated and allowing them to be compiled individually.
The inclusion mechanism is different from the conventional |\include|.
\item
The package \href{http://ctan.org/pkg/combine}{\textsf{combine}}
is an elaborate solution to combine several documents into one.
\end{itemize}
%
See also the CTAN topic \href{http://ctan.org/topic/subdocs}{\textsf{subdocs}}
for further related packages.
The present package differs from the above solutions in that
a document structure constructed with the conventional |\include| mechanism
just needs two extra commands at the top of every file
such that all constituent files can be compiled individually.

%%%%%%%%%%%%%%%%%%%%%%%%%%%%%%%%%%%%%%%%%%%%%%%%%%%%%%%%%%%%%%%%%%%%%%%%%%%%%%%%
%\subsection{Feature Suggestions}
%
%The following is a list of features which may be useful for future
%versions of this package:
%%
%\begin{itemize}
%\item
%\ldots
%\end{itemize}

%%%%%%%%%%%%%%%%%%%%%%%%%%%%%%%%%%%%%%%%%%%%%%%%%%%%%%%%%%%%%%%%%%%%%%%%%%%%%%%%
\subsection{Revision History}

%%%%%%%%%%%%%%%%%%%%%%%%%%%%%%%%%%%%%%%%
\paragraph{v2.0:} 2018/12/30

\begin{itemize}
\item
immediate forward processing
\item
added |\childdocby| mechanism
\item
manual restructured
\end{itemize}

%%%%%%%%%%%%%%%%%%%%%%%%%%%%%%%%%%%%%%%%
\paragraph{v1.6:} 2018/01/17

\begin{itemize}
\item
application for development of include files
\item
corrections to manual
\end{itemize}

%%%%%%%%%%%%%%%%%%%%%%%%%%%%%%%%%%%%%%%%
\paragraph{v1.5:} 2017/05/21

\begin{itemize}
\item
more complete structuring introduced
\item
|\childdocof| introduced
\item
|\childdoc| renamed to |\childdocmain|
\item
|\childredirect| renamed to |\childdocforward| and |\childdocforwardprefix|
and functionality expanded
\end{itemize}

%%%%%%%%%%%%%%%%%%%%%%%%%%%%%%%%%%%%%%%%
\paragraph{v1.0:} 2017/04/27

\begin{itemize}
\item
manual and install package
\item
first version published on CTAN
\end{itemize}

%%%%%%%%%%%%%%%%%%%%%%%%%%%%%%%%%%%%%%%%
\paragraph{v0.6:} 2017/04/26

\begin{itemize}
\item
redirection mechanism added
\end{itemize}

%%%%%%%%%%%%%%%%%%%%%%%%%%%%%%%%%%%%%%%%
\paragraph{v0.5:} 2017/04/26

\begin{itemize}
\item
functionality in definition file
\end{itemize}


%%%%%%%%%%%%%%%%%%%%%%%%%%%%%%%%%%%%%%%%%%%%%%%%%%%%%%%%%%%%%%%%%%%%%%%%%%%%%%%%
%%%%%%%%%%%%%%%%%%%%%%%%%%%%%%%%%%%%%%%%%%%%%%%%%%%%%%%%%%%%%%%%%%%%%%%%%%%%%%%%
%%%%%%%%%%%%%%%%%%%%%%%%%%%%%%%%%%%%%%%%%%%%%%%%%%%%%%%%%%%%%%%%%%%%%%%%%%%%%%%%
\appendix

\settowidth\MacroIndent{\rmfamily\scriptsize 000\ }

 \DocInput{childdoc.dtx}

\end{document}
%</driver>
% \fi
%
% %%%%%%%%%%%%%%%%%%%%%%%%%%%%%%%%%%%%%%%%%%%%%%%%%%%%%%%%%%%%%%%%%%%%%%%%%%%%%%
% %%%%%%%%%%%%%%%%%%%%%%%%%%%%%%%%%%%%%%%%%%%%%%%%%%%%%%%%%%%%%%%%%%%%%%%%%%%%%%
% \section{Sample}
%\iffalse
%<*samplemain>
%\fi
%
% The following presents a sample document
% with two chapters, two parts, a title page,
% a compile flag as well as three forwarding files to set the flag.
% It consists of eight |.tex| files:
% \begin{center}
% \begin{tabular}{ll}
% |cdocsamp.tex|&main file\\
% |cdocsch1.tex|&include file for chapter 1\\
% |cdocsch2.tex|&include file for chapter 2\\
% |cdocspt3.tex|&include file for part 3\\
% |cdocspt4.tex|&include file for part 4\\
% |cdocsdrf.tex|&forwarding file for main file in draft mode\\
% |cdocsfi1.tex|&forwarding file for final version of chapter 1\\
% |cdocsfi2.tex|&forwarding file for final version of chapter 2\\
% \end{tabular}
% \end{center}
% Each of the eight files can be compiled directly by the \LaTeX{} compiler.
%
% %%%%%%%%%%%%%%%%%%%%%%%%%%%%%%%%%%%%%%
% \paragraph{Main File.}
%
% The main file is called |cdocsamp.tex|.
%
% Load the \textsf{childdoc} definitions and
% declare the filename for the main document:
%    \begin{macrocode}
\input{childdoc.def}
\childdocmain{}
%    \end{macrocode}

% Optional override for |\version| flag:
%    \begin{macrocode}
%%\ifchilddoc\else\providecommand{\version}{draft}\fi
%    \end{macrocode}

% Define the default values for the |\version| flag
% (|final| for the main file and |draft| for childs):
%    \begin{macrocode}
\ifchilddoc
\providecommand{\version}{draft}
\else
\providecommand{\version}{final}
\fi
%    \end{macrocode}

% Load the standard document class:
%    \begin{macrocode}
\documentclass[12pt]{article}
%    \end{macrocode}

% Start the document body:
%    \begin{macrocode}
\begin{document}
%    \end{macrocode}

% Declare a title page.
% Print title, part of document being processed and version flag:
%    \begin{macrocode}
\addtocounter{page}{-1}
\begin{center}
{\LARGE\bfseries{}childdoc example\par}
\vspace{1cm}
\ifchilddoc
\ifchilddocmanual part\else chapter\fi:
`\childdocname' of `\childdocjob'\par
\else
main document: `\childdocjob'\par
\fi
version: \version\par
\end{center}
\newpage
%    \end{macrocode}

% Manually include selected file,
% otherwise process as usual:
%    \begin{macrocode}
\ifchilddocmanual
\section*{part `\childdocname'}
\input{\childdocname}
\else
%    \end{macrocode}

% Include the two chapters:
%    \begin{macrocode}
\include{cdocsch1}
\include{cdocsch2}
%    \end{macrocode}

% Include the two parts unless only chapters should be displayed:
%    \begin{macrocode}
\ifchilddoc\else
\section{part three}
\input{cdocspt3}
\section{part four}
\input{cdocspt4}
\fi
%    \end{macrocode}

% Process as usual until here:
%    \begin{macrocode}
\fi
%    \end{macrocode}

% End of document body:
%    \begin{macrocode}
\end{document}
%    \end{macrocode}
%\iffalse
%</samplemain>
%\fi
%
% %%%%%%%%%%%%%%%%%%%%%%%%%%%%%%%%%%%%%%
% \paragraph{Chapter Include Files.}
%
% The include files are called |cdocsch1.tex| and |cdocsch2.tex|.
%
%\iffalse
%<*samplechap1|samplechap2>
%\fi

% Optional override for |\version| flag:
%    \begin{macrocode}
%%\providecommand{\version}{final}
%    \end{macrocode}

% Include the main document:
%    \begin{macrocode}
\input{childdoc.def}
\childdocof{cdocsamp}
%    \end{macrocode}

%\iffalse
%</samplechap1|samplechap2>
%\fi
%
%\iffalse
%<*samplechap1>
%\fi
% Some text for chapter 1:
%    \begin{macrocode}
\section{one}
some text in chapter one
%    \end{macrocode}

%\iffalse
%</samplechap1>
%\fi
% Some text for chapter 2:
%\iffalse
%<*samplechap2>
%\fi
%    \begin{macrocode}
\section{two}
more text in chapter two
%    \end{macrocode}

%\iffalse
%</samplechap2>
%\fi
%
% %%%%%%%%%%%%%%%%%%%%%%%%%%%%%%%%%%%%%%
% \paragraph{Part Include Files.}
%
% The include files are called |cdocspt3.tex| and |cdocspt4.tex|.
%
%\iffalse
%<*samplepart3|samplepart4>
%\fi

% Optional override for |\version| flag:
%    \begin{macrocode}
%%\providecommand{\version}{final}
%    \end{macrocode}

% Include the main document:
%    \begin{macrocode}
\input{childdoc.def}
\childdocby{cdocsamp}
%    \end{macrocode}

%\iffalse
%</samplepart3|samplepart4>
%\fi
%
%\iffalse
%<*samplepart3>
%\fi
% Some text for part 3:
%    \begin{macrocode}
some text in part three
%    \end{macrocode}

%\iffalse
%</samplepart3>
%\fi
% Some text for part 4:
%\iffalse
%<*samplepart4>
%\fi
%    \begin{macrocode}
more text in part four
%    \end{macrocode}

%\iffalse
%</samplepart4>
%\fi
%
% %%%%%%%%%%%%%%%%%%%%%%%%%%%%%%%%%%%%%%
% \paragraph{Forwarding for a Complete Draft.}
%
% The following forwarding file |cdocsdrf.tex|
% compiles the main document in draft mode:
%\iffalse
%<*sampledraft>
%\fi
%    \begin{macrocode}
\def\version{draft}
\input{childdoc.def}
\childdocforward{cdocsamp}
%    \end{macrocode}

%\iffalse
%</sampledraft>
%\fi
%
% %%%%%%%%%%%%%%%%%%%%%%%%%%%%%%%%%%%%%%
% \paragraph{Forwarding for Final Version of the Chapters.}
%
% The following forwarding files |cdocsfn1.tex| and |cdocsfn2.tex|
% (with identical content)
% compile the final versions of the child documents
% |cdocsch1.tex| and |cdocsch2.tex|, respectively:
%\iffalse
%<*samplefinal>
%\fi
%    \begin{macrocode}
\def\version{final}
\input{childdoc.def}
\childdocforwardprefix[cdocsamp]{cdocsfn}{cdocsch}
%    \end{macrocode}

%\iffalse
%</samplefinal>
%\fi
%
% %%%%%%%%%%%%%%%%%%%%%%%%%%%%%%%%%%%%%%
% \paragraph{Command Line Processing.}
%
% The following three command lines generate the output files
% |cdocscld|, |cdocscl1| and |cdocscl2|
% which should be identical to
% |cdocsdrf|, |cdocsch1| and |cdocsfn2|, respectively:
% \begin{center}
% \begin{tabular}{l}
% |latex -jobname cdocscld \|\\
% |  "\def\version{draft}\input{childdoc.def}\childdocforward{cdocsamp}"|\\
% |latex -jobname cdocscl1 \|\\
% |  "\input{childdoc.def}\childdocforward[cdocsamp]{cdocsch1}"|\\
% |latex -jobname cdocscl2 \|\\
% |  "\def\version{final}\input{childdoc.def}\childdocforward{cdocsch2}"|
% \end{tabular}
% \end{center}
% Note that the trailing backslash on each first line
% merely continues the input to the second line
% (for convenient cut ant paste).
% Furthermore, the command |latex| can be replaced by any
% of its alternative versions such as |pdflatex|.
%
% %%%%%%%%%%%%%%%%%%%%%%%%%%%%%%%%%%%%%%%%%%%%%%%%%%%%%%%%%%%%%%%%%%%%%%%%%%%%%%
% %%%%%%%%%%%%%%%%%%%%%%%%%%%%%%%%%%%%%%%%%%%%%%%%%%%%%%%%%%%%%%%%%%%%%%%%%%%%%%
% \section{Implementation}
%\iffalse
%<*package>
%\fi
%
% This section describes the definitions file |childdoc.def|.

% The definitions cannot be loaded using |\usepackage| or |\RequirePackage|
% which has a mechanism to prevent loading a style file more than once.
% When loading the definitions by means of |\input|
% multiple instances have to be prevented manually:
%\iffalse
%This code needs to be before the `\ProvidesFile' directive
%which is defined at the beginning of this file.
%Therefore it is also placed there and commented out here.
%</package>
%<*discard>
%\fi
%    \begin{macrocode}
\ifdefined\childdocmain\endinput\fi
%    \end{macrocode}
%\iffalse
%</discard>
%<*package>
%\fi
%
% \macro{\ifchilddoc}
% \macro{\ifchilddocmanual}
% The conditional |\ifchilddoc| tells whether a
% child (true) or main (false) document is being compiled.
% The conditional |\ifchilddocmanual| tells whether
% the |\includeonly| mechanism is used (false) or
% the selection of child files must be performed manually (true).
% The definitions initialise to false:
%    \begin{macrocode}
\newif\ifchilddoc
\newif\ifchilddocmanual
%    \end{macrocode}

% \macro{\childdocname}
% \macro{\childdocjob}
% The macro |\childdocname| stores the name of the main document
% to be compiled. The macro |\childdocjob| stores the name of
% the document on which the \LaTeX{} compiler was originally invoked.
% The content of |\jobname| cannot be compared
% to filenames specified in the source due to different catcodes.
% The following code rescans |\jobname|, stores the result
% in |\childdocname| and saves a copy in |\childdocjob|:
%    \begin{macrocode}
\edef\childdocname{\scantokens\expandafter{\jobname\noexpand}}
\let\childdocjob\childdocname
%    \end{macrocode}

% \macro{\childdocdisable}
% The macro |\childdocdisable| prevents the main file
% from being processed more than once.
% At this stage, the main document command |\childdocmain|
% is assumed to be called once again where it should do nothing.
% Any subsequent call to it should prevent
% a secondary processing of the main document
% It overwrites the forwarding commands
% |\childdocof| and |\childdocforward|
% with empty macros to prevent further inclusions of the main document:
%    \begin{macrocode}
\newcommand{\childdocdisable}
{
  \renewcommand{\childdocmain}[1]{\renewcommand{\childdocmain}[1]{\endinput}}
  \renewcommand{\childdocof}[1]{}
  \renewcommand{\childdocby}[2][]{}
  \renewcommand{\childdocforward}[2][]{}
  \renewcommand{\childdocdisable}{}
}
%    \end{macrocode}

% \macro{\childdocmain}
% The macro |\childdocmain| is to be called at the top of the main file
% with nothing or the main filename (without extension) as argument.
% First, it breaks loops.
% If the argument is not empty and does not match |\childdocname|
% (which is set by the first inclusion of |childdoc.def|),
% |\ifchilddoc| is set to true, |\includeonly| is applied to the child file
% and |\jobname| is set to the main file
% (for proper handling of |.aux| files):
%    \begin{macrocode}
\newcommand{\childdocmain}[1]
{
  \childdocdisable\childdocmain{}
  \if?#1?\else
    \begingroup
      \def\childdoctmp{#1}
      \ifx\childdoctmp\childdocname
        \def\childdoctmp{}
      \else
        \def\childdoctmp
        {
          \childdoctrue
          \includeonly{\childdocname}
          \def\childdocjob{#1}
          \def\jobname{#1}
        }
      \fi
      \expandafter
    \endgroup
    \childdoctmp
  \fi
}
%    \end{macrocode}

% \macro{\childdocof}
% The command |\childdocof| redirects
% compilation to the main file |#1|.
%    \begin{macrocode}
\newcommand{\childdocof}[1]
{
  \childdocdisable
  \childdoctrue
  \includeonly{\childdocname}
  \def\jobname{#1}
  \def\childdocjob{#1}
  \input{#1}
}
%    \end{macrocode}

% \macro{\childdocby}
% The command |\childdocby| ....
%    \begin{macrocode}
\newcommand{\childdocby}[2][]
{
  \childdocdisable
  \childdoctrue
  \childdocmanualtrue
  \if?#1?\else
    \def\jobname{#2}
  \fi
  \def\childdocjob{#2}
  \input{#2}
  \endinput
}
%    \end{macrocode}

% \macro{\childdocforward}
% The command |\childdocforward| redirects
% compilation to the main file or
% (if the optional argument is given) a child file.
% Parameters are set as if the main file
% or a child file starting with |\childdocof| was compiled.
% Then compilation is handed over to the main file:
%    \begin{macrocode}
\newcommand{\childdocforward}[2][]
{
  \begingroup
    \if?#1?
      \def\childdoctmp
      {
        \def\childdocname{#2}
        \def\childdocjob{#2}
        \def\jobname{#2}
        \input{#2}
        \endinput
      }
    \else
      \def\childdoctmp
      {
        \childdocdisable
        \def\childdocname{#2}
        \childdoctrue
        \includeonly{#2}
        \def\childdocjob{#1}
        \def\jobname{#1}
        \input{#1}
        \endinput
      }
    \fi
    \expandafter
  \endgroup
  \childdoctmp
}
%    \end{macrocode}

% \macro{\childdocforwardprefix}
% The command |\childdocforwardprefix| redirects
% compilation to the main or a child file by means of a pattern.
% The prefix |#1| in the current filename is replaced by |#2|
% and the suffix of the current filename is kept
% (it is assumed that the filename does not contain the substring `|~~~|'
% which is used as a delimiter).
% Compilation is handed over to the new file by |\childdocforward|:
%    \begin{macrocode}
\newcommand{\childdocforwardprefix}[3][]
{
  \begingroup
    \def\childdocextract #2##1~~~{\def\childdoctmp{\childdocforward[#1]{#3##1}}}
    \expandafter\childdocextract\childdocname~~~
    \expandafter
  \endgroup
  \childdoctmp
}
%    \end{macrocode}

% \macro{\childdoc}
% The deprecated macro |\childdoc| is a legacy version of |\childdocmain|:
%    \begin{macrocode}
\newcommand{\childdoc}{\childdocmain}
%    \end{macrocode}

% \macro{\childdocredirect}
% The deprecated macro |\childdocredirect| is a legacy version
% of |\childdocforward| and |\childdocforwardprefix|:
%    \begin{macrocode}
\newcommand{\childdocredirect}[2][]
{
  \begingroup
    \if?#1?
      \def\childdoctmp{\childdocforward{#2}}
    \else
      \def\childdoctmp{\childdocforwardprefix{#1}{#2}}
    \fi
    \expandafter
  \endgroup
  \childdoctmp
}
%    \end{macrocode}

%\iffalse
%</package>
%\fi
%
\endinput

\childdocof{cdocsamp}
%    \end{macrocode}

%\iffalse
%</samplechap1|samplechap2>
%\fi
%
%\iffalse
%<*samplechap1>
%\fi
% Some text for chapter 1:
%    \begin{macrocode}
\section{one}
some text in chapter one
%    \end{macrocode}

%\iffalse
%</samplechap1>
%\fi
% Some text for chapter 2:
%\iffalse
%<*samplechap2>
%\fi
%    \begin{macrocode}
\section{two}
more text in chapter two
%    \end{macrocode}

%\iffalse
%</samplechap2>
%\fi
%
% %%%%%%%%%%%%%%%%%%%%%%%%%%%%%%%%%%%%%%
% \paragraph{Part Include Files.}
%
% The include files are called |cdocspt3.tex| and |cdocspt4.tex|.
%
%\iffalse
%<*samplepart3|samplepart4>
%\fi

% Optional override for |\version| flag:
%    \begin{macrocode}
%%\providecommand{\version}{final}
%    \end{macrocode}

% Include the main document:
%    \begin{macrocode}
% \iffalse
%
% childdoc.dtx Copyright (C) 2017-2018 Niklas Beisert
%
% This work may be distributed and/or modified under the
% conditions of the LaTeX Project Public License, either version 1.3
% of this license or (at your option) any later version.
% The latest version of this license is in
%   http://www.latex-project.org/lppl.txt
% and version 1.3 or later is part of all distributions of LaTeX
% version 2005/12/01 or later.
%
% This work has the LPPL maintenance status `maintained'.
%
% The Current Maintainer of this work is Niklas Beisert.
%
% This work consists of the files childdoc.dtx and childdoc.ins
% and the derived files childdoc.def and cdocsamp.tex with
% cdocsch1.tex, cdocsch2.tex, cdocsdrf.tex, cdocsfn1.tex, cdocsfn2.tex.
%
%<package>\ifdefined\childdocmain\endinput\fi
%<package>\ProvidesFile{childdoc.def}[2018/12/30 v2.0 child document driver]
%<samplemain>\ProvidesFile{cdocsamp.tex}[2018/12/30 v2.0 sample for childdoc]
%<*driver>
%\ProvidesFile{childdoc.drv}[2018/12/30 v2.0 childdoc reference manual file]
\PassOptionsToClass{10pt,a4paper}{article}
\documentclass{ltxdoc}

\usepackage[margin=35mm]{geometry}
\usepackage{hyperref}
\usepackage{hyperxmp}
\usepackage[usenames]{color}

\hypersetup{colorlinks=true}
\hypersetup{pdfstartview=FitH}
\hypersetup{pdfpagemode=UseNone}
\hypersetup{pdfsource={}}
\hypersetup{pdflang={en-UK}}
\hypersetup{pdfcopyright={Copyright 2017-2018 Niklas Beisert.
  This work may be distributed and/or modified under the
  conditions of the LaTeX Project Public License, either version 1.3
  of this license or (at your option) any later version.}}
\hypersetup{pdflicenseurl={http://www.latex-project.org/lppl.txt}}
\hypersetup{pdfcontactaddress={ETH Zurich, ITP, HIT K,
  Wolfgang-Pauli-Strasse 27}}
\hypersetup{pdfcontactpostcode={8093}}
\hypersetup{pdfcontactcity={Zurich}}
\hypersetup{pdfcontactcountry={Switzerland}}
\hypersetup{pdfcontactemail={nbeisert@itp.phys.ethz.ch}}
\hypersetup{pdfcontacturl={http://people.phys.ethz.ch/\xmptilde nbeisert/}}

\newcommand{\secref}[1]{\hyperref[#1]{section \ref*{#1}}}

\parskip1ex
\parindent0pt
\let\olditemize\itemize
\def\itemize{\olditemize\parskip0pt}

\begin{document}

\title{The \textsf{childdoc} Package}
\hypersetup{pdftitle={The childdoc Package}}
\author{Niklas Beisert\\[2ex]
  Institut f\"ur Theoretische Physik\\
  Eidgen\"ossische Technische Hochschule Z\"urich\\
  Wolfgang-Pauli-Strasse 27, 8093 Z\"urich, Switzerland\\[1ex]
  \href{mailto:nbeisert@itp.phys.ethz.ch}
  {\texttt{nbeisert@itp.phys.ethz.ch}}}
\hypersetup{pdfauthor={Niklas Beisert}}
\hypersetup{pdfsubject={Manual for the LaTeX2e Package childdoc}}
\date{30 December 2018, \textsf{v2.0}}
\maketitle

\begin{abstract}\noindent
\textsf{childdoc} is a \LaTeXe{} package
that enables the direct compilation
of document sections included by |\include|
to individual files.
\end{abstract}

\begingroup
\parskip0ex
\tableofcontents
\endgroup

%%%%%%%%%%%%%%%%%%%%%%%%%%%%%%%%%%%%%%%%%%%%%%%%%%%%%%%%%%%%%%%%%%%%%%%%%%%%%%%%
%%%%%%%%%%%%%%%%%%%%%%%%%%%%%%%%%%%%%%%%%%%%%%%%%%%%%%%%%%%%%%%%%%%%%%%%%%%%%%%%
\section{Introduction}

\LaTeX{} provides a mechanism to structure a large document (such as a book)
into a main file and several child files (containing the chapters)
using the |\include| command.
This mechanism is beneficial for documents
which span hundreds of pages in order to
make the source file(s) more manageable.
Moreover, compilation can be restricted to
selected child files by means of the |\includeonly| command.
The latter feature can be used to reduce the compilation time while editing
(this was significantly more useful in the earlier days of \LaTeX{})
or to generate a smaller document which is easier to navigate.
Another application of |\includeonly| is to generate
documents consisting of selected parts of the complete document.

However, there are a few drawbacks of the plain |\include| mechanism:
\begin{itemize}
\item
The child files cannot be compiled on their own,
they can only be compiled via the main file.
A naive editing environment
(such as a text editor with an option
to have the current file processed by \LaTeX)
may require one to switch to the main file before compiling;
attempting to compile the child file produces errors.
\item
The main file must be modified (each time)
to adjust the |\includeonly| command
to the present needs. This easily leaves the main file in a messy state.
\item
The generated document will always carry the filename
of the main document. This is inconvenient if
several child files are to be compiled and
to be kept for distribution.
\end{itemize}

The present package provides a simple interface
to make child files individually compilable by \LaTeX{}.
Compiling a child file then has the same effect as compiling
the main file with an |\includeonly| command
to select the appropriate child.
Moreover the generated document will carry the name of the child
rather than the main file.
This resolves all three above issues.

This feature is meant to make the editing of books,
thesis documents and lecture notes somewhat more convenient.
However, the package can also be used efficiently for
composing a series of documents (such as exercise sheets)
which are typically distributed individually.
It then assists the author in generating the individual documents
(potentially in different versions)
as well as a document containing the collected series.
Another application is in developing style files
or other kinds of included material
where compilation of the style file could redirect
to a sample or test file.

%%%%%%%%%%%%%%%%%%%%%%%%%%%%%%%%%%%%%%%%%%%%%%%%%%%%%%%%%%%%%%%%%%%%%%%%%%%%%%%%
%%%%%%%%%%%%%%%%%%%%%%%%%%%%%%%%%%%%%%%%%%%%%%%%%%%%%%%%%%%%%%%%%%%%%%%%%%%%%%%%
\section{Usage}

First of all, the package \textsf{childdoc} is \emph{not} a standard
\LaTeXe{} |.sty| style file! Therefore it needs to be invoked in
a non-standard way.

%%%%%%%%%%%%%%%%%%%%%%%%%%%%%%%%%%%%%%%%%%%%%%%%%%%%%%%%%%%%%%%%%%%%%%%%%%%%%%%%
\subsection{Included Files}
\label{sec:include}

%%%%%%%%%%%%%%%%%%%%%%%%%%%%%%%%%%%%%%%%
\DescribeMacro{\childdocmain}
To use the package, add the commands
\begin{center}
\begin{tabular}{l}
|\input{childdoc.def}|\\
|\childdocmain{}|\\
\end{tabular}
\end{center}
at the very top of the main \LaTeX{} file,
in particular \emph{before} the |\documentclass| statement!
The argument of |\childdocmain| should be left empty
(but it must be present).

%%%%%%%%%%%%%%%%%%%%%%%%%%%%%%%%%%%%%%%%
\DescribeMacro{\childdocof}
Furthermore, add the commands
\begin{center}
\begin{tabular}{l}
|\input{childdoc.def}|\\
|\childdocof{|\textit{main}|}|\\
\end{tabular}
\end{center}
at the top of every child file \textit{child}
which is included by |\include{|\textit{child}|}|
from within the main file
(or at least for those files to be compiled individually).
The argument \textit{main} must be the filename of the main file.

There are a couple of
considerations in setting up the main and child documents:

%%%%%%%%%%%%%%%%%%%%%%%%%%%%%%%%%%%%%%%%
\paragraph{Restrictions.}

Please note the following restrictions:
\begin{itemize}
\item
|\childdocmain| must be called with one argument \textit{main}
to ensure compatibility with earlier version of the package.
It must either be empty (|\childdocmain{}|)
or precisely match the filename of the main file in which it is specified.
See \secref{sec:detection} for further information.
\item
The filename \textit{main} must be specified without the |.tex| extension.
\item
The filename \textit{main} is case sensitive
(even in case-insensitive file systems)
due to internal string comparison.
\item
The argument \textit{main} should be fully expanded, it cannot be a macro.
\item
Subdirectories and special characters should be avoided in filenames.
\item
The command |\childdocmain{|\textit{main}|}| must be followed by a whitespace.
It should not be followed immediately by another command
or by a comment mark `|%|'.
This is because the \TeX{} parser reads the token immediately following
the argument of |\childdocmain| and puts it
at the beginning of every child section;
however, a white\-space is ignored.
\end{itemize}

%%%%%%%%%%%%%%%%%%%%%%%%%%%%%%%%%%%%%%%%
\paragraph{Content of Main File.}

It is advisable to place all content in the child files included by |\include|.
Any output contained in the main file will appear in all child documents
unless suppressed manually;
it cannot be suppressed automatically by the |\includeonly| directive
and thus should normally be avoided.
A method to include some content in the main file
by means of conditional processing is described in \secref{sec:conditional}.

%%%%%%%%%%%%%%%%%%%%%%%%%%%%%%%%%%%%%%%%
\paragraph{Page Numbering.}

When only a part of the document is compiled,
the appropriate numbering of pages
(as well as other status parameters)
is determined from the |.aux| files.
The latter contain information from previous passes.
However this information needs to propagate through
all intermediate child documents.
Therefore the page numbering in child documents may well
be inconsistent until the complete document is compiled at least once.

A useful (if unconventional) way to always ensure a consistent
page numbering is to restart the numbering in each child document
and denote the pages by `\textit{child}|.|\textit{page}'
where \textit{child} represents the chapter/section number of the child file.
This can be achieved by the command
|\numberwithin{page}{|\textit{child}|}|
of the \textsf{amsmath} package
where \textit{child} can be |chapter| or |section|
depending on the chosen structuring.
Alternatively, one can modify the macro |\thepage| appropriately
and reset the counter |page| at the start of each child file.

%%%%%%%%%%%%%%%%%%%%%%%%%%%%%%%%%%%%%%%%%%%%%%%%%%%%%%%%%%%%%%%%%%%%%%%%%%%%%%%%
\subsection{Conditional Processing}
\label{sec:conditional}

The package provides a mechanism to compile different versions
of a document. To customise the versions further some conditional processing
can come in handy to distinguish which version is being compiled.
The package provides two macros to describe the compilation context:

%%%%%%%%%%%%%%%%%%%%%%%%%%%%%%%%%%%%%%%%
\DescribeMacro{\ifchilddoc}
The conditional |\ifchilddoc| distinguishes between the compilation of
child documents and the main document:
%
\begin{center}
|\ifchilddoc |\textit{child-code}| |[|\||else |\textit{main-code}]| \||fi|
\end{center}

%%%%%%%%%%%%%%%%%%%%%%%%%%%%%%%%%%%%%%%%
\DescribeMacro{\childdocname}
\DescribeMacro{\childdocjob}
The macro |\childdocname| contains the filename (without extension)
of the main or child file being processed.
Note that |\childdocjob| will always contain the name of the main file.

%%%%%%%%%%%%%%%%%%%%%%%%%%%%%%%%%%%%%%%%
\paragraph{Title Page.}

Conditional processing can be used to include a title or banner page
in the main document when proper precautions are taken.
Importantly, the code in the main file should ensure that the page counter
(as well as other status parameters which are stored in the |.aux| files)
takes the same value after the conditional processing.
Otherwise the page numbers may take divergent values
depending on which part is compiled.

For example, a title page could be declared by:
%
\begin{center}
\begin{tabular}{l}
|\ifchilddoc\||else|\\
|\addtocounter{page}{-1}|\\
\textit{code for title page}\\
|\newpage|\\
|\||fi|
\end{tabular}
\end{center}
%
A banner page for the child documents can be generated by:
%
\begin{center}
\begin{tabular}{l}
|\ifchilddoc|\\
|\addtocounter{page}{-1}|\\
\textit{code for banner page}\\
|\newpage|\\
|\||fi|
\end{tabular}
\end{center}
%
Here one could write a message such as:
\begin{center}
|This is the part \childdocname{} of \childdocjob{}.|
\end{center}

%%%%%%%%%%%%%%%%%%%%%%%%%%%%%%%%%%%%%%%%%%%%%%%%%%%%%%%%%%%%%%%%%%%%%%%%%%%%%%%%
\subsection{Flags}
\label{sec:flags}

The package makes it easy to generate different versions
of the main or child documents.
To this end compilation flags can be defined
and assigned different default values.
They will be particularly useful in conjunction
with the forwarding mechanism described in \secref{sec:forward}.

For example, it may be useful to have a flag |\version|
which can be set to |draft| or |final|.
The document source will contain some conditional code
depending on the value of |\version|.
Suppose further, the flag should default to |final| for the main file
and to |draft| for child files
which is a natural assignment for editing the document.
This is achieved by placing the following code
in the preamble of the main document
(below the |\childdocmain| directive):
%
\begin{center}
\begin{tabular}{l}
|\ifchilddoc|\\
|\providecommand{\version}{draft}|\\
|\||else|\\
|\providecommand{\version}{final}|\\
|\||fi|
\end{tabular}
\end{center}
%
The definition by |\providecommand| makes sure
that previous definitions are not overwritten.
Further statements |\providecommand{\version}{...}|
can thus be added before the above code to override it.

For the main file, one might add a line
(between |\childdocmain| and the above block)
%
\begin{center}
|%\ifchilddoc\||else\providecommand{\version}{draft}\||fi|
\end{center}
%
which can be uncommented to produce a draft version.
Likewise one can add a line to the very top of a child file
(above the |\childdocof{|\textit{main}|}| directive)
%
\begin{center}
|%\providecommand{\version}{final}|
\end{center}
%
which can be uncommented to produce the final version of this child document.

%%%%%%%%%%%%%%%%%%%%%%%%%%%%%%%%%%%%%%%%%%%%%%%%%%%%%%%%%%%%%%%%%%%%%%%%%%%%%%%%
\subsection{Forwarding}
\label{sec:forward}

Different versions of the main or child documents
using compilation flags as described in \secref{sec:flags}
can be (permanently) stored in different files
for convenient compilation, viewing and distribution.
To this end, the package defines a command
to pass on compilation to a different file:

%%%%%%%%%%%%%%%%%%%%%%%%%%%%%%%%%%%%%%%%
\DescribeMacro{\childdocforward}
The command |\childdocforward| redirects processing to
another source file:
%
\begin{center}
\begin{tabular}{l}
|\input{childdoc.def}|\\
|\childdocforward[|\textit{main}|]{|\textit{dest}|}|\\
\end{tabular}
\end{center}
%
The argument \textit{dest} is the destination file
(without extension).
It should be the main file or one of the child files.
Note that further \textsf{childdoc} directives
such as |\childdocof| and |\childdocforward|
in the indicated file will be processed in this form.
The optional argument \textit{main}
passes on directly to the main file \textit{main}
while pretending to compile the child \textit{dest}.
This form behaves as if \textit{dest}
issues |\childdocof{|\textit{main}|}| right away,
and no further \textsf{childdoc} directives will be processed.

%%%%%%%%%%%%%%%%%%%%%%%%%%%%%%%%%%%%%%%%
\DescribeMacro{\...prefix}
In the alternative form |\childdocforwardprefix|,
%
\begin{center}
\begin{tabular}{l}
|\input{childdoc.def}|\\
|\childdocforwardprefix[|\textit{main}|]{|\textit{prefix}|}{|\textit{dest}|}|
\end{tabular}
\end{center}
%
the destination file is determined by a pattern
depending on the current file:
To make this work, the current file must be called
`{\textit{prefix}\hspace{0.2em}\textit{suffix}}'
with \textit{prefix} matching precisely the argument.
Processing is then passed on to the file
`{\textit{dest}\hspace{0.2em}\textit{suffix}}'.
Surely, the same effect is achieved by
directly specifying the
argument `{\textit{dest}\hspace{0.2em}\textit{suffix}}'
in the first form.
However, that requires to set up a different file
for each child. With the alternative form of the command
all these files can have exactly the same content
which simplifies setting them up and maintaining them.

For example, the following file |draft.tex|
with a compilation flag |\version| as described in \secref{sec:flags}
compiles the main document as a draft:
%
\begin{center}
\begin{tabular}{l}
|\def\version{draft}|\\
|\input{childdoc.def}|\\
|\childdocforward{|\textit{main}|}|
\end{tabular}
\end{center}
%
Likewise, the following files |final|\textit{nn}|.tex|
compile the final version of the child document
|child|\textit{nn}|.tex|:
%
\begin{center}
\begin{tabular}{l}
|\def\version{final}|\\
|\input{childdoc.def}|\\
|\childdocforwardprefix{final}{child}|
\end{tabular}
\end{center}
%

Note that when several versions of a main file and/or of each child file
are to be generated, it may be convenient to set up a |Makefile| or
shell script to automatise the process.

%%%%%%%%%%%%%%%%%%%%%%%%%%%%%%%%%%%%%%%%%%%%%%%%%%%%%%%%%%%%%%%%%%%%%%%%%%%%%%%%
\subsection{Command Line Processing}
\label{sec:commandline}

The effect of redirection files can also be achieved by invoking
the \LaTeX{} compiler with a more elaborate command line.
Most conveniently this should be done as part
of a shell script or a |Makefile|.

When using \textsf{childdoc} in the main file, the following
command lines effectively perform a redirection
(note that depending on the shell being used,
backslashes may have to be doubled: `|\|' $\to$ `|\\|'):
%
\begin{center}
|... -jobname "|\textit{target}|" |\\|"|[\textit{flags}]%
|\input{childdoc.def}\childdocforward[|\textit{main}|]{|\textit{dest}|}"|
\end{center}
%
Here \textit{target} is the name of the output file,
\textit{main} is the name of the main file
and \textit{dest} is the name of the main or child file to be processed
(all filenames without extensions).
The optional argument \textit{main} can be omitted
if \textit{main} matches \textit{dest}.
Optionally, compilation \textit{flags} can be defined via |\def| commands.
This command line makes the \TeX{} engine believe
it is compiling the file \textit{target}
whose content is specified as the latter parameter.
The provided code then forwards the processing to
\textit{main} or \textit{dest} as described in \secref{sec:forward}.

%%%%%%%%%%%%%%%%%%%%%%%%%%%%%%%%%%%%%%%%%%%%%%%%%%%%%%%%%%%%%%%%%%%%%%%%%%%%%%%%
\subsection{Include by Input}
\label{sec:input}

Including child documents by |\include| has some restrictions by design.
Most notably, the content of a child document always occupies
its own set of pages; pages cannot be shared between child documents.
Usually, this behaviour makes perfect sense
because each child document contain an essential part of the document.
However, in some situations it may be desirable to compose
a document from a collection of parts
without having mandatory page breaks between then.
For this case, the package
provides a mechanism to include parts
by |\input| which can also be processed individually.
However, by construction this mechanism
requires manual handling of the content to be output.

%%%%%%%%%%%%%%%%%%%%%%%%%%%%%%%%%%%%%%%%
\DescribeMacro{\ifchilddocmanual}
The main file should be prepared as usual, see \secref{sec:include}.
However, the document body must make a distinction
between processing of an individual part and of the main document, e.g.:
%
\begin{center}
\begin{tabular}{l}
|\ifchilddocmanual|\\
|\input{\childdocname}|\\
|\||else|\\
\textit{document body with }|\input{|\textit{part}|}|\\
|\||fi|
\end{tabular}
\end{center}
%
The conditional |\ifchilddocmanual| is true whenever
a part to be included by |\input| is being compiled,
and the name of the part is stored in |\childdocname|.

%%%%%%%%%%%%%%%%%%%%%%%%%%%%%%%%%%%%%%%%
\DescribeMacro{\childdocby}
Each part to be included by |\input| should start with:
%
\begin{center}
\begin{tabular}{l}
|\input{childdoc.def}|\\
|\childdocby{|\textit{main}|}|\\
\end{tabular}
\end{center}
%
The directive |\childdocby| is similar to |\childdocof|
described in \secref{sec:include},
but the subsequent selection of content must be done manually.
To that end, both |\ifchilddoc| and |\ifchilddocmanual|
will be true upon processing of a part,
and the name of the part is stored in |\childdocname|.
Note that |\jobname| will be set to the filename of the current part
so that each part receives an individual |.aux| file
that does not interfere with the |.aux| file(s) of the main document.
This behaviour can be altered by the alternative form
|\childdocby[*]{|\textit{main}|}| (with a non-empty optional argument)
which uses the |.aux| file of the main document
by setting |\jobname| to \textit{main}.

%%%%%%%%%%%%%%%%%%%%%%%%%%%%%%%%%%%%%%%%%%%%%%%%%%%%%%%%%%%%%%%%%%%%%%%%%%%%%%%%
\subsection{Driver Development}
\label{sec:driver}

The \textsf{childdoc} mechanism can also be use for the development
of definition files such as \LaTeX{} styles or classes.
This case differs from the above setup with multiple parts
included by |\include| in that no |\includeonly| should be invoked.
This can be achieved by starting the include file
(before |\ProvidesPackage|) with:
%
\begin{center}
\begin{tabular}{l}
|\input{childdoc.def}|\\
|\childdocforward{|\textit{main}|}|\\
\end{tabular}
\end{center}
%
or alternatively with:
%
\begin{center}
\begin{tabular}{l}
|\input{childdoc.def}|\\
|\childdocby{|\textit{main}|}|\\
\end{tabular}
\end{center}
%
Both forms have slightly different effects as described above.
The main file is prepared as usual, see \secref{sec:include}.

%%%%%%%%%%%%%%%%%%%%%%%%%%%%%%%%%%%%%%%%%%%%%%%%%%%%%%%%%%%%%%%%%%%%%%%%%%%%%%%%
\subsection{Legacy Detection}
\label{sec:detection}

The directive |\childdocmain| in the main file can detect
whether the complete document or merely a child is to be compiled
even without using the directive |\childdocof|.
This method is deprecated because it is less robust
and there is no compelling reason to use it;
it is merely provided for backward compatibility
and it may be removed in future versions.

If the detection mechanism is to be used,
it is mandatory to correctly specify
the filename of the main file as the argument of |\childdocmain|:
%
\begin{center}
\begin{tabular}{l}
|\input{childdoc.def}|\\
|\childdocmain{|\textit{main}|}|\\
\end{tabular}
\end{center}
%
If |\jobname| does not match the argument \textit{main} of |\childdocmain|,
it is assumed that |\jobname| points to the child file to be compiled.
When using |\childdocmain| with the main file specified as argument,
it suffices to start a child file
with just |\input{|\textit{main}|}|
without loading of the package and using |\childdocof|.
If instead all processing is done
with the appropriate \textsf{childdoc} directives,
the argument of \textit{main} of |\childdocmain| can be empty.

An alternative version of the command line processing described
in \secref{sec:commandline} using the detection mechanism reads:
%
\begin{center}
|... -jobname "|\textit{target}|" "|[\textit{flags}]%
[|\def\jobname{|\textit{dest}|}|]|\input{|\textit{main}|}"|
\end{center}

%%%%%%%%%%%%%%%%%%%%%%%%%%%%%%%%%%%%%%%%%%%%%%%%%%%%%%%%%%%%%%%%%%%%%%%%%%%%%%%%
\subsection{Manual Code}
\label{sec:manual}

In case one cannot be certain whether the definitions file |childdoc.def|
is installed on the target \TeX{} distribution
and one prefers not to ship it,
it is conceivable to paste a few relevant commands into the sources.

To that end, drop all statements |\input{childdoc.def}|
and perform the replacements as outlined below.
Instead of |\childdocmain{|\textit{main}|}| add the following code
to the top of the main file:
%
\begin{center}
\begin{tabular}{l}
|\||ifdefined\childdocname\endinput\||fi\newif\ifchilddoc|\\
|\edef\childdocname{\scantokens\expandafter{\jobname\noexpand}}|\\
|\def\childdocmain{|\textit{main}|}\||ifx\childdocmain\childdocname\||else|\\
|\childdoctrue\includeonly{\childdocname}\let\jobname\childdocmain\||fi|\\
\end{tabular}
\end{center}
%
Instead of |\childdocof{|\textit{main}|}| just include the main file
at the top of each child file:
%
\begin{center}
|\input{|\textit{main}|}|
\end{center}
%
A simple redirection |\childdocforward{|\textit{dest}|}| is achieved by:
%
\begin{center}
|\def\jobname{|\textit{dest}|}\input{\jobname}|
\end{center}
%
The redirection with prefix
|\childdocforwardprefix[|\textit{prefix}|]{|\textit{dest}|}|
is accomplished by:
%
\begin{center}
\begin{tabular}{l}
|{\edef\jobname{\scantokens\expandafter{\jobname\noexpand}}|\\
|\def\redirectjob |\textit{prefix}|#1~~~{\gdef\jobname{|\textit{dest}|#1}}|\\
|\expandafter\redirectjob\jobname~~~}\input{\jobname}|
\end{tabular}
\end{center}

In an alternative approach,
child documents can be compiled by a specific command line
without additional code or specific definitions:
%
\begin{center}
|... -jobname "|\textit{target}|" "|[\textit{flags}]%
|\includeonly{|\textit{dest}|}\input{|\textit{main}|}"|
\end{center}
%

%%%%%%%%%%%%%%%%%%%%%%%%%%%%%%%%%%%%%%%%%%%%%%%%%%%%%%%%%%%%%%%%%%%%%%%%%%%%%%%%
%%%%%%%%%%%%%%%%%%%%%%%%%%%%%%%%%%%%%%%%%%%%%%%%%%%%%%%%%%%%%%%%%%%%%%%%%%%%%%%%
\section{Information}

%%%%%%%%%%%%%%%%%%%%%%%%%%%%%%%%%%%%%%%%%%%%%%%%%%%%%%%%%%%%%%%%%%%%%%%%%%%%%%%%
\subsection{Copyright}

Copyright \copyright{} 2017--2018 Niklas Beisert

This work may be distributed and/or modified under the
conditions of the \LaTeX{} Project Public License, either version 1.3
of this license or (at your option) any later version.
The latest version of this license is in
  \url{http://www.latex-project.org/lppl.txt}
and version 1.3 or later is part of all distributions of \LaTeX{}
version 2005/12/01 or later.

This work has the LPPL maintenance status `maintained'.

The Current Maintainer of this work is Niklas Beisert.

This work consists of the files |README.txt|, |childdoc.ins| and |childdoc.dtx|
as well as the derived files |childdoc.def|, |cdocsamp.tex|
with |cdocsch1.tex|, |cdocsch2.tex|, |cdocspt3.tex|, |cdocspt4.tex|,
|cdocsdrf.tex|, |cdocsfn1.tex|, |cdocsfn2.tex|
as well as |childdoc.pdf|.

%%%%%%%%%%%%%%%%%%%%%%%%%%%%%%%%%%%%%%%%%%%%%%%%%%%%%%%%%%%%%%%%%%%%%%%%%%%%%%%%
\subsection{Files and Installation}

The package consists of the files:
%
\begin{center}
\begin{tabular}{ll}
    |README.txt|   & readme file \\
    |childdoc.ins| & installation file \\
    |childdoc.dtx| & source file \\
    |childdoc.def| & definition file \\
    |cdocsamp.tex| & sample main file \\
    |cdocsch1.tex| & sample include file \\
    |cdocsch2.tex| & sample include file \\
    |cdocspt3.tex| & sample part file \\
    |cdocspt4.tex| & sample part file \\
    |cdocsdrf.tex| & sample redirection file \\
    |cdocsfn1.tex| & sample redirection file \\
    |cdocsfn2.tex| & sample redirection file \\
    |childdoc.pdf| & manual
\end{tabular}
\end{center}
%
The distribution consists of the files
|README.txt|, |childdoc.ins| and |childdoc.dtx|.
%
\begin{itemize}
\item
Run (pdf)\LaTeX{} on |childdoc.dtx|
to compile the manual |childdoc.pdf| (this file).
\item
Run \LaTeX{} on |childdoc.ins| to create the definitions file |childdoc.def|
and the sample |cdocsamp.tex| with include files
|cdocsch1.tex|, |cdocsch2.tex|, |cdocspt3.tex|, |cdocspt4.tex|,
|cdocsdrf.tex|, |cdocsfn1.tex|, |cdocsfn2.tex|.
Then copy the file |childdoc.def| to an appropriate directory of your \LaTeX{}
distribution, e.g.\ \textit{texmf-root}|/tex/latex/childdoc|.
\end{itemize}

%%%%%%%%%%%%%%%%%%%%%%%%%%%%%%%%%%%%%%%%%%%%%%%%%%%%%%%%%%%%%%%%%%%%%%%%%%%%%%%%
\subsection{Related CTAN Packages}

There are several other packages which offer a similar functionality:
%
\begin{itemize}
\item
The packages
\href{http://ctan.org/pkg/docmute}{\textsf{docmute}},
\href{http://ctan.org/pkg/includex}{\textsf{includex}} and
\href{http://ctan.org/pkg/standalone}{\textsf{standalone}}
provide commands to include only the document body of
a child file thus allowing both files to be compiled individually.
\item
The packages \href{http://ctan.org/pkg/subdocs}{\textsf{subdocs}}
and \href{http://ctan.org/pkg/subfiles}{\textsf{subfiles}}
provide structures in which the main and child documents can be
encapsulated and allowing them to be compiled individually.
The inclusion mechanism is different from the conventional |\include|.
\item
The package \href{http://ctan.org/pkg/combine}{\textsf{combine}}
is an elaborate solution to combine several documents into one.
\end{itemize}
%
See also the CTAN topic \href{http://ctan.org/topic/subdocs}{\textsf{subdocs}}
for further related packages.
The present package differs from the above solutions in that
a document structure constructed with the conventional |\include| mechanism
just needs two extra commands at the top of every file
such that all constituent files can be compiled individually.

%%%%%%%%%%%%%%%%%%%%%%%%%%%%%%%%%%%%%%%%%%%%%%%%%%%%%%%%%%%%%%%%%%%%%%%%%%%%%%%%
%\subsection{Feature Suggestions}
%
%The following is a list of features which may be useful for future
%versions of this package:
%%
%\begin{itemize}
%\item
%\ldots
%\end{itemize}

%%%%%%%%%%%%%%%%%%%%%%%%%%%%%%%%%%%%%%%%%%%%%%%%%%%%%%%%%%%%%%%%%%%%%%%%%%%%%%%%
\subsection{Revision History}

%%%%%%%%%%%%%%%%%%%%%%%%%%%%%%%%%%%%%%%%
\paragraph{v2.0:} 2018/12/30

\begin{itemize}
\item
immediate forward processing
\item
added |\childdocby| mechanism
\item
manual restructured
\end{itemize}

%%%%%%%%%%%%%%%%%%%%%%%%%%%%%%%%%%%%%%%%
\paragraph{v1.6:} 2018/01/17

\begin{itemize}
\item
application for development of include files
\item
corrections to manual
\end{itemize}

%%%%%%%%%%%%%%%%%%%%%%%%%%%%%%%%%%%%%%%%
\paragraph{v1.5:} 2017/05/21

\begin{itemize}
\item
more complete structuring introduced
\item
|\childdocof| introduced
\item
|\childdoc| renamed to |\childdocmain|
\item
|\childredirect| renamed to |\childdocforward| and |\childdocforwardprefix|
and functionality expanded
\end{itemize}

%%%%%%%%%%%%%%%%%%%%%%%%%%%%%%%%%%%%%%%%
\paragraph{v1.0:} 2017/04/27

\begin{itemize}
\item
manual and install package
\item
first version published on CTAN
\end{itemize}

%%%%%%%%%%%%%%%%%%%%%%%%%%%%%%%%%%%%%%%%
\paragraph{v0.6:} 2017/04/26

\begin{itemize}
\item
redirection mechanism added
\end{itemize}

%%%%%%%%%%%%%%%%%%%%%%%%%%%%%%%%%%%%%%%%
\paragraph{v0.5:} 2017/04/26

\begin{itemize}
\item
functionality in definition file
\end{itemize}


%%%%%%%%%%%%%%%%%%%%%%%%%%%%%%%%%%%%%%%%%%%%%%%%%%%%%%%%%%%%%%%%%%%%%%%%%%%%%%%%
%%%%%%%%%%%%%%%%%%%%%%%%%%%%%%%%%%%%%%%%%%%%%%%%%%%%%%%%%%%%%%%%%%%%%%%%%%%%%%%%
%%%%%%%%%%%%%%%%%%%%%%%%%%%%%%%%%%%%%%%%%%%%%%%%%%%%%%%%%%%%%%%%%%%%%%%%%%%%%%%%
\appendix

\settowidth\MacroIndent{\rmfamily\scriptsize 000\ }

 \DocInput{childdoc.dtx}

\end{document}
%</driver>
% \fi
%
% %%%%%%%%%%%%%%%%%%%%%%%%%%%%%%%%%%%%%%%%%%%%%%%%%%%%%%%%%%%%%%%%%%%%%%%%%%%%%%
% %%%%%%%%%%%%%%%%%%%%%%%%%%%%%%%%%%%%%%%%%%%%%%%%%%%%%%%%%%%%%%%%%%%%%%%%%%%%%%
% \section{Sample}
%\iffalse
%<*samplemain>
%\fi
%
% The following presents a sample document
% with two chapters, two parts, a title page,
% a compile flag as well as three forwarding files to set the flag.
% It consists of eight |.tex| files:
% \begin{center}
% \begin{tabular}{ll}
% |cdocsamp.tex|&main file\\
% |cdocsch1.tex|&include file for chapter 1\\
% |cdocsch2.tex|&include file for chapter 2\\
% |cdocspt3.tex|&include file for part 3\\
% |cdocspt4.tex|&include file for part 4\\
% |cdocsdrf.tex|&forwarding file for main file in draft mode\\
% |cdocsfi1.tex|&forwarding file for final version of chapter 1\\
% |cdocsfi2.tex|&forwarding file for final version of chapter 2\\
% \end{tabular}
% \end{center}
% Each of the eight files can be compiled directly by the \LaTeX{} compiler.
%
% %%%%%%%%%%%%%%%%%%%%%%%%%%%%%%%%%%%%%%
% \paragraph{Main File.}
%
% The main file is called |cdocsamp.tex|.
%
% Load the \textsf{childdoc} definitions and
% declare the filename for the main document:
%    \begin{macrocode}
\input{childdoc.def}
\childdocmain{}
%    \end{macrocode}

% Optional override for |\version| flag:
%    \begin{macrocode}
%%\ifchilddoc\else\providecommand{\version}{draft}\fi
%    \end{macrocode}

% Define the default values for the |\version| flag
% (|final| for the main file and |draft| for childs):
%    \begin{macrocode}
\ifchilddoc
\providecommand{\version}{draft}
\else
\providecommand{\version}{final}
\fi
%    \end{macrocode}

% Load the standard document class:
%    \begin{macrocode}
\documentclass[12pt]{article}
%    \end{macrocode}

% Start the document body:
%    \begin{macrocode}
\begin{document}
%    \end{macrocode}

% Declare a title page.
% Print title, part of document being processed and version flag:
%    \begin{macrocode}
\addtocounter{page}{-1}
\begin{center}
{\LARGE\bfseries{}childdoc example\par}
\vspace{1cm}
\ifchilddoc
\ifchilddocmanual part\else chapter\fi:
`\childdocname' of `\childdocjob'\par
\else
main document: `\childdocjob'\par
\fi
version: \version\par
\end{center}
\newpage
%    \end{macrocode}

% Manually include selected file,
% otherwise process as usual:
%    \begin{macrocode}
\ifchilddocmanual
\section*{part `\childdocname'}
\input{\childdocname}
\else
%    \end{macrocode}

% Include the two chapters:
%    \begin{macrocode}
\include{cdocsch1}
\include{cdocsch2}
%    \end{macrocode}

% Include the two parts unless only chapters should be displayed:
%    \begin{macrocode}
\ifchilddoc\else
\section{part three}
\input{cdocspt3}
\section{part four}
\input{cdocspt4}
\fi
%    \end{macrocode}

% Process as usual until here:
%    \begin{macrocode}
\fi
%    \end{macrocode}

% End of document body:
%    \begin{macrocode}
\end{document}
%    \end{macrocode}
%\iffalse
%</samplemain>
%\fi
%
% %%%%%%%%%%%%%%%%%%%%%%%%%%%%%%%%%%%%%%
% \paragraph{Chapter Include Files.}
%
% The include files are called |cdocsch1.tex| and |cdocsch2.tex|.
%
%\iffalse
%<*samplechap1|samplechap2>
%\fi

% Optional override for |\version| flag:
%    \begin{macrocode}
%%\providecommand{\version}{final}
%    \end{macrocode}

% Include the main document:
%    \begin{macrocode}
\input{childdoc.def}
\childdocof{cdocsamp}
%    \end{macrocode}

%\iffalse
%</samplechap1|samplechap2>
%\fi
%
%\iffalse
%<*samplechap1>
%\fi
% Some text for chapter 1:
%    \begin{macrocode}
\section{one}
some text in chapter one
%    \end{macrocode}

%\iffalse
%</samplechap1>
%\fi
% Some text for chapter 2:
%\iffalse
%<*samplechap2>
%\fi
%    \begin{macrocode}
\section{two}
more text in chapter two
%    \end{macrocode}

%\iffalse
%</samplechap2>
%\fi
%
% %%%%%%%%%%%%%%%%%%%%%%%%%%%%%%%%%%%%%%
% \paragraph{Part Include Files.}
%
% The include files are called |cdocspt3.tex| and |cdocspt4.tex|.
%
%\iffalse
%<*samplepart3|samplepart4>
%\fi

% Optional override for |\version| flag:
%    \begin{macrocode}
%%\providecommand{\version}{final}
%    \end{macrocode}

% Include the main document:
%    \begin{macrocode}
\input{childdoc.def}
\childdocby{cdocsamp}
%    \end{macrocode}

%\iffalse
%</samplepart3|samplepart4>
%\fi
%
%\iffalse
%<*samplepart3>
%\fi
% Some text for part 3:
%    \begin{macrocode}
some text in part three
%    \end{macrocode}

%\iffalse
%</samplepart3>
%\fi
% Some text for part 4:
%\iffalse
%<*samplepart4>
%\fi
%    \begin{macrocode}
more text in part four
%    \end{macrocode}

%\iffalse
%</samplepart4>
%\fi
%
% %%%%%%%%%%%%%%%%%%%%%%%%%%%%%%%%%%%%%%
% \paragraph{Forwarding for a Complete Draft.}
%
% The following forwarding file |cdocsdrf.tex|
% compiles the main document in draft mode:
%\iffalse
%<*sampledraft>
%\fi
%    \begin{macrocode}
\def\version{draft}
\input{childdoc.def}
\childdocforward{cdocsamp}
%    \end{macrocode}

%\iffalse
%</sampledraft>
%\fi
%
% %%%%%%%%%%%%%%%%%%%%%%%%%%%%%%%%%%%%%%
% \paragraph{Forwarding for Final Version of the Chapters.}
%
% The following forwarding files |cdocsfn1.tex| and |cdocsfn2.tex|
% (with identical content)
% compile the final versions of the child documents
% |cdocsch1.tex| and |cdocsch2.tex|, respectively:
%\iffalse
%<*samplefinal>
%\fi
%    \begin{macrocode}
\def\version{final}
\input{childdoc.def}
\childdocforwardprefix[cdocsamp]{cdocsfn}{cdocsch}
%    \end{macrocode}

%\iffalse
%</samplefinal>
%\fi
%
% %%%%%%%%%%%%%%%%%%%%%%%%%%%%%%%%%%%%%%
% \paragraph{Command Line Processing.}
%
% The following three command lines generate the output files
% |cdocscld|, |cdocscl1| and |cdocscl2|
% which should be identical to
% |cdocsdrf|, |cdocsch1| and |cdocsfn2|, respectively:
% \begin{center}
% \begin{tabular}{l}
% |latex -jobname cdocscld \|\\
% |  "\def\version{draft}\input{childdoc.def}\childdocforward{cdocsamp}"|\\
% |latex -jobname cdocscl1 \|\\
% |  "\input{childdoc.def}\childdocforward[cdocsamp]{cdocsch1}"|\\
% |latex -jobname cdocscl2 \|\\
% |  "\def\version{final}\input{childdoc.def}\childdocforward{cdocsch2}"|
% \end{tabular}
% \end{center}
% Note that the trailing backslash on each first line
% merely continues the input to the second line
% (for convenient cut ant paste).
% Furthermore, the command |latex| can be replaced by any
% of its alternative versions such as |pdflatex|.
%
% %%%%%%%%%%%%%%%%%%%%%%%%%%%%%%%%%%%%%%%%%%%%%%%%%%%%%%%%%%%%%%%%%%%%%%%%%%%%%%
% %%%%%%%%%%%%%%%%%%%%%%%%%%%%%%%%%%%%%%%%%%%%%%%%%%%%%%%%%%%%%%%%%%%%%%%%%%%%%%
% \section{Implementation}
%\iffalse
%<*package>
%\fi
%
% This section describes the definitions file |childdoc.def|.

% The definitions cannot be loaded using |\usepackage| or |\RequirePackage|
% which has a mechanism to prevent loading a style file more than once.
% When loading the definitions by means of |\input|
% multiple instances have to be prevented manually:
%\iffalse
%This code needs to be before the `\ProvidesFile' directive
%which is defined at the beginning of this file.
%Therefore it is also placed there and commented out here.
%</package>
%<*discard>
%\fi
%    \begin{macrocode}
\ifdefined\childdocmain\endinput\fi
%    \end{macrocode}
%\iffalse
%</discard>
%<*package>
%\fi
%
% \macro{\ifchilddoc}
% \macro{\ifchilddocmanual}
% The conditional |\ifchilddoc| tells whether a
% child (true) or main (false) document is being compiled.
% The conditional |\ifchilddocmanual| tells whether
% the |\includeonly| mechanism is used (false) or
% the selection of child files must be performed manually (true).
% The definitions initialise to false:
%    \begin{macrocode}
\newif\ifchilddoc
\newif\ifchilddocmanual
%    \end{macrocode}

% \macro{\childdocname}
% \macro{\childdocjob}
% The macro |\childdocname| stores the name of the main document
% to be compiled. The macro |\childdocjob| stores the name of
% the document on which the \LaTeX{} compiler was originally invoked.
% The content of |\jobname| cannot be compared
% to filenames specified in the source due to different catcodes.
% The following code rescans |\jobname|, stores the result
% in |\childdocname| and saves a copy in |\childdocjob|:
%    \begin{macrocode}
\edef\childdocname{\scantokens\expandafter{\jobname\noexpand}}
\let\childdocjob\childdocname
%    \end{macrocode}

% \macro{\childdocdisable}
% The macro |\childdocdisable| prevents the main file
% from being processed more than once.
% At this stage, the main document command |\childdocmain|
% is assumed to be called once again where it should do nothing.
% Any subsequent call to it should prevent
% a secondary processing of the main document
% It overwrites the forwarding commands
% |\childdocof| and |\childdocforward|
% with empty macros to prevent further inclusions of the main document:
%    \begin{macrocode}
\newcommand{\childdocdisable}
{
  \renewcommand{\childdocmain}[1]{\renewcommand{\childdocmain}[1]{\endinput}}
  \renewcommand{\childdocof}[1]{}
  \renewcommand{\childdocby}[2][]{}
  \renewcommand{\childdocforward}[2][]{}
  \renewcommand{\childdocdisable}{}
}
%    \end{macrocode}

% \macro{\childdocmain}
% The macro |\childdocmain| is to be called at the top of the main file
% with nothing or the main filename (without extension) as argument.
% First, it breaks loops.
% If the argument is not empty and does not match |\childdocname|
% (which is set by the first inclusion of |childdoc.def|),
% |\ifchilddoc| is set to true, |\includeonly| is applied to the child file
% and |\jobname| is set to the main file
% (for proper handling of |.aux| files):
%    \begin{macrocode}
\newcommand{\childdocmain}[1]
{
  \childdocdisable\childdocmain{}
  \if?#1?\else
    \begingroup
      \def\childdoctmp{#1}
      \ifx\childdoctmp\childdocname
        \def\childdoctmp{}
      \else
        \def\childdoctmp
        {
          \childdoctrue
          \includeonly{\childdocname}
          \def\childdocjob{#1}
          \def\jobname{#1}
        }
      \fi
      \expandafter
    \endgroup
    \childdoctmp
  \fi
}
%    \end{macrocode}

% \macro{\childdocof}
% The command |\childdocof| redirects
% compilation to the main file |#1|.
%    \begin{macrocode}
\newcommand{\childdocof}[1]
{
  \childdocdisable
  \childdoctrue
  \includeonly{\childdocname}
  \def\jobname{#1}
  \def\childdocjob{#1}
  \input{#1}
}
%    \end{macrocode}

% \macro{\childdocby}
% The command |\childdocby| ....
%    \begin{macrocode}
\newcommand{\childdocby}[2][]
{
  \childdocdisable
  \childdoctrue
  \childdocmanualtrue
  \if?#1?\else
    \def\jobname{#2}
  \fi
  \def\childdocjob{#2}
  \input{#2}
  \endinput
}
%    \end{macrocode}

% \macro{\childdocforward}
% The command |\childdocforward| redirects
% compilation to the main file or
% (if the optional argument is given) a child file.
% Parameters are set as if the main file
% or a child file starting with |\childdocof| was compiled.
% Then compilation is handed over to the main file:
%    \begin{macrocode}
\newcommand{\childdocforward}[2][]
{
  \begingroup
    \if?#1?
      \def\childdoctmp
      {
        \def\childdocname{#2}
        \def\childdocjob{#2}
        \def\jobname{#2}
        \input{#2}
        \endinput
      }
    \else
      \def\childdoctmp
      {
        \childdocdisable
        \def\childdocname{#2}
        \childdoctrue
        \includeonly{#2}
        \def\childdocjob{#1}
        \def\jobname{#1}
        \input{#1}
        \endinput
      }
    \fi
    \expandafter
  \endgroup
  \childdoctmp
}
%    \end{macrocode}

% \macro{\childdocforwardprefix}
% The command |\childdocforwardprefix| redirects
% compilation to the main or a child file by means of a pattern.
% The prefix |#1| in the current filename is replaced by |#2|
% and the suffix of the current filename is kept
% (it is assumed that the filename does not contain the substring `|~~~|'
% which is used as a delimiter).
% Compilation is handed over to the new file by |\childdocforward|:
%    \begin{macrocode}
\newcommand{\childdocforwardprefix}[3][]
{
  \begingroup
    \def\childdocextract #2##1~~~{\def\childdoctmp{\childdocforward[#1]{#3##1}}}
    \expandafter\childdocextract\childdocname~~~
    \expandafter
  \endgroup
  \childdoctmp
}
%    \end{macrocode}

% \macro{\childdoc}
% The deprecated macro |\childdoc| is a legacy version of |\childdocmain|:
%    \begin{macrocode}
\newcommand{\childdoc}{\childdocmain}
%    \end{macrocode}

% \macro{\childdocredirect}
% The deprecated macro |\childdocredirect| is a legacy version
% of |\childdocforward| and |\childdocforwardprefix|:
%    \begin{macrocode}
\newcommand{\childdocredirect}[2][]
{
  \begingroup
    \if?#1?
      \def\childdoctmp{\childdocforward{#2}}
    \else
      \def\childdoctmp{\childdocforwardprefix{#1}{#2}}
    \fi
    \expandafter
  \endgroup
  \childdoctmp
}
%    \end{macrocode}

%\iffalse
%</package>
%\fi
%
\endinput

\childdocby{cdocsamp}
%    \end{macrocode}

%\iffalse
%</samplepart3|samplepart4>
%\fi
%
%\iffalse
%<*samplepart3>
%\fi
% Some text for part 3:
%    \begin{macrocode}
some text in part three
%    \end{macrocode}

%\iffalse
%</samplepart3>
%\fi
% Some text for part 4:
%\iffalse
%<*samplepart4>
%\fi
%    \begin{macrocode}
more text in part four
%    \end{macrocode}

%\iffalse
%</samplepart4>
%\fi
%
% %%%%%%%%%%%%%%%%%%%%%%%%%%%%%%%%%%%%%%
% \paragraph{Forwarding for a Complete Draft.}
%
% The following forwarding file |cdocsdrf.tex|
% compiles the main document in draft mode:
%\iffalse
%<*sampledraft>
%\fi
%    \begin{macrocode}
\def\version{draft}
% \iffalse
%
% childdoc.dtx Copyright (C) 2017-2018 Niklas Beisert
%
% This work may be distributed and/or modified under the
% conditions of the LaTeX Project Public License, either version 1.3
% of this license or (at your option) any later version.
% The latest version of this license is in
%   http://www.latex-project.org/lppl.txt
% and version 1.3 or later is part of all distributions of LaTeX
% version 2005/12/01 or later.
%
% This work has the LPPL maintenance status `maintained'.
%
% The Current Maintainer of this work is Niklas Beisert.
%
% This work consists of the files childdoc.dtx and childdoc.ins
% and the derived files childdoc.def and cdocsamp.tex with
% cdocsch1.tex, cdocsch2.tex, cdocsdrf.tex, cdocsfn1.tex, cdocsfn2.tex.
%
%<package>\ifdefined\childdocmain\endinput\fi
%<package>\ProvidesFile{childdoc.def}[2018/12/30 v2.0 child document driver]
%<samplemain>\ProvidesFile{cdocsamp.tex}[2018/12/30 v2.0 sample for childdoc]
%<*driver>
%\ProvidesFile{childdoc.drv}[2018/12/30 v2.0 childdoc reference manual file]
\PassOptionsToClass{10pt,a4paper}{article}
\documentclass{ltxdoc}

\usepackage[margin=35mm]{geometry}
\usepackage{hyperref}
\usepackage{hyperxmp}
\usepackage[usenames]{color}

\hypersetup{colorlinks=true}
\hypersetup{pdfstartview=FitH}
\hypersetup{pdfpagemode=UseNone}
\hypersetup{pdfsource={}}
\hypersetup{pdflang={en-UK}}
\hypersetup{pdfcopyright={Copyright 2017-2018 Niklas Beisert.
  This work may be distributed and/or modified under the
  conditions of the LaTeX Project Public License, either version 1.3
  of this license or (at your option) any later version.}}
\hypersetup{pdflicenseurl={http://www.latex-project.org/lppl.txt}}
\hypersetup{pdfcontactaddress={ETH Zurich, ITP, HIT K,
  Wolfgang-Pauli-Strasse 27}}
\hypersetup{pdfcontactpostcode={8093}}
\hypersetup{pdfcontactcity={Zurich}}
\hypersetup{pdfcontactcountry={Switzerland}}
\hypersetup{pdfcontactemail={nbeisert@itp.phys.ethz.ch}}
\hypersetup{pdfcontacturl={http://people.phys.ethz.ch/\xmptilde nbeisert/}}

\newcommand{\secref}[1]{\hyperref[#1]{section \ref*{#1}}}

\parskip1ex
\parindent0pt
\let\olditemize\itemize
\def\itemize{\olditemize\parskip0pt}

\begin{document}

\title{The \textsf{childdoc} Package}
\hypersetup{pdftitle={The childdoc Package}}
\author{Niklas Beisert\\[2ex]
  Institut f\"ur Theoretische Physik\\
  Eidgen\"ossische Technische Hochschule Z\"urich\\
  Wolfgang-Pauli-Strasse 27, 8093 Z\"urich, Switzerland\\[1ex]
  \href{mailto:nbeisert@itp.phys.ethz.ch}
  {\texttt{nbeisert@itp.phys.ethz.ch}}}
\hypersetup{pdfauthor={Niklas Beisert}}
\hypersetup{pdfsubject={Manual for the LaTeX2e Package childdoc}}
\date{30 December 2018, \textsf{v2.0}}
\maketitle

\begin{abstract}\noindent
\textsf{childdoc} is a \LaTeXe{} package
that enables the direct compilation
of document sections included by |\include|
to individual files.
\end{abstract}

\begingroup
\parskip0ex
\tableofcontents
\endgroup

%%%%%%%%%%%%%%%%%%%%%%%%%%%%%%%%%%%%%%%%%%%%%%%%%%%%%%%%%%%%%%%%%%%%%%%%%%%%%%%%
%%%%%%%%%%%%%%%%%%%%%%%%%%%%%%%%%%%%%%%%%%%%%%%%%%%%%%%%%%%%%%%%%%%%%%%%%%%%%%%%
\section{Introduction}

\LaTeX{} provides a mechanism to structure a large document (such as a book)
into a main file and several child files (containing the chapters)
using the |\include| command.
This mechanism is beneficial for documents
which span hundreds of pages in order to
make the source file(s) more manageable.
Moreover, compilation can be restricted to
selected child files by means of the |\includeonly| command.
The latter feature can be used to reduce the compilation time while editing
(this was significantly more useful in the earlier days of \LaTeX{})
or to generate a smaller document which is easier to navigate.
Another application of |\includeonly| is to generate
documents consisting of selected parts of the complete document.

However, there are a few drawbacks of the plain |\include| mechanism:
\begin{itemize}
\item
The child files cannot be compiled on their own,
they can only be compiled via the main file.
A naive editing environment
(such as a text editor with an option
to have the current file processed by \LaTeX)
may require one to switch to the main file before compiling;
attempting to compile the child file produces errors.
\item
The main file must be modified (each time)
to adjust the |\includeonly| command
to the present needs. This easily leaves the main file in a messy state.
\item
The generated document will always carry the filename
of the main document. This is inconvenient if
several child files are to be compiled and
to be kept for distribution.
\end{itemize}

The present package provides a simple interface
to make child files individually compilable by \LaTeX{}.
Compiling a child file then has the same effect as compiling
the main file with an |\includeonly| command
to select the appropriate child.
Moreover the generated document will carry the name of the child
rather than the main file.
This resolves all three above issues.

This feature is meant to make the editing of books,
thesis documents and lecture notes somewhat more convenient.
However, the package can also be used efficiently for
composing a series of documents (such as exercise sheets)
which are typically distributed individually.
It then assists the author in generating the individual documents
(potentially in different versions)
as well as a document containing the collected series.
Another application is in developing style files
or other kinds of included material
where compilation of the style file could redirect
to a sample or test file.

%%%%%%%%%%%%%%%%%%%%%%%%%%%%%%%%%%%%%%%%%%%%%%%%%%%%%%%%%%%%%%%%%%%%%%%%%%%%%%%%
%%%%%%%%%%%%%%%%%%%%%%%%%%%%%%%%%%%%%%%%%%%%%%%%%%%%%%%%%%%%%%%%%%%%%%%%%%%%%%%%
\section{Usage}

First of all, the package \textsf{childdoc} is \emph{not} a standard
\LaTeXe{} |.sty| style file! Therefore it needs to be invoked in
a non-standard way.

%%%%%%%%%%%%%%%%%%%%%%%%%%%%%%%%%%%%%%%%%%%%%%%%%%%%%%%%%%%%%%%%%%%%%%%%%%%%%%%%
\subsection{Included Files}
\label{sec:include}

%%%%%%%%%%%%%%%%%%%%%%%%%%%%%%%%%%%%%%%%
\DescribeMacro{\childdocmain}
To use the package, add the commands
\begin{center}
\begin{tabular}{l}
|\input{childdoc.def}|\\
|\childdocmain{}|\\
\end{tabular}
\end{center}
at the very top of the main \LaTeX{} file,
in particular \emph{before} the |\documentclass| statement!
The argument of |\childdocmain| should be left empty
(but it must be present).

%%%%%%%%%%%%%%%%%%%%%%%%%%%%%%%%%%%%%%%%
\DescribeMacro{\childdocof}
Furthermore, add the commands
\begin{center}
\begin{tabular}{l}
|\input{childdoc.def}|\\
|\childdocof{|\textit{main}|}|\\
\end{tabular}
\end{center}
at the top of every child file \textit{child}
which is included by |\include{|\textit{child}|}|
from within the main file
(or at least for those files to be compiled individually).
The argument \textit{main} must be the filename of the main file.

There are a couple of
considerations in setting up the main and child documents:

%%%%%%%%%%%%%%%%%%%%%%%%%%%%%%%%%%%%%%%%
\paragraph{Restrictions.}

Please note the following restrictions:
\begin{itemize}
\item
|\childdocmain| must be called with one argument \textit{main}
to ensure compatibility with earlier version of the package.
It must either be empty (|\childdocmain{}|)
or precisely match the filename of the main file in which it is specified.
See \secref{sec:detection} for further information.
\item
The filename \textit{main} must be specified without the |.tex| extension.
\item
The filename \textit{main} is case sensitive
(even in case-insensitive file systems)
due to internal string comparison.
\item
The argument \textit{main} should be fully expanded, it cannot be a macro.
\item
Subdirectories and special characters should be avoided in filenames.
\item
The command |\childdocmain{|\textit{main}|}| must be followed by a whitespace.
It should not be followed immediately by another command
or by a comment mark `|%|'.
This is because the \TeX{} parser reads the token immediately following
the argument of |\childdocmain| and puts it
at the beginning of every child section;
however, a white\-space is ignored.
\end{itemize}

%%%%%%%%%%%%%%%%%%%%%%%%%%%%%%%%%%%%%%%%
\paragraph{Content of Main File.}

It is advisable to place all content in the child files included by |\include|.
Any output contained in the main file will appear in all child documents
unless suppressed manually;
it cannot be suppressed automatically by the |\includeonly| directive
and thus should normally be avoided.
A method to include some content in the main file
by means of conditional processing is described in \secref{sec:conditional}.

%%%%%%%%%%%%%%%%%%%%%%%%%%%%%%%%%%%%%%%%
\paragraph{Page Numbering.}

When only a part of the document is compiled,
the appropriate numbering of pages
(as well as other status parameters)
is determined from the |.aux| files.
The latter contain information from previous passes.
However this information needs to propagate through
all intermediate child documents.
Therefore the page numbering in child documents may well
be inconsistent until the complete document is compiled at least once.

A useful (if unconventional) way to always ensure a consistent
page numbering is to restart the numbering in each child document
and denote the pages by `\textit{child}|.|\textit{page}'
where \textit{child} represents the chapter/section number of the child file.
This can be achieved by the command
|\numberwithin{page}{|\textit{child}|}|
of the \textsf{amsmath} package
where \textit{child} can be |chapter| or |section|
depending on the chosen structuring.
Alternatively, one can modify the macro |\thepage| appropriately
and reset the counter |page| at the start of each child file.

%%%%%%%%%%%%%%%%%%%%%%%%%%%%%%%%%%%%%%%%%%%%%%%%%%%%%%%%%%%%%%%%%%%%%%%%%%%%%%%%
\subsection{Conditional Processing}
\label{sec:conditional}

The package provides a mechanism to compile different versions
of a document. To customise the versions further some conditional processing
can come in handy to distinguish which version is being compiled.
The package provides two macros to describe the compilation context:

%%%%%%%%%%%%%%%%%%%%%%%%%%%%%%%%%%%%%%%%
\DescribeMacro{\ifchilddoc}
The conditional |\ifchilddoc| distinguishes between the compilation of
child documents and the main document:
%
\begin{center}
|\ifchilddoc |\textit{child-code}| |[|\||else |\textit{main-code}]| \||fi|
\end{center}

%%%%%%%%%%%%%%%%%%%%%%%%%%%%%%%%%%%%%%%%
\DescribeMacro{\childdocname}
\DescribeMacro{\childdocjob}
The macro |\childdocname| contains the filename (without extension)
of the main or child file being processed.
Note that |\childdocjob| will always contain the name of the main file.

%%%%%%%%%%%%%%%%%%%%%%%%%%%%%%%%%%%%%%%%
\paragraph{Title Page.}

Conditional processing can be used to include a title or banner page
in the main document when proper precautions are taken.
Importantly, the code in the main file should ensure that the page counter
(as well as other status parameters which are stored in the |.aux| files)
takes the same value after the conditional processing.
Otherwise the page numbers may take divergent values
depending on which part is compiled.

For example, a title page could be declared by:
%
\begin{center}
\begin{tabular}{l}
|\ifchilddoc\||else|\\
|\addtocounter{page}{-1}|\\
\textit{code for title page}\\
|\newpage|\\
|\||fi|
\end{tabular}
\end{center}
%
A banner page for the child documents can be generated by:
%
\begin{center}
\begin{tabular}{l}
|\ifchilddoc|\\
|\addtocounter{page}{-1}|\\
\textit{code for banner page}\\
|\newpage|\\
|\||fi|
\end{tabular}
\end{center}
%
Here one could write a message such as:
\begin{center}
|This is the part \childdocname{} of \childdocjob{}.|
\end{center}

%%%%%%%%%%%%%%%%%%%%%%%%%%%%%%%%%%%%%%%%%%%%%%%%%%%%%%%%%%%%%%%%%%%%%%%%%%%%%%%%
\subsection{Flags}
\label{sec:flags}

The package makes it easy to generate different versions
of the main or child documents.
To this end compilation flags can be defined
and assigned different default values.
They will be particularly useful in conjunction
with the forwarding mechanism described in \secref{sec:forward}.

For example, it may be useful to have a flag |\version|
which can be set to |draft| or |final|.
The document source will contain some conditional code
depending on the value of |\version|.
Suppose further, the flag should default to |final| for the main file
and to |draft| for child files
which is a natural assignment for editing the document.
This is achieved by placing the following code
in the preamble of the main document
(below the |\childdocmain| directive):
%
\begin{center}
\begin{tabular}{l}
|\ifchilddoc|\\
|\providecommand{\version}{draft}|\\
|\||else|\\
|\providecommand{\version}{final}|\\
|\||fi|
\end{tabular}
\end{center}
%
The definition by |\providecommand| makes sure
that previous definitions are not overwritten.
Further statements |\providecommand{\version}{...}|
can thus be added before the above code to override it.

For the main file, one might add a line
(between |\childdocmain| and the above block)
%
\begin{center}
|%\ifchilddoc\||else\providecommand{\version}{draft}\||fi|
\end{center}
%
which can be uncommented to produce a draft version.
Likewise one can add a line to the very top of a child file
(above the |\childdocof{|\textit{main}|}| directive)
%
\begin{center}
|%\providecommand{\version}{final}|
\end{center}
%
which can be uncommented to produce the final version of this child document.

%%%%%%%%%%%%%%%%%%%%%%%%%%%%%%%%%%%%%%%%%%%%%%%%%%%%%%%%%%%%%%%%%%%%%%%%%%%%%%%%
\subsection{Forwarding}
\label{sec:forward}

Different versions of the main or child documents
using compilation flags as described in \secref{sec:flags}
can be (permanently) stored in different files
for convenient compilation, viewing and distribution.
To this end, the package defines a command
to pass on compilation to a different file:

%%%%%%%%%%%%%%%%%%%%%%%%%%%%%%%%%%%%%%%%
\DescribeMacro{\childdocforward}
The command |\childdocforward| redirects processing to
another source file:
%
\begin{center}
\begin{tabular}{l}
|\input{childdoc.def}|\\
|\childdocforward[|\textit{main}|]{|\textit{dest}|}|\\
\end{tabular}
\end{center}
%
The argument \textit{dest} is the destination file
(without extension).
It should be the main file or one of the child files.
Note that further \textsf{childdoc} directives
such as |\childdocof| and |\childdocforward|
in the indicated file will be processed in this form.
The optional argument \textit{main}
passes on directly to the main file \textit{main}
while pretending to compile the child \textit{dest}.
This form behaves as if \textit{dest}
issues |\childdocof{|\textit{main}|}| right away,
and no further \textsf{childdoc} directives will be processed.

%%%%%%%%%%%%%%%%%%%%%%%%%%%%%%%%%%%%%%%%
\DescribeMacro{\...prefix}
In the alternative form |\childdocforwardprefix|,
%
\begin{center}
\begin{tabular}{l}
|\input{childdoc.def}|\\
|\childdocforwardprefix[|\textit{main}|]{|\textit{prefix}|}{|\textit{dest}|}|
\end{tabular}
\end{center}
%
the destination file is determined by a pattern
depending on the current file:
To make this work, the current file must be called
`{\textit{prefix}\hspace{0.2em}\textit{suffix}}'
with \textit{prefix} matching precisely the argument.
Processing is then passed on to the file
`{\textit{dest}\hspace{0.2em}\textit{suffix}}'.
Surely, the same effect is achieved by
directly specifying the
argument `{\textit{dest}\hspace{0.2em}\textit{suffix}}'
in the first form.
However, that requires to set up a different file
for each child. With the alternative form of the command
all these files can have exactly the same content
which simplifies setting them up and maintaining them.

For example, the following file |draft.tex|
with a compilation flag |\version| as described in \secref{sec:flags}
compiles the main document as a draft:
%
\begin{center}
\begin{tabular}{l}
|\def\version{draft}|\\
|\input{childdoc.def}|\\
|\childdocforward{|\textit{main}|}|
\end{tabular}
\end{center}
%
Likewise, the following files |final|\textit{nn}|.tex|
compile the final version of the child document
|child|\textit{nn}|.tex|:
%
\begin{center}
\begin{tabular}{l}
|\def\version{final}|\\
|\input{childdoc.def}|\\
|\childdocforwardprefix{final}{child}|
\end{tabular}
\end{center}
%

Note that when several versions of a main file and/or of each child file
are to be generated, it may be convenient to set up a |Makefile| or
shell script to automatise the process.

%%%%%%%%%%%%%%%%%%%%%%%%%%%%%%%%%%%%%%%%%%%%%%%%%%%%%%%%%%%%%%%%%%%%%%%%%%%%%%%%
\subsection{Command Line Processing}
\label{sec:commandline}

The effect of redirection files can also be achieved by invoking
the \LaTeX{} compiler with a more elaborate command line.
Most conveniently this should be done as part
of a shell script or a |Makefile|.

When using \textsf{childdoc} in the main file, the following
command lines effectively perform a redirection
(note that depending on the shell being used,
backslashes may have to be doubled: `|\|' $\to$ `|\\|'):
%
\begin{center}
|... -jobname "|\textit{target}|" |\\|"|[\textit{flags}]%
|\input{childdoc.def}\childdocforward[|\textit{main}|]{|\textit{dest}|}"|
\end{center}
%
Here \textit{target} is the name of the output file,
\textit{main} is the name of the main file
and \textit{dest} is the name of the main or child file to be processed
(all filenames without extensions).
The optional argument \textit{main} can be omitted
if \textit{main} matches \textit{dest}.
Optionally, compilation \textit{flags} can be defined via |\def| commands.
This command line makes the \TeX{} engine believe
it is compiling the file \textit{target}
whose content is specified as the latter parameter.
The provided code then forwards the processing to
\textit{main} or \textit{dest} as described in \secref{sec:forward}.

%%%%%%%%%%%%%%%%%%%%%%%%%%%%%%%%%%%%%%%%%%%%%%%%%%%%%%%%%%%%%%%%%%%%%%%%%%%%%%%%
\subsection{Include by Input}
\label{sec:input}

Including child documents by |\include| has some restrictions by design.
Most notably, the content of a child document always occupies
its own set of pages; pages cannot be shared between child documents.
Usually, this behaviour makes perfect sense
because each child document contain an essential part of the document.
However, in some situations it may be desirable to compose
a document from a collection of parts
without having mandatory page breaks between then.
For this case, the package
provides a mechanism to include parts
by |\input| which can also be processed individually.
However, by construction this mechanism
requires manual handling of the content to be output.

%%%%%%%%%%%%%%%%%%%%%%%%%%%%%%%%%%%%%%%%
\DescribeMacro{\ifchilddocmanual}
The main file should be prepared as usual, see \secref{sec:include}.
However, the document body must make a distinction
between processing of an individual part and of the main document, e.g.:
%
\begin{center}
\begin{tabular}{l}
|\ifchilddocmanual|\\
|\input{\childdocname}|\\
|\||else|\\
\textit{document body with }|\input{|\textit{part}|}|\\
|\||fi|
\end{tabular}
\end{center}
%
The conditional |\ifchilddocmanual| is true whenever
a part to be included by |\input| is being compiled,
and the name of the part is stored in |\childdocname|.

%%%%%%%%%%%%%%%%%%%%%%%%%%%%%%%%%%%%%%%%
\DescribeMacro{\childdocby}
Each part to be included by |\input| should start with:
%
\begin{center}
\begin{tabular}{l}
|\input{childdoc.def}|\\
|\childdocby{|\textit{main}|}|\\
\end{tabular}
\end{center}
%
The directive |\childdocby| is similar to |\childdocof|
described in \secref{sec:include},
but the subsequent selection of content must be done manually.
To that end, both |\ifchilddoc| and |\ifchilddocmanual|
will be true upon processing of a part,
and the name of the part is stored in |\childdocname|.
Note that |\jobname| will be set to the filename of the current part
so that each part receives an individual |.aux| file
that does not interfere with the |.aux| file(s) of the main document.
This behaviour can be altered by the alternative form
|\childdocby[*]{|\textit{main}|}| (with a non-empty optional argument)
which uses the |.aux| file of the main document
by setting |\jobname| to \textit{main}.

%%%%%%%%%%%%%%%%%%%%%%%%%%%%%%%%%%%%%%%%%%%%%%%%%%%%%%%%%%%%%%%%%%%%%%%%%%%%%%%%
\subsection{Driver Development}
\label{sec:driver}

The \textsf{childdoc} mechanism can also be use for the development
of definition files such as \LaTeX{} styles or classes.
This case differs from the above setup with multiple parts
included by |\include| in that no |\includeonly| should be invoked.
This can be achieved by starting the include file
(before |\ProvidesPackage|) with:
%
\begin{center}
\begin{tabular}{l}
|\input{childdoc.def}|\\
|\childdocforward{|\textit{main}|}|\\
\end{tabular}
\end{center}
%
or alternatively with:
%
\begin{center}
\begin{tabular}{l}
|\input{childdoc.def}|\\
|\childdocby{|\textit{main}|}|\\
\end{tabular}
\end{center}
%
Both forms have slightly different effects as described above.
The main file is prepared as usual, see \secref{sec:include}.

%%%%%%%%%%%%%%%%%%%%%%%%%%%%%%%%%%%%%%%%%%%%%%%%%%%%%%%%%%%%%%%%%%%%%%%%%%%%%%%%
\subsection{Legacy Detection}
\label{sec:detection}

The directive |\childdocmain| in the main file can detect
whether the complete document or merely a child is to be compiled
even without using the directive |\childdocof|.
This method is deprecated because it is less robust
and there is no compelling reason to use it;
it is merely provided for backward compatibility
and it may be removed in future versions.

If the detection mechanism is to be used,
it is mandatory to correctly specify
the filename of the main file as the argument of |\childdocmain|:
%
\begin{center}
\begin{tabular}{l}
|\input{childdoc.def}|\\
|\childdocmain{|\textit{main}|}|\\
\end{tabular}
\end{center}
%
If |\jobname| does not match the argument \textit{main} of |\childdocmain|,
it is assumed that |\jobname| points to the child file to be compiled.
When using |\childdocmain| with the main file specified as argument,
it suffices to start a child file
with just |\input{|\textit{main}|}|
without loading of the package and using |\childdocof|.
If instead all processing is done
with the appropriate \textsf{childdoc} directives,
the argument of \textit{main} of |\childdocmain| can be empty.

An alternative version of the command line processing described
in \secref{sec:commandline} using the detection mechanism reads:
%
\begin{center}
|... -jobname "|\textit{target}|" "|[\textit{flags}]%
[|\def\jobname{|\textit{dest}|}|]|\input{|\textit{main}|}"|
\end{center}

%%%%%%%%%%%%%%%%%%%%%%%%%%%%%%%%%%%%%%%%%%%%%%%%%%%%%%%%%%%%%%%%%%%%%%%%%%%%%%%%
\subsection{Manual Code}
\label{sec:manual}

In case one cannot be certain whether the definitions file |childdoc.def|
is installed on the target \TeX{} distribution
and one prefers not to ship it,
it is conceivable to paste a few relevant commands into the sources.

To that end, drop all statements |\input{childdoc.def}|
and perform the replacements as outlined below.
Instead of |\childdocmain{|\textit{main}|}| add the following code
to the top of the main file:
%
\begin{center}
\begin{tabular}{l}
|\||ifdefined\childdocname\endinput\||fi\newif\ifchilddoc|\\
|\edef\childdocname{\scantokens\expandafter{\jobname\noexpand}}|\\
|\def\childdocmain{|\textit{main}|}\||ifx\childdocmain\childdocname\||else|\\
|\childdoctrue\includeonly{\childdocname}\let\jobname\childdocmain\||fi|\\
\end{tabular}
\end{center}
%
Instead of |\childdocof{|\textit{main}|}| just include the main file
at the top of each child file:
%
\begin{center}
|\input{|\textit{main}|}|
\end{center}
%
A simple redirection |\childdocforward{|\textit{dest}|}| is achieved by:
%
\begin{center}
|\def\jobname{|\textit{dest}|}\input{\jobname}|
\end{center}
%
The redirection with prefix
|\childdocforwardprefix[|\textit{prefix}|]{|\textit{dest}|}|
is accomplished by:
%
\begin{center}
\begin{tabular}{l}
|{\edef\jobname{\scantokens\expandafter{\jobname\noexpand}}|\\
|\def\redirectjob |\textit{prefix}|#1~~~{\gdef\jobname{|\textit{dest}|#1}}|\\
|\expandafter\redirectjob\jobname~~~}\input{\jobname}|
\end{tabular}
\end{center}

In an alternative approach,
child documents can be compiled by a specific command line
without additional code or specific definitions:
%
\begin{center}
|... -jobname "|\textit{target}|" "|[\textit{flags}]%
|\includeonly{|\textit{dest}|}\input{|\textit{main}|}"|
\end{center}
%

%%%%%%%%%%%%%%%%%%%%%%%%%%%%%%%%%%%%%%%%%%%%%%%%%%%%%%%%%%%%%%%%%%%%%%%%%%%%%%%%
%%%%%%%%%%%%%%%%%%%%%%%%%%%%%%%%%%%%%%%%%%%%%%%%%%%%%%%%%%%%%%%%%%%%%%%%%%%%%%%%
\section{Information}

%%%%%%%%%%%%%%%%%%%%%%%%%%%%%%%%%%%%%%%%%%%%%%%%%%%%%%%%%%%%%%%%%%%%%%%%%%%%%%%%
\subsection{Copyright}

Copyright \copyright{} 2017--2018 Niklas Beisert

This work may be distributed and/or modified under the
conditions of the \LaTeX{} Project Public License, either version 1.3
of this license or (at your option) any later version.
The latest version of this license is in
  \url{http://www.latex-project.org/lppl.txt}
and version 1.3 or later is part of all distributions of \LaTeX{}
version 2005/12/01 or later.

This work has the LPPL maintenance status `maintained'.

The Current Maintainer of this work is Niklas Beisert.

This work consists of the files |README.txt|, |childdoc.ins| and |childdoc.dtx|
as well as the derived files |childdoc.def|, |cdocsamp.tex|
with |cdocsch1.tex|, |cdocsch2.tex|, |cdocspt3.tex|, |cdocspt4.tex|,
|cdocsdrf.tex|, |cdocsfn1.tex|, |cdocsfn2.tex|
as well as |childdoc.pdf|.

%%%%%%%%%%%%%%%%%%%%%%%%%%%%%%%%%%%%%%%%%%%%%%%%%%%%%%%%%%%%%%%%%%%%%%%%%%%%%%%%
\subsection{Files and Installation}

The package consists of the files:
%
\begin{center}
\begin{tabular}{ll}
    |README.txt|   & readme file \\
    |childdoc.ins| & installation file \\
    |childdoc.dtx| & source file \\
    |childdoc.def| & definition file \\
    |cdocsamp.tex| & sample main file \\
    |cdocsch1.tex| & sample include file \\
    |cdocsch2.tex| & sample include file \\
    |cdocspt3.tex| & sample part file \\
    |cdocspt4.tex| & sample part file \\
    |cdocsdrf.tex| & sample redirection file \\
    |cdocsfn1.tex| & sample redirection file \\
    |cdocsfn2.tex| & sample redirection file \\
    |childdoc.pdf| & manual
\end{tabular}
\end{center}
%
The distribution consists of the files
|README.txt|, |childdoc.ins| and |childdoc.dtx|.
%
\begin{itemize}
\item
Run (pdf)\LaTeX{} on |childdoc.dtx|
to compile the manual |childdoc.pdf| (this file).
\item
Run \LaTeX{} on |childdoc.ins| to create the definitions file |childdoc.def|
and the sample |cdocsamp.tex| with include files
|cdocsch1.tex|, |cdocsch2.tex|, |cdocspt3.tex|, |cdocspt4.tex|,
|cdocsdrf.tex|, |cdocsfn1.tex|, |cdocsfn2.tex|.
Then copy the file |childdoc.def| to an appropriate directory of your \LaTeX{}
distribution, e.g.\ \textit{texmf-root}|/tex/latex/childdoc|.
\end{itemize}

%%%%%%%%%%%%%%%%%%%%%%%%%%%%%%%%%%%%%%%%%%%%%%%%%%%%%%%%%%%%%%%%%%%%%%%%%%%%%%%%
\subsection{Related CTAN Packages}

There are several other packages which offer a similar functionality:
%
\begin{itemize}
\item
The packages
\href{http://ctan.org/pkg/docmute}{\textsf{docmute}},
\href{http://ctan.org/pkg/includex}{\textsf{includex}} and
\href{http://ctan.org/pkg/standalone}{\textsf{standalone}}
provide commands to include only the document body of
a child file thus allowing both files to be compiled individually.
\item
The packages \href{http://ctan.org/pkg/subdocs}{\textsf{subdocs}}
and \href{http://ctan.org/pkg/subfiles}{\textsf{subfiles}}
provide structures in which the main and child documents can be
encapsulated and allowing them to be compiled individually.
The inclusion mechanism is different from the conventional |\include|.
\item
The package \href{http://ctan.org/pkg/combine}{\textsf{combine}}
is an elaborate solution to combine several documents into one.
\end{itemize}
%
See also the CTAN topic \href{http://ctan.org/topic/subdocs}{\textsf{subdocs}}
for further related packages.
The present package differs from the above solutions in that
a document structure constructed with the conventional |\include| mechanism
just needs two extra commands at the top of every file
such that all constituent files can be compiled individually.

%%%%%%%%%%%%%%%%%%%%%%%%%%%%%%%%%%%%%%%%%%%%%%%%%%%%%%%%%%%%%%%%%%%%%%%%%%%%%%%%
%\subsection{Feature Suggestions}
%
%The following is a list of features which may be useful for future
%versions of this package:
%%
%\begin{itemize}
%\item
%\ldots
%\end{itemize}

%%%%%%%%%%%%%%%%%%%%%%%%%%%%%%%%%%%%%%%%%%%%%%%%%%%%%%%%%%%%%%%%%%%%%%%%%%%%%%%%
\subsection{Revision History}

%%%%%%%%%%%%%%%%%%%%%%%%%%%%%%%%%%%%%%%%
\paragraph{v2.0:} 2018/12/30

\begin{itemize}
\item
immediate forward processing
\item
added |\childdocby| mechanism
\item
manual restructured
\end{itemize}

%%%%%%%%%%%%%%%%%%%%%%%%%%%%%%%%%%%%%%%%
\paragraph{v1.6:} 2018/01/17

\begin{itemize}
\item
application for development of include files
\item
corrections to manual
\end{itemize}

%%%%%%%%%%%%%%%%%%%%%%%%%%%%%%%%%%%%%%%%
\paragraph{v1.5:} 2017/05/21

\begin{itemize}
\item
more complete structuring introduced
\item
|\childdocof| introduced
\item
|\childdoc| renamed to |\childdocmain|
\item
|\childredirect| renamed to |\childdocforward| and |\childdocforwardprefix|
and functionality expanded
\end{itemize}

%%%%%%%%%%%%%%%%%%%%%%%%%%%%%%%%%%%%%%%%
\paragraph{v1.0:} 2017/04/27

\begin{itemize}
\item
manual and install package
\item
first version published on CTAN
\end{itemize}

%%%%%%%%%%%%%%%%%%%%%%%%%%%%%%%%%%%%%%%%
\paragraph{v0.6:} 2017/04/26

\begin{itemize}
\item
redirection mechanism added
\end{itemize}

%%%%%%%%%%%%%%%%%%%%%%%%%%%%%%%%%%%%%%%%
\paragraph{v0.5:} 2017/04/26

\begin{itemize}
\item
functionality in definition file
\end{itemize}


%%%%%%%%%%%%%%%%%%%%%%%%%%%%%%%%%%%%%%%%%%%%%%%%%%%%%%%%%%%%%%%%%%%%%%%%%%%%%%%%
%%%%%%%%%%%%%%%%%%%%%%%%%%%%%%%%%%%%%%%%%%%%%%%%%%%%%%%%%%%%%%%%%%%%%%%%%%%%%%%%
%%%%%%%%%%%%%%%%%%%%%%%%%%%%%%%%%%%%%%%%%%%%%%%%%%%%%%%%%%%%%%%%%%%%%%%%%%%%%%%%
\appendix

\settowidth\MacroIndent{\rmfamily\scriptsize 000\ }

 \DocInput{childdoc.dtx}

\end{document}
%</driver>
% \fi
%
% %%%%%%%%%%%%%%%%%%%%%%%%%%%%%%%%%%%%%%%%%%%%%%%%%%%%%%%%%%%%%%%%%%%%%%%%%%%%%%
% %%%%%%%%%%%%%%%%%%%%%%%%%%%%%%%%%%%%%%%%%%%%%%%%%%%%%%%%%%%%%%%%%%%%%%%%%%%%%%
% \section{Sample}
%\iffalse
%<*samplemain>
%\fi
%
% The following presents a sample document
% with two chapters, two parts, a title page,
% a compile flag as well as three forwarding files to set the flag.
% It consists of eight |.tex| files:
% \begin{center}
% \begin{tabular}{ll}
% |cdocsamp.tex|&main file\\
% |cdocsch1.tex|&include file for chapter 1\\
% |cdocsch2.tex|&include file for chapter 2\\
% |cdocspt3.tex|&include file for part 3\\
% |cdocspt4.tex|&include file for part 4\\
% |cdocsdrf.tex|&forwarding file for main file in draft mode\\
% |cdocsfi1.tex|&forwarding file for final version of chapter 1\\
% |cdocsfi2.tex|&forwarding file for final version of chapter 2\\
% \end{tabular}
% \end{center}
% Each of the eight files can be compiled directly by the \LaTeX{} compiler.
%
% %%%%%%%%%%%%%%%%%%%%%%%%%%%%%%%%%%%%%%
% \paragraph{Main File.}
%
% The main file is called |cdocsamp.tex|.
%
% Load the \textsf{childdoc} definitions and
% declare the filename for the main document:
%    \begin{macrocode}
\input{childdoc.def}
\childdocmain{}
%    \end{macrocode}

% Optional override for |\version| flag:
%    \begin{macrocode}
%%\ifchilddoc\else\providecommand{\version}{draft}\fi
%    \end{macrocode}

% Define the default values for the |\version| flag
% (|final| for the main file and |draft| for childs):
%    \begin{macrocode}
\ifchilddoc
\providecommand{\version}{draft}
\else
\providecommand{\version}{final}
\fi
%    \end{macrocode}

% Load the standard document class:
%    \begin{macrocode}
\documentclass[12pt]{article}
%    \end{macrocode}

% Start the document body:
%    \begin{macrocode}
\begin{document}
%    \end{macrocode}

% Declare a title page.
% Print title, part of document being processed and version flag:
%    \begin{macrocode}
\addtocounter{page}{-1}
\begin{center}
{\LARGE\bfseries{}childdoc example\par}
\vspace{1cm}
\ifchilddoc
\ifchilddocmanual part\else chapter\fi:
`\childdocname' of `\childdocjob'\par
\else
main document: `\childdocjob'\par
\fi
version: \version\par
\end{center}
\newpage
%    \end{macrocode}

% Manually include selected file,
% otherwise process as usual:
%    \begin{macrocode}
\ifchilddocmanual
\section*{part `\childdocname'}
\input{\childdocname}
\else
%    \end{macrocode}

% Include the two chapters:
%    \begin{macrocode}
\include{cdocsch1}
\include{cdocsch2}
%    \end{macrocode}

% Include the two parts unless only chapters should be displayed:
%    \begin{macrocode}
\ifchilddoc\else
\section{part three}
\input{cdocspt3}
\section{part four}
\input{cdocspt4}
\fi
%    \end{macrocode}

% Process as usual until here:
%    \begin{macrocode}
\fi
%    \end{macrocode}

% End of document body:
%    \begin{macrocode}
\end{document}
%    \end{macrocode}
%\iffalse
%</samplemain>
%\fi
%
% %%%%%%%%%%%%%%%%%%%%%%%%%%%%%%%%%%%%%%
% \paragraph{Chapter Include Files.}
%
% The include files are called |cdocsch1.tex| and |cdocsch2.tex|.
%
%\iffalse
%<*samplechap1|samplechap2>
%\fi

% Optional override for |\version| flag:
%    \begin{macrocode}
%%\providecommand{\version}{final}
%    \end{macrocode}

% Include the main document:
%    \begin{macrocode}
\input{childdoc.def}
\childdocof{cdocsamp}
%    \end{macrocode}

%\iffalse
%</samplechap1|samplechap2>
%\fi
%
%\iffalse
%<*samplechap1>
%\fi
% Some text for chapter 1:
%    \begin{macrocode}
\section{one}
some text in chapter one
%    \end{macrocode}

%\iffalse
%</samplechap1>
%\fi
% Some text for chapter 2:
%\iffalse
%<*samplechap2>
%\fi
%    \begin{macrocode}
\section{two}
more text in chapter two
%    \end{macrocode}

%\iffalse
%</samplechap2>
%\fi
%
% %%%%%%%%%%%%%%%%%%%%%%%%%%%%%%%%%%%%%%
% \paragraph{Part Include Files.}
%
% The include files are called |cdocspt3.tex| and |cdocspt4.tex|.
%
%\iffalse
%<*samplepart3|samplepart4>
%\fi

% Optional override for |\version| flag:
%    \begin{macrocode}
%%\providecommand{\version}{final}
%    \end{macrocode}

% Include the main document:
%    \begin{macrocode}
\input{childdoc.def}
\childdocby{cdocsamp}
%    \end{macrocode}

%\iffalse
%</samplepart3|samplepart4>
%\fi
%
%\iffalse
%<*samplepart3>
%\fi
% Some text for part 3:
%    \begin{macrocode}
some text in part three
%    \end{macrocode}

%\iffalse
%</samplepart3>
%\fi
% Some text for part 4:
%\iffalse
%<*samplepart4>
%\fi
%    \begin{macrocode}
more text in part four
%    \end{macrocode}

%\iffalse
%</samplepart4>
%\fi
%
% %%%%%%%%%%%%%%%%%%%%%%%%%%%%%%%%%%%%%%
% \paragraph{Forwarding for a Complete Draft.}
%
% The following forwarding file |cdocsdrf.tex|
% compiles the main document in draft mode:
%\iffalse
%<*sampledraft>
%\fi
%    \begin{macrocode}
\def\version{draft}
\input{childdoc.def}
\childdocforward{cdocsamp}
%    \end{macrocode}

%\iffalse
%</sampledraft>
%\fi
%
% %%%%%%%%%%%%%%%%%%%%%%%%%%%%%%%%%%%%%%
% \paragraph{Forwarding for Final Version of the Chapters.}
%
% The following forwarding files |cdocsfn1.tex| and |cdocsfn2.tex|
% (with identical content)
% compile the final versions of the child documents
% |cdocsch1.tex| and |cdocsch2.tex|, respectively:
%\iffalse
%<*samplefinal>
%\fi
%    \begin{macrocode}
\def\version{final}
\input{childdoc.def}
\childdocforwardprefix[cdocsamp]{cdocsfn}{cdocsch}
%    \end{macrocode}

%\iffalse
%</samplefinal>
%\fi
%
% %%%%%%%%%%%%%%%%%%%%%%%%%%%%%%%%%%%%%%
% \paragraph{Command Line Processing.}
%
% The following three command lines generate the output files
% |cdocscld|, |cdocscl1| and |cdocscl2|
% which should be identical to
% |cdocsdrf|, |cdocsch1| and |cdocsfn2|, respectively:
% \begin{center}
% \begin{tabular}{l}
% |latex -jobname cdocscld \|\\
% |  "\def\version{draft}\input{childdoc.def}\childdocforward{cdocsamp}"|\\
% |latex -jobname cdocscl1 \|\\
% |  "\input{childdoc.def}\childdocforward[cdocsamp]{cdocsch1}"|\\
% |latex -jobname cdocscl2 \|\\
% |  "\def\version{final}\input{childdoc.def}\childdocforward{cdocsch2}"|
% \end{tabular}
% \end{center}
% Note that the trailing backslash on each first line
% merely continues the input to the second line
% (for convenient cut ant paste).
% Furthermore, the command |latex| can be replaced by any
% of its alternative versions such as |pdflatex|.
%
% %%%%%%%%%%%%%%%%%%%%%%%%%%%%%%%%%%%%%%%%%%%%%%%%%%%%%%%%%%%%%%%%%%%%%%%%%%%%%%
% %%%%%%%%%%%%%%%%%%%%%%%%%%%%%%%%%%%%%%%%%%%%%%%%%%%%%%%%%%%%%%%%%%%%%%%%%%%%%%
% \section{Implementation}
%\iffalse
%<*package>
%\fi
%
% This section describes the definitions file |childdoc.def|.

% The definitions cannot be loaded using |\usepackage| or |\RequirePackage|
% which has a mechanism to prevent loading a style file more than once.
% When loading the definitions by means of |\input|
% multiple instances have to be prevented manually:
%\iffalse
%This code needs to be before the `\ProvidesFile' directive
%which is defined at the beginning of this file.
%Therefore it is also placed there and commented out here.
%</package>
%<*discard>
%\fi
%    \begin{macrocode}
\ifdefined\childdocmain\endinput\fi
%    \end{macrocode}
%\iffalse
%</discard>
%<*package>
%\fi
%
% \macro{\ifchilddoc}
% \macro{\ifchilddocmanual}
% The conditional |\ifchilddoc| tells whether a
% child (true) or main (false) document is being compiled.
% The conditional |\ifchilddocmanual| tells whether
% the |\includeonly| mechanism is used (false) or
% the selection of child files must be performed manually (true).
% The definitions initialise to false:
%    \begin{macrocode}
\newif\ifchilddoc
\newif\ifchilddocmanual
%    \end{macrocode}

% \macro{\childdocname}
% \macro{\childdocjob}
% The macro |\childdocname| stores the name of the main document
% to be compiled. The macro |\childdocjob| stores the name of
% the document on which the \LaTeX{} compiler was originally invoked.
% The content of |\jobname| cannot be compared
% to filenames specified in the source due to different catcodes.
% The following code rescans |\jobname|, stores the result
% in |\childdocname| and saves a copy in |\childdocjob|:
%    \begin{macrocode}
\edef\childdocname{\scantokens\expandafter{\jobname\noexpand}}
\let\childdocjob\childdocname
%    \end{macrocode}

% \macro{\childdocdisable}
% The macro |\childdocdisable| prevents the main file
% from being processed more than once.
% At this stage, the main document command |\childdocmain|
% is assumed to be called once again where it should do nothing.
% Any subsequent call to it should prevent
% a secondary processing of the main document
% It overwrites the forwarding commands
% |\childdocof| and |\childdocforward|
% with empty macros to prevent further inclusions of the main document:
%    \begin{macrocode}
\newcommand{\childdocdisable}
{
  \renewcommand{\childdocmain}[1]{\renewcommand{\childdocmain}[1]{\endinput}}
  \renewcommand{\childdocof}[1]{}
  \renewcommand{\childdocby}[2][]{}
  \renewcommand{\childdocforward}[2][]{}
  \renewcommand{\childdocdisable}{}
}
%    \end{macrocode}

% \macro{\childdocmain}
% The macro |\childdocmain| is to be called at the top of the main file
% with nothing or the main filename (without extension) as argument.
% First, it breaks loops.
% If the argument is not empty and does not match |\childdocname|
% (which is set by the first inclusion of |childdoc.def|),
% |\ifchilddoc| is set to true, |\includeonly| is applied to the child file
% and |\jobname| is set to the main file
% (for proper handling of |.aux| files):
%    \begin{macrocode}
\newcommand{\childdocmain}[1]
{
  \childdocdisable\childdocmain{}
  \if?#1?\else
    \begingroup
      \def\childdoctmp{#1}
      \ifx\childdoctmp\childdocname
        \def\childdoctmp{}
      \else
        \def\childdoctmp
        {
          \childdoctrue
          \includeonly{\childdocname}
          \def\childdocjob{#1}
          \def\jobname{#1}
        }
      \fi
      \expandafter
    \endgroup
    \childdoctmp
  \fi
}
%    \end{macrocode}

% \macro{\childdocof}
% The command |\childdocof| redirects
% compilation to the main file |#1|.
%    \begin{macrocode}
\newcommand{\childdocof}[1]
{
  \childdocdisable
  \childdoctrue
  \includeonly{\childdocname}
  \def\jobname{#1}
  \def\childdocjob{#1}
  \input{#1}
}
%    \end{macrocode}

% \macro{\childdocby}
% The command |\childdocby| ....
%    \begin{macrocode}
\newcommand{\childdocby}[2][]
{
  \childdocdisable
  \childdoctrue
  \childdocmanualtrue
  \if?#1?\else
    \def\jobname{#2}
  \fi
  \def\childdocjob{#2}
  \input{#2}
  \endinput
}
%    \end{macrocode}

% \macro{\childdocforward}
% The command |\childdocforward| redirects
% compilation to the main file or
% (if the optional argument is given) a child file.
% Parameters are set as if the main file
% or a child file starting with |\childdocof| was compiled.
% Then compilation is handed over to the main file:
%    \begin{macrocode}
\newcommand{\childdocforward}[2][]
{
  \begingroup
    \if?#1?
      \def\childdoctmp
      {
        \def\childdocname{#2}
        \def\childdocjob{#2}
        \def\jobname{#2}
        \input{#2}
        \endinput
      }
    \else
      \def\childdoctmp
      {
        \childdocdisable
        \def\childdocname{#2}
        \childdoctrue
        \includeonly{#2}
        \def\childdocjob{#1}
        \def\jobname{#1}
        \input{#1}
        \endinput
      }
    \fi
    \expandafter
  \endgroup
  \childdoctmp
}
%    \end{macrocode}

% \macro{\childdocforwardprefix}
% The command |\childdocforwardprefix| redirects
% compilation to the main or a child file by means of a pattern.
% The prefix |#1| in the current filename is replaced by |#2|
% and the suffix of the current filename is kept
% (it is assumed that the filename does not contain the substring `|~~~|'
% which is used as a delimiter).
% Compilation is handed over to the new file by |\childdocforward|:
%    \begin{macrocode}
\newcommand{\childdocforwardprefix}[3][]
{
  \begingroup
    \def\childdocextract #2##1~~~{\def\childdoctmp{\childdocforward[#1]{#3##1}}}
    \expandafter\childdocextract\childdocname~~~
    \expandafter
  \endgroup
  \childdoctmp
}
%    \end{macrocode}

% \macro{\childdoc}
% The deprecated macro |\childdoc| is a legacy version of |\childdocmain|:
%    \begin{macrocode}
\newcommand{\childdoc}{\childdocmain}
%    \end{macrocode}

% \macro{\childdocredirect}
% The deprecated macro |\childdocredirect| is a legacy version
% of |\childdocforward| and |\childdocforwardprefix|:
%    \begin{macrocode}
\newcommand{\childdocredirect}[2][]
{
  \begingroup
    \if?#1?
      \def\childdoctmp{\childdocforward{#2}}
    \else
      \def\childdoctmp{\childdocforwardprefix{#1}{#2}}
    \fi
    \expandafter
  \endgroup
  \childdoctmp
}
%    \end{macrocode}

%\iffalse
%</package>
%\fi
%
\endinput

\childdocforward{cdocsamp}
%    \end{macrocode}

%\iffalse
%</sampledraft>
%\fi
%
% %%%%%%%%%%%%%%%%%%%%%%%%%%%%%%%%%%%%%%
% \paragraph{Forwarding for Final Version of the Chapters.}
%
% The following forwarding files |cdocsfn1.tex| and |cdocsfn2.tex|
% (with identical content)
% compile the final versions of the child documents
% |cdocsch1.tex| and |cdocsch2.tex|, respectively:
%\iffalse
%<*samplefinal>
%\fi
%    \begin{macrocode}
\def\version{final}
% \iffalse
%
% childdoc.dtx Copyright (C) 2017-2018 Niklas Beisert
%
% This work may be distributed and/or modified under the
% conditions of the LaTeX Project Public License, either version 1.3
% of this license or (at your option) any later version.
% The latest version of this license is in
%   http://www.latex-project.org/lppl.txt
% and version 1.3 or later is part of all distributions of LaTeX
% version 2005/12/01 or later.
%
% This work has the LPPL maintenance status `maintained'.
%
% The Current Maintainer of this work is Niklas Beisert.
%
% This work consists of the files childdoc.dtx and childdoc.ins
% and the derived files childdoc.def and cdocsamp.tex with
% cdocsch1.tex, cdocsch2.tex, cdocsdrf.tex, cdocsfn1.tex, cdocsfn2.tex.
%
%<package>\ifdefined\childdocmain\endinput\fi
%<package>\ProvidesFile{childdoc.def}[2018/12/30 v2.0 child document driver]
%<samplemain>\ProvidesFile{cdocsamp.tex}[2018/12/30 v2.0 sample for childdoc]
%<*driver>
%\ProvidesFile{childdoc.drv}[2018/12/30 v2.0 childdoc reference manual file]
\PassOptionsToClass{10pt,a4paper}{article}
\documentclass{ltxdoc}

\usepackage[margin=35mm]{geometry}
\usepackage{hyperref}
\usepackage{hyperxmp}
\usepackage[usenames]{color}

\hypersetup{colorlinks=true}
\hypersetup{pdfstartview=FitH}
\hypersetup{pdfpagemode=UseNone}
\hypersetup{pdfsource={}}
\hypersetup{pdflang={en-UK}}
\hypersetup{pdfcopyright={Copyright 2017-2018 Niklas Beisert.
  This work may be distributed and/or modified under the
  conditions of the LaTeX Project Public License, either version 1.3
  of this license or (at your option) any later version.}}
\hypersetup{pdflicenseurl={http://www.latex-project.org/lppl.txt}}
\hypersetup{pdfcontactaddress={ETH Zurich, ITP, HIT K,
  Wolfgang-Pauli-Strasse 27}}
\hypersetup{pdfcontactpostcode={8093}}
\hypersetup{pdfcontactcity={Zurich}}
\hypersetup{pdfcontactcountry={Switzerland}}
\hypersetup{pdfcontactemail={nbeisert@itp.phys.ethz.ch}}
\hypersetup{pdfcontacturl={http://people.phys.ethz.ch/\xmptilde nbeisert/}}

\newcommand{\secref}[1]{\hyperref[#1]{section \ref*{#1}}}

\parskip1ex
\parindent0pt
\let\olditemize\itemize
\def\itemize{\olditemize\parskip0pt}

\begin{document}

\title{The \textsf{childdoc} Package}
\hypersetup{pdftitle={The childdoc Package}}
\author{Niklas Beisert\\[2ex]
  Institut f\"ur Theoretische Physik\\
  Eidgen\"ossische Technische Hochschule Z\"urich\\
  Wolfgang-Pauli-Strasse 27, 8093 Z\"urich, Switzerland\\[1ex]
  \href{mailto:nbeisert@itp.phys.ethz.ch}
  {\texttt{nbeisert@itp.phys.ethz.ch}}}
\hypersetup{pdfauthor={Niklas Beisert}}
\hypersetup{pdfsubject={Manual for the LaTeX2e Package childdoc}}
\date{30 December 2018, \textsf{v2.0}}
\maketitle

\begin{abstract}\noindent
\textsf{childdoc} is a \LaTeXe{} package
that enables the direct compilation
of document sections included by |\include|
to individual files.
\end{abstract}

\begingroup
\parskip0ex
\tableofcontents
\endgroup

%%%%%%%%%%%%%%%%%%%%%%%%%%%%%%%%%%%%%%%%%%%%%%%%%%%%%%%%%%%%%%%%%%%%%%%%%%%%%%%%
%%%%%%%%%%%%%%%%%%%%%%%%%%%%%%%%%%%%%%%%%%%%%%%%%%%%%%%%%%%%%%%%%%%%%%%%%%%%%%%%
\section{Introduction}

\LaTeX{} provides a mechanism to structure a large document (such as a book)
into a main file and several child files (containing the chapters)
using the |\include| command.
This mechanism is beneficial for documents
which span hundreds of pages in order to
make the source file(s) more manageable.
Moreover, compilation can be restricted to
selected child files by means of the |\includeonly| command.
The latter feature can be used to reduce the compilation time while editing
(this was significantly more useful in the earlier days of \LaTeX{})
or to generate a smaller document which is easier to navigate.
Another application of |\includeonly| is to generate
documents consisting of selected parts of the complete document.

However, there are a few drawbacks of the plain |\include| mechanism:
\begin{itemize}
\item
The child files cannot be compiled on their own,
they can only be compiled via the main file.
A naive editing environment
(such as a text editor with an option
to have the current file processed by \LaTeX)
may require one to switch to the main file before compiling;
attempting to compile the child file produces errors.
\item
The main file must be modified (each time)
to adjust the |\includeonly| command
to the present needs. This easily leaves the main file in a messy state.
\item
The generated document will always carry the filename
of the main document. This is inconvenient if
several child files are to be compiled and
to be kept for distribution.
\end{itemize}

The present package provides a simple interface
to make child files individually compilable by \LaTeX{}.
Compiling a child file then has the same effect as compiling
the main file with an |\includeonly| command
to select the appropriate child.
Moreover the generated document will carry the name of the child
rather than the main file.
This resolves all three above issues.

This feature is meant to make the editing of books,
thesis documents and lecture notes somewhat more convenient.
However, the package can also be used efficiently for
composing a series of documents (such as exercise sheets)
which are typically distributed individually.
It then assists the author in generating the individual documents
(potentially in different versions)
as well as a document containing the collected series.
Another application is in developing style files
or other kinds of included material
where compilation of the style file could redirect
to a sample or test file.

%%%%%%%%%%%%%%%%%%%%%%%%%%%%%%%%%%%%%%%%%%%%%%%%%%%%%%%%%%%%%%%%%%%%%%%%%%%%%%%%
%%%%%%%%%%%%%%%%%%%%%%%%%%%%%%%%%%%%%%%%%%%%%%%%%%%%%%%%%%%%%%%%%%%%%%%%%%%%%%%%
\section{Usage}

First of all, the package \textsf{childdoc} is \emph{not} a standard
\LaTeXe{} |.sty| style file! Therefore it needs to be invoked in
a non-standard way.

%%%%%%%%%%%%%%%%%%%%%%%%%%%%%%%%%%%%%%%%%%%%%%%%%%%%%%%%%%%%%%%%%%%%%%%%%%%%%%%%
\subsection{Included Files}
\label{sec:include}

%%%%%%%%%%%%%%%%%%%%%%%%%%%%%%%%%%%%%%%%
\DescribeMacro{\childdocmain}
To use the package, add the commands
\begin{center}
\begin{tabular}{l}
|\input{childdoc.def}|\\
|\childdocmain{}|\\
\end{tabular}
\end{center}
at the very top of the main \LaTeX{} file,
in particular \emph{before} the |\documentclass| statement!
The argument of |\childdocmain| should be left empty
(but it must be present).

%%%%%%%%%%%%%%%%%%%%%%%%%%%%%%%%%%%%%%%%
\DescribeMacro{\childdocof}
Furthermore, add the commands
\begin{center}
\begin{tabular}{l}
|\input{childdoc.def}|\\
|\childdocof{|\textit{main}|}|\\
\end{tabular}
\end{center}
at the top of every child file \textit{child}
which is included by |\include{|\textit{child}|}|
from within the main file
(or at least for those files to be compiled individually).
The argument \textit{main} must be the filename of the main file.

There are a couple of
considerations in setting up the main and child documents:

%%%%%%%%%%%%%%%%%%%%%%%%%%%%%%%%%%%%%%%%
\paragraph{Restrictions.}

Please note the following restrictions:
\begin{itemize}
\item
|\childdocmain| must be called with one argument \textit{main}
to ensure compatibility with earlier version of the package.
It must either be empty (|\childdocmain{}|)
or precisely match the filename of the main file in which it is specified.
See \secref{sec:detection} for further information.
\item
The filename \textit{main} must be specified without the |.tex| extension.
\item
The filename \textit{main} is case sensitive
(even in case-insensitive file systems)
due to internal string comparison.
\item
The argument \textit{main} should be fully expanded, it cannot be a macro.
\item
Subdirectories and special characters should be avoided in filenames.
\item
The command |\childdocmain{|\textit{main}|}| must be followed by a whitespace.
It should not be followed immediately by another command
or by a comment mark `|%|'.
This is because the \TeX{} parser reads the token immediately following
the argument of |\childdocmain| and puts it
at the beginning of every child section;
however, a white\-space is ignored.
\end{itemize}

%%%%%%%%%%%%%%%%%%%%%%%%%%%%%%%%%%%%%%%%
\paragraph{Content of Main File.}

It is advisable to place all content in the child files included by |\include|.
Any output contained in the main file will appear in all child documents
unless suppressed manually;
it cannot be suppressed automatically by the |\includeonly| directive
and thus should normally be avoided.
A method to include some content in the main file
by means of conditional processing is described in \secref{sec:conditional}.

%%%%%%%%%%%%%%%%%%%%%%%%%%%%%%%%%%%%%%%%
\paragraph{Page Numbering.}

When only a part of the document is compiled,
the appropriate numbering of pages
(as well as other status parameters)
is determined from the |.aux| files.
The latter contain information from previous passes.
However this information needs to propagate through
all intermediate child documents.
Therefore the page numbering in child documents may well
be inconsistent until the complete document is compiled at least once.

A useful (if unconventional) way to always ensure a consistent
page numbering is to restart the numbering in each child document
and denote the pages by `\textit{child}|.|\textit{page}'
where \textit{child} represents the chapter/section number of the child file.
This can be achieved by the command
|\numberwithin{page}{|\textit{child}|}|
of the \textsf{amsmath} package
where \textit{child} can be |chapter| or |section|
depending on the chosen structuring.
Alternatively, one can modify the macro |\thepage| appropriately
and reset the counter |page| at the start of each child file.

%%%%%%%%%%%%%%%%%%%%%%%%%%%%%%%%%%%%%%%%%%%%%%%%%%%%%%%%%%%%%%%%%%%%%%%%%%%%%%%%
\subsection{Conditional Processing}
\label{sec:conditional}

The package provides a mechanism to compile different versions
of a document. To customise the versions further some conditional processing
can come in handy to distinguish which version is being compiled.
The package provides two macros to describe the compilation context:

%%%%%%%%%%%%%%%%%%%%%%%%%%%%%%%%%%%%%%%%
\DescribeMacro{\ifchilddoc}
The conditional |\ifchilddoc| distinguishes between the compilation of
child documents and the main document:
%
\begin{center}
|\ifchilddoc |\textit{child-code}| |[|\||else |\textit{main-code}]| \||fi|
\end{center}

%%%%%%%%%%%%%%%%%%%%%%%%%%%%%%%%%%%%%%%%
\DescribeMacro{\childdocname}
\DescribeMacro{\childdocjob}
The macro |\childdocname| contains the filename (without extension)
of the main or child file being processed.
Note that |\childdocjob| will always contain the name of the main file.

%%%%%%%%%%%%%%%%%%%%%%%%%%%%%%%%%%%%%%%%
\paragraph{Title Page.}

Conditional processing can be used to include a title or banner page
in the main document when proper precautions are taken.
Importantly, the code in the main file should ensure that the page counter
(as well as other status parameters which are stored in the |.aux| files)
takes the same value after the conditional processing.
Otherwise the page numbers may take divergent values
depending on which part is compiled.

For example, a title page could be declared by:
%
\begin{center}
\begin{tabular}{l}
|\ifchilddoc\||else|\\
|\addtocounter{page}{-1}|\\
\textit{code for title page}\\
|\newpage|\\
|\||fi|
\end{tabular}
\end{center}
%
A banner page for the child documents can be generated by:
%
\begin{center}
\begin{tabular}{l}
|\ifchilddoc|\\
|\addtocounter{page}{-1}|\\
\textit{code for banner page}\\
|\newpage|\\
|\||fi|
\end{tabular}
\end{center}
%
Here one could write a message such as:
\begin{center}
|This is the part \childdocname{} of \childdocjob{}.|
\end{center}

%%%%%%%%%%%%%%%%%%%%%%%%%%%%%%%%%%%%%%%%%%%%%%%%%%%%%%%%%%%%%%%%%%%%%%%%%%%%%%%%
\subsection{Flags}
\label{sec:flags}

The package makes it easy to generate different versions
of the main or child documents.
To this end compilation flags can be defined
and assigned different default values.
They will be particularly useful in conjunction
with the forwarding mechanism described in \secref{sec:forward}.

For example, it may be useful to have a flag |\version|
which can be set to |draft| or |final|.
The document source will contain some conditional code
depending on the value of |\version|.
Suppose further, the flag should default to |final| for the main file
and to |draft| for child files
which is a natural assignment for editing the document.
This is achieved by placing the following code
in the preamble of the main document
(below the |\childdocmain| directive):
%
\begin{center}
\begin{tabular}{l}
|\ifchilddoc|\\
|\providecommand{\version}{draft}|\\
|\||else|\\
|\providecommand{\version}{final}|\\
|\||fi|
\end{tabular}
\end{center}
%
The definition by |\providecommand| makes sure
that previous definitions are not overwritten.
Further statements |\providecommand{\version}{...}|
can thus be added before the above code to override it.

For the main file, one might add a line
(between |\childdocmain| and the above block)
%
\begin{center}
|%\ifchilddoc\||else\providecommand{\version}{draft}\||fi|
\end{center}
%
which can be uncommented to produce a draft version.
Likewise one can add a line to the very top of a child file
(above the |\childdocof{|\textit{main}|}| directive)
%
\begin{center}
|%\providecommand{\version}{final}|
\end{center}
%
which can be uncommented to produce the final version of this child document.

%%%%%%%%%%%%%%%%%%%%%%%%%%%%%%%%%%%%%%%%%%%%%%%%%%%%%%%%%%%%%%%%%%%%%%%%%%%%%%%%
\subsection{Forwarding}
\label{sec:forward}

Different versions of the main or child documents
using compilation flags as described in \secref{sec:flags}
can be (permanently) stored in different files
for convenient compilation, viewing and distribution.
To this end, the package defines a command
to pass on compilation to a different file:

%%%%%%%%%%%%%%%%%%%%%%%%%%%%%%%%%%%%%%%%
\DescribeMacro{\childdocforward}
The command |\childdocforward| redirects processing to
another source file:
%
\begin{center}
\begin{tabular}{l}
|\input{childdoc.def}|\\
|\childdocforward[|\textit{main}|]{|\textit{dest}|}|\\
\end{tabular}
\end{center}
%
The argument \textit{dest} is the destination file
(without extension).
It should be the main file or one of the child files.
Note that further \textsf{childdoc} directives
such as |\childdocof| and |\childdocforward|
in the indicated file will be processed in this form.
The optional argument \textit{main}
passes on directly to the main file \textit{main}
while pretending to compile the child \textit{dest}.
This form behaves as if \textit{dest}
issues |\childdocof{|\textit{main}|}| right away,
and no further \textsf{childdoc} directives will be processed.

%%%%%%%%%%%%%%%%%%%%%%%%%%%%%%%%%%%%%%%%
\DescribeMacro{\...prefix}
In the alternative form |\childdocforwardprefix|,
%
\begin{center}
\begin{tabular}{l}
|\input{childdoc.def}|\\
|\childdocforwardprefix[|\textit{main}|]{|\textit{prefix}|}{|\textit{dest}|}|
\end{tabular}
\end{center}
%
the destination file is determined by a pattern
depending on the current file:
To make this work, the current file must be called
`{\textit{prefix}\hspace{0.2em}\textit{suffix}}'
with \textit{prefix} matching precisely the argument.
Processing is then passed on to the file
`{\textit{dest}\hspace{0.2em}\textit{suffix}}'.
Surely, the same effect is achieved by
directly specifying the
argument `{\textit{dest}\hspace{0.2em}\textit{suffix}}'
in the first form.
However, that requires to set up a different file
for each child. With the alternative form of the command
all these files can have exactly the same content
which simplifies setting them up and maintaining them.

For example, the following file |draft.tex|
with a compilation flag |\version| as described in \secref{sec:flags}
compiles the main document as a draft:
%
\begin{center}
\begin{tabular}{l}
|\def\version{draft}|\\
|\input{childdoc.def}|\\
|\childdocforward{|\textit{main}|}|
\end{tabular}
\end{center}
%
Likewise, the following files |final|\textit{nn}|.tex|
compile the final version of the child document
|child|\textit{nn}|.tex|:
%
\begin{center}
\begin{tabular}{l}
|\def\version{final}|\\
|\input{childdoc.def}|\\
|\childdocforwardprefix{final}{child}|
\end{tabular}
\end{center}
%

Note that when several versions of a main file and/or of each child file
are to be generated, it may be convenient to set up a |Makefile| or
shell script to automatise the process.

%%%%%%%%%%%%%%%%%%%%%%%%%%%%%%%%%%%%%%%%%%%%%%%%%%%%%%%%%%%%%%%%%%%%%%%%%%%%%%%%
\subsection{Command Line Processing}
\label{sec:commandline}

The effect of redirection files can also be achieved by invoking
the \LaTeX{} compiler with a more elaborate command line.
Most conveniently this should be done as part
of a shell script or a |Makefile|.

When using \textsf{childdoc} in the main file, the following
command lines effectively perform a redirection
(note that depending on the shell being used,
backslashes may have to be doubled: `|\|' $\to$ `|\\|'):
%
\begin{center}
|... -jobname "|\textit{target}|" |\\|"|[\textit{flags}]%
|\input{childdoc.def}\childdocforward[|\textit{main}|]{|\textit{dest}|}"|
\end{center}
%
Here \textit{target} is the name of the output file,
\textit{main} is the name of the main file
and \textit{dest} is the name of the main or child file to be processed
(all filenames without extensions).
The optional argument \textit{main} can be omitted
if \textit{main} matches \textit{dest}.
Optionally, compilation \textit{flags} can be defined via |\def| commands.
This command line makes the \TeX{} engine believe
it is compiling the file \textit{target}
whose content is specified as the latter parameter.
The provided code then forwards the processing to
\textit{main} or \textit{dest} as described in \secref{sec:forward}.

%%%%%%%%%%%%%%%%%%%%%%%%%%%%%%%%%%%%%%%%%%%%%%%%%%%%%%%%%%%%%%%%%%%%%%%%%%%%%%%%
\subsection{Include by Input}
\label{sec:input}

Including child documents by |\include| has some restrictions by design.
Most notably, the content of a child document always occupies
its own set of pages; pages cannot be shared between child documents.
Usually, this behaviour makes perfect sense
because each child document contain an essential part of the document.
However, in some situations it may be desirable to compose
a document from a collection of parts
without having mandatory page breaks between then.
For this case, the package
provides a mechanism to include parts
by |\input| which can also be processed individually.
However, by construction this mechanism
requires manual handling of the content to be output.

%%%%%%%%%%%%%%%%%%%%%%%%%%%%%%%%%%%%%%%%
\DescribeMacro{\ifchilddocmanual}
The main file should be prepared as usual, see \secref{sec:include}.
However, the document body must make a distinction
between processing of an individual part and of the main document, e.g.:
%
\begin{center}
\begin{tabular}{l}
|\ifchilddocmanual|\\
|\input{\childdocname}|\\
|\||else|\\
\textit{document body with }|\input{|\textit{part}|}|\\
|\||fi|
\end{tabular}
\end{center}
%
The conditional |\ifchilddocmanual| is true whenever
a part to be included by |\input| is being compiled,
and the name of the part is stored in |\childdocname|.

%%%%%%%%%%%%%%%%%%%%%%%%%%%%%%%%%%%%%%%%
\DescribeMacro{\childdocby}
Each part to be included by |\input| should start with:
%
\begin{center}
\begin{tabular}{l}
|\input{childdoc.def}|\\
|\childdocby{|\textit{main}|}|\\
\end{tabular}
\end{center}
%
The directive |\childdocby| is similar to |\childdocof|
described in \secref{sec:include},
but the subsequent selection of content must be done manually.
To that end, both |\ifchilddoc| and |\ifchilddocmanual|
will be true upon processing of a part,
and the name of the part is stored in |\childdocname|.
Note that |\jobname| will be set to the filename of the current part
so that each part receives an individual |.aux| file
that does not interfere with the |.aux| file(s) of the main document.
This behaviour can be altered by the alternative form
|\childdocby[*]{|\textit{main}|}| (with a non-empty optional argument)
which uses the |.aux| file of the main document
by setting |\jobname| to \textit{main}.

%%%%%%%%%%%%%%%%%%%%%%%%%%%%%%%%%%%%%%%%%%%%%%%%%%%%%%%%%%%%%%%%%%%%%%%%%%%%%%%%
\subsection{Driver Development}
\label{sec:driver}

The \textsf{childdoc} mechanism can also be use for the development
of definition files such as \LaTeX{} styles or classes.
This case differs from the above setup with multiple parts
included by |\include| in that no |\includeonly| should be invoked.
This can be achieved by starting the include file
(before |\ProvidesPackage|) with:
%
\begin{center}
\begin{tabular}{l}
|\input{childdoc.def}|\\
|\childdocforward{|\textit{main}|}|\\
\end{tabular}
\end{center}
%
or alternatively with:
%
\begin{center}
\begin{tabular}{l}
|\input{childdoc.def}|\\
|\childdocby{|\textit{main}|}|\\
\end{tabular}
\end{center}
%
Both forms have slightly different effects as described above.
The main file is prepared as usual, see \secref{sec:include}.

%%%%%%%%%%%%%%%%%%%%%%%%%%%%%%%%%%%%%%%%%%%%%%%%%%%%%%%%%%%%%%%%%%%%%%%%%%%%%%%%
\subsection{Legacy Detection}
\label{sec:detection}

The directive |\childdocmain| in the main file can detect
whether the complete document or merely a child is to be compiled
even without using the directive |\childdocof|.
This method is deprecated because it is less robust
and there is no compelling reason to use it;
it is merely provided for backward compatibility
and it may be removed in future versions.

If the detection mechanism is to be used,
it is mandatory to correctly specify
the filename of the main file as the argument of |\childdocmain|:
%
\begin{center}
\begin{tabular}{l}
|\input{childdoc.def}|\\
|\childdocmain{|\textit{main}|}|\\
\end{tabular}
\end{center}
%
If |\jobname| does not match the argument \textit{main} of |\childdocmain|,
it is assumed that |\jobname| points to the child file to be compiled.
When using |\childdocmain| with the main file specified as argument,
it suffices to start a child file
with just |\input{|\textit{main}|}|
without loading of the package and using |\childdocof|.
If instead all processing is done
with the appropriate \textsf{childdoc} directives,
the argument of \textit{main} of |\childdocmain| can be empty.

An alternative version of the command line processing described
in \secref{sec:commandline} using the detection mechanism reads:
%
\begin{center}
|... -jobname "|\textit{target}|" "|[\textit{flags}]%
[|\def\jobname{|\textit{dest}|}|]|\input{|\textit{main}|}"|
\end{center}

%%%%%%%%%%%%%%%%%%%%%%%%%%%%%%%%%%%%%%%%%%%%%%%%%%%%%%%%%%%%%%%%%%%%%%%%%%%%%%%%
\subsection{Manual Code}
\label{sec:manual}

In case one cannot be certain whether the definitions file |childdoc.def|
is installed on the target \TeX{} distribution
and one prefers not to ship it,
it is conceivable to paste a few relevant commands into the sources.

To that end, drop all statements |\input{childdoc.def}|
and perform the replacements as outlined below.
Instead of |\childdocmain{|\textit{main}|}| add the following code
to the top of the main file:
%
\begin{center}
\begin{tabular}{l}
|\||ifdefined\childdocname\endinput\||fi\newif\ifchilddoc|\\
|\edef\childdocname{\scantokens\expandafter{\jobname\noexpand}}|\\
|\def\childdocmain{|\textit{main}|}\||ifx\childdocmain\childdocname\||else|\\
|\childdoctrue\includeonly{\childdocname}\let\jobname\childdocmain\||fi|\\
\end{tabular}
\end{center}
%
Instead of |\childdocof{|\textit{main}|}| just include the main file
at the top of each child file:
%
\begin{center}
|\input{|\textit{main}|}|
\end{center}
%
A simple redirection |\childdocforward{|\textit{dest}|}| is achieved by:
%
\begin{center}
|\def\jobname{|\textit{dest}|}\input{\jobname}|
\end{center}
%
The redirection with prefix
|\childdocforwardprefix[|\textit{prefix}|]{|\textit{dest}|}|
is accomplished by:
%
\begin{center}
\begin{tabular}{l}
|{\edef\jobname{\scantokens\expandafter{\jobname\noexpand}}|\\
|\def\redirectjob |\textit{prefix}|#1~~~{\gdef\jobname{|\textit{dest}|#1}}|\\
|\expandafter\redirectjob\jobname~~~}\input{\jobname}|
\end{tabular}
\end{center}

In an alternative approach,
child documents can be compiled by a specific command line
without additional code or specific definitions:
%
\begin{center}
|... -jobname "|\textit{target}|" "|[\textit{flags}]%
|\includeonly{|\textit{dest}|}\input{|\textit{main}|}"|
\end{center}
%

%%%%%%%%%%%%%%%%%%%%%%%%%%%%%%%%%%%%%%%%%%%%%%%%%%%%%%%%%%%%%%%%%%%%%%%%%%%%%%%%
%%%%%%%%%%%%%%%%%%%%%%%%%%%%%%%%%%%%%%%%%%%%%%%%%%%%%%%%%%%%%%%%%%%%%%%%%%%%%%%%
\section{Information}

%%%%%%%%%%%%%%%%%%%%%%%%%%%%%%%%%%%%%%%%%%%%%%%%%%%%%%%%%%%%%%%%%%%%%%%%%%%%%%%%
\subsection{Copyright}

Copyright \copyright{} 2017--2018 Niklas Beisert

This work may be distributed and/or modified under the
conditions of the \LaTeX{} Project Public License, either version 1.3
of this license or (at your option) any later version.
The latest version of this license is in
  \url{http://www.latex-project.org/lppl.txt}
and version 1.3 or later is part of all distributions of \LaTeX{}
version 2005/12/01 or later.

This work has the LPPL maintenance status `maintained'.

The Current Maintainer of this work is Niklas Beisert.

This work consists of the files |README.txt|, |childdoc.ins| and |childdoc.dtx|
as well as the derived files |childdoc.def|, |cdocsamp.tex|
with |cdocsch1.tex|, |cdocsch2.tex|, |cdocspt3.tex|, |cdocspt4.tex|,
|cdocsdrf.tex|, |cdocsfn1.tex|, |cdocsfn2.tex|
as well as |childdoc.pdf|.

%%%%%%%%%%%%%%%%%%%%%%%%%%%%%%%%%%%%%%%%%%%%%%%%%%%%%%%%%%%%%%%%%%%%%%%%%%%%%%%%
\subsection{Files and Installation}

The package consists of the files:
%
\begin{center}
\begin{tabular}{ll}
    |README.txt|   & readme file \\
    |childdoc.ins| & installation file \\
    |childdoc.dtx| & source file \\
    |childdoc.def| & definition file \\
    |cdocsamp.tex| & sample main file \\
    |cdocsch1.tex| & sample include file \\
    |cdocsch2.tex| & sample include file \\
    |cdocspt3.tex| & sample part file \\
    |cdocspt4.tex| & sample part file \\
    |cdocsdrf.tex| & sample redirection file \\
    |cdocsfn1.tex| & sample redirection file \\
    |cdocsfn2.tex| & sample redirection file \\
    |childdoc.pdf| & manual
\end{tabular}
\end{center}
%
The distribution consists of the files
|README.txt|, |childdoc.ins| and |childdoc.dtx|.
%
\begin{itemize}
\item
Run (pdf)\LaTeX{} on |childdoc.dtx|
to compile the manual |childdoc.pdf| (this file).
\item
Run \LaTeX{} on |childdoc.ins| to create the definitions file |childdoc.def|
and the sample |cdocsamp.tex| with include files
|cdocsch1.tex|, |cdocsch2.tex|, |cdocspt3.tex|, |cdocspt4.tex|,
|cdocsdrf.tex|, |cdocsfn1.tex|, |cdocsfn2.tex|.
Then copy the file |childdoc.def| to an appropriate directory of your \LaTeX{}
distribution, e.g.\ \textit{texmf-root}|/tex/latex/childdoc|.
\end{itemize}

%%%%%%%%%%%%%%%%%%%%%%%%%%%%%%%%%%%%%%%%%%%%%%%%%%%%%%%%%%%%%%%%%%%%%%%%%%%%%%%%
\subsection{Related CTAN Packages}

There are several other packages which offer a similar functionality:
%
\begin{itemize}
\item
The packages
\href{http://ctan.org/pkg/docmute}{\textsf{docmute}},
\href{http://ctan.org/pkg/includex}{\textsf{includex}} and
\href{http://ctan.org/pkg/standalone}{\textsf{standalone}}
provide commands to include only the document body of
a child file thus allowing both files to be compiled individually.
\item
The packages \href{http://ctan.org/pkg/subdocs}{\textsf{subdocs}}
and \href{http://ctan.org/pkg/subfiles}{\textsf{subfiles}}
provide structures in which the main and child documents can be
encapsulated and allowing them to be compiled individually.
The inclusion mechanism is different from the conventional |\include|.
\item
The package \href{http://ctan.org/pkg/combine}{\textsf{combine}}
is an elaborate solution to combine several documents into one.
\end{itemize}
%
See also the CTAN topic \href{http://ctan.org/topic/subdocs}{\textsf{subdocs}}
for further related packages.
The present package differs from the above solutions in that
a document structure constructed with the conventional |\include| mechanism
just needs two extra commands at the top of every file
such that all constituent files can be compiled individually.

%%%%%%%%%%%%%%%%%%%%%%%%%%%%%%%%%%%%%%%%%%%%%%%%%%%%%%%%%%%%%%%%%%%%%%%%%%%%%%%%
%\subsection{Feature Suggestions}
%
%The following is a list of features which may be useful for future
%versions of this package:
%%
%\begin{itemize}
%\item
%\ldots
%\end{itemize}

%%%%%%%%%%%%%%%%%%%%%%%%%%%%%%%%%%%%%%%%%%%%%%%%%%%%%%%%%%%%%%%%%%%%%%%%%%%%%%%%
\subsection{Revision History}

%%%%%%%%%%%%%%%%%%%%%%%%%%%%%%%%%%%%%%%%
\paragraph{v2.0:} 2018/12/30

\begin{itemize}
\item
immediate forward processing
\item
added |\childdocby| mechanism
\item
manual restructured
\end{itemize}

%%%%%%%%%%%%%%%%%%%%%%%%%%%%%%%%%%%%%%%%
\paragraph{v1.6:} 2018/01/17

\begin{itemize}
\item
application for development of include files
\item
corrections to manual
\end{itemize}

%%%%%%%%%%%%%%%%%%%%%%%%%%%%%%%%%%%%%%%%
\paragraph{v1.5:} 2017/05/21

\begin{itemize}
\item
more complete structuring introduced
\item
|\childdocof| introduced
\item
|\childdoc| renamed to |\childdocmain|
\item
|\childredirect| renamed to |\childdocforward| and |\childdocforwardprefix|
and functionality expanded
\end{itemize}

%%%%%%%%%%%%%%%%%%%%%%%%%%%%%%%%%%%%%%%%
\paragraph{v1.0:} 2017/04/27

\begin{itemize}
\item
manual and install package
\item
first version published on CTAN
\end{itemize}

%%%%%%%%%%%%%%%%%%%%%%%%%%%%%%%%%%%%%%%%
\paragraph{v0.6:} 2017/04/26

\begin{itemize}
\item
redirection mechanism added
\end{itemize}

%%%%%%%%%%%%%%%%%%%%%%%%%%%%%%%%%%%%%%%%
\paragraph{v0.5:} 2017/04/26

\begin{itemize}
\item
functionality in definition file
\end{itemize}


%%%%%%%%%%%%%%%%%%%%%%%%%%%%%%%%%%%%%%%%%%%%%%%%%%%%%%%%%%%%%%%%%%%%%%%%%%%%%%%%
%%%%%%%%%%%%%%%%%%%%%%%%%%%%%%%%%%%%%%%%%%%%%%%%%%%%%%%%%%%%%%%%%%%%%%%%%%%%%%%%
%%%%%%%%%%%%%%%%%%%%%%%%%%%%%%%%%%%%%%%%%%%%%%%%%%%%%%%%%%%%%%%%%%%%%%%%%%%%%%%%
\appendix

\settowidth\MacroIndent{\rmfamily\scriptsize 000\ }

 \DocInput{childdoc.dtx}

\end{document}
%</driver>
% \fi
%
% %%%%%%%%%%%%%%%%%%%%%%%%%%%%%%%%%%%%%%%%%%%%%%%%%%%%%%%%%%%%%%%%%%%%%%%%%%%%%%
% %%%%%%%%%%%%%%%%%%%%%%%%%%%%%%%%%%%%%%%%%%%%%%%%%%%%%%%%%%%%%%%%%%%%%%%%%%%%%%
% \section{Sample}
%\iffalse
%<*samplemain>
%\fi
%
% The following presents a sample document
% with two chapters, two parts, a title page,
% a compile flag as well as three forwarding files to set the flag.
% It consists of eight |.tex| files:
% \begin{center}
% \begin{tabular}{ll}
% |cdocsamp.tex|&main file\\
% |cdocsch1.tex|&include file for chapter 1\\
% |cdocsch2.tex|&include file for chapter 2\\
% |cdocspt3.tex|&include file for part 3\\
% |cdocspt4.tex|&include file for part 4\\
% |cdocsdrf.tex|&forwarding file for main file in draft mode\\
% |cdocsfi1.tex|&forwarding file for final version of chapter 1\\
% |cdocsfi2.tex|&forwarding file for final version of chapter 2\\
% \end{tabular}
% \end{center}
% Each of the eight files can be compiled directly by the \LaTeX{} compiler.
%
% %%%%%%%%%%%%%%%%%%%%%%%%%%%%%%%%%%%%%%
% \paragraph{Main File.}
%
% The main file is called |cdocsamp.tex|.
%
% Load the \textsf{childdoc} definitions and
% declare the filename for the main document:
%    \begin{macrocode}
\input{childdoc.def}
\childdocmain{}
%    \end{macrocode}

% Optional override for |\version| flag:
%    \begin{macrocode}
%%\ifchilddoc\else\providecommand{\version}{draft}\fi
%    \end{macrocode}

% Define the default values for the |\version| flag
% (|final| for the main file and |draft| for childs):
%    \begin{macrocode}
\ifchilddoc
\providecommand{\version}{draft}
\else
\providecommand{\version}{final}
\fi
%    \end{macrocode}

% Load the standard document class:
%    \begin{macrocode}
\documentclass[12pt]{article}
%    \end{macrocode}

% Start the document body:
%    \begin{macrocode}
\begin{document}
%    \end{macrocode}

% Declare a title page.
% Print title, part of document being processed and version flag:
%    \begin{macrocode}
\addtocounter{page}{-1}
\begin{center}
{\LARGE\bfseries{}childdoc example\par}
\vspace{1cm}
\ifchilddoc
\ifchilddocmanual part\else chapter\fi:
`\childdocname' of `\childdocjob'\par
\else
main document: `\childdocjob'\par
\fi
version: \version\par
\end{center}
\newpage
%    \end{macrocode}

% Manually include selected file,
% otherwise process as usual:
%    \begin{macrocode}
\ifchilddocmanual
\section*{part `\childdocname'}
\input{\childdocname}
\else
%    \end{macrocode}

% Include the two chapters:
%    \begin{macrocode}
\include{cdocsch1}
\include{cdocsch2}
%    \end{macrocode}

% Include the two parts unless only chapters should be displayed:
%    \begin{macrocode}
\ifchilddoc\else
\section{part three}
\input{cdocspt3}
\section{part four}
\input{cdocspt4}
\fi
%    \end{macrocode}

% Process as usual until here:
%    \begin{macrocode}
\fi
%    \end{macrocode}

% End of document body:
%    \begin{macrocode}
\end{document}
%    \end{macrocode}
%\iffalse
%</samplemain>
%\fi
%
% %%%%%%%%%%%%%%%%%%%%%%%%%%%%%%%%%%%%%%
% \paragraph{Chapter Include Files.}
%
% The include files are called |cdocsch1.tex| and |cdocsch2.tex|.
%
%\iffalse
%<*samplechap1|samplechap2>
%\fi

% Optional override for |\version| flag:
%    \begin{macrocode}
%%\providecommand{\version}{final}
%    \end{macrocode}

% Include the main document:
%    \begin{macrocode}
\input{childdoc.def}
\childdocof{cdocsamp}
%    \end{macrocode}

%\iffalse
%</samplechap1|samplechap2>
%\fi
%
%\iffalse
%<*samplechap1>
%\fi
% Some text for chapter 1:
%    \begin{macrocode}
\section{one}
some text in chapter one
%    \end{macrocode}

%\iffalse
%</samplechap1>
%\fi
% Some text for chapter 2:
%\iffalse
%<*samplechap2>
%\fi
%    \begin{macrocode}
\section{two}
more text in chapter two
%    \end{macrocode}

%\iffalse
%</samplechap2>
%\fi
%
% %%%%%%%%%%%%%%%%%%%%%%%%%%%%%%%%%%%%%%
% \paragraph{Part Include Files.}
%
% The include files are called |cdocspt3.tex| and |cdocspt4.tex|.
%
%\iffalse
%<*samplepart3|samplepart4>
%\fi

% Optional override for |\version| flag:
%    \begin{macrocode}
%%\providecommand{\version}{final}
%    \end{macrocode}

% Include the main document:
%    \begin{macrocode}
\input{childdoc.def}
\childdocby{cdocsamp}
%    \end{macrocode}

%\iffalse
%</samplepart3|samplepart4>
%\fi
%
%\iffalse
%<*samplepart3>
%\fi
% Some text for part 3:
%    \begin{macrocode}
some text in part three
%    \end{macrocode}

%\iffalse
%</samplepart3>
%\fi
% Some text for part 4:
%\iffalse
%<*samplepart4>
%\fi
%    \begin{macrocode}
more text in part four
%    \end{macrocode}

%\iffalse
%</samplepart4>
%\fi
%
% %%%%%%%%%%%%%%%%%%%%%%%%%%%%%%%%%%%%%%
% \paragraph{Forwarding for a Complete Draft.}
%
% The following forwarding file |cdocsdrf.tex|
% compiles the main document in draft mode:
%\iffalse
%<*sampledraft>
%\fi
%    \begin{macrocode}
\def\version{draft}
\input{childdoc.def}
\childdocforward{cdocsamp}
%    \end{macrocode}

%\iffalse
%</sampledraft>
%\fi
%
% %%%%%%%%%%%%%%%%%%%%%%%%%%%%%%%%%%%%%%
% \paragraph{Forwarding for Final Version of the Chapters.}
%
% The following forwarding files |cdocsfn1.tex| and |cdocsfn2.tex|
% (with identical content)
% compile the final versions of the child documents
% |cdocsch1.tex| and |cdocsch2.tex|, respectively:
%\iffalse
%<*samplefinal>
%\fi
%    \begin{macrocode}
\def\version{final}
\input{childdoc.def}
\childdocforwardprefix[cdocsamp]{cdocsfn}{cdocsch}
%    \end{macrocode}

%\iffalse
%</samplefinal>
%\fi
%
% %%%%%%%%%%%%%%%%%%%%%%%%%%%%%%%%%%%%%%
% \paragraph{Command Line Processing.}
%
% The following three command lines generate the output files
% |cdocscld|, |cdocscl1| and |cdocscl2|
% which should be identical to
% |cdocsdrf|, |cdocsch1| and |cdocsfn2|, respectively:
% \begin{center}
% \begin{tabular}{l}
% |latex -jobname cdocscld \|\\
% |  "\def\version{draft}\input{childdoc.def}\childdocforward{cdocsamp}"|\\
% |latex -jobname cdocscl1 \|\\
% |  "\input{childdoc.def}\childdocforward[cdocsamp]{cdocsch1}"|\\
% |latex -jobname cdocscl2 \|\\
% |  "\def\version{final}\input{childdoc.def}\childdocforward{cdocsch2}"|
% \end{tabular}
% \end{center}
% Note that the trailing backslash on each first line
% merely continues the input to the second line
% (for convenient cut ant paste).
% Furthermore, the command |latex| can be replaced by any
% of its alternative versions such as |pdflatex|.
%
% %%%%%%%%%%%%%%%%%%%%%%%%%%%%%%%%%%%%%%%%%%%%%%%%%%%%%%%%%%%%%%%%%%%%%%%%%%%%%%
% %%%%%%%%%%%%%%%%%%%%%%%%%%%%%%%%%%%%%%%%%%%%%%%%%%%%%%%%%%%%%%%%%%%%%%%%%%%%%%
% \section{Implementation}
%\iffalse
%<*package>
%\fi
%
% This section describes the definitions file |childdoc.def|.

% The definitions cannot be loaded using |\usepackage| or |\RequirePackage|
% which has a mechanism to prevent loading a style file more than once.
% When loading the definitions by means of |\input|
% multiple instances have to be prevented manually:
%\iffalse
%This code needs to be before the `\ProvidesFile' directive
%which is defined at the beginning of this file.
%Therefore it is also placed there and commented out here.
%</package>
%<*discard>
%\fi
%    \begin{macrocode}
\ifdefined\childdocmain\endinput\fi
%    \end{macrocode}
%\iffalse
%</discard>
%<*package>
%\fi
%
% \macro{\ifchilddoc}
% \macro{\ifchilddocmanual}
% The conditional |\ifchilddoc| tells whether a
% child (true) or main (false) document is being compiled.
% The conditional |\ifchilddocmanual| tells whether
% the |\includeonly| mechanism is used (false) or
% the selection of child files must be performed manually (true).
% The definitions initialise to false:
%    \begin{macrocode}
\newif\ifchilddoc
\newif\ifchilddocmanual
%    \end{macrocode}

% \macro{\childdocname}
% \macro{\childdocjob}
% The macro |\childdocname| stores the name of the main document
% to be compiled. The macro |\childdocjob| stores the name of
% the document on which the \LaTeX{} compiler was originally invoked.
% The content of |\jobname| cannot be compared
% to filenames specified in the source due to different catcodes.
% The following code rescans |\jobname|, stores the result
% in |\childdocname| and saves a copy in |\childdocjob|:
%    \begin{macrocode}
\edef\childdocname{\scantokens\expandafter{\jobname\noexpand}}
\let\childdocjob\childdocname
%    \end{macrocode}

% \macro{\childdocdisable}
% The macro |\childdocdisable| prevents the main file
% from being processed more than once.
% At this stage, the main document command |\childdocmain|
% is assumed to be called once again where it should do nothing.
% Any subsequent call to it should prevent
% a secondary processing of the main document
% It overwrites the forwarding commands
% |\childdocof| and |\childdocforward|
% with empty macros to prevent further inclusions of the main document:
%    \begin{macrocode}
\newcommand{\childdocdisable}
{
  \renewcommand{\childdocmain}[1]{\renewcommand{\childdocmain}[1]{\endinput}}
  \renewcommand{\childdocof}[1]{}
  \renewcommand{\childdocby}[2][]{}
  \renewcommand{\childdocforward}[2][]{}
  \renewcommand{\childdocdisable}{}
}
%    \end{macrocode}

% \macro{\childdocmain}
% The macro |\childdocmain| is to be called at the top of the main file
% with nothing or the main filename (without extension) as argument.
% First, it breaks loops.
% If the argument is not empty and does not match |\childdocname|
% (which is set by the first inclusion of |childdoc.def|),
% |\ifchilddoc| is set to true, |\includeonly| is applied to the child file
% and |\jobname| is set to the main file
% (for proper handling of |.aux| files):
%    \begin{macrocode}
\newcommand{\childdocmain}[1]
{
  \childdocdisable\childdocmain{}
  \if?#1?\else
    \begingroup
      \def\childdoctmp{#1}
      \ifx\childdoctmp\childdocname
        \def\childdoctmp{}
      \else
        \def\childdoctmp
        {
          \childdoctrue
          \includeonly{\childdocname}
          \def\childdocjob{#1}
          \def\jobname{#1}
        }
      \fi
      \expandafter
    \endgroup
    \childdoctmp
  \fi
}
%    \end{macrocode}

% \macro{\childdocof}
% The command |\childdocof| redirects
% compilation to the main file |#1|.
%    \begin{macrocode}
\newcommand{\childdocof}[1]
{
  \childdocdisable
  \childdoctrue
  \includeonly{\childdocname}
  \def\jobname{#1}
  \def\childdocjob{#1}
  \input{#1}
}
%    \end{macrocode}

% \macro{\childdocby}
% The command |\childdocby| ....
%    \begin{macrocode}
\newcommand{\childdocby}[2][]
{
  \childdocdisable
  \childdoctrue
  \childdocmanualtrue
  \if?#1?\else
    \def\jobname{#2}
  \fi
  \def\childdocjob{#2}
  \input{#2}
  \endinput
}
%    \end{macrocode}

% \macro{\childdocforward}
% The command |\childdocforward| redirects
% compilation to the main file or
% (if the optional argument is given) a child file.
% Parameters are set as if the main file
% or a child file starting with |\childdocof| was compiled.
% Then compilation is handed over to the main file:
%    \begin{macrocode}
\newcommand{\childdocforward}[2][]
{
  \begingroup
    \if?#1?
      \def\childdoctmp
      {
        \def\childdocname{#2}
        \def\childdocjob{#2}
        \def\jobname{#2}
        \input{#2}
        \endinput
      }
    \else
      \def\childdoctmp
      {
        \childdocdisable
        \def\childdocname{#2}
        \childdoctrue
        \includeonly{#2}
        \def\childdocjob{#1}
        \def\jobname{#1}
        \input{#1}
        \endinput
      }
    \fi
    \expandafter
  \endgroup
  \childdoctmp
}
%    \end{macrocode}

% \macro{\childdocforwardprefix}
% The command |\childdocforwardprefix| redirects
% compilation to the main or a child file by means of a pattern.
% The prefix |#1| in the current filename is replaced by |#2|
% and the suffix of the current filename is kept
% (it is assumed that the filename does not contain the substring `|~~~|'
% which is used as a delimiter).
% Compilation is handed over to the new file by |\childdocforward|:
%    \begin{macrocode}
\newcommand{\childdocforwardprefix}[3][]
{
  \begingroup
    \def\childdocextract #2##1~~~{\def\childdoctmp{\childdocforward[#1]{#3##1}}}
    \expandafter\childdocextract\childdocname~~~
    \expandafter
  \endgroup
  \childdoctmp
}
%    \end{macrocode}

% \macro{\childdoc}
% The deprecated macro |\childdoc| is a legacy version of |\childdocmain|:
%    \begin{macrocode}
\newcommand{\childdoc}{\childdocmain}
%    \end{macrocode}

% \macro{\childdocredirect}
% The deprecated macro |\childdocredirect| is a legacy version
% of |\childdocforward| and |\childdocforwardprefix|:
%    \begin{macrocode}
\newcommand{\childdocredirect}[2][]
{
  \begingroup
    \if?#1?
      \def\childdoctmp{\childdocforward{#2}}
    \else
      \def\childdoctmp{\childdocforwardprefix{#1}{#2}}
    \fi
    \expandafter
  \endgroup
  \childdoctmp
}
%    \end{macrocode}

%\iffalse
%</package>
%\fi
%
\endinput

\childdocforwardprefix[cdocsamp]{cdocsfn}{cdocsch}
%    \end{macrocode}

%\iffalse
%</samplefinal>
%\fi
%
% %%%%%%%%%%%%%%%%%%%%%%%%%%%%%%%%%%%%%%
% \paragraph{Command Line Processing.}
%
% The following three command lines generate the output files
% |cdocscld|, |cdocscl1| and |cdocscl2|
% which should be identical to
% |cdocsdrf|, |cdocsch1| and |cdocsfn2|, respectively:
% \begin{center}
% \begin{tabular}{l}
% |latex -jobname cdocscld \|\\
% |  "\def\version{draft}% \iffalse
%
% childdoc.dtx Copyright (C) 2017-2018 Niklas Beisert
%
% This work may be distributed and/or modified under the
% conditions of the LaTeX Project Public License, either version 1.3
% of this license or (at your option) any later version.
% The latest version of this license is in
%   http://www.latex-project.org/lppl.txt
% and version 1.3 or later is part of all distributions of LaTeX
% version 2005/12/01 or later.
%
% This work has the LPPL maintenance status `maintained'.
%
% The Current Maintainer of this work is Niklas Beisert.
%
% This work consists of the files childdoc.dtx and childdoc.ins
% and the derived files childdoc.def and cdocsamp.tex with
% cdocsch1.tex, cdocsch2.tex, cdocsdrf.tex, cdocsfn1.tex, cdocsfn2.tex.
%
%<package>\ifdefined\childdocmain\endinput\fi
%<package>\ProvidesFile{childdoc.def}[2018/12/30 v2.0 child document driver]
%<samplemain>\ProvidesFile{cdocsamp.tex}[2018/12/30 v2.0 sample for childdoc]
%<*driver>
%\ProvidesFile{childdoc.drv}[2018/12/30 v2.0 childdoc reference manual file]
\PassOptionsToClass{10pt,a4paper}{article}
\documentclass{ltxdoc}

\usepackage[margin=35mm]{geometry}
\usepackage{hyperref}
\usepackage{hyperxmp}
\usepackage[usenames]{color}

\hypersetup{colorlinks=true}
\hypersetup{pdfstartview=FitH}
\hypersetup{pdfpagemode=UseNone}
\hypersetup{pdfsource={}}
\hypersetup{pdflang={en-UK}}
\hypersetup{pdfcopyright={Copyright 2017-2018 Niklas Beisert.
  This work may be distributed and/or modified under the
  conditions of the LaTeX Project Public License, either version 1.3
  of this license or (at your option) any later version.}}
\hypersetup{pdflicenseurl={http://www.latex-project.org/lppl.txt}}
\hypersetup{pdfcontactaddress={ETH Zurich, ITP, HIT K,
  Wolfgang-Pauli-Strasse 27}}
\hypersetup{pdfcontactpostcode={8093}}
\hypersetup{pdfcontactcity={Zurich}}
\hypersetup{pdfcontactcountry={Switzerland}}
\hypersetup{pdfcontactemail={nbeisert@itp.phys.ethz.ch}}
\hypersetup{pdfcontacturl={http://people.phys.ethz.ch/\xmptilde nbeisert/}}

\newcommand{\secref}[1]{\hyperref[#1]{section \ref*{#1}}}

\parskip1ex
\parindent0pt
\let\olditemize\itemize
\def\itemize{\olditemize\parskip0pt}

\begin{document}

\title{The \textsf{childdoc} Package}
\hypersetup{pdftitle={The childdoc Package}}
\author{Niklas Beisert\\[2ex]
  Institut f\"ur Theoretische Physik\\
  Eidgen\"ossische Technische Hochschule Z\"urich\\
  Wolfgang-Pauli-Strasse 27, 8093 Z\"urich, Switzerland\\[1ex]
  \href{mailto:nbeisert@itp.phys.ethz.ch}
  {\texttt{nbeisert@itp.phys.ethz.ch}}}
\hypersetup{pdfauthor={Niklas Beisert}}
\hypersetup{pdfsubject={Manual for the LaTeX2e Package childdoc}}
\date{30 December 2018, \textsf{v2.0}}
\maketitle

\begin{abstract}\noindent
\textsf{childdoc} is a \LaTeXe{} package
that enables the direct compilation
of document sections included by |\include|
to individual files.
\end{abstract}

\begingroup
\parskip0ex
\tableofcontents
\endgroup

%%%%%%%%%%%%%%%%%%%%%%%%%%%%%%%%%%%%%%%%%%%%%%%%%%%%%%%%%%%%%%%%%%%%%%%%%%%%%%%%
%%%%%%%%%%%%%%%%%%%%%%%%%%%%%%%%%%%%%%%%%%%%%%%%%%%%%%%%%%%%%%%%%%%%%%%%%%%%%%%%
\section{Introduction}

\LaTeX{} provides a mechanism to structure a large document (such as a book)
into a main file and several child files (containing the chapters)
using the |\include| command.
This mechanism is beneficial for documents
which span hundreds of pages in order to
make the source file(s) more manageable.
Moreover, compilation can be restricted to
selected child files by means of the |\includeonly| command.
The latter feature can be used to reduce the compilation time while editing
(this was significantly more useful in the earlier days of \LaTeX{})
or to generate a smaller document which is easier to navigate.
Another application of |\includeonly| is to generate
documents consisting of selected parts of the complete document.

However, there are a few drawbacks of the plain |\include| mechanism:
\begin{itemize}
\item
The child files cannot be compiled on their own,
they can only be compiled via the main file.
A naive editing environment
(such as a text editor with an option
to have the current file processed by \LaTeX)
may require one to switch to the main file before compiling;
attempting to compile the child file produces errors.
\item
The main file must be modified (each time)
to adjust the |\includeonly| command
to the present needs. This easily leaves the main file in a messy state.
\item
The generated document will always carry the filename
of the main document. This is inconvenient if
several child files are to be compiled and
to be kept for distribution.
\end{itemize}

The present package provides a simple interface
to make child files individually compilable by \LaTeX{}.
Compiling a child file then has the same effect as compiling
the main file with an |\includeonly| command
to select the appropriate child.
Moreover the generated document will carry the name of the child
rather than the main file.
This resolves all three above issues.

This feature is meant to make the editing of books,
thesis documents and lecture notes somewhat more convenient.
However, the package can also be used efficiently for
composing a series of documents (such as exercise sheets)
which are typically distributed individually.
It then assists the author in generating the individual documents
(potentially in different versions)
as well as a document containing the collected series.
Another application is in developing style files
or other kinds of included material
where compilation of the style file could redirect
to a sample or test file.

%%%%%%%%%%%%%%%%%%%%%%%%%%%%%%%%%%%%%%%%%%%%%%%%%%%%%%%%%%%%%%%%%%%%%%%%%%%%%%%%
%%%%%%%%%%%%%%%%%%%%%%%%%%%%%%%%%%%%%%%%%%%%%%%%%%%%%%%%%%%%%%%%%%%%%%%%%%%%%%%%
\section{Usage}

First of all, the package \textsf{childdoc} is \emph{not} a standard
\LaTeXe{} |.sty| style file! Therefore it needs to be invoked in
a non-standard way.

%%%%%%%%%%%%%%%%%%%%%%%%%%%%%%%%%%%%%%%%%%%%%%%%%%%%%%%%%%%%%%%%%%%%%%%%%%%%%%%%
\subsection{Included Files}
\label{sec:include}

%%%%%%%%%%%%%%%%%%%%%%%%%%%%%%%%%%%%%%%%
\DescribeMacro{\childdocmain}
To use the package, add the commands
\begin{center}
\begin{tabular}{l}
|\input{childdoc.def}|\\
|\childdocmain{}|\\
\end{tabular}
\end{center}
at the very top of the main \LaTeX{} file,
in particular \emph{before} the |\documentclass| statement!
The argument of |\childdocmain| should be left empty
(but it must be present).

%%%%%%%%%%%%%%%%%%%%%%%%%%%%%%%%%%%%%%%%
\DescribeMacro{\childdocof}
Furthermore, add the commands
\begin{center}
\begin{tabular}{l}
|\input{childdoc.def}|\\
|\childdocof{|\textit{main}|}|\\
\end{tabular}
\end{center}
at the top of every child file \textit{child}
which is included by |\include{|\textit{child}|}|
from within the main file
(or at least for those files to be compiled individually).
The argument \textit{main} must be the filename of the main file.

There are a couple of
considerations in setting up the main and child documents:

%%%%%%%%%%%%%%%%%%%%%%%%%%%%%%%%%%%%%%%%
\paragraph{Restrictions.}

Please note the following restrictions:
\begin{itemize}
\item
|\childdocmain| must be called with one argument \textit{main}
to ensure compatibility with earlier version of the package.
It must either be empty (|\childdocmain{}|)
or precisely match the filename of the main file in which it is specified.
See \secref{sec:detection} for further information.
\item
The filename \textit{main} must be specified without the |.tex| extension.
\item
The filename \textit{main} is case sensitive
(even in case-insensitive file systems)
due to internal string comparison.
\item
The argument \textit{main} should be fully expanded, it cannot be a macro.
\item
Subdirectories and special characters should be avoided in filenames.
\item
The command |\childdocmain{|\textit{main}|}| must be followed by a whitespace.
It should not be followed immediately by another command
or by a comment mark `|%|'.
This is because the \TeX{} parser reads the token immediately following
the argument of |\childdocmain| and puts it
at the beginning of every child section;
however, a white\-space is ignored.
\end{itemize}

%%%%%%%%%%%%%%%%%%%%%%%%%%%%%%%%%%%%%%%%
\paragraph{Content of Main File.}

It is advisable to place all content in the child files included by |\include|.
Any output contained in the main file will appear in all child documents
unless suppressed manually;
it cannot be suppressed automatically by the |\includeonly| directive
and thus should normally be avoided.
A method to include some content in the main file
by means of conditional processing is described in \secref{sec:conditional}.

%%%%%%%%%%%%%%%%%%%%%%%%%%%%%%%%%%%%%%%%
\paragraph{Page Numbering.}

When only a part of the document is compiled,
the appropriate numbering of pages
(as well as other status parameters)
is determined from the |.aux| files.
The latter contain information from previous passes.
However this information needs to propagate through
all intermediate child documents.
Therefore the page numbering in child documents may well
be inconsistent until the complete document is compiled at least once.

A useful (if unconventional) way to always ensure a consistent
page numbering is to restart the numbering in each child document
and denote the pages by `\textit{child}|.|\textit{page}'
where \textit{child} represents the chapter/section number of the child file.
This can be achieved by the command
|\numberwithin{page}{|\textit{child}|}|
of the \textsf{amsmath} package
where \textit{child} can be |chapter| or |section|
depending on the chosen structuring.
Alternatively, one can modify the macro |\thepage| appropriately
and reset the counter |page| at the start of each child file.

%%%%%%%%%%%%%%%%%%%%%%%%%%%%%%%%%%%%%%%%%%%%%%%%%%%%%%%%%%%%%%%%%%%%%%%%%%%%%%%%
\subsection{Conditional Processing}
\label{sec:conditional}

The package provides a mechanism to compile different versions
of a document. To customise the versions further some conditional processing
can come in handy to distinguish which version is being compiled.
The package provides two macros to describe the compilation context:

%%%%%%%%%%%%%%%%%%%%%%%%%%%%%%%%%%%%%%%%
\DescribeMacro{\ifchilddoc}
The conditional |\ifchilddoc| distinguishes between the compilation of
child documents and the main document:
%
\begin{center}
|\ifchilddoc |\textit{child-code}| |[|\||else |\textit{main-code}]| \||fi|
\end{center}

%%%%%%%%%%%%%%%%%%%%%%%%%%%%%%%%%%%%%%%%
\DescribeMacro{\childdocname}
\DescribeMacro{\childdocjob}
The macro |\childdocname| contains the filename (without extension)
of the main or child file being processed.
Note that |\childdocjob| will always contain the name of the main file.

%%%%%%%%%%%%%%%%%%%%%%%%%%%%%%%%%%%%%%%%
\paragraph{Title Page.}

Conditional processing can be used to include a title or banner page
in the main document when proper precautions are taken.
Importantly, the code in the main file should ensure that the page counter
(as well as other status parameters which are stored in the |.aux| files)
takes the same value after the conditional processing.
Otherwise the page numbers may take divergent values
depending on which part is compiled.

For example, a title page could be declared by:
%
\begin{center}
\begin{tabular}{l}
|\ifchilddoc\||else|\\
|\addtocounter{page}{-1}|\\
\textit{code for title page}\\
|\newpage|\\
|\||fi|
\end{tabular}
\end{center}
%
A banner page for the child documents can be generated by:
%
\begin{center}
\begin{tabular}{l}
|\ifchilddoc|\\
|\addtocounter{page}{-1}|\\
\textit{code for banner page}\\
|\newpage|\\
|\||fi|
\end{tabular}
\end{center}
%
Here one could write a message such as:
\begin{center}
|This is the part \childdocname{} of \childdocjob{}.|
\end{center}

%%%%%%%%%%%%%%%%%%%%%%%%%%%%%%%%%%%%%%%%%%%%%%%%%%%%%%%%%%%%%%%%%%%%%%%%%%%%%%%%
\subsection{Flags}
\label{sec:flags}

The package makes it easy to generate different versions
of the main or child documents.
To this end compilation flags can be defined
and assigned different default values.
They will be particularly useful in conjunction
with the forwarding mechanism described in \secref{sec:forward}.

For example, it may be useful to have a flag |\version|
which can be set to |draft| or |final|.
The document source will contain some conditional code
depending on the value of |\version|.
Suppose further, the flag should default to |final| for the main file
and to |draft| for child files
which is a natural assignment for editing the document.
This is achieved by placing the following code
in the preamble of the main document
(below the |\childdocmain| directive):
%
\begin{center}
\begin{tabular}{l}
|\ifchilddoc|\\
|\providecommand{\version}{draft}|\\
|\||else|\\
|\providecommand{\version}{final}|\\
|\||fi|
\end{tabular}
\end{center}
%
The definition by |\providecommand| makes sure
that previous definitions are not overwritten.
Further statements |\providecommand{\version}{...}|
can thus be added before the above code to override it.

For the main file, one might add a line
(between |\childdocmain| and the above block)
%
\begin{center}
|%\ifchilddoc\||else\providecommand{\version}{draft}\||fi|
\end{center}
%
which can be uncommented to produce a draft version.
Likewise one can add a line to the very top of a child file
(above the |\childdocof{|\textit{main}|}| directive)
%
\begin{center}
|%\providecommand{\version}{final}|
\end{center}
%
which can be uncommented to produce the final version of this child document.

%%%%%%%%%%%%%%%%%%%%%%%%%%%%%%%%%%%%%%%%%%%%%%%%%%%%%%%%%%%%%%%%%%%%%%%%%%%%%%%%
\subsection{Forwarding}
\label{sec:forward}

Different versions of the main or child documents
using compilation flags as described in \secref{sec:flags}
can be (permanently) stored in different files
for convenient compilation, viewing and distribution.
To this end, the package defines a command
to pass on compilation to a different file:

%%%%%%%%%%%%%%%%%%%%%%%%%%%%%%%%%%%%%%%%
\DescribeMacro{\childdocforward}
The command |\childdocforward| redirects processing to
another source file:
%
\begin{center}
\begin{tabular}{l}
|\input{childdoc.def}|\\
|\childdocforward[|\textit{main}|]{|\textit{dest}|}|\\
\end{tabular}
\end{center}
%
The argument \textit{dest} is the destination file
(without extension).
It should be the main file or one of the child files.
Note that further \textsf{childdoc} directives
such as |\childdocof| and |\childdocforward|
in the indicated file will be processed in this form.
The optional argument \textit{main}
passes on directly to the main file \textit{main}
while pretending to compile the child \textit{dest}.
This form behaves as if \textit{dest}
issues |\childdocof{|\textit{main}|}| right away,
and no further \textsf{childdoc} directives will be processed.

%%%%%%%%%%%%%%%%%%%%%%%%%%%%%%%%%%%%%%%%
\DescribeMacro{\...prefix}
In the alternative form |\childdocforwardprefix|,
%
\begin{center}
\begin{tabular}{l}
|\input{childdoc.def}|\\
|\childdocforwardprefix[|\textit{main}|]{|\textit{prefix}|}{|\textit{dest}|}|
\end{tabular}
\end{center}
%
the destination file is determined by a pattern
depending on the current file:
To make this work, the current file must be called
`{\textit{prefix}\hspace{0.2em}\textit{suffix}}'
with \textit{prefix} matching precisely the argument.
Processing is then passed on to the file
`{\textit{dest}\hspace{0.2em}\textit{suffix}}'.
Surely, the same effect is achieved by
directly specifying the
argument `{\textit{dest}\hspace{0.2em}\textit{suffix}}'
in the first form.
However, that requires to set up a different file
for each child. With the alternative form of the command
all these files can have exactly the same content
which simplifies setting them up and maintaining them.

For example, the following file |draft.tex|
with a compilation flag |\version| as described in \secref{sec:flags}
compiles the main document as a draft:
%
\begin{center}
\begin{tabular}{l}
|\def\version{draft}|\\
|\input{childdoc.def}|\\
|\childdocforward{|\textit{main}|}|
\end{tabular}
\end{center}
%
Likewise, the following files |final|\textit{nn}|.tex|
compile the final version of the child document
|child|\textit{nn}|.tex|:
%
\begin{center}
\begin{tabular}{l}
|\def\version{final}|\\
|\input{childdoc.def}|\\
|\childdocforwardprefix{final}{child}|
\end{tabular}
\end{center}
%

Note that when several versions of a main file and/or of each child file
are to be generated, it may be convenient to set up a |Makefile| or
shell script to automatise the process.

%%%%%%%%%%%%%%%%%%%%%%%%%%%%%%%%%%%%%%%%%%%%%%%%%%%%%%%%%%%%%%%%%%%%%%%%%%%%%%%%
\subsection{Command Line Processing}
\label{sec:commandline}

The effect of redirection files can also be achieved by invoking
the \LaTeX{} compiler with a more elaborate command line.
Most conveniently this should be done as part
of a shell script or a |Makefile|.

When using \textsf{childdoc} in the main file, the following
command lines effectively perform a redirection
(note that depending on the shell being used,
backslashes may have to be doubled: `|\|' $\to$ `|\\|'):
%
\begin{center}
|... -jobname "|\textit{target}|" |\\|"|[\textit{flags}]%
|\input{childdoc.def}\childdocforward[|\textit{main}|]{|\textit{dest}|}"|
\end{center}
%
Here \textit{target} is the name of the output file,
\textit{main} is the name of the main file
and \textit{dest} is the name of the main or child file to be processed
(all filenames without extensions).
The optional argument \textit{main} can be omitted
if \textit{main} matches \textit{dest}.
Optionally, compilation \textit{flags} can be defined via |\def| commands.
This command line makes the \TeX{} engine believe
it is compiling the file \textit{target}
whose content is specified as the latter parameter.
The provided code then forwards the processing to
\textit{main} or \textit{dest} as described in \secref{sec:forward}.

%%%%%%%%%%%%%%%%%%%%%%%%%%%%%%%%%%%%%%%%%%%%%%%%%%%%%%%%%%%%%%%%%%%%%%%%%%%%%%%%
\subsection{Include by Input}
\label{sec:input}

Including child documents by |\include| has some restrictions by design.
Most notably, the content of a child document always occupies
its own set of pages; pages cannot be shared between child documents.
Usually, this behaviour makes perfect sense
because each child document contain an essential part of the document.
However, in some situations it may be desirable to compose
a document from a collection of parts
without having mandatory page breaks between then.
For this case, the package
provides a mechanism to include parts
by |\input| which can also be processed individually.
However, by construction this mechanism
requires manual handling of the content to be output.

%%%%%%%%%%%%%%%%%%%%%%%%%%%%%%%%%%%%%%%%
\DescribeMacro{\ifchilddocmanual}
The main file should be prepared as usual, see \secref{sec:include}.
However, the document body must make a distinction
between processing of an individual part and of the main document, e.g.:
%
\begin{center}
\begin{tabular}{l}
|\ifchilddocmanual|\\
|\input{\childdocname}|\\
|\||else|\\
\textit{document body with }|\input{|\textit{part}|}|\\
|\||fi|
\end{tabular}
\end{center}
%
The conditional |\ifchilddocmanual| is true whenever
a part to be included by |\input| is being compiled,
and the name of the part is stored in |\childdocname|.

%%%%%%%%%%%%%%%%%%%%%%%%%%%%%%%%%%%%%%%%
\DescribeMacro{\childdocby}
Each part to be included by |\input| should start with:
%
\begin{center}
\begin{tabular}{l}
|\input{childdoc.def}|\\
|\childdocby{|\textit{main}|}|\\
\end{tabular}
\end{center}
%
The directive |\childdocby| is similar to |\childdocof|
described in \secref{sec:include},
but the subsequent selection of content must be done manually.
To that end, both |\ifchilddoc| and |\ifchilddocmanual|
will be true upon processing of a part,
and the name of the part is stored in |\childdocname|.
Note that |\jobname| will be set to the filename of the current part
so that each part receives an individual |.aux| file
that does not interfere with the |.aux| file(s) of the main document.
This behaviour can be altered by the alternative form
|\childdocby[*]{|\textit{main}|}| (with a non-empty optional argument)
which uses the |.aux| file of the main document
by setting |\jobname| to \textit{main}.

%%%%%%%%%%%%%%%%%%%%%%%%%%%%%%%%%%%%%%%%%%%%%%%%%%%%%%%%%%%%%%%%%%%%%%%%%%%%%%%%
\subsection{Driver Development}
\label{sec:driver}

The \textsf{childdoc} mechanism can also be use for the development
of definition files such as \LaTeX{} styles or classes.
This case differs from the above setup with multiple parts
included by |\include| in that no |\includeonly| should be invoked.
This can be achieved by starting the include file
(before |\ProvidesPackage|) with:
%
\begin{center}
\begin{tabular}{l}
|\input{childdoc.def}|\\
|\childdocforward{|\textit{main}|}|\\
\end{tabular}
\end{center}
%
or alternatively with:
%
\begin{center}
\begin{tabular}{l}
|\input{childdoc.def}|\\
|\childdocby{|\textit{main}|}|\\
\end{tabular}
\end{center}
%
Both forms have slightly different effects as described above.
The main file is prepared as usual, see \secref{sec:include}.

%%%%%%%%%%%%%%%%%%%%%%%%%%%%%%%%%%%%%%%%%%%%%%%%%%%%%%%%%%%%%%%%%%%%%%%%%%%%%%%%
\subsection{Legacy Detection}
\label{sec:detection}

The directive |\childdocmain| in the main file can detect
whether the complete document or merely a child is to be compiled
even without using the directive |\childdocof|.
This method is deprecated because it is less robust
and there is no compelling reason to use it;
it is merely provided for backward compatibility
and it may be removed in future versions.

If the detection mechanism is to be used,
it is mandatory to correctly specify
the filename of the main file as the argument of |\childdocmain|:
%
\begin{center}
\begin{tabular}{l}
|\input{childdoc.def}|\\
|\childdocmain{|\textit{main}|}|\\
\end{tabular}
\end{center}
%
If |\jobname| does not match the argument \textit{main} of |\childdocmain|,
it is assumed that |\jobname| points to the child file to be compiled.
When using |\childdocmain| with the main file specified as argument,
it suffices to start a child file
with just |\input{|\textit{main}|}|
without loading of the package and using |\childdocof|.
If instead all processing is done
with the appropriate \textsf{childdoc} directives,
the argument of \textit{main} of |\childdocmain| can be empty.

An alternative version of the command line processing described
in \secref{sec:commandline} using the detection mechanism reads:
%
\begin{center}
|... -jobname "|\textit{target}|" "|[\textit{flags}]%
[|\def\jobname{|\textit{dest}|}|]|\input{|\textit{main}|}"|
\end{center}

%%%%%%%%%%%%%%%%%%%%%%%%%%%%%%%%%%%%%%%%%%%%%%%%%%%%%%%%%%%%%%%%%%%%%%%%%%%%%%%%
\subsection{Manual Code}
\label{sec:manual}

In case one cannot be certain whether the definitions file |childdoc.def|
is installed on the target \TeX{} distribution
and one prefers not to ship it,
it is conceivable to paste a few relevant commands into the sources.

To that end, drop all statements |\input{childdoc.def}|
and perform the replacements as outlined below.
Instead of |\childdocmain{|\textit{main}|}| add the following code
to the top of the main file:
%
\begin{center}
\begin{tabular}{l}
|\||ifdefined\childdocname\endinput\||fi\newif\ifchilddoc|\\
|\edef\childdocname{\scantokens\expandafter{\jobname\noexpand}}|\\
|\def\childdocmain{|\textit{main}|}\||ifx\childdocmain\childdocname\||else|\\
|\childdoctrue\includeonly{\childdocname}\let\jobname\childdocmain\||fi|\\
\end{tabular}
\end{center}
%
Instead of |\childdocof{|\textit{main}|}| just include the main file
at the top of each child file:
%
\begin{center}
|\input{|\textit{main}|}|
\end{center}
%
A simple redirection |\childdocforward{|\textit{dest}|}| is achieved by:
%
\begin{center}
|\def\jobname{|\textit{dest}|}\input{\jobname}|
\end{center}
%
The redirection with prefix
|\childdocforwardprefix[|\textit{prefix}|]{|\textit{dest}|}|
is accomplished by:
%
\begin{center}
\begin{tabular}{l}
|{\edef\jobname{\scantokens\expandafter{\jobname\noexpand}}|\\
|\def\redirectjob |\textit{prefix}|#1~~~{\gdef\jobname{|\textit{dest}|#1}}|\\
|\expandafter\redirectjob\jobname~~~}\input{\jobname}|
\end{tabular}
\end{center}

In an alternative approach,
child documents can be compiled by a specific command line
without additional code or specific definitions:
%
\begin{center}
|... -jobname "|\textit{target}|" "|[\textit{flags}]%
|\includeonly{|\textit{dest}|}\input{|\textit{main}|}"|
\end{center}
%

%%%%%%%%%%%%%%%%%%%%%%%%%%%%%%%%%%%%%%%%%%%%%%%%%%%%%%%%%%%%%%%%%%%%%%%%%%%%%%%%
%%%%%%%%%%%%%%%%%%%%%%%%%%%%%%%%%%%%%%%%%%%%%%%%%%%%%%%%%%%%%%%%%%%%%%%%%%%%%%%%
\section{Information}

%%%%%%%%%%%%%%%%%%%%%%%%%%%%%%%%%%%%%%%%%%%%%%%%%%%%%%%%%%%%%%%%%%%%%%%%%%%%%%%%
\subsection{Copyright}

Copyright \copyright{} 2017--2018 Niklas Beisert

This work may be distributed and/or modified under the
conditions of the \LaTeX{} Project Public License, either version 1.3
of this license or (at your option) any later version.
The latest version of this license is in
  \url{http://www.latex-project.org/lppl.txt}
and version 1.3 or later is part of all distributions of \LaTeX{}
version 2005/12/01 or later.

This work has the LPPL maintenance status `maintained'.

The Current Maintainer of this work is Niklas Beisert.

This work consists of the files |README.txt|, |childdoc.ins| and |childdoc.dtx|
as well as the derived files |childdoc.def|, |cdocsamp.tex|
with |cdocsch1.tex|, |cdocsch2.tex|, |cdocspt3.tex|, |cdocspt4.tex|,
|cdocsdrf.tex|, |cdocsfn1.tex|, |cdocsfn2.tex|
as well as |childdoc.pdf|.

%%%%%%%%%%%%%%%%%%%%%%%%%%%%%%%%%%%%%%%%%%%%%%%%%%%%%%%%%%%%%%%%%%%%%%%%%%%%%%%%
\subsection{Files and Installation}

The package consists of the files:
%
\begin{center}
\begin{tabular}{ll}
    |README.txt|   & readme file \\
    |childdoc.ins| & installation file \\
    |childdoc.dtx| & source file \\
    |childdoc.def| & definition file \\
    |cdocsamp.tex| & sample main file \\
    |cdocsch1.tex| & sample include file \\
    |cdocsch2.tex| & sample include file \\
    |cdocspt3.tex| & sample part file \\
    |cdocspt4.tex| & sample part file \\
    |cdocsdrf.tex| & sample redirection file \\
    |cdocsfn1.tex| & sample redirection file \\
    |cdocsfn2.tex| & sample redirection file \\
    |childdoc.pdf| & manual
\end{tabular}
\end{center}
%
The distribution consists of the files
|README.txt|, |childdoc.ins| and |childdoc.dtx|.
%
\begin{itemize}
\item
Run (pdf)\LaTeX{} on |childdoc.dtx|
to compile the manual |childdoc.pdf| (this file).
\item
Run \LaTeX{} on |childdoc.ins| to create the definitions file |childdoc.def|
and the sample |cdocsamp.tex| with include files
|cdocsch1.tex|, |cdocsch2.tex|, |cdocspt3.tex|, |cdocspt4.tex|,
|cdocsdrf.tex|, |cdocsfn1.tex|, |cdocsfn2.tex|.
Then copy the file |childdoc.def| to an appropriate directory of your \LaTeX{}
distribution, e.g.\ \textit{texmf-root}|/tex/latex/childdoc|.
\end{itemize}

%%%%%%%%%%%%%%%%%%%%%%%%%%%%%%%%%%%%%%%%%%%%%%%%%%%%%%%%%%%%%%%%%%%%%%%%%%%%%%%%
\subsection{Related CTAN Packages}

There are several other packages which offer a similar functionality:
%
\begin{itemize}
\item
The packages
\href{http://ctan.org/pkg/docmute}{\textsf{docmute}},
\href{http://ctan.org/pkg/includex}{\textsf{includex}} and
\href{http://ctan.org/pkg/standalone}{\textsf{standalone}}
provide commands to include only the document body of
a child file thus allowing both files to be compiled individually.
\item
The packages \href{http://ctan.org/pkg/subdocs}{\textsf{subdocs}}
and \href{http://ctan.org/pkg/subfiles}{\textsf{subfiles}}
provide structures in which the main and child documents can be
encapsulated and allowing them to be compiled individually.
The inclusion mechanism is different from the conventional |\include|.
\item
The package \href{http://ctan.org/pkg/combine}{\textsf{combine}}
is an elaborate solution to combine several documents into one.
\end{itemize}
%
See also the CTAN topic \href{http://ctan.org/topic/subdocs}{\textsf{subdocs}}
for further related packages.
The present package differs from the above solutions in that
a document structure constructed with the conventional |\include| mechanism
just needs two extra commands at the top of every file
such that all constituent files can be compiled individually.

%%%%%%%%%%%%%%%%%%%%%%%%%%%%%%%%%%%%%%%%%%%%%%%%%%%%%%%%%%%%%%%%%%%%%%%%%%%%%%%%
%\subsection{Feature Suggestions}
%
%The following is a list of features which may be useful for future
%versions of this package:
%%
%\begin{itemize}
%\item
%\ldots
%\end{itemize}

%%%%%%%%%%%%%%%%%%%%%%%%%%%%%%%%%%%%%%%%%%%%%%%%%%%%%%%%%%%%%%%%%%%%%%%%%%%%%%%%
\subsection{Revision History}

%%%%%%%%%%%%%%%%%%%%%%%%%%%%%%%%%%%%%%%%
\paragraph{v2.0:} 2018/12/30

\begin{itemize}
\item
immediate forward processing
\item
added |\childdocby| mechanism
\item
manual restructured
\end{itemize}

%%%%%%%%%%%%%%%%%%%%%%%%%%%%%%%%%%%%%%%%
\paragraph{v1.6:} 2018/01/17

\begin{itemize}
\item
application for development of include files
\item
corrections to manual
\end{itemize}

%%%%%%%%%%%%%%%%%%%%%%%%%%%%%%%%%%%%%%%%
\paragraph{v1.5:} 2017/05/21

\begin{itemize}
\item
more complete structuring introduced
\item
|\childdocof| introduced
\item
|\childdoc| renamed to |\childdocmain|
\item
|\childredirect| renamed to |\childdocforward| and |\childdocforwardprefix|
and functionality expanded
\end{itemize}

%%%%%%%%%%%%%%%%%%%%%%%%%%%%%%%%%%%%%%%%
\paragraph{v1.0:} 2017/04/27

\begin{itemize}
\item
manual and install package
\item
first version published on CTAN
\end{itemize}

%%%%%%%%%%%%%%%%%%%%%%%%%%%%%%%%%%%%%%%%
\paragraph{v0.6:} 2017/04/26

\begin{itemize}
\item
redirection mechanism added
\end{itemize}

%%%%%%%%%%%%%%%%%%%%%%%%%%%%%%%%%%%%%%%%
\paragraph{v0.5:} 2017/04/26

\begin{itemize}
\item
functionality in definition file
\end{itemize}


%%%%%%%%%%%%%%%%%%%%%%%%%%%%%%%%%%%%%%%%%%%%%%%%%%%%%%%%%%%%%%%%%%%%%%%%%%%%%%%%
%%%%%%%%%%%%%%%%%%%%%%%%%%%%%%%%%%%%%%%%%%%%%%%%%%%%%%%%%%%%%%%%%%%%%%%%%%%%%%%%
%%%%%%%%%%%%%%%%%%%%%%%%%%%%%%%%%%%%%%%%%%%%%%%%%%%%%%%%%%%%%%%%%%%%%%%%%%%%%%%%
\appendix

\settowidth\MacroIndent{\rmfamily\scriptsize 000\ }

 \DocInput{childdoc.dtx}

\end{document}
%</driver>
% \fi
%
% %%%%%%%%%%%%%%%%%%%%%%%%%%%%%%%%%%%%%%%%%%%%%%%%%%%%%%%%%%%%%%%%%%%%%%%%%%%%%%
% %%%%%%%%%%%%%%%%%%%%%%%%%%%%%%%%%%%%%%%%%%%%%%%%%%%%%%%%%%%%%%%%%%%%%%%%%%%%%%
% \section{Sample}
%\iffalse
%<*samplemain>
%\fi
%
% The following presents a sample document
% with two chapters, two parts, a title page,
% a compile flag as well as three forwarding files to set the flag.
% It consists of eight |.tex| files:
% \begin{center}
% \begin{tabular}{ll}
% |cdocsamp.tex|&main file\\
% |cdocsch1.tex|&include file for chapter 1\\
% |cdocsch2.tex|&include file for chapter 2\\
% |cdocspt3.tex|&include file for part 3\\
% |cdocspt4.tex|&include file for part 4\\
% |cdocsdrf.tex|&forwarding file for main file in draft mode\\
% |cdocsfi1.tex|&forwarding file for final version of chapter 1\\
% |cdocsfi2.tex|&forwarding file for final version of chapter 2\\
% \end{tabular}
% \end{center}
% Each of the eight files can be compiled directly by the \LaTeX{} compiler.
%
% %%%%%%%%%%%%%%%%%%%%%%%%%%%%%%%%%%%%%%
% \paragraph{Main File.}
%
% The main file is called |cdocsamp.tex|.
%
% Load the \textsf{childdoc} definitions and
% declare the filename for the main document:
%    \begin{macrocode}
\input{childdoc.def}
\childdocmain{}
%    \end{macrocode}

% Optional override for |\version| flag:
%    \begin{macrocode}
%%\ifchilddoc\else\providecommand{\version}{draft}\fi
%    \end{macrocode}

% Define the default values for the |\version| flag
% (|final| for the main file and |draft| for childs):
%    \begin{macrocode}
\ifchilddoc
\providecommand{\version}{draft}
\else
\providecommand{\version}{final}
\fi
%    \end{macrocode}

% Load the standard document class:
%    \begin{macrocode}
\documentclass[12pt]{article}
%    \end{macrocode}

% Start the document body:
%    \begin{macrocode}
\begin{document}
%    \end{macrocode}

% Declare a title page.
% Print title, part of document being processed and version flag:
%    \begin{macrocode}
\addtocounter{page}{-1}
\begin{center}
{\LARGE\bfseries{}childdoc example\par}
\vspace{1cm}
\ifchilddoc
\ifchilddocmanual part\else chapter\fi:
`\childdocname' of `\childdocjob'\par
\else
main document: `\childdocjob'\par
\fi
version: \version\par
\end{center}
\newpage
%    \end{macrocode}

% Manually include selected file,
% otherwise process as usual:
%    \begin{macrocode}
\ifchilddocmanual
\section*{part `\childdocname'}
\input{\childdocname}
\else
%    \end{macrocode}

% Include the two chapters:
%    \begin{macrocode}
\include{cdocsch1}
\include{cdocsch2}
%    \end{macrocode}

% Include the two parts unless only chapters should be displayed:
%    \begin{macrocode}
\ifchilddoc\else
\section{part three}
\input{cdocspt3}
\section{part four}
\input{cdocspt4}
\fi
%    \end{macrocode}

% Process as usual until here:
%    \begin{macrocode}
\fi
%    \end{macrocode}

% End of document body:
%    \begin{macrocode}
\end{document}
%    \end{macrocode}
%\iffalse
%</samplemain>
%\fi
%
% %%%%%%%%%%%%%%%%%%%%%%%%%%%%%%%%%%%%%%
% \paragraph{Chapter Include Files.}
%
% The include files are called |cdocsch1.tex| and |cdocsch2.tex|.
%
%\iffalse
%<*samplechap1|samplechap2>
%\fi

% Optional override for |\version| flag:
%    \begin{macrocode}
%%\providecommand{\version}{final}
%    \end{macrocode}

% Include the main document:
%    \begin{macrocode}
\input{childdoc.def}
\childdocof{cdocsamp}
%    \end{macrocode}

%\iffalse
%</samplechap1|samplechap2>
%\fi
%
%\iffalse
%<*samplechap1>
%\fi
% Some text for chapter 1:
%    \begin{macrocode}
\section{one}
some text in chapter one
%    \end{macrocode}

%\iffalse
%</samplechap1>
%\fi
% Some text for chapter 2:
%\iffalse
%<*samplechap2>
%\fi
%    \begin{macrocode}
\section{two}
more text in chapter two
%    \end{macrocode}

%\iffalse
%</samplechap2>
%\fi
%
% %%%%%%%%%%%%%%%%%%%%%%%%%%%%%%%%%%%%%%
% \paragraph{Part Include Files.}
%
% The include files are called |cdocspt3.tex| and |cdocspt4.tex|.
%
%\iffalse
%<*samplepart3|samplepart4>
%\fi

% Optional override for |\version| flag:
%    \begin{macrocode}
%%\providecommand{\version}{final}
%    \end{macrocode}

% Include the main document:
%    \begin{macrocode}
\input{childdoc.def}
\childdocby{cdocsamp}
%    \end{macrocode}

%\iffalse
%</samplepart3|samplepart4>
%\fi
%
%\iffalse
%<*samplepart3>
%\fi
% Some text for part 3:
%    \begin{macrocode}
some text in part three
%    \end{macrocode}

%\iffalse
%</samplepart3>
%\fi
% Some text for part 4:
%\iffalse
%<*samplepart4>
%\fi
%    \begin{macrocode}
more text in part four
%    \end{macrocode}

%\iffalse
%</samplepart4>
%\fi
%
% %%%%%%%%%%%%%%%%%%%%%%%%%%%%%%%%%%%%%%
% \paragraph{Forwarding for a Complete Draft.}
%
% The following forwarding file |cdocsdrf.tex|
% compiles the main document in draft mode:
%\iffalse
%<*sampledraft>
%\fi
%    \begin{macrocode}
\def\version{draft}
\input{childdoc.def}
\childdocforward{cdocsamp}
%    \end{macrocode}

%\iffalse
%</sampledraft>
%\fi
%
% %%%%%%%%%%%%%%%%%%%%%%%%%%%%%%%%%%%%%%
% \paragraph{Forwarding for Final Version of the Chapters.}
%
% The following forwarding files |cdocsfn1.tex| and |cdocsfn2.tex|
% (with identical content)
% compile the final versions of the child documents
% |cdocsch1.tex| and |cdocsch2.tex|, respectively:
%\iffalse
%<*samplefinal>
%\fi
%    \begin{macrocode}
\def\version{final}
\input{childdoc.def}
\childdocforwardprefix[cdocsamp]{cdocsfn}{cdocsch}
%    \end{macrocode}

%\iffalse
%</samplefinal>
%\fi
%
% %%%%%%%%%%%%%%%%%%%%%%%%%%%%%%%%%%%%%%
% \paragraph{Command Line Processing.}
%
% The following three command lines generate the output files
% |cdocscld|, |cdocscl1| and |cdocscl2|
% which should be identical to
% |cdocsdrf|, |cdocsch1| and |cdocsfn2|, respectively:
% \begin{center}
% \begin{tabular}{l}
% |latex -jobname cdocscld \|\\
% |  "\def\version{draft}\input{childdoc.def}\childdocforward{cdocsamp}"|\\
% |latex -jobname cdocscl1 \|\\
% |  "\input{childdoc.def}\childdocforward[cdocsamp]{cdocsch1}"|\\
% |latex -jobname cdocscl2 \|\\
% |  "\def\version{final}\input{childdoc.def}\childdocforward{cdocsch2}"|
% \end{tabular}
% \end{center}
% Note that the trailing backslash on each first line
% merely continues the input to the second line
% (for convenient cut ant paste).
% Furthermore, the command |latex| can be replaced by any
% of its alternative versions such as |pdflatex|.
%
% %%%%%%%%%%%%%%%%%%%%%%%%%%%%%%%%%%%%%%%%%%%%%%%%%%%%%%%%%%%%%%%%%%%%%%%%%%%%%%
% %%%%%%%%%%%%%%%%%%%%%%%%%%%%%%%%%%%%%%%%%%%%%%%%%%%%%%%%%%%%%%%%%%%%%%%%%%%%%%
% \section{Implementation}
%\iffalse
%<*package>
%\fi
%
% This section describes the definitions file |childdoc.def|.

% The definitions cannot be loaded using |\usepackage| or |\RequirePackage|
% which has a mechanism to prevent loading a style file more than once.
% When loading the definitions by means of |\input|
% multiple instances have to be prevented manually:
%\iffalse
%This code needs to be before the `\ProvidesFile' directive
%which is defined at the beginning of this file.
%Therefore it is also placed there and commented out here.
%</package>
%<*discard>
%\fi
%    \begin{macrocode}
\ifdefined\childdocmain\endinput\fi
%    \end{macrocode}
%\iffalse
%</discard>
%<*package>
%\fi
%
% \macro{\ifchilddoc}
% \macro{\ifchilddocmanual}
% The conditional |\ifchilddoc| tells whether a
% child (true) or main (false) document is being compiled.
% The conditional |\ifchilddocmanual| tells whether
% the |\includeonly| mechanism is used (false) or
% the selection of child files must be performed manually (true).
% The definitions initialise to false:
%    \begin{macrocode}
\newif\ifchilddoc
\newif\ifchilddocmanual
%    \end{macrocode}

% \macro{\childdocname}
% \macro{\childdocjob}
% The macro |\childdocname| stores the name of the main document
% to be compiled. The macro |\childdocjob| stores the name of
% the document on which the \LaTeX{} compiler was originally invoked.
% The content of |\jobname| cannot be compared
% to filenames specified in the source due to different catcodes.
% The following code rescans |\jobname|, stores the result
% in |\childdocname| and saves a copy in |\childdocjob|:
%    \begin{macrocode}
\edef\childdocname{\scantokens\expandafter{\jobname\noexpand}}
\let\childdocjob\childdocname
%    \end{macrocode}

% \macro{\childdocdisable}
% The macro |\childdocdisable| prevents the main file
% from being processed more than once.
% At this stage, the main document command |\childdocmain|
% is assumed to be called once again where it should do nothing.
% Any subsequent call to it should prevent
% a secondary processing of the main document
% It overwrites the forwarding commands
% |\childdocof| and |\childdocforward|
% with empty macros to prevent further inclusions of the main document:
%    \begin{macrocode}
\newcommand{\childdocdisable}
{
  \renewcommand{\childdocmain}[1]{\renewcommand{\childdocmain}[1]{\endinput}}
  \renewcommand{\childdocof}[1]{}
  \renewcommand{\childdocby}[2][]{}
  \renewcommand{\childdocforward}[2][]{}
  \renewcommand{\childdocdisable}{}
}
%    \end{macrocode}

% \macro{\childdocmain}
% The macro |\childdocmain| is to be called at the top of the main file
% with nothing or the main filename (without extension) as argument.
% First, it breaks loops.
% If the argument is not empty and does not match |\childdocname|
% (which is set by the first inclusion of |childdoc.def|),
% |\ifchilddoc| is set to true, |\includeonly| is applied to the child file
% and |\jobname| is set to the main file
% (for proper handling of |.aux| files):
%    \begin{macrocode}
\newcommand{\childdocmain}[1]
{
  \childdocdisable\childdocmain{}
  \if?#1?\else
    \begingroup
      \def\childdoctmp{#1}
      \ifx\childdoctmp\childdocname
        \def\childdoctmp{}
      \else
        \def\childdoctmp
        {
          \childdoctrue
          \includeonly{\childdocname}
          \def\childdocjob{#1}
          \def\jobname{#1}
        }
      \fi
      \expandafter
    \endgroup
    \childdoctmp
  \fi
}
%    \end{macrocode}

% \macro{\childdocof}
% The command |\childdocof| redirects
% compilation to the main file |#1|.
%    \begin{macrocode}
\newcommand{\childdocof}[1]
{
  \childdocdisable
  \childdoctrue
  \includeonly{\childdocname}
  \def\jobname{#1}
  \def\childdocjob{#1}
  \input{#1}
}
%    \end{macrocode}

% \macro{\childdocby}
% The command |\childdocby| ....
%    \begin{macrocode}
\newcommand{\childdocby}[2][]
{
  \childdocdisable
  \childdoctrue
  \childdocmanualtrue
  \if?#1?\else
    \def\jobname{#2}
  \fi
  \def\childdocjob{#2}
  \input{#2}
  \endinput
}
%    \end{macrocode}

% \macro{\childdocforward}
% The command |\childdocforward| redirects
% compilation to the main file or
% (if the optional argument is given) a child file.
% Parameters are set as if the main file
% or a child file starting with |\childdocof| was compiled.
% Then compilation is handed over to the main file:
%    \begin{macrocode}
\newcommand{\childdocforward}[2][]
{
  \begingroup
    \if?#1?
      \def\childdoctmp
      {
        \def\childdocname{#2}
        \def\childdocjob{#2}
        \def\jobname{#2}
        \input{#2}
        \endinput
      }
    \else
      \def\childdoctmp
      {
        \childdocdisable
        \def\childdocname{#2}
        \childdoctrue
        \includeonly{#2}
        \def\childdocjob{#1}
        \def\jobname{#1}
        \input{#1}
        \endinput
      }
    \fi
    \expandafter
  \endgroup
  \childdoctmp
}
%    \end{macrocode}

% \macro{\childdocforwardprefix}
% The command |\childdocforwardprefix| redirects
% compilation to the main or a child file by means of a pattern.
% The prefix |#1| in the current filename is replaced by |#2|
% and the suffix of the current filename is kept
% (it is assumed that the filename does not contain the substring `|~~~|'
% which is used as a delimiter).
% Compilation is handed over to the new file by |\childdocforward|:
%    \begin{macrocode}
\newcommand{\childdocforwardprefix}[3][]
{
  \begingroup
    \def\childdocextract #2##1~~~{\def\childdoctmp{\childdocforward[#1]{#3##1}}}
    \expandafter\childdocextract\childdocname~~~
    \expandafter
  \endgroup
  \childdoctmp
}
%    \end{macrocode}

% \macro{\childdoc}
% The deprecated macro |\childdoc| is a legacy version of |\childdocmain|:
%    \begin{macrocode}
\newcommand{\childdoc}{\childdocmain}
%    \end{macrocode}

% \macro{\childdocredirect}
% The deprecated macro |\childdocredirect| is a legacy version
% of |\childdocforward| and |\childdocforwardprefix|:
%    \begin{macrocode}
\newcommand{\childdocredirect}[2][]
{
  \begingroup
    \if?#1?
      \def\childdoctmp{\childdocforward{#2}}
    \else
      \def\childdoctmp{\childdocforwardprefix{#1}{#2}}
    \fi
    \expandafter
  \endgroup
  \childdoctmp
}
%    \end{macrocode}

%\iffalse
%</package>
%\fi
%
\endinput
\childdocforward{cdocsamp}"|\\
% |latex -jobname cdocscl1 \|\\
% |  "% \iffalse
%
% childdoc.dtx Copyright (C) 2017-2018 Niklas Beisert
%
% This work may be distributed and/or modified under the
% conditions of the LaTeX Project Public License, either version 1.3
% of this license or (at your option) any later version.
% The latest version of this license is in
%   http://www.latex-project.org/lppl.txt
% and version 1.3 or later is part of all distributions of LaTeX
% version 2005/12/01 or later.
%
% This work has the LPPL maintenance status `maintained'.
%
% The Current Maintainer of this work is Niklas Beisert.
%
% This work consists of the files childdoc.dtx and childdoc.ins
% and the derived files childdoc.def and cdocsamp.tex with
% cdocsch1.tex, cdocsch2.tex, cdocsdrf.tex, cdocsfn1.tex, cdocsfn2.tex.
%
%<package>\ifdefined\childdocmain\endinput\fi
%<package>\ProvidesFile{childdoc.def}[2018/12/30 v2.0 child document driver]
%<samplemain>\ProvidesFile{cdocsamp.tex}[2018/12/30 v2.0 sample for childdoc]
%<*driver>
%\ProvidesFile{childdoc.drv}[2018/12/30 v2.0 childdoc reference manual file]
\PassOptionsToClass{10pt,a4paper}{article}
\documentclass{ltxdoc}

\usepackage[margin=35mm]{geometry}
\usepackage{hyperref}
\usepackage{hyperxmp}
\usepackage[usenames]{color}

\hypersetup{colorlinks=true}
\hypersetup{pdfstartview=FitH}
\hypersetup{pdfpagemode=UseNone}
\hypersetup{pdfsource={}}
\hypersetup{pdflang={en-UK}}
\hypersetup{pdfcopyright={Copyright 2017-2018 Niklas Beisert.
  This work may be distributed and/or modified under the
  conditions of the LaTeX Project Public License, either version 1.3
  of this license or (at your option) any later version.}}
\hypersetup{pdflicenseurl={http://www.latex-project.org/lppl.txt}}
\hypersetup{pdfcontactaddress={ETH Zurich, ITP, HIT K,
  Wolfgang-Pauli-Strasse 27}}
\hypersetup{pdfcontactpostcode={8093}}
\hypersetup{pdfcontactcity={Zurich}}
\hypersetup{pdfcontactcountry={Switzerland}}
\hypersetup{pdfcontactemail={nbeisert@itp.phys.ethz.ch}}
\hypersetup{pdfcontacturl={http://people.phys.ethz.ch/\xmptilde nbeisert/}}

\newcommand{\secref}[1]{\hyperref[#1]{section \ref*{#1}}}

\parskip1ex
\parindent0pt
\let\olditemize\itemize
\def\itemize{\olditemize\parskip0pt}

\begin{document}

\title{The \textsf{childdoc} Package}
\hypersetup{pdftitle={The childdoc Package}}
\author{Niklas Beisert\\[2ex]
  Institut f\"ur Theoretische Physik\\
  Eidgen\"ossische Technische Hochschule Z\"urich\\
  Wolfgang-Pauli-Strasse 27, 8093 Z\"urich, Switzerland\\[1ex]
  \href{mailto:nbeisert@itp.phys.ethz.ch}
  {\texttt{nbeisert@itp.phys.ethz.ch}}}
\hypersetup{pdfauthor={Niklas Beisert}}
\hypersetup{pdfsubject={Manual for the LaTeX2e Package childdoc}}
\date{30 December 2018, \textsf{v2.0}}
\maketitle

\begin{abstract}\noindent
\textsf{childdoc} is a \LaTeXe{} package
that enables the direct compilation
of document sections included by |\include|
to individual files.
\end{abstract}

\begingroup
\parskip0ex
\tableofcontents
\endgroup

%%%%%%%%%%%%%%%%%%%%%%%%%%%%%%%%%%%%%%%%%%%%%%%%%%%%%%%%%%%%%%%%%%%%%%%%%%%%%%%%
%%%%%%%%%%%%%%%%%%%%%%%%%%%%%%%%%%%%%%%%%%%%%%%%%%%%%%%%%%%%%%%%%%%%%%%%%%%%%%%%
\section{Introduction}

\LaTeX{} provides a mechanism to structure a large document (such as a book)
into a main file and several child files (containing the chapters)
using the |\include| command.
This mechanism is beneficial for documents
which span hundreds of pages in order to
make the source file(s) more manageable.
Moreover, compilation can be restricted to
selected child files by means of the |\includeonly| command.
The latter feature can be used to reduce the compilation time while editing
(this was significantly more useful in the earlier days of \LaTeX{})
or to generate a smaller document which is easier to navigate.
Another application of |\includeonly| is to generate
documents consisting of selected parts of the complete document.

However, there are a few drawbacks of the plain |\include| mechanism:
\begin{itemize}
\item
The child files cannot be compiled on their own,
they can only be compiled via the main file.
A naive editing environment
(such as a text editor with an option
to have the current file processed by \LaTeX)
may require one to switch to the main file before compiling;
attempting to compile the child file produces errors.
\item
The main file must be modified (each time)
to adjust the |\includeonly| command
to the present needs. This easily leaves the main file in a messy state.
\item
The generated document will always carry the filename
of the main document. This is inconvenient if
several child files are to be compiled and
to be kept for distribution.
\end{itemize}

The present package provides a simple interface
to make child files individually compilable by \LaTeX{}.
Compiling a child file then has the same effect as compiling
the main file with an |\includeonly| command
to select the appropriate child.
Moreover the generated document will carry the name of the child
rather than the main file.
This resolves all three above issues.

This feature is meant to make the editing of books,
thesis documents and lecture notes somewhat more convenient.
However, the package can also be used efficiently for
composing a series of documents (such as exercise sheets)
which are typically distributed individually.
It then assists the author in generating the individual documents
(potentially in different versions)
as well as a document containing the collected series.
Another application is in developing style files
or other kinds of included material
where compilation of the style file could redirect
to a sample or test file.

%%%%%%%%%%%%%%%%%%%%%%%%%%%%%%%%%%%%%%%%%%%%%%%%%%%%%%%%%%%%%%%%%%%%%%%%%%%%%%%%
%%%%%%%%%%%%%%%%%%%%%%%%%%%%%%%%%%%%%%%%%%%%%%%%%%%%%%%%%%%%%%%%%%%%%%%%%%%%%%%%
\section{Usage}

First of all, the package \textsf{childdoc} is \emph{not} a standard
\LaTeXe{} |.sty| style file! Therefore it needs to be invoked in
a non-standard way.

%%%%%%%%%%%%%%%%%%%%%%%%%%%%%%%%%%%%%%%%%%%%%%%%%%%%%%%%%%%%%%%%%%%%%%%%%%%%%%%%
\subsection{Included Files}
\label{sec:include}

%%%%%%%%%%%%%%%%%%%%%%%%%%%%%%%%%%%%%%%%
\DescribeMacro{\childdocmain}
To use the package, add the commands
\begin{center}
\begin{tabular}{l}
|\input{childdoc.def}|\\
|\childdocmain{}|\\
\end{tabular}
\end{center}
at the very top of the main \LaTeX{} file,
in particular \emph{before} the |\documentclass| statement!
The argument of |\childdocmain| should be left empty
(but it must be present).

%%%%%%%%%%%%%%%%%%%%%%%%%%%%%%%%%%%%%%%%
\DescribeMacro{\childdocof}
Furthermore, add the commands
\begin{center}
\begin{tabular}{l}
|\input{childdoc.def}|\\
|\childdocof{|\textit{main}|}|\\
\end{tabular}
\end{center}
at the top of every child file \textit{child}
which is included by |\include{|\textit{child}|}|
from within the main file
(or at least for those files to be compiled individually).
The argument \textit{main} must be the filename of the main file.

There are a couple of
considerations in setting up the main and child documents:

%%%%%%%%%%%%%%%%%%%%%%%%%%%%%%%%%%%%%%%%
\paragraph{Restrictions.}

Please note the following restrictions:
\begin{itemize}
\item
|\childdocmain| must be called with one argument \textit{main}
to ensure compatibility with earlier version of the package.
It must either be empty (|\childdocmain{}|)
or precisely match the filename of the main file in which it is specified.
See \secref{sec:detection} for further information.
\item
The filename \textit{main} must be specified without the |.tex| extension.
\item
The filename \textit{main} is case sensitive
(even in case-insensitive file systems)
due to internal string comparison.
\item
The argument \textit{main} should be fully expanded, it cannot be a macro.
\item
Subdirectories and special characters should be avoided in filenames.
\item
The command |\childdocmain{|\textit{main}|}| must be followed by a whitespace.
It should not be followed immediately by another command
or by a comment mark `|%|'.
This is because the \TeX{} parser reads the token immediately following
the argument of |\childdocmain| and puts it
at the beginning of every child section;
however, a white\-space is ignored.
\end{itemize}

%%%%%%%%%%%%%%%%%%%%%%%%%%%%%%%%%%%%%%%%
\paragraph{Content of Main File.}

It is advisable to place all content in the child files included by |\include|.
Any output contained in the main file will appear in all child documents
unless suppressed manually;
it cannot be suppressed automatically by the |\includeonly| directive
and thus should normally be avoided.
A method to include some content in the main file
by means of conditional processing is described in \secref{sec:conditional}.

%%%%%%%%%%%%%%%%%%%%%%%%%%%%%%%%%%%%%%%%
\paragraph{Page Numbering.}

When only a part of the document is compiled,
the appropriate numbering of pages
(as well as other status parameters)
is determined from the |.aux| files.
The latter contain information from previous passes.
However this information needs to propagate through
all intermediate child documents.
Therefore the page numbering in child documents may well
be inconsistent until the complete document is compiled at least once.

A useful (if unconventional) way to always ensure a consistent
page numbering is to restart the numbering in each child document
and denote the pages by `\textit{child}|.|\textit{page}'
where \textit{child} represents the chapter/section number of the child file.
This can be achieved by the command
|\numberwithin{page}{|\textit{child}|}|
of the \textsf{amsmath} package
where \textit{child} can be |chapter| or |section|
depending on the chosen structuring.
Alternatively, one can modify the macro |\thepage| appropriately
and reset the counter |page| at the start of each child file.

%%%%%%%%%%%%%%%%%%%%%%%%%%%%%%%%%%%%%%%%%%%%%%%%%%%%%%%%%%%%%%%%%%%%%%%%%%%%%%%%
\subsection{Conditional Processing}
\label{sec:conditional}

The package provides a mechanism to compile different versions
of a document. To customise the versions further some conditional processing
can come in handy to distinguish which version is being compiled.
The package provides two macros to describe the compilation context:

%%%%%%%%%%%%%%%%%%%%%%%%%%%%%%%%%%%%%%%%
\DescribeMacro{\ifchilddoc}
The conditional |\ifchilddoc| distinguishes between the compilation of
child documents and the main document:
%
\begin{center}
|\ifchilddoc |\textit{child-code}| |[|\||else |\textit{main-code}]| \||fi|
\end{center}

%%%%%%%%%%%%%%%%%%%%%%%%%%%%%%%%%%%%%%%%
\DescribeMacro{\childdocname}
\DescribeMacro{\childdocjob}
The macro |\childdocname| contains the filename (without extension)
of the main or child file being processed.
Note that |\childdocjob| will always contain the name of the main file.

%%%%%%%%%%%%%%%%%%%%%%%%%%%%%%%%%%%%%%%%
\paragraph{Title Page.}

Conditional processing can be used to include a title or banner page
in the main document when proper precautions are taken.
Importantly, the code in the main file should ensure that the page counter
(as well as other status parameters which are stored in the |.aux| files)
takes the same value after the conditional processing.
Otherwise the page numbers may take divergent values
depending on which part is compiled.

For example, a title page could be declared by:
%
\begin{center}
\begin{tabular}{l}
|\ifchilddoc\||else|\\
|\addtocounter{page}{-1}|\\
\textit{code for title page}\\
|\newpage|\\
|\||fi|
\end{tabular}
\end{center}
%
A banner page for the child documents can be generated by:
%
\begin{center}
\begin{tabular}{l}
|\ifchilddoc|\\
|\addtocounter{page}{-1}|\\
\textit{code for banner page}\\
|\newpage|\\
|\||fi|
\end{tabular}
\end{center}
%
Here one could write a message such as:
\begin{center}
|This is the part \childdocname{} of \childdocjob{}.|
\end{center}

%%%%%%%%%%%%%%%%%%%%%%%%%%%%%%%%%%%%%%%%%%%%%%%%%%%%%%%%%%%%%%%%%%%%%%%%%%%%%%%%
\subsection{Flags}
\label{sec:flags}

The package makes it easy to generate different versions
of the main or child documents.
To this end compilation flags can be defined
and assigned different default values.
They will be particularly useful in conjunction
with the forwarding mechanism described in \secref{sec:forward}.

For example, it may be useful to have a flag |\version|
which can be set to |draft| or |final|.
The document source will contain some conditional code
depending on the value of |\version|.
Suppose further, the flag should default to |final| for the main file
and to |draft| for child files
which is a natural assignment for editing the document.
This is achieved by placing the following code
in the preamble of the main document
(below the |\childdocmain| directive):
%
\begin{center}
\begin{tabular}{l}
|\ifchilddoc|\\
|\providecommand{\version}{draft}|\\
|\||else|\\
|\providecommand{\version}{final}|\\
|\||fi|
\end{tabular}
\end{center}
%
The definition by |\providecommand| makes sure
that previous definitions are not overwritten.
Further statements |\providecommand{\version}{...}|
can thus be added before the above code to override it.

For the main file, one might add a line
(between |\childdocmain| and the above block)
%
\begin{center}
|%\ifchilddoc\||else\providecommand{\version}{draft}\||fi|
\end{center}
%
which can be uncommented to produce a draft version.
Likewise one can add a line to the very top of a child file
(above the |\childdocof{|\textit{main}|}| directive)
%
\begin{center}
|%\providecommand{\version}{final}|
\end{center}
%
which can be uncommented to produce the final version of this child document.

%%%%%%%%%%%%%%%%%%%%%%%%%%%%%%%%%%%%%%%%%%%%%%%%%%%%%%%%%%%%%%%%%%%%%%%%%%%%%%%%
\subsection{Forwarding}
\label{sec:forward}

Different versions of the main or child documents
using compilation flags as described in \secref{sec:flags}
can be (permanently) stored in different files
for convenient compilation, viewing and distribution.
To this end, the package defines a command
to pass on compilation to a different file:

%%%%%%%%%%%%%%%%%%%%%%%%%%%%%%%%%%%%%%%%
\DescribeMacro{\childdocforward}
The command |\childdocforward| redirects processing to
another source file:
%
\begin{center}
\begin{tabular}{l}
|\input{childdoc.def}|\\
|\childdocforward[|\textit{main}|]{|\textit{dest}|}|\\
\end{tabular}
\end{center}
%
The argument \textit{dest} is the destination file
(without extension).
It should be the main file or one of the child files.
Note that further \textsf{childdoc} directives
such as |\childdocof| and |\childdocforward|
in the indicated file will be processed in this form.
The optional argument \textit{main}
passes on directly to the main file \textit{main}
while pretending to compile the child \textit{dest}.
This form behaves as if \textit{dest}
issues |\childdocof{|\textit{main}|}| right away,
and no further \textsf{childdoc} directives will be processed.

%%%%%%%%%%%%%%%%%%%%%%%%%%%%%%%%%%%%%%%%
\DescribeMacro{\...prefix}
In the alternative form |\childdocforwardprefix|,
%
\begin{center}
\begin{tabular}{l}
|\input{childdoc.def}|\\
|\childdocforwardprefix[|\textit{main}|]{|\textit{prefix}|}{|\textit{dest}|}|
\end{tabular}
\end{center}
%
the destination file is determined by a pattern
depending on the current file:
To make this work, the current file must be called
`{\textit{prefix}\hspace{0.2em}\textit{suffix}}'
with \textit{prefix} matching precisely the argument.
Processing is then passed on to the file
`{\textit{dest}\hspace{0.2em}\textit{suffix}}'.
Surely, the same effect is achieved by
directly specifying the
argument `{\textit{dest}\hspace{0.2em}\textit{suffix}}'
in the first form.
However, that requires to set up a different file
for each child. With the alternative form of the command
all these files can have exactly the same content
which simplifies setting them up and maintaining them.

For example, the following file |draft.tex|
with a compilation flag |\version| as described in \secref{sec:flags}
compiles the main document as a draft:
%
\begin{center}
\begin{tabular}{l}
|\def\version{draft}|\\
|\input{childdoc.def}|\\
|\childdocforward{|\textit{main}|}|
\end{tabular}
\end{center}
%
Likewise, the following files |final|\textit{nn}|.tex|
compile the final version of the child document
|child|\textit{nn}|.tex|:
%
\begin{center}
\begin{tabular}{l}
|\def\version{final}|\\
|\input{childdoc.def}|\\
|\childdocforwardprefix{final}{child}|
\end{tabular}
\end{center}
%

Note that when several versions of a main file and/or of each child file
are to be generated, it may be convenient to set up a |Makefile| or
shell script to automatise the process.

%%%%%%%%%%%%%%%%%%%%%%%%%%%%%%%%%%%%%%%%%%%%%%%%%%%%%%%%%%%%%%%%%%%%%%%%%%%%%%%%
\subsection{Command Line Processing}
\label{sec:commandline}

The effect of redirection files can also be achieved by invoking
the \LaTeX{} compiler with a more elaborate command line.
Most conveniently this should be done as part
of a shell script or a |Makefile|.

When using \textsf{childdoc} in the main file, the following
command lines effectively perform a redirection
(note that depending on the shell being used,
backslashes may have to be doubled: `|\|' $\to$ `|\\|'):
%
\begin{center}
|... -jobname "|\textit{target}|" |\\|"|[\textit{flags}]%
|\input{childdoc.def}\childdocforward[|\textit{main}|]{|\textit{dest}|}"|
\end{center}
%
Here \textit{target} is the name of the output file,
\textit{main} is the name of the main file
and \textit{dest} is the name of the main or child file to be processed
(all filenames without extensions).
The optional argument \textit{main} can be omitted
if \textit{main} matches \textit{dest}.
Optionally, compilation \textit{flags} can be defined via |\def| commands.
This command line makes the \TeX{} engine believe
it is compiling the file \textit{target}
whose content is specified as the latter parameter.
The provided code then forwards the processing to
\textit{main} or \textit{dest} as described in \secref{sec:forward}.

%%%%%%%%%%%%%%%%%%%%%%%%%%%%%%%%%%%%%%%%%%%%%%%%%%%%%%%%%%%%%%%%%%%%%%%%%%%%%%%%
\subsection{Include by Input}
\label{sec:input}

Including child documents by |\include| has some restrictions by design.
Most notably, the content of a child document always occupies
its own set of pages; pages cannot be shared between child documents.
Usually, this behaviour makes perfect sense
because each child document contain an essential part of the document.
However, in some situations it may be desirable to compose
a document from a collection of parts
without having mandatory page breaks between then.
For this case, the package
provides a mechanism to include parts
by |\input| which can also be processed individually.
However, by construction this mechanism
requires manual handling of the content to be output.

%%%%%%%%%%%%%%%%%%%%%%%%%%%%%%%%%%%%%%%%
\DescribeMacro{\ifchilddocmanual}
The main file should be prepared as usual, see \secref{sec:include}.
However, the document body must make a distinction
between processing of an individual part and of the main document, e.g.:
%
\begin{center}
\begin{tabular}{l}
|\ifchilddocmanual|\\
|\input{\childdocname}|\\
|\||else|\\
\textit{document body with }|\input{|\textit{part}|}|\\
|\||fi|
\end{tabular}
\end{center}
%
The conditional |\ifchilddocmanual| is true whenever
a part to be included by |\input| is being compiled,
and the name of the part is stored in |\childdocname|.

%%%%%%%%%%%%%%%%%%%%%%%%%%%%%%%%%%%%%%%%
\DescribeMacro{\childdocby}
Each part to be included by |\input| should start with:
%
\begin{center}
\begin{tabular}{l}
|\input{childdoc.def}|\\
|\childdocby{|\textit{main}|}|\\
\end{tabular}
\end{center}
%
The directive |\childdocby| is similar to |\childdocof|
described in \secref{sec:include},
but the subsequent selection of content must be done manually.
To that end, both |\ifchilddoc| and |\ifchilddocmanual|
will be true upon processing of a part,
and the name of the part is stored in |\childdocname|.
Note that |\jobname| will be set to the filename of the current part
so that each part receives an individual |.aux| file
that does not interfere with the |.aux| file(s) of the main document.
This behaviour can be altered by the alternative form
|\childdocby[*]{|\textit{main}|}| (with a non-empty optional argument)
which uses the |.aux| file of the main document
by setting |\jobname| to \textit{main}.

%%%%%%%%%%%%%%%%%%%%%%%%%%%%%%%%%%%%%%%%%%%%%%%%%%%%%%%%%%%%%%%%%%%%%%%%%%%%%%%%
\subsection{Driver Development}
\label{sec:driver}

The \textsf{childdoc} mechanism can also be use for the development
of definition files such as \LaTeX{} styles or classes.
This case differs from the above setup with multiple parts
included by |\include| in that no |\includeonly| should be invoked.
This can be achieved by starting the include file
(before |\ProvidesPackage|) with:
%
\begin{center}
\begin{tabular}{l}
|\input{childdoc.def}|\\
|\childdocforward{|\textit{main}|}|\\
\end{tabular}
\end{center}
%
or alternatively with:
%
\begin{center}
\begin{tabular}{l}
|\input{childdoc.def}|\\
|\childdocby{|\textit{main}|}|\\
\end{tabular}
\end{center}
%
Both forms have slightly different effects as described above.
The main file is prepared as usual, see \secref{sec:include}.

%%%%%%%%%%%%%%%%%%%%%%%%%%%%%%%%%%%%%%%%%%%%%%%%%%%%%%%%%%%%%%%%%%%%%%%%%%%%%%%%
\subsection{Legacy Detection}
\label{sec:detection}

The directive |\childdocmain| in the main file can detect
whether the complete document or merely a child is to be compiled
even without using the directive |\childdocof|.
This method is deprecated because it is less robust
and there is no compelling reason to use it;
it is merely provided for backward compatibility
and it may be removed in future versions.

If the detection mechanism is to be used,
it is mandatory to correctly specify
the filename of the main file as the argument of |\childdocmain|:
%
\begin{center}
\begin{tabular}{l}
|\input{childdoc.def}|\\
|\childdocmain{|\textit{main}|}|\\
\end{tabular}
\end{center}
%
If |\jobname| does not match the argument \textit{main} of |\childdocmain|,
it is assumed that |\jobname| points to the child file to be compiled.
When using |\childdocmain| with the main file specified as argument,
it suffices to start a child file
with just |\input{|\textit{main}|}|
without loading of the package and using |\childdocof|.
If instead all processing is done
with the appropriate \textsf{childdoc} directives,
the argument of \textit{main} of |\childdocmain| can be empty.

An alternative version of the command line processing described
in \secref{sec:commandline} using the detection mechanism reads:
%
\begin{center}
|... -jobname "|\textit{target}|" "|[\textit{flags}]%
[|\def\jobname{|\textit{dest}|}|]|\input{|\textit{main}|}"|
\end{center}

%%%%%%%%%%%%%%%%%%%%%%%%%%%%%%%%%%%%%%%%%%%%%%%%%%%%%%%%%%%%%%%%%%%%%%%%%%%%%%%%
\subsection{Manual Code}
\label{sec:manual}

In case one cannot be certain whether the definitions file |childdoc.def|
is installed on the target \TeX{} distribution
and one prefers not to ship it,
it is conceivable to paste a few relevant commands into the sources.

To that end, drop all statements |\input{childdoc.def}|
and perform the replacements as outlined below.
Instead of |\childdocmain{|\textit{main}|}| add the following code
to the top of the main file:
%
\begin{center}
\begin{tabular}{l}
|\||ifdefined\childdocname\endinput\||fi\newif\ifchilddoc|\\
|\edef\childdocname{\scantokens\expandafter{\jobname\noexpand}}|\\
|\def\childdocmain{|\textit{main}|}\||ifx\childdocmain\childdocname\||else|\\
|\childdoctrue\includeonly{\childdocname}\let\jobname\childdocmain\||fi|\\
\end{tabular}
\end{center}
%
Instead of |\childdocof{|\textit{main}|}| just include the main file
at the top of each child file:
%
\begin{center}
|\input{|\textit{main}|}|
\end{center}
%
A simple redirection |\childdocforward{|\textit{dest}|}| is achieved by:
%
\begin{center}
|\def\jobname{|\textit{dest}|}\input{\jobname}|
\end{center}
%
The redirection with prefix
|\childdocforwardprefix[|\textit{prefix}|]{|\textit{dest}|}|
is accomplished by:
%
\begin{center}
\begin{tabular}{l}
|{\edef\jobname{\scantokens\expandafter{\jobname\noexpand}}|\\
|\def\redirectjob |\textit{prefix}|#1~~~{\gdef\jobname{|\textit{dest}|#1}}|\\
|\expandafter\redirectjob\jobname~~~}\input{\jobname}|
\end{tabular}
\end{center}

In an alternative approach,
child documents can be compiled by a specific command line
without additional code or specific definitions:
%
\begin{center}
|... -jobname "|\textit{target}|" "|[\textit{flags}]%
|\includeonly{|\textit{dest}|}\input{|\textit{main}|}"|
\end{center}
%

%%%%%%%%%%%%%%%%%%%%%%%%%%%%%%%%%%%%%%%%%%%%%%%%%%%%%%%%%%%%%%%%%%%%%%%%%%%%%%%%
%%%%%%%%%%%%%%%%%%%%%%%%%%%%%%%%%%%%%%%%%%%%%%%%%%%%%%%%%%%%%%%%%%%%%%%%%%%%%%%%
\section{Information}

%%%%%%%%%%%%%%%%%%%%%%%%%%%%%%%%%%%%%%%%%%%%%%%%%%%%%%%%%%%%%%%%%%%%%%%%%%%%%%%%
\subsection{Copyright}

Copyright \copyright{} 2017--2018 Niklas Beisert

This work may be distributed and/or modified under the
conditions of the \LaTeX{} Project Public License, either version 1.3
of this license or (at your option) any later version.
The latest version of this license is in
  \url{http://www.latex-project.org/lppl.txt}
and version 1.3 or later is part of all distributions of \LaTeX{}
version 2005/12/01 or later.

This work has the LPPL maintenance status `maintained'.

The Current Maintainer of this work is Niklas Beisert.

This work consists of the files |README.txt|, |childdoc.ins| and |childdoc.dtx|
as well as the derived files |childdoc.def|, |cdocsamp.tex|
with |cdocsch1.tex|, |cdocsch2.tex|, |cdocspt3.tex|, |cdocspt4.tex|,
|cdocsdrf.tex|, |cdocsfn1.tex|, |cdocsfn2.tex|
as well as |childdoc.pdf|.

%%%%%%%%%%%%%%%%%%%%%%%%%%%%%%%%%%%%%%%%%%%%%%%%%%%%%%%%%%%%%%%%%%%%%%%%%%%%%%%%
\subsection{Files and Installation}

The package consists of the files:
%
\begin{center}
\begin{tabular}{ll}
    |README.txt|   & readme file \\
    |childdoc.ins| & installation file \\
    |childdoc.dtx| & source file \\
    |childdoc.def| & definition file \\
    |cdocsamp.tex| & sample main file \\
    |cdocsch1.tex| & sample include file \\
    |cdocsch2.tex| & sample include file \\
    |cdocspt3.tex| & sample part file \\
    |cdocspt4.tex| & sample part file \\
    |cdocsdrf.tex| & sample redirection file \\
    |cdocsfn1.tex| & sample redirection file \\
    |cdocsfn2.tex| & sample redirection file \\
    |childdoc.pdf| & manual
\end{tabular}
\end{center}
%
The distribution consists of the files
|README.txt|, |childdoc.ins| and |childdoc.dtx|.
%
\begin{itemize}
\item
Run (pdf)\LaTeX{} on |childdoc.dtx|
to compile the manual |childdoc.pdf| (this file).
\item
Run \LaTeX{} on |childdoc.ins| to create the definitions file |childdoc.def|
and the sample |cdocsamp.tex| with include files
|cdocsch1.tex|, |cdocsch2.tex|, |cdocspt3.tex|, |cdocspt4.tex|,
|cdocsdrf.tex|, |cdocsfn1.tex|, |cdocsfn2.tex|.
Then copy the file |childdoc.def| to an appropriate directory of your \LaTeX{}
distribution, e.g.\ \textit{texmf-root}|/tex/latex/childdoc|.
\end{itemize}

%%%%%%%%%%%%%%%%%%%%%%%%%%%%%%%%%%%%%%%%%%%%%%%%%%%%%%%%%%%%%%%%%%%%%%%%%%%%%%%%
\subsection{Related CTAN Packages}

There are several other packages which offer a similar functionality:
%
\begin{itemize}
\item
The packages
\href{http://ctan.org/pkg/docmute}{\textsf{docmute}},
\href{http://ctan.org/pkg/includex}{\textsf{includex}} and
\href{http://ctan.org/pkg/standalone}{\textsf{standalone}}
provide commands to include only the document body of
a child file thus allowing both files to be compiled individually.
\item
The packages \href{http://ctan.org/pkg/subdocs}{\textsf{subdocs}}
and \href{http://ctan.org/pkg/subfiles}{\textsf{subfiles}}
provide structures in which the main and child documents can be
encapsulated and allowing them to be compiled individually.
The inclusion mechanism is different from the conventional |\include|.
\item
The package \href{http://ctan.org/pkg/combine}{\textsf{combine}}
is an elaborate solution to combine several documents into one.
\end{itemize}
%
See also the CTAN topic \href{http://ctan.org/topic/subdocs}{\textsf{subdocs}}
for further related packages.
The present package differs from the above solutions in that
a document structure constructed with the conventional |\include| mechanism
just needs two extra commands at the top of every file
such that all constituent files can be compiled individually.

%%%%%%%%%%%%%%%%%%%%%%%%%%%%%%%%%%%%%%%%%%%%%%%%%%%%%%%%%%%%%%%%%%%%%%%%%%%%%%%%
%\subsection{Feature Suggestions}
%
%The following is a list of features which may be useful for future
%versions of this package:
%%
%\begin{itemize}
%\item
%\ldots
%\end{itemize}

%%%%%%%%%%%%%%%%%%%%%%%%%%%%%%%%%%%%%%%%%%%%%%%%%%%%%%%%%%%%%%%%%%%%%%%%%%%%%%%%
\subsection{Revision History}

%%%%%%%%%%%%%%%%%%%%%%%%%%%%%%%%%%%%%%%%
\paragraph{v2.0:} 2018/12/30

\begin{itemize}
\item
immediate forward processing
\item
added |\childdocby| mechanism
\item
manual restructured
\end{itemize}

%%%%%%%%%%%%%%%%%%%%%%%%%%%%%%%%%%%%%%%%
\paragraph{v1.6:} 2018/01/17

\begin{itemize}
\item
application for development of include files
\item
corrections to manual
\end{itemize}

%%%%%%%%%%%%%%%%%%%%%%%%%%%%%%%%%%%%%%%%
\paragraph{v1.5:} 2017/05/21

\begin{itemize}
\item
more complete structuring introduced
\item
|\childdocof| introduced
\item
|\childdoc| renamed to |\childdocmain|
\item
|\childredirect| renamed to |\childdocforward| and |\childdocforwardprefix|
and functionality expanded
\end{itemize}

%%%%%%%%%%%%%%%%%%%%%%%%%%%%%%%%%%%%%%%%
\paragraph{v1.0:} 2017/04/27

\begin{itemize}
\item
manual and install package
\item
first version published on CTAN
\end{itemize}

%%%%%%%%%%%%%%%%%%%%%%%%%%%%%%%%%%%%%%%%
\paragraph{v0.6:} 2017/04/26

\begin{itemize}
\item
redirection mechanism added
\end{itemize}

%%%%%%%%%%%%%%%%%%%%%%%%%%%%%%%%%%%%%%%%
\paragraph{v0.5:} 2017/04/26

\begin{itemize}
\item
functionality in definition file
\end{itemize}


%%%%%%%%%%%%%%%%%%%%%%%%%%%%%%%%%%%%%%%%%%%%%%%%%%%%%%%%%%%%%%%%%%%%%%%%%%%%%%%%
%%%%%%%%%%%%%%%%%%%%%%%%%%%%%%%%%%%%%%%%%%%%%%%%%%%%%%%%%%%%%%%%%%%%%%%%%%%%%%%%
%%%%%%%%%%%%%%%%%%%%%%%%%%%%%%%%%%%%%%%%%%%%%%%%%%%%%%%%%%%%%%%%%%%%%%%%%%%%%%%%
\appendix

\settowidth\MacroIndent{\rmfamily\scriptsize 000\ }

 \DocInput{childdoc.dtx}

\end{document}
%</driver>
% \fi
%
% %%%%%%%%%%%%%%%%%%%%%%%%%%%%%%%%%%%%%%%%%%%%%%%%%%%%%%%%%%%%%%%%%%%%%%%%%%%%%%
% %%%%%%%%%%%%%%%%%%%%%%%%%%%%%%%%%%%%%%%%%%%%%%%%%%%%%%%%%%%%%%%%%%%%%%%%%%%%%%
% \section{Sample}
%\iffalse
%<*samplemain>
%\fi
%
% The following presents a sample document
% with two chapters, two parts, a title page,
% a compile flag as well as three forwarding files to set the flag.
% It consists of eight |.tex| files:
% \begin{center}
% \begin{tabular}{ll}
% |cdocsamp.tex|&main file\\
% |cdocsch1.tex|&include file for chapter 1\\
% |cdocsch2.tex|&include file for chapter 2\\
% |cdocspt3.tex|&include file for part 3\\
% |cdocspt4.tex|&include file for part 4\\
% |cdocsdrf.tex|&forwarding file for main file in draft mode\\
% |cdocsfi1.tex|&forwarding file for final version of chapter 1\\
% |cdocsfi2.tex|&forwarding file for final version of chapter 2\\
% \end{tabular}
% \end{center}
% Each of the eight files can be compiled directly by the \LaTeX{} compiler.
%
% %%%%%%%%%%%%%%%%%%%%%%%%%%%%%%%%%%%%%%
% \paragraph{Main File.}
%
% The main file is called |cdocsamp.tex|.
%
% Load the \textsf{childdoc} definitions and
% declare the filename for the main document:
%    \begin{macrocode}
\input{childdoc.def}
\childdocmain{}
%    \end{macrocode}

% Optional override for |\version| flag:
%    \begin{macrocode}
%%\ifchilddoc\else\providecommand{\version}{draft}\fi
%    \end{macrocode}

% Define the default values for the |\version| flag
% (|final| for the main file and |draft| for childs):
%    \begin{macrocode}
\ifchilddoc
\providecommand{\version}{draft}
\else
\providecommand{\version}{final}
\fi
%    \end{macrocode}

% Load the standard document class:
%    \begin{macrocode}
\documentclass[12pt]{article}
%    \end{macrocode}

% Start the document body:
%    \begin{macrocode}
\begin{document}
%    \end{macrocode}

% Declare a title page.
% Print title, part of document being processed and version flag:
%    \begin{macrocode}
\addtocounter{page}{-1}
\begin{center}
{\LARGE\bfseries{}childdoc example\par}
\vspace{1cm}
\ifchilddoc
\ifchilddocmanual part\else chapter\fi:
`\childdocname' of `\childdocjob'\par
\else
main document: `\childdocjob'\par
\fi
version: \version\par
\end{center}
\newpage
%    \end{macrocode}

% Manually include selected file,
% otherwise process as usual:
%    \begin{macrocode}
\ifchilddocmanual
\section*{part `\childdocname'}
\input{\childdocname}
\else
%    \end{macrocode}

% Include the two chapters:
%    \begin{macrocode}
\include{cdocsch1}
\include{cdocsch2}
%    \end{macrocode}

% Include the two parts unless only chapters should be displayed:
%    \begin{macrocode}
\ifchilddoc\else
\section{part three}
\input{cdocspt3}
\section{part four}
\input{cdocspt4}
\fi
%    \end{macrocode}

% Process as usual until here:
%    \begin{macrocode}
\fi
%    \end{macrocode}

% End of document body:
%    \begin{macrocode}
\end{document}
%    \end{macrocode}
%\iffalse
%</samplemain>
%\fi
%
% %%%%%%%%%%%%%%%%%%%%%%%%%%%%%%%%%%%%%%
% \paragraph{Chapter Include Files.}
%
% The include files are called |cdocsch1.tex| and |cdocsch2.tex|.
%
%\iffalse
%<*samplechap1|samplechap2>
%\fi

% Optional override for |\version| flag:
%    \begin{macrocode}
%%\providecommand{\version}{final}
%    \end{macrocode}

% Include the main document:
%    \begin{macrocode}
\input{childdoc.def}
\childdocof{cdocsamp}
%    \end{macrocode}

%\iffalse
%</samplechap1|samplechap2>
%\fi
%
%\iffalse
%<*samplechap1>
%\fi
% Some text for chapter 1:
%    \begin{macrocode}
\section{one}
some text in chapter one
%    \end{macrocode}

%\iffalse
%</samplechap1>
%\fi
% Some text for chapter 2:
%\iffalse
%<*samplechap2>
%\fi
%    \begin{macrocode}
\section{two}
more text in chapter two
%    \end{macrocode}

%\iffalse
%</samplechap2>
%\fi
%
% %%%%%%%%%%%%%%%%%%%%%%%%%%%%%%%%%%%%%%
% \paragraph{Part Include Files.}
%
% The include files are called |cdocspt3.tex| and |cdocspt4.tex|.
%
%\iffalse
%<*samplepart3|samplepart4>
%\fi

% Optional override for |\version| flag:
%    \begin{macrocode}
%%\providecommand{\version}{final}
%    \end{macrocode}

% Include the main document:
%    \begin{macrocode}
\input{childdoc.def}
\childdocby{cdocsamp}
%    \end{macrocode}

%\iffalse
%</samplepart3|samplepart4>
%\fi
%
%\iffalse
%<*samplepart3>
%\fi
% Some text for part 3:
%    \begin{macrocode}
some text in part three
%    \end{macrocode}

%\iffalse
%</samplepart3>
%\fi
% Some text for part 4:
%\iffalse
%<*samplepart4>
%\fi
%    \begin{macrocode}
more text in part four
%    \end{macrocode}

%\iffalse
%</samplepart4>
%\fi
%
% %%%%%%%%%%%%%%%%%%%%%%%%%%%%%%%%%%%%%%
% \paragraph{Forwarding for a Complete Draft.}
%
% The following forwarding file |cdocsdrf.tex|
% compiles the main document in draft mode:
%\iffalse
%<*sampledraft>
%\fi
%    \begin{macrocode}
\def\version{draft}
\input{childdoc.def}
\childdocforward{cdocsamp}
%    \end{macrocode}

%\iffalse
%</sampledraft>
%\fi
%
% %%%%%%%%%%%%%%%%%%%%%%%%%%%%%%%%%%%%%%
% \paragraph{Forwarding for Final Version of the Chapters.}
%
% The following forwarding files |cdocsfn1.tex| and |cdocsfn2.tex|
% (with identical content)
% compile the final versions of the child documents
% |cdocsch1.tex| and |cdocsch2.tex|, respectively:
%\iffalse
%<*samplefinal>
%\fi
%    \begin{macrocode}
\def\version{final}
\input{childdoc.def}
\childdocforwardprefix[cdocsamp]{cdocsfn}{cdocsch}
%    \end{macrocode}

%\iffalse
%</samplefinal>
%\fi
%
% %%%%%%%%%%%%%%%%%%%%%%%%%%%%%%%%%%%%%%
% \paragraph{Command Line Processing.}
%
% The following three command lines generate the output files
% |cdocscld|, |cdocscl1| and |cdocscl2|
% which should be identical to
% |cdocsdrf|, |cdocsch1| and |cdocsfn2|, respectively:
% \begin{center}
% \begin{tabular}{l}
% |latex -jobname cdocscld \|\\
% |  "\def\version{draft}\input{childdoc.def}\childdocforward{cdocsamp}"|\\
% |latex -jobname cdocscl1 \|\\
% |  "\input{childdoc.def}\childdocforward[cdocsamp]{cdocsch1}"|\\
% |latex -jobname cdocscl2 \|\\
% |  "\def\version{final}\input{childdoc.def}\childdocforward{cdocsch2}"|
% \end{tabular}
% \end{center}
% Note that the trailing backslash on each first line
% merely continues the input to the second line
% (for convenient cut ant paste).
% Furthermore, the command |latex| can be replaced by any
% of its alternative versions such as |pdflatex|.
%
% %%%%%%%%%%%%%%%%%%%%%%%%%%%%%%%%%%%%%%%%%%%%%%%%%%%%%%%%%%%%%%%%%%%%%%%%%%%%%%
% %%%%%%%%%%%%%%%%%%%%%%%%%%%%%%%%%%%%%%%%%%%%%%%%%%%%%%%%%%%%%%%%%%%%%%%%%%%%%%
% \section{Implementation}
%\iffalse
%<*package>
%\fi
%
% This section describes the definitions file |childdoc.def|.

% The definitions cannot be loaded using |\usepackage| or |\RequirePackage|
% which has a mechanism to prevent loading a style file more than once.
% When loading the definitions by means of |\input|
% multiple instances have to be prevented manually:
%\iffalse
%This code needs to be before the `\ProvidesFile' directive
%which is defined at the beginning of this file.
%Therefore it is also placed there and commented out here.
%</package>
%<*discard>
%\fi
%    \begin{macrocode}
\ifdefined\childdocmain\endinput\fi
%    \end{macrocode}
%\iffalse
%</discard>
%<*package>
%\fi
%
% \macro{\ifchilddoc}
% \macro{\ifchilddocmanual}
% The conditional |\ifchilddoc| tells whether a
% child (true) or main (false) document is being compiled.
% The conditional |\ifchilddocmanual| tells whether
% the |\includeonly| mechanism is used (false) or
% the selection of child files must be performed manually (true).
% The definitions initialise to false:
%    \begin{macrocode}
\newif\ifchilddoc
\newif\ifchilddocmanual
%    \end{macrocode}

% \macro{\childdocname}
% \macro{\childdocjob}
% The macro |\childdocname| stores the name of the main document
% to be compiled. The macro |\childdocjob| stores the name of
% the document on which the \LaTeX{} compiler was originally invoked.
% The content of |\jobname| cannot be compared
% to filenames specified in the source due to different catcodes.
% The following code rescans |\jobname|, stores the result
% in |\childdocname| and saves a copy in |\childdocjob|:
%    \begin{macrocode}
\edef\childdocname{\scantokens\expandafter{\jobname\noexpand}}
\let\childdocjob\childdocname
%    \end{macrocode}

% \macro{\childdocdisable}
% The macro |\childdocdisable| prevents the main file
% from being processed more than once.
% At this stage, the main document command |\childdocmain|
% is assumed to be called once again where it should do nothing.
% Any subsequent call to it should prevent
% a secondary processing of the main document
% It overwrites the forwarding commands
% |\childdocof| and |\childdocforward|
% with empty macros to prevent further inclusions of the main document:
%    \begin{macrocode}
\newcommand{\childdocdisable}
{
  \renewcommand{\childdocmain}[1]{\renewcommand{\childdocmain}[1]{\endinput}}
  \renewcommand{\childdocof}[1]{}
  \renewcommand{\childdocby}[2][]{}
  \renewcommand{\childdocforward}[2][]{}
  \renewcommand{\childdocdisable}{}
}
%    \end{macrocode}

% \macro{\childdocmain}
% The macro |\childdocmain| is to be called at the top of the main file
% with nothing or the main filename (without extension) as argument.
% First, it breaks loops.
% If the argument is not empty and does not match |\childdocname|
% (which is set by the first inclusion of |childdoc.def|),
% |\ifchilddoc| is set to true, |\includeonly| is applied to the child file
% and |\jobname| is set to the main file
% (for proper handling of |.aux| files):
%    \begin{macrocode}
\newcommand{\childdocmain}[1]
{
  \childdocdisable\childdocmain{}
  \if?#1?\else
    \begingroup
      \def\childdoctmp{#1}
      \ifx\childdoctmp\childdocname
        \def\childdoctmp{}
      \else
        \def\childdoctmp
        {
          \childdoctrue
          \includeonly{\childdocname}
          \def\childdocjob{#1}
          \def\jobname{#1}
        }
      \fi
      \expandafter
    \endgroup
    \childdoctmp
  \fi
}
%    \end{macrocode}

% \macro{\childdocof}
% The command |\childdocof| redirects
% compilation to the main file |#1|.
%    \begin{macrocode}
\newcommand{\childdocof}[1]
{
  \childdocdisable
  \childdoctrue
  \includeonly{\childdocname}
  \def\jobname{#1}
  \def\childdocjob{#1}
  \input{#1}
}
%    \end{macrocode}

% \macro{\childdocby}
% The command |\childdocby| ....
%    \begin{macrocode}
\newcommand{\childdocby}[2][]
{
  \childdocdisable
  \childdoctrue
  \childdocmanualtrue
  \if?#1?\else
    \def\jobname{#2}
  \fi
  \def\childdocjob{#2}
  \input{#2}
  \endinput
}
%    \end{macrocode}

% \macro{\childdocforward}
% The command |\childdocforward| redirects
% compilation to the main file or
% (if the optional argument is given) a child file.
% Parameters are set as if the main file
% or a child file starting with |\childdocof| was compiled.
% Then compilation is handed over to the main file:
%    \begin{macrocode}
\newcommand{\childdocforward}[2][]
{
  \begingroup
    \if?#1?
      \def\childdoctmp
      {
        \def\childdocname{#2}
        \def\childdocjob{#2}
        \def\jobname{#2}
        \input{#2}
        \endinput
      }
    \else
      \def\childdoctmp
      {
        \childdocdisable
        \def\childdocname{#2}
        \childdoctrue
        \includeonly{#2}
        \def\childdocjob{#1}
        \def\jobname{#1}
        \input{#1}
        \endinput
      }
    \fi
    \expandafter
  \endgroup
  \childdoctmp
}
%    \end{macrocode}

% \macro{\childdocforwardprefix}
% The command |\childdocforwardprefix| redirects
% compilation to the main or a child file by means of a pattern.
% The prefix |#1| in the current filename is replaced by |#2|
% and the suffix of the current filename is kept
% (it is assumed that the filename does not contain the substring `|~~~|'
% which is used as a delimiter).
% Compilation is handed over to the new file by |\childdocforward|:
%    \begin{macrocode}
\newcommand{\childdocforwardprefix}[3][]
{
  \begingroup
    \def\childdocextract #2##1~~~{\def\childdoctmp{\childdocforward[#1]{#3##1}}}
    \expandafter\childdocextract\childdocname~~~
    \expandafter
  \endgroup
  \childdoctmp
}
%    \end{macrocode}

% \macro{\childdoc}
% The deprecated macro |\childdoc| is a legacy version of |\childdocmain|:
%    \begin{macrocode}
\newcommand{\childdoc}{\childdocmain}
%    \end{macrocode}

% \macro{\childdocredirect}
% The deprecated macro |\childdocredirect| is a legacy version
% of |\childdocforward| and |\childdocforwardprefix|:
%    \begin{macrocode}
\newcommand{\childdocredirect}[2][]
{
  \begingroup
    \if?#1?
      \def\childdoctmp{\childdocforward{#2}}
    \else
      \def\childdoctmp{\childdocforwardprefix{#1}{#2}}
    \fi
    \expandafter
  \endgroup
  \childdoctmp
}
%    \end{macrocode}

%\iffalse
%</package>
%\fi
%
\endinput
\childdocforward[cdocsamp]{cdocsch1}"|\\
% |latex -jobname cdocscl2 \|\\
% |  "\def\version{final}% \iffalse
%
% childdoc.dtx Copyright (C) 2017-2018 Niklas Beisert
%
% This work may be distributed and/or modified under the
% conditions of the LaTeX Project Public License, either version 1.3
% of this license or (at your option) any later version.
% The latest version of this license is in
%   http://www.latex-project.org/lppl.txt
% and version 1.3 or later is part of all distributions of LaTeX
% version 2005/12/01 or later.
%
% This work has the LPPL maintenance status `maintained'.
%
% The Current Maintainer of this work is Niklas Beisert.
%
% This work consists of the files childdoc.dtx and childdoc.ins
% and the derived files childdoc.def and cdocsamp.tex with
% cdocsch1.tex, cdocsch2.tex, cdocsdrf.tex, cdocsfn1.tex, cdocsfn2.tex.
%
%<package>\ifdefined\childdocmain\endinput\fi
%<package>\ProvidesFile{childdoc.def}[2018/12/30 v2.0 child document driver]
%<samplemain>\ProvidesFile{cdocsamp.tex}[2018/12/30 v2.0 sample for childdoc]
%<*driver>
%\ProvidesFile{childdoc.drv}[2018/12/30 v2.0 childdoc reference manual file]
\PassOptionsToClass{10pt,a4paper}{article}
\documentclass{ltxdoc}

\usepackage[margin=35mm]{geometry}
\usepackage{hyperref}
\usepackage{hyperxmp}
\usepackage[usenames]{color}

\hypersetup{colorlinks=true}
\hypersetup{pdfstartview=FitH}
\hypersetup{pdfpagemode=UseNone}
\hypersetup{pdfsource={}}
\hypersetup{pdflang={en-UK}}
\hypersetup{pdfcopyright={Copyright 2017-2018 Niklas Beisert.
  This work may be distributed and/or modified under the
  conditions of the LaTeX Project Public License, either version 1.3
  of this license or (at your option) any later version.}}
\hypersetup{pdflicenseurl={http://www.latex-project.org/lppl.txt}}
\hypersetup{pdfcontactaddress={ETH Zurich, ITP, HIT K,
  Wolfgang-Pauli-Strasse 27}}
\hypersetup{pdfcontactpostcode={8093}}
\hypersetup{pdfcontactcity={Zurich}}
\hypersetup{pdfcontactcountry={Switzerland}}
\hypersetup{pdfcontactemail={nbeisert@itp.phys.ethz.ch}}
\hypersetup{pdfcontacturl={http://people.phys.ethz.ch/\xmptilde nbeisert/}}

\newcommand{\secref}[1]{\hyperref[#1]{section \ref*{#1}}}

\parskip1ex
\parindent0pt
\let\olditemize\itemize
\def\itemize{\olditemize\parskip0pt}

\begin{document}

\title{The \textsf{childdoc} Package}
\hypersetup{pdftitle={The childdoc Package}}
\author{Niklas Beisert\\[2ex]
  Institut f\"ur Theoretische Physik\\
  Eidgen\"ossische Technische Hochschule Z\"urich\\
  Wolfgang-Pauli-Strasse 27, 8093 Z\"urich, Switzerland\\[1ex]
  \href{mailto:nbeisert@itp.phys.ethz.ch}
  {\texttt{nbeisert@itp.phys.ethz.ch}}}
\hypersetup{pdfauthor={Niklas Beisert}}
\hypersetup{pdfsubject={Manual for the LaTeX2e Package childdoc}}
\date{30 December 2018, \textsf{v2.0}}
\maketitle

\begin{abstract}\noindent
\textsf{childdoc} is a \LaTeXe{} package
that enables the direct compilation
of document sections included by |\include|
to individual files.
\end{abstract}

\begingroup
\parskip0ex
\tableofcontents
\endgroup

%%%%%%%%%%%%%%%%%%%%%%%%%%%%%%%%%%%%%%%%%%%%%%%%%%%%%%%%%%%%%%%%%%%%%%%%%%%%%%%%
%%%%%%%%%%%%%%%%%%%%%%%%%%%%%%%%%%%%%%%%%%%%%%%%%%%%%%%%%%%%%%%%%%%%%%%%%%%%%%%%
\section{Introduction}

\LaTeX{} provides a mechanism to structure a large document (such as a book)
into a main file and several child files (containing the chapters)
using the |\include| command.
This mechanism is beneficial for documents
which span hundreds of pages in order to
make the source file(s) more manageable.
Moreover, compilation can be restricted to
selected child files by means of the |\includeonly| command.
The latter feature can be used to reduce the compilation time while editing
(this was significantly more useful in the earlier days of \LaTeX{})
or to generate a smaller document which is easier to navigate.
Another application of |\includeonly| is to generate
documents consisting of selected parts of the complete document.

However, there are a few drawbacks of the plain |\include| mechanism:
\begin{itemize}
\item
The child files cannot be compiled on their own,
they can only be compiled via the main file.
A naive editing environment
(such as a text editor with an option
to have the current file processed by \LaTeX)
may require one to switch to the main file before compiling;
attempting to compile the child file produces errors.
\item
The main file must be modified (each time)
to adjust the |\includeonly| command
to the present needs. This easily leaves the main file in a messy state.
\item
The generated document will always carry the filename
of the main document. This is inconvenient if
several child files are to be compiled and
to be kept for distribution.
\end{itemize}

The present package provides a simple interface
to make child files individually compilable by \LaTeX{}.
Compiling a child file then has the same effect as compiling
the main file with an |\includeonly| command
to select the appropriate child.
Moreover the generated document will carry the name of the child
rather than the main file.
This resolves all three above issues.

This feature is meant to make the editing of books,
thesis documents and lecture notes somewhat more convenient.
However, the package can also be used efficiently for
composing a series of documents (such as exercise sheets)
which are typically distributed individually.
It then assists the author in generating the individual documents
(potentially in different versions)
as well as a document containing the collected series.
Another application is in developing style files
or other kinds of included material
where compilation of the style file could redirect
to a sample or test file.

%%%%%%%%%%%%%%%%%%%%%%%%%%%%%%%%%%%%%%%%%%%%%%%%%%%%%%%%%%%%%%%%%%%%%%%%%%%%%%%%
%%%%%%%%%%%%%%%%%%%%%%%%%%%%%%%%%%%%%%%%%%%%%%%%%%%%%%%%%%%%%%%%%%%%%%%%%%%%%%%%
\section{Usage}

First of all, the package \textsf{childdoc} is \emph{not} a standard
\LaTeXe{} |.sty| style file! Therefore it needs to be invoked in
a non-standard way.

%%%%%%%%%%%%%%%%%%%%%%%%%%%%%%%%%%%%%%%%%%%%%%%%%%%%%%%%%%%%%%%%%%%%%%%%%%%%%%%%
\subsection{Included Files}
\label{sec:include}

%%%%%%%%%%%%%%%%%%%%%%%%%%%%%%%%%%%%%%%%
\DescribeMacro{\childdocmain}
To use the package, add the commands
\begin{center}
\begin{tabular}{l}
|\input{childdoc.def}|\\
|\childdocmain{}|\\
\end{tabular}
\end{center}
at the very top of the main \LaTeX{} file,
in particular \emph{before} the |\documentclass| statement!
The argument of |\childdocmain| should be left empty
(but it must be present).

%%%%%%%%%%%%%%%%%%%%%%%%%%%%%%%%%%%%%%%%
\DescribeMacro{\childdocof}
Furthermore, add the commands
\begin{center}
\begin{tabular}{l}
|\input{childdoc.def}|\\
|\childdocof{|\textit{main}|}|\\
\end{tabular}
\end{center}
at the top of every child file \textit{child}
which is included by |\include{|\textit{child}|}|
from within the main file
(or at least for those files to be compiled individually).
The argument \textit{main} must be the filename of the main file.

There are a couple of
considerations in setting up the main and child documents:

%%%%%%%%%%%%%%%%%%%%%%%%%%%%%%%%%%%%%%%%
\paragraph{Restrictions.}

Please note the following restrictions:
\begin{itemize}
\item
|\childdocmain| must be called with one argument \textit{main}
to ensure compatibility with earlier version of the package.
It must either be empty (|\childdocmain{}|)
or precisely match the filename of the main file in which it is specified.
See \secref{sec:detection} for further information.
\item
The filename \textit{main} must be specified without the |.tex| extension.
\item
The filename \textit{main} is case sensitive
(even in case-insensitive file systems)
due to internal string comparison.
\item
The argument \textit{main} should be fully expanded, it cannot be a macro.
\item
Subdirectories and special characters should be avoided in filenames.
\item
The command |\childdocmain{|\textit{main}|}| must be followed by a whitespace.
It should not be followed immediately by another command
or by a comment mark `|%|'.
This is because the \TeX{} parser reads the token immediately following
the argument of |\childdocmain| and puts it
at the beginning of every child section;
however, a white\-space is ignored.
\end{itemize}

%%%%%%%%%%%%%%%%%%%%%%%%%%%%%%%%%%%%%%%%
\paragraph{Content of Main File.}

It is advisable to place all content in the child files included by |\include|.
Any output contained in the main file will appear in all child documents
unless suppressed manually;
it cannot be suppressed automatically by the |\includeonly| directive
and thus should normally be avoided.
A method to include some content in the main file
by means of conditional processing is described in \secref{sec:conditional}.

%%%%%%%%%%%%%%%%%%%%%%%%%%%%%%%%%%%%%%%%
\paragraph{Page Numbering.}

When only a part of the document is compiled,
the appropriate numbering of pages
(as well as other status parameters)
is determined from the |.aux| files.
The latter contain information from previous passes.
However this information needs to propagate through
all intermediate child documents.
Therefore the page numbering in child documents may well
be inconsistent until the complete document is compiled at least once.

A useful (if unconventional) way to always ensure a consistent
page numbering is to restart the numbering in each child document
and denote the pages by `\textit{child}|.|\textit{page}'
where \textit{child} represents the chapter/section number of the child file.
This can be achieved by the command
|\numberwithin{page}{|\textit{child}|}|
of the \textsf{amsmath} package
where \textit{child} can be |chapter| or |section|
depending on the chosen structuring.
Alternatively, one can modify the macro |\thepage| appropriately
and reset the counter |page| at the start of each child file.

%%%%%%%%%%%%%%%%%%%%%%%%%%%%%%%%%%%%%%%%%%%%%%%%%%%%%%%%%%%%%%%%%%%%%%%%%%%%%%%%
\subsection{Conditional Processing}
\label{sec:conditional}

The package provides a mechanism to compile different versions
of a document. To customise the versions further some conditional processing
can come in handy to distinguish which version is being compiled.
The package provides two macros to describe the compilation context:

%%%%%%%%%%%%%%%%%%%%%%%%%%%%%%%%%%%%%%%%
\DescribeMacro{\ifchilddoc}
The conditional |\ifchilddoc| distinguishes between the compilation of
child documents and the main document:
%
\begin{center}
|\ifchilddoc |\textit{child-code}| |[|\||else |\textit{main-code}]| \||fi|
\end{center}

%%%%%%%%%%%%%%%%%%%%%%%%%%%%%%%%%%%%%%%%
\DescribeMacro{\childdocname}
\DescribeMacro{\childdocjob}
The macro |\childdocname| contains the filename (without extension)
of the main or child file being processed.
Note that |\childdocjob| will always contain the name of the main file.

%%%%%%%%%%%%%%%%%%%%%%%%%%%%%%%%%%%%%%%%
\paragraph{Title Page.}

Conditional processing can be used to include a title or banner page
in the main document when proper precautions are taken.
Importantly, the code in the main file should ensure that the page counter
(as well as other status parameters which are stored in the |.aux| files)
takes the same value after the conditional processing.
Otherwise the page numbers may take divergent values
depending on which part is compiled.

For example, a title page could be declared by:
%
\begin{center}
\begin{tabular}{l}
|\ifchilddoc\||else|\\
|\addtocounter{page}{-1}|\\
\textit{code for title page}\\
|\newpage|\\
|\||fi|
\end{tabular}
\end{center}
%
A banner page for the child documents can be generated by:
%
\begin{center}
\begin{tabular}{l}
|\ifchilddoc|\\
|\addtocounter{page}{-1}|\\
\textit{code for banner page}\\
|\newpage|\\
|\||fi|
\end{tabular}
\end{center}
%
Here one could write a message such as:
\begin{center}
|This is the part \childdocname{} of \childdocjob{}.|
\end{center}

%%%%%%%%%%%%%%%%%%%%%%%%%%%%%%%%%%%%%%%%%%%%%%%%%%%%%%%%%%%%%%%%%%%%%%%%%%%%%%%%
\subsection{Flags}
\label{sec:flags}

The package makes it easy to generate different versions
of the main or child documents.
To this end compilation flags can be defined
and assigned different default values.
They will be particularly useful in conjunction
with the forwarding mechanism described in \secref{sec:forward}.

For example, it may be useful to have a flag |\version|
which can be set to |draft| or |final|.
The document source will contain some conditional code
depending on the value of |\version|.
Suppose further, the flag should default to |final| for the main file
and to |draft| for child files
which is a natural assignment for editing the document.
This is achieved by placing the following code
in the preamble of the main document
(below the |\childdocmain| directive):
%
\begin{center}
\begin{tabular}{l}
|\ifchilddoc|\\
|\providecommand{\version}{draft}|\\
|\||else|\\
|\providecommand{\version}{final}|\\
|\||fi|
\end{tabular}
\end{center}
%
The definition by |\providecommand| makes sure
that previous definitions are not overwritten.
Further statements |\providecommand{\version}{...}|
can thus be added before the above code to override it.

For the main file, one might add a line
(between |\childdocmain| and the above block)
%
\begin{center}
|%\ifchilddoc\||else\providecommand{\version}{draft}\||fi|
\end{center}
%
which can be uncommented to produce a draft version.
Likewise one can add a line to the very top of a child file
(above the |\childdocof{|\textit{main}|}| directive)
%
\begin{center}
|%\providecommand{\version}{final}|
\end{center}
%
which can be uncommented to produce the final version of this child document.

%%%%%%%%%%%%%%%%%%%%%%%%%%%%%%%%%%%%%%%%%%%%%%%%%%%%%%%%%%%%%%%%%%%%%%%%%%%%%%%%
\subsection{Forwarding}
\label{sec:forward}

Different versions of the main or child documents
using compilation flags as described in \secref{sec:flags}
can be (permanently) stored in different files
for convenient compilation, viewing and distribution.
To this end, the package defines a command
to pass on compilation to a different file:

%%%%%%%%%%%%%%%%%%%%%%%%%%%%%%%%%%%%%%%%
\DescribeMacro{\childdocforward}
The command |\childdocforward| redirects processing to
another source file:
%
\begin{center}
\begin{tabular}{l}
|\input{childdoc.def}|\\
|\childdocforward[|\textit{main}|]{|\textit{dest}|}|\\
\end{tabular}
\end{center}
%
The argument \textit{dest} is the destination file
(without extension).
It should be the main file or one of the child files.
Note that further \textsf{childdoc} directives
such as |\childdocof| and |\childdocforward|
in the indicated file will be processed in this form.
The optional argument \textit{main}
passes on directly to the main file \textit{main}
while pretending to compile the child \textit{dest}.
This form behaves as if \textit{dest}
issues |\childdocof{|\textit{main}|}| right away,
and no further \textsf{childdoc} directives will be processed.

%%%%%%%%%%%%%%%%%%%%%%%%%%%%%%%%%%%%%%%%
\DescribeMacro{\...prefix}
In the alternative form |\childdocforwardprefix|,
%
\begin{center}
\begin{tabular}{l}
|\input{childdoc.def}|\\
|\childdocforwardprefix[|\textit{main}|]{|\textit{prefix}|}{|\textit{dest}|}|
\end{tabular}
\end{center}
%
the destination file is determined by a pattern
depending on the current file:
To make this work, the current file must be called
`{\textit{prefix}\hspace{0.2em}\textit{suffix}}'
with \textit{prefix} matching precisely the argument.
Processing is then passed on to the file
`{\textit{dest}\hspace{0.2em}\textit{suffix}}'.
Surely, the same effect is achieved by
directly specifying the
argument `{\textit{dest}\hspace{0.2em}\textit{suffix}}'
in the first form.
However, that requires to set up a different file
for each child. With the alternative form of the command
all these files can have exactly the same content
which simplifies setting them up and maintaining them.

For example, the following file |draft.tex|
with a compilation flag |\version| as described in \secref{sec:flags}
compiles the main document as a draft:
%
\begin{center}
\begin{tabular}{l}
|\def\version{draft}|\\
|\input{childdoc.def}|\\
|\childdocforward{|\textit{main}|}|
\end{tabular}
\end{center}
%
Likewise, the following files |final|\textit{nn}|.tex|
compile the final version of the child document
|child|\textit{nn}|.tex|:
%
\begin{center}
\begin{tabular}{l}
|\def\version{final}|\\
|\input{childdoc.def}|\\
|\childdocforwardprefix{final}{child}|
\end{tabular}
\end{center}
%

Note that when several versions of a main file and/or of each child file
are to be generated, it may be convenient to set up a |Makefile| or
shell script to automatise the process.

%%%%%%%%%%%%%%%%%%%%%%%%%%%%%%%%%%%%%%%%%%%%%%%%%%%%%%%%%%%%%%%%%%%%%%%%%%%%%%%%
\subsection{Command Line Processing}
\label{sec:commandline}

The effect of redirection files can also be achieved by invoking
the \LaTeX{} compiler with a more elaborate command line.
Most conveniently this should be done as part
of a shell script or a |Makefile|.

When using \textsf{childdoc} in the main file, the following
command lines effectively perform a redirection
(note that depending on the shell being used,
backslashes may have to be doubled: `|\|' $\to$ `|\\|'):
%
\begin{center}
|... -jobname "|\textit{target}|" |\\|"|[\textit{flags}]%
|\input{childdoc.def}\childdocforward[|\textit{main}|]{|\textit{dest}|}"|
\end{center}
%
Here \textit{target} is the name of the output file,
\textit{main} is the name of the main file
and \textit{dest} is the name of the main or child file to be processed
(all filenames without extensions).
The optional argument \textit{main} can be omitted
if \textit{main} matches \textit{dest}.
Optionally, compilation \textit{flags} can be defined via |\def| commands.
This command line makes the \TeX{} engine believe
it is compiling the file \textit{target}
whose content is specified as the latter parameter.
The provided code then forwards the processing to
\textit{main} or \textit{dest} as described in \secref{sec:forward}.

%%%%%%%%%%%%%%%%%%%%%%%%%%%%%%%%%%%%%%%%%%%%%%%%%%%%%%%%%%%%%%%%%%%%%%%%%%%%%%%%
\subsection{Include by Input}
\label{sec:input}

Including child documents by |\include| has some restrictions by design.
Most notably, the content of a child document always occupies
its own set of pages; pages cannot be shared between child documents.
Usually, this behaviour makes perfect sense
because each child document contain an essential part of the document.
However, in some situations it may be desirable to compose
a document from a collection of parts
without having mandatory page breaks between then.
For this case, the package
provides a mechanism to include parts
by |\input| which can also be processed individually.
However, by construction this mechanism
requires manual handling of the content to be output.

%%%%%%%%%%%%%%%%%%%%%%%%%%%%%%%%%%%%%%%%
\DescribeMacro{\ifchilddocmanual}
The main file should be prepared as usual, see \secref{sec:include}.
However, the document body must make a distinction
between processing of an individual part and of the main document, e.g.:
%
\begin{center}
\begin{tabular}{l}
|\ifchilddocmanual|\\
|\input{\childdocname}|\\
|\||else|\\
\textit{document body with }|\input{|\textit{part}|}|\\
|\||fi|
\end{tabular}
\end{center}
%
The conditional |\ifchilddocmanual| is true whenever
a part to be included by |\input| is being compiled,
and the name of the part is stored in |\childdocname|.

%%%%%%%%%%%%%%%%%%%%%%%%%%%%%%%%%%%%%%%%
\DescribeMacro{\childdocby}
Each part to be included by |\input| should start with:
%
\begin{center}
\begin{tabular}{l}
|\input{childdoc.def}|\\
|\childdocby{|\textit{main}|}|\\
\end{tabular}
\end{center}
%
The directive |\childdocby| is similar to |\childdocof|
described in \secref{sec:include},
but the subsequent selection of content must be done manually.
To that end, both |\ifchilddoc| and |\ifchilddocmanual|
will be true upon processing of a part,
and the name of the part is stored in |\childdocname|.
Note that |\jobname| will be set to the filename of the current part
so that each part receives an individual |.aux| file
that does not interfere with the |.aux| file(s) of the main document.
This behaviour can be altered by the alternative form
|\childdocby[*]{|\textit{main}|}| (with a non-empty optional argument)
which uses the |.aux| file of the main document
by setting |\jobname| to \textit{main}.

%%%%%%%%%%%%%%%%%%%%%%%%%%%%%%%%%%%%%%%%%%%%%%%%%%%%%%%%%%%%%%%%%%%%%%%%%%%%%%%%
\subsection{Driver Development}
\label{sec:driver}

The \textsf{childdoc} mechanism can also be use for the development
of definition files such as \LaTeX{} styles or classes.
This case differs from the above setup with multiple parts
included by |\include| in that no |\includeonly| should be invoked.
This can be achieved by starting the include file
(before |\ProvidesPackage|) with:
%
\begin{center}
\begin{tabular}{l}
|\input{childdoc.def}|\\
|\childdocforward{|\textit{main}|}|\\
\end{tabular}
\end{center}
%
or alternatively with:
%
\begin{center}
\begin{tabular}{l}
|\input{childdoc.def}|\\
|\childdocby{|\textit{main}|}|\\
\end{tabular}
\end{center}
%
Both forms have slightly different effects as described above.
The main file is prepared as usual, see \secref{sec:include}.

%%%%%%%%%%%%%%%%%%%%%%%%%%%%%%%%%%%%%%%%%%%%%%%%%%%%%%%%%%%%%%%%%%%%%%%%%%%%%%%%
\subsection{Legacy Detection}
\label{sec:detection}

The directive |\childdocmain| in the main file can detect
whether the complete document or merely a child is to be compiled
even without using the directive |\childdocof|.
This method is deprecated because it is less robust
and there is no compelling reason to use it;
it is merely provided for backward compatibility
and it may be removed in future versions.

If the detection mechanism is to be used,
it is mandatory to correctly specify
the filename of the main file as the argument of |\childdocmain|:
%
\begin{center}
\begin{tabular}{l}
|\input{childdoc.def}|\\
|\childdocmain{|\textit{main}|}|\\
\end{tabular}
\end{center}
%
If |\jobname| does not match the argument \textit{main} of |\childdocmain|,
it is assumed that |\jobname| points to the child file to be compiled.
When using |\childdocmain| with the main file specified as argument,
it suffices to start a child file
with just |\input{|\textit{main}|}|
without loading of the package and using |\childdocof|.
If instead all processing is done
with the appropriate \textsf{childdoc} directives,
the argument of \textit{main} of |\childdocmain| can be empty.

An alternative version of the command line processing described
in \secref{sec:commandline} using the detection mechanism reads:
%
\begin{center}
|... -jobname "|\textit{target}|" "|[\textit{flags}]%
[|\def\jobname{|\textit{dest}|}|]|\input{|\textit{main}|}"|
\end{center}

%%%%%%%%%%%%%%%%%%%%%%%%%%%%%%%%%%%%%%%%%%%%%%%%%%%%%%%%%%%%%%%%%%%%%%%%%%%%%%%%
\subsection{Manual Code}
\label{sec:manual}

In case one cannot be certain whether the definitions file |childdoc.def|
is installed on the target \TeX{} distribution
and one prefers not to ship it,
it is conceivable to paste a few relevant commands into the sources.

To that end, drop all statements |\input{childdoc.def}|
and perform the replacements as outlined below.
Instead of |\childdocmain{|\textit{main}|}| add the following code
to the top of the main file:
%
\begin{center}
\begin{tabular}{l}
|\||ifdefined\childdocname\endinput\||fi\newif\ifchilddoc|\\
|\edef\childdocname{\scantokens\expandafter{\jobname\noexpand}}|\\
|\def\childdocmain{|\textit{main}|}\||ifx\childdocmain\childdocname\||else|\\
|\childdoctrue\includeonly{\childdocname}\let\jobname\childdocmain\||fi|\\
\end{tabular}
\end{center}
%
Instead of |\childdocof{|\textit{main}|}| just include the main file
at the top of each child file:
%
\begin{center}
|\input{|\textit{main}|}|
\end{center}
%
A simple redirection |\childdocforward{|\textit{dest}|}| is achieved by:
%
\begin{center}
|\def\jobname{|\textit{dest}|}\input{\jobname}|
\end{center}
%
The redirection with prefix
|\childdocforwardprefix[|\textit{prefix}|]{|\textit{dest}|}|
is accomplished by:
%
\begin{center}
\begin{tabular}{l}
|{\edef\jobname{\scantokens\expandafter{\jobname\noexpand}}|\\
|\def\redirectjob |\textit{prefix}|#1~~~{\gdef\jobname{|\textit{dest}|#1}}|\\
|\expandafter\redirectjob\jobname~~~}\input{\jobname}|
\end{tabular}
\end{center}

In an alternative approach,
child documents can be compiled by a specific command line
without additional code or specific definitions:
%
\begin{center}
|... -jobname "|\textit{target}|" "|[\textit{flags}]%
|\includeonly{|\textit{dest}|}\input{|\textit{main}|}"|
\end{center}
%

%%%%%%%%%%%%%%%%%%%%%%%%%%%%%%%%%%%%%%%%%%%%%%%%%%%%%%%%%%%%%%%%%%%%%%%%%%%%%%%%
%%%%%%%%%%%%%%%%%%%%%%%%%%%%%%%%%%%%%%%%%%%%%%%%%%%%%%%%%%%%%%%%%%%%%%%%%%%%%%%%
\section{Information}

%%%%%%%%%%%%%%%%%%%%%%%%%%%%%%%%%%%%%%%%%%%%%%%%%%%%%%%%%%%%%%%%%%%%%%%%%%%%%%%%
\subsection{Copyright}

Copyright \copyright{} 2017--2018 Niklas Beisert

This work may be distributed and/or modified under the
conditions of the \LaTeX{} Project Public License, either version 1.3
of this license or (at your option) any later version.
The latest version of this license is in
  \url{http://www.latex-project.org/lppl.txt}
and version 1.3 or later is part of all distributions of \LaTeX{}
version 2005/12/01 or later.

This work has the LPPL maintenance status `maintained'.

The Current Maintainer of this work is Niklas Beisert.

This work consists of the files |README.txt|, |childdoc.ins| and |childdoc.dtx|
as well as the derived files |childdoc.def|, |cdocsamp.tex|
with |cdocsch1.tex|, |cdocsch2.tex|, |cdocspt3.tex|, |cdocspt4.tex|,
|cdocsdrf.tex|, |cdocsfn1.tex|, |cdocsfn2.tex|
as well as |childdoc.pdf|.

%%%%%%%%%%%%%%%%%%%%%%%%%%%%%%%%%%%%%%%%%%%%%%%%%%%%%%%%%%%%%%%%%%%%%%%%%%%%%%%%
\subsection{Files and Installation}

The package consists of the files:
%
\begin{center}
\begin{tabular}{ll}
    |README.txt|   & readme file \\
    |childdoc.ins| & installation file \\
    |childdoc.dtx| & source file \\
    |childdoc.def| & definition file \\
    |cdocsamp.tex| & sample main file \\
    |cdocsch1.tex| & sample include file \\
    |cdocsch2.tex| & sample include file \\
    |cdocspt3.tex| & sample part file \\
    |cdocspt4.tex| & sample part file \\
    |cdocsdrf.tex| & sample redirection file \\
    |cdocsfn1.tex| & sample redirection file \\
    |cdocsfn2.tex| & sample redirection file \\
    |childdoc.pdf| & manual
\end{tabular}
\end{center}
%
The distribution consists of the files
|README.txt|, |childdoc.ins| and |childdoc.dtx|.
%
\begin{itemize}
\item
Run (pdf)\LaTeX{} on |childdoc.dtx|
to compile the manual |childdoc.pdf| (this file).
\item
Run \LaTeX{} on |childdoc.ins| to create the definitions file |childdoc.def|
and the sample |cdocsamp.tex| with include files
|cdocsch1.tex|, |cdocsch2.tex|, |cdocspt3.tex|, |cdocspt4.tex|,
|cdocsdrf.tex|, |cdocsfn1.tex|, |cdocsfn2.tex|.
Then copy the file |childdoc.def| to an appropriate directory of your \LaTeX{}
distribution, e.g.\ \textit{texmf-root}|/tex/latex/childdoc|.
\end{itemize}

%%%%%%%%%%%%%%%%%%%%%%%%%%%%%%%%%%%%%%%%%%%%%%%%%%%%%%%%%%%%%%%%%%%%%%%%%%%%%%%%
\subsection{Related CTAN Packages}

There are several other packages which offer a similar functionality:
%
\begin{itemize}
\item
The packages
\href{http://ctan.org/pkg/docmute}{\textsf{docmute}},
\href{http://ctan.org/pkg/includex}{\textsf{includex}} and
\href{http://ctan.org/pkg/standalone}{\textsf{standalone}}
provide commands to include only the document body of
a child file thus allowing both files to be compiled individually.
\item
The packages \href{http://ctan.org/pkg/subdocs}{\textsf{subdocs}}
and \href{http://ctan.org/pkg/subfiles}{\textsf{subfiles}}
provide structures in which the main and child documents can be
encapsulated and allowing them to be compiled individually.
The inclusion mechanism is different from the conventional |\include|.
\item
The package \href{http://ctan.org/pkg/combine}{\textsf{combine}}
is an elaborate solution to combine several documents into one.
\end{itemize}
%
See also the CTAN topic \href{http://ctan.org/topic/subdocs}{\textsf{subdocs}}
for further related packages.
The present package differs from the above solutions in that
a document structure constructed with the conventional |\include| mechanism
just needs two extra commands at the top of every file
such that all constituent files can be compiled individually.

%%%%%%%%%%%%%%%%%%%%%%%%%%%%%%%%%%%%%%%%%%%%%%%%%%%%%%%%%%%%%%%%%%%%%%%%%%%%%%%%
%\subsection{Feature Suggestions}
%
%The following is a list of features which may be useful for future
%versions of this package:
%%
%\begin{itemize}
%\item
%\ldots
%\end{itemize}

%%%%%%%%%%%%%%%%%%%%%%%%%%%%%%%%%%%%%%%%%%%%%%%%%%%%%%%%%%%%%%%%%%%%%%%%%%%%%%%%
\subsection{Revision History}

%%%%%%%%%%%%%%%%%%%%%%%%%%%%%%%%%%%%%%%%
\paragraph{v2.0:} 2018/12/30

\begin{itemize}
\item
immediate forward processing
\item
added |\childdocby| mechanism
\item
manual restructured
\end{itemize}

%%%%%%%%%%%%%%%%%%%%%%%%%%%%%%%%%%%%%%%%
\paragraph{v1.6:} 2018/01/17

\begin{itemize}
\item
application for development of include files
\item
corrections to manual
\end{itemize}

%%%%%%%%%%%%%%%%%%%%%%%%%%%%%%%%%%%%%%%%
\paragraph{v1.5:} 2017/05/21

\begin{itemize}
\item
more complete structuring introduced
\item
|\childdocof| introduced
\item
|\childdoc| renamed to |\childdocmain|
\item
|\childredirect| renamed to |\childdocforward| and |\childdocforwardprefix|
and functionality expanded
\end{itemize}

%%%%%%%%%%%%%%%%%%%%%%%%%%%%%%%%%%%%%%%%
\paragraph{v1.0:} 2017/04/27

\begin{itemize}
\item
manual and install package
\item
first version published on CTAN
\end{itemize}

%%%%%%%%%%%%%%%%%%%%%%%%%%%%%%%%%%%%%%%%
\paragraph{v0.6:} 2017/04/26

\begin{itemize}
\item
redirection mechanism added
\end{itemize}

%%%%%%%%%%%%%%%%%%%%%%%%%%%%%%%%%%%%%%%%
\paragraph{v0.5:} 2017/04/26

\begin{itemize}
\item
functionality in definition file
\end{itemize}


%%%%%%%%%%%%%%%%%%%%%%%%%%%%%%%%%%%%%%%%%%%%%%%%%%%%%%%%%%%%%%%%%%%%%%%%%%%%%%%%
%%%%%%%%%%%%%%%%%%%%%%%%%%%%%%%%%%%%%%%%%%%%%%%%%%%%%%%%%%%%%%%%%%%%%%%%%%%%%%%%
%%%%%%%%%%%%%%%%%%%%%%%%%%%%%%%%%%%%%%%%%%%%%%%%%%%%%%%%%%%%%%%%%%%%%%%%%%%%%%%%
\appendix

\settowidth\MacroIndent{\rmfamily\scriptsize 000\ }

 \DocInput{childdoc.dtx}

\end{document}
%</driver>
% \fi
%
% %%%%%%%%%%%%%%%%%%%%%%%%%%%%%%%%%%%%%%%%%%%%%%%%%%%%%%%%%%%%%%%%%%%%%%%%%%%%%%
% %%%%%%%%%%%%%%%%%%%%%%%%%%%%%%%%%%%%%%%%%%%%%%%%%%%%%%%%%%%%%%%%%%%%%%%%%%%%%%
% \section{Sample}
%\iffalse
%<*samplemain>
%\fi
%
% The following presents a sample document
% with two chapters, two parts, a title page,
% a compile flag as well as three forwarding files to set the flag.
% It consists of eight |.tex| files:
% \begin{center}
% \begin{tabular}{ll}
% |cdocsamp.tex|&main file\\
% |cdocsch1.tex|&include file for chapter 1\\
% |cdocsch2.tex|&include file for chapter 2\\
% |cdocspt3.tex|&include file for part 3\\
% |cdocspt4.tex|&include file for part 4\\
% |cdocsdrf.tex|&forwarding file for main file in draft mode\\
% |cdocsfi1.tex|&forwarding file for final version of chapter 1\\
% |cdocsfi2.tex|&forwarding file for final version of chapter 2\\
% \end{tabular}
% \end{center}
% Each of the eight files can be compiled directly by the \LaTeX{} compiler.
%
% %%%%%%%%%%%%%%%%%%%%%%%%%%%%%%%%%%%%%%
% \paragraph{Main File.}
%
% The main file is called |cdocsamp.tex|.
%
% Load the \textsf{childdoc} definitions and
% declare the filename for the main document:
%    \begin{macrocode}
\input{childdoc.def}
\childdocmain{}
%    \end{macrocode}

% Optional override for |\version| flag:
%    \begin{macrocode}
%%\ifchilddoc\else\providecommand{\version}{draft}\fi
%    \end{macrocode}

% Define the default values for the |\version| flag
% (|final| for the main file and |draft| for childs):
%    \begin{macrocode}
\ifchilddoc
\providecommand{\version}{draft}
\else
\providecommand{\version}{final}
\fi
%    \end{macrocode}

% Load the standard document class:
%    \begin{macrocode}
\documentclass[12pt]{article}
%    \end{macrocode}

% Start the document body:
%    \begin{macrocode}
\begin{document}
%    \end{macrocode}

% Declare a title page.
% Print title, part of document being processed and version flag:
%    \begin{macrocode}
\addtocounter{page}{-1}
\begin{center}
{\LARGE\bfseries{}childdoc example\par}
\vspace{1cm}
\ifchilddoc
\ifchilddocmanual part\else chapter\fi:
`\childdocname' of `\childdocjob'\par
\else
main document: `\childdocjob'\par
\fi
version: \version\par
\end{center}
\newpage
%    \end{macrocode}

% Manually include selected file,
% otherwise process as usual:
%    \begin{macrocode}
\ifchilddocmanual
\section*{part `\childdocname'}
\input{\childdocname}
\else
%    \end{macrocode}

% Include the two chapters:
%    \begin{macrocode}
\include{cdocsch1}
\include{cdocsch2}
%    \end{macrocode}

% Include the two parts unless only chapters should be displayed:
%    \begin{macrocode}
\ifchilddoc\else
\section{part three}
\input{cdocspt3}
\section{part four}
\input{cdocspt4}
\fi
%    \end{macrocode}

% Process as usual until here:
%    \begin{macrocode}
\fi
%    \end{macrocode}

% End of document body:
%    \begin{macrocode}
\end{document}
%    \end{macrocode}
%\iffalse
%</samplemain>
%\fi
%
% %%%%%%%%%%%%%%%%%%%%%%%%%%%%%%%%%%%%%%
% \paragraph{Chapter Include Files.}
%
% The include files are called |cdocsch1.tex| and |cdocsch2.tex|.
%
%\iffalse
%<*samplechap1|samplechap2>
%\fi

% Optional override for |\version| flag:
%    \begin{macrocode}
%%\providecommand{\version}{final}
%    \end{macrocode}

% Include the main document:
%    \begin{macrocode}
\input{childdoc.def}
\childdocof{cdocsamp}
%    \end{macrocode}

%\iffalse
%</samplechap1|samplechap2>
%\fi
%
%\iffalse
%<*samplechap1>
%\fi
% Some text for chapter 1:
%    \begin{macrocode}
\section{one}
some text in chapter one
%    \end{macrocode}

%\iffalse
%</samplechap1>
%\fi
% Some text for chapter 2:
%\iffalse
%<*samplechap2>
%\fi
%    \begin{macrocode}
\section{two}
more text in chapter two
%    \end{macrocode}

%\iffalse
%</samplechap2>
%\fi
%
% %%%%%%%%%%%%%%%%%%%%%%%%%%%%%%%%%%%%%%
% \paragraph{Part Include Files.}
%
% The include files are called |cdocspt3.tex| and |cdocspt4.tex|.
%
%\iffalse
%<*samplepart3|samplepart4>
%\fi

% Optional override for |\version| flag:
%    \begin{macrocode}
%%\providecommand{\version}{final}
%    \end{macrocode}

% Include the main document:
%    \begin{macrocode}
\input{childdoc.def}
\childdocby{cdocsamp}
%    \end{macrocode}

%\iffalse
%</samplepart3|samplepart4>
%\fi
%
%\iffalse
%<*samplepart3>
%\fi
% Some text for part 3:
%    \begin{macrocode}
some text in part three
%    \end{macrocode}

%\iffalse
%</samplepart3>
%\fi
% Some text for part 4:
%\iffalse
%<*samplepart4>
%\fi
%    \begin{macrocode}
more text in part four
%    \end{macrocode}

%\iffalse
%</samplepart4>
%\fi
%
% %%%%%%%%%%%%%%%%%%%%%%%%%%%%%%%%%%%%%%
% \paragraph{Forwarding for a Complete Draft.}
%
% The following forwarding file |cdocsdrf.tex|
% compiles the main document in draft mode:
%\iffalse
%<*sampledraft>
%\fi
%    \begin{macrocode}
\def\version{draft}
\input{childdoc.def}
\childdocforward{cdocsamp}
%    \end{macrocode}

%\iffalse
%</sampledraft>
%\fi
%
% %%%%%%%%%%%%%%%%%%%%%%%%%%%%%%%%%%%%%%
% \paragraph{Forwarding for Final Version of the Chapters.}
%
% The following forwarding files |cdocsfn1.tex| and |cdocsfn2.tex|
% (with identical content)
% compile the final versions of the child documents
% |cdocsch1.tex| and |cdocsch2.tex|, respectively:
%\iffalse
%<*samplefinal>
%\fi
%    \begin{macrocode}
\def\version{final}
\input{childdoc.def}
\childdocforwardprefix[cdocsamp]{cdocsfn}{cdocsch}
%    \end{macrocode}

%\iffalse
%</samplefinal>
%\fi
%
% %%%%%%%%%%%%%%%%%%%%%%%%%%%%%%%%%%%%%%
% \paragraph{Command Line Processing.}
%
% The following three command lines generate the output files
% |cdocscld|, |cdocscl1| and |cdocscl2|
% which should be identical to
% |cdocsdrf|, |cdocsch1| and |cdocsfn2|, respectively:
% \begin{center}
% \begin{tabular}{l}
% |latex -jobname cdocscld \|\\
% |  "\def\version{draft}\input{childdoc.def}\childdocforward{cdocsamp}"|\\
% |latex -jobname cdocscl1 \|\\
% |  "\input{childdoc.def}\childdocforward[cdocsamp]{cdocsch1}"|\\
% |latex -jobname cdocscl2 \|\\
% |  "\def\version{final}\input{childdoc.def}\childdocforward{cdocsch2}"|
% \end{tabular}
% \end{center}
% Note that the trailing backslash on each first line
% merely continues the input to the second line
% (for convenient cut ant paste).
% Furthermore, the command |latex| can be replaced by any
% of its alternative versions such as |pdflatex|.
%
% %%%%%%%%%%%%%%%%%%%%%%%%%%%%%%%%%%%%%%%%%%%%%%%%%%%%%%%%%%%%%%%%%%%%%%%%%%%%%%
% %%%%%%%%%%%%%%%%%%%%%%%%%%%%%%%%%%%%%%%%%%%%%%%%%%%%%%%%%%%%%%%%%%%%%%%%%%%%%%
% \section{Implementation}
%\iffalse
%<*package>
%\fi
%
% This section describes the definitions file |childdoc.def|.

% The definitions cannot be loaded using |\usepackage| or |\RequirePackage|
% which has a mechanism to prevent loading a style file more than once.
% When loading the definitions by means of |\input|
% multiple instances have to be prevented manually:
%\iffalse
%This code needs to be before the `\ProvidesFile' directive
%which is defined at the beginning of this file.
%Therefore it is also placed there and commented out here.
%</package>
%<*discard>
%\fi
%    \begin{macrocode}
\ifdefined\childdocmain\endinput\fi
%    \end{macrocode}
%\iffalse
%</discard>
%<*package>
%\fi
%
% \macro{\ifchilddoc}
% \macro{\ifchilddocmanual}
% The conditional |\ifchilddoc| tells whether a
% child (true) or main (false) document is being compiled.
% The conditional |\ifchilddocmanual| tells whether
% the |\includeonly| mechanism is used (false) or
% the selection of child files must be performed manually (true).
% The definitions initialise to false:
%    \begin{macrocode}
\newif\ifchilddoc
\newif\ifchilddocmanual
%    \end{macrocode}

% \macro{\childdocname}
% \macro{\childdocjob}
% The macro |\childdocname| stores the name of the main document
% to be compiled. The macro |\childdocjob| stores the name of
% the document on which the \LaTeX{} compiler was originally invoked.
% The content of |\jobname| cannot be compared
% to filenames specified in the source due to different catcodes.
% The following code rescans |\jobname|, stores the result
% in |\childdocname| and saves a copy in |\childdocjob|:
%    \begin{macrocode}
\edef\childdocname{\scantokens\expandafter{\jobname\noexpand}}
\let\childdocjob\childdocname
%    \end{macrocode}

% \macro{\childdocdisable}
% The macro |\childdocdisable| prevents the main file
% from being processed more than once.
% At this stage, the main document command |\childdocmain|
% is assumed to be called once again where it should do nothing.
% Any subsequent call to it should prevent
% a secondary processing of the main document
% It overwrites the forwarding commands
% |\childdocof| and |\childdocforward|
% with empty macros to prevent further inclusions of the main document:
%    \begin{macrocode}
\newcommand{\childdocdisable}
{
  \renewcommand{\childdocmain}[1]{\renewcommand{\childdocmain}[1]{\endinput}}
  \renewcommand{\childdocof}[1]{}
  \renewcommand{\childdocby}[2][]{}
  \renewcommand{\childdocforward}[2][]{}
  \renewcommand{\childdocdisable}{}
}
%    \end{macrocode}

% \macro{\childdocmain}
% The macro |\childdocmain| is to be called at the top of the main file
% with nothing or the main filename (without extension) as argument.
% First, it breaks loops.
% If the argument is not empty and does not match |\childdocname|
% (which is set by the first inclusion of |childdoc.def|),
% |\ifchilddoc| is set to true, |\includeonly| is applied to the child file
% and |\jobname| is set to the main file
% (for proper handling of |.aux| files):
%    \begin{macrocode}
\newcommand{\childdocmain}[1]
{
  \childdocdisable\childdocmain{}
  \if?#1?\else
    \begingroup
      \def\childdoctmp{#1}
      \ifx\childdoctmp\childdocname
        \def\childdoctmp{}
      \else
        \def\childdoctmp
        {
          \childdoctrue
          \includeonly{\childdocname}
          \def\childdocjob{#1}
          \def\jobname{#1}
        }
      \fi
      \expandafter
    \endgroup
    \childdoctmp
  \fi
}
%    \end{macrocode}

% \macro{\childdocof}
% The command |\childdocof| redirects
% compilation to the main file |#1|.
%    \begin{macrocode}
\newcommand{\childdocof}[1]
{
  \childdocdisable
  \childdoctrue
  \includeonly{\childdocname}
  \def\jobname{#1}
  \def\childdocjob{#1}
  \input{#1}
}
%    \end{macrocode}

% \macro{\childdocby}
% The command |\childdocby| ....
%    \begin{macrocode}
\newcommand{\childdocby}[2][]
{
  \childdocdisable
  \childdoctrue
  \childdocmanualtrue
  \if?#1?\else
    \def\jobname{#2}
  \fi
  \def\childdocjob{#2}
  \input{#2}
  \endinput
}
%    \end{macrocode}

% \macro{\childdocforward}
% The command |\childdocforward| redirects
% compilation to the main file or
% (if the optional argument is given) a child file.
% Parameters are set as if the main file
% or a child file starting with |\childdocof| was compiled.
% Then compilation is handed over to the main file:
%    \begin{macrocode}
\newcommand{\childdocforward}[2][]
{
  \begingroup
    \if?#1?
      \def\childdoctmp
      {
        \def\childdocname{#2}
        \def\childdocjob{#2}
        \def\jobname{#2}
        \input{#2}
        \endinput
      }
    \else
      \def\childdoctmp
      {
        \childdocdisable
        \def\childdocname{#2}
        \childdoctrue
        \includeonly{#2}
        \def\childdocjob{#1}
        \def\jobname{#1}
        \input{#1}
        \endinput
      }
    \fi
    \expandafter
  \endgroup
  \childdoctmp
}
%    \end{macrocode}

% \macro{\childdocforwardprefix}
% The command |\childdocforwardprefix| redirects
% compilation to the main or a child file by means of a pattern.
% The prefix |#1| in the current filename is replaced by |#2|
% and the suffix of the current filename is kept
% (it is assumed that the filename does not contain the substring `|~~~|'
% which is used as a delimiter).
% Compilation is handed over to the new file by |\childdocforward|:
%    \begin{macrocode}
\newcommand{\childdocforwardprefix}[3][]
{
  \begingroup
    \def\childdocextract #2##1~~~{\def\childdoctmp{\childdocforward[#1]{#3##1}}}
    \expandafter\childdocextract\childdocname~~~
    \expandafter
  \endgroup
  \childdoctmp
}
%    \end{macrocode}

% \macro{\childdoc}
% The deprecated macro |\childdoc| is a legacy version of |\childdocmain|:
%    \begin{macrocode}
\newcommand{\childdoc}{\childdocmain}
%    \end{macrocode}

% \macro{\childdocredirect}
% The deprecated macro |\childdocredirect| is a legacy version
% of |\childdocforward| and |\childdocforwardprefix|:
%    \begin{macrocode}
\newcommand{\childdocredirect}[2][]
{
  \begingroup
    \if?#1?
      \def\childdoctmp{\childdocforward{#2}}
    \else
      \def\childdoctmp{\childdocforwardprefix{#1}{#2}}
    \fi
    \expandafter
  \endgroup
  \childdoctmp
}
%    \end{macrocode}

%\iffalse
%</package>
%\fi
%
\endinput
\childdocforward{cdocsch2}"|
% \end{tabular}
% \end{center}
% Note that the trailing backslash on each first line
% merely continues the input to the second line
% (for convenient cut ant paste).
% Furthermore, the command |latex| can be replaced by any
% of its alternative versions such as |pdflatex|.
%
% %%%%%%%%%%%%%%%%%%%%%%%%%%%%%%%%%%%%%%%%%%%%%%%%%%%%%%%%%%%%%%%%%%%%%%%%%%%%%%
% %%%%%%%%%%%%%%%%%%%%%%%%%%%%%%%%%%%%%%%%%%%%%%%%%%%%%%%%%%%%%%%%%%%%%%%%%%%%%%
% \section{Implementation}
%\iffalse
%<*package>
%\fi
%
% This section describes the definitions file |childdoc.def|.

% The definitions cannot be loaded using |\usepackage| or |\RequirePackage|
% which has a mechanism to prevent loading a style file more than once.
% When loading the definitions by means of |\input|
% multiple instances have to be prevented manually:
%\iffalse
%This code needs to be before the `\ProvidesFile' directive
%which is defined at the beginning of this file.
%Therefore it is also placed there and commented out here.
%</package>
%<*discard>
%\fi
%    \begin{macrocode}
\ifdefined\childdocmain\endinput\fi
%    \end{macrocode}
%\iffalse
%</discard>
%<*package>
%\fi
%
% \macro{\ifchilddoc}
% \macro{\ifchilddocmanual}
% The conditional |\ifchilddoc| tells whether a
% child (true) or main (false) document is being compiled.
% The conditional |\ifchilddocmanual| tells whether
% the |\includeonly| mechanism is used (false) or
% the selection of child files must be performed manually (true).
% The definitions initialise to false:
%    \begin{macrocode}
\newif\ifchilddoc
\newif\ifchilddocmanual
%    \end{macrocode}

% \macro{\childdocname}
% \macro{\childdocjob}
% The macro |\childdocname| stores the name of the main document
% to be compiled. The macro |\childdocjob| stores the name of
% the document on which the \LaTeX{} compiler was originally invoked.
% The content of |\jobname| cannot be compared
% to filenames specified in the source due to different catcodes.
% The following code rescans |\jobname|, stores the result
% in |\childdocname| and saves a copy in |\childdocjob|:
%    \begin{macrocode}
\edef\childdocname{\scantokens\expandafter{\jobname\noexpand}}
\let\childdocjob\childdocname
%    \end{macrocode}

% \macro{\childdocdisable}
% The macro |\childdocdisable| prevents the main file
% from being processed more than once.
% At this stage, the main document command |\childdocmain|
% is assumed to be called once again where it should do nothing.
% Any subsequent call to it should prevent
% a secondary processing of the main document
% It overwrites the forwarding commands
% |\childdocof| and |\childdocforward|
% with empty macros to prevent further inclusions of the main document:
%    \begin{macrocode}
\newcommand{\childdocdisable}
{
  \renewcommand{\childdocmain}[1]{\renewcommand{\childdocmain}[1]{\endinput}}
  \renewcommand{\childdocof}[1]{}
  \renewcommand{\childdocby}[2][]{}
  \renewcommand{\childdocforward}[2][]{}
  \renewcommand{\childdocdisable}{}
}
%    \end{macrocode}

% \macro{\childdocmain}
% The macro |\childdocmain| is to be called at the top of the main file
% with nothing or the main filename (without extension) as argument.
% First, it breaks loops.
% If the argument is not empty and does not match |\childdocname|
% (which is set by the first inclusion of |childdoc.def|),
% |\ifchilddoc| is set to true, |\includeonly| is applied to the child file
% and |\jobname| is set to the main file
% (for proper handling of |.aux| files):
%    \begin{macrocode}
\newcommand{\childdocmain}[1]
{
  \childdocdisable\childdocmain{}
  \if?#1?\else
    \begingroup
      \def\childdoctmp{#1}
      \ifx\childdoctmp\childdocname
        \def\childdoctmp{}
      \else
        \def\childdoctmp
        {
          \childdoctrue
          \includeonly{\childdocname}
          \def\childdocjob{#1}
          \def\jobname{#1}
        }
      \fi
      \expandafter
    \endgroup
    \childdoctmp
  \fi
}
%    \end{macrocode}

% \macro{\childdocof}
% The command |\childdocof| redirects
% compilation to the main file |#1|.
%    \begin{macrocode}
\newcommand{\childdocof}[1]
{
  \childdocdisable
  \childdoctrue
  \includeonly{\childdocname}
  \def\jobname{#1}
  \def\childdocjob{#1}
  \input{#1}
}
%    \end{macrocode}

% \macro{\childdocby}
% The command |\childdocby| ....
%    \begin{macrocode}
\newcommand{\childdocby}[2][]
{
  \childdocdisable
  \childdoctrue
  \childdocmanualtrue
  \if?#1?\else
    \def\jobname{#2}
  \fi
  \def\childdocjob{#2}
  \input{#2}
  \endinput
}
%    \end{macrocode}

% \macro{\childdocforward}
% The command |\childdocforward| redirects
% compilation to the main file or
% (if the optional argument is given) a child file.
% Parameters are set as if the main file
% or a child file starting with |\childdocof| was compiled.
% Then compilation is handed over to the main file:
%    \begin{macrocode}
\newcommand{\childdocforward}[2][]
{
  \begingroup
    \if?#1?
      \def\childdoctmp
      {
        \def\childdocname{#2}
        \def\childdocjob{#2}
        \def\jobname{#2}
        \input{#2}
        \endinput
      }
    \else
      \def\childdoctmp
      {
        \childdocdisable
        \def\childdocname{#2}
        \childdoctrue
        \includeonly{#2}
        \def\childdocjob{#1}
        \def\jobname{#1}
        \input{#1}
        \endinput
      }
    \fi
    \expandafter
  \endgroup
  \childdoctmp
}
%    \end{macrocode}

% \macro{\childdocforwardprefix}
% The command |\childdocforwardprefix| redirects
% compilation to the main or a child file by means of a pattern.
% The prefix |#1| in the current filename is replaced by |#2|
% and the suffix of the current filename is kept
% (it is assumed that the filename does not contain the substring `|~~~|'
% which is used as a delimiter).
% Compilation is handed over to the new file by |\childdocforward|:
%    \begin{macrocode}
\newcommand{\childdocforwardprefix}[3][]
{
  \begingroup
    \def\childdocextract #2##1~~~{\def\childdoctmp{\childdocforward[#1]{#3##1}}}
    \expandafter\childdocextract\childdocname~~~
    \expandafter
  \endgroup
  \childdoctmp
}
%    \end{macrocode}

% \macro{\childdoc}
% The deprecated macro |\childdoc| is a legacy version of |\childdocmain|:
%    \begin{macrocode}
\newcommand{\childdoc}{\childdocmain}
%    \end{macrocode}

% \macro{\childdocredirect}
% The deprecated macro |\childdocredirect| is a legacy version
% of |\childdocforward| and |\childdocforwardprefix|:
%    \begin{macrocode}
\newcommand{\childdocredirect}[2][]
{
  \begingroup
    \if?#1?
      \def\childdoctmp{\childdocforward{#2}}
    \else
      \def\childdoctmp{\childdocforwardprefix{#1}{#2}}
    \fi
    \expandafter
  \endgroup
  \childdoctmp
}
%    \end{macrocode}

%\iffalse
%</package>
%\fi
%
\endinput
|\\
|\childdocforwardprefix[|\textit{main}|]{|\textit{prefix}|}{|\textit{dest}|}|
\end{tabular}
\end{center}
%
the destination file is determined by a pattern
depending on the current file:
To make this work, the current file must be called
`{\textit{prefix}\hspace{0.2em}\textit{suffix}}'
with \textit{prefix} matching precisely the argument.
Processing is then passed on to the file
`{\textit{dest}\hspace{0.2em}\textit{suffix}}'.
Surely, the same effect is achieved by
directly specifying the
argument `{\textit{dest}\hspace{0.2em}\textit{suffix}}'
in the first form.
However, that requires to set up a different file
for each child. With the alternative form of the command
all these files can have exactly the same content
which simplifies setting them up and maintaining them.

For example, the following file |draft.tex|
with a compilation flag |\version| as described in \secref{sec:flags}
compiles the main document as a draft:
%
\begin{center}
\begin{tabular}{l}
|\def\version{draft}|\\
|% \iffalse
%
% childdoc.dtx Copyright (C) 2017-2018 Niklas Beisert
%
% This work may be distributed and/or modified under the
% conditions of the LaTeX Project Public License, either version 1.3
% of this license or (at your option) any later version.
% The latest version of this license is in
%   http://www.latex-project.org/lppl.txt
% and version 1.3 or later is part of all distributions of LaTeX
% version 2005/12/01 or later.
%
% This work has the LPPL maintenance status `maintained'.
%
% The Current Maintainer of this work is Niklas Beisert.
%
% This work consists of the files childdoc.dtx and childdoc.ins
% and the derived files childdoc.def and cdocsamp.tex with
% cdocsch1.tex, cdocsch2.tex, cdocsdrf.tex, cdocsfn1.tex, cdocsfn2.tex.
%
%<package>\ifdefined\childdocmain\endinput\fi
%<package>\ProvidesFile{childdoc.def}[2018/12/30 v2.0 child document driver]
%<samplemain>\ProvidesFile{cdocsamp.tex}[2018/12/30 v2.0 sample for childdoc]
%<*driver>
%\ProvidesFile{childdoc.drv}[2018/12/30 v2.0 childdoc reference manual file]
\PassOptionsToClass{10pt,a4paper}{article}
\documentclass{ltxdoc}

\usepackage[margin=35mm]{geometry}
\usepackage{hyperref}
\usepackage{hyperxmp}
\usepackage[usenames]{color}

\hypersetup{colorlinks=true}
\hypersetup{pdfstartview=FitH}
\hypersetup{pdfpagemode=UseNone}
\hypersetup{pdfsource={}}
\hypersetup{pdflang={en-UK}}
\hypersetup{pdfcopyright={Copyright 2017-2018 Niklas Beisert.
  This work may be distributed and/or modified under the
  conditions of the LaTeX Project Public License, either version 1.3
  of this license or (at your option) any later version.}}
\hypersetup{pdflicenseurl={http://www.latex-project.org/lppl.txt}}
\hypersetup{pdfcontactaddress={ETH Zurich, ITP, HIT K,
  Wolfgang-Pauli-Strasse 27}}
\hypersetup{pdfcontactpostcode={8093}}
\hypersetup{pdfcontactcity={Zurich}}
\hypersetup{pdfcontactcountry={Switzerland}}
\hypersetup{pdfcontactemail={nbeisert@itp.phys.ethz.ch}}
\hypersetup{pdfcontacturl={http://people.phys.ethz.ch/\xmptilde nbeisert/}}

\newcommand{\secref}[1]{\hyperref[#1]{section \ref*{#1}}}

\parskip1ex
\parindent0pt
\let\olditemize\itemize
\def\itemize{\olditemize\parskip0pt}

\begin{document}

\title{The \textsf{childdoc} Package}
\hypersetup{pdftitle={The childdoc Package}}
\author{Niklas Beisert\\[2ex]
  Institut f\"ur Theoretische Physik\\
  Eidgen\"ossische Technische Hochschule Z\"urich\\
  Wolfgang-Pauli-Strasse 27, 8093 Z\"urich, Switzerland\\[1ex]
  \href{mailto:nbeisert@itp.phys.ethz.ch}
  {\texttt{nbeisert@itp.phys.ethz.ch}}}
\hypersetup{pdfauthor={Niklas Beisert}}
\hypersetup{pdfsubject={Manual for the LaTeX2e Package childdoc}}
\date{30 December 2018, \textsf{v2.0}}
\maketitle

\begin{abstract}\noindent
\textsf{childdoc} is a \LaTeXe{} package
that enables the direct compilation
of document sections included by |\include|
to individual files.
\end{abstract}

\begingroup
\parskip0ex
\tableofcontents
\endgroup

%%%%%%%%%%%%%%%%%%%%%%%%%%%%%%%%%%%%%%%%%%%%%%%%%%%%%%%%%%%%%%%%%%%%%%%%%%%%%%%%
%%%%%%%%%%%%%%%%%%%%%%%%%%%%%%%%%%%%%%%%%%%%%%%%%%%%%%%%%%%%%%%%%%%%%%%%%%%%%%%%
\section{Introduction}

\LaTeX{} provides a mechanism to structure a large document (such as a book)
into a main file and several child files (containing the chapters)
using the |\include| command.
This mechanism is beneficial for documents
which span hundreds of pages in order to
make the source file(s) more manageable.
Moreover, compilation can be restricted to
selected child files by means of the |\includeonly| command.
The latter feature can be used to reduce the compilation time while editing
(this was significantly more useful in the earlier days of \LaTeX{})
or to generate a smaller document which is easier to navigate.
Another application of |\includeonly| is to generate
documents consisting of selected parts of the complete document.

However, there are a few drawbacks of the plain |\include| mechanism:
\begin{itemize}
\item
The child files cannot be compiled on their own,
they can only be compiled via the main file.
A naive editing environment
(such as a text editor with an option
to have the current file processed by \LaTeX)
may require one to switch to the main file before compiling;
attempting to compile the child file produces errors.
\item
The main file must be modified (each time)
to adjust the |\includeonly| command
to the present needs. This easily leaves the main file in a messy state.
\item
The generated document will always carry the filename
of the main document. This is inconvenient if
several child files are to be compiled and
to be kept for distribution.
\end{itemize}

The present package provides a simple interface
to make child files individually compilable by \LaTeX{}.
Compiling a child file then has the same effect as compiling
the main file with an |\includeonly| command
to select the appropriate child.
Moreover the generated document will carry the name of the child
rather than the main file.
This resolves all three above issues.

This feature is meant to make the editing of books,
thesis documents and lecture notes somewhat more convenient.
However, the package can also be used efficiently for
composing a series of documents (such as exercise sheets)
which are typically distributed individually.
It then assists the author in generating the individual documents
(potentially in different versions)
as well as a document containing the collected series.
Another application is in developing style files
or other kinds of included material
where compilation of the style file could redirect
to a sample or test file.

%%%%%%%%%%%%%%%%%%%%%%%%%%%%%%%%%%%%%%%%%%%%%%%%%%%%%%%%%%%%%%%%%%%%%%%%%%%%%%%%
%%%%%%%%%%%%%%%%%%%%%%%%%%%%%%%%%%%%%%%%%%%%%%%%%%%%%%%%%%%%%%%%%%%%%%%%%%%%%%%%
\section{Usage}

First of all, the package \textsf{childdoc} is \emph{not} a standard
\LaTeXe{} |.sty| style file! Therefore it needs to be invoked in
a non-standard way.

%%%%%%%%%%%%%%%%%%%%%%%%%%%%%%%%%%%%%%%%%%%%%%%%%%%%%%%%%%%%%%%%%%%%%%%%%%%%%%%%
\subsection{Included Files}
\label{sec:include}

%%%%%%%%%%%%%%%%%%%%%%%%%%%%%%%%%%%%%%%%
\DescribeMacro{\childdocmain}
To use the package, add the commands
\begin{center}
\begin{tabular}{l}
|% \iffalse
%
% childdoc.dtx Copyright (C) 2017-2018 Niklas Beisert
%
% This work may be distributed and/or modified under the
% conditions of the LaTeX Project Public License, either version 1.3
% of this license or (at your option) any later version.
% The latest version of this license is in
%   http://www.latex-project.org/lppl.txt
% and version 1.3 or later is part of all distributions of LaTeX
% version 2005/12/01 or later.
%
% This work has the LPPL maintenance status `maintained'.
%
% The Current Maintainer of this work is Niklas Beisert.
%
% This work consists of the files childdoc.dtx and childdoc.ins
% and the derived files childdoc.def and cdocsamp.tex with
% cdocsch1.tex, cdocsch2.tex, cdocsdrf.tex, cdocsfn1.tex, cdocsfn2.tex.
%
%<package>\ifdefined\childdocmain\endinput\fi
%<package>\ProvidesFile{childdoc.def}[2018/12/30 v2.0 child document driver]
%<samplemain>\ProvidesFile{cdocsamp.tex}[2018/12/30 v2.0 sample for childdoc]
%<*driver>
%\ProvidesFile{childdoc.drv}[2018/12/30 v2.0 childdoc reference manual file]
\PassOptionsToClass{10pt,a4paper}{article}
\documentclass{ltxdoc}

\usepackage[margin=35mm]{geometry}
\usepackage{hyperref}
\usepackage{hyperxmp}
\usepackage[usenames]{color}

\hypersetup{colorlinks=true}
\hypersetup{pdfstartview=FitH}
\hypersetup{pdfpagemode=UseNone}
\hypersetup{pdfsource={}}
\hypersetup{pdflang={en-UK}}
\hypersetup{pdfcopyright={Copyright 2017-2018 Niklas Beisert.
  This work may be distributed and/or modified under the
  conditions of the LaTeX Project Public License, either version 1.3
  of this license or (at your option) any later version.}}
\hypersetup{pdflicenseurl={http://www.latex-project.org/lppl.txt}}
\hypersetup{pdfcontactaddress={ETH Zurich, ITP, HIT K,
  Wolfgang-Pauli-Strasse 27}}
\hypersetup{pdfcontactpostcode={8093}}
\hypersetup{pdfcontactcity={Zurich}}
\hypersetup{pdfcontactcountry={Switzerland}}
\hypersetup{pdfcontactemail={nbeisert@itp.phys.ethz.ch}}
\hypersetup{pdfcontacturl={http://people.phys.ethz.ch/\xmptilde nbeisert/}}

\newcommand{\secref}[1]{\hyperref[#1]{section \ref*{#1}}}

\parskip1ex
\parindent0pt
\let\olditemize\itemize
\def\itemize{\olditemize\parskip0pt}

\begin{document}

\title{The \textsf{childdoc} Package}
\hypersetup{pdftitle={The childdoc Package}}
\author{Niklas Beisert\\[2ex]
  Institut f\"ur Theoretische Physik\\
  Eidgen\"ossische Technische Hochschule Z\"urich\\
  Wolfgang-Pauli-Strasse 27, 8093 Z\"urich, Switzerland\\[1ex]
  \href{mailto:nbeisert@itp.phys.ethz.ch}
  {\texttt{nbeisert@itp.phys.ethz.ch}}}
\hypersetup{pdfauthor={Niklas Beisert}}
\hypersetup{pdfsubject={Manual for the LaTeX2e Package childdoc}}
\date{30 December 2018, \textsf{v2.0}}
\maketitle

\begin{abstract}\noindent
\textsf{childdoc} is a \LaTeXe{} package
that enables the direct compilation
of document sections included by |\include|
to individual files.
\end{abstract}

\begingroup
\parskip0ex
\tableofcontents
\endgroup

%%%%%%%%%%%%%%%%%%%%%%%%%%%%%%%%%%%%%%%%%%%%%%%%%%%%%%%%%%%%%%%%%%%%%%%%%%%%%%%%
%%%%%%%%%%%%%%%%%%%%%%%%%%%%%%%%%%%%%%%%%%%%%%%%%%%%%%%%%%%%%%%%%%%%%%%%%%%%%%%%
\section{Introduction}

\LaTeX{} provides a mechanism to structure a large document (such as a book)
into a main file and several child files (containing the chapters)
using the |\include| command.
This mechanism is beneficial for documents
which span hundreds of pages in order to
make the source file(s) more manageable.
Moreover, compilation can be restricted to
selected child files by means of the |\includeonly| command.
The latter feature can be used to reduce the compilation time while editing
(this was significantly more useful in the earlier days of \LaTeX{})
or to generate a smaller document which is easier to navigate.
Another application of |\includeonly| is to generate
documents consisting of selected parts of the complete document.

However, there are a few drawbacks of the plain |\include| mechanism:
\begin{itemize}
\item
The child files cannot be compiled on their own,
they can only be compiled via the main file.
A naive editing environment
(such as a text editor with an option
to have the current file processed by \LaTeX)
may require one to switch to the main file before compiling;
attempting to compile the child file produces errors.
\item
The main file must be modified (each time)
to adjust the |\includeonly| command
to the present needs. This easily leaves the main file in a messy state.
\item
The generated document will always carry the filename
of the main document. This is inconvenient if
several child files are to be compiled and
to be kept for distribution.
\end{itemize}

The present package provides a simple interface
to make child files individually compilable by \LaTeX{}.
Compiling a child file then has the same effect as compiling
the main file with an |\includeonly| command
to select the appropriate child.
Moreover the generated document will carry the name of the child
rather than the main file.
This resolves all three above issues.

This feature is meant to make the editing of books,
thesis documents and lecture notes somewhat more convenient.
However, the package can also be used efficiently for
composing a series of documents (such as exercise sheets)
which are typically distributed individually.
It then assists the author in generating the individual documents
(potentially in different versions)
as well as a document containing the collected series.
Another application is in developing style files
or other kinds of included material
where compilation of the style file could redirect
to a sample or test file.

%%%%%%%%%%%%%%%%%%%%%%%%%%%%%%%%%%%%%%%%%%%%%%%%%%%%%%%%%%%%%%%%%%%%%%%%%%%%%%%%
%%%%%%%%%%%%%%%%%%%%%%%%%%%%%%%%%%%%%%%%%%%%%%%%%%%%%%%%%%%%%%%%%%%%%%%%%%%%%%%%
\section{Usage}

First of all, the package \textsf{childdoc} is \emph{not} a standard
\LaTeXe{} |.sty| style file! Therefore it needs to be invoked in
a non-standard way.

%%%%%%%%%%%%%%%%%%%%%%%%%%%%%%%%%%%%%%%%%%%%%%%%%%%%%%%%%%%%%%%%%%%%%%%%%%%%%%%%
\subsection{Included Files}
\label{sec:include}

%%%%%%%%%%%%%%%%%%%%%%%%%%%%%%%%%%%%%%%%
\DescribeMacro{\childdocmain}
To use the package, add the commands
\begin{center}
\begin{tabular}{l}
|\input{childdoc.def}|\\
|\childdocmain{}|\\
\end{tabular}
\end{center}
at the very top of the main \LaTeX{} file,
in particular \emph{before} the |\documentclass| statement!
The argument of |\childdocmain| should be left empty
(but it must be present).

%%%%%%%%%%%%%%%%%%%%%%%%%%%%%%%%%%%%%%%%
\DescribeMacro{\childdocof}
Furthermore, add the commands
\begin{center}
\begin{tabular}{l}
|\input{childdoc.def}|\\
|\childdocof{|\textit{main}|}|\\
\end{tabular}
\end{center}
at the top of every child file \textit{child}
which is included by |\include{|\textit{child}|}|
from within the main file
(or at least for those files to be compiled individually).
The argument \textit{main} must be the filename of the main file.

There are a couple of
considerations in setting up the main and child documents:

%%%%%%%%%%%%%%%%%%%%%%%%%%%%%%%%%%%%%%%%
\paragraph{Restrictions.}

Please note the following restrictions:
\begin{itemize}
\item
|\childdocmain| must be called with one argument \textit{main}
to ensure compatibility with earlier version of the package.
It must either be empty (|\childdocmain{}|)
or precisely match the filename of the main file in which it is specified.
See \secref{sec:detection} for further information.
\item
The filename \textit{main} must be specified without the |.tex| extension.
\item
The filename \textit{main} is case sensitive
(even in case-insensitive file systems)
due to internal string comparison.
\item
The argument \textit{main} should be fully expanded, it cannot be a macro.
\item
Subdirectories and special characters should be avoided in filenames.
\item
The command |\childdocmain{|\textit{main}|}| must be followed by a whitespace.
It should not be followed immediately by another command
or by a comment mark `|%|'.
This is because the \TeX{} parser reads the token immediately following
the argument of |\childdocmain| and puts it
at the beginning of every child section;
however, a white\-space is ignored.
\end{itemize}

%%%%%%%%%%%%%%%%%%%%%%%%%%%%%%%%%%%%%%%%
\paragraph{Content of Main File.}

It is advisable to place all content in the child files included by |\include|.
Any output contained in the main file will appear in all child documents
unless suppressed manually;
it cannot be suppressed automatically by the |\includeonly| directive
and thus should normally be avoided.
A method to include some content in the main file
by means of conditional processing is described in \secref{sec:conditional}.

%%%%%%%%%%%%%%%%%%%%%%%%%%%%%%%%%%%%%%%%
\paragraph{Page Numbering.}

When only a part of the document is compiled,
the appropriate numbering of pages
(as well as other status parameters)
is determined from the |.aux| files.
The latter contain information from previous passes.
However this information needs to propagate through
all intermediate child documents.
Therefore the page numbering in child documents may well
be inconsistent until the complete document is compiled at least once.

A useful (if unconventional) way to always ensure a consistent
page numbering is to restart the numbering in each child document
and denote the pages by `\textit{child}|.|\textit{page}'
where \textit{child} represents the chapter/section number of the child file.
This can be achieved by the command
|\numberwithin{page}{|\textit{child}|}|
of the \textsf{amsmath} package
where \textit{child} can be |chapter| or |section|
depending on the chosen structuring.
Alternatively, one can modify the macro |\thepage| appropriately
and reset the counter |page| at the start of each child file.

%%%%%%%%%%%%%%%%%%%%%%%%%%%%%%%%%%%%%%%%%%%%%%%%%%%%%%%%%%%%%%%%%%%%%%%%%%%%%%%%
\subsection{Conditional Processing}
\label{sec:conditional}

The package provides a mechanism to compile different versions
of a document. To customise the versions further some conditional processing
can come in handy to distinguish which version is being compiled.
The package provides two macros to describe the compilation context:

%%%%%%%%%%%%%%%%%%%%%%%%%%%%%%%%%%%%%%%%
\DescribeMacro{\ifchilddoc}
The conditional |\ifchilddoc| distinguishes between the compilation of
child documents and the main document:
%
\begin{center}
|\ifchilddoc |\textit{child-code}| |[|\||else |\textit{main-code}]| \||fi|
\end{center}

%%%%%%%%%%%%%%%%%%%%%%%%%%%%%%%%%%%%%%%%
\DescribeMacro{\childdocname}
\DescribeMacro{\childdocjob}
The macro |\childdocname| contains the filename (without extension)
of the main or child file being processed.
Note that |\childdocjob| will always contain the name of the main file.

%%%%%%%%%%%%%%%%%%%%%%%%%%%%%%%%%%%%%%%%
\paragraph{Title Page.}

Conditional processing can be used to include a title or banner page
in the main document when proper precautions are taken.
Importantly, the code in the main file should ensure that the page counter
(as well as other status parameters which are stored in the |.aux| files)
takes the same value after the conditional processing.
Otherwise the page numbers may take divergent values
depending on which part is compiled.

For example, a title page could be declared by:
%
\begin{center}
\begin{tabular}{l}
|\ifchilddoc\||else|\\
|\addtocounter{page}{-1}|\\
\textit{code for title page}\\
|\newpage|\\
|\||fi|
\end{tabular}
\end{center}
%
A banner page for the child documents can be generated by:
%
\begin{center}
\begin{tabular}{l}
|\ifchilddoc|\\
|\addtocounter{page}{-1}|\\
\textit{code for banner page}\\
|\newpage|\\
|\||fi|
\end{tabular}
\end{center}
%
Here one could write a message such as:
\begin{center}
|This is the part \childdocname{} of \childdocjob{}.|
\end{center}

%%%%%%%%%%%%%%%%%%%%%%%%%%%%%%%%%%%%%%%%%%%%%%%%%%%%%%%%%%%%%%%%%%%%%%%%%%%%%%%%
\subsection{Flags}
\label{sec:flags}

The package makes it easy to generate different versions
of the main or child documents.
To this end compilation flags can be defined
and assigned different default values.
They will be particularly useful in conjunction
with the forwarding mechanism described in \secref{sec:forward}.

For example, it may be useful to have a flag |\version|
which can be set to |draft| or |final|.
The document source will contain some conditional code
depending on the value of |\version|.
Suppose further, the flag should default to |final| for the main file
and to |draft| for child files
which is a natural assignment for editing the document.
This is achieved by placing the following code
in the preamble of the main document
(below the |\childdocmain| directive):
%
\begin{center}
\begin{tabular}{l}
|\ifchilddoc|\\
|\providecommand{\version}{draft}|\\
|\||else|\\
|\providecommand{\version}{final}|\\
|\||fi|
\end{tabular}
\end{center}
%
The definition by |\providecommand| makes sure
that previous definitions are not overwritten.
Further statements |\providecommand{\version}{...}|
can thus be added before the above code to override it.

For the main file, one might add a line
(between |\childdocmain| and the above block)
%
\begin{center}
|%\ifchilddoc\||else\providecommand{\version}{draft}\||fi|
\end{center}
%
which can be uncommented to produce a draft version.
Likewise one can add a line to the very top of a child file
(above the |\childdocof{|\textit{main}|}| directive)
%
\begin{center}
|%\providecommand{\version}{final}|
\end{center}
%
which can be uncommented to produce the final version of this child document.

%%%%%%%%%%%%%%%%%%%%%%%%%%%%%%%%%%%%%%%%%%%%%%%%%%%%%%%%%%%%%%%%%%%%%%%%%%%%%%%%
\subsection{Forwarding}
\label{sec:forward}

Different versions of the main or child documents
using compilation flags as described in \secref{sec:flags}
can be (permanently) stored in different files
for convenient compilation, viewing and distribution.
To this end, the package defines a command
to pass on compilation to a different file:

%%%%%%%%%%%%%%%%%%%%%%%%%%%%%%%%%%%%%%%%
\DescribeMacro{\childdocforward}
The command |\childdocforward| redirects processing to
another source file:
%
\begin{center}
\begin{tabular}{l}
|\input{childdoc.def}|\\
|\childdocforward[|\textit{main}|]{|\textit{dest}|}|\\
\end{tabular}
\end{center}
%
The argument \textit{dest} is the destination file
(without extension).
It should be the main file or one of the child files.
Note that further \textsf{childdoc} directives
such as |\childdocof| and |\childdocforward|
in the indicated file will be processed in this form.
The optional argument \textit{main}
passes on directly to the main file \textit{main}
while pretending to compile the child \textit{dest}.
This form behaves as if \textit{dest}
issues |\childdocof{|\textit{main}|}| right away,
and no further \textsf{childdoc} directives will be processed.

%%%%%%%%%%%%%%%%%%%%%%%%%%%%%%%%%%%%%%%%
\DescribeMacro{\...prefix}
In the alternative form |\childdocforwardprefix|,
%
\begin{center}
\begin{tabular}{l}
|\input{childdoc.def}|\\
|\childdocforwardprefix[|\textit{main}|]{|\textit{prefix}|}{|\textit{dest}|}|
\end{tabular}
\end{center}
%
the destination file is determined by a pattern
depending on the current file:
To make this work, the current file must be called
`{\textit{prefix}\hspace{0.2em}\textit{suffix}}'
with \textit{prefix} matching precisely the argument.
Processing is then passed on to the file
`{\textit{dest}\hspace{0.2em}\textit{suffix}}'.
Surely, the same effect is achieved by
directly specifying the
argument `{\textit{dest}\hspace{0.2em}\textit{suffix}}'
in the first form.
However, that requires to set up a different file
for each child. With the alternative form of the command
all these files can have exactly the same content
which simplifies setting them up and maintaining them.

For example, the following file |draft.tex|
with a compilation flag |\version| as described in \secref{sec:flags}
compiles the main document as a draft:
%
\begin{center}
\begin{tabular}{l}
|\def\version{draft}|\\
|\input{childdoc.def}|\\
|\childdocforward{|\textit{main}|}|
\end{tabular}
\end{center}
%
Likewise, the following files |final|\textit{nn}|.tex|
compile the final version of the child document
|child|\textit{nn}|.tex|:
%
\begin{center}
\begin{tabular}{l}
|\def\version{final}|\\
|\input{childdoc.def}|\\
|\childdocforwardprefix{final}{child}|
\end{tabular}
\end{center}
%

Note that when several versions of a main file and/or of each child file
are to be generated, it may be convenient to set up a |Makefile| or
shell script to automatise the process.

%%%%%%%%%%%%%%%%%%%%%%%%%%%%%%%%%%%%%%%%%%%%%%%%%%%%%%%%%%%%%%%%%%%%%%%%%%%%%%%%
\subsection{Command Line Processing}
\label{sec:commandline}

The effect of redirection files can also be achieved by invoking
the \LaTeX{} compiler with a more elaborate command line.
Most conveniently this should be done as part
of a shell script or a |Makefile|.

When using \textsf{childdoc} in the main file, the following
command lines effectively perform a redirection
(note that depending on the shell being used,
backslashes may have to be doubled: `|\|' $\to$ `|\\|'):
%
\begin{center}
|... -jobname "|\textit{target}|" |\\|"|[\textit{flags}]%
|\input{childdoc.def}\childdocforward[|\textit{main}|]{|\textit{dest}|}"|
\end{center}
%
Here \textit{target} is the name of the output file,
\textit{main} is the name of the main file
and \textit{dest} is the name of the main or child file to be processed
(all filenames without extensions).
The optional argument \textit{main} can be omitted
if \textit{main} matches \textit{dest}.
Optionally, compilation \textit{flags} can be defined via |\def| commands.
This command line makes the \TeX{} engine believe
it is compiling the file \textit{target}
whose content is specified as the latter parameter.
The provided code then forwards the processing to
\textit{main} or \textit{dest} as described in \secref{sec:forward}.

%%%%%%%%%%%%%%%%%%%%%%%%%%%%%%%%%%%%%%%%%%%%%%%%%%%%%%%%%%%%%%%%%%%%%%%%%%%%%%%%
\subsection{Include by Input}
\label{sec:input}

Including child documents by |\include| has some restrictions by design.
Most notably, the content of a child document always occupies
its own set of pages; pages cannot be shared between child documents.
Usually, this behaviour makes perfect sense
because each child document contain an essential part of the document.
However, in some situations it may be desirable to compose
a document from a collection of parts
without having mandatory page breaks between then.
For this case, the package
provides a mechanism to include parts
by |\input| which can also be processed individually.
However, by construction this mechanism
requires manual handling of the content to be output.

%%%%%%%%%%%%%%%%%%%%%%%%%%%%%%%%%%%%%%%%
\DescribeMacro{\ifchilddocmanual}
The main file should be prepared as usual, see \secref{sec:include}.
However, the document body must make a distinction
between processing of an individual part and of the main document, e.g.:
%
\begin{center}
\begin{tabular}{l}
|\ifchilddocmanual|\\
|\input{\childdocname}|\\
|\||else|\\
\textit{document body with }|\input{|\textit{part}|}|\\
|\||fi|
\end{tabular}
\end{center}
%
The conditional |\ifchilddocmanual| is true whenever
a part to be included by |\input| is being compiled,
and the name of the part is stored in |\childdocname|.

%%%%%%%%%%%%%%%%%%%%%%%%%%%%%%%%%%%%%%%%
\DescribeMacro{\childdocby}
Each part to be included by |\input| should start with:
%
\begin{center}
\begin{tabular}{l}
|\input{childdoc.def}|\\
|\childdocby{|\textit{main}|}|\\
\end{tabular}
\end{center}
%
The directive |\childdocby| is similar to |\childdocof|
described in \secref{sec:include},
but the subsequent selection of content must be done manually.
To that end, both |\ifchilddoc| and |\ifchilddocmanual|
will be true upon processing of a part,
and the name of the part is stored in |\childdocname|.
Note that |\jobname| will be set to the filename of the current part
so that each part receives an individual |.aux| file
that does not interfere with the |.aux| file(s) of the main document.
This behaviour can be altered by the alternative form
|\childdocby[*]{|\textit{main}|}| (with a non-empty optional argument)
which uses the |.aux| file of the main document
by setting |\jobname| to \textit{main}.

%%%%%%%%%%%%%%%%%%%%%%%%%%%%%%%%%%%%%%%%%%%%%%%%%%%%%%%%%%%%%%%%%%%%%%%%%%%%%%%%
\subsection{Driver Development}
\label{sec:driver}

The \textsf{childdoc} mechanism can also be use for the development
of definition files such as \LaTeX{} styles or classes.
This case differs from the above setup with multiple parts
included by |\include| in that no |\includeonly| should be invoked.
This can be achieved by starting the include file
(before |\ProvidesPackage|) with:
%
\begin{center}
\begin{tabular}{l}
|\input{childdoc.def}|\\
|\childdocforward{|\textit{main}|}|\\
\end{tabular}
\end{center}
%
or alternatively with:
%
\begin{center}
\begin{tabular}{l}
|\input{childdoc.def}|\\
|\childdocby{|\textit{main}|}|\\
\end{tabular}
\end{center}
%
Both forms have slightly different effects as described above.
The main file is prepared as usual, see \secref{sec:include}.

%%%%%%%%%%%%%%%%%%%%%%%%%%%%%%%%%%%%%%%%%%%%%%%%%%%%%%%%%%%%%%%%%%%%%%%%%%%%%%%%
\subsection{Legacy Detection}
\label{sec:detection}

The directive |\childdocmain| in the main file can detect
whether the complete document or merely a child is to be compiled
even without using the directive |\childdocof|.
This method is deprecated because it is less robust
and there is no compelling reason to use it;
it is merely provided for backward compatibility
and it may be removed in future versions.

If the detection mechanism is to be used,
it is mandatory to correctly specify
the filename of the main file as the argument of |\childdocmain|:
%
\begin{center}
\begin{tabular}{l}
|\input{childdoc.def}|\\
|\childdocmain{|\textit{main}|}|\\
\end{tabular}
\end{center}
%
If |\jobname| does not match the argument \textit{main} of |\childdocmain|,
it is assumed that |\jobname| points to the child file to be compiled.
When using |\childdocmain| with the main file specified as argument,
it suffices to start a child file
with just |\input{|\textit{main}|}|
without loading of the package and using |\childdocof|.
If instead all processing is done
with the appropriate \textsf{childdoc} directives,
the argument of \textit{main} of |\childdocmain| can be empty.

An alternative version of the command line processing described
in \secref{sec:commandline} using the detection mechanism reads:
%
\begin{center}
|... -jobname "|\textit{target}|" "|[\textit{flags}]%
[|\def\jobname{|\textit{dest}|}|]|\input{|\textit{main}|}"|
\end{center}

%%%%%%%%%%%%%%%%%%%%%%%%%%%%%%%%%%%%%%%%%%%%%%%%%%%%%%%%%%%%%%%%%%%%%%%%%%%%%%%%
\subsection{Manual Code}
\label{sec:manual}

In case one cannot be certain whether the definitions file |childdoc.def|
is installed on the target \TeX{} distribution
and one prefers not to ship it,
it is conceivable to paste a few relevant commands into the sources.

To that end, drop all statements |\input{childdoc.def}|
and perform the replacements as outlined below.
Instead of |\childdocmain{|\textit{main}|}| add the following code
to the top of the main file:
%
\begin{center}
\begin{tabular}{l}
|\||ifdefined\childdocname\endinput\||fi\newif\ifchilddoc|\\
|\edef\childdocname{\scantokens\expandafter{\jobname\noexpand}}|\\
|\def\childdocmain{|\textit{main}|}\||ifx\childdocmain\childdocname\||else|\\
|\childdoctrue\includeonly{\childdocname}\let\jobname\childdocmain\||fi|\\
\end{tabular}
\end{center}
%
Instead of |\childdocof{|\textit{main}|}| just include the main file
at the top of each child file:
%
\begin{center}
|\input{|\textit{main}|}|
\end{center}
%
A simple redirection |\childdocforward{|\textit{dest}|}| is achieved by:
%
\begin{center}
|\def\jobname{|\textit{dest}|}\input{\jobname}|
\end{center}
%
The redirection with prefix
|\childdocforwardprefix[|\textit{prefix}|]{|\textit{dest}|}|
is accomplished by:
%
\begin{center}
\begin{tabular}{l}
|{\edef\jobname{\scantokens\expandafter{\jobname\noexpand}}|\\
|\def\redirectjob |\textit{prefix}|#1~~~{\gdef\jobname{|\textit{dest}|#1}}|\\
|\expandafter\redirectjob\jobname~~~}\input{\jobname}|
\end{tabular}
\end{center}

In an alternative approach,
child documents can be compiled by a specific command line
without additional code or specific definitions:
%
\begin{center}
|... -jobname "|\textit{target}|" "|[\textit{flags}]%
|\includeonly{|\textit{dest}|}\input{|\textit{main}|}"|
\end{center}
%

%%%%%%%%%%%%%%%%%%%%%%%%%%%%%%%%%%%%%%%%%%%%%%%%%%%%%%%%%%%%%%%%%%%%%%%%%%%%%%%%
%%%%%%%%%%%%%%%%%%%%%%%%%%%%%%%%%%%%%%%%%%%%%%%%%%%%%%%%%%%%%%%%%%%%%%%%%%%%%%%%
\section{Information}

%%%%%%%%%%%%%%%%%%%%%%%%%%%%%%%%%%%%%%%%%%%%%%%%%%%%%%%%%%%%%%%%%%%%%%%%%%%%%%%%
\subsection{Copyright}

Copyright \copyright{} 2017--2018 Niklas Beisert

This work may be distributed and/or modified under the
conditions of the \LaTeX{} Project Public License, either version 1.3
of this license or (at your option) any later version.
The latest version of this license is in
  \url{http://www.latex-project.org/lppl.txt}
and version 1.3 or later is part of all distributions of \LaTeX{}
version 2005/12/01 or later.

This work has the LPPL maintenance status `maintained'.

The Current Maintainer of this work is Niklas Beisert.

This work consists of the files |README.txt|, |childdoc.ins| and |childdoc.dtx|
as well as the derived files |childdoc.def|, |cdocsamp.tex|
with |cdocsch1.tex|, |cdocsch2.tex|, |cdocspt3.tex|, |cdocspt4.tex|,
|cdocsdrf.tex|, |cdocsfn1.tex|, |cdocsfn2.tex|
as well as |childdoc.pdf|.

%%%%%%%%%%%%%%%%%%%%%%%%%%%%%%%%%%%%%%%%%%%%%%%%%%%%%%%%%%%%%%%%%%%%%%%%%%%%%%%%
\subsection{Files and Installation}

The package consists of the files:
%
\begin{center}
\begin{tabular}{ll}
    |README.txt|   & readme file \\
    |childdoc.ins| & installation file \\
    |childdoc.dtx| & source file \\
    |childdoc.def| & definition file \\
    |cdocsamp.tex| & sample main file \\
    |cdocsch1.tex| & sample include file \\
    |cdocsch2.tex| & sample include file \\
    |cdocspt3.tex| & sample part file \\
    |cdocspt4.tex| & sample part file \\
    |cdocsdrf.tex| & sample redirection file \\
    |cdocsfn1.tex| & sample redirection file \\
    |cdocsfn2.tex| & sample redirection file \\
    |childdoc.pdf| & manual
\end{tabular}
\end{center}
%
The distribution consists of the files
|README.txt|, |childdoc.ins| and |childdoc.dtx|.
%
\begin{itemize}
\item
Run (pdf)\LaTeX{} on |childdoc.dtx|
to compile the manual |childdoc.pdf| (this file).
\item
Run \LaTeX{} on |childdoc.ins| to create the definitions file |childdoc.def|
and the sample |cdocsamp.tex| with include files
|cdocsch1.tex|, |cdocsch2.tex|, |cdocspt3.tex|, |cdocspt4.tex|,
|cdocsdrf.tex|, |cdocsfn1.tex|, |cdocsfn2.tex|.
Then copy the file |childdoc.def| to an appropriate directory of your \LaTeX{}
distribution, e.g.\ \textit{texmf-root}|/tex/latex/childdoc|.
\end{itemize}

%%%%%%%%%%%%%%%%%%%%%%%%%%%%%%%%%%%%%%%%%%%%%%%%%%%%%%%%%%%%%%%%%%%%%%%%%%%%%%%%
\subsection{Related CTAN Packages}

There are several other packages which offer a similar functionality:
%
\begin{itemize}
\item
The packages
\href{http://ctan.org/pkg/docmute}{\textsf{docmute}},
\href{http://ctan.org/pkg/includex}{\textsf{includex}} and
\href{http://ctan.org/pkg/standalone}{\textsf{standalone}}
provide commands to include only the document body of
a child file thus allowing both files to be compiled individually.
\item
The packages \href{http://ctan.org/pkg/subdocs}{\textsf{subdocs}}
and \href{http://ctan.org/pkg/subfiles}{\textsf{subfiles}}
provide structures in which the main and child documents can be
encapsulated and allowing them to be compiled individually.
The inclusion mechanism is different from the conventional |\include|.
\item
The package \href{http://ctan.org/pkg/combine}{\textsf{combine}}
is an elaborate solution to combine several documents into one.
\end{itemize}
%
See also the CTAN topic \href{http://ctan.org/topic/subdocs}{\textsf{subdocs}}
for further related packages.
The present package differs from the above solutions in that
a document structure constructed with the conventional |\include| mechanism
just needs two extra commands at the top of every file
such that all constituent files can be compiled individually.

%%%%%%%%%%%%%%%%%%%%%%%%%%%%%%%%%%%%%%%%%%%%%%%%%%%%%%%%%%%%%%%%%%%%%%%%%%%%%%%%
%\subsection{Feature Suggestions}
%
%The following is a list of features which may be useful for future
%versions of this package:
%%
%\begin{itemize}
%\item
%\ldots
%\end{itemize}

%%%%%%%%%%%%%%%%%%%%%%%%%%%%%%%%%%%%%%%%%%%%%%%%%%%%%%%%%%%%%%%%%%%%%%%%%%%%%%%%
\subsection{Revision History}

%%%%%%%%%%%%%%%%%%%%%%%%%%%%%%%%%%%%%%%%
\paragraph{v2.0:} 2018/12/30

\begin{itemize}
\item
immediate forward processing
\item
added |\childdocby| mechanism
\item
manual restructured
\end{itemize}

%%%%%%%%%%%%%%%%%%%%%%%%%%%%%%%%%%%%%%%%
\paragraph{v1.6:} 2018/01/17

\begin{itemize}
\item
application for development of include files
\item
corrections to manual
\end{itemize}

%%%%%%%%%%%%%%%%%%%%%%%%%%%%%%%%%%%%%%%%
\paragraph{v1.5:} 2017/05/21

\begin{itemize}
\item
more complete structuring introduced
\item
|\childdocof| introduced
\item
|\childdoc| renamed to |\childdocmain|
\item
|\childredirect| renamed to |\childdocforward| and |\childdocforwardprefix|
and functionality expanded
\end{itemize}

%%%%%%%%%%%%%%%%%%%%%%%%%%%%%%%%%%%%%%%%
\paragraph{v1.0:} 2017/04/27

\begin{itemize}
\item
manual and install package
\item
first version published on CTAN
\end{itemize}

%%%%%%%%%%%%%%%%%%%%%%%%%%%%%%%%%%%%%%%%
\paragraph{v0.6:} 2017/04/26

\begin{itemize}
\item
redirection mechanism added
\end{itemize}

%%%%%%%%%%%%%%%%%%%%%%%%%%%%%%%%%%%%%%%%
\paragraph{v0.5:} 2017/04/26

\begin{itemize}
\item
functionality in definition file
\end{itemize}


%%%%%%%%%%%%%%%%%%%%%%%%%%%%%%%%%%%%%%%%%%%%%%%%%%%%%%%%%%%%%%%%%%%%%%%%%%%%%%%%
%%%%%%%%%%%%%%%%%%%%%%%%%%%%%%%%%%%%%%%%%%%%%%%%%%%%%%%%%%%%%%%%%%%%%%%%%%%%%%%%
%%%%%%%%%%%%%%%%%%%%%%%%%%%%%%%%%%%%%%%%%%%%%%%%%%%%%%%%%%%%%%%%%%%%%%%%%%%%%%%%
\appendix

\settowidth\MacroIndent{\rmfamily\scriptsize 000\ }

 \DocInput{childdoc.dtx}

\end{document}
%</driver>
% \fi
%
% %%%%%%%%%%%%%%%%%%%%%%%%%%%%%%%%%%%%%%%%%%%%%%%%%%%%%%%%%%%%%%%%%%%%%%%%%%%%%%
% %%%%%%%%%%%%%%%%%%%%%%%%%%%%%%%%%%%%%%%%%%%%%%%%%%%%%%%%%%%%%%%%%%%%%%%%%%%%%%
% \section{Sample}
%\iffalse
%<*samplemain>
%\fi
%
% The following presents a sample document
% with two chapters, two parts, a title page,
% a compile flag as well as three forwarding files to set the flag.
% It consists of eight |.tex| files:
% \begin{center}
% \begin{tabular}{ll}
% |cdocsamp.tex|&main file\\
% |cdocsch1.tex|&include file for chapter 1\\
% |cdocsch2.tex|&include file for chapter 2\\
% |cdocspt3.tex|&include file for part 3\\
% |cdocspt4.tex|&include file for part 4\\
% |cdocsdrf.tex|&forwarding file for main file in draft mode\\
% |cdocsfi1.tex|&forwarding file for final version of chapter 1\\
% |cdocsfi2.tex|&forwarding file for final version of chapter 2\\
% \end{tabular}
% \end{center}
% Each of the eight files can be compiled directly by the \LaTeX{} compiler.
%
% %%%%%%%%%%%%%%%%%%%%%%%%%%%%%%%%%%%%%%
% \paragraph{Main File.}
%
% The main file is called |cdocsamp.tex|.
%
% Load the \textsf{childdoc} definitions and
% declare the filename for the main document:
%    \begin{macrocode}
\input{childdoc.def}
\childdocmain{}
%    \end{macrocode}

% Optional override for |\version| flag:
%    \begin{macrocode}
%%\ifchilddoc\else\providecommand{\version}{draft}\fi
%    \end{macrocode}

% Define the default values for the |\version| flag
% (|final| for the main file and |draft| for childs):
%    \begin{macrocode}
\ifchilddoc
\providecommand{\version}{draft}
\else
\providecommand{\version}{final}
\fi
%    \end{macrocode}

% Load the standard document class:
%    \begin{macrocode}
\documentclass[12pt]{article}
%    \end{macrocode}

% Start the document body:
%    \begin{macrocode}
\begin{document}
%    \end{macrocode}

% Declare a title page.
% Print title, part of document being processed and version flag:
%    \begin{macrocode}
\addtocounter{page}{-1}
\begin{center}
{\LARGE\bfseries{}childdoc example\par}
\vspace{1cm}
\ifchilddoc
\ifchilddocmanual part\else chapter\fi:
`\childdocname' of `\childdocjob'\par
\else
main document: `\childdocjob'\par
\fi
version: \version\par
\end{center}
\newpage
%    \end{macrocode}

% Manually include selected file,
% otherwise process as usual:
%    \begin{macrocode}
\ifchilddocmanual
\section*{part `\childdocname'}
\input{\childdocname}
\else
%    \end{macrocode}

% Include the two chapters:
%    \begin{macrocode}
\include{cdocsch1}
\include{cdocsch2}
%    \end{macrocode}

% Include the two parts unless only chapters should be displayed:
%    \begin{macrocode}
\ifchilddoc\else
\section{part three}
\input{cdocspt3}
\section{part four}
\input{cdocspt4}
\fi
%    \end{macrocode}

% Process as usual until here:
%    \begin{macrocode}
\fi
%    \end{macrocode}

% End of document body:
%    \begin{macrocode}
\end{document}
%    \end{macrocode}
%\iffalse
%</samplemain>
%\fi
%
% %%%%%%%%%%%%%%%%%%%%%%%%%%%%%%%%%%%%%%
% \paragraph{Chapter Include Files.}
%
% The include files are called |cdocsch1.tex| and |cdocsch2.tex|.
%
%\iffalse
%<*samplechap1|samplechap2>
%\fi

% Optional override for |\version| flag:
%    \begin{macrocode}
%%\providecommand{\version}{final}
%    \end{macrocode}

% Include the main document:
%    \begin{macrocode}
\input{childdoc.def}
\childdocof{cdocsamp}
%    \end{macrocode}

%\iffalse
%</samplechap1|samplechap2>
%\fi
%
%\iffalse
%<*samplechap1>
%\fi
% Some text for chapter 1:
%    \begin{macrocode}
\section{one}
some text in chapter one
%    \end{macrocode}

%\iffalse
%</samplechap1>
%\fi
% Some text for chapter 2:
%\iffalse
%<*samplechap2>
%\fi
%    \begin{macrocode}
\section{two}
more text in chapter two
%    \end{macrocode}

%\iffalse
%</samplechap2>
%\fi
%
% %%%%%%%%%%%%%%%%%%%%%%%%%%%%%%%%%%%%%%
% \paragraph{Part Include Files.}
%
% The include files are called |cdocspt3.tex| and |cdocspt4.tex|.
%
%\iffalse
%<*samplepart3|samplepart4>
%\fi

% Optional override for |\version| flag:
%    \begin{macrocode}
%%\providecommand{\version}{final}
%    \end{macrocode}

% Include the main document:
%    \begin{macrocode}
\input{childdoc.def}
\childdocby{cdocsamp}
%    \end{macrocode}

%\iffalse
%</samplepart3|samplepart4>
%\fi
%
%\iffalse
%<*samplepart3>
%\fi
% Some text for part 3:
%    \begin{macrocode}
some text in part three
%    \end{macrocode}

%\iffalse
%</samplepart3>
%\fi
% Some text for part 4:
%\iffalse
%<*samplepart4>
%\fi
%    \begin{macrocode}
more text in part four
%    \end{macrocode}

%\iffalse
%</samplepart4>
%\fi
%
% %%%%%%%%%%%%%%%%%%%%%%%%%%%%%%%%%%%%%%
% \paragraph{Forwarding for a Complete Draft.}
%
% The following forwarding file |cdocsdrf.tex|
% compiles the main document in draft mode:
%\iffalse
%<*sampledraft>
%\fi
%    \begin{macrocode}
\def\version{draft}
\input{childdoc.def}
\childdocforward{cdocsamp}
%    \end{macrocode}

%\iffalse
%</sampledraft>
%\fi
%
% %%%%%%%%%%%%%%%%%%%%%%%%%%%%%%%%%%%%%%
% \paragraph{Forwarding for Final Version of the Chapters.}
%
% The following forwarding files |cdocsfn1.tex| and |cdocsfn2.tex|
% (with identical content)
% compile the final versions of the child documents
% |cdocsch1.tex| and |cdocsch2.tex|, respectively:
%\iffalse
%<*samplefinal>
%\fi
%    \begin{macrocode}
\def\version{final}
\input{childdoc.def}
\childdocforwardprefix[cdocsamp]{cdocsfn}{cdocsch}
%    \end{macrocode}

%\iffalse
%</samplefinal>
%\fi
%
% %%%%%%%%%%%%%%%%%%%%%%%%%%%%%%%%%%%%%%
% \paragraph{Command Line Processing.}
%
% The following three command lines generate the output files
% |cdocscld|, |cdocscl1| and |cdocscl2|
% which should be identical to
% |cdocsdrf|, |cdocsch1| and |cdocsfn2|, respectively:
% \begin{center}
% \begin{tabular}{l}
% |latex -jobname cdocscld \|\\
% |  "\def\version{draft}\input{childdoc.def}\childdocforward{cdocsamp}"|\\
% |latex -jobname cdocscl1 \|\\
% |  "\input{childdoc.def}\childdocforward[cdocsamp]{cdocsch1}"|\\
% |latex -jobname cdocscl2 \|\\
% |  "\def\version{final}\input{childdoc.def}\childdocforward{cdocsch2}"|
% \end{tabular}
% \end{center}
% Note that the trailing backslash on each first line
% merely continues the input to the second line
% (for convenient cut ant paste).
% Furthermore, the command |latex| can be replaced by any
% of its alternative versions such as |pdflatex|.
%
% %%%%%%%%%%%%%%%%%%%%%%%%%%%%%%%%%%%%%%%%%%%%%%%%%%%%%%%%%%%%%%%%%%%%%%%%%%%%%%
% %%%%%%%%%%%%%%%%%%%%%%%%%%%%%%%%%%%%%%%%%%%%%%%%%%%%%%%%%%%%%%%%%%%%%%%%%%%%%%
% \section{Implementation}
%\iffalse
%<*package>
%\fi
%
% This section describes the definitions file |childdoc.def|.

% The definitions cannot be loaded using |\usepackage| or |\RequirePackage|
% which has a mechanism to prevent loading a style file more than once.
% When loading the definitions by means of |\input|
% multiple instances have to be prevented manually:
%\iffalse
%This code needs to be before the `\ProvidesFile' directive
%which is defined at the beginning of this file.
%Therefore it is also placed there and commented out here.
%</package>
%<*discard>
%\fi
%    \begin{macrocode}
\ifdefined\childdocmain\endinput\fi
%    \end{macrocode}
%\iffalse
%</discard>
%<*package>
%\fi
%
% \macro{\ifchilddoc}
% \macro{\ifchilddocmanual}
% The conditional |\ifchilddoc| tells whether a
% child (true) or main (false) document is being compiled.
% The conditional |\ifchilddocmanual| tells whether
% the |\includeonly| mechanism is used (false) or
% the selection of child files must be performed manually (true).
% The definitions initialise to false:
%    \begin{macrocode}
\newif\ifchilddoc
\newif\ifchilddocmanual
%    \end{macrocode}

% \macro{\childdocname}
% \macro{\childdocjob}
% The macro |\childdocname| stores the name of the main document
% to be compiled. The macro |\childdocjob| stores the name of
% the document on which the \LaTeX{} compiler was originally invoked.
% The content of |\jobname| cannot be compared
% to filenames specified in the source due to different catcodes.
% The following code rescans |\jobname|, stores the result
% in |\childdocname| and saves a copy in |\childdocjob|:
%    \begin{macrocode}
\edef\childdocname{\scantokens\expandafter{\jobname\noexpand}}
\let\childdocjob\childdocname
%    \end{macrocode}

% \macro{\childdocdisable}
% The macro |\childdocdisable| prevents the main file
% from being processed more than once.
% At this stage, the main document command |\childdocmain|
% is assumed to be called once again where it should do nothing.
% Any subsequent call to it should prevent
% a secondary processing of the main document
% It overwrites the forwarding commands
% |\childdocof| and |\childdocforward|
% with empty macros to prevent further inclusions of the main document:
%    \begin{macrocode}
\newcommand{\childdocdisable}
{
  \renewcommand{\childdocmain}[1]{\renewcommand{\childdocmain}[1]{\endinput}}
  \renewcommand{\childdocof}[1]{}
  \renewcommand{\childdocby}[2][]{}
  \renewcommand{\childdocforward}[2][]{}
  \renewcommand{\childdocdisable}{}
}
%    \end{macrocode}

% \macro{\childdocmain}
% The macro |\childdocmain| is to be called at the top of the main file
% with nothing or the main filename (without extension) as argument.
% First, it breaks loops.
% If the argument is not empty and does not match |\childdocname|
% (which is set by the first inclusion of |childdoc.def|),
% |\ifchilddoc| is set to true, |\includeonly| is applied to the child file
% and |\jobname| is set to the main file
% (for proper handling of |.aux| files):
%    \begin{macrocode}
\newcommand{\childdocmain}[1]
{
  \childdocdisable\childdocmain{}
  \if?#1?\else
    \begingroup
      \def\childdoctmp{#1}
      \ifx\childdoctmp\childdocname
        \def\childdoctmp{}
      \else
        \def\childdoctmp
        {
          \childdoctrue
          \includeonly{\childdocname}
          \def\childdocjob{#1}
          \def\jobname{#1}
        }
      \fi
      \expandafter
    \endgroup
    \childdoctmp
  \fi
}
%    \end{macrocode}

% \macro{\childdocof}
% The command |\childdocof| redirects
% compilation to the main file |#1|.
%    \begin{macrocode}
\newcommand{\childdocof}[1]
{
  \childdocdisable
  \childdoctrue
  \includeonly{\childdocname}
  \def\jobname{#1}
  \def\childdocjob{#1}
  \input{#1}
}
%    \end{macrocode}

% \macro{\childdocby}
% The command |\childdocby| ....
%    \begin{macrocode}
\newcommand{\childdocby}[2][]
{
  \childdocdisable
  \childdoctrue
  \childdocmanualtrue
  \if?#1?\else
    \def\jobname{#2}
  \fi
  \def\childdocjob{#2}
  \input{#2}
  \endinput
}
%    \end{macrocode}

% \macro{\childdocforward}
% The command |\childdocforward| redirects
% compilation to the main file or
% (if the optional argument is given) a child file.
% Parameters are set as if the main file
% or a child file starting with |\childdocof| was compiled.
% Then compilation is handed over to the main file:
%    \begin{macrocode}
\newcommand{\childdocforward}[2][]
{
  \begingroup
    \if?#1?
      \def\childdoctmp
      {
        \def\childdocname{#2}
        \def\childdocjob{#2}
        \def\jobname{#2}
        \input{#2}
        \endinput
      }
    \else
      \def\childdoctmp
      {
        \childdocdisable
        \def\childdocname{#2}
        \childdoctrue
        \includeonly{#2}
        \def\childdocjob{#1}
        \def\jobname{#1}
        \input{#1}
        \endinput
      }
    \fi
    \expandafter
  \endgroup
  \childdoctmp
}
%    \end{macrocode}

% \macro{\childdocforwardprefix}
% The command |\childdocforwardprefix| redirects
% compilation to the main or a child file by means of a pattern.
% The prefix |#1| in the current filename is replaced by |#2|
% and the suffix of the current filename is kept
% (it is assumed that the filename does not contain the substring `|~~~|'
% which is used as a delimiter).
% Compilation is handed over to the new file by |\childdocforward|:
%    \begin{macrocode}
\newcommand{\childdocforwardprefix}[3][]
{
  \begingroup
    \def\childdocextract #2##1~~~{\def\childdoctmp{\childdocforward[#1]{#3##1}}}
    \expandafter\childdocextract\childdocname~~~
    \expandafter
  \endgroup
  \childdoctmp
}
%    \end{macrocode}

% \macro{\childdoc}
% The deprecated macro |\childdoc| is a legacy version of |\childdocmain|:
%    \begin{macrocode}
\newcommand{\childdoc}{\childdocmain}
%    \end{macrocode}

% \macro{\childdocredirect}
% The deprecated macro |\childdocredirect| is a legacy version
% of |\childdocforward| and |\childdocforwardprefix|:
%    \begin{macrocode}
\newcommand{\childdocredirect}[2][]
{
  \begingroup
    \if?#1?
      \def\childdoctmp{\childdocforward{#2}}
    \else
      \def\childdoctmp{\childdocforwardprefix{#1}{#2}}
    \fi
    \expandafter
  \endgroup
  \childdoctmp
}
%    \end{macrocode}

%\iffalse
%</package>
%\fi
%
\endinput
|\\
|\childdocmain{}|\\
\end{tabular}
\end{center}
at the very top of the main \LaTeX{} file,
in particular \emph{before} the |\documentclass| statement!
The argument of |\childdocmain| should be left empty
(but it must be present).

%%%%%%%%%%%%%%%%%%%%%%%%%%%%%%%%%%%%%%%%
\DescribeMacro{\childdocof}
Furthermore, add the commands
\begin{center}
\begin{tabular}{l}
|% \iffalse
%
% childdoc.dtx Copyright (C) 2017-2018 Niklas Beisert
%
% This work may be distributed and/or modified under the
% conditions of the LaTeX Project Public License, either version 1.3
% of this license or (at your option) any later version.
% The latest version of this license is in
%   http://www.latex-project.org/lppl.txt
% and version 1.3 or later is part of all distributions of LaTeX
% version 2005/12/01 or later.
%
% This work has the LPPL maintenance status `maintained'.
%
% The Current Maintainer of this work is Niklas Beisert.
%
% This work consists of the files childdoc.dtx and childdoc.ins
% and the derived files childdoc.def and cdocsamp.tex with
% cdocsch1.tex, cdocsch2.tex, cdocsdrf.tex, cdocsfn1.tex, cdocsfn2.tex.
%
%<package>\ifdefined\childdocmain\endinput\fi
%<package>\ProvidesFile{childdoc.def}[2018/12/30 v2.0 child document driver]
%<samplemain>\ProvidesFile{cdocsamp.tex}[2018/12/30 v2.0 sample for childdoc]
%<*driver>
%\ProvidesFile{childdoc.drv}[2018/12/30 v2.0 childdoc reference manual file]
\PassOptionsToClass{10pt,a4paper}{article}
\documentclass{ltxdoc}

\usepackage[margin=35mm]{geometry}
\usepackage{hyperref}
\usepackage{hyperxmp}
\usepackage[usenames]{color}

\hypersetup{colorlinks=true}
\hypersetup{pdfstartview=FitH}
\hypersetup{pdfpagemode=UseNone}
\hypersetup{pdfsource={}}
\hypersetup{pdflang={en-UK}}
\hypersetup{pdfcopyright={Copyright 2017-2018 Niklas Beisert.
  This work may be distributed and/or modified under the
  conditions of the LaTeX Project Public License, either version 1.3
  of this license or (at your option) any later version.}}
\hypersetup{pdflicenseurl={http://www.latex-project.org/lppl.txt}}
\hypersetup{pdfcontactaddress={ETH Zurich, ITP, HIT K,
  Wolfgang-Pauli-Strasse 27}}
\hypersetup{pdfcontactpostcode={8093}}
\hypersetup{pdfcontactcity={Zurich}}
\hypersetup{pdfcontactcountry={Switzerland}}
\hypersetup{pdfcontactemail={nbeisert@itp.phys.ethz.ch}}
\hypersetup{pdfcontacturl={http://people.phys.ethz.ch/\xmptilde nbeisert/}}

\newcommand{\secref}[1]{\hyperref[#1]{section \ref*{#1}}}

\parskip1ex
\parindent0pt
\let\olditemize\itemize
\def\itemize{\olditemize\parskip0pt}

\begin{document}

\title{The \textsf{childdoc} Package}
\hypersetup{pdftitle={The childdoc Package}}
\author{Niklas Beisert\\[2ex]
  Institut f\"ur Theoretische Physik\\
  Eidgen\"ossische Technische Hochschule Z\"urich\\
  Wolfgang-Pauli-Strasse 27, 8093 Z\"urich, Switzerland\\[1ex]
  \href{mailto:nbeisert@itp.phys.ethz.ch}
  {\texttt{nbeisert@itp.phys.ethz.ch}}}
\hypersetup{pdfauthor={Niklas Beisert}}
\hypersetup{pdfsubject={Manual for the LaTeX2e Package childdoc}}
\date{30 December 2018, \textsf{v2.0}}
\maketitle

\begin{abstract}\noindent
\textsf{childdoc} is a \LaTeXe{} package
that enables the direct compilation
of document sections included by |\include|
to individual files.
\end{abstract}

\begingroup
\parskip0ex
\tableofcontents
\endgroup

%%%%%%%%%%%%%%%%%%%%%%%%%%%%%%%%%%%%%%%%%%%%%%%%%%%%%%%%%%%%%%%%%%%%%%%%%%%%%%%%
%%%%%%%%%%%%%%%%%%%%%%%%%%%%%%%%%%%%%%%%%%%%%%%%%%%%%%%%%%%%%%%%%%%%%%%%%%%%%%%%
\section{Introduction}

\LaTeX{} provides a mechanism to structure a large document (such as a book)
into a main file and several child files (containing the chapters)
using the |\include| command.
This mechanism is beneficial for documents
which span hundreds of pages in order to
make the source file(s) more manageable.
Moreover, compilation can be restricted to
selected child files by means of the |\includeonly| command.
The latter feature can be used to reduce the compilation time while editing
(this was significantly more useful in the earlier days of \LaTeX{})
or to generate a smaller document which is easier to navigate.
Another application of |\includeonly| is to generate
documents consisting of selected parts of the complete document.

However, there are a few drawbacks of the plain |\include| mechanism:
\begin{itemize}
\item
The child files cannot be compiled on their own,
they can only be compiled via the main file.
A naive editing environment
(such as a text editor with an option
to have the current file processed by \LaTeX)
may require one to switch to the main file before compiling;
attempting to compile the child file produces errors.
\item
The main file must be modified (each time)
to adjust the |\includeonly| command
to the present needs. This easily leaves the main file in a messy state.
\item
The generated document will always carry the filename
of the main document. This is inconvenient if
several child files are to be compiled and
to be kept for distribution.
\end{itemize}

The present package provides a simple interface
to make child files individually compilable by \LaTeX{}.
Compiling a child file then has the same effect as compiling
the main file with an |\includeonly| command
to select the appropriate child.
Moreover the generated document will carry the name of the child
rather than the main file.
This resolves all three above issues.

This feature is meant to make the editing of books,
thesis documents and lecture notes somewhat more convenient.
However, the package can also be used efficiently for
composing a series of documents (such as exercise sheets)
which are typically distributed individually.
It then assists the author in generating the individual documents
(potentially in different versions)
as well as a document containing the collected series.
Another application is in developing style files
or other kinds of included material
where compilation of the style file could redirect
to a sample or test file.

%%%%%%%%%%%%%%%%%%%%%%%%%%%%%%%%%%%%%%%%%%%%%%%%%%%%%%%%%%%%%%%%%%%%%%%%%%%%%%%%
%%%%%%%%%%%%%%%%%%%%%%%%%%%%%%%%%%%%%%%%%%%%%%%%%%%%%%%%%%%%%%%%%%%%%%%%%%%%%%%%
\section{Usage}

First of all, the package \textsf{childdoc} is \emph{not} a standard
\LaTeXe{} |.sty| style file! Therefore it needs to be invoked in
a non-standard way.

%%%%%%%%%%%%%%%%%%%%%%%%%%%%%%%%%%%%%%%%%%%%%%%%%%%%%%%%%%%%%%%%%%%%%%%%%%%%%%%%
\subsection{Included Files}
\label{sec:include}

%%%%%%%%%%%%%%%%%%%%%%%%%%%%%%%%%%%%%%%%
\DescribeMacro{\childdocmain}
To use the package, add the commands
\begin{center}
\begin{tabular}{l}
|\input{childdoc.def}|\\
|\childdocmain{}|\\
\end{tabular}
\end{center}
at the very top of the main \LaTeX{} file,
in particular \emph{before} the |\documentclass| statement!
The argument of |\childdocmain| should be left empty
(but it must be present).

%%%%%%%%%%%%%%%%%%%%%%%%%%%%%%%%%%%%%%%%
\DescribeMacro{\childdocof}
Furthermore, add the commands
\begin{center}
\begin{tabular}{l}
|\input{childdoc.def}|\\
|\childdocof{|\textit{main}|}|\\
\end{tabular}
\end{center}
at the top of every child file \textit{child}
which is included by |\include{|\textit{child}|}|
from within the main file
(or at least for those files to be compiled individually).
The argument \textit{main} must be the filename of the main file.

There are a couple of
considerations in setting up the main and child documents:

%%%%%%%%%%%%%%%%%%%%%%%%%%%%%%%%%%%%%%%%
\paragraph{Restrictions.}

Please note the following restrictions:
\begin{itemize}
\item
|\childdocmain| must be called with one argument \textit{main}
to ensure compatibility with earlier version of the package.
It must either be empty (|\childdocmain{}|)
or precisely match the filename of the main file in which it is specified.
See \secref{sec:detection} for further information.
\item
The filename \textit{main} must be specified without the |.tex| extension.
\item
The filename \textit{main} is case sensitive
(even in case-insensitive file systems)
due to internal string comparison.
\item
The argument \textit{main} should be fully expanded, it cannot be a macro.
\item
Subdirectories and special characters should be avoided in filenames.
\item
The command |\childdocmain{|\textit{main}|}| must be followed by a whitespace.
It should not be followed immediately by another command
or by a comment mark `|%|'.
This is because the \TeX{} parser reads the token immediately following
the argument of |\childdocmain| and puts it
at the beginning of every child section;
however, a white\-space is ignored.
\end{itemize}

%%%%%%%%%%%%%%%%%%%%%%%%%%%%%%%%%%%%%%%%
\paragraph{Content of Main File.}

It is advisable to place all content in the child files included by |\include|.
Any output contained in the main file will appear in all child documents
unless suppressed manually;
it cannot be suppressed automatically by the |\includeonly| directive
and thus should normally be avoided.
A method to include some content in the main file
by means of conditional processing is described in \secref{sec:conditional}.

%%%%%%%%%%%%%%%%%%%%%%%%%%%%%%%%%%%%%%%%
\paragraph{Page Numbering.}

When only a part of the document is compiled,
the appropriate numbering of pages
(as well as other status parameters)
is determined from the |.aux| files.
The latter contain information from previous passes.
However this information needs to propagate through
all intermediate child documents.
Therefore the page numbering in child documents may well
be inconsistent until the complete document is compiled at least once.

A useful (if unconventional) way to always ensure a consistent
page numbering is to restart the numbering in each child document
and denote the pages by `\textit{child}|.|\textit{page}'
where \textit{child} represents the chapter/section number of the child file.
This can be achieved by the command
|\numberwithin{page}{|\textit{child}|}|
of the \textsf{amsmath} package
where \textit{child} can be |chapter| or |section|
depending on the chosen structuring.
Alternatively, one can modify the macro |\thepage| appropriately
and reset the counter |page| at the start of each child file.

%%%%%%%%%%%%%%%%%%%%%%%%%%%%%%%%%%%%%%%%%%%%%%%%%%%%%%%%%%%%%%%%%%%%%%%%%%%%%%%%
\subsection{Conditional Processing}
\label{sec:conditional}

The package provides a mechanism to compile different versions
of a document. To customise the versions further some conditional processing
can come in handy to distinguish which version is being compiled.
The package provides two macros to describe the compilation context:

%%%%%%%%%%%%%%%%%%%%%%%%%%%%%%%%%%%%%%%%
\DescribeMacro{\ifchilddoc}
The conditional |\ifchilddoc| distinguishes between the compilation of
child documents and the main document:
%
\begin{center}
|\ifchilddoc |\textit{child-code}| |[|\||else |\textit{main-code}]| \||fi|
\end{center}

%%%%%%%%%%%%%%%%%%%%%%%%%%%%%%%%%%%%%%%%
\DescribeMacro{\childdocname}
\DescribeMacro{\childdocjob}
The macro |\childdocname| contains the filename (without extension)
of the main or child file being processed.
Note that |\childdocjob| will always contain the name of the main file.

%%%%%%%%%%%%%%%%%%%%%%%%%%%%%%%%%%%%%%%%
\paragraph{Title Page.}

Conditional processing can be used to include a title or banner page
in the main document when proper precautions are taken.
Importantly, the code in the main file should ensure that the page counter
(as well as other status parameters which are stored in the |.aux| files)
takes the same value after the conditional processing.
Otherwise the page numbers may take divergent values
depending on which part is compiled.

For example, a title page could be declared by:
%
\begin{center}
\begin{tabular}{l}
|\ifchilddoc\||else|\\
|\addtocounter{page}{-1}|\\
\textit{code for title page}\\
|\newpage|\\
|\||fi|
\end{tabular}
\end{center}
%
A banner page for the child documents can be generated by:
%
\begin{center}
\begin{tabular}{l}
|\ifchilddoc|\\
|\addtocounter{page}{-1}|\\
\textit{code for banner page}\\
|\newpage|\\
|\||fi|
\end{tabular}
\end{center}
%
Here one could write a message such as:
\begin{center}
|This is the part \childdocname{} of \childdocjob{}.|
\end{center}

%%%%%%%%%%%%%%%%%%%%%%%%%%%%%%%%%%%%%%%%%%%%%%%%%%%%%%%%%%%%%%%%%%%%%%%%%%%%%%%%
\subsection{Flags}
\label{sec:flags}

The package makes it easy to generate different versions
of the main or child documents.
To this end compilation flags can be defined
and assigned different default values.
They will be particularly useful in conjunction
with the forwarding mechanism described in \secref{sec:forward}.

For example, it may be useful to have a flag |\version|
which can be set to |draft| or |final|.
The document source will contain some conditional code
depending on the value of |\version|.
Suppose further, the flag should default to |final| for the main file
and to |draft| for child files
which is a natural assignment for editing the document.
This is achieved by placing the following code
in the preamble of the main document
(below the |\childdocmain| directive):
%
\begin{center}
\begin{tabular}{l}
|\ifchilddoc|\\
|\providecommand{\version}{draft}|\\
|\||else|\\
|\providecommand{\version}{final}|\\
|\||fi|
\end{tabular}
\end{center}
%
The definition by |\providecommand| makes sure
that previous definitions are not overwritten.
Further statements |\providecommand{\version}{...}|
can thus be added before the above code to override it.

For the main file, one might add a line
(between |\childdocmain| and the above block)
%
\begin{center}
|%\ifchilddoc\||else\providecommand{\version}{draft}\||fi|
\end{center}
%
which can be uncommented to produce a draft version.
Likewise one can add a line to the very top of a child file
(above the |\childdocof{|\textit{main}|}| directive)
%
\begin{center}
|%\providecommand{\version}{final}|
\end{center}
%
which can be uncommented to produce the final version of this child document.

%%%%%%%%%%%%%%%%%%%%%%%%%%%%%%%%%%%%%%%%%%%%%%%%%%%%%%%%%%%%%%%%%%%%%%%%%%%%%%%%
\subsection{Forwarding}
\label{sec:forward}

Different versions of the main or child documents
using compilation flags as described in \secref{sec:flags}
can be (permanently) stored in different files
for convenient compilation, viewing and distribution.
To this end, the package defines a command
to pass on compilation to a different file:

%%%%%%%%%%%%%%%%%%%%%%%%%%%%%%%%%%%%%%%%
\DescribeMacro{\childdocforward}
The command |\childdocforward| redirects processing to
another source file:
%
\begin{center}
\begin{tabular}{l}
|\input{childdoc.def}|\\
|\childdocforward[|\textit{main}|]{|\textit{dest}|}|\\
\end{tabular}
\end{center}
%
The argument \textit{dest} is the destination file
(without extension).
It should be the main file or one of the child files.
Note that further \textsf{childdoc} directives
such as |\childdocof| and |\childdocforward|
in the indicated file will be processed in this form.
The optional argument \textit{main}
passes on directly to the main file \textit{main}
while pretending to compile the child \textit{dest}.
This form behaves as if \textit{dest}
issues |\childdocof{|\textit{main}|}| right away,
and no further \textsf{childdoc} directives will be processed.

%%%%%%%%%%%%%%%%%%%%%%%%%%%%%%%%%%%%%%%%
\DescribeMacro{\...prefix}
In the alternative form |\childdocforwardprefix|,
%
\begin{center}
\begin{tabular}{l}
|\input{childdoc.def}|\\
|\childdocforwardprefix[|\textit{main}|]{|\textit{prefix}|}{|\textit{dest}|}|
\end{tabular}
\end{center}
%
the destination file is determined by a pattern
depending on the current file:
To make this work, the current file must be called
`{\textit{prefix}\hspace{0.2em}\textit{suffix}}'
with \textit{prefix} matching precisely the argument.
Processing is then passed on to the file
`{\textit{dest}\hspace{0.2em}\textit{suffix}}'.
Surely, the same effect is achieved by
directly specifying the
argument `{\textit{dest}\hspace{0.2em}\textit{suffix}}'
in the first form.
However, that requires to set up a different file
for each child. With the alternative form of the command
all these files can have exactly the same content
which simplifies setting them up and maintaining them.

For example, the following file |draft.tex|
with a compilation flag |\version| as described in \secref{sec:flags}
compiles the main document as a draft:
%
\begin{center}
\begin{tabular}{l}
|\def\version{draft}|\\
|\input{childdoc.def}|\\
|\childdocforward{|\textit{main}|}|
\end{tabular}
\end{center}
%
Likewise, the following files |final|\textit{nn}|.tex|
compile the final version of the child document
|child|\textit{nn}|.tex|:
%
\begin{center}
\begin{tabular}{l}
|\def\version{final}|\\
|\input{childdoc.def}|\\
|\childdocforwardprefix{final}{child}|
\end{tabular}
\end{center}
%

Note that when several versions of a main file and/or of each child file
are to be generated, it may be convenient to set up a |Makefile| or
shell script to automatise the process.

%%%%%%%%%%%%%%%%%%%%%%%%%%%%%%%%%%%%%%%%%%%%%%%%%%%%%%%%%%%%%%%%%%%%%%%%%%%%%%%%
\subsection{Command Line Processing}
\label{sec:commandline}

The effect of redirection files can also be achieved by invoking
the \LaTeX{} compiler with a more elaborate command line.
Most conveniently this should be done as part
of a shell script or a |Makefile|.

When using \textsf{childdoc} in the main file, the following
command lines effectively perform a redirection
(note that depending on the shell being used,
backslashes may have to be doubled: `|\|' $\to$ `|\\|'):
%
\begin{center}
|... -jobname "|\textit{target}|" |\\|"|[\textit{flags}]%
|\input{childdoc.def}\childdocforward[|\textit{main}|]{|\textit{dest}|}"|
\end{center}
%
Here \textit{target} is the name of the output file,
\textit{main} is the name of the main file
and \textit{dest} is the name of the main or child file to be processed
(all filenames without extensions).
The optional argument \textit{main} can be omitted
if \textit{main} matches \textit{dest}.
Optionally, compilation \textit{flags} can be defined via |\def| commands.
This command line makes the \TeX{} engine believe
it is compiling the file \textit{target}
whose content is specified as the latter parameter.
The provided code then forwards the processing to
\textit{main} or \textit{dest} as described in \secref{sec:forward}.

%%%%%%%%%%%%%%%%%%%%%%%%%%%%%%%%%%%%%%%%%%%%%%%%%%%%%%%%%%%%%%%%%%%%%%%%%%%%%%%%
\subsection{Include by Input}
\label{sec:input}

Including child documents by |\include| has some restrictions by design.
Most notably, the content of a child document always occupies
its own set of pages; pages cannot be shared between child documents.
Usually, this behaviour makes perfect sense
because each child document contain an essential part of the document.
However, in some situations it may be desirable to compose
a document from a collection of parts
without having mandatory page breaks between then.
For this case, the package
provides a mechanism to include parts
by |\input| which can also be processed individually.
However, by construction this mechanism
requires manual handling of the content to be output.

%%%%%%%%%%%%%%%%%%%%%%%%%%%%%%%%%%%%%%%%
\DescribeMacro{\ifchilddocmanual}
The main file should be prepared as usual, see \secref{sec:include}.
However, the document body must make a distinction
between processing of an individual part and of the main document, e.g.:
%
\begin{center}
\begin{tabular}{l}
|\ifchilddocmanual|\\
|\input{\childdocname}|\\
|\||else|\\
\textit{document body with }|\input{|\textit{part}|}|\\
|\||fi|
\end{tabular}
\end{center}
%
The conditional |\ifchilddocmanual| is true whenever
a part to be included by |\input| is being compiled,
and the name of the part is stored in |\childdocname|.

%%%%%%%%%%%%%%%%%%%%%%%%%%%%%%%%%%%%%%%%
\DescribeMacro{\childdocby}
Each part to be included by |\input| should start with:
%
\begin{center}
\begin{tabular}{l}
|\input{childdoc.def}|\\
|\childdocby{|\textit{main}|}|\\
\end{tabular}
\end{center}
%
The directive |\childdocby| is similar to |\childdocof|
described in \secref{sec:include},
but the subsequent selection of content must be done manually.
To that end, both |\ifchilddoc| and |\ifchilddocmanual|
will be true upon processing of a part,
and the name of the part is stored in |\childdocname|.
Note that |\jobname| will be set to the filename of the current part
so that each part receives an individual |.aux| file
that does not interfere with the |.aux| file(s) of the main document.
This behaviour can be altered by the alternative form
|\childdocby[*]{|\textit{main}|}| (with a non-empty optional argument)
which uses the |.aux| file of the main document
by setting |\jobname| to \textit{main}.

%%%%%%%%%%%%%%%%%%%%%%%%%%%%%%%%%%%%%%%%%%%%%%%%%%%%%%%%%%%%%%%%%%%%%%%%%%%%%%%%
\subsection{Driver Development}
\label{sec:driver}

The \textsf{childdoc} mechanism can also be use for the development
of definition files such as \LaTeX{} styles or classes.
This case differs from the above setup with multiple parts
included by |\include| in that no |\includeonly| should be invoked.
This can be achieved by starting the include file
(before |\ProvidesPackage|) with:
%
\begin{center}
\begin{tabular}{l}
|\input{childdoc.def}|\\
|\childdocforward{|\textit{main}|}|\\
\end{tabular}
\end{center}
%
or alternatively with:
%
\begin{center}
\begin{tabular}{l}
|\input{childdoc.def}|\\
|\childdocby{|\textit{main}|}|\\
\end{tabular}
\end{center}
%
Both forms have slightly different effects as described above.
The main file is prepared as usual, see \secref{sec:include}.

%%%%%%%%%%%%%%%%%%%%%%%%%%%%%%%%%%%%%%%%%%%%%%%%%%%%%%%%%%%%%%%%%%%%%%%%%%%%%%%%
\subsection{Legacy Detection}
\label{sec:detection}

The directive |\childdocmain| in the main file can detect
whether the complete document or merely a child is to be compiled
even without using the directive |\childdocof|.
This method is deprecated because it is less robust
and there is no compelling reason to use it;
it is merely provided for backward compatibility
and it may be removed in future versions.

If the detection mechanism is to be used,
it is mandatory to correctly specify
the filename of the main file as the argument of |\childdocmain|:
%
\begin{center}
\begin{tabular}{l}
|\input{childdoc.def}|\\
|\childdocmain{|\textit{main}|}|\\
\end{tabular}
\end{center}
%
If |\jobname| does not match the argument \textit{main} of |\childdocmain|,
it is assumed that |\jobname| points to the child file to be compiled.
When using |\childdocmain| with the main file specified as argument,
it suffices to start a child file
with just |\input{|\textit{main}|}|
without loading of the package and using |\childdocof|.
If instead all processing is done
with the appropriate \textsf{childdoc} directives,
the argument of \textit{main} of |\childdocmain| can be empty.

An alternative version of the command line processing described
in \secref{sec:commandline} using the detection mechanism reads:
%
\begin{center}
|... -jobname "|\textit{target}|" "|[\textit{flags}]%
[|\def\jobname{|\textit{dest}|}|]|\input{|\textit{main}|}"|
\end{center}

%%%%%%%%%%%%%%%%%%%%%%%%%%%%%%%%%%%%%%%%%%%%%%%%%%%%%%%%%%%%%%%%%%%%%%%%%%%%%%%%
\subsection{Manual Code}
\label{sec:manual}

In case one cannot be certain whether the definitions file |childdoc.def|
is installed on the target \TeX{} distribution
and one prefers not to ship it,
it is conceivable to paste a few relevant commands into the sources.

To that end, drop all statements |\input{childdoc.def}|
and perform the replacements as outlined below.
Instead of |\childdocmain{|\textit{main}|}| add the following code
to the top of the main file:
%
\begin{center}
\begin{tabular}{l}
|\||ifdefined\childdocname\endinput\||fi\newif\ifchilddoc|\\
|\edef\childdocname{\scantokens\expandafter{\jobname\noexpand}}|\\
|\def\childdocmain{|\textit{main}|}\||ifx\childdocmain\childdocname\||else|\\
|\childdoctrue\includeonly{\childdocname}\let\jobname\childdocmain\||fi|\\
\end{tabular}
\end{center}
%
Instead of |\childdocof{|\textit{main}|}| just include the main file
at the top of each child file:
%
\begin{center}
|\input{|\textit{main}|}|
\end{center}
%
A simple redirection |\childdocforward{|\textit{dest}|}| is achieved by:
%
\begin{center}
|\def\jobname{|\textit{dest}|}\input{\jobname}|
\end{center}
%
The redirection with prefix
|\childdocforwardprefix[|\textit{prefix}|]{|\textit{dest}|}|
is accomplished by:
%
\begin{center}
\begin{tabular}{l}
|{\edef\jobname{\scantokens\expandafter{\jobname\noexpand}}|\\
|\def\redirectjob |\textit{prefix}|#1~~~{\gdef\jobname{|\textit{dest}|#1}}|\\
|\expandafter\redirectjob\jobname~~~}\input{\jobname}|
\end{tabular}
\end{center}

In an alternative approach,
child documents can be compiled by a specific command line
without additional code or specific definitions:
%
\begin{center}
|... -jobname "|\textit{target}|" "|[\textit{flags}]%
|\includeonly{|\textit{dest}|}\input{|\textit{main}|}"|
\end{center}
%

%%%%%%%%%%%%%%%%%%%%%%%%%%%%%%%%%%%%%%%%%%%%%%%%%%%%%%%%%%%%%%%%%%%%%%%%%%%%%%%%
%%%%%%%%%%%%%%%%%%%%%%%%%%%%%%%%%%%%%%%%%%%%%%%%%%%%%%%%%%%%%%%%%%%%%%%%%%%%%%%%
\section{Information}

%%%%%%%%%%%%%%%%%%%%%%%%%%%%%%%%%%%%%%%%%%%%%%%%%%%%%%%%%%%%%%%%%%%%%%%%%%%%%%%%
\subsection{Copyright}

Copyright \copyright{} 2017--2018 Niklas Beisert

This work may be distributed and/or modified under the
conditions of the \LaTeX{} Project Public License, either version 1.3
of this license or (at your option) any later version.
The latest version of this license is in
  \url{http://www.latex-project.org/lppl.txt}
and version 1.3 or later is part of all distributions of \LaTeX{}
version 2005/12/01 or later.

This work has the LPPL maintenance status `maintained'.

The Current Maintainer of this work is Niklas Beisert.

This work consists of the files |README.txt|, |childdoc.ins| and |childdoc.dtx|
as well as the derived files |childdoc.def|, |cdocsamp.tex|
with |cdocsch1.tex|, |cdocsch2.tex|, |cdocspt3.tex|, |cdocspt4.tex|,
|cdocsdrf.tex|, |cdocsfn1.tex|, |cdocsfn2.tex|
as well as |childdoc.pdf|.

%%%%%%%%%%%%%%%%%%%%%%%%%%%%%%%%%%%%%%%%%%%%%%%%%%%%%%%%%%%%%%%%%%%%%%%%%%%%%%%%
\subsection{Files and Installation}

The package consists of the files:
%
\begin{center}
\begin{tabular}{ll}
    |README.txt|   & readme file \\
    |childdoc.ins| & installation file \\
    |childdoc.dtx| & source file \\
    |childdoc.def| & definition file \\
    |cdocsamp.tex| & sample main file \\
    |cdocsch1.tex| & sample include file \\
    |cdocsch2.tex| & sample include file \\
    |cdocspt3.tex| & sample part file \\
    |cdocspt4.tex| & sample part file \\
    |cdocsdrf.tex| & sample redirection file \\
    |cdocsfn1.tex| & sample redirection file \\
    |cdocsfn2.tex| & sample redirection file \\
    |childdoc.pdf| & manual
\end{tabular}
\end{center}
%
The distribution consists of the files
|README.txt|, |childdoc.ins| and |childdoc.dtx|.
%
\begin{itemize}
\item
Run (pdf)\LaTeX{} on |childdoc.dtx|
to compile the manual |childdoc.pdf| (this file).
\item
Run \LaTeX{} on |childdoc.ins| to create the definitions file |childdoc.def|
and the sample |cdocsamp.tex| with include files
|cdocsch1.tex|, |cdocsch2.tex|, |cdocspt3.tex|, |cdocspt4.tex|,
|cdocsdrf.tex|, |cdocsfn1.tex|, |cdocsfn2.tex|.
Then copy the file |childdoc.def| to an appropriate directory of your \LaTeX{}
distribution, e.g.\ \textit{texmf-root}|/tex/latex/childdoc|.
\end{itemize}

%%%%%%%%%%%%%%%%%%%%%%%%%%%%%%%%%%%%%%%%%%%%%%%%%%%%%%%%%%%%%%%%%%%%%%%%%%%%%%%%
\subsection{Related CTAN Packages}

There are several other packages which offer a similar functionality:
%
\begin{itemize}
\item
The packages
\href{http://ctan.org/pkg/docmute}{\textsf{docmute}},
\href{http://ctan.org/pkg/includex}{\textsf{includex}} and
\href{http://ctan.org/pkg/standalone}{\textsf{standalone}}
provide commands to include only the document body of
a child file thus allowing both files to be compiled individually.
\item
The packages \href{http://ctan.org/pkg/subdocs}{\textsf{subdocs}}
and \href{http://ctan.org/pkg/subfiles}{\textsf{subfiles}}
provide structures in which the main and child documents can be
encapsulated and allowing them to be compiled individually.
The inclusion mechanism is different from the conventional |\include|.
\item
The package \href{http://ctan.org/pkg/combine}{\textsf{combine}}
is an elaborate solution to combine several documents into one.
\end{itemize}
%
See also the CTAN topic \href{http://ctan.org/topic/subdocs}{\textsf{subdocs}}
for further related packages.
The present package differs from the above solutions in that
a document structure constructed with the conventional |\include| mechanism
just needs two extra commands at the top of every file
such that all constituent files can be compiled individually.

%%%%%%%%%%%%%%%%%%%%%%%%%%%%%%%%%%%%%%%%%%%%%%%%%%%%%%%%%%%%%%%%%%%%%%%%%%%%%%%%
%\subsection{Feature Suggestions}
%
%The following is a list of features which may be useful for future
%versions of this package:
%%
%\begin{itemize}
%\item
%\ldots
%\end{itemize}

%%%%%%%%%%%%%%%%%%%%%%%%%%%%%%%%%%%%%%%%%%%%%%%%%%%%%%%%%%%%%%%%%%%%%%%%%%%%%%%%
\subsection{Revision History}

%%%%%%%%%%%%%%%%%%%%%%%%%%%%%%%%%%%%%%%%
\paragraph{v2.0:} 2018/12/30

\begin{itemize}
\item
immediate forward processing
\item
added |\childdocby| mechanism
\item
manual restructured
\end{itemize}

%%%%%%%%%%%%%%%%%%%%%%%%%%%%%%%%%%%%%%%%
\paragraph{v1.6:} 2018/01/17

\begin{itemize}
\item
application for development of include files
\item
corrections to manual
\end{itemize}

%%%%%%%%%%%%%%%%%%%%%%%%%%%%%%%%%%%%%%%%
\paragraph{v1.5:} 2017/05/21

\begin{itemize}
\item
more complete structuring introduced
\item
|\childdocof| introduced
\item
|\childdoc| renamed to |\childdocmain|
\item
|\childredirect| renamed to |\childdocforward| and |\childdocforwardprefix|
and functionality expanded
\end{itemize}

%%%%%%%%%%%%%%%%%%%%%%%%%%%%%%%%%%%%%%%%
\paragraph{v1.0:} 2017/04/27

\begin{itemize}
\item
manual and install package
\item
first version published on CTAN
\end{itemize}

%%%%%%%%%%%%%%%%%%%%%%%%%%%%%%%%%%%%%%%%
\paragraph{v0.6:} 2017/04/26

\begin{itemize}
\item
redirection mechanism added
\end{itemize}

%%%%%%%%%%%%%%%%%%%%%%%%%%%%%%%%%%%%%%%%
\paragraph{v0.5:} 2017/04/26

\begin{itemize}
\item
functionality in definition file
\end{itemize}


%%%%%%%%%%%%%%%%%%%%%%%%%%%%%%%%%%%%%%%%%%%%%%%%%%%%%%%%%%%%%%%%%%%%%%%%%%%%%%%%
%%%%%%%%%%%%%%%%%%%%%%%%%%%%%%%%%%%%%%%%%%%%%%%%%%%%%%%%%%%%%%%%%%%%%%%%%%%%%%%%
%%%%%%%%%%%%%%%%%%%%%%%%%%%%%%%%%%%%%%%%%%%%%%%%%%%%%%%%%%%%%%%%%%%%%%%%%%%%%%%%
\appendix

\settowidth\MacroIndent{\rmfamily\scriptsize 000\ }

 \DocInput{childdoc.dtx}

\end{document}
%</driver>
% \fi
%
% %%%%%%%%%%%%%%%%%%%%%%%%%%%%%%%%%%%%%%%%%%%%%%%%%%%%%%%%%%%%%%%%%%%%%%%%%%%%%%
% %%%%%%%%%%%%%%%%%%%%%%%%%%%%%%%%%%%%%%%%%%%%%%%%%%%%%%%%%%%%%%%%%%%%%%%%%%%%%%
% \section{Sample}
%\iffalse
%<*samplemain>
%\fi
%
% The following presents a sample document
% with two chapters, two parts, a title page,
% a compile flag as well as three forwarding files to set the flag.
% It consists of eight |.tex| files:
% \begin{center}
% \begin{tabular}{ll}
% |cdocsamp.tex|&main file\\
% |cdocsch1.tex|&include file for chapter 1\\
% |cdocsch2.tex|&include file for chapter 2\\
% |cdocspt3.tex|&include file for part 3\\
% |cdocspt4.tex|&include file for part 4\\
% |cdocsdrf.tex|&forwarding file for main file in draft mode\\
% |cdocsfi1.tex|&forwarding file for final version of chapter 1\\
% |cdocsfi2.tex|&forwarding file for final version of chapter 2\\
% \end{tabular}
% \end{center}
% Each of the eight files can be compiled directly by the \LaTeX{} compiler.
%
% %%%%%%%%%%%%%%%%%%%%%%%%%%%%%%%%%%%%%%
% \paragraph{Main File.}
%
% The main file is called |cdocsamp.tex|.
%
% Load the \textsf{childdoc} definitions and
% declare the filename for the main document:
%    \begin{macrocode}
\input{childdoc.def}
\childdocmain{}
%    \end{macrocode}

% Optional override for |\version| flag:
%    \begin{macrocode}
%%\ifchilddoc\else\providecommand{\version}{draft}\fi
%    \end{macrocode}

% Define the default values for the |\version| flag
% (|final| for the main file and |draft| for childs):
%    \begin{macrocode}
\ifchilddoc
\providecommand{\version}{draft}
\else
\providecommand{\version}{final}
\fi
%    \end{macrocode}

% Load the standard document class:
%    \begin{macrocode}
\documentclass[12pt]{article}
%    \end{macrocode}

% Start the document body:
%    \begin{macrocode}
\begin{document}
%    \end{macrocode}

% Declare a title page.
% Print title, part of document being processed and version flag:
%    \begin{macrocode}
\addtocounter{page}{-1}
\begin{center}
{\LARGE\bfseries{}childdoc example\par}
\vspace{1cm}
\ifchilddoc
\ifchilddocmanual part\else chapter\fi:
`\childdocname' of `\childdocjob'\par
\else
main document: `\childdocjob'\par
\fi
version: \version\par
\end{center}
\newpage
%    \end{macrocode}

% Manually include selected file,
% otherwise process as usual:
%    \begin{macrocode}
\ifchilddocmanual
\section*{part `\childdocname'}
\input{\childdocname}
\else
%    \end{macrocode}

% Include the two chapters:
%    \begin{macrocode}
\include{cdocsch1}
\include{cdocsch2}
%    \end{macrocode}

% Include the two parts unless only chapters should be displayed:
%    \begin{macrocode}
\ifchilddoc\else
\section{part three}
\input{cdocspt3}
\section{part four}
\input{cdocspt4}
\fi
%    \end{macrocode}

% Process as usual until here:
%    \begin{macrocode}
\fi
%    \end{macrocode}

% End of document body:
%    \begin{macrocode}
\end{document}
%    \end{macrocode}
%\iffalse
%</samplemain>
%\fi
%
% %%%%%%%%%%%%%%%%%%%%%%%%%%%%%%%%%%%%%%
% \paragraph{Chapter Include Files.}
%
% The include files are called |cdocsch1.tex| and |cdocsch2.tex|.
%
%\iffalse
%<*samplechap1|samplechap2>
%\fi

% Optional override for |\version| flag:
%    \begin{macrocode}
%%\providecommand{\version}{final}
%    \end{macrocode}

% Include the main document:
%    \begin{macrocode}
\input{childdoc.def}
\childdocof{cdocsamp}
%    \end{macrocode}

%\iffalse
%</samplechap1|samplechap2>
%\fi
%
%\iffalse
%<*samplechap1>
%\fi
% Some text for chapter 1:
%    \begin{macrocode}
\section{one}
some text in chapter one
%    \end{macrocode}

%\iffalse
%</samplechap1>
%\fi
% Some text for chapter 2:
%\iffalse
%<*samplechap2>
%\fi
%    \begin{macrocode}
\section{two}
more text in chapter two
%    \end{macrocode}

%\iffalse
%</samplechap2>
%\fi
%
% %%%%%%%%%%%%%%%%%%%%%%%%%%%%%%%%%%%%%%
% \paragraph{Part Include Files.}
%
% The include files are called |cdocspt3.tex| and |cdocspt4.tex|.
%
%\iffalse
%<*samplepart3|samplepart4>
%\fi

% Optional override for |\version| flag:
%    \begin{macrocode}
%%\providecommand{\version}{final}
%    \end{macrocode}

% Include the main document:
%    \begin{macrocode}
\input{childdoc.def}
\childdocby{cdocsamp}
%    \end{macrocode}

%\iffalse
%</samplepart3|samplepart4>
%\fi
%
%\iffalse
%<*samplepart3>
%\fi
% Some text for part 3:
%    \begin{macrocode}
some text in part three
%    \end{macrocode}

%\iffalse
%</samplepart3>
%\fi
% Some text for part 4:
%\iffalse
%<*samplepart4>
%\fi
%    \begin{macrocode}
more text in part four
%    \end{macrocode}

%\iffalse
%</samplepart4>
%\fi
%
% %%%%%%%%%%%%%%%%%%%%%%%%%%%%%%%%%%%%%%
% \paragraph{Forwarding for a Complete Draft.}
%
% The following forwarding file |cdocsdrf.tex|
% compiles the main document in draft mode:
%\iffalse
%<*sampledraft>
%\fi
%    \begin{macrocode}
\def\version{draft}
\input{childdoc.def}
\childdocforward{cdocsamp}
%    \end{macrocode}

%\iffalse
%</sampledraft>
%\fi
%
% %%%%%%%%%%%%%%%%%%%%%%%%%%%%%%%%%%%%%%
% \paragraph{Forwarding for Final Version of the Chapters.}
%
% The following forwarding files |cdocsfn1.tex| and |cdocsfn2.tex|
% (with identical content)
% compile the final versions of the child documents
% |cdocsch1.tex| and |cdocsch2.tex|, respectively:
%\iffalse
%<*samplefinal>
%\fi
%    \begin{macrocode}
\def\version{final}
\input{childdoc.def}
\childdocforwardprefix[cdocsamp]{cdocsfn}{cdocsch}
%    \end{macrocode}

%\iffalse
%</samplefinal>
%\fi
%
% %%%%%%%%%%%%%%%%%%%%%%%%%%%%%%%%%%%%%%
% \paragraph{Command Line Processing.}
%
% The following three command lines generate the output files
% |cdocscld|, |cdocscl1| and |cdocscl2|
% which should be identical to
% |cdocsdrf|, |cdocsch1| and |cdocsfn2|, respectively:
% \begin{center}
% \begin{tabular}{l}
% |latex -jobname cdocscld \|\\
% |  "\def\version{draft}\input{childdoc.def}\childdocforward{cdocsamp}"|\\
% |latex -jobname cdocscl1 \|\\
% |  "\input{childdoc.def}\childdocforward[cdocsamp]{cdocsch1}"|\\
% |latex -jobname cdocscl2 \|\\
% |  "\def\version{final}\input{childdoc.def}\childdocforward{cdocsch2}"|
% \end{tabular}
% \end{center}
% Note that the trailing backslash on each first line
% merely continues the input to the second line
% (for convenient cut ant paste).
% Furthermore, the command |latex| can be replaced by any
% of its alternative versions such as |pdflatex|.
%
% %%%%%%%%%%%%%%%%%%%%%%%%%%%%%%%%%%%%%%%%%%%%%%%%%%%%%%%%%%%%%%%%%%%%%%%%%%%%%%
% %%%%%%%%%%%%%%%%%%%%%%%%%%%%%%%%%%%%%%%%%%%%%%%%%%%%%%%%%%%%%%%%%%%%%%%%%%%%%%
% \section{Implementation}
%\iffalse
%<*package>
%\fi
%
% This section describes the definitions file |childdoc.def|.

% The definitions cannot be loaded using |\usepackage| or |\RequirePackage|
% which has a mechanism to prevent loading a style file more than once.
% When loading the definitions by means of |\input|
% multiple instances have to be prevented manually:
%\iffalse
%This code needs to be before the `\ProvidesFile' directive
%which is defined at the beginning of this file.
%Therefore it is also placed there and commented out here.
%</package>
%<*discard>
%\fi
%    \begin{macrocode}
\ifdefined\childdocmain\endinput\fi
%    \end{macrocode}
%\iffalse
%</discard>
%<*package>
%\fi
%
% \macro{\ifchilddoc}
% \macro{\ifchilddocmanual}
% The conditional |\ifchilddoc| tells whether a
% child (true) or main (false) document is being compiled.
% The conditional |\ifchilddocmanual| tells whether
% the |\includeonly| mechanism is used (false) or
% the selection of child files must be performed manually (true).
% The definitions initialise to false:
%    \begin{macrocode}
\newif\ifchilddoc
\newif\ifchilddocmanual
%    \end{macrocode}

% \macro{\childdocname}
% \macro{\childdocjob}
% The macro |\childdocname| stores the name of the main document
% to be compiled. The macro |\childdocjob| stores the name of
% the document on which the \LaTeX{} compiler was originally invoked.
% The content of |\jobname| cannot be compared
% to filenames specified in the source due to different catcodes.
% The following code rescans |\jobname|, stores the result
% in |\childdocname| and saves a copy in |\childdocjob|:
%    \begin{macrocode}
\edef\childdocname{\scantokens\expandafter{\jobname\noexpand}}
\let\childdocjob\childdocname
%    \end{macrocode}

% \macro{\childdocdisable}
% The macro |\childdocdisable| prevents the main file
% from being processed more than once.
% At this stage, the main document command |\childdocmain|
% is assumed to be called once again where it should do nothing.
% Any subsequent call to it should prevent
% a secondary processing of the main document
% It overwrites the forwarding commands
% |\childdocof| and |\childdocforward|
% with empty macros to prevent further inclusions of the main document:
%    \begin{macrocode}
\newcommand{\childdocdisable}
{
  \renewcommand{\childdocmain}[1]{\renewcommand{\childdocmain}[1]{\endinput}}
  \renewcommand{\childdocof}[1]{}
  \renewcommand{\childdocby}[2][]{}
  \renewcommand{\childdocforward}[2][]{}
  \renewcommand{\childdocdisable}{}
}
%    \end{macrocode}

% \macro{\childdocmain}
% The macro |\childdocmain| is to be called at the top of the main file
% with nothing or the main filename (without extension) as argument.
% First, it breaks loops.
% If the argument is not empty and does not match |\childdocname|
% (which is set by the first inclusion of |childdoc.def|),
% |\ifchilddoc| is set to true, |\includeonly| is applied to the child file
% and |\jobname| is set to the main file
% (for proper handling of |.aux| files):
%    \begin{macrocode}
\newcommand{\childdocmain}[1]
{
  \childdocdisable\childdocmain{}
  \if?#1?\else
    \begingroup
      \def\childdoctmp{#1}
      \ifx\childdoctmp\childdocname
        \def\childdoctmp{}
      \else
        \def\childdoctmp
        {
          \childdoctrue
          \includeonly{\childdocname}
          \def\childdocjob{#1}
          \def\jobname{#1}
        }
      \fi
      \expandafter
    \endgroup
    \childdoctmp
  \fi
}
%    \end{macrocode}

% \macro{\childdocof}
% The command |\childdocof| redirects
% compilation to the main file |#1|.
%    \begin{macrocode}
\newcommand{\childdocof}[1]
{
  \childdocdisable
  \childdoctrue
  \includeonly{\childdocname}
  \def\jobname{#1}
  \def\childdocjob{#1}
  \input{#1}
}
%    \end{macrocode}

% \macro{\childdocby}
% The command |\childdocby| ....
%    \begin{macrocode}
\newcommand{\childdocby}[2][]
{
  \childdocdisable
  \childdoctrue
  \childdocmanualtrue
  \if?#1?\else
    \def\jobname{#2}
  \fi
  \def\childdocjob{#2}
  \input{#2}
  \endinput
}
%    \end{macrocode}

% \macro{\childdocforward}
% The command |\childdocforward| redirects
% compilation to the main file or
% (if the optional argument is given) a child file.
% Parameters are set as if the main file
% or a child file starting with |\childdocof| was compiled.
% Then compilation is handed over to the main file:
%    \begin{macrocode}
\newcommand{\childdocforward}[2][]
{
  \begingroup
    \if?#1?
      \def\childdoctmp
      {
        \def\childdocname{#2}
        \def\childdocjob{#2}
        \def\jobname{#2}
        \input{#2}
        \endinput
      }
    \else
      \def\childdoctmp
      {
        \childdocdisable
        \def\childdocname{#2}
        \childdoctrue
        \includeonly{#2}
        \def\childdocjob{#1}
        \def\jobname{#1}
        \input{#1}
        \endinput
      }
    \fi
    \expandafter
  \endgroup
  \childdoctmp
}
%    \end{macrocode}

% \macro{\childdocforwardprefix}
% The command |\childdocforwardprefix| redirects
% compilation to the main or a child file by means of a pattern.
% The prefix |#1| in the current filename is replaced by |#2|
% and the suffix of the current filename is kept
% (it is assumed that the filename does not contain the substring `|~~~|'
% which is used as a delimiter).
% Compilation is handed over to the new file by |\childdocforward|:
%    \begin{macrocode}
\newcommand{\childdocforwardprefix}[3][]
{
  \begingroup
    \def\childdocextract #2##1~~~{\def\childdoctmp{\childdocforward[#1]{#3##1}}}
    \expandafter\childdocextract\childdocname~~~
    \expandafter
  \endgroup
  \childdoctmp
}
%    \end{macrocode}

% \macro{\childdoc}
% The deprecated macro |\childdoc| is a legacy version of |\childdocmain|:
%    \begin{macrocode}
\newcommand{\childdoc}{\childdocmain}
%    \end{macrocode}

% \macro{\childdocredirect}
% The deprecated macro |\childdocredirect| is a legacy version
% of |\childdocforward| and |\childdocforwardprefix|:
%    \begin{macrocode}
\newcommand{\childdocredirect}[2][]
{
  \begingroup
    \if?#1?
      \def\childdoctmp{\childdocforward{#2}}
    \else
      \def\childdoctmp{\childdocforwardprefix{#1}{#2}}
    \fi
    \expandafter
  \endgroup
  \childdoctmp
}
%    \end{macrocode}

%\iffalse
%</package>
%\fi
%
\endinput
|\\
|\childdocof{|\textit{main}|}|\\
\end{tabular}
\end{center}
at the top of every child file \textit{child}
which is included by |\include{|\textit{child}|}|
from within the main file
(or at least for those files to be compiled individually).
The argument \textit{main} must be the filename of the main file.

There are a couple of
considerations in setting up the main and child documents:

%%%%%%%%%%%%%%%%%%%%%%%%%%%%%%%%%%%%%%%%
\paragraph{Restrictions.}

Please note the following restrictions:
\begin{itemize}
\item
|\childdocmain| must be called with one argument \textit{main}
to ensure compatibility with earlier version of the package.
It must either be empty (|\childdocmain{}|)
or precisely match the filename of the main file in which it is specified.
See \secref{sec:detection} for further information.
\item
The filename \textit{main} must be specified without the |.tex| extension.
\item
The filename \textit{main} is case sensitive
(even in case-insensitive file systems)
due to internal string comparison.
\item
The argument \textit{main} should be fully expanded, it cannot be a macro.
\item
Subdirectories and special characters should be avoided in filenames.
\item
The command |\childdocmain{|\textit{main}|}| must be followed by a whitespace.
It should not be followed immediately by another command
or by a comment mark `|%|'.
This is because the \TeX{} parser reads the token immediately following
the argument of |\childdocmain| and puts it
at the beginning of every child section;
however, a white\-space is ignored.
\end{itemize}

%%%%%%%%%%%%%%%%%%%%%%%%%%%%%%%%%%%%%%%%
\paragraph{Content of Main File.}

It is advisable to place all content in the child files included by |\include|.
Any output contained in the main file will appear in all child documents
unless suppressed manually;
it cannot be suppressed automatically by the |\includeonly| directive
and thus should normally be avoided.
A method to include some content in the main file
by means of conditional processing is described in \secref{sec:conditional}.

%%%%%%%%%%%%%%%%%%%%%%%%%%%%%%%%%%%%%%%%
\paragraph{Page Numbering.}

When only a part of the document is compiled,
the appropriate numbering of pages
(as well as other status parameters)
is determined from the |.aux| files.
The latter contain information from previous passes.
However this information needs to propagate through
all intermediate child documents.
Therefore the page numbering in child documents may well
be inconsistent until the complete document is compiled at least once.

A useful (if unconventional) way to always ensure a consistent
page numbering is to restart the numbering in each child document
and denote the pages by `\textit{child}|.|\textit{page}'
where \textit{child} represents the chapter/section number of the child file.
This can be achieved by the command
|\numberwithin{page}{|\textit{child}|}|
of the \textsf{amsmath} package
where \textit{child} can be |chapter| or |section|
depending on the chosen structuring.
Alternatively, one can modify the macro |\thepage| appropriately
and reset the counter |page| at the start of each child file.

%%%%%%%%%%%%%%%%%%%%%%%%%%%%%%%%%%%%%%%%%%%%%%%%%%%%%%%%%%%%%%%%%%%%%%%%%%%%%%%%
\subsection{Conditional Processing}
\label{sec:conditional}

The package provides a mechanism to compile different versions
of a document. To customise the versions further some conditional processing
can come in handy to distinguish which version is being compiled.
The package provides two macros to describe the compilation context:

%%%%%%%%%%%%%%%%%%%%%%%%%%%%%%%%%%%%%%%%
\DescribeMacro{\ifchilddoc}
The conditional |\ifchilddoc| distinguishes between the compilation of
child documents and the main document:
%
\begin{center}
|\ifchilddoc |\textit{child-code}| |[|\||else |\textit{main-code}]| \||fi|
\end{center}

%%%%%%%%%%%%%%%%%%%%%%%%%%%%%%%%%%%%%%%%
\DescribeMacro{\childdocname}
\DescribeMacro{\childdocjob}
The macro |\childdocname| contains the filename (without extension)
of the main or child file being processed.
Note that |\childdocjob| will always contain the name of the main file.

%%%%%%%%%%%%%%%%%%%%%%%%%%%%%%%%%%%%%%%%
\paragraph{Title Page.}

Conditional processing can be used to include a title or banner page
in the main document when proper precautions are taken.
Importantly, the code in the main file should ensure that the page counter
(as well as other status parameters which are stored in the |.aux| files)
takes the same value after the conditional processing.
Otherwise the page numbers may take divergent values
depending on which part is compiled.

For example, a title page could be declared by:
%
\begin{center}
\begin{tabular}{l}
|\ifchilddoc\||else|\\
|\addtocounter{page}{-1}|\\
\textit{code for title page}\\
|\newpage|\\
|\||fi|
\end{tabular}
\end{center}
%
A banner page for the child documents can be generated by:
%
\begin{center}
\begin{tabular}{l}
|\ifchilddoc|\\
|\addtocounter{page}{-1}|\\
\textit{code for banner page}\\
|\newpage|\\
|\||fi|
\end{tabular}
\end{center}
%
Here one could write a message such as:
\begin{center}
|This is the part \childdocname{} of \childdocjob{}.|
\end{center}

%%%%%%%%%%%%%%%%%%%%%%%%%%%%%%%%%%%%%%%%%%%%%%%%%%%%%%%%%%%%%%%%%%%%%%%%%%%%%%%%
\subsection{Flags}
\label{sec:flags}

The package makes it easy to generate different versions
of the main or child documents.
To this end compilation flags can be defined
and assigned different default values.
They will be particularly useful in conjunction
with the forwarding mechanism described in \secref{sec:forward}.

For example, it may be useful to have a flag |\version|
which can be set to |draft| or |final|.
The document source will contain some conditional code
depending on the value of |\version|.
Suppose further, the flag should default to |final| for the main file
and to |draft| for child files
which is a natural assignment for editing the document.
This is achieved by placing the following code
in the preamble of the main document
(below the |\childdocmain| directive):
%
\begin{center}
\begin{tabular}{l}
|\ifchilddoc|\\
|\providecommand{\version}{draft}|\\
|\||else|\\
|\providecommand{\version}{final}|\\
|\||fi|
\end{tabular}
\end{center}
%
The definition by |\providecommand| makes sure
that previous definitions are not overwritten.
Further statements |\providecommand{\version}{...}|
can thus be added before the above code to override it.

For the main file, one might add a line
(between |\childdocmain| and the above block)
%
\begin{center}
|%\ifchilddoc\||else\providecommand{\version}{draft}\||fi|
\end{center}
%
which can be uncommented to produce a draft version.
Likewise one can add a line to the very top of a child file
(above the |\childdocof{|\textit{main}|}| directive)
%
\begin{center}
|%\providecommand{\version}{final}|
\end{center}
%
which can be uncommented to produce the final version of this child document.

%%%%%%%%%%%%%%%%%%%%%%%%%%%%%%%%%%%%%%%%%%%%%%%%%%%%%%%%%%%%%%%%%%%%%%%%%%%%%%%%
\subsection{Forwarding}
\label{sec:forward}

Different versions of the main or child documents
using compilation flags as described in \secref{sec:flags}
can be (permanently) stored in different files
for convenient compilation, viewing and distribution.
To this end, the package defines a command
to pass on compilation to a different file:

%%%%%%%%%%%%%%%%%%%%%%%%%%%%%%%%%%%%%%%%
\DescribeMacro{\childdocforward}
The command |\childdocforward| redirects processing to
another source file:
%
\begin{center}
\begin{tabular}{l}
|% \iffalse
%
% childdoc.dtx Copyright (C) 2017-2018 Niklas Beisert
%
% This work may be distributed and/or modified under the
% conditions of the LaTeX Project Public License, either version 1.3
% of this license or (at your option) any later version.
% The latest version of this license is in
%   http://www.latex-project.org/lppl.txt
% and version 1.3 or later is part of all distributions of LaTeX
% version 2005/12/01 or later.
%
% This work has the LPPL maintenance status `maintained'.
%
% The Current Maintainer of this work is Niklas Beisert.
%
% This work consists of the files childdoc.dtx and childdoc.ins
% and the derived files childdoc.def and cdocsamp.tex with
% cdocsch1.tex, cdocsch2.tex, cdocsdrf.tex, cdocsfn1.tex, cdocsfn2.tex.
%
%<package>\ifdefined\childdocmain\endinput\fi
%<package>\ProvidesFile{childdoc.def}[2018/12/30 v2.0 child document driver]
%<samplemain>\ProvidesFile{cdocsamp.tex}[2018/12/30 v2.0 sample for childdoc]
%<*driver>
%\ProvidesFile{childdoc.drv}[2018/12/30 v2.0 childdoc reference manual file]
\PassOptionsToClass{10pt,a4paper}{article}
\documentclass{ltxdoc}

\usepackage[margin=35mm]{geometry}
\usepackage{hyperref}
\usepackage{hyperxmp}
\usepackage[usenames]{color}

\hypersetup{colorlinks=true}
\hypersetup{pdfstartview=FitH}
\hypersetup{pdfpagemode=UseNone}
\hypersetup{pdfsource={}}
\hypersetup{pdflang={en-UK}}
\hypersetup{pdfcopyright={Copyright 2017-2018 Niklas Beisert.
  This work may be distributed and/or modified under the
  conditions of the LaTeX Project Public License, either version 1.3
  of this license or (at your option) any later version.}}
\hypersetup{pdflicenseurl={http://www.latex-project.org/lppl.txt}}
\hypersetup{pdfcontactaddress={ETH Zurich, ITP, HIT K,
  Wolfgang-Pauli-Strasse 27}}
\hypersetup{pdfcontactpostcode={8093}}
\hypersetup{pdfcontactcity={Zurich}}
\hypersetup{pdfcontactcountry={Switzerland}}
\hypersetup{pdfcontactemail={nbeisert@itp.phys.ethz.ch}}
\hypersetup{pdfcontacturl={http://people.phys.ethz.ch/\xmptilde nbeisert/}}

\newcommand{\secref}[1]{\hyperref[#1]{section \ref*{#1}}}

\parskip1ex
\parindent0pt
\let\olditemize\itemize
\def\itemize{\olditemize\parskip0pt}

\begin{document}

\title{The \textsf{childdoc} Package}
\hypersetup{pdftitle={The childdoc Package}}
\author{Niklas Beisert\\[2ex]
  Institut f\"ur Theoretische Physik\\
  Eidgen\"ossische Technische Hochschule Z\"urich\\
  Wolfgang-Pauli-Strasse 27, 8093 Z\"urich, Switzerland\\[1ex]
  \href{mailto:nbeisert@itp.phys.ethz.ch}
  {\texttt{nbeisert@itp.phys.ethz.ch}}}
\hypersetup{pdfauthor={Niklas Beisert}}
\hypersetup{pdfsubject={Manual for the LaTeX2e Package childdoc}}
\date{30 December 2018, \textsf{v2.0}}
\maketitle

\begin{abstract}\noindent
\textsf{childdoc} is a \LaTeXe{} package
that enables the direct compilation
of document sections included by |\include|
to individual files.
\end{abstract}

\begingroup
\parskip0ex
\tableofcontents
\endgroup

%%%%%%%%%%%%%%%%%%%%%%%%%%%%%%%%%%%%%%%%%%%%%%%%%%%%%%%%%%%%%%%%%%%%%%%%%%%%%%%%
%%%%%%%%%%%%%%%%%%%%%%%%%%%%%%%%%%%%%%%%%%%%%%%%%%%%%%%%%%%%%%%%%%%%%%%%%%%%%%%%
\section{Introduction}

\LaTeX{} provides a mechanism to structure a large document (such as a book)
into a main file and several child files (containing the chapters)
using the |\include| command.
This mechanism is beneficial for documents
which span hundreds of pages in order to
make the source file(s) more manageable.
Moreover, compilation can be restricted to
selected child files by means of the |\includeonly| command.
The latter feature can be used to reduce the compilation time while editing
(this was significantly more useful in the earlier days of \LaTeX{})
or to generate a smaller document which is easier to navigate.
Another application of |\includeonly| is to generate
documents consisting of selected parts of the complete document.

However, there are a few drawbacks of the plain |\include| mechanism:
\begin{itemize}
\item
The child files cannot be compiled on their own,
they can only be compiled via the main file.
A naive editing environment
(such as a text editor with an option
to have the current file processed by \LaTeX)
may require one to switch to the main file before compiling;
attempting to compile the child file produces errors.
\item
The main file must be modified (each time)
to adjust the |\includeonly| command
to the present needs. This easily leaves the main file in a messy state.
\item
The generated document will always carry the filename
of the main document. This is inconvenient if
several child files are to be compiled and
to be kept for distribution.
\end{itemize}

The present package provides a simple interface
to make child files individually compilable by \LaTeX{}.
Compiling a child file then has the same effect as compiling
the main file with an |\includeonly| command
to select the appropriate child.
Moreover the generated document will carry the name of the child
rather than the main file.
This resolves all three above issues.

This feature is meant to make the editing of books,
thesis documents and lecture notes somewhat more convenient.
However, the package can also be used efficiently for
composing a series of documents (such as exercise sheets)
which are typically distributed individually.
It then assists the author in generating the individual documents
(potentially in different versions)
as well as a document containing the collected series.
Another application is in developing style files
or other kinds of included material
where compilation of the style file could redirect
to a sample or test file.

%%%%%%%%%%%%%%%%%%%%%%%%%%%%%%%%%%%%%%%%%%%%%%%%%%%%%%%%%%%%%%%%%%%%%%%%%%%%%%%%
%%%%%%%%%%%%%%%%%%%%%%%%%%%%%%%%%%%%%%%%%%%%%%%%%%%%%%%%%%%%%%%%%%%%%%%%%%%%%%%%
\section{Usage}

First of all, the package \textsf{childdoc} is \emph{not} a standard
\LaTeXe{} |.sty| style file! Therefore it needs to be invoked in
a non-standard way.

%%%%%%%%%%%%%%%%%%%%%%%%%%%%%%%%%%%%%%%%%%%%%%%%%%%%%%%%%%%%%%%%%%%%%%%%%%%%%%%%
\subsection{Included Files}
\label{sec:include}

%%%%%%%%%%%%%%%%%%%%%%%%%%%%%%%%%%%%%%%%
\DescribeMacro{\childdocmain}
To use the package, add the commands
\begin{center}
\begin{tabular}{l}
|\input{childdoc.def}|\\
|\childdocmain{}|\\
\end{tabular}
\end{center}
at the very top of the main \LaTeX{} file,
in particular \emph{before} the |\documentclass| statement!
The argument of |\childdocmain| should be left empty
(but it must be present).

%%%%%%%%%%%%%%%%%%%%%%%%%%%%%%%%%%%%%%%%
\DescribeMacro{\childdocof}
Furthermore, add the commands
\begin{center}
\begin{tabular}{l}
|\input{childdoc.def}|\\
|\childdocof{|\textit{main}|}|\\
\end{tabular}
\end{center}
at the top of every child file \textit{child}
which is included by |\include{|\textit{child}|}|
from within the main file
(or at least for those files to be compiled individually).
The argument \textit{main} must be the filename of the main file.

There are a couple of
considerations in setting up the main and child documents:

%%%%%%%%%%%%%%%%%%%%%%%%%%%%%%%%%%%%%%%%
\paragraph{Restrictions.}

Please note the following restrictions:
\begin{itemize}
\item
|\childdocmain| must be called with one argument \textit{main}
to ensure compatibility with earlier version of the package.
It must either be empty (|\childdocmain{}|)
or precisely match the filename of the main file in which it is specified.
See \secref{sec:detection} for further information.
\item
The filename \textit{main} must be specified without the |.tex| extension.
\item
The filename \textit{main} is case sensitive
(even in case-insensitive file systems)
due to internal string comparison.
\item
The argument \textit{main} should be fully expanded, it cannot be a macro.
\item
Subdirectories and special characters should be avoided in filenames.
\item
The command |\childdocmain{|\textit{main}|}| must be followed by a whitespace.
It should not be followed immediately by another command
or by a comment mark `|%|'.
This is because the \TeX{} parser reads the token immediately following
the argument of |\childdocmain| and puts it
at the beginning of every child section;
however, a white\-space is ignored.
\end{itemize}

%%%%%%%%%%%%%%%%%%%%%%%%%%%%%%%%%%%%%%%%
\paragraph{Content of Main File.}

It is advisable to place all content in the child files included by |\include|.
Any output contained in the main file will appear in all child documents
unless suppressed manually;
it cannot be suppressed automatically by the |\includeonly| directive
and thus should normally be avoided.
A method to include some content in the main file
by means of conditional processing is described in \secref{sec:conditional}.

%%%%%%%%%%%%%%%%%%%%%%%%%%%%%%%%%%%%%%%%
\paragraph{Page Numbering.}

When only a part of the document is compiled,
the appropriate numbering of pages
(as well as other status parameters)
is determined from the |.aux| files.
The latter contain information from previous passes.
However this information needs to propagate through
all intermediate child documents.
Therefore the page numbering in child documents may well
be inconsistent until the complete document is compiled at least once.

A useful (if unconventional) way to always ensure a consistent
page numbering is to restart the numbering in each child document
and denote the pages by `\textit{child}|.|\textit{page}'
where \textit{child} represents the chapter/section number of the child file.
This can be achieved by the command
|\numberwithin{page}{|\textit{child}|}|
of the \textsf{amsmath} package
where \textit{child} can be |chapter| or |section|
depending on the chosen structuring.
Alternatively, one can modify the macro |\thepage| appropriately
and reset the counter |page| at the start of each child file.

%%%%%%%%%%%%%%%%%%%%%%%%%%%%%%%%%%%%%%%%%%%%%%%%%%%%%%%%%%%%%%%%%%%%%%%%%%%%%%%%
\subsection{Conditional Processing}
\label{sec:conditional}

The package provides a mechanism to compile different versions
of a document. To customise the versions further some conditional processing
can come in handy to distinguish which version is being compiled.
The package provides two macros to describe the compilation context:

%%%%%%%%%%%%%%%%%%%%%%%%%%%%%%%%%%%%%%%%
\DescribeMacro{\ifchilddoc}
The conditional |\ifchilddoc| distinguishes between the compilation of
child documents and the main document:
%
\begin{center}
|\ifchilddoc |\textit{child-code}| |[|\||else |\textit{main-code}]| \||fi|
\end{center}

%%%%%%%%%%%%%%%%%%%%%%%%%%%%%%%%%%%%%%%%
\DescribeMacro{\childdocname}
\DescribeMacro{\childdocjob}
The macro |\childdocname| contains the filename (without extension)
of the main or child file being processed.
Note that |\childdocjob| will always contain the name of the main file.

%%%%%%%%%%%%%%%%%%%%%%%%%%%%%%%%%%%%%%%%
\paragraph{Title Page.}

Conditional processing can be used to include a title or banner page
in the main document when proper precautions are taken.
Importantly, the code in the main file should ensure that the page counter
(as well as other status parameters which are stored in the |.aux| files)
takes the same value after the conditional processing.
Otherwise the page numbers may take divergent values
depending on which part is compiled.

For example, a title page could be declared by:
%
\begin{center}
\begin{tabular}{l}
|\ifchilddoc\||else|\\
|\addtocounter{page}{-1}|\\
\textit{code for title page}\\
|\newpage|\\
|\||fi|
\end{tabular}
\end{center}
%
A banner page for the child documents can be generated by:
%
\begin{center}
\begin{tabular}{l}
|\ifchilddoc|\\
|\addtocounter{page}{-1}|\\
\textit{code for banner page}\\
|\newpage|\\
|\||fi|
\end{tabular}
\end{center}
%
Here one could write a message such as:
\begin{center}
|This is the part \childdocname{} of \childdocjob{}.|
\end{center}

%%%%%%%%%%%%%%%%%%%%%%%%%%%%%%%%%%%%%%%%%%%%%%%%%%%%%%%%%%%%%%%%%%%%%%%%%%%%%%%%
\subsection{Flags}
\label{sec:flags}

The package makes it easy to generate different versions
of the main or child documents.
To this end compilation flags can be defined
and assigned different default values.
They will be particularly useful in conjunction
with the forwarding mechanism described in \secref{sec:forward}.

For example, it may be useful to have a flag |\version|
which can be set to |draft| or |final|.
The document source will contain some conditional code
depending on the value of |\version|.
Suppose further, the flag should default to |final| for the main file
and to |draft| for child files
which is a natural assignment for editing the document.
This is achieved by placing the following code
in the preamble of the main document
(below the |\childdocmain| directive):
%
\begin{center}
\begin{tabular}{l}
|\ifchilddoc|\\
|\providecommand{\version}{draft}|\\
|\||else|\\
|\providecommand{\version}{final}|\\
|\||fi|
\end{tabular}
\end{center}
%
The definition by |\providecommand| makes sure
that previous definitions are not overwritten.
Further statements |\providecommand{\version}{...}|
can thus be added before the above code to override it.

For the main file, one might add a line
(between |\childdocmain| and the above block)
%
\begin{center}
|%\ifchilddoc\||else\providecommand{\version}{draft}\||fi|
\end{center}
%
which can be uncommented to produce a draft version.
Likewise one can add a line to the very top of a child file
(above the |\childdocof{|\textit{main}|}| directive)
%
\begin{center}
|%\providecommand{\version}{final}|
\end{center}
%
which can be uncommented to produce the final version of this child document.

%%%%%%%%%%%%%%%%%%%%%%%%%%%%%%%%%%%%%%%%%%%%%%%%%%%%%%%%%%%%%%%%%%%%%%%%%%%%%%%%
\subsection{Forwarding}
\label{sec:forward}

Different versions of the main or child documents
using compilation flags as described in \secref{sec:flags}
can be (permanently) stored in different files
for convenient compilation, viewing and distribution.
To this end, the package defines a command
to pass on compilation to a different file:

%%%%%%%%%%%%%%%%%%%%%%%%%%%%%%%%%%%%%%%%
\DescribeMacro{\childdocforward}
The command |\childdocforward| redirects processing to
another source file:
%
\begin{center}
\begin{tabular}{l}
|\input{childdoc.def}|\\
|\childdocforward[|\textit{main}|]{|\textit{dest}|}|\\
\end{tabular}
\end{center}
%
The argument \textit{dest} is the destination file
(without extension).
It should be the main file or one of the child files.
Note that further \textsf{childdoc} directives
such as |\childdocof| and |\childdocforward|
in the indicated file will be processed in this form.
The optional argument \textit{main}
passes on directly to the main file \textit{main}
while pretending to compile the child \textit{dest}.
This form behaves as if \textit{dest}
issues |\childdocof{|\textit{main}|}| right away,
and no further \textsf{childdoc} directives will be processed.

%%%%%%%%%%%%%%%%%%%%%%%%%%%%%%%%%%%%%%%%
\DescribeMacro{\...prefix}
In the alternative form |\childdocforwardprefix|,
%
\begin{center}
\begin{tabular}{l}
|\input{childdoc.def}|\\
|\childdocforwardprefix[|\textit{main}|]{|\textit{prefix}|}{|\textit{dest}|}|
\end{tabular}
\end{center}
%
the destination file is determined by a pattern
depending on the current file:
To make this work, the current file must be called
`{\textit{prefix}\hspace{0.2em}\textit{suffix}}'
with \textit{prefix} matching precisely the argument.
Processing is then passed on to the file
`{\textit{dest}\hspace{0.2em}\textit{suffix}}'.
Surely, the same effect is achieved by
directly specifying the
argument `{\textit{dest}\hspace{0.2em}\textit{suffix}}'
in the first form.
However, that requires to set up a different file
for each child. With the alternative form of the command
all these files can have exactly the same content
which simplifies setting them up and maintaining them.

For example, the following file |draft.tex|
with a compilation flag |\version| as described in \secref{sec:flags}
compiles the main document as a draft:
%
\begin{center}
\begin{tabular}{l}
|\def\version{draft}|\\
|\input{childdoc.def}|\\
|\childdocforward{|\textit{main}|}|
\end{tabular}
\end{center}
%
Likewise, the following files |final|\textit{nn}|.tex|
compile the final version of the child document
|child|\textit{nn}|.tex|:
%
\begin{center}
\begin{tabular}{l}
|\def\version{final}|\\
|\input{childdoc.def}|\\
|\childdocforwardprefix{final}{child}|
\end{tabular}
\end{center}
%

Note that when several versions of a main file and/or of each child file
are to be generated, it may be convenient to set up a |Makefile| or
shell script to automatise the process.

%%%%%%%%%%%%%%%%%%%%%%%%%%%%%%%%%%%%%%%%%%%%%%%%%%%%%%%%%%%%%%%%%%%%%%%%%%%%%%%%
\subsection{Command Line Processing}
\label{sec:commandline}

The effect of redirection files can also be achieved by invoking
the \LaTeX{} compiler with a more elaborate command line.
Most conveniently this should be done as part
of a shell script or a |Makefile|.

When using \textsf{childdoc} in the main file, the following
command lines effectively perform a redirection
(note that depending on the shell being used,
backslashes may have to be doubled: `|\|' $\to$ `|\\|'):
%
\begin{center}
|... -jobname "|\textit{target}|" |\\|"|[\textit{flags}]%
|\input{childdoc.def}\childdocforward[|\textit{main}|]{|\textit{dest}|}"|
\end{center}
%
Here \textit{target} is the name of the output file,
\textit{main} is the name of the main file
and \textit{dest} is the name of the main or child file to be processed
(all filenames without extensions).
The optional argument \textit{main} can be omitted
if \textit{main} matches \textit{dest}.
Optionally, compilation \textit{flags} can be defined via |\def| commands.
This command line makes the \TeX{} engine believe
it is compiling the file \textit{target}
whose content is specified as the latter parameter.
The provided code then forwards the processing to
\textit{main} or \textit{dest} as described in \secref{sec:forward}.

%%%%%%%%%%%%%%%%%%%%%%%%%%%%%%%%%%%%%%%%%%%%%%%%%%%%%%%%%%%%%%%%%%%%%%%%%%%%%%%%
\subsection{Include by Input}
\label{sec:input}

Including child documents by |\include| has some restrictions by design.
Most notably, the content of a child document always occupies
its own set of pages; pages cannot be shared between child documents.
Usually, this behaviour makes perfect sense
because each child document contain an essential part of the document.
However, in some situations it may be desirable to compose
a document from a collection of parts
without having mandatory page breaks between then.
For this case, the package
provides a mechanism to include parts
by |\input| which can also be processed individually.
However, by construction this mechanism
requires manual handling of the content to be output.

%%%%%%%%%%%%%%%%%%%%%%%%%%%%%%%%%%%%%%%%
\DescribeMacro{\ifchilddocmanual}
The main file should be prepared as usual, see \secref{sec:include}.
However, the document body must make a distinction
between processing of an individual part and of the main document, e.g.:
%
\begin{center}
\begin{tabular}{l}
|\ifchilddocmanual|\\
|\input{\childdocname}|\\
|\||else|\\
\textit{document body with }|\input{|\textit{part}|}|\\
|\||fi|
\end{tabular}
\end{center}
%
The conditional |\ifchilddocmanual| is true whenever
a part to be included by |\input| is being compiled,
and the name of the part is stored in |\childdocname|.

%%%%%%%%%%%%%%%%%%%%%%%%%%%%%%%%%%%%%%%%
\DescribeMacro{\childdocby}
Each part to be included by |\input| should start with:
%
\begin{center}
\begin{tabular}{l}
|\input{childdoc.def}|\\
|\childdocby{|\textit{main}|}|\\
\end{tabular}
\end{center}
%
The directive |\childdocby| is similar to |\childdocof|
described in \secref{sec:include},
but the subsequent selection of content must be done manually.
To that end, both |\ifchilddoc| and |\ifchilddocmanual|
will be true upon processing of a part,
and the name of the part is stored in |\childdocname|.
Note that |\jobname| will be set to the filename of the current part
so that each part receives an individual |.aux| file
that does not interfere with the |.aux| file(s) of the main document.
This behaviour can be altered by the alternative form
|\childdocby[*]{|\textit{main}|}| (with a non-empty optional argument)
which uses the |.aux| file of the main document
by setting |\jobname| to \textit{main}.

%%%%%%%%%%%%%%%%%%%%%%%%%%%%%%%%%%%%%%%%%%%%%%%%%%%%%%%%%%%%%%%%%%%%%%%%%%%%%%%%
\subsection{Driver Development}
\label{sec:driver}

The \textsf{childdoc} mechanism can also be use for the development
of definition files such as \LaTeX{} styles or classes.
This case differs from the above setup with multiple parts
included by |\include| in that no |\includeonly| should be invoked.
This can be achieved by starting the include file
(before |\ProvidesPackage|) with:
%
\begin{center}
\begin{tabular}{l}
|\input{childdoc.def}|\\
|\childdocforward{|\textit{main}|}|\\
\end{tabular}
\end{center}
%
or alternatively with:
%
\begin{center}
\begin{tabular}{l}
|\input{childdoc.def}|\\
|\childdocby{|\textit{main}|}|\\
\end{tabular}
\end{center}
%
Both forms have slightly different effects as described above.
The main file is prepared as usual, see \secref{sec:include}.

%%%%%%%%%%%%%%%%%%%%%%%%%%%%%%%%%%%%%%%%%%%%%%%%%%%%%%%%%%%%%%%%%%%%%%%%%%%%%%%%
\subsection{Legacy Detection}
\label{sec:detection}

The directive |\childdocmain| in the main file can detect
whether the complete document or merely a child is to be compiled
even without using the directive |\childdocof|.
This method is deprecated because it is less robust
and there is no compelling reason to use it;
it is merely provided for backward compatibility
and it may be removed in future versions.

If the detection mechanism is to be used,
it is mandatory to correctly specify
the filename of the main file as the argument of |\childdocmain|:
%
\begin{center}
\begin{tabular}{l}
|\input{childdoc.def}|\\
|\childdocmain{|\textit{main}|}|\\
\end{tabular}
\end{center}
%
If |\jobname| does not match the argument \textit{main} of |\childdocmain|,
it is assumed that |\jobname| points to the child file to be compiled.
When using |\childdocmain| with the main file specified as argument,
it suffices to start a child file
with just |\input{|\textit{main}|}|
without loading of the package and using |\childdocof|.
If instead all processing is done
with the appropriate \textsf{childdoc} directives,
the argument of \textit{main} of |\childdocmain| can be empty.

An alternative version of the command line processing described
in \secref{sec:commandline} using the detection mechanism reads:
%
\begin{center}
|... -jobname "|\textit{target}|" "|[\textit{flags}]%
[|\def\jobname{|\textit{dest}|}|]|\input{|\textit{main}|}"|
\end{center}

%%%%%%%%%%%%%%%%%%%%%%%%%%%%%%%%%%%%%%%%%%%%%%%%%%%%%%%%%%%%%%%%%%%%%%%%%%%%%%%%
\subsection{Manual Code}
\label{sec:manual}

In case one cannot be certain whether the definitions file |childdoc.def|
is installed on the target \TeX{} distribution
and one prefers not to ship it,
it is conceivable to paste a few relevant commands into the sources.

To that end, drop all statements |\input{childdoc.def}|
and perform the replacements as outlined below.
Instead of |\childdocmain{|\textit{main}|}| add the following code
to the top of the main file:
%
\begin{center}
\begin{tabular}{l}
|\||ifdefined\childdocname\endinput\||fi\newif\ifchilddoc|\\
|\edef\childdocname{\scantokens\expandafter{\jobname\noexpand}}|\\
|\def\childdocmain{|\textit{main}|}\||ifx\childdocmain\childdocname\||else|\\
|\childdoctrue\includeonly{\childdocname}\let\jobname\childdocmain\||fi|\\
\end{tabular}
\end{center}
%
Instead of |\childdocof{|\textit{main}|}| just include the main file
at the top of each child file:
%
\begin{center}
|\input{|\textit{main}|}|
\end{center}
%
A simple redirection |\childdocforward{|\textit{dest}|}| is achieved by:
%
\begin{center}
|\def\jobname{|\textit{dest}|}\input{\jobname}|
\end{center}
%
The redirection with prefix
|\childdocforwardprefix[|\textit{prefix}|]{|\textit{dest}|}|
is accomplished by:
%
\begin{center}
\begin{tabular}{l}
|{\edef\jobname{\scantokens\expandafter{\jobname\noexpand}}|\\
|\def\redirectjob |\textit{prefix}|#1~~~{\gdef\jobname{|\textit{dest}|#1}}|\\
|\expandafter\redirectjob\jobname~~~}\input{\jobname}|
\end{tabular}
\end{center}

In an alternative approach,
child documents can be compiled by a specific command line
without additional code or specific definitions:
%
\begin{center}
|... -jobname "|\textit{target}|" "|[\textit{flags}]%
|\includeonly{|\textit{dest}|}\input{|\textit{main}|}"|
\end{center}
%

%%%%%%%%%%%%%%%%%%%%%%%%%%%%%%%%%%%%%%%%%%%%%%%%%%%%%%%%%%%%%%%%%%%%%%%%%%%%%%%%
%%%%%%%%%%%%%%%%%%%%%%%%%%%%%%%%%%%%%%%%%%%%%%%%%%%%%%%%%%%%%%%%%%%%%%%%%%%%%%%%
\section{Information}

%%%%%%%%%%%%%%%%%%%%%%%%%%%%%%%%%%%%%%%%%%%%%%%%%%%%%%%%%%%%%%%%%%%%%%%%%%%%%%%%
\subsection{Copyright}

Copyright \copyright{} 2017--2018 Niklas Beisert

This work may be distributed and/or modified under the
conditions of the \LaTeX{} Project Public License, either version 1.3
of this license or (at your option) any later version.
The latest version of this license is in
  \url{http://www.latex-project.org/lppl.txt}
and version 1.3 or later is part of all distributions of \LaTeX{}
version 2005/12/01 or later.

This work has the LPPL maintenance status `maintained'.

The Current Maintainer of this work is Niklas Beisert.

This work consists of the files |README.txt|, |childdoc.ins| and |childdoc.dtx|
as well as the derived files |childdoc.def|, |cdocsamp.tex|
with |cdocsch1.tex|, |cdocsch2.tex|, |cdocspt3.tex|, |cdocspt4.tex|,
|cdocsdrf.tex|, |cdocsfn1.tex|, |cdocsfn2.tex|
as well as |childdoc.pdf|.

%%%%%%%%%%%%%%%%%%%%%%%%%%%%%%%%%%%%%%%%%%%%%%%%%%%%%%%%%%%%%%%%%%%%%%%%%%%%%%%%
\subsection{Files and Installation}

The package consists of the files:
%
\begin{center}
\begin{tabular}{ll}
    |README.txt|   & readme file \\
    |childdoc.ins| & installation file \\
    |childdoc.dtx| & source file \\
    |childdoc.def| & definition file \\
    |cdocsamp.tex| & sample main file \\
    |cdocsch1.tex| & sample include file \\
    |cdocsch2.tex| & sample include file \\
    |cdocspt3.tex| & sample part file \\
    |cdocspt4.tex| & sample part file \\
    |cdocsdrf.tex| & sample redirection file \\
    |cdocsfn1.tex| & sample redirection file \\
    |cdocsfn2.tex| & sample redirection file \\
    |childdoc.pdf| & manual
\end{tabular}
\end{center}
%
The distribution consists of the files
|README.txt|, |childdoc.ins| and |childdoc.dtx|.
%
\begin{itemize}
\item
Run (pdf)\LaTeX{} on |childdoc.dtx|
to compile the manual |childdoc.pdf| (this file).
\item
Run \LaTeX{} on |childdoc.ins| to create the definitions file |childdoc.def|
and the sample |cdocsamp.tex| with include files
|cdocsch1.tex|, |cdocsch2.tex|, |cdocspt3.tex|, |cdocspt4.tex|,
|cdocsdrf.tex|, |cdocsfn1.tex|, |cdocsfn2.tex|.
Then copy the file |childdoc.def| to an appropriate directory of your \LaTeX{}
distribution, e.g.\ \textit{texmf-root}|/tex/latex/childdoc|.
\end{itemize}

%%%%%%%%%%%%%%%%%%%%%%%%%%%%%%%%%%%%%%%%%%%%%%%%%%%%%%%%%%%%%%%%%%%%%%%%%%%%%%%%
\subsection{Related CTAN Packages}

There are several other packages which offer a similar functionality:
%
\begin{itemize}
\item
The packages
\href{http://ctan.org/pkg/docmute}{\textsf{docmute}},
\href{http://ctan.org/pkg/includex}{\textsf{includex}} and
\href{http://ctan.org/pkg/standalone}{\textsf{standalone}}
provide commands to include only the document body of
a child file thus allowing both files to be compiled individually.
\item
The packages \href{http://ctan.org/pkg/subdocs}{\textsf{subdocs}}
and \href{http://ctan.org/pkg/subfiles}{\textsf{subfiles}}
provide structures in which the main and child documents can be
encapsulated and allowing them to be compiled individually.
The inclusion mechanism is different from the conventional |\include|.
\item
The package \href{http://ctan.org/pkg/combine}{\textsf{combine}}
is an elaborate solution to combine several documents into one.
\end{itemize}
%
See also the CTAN topic \href{http://ctan.org/topic/subdocs}{\textsf{subdocs}}
for further related packages.
The present package differs from the above solutions in that
a document structure constructed with the conventional |\include| mechanism
just needs two extra commands at the top of every file
such that all constituent files can be compiled individually.

%%%%%%%%%%%%%%%%%%%%%%%%%%%%%%%%%%%%%%%%%%%%%%%%%%%%%%%%%%%%%%%%%%%%%%%%%%%%%%%%
%\subsection{Feature Suggestions}
%
%The following is a list of features which may be useful for future
%versions of this package:
%%
%\begin{itemize}
%\item
%\ldots
%\end{itemize}

%%%%%%%%%%%%%%%%%%%%%%%%%%%%%%%%%%%%%%%%%%%%%%%%%%%%%%%%%%%%%%%%%%%%%%%%%%%%%%%%
\subsection{Revision History}

%%%%%%%%%%%%%%%%%%%%%%%%%%%%%%%%%%%%%%%%
\paragraph{v2.0:} 2018/12/30

\begin{itemize}
\item
immediate forward processing
\item
added |\childdocby| mechanism
\item
manual restructured
\end{itemize}

%%%%%%%%%%%%%%%%%%%%%%%%%%%%%%%%%%%%%%%%
\paragraph{v1.6:} 2018/01/17

\begin{itemize}
\item
application for development of include files
\item
corrections to manual
\end{itemize}

%%%%%%%%%%%%%%%%%%%%%%%%%%%%%%%%%%%%%%%%
\paragraph{v1.5:} 2017/05/21

\begin{itemize}
\item
more complete structuring introduced
\item
|\childdocof| introduced
\item
|\childdoc| renamed to |\childdocmain|
\item
|\childredirect| renamed to |\childdocforward| and |\childdocforwardprefix|
and functionality expanded
\end{itemize}

%%%%%%%%%%%%%%%%%%%%%%%%%%%%%%%%%%%%%%%%
\paragraph{v1.0:} 2017/04/27

\begin{itemize}
\item
manual and install package
\item
first version published on CTAN
\end{itemize}

%%%%%%%%%%%%%%%%%%%%%%%%%%%%%%%%%%%%%%%%
\paragraph{v0.6:} 2017/04/26

\begin{itemize}
\item
redirection mechanism added
\end{itemize}

%%%%%%%%%%%%%%%%%%%%%%%%%%%%%%%%%%%%%%%%
\paragraph{v0.5:} 2017/04/26

\begin{itemize}
\item
functionality in definition file
\end{itemize}


%%%%%%%%%%%%%%%%%%%%%%%%%%%%%%%%%%%%%%%%%%%%%%%%%%%%%%%%%%%%%%%%%%%%%%%%%%%%%%%%
%%%%%%%%%%%%%%%%%%%%%%%%%%%%%%%%%%%%%%%%%%%%%%%%%%%%%%%%%%%%%%%%%%%%%%%%%%%%%%%%
%%%%%%%%%%%%%%%%%%%%%%%%%%%%%%%%%%%%%%%%%%%%%%%%%%%%%%%%%%%%%%%%%%%%%%%%%%%%%%%%
\appendix

\settowidth\MacroIndent{\rmfamily\scriptsize 000\ }

 \DocInput{childdoc.dtx}

\end{document}
%</driver>
% \fi
%
% %%%%%%%%%%%%%%%%%%%%%%%%%%%%%%%%%%%%%%%%%%%%%%%%%%%%%%%%%%%%%%%%%%%%%%%%%%%%%%
% %%%%%%%%%%%%%%%%%%%%%%%%%%%%%%%%%%%%%%%%%%%%%%%%%%%%%%%%%%%%%%%%%%%%%%%%%%%%%%
% \section{Sample}
%\iffalse
%<*samplemain>
%\fi
%
% The following presents a sample document
% with two chapters, two parts, a title page,
% a compile flag as well as three forwarding files to set the flag.
% It consists of eight |.tex| files:
% \begin{center}
% \begin{tabular}{ll}
% |cdocsamp.tex|&main file\\
% |cdocsch1.tex|&include file for chapter 1\\
% |cdocsch2.tex|&include file for chapter 2\\
% |cdocspt3.tex|&include file for part 3\\
% |cdocspt4.tex|&include file for part 4\\
% |cdocsdrf.tex|&forwarding file for main file in draft mode\\
% |cdocsfi1.tex|&forwarding file for final version of chapter 1\\
% |cdocsfi2.tex|&forwarding file for final version of chapter 2\\
% \end{tabular}
% \end{center}
% Each of the eight files can be compiled directly by the \LaTeX{} compiler.
%
% %%%%%%%%%%%%%%%%%%%%%%%%%%%%%%%%%%%%%%
% \paragraph{Main File.}
%
% The main file is called |cdocsamp.tex|.
%
% Load the \textsf{childdoc} definitions and
% declare the filename for the main document:
%    \begin{macrocode}
\input{childdoc.def}
\childdocmain{}
%    \end{macrocode}

% Optional override for |\version| flag:
%    \begin{macrocode}
%%\ifchilddoc\else\providecommand{\version}{draft}\fi
%    \end{macrocode}

% Define the default values for the |\version| flag
% (|final| for the main file and |draft| for childs):
%    \begin{macrocode}
\ifchilddoc
\providecommand{\version}{draft}
\else
\providecommand{\version}{final}
\fi
%    \end{macrocode}

% Load the standard document class:
%    \begin{macrocode}
\documentclass[12pt]{article}
%    \end{macrocode}

% Start the document body:
%    \begin{macrocode}
\begin{document}
%    \end{macrocode}

% Declare a title page.
% Print title, part of document being processed and version flag:
%    \begin{macrocode}
\addtocounter{page}{-1}
\begin{center}
{\LARGE\bfseries{}childdoc example\par}
\vspace{1cm}
\ifchilddoc
\ifchilddocmanual part\else chapter\fi:
`\childdocname' of `\childdocjob'\par
\else
main document: `\childdocjob'\par
\fi
version: \version\par
\end{center}
\newpage
%    \end{macrocode}

% Manually include selected file,
% otherwise process as usual:
%    \begin{macrocode}
\ifchilddocmanual
\section*{part `\childdocname'}
\input{\childdocname}
\else
%    \end{macrocode}

% Include the two chapters:
%    \begin{macrocode}
\include{cdocsch1}
\include{cdocsch2}
%    \end{macrocode}

% Include the two parts unless only chapters should be displayed:
%    \begin{macrocode}
\ifchilddoc\else
\section{part three}
\input{cdocspt3}
\section{part four}
\input{cdocspt4}
\fi
%    \end{macrocode}

% Process as usual until here:
%    \begin{macrocode}
\fi
%    \end{macrocode}

% End of document body:
%    \begin{macrocode}
\end{document}
%    \end{macrocode}
%\iffalse
%</samplemain>
%\fi
%
% %%%%%%%%%%%%%%%%%%%%%%%%%%%%%%%%%%%%%%
% \paragraph{Chapter Include Files.}
%
% The include files are called |cdocsch1.tex| and |cdocsch2.tex|.
%
%\iffalse
%<*samplechap1|samplechap2>
%\fi

% Optional override for |\version| flag:
%    \begin{macrocode}
%%\providecommand{\version}{final}
%    \end{macrocode}

% Include the main document:
%    \begin{macrocode}
\input{childdoc.def}
\childdocof{cdocsamp}
%    \end{macrocode}

%\iffalse
%</samplechap1|samplechap2>
%\fi
%
%\iffalse
%<*samplechap1>
%\fi
% Some text for chapter 1:
%    \begin{macrocode}
\section{one}
some text in chapter one
%    \end{macrocode}

%\iffalse
%</samplechap1>
%\fi
% Some text for chapter 2:
%\iffalse
%<*samplechap2>
%\fi
%    \begin{macrocode}
\section{two}
more text in chapter two
%    \end{macrocode}

%\iffalse
%</samplechap2>
%\fi
%
% %%%%%%%%%%%%%%%%%%%%%%%%%%%%%%%%%%%%%%
% \paragraph{Part Include Files.}
%
% The include files are called |cdocspt3.tex| and |cdocspt4.tex|.
%
%\iffalse
%<*samplepart3|samplepart4>
%\fi

% Optional override for |\version| flag:
%    \begin{macrocode}
%%\providecommand{\version}{final}
%    \end{macrocode}

% Include the main document:
%    \begin{macrocode}
\input{childdoc.def}
\childdocby{cdocsamp}
%    \end{macrocode}

%\iffalse
%</samplepart3|samplepart4>
%\fi
%
%\iffalse
%<*samplepart3>
%\fi
% Some text for part 3:
%    \begin{macrocode}
some text in part three
%    \end{macrocode}

%\iffalse
%</samplepart3>
%\fi
% Some text for part 4:
%\iffalse
%<*samplepart4>
%\fi
%    \begin{macrocode}
more text in part four
%    \end{macrocode}

%\iffalse
%</samplepart4>
%\fi
%
% %%%%%%%%%%%%%%%%%%%%%%%%%%%%%%%%%%%%%%
% \paragraph{Forwarding for a Complete Draft.}
%
% The following forwarding file |cdocsdrf.tex|
% compiles the main document in draft mode:
%\iffalse
%<*sampledraft>
%\fi
%    \begin{macrocode}
\def\version{draft}
\input{childdoc.def}
\childdocforward{cdocsamp}
%    \end{macrocode}

%\iffalse
%</sampledraft>
%\fi
%
% %%%%%%%%%%%%%%%%%%%%%%%%%%%%%%%%%%%%%%
% \paragraph{Forwarding for Final Version of the Chapters.}
%
% The following forwarding files |cdocsfn1.tex| and |cdocsfn2.tex|
% (with identical content)
% compile the final versions of the child documents
% |cdocsch1.tex| and |cdocsch2.tex|, respectively:
%\iffalse
%<*samplefinal>
%\fi
%    \begin{macrocode}
\def\version{final}
\input{childdoc.def}
\childdocforwardprefix[cdocsamp]{cdocsfn}{cdocsch}
%    \end{macrocode}

%\iffalse
%</samplefinal>
%\fi
%
% %%%%%%%%%%%%%%%%%%%%%%%%%%%%%%%%%%%%%%
% \paragraph{Command Line Processing.}
%
% The following three command lines generate the output files
% |cdocscld|, |cdocscl1| and |cdocscl2|
% which should be identical to
% |cdocsdrf|, |cdocsch1| and |cdocsfn2|, respectively:
% \begin{center}
% \begin{tabular}{l}
% |latex -jobname cdocscld \|\\
% |  "\def\version{draft}\input{childdoc.def}\childdocforward{cdocsamp}"|\\
% |latex -jobname cdocscl1 \|\\
% |  "\input{childdoc.def}\childdocforward[cdocsamp]{cdocsch1}"|\\
% |latex -jobname cdocscl2 \|\\
% |  "\def\version{final}\input{childdoc.def}\childdocforward{cdocsch2}"|
% \end{tabular}
% \end{center}
% Note that the trailing backslash on each first line
% merely continues the input to the second line
% (for convenient cut ant paste).
% Furthermore, the command |latex| can be replaced by any
% of its alternative versions such as |pdflatex|.
%
% %%%%%%%%%%%%%%%%%%%%%%%%%%%%%%%%%%%%%%%%%%%%%%%%%%%%%%%%%%%%%%%%%%%%%%%%%%%%%%
% %%%%%%%%%%%%%%%%%%%%%%%%%%%%%%%%%%%%%%%%%%%%%%%%%%%%%%%%%%%%%%%%%%%%%%%%%%%%%%
% \section{Implementation}
%\iffalse
%<*package>
%\fi
%
% This section describes the definitions file |childdoc.def|.

% The definitions cannot be loaded using |\usepackage| or |\RequirePackage|
% which has a mechanism to prevent loading a style file more than once.
% When loading the definitions by means of |\input|
% multiple instances have to be prevented manually:
%\iffalse
%This code needs to be before the `\ProvidesFile' directive
%which is defined at the beginning of this file.
%Therefore it is also placed there and commented out here.
%</package>
%<*discard>
%\fi
%    \begin{macrocode}
\ifdefined\childdocmain\endinput\fi
%    \end{macrocode}
%\iffalse
%</discard>
%<*package>
%\fi
%
% \macro{\ifchilddoc}
% \macro{\ifchilddocmanual}
% The conditional |\ifchilddoc| tells whether a
% child (true) or main (false) document is being compiled.
% The conditional |\ifchilddocmanual| tells whether
% the |\includeonly| mechanism is used (false) or
% the selection of child files must be performed manually (true).
% The definitions initialise to false:
%    \begin{macrocode}
\newif\ifchilddoc
\newif\ifchilddocmanual
%    \end{macrocode}

% \macro{\childdocname}
% \macro{\childdocjob}
% The macro |\childdocname| stores the name of the main document
% to be compiled. The macro |\childdocjob| stores the name of
% the document on which the \LaTeX{} compiler was originally invoked.
% The content of |\jobname| cannot be compared
% to filenames specified in the source due to different catcodes.
% The following code rescans |\jobname|, stores the result
% in |\childdocname| and saves a copy in |\childdocjob|:
%    \begin{macrocode}
\edef\childdocname{\scantokens\expandafter{\jobname\noexpand}}
\let\childdocjob\childdocname
%    \end{macrocode}

% \macro{\childdocdisable}
% The macro |\childdocdisable| prevents the main file
% from being processed more than once.
% At this stage, the main document command |\childdocmain|
% is assumed to be called once again where it should do nothing.
% Any subsequent call to it should prevent
% a secondary processing of the main document
% It overwrites the forwarding commands
% |\childdocof| and |\childdocforward|
% with empty macros to prevent further inclusions of the main document:
%    \begin{macrocode}
\newcommand{\childdocdisable}
{
  \renewcommand{\childdocmain}[1]{\renewcommand{\childdocmain}[1]{\endinput}}
  \renewcommand{\childdocof}[1]{}
  \renewcommand{\childdocby}[2][]{}
  \renewcommand{\childdocforward}[2][]{}
  \renewcommand{\childdocdisable}{}
}
%    \end{macrocode}

% \macro{\childdocmain}
% The macro |\childdocmain| is to be called at the top of the main file
% with nothing or the main filename (without extension) as argument.
% First, it breaks loops.
% If the argument is not empty and does not match |\childdocname|
% (which is set by the first inclusion of |childdoc.def|),
% |\ifchilddoc| is set to true, |\includeonly| is applied to the child file
% and |\jobname| is set to the main file
% (for proper handling of |.aux| files):
%    \begin{macrocode}
\newcommand{\childdocmain}[1]
{
  \childdocdisable\childdocmain{}
  \if?#1?\else
    \begingroup
      \def\childdoctmp{#1}
      \ifx\childdoctmp\childdocname
        \def\childdoctmp{}
      \else
        \def\childdoctmp
        {
          \childdoctrue
          \includeonly{\childdocname}
          \def\childdocjob{#1}
          \def\jobname{#1}
        }
      \fi
      \expandafter
    \endgroup
    \childdoctmp
  \fi
}
%    \end{macrocode}

% \macro{\childdocof}
% The command |\childdocof| redirects
% compilation to the main file |#1|.
%    \begin{macrocode}
\newcommand{\childdocof}[1]
{
  \childdocdisable
  \childdoctrue
  \includeonly{\childdocname}
  \def\jobname{#1}
  \def\childdocjob{#1}
  \input{#1}
}
%    \end{macrocode}

% \macro{\childdocby}
% The command |\childdocby| ....
%    \begin{macrocode}
\newcommand{\childdocby}[2][]
{
  \childdocdisable
  \childdoctrue
  \childdocmanualtrue
  \if?#1?\else
    \def\jobname{#2}
  \fi
  \def\childdocjob{#2}
  \input{#2}
  \endinput
}
%    \end{macrocode}

% \macro{\childdocforward}
% The command |\childdocforward| redirects
% compilation to the main file or
% (if the optional argument is given) a child file.
% Parameters are set as if the main file
% or a child file starting with |\childdocof| was compiled.
% Then compilation is handed over to the main file:
%    \begin{macrocode}
\newcommand{\childdocforward}[2][]
{
  \begingroup
    \if?#1?
      \def\childdoctmp
      {
        \def\childdocname{#2}
        \def\childdocjob{#2}
        \def\jobname{#2}
        \input{#2}
        \endinput
      }
    \else
      \def\childdoctmp
      {
        \childdocdisable
        \def\childdocname{#2}
        \childdoctrue
        \includeonly{#2}
        \def\childdocjob{#1}
        \def\jobname{#1}
        \input{#1}
        \endinput
      }
    \fi
    \expandafter
  \endgroup
  \childdoctmp
}
%    \end{macrocode}

% \macro{\childdocforwardprefix}
% The command |\childdocforwardprefix| redirects
% compilation to the main or a child file by means of a pattern.
% The prefix |#1| in the current filename is replaced by |#2|
% and the suffix of the current filename is kept
% (it is assumed that the filename does not contain the substring `|~~~|'
% which is used as a delimiter).
% Compilation is handed over to the new file by |\childdocforward|:
%    \begin{macrocode}
\newcommand{\childdocforwardprefix}[3][]
{
  \begingroup
    \def\childdocextract #2##1~~~{\def\childdoctmp{\childdocforward[#1]{#3##1}}}
    \expandafter\childdocextract\childdocname~~~
    \expandafter
  \endgroup
  \childdoctmp
}
%    \end{macrocode}

% \macro{\childdoc}
% The deprecated macro |\childdoc| is a legacy version of |\childdocmain|:
%    \begin{macrocode}
\newcommand{\childdoc}{\childdocmain}
%    \end{macrocode}

% \macro{\childdocredirect}
% The deprecated macro |\childdocredirect| is a legacy version
% of |\childdocforward| and |\childdocforwardprefix|:
%    \begin{macrocode}
\newcommand{\childdocredirect}[2][]
{
  \begingroup
    \if?#1?
      \def\childdoctmp{\childdocforward{#2}}
    \else
      \def\childdoctmp{\childdocforwardprefix{#1}{#2}}
    \fi
    \expandafter
  \endgroup
  \childdoctmp
}
%    \end{macrocode}

%\iffalse
%</package>
%\fi
%
\endinput
|\\
|\childdocforward[|\textit{main}|]{|\textit{dest}|}|\\
\end{tabular}
\end{center}
%
The argument \textit{dest} is the destination file
(without extension).
It should be the main file or one of the child files.
Note that further \textsf{childdoc} directives
such as |\childdocof| and |\childdocforward|
in the indicated file will be processed in this form.
The optional argument \textit{main}
passes on directly to the main file \textit{main}
while pretending to compile the child \textit{dest}.
This form behaves as if \textit{dest}
issues |\childdocof{|\textit{main}|}| right away,
and no further \textsf{childdoc} directives will be processed.

%%%%%%%%%%%%%%%%%%%%%%%%%%%%%%%%%%%%%%%%
\DescribeMacro{\...prefix}
In the alternative form |\childdocforwardprefix|,
%
\begin{center}
\begin{tabular}{l}
|% \iffalse
%
% childdoc.dtx Copyright (C) 2017-2018 Niklas Beisert
%
% This work may be distributed and/or modified under the
% conditions of the LaTeX Project Public License, either version 1.3
% of this license or (at your option) any later version.
% The latest version of this license is in
%   http://www.latex-project.org/lppl.txt
% and version 1.3 or later is part of all distributions of LaTeX
% version 2005/12/01 or later.
%
% This work has the LPPL maintenance status `maintained'.
%
% The Current Maintainer of this work is Niklas Beisert.
%
% This work consists of the files childdoc.dtx and childdoc.ins
% and the derived files childdoc.def and cdocsamp.tex with
% cdocsch1.tex, cdocsch2.tex, cdocsdrf.tex, cdocsfn1.tex, cdocsfn2.tex.
%
%<package>\ifdefined\childdocmain\endinput\fi
%<package>\ProvidesFile{childdoc.def}[2018/12/30 v2.0 child document driver]
%<samplemain>\ProvidesFile{cdocsamp.tex}[2018/12/30 v2.0 sample for childdoc]
%<*driver>
%\ProvidesFile{childdoc.drv}[2018/12/30 v2.0 childdoc reference manual file]
\PassOptionsToClass{10pt,a4paper}{article}
\documentclass{ltxdoc}

\usepackage[margin=35mm]{geometry}
\usepackage{hyperref}
\usepackage{hyperxmp}
\usepackage[usenames]{color}

\hypersetup{colorlinks=true}
\hypersetup{pdfstartview=FitH}
\hypersetup{pdfpagemode=UseNone}
\hypersetup{pdfsource={}}
\hypersetup{pdflang={en-UK}}
\hypersetup{pdfcopyright={Copyright 2017-2018 Niklas Beisert.
  This work may be distributed and/or modified under the
  conditions of the LaTeX Project Public License, either version 1.3
  of this license or (at your option) any later version.}}
\hypersetup{pdflicenseurl={http://www.latex-project.org/lppl.txt}}
\hypersetup{pdfcontactaddress={ETH Zurich, ITP, HIT K,
  Wolfgang-Pauli-Strasse 27}}
\hypersetup{pdfcontactpostcode={8093}}
\hypersetup{pdfcontactcity={Zurich}}
\hypersetup{pdfcontactcountry={Switzerland}}
\hypersetup{pdfcontactemail={nbeisert@itp.phys.ethz.ch}}
\hypersetup{pdfcontacturl={http://people.phys.ethz.ch/\xmptilde nbeisert/}}

\newcommand{\secref}[1]{\hyperref[#1]{section \ref*{#1}}}

\parskip1ex
\parindent0pt
\let\olditemize\itemize
\def\itemize{\olditemize\parskip0pt}

\begin{document}

\title{The \textsf{childdoc} Package}
\hypersetup{pdftitle={The childdoc Package}}
\author{Niklas Beisert\\[2ex]
  Institut f\"ur Theoretische Physik\\
  Eidgen\"ossische Technische Hochschule Z\"urich\\
  Wolfgang-Pauli-Strasse 27, 8093 Z\"urich, Switzerland\\[1ex]
  \href{mailto:nbeisert@itp.phys.ethz.ch}
  {\texttt{nbeisert@itp.phys.ethz.ch}}}
\hypersetup{pdfauthor={Niklas Beisert}}
\hypersetup{pdfsubject={Manual for the LaTeX2e Package childdoc}}
\date{30 December 2018, \textsf{v2.0}}
\maketitle

\begin{abstract}\noindent
\textsf{childdoc} is a \LaTeXe{} package
that enables the direct compilation
of document sections included by |\include|
to individual files.
\end{abstract}

\begingroup
\parskip0ex
\tableofcontents
\endgroup

%%%%%%%%%%%%%%%%%%%%%%%%%%%%%%%%%%%%%%%%%%%%%%%%%%%%%%%%%%%%%%%%%%%%%%%%%%%%%%%%
%%%%%%%%%%%%%%%%%%%%%%%%%%%%%%%%%%%%%%%%%%%%%%%%%%%%%%%%%%%%%%%%%%%%%%%%%%%%%%%%
\section{Introduction}

\LaTeX{} provides a mechanism to structure a large document (such as a book)
into a main file and several child files (containing the chapters)
using the |\include| command.
This mechanism is beneficial for documents
which span hundreds of pages in order to
make the source file(s) more manageable.
Moreover, compilation can be restricted to
selected child files by means of the |\includeonly| command.
The latter feature can be used to reduce the compilation time while editing
(this was significantly more useful in the earlier days of \LaTeX{})
or to generate a smaller document which is easier to navigate.
Another application of |\includeonly| is to generate
documents consisting of selected parts of the complete document.

However, there are a few drawbacks of the plain |\include| mechanism:
\begin{itemize}
\item
The child files cannot be compiled on their own,
they can only be compiled via the main file.
A naive editing environment
(such as a text editor with an option
to have the current file processed by \LaTeX)
may require one to switch to the main file before compiling;
attempting to compile the child file produces errors.
\item
The main file must be modified (each time)
to adjust the |\includeonly| command
to the present needs. This easily leaves the main file in a messy state.
\item
The generated document will always carry the filename
of the main document. This is inconvenient if
several child files are to be compiled and
to be kept for distribution.
\end{itemize}

The present package provides a simple interface
to make child files individually compilable by \LaTeX{}.
Compiling a child file then has the same effect as compiling
the main file with an |\includeonly| command
to select the appropriate child.
Moreover the generated document will carry the name of the child
rather than the main file.
This resolves all three above issues.

This feature is meant to make the editing of books,
thesis documents and lecture notes somewhat more convenient.
However, the package can also be used efficiently for
composing a series of documents (such as exercise sheets)
which are typically distributed individually.
It then assists the author in generating the individual documents
(potentially in different versions)
as well as a document containing the collected series.
Another application is in developing style files
or other kinds of included material
where compilation of the style file could redirect
to a sample or test file.

%%%%%%%%%%%%%%%%%%%%%%%%%%%%%%%%%%%%%%%%%%%%%%%%%%%%%%%%%%%%%%%%%%%%%%%%%%%%%%%%
%%%%%%%%%%%%%%%%%%%%%%%%%%%%%%%%%%%%%%%%%%%%%%%%%%%%%%%%%%%%%%%%%%%%%%%%%%%%%%%%
\section{Usage}

First of all, the package \textsf{childdoc} is \emph{not} a standard
\LaTeXe{} |.sty| style file! Therefore it needs to be invoked in
a non-standard way.

%%%%%%%%%%%%%%%%%%%%%%%%%%%%%%%%%%%%%%%%%%%%%%%%%%%%%%%%%%%%%%%%%%%%%%%%%%%%%%%%
\subsection{Included Files}
\label{sec:include}

%%%%%%%%%%%%%%%%%%%%%%%%%%%%%%%%%%%%%%%%
\DescribeMacro{\childdocmain}
To use the package, add the commands
\begin{center}
\begin{tabular}{l}
|\input{childdoc.def}|\\
|\childdocmain{}|\\
\end{tabular}
\end{center}
at the very top of the main \LaTeX{} file,
in particular \emph{before} the |\documentclass| statement!
The argument of |\childdocmain| should be left empty
(but it must be present).

%%%%%%%%%%%%%%%%%%%%%%%%%%%%%%%%%%%%%%%%
\DescribeMacro{\childdocof}
Furthermore, add the commands
\begin{center}
\begin{tabular}{l}
|\input{childdoc.def}|\\
|\childdocof{|\textit{main}|}|\\
\end{tabular}
\end{center}
at the top of every child file \textit{child}
which is included by |\include{|\textit{child}|}|
from within the main file
(or at least for those files to be compiled individually).
The argument \textit{main} must be the filename of the main file.

There are a couple of
considerations in setting up the main and child documents:

%%%%%%%%%%%%%%%%%%%%%%%%%%%%%%%%%%%%%%%%
\paragraph{Restrictions.}

Please note the following restrictions:
\begin{itemize}
\item
|\childdocmain| must be called with one argument \textit{main}
to ensure compatibility with earlier version of the package.
It must either be empty (|\childdocmain{}|)
or precisely match the filename of the main file in which it is specified.
See \secref{sec:detection} for further information.
\item
The filename \textit{main} must be specified without the |.tex| extension.
\item
The filename \textit{main} is case sensitive
(even in case-insensitive file systems)
due to internal string comparison.
\item
The argument \textit{main} should be fully expanded, it cannot be a macro.
\item
Subdirectories and special characters should be avoided in filenames.
\item
The command |\childdocmain{|\textit{main}|}| must be followed by a whitespace.
It should not be followed immediately by another command
or by a comment mark `|%|'.
This is because the \TeX{} parser reads the token immediately following
the argument of |\childdocmain| and puts it
at the beginning of every child section;
however, a white\-space is ignored.
\end{itemize}

%%%%%%%%%%%%%%%%%%%%%%%%%%%%%%%%%%%%%%%%
\paragraph{Content of Main File.}

It is advisable to place all content in the child files included by |\include|.
Any output contained in the main file will appear in all child documents
unless suppressed manually;
it cannot be suppressed automatically by the |\includeonly| directive
and thus should normally be avoided.
A method to include some content in the main file
by means of conditional processing is described in \secref{sec:conditional}.

%%%%%%%%%%%%%%%%%%%%%%%%%%%%%%%%%%%%%%%%
\paragraph{Page Numbering.}

When only a part of the document is compiled,
the appropriate numbering of pages
(as well as other status parameters)
is determined from the |.aux| files.
The latter contain information from previous passes.
However this information needs to propagate through
all intermediate child documents.
Therefore the page numbering in child documents may well
be inconsistent until the complete document is compiled at least once.

A useful (if unconventional) way to always ensure a consistent
page numbering is to restart the numbering in each child document
and denote the pages by `\textit{child}|.|\textit{page}'
where \textit{child} represents the chapter/section number of the child file.
This can be achieved by the command
|\numberwithin{page}{|\textit{child}|}|
of the \textsf{amsmath} package
where \textit{child} can be |chapter| or |section|
depending on the chosen structuring.
Alternatively, one can modify the macro |\thepage| appropriately
and reset the counter |page| at the start of each child file.

%%%%%%%%%%%%%%%%%%%%%%%%%%%%%%%%%%%%%%%%%%%%%%%%%%%%%%%%%%%%%%%%%%%%%%%%%%%%%%%%
\subsection{Conditional Processing}
\label{sec:conditional}

The package provides a mechanism to compile different versions
of a document. To customise the versions further some conditional processing
can come in handy to distinguish which version is being compiled.
The package provides two macros to describe the compilation context:

%%%%%%%%%%%%%%%%%%%%%%%%%%%%%%%%%%%%%%%%
\DescribeMacro{\ifchilddoc}
The conditional |\ifchilddoc| distinguishes between the compilation of
child documents and the main document:
%
\begin{center}
|\ifchilddoc |\textit{child-code}| |[|\||else |\textit{main-code}]| \||fi|
\end{center}

%%%%%%%%%%%%%%%%%%%%%%%%%%%%%%%%%%%%%%%%
\DescribeMacro{\childdocname}
\DescribeMacro{\childdocjob}
The macro |\childdocname| contains the filename (without extension)
of the main or child file being processed.
Note that |\childdocjob| will always contain the name of the main file.

%%%%%%%%%%%%%%%%%%%%%%%%%%%%%%%%%%%%%%%%
\paragraph{Title Page.}

Conditional processing can be used to include a title or banner page
in the main document when proper precautions are taken.
Importantly, the code in the main file should ensure that the page counter
(as well as other status parameters which are stored in the |.aux| files)
takes the same value after the conditional processing.
Otherwise the page numbers may take divergent values
depending on which part is compiled.

For example, a title page could be declared by:
%
\begin{center}
\begin{tabular}{l}
|\ifchilddoc\||else|\\
|\addtocounter{page}{-1}|\\
\textit{code for title page}\\
|\newpage|\\
|\||fi|
\end{tabular}
\end{center}
%
A banner page for the child documents can be generated by:
%
\begin{center}
\begin{tabular}{l}
|\ifchilddoc|\\
|\addtocounter{page}{-1}|\\
\textit{code for banner page}\\
|\newpage|\\
|\||fi|
\end{tabular}
\end{center}
%
Here one could write a message such as:
\begin{center}
|This is the part \childdocname{} of \childdocjob{}.|
\end{center}

%%%%%%%%%%%%%%%%%%%%%%%%%%%%%%%%%%%%%%%%%%%%%%%%%%%%%%%%%%%%%%%%%%%%%%%%%%%%%%%%
\subsection{Flags}
\label{sec:flags}

The package makes it easy to generate different versions
of the main or child documents.
To this end compilation flags can be defined
and assigned different default values.
They will be particularly useful in conjunction
with the forwarding mechanism described in \secref{sec:forward}.

For example, it may be useful to have a flag |\version|
which can be set to |draft| or |final|.
The document source will contain some conditional code
depending on the value of |\version|.
Suppose further, the flag should default to |final| for the main file
and to |draft| for child files
which is a natural assignment for editing the document.
This is achieved by placing the following code
in the preamble of the main document
(below the |\childdocmain| directive):
%
\begin{center}
\begin{tabular}{l}
|\ifchilddoc|\\
|\providecommand{\version}{draft}|\\
|\||else|\\
|\providecommand{\version}{final}|\\
|\||fi|
\end{tabular}
\end{center}
%
The definition by |\providecommand| makes sure
that previous definitions are not overwritten.
Further statements |\providecommand{\version}{...}|
can thus be added before the above code to override it.

For the main file, one might add a line
(between |\childdocmain| and the above block)
%
\begin{center}
|%\ifchilddoc\||else\providecommand{\version}{draft}\||fi|
\end{center}
%
which can be uncommented to produce a draft version.
Likewise one can add a line to the very top of a child file
(above the |\childdocof{|\textit{main}|}| directive)
%
\begin{center}
|%\providecommand{\version}{final}|
\end{center}
%
which can be uncommented to produce the final version of this child document.

%%%%%%%%%%%%%%%%%%%%%%%%%%%%%%%%%%%%%%%%%%%%%%%%%%%%%%%%%%%%%%%%%%%%%%%%%%%%%%%%
\subsection{Forwarding}
\label{sec:forward}

Different versions of the main or child documents
using compilation flags as described in \secref{sec:flags}
can be (permanently) stored in different files
for convenient compilation, viewing and distribution.
To this end, the package defines a command
to pass on compilation to a different file:

%%%%%%%%%%%%%%%%%%%%%%%%%%%%%%%%%%%%%%%%
\DescribeMacro{\childdocforward}
The command |\childdocforward| redirects processing to
another source file:
%
\begin{center}
\begin{tabular}{l}
|\input{childdoc.def}|\\
|\childdocforward[|\textit{main}|]{|\textit{dest}|}|\\
\end{tabular}
\end{center}
%
The argument \textit{dest} is the destination file
(without extension).
It should be the main file or one of the child files.
Note that further \textsf{childdoc} directives
such as |\childdocof| and |\childdocforward|
in the indicated file will be processed in this form.
The optional argument \textit{main}
passes on directly to the main file \textit{main}
while pretending to compile the child \textit{dest}.
This form behaves as if \textit{dest}
issues |\childdocof{|\textit{main}|}| right away,
and no further \textsf{childdoc} directives will be processed.

%%%%%%%%%%%%%%%%%%%%%%%%%%%%%%%%%%%%%%%%
\DescribeMacro{\...prefix}
In the alternative form |\childdocforwardprefix|,
%
\begin{center}
\begin{tabular}{l}
|\input{childdoc.def}|\\
|\childdocforwardprefix[|\textit{main}|]{|\textit{prefix}|}{|\textit{dest}|}|
\end{tabular}
\end{center}
%
the destination file is determined by a pattern
depending on the current file:
To make this work, the current file must be called
`{\textit{prefix}\hspace{0.2em}\textit{suffix}}'
with \textit{prefix} matching precisely the argument.
Processing is then passed on to the file
`{\textit{dest}\hspace{0.2em}\textit{suffix}}'.
Surely, the same effect is achieved by
directly specifying the
argument `{\textit{dest}\hspace{0.2em}\textit{suffix}}'
in the first form.
However, that requires to set up a different file
for each child. With the alternative form of the command
all these files can have exactly the same content
which simplifies setting them up and maintaining them.

For example, the following file |draft.tex|
with a compilation flag |\version| as described in \secref{sec:flags}
compiles the main document as a draft:
%
\begin{center}
\begin{tabular}{l}
|\def\version{draft}|\\
|\input{childdoc.def}|\\
|\childdocforward{|\textit{main}|}|
\end{tabular}
\end{center}
%
Likewise, the following files |final|\textit{nn}|.tex|
compile the final version of the child document
|child|\textit{nn}|.tex|:
%
\begin{center}
\begin{tabular}{l}
|\def\version{final}|\\
|\input{childdoc.def}|\\
|\childdocforwardprefix{final}{child}|
\end{tabular}
\end{center}
%

Note that when several versions of a main file and/or of each child file
are to be generated, it may be convenient to set up a |Makefile| or
shell script to automatise the process.

%%%%%%%%%%%%%%%%%%%%%%%%%%%%%%%%%%%%%%%%%%%%%%%%%%%%%%%%%%%%%%%%%%%%%%%%%%%%%%%%
\subsection{Command Line Processing}
\label{sec:commandline}

The effect of redirection files can also be achieved by invoking
the \LaTeX{} compiler with a more elaborate command line.
Most conveniently this should be done as part
of a shell script or a |Makefile|.

When using \textsf{childdoc} in the main file, the following
command lines effectively perform a redirection
(note that depending on the shell being used,
backslashes may have to be doubled: `|\|' $\to$ `|\\|'):
%
\begin{center}
|... -jobname "|\textit{target}|" |\\|"|[\textit{flags}]%
|\input{childdoc.def}\childdocforward[|\textit{main}|]{|\textit{dest}|}"|
\end{center}
%
Here \textit{target} is the name of the output file,
\textit{main} is the name of the main file
and \textit{dest} is the name of the main or child file to be processed
(all filenames without extensions).
The optional argument \textit{main} can be omitted
if \textit{main} matches \textit{dest}.
Optionally, compilation \textit{flags} can be defined via |\def| commands.
This command line makes the \TeX{} engine believe
it is compiling the file \textit{target}
whose content is specified as the latter parameter.
The provided code then forwards the processing to
\textit{main} or \textit{dest} as described in \secref{sec:forward}.

%%%%%%%%%%%%%%%%%%%%%%%%%%%%%%%%%%%%%%%%%%%%%%%%%%%%%%%%%%%%%%%%%%%%%%%%%%%%%%%%
\subsection{Include by Input}
\label{sec:input}

Including child documents by |\include| has some restrictions by design.
Most notably, the content of a child document always occupies
its own set of pages; pages cannot be shared between child documents.
Usually, this behaviour makes perfect sense
because each child document contain an essential part of the document.
However, in some situations it may be desirable to compose
a document from a collection of parts
without having mandatory page breaks between then.
For this case, the package
provides a mechanism to include parts
by |\input| which can also be processed individually.
However, by construction this mechanism
requires manual handling of the content to be output.

%%%%%%%%%%%%%%%%%%%%%%%%%%%%%%%%%%%%%%%%
\DescribeMacro{\ifchilddocmanual}
The main file should be prepared as usual, see \secref{sec:include}.
However, the document body must make a distinction
between processing of an individual part and of the main document, e.g.:
%
\begin{center}
\begin{tabular}{l}
|\ifchilddocmanual|\\
|\input{\childdocname}|\\
|\||else|\\
\textit{document body with }|\input{|\textit{part}|}|\\
|\||fi|
\end{tabular}
\end{center}
%
The conditional |\ifchilddocmanual| is true whenever
a part to be included by |\input| is being compiled,
and the name of the part is stored in |\childdocname|.

%%%%%%%%%%%%%%%%%%%%%%%%%%%%%%%%%%%%%%%%
\DescribeMacro{\childdocby}
Each part to be included by |\input| should start with:
%
\begin{center}
\begin{tabular}{l}
|\input{childdoc.def}|\\
|\childdocby{|\textit{main}|}|\\
\end{tabular}
\end{center}
%
The directive |\childdocby| is similar to |\childdocof|
described in \secref{sec:include},
but the subsequent selection of content must be done manually.
To that end, both |\ifchilddoc| and |\ifchilddocmanual|
will be true upon processing of a part,
and the name of the part is stored in |\childdocname|.
Note that |\jobname| will be set to the filename of the current part
so that each part receives an individual |.aux| file
that does not interfere with the |.aux| file(s) of the main document.
This behaviour can be altered by the alternative form
|\childdocby[*]{|\textit{main}|}| (with a non-empty optional argument)
which uses the |.aux| file of the main document
by setting |\jobname| to \textit{main}.

%%%%%%%%%%%%%%%%%%%%%%%%%%%%%%%%%%%%%%%%%%%%%%%%%%%%%%%%%%%%%%%%%%%%%%%%%%%%%%%%
\subsection{Driver Development}
\label{sec:driver}

The \textsf{childdoc} mechanism can also be use for the development
of definition files such as \LaTeX{} styles or classes.
This case differs from the above setup with multiple parts
included by |\include| in that no |\includeonly| should be invoked.
This can be achieved by starting the include file
(before |\ProvidesPackage|) with:
%
\begin{center}
\begin{tabular}{l}
|\input{childdoc.def}|\\
|\childdocforward{|\textit{main}|}|\\
\end{tabular}
\end{center}
%
or alternatively with:
%
\begin{center}
\begin{tabular}{l}
|\input{childdoc.def}|\\
|\childdocby{|\textit{main}|}|\\
\end{tabular}
\end{center}
%
Both forms have slightly different effects as described above.
The main file is prepared as usual, see \secref{sec:include}.

%%%%%%%%%%%%%%%%%%%%%%%%%%%%%%%%%%%%%%%%%%%%%%%%%%%%%%%%%%%%%%%%%%%%%%%%%%%%%%%%
\subsection{Legacy Detection}
\label{sec:detection}

The directive |\childdocmain| in the main file can detect
whether the complete document or merely a child is to be compiled
even without using the directive |\childdocof|.
This method is deprecated because it is less robust
and there is no compelling reason to use it;
it is merely provided for backward compatibility
and it may be removed in future versions.

If the detection mechanism is to be used,
it is mandatory to correctly specify
the filename of the main file as the argument of |\childdocmain|:
%
\begin{center}
\begin{tabular}{l}
|\input{childdoc.def}|\\
|\childdocmain{|\textit{main}|}|\\
\end{tabular}
\end{center}
%
If |\jobname| does not match the argument \textit{main} of |\childdocmain|,
it is assumed that |\jobname| points to the child file to be compiled.
When using |\childdocmain| with the main file specified as argument,
it suffices to start a child file
with just |\input{|\textit{main}|}|
without loading of the package and using |\childdocof|.
If instead all processing is done
with the appropriate \textsf{childdoc} directives,
the argument of \textit{main} of |\childdocmain| can be empty.

An alternative version of the command line processing described
in \secref{sec:commandline} using the detection mechanism reads:
%
\begin{center}
|... -jobname "|\textit{target}|" "|[\textit{flags}]%
[|\def\jobname{|\textit{dest}|}|]|\input{|\textit{main}|}"|
\end{center}

%%%%%%%%%%%%%%%%%%%%%%%%%%%%%%%%%%%%%%%%%%%%%%%%%%%%%%%%%%%%%%%%%%%%%%%%%%%%%%%%
\subsection{Manual Code}
\label{sec:manual}

In case one cannot be certain whether the definitions file |childdoc.def|
is installed on the target \TeX{} distribution
and one prefers not to ship it,
it is conceivable to paste a few relevant commands into the sources.

To that end, drop all statements |\input{childdoc.def}|
and perform the replacements as outlined below.
Instead of |\childdocmain{|\textit{main}|}| add the following code
to the top of the main file:
%
\begin{center}
\begin{tabular}{l}
|\||ifdefined\childdocname\endinput\||fi\newif\ifchilddoc|\\
|\edef\childdocname{\scantokens\expandafter{\jobname\noexpand}}|\\
|\def\childdocmain{|\textit{main}|}\||ifx\childdocmain\childdocname\||else|\\
|\childdoctrue\includeonly{\childdocname}\let\jobname\childdocmain\||fi|\\
\end{tabular}
\end{center}
%
Instead of |\childdocof{|\textit{main}|}| just include the main file
at the top of each child file:
%
\begin{center}
|\input{|\textit{main}|}|
\end{center}
%
A simple redirection |\childdocforward{|\textit{dest}|}| is achieved by:
%
\begin{center}
|\def\jobname{|\textit{dest}|}\input{\jobname}|
\end{center}
%
The redirection with prefix
|\childdocforwardprefix[|\textit{prefix}|]{|\textit{dest}|}|
is accomplished by:
%
\begin{center}
\begin{tabular}{l}
|{\edef\jobname{\scantokens\expandafter{\jobname\noexpand}}|\\
|\def\redirectjob |\textit{prefix}|#1~~~{\gdef\jobname{|\textit{dest}|#1}}|\\
|\expandafter\redirectjob\jobname~~~}\input{\jobname}|
\end{tabular}
\end{center}

In an alternative approach,
child documents can be compiled by a specific command line
without additional code or specific definitions:
%
\begin{center}
|... -jobname "|\textit{target}|" "|[\textit{flags}]%
|\includeonly{|\textit{dest}|}\input{|\textit{main}|}"|
\end{center}
%

%%%%%%%%%%%%%%%%%%%%%%%%%%%%%%%%%%%%%%%%%%%%%%%%%%%%%%%%%%%%%%%%%%%%%%%%%%%%%%%%
%%%%%%%%%%%%%%%%%%%%%%%%%%%%%%%%%%%%%%%%%%%%%%%%%%%%%%%%%%%%%%%%%%%%%%%%%%%%%%%%
\section{Information}

%%%%%%%%%%%%%%%%%%%%%%%%%%%%%%%%%%%%%%%%%%%%%%%%%%%%%%%%%%%%%%%%%%%%%%%%%%%%%%%%
\subsection{Copyright}

Copyright \copyright{} 2017--2018 Niklas Beisert

This work may be distributed and/or modified under the
conditions of the \LaTeX{} Project Public License, either version 1.3
of this license or (at your option) any later version.
The latest version of this license is in
  \url{http://www.latex-project.org/lppl.txt}
and version 1.3 or later is part of all distributions of \LaTeX{}
version 2005/12/01 or later.

This work has the LPPL maintenance status `maintained'.

The Current Maintainer of this work is Niklas Beisert.

This work consists of the files |README.txt|, |childdoc.ins| and |childdoc.dtx|
as well as the derived files |childdoc.def|, |cdocsamp.tex|
with |cdocsch1.tex|, |cdocsch2.tex|, |cdocspt3.tex|, |cdocspt4.tex|,
|cdocsdrf.tex|, |cdocsfn1.tex|, |cdocsfn2.tex|
as well as |childdoc.pdf|.

%%%%%%%%%%%%%%%%%%%%%%%%%%%%%%%%%%%%%%%%%%%%%%%%%%%%%%%%%%%%%%%%%%%%%%%%%%%%%%%%
\subsection{Files and Installation}

The package consists of the files:
%
\begin{center}
\begin{tabular}{ll}
    |README.txt|   & readme file \\
    |childdoc.ins| & installation file \\
    |childdoc.dtx| & source file \\
    |childdoc.def| & definition file \\
    |cdocsamp.tex| & sample main file \\
    |cdocsch1.tex| & sample include file \\
    |cdocsch2.tex| & sample include file \\
    |cdocspt3.tex| & sample part file \\
    |cdocspt4.tex| & sample part file \\
    |cdocsdrf.tex| & sample redirection file \\
    |cdocsfn1.tex| & sample redirection file \\
    |cdocsfn2.tex| & sample redirection file \\
    |childdoc.pdf| & manual
\end{tabular}
\end{center}
%
The distribution consists of the files
|README.txt|, |childdoc.ins| and |childdoc.dtx|.
%
\begin{itemize}
\item
Run (pdf)\LaTeX{} on |childdoc.dtx|
to compile the manual |childdoc.pdf| (this file).
\item
Run \LaTeX{} on |childdoc.ins| to create the definitions file |childdoc.def|
and the sample |cdocsamp.tex| with include files
|cdocsch1.tex|, |cdocsch2.tex|, |cdocspt3.tex|, |cdocspt4.tex|,
|cdocsdrf.tex|, |cdocsfn1.tex|, |cdocsfn2.tex|.
Then copy the file |childdoc.def| to an appropriate directory of your \LaTeX{}
distribution, e.g.\ \textit{texmf-root}|/tex/latex/childdoc|.
\end{itemize}

%%%%%%%%%%%%%%%%%%%%%%%%%%%%%%%%%%%%%%%%%%%%%%%%%%%%%%%%%%%%%%%%%%%%%%%%%%%%%%%%
\subsection{Related CTAN Packages}

There are several other packages which offer a similar functionality:
%
\begin{itemize}
\item
The packages
\href{http://ctan.org/pkg/docmute}{\textsf{docmute}},
\href{http://ctan.org/pkg/includex}{\textsf{includex}} and
\href{http://ctan.org/pkg/standalone}{\textsf{standalone}}
provide commands to include only the document body of
a child file thus allowing both files to be compiled individually.
\item
The packages \href{http://ctan.org/pkg/subdocs}{\textsf{subdocs}}
and \href{http://ctan.org/pkg/subfiles}{\textsf{subfiles}}
provide structures in which the main and child documents can be
encapsulated and allowing them to be compiled individually.
The inclusion mechanism is different from the conventional |\include|.
\item
The package \href{http://ctan.org/pkg/combine}{\textsf{combine}}
is an elaborate solution to combine several documents into one.
\end{itemize}
%
See also the CTAN topic \href{http://ctan.org/topic/subdocs}{\textsf{subdocs}}
for further related packages.
The present package differs from the above solutions in that
a document structure constructed with the conventional |\include| mechanism
just needs two extra commands at the top of every file
such that all constituent files can be compiled individually.

%%%%%%%%%%%%%%%%%%%%%%%%%%%%%%%%%%%%%%%%%%%%%%%%%%%%%%%%%%%%%%%%%%%%%%%%%%%%%%%%
%\subsection{Feature Suggestions}
%
%The following is a list of features which may be useful for future
%versions of this package:
%%
%\begin{itemize}
%\item
%\ldots
%\end{itemize}

%%%%%%%%%%%%%%%%%%%%%%%%%%%%%%%%%%%%%%%%%%%%%%%%%%%%%%%%%%%%%%%%%%%%%%%%%%%%%%%%
\subsection{Revision History}

%%%%%%%%%%%%%%%%%%%%%%%%%%%%%%%%%%%%%%%%
\paragraph{v2.0:} 2018/12/30

\begin{itemize}
\item
immediate forward processing
\item
added |\childdocby| mechanism
\item
manual restructured
\end{itemize}

%%%%%%%%%%%%%%%%%%%%%%%%%%%%%%%%%%%%%%%%
\paragraph{v1.6:} 2018/01/17

\begin{itemize}
\item
application for development of include files
\item
corrections to manual
\end{itemize}

%%%%%%%%%%%%%%%%%%%%%%%%%%%%%%%%%%%%%%%%
\paragraph{v1.5:} 2017/05/21

\begin{itemize}
\item
more complete structuring introduced
\item
|\childdocof| introduced
\item
|\childdoc| renamed to |\childdocmain|
\item
|\childredirect| renamed to |\childdocforward| and |\childdocforwardprefix|
and functionality expanded
\end{itemize}

%%%%%%%%%%%%%%%%%%%%%%%%%%%%%%%%%%%%%%%%
\paragraph{v1.0:} 2017/04/27

\begin{itemize}
\item
manual and install package
\item
first version published on CTAN
\end{itemize}

%%%%%%%%%%%%%%%%%%%%%%%%%%%%%%%%%%%%%%%%
\paragraph{v0.6:} 2017/04/26

\begin{itemize}
\item
redirection mechanism added
\end{itemize}

%%%%%%%%%%%%%%%%%%%%%%%%%%%%%%%%%%%%%%%%
\paragraph{v0.5:} 2017/04/26

\begin{itemize}
\item
functionality in definition file
\end{itemize}


%%%%%%%%%%%%%%%%%%%%%%%%%%%%%%%%%%%%%%%%%%%%%%%%%%%%%%%%%%%%%%%%%%%%%%%%%%%%%%%%
%%%%%%%%%%%%%%%%%%%%%%%%%%%%%%%%%%%%%%%%%%%%%%%%%%%%%%%%%%%%%%%%%%%%%%%%%%%%%%%%
%%%%%%%%%%%%%%%%%%%%%%%%%%%%%%%%%%%%%%%%%%%%%%%%%%%%%%%%%%%%%%%%%%%%%%%%%%%%%%%%
\appendix

\settowidth\MacroIndent{\rmfamily\scriptsize 000\ }

 \DocInput{childdoc.dtx}

\end{document}
%</driver>
% \fi
%
% %%%%%%%%%%%%%%%%%%%%%%%%%%%%%%%%%%%%%%%%%%%%%%%%%%%%%%%%%%%%%%%%%%%%%%%%%%%%%%
% %%%%%%%%%%%%%%%%%%%%%%%%%%%%%%%%%%%%%%%%%%%%%%%%%%%%%%%%%%%%%%%%%%%%%%%%%%%%%%
% \section{Sample}
%\iffalse
%<*samplemain>
%\fi
%
% The following presents a sample document
% with two chapters, two parts, a title page,
% a compile flag as well as three forwarding files to set the flag.
% It consists of eight |.tex| files:
% \begin{center}
% \begin{tabular}{ll}
% |cdocsamp.tex|&main file\\
% |cdocsch1.tex|&include file for chapter 1\\
% |cdocsch2.tex|&include file for chapter 2\\
% |cdocspt3.tex|&include file for part 3\\
% |cdocspt4.tex|&include file for part 4\\
% |cdocsdrf.tex|&forwarding file for main file in draft mode\\
% |cdocsfi1.tex|&forwarding file for final version of chapter 1\\
% |cdocsfi2.tex|&forwarding file for final version of chapter 2\\
% \end{tabular}
% \end{center}
% Each of the eight files can be compiled directly by the \LaTeX{} compiler.
%
% %%%%%%%%%%%%%%%%%%%%%%%%%%%%%%%%%%%%%%
% \paragraph{Main File.}
%
% The main file is called |cdocsamp.tex|.
%
% Load the \textsf{childdoc} definitions and
% declare the filename for the main document:
%    \begin{macrocode}
\input{childdoc.def}
\childdocmain{}
%    \end{macrocode}

% Optional override for |\version| flag:
%    \begin{macrocode}
%%\ifchilddoc\else\providecommand{\version}{draft}\fi
%    \end{macrocode}

% Define the default values for the |\version| flag
% (|final| for the main file and |draft| for childs):
%    \begin{macrocode}
\ifchilddoc
\providecommand{\version}{draft}
\else
\providecommand{\version}{final}
\fi
%    \end{macrocode}

% Load the standard document class:
%    \begin{macrocode}
\documentclass[12pt]{article}
%    \end{macrocode}

% Start the document body:
%    \begin{macrocode}
\begin{document}
%    \end{macrocode}

% Declare a title page.
% Print title, part of document being processed and version flag:
%    \begin{macrocode}
\addtocounter{page}{-1}
\begin{center}
{\LARGE\bfseries{}childdoc example\par}
\vspace{1cm}
\ifchilddoc
\ifchilddocmanual part\else chapter\fi:
`\childdocname' of `\childdocjob'\par
\else
main document: `\childdocjob'\par
\fi
version: \version\par
\end{center}
\newpage
%    \end{macrocode}

% Manually include selected file,
% otherwise process as usual:
%    \begin{macrocode}
\ifchilddocmanual
\section*{part `\childdocname'}
\input{\childdocname}
\else
%    \end{macrocode}

% Include the two chapters:
%    \begin{macrocode}
\include{cdocsch1}
\include{cdocsch2}
%    \end{macrocode}

% Include the two parts unless only chapters should be displayed:
%    \begin{macrocode}
\ifchilddoc\else
\section{part three}
\input{cdocspt3}
\section{part four}
\input{cdocspt4}
\fi
%    \end{macrocode}

% Process as usual until here:
%    \begin{macrocode}
\fi
%    \end{macrocode}

% End of document body:
%    \begin{macrocode}
\end{document}
%    \end{macrocode}
%\iffalse
%</samplemain>
%\fi
%
% %%%%%%%%%%%%%%%%%%%%%%%%%%%%%%%%%%%%%%
% \paragraph{Chapter Include Files.}
%
% The include files are called |cdocsch1.tex| and |cdocsch2.tex|.
%
%\iffalse
%<*samplechap1|samplechap2>
%\fi

% Optional override for |\version| flag:
%    \begin{macrocode}
%%\providecommand{\version}{final}
%    \end{macrocode}

% Include the main document:
%    \begin{macrocode}
\input{childdoc.def}
\childdocof{cdocsamp}
%    \end{macrocode}

%\iffalse
%</samplechap1|samplechap2>
%\fi
%
%\iffalse
%<*samplechap1>
%\fi
% Some text for chapter 1:
%    \begin{macrocode}
\section{one}
some text in chapter one
%    \end{macrocode}

%\iffalse
%</samplechap1>
%\fi
% Some text for chapter 2:
%\iffalse
%<*samplechap2>
%\fi
%    \begin{macrocode}
\section{two}
more text in chapter two
%    \end{macrocode}

%\iffalse
%</samplechap2>
%\fi
%
% %%%%%%%%%%%%%%%%%%%%%%%%%%%%%%%%%%%%%%
% \paragraph{Part Include Files.}
%
% The include files are called |cdocspt3.tex| and |cdocspt4.tex|.
%
%\iffalse
%<*samplepart3|samplepart4>
%\fi

% Optional override for |\version| flag:
%    \begin{macrocode}
%%\providecommand{\version}{final}
%    \end{macrocode}

% Include the main document:
%    \begin{macrocode}
\input{childdoc.def}
\childdocby{cdocsamp}
%    \end{macrocode}

%\iffalse
%</samplepart3|samplepart4>
%\fi
%
%\iffalse
%<*samplepart3>
%\fi
% Some text for part 3:
%    \begin{macrocode}
some text in part three
%    \end{macrocode}

%\iffalse
%</samplepart3>
%\fi
% Some text for part 4:
%\iffalse
%<*samplepart4>
%\fi
%    \begin{macrocode}
more text in part four
%    \end{macrocode}

%\iffalse
%</samplepart4>
%\fi
%
% %%%%%%%%%%%%%%%%%%%%%%%%%%%%%%%%%%%%%%
% \paragraph{Forwarding for a Complete Draft.}
%
% The following forwarding file |cdocsdrf.tex|
% compiles the main document in draft mode:
%\iffalse
%<*sampledraft>
%\fi
%    \begin{macrocode}
\def\version{draft}
\input{childdoc.def}
\childdocforward{cdocsamp}
%    \end{macrocode}

%\iffalse
%</sampledraft>
%\fi
%
% %%%%%%%%%%%%%%%%%%%%%%%%%%%%%%%%%%%%%%
% \paragraph{Forwarding for Final Version of the Chapters.}
%
% The following forwarding files |cdocsfn1.tex| and |cdocsfn2.tex|
% (with identical content)
% compile the final versions of the child documents
% |cdocsch1.tex| and |cdocsch2.tex|, respectively:
%\iffalse
%<*samplefinal>
%\fi
%    \begin{macrocode}
\def\version{final}
\input{childdoc.def}
\childdocforwardprefix[cdocsamp]{cdocsfn}{cdocsch}
%    \end{macrocode}

%\iffalse
%</samplefinal>
%\fi
%
% %%%%%%%%%%%%%%%%%%%%%%%%%%%%%%%%%%%%%%
% \paragraph{Command Line Processing.}
%
% The following three command lines generate the output files
% |cdocscld|, |cdocscl1| and |cdocscl2|
% which should be identical to
% |cdocsdrf|, |cdocsch1| and |cdocsfn2|, respectively:
% \begin{center}
% \begin{tabular}{l}
% |latex -jobname cdocscld \|\\
% |  "\def\version{draft}\input{childdoc.def}\childdocforward{cdocsamp}"|\\
% |latex -jobname cdocscl1 \|\\
% |  "\input{childdoc.def}\childdocforward[cdocsamp]{cdocsch1}"|\\
% |latex -jobname cdocscl2 \|\\
% |  "\def\version{final}\input{childdoc.def}\childdocforward{cdocsch2}"|
% \end{tabular}
% \end{center}
% Note that the trailing backslash on each first line
% merely continues the input to the second line
% (for convenient cut ant paste).
% Furthermore, the command |latex| can be replaced by any
% of its alternative versions such as |pdflatex|.
%
% %%%%%%%%%%%%%%%%%%%%%%%%%%%%%%%%%%%%%%%%%%%%%%%%%%%%%%%%%%%%%%%%%%%%%%%%%%%%%%
% %%%%%%%%%%%%%%%%%%%%%%%%%%%%%%%%%%%%%%%%%%%%%%%%%%%%%%%%%%%%%%%%%%%%%%%%%%%%%%
% \section{Implementation}
%\iffalse
%<*package>
%\fi
%
% This section describes the definitions file |childdoc.def|.

% The definitions cannot be loaded using |\usepackage| or |\RequirePackage|
% which has a mechanism to prevent loading a style file more than once.
% When loading the definitions by means of |\input|
% multiple instances have to be prevented manually:
%\iffalse
%This code needs to be before the `\ProvidesFile' directive
%which is defined at the beginning of this file.
%Therefore it is also placed there and commented out here.
%</package>
%<*discard>
%\fi
%    \begin{macrocode}
\ifdefined\childdocmain\endinput\fi
%    \end{macrocode}
%\iffalse
%</discard>
%<*package>
%\fi
%
% \macro{\ifchilddoc}
% \macro{\ifchilddocmanual}
% The conditional |\ifchilddoc| tells whether a
% child (true) or main (false) document is being compiled.
% The conditional |\ifchilddocmanual| tells whether
% the |\includeonly| mechanism is used (false) or
% the selection of child files must be performed manually (true).
% The definitions initialise to false:
%    \begin{macrocode}
\newif\ifchilddoc
\newif\ifchilddocmanual
%    \end{macrocode}

% \macro{\childdocname}
% \macro{\childdocjob}
% The macro |\childdocname| stores the name of the main document
% to be compiled. The macro |\childdocjob| stores the name of
% the document on which the \LaTeX{} compiler was originally invoked.
% The content of |\jobname| cannot be compared
% to filenames specified in the source due to different catcodes.
% The following code rescans |\jobname|, stores the result
% in |\childdocname| and saves a copy in |\childdocjob|:
%    \begin{macrocode}
\edef\childdocname{\scantokens\expandafter{\jobname\noexpand}}
\let\childdocjob\childdocname
%    \end{macrocode}

% \macro{\childdocdisable}
% The macro |\childdocdisable| prevents the main file
% from being processed more than once.
% At this stage, the main document command |\childdocmain|
% is assumed to be called once again where it should do nothing.
% Any subsequent call to it should prevent
% a secondary processing of the main document
% It overwrites the forwarding commands
% |\childdocof| and |\childdocforward|
% with empty macros to prevent further inclusions of the main document:
%    \begin{macrocode}
\newcommand{\childdocdisable}
{
  \renewcommand{\childdocmain}[1]{\renewcommand{\childdocmain}[1]{\endinput}}
  \renewcommand{\childdocof}[1]{}
  \renewcommand{\childdocby}[2][]{}
  \renewcommand{\childdocforward}[2][]{}
  \renewcommand{\childdocdisable}{}
}
%    \end{macrocode}

% \macro{\childdocmain}
% The macro |\childdocmain| is to be called at the top of the main file
% with nothing or the main filename (without extension) as argument.
% First, it breaks loops.
% If the argument is not empty and does not match |\childdocname|
% (which is set by the first inclusion of |childdoc.def|),
% |\ifchilddoc| is set to true, |\includeonly| is applied to the child file
% and |\jobname| is set to the main file
% (for proper handling of |.aux| files):
%    \begin{macrocode}
\newcommand{\childdocmain}[1]
{
  \childdocdisable\childdocmain{}
  \if?#1?\else
    \begingroup
      \def\childdoctmp{#1}
      \ifx\childdoctmp\childdocname
        \def\childdoctmp{}
      \else
        \def\childdoctmp
        {
          \childdoctrue
          \includeonly{\childdocname}
          \def\childdocjob{#1}
          \def\jobname{#1}
        }
      \fi
      \expandafter
    \endgroup
    \childdoctmp
  \fi
}
%    \end{macrocode}

% \macro{\childdocof}
% The command |\childdocof| redirects
% compilation to the main file |#1|.
%    \begin{macrocode}
\newcommand{\childdocof}[1]
{
  \childdocdisable
  \childdoctrue
  \includeonly{\childdocname}
  \def\jobname{#1}
  \def\childdocjob{#1}
  \input{#1}
}
%    \end{macrocode}

% \macro{\childdocby}
% The command |\childdocby| ....
%    \begin{macrocode}
\newcommand{\childdocby}[2][]
{
  \childdocdisable
  \childdoctrue
  \childdocmanualtrue
  \if?#1?\else
    \def\jobname{#2}
  \fi
  \def\childdocjob{#2}
  \input{#2}
  \endinput
}
%    \end{macrocode}

% \macro{\childdocforward}
% The command |\childdocforward| redirects
% compilation to the main file or
% (if the optional argument is given) a child file.
% Parameters are set as if the main file
% or a child file starting with |\childdocof| was compiled.
% Then compilation is handed over to the main file:
%    \begin{macrocode}
\newcommand{\childdocforward}[2][]
{
  \begingroup
    \if?#1?
      \def\childdoctmp
      {
        \def\childdocname{#2}
        \def\childdocjob{#2}
        \def\jobname{#2}
        \input{#2}
        \endinput
      }
    \else
      \def\childdoctmp
      {
        \childdocdisable
        \def\childdocname{#2}
        \childdoctrue
        \includeonly{#2}
        \def\childdocjob{#1}
        \def\jobname{#1}
        \input{#1}
        \endinput
      }
    \fi
    \expandafter
  \endgroup
  \childdoctmp
}
%    \end{macrocode}

% \macro{\childdocforwardprefix}
% The command |\childdocforwardprefix| redirects
% compilation to the main or a child file by means of a pattern.
% The prefix |#1| in the current filename is replaced by |#2|
% and the suffix of the current filename is kept
% (it is assumed that the filename does not contain the substring `|~~~|'
% which is used as a delimiter).
% Compilation is handed over to the new file by |\childdocforward|:
%    \begin{macrocode}
\newcommand{\childdocforwardprefix}[3][]
{
  \begingroup
    \def\childdocextract #2##1~~~{\def\childdoctmp{\childdocforward[#1]{#3##1}}}
    \expandafter\childdocextract\childdocname~~~
    \expandafter
  \endgroup
  \childdoctmp
}
%    \end{macrocode}

% \macro{\childdoc}
% The deprecated macro |\childdoc| is a legacy version of |\childdocmain|:
%    \begin{macrocode}
\newcommand{\childdoc}{\childdocmain}
%    \end{macrocode}

% \macro{\childdocredirect}
% The deprecated macro |\childdocredirect| is a legacy version
% of |\childdocforward| and |\childdocforwardprefix|:
%    \begin{macrocode}
\newcommand{\childdocredirect}[2][]
{
  \begingroup
    \if?#1?
      \def\childdoctmp{\childdocforward{#2}}
    \else
      \def\childdoctmp{\childdocforwardprefix{#1}{#2}}
    \fi
    \expandafter
  \endgroup
  \childdoctmp
}
%    \end{macrocode}

%\iffalse
%</package>
%\fi
%
\endinput
|\\
|\childdocforwardprefix[|\textit{main}|]{|\textit{prefix}|}{|\textit{dest}|}|
\end{tabular}
\end{center}
%
the destination file is determined by a pattern
depending on the current file:
To make this work, the current file must be called
`{\textit{prefix}\hspace{0.2em}\textit{suffix}}'
with \textit{prefix} matching precisely the argument.
Processing is then passed on to the file
`{\textit{dest}\hspace{0.2em}\textit{suffix}}'.
Surely, the same effect is achieved by
directly specifying the
argument `{\textit{dest}\hspace{0.2em}\textit{suffix}}'
in the first form.
However, that requires to set up a different file
for each child. With the alternative form of the command
all these files can have exactly the same content
which simplifies setting them up and maintaining them.

For example, the following file |draft.tex|
with a compilation flag |\version| as described in \secref{sec:flags}
compiles the main document as a draft:
%
\begin{center}
\begin{tabular}{l}
|\def\version{draft}|\\
|% \iffalse
%
% childdoc.dtx Copyright (C) 2017-2018 Niklas Beisert
%
% This work may be distributed and/or modified under the
% conditions of the LaTeX Project Public License, either version 1.3
% of this license or (at your option) any later version.
% The latest version of this license is in
%   http://www.latex-project.org/lppl.txt
% and version 1.3 or later is part of all distributions of LaTeX
% version 2005/12/01 or later.
%
% This work has the LPPL maintenance status `maintained'.
%
% The Current Maintainer of this work is Niklas Beisert.
%
% This work consists of the files childdoc.dtx and childdoc.ins
% and the derived files childdoc.def and cdocsamp.tex with
% cdocsch1.tex, cdocsch2.tex, cdocsdrf.tex, cdocsfn1.tex, cdocsfn2.tex.
%
%<package>\ifdefined\childdocmain\endinput\fi
%<package>\ProvidesFile{childdoc.def}[2018/12/30 v2.0 child document driver]
%<samplemain>\ProvidesFile{cdocsamp.tex}[2018/12/30 v2.0 sample for childdoc]
%<*driver>
%\ProvidesFile{childdoc.drv}[2018/12/30 v2.0 childdoc reference manual file]
\PassOptionsToClass{10pt,a4paper}{article}
\documentclass{ltxdoc}

\usepackage[margin=35mm]{geometry}
\usepackage{hyperref}
\usepackage{hyperxmp}
\usepackage[usenames]{color}

\hypersetup{colorlinks=true}
\hypersetup{pdfstartview=FitH}
\hypersetup{pdfpagemode=UseNone}
\hypersetup{pdfsource={}}
\hypersetup{pdflang={en-UK}}
\hypersetup{pdfcopyright={Copyright 2017-2018 Niklas Beisert.
  This work may be distributed and/or modified under the
  conditions of the LaTeX Project Public License, either version 1.3
  of this license or (at your option) any later version.}}
\hypersetup{pdflicenseurl={http://www.latex-project.org/lppl.txt}}
\hypersetup{pdfcontactaddress={ETH Zurich, ITP, HIT K,
  Wolfgang-Pauli-Strasse 27}}
\hypersetup{pdfcontactpostcode={8093}}
\hypersetup{pdfcontactcity={Zurich}}
\hypersetup{pdfcontactcountry={Switzerland}}
\hypersetup{pdfcontactemail={nbeisert@itp.phys.ethz.ch}}
\hypersetup{pdfcontacturl={http://people.phys.ethz.ch/\xmptilde nbeisert/}}

\newcommand{\secref}[1]{\hyperref[#1]{section \ref*{#1}}}

\parskip1ex
\parindent0pt
\let\olditemize\itemize
\def\itemize{\olditemize\parskip0pt}

\begin{document}

\title{The \textsf{childdoc} Package}
\hypersetup{pdftitle={The childdoc Package}}
\author{Niklas Beisert\\[2ex]
  Institut f\"ur Theoretische Physik\\
  Eidgen\"ossische Technische Hochschule Z\"urich\\
  Wolfgang-Pauli-Strasse 27, 8093 Z\"urich, Switzerland\\[1ex]
  \href{mailto:nbeisert@itp.phys.ethz.ch}
  {\texttt{nbeisert@itp.phys.ethz.ch}}}
\hypersetup{pdfauthor={Niklas Beisert}}
\hypersetup{pdfsubject={Manual for the LaTeX2e Package childdoc}}
\date{30 December 2018, \textsf{v2.0}}
\maketitle

\begin{abstract}\noindent
\textsf{childdoc} is a \LaTeXe{} package
that enables the direct compilation
of document sections included by |\include|
to individual files.
\end{abstract}

\begingroup
\parskip0ex
\tableofcontents
\endgroup

%%%%%%%%%%%%%%%%%%%%%%%%%%%%%%%%%%%%%%%%%%%%%%%%%%%%%%%%%%%%%%%%%%%%%%%%%%%%%%%%
%%%%%%%%%%%%%%%%%%%%%%%%%%%%%%%%%%%%%%%%%%%%%%%%%%%%%%%%%%%%%%%%%%%%%%%%%%%%%%%%
\section{Introduction}

\LaTeX{} provides a mechanism to structure a large document (such as a book)
into a main file and several child files (containing the chapters)
using the |\include| command.
This mechanism is beneficial for documents
which span hundreds of pages in order to
make the source file(s) more manageable.
Moreover, compilation can be restricted to
selected child files by means of the |\includeonly| command.
The latter feature can be used to reduce the compilation time while editing
(this was significantly more useful in the earlier days of \LaTeX{})
or to generate a smaller document which is easier to navigate.
Another application of |\includeonly| is to generate
documents consisting of selected parts of the complete document.

However, there are a few drawbacks of the plain |\include| mechanism:
\begin{itemize}
\item
The child files cannot be compiled on their own,
they can only be compiled via the main file.
A naive editing environment
(such as a text editor with an option
to have the current file processed by \LaTeX)
may require one to switch to the main file before compiling;
attempting to compile the child file produces errors.
\item
The main file must be modified (each time)
to adjust the |\includeonly| command
to the present needs. This easily leaves the main file in a messy state.
\item
The generated document will always carry the filename
of the main document. This is inconvenient if
several child files are to be compiled and
to be kept for distribution.
\end{itemize}

The present package provides a simple interface
to make child files individually compilable by \LaTeX{}.
Compiling a child file then has the same effect as compiling
the main file with an |\includeonly| command
to select the appropriate child.
Moreover the generated document will carry the name of the child
rather than the main file.
This resolves all three above issues.

This feature is meant to make the editing of books,
thesis documents and lecture notes somewhat more convenient.
However, the package can also be used efficiently for
composing a series of documents (such as exercise sheets)
which are typically distributed individually.
It then assists the author in generating the individual documents
(potentially in different versions)
as well as a document containing the collected series.
Another application is in developing style files
or other kinds of included material
where compilation of the style file could redirect
to a sample or test file.

%%%%%%%%%%%%%%%%%%%%%%%%%%%%%%%%%%%%%%%%%%%%%%%%%%%%%%%%%%%%%%%%%%%%%%%%%%%%%%%%
%%%%%%%%%%%%%%%%%%%%%%%%%%%%%%%%%%%%%%%%%%%%%%%%%%%%%%%%%%%%%%%%%%%%%%%%%%%%%%%%
\section{Usage}

First of all, the package \textsf{childdoc} is \emph{not} a standard
\LaTeXe{} |.sty| style file! Therefore it needs to be invoked in
a non-standard way.

%%%%%%%%%%%%%%%%%%%%%%%%%%%%%%%%%%%%%%%%%%%%%%%%%%%%%%%%%%%%%%%%%%%%%%%%%%%%%%%%
\subsection{Included Files}
\label{sec:include}

%%%%%%%%%%%%%%%%%%%%%%%%%%%%%%%%%%%%%%%%
\DescribeMacro{\childdocmain}
To use the package, add the commands
\begin{center}
\begin{tabular}{l}
|\input{childdoc.def}|\\
|\childdocmain{}|\\
\end{tabular}
\end{center}
at the very top of the main \LaTeX{} file,
in particular \emph{before} the |\documentclass| statement!
The argument of |\childdocmain| should be left empty
(but it must be present).

%%%%%%%%%%%%%%%%%%%%%%%%%%%%%%%%%%%%%%%%
\DescribeMacro{\childdocof}
Furthermore, add the commands
\begin{center}
\begin{tabular}{l}
|\input{childdoc.def}|\\
|\childdocof{|\textit{main}|}|\\
\end{tabular}
\end{center}
at the top of every child file \textit{child}
which is included by |\include{|\textit{child}|}|
from within the main file
(or at least for those files to be compiled individually).
The argument \textit{main} must be the filename of the main file.

There are a couple of
considerations in setting up the main and child documents:

%%%%%%%%%%%%%%%%%%%%%%%%%%%%%%%%%%%%%%%%
\paragraph{Restrictions.}

Please note the following restrictions:
\begin{itemize}
\item
|\childdocmain| must be called with one argument \textit{main}
to ensure compatibility with earlier version of the package.
It must either be empty (|\childdocmain{}|)
or precisely match the filename of the main file in which it is specified.
See \secref{sec:detection} for further information.
\item
The filename \textit{main} must be specified without the |.tex| extension.
\item
The filename \textit{main} is case sensitive
(even in case-insensitive file systems)
due to internal string comparison.
\item
The argument \textit{main} should be fully expanded, it cannot be a macro.
\item
Subdirectories and special characters should be avoided in filenames.
\item
The command |\childdocmain{|\textit{main}|}| must be followed by a whitespace.
It should not be followed immediately by another command
or by a comment mark `|%|'.
This is because the \TeX{} parser reads the token immediately following
the argument of |\childdocmain| and puts it
at the beginning of every child section;
however, a white\-space is ignored.
\end{itemize}

%%%%%%%%%%%%%%%%%%%%%%%%%%%%%%%%%%%%%%%%
\paragraph{Content of Main File.}

It is advisable to place all content in the child files included by |\include|.
Any output contained in the main file will appear in all child documents
unless suppressed manually;
it cannot be suppressed automatically by the |\includeonly| directive
and thus should normally be avoided.
A method to include some content in the main file
by means of conditional processing is described in \secref{sec:conditional}.

%%%%%%%%%%%%%%%%%%%%%%%%%%%%%%%%%%%%%%%%
\paragraph{Page Numbering.}

When only a part of the document is compiled,
the appropriate numbering of pages
(as well as other status parameters)
is determined from the |.aux| files.
The latter contain information from previous passes.
However this information needs to propagate through
all intermediate child documents.
Therefore the page numbering in child documents may well
be inconsistent until the complete document is compiled at least once.

A useful (if unconventional) way to always ensure a consistent
page numbering is to restart the numbering in each child document
and denote the pages by `\textit{child}|.|\textit{page}'
where \textit{child} represents the chapter/section number of the child file.
This can be achieved by the command
|\numberwithin{page}{|\textit{child}|}|
of the \textsf{amsmath} package
where \textit{child} can be |chapter| or |section|
depending on the chosen structuring.
Alternatively, one can modify the macro |\thepage| appropriately
and reset the counter |page| at the start of each child file.

%%%%%%%%%%%%%%%%%%%%%%%%%%%%%%%%%%%%%%%%%%%%%%%%%%%%%%%%%%%%%%%%%%%%%%%%%%%%%%%%
\subsection{Conditional Processing}
\label{sec:conditional}

The package provides a mechanism to compile different versions
of a document. To customise the versions further some conditional processing
can come in handy to distinguish which version is being compiled.
The package provides two macros to describe the compilation context:

%%%%%%%%%%%%%%%%%%%%%%%%%%%%%%%%%%%%%%%%
\DescribeMacro{\ifchilddoc}
The conditional |\ifchilddoc| distinguishes between the compilation of
child documents and the main document:
%
\begin{center}
|\ifchilddoc |\textit{child-code}| |[|\||else |\textit{main-code}]| \||fi|
\end{center}

%%%%%%%%%%%%%%%%%%%%%%%%%%%%%%%%%%%%%%%%
\DescribeMacro{\childdocname}
\DescribeMacro{\childdocjob}
The macro |\childdocname| contains the filename (without extension)
of the main or child file being processed.
Note that |\childdocjob| will always contain the name of the main file.

%%%%%%%%%%%%%%%%%%%%%%%%%%%%%%%%%%%%%%%%
\paragraph{Title Page.}

Conditional processing can be used to include a title or banner page
in the main document when proper precautions are taken.
Importantly, the code in the main file should ensure that the page counter
(as well as other status parameters which are stored in the |.aux| files)
takes the same value after the conditional processing.
Otherwise the page numbers may take divergent values
depending on which part is compiled.

For example, a title page could be declared by:
%
\begin{center}
\begin{tabular}{l}
|\ifchilddoc\||else|\\
|\addtocounter{page}{-1}|\\
\textit{code for title page}\\
|\newpage|\\
|\||fi|
\end{tabular}
\end{center}
%
A banner page for the child documents can be generated by:
%
\begin{center}
\begin{tabular}{l}
|\ifchilddoc|\\
|\addtocounter{page}{-1}|\\
\textit{code for banner page}\\
|\newpage|\\
|\||fi|
\end{tabular}
\end{center}
%
Here one could write a message such as:
\begin{center}
|This is the part \childdocname{} of \childdocjob{}.|
\end{center}

%%%%%%%%%%%%%%%%%%%%%%%%%%%%%%%%%%%%%%%%%%%%%%%%%%%%%%%%%%%%%%%%%%%%%%%%%%%%%%%%
\subsection{Flags}
\label{sec:flags}

The package makes it easy to generate different versions
of the main or child documents.
To this end compilation flags can be defined
and assigned different default values.
They will be particularly useful in conjunction
with the forwarding mechanism described in \secref{sec:forward}.

For example, it may be useful to have a flag |\version|
which can be set to |draft| or |final|.
The document source will contain some conditional code
depending on the value of |\version|.
Suppose further, the flag should default to |final| for the main file
and to |draft| for child files
which is a natural assignment for editing the document.
This is achieved by placing the following code
in the preamble of the main document
(below the |\childdocmain| directive):
%
\begin{center}
\begin{tabular}{l}
|\ifchilddoc|\\
|\providecommand{\version}{draft}|\\
|\||else|\\
|\providecommand{\version}{final}|\\
|\||fi|
\end{tabular}
\end{center}
%
The definition by |\providecommand| makes sure
that previous definitions are not overwritten.
Further statements |\providecommand{\version}{...}|
can thus be added before the above code to override it.

For the main file, one might add a line
(between |\childdocmain| and the above block)
%
\begin{center}
|%\ifchilddoc\||else\providecommand{\version}{draft}\||fi|
\end{center}
%
which can be uncommented to produce a draft version.
Likewise one can add a line to the very top of a child file
(above the |\childdocof{|\textit{main}|}| directive)
%
\begin{center}
|%\providecommand{\version}{final}|
\end{center}
%
which can be uncommented to produce the final version of this child document.

%%%%%%%%%%%%%%%%%%%%%%%%%%%%%%%%%%%%%%%%%%%%%%%%%%%%%%%%%%%%%%%%%%%%%%%%%%%%%%%%
\subsection{Forwarding}
\label{sec:forward}

Different versions of the main or child documents
using compilation flags as described in \secref{sec:flags}
can be (permanently) stored in different files
for convenient compilation, viewing and distribution.
To this end, the package defines a command
to pass on compilation to a different file:

%%%%%%%%%%%%%%%%%%%%%%%%%%%%%%%%%%%%%%%%
\DescribeMacro{\childdocforward}
The command |\childdocforward| redirects processing to
another source file:
%
\begin{center}
\begin{tabular}{l}
|\input{childdoc.def}|\\
|\childdocforward[|\textit{main}|]{|\textit{dest}|}|\\
\end{tabular}
\end{center}
%
The argument \textit{dest} is the destination file
(without extension).
It should be the main file or one of the child files.
Note that further \textsf{childdoc} directives
such as |\childdocof| and |\childdocforward|
in the indicated file will be processed in this form.
The optional argument \textit{main}
passes on directly to the main file \textit{main}
while pretending to compile the child \textit{dest}.
This form behaves as if \textit{dest}
issues |\childdocof{|\textit{main}|}| right away,
and no further \textsf{childdoc} directives will be processed.

%%%%%%%%%%%%%%%%%%%%%%%%%%%%%%%%%%%%%%%%
\DescribeMacro{\...prefix}
In the alternative form |\childdocforwardprefix|,
%
\begin{center}
\begin{tabular}{l}
|\input{childdoc.def}|\\
|\childdocforwardprefix[|\textit{main}|]{|\textit{prefix}|}{|\textit{dest}|}|
\end{tabular}
\end{center}
%
the destination file is determined by a pattern
depending on the current file:
To make this work, the current file must be called
`{\textit{prefix}\hspace{0.2em}\textit{suffix}}'
with \textit{prefix} matching precisely the argument.
Processing is then passed on to the file
`{\textit{dest}\hspace{0.2em}\textit{suffix}}'.
Surely, the same effect is achieved by
directly specifying the
argument `{\textit{dest}\hspace{0.2em}\textit{suffix}}'
in the first form.
However, that requires to set up a different file
for each child. With the alternative form of the command
all these files can have exactly the same content
which simplifies setting them up and maintaining them.

For example, the following file |draft.tex|
with a compilation flag |\version| as described in \secref{sec:flags}
compiles the main document as a draft:
%
\begin{center}
\begin{tabular}{l}
|\def\version{draft}|\\
|\input{childdoc.def}|\\
|\childdocforward{|\textit{main}|}|
\end{tabular}
\end{center}
%
Likewise, the following files |final|\textit{nn}|.tex|
compile the final version of the child document
|child|\textit{nn}|.tex|:
%
\begin{center}
\begin{tabular}{l}
|\def\version{final}|\\
|\input{childdoc.def}|\\
|\childdocforwardprefix{final}{child}|
\end{tabular}
\end{center}
%

Note that when several versions of a main file and/or of each child file
are to be generated, it may be convenient to set up a |Makefile| or
shell script to automatise the process.

%%%%%%%%%%%%%%%%%%%%%%%%%%%%%%%%%%%%%%%%%%%%%%%%%%%%%%%%%%%%%%%%%%%%%%%%%%%%%%%%
\subsection{Command Line Processing}
\label{sec:commandline}

The effect of redirection files can also be achieved by invoking
the \LaTeX{} compiler with a more elaborate command line.
Most conveniently this should be done as part
of a shell script or a |Makefile|.

When using \textsf{childdoc} in the main file, the following
command lines effectively perform a redirection
(note that depending on the shell being used,
backslashes may have to be doubled: `|\|' $\to$ `|\\|'):
%
\begin{center}
|... -jobname "|\textit{target}|" |\\|"|[\textit{flags}]%
|\input{childdoc.def}\childdocforward[|\textit{main}|]{|\textit{dest}|}"|
\end{center}
%
Here \textit{target} is the name of the output file,
\textit{main} is the name of the main file
and \textit{dest} is the name of the main or child file to be processed
(all filenames without extensions).
The optional argument \textit{main} can be omitted
if \textit{main} matches \textit{dest}.
Optionally, compilation \textit{flags} can be defined via |\def| commands.
This command line makes the \TeX{} engine believe
it is compiling the file \textit{target}
whose content is specified as the latter parameter.
The provided code then forwards the processing to
\textit{main} or \textit{dest} as described in \secref{sec:forward}.

%%%%%%%%%%%%%%%%%%%%%%%%%%%%%%%%%%%%%%%%%%%%%%%%%%%%%%%%%%%%%%%%%%%%%%%%%%%%%%%%
\subsection{Include by Input}
\label{sec:input}

Including child documents by |\include| has some restrictions by design.
Most notably, the content of a child document always occupies
its own set of pages; pages cannot be shared between child documents.
Usually, this behaviour makes perfect sense
because each child document contain an essential part of the document.
However, in some situations it may be desirable to compose
a document from a collection of parts
without having mandatory page breaks between then.
For this case, the package
provides a mechanism to include parts
by |\input| which can also be processed individually.
However, by construction this mechanism
requires manual handling of the content to be output.

%%%%%%%%%%%%%%%%%%%%%%%%%%%%%%%%%%%%%%%%
\DescribeMacro{\ifchilddocmanual}
The main file should be prepared as usual, see \secref{sec:include}.
However, the document body must make a distinction
between processing of an individual part and of the main document, e.g.:
%
\begin{center}
\begin{tabular}{l}
|\ifchilddocmanual|\\
|\input{\childdocname}|\\
|\||else|\\
\textit{document body with }|\input{|\textit{part}|}|\\
|\||fi|
\end{tabular}
\end{center}
%
The conditional |\ifchilddocmanual| is true whenever
a part to be included by |\input| is being compiled,
and the name of the part is stored in |\childdocname|.

%%%%%%%%%%%%%%%%%%%%%%%%%%%%%%%%%%%%%%%%
\DescribeMacro{\childdocby}
Each part to be included by |\input| should start with:
%
\begin{center}
\begin{tabular}{l}
|\input{childdoc.def}|\\
|\childdocby{|\textit{main}|}|\\
\end{tabular}
\end{center}
%
The directive |\childdocby| is similar to |\childdocof|
described in \secref{sec:include},
but the subsequent selection of content must be done manually.
To that end, both |\ifchilddoc| and |\ifchilddocmanual|
will be true upon processing of a part,
and the name of the part is stored in |\childdocname|.
Note that |\jobname| will be set to the filename of the current part
so that each part receives an individual |.aux| file
that does not interfere with the |.aux| file(s) of the main document.
This behaviour can be altered by the alternative form
|\childdocby[*]{|\textit{main}|}| (with a non-empty optional argument)
which uses the |.aux| file of the main document
by setting |\jobname| to \textit{main}.

%%%%%%%%%%%%%%%%%%%%%%%%%%%%%%%%%%%%%%%%%%%%%%%%%%%%%%%%%%%%%%%%%%%%%%%%%%%%%%%%
\subsection{Driver Development}
\label{sec:driver}

The \textsf{childdoc} mechanism can also be use for the development
of definition files such as \LaTeX{} styles or classes.
This case differs from the above setup with multiple parts
included by |\include| in that no |\includeonly| should be invoked.
This can be achieved by starting the include file
(before |\ProvidesPackage|) with:
%
\begin{center}
\begin{tabular}{l}
|\input{childdoc.def}|\\
|\childdocforward{|\textit{main}|}|\\
\end{tabular}
\end{center}
%
or alternatively with:
%
\begin{center}
\begin{tabular}{l}
|\input{childdoc.def}|\\
|\childdocby{|\textit{main}|}|\\
\end{tabular}
\end{center}
%
Both forms have slightly different effects as described above.
The main file is prepared as usual, see \secref{sec:include}.

%%%%%%%%%%%%%%%%%%%%%%%%%%%%%%%%%%%%%%%%%%%%%%%%%%%%%%%%%%%%%%%%%%%%%%%%%%%%%%%%
\subsection{Legacy Detection}
\label{sec:detection}

The directive |\childdocmain| in the main file can detect
whether the complete document or merely a child is to be compiled
even without using the directive |\childdocof|.
This method is deprecated because it is less robust
and there is no compelling reason to use it;
it is merely provided for backward compatibility
and it may be removed in future versions.

If the detection mechanism is to be used,
it is mandatory to correctly specify
the filename of the main file as the argument of |\childdocmain|:
%
\begin{center}
\begin{tabular}{l}
|\input{childdoc.def}|\\
|\childdocmain{|\textit{main}|}|\\
\end{tabular}
\end{center}
%
If |\jobname| does not match the argument \textit{main} of |\childdocmain|,
it is assumed that |\jobname| points to the child file to be compiled.
When using |\childdocmain| with the main file specified as argument,
it suffices to start a child file
with just |\input{|\textit{main}|}|
without loading of the package and using |\childdocof|.
If instead all processing is done
with the appropriate \textsf{childdoc} directives,
the argument of \textit{main} of |\childdocmain| can be empty.

An alternative version of the command line processing described
in \secref{sec:commandline} using the detection mechanism reads:
%
\begin{center}
|... -jobname "|\textit{target}|" "|[\textit{flags}]%
[|\def\jobname{|\textit{dest}|}|]|\input{|\textit{main}|}"|
\end{center}

%%%%%%%%%%%%%%%%%%%%%%%%%%%%%%%%%%%%%%%%%%%%%%%%%%%%%%%%%%%%%%%%%%%%%%%%%%%%%%%%
\subsection{Manual Code}
\label{sec:manual}

In case one cannot be certain whether the definitions file |childdoc.def|
is installed on the target \TeX{} distribution
and one prefers not to ship it,
it is conceivable to paste a few relevant commands into the sources.

To that end, drop all statements |\input{childdoc.def}|
and perform the replacements as outlined below.
Instead of |\childdocmain{|\textit{main}|}| add the following code
to the top of the main file:
%
\begin{center}
\begin{tabular}{l}
|\||ifdefined\childdocname\endinput\||fi\newif\ifchilddoc|\\
|\edef\childdocname{\scantokens\expandafter{\jobname\noexpand}}|\\
|\def\childdocmain{|\textit{main}|}\||ifx\childdocmain\childdocname\||else|\\
|\childdoctrue\includeonly{\childdocname}\let\jobname\childdocmain\||fi|\\
\end{tabular}
\end{center}
%
Instead of |\childdocof{|\textit{main}|}| just include the main file
at the top of each child file:
%
\begin{center}
|\input{|\textit{main}|}|
\end{center}
%
A simple redirection |\childdocforward{|\textit{dest}|}| is achieved by:
%
\begin{center}
|\def\jobname{|\textit{dest}|}\input{\jobname}|
\end{center}
%
The redirection with prefix
|\childdocforwardprefix[|\textit{prefix}|]{|\textit{dest}|}|
is accomplished by:
%
\begin{center}
\begin{tabular}{l}
|{\edef\jobname{\scantokens\expandafter{\jobname\noexpand}}|\\
|\def\redirectjob |\textit{prefix}|#1~~~{\gdef\jobname{|\textit{dest}|#1}}|\\
|\expandafter\redirectjob\jobname~~~}\input{\jobname}|
\end{tabular}
\end{center}

In an alternative approach,
child documents can be compiled by a specific command line
without additional code or specific definitions:
%
\begin{center}
|... -jobname "|\textit{target}|" "|[\textit{flags}]%
|\includeonly{|\textit{dest}|}\input{|\textit{main}|}"|
\end{center}
%

%%%%%%%%%%%%%%%%%%%%%%%%%%%%%%%%%%%%%%%%%%%%%%%%%%%%%%%%%%%%%%%%%%%%%%%%%%%%%%%%
%%%%%%%%%%%%%%%%%%%%%%%%%%%%%%%%%%%%%%%%%%%%%%%%%%%%%%%%%%%%%%%%%%%%%%%%%%%%%%%%
\section{Information}

%%%%%%%%%%%%%%%%%%%%%%%%%%%%%%%%%%%%%%%%%%%%%%%%%%%%%%%%%%%%%%%%%%%%%%%%%%%%%%%%
\subsection{Copyright}

Copyright \copyright{} 2017--2018 Niklas Beisert

This work may be distributed and/or modified under the
conditions of the \LaTeX{} Project Public License, either version 1.3
of this license or (at your option) any later version.
The latest version of this license is in
  \url{http://www.latex-project.org/lppl.txt}
and version 1.3 or later is part of all distributions of \LaTeX{}
version 2005/12/01 or later.

This work has the LPPL maintenance status `maintained'.

The Current Maintainer of this work is Niklas Beisert.

This work consists of the files |README.txt|, |childdoc.ins| and |childdoc.dtx|
as well as the derived files |childdoc.def|, |cdocsamp.tex|
with |cdocsch1.tex|, |cdocsch2.tex|, |cdocspt3.tex|, |cdocspt4.tex|,
|cdocsdrf.tex|, |cdocsfn1.tex|, |cdocsfn2.tex|
as well as |childdoc.pdf|.

%%%%%%%%%%%%%%%%%%%%%%%%%%%%%%%%%%%%%%%%%%%%%%%%%%%%%%%%%%%%%%%%%%%%%%%%%%%%%%%%
\subsection{Files and Installation}

The package consists of the files:
%
\begin{center}
\begin{tabular}{ll}
    |README.txt|   & readme file \\
    |childdoc.ins| & installation file \\
    |childdoc.dtx| & source file \\
    |childdoc.def| & definition file \\
    |cdocsamp.tex| & sample main file \\
    |cdocsch1.tex| & sample include file \\
    |cdocsch2.tex| & sample include file \\
    |cdocspt3.tex| & sample part file \\
    |cdocspt4.tex| & sample part file \\
    |cdocsdrf.tex| & sample redirection file \\
    |cdocsfn1.tex| & sample redirection file \\
    |cdocsfn2.tex| & sample redirection file \\
    |childdoc.pdf| & manual
\end{tabular}
\end{center}
%
The distribution consists of the files
|README.txt|, |childdoc.ins| and |childdoc.dtx|.
%
\begin{itemize}
\item
Run (pdf)\LaTeX{} on |childdoc.dtx|
to compile the manual |childdoc.pdf| (this file).
\item
Run \LaTeX{} on |childdoc.ins| to create the definitions file |childdoc.def|
and the sample |cdocsamp.tex| with include files
|cdocsch1.tex|, |cdocsch2.tex|, |cdocspt3.tex|, |cdocspt4.tex|,
|cdocsdrf.tex|, |cdocsfn1.tex|, |cdocsfn2.tex|.
Then copy the file |childdoc.def| to an appropriate directory of your \LaTeX{}
distribution, e.g.\ \textit{texmf-root}|/tex/latex/childdoc|.
\end{itemize}

%%%%%%%%%%%%%%%%%%%%%%%%%%%%%%%%%%%%%%%%%%%%%%%%%%%%%%%%%%%%%%%%%%%%%%%%%%%%%%%%
\subsection{Related CTAN Packages}

There are several other packages which offer a similar functionality:
%
\begin{itemize}
\item
The packages
\href{http://ctan.org/pkg/docmute}{\textsf{docmute}},
\href{http://ctan.org/pkg/includex}{\textsf{includex}} and
\href{http://ctan.org/pkg/standalone}{\textsf{standalone}}
provide commands to include only the document body of
a child file thus allowing both files to be compiled individually.
\item
The packages \href{http://ctan.org/pkg/subdocs}{\textsf{subdocs}}
and \href{http://ctan.org/pkg/subfiles}{\textsf{subfiles}}
provide structures in which the main and child documents can be
encapsulated and allowing them to be compiled individually.
The inclusion mechanism is different from the conventional |\include|.
\item
The package \href{http://ctan.org/pkg/combine}{\textsf{combine}}
is an elaborate solution to combine several documents into one.
\end{itemize}
%
See also the CTAN topic \href{http://ctan.org/topic/subdocs}{\textsf{subdocs}}
for further related packages.
The present package differs from the above solutions in that
a document structure constructed with the conventional |\include| mechanism
just needs two extra commands at the top of every file
such that all constituent files can be compiled individually.

%%%%%%%%%%%%%%%%%%%%%%%%%%%%%%%%%%%%%%%%%%%%%%%%%%%%%%%%%%%%%%%%%%%%%%%%%%%%%%%%
%\subsection{Feature Suggestions}
%
%The following is a list of features which may be useful for future
%versions of this package:
%%
%\begin{itemize}
%\item
%\ldots
%\end{itemize}

%%%%%%%%%%%%%%%%%%%%%%%%%%%%%%%%%%%%%%%%%%%%%%%%%%%%%%%%%%%%%%%%%%%%%%%%%%%%%%%%
\subsection{Revision History}

%%%%%%%%%%%%%%%%%%%%%%%%%%%%%%%%%%%%%%%%
\paragraph{v2.0:} 2018/12/30

\begin{itemize}
\item
immediate forward processing
\item
added |\childdocby| mechanism
\item
manual restructured
\end{itemize}

%%%%%%%%%%%%%%%%%%%%%%%%%%%%%%%%%%%%%%%%
\paragraph{v1.6:} 2018/01/17

\begin{itemize}
\item
application for development of include files
\item
corrections to manual
\end{itemize}

%%%%%%%%%%%%%%%%%%%%%%%%%%%%%%%%%%%%%%%%
\paragraph{v1.5:} 2017/05/21

\begin{itemize}
\item
more complete structuring introduced
\item
|\childdocof| introduced
\item
|\childdoc| renamed to |\childdocmain|
\item
|\childredirect| renamed to |\childdocforward| and |\childdocforwardprefix|
and functionality expanded
\end{itemize}

%%%%%%%%%%%%%%%%%%%%%%%%%%%%%%%%%%%%%%%%
\paragraph{v1.0:} 2017/04/27

\begin{itemize}
\item
manual and install package
\item
first version published on CTAN
\end{itemize}

%%%%%%%%%%%%%%%%%%%%%%%%%%%%%%%%%%%%%%%%
\paragraph{v0.6:} 2017/04/26

\begin{itemize}
\item
redirection mechanism added
\end{itemize}

%%%%%%%%%%%%%%%%%%%%%%%%%%%%%%%%%%%%%%%%
\paragraph{v0.5:} 2017/04/26

\begin{itemize}
\item
functionality in definition file
\end{itemize}


%%%%%%%%%%%%%%%%%%%%%%%%%%%%%%%%%%%%%%%%%%%%%%%%%%%%%%%%%%%%%%%%%%%%%%%%%%%%%%%%
%%%%%%%%%%%%%%%%%%%%%%%%%%%%%%%%%%%%%%%%%%%%%%%%%%%%%%%%%%%%%%%%%%%%%%%%%%%%%%%%
%%%%%%%%%%%%%%%%%%%%%%%%%%%%%%%%%%%%%%%%%%%%%%%%%%%%%%%%%%%%%%%%%%%%%%%%%%%%%%%%
\appendix

\settowidth\MacroIndent{\rmfamily\scriptsize 000\ }

 \DocInput{childdoc.dtx}

\end{document}
%</driver>
% \fi
%
% %%%%%%%%%%%%%%%%%%%%%%%%%%%%%%%%%%%%%%%%%%%%%%%%%%%%%%%%%%%%%%%%%%%%%%%%%%%%%%
% %%%%%%%%%%%%%%%%%%%%%%%%%%%%%%%%%%%%%%%%%%%%%%%%%%%%%%%%%%%%%%%%%%%%%%%%%%%%%%
% \section{Sample}
%\iffalse
%<*samplemain>
%\fi
%
% The following presents a sample document
% with two chapters, two parts, a title page,
% a compile flag as well as three forwarding files to set the flag.
% It consists of eight |.tex| files:
% \begin{center}
% \begin{tabular}{ll}
% |cdocsamp.tex|&main file\\
% |cdocsch1.tex|&include file for chapter 1\\
% |cdocsch2.tex|&include file for chapter 2\\
% |cdocspt3.tex|&include file for part 3\\
% |cdocspt4.tex|&include file for part 4\\
% |cdocsdrf.tex|&forwarding file for main file in draft mode\\
% |cdocsfi1.tex|&forwarding file for final version of chapter 1\\
% |cdocsfi2.tex|&forwarding file for final version of chapter 2\\
% \end{tabular}
% \end{center}
% Each of the eight files can be compiled directly by the \LaTeX{} compiler.
%
% %%%%%%%%%%%%%%%%%%%%%%%%%%%%%%%%%%%%%%
% \paragraph{Main File.}
%
% The main file is called |cdocsamp.tex|.
%
% Load the \textsf{childdoc} definitions and
% declare the filename for the main document:
%    \begin{macrocode}
\input{childdoc.def}
\childdocmain{}
%    \end{macrocode}

% Optional override for |\version| flag:
%    \begin{macrocode}
%%\ifchilddoc\else\providecommand{\version}{draft}\fi
%    \end{macrocode}

% Define the default values for the |\version| flag
% (|final| for the main file and |draft| for childs):
%    \begin{macrocode}
\ifchilddoc
\providecommand{\version}{draft}
\else
\providecommand{\version}{final}
\fi
%    \end{macrocode}

% Load the standard document class:
%    \begin{macrocode}
\documentclass[12pt]{article}
%    \end{macrocode}

% Start the document body:
%    \begin{macrocode}
\begin{document}
%    \end{macrocode}

% Declare a title page.
% Print title, part of document being processed and version flag:
%    \begin{macrocode}
\addtocounter{page}{-1}
\begin{center}
{\LARGE\bfseries{}childdoc example\par}
\vspace{1cm}
\ifchilddoc
\ifchilddocmanual part\else chapter\fi:
`\childdocname' of `\childdocjob'\par
\else
main document: `\childdocjob'\par
\fi
version: \version\par
\end{center}
\newpage
%    \end{macrocode}

% Manually include selected file,
% otherwise process as usual:
%    \begin{macrocode}
\ifchilddocmanual
\section*{part `\childdocname'}
\input{\childdocname}
\else
%    \end{macrocode}

% Include the two chapters:
%    \begin{macrocode}
\include{cdocsch1}
\include{cdocsch2}
%    \end{macrocode}

% Include the two parts unless only chapters should be displayed:
%    \begin{macrocode}
\ifchilddoc\else
\section{part three}
\input{cdocspt3}
\section{part four}
\input{cdocspt4}
\fi
%    \end{macrocode}

% Process as usual until here:
%    \begin{macrocode}
\fi
%    \end{macrocode}

% End of document body:
%    \begin{macrocode}
\end{document}
%    \end{macrocode}
%\iffalse
%</samplemain>
%\fi
%
% %%%%%%%%%%%%%%%%%%%%%%%%%%%%%%%%%%%%%%
% \paragraph{Chapter Include Files.}
%
% The include files are called |cdocsch1.tex| and |cdocsch2.tex|.
%
%\iffalse
%<*samplechap1|samplechap2>
%\fi

% Optional override for |\version| flag:
%    \begin{macrocode}
%%\providecommand{\version}{final}
%    \end{macrocode}

% Include the main document:
%    \begin{macrocode}
\input{childdoc.def}
\childdocof{cdocsamp}
%    \end{macrocode}

%\iffalse
%</samplechap1|samplechap2>
%\fi
%
%\iffalse
%<*samplechap1>
%\fi
% Some text for chapter 1:
%    \begin{macrocode}
\section{one}
some text in chapter one
%    \end{macrocode}

%\iffalse
%</samplechap1>
%\fi
% Some text for chapter 2:
%\iffalse
%<*samplechap2>
%\fi
%    \begin{macrocode}
\section{two}
more text in chapter two
%    \end{macrocode}

%\iffalse
%</samplechap2>
%\fi
%
% %%%%%%%%%%%%%%%%%%%%%%%%%%%%%%%%%%%%%%
% \paragraph{Part Include Files.}
%
% The include files are called |cdocspt3.tex| and |cdocspt4.tex|.
%
%\iffalse
%<*samplepart3|samplepart4>
%\fi

% Optional override for |\version| flag:
%    \begin{macrocode}
%%\providecommand{\version}{final}
%    \end{macrocode}

% Include the main document:
%    \begin{macrocode}
\input{childdoc.def}
\childdocby{cdocsamp}
%    \end{macrocode}

%\iffalse
%</samplepart3|samplepart4>
%\fi
%
%\iffalse
%<*samplepart3>
%\fi
% Some text for part 3:
%    \begin{macrocode}
some text in part three
%    \end{macrocode}

%\iffalse
%</samplepart3>
%\fi
% Some text for part 4:
%\iffalse
%<*samplepart4>
%\fi
%    \begin{macrocode}
more text in part four
%    \end{macrocode}

%\iffalse
%</samplepart4>
%\fi
%
% %%%%%%%%%%%%%%%%%%%%%%%%%%%%%%%%%%%%%%
% \paragraph{Forwarding for a Complete Draft.}
%
% The following forwarding file |cdocsdrf.tex|
% compiles the main document in draft mode:
%\iffalse
%<*sampledraft>
%\fi
%    \begin{macrocode}
\def\version{draft}
\input{childdoc.def}
\childdocforward{cdocsamp}
%    \end{macrocode}

%\iffalse
%</sampledraft>
%\fi
%
% %%%%%%%%%%%%%%%%%%%%%%%%%%%%%%%%%%%%%%
% \paragraph{Forwarding for Final Version of the Chapters.}
%
% The following forwarding files |cdocsfn1.tex| and |cdocsfn2.tex|
% (with identical content)
% compile the final versions of the child documents
% |cdocsch1.tex| and |cdocsch2.tex|, respectively:
%\iffalse
%<*samplefinal>
%\fi
%    \begin{macrocode}
\def\version{final}
\input{childdoc.def}
\childdocforwardprefix[cdocsamp]{cdocsfn}{cdocsch}
%    \end{macrocode}

%\iffalse
%</samplefinal>
%\fi
%
% %%%%%%%%%%%%%%%%%%%%%%%%%%%%%%%%%%%%%%
% \paragraph{Command Line Processing.}
%
% The following three command lines generate the output files
% |cdocscld|, |cdocscl1| and |cdocscl2|
% which should be identical to
% |cdocsdrf|, |cdocsch1| and |cdocsfn2|, respectively:
% \begin{center}
% \begin{tabular}{l}
% |latex -jobname cdocscld \|\\
% |  "\def\version{draft}\input{childdoc.def}\childdocforward{cdocsamp}"|\\
% |latex -jobname cdocscl1 \|\\
% |  "\input{childdoc.def}\childdocforward[cdocsamp]{cdocsch1}"|\\
% |latex -jobname cdocscl2 \|\\
% |  "\def\version{final}\input{childdoc.def}\childdocforward{cdocsch2}"|
% \end{tabular}
% \end{center}
% Note that the trailing backslash on each first line
% merely continues the input to the second line
% (for convenient cut ant paste).
% Furthermore, the command |latex| can be replaced by any
% of its alternative versions such as |pdflatex|.
%
% %%%%%%%%%%%%%%%%%%%%%%%%%%%%%%%%%%%%%%%%%%%%%%%%%%%%%%%%%%%%%%%%%%%%%%%%%%%%%%
% %%%%%%%%%%%%%%%%%%%%%%%%%%%%%%%%%%%%%%%%%%%%%%%%%%%%%%%%%%%%%%%%%%%%%%%%%%%%%%
% \section{Implementation}
%\iffalse
%<*package>
%\fi
%
% This section describes the definitions file |childdoc.def|.

% The definitions cannot be loaded using |\usepackage| or |\RequirePackage|
% which has a mechanism to prevent loading a style file more than once.
% When loading the definitions by means of |\input|
% multiple instances have to be prevented manually:
%\iffalse
%This code needs to be before the `\ProvidesFile' directive
%which is defined at the beginning of this file.
%Therefore it is also placed there and commented out here.
%</package>
%<*discard>
%\fi
%    \begin{macrocode}
\ifdefined\childdocmain\endinput\fi
%    \end{macrocode}
%\iffalse
%</discard>
%<*package>
%\fi
%
% \macro{\ifchilddoc}
% \macro{\ifchilddocmanual}
% The conditional |\ifchilddoc| tells whether a
% child (true) or main (false) document is being compiled.
% The conditional |\ifchilddocmanual| tells whether
% the |\includeonly| mechanism is used (false) or
% the selection of child files must be performed manually (true).
% The definitions initialise to false:
%    \begin{macrocode}
\newif\ifchilddoc
\newif\ifchilddocmanual
%    \end{macrocode}

% \macro{\childdocname}
% \macro{\childdocjob}
% The macro |\childdocname| stores the name of the main document
% to be compiled. The macro |\childdocjob| stores the name of
% the document on which the \LaTeX{} compiler was originally invoked.
% The content of |\jobname| cannot be compared
% to filenames specified in the source due to different catcodes.
% The following code rescans |\jobname|, stores the result
% in |\childdocname| and saves a copy in |\childdocjob|:
%    \begin{macrocode}
\edef\childdocname{\scantokens\expandafter{\jobname\noexpand}}
\let\childdocjob\childdocname
%    \end{macrocode}

% \macro{\childdocdisable}
% The macro |\childdocdisable| prevents the main file
% from being processed more than once.
% At this stage, the main document command |\childdocmain|
% is assumed to be called once again where it should do nothing.
% Any subsequent call to it should prevent
% a secondary processing of the main document
% It overwrites the forwarding commands
% |\childdocof| and |\childdocforward|
% with empty macros to prevent further inclusions of the main document:
%    \begin{macrocode}
\newcommand{\childdocdisable}
{
  \renewcommand{\childdocmain}[1]{\renewcommand{\childdocmain}[1]{\endinput}}
  \renewcommand{\childdocof}[1]{}
  \renewcommand{\childdocby}[2][]{}
  \renewcommand{\childdocforward}[2][]{}
  \renewcommand{\childdocdisable}{}
}
%    \end{macrocode}

% \macro{\childdocmain}
% The macro |\childdocmain| is to be called at the top of the main file
% with nothing or the main filename (without extension) as argument.
% First, it breaks loops.
% If the argument is not empty and does not match |\childdocname|
% (which is set by the first inclusion of |childdoc.def|),
% |\ifchilddoc| is set to true, |\includeonly| is applied to the child file
% and |\jobname| is set to the main file
% (for proper handling of |.aux| files):
%    \begin{macrocode}
\newcommand{\childdocmain}[1]
{
  \childdocdisable\childdocmain{}
  \if?#1?\else
    \begingroup
      \def\childdoctmp{#1}
      \ifx\childdoctmp\childdocname
        \def\childdoctmp{}
      \else
        \def\childdoctmp
        {
          \childdoctrue
          \includeonly{\childdocname}
          \def\childdocjob{#1}
          \def\jobname{#1}
        }
      \fi
      \expandafter
    \endgroup
    \childdoctmp
  \fi
}
%    \end{macrocode}

% \macro{\childdocof}
% The command |\childdocof| redirects
% compilation to the main file |#1|.
%    \begin{macrocode}
\newcommand{\childdocof}[1]
{
  \childdocdisable
  \childdoctrue
  \includeonly{\childdocname}
  \def\jobname{#1}
  \def\childdocjob{#1}
  \input{#1}
}
%    \end{macrocode}

% \macro{\childdocby}
% The command |\childdocby| ....
%    \begin{macrocode}
\newcommand{\childdocby}[2][]
{
  \childdocdisable
  \childdoctrue
  \childdocmanualtrue
  \if?#1?\else
    \def\jobname{#2}
  \fi
  \def\childdocjob{#2}
  \input{#2}
  \endinput
}
%    \end{macrocode}

% \macro{\childdocforward}
% The command |\childdocforward| redirects
% compilation to the main file or
% (if the optional argument is given) a child file.
% Parameters are set as if the main file
% or a child file starting with |\childdocof| was compiled.
% Then compilation is handed over to the main file:
%    \begin{macrocode}
\newcommand{\childdocforward}[2][]
{
  \begingroup
    \if?#1?
      \def\childdoctmp
      {
        \def\childdocname{#2}
        \def\childdocjob{#2}
        \def\jobname{#2}
        \input{#2}
        \endinput
      }
    \else
      \def\childdoctmp
      {
        \childdocdisable
        \def\childdocname{#2}
        \childdoctrue
        \includeonly{#2}
        \def\childdocjob{#1}
        \def\jobname{#1}
        \input{#1}
        \endinput
      }
    \fi
    \expandafter
  \endgroup
  \childdoctmp
}
%    \end{macrocode}

% \macro{\childdocforwardprefix}
% The command |\childdocforwardprefix| redirects
% compilation to the main or a child file by means of a pattern.
% The prefix |#1| in the current filename is replaced by |#2|
% and the suffix of the current filename is kept
% (it is assumed that the filename does not contain the substring `|~~~|'
% which is used as a delimiter).
% Compilation is handed over to the new file by |\childdocforward|:
%    \begin{macrocode}
\newcommand{\childdocforwardprefix}[3][]
{
  \begingroup
    \def\childdocextract #2##1~~~{\def\childdoctmp{\childdocforward[#1]{#3##1}}}
    \expandafter\childdocextract\childdocname~~~
    \expandafter
  \endgroup
  \childdoctmp
}
%    \end{macrocode}

% \macro{\childdoc}
% The deprecated macro |\childdoc| is a legacy version of |\childdocmain|:
%    \begin{macrocode}
\newcommand{\childdoc}{\childdocmain}
%    \end{macrocode}

% \macro{\childdocredirect}
% The deprecated macro |\childdocredirect| is a legacy version
% of |\childdocforward| and |\childdocforwardprefix|:
%    \begin{macrocode}
\newcommand{\childdocredirect}[2][]
{
  \begingroup
    \if?#1?
      \def\childdoctmp{\childdocforward{#2}}
    \else
      \def\childdoctmp{\childdocforwardprefix{#1}{#2}}
    \fi
    \expandafter
  \endgroup
  \childdoctmp
}
%    \end{macrocode}

%\iffalse
%</package>
%\fi
%
\endinput
|\\
|\childdocforward{|\textit{main}|}|
\end{tabular}
\end{center}
%
Likewise, the following files |final|\textit{nn}|.tex|
compile the final version of the child document
|child|\textit{nn}|.tex|:
%
\begin{center}
\begin{tabular}{l}
|\def\version{final}|\\
|% \iffalse
%
% childdoc.dtx Copyright (C) 2017-2018 Niklas Beisert
%
% This work may be distributed and/or modified under the
% conditions of the LaTeX Project Public License, either version 1.3
% of this license or (at your option) any later version.
% The latest version of this license is in
%   http://www.latex-project.org/lppl.txt
% and version 1.3 or later is part of all distributions of LaTeX
% version 2005/12/01 or later.
%
% This work has the LPPL maintenance status `maintained'.
%
% The Current Maintainer of this work is Niklas Beisert.
%
% This work consists of the files childdoc.dtx and childdoc.ins
% and the derived files childdoc.def and cdocsamp.tex with
% cdocsch1.tex, cdocsch2.tex, cdocsdrf.tex, cdocsfn1.tex, cdocsfn2.tex.
%
%<package>\ifdefined\childdocmain\endinput\fi
%<package>\ProvidesFile{childdoc.def}[2018/12/30 v2.0 child document driver]
%<samplemain>\ProvidesFile{cdocsamp.tex}[2018/12/30 v2.0 sample for childdoc]
%<*driver>
%\ProvidesFile{childdoc.drv}[2018/12/30 v2.0 childdoc reference manual file]
\PassOptionsToClass{10pt,a4paper}{article}
\documentclass{ltxdoc}

\usepackage[margin=35mm]{geometry}
\usepackage{hyperref}
\usepackage{hyperxmp}
\usepackage[usenames]{color}

\hypersetup{colorlinks=true}
\hypersetup{pdfstartview=FitH}
\hypersetup{pdfpagemode=UseNone}
\hypersetup{pdfsource={}}
\hypersetup{pdflang={en-UK}}
\hypersetup{pdfcopyright={Copyright 2017-2018 Niklas Beisert.
  This work may be distributed and/or modified under the
  conditions of the LaTeX Project Public License, either version 1.3
  of this license or (at your option) any later version.}}
\hypersetup{pdflicenseurl={http://www.latex-project.org/lppl.txt}}
\hypersetup{pdfcontactaddress={ETH Zurich, ITP, HIT K,
  Wolfgang-Pauli-Strasse 27}}
\hypersetup{pdfcontactpostcode={8093}}
\hypersetup{pdfcontactcity={Zurich}}
\hypersetup{pdfcontactcountry={Switzerland}}
\hypersetup{pdfcontactemail={nbeisert@itp.phys.ethz.ch}}
\hypersetup{pdfcontacturl={http://people.phys.ethz.ch/\xmptilde nbeisert/}}

\newcommand{\secref}[1]{\hyperref[#1]{section \ref*{#1}}}

\parskip1ex
\parindent0pt
\let\olditemize\itemize
\def\itemize{\olditemize\parskip0pt}

\begin{document}

\title{The \textsf{childdoc} Package}
\hypersetup{pdftitle={The childdoc Package}}
\author{Niklas Beisert\\[2ex]
  Institut f\"ur Theoretische Physik\\
  Eidgen\"ossische Technische Hochschule Z\"urich\\
  Wolfgang-Pauli-Strasse 27, 8093 Z\"urich, Switzerland\\[1ex]
  \href{mailto:nbeisert@itp.phys.ethz.ch}
  {\texttt{nbeisert@itp.phys.ethz.ch}}}
\hypersetup{pdfauthor={Niklas Beisert}}
\hypersetup{pdfsubject={Manual for the LaTeX2e Package childdoc}}
\date{30 December 2018, \textsf{v2.0}}
\maketitle

\begin{abstract}\noindent
\textsf{childdoc} is a \LaTeXe{} package
that enables the direct compilation
of document sections included by |\include|
to individual files.
\end{abstract}

\begingroup
\parskip0ex
\tableofcontents
\endgroup

%%%%%%%%%%%%%%%%%%%%%%%%%%%%%%%%%%%%%%%%%%%%%%%%%%%%%%%%%%%%%%%%%%%%%%%%%%%%%%%%
%%%%%%%%%%%%%%%%%%%%%%%%%%%%%%%%%%%%%%%%%%%%%%%%%%%%%%%%%%%%%%%%%%%%%%%%%%%%%%%%
\section{Introduction}

\LaTeX{} provides a mechanism to structure a large document (such as a book)
into a main file and several child files (containing the chapters)
using the |\include| command.
This mechanism is beneficial for documents
which span hundreds of pages in order to
make the source file(s) more manageable.
Moreover, compilation can be restricted to
selected child files by means of the |\includeonly| command.
The latter feature can be used to reduce the compilation time while editing
(this was significantly more useful in the earlier days of \LaTeX{})
or to generate a smaller document which is easier to navigate.
Another application of |\includeonly| is to generate
documents consisting of selected parts of the complete document.

However, there are a few drawbacks of the plain |\include| mechanism:
\begin{itemize}
\item
The child files cannot be compiled on their own,
they can only be compiled via the main file.
A naive editing environment
(such as a text editor with an option
to have the current file processed by \LaTeX)
may require one to switch to the main file before compiling;
attempting to compile the child file produces errors.
\item
The main file must be modified (each time)
to adjust the |\includeonly| command
to the present needs. This easily leaves the main file in a messy state.
\item
The generated document will always carry the filename
of the main document. This is inconvenient if
several child files are to be compiled and
to be kept for distribution.
\end{itemize}

The present package provides a simple interface
to make child files individually compilable by \LaTeX{}.
Compiling a child file then has the same effect as compiling
the main file with an |\includeonly| command
to select the appropriate child.
Moreover the generated document will carry the name of the child
rather than the main file.
This resolves all three above issues.

This feature is meant to make the editing of books,
thesis documents and lecture notes somewhat more convenient.
However, the package can also be used efficiently for
composing a series of documents (such as exercise sheets)
which are typically distributed individually.
It then assists the author in generating the individual documents
(potentially in different versions)
as well as a document containing the collected series.
Another application is in developing style files
or other kinds of included material
where compilation of the style file could redirect
to a sample or test file.

%%%%%%%%%%%%%%%%%%%%%%%%%%%%%%%%%%%%%%%%%%%%%%%%%%%%%%%%%%%%%%%%%%%%%%%%%%%%%%%%
%%%%%%%%%%%%%%%%%%%%%%%%%%%%%%%%%%%%%%%%%%%%%%%%%%%%%%%%%%%%%%%%%%%%%%%%%%%%%%%%
\section{Usage}

First of all, the package \textsf{childdoc} is \emph{not} a standard
\LaTeXe{} |.sty| style file! Therefore it needs to be invoked in
a non-standard way.

%%%%%%%%%%%%%%%%%%%%%%%%%%%%%%%%%%%%%%%%%%%%%%%%%%%%%%%%%%%%%%%%%%%%%%%%%%%%%%%%
\subsection{Included Files}
\label{sec:include}

%%%%%%%%%%%%%%%%%%%%%%%%%%%%%%%%%%%%%%%%
\DescribeMacro{\childdocmain}
To use the package, add the commands
\begin{center}
\begin{tabular}{l}
|\input{childdoc.def}|\\
|\childdocmain{}|\\
\end{tabular}
\end{center}
at the very top of the main \LaTeX{} file,
in particular \emph{before} the |\documentclass| statement!
The argument of |\childdocmain| should be left empty
(but it must be present).

%%%%%%%%%%%%%%%%%%%%%%%%%%%%%%%%%%%%%%%%
\DescribeMacro{\childdocof}
Furthermore, add the commands
\begin{center}
\begin{tabular}{l}
|\input{childdoc.def}|\\
|\childdocof{|\textit{main}|}|\\
\end{tabular}
\end{center}
at the top of every child file \textit{child}
which is included by |\include{|\textit{child}|}|
from within the main file
(or at least for those files to be compiled individually).
The argument \textit{main} must be the filename of the main file.

There are a couple of
considerations in setting up the main and child documents:

%%%%%%%%%%%%%%%%%%%%%%%%%%%%%%%%%%%%%%%%
\paragraph{Restrictions.}

Please note the following restrictions:
\begin{itemize}
\item
|\childdocmain| must be called with one argument \textit{main}
to ensure compatibility with earlier version of the package.
It must either be empty (|\childdocmain{}|)
or precisely match the filename of the main file in which it is specified.
See \secref{sec:detection} for further information.
\item
The filename \textit{main} must be specified without the |.tex| extension.
\item
The filename \textit{main} is case sensitive
(even in case-insensitive file systems)
due to internal string comparison.
\item
The argument \textit{main} should be fully expanded, it cannot be a macro.
\item
Subdirectories and special characters should be avoided in filenames.
\item
The command |\childdocmain{|\textit{main}|}| must be followed by a whitespace.
It should not be followed immediately by another command
or by a comment mark `|%|'.
This is because the \TeX{} parser reads the token immediately following
the argument of |\childdocmain| and puts it
at the beginning of every child section;
however, a white\-space is ignored.
\end{itemize}

%%%%%%%%%%%%%%%%%%%%%%%%%%%%%%%%%%%%%%%%
\paragraph{Content of Main File.}

It is advisable to place all content in the child files included by |\include|.
Any output contained in the main file will appear in all child documents
unless suppressed manually;
it cannot be suppressed automatically by the |\includeonly| directive
and thus should normally be avoided.
A method to include some content in the main file
by means of conditional processing is described in \secref{sec:conditional}.

%%%%%%%%%%%%%%%%%%%%%%%%%%%%%%%%%%%%%%%%
\paragraph{Page Numbering.}

When only a part of the document is compiled,
the appropriate numbering of pages
(as well as other status parameters)
is determined from the |.aux| files.
The latter contain information from previous passes.
However this information needs to propagate through
all intermediate child documents.
Therefore the page numbering in child documents may well
be inconsistent until the complete document is compiled at least once.

A useful (if unconventional) way to always ensure a consistent
page numbering is to restart the numbering in each child document
and denote the pages by `\textit{child}|.|\textit{page}'
where \textit{child} represents the chapter/section number of the child file.
This can be achieved by the command
|\numberwithin{page}{|\textit{child}|}|
of the \textsf{amsmath} package
where \textit{child} can be |chapter| or |section|
depending on the chosen structuring.
Alternatively, one can modify the macro |\thepage| appropriately
and reset the counter |page| at the start of each child file.

%%%%%%%%%%%%%%%%%%%%%%%%%%%%%%%%%%%%%%%%%%%%%%%%%%%%%%%%%%%%%%%%%%%%%%%%%%%%%%%%
\subsection{Conditional Processing}
\label{sec:conditional}

The package provides a mechanism to compile different versions
of a document. To customise the versions further some conditional processing
can come in handy to distinguish which version is being compiled.
The package provides two macros to describe the compilation context:

%%%%%%%%%%%%%%%%%%%%%%%%%%%%%%%%%%%%%%%%
\DescribeMacro{\ifchilddoc}
The conditional |\ifchilddoc| distinguishes between the compilation of
child documents and the main document:
%
\begin{center}
|\ifchilddoc |\textit{child-code}| |[|\||else |\textit{main-code}]| \||fi|
\end{center}

%%%%%%%%%%%%%%%%%%%%%%%%%%%%%%%%%%%%%%%%
\DescribeMacro{\childdocname}
\DescribeMacro{\childdocjob}
The macro |\childdocname| contains the filename (without extension)
of the main or child file being processed.
Note that |\childdocjob| will always contain the name of the main file.

%%%%%%%%%%%%%%%%%%%%%%%%%%%%%%%%%%%%%%%%
\paragraph{Title Page.}

Conditional processing can be used to include a title or banner page
in the main document when proper precautions are taken.
Importantly, the code in the main file should ensure that the page counter
(as well as other status parameters which are stored in the |.aux| files)
takes the same value after the conditional processing.
Otherwise the page numbers may take divergent values
depending on which part is compiled.

For example, a title page could be declared by:
%
\begin{center}
\begin{tabular}{l}
|\ifchilddoc\||else|\\
|\addtocounter{page}{-1}|\\
\textit{code for title page}\\
|\newpage|\\
|\||fi|
\end{tabular}
\end{center}
%
A banner page for the child documents can be generated by:
%
\begin{center}
\begin{tabular}{l}
|\ifchilddoc|\\
|\addtocounter{page}{-1}|\\
\textit{code for banner page}\\
|\newpage|\\
|\||fi|
\end{tabular}
\end{center}
%
Here one could write a message such as:
\begin{center}
|This is the part \childdocname{} of \childdocjob{}.|
\end{center}

%%%%%%%%%%%%%%%%%%%%%%%%%%%%%%%%%%%%%%%%%%%%%%%%%%%%%%%%%%%%%%%%%%%%%%%%%%%%%%%%
\subsection{Flags}
\label{sec:flags}

The package makes it easy to generate different versions
of the main or child documents.
To this end compilation flags can be defined
and assigned different default values.
They will be particularly useful in conjunction
with the forwarding mechanism described in \secref{sec:forward}.

For example, it may be useful to have a flag |\version|
which can be set to |draft| or |final|.
The document source will contain some conditional code
depending on the value of |\version|.
Suppose further, the flag should default to |final| for the main file
and to |draft| for child files
which is a natural assignment for editing the document.
This is achieved by placing the following code
in the preamble of the main document
(below the |\childdocmain| directive):
%
\begin{center}
\begin{tabular}{l}
|\ifchilddoc|\\
|\providecommand{\version}{draft}|\\
|\||else|\\
|\providecommand{\version}{final}|\\
|\||fi|
\end{tabular}
\end{center}
%
The definition by |\providecommand| makes sure
that previous definitions are not overwritten.
Further statements |\providecommand{\version}{...}|
can thus be added before the above code to override it.

For the main file, one might add a line
(between |\childdocmain| and the above block)
%
\begin{center}
|%\ifchilddoc\||else\providecommand{\version}{draft}\||fi|
\end{center}
%
which can be uncommented to produce a draft version.
Likewise one can add a line to the very top of a child file
(above the |\childdocof{|\textit{main}|}| directive)
%
\begin{center}
|%\providecommand{\version}{final}|
\end{center}
%
which can be uncommented to produce the final version of this child document.

%%%%%%%%%%%%%%%%%%%%%%%%%%%%%%%%%%%%%%%%%%%%%%%%%%%%%%%%%%%%%%%%%%%%%%%%%%%%%%%%
\subsection{Forwarding}
\label{sec:forward}

Different versions of the main or child documents
using compilation flags as described in \secref{sec:flags}
can be (permanently) stored in different files
for convenient compilation, viewing and distribution.
To this end, the package defines a command
to pass on compilation to a different file:

%%%%%%%%%%%%%%%%%%%%%%%%%%%%%%%%%%%%%%%%
\DescribeMacro{\childdocforward}
The command |\childdocforward| redirects processing to
another source file:
%
\begin{center}
\begin{tabular}{l}
|\input{childdoc.def}|\\
|\childdocforward[|\textit{main}|]{|\textit{dest}|}|\\
\end{tabular}
\end{center}
%
The argument \textit{dest} is the destination file
(without extension).
It should be the main file or one of the child files.
Note that further \textsf{childdoc} directives
such as |\childdocof| and |\childdocforward|
in the indicated file will be processed in this form.
The optional argument \textit{main}
passes on directly to the main file \textit{main}
while pretending to compile the child \textit{dest}.
This form behaves as if \textit{dest}
issues |\childdocof{|\textit{main}|}| right away,
and no further \textsf{childdoc} directives will be processed.

%%%%%%%%%%%%%%%%%%%%%%%%%%%%%%%%%%%%%%%%
\DescribeMacro{\...prefix}
In the alternative form |\childdocforwardprefix|,
%
\begin{center}
\begin{tabular}{l}
|\input{childdoc.def}|\\
|\childdocforwardprefix[|\textit{main}|]{|\textit{prefix}|}{|\textit{dest}|}|
\end{tabular}
\end{center}
%
the destination file is determined by a pattern
depending on the current file:
To make this work, the current file must be called
`{\textit{prefix}\hspace{0.2em}\textit{suffix}}'
with \textit{prefix} matching precisely the argument.
Processing is then passed on to the file
`{\textit{dest}\hspace{0.2em}\textit{suffix}}'.
Surely, the same effect is achieved by
directly specifying the
argument `{\textit{dest}\hspace{0.2em}\textit{suffix}}'
in the first form.
However, that requires to set up a different file
for each child. With the alternative form of the command
all these files can have exactly the same content
which simplifies setting them up and maintaining them.

For example, the following file |draft.tex|
with a compilation flag |\version| as described in \secref{sec:flags}
compiles the main document as a draft:
%
\begin{center}
\begin{tabular}{l}
|\def\version{draft}|\\
|\input{childdoc.def}|\\
|\childdocforward{|\textit{main}|}|
\end{tabular}
\end{center}
%
Likewise, the following files |final|\textit{nn}|.tex|
compile the final version of the child document
|child|\textit{nn}|.tex|:
%
\begin{center}
\begin{tabular}{l}
|\def\version{final}|\\
|\input{childdoc.def}|\\
|\childdocforwardprefix{final}{child}|
\end{tabular}
\end{center}
%

Note that when several versions of a main file and/or of each child file
are to be generated, it may be convenient to set up a |Makefile| or
shell script to automatise the process.

%%%%%%%%%%%%%%%%%%%%%%%%%%%%%%%%%%%%%%%%%%%%%%%%%%%%%%%%%%%%%%%%%%%%%%%%%%%%%%%%
\subsection{Command Line Processing}
\label{sec:commandline}

The effect of redirection files can also be achieved by invoking
the \LaTeX{} compiler with a more elaborate command line.
Most conveniently this should be done as part
of a shell script or a |Makefile|.

When using \textsf{childdoc} in the main file, the following
command lines effectively perform a redirection
(note that depending on the shell being used,
backslashes may have to be doubled: `|\|' $\to$ `|\\|'):
%
\begin{center}
|... -jobname "|\textit{target}|" |\\|"|[\textit{flags}]%
|\input{childdoc.def}\childdocforward[|\textit{main}|]{|\textit{dest}|}"|
\end{center}
%
Here \textit{target} is the name of the output file,
\textit{main} is the name of the main file
and \textit{dest} is the name of the main or child file to be processed
(all filenames without extensions).
The optional argument \textit{main} can be omitted
if \textit{main} matches \textit{dest}.
Optionally, compilation \textit{flags} can be defined via |\def| commands.
This command line makes the \TeX{} engine believe
it is compiling the file \textit{target}
whose content is specified as the latter parameter.
The provided code then forwards the processing to
\textit{main} or \textit{dest} as described in \secref{sec:forward}.

%%%%%%%%%%%%%%%%%%%%%%%%%%%%%%%%%%%%%%%%%%%%%%%%%%%%%%%%%%%%%%%%%%%%%%%%%%%%%%%%
\subsection{Include by Input}
\label{sec:input}

Including child documents by |\include| has some restrictions by design.
Most notably, the content of a child document always occupies
its own set of pages; pages cannot be shared between child documents.
Usually, this behaviour makes perfect sense
because each child document contain an essential part of the document.
However, in some situations it may be desirable to compose
a document from a collection of parts
without having mandatory page breaks between then.
For this case, the package
provides a mechanism to include parts
by |\input| which can also be processed individually.
However, by construction this mechanism
requires manual handling of the content to be output.

%%%%%%%%%%%%%%%%%%%%%%%%%%%%%%%%%%%%%%%%
\DescribeMacro{\ifchilddocmanual}
The main file should be prepared as usual, see \secref{sec:include}.
However, the document body must make a distinction
between processing of an individual part and of the main document, e.g.:
%
\begin{center}
\begin{tabular}{l}
|\ifchilddocmanual|\\
|\input{\childdocname}|\\
|\||else|\\
\textit{document body with }|\input{|\textit{part}|}|\\
|\||fi|
\end{tabular}
\end{center}
%
The conditional |\ifchilddocmanual| is true whenever
a part to be included by |\input| is being compiled,
and the name of the part is stored in |\childdocname|.

%%%%%%%%%%%%%%%%%%%%%%%%%%%%%%%%%%%%%%%%
\DescribeMacro{\childdocby}
Each part to be included by |\input| should start with:
%
\begin{center}
\begin{tabular}{l}
|\input{childdoc.def}|\\
|\childdocby{|\textit{main}|}|\\
\end{tabular}
\end{center}
%
The directive |\childdocby| is similar to |\childdocof|
described in \secref{sec:include},
but the subsequent selection of content must be done manually.
To that end, both |\ifchilddoc| and |\ifchilddocmanual|
will be true upon processing of a part,
and the name of the part is stored in |\childdocname|.
Note that |\jobname| will be set to the filename of the current part
so that each part receives an individual |.aux| file
that does not interfere with the |.aux| file(s) of the main document.
This behaviour can be altered by the alternative form
|\childdocby[*]{|\textit{main}|}| (with a non-empty optional argument)
which uses the |.aux| file of the main document
by setting |\jobname| to \textit{main}.

%%%%%%%%%%%%%%%%%%%%%%%%%%%%%%%%%%%%%%%%%%%%%%%%%%%%%%%%%%%%%%%%%%%%%%%%%%%%%%%%
\subsection{Driver Development}
\label{sec:driver}

The \textsf{childdoc} mechanism can also be use for the development
of definition files such as \LaTeX{} styles or classes.
This case differs from the above setup with multiple parts
included by |\include| in that no |\includeonly| should be invoked.
This can be achieved by starting the include file
(before |\ProvidesPackage|) with:
%
\begin{center}
\begin{tabular}{l}
|\input{childdoc.def}|\\
|\childdocforward{|\textit{main}|}|\\
\end{tabular}
\end{center}
%
or alternatively with:
%
\begin{center}
\begin{tabular}{l}
|\input{childdoc.def}|\\
|\childdocby{|\textit{main}|}|\\
\end{tabular}
\end{center}
%
Both forms have slightly different effects as described above.
The main file is prepared as usual, see \secref{sec:include}.

%%%%%%%%%%%%%%%%%%%%%%%%%%%%%%%%%%%%%%%%%%%%%%%%%%%%%%%%%%%%%%%%%%%%%%%%%%%%%%%%
\subsection{Legacy Detection}
\label{sec:detection}

The directive |\childdocmain| in the main file can detect
whether the complete document or merely a child is to be compiled
even without using the directive |\childdocof|.
This method is deprecated because it is less robust
and there is no compelling reason to use it;
it is merely provided for backward compatibility
and it may be removed in future versions.

If the detection mechanism is to be used,
it is mandatory to correctly specify
the filename of the main file as the argument of |\childdocmain|:
%
\begin{center}
\begin{tabular}{l}
|\input{childdoc.def}|\\
|\childdocmain{|\textit{main}|}|\\
\end{tabular}
\end{center}
%
If |\jobname| does not match the argument \textit{main} of |\childdocmain|,
it is assumed that |\jobname| points to the child file to be compiled.
When using |\childdocmain| with the main file specified as argument,
it suffices to start a child file
with just |\input{|\textit{main}|}|
without loading of the package and using |\childdocof|.
If instead all processing is done
with the appropriate \textsf{childdoc} directives,
the argument of \textit{main} of |\childdocmain| can be empty.

An alternative version of the command line processing described
in \secref{sec:commandline} using the detection mechanism reads:
%
\begin{center}
|... -jobname "|\textit{target}|" "|[\textit{flags}]%
[|\def\jobname{|\textit{dest}|}|]|\input{|\textit{main}|}"|
\end{center}

%%%%%%%%%%%%%%%%%%%%%%%%%%%%%%%%%%%%%%%%%%%%%%%%%%%%%%%%%%%%%%%%%%%%%%%%%%%%%%%%
\subsection{Manual Code}
\label{sec:manual}

In case one cannot be certain whether the definitions file |childdoc.def|
is installed on the target \TeX{} distribution
and one prefers not to ship it,
it is conceivable to paste a few relevant commands into the sources.

To that end, drop all statements |\input{childdoc.def}|
and perform the replacements as outlined below.
Instead of |\childdocmain{|\textit{main}|}| add the following code
to the top of the main file:
%
\begin{center}
\begin{tabular}{l}
|\||ifdefined\childdocname\endinput\||fi\newif\ifchilddoc|\\
|\edef\childdocname{\scantokens\expandafter{\jobname\noexpand}}|\\
|\def\childdocmain{|\textit{main}|}\||ifx\childdocmain\childdocname\||else|\\
|\childdoctrue\includeonly{\childdocname}\let\jobname\childdocmain\||fi|\\
\end{tabular}
\end{center}
%
Instead of |\childdocof{|\textit{main}|}| just include the main file
at the top of each child file:
%
\begin{center}
|\input{|\textit{main}|}|
\end{center}
%
A simple redirection |\childdocforward{|\textit{dest}|}| is achieved by:
%
\begin{center}
|\def\jobname{|\textit{dest}|}\input{\jobname}|
\end{center}
%
The redirection with prefix
|\childdocforwardprefix[|\textit{prefix}|]{|\textit{dest}|}|
is accomplished by:
%
\begin{center}
\begin{tabular}{l}
|{\edef\jobname{\scantokens\expandafter{\jobname\noexpand}}|\\
|\def\redirectjob |\textit{prefix}|#1~~~{\gdef\jobname{|\textit{dest}|#1}}|\\
|\expandafter\redirectjob\jobname~~~}\input{\jobname}|
\end{tabular}
\end{center}

In an alternative approach,
child documents can be compiled by a specific command line
without additional code or specific definitions:
%
\begin{center}
|... -jobname "|\textit{target}|" "|[\textit{flags}]%
|\includeonly{|\textit{dest}|}\input{|\textit{main}|}"|
\end{center}
%

%%%%%%%%%%%%%%%%%%%%%%%%%%%%%%%%%%%%%%%%%%%%%%%%%%%%%%%%%%%%%%%%%%%%%%%%%%%%%%%%
%%%%%%%%%%%%%%%%%%%%%%%%%%%%%%%%%%%%%%%%%%%%%%%%%%%%%%%%%%%%%%%%%%%%%%%%%%%%%%%%
\section{Information}

%%%%%%%%%%%%%%%%%%%%%%%%%%%%%%%%%%%%%%%%%%%%%%%%%%%%%%%%%%%%%%%%%%%%%%%%%%%%%%%%
\subsection{Copyright}

Copyright \copyright{} 2017--2018 Niklas Beisert

This work may be distributed and/or modified under the
conditions of the \LaTeX{} Project Public License, either version 1.3
of this license or (at your option) any later version.
The latest version of this license is in
  \url{http://www.latex-project.org/lppl.txt}
and version 1.3 or later is part of all distributions of \LaTeX{}
version 2005/12/01 or later.

This work has the LPPL maintenance status `maintained'.

The Current Maintainer of this work is Niklas Beisert.

This work consists of the files |README.txt|, |childdoc.ins| and |childdoc.dtx|
as well as the derived files |childdoc.def|, |cdocsamp.tex|
with |cdocsch1.tex|, |cdocsch2.tex|, |cdocspt3.tex|, |cdocspt4.tex|,
|cdocsdrf.tex|, |cdocsfn1.tex|, |cdocsfn2.tex|
as well as |childdoc.pdf|.

%%%%%%%%%%%%%%%%%%%%%%%%%%%%%%%%%%%%%%%%%%%%%%%%%%%%%%%%%%%%%%%%%%%%%%%%%%%%%%%%
\subsection{Files and Installation}

The package consists of the files:
%
\begin{center}
\begin{tabular}{ll}
    |README.txt|   & readme file \\
    |childdoc.ins| & installation file \\
    |childdoc.dtx| & source file \\
    |childdoc.def| & definition file \\
    |cdocsamp.tex| & sample main file \\
    |cdocsch1.tex| & sample include file \\
    |cdocsch2.tex| & sample include file \\
    |cdocspt3.tex| & sample part file \\
    |cdocspt4.tex| & sample part file \\
    |cdocsdrf.tex| & sample redirection file \\
    |cdocsfn1.tex| & sample redirection file \\
    |cdocsfn2.tex| & sample redirection file \\
    |childdoc.pdf| & manual
\end{tabular}
\end{center}
%
The distribution consists of the files
|README.txt|, |childdoc.ins| and |childdoc.dtx|.
%
\begin{itemize}
\item
Run (pdf)\LaTeX{} on |childdoc.dtx|
to compile the manual |childdoc.pdf| (this file).
\item
Run \LaTeX{} on |childdoc.ins| to create the definitions file |childdoc.def|
and the sample |cdocsamp.tex| with include files
|cdocsch1.tex|, |cdocsch2.tex|, |cdocspt3.tex|, |cdocspt4.tex|,
|cdocsdrf.tex|, |cdocsfn1.tex|, |cdocsfn2.tex|.
Then copy the file |childdoc.def| to an appropriate directory of your \LaTeX{}
distribution, e.g.\ \textit{texmf-root}|/tex/latex/childdoc|.
\end{itemize}

%%%%%%%%%%%%%%%%%%%%%%%%%%%%%%%%%%%%%%%%%%%%%%%%%%%%%%%%%%%%%%%%%%%%%%%%%%%%%%%%
\subsection{Related CTAN Packages}

There are several other packages which offer a similar functionality:
%
\begin{itemize}
\item
The packages
\href{http://ctan.org/pkg/docmute}{\textsf{docmute}},
\href{http://ctan.org/pkg/includex}{\textsf{includex}} and
\href{http://ctan.org/pkg/standalone}{\textsf{standalone}}
provide commands to include only the document body of
a child file thus allowing both files to be compiled individually.
\item
The packages \href{http://ctan.org/pkg/subdocs}{\textsf{subdocs}}
and \href{http://ctan.org/pkg/subfiles}{\textsf{subfiles}}
provide structures in which the main and child documents can be
encapsulated and allowing them to be compiled individually.
The inclusion mechanism is different from the conventional |\include|.
\item
The package \href{http://ctan.org/pkg/combine}{\textsf{combine}}
is an elaborate solution to combine several documents into one.
\end{itemize}
%
See also the CTAN topic \href{http://ctan.org/topic/subdocs}{\textsf{subdocs}}
for further related packages.
The present package differs from the above solutions in that
a document structure constructed with the conventional |\include| mechanism
just needs two extra commands at the top of every file
such that all constituent files can be compiled individually.

%%%%%%%%%%%%%%%%%%%%%%%%%%%%%%%%%%%%%%%%%%%%%%%%%%%%%%%%%%%%%%%%%%%%%%%%%%%%%%%%
%\subsection{Feature Suggestions}
%
%The following is a list of features which may be useful for future
%versions of this package:
%%
%\begin{itemize}
%\item
%\ldots
%\end{itemize}

%%%%%%%%%%%%%%%%%%%%%%%%%%%%%%%%%%%%%%%%%%%%%%%%%%%%%%%%%%%%%%%%%%%%%%%%%%%%%%%%
\subsection{Revision History}

%%%%%%%%%%%%%%%%%%%%%%%%%%%%%%%%%%%%%%%%
\paragraph{v2.0:} 2018/12/30

\begin{itemize}
\item
immediate forward processing
\item
added |\childdocby| mechanism
\item
manual restructured
\end{itemize}

%%%%%%%%%%%%%%%%%%%%%%%%%%%%%%%%%%%%%%%%
\paragraph{v1.6:} 2018/01/17

\begin{itemize}
\item
application for development of include files
\item
corrections to manual
\end{itemize}

%%%%%%%%%%%%%%%%%%%%%%%%%%%%%%%%%%%%%%%%
\paragraph{v1.5:} 2017/05/21

\begin{itemize}
\item
more complete structuring introduced
\item
|\childdocof| introduced
\item
|\childdoc| renamed to |\childdocmain|
\item
|\childredirect| renamed to |\childdocforward| and |\childdocforwardprefix|
and functionality expanded
\end{itemize}

%%%%%%%%%%%%%%%%%%%%%%%%%%%%%%%%%%%%%%%%
\paragraph{v1.0:} 2017/04/27

\begin{itemize}
\item
manual and install package
\item
first version published on CTAN
\end{itemize}

%%%%%%%%%%%%%%%%%%%%%%%%%%%%%%%%%%%%%%%%
\paragraph{v0.6:} 2017/04/26

\begin{itemize}
\item
redirection mechanism added
\end{itemize}

%%%%%%%%%%%%%%%%%%%%%%%%%%%%%%%%%%%%%%%%
\paragraph{v0.5:} 2017/04/26

\begin{itemize}
\item
functionality in definition file
\end{itemize}


%%%%%%%%%%%%%%%%%%%%%%%%%%%%%%%%%%%%%%%%%%%%%%%%%%%%%%%%%%%%%%%%%%%%%%%%%%%%%%%%
%%%%%%%%%%%%%%%%%%%%%%%%%%%%%%%%%%%%%%%%%%%%%%%%%%%%%%%%%%%%%%%%%%%%%%%%%%%%%%%%
%%%%%%%%%%%%%%%%%%%%%%%%%%%%%%%%%%%%%%%%%%%%%%%%%%%%%%%%%%%%%%%%%%%%%%%%%%%%%%%%
\appendix

\settowidth\MacroIndent{\rmfamily\scriptsize 000\ }

 \DocInput{childdoc.dtx}

\end{document}
%</driver>
% \fi
%
% %%%%%%%%%%%%%%%%%%%%%%%%%%%%%%%%%%%%%%%%%%%%%%%%%%%%%%%%%%%%%%%%%%%%%%%%%%%%%%
% %%%%%%%%%%%%%%%%%%%%%%%%%%%%%%%%%%%%%%%%%%%%%%%%%%%%%%%%%%%%%%%%%%%%%%%%%%%%%%
% \section{Sample}
%\iffalse
%<*samplemain>
%\fi
%
% The following presents a sample document
% with two chapters, two parts, a title page,
% a compile flag as well as three forwarding files to set the flag.
% It consists of eight |.tex| files:
% \begin{center}
% \begin{tabular}{ll}
% |cdocsamp.tex|&main file\\
% |cdocsch1.tex|&include file for chapter 1\\
% |cdocsch2.tex|&include file for chapter 2\\
% |cdocspt3.tex|&include file for part 3\\
% |cdocspt4.tex|&include file for part 4\\
% |cdocsdrf.tex|&forwarding file for main file in draft mode\\
% |cdocsfi1.tex|&forwarding file for final version of chapter 1\\
% |cdocsfi2.tex|&forwarding file for final version of chapter 2\\
% \end{tabular}
% \end{center}
% Each of the eight files can be compiled directly by the \LaTeX{} compiler.
%
% %%%%%%%%%%%%%%%%%%%%%%%%%%%%%%%%%%%%%%
% \paragraph{Main File.}
%
% The main file is called |cdocsamp.tex|.
%
% Load the \textsf{childdoc} definitions and
% declare the filename for the main document:
%    \begin{macrocode}
\input{childdoc.def}
\childdocmain{}
%    \end{macrocode}

% Optional override for |\version| flag:
%    \begin{macrocode}
%%\ifchilddoc\else\providecommand{\version}{draft}\fi
%    \end{macrocode}

% Define the default values for the |\version| flag
% (|final| for the main file and |draft| for childs):
%    \begin{macrocode}
\ifchilddoc
\providecommand{\version}{draft}
\else
\providecommand{\version}{final}
\fi
%    \end{macrocode}

% Load the standard document class:
%    \begin{macrocode}
\documentclass[12pt]{article}
%    \end{macrocode}

% Start the document body:
%    \begin{macrocode}
\begin{document}
%    \end{macrocode}

% Declare a title page.
% Print title, part of document being processed and version flag:
%    \begin{macrocode}
\addtocounter{page}{-1}
\begin{center}
{\LARGE\bfseries{}childdoc example\par}
\vspace{1cm}
\ifchilddoc
\ifchilddocmanual part\else chapter\fi:
`\childdocname' of `\childdocjob'\par
\else
main document: `\childdocjob'\par
\fi
version: \version\par
\end{center}
\newpage
%    \end{macrocode}

% Manually include selected file,
% otherwise process as usual:
%    \begin{macrocode}
\ifchilddocmanual
\section*{part `\childdocname'}
\input{\childdocname}
\else
%    \end{macrocode}

% Include the two chapters:
%    \begin{macrocode}
\include{cdocsch1}
\include{cdocsch2}
%    \end{macrocode}

% Include the two parts unless only chapters should be displayed:
%    \begin{macrocode}
\ifchilddoc\else
\section{part three}
\input{cdocspt3}
\section{part four}
\input{cdocspt4}
\fi
%    \end{macrocode}

% Process as usual until here:
%    \begin{macrocode}
\fi
%    \end{macrocode}

% End of document body:
%    \begin{macrocode}
\end{document}
%    \end{macrocode}
%\iffalse
%</samplemain>
%\fi
%
% %%%%%%%%%%%%%%%%%%%%%%%%%%%%%%%%%%%%%%
% \paragraph{Chapter Include Files.}
%
% The include files are called |cdocsch1.tex| and |cdocsch2.tex|.
%
%\iffalse
%<*samplechap1|samplechap2>
%\fi

% Optional override for |\version| flag:
%    \begin{macrocode}
%%\providecommand{\version}{final}
%    \end{macrocode}

% Include the main document:
%    \begin{macrocode}
\input{childdoc.def}
\childdocof{cdocsamp}
%    \end{macrocode}

%\iffalse
%</samplechap1|samplechap2>
%\fi
%
%\iffalse
%<*samplechap1>
%\fi
% Some text for chapter 1:
%    \begin{macrocode}
\section{one}
some text in chapter one
%    \end{macrocode}

%\iffalse
%</samplechap1>
%\fi
% Some text for chapter 2:
%\iffalse
%<*samplechap2>
%\fi
%    \begin{macrocode}
\section{two}
more text in chapter two
%    \end{macrocode}

%\iffalse
%</samplechap2>
%\fi
%
% %%%%%%%%%%%%%%%%%%%%%%%%%%%%%%%%%%%%%%
% \paragraph{Part Include Files.}
%
% The include files are called |cdocspt3.tex| and |cdocspt4.tex|.
%
%\iffalse
%<*samplepart3|samplepart4>
%\fi

% Optional override for |\version| flag:
%    \begin{macrocode}
%%\providecommand{\version}{final}
%    \end{macrocode}

% Include the main document:
%    \begin{macrocode}
\input{childdoc.def}
\childdocby{cdocsamp}
%    \end{macrocode}

%\iffalse
%</samplepart3|samplepart4>
%\fi
%
%\iffalse
%<*samplepart3>
%\fi
% Some text for part 3:
%    \begin{macrocode}
some text in part three
%    \end{macrocode}

%\iffalse
%</samplepart3>
%\fi
% Some text for part 4:
%\iffalse
%<*samplepart4>
%\fi
%    \begin{macrocode}
more text in part four
%    \end{macrocode}

%\iffalse
%</samplepart4>
%\fi
%
% %%%%%%%%%%%%%%%%%%%%%%%%%%%%%%%%%%%%%%
% \paragraph{Forwarding for a Complete Draft.}
%
% The following forwarding file |cdocsdrf.tex|
% compiles the main document in draft mode:
%\iffalse
%<*sampledraft>
%\fi
%    \begin{macrocode}
\def\version{draft}
\input{childdoc.def}
\childdocforward{cdocsamp}
%    \end{macrocode}

%\iffalse
%</sampledraft>
%\fi
%
% %%%%%%%%%%%%%%%%%%%%%%%%%%%%%%%%%%%%%%
% \paragraph{Forwarding for Final Version of the Chapters.}
%
% The following forwarding files |cdocsfn1.tex| and |cdocsfn2.tex|
% (with identical content)
% compile the final versions of the child documents
% |cdocsch1.tex| and |cdocsch2.tex|, respectively:
%\iffalse
%<*samplefinal>
%\fi
%    \begin{macrocode}
\def\version{final}
\input{childdoc.def}
\childdocforwardprefix[cdocsamp]{cdocsfn}{cdocsch}
%    \end{macrocode}

%\iffalse
%</samplefinal>
%\fi
%
% %%%%%%%%%%%%%%%%%%%%%%%%%%%%%%%%%%%%%%
% \paragraph{Command Line Processing.}
%
% The following three command lines generate the output files
% |cdocscld|, |cdocscl1| and |cdocscl2|
% which should be identical to
% |cdocsdrf|, |cdocsch1| and |cdocsfn2|, respectively:
% \begin{center}
% \begin{tabular}{l}
% |latex -jobname cdocscld \|\\
% |  "\def\version{draft}\input{childdoc.def}\childdocforward{cdocsamp}"|\\
% |latex -jobname cdocscl1 \|\\
% |  "\input{childdoc.def}\childdocforward[cdocsamp]{cdocsch1}"|\\
% |latex -jobname cdocscl2 \|\\
% |  "\def\version{final}\input{childdoc.def}\childdocforward{cdocsch2}"|
% \end{tabular}
% \end{center}
% Note that the trailing backslash on each first line
% merely continues the input to the second line
% (for convenient cut ant paste).
% Furthermore, the command |latex| can be replaced by any
% of its alternative versions such as |pdflatex|.
%
% %%%%%%%%%%%%%%%%%%%%%%%%%%%%%%%%%%%%%%%%%%%%%%%%%%%%%%%%%%%%%%%%%%%%%%%%%%%%%%
% %%%%%%%%%%%%%%%%%%%%%%%%%%%%%%%%%%%%%%%%%%%%%%%%%%%%%%%%%%%%%%%%%%%%%%%%%%%%%%
% \section{Implementation}
%\iffalse
%<*package>
%\fi
%
% This section describes the definitions file |childdoc.def|.

% The definitions cannot be loaded using |\usepackage| or |\RequirePackage|
% which has a mechanism to prevent loading a style file more than once.
% When loading the definitions by means of |\input|
% multiple instances have to be prevented manually:
%\iffalse
%This code needs to be before the `\ProvidesFile' directive
%which is defined at the beginning of this file.
%Therefore it is also placed there and commented out here.
%</package>
%<*discard>
%\fi
%    \begin{macrocode}
\ifdefined\childdocmain\endinput\fi
%    \end{macrocode}
%\iffalse
%</discard>
%<*package>
%\fi
%
% \macro{\ifchilddoc}
% \macro{\ifchilddocmanual}
% The conditional |\ifchilddoc| tells whether a
% child (true) or main (false) document is being compiled.
% The conditional |\ifchilddocmanual| tells whether
% the |\includeonly| mechanism is used (false) or
% the selection of child files must be performed manually (true).
% The definitions initialise to false:
%    \begin{macrocode}
\newif\ifchilddoc
\newif\ifchilddocmanual
%    \end{macrocode}

% \macro{\childdocname}
% \macro{\childdocjob}
% The macro |\childdocname| stores the name of the main document
% to be compiled. The macro |\childdocjob| stores the name of
% the document on which the \LaTeX{} compiler was originally invoked.
% The content of |\jobname| cannot be compared
% to filenames specified in the source due to different catcodes.
% The following code rescans |\jobname|, stores the result
% in |\childdocname| and saves a copy in |\childdocjob|:
%    \begin{macrocode}
\edef\childdocname{\scantokens\expandafter{\jobname\noexpand}}
\let\childdocjob\childdocname
%    \end{macrocode}

% \macro{\childdocdisable}
% The macro |\childdocdisable| prevents the main file
% from being processed more than once.
% At this stage, the main document command |\childdocmain|
% is assumed to be called once again where it should do nothing.
% Any subsequent call to it should prevent
% a secondary processing of the main document
% It overwrites the forwarding commands
% |\childdocof| and |\childdocforward|
% with empty macros to prevent further inclusions of the main document:
%    \begin{macrocode}
\newcommand{\childdocdisable}
{
  \renewcommand{\childdocmain}[1]{\renewcommand{\childdocmain}[1]{\endinput}}
  \renewcommand{\childdocof}[1]{}
  \renewcommand{\childdocby}[2][]{}
  \renewcommand{\childdocforward}[2][]{}
  \renewcommand{\childdocdisable}{}
}
%    \end{macrocode}

% \macro{\childdocmain}
% The macro |\childdocmain| is to be called at the top of the main file
% with nothing or the main filename (without extension) as argument.
% First, it breaks loops.
% If the argument is not empty and does not match |\childdocname|
% (which is set by the first inclusion of |childdoc.def|),
% |\ifchilddoc| is set to true, |\includeonly| is applied to the child file
% and |\jobname| is set to the main file
% (for proper handling of |.aux| files):
%    \begin{macrocode}
\newcommand{\childdocmain}[1]
{
  \childdocdisable\childdocmain{}
  \if?#1?\else
    \begingroup
      \def\childdoctmp{#1}
      \ifx\childdoctmp\childdocname
        \def\childdoctmp{}
      \else
        \def\childdoctmp
        {
          \childdoctrue
          \includeonly{\childdocname}
          \def\childdocjob{#1}
          \def\jobname{#1}
        }
      \fi
      \expandafter
    \endgroup
    \childdoctmp
  \fi
}
%    \end{macrocode}

% \macro{\childdocof}
% The command |\childdocof| redirects
% compilation to the main file |#1|.
%    \begin{macrocode}
\newcommand{\childdocof}[1]
{
  \childdocdisable
  \childdoctrue
  \includeonly{\childdocname}
  \def\jobname{#1}
  \def\childdocjob{#1}
  \input{#1}
}
%    \end{macrocode}

% \macro{\childdocby}
% The command |\childdocby| ....
%    \begin{macrocode}
\newcommand{\childdocby}[2][]
{
  \childdocdisable
  \childdoctrue
  \childdocmanualtrue
  \if?#1?\else
    \def\jobname{#2}
  \fi
  \def\childdocjob{#2}
  \input{#2}
  \endinput
}
%    \end{macrocode}

% \macro{\childdocforward}
% The command |\childdocforward| redirects
% compilation to the main file or
% (if the optional argument is given) a child file.
% Parameters are set as if the main file
% or a child file starting with |\childdocof| was compiled.
% Then compilation is handed over to the main file:
%    \begin{macrocode}
\newcommand{\childdocforward}[2][]
{
  \begingroup
    \if?#1?
      \def\childdoctmp
      {
        \def\childdocname{#2}
        \def\childdocjob{#2}
        \def\jobname{#2}
        \input{#2}
        \endinput
      }
    \else
      \def\childdoctmp
      {
        \childdocdisable
        \def\childdocname{#2}
        \childdoctrue
        \includeonly{#2}
        \def\childdocjob{#1}
        \def\jobname{#1}
        \input{#1}
        \endinput
      }
    \fi
    \expandafter
  \endgroup
  \childdoctmp
}
%    \end{macrocode}

% \macro{\childdocforwardprefix}
% The command |\childdocforwardprefix| redirects
% compilation to the main or a child file by means of a pattern.
% The prefix |#1| in the current filename is replaced by |#2|
% and the suffix of the current filename is kept
% (it is assumed that the filename does not contain the substring `|~~~|'
% which is used as a delimiter).
% Compilation is handed over to the new file by |\childdocforward|:
%    \begin{macrocode}
\newcommand{\childdocforwardprefix}[3][]
{
  \begingroup
    \def\childdocextract #2##1~~~{\def\childdoctmp{\childdocforward[#1]{#3##1}}}
    \expandafter\childdocextract\childdocname~~~
    \expandafter
  \endgroup
  \childdoctmp
}
%    \end{macrocode}

% \macro{\childdoc}
% The deprecated macro |\childdoc| is a legacy version of |\childdocmain|:
%    \begin{macrocode}
\newcommand{\childdoc}{\childdocmain}
%    \end{macrocode}

% \macro{\childdocredirect}
% The deprecated macro |\childdocredirect| is a legacy version
% of |\childdocforward| and |\childdocforwardprefix|:
%    \begin{macrocode}
\newcommand{\childdocredirect}[2][]
{
  \begingroup
    \if?#1?
      \def\childdoctmp{\childdocforward{#2}}
    \else
      \def\childdoctmp{\childdocforwardprefix{#1}{#2}}
    \fi
    \expandafter
  \endgroup
  \childdoctmp
}
%    \end{macrocode}

%\iffalse
%</package>
%\fi
%
\endinput
|\\
|\childdocforwardprefix{final}{child}|
\end{tabular}
\end{center}
%

Note that when several versions of a main file and/or of each child file
are to be generated, it may be convenient to set up a |Makefile| or
shell script to automatise the process.

%%%%%%%%%%%%%%%%%%%%%%%%%%%%%%%%%%%%%%%%%%%%%%%%%%%%%%%%%%%%%%%%%%%%%%%%%%%%%%%%
\subsection{Command Line Processing}
\label{sec:commandline}

The effect of redirection files can also be achieved by invoking
the \LaTeX{} compiler with a more elaborate command line.
Most conveniently this should be done as part
of a shell script or a |Makefile|.

When using \textsf{childdoc} in the main file, the following
command lines effectively perform a redirection
(note that depending on the shell being used,
backslashes may have to be doubled: `|\|' $\to$ `|\\|'):
%
\begin{center}
|... -jobname "|\textit{target}|" |\\|"|[\textit{flags}]%
|% \iffalse
%
% childdoc.dtx Copyright (C) 2017-2018 Niklas Beisert
%
% This work may be distributed and/or modified under the
% conditions of the LaTeX Project Public License, either version 1.3
% of this license or (at your option) any later version.
% The latest version of this license is in
%   http://www.latex-project.org/lppl.txt
% and version 1.3 or later is part of all distributions of LaTeX
% version 2005/12/01 or later.
%
% This work has the LPPL maintenance status `maintained'.
%
% The Current Maintainer of this work is Niklas Beisert.
%
% This work consists of the files childdoc.dtx and childdoc.ins
% and the derived files childdoc.def and cdocsamp.tex with
% cdocsch1.tex, cdocsch2.tex, cdocsdrf.tex, cdocsfn1.tex, cdocsfn2.tex.
%
%<package>\ifdefined\childdocmain\endinput\fi
%<package>\ProvidesFile{childdoc.def}[2018/12/30 v2.0 child document driver]
%<samplemain>\ProvidesFile{cdocsamp.tex}[2018/12/30 v2.0 sample for childdoc]
%<*driver>
%\ProvidesFile{childdoc.drv}[2018/12/30 v2.0 childdoc reference manual file]
\PassOptionsToClass{10pt,a4paper}{article}
\documentclass{ltxdoc}

\usepackage[margin=35mm]{geometry}
\usepackage{hyperref}
\usepackage{hyperxmp}
\usepackage[usenames]{color}

\hypersetup{colorlinks=true}
\hypersetup{pdfstartview=FitH}
\hypersetup{pdfpagemode=UseNone}
\hypersetup{pdfsource={}}
\hypersetup{pdflang={en-UK}}
\hypersetup{pdfcopyright={Copyright 2017-2018 Niklas Beisert.
  This work may be distributed and/or modified under the
  conditions of the LaTeX Project Public License, either version 1.3
  of this license or (at your option) any later version.}}
\hypersetup{pdflicenseurl={http://www.latex-project.org/lppl.txt}}
\hypersetup{pdfcontactaddress={ETH Zurich, ITP, HIT K,
  Wolfgang-Pauli-Strasse 27}}
\hypersetup{pdfcontactpostcode={8093}}
\hypersetup{pdfcontactcity={Zurich}}
\hypersetup{pdfcontactcountry={Switzerland}}
\hypersetup{pdfcontactemail={nbeisert@itp.phys.ethz.ch}}
\hypersetup{pdfcontacturl={http://people.phys.ethz.ch/\xmptilde nbeisert/}}

\newcommand{\secref}[1]{\hyperref[#1]{section \ref*{#1}}}

\parskip1ex
\parindent0pt
\let\olditemize\itemize
\def\itemize{\olditemize\parskip0pt}

\begin{document}

\title{The \textsf{childdoc} Package}
\hypersetup{pdftitle={The childdoc Package}}
\author{Niklas Beisert\\[2ex]
  Institut f\"ur Theoretische Physik\\
  Eidgen\"ossische Technische Hochschule Z\"urich\\
  Wolfgang-Pauli-Strasse 27, 8093 Z\"urich, Switzerland\\[1ex]
  \href{mailto:nbeisert@itp.phys.ethz.ch}
  {\texttt{nbeisert@itp.phys.ethz.ch}}}
\hypersetup{pdfauthor={Niklas Beisert}}
\hypersetup{pdfsubject={Manual for the LaTeX2e Package childdoc}}
\date{30 December 2018, \textsf{v2.0}}
\maketitle

\begin{abstract}\noindent
\textsf{childdoc} is a \LaTeXe{} package
that enables the direct compilation
of document sections included by |\include|
to individual files.
\end{abstract}

\begingroup
\parskip0ex
\tableofcontents
\endgroup

%%%%%%%%%%%%%%%%%%%%%%%%%%%%%%%%%%%%%%%%%%%%%%%%%%%%%%%%%%%%%%%%%%%%%%%%%%%%%%%%
%%%%%%%%%%%%%%%%%%%%%%%%%%%%%%%%%%%%%%%%%%%%%%%%%%%%%%%%%%%%%%%%%%%%%%%%%%%%%%%%
\section{Introduction}

\LaTeX{} provides a mechanism to structure a large document (such as a book)
into a main file and several child files (containing the chapters)
using the |\include| command.
This mechanism is beneficial for documents
which span hundreds of pages in order to
make the source file(s) more manageable.
Moreover, compilation can be restricted to
selected child files by means of the |\includeonly| command.
The latter feature can be used to reduce the compilation time while editing
(this was significantly more useful in the earlier days of \LaTeX{})
or to generate a smaller document which is easier to navigate.
Another application of |\includeonly| is to generate
documents consisting of selected parts of the complete document.

However, there are a few drawbacks of the plain |\include| mechanism:
\begin{itemize}
\item
The child files cannot be compiled on their own,
they can only be compiled via the main file.
A naive editing environment
(such as a text editor with an option
to have the current file processed by \LaTeX)
may require one to switch to the main file before compiling;
attempting to compile the child file produces errors.
\item
The main file must be modified (each time)
to adjust the |\includeonly| command
to the present needs. This easily leaves the main file in a messy state.
\item
The generated document will always carry the filename
of the main document. This is inconvenient if
several child files are to be compiled and
to be kept for distribution.
\end{itemize}

The present package provides a simple interface
to make child files individually compilable by \LaTeX{}.
Compiling a child file then has the same effect as compiling
the main file with an |\includeonly| command
to select the appropriate child.
Moreover the generated document will carry the name of the child
rather than the main file.
This resolves all three above issues.

This feature is meant to make the editing of books,
thesis documents and lecture notes somewhat more convenient.
However, the package can also be used efficiently for
composing a series of documents (such as exercise sheets)
which are typically distributed individually.
It then assists the author in generating the individual documents
(potentially in different versions)
as well as a document containing the collected series.
Another application is in developing style files
or other kinds of included material
where compilation of the style file could redirect
to a sample or test file.

%%%%%%%%%%%%%%%%%%%%%%%%%%%%%%%%%%%%%%%%%%%%%%%%%%%%%%%%%%%%%%%%%%%%%%%%%%%%%%%%
%%%%%%%%%%%%%%%%%%%%%%%%%%%%%%%%%%%%%%%%%%%%%%%%%%%%%%%%%%%%%%%%%%%%%%%%%%%%%%%%
\section{Usage}

First of all, the package \textsf{childdoc} is \emph{not} a standard
\LaTeXe{} |.sty| style file! Therefore it needs to be invoked in
a non-standard way.

%%%%%%%%%%%%%%%%%%%%%%%%%%%%%%%%%%%%%%%%%%%%%%%%%%%%%%%%%%%%%%%%%%%%%%%%%%%%%%%%
\subsection{Included Files}
\label{sec:include}

%%%%%%%%%%%%%%%%%%%%%%%%%%%%%%%%%%%%%%%%
\DescribeMacro{\childdocmain}
To use the package, add the commands
\begin{center}
\begin{tabular}{l}
|\input{childdoc.def}|\\
|\childdocmain{}|\\
\end{tabular}
\end{center}
at the very top of the main \LaTeX{} file,
in particular \emph{before} the |\documentclass| statement!
The argument of |\childdocmain| should be left empty
(but it must be present).

%%%%%%%%%%%%%%%%%%%%%%%%%%%%%%%%%%%%%%%%
\DescribeMacro{\childdocof}
Furthermore, add the commands
\begin{center}
\begin{tabular}{l}
|\input{childdoc.def}|\\
|\childdocof{|\textit{main}|}|\\
\end{tabular}
\end{center}
at the top of every child file \textit{child}
which is included by |\include{|\textit{child}|}|
from within the main file
(or at least for those files to be compiled individually).
The argument \textit{main} must be the filename of the main file.

There are a couple of
considerations in setting up the main and child documents:

%%%%%%%%%%%%%%%%%%%%%%%%%%%%%%%%%%%%%%%%
\paragraph{Restrictions.}

Please note the following restrictions:
\begin{itemize}
\item
|\childdocmain| must be called with one argument \textit{main}
to ensure compatibility with earlier version of the package.
It must either be empty (|\childdocmain{}|)
or precisely match the filename of the main file in which it is specified.
See \secref{sec:detection} for further information.
\item
The filename \textit{main} must be specified without the |.tex| extension.
\item
The filename \textit{main} is case sensitive
(even in case-insensitive file systems)
due to internal string comparison.
\item
The argument \textit{main} should be fully expanded, it cannot be a macro.
\item
Subdirectories and special characters should be avoided in filenames.
\item
The command |\childdocmain{|\textit{main}|}| must be followed by a whitespace.
It should not be followed immediately by another command
or by a comment mark `|%|'.
This is because the \TeX{} parser reads the token immediately following
the argument of |\childdocmain| and puts it
at the beginning of every child section;
however, a white\-space is ignored.
\end{itemize}

%%%%%%%%%%%%%%%%%%%%%%%%%%%%%%%%%%%%%%%%
\paragraph{Content of Main File.}

It is advisable to place all content in the child files included by |\include|.
Any output contained in the main file will appear in all child documents
unless suppressed manually;
it cannot be suppressed automatically by the |\includeonly| directive
and thus should normally be avoided.
A method to include some content in the main file
by means of conditional processing is described in \secref{sec:conditional}.

%%%%%%%%%%%%%%%%%%%%%%%%%%%%%%%%%%%%%%%%
\paragraph{Page Numbering.}

When only a part of the document is compiled,
the appropriate numbering of pages
(as well as other status parameters)
is determined from the |.aux| files.
The latter contain information from previous passes.
However this information needs to propagate through
all intermediate child documents.
Therefore the page numbering in child documents may well
be inconsistent until the complete document is compiled at least once.

A useful (if unconventional) way to always ensure a consistent
page numbering is to restart the numbering in each child document
and denote the pages by `\textit{child}|.|\textit{page}'
where \textit{child} represents the chapter/section number of the child file.
This can be achieved by the command
|\numberwithin{page}{|\textit{child}|}|
of the \textsf{amsmath} package
where \textit{child} can be |chapter| or |section|
depending on the chosen structuring.
Alternatively, one can modify the macro |\thepage| appropriately
and reset the counter |page| at the start of each child file.

%%%%%%%%%%%%%%%%%%%%%%%%%%%%%%%%%%%%%%%%%%%%%%%%%%%%%%%%%%%%%%%%%%%%%%%%%%%%%%%%
\subsection{Conditional Processing}
\label{sec:conditional}

The package provides a mechanism to compile different versions
of a document. To customise the versions further some conditional processing
can come in handy to distinguish which version is being compiled.
The package provides two macros to describe the compilation context:

%%%%%%%%%%%%%%%%%%%%%%%%%%%%%%%%%%%%%%%%
\DescribeMacro{\ifchilddoc}
The conditional |\ifchilddoc| distinguishes between the compilation of
child documents and the main document:
%
\begin{center}
|\ifchilddoc |\textit{child-code}| |[|\||else |\textit{main-code}]| \||fi|
\end{center}

%%%%%%%%%%%%%%%%%%%%%%%%%%%%%%%%%%%%%%%%
\DescribeMacro{\childdocname}
\DescribeMacro{\childdocjob}
The macro |\childdocname| contains the filename (without extension)
of the main or child file being processed.
Note that |\childdocjob| will always contain the name of the main file.

%%%%%%%%%%%%%%%%%%%%%%%%%%%%%%%%%%%%%%%%
\paragraph{Title Page.}

Conditional processing can be used to include a title or banner page
in the main document when proper precautions are taken.
Importantly, the code in the main file should ensure that the page counter
(as well as other status parameters which are stored in the |.aux| files)
takes the same value after the conditional processing.
Otherwise the page numbers may take divergent values
depending on which part is compiled.

For example, a title page could be declared by:
%
\begin{center}
\begin{tabular}{l}
|\ifchilddoc\||else|\\
|\addtocounter{page}{-1}|\\
\textit{code for title page}\\
|\newpage|\\
|\||fi|
\end{tabular}
\end{center}
%
A banner page for the child documents can be generated by:
%
\begin{center}
\begin{tabular}{l}
|\ifchilddoc|\\
|\addtocounter{page}{-1}|\\
\textit{code for banner page}\\
|\newpage|\\
|\||fi|
\end{tabular}
\end{center}
%
Here one could write a message such as:
\begin{center}
|This is the part \childdocname{} of \childdocjob{}.|
\end{center}

%%%%%%%%%%%%%%%%%%%%%%%%%%%%%%%%%%%%%%%%%%%%%%%%%%%%%%%%%%%%%%%%%%%%%%%%%%%%%%%%
\subsection{Flags}
\label{sec:flags}

The package makes it easy to generate different versions
of the main or child documents.
To this end compilation flags can be defined
and assigned different default values.
They will be particularly useful in conjunction
with the forwarding mechanism described in \secref{sec:forward}.

For example, it may be useful to have a flag |\version|
which can be set to |draft| or |final|.
The document source will contain some conditional code
depending on the value of |\version|.
Suppose further, the flag should default to |final| for the main file
and to |draft| for child files
which is a natural assignment for editing the document.
This is achieved by placing the following code
in the preamble of the main document
(below the |\childdocmain| directive):
%
\begin{center}
\begin{tabular}{l}
|\ifchilddoc|\\
|\providecommand{\version}{draft}|\\
|\||else|\\
|\providecommand{\version}{final}|\\
|\||fi|
\end{tabular}
\end{center}
%
The definition by |\providecommand| makes sure
that previous definitions are not overwritten.
Further statements |\providecommand{\version}{...}|
can thus be added before the above code to override it.

For the main file, one might add a line
(between |\childdocmain| and the above block)
%
\begin{center}
|%\ifchilddoc\||else\providecommand{\version}{draft}\||fi|
\end{center}
%
which can be uncommented to produce a draft version.
Likewise one can add a line to the very top of a child file
(above the |\childdocof{|\textit{main}|}| directive)
%
\begin{center}
|%\providecommand{\version}{final}|
\end{center}
%
which can be uncommented to produce the final version of this child document.

%%%%%%%%%%%%%%%%%%%%%%%%%%%%%%%%%%%%%%%%%%%%%%%%%%%%%%%%%%%%%%%%%%%%%%%%%%%%%%%%
\subsection{Forwarding}
\label{sec:forward}

Different versions of the main or child documents
using compilation flags as described in \secref{sec:flags}
can be (permanently) stored in different files
for convenient compilation, viewing and distribution.
To this end, the package defines a command
to pass on compilation to a different file:

%%%%%%%%%%%%%%%%%%%%%%%%%%%%%%%%%%%%%%%%
\DescribeMacro{\childdocforward}
The command |\childdocforward| redirects processing to
another source file:
%
\begin{center}
\begin{tabular}{l}
|\input{childdoc.def}|\\
|\childdocforward[|\textit{main}|]{|\textit{dest}|}|\\
\end{tabular}
\end{center}
%
The argument \textit{dest} is the destination file
(without extension).
It should be the main file or one of the child files.
Note that further \textsf{childdoc} directives
such as |\childdocof| and |\childdocforward|
in the indicated file will be processed in this form.
The optional argument \textit{main}
passes on directly to the main file \textit{main}
while pretending to compile the child \textit{dest}.
This form behaves as if \textit{dest}
issues |\childdocof{|\textit{main}|}| right away,
and no further \textsf{childdoc} directives will be processed.

%%%%%%%%%%%%%%%%%%%%%%%%%%%%%%%%%%%%%%%%
\DescribeMacro{\...prefix}
In the alternative form |\childdocforwardprefix|,
%
\begin{center}
\begin{tabular}{l}
|\input{childdoc.def}|\\
|\childdocforwardprefix[|\textit{main}|]{|\textit{prefix}|}{|\textit{dest}|}|
\end{tabular}
\end{center}
%
the destination file is determined by a pattern
depending on the current file:
To make this work, the current file must be called
`{\textit{prefix}\hspace{0.2em}\textit{suffix}}'
with \textit{prefix} matching precisely the argument.
Processing is then passed on to the file
`{\textit{dest}\hspace{0.2em}\textit{suffix}}'.
Surely, the same effect is achieved by
directly specifying the
argument `{\textit{dest}\hspace{0.2em}\textit{suffix}}'
in the first form.
However, that requires to set up a different file
for each child. With the alternative form of the command
all these files can have exactly the same content
which simplifies setting them up and maintaining them.

For example, the following file |draft.tex|
with a compilation flag |\version| as described in \secref{sec:flags}
compiles the main document as a draft:
%
\begin{center}
\begin{tabular}{l}
|\def\version{draft}|\\
|\input{childdoc.def}|\\
|\childdocforward{|\textit{main}|}|
\end{tabular}
\end{center}
%
Likewise, the following files |final|\textit{nn}|.tex|
compile the final version of the child document
|child|\textit{nn}|.tex|:
%
\begin{center}
\begin{tabular}{l}
|\def\version{final}|\\
|\input{childdoc.def}|\\
|\childdocforwardprefix{final}{child}|
\end{tabular}
\end{center}
%

Note that when several versions of a main file and/or of each child file
are to be generated, it may be convenient to set up a |Makefile| or
shell script to automatise the process.

%%%%%%%%%%%%%%%%%%%%%%%%%%%%%%%%%%%%%%%%%%%%%%%%%%%%%%%%%%%%%%%%%%%%%%%%%%%%%%%%
\subsection{Command Line Processing}
\label{sec:commandline}

The effect of redirection files can also be achieved by invoking
the \LaTeX{} compiler with a more elaborate command line.
Most conveniently this should be done as part
of a shell script or a |Makefile|.

When using \textsf{childdoc} in the main file, the following
command lines effectively perform a redirection
(note that depending on the shell being used,
backslashes may have to be doubled: `|\|' $\to$ `|\\|'):
%
\begin{center}
|... -jobname "|\textit{target}|" |\\|"|[\textit{flags}]%
|\input{childdoc.def}\childdocforward[|\textit{main}|]{|\textit{dest}|}"|
\end{center}
%
Here \textit{target} is the name of the output file,
\textit{main} is the name of the main file
and \textit{dest} is the name of the main or child file to be processed
(all filenames without extensions).
The optional argument \textit{main} can be omitted
if \textit{main} matches \textit{dest}.
Optionally, compilation \textit{flags} can be defined via |\def| commands.
This command line makes the \TeX{} engine believe
it is compiling the file \textit{target}
whose content is specified as the latter parameter.
The provided code then forwards the processing to
\textit{main} or \textit{dest} as described in \secref{sec:forward}.

%%%%%%%%%%%%%%%%%%%%%%%%%%%%%%%%%%%%%%%%%%%%%%%%%%%%%%%%%%%%%%%%%%%%%%%%%%%%%%%%
\subsection{Include by Input}
\label{sec:input}

Including child documents by |\include| has some restrictions by design.
Most notably, the content of a child document always occupies
its own set of pages; pages cannot be shared between child documents.
Usually, this behaviour makes perfect sense
because each child document contain an essential part of the document.
However, in some situations it may be desirable to compose
a document from a collection of parts
without having mandatory page breaks between then.
For this case, the package
provides a mechanism to include parts
by |\input| which can also be processed individually.
However, by construction this mechanism
requires manual handling of the content to be output.

%%%%%%%%%%%%%%%%%%%%%%%%%%%%%%%%%%%%%%%%
\DescribeMacro{\ifchilddocmanual}
The main file should be prepared as usual, see \secref{sec:include}.
However, the document body must make a distinction
between processing of an individual part and of the main document, e.g.:
%
\begin{center}
\begin{tabular}{l}
|\ifchilddocmanual|\\
|\input{\childdocname}|\\
|\||else|\\
\textit{document body with }|\input{|\textit{part}|}|\\
|\||fi|
\end{tabular}
\end{center}
%
The conditional |\ifchilddocmanual| is true whenever
a part to be included by |\input| is being compiled,
and the name of the part is stored in |\childdocname|.

%%%%%%%%%%%%%%%%%%%%%%%%%%%%%%%%%%%%%%%%
\DescribeMacro{\childdocby}
Each part to be included by |\input| should start with:
%
\begin{center}
\begin{tabular}{l}
|\input{childdoc.def}|\\
|\childdocby{|\textit{main}|}|\\
\end{tabular}
\end{center}
%
The directive |\childdocby| is similar to |\childdocof|
described in \secref{sec:include},
but the subsequent selection of content must be done manually.
To that end, both |\ifchilddoc| and |\ifchilddocmanual|
will be true upon processing of a part,
and the name of the part is stored in |\childdocname|.
Note that |\jobname| will be set to the filename of the current part
so that each part receives an individual |.aux| file
that does not interfere with the |.aux| file(s) of the main document.
This behaviour can be altered by the alternative form
|\childdocby[*]{|\textit{main}|}| (with a non-empty optional argument)
which uses the |.aux| file of the main document
by setting |\jobname| to \textit{main}.

%%%%%%%%%%%%%%%%%%%%%%%%%%%%%%%%%%%%%%%%%%%%%%%%%%%%%%%%%%%%%%%%%%%%%%%%%%%%%%%%
\subsection{Driver Development}
\label{sec:driver}

The \textsf{childdoc} mechanism can also be use for the development
of definition files such as \LaTeX{} styles or classes.
This case differs from the above setup with multiple parts
included by |\include| in that no |\includeonly| should be invoked.
This can be achieved by starting the include file
(before |\ProvidesPackage|) with:
%
\begin{center}
\begin{tabular}{l}
|\input{childdoc.def}|\\
|\childdocforward{|\textit{main}|}|\\
\end{tabular}
\end{center}
%
or alternatively with:
%
\begin{center}
\begin{tabular}{l}
|\input{childdoc.def}|\\
|\childdocby{|\textit{main}|}|\\
\end{tabular}
\end{center}
%
Both forms have slightly different effects as described above.
The main file is prepared as usual, see \secref{sec:include}.

%%%%%%%%%%%%%%%%%%%%%%%%%%%%%%%%%%%%%%%%%%%%%%%%%%%%%%%%%%%%%%%%%%%%%%%%%%%%%%%%
\subsection{Legacy Detection}
\label{sec:detection}

The directive |\childdocmain| in the main file can detect
whether the complete document or merely a child is to be compiled
even without using the directive |\childdocof|.
This method is deprecated because it is less robust
and there is no compelling reason to use it;
it is merely provided for backward compatibility
and it may be removed in future versions.

If the detection mechanism is to be used,
it is mandatory to correctly specify
the filename of the main file as the argument of |\childdocmain|:
%
\begin{center}
\begin{tabular}{l}
|\input{childdoc.def}|\\
|\childdocmain{|\textit{main}|}|\\
\end{tabular}
\end{center}
%
If |\jobname| does not match the argument \textit{main} of |\childdocmain|,
it is assumed that |\jobname| points to the child file to be compiled.
When using |\childdocmain| with the main file specified as argument,
it suffices to start a child file
with just |\input{|\textit{main}|}|
without loading of the package and using |\childdocof|.
If instead all processing is done
with the appropriate \textsf{childdoc} directives,
the argument of \textit{main} of |\childdocmain| can be empty.

An alternative version of the command line processing described
in \secref{sec:commandline} using the detection mechanism reads:
%
\begin{center}
|... -jobname "|\textit{target}|" "|[\textit{flags}]%
[|\def\jobname{|\textit{dest}|}|]|\input{|\textit{main}|}"|
\end{center}

%%%%%%%%%%%%%%%%%%%%%%%%%%%%%%%%%%%%%%%%%%%%%%%%%%%%%%%%%%%%%%%%%%%%%%%%%%%%%%%%
\subsection{Manual Code}
\label{sec:manual}

In case one cannot be certain whether the definitions file |childdoc.def|
is installed on the target \TeX{} distribution
and one prefers not to ship it,
it is conceivable to paste a few relevant commands into the sources.

To that end, drop all statements |\input{childdoc.def}|
and perform the replacements as outlined below.
Instead of |\childdocmain{|\textit{main}|}| add the following code
to the top of the main file:
%
\begin{center}
\begin{tabular}{l}
|\||ifdefined\childdocname\endinput\||fi\newif\ifchilddoc|\\
|\edef\childdocname{\scantokens\expandafter{\jobname\noexpand}}|\\
|\def\childdocmain{|\textit{main}|}\||ifx\childdocmain\childdocname\||else|\\
|\childdoctrue\includeonly{\childdocname}\let\jobname\childdocmain\||fi|\\
\end{tabular}
\end{center}
%
Instead of |\childdocof{|\textit{main}|}| just include the main file
at the top of each child file:
%
\begin{center}
|\input{|\textit{main}|}|
\end{center}
%
A simple redirection |\childdocforward{|\textit{dest}|}| is achieved by:
%
\begin{center}
|\def\jobname{|\textit{dest}|}\input{\jobname}|
\end{center}
%
The redirection with prefix
|\childdocforwardprefix[|\textit{prefix}|]{|\textit{dest}|}|
is accomplished by:
%
\begin{center}
\begin{tabular}{l}
|{\edef\jobname{\scantokens\expandafter{\jobname\noexpand}}|\\
|\def\redirectjob |\textit{prefix}|#1~~~{\gdef\jobname{|\textit{dest}|#1}}|\\
|\expandafter\redirectjob\jobname~~~}\input{\jobname}|
\end{tabular}
\end{center}

In an alternative approach,
child documents can be compiled by a specific command line
without additional code or specific definitions:
%
\begin{center}
|... -jobname "|\textit{target}|" "|[\textit{flags}]%
|\includeonly{|\textit{dest}|}\input{|\textit{main}|}"|
\end{center}
%

%%%%%%%%%%%%%%%%%%%%%%%%%%%%%%%%%%%%%%%%%%%%%%%%%%%%%%%%%%%%%%%%%%%%%%%%%%%%%%%%
%%%%%%%%%%%%%%%%%%%%%%%%%%%%%%%%%%%%%%%%%%%%%%%%%%%%%%%%%%%%%%%%%%%%%%%%%%%%%%%%
\section{Information}

%%%%%%%%%%%%%%%%%%%%%%%%%%%%%%%%%%%%%%%%%%%%%%%%%%%%%%%%%%%%%%%%%%%%%%%%%%%%%%%%
\subsection{Copyright}

Copyright \copyright{} 2017--2018 Niklas Beisert

This work may be distributed and/or modified under the
conditions of the \LaTeX{} Project Public License, either version 1.3
of this license or (at your option) any later version.
The latest version of this license is in
  \url{http://www.latex-project.org/lppl.txt}
and version 1.3 or later is part of all distributions of \LaTeX{}
version 2005/12/01 or later.

This work has the LPPL maintenance status `maintained'.

The Current Maintainer of this work is Niklas Beisert.

This work consists of the files |README.txt|, |childdoc.ins| and |childdoc.dtx|
as well as the derived files |childdoc.def|, |cdocsamp.tex|
with |cdocsch1.tex|, |cdocsch2.tex|, |cdocspt3.tex|, |cdocspt4.tex|,
|cdocsdrf.tex|, |cdocsfn1.tex|, |cdocsfn2.tex|
as well as |childdoc.pdf|.

%%%%%%%%%%%%%%%%%%%%%%%%%%%%%%%%%%%%%%%%%%%%%%%%%%%%%%%%%%%%%%%%%%%%%%%%%%%%%%%%
\subsection{Files and Installation}

The package consists of the files:
%
\begin{center}
\begin{tabular}{ll}
    |README.txt|   & readme file \\
    |childdoc.ins| & installation file \\
    |childdoc.dtx| & source file \\
    |childdoc.def| & definition file \\
    |cdocsamp.tex| & sample main file \\
    |cdocsch1.tex| & sample include file \\
    |cdocsch2.tex| & sample include file \\
    |cdocspt3.tex| & sample part file \\
    |cdocspt4.tex| & sample part file \\
    |cdocsdrf.tex| & sample redirection file \\
    |cdocsfn1.tex| & sample redirection file \\
    |cdocsfn2.tex| & sample redirection file \\
    |childdoc.pdf| & manual
\end{tabular}
\end{center}
%
The distribution consists of the files
|README.txt|, |childdoc.ins| and |childdoc.dtx|.
%
\begin{itemize}
\item
Run (pdf)\LaTeX{} on |childdoc.dtx|
to compile the manual |childdoc.pdf| (this file).
\item
Run \LaTeX{} on |childdoc.ins| to create the definitions file |childdoc.def|
and the sample |cdocsamp.tex| with include files
|cdocsch1.tex|, |cdocsch2.tex|, |cdocspt3.tex|, |cdocspt4.tex|,
|cdocsdrf.tex|, |cdocsfn1.tex|, |cdocsfn2.tex|.
Then copy the file |childdoc.def| to an appropriate directory of your \LaTeX{}
distribution, e.g.\ \textit{texmf-root}|/tex/latex/childdoc|.
\end{itemize}

%%%%%%%%%%%%%%%%%%%%%%%%%%%%%%%%%%%%%%%%%%%%%%%%%%%%%%%%%%%%%%%%%%%%%%%%%%%%%%%%
\subsection{Related CTAN Packages}

There are several other packages which offer a similar functionality:
%
\begin{itemize}
\item
The packages
\href{http://ctan.org/pkg/docmute}{\textsf{docmute}},
\href{http://ctan.org/pkg/includex}{\textsf{includex}} and
\href{http://ctan.org/pkg/standalone}{\textsf{standalone}}
provide commands to include only the document body of
a child file thus allowing both files to be compiled individually.
\item
The packages \href{http://ctan.org/pkg/subdocs}{\textsf{subdocs}}
and \href{http://ctan.org/pkg/subfiles}{\textsf{subfiles}}
provide structures in which the main and child documents can be
encapsulated and allowing them to be compiled individually.
The inclusion mechanism is different from the conventional |\include|.
\item
The package \href{http://ctan.org/pkg/combine}{\textsf{combine}}
is an elaborate solution to combine several documents into one.
\end{itemize}
%
See also the CTAN topic \href{http://ctan.org/topic/subdocs}{\textsf{subdocs}}
for further related packages.
The present package differs from the above solutions in that
a document structure constructed with the conventional |\include| mechanism
just needs two extra commands at the top of every file
such that all constituent files can be compiled individually.

%%%%%%%%%%%%%%%%%%%%%%%%%%%%%%%%%%%%%%%%%%%%%%%%%%%%%%%%%%%%%%%%%%%%%%%%%%%%%%%%
%\subsection{Feature Suggestions}
%
%The following is a list of features which may be useful for future
%versions of this package:
%%
%\begin{itemize}
%\item
%\ldots
%\end{itemize}

%%%%%%%%%%%%%%%%%%%%%%%%%%%%%%%%%%%%%%%%%%%%%%%%%%%%%%%%%%%%%%%%%%%%%%%%%%%%%%%%
\subsection{Revision History}

%%%%%%%%%%%%%%%%%%%%%%%%%%%%%%%%%%%%%%%%
\paragraph{v2.0:} 2018/12/30

\begin{itemize}
\item
immediate forward processing
\item
added |\childdocby| mechanism
\item
manual restructured
\end{itemize}

%%%%%%%%%%%%%%%%%%%%%%%%%%%%%%%%%%%%%%%%
\paragraph{v1.6:} 2018/01/17

\begin{itemize}
\item
application for development of include files
\item
corrections to manual
\end{itemize}

%%%%%%%%%%%%%%%%%%%%%%%%%%%%%%%%%%%%%%%%
\paragraph{v1.5:} 2017/05/21

\begin{itemize}
\item
more complete structuring introduced
\item
|\childdocof| introduced
\item
|\childdoc| renamed to |\childdocmain|
\item
|\childredirect| renamed to |\childdocforward| and |\childdocforwardprefix|
and functionality expanded
\end{itemize}

%%%%%%%%%%%%%%%%%%%%%%%%%%%%%%%%%%%%%%%%
\paragraph{v1.0:} 2017/04/27

\begin{itemize}
\item
manual and install package
\item
first version published on CTAN
\end{itemize}

%%%%%%%%%%%%%%%%%%%%%%%%%%%%%%%%%%%%%%%%
\paragraph{v0.6:} 2017/04/26

\begin{itemize}
\item
redirection mechanism added
\end{itemize}

%%%%%%%%%%%%%%%%%%%%%%%%%%%%%%%%%%%%%%%%
\paragraph{v0.5:} 2017/04/26

\begin{itemize}
\item
functionality in definition file
\end{itemize}


%%%%%%%%%%%%%%%%%%%%%%%%%%%%%%%%%%%%%%%%%%%%%%%%%%%%%%%%%%%%%%%%%%%%%%%%%%%%%%%%
%%%%%%%%%%%%%%%%%%%%%%%%%%%%%%%%%%%%%%%%%%%%%%%%%%%%%%%%%%%%%%%%%%%%%%%%%%%%%%%%
%%%%%%%%%%%%%%%%%%%%%%%%%%%%%%%%%%%%%%%%%%%%%%%%%%%%%%%%%%%%%%%%%%%%%%%%%%%%%%%%
\appendix

\settowidth\MacroIndent{\rmfamily\scriptsize 000\ }

 \DocInput{childdoc.dtx}

\end{document}
%</driver>
% \fi
%
% %%%%%%%%%%%%%%%%%%%%%%%%%%%%%%%%%%%%%%%%%%%%%%%%%%%%%%%%%%%%%%%%%%%%%%%%%%%%%%
% %%%%%%%%%%%%%%%%%%%%%%%%%%%%%%%%%%%%%%%%%%%%%%%%%%%%%%%%%%%%%%%%%%%%%%%%%%%%%%
% \section{Sample}
%\iffalse
%<*samplemain>
%\fi
%
% The following presents a sample document
% with two chapters, two parts, a title page,
% a compile flag as well as three forwarding files to set the flag.
% It consists of eight |.tex| files:
% \begin{center}
% \begin{tabular}{ll}
% |cdocsamp.tex|&main file\\
% |cdocsch1.tex|&include file for chapter 1\\
% |cdocsch2.tex|&include file for chapter 2\\
% |cdocspt3.tex|&include file for part 3\\
% |cdocspt4.tex|&include file for part 4\\
% |cdocsdrf.tex|&forwarding file for main file in draft mode\\
% |cdocsfi1.tex|&forwarding file for final version of chapter 1\\
% |cdocsfi2.tex|&forwarding file for final version of chapter 2\\
% \end{tabular}
% \end{center}
% Each of the eight files can be compiled directly by the \LaTeX{} compiler.
%
% %%%%%%%%%%%%%%%%%%%%%%%%%%%%%%%%%%%%%%
% \paragraph{Main File.}
%
% The main file is called |cdocsamp.tex|.
%
% Load the \textsf{childdoc} definitions and
% declare the filename for the main document:
%    \begin{macrocode}
\input{childdoc.def}
\childdocmain{}
%    \end{macrocode}

% Optional override for |\version| flag:
%    \begin{macrocode}
%%\ifchilddoc\else\providecommand{\version}{draft}\fi
%    \end{macrocode}

% Define the default values for the |\version| flag
% (|final| for the main file and |draft| for childs):
%    \begin{macrocode}
\ifchilddoc
\providecommand{\version}{draft}
\else
\providecommand{\version}{final}
\fi
%    \end{macrocode}

% Load the standard document class:
%    \begin{macrocode}
\documentclass[12pt]{article}
%    \end{macrocode}

% Start the document body:
%    \begin{macrocode}
\begin{document}
%    \end{macrocode}

% Declare a title page.
% Print title, part of document being processed and version flag:
%    \begin{macrocode}
\addtocounter{page}{-1}
\begin{center}
{\LARGE\bfseries{}childdoc example\par}
\vspace{1cm}
\ifchilddoc
\ifchilddocmanual part\else chapter\fi:
`\childdocname' of `\childdocjob'\par
\else
main document: `\childdocjob'\par
\fi
version: \version\par
\end{center}
\newpage
%    \end{macrocode}

% Manually include selected file,
% otherwise process as usual:
%    \begin{macrocode}
\ifchilddocmanual
\section*{part `\childdocname'}
\input{\childdocname}
\else
%    \end{macrocode}

% Include the two chapters:
%    \begin{macrocode}
\include{cdocsch1}
\include{cdocsch2}
%    \end{macrocode}

% Include the two parts unless only chapters should be displayed:
%    \begin{macrocode}
\ifchilddoc\else
\section{part three}
\input{cdocspt3}
\section{part four}
\input{cdocspt4}
\fi
%    \end{macrocode}

% Process as usual until here:
%    \begin{macrocode}
\fi
%    \end{macrocode}

% End of document body:
%    \begin{macrocode}
\end{document}
%    \end{macrocode}
%\iffalse
%</samplemain>
%\fi
%
% %%%%%%%%%%%%%%%%%%%%%%%%%%%%%%%%%%%%%%
% \paragraph{Chapter Include Files.}
%
% The include files are called |cdocsch1.tex| and |cdocsch2.tex|.
%
%\iffalse
%<*samplechap1|samplechap2>
%\fi

% Optional override for |\version| flag:
%    \begin{macrocode}
%%\providecommand{\version}{final}
%    \end{macrocode}

% Include the main document:
%    \begin{macrocode}
\input{childdoc.def}
\childdocof{cdocsamp}
%    \end{macrocode}

%\iffalse
%</samplechap1|samplechap2>
%\fi
%
%\iffalse
%<*samplechap1>
%\fi
% Some text for chapter 1:
%    \begin{macrocode}
\section{one}
some text in chapter one
%    \end{macrocode}

%\iffalse
%</samplechap1>
%\fi
% Some text for chapter 2:
%\iffalse
%<*samplechap2>
%\fi
%    \begin{macrocode}
\section{two}
more text in chapter two
%    \end{macrocode}

%\iffalse
%</samplechap2>
%\fi
%
% %%%%%%%%%%%%%%%%%%%%%%%%%%%%%%%%%%%%%%
% \paragraph{Part Include Files.}
%
% The include files are called |cdocspt3.tex| and |cdocspt4.tex|.
%
%\iffalse
%<*samplepart3|samplepart4>
%\fi

% Optional override for |\version| flag:
%    \begin{macrocode}
%%\providecommand{\version}{final}
%    \end{macrocode}

% Include the main document:
%    \begin{macrocode}
\input{childdoc.def}
\childdocby{cdocsamp}
%    \end{macrocode}

%\iffalse
%</samplepart3|samplepart4>
%\fi
%
%\iffalse
%<*samplepart3>
%\fi
% Some text for part 3:
%    \begin{macrocode}
some text in part three
%    \end{macrocode}

%\iffalse
%</samplepart3>
%\fi
% Some text for part 4:
%\iffalse
%<*samplepart4>
%\fi
%    \begin{macrocode}
more text in part four
%    \end{macrocode}

%\iffalse
%</samplepart4>
%\fi
%
% %%%%%%%%%%%%%%%%%%%%%%%%%%%%%%%%%%%%%%
% \paragraph{Forwarding for a Complete Draft.}
%
% The following forwarding file |cdocsdrf.tex|
% compiles the main document in draft mode:
%\iffalse
%<*sampledraft>
%\fi
%    \begin{macrocode}
\def\version{draft}
\input{childdoc.def}
\childdocforward{cdocsamp}
%    \end{macrocode}

%\iffalse
%</sampledraft>
%\fi
%
% %%%%%%%%%%%%%%%%%%%%%%%%%%%%%%%%%%%%%%
% \paragraph{Forwarding for Final Version of the Chapters.}
%
% The following forwarding files |cdocsfn1.tex| and |cdocsfn2.tex|
% (with identical content)
% compile the final versions of the child documents
% |cdocsch1.tex| and |cdocsch2.tex|, respectively:
%\iffalse
%<*samplefinal>
%\fi
%    \begin{macrocode}
\def\version{final}
\input{childdoc.def}
\childdocforwardprefix[cdocsamp]{cdocsfn}{cdocsch}
%    \end{macrocode}

%\iffalse
%</samplefinal>
%\fi
%
% %%%%%%%%%%%%%%%%%%%%%%%%%%%%%%%%%%%%%%
% \paragraph{Command Line Processing.}
%
% The following three command lines generate the output files
% |cdocscld|, |cdocscl1| and |cdocscl2|
% which should be identical to
% |cdocsdrf|, |cdocsch1| and |cdocsfn2|, respectively:
% \begin{center}
% \begin{tabular}{l}
% |latex -jobname cdocscld \|\\
% |  "\def\version{draft}\input{childdoc.def}\childdocforward{cdocsamp}"|\\
% |latex -jobname cdocscl1 \|\\
% |  "\input{childdoc.def}\childdocforward[cdocsamp]{cdocsch1}"|\\
% |latex -jobname cdocscl2 \|\\
% |  "\def\version{final}\input{childdoc.def}\childdocforward{cdocsch2}"|
% \end{tabular}
% \end{center}
% Note that the trailing backslash on each first line
% merely continues the input to the second line
% (for convenient cut ant paste).
% Furthermore, the command |latex| can be replaced by any
% of its alternative versions such as |pdflatex|.
%
% %%%%%%%%%%%%%%%%%%%%%%%%%%%%%%%%%%%%%%%%%%%%%%%%%%%%%%%%%%%%%%%%%%%%%%%%%%%%%%
% %%%%%%%%%%%%%%%%%%%%%%%%%%%%%%%%%%%%%%%%%%%%%%%%%%%%%%%%%%%%%%%%%%%%%%%%%%%%%%
% \section{Implementation}
%\iffalse
%<*package>
%\fi
%
% This section describes the definitions file |childdoc.def|.

% The definitions cannot be loaded using |\usepackage| or |\RequirePackage|
% which has a mechanism to prevent loading a style file more than once.
% When loading the definitions by means of |\input|
% multiple instances have to be prevented manually:
%\iffalse
%This code needs to be before the `\ProvidesFile' directive
%which is defined at the beginning of this file.
%Therefore it is also placed there and commented out here.
%</package>
%<*discard>
%\fi
%    \begin{macrocode}
\ifdefined\childdocmain\endinput\fi
%    \end{macrocode}
%\iffalse
%</discard>
%<*package>
%\fi
%
% \macro{\ifchilddoc}
% \macro{\ifchilddocmanual}
% The conditional |\ifchilddoc| tells whether a
% child (true) or main (false) document is being compiled.
% The conditional |\ifchilddocmanual| tells whether
% the |\includeonly| mechanism is used (false) or
% the selection of child files must be performed manually (true).
% The definitions initialise to false:
%    \begin{macrocode}
\newif\ifchilddoc
\newif\ifchilddocmanual
%    \end{macrocode}

% \macro{\childdocname}
% \macro{\childdocjob}
% The macro |\childdocname| stores the name of the main document
% to be compiled. The macro |\childdocjob| stores the name of
% the document on which the \LaTeX{} compiler was originally invoked.
% The content of |\jobname| cannot be compared
% to filenames specified in the source due to different catcodes.
% The following code rescans |\jobname|, stores the result
% in |\childdocname| and saves a copy in |\childdocjob|:
%    \begin{macrocode}
\edef\childdocname{\scantokens\expandafter{\jobname\noexpand}}
\let\childdocjob\childdocname
%    \end{macrocode}

% \macro{\childdocdisable}
% The macro |\childdocdisable| prevents the main file
% from being processed more than once.
% At this stage, the main document command |\childdocmain|
% is assumed to be called once again where it should do nothing.
% Any subsequent call to it should prevent
% a secondary processing of the main document
% It overwrites the forwarding commands
% |\childdocof| and |\childdocforward|
% with empty macros to prevent further inclusions of the main document:
%    \begin{macrocode}
\newcommand{\childdocdisable}
{
  \renewcommand{\childdocmain}[1]{\renewcommand{\childdocmain}[1]{\endinput}}
  \renewcommand{\childdocof}[1]{}
  \renewcommand{\childdocby}[2][]{}
  \renewcommand{\childdocforward}[2][]{}
  \renewcommand{\childdocdisable}{}
}
%    \end{macrocode}

% \macro{\childdocmain}
% The macro |\childdocmain| is to be called at the top of the main file
% with nothing or the main filename (without extension) as argument.
% First, it breaks loops.
% If the argument is not empty and does not match |\childdocname|
% (which is set by the first inclusion of |childdoc.def|),
% |\ifchilddoc| is set to true, |\includeonly| is applied to the child file
% and |\jobname| is set to the main file
% (for proper handling of |.aux| files):
%    \begin{macrocode}
\newcommand{\childdocmain}[1]
{
  \childdocdisable\childdocmain{}
  \if?#1?\else
    \begingroup
      \def\childdoctmp{#1}
      \ifx\childdoctmp\childdocname
        \def\childdoctmp{}
      \else
        \def\childdoctmp
        {
          \childdoctrue
          \includeonly{\childdocname}
          \def\childdocjob{#1}
          \def\jobname{#1}
        }
      \fi
      \expandafter
    \endgroup
    \childdoctmp
  \fi
}
%    \end{macrocode}

% \macro{\childdocof}
% The command |\childdocof| redirects
% compilation to the main file |#1|.
%    \begin{macrocode}
\newcommand{\childdocof}[1]
{
  \childdocdisable
  \childdoctrue
  \includeonly{\childdocname}
  \def\jobname{#1}
  \def\childdocjob{#1}
  \input{#1}
}
%    \end{macrocode}

% \macro{\childdocby}
% The command |\childdocby| ....
%    \begin{macrocode}
\newcommand{\childdocby}[2][]
{
  \childdocdisable
  \childdoctrue
  \childdocmanualtrue
  \if?#1?\else
    \def\jobname{#2}
  \fi
  \def\childdocjob{#2}
  \input{#2}
  \endinput
}
%    \end{macrocode}

% \macro{\childdocforward}
% The command |\childdocforward| redirects
% compilation to the main file or
% (if the optional argument is given) a child file.
% Parameters are set as if the main file
% or a child file starting with |\childdocof| was compiled.
% Then compilation is handed over to the main file:
%    \begin{macrocode}
\newcommand{\childdocforward}[2][]
{
  \begingroup
    \if?#1?
      \def\childdoctmp
      {
        \def\childdocname{#2}
        \def\childdocjob{#2}
        \def\jobname{#2}
        \input{#2}
        \endinput
      }
    \else
      \def\childdoctmp
      {
        \childdocdisable
        \def\childdocname{#2}
        \childdoctrue
        \includeonly{#2}
        \def\childdocjob{#1}
        \def\jobname{#1}
        \input{#1}
        \endinput
      }
    \fi
    \expandafter
  \endgroup
  \childdoctmp
}
%    \end{macrocode}

% \macro{\childdocforwardprefix}
% The command |\childdocforwardprefix| redirects
% compilation to the main or a child file by means of a pattern.
% The prefix |#1| in the current filename is replaced by |#2|
% and the suffix of the current filename is kept
% (it is assumed that the filename does not contain the substring `|~~~|'
% which is used as a delimiter).
% Compilation is handed over to the new file by |\childdocforward|:
%    \begin{macrocode}
\newcommand{\childdocforwardprefix}[3][]
{
  \begingroup
    \def\childdocextract #2##1~~~{\def\childdoctmp{\childdocforward[#1]{#3##1}}}
    \expandafter\childdocextract\childdocname~~~
    \expandafter
  \endgroup
  \childdoctmp
}
%    \end{macrocode}

% \macro{\childdoc}
% The deprecated macro |\childdoc| is a legacy version of |\childdocmain|:
%    \begin{macrocode}
\newcommand{\childdoc}{\childdocmain}
%    \end{macrocode}

% \macro{\childdocredirect}
% The deprecated macro |\childdocredirect| is a legacy version
% of |\childdocforward| and |\childdocforwardprefix|:
%    \begin{macrocode}
\newcommand{\childdocredirect}[2][]
{
  \begingroup
    \if?#1?
      \def\childdoctmp{\childdocforward{#2}}
    \else
      \def\childdoctmp{\childdocforwardprefix{#1}{#2}}
    \fi
    \expandafter
  \endgroup
  \childdoctmp
}
%    \end{macrocode}

%\iffalse
%</package>
%\fi
%
\endinput
\childdocforward[|\textit{main}|]{|\textit{dest}|}"|
\end{center}
%
Here \textit{target} is the name of the output file,
\textit{main} is the name of the main file
and \textit{dest} is the name of the main or child file to be processed
(all filenames without extensions).
The optional argument \textit{main} can be omitted
if \textit{main} matches \textit{dest}.
Optionally, compilation \textit{flags} can be defined via |\def| commands.
This command line makes the \TeX{} engine believe
it is compiling the file \textit{target}
whose content is specified as the latter parameter.
The provided code then forwards the processing to
\textit{main} or \textit{dest} as described in \secref{sec:forward}.

%%%%%%%%%%%%%%%%%%%%%%%%%%%%%%%%%%%%%%%%%%%%%%%%%%%%%%%%%%%%%%%%%%%%%%%%%%%%%%%%
\subsection{Include by Input}
\label{sec:input}

Including child documents by |\include| has some restrictions by design.
Most notably, the content of a child document always occupies
its own set of pages; pages cannot be shared between child documents.
Usually, this behaviour makes perfect sense
because each child document contain an essential part of the document.
However, in some situations it may be desirable to compose
a document from a collection of parts
without having mandatory page breaks between then.
For this case, the package
provides a mechanism to include parts
by |\input| which can also be processed individually.
However, by construction this mechanism
requires manual handling of the content to be output.

%%%%%%%%%%%%%%%%%%%%%%%%%%%%%%%%%%%%%%%%
\DescribeMacro{\ifchilddocmanual}
The main file should be prepared as usual, see \secref{sec:include}.
However, the document body must make a distinction
between processing of an individual part and of the main document, e.g.:
%
\begin{center}
\begin{tabular}{l}
|\ifchilddocmanual|\\
|\input{\childdocname}|\\
|\||else|\\
\textit{document body with }|\input{|\textit{part}|}|\\
|\||fi|
\end{tabular}
\end{center}
%
The conditional |\ifchilddocmanual| is true whenever
a part to be included by |\input| is being compiled,
and the name of the part is stored in |\childdocname|.

%%%%%%%%%%%%%%%%%%%%%%%%%%%%%%%%%%%%%%%%
\DescribeMacro{\childdocby}
Each part to be included by |\input| should start with:
%
\begin{center}
\begin{tabular}{l}
|% \iffalse
%
% childdoc.dtx Copyright (C) 2017-2018 Niklas Beisert
%
% This work may be distributed and/or modified under the
% conditions of the LaTeX Project Public License, either version 1.3
% of this license or (at your option) any later version.
% The latest version of this license is in
%   http://www.latex-project.org/lppl.txt
% and version 1.3 or later is part of all distributions of LaTeX
% version 2005/12/01 or later.
%
% This work has the LPPL maintenance status `maintained'.
%
% The Current Maintainer of this work is Niklas Beisert.
%
% This work consists of the files childdoc.dtx and childdoc.ins
% and the derived files childdoc.def and cdocsamp.tex with
% cdocsch1.tex, cdocsch2.tex, cdocsdrf.tex, cdocsfn1.tex, cdocsfn2.tex.
%
%<package>\ifdefined\childdocmain\endinput\fi
%<package>\ProvidesFile{childdoc.def}[2018/12/30 v2.0 child document driver]
%<samplemain>\ProvidesFile{cdocsamp.tex}[2018/12/30 v2.0 sample for childdoc]
%<*driver>
%\ProvidesFile{childdoc.drv}[2018/12/30 v2.0 childdoc reference manual file]
\PassOptionsToClass{10pt,a4paper}{article}
\documentclass{ltxdoc}

\usepackage[margin=35mm]{geometry}
\usepackage{hyperref}
\usepackage{hyperxmp}
\usepackage[usenames]{color}

\hypersetup{colorlinks=true}
\hypersetup{pdfstartview=FitH}
\hypersetup{pdfpagemode=UseNone}
\hypersetup{pdfsource={}}
\hypersetup{pdflang={en-UK}}
\hypersetup{pdfcopyright={Copyright 2017-2018 Niklas Beisert.
  This work may be distributed and/or modified under the
  conditions of the LaTeX Project Public License, either version 1.3
  of this license or (at your option) any later version.}}
\hypersetup{pdflicenseurl={http://www.latex-project.org/lppl.txt}}
\hypersetup{pdfcontactaddress={ETH Zurich, ITP, HIT K,
  Wolfgang-Pauli-Strasse 27}}
\hypersetup{pdfcontactpostcode={8093}}
\hypersetup{pdfcontactcity={Zurich}}
\hypersetup{pdfcontactcountry={Switzerland}}
\hypersetup{pdfcontactemail={nbeisert@itp.phys.ethz.ch}}
\hypersetup{pdfcontacturl={http://people.phys.ethz.ch/\xmptilde nbeisert/}}

\newcommand{\secref}[1]{\hyperref[#1]{section \ref*{#1}}}

\parskip1ex
\parindent0pt
\let\olditemize\itemize
\def\itemize{\olditemize\parskip0pt}

\begin{document}

\title{The \textsf{childdoc} Package}
\hypersetup{pdftitle={The childdoc Package}}
\author{Niklas Beisert\\[2ex]
  Institut f\"ur Theoretische Physik\\
  Eidgen\"ossische Technische Hochschule Z\"urich\\
  Wolfgang-Pauli-Strasse 27, 8093 Z\"urich, Switzerland\\[1ex]
  \href{mailto:nbeisert@itp.phys.ethz.ch}
  {\texttt{nbeisert@itp.phys.ethz.ch}}}
\hypersetup{pdfauthor={Niklas Beisert}}
\hypersetup{pdfsubject={Manual for the LaTeX2e Package childdoc}}
\date{30 December 2018, \textsf{v2.0}}
\maketitle

\begin{abstract}\noindent
\textsf{childdoc} is a \LaTeXe{} package
that enables the direct compilation
of document sections included by |\include|
to individual files.
\end{abstract}

\begingroup
\parskip0ex
\tableofcontents
\endgroup

%%%%%%%%%%%%%%%%%%%%%%%%%%%%%%%%%%%%%%%%%%%%%%%%%%%%%%%%%%%%%%%%%%%%%%%%%%%%%%%%
%%%%%%%%%%%%%%%%%%%%%%%%%%%%%%%%%%%%%%%%%%%%%%%%%%%%%%%%%%%%%%%%%%%%%%%%%%%%%%%%
\section{Introduction}

\LaTeX{} provides a mechanism to structure a large document (such as a book)
into a main file and several child files (containing the chapters)
using the |\include| command.
This mechanism is beneficial for documents
which span hundreds of pages in order to
make the source file(s) more manageable.
Moreover, compilation can be restricted to
selected child files by means of the |\includeonly| command.
The latter feature can be used to reduce the compilation time while editing
(this was significantly more useful in the earlier days of \LaTeX{})
or to generate a smaller document which is easier to navigate.
Another application of |\includeonly| is to generate
documents consisting of selected parts of the complete document.

However, there are a few drawbacks of the plain |\include| mechanism:
\begin{itemize}
\item
The child files cannot be compiled on their own,
they can only be compiled via the main file.
A naive editing environment
(such as a text editor with an option
to have the current file processed by \LaTeX)
may require one to switch to the main file before compiling;
attempting to compile the child file produces errors.
\item
The main file must be modified (each time)
to adjust the |\includeonly| command
to the present needs. This easily leaves the main file in a messy state.
\item
The generated document will always carry the filename
of the main document. This is inconvenient if
several child files are to be compiled and
to be kept for distribution.
\end{itemize}

The present package provides a simple interface
to make child files individually compilable by \LaTeX{}.
Compiling a child file then has the same effect as compiling
the main file with an |\includeonly| command
to select the appropriate child.
Moreover the generated document will carry the name of the child
rather than the main file.
This resolves all three above issues.

This feature is meant to make the editing of books,
thesis documents and lecture notes somewhat more convenient.
However, the package can also be used efficiently for
composing a series of documents (such as exercise sheets)
which are typically distributed individually.
It then assists the author in generating the individual documents
(potentially in different versions)
as well as a document containing the collected series.
Another application is in developing style files
or other kinds of included material
where compilation of the style file could redirect
to a sample or test file.

%%%%%%%%%%%%%%%%%%%%%%%%%%%%%%%%%%%%%%%%%%%%%%%%%%%%%%%%%%%%%%%%%%%%%%%%%%%%%%%%
%%%%%%%%%%%%%%%%%%%%%%%%%%%%%%%%%%%%%%%%%%%%%%%%%%%%%%%%%%%%%%%%%%%%%%%%%%%%%%%%
\section{Usage}

First of all, the package \textsf{childdoc} is \emph{not} a standard
\LaTeXe{} |.sty| style file! Therefore it needs to be invoked in
a non-standard way.

%%%%%%%%%%%%%%%%%%%%%%%%%%%%%%%%%%%%%%%%%%%%%%%%%%%%%%%%%%%%%%%%%%%%%%%%%%%%%%%%
\subsection{Included Files}
\label{sec:include}

%%%%%%%%%%%%%%%%%%%%%%%%%%%%%%%%%%%%%%%%
\DescribeMacro{\childdocmain}
To use the package, add the commands
\begin{center}
\begin{tabular}{l}
|\input{childdoc.def}|\\
|\childdocmain{}|\\
\end{tabular}
\end{center}
at the very top of the main \LaTeX{} file,
in particular \emph{before} the |\documentclass| statement!
The argument of |\childdocmain| should be left empty
(but it must be present).

%%%%%%%%%%%%%%%%%%%%%%%%%%%%%%%%%%%%%%%%
\DescribeMacro{\childdocof}
Furthermore, add the commands
\begin{center}
\begin{tabular}{l}
|\input{childdoc.def}|\\
|\childdocof{|\textit{main}|}|\\
\end{tabular}
\end{center}
at the top of every child file \textit{child}
which is included by |\include{|\textit{child}|}|
from within the main file
(or at least for those files to be compiled individually).
The argument \textit{main} must be the filename of the main file.

There are a couple of
considerations in setting up the main and child documents:

%%%%%%%%%%%%%%%%%%%%%%%%%%%%%%%%%%%%%%%%
\paragraph{Restrictions.}

Please note the following restrictions:
\begin{itemize}
\item
|\childdocmain| must be called with one argument \textit{main}
to ensure compatibility with earlier version of the package.
It must either be empty (|\childdocmain{}|)
or precisely match the filename of the main file in which it is specified.
See \secref{sec:detection} for further information.
\item
The filename \textit{main} must be specified without the |.tex| extension.
\item
The filename \textit{main} is case sensitive
(even in case-insensitive file systems)
due to internal string comparison.
\item
The argument \textit{main} should be fully expanded, it cannot be a macro.
\item
Subdirectories and special characters should be avoided in filenames.
\item
The command |\childdocmain{|\textit{main}|}| must be followed by a whitespace.
It should not be followed immediately by another command
or by a comment mark `|%|'.
This is because the \TeX{} parser reads the token immediately following
the argument of |\childdocmain| and puts it
at the beginning of every child section;
however, a white\-space is ignored.
\end{itemize}

%%%%%%%%%%%%%%%%%%%%%%%%%%%%%%%%%%%%%%%%
\paragraph{Content of Main File.}

It is advisable to place all content in the child files included by |\include|.
Any output contained in the main file will appear in all child documents
unless suppressed manually;
it cannot be suppressed automatically by the |\includeonly| directive
and thus should normally be avoided.
A method to include some content in the main file
by means of conditional processing is described in \secref{sec:conditional}.

%%%%%%%%%%%%%%%%%%%%%%%%%%%%%%%%%%%%%%%%
\paragraph{Page Numbering.}

When only a part of the document is compiled,
the appropriate numbering of pages
(as well as other status parameters)
is determined from the |.aux| files.
The latter contain information from previous passes.
However this information needs to propagate through
all intermediate child documents.
Therefore the page numbering in child documents may well
be inconsistent until the complete document is compiled at least once.

A useful (if unconventional) way to always ensure a consistent
page numbering is to restart the numbering in each child document
and denote the pages by `\textit{child}|.|\textit{page}'
where \textit{child} represents the chapter/section number of the child file.
This can be achieved by the command
|\numberwithin{page}{|\textit{child}|}|
of the \textsf{amsmath} package
where \textit{child} can be |chapter| or |section|
depending on the chosen structuring.
Alternatively, one can modify the macro |\thepage| appropriately
and reset the counter |page| at the start of each child file.

%%%%%%%%%%%%%%%%%%%%%%%%%%%%%%%%%%%%%%%%%%%%%%%%%%%%%%%%%%%%%%%%%%%%%%%%%%%%%%%%
\subsection{Conditional Processing}
\label{sec:conditional}

The package provides a mechanism to compile different versions
of a document. To customise the versions further some conditional processing
can come in handy to distinguish which version is being compiled.
The package provides two macros to describe the compilation context:

%%%%%%%%%%%%%%%%%%%%%%%%%%%%%%%%%%%%%%%%
\DescribeMacro{\ifchilddoc}
The conditional |\ifchilddoc| distinguishes between the compilation of
child documents and the main document:
%
\begin{center}
|\ifchilddoc |\textit{child-code}| |[|\||else |\textit{main-code}]| \||fi|
\end{center}

%%%%%%%%%%%%%%%%%%%%%%%%%%%%%%%%%%%%%%%%
\DescribeMacro{\childdocname}
\DescribeMacro{\childdocjob}
The macro |\childdocname| contains the filename (without extension)
of the main or child file being processed.
Note that |\childdocjob| will always contain the name of the main file.

%%%%%%%%%%%%%%%%%%%%%%%%%%%%%%%%%%%%%%%%
\paragraph{Title Page.}

Conditional processing can be used to include a title or banner page
in the main document when proper precautions are taken.
Importantly, the code in the main file should ensure that the page counter
(as well as other status parameters which are stored in the |.aux| files)
takes the same value after the conditional processing.
Otherwise the page numbers may take divergent values
depending on which part is compiled.

For example, a title page could be declared by:
%
\begin{center}
\begin{tabular}{l}
|\ifchilddoc\||else|\\
|\addtocounter{page}{-1}|\\
\textit{code for title page}\\
|\newpage|\\
|\||fi|
\end{tabular}
\end{center}
%
A banner page for the child documents can be generated by:
%
\begin{center}
\begin{tabular}{l}
|\ifchilddoc|\\
|\addtocounter{page}{-1}|\\
\textit{code for banner page}\\
|\newpage|\\
|\||fi|
\end{tabular}
\end{center}
%
Here one could write a message such as:
\begin{center}
|This is the part \childdocname{} of \childdocjob{}.|
\end{center}

%%%%%%%%%%%%%%%%%%%%%%%%%%%%%%%%%%%%%%%%%%%%%%%%%%%%%%%%%%%%%%%%%%%%%%%%%%%%%%%%
\subsection{Flags}
\label{sec:flags}

The package makes it easy to generate different versions
of the main or child documents.
To this end compilation flags can be defined
and assigned different default values.
They will be particularly useful in conjunction
with the forwarding mechanism described in \secref{sec:forward}.

For example, it may be useful to have a flag |\version|
which can be set to |draft| or |final|.
The document source will contain some conditional code
depending on the value of |\version|.
Suppose further, the flag should default to |final| for the main file
and to |draft| for child files
which is a natural assignment for editing the document.
This is achieved by placing the following code
in the preamble of the main document
(below the |\childdocmain| directive):
%
\begin{center}
\begin{tabular}{l}
|\ifchilddoc|\\
|\providecommand{\version}{draft}|\\
|\||else|\\
|\providecommand{\version}{final}|\\
|\||fi|
\end{tabular}
\end{center}
%
The definition by |\providecommand| makes sure
that previous definitions are not overwritten.
Further statements |\providecommand{\version}{...}|
can thus be added before the above code to override it.

For the main file, one might add a line
(between |\childdocmain| and the above block)
%
\begin{center}
|%\ifchilddoc\||else\providecommand{\version}{draft}\||fi|
\end{center}
%
which can be uncommented to produce a draft version.
Likewise one can add a line to the very top of a child file
(above the |\childdocof{|\textit{main}|}| directive)
%
\begin{center}
|%\providecommand{\version}{final}|
\end{center}
%
which can be uncommented to produce the final version of this child document.

%%%%%%%%%%%%%%%%%%%%%%%%%%%%%%%%%%%%%%%%%%%%%%%%%%%%%%%%%%%%%%%%%%%%%%%%%%%%%%%%
\subsection{Forwarding}
\label{sec:forward}

Different versions of the main or child documents
using compilation flags as described in \secref{sec:flags}
can be (permanently) stored in different files
for convenient compilation, viewing and distribution.
To this end, the package defines a command
to pass on compilation to a different file:

%%%%%%%%%%%%%%%%%%%%%%%%%%%%%%%%%%%%%%%%
\DescribeMacro{\childdocforward}
The command |\childdocforward| redirects processing to
another source file:
%
\begin{center}
\begin{tabular}{l}
|\input{childdoc.def}|\\
|\childdocforward[|\textit{main}|]{|\textit{dest}|}|\\
\end{tabular}
\end{center}
%
The argument \textit{dest} is the destination file
(without extension).
It should be the main file or one of the child files.
Note that further \textsf{childdoc} directives
such as |\childdocof| and |\childdocforward|
in the indicated file will be processed in this form.
The optional argument \textit{main}
passes on directly to the main file \textit{main}
while pretending to compile the child \textit{dest}.
This form behaves as if \textit{dest}
issues |\childdocof{|\textit{main}|}| right away,
and no further \textsf{childdoc} directives will be processed.

%%%%%%%%%%%%%%%%%%%%%%%%%%%%%%%%%%%%%%%%
\DescribeMacro{\...prefix}
In the alternative form |\childdocforwardprefix|,
%
\begin{center}
\begin{tabular}{l}
|\input{childdoc.def}|\\
|\childdocforwardprefix[|\textit{main}|]{|\textit{prefix}|}{|\textit{dest}|}|
\end{tabular}
\end{center}
%
the destination file is determined by a pattern
depending on the current file:
To make this work, the current file must be called
`{\textit{prefix}\hspace{0.2em}\textit{suffix}}'
with \textit{prefix} matching precisely the argument.
Processing is then passed on to the file
`{\textit{dest}\hspace{0.2em}\textit{suffix}}'.
Surely, the same effect is achieved by
directly specifying the
argument `{\textit{dest}\hspace{0.2em}\textit{suffix}}'
in the first form.
However, that requires to set up a different file
for each child. With the alternative form of the command
all these files can have exactly the same content
which simplifies setting them up and maintaining them.

For example, the following file |draft.tex|
with a compilation flag |\version| as described in \secref{sec:flags}
compiles the main document as a draft:
%
\begin{center}
\begin{tabular}{l}
|\def\version{draft}|\\
|\input{childdoc.def}|\\
|\childdocforward{|\textit{main}|}|
\end{tabular}
\end{center}
%
Likewise, the following files |final|\textit{nn}|.tex|
compile the final version of the child document
|child|\textit{nn}|.tex|:
%
\begin{center}
\begin{tabular}{l}
|\def\version{final}|\\
|\input{childdoc.def}|\\
|\childdocforwardprefix{final}{child}|
\end{tabular}
\end{center}
%

Note that when several versions of a main file and/or of each child file
are to be generated, it may be convenient to set up a |Makefile| or
shell script to automatise the process.

%%%%%%%%%%%%%%%%%%%%%%%%%%%%%%%%%%%%%%%%%%%%%%%%%%%%%%%%%%%%%%%%%%%%%%%%%%%%%%%%
\subsection{Command Line Processing}
\label{sec:commandline}

The effect of redirection files can also be achieved by invoking
the \LaTeX{} compiler with a more elaborate command line.
Most conveniently this should be done as part
of a shell script or a |Makefile|.

When using \textsf{childdoc} in the main file, the following
command lines effectively perform a redirection
(note that depending on the shell being used,
backslashes may have to be doubled: `|\|' $\to$ `|\\|'):
%
\begin{center}
|... -jobname "|\textit{target}|" |\\|"|[\textit{flags}]%
|\input{childdoc.def}\childdocforward[|\textit{main}|]{|\textit{dest}|}"|
\end{center}
%
Here \textit{target} is the name of the output file,
\textit{main} is the name of the main file
and \textit{dest} is the name of the main or child file to be processed
(all filenames without extensions).
The optional argument \textit{main} can be omitted
if \textit{main} matches \textit{dest}.
Optionally, compilation \textit{flags} can be defined via |\def| commands.
This command line makes the \TeX{} engine believe
it is compiling the file \textit{target}
whose content is specified as the latter parameter.
The provided code then forwards the processing to
\textit{main} or \textit{dest} as described in \secref{sec:forward}.

%%%%%%%%%%%%%%%%%%%%%%%%%%%%%%%%%%%%%%%%%%%%%%%%%%%%%%%%%%%%%%%%%%%%%%%%%%%%%%%%
\subsection{Include by Input}
\label{sec:input}

Including child documents by |\include| has some restrictions by design.
Most notably, the content of a child document always occupies
its own set of pages; pages cannot be shared between child documents.
Usually, this behaviour makes perfect sense
because each child document contain an essential part of the document.
However, in some situations it may be desirable to compose
a document from a collection of parts
without having mandatory page breaks between then.
For this case, the package
provides a mechanism to include parts
by |\input| which can also be processed individually.
However, by construction this mechanism
requires manual handling of the content to be output.

%%%%%%%%%%%%%%%%%%%%%%%%%%%%%%%%%%%%%%%%
\DescribeMacro{\ifchilddocmanual}
The main file should be prepared as usual, see \secref{sec:include}.
However, the document body must make a distinction
between processing of an individual part and of the main document, e.g.:
%
\begin{center}
\begin{tabular}{l}
|\ifchilddocmanual|\\
|\input{\childdocname}|\\
|\||else|\\
\textit{document body with }|\input{|\textit{part}|}|\\
|\||fi|
\end{tabular}
\end{center}
%
The conditional |\ifchilddocmanual| is true whenever
a part to be included by |\input| is being compiled,
and the name of the part is stored in |\childdocname|.

%%%%%%%%%%%%%%%%%%%%%%%%%%%%%%%%%%%%%%%%
\DescribeMacro{\childdocby}
Each part to be included by |\input| should start with:
%
\begin{center}
\begin{tabular}{l}
|\input{childdoc.def}|\\
|\childdocby{|\textit{main}|}|\\
\end{tabular}
\end{center}
%
The directive |\childdocby| is similar to |\childdocof|
described in \secref{sec:include},
but the subsequent selection of content must be done manually.
To that end, both |\ifchilddoc| and |\ifchilddocmanual|
will be true upon processing of a part,
and the name of the part is stored in |\childdocname|.
Note that |\jobname| will be set to the filename of the current part
so that each part receives an individual |.aux| file
that does not interfere with the |.aux| file(s) of the main document.
This behaviour can be altered by the alternative form
|\childdocby[*]{|\textit{main}|}| (with a non-empty optional argument)
which uses the |.aux| file of the main document
by setting |\jobname| to \textit{main}.

%%%%%%%%%%%%%%%%%%%%%%%%%%%%%%%%%%%%%%%%%%%%%%%%%%%%%%%%%%%%%%%%%%%%%%%%%%%%%%%%
\subsection{Driver Development}
\label{sec:driver}

The \textsf{childdoc} mechanism can also be use for the development
of definition files such as \LaTeX{} styles or classes.
This case differs from the above setup with multiple parts
included by |\include| in that no |\includeonly| should be invoked.
This can be achieved by starting the include file
(before |\ProvidesPackage|) with:
%
\begin{center}
\begin{tabular}{l}
|\input{childdoc.def}|\\
|\childdocforward{|\textit{main}|}|\\
\end{tabular}
\end{center}
%
or alternatively with:
%
\begin{center}
\begin{tabular}{l}
|\input{childdoc.def}|\\
|\childdocby{|\textit{main}|}|\\
\end{tabular}
\end{center}
%
Both forms have slightly different effects as described above.
The main file is prepared as usual, see \secref{sec:include}.

%%%%%%%%%%%%%%%%%%%%%%%%%%%%%%%%%%%%%%%%%%%%%%%%%%%%%%%%%%%%%%%%%%%%%%%%%%%%%%%%
\subsection{Legacy Detection}
\label{sec:detection}

The directive |\childdocmain| in the main file can detect
whether the complete document or merely a child is to be compiled
even without using the directive |\childdocof|.
This method is deprecated because it is less robust
and there is no compelling reason to use it;
it is merely provided for backward compatibility
and it may be removed in future versions.

If the detection mechanism is to be used,
it is mandatory to correctly specify
the filename of the main file as the argument of |\childdocmain|:
%
\begin{center}
\begin{tabular}{l}
|\input{childdoc.def}|\\
|\childdocmain{|\textit{main}|}|\\
\end{tabular}
\end{center}
%
If |\jobname| does not match the argument \textit{main} of |\childdocmain|,
it is assumed that |\jobname| points to the child file to be compiled.
When using |\childdocmain| with the main file specified as argument,
it suffices to start a child file
with just |\input{|\textit{main}|}|
without loading of the package and using |\childdocof|.
If instead all processing is done
with the appropriate \textsf{childdoc} directives,
the argument of \textit{main} of |\childdocmain| can be empty.

An alternative version of the command line processing described
in \secref{sec:commandline} using the detection mechanism reads:
%
\begin{center}
|... -jobname "|\textit{target}|" "|[\textit{flags}]%
[|\def\jobname{|\textit{dest}|}|]|\input{|\textit{main}|}"|
\end{center}

%%%%%%%%%%%%%%%%%%%%%%%%%%%%%%%%%%%%%%%%%%%%%%%%%%%%%%%%%%%%%%%%%%%%%%%%%%%%%%%%
\subsection{Manual Code}
\label{sec:manual}

In case one cannot be certain whether the definitions file |childdoc.def|
is installed on the target \TeX{} distribution
and one prefers not to ship it,
it is conceivable to paste a few relevant commands into the sources.

To that end, drop all statements |\input{childdoc.def}|
and perform the replacements as outlined below.
Instead of |\childdocmain{|\textit{main}|}| add the following code
to the top of the main file:
%
\begin{center}
\begin{tabular}{l}
|\||ifdefined\childdocname\endinput\||fi\newif\ifchilddoc|\\
|\edef\childdocname{\scantokens\expandafter{\jobname\noexpand}}|\\
|\def\childdocmain{|\textit{main}|}\||ifx\childdocmain\childdocname\||else|\\
|\childdoctrue\includeonly{\childdocname}\let\jobname\childdocmain\||fi|\\
\end{tabular}
\end{center}
%
Instead of |\childdocof{|\textit{main}|}| just include the main file
at the top of each child file:
%
\begin{center}
|\input{|\textit{main}|}|
\end{center}
%
A simple redirection |\childdocforward{|\textit{dest}|}| is achieved by:
%
\begin{center}
|\def\jobname{|\textit{dest}|}\input{\jobname}|
\end{center}
%
The redirection with prefix
|\childdocforwardprefix[|\textit{prefix}|]{|\textit{dest}|}|
is accomplished by:
%
\begin{center}
\begin{tabular}{l}
|{\edef\jobname{\scantokens\expandafter{\jobname\noexpand}}|\\
|\def\redirectjob |\textit{prefix}|#1~~~{\gdef\jobname{|\textit{dest}|#1}}|\\
|\expandafter\redirectjob\jobname~~~}\input{\jobname}|
\end{tabular}
\end{center}

In an alternative approach,
child documents can be compiled by a specific command line
without additional code or specific definitions:
%
\begin{center}
|... -jobname "|\textit{target}|" "|[\textit{flags}]%
|\includeonly{|\textit{dest}|}\input{|\textit{main}|}"|
\end{center}
%

%%%%%%%%%%%%%%%%%%%%%%%%%%%%%%%%%%%%%%%%%%%%%%%%%%%%%%%%%%%%%%%%%%%%%%%%%%%%%%%%
%%%%%%%%%%%%%%%%%%%%%%%%%%%%%%%%%%%%%%%%%%%%%%%%%%%%%%%%%%%%%%%%%%%%%%%%%%%%%%%%
\section{Information}

%%%%%%%%%%%%%%%%%%%%%%%%%%%%%%%%%%%%%%%%%%%%%%%%%%%%%%%%%%%%%%%%%%%%%%%%%%%%%%%%
\subsection{Copyright}

Copyright \copyright{} 2017--2018 Niklas Beisert

This work may be distributed and/or modified under the
conditions of the \LaTeX{} Project Public License, either version 1.3
of this license or (at your option) any later version.
The latest version of this license is in
  \url{http://www.latex-project.org/lppl.txt}
and version 1.3 or later is part of all distributions of \LaTeX{}
version 2005/12/01 or later.

This work has the LPPL maintenance status `maintained'.

The Current Maintainer of this work is Niklas Beisert.

This work consists of the files |README.txt|, |childdoc.ins| and |childdoc.dtx|
as well as the derived files |childdoc.def|, |cdocsamp.tex|
with |cdocsch1.tex|, |cdocsch2.tex|, |cdocspt3.tex|, |cdocspt4.tex|,
|cdocsdrf.tex|, |cdocsfn1.tex|, |cdocsfn2.tex|
as well as |childdoc.pdf|.

%%%%%%%%%%%%%%%%%%%%%%%%%%%%%%%%%%%%%%%%%%%%%%%%%%%%%%%%%%%%%%%%%%%%%%%%%%%%%%%%
\subsection{Files and Installation}

The package consists of the files:
%
\begin{center}
\begin{tabular}{ll}
    |README.txt|   & readme file \\
    |childdoc.ins| & installation file \\
    |childdoc.dtx| & source file \\
    |childdoc.def| & definition file \\
    |cdocsamp.tex| & sample main file \\
    |cdocsch1.tex| & sample include file \\
    |cdocsch2.tex| & sample include file \\
    |cdocspt3.tex| & sample part file \\
    |cdocspt4.tex| & sample part file \\
    |cdocsdrf.tex| & sample redirection file \\
    |cdocsfn1.tex| & sample redirection file \\
    |cdocsfn2.tex| & sample redirection file \\
    |childdoc.pdf| & manual
\end{tabular}
\end{center}
%
The distribution consists of the files
|README.txt|, |childdoc.ins| and |childdoc.dtx|.
%
\begin{itemize}
\item
Run (pdf)\LaTeX{} on |childdoc.dtx|
to compile the manual |childdoc.pdf| (this file).
\item
Run \LaTeX{} on |childdoc.ins| to create the definitions file |childdoc.def|
and the sample |cdocsamp.tex| with include files
|cdocsch1.tex|, |cdocsch2.tex|, |cdocspt3.tex|, |cdocspt4.tex|,
|cdocsdrf.tex|, |cdocsfn1.tex|, |cdocsfn2.tex|.
Then copy the file |childdoc.def| to an appropriate directory of your \LaTeX{}
distribution, e.g.\ \textit{texmf-root}|/tex/latex/childdoc|.
\end{itemize}

%%%%%%%%%%%%%%%%%%%%%%%%%%%%%%%%%%%%%%%%%%%%%%%%%%%%%%%%%%%%%%%%%%%%%%%%%%%%%%%%
\subsection{Related CTAN Packages}

There are several other packages which offer a similar functionality:
%
\begin{itemize}
\item
The packages
\href{http://ctan.org/pkg/docmute}{\textsf{docmute}},
\href{http://ctan.org/pkg/includex}{\textsf{includex}} and
\href{http://ctan.org/pkg/standalone}{\textsf{standalone}}
provide commands to include only the document body of
a child file thus allowing both files to be compiled individually.
\item
The packages \href{http://ctan.org/pkg/subdocs}{\textsf{subdocs}}
and \href{http://ctan.org/pkg/subfiles}{\textsf{subfiles}}
provide structures in which the main and child documents can be
encapsulated and allowing them to be compiled individually.
The inclusion mechanism is different from the conventional |\include|.
\item
The package \href{http://ctan.org/pkg/combine}{\textsf{combine}}
is an elaborate solution to combine several documents into one.
\end{itemize}
%
See also the CTAN topic \href{http://ctan.org/topic/subdocs}{\textsf{subdocs}}
for further related packages.
The present package differs from the above solutions in that
a document structure constructed with the conventional |\include| mechanism
just needs two extra commands at the top of every file
such that all constituent files can be compiled individually.

%%%%%%%%%%%%%%%%%%%%%%%%%%%%%%%%%%%%%%%%%%%%%%%%%%%%%%%%%%%%%%%%%%%%%%%%%%%%%%%%
%\subsection{Feature Suggestions}
%
%The following is a list of features which may be useful for future
%versions of this package:
%%
%\begin{itemize}
%\item
%\ldots
%\end{itemize}

%%%%%%%%%%%%%%%%%%%%%%%%%%%%%%%%%%%%%%%%%%%%%%%%%%%%%%%%%%%%%%%%%%%%%%%%%%%%%%%%
\subsection{Revision History}

%%%%%%%%%%%%%%%%%%%%%%%%%%%%%%%%%%%%%%%%
\paragraph{v2.0:} 2018/12/30

\begin{itemize}
\item
immediate forward processing
\item
added |\childdocby| mechanism
\item
manual restructured
\end{itemize}

%%%%%%%%%%%%%%%%%%%%%%%%%%%%%%%%%%%%%%%%
\paragraph{v1.6:} 2018/01/17

\begin{itemize}
\item
application for development of include files
\item
corrections to manual
\end{itemize}

%%%%%%%%%%%%%%%%%%%%%%%%%%%%%%%%%%%%%%%%
\paragraph{v1.5:} 2017/05/21

\begin{itemize}
\item
more complete structuring introduced
\item
|\childdocof| introduced
\item
|\childdoc| renamed to |\childdocmain|
\item
|\childredirect| renamed to |\childdocforward| and |\childdocforwardprefix|
and functionality expanded
\end{itemize}

%%%%%%%%%%%%%%%%%%%%%%%%%%%%%%%%%%%%%%%%
\paragraph{v1.0:} 2017/04/27

\begin{itemize}
\item
manual and install package
\item
first version published on CTAN
\end{itemize}

%%%%%%%%%%%%%%%%%%%%%%%%%%%%%%%%%%%%%%%%
\paragraph{v0.6:} 2017/04/26

\begin{itemize}
\item
redirection mechanism added
\end{itemize}

%%%%%%%%%%%%%%%%%%%%%%%%%%%%%%%%%%%%%%%%
\paragraph{v0.5:} 2017/04/26

\begin{itemize}
\item
functionality in definition file
\end{itemize}


%%%%%%%%%%%%%%%%%%%%%%%%%%%%%%%%%%%%%%%%%%%%%%%%%%%%%%%%%%%%%%%%%%%%%%%%%%%%%%%%
%%%%%%%%%%%%%%%%%%%%%%%%%%%%%%%%%%%%%%%%%%%%%%%%%%%%%%%%%%%%%%%%%%%%%%%%%%%%%%%%
%%%%%%%%%%%%%%%%%%%%%%%%%%%%%%%%%%%%%%%%%%%%%%%%%%%%%%%%%%%%%%%%%%%%%%%%%%%%%%%%
\appendix

\settowidth\MacroIndent{\rmfamily\scriptsize 000\ }

 \DocInput{childdoc.dtx}

\end{document}
%</driver>
% \fi
%
% %%%%%%%%%%%%%%%%%%%%%%%%%%%%%%%%%%%%%%%%%%%%%%%%%%%%%%%%%%%%%%%%%%%%%%%%%%%%%%
% %%%%%%%%%%%%%%%%%%%%%%%%%%%%%%%%%%%%%%%%%%%%%%%%%%%%%%%%%%%%%%%%%%%%%%%%%%%%%%
% \section{Sample}
%\iffalse
%<*samplemain>
%\fi
%
% The following presents a sample document
% with two chapters, two parts, a title page,
% a compile flag as well as three forwarding files to set the flag.
% It consists of eight |.tex| files:
% \begin{center}
% \begin{tabular}{ll}
% |cdocsamp.tex|&main file\\
% |cdocsch1.tex|&include file for chapter 1\\
% |cdocsch2.tex|&include file for chapter 2\\
% |cdocspt3.tex|&include file for part 3\\
% |cdocspt4.tex|&include file for part 4\\
% |cdocsdrf.tex|&forwarding file for main file in draft mode\\
% |cdocsfi1.tex|&forwarding file for final version of chapter 1\\
% |cdocsfi2.tex|&forwarding file for final version of chapter 2\\
% \end{tabular}
% \end{center}
% Each of the eight files can be compiled directly by the \LaTeX{} compiler.
%
% %%%%%%%%%%%%%%%%%%%%%%%%%%%%%%%%%%%%%%
% \paragraph{Main File.}
%
% The main file is called |cdocsamp.tex|.
%
% Load the \textsf{childdoc} definitions and
% declare the filename for the main document:
%    \begin{macrocode}
\input{childdoc.def}
\childdocmain{}
%    \end{macrocode}

% Optional override for |\version| flag:
%    \begin{macrocode}
%%\ifchilddoc\else\providecommand{\version}{draft}\fi
%    \end{macrocode}

% Define the default values for the |\version| flag
% (|final| for the main file and |draft| for childs):
%    \begin{macrocode}
\ifchilddoc
\providecommand{\version}{draft}
\else
\providecommand{\version}{final}
\fi
%    \end{macrocode}

% Load the standard document class:
%    \begin{macrocode}
\documentclass[12pt]{article}
%    \end{macrocode}

% Start the document body:
%    \begin{macrocode}
\begin{document}
%    \end{macrocode}

% Declare a title page.
% Print title, part of document being processed and version flag:
%    \begin{macrocode}
\addtocounter{page}{-1}
\begin{center}
{\LARGE\bfseries{}childdoc example\par}
\vspace{1cm}
\ifchilddoc
\ifchilddocmanual part\else chapter\fi:
`\childdocname' of `\childdocjob'\par
\else
main document: `\childdocjob'\par
\fi
version: \version\par
\end{center}
\newpage
%    \end{macrocode}

% Manually include selected file,
% otherwise process as usual:
%    \begin{macrocode}
\ifchilddocmanual
\section*{part `\childdocname'}
\input{\childdocname}
\else
%    \end{macrocode}

% Include the two chapters:
%    \begin{macrocode}
\include{cdocsch1}
\include{cdocsch2}
%    \end{macrocode}

% Include the two parts unless only chapters should be displayed:
%    \begin{macrocode}
\ifchilddoc\else
\section{part three}
\input{cdocspt3}
\section{part four}
\input{cdocspt4}
\fi
%    \end{macrocode}

% Process as usual until here:
%    \begin{macrocode}
\fi
%    \end{macrocode}

% End of document body:
%    \begin{macrocode}
\end{document}
%    \end{macrocode}
%\iffalse
%</samplemain>
%\fi
%
% %%%%%%%%%%%%%%%%%%%%%%%%%%%%%%%%%%%%%%
% \paragraph{Chapter Include Files.}
%
% The include files are called |cdocsch1.tex| and |cdocsch2.tex|.
%
%\iffalse
%<*samplechap1|samplechap2>
%\fi

% Optional override for |\version| flag:
%    \begin{macrocode}
%%\providecommand{\version}{final}
%    \end{macrocode}

% Include the main document:
%    \begin{macrocode}
\input{childdoc.def}
\childdocof{cdocsamp}
%    \end{macrocode}

%\iffalse
%</samplechap1|samplechap2>
%\fi
%
%\iffalse
%<*samplechap1>
%\fi
% Some text for chapter 1:
%    \begin{macrocode}
\section{one}
some text in chapter one
%    \end{macrocode}

%\iffalse
%</samplechap1>
%\fi
% Some text for chapter 2:
%\iffalse
%<*samplechap2>
%\fi
%    \begin{macrocode}
\section{two}
more text in chapter two
%    \end{macrocode}

%\iffalse
%</samplechap2>
%\fi
%
% %%%%%%%%%%%%%%%%%%%%%%%%%%%%%%%%%%%%%%
% \paragraph{Part Include Files.}
%
% The include files are called |cdocspt3.tex| and |cdocspt4.tex|.
%
%\iffalse
%<*samplepart3|samplepart4>
%\fi

% Optional override for |\version| flag:
%    \begin{macrocode}
%%\providecommand{\version}{final}
%    \end{macrocode}

% Include the main document:
%    \begin{macrocode}
\input{childdoc.def}
\childdocby{cdocsamp}
%    \end{macrocode}

%\iffalse
%</samplepart3|samplepart4>
%\fi
%
%\iffalse
%<*samplepart3>
%\fi
% Some text for part 3:
%    \begin{macrocode}
some text in part three
%    \end{macrocode}

%\iffalse
%</samplepart3>
%\fi
% Some text for part 4:
%\iffalse
%<*samplepart4>
%\fi
%    \begin{macrocode}
more text in part four
%    \end{macrocode}

%\iffalse
%</samplepart4>
%\fi
%
% %%%%%%%%%%%%%%%%%%%%%%%%%%%%%%%%%%%%%%
% \paragraph{Forwarding for a Complete Draft.}
%
% The following forwarding file |cdocsdrf.tex|
% compiles the main document in draft mode:
%\iffalse
%<*sampledraft>
%\fi
%    \begin{macrocode}
\def\version{draft}
\input{childdoc.def}
\childdocforward{cdocsamp}
%    \end{macrocode}

%\iffalse
%</sampledraft>
%\fi
%
% %%%%%%%%%%%%%%%%%%%%%%%%%%%%%%%%%%%%%%
% \paragraph{Forwarding for Final Version of the Chapters.}
%
% The following forwarding files |cdocsfn1.tex| and |cdocsfn2.tex|
% (with identical content)
% compile the final versions of the child documents
% |cdocsch1.tex| and |cdocsch2.tex|, respectively:
%\iffalse
%<*samplefinal>
%\fi
%    \begin{macrocode}
\def\version{final}
\input{childdoc.def}
\childdocforwardprefix[cdocsamp]{cdocsfn}{cdocsch}
%    \end{macrocode}

%\iffalse
%</samplefinal>
%\fi
%
% %%%%%%%%%%%%%%%%%%%%%%%%%%%%%%%%%%%%%%
% \paragraph{Command Line Processing.}
%
% The following three command lines generate the output files
% |cdocscld|, |cdocscl1| and |cdocscl2|
% which should be identical to
% |cdocsdrf|, |cdocsch1| and |cdocsfn2|, respectively:
% \begin{center}
% \begin{tabular}{l}
% |latex -jobname cdocscld \|\\
% |  "\def\version{draft}\input{childdoc.def}\childdocforward{cdocsamp}"|\\
% |latex -jobname cdocscl1 \|\\
% |  "\input{childdoc.def}\childdocforward[cdocsamp]{cdocsch1}"|\\
% |latex -jobname cdocscl2 \|\\
% |  "\def\version{final}\input{childdoc.def}\childdocforward{cdocsch2}"|
% \end{tabular}
% \end{center}
% Note that the trailing backslash on each first line
% merely continues the input to the second line
% (for convenient cut ant paste).
% Furthermore, the command |latex| can be replaced by any
% of its alternative versions such as |pdflatex|.
%
% %%%%%%%%%%%%%%%%%%%%%%%%%%%%%%%%%%%%%%%%%%%%%%%%%%%%%%%%%%%%%%%%%%%%%%%%%%%%%%
% %%%%%%%%%%%%%%%%%%%%%%%%%%%%%%%%%%%%%%%%%%%%%%%%%%%%%%%%%%%%%%%%%%%%%%%%%%%%%%
% \section{Implementation}
%\iffalse
%<*package>
%\fi
%
% This section describes the definitions file |childdoc.def|.

% The definitions cannot be loaded using |\usepackage| or |\RequirePackage|
% which has a mechanism to prevent loading a style file more than once.
% When loading the definitions by means of |\input|
% multiple instances have to be prevented manually:
%\iffalse
%This code needs to be before the `\ProvidesFile' directive
%which is defined at the beginning of this file.
%Therefore it is also placed there and commented out here.
%</package>
%<*discard>
%\fi
%    \begin{macrocode}
\ifdefined\childdocmain\endinput\fi
%    \end{macrocode}
%\iffalse
%</discard>
%<*package>
%\fi
%
% \macro{\ifchilddoc}
% \macro{\ifchilddocmanual}
% The conditional |\ifchilddoc| tells whether a
% child (true) or main (false) document is being compiled.
% The conditional |\ifchilddocmanual| tells whether
% the |\includeonly| mechanism is used (false) or
% the selection of child files must be performed manually (true).
% The definitions initialise to false:
%    \begin{macrocode}
\newif\ifchilddoc
\newif\ifchilddocmanual
%    \end{macrocode}

% \macro{\childdocname}
% \macro{\childdocjob}
% The macro |\childdocname| stores the name of the main document
% to be compiled. The macro |\childdocjob| stores the name of
% the document on which the \LaTeX{} compiler was originally invoked.
% The content of |\jobname| cannot be compared
% to filenames specified in the source due to different catcodes.
% The following code rescans |\jobname|, stores the result
% in |\childdocname| and saves a copy in |\childdocjob|:
%    \begin{macrocode}
\edef\childdocname{\scantokens\expandafter{\jobname\noexpand}}
\let\childdocjob\childdocname
%    \end{macrocode}

% \macro{\childdocdisable}
% The macro |\childdocdisable| prevents the main file
% from being processed more than once.
% At this stage, the main document command |\childdocmain|
% is assumed to be called once again where it should do nothing.
% Any subsequent call to it should prevent
% a secondary processing of the main document
% It overwrites the forwarding commands
% |\childdocof| and |\childdocforward|
% with empty macros to prevent further inclusions of the main document:
%    \begin{macrocode}
\newcommand{\childdocdisable}
{
  \renewcommand{\childdocmain}[1]{\renewcommand{\childdocmain}[1]{\endinput}}
  \renewcommand{\childdocof}[1]{}
  \renewcommand{\childdocby}[2][]{}
  \renewcommand{\childdocforward}[2][]{}
  \renewcommand{\childdocdisable}{}
}
%    \end{macrocode}

% \macro{\childdocmain}
% The macro |\childdocmain| is to be called at the top of the main file
% with nothing or the main filename (without extension) as argument.
% First, it breaks loops.
% If the argument is not empty and does not match |\childdocname|
% (which is set by the first inclusion of |childdoc.def|),
% |\ifchilddoc| is set to true, |\includeonly| is applied to the child file
% and |\jobname| is set to the main file
% (for proper handling of |.aux| files):
%    \begin{macrocode}
\newcommand{\childdocmain}[1]
{
  \childdocdisable\childdocmain{}
  \if?#1?\else
    \begingroup
      \def\childdoctmp{#1}
      \ifx\childdoctmp\childdocname
        \def\childdoctmp{}
      \else
        \def\childdoctmp
        {
          \childdoctrue
          \includeonly{\childdocname}
          \def\childdocjob{#1}
          \def\jobname{#1}
        }
      \fi
      \expandafter
    \endgroup
    \childdoctmp
  \fi
}
%    \end{macrocode}

% \macro{\childdocof}
% The command |\childdocof| redirects
% compilation to the main file |#1|.
%    \begin{macrocode}
\newcommand{\childdocof}[1]
{
  \childdocdisable
  \childdoctrue
  \includeonly{\childdocname}
  \def\jobname{#1}
  \def\childdocjob{#1}
  \input{#1}
}
%    \end{macrocode}

% \macro{\childdocby}
% The command |\childdocby| ....
%    \begin{macrocode}
\newcommand{\childdocby}[2][]
{
  \childdocdisable
  \childdoctrue
  \childdocmanualtrue
  \if?#1?\else
    \def\jobname{#2}
  \fi
  \def\childdocjob{#2}
  \input{#2}
  \endinput
}
%    \end{macrocode}

% \macro{\childdocforward}
% The command |\childdocforward| redirects
% compilation to the main file or
% (if the optional argument is given) a child file.
% Parameters are set as if the main file
% or a child file starting with |\childdocof| was compiled.
% Then compilation is handed over to the main file:
%    \begin{macrocode}
\newcommand{\childdocforward}[2][]
{
  \begingroup
    \if?#1?
      \def\childdoctmp
      {
        \def\childdocname{#2}
        \def\childdocjob{#2}
        \def\jobname{#2}
        \input{#2}
        \endinput
      }
    \else
      \def\childdoctmp
      {
        \childdocdisable
        \def\childdocname{#2}
        \childdoctrue
        \includeonly{#2}
        \def\childdocjob{#1}
        \def\jobname{#1}
        \input{#1}
        \endinput
      }
    \fi
    \expandafter
  \endgroup
  \childdoctmp
}
%    \end{macrocode}

% \macro{\childdocforwardprefix}
% The command |\childdocforwardprefix| redirects
% compilation to the main or a child file by means of a pattern.
% The prefix |#1| in the current filename is replaced by |#2|
% and the suffix of the current filename is kept
% (it is assumed that the filename does not contain the substring `|~~~|'
% which is used as a delimiter).
% Compilation is handed over to the new file by |\childdocforward|:
%    \begin{macrocode}
\newcommand{\childdocforwardprefix}[3][]
{
  \begingroup
    \def\childdocextract #2##1~~~{\def\childdoctmp{\childdocforward[#1]{#3##1}}}
    \expandafter\childdocextract\childdocname~~~
    \expandafter
  \endgroup
  \childdoctmp
}
%    \end{macrocode}

% \macro{\childdoc}
% The deprecated macro |\childdoc| is a legacy version of |\childdocmain|:
%    \begin{macrocode}
\newcommand{\childdoc}{\childdocmain}
%    \end{macrocode}

% \macro{\childdocredirect}
% The deprecated macro |\childdocredirect| is a legacy version
% of |\childdocforward| and |\childdocforwardprefix|:
%    \begin{macrocode}
\newcommand{\childdocredirect}[2][]
{
  \begingroup
    \if?#1?
      \def\childdoctmp{\childdocforward{#2}}
    \else
      \def\childdoctmp{\childdocforwardprefix{#1}{#2}}
    \fi
    \expandafter
  \endgroup
  \childdoctmp
}
%    \end{macrocode}

%\iffalse
%</package>
%\fi
%
\endinput
|\\
|\childdocby{|\textit{main}|}|\\
\end{tabular}
\end{center}
%
The directive |\childdocby| is similar to |\childdocof|
described in \secref{sec:include},
but the subsequent selection of content must be done manually.
To that end, both |\ifchilddoc| and |\ifchilddocmanual|
will be true upon processing of a part,
and the name of the part is stored in |\childdocname|.
Note that |\jobname| will be set to the filename of the current part
so that each part receives an individual |.aux| file
that does not interfere with the |.aux| file(s) of the main document.
This behaviour can be altered by the alternative form
|\childdocby[*]{|\textit{main}|}| (with a non-empty optional argument)
which uses the |.aux| file of the main document
by setting |\jobname| to \textit{main}.

%%%%%%%%%%%%%%%%%%%%%%%%%%%%%%%%%%%%%%%%%%%%%%%%%%%%%%%%%%%%%%%%%%%%%%%%%%%%%%%%
\subsection{Driver Development}
\label{sec:driver}

The \textsf{childdoc} mechanism can also be use for the development
of definition files such as \LaTeX{} styles or classes.
This case differs from the above setup with multiple parts
included by |\include| in that no |\includeonly| should be invoked.
This can be achieved by starting the include file
(before |\ProvidesPackage|) with:
%
\begin{center}
\begin{tabular}{l}
|% \iffalse
%
% childdoc.dtx Copyright (C) 2017-2018 Niklas Beisert
%
% This work may be distributed and/or modified under the
% conditions of the LaTeX Project Public License, either version 1.3
% of this license or (at your option) any later version.
% The latest version of this license is in
%   http://www.latex-project.org/lppl.txt
% and version 1.3 or later is part of all distributions of LaTeX
% version 2005/12/01 or later.
%
% This work has the LPPL maintenance status `maintained'.
%
% The Current Maintainer of this work is Niklas Beisert.
%
% This work consists of the files childdoc.dtx and childdoc.ins
% and the derived files childdoc.def and cdocsamp.tex with
% cdocsch1.tex, cdocsch2.tex, cdocsdrf.tex, cdocsfn1.tex, cdocsfn2.tex.
%
%<package>\ifdefined\childdocmain\endinput\fi
%<package>\ProvidesFile{childdoc.def}[2018/12/30 v2.0 child document driver]
%<samplemain>\ProvidesFile{cdocsamp.tex}[2018/12/30 v2.0 sample for childdoc]
%<*driver>
%\ProvidesFile{childdoc.drv}[2018/12/30 v2.0 childdoc reference manual file]
\PassOptionsToClass{10pt,a4paper}{article}
\documentclass{ltxdoc}

\usepackage[margin=35mm]{geometry}
\usepackage{hyperref}
\usepackage{hyperxmp}
\usepackage[usenames]{color}

\hypersetup{colorlinks=true}
\hypersetup{pdfstartview=FitH}
\hypersetup{pdfpagemode=UseNone}
\hypersetup{pdfsource={}}
\hypersetup{pdflang={en-UK}}
\hypersetup{pdfcopyright={Copyright 2017-2018 Niklas Beisert.
  This work may be distributed and/or modified under the
  conditions of the LaTeX Project Public License, either version 1.3
  of this license or (at your option) any later version.}}
\hypersetup{pdflicenseurl={http://www.latex-project.org/lppl.txt}}
\hypersetup{pdfcontactaddress={ETH Zurich, ITP, HIT K,
  Wolfgang-Pauli-Strasse 27}}
\hypersetup{pdfcontactpostcode={8093}}
\hypersetup{pdfcontactcity={Zurich}}
\hypersetup{pdfcontactcountry={Switzerland}}
\hypersetup{pdfcontactemail={nbeisert@itp.phys.ethz.ch}}
\hypersetup{pdfcontacturl={http://people.phys.ethz.ch/\xmptilde nbeisert/}}

\newcommand{\secref}[1]{\hyperref[#1]{section \ref*{#1}}}

\parskip1ex
\parindent0pt
\let\olditemize\itemize
\def\itemize{\olditemize\parskip0pt}

\begin{document}

\title{The \textsf{childdoc} Package}
\hypersetup{pdftitle={The childdoc Package}}
\author{Niklas Beisert\\[2ex]
  Institut f\"ur Theoretische Physik\\
  Eidgen\"ossische Technische Hochschule Z\"urich\\
  Wolfgang-Pauli-Strasse 27, 8093 Z\"urich, Switzerland\\[1ex]
  \href{mailto:nbeisert@itp.phys.ethz.ch}
  {\texttt{nbeisert@itp.phys.ethz.ch}}}
\hypersetup{pdfauthor={Niklas Beisert}}
\hypersetup{pdfsubject={Manual for the LaTeX2e Package childdoc}}
\date{30 December 2018, \textsf{v2.0}}
\maketitle

\begin{abstract}\noindent
\textsf{childdoc} is a \LaTeXe{} package
that enables the direct compilation
of document sections included by |\include|
to individual files.
\end{abstract}

\begingroup
\parskip0ex
\tableofcontents
\endgroup

%%%%%%%%%%%%%%%%%%%%%%%%%%%%%%%%%%%%%%%%%%%%%%%%%%%%%%%%%%%%%%%%%%%%%%%%%%%%%%%%
%%%%%%%%%%%%%%%%%%%%%%%%%%%%%%%%%%%%%%%%%%%%%%%%%%%%%%%%%%%%%%%%%%%%%%%%%%%%%%%%
\section{Introduction}

\LaTeX{} provides a mechanism to structure a large document (such as a book)
into a main file and several child files (containing the chapters)
using the |\include| command.
This mechanism is beneficial for documents
which span hundreds of pages in order to
make the source file(s) more manageable.
Moreover, compilation can be restricted to
selected child files by means of the |\includeonly| command.
The latter feature can be used to reduce the compilation time while editing
(this was significantly more useful in the earlier days of \LaTeX{})
or to generate a smaller document which is easier to navigate.
Another application of |\includeonly| is to generate
documents consisting of selected parts of the complete document.

However, there are a few drawbacks of the plain |\include| mechanism:
\begin{itemize}
\item
The child files cannot be compiled on their own,
they can only be compiled via the main file.
A naive editing environment
(such as a text editor with an option
to have the current file processed by \LaTeX)
may require one to switch to the main file before compiling;
attempting to compile the child file produces errors.
\item
The main file must be modified (each time)
to adjust the |\includeonly| command
to the present needs. This easily leaves the main file in a messy state.
\item
The generated document will always carry the filename
of the main document. This is inconvenient if
several child files are to be compiled and
to be kept for distribution.
\end{itemize}

The present package provides a simple interface
to make child files individually compilable by \LaTeX{}.
Compiling a child file then has the same effect as compiling
the main file with an |\includeonly| command
to select the appropriate child.
Moreover the generated document will carry the name of the child
rather than the main file.
This resolves all three above issues.

This feature is meant to make the editing of books,
thesis documents and lecture notes somewhat more convenient.
However, the package can also be used efficiently for
composing a series of documents (such as exercise sheets)
which are typically distributed individually.
It then assists the author in generating the individual documents
(potentially in different versions)
as well as a document containing the collected series.
Another application is in developing style files
or other kinds of included material
where compilation of the style file could redirect
to a sample or test file.

%%%%%%%%%%%%%%%%%%%%%%%%%%%%%%%%%%%%%%%%%%%%%%%%%%%%%%%%%%%%%%%%%%%%%%%%%%%%%%%%
%%%%%%%%%%%%%%%%%%%%%%%%%%%%%%%%%%%%%%%%%%%%%%%%%%%%%%%%%%%%%%%%%%%%%%%%%%%%%%%%
\section{Usage}

First of all, the package \textsf{childdoc} is \emph{not} a standard
\LaTeXe{} |.sty| style file! Therefore it needs to be invoked in
a non-standard way.

%%%%%%%%%%%%%%%%%%%%%%%%%%%%%%%%%%%%%%%%%%%%%%%%%%%%%%%%%%%%%%%%%%%%%%%%%%%%%%%%
\subsection{Included Files}
\label{sec:include}

%%%%%%%%%%%%%%%%%%%%%%%%%%%%%%%%%%%%%%%%
\DescribeMacro{\childdocmain}
To use the package, add the commands
\begin{center}
\begin{tabular}{l}
|\input{childdoc.def}|\\
|\childdocmain{}|\\
\end{tabular}
\end{center}
at the very top of the main \LaTeX{} file,
in particular \emph{before} the |\documentclass| statement!
The argument of |\childdocmain| should be left empty
(but it must be present).

%%%%%%%%%%%%%%%%%%%%%%%%%%%%%%%%%%%%%%%%
\DescribeMacro{\childdocof}
Furthermore, add the commands
\begin{center}
\begin{tabular}{l}
|\input{childdoc.def}|\\
|\childdocof{|\textit{main}|}|\\
\end{tabular}
\end{center}
at the top of every child file \textit{child}
which is included by |\include{|\textit{child}|}|
from within the main file
(or at least for those files to be compiled individually).
The argument \textit{main} must be the filename of the main file.

There are a couple of
considerations in setting up the main and child documents:

%%%%%%%%%%%%%%%%%%%%%%%%%%%%%%%%%%%%%%%%
\paragraph{Restrictions.}

Please note the following restrictions:
\begin{itemize}
\item
|\childdocmain| must be called with one argument \textit{main}
to ensure compatibility with earlier version of the package.
It must either be empty (|\childdocmain{}|)
or precisely match the filename of the main file in which it is specified.
See \secref{sec:detection} for further information.
\item
The filename \textit{main} must be specified without the |.tex| extension.
\item
The filename \textit{main} is case sensitive
(even in case-insensitive file systems)
due to internal string comparison.
\item
The argument \textit{main} should be fully expanded, it cannot be a macro.
\item
Subdirectories and special characters should be avoided in filenames.
\item
The command |\childdocmain{|\textit{main}|}| must be followed by a whitespace.
It should not be followed immediately by another command
or by a comment mark `|%|'.
This is because the \TeX{} parser reads the token immediately following
the argument of |\childdocmain| and puts it
at the beginning of every child section;
however, a white\-space is ignored.
\end{itemize}

%%%%%%%%%%%%%%%%%%%%%%%%%%%%%%%%%%%%%%%%
\paragraph{Content of Main File.}

It is advisable to place all content in the child files included by |\include|.
Any output contained in the main file will appear in all child documents
unless suppressed manually;
it cannot be suppressed automatically by the |\includeonly| directive
and thus should normally be avoided.
A method to include some content in the main file
by means of conditional processing is described in \secref{sec:conditional}.

%%%%%%%%%%%%%%%%%%%%%%%%%%%%%%%%%%%%%%%%
\paragraph{Page Numbering.}

When only a part of the document is compiled,
the appropriate numbering of pages
(as well as other status parameters)
is determined from the |.aux| files.
The latter contain information from previous passes.
However this information needs to propagate through
all intermediate child documents.
Therefore the page numbering in child documents may well
be inconsistent until the complete document is compiled at least once.

A useful (if unconventional) way to always ensure a consistent
page numbering is to restart the numbering in each child document
and denote the pages by `\textit{child}|.|\textit{page}'
where \textit{child} represents the chapter/section number of the child file.
This can be achieved by the command
|\numberwithin{page}{|\textit{child}|}|
of the \textsf{amsmath} package
where \textit{child} can be |chapter| or |section|
depending on the chosen structuring.
Alternatively, one can modify the macro |\thepage| appropriately
and reset the counter |page| at the start of each child file.

%%%%%%%%%%%%%%%%%%%%%%%%%%%%%%%%%%%%%%%%%%%%%%%%%%%%%%%%%%%%%%%%%%%%%%%%%%%%%%%%
\subsection{Conditional Processing}
\label{sec:conditional}

The package provides a mechanism to compile different versions
of a document. To customise the versions further some conditional processing
can come in handy to distinguish which version is being compiled.
The package provides two macros to describe the compilation context:

%%%%%%%%%%%%%%%%%%%%%%%%%%%%%%%%%%%%%%%%
\DescribeMacro{\ifchilddoc}
The conditional |\ifchilddoc| distinguishes between the compilation of
child documents and the main document:
%
\begin{center}
|\ifchilddoc |\textit{child-code}| |[|\||else |\textit{main-code}]| \||fi|
\end{center}

%%%%%%%%%%%%%%%%%%%%%%%%%%%%%%%%%%%%%%%%
\DescribeMacro{\childdocname}
\DescribeMacro{\childdocjob}
The macro |\childdocname| contains the filename (without extension)
of the main or child file being processed.
Note that |\childdocjob| will always contain the name of the main file.

%%%%%%%%%%%%%%%%%%%%%%%%%%%%%%%%%%%%%%%%
\paragraph{Title Page.}

Conditional processing can be used to include a title or banner page
in the main document when proper precautions are taken.
Importantly, the code in the main file should ensure that the page counter
(as well as other status parameters which are stored in the |.aux| files)
takes the same value after the conditional processing.
Otherwise the page numbers may take divergent values
depending on which part is compiled.

For example, a title page could be declared by:
%
\begin{center}
\begin{tabular}{l}
|\ifchilddoc\||else|\\
|\addtocounter{page}{-1}|\\
\textit{code for title page}\\
|\newpage|\\
|\||fi|
\end{tabular}
\end{center}
%
A banner page for the child documents can be generated by:
%
\begin{center}
\begin{tabular}{l}
|\ifchilddoc|\\
|\addtocounter{page}{-1}|\\
\textit{code for banner page}\\
|\newpage|\\
|\||fi|
\end{tabular}
\end{center}
%
Here one could write a message such as:
\begin{center}
|This is the part \childdocname{} of \childdocjob{}.|
\end{center}

%%%%%%%%%%%%%%%%%%%%%%%%%%%%%%%%%%%%%%%%%%%%%%%%%%%%%%%%%%%%%%%%%%%%%%%%%%%%%%%%
\subsection{Flags}
\label{sec:flags}

The package makes it easy to generate different versions
of the main or child documents.
To this end compilation flags can be defined
and assigned different default values.
They will be particularly useful in conjunction
with the forwarding mechanism described in \secref{sec:forward}.

For example, it may be useful to have a flag |\version|
which can be set to |draft| or |final|.
The document source will contain some conditional code
depending on the value of |\version|.
Suppose further, the flag should default to |final| for the main file
and to |draft| for child files
which is a natural assignment for editing the document.
This is achieved by placing the following code
in the preamble of the main document
(below the |\childdocmain| directive):
%
\begin{center}
\begin{tabular}{l}
|\ifchilddoc|\\
|\providecommand{\version}{draft}|\\
|\||else|\\
|\providecommand{\version}{final}|\\
|\||fi|
\end{tabular}
\end{center}
%
The definition by |\providecommand| makes sure
that previous definitions are not overwritten.
Further statements |\providecommand{\version}{...}|
can thus be added before the above code to override it.

For the main file, one might add a line
(between |\childdocmain| and the above block)
%
\begin{center}
|%\ifchilddoc\||else\providecommand{\version}{draft}\||fi|
\end{center}
%
which can be uncommented to produce a draft version.
Likewise one can add a line to the very top of a child file
(above the |\childdocof{|\textit{main}|}| directive)
%
\begin{center}
|%\providecommand{\version}{final}|
\end{center}
%
which can be uncommented to produce the final version of this child document.

%%%%%%%%%%%%%%%%%%%%%%%%%%%%%%%%%%%%%%%%%%%%%%%%%%%%%%%%%%%%%%%%%%%%%%%%%%%%%%%%
\subsection{Forwarding}
\label{sec:forward}

Different versions of the main or child documents
using compilation flags as described in \secref{sec:flags}
can be (permanently) stored in different files
for convenient compilation, viewing and distribution.
To this end, the package defines a command
to pass on compilation to a different file:

%%%%%%%%%%%%%%%%%%%%%%%%%%%%%%%%%%%%%%%%
\DescribeMacro{\childdocforward}
The command |\childdocforward| redirects processing to
another source file:
%
\begin{center}
\begin{tabular}{l}
|\input{childdoc.def}|\\
|\childdocforward[|\textit{main}|]{|\textit{dest}|}|\\
\end{tabular}
\end{center}
%
The argument \textit{dest} is the destination file
(without extension).
It should be the main file or one of the child files.
Note that further \textsf{childdoc} directives
such as |\childdocof| and |\childdocforward|
in the indicated file will be processed in this form.
The optional argument \textit{main}
passes on directly to the main file \textit{main}
while pretending to compile the child \textit{dest}.
This form behaves as if \textit{dest}
issues |\childdocof{|\textit{main}|}| right away,
and no further \textsf{childdoc} directives will be processed.

%%%%%%%%%%%%%%%%%%%%%%%%%%%%%%%%%%%%%%%%
\DescribeMacro{\...prefix}
In the alternative form |\childdocforwardprefix|,
%
\begin{center}
\begin{tabular}{l}
|\input{childdoc.def}|\\
|\childdocforwardprefix[|\textit{main}|]{|\textit{prefix}|}{|\textit{dest}|}|
\end{tabular}
\end{center}
%
the destination file is determined by a pattern
depending on the current file:
To make this work, the current file must be called
`{\textit{prefix}\hspace{0.2em}\textit{suffix}}'
with \textit{prefix} matching precisely the argument.
Processing is then passed on to the file
`{\textit{dest}\hspace{0.2em}\textit{suffix}}'.
Surely, the same effect is achieved by
directly specifying the
argument `{\textit{dest}\hspace{0.2em}\textit{suffix}}'
in the first form.
However, that requires to set up a different file
for each child. With the alternative form of the command
all these files can have exactly the same content
which simplifies setting them up and maintaining them.

For example, the following file |draft.tex|
with a compilation flag |\version| as described in \secref{sec:flags}
compiles the main document as a draft:
%
\begin{center}
\begin{tabular}{l}
|\def\version{draft}|\\
|\input{childdoc.def}|\\
|\childdocforward{|\textit{main}|}|
\end{tabular}
\end{center}
%
Likewise, the following files |final|\textit{nn}|.tex|
compile the final version of the child document
|child|\textit{nn}|.tex|:
%
\begin{center}
\begin{tabular}{l}
|\def\version{final}|\\
|\input{childdoc.def}|\\
|\childdocforwardprefix{final}{child}|
\end{tabular}
\end{center}
%

Note that when several versions of a main file and/or of each child file
are to be generated, it may be convenient to set up a |Makefile| or
shell script to automatise the process.

%%%%%%%%%%%%%%%%%%%%%%%%%%%%%%%%%%%%%%%%%%%%%%%%%%%%%%%%%%%%%%%%%%%%%%%%%%%%%%%%
\subsection{Command Line Processing}
\label{sec:commandline}

The effect of redirection files can also be achieved by invoking
the \LaTeX{} compiler with a more elaborate command line.
Most conveniently this should be done as part
of a shell script or a |Makefile|.

When using \textsf{childdoc} in the main file, the following
command lines effectively perform a redirection
(note that depending on the shell being used,
backslashes may have to be doubled: `|\|' $\to$ `|\\|'):
%
\begin{center}
|... -jobname "|\textit{target}|" |\\|"|[\textit{flags}]%
|\input{childdoc.def}\childdocforward[|\textit{main}|]{|\textit{dest}|}"|
\end{center}
%
Here \textit{target} is the name of the output file,
\textit{main} is the name of the main file
and \textit{dest} is the name of the main or child file to be processed
(all filenames without extensions).
The optional argument \textit{main} can be omitted
if \textit{main} matches \textit{dest}.
Optionally, compilation \textit{flags} can be defined via |\def| commands.
This command line makes the \TeX{} engine believe
it is compiling the file \textit{target}
whose content is specified as the latter parameter.
The provided code then forwards the processing to
\textit{main} or \textit{dest} as described in \secref{sec:forward}.

%%%%%%%%%%%%%%%%%%%%%%%%%%%%%%%%%%%%%%%%%%%%%%%%%%%%%%%%%%%%%%%%%%%%%%%%%%%%%%%%
\subsection{Include by Input}
\label{sec:input}

Including child documents by |\include| has some restrictions by design.
Most notably, the content of a child document always occupies
its own set of pages; pages cannot be shared between child documents.
Usually, this behaviour makes perfect sense
because each child document contain an essential part of the document.
However, in some situations it may be desirable to compose
a document from a collection of parts
without having mandatory page breaks between then.
For this case, the package
provides a mechanism to include parts
by |\input| which can also be processed individually.
However, by construction this mechanism
requires manual handling of the content to be output.

%%%%%%%%%%%%%%%%%%%%%%%%%%%%%%%%%%%%%%%%
\DescribeMacro{\ifchilddocmanual}
The main file should be prepared as usual, see \secref{sec:include}.
However, the document body must make a distinction
between processing of an individual part and of the main document, e.g.:
%
\begin{center}
\begin{tabular}{l}
|\ifchilddocmanual|\\
|\input{\childdocname}|\\
|\||else|\\
\textit{document body with }|\input{|\textit{part}|}|\\
|\||fi|
\end{tabular}
\end{center}
%
The conditional |\ifchilddocmanual| is true whenever
a part to be included by |\input| is being compiled,
and the name of the part is stored in |\childdocname|.

%%%%%%%%%%%%%%%%%%%%%%%%%%%%%%%%%%%%%%%%
\DescribeMacro{\childdocby}
Each part to be included by |\input| should start with:
%
\begin{center}
\begin{tabular}{l}
|\input{childdoc.def}|\\
|\childdocby{|\textit{main}|}|\\
\end{tabular}
\end{center}
%
The directive |\childdocby| is similar to |\childdocof|
described in \secref{sec:include},
but the subsequent selection of content must be done manually.
To that end, both |\ifchilddoc| and |\ifchilddocmanual|
will be true upon processing of a part,
and the name of the part is stored in |\childdocname|.
Note that |\jobname| will be set to the filename of the current part
so that each part receives an individual |.aux| file
that does not interfere with the |.aux| file(s) of the main document.
This behaviour can be altered by the alternative form
|\childdocby[*]{|\textit{main}|}| (with a non-empty optional argument)
which uses the |.aux| file of the main document
by setting |\jobname| to \textit{main}.

%%%%%%%%%%%%%%%%%%%%%%%%%%%%%%%%%%%%%%%%%%%%%%%%%%%%%%%%%%%%%%%%%%%%%%%%%%%%%%%%
\subsection{Driver Development}
\label{sec:driver}

The \textsf{childdoc} mechanism can also be use for the development
of definition files such as \LaTeX{} styles or classes.
This case differs from the above setup with multiple parts
included by |\include| in that no |\includeonly| should be invoked.
This can be achieved by starting the include file
(before |\ProvidesPackage|) with:
%
\begin{center}
\begin{tabular}{l}
|\input{childdoc.def}|\\
|\childdocforward{|\textit{main}|}|\\
\end{tabular}
\end{center}
%
or alternatively with:
%
\begin{center}
\begin{tabular}{l}
|\input{childdoc.def}|\\
|\childdocby{|\textit{main}|}|\\
\end{tabular}
\end{center}
%
Both forms have slightly different effects as described above.
The main file is prepared as usual, see \secref{sec:include}.

%%%%%%%%%%%%%%%%%%%%%%%%%%%%%%%%%%%%%%%%%%%%%%%%%%%%%%%%%%%%%%%%%%%%%%%%%%%%%%%%
\subsection{Legacy Detection}
\label{sec:detection}

The directive |\childdocmain| in the main file can detect
whether the complete document or merely a child is to be compiled
even without using the directive |\childdocof|.
This method is deprecated because it is less robust
and there is no compelling reason to use it;
it is merely provided for backward compatibility
and it may be removed in future versions.

If the detection mechanism is to be used,
it is mandatory to correctly specify
the filename of the main file as the argument of |\childdocmain|:
%
\begin{center}
\begin{tabular}{l}
|\input{childdoc.def}|\\
|\childdocmain{|\textit{main}|}|\\
\end{tabular}
\end{center}
%
If |\jobname| does not match the argument \textit{main} of |\childdocmain|,
it is assumed that |\jobname| points to the child file to be compiled.
When using |\childdocmain| with the main file specified as argument,
it suffices to start a child file
with just |\input{|\textit{main}|}|
without loading of the package and using |\childdocof|.
If instead all processing is done
with the appropriate \textsf{childdoc} directives,
the argument of \textit{main} of |\childdocmain| can be empty.

An alternative version of the command line processing described
in \secref{sec:commandline} using the detection mechanism reads:
%
\begin{center}
|... -jobname "|\textit{target}|" "|[\textit{flags}]%
[|\def\jobname{|\textit{dest}|}|]|\input{|\textit{main}|}"|
\end{center}

%%%%%%%%%%%%%%%%%%%%%%%%%%%%%%%%%%%%%%%%%%%%%%%%%%%%%%%%%%%%%%%%%%%%%%%%%%%%%%%%
\subsection{Manual Code}
\label{sec:manual}

In case one cannot be certain whether the definitions file |childdoc.def|
is installed on the target \TeX{} distribution
and one prefers not to ship it,
it is conceivable to paste a few relevant commands into the sources.

To that end, drop all statements |\input{childdoc.def}|
and perform the replacements as outlined below.
Instead of |\childdocmain{|\textit{main}|}| add the following code
to the top of the main file:
%
\begin{center}
\begin{tabular}{l}
|\||ifdefined\childdocname\endinput\||fi\newif\ifchilddoc|\\
|\edef\childdocname{\scantokens\expandafter{\jobname\noexpand}}|\\
|\def\childdocmain{|\textit{main}|}\||ifx\childdocmain\childdocname\||else|\\
|\childdoctrue\includeonly{\childdocname}\let\jobname\childdocmain\||fi|\\
\end{tabular}
\end{center}
%
Instead of |\childdocof{|\textit{main}|}| just include the main file
at the top of each child file:
%
\begin{center}
|\input{|\textit{main}|}|
\end{center}
%
A simple redirection |\childdocforward{|\textit{dest}|}| is achieved by:
%
\begin{center}
|\def\jobname{|\textit{dest}|}\input{\jobname}|
\end{center}
%
The redirection with prefix
|\childdocforwardprefix[|\textit{prefix}|]{|\textit{dest}|}|
is accomplished by:
%
\begin{center}
\begin{tabular}{l}
|{\edef\jobname{\scantokens\expandafter{\jobname\noexpand}}|\\
|\def\redirectjob |\textit{prefix}|#1~~~{\gdef\jobname{|\textit{dest}|#1}}|\\
|\expandafter\redirectjob\jobname~~~}\input{\jobname}|
\end{tabular}
\end{center}

In an alternative approach,
child documents can be compiled by a specific command line
without additional code or specific definitions:
%
\begin{center}
|... -jobname "|\textit{target}|" "|[\textit{flags}]%
|\includeonly{|\textit{dest}|}\input{|\textit{main}|}"|
\end{center}
%

%%%%%%%%%%%%%%%%%%%%%%%%%%%%%%%%%%%%%%%%%%%%%%%%%%%%%%%%%%%%%%%%%%%%%%%%%%%%%%%%
%%%%%%%%%%%%%%%%%%%%%%%%%%%%%%%%%%%%%%%%%%%%%%%%%%%%%%%%%%%%%%%%%%%%%%%%%%%%%%%%
\section{Information}

%%%%%%%%%%%%%%%%%%%%%%%%%%%%%%%%%%%%%%%%%%%%%%%%%%%%%%%%%%%%%%%%%%%%%%%%%%%%%%%%
\subsection{Copyright}

Copyright \copyright{} 2017--2018 Niklas Beisert

This work may be distributed and/or modified under the
conditions of the \LaTeX{} Project Public License, either version 1.3
of this license or (at your option) any later version.
The latest version of this license is in
  \url{http://www.latex-project.org/lppl.txt}
and version 1.3 or later is part of all distributions of \LaTeX{}
version 2005/12/01 or later.

This work has the LPPL maintenance status `maintained'.

The Current Maintainer of this work is Niklas Beisert.

This work consists of the files |README.txt|, |childdoc.ins| and |childdoc.dtx|
as well as the derived files |childdoc.def|, |cdocsamp.tex|
with |cdocsch1.tex|, |cdocsch2.tex|, |cdocspt3.tex|, |cdocspt4.tex|,
|cdocsdrf.tex|, |cdocsfn1.tex|, |cdocsfn2.tex|
as well as |childdoc.pdf|.

%%%%%%%%%%%%%%%%%%%%%%%%%%%%%%%%%%%%%%%%%%%%%%%%%%%%%%%%%%%%%%%%%%%%%%%%%%%%%%%%
\subsection{Files and Installation}

The package consists of the files:
%
\begin{center}
\begin{tabular}{ll}
    |README.txt|   & readme file \\
    |childdoc.ins| & installation file \\
    |childdoc.dtx| & source file \\
    |childdoc.def| & definition file \\
    |cdocsamp.tex| & sample main file \\
    |cdocsch1.tex| & sample include file \\
    |cdocsch2.tex| & sample include file \\
    |cdocspt3.tex| & sample part file \\
    |cdocspt4.tex| & sample part file \\
    |cdocsdrf.tex| & sample redirection file \\
    |cdocsfn1.tex| & sample redirection file \\
    |cdocsfn2.tex| & sample redirection file \\
    |childdoc.pdf| & manual
\end{tabular}
\end{center}
%
The distribution consists of the files
|README.txt|, |childdoc.ins| and |childdoc.dtx|.
%
\begin{itemize}
\item
Run (pdf)\LaTeX{} on |childdoc.dtx|
to compile the manual |childdoc.pdf| (this file).
\item
Run \LaTeX{} on |childdoc.ins| to create the definitions file |childdoc.def|
and the sample |cdocsamp.tex| with include files
|cdocsch1.tex|, |cdocsch2.tex|, |cdocspt3.tex|, |cdocspt4.tex|,
|cdocsdrf.tex|, |cdocsfn1.tex|, |cdocsfn2.tex|.
Then copy the file |childdoc.def| to an appropriate directory of your \LaTeX{}
distribution, e.g.\ \textit{texmf-root}|/tex/latex/childdoc|.
\end{itemize}

%%%%%%%%%%%%%%%%%%%%%%%%%%%%%%%%%%%%%%%%%%%%%%%%%%%%%%%%%%%%%%%%%%%%%%%%%%%%%%%%
\subsection{Related CTAN Packages}

There are several other packages which offer a similar functionality:
%
\begin{itemize}
\item
The packages
\href{http://ctan.org/pkg/docmute}{\textsf{docmute}},
\href{http://ctan.org/pkg/includex}{\textsf{includex}} and
\href{http://ctan.org/pkg/standalone}{\textsf{standalone}}
provide commands to include only the document body of
a child file thus allowing both files to be compiled individually.
\item
The packages \href{http://ctan.org/pkg/subdocs}{\textsf{subdocs}}
and \href{http://ctan.org/pkg/subfiles}{\textsf{subfiles}}
provide structures in which the main and child documents can be
encapsulated and allowing them to be compiled individually.
The inclusion mechanism is different from the conventional |\include|.
\item
The package \href{http://ctan.org/pkg/combine}{\textsf{combine}}
is an elaborate solution to combine several documents into one.
\end{itemize}
%
See also the CTAN topic \href{http://ctan.org/topic/subdocs}{\textsf{subdocs}}
for further related packages.
The present package differs from the above solutions in that
a document structure constructed with the conventional |\include| mechanism
just needs two extra commands at the top of every file
such that all constituent files can be compiled individually.

%%%%%%%%%%%%%%%%%%%%%%%%%%%%%%%%%%%%%%%%%%%%%%%%%%%%%%%%%%%%%%%%%%%%%%%%%%%%%%%%
%\subsection{Feature Suggestions}
%
%The following is a list of features which may be useful for future
%versions of this package:
%%
%\begin{itemize}
%\item
%\ldots
%\end{itemize}

%%%%%%%%%%%%%%%%%%%%%%%%%%%%%%%%%%%%%%%%%%%%%%%%%%%%%%%%%%%%%%%%%%%%%%%%%%%%%%%%
\subsection{Revision History}

%%%%%%%%%%%%%%%%%%%%%%%%%%%%%%%%%%%%%%%%
\paragraph{v2.0:} 2018/12/30

\begin{itemize}
\item
immediate forward processing
\item
added |\childdocby| mechanism
\item
manual restructured
\end{itemize}

%%%%%%%%%%%%%%%%%%%%%%%%%%%%%%%%%%%%%%%%
\paragraph{v1.6:} 2018/01/17

\begin{itemize}
\item
application for development of include files
\item
corrections to manual
\end{itemize}

%%%%%%%%%%%%%%%%%%%%%%%%%%%%%%%%%%%%%%%%
\paragraph{v1.5:} 2017/05/21

\begin{itemize}
\item
more complete structuring introduced
\item
|\childdocof| introduced
\item
|\childdoc| renamed to |\childdocmain|
\item
|\childredirect| renamed to |\childdocforward| and |\childdocforwardprefix|
and functionality expanded
\end{itemize}

%%%%%%%%%%%%%%%%%%%%%%%%%%%%%%%%%%%%%%%%
\paragraph{v1.0:} 2017/04/27

\begin{itemize}
\item
manual and install package
\item
first version published on CTAN
\end{itemize}

%%%%%%%%%%%%%%%%%%%%%%%%%%%%%%%%%%%%%%%%
\paragraph{v0.6:} 2017/04/26

\begin{itemize}
\item
redirection mechanism added
\end{itemize}

%%%%%%%%%%%%%%%%%%%%%%%%%%%%%%%%%%%%%%%%
\paragraph{v0.5:} 2017/04/26

\begin{itemize}
\item
functionality in definition file
\end{itemize}


%%%%%%%%%%%%%%%%%%%%%%%%%%%%%%%%%%%%%%%%%%%%%%%%%%%%%%%%%%%%%%%%%%%%%%%%%%%%%%%%
%%%%%%%%%%%%%%%%%%%%%%%%%%%%%%%%%%%%%%%%%%%%%%%%%%%%%%%%%%%%%%%%%%%%%%%%%%%%%%%%
%%%%%%%%%%%%%%%%%%%%%%%%%%%%%%%%%%%%%%%%%%%%%%%%%%%%%%%%%%%%%%%%%%%%%%%%%%%%%%%%
\appendix

\settowidth\MacroIndent{\rmfamily\scriptsize 000\ }

 \DocInput{childdoc.dtx}

\end{document}
%</driver>
% \fi
%
% %%%%%%%%%%%%%%%%%%%%%%%%%%%%%%%%%%%%%%%%%%%%%%%%%%%%%%%%%%%%%%%%%%%%%%%%%%%%%%
% %%%%%%%%%%%%%%%%%%%%%%%%%%%%%%%%%%%%%%%%%%%%%%%%%%%%%%%%%%%%%%%%%%%%%%%%%%%%%%
% \section{Sample}
%\iffalse
%<*samplemain>
%\fi
%
% The following presents a sample document
% with two chapters, two parts, a title page,
% a compile flag as well as three forwarding files to set the flag.
% It consists of eight |.tex| files:
% \begin{center}
% \begin{tabular}{ll}
% |cdocsamp.tex|&main file\\
% |cdocsch1.tex|&include file for chapter 1\\
% |cdocsch2.tex|&include file for chapter 2\\
% |cdocspt3.tex|&include file for part 3\\
% |cdocspt4.tex|&include file for part 4\\
% |cdocsdrf.tex|&forwarding file for main file in draft mode\\
% |cdocsfi1.tex|&forwarding file for final version of chapter 1\\
% |cdocsfi2.tex|&forwarding file for final version of chapter 2\\
% \end{tabular}
% \end{center}
% Each of the eight files can be compiled directly by the \LaTeX{} compiler.
%
% %%%%%%%%%%%%%%%%%%%%%%%%%%%%%%%%%%%%%%
% \paragraph{Main File.}
%
% The main file is called |cdocsamp.tex|.
%
% Load the \textsf{childdoc} definitions and
% declare the filename for the main document:
%    \begin{macrocode}
\input{childdoc.def}
\childdocmain{}
%    \end{macrocode}

% Optional override for |\version| flag:
%    \begin{macrocode}
%%\ifchilddoc\else\providecommand{\version}{draft}\fi
%    \end{macrocode}

% Define the default values for the |\version| flag
% (|final| for the main file and |draft| for childs):
%    \begin{macrocode}
\ifchilddoc
\providecommand{\version}{draft}
\else
\providecommand{\version}{final}
\fi
%    \end{macrocode}

% Load the standard document class:
%    \begin{macrocode}
\documentclass[12pt]{article}
%    \end{macrocode}

% Start the document body:
%    \begin{macrocode}
\begin{document}
%    \end{macrocode}

% Declare a title page.
% Print title, part of document being processed and version flag:
%    \begin{macrocode}
\addtocounter{page}{-1}
\begin{center}
{\LARGE\bfseries{}childdoc example\par}
\vspace{1cm}
\ifchilddoc
\ifchilddocmanual part\else chapter\fi:
`\childdocname' of `\childdocjob'\par
\else
main document: `\childdocjob'\par
\fi
version: \version\par
\end{center}
\newpage
%    \end{macrocode}

% Manually include selected file,
% otherwise process as usual:
%    \begin{macrocode}
\ifchilddocmanual
\section*{part `\childdocname'}
\input{\childdocname}
\else
%    \end{macrocode}

% Include the two chapters:
%    \begin{macrocode}
\include{cdocsch1}
\include{cdocsch2}
%    \end{macrocode}

% Include the two parts unless only chapters should be displayed:
%    \begin{macrocode}
\ifchilddoc\else
\section{part three}
\input{cdocspt3}
\section{part four}
\input{cdocspt4}
\fi
%    \end{macrocode}

% Process as usual until here:
%    \begin{macrocode}
\fi
%    \end{macrocode}

% End of document body:
%    \begin{macrocode}
\end{document}
%    \end{macrocode}
%\iffalse
%</samplemain>
%\fi
%
% %%%%%%%%%%%%%%%%%%%%%%%%%%%%%%%%%%%%%%
% \paragraph{Chapter Include Files.}
%
% The include files are called |cdocsch1.tex| and |cdocsch2.tex|.
%
%\iffalse
%<*samplechap1|samplechap2>
%\fi

% Optional override for |\version| flag:
%    \begin{macrocode}
%%\providecommand{\version}{final}
%    \end{macrocode}

% Include the main document:
%    \begin{macrocode}
\input{childdoc.def}
\childdocof{cdocsamp}
%    \end{macrocode}

%\iffalse
%</samplechap1|samplechap2>
%\fi
%
%\iffalse
%<*samplechap1>
%\fi
% Some text for chapter 1:
%    \begin{macrocode}
\section{one}
some text in chapter one
%    \end{macrocode}

%\iffalse
%</samplechap1>
%\fi
% Some text for chapter 2:
%\iffalse
%<*samplechap2>
%\fi
%    \begin{macrocode}
\section{two}
more text in chapter two
%    \end{macrocode}

%\iffalse
%</samplechap2>
%\fi
%
% %%%%%%%%%%%%%%%%%%%%%%%%%%%%%%%%%%%%%%
% \paragraph{Part Include Files.}
%
% The include files are called |cdocspt3.tex| and |cdocspt4.tex|.
%
%\iffalse
%<*samplepart3|samplepart4>
%\fi

% Optional override for |\version| flag:
%    \begin{macrocode}
%%\providecommand{\version}{final}
%    \end{macrocode}

% Include the main document:
%    \begin{macrocode}
\input{childdoc.def}
\childdocby{cdocsamp}
%    \end{macrocode}

%\iffalse
%</samplepart3|samplepart4>
%\fi
%
%\iffalse
%<*samplepart3>
%\fi
% Some text for part 3:
%    \begin{macrocode}
some text in part three
%    \end{macrocode}

%\iffalse
%</samplepart3>
%\fi
% Some text for part 4:
%\iffalse
%<*samplepart4>
%\fi
%    \begin{macrocode}
more text in part four
%    \end{macrocode}

%\iffalse
%</samplepart4>
%\fi
%
% %%%%%%%%%%%%%%%%%%%%%%%%%%%%%%%%%%%%%%
% \paragraph{Forwarding for a Complete Draft.}
%
% The following forwarding file |cdocsdrf.tex|
% compiles the main document in draft mode:
%\iffalse
%<*sampledraft>
%\fi
%    \begin{macrocode}
\def\version{draft}
\input{childdoc.def}
\childdocforward{cdocsamp}
%    \end{macrocode}

%\iffalse
%</sampledraft>
%\fi
%
% %%%%%%%%%%%%%%%%%%%%%%%%%%%%%%%%%%%%%%
% \paragraph{Forwarding for Final Version of the Chapters.}
%
% The following forwarding files |cdocsfn1.tex| and |cdocsfn2.tex|
% (with identical content)
% compile the final versions of the child documents
% |cdocsch1.tex| and |cdocsch2.tex|, respectively:
%\iffalse
%<*samplefinal>
%\fi
%    \begin{macrocode}
\def\version{final}
\input{childdoc.def}
\childdocforwardprefix[cdocsamp]{cdocsfn}{cdocsch}
%    \end{macrocode}

%\iffalse
%</samplefinal>
%\fi
%
% %%%%%%%%%%%%%%%%%%%%%%%%%%%%%%%%%%%%%%
% \paragraph{Command Line Processing.}
%
% The following three command lines generate the output files
% |cdocscld|, |cdocscl1| and |cdocscl2|
% which should be identical to
% |cdocsdrf|, |cdocsch1| and |cdocsfn2|, respectively:
% \begin{center}
% \begin{tabular}{l}
% |latex -jobname cdocscld \|\\
% |  "\def\version{draft}\input{childdoc.def}\childdocforward{cdocsamp}"|\\
% |latex -jobname cdocscl1 \|\\
% |  "\input{childdoc.def}\childdocforward[cdocsamp]{cdocsch1}"|\\
% |latex -jobname cdocscl2 \|\\
% |  "\def\version{final}\input{childdoc.def}\childdocforward{cdocsch2}"|
% \end{tabular}
% \end{center}
% Note that the trailing backslash on each first line
% merely continues the input to the second line
% (for convenient cut ant paste).
% Furthermore, the command |latex| can be replaced by any
% of its alternative versions such as |pdflatex|.
%
% %%%%%%%%%%%%%%%%%%%%%%%%%%%%%%%%%%%%%%%%%%%%%%%%%%%%%%%%%%%%%%%%%%%%%%%%%%%%%%
% %%%%%%%%%%%%%%%%%%%%%%%%%%%%%%%%%%%%%%%%%%%%%%%%%%%%%%%%%%%%%%%%%%%%%%%%%%%%%%
% \section{Implementation}
%\iffalse
%<*package>
%\fi
%
% This section describes the definitions file |childdoc.def|.

% The definitions cannot be loaded using |\usepackage| or |\RequirePackage|
% which has a mechanism to prevent loading a style file more than once.
% When loading the definitions by means of |\input|
% multiple instances have to be prevented manually:
%\iffalse
%This code needs to be before the `\ProvidesFile' directive
%which is defined at the beginning of this file.
%Therefore it is also placed there and commented out here.
%</package>
%<*discard>
%\fi
%    \begin{macrocode}
\ifdefined\childdocmain\endinput\fi
%    \end{macrocode}
%\iffalse
%</discard>
%<*package>
%\fi
%
% \macro{\ifchilddoc}
% \macro{\ifchilddocmanual}
% The conditional |\ifchilddoc| tells whether a
% child (true) or main (false) document is being compiled.
% The conditional |\ifchilddocmanual| tells whether
% the |\includeonly| mechanism is used (false) or
% the selection of child files must be performed manually (true).
% The definitions initialise to false:
%    \begin{macrocode}
\newif\ifchilddoc
\newif\ifchilddocmanual
%    \end{macrocode}

% \macro{\childdocname}
% \macro{\childdocjob}
% The macro |\childdocname| stores the name of the main document
% to be compiled. The macro |\childdocjob| stores the name of
% the document on which the \LaTeX{} compiler was originally invoked.
% The content of |\jobname| cannot be compared
% to filenames specified in the source due to different catcodes.
% The following code rescans |\jobname|, stores the result
% in |\childdocname| and saves a copy in |\childdocjob|:
%    \begin{macrocode}
\edef\childdocname{\scantokens\expandafter{\jobname\noexpand}}
\let\childdocjob\childdocname
%    \end{macrocode}

% \macro{\childdocdisable}
% The macro |\childdocdisable| prevents the main file
% from being processed more than once.
% At this stage, the main document command |\childdocmain|
% is assumed to be called once again where it should do nothing.
% Any subsequent call to it should prevent
% a secondary processing of the main document
% It overwrites the forwarding commands
% |\childdocof| and |\childdocforward|
% with empty macros to prevent further inclusions of the main document:
%    \begin{macrocode}
\newcommand{\childdocdisable}
{
  \renewcommand{\childdocmain}[1]{\renewcommand{\childdocmain}[1]{\endinput}}
  \renewcommand{\childdocof}[1]{}
  \renewcommand{\childdocby}[2][]{}
  \renewcommand{\childdocforward}[2][]{}
  \renewcommand{\childdocdisable}{}
}
%    \end{macrocode}

% \macro{\childdocmain}
% The macro |\childdocmain| is to be called at the top of the main file
% with nothing or the main filename (without extension) as argument.
% First, it breaks loops.
% If the argument is not empty and does not match |\childdocname|
% (which is set by the first inclusion of |childdoc.def|),
% |\ifchilddoc| is set to true, |\includeonly| is applied to the child file
% and |\jobname| is set to the main file
% (for proper handling of |.aux| files):
%    \begin{macrocode}
\newcommand{\childdocmain}[1]
{
  \childdocdisable\childdocmain{}
  \if?#1?\else
    \begingroup
      \def\childdoctmp{#1}
      \ifx\childdoctmp\childdocname
        \def\childdoctmp{}
      \else
        \def\childdoctmp
        {
          \childdoctrue
          \includeonly{\childdocname}
          \def\childdocjob{#1}
          \def\jobname{#1}
        }
      \fi
      \expandafter
    \endgroup
    \childdoctmp
  \fi
}
%    \end{macrocode}

% \macro{\childdocof}
% The command |\childdocof| redirects
% compilation to the main file |#1|.
%    \begin{macrocode}
\newcommand{\childdocof}[1]
{
  \childdocdisable
  \childdoctrue
  \includeonly{\childdocname}
  \def\jobname{#1}
  \def\childdocjob{#1}
  \input{#1}
}
%    \end{macrocode}

% \macro{\childdocby}
% The command |\childdocby| ....
%    \begin{macrocode}
\newcommand{\childdocby}[2][]
{
  \childdocdisable
  \childdoctrue
  \childdocmanualtrue
  \if?#1?\else
    \def\jobname{#2}
  \fi
  \def\childdocjob{#2}
  \input{#2}
  \endinput
}
%    \end{macrocode}

% \macro{\childdocforward}
% The command |\childdocforward| redirects
% compilation to the main file or
% (if the optional argument is given) a child file.
% Parameters are set as if the main file
% or a child file starting with |\childdocof| was compiled.
% Then compilation is handed over to the main file:
%    \begin{macrocode}
\newcommand{\childdocforward}[2][]
{
  \begingroup
    \if?#1?
      \def\childdoctmp
      {
        \def\childdocname{#2}
        \def\childdocjob{#2}
        \def\jobname{#2}
        \input{#2}
        \endinput
      }
    \else
      \def\childdoctmp
      {
        \childdocdisable
        \def\childdocname{#2}
        \childdoctrue
        \includeonly{#2}
        \def\childdocjob{#1}
        \def\jobname{#1}
        \input{#1}
        \endinput
      }
    \fi
    \expandafter
  \endgroup
  \childdoctmp
}
%    \end{macrocode}

% \macro{\childdocforwardprefix}
% The command |\childdocforwardprefix| redirects
% compilation to the main or a child file by means of a pattern.
% The prefix |#1| in the current filename is replaced by |#2|
% and the suffix of the current filename is kept
% (it is assumed that the filename does not contain the substring `|~~~|'
% which is used as a delimiter).
% Compilation is handed over to the new file by |\childdocforward|:
%    \begin{macrocode}
\newcommand{\childdocforwardprefix}[3][]
{
  \begingroup
    \def\childdocextract #2##1~~~{\def\childdoctmp{\childdocforward[#1]{#3##1}}}
    \expandafter\childdocextract\childdocname~~~
    \expandafter
  \endgroup
  \childdoctmp
}
%    \end{macrocode}

% \macro{\childdoc}
% The deprecated macro |\childdoc| is a legacy version of |\childdocmain|:
%    \begin{macrocode}
\newcommand{\childdoc}{\childdocmain}
%    \end{macrocode}

% \macro{\childdocredirect}
% The deprecated macro |\childdocredirect| is a legacy version
% of |\childdocforward| and |\childdocforwardprefix|:
%    \begin{macrocode}
\newcommand{\childdocredirect}[2][]
{
  \begingroup
    \if?#1?
      \def\childdoctmp{\childdocforward{#2}}
    \else
      \def\childdoctmp{\childdocforwardprefix{#1}{#2}}
    \fi
    \expandafter
  \endgroup
  \childdoctmp
}
%    \end{macrocode}

%\iffalse
%</package>
%\fi
%
\endinput
|\\
|\childdocforward{|\textit{main}|}|\\
\end{tabular}
\end{center}
%
or alternatively with:
%
\begin{center}
\begin{tabular}{l}
|% \iffalse
%
% childdoc.dtx Copyright (C) 2017-2018 Niklas Beisert
%
% This work may be distributed and/or modified under the
% conditions of the LaTeX Project Public License, either version 1.3
% of this license or (at your option) any later version.
% The latest version of this license is in
%   http://www.latex-project.org/lppl.txt
% and version 1.3 or later is part of all distributions of LaTeX
% version 2005/12/01 or later.
%
% This work has the LPPL maintenance status `maintained'.
%
% The Current Maintainer of this work is Niklas Beisert.
%
% This work consists of the files childdoc.dtx and childdoc.ins
% and the derived files childdoc.def and cdocsamp.tex with
% cdocsch1.tex, cdocsch2.tex, cdocsdrf.tex, cdocsfn1.tex, cdocsfn2.tex.
%
%<package>\ifdefined\childdocmain\endinput\fi
%<package>\ProvidesFile{childdoc.def}[2018/12/30 v2.0 child document driver]
%<samplemain>\ProvidesFile{cdocsamp.tex}[2018/12/30 v2.0 sample for childdoc]
%<*driver>
%\ProvidesFile{childdoc.drv}[2018/12/30 v2.0 childdoc reference manual file]
\PassOptionsToClass{10pt,a4paper}{article}
\documentclass{ltxdoc}

\usepackage[margin=35mm]{geometry}
\usepackage{hyperref}
\usepackage{hyperxmp}
\usepackage[usenames]{color}

\hypersetup{colorlinks=true}
\hypersetup{pdfstartview=FitH}
\hypersetup{pdfpagemode=UseNone}
\hypersetup{pdfsource={}}
\hypersetup{pdflang={en-UK}}
\hypersetup{pdfcopyright={Copyright 2017-2018 Niklas Beisert.
  This work may be distributed and/or modified under the
  conditions of the LaTeX Project Public License, either version 1.3
  of this license or (at your option) any later version.}}
\hypersetup{pdflicenseurl={http://www.latex-project.org/lppl.txt}}
\hypersetup{pdfcontactaddress={ETH Zurich, ITP, HIT K,
  Wolfgang-Pauli-Strasse 27}}
\hypersetup{pdfcontactpostcode={8093}}
\hypersetup{pdfcontactcity={Zurich}}
\hypersetup{pdfcontactcountry={Switzerland}}
\hypersetup{pdfcontactemail={nbeisert@itp.phys.ethz.ch}}
\hypersetup{pdfcontacturl={http://people.phys.ethz.ch/\xmptilde nbeisert/}}

\newcommand{\secref}[1]{\hyperref[#1]{section \ref*{#1}}}

\parskip1ex
\parindent0pt
\let\olditemize\itemize
\def\itemize{\olditemize\parskip0pt}

\begin{document}

\title{The \textsf{childdoc} Package}
\hypersetup{pdftitle={The childdoc Package}}
\author{Niklas Beisert\\[2ex]
  Institut f\"ur Theoretische Physik\\
  Eidgen\"ossische Technische Hochschule Z\"urich\\
  Wolfgang-Pauli-Strasse 27, 8093 Z\"urich, Switzerland\\[1ex]
  \href{mailto:nbeisert@itp.phys.ethz.ch}
  {\texttt{nbeisert@itp.phys.ethz.ch}}}
\hypersetup{pdfauthor={Niklas Beisert}}
\hypersetup{pdfsubject={Manual for the LaTeX2e Package childdoc}}
\date{30 December 2018, \textsf{v2.0}}
\maketitle

\begin{abstract}\noindent
\textsf{childdoc} is a \LaTeXe{} package
that enables the direct compilation
of document sections included by |\include|
to individual files.
\end{abstract}

\begingroup
\parskip0ex
\tableofcontents
\endgroup

%%%%%%%%%%%%%%%%%%%%%%%%%%%%%%%%%%%%%%%%%%%%%%%%%%%%%%%%%%%%%%%%%%%%%%%%%%%%%%%%
%%%%%%%%%%%%%%%%%%%%%%%%%%%%%%%%%%%%%%%%%%%%%%%%%%%%%%%%%%%%%%%%%%%%%%%%%%%%%%%%
\section{Introduction}

\LaTeX{} provides a mechanism to structure a large document (such as a book)
into a main file and several child files (containing the chapters)
using the |\include| command.
This mechanism is beneficial for documents
which span hundreds of pages in order to
make the source file(s) more manageable.
Moreover, compilation can be restricted to
selected child files by means of the |\includeonly| command.
The latter feature can be used to reduce the compilation time while editing
(this was significantly more useful in the earlier days of \LaTeX{})
or to generate a smaller document which is easier to navigate.
Another application of |\includeonly| is to generate
documents consisting of selected parts of the complete document.

However, there are a few drawbacks of the plain |\include| mechanism:
\begin{itemize}
\item
The child files cannot be compiled on their own,
they can only be compiled via the main file.
A naive editing environment
(such as a text editor with an option
to have the current file processed by \LaTeX)
may require one to switch to the main file before compiling;
attempting to compile the child file produces errors.
\item
The main file must be modified (each time)
to adjust the |\includeonly| command
to the present needs. This easily leaves the main file in a messy state.
\item
The generated document will always carry the filename
of the main document. This is inconvenient if
several child files are to be compiled and
to be kept for distribution.
\end{itemize}

The present package provides a simple interface
to make child files individually compilable by \LaTeX{}.
Compiling a child file then has the same effect as compiling
the main file with an |\includeonly| command
to select the appropriate child.
Moreover the generated document will carry the name of the child
rather than the main file.
This resolves all three above issues.

This feature is meant to make the editing of books,
thesis documents and lecture notes somewhat more convenient.
However, the package can also be used efficiently for
composing a series of documents (such as exercise sheets)
which are typically distributed individually.
It then assists the author in generating the individual documents
(potentially in different versions)
as well as a document containing the collected series.
Another application is in developing style files
or other kinds of included material
where compilation of the style file could redirect
to a sample or test file.

%%%%%%%%%%%%%%%%%%%%%%%%%%%%%%%%%%%%%%%%%%%%%%%%%%%%%%%%%%%%%%%%%%%%%%%%%%%%%%%%
%%%%%%%%%%%%%%%%%%%%%%%%%%%%%%%%%%%%%%%%%%%%%%%%%%%%%%%%%%%%%%%%%%%%%%%%%%%%%%%%
\section{Usage}

First of all, the package \textsf{childdoc} is \emph{not} a standard
\LaTeXe{} |.sty| style file! Therefore it needs to be invoked in
a non-standard way.

%%%%%%%%%%%%%%%%%%%%%%%%%%%%%%%%%%%%%%%%%%%%%%%%%%%%%%%%%%%%%%%%%%%%%%%%%%%%%%%%
\subsection{Included Files}
\label{sec:include}

%%%%%%%%%%%%%%%%%%%%%%%%%%%%%%%%%%%%%%%%
\DescribeMacro{\childdocmain}
To use the package, add the commands
\begin{center}
\begin{tabular}{l}
|\input{childdoc.def}|\\
|\childdocmain{}|\\
\end{tabular}
\end{center}
at the very top of the main \LaTeX{} file,
in particular \emph{before} the |\documentclass| statement!
The argument of |\childdocmain| should be left empty
(but it must be present).

%%%%%%%%%%%%%%%%%%%%%%%%%%%%%%%%%%%%%%%%
\DescribeMacro{\childdocof}
Furthermore, add the commands
\begin{center}
\begin{tabular}{l}
|\input{childdoc.def}|\\
|\childdocof{|\textit{main}|}|\\
\end{tabular}
\end{center}
at the top of every child file \textit{child}
which is included by |\include{|\textit{child}|}|
from within the main file
(or at least for those files to be compiled individually).
The argument \textit{main} must be the filename of the main file.

There are a couple of
considerations in setting up the main and child documents:

%%%%%%%%%%%%%%%%%%%%%%%%%%%%%%%%%%%%%%%%
\paragraph{Restrictions.}

Please note the following restrictions:
\begin{itemize}
\item
|\childdocmain| must be called with one argument \textit{main}
to ensure compatibility with earlier version of the package.
It must either be empty (|\childdocmain{}|)
or precisely match the filename of the main file in which it is specified.
See \secref{sec:detection} for further information.
\item
The filename \textit{main} must be specified without the |.tex| extension.
\item
The filename \textit{main} is case sensitive
(even in case-insensitive file systems)
due to internal string comparison.
\item
The argument \textit{main} should be fully expanded, it cannot be a macro.
\item
Subdirectories and special characters should be avoided in filenames.
\item
The command |\childdocmain{|\textit{main}|}| must be followed by a whitespace.
It should not be followed immediately by another command
or by a comment mark `|%|'.
This is because the \TeX{} parser reads the token immediately following
the argument of |\childdocmain| and puts it
at the beginning of every child section;
however, a white\-space is ignored.
\end{itemize}

%%%%%%%%%%%%%%%%%%%%%%%%%%%%%%%%%%%%%%%%
\paragraph{Content of Main File.}

It is advisable to place all content in the child files included by |\include|.
Any output contained in the main file will appear in all child documents
unless suppressed manually;
it cannot be suppressed automatically by the |\includeonly| directive
and thus should normally be avoided.
A method to include some content in the main file
by means of conditional processing is described in \secref{sec:conditional}.

%%%%%%%%%%%%%%%%%%%%%%%%%%%%%%%%%%%%%%%%
\paragraph{Page Numbering.}

When only a part of the document is compiled,
the appropriate numbering of pages
(as well as other status parameters)
is determined from the |.aux| files.
The latter contain information from previous passes.
However this information needs to propagate through
all intermediate child documents.
Therefore the page numbering in child documents may well
be inconsistent until the complete document is compiled at least once.

A useful (if unconventional) way to always ensure a consistent
page numbering is to restart the numbering in each child document
and denote the pages by `\textit{child}|.|\textit{page}'
where \textit{child} represents the chapter/section number of the child file.
This can be achieved by the command
|\numberwithin{page}{|\textit{child}|}|
of the \textsf{amsmath} package
where \textit{child} can be |chapter| or |section|
depending on the chosen structuring.
Alternatively, one can modify the macro |\thepage| appropriately
and reset the counter |page| at the start of each child file.

%%%%%%%%%%%%%%%%%%%%%%%%%%%%%%%%%%%%%%%%%%%%%%%%%%%%%%%%%%%%%%%%%%%%%%%%%%%%%%%%
\subsection{Conditional Processing}
\label{sec:conditional}

The package provides a mechanism to compile different versions
of a document. To customise the versions further some conditional processing
can come in handy to distinguish which version is being compiled.
The package provides two macros to describe the compilation context:

%%%%%%%%%%%%%%%%%%%%%%%%%%%%%%%%%%%%%%%%
\DescribeMacro{\ifchilddoc}
The conditional |\ifchilddoc| distinguishes between the compilation of
child documents and the main document:
%
\begin{center}
|\ifchilddoc |\textit{child-code}| |[|\||else |\textit{main-code}]| \||fi|
\end{center}

%%%%%%%%%%%%%%%%%%%%%%%%%%%%%%%%%%%%%%%%
\DescribeMacro{\childdocname}
\DescribeMacro{\childdocjob}
The macro |\childdocname| contains the filename (without extension)
of the main or child file being processed.
Note that |\childdocjob| will always contain the name of the main file.

%%%%%%%%%%%%%%%%%%%%%%%%%%%%%%%%%%%%%%%%
\paragraph{Title Page.}

Conditional processing can be used to include a title or banner page
in the main document when proper precautions are taken.
Importantly, the code in the main file should ensure that the page counter
(as well as other status parameters which are stored in the |.aux| files)
takes the same value after the conditional processing.
Otherwise the page numbers may take divergent values
depending on which part is compiled.

For example, a title page could be declared by:
%
\begin{center}
\begin{tabular}{l}
|\ifchilddoc\||else|\\
|\addtocounter{page}{-1}|\\
\textit{code for title page}\\
|\newpage|\\
|\||fi|
\end{tabular}
\end{center}
%
A banner page for the child documents can be generated by:
%
\begin{center}
\begin{tabular}{l}
|\ifchilddoc|\\
|\addtocounter{page}{-1}|\\
\textit{code for banner page}\\
|\newpage|\\
|\||fi|
\end{tabular}
\end{center}
%
Here one could write a message such as:
\begin{center}
|This is the part \childdocname{} of \childdocjob{}.|
\end{center}

%%%%%%%%%%%%%%%%%%%%%%%%%%%%%%%%%%%%%%%%%%%%%%%%%%%%%%%%%%%%%%%%%%%%%%%%%%%%%%%%
\subsection{Flags}
\label{sec:flags}

The package makes it easy to generate different versions
of the main or child documents.
To this end compilation flags can be defined
and assigned different default values.
They will be particularly useful in conjunction
with the forwarding mechanism described in \secref{sec:forward}.

For example, it may be useful to have a flag |\version|
which can be set to |draft| or |final|.
The document source will contain some conditional code
depending on the value of |\version|.
Suppose further, the flag should default to |final| for the main file
and to |draft| for child files
which is a natural assignment for editing the document.
This is achieved by placing the following code
in the preamble of the main document
(below the |\childdocmain| directive):
%
\begin{center}
\begin{tabular}{l}
|\ifchilddoc|\\
|\providecommand{\version}{draft}|\\
|\||else|\\
|\providecommand{\version}{final}|\\
|\||fi|
\end{tabular}
\end{center}
%
The definition by |\providecommand| makes sure
that previous definitions are not overwritten.
Further statements |\providecommand{\version}{...}|
can thus be added before the above code to override it.

For the main file, one might add a line
(between |\childdocmain| and the above block)
%
\begin{center}
|%\ifchilddoc\||else\providecommand{\version}{draft}\||fi|
\end{center}
%
which can be uncommented to produce a draft version.
Likewise one can add a line to the very top of a child file
(above the |\childdocof{|\textit{main}|}| directive)
%
\begin{center}
|%\providecommand{\version}{final}|
\end{center}
%
which can be uncommented to produce the final version of this child document.

%%%%%%%%%%%%%%%%%%%%%%%%%%%%%%%%%%%%%%%%%%%%%%%%%%%%%%%%%%%%%%%%%%%%%%%%%%%%%%%%
\subsection{Forwarding}
\label{sec:forward}

Different versions of the main or child documents
using compilation flags as described in \secref{sec:flags}
can be (permanently) stored in different files
for convenient compilation, viewing and distribution.
To this end, the package defines a command
to pass on compilation to a different file:

%%%%%%%%%%%%%%%%%%%%%%%%%%%%%%%%%%%%%%%%
\DescribeMacro{\childdocforward}
The command |\childdocforward| redirects processing to
another source file:
%
\begin{center}
\begin{tabular}{l}
|\input{childdoc.def}|\\
|\childdocforward[|\textit{main}|]{|\textit{dest}|}|\\
\end{tabular}
\end{center}
%
The argument \textit{dest} is the destination file
(without extension).
It should be the main file or one of the child files.
Note that further \textsf{childdoc} directives
such as |\childdocof| and |\childdocforward|
in the indicated file will be processed in this form.
The optional argument \textit{main}
passes on directly to the main file \textit{main}
while pretending to compile the child \textit{dest}.
This form behaves as if \textit{dest}
issues |\childdocof{|\textit{main}|}| right away,
and no further \textsf{childdoc} directives will be processed.

%%%%%%%%%%%%%%%%%%%%%%%%%%%%%%%%%%%%%%%%
\DescribeMacro{\...prefix}
In the alternative form |\childdocforwardprefix|,
%
\begin{center}
\begin{tabular}{l}
|\input{childdoc.def}|\\
|\childdocforwardprefix[|\textit{main}|]{|\textit{prefix}|}{|\textit{dest}|}|
\end{tabular}
\end{center}
%
the destination file is determined by a pattern
depending on the current file:
To make this work, the current file must be called
`{\textit{prefix}\hspace{0.2em}\textit{suffix}}'
with \textit{prefix} matching precisely the argument.
Processing is then passed on to the file
`{\textit{dest}\hspace{0.2em}\textit{suffix}}'.
Surely, the same effect is achieved by
directly specifying the
argument `{\textit{dest}\hspace{0.2em}\textit{suffix}}'
in the first form.
However, that requires to set up a different file
for each child. With the alternative form of the command
all these files can have exactly the same content
which simplifies setting them up and maintaining them.

For example, the following file |draft.tex|
with a compilation flag |\version| as described in \secref{sec:flags}
compiles the main document as a draft:
%
\begin{center}
\begin{tabular}{l}
|\def\version{draft}|\\
|\input{childdoc.def}|\\
|\childdocforward{|\textit{main}|}|
\end{tabular}
\end{center}
%
Likewise, the following files |final|\textit{nn}|.tex|
compile the final version of the child document
|child|\textit{nn}|.tex|:
%
\begin{center}
\begin{tabular}{l}
|\def\version{final}|\\
|\input{childdoc.def}|\\
|\childdocforwardprefix{final}{child}|
\end{tabular}
\end{center}
%

Note that when several versions of a main file and/or of each child file
are to be generated, it may be convenient to set up a |Makefile| or
shell script to automatise the process.

%%%%%%%%%%%%%%%%%%%%%%%%%%%%%%%%%%%%%%%%%%%%%%%%%%%%%%%%%%%%%%%%%%%%%%%%%%%%%%%%
\subsection{Command Line Processing}
\label{sec:commandline}

The effect of redirection files can also be achieved by invoking
the \LaTeX{} compiler with a more elaborate command line.
Most conveniently this should be done as part
of a shell script or a |Makefile|.

When using \textsf{childdoc} in the main file, the following
command lines effectively perform a redirection
(note that depending on the shell being used,
backslashes may have to be doubled: `|\|' $\to$ `|\\|'):
%
\begin{center}
|... -jobname "|\textit{target}|" |\\|"|[\textit{flags}]%
|\input{childdoc.def}\childdocforward[|\textit{main}|]{|\textit{dest}|}"|
\end{center}
%
Here \textit{target} is the name of the output file,
\textit{main} is the name of the main file
and \textit{dest} is the name of the main or child file to be processed
(all filenames without extensions).
The optional argument \textit{main} can be omitted
if \textit{main} matches \textit{dest}.
Optionally, compilation \textit{flags} can be defined via |\def| commands.
This command line makes the \TeX{} engine believe
it is compiling the file \textit{target}
whose content is specified as the latter parameter.
The provided code then forwards the processing to
\textit{main} or \textit{dest} as described in \secref{sec:forward}.

%%%%%%%%%%%%%%%%%%%%%%%%%%%%%%%%%%%%%%%%%%%%%%%%%%%%%%%%%%%%%%%%%%%%%%%%%%%%%%%%
\subsection{Include by Input}
\label{sec:input}

Including child documents by |\include| has some restrictions by design.
Most notably, the content of a child document always occupies
its own set of pages; pages cannot be shared between child documents.
Usually, this behaviour makes perfect sense
because each child document contain an essential part of the document.
However, in some situations it may be desirable to compose
a document from a collection of parts
without having mandatory page breaks between then.
For this case, the package
provides a mechanism to include parts
by |\input| which can also be processed individually.
However, by construction this mechanism
requires manual handling of the content to be output.

%%%%%%%%%%%%%%%%%%%%%%%%%%%%%%%%%%%%%%%%
\DescribeMacro{\ifchilddocmanual}
The main file should be prepared as usual, see \secref{sec:include}.
However, the document body must make a distinction
between processing of an individual part and of the main document, e.g.:
%
\begin{center}
\begin{tabular}{l}
|\ifchilddocmanual|\\
|\input{\childdocname}|\\
|\||else|\\
\textit{document body with }|\input{|\textit{part}|}|\\
|\||fi|
\end{tabular}
\end{center}
%
The conditional |\ifchilddocmanual| is true whenever
a part to be included by |\input| is being compiled,
and the name of the part is stored in |\childdocname|.

%%%%%%%%%%%%%%%%%%%%%%%%%%%%%%%%%%%%%%%%
\DescribeMacro{\childdocby}
Each part to be included by |\input| should start with:
%
\begin{center}
\begin{tabular}{l}
|\input{childdoc.def}|\\
|\childdocby{|\textit{main}|}|\\
\end{tabular}
\end{center}
%
The directive |\childdocby| is similar to |\childdocof|
described in \secref{sec:include},
but the subsequent selection of content must be done manually.
To that end, both |\ifchilddoc| and |\ifchilddocmanual|
will be true upon processing of a part,
and the name of the part is stored in |\childdocname|.
Note that |\jobname| will be set to the filename of the current part
so that each part receives an individual |.aux| file
that does not interfere with the |.aux| file(s) of the main document.
This behaviour can be altered by the alternative form
|\childdocby[*]{|\textit{main}|}| (with a non-empty optional argument)
which uses the |.aux| file of the main document
by setting |\jobname| to \textit{main}.

%%%%%%%%%%%%%%%%%%%%%%%%%%%%%%%%%%%%%%%%%%%%%%%%%%%%%%%%%%%%%%%%%%%%%%%%%%%%%%%%
\subsection{Driver Development}
\label{sec:driver}

The \textsf{childdoc} mechanism can also be use for the development
of definition files such as \LaTeX{} styles or classes.
This case differs from the above setup with multiple parts
included by |\include| in that no |\includeonly| should be invoked.
This can be achieved by starting the include file
(before |\ProvidesPackage|) with:
%
\begin{center}
\begin{tabular}{l}
|\input{childdoc.def}|\\
|\childdocforward{|\textit{main}|}|\\
\end{tabular}
\end{center}
%
or alternatively with:
%
\begin{center}
\begin{tabular}{l}
|\input{childdoc.def}|\\
|\childdocby{|\textit{main}|}|\\
\end{tabular}
\end{center}
%
Both forms have slightly different effects as described above.
The main file is prepared as usual, see \secref{sec:include}.

%%%%%%%%%%%%%%%%%%%%%%%%%%%%%%%%%%%%%%%%%%%%%%%%%%%%%%%%%%%%%%%%%%%%%%%%%%%%%%%%
\subsection{Legacy Detection}
\label{sec:detection}

The directive |\childdocmain| in the main file can detect
whether the complete document or merely a child is to be compiled
even without using the directive |\childdocof|.
This method is deprecated because it is less robust
and there is no compelling reason to use it;
it is merely provided for backward compatibility
and it may be removed in future versions.

If the detection mechanism is to be used,
it is mandatory to correctly specify
the filename of the main file as the argument of |\childdocmain|:
%
\begin{center}
\begin{tabular}{l}
|\input{childdoc.def}|\\
|\childdocmain{|\textit{main}|}|\\
\end{tabular}
\end{center}
%
If |\jobname| does not match the argument \textit{main} of |\childdocmain|,
it is assumed that |\jobname| points to the child file to be compiled.
When using |\childdocmain| with the main file specified as argument,
it suffices to start a child file
with just |\input{|\textit{main}|}|
without loading of the package and using |\childdocof|.
If instead all processing is done
with the appropriate \textsf{childdoc} directives,
the argument of \textit{main} of |\childdocmain| can be empty.

An alternative version of the command line processing described
in \secref{sec:commandline} using the detection mechanism reads:
%
\begin{center}
|... -jobname "|\textit{target}|" "|[\textit{flags}]%
[|\def\jobname{|\textit{dest}|}|]|\input{|\textit{main}|}"|
\end{center}

%%%%%%%%%%%%%%%%%%%%%%%%%%%%%%%%%%%%%%%%%%%%%%%%%%%%%%%%%%%%%%%%%%%%%%%%%%%%%%%%
\subsection{Manual Code}
\label{sec:manual}

In case one cannot be certain whether the definitions file |childdoc.def|
is installed on the target \TeX{} distribution
and one prefers not to ship it,
it is conceivable to paste a few relevant commands into the sources.

To that end, drop all statements |\input{childdoc.def}|
and perform the replacements as outlined below.
Instead of |\childdocmain{|\textit{main}|}| add the following code
to the top of the main file:
%
\begin{center}
\begin{tabular}{l}
|\||ifdefined\childdocname\endinput\||fi\newif\ifchilddoc|\\
|\edef\childdocname{\scantokens\expandafter{\jobname\noexpand}}|\\
|\def\childdocmain{|\textit{main}|}\||ifx\childdocmain\childdocname\||else|\\
|\childdoctrue\includeonly{\childdocname}\let\jobname\childdocmain\||fi|\\
\end{tabular}
\end{center}
%
Instead of |\childdocof{|\textit{main}|}| just include the main file
at the top of each child file:
%
\begin{center}
|\input{|\textit{main}|}|
\end{center}
%
A simple redirection |\childdocforward{|\textit{dest}|}| is achieved by:
%
\begin{center}
|\def\jobname{|\textit{dest}|}\input{\jobname}|
\end{center}
%
The redirection with prefix
|\childdocforwardprefix[|\textit{prefix}|]{|\textit{dest}|}|
is accomplished by:
%
\begin{center}
\begin{tabular}{l}
|{\edef\jobname{\scantokens\expandafter{\jobname\noexpand}}|\\
|\def\redirectjob |\textit{prefix}|#1~~~{\gdef\jobname{|\textit{dest}|#1}}|\\
|\expandafter\redirectjob\jobname~~~}\input{\jobname}|
\end{tabular}
\end{center}

In an alternative approach,
child documents can be compiled by a specific command line
without additional code or specific definitions:
%
\begin{center}
|... -jobname "|\textit{target}|" "|[\textit{flags}]%
|\includeonly{|\textit{dest}|}\input{|\textit{main}|}"|
\end{center}
%

%%%%%%%%%%%%%%%%%%%%%%%%%%%%%%%%%%%%%%%%%%%%%%%%%%%%%%%%%%%%%%%%%%%%%%%%%%%%%%%%
%%%%%%%%%%%%%%%%%%%%%%%%%%%%%%%%%%%%%%%%%%%%%%%%%%%%%%%%%%%%%%%%%%%%%%%%%%%%%%%%
\section{Information}

%%%%%%%%%%%%%%%%%%%%%%%%%%%%%%%%%%%%%%%%%%%%%%%%%%%%%%%%%%%%%%%%%%%%%%%%%%%%%%%%
\subsection{Copyright}

Copyright \copyright{} 2017--2018 Niklas Beisert

This work may be distributed and/or modified under the
conditions of the \LaTeX{} Project Public License, either version 1.3
of this license or (at your option) any later version.
The latest version of this license is in
  \url{http://www.latex-project.org/lppl.txt}
and version 1.3 or later is part of all distributions of \LaTeX{}
version 2005/12/01 or later.

This work has the LPPL maintenance status `maintained'.

The Current Maintainer of this work is Niklas Beisert.

This work consists of the files |README.txt|, |childdoc.ins| and |childdoc.dtx|
as well as the derived files |childdoc.def|, |cdocsamp.tex|
with |cdocsch1.tex|, |cdocsch2.tex|, |cdocspt3.tex|, |cdocspt4.tex|,
|cdocsdrf.tex|, |cdocsfn1.tex|, |cdocsfn2.tex|
as well as |childdoc.pdf|.

%%%%%%%%%%%%%%%%%%%%%%%%%%%%%%%%%%%%%%%%%%%%%%%%%%%%%%%%%%%%%%%%%%%%%%%%%%%%%%%%
\subsection{Files and Installation}

The package consists of the files:
%
\begin{center}
\begin{tabular}{ll}
    |README.txt|   & readme file \\
    |childdoc.ins| & installation file \\
    |childdoc.dtx| & source file \\
    |childdoc.def| & definition file \\
    |cdocsamp.tex| & sample main file \\
    |cdocsch1.tex| & sample include file \\
    |cdocsch2.tex| & sample include file \\
    |cdocspt3.tex| & sample part file \\
    |cdocspt4.tex| & sample part file \\
    |cdocsdrf.tex| & sample redirection file \\
    |cdocsfn1.tex| & sample redirection file \\
    |cdocsfn2.tex| & sample redirection file \\
    |childdoc.pdf| & manual
\end{tabular}
\end{center}
%
The distribution consists of the files
|README.txt|, |childdoc.ins| and |childdoc.dtx|.
%
\begin{itemize}
\item
Run (pdf)\LaTeX{} on |childdoc.dtx|
to compile the manual |childdoc.pdf| (this file).
\item
Run \LaTeX{} on |childdoc.ins| to create the definitions file |childdoc.def|
and the sample |cdocsamp.tex| with include files
|cdocsch1.tex|, |cdocsch2.tex|, |cdocspt3.tex|, |cdocspt4.tex|,
|cdocsdrf.tex|, |cdocsfn1.tex|, |cdocsfn2.tex|.
Then copy the file |childdoc.def| to an appropriate directory of your \LaTeX{}
distribution, e.g.\ \textit{texmf-root}|/tex/latex/childdoc|.
\end{itemize}

%%%%%%%%%%%%%%%%%%%%%%%%%%%%%%%%%%%%%%%%%%%%%%%%%%%%%%%%%%%%%%%%%%%%%%%%%%%%%%%%
\subsection{Related CTAN Packages}

There are several other packages which offer a similar functionality:
%
\begin{itemize}
\item
The packages
\href{http://ctan.org/pkg/docmute}{\textsf{docmute}},
\href{http://ctan.org/pkg/includex}{\textsf{includex}} and
\href{http://ctan.org/pkg/standalone}{\textsf{standalone}}
provide commands to include only the document body of
a child file thus allowing both files to be compiled individually.
\item
The packages \href{http://ctan.org/pkg/subdocs}{\textsf{subdocs}}
and \href{http://ctan.org/pkg/subfiles}{\textsf{subfiles}}
provide structures in which the main and child documents can be
encapsulated and allowing them to be compiled individually.
The inclusion mechanism is different from the conventional |\include|.
\item
The package \href{http://ctan.org/pkg/combine}{\textsf{combine}}
is an elaborate solution to combine several documents into one.
\end{itemize}
%
See also the CTAN topic \href{http://ctan.org/topic/subdocs}{\textsf{subdocs}}
for further related packages.
The present package differs from the above solutions in that
a document structure constructed with the conventional |\include| mechanism
just needs two extra commands at the top of every file
such that all constituent files can be compiled individually.

%%%%%%%%%%%%%%%%%%%%%%%%%%%%%%%%%%%%%%%%%%%%%%%%%%%%%%%%%%%%%%%%%%%%%%%%%%%%%%%%
%\subsection{Feature Suggestions}
%
%The following is a list of features which may be useful for future
%versions of this package:
%%
%\begin{itemize}
%\item
%\ldots
%\end{itemize}

%%%%%%%%%%%%%%%%%%%%%%%%%%%%%%%%%%%%%%%%%%%%%%%%%%%%%%%%%%%%%%%%%%%%%%%%%%%%%%%%
\subsection{Revision History}

%%%%%%%%%%%%%%%%%%%%%%%%%%%%%%%%%%%%%%%%
\paragraph{v2.0:} 2018/12/30

\begin{itemize}
\item
immediate forward processing
\item
added |\childdocby| mechanism
\item
manual restructured
\end{itemize}

%%%%%%%%%%%%%%%%%%%%%%%%%%%%%%%%%%%%%%%%
\paragraph{v1.6:} 2018/01/17

\begin{itemize}
\item
application for development of include files
\item
corrections to manual
\end{itemize}

%%%%%%%%%%%%%%%%%%%%%%%%%%%%%%%%%%%%%%%%
\paragraph{v1.5:} 2017/05/21

\begin{itemize}
\item
more complete structuring introduced
\item
|\childdocof| introduced
\item
|\childdoc| renamed to |\childdocmain|
\item
|\childredirect| renamed to |\childdocforward| and |\childdocforwardprefix|
and functionality expanded
\end{itemize}

%%%%%%%%%%%%%%%%%%%%%%%%%%%%%%%%%%%%%%%%
\paragraph{v1.0:} 2017/04/27

\begin{itemize}
\item
manual and install package
\item
first version published on CTAN
\end{itemize}

%%%%%%%%%%%%%%%%%%%%%%%%%%%%%%%%%%%%%%%%
\paragraph{v0.6:} 2017/04/26

\begin{itemize}
\item
redirection mechanism added
\end{itemize}

%%%%%%%%%%%%%%%%%%%%%%%%%%%%%%%%%%%%%%%%
\paragraph{v0.5:} 2017/04/26

\begin{itemize}
\item
functionality in definition file
\end{itemize}


%%%%%%%%%%%%%%%%%%%%%%%%%%%%%%%%%%%%%%%%%%%%%%%%%%%%%%%%%%%%%%%%%%%%%%%%%%%%%%%%
%%%%%%%%%%%%%%%%%%%%%%%%%%%%%%%%%%%%%%%%%%%%%%%%%%%%%%%%%%%%%%%%%%%%%%%%%%%%%%%%
%%%%%%%%%%%%%%%%%%%%%%%%%%%%%%%%%%%%%%%%%%%%%%%%%%%%%%%%%%%%%%%%%%%%%%%%%%%%%%%%
\appendix

\settowidth\MacroIndent{\rmfamily\scriptsize 000\ }

 \DocInput{childdoc.dtx}

\end{document}
%</driver>
% \fi
%
% %%%%%%%%%%%%%%%%%%%%%%%%%%%%%%%%%%%%%%%%%%%%%%%%%%%%%%%%%%%%%%%%%%%%%%%%%%%%%%
% %%%%%%%%%%%%%%%%%%%%%%%%%%%%%%%%%%%%%%%%%%%%%%%%%%%%%%%%%%%%%%%%%%%%%%%%%%%%%%
% \section{Sample}
%\iffalse
%<*samplemain>
%\fi
%
% The following presents a sample document
% with two chapters, two parts, a title page,
% a compile flag as well as three forwarding files to set the flag.
% It consists of eight |.tex| files:
% \begin{center}
% \begin{tabular}{ll}
% |cdocsamp.tex|&main file\\
% |cdocsch1.tex|&include file for chapter 1\\
% |cdocsch2.tex|&include file for chapter 2\\
% |cdocspt3.tex|&include file for part 3\\
% |cdocspt4.tex|&include file for part 4\\
% |cdocsdrf.tex|&forwarding file for main file in draft mode\\
% |cdocsfi1.tex|&forwarding file for final version of chapter 1\\
% |cdocsfi2.tex|&forwarding file for final version of chapter 2\\
% \end{tabular}
% \end{center}
% Each of the eight files can be compiled directly by the \LaTeX{} compiler.
%
% %%%%%%%%%%%%%%%%%%%%%%%%%%%%%%%%%%%%%%
% \paragraph{Main File.}
%
% The main file is called |cdocsamp.tex|.
%
% Load the \textsf{childdoc} definitions and
% declare the filename for the main document:
%    \begin{macrocode}
\input{childdoc.def}
\childdocmain{}
%    \end{macrocode}

% Optional override for |\version| flag:
%    \begin{macrocode}
%%\ifchilddoc\else\providecommand{\version}{draft}\fi
%    \end{macrocode}

% Define the default values for the |\version| flag
% (|final| for the main file and |draft| for childs):
%    \begin{macrocode}
\ifchilddoc
\providecommand{\version}{draft}
\else
\providecommand{\version}{final}
\fi
%    \end{macrocode}

% Load the standard document class:
%    \begin{macrocode}
\documentclass[12pt]{article}
%    \end{macrocode}

% Start the document body:
%    \begin{macrocode}
\begin{document}
%    \end{macrocode}

% Declare a title page.
% Print title, part of document being processed and version flag:
%    \begin{macrocode}
\addtocounter{page}{-1}
\begin{center}
{\LARGE\bfseries{}childdoc example\par}
\vspace{1cm}
\ifchilddoc
\ifchilddocmanual part\else chapter\fi:
`\childdocname' of `\childdocjob'\par
\else
main document: `\childdocjob'\par
\fi
version: \version\par
\end{center}
\newpage
%    \end{macrocode}

% Manually include selected file,
% otherwise process as usual:
%    \begin{macrocode}
\ifchilddocmanual
\section*{part `\childdocname'}
\input{\childdocname}
\else
%    \end{macrocode}

% Include the two chapters:
%    \begin{macrocode}
\include{cdocsch1}
\include{cdocsch2}
%    \end{macrocode}

% Include the two parts unless only chapters should be displayed:
%    \begin{macrocode}
\ifchilddoc\else
\section{part three}
\input{cdocspt3}
\section{part four}
\input{cdocspt4}
\fi
%    \end{macrocode}

% Process as usual until here:
%    \begin{macrocode}
\fi
%    \end{macrocode}

% End of document body:
%    \begin{macrocode}
\end{document}
%    \end{macrocode}
%\iffalse
%</samplemain>
%\fi
%
% %%%%%%%%%%%%%%%%%%%%%%%%%%%%%%%%%%%%%%
% \paragraph{Chapter Include Files.}
%
% The include files are called |cdocsch1.tex| and |cdocsch2.tex|.
%
%\iffalse
%<*samplechap1|samplechap2>
%\fi

% Optional override for |\version| flag:
%    \begin{macrocode}
%%\providecommand{\version}{final}
%    \end{macrocode}

% Include the main document:
%    \begin{macrocode}
\input{childdoc.def}
\childdocof{cdocsamp}
%    \end{macrocode}

%\iffalse
%</samplechap1|samplechap2>
%\fi
%
%\iffalse
%<*samplechap1>
%\fi
% Some text for chapter 1:
%    \begin{macrocode}
\section{one}
some text in chapter one
%    \end{macrocode}

%\iffalse
%</samplechap1>
%\fi
% Some text for chapter 2:
%\iffalse
%<*samplechap2>
%\fi
%    \begin{macrocode}
\section{two}
more text in chapter two
%    \end{macrocode}

%\iffalse
%</samplechap2>
%\fi
%
% %%%%%%%%%%%%%%%%%%%%%%%%%%%%%%%%%%%%%%
% \paragraph{Part Include Files.}
%
% The include files are called |cdocspt3.tex| and |cdocspt4.tex|.
%
%\iffalse
%<*samplepart3|samplepart4>
%\fi

% Optional override for |\version| flag:
%    \begin{macrocode}
%%\providecommand{\version}{final}
%    \end{macrocode}

% Include the main document:
%    \begin{macrocode}
\input{childdoc.def}
\childdocby{cdocsamp}
%    \end{macrocode}

%\iffalse
%</samplepart3|samplepart4>
%\fi
%
%\iffalse
%<*samplepart3>
%\fi
% Some text for part 3:
%    \begin{macrocode}
some text in part three
%    \end{macrocode}

%\iffalse
%</samplepart3>
%\fi
% Some text for part 4:
%\iffalse
%<*samplepart4>
%\fi
%    \begin{macrocode}
more text in part four
%    \end{macrocode}

%\iffalse
%</samplepart4>
%\fi
%
% %%%%%%%%%%%%%%%%%%%%%%%%%%%%%%%%%%%%%%
% \paragraph{Forwarding for a Complete Draft.}
%
% The following forwarding file |cdocsdrf.tex|
% compiles the main document in draft mode:
%\iffalse
%<*sampledraft>
%\fi
%    \begin{macrocode}
\def\version{draft}
\input{childdoc.def}
\childdocforward{cdocsamp}
%    \end{macrocode}

%\iffalse
%</sampledraft>
%\fi
%
% %%%%%%%%%%%%%%%%%%%%%%%%%%%%%%%%%%%%%%
% \paragraph{Forwarding for Final Version of the Chapters.}
%
% The following forwarding files |cdocsfn1.tex| and |cdocsfn2.tex|
% (with identical content)
% compile the final versions of the child documents
% |cdocsch1.tex| and |cdocsch2.tex|, respectively:
%\iffalse
%<*samplefinal>
%\fi
%    \begin{macrocode}
\def\version{final}
\input{childdoc.def}
\childdocforwardprefix[cdocsamp]{cdocsfn}{cdocsch}
%    \end{macrocode}

%\iffalse
%</samplefinal>
%\fi
%
% %%%%%%%%%%%%%%%%%%%%%%%%%%%%%%%%%%%%%%
% \paragraph{Command Line Processing.}
%
% The following three command lines generate the output files
% |cdocscld|, |cdocscl1| and |cdocscl2|
% which should be identical to
% |cdocsdrf|, |cdocsch1| and |cdocsfn2|, respectively:
% \begin{center}
% \begin{tabular}{l}
% |latex -jobname cdocscld \|\\
% |  "\def\version{draft}\input{childdoc.def}\childdocforward{cdocsamp}"|\\
% |latex -jobname cdocscl1 \|\\
% |  "\input{childdoc.def}\childdocforward[cdocsamp]{cdocsch1}"|\\
% |latex -jobname cdocscl2 \|\\
% |  "\def\version{final}\input{childdoc.def}\childdocforward{cdocsch2}"|
% \end{tabular}
% \end{center}
% Note that the trailing backslash on each first line
% merely continues the input to the second line
% (for convenient cut ant paste).
% Furthermore, the command |latex| can be replaced by any
% of its alternative versions such as |pdflatex|.
%
% %%%%%%%%%%%%%%%%%%%%%%%%%%%%%%%%%%%%%%%%%%%%%%%%%%%%%%%%%%%%%%%%%%%%%%%%%%%%%%
% %%%%%%%%%%%%%%%%%%%%%%%%%%%%%%%%%%%%%%%%%%%%%%%%%%%%%%%%%%%%%%%%%%%%%%%%%%%%%%
% \section{Implementation}
%\iffalse
%<*package>
%\fi
%
% This section describes the definitions file |childdoc.def|.

% The definitions cannot be loaded using |\usepackage| or |\RequirePackage|
% which has a mechanism to prevent loading a style file more than once.
% When loading the definitions by means of |\input|
% multiple instances have to be prevented manually:
%\iffalse
%This code needs to be before the `\ProvidesFile' directive
%which is defined at the beginning of this file.
%Therefore it is also placed there and commented out here.
%</package>
%<*discard>
%\fi
%    \begin{macrocode}
\ifdefined\childdocmain\endinput\fi
%    \end{macrocode}
%\iffalse
%</discard>
%<*package>
%\fi
%
% \macro{\ifchilddoc}
% \macro{\ifchilddocmanual}
% The conditional |\ifchilddoc| tells whether a
% child (true) or main (false) document is being compiled.
% The conditional |\ifchilddocmanual| tells whether
% the |\includeonly| mechanism is used (false) or
% the selection of child files must be performed manually (true).
% The definitions initialise to false:
%    \begin{macrocode}
\newif\ifchilddoc
\newif\ifchilddocmanual
%    \end{macrocode}

% \macro{\childdocname}
% \macro{\childdocjob}
% The macro |\childdocname| stores the name of the main document
% to be compiled. The macro |\childdocjob| stores the name of
% the document on which the \LaTeX{} compiler was originally invoked.
% The content of |\jobname| cannot be compared
% to filenames specified in the source due to different catcodes.
% The following code rescans |\jobname|, stores the result
% in |\childdocname| and saves a copy in |\childdocjob|:
%    \begin{macrocode}
\edef\childdocname{\scantokens\expandafter{\jobname\noexpand}}
\let\childdocjob\childdocname
%    \end{macrocode}

% \macro{\childdocdisable}
% The macro |\childdocdisable| prevents the main file
% from being processed more than once.
% At this stage, the main document command |\childdocmain|
% is assumed to be called once again where it should do nothing.
% Any subsequent call to it should prevent
% a secondary processing of the main document
% It overwrites the forwarding commands
% |\childdocof| and |\childdocforward|
% with empty macros to prevent further inclusions of the main document:
%    \begin{macrocode}
\newcommand{\childdocdisable}
{
  \renewcommand{\childdocmain}[1]{\renewcommand{\childdocmain}[1]{\endinput}}
  \renewcommand{\childdocof}[1]{}
  \renewcommand{\childdocby}[2][]{}
  \renewcommand{\childdocforward}[2][]{}
  \renewcommand{\childdocdisable}{}
}
%    \end{macrocode}

% \macro{\childdocmain}
% The macro |\childdocmain| is to be called at the top of the main file
% with nothing or the main filename (without extension) as argument.
% First, it breaks loops.
% If the argument is not empty and does not match |\childdocname|
% (which is set by the first inclusion of |childdoc.def|),
% |\ifchilddoc| is set to true, |\includeonly| is applied to the child file
% and |\jobname| is set to the main file
% (for proper handling of |.aux| files):
%    \begin{macrocode}
\newcommand{\childdocmain}[1]
{
  \childdocdisable\childdocmain{}
  \if?#1?\else
    \begingroup
      \def\childdoctmp{#1}
      \ifx\childdoctmp\childdocname
        \def\childdoctmp{}
      \else
        \def\childdoctmp
        {
          \childdoctrue
          \includeonly{\childdocname}
          \def\childdocjob{#1}
          \def\jobname{#1}
        }
      \fi
      \expandafter
    \endgroup
    \childdoctmp
  \fi
}
%    \end{macrocode}

% \macro{\childdocof}
% The command |\childdocof| redirects
% compilation to the main file |#1|.
%    \begin{macrocode}
\newcommand{\childdocof}[1]
{
  \childdocdisable
  \childdoctrue
  \includeonly{\childdocname}
  \def\jobname{#1}
  \def\childdocjob{#1}
  \input{#1}
}
%    \end{macrocode}

% \macro{\childdocby}
% The command |\childdocby| ....
%    \begin{macrocode}
\newcommand{\childdocby}[2][]
{
  \childdocdisable
  \childdoctrue
  \childdocmanualtrue
  \if?#1?\else
    \def\jobname{#2}
  \fi
  \def\childdocjob{#2}
  \input{#2}
  \endinput
}
%    \end{macrocode}

% \macro{\childdocforward}
% The command |\childdocforward| redirects
% compilation to the main file or
% (if the optional argument is given) a child file.
% Parameters are set as if the main file
% or a child file starting with |\childdocof| was compiled.
% Then compilation is handed over to the main file:
%    \begin{macrocode}
\newcommand{\childdocforward}[2][]
{
  \begingroup
    \if?#1?
      \def\childdoctmp
      {
        \def\childdocname{#2}
        \def\childdocjob{#2}
        \def\jobname{#2}
        \input{#2}
        \endinput
      }
    \else
      \def\childdoctmp
      {
        \childdocdisable
        \def\childdocname{#2}
        \childdoctrue
        \includeonly{#2}
        \def\childdocjob{#1}
        \def\jobname{#1}
        \input{#1}
        \endinput
      }
    \fi
    \expandafter
  \endgroup
  \childdoctmp
}
%    \end{macrocode}

% \macro{\childdocforwardprefix}
% The command |\childdocforwardprefix| redirects
% compilation to the main or a child file by means of a pattern.
% The prefix |#1| in the current filename is replaced by |#2|
% and the suffix of the current filename is kept
% (it is assumed that the filename does not contain the substring `|~~~|'
% which is used as a delimiter).
% Compilation is handed over to the new file by |\childdocforward|:
%    \begin{macrocode}
\newcommand{\childdocforwardprefix}[3][]
{
  \begingroup
    \def\childdocextract #2##1~~~{\def\childdoctmp{\childdocforward[#1]{#3##1}}}
    \expandafter\childdocextract\childdocname~~~
    \expandafter
  \endgroup
  \childdoctmp
}
%    \end{macrocode}

% \macro{\childdoc}
% The deprecated macro |\childdoc| is a legacy version of |\childdocmain|:
%    \begin{macrocode}
\newcommand{\childdoc}{\childdocmain}
%    \end{macrocode}

% \macro{\childdocredirect}
% The deprecated macro |\childdocredirect| is a legacy version
% of |\childdocforward| and |\childdocforwardprefix|:
%    \begin{macrocode}
\newcommand{\childdocredirect}[2][]
{
  \begingroup
    \if?#1?
      \def\childdoctmp{\childdocforward{#2}}
    \else
      \def\childdoctmp{\childdocforwardprefix{#1}{#2}}
    \fi
    \expandafter
  \endgroup
  \childdoctmp
}
%    \end{macrocode}

%\iffalse
%</package>
%\fi
%
\endinput
|\\
|\childdocby{|\textit{main}|}|\\
\end{tabular}
\end{center}
%
Both forms have slightly different effects as described above.
The main file is prepared as usual, see \secref{sec:include}.

%%%%%%%%%%%%%%%%%%%%%%%%%%%%%%%%%%%%%%%%%%%%%%%%%%%%%%%%%%%%%%%%%%%%%%%%%%%%%%%%
\subsection{Legacy Detection}
\label{sec:detection}

The directive |\childdocmain| in the main file can detect
whether the complete document or merely a child is to be compiled
even without using the directive |\childdocof|.
This method is deprecated because it is less robust
and there is no compelling reason to use it;
it is merely provided for backward compatibility
and it may be removed in future versions.

If the detection mechanism is to be used,
it is mandatory to correctly specify
the filename of the main file as the argument of |\childdocmain|:
%
\begin{center}
\begin{tabular}{l}
|% \iffalse
%
% childdoc.dtx Copyright (C) 2017-2018 Niklas Beisert
%
% This work may be distributed and/or modified under the
% conditions of the LaTeX Project Public License, either version 1.3
% of this license or (at your option) any later version.
% The latest version of this license is in
%   http://www.latex-project.org/lppl.txt
% and version 1.3 or later is part of all distributions of LaTeX
% version 2005/12/01 or later.
%
% This work has the LPPL maintenance status `maintained'.
%
% The Current Maintainer of this work is Niklas Beisert.
%
% This work consists of the files childdoc.dtx and childdoc.ins
% and the derived files childdoc.def and cdocsamp.tex with
% cdocsch1.tex, cdocsch2.tex, cdocsdrf.tex, cdocsfn1.tex, cdocsfn2.tex.
%
%<package>\ifdefined\childdocmain\endinput\fi
%<package>\ProvidesFile{childdoc.def}[2018/12/30 v2.0 child document driver]
%<samplemain>\ProvidesFile{cdocsamp.tex}[2018/12/30 v2.0 sample for childdoc]
%<*driver>
%\ProvidesFile{childdoc.drv}[2018/12/30 v2.0 childdoc reference manual file]
\PassOptionsToClass{10pt,a4paper}{article}
\documentclass{ltxdoc}

\usepackage[margin=35mm]{geometry}
\usepackage{hyperref}
\usepackage{hyperxmp}
\usepackage[usenames]{color}

\hypersetup{colorlinks=true}
\hypersetup{pdfstartview=FitH}
\hypersetup{pdfpagemode=UseNone}
\hypersetup{pdfsource={}}
\hypersetup{pdflang={en-UK}}
\hypersetup{pdfcopyright={Copyright 2017-2018 Niklas Beisert.
  This work may be distributed and/or modified under the
  conditions of the LaTeX Project Public License, either version 1.3
  of this license or (at your option) any later version.}}
\hypersetup{pdflicenseurl={http://www.latex-project.org/lppl.txt}}
\hypersetup{pdfcontactaddress={ETH Zurich, ITP, HIT K,
  Wolfgang-Pauli-Strasse 27}}
\hypersetup{pdfcontactpostcode={8093}}
\hypersetup{pdfcontactcity={Zurich}}
\hypersetup{pdfcontactcountry={Switzerland}}
\hypersetup{pdfcontactemail={nbeisert@itp.phys.ethz.ch}}
\hypersetup{pdfcontacturl={http://people.phys.ethz.ch/\xmptilde nbeisert/}}

\newcommand{\secref}[1]{\hyperref[#1]{section \ref*{#1}}}

\parskip1ex
\parindent0pt
\let\olditemize\itemize
\def\itemize{\olditemize\parskip0pt}

\begin{document}

\title{The \textsf{childdoc} Package}
\hypersetup{pdftitle={The childdoc Package}}
\author{Niklas Beisert\\[2ex]
  Institut f\"ur Theoretische Physik\\
  Eidgen\"ossische Technische Hochschule Z\"urich\\
  Wolfgang-Pauli-Strasse 27, 8093 Z\"urich, Switzerland\\[1ex]
  \href{mailto:nbeisert@itp.phys.ethz.ch}
  {\texttt{nbeisert@itp.phys.ethz.ch}}}
\hypersetup{pdfauthor={Niklas Beisert}}
\hypersetup{pdfsubject={Manual for the LaTeX2e Package childdoc}}
\date{30 December 2018, \textsf{v2.0}}
\maketitle

\begin{abstract}\noindent
\textsf{childdoc} is a \LaTeXe{} package
that enables the direct compilation
of document sections included by |\include|
to individual files.
\end{abstract}

\begingroup
\parskip0ex
\tableofcontents
\endgroup

%%%%%%%%%%%%%%%%%%%%%%%%%%%%%%%%%%%%%%%%%%%%%%%%%%%%%%%%%%%%%%%%%%%%%%%%%%%%%%%%
%%%%%%%%%%%%%%%%%%%%%%%%%%%%%%%%%%%%%%%%%%%%%%%%%%%%%%%%%%%%%%%%%%%%%%%%%%%%%%%%
\section{Introduction}

\LaTeX{} provides a mechanism to structure a large document (such as a book)
into a main file and several child files (containing the chapters)
using the |\include| command.
This mechanism is beneficial for documents
which span hundreds of pages in order to
make the source file(s) more manageable.
Moreover, compilation can be restricted to
selected child files by means of the |\includeonly| command.
The latter feature can be used to reduce the compilation time while editing
(this was significantly more useful in the earlier days of \LaTeX{})
or to generate a smaller document which is easier to navigate.
Another application of |\includeonly| is to generate
documents consisting of selected parts of the complete document.

However, there are a few drawbacks of the plain |\include| mechanism:
\begin{itemize}
\item
The child files cannot be compiled on their own,
they can only be compiled via the main file.
A naive editing environment
(such as a text editor with an option
to have the current file processed by \LaTeX)
may require one to switch to the main file before compiling;
attempting to compile the child file produces errors.
\item
The main file must be modified (each time)
to adjust the |\includeonly| command
to the present needs. This easily leaves the main file in a messy state.
\item
The generated document will always carry the filename
of the main document. This is inconvenient if
several child files are to be compiled and
to be kept for distribution.
\end{itemize}

The present package provides a simple interface
to make child files individually compilable by \LaTeX{}.
Compiling a child file then has the same effect as compiling
the main file with an |\includeonly| command
to select the appropriate child.
Moreover the generated document will carry the name of the child
rather than the main file.
This resolves all three above issues.

This feature is meant to make the editing of books,
thesis documents and lecture notes somewhat more convenient.
However, the package can also be used efficiently for
composing a series of documents (such as exercise sheets)
which are typically distributed individually.
It then assists the author in generating the individual documents
(potentially in different versions)
as well as a document containing the collected series.
Another application is in developing style files
or other kinds of included material
where compilation of the style file could redirect
to a sample or test file.

%%%%%%%%%%%%%%%%%%%%%%%%%%%%%%%%%%%%%%%%%%%%%%%%%%%%%%%%%%%%%%%%%%%%%%%%%%%%%%%%
%%%%%%%%%%%%%%%%%%%%%%%%%%%%%%%%%%%%%%%%%%%%%%%%%%%%%%%%%%%%%%%%%%%%%%%%%%%%%%%%
\section{Usage}

First of all, the package \textsf{childdoc} is \emph{not} a standard
\LaTeXe{} |.sty| style file! Therefore it needs to be invoked in
a non-standard way.

%%%%%%%%%%%%%%%%%%%%%%%%%%%%%%%%%%%%%%%%%%%%%%%%%%%%%%%%%%%%%%%%%%%%%%%%%%%%%%%%
\subsection{Included Files}
\label{sec:include}

%%%%%%%%%%%%%%%%%%%%%%%%%%%%%%%%%%%%%%%%
\DescribeMacro{\childdocmain}
To use the package, add the commands
\begin{center}
\begin{tabular}{l}
|\input{childdoc.def}|\\
|\childdocmain{}|\\
\end{tabular}
\end{center}
at the very top of the main \LaTeX{} file,
in particular \emph{before} the |\documentclass| statement!
The argument of |\childdocmain| should be left empty
(but it must be present).

%%%%%%%%%%%%%%%%%%%%%%%%%%%%%%%%%%%%%%%%
\DescribeMacro{\childdocof}
Furthermore, add the commands
\begin{center}
\begin{tabular}{l}
|\input{childdoc.def}|\\
|\childdocof{|\textit{main}|}|\\
\end{tabular}
\end{center}
at the top of every child file \textit{child}
which is included by |\include{|\textit{child}|}|
from within the main file
(or at least for those files to be compiled individually).
The argument \textit{main} must be the filename of the main file.

There are a couple of
considerations in setting up the main and child documents:

%%%%%%%%%%%%%%%%%%%%%%%%%%%%%%%%%%%%%%%%
\paragraph{Restrictions.}

Please note the following restrictions:
\begin{itemize}
\item
|\childdocmain| must be called with one argument \textit{main}
to ensure compatibility with earlier version of the package.
It must either be empty (|\childdocmain{}|)
or precisely match the filename of the main file in which it is specified.
See \secref{sec:detection} for further information.
\item
The filename \textit{main} must be specified without the |.tex| extension.
\item
The filename \textit{main} is case sensitive
(even in case-insensitive file systems)
due to internal string comparison.
\item
The argument \textit{main} should be fully expanded, it cannot be a macro.
\item
Subdirectories and special characters should be avoided in filenames.
\item
The command |\childdocmain{|\textit{main}|}| must be followed by a whitespace.
It should not be followed immediately by another command
or by a comment mark `|%|'.
This is because the \TeX{} parser reads the token immediately following
the argument of |\childdocmain| and puts it
at the beginning of every child section;
however, a white\-space is ignored.
\end{itemize}

%%%%%%%%%%%%%%%%%%%%%%%%%%%%%%%%%%%%%%%%
\paragraph{Content of Main File.}

It is advisable to place all content in the child files included by |\include|.
Any output contained in the main file will appear in all child documents
unless suppressed manually;
it cannot be suppressed automatically by the |\includeonly| directive
and thus should normally be avoided.
A method to include some content in the main file
by means of conditional processing is described in \secref{sec:conditional}.

%%%%%%%%%%%%%%%%%%%%%%%%%%%%%%%%%%%%%%%%
\paragraph{Page Numbering.}

When only a part of the document is compiled,
the appropriate numbering of pages
(as well as other status parameters)
is determined from the |.aux| files.
The latter contain information from previous passes.
However this information needs to propagate through
all intermediate child documents.
Therefore the page numbering in child documents may well
be inconsistent until the complete document is compiled at least once.

A useful (if unconventional) way to always ensure a consistent
page numbering is to restart the numbering in each child document
and denote the pages by `\textit{child}|.|\textit{page}'
where \textit{child} represents the chapter/section number of the child file.
This can be achieved by the command
|\numberwithin{page}{|\textit{child}|}|
of the \textsf{amsmath} package
where \textit{child} can be |chapter| or |section|
depending on the chosen structuring.
Alternatively, one can modify the macro |\thepage| appropriately
and reset the counter |page| at the start of each child file.

%%%%%%%%%%%%%%%%%%%%%%%%%%%%%%%%%%%%%%%%%%%%%%%%%%%%%%%%%%%%%%%%%%%%%%%%%%%%%%%%
\subsection{Conditional Processing}
\label{sec:conditional}

The package provides a mechanism to compile different versions
of a document. To customise the versions further some conditional processing
can come in handy to distinguish which version is being compiled.
The package provides two macros to describe the compilation context:

%%%%%%%%%%%%%%%%%%%%%%%%%%%%%%%%%%%%%%%%
\DescribeMacro{\ifchilddoc}
The conditional |\ifchilddoc| distinguishes between the compilation of
child documents and the main document:
%
\begin{center}
|\ifchilddoc |\textit{child-code}| |[|\||else |\textit{main-code}]| \||fi|
\end{center}

%%%%%%%%%%%%%%%%%%%%%%%%%%%%%%%%%%%%%%%%
\DescribeMacro{\childdocname}
\DescribeMacro{\childdocjob}
The macro |\childdocname| contains the filename (without extension)
of the main or child file being processed.
Note that |\childdocjob| will always contain the name of the main file.

%%%%%%%%%%%%%%%%%%%%%%%%%%%%%%%%%%%%%%%%
\paragraph{Title Page.}

Conditional processing can be used to include a title or banner page
in the main document when proper precautions are taken.
Importantly, the code in the main file should ensure that the page counter
(as well as other status parameters which are stored in the |.aux| files)
takes the same value after the conditional processing.
Otherwise the page numbers may take divergent values
depending on which part is compiled.

For example, a title page could be declared by:
%
\begin{center}
\begin{tabular}{l}
|\ifchilddoc\||else|\\
|\addtocounter{page}{-1}|\\
\textit{code for title page}\\
|\newpage|\\
|\||fi|
\end{tabular}
\end{center}
%
A banner page for the child documents can be generated by:
%
\begin{center}
\begin{tabular}{l}
|\ifchilddoc|\\
|\addtocounter{page}{-1}|\\
\textit{code for banner page}\\
|\newpage|\\
|\||fi|
\end{tabular}
\end{center}
%
Here one could write a message such as:
\begin{center}
|This is the part \childdocname{} of \childdocjob{}.|
\end{center}

%%%%%%%%%%%%%%%%%%%%%%%%%%%%%%%%%%%%%%%%%%%%%%%%%%%%%%%%%%%%%%%%%%%%%%%%%%%%%%%%
\subsection{Flags}
\label{sec:flags}

The package makes it easy to generate different versions
of the main or child documents.
To this end compilation flags can be defined
and assigned different default values.
They will be particularly useful in conjunction
with the forwarding mechanism described in \secref{sec:forward}.

For example, it may be useful to have a flag |\version|
which can be set to |draft| or |final|.
The document source will contain some conditional code
depending on the value of |\version|.
Suppose further, the flag should default to |final| for the main file
and to |draft| for child files
which is a natural assignment for editing the document.
This is achieved by placing the following code
in the preamble of the main document
(below the |\childdocmain| directive):
%
\begin{center}
\begin{tabular}{l}
|\ifchilddoc|\\
|\providecommand{\version}{draft}|\\
|\||else|\\
|\providecommand{\version}{final}|\\
|\||fi|
\end{tabular}
\end{center}
%
The definition by |\providecommand| makes sure
that previous definitions are not overwritten.
Further statements |\providecommand{\version}{...}|
can thus be added before the above code to override it.

For the main file, one might add a line
(between |\childdocmain| and the above block)
%
\begin{center}
|%\ifchilddoc\||else\providecommand{\version}{draft}\||fi|
\end{center}
%
which can be uncommented to produce a draft version.
Likewise one can add a line to the very top of a child file
(above the |\childdocof{|\textit{main}|}| directive)
%
\begin{center}
|%\providecommand{\version}{final}|
\end{center}
%
which can be uncommented to produce the final version of this child document.

%%%%%%%%%%%%%%%%%%%%%%%%%%%%%%%%%%%%%%%%%%%%%%%%%%%%%%%%%%%%%%%%%%%%%%%%%%%%%%%%
\subsection{Forwarding}
\label{sec:forward}

Different versions of the main or child documents
using compilation flags as described in \secref{sec:flags}
can be (permanently) stored in different files
for convenient compilation, viewing and distribution.
To this end, the package defines a command
to pass on compilation to a different file:

%%%%%%%%%%%%%%%%%%%%%%%%%%%%%%%%%%%%%%%%
\DescribeMacro{\childdocforward}
The command |\childdocforward| redirects processing to
another source file:
%
\begin{center}
\begin{tabular}{l}
|\input{childdoc.def}|\\
|\childdocforward[|\textit{main}|]{|\textit{dest}|}|\\
\end{tabular}
\end{center}
%
The argument \textit{dest} is the destination file
(without extension).
It should be the main file or one of the child files.
Note that further \textsf{childdoc} directives
such as |\childdocof| and |\childdocforward|
in the indicated file will be processed in this form.
The optional argument \textit{main}
passes on directly to the main file \textit{main}
while pretending to compile the child \textit{dest}.
This form behaves as if \textit{dest}
issues |\childdocof{|\textit{main}|}| right away,
and no further \textsf{childdoc} directives will be processed.

%%%%%%%%%%%%%%%%%%%%%%%%%%%%%%%%%%%%%%%%
\DescribeMacro{\...prefix}
In the alternative form |\childdocforwardprefix|,
%
\begin{center}
\begin{tabular}{l}
|\input{childdoc.def}|\\
|\childdocforwardprefix[|\textit{main}|]{|\textit{prefix}|}{|\textit{dest}|}|
\end{tabular}
\end{center}
%
the destination file is determined by a pattern
depending on the current file:
To make this work, the current file must be called
`{\textit{prefix}\hspace{0.2em}\textit{suffix}}'
with \textit{prefix} matching precisely the argument.
Processing is then passed on to the file
`{\textit{dest}\hspace{0.2em}\textit{suffix}}'.
Surely, the same effect is achieved by
directly specifying the
argument `{\textit{dest}\hspace{0.2em}\textit{suffix}}'
in the first form.
However, that requires to set up a different file
for each child. With the alternative form of the command
all these files can have exactly the same content
which simplifies setting them up and maintaining them.

For example, the following file |draft.tex|
with a compilation flag |\version| as described in \secref{sec:flags}
compiles the main document as a draft:
%
\begin{center}
\begin{tabular}{l}
|\def\version{draft}|\\
|\input{childdoc.def}|\\
|\childdocforward{|\textit{main}|}|
\end{tabular}
\end{center}
%
Likewise, the following files |final|\textit{nn}|.tex|
compile the final version of the child document
|child|\textit{nn}|.tex|:
%
\begin{center}
\begin{tabular}{l}
|\def\version{final}|\\
|\input{childdoc.def}|\\
|\childdocforwardprefix{final}{child}|
\end{tabular}
\end{center}
%

Note that when several versions of a main file and/or of each child file
are to be generated, it may be convenient to set up a |Makefile| or
shell script to automatise the process.

%%%%%%%%%%%%%%%%%%%%%%%%%%%%%%%%%%%%%%%%%%%%%%%%%%%%%%%%%%%%%%%%%%%%%%%%%%%%%%%%
\subsection{Command Line Processing}
\label{sec:commandline}

The effect of redirection files can also be achieved by invoking
the \LaTeX{} compiler with a more elaborate command line.
Most conveniently this should be done as part
of a shell script or a |Makefile|.

When using \textsf{childdoc} in the main file, the following
command lines effectively perform a redirection
(note that depending on the shell being used,
backslashes may have to be doubled: `|\|' $\to$ `|\\|'):
%
\begin{center}
|... -jobname "|\textit{target}|" |\\|"|[\textit{flags}]%
|\input{childdoc.def}\childdocforward[|\textit{main}|]{|\textit{dest}|}"|
\end{center}
%
Here \textit{target} is the name of the output file,
\textit{main} is the name of the main file
and \textit{dest} is the name of the main or child file to be processed
(all filenames without extensions).
The optional argument \textit{main} can be omitted
if \textit{main} matches \textit{dest}.
Optionally, compilation \textit{flags} can be defined via |\def| commands.
This command line makes the \TeX{} engine believe
it is compiling the file \textit{target}
whose content is specified as the latter parameter.
The provided code then forwards the processing to
\textit{main} or \textit{dest} as described in \secref{sec:forward}.

%%%%%%%%%%%%%%%%%%%%%%%%%%%%%%%%%%%%%%%%%%%%%%%%%%%%%%%%%%%%%%%%%%%%%%%%%%%%%%%%
\subsection{Include by Input}
\label{sec:input}

Including child documents by |\include| has some restrictions by design.
Most notably, the content of a child document always occupies
its own set of pages; pages cannot be shared between child documents.
Usually, this behaviour makes perfect sense
because each child document contain an essential part of the document.
However, in some situations it may be desirable to compose
a document from a collection of parts
without having mandatory page breaks between then.
For this case, the package
provides a mechanism to include parts
by |\input| which can also be processed individually.
However, by construction this mechanism
requires manual handling of the content to be output.

%%%%%%%%%%%%%%%%%%%%%%%%%%%%%%%%%%%%%%%%
\DescribeMacro{\ifchilddocmanual}
The main file should be prepared as usual, see \secref{sec:include}.
However, the document body must make a distinction
between processing of an individual part and of the main document, e.g.:
%
\begin{center}
\begin{tabular}{l}
|\ifchilddocmanual|\\
|\input{\childdocname}|\\
|\||else|\\
\textit{document body with }|\input{|\textit{part}|}|\\
|\||fi|
\end{tabular}
\end{center}
%
The conditional |\ifchilddocmanual| is true whenever
a part to be included by |\input| is being compiled,
and the name of the part is stored in |\childdocname|.

%%%%%%%%%%%%%%%%%%%%%%%%%%%%%%%%%%%%%%%%
\DescribeMacro{\childdocby}
Each part to be included by |\input| should start with:
%
\begin{center}
\begin{tabular}{l}
|\input{childdoc.def}|\\
|\childdocby{|\textit{main}|}|\\
\end{tabular}
\end{center}
%
The directive |\childdocby| is similar to |\childdocof|
described in \secref{sec:include},
but the subsequent selection of content must be done manually.
To that end, both |\ifchilddoc| and |\ifchilddocmanual|
will be true upon processing of a part,
and the name of the part is stored in |\childdocname|.
Note that |\jobname| will be set to the filename of the current part
so that each part receives an individual |.aux| file
that does not interfere with the |.aux| file(s) of the main document.
This behaviour can be altered by the alternative form
|\childdocby[*]{|\textit{main}|}| (with a non-empty optional argument)
which uses the |.aux| file of the main document
by setting |\jobname| to \textit{main}.

%%%%%%%%%%%%%%%%%%%%%%%%%%%%%%%%%%%%%%%%%%%%%%%%%%%%%%%%%%%%%%%%%%%%%%%%%%%%%%%%
\subsection{Driver Development}
\label{sec:driver}

The \textsf{childdoc} mechanism can also be use for the development
of definition files such as \LaTeX{} styles or classes.
This case differs from the above setup with multiple parts
included by |\include| in that no |\includeonly| should be invoked.
This can be achieved by starting the include file
(before |\ProvidesPackage|) with:
%
\begin{center}
\begin{tabular}{l}
|\input{childdoc.def}|\\
|\childdocforward{|\textit{main}|}|\\
\end{tabular}
\end{center}
%
or alternatively with:
%
\begin{center}
\begin{tabular}{l}
|\input{childdoc.def}|\\
|\childdocby{|\textit{main}|}|\\
\end{tabular}
\end{center}
%
Both forms have slightly different effects as described above.
The main file is prepared as usual, see \secref{sec:include}.

%%%%%%%%%%%%%%%%%%%%%%%%%%%%%%%%%%%%%%%%%%%%%%%%%%%%%%%%%%%%%%%%%%%%%%%%%%%%%%%%
\subsection{Legacy Detection}
\label{sec:detection}

The directive |\childdocmain| in the main file can detect
whether the complete document or merely a child is to be compiled
even without using the directive |\childdocof|.
This method is deprecated because it is less robust
and there is no compelling reason to use it;
it is merely provided for backward compatibility
and it may be removed in future versions.

If the detection mechanism is to be used,
it is mandatory to correctly specify
the filename of the main file as the argument of |\childdocmain|:
%
\begin{center}
\begin{tabular}{l}
|\input{childdoc.def}|\\
|\childdocmain{|\textit{main}|}|\\
\end{tabular}
\end{center}
%
If |\jobname| does not match the argument \textit{main} of |\childdocmain|,
it is assumed that |\jobname| points to the child file to be compiled.
When using |\childdocmain| with the main file specified as argument,
it suffices to start a child file
with just |\input{|\textit{main}|}|
without loading of the package and using |\childdocof|.
If instead all processing is done
with the appropriate \textsf{childdoc} directives,
the argument of \textit{main} of |\childdocmain| can be empty.

An alternative version of the command line processing described
in \secref{sec:commandline} using the detection mechanism reads:
%
\begin{center}
|... -jobname "|\textit{target}|" "|[\textit{flags}]%
[|\def\jobname{|\textit{dest}|}|]|\input{|\textit{main}|}"|
\end{center}

%%%%%%%%%%%%%%%%%%%%%%%%%%%%%%%%%%%%%%%%%%%%%%%%%%%%%%%%%%%%%%%%%%%%%%%%%%%%%%%%
\subsection{Manual Code}
\label{sec:manual}

In case one cannot be certain whether the definitions file |childdoc.def|
is installed on the target \TeX{} distribution
and one prefers not to ship it,
it is conceivable to paste a few relevant commands into the sources.

To that end, drop all statements |\input{childdoc.def}|
and perform the replacements as outlined below.
Instead of |\childdocmain{|\textit{main}|}| add the following code
to the top of the main file:
%
\begin{center}
\begin{tabular}{l}
|\||ifdefined\childdocname\endinput\||fi\newif\ifchilddoc|\\
|\edef\childdocname{\scantokens\expandafter{\jobname\noexpand}}|\\
|\def\childdocmain{|\textit{main}|}\||ifx\childdocmain\childdocname\||else|\\
|\childdoctrue\includeonly{\childdocname}\let\jobname\childdocmain\||fi|\\
\end{tabular}
\end{center}
%
Instead of |\childdocof{|\textit{main}|}| just include the main file
at the top of each child file:
%
\begin{center}
|\input{|\textit{main}|}|
\end{center}
%
A simple redirection |\childdocforward{|\textit{dest}|}| is achieved by:
%
\begin{center}
|\def\jobname{|\textit{dest}|}\input{\jobname}|
\end{center}
%
The redirection with prefix
|\childdocforwardprefix[|\textit{prefix}|]{|\textit{dest}|}|
is accomplished by:
%
\begin{center}
\begin{tabular}{l}
|{\edef\jobname{\scantokens\expandafter{\jobname\noexpand}}|\\
|\def\redirectjob |\textit{prefix}|#1~~~{\gdef\jobname{|\textit{dest}|#1}}|\\
|\expandafter\redirectjob\jobname~~~}\input{\jobname}|
\end{tabular}
\end{center}

In an alternative approach,
child documents can be compiled by a specific command line
without additional code or specific definitions:
%
\begin{center}
|... -jobname "|\textit{target}|" "|[\textit{flags}]%
|\includeonly{|\textit{dest}|}\input{|\textit{main}|}"|
\end{center}
%

%%%%%%%%%%%%%%%%%%%%%%%%%%%%%%%%%%%%%%%%%%%%%%%%%%%%%%%%%%%%%%%%%%%%%%%%%%%%%%%%
%%%%%%%%%%%%%%%%%%%%%%%%%%%%%%%%%%%%%%%%%%%%%%%%%%%%%%%%%%%%%%%%%%%%%%%%%%%%%%%%
\section{Information}

%%%%%%%%%%%%%%%%%%%%%%%%%%%%%%%%%%%%%%%%%%%%%%%%%%%%%%%%%%%%%%%%%%%%%%%%%%%%%%%%
\subsection{Copyright}

Copyright \copyright{} 2017--2018 Niklas Beisert

This work may be distributed and/or modified under the
conditions of the \LaTeX{} Project Public License, either version 1.3
of this license or (at your option) any later version.
The latest version of this license is in
  \url{http://www.latex-project.org/lppl.txt}
and version 1.3 or later is part of all distributions of \LaTeX{}
version 2005/12/01 or later.

This work has the LPPL maintenance status `maintained'.

The Current Maintainer of this work is Niklas Beisert.

This work consists of the files |README.txt|, |childdoc.ins| and |childdoc.dtx|
as well as the derived files |childdoc.def|, |cdocsamp.tex|
with |cdocsch1.tex|, |cdocsch2.tex|, |cdocspt3.tex|, |cdocspt4.tex|,
|cdocsdrf.tex|, |cdocsfn1.tex|, |cdocsfn2.tex|
as well as |childdoc.pdf|.

%%%%%%%%%%%%%%%%%%%%%%%%%%%%%%%%%%%%%%%%%%%%%%%%%%%%%%%%%%%%%%%%%%%%%%%%%%%%%%%%
\subsection{Files and Installation}

The package consists of the files:
%
\begin{center}
\begin{tabular}{ll}
    |README.txt|   & readme file \\
    |childdoc.ins| & installation file \\
    |childdoc.dtx| & source file \\
    |childdoc.def| & definition file \\
    |cdocsamp.tex| & sample main file \\
    |cdocsch1.tex| & sample include file \\
    |cdocsch2.tex| & sample include file \\
    |cdocspt3.tex| & sample part file \\
    |cdocspt4.tex| & sample part file \\
    |cdocsdrf.tex| & sample redirection file \\
    |cdocsfn1.tex| & sample redirection file \\
    |cdocsfn2.tex| & sample redirection file \\
    |childdoc.pdf| & manual
\end{tabular}
\end{center}
%
The distribution consists of the files
|README.txt|, |childdoc.ins| and |childdoc.dtx|.
%
\begin{itemize}
\item
Run (pdf)\LaTeX{} on |childdoc.dtx|
to compile the manual |childdoc.pdf| (this file).
\item
Run \LaTeX{} on |childdoc.ins| to create the definitions file |childdoc.def|
and the sample |cdocsamp.tex| with include files
|cdocsch1.tex|, |cdocsch2.tex|, |cdocspt3.tex|, |cdocspt4.tex|,
|cdocsdrf.tex|, |cdocsfn1.tex|, |cdocsfn2.tex|.
Then copy the file |childdoc.def| to an appropriate directory of your \LaTeX{}
distribution, e.g.\ \textit{texmf-root}|/tex/latex/childdoc|.
\end{itemize}

%%%%%%%%%%%%%%%%%%%%%%%%%%%%%%%%%%%%%%%%%%%%%%%%%%%%%%%%%%%%%%%%%%%%%%%%%%%%%%%%
\subsection{Related CTAN Packages}

There are several other packages which offer a similar functionality:
%
\begin{itemize}
\item
The packages
\href{http://ctan.org/pkg/docmute}{\textsf{docmute}},
\href{http://ctan.org/pkg/includex}{\textsf{includex}} and
\href{http://ctan.org/pkg/standalone}{\textsf{standalone}}
provide commands to include only the document body of
a child file thus allowing both files to be compiled individually.
\item
The packages \href{http://ctan.org/pkg/subdocs}{\textsf{subdocs}}
and \href{http://ctan.org/pkg/subfiles}{\textsf{subfiles}}
provide structures in which the main and child documents can be
encapsulated and allowing them to be compiled individually.
The inclusion mechanism is different from the conventional |\include|.
\item
The package \href{http://ctan.org/pkg/combine}{\textsf{combine}}
is an elaborate solution to combine several documents into one.
\end{itemize}
%
See also the CTAN topic \href{http://ctan.org/topic/subdocs}{\textsf{subdocs}}
for further related packages.
The present package differs from the above solutions in that
a document structure constructed with the conventional |\include| mechanism
just needs two extra commands at the top of every file
such that all constituent files can be compiled individually.

%%%%%%%%%%%%%%%%%%%%%%%%%%%%%%%%%%%%%%%%%%%%%%%%%%%%%%%%%%%%%%%%%%%%%%%%%%%%%%%%
%\subsection{Feature Suggestions}
%
%The following is a list of features which may be useful for future
%versions of this package:
%%
%\begin{itemize}
%\item
%\ldots
%\end{itemize}

%%%%%%%%%%%%%%%%%%%%%%%%%%%%%%%%%%%%%%%%%%%%%%%%%%%%%%%%%%%%%%%%%%%%%%%%%%%%%%%%
\subsection{Revision History}

%%%%%%%%%%%%%%%%%%%%%%%%%%%%%%%%%%%%%%%%
\paragraph{v2.0:} 2018/12/30

\begin{itemize}
\item
immediate forward processing
\item
added |\childdocby| mechanism
\item
manual restructured
\end{itemize}

%%%%%%%%%%%%%%%%%%%%%%%%%%%%%%%%%%%%%%%%
\paragraph{v1.6:} 2018/01/17

\begin{itemize}
\item
application for development of include files
\item
corrections to manual
\end{itemize}

%%%%%%%%%%%%%%%%%%%%%%%%%%%%%%%%%%%%%%%%
\paragraph{v1.5:} 2017/05/21

\begin{itemize}
\item
more complete structuring introduced
\item
|\childdocof| introduced
\item
|\childdoc| renamed to |\childdocmain|
\item
|\childredirect| renamed to |\childdocforward| and |\childdocforwardprefix|
and functionality expanded
\end{itemize}

%%%%%%%%%%%%%%%%%%%%%%%%%%%%%%%%%%%%%%%%
\paragraph{v1.0:} 2017/04/27

\begin{itemize}
\item
manual and install package
\item
first version published on CTAN
\end{itemize}

%%%%%%%%%%%%%%%%%%%%%%%%%%%%%%%%%%%%%%%%
\paragraph{v0.6:} 2017/04/26

\begin{itemize}
\item
redirection mechanism added
\end{itemize}

%%%%%%%%%%%%%%%%%%%%%%%%%%%%%%%%%%%%%%%%
\paragraph{v0.5:} 2017/04/26

\begin{itemize}
\item
functionality in definition file
\end{itemize}


%%%%%%%%%%%%%%%%%%%%%%%%%%%%%%%%%%%%%%%%%%%%%%%%%%%%%%%%%%%%%%%%%%%%%%%%%%%%%%%%
%%%%%%%%%%%%%%%%%%%%%%%%%%%%%%%%%%%%%%%%%%%%%%%%%%%%%%%%%%%%%%%%%%%%%%%%%%%%%%%%
%%%%%%%%%%%%%%%%%%%%%%%%%%%%%%%%%%%%%%%%%%%%%%%%%%%%%%%%%%%%%%%%%%%%%%%%%%%%%%%%
\appendix

\settowidth\MacroIndent{\rmfamily\scriptsize 000\ }

 \DocInput{childdoc.dtx}

\end{document}
%</driver>
% \fi
%
% %%%%%%%%%%%%%%%%%%%%%%%%%%%%%%%%%%%%%%%%%%%%%%%%%%%%%%%%%%%%%%%%%%%%%%%%%%%%%%
% %%%%%%%%%%%%%%%%%%%%%%%%%%%%%%%%%%%%%%%%%%%%%%%%%%%%%%%%%%%%%%%%%%%%%%%%%%%%%%
% \section{Sample}
%\iffalse
%<*samplemain>
%\fi
%
% The following presents a sample document
% with two chapters, two parts, a title page,
% a compile flag as well as three forwarding files to set the flag.
% It consists of eight |.tex| files:
% \begin{center}
% \begin{tabular}{ll}
% |cdocsamp.tex|&main file\\
% |cdocsch1.tex|&include file for chapter 1\\
% |cdocsch2.tex|&include file for chapter 2\\
% |cdocspt3.tex|&include file for part 3\\
% |cdocspt4.tex|&include file for part 4\\
% |cdocsdrf.tex|&forwarding file for main file in draft mode\\
% |cdocsfi1.tex|&forwarding file for final version of chapter 1\\
% |cdocsfi2.tex|&forwarding file for final version of chapter 2\\
% \end{tabular}
% \end{center}
% Each of the eight files can be compiled directly by the \LaTeX{} compiler.
%
% %%%%%%%%%%%%%%%%%%%%%%%%%%%%%%%%%%%%%%
% \paragraph{Main File.}
%
% The main file is called |cdocsamp.tex|.
%
% Load the \textsf{childdoc} definitions and
% declare the filename for the main document:
%    \begin{macrocode}
\input{childdoc.def}
\childdocmain{}
%    \end{macrocode}

% Optional override for |\version| flag:
%    \begin{macrocode}
%%\ifchilddoc\else\providecommand{\version}{draft}\fi
%    \end{macrocode}

% Define the default values for the |\version| flag
% (|final| for the main file and |draft| for childs):
%    \begin{macrocode}
\ifchilddoc
\providecommand{\version}{draft}
\else
\providecommand{\version}{final}
\fi
%    \end{macrocode}

% Load the standard document class:
%    \begin{macrocode}
\documentclass[12pt]{article}
%    \end{macrocode}

% Start the document body:
%    \begin{macrocode}
\begin{document}
%    \end{macrocode}

% Declare a title page.
% Print title, part of document being processed and version flag:
%    \begin{macrocode}
\addtocounter{page}{-1}
\begin{center}
{\LARGE\bfseries{}childdoc example\par}
\vspace{1cm}
\ifchilddoc
\ifchilddocmanual part\else chapter\fi:
`\childdocname' of `\childdocjob'\par
\else
main document: `\childdocjob'\par
\fi
version: \version\par
\end{center}
\newpage
%    \end{macrocode}

% Manually include selected file,
% otherwise process as usual:
%    \begin{macrocode}
\ifchilddocmanual
\section*{part `\childdocname'}
\input{\childdocname}
\else
%    \end{macrocode}

% Include the two chapters:
%    \begin{macrocode}
\include{cdocsch1}
\include{cdocsch2}
%    \end{macrocode}

% Include the two parts unless only chapters should be displayed:
%    \begin{macrocode}
\ifchilddoc\else
\section{part three}
\input{cdocspt3}
\section{part four}
\input{cdocspt4}
\fi
%    \end{macrocode}

% Process as usual until here:
%    \begin{macrocode}
\fi
%    \end{macrocode}

% End of document body:
%    \begin{macrocode}
\end{document}
%    \end{macrocode}
%\iffalse
%</samplemain>
%\fi
%
% %%%%%%%%%%%%%%%%%%%%%%%%%%%%%%%%%%%%%%
% \paragraph{Chapter Include Files.}
%
% The include files are called |cdocsch1.tex| and |cdocsch2.tex|.
%
%\iffalse
%<*samplechap1|samplechap2>
%\fi

% Optional override for |\version| flag:
%    \begin{macrocode}
%%\providecommand{\version}{final}
%    \end{macrocode}

% Include the main document:
%    \begin{macrocode}
\input{childdoc.def}
\childdocof{cdocsamp}
%    \end{macrocode}

%\iffalse
%</samplechap1|samplechap2>
%\fi
%
%\iffalse
%<*samplechap1>
%\fi
% Some text for chapter 1:
%    \begin{macrocode}
\section{one}
some text in chapter one
%    \end{macrocode}

%\iffalse
%</samplechap1>
%\fi
% Some text for chapter 2:
%\iffalse
%<*samplechap2>
%\fi
%    \begin{macrocode}
\section{two}
more text in chapter two
%    \end{macrocode}

%\iffalse
%</samplechap2>
%\fi
%
% %%%%%%%%%%%%%%%%%%%%%%%%%%%%%%%%%%%%%%
% \paragraph{Part Include Files.}
%
% The include files are called |cdocspt3.tex| and |cdocspt4.tex|.
%
%\iffalse
%<*samplepart3|samplepart4>
%\fi

% Optional override for |\version| flag:
%    \begin{macrocode}
%%\providecommand{\version}{final}
%    \end{macrocode}

% Include the main document:
%    \begin{macrocode}
\input{childdoc.def}
\childdocby{cdocsamp}
%    \end{macrocode}

%\iffalse
%</samplepart3|samplepart4>
%\fi
%
%\iffalse
%<*samplepart3>
%\fi
% Some text for part 3:
%    \begin{macrocode}
some text in part three
%    \end{macrocode}

%\iffalse
%</samplepart3>
%\fi
% Some text for part 4:
%\iffalse
%<*samplepart4>
%\fi
%    \begin{macrocode}
more text in part four
%    \end{macrocode}

%\iffalse
%</samplepart4>
%\fi
%
% %%%%%%%%%%%%%%%%%%%%%%%%%%%%%%%%%%%%%%
% \paragraph{Forwarding for a Complete Draft.}
%
% The following forwarding file |cdocsdrf.tex|
% compiles the main document in draft mode:
%\iffalse
%<*sampledraft>
%\fi
%    \begin{macrocode}
\def\version{draft}
\input{childdoc.def}
\childdocforward{cdocsamp}
%    \end{macrocode}

%\iffalse
%</sampledraft>
%\fi
%
% %%%%%%%%%%%%%%%%%%%%%%%%%%%%%%%%%%%%%%
% \paragraph{Forwarding for Final Version of the Chapters.}
%
% The following forwarding files |cdocsfn1.tex| and |cdocsfn2.tex|
% (with identical content)
% compile the final versions of the child documents
% |cdocsch1.tex| and |cdocsch2.tex|, respectively:
%\iffalse
%<*samplefinal>
%\fi
%    \begin{macrocode}
\def\version{final}
\input{childdoc.def}
\childdocforwardprefix[cdocsamp]{cdocsfn}{cdocsch}
%    \end{macrocode}

%\iffalse
%</samplefinal>
%\fi
%
% %%%%%%%%%%%%%%%%%%%%%%%%%%%%%%%%%%%%%%
% \paragraph{Command Line Processing.}
%
% The following three command lines generate the output files
% |cdocscld|, |cdocscl1| and |cdocscl2|
% which should be identical to
% |cdocsdrf|, |cdocsch1| and |cdocsfn2|, respectively:
% \begin{center}
% \begin{tabular}{l}
% |latex -jobname cdocscld \|\\
% |  "\def\version{draft}\input{childdoc.def}\childdocforward{cdocsamp}"|\\
% |latex -jobname cdocscl1 \|\\
% |  "\input{childdoc.def}\childdocforward[cdocsamp]{cdocsch1}"|\\
% |latex -jobname cdocscl2 \|\\
% |  "\def\version{final}\input{childdoc.def}\childdocforward{cdocsch2}"|
% \end{tabular}
% \end{center}
% Note that the trailing backslash on each first line
% merely continues the input to the second line
% (for convenient cut ant paste).
% Furthermore, the command |latex| can be replaced by any
% of its alternative versions such as |pdflatex|.
%
% %%%%%%%%%%%%%%%%%%%%%%%%%%%%%%%%%%%%%%%%%%%%%%%%%%%%%%%%%%%%%%%%%%%%%%%%%%%%%%
% %%%%%%%%%%%%%%%%%%%%%%%%%%%%%%%%%%%%%%%%%%%%%%%%%%%%%%%%%%%%%%%%%%%%%%%%%%%%%%
% \section{Implementation}
%\iffalse
%<*package>
%\fi
%
% This section describes the definitions file |childdoc.def|.

% The definitions cannot be loaded using |\usepackage| or |\RequirePackage|
% which has a mechanism to prevent loading a style file more than once.
% When loading the definitions by means of |\input|
% multiple instances have to be prevented manually:
%\iffalse
%This code needs to be before the `\ProvidesFile' directive
%which is defined at the beginning of this file.
%Therefore it is also placed there and commented out here.
%</package>
%<*discard>
%\fi
%    \begin{macrocode}
\ifdefined\childdocmain\endinput\fi
%    \end{macrocode}
%\iffalse
%</discard>
%<*package>
%\fi
%
% \macro{\ifchilddoc}
% \macro{\ifchilddocmanual}
% The conditional |\ifchilddoc| tells whether a
% child (true) or main (false) document is being compiled.
% The conditional |\ifchilddocmanual| tells whether
% the |\includeonly| mechanism is used (false) or
% the selection of child files must be performed manually (true).
% The definitions initialise to false:
%    \begin{macrocode}
\newif\ifchilddoc
\newif\ifchilddocmanual
%    \end{macrocode}

% \macro{\childdocname}
% \macro{\childdocjob}
% The macro |\childdocname| stores the name of the main document
% to be compiled. The macro |\childdocjob| stores the name of
% the document on which the \LaTeX{} compiler was originally invoked.
% The content of |\jobname| cannot be compared
% to filenames specified in the source due to different catcodes.
% The following code rescans |\jobname|, stores the result
% in |\childdocname| and saves a copy in |\childdocjob|:
%    \begin{macrocode}
\edef\childdocname{\scantokens\expandafter{\jobname\noexpand}}
\let\childdocjob\childdocname
%    \end{macrocode}

% \macro{\childdocdisable}
% The macro |\childdocdisable| prevents the main file
% from being processed more than once.
% At this stage, the main document command |\childdocmain|
% is assumed to be called once again where it should do nothing.
% Any subsequent call to it should prevent
% a secondary processing of the main document
% It overwrites the forwarding commands
% |\childdocof| and |\childdocforward|
% with empty macros to prevent further inclusions of the main document:
%    \begin{macrocode}
\newcommand{\childdocdisable}
{
  \renewcommand{\childdocmain}[1]{\renewcommand{\childdocmain}[1]{\endinput}}
  \renewcommand{\childdocof}[1]{}
  \renewcommand{\childdocby}[2][]{}
  \renewcommand{\childdocforward}[2][]{}
  \renewcommand{\childdocdisable}{}
}
%    \end{macrocode}

% \macro{\childdocmain}
% The macro |\childdocmain| is to be called at the top of the main file
% with nothing or the main filename (without extension) as argument.
% First, it breaks loops.
% If the argument is not empty and does not match |\childdocname|
% (which is set by the first inclusion of |childdoc.def|),
% |\ifchilddoc| is set to true, |\includeonly| is applied to the child file
% and |\jobname| is set to the main file
% (for proper handling of |.aux| files):
%    \begin{macrocode}
\newcommand{\childdocmain}[1]
{
  \childdocdisable\childdocmain{}
  \if?#1?\else
    \begingroup
      \def\childdoctmp{#1}
      \ifx\childdoctmp\childdocname
        \def\childdoctmp{}
      \else
        \def\childdoctmp
        {
          \childdoctrue
          \includeonly{\childdocname}
          \def\childdocjob{#1}
          \def\jobname{#1}
        }
      \fi
      \expandafter
    \endgroup
    \childdoctmp
  \fi
}
%    \end{macrocode}

% \macro{\childdocof}
% The command |\childdocof| redirects
% compilation to the main file |#1|.
%    \begin{macrocode}
\newcommand{\childdocof}[1]
{
  \childdocdisable
  \childdoctrue
  \includeonly{\childdocname}
  \def\jobname{#1}
  \def\childdocjob{#1}
  \input{#1}
}
%    \end{macrocode}

% \macro{\childdocby}
% The command |\childdocby| ....
%    \begin{macrocode}
\newcommand{\childdocby}[2][]
{
  \childdocdisable
  \childdoctrue
  \childdocmanualtrue
  \if?#1?\else
    \def\jobname{#2}
  \fi
  \def\childdocjob{#2}
  \input{#2}
  \endinput
}
%    \end{macrocode}

% \macro{\childdocforward}
% The command |\childdocforward| redirects
% compilation to the main file or
% (if the optional argument is given) a child file.
% Parameters are set as if the main file
% or a child file starting with |\childdocof| was compiled.
% Then compilation is handed over to the main file:
%    \begin{macrocode}
\newcommand{\childdocforward}[2][]
{
  \begingroup
    \if?#1?
      \def\childdoctmp
      {
        \def\childdocname{#2}
        \def\childdocjob{#2}
        \def\jobname{#2}
        \input{#2}
        \endinput
      }
    \else
      \def\childdoctmp
      {
        \childdocdisable
        \def\childdocname{#2}
        \childdoctrue
        \includeonly{#2}
        \def\childdocjob{#1}
        \def\jobname{#1}
        \input{#1}
        \endinput
      }
    \fi
    \expandafter
  \endgroup
  \childdoctmp
}
%    \end{macrocode}

% \macro{\childdocforwardprefix}
% The command |\childdocforwardprefix| redirects
% compilation to the main or a child file by means of a pattern.
% The prefix |#1| in the current filename is replaced by |#2|
% and the suffix of the current filename is kept
% (it is assumed that the filename does not contain the substring `|~~~|'
% which is used as a delimiter).
% Compilation is handed over to the new file by |\childdocforward|:
%    \begin{macrocode}
\newcommand{\childdocforwardprefix}[3][]
{
  \begingroup
    \def\childdocextract #2##1~~~{\def\childdoctmp{\childdocforward[#1]{#3##1}}}
    \expandafter\childdocextract\childdocname~~~
    \expandafter
  \endgroup
  \childdoctmp
}
%    \end{macrocode}

% \macro{\childdoc}
% The deprecated macro |\childdoc| is a legacy version of |\childdocmain|:
%    \begin{macrocode}
\newcommand{\childdoc}{\childdocmain}
%    \end{macrocode}

% \macro{\childdocredirect}
% The deprecated macro |\childdocredirect| is a legacy version
% of |\childdocforward| and |\childdocforwardprefix|:
%    \begin{macrocode}
\newcommand{\childdocredirect}[2][]
{
  \begingroup
    \if?#1?
      \def\childdoctmp{\childdocforward{#2}}
    \else
      \def\childdoctmp{\childdocforwardprefix{#1}{#2}}
    \fi
    \expandafter
  \endgroup
  \childdoctmp
}
%    \end{macrocode}

%\iffalse
%</package>
%\fi
%
\endinput
|\\
|\childdocmain{|\textit{main}|}|\\
\end{tabular}
\end{center}
%
If |\jobname| does not match the argument \textit{main} of |\childdocmain|,
it is assumed that |\jobname| points to the child file to be compiled.
When using |\childdocmain| with the main file specified as argument,
it suffices to start a child file
with just |\input{|\textit{main}|}|
without loading of the package and using |\childdocof|.
If instead all processing is done
with the appropriate \textsf{childdoc} directives,
the argument of \textit{main} of |\childdocmain| can be empty.

An alternative version of the command line processing described
in \secref{sec:commandline} using the detection mechanism reads:
%
\begin{center}
|... -jobname "|\textit{target}|" "|[\textit{flags}]%
[|\def\jobname{|\textit{dest}|}|]|\input{|\textit{main}|}"|
\end{center}

%%%%%%%%%%%%%%%%%%%%%%%%%%%%%%%%%%%%%%%%%%%%%%%%%%%%%%%%%%%%%%%%%%%%%%%%%%%%%%%%
\subsection{Manual Code}
\label{sec:manual}

In case one cannot be certain whether the definitions file |childdoc.def|
is installed on the target \TeX{} distribution
and one prefers not to ship it,
it is conceivable to paste a few relevant commands into the sources.

To that end, drop all statements |% \iffalse
%
% childdoc.dtx Copyright (C) 2017-2018 Niklas Beisert
%
% This work may be distributed and/or modified under the
% conditions of the LaTeX Project Public License, either version 1.3
% of this license or (at your option) any later version.
% The latest version of this license is in
%   http://www.latex-project.org/lppl.txt
% and version 1.3 or later is part of all distributions of LaTeX
% version 2005/12/01 or later.
%
% This work has the LPPL maintenance status `maintained'.
%
% The Current Maintainer of this work is Niklas Beisert.
%
% This work consists of the files childdoc.dtx and childdoc.ins
% and the derived files childdoc.def and cdocsamp.tex with
% cdocsch1.tex, cdocsch2.tex, cdocsdrf.tex, cdocsfn1.tex, cdocsfn2.tex.
%
%<package>\ifdefined\childdocmain\endinput\fi
%<package>\ProvidesFile{childdoc.def}[2018/12/30 v2.0 child document driver]
%<samplemain>\ProvidesFile{cdocsamp.tex}[2018/12/30 v2.0 sample for childdoc]
%<*driver>
%\ProvidesFile{childdoc.drv}[2018/12/30 v2.0 childdoc reference manual file]
\PassOptionsToClass{10pt,a4paper}{article}
\documentclass{ltxdoc}

\usepackage[margin=35mm]{geometry}
\usepackage{hyperref}
\usepackage{hyperxmp}
\usepackage[usenames]{color}

\hypersetup{colorlinks=true}
\hypersetup{pdfstartview=FitH}
\hypersetup{pdfpagemode=UseNone}
\hypersetup{pdfsource={}}
\hypersetup{pdflang={en-UK}}
\hypersetup{pdfcopyright={Copyright 2017-2018 Niklas Beisert.
  This work may be distributed and/or modified under the
  conditions of the LaTeX Project Public License, either version 1.3
  of this license or (at your option) any later version.}}
\hypersetup{pdflicenseurl={http://www.latex-project.org/lppl.txt}}
\hypersetup{pdfcontactaddress={ETH Zurich, ITP, HIT K,
  Wolfgang-Pauli-Strasse 27}}
\hypersetup{pdfcontactpostcode={8093}}
\hypersetup{pdfcontactcity={Zurich}}
\hypersetup{pdfcontactcountry={Switzerland}}
\hypersetup{pdfcontactemail={nbeisert@itp.phys.ethz.ch}}
\hypersetup{pdfcontacturl={http://people.phys.ethz.ch/\xmptilde nbeisert/}}

\newcommand{\secref}[1]{\hyperref[#1]{section \ref*{#1}}}

\parskip1ex
\parindent0pt
\let\olditemize\itemize
\def\itemize{\olditemize\parskip0pt}

\begin{document}

\title{The \textsf{childdoc} Package}
\hypersetup{pdftitle={The childdoc Package}}
\author{Niklas Beisert\\[2ex]
  Institut f\"ur Theoretische Physik\\
  Eidgen\"ossische Technische Hochschule Z\"urich\\
  Wolfgang-Pauli-Strasse 27, 8093 Z\"urich, Switzerland\\[1ex]
  \href{mailto:nbeisert@itp.phys.ethz.ch}
  {\texttt{nbeisert@itp.phys.ethz.ch}}}
\hypersetup{pdfauthor={Niklas Beisert}}
\hypersetup{pdfsubject={Manual for the LaTeX2e Package childdoc}}
\date{30 December 2018, \textsf{v2.0}}
\maketitle

\begin{abstract}\noindent
\textsf{childdoc} is a \LaTeXe{} package
that enables the direct compilation
of document sections included by |\include|
to individual files.
\end{abstract}

\begingroup
\parskip0ex
\tableofcontents
\endgroup

%%%%%%%%%%%%%%%%%%%%%%%%%%%%%%%%%%%%%%%%%%%%%%%%%%%%%%%%%%%%%%%%%%%%%%%%%%%%%%%%
%%%%%%%%%%%%%%%%%%%%%%%%%%%%%%%%%%%%%%%%%%%%%%%%%%%%%%%%%%%%%%%%%%%%%%%%%%%%%%%%
\section{Introduction}

\LaTeX{} provides a mechanism to structure a large document (such as a book)
into a main file and several child files (containing the chapters)
using the |\include| command.
This mechanism is beneficial for documents
which span hundreds of pages in order to
make the source file(s) more manageable.
Moreover, compilation can be restricted to
selected child files by means of the |\includeonly| command.
The latter feature can be used to reduce the compilation time while editing
(this was significantly more useful in the earlier days of \LaTeX{})
or to generate a smaller document which is easier to navigate.
Another application of |\includeonly| is to generate
documents consisting of selected parts of the complete document.

However, there are a few drawbacks of the plain |\include| mechanism:
\begin{itemize}
\item
The child files cannot be compiled on their own,
they can only be compiled via the main file.
A naive editing environment
(such as a text editor with an option
to have the current file processed by \LaTeX)
may require one to switch to the main file before compiling;
attempting to compile the child file produces errors.
\item
The main file must be modified (each time)
to adjust the |\includeonly| command
to the present needs. This easily leaves the main file in a messy state.
\item
The generated document will always carry the filename
of the main document. This is inconvenient if
several child files are to be compiled and
to be kept for distribution.
\end{itemize}

The present package provides a simple interface
to make child files individually compilable by \LaTeX{}.
Compiling a child file then has the same effect as compiling
the main file with an |\includeonly| command
to select the appropriate child.
Moreover the generated document will carry the name of the child
rather than the main file.
This resolves all three above issues.

This feature is meant to make the editing of books,
thesis documents and lecture notes somewhat more convenient.
However, the package can also be used efficiently for
composing a series of documents (such as exercise sheets)
which are typically distributed individually.
It then assists the author in generating the individual documents
(potentially in different versions)
as well as a document containing the collected series.
Another application is in developing style files
or other kinds of included material
where compilation of the style file could redirect
to a sample or test file.

%%%%%%%%%%%%%%%%%%%%%%%%%%%%%%%%%%%%%%%%%%%%%%%%%%%%%%%%%%%%%%%%%%%%%%%%%%%%%%%%
%%%%%%%%%%%%%%%%%%%%%%%%%%%%%%%%%%%%%%%%%%%%%%%%%%%%%%%%%%%%%%%%%%%%%%%%%%%%%%%%
\section{Usage}

First of all, the package \textsf{childdoc} is \emph{not} a standard
\LaTeXe{} |.sty| style file! Therefore it needs to be invoked in
a non-standard way.

%%%%%%%%%%%%%%%%%%%%%%%%%%%%%%%%%%%%%%%%%%%%%%%%%%%%%%%%%%%%%%%%%%%%%%%%%%%%%%%%
\subsection{Included Files}
\label{sec:include}

%%%%%%%%%%%%%%%%%%%%%%%%%%%%%%%%%%%%%%%%
\DescribeMacro{\childdocmain}
To use the package, add the commands
\begin{center}
\begin{tabular}{l}
|\input{childdoc.def}|\\
|\childdocmain{}|\\
\end{tabular}
\end{center}
at the very top of the main \LaTeX{} file,
in particular \emph{before} the |\documentclass| statement!
The argument of |\childdocmain| should be left empty
(but it must be present).

%%%%%%%%%%%%%%%%%%%%%%%%%%%%%%%%%%%%%%%%
\DescribeMacro{\childdocof}
Furthermore, add the commands
\begin{center}
\begin{tabular}{l}
|\input{childdoc.def}|\\
|\childdocof{|\textit{main}|}|\\
\end{tabular}
\end{center}
at the top of every child file \textit{child}
which is included by |\include{|\textit{child}|}|
from within the main file
(or at least for those files to be compiled individually).
The argument \textit{main} must be the filename of the main file.

There are a couple of
considerations in setting up the main and child documents:

%%%%%%%%%%%%%%%%%%%%%%%%%%%%%%%%%%%%%%%%
\paragraph{Restrictions.}

Please note the following restrictions:
\begin{itemize}
\item
|\childdocmain| must be called with one argument \textit{main}
to ensure compatibility with earlier version of the package.
It must either be empty (|\childdocmain{}|)
or precisely match the filename of the main file in which it is specified.
See \secref{sec:detection} for further information.
\item
The filename \textit{main} must be specified without the |.tex| extension.
\item
The filename \textit{main} is case sensitive
(even in case-insensitive file systems)
due to internal string comparison.
\item
The argument \textit{main} should be fully expanded, it cannot be a macro.
\item
Subdirectories and special characters should be avoided in filenames.
\item
The command |\childdocmain{|\textit{main}|}| must be followed by a whitespace.
It should not be followed immediately by another command
or by a comment mark `|%|'.
This is because the \TeX{} parser reads the token immediately following
the argument of |\childdocmain| and puts it
at the beginning of every child section;
however, a white\-space is ignored.
\end{itemize}

%%%%%%%%%%%%%%%%%%%%%%%%%%%%%%%%%%%%%%%%
\paragraph{Content of Main File.}

It is advisable to place all content in the child files included by |\include|.
Any output contained in the main file will appear in all child documents
unless suppressed manually;
it cannot be suppressed automatically by the |\includeonly| directive
and thus should normally be avoided.
A method to include some content in the main file
by means of conditional processing is described in \secref{sec:conditional}.

%%%%%%%%%%%%%%%%%%%%%%%%%%%%%%%%%%%%%%%%
\paragraph{Page Numbering.}

When only a part of the document is compiled,
the appropriate numbering of pages
(as well as other status parameters)
is determined from the |.aux| files.
The latter contain information from previous passes.
However this information needs to propagate through
all intermediate child documents.
Therefore the page numbering in child documents may well
be inconsistent until the complete document is compiled at least once.

A useful (if unconventional) way to always ensure a consistent
page numbering is to restart the numbering in each child document
and denote the pages by `\textit{child}|.|\textit{page}'
where \textit{child} represents the chapter/section number of the child file.
This can be achieved by the command
|\numberwithin{page}{|\textit{child}|}|
of the \textsf{amsmath} package
where \textit{child} can be |chapter| or |section|
depending on the chosen structuring.
Alternatively, one can modify the macro |\thepage| appropriately
and reset the counter |page| at the start of each child file.

%%%%%%%%%%%%%%%%%%%%%%%%%%%%%%%%%%%%%%%%%%%%%%%%%%%%%%%%%%%%%%%%%%%%%%%%%%%%%%%%
\subsection{Conditional Processing}
\label{sec:conditional}

The package provides a mechanism to compile different versions
of a document. To customise the versions further some conditional processing
can come in handy to distinguish which version is being compiled.
The package provides two macros to describe the compilation context:

%%%%%%%%%%%%%%%%%%%%%%%%%%%%%%%%%%%%%%%%
\DescribeMacro{\ifchilddoc}
The conditional |\ifchilddoc| distinguishes between the compilation of
child documents and the main document:
%
\begin{center}
|\ifchilddoc |\textit{child-code}| |[|\||else |\textit{main-code}]| \||fi|
\end{center}

%%%%%%%%%%%%%%%%%%%%%%%%%%%%%%%%%%%%%%%%
\DescribeMacro{\childdocname}
\DescribeMacro{\childdocjob}
The macro |\childdocname| contains the filename (without extension)
of the main or child file being processed.
Note that |\childdocjob| will always contain the name of the main file.

%%%%%%%%%%%%%%%%%%%%%%%%%%%%%%%%%%%%%%%%
\paragraph{Title Page.}

Conditional processing can be used to include a title or banner page
in the main document when proper precautions are taken.
Importantly, the code in the main file should ensure that the page counter
(as well as other status parameters which are stored in the |.aux| files)
takes the same value after the conditional processing.
Otherwise the page numbers may take divergent values
depending on which part is compiled.

For example, a title page could be declared by:
%
\begin{center}
\begin{tabular}{l}
|\ifchilddoc\||else|\\
|\addtocounter{page}{-1}|\\
\textit{code for title page}\\
|\newpage|\\
|\||fi|
\end{tabular}
\end{center}
%
A banner page for the child documents can be generated by:
%
\begin{center}
\begin{tabular}{l}
|\ifchilddoc|\\
|\addtocounter{page}{-1}|\\
\textit{code for banner page}\\
|\newpage|\\
|\||fi|
\end{tabular}
\end{center}
%
Here one could write a message such as:
\begin{center}
|This is the part \childdocname{} of \childdocjob{}.|
\end{center}

%%%%%%%%%%%%%%%%%%%%%%%%%%%%%%%%%%%%%%%%%%%%%%%%%%%%%%%%%%%%%%%%%%%%%%%%%%%%%%%%
\subsection{Flags}
\label{sec:flags}

The package makes it easy to generate different versions
of the main or child documents.
To this end compilation flags can be defined
and assigned different default values.
They will be particularly useful in conjunction
with the forwarding mechanism described in \secref{sec:forward}.

For example, it may be useful to have a flag |\version|
which can be set to |draft| or |final|.
The document source will contain some conditional code
depending on the value of |\version|.
Suppose further, the flag should default to |final| for the main file
and to |draft| for child files
which is a natural assignment for editing the document.
This is achieved by placing the following code
in the preamble of the main document
(below the |\childdocmain| directive):
%
\begin{center}
\begin{tabular}{l}
|\ifchilddoc|\\
|\providecommand{\version}{draft}|\\
|\||else|\\
|\providecommand{\version}{final}|\\
|\||fi|
\end{tabular}
\end{center}
%
The definition by |\providecommand| makes sure
that previous definitions are not overwritten.
Further statements |\providecommand{\version}{...}|
can thus be added before the above code to override it.

For the main file, one might add a line
(between |\childdocmain| and the above block)
%
\begin{center}
|%\ifchilddoc\||else\providecommand{\version}{draft}\||fi|
\end{center}
%
which can be uncommented to produce a draft version.
Likewise one can add a line to the very top of a child file
(above the |\childdocof{|\textit{main}|}| directive)
%
\begin{center}
|%\providecommand{\version}{final}|
\end{center}
%
which can be uncommented to produce the final version of this child document.

%%%%%%%%%%%%%%%%%%%%%%%%%%%%%%%%%%%%%%%%%%%%%%%%%%%%%%%%%%%%%%%%%%%%%%%%%%%%%%%%
\subsection{Forwarding}
\label{sec:forward}

Different versions of the main or child documents
using compilation flags as described in \secref{sec:flags}
can be (permanently) stored in different files
for convenient compilation, viewing and distribution.
To this end, the package defines a command
to pass on compilation to a different file:

%%%%%%%%%%%%%%%%%%%%%%%%%%%%%%%%%%%%%%%%
\DescribeMacro{\childdocforward}
The command |\childdocforward| redirects processing to
another source file:
%
\begin{center}
\begin{tabular}{l}
|\input{childdoc.def}|\\
|\childdocforward[|\textit{main}|]{|\textit{dest}|}|\\
\end{tabular}
\end{center}
%
The argument \textit{dest} is the destination file
(without extension).
It should be the main file or one of the child files.
Note that further \textsf{childdoc} directives
such as |\childdocof| and |\childdocforward|
in the indicated file will be processed in this form.
The optional argument \textit{main}
passes on directly to the main file \textit{main}
while pretending to compile the child \textit{dest}.
This form behaves as if \textit{dest}
issues |\childdocof{|\textit{main}|}| right away,
and no further \textsf{childdoc} directives will be processed.

%%%%%%%%%%%%%%%%%%%%%%%%%%%%%%%%%%%%%%%%
\DescribeMacro{\...prefix}
In the alternative form |\childdocforwardprefix|,
%
\begin{center}
\begin{tabular}{l}
|\input{childdoc.def}|\\
|\childdocforwardprefix[|\textit{main}|]{|\textit{prefix}|}{|\textit{dest}|}|
\end{tabular}
\end{center}
%
the destination file is determined by a pattern
depending on the current file:
To make this work, the current file must be called
`{\textit{prefix}\hspace{0.2em}\textit{suffix}}'
with \textit{prefix} matching precisely the argument.
Processing is then passed on to the file
`{\textit{dest}\hspace{0.2em}\textit{suffix}}'.
Surely, the same effect is achieved by
directly specifying the
argument `{\textit{dest}\hspace{0.2em}\textit{suffix}}'
in the first form.
However, that requires to set up a different file
for each child. With the alternative form of the command
all these files can have exactly the same content
which simplifies setting them up and maintaining them.

For example, the following file |draft.tex|
with a compilation flag |\version| as described in \secref{sec:flags}
compiles the main document as a draft:
%
\begin{center}
\begin{tabular}{l}
|\def\version{draft}|\\
|\input{childdoc.def}|\\
|\childdocforward{|\textit{main}|}|
\end{tabular}
\end{center}
%
Likewise, the following files |final|\textit{nn}|.tex|
compile the final version of the child document
|child|\textit{nn}|.tex|:
%
\begin{center}
\begin{tabular}{l}
|\def\version{final}|\\
|\input{childdoc.def}|\\
|\childdocforwardprefix{final}{child}|
\end{tabular}
\end{center}
%

Note that when several versions of a main file and/or of each child file
are to be generated, it may be convenient to set up a |Makefile| or
shell script to automatise the process.

%%%%%%%%%%%%%%%%%%%%%%%%%%%%%%%%%%%%%%%%%%%%%%%%%%%%%%%%%%%%%%%%%%%%%%%%%%%%%%%%
\subsection{Command Line Processing}
\label{sec:commandline}

The effect of redirection files can also be achieved by invoking
the \LaTeX{} compiler with a more elaborate command line.
Most conveniently this should be done as part
of a shell script or a |Makefile|.

When using \textsf{childdoc} in the main file, the following
command lines effectively perform a redirection
(note that depending on the shell being used,
backslashes may have to be doubled: `|\|' $\to$ `|\\|'):
%
\begin{center}
|... -jobname "|\textit{target}|" |\\|"|[\textit{flags}]%
|\input{childdoc.def}\childdocforward[|\textit{main}|]{|\textit{dest}|}"|
\end{center}
%
Here \textit{target} is the name of the output file,
\textit{main} is the name of the main file
and \textit{dest} is the name of the main or child file to be processed
(all filenames without extensions).
The optional argument \textit{main} can be omitted
if \textit{main} matches \textit{dest}.
Optionally, compilation \textit{flags} can be defined via |\def| commands.
This command line makes the \TeX{} engine believe
it is compiling the file \textit{target}
whose content is specified as the latter parameter.
The provided code then forwards the processing to
\textit{main} or \textit{dest} as described in \secref{sec:forward}.

%%%%%%%%%%%%%%%%%%%%%%%%%%%%%%%%%%%%%%%%%%%%%%%%%%%%%%%%%%%%%%%%%%%%%%%%%%%%%%%%
\subsection{Include by Input}
\label{sec:input}

Including child documents by |\include| has some restrictions by design.
Most notably, the content of a child document always occupies
its own set of pages; pages cannot be shared between child documents.
Usually, this behaviour makes perfect sense
because each child document contain an essential part of the document.
However, in some situations it may be desirable to compose
a document from a collection of parts
without having mandatory page breaks between then.
For this case, the package
provides a mechanism to include parts
by |\input| which can also be processed individually.
However, by construction this mechanism
requires manual handling of the content to be output.

%%%%%%%%%%%%%%%%%%%%%%%%%%%%%%%%%%%%%%%%
\DescribeMacro{\ifchilddocmanual}
The main file should be prepared as usual, see \secref{sec:include}.
However, the document body must make a distinction
between processing of an individual part and of the main document, e.g.:
%
\begin{center}
\begin{tabular}{l}
|\ifchilddocmanual|\\
|\input{\childdocname}|\\
|\||else|\\
\textit{document body with }|\input{|\textit{part}|}|\\
|\||fi|
\end{tabular}
\end{center}
%
The conditional |\ifchilddocmanual| is true whenever
a part to be included by |\input| is being compiled,
and the name of the part is stored in |\childdocname|.

%%%%%%%%%%%%%%%%%%%%%%%%%%%%%%%%%%%%%%%%
\DescribeMacro{\childdocby}
Each part to be included by |\input| should start with:
%
\begin{center}
\begin{tabular}{l}
|\input{childdoc.def}|\\
|\childdocby{|\textit{main}|}|\\
\end{tabular}
\end{center}
%
The directive |\childdocby| is similar to |\childdocof|
described in \secref{sec:include},
but the subsequent selection of content must be done manually.
To that end, both |\ifchilddoc| and |\ifchilddocmanual|
will be true upon processing of a part,
and the name of the part is stored in |\childdocname|.
Note that |\jobname| will be set to the filename of the current part
so that each part receives an individual |.aux| file
that does not interfere with the |.aux| file(s) of the main document.
This behaviour can be altered by the alternative form
|\childdocby[*]{|\textit{main}|}| (with a non-empty optional argument)
which uses the |.aux| file of the main document
by setting |\jobname| to \textit{main}.

%%%%%%%%%%%%%%%%%%%%%%%%%%%%%%%%%%%%%%%%%%%%%%%%%%%%%%%%%%%%%%%%%%%%%%%%%%%%%%%%
\subsection{Driver Development}
\label{sec:driver}

The \textsf{childdoc} mechanism can also be use for the development
of definition files such as \LaTeX{} styles or classes.
This case differs from the above setup with multiple parts
included by |\include| in that no |\includeonly| should be invoked.
This can be achieved by starting the include file
(before |\ProvidesPackage|) with:
%
\begin{center}
\begin{tabular}{l}
|\input{childdoc.def}|\\
|\childdocforward{|\textit{main}|}|\\
\end{tabular}
\end{center}
%
or alternatively with:
%
\begin{center}
\begin{tabular}{l}
|\input{childdoc.def}|\\
|\childdocby{|\textit{main}|}|\\
\end{tabular}
\end{center}
%
Both forms have slightly different effects as described above.
The main file is prepared as usual, see \secref{sec:include}.

%%%%%%%%%%%%%%%%%%%%%%%%%%%%%%%%%%%%%%%%%%%%%%%%%%%%%%%%%%%%%%%%%%%%%%%%%%%%%%%%
\subsection{Legacy Detection}
\label{sec:detection}

The directive |\childdocmain| in the main file can detect
whether the complete document or merely a child is to be compiled
even without using the directive |\childdocof|.
This method is deprecated because it is less robust
and there is no compelling reason to use it;
it is merely provided for backward compatibility
and it may be removed in future versions.

If the detection mechanism is to be used,
it is mandatory to correctly specify
the filename of the main file as the argument of |\childdocmain|:
%
\begin{center}
\begin{tabular}{l}
|\input{childdoc.def}|\\
|\childdocmain{|\textit{main}|}|\\
\end{tabular}
\end{center}
%
If |\jobname| does not match the argument \textit{main} of |\childdocmain|,
it is assumed that |\jobname| points to the child file to be compiled.
When using |\childdocmain| with the main file specified as argument,
it suffices to start a child file
with just |\input{|\textit{main}|}|
without loading of the package and using |\childdocof|.
If instead all processing is done
with the appropriate \textsf{childdoc} directives,
the argument of \textit{main} of |\childdocmain| can be empty.

An alternative version of the command line processing described
in \secref{sec:commandline} using the detection mechanism reads:
%
\begin{center}
|... -jobname "|\textit{target}|" "|[\textit{flags}]%
[|\def\jobname{|\textit{dest}|}|]|\input{|\textit{main}|}"|
\end{center}

%%%%%%%%%%%%%%%%%%%%%%%%%%%%%%%%%%%%%%%%%%%%%%%%%%%%%%%%%%%%%%%%%%%%%%%%%%%%%%%%
\subsection{Manual Code}
\label{sec:manual}

In case one cannot be certain whether the definitions file |childdoc.def|
is installed on the target \TeX{} distribution
and one prefers not to ship it,
it is conceivable to paste a few relevant commands into the sources.

To that end, drop all statements |\input{childdoc.def}|
and perform the replacements as outlined below.
Instead of |\childdocmain{|\textit{main}|}| add the following code
to the top of the main file:
%
\begin{center}
\begin{tabular}{l}
|\||ifdefined\childdocname\endinput\||fi\newif\ifchilddoc|\\
|\edef\childdocname{\scantokens\expandafter{\jobname\noexpand}}|\\
|\def\childdocmain{|\textit{main}|}\||ifx\childdocmain\childdocname\||else|\\
|\childdoctrue\includeonly{\childdocname}\let\jobname\childdocmain\||fi|\\
\end{tabular}
\end{center}
%
Instead of |\childdocof{|\textit{main}|}| just include the main file
at the top of each child file:
%
\begin{center}
|\input{|\textit{main}|}|
\end{center}
%
A simple redirection |\childdocforward{|\textit{dest}|}| is achieved by:
%
\begin{center}
|\def\jobname{|\textit{dest}|}\input{\jobname}|
\end{center}
%
The redirection with prefix
|\childdocforwardprefix[|\textit{prefix}|]{|\textit{dest}|}|
is accomplished by:
%
\begin{center}
\begin{tabular}{l}
|{\edef\jobname{\scantokens\expandafter{\jobname\noexpand}}|\\
|\def\redirectjob |\textit{prefix}|#1~~~{\gdef\jobname{|\textit{dest}|#1}}|\\
|\expandafter\redirectjob\jobname~~~}\input{\jobname}|
\end{tabular}
\end{center}

In an alternative approach,
child documents can be compiled by a specific command line
without additional code or specific definitions:
%
\begin{center}
|... -jobname "|\textit{target}|" "|[\textit{flags}]%
|\includeonly{|\textit{dest}|}\input{|\textit{main}|}"|
\end{center}
%

%%%%%%%%%%%%%%%%%%%%%%%%%%%%%%%%%%%%%%%%%%%%%%%%%%%%%%%%%%%%%%%%%%%%%%%%%%%%%%%%
%%%%%%%%%%%%%%%%%%%%%%%%%%%%%%%%%%%%%%%%%%%%%%%%%%%%%%%%%%%%%%%%%%%%%%%%%%%%%%%%
\section{Information}

%%%%%%%%%%%%%%%%%%%%%%%%%%%%%%%%%%%%%%%%%%%%%%%%%%%%%%%%%%%%%%%%%%%%%%%%%%%%%%%%
\subsection{Copyright}

Copyright \copyright{} 2017--2018 Niklas Beisert

This work may be distributed and/or modified under the
conditions of the \LaTeX{} Project Public License, either version 1.3
of this license or (at your option) any later version.
The latest version of this license is in
  \url{http://www.latex-project.org/lppl.txt}
and version 1.3 or later is part of all distributions of \LaTeX{}
version 2005/12/01 or later.

This work has the LPPL maintenance status `maintained'.

The Current Maintainer of this work is Niklas Beisert.

This work consists of the files |README.txt|, |childdoc.ins| and |childdoc.dtx|
as well as the derived files |childdoc.def|, |cdocsamp.tex|
with |cdocsch1.tex|, |cdocsch2.tex|, |cdocspt3.tex|, |cdocspt4.tex|,
|cdocsdrf.tex|, |cdocsfn1.tex|, |cdocsfn2.tex|
as well as |childdoc.pdf|.

%%%%%%%%%%%%%%%%%%%%%%%%%%%%%%%%%%%%%%%%%%%%%%%%%%%%%%%%%%%%%%%%%%%%%%%%%%%%%%%%
\subsection{Files and Installation}

The package consists of the files:
%
\begin{center}
\begin{tabular}{ll}
    |README.txt|   & readme file \\
    |childdoc.ins| & installation file \\
    |childdoc.dtx| & source file \\
    |childdoc.def| & definition file \\
    |cdocsamp.tex| & sample main file \\
    |cdocsch1.tex| & sample include file \\
    |cdocsch2.tex| & sample include file \\
    |cdocspt3.tex| & sample part file \\
    |cdocspt4.tex| & sample part file \\
    |cdocsdrf.tex| & sample redirection file \\
    |cdocsfn1.tex| & sample redirection file \\
    |cdocsfn2.tex| & sample redirection file \\
    |childdoc.pdf| & manual
\end{tabular}
\end{center}
%
The distribution consists of the files
|README.txt|, |childdoc.ins| and |childdoc.dtx|.
%
\begin{itemize}
\item
Run (pdf)\LaTeX{} on |childdoc.dtx|
to compile the manual |childdoc.pdf| (this file).
\item
Run \LaTeX{} on |childdoc.ins| to create the definitions file |childdoc.def|
and the sample |cdocsamp.tex| with include files
|cdocsch1.tex|, |cdocsch2.tex|, |cdocspt3.tex|, |cdocspt4.tex|,
|cdocsdrf.tex|, |cdocsfn1.tex|, |cdocsfn2.tex|.
Then copy the file |childdoc.def| to an appropriate directory of your \LaTeX{}
distribution, e.g.\ \textit{texmf-root}|/tex/latex/childdoc|.
\end{itemize}

%%%%%%%%%%%%%%%%%%%%%%%%%%%%%%%%%%%%%%%%%%%%%%%%%%%%%%%%%%%%%%%%%%%%%%%%%%%%%%%%
\subsection{Related CTAN Packages}

There are several other packages which offer a similar functionality:
%
\begin{itemize}
\item
The packages
\href{http://ctan.org/pkg/docmute}{\textsf{docmute}},
\href{http://ctan.org/pkg/includex}{\textsf{includex}} and
\href{http://ctan.org/pkg/standalone}{\textsf{standalone}}
provide commands to include only the document body of
a child file thus allowing both files to be compiled individually.
\item
The packages \href{http://ctan.org/pkg/subdocs}{\textsf{subdocs}}
and \href{http://ctan.org/pkg/subfiles}{\textsf{subfiles}}
provide structures in which the main and child documents can be
encapsulated and allowing them to be compiled individually.
The inclusion mechanism is different from the conventional |\include|.
\item
The package \href{http://ctan.org/pkg/combine}{\textsf{combine}}
is an elaborate solution to combine several documents into one.
\end{itemize}
%
See also the CTAN topic \href{http://ctan.org/topic/subdocs}{\textsf{subdocs}}
for further related packages.
The present package differs from the above solutions in that
a document structure constructed with the conventional |\include| mechanism
just needs two extra commands at the top of every file
such that all constituent files can be compiled individually.

%%%%%%%%%%%%%%%%%%%%%%%%%%%%%%%%%%%%%%%%%%%%%%%%%%%%%%%%%%%%%%%%%%%%%%%%%%%%%%%%
%\subsection{Feature Suggestions}
%
%The following is a list of features which may be useful for future
%versions of this package:
%%
%\begin{itemize}
%\item
%\ldots
%\end{itemize}

%%%%%%%%%%%%%%%%%%%%%%%%%%%%%%%%%%%%%%%%%%%%%%%%%%%%%%%%%%%%%%%%%%%%%%%%%%%%%%%%
\subsection{Revision History}

%%%%%%%%%%%%%%%%%%%%%%%%%%%%%%%%%%%%%%%%
\paragraph{v2.0:} 2018/12/30

\begin{itemize}
\item
immediate forward processing
\item
added |\childdocby| mechanism
\item
manual restructured
\end{itemize}

%%%%%%%%%%%%%%%%%%%%%%%%%%%%%%%%%%%%%%%%
\paragraph{v1.6:} 2018/01/17

\begin{itemize}
\item
application for development of include files
\item
corrections to manual
\end{itemize}

%%%%%%%%%%%%%%%%%%%%%%%%%%%%%%%%%%%%%%%%
\paragraph{v1.5:} 2017/05/21

\begin{itemize}
\item
more complete structuring introduced
\item
|\childdocof| introduced
\item
|\childdoc| renamed to |\childdocmain|
\item
|\childredirect| renamed to |\childdocforward| and |\childdocforwardprefix|
and functionality expanded
\end{itemize}

%%%%%%%%%%%%%%%%%%%%%%%%%%%%%%%%%%%%%%%%
\paragraph{v1.0:} 2017/04/27

\begin{itemize}
\item
manual and install package
\item
first version published on CTAN
\end{itemize}

%%%%%%%%%%%%%%%%%%%%%%%%%%%%%%%%%%%%%%%%
\paragraph{v0.6:} 2017/04/26

\begin{itemize}
\item
redirection mechanism added
\end{itemize}

%%%%%%%%%%%%%%%%%%%%%%%%%%%%%%%%%%%%%%%%
\paragraph{v0.5:} 2017/04/26

\begin{itemize}
\item
functionality in definition file
\end{itemize}


%%%%%%%%%%%%%%%%%%%%%%%%%%%%%%%%%%%%%%%%%%%%%%%%%%%%%%%%%%%%%%%%%%%%%%%%%%%%%%%%
%%%%%%%%%%%%%%%%%%%%%%%%%%%%%%%%%%%%%%%%%%%%%%%%%%%%%%%%%%%%%%%%%%%%%%%%%%%%%%%%
%%%%%%%%%%%%%%%%%%%%%%%%%%%%%%%%%%%%%%%%%%%%%%%%%%%%%%%%%%%%%%%%%%%%%%%%%%%%%%%%
\appendix

\settowidth\MacroIndent{\rmfamily\scriptsize 000\ }

 \DocInput{childdoc.dtx}

\end{document}
%</driver>
% \fi
%
% %%%%%%%%%%%%%%%%%%%%%%%%%%%%%%%%%%%%%%%%%%%%%%%%%%%%%%%%%%%%%%%%%%%%%%%%%%%%%%
% %%%%%%%%%%%%%%%%%%%%%%%%%%%%%%%%%%%%%%%%%%%%%%%%%%%%%%%%%%%%%%%%%%%%%%%%%%%%%%
% \section{Sample}
%\iffalse
%<*samplemain>
%\fi
%
% The following presents a sample document
% with two chapters, two parts, a title page,
% a compile flag as well as three forwarding files to set the flag.
% It consists of eight |.tex| files:
% \begin{center}
% \begin{tabular}{ll}
% |cdocsamp.tex|&main file\\
% |cdocsch1.tex|&include file for chapter 1\\
% |cdocsch2.tex|&include file for chapter 2\\
% |cdocspt3.tex|&include file for part 3\\
% |cdocspt4.tex|&include file for part 4\\
% |cdocsdrf.tex|&forwarding file for main file in draft mode\\
% |cdocsfi1.tex|&forwarding file for final version of chapter 1\\
% |cdocsfi2.tex|&forwarding file for final version of chapter 2\\
% \end{tabular}
% \end{center}
% Each of the eight files can be compiled directly by the \LaTeX{} compiler.
%
% %%%%%%%%%%%%%%%%%%%%%%%%%%%%%%%%%%%%%%
% \paragraph{Main File.}
%
% The main file is called |cdocsamp.tex|.
%
% Load the \textsf{childdoc} definitions and
% declare the filename for the main document:
%    \begin{macrocode}
\input{childdoc.def}
\childdocmain{}
%    \end{macrocode}

% Optional override for |\version| flag:
%    \begin{macrocode}
%%\ifchilddoc\else\providecommand{\version}{draft}\fi
%    \end{macrocode}

% Define the default values for the |\version| flag
% (|final| for the main file and |draft| for childs):
%    \begin{macrocode}
\ifchilddoc
\providecommand{\version}{draft}
\else
\providecommand{\version}{final}
\fi
%    \end{macrocode}

% Load the standard document class:
%    \begin{macrocode}
\documentclass[12pt]{article}
%    \end{macrocode}

% Start the document body:
%    \begin{macrocode}
\begin{document}
%    \end{macrocode}

% Declare a title page.
% Print title, part of document being processed and version flag:
%    \begin{macrocode}
\addtocounter{page}{-1}
\begin{center}
{\LARGE\bfseries{}childdoc example\par}
\vspace{1cm}
\ifchilddoc
\ifchilddocmanual part\else chapter\fi:
`\childdocname' of `\childdocjob'\par
\else
main document: `\childdocjob'\par
\fi
version: \version\par
\end{center}
\newpage
%    \end{macrocode}

% Manually include selected file,
% otherwise process as usual:
%    \begin{macrocode}
\ifchilddocmanual
\section*{part `\childdocname'}
\input{\childdocname}
\else
%    \end{macrocode}

% Include the two chapters:
%    \begin{macrocode}
\include{cdocsch1}
\include{cdocsch2}
%    \end{macrocode}

% Include the two parts unless only chapters should be displayed:
%    \begin{macrocode}
\ifchilddoc\else
\section{part three}
\input{cdocspt3}
\section{part four}
\input{cdocspt4}
\fi
%    \end{macrocode}

% Process as usual until here:
%    \begin{macrocode}
\fi
%    \end{macrocode}

% End of document body:
%    \begin{macrocode}
\end{document}
%    \end{macrocode}
%\iffalse
%</samplemain>
%\fi
%
% %%%%%%%%%%%%%%%%%%%%%%%%%%%%%%%%%%%%%%
% \paragraph{Chapter Include Files.}
%
% The include files are called |cdocsch1.tex| and |cdocsch2.tex|.
%
%\iffalse
%<*samplechap1|samplechap2>
%\fi

% Optional override for |\version| flag:
%    \begin{macrocode}
%%\providecommand{\version}{final}
%    \end{macrocode}

% Include the main document:
%    \begin{macrocode}
\input{childdoc.def}
\childdocof{cdocsamp}
%    \end{macrocode}

%\iffalse
%</samplechap1|samplechap2>
%\fi
%
%\iffalse
%<*samplechap1>
%\fi
% Some text for chapter 1:
%    \begin{macrocode}
\section{one}
some text in chapter one
%    \end{macrocode}

%\iffalse
%</samplechap1>
%\fi
% Some text for chapter 2:
%\iffalse
%<*samplechap2>
%\fi
%    \begin{macrocode}
\section{two}
more text in chapter two
%    \end{macrocode}

%\iffalse
%</samplechap2>
%\fi
%
% %%%%%%%%%%%%%%%%%%%%%%%%%%%%%%%%%%%%%%
% \paragraph{Part Include Files.}
%
% The include files are called |cdocspt3.tex| and |cdocspt4.tex|.
%
%\iffalse
%<*samplepart3|samplepart4>
%\fi

% Optional override for |\version| flag:
%    \begin{macrocode}
%%\providecommand{\version}{final}
%    \end{macrocode}

% Include the main document:
%    \begin{macrocode}
\input{childdoc.def}
\childdocby{cdocsamp}
%    \end{macrocode}

%\iffalse
%</samplepart3|samplepart4>
%\fi
%
%\iffalse
%<*samplepart3>
%\fi
% Some text for part 3:
%    \begin{macrocode}
some text in part three
%    \end{macrocode}

%\iffalse
%</samplepart3>
%\fi
% Some text for part 4:
%\iffalse
%<*samplepart4>
%\fi
%    \begin{macrocode}
more text in part four
%    \end{macrocode}

%\iffalse
%</samplepart4>
%\fi
%
% %%%%%%%%%%%%%%%%%%%%%%%%%%%%%%%%%%%%%%
% \paragraph{Forwarding for a Complete Draft.}
%
% The following forwarding file |cdocsdrf.tex|
% compiles the main document in draft mode:
%\iffalse
%<*sampledraft>
%\fi
%    \begin{macrocode}
\def\version{draft}
\input{childdoc.def}
\childdocforward{cdocsamp}
%    \end{macrocode}

%\iffalse
%</sampledraft>
%\fi
%
% %%%%%%%%%%%%%%%%%%%%%%%%%%%%%%%%%%%%%%
% \paragraph{Forwarding for Final Version of the Chapters.}
%
% The following forwarding files |cdocsfn1.tex| and |cdocsfn2.tex|
% (with identical content)
% compile the final versions of the child documents
% |cdocsch1.tex| and |cdocsch2.tex|, respectively:
%\iffalse
%<*samplefinal>
%\fi
%    \begin{macrocode}
\def\version{final}
\input{childdoc.def}
\childdocforwardprefix[cdocsamp]{cdocsfn}{cdocsch}
%    \end{macrocode}

%\iffalse
%</samplefinal>
%\fi
%
% %%%%%%%%%%%%%%%%%%%%%%%%%%%%%%%%%%%%%%
% \paragraph{Command Line Processing.}
%
% The following three command lines generate the output files
% |cdocscld|, |cdocscl1| and |cdocscl2|
% which should be identical to
% |cdocsdrf|, |cdocsch1| and |cdocsfn2|, respectively:
% \begin{center}
% \begin{tabular}{l}
% |latex -jobname cdocscld \|\\
% |  "\def\version{draft}\input{childdoc.def}\childdocforward{cdocsamp}"|\\
% |latex -jobname cdocscl1 \|\\
% |  "\input{childdoc.def}\childdocforward[cdocsamp]{cdocsch1}"|\\
% |latex -jobname cdocscl2 \|\\
% |  "\def\version{final}\input{childdoc.def}\childdocforward{cdocsch2}"|
% \end{tabular}
% \end{center}
% Note that the trailing backslash on each first line
% merely continues the input to the second line
% (for convenient cut ant paste).
% Furthermore, the command |latex| can be replaced by any
% of its alternative versions such as |pdflatex|.
%
% %%%%%%%%%%%%%%%%%%%%%%%%%%%%%%%%%%%%%%%%%%%%%%%%%%%%%%%%%%%%%%%%%%%%%%%%%%%%%%
% %%%%%%%%%%%%%%%%%%%%%%%%%%%%%%%%%%%%%%%%%%%%%%%%%%%%%%%%%%%%%%%%%%%%%%%%%%%%%%
% \section{Implementation}
%\iffalse
%<*package>
%\fi
%
% This section describes the definitions file |childdoc.def|.

% The definitions cannot be loaded using |\usepackage| or |\RequirePackage|
% which has a mechanism to prevent loading a style file more than once.
% When loading the definitions by means of |\input|
% multiple instances have to be prevented manually:
%\iffalse
%This code needs to be before the `\ProvidesFile' directive
%which is defined at the beginning of this file.
%Therefore it is also placed there and commented out here.
%</package>
%<*discard>
%\fi
%    \begin{macrocode}
\ifdefined\childdocmain\endinput\fi
%    \end{macrocode}
%\iffalse
%</discard>
%<*package>
%\fi
%
% \macro{\ifchilddoc}
% \macro{\ifchilddocmanual}
% The conditional |\ifchilddoc| tells whether a
% child (true) or main (false) document is being compiled.
% The conditional |\ifchilddocmanual| tells whether
% the |\includeonly| mechanism is used (false) or
% the selection of child files must be performed manually (true).
% The definitions initialise to false:
%    \begin{macrocode}
\newif\ifchilddoc
\newif\ifchilddocmanual
%    \end{macrocode}

% \macro{\childdocname}
% \macro{\childdocjob}
% The macro |\childdocname| stores the name of the main document
% to be compiled. The macro |\childdocjob| stores the name of
% the document on which the \LaTeX{} compiler was originally invoked.
% The content of |\jobname| cannot be compared
% to filenames specified in the source due to different catcodes.
% The following code rescans |\jobname|, stores the result
% in |\childdocname| and saves a copy in |\childdocjob|:
%    \begin{macrocode}
\edef\childdocname{\scantokens\expandafter{\jobname\noexpand}}
\let\childdocjob\childdocname
%    \end{macrocode}

% \macro{\childdocdisable}
% The macro |\childdocdisable| prevents the main file
% from being processed more than once.
% At this stage, the main document command |\childdocmain|
% is assumed to be called once again where it should do nothing.
% Any subsequent call to it should prevent
% a secondary processing of the main document
% It overwrites the forwarding commands
% |\childdocof| and |\childdocforward|
% with empty macros to prevent further inclusions of the main document:
%    \begin{macrocode}
\newcommand{\childdocdisable}
{
  \renewcommand{\childdocmain}[1]{\renewcommand{\childdocmain}[1]{\endinput}}
  \renewcommand{\childdocof}[1]{}
  \renewcommand{\childdocby}[2][]{}
  \renewcommand{\childdocforward}[2][]{}
  \renewcommand{\childdocdisable}{}
}
%    \end{macrocode}

% \macro{\childdocmain}
% The macro |\childdocmain| is to be called at the top of the main file
% with nothing or the main filename (without extension) as argument.
% First, it breaks loops.
% If the argument is not empty and does not match |\childdocname|
% (which is set by the first inclusion of |childdoc.def|),
% |\ifchilddoc| is set to true, |\includeonly| is applied to the child file
% and |\jobname| is set to the main file
% (for proper handling of |.aux| files):
%    \begin{macrocode}
\newcommand{\childdocmain}[1]
{
  \childdocdisable\childdocmain{}
  \if?#1?\else
    \begingroup
      \def\childdoctmp{#1}
      \ifx\childdoctmp\childdocname
        \def\childdoctmp{}
      \else
        \def\childdoctmp
        {
          \childdoctrue
          \includeonly{\childdocname}
          \def\childdocjob{#1}
          \def\jobname{#1}
        }
      \fi
      \expandafter
    \endgroup
    \childdoctmp
  \fi
}
%    \end{macrocode}

% \macro{\childdocof}
% The command |\childdocof| redirects
% compilation to the main file |#1|.
%    \begin{macrocode}
\newcommand{\childdocof}[1]
{
  \childdocdisable
  \childdoctrue
  \includeonly{\childdocname}
  \def\jobname{#1}
  \def\childdocjob{#1}
  \input{#1}
}
%    \end{macrocode}

% \macro{\childdocby}
% The command |\childdocby| ....
%    \begin{macrocode}
\newcommand{\childdocby}[2][]
{
  \childdocdisable
  \childdoctrue
  \childdocmanualtrue
  \if?#1?\else
    \def\jobname{#2}
  \fi
  \def\childdocjob{#2}
  \input{#2}
  \endinput
}
%    \end{macrocode}

% \macro{\childdocforward}
% The command |\childdocforward| redirects
% compilation to the main file or
% (if the optional argument is given) a child file.
% Parameters are set as if the main file
% or a child file starting with |\childdocof| was compiled.
% Then compilation is handed over to the main file:
%    \begin{macrocode}
\newcommand{\childdocforward}[2][]
{
  \begingroup
    \if?#1?
      \def\childdoctmp
      {
        \def\childdocname{#2}
        \def\childdocjob{#2}
        \def\jobname{#2}
        \input{#2}
        \endinput
      }
    \else
      \def\childdoctmp
      {
        \childdocdisable
        \def\childdocname{#2}
        \childdoctrue
        \includeonly{#2}
        \def\childdocjob{#1}
        \def\jobname{#1}
        \input{#1}
        \endinput
      }
    \fi
    \expandafter
  \endgroup
  \childdoctmp
}
%    \end{macrocode}

% \macro{\childdocforwardprefix}
% The command |\childdocforwardprefix| redirects
% compilation to the main or a child file by means of a pattern.
% The prefix |#1| in the current filename is replaced by |#2|
% and the suffix of the current filename is kept
% (it is assumed that the filename does not contain the substring `|~~~|'
% which is used as a delimiter).
% Compilation is handed over to the new file by |\childdocforward|:
%    \begin{macrocode}
\newcommand{\childdocforwardprefix}[3][]
{
  \begingroup
    \def\childdocextract #2##1~~~{\def\childdoctmp{\childdocforward[#1]{#3##1}}}
    \expandafter\childdocextract\childdocname~~~
    \expandafter
  \endgroup
  \childdoctmp
}
%    \end{macrocode}

% \macro{\childdoc}
% The deprecated macro |\childdoc| is a legacy version of |\childdocmain|:
%    \begin{macrocode}
\newcommand{\childdoc}{\childdocmain}
%    \end{macrocode}

% \macro{\childdocredirect}
% The deprecated macro |\childdocredirect| is a legacy version
% of |\childdocforward| and |\childdocforwardprefix|:
%    \begin{macrocode}
\newcommand{\childdocredirect}[2][]
{
  \begingroup
    \if?#1?
      \def\childdoctmp{\childdocforward{#2}}
    \else
      \def\childdoctmp{\childdocforwardprefix{#1}{#2}}
    \fi
    \expandafter
  \endgroup
  \childdoctmp
}
%    \end{macrocode}

%\iffalse
%</package>
%\fi
%
\endinput
|
and perform the replacements as outlined below.
Instead of |\childdocmain{|\textit{main}|}| add the following code
to the top of the main file:
%
\begin{center}
\begin{tabular}{l}
|\||ifdefined\childdocname\endinput\||fi\newif\ifchilddoc|\\
|\edef\childdocname{\scantokens\expandafter{\jobname\noexpand}}|\\
|\def\childdocmain{|\textit{main}|}\||ifx\childdocmain\childdocname\||else|\\
|\childdoctrue\includeonly{\childdocname}\let\jobname\childdocmain\||fi|\\
\end{tabular}
\end{center}
%
Instead of |\childdocof{|\textit{main}|}| just include the main file
at the top of each child file:
%
\begin{center}
|\input{|\textit{main}|}|
\end{center}
%
A simple redirection |\childdocforward{|\textit{dest}|}| is achieved by:
%
\begin{center}
|\def\jobname{|\textit{dest}|}\input{\jobname}|
\end{center}
%
The redirection with prefix
|\childdocforwardprefix[|\textit{prefix}|]{|\textit{dest}|}|
is accomplished by:
%
\begin{center}
\begin{tabular}{l}
|{\edef\jobname{\scantokens\expandafter{\jobname\noexpand}}|\\
|\def\redirectjob |\textit{prefix}|#1~~~{\gdef\jobname{|\textit{dest}|#1}}|\\
|\expandafter\redirectjob\jobname~~~}\input{\jobname}|
\end{tabular}
\end{center}

In an alternative approach,
child documents can be compiled by a specific command line
without additional code or specific definitions:
%
\begin{center}
|... -jobname "|\textit{target}|" "|[\textit{flags}]%
|\includeonly{|\textit{dest}|}\input{|\textit{main}|}"|
\end{center}
%

%%%%%%%%%%%%%%%%%%%%%%%%%%%%%%%%%%%%%%%%%%%%%%%%%%%%%%%%%%%%%%%%%%%%%%%%%%%%%%%%
%%%%%%%%%%%%%%%%%%%%%%%%%%%%%%%%%%%%%%%%%%%%%%%%%%%%%%%%%%%%%%%%%%%%%%%%%%%%%%%%
\section{Information}

%%%%%%%%%%%%%%%%%%%%%%%%%%%%%%%%%%%%%%%%%%%%%%%%%%%%%%%%%%%%%%%%%%%%%%%%%%%%%%%%
\subsection{Copyright}

Copyright \copyright{} 2017--2018 Niklas Beisert

This work may be distributed and/or modified under the
conditions of the \LaTeX{} Project Public License, either version 1.3
of this license or (at your option) any later version.
The latest version of this license is in
  \url{http://www.latex-project.org/lppl.txt}
and version 1.3 or later is part of all distributions of \LaTeX{}
version 2005/12/01 or later.

This work has the LPPL maintenance status `maintained'.

The Current Maintainer of this work is Niklas Beisert.

This work consists of the files |README.txt|, |childdoc.ins| and |childdoc.dtx|
as well as the derived files |childdoc.def|, |cdocsamp.tex|
with |cdocsch1.tex|, |cdocsch2.tex|, |cdocspt3.tex|, |cdocspt4.tex|,
|cdocsdrf.tex|, |cdocsfn1.tex|, |cdocsfn2.tex|
as well as |childdoc.pdf|.

%%%%%%%%%%%%%%%%%%%%%%%%%%%%%%%%%%%%%%%%%%%%%%%%%%%%%%%%%%%%%%%%%%%%%%%%%%%%%%%%
\subsection{Files and Installation}

The package consists of the files:
%
\begin{center}
\begin{tabular}{ll}
    |README.txt|   & readme file \\
    |childdoc.ins| & installation file \\
    |childdoc.dtx| & source file \\
    |childdoc.def| & definition file \\
    |cdocsamp.tex| & sample main file \\
    |cdocsch1.tex| & sample include file \\
    |cdocsch2.tex| & sample include file \\
    |cdocspt3.tex| & sample part file \\
    |cdocspt4.tex| & sample part file \\
    |cdocsdrf.tex| & sample redirection file \\
    |cdocsfn1.tex| & sample redirection file \\
    |cdocsfn2.tex| & sample redirection file \\
    |childdoc.pdf| & manual
\end{tabular}
\end{center}
%
The distribution consists of the files
|README.txt|, |childdoc.ins| and |childdoc.dtx|.
%
\begin{itemize}
\item
Run (pdf)\LaTeX{} on |childdoc.dtx|
to compile the manual |childdoc.pdf| (this file).
\item
Run \LaTeX{} on |childdoc.ins| to create the definitions file |childdoc.def|
and the sample |cdocsamp.tex| with include files
|cdocsch1.tex|, |cdocsch2.tex|, |cdocspt3.tex|, |cdocspt4.tex|,
|cdocsdrf.tex|, |cdocsfn1.tex|, |cdocsfn2.tex|.
Then copy the file |childdoc.def| to an appropriate directory of your \LaTeX{}
distribution, e.g.\ \textit{texmf-root}|/tex/latex/childdoc|.
\end{itemize}

%%%%%%%%%%%%%%%%%%%%%%%%%%%%%%%%%%%%%%%%%%%%%%%%%%%%%%%%%%%%%%%%%%%%%%%%%%%%%%%%
\subsection{Related CTAN Packages}

There are several other packages which offer a similar functionality:
%
\begin{itemize}
\item
The packages
\href{http://ctan.org/pkg/docmute}{\textsf{docmute}},
\href{http://ctan.org/pkg/includex}{\textsf{includex}} and
\href{http://ctan.org/pkg/standalone}{\textsf{standalone}}
provide commands to include only the document body of
a child file thus allowing both files to be compiled individually.
\item
The packages \href{http://ctan.org/pkg/subdocs}{\textsf{subdocs}}
and \href{http://ctan.org/pkg/subfiles}{\textsf{subfiles}}
provide structures in which the main and child documents can be
encapsulated and allowing them to be compiled individually.
The inclusion mechanism is different from the conventional |\include|.
\item
The package \href{http://ctan.org/pkg/combine}{\textsf{combine}}
is an elaborate solution to combine several documents into one.
\end{itemize}
%
See also the CTAN topic \href{http://ctan.org/topic/subdocs}{\textsf{subdocs}}
for further related packages.
The present package differs from the above solutions in that
a document structure constructed with the conventional |\include| mechanism
just needs two extra commands at the top of every file
such that all constituent files can be compiled individually.

%%%%%%%%%%%%%%%%%%%%%%%%%%%%%%%%%%%%%%%%%%%%%%%%%%%%%%%%%%%%%%%%%%%%%%%%%%%%%%%%
%\subsection{Feature Suggestions}
%
%The following is a list of features which may be useful for future
%versions of this package:
%%
%\begin{itemize}
%\item
%\ldots
%\end{itemize}

%%%%%%%%%%%%%%%%%%%%%%%%%%%%%%%%%%%%%%%%%%%%%%%%%%%%%%%%%%%%%%%%%%%%%%%%%%%%%%%%
\subsection{Revision History}

%%%%%%%%%%%%%%%%%%%%%%%%%%%%%%%%%%%%%%%%
\paragraph{v2.0:} 2018/12/30

\begin{itemize}
\item
immediate forward processing
\item
added |\childdocby| mechanism
\item
manual restructured
\end{itemize}

%%%%%%%%%%%%%%%%%%%%%%%%%%%%%%%%%%%%%%%%
\paragraph{v1.6:} 2018/01/17

\begin{itemize}
\item
application for development of include files
\item
corrections to manual
\end{itemize}

%%%%%%%%%%%%%%%%%%%%%%%%%%%%%%%%%%%%%%%%
\paragraph{v1.5:} 2017/05/21

\begin{itemize}
\item
more complete structuring introduced
\item
|\childdocof| introduced
\item
|\childdoc| renamed to |\childdocmain|
\item
|\childredirect| renamed to |\childdocforward| and |\childdocforwardprefix|
and functionality expanded
\end{itemize}

%%%%%%%%%%%%%%%%%%%%%%%%%%%%%%%%%%%%%%%%
\paragraph{v1.0:} 2017/04/27

\begin{itemize}
\item
manual and install package
\item
first version published on CTAN
\end{itemize}

%%%%%%%%%%%%%%%%%%%%%%%%%%%%%%%%%%%%%%%%
\paragraph{v0.6:} 2017/04/26

\begin{itemize}
\item
redirection mechanism added
\end{itemize}

%%%%%%%%%%%%%%%%%%%%%%%%%%%%%%%%%%%%%%%%
\paragraph{v0.5:} 2017/04/26

\begin{itemize}
\item
functionality in definition file
\end{itemize}


%%%%%%%%%%%%%%%%%%%%%%%%%%%%%%%%%%%%%%%%%%%%%%%%%%%%%%%%%%%%%%%%%%%%%%%%%%%%%%%%
%%%%%%%%%%%%%%%%%%%%%%%%%%%%%%%%%%%%%%%%%%%%%%%%%%%%%%%%%%%%%%%%%%%%%%%%%%%%%%%%
%%%%%%%%%%%%%%%%%%%%%%%%%%%%%%%%%%%%%%%%%%%%%%%%%%%%%%%%%%%%%%%%%%%%%%%%%%%%%%%%
\appendix

\settowidth\MacroIndent{\rmfamily\scriptsize 000\ }

 \DocInput{childdoc.dtx}

\end{document}
%</driver>
% \fi
%
% %%%%%%%%%%%%%%%%%%%%%%%%%%%%%%%%%%%%%%%%%%%%%%%%%%%%%%%%%%%%%%%%%%%%%%%%%%%%%%
% %%%%%%%%%%%%%%%%%%%%%%%%%%%%%%%%%%%%%%%%%%%%%%%%%%%%%%%%%%%%%%%%%%%%%%%%%%%%%%
% \section{Sample}
%\iffalse
%<*samplemain>
%\fi
%
% The following presents a sample document
% with two chapters, two parts, a title page,
% a compile flag as well as three forwarding files to set the flag.
% It consists of eight |.tex| files:
% \begin{center}
% \begin{tabular}{ll}
% |cdocsamp.tex|&main file\\
% |cdocsch1.tex|&include file for chapter 1\\
% |cdocsch2.tex|&include file for chapter 2\\
% |cdocspt3.tex|&include file for part 3\\
% |cdocspt4.tex|&include file for part 4\\
% |cdocsdrf.tex|&forwarding file for main file in draft mode\\
% |cdocsfi1.tex|&forwarding file for final version of chapter 1\\
% |cdocsfi2.tex|&forwarding file for final version of chapter 2\\
% \end{tabular}
% \end{center}
% Each of the eight files can be compiled directly by the \LaTeX{} compiler.
%
% %%%%%%%%%%%%%%%%%%%%%%%%%%%%%%%%%%%%%%
% \paragraph{Main File.}
%
% The main file is called |cdocsamp.tex|.
%
% Load the \textsf{childdoc} definitions and
% declare the filename for the main document:
%    \begin{macrocode}
% \iffalse
%
% childdoc.dtx Copyright (C) 2017-2018 Niklas Beisert
%
% This work may be distributed and/or modified under the
% conditions of the LaTeX Project Public License, either version 1.3
% of this license or (at your option) any later version.
% The latest version of this license is in
%   http://www.latex-project.org/lppl.txt
% and version 1.3 or later is part of all distributions of LaTeX
% version 2005/12/01 or later.
%
% This work has the LPPL maintenance status `maintained'.
%
% The Current Maintainer of this work is Niklas Beisert.
%
% This work consists of the files childdoc.dtx and childdoc.ins
% and the derived files childdoc.def and cdocsamp.tex with
% cdocsch1.tex, cdocsch2.tex, cdocsdrf.tex, cdocsfn1.tex, cdocsfn2.tex.
%
%<package>\ifdefined\childdocmain\endinput\fi
%<package>\ProvidesFile{childdoc.def}[2018/12/30 v2.0 child document driver]
%<samplemain>\ProvidesFile{cdocsamp.tex}[2018/12/30 v2.0 sample for childdoc]
%<*driver>
%\ProvidesFile{childdoc.drv}[2018/12/30 v2.0 childdoc reference manual file]
\PassOptionsToClass{10pt,a4paper}{article}
\documentclass{ltxdoc}

\usepackage[margin=35mm]{geometry}
\usepackage{hyperref}
\usepackage{hyperxmp}
\usepackage[usenames]{color}

\hypersetup{colorlinks=true}
\hypersetup{pdfstartview=FitH}
\hypersetup{pdfpagemode=UseNone}
\hypersetup{pdfsource={}}
\hypersetup{pdflang={en-UK}}
\hypersetup{pdfcopyright={Copyright 2017-2018 Niklas Beisert.
  This work may be distributed and/or modified under the
  conditions of the LaTeX Project Public License, either version 1.3
  of this license or (at your option) any later version.}}
\hypersetup{pdflicenseurl={http://www.latex-project.org/lppl.txt}}
\hypersetup{pdfcontactaddress={ETH Zurich, ITP, HIT K,
  Wolfgang-Pauli-Strasse 27}}
\hypersetup{pdfcontactpostcode={8093}}
\hypersetup{pdfcontactcity={Zurich}}
\hypersetup{pdfcontactcountry={Switzerland}}
\hypersetup{pdfcontactemail={nbeisert@itp.phys.ethz.ch}}
\hypersetup{pdfcontacturl={http://people.phys.ethz.ch/\xmptilde nbeisert/}}

\newcommand{\secref}[1]{\hyperref[#1]{section \ref*{#1}}}

\parskip1ex
\parindent0pt
\let\olditemize\itemize
\def\itemize{\olditemize\parskip0pt}

\begin{document}

\title{The \textsf{childdoc} Package}
\hypersetup{pdftitle={The childdoc Package}}
\author{Niklas Beisert\\[2ex]
  Institut f\"ur Theoretische Physik\\
  Eidgen\"ossische Technische Hochschule Z\"urich\\
  Wolfgang-Pauli-Strasse 27, 8093 Z\"urich, Switzerland\\[1ex]
  \href{mailto:nbeisert@itp.phys.ethz.ch}
  {\texttt{nbeisert@itp.phys.ethz.ch}}}
\hypersetup{pdfauthor={Niklas Beisert}}
\hypersetup{pdfsubject={Manual for the LaTeX2e Package childdoc}}
\date{30 December 2018, \textsf{v2.0}}
\maketitle

\begin{abstract}\noindent
\textsf{childdoc} is a \LaTeXe{} package
that enables the direct compilation
of document sections included by |\include|
to individual files.
\end{abstract}

\begingroup
\parskip0ex
\tableofcontents
\endgroup

%%%%%%%%%%%%%%%%%%%%%%%%%%%%%%%%%%%%%%%%%%%%%%%%%%%%%%%%%%%%%%%%%%%%%%%%%%%%%%%%
%%%%%%%%%%%%%%%%%%%%%%%%%%%%%%%%%%%%%%%%%%%%%%%%%%%%%%%%%%%%%%%%%%%%%%%%%%%%%%%%
\section{Introduction}

\LaTeX{} provides a mechanism to structure a large document (such as a book)
into a main file and several child files (containing the chapters)
using the |\include| command.
This mechanism is beneficial for documents
which span hundreds of pages in order to
make the source file(s) more manageable.
Moreover, compilation can be restricted to
selected child files by means of the |\includeonly| command.
The latter feature can be used to reduce the compilation time while editing
(this was significantly more useful in the earlier days of \LaTeX{})
or to generate a smaller document which is easier to navigate.
Another application of |\includeonly| is to generate
documents consisting of selected parts of the complete document.

However, there are a few drawbacks of the plain |\include| mechanism:
\begin{itemize}
\item
The child files cannot be compiled on their own,
they can only be compiled via the main file.
A naive editing environment
(such as a text editor with an option
to have the current file processed by \LaTeX)
may require one to switch to the main file before compiling;
attempting to compile the child file produces errors.
\item
The main file must be modified (each time)
to adjust the |\includeonly| command
to the present needs. This easily leaves the main file in a messy state.
\item
The generated document will always carry the filename
of the main document. This is inconvenient if
several child files are to be compiled and
to be kept for distribution.
\end{itemize}

The present package provides a simple interface
to make child files individually compilable by \LaTeX{}.
Compiling a child file then has the same effect as compiling
the main file with an |\includeonly| command
to select the appropriate child.
Moreover the generated document will carry the name of the child
rather than the main file.
This resolves all three above issues.

This feature is meant to make the editing of books,
thesis documents and lecture notes somewhat more convenient.
However, the package can also be used efficiently for
composing a series of documents (such as exercise sheets)
which are typically distributed individually.
It then assists the author in generating the individual documents
(potentially in different versions)
as well as a document containing the collected series.
Another application is in developing style files
or other kinds of included material
where compilation of the style file could redirect
to a sample or test file.

%%%%%%%%%%%%%%%%%%%%%%%%%%%%%%%%%%%%%%%%%%%%%%%%%%%%%%%%%%%%%%%%%%%%%%%%%%%%%%%%
%%%%%%%%%%%%%%%%%%%%%%%%%%%%%%%%%%%%%%%%%%%%%%%%%%%%%%%%%%%%%%%%%%%%%%%%%%%%%%%%
\section{Usage}

First of all, the package \textsf{childdoc} is \emph{not} a standard
\LaTeXe{} |.sty| style file! Therefore it needs to be invoked in
a non-standard way.

%%%%%%%%%%%%%%%%%%%%%%%%%%%%%%%%%%%%%%%%%%%%%%%%%%%%%%%%%%%%%%%%%%%%%%%%%%%%%%%%
\subsection{Included Files}
\label{sec:include}

%%%%%%%%%%%%%%%%%%%%%%%%%%%%%%%%%%%%%%%%
\DescribeMacro{\childdocmain}
To use the package, add the commands
\begin{center}
\begin{tabular}{l}
|\input{childdoc.def}|\\
|\childdocmain{}|\\
\end{tabular}
\end{center}
at the very top of the main \LaTeX{} file,
in particular \emph{before} the |\documentclass| statement!
The argument of |\childdocmain| should be left empty
(but it must be present).

%%%%%%%%%%%%%%%%%%%%%%%%%%%%%%%%%%%%%%%%
\DescribeMacro{\childdocof}
Furthermore, add the commands
\begin{center}
\begin{tabular}{l}
|\input{childdoc.def}|\\
|\childdocof{|\textit{main}|}|\\
\end{tabular}
\end{center}
at the top of every child file \textit{child}
which is included by |\include{|\textit{child}|}|
from within the main file
(or at least for those files to be compiled individually).
The argument \textit{main} must be the filename of the main file.

There are a couple of
considerations in setting up the main and child documents:

%%%%%%%%%%%%%%%%%%%%%%%%%%%%%%%%%%%%%%%%
\paragraph{Restrictions.}

Please note the following restrictions:
\begin{itemize}
\item
|\childdocmain| must be called with one argument \textit{main}
to ensure compatibility with earlier version of the package.
It must either be empty (|\childdocmain{}|)
or precisely match the filename of the main file in which it is specified.
See \secref{sec:detection} for further information.
\item
The filename \textit{main} must be specified without the |.tex| extension.
\item
The filename \textit{main} is case sensitive
(even in case-insensitive file systems)
due to internal string comparison.
\item
The argument \textit{main} should be fully expanded, it cannot be a macro.
\item
Subdirectories and special characters should be avoided in filenames.
\item
The command |\childdocmain{|\textit{main}|}| must be followed by a whitespace.
It should not be followed immediately by another command
or by a comment mark `|%|'.
This is because the \TeX{} parser reads the token immediately following
the argument of |\childdocmain| and puts it
at the beginning of every child section;
however, a white\-space is ignored.
\end{itemize}

%%%%%%%%%%%%%%%%%%%%%%%%%%%%%%%%%%%%%%%%
\paragraph{Content of Main File.}

It is advisable to place all content in the child files included by |\include|.
Any output contained in the main file will appear in all child documents
unless suppressed manually;
it cannot be suppressed automatically by the |\includeonly| directive
and thus should normally be avoided.
A method to include some content in the main file
by means of conditional processing is described in \secref{sec:conditional}.

%%%%%%%%%%%%%%%%%%%%%%%%%%%%%%%%%%%%%%%%
\paragraph{Page Numbering.}

When only a part of the document is compiled,
the appropriate numbering of pages
(as well as other status parameters)
is determined from the |.aux| files.
The latter contain information from previous passes.
However this information needs to propagate through
all intermediate child documents.
Therefore the page numbering in child documents may well
be inconsistent until the complete document is compiled at least once.

A useful (if unconventional) way to always ensure a consistent
page numbering is to restart the numbering in each child document
and denote the pages by `\textit{child}|.|\textit{page}'
where \textit{child} represents the chapter/section number of the child file.
This can be achieved by the command
|\numberwithin{page}{|\textit{child}|}|
of the \textsf{amsmath} package
where \textit{child} can be |chapter| or |section|
depending on the chosen structuring.
Alternatively, one can modify the macro |\thepage| appropriately
and reset the counter |page| at the start of each child file.

%%%%%%%%%%%%%%%%%%%%%%%%%%%%%%%%%%%%%%%%%%%%%%%%%%%%%%%%%%%%%%%%%%%%%%%%%%%%%%%%
\subsection{Conditional Processing}
\label{sec:conditional}

The package provides a mechanism to compile different versions
of a document. To customise the versions further some conditional processing
can come in handy to distinguish which version is being compiled.
The package provides two macros to describe the compilation context:

%%%%%%%%%%%%%%%%%%%%%%%%%%%%%%%%%%%%%%%%
\DescribeMacro{\ifchilddoc}
The conditional |\ifchilddoc| distinguishes between the compilation of
child documents and the main document:
%
\begin{center}
|\ifchilddoc |\textit{child-code}| |[|\||else |\textit{main-code}]| \||fi|
\end{center}

%%%%%%%%%%%%%%%%%%%%%%%%%%%%%%%%%%%%%%%%
\DescribeMacro{\childdocname}
\DescribeMacro{\childdocjob}
The macro |\childdocname| contains the filename (without extension)
of the main or child file being processed.
Note that |\childdocjob| will always contain the name of the main file.

%%%%%%%%%%%%%%%%%%%%%%%%%%%%%%%%%%%%%%%%
\paragraph{Title Page.}

Conditional processing can be used to include a title or banner page
in the main document when proper precautions are taken.
Importantly, the code in the main file should ensure that the page counter
(as well as other status parameters which are stored in the |.aux| files)
takes the same value after the conditional processing.
Otherwise the page numbers may take divergent values
depending on which part is compiled.

For example, a title page could be declared by:
%
\begin{center}
\begin{tabular}{l}
|\ifchilddoc\||else|\\
|\addtocounter{page}{-1}|\\
\textit{code for title page}\\
|\newpage|\\
|\||fi|
\end{tabular}
\end{center}
%
A banner page for the child documents can be generated by:
%
\begin{center}
\begin{tabular}{l}
|\ifchilddoc|\\
|\addtocounter{page}{-1}|\\
\textit{code for banner page}\\
|\newpage|\\
|\||fi|
\end{tabular}
\end{center}
%
Here one could write a message such as:
\begin{center}
|This is the part \childdocname{} of \childdocjob{}.|
\end{center}

%%%%%%%%%%%%%%%%%%%%%%%%%%%%%%%%%%%%%%%%%%%%%%%%%%%%%%%%%%%%%%%%%%%%%%%%%%%%%%%%
\subsection{Flags}
\label{sec:flags}

The package makes it easy to generate different versions
of the main or child documents.
To this end compilation flags can be defined
and assigned different default values.
They will be particularly useful in conjunction
with the forwarding mechanism described in \secref{sec:forward}.

For example, it may be useful to have a flag |\version|
which can be set to |draft| or |final|.
The document source will contain some conditional code
depending on the value of |\version|.
Suppose further, the flag should default to |final| for the main file
and to |draft| for child files
which is a natural assignment for editing the document.
This is achieved by placing the following code
in the preamble of the main document
(below the |\childdocmain| directive):
%
\begin{center}
\begin{tabular}{l}
|\ifchilddoc|\\
|\providecommand{\version}{draft}|\\
|\||else|\\
|\providecommand{\version}{final}|\\
|\||fi|
\end{tabular}
\end{center}
%
The definition by |\providecommand| makes sure
that previous definitions are not overwritten.
Further statements |\providecommand{\version}{...}|
can thus be added before the above code to override it.

For the main file, one might add a line
(between |\childdocmain| and the above block)
%
\begin{center}
|%\ifchilddoc\||else\providecommand{\version}{draft}\||fi|
\end{center}
%
which can be uncommented to produce a draft version.
Likewise one can add a line to the very top of a child file
(above the |\childdocof{|\textit{main}|}| directive)
%
\begin{center}
|%\providecommand{\version}{final}|
\end{center}
%
which can be uncommented to produce the final version of this child document.

%%%%%%%%%%%%%%%%%%%%%%%%%%%%%%%%%%%%%%%%%%%%%%%%%%%%%%%%%%%%%%%%%%%%%%%%%%%%%%%%
\subsection{Forwarding}
\label{sec:forward}

Different versions of the main or child documents
using compilation flags as described in \secref{sec:flags}
can be (permanently) stored in different files
for convenient compilation, viewing and distribution.
To this end, the package defines a command
to pass on compilation to a different file:

%%%%%%%%%%%%%%%%%%%%%%%%%%%%%%%%%%%%%%%%
\DescribeMacro{\childdocforward}
The command |\childdocforward| redirects processing to
another source file:
%
\begin{center}
\begin{tabular}{l}
|\input{childdoc.def}|\\
|\childdocforward[|\textit{main}|]{|\textit{dest}|}|\\
\end{tabular}
\end{center}
%
The argument \textit{dest} is the destination file
(without extension).
It should be the main file or one of the child files.
Note that further \textsf{childdoc} directives
such as |\childdocof| and |\childdocforward|
in the indicated file will be processed in this form.
The optional argument \textit{main}
passes on directly to the main file \textit{main}
while pretending to compile the child \textit{dest}.
This form behaves as if \textit{dest}
issues |\childdocof{|\textit{main}|}| right away,
and no further \textsf{childdoc} directives will be processed.

%%%%%%%%%%%%%%%%%%%%%%%%%%%%%%%%%%%%%%%%
\DescribeMacro{\...prefix}
In the alternative form |\childdocforwardprefix|,
%
\begin{center}
\begin{tabular}{l}
|\input{childdoc.def}|\\
|\childdocforwardprefix[|\textit{main}|]{|\textit{prefix}|}{|\textit{dest}|}|
\end{tabular}
\end{center}
%
the destination file is determined by a pattern
depending on the current file:
To make this work, the current file must be called
`{\textit{prefix}\hspace{0.2em}\textit{suffix}}'
with \textit{prefix} matching precisely the argument.
Processing is then passed on to the file
`{\textit{dest}\hspace{0.2em}\textit{suffix}}'.
Surely, the same effect is achieved by
directly specifying the
argument `{\textit{dest}\hspace{0.2em}\textit{suffix}}'
in the first form.
However, that requires to set up a different file
for each child. With the alternative form of the command
all these files can have exactly the same content
which simplifies setting them up and maintaining them.

For example, the following file |draft.tex|
with a compilation flag |\version| as described in \secref{sec:flags}
compiles the main document as a draft:
%
\begin{center}
\begin{tabular}{l}
|\def\version{draft}|\\
|\input{childdoc.def}|\\
|\childdocforward{|\textit{main}|}|
\end{tabular}
\end{center}
%
Likewise, the following files |final|\textit{nn}|.tex|
compile the final version of the child document
|child|\textit{nn}|.tex|:
%
\begin{center}
\begin{tabular}{l}
|\def\version{final}|\\
|\input{childdoc.def}|\\
|\childdocforwardprefix{final}{child}|
\end{tabular}
\end{center}
%

Note that when several versions of a main file and/or of each child file
are to be generated, it may be convenient to set up a |Makefile| or
shell script to automatise the process.

%%%%%%%%%%%%%%%%%%%%%%%%%%%%%%%%%%%%%%%%%%%%%%%%%%%%%%%%%%%%%%%%%%%%%%%%%%%%%%%%
\subsection{Command Line Processing}
\label{sec:commandline}

The effect of redirection files can also be achieved by invoking
the \LaTeX{} compiler with a more elaborate command line.
Most conveniently this should be done as part
of a shell script or a |Makefile|.

When using \textsf{childdoc} in the main file, the following
command lines effectively perform a redirection
(note that depending on the shell being used,
backslashes may have to be doubled: `|\|' $\to$ `|\\|'):
%
\begin{center}
|... -jobname "|\textit{target}|" |\\|"|[\textit{flags}]%
|\input{childdoc.def}\childdocforward[|\textit{main}|]{|\textit{dest}|}"|
\end{center}
%
Here \textit{target} is the name of the output file,
\textit{main} is the name of the main file
and \textit{dest} is the name of the main or child file to be processed
(all filenames without extensions).
The optional argument \textit{main} can be omitted
if \textit{main} matches \textit{dest}.
Optionally, compilation \textit{flags} can be defined via |\def| commands.
This command line makes the \TeX{} engine believe
it is compiling the file \textit{target}
whose content is specified as the latter parameter.
The provided code then forwards the processing to
\textit{main} or \textit{dest} as described in \secref{sec:forward}.

%%%%%%%%%%%%%%%%%%%%%%%%%%%%%%%%%%%%%%%%%%%%%%%%%%%%%%%%%%%%%%%%%%%%%%%%%%%%%%%%
\subsection{Include by Input}
\label{sec:input}

Including child documents by |\include| has some restrictions by design.
Most notably, the content of a child document always occupies
its own set of pages; pages cannot be shared between child documents.
Usually, this behaviour makes perfect sense
because each child document contain an essential part of the document.
However, in some situations it may be desirable to compose
a document from a collection of parts
without having mandatory page breaks between then.
For this case, the package
provides a mechanism to include parts
by |\input| which can also be processed individually.
However, by construction this mechanism
requires manual handling of the content to be output.

%%%%%%%%%%%%%%%%%%%%%%%%%%%%%%%%%%%%%%%%
\DescribeMacro{\ifchilddocmanual}
The main file should be prepared as usual, see \secref{sec:include}.
However, the document body must make a distinction
between processing of an individual part and of the main document, e.g.:
%
\begin{center}
\begin{tabular}{l}
|\ifchilddocmanual|\\
|\input{\childdocname}|\\
|\||else|\\
\textit{document body with }|\input{|\textit{part}|}|\\
|\||fi|
\end{tabular}
\end{center}
%
The conditional |\ifchilddocmanual| is true whenever
a part to be included by |\input| is being compiled,
and the name of the part is stored in |\childdocname|.

%%%%%%%%%%%%%%%%%%%%%%%%%%%%%%%%%%%%%%%%
\DescribeMacro{\childdocby}
Each part to be included by |\input| should start with:
%
\begin{center}
\begin{tabular}{l}
|\input{childdoc.def}|\\
|\childdocby{|\textit{main}|}|\\
\end{tabular}
\end{center}
%
The directive |\childdocby| is similar to |\childdocof|
described in \secref{sec:include},
but the subsequent selection of content must be done manually.
To that end, both |\ifchilddoc| and |\ifchilddocmanual|
will be true upon processing of a part,
and the name of the part is stored in |\childdocname|.
Note that |\jobname| will be set to the filename of the current part
so that each part receives an individual |.aux| file
that does not interfere with the |.aux| file(s) of the main document.
This behaviour can be altered by the alternative form
|\childdocby[*]{|\textit{main}|}| (with a non-empty optional argument)
which uses the |.aux| file of the main document
by setting |\jobname| to \textit{main}.

%%%%%%%%%%%%%%%%%%%%%%%%%%%%%%%%%%%%%%%%%%%%%%%%%%%%%%%%%%%%%%%%%%%%%%%%%%%%%%%%
\subsection{Driver Development}
\label{sec:driver}

The \textsf{childdoc} mechanism can also be use for the development
of definition files such as \LaTeX{} styles or classes.
This case differs from the above setup with multiple parts
included by |\include| in that no |\includeonly| should be invoked.
This can be achieved by starting the include file
(before |\ProvidesPackage|) with:
%
\begin{center}
\begin{tabular}{l}
|\input{childdoc.def}|\\
|\childdocforward{|\textit{main}|}|\\
\end{tabular}
\end{center}
%
or alternatively with:
%
\begin{center}
\begin{tabular}{l}
|\input{childdoc.def}|\\
|\childdocby{|\textit{main}|}|\\
\end{tabular}
\end{center}
%
Both forms have slightly different effects as described above.
The main file is prepared as usual, see \secref{sec:include}.

%%%%%%%%%%%%%%%%%%%%%%%%%%%%%%%%%%%%%%%%%%%%%%%%%%%%%%%%%%%%%%%%%%%%%%%%%%%%%%%%
\subsection{Legacy Detection}
\label{sec:detection}

The directive |\childdocmain| in the main file can detect
whether the complete document or merely a child is to be compiled
even without using the directive |\childdocof|.
This method is deprecated because it is less robust
and there is no compelling reason to use it;
it is merely provided for backward compatibility
and it may be removed in future versions.

If the detection mechanism is to be used,
it is mandatory to correctly specify
the filename of the main file as the argument of |\childdocmain|:
%
\begin{center}
\begin{tabular}{l}
|\input{childdoc.def}|\\
|\childdocmain{|\textit{main}|}|\\
\end{tabular}
\end{center}
%
If |\jobname| does not match the argument \textit{main} of |\childdocmain|,
it is assumed that |\jobname| points to the child file to be compiled.
When using |\childdocmain| with the main file specified as argument,
it suffices to start a child file
with just |\input{|\textit{main}|}|
without loading of the package and using |\childdocof|.
If instead all processing is done
with the appropriate \textsf{childdoc} directives,
the argument of \textit{main} of |\childdocmain| can be empty.

An alternative version of the command line processing described
in \secref{sec:commandline} using the detection mechanism reads:
%
\begin{center}
|... -jobname "|\textit{target}|" "|[\textit{flags}]%
[|\def\jobname{|\textit{dest}|}|]|\input{|\textit{main}|}"|
\end{center}

%%%%%%%%%%%%%%%%%%%%%%%%%%%%%%%%%%%%%%%%%%%%%%%%%%%%%%%%%%%%%%%%%%%%%%%%%%%%%%%%
\subsection{Manual Code}
\label{sec:manual}

In case one cannot be certain whether the definitions file |childdoc.def|
is installed on the target \TeX{} distribution
and one prefers not to ship it,
it is conceivable to paste a few relevant commands into the sources.

To that end, drop all statements |\input{childdoc.def}|
and perform the replacements as outlined below.
Instead of |\childdocmain{|\textit{main}|}| add the following code
to the top of the main file:
%
\begin{center}
\begin{tabular}{l}
|\||ifdefined\childdocname\endinput\||fi\newif\ifchilddoc|\\
|\edef\childdocname{\scantokens\expandafter{\jobname\noexpand}}|\\
|\def\childdocmain{|\textit{main}|}\||ifx\childdocmain\childdocname\||else|\\
|\childdoctrue\includeonly{\childdocname}\let\jobname\childdocmain\||fi|\\
\end{tabular}
\end{center}
%
Instead of |\childdocof{|\textit{main}|}| just include the main file
at the top of each child file:
%
\begin{center}
|\input{|\textit{main}|}|
\end{center}
%
A simple redirection |\childdocforward{|\textit{dest}|}| is achieved by:
%
\begin{center}
|\def\jobname{|\textit{dest}|}\input{\jobname}|
\end{center}
%
The redirection with prefix
|\childdocforwardprefix[|\textit{prefix}|]{|\textit{dest}|}|
is accomplished by:
%
\begin{center}
\begin{tabular}{l}
|{\edef\jobname{\scantokens\expandafter{\jobname\noexpand}}|\\
|\def\redirectjob |\textit{prefix}|#1~~~{\gdef\jobname{|\textit{dest}|#1}}|\\
|\expandafter\redirectjob\jobname~~~}\input{\jobname}|
\end{tabular}
\end{center}

In an alternative approach,
child documents can be compiled by a specific command line
without additional code or specific definitions:
%
\begin{center}
|... -jobname "|\textit{target}|" "|[\textit{flags}]%
|\includeonly{|\textit{dest}|}\input{|\textit{main}|}"|
\end{center}
%

%%%%%%%%%%%%%%%%%%%%%%%%%%%%%%%%%%%%%%%%%%%%%%%%%%%%%%%%%%%%%%%%%%%%%%%%%%%%%%%%
%%%%%%%%%%%%%%%%%%%%%%%%%%%%%%%%%%%%%%%%%%%%%%%%%%%%%%%%%%%%%%%%%%%%%%%%%%%%%%%%
\section{Information}

%%%%%%%%%%%%%%%%%%%%%%%%%%%%%%%%%%%%%%%%%%%%%%%%%%%%%%%%%%%%%%%%%%%%%%%%%%%%%%%%
\subsection{Copyright}

Copyright \copyright{} 2017--2018 Niklas Beisert

This work may be distributed and/or modified under the
conditions of the \LaTeX{} Project Public License, either version 1.3
of this license or (at your option) any later version.
The latest version of this license is in
  \url{http://www.latex-project.org/lppl.txt}
and version 1.3 or later is part of all distributions of \LaTeX{}
version 2005/12/01 or later.

This work has the LPPL maintenance status `maintained'.

The Current Maintainer of this work is Niklas Beisert.

This work consists of the files |README.txt|, |childdoc.ins| and |childdoc.dtx|
as well as the derived files |childdoc.def|, |cdocsamp.tex|
with |cdocsch1.tex|, |cdocsch2.tex|, |cdocspt3.tex|, |cdocspt4.tex|,
|cdocsdrf.tex|, |cdocsfn1.tex|, |cdocsfn2.tex|
as well as |childdoc.pdf|.

%%%%%%%%%%%%%%%%%%%%%%%%%%%%%%%%%%%%%%%%%%%%%%%%%%%%%%%%%%%%%%%%%%%%%%%%%%%%%%%%
\subsection{Files and Installation}

The package consists of the files:
%
\begin{center}
\begin{tabular}{ll}
    |README.txt|   & readme file \\
    |childdoc.ins| & installation file \\
    |childdoc.dtx| & source file \\
    |childdoc.def| & definition file \\
    |cdocsamp.tex| & sample main file \\
    |cdocsch1.tex| & sample include file \\
    |cdocsch2.tex| & sample include file \\
    |cdocspt3.tex| & sample part file \\
    |cdocspt4.tex| & sample part file \\
    |cdocsdrf.tex| & sample redirection file \\
    |cdocsfn1.tex| & sample redirection file \\
    |cdocsfn2.tex| & sample redirection file \\
    |childdoc.pdf| & manual
\end{tabular}
\end{center}
%
The distribution consists of the files
|README.txt|, |childdoc.ins| and |childdoc.dtx|.
%
\begin{itemize}
\item
Run (pdf)\LaTeX{} on |childdoc.dtx|
to compile the manual |childdoc.pdf| (this file).
\item
Run \LaTeX{} on |childdoc.ins| to create the definitions file |childdoc.def|
and the sample |cdocsamp.tex| with include files
|cdocsch1.tex|, |cdocsch2.tex|, |cdocspt3.tex|, |cdocspt4.tex|,
|cdocsdrf.tex|, |cdocsfn1.tex|, |cdocsfn2.tex|.
Then copy the file |childdoc.def| to an appropriate directory of your \LaTeX{}
distribution, e.g.\ \textit{texmf-root}|/tex/latex/childdoc|.
\end{itemize}

%%%%%%%%%%%%%%%%%%%%%%%%%%%%%%%%%%%%%%%%%%%%%%%%%%%%%%%%%%%%%%%%%%%%%%%%%%%%%%%%
\subsection{Related CTAN Packages}

There are several other packages which offer a similar functionality:
%
\begin{itemize}
\item
The packages
\href{http://ctan.org/pkg/docmute}{\textsf{docmute}},
\href{http://ctan.org/pkg/includex}{\textsf{includex}} and
\href{http://ctan.org/pkg/standalone}{\textsf{standalone}}
provide commands to include only the document body of
a child file thus allowing both files to be compiled individually.
\item
The packages \href{http://ctan.org/pkg/subdocs}{\textsf{subdocs}}
and \href{http://ctan.org/pkg/subfiles}{\textsf{subfiles}}
provide structures in which the main and child documents can be
encapsulated and allowing them to be compiled individually.
The inclusion mechanism is different from the conventional |\include|.
\item
The package \href{http://ctan.org/pkg/combine}{\textsf{combine}}
is an elaborate solution to combine several documents into one.
\end{itemize}
%
See also the CTAN topic \href{http://ctan.org/topic/subdocs}{\textsf{subdocs}}
for further related packages.
The present package differs from the above solutions in that
a document structure constructed with the conventional |\include| mechanism
just needs two extra commands at the top of every file
such that all constituent files can be compiled individually.

%%%%%%%%%%%%%%%%%%%%%%%%%%%%%%%%%%%%%%%%%%%%%%%%%%%%%%%%%%%%%%%%%%%%%%%%%%%%%%%%
%\subsection{Feature Suggestions}
%
%The following is a list of features which may be useful for future
%versions of this package:
%%
%\begin{itemize}
%\item
%\ldots
%\end{itemize}

%%%%%%%%%%%%%%%%%%%%%%%%%%%%%%%%%%%%%%%%%%%%%%%%%%%%%%%%%%%%%%%%%%%%%%%%%%%%%%%%
\subsection{Revision History}

%%%%%%%%%%%%%%%%%%%%%%%%%%%%%%%%%%%%%%%%
\paragraph{v2.0:} 2018/12/30

\begin{itemize}
\item
immediate forward processing
\item
added |\childdocby| mechanism
\item
manual restructured
\end{itemize}

%%%%%%%%%%%%%%%%%%%%%%%%%%%%%%%%%%%%%%%%
\paragraph{v1.6:} 2018/01/17

\begin{itemize}
\item
application for development of include files
\item
corrections to manual
\end{itemize}

%%%%%%%%%%%%%%%%%%%%%%%%%%%%%%%%%%%%%%%%
\paragraph{v1.5:} 2017/05/21

\begin{itemize}
\item
more complete structuring introduced
\item
|\childdocof| introduced
\item
|\childdoc| renamed to |\childdocmain|
\item
|\childredirect| renamed to |\childdocforward| and |\childdocforwardprefix|
and functionality expanded
\end{itemize}

%%%%%%%%%%%%%%%%%%%%%%%%%%%%%%%%%%%%%%%%
\paragraph{v1.0:} 2017/04/27

\begin{itemize}
\item
manual and install package
\item
first version published on CTAN
\end{itemize}

%%%%%%%%%%%%%%%%%%%%%%%%%%%%%%%%%%%%%%%%
\paragraph{v0.6:} 2017/04/26

\begin{itemize}
\item
redirection mechanism added
\end{itemize}

%%%%%%%%%%%%%%%%%%%%%%%%%%%%%%%%%%%%%%%%
\paragraph{v0.5:} 2017/04/26

\begin{itemize}
\item
functionality in definition file
\end{itemize}


%%%%%%%%%%%%%%%%%%%%%%%%%%%%%%%%%%%%%%%%%%%%%%%%%%%%%%%%%%%%%%%%%%%%%%%%%%%%%%%%
%%%%%%%%%%%%%%%%%%%%%%%%%%%%%%%%%%%%%%%%%%%%%%%%%%%%%%%%%%%%%%%%%%%%%%%%%%%%%%%%
%%%%%%%%%%%%%%%%%%%%%%%%%%%%%%%%%%%%%%%%%%%%%%%%%%%%%%%%%%%%%%%%%%%%%%%%%%%%%%%%
\appendix

\settowidth\MacroIndent{\rmfamily\scriptsize 000\ }

 \DocInput{childdoc.dtx}

\end{document}
%</driver>
% \fi
%
% %%%%%%%%%%%%%%%%%%%%%%%%%%%%%%%%%%%%%%%%%%%%%%%%%%%%%%%%%%%%%%%%%%%%%%%%%%%%%%
% %%%%%%%%%%%%%%%%%%%%%%%%%%%%%%%%%%%%%%%%%%%%%%%%%%%%%%%%%%%%%%%%%%%%%%%%%%%%%%
% \section{Sample}
%\iffalse
%<*samplemain>
%\fi
%
% The following presents a sample document
% with two chapters, two parts, a title page,
% a compile flag as well as three forwarding files to set the flag.
% It consists of eight |.tex| files:
% \begin{center}
% \begin{tabular}{ll}
% |cdocsamp.tex|&main file\\
% |cdocsch1.tex|&include file for chapter 1\\
% |cdocsch2.tex|&include file for chapter 2\\
% |cdocspt3.tex|&include file for part 3\\
% |cdocspt4.tex|&include file for part 4\\
% |cdocsdrf.tex|&forwarding file for main file in draft mode\\
% |cdocsfi1.tex|&forwarding file for final version of chapter 1\\
% |cdocsfi2.tex|&forwarding file for final version of chapter 2\\
% \end{tabular}
% \end{center}
% Each of the eight files can be compiled directly by the \LaTeX{} compiler.
%
% %%%%%%%%%%%%%%%%%%%%%%%%%%%%%%%%%%%%%%
% \paragraph{Main File.}
%
% The main file is called |cdocsamp.tex|.
%
% Load the \textsf{childdoc} definitions and
% declare the filename for the main document:
%    \begin{macrocode}
\input{childdoc.def}
\childdocmain{}
%    \end{macrocode}

% Optional override for |\version| flag:
%    \begin{macrocode}
%%\ifchilddoc\else\providecommand{\version}{draft}\fi
%    \end{macrocode}

% Define the default values for the |\version| flag
% (|final| for the main file and |draft| for childs):
%    \begin{macrocode}
\ifchilddoc
\providecommand{\version}{draft}
\else
\providecommand{\version}{final}
\fi
%    \end{macrocode}

% Load the standard document class:
%    \begin{macrocode}
\documentclass[12pt]{article}
%    \end{macrocode}

% Start the document body:
%    \begin{macrocode}
\begin{document}
%    \end{macrocode}

% Declare a title page.
% Print title, part of document being processed and version flag:
%    \begin{macrocode}
\addtocounter{page}{-1}
\begin{center}
{\LARGE\bfseries{}childdoc example\par}
\vspace{1cm}
\ifchilddoc
\ifchilddocmanual part\else chapter\fi:
`\childdocname' of `\childdocjob'\par
\else
main document: `\childdocjob'\par
\fi
version: \version\par
\end{center}
\newpage
%    \end{macrocode}

% Manually include selected file,
% otherwise process as usual:
%    \begin{macrocode}
\ifchilddocmanual
\section*{part `\childdocname'}
\input{\childdocname}
\else
%    \end{macrocode}

% Include the two chapters:
%    \begin{macrocode}
\include{cdocsch1}
\include{cdocsch2}
%    \end{macrocode}

% Include the two parts unless only chapters should be displayed:
%    \begin{macrocode}
\ifchilddoc\else
\section{part three}
\input{cdocspt3}
\section{part four}
\input{cdocspt4}
\fi
%    \end{macrocode}

% Process as usual until here:
%    \begin{macrocode}
\fi
%    \end{macrocode}

% End of document body:
%    \begin{macrocode}
\end{document}
%    \end{macrocode}
%\iffalse
%</samplemain>
%\fi
%
% %%%%%%%%%%%%%%%%%%%%%%%%%%%%%%%%%%%%%%
% \paragraph{Chapter Include Files.}
%
% The include files are called |cdocsch1.tex| and |cdocsch2.tex|.
%
%\iffalse
%<*samplechap1|samplechap2>
%\fi

% Optional override for |\version| flag:
%    \begin{macrocode}
%%\providecommand{\version}{final}
%    \end{macrocode}

% Include the main document:
%    \begin{macrocode}
\input{childdoc.def}
\childdocof{cdocsamp}
%    \end{macrocode}

%\iffalse
%</samplechap1|samplechap2>
%\fi
%
%\iffalse
%<*samplechap1>
%\fi
% Some text for chapter 1:
%    \begin{macrocode}
\section{one}
some text in chapter one
%    \end{macrocode}

%\iffalse
%</samplechap1>
%\fi
% Some text for chapter 2:
%\iffalse
%<*samplechap2>
%\fi
%    \begin{macrocode}
\section{two}
more text in chapter two
%    \end{macrocode}

%\iffalse
%</samplechap2>
%\fi
%
% %%%%%%%%%%%%%%%%%%%%%%%%%%%%%%%%%%%%%%
% \paragraph{Part Include Files.}
%
% The include files are called |cdocspt3.tex| and |cdocspt4.tex|.
%
%\iffalse
%<*samplepart3|samplepart4>
%\fi

% Optional override for |\version| flag:
%    \begin{macrocode}
%%\providecommand{\version}{final}
%    \end{macrocode}

% Include the main document:
%    \begin{macrocode}
\input{childdoc.def}
\childdocby{cdocsamp}
%    \end{macrocode}

%\iffalse
%</samplepart3|samplepart4>
%\fi
%
%\iffalse
%<*samplepart3>
%\fi
% Some text for part 3:
%    \begin{macrocode}
some text in part three
%    \end{macrocode}

%\iffalse
%</samplepart3>
%\fi
% Some text for part 4:
%\iffalse
%<*samplepart4>
%\fi
%    \begin{macrocode}
more text in part four
%    \end{macrocode}

%\iffalse
%</samplepart4>
%\fi
%
% %%%%%%%%%%%%%%%%%%%%%%%%%%%%%%%%%%%%%%
% \paragraph{Forwarding for a Complete Draft.}
%
% The following forwarding file |cdocsdrf.tex|
% compiles the main document in draft mode:
%\iffalse
%<*sampledraft>
%\fi
%    \begin{macrocode}
\def\version{draft}
\input{childdoc.def}
\childdocforward{cdocsamp}
%    \end{macrocode}

%\iffalse
%</sampledraft>
%\fi
%
% %%%%%%%%%%%%%%%%%%%%%%%%%%%%%%%%%%%%%%
% \paragraph{Forwarding for Final Version of the Chapters.}
%
% The following forwarding files |cdocsfn1.tex| and |cdocsfn2.tex|
% (with identical content)
% compile the final versions of the child documents
% |cdocsch1.tex| and |cdocsch2.tex|, respectively:
%\iffalse
%<*samplefinal>
%\fi
%    \begin{macrocode}
\def\version{final}
\input{childdoc.def}
\childdocforwardprefix[cdocsamp]{cdocsfn}{cdocsch}
%    \end{macrocode}

%\iffalse
%</samplefinal>
%\fi
%
% %%%%%%%%%%%%%%%%%%%%%%%%%%%%%%%%%%%%%%
% \paragraph{Command Line Processing.}
%
% The following three command lines generate the output files
% |cdocscld|, |cdocscl1| and |cdocscl2|
% which should be identical to
% |cdocsdrf|, |cdocsch1| and |cdocsfn2|, respectively:
% \begin{center}
% \begin{tabular}{l}
% |latex -jobname cdocscld \|\\
% |  "\def\version{draft}\input{childdoc.def}\childdocforward{cdocsamp}"|\\
% |latex -jobname cdocscl1 \|\\
% |  "\input{childdoc.def}\childdocforward[cdocsamp]{cdocsch1}"|\\
% |latex -jobname cdocscl2 \|\\
% |  "\def\version{final}\input{childdoc.def}\childdocforward{cdocsch2}"|
% \end{tabular}
% \end{center}
% Note that the trailing backslash on each first line
% merely continues the input to the second line
% (for convenient cut ant paste).
% Furthermore, the command |latex| can be replaced by any
% of its alternative versions such as |pdflatex|.
%
% %%%%%%%%%%%%%%%%%%%%%%%%%%%%%%%%%%%%%%%%%%%%%%%%%%%%%%%%%%%%%%%%%%%%%%%%%%%%%%
% %%%%%%%%%%%%%%%%%%%%%%%%%%%%%%%%%%%%%%%%%%%%%%%%%%%%%%%%%%%%%%%%%%%%%%%%%%%%%%
% \section{Implementation}
%\iffalse
%<*package>
%\fi
%
% This section describes the definitions file |childdoc.def|.

% The definitions cannot be loaded using |\usepackage| or |\RequirePackage|
% which has a mechanism to prevent loading a style file more than once.
% When loading the definitions by means of |\input|
% multiple instances have to be prevented manually:
%\iffalse
%This code needs to be before the `\ProvidesFile' directive
%which is defined at the beginning of this file.
%Therefore it is also placed there and commented out here.
%</package>
%<*discard>
%\fi
%    \begin{macrocode}
\ifdefined\childdocmain\endinput\fi
%    \end{macrocode}
%\iffalse
%</discard>
%<*package>
%\fi
%
% \macro{\ifchilddoc}
% \macro{\ifchilddocmanual}
% The conditional |\ifchilddoc| tells whether a
% child (true) or main (false) document is being compiled.
% The conditional |\ifchilddocmanual| tells whether
% the |\includeonly| mechanism is used (false) or
% the selection of child files must be performed manually (true).
% The definitions initialise to false:
%    \begin{macrocode}
\newif\ifchilddoc
\newif\ifchilddocmanual
%    \end{macrocode}

% \macro{\childdocname}
% \macro{\childdocjob}
% The macro |\childdocname| stores the name of the main document
% to be compiled. The macro |\childdocjob| stores the name of
% the document on which the \LaTeX{} compiler was originally invoked.
% The content of |\jobname| cannot be compared
% to filenames specified in the source due to different catcodes.
% The following code rescans |\jobname|, stores the result
% in |\childdocname| and saves a copy in |\childdocjob|:
%    \begin{macrocode}
\edef\childdocname{\scantokens\expandafter{\jobname\noexpand}}
\let\childdocjob\childdocname
%    \end{macrocode}

% \macro{\childdocdisable}
% The macro |\childdocdisable| prevents the main file
% from being processed more than once.
% At this stage, the main document command |\childdocmain|
% is assumed to be called once again where it should do nothing.
% Any subsequent call to it should prevent
% a secondary processing of the main document
% It overwrites the forwarding commands
% |\childdocof| and |\childdocforward|
% with empty macros to prevent further inclusions of the main document:
%    \begin{macrocode}
\newcommand{\childdocdisable}
{
  \renewcommand{\childdocmain}[1]{\renewcommand{\childdocmain}[1]{\endinput}}
  \renewcommand{\childdocof}[1]{}
  \renewcommand{\childdocby}[2][]{}
  \renewcommand{\childdocforward}[2][]{}
  \renewcommand{\childdocdisable}{}
}
%    \end{macrocode}

% \macro{\childdocmain}
% The macro |\childdocmain| is to be called at the top of the main file
% with nothing or the main filename (without extension) as argument.
% First, it breaks loops.
% If the argument is not empty and does not match |\childdocname|
% (which is set by the first inclusion of |childdoc.def|),
% |\ifchilddoc| is set to true, |\includeonly| is applied to the child file
% and |\jobname| is set to the main file
% (for proper handling of |.aux| files):
%    \begin{macrocode}
\newcommand{\childdocmain}[1]
{
  \childdocdisable\childdocmain{}
  \if?#1?\else
    \begingroup
      \def\childdoctmp{#1}
      \ifx\childdoctmp\childdocname
        \def\childdoctmp{}
      \else
        \def\childdoctmp
        {
          \childdoctrue
          \includeonly{\childdocname}
          \def\childdocjob{#1}
          \def\jobname{#1}
        }
      \fi
      \expandafter
    \endgroup
    \childdoctmp
  \fi
}
%    \end{macrocode}

% \macro{\childdocof}
% The command |\childdocof| redirects
% compilation to the main file |#1|.
%    \begin{macrocode}
\newcommand{\childdocof}[1]
{
  \childdocdisable
  \childdoctrue
  \includeonly{\childdocname}
  \def\jobname{#1}
  \def\childdocjob{#1}
  \input{#1}
}
%    \end{macrocode}

% \macro{\childdocby}
% The command |\childdocby| ....
%    \begin{macrocode}
\newcommand{\childdocby}[2][]
{
  \childdocdisable
  \childdoctrue
  \childdocmanualtrue
  \if?#1?\else
    \def\jobname{#2}
  \fi
  \def\childdocjob{#2}
  \input{#2}
  \endinput
}
%    \end{macrocode}

% \macro{\childdocforward}
% The command |\childdocforward| redirects
% compilation to the main file or
% (if the optional argument is given) a child file.
% Parameters are set as if the main file
% or a child file starting with |\childdocof| was compiled.
% Then compilation is handed over to the main file:
%    \begin{macrocode}
\newcommand{\childdocforward}[2][]
{
  \begingroup
    \if?#1?
      \def\childdoctmp
      {
        \def\childdocname{#2}
        \def\childdocjob{#2}
        \def\jobname{#2}
        \input{#2}
        \endinput
      }
    \else
      \def\childdoctmp
      {
        \childdocdisable
        \def\childdocname{#2}
        \childdoctrue
        \includeonly{#2}
        \def\childdocjob{#1}
        \def\jobname{#1}
        \input{#1}
        \endinput
      }
    \fi
    \expandafter
  \endgroup
  \childdoctmp
}
%    \end{macrocode}

% \macro{\childdocforwardprefix}
% The command |\childdocforwardprefix| redirects
% compilation to the main or a child file by means of a pattern.
% The prefix |#1| in the current filename is replaced by |#2|
% and the suffix of the current filename is kept
% (it is assumed that the filename does not contain the substring `|~~~|'
% which is used as a delimiter).
% Compilation is handed over to the new file by |\childdocforward|:
%    \begin{macrocode}
\newcommand{\childdocforwardprefix}[3][]
{
  \begingroup
    \def\childdocextract #2##1~~~{\def\childdoctmp{\childdocforward[#1]{#3##1}}}
    \expandafter\childdocextract\childdocname~~~
    \expandafter
  \endgroup
  \childdoctmp
}
%    \end{macrocode}

% \macro{\childdoc}
% The deprecated macro |\childdoc| is a legacy version of |\childdocmain|:
%    \begin{macrocode}
\newcommand{\childdoc}{\childdocmain}
%    \end{macrocode}

% \macro{\childdocredirect}
% The deprecated macro |\childdocredirect| is a legacy version
% of |\childdocforward| and |\childdocforwardprefix|:
%    \begin{macrocode}
\newcommand{\childdocredirect}[2][]
{
  \begingroup
    \if?#1?
      \def\childdoctmp{\childdocforward{#2}}
    \else
      \def\childdoctmp{\childdocforwardprefix{#1}{#2}}
    \fi
    \expandafter
  \endgroup
  \childdoctmp
}
%    \end{macrocode}

%\iffalse
%</package>
%\fi
%
\endinput

\childdocmain{}
%    \end{macrocode}

% Optional override for |\version| flag:
%    \begin{macrocode}
%%\ifchilddoc\else\providecommand{\version}{draft}\fi
%    \end{macrocode}

% Define the default values for the |\version| flag
% (|final| for the main file and |draft| for childs):
%    \begin{macrocode}
\ifchilddoc
\providecommand{\version}{draft}
\else
\providecommand{\version}{final}
\fi
%    \end{macrocode}

% Load the standard document class:
%    \begin{macrocode}
\documentclass[12pt]{article}
%    \end{macrocode}

% Start the document body:
%    \begin{macrocode}
\begin{document}
%    \end{macrocode}

% Declare a title page.
% Print title, part of document being processed and version flag:
%    \begin{macrocode}
\addtocounter{page}{-1}
\begin{center}
{\LARGE\bfseries{}childdoc example\par}
\vspace{1cm}
\ifchilddoc
\ifchilddocmanual part\else chapter\fi:
`\childdocname' of `\childdocjob'\par
\else
main document: `\childdocjob'\par
\fi
version: \version\par
\end{center}
\newpage
%    \end{macrocode}

% Manually include selected file,
% otherwise process as usual:
%    \begin{macrocode}
\ifchilddocmanual
\section*{part `\childdocname'}
\input{\childdocname}
\else
%    \end{macrocode}

% Include the two chapters:
%    \begin{macrocode}
\include{cdocsch1}
\include{cdocsch2}
%    \end{macrocode}

% Include the two parts unless only chapters should be displayed:
%    \begin{macrocode}
\ifchilddoc\else
\section{part three}
\input{cdocspt3}
\section{part four}
\input{cdocspt4}
\fi
%    \end{macrocode}

% Process as usual until here:
%    \begin{macrocode}
\fi
%    \end{macrocode}

% End of document body:
%    \begin{macrocode}
\end{document}
%    \end{macrocode}
%\iffalse
%</samplemain>
%\fi
%
% %%%%%%%%%%%%%%%%%%%%%%%%%%%%%%%%%%%%%%
% \paragraph{Chapter Include Files.}
%
% The include files are called |cdocsch1.tex| and |cdocsch2.tex|.
%
%\iffalse
%<*samplechap1|samplechap2>
%\fi

% Optional override for |\version| flag:
%    \begin{macrocode}
%%\providecommand{\version}{final}
%    \end{macrocode}

% Include the main document:
%    \begin{macrocode}
% \iffalse
%
% childdoc.dtx Copyright (C) 2017-2018 Niklas Beisert
%
% This work may be distributed and/or modified under the
% conditions of the LaTeX Project Public License, either version 1.3
% of this license or (at your option) any later version.
% The latest version of this license is in
%   http://www.latex-project.org/lppl.txt
% and version 1.3 or later is part of all distributions of LaTeX
% version 2005/12/01 or later.
%
% This work has the LPPL maintenance status `maintained'.
%
% The Current Maintainer of this work is Niklas Beisert.
%
% This work consists of the files childdoc.dtx and childdoc.ins
% and the derived files childdoc.def and cdocsamp.tex with
% cdocsch1.tex, cdocsch2.tex, cdocsdrf.tex, cdocsfn1.tex, cdocsfn2.tex.
%
%<package>\ifdefined\childdocmain\endinput\fi
%<package>\ProvidesFile{childdoc.def}[2018/12/30 v2.0 child document driver]
%<samplemain>\ProvidesFile{cdocsamp.tex}[2018/12/30 v2.0 sample for childdoc]
%<*driver>
%\ProvidesFile{childdoc.drv}[2018/12/30 v2.0 childdoc reference manual file]
\PassOptionsToClass{10pt,a4paper}{article}
\documentclass{ltxdoc}

\usepackage[margin=35mm]{geometry}
\usepackage{hyperref}
\usepackage{hyperxmp}
\usepackage[usenames]{color}

\hypersetup{colorlinks=true}
\hypersetup{pdfstartview=FitH}
\hypersetup{pdfpagemode=UseNone}
\hypersetup{pdfsource={}}
\hypersetup{pdflang={en-UK}}
\hypersetup{pdfcopyright={Copyright 2017-2018 Niklas Beisert.
  This work may be distributed and/or modified under the
  conditions of the LaTeX Project Public License, either version 1.3
  of this license or (at your option) any later version.}}
\hypersetup{pdflicenseurl={http://www.latex-project.org/lppl.txt}}
\hypersetup{pdfcontactaddress={ETH Zurich, ITP, HIT K,
  Wolfgang-Pauli-Strasse 27}}
\hypersetup{pdfcontactpostcode={8093}}
\hypersetup{pdfcontactcity={Zurich}}
\hypersetup{pdfcontactcountry={Switzerland}}
\hypersetup{pdfcontactemail={nbeisert@itp.phys.ethz.ch}}
\hypersetup{pdfcontacturl={http://people.phys.ethz.ch/\xmptilde nbeisert/}}

\newcommand{\secref}[1]{\hyperref[#1]{section \ref*{#1}}}

\parskip1ex
\parindent0pt
\let\olditemize\itemize
\def\itemize{\olditemize\parskip0pt}

\begin{document}

\title{The \textsf{childdoc} Package}
\hypersetup{pdftitle={The childdoc Package}}
\author{Niklas Beisert\\[2ex]
  Institut f\"ur Theoretische Physik\\
  Eidgen\"ossische Technische Hochschule Z\"urich\\
  Wolfgang-Pauli-Strasse 27, 8093 Z\"urich, Switzerland\\[1ex]
  \href{mailto:nbeisert@itp.phys.ethz.ch}
  {\texttt{nbeisert@itp.phys.ethz.ch}}}
\hypersetup{pdfauthor={Niklas Beisert}}
\hypersetup{pdfsubject={Manual for the LaTeX2e Package childdoc}}
\date{30 December 2018, \textsf{v2.0}}
\maketitle

\begin{abstract}\noindent
\textsf{childdoc} is a \LaTeXe{} package
that enables the direct compilation
of document sections included by |\include|
to individual files.
\end{abstract}

\begingroup
\parskip0ex
\tableofcontents
\endgroup

%%%%%%%%%%%%%%%%%%%%%%%%%%%%%%%%%%%%%%%%%%%%%%%%%%%%%%%%%%%%%%%%%%%%%%%%%%%%%%%%
%%%%%%%%%%%%%%%%%%%%%%%%%%%%%%%%%%%%%%%%%%%%%%%%%%%%%%%%%%%%%%%%%%%%%%%%%%%%%%%%
\section{Introduction}

\LaTeX{} provides a mechanism to structure a large document (such as a book)
into a main file and several child files (containing the chapters)
using the |\include| command.
This mechanism is beneficial for documents
which span hundreds of pages in order to
make the source file(s) more manageable.
Moreover, compilation can be restricted to
selected child files by means of the |\includeonly| command.
The latter feature can be used to reduce the compilation time while editing
(this was significantly more useful in the earlier days of \LaTeX{})
or to generate a smaller document which is easier to navigate.
Another application of |\includeonly| is to generate
documents consisting of selected parts of the complete document.

However, there are a few drawbacks of the plain |\include| mechanism:
\begin{itemize}
\item
The child files cannot be compiled on their own,
they can only be compiled via the main file.
A naive editing environment
(such as a text editor with an option
to have the current file processed by \LaTeX)
may require one to switch to the main file before compiling;
attempting to compile the child file produces errors.
\item
The main file must be modified (each time)
to adjust the |\includeonly| command
to the present needs. This easily leaves the main file in a messy state.
\item
The generated document will always carry the filename
of the main document. This is inconvenient if
several child files are to be compiled and
to be kept for distribution.
\end{itemize}

The present package provides a simple interface
to make child files individually compilable by \LaTeX{}.
Compiling a child file then has the same effect as compiling
the main file with an |\includeonly| command
to select the appropriate child.
Moreover the generated document will carry the name of the child
rather than the main file.
This resolves all three above issues.

This feature is meant to make the editing of books,
thesis documents and lecture notes somewhat more convenient.
However, the package can also be used efficiently for
composing a series of documents (such as exercise sheets)
which are typically distributed individually.
It then assists the author in generating the individual documents
(potentially in different versions)
as well as a document containing the collected series.
Another application is in developing style files
or other kinds of included material
where compilation of the style file could redirect
to a sample or test file.

%%%%%%%%%%%%%%%%%%%%%%%%%%%%%%%%%%%%%%%%%%%%%%%%%%%%%%%%%%%%%%%%%%%%%%%%%%%%%%%%
%%%%%%%%%%%%%%%%%%%%%%%%%%%%%%%%%%%%%%%%%%%%%%%%%%%%%%%%%%%%%%%%%%%%%%%%%%%%%%%%
\section{Usage}

First of all, the package \textsf{childdoc} is \emph{not} a standard
\LaTeXe{} |.sty| style file! Therefore it needs to be invoked in
a non-standard way.

%%%%%%%%%%%%%%%%%%%%%%%%%%%%%%%%%%%%%%%%%%%%%%%%%%%%%%%%%%%%%%%%%%%%%%%%%%%%%%%%
\subsection{Included Files}
\label{sec:include}

%%%%%%%%%%%%%%%%%%%%%%%%%%%%%%%%%%%%%%%%
\DescribeMacro{\childdocmain}
To use the package, add the commands
\begin{center}
\begin{tabular}{l}
|\input{childdoc.def}|\\
|\childdocmain{}|\\
\end{tabular}
\end{center}
at the very top of the main \LaTeX{} file,
in particular \emph{before} the |\documentclass| statement!
The argument of |\childdocmain| should be left empty
(but it must be present).

%%%%%%%%%%%%%%%%%%%%%%%%%%%%%%%%%%%%%%%%
\DescribeMacro{\childdocof}
Furthermore, add the commands
\begin{center}
\begin{tabular}{l}
|\input{childdoc.def}|\\
|\childdocof{|\textit{main}|}|\\
\end{tabular}
\end{center}
at the top of every child file \textit{child}
which is included by |\include{|\textit{child}|}|
from within the main file
(or at least for those files to be compiled individually).
The argument \textit{main} must be the filename of the main file.

There are a couple of
considerations in setting up the main and child documents:

%%%%%%%%%%%%%%%%%%%%%%%%%%%%%%%%%%%%%%%%
\paragraph{Restrictions.}

Please note the following restrictions:
\begin{itemize}
\item
|\childdocmain| must be called with one argument \textit{main}
to ensure compatibility with earlier version of the package.
It must either be empty (|\childdocmain{}|)
or precisely match the filename of the main file in which it is specified.
See \secref{sec:detection} for further information.
\item
The filename \textit{main} must be specified without the |.tex| extension.
\item
The filename \textit{main} is case sensitive
(even in case-insensitive file systems)
due to internal string comparison.
\item
The argument \textit{main} should be fully expanded, it cannot be a macro.
\item
Subdirectories and special characters should be avoided in filenames.
\item
The command |\childdocmain{|\textit{main}|}| must be followed by a whitespace.
It should not be followed immediately by another command
or by a comment mark `|%|'.
This is because the \TeX{} parser reads the token immediately following
the argument of |\childdocmain| and puts it
at the beginning of every child section;
however, a white\-space is ignored.
\end{itemize}

%%%%%%%%%%%%%%%%%%%%%%%%%%%%%%%%%%%%%%%%
\paragraph{Content of Main File.}

It is advisable to place all content in the child files included by |\include|.
Any output contained in the main file will appear in all child documents
unless suppressed manually;
it cannot be suppressed automatically by the |\includeonly| directive
and thus should normally be avoided.
A method to include some content in the main file
by means of conditional processing is described in \secref{sec:conditional}.

%%%%%%%%%%%%%%%%%%%%%%%%%%%%%%%%%%%%%%%%
\paragraph{Page Numbering.}

When only a part of the document is compiled,
the appropriate numbering of pages
(as well as other status parameters)
is determined from the |.aux| files.
The latter contain information from previous passes.
However this information needs to propagate through
all intermediate child documents.
Therefore the page numbering in child documents may well
be inconsistent until the complete document is compiled at least once.

A useful (if unconventional) way to always ensure a consistent
page numbering is to restart the numbering in each child document
and denote the pages by `\textit{child}|.|\textit{page}'
where \textit{child} represents the chapter/section number of the child file.
This can be achieved by the command
|\numberwithin{page}{|\textit{child}|}|
of the \textsf{amsmath} package
where \textit{child} can be |chapter| or |section|
depending on the chosen structuring.
Alternatively, one can modify the macro |\thepage| appropriately
and reset the counter |page| at the start of each child file.

%%%%%%%%%%%%%%%%%%%%%%%%%%%%%%%%%%%%%%%%%%%%%%%%%%%%%%%%%%%%%%%%%%%%%%%%%%%%%%%%
\subsection{Conditional Processing}
\label{sec:conditional}

The package provides a mechanism to compile different versions
of a document. To customise the versions further some conditional processing
can come in handy to distinguish which version is being compiled.
The package provides two macros to describe the compilation context:

%%%%%%%%%%%%%%%%%%%%%%%%%%%%%%%%%%%%%%%%
\DescribeMacro{\ifchilddoc}
The conditional |\ifchilddoc| distinguishes between the compilation of
child documents and the main document:
%
\begin{center}
|\ifchilddoc |\textit{child-code}| |[|\||else |\textit{main-code}]| \||fi|
\end{center}

%%%%%%%%%%%%%%%%%%%%%%%%%%%%%%%%%%%%%%%%
\DescribeMacro{\childdocname}
\DescribeMacro{\childdocjob}
The macro |\childdocname| contains the filename (without extension)
of the main or child file being processed.
Note that |\childdocjob| will always contain the name of the main file.

%%%%%%%%%%%%%%%%%%%%%%%%%%%%%%%%%%%%%%%%
\paragraph{Title Page.}

Conditional processing can be used to include a title or banner page
in the main document when proper precautions are taken.
Importantly, the code in the main file should ensure that the page counter
(as well as other status parameters which are stored in the |.aux| files)
takes the same value after the conditional processing.
Otherwise the page numbers may take divergent values
depending on which part is compiled.

For example, a title page could be declared by:
%
\begin{center}
\begin{tabular}{l}
|\ifchilddoc\||else|\\
|\addtocounter{page}{-1}|\\
\textit{code for title page}\\
|\newpage|\\
|\||fi|
\end{tabular}
\end{center}
%
A banner page for the child documents can be generated by:
%
\begin{center}
\begin{tabular}{l}
|\ifchilddoc|\\
|\addtocounter{page}{-1}|\\
\textit{code for banner page}\\
|\newpage|\\
|\||fi|
\end{tabular}
\end{center}
%
Here one could write a message such as:
\begin{center}
|This is the part \childdocname{} of \childdocjob{}.|
\end{center}

%%%%%%%%%%%%%%%%%%%%%%%%%%%%%%%%%%%%%%%%%%%%%%%%%%%%%%%%%%%%%%%%%%%%%%%%%%%%%%%%
\subsection{Flags}
\label{sec:flags}

The package makes it easy to generate different versions
of the main or child documents.
To this end compilation flags can be defined
and assigned different default values.
They will be particularly useful in conjunction
with the forwarding mechanism described in \secref{sec:forward}.

For example, it may be useful to have a flag |\version|
which can be set to |draft| or |final|.
The document source will contain some conditional code
depending on the value of |\version|.
Suppose further, the flag should default to |final| for the main file
and to |draft| for child files
which is a natural assignment for editing the document.
This is achieved by placing the following code
in the preamble of the main document
(below the |\childdocmain| directive):
%
\begin{center}
\begin{tabular}{l}
|\ifchilddoc|\\
|\providecommand{\version}{draft}|\\
|\||else|\\
|\providecommand{\version}{final}|\\
|\||fi|
\end{tabular}
\end{center}
%
The definition by |\providecommand| makes sure
that previous definitions are not overwritten.
Further statements |\providecommand{\version}{...}|
can thus be added before the above code to override it.

For the main file, one might add a line
(between |\childdocmain| and the above block)
%
\begin{center}
|%\ifchilddoc\||else\providecommand{\version}{draft}\||fi|
\end{center}
%
which can be uncommented to produce a draft version.
Likewise one can add a line to the very top of a child file
(above the |\childdocof{|\textit{main}|}| directive)
%
\begin{center}
|%\providecommand{\version}{final}|
\end{center}
%
which can be uncommented to produce the final version of this child document.

%%%%%%%%%%%%%%%%%%%%%%%%%%%%%%%%%%%%%%%%%%%%%%%%%%%%%%%%%%%%%%%%%%%%%%%%%%%%%%%%
\subsection{Forwarding}
\label{sec:forward}

Different versions of the main or child documents
using compilation flags as described in \secref{sec:flags}
can be (permanently) stored in different files
for convenient compilation, viewing and distribution.
To this end, the package defines a command
to pass on compilation to a different file:

%%%%%%%%%%%%%%%%%%%%%%%%%%%%%%%%%%%%%%%%
\DescribeMacro{\childdocforward}
The command |\childdocforward| redirects processing to
another source file:
%
\begin{center}
\begin{tabular}{l}
|\input{childdoc.def}|\\
|\childdocforward[|\textit{main}|]{|\textit{dest}|}|\\
\end{tabular}
\end{center}
%
The argument \textit{dest} is the destination file
(without extension).
It should be the main file or one of the child files.
Note that further \textsf{childdoc} directives
such as |\childdocof| and |\childdocforward|
in the indicated file will be processed in this form.
The optional argument \textit{main}
passes on directly to the main file \textit{main}
while pretending to compile the child \textit{dest}.
This form behaves as if \textit{dest}
issues |\childdocof{|\textit{main}|}| right away,
and no further \textsf{childdoc} directives will be processed.

%%%%%%%%%%%%%%%%%%%%%%%%%%%%%%%%%%%%%%%%
\DescribeMacro{\...prefix}
In the alternative form |\childdocforwardprefix|,
%
\begin{center}
\begin{tabular}{l}
|\input{childdoc.def}|\\
|\childdocforwardprefix[|\textit{main}|]{|\textit{prefix}|}{|\textit{dest}|}|
\end{tabular}
\end{center}
%
the destination file is determined by a pattern
depending on the current file:
To make this work, the current file must be called
`{\textit{prefix}\hspace{0.2em}\textit{suffix}}'
with \textit{prefix} matching precisely the argument.
Processing is then passed on to the file
`{\textit{dest}\hspace{0.2em}\textit{suffix}}'.
Surely, the same effect is achieved by
directly specifying the
argument `{\textit{dest}\hspace{0.2em}\textit{suffix}}'
in the first form.
However, that requires to set up a different file
for each child. With the alternative form of the command
all these files can have exactly the same content
which simplifies setting them up and maintaining them.

For example, the following file |draft.tex|
with a compilation flag |\version| as described in \secref{sec:flags}
compiles the main document as a draft:
%
\begin{center}
\begin{tabular}{l}
|\def\version{draft}|\\
|\input{childdoc.def}|\\
|\childdocforward{|\textit{main}|}|
\end{tabular}
\end{center}
%
Likewise, the following files |final|\textit{nn}|.tex|
compile the final version of the child document
|child|\textit{nn}|.tex|:
%
\begin{center}
\begin{tabular}{l}
|\def\version{final}|\\
|\input{childdoc.def}|\\
|\childdocforwardprefix{final}{child}|
\end{tabular}
\end{center}
%

Note that when several versions of a main file and/or of each child file
are to be generated, it may be convenient to set up a |Makefile| or
shell script to automatise the process.

%%%%%%%%%%%%%%%%%%%%%%%%%%%%%%%%%%%%%%%%%%%%%%%%%%%%%%%%%%%%%%%%%%%%%%%%%%%%%%%%
\subsection{Command Line Processing}
\label{sec:commandline}

The effect of redirection files can also be achieved by invoking
the \LaTeX{} compiler with a more elaborate command line.
Most conveniently this should be done as part
of a shell script or a |Makefile|.

When using \textsf{childdoc} in the main file, the following
command lines effectively perform a redirection
(note that depending on the shell being used,
backslashes may have to be doubled: `|\|' $\to$ `|\\|'):
%
\begin{center}
|... -jobname "|\textit{target}|" |\\|"|[\textit{flags}]%
|\input{childdoc.def}\childdocforward[|\textit{main}|]{|\textit{dest}|}"|
\end{center}
%
Here \textit{target} is the name of the output file,
\textit{main} is the name of the main file
and \textit{dest} is the name of the main or child file to be processed
(all filenames without extensions).
The optional argument \textit{main} can be omitted
if \textit{main} matches \textit{dest}.
Optionally, compilation \textit{flags} can be defined via |\def| commands.
This command line makes the \TeX{} engine believe
it is compiling the file \textit{target}
whose content is specified as the latter parameter.
The provided code then forwards the processing to
\textit{main} or \textit{dest} as described in \secref{sec:forward}.

%%%%%%%%%%%%%%%%%%%%%%%%%%%%%%%%%%%%%%%%%%%%%%%%%%%%%%%%%%%%%%%%%%%%%%%%%%%%%%%%
\subsection{Include by Input}
\label{sec:input}

Including child documents by |\include| has some restrictions by design.
Most notably, the content of a child document always occupies
its own set of pages; pages cannot be shared between child documents.
Usually, this behaviour makes perfect sense
because each child document contain an essential part of the document.
However, in some situations it may be desirable to compose
a document from a collection of parts
without having mandatory page breaks between then.
For this case, the package
provides a mechanism to include parts
by |\input| which can also be processed individually.
However, by construction this mechanism
requires manual handling of the content to be output.

%%%%%%%%%%%%%%%%%%%%%%%%%%%%%%%%%%%%%%%%
\DescribeMacro{\ifchilddocmanual}
The main file should be prepared as usual, see \secref{sec:include}.
However, the document body must make a distinction
between processing of an individual part and of the main document, e.g.:
%
\begin{center}
\begin{tabular}{l}
|\ifchilddocmanual|\\
|\input{\childdocname}|\\
|\||else|\\
\textit{document body with }|\input{|\textit{part}|}|\\
|\||fi|
\end{tabular}
\end{center}
%
The conditional |\ifchilddocmanual| is true whenever
a part to be included by |\input| is being compiled,
and the name of the part is stored in |\childdocname|.

%%%%%%%%%%%%%%%%%%%%%%%%%%%%%%%%%%%%%%%%
\DescribeMacro{\childdocby}
Each part to be included by |\input| should start with:
%
\begin{center}
\begin{tabular}{l}
|\input{childdoc.def}|\\
|\childdocby{|\textit{main}|}|\\
\end{tabular}
\end{center}
%
The directive |\childdocby| is similar to |\childdocof|
described in \secref{sec:include},
but the subsequent selection of content must be done manually.
To that end, both |\ifchilddoc| and |\ifchilddocmanual|
will be true upon processing of a part,
and the name of the part is stored in |\childdocname|.
Note that |\jobname| will be set to the filename of the current part
so that each part receives an individual |.aux| file
that does not interfere with the |.aux| file(s) of the main document.
This behaviour can be altered by the alternative form
|\childdocby[*]{|\textit{main}|}| (with a non-empty optional argument)
which uses the |.aux| file of the main document
by setting |\jobname| to \textit{main}.

%%%%%%%%%%%%%%%%%%%%%%%%%%%%%%%%%%%%%%%%%%%%%%%%%%%%%%%%%%%%%%%%%%%%%%%%%%%%%%%%
\subsection{Driver Development}
\label{sec:driver}

The \textsf{childdoc} mechanism can also be use for the development
of definition files such as \LaTeX{} styles or classes.
This case differs from the above setup with multiple parts
included by |\include| in that no |\includeonly| should be invoked.
This can be achieved by starting the include file
(before |\ProvidesPackage|) with:
%
\begin{center}
\begin{tabular}{l}
|\input{childdoc.def}|\\
|\childdocforward{|\textit{main}|}|\\
\end{tabular}
\end{center}
%
or alternatively with:
%
\begin{center}
\begin{tabular}{l}
|\input{childdoc.def}|\\
|\childdocby{|\textit{main}|}|\\
\end{tabular}
\end{center}
%
Both forms have slightly different effects as described above.
The main file is prepared as usual, see \secref{sec:include}.

%%%%%%%%%%%%%%%%%%%%%%%%%%%%%%%%%%%%%%%%%%%%%%%%%%%%%%%%%%%%%%%%%%%%%%%%%%%%%%%%
\subsection{Legacy Detection}
\label{sec:detection}

The directive |\childdocmain| in the main file can detect
whether the complete document or merely a child is to be compiled
even without using the directive |\childdocof|.
This method is deprecated because it is less robust
and there is no compelling reason to use it;
it is merely provided for backward compatibility
and it may be removed in future versions.

If the detection mechanism is to be used,
it is mandatory to correctly specify
the filename of the main file as the argument of |\childdocmain|:
%
\begin{center}
\begin{tabular}{l}
|\input{childdoc.def}|\\
|\childdocmain{|\textit{main}|}|\\
\end{tabular}
\end{center}
%
If |\jobname| does not match the argument \textit{main} of |\childdocmain|,
it is assumed that |\jobname| points to the child file to be compiled.
When using |\childdocmain| with the main file specified as argument,
it suffices to start a child file
with just |\input{|\textit{main}|}|
without loading of the package and using |\childdocof|.
If instead all processing is done
with the appropriate \textsf{childdoc} directives,
the argument of \textit{main} of |\childdocmain| can be empty.

An alternative version of the command line processing described
in \secref{sec:commandline} using the detection mechanism reads:
%
\begin{center}
|... -jobname "|\textit{target}|" "|[\textit{flags}]%
[|\def\jobname{|\textit{dest}|}|]|\input{|\textit{main}|}"|
\end{center}

%%%%%%%%%%%%%%%%%%%%%%%%%%%%%%%%%%%%%%%%%%%%%%%%%%%%%%%%%%%%%%%%%%%%%%%%%%%%%%%%
\subsection{Manual Code}
\label{sec:manual}

In case one cannot be certain whether the definitions file |childdoc.def|
is installed on the target \TeX{} distribution
and one prefers not to ship it,
it is conceivable to paste a few relevant commands into the sources.

To that end, drop all statements |\input{childdoc.def}|
and perform the replacements as outlined below.
Instead of |\childdocmain{|\textit{main}|}| add the following code
to the top of the main file:
%
\begin{center}
\begin{tabular}{l}
|\||ifdefined\childdocname\endinput\||fi\newif\ifchilddoc|\\
|\edef\childdocname{\scantokens\expandafter{\jobname\noexpand}}|\\
|\def\childdocmain{|\textit{main}|}\||ifx\childdocmain\childdocname\||else|\\
|\childdoctrue\includeonly{\childdocname}\let\jobname\childdocmain\||fi|\\
\end{tabular}
\end{center}
%
Instead of |\childdocof{|\textit{main}|}| just include the main file
at the top of each child file:
%
\begin{center}
|\input{|\textit{main}|}|
\end{center}
%
A simple redirection |\childdocforward{|\textit{dest}|}| is achieved by:
%
\begin{center}
|\def\jobname{|\textit{dest}|}\input{\jobname}|
\end{center}
%
The redirection with prefix
|\childdocforwardprefix[|\textit{prefix}|]{|\textit{dest}|}|
is accomplished by:
%
\begin{center}
\begin{tabular}{l}
|{\edef\jobname{\scantokens\expandafter{\jobname\noexpand}}|\\
|\def\redirectjob |\textit{prefix}|#1~~~{\gdef\jobname{|\textit{dest}|#1}}|\\
|\expandafter\redirectjob\jobname~~~}\input{\jobname}|
\end{tabular}
\end{center}

In an alternative approach,
child documents can be compiled by a specific command line
without additional code or specific definitions:
%
\begin{center}
|... -jobname "|\textit{target}|" "|[\textit{flags}]%
|\includeonly{|\textit{dest}|}\input{|\textit{main}|}"|
\end{center}
%

%%%%%%%%%%%%%%%%%%%%%%%%%%%%%%%%%%%%%%%%%%%%%%%%%%%%%%%%%%%%%%%%%%%%%%%%%%%%%%%%
%%%%%%%%%%%%%%%%%%%%%%%%%%%%%%%%%%%%%%%%%%%%%%%%%%%%%%%%%%%%%%%%%%%%%%%%%%%%%%%%
\section{Information}

%%%%%%%%%%%%%%%%%%%%%%%%%%%%%%%%%%%%%%%%%%%%%%%%%%%%%%%%%%%%%%%%%%%%%%%%%%%%%%%%
\subsection{Copyright}

Copyright \copyright{} 2017--2018 Niklas Beisert

This work may be distributed and/or modified under the
conditions of the \LaTeX{} Project Public License, either version 1.3
of this license or (at your option) any later version.
The latest version of this license is in
  \url{http://www.latex-project.org/lppl.txt}
and version 1.3 or later is part of all distributions of \LaTeX{}
version 2005/12/01 or later.

This work has the LPPL maintenance status `maintained'.

The Current Maintainer of this work is Niklas Beisert.

This work consists of the files |README.txt|, |childdoc.ins| and |childdoc.dtx|
as well as the derived files |childdoc.def|, |cdocsamp.tex|
with |cdocsch1.tex|, |cdocsch2.tex|, |cdocspt3.tex|, |cdocspt4.tex|,
|cdocsdrf.tex|, |cdocsfn1.tex|, |cdocsfn2.tex|
as well as |childdoc.pdf|.

%%%%%%%%%%%%%%%%%%%%%%%%%%%%%%%%%%%%%%%%%%%%%%%%%%%%%%%%%%%%%%%%%%%%%%%%%%%%%%%%
\subsection{Files and Installation}

The package consists of the files:
%
\begin{center}
\begin{tabular}{ll}
    |README.txt|   & readme file \\
    |childdoc.ins| & installation file \\
    |childdoc.dtx| & source file \\
    |childdoc.def| & definition file \\
    |cdocsamp.tex| & sample main file \\
    |cdocsch1.tex| & sample include file \\
    |cdocsch2.tex| & sample include file \\
    |cdocspt3.tex| & sample part file \\
    |cdocspt4.tex| & sample part file \\
    |cdocsdrf.tex| & sample redirection file \\
    |cdocsfn1.tex| & sample redirection file \\
    |cdocsfn2.tex| & sample redirection file \\
    |childdoc.pdf| & manual
\end{tabular}
\end{center}
%
The distribution consists of the files
|README.txt|, |childdoc.ins| and |childdoc.dtx|.
%
\begin{itemize}
\item
Run (pdf)\LaTeX{} on |childdoc.dtx|
to compile the manual |childdoc.pdf| (this file).
\item
Run \LaTeX{} on |childdoc.ins| to create the definitions file |childdoc.def|
and the sample |cdocsamp.tex| with include files
|cdocsch1.tex|, |cdocsch2.tex|, |cdocspt3.tex|, |cdocspt4.tex|,
|cdocsdrf.tex|, |cdocsfn1.tex|, |cdocsfn2.tex|.
Then copy the file |childdoc.def| to an appropriate directory of your \LaTeX{}
distribution, e.g.\ \textit{texmf-root}|/tex/latex/childdoc|.
\end{itemize}

%%%%%%%%%%%%%%%%%%%%%%%%%%%%%%%%%%%%%%%%%%%%%%%%%%%%%%%%%%%%%%%%%%%%%%%%%%%%%%%%
\subsection{Related CTAN Packages}

There are several other packages which offer a similar functionality:
%
\begin{itemize}
\item
The packages
\href{http://ctan.org/pkg/docmute}{\textsf{docmute}},
\href{http://ctan.org/pkg/includex}{\textsf{includex}} and
\href{http://ctan.org/pkg/standalone}{\textsf{standalone}}
provide commands to include only the document body of
a child file thus allowing both files to be compiled individually.
\item
The packages \href{http://ctan.org/pkg/subdocs}{\textsf{subdocs}}
and \href{http://ctan.org/pkg/subfiles}{\textsf{subfiles}}
provide structures in which the main and child documents can be
encapsulated and allowing them to be compiled individually.
The inclusion mechanism is different from the conventional |\include|.
\item
The package \href{http://ctan.org/pkg/combine}{\textsf{combine}}
is an elaborate solution to combine several documents into one.
\end{itemize}
%
See also the CTAN topic \href{http://ctan.org/topic/subdocs}{\textsf{subdocs}}
for further related packages.
The present package differs from the above solutions in that
a document structure constructed with the conventional |\include| mechanism
just needs two extra commands at the top of every file
such that all constituent files can be compiled individually.

%%%%%%%%%%%%%%%%%%%%%%%%%%%%%%%%%%%%%%%%%%%%%%%%%%%%%%%%%%%%%%%%%%%%%%%%%%%%%%%%
%\subsection{Feature Suggestions}
%
%The following is a list of features which may be useful for future
%versions of this package:
%%
%\begin{itemize}
%\item
%\ldots
%\end{itemize}

%%%%%%%%%%%%%%%%%%%%%%%%%%%%%%%%%%%%%%%%%%%%%%%%%%%%%%%%%%%%%%%%%%%%%%%%%%%%%%%%
\subsection{Revision History}

%%%%%%%%%%%%%%%%%%%%%%%%%%%%%%%%%%%%%%%%
\paragraph{v2.0:} 2018/12/30

\begin{itemize}
\item
immediate forward processing
\item
added |\childdocby| mechanism
\item
manual restructured
\end{itemize}

%%%%%%%%%%%%%%%%%%%%%%%%%%%%%%%%%%%%%%%%
\paragraph{v1.6:} 2018/01/17

\begin{itemize}
\item
application for development of include files
\item
corrections to manual
\end{itemize}

%%%%%%%%%%%%%%%%%%%%%%%%%%%%%%%%%%%%%%%%
\paragraph{v1.5:} 2017/05/21

\begin{itemize}
\item
more complete structuring introduced
\item
|\childdocof| introduced
\item
|\childdoc| renamed to |\childdocmain|
\item
|\childredirect| renamed to |\childdocforward| and |\childdocforwardprefix|
and functionality expanded
\end{itemize}

%%%%%%%%%%%%%%%%%%%%%%%%%%%%%%%%%%%%%%%%
\paragraph{v1.0:} 2017/04/27

\begin{itemize}
\item
manual and install package
\item
first version published on CTAN
\end{itemize}

%%%%%%%%%%%%%%%%%%%%%%%%%%%%%%%%%%%%%%%%
\paragraph{v0.6:} 2017/04/26

\begin{itemize}
\item
redirection mechanism added
\end{itemize}

%%%%%%%%%%%%%%%%%%%%%%%%%%%%%%%%%%%%%%%%
\paragraph{v0.5:} 2017/04/26

\begin{itemize}
\item
functionality in definition file
\end{itemize}


%%%%%%%%%%%%%%%%%%%%%%%%%%%%%%%%%%%%%%%%%%%%%%%%%%%%%%%%%%%%%%%%%%%%%%%%%%%%%%%%
%%%%%%%%%%%%%%%%%%%%%%%%%%%%%%%%%%%%%%%%%%%%%%%%%%%%%%%%%%%%%%%%%%%%%%%%%%%%%%%%
%%%%%%%%%%%%%%%%%%%%%%%%%%%%%%%%%%%%%%%%%%%%%%%%%%%%%%%%%%%%%%%%%%%%%%%%%%%%%%%%
\appendix

\settowidth\MacroIndent{\rmfamily\scriptsize 000\ }

 \DocInput{childdoc.dtx}

\end{document}
%</driver>
% \fi
%
% %%%%%%%%%%%%%%%%%%%%%%%%%%%%%%%%%%%%%%%%%%%%%%%%%%%%%%%%%%%%%%%%%%%%%%%%%%%%%%
% %%%%%%%%%%%%%%%%%%%%%%%%%%%%%%%%%%%%%%%%%%%%%%%%%%%%%%%%%%%%%%%%%%%%%%%%%%%%%%
% \section{Sample}
%\iffalse
%<*samplemain>
%\fi
%
% The following presents a sample document
% with two chapters, two parts, a title page,
% a compile flag as well as three forwarding files to set the flag.
% It consists of eight |.tex| files:
% \begin{center}
% \begin{tabular}{ll}
% |cdocsamp.tex|&main file\\
% |cdocsch1.tex|&include file for chapter 1\\
% |cdocsch2.tex|&include file for chapter 2\\
% |cdocspt3.tex|&include file for part 3\\
% |cdocspt4.tex|&include file for part 4\\
% |cdocsdrf.tex|&forwarding file for main file in draft mode\\
% |cdocsfi1.tex|&forwarding file for final version of chapter 1\\
% |cdocsfi2.tex|&forwarding file for final version of chapter 2\\
% \end{tabular}
% \end{center}
% Each of the eight files can be compiled directly by the \LaTeX{} compiler.
%
% %%%%%%%%%%%%%%%%%%%%%%%%%%%%%%%%%%%%%%
% \paragraph{Main File.}
%
% The main file is called |cdocsamp.tex|.
%
% Load the \textsf{childdoc} definitions and
% declare the filename for the main document:
%    \begin{macrocode}
\input{childdoc.def}
\childdocmain{}
%    \end{macrocode}

% Optional override for |\version| flag:
%    \begin{macrocode}
%%\ifchilddoc\else\providecommand{\version}{draft}\fi
%    \end{macrocode}

% Define the default values for the |\version| flag
% (|final| for the main file and |draft| for childs):
%    \begin{macrocode}
\ifchilddoc
\providecommand{\version}{draft}
\else
\providecommand{\version}{final}
\fi
%    \end{macrocode}

% Load the standard document class:
%    \begin{macrocode}
\documentclass[12pt]{article}
%    \end{macrocode}

% Start the document body:
%    \begin{macrocode}
\begin{document}
%    \end{macrocode}

% Declare a title page.
% Print title, part of document being processed and version flag:
%    \begin{macrocode}
\addtocounter{page}{-1}
\begin{center}
{\LARGE\bfseries{}childdoc example\par}
\vspace{1cm}
\ifchilddoc
\ifchilddocmanual part\else chapter\fi:
`\childdocname' of `\childdocjob'\par
\else
main document: `\childdocjob'\par
\fi
version: \version\par
\end{center}
\newpage
%    \end{macrocode}

% Manually include selected file,
% otherwise process as usual:
%    \begin{macrocode}
\ifchilddocmanual
\section*{part `\childdocname'}
\input{\childdocname}
\else
%    \end{macrocode}

% Include the two chapters:
%    \begin{macrocode}
\include{cdocsch1}
\include{cdocsch2}
%    \end{macrocode}

% Include the two parts unless only chapters should be displayed:
%    \begin{macrocode}
\ifchilddoc\else
\section{part three}
\input{cdocspt3}
\section{part four}
\input{cdocspt4}
\fi
%    \end{macrocode}

% Process as usual until here:
%    \begin{macrocode}
\fi
%    \end{macrocode}

% End of document body:
%    \begin{macrocode}
\end{document}
%    \end{macrocode}
%\iffalse
%</samplemain>
%\fi
%
% %%%%%%%%%%%%%%%%%%%%%%%%%%%%%%%%%%%%%%
% \paragraph{Chapter Include Files.}
%
% The include files are called |cdocsch1.tex| and |cdocsch2.tex|.
%
%\iffalse
%<*samplechap1|samplechap2>
%\fi

% Optional override for |\version| flag:
%    \begin{macrocode}
%%\providecommand{\version}{final}
%    \end{macrocode}

% Include the main document:
%    \begin{macrocode}
\input{childdoc.def}
\childdocof{cdocsamp}
%    \end{macrocode}

%\iffalse
%</samplechap1|samplechap2>
%\fi
%
%\iffalse
%<*samplechap1>
%\fi
% Some text for chapter 1:
%    \begin{macrocode}
\section{one}
some text in chapter one
%    \end{macrocode}

%\iffalse
%</samplechap1>
%\fi
% Some text for chapter 2:
%\iffalse
%<*samplechap2>
%\fi
%    \begin{macrocode}
\section{two}
more text in chapter two
%    \end{macrocode}

%\iffalse
%</samplechap2>
%\fi
%
% %%%%%%%%%%%%%%%%%%%%%%%%%%%%%%%%%%%%%%
% \paragraph{Part Include Files.}
%
% The include files are called |cdocspt3.tex| and |cdocspt4.tex|.
%
%\iffalse
%<*samplepart3|samplepart4>
%\fi

% Optional override for |\version| flag:
%    \begin{macrocode}
%%\providecommand{\version}{final}
%    \end{macrocode}

% Include the main document:
%    \begin{macrocode}
\input{childdoc.def}
\childdocby{cdocsamp}
%    \end{macrocode}

%\iffalse
%</samplepart3|samplepart4>
%\fi
%
%\iffalse
%<*samplepart3>
%\fi
% Some text for part 3:
%    \begin{macrocode}
some text in part three
%    \end{macrocode}

%\iffalse
%</samplepart3>
%\fi
% Some text for part 4:
%\iffalse
%<*samplepart4>
%\fi
%    \begin{macrocode}
more text in part four
%    \end{macrocode}

%\iffalse
%</samplepart4>
%\fi
%
% %%%%%%%%%%%%%%%%%%%%%%%%%%%%%%%%%%%%%%
% \paragraph{Forwarding for a Complete Draft.}
%
% The following forwarding file |cdocsdrf.tex|
% compiles the main document in draft mode:
%\iffalse
%<*sampledraft>
%\fi
%    \begin{macrocode}
\def\version{draft}
\input{childdoc.def}
\childdocforward{cdocsamp}
%    \end{macrocode}

%\iffalse
%</sampledraft>
%\fi
%
% %%%%%%%%%%%%%%%%%%%%%%%%%%%%%%%%%%%%%%
% \paragraph{Forwarding for Final Version of the Chapters.}
%
% The following forwarding files |cdocsfn1.tex| and |cdocsfn2.tex|
% (with identical content)
% compile the final versions of the child documents
% |cdocsch1.tex| and |cdocsch2.tex|, respectively:
%\iffalse
%<*samplefinal>
%\fi
%    \begin{macrocode}
\def\version{final}
\input{childdoc.def}
\childdocforwardprefix[cdocsamp]{cdocsfn}{cdocsch}
%    \end{macrocode}

%\iffalse
%</samplefinal>
%\fi
%
% %%%%%%%%%%%%%%%%%%%%%%%%%%%%%%%%%%%%%%
% \paragraph{Command Line Processing.}
%
% The following three command lines generate the output files
% |cdocscld|, |cdocscl1| and |cdocscl2|
% which should be identical to
% |cdocsdrf|, |cdocsch1| and |cdocsfn2|, respectively:
% \begin{center}
% \begin{tabular}{l}
% |latex -jobname cdocscld \|\\
% |  "\def\version{draft}\input{childdoc.def}\childdocforward{cdocsamp}"|\\
% |latex -jobname cdocscl1 \|\\
% |  "\input{childdoc.def}\childdocforward[cdocsamp]{cdocsch1}"|\\
% |latex -jobname cdocscl2 \|\\
% |  "\def\version{final}\input{childdoc.def}\childdocforward{cdocsch2}"|
% \end{tabular}
% \end{center}
% Note that the trailing backslash on each first line
% merely continues the input to the second line
% (for convenient cut ant paste).
% Furthermore, the command |latex| can be replaced by any
% of its alternative versions such as |pdflatex|.
%
% %%%%%%%%%%%%%%%%%%%%%%%%%%%%%%%%%%%%%%%%%%%%%%%%%%%%%%%%%%%%%%%%%%%%%%%%%%%%%%
% %%%%%%%%%%%%%%%%%%%%%%%%%%%%%%%%%%%%%%%%%%%%%%%%%%%%%%%%%%%%%%%%%%%%%%%%%%%%%%
% \section{Implementation}
%\iffalse
%<*package>
%\fi
%
% This section describes the definitions file |childdoc.def|.

% The definitions cannot be loaded using |\usepackage| or |\RequirePackage|
% which has a mechanism to prevent loading a style file more than once.
% When loading the definitions by means of |\input|
% multiple instances have to be prevented manually:
%\iffalse
%This code needs to be before the `\ProvidesFile' directive
%which is defined at the beginning of this file.
%Therefore it is also placed there and commented out here.
%</package>
%<*discard>
%\fi
%    \begin{macrocode}
\ifdefined\childdocmain\endinput\fi
%    \end{macrocode}
%\iffalse
%</discard>
%<*package>
%\fi
%
% \macro{\ifchilddoc}
% \macro{\ifchilddocmanual}
% The conditional |\ifchilddoc| tells whether a
% child (true) or main (false) document is being compiled.
% The conditional |\ifchilddocmanual| tells whether
% the |\includeonly| mechanism is used (false) or
% the selection of child files must be performed manually (true).
% The definitions initialise to false:
%    \begin{macrocode}
\newif\ifchilddoc
\newif\ifchilddocmanual
%    \end{macrocode}

% \macro{\childdocname}
% \macro{\childdocjob}
% The macro |\childdocname| stores the name of the main document
% to be compiled. The macro |\childdocjob| stores the name of
% the document on which the \LaTeX{} compiler was originally invoked.
% The content of |\jobname| cannot be compared
% to filenames specified in the source due to different catcodes.
% The following code rescans |\jobname|, stores the result
% in |\childdocname| and saves a copy in |\childdocjob|:
%    \begin{macrocode}
\edef\childdocname{\scantokens\expandafter{\jobname\noexpand}}
\let\childdocjob\childdocname
%    \end{macrocode}

% \macro{\childdocdisable}
% The macro |\childdocdisable| prevents the main file
% from being processed more than once.
% At this stage, the main document command |\childdocmain|
% is assumed to be called once again where it should do nothing.
% Any subsequent call to it should prevent
% a secondary processing of the main document
% It overwrites the forwarding commands
% |\childdocof| and |\childdocforward|
% with empty macros to prevent further inclusions of the main document:
%    \begin{macrocode}
\newcommand{\childdocdisable}
{
  \renewcommand{\childdocmain}[1]{\renewcommand{\childdocmain}[1]{\endinput}}
  \renewcommand{\childdocof}[1]{}
  \renewcommand{\childdocby}[2][]{}
  \renewcommand{\childdocforward}[2][]{}
  \renewcommand{\childdocdisable}{}
}
%    \end{macrocode}

% \macro{\childdocmain}
% The macro |\childdocmain| is to be called at the top of the main file
% with nothing or the main filename (without extension) as argument.
% First, it breaks loops.
% If the argument is not empty and does not match |\childdocname|
% (which is set by the first inclusion of |childdoc.def|),
% |\ifchilddoc| is set to true, |\includeonly| is applied to the child file
% and |\jobname| is set to the main file
% (for proper handling of |.aux| files):
%    \begin{macrocode}
\newcommand{\childdocmain}[1]
{
  \childdocdisable\childdocmain{}
  \if?#1?\else
    \begingroup
      \def\childdoctmp{#1}
      \ifx\childdoctmp\childdocname
        \def\childdoctmp{}
      \else
        \def\childdoctmp
        {
          \childdoctrue
          \includeonly{\childdocname}
          \def\childdocjob{#1}
          \def\jobname{#1}
        }
      \fi
      \expandafter
    \endgroup
    \childdoctmp
  \fi
}
%    \end{macrocode}

% \macro{\childdocof}
% The command |\childdocof| redirects
% compilation to the main file |#1|.
%    \begin{macrocode}
\newcommand{\childdocof}[1]
{
  \childdocdisable
  \childdoctrue
  \includeonly{\childdocname}
  \def\jobname{#1}
  \def\childdocjob{#1}
  \input{#1}
}
%    \end{macrocode}

% \macro{\childdocby}
% The command |\childdocby| ....
%    \begin{macrocode}
\newcommand{\childdocby}[2][]
{
  \childdocdisable
  \childdoctrue
  \childdocmanualtrue
  \if?#1?\else
    \def\jobname{#2}
  \fi
  \def\childdocjob{#2}
  \input{#2}
  \endinput
}
%    \end{macrocode}

% \macro{\childdocforward}
% The command |\childdocforward| redirects
% compilation to the main file or
% (if the optional argument is given) a child file.
% Parameters are set as if the main file
% or a child file starting with |\childdocof| was compiled.
% Then compilation is handed over to the main file:
%    \begin{macrocode}
\newcommand{\childdocforward}[2][]
{
  \begingroup
    \if?#1?
      \def\childdoctmp
      {
        \def\childdocname{#2}
        \def\childdocjob{#2}
        \def\jobname{#2}
        \input{#2}
        \endinput
      }
    \else
      \def\childdoctmp
      {
        \childdocdisable
        \def\childdocname{#2}
        \childdoctrue
        \includeonly{#2}
        \def\childdocjob{#1}
        \def\jobname{#1}
        \input{#1}
        \endinput
      }
    \fi
    \expandafter
  \endgroup
  \childdoctmp
}
%    \end{macrocode}

% \macro{\childdocforwardprefix}
% The command |\childdocforwardprefix| redirects
% compilation to the main or a child file by means of a pattern.
% The prefix |#1| in the current filename is replaced by |#2|
% and the suffix of the current filename is kept
% (it is assumed that the filename does not contain the substring `|~~~|'
% which is used as a delimiter).
% Compilation is handed over to the new file by |\childdocforward|:
%    \begin{macrocode}
\newcommand{\childdocforwardprefix}[3][]
{
  \begingroup
    \def\childdocextract #2##1~~~{\def\childdoctmp{\childdocforward[#1]{#3##1}}}
    \expandafter\childdocextract\childdocname~~~
    \expandafter
  \endgroup
  \childdoctmp
}
%    \end{macrocode}

% \macro{\childdoc}
% The deprecated macro |\childdoc| is a legacy version of |\childdocmain|:
%    \begin{macrocode}
\newcommand{\childdoc}{\childdocmain}
%    \end{macrocode}

% \macro{\childdocredirect}
% The deprecated macro |\childdocredirect| is a legacy version
% of |\childdocforward| and |\childdocforwardprefix|:
%    \begin{macrocode}
\newcommand{\childdocredirect}[2][]
{
  \begingroup
    \if?#1?
      \def\childdoctmp{\childdocforward{#2}}
    \else
      \def\childdoctmp{\childdocforwardprefix{#1}{#2}}
    \fi
    \expandafter
  \endgroup
  \childdoctmp
}
%    \end{macrocode}

%\iffalse
%</package>
%\fi
%
\endinput

\childdocof{cdocsamp}
%    \end{macrocode}

%\iffalse
%</samplechap1|samplechap2>
%\fi
%
%\iffalse
%<*samplechap1>
%\fi
% Some text for chapter 1:
%    \begin{macrocode}
\section{one}
some text in chapter one
%    \end{macrocode}

%\iffalse
%</samplechap1>
%\fi
% Some text for chapter 2:
%\iffalse
%<*samplechap2>
%\fi
%    \begin{macrocode}
\section{two}
more text in chapter two
%    \end{macrocode}

%\iffalse
%</samplechap2>
%\fi
%
% %%%%%%%%%%%%%%%%%%%%%%%%%%%%%%%%%%%%%%
% \paragraph{Part Include Files.}
%
% The include files are called |cdocspt3.tex| and |cdocspt4.tex|.
%
%\iffalse
%<*samplepart3|samplepart4>
%\fi

% Optional override for |\version| flag:
%    \begin{macrocode}
%%\providecommand{\version}{final}
%    \end{macrocode}

% Include the main document:
%    \begin{macrocode}
% \iffalse
%
% childdoc.dtx Copyright (C) 2017-2018 Niklas Beisert
%
% This work may be distributed and/or modified under the
% conditions of the LaTeX Project Public License, either version 1.3
% of this license or (at your option) any later version.
% The latest version of this license is in
%   http://www.latex-project.org/lppl.txt
% and version 1.3 or later is part of all distributions of LaTeX
% version 2005/12/01 or later.
%
% This work has the LPPL maintenance status `maintained'.
%
% The Current Maintainer of this work is Niklas Beisert.
%
% This work consists of the files childdoc.dtx and childdoc.ins
% and the derived files childdoc.def and cdocsamp.tex with
% cdocsch1.tex, cdocsch2.tex, cdocsdrf.tex, cdocsfn1.tex, cdocsfn2.tex.
%
%<package>\ifdefined\childdocmain\endinput\fi
%<package>\ProvidesFile{childdoc.def}[2018/12/30 v2.0 child document driver]
%<samplemain>\ProvidesFile{cdocsamp.tex}[2018/12/30 v2.0 sample for childdoc]
%<*driver>
%\ProvidesFile{childdoc.drv}[2018/12/30 v2.0 childdoc reference manual file]
\PassOptionsToClass{10pt,a4paper}{article}
\documentclass{ltxdoc}

\usepackage[margin=35mm]{geometry}
\usepackage{hyperref}
\usepackage{hyperxmp}
\usepackage[usenames]{color}

\hypersetup{colorlinks=true}
\hypersetup{pdfstartview=FitH}
\hypersetup{pdfpagemode=UseNone}
\hypersetup{pdfsource={}}
\hypersetup{pdflang={en-UK}}
\hypersetup{pdfcopyright={Copyright 2017-2018 Niklas Beisert.
  This work may be distributed and/or modified under the
  conditions of the LaTeX Project Public License, either version 1.3
  of this license or (at your option) any later version.}}
\hypersetup{pdflicenseurl={http://www.latex-project.org/lppl.txt}}
\hypersetup{pdfcontactaddress={ETH Zurich, ITP, HIT K,
  Wolfgang-Pauli-Strasse 27}}
\hypersetup{pdfcontactpostcode={8093}}
\hypersetup{pdfcontactcity={Zurich}}
\hypersetup{pdfcontactcountry={Switzerland}}
\hypersetup{pdfcontactemail={nbeisert@itp.phys.ethz.ch}}
\hypersetup{pdfcontacturl={http://people.phys.ethz.ch/\xmptilde nbeisert/}}

\newcommand{\secref}[1]{\hyperref[#1]{section \ref*{#1}}}

\parskip1ex
\parindent0pt
\let\olditemize\itemize
\def\itemize{\olditemize\parskip0pt}

\begin{document}

\title{The \textsf{childdoc} Package}
\hypersetup{pdftitle={The childdoc Package}}
\author{Niklas Beisert\\[2ex]
  Institut f\"ur Theoretische Physik\\
  Eidgen\"ossische Technische Hochschule Z\"urich\\
  Wolfgang-Pauli-Strasse 27, 8093 Z\"urich, Switzerland\\[1ex]
  \href{mailto:nbeisert@itp.phys.ethz.ch}
  {\texttt{nbeisert@itp.phys.ethz.ch}}}
\hypersetup{pdfauthor={Niklas Beisert}}
\hypersetup{pdfsubject={Manual for the LaTeX2e Package childdoc}}
\date{30 December 2018, \textsf{v2.0}}
\maketitle

\begin{abstract}\noindent
\textsf{childdoc} is a \LaTeXe{} package
that enables the direct compilation
of document sections included by |\include|
to individual files.
\end{abstract}

\begingroup
\parskip0ex
\tableofcontents
\endgroup

%%%%%%%%%%%%%%%%%%%%%%%%%%%%%%%%%%%%%%%%%%%%%%%%%%%%%%%%%%%%%%%%%%%%%%%%%%%%%%%%
%%%%%%%%%%%%%%%%%%%%%%%%%%%%%%%%%%%%%%%%%%%%%%%%%%%%%%%%%%%%%%%%%%%%%%%%%%%%%%%%
\section{Introduction}

\LaTeX{} provides a mechanism to structure a large document (such as a book)
into a main file and several child files (containing the chapters)
using the |\include| command.
This mechanism is beneficial for documents
which span hundreds of pages in order to
make the source file(s) more manageable.
Moreover, compilation can be restricted to
selected child files by means of the |\includeonly| command.
The latter feature can be used to reduce the compilation time while editing
(this was significantly more useful in the earlier days of \LaTeX{})
or to generate a smaller document which is easier to navigate.
Another application of |\includeonly| is to generate
documents consisting of selected parts of the complete document.

However, there are a few drawbacks of the plain |\include| mechanism:
\begin{itemize}
\item
The child files cannot be compiled on their own,
they can only be compiled via the main file.
A naive editing environment
(such as a text editor with an option
to have the current file processed by \LaTeX)
may require one to switch to the main file before compiling;
attempting to compile the child file produces errors.
\item
The main file must be modified (each time)
to adjust the |\includeonly| command
to the present needs. This easily leaves the main file in a messy state.
\item
The generated document will always carry the filename
of the main document. This is inconvenient if
several child files are to be compiled and
to be kept for distribution.
\end{itemize}

The present package provides a simple interface
to make child files individually compilable by \LaTeX{}.
Compiling a child file then has the same effect as compiling
the main file with an |\includeonly| command
to select the appropriate child.
Moreover the generated document will carry the name of the child
rather than the main file.
This resolves all three above issues.

This feature is meant to make the editing of books,
thesis documents and lecture notes somewhat more convenient.
However, the package can also be used efficiently for
composing a series of documents (such as exercise sheets)
which are typically distributed individually.
It then assists the author in generating the individual documents
(potentially in different versions)
as well as a document containing the collected series.
Another application is in developing style files
or other kinds of included material
where compilation of the style file could redirect
to a sample or test file.

%%%%%%%%%%%%%%%%%%%%%%%%%%%%%%%%%%%%%%%%%%%%%%%%%%%%%%%%%%%%%%%%%%%%%%%%%%%%%%%%
%%%%%%%%%%%%%%%%%%%%%%%%%%%%%%%%%%%%%%%%%%%%%%%%%%%%%%%%%%%%%%%%%%%%%%%%%%%%%%%%
\section{Usage}

First of all, the package \textsf{childdoc} is \emph{not} a standard
\LaTeXe{} |.sty| style file! Therefore it needs to be invoked in
a non-standard way.

%%%%%%%%%%%%%%%%%%%%%%%%%%%%%%%%%%%%%%%%%%%%%%%%%%%%%%%%%%%%%%%%%%%%%%%%%%%%%%%%
\subsection{Included Files}
\label{sec:include}

%%%%%%%%%%%%%%%%%%%%%%%%%%%%%%%%%%%%%%%%
\DescribeMacro{\childdocmain}
To use the package, add the commands
\begin{center}
\begin{tabular}{l}
|\input{childdoc.def}|\\
|\childdocmain{}|\\
\end{tabular}
\end{center}
at the very top of the main \LaTeX{} file,
in particular \emph{before} the |\documentclass| statement!
The argument of |\childdocmain| should be left empty
(but it must be present).

%%%%%%%%%%%%%%%%%%%%%%%%%%%%%%%%%%%%%%%%
\DescribeMacro{\childdocof}
Furthermore, add the commands
\begin{center}
\begin{tabular}{l}
|\input{childdoc.def}|\\
|\childdocof{|\textit{main}|}|\\
\end{tabular}
\end{center}
at the top of every child file \textit{child}
which is included by |\include{|\textit{child}|}|
from within the main file
(or at least for those files to be compiled individually).
The argument \textit{main} must be the filename of the main file.

There are a couple of
considerations in setting up the main and child documents:

%%%%%%%%%%%%%%%%%%%%%%%%%%%%%%%%%%%%%%%%
\paragraph{Restrictions.}

Please note the following restrictions:
\begin{itemize}
\item
|\childdocmain| must be called with one argument \textit{main}
to ensure compatibility with earlier version of the package.
It must either be empty (|\childdocmain{}|)
or precisely match the filename of the main file in which it is specified.
See \secref{sec:detection} for further information.
\item
The filename \textit{main} must be specified without the |.tex| extension.
\item
The filename \textit{main} is case sensitive
(even in case-insensitive file systems)
due to internal string comparison.
\item
The argument \textit{main} should be fully expanded, it cannot be a macro.
\item
Subdirectories and special characters should be avoided in filenames.
\item
The command |\childdocmain{|\textit{main}|}| must be followed by a whitespace.
It should not be followed immediately by another command
or by a comment mark `|%|'.
This is because the \TeX{} parser reads the token immediately following
the argument of |\childdocmain| and puts it
at the beginning of every child section;
however, a white\-space is ignored.
\end{itemize}

%%%%%%%%%%%%%%%%%%%%%%%%%%%%%%%%%%%%%%%%
\paragraph{Content of Main File.}

It is advisable to place all content in the child files included by |\include|.
Any output contained in the main file will appear in all child documents
unless suppressed manually;
it cannot be suppressed automatically by the |\includeonly| directive
and thus should normally be avoided.
A method to include some content in the main file
by means of conditional processing is described in \secref{sec:conditional}.

%%%%%%%%%%%%%%%%%%%%%%%%%%%%%%%%%%%%%%%%
\paragraph{Page Numbering.}

When only a part of the document is compiled,
the appropriate numbering of pages
(as well as other status parameters)
is determined from the |.aux| files.
The latter contain information from previous passes.
However this information needs to propagate through
all intermediate child documents.
Therefore the page numbering in child documents may well
be inconsistent until the complete document is compiled at least once.

A useful (if unconventional) way to always ensure a consistent
page numbering is to restart the numbering in each child document
and denote the pages by `\textit{child}|.|\textit{page}'
where \textit{child} represents the chapter/section number of the child file.
This can be achieved by the command
|\numberwithin{page}{|\textit{child}|}|
of the \textsf{amsmath} package
where \textit{child} can be |chapter| or |section|
depending on the chosen structuring.
Alternatively, one can modify the macro |\thepage| appropriately
and reset the counter |page| at the start of each child file.

%%%%%%%%%%%%%%%%%%%%%%%%%%%%%%%%%%%%%%%%%%%%%%%%%%%%%%%%%%%%%%%%%%%%%%%%%%%%%%%%
\subsection{Conditional Processing}
\label{sec:conditional}

The package provides a mechanism to compile different versions
of a document. To customise the versions further some conditional processing
can come in handy to distinguish which version is being compiled.
The package provides two macros to describe the compilation context:

%%%%%%%%%%%%%%%%%%%%%%%%%%%%%%%%%%%%%%%%
\DescribeMacro{\ifchilddoc}
The conditional |\ifchilddoc| distinguishes between the compilation of
child documents and the main document:
%
\begin{center}
|\ifchilddoc |\textit{child-code}| |[|\||else |\textit{main-code}]| \||fi|
\end{center}

%%%%%%%%%%%%%%%%%%%%%%%%%%%%%%%%%%%%%%%%
\DescribeMacro{\childdocname}
\DescribeMacro{\childdocjob}
The macro |\childdocname| contains the filename (without extension)
of the main or child file being processed.
Note that |\childdocjob| will always contain the name of the main file.

%%%%%%%%%%%%%%%%%%%%%%%%%%%%%%%%%%%%%%%%
\paragraph{Title Page.}

Conditional processing can be used to include a title or banner page
in the main document when proper precautions are taken.
Importantly, the code in the main file should ensure that the page counter
(as well as other status parameters which are stored in the |.aux| files)
takes the same value after the conditional processing.
Otherwise the page numbers may take divergent values
depending on which part is compiled.

For example, a title page could be declared by:
%
\begin{center}
\begin{tabular}{l}
|\ifchilddoc\||else|\\
|\addtocounter{page}{-1}|\\
\textit{code for title page}\\
|\newpage|\\
|\||fi|
\end{tabular}
\end{center}
%
A banner page for the child documents can be generated by:
%
\begin{center}
\begin{tabular}{l}
|\ifchilddoc|\\
|\addtocounter{page}{-1}|\\
\textit{code for banner page}\\
|\newpage|\\
|\||fi|
\end{tabular}
\end{center}
%
Here one could write a message such as:
\begin{center}
|This is the part \childdocname{} of \childdocjob{}.|
\end{center}

%%%%%%%%%%%%%%%%%%%%%%%%%%%%%%%%%%%%%%%%%%%%%%%%%%%%%%%%%%%%%%%%%%%%%%%%%%%%%%%%
\subsection{Flags}
\label{sec:flags}

The package makes it easy to generate different versions
of the main or child documents.
To this end compilation flags can be defined
and assigned different default values.
They will be particularly useful in conjunction
with the forwarding mechanism described in \secref{sec:forward}.

For example, it may be useful to have a flag |\version|
which can be set to |draft| or |final|.
The document source will contain some conditional code
depending on the value of |\version|.
Suppose further, the flag should default to |final| for the main file
and to |draft| for child files
which is a natural assignment for editing the document.
This is achieved by placing the following code
in the preamble of the main document
(below the |\childdocmain| directive):
%
\begin{center}
\begin{tabular}{l}
|\ifchilddoc|\\
|\providecommand{\version}{draft}|\\
|\||else|\\
|\providecommand{\version}{final}|\\
|\||fi|
\end{tabular}
\end{center}
%
The definition by |\providecommand| makes sure
that previous definitions are not overwritten.
Further statements |\providecommand{\version}{...}|
can thus be added before the above code to override it.

For the main file, one might add a line
(between |\childdocmain| and the above block)
%
\begin{center}
|%\ifchilddoc\||else\providecommand{\version}{draft}\||fi|
\end{center}
%
which can be uncommented to produce a draft version.
Likewise one can add a line to the very top of a child file
(above the |\childdocof{|\textit{main}|}| directive)
%
\begin{center}
|%\providecommand{\version}{final}|
\end{center}
%
which can be uncommented to produce the final version of this child document.

%%%%%%%%%%%%%%%%%%%%%%%%%%%%%%%%%%%%%%%%%%%%%%%%%%%%%%%%%%%%%%%%%%%%%%%%%%%%%%%%
\subsection{Forwarding}
\label{sec:forward}

Different versions of the main or child documents
using compilation flags as described in \secref{sec:flags}
can be (permanently) stored in different files
for convenient compilation, viewing and distribution.
To this end, the package defines a command
to pass on compilation to a different file:

%%%%%%%%%%%%%%%%%%%%%%%%%%%%%%%%%%%%%%%%
\DescribeMacro{\childdocforward}
The command |\childdocforward| redirects processing to
another source file:
%
\begin{center}
\begin{tabular}{l}
|\input{childdoc.def}|\\
|\childdocforward[|\textit{main}|]{|\textit{dest}|}|\\
\end{tabular}
\end{center}
%
The argument \textit{dest} is the destination file
(without extension).
It should be the main file or one of the child files.
Note that further \textsf{childdoc} directives
such as |\childdocof| and |\childdocforward|
in the indicated file will be processed in this form.
The optional argument \textit{main}
passes on directly to the main file \textit{main}
while pretending to compile the child \textit{dest}.
This form behaves as if \textit{dest}
issues |\childdocof{|\textit{main}|}| right away,
and no further \textsf{childdoc} directives will be processed.

%%%%%%%%%%%%%%%%%%%%%%%%%%%%%%%%%%%%%%%%
\DescribeMacro{\...prefix}
In the alternative form |\childdocforwardprefix|,
%
\begin{center}
\begin{tabular}{l}
|\input{childdoc.def}|\\
|\childdocforwardprefix[|\textit{main}|]{|\textit{prefix}|}{|\textit{dest}|}|
\end{tabular}
\end{center}
%
the destination file is determined by a pattern
depending on the current file:
To make this work, the current file must be called
`{\textit{prefix}\hspace{0.2em}\textit{suffix}}'
with \textit{prefix} matching precisely the argument.
Processing is then passed on to the file
`{\textit{dest}\hspace{0.2em}\textit{suffix}}'.
Surely, the same effect is achieved by
directly specifying the
argument `{\textit{dest}\hspace{0.2em}\textit{suffix}}'
in the first form.
However, that requires to set up a different file
for each child. With the alternative form of the command
all these files can have exactly the same content
which simplifies setting them up and maintaining them.

For example, the following file |draft.tex|
with a compilation flag |\version| as described in \secref{sec:flags}
compiles the main document as a draft:
%
\begin{center}
\begin{tabular}{l}
|\def\version{draft}|\\
|\input{childdoc.def}|\\
|\childdocforward{|\textit{main}|}|
\end{tabular}
\end{center}
%
Likewise, the following files |final|\textit{nn}|.tex|
compile the final version of the child document
|child|\textit{nn}|.tex|:
%
\begin{center}
\begin{tabular}{l}
|\def\version{final}|\\
|\input{childdoc.def}|\\
|\childdocforwardprefix{final}{child}|
\end{tabular}
\end{center}
%

Note that when several versions of a main file and/or of each child file
are to be generated, it may be convenient to set up a |Makefile| or
shell script to automatise the process.

%%%%%%%%%%%%%%%%%%%%%%%%%%%%%%%%%%%%%%%%%%%%%%%%%%%%%%%%%%%%%%%%%%%%%%%%%%%%%%%%
\subsection{Command Line Processing}
\label{sec:commandline}

The effect of redirection files can also be achieved by invoking
the \LaTeX{} compiler with a more elaborate command line.
Most conveniently this should be done as part
of a shell script or a |Makefile|.

When using \textsf{childdoc} in the main file, the following
command lines effectively perform a redirection
(note that depending on the shell being used,
backslashes may have to be doubled: `|\|' $\to$ `|\\|'):
%
\begin{center}
|... -jobname "|\textit{target}|" |\\|"|[\textit{flags}]%
|\input{childdoc.def}\childdocforward[|\textit{main}|]{|\textit{dest}|}"|
\end{center}
%
Here \textit{target} is the name of the output file,
\textit{main} is the name of the main file
and \textit{dest} is the name of the main or child file to be processed
(all filenames without extensions).
The optional argument \textit{main} can be omitted
if \textit{main} matches \textit{dest}.
Optionally, compilation \textit{flags} can be defined via |\def| commands.
This command line makes the \TeX{} engine believe
it is compiling the file \textit{target}
whose content is specified as the latter parameter.
The provided code then forwards the processing to
\textit{main} or \textit{dest} as described in \secref{sec:forward}.

%%%%%%%%%%%%%%%%%%%%%%%%%%%%%%%%%%%%%%%%%%%%%%%%%%%%%%%%%%%%%%%%%%%%%%%%%%%%%%%%
\subsection{Include by Input}
\label{sec:input}

Including child documents by |\include| has some restrictions by design.
Most notably, the content of a child document always occupies
its own set of pages; pages cannot be shared between child documents.
Usually, this behaviour makes perfect sense
because each child document contain an essential part of the document.
However, in some situations it may be desirable to compose
a document from a collection of parts
without having mandatory page breaks between then.
For this case, the package
provides a mechanism to include parts
by |\input| which can also be processed individually.
However, by construction this mechanism
requires manual handling of the content to be output.

%%%%%%%%%%%%%%%%%%%%%%%%%%%%%%%%%%%%%%%%
\DescribeMacro{\ifchilddocmanual}
The main file should be prepared as usual, see \secref{sec:include}.
However, the document body must make a distinction
between processing of an individual part and of the main document, e.g.:
%
\begin{center}
\begin{tabular}{l}
|\ifchilddocmanual|\\
|\input{\childdocname}|\\
|\||else|\\
\textit{document body with }|\input{|\textit{part}|}|\\
|\||fi|
\end{tabular}
\end{center}
%
The conditional |\ifchilddocmanual| is true whenever
a part to be included by |\input| is being compiled,
and the name of the part is stored in |\childdocname|.

%%%%%%%%%%%%%%%%%%%%%%%%%%%%%%%%%%%%%%%%
\DescribeMacro{\childdocby}
Each part to be included by |\input| should start with:
%
\begin{center}
\begin{tabular}{l}
|\input{childdoc.def}|\\
|\childdocby{|\textit{main}|}|\\
\end{tabular}
\end{center}
%
The directive |\childdocby| is similar to |\childdocof|
described in \secref{sec:include},
but the subsequent selection of content must be done manually.
To that end, both |\ifchilddoc| and |\ifchilddocmanual|
will be true upon processing of a part,
and the name of the part is stored in |\childdocname|.
Note that |\jobname| will be set to the filename of the current part
so that each part receives an individual |.aux| file
that does not interfere with the |.aux| file(s) of the main document.
This behaviour can be altered by the alternative form
|\childdocby[*]{|\textit{main}|}| (with a non-empty optional argument)
which uses the |.aux| file of the main document
by setting |\jobname| to \textit{main}.

%%%%%%%%%%%%%%%%%%%%%%%%%%%%%%%%%%%%%%%%%%%%%%%%%%%%%%%%%%%%%%%%%%%%%%%%%%%%%%%%
\subsection{Driver Development}
\label{sec:driver}

The \textsf{childdoc} mechanism can also be use for the development
of definition files such as \LaTeX{} styles or classes.
This case differs from the above setup with multiple parts
included by |\include| in that no |\includeonly| should be invoked.
This can be achieved by starting the include file
(before |\ProvidesPackage|) with:
%
\begin{center}
\begin{tabular}{l}
|\input{childdoc.def}|\\
|\childdocforward{|\textit{main}|}|\\
\end{tabular}
\end{center}
%
or alternatively with:
%
\begin{center}
\begin{tabular}{l}
|\input{childdoc.def}|\\
|\childdocby{|\textit{main}|}|\\
\end{tabular}
\end{center}
%
Both forms have slightly different effects as described above.
The main file is prepared as usual, see \secref{sec:include}.

%%%%%%%%%%%%%%%%%%%%%%%%%%%%%%%%%%%%%%%%%%%%%%%%%%%%%%%%%%%%%%%%%%%%%%%%%%%%%%%%
\subsection{Legacy Detection}
\label{sec:detection}

The directive |\childdocmain| in the main file can detect
whether the complete document or merely a child is to be compiled
even without using the directive |\childdocof|.
This method is deprecated because it is less robust
and there is no compelling reason to use it;
it is merely provided for backward compatibility
and it may be removed in future versions.

If the detection mechanism is to be used,
it is mandatory to correctly specify
the filename of the main file as the argument of |\childdocmain|:
%
\begin{center}
\begin{tabular}{l}
|\input{childdoc.def}|\\
|\childdocmain{|\textit{main}|}|\\
\end{tabular}
\end{center}
%
If |\jobname| does not match the argument \textit{main} of |\childdocmain|,
it is assumed that |\jobname| points to the child file to be compiled.
When using |\childdocmain| with the main file specified as argument,
it suffices to start a child file
with just |\input{|\textit{main}|}|
without loading of the package and using |\childdocof|.
If instead all processing is done
with the appropriate \textsf{childdoc} directives,
the argument of \textit{main} of |\childdocmain| can be empty.

An alternative version of the command line processing described
in \secref{sec:commandline} using the detection mechanism reads:
%
\begin{center}
|... -jobname "|\textit{target}|" "|[\textit{flags}]%
[|\def\jobname{|\textit{dest}|}|]|\input{|\textit{main}|}"|
\end{center}

%%%%%%%%%%%%%%%%%%%%%%%%%%%%%%%%%%%%%%%%%%%%%%%%%%%%%%%%%%%%%%%%%%%%%%%%%%%%%%%%
\subsection{Manual Code}
\label{sec:manual}

In case one cannot be certain whether the definitions file |childdoc.def|
is installed on the target \TeX{} distribution
and one prefers not to ship it,
it is conceivable to paste a few relevant commands into the sources.

To that end, drop all statements |\input{childdoc.def}|
and perform the replacements as outlined below.
Instead of |\childdocmain{|\textit{main}|}| add the following code
to the top of the main file:
%
\begin{center}
\begin{tabular}{l}
|\||ifdefined\childdocname\endinput\||fi\newif\ifchilddoc|\\
|\edef\childdocname{\scantokens\expandafter{\jobname\noexpand}}|\\
|\def\childdocmain{|\textit{main}|}\||ifx\childdocmain\childdocname\||else|\\
|\childdoctrue\includeonly{\childdocname}\let\jobname\childdocmain\||fi|\\
\end{tabular}
\end{center}
%
Instead of |\childdocof{|\textit{main}|}| just include the main file
at the top of each child file:
%
\begin{center}
|\input{|\textit{main}|}|
\end{center}
%
A simple redirection |\childdocforward{|\textit{dest}|}| is achieved by:
%
\begin{center}
|\def\jobname{|\textit{dest}|}\input{\jobname}|
\end{center}
%
The redirection with prefix
|\childdocforwardprefix[|\textit{prefix}|]{|\textit{dest}|}|
is accomplished by:
%
\begin{center}
\begin{tabular}{l}
|{\edef\jobname{\scantokens\expandafter{\jobname\noexpand}}|\\
|\def\redirectjob |\textit{prefix}|#1~~~{\gdef\jobname{|\textit{dest}|#1}}|\\
|\expandafter\redirectjob\jobname~~~}\input{\jobname}|
\end{tabular}
\end{center}

In an alternative approach,
child documents can be compiled by a specific command line
without additional code or specific definitions:
%
\begin{center}
|... -jobname "|\textit{target}|" "|[\textit{flags}]%
|\includeonly{|\textit{dest}|}\input{|\textit{main}|}"|
\end{center}
%

%%%%%%%%%%%%%%%%%%%%%%%%%%%%%%%%%%%%%%%%%%%%%%%%%%%%%%%%%%%%%%%%%%%%%%%%%%%%%%%%
%%%%%%%%%%%%%%%%%%%%%%%%%%%%%%%%%%%%%%%%%%%%%%%%%%%%%%%%%%%%%%%%%%%%%%%%%%%%%%%%
\section{Information}

%%%%%%%%%%%%%%%%%%%%%%%%%%%%%%%%%%%%%%%%%%%%%%%%%%%%%%%%%%%%%%%%%%%%%%%%%%%%%%%%
\subsection{Copyright}

Copyright \copyright{} 2017--2018 Niklas Beisert

This work may be distributed and/or modified under the
conditions of the \LaTeX{} Project Public License, either version 1.3
of this license or (at your option) any later version.
The latest version of this license is in
  \url{http://www.latex-project.org/lppl.txt}
and version 1.3 or later is part of all distributions of \LaTeX{}
version 2005/12/01 or later.

This work has the LPPL maintenance status `maintained'.

The Current Maintainer of this work is Niklas Beisert.

This work consists of the files |README.txt|, |childdoc.ins| and |childdoc.dtx|
as well as the derived files |childdoc.def|, |cdocsamp.tex|
with |cdocsch1.tex|, |cdocsch2.tex|, |cdocspt3.tex|, |cdocspt4.tex|,
|cdocsdrf.tex|, |cdocsfn1.tex|, |cdocsfn2.tex|
as well as |childdoc.pdf|.

%%%%%%%%%%%%%%%%%%%%%%%%%%%%%%%%%%%%%%%%%%%%%%%%%%%%%%%%%%%%%%%%%%%%%%%%%%%%%%%%
\subsection{Files and Installation}

The package consists of the files:
%
\begin{center}
\begin{tabular}{ll}
    |README.txt|   & readme file \\
    |childdoc.ins| & installation file \\
    |childdoc.dtx| & source file \\
    |childdoc.def| & definition file \\
    |cdocsamp.tex| & sample main file \\
    |cdocsch1.tex| & sample include file \\
    |cdocsch2.tex| & sample include file \\
    |cdocspt3.tex| & sample part file \\
    |cdocspt4.tex| & sample part file \\
    |cdocsdrf.tex| & sample redirection file \\
    |cdocsfn1.tex| & sample redirection file \\
    |cdocsfn2.tex| & sample redirection file \\
    |childdoc.pdf| & manual
\end{tabular}
\end{center}
%
The distribution consists of the files
|README.txt|, |childdoc.ins| and |childdoc.dtx|.
%
\begin{itemize}
\item
Run (pdf)\LaTeX{} on |childdoc.dtx|
to compile the manual |childdoc.pdf| (this file).
\item
Run \LaTeX{} on |childdoc.ins| to create the definitions file |childdoc.def|
and the sample |cdocsamp.tex| with include files
|cdocsch1.tex|, |cdocsch2.tex|, |cdocspt3.tex|, |cdocspt4.tex|,
|cdocsdrf.tex|, |cdocsfn1.tex|, |cdocsfn2.tex|.
Then copy the file |childdoc.def| to an appropriate directory of your \LaTeX{}
distribution, e.g.\ \textit{texmf-root}|/tex/latex/childdoc|.
\end{itemize}

%%%%%%%%%%%%%%%%%%%%%%%%%%%%%%%%%%%%%%%%%%%%%%%%%%%%%%%%%%%%%%%%%%%%%%%%%%%%%%%%
\subsection{Related CTAN Packages}

There are several other packages which offer a similar functionality:
%
\begin{itemize}
\item
The packages
\href{http://ctan.org/pkg/docmute}{\textsf{docmute}},
\href{http://ctan.org/pkg/includex}{\textsf{includex}} and
\href{http://ctan.org/pkg/standalone}{\textsf{standalone}}
provide commands to include only the document body of
a child file thus allowing both files to be compiled individually.
\item
The packages \href{http://ctan.org/pkg/subdocs}{\textsf{subdocs}}
and \href{http://ctan.org/pkg/subfiles}{\textsf{subfiles}}
provide structures in which the main and child documents can be
encapsulated and allowing them to be compiled individually.
The inclusion mechanism is different from the conventional |\include|.
\item
The package \href{http://ctan.org/pkg/combine}{\textsf{combine}}
is an elaborate solution to combine several documents into one.
\end{itemize}
%
See also the CTAN topic \href{http://ctan.org/topic/subdocs}{\textsf{subdocs}}
for further related packages.
The present package differs from the above solutions in that
a document structure constructed with the conventional |\include| mechanism
just needs two extra commands at the top of every file
such that all constituent files can be compiled individually.

%%%%%%%%%%%%%%%%%%%%%%%%%%%%%%%%%%%%%%%%%%%%%%%%%%%%%%%%%%%%%%%%%%%%%%%%%%%%%%%%
%\subsection{Feature Suggestions}
%
%The following is a list of features which may be useful for future
%versions of this package:
%%
%\begin{itemize}
%\item
%\ldots
%\end{itemize}

%%%%%%%%%%%%%%%%%%%%%%%%%%%%%%%%%%%%%%%%%%%%%%%%%%%%%%%%%%%%%%%%%%%%%%%%%%%%%%%%
\subsection{Revision History}

%%%%%%%%%%%%%%%%%%%%%%%%%%%%%%%%%%%%%%%%
\paragraph{v2.0:} 2018/12/30

\begin{itemize}
\item
immediate forward processing
\item
added |\childdocby| mechanism
\item
manual restructured
\end{itemize}

%%%%%%%%%%%%%%%%%%%%%%%%%%%%%%%%%%%%%%%%
\paragraph{v1.6:} 2018/01/17

\begin{itemize}
\item
application for development of include files
\item
corrections to manual
\end{itemize}

%%%%%%%%%%%%%%%%%%%%%%%%%%%%%%%%%%%%%%%%
\paragraph{v1.5:} 2017/05/21

\begin{itemize}
\item
more complete structuring introduced
\item
|\childdocof| introduced
\item
|\childdoc| renamed to |\childdocmain|
\item
|\childredirect| renamed to |\childdocforward| and |\childdocforwardprefix|
and functionality expanded
\end{itemize}

%%%%%%%%%%%%%%%%%%%%%%%%%%%%%%%%%%%%%%%%
\paragraph{v1.0:} 2017/04/27

\begin{itemize}
\item
manual and install package
\item
first version published on CTAN
\end{itemize}

%%%%%%%%%%%%%%%%%%%%%%%%%%%%%%%%%%%%%%%%
\paragraph{v0.6:} 2017/04/26

\begin{itemize}
\item
redirection mechanism added
\end{itemize}

%%%%%%%%%%%%%%%%%%%%%%%%%%%%%%%%%%%%%%%%
\paragraph{v0.5:} 2017/04/26

\begin{itemize}
\item
functionality in definition file
\end{itemize}


%%%%%%%%%%%%%%%%%%%%%%%%%%%%%%%%%%%%%%%%%%%%%%%%%%%%%%%%%%%%%%%%%%%%%%%%%%%%%%%%
%%%%%%%%%%%%%%%%%%%%%%%%%%%%%%%%%%%%%%%%%%%%%%%%%%%%%%%%%%%%%%%%%%%%%%%%%%%%%%%%
%%%%%%%%%%%%%%%%%%%%%%%%%%%%%%%%%%%%%%%%%%%%%%%%%%%%%%%%%%%%%%%%%%%%%%%%%%%%%%%%
\appendix

\settowidth\MacroIndent{\rmfamily\scriptsize 000\ }

 \DocInput{childdoc.dtx}

\end{document}
%</driver>
% \fi
%
% %%%%%%%%%%%%%%%%%%%%%%%%%%%%%%%%%%%%%%%%%%%%%%%%%%%%%%%%%%%%%%%%%%%%%%%%%%%%%%
% %%%%%%%%%%%%%%%%%%%%%%%%%%%%%%%%%%%%%%%%%%%%%%%%%%%%%%%%%%%%%%%%%%%%%%%%%%%%%%
% \section{Sample}
%\iffalse
%<*samplemain>
%\fi
%
% The following presents a sample document
% with two chapters, two parts, a title page,
% a compile flag as well as three forwarding files to set the flag.
% It consists of eight |.tex| files:
% \begin{center}
% \begin{tabular}{ll}
% |cdocsamp.tex|&main file\\
% |cdocsch1.tex|&include file for chapter 1\\
% |cdocsch2.tex|&include file for chapter 2\\
% |cdocspt3.tex|&include file for part 3\\
% |cdocspt4.tex|&include file for part 4\\
% |cdocsdrf.tex|&forwarding file for main file in draft mode\\
% |cdocsfi1.tex|&forwarding file for final version of chapter 1\\
% |cdocsfi2.tex|&forwarding file for final version of chapter 2\\
% \end{tabular}
% \end{center}
% Each of the eight files can be compiled directly by the \LaTeX{} compiler.
%
% %%%%%%%%%%%%%%%%%%%%%%%%%%%%%%%%%%%%%%
% \paragraph{Main File.}
%
% The main file is called |cdocsamp.tex|.
%
% Load the \textsf{childdoc} definitions and
% declare the filename for the main document:
%    \begin{macrocode}
\input{childdoc.def}
\childdocmain{}
%    \end{macrocode}

% Optional override for |\version| flag:
%    \begin{macrocode}
%%\ifchilddoc\else\providecommand{\version}{draft}\fi
%    \end{macrocode}

% Define the default values for the |\version| flag
% (|final| for the main file and |draft| for childs):
%    \begin{macrocode}
\ifchilddoc
\providecommand{\version}{draft}
\else
\providecommand{\version}{final}
\fi
%    \end{macrocode}

% Load the standard document class:
%    \begin{macrocode}
\documentclass[12pt]{article}
%    \end{macrocode}

% Start the document body:
%    \begin{macrocode}
\begin{document}
%    \end{macrocode}

% Declare a title page.
% Print title, part of document being processed and version flag:
%    \begin{macrocode}
\addtocounter{page}{-1}
\begin{center}
{\LARGE\bfseries{}childdoc example\par}
\vspace{1cm}
\ifchilddoc
\ifchilddocmanual part\else chapter\fi:
`\childdocname' of `\childdocjob'\par
\else
main document: `\childdocjob'\par
\fi
version: \version\par
\end{center}
\newpage
%    \end{macrocode}

% Manually include selected file,
% otherwise process as usual:
%    \begin{macrocode}
\ifchilddocmanual
\section*{part `\childdocname'}
\input{\childdocname}
\else
%    \end{macrocode}

% Include the two chapters:
%    \begin{macrocode}
\include{cdocsch1}
\include{cdocsch2}
%    \end{macrocode}

% Include the two parts unless only chapters should be displayed:
%    \begin{macrocode}
\ifchilddoc\else
\section{part three}
\input{cdocspt3}
\section{part four}
\input{cdocspt4}
\fi
%    \end{macrocode}

% Process as usual until here:
%    \begin{macrocode}
\fi
%    \end{macrocode}

% End of document body:
%    \begin{macrocode}
\end{document}
%    \end{macrocode}
%\iffalse
%</samplemain>
%\fi
%
% %%%%%%%%%%%%%%%%%%%%%%%%%%%%%%%%%%%%%%
% \paragraph{Chapter Include Files.}
%
% The include files are called |cdocsch1.tex| and |cdocsch2.tex|.
%
%\iffalse
%<*samplechap1|samplechap2>
%\fi

% Optional override for |\version| flag:
%    \begin{macrocode}
%%\providecommand{\version}{final}
%    \end{macrocode}

% Include the main document:
%    \begin{macrocode}
\input{childdoc.def}
\childdocof{cdocsamp}
%    \end{macrocode}

%\iffalse
%</samplechap1|samplechap2>
%\fi
%
%\iffalse
%<*samplechap1>
%\fi
% Some text for chapter 1:
%    \begin{macrocode}
\section{one}
some text in chapter one
%    \end{macrocode}

%\iffalse
%</samplechap1>
%\fi
% Some text for chapter 2:
%\iffalse
%<*samplechap2>
%\fi
%    \begin{macrocode}
\section{two}
more text in chapter two
%    \end{macrocode}

%\iffalse
%</samplechap2>
%\fi
%
% %%%%%%%%%%%%%%%%%%%%%%%%%%%%%%%%%%%%%%
% \paragraph{Part Include Files.}
%
% The include files are called |cdocspt3.tex| and |cdocspt4.tex|.
%
%\iffalse
%<*samplepart3|samplepart4>
%\fi

% Optional override for |\version| flag:
%    \begin{macrocode}
%%\providecommand{\version}{final}
%    \end{macrocode}

% Include the main document:
%    \begin{macrocode}
\input{childdoc.def}
\childdocby{cdocsamp}
%    \end{macrocode}

%\iffalse
%</samplepart3|samplepart4>
%\fi
%
%\iffalse
%<*samplepart3>
%\fi
% Some text for part 3:
%    \begin{macrocode}
some text in part three
%    \end{macrocode}

%\iffalse
%</samplepart3>
%\fi
% Some text for part 4:
%\iffalse
%<*samplepart4>
%\fi
%    \begin{macrocode}
more text in part four
%    \end{macrocode}

%\iffalse
%</samplepart4>
%\fi
%
% %%%%%%%%%%%%%%%%%%%%%%%%%%%%%%%%%%%%%%
% \paragraph{Forwarding for a Complete Draft.}
%
% The following forwarding file |cdocsdrf.tex|
% compiles the main document in draft mode:
%\iffalse
%<*sampledraft>
%\fi
%    \begin{macrocode}
\def\version{draft}
\input{childdoc.def}
\childdocforward{cdocsamp}
%    \end{macrocode}

%\iffalse
%</sampledraft>
%\fi
%
% %%%%%%%%%%%%%%%%%%%%%%%%%%%%%%%%%%%%%%
% \paragraph{Forwarding for Final Version of the Chapters.}
%
% The following forwarding files |cdocsfn1.tex| and |cdocsfn2.tex|
% (with identical content)
% compile the final versions of the child documents
% |cdocsch1.tex| and |cdocsch2.tex|, respectively:
%\iffalse
%<*samplefinal>
%\fi
%    \begin{macrocode}
\def\version{final}
\input{childdoc.def}
\childdocforwardprefix[cdocsamp]{cdocsfn}{cdocsch}
%    \end{macrocode}

%\iffalse
%</samplefinal>
%\fi
%
% %%%%%%%%%%%%%%%%%%%%%%%%%%%%%%%%%%%%%%
% \paragraph{Command Line Processing.}
%
% The following three command lines generate the output files
% |cdocscld|, |cdocscl1| and |cdocscl2|
% which should be identical to
% |cdocsdrf|, |cdocsch1| and |cdocsfn2|, respectively:
% \begin{center}
% \begin{tabular}{l}
% |latex -jobname cdocscld \|\\
% |  "\def\version{draft}\input{childdoc.def}\childdocforward{cdocsamp}"|\\
% |latex -jobname cdocscl1 \|\\
% |  "\input{childdoc.def}\childdocforward[cdocsamp]{cdocsch1}"|\\
% |latex -jobname cdocscl2 \|\\
% |  "\def\version{final}\input{childdoc.def}\childdocforward{cdocsch2}"|
% \end{tabular}
% \end{center}
% Note that the trailing backslash on each first line
% merely continues the input to the second line
% (for convenient cut ant paste).
% Furthermore, the command |latex| can be replaced by any
% of its alternative versions such as |pdflatex|.
%
% %%%%%%%%%%%%%%%%%%%%%%%%%%%%%%%%%%%%%%%%%%%%%%%%%%%%%%%%%%%%%%%%%%%%%%%%%%%%%%
% %%%%%%%%%%%%%%%%%%%%%%%%%%%%%%%%%%%%%%%%%%%%%%%%%%%%%%%%%%%%%%%%%%%%%%%%%%%%%%
% \section{Implementation}
%\iffalse
%<*package>
%\fi
%
% This section describes the definitions file |childdoc.def|.

% The definitions cannot be loaded using |\usepackage| or |\RequirePackage|
% which has a mechanism to prevent loading a style file more than once.
% When loading the definitions by means of |\input|
% multiple instances have to be prevented manually:
%\iffalse
%This code needs to be before the `\ProvidesFile' directive
%which is defined at the beginning of this file.
%Therefore it is also placed there and commented out here.
%</package>
%<*discard>
%\fi
%    \begin{macrocode}
\ifdefined\childdocmain\endinput\fi
%    \end{macrocode}
%\iffalse
%</discard>
%<*package>
%\fi
%
% \macro{\ifchilddoc}
% \macro{\ifchilddocmanual}
% The conditional |\ifchilddoc| tells whether a
% child (true) or main (false) document is being compiled.
% The conditional |\ifchilddocmanual| tells whether
% the |\includeonly| mechanism is used (false) or
% the selection of child files must be performed manually (true).
% The definitions initialise to false:
%    \begin{macrocode}
\newif\ifchilddoc
\newif\ifchilddocmanual
%    \end{macrocode}

% \macro{\childdocname}
% \macro{\childdocjob}
% The macro |\childdocname| stores the name of the main document
% to be compiled. The macro |\childdocjob| stores the name of
% the document on which the \LaTeX{} compiler was originally invoked.
% The content of |\jobname| cannot be compared
% to filenames specified in the source due to different catcodes.
% The following code rescans |\jobname|, stores the result
% in |\childdocname| and saves a copy in |\childdocjob|:
%    \begin{macrocode}
\edef\childdocname{\scantokens\expandafter{\jobname\noexpand}}
\let\childdocjob\childdocname
%    \end{macrocode}

% \macro{\childdocdisable}
% The macro |\childdocdisable| prevents the main file
% from being processed more than once.
% At this stage, the main document command |\childdocmain|
% is assumed to be called once again where it should do nothing.
% Any subsequent call to it should prevent
% a secondary processing of the main document
% It overwrites the forwarding commands
% |\childdocof| and |\childdocforward|
% with empty macros to prevent further inclusions of the main document:
%    \begin{macrocode}
\newcommand{\childdocdisable}
{
  \renewcommand{\childdocmain}[1]{\renewcommand{\childdocmain}[1]{\endinput}}
  \renewcommand{\childdocof}[1]{}
  \renewcommand{\childdocby}[2][]{}
  \renewcommand{\childdocforward}[2][]{}
  \renewcommand{\childdocdisable}{}
}
%    \end{macrocode}

% \macro{\childdocmain}
% The macro |\childdocmain| is to be called at the top of the main file
% with nothing or the main filename (without extension) as argument.
% First, it breaks loops.
% If the argument is not empty and does not match |\childdocname|
% (which is set by the first inclusion of |childdoc.def|),
% |\ifchilddoc| is set to true, |\includeonly| is applied to the child file
% and |\jobname| is set to the main file
% (for proper handling of |.aux| files):
%    \begin{macrocode}
\newcommand{\childdocmain}[1]
{
  \childdocdisable\childdocmain{}
  \if?#1?\else
    \begingroup
      \def\childdoctmp{#1}
      \ifx\childdoctmp\childdocname
        \def\childdoctmp{}
      \else
        \def\childdoctmp
        {
          \childdoctrue
          \includeonly{\childdocname}
          \def\childdocjob{#1}
          \def\jobname{#1}
        }
      \fi
      \expandafter
    \endgroup
    \childdoctmp
  \fi
}
%    \end{macrocode}

% \macro{\childdocof}
% The command |\childdocof| redirects
% compilation to the main file |#1|.
%    \begin{macrocode}
\newcommand{\childdocof}[1]
{
  \childdocdisable
  \childdoctrue
  \includeonly{\childdocname}
  \def\jobname{#1}
  \def\childdocjob{#1}
  \input{#1}
}
%    \end{macrocode}

% \macro{\childdocby}
% The command |\childdocby| ....
%    \begin{macrocode}
\newcommand{\childdocby}[2][]
{
  \childdocdisable
  \childdoctrue
  \childdocmanualtrue
  \if?#1?\else
    \def\jobname{#2}
  \fi
  \def\childdocjob{#2}
  \input{#2}
  \endinput
}
%    \end{macrocode}

% \macro{\childdocforward}
% The command |\childdocforward| redirects
% compilation to the main file or
% (if the optional argument is given) a child file.
% Parameters are set as if the main file
% or a child file starting with |\childdocof| was compiled.
% Then compilation is handed over to the main file:
%    \begin{macrocode}
\newcommand{\childdocforward}[2][]
{
  \begingroup
    \if?#1?
      \def\childdoctmp
      {
        \def\childdocname{#2}
        \def\childdocjob{#2}
        \def\jobname{#2}
        \input{#2}
        \endinput
      }
    \else
      \def\childdoctmp
      {
        \childdocdisable
        \def\childdocname{#2}
        \childdoctrue
        \includeonly{#2}
        \def\childdocjob{#1}
        \def\jobname{#1}
        \input{#1}
        \endinput
      }
    \fi
    \expandafter
  \endgroup
  \childdoctmp
}
%    \end{macrocode}

% \macro{\childdocforwardprefix}
% The command |\childdocforwardprefix| redirects
% compilation to the main or a child file by means of a pattern.
% The prefix |#1| in the current filename is replaced by |#2|
% and the suffix of the current filename is kept
% (it is assumed that the filename does not contain the substring `|~~~|'
% which is used as a delimiter).
% Compilation is handed over to the new file by |\childdocforward|:
%    \begin{macrocode}
\newcommand{\childdocforwardprefix}[3][]
{
  \begingroup
    \def\childdocextract #2##1~~~{\def\childdoctmp{\childdocforward[#1]{#3##1}}}
    \expandafter\childdocextract\childdocname~~~
    \expandafter
  \endgroup
  \childdoctmp
}
%    \end{macrocode}

% \macro{\childdoc}
% The deprecated macro |\childdoc| is a legacy version of |\childdocmain|:
%    \begin{macrocode}
\newcommand{\childdoc}{\childdocmain}
%    \end{macrocode}

% \macro{\childdocredirect}
% The deprecated macro |\childdocredirect| is a legacy version
% of |\childdocforward| and |\childdocforwardprefix|:
%    \begin{macrocode}
\newcommand{\childdocredirect}[2][]
{
  \begingroup
    \if?#1?
      \def\childdoctmp{\childdocforward{#2}}
    \else
      \def\childdoctmp{\childdocforwardprefix{#1}{#2}}
    \fi
    \expandafter
  \endgroup
  \childdoctmp
}
%    \end{macrocode}

%\iffalse
%</package>
%\fi
%
\endinput

\childdocby{cdocsamp}
%    \end{macrocode}

%\iffalse
%</samplepart3|samplepart4>
%\fi
%
%\iffalse
%<*samplepart3>
%\fi
% Some text for part 3:
%    \begin{macrocode}
some text in part three
%    \end{macrocode}

%\iffalse
%</samplepart3>
%\fi
% Some text for part 4:
%\iffalse
%<*samplepart4>
%\fi
%    \begin{macrocode}
more text in part four
%    \end{macrocode}

%\iffalse
%</samplepart4>
%\fi
%
% %%%%%%%%%%%%%%%%%%%%%%%%%%%%%%%%%%%%%%
% \paragraph{Forwarding for a Complete Draft.}
%
% The following forwarding file |cdocsdrf.tex|
% compiles the main document in draft mode:
%\iffalse
%<*sampledraft>
%\fi
%    \begin{macrocode}
\def\version{draft}
% \iffalse
%
% childdoc.dtx Copyright (C) 2017-2018 Niklas Beisert
%
% This work may be distributed and/or modified under the
% conditions of the LaTeX Project Public License, either version 1.3
% of this license or (at your option) any later version.
% The latest version of this license is in
%   http://www.latex-project.org/lppl.txt
% and version 1.3 or later is part of all distributions of LaTeX
% version 2005/12/01 or later.
%
% This work has the LPPL maintenance status `maintained'.
%
% The Current Maintainer of this work is Niklas Beisert.
%
% This work consists of the files childdoc.dtx and childdoc.ins
% and the derived files childdoc.def and cdocsamp.tex with
% cdocsch1.tex, cdocsch2.tex, cdocsdrf.tex, cdocsfn1.tex, cdocsfn2.tex.
%
%<package>\ifdefined\childdocmain\endinput\fi
%<package>\ProvidesFile{childdoc.def}[2018/12/30 v2.0 child document driver]
%<samplemain>\ProvidesFile{cdocsamp.tex}[2018/12/30 v2.0 sample for childdoc]
%<*driver>
%\ProvidesFile{childdoc.drv}[2018/12/30 v2.0 childdoc reference manual file]
\PassOptionsToClass{10pt,a4paper}{article}
\documentclass{ltxdoc}

\usepackage[margin=35mm]{geometry}
\usepackage{hyperref}
\usepackage{hyperxmp}
\usepackage[usenames]{color}

\hypersetup{colorlinks=true}
\hypersetup{pdfstartview=FitH}
\hypersetup{pdfpagemode=UseNone}
\hypersetup{pdfsource={}}
\hypersetup{pdflang={en-UK}}
\hypersetup{pdfcopyright={Copyright 2017-2018 Niklas Beisert.
  This work may be distributed and/or modified under the
  conditions of the LaTeX Project Public License, either version 1.3
  of this license or (at your option) any later version.}}
\hypersetup{pdflicenseurl={http://www.latex-project.org/lppl.txt}}
\hypersetup{pdfcontactaddress={ETH Zurich, ITP, HIT K,
  Wolfgang-Pauli-Strasse 27}}
\hypersetup{pdfcontactpostcode={8093}}
\hypersetup{pdfcontactcity={Zurich}}
\hypersetup{pdfcontactcountry={Switzerland}}
\hypersetup{pdfcontactemail={nbeisert@itp.phys.ethz.ch}}
\hypersetup{pdfcontacturl={http://people.phys.ethz.ch/\xmptilde nbeisert/}}

\newcommand{\secref}[1]{\hyperref[#1]{section \ref*{#1}}}

\parskip1ex
\parindent0pt
\let\olditemize\itemize
\def\itemize{\olditemize\parskip0pt}

\begin{document}

\title{The \textsf{childdoc} Package}
\hypersetup{pdftitle={The childdoc Package}}
\author{Niklas Beisert\\[2ex]
  Institut f\"ur Theoretische Physik\\
  Eidgen\"ossische Technische Hochschule Z\"urich\\
  Wolfgang-Pauli-Strasse 27, 8093 Z\"urich, Switzerland\\[1ex]
  \href{mailto:nbeisert@itp.phys.ethz.ch}
  {\texttt{nbeisert@itp.phys.ethz.ch}}}
\hypersetup{pdfauthor={Niklas Beisert}}
\hypersetup{pdfsubject={Manual for the LaTeX2e Package childdoc}}
\date{30 December 2018, \textsf{v2.0}}
\maketitle

\begin{abstract}\noindent
\textsf{childdoc} is a \LaTeXe{} package
that enables the direct compilation
of document sections included by |\include|
to individual files.
\end{abstract}

\begingroup
\parskip0ex
\tableofcontents
\endgroup

%%%%%%%%%%%%%%%%%%%%%%%%%%%%%%%%%%%%%%%%%%%%%%%%%%%%%%%%%%%%%%%%%%%%%%%%%%%%%%%%
%%%%%%%%%%%%%%%%%%%%%%%%%%%%%%%%%%%%%%%%%%%%%%%%%%%%%%%%%%%%%%%%%%%%%%%%%%%%%%%%
\section{Introduction}

\LaTeX{} provides a mechanism to structure a large document (such as a book)
into a main file and several child files (containing the chapters)
using the |\include| command.
This mechanism is beneficial for documents
which span hundreds of pages in order to
make the source file(s) more manageable.
Moreover, compilation can be restricted to
selected child files by means of the |\includeonly| command.
The latter feature can be used to reduce the compilation time while editing
(this was significantly more useful in the earlier days of \LaTeX{})
or to generate a smaller document which is easier to navigate.
Another application of |\includeonly| is to generate
documents consisting of selected parts of the complete document.

However, there are a few drawbacks of the plain |\include| mechanism:
\begin{itemize}
\item
The child files cannot be compiled on their own,
they can only be compiled via the main file.
A naive editing environment
(such as a text editor with an option
to have the current file processed by \LaTeX)
may require one to switch to the main file before compiling;
attempting to compile the child file produces errors.
\item
The main file must be modified (each time)
to adjust the |\includeonly| command
to the present needs. This easily leaves the main file in a messy state.
\item
The generated document will always carry the filename
of the main document. This is inconvenient if
several child files are to be compiled and
to be kept for distribution.
\end{itemize}

The present package provides a simple interface
to make child files individually compilable by \LaTeX{}.
Compiling a child file then has the same effect as compiling
the main file with an |\includeonly| command
to select the appropriate child.
Moreover the generated document will carry the name of the child
rather than the main file.
This resolves all three above issues.

This feature is meant to make the editing of books,
thesis documents and lecture notes somewhat more convenient.
However, the package can also be used efficiently for
composing a series of documents (such as exercise sheets)
which are typically distributed individually.
It then assists the author in generating the individual documents
(potentially in different versions)
as well as a document containing the collected series.
Another application is in developing style files
or other kinds of included material
where compilation of the style file could redirect
to a sample or test file.

%%%%%%%%%%%%%%%%%%%%%%%%%%%%%%%%%%%%%%%%%%%%%%%%%%%%%%%%%%%%%%%%%%%%%%%%%%%%%%%%
%%%%%%%%%%%%%%%%%%%%%%%%%%%%%%%%%%%%%%%%%%%%%%%%%%%%%%%%%%%%%%%%%%%%%%%%%%%%%%%%
\section{Usage}

First of all, the package \textsf{childdoc} is \emph{not} a standard
\LaTeXe{} |.sty| style file! Therefore it needs to be invoked in
a non-standard way.

%%%%%%%%%%%%%%%%%%%%%%%%%%%%%%%%%%%%%%%%%%%%%%%%%%%%%%%%%%%%%%%%%%%%%%%%%%%%%%%%
\subsection{Included Files}
\label{sec:include}

%%%%%%%%%%%%%%%%%%%%%%%%%%%%%%%%%%%%%%%%
\DescribeMacro{\childdocmain}
To use the package, add the commands
\begin{center}
\begin{tabular}{l}
|\input{childdoc.def}|\\
|\childdocmain{}|\\
\end{tabular}
\end{center}
at the very top of the main \LaTeX{} file,
in particular \emph{before} the |\documentclass| statement!
The argument of |\childdocmain| should be left empty
(but it must be present).

%%%%%%%%%%%%%%%%%%%%%%%%%%%%%%%%%%%%%%%%
\DescribeMacro{\childdocof}
Furthermore, add the commands
\begin{center}
\begin{tabular}{l}
|\input{childdoc.def}|\\
|\childdocof{|\textit{main}|}|\\
\end{tabular}
\end{center}
at the top of every child file \textit{child}
which is included by |\include{|\textit{child}|}|
from within the main file
(or at least for those files to be compiled individually).
The argument \textit{main} must be the filename of the main file.

There are a couple of
considerations in setting up the main and child documents:

%%%%%%%%%%%%%%%%%%%%%%%%%%%%%%%%%%%%%%%%
\paragraph{Restrictions.}

Please note the following restrictions:
\begin{itemize}
\item
|\childdocmain| must be called with one argument \textit{main}
to ensure compatibility with earlier version of the package.
It must either be empty (|\childdocmain{}|)
or precisely match the filename of the main file in which it is specified.
See \secref{sec:detection} for further information.
\item
The filename \textit{main} must be specified without the |.tex| extension.
\item
The filename \textit{main} is case sensitive
(even in case-insensitive file systems)
due to internal string comparison.
\item
The argument \textit{main} should be fully expanded, it cannot be a macro.
\item
Subdirectories and special characters should be avoided in filenames.
\item
The command |\childdocmain{|\textit{main}|}| must be followed by a whitespace.
It should not be followed immediately by another command
or by a comment mark `|%|'.
This is because the \TeX{} parser reads the token immediately following
the argument of |\childdocmain| and puts it
at the beginning of every child section;
however, a white\-space is ignored.
\end{itemize}

%%%%%%%%%%%%%%%%%%%%%%%%%%%%%%%%%%%%%%%%
\paragraph{Content of Main File.}

It is advisable to place all content in the child files included by |\include|.
Any output contained in the main file will appear in all child documents
unless suppressed manually;
it cannot be suppressed automatically by the |\includeonly| directive
and thus should normally be avoided.
A method to include some content in the main file
by means of conditional processing is described in \secref{sec:conditional}.

%%%%%%%%%%%%%%%%%%%%%%%%%%%%%%%%%%%%%%%%
\paragraph{Page Numbering.}

When only a part of the document is compiled,
the appropriate numbering of pages
(as well as other status parameters)
is determined from the |.aux| files.
The latter contain information from previous passes.
However this information needs to propagate through
all intermediate child documents.
Therefore the page numbering in child documents may well
be inconsistent until the complete document is compiled at least once.

A useful (if unconventional) way to always ensure a consistent
page numbering is to restart the numbering in each child document
and denote the pages by `\textit{child}|.|\textit{page}'
where \textit{child} represents the chapter/section number of the child file.
This can be achieved by the command
|\numberwithin{page}{|\textit{child}|}|
of the \textsf{amsmath} package
where \textit{child} can be |chapter| or |section|
depending on the chosen structuring.
Alternatively, one can modify the macro |\thepage| appropriately
and reset the counter |page| at the start of each child file.

%%%%%%%%%%%%%%%%%%%%%%%%%%%%%%%%%%%%%%%%%%%%%%%%%%%%%%%%%%%%%%%%%%%%%%%%%%%%%%%%
\subsection{Conditional Processing}
\label{sec:conditional}

The package provides a mechanism to compile different versions
of a document. To customise the versions further some conditional processing
can come in handy to distinguish which version is being compiled.
The package provides two macros to describe the compilation context:

%%%%%%%%%%%%%%%%%%%%%%%%%%%%%%%%%%%%%%%%
\DescribeMacro{\ifchilddoc}
The conditional |\ifchilddoc| distinguishes between the compilation of
child documents and the main document:
%
\begin{center}
|\ifchilddoc |\textit{child-code}| |[|\||else |\textit{main-code}]| \||fi|
\end{center}

%%%%%%%%%%%%%%%%%%%%%%%%%%%%%%%%%%%%%%%%
\DescribeMacro{\childdocname}
\DescribeMacro{\childdocjob}
The macro |\childdocname| contains the filename (without extension)
of the main or child file being processed.
Note that |\childdocjob| will always contain the name of the main file.

%%%%%%%%%%%%%%%%%%%%%%%%%%%%%%%%%%%%%%%%
\paragraph{Title Page.}

Conditional processing can be used to include a title or banner page
in the main document when proper precautions are taken.
Importantly, the code in the main file should ensure that the page counter
(as well as other status parameters which are stored in the |.aux| files)
takes the same value after the conditional processing.
Otherwise the page numbers may take divergent values
depending on which part is compiled.

For example, a title page could be declared by:
%
\begin{center}
\begin{tabular}{l}
|\ifchilddoc\||else|\\
|\addtocounter{page}{-1}|\\
\textit{code for title page}\\
|\newpage|\\
|\||fi|
\end{tabular}
\end{center}
%
A banner page for the child documents can be generated by:
%
\begin{center}
\begin{tabular}{l}
|\ifchilddoc|\\
|\addtocounter{page}{-1}|\\
\textit{code for banner page}\\
|\newpage|\\
|\||fi|
\end{tabular}
\end{center}
%
Here one could write a message such as:
\begin{center}
|This is the part \childdocname{} of \childdocjob{}.|
\end{center}

%%%%%%%%%%%%%%%%%%%%%%%%%%%%%%%%%%%%%%%%%%%%%%%%%%%%%%%%%%%%%%%%%%%%%%%%%%%%%%%%
\subsection{Flags}
\label{sec:flags}

The package makes it easy to generate different versions
of the main or child documents.
To this end compilation flags can be defined
and assigned different default values.
They will be particularly useful in conjunction
with the forwarding mechanism described in \secref{sec:forward}.

For example, it may be useful to have a flag |\version|
which can be set to |draft| or |final|.
The document source will contain some conditional code
depending on the value of |\version|.
Suppose further, the flag should default to |final| for the main file
and to |draft| for child files
which is a natural assignment for editing the document.
This is achieved by placing the following code
in the preamble of the main document
(below the |\childdocmain| directive):
%
\begin{center}
\begin{tabular}{l}
|\ifchilddoc|\\
|\providecommand{\version}{draft}|\\
|\||else|\\
|\providecommand{\version}{final}|\\
|\||fi|
\end{tabular}
\end{center}
%
The definition by |\providecommand| makes sure
that previous definitions are not overwritten.
Further statements |\providecommand{\version}{...}|
can thus be added before the above code to override it.

For the main file, one might add a line
(between |\childdocmain| and the above block)
%
\begin{center}
|%\ifchilddoc\||else\providecommand{\version}{draft}\||fi|
\end{center}
%
which can be uncommented to produce a draft version.
Likewise one can add a line to the very top of a child file
(above the |\childdocof{|\textit{main}|}| directive)
%
\begin{center}
|%\providecommand{\version}{final}|
\end{center}
%
which can be uncommented to produce the final version of this child document.

%%%%%%%%%%%%%%%%%%%%%%%%%%%%%%%%%%%%%%%%%%%%%%%%%%%%%%%%%%%%%%%%%%%%%%%%%%%%%%%%
\subsection{Forwarding}
\label{sec:forward}

Different versions of the main or child documents
using compilation flags as described in \secref{sec:flags}
can be (permanently) stored in different files
for convenient compilation, viewing and distribution.
To this end, the package defines a command
to pass on compilation to a different file:

%%%%%%%%%%%%%%%%%%%%%%%%%%%%%%%%%%%%%%%%
\DescribeMacro{\childdocforward}
The command |\childdocforward| redirects processing to
another source file:
%
\begin{center}
\begin{tabular}{l}
|\input{childdoc.def}|\\
|\childdocforward[|\textit{main}|]{|\textit{dest}|}|\\
\end{tabular}
\end{center}
%
The argument \textit{dest} is the destination file
(without extension).
It should be the main file or one of the child files.
Note that further \textsf{childdoc} directives
such as |\childdocof| and |\childdocforward|
in the indicated file will be processed in this form.
The optional argument \textit{main}
passes on directly to the main file \textit{main}
while pretending to compile the child \textit{dest}.
This form behaves as if \textit{dest}
issues |\childdocof{|\textit{main}|}| right away,
and no further \textsf{childdoc} directives will be processed.

%%%%%%%%%%%%%%%%%%%%%%%%%%%%%%%%%%%%%%%%
\DescribeMacro{\...prefix}
In the alternative form |\childdocforwardprefix|,
%
\begin{center}
\begin{tabular}{l}
|\input{childdoc.def}|\\
|\childdocforwardprefix[|\textit{main}|]{|\textit{prefix}|}{|\textit{dest}|}|
\end{tabular}
\end{center}
%
the destination file is determined by a pattern
depending on the current file:
To make this work, the current file must be called
`{\textit{prefix}\hspace{0.2em}\textit{suffix}}'
with \textit{prefix} matching precisely the argument.
Processing is then passed on to the file
`{\textit{dest}\hspace{0.2em}\textit{suffix}}'.
Surely, the same effect is achieved by
directly specifying the
argument `{\textit{dest}\hspace{0.2em}\textit{suffix}}'
in the first form.
However, that requires to set up a different file
for each child. With the alternative form of the command
all these files can have exactly the same content
which simplifies setting them up and maintaining them.

For example, the following file |draft.tex|
with a compilation flag |\version| as described in \secref{sec:flags}
compiles the main document as a draft:
%
\begin{center}
\begin{tabular}{l}
|\def\version{draft}|\\
|\input{childdoc.def}|\\
|\childdocforward{|\textit{main}|}|
\end{tabular}
\end{center}
%
Likewise, the following files |final|\textit{nn}|.tex|
compile the final version of the child document
|child|\textit{nn}|.tex|:
%
\begin{center}
\begin{tabular}{l}
|\def\version{final}|\\
|\input{childdoc.def}|\\
|\childdocforwardprefix{final}{child}|
\end{tabular}
\end{center}
%

Note that when several versions of a main file and/or of each child file
are to be generated, it may be convenient to set up a |Makefile| or
shell script to automatise the process.

%%%%%%%%%%%%%%%%%%%%%%%%%%%%%%%%%%%%%%%%%%%%%%%%%%%%%%%%%%%%%%%%%%%%%%%%%%%%%%%%
\subsection{Command Line Processing}
\label{sec:commandline}

The effect of redirection files can also be achieved by invoking
the \LaTeX{} compiler with a more elaborate command line.
Most conveniently this should be done as part
of a shell script or a |Makefile|.

When using \textsf{childdoc} in the main file, the following
command lines effectively perform a redirection
(note that depending on the shell being used,
backslashes may have to be doubled: `|\|' $\to$ `|\\|'):
%
\begin{center}
|... -jobname "|\textit{target}|" |\\|"|[\textit{flags}]%
|\input{childdoc.def}\childdocforward[|\textit{main}|]{|\textit{dest}|}"|
\end{center}
%
Here \textit{target} is the name of the output file,
\textit{main} is the name of the main file
and \textit{dest} is the name of the main or child file to be processed
(all filenames without extensions).
The optional argument \textit{main} can be omitted
if \textit{main} matches \textit{dest}.
Optionally, compilation \textit{flags} can be defined via |\def| commands.
This command line makes the \TeX{} engine believe
it is compiling the file \textit{target}
whose content is specified as the latter parameter.
The provided code then forwards the processing to
\textit{main} or \textit{dest} as described in \secref{sec:forward}.

%%%%%%%%%%%%%%%%%%%%%%%%%%%%%%%%%%%%%%%%%%%%%%%%%%%%%%%%%%%%%%%%%%%%%%%%%%%%%%%%
\subsection{Include by Input}
\label{sec:input}

Including child documents by |\include| has some restrictions by design.
Most notably, the content of a child document always occupies
its own set of pages; pages cannot be shared between child documents.
Usually, this behaviour makes perfect sense
because each child document contain an essential part of the document.
However, in some situations it may be desirable to compose
a document from a collection of parts
without having mandatory page breaks between then.
For this case, the package
provides a mechanism to include parts
by |\input| which can also be processed individually.
However, by construction this mechanism
requires manual handling of the content to be output.

%%%%%%%%%%%%%%%%%%%%%%%%%%%%%%%%%%%%%%%%
\DescribeMacro{\ifchilddocmanual}
The main file should be prepared as usual, see \secref{sec:include}.
However, the document body must make a distinction
between processing of an individual part and of the main document, e.g.:
%
\begin{center}
\begin{tabular}{l}
|\ifchilddocmanual|\\
|\input{\childdocname}|\\
|\||else|\\
\textit{document body with }|\input{|\textit{part}|}|\\
|\||fi|
\end{tabular}
\end{center}
%
The conditional |\ifchilddocmanual| is true whenever
a part to be included by |\input| is being compiled,
and the name of the part is stored in |\childdocname|.

%%%%%%%%%%%%%%%%%%%%%%%%%%%%%%%%%%%%%%%%
\DescribeMacro{\childdocby}
Each part to be included by |\input| should start with:
%
\begin{center}
\begin{tabular}{l}
|\input{childdoc.def}|\\
|\childdocby{|\textit{main}|}|\\
\end{tabular}
\end{center}
%
The directive |\childdocby| is similar to |\childdocof|
described in \secref{sec:include},
but the subsequent selection of content must be done manually.
To that end, both |\ifchilddoc| and |\ifchilddocmanual|
will be true upon processing of a part,
and the name of the part is stored in |\childdocname|.
Note that |\jobname| will be set to the filename of the current part
so that each part receives an individual |.aux| file
that does not interfere with the |.aux| file(s) of the main document.
This behaviour can be altered by the alternative form
|\childdocby[*]{|\textit{main}|}| (with a non-empty optional argument)
which uses the |.aux| file of the main document
by setting |\jobname| to \textit{main}.

%%%%%%%%%%%%%%%%%%%%%%%%%%%%%%%%%%%%%%%%%%%%%%%%%%%%%%%%%%%%%%%%%%%%%%%%%%%%%%%%
\subsection{Driver Development}
\label{sec:driver}

The \textsf{childdoc} mechanism can also be use for the development
of definition files such as \LaTeX{} styles or classes.
This case differs from the above setup with multiple parts
included by |\include| in that no |\includeonly| should be invoked.
This can be achieved by starting the include file
(before |\ProvidesPackage|) with:
%
\begin{center}
\begin{tabular}{l}
|\input{childdoc.def}|\\
|\childdocforward{|\textit{main}|}|\\
\end{tabular}
\end{center}
%
or alternatively with:
%
\begin{center}
\begin{tabular}{l}
|\input{childdoc.def}|\\
|\childdocby{|\textit{main}|}|\\
\end{tabular}
\end{center}
%
Both forms have slightly different effects as described above.
The main file is prepared as usual, see \secref{sec:include}.

%%%%%%%%%%%%%%%%%%%%%%%%%%%%%%%%%%%%%%%%%%%%%%%%%%%%%%%%%%%%%%%%%%%%%%%%%%%%%%%%
\subsection{Legacy Detection}
\label{sec:detection}

The directive |\childdocmain| in the main file can detect
whether the complete document or merely a child is to be compiled
even without using the directive |\childdocof|.
This method is deprecated because it is less robust
and there is no compelling reason to use it;
it is merely provided for backward compatibility
and it may be removed in future versions.

If the detection mechanism is to be used,
it is mandatory to correctly specify
the filename of the main file as the argument of |\childdocmain|:
%
\begin{center}
\begin{tabular}{l}
|\input{childdoc.def}|\\
|\childdocmain{|\textit{main}|}|\\
\end{tabular}
\end{center}
%
If |\jobname| does not match the argument \textit{main} of |\childdocmain|,
it is assumed that |\jobname| points to the child file to be compiled.
When using |\childdocmain| with the main file specified as argument,
it suffices to start a child file
with just |\input{|\textit{main}|}|
without loading of the package and using |\childdocof|.
If instead all processing is done
with the appropriate \textsf{childdoc} directives,
the argument of \textit{main} of |\childdocmain| can be empty.

An alternative version of the command line processing described
in \secref{sec:commandline} using the detection mechanism reads:
%
\begin{center}
|... -jobname "|\textit{target}|" "|[\textit{flags}]%
[|\def\jobname{|\textit{dest}|}|]|\input{|\textit{main}|}"|
\end{center}

%%%%%%%%%%%%%%%%%%%%%%%%%%%%%%%%%%%%%%%%%%%%%%%%%%%%%%%%%%%%%%%%%%%%%%%%%%%%%%%%
\subsection{Manual Code}
\label{sec:manual}

In case one cannot be certain whether the definitions file |childdoc.def|
is installed on the target \TeX{} distribution
and one prefers not to ship it,
it is conceivable to paste a few relevant commands into the sources.

To that end, drop all statements |\input{childdoc.def}|
and perform the replacements as outlined below.
Instead of |\childdocmain{|\textit{main}|}| add the following code
to the top of the main file:
%
\begin{center}
\begin{tabular}{l}
|\||ifdefined\childdocname\endinput\||fi\newif\ifchilddoc|\\
|\edef\childdocname{\scantokens\expandafter{\jobname\noexpand}}|\\
|\def\childdocmain{|\textit{main}|}\||ifx\childdocmain\childdocname\||else|\\
|\childdoctrue\includeonly{\childdocname}\let\jobname\childdocmain\||fi|\\
\end{tabular}
\end{center}
%
Instead of |\childdocof{|\textit{main}|}| just include the main file
at the top of each child file:
%
\begin{center}
|\input{|\textit{main}|}|
\end{center}
%
A simple redirection |\childdocforward{|\textit{dest}|}| is achieved by:
%
\begin{center}
|\def\jobname{|\textit{dest}|}\input{\jobname}|
\end{center}
%
The redirection with prefix
|\childdocforwardprefix[|\textit{prefix}|]{|\textit{dest}|}|
is accomplished by:
%
\begin{center}
\begin{tabular}{l}
|{\edef\jobname{\scantokens\expandafter{\jobname\noexpand}}|\\
|\def\redirectjob |\textit{prefix}|#1~~~{\gdef\jobname{|\textit{dest}|#1}}|\\
|\expandafter\redirectjob\jobname~~~}\input{\jobname}|
\end{tabular}
\end{center}

In an alternative approach,
child documents can be compiled by a specific command line
without additional code or specific definitions:
%
\begin{center}
|... -jobname "|\textit{target}|" "|[\textit{flags}]%
|\includeonly{|\textit{dest}|}\input{|\textit{main}|}"|
\end{center}
%

%%%%%%%%%%%%%%%%%%%%%%%%%%%%%%%%%%%%%%%%%%%%%%%%%%%%%%%%%%%%%%%%%%%%%%%%%%%%%%%%
%%%%%%%%%%%%%%%%%%%%%%%%%%%%%%%%%%%%%%%%%%%%%%%%%%%%%%%%%%%%%%%%%%%%%%%%%%%%%%%%
\section{Information}

%%%%%%%%%%%%%%%%%%%%%%%%%%%%%%%%%%%%%%%%%%%%%%%%%%%%%%%%%%%%%%%%%%%%%%%%%%%%%%%%
\subsection{Copyright}

Copyright \copyright{} 2017--2018 Niklas Beisert

This work may be distributed and/or modified under the
conditions of the \LaTeX{} Project Public License, either version 1.3
of this license or (at your option) any later version.
The latest version of this license is in
  \url{http://www.latex-project.org/lppl.txt}
and version 1.3 or later is part of all distributions of \LaTeX{}
version 2005/12/01 or later.

This work has the LPPL maintenance status `maintained'.

The Current Maintainer of this work is Niklas Beisert.

This work consists of the files |README.txt|, |childdoc.ins| and |childdoc.dtx|
as well as the derived files |childdoc.def|, |cdocsamp.tex|
with |cdocsch1.tex|, |cdocsch2.tex|, |cdocspt3.tex|, |cdocspt4.tex|,
|cdocsdrf.tex|, |cdocsfn1.tex|, |cdocsfn2.tex|
as well as |childdoc.pdf|.

%%%%%%%%%%%%%%%%%%%%%%%%%%%%%%%%%%%%%%%%%%%%%%%%%%%%%%%%%%%%%%%%%%%%%%%%%%%%%%%%
\subsection{Files and Installation}

The package consists of the files:
%
\begin{center}
\begin{tabular}{ll}
    |README.txt|   & readme file \\
    |childdoc.ins| & installation file \\
    |childdoc.dtx| & source file \\
    |childdoc.def| & definition file \\
    |cdocsamp.tex| & sample main file \\
    |cdocsch1.tex| & sample include file \\
    |cdocsch2.tex| & sample include file \\
    |cdocspt3.tex| & sample part file \\
    |cdocspt4.tex| & sample part file \\
    |cdocsdrf.tex| & sample redirection file \\
    |cdocsfn1.tex| & sample redirection file \\
    |cdocsfn2.tex| & sample redirection file \\
    |childdoc.pdf| & manual
\end{tabular}
\end{center}
%
The distribution consists of the files
|README.txt|, |childdoc.ins| and |childdoc.dtx|.
%
\begin{itemize}
\item
Run (pdf)\LaTeX{} on |childdoc.dtx|
to compile the manual |childdoc.pdf| (this file).
\item
Run \LaTeX{} on |childdoc.ins| to create the definitions file |childdoc.def|
and the sample |cdocsamp.tex| with include files
|cdocsch1.tex|, |cdocsch2.tex|, |cdocspt3.tex|, |cdocspt4.tex|,
|cdocsdrf.tex|, |cdocsfn1.tex|, |cdocsfn2.tex|.
Then copy the file |childdoc.def| to an appropriate directory of your \LaTeX{}
distribution, e.g.\ \textit{texmf-root}|/tex/latex/childdoc|.
\end{itemize}

%%%%%%%%%%%%%%%%%%%%%%%%%%%%%%%%%%%%%%%%%%%%%%%%%%%%%%%%%%%%%%%%%%%%%%%%%%%%%%%%
\subsection{Related CTAN Packages}

There are several other packages which offer a similar functionality:
%
\begin{itemize}
\item
The packages
\href{http://ctan.org/pkg/docmute}{\textsf{docmute}},
\href{http://ctan.org/pkg/includex}{\textsf{includex}} and
\href{http://ctan.org/pkg/standalone}{\textsf{standalone}}
provide commands to include only the document body of
a child file thus allowing both files to be compiled individually.
\item
The packages \href{http://ctan.org/pkg/subdocs}{\textsf{subdocs}}
and \href{http://ctan.org/pkg/subfiles}{\textsf{subfiles}}
provide structures in which the main and child documents can be
encapsulated and allowing them to be compiled individually.
The inclusion mechanism is different from the conventional |\include|.
\item
The package \href{http://ctan.org/pkg/combine}{\textsf{combine}}
is an elaborate solution to combine several documents into one.
\end{itemize}
%
See also the CTAN topic \href{http://ctan.org/topic/subdocs}{\textsf{subdocs}}
for further related packages.
The present package differs from the above solutions in that
a document structure constructed with the conventional |\include| mechanism
just needs two extra commands at the top of every file
such that all constituent files can be compiled individually.

%%%%%%%%%%%%%%%%%%%%%%%%%%%%%%%%%%%%%%%%%%%%%%%%%%%%%%%%%%%%%%%%%%%%%%%%%%%%%%%%
%\subsection{Feature Suggestions}
%
%The following is a list of features which may be useful for future
%versions of this package:
%%
%\begin{itemize}
%\item
%\ldots
%\end{itemize}

%%%%%%%%%%%%%%%%%%%%%%%%%%%%%%%%%%%%%%%%%%%%%%%%%%%%%%%%%%%%%%%%%%%%%%%%%%%%%%%%
\subsection{Revision History}

%%%%%%%%%%%%%%%%%%%%%%%%%%%%%%%%%%%%%%%%
\paragraph{v2.0:} 2018/12/30

\begin{itemize}
\item
immediate forward processing
\item
added |\childdocby| mechanism
\item
manual restructured
\end{itemize}

%%%%%%%%%%%%%%%%%%%%%%%%%%%%%%%%%%%%%%%%
\paragraph{v1.6:} 2018/01/17

\begin{itemize}
\item
application for development of include files
\item
corrections to manual
\end{itemize}

%%%%%%%%%%%%%%%%%%%%%%%%%%%%%%%%%%%%%%%%
\paragraph{v1.5:} 2017/05/21

\begin{itemize}
\item
more complete structuring introduced
\item
|\childdocof| introduced
\item
|\childdoc| renamed to |\childdocmain|
\item
|\childredirect| renamed to |\childdocforward| and |\childdocforwardprefix|
and functionality expanded
\end{itemize}

%%%%%%%%%%%%%%%%%%%%%%%%%%%%%%%%%%%%%%%%
\paragraph{v1.0:} 2017/04/27

\begin{itemize}
\item
manual and install package
\item
first version published on CTAN
\end{itemize}

%%%%%%%%%%%%%%%%%%%%%%%%%%%%%%%%%%%%%%%%
\paragraph{v0.6:} 2017/04/26

\begin{itemize}
\item
redirection mechanism added
\end{itemize}

%%%%%%%%%%%%%%%%%%%%%%%%%%%%%%%%%%%%%%%%
\paragraph{v0.5:} 2017/04/26

\begin{itemize}
\item
functionality in definition file
\end{itemize}


%%%%%%%%%%%%%%%%%%%%%%%%%%%%%%%%%%%%%%%%%%%%%%%%%%%%%%%%%%%%%%%%%%%%%%%%%%%%%%%%
%%%%%%%%%%%%%%%%%%%%%%%%%%%%%%%%%%%%%%%%%%%%%%%%%%%%%%%%%%%%%%%%%%%%%%%%%%%%%%%%
%%%%%%%%%%%%%%%%%%%%%%%%%%%%%%%%%%%%%%%%%%%%%%%%%%%%%%%%%%%%%%%%%%%%%%%%%%%%%%%%
\appendix

\settowidth\MacroIndent{\rmfamily\scriptsize 000\ }

 \DocInput{childdoc.dtx}

\end{document}
%</driver>
% \fi
%
% %%%%%%%%%%%%%%%%%%%%%%%%%%%%%%%%%%%%%%%%%%%%%%%%%%%%%%%%%%%%%%%%%%%%%%%%%%%%%%
% %%%%%%%%%%%%%%%%%%%%%%%%%%%%%%%%%%%%%%%%%%%%%%%%%%%%%%%%%%%%%%%%%%%%%%%%%%%%%%
% \section{Sample}
%\iffalse
%<*samplemain>
%\fi
%
% The following presents a sample document
% with two chapters, two parts, a title page,
% a compile flag as well as three forwarding files to set the flag.
% It consists of eight |.tex| files:
% \begin{center}
% \begin{tabular}{ll}
% |cdocsamp.tex|&main file\\
% |cdocsch1.tex|&include file for chapter 1\\
% |cdocsch2.tex|&include file for chapter 2\\
% |cdocspt3.tex|&include file for part 3\\
% |cdocspt4.tex|&include file for part 4\\
% |cdocsdrf.tex|&forwarding file for main file in draft mode\\
% |cdocsfi1.tex|&forwarding file for final version of chapter 1\\
% |cdocsfi2.tex|&forwarding file for final version of chapter 2\\
% \end{tabular}
% \end{center}
% Each of the eight files can be compiled directly by the \LaTeX{} compiler.
%
% %%%%%%%%%%%%%%%%%%%%%%%%%%%%%%%%%%%%%%
% \paragraph{Main File.}
%
% The main file is called |cdocsamp.tex|.
%
% Load the \textsf{childdoc} definitions and
% declare the filename for the main document:
%    \begin{macrocode}
\input{childdoc.def}
\childdocmain{}
%    \end{macrocode}

% Optional override for |\version| flag:
%    \begin{macrocode}
%%\ifchilddoc\else\providecommand{\version}{draft}\fi
%    \end{macrocode}

% Define the default values for the |\version| flag
% (|final| for the main file and |draft| for childs):
%    \begin{macrocode}
\ifchilddoc
\providecommand{\version}{draft}
\else
\providecommand{\version}{final}
\fi
%    \end{macrocode}

% Load the standard document class:
%    \begin{macrocode}
\documentclass[12pt]{article}
%    \end{macrocode}

% Start the document body:
%    \begin{macrocode}
\begin{document}
%    \end{macrocode}

% Declare a title page.
% Print title, part of document being processed and version flag:
%    \begin{macrocode}
\addtocounter{page}{-1}
\begin{center}
{\LARGE\bfseries{}childdoc example\par}
\vspace{1cm}
\ifchilddoc
\ifchilddocmanual part\else chapter\fi:
`\childdocname' of `\childdocjob'\par
\else
main document: `\childdocjob'\par
\fi
version: \version\par
\end{center}
\newpage
%    \end{macrocode}

% Manually include selected file,
% otherwise process as usual:
%    \begin{macrocode}
\ifchilddocmanual
\section*{part `\childdocname'}
\input{\childdocname}
\else
%    \end{macrocode}

% Include the two chapters:
%    \begin{macrocode}
\include{cdocsch1}
\include{cdocsch2}
%    \end{macrocode}

% Include the two parts unless only chapters should be displayed:
%    \begin{macrocode}
\ifchilddoc\else
\section{part three}
\input{cdocspt3}
\section{part four}
\input{cdocspt4}
\fi
%    \end{macrocode}

% Process as usual until here:
%    \begin{macrocode}
\fi
%    \end{macrocode}

% End of document body:
%    \begin{macrocode}
\end{document}
%    \end{macrocode}
%\iffalse
%</samplemain>
%\fi
%
% %%%%%%%%%%%%%%%%%%%%%%%%%%%%%%%%%%%%%%
% \paragraph{Chapter Include Files.}
%
% The include files are called |cdocsch1.tex| and |cdocsch2.tex|.
%
%\iffalse
%<*samplechap1|samplechap2>
%\fi

% Optional override for |\version| flag:
%    \begin{macrocode}
%%\providecommand{\version}{final}
%    \end{macrocode}

% Include the main document:
%    \begin{macrocode}
\input{childdoc.def}
\childdocof{cdocsamp}
%    \end{macrocode}

%\iffalse
%</samplechap1|samplechap2>
%\fi
%
%\iffalse
%<*samplechap1>
%\fi
% Some text for chapter 1:
%    \begin{macrocode}
\section{one}
some text in chapter one
%    \end{macrocode}

%\iffalse
%</samplechap1>
%\fi
% Some text for chapter 2:
%\iffalse
%<*samplechap2>
%\fi
%    \begin{macrocode}
\section{two}
more text in chapter two
%    \end{macrocode}

%\iffalse
%</samplechap2>
%\fi
%
% %%%%%%%%%%%%%%%%%%%%%%%%%%%%%%%%%%%%%%
% \paragraph{Part Include Files.}
%
% The include files are called |cdocspt3.tex| and |cdocspt4.tex|.
%
%\iffalse
%<*samplepart3|samplepart4>
%\fi

% Optional override for |\version| flag:
%    \begin{macrocode}
%%\providecommand{\version}{final}
%    \end{macrocode}

% Include the main document:
%    \begin{macrocode}
\input{childdoc.def}
\childdocby{cdocsamp}
%    \end{macrocode}

%\iffalse
%</samplepart3|samplepart4>
%\fi
%
%\iffalse
%<*samplepart3>
%\fi
% Some text for part 3:
%    \begin{macrocode}
some text in part three
%    \end{macrocode}

%\iffalse
%</samplepart3>
%\fi
% Some text for part 4:
%\iffalse
%<*samplepart4>
%\fi
%    \begin{macrocode}
more text in part four
%    \end{macrocode}

%\iffalse
%</samplepart4>
%\fi
%
% %%%%%%%%%%%%%%%%%%%%%%%%%%%%%%%%%%%%%%
% \paragraph{Forwarding for a Complete Draft.}
%
% The following forwarding file |cdocsdrf.tex|
% compiles the main document in draft mode:
%\iffalse
%<*sampledraft>
%\fi
%    \begin{macrocode}
\def\version{draft}
\input{childdoc.def}
\childdocforward{cdocsamp}
%    \end{macrocode}

%\iffalse
%</sampledraft>
%\fi
%
% %%%%%%%%%%%%%%%%%%%%%%%%%%%%%%%%%%%%%%
% \paragraph{Forwarding for Final Version of the Chapters.}
%
% The following forwarding files |cdocsfn1.tex| and |cdocsfn2.tex|
% (with identical content)
% compile the final versions of the child documents
% |cdocsch1.tex| and |cdocsch2.tex|, respectively:
%\iffalse
%<*samplefinal>
%\fi
%    \begin{macrocode}
\def\version{final}
\input{childdoc.def}
\childdocforwardprefix[cdocsamp]{cdocsfn}{cdocsch}
%    \end{macrocode}

%\iffalse
%</samplefinal>
%\fi
%
% %%%%%%%%%%%%%%%%%%%%%%%%%%%%%%%%%%%%%%
% \paragraph{Command Line Processing.}
%
% The following three command lines generate the output files
% |cdocscld|, |cdocscl1| and |cdocscl2|
% which should be identical to
% |cdocsdrf|, |cdocsch1| and |cdocsfn2|, respectively:
% \begin{center}
% \begin{tabular}{l}
% |latex -jobname cdocscld \|\\
% |  "\def\version{draft}\input{childdoc.def}\childdocforward{cdocsamp}"|\\
% |latex -jobname cdocscl1 \|\\
% |  "\input{childdoc.def}\childdocforward[cdocsamp]{cdocsch1}"|\\
% |latex -jobname cdocscl2 \|\\
% |  "\def\version{final}\input{childdoc.def}\childdocforward{cdocsch2}"|
% \end{tabular}
% \end{center}
% Note that the trailing backslash on each first line
% merely continues the input to the second line
% (for convenient cut ant paste).
% Furthermore, the command |latex| can be replaced by any
% of its alternative versions such as |pdflatex|.
%
% %%%%%%%%%%%%%%%%%%%%%%%%%%%%%%%%%%%%%%%%%%%%%%%%%%%%%%%%%%%%%%%%%%%%%%%%%%%%%%
% %%%%%%%%%%%%%%%%%%%%%%%%%%%%%%%%%%%%%%%%%%%%%%%%%%%%%%%%%%%%%%%%%%%%%%%%%%%%%%
% \section{Implementation}
%\iffalse
%<*package>
%\fi
%
% This section describes the definitions file |childdoc.def|.

% The definitions cannot be loaded using |\usepackage| or |\RequirePackage|
% which has a mechanism to prevent loading a style file more than once.
% When loading the definitions by means of |\input|
% multiple instances have to be prevented manually:
%\iffalse
%This code needs to be before the `\ProvidesFile' directive
%which is defined at the beginning of this file.
%Therefore it is also placed there and commented out here.
%</package>
%<*discard>
%\fi
%    \begin{macrocode}
\ifdefined\childdocmain\endinput\fi
%    \end{macrocode}
%\iffalse
%</discard>
%<*package>
%\fi
%
% \macro{\ifchilddoc}
% \macro{\ifchilddocmanual}
% The conditional |\ifchilddoc| tells whether a
% child (true) or main (false) document is being compiled.
% The conditional |\ifchilddocmanual| tells whether
% the |\includeonly| mechanism is used (false) or
% the selection of child files must be performed manually (true).
% The definitions initialise to false:
%    \begin{macrocode}
\newif\ifchilddoc
\newif\ifchilddocmanual
%    \end{macrocode}

% \macro{\childdocname}
% \macro{\childdocjob}
% The macro |\childdocname| stores the name of the main document
% to be compiled. The macro |\childdocjob| stores the name of
% the document on which the \LaTeX{} compiler was originally invoked.
% The content of |\jobname| cannot be compared
% to filenames specified in the source due to different catcodes.
% The following code rescans |\jobname|, stores the result
% in |\childdocname| and saves a copy in |\childdocjob|:
%    \begin{macrocode}
\edef\childdocname{\scantokens\expandafter{\jobname\noexpand}}
\let\childdocjob\childdocname
%    \end{macrocode}

% \macro{\childdocdisable}
% The macro |\childdocdisable| prevents the main file
% from being processed more than once.
% At this stage, the main document command |\childdocmain|
% is assumed to be called once again where it should do nothing.
% Any subsequent call to it should prevent
% a secondary processing of the main document
% It overwrites the forwarding commands
% |\childdocof| and |\childdocforward|
% with empty macros to prevent further inclusions of the main document:
%    \begin{macrocode}
\newcommand{\childdocdisable}
{
  \renewcommand{\childdocmain}[1]{\renewcommand{\childdocmain}[1]{\endinput}}
  \renewcommand{\childdocof}[1]{}
  \renewcommand{\childdocby}[2][]{}
  \renewcommand{\childdocforward}[2][]{}
  \renewcommand{\childdocdisable}{}
}
%    \end{macrocode}

% \macro{\childdocmain}
% The macro |\childdocmain| is to be called at the top of the main file
% with nothing or the main filename (without extension) as argument.
% First, it breaks loops.
% If the argument is not empty and does not match |\childdocname|
% (which is set by the first inclusion of |childdoc.def|),
% |\ifchilddoc| is set to true, |\includeonly| is applied to the child file
% and |\jobname| is set to the main file
% (for proper handling of |.aux| files):
%    \begin{macrocode}
\newcommand{\childdocmain}[1]
{
  \childdocdisable\childdocmain{}
  \if?#1?\else
    \begingroup
      \def\childdoctmp{#1}
      \ifx\childdoctmp\childdocname
        \def\childdoctmp{}
      \else
        \def\childdoctmp
        {
          \childdoctrue
          \includeonly{\childdocname}
          \def\childdocjob{#1}
          \def\jobname{#1}
        }
      \fi
      \expandafter
    \endgroup
    \childdoctmp
  \fi
}
%    \end{macrocode}

% \macro{\childdocof}
% The command |\childdocof| redirects
% compilation to the main file |#1|.
%    \begin{macrocode}
\newcommand{\childdocof}[1]
{
  \childdocdisable
  \childdoctrue
  \includeonly{\childdocname}
  \def\jobname{#1}
  \def\childdocjob{#1}
  \input{#1}
}
%    \end{macrocode}

% \macro{\childdocby}
% The command |\childdocby| ....
%    \begin{macrocode}
\newcommand{\childdocby}[2][]
{
  \childdocdisable
  \childdoctrue
  \childdocmanualtrue
  \if?#1?\else
    \def\jobname{#2}
  \fi
  \def\childdocjob{#2}
  \input{#2}
  \endinput
}
%    \end{macrocode}

% \macro{\childdocforward}
% The command |\childdocforward| redirects
% compilation to the main file or
% (if the optional argument is given) a child file.
% Parameters are set as if the main file
% or a child file starting with |\childdocof| was compiled.
% Then compilation is handed over to the main file:
%    \begin{macrocode}
\newcommand{\childdocforward}[2][]
{
  \begingroup
    \if?#1?
      \def\childdoctmp
      {
        \def\childdocname{#2}
        \def\childdocjob{#2}
        \def\jobname{#2}
        \input{#2}
        \endinput
      }
    \else
      \def\childdoctmp
      {
        \childdocdisable
        \def\childdocname{#2}
        \childdoctrue
        \includeonly{#2}
        \def\childdocjob{#1}
        \def\jobname{#1}
        \input{#1}
        \endinput
      }
    \fi
    \expandafter
  \endgroup
  \childdoctmp
}
%    \end{macrocode}

% \macro{\childdocforwardprefix}
% The command |\childdocforwardprefix| redirects
% compilation to the main or a child file by means of a pattern.
% The prefix |#1| in the current filename is replaced by |#2|
% and the suffix of the current filename is kept
% (it is assumed that the filename does not contain the substring `|~~~|'
% which is used as a delimiter).
% Compilation is handed over to the new file by |\childdocforward|:
%    \begin{macrocode}
\newcommand{\childdocforwardprefix}[3][]
{
  \begingroup
    \def\childdocextract #2##1~~~{\def\childdoctmp{\childdocforward[#1]{#3##1}}}
    \expandafter\childdocextract\childdocname~~~
    \expandafter
  \endgroup
  \childdoctmp
}
%    \end{macrocode}

% \macro{\childdoc}
% The deprecated macro |\childdoc| is a legacy version of |\childdocmain|:
%    \begin{macrocode}
\newcommand{\childdoc}{\childdocmain}
%    \end{macrocode}

% \macro{\childdocredirect}
% The deprecated macro |\childdocredirect| is a legacy version
% of |\childdocforward| and |\childdocforwardprefix|:
%    \begin{macrocode}
\newcommand{\childdocredirect}[2][]
{
  \begingroup
    \if?#1?
      \def\childdoctmp{\childdocforward{#2}}
    \else
      \def\childdoctmp{\childdocforwardprefix{#1}{#2}}
    \fi
    \expandafter
  \endgroup
  \childdoctmp
}
%    \end{macrocode}

%\iffalse
%</package>
%\fi
%
\endinput

\childdocforward{cdocsamp}
%    \end{macrocode}

%\iffalse
%</sampledraft>
%\fi
%
% %%%%%%%%%%%%%%%%%%%%%%%%%%%%%%%%%%%%%%
% \paragraph{Forwarding for Final Version of the Chapters.}
%
% The following forwarding files |cdocsfn1.tex| and |cdocsfn2.tex|
% (with identical content)
% compile the final versions of the child documents
% |cdocsch1.tex| and |cdocsch2.tex|, respectively:
%\iffalse
%<*samplefinal>
%\fi
%    \begin{macrocode}
\def\version{final}
% \iffalse
%
% childdoc.dtx Copyright (C) 2017-2018 Niklas Beisert
%
% This work may be distributed and/or modified under the
% conditions of the LaTeX Project Public License, either version 1.3
% of this license or (at your option) any later version.
% The latest version of this license is in
%   http://www.latex-project.org/lppl.txt
% and version 1.3 or later is part of all distributions of LaTeX
% version 2005/12/01 or later.
%
% This work has the LPPL maintenance status `maintained'.
%
% The Current Maintainer of this work is Niklas Beisert.
%
% This work consists of the files childdoc.dtx and childdoc.ins
% and the derived files childdoc.def and cdocsamp.tex with
% cdocsch1.tex, cdocsch2.tex, cdocsdrf.tex, cdocsfn1.tex, cdocsfn2.tex.
%
%<package>\ifdefined\childdocmain\endinput\fi
%<package>\ProvidesFile{childdoc.def}[2018/12/30 v2.0 child document driver]
%<samplemain>\ProvidesFile{cdocsamp.tex}[2018/12/30 v2.0 sample for childdoc]
%<*driver>
%\ProvidesFile{childdoc.drv}[2018/12/30 v2.0 childdoc reference manual file]
\PassOptionsToClass{10pt,a4paper}{article}
\documentclass{ltxdoc}

\usepackage[margin=35mm]{geometry}
\usepackage{hyperref}
\usepackage{hyperxmp}
\usepackage[usenames]{color}

\hypersetup{colorlinks=true}
\hypersetup{pdfstartview=FitH}
\hypersetup{pdfpagemode=UseNone}
\hypersetup{pdfsource={}}
\hypersetup{pdflang={en-UK}}
\hypersetup{pdfcopyright={Copyright 2017-2018 Niklas Beisert.
  This work may be distributed and/or modified under the
  conditions of the LaTeX Project Public License, either version 1.3
  of this license or (at your option) any later version.}}
\hypersetup{pdflicenseurl={http://www.latex-project.org/lppl.txt}}
\hypersetup{pdfcontactaddress={ETH Zurich, ITP, HIT K,
  Wolfgang-Pauli-Strasse 27}}
\hypersetup{pdfcontactpostcode={8093}}
\hypersetup{pdfcontactcity={Zurich}}
\hypersetup{pdfcontactcountry={Switzerland}}
\hypersetup{pdfcontactemail={nbeisert@itp.phys.ethz.ch}}
\hypersetup{pdfcontacturl={http://people.phys.ethz.ch/\xmptilde nbeisert/}}

\newcommand{\secref}[1]{\hyperref[#1]{section \ref*{#1}}}

\parskip1ex
\parindent0pt
\let\olditemize\itemize
\def\itemize{\olditemize\parskip0pt}

\begin{document}

\title{The \textsf{childdoc} Package}
\hypersetup{pdftitle={The childdoc Package}}
\author{Niklas Beisert\\[2ex]
  Institut f\"ur Theoretische Physik\\
  Eidgen\"ossische Technische Hochschule Z\"urich\\
  Wolfgang-Pauli-Strasse 27, 8093 Z\"urich, Switzerland\\[1ex]
  \href{mailto:nbeisert@itp.phys.ethz.ch}
  {\texttt{nbeisert@itp.phys.ethz.ch}}}
\hypersetup{pdfauthor={Niklas Beisert}}
\hypersetup{pdfsubject={Manual for the LaTeX2e Package childdoc}}
\date{30 December 2018, \textsf{v2.0}}
\maketitle

\begin{abstract}\noindent
\textsf{childdoc} is a \LaTeXe{} package
that enables the direct compilation
of document sections included by |\include|
to individual files.
\end{abstract}

\begingroup
\parskip0ex
\tableofcontents
\endgroup

%%%%%%%%%%%%%%%%%%%%%%%%%%%%%%%%%%%%%%%%%%%%%%%%%%%%%%%%%%%%%%%%%%%%%%%%%%%%%%%%
%%%%%%%%%%%%%%%%%%%%%%%%%%%%%%%%%%%%%%%%%%%%%%%%%%%%%%%%%%%%%%%%%%%%%%%%%%%%%%%%
\section{Introduction}

\LaTeX{} provides a mechanism to structure a large document (such as a book)
into a main file and several child files (containing the chapters)
using the |\include| command.
This mechanism is beneficial for documents
which span hundreds of pages in order to
make the source file(s) more manageable.
Moreover, compilation can be restricted to
selected child files by means of the |\includeonly| command.
The latter feature can be used to reduce the compilation time while editing
(this was significantly more useful in the earlier days of \LaTeX{})
or to generate a smaller document which is easier to navigate.
Another application of |\includeonly| is to generate
documents consisting of selected parts of the complete document.

However, there are a few drawbacks of the plain |\include| mechanism:
\begin{itemize}
\item
The child files cannot be compiled on their own,
they can only be compiled via the main file.
A naive editing environment
(such as a text editor with an option
to have the current file processed by \LaTeX)
may require one to switch to the main file before compiling;
attempting to compile the child file produces errors.
\item
The main file must be modified (each time)
to adjust the |\includeonly| command
to the present needs. This easily leaves the main file in a messy state.
\item
The generated document will always carry the filename
of the main document. This is inconvenient if
several child files are to be compiled and
to be kept for distribution.
\end{itemize}

The present package provides a simple interface
to make child files individually compilable by \LaTeX{}.
Compiling a child file then has the same effect as compiling
the main file with an |\includeonly| command
to select the appropriate child.
Moreover the generated document will carry the name of the child
rather than the main file.
This resolves all three above issues.

This feature is meant to make the editing of books,
thesis documents and lecture notes somewhat more convenient.
However, the package can also be used efficiently for
composing a series of documents (such as exercise sheets)
which are typically distributed individually.
It then assists the author in generating the individual documents
(potentially in different versions)
as well as a document containing the collected series.
Another application is in developing style files
or other kinds of included material
where compilation of the style file could redirect
to a sample or test file.

%%%%%%%%%%%%%%%%%%%%%%%%%%%%%%%%%%%%%%%%%%%%%%%%%%%%%%%%%%%%%%%%%%%%%%%%%%%%%%%%
%%%%%%%%%%%%%%%%%%%%%%%%%%%%%%%%%%%%%%%%%%%%%%%%%%%%%%%%%%%%%%%%%%%%%%%%%%%%%%%%
\section{Usage}

First of all, the package \textsf{childdoc} is \emph{not} a standard
\LaTeXe{} |.sty| style file! Therefore it needs to be invoked in
a non-standard way.

%%%%%%%%%%%%%%%%%%%%%%%%%%%%%%%%%%%%%%%%%%%%%%%%%%%%%%%%%%%%%%%%%%%%%%%%%%%%%%%%
\subsection{Included Files}
\label{sec:include}

%%%%%%%%%%%%%%%%%%%%%%%%%%%%%%%%%%%%%%%%
\DescribeMacro{\childdocmain}
To use the package, add the commands
\begin{center}
\begin{tabular}{l}
|\input{childdoc.def}|\\
|\childdocmain{}|\\
\end{tabular}
\end{center}
at the very top of the main \LaTeX{} file,
in particular \emph{before} the |\documentclass| statement!
The argument of |\childdocmain| should be left empty
(but it must be present).

%%%%%%%%%%%%%%%%%%%%%%%%%%%%%%%%%%%%%%%%
\DescribeMacro{\childdocof}
Furthermore, add the commands
\begin{center}
\begin{tabular}{l}
|\input{childdoc.def}|\\
|\childdocof{|\textit{main}|}|\\
\end{tabular}
\end{center}
at the top of every child file \textit{child}
which is included by |\include{|\textit{child}|}|
from within the main file
(or at least for those files to be compiled individually).
The argument \textit{main} must be the filename of the main file.

There are a couple of
considerations in setting up the main and child documents:

%%%%%%%%%%%%%%%%%%%%%%%%%%%%%%%%%%%%%%%%
\paragraph{Restrictions.}

Please note the following restrictions:
\begin{itemize}
\item
|\childdocmain| must be called with one argument \textit{main}
to ensure compatibility with earlier version of the package.
It must either be empty (|\childdocmain{}|)
or precisely match the filename of the main file in which it is specified.
See \secref{sec:detection} for further information.
\item
The filename \textit{main} must be specified without the |.tex| extension.
\item
The filename \textit{main} is case sensitive
(even in case-insensitive file systems)
due to internal string comparison.
\item
The argument \textit{main} should be fully expanded, it cannot be a macro.
\item
Subdirectories and special characters should be avoided in filenames.
\item
The command |\childdocmain{|\textit{main}|}| must be followed by a whitespace.
It should not be followed immediately by another command
or by a comment mark `|%|'.
This is because the \TeX{} parser reads the token immediately following
the argument of |\childdocmain| and puts it
at the beginning of every child section;
however, a white\-space is ignored.
\end{itemize}

%%%%%%%%%%%%%%%%%%%%%%%%%%%%%%%%%%%%%%%%
\paragraph{Content of Main File.}

It is advisable to place all content in the child files included by |\include|.
Any output contained in the main file will appear in all child documents
unless suppressed manually;
it cannot be suppressed automatically by the |\includeonly| directive
and thus should normally be avoided.
A method to include some content in the main file
by means of conditional processing is described in \secref{sec:conditional}.

%%%%%%%%%%%%%%%%%%%%%%%%%%%%%%%%%%%%%%%%
\paragraph{Page Numbering.}

When only a part of the document is compiled,
the appropriate numbering of pages
(as well as other status parameters)
is determined from the |.aux| files.
The latter contain information from previous passes.
However this information needs to propagate through
all intermediate child documents.
Therefore the page numbering in child documents may well
be inconsistent until the complete document is compiled at least once.

A useful (if unconventional) way to always ensure a consistent
page numbering is to restart the numbering in each child document
and denote the pages by `\textit{child}|.|\textit{page}'
where \textit{child} represents the chapter/section number of the child file.
This can be achieved by the command
|\numberwithin{page}{|\textit{child}|}|
of the \textsf{amsmath} package
where \textit{child} can be |chapter| or |section|
depending on the chosen structuring.
Alternatively, one can modify the macro |\thepage| appropriately
and reset the counter |page| at the start of each child file.

%%%%%%%%%%%%%%%%%%%%%%%%%%%%%%%%%%%%%%%%%%%%%%%%%%%%%%%%%%%%%%%%%%%%%%%%%%%%%%%%
\subsection{Conditional Processing}
\label{sec:conditional}

The package provides a mechanism to compile different versions
of a document. To customise the versions further some conditional processing
can come in handy to distinguish which version is being compiled.
The package provides two macros to describe the compilation context:

%%%%%%%%%%%%%%%%%%%%%%%%%%%%%%%%%%%%%%%%
\DescribeMacro{\ifchilddoc}
The conditional |\ifchilddoc| distinguishes between the compilation of
child documents and the main document:
%
\begin{center}
|\ifchilddoc |\textit{child-code}| |[|\||else |\textit{main-code}]| \||fi|
\end{center}

%%%%%%%%%%%%%%%%%%%%%%%%%%%%%%%%%%%%%%%%
\DescribeMacro{\childdocname}
\DescribeMacro{\childdocjob}
The macro |\childdocname| contains the filename (without extension)
of the main or child file being processed.
Note that |\childdocjob| will always contain the name of the main file.

%%%%%%%%%%%%%%%%%%%%%%%%%%%%%%%%%%%%%%%%
\paragraph{Title Page.}

Conditional processing can be used to include a title or banner page
in the main document when proper precautions are taken.
Importantly, the code in the main file should ensure that the page counter
(as well as other status parameters which are stored in the |.aux| files)
takes the same value after the conditional processing.
Otherwise the page numbers may take divergent values
depending on which part is compiled.

For example, a title page could be declared by:
%
\begin{center}
\begin{tabular}{l}
|\ifchilddoc\||else|\\
|\addtocounter{page}{-1}|\\
\textit{code for title page}\\
|\newpage|\\
|\||fi|
\end{tabular}
\end{center}
%
A banner page for the child documents can be generated by:
%
\begin{center}
\begin{tabular}{l}
|\ifchilddoc|\\
|\addtocounter{page}{-1}|\\
\textit{code for banner page}\\
|\newpage|\\
|\||fi|
\end{tabular}
\end{center}
%
Here one could write a message such as:
\begin{center}
|This is the part \childdocname{} of \childdocjob{}.|
\end{center}

%%%%%%%%%%%%%%%%%%%%%%%%%%%%%%%%%%%%%%%%%%%%%%%%%%%%%%%%%%%%%%%%%%%%%%%%%%%%%%%%
\subsection{Flags}
\label{sec:flags}

The package makes it easy to generate different versions
of the main or child documents.
To this end compilation flags can be defined
and assigned different default values.
They will be particularly useful in conjunction
with the forwarding mechanism described in \secref{sec:forward}.

For example, it may be useful to have a flag |\version|
which can be set to |draft| or |final|.
The document source will contain some conditional code
depending on the value of |\version|.
Suppose further, the flag should default to |final| for the main file
and to |draft| for child files
which is a natural assignment for editing the document.
This is achieved by placing the following code
in the preamble of the main document
(below the |\childdocmain| directive):
%
\begin{center}
\begin{tabular}{l}
|\ifchilddoc|\\
|\providecommand{\version}{draft}|\\
|\||else|\\
|\providecommand{\version}{final}|\\
|\||fi|
\end{tabular}
\end{center}
%
The definition by |\providecommand| makes sure
that previous definitions are not overwritten.
Further statements |\providecommand{\version}{...}|
can thus be added before the above code to override it.

For the main file, one might add a line
(between |\childdocmain| and the above block)
%
\begin{center}
|%\ifchilddoc\||else\providecommand{\version}{draft}\||fi|
\end{center}
%
which can be uncommented to produce a draft version.
Likewise one can add a line to the very top of a child file
(above the |\childdocof{|\textit{main}|}| directive)
%
\begin{center}
|%\providecommand{\version}{final}|
\end{center}
%
which can be uncommented to produce the final version of this child document.

%%%%%%%%%%%%%%%%%%%%%%%%%%%%%%%%%%%%%%%%%%%%%%%%%%%%%%%%%%%%%%%%%%%%%%%%%%%%%%%%
\subsection{Forwarding}
\label{sec:forward}

Different versions of the main or child documents
using compilation flags as described in \secref{sec:flags}
can be (permanently) stored in different files
for convenient compilation, viewing and distribution.
To this end, the package defines a command
to pass on compilation to a different file:

%%%%%%%%%%%%%%%%%%%%%%%%%%%%%%%%%%%%%%%%
\DescribeMacro{\childdocforward}
The command |\childdocforward| redirects processing to
another source file:
%
\begin{center}
\begin{tabular}{l}
|\input{childdoc.def}|\\
|\childdocforward[|\textit{main}|]{|\textit{dest}|}|\\
\end{tabular}
\end{center}
%
The argument \textit{dest} is the destination file
(without extension).
It should be the main file or one of the child files.
Note that further \textsf{childdoc} directives
such as |\childdocof| and |\childdocforward|
in the indicated file will be processed in this form.
The optional argument \textit{main}
passes on directly to the main file \textit{main}
while pretending to compile the child \textit{dest}.
This form behaves as if \textit{dest}
issues |\childdocof{|\textit{main}|}| right away,
and no further \textsf{childdoc} directives will be processed.

%%%%%%%%%%%%%%%%%%%%%%%%%%%%%%%%%%%%%%%%
\DescribeMacro{\...prefix}
In the alternative form |\childdocforwardprefix|,
%
\begin{center}
\begin{tabular}{l}
|\input{childdoc.def}|\\
|\childdocforwardprefix[|\textit{main}|]{|\textit{prefix}|}{|\textit{dest}|}|
\end{tabular}
\end{center}
%
the destination file is determined by a pattern
depending on the current file:
To make this work, the current file must be called
`{\textit{prefix}\hspace{0.2em}\textit{suffix}}'
with \textit{prefix} matching precisely the argument.
Processing is then passed on to the file
`{\textit{dest}\hspace{0.2em}\textit{suffix}}'.
Surely, the same effect is achieved by
directly specifying the
argument `{\textit{dest}\hspace{0.2em}\textit{suffix}}'
in the first form.
However, that requires to set up a different file
for each child. With the alternative form of the command
all these files can have exactly the same content
which simplifies setting them up and maintaining them.

For example, the following file |draft.tex|
with a compilation flag |\version| as described in \secref{sec:flags}
compiles the main document as a draft:
%
\begin{center}
\begin{tabular}{l}
|\def\version{draft}|\\
|\input{childdoc.def}|\\
|\childdocforward{|\textit{main}|}|
\end{tabular}
\end{center}
%
Likewise, the following files |final|\textit{nn}|.tex|
compile the final version of the child document
|child|\textit{nn}|.tex|:
%
\begin{center}
\begin{tabular}{l}
|\def\version{final}|\\
|\input{childdoc.def}|\\
|\childdocforwardprefix{final}{child}|
\end{tabular}
\end{center}
%

Note that when several versions of a main file and/or of each child file
are to be generated, it may be convenient to set up a |Makefile| or
shell script to automatise the process.

%%%%%%%%%%%%%%%%%%%%%%%%%%%%%%%%%%%%%%%%%%%%%%%%%%%%%%%%%%%%%%%%%%%%%%%%%%%%%%%%
\subsection{Command Line Processing}
\label{sec:commandline}

The effect of redirection files can also be achieved by invoking
the \LaTeX{} compiler with a more elaborate command line.
Most conveniently this should be done as part
of a shell script or a |Makefile|.

When using \textsf{childdoc} in the main file, the following
command lines effectively perform a redirection
(note that depending on the shell being used,
backslashes may have to be doubled: `|\|' $\to$ `|\\|'):
%
\begin{center}
|... -jobname "|\textit{target}|" |\\|"|[\textit{flags}]%
|\input{childdoc.def}\childdocforward[|\textit{main}|]{|\textit{dest}|}"|
\end{center}
%
Here \textit{target} is the name of the output file,
\textit{main} is the name of the main file
and \textit{dest} is the name of the main or child file to be processed
(all filenames without extensions).
The optional argument \textit{main} can be omitted
if \textit{main} matches \textit{dest}.
Optionally, compilation \textit{flags} can be defined via |\def| commands.
This command line makes the \TeX{} engine believe
it is compiling the file \textit{target}
whose content is specified as the latter parameter.
The provided code then forwards the processing to
\textit{main} or \textit{dest} as described in \secref{sec:forward}.

%%%%%%%%%%%%%%%%%%%%%%%%%%%%%%%%%%%%%%%%%%%%%%%%%%%%%%%%%%%%%%%%%%%%%%%%%%%%%%%%
\subsection{Include by Input}
\label{sec:input}

Including child documents by |\include| has some restrictions by design.
Most notably, the content of a child document always occupies
its own set of pages; pages cannot be shared between child documents.
Usually, this behaviour makes perfect sense
because each child document contain an essential part of the document.
However, in some situations it may be desirable to compose
a document from a collection of parts
without having mandatory page breaks between then.
For this case, the package
provides a mechanism to include parts
by |\input| which can also be processed individually.
However, by construction this mechanism
requires manual handling of the content to be output.

%%%%%%%%%%%%%%%%%%%%%%%%%%%%%%%%%%%%%%%%
\DescribeMacro{\ifchilddocmanual}
The main file should be prepared as usual, see \secref{sec:include}.
However, the document body must make a distinction
between processing of an individual part and of the main document, e.g.:
%
\begin{center}
\begin{tabular}{l}
|\ifchilddocmanual|\\
|\input{\childdocname}|\\
|\||else|\\
\textit{document body with }|\input{|\textit{part}|}|\\
|\||fi|
\end{tabular}
\end{center}
%
The conditional |\ifchilddocmanual| is true whenever
a part to be included by |\input| is being compiled,
and the name of the part is stored in |\childdocname|.

%%%%%%%%%%%%%%%%%%%%%%%%%%%%%%%%%%%%%%%%
\DescribeMacro{\childdocby}
Each part to be included by |\input| should start with:
%
\begin{center}
\begin{tabular}{l}
|\input{childdoc.def}|\\
|\childdocby{|\textit{main}|}|\\
\end{tabular}
\end{center}
%
The directive |\childdocby| is similar to |\childdocof|
described in \secref{sec:include},
but the subsequent selection of content must be done manually.
To that end, both |\ifchilddoc| and |\ifchilddocmanual|
will be true upon processing of a part,
and the name of the part is stored in |\childdocname|.
Note that |\jobname| will be set to the filename of the current part
so that each part receives an individual |.aux| file
that does not interfere with the |.aux| file(s) of the main document.
This behaviour can be altered by the alternative form
|\childdocby[*]{|\textit{main}|}| (with a non-empty optional argument)
which uses the |.aux| file of the main document
by setting |\jobname| to \textit{main}.

%%%%%%%%%%%%%%%%%%%%%%%%%%%%%%%%%%%%%%%%%%%%%%%%%%%%%%%%%%%%%%%%%%%%%%%%%%%%%%%%
\subsection{Driver Development}
\label{sec:driver}

The \textsf{childdoc} mechanism can also be use for the development
of definition files such as \LaTeX{} styles or classes.
This case differs from the above setup with multiple parts
included by |\include| in that no |\includeonly| should be invoked.
This can be achieved by starting the include file
(before |\ProvidesPackage|) with:
%
\begin{center}
\begin{tabular}{l}
|\input{childdoc.def}|\\
|\childdocforward{|\textit{main}|}|\\
\end{tabular}
\end{center}
%
or alternatively with:
%
\begin{center}
\begin{tabular}{l}
|\input{childdoc.def}|\\
|\childdocby{|\textit{main}|}|\\
\end{tabular}
\end{center}
%
Both forms have slightly different effects as described above.
The main file is prepared as usual, see \secref{sec:include}.

%%%%%%%%%%%%%%%%%%%%%%%%%%%%%%%%%%%%%%%%%%%%%%%%%%%%%%%%%%%%%%%%%%%%%%%%%%%%%%%%
\subsection{Legacy Detection}
\label{sec:detection}

The directive |\childdocmain| in the main file can detect
whether the complete document or merely a child is to be compiled
even without using the directive |\childdocof|.
This method is deprecated because it is less robust
and there is no compelling reason to use it;
it is merely provided for backward compatibility
and it may be removed in future versions.

If the detection mechanism is to be used,
it is mandatory to correctly specify
the filename of the main file as the argument of |\childdocmain|:
%
\begin{center}
\begin{tabular}{l}
|\input{childdoc.def}|\\
|\childdocmain{|\textit{main}|}|\\
\end{tabular}
\end{center}
%
If |\jobname| does not match the argument \textit{main} of |\childdocmain|,
it is assumed that |\jobname| points to the child file to be compiled.
When using |\childdocmain| with the main file specified as argument,
it suffices to start a child file
with just |\input{|\textit{main}|}|
without loading of the package and using |\childdocof|.
If instead all processing is done
with the appropriate \textsf{childdoc} directives,
the argument of \textit{main} of |\childdocmain| can be empty.

An alternative version of the command line processing described
in \secref{sec:commandline} using the detection mechanism reads:
%
\begin{center}
|... -jobname "|\textit{target}|" "|[\textit{flags}]%
[|\def\jobname{|\textit{dest}|}|]|\input{|\textit{main}|}"|
\end{center}

%%%%%%%%%%%%%%%%%%%%%%%%%%%%%%%%%%%%%%%%%%%%%%%%%%%%%%%%%%%%%%%%%%%%%%%%%%%%%%%%
\subsection{Manual Code}
\label{sec:manual}

In case one cannot be certain whether the definitions file |childdoc.def|
is installed on the target \TeX{} distribution
and one prefers not to ship it,
it is conceivable to paste a few relevant commands into the sources.

To that end, drop all statements |\input{childdoc.def}|
and perform the replacements as outlined below.
Instead of |\childdocmain{|\textit{main}|}| add the following code
to the top of the main file:
%
\begin{center}
\begin{tabular}{l}
|\||ifdefined\childdocname\endinput\||fi\newif\ifchilddoc|\\
|\edef\childdocname{\scantokens\expandafter{\jobname\noexpand}}|\\
|\def\childdocmain{|\textit{main}|}\||ifx\childdocmain\childdocname\||else|\\
|\childdoctrue\includeonly{\childdocname}\let\jobname\childdocmain\||fi|\\
\end{tabular}
\end{center}
%
Instead of |\childdocof{|\textit{main}|}| just include the main file
at the top of each child file:
%
\begin{center}
|\input{|\textit{main}|}|
\end{center}
%
A simple redirection |\childdocforward{|\textit{dest}|}| is achieved by:
%
\begin{center}
|\def\jobname{|\textit{dest}|}\input{\jobname}|
\end{center}
%
The redirection with prefix
|\childdocforwardprefix[|\textit{prefix}|]{|\textit{dest}|}|
is accomplished by:
%
\begin{center}
\begin{tabular}{l}
|{\edef\jobname{\scantokens\expandafter{\jobname\noexpand}}|\\
|\def\redirectjob |\textit{prefix}|#1~~~{\gdef\jobname{|\textit{dest}|#1}}|\\
|\expandafter\redirectjob\jobname~~~}\input{\jobname}|
\end{tabular}
\end{center}

In an alternative approach,
child documents can be compiled by a specific command line
without additional code or specific definitions:
%
\begin{center}
|... -jobname "|\textit{target}|" "|[\textit{flags}]%
|\includeonly{|\textit{dest}|}\input{|\textit{main}|}"|
\end{center}
%

%%%%%%%%%%%%%%%%%%%%%%%%%%%%%%%%%%%%%%%%%%%%%%%%%%%%%%%%%%%%%%%%%%%%%%%%%%%%%%%%
%%%%%%%%%%%%%%%%%%%%%%%%%%%%%%%%%%%%%%%%%%%%%%%%%%%%%%%%%%%%%%%%%%%%%%%%%%%%%%%%
\section{Information}

%%%%%%%%%%%%%%%%%%%%%%%%%%%%%%%%%%%%%%%%%%%%%%%%%%%%%%%%%%%%%%%%%%%%%%%%%%%%%%%%
\subsection{Copyright}

Copyright \copyright{} 2017--2018 Niklas Beisert

This work may be distributed and/or modified under the
conditions of the \LaTeX{} Project Public License, either version 1.3
of this license or (at your option) any later version.
The latest version of this license is in
  \url{http://www.latex-project.org/lppl.txt}
and version 1.3 or later is part of all distributions of \LaTeX{}
version 2005/12/01 or later.

This work has the LPPL maintenance status `maintained'.

The Current Maintainer of this work is Niklas Beisert.

This work consists of the files |README.txt|, |childdoc.ins| and |childdoc.dtx|
as well as the derived files |childdoc.def|, |cdocsamp.tex|
with |cdocsch1.tex|, |cdocsch2.tex|, |cdocspt3.tex|, |cdocspt4.tex|,
|cdocsdrf.tex|, |cdocsfn1.tex|, |cdocsfn2.tex|
as well as |childdoc.pdf|.

%%%%%%%%%%%%%%%%%%%%%%%%%%%%%%%%%%%%%%%%%%%%%%%%%%%%%%%%%%%%%%%%%%%%%%%%%%%%%%%%
\subsection{Files and Installation}

The package consists of the files:
%
\begin{center}
\begin{tabular}{ll}
    |README.txt|   & readme file \\
    |childdoc.ins| & installation file \\
    |childdoc.dtx| & source file \\
    |childdoc.def| & definition file \\
    |cdocsamp.tex| & sample main file \\
    |cdocsch1.tex| & sample include file \\
    |cdocsch2.tex| & sample include file \\
    |cdocspt3.tex| & sample part file \\
    |cdocspt4.tex| & sample part file \\
    |cdocsdrf.tex| & sample redirection file \\
    |cdocsfn1.tex| & sample redirection file \\
    |cdocsfn2.tex| & sample redirection file \\
    |childdoc.pdf| & manual
\end{tabular}
\end{center}
%
The distribution consists of the files
|README.txt|, |childdoc.ins| and |childdoc.dtx|.
%
\begin{itemize}
\item
Run (pdf)\LaTeX{} on |childdoc.dtx|
to compile the manual |childdoc.pdf| (this file).
\item
Run \LaTeX{} on |childdoc.ins| to create the definitions file |childdoc.def|
and the sample |cdocsamp.tex| with include files
|cdocsch1.tex|, |cdocsch2.tex|, |cdocspt3.tex|, |cdocspt4.tex|,
|cdocsdrf.tex|, |cdocsfn1.tex|, |cdocsfn2.tex|.
Then copy the file |childdoc.def| to an appropriate directory of your \LaTeX{}
distribution, e.g.\ \textit{texmf-root}|/tex/latex/childdoc|.
\end{itemize}

%%%%%%%%%%%%%%%%%%%%%%%%%%%%%%%%%%%%%%%%%%%%%%%%%%%%%%%%%%%%%%%%%%%%%%%%%%%%%%%%
\subsection{Related CTAN Packages}

There are several other packages which offer a similar functionality:
%
\begin{itemize}
\item
The packages
\href{http://ctan.org/pkg/docmute}{\textsf{docmute}},
\href{http://ctan.org/pkg/includex}{\textsf{includex}} and
\href{http://ctan.org/pkg/standalone}{\textsf{standalone}}
provide commands to include only the document body of
a child file thus allowing both files to be compiled individually.
\item
The packages \href{http://ctan.org/pkg/subdocs}{\textsf{subdocs}}
and \href{http://ctan.org/pkg/subfiles}{\textsf{subfiles}}
provide structures in which the main and child documents can be
encapsulated and allowing them to be compiled individually.
The inclusion mechanism is different from the conventional |\include|.
\item
The package \href{http://ctan.org/pkg/combine}{\textsf{combine}}
is an elaborate solution to combine several documents into one.
\end{itemize}
%
See also the CTAN topic \href{http://ctan.org/topic/subdocs}{\textsf{subdocs}}
for further related packages.
The present package differs from the above solutions in that
a document structure constructed with the conventional |\include| mechanism
just needs two extra commands at the top of every file
such that all constituent files can be compiled individually.

%%%%%%%%%%%%%%%%%%%%%%%%%%%%%%%%%%%%%%%%%%%%%%%%%%%%%%%%%%%%%%%%%%%%%%%%%%%%%%%%
%\subsection{Feature Suggestions}
%
%The following is a list of features which may be useful for future
%versions of this package:
%%
%\begin{itemize}
%\item
%\ldots
%\end{itemize}

%%%%%%%%%%%%%%%%%%%%%%%%%%%%%%%%%%%%%%%%%%%%%%%%%%%%%%%%%%%%%%%%%%%%%%%%%%%%%%%%
\subsection{Revision History}

%%%%%%%%%%%%%%%%%%%%%%%%%%%%%%%%%%%%%%%%
\paragraph{v2.0:} 2018/12/30

\begin{itemize}
\item
immediate forward processing
\item
added |\childdocby| mechanism
\item
manual restructured
\end{itemize}

%%%%%%%%%%%%%%%%%%%%%%%%%%%%%%%%%%%%%%%%
\paragraph{v1.6:} 2018/01/17

\begin{itemize}
\item
application for development of include files
\item
corrections to manual
\end{itemize}

%%%%%%%%%%%%%%%%%%%%%%%%%%%%%%%%%%%%%%%%
\paragraph{v1.5:} 2017/05/21

\begin{itemize}
\item
more complete structuring introduced
\item
|\childdocof| introduced
\item
|\childdoc| renamed to |\childdocmain|
\item
|\childredirect| renamed to |\childdocforward| and |\childdocforwardprefix|
and functionality expanded
\end{itemize}

%%%%%%%%%%%%%%%%%%%%%%%%%%%%%%%%%%%%%%%%
\paragraph{v1.0:} 2017/04/27

\begin{itemize}
\item
manual and install package
\item
first version published on CTAN
\end{itemize}

%%%%%%%%%%%%%%%%%%%%%%%%%%%%%%%%%%%%%%%%
\paragraph{v0.6:} 2017/04/26

\begin{itemize}
\item
redirection mechanism added
\end{itemize}

%%%%%%%%%%%%%%%%%%%%%%%%%%%%%%%%%%%%%%%%
\paragraph{v0.5:} 2017/04/26

\begin{itemize}
\item
functionality in definition file
\end{itemize}


%%%%%%%%%%%%%%%%%%%%%%%%%%%%%%%%%%%%%%%%%%%%%%%%%%%%%%%%%%%%%%%%%%%%%%%%%%%%%%%%
%%%%%%%%%%%%%%%%%%%%%%%%%%%%%%%%%%%%%%%%%%%%%%%%%%%%%%%%%%%%%%%%%%%%%%%%%%%%%%%%
%%%%%%%%%%%%%%%%%%%%%%%%%%%%%%%%%%%%%%%%%%%%%%%%%%%%%%%%%%%%%%%%%%%%%%%%%%%%%%%%
\appendix

\settowidth\MacroIndent{\rmfamily\scriptsize 000\ }

 \DocInput{childdoc.dtx}

\end{document}
%</driver>
% \fi
%
% %%%%%%%%%%%%%%%%%%%%%%%%%%%%%%%%%%%%%%%%%%%%%%%%%%%%%%%%%%%%%%%%%%%%%%%%%%%%%%
% %%%%%%%%%%%%%%%%%%%%%%%%%%%%%%%%%%%%%%%%%%%%%%%%%%%%%%%%%%%%%%%%%%%%%%%%%%%%%%
% \section{Sample}
%\iffalse
%<*samplemain>
%\fi
%
% The following presents a sample document
% with two chapters, two parts, a title page,
% a compile flag as well as three forwarding files to set the flag.
% It consists of eight |.tex| files:
% \begin{center}
% \begin{tabular}{ll}
% |cdocsamp.tex|&main file\\
% |cdocsch1.tex|&include file for chapter 1\\
% |cdocsch2.tex|&include file for chapter 2\\
% |cdocspt3.tex|&include file for part 3\\
% |cdocspt4.tex|&include file for part 4\\
% |cdocsdrf.tex|&forwarding file for main file in draft mode\\
% |cdocsfi1.tex|&forwarding file for final version of chapter 1\\
% |cdocsfi2.tex|&forwarding file for final version of chapter 2\\
% \end{tabular}
% \end{center}
% Each of the eight files can be compiled directly by the \LaTeX{} compiler.
%
% %%%%%%%%%%%%%%%%%%%%%%%%%%%%%%%%%%%%%%
% \paragraph{Main File.}
%
% The main file is called |cdocsamp.tex|.
%
% Load the \textsf{childdoc} definitions and
% declare the filename for the main document:
%    \begin{macrocode}
\input{childdoc.def}
\childdocmain{}
%    \end{macrocode}

% Optional override for |\version| flag:
%    \begin{macrocode}
%%\ifchilddoc\else\providecommand{\version}{draft}\fi
%    \end{macrocode}

% Define the default values for the |\version| flag
% (|final| for the main file and |draft| for childs):
%    \begin{macrocode}
\ifchilddoc
\providecommand{\version}{draft}
\else
\providecommand{\version}{final}
\fi
%    \end{macrocode}

% Load the standard document class:
%    \begin{macrocode}
\documentclass[12pt]{article}
%    \end{macrocode}

% Start the document body:
%    \begin{macrocode}
\begin{document}
%    \end{macrocode}

% Declare a title page.
% Print title, part of document being processed and version flag:
%    \begin{macrocode}
\addtocounter{page}{-1}
\begin{center}
{\LARGE\bfseries{}childdoc example\par}
\vspace{1cm}
\ifchilddoc
\ifchilddocmanual part\else chapter\fi:
`\childdocname' of `\childdocjob'\par
\else
main document: `\childdocjob'\par
\fi
version: \version\par
\end{center}
\newpage
%    \end{macrocode}

% Manually include selected file,
% otherwise process as usual:
%    \begin{macrocode}
\ifchilddocmanual
\section*{part `\childdocname'}
\input{\childdocname}
\else
%    \end{macrocode}

% Include the two chapters:
%    \begin{macrocode}
\include{cdocsch1}
\include{cdocsch2}
%    \end{macrocode}

% Include the two parts unless only chapters should be displayed:
%    \begin{macrocode}
\ifchilddoc\else
\section{part three}
\input{cdocspt3}
\section{part four}
\input{cdocspt4}
\fi
%    \end{macrocode}

% Process as usual until here:
%    \begin{macrocode}
\fi
%    \end{macrocode}

% End of document body:
%    \begin{macrocode}
\end{document}
%    \end{macrocode}
%\iffalse
%</samplemain>
%\fi
%
% %%%%%%%%%%%%%%%%%%%%%%%%%%%%%%%%%%%%%%
% \paragraph{Chapter Include Files.}
%
% The include files are called |cdocsch1.tex| and |cdocsch2.tex|.
%
%\iffalse
%<*samplechap1|samplechap2>
%\fi

% Optional override for |\version| flag:
%    \begin{macrocode}
%%\providecommand{\version}{final}
%    \end{macrocode}

% Include the main document:
%    \begin{macrocode}
\input{childdoc.def}
\childdocof{cdocsamp}
%    \end{macrocode}

%\iffalse
%</samplechap1|samplechap2>
%\fi
%
%\iffalse
%<*samplechap1>
%\fi
% Some text for chapter 1:
%    \begin{macrocode}
\section{one}
some text in chapter one
%    \end{macrocode}

%\iffalse
%</samplechap1>
%\fi
% Some text for chapter 2:
%\iffalse
%<*samplechap2>
%\fi
%    \begin{macrocode}
\section{two}
more text in chapter two
%    \end{macrocode}

%\iffalse
%</samplechap2>
%\fi
%
% %%%%%%%%%%%%%%%%%%%%%%%%%%%%%%%%%%%%%%
% \paragraph{Part Include Files.}
%
% The include files are called |cdocspt3.tex| and |cdocspt4.tex|.
%
%\iffalse
%<*samplepart3|samplepart4>
%\fi

% Optional override for |\version| flag:
%    \begin{macrocode}
%%\providecommand{\version}{final}
%    \end{macrocode}

% Include the main document:
%    \begin{macrocode}
\input{childdoc.def}
\childdocby{cdocsamp}
%    \end{macrocode}

%\iffalse
%</samplepart3|samplepart4>
%\fi
%
%\iffalse
%<*samplepart3>
%\fi
% Some text for part 3:
%    \begin{macrocode}
some text in part three
%    \end{macrocode}

%\iffalse
%</samplepart3>
%\fi
% Some text for part 4:
%\iffalse
%<*samplepart4>
%\fi
%    \begin{macrocode}
more text in part four
%    \end{macrocode}

%\iffalse
%</samplepart4>
%\fi
%
% %%%%%%%%%%%%%%%%%%%%%%%%%%%%%%%%%%%%%%
% \paragraph{Forwarding for a Complete Draft.}
%
% The following forwarding file |cdocsdrf.tex|
% compiles the main document in draft mode:
%\iffalse
%<*sampledraft>
%\fi
%    \begin{macrocode}
\def\version{draft}
\input{childdoc.def}
\childdocforward{cdocsamp}
%    \end{macrocode}

%\iffalse
%</sampledraft>
%\fi
%
% %%%%%%%%%%%%%%%%%%%%%%%%%%%%%%%%%%%%%%
% \paragraph{Forwarding for Final Version of the Chapters.}
%
% The following forwarding files |cdocsfn1.tex| and |cdocsfn2.tex|
% (with identical content)
% compile the final versions of the child documents
% |cdocsch1.tex| and |cdocsch2.tex|, respectively:
%\iffalse
%<*samplefinal>
%\fi
%    \begin{macrocode}
\def\version{final}
\input{childdoc.def}
\childdocforwardprefix[cdocsamp]{cdocsfn}{cdocsch}
%    \end{macrocode}

%\iffalse
%</samplefinal>
%\fi
%
% %%%%%%%%%%%%%%%%%%%%%%%%%%%%%%%%%%%%%%
% \paragraph{Command Line Processing.}
%
% The following three command lines generate the output files
% |cdocscld|, |cdocscl1| and |cdocscl2|
% which should be identical to
% |cdocsdrf|, |cdocsch1| and |cdocsfn2|, respectively:
% \begin{center}
% \begin{tabular}{l}
% |latex -jobname cdocscld \|\\
% |  "\def\version{draft}\input{childdoc.def}\childdocforward{cdocsamp}"|\\
% |latex -jobname cdocscl1 \|\\
% |  "\input{childdoc.def}\childdocforward[cdocsamp]{cdocsch1}"|\\
% |latex -jobname cdocscl2 \|\\
% |  "\def\version{final}\input{childdoc.def}\childdocforward{cdocsch2}"|
% \end{tabular}
% \end{center}
% Note that the trailing backslash on each first line
% merely continues the input to the second line
% (for convenient cut ant paste).
% Furthermore, the command |latex| can be replaced by any
% of its alternative versions such as |pdflatex|.
%
% %%%%%%%%%%%%%%%%%%%%%%%%%%%%%%%%%%%%%%%%%%%%%%%%%%%%%%%%%%%%%%%%%%%%%%%%%%%%%%
% %%%%%%%%%%%%%%%%%%%%%%%%%%%%%%%%%%%%%%%%%%%%%%%%%%%%%%%%%%%%%%%%%%%%%%%%%%%%%%
% \section{Implementation}
%\iffalse
%<*package>
%\fi
%
% This section describes the definitions file |childdoc.def|.

% The definitions cannot be loaded using |\usepackage| or |\RequirePackage|
% which has a mechanism to prevent loading a style file more than once.
% When loading the definitions by means of |\input|
% multiple instances have to be prevented manually:
%\iffalse
%This code needs to be before the `\ProvidesFile' directive
%which is defined at the beginning of this file.
%Therefore it is also placed there and commented out here.
%</package>
%<*discard>
%\fi
%    \begin{macrocode}
\ifdefined\childdocmain\endinput\fi
%    \end{macrocode}
%\iffalse
%</discard>
%<*package>
%\fi
%
% \macro{\ifchilddoc}
% \macro{\ifchilddocmanual}
% The conditional |\ifchilddoc| tells whether a
% child (true) or main (false) document is being compiled.
% The conditional |\ifchilddocmanual| tells whether
% the |\includeonly| mechanism is used (false) or
% the selection of child files must be performed manually (true).
% The definitions initialise to false:
%    \begin{macrocode}
\newif\ifchilddoc
\newif\ifchilddocmanual
%    \end{macrocode}

% \macro{\childdocname}
% \macro{\childdocjob}
% The macro |\childdocname| stores the name of the main document
% to be compiled. The macro |\childdocjob| stores the name of
% the document on which the \LaTeX{} compiler was originally invoked.
% The content of |\jobname| cannot be compared
% to filenames specified in the source due to different catcodes.
% The following code rescans |\jobname|, stores the result
% in |\childdocname| and saves a copy in |\childdocjob|:
%    \begin{macrocode}
\edef\childdocname{\scantokens\expandafter{\jobname\noexpand}}
\let\childdocjob\childdocname
%    \end{macrocode}

% \macro{\childdocdisable}
% The macro |\childdocdisable| prevents the main file
% from being processed more than once.
% At this stage, the main document command |\childdocmain|
% is assumed to be called once again where it should do nothing.
% Any subsequent call to it should prevent
% a secondary processing of the main document
% It overwrites the forwarding commands
% |\childdocof| and |\childdocforward|
% with empty macros to prevent further inclusions of the main document:
%    \begin{macrocode}
\newcommand{\childdocdisable}
{
  \renewcommand{\childdocmain}[1]{\renewcommand{\childdocmain}[1]{\endinput}}
  \renewcommand{\childdocof}[1]{}
  \renewcommand{\childdocby}[2][]{}
  \renewcommand{\childdocforward}[2][]{}
  \renewcommand{\childdocdisable}{}
}
%    \end{macrocode}

% \macro{\childdocmain}
% The macro |\childdocmain| is to be called at the top of the main file
% with nothing or the main filename (without extension) as argument.
% First, it breaks loops.
% If the argument is not empty and does not match |\childdocname|
% (which is set by the first inclusion of |childdoc.def|),
% |\ifchilddoc| is set to true, |\includeonly| is applied to the child file
% and |\jobname| is set to the main file
% (for proper handling of |.aux| files):
%    \begin{macrocode}
\newcommand{\childdocmain}[1]
{
  \childdocdisable\childdocmain{}
  \if?#1?\else
    \begingroup
      \def\childdoctmp{#1}
      \ifx\childdoctmp\childdocname
        \def\childdoctmp{}
      \else
        \def\childdoctmp
        {
          \childdoctrue
          \includeonly{\childdocname}
          \def\childdocjob{#1}
          \def\jobname{#1}
        }
      \fi
      \expandafter
    \endgroup
    \childdoctmp
  \fi
}
%    \end{macrocode}

% \macro{\childdocof}
% The command |\childdocof| redirects
% compilation to the main file |#1|.
%    \begin{macrocode}
\newcommand{\childdocof}[1]
{
  \childdocdisable
  \childdoctrue
  \includeonly{\childdocname}
  \def\jobname{#1}
  \def\childdocjob{#1}
  \input{#1}
}
%    \end{macrocode}

% \macro{\childdocby}
% The command |\childdocby| ....
%    \begin{macrocode}
\newcommand{\childdocby}[2][]
{
  \childdocdisable
  \childdoctrue
  \childdocmanualtrue
  \if?#1?\else
    \def\jobname{#2}
  \fi
  \def\childdocjob{#2}
  \input{#2}
  \endinput
}
%    \end{macrocode}

% \macro{\childdocforward}
% The command |\childdocforward| redirects
% compilation to the main file or
% (if the optional argument is given) a child file.
% Parameters are set as if the main file
% or a child file starting with |\childdocof| was compiled.
% Then compilation is handed over to the main file:
%    \begin{macrocode}
\newcommand{\childdocforward}[2][]
{
  \begingroup
    \if?#1?
      \def\childdoctmp
      {
        \def\childdocname{#2}
        \def\childdocjob{#2}
        \def\jobname{#2}
        \input{#2}
        \endinput
      }
    \else
      \def\childdoctmp
      {
        \childdocdisable
        \def\childdocname{#2}
        \childdoctrue
        \includeonly{#2}
        \def\childdocjob{#1}
        \def\jobname{#1}
        \input{#1}
        \endinput
      }
    \fi
    \expandafter
  \endgroup
  \childdoctmp
}
%    \end{macrocode}

% \macro{\childdocforwardprefix}
% The command |\childdocforwardprefix| redirects
% compilation to the main or a child file by means of a pattern.
% The prefix |#1| in the current filename is replaced by |#2|
% and the suffix of the current filename is kept
% (it is assumed that the filename does not contain the substring `|~~~|'
% which is used as a delimiter).
% Compilation is handed over to the new file by |\childdocforward|:
%    \begin{macrocode}
\newcommand{\childdocforwardprefix}[3][]
{
  \begingroup
    \def\childdocextract #2##1~~~{\def\childdoctmp{\childdocforward[#1]{#3##1}}}
    \expandafter\childdocextract\childdocname~~~
    \expandafter
  \endgroup
  \childdoctmp
}
%    \end{macrocode}

% \macro{\childdoc}
% The deprecated macro |\childdoc| is a legacy version of |\childdocmain|:
%    \begin{macrocode}
\newcommand{\childdoc}{\childdocmain}
%    \end{macrocode}

% \macro{\childdocredirect}
% The deprecated macro |\childdocredirect| is a legacy version
% of |\childdocforward| and |\childdocforwardprefix|:
%    \begin{macrocode}
\newcommand{\childdocredirect}[2][]
{
  \begingroup
    \if?#1?
      \def\childdoctmp{\childdocforward{#2}}
    \else
      \def\childdoctmp{\childdocforwardprefix{#1}{#2}}
    \fi
    \expandafter
  \endgroup
  \childdoctmp
}
%    \end{macrocode}

%\iffalse
%</package>
%\fi
%
\endinput

\childdocforwardprefix[cdocsamp]{cdocsfn}{cdocsch}
%    \end{macrocode}

%\iffalse
%</samplefinal>
%\fi
%
% %%%%%%%%%%%%%%%%%%%%%%%%%%%%%%%%%%%%%%
% \paragraph{Command Line Processing.}
%
% The following three command lines generate the output files
% |cdocscld|, |cdocscl1| and |cdocscl2|
% which should be identical to
% |cdocsdrf|, |cdocsch1| and |cdocsfn2|, respectively:
% \begin{center}
% \begin{tabular}{l}
% |latex -jobname cdocscld \|\\
% |  "\def\version{draft}% \iffalse
%
% childdoc.dtx Copyright (C) 2017-2018 Niklas Beisert
%
% This work may be distributed and/or modified under the
% conditions of the LaTeX Project Public License, either version 1.3
% of this license or (at your option) any later version.
% The latest version of this license is in
%   http://www.latex-project.org/lppl.txt
% and version 1.3 or later is part of all distributions of LaTeX
% version 2005/12/01 or later.
%
% This work has the LPPL maintenance status `maintained'.
%
% The Current Maintainer of this work is Niklas Beisert.
%
% This work consists of the files childdoc.dtx and childdoc.ins
% and the derived files childdoc.def and cdocsamp.tex with
% cdocsch1.tex, cdocsch2.tex, cdocsdrf.tex, cdocsfn1.tex, cdocsfn2.tex.
%
%<package>\ifdefined\childdocmain\endinput\fi
%<package>\ProvidesFile{childdoc.def}[2018/12/30 v2.0 child document driver]
%<samplemain>\ProvidesFile{cdocsamp.tex}[2018/12/30 v2.0 sample for childdoc]
%<*driver>
%\ProvidesFile{childdoc.drv}[2018/12/30 v2.0 childdoc reference manual file]
\PassOptionsToClass{10pt,a4paper}{article}
\documentclass{ltxdoc}

\usepackage[margin=35mm]{geometry}
\usepackage{hyperref}
\usepackage{hyperxmp}
\usepackage[usenames]{color}

\hypersetup{colorlinks=true}
\hypersetup{pdfstartview=FitH}
\hypersetup{pdfpagemode=UseNone}
\hypersetup{pdfsource={}}
\hypersetup{pdflang={en-UK}}
\hypersetup{pdfcopyright={Copyright 2017-2018 Niklas Beisert.
  This work may be distributed and/or modified under the
  conditions of the LaTeX Project Public License, either version 1.3
  of this license or (at your option) any later version.}}
\hypersetup{pdflicenseurl={http://www.latex-project.org/lppl.txt}}
\hypersetup{pdfcontactaddress={ETH Zurich, ITP, HIT K,
  Wolfgang-Pauli-Strasse 27}}
\hypersetup{pdfcontactpostcode={8093}}
\hypersetup{pdfcontactcity={Zurich}}
\hypersetup{pdfcontactcountry={Switzerland}}
\hypersetup{pdfcontactemail={nbeisert@itp.phys.ethz.ch}}
\hypersetup{pdfcontacturl={http://people.phys.ethz.ch/\xmptilde nbeisert/}}

\newcommand{\secref}[1]{\hyperref[#1]{section \ref*{#1}}}

\parskip1ex
\parindent0pt
\let\olditemize\itemize
\def\itemize{\olditemize\parskip0pt}

\begin{document}

\title{The \textsf{childdoc} Package}
\hypersetup{pdftitle={The childdoc Package}}
\author{Niklas Beisert\\[2ex]
  Institut f\"ur Theoretische Physik\\
  Eidgen\"ossische Technische Hochschule Z\"urich\\
  Wolfgang-Pauli-Strasse 27, 8093 Z\"urich, Switzerland\\[1ex]
  \href{mailto:nbeisert@itp.phys.ethz.ch}
  {\texttt{nbeisert@itp.phys.ethz.ch}}}
\hypersetup{pdfauthor={Niklas Beisert}}
\hypersetup{pdfsubject={Manual for the LaTeX2e Package childdoc}}
\date{30 December 2018, \textsf{v2.0}}
\maketitle

\begin{abstract}\noindent
\textsf{childdoc} is a \LaTeXe{} package
that enables the direct compilation
of document sections included by |\include|
to individual files.
\end{abstract}

\begingroup
\parskip0ex
\tableofcontents
\endgroup

%%%%%%%%%%%%%%%%%%%%%%%%%%%%%%%%%%%%%%%%%%%%%%%%%%%%%%%%%%%%%%%%%%%%%%%%%%%%%%%%
%%%%%%%%%%%%%%%%%%%%%%%%%%%%%%%%%%%%%%%%%%%%%%%%%%%%%%%%%%%%%%%%%%%%%%%%%%%%%%%%
\section{Introduction}

\LaTeX{} provides a mechanism to structure a large document (such as a book)
into a main file and several child files (containing the chapters)
using the |\include| command.
This mechanism is beneficial for documents
which span hundreds of pages in order to
make the source file(s) more manageable.
Moreover, compilation can be restricted to
selected child files by means of the |\includeonly| command.
The latter feature can be used to reduce the compilation time while editing
(this was significantly more useful in the earlier days of \LaTeX{})
or to generate a smaller document which is easier to navigate.
Another application of |\includeonly| is to generate
documents consisting of selected parts of the complete document.

However, there are a few drawbacks of the plain |\include| mechanism:
\begin{itemize}
\item
The child files cannot be compiled on their own,
they can only be compiled via the main file.
A naive editing environment
(such as a text editor with an option
to have the current file processed by \LaTeX)
may require one to switch to the main file before compiling;
attempting to compile the child file produces errors.
\item
The main file must be modified (each time)
to adjust the |\includeonly| command
to the present needs. This easily leaves the main file in a messy state.
\item
The generated document will always carry the filename
of the main document. This is inconvenient if
several child files are to be compiled and
to be kept for distribution.
\end{itemize}

The present package provides a simple interface
to make child files individually compilable by \LaTeX{}.
Compiling a child file then has the same effect as compiling
the main file with an |\includeonly| command
to select the appropriate child.
Moreover the generated document will carry the name of the child
rather than the main file.
This resolves all three above issues.

This feature is meant to make the editing of books,
thesis documents and lecture notes somewhat more convenient.
However, the package can also be used efficiently for
composing a series of documents (such as exercise sheets)
which are typically distributed individually.
It then assists the author in generating the individual documents
(potentially in different versions)
as well as a document containing the collected series.
Another application is in developing style files
or other kinds of included material
where compilation of the style file could redirect
to a sample or test file.

%%%%%%%%%%%%%%%%%%%%%%%%%%%%%%%%%%%%%%%%%%%%%%%%%%%%%%%%%%%%%%%%%%%%%%%%%%%%%%%%
%%%%%%%%%%%%%%%%%%%%%%%%%%%%%%%%%%%%%%%%%%%%%%%%%%%%%%%%%%%%%%%%%%%%%%%%%%%%%%%%
\section{Usage}

First of all, the package \textsf{childdoc} is \emph{not} a standard
\LaTeXe{} |.sty| style file! Therefore it needs to be invoked in
a non-standard way.

%%%%%%%%%%%%%%%%%%%%%%%%%%%%%%%%%%%%%%%%%%%%%%%%%%%%%%%%%%%%%%%%%%%%%%%%%%%%%%%%
\subsection{Included Files}
\label{sec:include}

%%%%%%%%%%%%%%%%%%%%%%%%%%%%%%%%%%%%%%%%
\DescribeMacro{\childdocmain}
To use the package, add the commands
\begin{center}
\begin{tabular}{l}
|\input{childdoc.def}|\\
|\childdocmain{}|\\
\end{tabular}
\end{center}
at the very top of the main \LaTeX{} file,
in particular \emph{before} the |\documentclass| statement!
The argument of |\childdocmain| should be left empty
(but it must be present).

%%%%%%%%%%%%%%%%%%%%%%%%%%%%%%%%%%%%%%%%
\DescribeMacro{\childdocof}
Furthermore, add the commands
\begin{center}
\begin{tabular}{l}
|\input{childdoc.def}|\\
|\childdocof{|\textit{main}|}|\\
\end{tabular}
\end{center}
at the top of every child file \textit{child}
which is included by |\include{|\textit{child}|}|
from within the main file
(or at least for those files to be compiled individually).
The argument \textit{main} must be the filename of the main file.

There are a couple of
considerations in setting up the main and child documents:

%%%%%%%%%%%%%%%%%%%%%%%%%%%%%%%%%%%%%%%%
\paragraph{Restrictions.}

Please note the following restrictions:
\begin{itemize}
\item
|\childdocmain| must be called with one argument \textit{main}
to ensure compatibility with earlier version of the package.
It must either be empty (|\childdocmain{}|)
or precisely match the filename of the main file in which it is specified.
See \secref{sec:detection} for further information.
\item
The filename \textit{main} must be specified without the |.tex| extension.
\item
The filename \textit{main} is case sensitive
(even in case-insensitive file systems)
due to internal string comparison.
\item
The argument \textit{main} should be fully expanded, it cannot be a macro.
\item
Subdirectories and special characters should be avoided in filenames.
\item
The command |\childdocmain{|\textit{main}|}| must be followed by a whitespace.
It should not be followed immediately by another command
or by a comment mark `|%|'.
This is because the \TeX{} parser reads the token immediately following
the argument of |\childdocmain| and puts it
at the beginning of every child section;
however, a white\-space is ignored.
\end{itemize}

%%%%%%%%%%%%%%%%%%%%%%%%%%%%%%%%%%%%%%%%
\paragraph{Content of Main File.}

It is advisable to place all content in the child files included by |\include|.
Any output contained in the main file will appear in all child documents
unless suppressed manually;
it cannot be suppressed automatically by the |\includeonly| directive
and thus should normally be avoided.
A method to include some content in the main file
by means of conditional processing is described in \secref{sec:conditional}.

%%%%%%%%%%%%%%%%%%%%%%%%%%%%%%%%%%%%%%%%
\paragraph{Page Numbering.}

When only a part of the document is compiled,
the appropriate numbering of pages
(as well as other status parameters)
is determined from the |.aux| files.
The latter contain information from previous passes.
However this information needs to propagate through
all intermediate child documents.
Therefore the page numbering in child documents may well
be inconsistent until the complete document is compiled at least once.

A useful (if unconventional) way to always ensure a consistent
page numbering is to restart the numbering in each child document
and denote the pages by `\textit{child}|.|\textit{page}'
where \textit{child} represents the chapter/section number of the child file.
This can be achieved by the command
|\numberwithin{page}{|\textit{child}|}|
of the \textsf{amsmath} package
where \textit{child} can be |chapter| or |section|
depending on the chosen structuring.
Alternatively, one can modify the macro |\thepage| appropriately
and reset the counter |page| at the start of each child file.

%%%%%%%%%%%%%%%%%%%%%%%%%%%%%%%%%%%%%%%%%%%%%%%%%%%%%%%%%%%%%%%%%%%%%%%%%%%%%%%%
\subsection{Conditional Processing}
\label{sec:conditional}

The package provides a mechanism to compile different versions
of a document. To customise the versions further some conditional processing
can come in handy to distinguish which version is being compiled.
The package provides two macros to describe the compilation context:

%%%%%%%%%%%%%%%%%%%%%%%%%%%%%%%%%%%%%%%%
\DescribeMacro{\ifchilddoc}
The conditional |\ifchilddoc| distinguishes between the compilation of
child documents and the main document:
%
\begin{center}
|\ifchilddoc |\textit{child-code}| |[|\||else |\textit{main-code}]| \||fi|
\end{center}

%%%%%%%%%%%%%%%%%%%%%%%%%%%%%%%%%%%%%%%%
\DescribeMacro{\childdocname}
\DescribeMacro{\childdocjob}
The macro |\childdocname| contains the filename (without extension)
of the main or child file being processed.
Note that |\childdocjob| will always contain the name of the main file.

%%%%%%%%%%%%%%%%%%%%%%%%%%%%%%%%%%%%%%%%
\paragraph{Title Page.}

Conditional processing can be used to include a title or banner page
in the main document when proper precautions are taken.
Importantly, the code in the main file should ensure that the page counter
(as well as other status parameters which are stored in the |.aux| files)
takes the same value after the conditional processing.
Otherwise the page numbers may take divergent values
depending on which part is compiled.

For example, a title page could be declared by:
%
\begin{center}
\begin{tabular}{l}
|\ifchilddoc\||else|\\
|\addtocounter{page}{-1}|\\
\textit{code for title page}\\
|\newpage|\\
|\||fi|
\end{tabular}
\end{center}
%
A banner page for the child documents can be generated by:
%
\begin{center}
\begin{tabular}{l}
|\ifchilddoc|\\
|\addtocounter{page}{-1}|\\
\textit{code for banner page}\\
|\newpage|\\
|\||fi|
\end{tabular}
\end{center}
%
Here one could write a message such as:
\begin{center}
|This is the part \childdocname{} of \childdocjob{}.|
\end{center}

%%%%%%%%%%%%%%%%%%%%%%%%%%%%%%%%%%%%%%%%%%%%%%%%%%%%%%%%%%%%%%%%%%%%%%%%%%%%%%%%
\subsection{Flags}
\label{sec:flags}

The package makes it easy to generate different versions
of the main or child documents.
To this end compilation flags can be defined
and assigned different default values.
They will be particularly useful in conjunction
with the forwarding mechanism described in \secref{sec:forward}.

For example, it may be useful to have a flag |\version|
which can be set to |draft| or |final|.
The document source will contain some conditional code
depending on the value of |\version|.
Suppose further, the flag should default to |final| for the main file
and to |draft| for child files
which is a natural assignment for editing the document.
This is achieved by placing the following code
in the preamble of the main document
(below the |\childdocmain| directive):
%
\begin{center}
\begin{tabular}{l}
|\ifchilddoc|\\
|\providecommand{\version}{draft}|\\
|\||else|\\
|\providecommand{\version}{final}|\\
|\||fi|
\end{tabular}
\end{center}
%
The definition by |\providecommand| makes sure
that previous definitions are not overwritten.
Further statements |\providecommand{\version}{...}|
can thus be added before the above code to override it.

For the main file, one might add a line
(between |\childdocmain| and the above block)
%
\begin{center}
|%\ifchilddoc\||else\providecommand{\version}{draft}\||fi|
\end{center}
%
which can be uncommented to produce a draft version.
Likewise one can add a line to the very top of a child file
(above the |\childdocof{|\textit{main}|}| directive)
%
\begin{center}
|%\providecommand{\version}{final}|
\end{center}
%
which can be uncommented to produce the final version of this child document.

%%%%%%%%%%%%%%%%%%%%%%%%%%%%%%%%%%%%%%%%%%%%%%%%%%%%%%%%%%%%%%%%%%%%%%%%%%%%%%%%
\subsection{Forwarding}
\label{sec:forward}

Different versions of the main or child documents
using compilation flags as described in \secref{sec:flags}
can be (permanently) stored in different files
for convenient compilation, viewing and distribution.
To this end, the package defines a command
to pass on compilation to a different file:

%%%%%%%%%%%%%%%%%%%%%%%%%%%%%%%%%%%%%%%%
\DescribeMacro{\childdocforward}
The command |\childdocforward| redirects processing to
another source file:
%
\begin{center}
\begin{tabular}{l}
|\input{childdoc.def}|\\
|\childdocforward[|\textit{main}|]{|\textit{dest}|}|\\
\end{tabular}
\end{center}
%
The argument \textit{dest} is the destination file
(without extension).
It should be the main file or one of the child files.
Note that further \textsf{childdoc} directives
such as |\childdocof| and |\childdocforward|
in the indicated file will be processed in this form.
The optional argument \textit{main}
passes on directly to the main file \textit{main}
while pretending to compile the child \textit{dest}.
This form behaves as if \textit{dest}
issues |\childdocof{|\textit{main}|}| right away,
and no further \textsf{childdoc} directives will be processed.

%%%%%%%%%%%%%%%%%%%%%%%%%%%%%%%%%%%%%%%%
\DescribeMacro{\...prefix}
In the alternative form |\childdocforwardprefix|,
%
\begin{center}
\begin{tabular}{l}
|\input{childdoc.def}|\\
|\childdocforwardprefix[|\textit{main}|]{|\textit{prefix}|}{|\textit{dest}|}|
\end{tabular}
\end{center}
%
the destination file is determined by a pattern
depending on the current file:
To make this work, the current file must be called
`{\textit{prefix}\hspace{0.2em}\textit{suffix}}'
with \textit{prefix} matching precisely the argument.
Processing is then passed on to the file
`{\textit{dest}\hspace{0.2em}\textit{suffix}}'.
Surely, the same effect is achieved by
directly specifying the
argument `{\textit{dest}\hspace{0.2em}\textit{suffix}}'
in the first form.
However, that requires to set up a different file
for each child. With the alternative form of the command
all these files can have exactly the same content
which simplifies setting them up and maintaining them.

For example, the following file |draft.tex|
with a compilation flag |\version| as described in \secref{sec:flags}
compiles the main document as a draft:
%
\begin{center}
\begin{tabular}{l}
|\def\version{draft}|\\
|\input{childdoc.def}|\\
|\childdocforward{|\textit{main}|}|
\end{tabular}
\end{center}
%
Likewise, the following files |final|\textit{nn}|.tex|
compile the final version of the child document
|child|\textit{nn}|.tex|:
%
\begin{center}
\begin{tabular}{l}
|\def\version{final}|\\
|\input{childdoc.def}|\\
|\childdocforwardprefix{final}{child}|
\end{tabular}
\end{center}
%

Note that when several versions of a main file and/or of each child file
are to be generated, it may be convenient to set up a |Makefile| or
shell script to automatise the process.

%%%%%%%%%%%%%%%%%%%%%%%%%%%%%%%%%%%%%%%%%%%%%%%%%%%%%%%%%%%%%%%%%%%%%%%%%%%%%%%%
\subsection{Command Line Processing}
\label{sec:commandline}

The effect of redirection files can also be achieved by invoking
the \LaTeX{} compiler with a more elaborate command line.
Most conveniently this should be done as part
of a shell script or a |Makefile|.

When using \textsf{childdoc} in the main file, the following
command lines effectively perform a redirection
(note that depending on the shell being used,
backslashes may have to be doubled: `|\|' $\to$ `|\\|'):
%
\begin{center}
|... -jobname "|\textit{target}|" |\\|"|[\textit{flags}]%
|\input{childdoc.def}\childdocforward[|\textit{main}|]{|\textit{dest}|}"|
\end{center}
%
Here \textit{target} is the name of the output file,
\textit{main} is the name of the main file
and \textit{dest} is the name of the main or child file to be processed
(all filenames without extensions).
The optional argument \textit{main} can be omitted
if \textit{main} matches \textit{dest}.
Optionally, compilation \textit{flags} can be defined via |\def| commands.
This command line makes the \TeX{} engine believe
it is compiling the file \textit{target}
whose content is specified as the latter parameter.
The provided code then forwards the processing to
\textit{main} or \textit{dest} as described in \secref{sec:forward}.

%%%%%%%%%%%%%%%%%%%%%%%%%%%%%%%%%%%%%%%%%%%%%%%%%%%%%%%%%%%%%%%%%%%%%%%%%%%%%%%%
\subsection{Include by Input}
\label{sec:input}

Including child documents by |\include| has some restrictions by design.
Most notably, the content of a child document always occupies
its own set of pages; pages cannot be shared between child documents.
Usually, this behaviour makes perfect sense
because each child document contain an essential part of the document.
However, in some situations it may be desirable to compose
a document from a collection of parts
without having mandatory page breaks between then.
For this case, the package
provides a mechanism to include parts
by |\input| which can also be processed individually.
However, by construction this mechanism
requires manual handling of the content to be output.

%%%%%%%%%%%%%%%%%%%%%%%%%%%%%%%%%%%%%%%%
\DescribeMacro{\ifchilddocmanual}
The main file should be prepared as usual, see \secref{sec:include}.
However, the document body must make a distinction
between processing of an individual part and of the main document, e.g.:
%
\begin{center}
\begin{tabular}{l}
|\ifchilddocmanual|\\
|\input{\childdocname}|\\
|\||else|\\
\textit{document body with }|\input{|\textit{part}|}|\\
|\||fi|
\end{tabular}
\end{center}
%
The conditional |\ifchilddocmanual| is true whenever
a part to be included by |\input| is being compiled,
and the name of the part is stored in |\childdocname|.

%%%%%%%%%%%%%%%%%%%%%%%%%%%%%%%%%%%%%%%%
\DescribeMacro{\childdocby}
Each part to be included by |\input| should start with:
%
\begin{center}
\begin{tabular}{l}
|\input{childdoc.def}|\\
|\childdocby{|\textit{main}|}|\\
\end{tabular}
\end{center}
%
The directive |\childdocby| is similar to |\childdocof|
described in \secref{sec:include},
but the subsequent selection of content must be done manually.
To that end, both |\ifchilddoc| and |\ifchilddocmanual|
will be true upon processing of a part,
and the name of the part is stored in |\childdocname|.
Note that |\jobname| will be set to the filename of the current part
so that each part receives an individual |.aux| file
that does not interfere with the |.aux| file(s) of the main document.
This behaviour can be altered by the alternative form
|\childdocby[*]{|\textit{main}|}| (with a non-empty optional argument)
which uses the |.aux| file of the main document
by setting |\jobname| to \textit{main}.

%%%%%%%%%%%%%%%%%%%%%%%%%%%%%%%%%%%%%%%%%%%%%%%%%%%%%%%%%%%%%%%%%%%%%%%%%%%%%%%%
\subsection{Driver Development}
\label{sec:driver}

The \textsf{childdoc} mechanism can also be use for the development
of definition files such as \LaTeX{} styles or classes.
This case differs from the above setup with multiple parts
included by |\include| in that no |\includeonly| should be invoked.
This can be achieved by starting the include file
(before |\ProvidesPackage|) with:
%
\begin{center}
\begin{tabular}{l}
|\input{childdoc.def}|\\
|\childdocforward{|\textit{main}|}|\\
\end{tabular}
\end{center}
%
or alternatively with:
%
\begin{center}
\begin{tabular}{l}
|\input{childdoc.def}|\\
|\childdocby{|\textit{main}|}|\\
\end{tabular}
\end{center}
%
Both forms have slightly different effects as described above.
The main file is prepared as usual, see \secref{sec:include}.

%%%%%%%%%%%%%%%%%%%%%%%%%%%%%%%%%%%%%%%%%%%%%%%%%%%%%%%%%%%%%%%%%%%%%%%%%%%%%%%%
\subsection{Legacy Detection}
\label{sec:detection}

The directive |\childdocmain| in the main file can detect
whether the complete document or merely a child is to be compiled
even without using the directive |\childdocof|.
This method is deprecated because it is less robust
and there is no compelling reason to use it;
it is merely provided for backward compatibility
and it may be removed in future versions.

If the detection mechanism is to be used,
it is mandatory to correctly specify
the filename of the main file as the argument of |\childdocmain|:
%
\begin{center}
\begin{tabular}{l}
|\input{childdoc.def}|\\
|\childdocmain{|\textit{main}|}|\\
\end{tabular}
\end{center}
%
If |\jobname| does not match the argument \textit{main} of |\childdocmain|,
it is assumed that |\jobname| points to the child file to be compiled.
When using |\childdocmain| with the main file specified as argument,
it suffices to start a child file
with just |\input{|\textit{main}|}|
without loading of the package and using |\childdocof|.
If instead all processing is done
with the appropriate \textsf{childdoc} directives,
the argument of \textit{main} of |\childdocmain| can be empty.

An alternative version of the command line processing described
in \secref{sec:commandline} using the detection mechanism reads:
%
\begin{center}
|... -jobname "|\textit{target}|" "|[\textit{flags}]%
[|\def\jobname{|\textit{dest}|}|]|\input{|\textit{main}|}"|
\end{center}

%%%%%%%%%%%%%%%%%%%%%%%%%%%%%%%%%%%%%%%%%%%%%%%%%%%%%%%%%%%%%%%%%%%%%%%%%%%%%%%%
\subsection{Manual Code}
\label{sec:manual}

In case one cannot be certain whether the definitions file |childdoc.def|
is installed on the target \TeX{} distribution
and one prefers not to ship it,
it is conceivable to paste a few relevant commands into the sources.

To that end, drop all statements |\input{childdoc.def}|
and perform the replacements as outlined below.
Instead of |\childdocmain{|\textit{main}|}| add the following code
to the top of the main file:
%
\begin{center}
\begin{tabular}{l}
|\||ifdefined\childdocname\endinput\||fi\newif\ifchilddoc|\\
|\edef\childdocname{\scantokens\expandafter{\jobname\noexpand}}|\\
|\def\childdocmain{|\textit{main}|}\||ifx\childdocmain\childdocname\||else|\\
|\childdoctrue\includeonly{\childdocname}\let\jobname\childdocmain\||fi|\\
\end{tabular}
\end{center}
%
Instead of |\childdocof{|\textit{main}|}| just include the main file
at the top of each child file:
%
\begin{center}
|\input{|\textit{main}|}|
\end{center}
%
A simple redirection |\childdocforward{|\textit{dest}|}| is achieved by:
%
\begin{center}
|\def\jobname{|\textit{dest}|}\input{\jobname}|
\end{center}
%
The redirection with prefix
|\childdocforwardprefix[|\textit{prefix}|]{|\textit{dest}|}|
is accomplished by:
%
\begin{center}
\begin{tabular}{l}
|{\edef\jobname{\scantokens\expandafter{\jobname\noexpand}}|\\
|\def\redirectjob |\textit{prefix}|#1~~~{\gdef\jobname{|\textit{dest}|#1}}|\\
|\expandafter\redirectjob\jobname~~~}\input{\jobname}|
\end{tabular}
\end{center}

In an alternative approach,
child documents can be compiled by a specific command line
without additional code or specific definitions:
%
\begin{center}
|... -jobname "|\textit{target}|" "|[\textit{flags}]%
|\includeonly{|\textit{dest}|}\input{|\textit{main}|}"|
\end{center}
%

%%%%%%%%%%%%%%%%%%%%%%%%%%%%%%%%%%%%%%%%%%%%%%%%%%%%%%%%%%%%%%%%%%%%%%%%%%%%%%%%
%%%%%%%%%%%%%%%%%%%%%%%%%%%%%%%%%%%%%%%%%%%%%%%%%%%%%%%%%%%%%%%%%%%%%%%%%%%%%%%%
\section{Information}

%%%%%%%%%%%%%%%%%%%%%%%%%%%%%%%%%%%%%%%%%%%%%%%%%%%%%%%%%%%%%%%%%%%%%%%%%%%%%%%%
\subsection{Copyright}

Copyright \copyright{} 2017--2018 Niklas Beisert

This work may be distributed and/or modified under the
conditions of the \LaTeX{} Project Public License, either version 1.3
of this license or (at your option) any later version.
The latest version of this license is in
  \url{http://www.latex-project.org/lppl.txt}
and version 1.3 or later is part of all distributions of \LaTeX{}
version 2005/12/01 or later.

This work has the LPPL maintenance status `maintained'.

The Current Maintainer of this work is Niklas Beisert.

This work consists of the files |README.txt|, |childdoc.ins| and |childdoc.dtx|
as well as the derived files |childdoc.def|, |cdocsamp.tex|
with |cdocsch1.tex|, |cdocsch2.tex|, |cdocspt3.tex|, |cdocspt4.tex|,
|cdocsdrf.tex|, |cdocsfn1.tex|, |cdocsfn2.tex|
as well as |childdoc.pdf|.

%%%%%%%%%%%%%%%%%%%%%%%%%%%%%%%%%%%%%%%%%%%%%%%%%%%%%%%%%%%%%%%%%%%%%%%%%%%%%%%%
\subsection{Files and Installation}

The package consists of the files:
%
\begin{center}
\begin{tabular}{ll}
    |README.txt|   & readme file \\
    |childdoc.ins| & installation file \\
    |childdoc.dtx| & source file \\
    |childdoc.def| & definition file \\
    |cdocsamp.tex| & sample main file \\
    |cdocsch1.tex| & sample include file \\
    |cdocsch2.tex| & sample include file \\
    |cdocspt3.tex| & sample part file \\
    |cdocspt4.tex| & sample part file \\
    |cdocsdrf.tex| & sample redirection file \\
    |cdocsfn1.tex| & sample redirection file \\
    |cdocsfn2.tex| & sample redirection file \\
    |childdoc.pdf| & manual
\end{tabular}
\end{center}
%
The distribution consists of the files
|README.txt|, |childdoc.ins| and |childdoc.dtx|.
%
\begin{itemize}
\item
Run (pdf)\LaTeX{} on |childdoc.dtx|
to compile the manual |childdoc.pdf| (this file).
\item
Run \LaTeX{} on |childdoc.ins| to create the definitions file |childdoc.def|
and the sample |cdocsamp.tex| with include files
|cdocsch1.tex|, |cdocsch2.tex|, |cdocspt3.tex|, |cdocspt4.tex|,
|cdocsdrf.tex|, |cdocsfn1.tex|, |cdocsfn2.tex|.
Then copy the file |childdoc.def| to an appropriate directory of your \LaTeX{}
distribution, e.g.\ \textit{texmf-root}|/tex/latex/childdoc|.
\end{itemize}

%%%%%%%%%%%%%%%%%%%%%%%%%%%%%%%%%%%%%%%%%%%%%%%%%%%%%%%%%%%%%%%%%%%%%%%%%%%%%%%%
\subsection{Related CTAN Packages}

There are several other packages which offer a similar functionality:
%
\begin{itemize}
\item
The packages
\href{http://ctan.org/pkg/docmute}{\textsf{docmute}},
\href{http://ctan.org/pkg/includex}{\textsf{includex}} and
\href{http://ctan.org/pkg/standalone}{\textsf{standalone}}
provide commands to include only the document body of
a child file thus allowing both files to be compiled individually.
\item
The packages \href{http://ctan.org/pkg/subdocs}{\textsf{subdocs}}
and \href{http://ctan.org/pkg/subfiles}{\textsf{subfiles}}
provide structures in which the main and child documents can be
encapsulated and allowing them to be compiled individually.
The inclusion mechanism is different from the conventional |\include|.
\item
The package \href{http://ctan.org/pkg/combine}{\textsf{combine}}
is an elaborate solution to combine several documents into one.
\end{itemize}
%
See also the CTAN topic \href{http://ctan.org/topic/subdocs}{\textsf{subdocs}}
for further related packages.
The present package differs from the above solutions in that
a document structure constructed with the conventional |\include| mechanism
just needs two extra commands at the top of every file
such that all constituent files can be compiled individually.

%%%%%%%%%%%%%%%%%%%%%%%%%%%%%%%%%%%%%%%%%%%%%%%%%%%%%%%%%%%%%%%%%%%%%%%%%%%%%%%%
%\subsection{Feature Suggestions}
%
%The following is a list of features which may be useful for future
%versions of this package:
%%
%\begin{itemize}
%\item
%\ldots
%\end{itemize}

%%%%%%%%%%%%%%%%%%%%%%%%%%%%%%%%%%%%%%%%%%%%%%%%%%%%%%%%%%%%%%%%%%%%%%%%%%%%%%%%
\subsection{Revision History}

%%%%%%%%%%%%%%%%%%%%%%%%%%%%%%%%%%%%%%%%
\paragraph{v2.0:} 2018/12/30

\begin{itemize}
\item
immediate forward processing
\item
added |\childdocby| mechanism
\item
manual restructured
\end{itemize}

%%%%%%%%%%%%%%%%%%%%%%%%%%%%%%%%%%%%%%%%
\paragraph{v1.6:} 2018/01/17

\begin{itemize}
\item
application for development of include files
\item
corrections to manual
\end{itemize}

%%%%%%%%%%%%%%%%%%%%%%%%%%%%%%%%%%%%%%%%
\paragraph{v1.5:} 2017/05/21

\begin{itemize}
\item
more complete structuring introduced
\item
|\childdocof| introduced
\item
|\childdoc| renamed to |\childdocmain|
\item
|\childredirect| renamed to |\childdocforward| and |\childdocforwardprefix|
and functionality expanded
\end{itemize}

%%%%%%%%%%%%%%%%%%%%%%%%%%%%%%%%%%%%%%%%
\paragraph{v1.0:} 2017/04/27

\begin{itemize}
\item
manual and install package
\item
first version published on CTAN
\end{itemize}

%%%%%%%%%%%%%%%%%%%%%%%%%%%%%%%%%%%%%%%%
\paragraph{v0.6:} 2017/04/26

\begin{itemize}
\item
redirection mechanism added
\end{itemize}

%%%%%%%%%%%%%%%%%%%%%%%%%%%%%%%%%%%%%%%%
\paragraph{v0.5:} 2017/04/26

\begin{itemize}
\item
functionality in definition file
\end{itemize}


%%%%%%%%%%%%%%%%%%%%%%%%%%%%%%%%%%%%%%%%%%%%%%%%%%%%%%%%%%%%%%%%%%%%%%%%%%%%%%%%
%%%%%%%%%%%%%%%%%%%%%%%%%%%%%%%%%%%%%%%%%%%%%%%%%%%%%%%%%%%%%%%%%%%%%%%%%%%%%%%%
%%%%%%%%%%%%%%%%%%%%%%%%%%%%%%%%%%%%%%%%%%%%%%%%%%%%%%%%%%%%%%%%%%%%%%%%%%%%%%%%
\appendix

\settowidth\MacroIndent{\rmfamily\scriptsize 000\ }

 \DocInput{childdoc.dtx}

\end{document}
%</driver>
% \fi
%
% %%%%%%%%%%%%%%%%%%%%%%%%%%%%%%%%%%%%%%%%%%%%%%%%%%%%%%%%%%%%%%%%%%%%%%%%%%%%%%
% %%%%%%%%%%%%%%%%%%%%%%%%%%%%%%%%%%%%%%%%%%%%%%%%%%%%%%%%%%%%%%%%%%%%%%%%%%%%%%
% \section{Sample}
%\iffalse
%<*samplemain>
%\fi
%
% The following presents a sample document
% with two chapters, two parts, a title page,
% a compile flag as well as three forwarding files to set the flag.
% It consists of eight |.tex| files:
% \begin{center}
% \begin{tabular}{ll}
% |cdocsamp.tex|&main file\\
% |cdocsch1.tex|&include file for chapter 1\\
% |cdocsch2.tex|&include file for chapter 2\\
% |cdocspt3.tex|&include file for part 3\\
% |cdocspt4.tex|&include file for part 4\\
% |cdocsdrf.tex|&forwarding file for main file in draft mode\\
% |cdocsfi1.tex|&forwarding file for final version of chapter 1\\
% |cdocsfi2.tex|&forwarding file for final version of chapter 2\\
% \end{tabular}
% \end{center}
% Each of the eight files can be compiled directly by the \LaTeX{} compiler.
%
% %%%%%%%%%%%%%%%%%%%%%%%%%%%%%%%%%%%%%%
% \paragraph{Main File.}
%
% The main file is called |cdocsamp.tex|.
%
% Load the \textsf{childdoc} definitions and
% declare the filename for the main document:
%    \begin{macrocode}
\input{childdoc.def}
\childdocmain{}
%    \end{macrocode}

% Optional override for |\version| flag:
%    \begin{macrocode}
%%\ifchilddoc\else\providecommand{\version}{draft}\fi
%    \end{macrocode}

% Define the default values for the |\version| flag
% (|final| for the main file and |draft| for childs):
%    \begin{macrocode}
\ifchilddoc
\providecommand{\version}{draft}
\else
\providecommand{\version}{final}
\fi
%    \end{macrocode}

% Load the standard document class:
%    \begin{macrocode}
\documentclass[12pt]{article}
%    \end{macrocode}

% Start the document body:
%    \begin{macrocode}
\begin{document}
%    \end{macrocode}

% Declare a title page.
% Print title, part of document being processed and version flag:
%    \begin{macrocode}
\addtocounter{page}{-1}
\begin{center}
{\LARGE\bfseries{}childdoc example\par}
\vspace{1cm}
\ifchilddoc
\ifchilddocmanual part\else chapter\fi:
`\childdocname' of `\childdocjob'\par
\else
main document: `\childdocjob'\par
\fi
version: \version\par
\end{center}
\newpage
%    \end{macrocode}

% Manually include selected file,
% otherwise process as usual:
%    \begin{macrocode}
\ifchilddocmanual
\section*{part `\childdocname'}
\input{\childdocname}
\else
%    \end{macrocode}

% Include the two chapters:
%    \begin{macrocode}
\include{cdocsch1}
\include{cdocsch2}
%    \end{macrocode}

% Include the two parts unless only chapters should be displayed:
%    \begin{macrocode}
\ifchilddoc\else
\section{part three}
\input{cdocspt3}
\section{part four}
\input{cdocspt4}
\fi
%    \end{macrocode}

% Process as usual until here:
%    \begin{macrocode}
\fi
%    \end{macrocode}

% End of document body:
%    \begin{macrocode}
\end{document}
%    \end{macrocode}
%\iffalse
%</samplemain>
%\fi
%
% %%%%%%%%%%%%%%%%%%%%%%%%%%%%%%%%%%%%%%
% \paragraph{Chapter Include Files.}
%
% The include files are called |cdocsch1.tex| and |cdocsch2.tex|.
%
%\iffalse
%<*samplechap1|samplechap2>
%\fi

% Optional override for |\version| flag:
%    \begin{macrocode}
%%\providecommand{\version}{final}
%    \end{macrocode}

% Include the main document:
%    \begin{macrocode}
\input{childdoc.def}
\childdocof{cdocsamp}
%    \end{macrocode}

%\iffalse
%</samplechap1|samplechap2>
%\fi
%
%\iffalse
%<*samplechap1>
%\fi
% Some text for chapter 1:
%    \begin{macrocode}
\section{one}
some text in chapter one
%    \end{macrocode}

%\iffalse
%</samplechap1>
%\fi
% Some text for chapter 2:
%\iffalse
%<*samplechap2>
%\fi
%    \begin{macrocode}
\section{two}
more text in chapter two
%    \end{macrocode}

%\iffalse
%</samplechap2>
%\fi
%
% %%%%%%%%%%%%%%%%%%%%%%%%%%%%%%%%%%%%%%
% \paragraph{Part Include Files.}
%
% The include files are called |cdocspt3.tex| and |cdocspt4.tex|.
%
%\iffalse
%<*samplepart3|samplepart4>
%\fi

% Optional override for |\version| flag:
%    \begin{macrocode}
%%\providecommand{\version}{final}
%    \end{macrocode}

% Include the main document:
%    \begin{macrocode}
\input{childdoc.def}
\childdocby{cdocsamp}
%    \end{macrocode}

%\iffalse
%</samplepart3|samplepart4>
%\fi
%
%\iffalse
%<*samplepart3>
%\fi
% Some text for part 3:
%    \begin{macrocode}
some text in part three
%    \end{macrocode}

%\iffalse
%</samplepart3>
%\fi
% Some text for part 4:
%\iffalse
%<*samplepart4>
%\fi
%    \begin{macrocode}
more text in part four
%    \end{macrocode}

%\iffalse
%</samplepart4>
%\fi
%
% %%%%%%%%%%%%%%%%%%%%%%%%%%%%%%%%%%%%%%
% \paragraph{Forwarding for a Complete Draft.}
%
% The following forwarding file |cdocsdrf.tex|
% compiles the main document in draft mode:
%\iffalse
%<*sampledraft>
%\fi
%    \begin{macrocode}
\def\version{draft}
\input{childdoc.def}
\childdocforward{cdocsamp}
%    \end{macrocode}

%\iffalse
%</sampledraft>
%\fi
%
% %%%%%%%%%%%%%%%%%%%%%%%%%%%%%%%%%%%%%%
% \paragraph{Forwarding for Final Version of the Chapters.}
%
% The following forwarding files |cdocsfn1.tex| and |cdocsfn2.tex|
% (with identical content)
% compile the final versions of the child documents
% |cdocsch1.tex| and |cdocsch2.tex|, respectively:
%\iffalse
%<*samplefinal>
%\fi
%    \begin{macrocode}
\def\version{final}
\input{childdoc.def}
\childdocforwardprefix[cdocsamp]{cdocsfn}{cdocsch}
%    \end{macrocode}

%\iffalse
%</samplefinal>
%\fi
%
% %%%%%%%%%%%%%%%%%%%%%%%%%%%%%%%%%%%%%%
% \paragraph{Command Line Processing.}
%
% The following three command lines generate the output files
% |cdocscld|, |cdocscl1| and |cdocscl2|
% which should be identical to
% |cdocsdrf|, |cdocsch1| and |cdocsfn2|, respectively:
% \begin{center}
% \begin{tabular}{l}
% |latex -jobname cdocscld \|\\
% |  "\def\version{draft}\input{childdoc.def}\childdocforward{cdocsamp}"|\\
% |latex -jobname cdocscl1 \|\\
% |  "\input{childdoc.def}\childdocforward[cdocsamp]{cdocsch1}"|\\
% |latex -jobname cdocscl2 \|\\
% |  "\def\version{final}\input{childdoc.def}\childdocforward{cdocsch2}"|
% \end{tabular}
% \end{center}
% Note that the trailing backslash on each first line
% merely continues the input to the second line
% (for convenient cut ant paste).
% Furthermore, the command |latex| can be replaced by any
% of its alternative versions such as |pdflatex|.
%
% %%%%%%%%%%%%%%%%%%%%%%%%%%%%%%%%%%%%%%%%%%%%%%%%%%%%%%%%%%%%%%%%%%%%%%%%%%%%%%
% %%%%%%%%%%%%%%%%%%%%%%%%%%%%%%%%%%%%%%%%%%%%%%%%%%%%%%%%%%%%%%%%%%%%%%%%%%%%%%
% \section{Implementation}
%\iffalse
%<*package>
%\fi
%
% This section describes the definitions file |childdoc.def|.

% The definitions cannot be loaded using |\usepackage| or |\RequirePackage|
% which has a mechanism to prevent loading a style file more than once.
% When loading the definitions by means of |\input|
% multiple instances have to be prevented manually:
%\iffalse
%This code needs to be before the `\ProvidesFile' directive
%which is defined at the beginning of this file.
%Therefore it is also placed there and commented out here.
%</package>
%<*discard>
%\fi
%    \begin{macrocode}
\ifdefined\childdocmain\endinput\fi
%    \end{macrocode}
%\iffalse
%</discard>
%<*package>
%\fi
%
% \macro{\ifchilddoc}
% \macro{\ifchilddocmanual}
% The conditional |\ifchilddoc| tells whether a
% child (true) or main (false) document is being compiled.
% The conditional |\ifchilddocmanual| tells whether
% the |\includeonly| mechanism is used (false) or
% the selection of child files must be performed manually (true).
% The definitions initialise to false:
%    \begin{macrocode}
\newif\ifchilddoc
\newif\ifchilddocmanual
%    \end{macrocode}

% \macro{\childdocname}
% \macro{\childdocjob}
% The macro |\childdocname| stores the name of the main document
% to be compiled. The macro |\childdocjob| stores the name of
% the document on which the \LaTeX{} compiler was originally invoked.
% The content of |\jobname| cannot be compared
% to filenames specified in the source due to different catcodes.
% The following code rescans |\jobname|, stores the result
% in |\childdocname| and saves a copy in |\childdocjob|:
%    \begin{macrocode}
\edef\childdocname{\scantokens\expandafter{\jobname\noexpand}}
\let\childdocjob\childdocname
%    \end{macrocode}

% \macro{\childdocdisable}
% The macro |\childdocdisable| prevents the main file
% from being processed more than once.
% At this stage, the main document command |\childdocmain|
% is assumed to be called once again where it should do nothing.
% Any subsequent call to it should prevent
% a secondary processing of the main document
% It overwrites the forwarding commands
% |\childdocof| and |\childdocforward|
% with empty macros to prevent further inclusions of the main document:
%    \begin{macrocode}
\newcommand{\childdocdisable}
{
  \renewcommand{\childdocmain}[1]{\renewcommand{\childdocmain}[1]{\endinput}}
  \renewcommand{\childdocof}[1]{}
  \renewcommand{\childdocby}[2][]{}
  \renewcommand{\childdocforward}[2][]{}
  \renewcommand{\childdocdisable}{}
}
%    \end{macrocode}

% \macro{\childdocmain}
% The macro |\childdocmain| is to be called at the top of the main file
% with nothing or the main filename (without extension) as argument.
% First, it breaks loops.
% If the argument is not empty and does not match |\childdocname|
% (which is set by the first inclusion of |childdoc.def|),
% |\ifchilddoc| is set to true, |\includeonly| is applied to the child file
% and |\jobname| is set to the main file
% (for proper handling of |.aux| files):
%    \begin{macrocode}
\newcommand{\childdocmain}[1]
{
  \childdocdisable\childdocmain{}
  \if?#1?\else
    \begingroup
      \def\childdoctmp{#1}
      \ifx\childdoctmp\childdocname
        \def\childdoctmp{}
      \else
        \def\childdoctmp
        {
          \childdoctrue
          \includeonly{\childdocname}
          \def\childdocjob{#1}
          \def\jobname{#1}
        }
      \fi
      \expandafter
    \endgroup
    \childdoctmp
  \fi
}
%    \end{macrocode}

% \macro{\childdocof}
% The command |\childdocof| redirects
% compilation to the main file |#1|.
%    \begin{macrocode}
\newcommand{\childdocof}[1]
{
  \childdocdisable
  \childdoctrue
  \includeonly{\childdocname}
  \def\jobname{#1}
  \def\childdocjob{#1}
  \input{#1}
}
%    \end{macrocode}

% \macro{\childdocby}
% The command |\childdocby| ....
%    \begin{macrocode}
\newcommand{\childdocby}[2][]
{
  \childdocdisable
  \childdoctrue
  \childdocmanualtrue
  \if?#1?\else
    \def\jobname{#2}
  \fi
  \def\childdocjob{#2}
  \input{#2}
  \endinput
}
%    \end{macrocode}

% \macro{\childdocforward}
% The command |\childdocforward| redirects
% compilation to the main file or
% (if the optional argument is given) a child file.
% Parameters are set as if the main file
% or a child file starting with |\childdocof| was compiled.
% Then compilation is handed over to the main file:
%    \begin{macrocode}
\newcommand{\childdocforward}[2][]
{
  \begingroup
    \if?#1?
      \def\childdoctmp
      {
        \def\childdocname{#2}
        \def\childdocjob{#2}
        \def\jobname{#2}
        \input{#2}
        \endinput
      }
    \else
      \def\childdoctmp
      {
        \childdocdisable
        \def\childdocname{#2}
        \childdoctrue
        \includeonly{#2}
        \def\childdocjob{#1}
        \def\jobname{#1}
        \input{#1}
        \endinput
      }
    \fi
    \expandafter
  \endgroup
  \childdoctmp
}
%    \end{macrocode}

% \macro{\childdocforwardprefix}
% The command |\childdocforwardprefix| redirects
% compilation to the main or a child file by means of a pattern.
% The prefix |#1| in the current filename is replaced by |#2|
% and the suffix of the current filename is kept
% (it is assumed that the filename does not contain the substring `|~~~|'
% which is used as a delimiter).
% Compilation is handed over to the new file by |\childdocforward|:
%    \begin{macrocode}
\newcommand{\childdocforwardprefix}[3][]
{
  \begingroup
    \def\childdocextract #2##1~~~{\def\childdoctmp{\childdocforward[#1]{#3##1}}}
    \expandafter\childdocextract\childdocname~~~
    \expandafter
  \endgroup
  \childdoctmp
}
%    \end{macrocode}

% \macro{\childdoc}
% The deprecated macro |\childdoc| is a legacy version of |\childdocmain|:
%    \begin{macrocode}
\newcommand{\childdoc}{\childdocmain}
%    \end{macrocode}

% \macro{\childdocredirect}
% The deprecated macro |\childdocredirect| is a legacy version
% of |\childdocforward| and |\childdocforwardprefix|:
%    \begin{macrocode}
\newcommand{\childdocredirect}[2][]
{
  \begingroup
    \if?#1?
      \def\childdoctmp{\childdocforward{#2}}
    \else
      \def\childdoctmp{\childdocforwardprefix{#1}{#2}}
    \fi
    \expandafter
  \endgroup
  \childdoctmp
}
%    \end{macrocode}

%\iffalse
%</package>
%\fi
%
\endinput
\childdocforward{cdocsamp}"|\\
% |latex -jobname cdocscl1 \|\\
% |  "% \iffalse
%
% childdoc.dtx Copyright (C) 2017-2018 Niklas Beisert
%
% This work may be distributed and/or modified under the
% conditions of the LaTeX Project Public License, either version 1.3
% of this license or (at your option) any later version.
% The latest version of this license is in
%   http://www.latex-project.org/lppl.txt
% and version 1.3 or later is part of all distributions of LaTeX
% version 2005/12/01 or later.
%
% This work has the LPPL maintenance status `maintained'.
%
% The Current Maintainer of this work is Niklas Beisert.
%
% This work consists of the files childdoc.dtx and childdoc.ins
% and the derived files childdoc.def and cdocsamp.tex with
% cdocsch1.tex, cdocsch2.tex, cdocsdrf.tex, cdocsfn1.tex, cdocsfn2.tex.
%
%<package>\ifdefined\childdocmain\endinput\fi
%<package>\ProvidesFile{childdoc.def}[2018/12/30 v2.0 child document driver]
%<samplemain>\ProvidesFile{cdocsamp.tex}[2018/12/30 v2.0 sample for childdoc]
%<*driver>
%\ProvidesFile{childdoc.drv}[2018/12/30 v2.0 childdoc reference manual file]
\PassOptionsToClass{10pt,a4paper}{article}
\documentclass{ltxdoc}

\usepackage[margin=35mm]{geometry}
\usepackage{hyperref}
\usepackage{hyperxmp}
\usepackage[usenames]{color}

\hypersetup{colorlinks=true}
\hypersetup{pdfstartview=FitH}
\hypersetup{pdfpagemode=UseNone}
\hypersetup{pdfsource={}}
\hypersetup{pdflang={en-UK}}
\hypersetup{pdfcopyright={Copyright 2017-2018 Niklas Beisert.
  This work may be distributed and/or modified under the
  conditions of the LaTeX Project Public License, either version 1.3
  of this license or (at your option) any later version.}}
\hypersetup{pdflicenseurl={http://www.latex-project.org/lppl.txt}}
\hypersetup{pdfcontactaddress={ETH Zurich, ITP, HIT K,
  Wolfgang-Pauli-Strasse 27}}
\hypersetup{pdfcontactpostcode={8093}}
\hypersetup{pdfcontactcity={Zurich}}
\hypersetup{pdfcontactcountry={Switzerland}}
\hypersetup{pdfcontactemail={nbeisert@itp.phys.ethz.ch}}
\hypersetup{pdfcontacturl={http://people.phys.ethz.ch/\xmptilde nbeisert/}}

\newcommand{\secref}[1]{\hyperref[#1]{section \ref*{#1}}}

\parskip1ex
\parindent0pt
\let\olditemize\itemize
\def\itemize{\olditemize\parskip0pt}

\begin{document}

\title{The \textsf{childdoc} Package}
\hypersetup{pdftitle={The childdoc Package}}
\author{Niklas Beisert\\[2ex]
  Institut f\"ur Theoretische Physik\\
  Eidgen\"ossische Technische Hochschule Z\"urich\\
  Wolfgang-Pauli-Strasse 27, 8093 Z\"urich, Switzerland\\[1ex]
  \href{mailto:nbeisert@itp.phys.ethz.ch}
  {\texttt{nbeisert@itp.phys.ethz.ch}}}
\hypersetup{pdfauthor={Niklas Beisert}}
\hypersetup{pdfsubject={Manual for the LaTeX2e Package childdoc}}
\date{30 December 2018, \textsf{v2.0}}
\maketitle

\begin{abstract}\noindent
\textsf{childdoc} is a \LaTeXe{} package
that enables the direct compilation
of document sections included by |\include|
to individual files.
\end{abstract}

\begingroup
\parskip0ex
\tableofcontents
\endgroup

%%%%%%%%%%%%%%%%%%%%%%%%%%%%%%%%%%%%%%%%%%%%%%%%%%%%%%%%%%%%%%%%%%%%%%%%%%%%%%%%
%%%%%%%%%%%%%%%%%%%%%%%%%%%%%%%%%%%%%%%%%%%%%%%%%%%%%%%%%%%%%%%%%%%%%%%%%%%%%%%%
\section{Introduction}

\LaTeX{} provides a mechanism to structure a large document (such as a book)
into a main file and several child files (containing the chapters)
using the |\include| command.
This mechanism is beneficial for documents
which span hundreds of pages in order to
make the source file(s) more manageable.
Moreover, compilation can be restricted to
selected child files by means of the |\includeonly| command.
The latter feature can be used to reduce the compilation time while editing
(this was significantly more useful in the earlier days of \LaTeX{})
or to generate a smaller document which is easier to navigate.
Another application of |\includeonly| is to generate
documents consisting of selected parts of the complete document.

However, there are a few drawbacks of the plain |\include| mechanism:
\begin{itemize}
\item
The child files cannot be compiled on their own,
they can only be compiled via the main file.
A naive editing environment
(such as a text editor with an option
to have the current file processed by \LaTeX)
may require one to switch to the main file before compiling;
attempting to compile the child file produces errors.
\item
The main file must be modified (each time)
to adjust the |\includeonly| command
to the present needs. This easily leaves the main file in a messy state.
\item
The generated document will always carry the filename
of the main document. This is inconvenient if
several child files are to be compiled and
to be kept for distribution.
\end{itemize}

The present package provides a simple interface
to make child files individually compilable by \LaTeX{}.
Compiling a child file then has the same effect as compiling
the main file with an |\includeonly| command
to select the appropriate child.
Moreover the generated document will carry the name of the child
rather than the main file.
This resolves all three above issues.

This feature is meant to make the editing of books,
thesis documents and lecture notes somewhat more convenient.
However, the package can also be used efficiently for
composing a series of documents (such as exercise sheets)
which are typically distributed individually.
It then assists the author in generating the individual documents
(potentially in different versions)
as well as a document containing the collected series.
Another application is in developing style files
or other kinds of included material
where compilation of the style file could redirect
to a sample or test file.

%%%%%%%%%%%%%%%%%%%%%%%%%%%%%%%%%%%%%%%%%%%%%%%%%%%%%%%%%%%%%%%%%%%%%%%%%%%%%%%%
%%%%%%%%%%%%%%%%%%%%%%%%%%%%%%%%%%%%%%%%%%%%%%%%%%%%%%%%%%%%%%%%%%%%%%%%%%%%%%%%
\section{Usage}

First of all, the package \textsf{childdoc} is \emph{not} a standard
\LaTeXe{} |.sty| style file! Therefore it needs to be invoked in
a non-standard way.

%%%%%%%%%%%%%%%%%%%%%%%%%%%%%%%%%%%%%%%%%%%%%%%%%%%%%%%%%%%%%%%%%%%%%%%%%%%%%%%%
\subsection{Included Files}
\label{sec:include}

%%%%%%%%%%%%%%%%%%%%%%%%%%%%%%%%%%%%%%%%
\DescribeMacro{\childdocmain}
To use the package, add the commands
\begin{center}
\begin{tabular}{l}
|\input{childdoc.def}|\\
|\childdocmain{}|\\
\end{tabular}
\end{center}
at the very top of the main \LaTeX{} file,
in particular \emph{before} the |\documentclass| statement!
The argument of |\childdocmain| should be left empty
(but it must be present).

%%%%%%%%%%%%%%%%%%%%%%%%%%%%%%%%%%%%%%%%
\DescribeMacro{\childdocof}
Furthermore, add the commands
\begin{center}
\begin{tabular}{l}
|\input{childdoc.def}|\\
|\childdocof{|\textit{main}|}|\\
\end{tabular}
\end{center}
at the top of every child file \textit{child}
which is included by |\include{|\textit{child}|}|
from within the main file
(or at least for those files to be compiled individually).
The argument \textit{main} must be the filename of the main file.

There are a couple of
considerations in setting up the main and child documents:

%%%%%%%%%%%%%%%%%%%%%%%%%%%%%%%%%%%%%%%%
\paragraph{Restrictions.}

Please note the following restrictions:
\begin{itemize}
\item
|\childdocmain| must be called with one argument \textit{main}
to ensure compatibility with earlier version of the package.
It must either be empty (|\childdocmain{}|)
or precisely match the filename of the main file in which it is specified.
See \secref{sec:detection} for further information.
\item
The filename \textit{main} must be specified without the |.tex| extension.
\item
The filename \textit{main} is case sensitive
(even in case-insensitive file systems)
due to internal string comparison.
\item
The argument \textit{main} should be fully expanded, it cannot be a macro.
\item
Subdirectories and special characters should be avoided in filenames.
\item
The command |\childdocmain{|\textit{main}|}| must be followed by a whitespace.
It should not be followed immediately by another command
or by a comment mark `|%|'.
This is because the \TeX{} parser reads the token immediately following
the argument of |\childdocmain| and puts it
at the beginning of every child section;
however, a white\-space is ignored.
\end{itemize}

%%%%%%%%%%%%%%%%%%%%%%%%%%%%%%%%%%%%%%%%
\paragraph{Content of Main File.}

It is advisable to place all content in the child files included by |\include|.
Any output contained in the main file will appear in all child documents
unless suppressed manually;
it cannot be suppressed automatically by the |\includeonly| directive
and thus should normally be avoided.
A method to include some content in the main file
by means of conditional processing is described in \secref{sec:conditional}.

%%%%%%%%%%%%%%%%%%%%%%%%%%%%%%%%%%%%%%%%
\paragraph{Page Numbering.}

When only a part of the document is compiled,
the appropriate numbering of pages
(as well as other status parameters)
is determined from the |.aux| files.
The latter contain information from previous passes.
However this information needs to propagate through
all intermediate child documents.
Therefore the page numbering in child documents may well
be inconsistent until the complete document is compiled at least once.

A useful (if unconventional) way to always ensure a consistent
page numbering is to restart the numbering in each child document
and denote the pages by `\textit{child}|.|\textit{page}'
where \textit{child} represents the chapter/section number of the child file.
This can be achieved by the command
|\numberwithin{page}{|\textit{child}|}|
of the \textsf{amsmath} package
where \textit{child} can be |chapter| or |section|
depending on the chosen structuring.
Alternatively, one can modify the macro |\thepage| appropriately
and reset the counter |page| at the start of each child file.

%%%%%%%%%%%%%%%%%%%%%%%%%%%%%%%%%%%%%%%%%%%%%%%%%%%%%%%%%%%%%%%%%%%%%%%%%%%%%%%%
\subsection{Conditional Processing}
\label{sec:conditional}

The package provides a mechanism to compile different versions
of a document. To customise the versions further some conditional processing
can come in handy to distinguish which version is being compiled.
The package provides two macros to describe the compilation context:

%%%%%%%%%%%%%%%%%%%%%%%%%%%%%%%%%%%%%%%%
\DescribeMacro{\ifchilddoc}
The conditional |\ifchilddoc| distinguishes between the compilation of
child documents and the main document:
%
\begin{center}
|\ifchilddoc |\textit{child-code}| |[|\||else |\textit{main-code}]| \||fi|
\end{center}

%%%%%%%%%%%%%%%%%%%%%%%%%%%%%%%%%%%%%%%%
\DescribeMacro{\childdocname}
\DescribeMacro{\childdocjob}
The macro |\childdocname| contains the filename (without extension)
of the main or child file being processed.
Note that |\childdocjob| will always contain the name of the main file.

%%%%%%%%%%%%%%%%%%%%%%%%%%%%%%%%%%%%%%%%
\paragraph{Title Page.}

Conditional processing can be used to include a title or banner page
in the main document when proper precautions are taken.
Importantly, the code in the main file should ensure that the page counter
(as well as other status parameters which are stored in the |.aux| files)
takes the same value after the conditional processing.
Otherwise the page numbers may take divergent values
depending on which part is compiled.

For example, a title page could be declared by:
%
\begin{center}
\begin{tabular}{l}
|\ifchilddoc\||else|\\
|\addtocounter{page}{-1}|\\
\textit{code for title page}\\
|\newpage|\\
|\||fi|
\end{tabular}
\end{center}
%
A banner page for the child documents can be generated by:
%
\begin{center}
\begin{tabular}{l}
|\ifchilddoc|\\
|\addtocounter{page}{-1}|\\
\textit{code for banner page}\\
|\newpage|\\
|\||fi|
\end{tabular}
\end{center}
%
Here one could write a message such as:
\begin{center}
|This is the part \childdocname{} of \childdocjob{}.|
\end{center}

%%%%%%%%%%%%%%%%%%%%%%%%%%%%%%%%%%%%%%%%%%%%%%%%%%%%%%%%%%%%%%%%%%%%%%%%%%%%%%%%
\subsection{Flags}
\label{sec:flags}

The package makes it easy to generate different versions
of the main or child documents.
To this end compilation flags can be defined
and assigned different default values.
They will be particularly useful in conjunction
with the forwarding mechanism described in \secref{sec:forward}.

For example, it may be useful to have a flag |\version|
which can be set to |draft| or |final|.
The document source will contain some conditional code
depending on the value of |\version|.
Suppose further, the flag should default to |final| for the main file
and to |draft| for child files
which is a natural assignment for editing the document.
This is achieved by placing the following code
in the preamble of the main document
(below the |\childdocmain| directive):
%
\begin{center}
\begin{tabular}{l}
|\ifchilddoc|\\
|\providecommand{\version}{draft}|\\
|\||else|\\
|\providecommand{\version}{final}|\\
|\||fi|
\end{tabular}
\end{center}
%
The definition by |\providecommand| makes sure
that previous definitions are not overwritten.
Further statements |\providecommand{\version}{...}|
can thus be added before the above code to override it.

For the main file, one might add a line
(between |\childdocmain| and the above block)
%
\begin{center}
|%\ifchilddoc\||else\providecommand{\version}{draft}\||fi|
\end{center}
%
which can be uncommented to produce a draft version.
Likewise one can add a line to the very top of a child file
(above the |\childdocof{|\textit{main}|}| directive)
%
\begin{center}
|%\providecommand{\version}{final}|
\end{center}
%
which can be uncommented to produce the final version of this child document.

%%%%%%%%%%%%%%%%%%%%%%%%%%%%%%%%%%%%%%%%%%%%%%%%%%%%%%%%%%%%%%%%%%%%%%%%%%%%%%%%
\subsection{Forwarding}
\label{sec:forward}

Different versions of the main or child documents
using compilation flags as described in \secref{sec:flags}
can be (permanently) stored in different files
for convenient compilation, viewing and distribution.
To this end, the package defines a command
to pass on compilation to a different file:

%%%%%%%%%%%%%%%%%%%%%%%%%%%%%%%%%%%%%%%%
\DescribeMacro{\childdocforward}
The command |\childdocforward| redirects processing to
another source file:
%
\begin{center}
\begin{tabular}{l}
|\input{childdoc.def}|\\
|\childdocforward[|\textit{main}|]{|\textit{dest}|}|\\
\end{tabular}
\end{center}
%
The argument \textit{dest} is the destination file
(without extension).
It should be the main file or one of the child files.
Note that further \textsf{childdoc} directives
such as |\childdocof| and |\childdocforward|
in the indicated file will be processed in this form.
The optional argument \textit{main}
passes on directly to the main file \textit{main}
while pretending to compile the child \textit{dest}.
This form behaves as if \textit{dest}
issues |\childdocof{|\textit{main}|}| right away,
and no further \textsf{childdoc} directives will be processed.

%%%%%%%%%%%%%%%%%%%%%%%%%%%%%%%%%%%%%%%%
\DescribeMacro{\...prefix}
In the alternative form |\childdocforwardprefix|,
%
\begin{center}
\begin{tabular}{l}
|\input{childdoc.def}|\\
|\childdocforwardprefix[|\textit{main}|]{|\textit{prefix}|}{|\textit{dest}|}|
\end{tabular}
\end{center}
%
the destination file is determined by a pattern
depending on the current file:
To make this work, the current file must be called
`{\textit{prefix}\hspace{0.2em}\textit{suffix}}'
with \textit{prefix} matching precisely the argument.
Processing is then passed on to the file
`{\textit{dest}\hspace{0.2em}\textit{suffix}}'.
Surely, the same effect is achieved by
directly specifying the
argument `{\textit{dest}\hspace{0.2em}\textit{suffix}}'
in the first form.
However, that requires to set up a different file
for each child. With the alternative form of the command
all these files can have exactly the same content
which simplifies setting them up and maintaining them.

For example, the following file |draft.tex|
with a compilation flag |\version| as described in \secref{sec:flags}
compiles the main document as a draft:
%
\begin{center}
\begin{tabular}{l}
|\def\version{draft}|\\
|\input{childdoc.def}|\\
|\childdocforward{|\textit{main}|}|
\end{tabular}
\end{center}
%
Likewise, the following files |final|\textit{nn}|.tex|
compile the final version of the child document
|child|\textit{nn}|.tex|:
%
\begin{center}
\begin{tabular}{l}
|\def\version{final}|\\
|\input{childdoc.def}|\\
|\childdocforwardprefix{final}{child}|
\end{tabular}
\end{center}
%

Note that when several versions of a main file and/or of each child file
are to be generated, it may be convenient to set up a |Makefile| or
shell script to automatise the process.

%%%%%%%%%%%%%%%%%%%%%%%%%%%%%%%%%%%%%%%%%%%%%%%%%%%%%%%%%%%%%%%%%%%%%%%%%%%%%%%%
\subsection{Command Line Processing}
\label{sec:commandline}

The effect of redirection files can also be achieved by invoking
the \LaTeX{} compiler with a more elaborate command line.
Most conveniently this should be done as part
of a shell script or a |Makefile|.

When using \textsf{childdoc} in the main file, the following
command lines effectively perform a redirection
(note that depending on the shell being used,
backslashes may have to be doubled: `|\|' $\to$ `|\\|'):
%
\begin{center}
|... -jobname "|\textit{target}|" |\\|"|[\textit{flags}]%
|\input{childdoc.def}\childdocforward[|\textit{main}|]{|\textit{dest}|}"|
\end{center}
%
Here \textit{target} is the name of the output file,
\textit{main} is the name of the main file
and \textit{dest} is the name of the main or child file to be processed
(all filenames without extensions).
The optional argument \textit{main} can be omitted
if \textit{main} matches \textit{dest}.
Optionally, compilation \textit{flags} can be defined via |\def| commands.
This command line makes the \TeX{} engine believe
it is compiling the file \textit{target}
whose content is specified as the latter parameter.
The provided code then forwards the processing to
\textit{main} or \textit{dest} as described in \secref{sec:forward}.

%%%%%%%%%%%%%%%%%%%%%%%%%%%%%%%%%%%%%%%%%%%%%%%%%%%%%%%%%%%%%%%%%%%%%%%%%%%%%%%%
\subsection{Include by Input}
\label{sec:input}

Including child documents by |\include| has some restrictions by design.
Most notably, the content of a child document always occupies
its own set of pages; pages cannot be shared between child documents.
Usually, this behaviour makes perfect sense
because each child document contain an essential part of the document.
However, in some situations it may be desirable to compose
a document from a collection of parts
without having mandatory page breaks between then.
For this case, the package
provides a mechanism to include parts
by |\input| which can also be processed individually.
However, by construction this mechanism
requires manual handling of the content to be output.

%%%%%%%%%%%%%%%%%%%%%%%%%%%%%%%%%%%%%%%%
\DescribeMacro{\ifchilddocmanual}
The main file should be prepared as usual, see \secref{sec:include}.
However, the document body must make a distinction
between processing of an individual part and of the main document, e.g.:
%
\begin{center}
\begin{tabular}{l}
|\ifchilddocmanual|\\
|\input{\childdocname}|\\
|\||else|\\
\textit{document body with }|\input{|\textit{part}|}|\\
|\||fi|
\end{tabular}
\end{center}
%
The conditional |\ifchilddocmanual| is true whenever
a part to be included by |\input| is being compiled,
and the name of the part is stored in |\childdocname|.

%%%%%%%%%%%%%%%%%%%%%%%%%%%%%%%%%%%%%%%%
\DescribeMacro{\childdocby}
Each part to be included by |\input| should start with:
%
\begin{center}
\begin{tabular}{l}
|\input{childdoc.def}|\\
|\childdocby{|\textit{main}|}|\\
\end{tabular}
\end{center}
%
The directive |\childdocby| is similar to |\childdocof|
described in \secref{sec:include},
but the subsequent selection of content must be done manually.
To that end, both |\ifchilddoc| and |\ifchilddocmanual|
will be true upon processing of a part,
and the name of the part is stored in |\childdocname|.
Note that |\jobname| will be set to the filename of the current part
so that each part receives an individual |.aux| file
that does not interfere with the |.aux| file(s) of the main document.
This behaviour can be altered by the alternative form
|\childdocby[*]{|\textit{main}|}| (with a non-empty optional argument)
which uses the |.aux| file of the main document
by setting |\jobname| to \textit{main}.

%%%%%%%%%%%%%%%%%%%%%%%%%%%%%%%%%%%%%%%%%%%%%%%%%%%%%%%%%%%%%%%%%%%%%%%%%%%%%%%%
\subsection{Driver Development}
\label{sec:driver}

The \textsf{childdoc} mechanism can also be use for the development
of definition files such as \LaTeX{} styles or classes.
This case differs from the above setup with multiple parts
included by |\include| in that no |\includeonly| should be invoked.
This can be achieved by starting the include file
(before |\ProvidesPackage|) with:
%
\begin{center}
\begin{tabular}{l}
|\input{childdoc.def}|\\
|\childdocforward{|\textit{main}|}|\\
\end{tabular}
\end{center}
%
or alternatively with:
%
\begin{center}
\begin{tabular}{l}
|\input{childdoc.def}|\\
|\childdocby{|\textit{main}|}|\\
\end{tabular}
\end{center}
%
Both forms have slightly different effects as described above.
The main file is prepared as usual, see \secref{sec:include}.

%%%%%%%%%%%%%%%%%%%%%%%%%%%%%%%%%%%%%%%%%%%%%%%%%%%%%%%%%%%%%%%%%%%%%%%%%%%%%%%%
\subsection{Legacy Detection}
\label{sec:detection}

The directive |\childdocmain| in the main file can detect
whether the complete document or merely a child is to be compiled
even without using the directive |\childdocof|.
This method is deprecated because it is less robust
and there is no compelling reason to use it;
it is merely provided for backward compatibility
and it may be removed in future versions.

If the detection mechanism is to be used,
it is mandatory to correctly specify
the filename of the main file as the argument of |\childdocmain|:
%
\begin{center}
\begin{tabular}{l}
|\input{childdoc.def}|\\
|\childdocmain{|\textit{main}|}|\\
\end{tabular}
\end{center}
%
If |\jobname| does not match the argument \textit{main} of |\childdocmain|,
it is assumed that |\jobname| points to the child file to be compiled.
When using |\childdocmain| with the main file specified as argument,
it suffices to start a child file
with just |\input{|\textit{main}|}|
without loading of the package and using |\childdocof|.
If instead all processing is done
with the appropriate \textsf{childdoc} directives,
the argument of \textit{main} of |\childdocmain| can be empty.

An alternative version of the command line processing described
in \secref{sec:commandline} using the detection mechanism reads:
%
\begin{center}
|... -jobname "|\textit{target}|" "|[\textit{flags}]%
[|\def\jobname{|\textit{dest}|}|]|\input{|\textit{main}|}"|
\end{center}

%%%%%%%%%%%%%%%%%%%%%%%%%%%%%%%%%%%%%%%%%%%%%%%%%%%%%%%%%%%%%%%%%%%%%%%%%%%%%%%%
\subsection{Manual Code}
\label{sec:manual}

In case one cannot be certain whether the definitions file |childdoc.def|
is installed on the target \TeX{} distribution
and one prefers not to ship it,
it is conceivable to paste a few relevant commands into the sources.

To that end, drop all statements |\input{childdoc.def}|
and perform the replacements as outlined below.
Instead of |\childdocmain{|\textit{main}|}| add the following code
to the top of the main file:
%
\begin{center}
\begin{tabular}{l}
|\||ifdefined\childdocname\endinput\||fi\newif\ifchilddoc|\\
|\edef\childdocname{\scantokens\expandafter{\jobname\noexpand}}|\\
|\def\childdocmain{|\textit{main}|}\||ifx\childdocmain\childdocname\||else|\\
|\childdoctrue\includeonly{\childdocname}\let\jobname\childdocmain\||fi|\\
\end{tabular}
\end{center}
%
Instead of |\childdocof{|\textit{main}|}| just include the main file
at the top of each child file:
%
\begin{center}
|\input{|\textit{main}|}|
\end{center}
%
A simple redirection |\childdocforward{|\textit{dest}|}| is achieved by:
%
\begin{center}
|\def\jobname{|\textit{dest}|}\input{\jobname}|
\end{center}
%
The redirection with prefix
|\childdocforwardprefix[|\textit{prefix}|]{|\textit{dest}|}|
is accomplished by:
%
\begin{center}
\begin{tabular}{l}
|{\edef\jobname{\scantokens\expandafter{\jobname\noexpand}}|\\
|\def\redirectjob |\textit{prefix}|#1~~~{\gdef\jobname{|\textit{dest}|#1}}|\\
|\expandafter\redirectjob\jobname~~~}\input{\jobname}|
\end{tabular}
\end{center}

In an alternative approach,
child documents can be compiled by a specific command line
without additional code or specific definitions:
%
\begin{center}
|... -jobname "|\textit{target}|" "|[\textit{flags}]%
|\includeonly{|\textit{dest}|}\input{|\textit{main}|}"|
\end{center}
%

%%%%%%%%%%%%%%%%%%%%%%%%%%%%%%%%%%%%%%%%%%%%%%%%%%%%%%%%%%%%%%%%%%%%%%%%%%%%%%%%
%%%%%%%%%%%%%%%%%%%%%%%%%%%%%%%%%%%%%%%%%%%%%%%%%%%%%%%%%%%%%%%%%%%%%%%%%%%%%%%%
\section{Information}

%%%%%%%%%%%%%%%%%%%%%%%%%%%%%%%%%%%%%%%%%%%%%%%%%%%%%%%%%%%%%%%%%%%%%%%%%%%%%%%%
\subsection{Copyright}

Copyright \copyright{} 2017--2018 Niklas Beisert

This work may be distributed and/or modified under the
conditions of the \LaTeX{} Project Public License, either version 1.3
of this license or (at your option) any later version.
The latest version of this license is in
  \url{http://www.latex-project.org/lppl.txt}
and version 1.3 or later is part of all distributions of \LaTeX{}
version 2005/12/01 or later.

This work has the LPPL maintenance status `maintained'.

The Current Maintainer of this work is Niklas Beisert.

This work consists of the files |README.txt|, |childdoc.ins| and |childdoc.dtx|
as well as the derived files |childdoc.def|, |cdocsamp.tex|
with |cdocsch1.tex|, |cdocsch2.tex|, |cdocspt3.tex|, |cdocspt4.tex|,
|cdocsdrf.tex|, |cdocsfn1.tex|, |cdocsfn2.tex|
as well as |childdoc.pdf|.

%%%%%%%%%%%%%%%%%%%%%%%%%%%%%%%%%%%%%%%%%%%%%%%%%%%%%%%%%%%%%%%%%%%%%%%%%%%%%%%%
\subsection{Files and Installation}

The package consists of the files:
%
\begin{center}
\begin{tabular}{ll}
    |README.txt|   & readme file \\
    |childdoc.ins| & installation file \\
    |childdoc.dtx| & source file \\
    |childdoc.def| & definition file \\
    |cdocsamp.tex| & sample main file \\
    |cdocsch1.tex| & sample include file \\
    |cdocsch2.tex| & sample include file \\
    |cdocspt3.tex| & sample part file \\
    |cdocspt4.tex| & sample part file \\
    |cdocsdrf.tex| & sample redirection file \\
    |cdocsfn1.tex| & sample redirection file \\
    |cdocsfn2.tex| & sample redirection file \\
    |childdoc.pdf| & manual
\end{tabular}
\end{center}
%
The distribution consists of the files
|README.txt|, |childdoc.ins| and |childdoc.dtx|.
%
\begin{itemize}
\item
Run (pdf)\LaTeX{} on |childdoc.dtx|
to compile the manual |childdoc.pdf| (this file).
\item
Run \LaTeX{} on |childdoc.ins| to create the definitions file |childdoc.def|
and the sample |cdocsamp.tex| with include files
|cdocsch1.tex|, |cdocsch2.tex|, |cdocspt3.tex|, |cdocspt4.tex|,
|cdocsdrf.tex|, |cdocsfn1.tex|, |cdocsfn2.tex|.
Then copy the file |childdoc.def| to an appropriate directory of your \LaTeX{}
distribution, e.g.\ \textit{texmf-root}|/tex/latex/childdoc|.
\end{itemize}

%%%%%%%%%%%%%%%%%%%%%%%%%%%%%%%%%%%%%%%%%%%%%%%%%%%%%%%%%%%%%%%%%%%%%%%%%%%%%%%%
\subsection{Related CTAN Packages}

There are several other packages which offer a similar functionality:
%
\begin{itemize}
\item
The packages
\href{http://ctan.org/pkg/docmute}{\textsf{docmute}},
\href{http://ctan.org/pkg/includex}{\textsf{includex}} and
\href{http://ctan.org/pkg/standalone}{\textsf{standalone}}
provide commands to include only the document body of
a child file thus allowing both files to be compiled individually.
\item
The packages \href{http://ctan.org/pkg/subdocs}{\textsf{subdocs}}
and \href{http://ctan.org/pkg/subfiles}{\textsf{subfiles}}
provide structures in which the main and child documents can be
encapsulated and allowing them to be compiled individually.
The inclusion mechanism is different from the conventional |\include|.
\item
The package \href{http://ctan.org/pkg/combine}{\textsf{combine}}
is an elaborate solution to combine several documents into one.
\end{itemize}
%
See also the CTAN topic \href{http://ctan.org/topic/subdocs}{\textsf{subdocs}}
for further related packages.
The present package differs from the above solutions in that
a document structure constructed with the conventional |\include| mechanism
just needs two extra commands at the top of every file
such that all constituent files can be compiled individually.

%%%%%%%%%%%%%%%%%%%%%%%%%%%%%%%%%%%%%%%%%%%%%%%%%%%%%%%%%%%%%%%%%%%%%%%%%%%%%%%%
%\subsection{Feature Suggestions}
%
%The following is a list of features which may be useful for future
%versions of this package:
%%
%\begin{itemize}
%\item
%\ldots
%\end{itemize}

%%%%%%%%%%%%%%%%%%%%%%%%%%%%%%%%%%%%%%%%%%%%%%%%%%%%%%%%%%%%%%%%%%%%%%%%%%%%%%%%
\subsection{Revision History}

%%%%%%%%%%%%%%%%%%%%%%%%%%%%%%%%%%%%%%%%
\paragraph{v2.0:} 2018/12/30

\begin{itemize}
\item
immediate forward processing
\item
added |\childdocby| mechanism
\item
manual restructured
\end{itemize}

%%%%%%%%%%%%%%%%%%%%%%%%%%%%%%%%%%%%%%%%
\paragraph{v1.6:} 2018/01/17

\begin{itemize}
\item
application for development of include files
\item
corrections to manual
\end{itemize}

%%%%%%%%%%%%%%%%%%%%%%%%%%%%%%%%%%%%%%%%
\paragraph{v1.5:} 2017/05/21

\begin{itemize}
\item
more complete structuring introduced
\item
|\childdocof| introduced
\item
|\childdoc| renamed to |\childdocmain|
\item
|\childredirect| renamed to |\childdocforward| and |\childdocforwardprefix|
and functionality expanded
\end{itemize}

%%%%%%%%%%%%%%%%%%%%%%%%%%%%%%%%%%%%%%%%
\paragraph{v1.0:} 2017/04/27

\begin{itemize}
\item
manual and install package
\item
first version published on CTAN
\end{itemize}

%%%%%%%%%%%%%%%%%%%%%%%%%%%%%%%%%%%%%%%%
\paragraph{v0.6:} 2017/04/26

\begin{itemize}
\item
redirection mechanism added
\end{itemize}

%%%%%%%%%%%%%%%%%%%%%%%%%%%%%%%%%%%%%%%%
\paragraph{v0.5:} 2017/04/26

\begin{itemize}
\item
functionality in definition file
\end{itemize}


%%%%%%%%%%%%%%%%%%%%%%%%%%%%%%%%%%%%%%%%%%%%%%%%%%%%%%%%%%%%%%%%%%%%%%%%%%%%%%%%
%%%%%%%%%%%%%%%%%%%%%%%%%%%%%%%%%%%%%%%%%%%%%%%%%%%%%%%%%%%%%%%%%%%%%%%%%%%%%%%%
%%%%%%%%%%%%%%%%%%%%%%%%%%%%%%%%%%%%%%%%%%%%%%%%%%%%%%%%%%%%%%%%%%%%%%%%%%%%%%%%
\appendix

\settowidth\MacroIndent{\rmfamily\scriptsize 000\ }

 \DocInput{childdoc.dtx}

\end{document}
%</driver>
% \fi
%
% %%%%%%%%%%%%%%%%%%%%%%%%%%%%%%%%%%%%%%%%%%%%%%%%%%%%%%%%%%%%%%%%%%%%%%%%%%%%%%
% %%%%%%%%%%%%%%%%%%%%%%%%%%%%%%%%%%%%%%%%%%%%%%%%%%%%%%%%%%%%%%%%%%%%%%%%%%%%%%
% \section{Sample}
%\iffalse
%<*samplemain>
%\fi
%
% The following presents a sample document
% with two chapters, two parts, a title page,
% a compile flag as well as three forwarding files to set the flag.
% It consists of eight |.tex| files:
% \begin{center}
% \begin{tabular}{ll}
% |cdocsamp.tex|&main file\\
% |cdocsch1.tex|&include file for chapter 1\\
% |cdocsch2.tex|&include file for chapter 2\\
% |cdocspt3.tex|&include file for part 3\\
% |cdocspt4.tex|&include file for part 4\\
% |cdocsdrf.tex|&forwarding file for main file in draft mode\\
% |cdocsfi1.tex|&forwarding file for final version of chapter 1\\
% |cdocsfi2.tex|&forwarding file for final version of chapter 2\\
% \end{tabular}
% \end{center}
% Each of the eight files can be compiled directly by the \LaTeX{} compiler.
%
% %%%%%%%%%%%%%%%%%%%%%%%%%%%%%%%%%%%%%%
% \paragraph{Main File.}
%
% The main file is called |cdocsamp.tex|.
%
% Load the \textsf{childdoc} definitions and
% declare the filename for the main document:
%    \begin{macrocode}
\input{childdoc.def}
\childdocmain{}
%    \end{macrocode}

% Optional override for |\version| flag:
%    \begin{macrocode}
%%\ifchilddoc\else\providecommand{\version}{draft}\fi
%    \end{macrocode}

% Define the default values for the |\version| flag
% (|final| for the main file and |draft| for childs):
%    \begin{macrocode}
\ifchilddoc
\providecommand{\version}{draft}
\else
\providecommand{\version}{final}
\fi
%    \end{macrocode}

% Load the standard document class:
%    \begin{macrocode}
\documentclass[12pt]{article}
%    \end{macrocode}

% Start the document body:
%    \begin{macrocode}
\begin{document}
%    \end{macrocode}

% Declare a title page.
% Print title, part of document being processed and version flag:
%    \begin{macrocode}
\addtocounter{page}{-1}
\begin{center}
{\LARGE\bfseries{}childdoc example\par}
\vspace{1cm}
\ifchilddoc
\ifchilddocmanual part\else chapter\fi:
`\childdocname' of `\childdocjob'\par
\else
main document: `\childdocjob'\par
\fi
version: \version\par
\end{center}
\newpage
%    \end{macrocode}

% Manually include selected file,
% otherwise process as usual:
%    \begin{macrocode}
\ifchilddocmanual
\section*{part `\childdocname'}
\input{\childdocname}
\else
%    \end{macrocode}

% Include the two chapters:
%    \begin{macrocode}
\include{cdocsch1}
\include{cdocsch2}
%    \end{macrocode}

% Include the two parts unless only chapters should be displayed:
%    \begin{macrocode}
\ifchilddoc\else
\section{part three}
\input{cdocspt3}
\section{part four}
\input{cdocspt4}
\fi
%    \end{macrocode}

% Process as usual until here:
%    \begin{macrocode}
\fi
%    \end{macrocode}

% End of document body:
%    \begin{macrocode}
\end{document}
%    \end{macrocode}
%\iffalse
%</samplemain>
%\fi
%
% %%%%%%%%%%%%%%%%%%%%%%%%%%%%%%%%%%%%%%
% \paragraph{Chapter Include Files.}
%
% The include files are called |cdocsch1.tex| and |cdocsch2.tex|.
%
%\iffalse
%<*samplechap1|samplechap2>
%\fi

% Optional override for |\version| flag:
%    \begin{macrocode}
%%\providecommand{\version}{final}
%    \end{macrocode}

% Include the main document:
%    \begin{macrocode}
\input{childdoc.def}
\childdocof{cdocsamp}
%    \end{macrocode}

%\iffalse
%</samplechap1|samplechap2>
%\fi
%
%\iffalse
%<*samplechap1>
%\fi
% Some text for chapter 1:
%    \begin{macrocode}
\section{one}
some text in chapter one
%    \end{macrocode}

%\iffalse
%</samplechap1>
%\fi
% Some text for chapter 2:
%\iffalse
%<*samplechap2>
%\fi
%    \begin{macrocode}
\section{two}
more text in chapter two
%    \end{macrocode}

%\iffalse
%</samplechap2>
%\fi
%
% %%%%%%%%%%%%%%%%%%%%%%%%%%%%%%%%%%%%%%
% \paragraph{Part Include Files.}
%
% The include files are called |cdocspt3.tex| and |cdocspt4.tex|.
%
%\iffalse
%<*samplepart3|samplepart4>
%\fi

% Optional override for |\version| flag:
%    \begin{macrocode}
%%\providecommand{\version}{final}
%    \end{macrocode}

% Include the main document:
%    \begin{macrocode}
\input{childdoc.def}
\childdocby{cdocsamp}
%    \end{macrocode}

%\iffalse
%</samplepart3|samplepart4>
%\fi
%
%\iffalse
%<*samplepart3>
%\fi
% Some text for part 3:
%    \begin{macrocode}
some text in part three
%    \end{macrocode}

%\iffalse
%</samplepart3>
%\fi
% Some text for part 4:
%\iffalse
%<*samplepart4>
%\fi
%    \begin{macrocode}
more text in part four
%    \end{macrocode}

%\iffalse
%</samplepart4>
%\fi
%
% %%%%%%%%%%%%%%%%%%%%%%%%%%%%%%%%%%%%%%
% \paragraph{Forwarding for a Complete Draft.}
%
% The following forwarding file |cdocsdrf.tex|
% compiles the main document in draft mode:
%\iffalse
%<*sampledraft>
%\fi
%    \begin{macrocode}
\def\version{draft}
\input{childdoc.def}
\childdocforward{cdocsamp}
%    \end{macrocode}

%\iffalse
%</sampledraft>
%\fi
%
% %%%%%%%%%%%%%%%%%%%%%%%%%%%%%%%%%%%%%%
% \paragraph{Forwarding for Final Version of the Chapters.}
%
% The following forwarding files |cdocsfn1.tex| and |cdocsfn2.tex|
% (with identical content)
% compile the final versions of the child documents
% |cdocsch1.tex| and |cdocsch2.tex|, respectively:
%\iffalse
%<*samplefinal>
%\fi
%    \begin{macrocode}
\def\version{final}
\input{childdoc.def}
\childdocforwardprefix[cdocsamp]{cdocsfn}{cdocsch}
%    \end{macrocode}

%\iffalse
%</samplefinal>
%\fi
%
% %%%%%%%%%%%%%%%%%%%%%%%%%%%%%%%%%%%%%%
% \paragraph{Command Line Processing.}
%
% The following three command lines generate the output files
% |cdocscld|, |cdocscl1| and |cdocscl2|
% which should be identical to
% |cdocsdrf|, |cdocsch1| and |cdocsfn2|, respectively:
% \begin{center}
% \begin{tabular}{l}
% |latex -jobname cdocscld \|\\
% |  "\def\version{draft}\input{childdoc.def}\childdocforward{cdocsamp}"|\\
% |latex -jobname cdocscl1 \|\\
% |  "\input{childdoc.def}\childdocforward[cdocsamp]{cdocsch1}"|\\
% |latex -jobname cdocscl2 \|\\
% |  "\def\version{final}\input{childdoc.def}\childdocforward{cdocsch2}"|
% \end{tabular}
% \end{center}
% Note that the trailing backslash on each first line
% merely continues the input to the second line
% (for convenient cut ant paste).
% Furthermore, the command |latex| can be replaced by any
% of its alternative versions such as |pdflatex|.
%
% %%%%%%%%%%%%%%%%%%%%%%%%%%%%%%%%%%%%%%%%%%%%%%%%%%%%%%%%%%%%%%%%%%%%%%%%%%%%%%
% %%%%%%%%%%%%%%%%%%%%%%%%%%%%%%%%%%%%%%%%%%%%%%%%%%%%%%%%%%%%%%%%%%%%%%%%%%%%%%
% \section{Implementation}
%\iffalse
%<*package>
%\fi
%
% This section describes the definitions file |childdoc.def|.

% The definitions cannot be loaded using |\usepackage| or |\RequirePackage|
% which has a mechanism to prevent loading a style file more than once.
% When loading the definitions by means of |\input|
% multiple instances have to be prevented manually:
%\iffalse
%This code needs to be before the `\ProvidesFile' directive
%which is defined at the beginning of this file.
%Therefore it is also placed there and commented out here.
%</package>
%<*discard>
%\fi
%    \begin{macrocode}
\ifdefined\childdocmain\endinput\fi
%    \end{macrocode}
%\iffalse
%</discard>
%<*package>
%\fi
%
% \macro{\ifchilddoc}
% \macro{\ifchilddocmanual}
% The conditional |\ifchilddoc| tells whether a
% child (true) or main (false) document is being compiled.
% The conditional |\ifchilddocmanual| tells whether
% the |\includeonly| mechanism is used (false) or
% the selection of child files must be performed manually (true).
% The definitions initialise to false:
%    \begin{macrocode}
\newif\ifchilddoc
\newif\ifchilddocmanual
%    \end{macrocode}

% \macro{\childdocname}
% \macro{\childdocjob}
% The macro |\childdocname| stores the name of the main document
% to be compiled. The macro |\childdocjob| stores the name of
% the document on which the \LaTeX{} compiler was originally invoked.
% The content of |\jobname| cannot be compared
% to filenames specified in the source due to different catcodes.
% The following code rescans |\jobname|, stores the result
% in |\childdocname| and saves a copy in |\childdocjob|:
%    \begin{macrocode}
\edef\childdocname{\scantokens\expandafter{\jobname\noexpand}}
\let\childdocjob\childdocname
%    \end{macrocode}

% \macro{\childdocdisable}
% The macro |\childdocdisable| prevents the main file
% from being processed more than once.
% At this stage, the main document command |\childdocmain|
% is assumed to be called once again where it should do nothing.
% Any subsequent call to it should prevent
% a secondary processing of the main document
% It overwrites the forwarding commands
% |\childdocof| and |\childdocforward|
% with empty macros to prevent further inclusions of the main document:
%    \begin{macrocode}
\newcommand{\childdocdisable}
{
  \renewcommand{\childdocmain}[1]{\renewcommand{\childdocmain}[1]{\endinput}}
  \renewcommand{\childdocof}[1]{}
  \renewcommand{\childdocby}[2][]{}
  \renewcommand{\childdocforward}[2][]{}
  \renewcommand{\childdocdisable}{}
}
%    \end{macrocode}

% \macro{\childdocmain}
% The macro |\childdocmain| is to be called at the top of the main file
% with nothing or the main filename (without extension) as argument.
% First, it breaks loops.
% If the argument is not empty and does not match |\childdocname|
% (which is set by the first inclusion of |childdoc.def|),
% |\ifchilddoc| is set to true, |\includeonly| is applied to the child file
% and |\jobname| is set to the main file
% (for proper handling of |.aux| files):
%    \begin{macrocode}
\newcommand{\childdocmain}[1]
{
  \childdocdisable\childdocmain{}
  \if?#1?\else
    \begingroup
      \def\childdoctmp{#1}
      \ifx\childdoctmp\childdocname
        \def\childdoctmp{}
      \else
        \def\childdoctmp
        {
          \childdoctrue
          \includeonly{\childdocname}
          \def\childdocjob{#1}
          \def\jobname{#1}
        }
      \fi
      \expandafter
    \endgroup
    \childdoctmp
  \fi
}
%    \end{macrocode}

% \macro{\childdocof}
% The command |\childdocof| redirects
% compilation to the main file |#1|.
%    \begin{macrocode}
\newcommand{\childdocof}[1]
{
  \childdocdisable
  \childdoctrue
  \includeonly{\childdocname}
  \def\jobname{#1}
  \def\childdocjob{#1}
  \input{#1}
}
%    \end{macrocode}

% \macro{\childdocby}
% The command |\childdocby| ....
%    \begin{macrocode}
\newcommand{\childdocby}[2][]
{
  \childdocdisable
  \childdoctrue
  \childdocmanualtrue
  \if?#1?\else
    \def\jobname{#2}
  \fi
  \def\childdocjob{#2}
  \input{#2}
  \endinput
}
%    \end{macrocode}

% \macro{\childdocforward}
% The command |\childdocforward| redirects
% compilation to the main file or
% (if the optional argument is given) a child file.
% Parameters are set as if the main file
% or a child file starting with |\childdocof| was compiled.
% Then compilation is handed over to the main file:
%    \begin{macrocode}
\newcommand{\childdocforward}[2][]
{
  \begingroup
    \if?#1?
      \def\childdoctmp
      {
        \def\childdocname{#2}
        \def\childdocjob{#2}
        \def\jobname{#2}
        \input{#2}
        \endinput
      }
    \else
      \def\childdoctmp
      {
        \childdocdisable
        \def\childdocname{#2}
        \childdoctrue
        \includeonly{#2}
        \def\childdocjob{#1}
        \def\jobname{#1}
        \input{#1}
        \endinput
      }
    \fi
    \expandafter
  \endgroup
  \childdoctmp
}
%    \end{macrocode}

% \macro{\childdocforwardprefix}
% The command |\childdocforwardprefix| redirects
% compilation to the main or a child file by means of a pattern.
% The prefix |#1| in the current filename is replaced by |#2|
% and the suffix of the current filename is kept
% (it is assumed that the filename does not contain the substring `|~~~|'
% which is used as a delimiter).
% Compilation is handed over to the new file by |\childdocforward|:
%    \begin{macrocode}
\newcommand{\childdocforwardprefix}[3][]
{
  \begingroup
    \def\childdocextract #2##1~~~{\def\childdoctmp{\childdocforward[#1]{#3##1}}}
    \expandafter\childdocextract\childdocname~~~
    \expandafter
  \endgroup
  \childdoctmp
}
%    \end{macrocode}

% \macro{\childdoc}
% The deprecated macro |\childdoc| is a legacy version of |\childdocmain|:
%    \begin{macrocode}
\newcommand{\childdoc}{\childdocmain}
%    \end{macrocode}

% \macro{\childdocredirect}
% The deprecated macro |\childdocredirect| is a legacy version
% of |\childdocforward| and |\childdocforwardprefix|:
%    \begin{macrocode}
\newcommand{\childdocredirect}[2][]
{
  \begingroup
    \if?#1?
      \def\childdoctmp{\childdocforward{#2}}
    \else
      \def\childdoctmp{\childdocforwardprefix{#1}{#2}}
    \fi
    \expandafter
  \endgroup
  \childdoctmp
}
%    \end{macrocode}

%\iffalse
%</package>
%\fi
%
\endinput
\childdocforward[cdocsamp]{cdocsch1}"|\\
% |latex -jobname cdocscl2 \|\\
% |  "\def\version{final}% \iffalse
%
% childdoc.dtx Copyright (C) 2017-2018 Niklas Beisert
%
% This work may be distributed and/or modified under the
% conditions of the LaTeX Project Public License, either version 1.3
% of this license or (at your option) any later version.
% The latest version of this license is in
%   http://www.latex-project.org/lppl.txt
% and version 1.3 or later is part of all distributions of LaTeX
% version 2005/12/01 or later.
%
% This work has the LPPL maintenance status `maintained'.
%
% The Current Maintainer of this work is Niklas Beisert.
%
% This work consists of the files childdoc.dtx and childdoc.ins
% and the derived files childdoc.def and cdocsamp.tex with
% cdocsch1.tex, cdocsch2.tex, cdocsdrf.tex, cdocsfn1.tex, cdocsfn2.tex.
%
%<package>\ifdefined\childdocmain\endinput\fi
%<package>\ProvidesFile{childdoc.def}[2018/12/30 v2.0 child document driver]
%<samplemain>\ProvidesFile{cdocsamp.tex}[2018/12/30 v2.0 sample for childdoc]
%<*driver>
%\ProvidesFile{childdoc.drv}[2018/12/30 v2.0 childdoc reference manual file]
\PassOptionsToClass{10pt,a4paper}{article}
\documentclass{ltxdoc}

\usepackage[margin=35mm]{geometry}
\usepackage{hyperref}
\usepackage{hyperxmp}
\usepackage[usenames]{color}

\hypersetup{colorlinks=true}
\hypersetup{pdfstartview=FitH}
\hypersetup{pdfpagemode=UseNone}
\hypersetup{pdfsource={}}
\hypersetup{pdflang={en-UK}}
\hypersetup{pdfcopyright={Copyright 2017-2018 Niklas Beisert.
  This work may be distributed and/or modified under the
  conditions of the LaTeX Project Public License, either version 1.3
  of this license or (at your option) any later version.}}
\hypersetup{pdflicenseurl={http://www.latex-project.org/lppl.txt}}
\hypersetup{pdfcontactaddress={ETH Zurich, ITP, HIT K,
  Wolfgang-Pauli-Strasse 27}}
\hypersetup{pdfcontactpostcode={8093}}
\hypersetup{pdfcontactcity={Zurich}}
\hypersetup{pdfcontactcountry={Switzerland}}
\hypersetup{pdfcontactemail={nbeisert@itp.phys.ethz.ch}}
\hypersetup{pdfcontacturl={http://people.phys.ethz.ch/\xmptilde nbeisert/}}

\newcommand{\secref}[1]{\hyperref[#1]{section \ref*{#1}}}

\parskip1ex
\parindent0pt
\let\olditemize\itemize
\def\itemize{\olditemize\parskip0pt}

\begin{document}

\title{The \textsf{childdoc} Package}
\hypersetup{pdftitle={The childdoc Package}}
\author{Niklas Beisert\\[2ex]
  Institut f\"ur Theoretische Physik\\
  Eidgen\"ossische Technische Hochschule Z\"urich\\
  Wolfgang-Pauli-Strasse 27, 8093 Z\"urich, Switzerland\\[1ex]
  \href{mailto:nbeisert@itp.phys.ethz.ch}
  {\texttt{nbeisert@itp.phys.ethz.ch}}}
\hypersetup{pdfauthor={Niklas Beisert}}
\hypersetup{pdfsubject={Manual for the LaTeX2e Package childdoc}}
\date{30 December 2018, \textsf{v2.0}}
\maketitle

\begin{abstract}\noindent
\textsf{childdoc} is a \LaTeXe{} package
that enables the direct compilation
of document sections included by |\include|
to individual files.
\end{abstract}

\begingroup
\parskip0ex
\tableofcontents
\endgroup

%%%%%%%%%%%%%%%%%%%%%%%%%%%%%%%%%%%%%%%%%%%%%%%%%%%%%%%%%%%%%%%%%%%%%%%%%%%%%%%%
%%%%%%%%%%%%%%%%%%%%%%%%%%%%%%%%%%%%%%%%%%%%%%%%%%%%%%%%%%%%%%%%%%%%%%%%%%%%%%%%
\section{Introduction}

\LaTeX{} provides a mechanism to structure a large document (such as a book)
into a main file and several child files (containing the chapters)
using the |\include| command.
This mechanism is beneficial for documents
which span hundreds of pages in order to
make the source file(s) more manageable.
Moreover, compilation can be restricted to
selected child files by means of the |\includeonly| command.
The latter feature can be used to reduce the compilation time while editing
(this was significantly more useful in the earlier days of \LaTeX{})
or to generate a smaller document which is easier to navigate.
Another application of |\includeonly| is to generate
documents consisting of selected parts of the complete document.

However, there are a few drawbacks of the plain |\include| mechanism:
\begin{itemize}
\item
The child files cannot be compiled on their own,
they can only be compiled via the main file.
A naive editing environment
(such as a text editor with an option
to have the current file processed by \LaTeX)
may require one to switch to the main file before compiling;
attempting to compile the child file produces errors.
\item
The main file must be modified (each time)
to adjust the |\includeonly| command
to the present needs. This easily leaves the main file in a messy state.
\item
The generated document will always carry the filename
of the main document. This is inconvenient if
several child files are to be compiled and
to be kept for distribution.
\end{itemize}

The present package provides a simple interface
to make child files individually compilable by \LaTeX{}.
Compiling a child file then has the same effect as compiling
the main file with an |\includeonly| command
to select the appropriate child.
Moreover the generated document will carry the name of the child
rather than the main file.
This resolves all three above issues.

This feature is meant to make the editing of books,
thesis documents and lecture notes somewhat more convenient.
However, the package can also be used efficiently for
composing a series of documents (such as exercise sheets)
which are typically distributed individually.
It then assists the author in generating the individual documents
(potentially in different versions)
as well as a document containing the collected series.
Another application is in developing style files
or other kinds of included material
where compilation of the style file could redirect
to a sample or test file.

%%%%%%%%%%%%%%%%%%%%%%%%%%%%%%%%%%%%%%%%%%%%%%%%%%%%%%%%%%%%%%%%%%%%%%%%%%%%%%%%
%%%%%%%%%%%%%%%%%%%%%%%%%%%%%%%%%%%%%%%%%%%%%%%%%%%%%%%%%%%%%%%%%%%%%%%%%%%%%%%%
\section{Usage}

First of all, the package \textsf{childdoc} is \emph{not} a standard
\LaTeXe{} |.sty| style file! Therefore it needs to be invoked in
a non-standard way.

%%%%%%%%%%%%%%%%%%%%%%%%%%%%%%%%%%%%%%%%%%%%%%%%%%%%%%%%%%%%%%%%%%%%%%%%%%%%%%%%
\subsection{Included Files}
\label{sec:include}

%%%%%%%%%%%%%%%%%%%%%%%%%%%%%%%%%%%%%%%%
\DescribeMacro{\childdocmain}
To use the package, add the commands
\begin{center}
\begin{tabular}{l}
|\input{childdoc.def}|\\
|\childdocmain{}|\\
\end{tabular}
\end{center}
at the very top of the main \LaTeX{} file,
in particular \emph{before} the |\documentclass| statement!
The argument of |\childdocmain| should be left empty
(but it must be present).

%%%%%%%%%%%%%%%%%%%%%%%%%%%%%%%%%%%%%%%%
\DescribeMacro{\childdocof}
Furthermore, add the commands
\begin{center}
\begin{tabular}{l}
|\input{childdoc.def}|\\
|\childdocof{|\textit{main}|}|\\
\end{tabular}
\end{center}
at the top of every child file \textit{child}
which is included by |\include{|\textit{child}|}|
from within the main file
(or at least for those files to be compiled individually).
The argument \textit{main} must be the filename of the main file.

There are a couple of
considerations in setting up the main and child documents:

%%%%%%%%%%%%%%%%%%%%%%%%%%%%%%%%%%%%%%%%
\paragraph{Restrictions.}

Please note the following restrictions:
\begin{itemize}
\item
|\childdocmain| must be called with one argument \textit{main}
to ensure compatibility with earlier version of the package.
It must either be empty (|\childdocmain{}|)
or precisely match the filename of the main file in which it is specified.
See \secref{sec:detection} for further information.
\item
The filename \textit{main} must be specified without the |.tex| extension.
\item
The filename \textit{main} is case sensitive
(even in case-insensitive file systems)
due to internal string comparison.
\item
The argument \textit{main} should be fully expanded, it cannot be a macro.
\item
Subdirectories and special characters should be avoided in filenames.
\item
The command |\childdocmain{|\textit{main}|}| must be followed by a whitespace.
It should not be followed immediately by another command
or by a comment mark `|%|'.
This is because the \TeX{} parser reads the token immediately following
the argument of |\childdocmain| and puts it
at the beginning of every child section;
however, a white\-space is ignored.
\end{itemize}

%%%%%%%%%%%%%%%%%%%%%%%%%%%%%%%%%%%%%%%%
\paragraph{Content of Main File.}

It is advisable to place all content in the child files included by |\include|.
Any output contained in the main file will appear in all child documents
unless suppressed manually;
it cannot be suppressed automatically by the |\includeonly| directive
and thus should normally be avoided.
A method to include some content in the main file
by means of conditional processing is described in \secref{sec:conditional}.

%%%%%%%%%%%%%%%%%%%%%%%%%%%%%%%%%%%%%%%%
\paragraph{Page Numbering.}

When only a part of the document is compiled,
the appropriate numbering of pages
(as well as other status parameters)
is determined from the |.aux| files.
The latter contain information from previous passes.
However this information needs to propagate through
all intermediate child documents.
Therefore the page numbering in child documents may well
be inconsistent until the complete document is compiled at least once.

A useful (if unconventional) way to always ensure a consistent
page numbering is to restart the numbering in each child document
and denote the pages by `\textit{child}|.|\textit{page}'
where \textit{child} represents the chapter/section number of the child file.
This can be achieved by the command
|\numberwithin{page}{|\textit{child}|}|
of the \textsf{amsmath} package
where \textit{child} can be |chapter| or |section|
depending on the chosen structuring.
Alternatively, one can modify the macro |\thepage| appropriately
and reset the counter |page| at the start of each child file.

%%%%%%%%%%%%%%%%%%%%%%%%%%%%%%%%%%%%%%%%%%%%%%%%%%%%%%%%%%%%%%%%%%%%%%%%%%%%%%%%
\subsection{Conditional Processing}
\label{sec:conditional}

The package provides a mechanism to compile different versions
of a document. To customise the versions further some conditional processing
can come in handy to distinguish which version is being compiled.
The package provides two macros to describe the compilation context:

%%%%%%%%%%%%%%%%%%%%%%%%%%%%%%%%%%%%%%%%
\DescribeMacro{\ifchilddoc}
The conditional |\ifchilddoc| distinguishes between the compilation of
child documents and the main document:
%
\begin{center}
|\ifchilddoc |\textit{child-code}| |[|\||else |\textit{main-code}]| \||fi|
\end{center}

%%%%%%%%%%%%%%%%%%%%%%%%%%%%%%%%%%%%%%%%
\DescribeMacro{\childdocname}
\DescribeMacro{\childdocjob}
The macro |\childdocname| contains the filename (without extension)
of the main or child file being processed.
Note that |\childdocjob| will always contain the name of the main file.

%%%%%%%%%%%%%%%%%%%%%%%%%%%%%%%%%%%%%%%%
\paragraph{Title Page.}

Conditional processing can be used to include a title or banner page
in the main document when proper precautions are taken.
Importantly, the code in the main file should ensure that the page counter
(as well as other status parameters which are stored in the |.aux| files)
takes the same value after the conditional processing.
Otherwise the page numbers may take divergent values
depending on which part is compiled.

For example, a title page could be declared by:
%
\begin{center}
\begin{tabular}{l}
|\ifchilddoc\||else|\\
|\addtocounter{page}{-1}|\\
\textit{code for title page}\\
|\newpage|\\
|\||fi|
\end{tabular}
\end{center}
%
A banner page for the child documents can be generated by:
%
\begin{center}
\begin{tabular}{l}
|\ifchilddoc|\\
|\addtocounter{page}{-1}|\\
\textit{code for banner page}\\
|\newpage|\\
|\||fi|
\end{tabular}
\end{center}
%
Here one could write a message such as:
\begin{center}
|This is the part \childdocname{} of \childdocjob{}.|
\end{center}

%%%%%%%%%%%%%%%%%%%%%%%%%%%%%%%%%%%%%%%%%%%%%%%%%%%%%%%%%%%%%%%%%%%%%%%%%%%%%%%%
\subsection{Flags}
\label{sec:flags}

The package makes it easy to generate different versions
of the main or child documents.
To this end compilation flags can be defined
and assigned different default values.
They will be particularly useful in conjunction
with the forwarding mechanism described in \secref{sec:forward}.

For example, it may be useful to have a flag |\version|
which can be set to |draft| or |final|.
The document source will contain some conditional code
depending on the value of |\version|.
Suppose further, the flag should default to |final| for the main file
and to |draft| for child files
which is a natural assignment for editing the document.
This is achieved by placing the following code
in the preamble of the main document
(below the |\childdocmain| directive):
%
\begin{center}
\begin{tabular}{l}
|\ifchilddoc|\\
|\providecommand{\version}{draft}|\\
|\||else|\\
|\providecommand{\version}{final}|\\
|\||fi|
\end{tabular}
\end{center}
%
The definition by |\providecommand| makes sure
that previous definitions are not overwritten.
Further statements |\providecommand{\version}{...}|
can thus be added before the above code to override it.

For the main file, one might add a line
(between |\childdocmain| and the above block)
%
\begin{center}
|%\ifchilddoc\||else\providecommand{\version}{draft}\||fi|
\end{center}
%
which can be uncommented to produce a draft version.
Likewise one can add a line to the very top of a child file
(above the |\childdocof{|\textit{main}|}| directive)
%
\begin{center}
|%\providecommand{\version}{final}|
\end{center}
%
which can be uncommented to produce the final version of this child document.

%%%%%%%%%%%%%%%%%%%%%%%%%%%%%%%%%%%%%%%%%%%%%%%%%%%%%%%%%%%%%%%%%%%%%%%%%%%%%%%%
\subsection{Forwarding}
\label{sec:forward}

Different versions of the main or child documents
using compilation flags as described in \secref{sec:flags}
can be (permanently) stored in different files
for convenient compilation, viewing and distribution.
To this end, the package defines a command
to pass on compilation to a different file:

%%%%%%%%%%%%%%%%%%%%%%%%%%%%%%%%%%%%%%%%
\DescribeMacro{\childdocforward}
The command |\childdocforward| redirects processing to
another source file:
%
\begin{center}
\begin{tabular}{l}
|\input{childdoc.def}|\\
|\childdocforward[|\textit{main}|]{|\textit{dest}|}|\\
\end{tabular}
\end{center}
%
The argument \textit{dest} is the destination file
(without extension).
It should be the main file or one of the child files.
Note that further \textsf{childdoc} directives
such as |\childdocof| and |\childdocforward|
in the indicated file will be processed in this form.
The optional argument \textit{main}
passes on directly to the main file \textit{main}
while pretending to compile the child \textit{dest}.
This form behaves as if \textit{dest}
issues |\childdocof{|\textit{main}|}| right away,
and no further \textsf{childdoc} directives will be processed.

%%%%%%%%%%%%%%%%%%%%%%%%%%%%%%%%%%%%%%%%
\DescribeMacro{\...prefix}
In the alternative form |\childdocforwardprefix|,
%
\begin{center}
\begin{tabular}{l}
|\input{childdoc.def}|\\
|\childdocforwardprefix[|\textit{main}|]{|\textit{prefix}|}{|\textit{dest}|}|
\end{tabular}
\end{center}
%
the destination file is determined by a pattern
depending on the current file:
To make this work, the current file must be called
`{\textit{prefix}\hspace{0.2em}\textit{suffix}}'
with \textit{prefix} matching precisely the argument.
Processing is then passed on to the file
`{\textit{dest}\hspace{0.2em}\textit{suffix}}'.
Surely, the same effect is achieved by
directly specifying the
argument `{\textit{dest}\hspace{0.2em}\textit{suffix}}'
in the first form.
However, that requires to set up a different file
for each child. With the alternative form of the command
all these files can have exactly the same content
which simplifies setting them up and maintaining them.

For example, the following file |draft.tex|
with a compilation flag |\version| as described in \secref{sec:flags}
compiles the main document as a draft:
%
\begin{center}
\begin{tabular}{l}
|\def\version{draft}|\\
|\input{childdoc.def}|\\
|\childdocforward{|\textit{main}|}|
\end{tabular}
\end{center}
%
Likewise, the following files |final|\textit{nn}|.tex|
compile the final version of the child document
|child|\textit{nn}|.tex|:
%
\begin{center}
\begin{tabular}{l}
|\def\version{final}|\\
|\input{childdoc.def}|\\
|\childdocforwardprefix{final}{child}|
\end{tabular}
\end{center}
%

Note that when several versions of a main file and/or of each child file
are to be generated, it may be convenient to set up a |Makefile| or
shell script to automatise the process.

%%%%%%%%%%%%%%%%%%%%%%%%%%%%%%%%%%%%%%%%%%%%%%%%%%%%%%%%%%%%%%%%%%%%%%%%%%%%%%%%
\subsection{Command Line Processing}
\label{sec:commandline}

The effect of redirection files can also be achieved by invoking
the \LaTeX{} compiler with a more elaborate command line.
Most conveniently this should be done as part
of a shell script or a |Makefile|.

When using \textsf{childdoc} in the main file, the following
command lines effectively perform a redirection
(note that depending on the shell being used,
backslashes may have to be doubled: `|\|' $\to$ `|\\|'):
%
\begin{center}
|... -jobname "|\textit{target}|" |\\|"|[\textit{flags}]%
|\input{childdoc.def}\childdocforward[|\textit{main}|]{|\textit{dest}|}"|
\end{center}
%
Here \textit{target} is the name of the output file,
\textit{main} is the name of the main file
and \textit{dest} is the name of the main or child file to be processed
(all filenames without extensions).
The optional argument \textit{main} can be omitted
if \textit{main} matches \textit{dest}.
Optionally, compilation \textit{flags} can be defined via |\def| commands.
This command line makes the \TeX{} engine believe
it is compiling the file \textit{target}
whose content is specified as the latter parameter.
The provided code then forwards the processing to
\textit{main} or \textit{dest} as described in \secref{sec:forward}.

%%%%%%%%%%%%%%%%%%%%%%%%%%%%%%%%%%%%%%%%%%%%%%%%%%%%%%%%%%%%%%%%%%%%%%%%%%%%%%%%
\subsection{Include by Input}
\label{sec:input}

Including child documents by |\include| has some restrictions by design.
Most notably, the content of a child document always occupies
its own set of pages; pages cannot be shared between child documents.
Usually, this behaviour makes perfect sense
because each child document contain an essential part of the document.
However, in some situations it may be desirable to compose
a document from a collection of parts
without having mandatory page breaks between then.
For this case, the package
provides a mechanism to include parts
by |\input| which can also be processed individually.
However, by construction this mechanism
requires manual handling of the content to be output.

%%%%%%%%%%%%%%%%%%%%%%%%%%%%%%%%%%%%%%%%
\DescribeMacro{\ifchilddocmanual}
The main file should be prepared as usual, see \secref{sec:include}.
However, the document body must make a distinction
between processing of an individual part and of the main document, e.g.:
%
\begin{center}
\begin{tabular}{l}
|\ifchilddocmanual|\\
|\input{\childdocname}|\\
|\||else|\\
\textit{document body with }|\input{|\textit{part}|}|\\
|\||fi|
\end{tabular}
\end{center}
%
The conditional |\ifchilddocmanual| is true whenever
a part to be included by |\input| is being compiled,
and the name of the part is stored in |\childdocname|.

%%%%%%%%%%%%%%%%%%%%%%%%%%%%%%%%%%%%%%%%
\DescribeMacro{\childdocby}
Each part to be included by |\input| should start with:
%
\begin{center}
\begin{tabular}{l}
|\input{childdoc.def}|\\
|\childdocby{|\textit{main}|}|\\
\end{tabular}
\end{center}
%
The directive |\childdocby| is similar to |\childdocof|
described in \secref{sec:include},
but the subsequent selection of content must be done manually.
To that end, both |\ifchilddoc| and |\ifchilddocmanual|
will be true upon processing of a part,
and the name of the part is stored in |\childdocname|.
Note that |\jobname| will be set to the filename of the current part
so that each part receives an individual |.aux| file
that does not interfere with the |.aux| file(s) of the main document.
This behaviour can be altered by the alternative form
|\childdocby[*]{|\textit{main}|}| (with a non-empty optional argument)
which uses the |.aux| file of the main document
by setting |\jobname| to \textit{main}.

%%%%%%%%%%%%%%%%%%%%%%%%%%%%%%%%%%%%%%%%%%%%%%%%%%%%%%%%%%%%%%%%%%%%%%%%%%%%%%%%
\subsection{Driver Development}
\label{sec:driver}

The \textsf{childdoc} mechanism can also be use for the development
of definition files such as \LaTeX{} styles or classes.
This case differs from the above setup with multiple parts
included by |\include| in that no |\includeonly| should be invoked.
This can be achieved by starting the include file
(before |\ProvidesPackage|) with:
%
\begin{center}
\begin{tabular}{l}
|\input{childdoc.def}|\\
|\childdocforward{|\textit{main}|}|\\
\end{tabular}
\end{center}
%
or alternatively with:
%
\begin{center}
\begin{tabular}{l}
|\input{childdoc.def}|\\
|\childdocby{|\textit{main}|}|\\
\end{tabular}
\end{center}
%
Both forms have slightly different effects as described above.
The main file is prepared as usual, see \secref{sec:include}.

%%%%%%%%%%%%%%%%%%%%%%%%%%%%%%%%%%%%%%%%%%%%%%%%%%%%%%%%%%%%%%%%%%%%%%%%%%%%%%%%
\subsection{Legacy Detection}
\label{sec:detection}

The directive |\childdocmain| in the main file can detect
whether the complete document or merely a child is to be compiled
even without using the directive |\childdocof|.
This method is deprecated because it is less robust
and there is no compelling reason to use it;
it is merely provided for backward compatibility
and it may be removed in future versions.

If the detection mechanism is to be used,
it is mandatory to correctly specify
the filename of the main file as the argument of |\childdocmain|:
%
\begin{center}
\begin{tabular}{l}
|\input{childdoc.def}|\\
|\childdocmain{|\textit{main}|}|\\
\end{tabular}
\end{center}
%
If |\jobname| does not match the argument \textit{main} of |\childdocmain|,
it is assumed that |\jobname| points to the child file to be compiled.
When using |\childdocmain| with the main file specified as argument,
it suffices to start a child file
with just |\input{|\textit{main}|}|
without loading of the package and using |\childdocof|.
If instead all processing is done
with the appropriate \textsf{childdoc} directives,
the argument of \textit{main} of |\childdocmain| can be empty.

An alternative version of the command line processing described
in \secref{sec:commandline} using the detection mechanism reads:
%
\begin{center}
|... -jobname "|\textit{target}|" "|[\textit{flags}]%
[|\def\jobname{|\textit{dest}|}|]|\input{|\textit{main}|}"|
\end{center}

%%%%%%%%%%%%%%%%%%%%%%%%%%%%%%%%%%%%%%%%%%%%%%%%%%%%%%%%%%%%%%%%%%%%%%%%%%%%%%%%
\subsection{Manual Code}
\label{sec:manual}

In case one cannot be certain whether the definitions file |childdoc.def|
is installed on the target \TeX{} distribution
and one prefers not to ship it,
it is conceivable to paste a few relevant commands into the sources.

To that end, drop all statements |\input{childdoc.def}|
and perform the replacements as outlined below.
Instead of |\childdocmain{|\textit{main}|}| add the following code
to the top of the main file:
%
\begin{center}
\begin{tabular}{l}
|\||ifdefined\childdocname\endinput\||fi\newif\ifchilddoc|\\
|\edef\childdocname{\scantokens\expandafter{\jobname\noexpand}}|\\
|\def\childdocmain{|\textit{main}|}\||ifx\childdocmain\childdocname\||else|\\
|\childdoctrue\includeonly{\childdocname}\let\jobname\childdocmain\||fi|\\
\end{tabular}
\end{center}
%
Instead of |\childdocof{|\textit{main}|}| just include the main file
at the top of each child file:
%
\begin{center}
|\input{|\textit{main}|}|
\end{center}
%
A simple redirection |\childdocforward{|\textit{dest}|}| is achieved by:
%
\begin{center}
|\def\jobname{|\textit{dest}|}\input{\jobname}|
\end{center}
%
The redirection with prefix
|\childdocforwardprefix[|\textit{prefix}|]{|\textit{dest}|}|
is accomplished by:
%
\begin{center}
\begin{tabular}{l}
|{\edef\jobname{\scantokens\expandafter{\jobname\noexpand}}|\\
|\def\redirectjob |\textit{prefix}|#1~~~{\gdef\jobname{|\textit{dest}|#1}}|\\
|\expandafter\redirectjob\jobname~~~}\input{\jobname}|
\end{tabular}
\end{center}

In an alternative approach,
child documents can be compiled by a specific command line
without additional code or specific definitions:
%
\begin{center}
|... -jobname "|\textit{target}|" "|[\textit{flags}]%
|\includeonly{|\textit{dest}|}\input{|\textit{main}|}"|
\end{center}
%

%%%%%%%%%%%%%%%%%%%%%%%%%%%%%%%%%%%%%%%%%%%%%%%%%%%%%%%%%%%%%%%%%%%%%%%%%%%%%%%%
%%%%%%%%%%%%%%%%%%%%%%%%%%%%%%%%%%%%%%%%%%%%%%%%%%%%%%%%%%%%%%%%%%%%%%%%%%%%%%%%
\section{Information}

%%%%%%%%%%%%%%%%%%%%%%%%%%%%%%%%%%%%%%%%%%%%%%%%%%%%%%%%%%%%%%%%%%%%%%%%%%%%%%%%
\subsection{Copyright}

Copyright \copyright{} 2017--2018 Niklas Beisert

This work may be distributed and/or modified under the
conditions of the \LaTeX{} Project Public License, either version 1.3
of this license or (at your option) any later version.
The latest version of this license is in
  \url{http://www.latex-project.org/lppl.txt}
and version 1.3 or later is part of all distributions of \LaTeX{}
version 2005/12/01 or later.

This work has the LPPL maintenance status `maintained'.

The Current Maintainer of this work is Niklas Beisert.

This work consists of the files |README.txt|, |childdoc.ins| and |childdoc.dtx|
as well as the derived files |childdoc.def|, |cdocsamp.tex|
with |cdocsch1.tex|, |cdocsch2.tex|, |cdocspt3.tex|, |cdocspt4.tex|,
|cdocsdrf.tex|, |cdocsfn1.tex|, |cdocsfn2.tex|
as well as |childdoc.pdf|.

%%%%%%%%%%%%%%%%%%%%%%%%%%%%%%%%%%%%%%%%%%%%%%%%%%%%%%%%%%%%%%%%%%%%%%%%%%%%%%%%
\subsection{Files and Installation}

The package consists of the files:
%
\begin{center}
\begin{tabular}{ll}
    |README.txt|   & readme file \\
    |childdoc.ins| & installation file \\
    |childdoc.dtx| & source file \\
    |childdoc.def| & definition file \\
    |cdocsamp.tex| & sample main file \\
    |cdocsch1.tex| & sample include file \\
    |cdocsch2.tex| & sample include file \\
    |cdocspt3.tex| & sample part file \\
    |cdocspt4.tex| & sample part file \\
    |cdocsdrf.tex| & sample redirection file \\
    |cdocsfn1.tex| & sample redirection file \\
    |cdocsfn2.tex| & sample redirection file \\
    |childdoc.pdf| & manual
\end{tabular}
\end{center}
%
The distribution consists of the files
|README.txt|, |childdoc.ins| and |childdoc.dtx|.
%
\begin{itemize}
\item
Run (pdf)\LaTeX{} on |childdoc.dtx|
to compile the manual |childdoc.pdf| (this file).
\item
Run \LaTeX{} on |childdoc.ins| to create the definitions file |childdoc.def|
and the sample |cdocsamp.tex| with include files
|cdocsch1.tex|, |cdocsch2.tex|, |cdocspt3.tex|, |cdocspt4.tex|,
|cdocsdrf.tex|, |cdocsfn1.tex|, |cdocsfn2.tex|.
Then copy the file |childdoc.def| to an appropriate directory of your \LaTeX{}
distribution, e.g.\ \textit{texmf-root}|/tex/latex/childdoc|.
\end{itemize}

%%%%%%%%%%%%%%%%%%%%%%%%%%%%%%%%%%%%%%%%%%%%%%%%%%%%%%%%%%%%%%%%%%%%%%%%%%%%%%%%
\subsection{Related CTAN Packages}

There are several other packages which offer a similar functionality:
%
\begin{itemize}
\item
The packages
\href{http://ctan.org/pkg/docmute}{\textsf{docmute}},
\href{http://ctan.org/pkg/includex}{\textsf{includex}} and
\href{http://ctan.org/pkg/standalone}{\textsf{standalone}}
provide commands to include only the document body of
a child file thus allowing both files to be compiled individually.
\item
The packages \href{http://ctan.org/pkg/subdocs}{\textsf{subdocs}}
and \href{http://ctan.org/pkg/subfiles}{\textsf{subfiles}}
provide structures in which the main and child documents can be
encapsulated and allowing them to be compiled individually.
The inclusion mechanism is different from the conventional |\include|.
\item
The package \href{http://ctan.org/pkg/combine}{\textsf{combine}}
is an elaborate solution to combine several documents into one.
\end{itemize}
%
See also the CTAN topic \href{http://ctan.org/topic/subdocs}{\textsf{subdocs}}
for further related packages.
The present package differs from the above solutions in that
a document structure constructed with the conventional |\include| mechanism
just needs two extra commands at the top of every file
such that all constituent files can be compiled individually.

%%%%%%%%%%%%%%%%%%%%%%%%%%%%%%%%%%%%%%%%%%%%%%%%%%%%%%%%%%%%%%%%%%%%%%%%%%%%%%%%
%\subsection{Feature Suggestions}
%
%The following is a list of features which may be useful for future
%versions of this package:
%%
%\begin{itemize}
%\item
%\ldots
%\end{itemize}

%%%%%%%%%%%%%%%%%%%%%%%%%%%%%%%%%%%%%%%%%%%%%%%%%%%%%%%%%%%%%%%%%%%%%%%%%%%%%%%%
\subsection{Revision History}

%%%%%%%%%%%%%%%%%%%%%%%%%%%%%%%%%%%%%%%%
\paragraph{v2.0:} 2018/12/30

\begin{itemize}
\item
immediate forward processing
\item
added |\childdocby| mechanism
\item
manual restructured
\end{itemize}

%%%%%%%%%%%%%%%%%%%%%%%%%%%%%%%%%%%%%%%%
\paragraph{v1.6:} 2018/01/17

\begin{itemize}
\item
application for development of include files
\item
corrections to manual
\end{itemize}

%%%%%%%%%%%%%%%%%%%%%%%%%%%%%%%%%%%%%%%%
\paragraph{v1.5:} 2017/05/21

\begin{itemize}
\item
more complete structuring introduced
\item
|\childdocof| introduced
\item
|\childdoc| renamed to |\childdocmain|
\item
|\childredirect| renamed to |\childdocforward| and |\childdocforwardprefix|
and functionality expanded
\end{itemize}

%%%%%%%%%%%%%%%%%%%%%%%%%%%%%%%%%%%%%%%%
\paragraph{v1.0:} 2017/04/27

\begin{itemize}
\item
manual and install package
\item
first version published on CTAN
\end{itemize}

%%%%%%%%%%%%%%%%%%%%%%%%%%%%%%%%%%%%%%%%
\paragraph{v0.6:} 2017/04/26

\begin{itemize}
\item
redirection mechanism added
\end{itemize}

%%%%%%%%%%%%%%%%%%%%%%%%%%%%%%%%%%%%%%%%
\paragraph{v0.5:} 2017/04/26

\begin{itemize}
\item
functionality in definition file
\end{itemize}


%%%%%%%%%%%%%%%%%%%%%%%%%%%%%%%%%%%%%%%%%%%%%%%%%%%%%%%%%%%%%%%%%%%%%%%%%%%%%%%%
%%%%%%%%%%%%%%%%%%%%%%%%%%%%%%%%%%%%%%%%%%%%%%%%%%%%%%%%%%%%%%%%%%%%%%%%%%%%%%%%
%%%%%%%%%%%%%%%%%%%%%%%%%%%%%%%%%%%%%%%%%%%%%%%%%%%%%%%%%%%%%%%%%%%%%%%%%%%%%%%%
\appendix

\settowidth\MacroIndent{\rmfamily\scriptsize 000\ }

 \DocInput{childdoc.dtx}

\end{document}
%</driver>
% \fi
%
% %%%%%%%%%%%%%%%%%%%%%%%%%%%%%%%%%%%%%%%%%%%%%%%%%%%%%%%%%%%%%%%%%%%%%%%%%%%%%%
% %%%%%%%%%%%%%%%%%%%%%%%%%%%%%%%%%%%%%%%%%%%%%%%%%%%%%%%%%%%%%%%%%%%%%%%%%%%%%%
% \section{Sample}
%\iffalse
%<*samplemain>
%\fi
%
% The following presents a sample document
% with two chapters, two parts, a title page,
% a compile flag as well as three forwarding files to set the flag.
% It consists of eight |.tex| files:
% \begin{center}
% \begin{tabular}{ll}
% |cdocsamp.tex|&main file\\
% |cdocsch1.tex|&include file for chapter 1\\
% |cdocsch2.tex|&include file for chapter 2\\
% |cdocspt3.tex|&include file for part 3\\
% |cdocspt4.tex|&include file for part 4\\
% |cdocsdrf.tex|&forwarding file for main file in draft mode\\
% |cdocsfi1.tex|&forwarding file for final version of chapter 1\\
% |cdocsfi2.tex|&forwarding file for final version of chapter 2\\
% \end{tabular}
% \end{center}
% Each of the eight files can be compiled directly by the \LaTeX{} compiler.
%
% %%%%%%%%%%%%%%%%%%%%%%%%%%%%%%%%%%%%%%
% \paragraph{Main File.}
%
% The main file is called |cdocsamp.tex|.
%
% Load the \textsf{childdoc} definitions and
% declare the filename for the main document:
%    \begin{macrocode}
\input{childdoc.def}
\childdocmain{}
%    \end{macrocode}

% Optional override for |\version| flag:
%    \begin{macrocode}
%%\ifchilddoc\else\providecommand{\version}{draft}\fi
%    \end{macrocode}

% Define the default values for the |\version| flag
% (|final| for the main file and |draft| for childs):
%    \begin{macrocode}
\ifchilddoc
\providecommand{\version}{draft}
\else
\providecommand{\version}{final}
\fi
%    \end{macrocode}

% Load the standard document class:
%    \begin{macrocode}
\documentclass[12pt]{article}
%    \end{macrocode}

% Start the document body:
%    \begin{macrocode}
\begin{document}
%    \end{macrocode}

% Declare a title page.
% Print title, part of document being processed and version flag:
%    \begin{macrocode}
\addtocounter{page}{-1}
\begin{center}
{\LARGE\bfseries{}childdoc example\par}
\vspace{1cm}
\ifchilddoc
\ifchilddocmanual part\else chapter\fi:
`\childdocname' of `\childdocjob'\par
\else
main document: `\childdocjob'\par
\fi
version: \version\par
\end{center}
\newpage
%    \end{macrocode}

% Manually include selected file,
% otherwise process as usual:
%    \begin{macrocode}
\ifchilddocmanual
\section*{part `\childdocname'}
\input{\childdocname}
\else
%    \end{macrocode}

% Include the two chapters:
%    \begin{macrocode}
\include{cdocsch1}
\include{cdocsch2}
%    \end{macrocode}

% Include the two parts unless only chapters should be displayed:
%    \begin{macrocode}
\ifchilddoc\else
\section{part three}
\input{cdocspt3}
\section{part four}
\input{cdocspt4}
\fi
%    \end{macrocode}

% Process as usual until here:
%    \begin{macrocode}
\fi
%    \end{macrocode}

% End of document body:
%    \begin{macrocode}
\end{document}
%    \end{macrocode}
%\iffalse
%</samplemain>
%\fi
%
% %%%%%%%%%%%%%%%%%%%%%%%%%%%%%%%%%%%%%%
% \paragraph{Chapter Include Files.}
%
% The include files are called |cdocsch1.tex| and |cdocsch2.tex|.
%
%\iffalse
%<*samplechap1|samplechap2>
%\fi

% Optional override for |\version| flag:
%    \begin{macrocode}
%%\providecommand{\version}{final}
%    \end{macrocode}

% Include the main document:
%    \begin{macrocode}
\input{childdoc.def}
\childdocof{cdocsamp}
%    \end{macrocode}

%\iffalse
%</samplechap1|samplechap2>
%\fi
%
%\iffalse
%<*samplechap1>
%\fi
% Some text for chapter 1:
%    \begin{macrocode}
\section{one}
some text in chapter one
%    \end{macrocode}

%\iffalse
%</samplechap1>
%\fi
% Some text for chapter 2:
%\iffalse
%<*samplechap2>
%\fi
%    \begin{macrocode}
\section{two}
more text in chapter two
%    \end{macrocode}

%\iffalse
%</samplechap2>
%\fi
%
% %%%%%%%%%%%%%%%%%%%%%%%%%%%%%%%%%%%%%%
% \paragraph{Part Include Files.}
%
% The include files are called |cdocspt3.tex| and |cdocspt4.tex|.
%
%\iffalse
%<*samplepart3|samplepart4>
%\fi

% Optional override for |\version| flag:
%    \begin{macrocode}
%%\providecommand{\version}{final}
%    \end{macrocode}

% Include the main document:
%    \begin{macrocode}
\input{childdoc.def}
\childdocby{cdocsamp}
%    \end{macrocode}

%\iffalse
%</samplepart3|samplepart4>
%\fi
%
%\iffalse
%<*samplepart3>
%\fi
% Some text for part 3:
%    \begin{macrocode}
some text in part three
%    \end{macrocode}

%\iffalse
%</samplepart3>
%\fi
% Some text for part 4:
%\iffalse
%<*samplepart4>
%\fi
%    \begin{macrocode}
more text in part four
%    \end{macrocode}

%\iffalse
%</samplepart4>
%\fi
%
% %%%%%%%%%%%%%%%%%%%%%%%%%%%%%%%%%%%%%%
% \paragraph{Forwarding for a Complete Draft.}
%
% The following forwarding file |cdocsdrf.tex|
% compiles the main document in draft mode:
%\iffalse
%<*sampledraft>
%\fi
%    \begin{macrocode}
\def\version{draft}
\input{childdoc.def}
\childdocforward{cdocsamp}
%    \end{macrocode}

%\iffalse
%</sampledraft>
%\fi
%
% %%%%%%%%%%%%%%%%%%%%%%%%%%%%%%%%%%%%%%
% \paragraph{Forwarding for Final Version of the Chapters.}
%
% The following forwarding files |cdocsfn1.tex| and |cdocsfn2.tex|
% (with identical content)
% compile the final versions of the child documents
% |cdocsch1.tex| and |cdocsch2.tex|, respectively:
%\iffalse
%<*samplefinal>
%\fi
%    \begin{macrocode}
\def\version{final}
\input{childdoc.def}
\childdocforwardprefix[cdocsamp]{cdocsfn}{cdocsch}
%    \end{macrocode}

%\iffalse
%</samplefinal>
%\fi
%
% %%%%%%%%%%%%%%%%%%%%%%%%%%%%%%%%%%%%%%
% \paragraph{Command Line Processing.}
%
% The following three command lines generate the output files
% |cdocscld|, |cdocscl1| and |cdocscl2|
% which should be identical to
% |cdocsdrf|, |cdocsch1| and |cdocsfn2|, respectively:
% \begin{center}
% \begin{tabular}{l}
% |latex -jobname cdocscld \|\\
% |  "\def\version{draft}\input{childdoc.def}\childdocforward{cdocsamp}"|\\
% |latex -jobname cdocscl1 \|\\
% |  "\input{childdoc.def}\childdocforward[cdocsamp]{cdocsch1}"|\\
% |latex -jobname cdocscl2 \|\\
% |  "\def\version{final}\input{childdoc.def}\childdocforward{cdocsch2}"|
% \end{tabular}
% \end{center}
% Note that the trailing backslash on each first line
% merely continues the input to the second line
% (for convenient cut ant paste).
% Furthermore, the command |latex| can be replaced by any
% of its alternative versions such as |pdflatex|.
%
% %%%%%%%%%%%%%%%%%%%%%%%%%%%%%%%%%%%%%%%%%%%%%%%%%%%%%%%%%%%%%%%%%%%%%%%%%%%%%%
% %%%%%%%%%%%%%%%%%%%%%%%%%%%%%%%%%%%%%%%%%%%%%%%%%%%%%%%%%%%%%%%%%%%%%%%%%%%%%%
% \section{Implementation}
%\iffalse
%<*package>
%\fi
%
% This section describes the definitions file |childdoc.def|.

% The definitions cannot be loaded using |\usepackage| or |\RequirePackage|
% which has a mechanism to prevent loading a style file more than once.
% When loading the definitions by means of |\input|
% multiple instances have to be prevented manually:
%\iffalse
%This code needs to be before the `\ProvidesFile' directive
%which is defined at the beginning of this file.
%Therefore it is also placed there and commented out here.
%</package>
%<*discard>
%\fi
%    \begin{macrocode}
\ifdefined\childdocmain\endinput\fi
%    \end{macrocode}
%\iffalse
%</discard>
%<*package>
%\fi
%
% \macro{\ifchilddoc}
% \macro{\ifchilddocmanual}
% The conditional |\ifchilddoc| tells whether a
% child (true) or main (false) document is being compiled.
% The conditional |\ifchilddocmanual| tells whether
% the |\includeonly| mechanism is used (false) or
% the selection of child files must be performed manually (true).
% The definitions initialise to false:
%    \begin{macrocode}
\newif\ifchilddoc
\newif\ifchilddocmanual
%    \end{macrocode}

% \macro{\childdocname}
% \macro{\childdocjob}
% The macro |\childdocname| stores the name of the main document
% to be compiled. The macro |\childdocjob| stores the name of
% the document on which the \LaTeX{} compiler was originally invoked.
% The content of |\jobname| cannot be compared
% to filenames specified in the source due to different catcodes.
% The following code rescans |\jobname|, stores the result
% in |\childdocname| and saves a copy in |\childdocjob|:
%    \begin{macrocode}
\edef\childdocname{\scantokens\expandafter{\jobname\noexpand}}
\let\childdocjob\childdocname
%    \end{macrocode}

% \macro{\childdocdisable}
% The macro |\childdocdisable| prevents the main file
% from being processed more than once.
% At this stage, the main document command |\childdocmain|
% is assumed to be called once again where it should do nothing.
% Any subsequent call to it should prevent
% a secondary processing of the main document
% It overwrites the forwarding commands
% |\childdocof| and |\childdocforward|
% with empty macros to prevent further inclusions of the main document:
%    \begin{macrocode}
\newcommand{\childdocdisable}
{
  \renewcommand{\childdocmain}[1]{\renewcommand{\childdocmain}[1]{\endinput}}
  \renewcommand{\childdocof}[1]{}
  \renewcommand{\childdocby}[2][]{}
  \renewcommand{\childdocforward}[2][]{}
  \renewcommand{\childdocdisable}{}
}
%    \end{macrocode}

% \macro{\childdocmain}
% The macro |\childdocmain| is to be called at the top of the main file
% with nothing or the main filename (without extension) as argument.
% First, it breaks loops.
% If the argument is not empty and does not match |\childdocname|
% (which is set by the first inclusion of |childdoc.def|),
% |\ifchilddoc| is set to true, |\includeonly| is applied to the child file
% and |\jobname| is set to the main file
% (for proper handling of |.aux| files):
%    \begin{macrocode}
\newcommand{\childdocmain}[1]
{
  \childdocdisable\childdocmain{}
  \if?#1?\else
    \begingroup
      \def\childdoctmp{#1}
      \ifx\childdoctmp\childdocname
        \def\childdoctmp{}
      \else
        \def\childdoctmp
        {
          \childdoctrue
          \includeonly{\childdocname}
          \def\childdocjob{#1}
          \def\jobname{#1}
        }
      \fi
      \expandafter
    \endgroup
    \childdoctmp
  \fi
}
%    \end{macrocode}

% \macro{\childdocof}
% The command |\childdocof| redirects
% compilation to the main file |#1|.
%    \begin{macrocode}
\newcommand{\childdocof}[1]
{
  \childdocdisable
  \childdoctrue
  \includeonly{\childdocname}
  \def\jobname{#1}
  \def\childdocjob{#1}
  \input{#1}
}
%    \end{macrocode}

% \macro{\childdocby}
% The command |\childdocby| ....
%    \begin{macrocode}
\newcommand{\childdocby}[2][]
{
  \childdocdisable
  \childdoctrue
  \childdocmanualtrue
  \if?#1?\else
    \def\jobname{#2}
  \fi
  \def\childdocjob{#2}
  \input{#2}
  \endinput
}
%    \end{macrocode}

% \macro{\childdocforward}
% The command |\childdocforward| redirects
% compilation to the main file or
% (if the optional argument is given) a child file.
% Parameters are set as if the main file
% or a child file starting with |\childdocof| was compiled.
% Then compilation is handed over to the main file:
%    \begin{macrocode}
\newcommand{\childdocforward}[2][]
{
  \begingroup
    \if?#1?
      \def\childdoctmp
      {
        \def\childdocname{#2}
        \def\childdocjob{#2}
        \def\jobname{#2}
        \input{#2}
        \endinput
      }
    \else
      \def\childdoctmp
      {
        \childdocdisable
        \def\childdocname{#2}
        \childdoctrue
        \includeonly{#2}
        \def\childdocjob{#1}
        \def\jobname{#1}
        \input{#1}
        \endinput
      }
    \fi
    \expandafter
  \endgroup
  \childdoctmp
}
%    \end{macrocode}

% \macro{\childdocforwardprefix}
% The command |\childdocforwardprefix| redirects
% compilation to the main or a child file by means of a pattern.
% The prefix |#1| in the current filename is replaced by |#2|
% and the suffix of the current filename is kept
% (it is assumed that the filename does not contain the substring `|~~~|'
% which is used as a delimiter).
% Compilation is handed over to the new file by |\childdocforward|:
%    \begin{macrocode}
\newcommand{\childdocforwardprefix}[3][]
{
  \begingroup
    \def\childdocextract #2##1~~~{\def\childdoctmp{\childdocforward[#1]{#3##1}}}
    \expandafter\childdocextract\childdocname~~~
    \expandafter
  \endgroup
  \childdoctmp
}
%    \end{macrocode}

% \macro{\childdoc}
% The deprecated macro |\childdoc| is a legacy version of |\childdocmain|:
%    \begin{macrocode}
\newcommand{\childdoc}{\childdocmain}
%    \end{macrocode}

% \macro{\childdocredirect}
% The deprecated macro |\childdocredirect| is a legacy version
% of |\childdocforward| and |\childdocforwardprefix|:
%    \begin{macrocode}
\newcommand{\childdocredirect}[2][]
{
  \begingroup
    \if?#1?
      \def\childdoctmp{\childdocforward{#2}}
    \else
      \def\childdoctmp{\childdocforwardprefix{#1}{#2}}
    \fi
    \expandafter
  \endgroup
  \childdoctmp
}
%    \end{macrocode}

%\iffalse
%</package>
%\fi
%
\endinput
\childdocforward{cdocsch2}"|
% \end{tabular}
% \end{center}
% Note that the trailing backslash on each first line
% merely continues the input to the second line
% (for convenient cut ant paste).
% Furthermore, the command |latex| can be replaced by any
% of its alternative versions such as |pdflatex|.
%
% %%%%%%%%%%%%%%%%%%%%%%%%%%%%%%%%%%%%%%%%%%%%%%%%%%%%%%%%%%%%%%%%%%%%%%%%%%%%%%
% %%%%%%%%%%%%%%%%%%%%%%%%%%%%%%%%%%%%%%%%%%%%%%%%%%%%%%%%%%%%%%%%%%%%%%%%%%%%%%
% \section{Implementation}
%\iffalse
%<*package>
%\fi
%
% This section describes the definitions file |childdoc.def|.

% The definitions cannot be loaded using |\usepackage| or |\RequirePackage|
% which has a mechanism to prevent loading a style file more than once.
% When loading the definitions by means of |\input|
% multiple instances have to be prevented manually:
%\iffalse
%This code needs to be before the `\ProvidesFile' directive
%which is defined at the beginning of this file.
%Therefore it is also placed there and commented out here.
%</package>
%<*discard>
%\fi
%    \begin{macrocode}
\ifdefined\childdocmain\endinput\fi
%    \end{macrocode}
%\iffalse
%</discard>
%<*package>
%\fi
%
% \macro{\ifchilddoc}
% \macro{\ifchilddocmanual}
% The conditional |\ifchilddoc| tells whether a
% child (true) or main (false) document is being compiled.
% The conditional |\ifchilddocmanual| tells whether
% the |\includeonly| mechanism is used (false) or
% the selection of child files must be performed manually (true).
% The definitions initialise to false:
%    \begin{macrocode}
\newif\ifchilddoc
\newif\ifchilddocmanual
%    \end{macrocode}

% \macro{\childdocname}
% \macro{\childdocjob}
% The macro |\childdocname| stores the name of the main document
% to be compiled. The macro |\childdocjob| stores the name of
% the document on which the \LaTeX{} compiler was originally invoked.
% The content of |\jobname| cannot be compared
% to filenames specified in the source due to different catcodes.
% The following code rescans |\jobname|, stores the result
% in |\childdocname| and saves a copy in |\childdocjob|:
%    \begin{macrocode}
\edef\childdocname{\scantokens\expandafter{\jobname\noexpand}}
\let\childdocjob\childdocname
%    \end{macrocode}

% \macro{\childdocdisable}
% The macro |\childdocdisable| prevents the main file
% from being processed more than once.
% At this stage, the main document command |\childdocmain|
% is assumed to be called once again where it should do nothing.
% Any subsequent call to it should prevent
% a secondary processing of the main document
% It overwrites the forwarding commands
% |\childdocof| and |\childdocforward|
% with empty macros to prevent further inclusions of the main document:
%    \begin{macrocode}
\newcommand{\childdocdisable}
{
  \renewcommand{\childdocmain}[1]{\renewcommand{\childdocmain}[1]{\endinput}}
  \renewcommand{\childdocof}[1]{}
  \renewcommand{\childdocby}[2][]{}
  \renewcommand{\childdocforward}[2][]{}
  \renewcommand{\childdocdisable}{}
}
%    \end{macrocode}

% \macro{\childdocmain}
% The macro |\childdocmain| is to be called at the top of the main file
% with nothing or the main filename (without extension) as argument.
% First, it breaks loops.
% If the argument is not empty and does not match |\childdocname|
% (which is set by the first inclusion of |childdoc.def|),
% |\ifchilddoc| is set to true, |\includeonly| is applied to the child file
% and |\jobname| is set to the main file
% (for proper handling of |.aux| files):
%    \begin{macrocode}
\newcommand{\childdocmain}[1]
{
  \childdocdisable\childdocmain{}
  \if?#1?\else
    \begingroup
      \def\childdoctmp{#1}
      \ifx\childdoctmp\childdocname
        \def\childdoctmp{}
      \else
        \def\childdoctmp
        {
          \childdoctrue
          \includeonly{\childdocname}
          \def\childdocjob{#1}
          \def\jobname{#1}
        }
      \fi
      \expandafter
    \endgroup
    \childdoctmp
  \fi
}
%    \end{macrocode}

% \macro{\childdocof}
% The command |\childdocof| redirects
% compilation to the main file |#1|.
%    \begin{macrocode}
\newcommand{\childdocof}[1]
{
  \childdocdisable
  \childdoctrue
  \includeonly{\childdocname}
  \def\jobname{#1}
  \def\childdocjob{#1}
  \input{#1}
}
%    \end{macrocode}

% \macro{\childdocby}
% The command |\childdocby| ....
%    \begin{macrocode}
\newcommand{\childdocby}[2][]
{
  \childdocdisable
  \childdoctrue
  \childdocmanualtrue
  \if?#1?\else
    \def\jobname{#2}
  \fi
  \def\childdocjob{#2}
  \input{#2}
  \endinput
}
%    \end{macrocode}

% \macro{\childdocforward}
% The command |\childdocforward| redirects
% compilation to the main file or
% (if the optional argument is given) a child file.
% Parameters are set as if the main file
% or a child file starting with |\childdocof| was compiled.
% Then compilation is handed over to the main file:
%    \begin{macrocode}
\newcommand{\childdocforward}[2][]
{
  \begingroup
    \if?#1?
      \def\childdoctmp
      {
        \def\childdocname{#2}
        \def\childdocjob{#2}
        \def\jobname{#2}
        \input{#2}
        \endinput
      }
    \else
      \def\childdoctmp
      {
        \childdocdisable
        \def\childdocname{#2}
        \childdoctrue
        \includeonly{#2}
        \def\childdocjob{#1}
        \def\jobname{#1}
        \input{#1}
        \endinput
      }
    \fi
    \expandafter
  \endgroup
  \childdoctmp
}
%    \end{macrocode}

% \macro{\childdocforwardprefix}
% The command |\childdocforwardprefix| redirects
% compilation to the main or a child file by means of a pattern.
% The prefix |#1| in the current filename is replaced by |#2|
% and the suffix of the current filename is kept
% (it is assumed that the filename does not contain the substring `|~~~|'
% which is used as a delimiter).
% Compilation is handed over to the new file by |\childdocforward|:
%    \begin{macrocode}
\newcommand{\childdocforwardprefix}[3][]
{
  \begingroup
    \def\childdocextract #2##1~~~{\def\childdoctmp{\childdocforward[#1]{#3##1}}}
    \expandafter\childdocextract\childdocname~~~
    \expandafter
  \endgroup
  \childdoctmp
}
%    \end{macrocode}

% \macro{\childdoc}
% The deprecated macro |\childdoc| is a legacy version of |\childdocmain|:
%    \begin{macrocode}
\newcommand{\childdoc}{\childdocmain}
%    \end{macrocode}

% \macro{\childdocredirect}
% The deprecated macro |\childdocredirect| is a legacy version
% of |\childdocforward| and |\childdocforwardprefix|:
%    \begin{macrocode}
\newcommand{\childdocredirect}[2][]
{
  \begingroup
    \if?#1?
      \def\childdoctmp{\childdocforward{#2}}
    \else
      \def\childdoctmp{\childdocforwardprefix{#1}{#2}}
    \fi
    \expandafter
  \endgroup
  \childdoctmp
}
%    \end{macrocode}

%\iffalse
%</package>
%\fi
%
\endinput
|\\
|\childdocforward{|\textit{main}|}|
\end{tabular}
\end{center}
%
Likewise, the following files |final|\textit{nn}|.tex|
compile the final version of the child document
|child|\textit{nn}|.tex|:
%
\begin{center}
\begin{tabular}{l}
|\def\version{final}|\\
|% \iffalse
%
% childdoc.dtx Copyright (C) 2017-2018 Niklas Beisert
%
% This work may be distributed and/or modified under the
% conditions of the LaTeX Project Public License, either version 1.3
% of this license or (at your option) any later version.
% The latest version of this license is in
%   http://www.latex-project.org/lppl.txt
% and version 1.3 or later is part of all distributions of LaTeX
% version 2005/12/01 or later.
%
% This work has the LPPL maintenance status `maintained'.
%
% The Current Maintainer of this work is Niklas Beisert.
%
% This work consists of the files childdoc.dtx and childdoc.ins
% and the derived files childdoc.def and cdocsamp.tex with
% cdocsch1.tex, cdocsch2.tex, cdocsdrf.tex, cdocsfn1.tex, cdocsfn2.tex.
%
%<package>\ifdefined\childdocmain\endinput\fi
%<package>\ProvidesFile{childdoc.def}[2018/12/30 v2.0 child document driver]
%<samplemain>\ProvidesFile{cdocsamp.tex}[2018/12/30 v2.0 sample for childdoc]
%<*driver>
%\ProvidesFile{childdoc.drv}[2018/12/30 v2.0 childdoc reference manual file]
\PassOptionsToClass{10pt,a4paper}{article}
\documentclass{ltxdoc}

\usepackage[margin=35mm]{geometry}
\usepackage{hyperref}
\usepackage{hyperxmp}
\usepackage[usenames]{color}

\hypersetup{colorlinks=true}
\hypersetup{pdfstartview=FitH}
\hypersetup{pdfpagemode=UseNone}
\hypersetup{pdfsource={}}
\hypersetup{pdflang={en-UK}}
\hypersetup{pdfcopyright={Copyright 2017-2018 Niklas Beisert.
  This work may be distributed and/or modified under the
  conditions of the LaTeX Project Public License, either version 1.3
  of this license or (at your option) any later version.}}
\hypersetup{pdflicenseurl={http://www.latex-project.org/lppl.txt}}
\hypersetup{pdfcontactaddress={ETH Zurich, ITP, HIT K,
  Wolfgang-Pauli-Strasse 27}}
\hypersetup{pdfcontactpostcode={8093}}
\hypersetup{pdfcontactcity={Zurich}}
\hypersetup{pdfcontactcountry={Switzerland}}
\hypersetup{pdfcontactemail={nbeisert@itp.phys.ethz.ch}}
\hypersetup{pdfcontacturl={http://people.phys.ethz.ch/\xmptilde nbeisert/}}

\newcommand{\secref}[1]{\hyperref[#1]{section \ref*{#1}}}

\parskip1ex
\parindent0pt
\let\olditemize\itemize
\def\itemize{\olditemize\parskip0pt}

\begin{document}

\title{The \textsf{childdoc} Package}
\hypersetup{pdftitle={The childdoc Package}}
\author{Niklas Beisert\\[2ex]
  Institut f\"ur Theoretische Physik\\
  Eidgen\"ossische Technische Hochschule Z\"urich\\
  Wolfgang-Pauli-Strasse 27, 8093 Z\"urich, Switzerland\\[1ex]
  \href{mailto:nbeisert@itp.phys.ethz.ch}
  {\texttt{nbeisert@itp.phys.ethz.ch}}}
\hypersetup{pdfauthor={Niklas Beisert}}
\hypersetup{pdfsubject={Manual for the LaTeX2e Package childdoc}}
\date{30 December 2018, \textsf{v2.0}}
\maketitle

\begin{abstract}\noindent
\textsf{childdoc} is a \LaTeXe{} package
that enables the direct compilation
of document sections included by |\include|
to individual files.
\end{abstract}

\begingroup
\parskip0ex
\tableofcontents
\endgroup

%%%%%%%%%%%%%%%%%%%%%%%%%%%%%%%%%%%%%%%%%%%%%%%%%%%%%%%%%%%%%%%%%%%%%%%%%%%%%%%%
%%%%%%%%%%%%%%%%%%%%%%%%%%%%%%%%%%%%%%%%%%%%%%%%%%%%%%%%%%%%%%%%%%%%%%%%%%%%%%%%
\section{Introduction}

\LaTeX{} provides a mechanism to structure a large document (such as a book)
into a main file and several child files (containing the chapters)
using the |\include| command.
This mechanism is beneficial for documents
which span hundreds of pages in order to
make the source file(s) more manageable.
Moreover, compilation can be restricted to
selected child files by means of the |\includeonly| command.
The latter feature can be used to reduce the compilation time while editing
(this was significantly more useful in the earlier days of \LaTeX{})
or to generate a smaller document which is easier to navigate.
Another application of |\includeonly| is to generate
documents consisting of selected parts of the complete document.

However, there are a few drawbacks of the plain |\include| mechanism:
\begin{itemize}
\item
The child files cannot be compiled on their own,
they can only be compiled via the main file.
A naive editing environment
(such as a text editor with an option
to have the current file processed by \LaTeX)
may require one to switch to the main file before compiling;
attempting to compile the child file produces errors.
\item
The main file must be modified (each time)
to adjust the |\includeonly| command
to the present needs. This easily leaves the main file in a messy state.
\item
The generated document will always carry the filename
of the main document. This is inconvenient if
several child files are to be compiled and
to be kept for distribution.
\end{itemize}

The present package provides a simple interface
to make child files individually compilable by \LaTeX{}.
Compiling a child file then has the same effect as compiling
the main file with an |\includeonly| command
to select the appropriate child.
Moreover the generated document will carry the name of the child
rather than the main file.
This resolves all three above issues.

This feature is meant to make the editing of books,
thesis documents and lecture notes somewhat more convenient.
However, the package can also be used efficiently for
composing a series of documents (such as exercise sheets)
which are typically distributed individually.
It then assists the author in generating the individual documents
(potentially in different versions)
as well as a document containing the collected series.
Another application is in developing style files
or other kinds of included material
where compilation of the style file could redirect
to a sample or test file.

%%%%%%%%%%%%%%%%%%%%%%%%%%%%%%%%%%%%%%%%%%%%%%%%%%%%%%%%%%%%%%%%%%%%%%%%%%%%%%%%
%%%%%%%%%%%%%%%%%%%%%%%%%%%%%%%%%%%%%%%%%%%%%%%%%%%%%%%%%%%%%%%%%%%%%%%%%%%%%%%%
\section{Usage}

First of all, the package \textsf{childdoc} is \emph{not} a standard
\LaTeXe{} |.sty| style file! Therefore it needs to be invoked in
a non-standard way.

%%%%%%%%%%%%%%%%%%%%%%%%%%%%%%%%%%%%%%%%%%%%%%%%%%%%%%%%%%%%%%%%%%%%%%%%%%%%%%%%
\subsection{Included Files}
\label{sec:include}

%%%%%%%%%%%%%%%%%%%%%%%%%%%%%%%%%%%%%%%%
\DescribeMacro{\childdocmain}
To use the package, add the commands
\begin{center}
\begin{tabular}{l}
|% \iffalse
%
% childdoc.dtx Copyright (C) 2017-2018 Niklas Beisert
%
% This work may be distributed and/or modified under the
% conditions of the LaTeX Project Public License, either version 1.3
% of this license or (at your option) any later version.
% The latest version of this license is in
%   http://www.latex-project.org/lppl.txt
% and version 1.3 or later is part of all distributions of LaTeX
% version 2005/12/01 or later.
%
% This work has the LPPL maintenance status `maintained'.
%
% The Current Maintainer of this work is Niklas Beisert.
%
% This work consists of the files childdoc.dtx and childdoc.ins
% and the derived files childdoc.def and cdocsamp.tex with
% cdocsch1.tex, cdocsch2.tex, cdocsdrf.tex, cdocsfn1.tex, cdocsfn2.tex.
%
%<package>\ifdefined\childdocmain\endinput\fi
%<package>\ProvidesFile{childdoc.def}[2018/12/30 v2.0 child document driver]
%<samplemain>\ProvidesFile{cdocsamp.tex}[2018/12/30 v2.0 sample for childdoc]
%<*driver>
%\ProvidesFile{childdoc.drv}[2018/12/30 v2.0 childdoc reference manual file]
\PassOptionsToClass{10pt,a4paper}{article}
\documentclass{ltxdoc}

\usepackage[margin=35mm]{geometry}
\usepackage{hyperref}
\usepackage{hyperxmp}
\usepackage[usenames]{color}

\hypersetup{colorlinks=true}
\hypersetup{pdfstartview=FitH}
\hypersetup{pdfpagemode=UseNone}
\hypersetup{pdfsource={}}
\hypersetup{pdflang={en-UK}}
\hypersetup{pdfcopyright={Copyright 2017-2018 Niklas Beisert.
  This work may be distributed and/or modified under the
  conditions of the LaTeX Project Public License, either version 1.3
  of this license or (at your option) any later version.}}
\hypersetup{pdflicenseurl={http://www.latex-project.org/lppl.txt}}
\hypersetup{pdfcontactaddress={ETH Zurich, ITP, HIT K,
  Wolfgang-Pauli-Strasse 27}}
\hypersetup{pdfcontactpostcode={8093}}
\hypersetup{pdfcontactcity={Zurich}}
\hypersetup{pdfcontactcountry={Switzerland}}
\hypersetup{pdfcontactemail={nbeisert@itp.phys.ethz.ch}}
\hypersetup{pdfcontacturl={http://people.phys.ethz.ch/\xmptilde nbeisert/}}

\newcommand{\secref}[1]{\hyperref[#1]{section \ref*{#1}}}

\parskip1ex
\parindent0pt
\let\olditemize\itemize
\def\itemize{\olditemize\parskip0pt}

\begin{document}

\title{The \textsf{childdoc} Package}
\hypersetup{pdftitle={The childdoc Package}}
\author{Niklas Beisert\\[2ex]
  Institut f\"ur Theoretische Physik\\
  Eidgen\"ossische Technische Hochschule Z\"urich\\
  Wolfgang-Pauli-Strasse 27, 8093 Z\"urich, Switzerland\\[1ex]
  \href{mailto:nbeisert@itp.phys.ethz.ch}
  {\texttt{nbeisert@itp.phys.ethz.ch}}}
\hypersetup{pdfauthor={Niklas Beisert}}
\hypersetup{pdfsubject={Manual for the LaTeX2e Package childdoc}}
\date{30 December 2018, \textsf{v2.0}}
\maketitle

\begin{abstract}\noindent
\textsf{childdoc} is a \LaTeXe{} package
that enables the direct compilation
of document sections included by |\include|
to individual files.
\end{abstract}

\begingroup
\parskip0ex
\tableofcontents
\endgroup

%%%%%%%%%%%%%%%%%%%%%%%%%%%%%%%%%%%%%%%%%%%%%%%%%%%%%%%%%%%%%%%%%%%%%%%%%%%%%%%%
%%%%%%%%%%%%%%%%%%%%%%%%%%%%%%%%%%%%%%%%%%%%%%%%%%%%%%%%%%%%%%%%%%%%%%%%%%%%%%%%
\section{Introduction}

\LaTeX{} provides a mechanism to structure a large document (such as a book)
into a main file and several child files (containing the chapters)
using the |\include| command.
This mechanism is beneficial for documents
which span hundreds of pages in order to
make the source file(s) more manageable.
Moreover, compilation can be restricted to
selected child files by means of the |\includeonly| command.
The latter feature can be used to reduce the compilation time while editing
(this was significantly more useful in the earlier days of \LaTeX{})
or to generate a smaller document which is easier to navigate.
Another application of |\includeonly| is to generate
documents consisting of selected parts of the complete document.

However, there are a few drawbacks of the plain |\include| mechanism:
\begin{itemize}
\item
The child files cannot be compiled on their own,
they can only be compiled via the main file.
A naive editing environment
(such as a text editor with an option
to have the current file processed by \LaTeX)
may require one to switch to the main file before compiling;
attempting to compile the child file produces errors.
\item
The main file must be modified (each time)
to adjust the |\includeonly| command
to the present needs. This easily leaves the main file in a messy state.
\item
The generated document will always carry the filename
of the main document. This is inconvenient if
several child files are to be compiled and
to be kept for distribution.
\end{itemize}

The present package provides a simple interface
to make child files individually compilable by \LaTeX{}.
Compiling a child file then has the same effect as compiling
the main file with an |\includeonly| command
to select the appropriate child.
Moreover the generated document will carry the name of the child
rather than the main file.
This resolves all three above issues.

This feature is meant to make the editing of books,
thesis documents and lecture notes somewhat more convenient.
However, the package can also be used efficiently for
composing a series of documents (such as exercise sheets)
which are typically distributed individually.
It then assists the author in generating the individual documents
(potentially in different versions)
as well as a document containing the collected series.
Another application is in developing style files
or other kinds of included material
where compilation of the style file could redirect
to a sample or test file.

%%%%%%%%%%%%%%%%%%%%%%%%%%%%%%%%%%%%%%%%%%%%%%%%%%%%%%%%%%%%%%%%%%%%%%%%%%%%%%%%
%%%%%%%%%%%%%%%%%%%%%%%%%%%%%%%%%%%%%%%%%%%%%%%%%%%%%%%%%%%%%%%%%%%%%%%%%%%%%%%%
\section{Usage}

First of all, the package \textsf{childdoc} is \emph{not} a standard
\LaTeXe{} |.sty| style file! Therefore it needs to be invoked in
a non-standard way.

%%%%%%%%%%%%%%%%%%%%%%%%%%%%%%%%%%%%%%%%%%%%%%%%%%%%%%%%%%%%%%%%%%%%%%%%%%%%%%%%
\subsection{Included Files}
\label{sec:include}

%%%%%%%%%%%%%%%%%%%%%%%%%%%%%%%%%%%%%%%%
\DescribeMacro{\childdocmain}
To use the package, add the commands
\begin{center}
\begin{tabular}{l}
|\input{childdoc.def}|\\
|\childdocmain{}|\\
\end{tabular}
\end{center}
at the very top of the main \LaTeX{} file,
in particular \emph{before} the |\documentclass| statement!
The argument of |\childdocmain| should be left empty
(but it must be present).

%%%%%%%%%%%%%%%%%%%%%%%%%%%%%%%%%%%%%%%%
\DescribeMacro{\childdocof}
Furthermore, add the commands
\begin{center}
\begin{tabular}{l}
|\input{childdoc.def}|\\
|\childdocof{|\textit{main}|}|\\
\end{tabular}
\end{center}
at the top of every child file \textit{child}
which is included by |\include{|\textit{child}|}|
from within the main file
(or at least for those files to be compiled individually).
The argument \textit{main} must be the filename of the main file.

There are a couple of
considerations in setting up the main and child documents:

%%%%%%%%%%%%%%%%%%%%%%%%%%%%%%%%%%%%%%%%
\paragraph{Restrictions.}

Please note the following restrictions:
\begin{itemize}
\item
|\childdocmain| must be called with one argument \textit{main}
to ensure compatibility with earlier version of the package.
It must either be empty (|\childdocmain{}|)
or precisely match the filename of the main file in which it is specified.
See \secref{sec:detection} for further information.
\item
The filename \textit{main} must be specified without the |.tex| extension.
\item
The filename \textit{main} is case sensitive
(even in case-insensitive file systems)
due to internal string comparison.
\item
The argument \textit{main} should be fully expanded, it cannot be a macro.
\item
Subdirectories and special characters should be avoided in filenames.
\item
The command |\childdocmain{|\textit{main}|}| must be followed by a whitespace.
It should not be followed immediately by another command
or by a comment mark `|%|'.
This is because the \TeX{} parser reads the token immediately following
the argument of |\childdocmain| and puts it
at the beginning of every child section;
however, a white\-space is ignored.
\end{itemize}

%%%%%%%%%%%%%%%%%%%%%%%%%%%%%%%%%%%%%%%%
\paragraph{Content of Main File.}

It is advisable to place all content in the child files included by |\include|.
Any output contained in the main file will appear in all child documents
unless suppressed manually;
it cannot be suppressed automatically by the |\includeonly| directive
and thus should normally be avoided.
A method to include some content in the main file
by means of conditional processing is described in \secref{sec:conditional}.

%%%%%%%%%%%%%%%%%%%%%%%%%%%%%%%%%%%%%%%%
\paragraph{Page Numbering.}

When only a part of the document is compiled,
the appropriate numbering of pages
(as well as other status parameters)
is determined from the |.aux| files.
The latter contain information from previous passes.
However this information needs to propagate through
all intermediate child documents.
Therefore the page numbering in child documents may well
be inconsistent until the complete document is compiled at least once.

A useful (if unconventional) way to always ensure a consistent
page numbering is to restart the numbering in each child document
and denote the pages by `\textit{child}|.|\textit{page}'
where \textit{child} represents the chapter/section number of the child file.
This can be achieved by the command
|\numberwithin{page}{|\textit{child}|}|
of the \textsf{amsmath} package
where \textit{child} can be |chapter| or |section|
depending on the chosen structuring.
Alternatively, one can modify the macro |\thepage| appropriately
and reset the counter |page| at the start of each child file.

%%%%%%%%%%%%%%%%%%%%%%%%%%%%%%%%%%%%%%%%%%%%%%%%%%%%%%%%%%%%%%%%%%%%%%%%%%%%%%%%
\subsection{Conditional Processing}
\label{sec:conditional}

The package provides a mechanism to compile different versions
of a document. To customise the versions further some conditional processing
can come in handy to distinguish which version is being compiled.
The package provides two macros to describe the compilation context:

%%%%%%%%%%%%%%%%%%%%%%%%%%%%%%%%%%%%%%%%
\DescribeMacro{\ifchilddoc}
The conditional |\ifchilddoc| distinguishes between the compilation of
child documents and the main document:
%
\begin{center}
|\ifchilddoc |\textit{child-code}| |[|\||else |\textit{main-code}]| \||fi|
\end{center}

%%%%%%%%%%%%%%%%%%%%%%%%%%%%%%%%%%%%%%%%
\DescribeMacro{\childdocname}
\DescribeMacro{\childdocjob}
The macro |\childdocname| contains the filename (without extension)
of the main or child file being processed.
Note that |\childdocjob| will always contain the name of the main file.

%%%%%%%%%%%%%%%%%%%%%%%%%%%%%%%%%%%%%%%%
\paragraph{Title Page.}

Conditional processing can be used to include a title or banner page
in the main document when proper precautions are taken.
Importantly, the code in the main file should ensure that the page counter
(as well as other status parameters which are stored in the |.aux| files)
takes the same value after the conditional processing.
Otherwise the page numbers may take divergent values
depending on which part is compiled.

For example, a title page could be declared by:
%
\begin{center}
\begin{tabular}{l}
|\ifchilddoc\||else|\\
|\addtocounter{page}{-1}|\\
\textit{code for title page}\\
|\newpage|\\
|\||fi|
\end{tabular}
\end{center}
%
A banner page for the child documents can be generated by:
%
\begin{center}
\begin{tabular}{l}
|\ifchilddoc|\\
|\addtocounter{page}{-1}|\\
\textit{code for banner page}\\
|\newpage|\\
|\||fi|
\end{tabular}
\end{center}
%
Here one could write a message such as:
\begin{center}
|This is the part \childdocname{} of \childdocjob{}.|
\end{center}

%%%%%%%%%%%%%%%%%%%%%%%%%%%%%%%%%%%%%%%%%%%%%%%%%%%%%%%%%%%%%%%%%%%%%%%%%%%%%%%%
\subsection{Flags}
\label{sec:flags}

The package makes it easy to generate different versions
of the main or child documents.
To this end compilation flags can be defined
and assigned different default values.
They will be particularly useful in conjunction
with the forwarding mechanism described in \secref{sec:forward}.

For example, it may be useful to have a flag |\version|
which can be set to |draft| or |final|.
The document source will contain some conditional code
depending on the value of |\version|.
Suppose further, the flag should default to |final| for the main file
and to |draft| for child files
which is a natural assignment for editing the document.
This is achieved by placing the following code
in the preamble of the main document
(below the |\childdocmain| directive):
%
\begin{center}
\begin{tabular}{l}
|\ifchilddoc|\\
|\providecommand{\version}{draft}|\\
|\||else|\\
|\providecommand{\version}{final}|\\
|\||fi|
\end{tabular}
\end{center}
%
The definition by |\providecommand| makes sure
that previous definitions are not overwritten.
Further statements |\providecommand{\version}{...}|
can thus be added before the above code to override it.

For the main file, one might add a line
(between |\childdocmain| and the above block)
%
\begin{center}
|%\ifchilddoc\||else\providecommand{\version}{draft}\||fi|
\end{center}
%
which can be uncommented to produce a draft version.
Likewise one can add a line to the very top of a child file
(above the |\childdocof{|\textit{main}|}| directive)
%
\begin{center}
|%\providecommand{\version}{final}|
\end{center}
%
which can be uncommented to produce the final version of this child document.

%%%%%%%%%%%%%%%%%%%%%%%%%%%%%%%%%%%%%%%%%%%%%%%%%%%%%%%%%%%%%%%%%%%%%%%%%%%%%%%%
\subsection{Forwarding}
\label{sec:forward}

Different versions of the main or child documents
using compilation flags as described in \secref{sec:flags}
can be (permanently) stored in different files
for convenient compilation, viewing and distribution.
To this end, the package defines a command
to pass on compilation to a different file:

%%%%%%%%%%%%%%%%%%%%%%%%%%%%%%%%%%%%%%%%
\DescribeMacro{\childdocforward}
The command |\childdocforward| redirects processing to
another source file:
%
\begin{center}
\begin{tabular}{l}
|\input{childdoc.def}|\\
|\childdocforward[|\textit{main}|]{|\textit{dest}|}|\\
\end{tabular}
\end{center}
%
The argument \textit{dest} is the destination file
(without extension).
It should be the main file or one of the child files.
Note that further \textsf{childdoc} directives
such as |\childdocof| and |\childdocforward|
in the indicated file will be processed in this form.
The optional argument \textit{main}
passes on directly to the main file \textit{main}
while pretending to compile the child \textit{dest}.
This form behaves as if \textit{dest}
issues |\childdocof{|\textit{main}|}| right away,
and no further \textsf{childdoc} directives will be processed.

%%%%%%%%%%%%%%%%%%%%%%%%%%%%%%%%%%%%%%%%
\DescribeMacro{\...prefix}
In the alternative form |\childdocforwardprefix|,
%
\begin{center}
\begin{tabular}{l}
|\input{childdoc.def}|\\
|\childdocforwardprefix[|\textit{main}|]{|\textit{prefix}|}{|\textit{dest}|}|
\end{tabular}
\end{center}
%
the destination file is determined by a pattern
depending on the current file:
To make this work, the current file must be called
`{\textit{prefix}\hspace{0.2em}\textit{suffix}}'
with \textit{prefix} matching precisely the argument.
Processing is then passed on to the file
`{\textit{dest}\hspace{0.2em}\textit{suffix}}'.
Surely, the same effect is achieved by
directly specifying the
argument `{\textit{dest}\hspace{0.2em}\textit{suffix}}'
in the first form.
However, that requires to set up a different file
for each child. With the alternative form of the command
all these files can have exactly the same content
which simplifies setting them up and maintaining them.

For example, the following file |draft.tex|
with a compilation flag |\version| as described in \secref{sec:flags}
compiles the main document as a draft:
%
\begin{center}
\begin{tabular}{l}
|\def\version{draft}|\\
|\input{childdoc.def}|\\
|\childdocforward{|\textit{main}|}|
\end{tabular}
\end{center}
%
Likewise, the following files |final|\textit{nn}|.tex|
compile the final version of the child document
|child|\textit{nn}|.tex|:
%
\begin{center}
\begin{tabular}{l}
|\def\version{final}|\\
|\input{childdoc.def}|\\
|\childdocforwardprefix{final}{child}|
\end{tabular}
\end{center}
%

Note that when several versions of a main file and/or of each child file
are to be generated, it may be convenient to set up a |Makefile| or
shell script to automatise the process.

%%%%%%%%%%%%%%%%%%%%%%%%%%%%%%%%%%%%%%%%%%%%%%%%%%%%%%%%%%%%%%%%%%%%%%%%%%%%%%%%
\subsection{Command Line Processing}
\label{sec:commandline}

The effect of redirection files can also be achieved by invoking
the \LaTeX{} compiler with a more elaborate command line.
Most conveniently this should be done as part
of a shell script or a |Makefile|.

When using \textsf{childdoc} in the main file, the following
command lines effectively perform a redirection
(note that depending on the shell being used,
backslashes may have to be doubled: `|\|' $\to$ `|\\|'):
%
\begin{center}
|... -jobname "|\textit{target}|" |\\|"|[\textit{flags}]%
|\input{childdoc.def}\childdocforward[|\textit{main}|]{|\textit{dest}|}"|
\end{center}
%
Here \textit{target} is the name of the output file,
\textit{main} is the name of the main file
and \textit{dest} is the name of the main or child file to be processed
(all filenames without extensions).
The optional argument \textit{main} can be omitted
if \textit{main} matches \textit{dest}.
Optionally, compilation \textit{flags} can be defined via |\def| commands.
This command line makes the \TeX{} engine believe
it is compiling the file \textit{target}
whose content is specified as the latter parameter.
The provided code then forwards the processing to
\textit{main} or \textit{dest} as described in \secref{sec:forward}.

%%%%%%%%%%%%%%%%%%%%%%%%%%%%%%%%%%%%%%%%%%%%%%%%%%%%%%%%%%%%%%%%%%%%%%%%%%%%%%%%
\subsection{Include by Input}
\label{sec:input}

Including child documents by |\include| has some restrictions by design.
Most notably, the content of a child document always occupies
its own set of pages; pages cannot be shared between child documents.
Usually, this behaviour makes perfect sense
because each child document contain an essential part of the document.
However, in some situations it may be desirable to compose
a document from a collection of parts
without having mandatory page breaks between then.
For this case, the package
provides a mechanism to include parts
by |\input| which can also be processed individually.
However, by construction this mechanism
requires manual handling of the content to be output.

%%%%%%%%%%%%%%%%%%%%%%%%%%%%%%%%%%%%%%%%
\DescribeMacro{\ifchilddocmanual}
The main file should be prepared as usual, see \secref{sec:include}.
However, the document body must make a distinction
between processing of an individual part and of the main document, e.g.:
%
\begin{center}
\begin{tabular}{l}
|\ifchilddocmanual|\\
|\input{\childdocname}|\\
|\||else|\\
\textit{document body with }|\input{|\textit{part}|}|\\
|\||fi|
\end{tabular}
\end{center}
%
The conditional |\ifchilddocmanual| is true whenever
a part to be included by |\input| is being compiled,
and the name of the part is stored in |\childdocname|.

%%%%%%%%%%%%%%%%%%%%%%%%%%%%%%%%%%%%%%%%
\DescribeMacro{\childdocby}
Each part to be included by |\input| should start with:
%
\begin{center}
\begin{tabular}{l}
|\input{childdoc.def}|\\
|\childdocby{|\textit{main}|}|\\
\end{tabular}
\end{center}
%
The directive |\childdocby| is similar to |\childdocof|
described in \secref{sec:include},
but the subsequent selection of content must be done manually.
To that end, both |\ifchilddoc| and |\ifchilddocmanual|
will be true upon processing of a part,
and the name of the part is stored in |\childdocname|.
Note that |\jobname| will be set to the filename of the current part
so that each part receives an individual |.aux| file
that does not interfere with the |.aux| file(s) of the main document.
This behaviour can be altered by the alternative form
|\childdocby[*]{|\textit{main}|}| (with a non-empty optional argument)
which uses the |.aux| file of the main document
by setting |\jobname| to \textit{main}.

%%%%%%%%%%%%%%%%%%%%%%%%%%%%%%%%%%%%%%%%%%%%%%%%%%%%%%%%%%%%%%%%%%%%%%%%%%%%%%%%
\subsection{Driver Development}
\label{sec:driver}

The \textsf{childdoc} mechanism can also be use for the development
of definition files such as \LaTeX{} styles or classes.
This case differs from the above setup with multiple parts
included by |\include| in that no |\includeonly| should be invoked.
This can be achieved by starting the include file
(before |\ProvidesPackage|) with:
%
\begin{center}
\begin{tabular}{l}
|\input{childdoc.def}|\\
|\childdocforward{|\textit{main}|}|\\
\end{tabular}
\end{center}
%
or alternatively with:
%
\begin{center}
\begin{tabular}{l}
|\input{childdoc.def}|\\
|\childdocby{|\textit{main}|}|\\
\end{tabular}
\end{center}
%
Both forms have slightly different effects as described above.
The main file is prepared as usual, see \secref{sec:include}.

%%%%%%%%%%%%%%%%%%%%%%%%%%%%%%%%%%%%%%%%%%%%%%%%%%%%%%%%%%%%%%%%%%%%%%%%%%%%%%%%
\subsection{Legacy Detection}
\label{sec:detection}

The directive |\childdocmain| in the main file can detect
whether the complete document or merely a child is to be compiled
even without using the directive |\childdocof|.
This method is deprecated because it is less robust
and there is no compelling reason to use it;
it is merely provided for backward compatibility
and it may be removed in future versions.

If the detection mechanism is to be used,
it is mandatory to correctly specify
the filename of the main file as the argument of |\childdocmain|:
%
\begin{center}
\begin{tabular}{l}
|\input{childdoc.def}|\\
|\childdocmain{|\textit{main}|}|\\
\end{tabular}
\end{center}
%
If |\jobname| does not match the argument \textit{main} of |\childdocmain|,
it is assumed that |\jobname| points to the child file to be compiled.
When using |\childdocmain| with the main file specified as argument,
it suffices to start a child file
with just |\input{|\textit{main}|}|
without loading of the package and using |\childdocof|.
If instead all processing is done
with the appropriate \textsf{childdoc} directives,
the argument of \textit{main} of |\childdocmain| can be empty.

An alternative version of the command line processing described
in \secref{sec:commandline} using the detection mechanism reads:
%
\begin{center}
|... -jobname "|\textit{target}|" "|[\textit{flags}]%
[|\def\jobname{|\textit{dest}|}|]|\input{|\textit{main}|}"|
\end{center}

%%%%%%%%%%%%%%%%%%%%%%%%%%%%%%%%%%%%%%%%%%%%%%%%%%%%%%%%%%%%%%%%%%%%%%%%%%%%%%%%
\subsection{Manual Code}
\label{sec:manual}

In case one cannot be certain whether the definitions file |childdoc.def|
is installed on the target \TeX{} distribution
and one prefers not to ship it,
it is conceivable to paste a few relevant commands into the sources.

To that end, drop all statements |\input{childdoc.def}|
and perform the replacements as outlined below.
Instead of |\childdocmain{|\textit{main}|}| add the following code
to the top of the main file:
%
\begin{center}
\begin{tabular}{l}
|\||ifdefined\childdocname\endinput\||fi\newif\ifchilddoc|\\
|\edef\childdocname{\scantokens\expandafter{\jobname\noexpand}}|\\
|\def\childdocmain{|\textit{main}|}\||ifx\childdocmain\childdocname\||else|\\
|\childdoctrue\includeonly{\childdocname}\let\jobname\childdocmain\||fi|\\
\end{tabular}
\end{center}
%
Instead of |\childdocof{|\textit{main}|}| just include the main file
at the top of each child file:
%
\begin{center}
|\input{|\textit{main}|}|
\end{center}
%
A simple redirection |\childdocforward{|\textit{dest}|}| is achieved by:
%
\begin{center}
|\def\jobname{|\textit{dest}|}\input{\jobname}|
\end{center}
%
The redirection with prefix
|\childdocforwardprefix[|\textit{prefix}|]{|\textit{dest}|}|
is accomplished by:
%
\begin{center}
\begin{tabular}{l}
|{\edef\jobname{\scantokens\expandafter{\jobname\noexpand}}|\\
|\def\redirectjob |\textit{prefix}|#1~~~{\gdef\jobname{|\textit{dest}|#1}}|\\
|\expandafter\redirectjob\jobname~~~}\input{\jobname}|
\end{tabular}
\end{center}

In an alternative approach,
child documents can be compiled by a specific command line
without additional code or specific definitions:
%
\begin{center}
|... -jobname "|\textit{target}|" "|[\textit{flags}]%
|\includeonly{|\textit{dest}|}\input{|\textit{main}|}"|
\end{center}
%

%%%%%%%%%%%%%%%%%%%%%%%%%%%%%%%%%%%%%%%%%%%%%%%%%%%%%%%%%%%%%%%%%%%%%%%%%%%%%%%%
%%%%%%%%%%%%%%%%%%%%%%%%%%%%%%%%%%%%%%%%%%%%%%%%%%%%%%%%%%%%%%%%%%%%%%%%%%%%%%%%
\section{Information}

%%%%%%%%%%%%%%%%%%%%%%%%%%%%%%%%%%%%%%%%%%%%%%%%%%%%%%%%%%%%%%%%%%%%%%%%%%%%%%%%
\subsection{Copyright}

Copyright \copyright{} 2017--2018 Niklas Beisert

This work may be distributed and/or modified under the
conditions of the \LaTeX{} Project Public License, either version 1.3
of this license or (at your option) any later version.
The latest version of this license is in
  \url{http://www.latex-project.org/lppl.txt}
and version 1.3 or later is part of all distributions of \LaTeX{}
version 2005/12/01 or later.

This work has the LPPL maintenance status `maintained'.

The Current Maintainer of this work is Niklas Beisert.

This work consists of the files |README.txt|, |childdoc.ins| and |childdoc.dtx|
as well as the derived files |childdoc.def|, |cdocsamp.tex|
with |cdocsch1.tex|, |cdocsch2.tex|, |cdocspt3.tex|, |cdocspt4.tex|,
|cdocsdrf.tex|, |cdocsfn1.tex|, |cdocsfn2.tex|
as well as |childdoc.pdf|.

%%%%%%%%%%%%%%%%%%%%%%%%%%%%%%%%%%%%%%%%%%%%%%%%%%%%%%%%%%%%%%%%%%%%%%%%%%%%%%%%
\subsection{Files and Installation}

The package consists of the files:
%
\begin{center}
\begin{tabular}{ll}
    |README.txt|   & readme file \\
    |childdoc.ins| & installation file \\
    |childdoc.dtx| & source file \\
    |childdoc.def| & definition file \\
    |cdocsamp.tex| & sample main file \\
    |cdocsch1.tex| & sample include file \\
    |cdocsch2.tex| & sample include file \\
    |cdocspt3.tex| & sample part file \\
    |cdocspt4.tex| & sample part file \\
    |cdocsdrf.tex| & sample redirection file \\
    |cdocsfn1.tex| & sample redirection file \\
    |cdocsfn2.tex| & sample redirection file \\
    |childdoc.pdf| & manual
\end{tabular}
\end{center}
%
The distribution consists of the files
|README.txt|, |childdoc.ins| and |childdoc.dtx|.
%
\begin{itemize}
\item
Run (pdf)\LaTeX{} on |childdoc.dtx|
to compile the manual |childdoc.pdf| (this file).
\item
Run \LaTeX{} on |childdoc.ins| to create the definitions file |childdoc.def|
and the sample |cdocsamp.tex| with include files
|cdocsch1.tex|, |cdocsch2.tex|, |cdocspt3.tex|, |cdocspt4.tex|,
|cdocsdrf.tex|, |cdocsfn1.tex|, |cdocsfn2.tex|.
Then copy the file |childdoc.def| to an appropriate directory of your \LaTeX{}
distribution, e.g.\ \textit{texmf-root}|/tex/latex/childdoc|.
\end{itemize}

%%%%%%%%%%%%%%%%%%%%%%%%%%%%%%%%%%%%%%%%%%%%%%%%%%%%%%%%%%%%%%%%%%%%%%%%%%%%%%%%
\subsection{Related CTAN Packages}

There are several other packages which offer a similar functionality:
%
\begin{itemize}
\item
The packages
\href{http://ctan.org/pkg/docmute}{\textsf{docmute}},
\href{http://ctan.org/pkg/includex}{\textsf{includex}} and
\href{http://ctan.org/pkg/standalone}{\textsf{standalone}}
provide commands to include only the document body of
a child file thus allowing both files to be compiled individually.
\item
The packages \href{http://ctan.org/pkg/subdocs}{\textsf{subdocs}}
and \href{http://ctan.org/pkg/subfiles}{\textsf{subfiles}}
provide structures in which the main and child documents can be
encapsulated and allowing them to be compiled individually.
The inclusion mechanism is different from the conventional |\include|.
\item
The package \href{http://ctan.org/pkg/combine}{\textsf{combine}}
is an elaborate solution to combine several documents into one.
\end{itemize}
%
See also the CTAN topic \href{http://ctan.org/topic/subdocs}{\textsf{subdocs}}
for further related packages.
The present package differs from the above solutions in that
a document structure constructed with the conventional |\include| mechanism
just needs two extra commands at the top of every file
such that all constituent files can be compiled individually.

%%%%%%%%%%%%%%%%%%%%%%%%%%%%%%%%%%%%%%%%%%%%%%%%%%%%%%%%%%%%%%%%%%%%%%%%%%%%%%%%
%\subsection{Feature Suggestions}
%
%The following is a list of features which may be useful for future
%versions of this package:
%%
%\begin{itemize}
%\item
%\ldots
%\end{itemize}

%%%%%%%%%%%%%%%%%%%%%%%%%%%%%%%%%%%%%%%%%%%%%%%%%%%%%%%%%%%%%%%%%%%%%%%%%%%%%%%%
\subsection{Revision History}

%%%%%%%%%%%%%%%%%%%%%%%%%%%%%%%%%%%%%%%%
\paragraph{v2.0:} 2018/12/30

\begin{itemize}
\item
immediate forward processing
\item
added |\childdocby| mechanism
\item
manual restructured
\end{itemize}

%%%%%%%%%%%%%%%%%%%%%%%%%%%%%%%%%%%%%%%%
\paragraph{v1.6:} 2018/01/17

\begin{itemize}
\item
application for development of include files
\item
corrections to manual
\end{itemize}

%%%%%%%%%%%%%%%%%%%%%%%%%%%%%%%%%%%%%%%%
\paragraph{v1.5:} 2017/05/21

\begin{itemize}
\item
more complete structuring introduced
\item
|\childdocof| introduced
\item
|\childdoc| renamed to |\childdocmain|
\item
|\childredirect| renamed to |\childdocforward| and |\childdocforwardprefix|
and functionality expanded
\end{itemize}

%%%%%%%%%%%%%%%%%%%%%%%%%%%%%%%%%%%%%%%%
\paragraph{v1.0:} 2017/04/27

\begin{itemize}
\item
manual and install package
\item
first version published on CTAN
\end{itemize}

%%%%%%%%%%%%%%%%%%%%%%%%%%%%%%%%%%%%%%%%
\paragraph{v0.6:} 2017/04/26

\begin{itemize}
\item
redirection mechanism added
\end{itemize}

%%%%%%%%%%%%%%%%%%%%%%%%%%%%%%%%%%%%%%%%
\paragraph{v0.5:} 2017/04/26

\begin{itemize}
\item
functionality in definition file
\end{itemize}


%%%%%%%%%%%%%%%%%%%%%%%%%%%%%%%%%%%%%%%%%%%%%%%%%%%%%%%%%%%%%%%%%%%%%%%%%%%%%%%%
%%%%%%%%%%%%%%%%%%%%%%%%%%%%%%%%%%%%%%%%%%%%%%%%%%%%%%%%%%%%%%%%%%%%%%%%%%%%%%%%
%%%%%%%%%%%%%%%%%%%%%%%%%%%%%%%%%%%%%%%%%%%%%%%%%%%%%%%%%%%%%%%%%%%%%%%%%%%%%%%%
\appendix

\settowidth\MacroIndent{\rmfamily\scriptsize 000\ }

 \DocInput{childdoc.dtx}

\end{document}
%</driver>
% \fi
%
% %%%%%%%%%%%%%%%%%%%%%%%%%%%%%%%%%%%%%%%%%%%%%%%%%%%%%%%%%%%%%%%%%%%%%%%%%%%%%%
% %%%%%%%%%%%%%%%%%%%%%%%%%%%%%%%%%%%%%%%%%%%%%%%%%%%%%%%%%%%%%%%%%%%%%%%%%%%%%%
% \section{Sample}
%\iffalse
%<*samplemain>
%\fi
%
% The following presents a sample document
% with two chapters, two parts, a title page,
% a compile flag as well as three forwarding files to set the flag.
% It consists of eight |.tex| files:
% \begin{center}
% \begin{tabular}{ll}
% |cdocsamp.tex|&main file\\
% |cdocsch1.tex|&include file for chapter 1\\
% |cdocsch2.tex|&include file for chapter 2\\
% |cdocspt3.tex|&include file for part 3\\
% |cdocspt4.tex|&include file for part 4\\
% |cdocsdrf.tex|&forwarding file for main file in draft mode\\
% |cdocsfi1.tex|&forwarding file for final version of chapter 1\\
% |cdocsfi2.tex|&forwarding file for final version of chapter 2\\
% \end{tabular}
% \end{center}
% Each of the eight files can be compiled directly by the \LaTeX{} compiler.
%
% %%%%%%%%%%%%%%%%%%%%%%%%%%%%%%%%%%%%%%
% \paragraph{Main File.}
%
% The main file is called |cdocsamp.tex|.
%
% Load the \textsf{childdoc} definitions and
% declare the filename for the main document:
%    \begin{macrocode}
\input{childdoc.def}
\childdocmain{}
%    \end{macrocode}

% Optional override for |\version| flag:
%    \begin{macrocode}
%%\ifchilddoc\else\providecommand{\version}{draft}\fi
%    \end{macrocode}

% Define the default values for the |\version| flag
% (|final| for the main file and |draft| for childs):
%    \begin{macrocode}
\ifchilddoc
\providecommand{\version}{draft}
\else
\providecommand{\version}{final}
\fi
%    \end{macrocode}

% Load the standard document class:
%    \begin{macrocode}
\documentclass[12pt]{article}
%    \end{macrocode}

% Start the document body:
%    \begin{macrocode}
\begin{document}
%    \end{macrocode}

% Declare a title page.
% Print title, part of document being processed and version flag:
%    \begin{macrocode}
\addtocounter{page}{-1}
\begin{center}
{\LARGE\bfseries{}childdoc example\par}
\vspace{1cm}
\ifchilddoc
\ifchilddocmanual part\else chapter\fi:
`\childdocname' of `\childdocjob'\par
\else
main document: `\childdocjob'\par
\fi
version: \version\par
\end{center}
\newpage
%    \end{macrocode}

% Manually include selected file,
% otherwise process as usual:
%    \begin{macrocode}
\ifchilddocmanual
\section*{part `\childdocname'}
\input{\childdocname}
\else
%    \end{macrocode}

% Include the two chapters:
%    \begin{macrocode}
\include{cdocsch1}
\include{cdocsch2}
%    \end{macrocode}

% Include the two parts unless only chapters should be displayed:
%    \begin{macrocode}
\ifchilddoc\else
\section{part three}
\input{cdocspt3}
\section{part four}
\input{cdocspt4}
\fi
%    \end{macrocode}

% Process as usual until here:
%    \begin{macrocode}
\fi
%    \end{macrocode}

% End of document body:
%    \begin{macrocode}
\end{document}
%    \end{macrocode}
%\iffalse
%</samplemain>
%\fi
%
% %%%%%%%%%%%%%%%%%%%%%%%%%%%%%%%%%%%%%%
% \paragraph{Chapter Include Files.}
%
% The include files are called |cdocsch1.tex| and |cdocsch2.tex|.
%
%\iffalse
%<*samplechap1|samplechap2>
%\fi

% Optional override for |\version| flag:
%    \begin{macrocode}
%%\providecommand{\version}{final}
%    \end{macrocode}

% Include the main document:
%    \begin{macrocode}
\input{childdoc.def}
\childdocof{cdocsamp}
%    \end{macrocode}

%\iffalse
%</samplechap1|samplechap2>
%\fi
%
%\iffalse
%<*samplechap1>
%\fi
% Some text for chapter 1:
%    \begin{macrocode}
\section{one}
some text in chapter one
%    \end{macrocode}

%\iffalse
%</samplechap1>
%\fi
% Some text for chapter 2:
%\iffalse
%<*samplechap2>
%\fi
%    \begin{macrocode}
\section{two}
more text in chapter two
%    \end{macrocode}

%\iffalse
%</samplechap2>
%\fi
%
% %%%%%%%%%%%%%%%%%%%%%%%%%%%%%%%%%%%%%%
% \paragraph{Part Include Files.}
%
% The include files are called |cdocspt3.tex| and |cdocspt4.tex|.
%
%\iffalse
%<*samplepart3|samplepart4>
%\fi

% Optional override for |\version| flag:
%    \begin{macrocode}
%%\providecommand{\version}{final}
%    \end{macrocode}

% Include the main document:
%    \begin{macrocode}
\input{childdoc.def}
\childdocby{cdocsamp}
%    \end{macrocode}

%\iffalse
%</samplepart3|samplepart4>
%\fi
%
%\iffalse
%<*samplepart3>
%\fi
% Some text for part 3:
%    \begin{macrocode}
some text in part three
%    \end{macrocode}

%\iffalse
%</samplepart3>
%\fi
% Some text for part 4:
%\iffalse
%<*samplepart4>
%\fi
%    \begin{macrocode}
more text in part four
%    \end{macrocode}

%\iffalse
%</samplepart4>
%\fi
%
% %%%%%%%%%%%%%%%%%%%%%%%%%%%%%%%%%%%%%%
% \paragraph{Forwarding for a Complete Draft.}
%
% The following forwarding file |cdocsdrf.tex|
% compiles the main document in draft mode:
%\iffalse
%<*sampledraft>
%\fi
%    \begin{macrocode}
\def\version{draft}
\input{childdoc.def}
\childdocforward{cdocsamp}
%    \end{macrocode}

%\iffalse
%</sampledraft>
%\fi
%
% %%%%%%%%%%%%%%%%%%%%%%%%%%%%%%%%%%%%%%
% \paragraph{Forwarding for Final Version of the Chapters.}
%
% The following forwarding files |cdocsfn1.tex| and |cdocsfn2.tex|
% (with identical content)
% compile the final versions of the child documents
% |cdocsch1.tex| and |cdocsch2.tex|, respectively:
%\iffalse
%<*samplefinal>
%\fi
%    \begin{macrocode}
\def\version{final}
\input{childdoc.def}
\childdocforwardprefix[cdocsamp]{cdocsfn}{cdocsch}
%    \end{macrocode}

%\iffalse
%</samplefinal>
%\fi
%
% %%%%%%%%%%%%%%%%%%%%%%%%%%%%%%%%%%%%%%
% \paragraph{Command Line Processing.}
%
% The following three command lines generate the output files
% |cdocscld|, |cdocscl1| and |cdocscl2|
% which should be identical to
% |cdocsdrf|, |cdocsch1| and |cdocsfn2|, respectively:
% \begin{center}
% \begin{tabular}{l}
% |latex -jobname cdocscld \|\\
% |  "\def\version{draft}\input{childdoc.def}\childdocforward{cdocsamp}"|\\
% |latex -jobname cdocscl1 \|\\
% |  "\input{childdoc.def}\childdocforward[cdocsamp]{cdocsch1}"|\\
% |latex -jobname cdocscl2 \|\\
% |  "\def\version{final}\input{childdoc.def}\childdocforward{cdocsch2}"|
% \end{tabular}
% \end{center}
% Note that the trailing backslash on each first line
% merely continues the input to the second line
% (for convenient cut ant paste).
% Furthermore, the command |latex| can be replaced by any
% of its alternative versions such as |pdflatex|.
%
% %%%%%%%%%%%%%%%%%%%%%%%%%%%%%%%%%%%%%%%%%%%%%%%%%%%%%%%%%%%%%%%%%%%%%%%%%%%%%%
% %%%%%%%%%%%%%%%%%%%%%%%%%%%%%%%%%%%%%%%%%%%%%%%%%%%%%%%%%%%%%%%%%%%%%%%%%%%%%%
% \section{Implementation}
%\iffalse
%<*package>
%\fi
%
% This section describes the definitions file |childdoc.def|.

% The definitions cannot be loaded using |\usepackage| or |\RequirePackage|
% which has a mechanism to prevent loading a style file more than once.
% When loading the definitions by means of |\input|
% multiple instances have to be prevented manually:
%\iffalse
%This code needs to be before the `\ProvidesFile' directive
%which is defined at the beginning of this file.
%Therefore it is also placed there and commented out here.
%</package>
%<*discard>
%\fi
%    \begin{macrocode}
\ifdefined\childdocmain\endinput\fi
%    \end{macrocode}
%\iffalse
%</discard>
%<*package>
%\fi
%
% \macro{\ifchilddoc}
% \macro{\ifchilddocmanual}
% The conditional |\ifchilddoc| tells whether a
% child (true) or main (false) document is being compiled.
% The conditional |\ifchilddocmanual| tells whether
% the |\includeonly| mechanism is used (false) or
% the selection of child files must be performed manually (true).
% The definitions initialise to false:
%    \begin{macrocode}
\newif\ifchilddoc
\newif\ifchilddocmanual
%    \end{macrocode}

% \macro{\childdocname}
% \macro{\childdocjob}
% The macro |\childdocname| stores the name of the main document
% to be compiled. The macro |\childdocjob| stores the name of
% the document on which the \LaTeX{} compiler was originally invoked.
% The content of |\jobname| cannot be compared
% to filenames specified in the source due to different catcodes.
% The following code rescans |\jobname|, stores the result
% in |\childdocname| and saves a copy in |\childdocjob|:
%    \begin{macrocode}
\edef\childdocname{\scantokens\expandafter{\jobname\noexpand}}
\let\childdocjob\childdocname
%    \end{macrocode}

% \macro{\childdocdisable}
% The macro |\childdocdisable| prevents the main file
% from being processed more than once.
% At this stage, the main document command |\childdocmain|
% is assumed to be called once again where it should do nothing.
% Any subsequent call to it should prevent
% a secondary processing of the main document
% It overwrites the forwarding commands
% |\childdocof| and |\childdocforward|
% with empty macros to prevent further inclusions of the main document:
%    \begin{macrocode}
\newcommand{\childdocdisable}
{
  \renewcommand{\childdocmain}[1]{\renewcommand{\childdocmain}[1]{\endinput}}
  \renewcommand{\childdocof}[1]{}
  \renewcommand{\childdocby}[2][]{}
  \renewcommand{\childdocforward}[2][]{}
  \renewcommand{\childdocdisable}{}
}
%    \end{macrocode}

% \macro{\childdocmain}
% The macro |\childdocmain| is to be called at the top of the main file
% with nothing or the main filename (without extension) as argument.
% First, it breaks loops.
% If the argument is not empty and does not match |\childdocname|
% (which is set by the first inclusion of |childdoc.def|),
% |\ifchilddoc| is set to true, |\includeonly| is applied to the child file
% and |\jobname| is set to the main file
% (for proper handling of |.aux| files):
%    \begin{macrocode}
\newcommand{\childdocmain}[1]
{
  \childdocdisable\childdocmain{}
  \if?#1?\else
    \begingroup
      \def\childdoctmp{#1}
      \ifx\childdoctmp\childdocname
        \def\childdoctmp{}
      \else
        \def\childdoctmp
        {
          \childdoctrue
          \includeonly{\childdocname}
          \def\childdocjob{#1}
          \def\jobname{#1}
        }
      \fi
      \expandafter
    \endgroup
    \childdoctmp
  \fi
}
%    \end{macrocode}

% \macro{\childdocof}
% The command |\childdocof| redirects
% compilation to the main file |#1|.
%    \begin{macrocode}
\newcommand{\childdocof}[1]
{
  \childdocdisable
  \childdoctrue
  \includeonly{\childdocname}
  \def\jobname{#1}
  \def\childdocjob{#1}
  \input{#1}
}
%    \end{macrocode}

% \macro{\childdocby}
% The command |\childdocby| ....
%    \begin{macrocode}
\newcommand{\childdocby}[2][]
{
  \childdocdisable
  \childdoctrue
  \childdocmanualtrue
  \if?#1?\else
    \def\jobname{#2}
  \fi
  \def\childdocjob{#2}
  \input{#2}
  \endinput
}
%    \end{macrocode}

% \macro{\childdocforward}
% The command |\childdocforward| redirects
% compilation to the main file or
% (if the optional argument is given) a child file.
% Parameters are set as if the main file
% or a child file starting with |\childdocof| was compiled.
% Then compilation is handed over to the main file:
%    \begin{macrocode}
\newcommand{\childdocforward}[2][]
{
  \begingroup
    \if?#1?
      \def\childdoctmp
      {
        \def\childdocname{#2}
        \def\childdocjob{#2}
        \def\jobname{#2}
        \input{#2}
        \endinput
      }
    \else
      \def\childdoctmp
      {
        \childdocdisable
        \def\childdocname{#2}
        \childdoctrue
        \includeonly{#2}
        \def\childdocjob{#1}
        \def\jobname{#1}
        \input{#1}
        \endinput
      }
    \fi
    \expandafter
  \endgroup
  \childdoctmp
}
%    \end{macrocode}

% \macro{\childdocforwardprefix}
% The command |\childdocforwardprefix| redirects
% compilation to the main or a child file by means of a pattern.
% The prefix |#1| in the current filename is replaced by |#2|
% and the suffix of the current filename is kept
% (it is assumed that the filename does not contain the substring `|~~~|'
% which is used as a delimiter).
% Compilation is handed over to the new file by |\childdocforward|:
%    \begin{macrocode}
\newcommand{\childdocforwardprefix}[3][]
{
  \begingroup
    \def\childdocextract #2##1~~~{\def\childdoctmp{\childdocforward[#1]{#3##1}}}
    \expandafter\childdocextract\childdocname~~~
    \expandafter
  \endgroup
  \childdoctmp
}
%    \end{macrocode}

% \macro{\childdoc}
% The deprecated macro |\childdoc| is a legacy version of |\childdocmain|:
%    \begin{macrocode}
\newcommand{\childdoc}{\childdocmain}
%    \end{macrocode}

% \macro{\childdocredirect}
% The deprecated macro |\childdocredirect| is a legacy version
% of |\childdocforward| and |\childdocforwardprefix|:
%    \begin{macrocode}
\newcommand{\childdocredirect}[2][]
{
  \begingroup
    \if?#1?
      \def\childdoctmp{\childdocforward{#2}}
    \else
      \def\childdoctmp{\childdocforwardprefix{#1}{#2}}
    \fi
    \expandafter
  \endgroup
  \childdoctmp
}
%    \end{macrocode}

%\iffalse
%</package>
%\fi
%
\endinput
|\\
|\childdocmain{}|\\
\end{tabular}
\end{center}
at the very top of the main \LaTeX{} file,
in particular \emph{before} the |\documentclass| statement!
The argument of |\childdocmain| should be left empty
(but it must be present).

%%%%%%%%%%%%%%%%%%%%%%%%%%%%%%%%%%%%%%%%
\DescribeMacro{\childdocof}
Furthermore, add the commands
\begin{center}
\begin{tabular}{l}
|% \iffalse
%
% childdoc.dtx Copyright (C) 2017-2018 Niklas Beisert
%
% This work may be distributed and/or modified under the
% conditions of the LaTeX Project Public License, either version 1.3
% of this license or (at your option) any later version.
% The latest version of this license is in
%   http://www.latex-project.org/lppl.txt
% and version 1.3 or later is part of all distributions of LaTeX
% version 2005/12/01 or later.
%
% This work has the LPPL maintenance status `maintained'.
%
% The Current Maintainer of this work is Niklas Beisert.
%
% This work consists of the files childdoc.dtx and childdoc.ins
% and the derived files childdoc.def and cdocsamp.tex with
% cdocsch1.tex, cdocsch2.tex, cdocsdrf.tex, cdocsfn1.tex, cdocsfn2.tex.
%
%<package>\ifdefined\childdocmain\endinput\fi
%<package>\ProvidesFile{childdoc.def}[2018/12/30 v2.0 child document driver]
%<samplemain>\ProvidesFile{cdocsamp.tex}[2018/12/30 v2.0 sample for childdoc]
%<*driver>
%\ProvidesFile{childdoc.drv}[2018/12/30 v2.0 childdoc reference manual file]
\PassOptionsToClass{10pt,a4paper}{article}
\documentclass{ltxdoc}

\usepackage[margin=35mm]{geometry}
\usepackage{hyperref}
\usepackage{hyperxmp}
\usepackage[usenames]{color}

\hypersetup{colorlinks=true}
\hypersetup{pdfstartview=FitH}
\hypersetup{pdfpagemode=UseNone}
\hypersetup{pdfsource={}}
\hypersetup{pdflang={en-UK}}
\hypersetup{pdfcopyright={Copyright 2017-2018 Niklas Beisert.
  This work may be distributed and/or modified under the
  conditions of the LaTeX Project Public License, either version 1.3
  of this license or (at your option) any later version.}}
\hypersetup{pdflicenseurl={http://www.latex-project.org/lppl.txt}}
\hypersetup{pdfcontactaddress={ETH Zurich, ITP, HIT K,
  Wolfgang-Pauli-Strasse 27}}
\hypersetup{pdfcontactpostcode={8093}}
\hypersetup{pdfcontactcity={Zurich}}
\hypersetup{pdfcontactcountry={Switzerland}}
\hypersetup{pdfcontactemail={nbeisert@itp.phys.ethz.ch}}
\hypersetup{pdfcontacturl={http://people.phys.ethz.ch/\xmptilde nbeisert/}}

\newcommand{\secref}[1]{\hyperref[#1]{section \ref*{#1}}}

\parskip1ex
\parindent0pt
\let\olditemize\itemize
\def\itemize{\olditemize\parskip0pt}

\begin{document}

\title{The \textsf{childdoc} Package}
\hypersetup{pdftitle={The childdoc Package}}
\author{Niklas Beisert\\[2ex]
  Institut f\"ur Theoretische Physik\\
  Eidgen\"ossische Technische Hochschule Z\"urich\\
  Wolfgang-Pauli-Strasse 27, 8093 Z\"urich, Switzerland\\[1ex]
  \href{mailto:nbeisert@itp.phys.ethz.ch}
  {\texttt{nbeisert@itp.phys.ethz.ch}}}
\hypersetup{pdfauthor={Niklas Beisert}}
\hypersetup{pdfsubject={Manual for the LaTeX2e Package childdoc}}
\date{30 December 2018, \textsf{v2.0}}
\maketitle

\begin{abstract}\noindent
\textsf{childdoc} is a \LaTeXe{} package
that enables the direct compilation
of document sections included by |\include|
to individual files.
\end{abstract}

\begingroup
\parskip0ex
\tableofcontents
\endgroup

%%%%%%%%%%%%%%%%%%%%%%%%%%%%%%%%%%%%%%%%%%%%%%%%%%%%%%%%%%%%%%%%%%%%%%%%%%%%%%%%
%%%%%%%%%%%%%%%%%%%%%%%%%%%%%%%%%%%%%%%%%%%%%%%%%%%%%%%%%%%%%%%%%%%%%%%%%%%%%%%%
\section{Introduction}

\LaTeX{} provides a mechanism to structure a large document (such as a book)
into a main file and several child files (containing the chapters)
using the |\include| command.
This mechanism is beneficial for documents
which span hundreds of pages in order to
make the source file(s) more manageable.
Moreover, compilation can be restricted to
selected child files by means of the |\includeonly| command.
The latter feature can be used to reduce the compilation time while editing
(this was significantly more useful in the earlier days of \LaTeX{})
or to generate a smaller document which is easier to navigate.
Another application of |\includeonly| is to generate
documents consisting of selected parts of the complete document.

However, there are a few drawbacks of the plain |\include| mechanism:
\begin{itemize}
\item
The child files cannot be compiled on their own,
they can only be compiled via the main file.
A naive editing environment
(such as a text editor with an option
to have the current file processed by \LaTeX)
may require one to switch to the main file before compiling;
attempting to compile the child file produces errors.
\item
The main file must be modified (each time)
to adjust the |\includeonly| command
to the present needs. This easily leaves the main file in a messy state.
\item
The generated document will always carry the filename
of the main document. This is inconvenient if
several child files are to be compiled and
to be kept for distribution.
\end{itemize}

The present package provides a simple interface
to make child files individually compilable by \LaTeX{}.
Compiling a child file then has the same effect as compiling
the main file with an |\includeonly| command
to select the appropriate child.
Moreover the generated document will carry the name of the child
rather than the main file.
This resolves all three above issues.

This feature is meant to make the editing of books,
thesis documents and lecture notes somewhat more convenient.
However, the package can also be used efficiently for
composing a series of documents (such as exercise sheets)
which are typically distributed individually.
It then assists the author in generating the individual documents
(potentially in different versions)
as well as a document containing the collected series.
Another application is in developing style files
or other kinds of included material
where compilation of the style file could redirect
to a sample or test file.

%%%%%%%%%%%%%%%%%%%%%%%%%%%%%%%%%%%%%%%%%%%%%%%%%%%%%%%%%%%%%%%%%%%%%%%%%%%%%%%%
%%%%%%%%%%%%%%%%%%%%%%%%%%%%%%%%%%%%%%%%%%%%%%%%%%%%%%%%%%%%%%%%%%%%%%%%%%%%%%%%
\section{Usage}

First of all, the package \textsf{childdoc} is \emph{not} a standard
\LaTeXe{} |.sty| style file! Therefore it needs to be invoked in
a non-standard way.

%%%%%%%%%%%%%%%%%%%%%%%%%%%%%%%%%%%%%%%%%%%%%%%%%%%%%%%%%%%%%%%%%%%%%%%%%%%%%%%%
\subsection{Included Files}
\label{sec:include}

%%%%%%%%%%%%%%%%%%%%%%%%%%%%%%%%%%%%%%%%
\DescribeMacro{\childdocmain}
To use the package, add the commands
\begin{center}
\begin{tabular}{l}
|\input{childdoc.def}|\\
|\childdocmain{}|\\
\end{tabular}
\end{center}
at the very top of the main \LaTeX{} file,
in particular \emph{before} the |\documentclass| statement!
The argument of |\childdocmain| should be left empty
(but it must be present).

%%%%%%%%%%%%%%%%%%%%%%%%%%%%%%%%%%%%%%%%
\DescribeMacro{\childdocof}
Furthermore, add the commands
\begin{center}
\begin{tabular}{l}
|\input{childdoc.def}|\\
|\childdocof{|\textit{main}|}|\\
\end{tabular}
\end{center}
at the top of every child file \textit{child}
which is included by |\include{|\textit{child}|}|
from within the main file
(or at least for those files to be compiled individually).
The argument \textit{main} must be the filename of the main file.

There are a couple of
considerations in setting up the main and child documents:

%%%%%%%%%%%%%%%%%%%%%%%%%%%%%%%%%%%%%%%%
\paragraph{Restrictions.}

Please note the following restrictions:
\begin{itemize}
\item
|\childdocmain| must be called with one argument \textit{main}
to ensure compatibility with earlier version of the package.
It must either be empty (|\childdocmain{}|)
or precisely match the filename of the main file in which it is specified.
See \secref{sec:detection} for further information.
\item
The filename \textit{main} must be specified without the |.tex| extension.
\item
The filename \textit{main} is case sensitive
(even in case-insensitive file systems)
due to internal string comparison.
\item
The argument \textit{main} should be fully expanded, it cannot be a macro.
\item
Subdirectories and special characters should be avoided in filenames.
\item
The command |\childdocmain{|\textit{main}|}| must be followed by a whitespace.
It should not be followed immediately by another command
or by a comment mark `|%|'.
This is because the \TeX{} parser reads the token immediately following
the argument of |\childdocmain| and puts it
at the beginning of every child section;
however, a white\-space is ignored.
\end{itemize}

%%%%%%%%%%%%%%%%%%%%%%%%%%%%%%%%%%%%%%%%
\paragraph{Content of Main File.}

It is advisable to place all content in the child files included by |\include|.
Any output contained in the main file will appear in all child documents
unless suppressed manually;
it cannot be suppressed automatically by the |\includeonly| directive
and thus should normally be avoided.
A method to include some content in the main file
by means of conditional processing is described in \secref{sec:conditional}.

%%%%%%%%%%%%%%%%%%%%%%%%%%%%%%%%%%%%%%%%
\paragraph{Page Numbering.}

When only a part of the document is compiled,
the appropriate numbering of pages
(as well as other status parameters)
is determined from the |.aux| files.
The latter contain information from previous passes.
However this information needs to propagate through
all intermediate child documents.
Therefore the page numbering in child documents may well
be inconsistent until the complete document is compiled at least once.

A useful (if unconventional) way to always ensure a consistent
page numbering is to restart the numbering in each child document
and denote the pages by `\textit{child}|.|\textit{page}'
where \textit{child} represents the chapter/section number of the child file.
This can be achieved by the command
|\numberwithin{page}{|\textit{child}|}|
of the \textsf{amsmath} package
where \textit{child} can be |chapter| or |section|
depending on the chosen structuring.
Alternatively, one can modify the macro |\thepage| appropriately
and reset the counter |page| at the start of each child file.

%%%%%%%%%%%%%%%%%%%%%%%%%%%%%%%%%%%%%%%%%%%%%%%%%%%%%%%%%%%%%%%%%%%%%%%%%%%%%%%%
\subsection{Conditional Processing}
\label{sec:conditional}

The package provides a mechanism to compile different versions
of a document. To customise the versions further some conditional processing
can come in handy to distinguish which version is being compiled.
The package provides two macros to describe the compilation context:

%%%%%%%%%%%%%%%%%%%%%%%%%%%%%%%%%%%%%%%%
\DescribeMacro{\ifchilddoc}
The conditional |\ifchilddoc| distinguishes between the compilation of
child documents and the main document:
%
\begin{center}
|\ifchilddoc |\textit{child-code}| |[|\||else |\textit{main-code}]| \||fi|
\end{center}

%%%%%%%%%%%%%%%%%%%%%%%%%%%%%%%%%%%%%%%%
\DescribeMacro{\childdocname}
\DescribeMacro{\childdocjob}
The macro |\childdocname| contains the filename (without extension)
of the main or child file being processed.
Note that |\childdocjob| will always contain the name of the main file.

%%%%%%%%%%%%%%%%%%%%%%%%%%%%%%%%%%%%%%%%
\paragraph{Title Page.}

Conditional processing can be used to include a title or banner page
in the main document when proper precautions are taken.
Importantly, the code in the main file should ensure that the page counter
(as well as other status parameters which are stored in the |.aux| files)
takes the same value after the conditional processing.
Otherwise the page numbers may take divergent values
depending on which part is compiled.

For example, a title page could be declared by:
%
\begin{center}
\begin{tabular}{l}
|\ifchilddoc\||else|\\
|\addtocounter{page}{-1}|\\
\textit{code for title page}\\
|\newpage|\\
|\||fi|
\end{tabular}
\end{center}
%
A banner page for the child documents can be generated by:
%
\begin{center}
\begin{tabular}{l}
|\ifchilddoc|\\
|\addtocounter{page}{-1}|\\
\textit{code for banner page}\\
|\newpage|\\
|\||fi|
\end{tabular}
\end{center}
%
Here one could write a message such as:
\begin{center}
|This is the part \childdocname{} of \childdocjob{}.|
\end{center}

%%%%%%%%%%%%%%%%%%%%%%%%%%%%%%%%%%%%%%%%%%%%%%%%%%%%%%%%%%%%%%%%%%%%%%%%%%%%%%%%
\subsection{Flags}
\label{sec:flags}

The package makes it easy to generate different versions
of the main or child documents.
To this end compilation flags can be defined
and assigned different default values.
They will be particularly useful in conjunction
with the forwarding mechanism described in \secref{sec:forward}.

For example, it may be useful to have a flag |\version|
which can be set to |draft| or |final|.
The document source will contain some conditional code
depending on the value of |\version|.
Suppose further, the flag should default to |final| for the main file
and to |draft| for child files
which is a natural assignment for editing the document.
This is achieved by placing the following code
in the preamble of the main document
(below the |\childdocmain| directive):
%
\begin{center}
\begin{tabular}{l}
|\ifchilddoc|\\
|\providecommand{\version}{draft}|\\
|\||else|\\
|\providecommand{\version}{final}|\\
|\||fi|
\end{tabular}
\end{center}
%
The definition by |\providecommand| makes sure
that previous definitions are not overwritten.
Further statements |\providecommand{\version}{...}|
can thus be added before the above code to override it.

For the main file, one might add a line
(between |\childdocmain| and the above block)
%
\begin{center}
|%\ifchilddoc\||else\providecommand{\version}{draft}\||fi|
\end{center}
%
which can be uncommented to produce a draft version.
Likewise one can add a line to the very top of a child file
(above the |\childdocof{|\textit{main}|}| directive)
%
\begin{center}
|%\providecommand{\version}{final}|
\end{center}
%
which can be uncommented to produce the final version of this child document.

%%%%%%%%%%%%%%%%%%%%%%%%%%%%%%%%%%%%%%%%%%%%%%%%%%%%%%%%%%%%%%%%%%%%%%%%%%%%%%%%
\subsection{Forwarding}
\label{sec:forward}

Different versions of the main or child documents
using compilation flags as described in \secref{sec:flags}
can be (permanently) stored in different files
for convenient compilation, viewing and distribution.
To this end, the package defines a command
to pass on compilation to a different file:

%%%%%%%%%%%%%%%%%%%%%%%%%%%%%%%%%%%%%%%%
\DescribeMacro{\childdocforward}
The command |\childdocforward| redirects processing to
another source file:
%
\begin{center}
\begin{tabular}{l}
|\input{childdoc.def}|\\
|\childdocforward[|\textit{main}|]{|\textit{dest}|}|\\
\end{tabular}
\end{center}
%
The argument \textit{dest} is the destination file
(without extension).
It should be the main file or one of the child files.
Note that further \textsf{childdoc} directives
such as |\childdocof| and |\childdocforward|
in the indicated file will be processed in this form.
The optional argument \textit{main}
passes on directly to the main file \textit{main}
while pretending to compile the child \textit{dest}.
This form behaves as if \textit{dest}
issues |\childdocof{|\textit{main}|}| right away,
and no further \textsf{childdoc} directives will be processed.

%%%%%%%%%%%%%%%%%%%%%%%%%%%%%%%%%%%%%%%%
\DescribeMacro{\...prefix}
In the alternative form |\childdocforwardprefix|,
%
\begin{center}
\begin{tabular}{l}
|\input{childdoc.def}|\\
|\childdocforwardprefix[|\textit{main}|]{|\textit{prefix}|}{|\textit{dest}|}|
\end{tabular}
\end{center}
%
the destination file is determined by a pattern
depending on the current file:
To make this work, the current file must be called
`{\textit{prefix}\hspace{0.2em}\textit{suffix}}'
with \textit{prefix} matching precisely the argument.
Processing is then passed on to the file
`{\textit{dest}\hspace{0.2em}\textit{suffix}}'.
Surely, the same effect is achieved by
directly specifying the
argument `{\textit{dest}\hspace{0.2em}\textit{suffix}}'
in the first form.
However, that requires to set up a different file
for each child. With the alternative form of the command
all these files can have exactly the same content
which simplifies setting them up and maintaining them.

For example, the following file |draft.tex|
with a compilation flag |\version| as described in \secref{sec:flags}
compiles the main document as a draft:
%
\begin{center}
\begin{tabular}{l}
|\def\version{draft}|\\
|\input{childdoc.def}|\\
|\childdocforward{|\textit{main}|}|
\end{tabular}
\end{center}
%
Likewise, the following files |final|\textit{nn}|.tex|
compile the final version of the child document
|child|\textit{nn}|.tex|:
%
\begin{center}
\begin{tabular}{l}
|\def\version{final}|\\
|\input{childdoc.def}|\\
|\childdocforwardprefix{final}{child}|
\end{tabular}
\end{center}
%

Note that when several versions of a main file and/or of each child file
are to be generated, it may be convenient to set up a |Makefile| or
shell script to automatise the process.

%%%%%%%%%%%%%%%%%%%%%%%%%%%%%%%%%%%%%%%%%%%%%%%%%%%%%%%%%%%%%%%%%%%%%%%%%%%%%%%%
\subsection{Command Line Processing}
\label{sec:commandline}

The effect of redirection files can also be achieved by invoking
the \LaTeX{} compiler with a more elaborate command line.
Most conveniently this should be done as part
of a shell script or a |Makefile|.

When using \textsf{childdoc} in the main file, the following
command lines effectively perform a redirection
(note that depending on the shell being used,
backslashes may have to be doubled: `|\|' $\to$ `|\\|'):
%
\begin{center}
|... -jobname "|\textit{target}|" |\\|"|[\textit{flags}]%
|\input{childdoc.def}\childdocforward[|\textit{main}|]{|\textit{dest}|}"|
\end{center}
%
Here \textit{target} is the name of the output file,
\textit{main} is the name of the main file
and \textit{dest} is the name of the main or child file to be processed
(all filenames without extensions).
The optional argument \textit{main} can be omitted
if \textit{main} matches \textit{dest}.
Optionally, compilation \textit{flags} can be defined via |\def| commands.
This command line makes the \TeX{} engine believe
it is compiling the file \textit{target}
whose content is specified as the latter parameter.
The provided code then forwards the processing to
\textit{main} or \textit{dest} as described in \secref{sec:forward}.

%%%%%%%%%%%%%%%%%%%%%%%%%%%%%%%%%%%%%%%%%%%%%%%%%%%%%%%%%%%%%%%%%%%%%%%%%%%%%%%%
\subsection{Include by Input}
\label{sec:input}

Including child documents by |\include| has some restrictions by design.
Most notably, the content of a child document always occupies
its own set of pages; pages cannot be shared between child documents.
Usually, this behaviour makes perfect sense
because each child document contain an essential part of the document.
However, in some situations it may be desirable to compose
a document from a collection of parts
without having mandatory page breaks between then.
For this case, the package
provides a mechanism to include parts
by |\input| which can also be processed individually.
However, by construction this mechanism
requires manual handling of the content to be output.

%%%%%%%%%%%%%%%%%%%%%%%%%%%%%%%%%%%%%%%%
\DescribeMacro{\ifchilddocmanual}
The main file should be prepared as usual, see \secref{sec:include}.
However, the document body must make a distinction
between processing of an individual part and of the main document, e.g.:
%
\begin{center}
\begin{tabular}{l}
|\ifchilddocmanual|\\
|\input{\childdocname}|\\
|\||else|\\
\textit{document body with }|\input{|\textit{part}|}|\\
|\||fi|
\end{tabular}
\end{center}
%
The conditional |\ifchilddocmanual| is true whenever
a part to be included by |\input| is being compiled,
and the name of the part is stored in |\childdocname|.

%%%%%%%%%%%%%%%%%%%%%%%%%%%%%%%%%%%%%%%%
\DescribeMacro{\childdocby}
Each part to be included by |\input| should start with:
%
\begin{center}
\begin{tabular}{l}
|\input{childdoc.def}|\\
|\childdocby{|\textit{main}|}|\\
\end{tabular}
\end{center}
%
The directive |\childdocby| is similar to |\childdocof|
described in \secref{sec:include},
but the subsequent selection of content must be done manually.
To that end, both |\ifchilddoc| and |\ifchilddocmanual|
will be true upon processing of a part,
and the name of the part is stored in |\childdocname|.
Note that |\jobname| will be set to the filename of the current part
so that each part receives an individual |.aux| file
that does not interfere with the |.aux| file(s) of the main document.
This behaviour can be altered by the alternative form
|\childdocby[*]{|\textit{main}|}| (with a non-empty optional argument)
which uses the |.aux| file of the main document
by setting |\jobname| to \textit{main}.

%%%%%%%%%%%%%%%%%%%%%%%%%%%%%%%%%%%%%%%%%%%%%%%%%%%%%%%%%%%%%%%%%%%%%%%%%%%%%%%%
\subsection{Driver Development}
\label{sec:driver}

The \textsf{childdoc} mechanism can also be use for the development
of definition files such as \LaTeX{} styles or classes.
This case differs from the above setup with multiple parts
included by |\include| in that no |\includeonly| should be invoked.
This can be achieved by starting the include file
(before |\ProvidesPackage|) with:
%
\begin{center}
\begin{tabular}{l}
|\input{childdoc.def}|\\
|\childdocforward{|\textit{main}|}|\\
\end{tabular}
\end{center}
%
or alternatively with:
%
\begin{center}
\begin{tabular}{l}
|\input{childdoc.def}|\\
|\childdocby{|\textit{main}|}|\\
\end{tabular}
\end{center}
%
Both forms have slightly different effects as described above.
The main file is prepared as usual, see \secref{sec:include}.

%%%%%%%%%%%%%%%%%%%%%%%%%%%%%%%%%%%%%%%%%%%%%%%%%%%%%%%%%%%%%%%%%%%%%%%%%%%%%%%%
\subsection{Legacy Detection}
\label{sec:detection}

The directive |\childdocmain| in the main file can detect
whether the complete document or merely a child is to be compiled
even without using the directive |\childdocof|.
This method is deprecated because it is less robust
and there is no compelling reason to use it;
it is merely provided for backward compatibility
and it may be removed in future versions.

If the detection mechanism is to be used,
it is mandatory to correctly specify
the filename of the main file as the argument of |\childdocmain|:
%
\begin{center}
\begin{tabular}{l}
|\input{childdoc.def}|\\
|\childdocmain{|\textit{main}|}|\\
\end{tabular}
\end{center}
%
If |\jobname| does not match the argument \textit{main} of |\childdocmain|,
it is assumed that |\jobname| points to the child file to be compiled.
When using |\childdocmain| with the main file specified as argument,
it suffices to start a child file
with just |\input{|\textit{main}|}|
without loading of the package and using |\childdocof|.
If instead all processing is done
with the appropriate \textsf{childdoc} directives,
the argument of \textit{main} of |\childdocmain| can be empty.

An alternative version of the command line processing described
in \secref{sec:commandline} using the detection mechanism reads:
%
\begin{center}
|... -jobname "|\textit{target}|" "|[\textit{flags}]%
[|\def\jobname{|\textit{dest}|}|]|\input{|\textit{main}|}"|
\end{center}

%%%%%%%%%%%%%%%%%%%%%%%%%%%%%%%%%%%%%%%%%%%%%%%%%%%%%%%%%%%%%%%%%%%%%%%%%%%%%%%%
\subsection{Manual Code}
\label{sec:manual}

In case one cannot be certain whether the definitions file |childdoc.def|
is installed on the target \TeX{} distribution
and one prefers not to ship it,
it is conceivable to paste a few relevant commands into the sources.

To that end, drop all statements |\input{childdoc.def}|
and perform the replacements as outlined below.
Instead of |\childdocmain{|\textit{main}|}| add the following code
to the top of the main file:
%
\begin{center}
\begin{tabular}{l}
|\||ifdefined\childdocname\endinput\||fi\newif\ifchilddoc|\\
|\edef\childdocname{\scantokens\expandafter{\jobname\noexpand}}|\\
|\def\childdocmain{|\textit{main}|}\||ifx\childdocmain\childdocname\||else|\\
|\childdoctrue\includeonly{\childdocname}\let\jobname\childdocmain\||fi|\\
\end{tabular}
\end{center}
%
Instead of |\childdocof{|\textit{main}|}| just include the main file
at the top of each child file:
%
\begin{center}
|\input{|\textit{main}|}|
\end{center}
%
A simple redirection |\childdocforward{|\textit{dest}|}| is achieved by:
%
\begin{center}
|\def\jobname{|\textit{dest}|}\input{\jobname}|
\end{center}
%
The redirection with prefix
|\childdocforwardprefix[|\textit{prefix}|]{|\textit{dest}|}|
is accomplished by:
%
\begin{center}
\begin{tabular}{l}
|{\edef\jobname{\scantokens\expandafter{\jobname\noexpand}}|\\
|\def\redirectjob |\textit{prefix}|#1~~~{\gdef\jobname{|\textit{dest}|#1}}|\\
|\expandafter\redirectjob\jobname~~~}\input{\jobname}|
\end{tabular}
\end{center}

In an alternative approach,
child documents can be compiled by a specific command line
without additional code or specific definitions:
%
\begin{center}
|... -jobname "|\textit{target}|" "|[\textit{flags}]%
|\includeonly{|\textit{dest}|}\input{|\textit{main}|}"|
\end{center}
%

%%%%%%%%%%%%%%%%%%%%%%%%%%%%%%%%%%%%%%%%%%%%%%%%%%%%%%%%%%%%%%%%%%%%%%%%%%%%%%%%
%%%%%%%%%%%%%%%%%%%%%%%%%%%%%%%%%%%%%%%%%%%%%%%%%%%%%%%%%%%%%%%%%%%%%%%%%%%%%%%%
\section{Information}

%%%%%%%%%%%%%%%%%%%%%%%%%%%%%%%%%%%%%%%%%%%%%%%%%%%%%%%%%%%%%%%%%%%%%%%%%%%%%%%%
\subsection{Copyright}

Copyright \copyright{} 2017--2018 Niklas Beisert

This work may be distributed and/or modified under the
conditions of the \LaTeX{} Project Public License, either version 1.3
of this license or (at your option) any later version.
The latest version of this license is in
  \url{http://www.latex-project.org/lppl.txt}
and version 1.3 or later is part of all distributions of \LaTeX{}
version 2005/12/01 or later.

This work has the LPPL maintenance status `maintained'.

The Current Maintainer of this work is Niklas Beisert.

This work consists of the files |README.txt|, |childdoc.ins| and |childdoc.dtx|
as well as the derived files |childdoc.def|, |cdocsamp.tex|
with |cdocsch1.tex|, |cdocsch2.tex|, |cdocspt3.tex|, |cdocspt4.tex|,
|cdocsdrf.tex|, |cdocsfn1.tex|, |cdocsfn2.tex|
as well as |childdoc.pdf|.

%%%%%%%%%%%%%%%%%%%%%%%%%%%%%%%%%%%%%%%%%%%%%%%%%%%%%%%%%%%%%%%%%%%%%%%%%%%%%%%%
\subsection{Files and Installation}

The package consists of the files:
%
\begin{center}
\begin{tabular}{ll}
    |README.txt|   & readme file \\
    |childdoc.ins| & installation file \\
    |childdoc.dtx| & source file \\
    |childdoc.def| & definition file \\
    |cdocsamp.tex| & sample main file \\
    |cdocsch1.tex| & sample include file \\
    |cdocsch2.tex| & sample include file \\
    |cdocspt3.tex| & sample part file \\
    |cdocspt4.tex| & sample part file \\
    |cdocsdrf.tex| & sample redirection file \\
    |cdocsfn1.tex| & sample redirection file \\
    |cdocsfn2.tex| & sample redirection file \\
    |childdoc.pdf| & manual
\end{tabular}
\end{center}
%
The distribution consists of the files
|README.txt|, |childdoc.ins| and |childdoc.dtx|.
%
\begin{itemize}
\item
Run (pdf)\LaTeX{} on |childdoc.dtx|
to compile the manual |childdoc.pdf| (this file).
\item
Run \LaTeX{} on |childdoc.ins| to create the definitions file |childdoc.def|
and the sample |cdocsamp.tex| with include files
|cdocsch1.tex|, |cdocsch2.tex|, |cdocspt3.tex|, |cdocspt4.tex|,
|cdocsdrf.tex|, |cdocsfn1.tex|, |cdocsfn2.tex|.
Then copy the file |childdoc.def| to an appropriate directory of your \LaTeX{}
distribution, e.g.\ \textit{texmf-root}|/tex/latex/childdoc|.
\end{itemize}

%%%%%%%%%%%%%%%%%%%%%%%%%%%%%%%%%%%%%%%%%%%%%%%%%%%%%%%%%%%%%%%%%%%%%%%%%%%%%%%%
\subsection{Related CTAN Packages}

There are several other packages which offer a similar functionality:
%
\begin{itemize}
\item
The packages
\href{http://ctan.org/pkg/docmute}{\textsf{docmute}},
\href{http://ctan.org/pkg/includex}{\textsf{includex}} and
\href{http://ctan.org/pkg/standalone}{\textsf{standalone}}
provide commands to include only the document body of
a child file thus allowing both files to be compiled individually.
\item
The packages \href{http://ctan.org/pkg/subdocs}{\textsf{subdocs}}
and \href{http://ctan.org/pkg/subfiles}{\textsf{subfiles}}
provide structures in which the main and child documents can be
encapsulated and allowing them to be compiled individually.
The inclusion mechanism is different from the conventional |\include|.
\item
The package \href{http://ctan.org/pkg/combine}{\textsf{combine}}
is an elaborate solution to combine several documents into one.
\end{itemize}
%
See also the CTAN topic \href{http://ctan.org/topic/subdocs}{\textsf{subdocs}}
for further related packages.
The present package differs from the above solutions in that
a document structure constructed with the conventional |\include| mechanism
just needs two extra commands at the top of every file
such that all constituent files can be compiled individually.

%%%%%%%%%%%%%%%%%%%%%%%%%%%%%%%%%%%%%%%%%%%%%%%%%%%%%%%%%%%%%%%%%%%%%%%%%%%%%%%%
%\subsection{Feature Suggestions}
%
%The following is a list of features which may be useful for future
%versions of this package:
%%
%\begin{itemize}
%\item
%\ldots
%\end{itemize}

%%%%%%%%%%%%%%%%%%%%%%%%%%%%%%%%%%%%%%%%%%%%%%%%%%%%%%%%%%%%%%%%%%%%%%%%%%%%%%%%
\subsection{Revision History}

%%%%%%%%%%%%%%%%%%%%%%%%%%%%%%%%%%%%%%%%
\paragraph{v2.0:} 2018/12/30

\begin{itemize}
\item
immediate forward processing
\item
added |\childdocby| mechanism
\item
manual restructured
\end{itemize}

%%%%%%%%%%%%%%%%%%%%%%%%%%%%%%%%%%%%%%%%
\paragraph{v1.6:} 2018/01/17

\begin{itemize}
\item
application for development of include files
\item
corrections to manual
\end{itemize}

%%%%%%%%%%%%%%%%%%%%%%%%%%%%%%%%%%%%%%%%
\paragraph{v1.5:} 2017/05/21

\begin{itemize}
\item
more complete structuring introduced
\item
|\childdocof| introduced
\item
|\childdoc| renamed to |\childdocmain|
\item
|\childredirect| renamed to |\childdocforward| and |\childdocforwardprefix|
and functionality expanded
\end{itemize}

%%%%%%%%%%%%%%%%%%%%%%%%%%%%%%%%%%%%%%%%
\paragraph{v1.0:} 2017/04/27

\begin{itemize}
\item
manual and install package
\item
first version published on CTAN
\end{itemize}

%%%%%%%%%%%%%%%%%%%%%%%%%%%%%%%%%%%%%%%%
\paragraph{v0.6:} 2017/04/26

\begin{itemize}
\item
redirection mechanism added
\end{itemize}

%%%%%%%%%%%%%%%%%%%%%%%%%%%%%%%%%%%%%%%%
\paragraph{v0.5:} 2017/04/26

\begin{itemize}
\item
functionality in definition file
\end{itemize}


%%%%%%%%%%%%%%%%%%%%%%%%%%%%%%%%%%%%%%%%%%%%%%%%%%%%%%%%%%%%%%%%%%%%%%%%%%%%%%%%
%%%%%%%%%%%%%%%%%%%%%%%%%%%%%%%%%%%%%%%%%%%%%%%%%%%%%%%%%%%%%%%%%%%%%%%%%%%%%%%%
%%%%%%%%%%%%%%%%%%%%%%%%%%%%%%%%%%%%%%%%%%%%%%%%%%%%%%%%%%%%%%%%%%%%%%%%%%%%%%%%
\appendix

\settowidth\MacroIndent{\rmfamily\scriptsize 000\ }

 \DocInput{childdoc.dtx}

\end{document}
%</driver>
% \fi
%
% %%%%%%%%%%%%%%%%%%%%%%%%%%%%%%%%%%%%%%%%%%%%%%%%%%%%%%%%%%%%%%%%%%%%%%%%%%%%%%
% %%%%%%%%%%%%%%%%%%%%%%%%%%%%%%%%%%%%%%%%%%%%%%%%%%%%%%%%%%%%%%%%%%%%%%%%%%%%%%
% \section{Sample}
%\iffalse
%<*samplemain>
%\fi
%
% The following presents a sample document
% with two chapters, two parts, a title page,
% a compile flag as well as three forwarding files to set the flag.
% It consists of eight |.tex| files:
% \begin{center}
% \begin{tabular}{ll}
% |cdocsamp.tex|&main file\\
% |cdocsch1.tex|&include file for chapter 1\\
% |cdocsch2.tex|&include file for chapter 2\\
% |cdocspt3.tex|&include file for part 3\\
% |cdocspt4.tex|&include file for part 4\\
% |cdocsdrf.tex|&forwarding file for main file in draft mode\\
% |cdocsfi1.tex|&forwarding file for final version of chapter 1\\
% |cdocsfi2.tex|&forwarding file for final version of chapter 2\\
% \end{tabular}
% \end{center}
% Each of the eight files can be compiled directly by the \LaTeX{} compiler.
%
% %%%%%%%%%%%%%%%%%%%%%%%%%%%%%%%%%%%%%%
% \paragraph{Main File.}
%
% The main file is called |cdocsamp.tex|.
%
% Load the \textsf{childdoc} definitions and
% declare the filename for the main document:
%    \begin{macrocode}
\input{childdoc.def}
\childdocmain{}
%    \end{macrocode}

% Optional override for |\version| flag:
%    \begin{macrocode}
%%\ifchilddoc\else\providecommand{\version}{draft}\fi
%    \end{macrocode}

% Define the default values for the |\version| flag
% (|final| for the main file and |draft| for childs):
%    \begin{macrocode}
\ifchilddoc
\providecommand{\version}{draft}
\else
\providecommand{\version}{final}
\fi
%    \end{macrocode}

% Load the standard document class:
%    \begin{macrocode}
\documentclass[12pt]{article}
%    \end{macrocode}

% Start the document body:
%    \begin{macrocode}
\begin{document}
%    \end{macrocode}

% Declare a title page.
% Print title, part of document being processed and version flag:
%    \begin{macrocode}
\addtocounter{page}{-1}
\begin{center}
{\LARGE\bfseries{}childdoc example\par}
\vspace{1cm}
\ifchilddoc
\ifchilddocmanual part\else chapter\fi:
`\childdocname' of `\childdocjob'\par
\else
main document: `\childdocjob'\par
\fi
version: \version\par
\end{center}
\newpage
%    \end{macrocode}

% Manually include selected file,
% otherwise process as usual:
%    \begin{macrocode}
\ifchilddocmanual
\section*{part `\childdocname'}
\input{\childdocname}
\else
%    \end{macrocode}

% Include the two chapters:
%    \begin{macrocode}
\include{cdocsch1}
\include{cdocsch2}
%    \end{macrocode}

% Include the two parts unless only chapters should be displayed:
%    \begin{macrocode}
\ifchilddoc\else
\section{part three}
\input{cdocspt3}
\section{part four}
\input{cdocspt4}
\fi
%    \end{macrocode}

% Process as usual until here:
%    \begin{macrocode}
\fi
%    \end{macrocode}

% End of document body:
%    \begin{macrocode}
\end{document}
%    \end{macrocode}
%\iffalse
%</samplemain>
%\fi
%
% %%%%%%%%%%%%%%%%%%%%%%%%%%%%%%%%%%%%%%
% \paragraph{Chapter Include Files.}
%
% The include files are called |cdocsch1.tex| and |cdocsch2.tex|.
%
%\iffalse
%<*samplechap1|samplechap2>
%\fi

% Optional override for |\version| flag:
%    \begin{macrocode}
%%\providecommand{\version}{final}
%    \end{macrocode}

% Include the main document:
%    \begin{macrocode}
\input{childdoc.def}
\childdocof{cdocsamp}
%    \end{macrocode}

%\iffalse
%</samplechap1|samplechap2>
%\fi
%
%\iffalse
%<*samplechap1>
%\fi
% Some text for chapter 1:
%    \begin{macrocode}
\section{one}
some text in chapter one
%    \end{macrocode}

%\iffalse
%</samplechap1>
%\fi
% Some text for chapter 2:
%\iffalse
%<*samplechap2>
%\fi
%    \begin{macrocode}
\section{two}
more text in chapter two
%    \end{macrocode}

%\iffalse
%</samplechap2>
%\fi
%
% %%%%%%%%%%%%%%%%%%%%%%%%%%%%%%%%%%%%%%
% \paragraph{Part Include Files.}
%
% The include files are called |cdocspt3.tex| and |cdocspt4.tex|.
%
%\iffalse
%<*samplepart3|samplepart4>
%\fi

% Optional override for |\version| flag:
%    \begin{macrocode}
%%\providecommand{\version}{final}
%    \end{macrocode}

% Include the main document:
%    \begin{macrocode}
\input{childdoc.def}
\childdocby{cdocsamp}
%    \end{macrocode}

%\iffalse
%</samplepart3|samplepart4>
%\fi
%
%\iffalse
%<*samplepart3>
%\fi
% Some text for part 3:
%    \begin{macrocode}
some text in part three
%    \end{macrocode}

%\iffalse
%</samplepart3>
%\fi
% Some text for part 4:
%\iffalse
%<*samplepart4>
%\fi
%    \begin{macrocode}
more text in part four
%    \end{macrocode}

%\iffalse
%</samplepart4>
%\fi
%
% %%%%%%%%%%%%%%%%%%%%%%%%%%%%%%%%%%%%%%
% \paragraph{Forwarding for a Complete Draft.}
%
% The following forwarding file |cdocsdrf.tex|
% compiles the main document in draft mode:
%\iffalse
%<*sampledraft>
%\fi
%    \begin{macrocode}
\def\version{draft}
\input{childdoc.def}
\childdocforward{cdocsamp}
%    \end{macrocode}

%\iffalse
%</sampledraft>
%\fi
%
% %%%%%%%%%%%%%%%%%%%%%%%%%%%%%%%%%%%%%%
% \paragraph{Forwarding for Final Version of the Chapters.}
%
% The following forwarding files |cdocsfn1.tex| and |cdocsfn2.tex|
% (with identical content)
% compile the final versions of the child documents
% |cdocsch1.tex| and |cdocsch2.tex|, respectively:
%\iffalse
%<*samplefinal>
%\fi
%    \begin{macrocode}
\def\version{final}
\input{childdoc.def}
\childdocforwardprefix[cdocsamp]{cdocsfn}{cdocsch}
%    \end{macrocode}

%\iffalse
%</samplefinal>
%\fi
%
% %%%%%%%%%%%%%%%%%%%%%%%%%%%%%%%%%%%%%%
% \paragraph{Command Line Processing.}
%
% The following three command lines generate the output files
% |cdocscld|, |cdocscl1| and |cdocscl2|
% which should be identical to
% |cdocsdrf|, |cdocsch1| and |cdocsfn2|, respectively:
% \begin{center}
% \begin{tabular}{l}
% |latex -jobname cdocscld \|\\
% |  "\def\version{draft}\input{childdoc.def}\childdocforward{cdocsamp}"|\\
% |latex -jobname cdocscl1 \|\\
% |  "\input{childdoc.def}\childdocforward[cdocsamp]{cdocsch1}"|\\
% |latex -jobname cdocscl2 \|\\
% |  "\def\version{final}\input{childdoc.def}\childdocforward{cdocsch2}"|
% \end{tabular}
% \end{center}
% Note that the trailing backslash on each first line
% merely continues the input to the second line
% (for convenient cut ant paste).
% Furthermore, the command |latex| can be replaced by any
% of its alternative versions such as |pdflatex|.
%
% %%%%%%%%%%%%%%%%%%%%%%%%%%%%%%%%%%%%%%%%%%%%%%%%%%%%%%%%%%%%%%%%%%%%%%%%%%%%%%
% %%%%%%%%%%%%%%%%%%%%%%%%%%%%%%%%%%%%%%%%%%%%%%%%%%%%%%%%%%%%%%%%%%%%%%%%%%%%%%
% \section{Implementation}
%\iffalse
%<*package>
%\fi
%
% This section describes the definitions file |childdoc.def|.

% The definitions cannot be loaded using |\usepackage| or |\RequirePackage|
% which has a mechanism to prevent loading a style file more than once.
% When loading the definitions by means of |\input|
% multiple instances have to be prevented manually:
%\iffalse
%This code needs to be before the `\ProvidesFile' directive
%which is defined at the beginning of this file.
%Therefore it is also placed there and commented out here.
%</package>
%<*discard>
%\fi
%    \begin{macrocode}
\ifdefined\childdocmain\endinput\fi
%    \end{macrocode}
%\iffalse
%</discard>
%<*package>
%\fi
%
% \macro{\ifchilddoc}
% \macro{\ifchilddocmanual}
% The conditional |\ifchilddoc| tells whether a
% child (true) or main (false) document is being compiled.
% The conditional |\ifchilddocmanual| tells whether
% the |\includeonly| mechanism is used (false) or
% the selection of child files must be performed manually (true).
% The definitions initialise to false:
%    \begin{macrocode}
\newif\ifchilddoc
\newif\ifchilddocmanual
%    \end{macrocode}

% \macro{\childdocname}
% \macro{\childdocjob}
% The macro |\childdocname| stores the name of the main document
% to be compiled. The macro |\childdocjob| stores the name of
% the document on which the \LaTeX{} compiler was originally invoked.
% The content of |\jobname| cannot be compared
% to filenames specified in the source due to different catcodes.
% The following code rescans |\jobname|, stores the result
% in |\childdocname| and saves a copy in |\childdocjob|:
%    \begin{macrocode}
\edef\childdocname{\scantokens\expandafter{\jobname\noexpand}}
\let\childdocjob\childdocname
%    \end{macrocode}

% \macro{\childdocdisable}
% The macro |\childdocdisable| prevents the main file
% from being processed more than once.
% At this stage, the main document command |\childdocmain|
% is assumed to be called once again where it should do nothing.
% Any subsequent call to it should prevent
% a secondary processing of the main document
% It overwrites the forwarding commands
% |\childdocof| and |\childdocforward|
% with empty macros to prevent further inclusions of the main document:
%    \begin{macrocode}
\newcommand{\childdocdisable}
{
  \renewcommand{\childdocmain}[1]{\renewcommand{\childdocmain}[1]{\endinput}}
  \renewcommand{\childdocof}[1]{}
  \renewcommand{\childdocby}[2][]{}
  \renewcommand{\childdocforward}[2][]{}
  \renewcommand{\childdocdisable}{}
}
%    \end{macrocode}

% \macro{\childdocmain}
% The macro |\childdocmain| is to be called at the top of the main file
% with nothing or the main filename (without extension) as argument.
% First, it breaks loops.
% If the argument is not empty and does not match |\childdocname|
% (which is set by the first inclusion of |childdoc.def|),
% |\ifchilddoc| is set to true, |\includeonly| is applied to the child file
% and |\jobname| is set to the main file
% (for proper handling of |.aux| files):
%    \begin{macrocode}
\newcommand{\childdocmain}[1]
{
  \childdocdisable\childdocmain{}
  \if?#1?\else
    \begingroup
      \def\childdoctmp{#1}
      \ifx\childdoctmp\childdocname
        \def\childdoctmp{}
      \else
        \def\childdoctmp
        {
          \childdoctrue
          \includeonly{\childdocname}
          \def\childdocjob{#1}
          \def\jobname{#1}
        }
      \fi
      \expandafter
    \endgroup
    \childdoctmp
  \fi
}
%    \end{macrocode}

% \macro{\childdocof}
% The command |\childdocof| redirects
% compilation to the main file |#1|.
%    \begin{macrocode}
\newcommand{\childdocof}[1]
{
  \childdocdisable
  \childdoctrue
  \includeonly{\childdocname}
  \def\jobname{#1}
  \def\childdocjob{#1}
  \input{#1}
}
%    \end{macrocode}

% \macro{\childdocby}
% The command |\childdocby| ....
%    \begin{macrocode}
\newcommand{\childdocby}[2][]
{
  \childdocdisable
  \childdoctrue
  \childdocmanualtrue
  \if?#1?\else
    \def\jobname{#2}
  \fi
  \def\childdocjob{#2}
  \input{#2}
  \endinput
}
%    \end{macrocode}

% \macro{\childdocforward}
% The command |\childdocforward| redirects
% compilation to the main file or
% (if the optional argument is given) a child file.
% Parameters are set as if the main file
% or a child file starting with |\childdocof| was compiled.
% Then compilation is handed over to the main file:
%    \begin{macrocode}
\newcommand{\childdocforward}[2][]
{
  \begingroup
    \if?#1?
      \def\childdoctmp
      {
        \def\childdocname{#2}
        \def\childdocjob{#2}
        \def\jobname{#2}
        \input{#2}
        \endinput
      }
    \else
      \def\childdoctmp
      {
        \childdocdisable
        \def\childdocname{#2}
        \childdoctrue
        \includeonly{#2}
        \def\childdocjob{#1}
        \def\jobname{#1}
        \input{#1}
        \endinput
      }
    \fi
    \expandafter
  \endgroup
  \childdoctmp
}
%    \end{macrocode}

% \macro{\childdocforwardprefix}
% The command |\childdocforwardprefix| redirects
% compilation to the main or a child file by means of a pattern.
% The prefix |#1| in the current filename is replaced by |#2|
% and the suffix of the current filename is kept
% (it is assumed that the filename does not contain the substring `|~~~|'
% which is used as a delimiter).
% Compilation is handed over to the new file by |\childdocforward|:
%    \begin{macrocode}
\newcommand{\childdocforwardprefix}[3][]
{
  \begingroup
    \def\childdocextract #2##1~~~{\def\childdoctmp{\childdocforward[#1]{#3##1}}}
    \expandafter\childdocextract\childdocname~~~
    \expandafter
  \endgroup
  \childdoctmp
}
%    \end{macrocode}

% \macro{\childdoc}
% The deprecated macro |\childdoc| is a legacy version of |\childdocmain|:
%    \begin{macrocode}
\newcommand{\childdoc}{\childdocmain}
%    \end{macrocode}

% \macro{\childdocredirect}
% The deprecated macro |\childdocredirect| is a legacy version
% of |\childdocforward| and |\childdocforwardprefix|:
%    \begin{macrocode}
\newcommand{\childdocredirect}[2][]
{
  \begingroup
    \if?#1?
      \def\childdoctmp{\childdocforward{#2}}
    \else
      \def\childdoctmp{\childdocforwardprefix{#1}{#2}}
    \fi
    \expandafter
  \endgroup
  \childdoctmp
}
%    \end{macrocode}

%\iffalse
%</package>
%\fi
%
\endinput
|\\
|\childdocof{|\textit{main}|}|\\
\end{tabular}
\end{center}
at the top of every child file \textit{child}
which is included by |\include{|\textit{child}|}|
from within the main file
(or at least for those files to be compiled individually).
The argument \textit{main} must be the filename of the main file.

There are a couple of
considerations in setting up the main and child documents:

%%%%%%%%%%%%%%%%%%%%%%%%%%%%%%%%%%%%%%%%
\paragraph{Restrictions.}

Please note the following restrictions:
\begin{itemize}
\item
|\childdocmain| must be called with one argument \textit{main}
to ensure compatibility with earlier version of the package.
It must either be empty (|\childdocmain{}|)
or precisely match the filename of the main file in which it is specified.
See \secref{sec:detection} for further information.
\item
The filename \textit{main} must be specified without the |.tex| extension.
\item
The filename \textit{main} is case sensitive
(even in case-insensitive file systems)
due to internal string comparison.
\item
The argument \textit{main} should be fully expanded, it cannot be a macro.
\item
Subdirectories and special characters should be avoided in filenames.
\item
The command |\childdocmain{|\textit{main}|}| must be followed by a whitespace.
It should not be followed immediately by another command
or by a comment mark `|%|'.
This is because the \TeX{} parser reads the token immediately following
the argument of |\childdocmain| and puts it
at the beginning of every child section;
however, a white\-space is ignored.
\end{itemize}

%%%%%%%%%%%%%%%%%%%%%%%%%%%%%%%%%%%%%%%%
\paragraph{Content of Main File.}

It is advisable to place all content in the child files included by |\include|.
Any output contained in the main file will appear in all child documents
unless suppressed manually;
it cannot be suppressed automatically by the |\includeonly| directive
and thus should normally be avoided.
A method to include some content in the main file
by means of conditional processing is described in \secref{sec:conditional}.

%%%%%%%%%%%%%%%%%%%%%%%%%%%%%%%%%%%%%%%%
\paragraph{Page Numbering.}

When only a part of the document is compiled,
the appropriate numbering of pages
(as well as other status parameters)
is determined from the |.aux| files.
The latter contain information from previous passes.
However this information needs to propagate through
all intermediate child documents.
Therefore the page numbering in child documents may well
be inconsistent until the complete document is compiled at least once.

A useful (if unconventional) way to always ensure a consistent
page numbering is to restart the numbering in each child document
and denote the pages by `\textit{child}|.|\textit{page}'
where \textit{child} represents the chapter/section number of the child file.
This can be achieved by the command
|\numberwithin{page}{|\textit{child}|}|
of the \textsf{amsmath} package
where \textit{child} can be |chapter| or |section|
depending on the chosen structuring.
Alternatively, one can modify the macro |\thepage| appropriately
and reset the counter |page| at the start of each child file.

%%%%%%%%%%%%%%%%%%%%%%%%%%%%%%%%%%%%%%%%%%%%%%%%%%%%%%%%%%%%%%%%%%%%%%%%%%%%%%%%
\subsection{Conditional Processing}
\label{sec:conditional}

The package provides a mechanism to compile different versions
of a document. To customise the versions further some conditional processing
can come in handy to distinguish which version is being compiled.
The package provides two macros to describe the compilation context:

%%%%%%%%%%%%%%%%%%%%%%%%%%%%%%%%%%%%%%%%
\DescribeMacro{\ifchilddoc}
The conditional |\ifchilddoc| distinguishes between the compilation of
child documents and the main document:
%
\begin{center}
|\ifchilddoc |\textit{child-code}| |[|\||else |\textit{main-code}]| \||fi|
\end{center}

%%%%%%%%%%%%%%%%%%%%%%%%%%%%%%%%%%%%%%%%
\DescribeMacro{\childdocname}
\DescribeMacro{\childdocjob}
The macro |\childdocname| contains the filename (without extension)
of the main or child file being processed.
Note that |\childdocjob| will always contain the name of the main file.

%%%%%%%%%%%%%%%%%%%%%%%%%%%%%%%%%%%%%%%%
\paragraph{Title Page.}

Conditional processing can be used to include a title or banner page
in the main document when proper precautions are taken.
Importantly, the code in the main file should ensure that the page counter
(as well as other status parameters which are stored in the |.aux| files)
takes the same value after the conditional processing.
Otherwise the page numbers may take divergent values
depending on which part is compiled.

For example, a title page could be declared by:
%
\begin{center}
\begin{tabular}{l}
|\ifchilddoc\||else|\\
|\addtocounter{page}{-1}|\\
\textit{code for title page}\\
|\newpage|\\
|\||fi|
\end{tabular}
\end{center}
%
A banner page for the child documents can be generated by:
%
\begin{center}
\begin{tabular}{l}
|\ifchilddoc|\\
|\addtocounter{page}{-1}|\\
\textit{code for banner page}\\
|\newpage|\\
|\||fi|
\end{tabular}
\end{center}
%
Here one could write a message such as:
\begin{center}
|This is the part \childdocname{} of \childdocjob{}.|
\end{center}

%%%%%%%%%%%%%%%%%%%%%%%%%%%%%%%%%%%%%%%%%%%%%%%%%%%%%%%%%%%%%%%%%%%%%%%%%%%%%%%%
\subsection{Flags}
\label{sec:flags}

The package makes it easy to generate different versions
of the main or child documents.
To this end compilation flags can be defined
and assigned different default values.
They will be particularly useful in conjunction
with the forwarding mechanism described in \secref{sec:forward}.

For example, it may be useful to have a flag |\version|
which can be set to |draft| or |final|.
The document source will contain some conditional code
depending on the value of |\version|.
Suppose further, the flag should default to |final| for the main file
and to |draft| for child files
which is a natural assignment for editing the document.
This is achieved by placing the following code
in the preamble of the main document
(below the |\childdocmain| directive):
%
\begin{center}
\begin{tabular}{l}
|\ifchilddoc|\\
|\providecommand{\version}{draft}|\\
|\||else|\\
|\providecommand{\version}{final}|\\
|\||fi|
\end{tabular}
\end{center}
%
The definition by |\providecommand| makes sure
that previous definitions are not overwritten.
Further statements |\providecommand{\version}{...}|
can thus be added before the above code to override it.

For the main file, one might add a line
(between |\childdocmain| and the above block)
%
\begin{center}
|%\ifchilddoc\||else\providecommand{\version}{draft}\||fi|
\end{center}
%
which can be uncommented to produce a draft version.
Likewise one can add a line to the very top of a child file
(above the |\childdocof{|\textit{main}|}| directive)
%
\begin{center}
|%\providecommand{\version}{final}|
\end{center}
%
which can be uncommented to produce the final version of this child document.

%%%%%%%%%%%%%%%%%%%%%%%%%%%%%%%%%%%%%%%%%%%%%%%%%%%%%%%%%%%%%%%%%%%%%%%%%%%%%%%%
\subsection{Forwarding}
\label{sec:forward}

Different versions of the main or child documents
using compilation flags as described in \secref{sec:flags}
can be (permanently) stored in different files
for convenient compilation, viewing and distribution.
To this end, the package defines a command
to pass on compilation to a different file:

%%%%%%%%%%%%%%%%%%%%%%%%%%%%%%%%%%%%%%%%
\DescribeMacro{\childdocforward}
The command |\childdocforward| redirects processing to
another source file:
%
\begin{center}
\begin{tabular}{l}
|% \iffalse
%
% childdoc.dtx Copyright (C) 2017-2018 Niklas Beisert
%
% This work may be distributed and/or modified under the
% conditions of the LaTeX Project Public License, either version 1.3
% of this license or (at your option) any later version.
% The latest version of this license is in
%   http://www.latex-project.org/lppl.txt
% and version 1.3 or later is part of all distributions of LaTeX
% version 2005/12/01 or later.
%
% This work has the LPPL maintenance status `maintained'.
%
% The Current Maintainer of this work is Niklas Beisert.
%
% This work consists of the files childdoc.dtx and childdoc.ins
% and the derived files childdoc.def and cdocsamp.tex with
% cdocsch1.tex, cdocsch2.tex, cdocsdrf.tex, cdocsfn1.tex, cdocsfn2.tex.
%
%<package>\ifdefined\childdocmain\endinput\fi
%<package>\ProvidesFile{childdoc.def}[2018/12/30 v2.0 child document driver]
%<samplemain>\ProvidesFile{cdocsamp.tex}[2018/12/30 v2.0 sample for childdoc]
%<*driver>
%\ProvidesFile{childdoc.drv}[2018/12/30 v2.0 childdoc reference manual file]
\PassOptionsToClass{10pt,a4paper}{article}
\documentclass{ltxdoc}

\usepackage[margin=35mm]{geometry}
\usepackage{hyperref}
\usepackage{hyperxmp}
\usepackage[usenames]{color}

\hypersetup{colorlinks=true}
\hypersetup{pdfstartview=FitH}
\hypersetup{pdfpagemode=UseNone}
\hypersetup{pdfsource={}}
\hypersetup{pdflang={en-UK}}
\hypersetup{pdfcopyright={Copyright 2017-2018 Niklas Beisert.
  This work may be distributed and/or modified under the
  conditions of the LaTeX Project Public License, either version 1.3
  of this license or (at your option) any later version.}}
\hypersetup{pdflicenseurl={http://www.latex-project.org/lppl.txt}}
\hypersetup{pdfcontactaddress={ETH Zurich, ITP, HIT K,
  Wolfgang-Pauli-Strasse 27}}
\hypersetup{pdfcontactpostcode={8093}}
\hypersetup{pdfcontactcity={Zurich}}
\hypersetup{pdfcontactcountry={Switzerland}}
\hypersetup{pdfcontactemail={nbeisert@itp.phys.ethz.ch}}
\hypersetup{pdfcontacturl={http://people.phys.ethz.ch/\xmptilde nbeisert/}}

\newcommand{\secref}[1]{\hyperref[#1]{section \ref*{#1}}}

\parskip1ex
\parindent0pt
\let\olditemize\itemize
\def\itemize{\olditemize\parskip0pt}

\begin{document}

\title{The \textsf{childdoc} Package}
\hypersetup{pdftitle={The childdoc Package}}
\author{Niklas Beisert\\[2ex]
  Institut f\"ur Theoretische Physik\\
  Eidgen\"ossische Technische Hochschule Z\"urich\\
  Wolfgang-Pauli-Strasse 27, 8093 Z\"urich, Switzerland\\[1ex]
  \href{mailto:nbeisert@itp.phys.ethz.ch}
  {\texttt{nbeisert@itp.phys.ethz.ch}}}
\hypersetup{pdfauthor={Niklas Beisert}}
\hypersetup{pdfsubject={Manual for the LaTeX2e Package childdoc}}
\date{30 December 2018, \textsf{v2.0}}
\maketitle

\begin{abstract}\noindent
\textsf{childdoc} is a \LaTeXe{} package
that enables the direct compilation
of document sections included by |\include|
to individual files.
\end{abstract}

\begingroup
\parskip0ex
\tableofcontents
\endgroup

%%%%%%%%%%%%%%%%%%%%%%%%%%%%%%%%%%%%%%%%%%%%%%%%%%%%%%%%%%%%%%%%%%%%%%%%%%%%%%%%
%%%%%%%%%%%%%%%%%%%%%%%%%%%%%%%%%%%%%%%%%%%%%%%%%%%%%%%%%%%%%%%%%%%%%%%%%%%%%%%%
\section{Introduction}

\LaTeX{} provides a mechanism to structure a large document (such as a book)
into a main file and several child files (containing the chapters)
using the |\include| command.
This mechanism is beneficial for documents
which span hundreds of pages in order to
make the source file(s) more manageable.
Moreover, compilation can be restricted to
selected child files by means of the |\includeonly| command.
The latter feature can be used to reduce the compilation time while editing
(this was significantly more useful in the earlier days of \LaTeX{})
or to generate a smaller document which is easier to navigate.
Another application of |\includeonly| is to generate
documents consisting of selected parts of the complete document.

However, there are a few drawbacks of the plain |\include| mechanism:
\begin{itemize}
\item
The child files cannot be compiled on their own,
they can only be compiled via the main file.
A naive editing environment
(such as a text editor with an option
to have the current file processed by \LaTeX)
may require one to switch to the main file before compiling;
attempting to compile the child file produces errors.
\item
The main file must be modified (each time)
to adjust the |\includeonly| command
to the present needs. This easily leaves the main file in a messy state.
\item
The generated document will always carry the filename
of the main document. This is inconvenient if
several child files are to be compiled and
to be kept for distribution.
\end{itemize}

The present package provides a simple interface
to make child files individually compilable by \LaTeX{}.
Compiling a child file then has the same effect as compiling
the main file with an |\includeonly| command
to select the appropriate child.
Moreover the generated document will carry the name of the child
rather than the main file.
This resolves all three above issues.

This feature is meant to make the editing of books,
thesis documents and lecture notes somewhat more convenient.
However, the package can also be used efficiently for
composing a series of documents (such as exercise sheets)
which are typically distributed individually.
It then assists the author in generating the individual documents
(potentially in different versions)
as well as a document containing the collected series.
Another application is in developing style files
or other kinds of included material
where compilation of the style file could redirect
to a sample or test file.

%%%%%%%%%%%%%%%%%%%%%%%%%%%%%%%%%%%%%%%%%%%%%%%%%%%%%%%%%%%%%%%%%%%%%%%%%%%%%%%%
%%%%%%%%%%%%%%%%%%%%%%%%%%%%%%%%%%%%%%%%%%%%%%%%%%%%%%%%%%%%%%%%%%%%%%%%%%%%%%%%
\section{Usage}

First of all, the package \textsf{childdoc} is \emph{not} a standard
\LaTeXe{} |.sty| style file! Therefore it needs to be invoked in
a non-standard way.

%%%%%%%%%%%%%%%%%%%%%%%%%%%%%%%%%%%%%%%%%%%%%%%%%%%%%%%%%%%%%%%%%%%%%%%%%%%%%%%%
\subsection{Included Files}
\label{sec:include}

%%%%%%%%%%%%%%%%%%%%%%%%%%%%%%%%%%%%%%%%
\DescribeMacro{\childdocmain}
To use the package, add the commands
\begin{center}
\begin{tabular}{l}
|\input{childdoc.def}|\\
|\childdocmain{}|\\
\end{tabular}
\end{center}
at the very top of the main \LaTeX{} file,
in particular \emph{before} the |\documentclass| statement!
The argument of |\childdocmain| should be left empty
(but it must be present).

%%%%%%%%%%%%%%%%%%%%%%%%%%%%%%%%%%%%%%%%
\DescribeMacro{\childdocof}
Furthermore, add the commands
\begin{center}
\begin{tabular}{l}
|\input{childdoc.def}|\\
|\childdocof{|\textit{main}|}|\\
\end{tabular}
\end{center}
at the top of every child file \textit{child}
which is included by |\include{|\textit{child}|}|
from within the main file
(or at least for those files to be compiled individually).
The argument \textit{main} must be the filename of the main file.

There are a couple of
considerations in setting up the main and child documents:

%%%%%%%%%%%%%%%%%%%%%%%%%%%%%%%%%%%%%%%%
\paragraph{Restrictions.}

Please note the following restrictions:
\begin{itemize}
\item
|\childdocmain| must be called with one argument \textit{main}
to ensure compatibility with earlier version of the package.
It must either be empty (|\childdocmain{}|)
or precisely match the filename of the main file in which it is specified.
See \secref{sec:detection} for further information.
\item
The filename \textit{main} must be specified without the |.tex| extension.
\item
The filename \textit{main} is case sensitive
(even in case-insensitive file systems)
due to internal string comparison.
\item
The argument \textit{main} should be fully expanded, it cannot be a macro.
\item
Subdirectories and special characters should be avoided in filenames.
\item
The command |\childdocmain{|\textit{main}|}| must be followed by a whitespace.
It should not be followed immediately by another command
or by a comment mark `|%|'.
This is because the \TeX{} parser reads the token immediately following
the argument of |\childdocmain| and puts it
at the beginning of every child section;
however, a white\-space is ignored.
\end{itemize}

%%%%%%%%%%%%%%%%%%%%%%%%%%%%%%%%%%%%%%%%
\paragraph{Content of Main File.}

It is advisable to place all content in the child files included by |\include|.
Any output contained in the main file will appear in all child documents
unless suppressed manually;
it cannot be suppressed automatically by the |\includeonly| directive
and thus should normally be avoided.
A method to include some content in the main file
by means of conditional processing is described in \secref{sec:conditional}.

%%%%%%%%%%%%%%%%%%%%%%%%%%%%%%%%%%%%%%%%
\paragraph{Page Numbering.}

When only a part of the document is compiled,
the appropriate numbering of pages
(as well as other status parameters)
is determined from the |.aux| files.
The latter contain information from previous passes.
However this information needs to propagate through
all intermediate child documents.
Therefore the page numbering in child documents may well
be inconsistent until the complete document is compiled at least once.

A useful (if unconventional) way to always ensure a consistent
page numbering is to restart the numbering in each child document
and denote the pages by `\textit{child}|.|\textit{page}'
where \textit{child} represents the chapter/section number of the child file.
This can be achieved by the command
|\numberwithin{page}{|\textit{child}|}|
of the \textsf{amsmath} package
where \textit{child} can be |chapter| or |section|
depending on the chosen structuring.
Alternatively, one can modify the macro |\thepage| appropriately
and reset the counter |page| at the start of each child file.

%%%%%%%%%%%%%%%%%%%%%%%%%%%%%%%%%%%%%%%%%%%%%%%%%%%%%%%%%%%%%%%%%%%%%%%%%%%%%%%%
\subsection{Conditional Processing}
\label{sec:conditional}

The package provides a mechanism to compile different versions
of a document. To customise the versions further some conditional processing
can come in handy to distinguish which version is being compiled.
The package provides two macros to describe the compilation context:

%%%%%%%%%%%%%%%%%%%%%%%%%%%%%%%%%%%%%%%%
\DescribeMacro{\ifchilddoc}
The conditional |\ifchilddoc| distinguishes between the compilation of
child documents and the main document:
%
\begin{center}
|\ifchilddoc |\textit{child-code}| |[|\||else |\textit{main-code}]| \||fi|
\end{center}

%%%%%%%%%%%%%%%%%%%%%%%%%%%%%%%%%%%%%%%%
\DescribeMacro{\childdocname}
\DescribeMacro{\childdocjob}
The macro |\childdocname| contains the filename (without extension)
of the main or child file being processed.
Note that |\childdocjob| will always contain the name of the main file.

%%%%%%%%%%%%%%%%%%%%%%%%%%%%%%%%%%%%%%%%
\paragraph{Title Page.}

Conditional processing can be used to include a title or banner page
in the main document when proper precautions are taken.
Importantly, the code in the main file should ensure that the page counter
(as well as other status parameters which are stored in the |.aux| files)
takes the same value after the conditional processing.
Otherwise the page numbers may take divergent values
depending on which part is compiled.

For example, a title page could be declared by:
%
\begin{center}
\begin{tabular}{l}
|\ifchilddoc\||else|\\
|\addtocounter{page}{-1}|\\
\textit{code for title page}\\
|\newpage|\\
|\||fi|
\end{tabular}
\end{center}
%
A banner page for the child documents can be generated by:
%
\begin{center}
\begin{tabular}{l}
|\ifchilddoc|\\
|\addtocounter{page}{-1}|\\
\textit{code for banner page}\\
|\newpage|\\
|\||fi|
\end{tabular}
\end{center}
%
Here one could write a message such as:
\begin{center}
|This is the part \childdocname{} of \childdocjob{}.|
\end{center}

%%%%%%%%%%%%%%%%%%%%%%%%%%%%%%%%%%%%%%%%%%%%%%%%%%%%%%%%%%%%%%%%%%%%%%%%%%%%%%%%
\subsection{Flags}
\label{sec:flags}

The package makes it easy to generate different versions
of the main or child documents.
To this end compilation flags can be defined
and assigned different default values.
They will be particularly useful in conjunction
with the forwarding mechanism described in \secref{sec:forward}.

For example, it may be useful to have a flag |\version|
which can be set to |draft| or |final|.
The document source will contain some conditional code
depending on the value of |\version|.
Suppose further, the flag should default to |final| for the main file
and to |draft| for child files
which is a natural assignment for editing the document.
This is achieved by placing the following code
in the preamble of the main document
(below the |\childdocmain| directive):
%
\begin{center}
\begin{tabular}{l}
|\ifchilddoc|\\
|\providecommand{\version}{draft}|\\
|\||else|\\
|\providecommand{\version}{final}|\\
|\||fi|
\end{tabular}
\end{center}
%
The definition by |\providecommand| makes sure
that previous definitions are not overwritten.
Further statements |\providecommand{\version}{...}|
can thus be added before the above code to override it.

For the main file, one might add a line
(between |\childdocmain| and the above block)
%
\begin{center}
|%\ifchilddoc\||else\providecommand{\version}{draft}\||fi|
\end{center}
%
which can be uncommented to produce a draft version.
Likewise one can add a line to the very top of a child file
(above the |\childdocof{|\textit{main}|}| directive)
%
\begin{center}
|%\providecommand{\version}{final}|
\end{center}
%
which can be uncommented to produce the final version of this child document.

%%%%%%%%%%%%%%%%%%%%%%%%%%%%%%%%%%%%%%%%%%%%%%%%%%%%%%%%%%%%%%%%%%%%%%%%%%%%%%%%
\subsection{Forwarding}
\label{sec:forward}

Different versions of the main or child documents
using compilation flags as described in \secref{sec:flags}
can be (permanently) stored in different files
for convenient compilation, viewing and distribution.
To this end, the package defines a command
to pass on compilation to a different file:

%%%%%%%%%%%%%%%%%%%%%%%%%%%%%%%%%%%%%%%%
\DescribeMacro{\childdocforward}
The command |\childdocforward| redirects processing to
another source file:
%
\begin{center}
\begin{tabular}{l}
|\input{childdoc.def}|\\
|\childdocforward[|\textit{main}|]{|\textit{dest}|}|\\
\end{tabular}
\end{center}
%
The argument \textit{dest} is the destination file
(without extension).
It should be the main file or one of the child files.
Note that further \textsf{childdoc} directives
such as |\childdocof| and |\childdocforward|
in the indicated file will be processed in this form.
The optional argument \textit{main}
passes on directly to the main file \textit{main}
while pretending to compile the child \textit{dest}.
This form behaves as if \textit{dest}
issues |\childdocof{|\textit{main}|}| right away,
and no further \textsf{childdoc} directives will be processed.

%%%%%%%%%%%%%%%%%%%%%%%%%%%%%%%%%%%%%%%%
\DescribeMacro{\...prefix}
In the alternative form |\childdocforwardprefix|,
%
\begin{center}
\begin{tabular}{l}
|\input{childdoc.def}|\\
|\childdocforwardprefix[|\textit{main}|]{|\textit{prefix}|}{|\textit{dest}|}|
\end{tabular}
\end{center}
%
the destination file is determined by a pattern
depending on the current file:
To make this work, the current file must be called
`{\textit{prefix}\hspace{0.2em}\textit{suffix}}'
with \textit{prefix} matching precisely the argument.
Processing is then passed on to the file
`{\textit{dest}\hspace{0.2em}\textit{suffix}}'.
Surely, the same effect is achieved by
directly specifying the
argument `{\textit{dest}\hspace{0.2em}\textit{suffix}}'
in the first form.
However, that requires to set up a different file
for each child. With the alternative form of the command
all these files can have exactly the same content
which simplifies setting them up and maintaining them.

For example, the following file |draft.tex|
with a compilation flag |\version| as described in \secref{sec:flags}
compiles the main document as a draft:
%
\begin{center}
\begin{tabular}{l}
|\def\version{draft}|\\
|\input{childdoc.def}|\\
|\childdocforward{|\textit{main}|}|
\end{tabular}
\end{center}
%
Likewise, the following files |final|\textit{nn}|.tex|
compile the final version of the child document
|child|\textit{nn}|.tex|:
%
\begin{center}
\begin{tabular}{l}
|\def\version{final}|\\
|\input{childdoc.def}|\\
|\childdocforwardprefix{final}{child}|
\end{tabular}
\end{center}
%

Note that when several versions of a main file and/or of each child file
are to be generated, it may be convenient to set up a |Makefile| or
shell script to automatise the process.

%%%%%%%%%%%%%%%%%%%%%%%%%%%%%%%%%%%%%%%%%%%%%%%%%%%%%%%%%%%%%%%%%%%%%%%%%%%%%%%%
\subsection{Command Line Processing}
\label{sec:commandline}

The effect of redirection files can also be achieved by invoking
the \LaTeX{} compiler with a more elaborate command line.
Most conveniently this should be done as part
of a shell script or a |Makefile|.

When using \textsf{childdoc} in the main file, the following
command lines effectively perform a redirection
(note that depending on the shell being used,
backslashes may have to be doubled: `|\|' $\to$ `|\\|'):
%
\begin{center}
|... -jobname "|\textit{target}|" |\\|"|[\textit{flags}]%
|\input{childdoc.def}\childdocforward[|\textit{main}|]{|\textit{dest}|}"|
\end{center}
%
Here \textit{target} is the name of the output file,
\textit{main} is the name of the main file
and \textit{dest} is the name of the main or child file to be processed
(all filenames without extensions).
The optional argument \textit{main} can be omitted
if \textit{main} matches \textit{dest}.
Optionally, compilation \textit{flags} can be defined via |\def| commands.
This command line makes the \TeX{} engine believe
it is compiling the file \textit{target}
whose content is specified as the latter parameter.
The provided code then forwards the processing to
\textit{main} or \textit{dest} as described in \secref{sec:forward}.

%%%%%%%%%%%%%%%%%%%%%%%%%%%%%%%%%%%%%%%%%%%%%%%%%%%%%%%%%%%%%%%%%%%%%%%%%%%%%%%%
\subsection{Include by Input}
\label{sec:input}

Including child documents by |\include| has some restrictions by design.
Most notably, the content of a child document always occupies
its own set of pages; pages cannot be shared between child documents.
Usually, this behaviour makes perfect sense
because each child document contain an essential part of the document.
However, in some situations it may be desirable to compose
a document from a collection of parts
without having mandatory page breaks between then.
For this case, the package
provides a mechanism to include parts
by |\input| which can also be processed individually.
However, by construction this mechanism
requires manual handling of the content to be output.

%%%%%%%%%%%%%%%%%%%%%%%%%%%%%%%%%%%%%%%%
\DescribeMacro{\ifchilddocmanual}
The main file should be prepared as usual, see \secref{sec:include}.
However, the document body must make a distinction
between processing of an individual part and of the main document, e.g.:
%
\begin{center}
\begin{tabular}{l}
|\ifchilddocmanual|\\
|\input{\childdocname}|\\
|\||else|\\
\textit{document body with }|\input{|\textit{part}|}|\\
|\||fi|
\end{tabular}
\end{center}
%
The conditional |\ifchilddocmanual| is true whenever
a part to be included by |\input| is being compiled,
and the name of the part is stored in |\childdocname|.

%%%%%%%%%%%%%%%%%%%%%%%%%%%%%%%%%%%%%%%%
\DescribeMacro{\childdocby}
Each part to be included by |\input| should start with:
%
\begin{center}
\begin{tabular}{l}
|\input{childdoc.def}|\\
|\childdocby{|\textit{main}|}|\\
\end{tabular}
\end{center}
%
The directive |\childdocby| is similar to |\childdocof|
described in \secref{sec:include},
but the subsequent selection of content must be done manually.
To that end, both |\ifchilddoc| and |\ifchilddocmanual|
will be true upon processing of a part,
and the name of the part is stored in |\childdocname|.
Note that |\jobname| will be set to the filename of the current part
so that each part receives an individual |.aux| file
that does not interfere with the |.aux| file(s) of the main document.
This behaviour can be altered by the alternative form
|\childdocby[*]{|\textit{main}|}| (with a non-empty optional argument)
which uses the |.aux| file of the main document
by setting |\jobname| to \textit{main}.

%%%%%%%%%%%%%%%%%%%%%%%%%%%%%%%%%%%%%%%%%%%%%%%%%%%%%%%%%%%%%%%%%%%%%%%%%%%%%%%%
\subsection{Driver Development}
\label{sec:driver}

The \textsf{childdoc} mechanism can also be use for the development
of definition files such as \LaTeX{} styles or classes.
This case differs from the above setup with multiple parts
included by |\include| in that no |\includeonly| should be invoked.
This can be achieved by starting the include file
(before |\ProvidesPackage|) with:
%
\begin{center}
\begin{tabular}{l}
|\input{childdoc.def}|\\
|\childdocforward{|\textit{main}|}|\\
\end{tabular}
\end{center}
%
or alternatively with:
%
\begin{center}
\begin{tabular}{l}
|\input{childdoc.def}|\\
|\childdocby{|\textit{main}|}|\\
\end{tabular}
\end{center}
%
Both forms have slightly different effects as described above.
The main file is prepared as usual, see \secref{sec:include}.

%%%%%%%%%%%%%%%%%%%%%%%%%%%%%%%%%%%%%%%%%%%%%%%%%%%%%%%%%%%%%%%%%%%%%%%%%%%%%%%%
\subsection{Legacy Detection}
\label{sec:detection}

The directive |\childdocmain| in the main file can detect
whether the complete document or merely a child is to be compiled
even without using the directive |\childdocof|.
This method is deprecated because it is less robust
and there is no compelling reason to use it;
it is merely provided for backward compatibility
and it may be removed in future versions.

If the detection mechanism is to be used,
it is mandatory to correctly specify
the filename of the main file as the argument of |\childdocmain|:
%
\begin{center}
\begin{tabular}{l}
|\input{childdoc.def}|\\
|\childdocmain{|\textit{main}|}|\\
\end{tabular}
\end{center}
%
If |\jobname| does not match the argument \textit{main} of |\childdocmain|,
it is assumed that |\jobname| points to the child file to be compiled.
When using |\childdocmain| with the main file specified as argument,
it suffices to start a child file
with just |\input{|\textit{main}|}|
without loading of the package and using |\childdocof|.
If instead all processing is done
with the appropriate \textsf{childdoc} directives,
the argument of \textit{main} of |\childdocmain| can be empty.

An alternative version of the command line processing described
in \secref{sec:commandline} using the detection mechanism reads:
%
\begin{center}
|... -jobname "|\textit{target}|" "|[\textit{flags}]%
[|\def\jobname{|\textit{dest}|}|]|\input{|\textit{main}|}"|
\end{center}

%%%%%%%%%%%%%%%%%%%%%%%%%%%%%%%%%%%%%%%%%%%%%%%%%%%%%%%%%%%%%%%%%%%%%%%%%%%%%%%%
\subsection{Manual Code}
\label{sec:manual}

In case one cannot be certain whether the definitions file |childdoc.def|
is installed on the target \TeX{} distribution
and one prefers not to ship it,
it is conceivable to paste a few relevant commands into the sources.

To that end, drop all statements |\input{childdoc.def}|
and perform the replacements as outlined below.
Instead of |\childdocmain{|\textit{main}|}| add the following code
to the top of the main file:
%
\begin{center}
\begin{tabular}{l}
|\||ifdefined\childdocname\endinput\||fi\newif\ifchilddoc|\\
|\edef\childdocname{\scantokens\expandafter{\jobname\noexpand}}|\\
|\def\childdocmain{|\textit{main}|}\||ifx\childdocmain\childdocname\||else|\\
|\childdoctrue\includeonly{\childdocname}\let\jobname\childdocmain\||fi|\\
\end{tabular}
\end{center}
%
Instead of |\childdocof{|\textit{main}|}| just include the main file
at the top of each child file:
%
\begin{center}
|\input{|\textit{main}|}|
\end{center}
%
A simple redirection |\childdocforward{|\textit{dest}|}| is achieved by:
%
\begin{center}
|\def\jobname{|\textit{dest}|}\input{\jobname}|
\end{center}
%
The redirection with prefix
|\childdocforwardprefix[|\textit{prefix}|]{|\textit{dest}|}|
is accomplished by:
%
\begin{center}
\begin{tabular}{l}
|{\edef\jobname{\scantokens\expandafter{\jobname\noexpand}}|\\
|\def\redirectjob |\textit{prefix}|#1~~~{\gdef\jobname{|\textit{dest}|#1}}|\\
|\expandafter\redirectjob\jobname~~~}\input{\jobname}|
\end{tabular}
\end{center}

In an alternative approach,
child documents can be compiled by a specific command line
without additional code or specific definitions:
%
\begin{center}
|... -jobname "|\textit{target}|" "|[\textit{flags}]%
|\includeonly{|\textit{dest}|}\input{|\textit{main}|}"|
\end{center}
%

%%%%%%%%%%%%%%%%%%%%%%%%%%%%%%%%%%%%%%%%%%%%%%%%%%%%%%%%%%%%%%%%%%%%%%%%%%%%%%%%
%%%%%%%%%%%%%%%%%%%%%%%%%%%%%%%%%%%%%%%%%%%%%%%%%%%%%%%%%%%%%%%%%%%%%%%%%%%%%%%%
\section{Information}

%%%%%%%%%%%%%%%%%%%%%%%%%%%%%%%%%%%%%%%%%%%%%%%%%%%%%%%%%%%%%%%%%%%%%%%%%%%%%%%%
\subsection{Copyright}

Copyright \copyright{} 2017--2018 Niklas Beisert

This work may be distributed and/or modified under the
conditions of the \LaTeX{} Project Public License, either version 1.3
of this license or (at your option) any later version.
The latest version of this license is in
  \url{http://www.latex-project.org/lppl.txt}
and version 1.3 or later is part of all distributions of \LaTeX{}
version 2005/12/01 or later.

This work has the LPPL maintenance status `maintained'.

The Current Maintainer of this work is Niklas Beisert.

This work consists of the files |README.txt|, |childdoc.ins| and |childdoc.dtx|
as well as the derived files |childdoc.def|, |cdocsamp.tex|
with |cdocsch1.tex|, |cdocsch2.tex|, |cdocspt3.tex|, |cdocspt4.tex|,
|cdocsdrf.tex|, |cdocsfn1.tex|, |cdocsfn2.tex|
as well as |childdoc.pdf|.

%%%%%%%%%%%%%%%%%%%%%%%%%%%%%%%%%%%%%%%%%%%%%%%%%%%%%%%%%%%%%%%%%%%%%%%%%%%%%%%%
\subsection{Files and Installation}

The package consists of the files:
%
\begin{center}
\begin{tabular}{ll}
    |README.txt|   & readme file \\
    |childdoc.ins| & installation file \\
    |childdoc.dtx| & source file \\
    |childdoc.def| & definition file \\
    |cdocsamp.tex| & sample main file \\
    |cdocsch1.tex| & sample include file \\
    |cdocsch2.tex| & sample include file \\
    |cdocspt3.tex| & sample part file \\
    |cdocspt4.tex| & sample part file \\
    |cdocsdrf.tex| & sample redirection file \\
    |cdocsfn1.tex| & sample redirection file \\
    |cdocsfn2.tex| & sample redirection file \\
    |childdoc.pdf| & manual
\end{tabular}
\end{center}
%
The distribution consists of the files
|README.txt|, |childdoc.ins| and |childdoc.dtx|.
%
\begin{itemize}
\item
Run (pdf)\LaTeX{} on |childdoc.dtx|
to compile the manual |childdoc.pdf| (this file).
\item
Run \LaTeX{} on |childdoc.ins| to create the definitions file |childdoc.def|
and the sample |cdocsamp.tex| with include files
|cdocsch1.tex|, |cdocsch2.tex|, |cdocspt3.tex|, |cdocspt4.tex|,
|cdocsdrf.tex|, |cdocsfn1.tex|, |cdocsfn2.tex|.
Then copy the file |childdoc.def| to an appropriate directory of your \LaTeX{}
distribution, e.g.\ \textit{texmf-root}|/tex/latex/childdoc|.
\end{itemize}

%%%%%%%%%%%%%%%%%%%%%%%%%%%%%%%%%%%%%%%%%%%%%%%%%%%%%%%%%%%%%%%%%%%%%%%%%%%%%%%%
\subsection{Related CTAN Packages}

There are several other packages which offer a similar functionality:
%
\begin{itemize}
\item
The packages
\href{http://ctan.org/pkg/docmute}{\textsf{docmute}},
\href{http://ctan.org/pkg/includex}{\textsf{includex}} and
\href{http://ctan.org/pkg/standalone}{\textsf{standalone}}
provide commands to include only the document body of
a child file thus allowing both files to be compiled individually.
\item
The packages \href{http://ctan.org/pkg/subdocs}{\textsf{subdocs}}
and \href{http://ctan.org/pkg/subfiles}{\textsf{subfiles}}
provide structures in which the main and child documents can be
encapsulated and allowing them to be compiled individually.
The inclusion mechanism is different from the conventional |\include|.
\item
The package \href{http://ctan.org/pkg/combine}{\textsf{combine}}
is an elaborate solution to combine several documents into one.
\end{itemize}
%
See also the CTAN topic \href{http://ctan.org/topic/subdocs}{\textsf{subdocs}}
for further related packages.
The present package differs from the above solutions in that
a document structure constructed with the conventional |\include| mechanism
just needs two extra commands at the top of every file
such that all constituent files can be compiled individually.

%%%%%%%%%%%%%%%%%%%%%%%%%%%%%%%%%%%%%%%%%%%%%%%%%%%%%%%%%%%%%%%%%%%%%%%%%%%%%%%%
%\subsection{Feature Suggestions}
%
%The following is a list of features which may be useful for future
%versions of this package:
%%
%\begin{itemize}
%\item
%\ldots
%\end{itemize}

%%%%%%%%%%%%%%%%%%%%%%%%%%%%%%%%%%%%%%%%%%%%%%%%%%%%%%%%%%%%%%%%%%%%%%%%%%%%%%%%
\subsection{Revision History}

%%%%%%%%%%%%%%%%%%%%%%%%%%%%%%%%%%%%%%%%
\paragraph{v2.0:} 2018/12/30

\begin{itemize}
\item
immediate forward processing
\item
added |\childdocby| mechanism
\item
manual restructured
\end{itemize}

%%%%%%%%%%%%%%%%%%%%%%%%%%%%%%%%%%%%%%%%
\paragraph{v1.6:} 2018/01/17

\begin{itemize}
\item
application for development of include files
\item
corrections to manual
\end{itemize}

%%%%%%%%%%%%%%%%%%%%%%%%%%%%%%%%%%%%%%%%
\paragraph{v1.5:} 2017/05/21

\begin{itemize}
\item
more complete structuring introduced
\item
|\childdocof| introduced
\item
|\childdoc| renamed to |\childdocmain|
\item
|\childredirect| renamed to |\childdocforward| and |\childdocforwardprefix|
and functionality expanded
\end{itemize}

%%%%%%%%%%%%%%%%%%%%%%%%%%%%%%%%%%%%%%%%
\paragraph{v1.0:} 2017/04/27

\begin{itemize}
\item
manual and install package
\item
first version published on CTAN
\end{itemize}

%%%%%%%%%%%%%%%%%%%%%%%%%%%%%%%%%%%%%%%%
\paragraph{v0.6:} 2017/04/26

\begin{itemize}
\item
redirection mechanism added
\end{itemize}

%%%%%%%%%%%%%%%%%%%%%%%%%%%%%%%%%%%%%%%%
\paragraph{v0.5:} 2017/04/26

\begin{itemize}
\item
functionality in definition file
\end{itemize}


%%%%%%%%%%%%%%%%%%%%%%%%%%%%%%%%%%%%%%%%%%%%%%%%%%%%%%%%%%%%%%%%%%%%%%%%%%%%%%%%
%%%%%%%%%%%%%%%%%%%%%%%%%%%%%%%%%%%%%%%%%%%%%%%%%%%%%%%%%%%%%%%%%%%%%%%%%%%%%%%%
%%%%%%%%%%%%%%%%%%%%%%%%%%%%%%%%%%%%%%%%%%%%%%%%%%%%%%%%%%%%%%%%%%%%%%%%%%%%%%%%
\appendix

\settowidth\MacroIndent{\rmfamily\scriptsize 000\ }

 \DocInput{childdoc.dtx}

\end{document}
%</driver>
% \fi
%
% %%%%%%%%%%%%%%%%%%%%%%%%%%%%%%%%%%%%%%%%%%%%%%%%%%%%%%%%%%%%%%%%%%%%%%%%%%%%%%
% %%%%%%%%%%%%%%%%%%%%%%%%%%%%%%%%%%%%%%%%%%%%%%%%%%%%%%%%%%%%%%%%%%%%%%%%%%%%%%
% \section{Sample}
%\iffalse
%<*samplemain>
%\fi
%
% The following presents a sample document
% with two chapters, two parts, a title page,
% a compile flag as well as three forwarding files to set the flag.
% It consists of eight |.tex| files:
% \begin{center}
% \begin{tabular}{ll}
% |cdocsamp.tex|&main file\\
% |cdocsch1.tex|&include file for chapter 1\\
% |cdocsch2.tex|&include file for chapter 2\\
% |cdocspt3.tex|&include file for part 3\\
% |cdocspt4.tex|&include file for part 4\\
% |cdocsdrf.tex|&forwarding file for main file in draft mode\\
% |cdocsfi1.tex|&forwarding file for final version of chapter 1\\
% |cdocsfi2.tex|&forwarding file for final version of chapter 2\\
% \end{tabular}
% \end{center}
% Each of the eight files can be compiled directly by the \LaTeX{} compiler.
%
% %%%%%%%%%%%%%%%%%%%%%%%%%%%%%%%%%%%%%%
% \paragraph{Main File.}
%
% The main file is called |cdocsamp.tex|.
%
% Load the \textsf{childdoc} definitions and
% declare the filename for the main document:
%    \begin{macrocode}
\input{childdoc.def}
\childdocmain{}
%    \end{macrocode}

% Optional override for |\version| flag:
%    \begin{macrocode}
%%\ifchilddoc\else\providecommand{\version}{draft}\fi
%    \end{macrocode}

% Define the default values for the |\version| flag
% (|final| for the main file and |draft| for childs):
%    \begin{macrocode}
\ifchilddoc
\providecommand{\version}{draft}
\else
\providecommand{\version}{final}
\fi
%    \end{macrocode}

% Load the standard document class:
%    \begin{macrocode}
\documentclass[12pt]{article}
%    \end{macrocode}

% Start the document body:
%    \begin{macrocode}
\begin{document}
%    \end{macrocode}

% Declare a title page.
% Print title, part of document being processed and version flag:
%    \begin{macrocode}
\addtocounter{page}{-1}
\begin{center}
{\LARGE\bfseries{}childdoc example\par}
\vspace{1cm}
\ifchilddoc
\ifchilddocmanual part\else chapter\fi:
`\childdocname' of `\childdocjob'\par
\else
main document: `\childdocjob'\par
\fi
version: \version\par
\end{center}
\newpage
%    \end{macrocode}

% Manually include selected file,
% otherwise process as usual:
%    \begin{macrocode}
\ifchilddocmanual
\section*{part `\childdocname'}
\input{\childdocname}
\else
%    \end{macrocode}

% Include the two chapters:
%    \begin{macrocode}
\include{cdocsch1}
\include{cdocsch2}
%    \end{macrocode}

% Include the two parts unless only chapters should be displayed:
%    \begin{macrocode}
\ifchilddoc\else
\section{part three}
\input{cdocspt3}
\section{part four}
\input{cdocspt4}
\fi
%    \end{macrocode}

% Process as usual until here:
%    \begin{macrocode}
\fi
%    \end{macrocode}

% End of document body:
%    \begin{macrocode}
\end{document}
%    \end{macrocode}
%\iffalse
%</samplemain>
%\fi
%
% %%%%%%%%%%%%%%%%%%%%%%%%%%%%%%%%%%%%%%
% \paragraph{Chapter Include Files.}
%
% The include files are called |cdocsch1.tex| and |cdocsch2.tex|.
%
%\iffalse
%<*samplechap1|samplechap2>
%\fi

% Optional override for |\version| flag:
%    \begin{macrocode}
%%\providecommand{\version}{final}
%    \end{macrocode}

% Include the main document:
%    \begin{macrocode}
\input{childdoc.def}
\childdocof{cdocsamp}
%    \end{macrocode}

%\iffalse
%</samplechap1|samplechap2>
%\fi
%
%\iffalse
%<*samplechap1>
%\fi
% Some text for chapter 1:
%    \begin{macrocode}
\section{one}
some text in chapter one
%    \end{macrocode}

%\iffalse
%</samplechap1>
%\fi
% Some text for chapter 2:
%\iffalse
%<*samplechap2>
%\fi
%    \begin{macrocode}
\section{two}
more text in chapter two
%    \end{macrocode}

%\iffalse
%</samplechap2>
%\fi
%
% %%%%%%%%%%%%%%%%%%%%%%%%%%%%%%%%%%%%%%
% \paragraph{Part Include Files.}
%
% The include files are called |cdocspt3.tex| and |cdocspt4.tex|.
%
%\iffalse
%<*samplepart3|samplepart4>
%\fi

% Optional override for |\version| flag:
%    \begin{macrocode}
%%\providecommand{\version}{final}
%    \end{macrocode}

% Include the main document:
%    \begin{macrocode}
\input{childdoc.def}
\childdocby{cdocsamp}
%    \end{macrocode}

%\iffalse
%</samplepart3|samplepart4>
%\fi
%
%\iffalse
%<*samplepart3>
%\fi
% Some text for part 3:
%    \begin{macrocode}
some text in part three
%    \end{macrocode}

%\iffalse
%</samplepart3>
%\fi
% Some text for part 4:
%\iffalse
%<*samplepart4>
%\fi
%    \begin{macrocode}
more text in part four
%    \end{macrocode}

%\iffalse
%</samplepart4>
%\fi
%
% %%%%%%%%%%%%%%%%%%%%%%%%%%%%%%%%%%%%%%
% \paragraph{Forwarding for a Complete Draft.}
%
% The following forwarding file |cdocsdrf.tex|
% compiles the main document in draft mode:
%\iffalse
%<*sampledraft>
%\fi
%    \begin{macrocode}
\def\version{draft}
\input{childdoc.def}
\childdocforward{cdocsamp}
%    \end{macrocode}

%\iffalse
%</sampledraft>
%\fi
%
% %%%%%%%%%%%%%%%%%%%%%%%%%%%%%%%%%%%%%%
% \paragraph{Forwarding for Final Version of the Chapters.}
%
% The following forwarding files |cdocsfn1.tex| and |cdocsfn2.tex|
% (with identical content)
% compile the final versions of the child documents
% |cdocsch1.tex| and |cdocsch2.tex|, respectively:
%\iffalse
%<*samplefinal>
%\fi
%    \begin{macrocode}
\def\version{final}
\input{childdoc.def}
\childdocforwardprefix[cdocsamp]{cdocsfn}{cdocsch}
%    \end{macrocode}

%\iffalse
%</samplefinal>
%\fi
%
% %%%%%%%%%%%%%%%%%%%%%%%%%%%%%%%%%%%%%%
% \paragraph{Command Line Processing.}
%
% The following three command lines generate the output files
% |cdocscld|, |cdocscl1| and |cdocscl2|
% which should be identical to
% |cdocsdrf|, |cdocsch1| and |cdocsfn2|, respectively:
% \begin{center}
% \begin{tabular}{l}
% |latex -jobname cdocscld \|\\
% |  "\def\version{draft}\input{childdoc.def}\childdocforward{cdocsamp}"|\\
% |latex -jobname cdocscl1 \|\\
% |  "\input{childdoc.def}\childdocforward[cdocsamp]{cdocsch1}"|\\
% |latex -jobname cdocscl2 \|\\
% |  "\def\version{final}\input{childdoc.def}\childdocforward{cdocsch2}"|
% \end{tabular}
% \end{center}
% Note that the trailing backslash on each first line
% merely continues the input to the second line
% (for convenient cut ant paste).
% Furthermore, the command |latex| can be replaced by any
% of its alternative versions such as |pdflatex|.
%
% %%%%%%%%%%%%%%%%%%%%%%%%%%%%%%%%%%%%%%%%%%%%%%%%%%%%%%%%%%%%%%%%%%%%%%%%%%%%%%
% %%%%%%%%%%%%%%%%%%%%%%%%%%%%%%%%%%%%%%%%%%%%%%%%%%%%%%%%%%%%%%%%%%%%%%%%%%%%%%
% \section{Implementation}
%\iffalse
%<*package>
%\fi
%
% This section describes the definitions file |childdoc.def|.

% The definitions cannot be loaded using |\usepackage| or |\RequirePackage|
% which has a mechanism to prevent loading a style file more than once.
% When loading the definitions by means of |\input|
% multiple instances have to be prevented manually:
%\iffalse
%This code needs to be before the `\ProvidesFile' directive
%which is defined at the beginning of this file.
%Therefore it is also placed there and commented out here.
%</package>
%<*discard>
%\fi
%    \begin{macrocode}
\ifdefined\childdocmain\endinput\fi
%    \end{macrocode}
%\iffalse
%</discard>
%<*package>
%\fi
%
% \macro{\ifchilddoc}
% \macro{\ifchilddocmanual}
% The conditional |\ifchilddoc| tells whether a
% child (true) or main (false) document is being compiled.
% The conditional |\ifchilddocmanual| tells whether
% the |\includeonly| mechanism is used (false) or
% the selection of child files must be performed manually (true).
% The definitions initialise to false:
%    \begin{macrocode}
\newif\ifchilddoc
\newif\ifchilddocmanual
%    \end{macrocode}

% \macro{\childdocname}
% \macro{\childdocjob}
% The macro |\childdocname| stores the name of the main document
% to be compiled. The macro |\childdocjob| stores the name of
% the document on which the \LaTeX{} compiler was originally invoked.
% The content of |\jobname| cannot be compared
% to filenames specified in the source due to different catcodes.
% The following code rescans |\jobname|, stores the result
% in |\childdocname| and saves a copy in |\childdocjob|:
%    \begin{macrocode}
\edef\childdocname{\scantokens\expandafter{\jobname\noexpand}}
\let\childdocjob\childdocname
%    \end{macrocode}

% \macro{\childdocdisable}
% The macro |\childdocdisable| prevents the main file
% from being processed more than once.
% At this stage, the main document command |\childdocmain|
% is assumed to be called once again where it should do nothing.
% Any subsequent call to it should prevent
% a secondary processing of the main document
% It overwrites the forwarding commands
% |\childdocof| and |\childdocforward|
% with empty macros to prevent further inclusions of the main document:
%    \begin{macrocode}
\newcommand{\childdocdisable}
{
  \renewcommand{\childdocmain}[1]{\renewcommand{\childdocmain}[1]{\endinput}}
  \renewcommand{\childdocof}[1]{}
  \renewcommand{\childdocby}[2][]{}
  \renewcommand{\childdocforward}[2][]{}
  \renewcommand{\childdocdisable}{}
}
%    \end{macrocode}

% \macro{\childdocmain}
% The macro |\childdocmain| is to be called at the top of the main file
% with nothing or the main filename (without extension) as argument.
% First, it breaks loops.
% If the argument is not empty and does not match |\childdocname|
% (which is set by the first inclusion of |childdoc.def|),
% |\ifchilddoc| is set to true, |\includeonly| is applied to the child file
% and |\jobname| is set to the main file
% (for proper handling of |.aux| files):
%    \begin{macrocode}
\newcommand{\childdocmain}[1]
{
  \childdocdisable\childdocmain{}
  \if?#1?\else
    \begingroup
      \def\childdoctmp{#1}
      \ifx\childdoctmp\childdocname
        \def\childdoctmp{}
      \else
        \def\childdoctmp
        {
          \childdoctrue
          \includeonly{\childdocname}
          \def\childdocjob{#1}
          \def\jobname{#1}
        }
      \fi
      \expandafter
    \endgroup
    \childdoctmp
  \fi
}
%    \end{macrocode}

% \macro{\childdocof}
% The command |\childdocof| redirects
% compilation to the main file |#1|.
%    \begin{macrocode}
\newcommand{\childdocof}[1]
{
  \childdocdisable
  \childdoctrue
  \includeonly{\childdocname}
  \def\jobname{#1}
  \def\childdocjob{#1}
  \input{#1}
}
%    \end{macrocode}

% \macro{\childdocby}
% The command |\childdocby| ....
%    \begin{macrocode}
\newcommand{\childdocby}[2][]
{
  \childdocdisable
  \childdoctrue
  \childdocmanualtrue
  \if?#1?\else
    \def\jobname{#2}
  \fi
  \def\childdocjob{#2}
  \input{#2}
  \endinput
}
%    \end{macrocode}

% \macro{\childdocforward}
% The command |\childdocforward| redirects
% compilation to the main file or
% (if the optional argument is given) a child file.
% Parameters are set as if the main file
% or a child file starting with |\childdocof| was compiled.
% Then compilation is handed over to the main file:
%    \begin{macrocode}
\newcommand{\childdocforward}[2][]
{
  \begingroup
    \if?#1?
      \def\childdoctmp
      {
        \def\childdocname{#2}
        \def\childdocjob{#2}
        \def\jobname{#2}
        \input{#2}
        \endinput
      }
    \else
      \def\childdoctmp
      {
        \childdocdisable
        \def\childdocname{#2}
        \childdoctrue
        \includeonly{#2}
        \def\childdocjob{#1}
        \def\jobname{#1}
        \input{#1}
        \endinput
      }
    \fi
    \expandafter
  \endgroup
  \childdoctmp
}
%    \end{macrocode}

% \macro{\childdocforwardprefix}
% The command |\childdocforwardprefix| redirects
% compilation to the main or a child file by means of a pattern.
% The prefix |#1| in the current filename is replaced by |#2|
% and the suffix of the current filename is kept
% (it is assumed that the filename does not contain the substring `|~~~|'
% which is used as a delimiter).
% Compilation is handed over to the new file by |\childdocforward|:
%    \begin{macrocode}
\newcommand{\childdocforwardprefix}[3][]
{
  \begingroup
    \def\childdocextract #2##1~~~{\def\childdoctmp{\childdocforward[#1]{#3##1}}}
    \expandafter\childdocextract\childdocname~~~
    \expandafter
  \endgroup
  \childdoctmp
}
%    \end{macrocode}

% \macro{\childdoc}
% The deprecated macro |\childdoc| is a legacy version of |\childdocmain|:
%    \begin{macrocode}
\newcommand{\childdoc}{\childdocmain}
%    \end{macrocode}

% \macro{\childdocredirect}
% The deprecated macro |\childdocredirect| is a legacy version
% of |\childdocforward| and |\childdocforwardprefix|:
%    \begin{macrocode}
\newcommand{\childdocredirect}[2][]
{
  \begingroup
    \if?#1?
      \def\childdoctmp{\childdocforward{#2}}
    \else
      \def\childdoctmp{\childdocforwardprefix{#1}{#2}}
    \fi
    \expandafter
  \endgroup
  \childdoctmp
}
%    \end{macrocode}

%\iffalse
%</package>
%\fi
%
\endinput
|\\
|\childdocforward[|\textit{main}|]{|\textit{dest}|}|\\
\end{tabular}
\end{center}
%
The argument \textit{dest} is the destination file
(without extension).
It should be the main file or one of the child files.
Note that further \textsf{childdoc} directives
such as |\childdocof| and |\childdocforward|
in the indicated file will be processed in this form.
The optional argument \textit{main}
passes on directly to the main file \textit{main}
while pretending to compile the child \textit{dest}.
This form behaves as if \textit{dest}
issues |\childdocof{|\textit{main}|}| right away,
and no further \textsf{childdoc} directives will be processed.

%%%%%%%%%%%%%%%%%%%%%%%%%%%%%%%%%%%%%%%%
\DescribeMacro{\...prefix}
In the alternative form |\childdocforwardprefix|,
%
\begin{center}
\begin{tabular}{l}
|% \iffalse
%
% childdoc.dtx Copyright (C) 2017-2018 Niklas Beisert
%
% This work may be distributed and/or modified under the
% conditions of the LaTeX Project Public License, either version 1.3
% of this license or (at your option) any later version.
% The latest version of this license is in
%   http://www.latex-project.org/lppl.txt
% and version 1.3 or later is part of all distributions of LaTeX
% version 2005/12/01 or later.
%
% This work has the LPPL maintenance status `maintained'.
%
% The Current Maintainer of this work is Niklas Beisert.
%
% This work consists of the files childdoc.dtx and childdoc.ins
% and the derived files childdoc.def and cdocsamp.tex with
% cdocsch1.tex, cdocsch2.tex, cdocsdrf.tex, cdocsfn1.tex, cdocsfn2.tex.
%
%<package>\ifdefined\childdocmain\endinput\fi
%<package>\ProvidesFile{childdoc.def}[2018/12/30 v2.0 child document driver]
%<samplemain>\ProvidesFile{cdocsamp.tex}[2018/12/30 v2.0 sample for childdoc]
%<*driver>
%\ProvidesFile{childdoc.drv}[2018/12/30 v2.0 childdoc reference manual file]
\PassOptionsToClass{10pt,a4paper}{article}
\documentclass{ltxdoc}

\usepackage[margin=35mm]{geometry}
\usepackage{hyperref}
\usepackage{hyperxmp}
\usepackage[usenames]{color}

\hypersetup{colorlinks=true}
\hypersetup{pdfstartview=FitH}
\hypersetup{pdfpagemode=UseNone}
\hypersetup{pdfsource={}}
\hypersetup{pdflang={en-UK}}
\hypersetup{pdfcopyright={Copyright 2017-2018 Niklas Beisert.
  This work may be distributed and/or modified under the
  conditions of the LaTeX Project Public License, either version 1.3
  of this license or (at your option) any later version.}}
\hypersetup{pdflicenseurl={http://www.latex-project.org/lppl.txt}}
\hypersetup{pdfcontactaddress={ETH Zurich, ITP, HIT K,
  Wolfgang-Pauli-Strasse 27}}
\hypersetup{pdfcontactpostcode={8093}}
\hypersetup{pdfcontactcity={Zurich}}
\hypersetup{pdfcontactcountry={Switzerland}}
\hypersetup{pdfcontactemail={nbeisert@itp.phys.ethz.ch}}
\hypersetup{pdfcontacturl={http://people.phys.ethz.ch/\xmptilde nbeisert/}}

\newcommand{\secref}[1]{\hyperref[#1]{section \ref*{#1}}}

\parskip1ex
\parindent0pt
\let\olditemize\itemize
\def\itemize{\olditemize\parskip0pt}

\begin{document}

\title{The \textsf{childdoc} Package}
\hypersetup{pdftitle={The childdoc Package}}
\author{Niklas Beisert\\[2ex]
  Institut f\"ur Theoretische Physik\\
  Eidgen\"ossische Technische Hochschule Z\"urich\\
  Wolfgang-Pauli-Strasse 27, 8093 Z\"urich, Switzerland\\[1ex]
  \href{mailto:nbeisert@itp.phys.ethz.ch}
  {\texttt{nbeisert@itp.phys.ethz.ch}}}
\hypersetup{pdfauthor={Niklas Beisert}}
\hypersetup{pdfsubject={Manual for the LaTeX2e Package childdoc}}
\date{30 December 2018, \textsf{v2.0}}
\maketitle

\begin{abstract}\noindent
\textsf{childdoc} is a \LaTeXe{} package
that enables the direct compilation
of document sections included by |\include|
to individual files.
\end{abstract}

\begingroup
\parskip0ex
\tableofcontents
\endgroup

%%%%%%%%%%%%%%%%%%%%%%%%%%%%%%%%%%%%%%%%%%%%%%%%%%%%%%%%%%%%%%%%%%%%%%%%%%%%%%%%
%%%%%%%%%%%%%%%%%%%%%%%%%%%%%%%%%%%%%%%%%%%%%%%%%%%%%%%%%%%%%%%%%%%%%%%%%%%%%%%%
\section{Introduction}

\LaTeX{} provides a mechanism to structure a large document (such as a book)
into a main file and several child files (containing the chapters)
using the |\include| command.
This mechanism is beneficial for documents
which span hundreds of pages in order to
make the source file(s) more manageable.
Moreover, compilation can be restricted to
selected child files by means of the |\includeonly| command.
The latter feature can be used to reduce the compilation time while editing
(this was significantly more useful in the earlier days of \LaTeX{})
or to generate a smaller document which is easier to navigate.
Another application of |\includeonly| is to generate
documents consisting of selected parts of the complete document.

However, there are a few drawbacks of the plain |\include| mechanism:
\begin{itemize}
\item
The child files cannot be compiled on their own,
they can only be compiled via the main file.
A naive editing environment
(such as a text editor with an option
to have the current file processed by \LaTeX)
may require one to switch to the main file before compiling;
attempting to compile the child file produces errors.
\item
The main file must be modified (each time)
to adjust the |\includeonly| command
to the present needs. This easily leaves the main file in a messy state.
\item
The generated document will always carry the filename
of the main document. This is inconvenient if
several child files are to be compiled and
to be kept for distribution.
\end{itemize}

The present package provides a simple interface
to make child files individually compilable by \LaTeX{}.
Compiling a child file then has the same effect as compiling
the main file with an |\includeonly| command
to select the appropriate child.
Moreover the generated document will carry the name of the child
rather than the main file.
This resolves all three above issues.

This feature is meant to make the editing of books,
thesis documents and lecture notes somewhat more convenient.
However, the package can also be used efficiently for
composing a series of documents (such as exercise sheets)
which are typically distributed individually.
It then assists the author in generating the individual documents
(potentially in different versions)
as well as a document containing the collected series.
Another application is in developing style files
or other kinds of included material
where compilation of the style file could redirect
to a sample or test file.

%%%%%%%%%%%%%%%%%%%%%%%%%%%%%%%%%%%%%%%%%%%%%%%%%%%%%%%%%%%%%%%%%%%%%%%%%%%%%%%%
%%%%%%%%%%%%%%%%%%%%%%%%%%%%%%%%%%%%%%%%%%%%%%%%%%%%%%%%%%%%%%%%%%%%%%%%%%%%%%%%
\section{Usage}

First of all, the package \textsf{childdoc} is \emph{not} a standard
\LaTeXe{} |.sty| style file! Therefore it needs to be invoked in
a non-standard way.

%%%%%%%%%%%%%%%%%%%%%%%%%%%%%%%%%%%%%%%%%%%%%%%%%%%%%%%%%%%%%%%%%%%%%%%%%%%%%%%%
\subsection{Included Files}
\label{sec:include}

%%%%%%%%%%%%%%%%%%%%%%%%%%%%%%%%%%%%%%%%
\DescribeMacro{\childdocmain}
To use the package, add the commands
\begin{center}
\begin{tabular}{l}
|\input{childdoc.def}|\\
|\childdocmain{}|\\
\end{tabular}
\end{center}
at the very top of the main \LaTeX{} file,
in particular \emph{before} the |\documentclass| statement!
The argument of |\childdocmain| should be left empty
(but it must be present).

%%%%%%%%%%%%%%%%%%%%%%%%%%%%%%%%%%%%%%%%
\DescribeMacro{\childdocof}
Furthermore, add the commands
\begin{center}
\begin{tabular}{l}
|\input{childdoc.def}|\\
|\childdocof{|\textit{main}|}|\\
\end{tabular}
\end{center}
at the top of every child file \textit{child}
which is included by |\include{|\textit{child}|}|
from within the main file
(or at least for those files to be compiled individually).
The argument \textit{main} must be the filename of the main file.

There are a couple of
considerations in setting up the main and child documents:

%%%%%%%%%%%%%%%%%%%%%%%%%%%%%%%%%%%%%%%%
\paragraph{Restrictions.}

Please note the following restrictions:
\begin{itemize}
\item
|\childdocmain| must be called with one argument \textit{main}
to ensure compatibility with earlier version of the package.
It must either be empty (|\childdocmain{}|)
or precisely match the filename of the main file in which it is specified.
See \secref{sec:detection} for further information.
\item
The filename \textit{main} must be specified without the |.tex| extension.
\item
The filename \textit{main} is case sensitive
(even in case-insensitive file systems)
due to internal string comparison.
\item
The argument \textit{main} should be fully expanded, it cannot be a macro.
\item
Subdirectories and special characters should be avoided in filenames.
\item
The command |\childdocmain{|\textit{main}|}| must be followed by a whitespace.
It should not be followed immediately by another command
or by a comment mark `|%|'.
This is because the \TeX{} parser reads the token immediately following
the argument of |\childdocmain| and puts it
at the beginning of every child section;
however, a white\-space is ignored.
\end{itemize}

%%%%%%%%%%%%%%%%%%%%%%%%%%%%%%%%%%%%%%%%
\paragraph{Content of Main File.}

It is advisable to place all content in the child files included by |\include|.
Any output contained in the main file will appear in all child documents
unless suppressed manually;
it cannot be suppressed automatically by the |\includeonly| directive
and thus should normally be avoided.
A method to include some content in the main file
by means of conditional processing is described in \secref{sec:conditional}.

%%%%%%%%%%%%%%%%%%%%%%%%%%%%%%%%%%%%%%%%
\paragraph{Page Numbering.}

When only a part of the document is compiled,
the appropriate numbering of pages
(as well as other status parameters)
is determined from the |.aux| files.
The latter contain information from previous passes.
However this information needs to propagate through
all intermediate child documents.
Therefore the page numbering in child documents may well
be inconsistent until the complete document is compiled at least once.

A useful (if unconventional) way to always ensure a consistent
page numbering is to restart the numbering in each child document
and denote the pages by `\textit{child}|.|\textit{page}'
where \textit{child} represents the chapter/section number of the child file.
This can be achieved by the command
|\numberwithin{page}{|\textit{child}|}|
of the \textsf{amsmath} package
where \textit{child} can be |chapter| or |section|
depending on the chosen structuring.
Alternatively, one can modify the macro |\thepage| appropriately
and reset the counter |page| at the start of each child file.

%%%%%%%%%%%%%%%%%%%%%%%%%%%%%%%%%%%%%%%%%%%%%%%%%%%%%%%%%%%%%%%%%%%%%%%%%%%%%%%%
\subsection{Conditional Processing}
\label{sec:conditional}

The package provides a mechanism to compile different versions
of a document. To customise the versions further some conditional processing
can come in handy to distinguish which version is being compiled.
The package provides two macros to describe the compilation context:

%%%%%%%%%%%%%%%%%%%%%%%%%%%%%%%%%%%%%%%%
\DescribeMacro{\ifchilddoc}
The conditional |\ifchilddoc| distinguishes between the compilation of
child documents and the main document:
%
\begin{center}
|\ifchilddoc |\textit{child-code}| |[|\||else |\textit{main-code}]| \||fi|
\end{center}

%%%%%%%%%%%%%%%%%%%%%%%%%%%%%%%%%%%%%%%%
\DescribeMacro{\childdocname}
\DescribeMacro{\childdocjob}
The macro |\childdocname| contains the filename (without extension)
of the main or child file being processed.
Note that |\childdocjob| will always contain the name of the main file.

%%%%%%%%%%%%%%%%%%%%%%%%%%%%%%%%%%%%%%%%
\paragraph{Title Page.}

Conditional processing can be used to include a title or banner page
in the main document when proper precautions are taken.
Importantly, the code in the main file should ensure that the page counter
(as well as other status parameters which are stored in the |.aux| files)
takes the same value after the conditional processing.
Otherwise the page numbers may take divergent values
depending on which part is compiled.

For example, a title page could be declared by:
%
\begin{center}
\begin{tabular}{l}
|\ifchilddoc\||else|\\
|\addtocounter{page}{-1}|\\
\textit{code for title page}\\
|\newpage|\\
|\||fi|
\end{tabular}
\end{center}
%
A banner page for the child documents can be generated by:
%
\begin{center}
\begin{tabular}{l}
|\ifchilddoc|\\
|\addtocounter{page}{-1}|\\
\textit{code for banner page}\\
|\newpage|\\
|\||fi|
\end{tabular}
\end{center}
%
Here one could write a message such as:
\begin{center}
|This is the part \childdocname{} of \childdocjob{}.|
\end{center}

%%%%%%%%%%%%%%%%%%%%%%%%%%%%%%%%%%%%%%%%%%%%%%%%%%%%%%%%%%%%%%%%%%%%%%%%%%%%%%%%
\subsection{Flags}
\label{sec:flags}

The package makes it easy to generate different versions
of the main or child documents.
To this end compilation flags can be defined
and assigned different default values.
They will be particularly useful in conjunction
with the forwarding mechanism described in \secref{sec:forward}.

For example, it may be useful to have a flag |\version|
which can be set to |draft| or |final|.
The document source will contain some conditional code
depending on the value of |\version|.
Suppose further, the flag should default to |final| for the main file
and to |draft| for child files
which is a natural assignment for editing the document.
This is achieved by placing the following code
in the preamble of the main document
(below the |\childdocmain| directive):
%
\begin{center}
\begin{tabular}{l}
|\ifchilddoc|\\
|\providecommand{\version}{draft}|\\
|\||else|\\
|\providecommand{\version}{final}|\\
|\||fi|
\end{tabular}
\end{center}
%
The definition by |\providecommand| makes sure
that previous definitions are not overwritten.
Further statements |\providecommand{\version}{...}|
can thus be added before the above code to override it.

For the main file, one might add a line
(between |\childdocmain| and the above block)
%
\begin{center}
|%\ifchilddoc\||else\providecommand{\version}{draft}\||fi|
\end{center}
%
which can be uncommented to produce a draft version.
Likewise one can add a line to the very top of a child file
(above the |\childdocof{|\textit{main}|}| directive)
%
\begin{center}
|%\providecommand{\version}{final}|
\end{center}
%
which can be uncommented to produce the final version of this child document.

%%%%%%%%%%%%%%%%%%%%%%%%%%%%%%%%%%%%%%%%%%%%%%%%%%%%%%%%%%%%%%%%%%%%%%%%%%%%%%%%
\subsection{Forwarding}
\label{sec:forward}

Different versions of the main or child documents
using compilation flags as described in \secref{sec:flags}
can be (permanently) stored in different files
for convenient compilation, viewing and distribution.
To this end, the package defines a command
to pass on compilation to a different file:

%%%%%%%%%%%%%%%%%%%%%%%%%%%%%%%%%%%%%%%%
\DescribeMacro{\childdocforward}
The command |\childdocforward| redirects processing to
another source file:
%
\begin{center}
\begin{tabular}{l}
|\input{childdoc.def}|\\
|\childdocforward[|\textit{main}|]{|\textit{dest}|}|\\
\end{tabular}
\end{center}
%
The argument \textit{dest} is the destination file
(without extension).
It should be the main file or one of the child files.
Note that further \textsf{childdoc} directives
such as |\childdocof| and |\childdocforward|
in the indicated file will be processed in this form.
The optional argument \textit{main}
passes on directly to the main file \textit{main}
while pretending to compile the child \textit{dest}.
This form behaves as if \textit{dest}
issues |\childdocof{|\textit{main}|}| right away,
and no further \textsf{childdoc} directives will be processed.

%%%%%%%%%%%%%%%%%%%%%%%%%%%%%%%%%%%%%%%%
\DescribeMacro{\...prefix}
In the alternative form |\childdocforwardprefix|,
%
\begin{center}
\begin{tabular}{l}
|\input{childdoc.def}|\\
|\childdocforwardprefix[|\textit{main}|]{|\textit{prefix}|}{|\textit{dest}|}|
\end{tabular}
\end{center}
%
the destination file is determined by a pattern
depending on the current file:
To make this work, the current file must be called
`{\textit{prefix}\hspace{0.2em}\textit{suffix}}'
with \textit{prefix} matching precisely the argument.
Processing is then passed on to the file
`{\textit{dest}\hspace{0.2em}\textit{suffix}}'.
Surely, the same effect is achieved by
directly specifying the
argument `{\textit{dest}\hspace{0.2em}\textit{suffix}}'
in the first form.
However, that requires to set up a different file
for each child. With the alternative form of the command
all these files can have exactly the same content
which simplifies setting them up and maintaining them.

For example, the following file |draft.tex|
with a compilation flag |\version| as described in \secref{sec:flags}
compiles the main document as a draft:
%
\begin{center}
\begin{tabular}{l}
|\def\version{draft}|\\
|\input{childdoc.def}|\\
|\childdocforward{|\textit{main}|}|
\end{tabular}
\end{center}
%
Likewise, the following files |final|\textit{nn}|.tex|
compile the final version of the child document
|child|\textit{nn}|.tex|:
%
\begin{center}
\begin{tabular}{l}
|\def\version{final}|\\
|\input{childdoc.def}|\\
|\childdocforwardprefix{final}{child}|
\end{tabular}
\end{center}
%

Note that when several versions of a main file and/or of each child file
are to be generated, it may be convenient to set up a |Makefile| or
shell script to automatise the process.

%%%%%%%%%%%%%%%%%%%%%%%%%%%%%%%%%%%%%%%%%%%%%%%%%%%%%%%%%%%%%%%%%%%%%%%%%%%%%%%%
\subsection{Command Line Processing}
\label{sec:commandline}

The effect of redirection files can also be achieved by invoking
the \LaTeX{} compiler with a more elaborate command line.
Most conveniently this should be done as part
of a shell script or a |Makefile|.

When using \textsf{childdoc} in the main file, the following
command lines effectively perform a redirection
(note that depending on the shell being used,
backslashes may have to be doubled: `|\|' $\to$ `|\\|'):
%
\begin{center}
|... -jobname "|\textit{target}|" |\\|"|[\textit{flags}]%
|\input{childdoc.def}\childdocforward[|\textit{main}|]{|\textit{dest}|}"|
\end{center}
%
Here \textit{target} is the name of the output file,
\textit{main} is the name of the main file
and \textit{dest} is the name of the main or child file to be processed
(all filenames without extensions).
The optional argument \textit{main} can be omitted
if \textit{main} matches \textit{dest}.
Optionally, compilation \textit{flags} can be defined via |\def| commands.
This command line makes the \TeX{} engine believe
it is compiling the file \textit{target}
whose content is specified as the latter parameter.
The provided code then forwards the processing to
\textit{main} or \textit{dest} as described in \secref{sec:forward}.

%%%%%%%%%%%%%%%%%%%%%%%%%%%%%%%%%%%%%%%%%%%%%%%%%%%%%%%%%%%%%%%%%%%%%%%%%%%%%%%%
\subsection{Include by Input}
\label{sec:input}

Including child documents by |\include| has some restrictions by design.
Most notably, the content of a child document always occupies
its own set of pages; pages cannot be shared between child documents.
Usually, this behaviour makes perfect sense
because each child document contain an essential part of the document.
However, in some situations it may be desirable to compose
a document from a collection of parts
without having mandatory page breaks between then.
For this case, the package
provides a mechanism to include parts
by |\input| which can also be processed individually.
However, by construction this mechanism
requires manual handling of the content to be output.

%%%%%%%%%%%%%%%%%%%%%%%%%%%%%%%%%%%%%%%%
\DescribeMacro{\ifchilddocmanual}
The main file should be prepared as usual, see \secref{sec:include}.
However, the document body must make a distinction
between processing of an individual part and of the main document, e.g.:
%
\begin{center}
\begin{tabular}{l}
|\ifchilddocmanual|\\
|\input{\childdocname}|\\
|\||else|\\
\textit{document body with }|\input{|\textit{part}|}|\\
|\||fi|
\end{tabular}
\end{center}
%
The conditional |\ifchilddocmanual| is true whenever
a part to be included by |\input| is being compiled,
and the name of the part is stored in |\childdocname|.

%%%%%%%%%%%%%%%%%%%%%%%%%%%%%%%%%%%%%%%%
\DescribeMacro{\childdocby}
Each part to be included by |\input| should start with:
%
\begin{center}
\begin{tabular}{l}
|\input{childdoc.def}|\\
|\childdocby{|\textit{main}|}|\\
\end{tabular}
\end{center}
%
The directive |\childdocby| is similar to |\childdocof|
described in \secref{sec:include},
but the subsequent selection of content must be done manually.
To that end, both |\ifchilddoc| and |\ifchilddocmanual|
will be true upon processing of a part,
and the name of the part is stored in |\childdocname|.
Note that |\jobname| will be set to the filename of the current part
so that each part receives an individual |.aux| file
that does not interfere with the |.aux| file(s) of the main document.
This behaviour can be altered by the alternative form
|\childdocby[*]{|\textit{main}|}| (with a non-empty optional argument)
which uses the |.aux| file of the main document
by setting |\jobname| to \textit{main}.

%%%%%%%%%%%%%%%%%%%%%%%%%%%%%%%%%%%%%%%%%%%%%%%%%%%%%%%%%%%%%%%%%%%%%%%%%%%%%%%%
\subsection{Driver Development}
\label{sec:driver}

The \textsf{childdoc} mechanism can also be use for the development
of definition files such as \LaTeX{} styles or classes.
This case differs from the above setup with multiple parts
included by |\include| in that no |\includeonly| should be invoked.
This can be achieved by starting the include file
(before |\ProvidesPackage|) with:
%
\begin{center}
\begin{tabular}{l}
|\input{childdoc.def}|\\
|\childdocforward{|\textit{main}|}|\\
\end{tabular}
\end{center}
%
or alternatively with:
%
\begin{center}
\begin{tabular}{l}
|\input{childdoc.def}|\\
|\childdocby{|\textit{main}|}|\\
\end{tabular}
\end{center}
%
Both forms have slightly different effects as described above.
The main file is prepared as usual, see \secref{sec:include}.

%%%%%%%%%%%%%%%%%%%%%%%%%%%%%%%%%%%%%%%%%%%%%%%%%%%%%%%%%%%%%%%%%%%%%%%%%%%%%%%%
\subsection{Legacy Detection}
\label{sec:detection}

The directive |\childdocmain| in the main file can detect
whether the complete document or merely a child is to be compiled
even without using the directive |\childdocof|.
This method is deprecated because it is less robust
and there is no compelling reason to use it;
it is merely provided for backward compatibility
and it may be removed in future versions.

If the detection mechanism is to be used,
it is mandatory to correctly specify
the filename of the main file as the argument of |\childdocmain|:
%
\begin{center}
\begin{tabular}{l}
|\input{childdoc.def}|\\
|\childdocmain{|\textit{main}|}|\\
\end{tabular}
\end{center}
%
If |\jobname| does not match the argument \textit{main} of |\childdocmain|,
it is assumed that |\jobname| points to the child file to be compiled.
When using |\childdocmain| with the main file specified as argument,
it suffices to start a child file
with just |\input{|\textit{main}|}|
without loading of the package and using |\childdocof|.
If instead all processing is done
with the appropriate \textsf{childdoc} directives,
the argument of \textit{main} of |\childdocmain| can be empty.

An alternative version of the command line processing described
in \secref{sec:commandline} using the detection mechanism reads:
%
\begin{center}
|... -jobname "|\textit{target}|" "|[\textit{flags}]%
[|\def\jobname{|\textit{dest}|}|]|\input{|\textit{main}|}"|
\end{center}

%%%%%%%%%%%%%%%%%%%%%%%%%%%%%%%%%%%%%%%%%%%%%%%%%%%%%%%%%%%%%%%%%%%%%%%%%%%%%%%%
\subsection{Manual Code}
\label{sec:manual}

In case one cannot be certain whether the definitions file |childdoc.def|
is installed on the target \TeX{} distribution
and one prefers not to ship it,
it is conceivable to paste a few relevant commands into the sources.

To that end, drop all statements |\input{childdoc.def}|
and perform the replacements as outlined below.
Instead of |\childdocmain{|\textit{main}|}| add the following code
to the top of the main file:
%
\begin{center}
\begin{tabular}{l}
|\||ifdefined\childdocname\endinput\||fi\newif\ifchilddoc|\\
|\edef\childdocname{\scantokens\expandafter{\jobname\noexpand}}|\\
|\def\childdocmain{|\textit{main}|}\||ifx\childdocmain\childdocname\||else|\\
|\childdoctrue\includeonly{\childdocname}\let\jobname\childdocmain\||fi|\\
\end{tabular}
\end{center}
%
Instead of |\childdocof{|\textit{main}|}| just include the main file
at the top of each child file:
%
\begin{center}
|\input{|\textit{main}|}|
\end{center}
%
A simple redirection |\childdocforward{|\textit{dest}|}| is achieved by:
%
\begin{center}
|\def\jobname{|\textit{dest}|}\input{\jobname}|
\end{center}
%
The redirection with prefix
|\childdocforwardprefix[|\textit{prefix}|]{|\textit{dest}|}|
is accomplished by:
%
\begin{center}
\begin{tabular}{l}
|{\edef\jobname{\scantokens\expandafter{\jobname\noexpand}}|\\
|\def\redirectjob |\textit{prefix}|#1~~~{\gdef\jobname{|\textit{dest}|#1}}|\\
|\expandafter\redirectjob\jobname~~~}\input{\jobname}|
\end{tabular}
\end{center}

In an alternative approach,
child documents can be compiled by a specific command line
without additional code or specific definitions:
%
\begin{center}
|... -jobname "|\textit{target}|" "|[\textit{flags}]%
|\includeonly{|\textit{dest}|}\input{|\textit{main}|}"|
\end{center}
%

%%%%%%%%%%%%%%%%%%%%%%%%%%%%%%%%%%%%%%%%%%%%%%%%%%%%%%%%%%%%%%%%%%%%%%%%%%%%%%%%
%%%%%%%%%%%%%%%%%%%%%%%%%%%%%%%%%%%%%%%%%%%%%%%%%%%%%%%%%%%%%%%%%%%%%%%%%%%%%%%%
\section{Information}

%%%%%%%%%%%%%%%%%%%%%%%%%%%%%%%%%%%%%%%%%%%%%%%%%%%%%%%%%%%%%%%%%%%%%%%%%%%%%%%%
\subsection{Copyright}

Copyright \copyright{} 2017--2018 Niklas Beisert

This work may be distributed and/or modified under the
conditions of the \LaTeX{} Project Public License, either version 1.3
of this license or (at your option) any later version.
The latest version of this license is in
  \url{http://www.latex-project.org/lppl.txt}
and version 1.3 or later is part of all distributions of \LaTeX{}
version 2005/12/01 or later.

This work has the LPPL maintenance status `maintained'.

The Current Maintainer of this work is Niklas Beisert.

This work consists of the files |README.txt|, |childdoc.ins| and |childdoc.dtx|
as well as the derived files |childdoc.def|, |cdocsamp.tex|
with |cdocsch1.tex|, |cdocsch2.tex|, |cdocspt3.tex|, |cdocspt4.tex|,
|cdocsdrf.tex|, |cdocsfn1.tex|, |cdocsfn2.tex|
as well as |childdoc.pdf|.

%%%%%%%%%%%%%%%%%%%%%%%%%%%%%%%%%%%%%%%%%%%%%%%%%%%%%%%%%%%%%%%%%%%%%%%%%%%%%%%%
\subsection{Files and Installation}

The package consists of the files:
%
\begin{center}
\begin{tabular}{ll}
    |README.txt|   & readme file \\
    |childdoc.ins| & installation file \\
    |childdoc.dtx| & source file \\
    |childdoc.def| & definition file \\
    |cdocsamp.tex| & sample main file \\
    |cdocsch1.tex| & sample include file \\
    |cdocsch2.tex| & sample include file \\
    |cdocspt3.tex| & sample part file \\
    |cdocspt4.tex| & sample part file \\
    |cdocsdrf.tex| & sample redirection file \\
    |cdocsfn1.tex| & sample redirection file \\
    |cdocsfn2.tex| & sample redirection file \\
    |childdoc.pdf| & manual
\end{tabular}
\end{center}
%
The distribution consists of the files
|README.txt|, |childdoc.ins| and |childdoc.dtx|.
%
\begin{itemize}
\item
Run (pdf)\LaTeX{} on |childdoc.dtx|
to compile the manual |childdoc.pdf| (this file).
\item
Run \LaTeX{} on |childdoc.ins| to create the definitions file |childdoc.def|
and the sample |cdocsamp.tex| with include files
|cdocsch1.tex|, |cdocsch2.tex|, |cdocspt3.tex|, |cdocspt4.tex|,
|cdocsdrf.tex|, |cdocsfn1.tex|, |cdocsfn2.tex|.
Then copy the file |childdoc.def| to an appropriate directory of your \LaTeX{}
distribution, e.g.\ \textit{texmf-root}|/tex/latex/childdoc|.
\end{itemize}

%%%%%%%%%%%%%%%%%%%%%%%%%%%%%%%%%%%%%%%%%%%%%%%%%%%%%%%%%%%%%%%%%%%%%%%%%%%%%%%%
\subsection{Related CTAN Packages}

There are several other packages which offer a similar functionality:
%
\begin{itemize}
\item
The packages
\href{http://ctan.org/pkg/docmute}{\textsf{docmute}},
\href{http://ctan.org/pkg/includex}{\textsf{includex}} and
\href{http://ctan.org/pkg/standalone}{\textsf{standalone}}
provide commands to include only the document body of
a child file thus allowing both files to be compiled individually.
\item
The packages \href{http://ctan.org/pkg/subdocs}{\textsf{subdocs}}
and \href{http://ctan.org/pkg/subfiles}{\textsf{subfiles}}
provide structures in which the main and child documents can be
encapsulated and allowing them to be compiled individually.
The inclusion mechanism is different from the conventional |\include|.
\item
The package \href{http://ctan.org/pkg/combine}{\textsf{combine}}
is an elaborate solution to combine several documents into one.
\end{itemize}
%
See also the CTAN topic \href{http://ctan.org/topic/subdocs}{\textsf{subdocs}}
for further related packages.
The present package differs from the above solutions in that
a document structure constructed with the conventional |\include| mechanism
just needs two extra commands at the top of every file
such that all constituent files can be compiled individually.

%%%%%%%%%%%%%%%%%%%%%%%%%%%%%%%%%%%%%%%%%%%%%%%%%%%%%%%%%%%%%%%%%%%%%%%%%%%%%%%%
%\subsection{Feature Suggestions}
%
%The following is a list of features which may be useful for future
%versions of this package:
%%
%\begin{itemize}
%\item
%\ldots
%\end{itemize}

%%%%%%%%%%%%%%%%%%%%%%%%%%%%%%%%%%%%%%%%%%%%%%%%%%%%%%%%%%%%%%%%%%%%%%%%%%%%%%%%
\subsection{Revision History}

%%%%%%%%%%%%%%%%%%%%%%%%%%%%%%%%%%%%%%%%
\paragraph{v2.0:} 2018/12/30

\begin{itemize}
\item
immediate forward processing
\item
added |\childdocby| mechanism
\item
manual restructured
\end{itemize}

%%%%%%%%%%%%%%%%%%%%%%%%%%%%%%%%%%%%%%%%
\paragraph{v1.6:} 2018/01/17

\begin{itemize}
\item
application for development of include files
\item
corrections to manual
\end{itemize}

%%%%%%%%%%%%%%%%%%%%%%%%%%%%%%%%%%%%%%%%
\paragraph{v1.5:} 2017/05/21

\begin{itemize}
\item
more complete structuring introduced
\item
|\childdocof| introduced
\item
|\childdoc| renamed to |\childdocmain|
\item
|\childredirect| renamed to |\childdocforward| and |\childdocforwardprefix|
and functionality expanded
\end{itemize}

%%%%%%%%%%%%%%%%%%%%%%%%%%%%%%%%%%%%%%%%
\paragraph{v1.0:} 2017/04/27

\begin{itemize}
\item
manual and install package
\item
first version published on CTAN
\end{itemize}

%%%%%%%%%%%%%%%%%%%%%%%%%%%%%%%%%%%%%%%%
\paragraph{v0.6:} 2017/04/26

\begin{itemize}
\item
redirection mechanism added
\end{itemize}

%%%%%%%%%%%%%%%%%%%%%%%%%%%%%%%%%%%%%%%%
\paragraph{v0.5:} 2017/04/26

\begin{itemize}
\item
functionality in definition file
\end{itemize}


%%%%%%%%%%%%%%%%%%%%%%%%%%%%%%%%%%%%%%%%%%%%%%%%%%%%%%%%%%%%%%%%%%%%%%%%%%%%%%%%
%%%%%%%%%%%%%%%%%%%%%%%%%%%%%%%%%%%%%%%%%%%%%%%%%%%%%%%%%%%%%%%%%%%%%%%%%%%%%%%%
%%%%%%%%%%%%%%%%%%%%%%%%%%%%%%%%%%%%%%%%%%%%%%%%%%%%%%%%%%%%%%%%%%%%%%%%%%%%%%%%
\appendix

\settowidth\MacroIndent{\rmfamily\scriptsize 000\ }

 \DocInput{childdoc.dtx}

\end{document}
%</driver>
% \fi
%
% %%%%%%%%%%%%%%%%%%%%%%%%%%%%%%%%%%%%%%%%%%%%%%%%%%%%%%%%%%%%%%%%%%%%%%%%%%%%%%
% %%%%%%%%%%%%%%%%%%%%%%%%%%%%%%%%%%%%%%%%%%%%%%%%%%%%%%%%%%%%%%%%%%%%%%%%%%%%%%
% \section{Sample}
%\iffalse
%<*samplemain>
%\fi
%
% The following presents a sample document
% with two chapters, two parts, a title page,
% a compile flag as well as three forwarding files to set the flag.
% It consists of eight |.tex| files:
% \begin{center}
% \begin{tabular}{ll}
% |cdocsamp.tex|&main file\\
% |cdocsch1.tex|&include file for chapter 1\\
% |cdocsch2.tex|&include file for chapter 2\\
% |cdocspt3.tex|&include file for part 3\\
% |cdocspt4.tex|&include file for part 4\\
% |cdocsdrf.tex|&forwarding file for main file in draft mode\\
% |cdocsfi1.tex|&forwarding file for final version of chapter 1\\
% |cdocsfi2.tex|&forwarding file for final version of chapter 2\\
% \end{tabular}
% \end{center}
% Each of the eight files can be compiled directly by the \LaTeX{} compiler.
%
% %%%%%%%%%%%%%%%%%%%%%%%%%%%%%%%%%%%%%%
% \paragraph{Main File.}
%
% The main file is called |cdocsamp.tex|.
%
% Load the \textsf{childdoc} definitions and
% declare the filename for the main document:
%    \begin{macrocode}
\input{childdoc.def}
\childdocmain{}
%    \end{macrocode}

% Optional override for |\version| flag:
%    \begin{macrocode}
%%\ifchilddoc\else\providecommand{\version}{draft}\fi
%    \end{macrocode}

% Define the default values for the |\version| flag
% (|final| for the main file and |draft| for childs):
%    \begin{macrocode}
\ifchilddoc
\providecommand{\version}{draft}
\else
\providecommand{\version}{final}
\fi
%    \end{macrocode}

% Load the standard document class:
%    \begin{macrocode}
\documentclass[12pt]{article}
%    \end{macrocode}

% Start the document body:
%    \begin{macrocode}
\begin{document}
%    \end{macrocode}

% Declare a title page.
% Print title, part of document being processed and version flag:
%    \begin{macrocode}
\addtocounter{page}{-1}
\begin{center}
{\LARGE\bfseries{}childdoc example\par}
\vspace{1cm}
\ifchilddoc
\ifchilddocmanual part\else chapter\fi:
`\childdocname' of `\childdocjob'\par
\else
main document: `\childdocjob'\par
\fi
version: \version\par
\end{center}
\newpage
%    \end{macrocode}

% Manually include selected file,
% otherwise process as usual:
%    \begin{macrocode}
\ifchilddocmanual
\section*{part `\childdocname'}
\input{\childdocname}
\else
%    \end{macrocode}

% Include the two chapters:
%    \begin{macrocode}
\include{cdocsch1}
\include{cdocsch2}
%    \end{macrocode}

% Include the two parts unless only chapters should be displayed:
%    \begin{macrocode}
\ifchilddoc\else
\section{part three}
\input{cdocspt3}
\section{part four}
\input{cdocspt4}
\fi
%    \end{macrocode}

% Process as usual until here:
%    \begin{macrocode}
\fi
%    \end{macrocode}

% End of document body:
%    \begin{macrocode}
\end{document}
%    \end{macrocode}
%\iffalse
%</samplemain>
%\fi
%
% %%%%%%%%%%%%%%%%%%%%%%%%%%%%%%%%%%%%%%
% \paragraph{Chapter Include Files.}
%
% The include files are called |cdocsch1.tex| and |cdocsch2.tex|.
%
%\iffalse
%<*samplechap1|samplechap2>
%\fi

% Optional override for |\version| flag:
%    \begin{macrocode}
%%\providecommand{\version}{final}
%    \end{macrocode}

% Include the main document:
%    \begin{macrocode}
\input{childdoc.def}
\childdocof{cdocsamp}
%    \end{macrocode}

%\iffalse
%</samplechap1|samplechap2>
%\fi
%
%\iffalse
%<*samplechap1>
%\fi
% Some text for chapter 1:
%    \begin{macrocode}
\section{one}
some text in chapter one
%    \end{macrocode}

%\iffalse
%</samplechap1>
%\fi
% Some text for chapter 2:
%\iffalse
%<*samplechap2>
%\fi
%    \begin{macrocode}
\section{two}
more text in chapter two
%    \end{macrocode}

%\iffalse
%</samplechap2>
%\fi
%
% %%%%%%%%%%%%%%%%%%%%%%%%%%%%%%%%%%%%%%
% \paragraph{Part Include Files.}
%
% The include files are called |cdocspt3.tex| and |cdocspt4.tex|.
%
%\iffalse
%<*samplepart3|samplepart4>
%\fi

% Optional override for |\version| flag:
%    \begin{macrocode}
%%\providecommand{\version}{final}
%    \end{macrocode}

% Include the main document:
%    \begin{macrocode}
\input{childdoc.def}
\childdocby{cdocsamp}
%    \end{macrocode}

%\iffalse
%</samplepart3|samplepart4>
%\fi
%
%\iffalse
%<*samplepart3>
%\fi
% Some text for part 3:
%    \begin{macrocode}
some text in part three
%    \end{macrocode}

%\iffalse
%</samplepart3>
%\fi
% Some text for part 4:
%\iffalse
%<*samplepart4>
%\fi
%    \begin{macrocode}
more text in part four
%    \end{macrocode}

%\iffalse
%</samplepart4>
%\fi
%
% %%%%%%%%%%%%%%%%%%%%%%%%%%%%%%%%%%%%%%
% \paragraph{Forwarding for a Complete Draft.}
%
% The following forwarding file |cdocsdrf.tex|
% compiles the main document in draft mode:
%\iffalse
%<*sampledraft>
%\fi
%    \begin{macrocode}
\def\version{draft}
\input{childdoc.def}
\childdocforward{cdocsamp}
%    \end{macrocode}

%\iffalse
%</sampledraft>
%\fi
%
% %%%%%%%%%%%%%%%%%%%%%%%%%%%%%%%%%%%%%%
% \paragraph{Forwarding for Final Version of the Chapters.}
%
% The following forwarding files |cdocsfn1.tex| and |cdocsfn2.tex|
% (with identical content)
% compile the final versions of the child documents
% |cdocsch1.tex| and |cdocsch2.tex|, respectively:
%\iffalse
%<*samplefinal>
%\fi
%    \begin{macrocode}
\def\version{final}
\input{childdoc.def}
\childdocforwardprefix[cdocsamp]{cdocsfn}{cdocsch}
%    \end{macrocode}

%\iffalse
%</samplefinal>
%\fi
%
% %%%%%%%%%%%%%%%%%%%%%%%%%%%%%%%%%%%%%%
% \paragraph{Command Line Processing.}
%
% The following three command lines generate the output files
% |cdocscld|, |cdocscl1| and |cdocscl2|
% which should be identical to
% |cdocsdrf|, |cdocsch1| and |cdocsfn2|, respectively:
% \begin{center}
% \begin{tabular}{l}
% |latex -jobname cdocscld \|\\
% |  "\def\version{draft}\input{childdoc.def}\childdocforward{cdocsamp}"|\\
% |latex -jobname cdocscl1 \|\\
% |  "\input{childdoc.def}\childdocforward[cdocsamp]{cdocsch1}"|\\
% |latex -jobname cdocscl2 \|\\
% |  "\def\version{final}\input{childdoc.def}\childdocforward{cdocsch2}"|
% \end{tabular}
% \end{center}
% Note that the trailing backslash on each first line
% merely continues the input to the second line
% (for convenient cut ant paste).
% Furthermore, the command |latex| can be replaced by any
% of its alternative versions such as |pdflatex|.
%
% %%%%%%%%%%%%%%%%%%%%%%%%%%%%%%%%%%%%%%%%%%%%%%%%%%%%%%%%%%%%%%%%%%%%%%%%%%%%%%
% %%%%%%%%%%%%%%%%%%%%%%%%%%%%%%%%%%%%%%%%%%%%%%%%%%%%%%%%%%%%%%%%%%%%%%%%%%%%%%
% \section{Implementation}
%\iffalse
%<*package>
%\fi
%
% This section describes the definitions file |childdoc.def|.

% The definitions cannot be loaded using |\usepackage| or |\RequirePackage|
% which has a mechanism to prevent loading a style file more than once.
% When loading the definitions by means of |\input|
% multiple instances have to be prevented manually:
%\iffalse
%This code needs to be before the `\ProvidesFile' directive
%which is defined at the beginning of this file.
%Therefore it is also placed there and commented out here.
%</package>
%<*discard>
%\fi
%    \begin{macrocode}
\ifdefined\childdocmain\endinput\fi
%    \end{macrocode}
%\iffalse
%</discard>
%<*package>
%\fi
%
% \macro{\ifchilddoc}
% \macro{\ifchilddocmanual}
% The conditional |\ifchilddoc| tells whether a
% child (true) or main (false) document is being compiled.
% The conditional |\ifchilddocmanual| tells whether
% the |\includeonly| mechanism is used (false) or
% the selection of child files must be performed manually (true).
% The definitions initialise to false:
%    \begin{macrocode}
\newif\ifchilddoc
\newif\ifchilddocmanual
%    \end{macrocode}

% \macro{\childdocname}
% \macro{\childdocjob}
% The macro |\childdocname| stores the name of the main document
% to be compiled. The macro |\childdocjob| stores the name of
% the document on which the \LaTeX{} compiler was originally invoked.
% The content of |\jobname| cannot be compared
% to filenames specified in the source due to different catcodes.
% The following code rescans |\jobname|, stores the result
% in |\childdocname| and saves a copy in |\childdocjob|:
%    \begin{macrocode}
\edef\childdocname{\scantokens\expandafter{\jobname\noexpand}}
\let\childdocjob\childdocname
%    \end{macrocode}

% \macro{\childdocdisable}
% The macro |\childdocdisable| prevents the main file
% from being processed more than once.
% At this stage, the main document command |\childdocmain|
% is assumed to be called once again where it should do nothing.
% Any subsequent call to it should prevent
% a secondary processing of the main document
% It overwrites the forwarding commands
% |\childdocof| and |\childdocforward|
% with empty macros to prevent further inclusions of the main document:
%    \begin{macrocode}
\newcommand{\childdocdisable}
{
  \renewcommand{\childdocmain}[1]{\renewcommand{\childdocmain}[1]{\endinput}}
  \renewcommand{\childdocof}[1]{}
  \renewcommand{\childdocby}[2][]{}
  \renewcommand{\childdocforward}[2][]{}
  \renewcommand{\childdocdisable}{}
}
%    \end{macrocode}

% \macro{\childdocmain}
% The macro |\childdocmain| is to be called at the top of the main file
% with nothing or the main filename (without extension) as argument.
% First, it breaks loops.
% If the argument is not empty and does not match |\childdocname|
% (which is set by the first inclusion of |childdoc.def|),
% |\ifchilddoc| is set to true, |\includeonly| is applied to the child file
% and |\jobname| is set to the main file
% (for proper handling of |.aux| files):
%    \begin{macrocode}
\newcommand{\childdocmain}[1]
{
  \childdocdisable\childdocmain{}
  \if?#1?\else
    \begingroup
      \def\childdoctmp{#1}
      \ifx\childdoctmp\childdocname
        \def\childdoctmp{}
      \else
        \def\childdoctmp
        {
          \childdoctrue
          \includeonly{\childdocname}
          \def\childdocjob{#1}
          \def\jobname{#1}
        }
      \fi
      \expandafter
    \endgroup
    \childdoctmp
  \fi
}
%    \end{macrocode}

% \macro{\childdocof}
% The command |\childdocof| redirects
% compilation to the main file |#1|.
%    \begin{macrocode}
\newcommand{\childdocof}[1]
{
  \childdocdisable
  \childdoctrue
  \includeonly{\childdocname}
  \def\jobname{#1}
  \def\childdocjob{#1}
  \input{#1}
}
%    \end{macrocode}

% \macro{\childdocby}
% The command |\childdocby| ....
%    \begin{macrocode}
\newcommand{\childdocby}[2][]
{
  \childdocdisable
  \childdoctrue
  \childdocmanualtrue
  \if?#1?\else
    \def\jobname{#2}
  \fi
  \def\childdocjob{#2}
  \input{#2}
  \endinput
}
%    \end{macrocode}

% \macro{\childdocforward}
% The command |\childdocforward| redirects
% compilation to the main file or
% (if the optional argument is given) a child file.
% Parameters are set as if the main file
% or a child file starting with |\childdocof| was compiled.
% Then compilation is handed over to the main file:
%    \begin{macrocode}
\newcommand{\childdocforward}[2][]
{
  \begingroup
    \if?#1?
      \def\childdoctmp
      {
        \def\childdocname{#2}
        \def\childdocjob{#2}
        \def\jobname{#2}
        \input{#2}
        \endinput
      }
    \else
      \def\childdoctmp
      {
        \childdocdisable
        \def\childdocname{#2}
        \childdoctrue
        \includeonly{#2}
        \def\childdocjob{#1}
        \def\jobname{#1}
        \input{#1}
        \endinput
      }
    \fi
    \expandafter
  \endgroup
  \childdoctmp
}
%    \end{macrocode}

% \macro{\childdocforwardprefix}
% The command |\childdocforwardprefix| redirects
% compilation to the main or a child file by means of a pattern.
% The prefix |#1| in the current filename is replaced by |#2|
% and the suffix of the current filename is kept
% (it is assumed that the filename does not contain the substring `|~~~|'
% which is used as a delimiter).
% Compilation is handed over to the new file by |\childdocforward|:
%    \begin{macrocode}
\newcommand{\childdocforwardprefix}[3][]
{
  \begingroup
    \def\childdocextract #2##1~~~{\def\childdoctmp{\childdocforward[#1]{#3##1}}}
    \expandafter\childdocextract\childdocname~~~
    \expandafter
  \endgroup
  \childdoctmp
}
%    \end{macrocode}

% \macro{\childdoc}
% The deprecated macro |\childdoc| is a legacy version of |\childdocmain|:
%    \begin{macrocode}
\newcommand{\childdoc}{\childdocmain}
%    \end{macrocode}

% \macro{\childdocredirect}
% The deprecated macro |\childdocredirect| is a legacy version
% of |\childdocforward| and |\childdocforwardprefix|:
%    \begin{macrocode}
\newcommand{\childdocredirect}[2][]
{
  \begingroup
    \if?#1?
      \def\childdoctmp{\childdocforward{#2}}
    \else
      \def\childdoctmp{\childdocforwardprefix{#1}{#2}}
    \fi
    \expandafter
  \endgroup
  \childdoctmp
}
%    \end{macrocode}

%\iffalse
%</package>
%\fi
%
\endinput
|\\
|\childdocforwardprefix[|\textit{main}|]{|\textit{prefix}|}{|\textit{dest}|}|
\end{tabular}
\end{center}
%
the destination file is determined by a pattern
depending on the current file:
To make this work, the current file must be called
`{\textit{prefix}\hspace{0.2em}\textit{suffix}}'
with \textit{prefix} matching precisely the argument.
Processing is then passed on to the file
`{\textit{dest}\hspace{0.2em}\textit{suffix}}'.
Surely, the same effect is achieved by
directly specifying the
argument `{\textit{dest}\hspace{0.2em}\textit{suffix}}'
in the first form.
However, that requires to set up a different file
for each child. With the alternative form of the command
all these files can have exactly the same content
which simplifies setting them up and maintaining them.

For example, the following file |draft.tex|
with a compilation flag |\version| as described in \secref{sec:flags}
compiles the main document as a draft:
%
\begin{center}
\begin{tabular}{l}
|\def\version{draft}|\\
|% \iffalse
%
% childdoc.dtx Copyright (C) 2017-2018 Niklas Beisert
%
% This work may be distributed and/or modified under the
% conditions of the LaTeX Project Public License, either version 1.3
% of this license or (at your option) any later version.
% The latest version of this license is in
%   http://www.latex-project.org/lppl.txt
% and version 1.3 or later is part of all distributions of LaTeX
% version 2005/12/01 or later.
%
% This work has the LPPL maintenance status `maintained'.
%
% The Current Maintainer of this work is Niklas Beisert.
%
% This work consists of the files childdoc.dtx and childdoc.ins
% and the derived files childdoc.def and cdocsamp.tex with
% cdocsch1.tex, cdocsch2.tex, cdocsdrf.tex, cdocsfn1.tex, cdocsfn2.tex.
%
%<package>\ifdefined\childdocmain\endinput\fi
%<package>\ProvidesFile{childdoc.def}[2018/12/30 v2.0 child document driver]
%<samplemain>\ProvidesFile{cdocsamp.tex}[2018/12/30 v2.0 sample for childdoc]
%<*driver>
%\ProvidesFile{childdoc.drv}[2018/12/30 v2.0 childdoc reference manual file]
\PassOptionsToClass{10pt,a4paper}{article}
\documentclass{ltxdoc}

\usepackage[margin=35mm]{geometry}
\usepackage{hyperref}
\usepackage{hyperxmp}
\usepackage[usenames]{color}

\hypersetup{colorlinks=true}
\hypersetup{pdfstartview=FitH}
\hypersetup{pdfpagemode=UseNone}
\hypersetup{pdfsource={}}
\hypersetup{pdflang={en-UK}}
\hypersetup{pdfcopyright={Copyright 2017-2018 Niklas Beisert.
  This work may be distributed and/or modified under the
  conditions of the LaTeX Project Public License, either version 1.3
  of this license or (at your option) any later version.}}
\hypersetup{pdflicenseurl={http://www.latex-project.org/lppl.txt}}
\hypersetup{pdfcontactaddress={ETH Zurich, ITP, HIT K,
  Wolfgang-Pauli-Strasse 27}}
\hypersetup{pdfcontactpostcode={8093}}
\hypersetup{pdfcontactcity={Zurich}}
\hypersetup{pdfcontactcountry={Switzerland}}
\hypersetup{pdfcontactemail={nbeisert@itp.phys.ethz.ch}}
\hypersetup{pdfcontacturl={http://people.phys.ethz.ch/\xmptilde nbeisert/}}

\newcommand{\secref}[1]{\hyperref[#1]{section \ref*{#1}}}

\parskip1ex
\parindent0pt
\let\olditemize\itemize
\def\itemize{\olditemize\parskip0pt}

\begin{document}

\title{The \textsf{childdoc} Package}
\hypersetup{pdftitle={The childdoc Package}}
\author{Niklas Beisert\\[2ex]
  Institut f\"ur Theoretische Physik\\
  Eidgen\"ossische Technische Hochschule Z\"urich\\
  Wolfgang-Pauli-Strasse 27, 8093 Z\"urich, Switzerland\\[1ex]
  \href{mailto:nbeisert@itp.phys.ethz.ch}
  {\texttt{nbeisert@itp.phys.ethz.ch}}}
\hypersetup{pdfauthor={Niklas Beisert}}
\hypersetup{pdfsubject={Manual for the LaTeX2e Package childdoc}}
\date{30 December 2018, \textsf{v2.0}}
\maketitle

\begin{abstract}\noindent
\textsf{childdoc} is a \LaTeXe{} package
that enables the direct compilation
of document sections included by |\include|
to individual files.
\end{abstract}

\begingroup
\parskip0ex
\tableofcontents
\endgroup

%%%%%%%%%%%%%%%%%%%%%%%%%%%%%%%%%%%%%%%%%%%%%%%%%%%%%%%%%%%%%%%%%%%%%%%%%%%%%%%%
%%%%%%%%%%%%%%%%%%%%%%%%%%%%%%%%%%%%%%%%%%%%%%%%%%%%%%%%%%%%%%%%%%%%%%%%%%%%%%%%
\section{Introduction}

\LaTeX{} provides a mechanism to structure a large document (such as a book)
into a main file and several child files (containing the chapters)
using the |\include| command.
This mechanism is beneficial for documents
which span hundreds of pages in order to
make the source file(s) more manageable.
Moreover, compilation can be restricted to
selected child files by means of the |\includeonly| command.
The latter feature can be used to reduce the compilation time while editing
(this was significantly more useful in the earlier days of \LaTeX{})
or to generate a smaller document which is easier to navigate.
Another application of |\includeonly| is to generate
documents consisting of selected parts of the complete document.

However, there are a few drawbacks of the plain |\include| mechanism:
\begin{itemize}
\item
The child files cannot be compiled on their own,
they can only be compiled via the main file.
A naive editing environment
(such as a text editor with an option
to have the current file processed by \LaTeX)
may require one to switch to the main file before compiling;
attempting to compile the child file produces errors.
\item
The main file must be modified (each time)
to adjust the |\includeonly| command
to the present needs. This easily leaves the main file in a messy state.
\item
The generated document will always carry the filename
of the main document. This is inconvenient if
several child files are to be compiled and
to be kept for distribution.
\end{itemize}

The present package provides a simple interface
to make child files individually compilable by \LaTeX{}.
Compiling a child file then has the same effect as compiling
the main file with an |\includeonly| command
to select the appropriate child.
Moreover the generated document will carry the name of the child
rather than the main file.
This resolves all three above issues.

This feature is meant to make the editing of books,
thesis documents and lecture notes somewhat more convenient.
However, the package can also be used efficiently for
composing a series of documents (such as exercise sheets)
which are typically distributed individually.
It then assists the author in generating the individual documents
(potentially in different versions)
as well as a document containing the collected series.
Another application is in developing style files
or other kinds of included material
where compilation of the style file could redirect
to a sample or test file.

%%%%%%%%%%%%%%%%%%%%%%%%%%%%%%%%%%%%%%%%%%%%%%%%%%%%%%%%%%%%%%%%%%%%%%%%%%%%%%%%
%%%%%%%%%%%%%%%%%%%%%%%%%%%%%%%%%%%%%%%%%%%%%%%%%%%%%%%%%%%%%%%%%%%%%%%%%%%%%%%%
\section{Usage}

First of all, the package \textsf{childdoc} is \emph{not} a standard
\LaTeXe{} |.sty| style file! Therefore it needs to be invoked in
a non-standard way.

%%%%%%%%%%%%%%%%%%%%%%%%%%%%%%%%%%%%%%%%%%%%%%%%%%%%%%%%%%%%%%%%%%%%%%%%%%%%%%%%
\subsection{Included Files}
\label{sec:include}

%%%%%%%%%%%%%%%%%%%%%%%%%%%%%%%%%%%%%%%%
\DescribeMacro{\childdocmain}
To use the package, add the commands
\begin{center}
\begin{tabular}{l}
|\input{childdoc.def}|\\
|\childdocmain{}|\\
\end{tabular}
\end{center}
at the very top of the main \LaTeX{} file,
in particular \emph{before} the |\documentclass| statement!
The argument of |\childdocmain| should be left empty
(but it must be present).

%%%%%%%%%%%%%%%%%%%%%%%%%%%%%%%%%%%%%%%%
\DescribeMacro{\childdocof}
Furthermore, add the commands
\begin{center}
\begin{tabular}{l}
|\input{childdoc.def}|\\
|\childdocof{|\textit{main}|}|\\
\end{tabular}
\end{center}
at the top of every child file \textit{child}
which is included by |\include{|\textit{child}|}|
from within the main file
(or at least for those files to be compiled individually).
The argument \textit{main} must be the filename of the main file.

There are a couple of
considerations in setting up the main and child documents:

%%%%%%%%%%%%%%%%%%%%%%%%%%%%%%%%%%%%%%%%
\paragraph{Restrictions.}

Please note the following restrictions:
\begin{itemize}
\item
|\childdocmain| must be called with one argument \textit{main}
to ensure compatibility with earlier version of the package.
It must either be empty (|\childdocmain{}|)
or precisely match the filename of the main file in which it is specified.
See \secref{sec:detection} for further information.
\item
The filename \textit{main} must be specified without the |.tex| extension.
\item
The filename \textit{main} is case sensitive
(even in case-insensitive file systems)
due to internal string comparison.
\item
The argument \textit{main} should be fully expanded, it cannot be a macro.
\item
Subdirectories and special characters should be avoided in filenames.
\item
The command |\childdocmain{|\textit{main}|}| must be followed by a whitespace.
It should not be followed immediately by another command
or by a comment mark `|%|'.
This is because the \TeX{} parser reads the token immediately following
the argument of |\childdocmain| and puts it
at the beginning of every child section;
however, a white\-space is ignored.
\end{itemize}

%%%%%%%%%%%%%%%%%%%%%%%%%%%%%%%%%%%%%%%%
\paragraph{Content of Main File.}

It is advisable to place all content in the child files included by |\include|.
Any output contained in the main file will appear in all child documents
unless suppressed manually;
it cannot be suppressed automatically by the |\includeonly| directive
and thus should normally be avoided.
A method to include some content in the main file
by means of conditional processing is described in \secref{sec:conditional}.

%%%%%%%%%%%%%%%%%%%%%%%%%%%%%%%%%%%%%%%%
\paragraph{Page Numbering.}

When only a part of the document is compiled,
the appropriate numbering of pages
(as well as other status parameters)
is determined from the |.aux| files.
The latter contain information from previous passes.
However this information needs to propagate through
all intermediate child documents.
Therefore the page numbering in child documents may well
be inconsistent until the complete document is compiled at least once.

A useful (if unconventional) way to always ensure a consistent
page numbering is to restart the numbering in each child document
and denote the pages by `\textit{child}|.|\textit{page}'
where \textit{child} represents the chapter/section number of the child file.
This can be achieved by the command
|\numberwithin{page}{|\textit{child}|}|
of the \textsf{amsmath} package
where \textit{child} can be |chapter| or |section|
depending on the chosen structuring.
Alternatively, one can modify the macro |\thepage| appropriately
and reset the counter |page| at the start of each child file.

%%%%%%%%%%%%%%%%%%%%%%%%%%%%%%%%%%%%%%%%%%%%%%%%%%%%%%%%%%%%%%%%%%%%%%%%%%%%%%%%
\subsection{Conditional Processing}
\label{sec:conditional}

The package provides a mechanism to compile different versions
of a document. To customise the versions further some conditional processing
can come in handy to distinguish which version is being compiled.
The package provides two macros to describe the compilation context:

%%%%%%%%%%%%%%%%%%%%%%%%%%%%%%%%%%%%%%%%
\DescribeMacro{\ifchilddoc}
The conditional |\ifchilddoc| distinguishes between the compilation of
child documents and the main document:
%
\begin{center}
|\ifchilddoc |\textit{child-code}| |[|\||else |\textit{main-code}]| \||fi|
\end{center}

%%%%%%%%%%%%%%%%%%%%%%%%%%%%%%%%%%%%%%%%
\DescribeMacro{\childdocname}
\DescribeMacro{\childdocjob}
The macro |\childdocname| contains the filename (without extension)
of the main or child file being processed.
Note that |\childdocjob| will always contain the name of the main file.

%%%%%%%%%%%%%%%%%%%%%%%%%%%%%%%%%%%%%%%%
\paragraph{Title Page.}

Conditional processing can be used to include a title or banner page
in the main document when proper precautions are taken.
Importantly, the code in the main file should ensure that the page counter
(as well as other status parameters which are stored in the |.aux| files)
takes the same value after the conditional processing.
Otherwise the page numbers may take divergent values
depending on which part is compiled.

For example, a title page could be declared by:
%
\begin{center}
\begin{tabular}{l}
|\ifchilddoc\||else|\\
|\addtocounter{page}{-1}|\\
\textit{code for title page}\\
|\newpage|\\
|\||fi|
\end{tabular}
\end{center}
%
A banner page for the child documents can be generated by:
%
\begin{center}
\begin{tabular}{l}
|\ifchilddoc|\\
|\addtocounter{page}{-1}|\\
\textit{code for banner page}\\
|\newpage|\\
|\||fi|
\end{tabular}
\end{center}
%
Here one could write a message such as:
\begin{center}
|This is the part \childdocname{} of \childdocjob{}.|
\end{center}

%%%%%%%%%%%%%%%%%%%%%%%%%%%%%%%%%%%%%%%%%%%%%%%%%%%%%%%%%%%%%%%%%%%%%%%%%%%%%%%%
\subsection{Flags}
\label{sec:flags}

The package makes it easy to generate different versions
of the main or child documents.
To this end compilation flags can be defined
and assigned different default values.
They will be particularly useful in conjunction
with the forwarding mechanism described in \secref{sec:forward}.

For example, it may be useful to have a flag |\version|
which can be set to |draft| or |final|.
The document source will contain some conditional code
depending on the value of |\version|.
Suppose further, the flag should default to |final| for the main file
and to |draft| for child files
which is a natural assignment for editing the document.
This is achieved by placing the following code
in the preamble of the main document
(below the |\childdocmain| directive):
%
\begin{center}
\begin{tabular}{l}
|\ifchilddoc|\\
|\providecommand{\version}{draft}|\\
|\||else|\\
|\providecommand{\version}{final}|\\
|\||fi|
\end{tabular}
\end{center}
%
The definition by |\providecommand| makes sure
that previous definitions are not overwritten.
Further statements |\providecommand{\version}{...}|
can thus be added before the above code to override it.

For the main file, one might add a line
(between |\childdocmain| and the above block)
%
\begin{center}
|%\ifchilddoc\||else\providecommand{\version}{draft}\||fi|
\end{center}
%
which can be uncommented to produce a draft version.
Likewise one can add a line to the very top of a child file
(above the |\childdocof{|\textit{main}|}| directive)
%
\begin{center}
|%\providecommand{\version}{final}|
\end{center}
%
which can be uncommented to produce the final version of this child document.

%%%%%%%%%%%%%%%%%%%%%%%%%%%%%%%%%%%%%%%%%%%%%%%%%%%%%%%%%%%%%%%%%%%%%%%%%%%%%%%%
\subsection{Forwarding}
\label{sec:forward}

Different versions of the main or child documents
using compilation flags as described in \secref{sec:flags}
can be (permanently) stored in different files
for convenient compilation, viewing and distribution.
To this end, the package defines a command
to pass on compilation to a different file:

%%%%%%%%%%%%%%%%%%%%%%%%%%%%%%%%%%%%%%%%
\DescribeMacro{\childdocforward}
The command |\childdocforward| redirects processing to
another source file:
%
\begin{center}
\begin{tabular}{l}
|\input{childdoc.def}|\\
|\childdocforward[|\textit{main}|]{|\textit{dest}|}|\\
\end{tabular}
\end{center}
%
The argument \textit{dest} is the destination file
(without extension).
It should be the main file or one of the child files.
Note that further \textsf{childdoc} directives
such as |\childdocof| and |\childdocforward|
in the indicated file will be processed in this form.
The optional argument \textit{main}
passes on directly to the main file \textit{main}
while pretending to compile the child \textit{dest}.
This form behaves as if \textit{dest}
issues |\childdocof{|\textit{main}|}| right away,
and no further \textsf{childdoc} directives will be processed.

%%%%%%%%%%%%%%%%%%%%%%%%%%%%%%%%%%%%%%%%
\DescribeMacro{\...prefix}
In the alternative form |\childdocforwardprefix|,
%
\begin{center}
\begin{tabular}{l}
|\input{childdoc.def}|\\
|\childdocforwardprefix[|\textit{main}|]{|\textit{prefix}|}{|\textit{dest}|}|
\end{tabular}
\end{center}
%
the destination file is determined by a pattern
depending on the current file:
To make this work, the current file must be called
`{\textit{prefix}\hspace{0.2em}\textit{suffix}}'
with \textit{prefix} matching precisely the argument.
Processing is then passed on to the file
`{\textit{dest}\hspace{0.2em}\textit{suffix}}'.
Surely, the same effect is achieved by
directly specifying the
argument `{\textit{dest}\hspace{0.2em}\textit{suffix}}'
in the first form.
However, that requires to set up a different file
for each child. With the alternative form of the command
all these files can have exactly the same content
which simplifies setting them up and maintaining them.

For example, the following file |draft.tex|
with a compilation flag |\version| as described in \secref{sec:flags}
compiles the main document as a draft:
%
\begin{center}
\begin{tabular}{l}
|\def\version{draft}|\\
|\input{childdoc.def}|\\
|\childdocforward{|\textit{main}|}|
\end{tabular}
\end{center}
%
Likewise, the following files |final|\textit{nn}|.tex|
compile the final version of the child document
|child|\textit{nn}|.tex|:
%
\begin{center}
\begin{tabular}{l}
|\def\version{final}|\\
|\input{childdoc.def}|\\
|\childdocforwardprefix{final}{child}|
\end{tabular}
\end{center}
%

Note that when several versions of a main file and/or of each child file
are to be generated, it may be convenient to set up a |Makefile| or
shell script to automatise the process.

%%%%%%%%%%%%%%%%%%%%%%%%%%%%%%%%%%%%%%%%%%%%%%%%%%%%%%%%%%%%%%%%%%%%%%%%%%%%%%%%
\subsection{Command Line Processing}
\label{sec:commandline}

The effect of redirection files can also be achieved by invoking
the \LaTeX{} compiler with a more elaborate command line.
Most conveniently this should be done as part
of a shell script or a |Makefile|.

When using \textsf{childdoc} in the main file, the following
command lines effectively perform a redirection
(note that depending on the shell being used,
backslashes may have to be doubled: `|\|' $\to$ `|\\|'):
%
\begin{center}
|... -jobname "|\textit{target}|" |\\|"|[\textit{flags}]%
|\input{childdoc.def}\childdocforward[|\textit{main}|]{|\textit{dest}|}"|
\end{center}
%
Here \textit{target} is the name of the output file,
\textit{main} is the name of the main file
and \textit{dest} is the name of the main or child file to be processed
(all filenames without extensions).
The optional argument \textit{main} can be omitted
if \textit{main} matches \textit{dest}.
Optionally, compilation \textit{flags} can be defined via |\def| commands.
This command line makes the \TeX{} engine believe
it is compiling the file \textit{target}
whose content is specified as the latter parameter.
The provided code then forwards the processing to
\textit{main} or \textit{dest} as described in \secref{sec:forward}.

%%%%%%%%%%%%%%%%%%%%%%%%%%%%%%%%%%%%%%%%%%%%%%%%%%%%%%%%%%%%%%%%%%%%%%%%%%%%%%%%
\subsection{Include by Input}
\label{sec:input}

Including child documents by |\include| has some restrictions by design.
Most notably, the content of a child document always occupies
its own set of pages; pages cannot be shared between child documents.
Usually, this behaviour makes perfect sense
because each child document contain an essential part of the document.
However, in some situations it may be desirable to compose
a document from a collection of parts
without having mandatory page breaks between then.
For this case, the package
provides a mechanism to include parts
by |\input| which can also be processed individually.
However, by construction this mechanism
requires manual handling of the content to be output.

%%%%%%%%%%%%%%%%%%%%%%%%%%%%%%%%%%%%%%%%
\DescribeMacro{\ifchilddocmanual}
The main file should be prepared as usual, see \secref{sec:include}.
However, the document body must make a distinction
between processing of an individual part and of the main document, e.g.:
%
\begin{center}
\begin{tabular}{l}
|\ifchilddocmanual|\\
|\input{\childdocname}|\\
|\||else|\\
\textit{document body with }|\input{|\textit{part}|}|\\
|\||fi|
\end{tabular}
\end{center}
%
The conditional |\ifchilddocmanual| is true whenever
a part to be included by |\input| is being compiled,
and the name of the part is stored in |\childdocname|.

%%%%%%%%%%%%%%%%%%%%%%%%%%%%%%%%%%%%%%%%
\DescribeMacro{\childdocby}
Each part to be included by |\input| should start with:
%
\begin{center}
\begin{tabular}{l}
|\input{childdoc.def}|\\
|\childdocby{|\textit{main}|}|\\
\end{tabular}
\end{center}
%
The directive |\childdocby| is similar to |\childdocof|
described in \secref{sec:include},
but the subsequent selection of content must be done manually.
To that end, both |\ifchilddoc| and |\ifchilddocmanual|
will be true upon processing of a part,
and the name of the part is stored in |\childdocname|.
Note that |\jobname| will be set to the filename of the current part
so that each part receives an individual |.aux| file
that does not interfere with the |.aux| file(s) of the main document.
This behaviour can be altered by the alternative form
|\childdocby[*]{|\textit{main}|}| (with a non-empty optional argument)
which uses the |.aux| file of the main document
by setting |\jobname| to \textit{main}.

%%%%%%%%%%%%%%%%%%%%%%%%%%%%%%%%%%%%%%%%%%%%%%%%%%%%%%%%%%%%%%%%%%%%%%%%%%%%%%%%
\subsection{Driver Development}
\label{sec:driver}

The \textsf{childdoc} mechanism can also be use for the development
of definition files such as \LaTeX{} styles or classes.
This case differs from the above setup with multiple parts
included by |\include| in that no |\includeonly| should be invoked.
This can be achieved by starting the include file
(before |\ProvidesPackage|) with:
%
\begin{center}
\begin{tabular}{l}
|\input{childdoc.def}|\\
|\childdocforward{|\textit{main}|}|\\
\end{tabular}
\end{center}
%
or alternatively with:
%
\begin{center}
\begin{tabular}{l}
|\input{childdoc.def}|\\
|\childdocby{|\textit{main}|}|\\
\end{tabular}
\end{center}
%
Both forms have slightly different effects as described above.
The main file is prepared as usual, see \secref{sec:include}.

%%%%%%%%%%%%%%%%%%%%%%%%%%%%%%%%%%%%%%%%%%%%%%%%%%%%%%%%%%%%%%%%%%%%%%%%%%%%%%%%
\subsection{Legacy Detection}
\label{sec:detection}

The directive |\childdocmain| in the main file can detect
whether the complete document or merely a child is to be compiled
even without using the directive |\childdocof|.
This method is deprecated because it is less robust
and there is no compelling reason to use it;
it is merely provided for backward compatibility
and it may be removed in future versions.

If the detection mechanism is to be used,
it is mandatory to correctly specify
the filename of the main file as the argument of |\childdocmain|:
%
\begin{center}
\begin{tabular}{l}
|\input{childdoc.def}|\\
|\childdocmain{|\textit{main}|}|\\
\end{tabular}
\end{center}
%
If |\jobname| does not match the argument \textit{main} of |\childdocmain|,
it is assumed that |\jobname| points to the child file to be compiled.
When using |\childdocmain| with the main file specified as argument,
it suffices to start a child file
with just |\input{|\textit{main}|}|
without loading of the package and using |\childdocof|.
If instead all processing is done
with the appropriate \textsf{childdoc} directives,
the argument of \textit{main} of |\childdocmain| can be empty.

An alternative version of the command line processing described
in \secref{sec:commandline} using the detection mechanism reads:
%
\begin{center}
|... -jobname "|\textit{target}|" "|[\textit{flags}]%
[|\def\jobname{|\textit{dest}|}|]|\input{|\textit{main}|}"|
\end{center}

%%%%%%%%%%%%%%%%%%%%%%%%%%%%%%%%%%%%%%%%%%%%%%%%%%%%%%%%%%%%%%%%%%%%%%%%%%%%%%%%
\subsection{Manual Code}
\label{sec:manual}

In case one cannot be certain whether the definitions file |childdoc.def|
is installed on the target \TeX{} distribution
and one prefers not to ship it,
it is conceivable to paste a few relevant commands into the sources.

To that end, drop all statements |\input{childdoc.def}|
and perform the replacements as outlined below.
Instead of |\childdocmain{|\textit{main}|}| add the following code
to the top of the main file:
%
\begin{center}
\begin{tabular}{l}
|\||ifdefined\childdocname\endinput\||fi\newif\ifchilddoc|\\
|\edef\childdocname{\scantokens\expandafter{\jobname\noexpand}}|\\
|\def\childdocmain{|\textit{main}|}\||ifx\childdocmain\childdocname\||else|\\
|\childdoctrue\includeonly{\childdocname}\let\jobname\childdocmain\||fi|\\
\end{tabular}
\end{center}
%
Instead of |\childdocof{|\textit{main}|}| just include the main file
at the top of each child file:
%
\begin{center}
|\input{|\textit{main}|}|
\end{center}
%
A simple redirection |\childdocforward{|\textit{dest}|}| is achieved by:
%
\begin{center}
|\def\jobname{|\textit{dest}|}\input{\jobname}|
\end{center}
%
The redirection with prefix
|\childdocforwardprefix[|\textit{prefix}|]{|\textit{dest}|}|
is accomplished by:
%
\begin{center}
\begin{tabular}{l}
|{\edef\jobname{\scantokens\expandafter{\jobname\noexpand}}|\\
|\def\redirectjob |\textit{prefix}|#1~~~{\gdef\jobname{|\textit{dest}|#1}}|\\
|\expandafter\redirectjob\jobname~~~}\input{\jobname}|
\end{tabular}
\end{center}

In an alternative approach,
child documents can be compiled by a specific command line
without additional code or specific definitions:
%
\begin{center}
|... -jobname "|\textit{target}|" "|[\textit{flags}]%
|\includeonly{|\textit{dest}|}\input{|\textit{main}|}"|
\end{center}
%

%%%%%%%%%%%%%%%%%%%%%%%%%%%%%%%%%%%%%%%%%%%%%%%%%%%%%%%%%%%%%%%%%%%%%%%%%%%%%%%%
%%%%%%%%%%%%%%%%%%%%%%%%%%%%%%%%%%%%%%%%%%%%%%%%%%%%%%%%%%%%%%%%%%%%%%%%%%%%%%%%
\section{Information}

%%%%%%%%%%%%%%%%%%%%%%%%%%%%%%%%%%%%%%%%%%%%%%%%%%%%%%%%%%%%%%%%%%%%%%%%%%%%%%%%
\subsection{Copyright}

Copyright \copyright{} 2017--2018 Niklas Beisert

This work may be distributed and/or modified under the
conditions of the \LaTeX{} Project Public License, either version 1.3
of this license or (at your option) any later version.
The latest version of this license is in
  \url{http://www.latex-project.org/lppl.txt}
and version 1.3 or later is part of all distributions of \LaTeX{}
version 2005/12/01 or later.

This work has the LPPL maintenance status `maintained'.

The Current Maintainer of this work is Niklas Beisert.

This work consists of the files |README.txt|, |childdoc.ins| and |childdoc.dtx|
as well as the derived files |childdoc.def|, |cdocsamp.tex|
with |cdocsch1.tex|, |cdocsch2.tex|, |cdocspt3.tex|, |cdocspt4.tex|,
|cdocsdrf.tex|, |cdocsfn1.tex|, |cdocsfn2.tex|
as well as |childdoc.pdf|.

%%%%%%%%%%%%%%%%%%%%%%%%%%%%%%%%%%%%%%%%%%%%%%%%%%%%%%%%%%%%%%%%%%%%%%%%%%%%%%%%
\subsection{Files and Installation}

The package consists of the files:
%
\begin{center}
\begin{tabular}{ll}
    |README.txt|   & readme file \\
    |childdoc.ins| & installation file \\
    |childdoc.dtx| & source file \\
    |childdoc.def| & definition file \\
    |cdocsamp.tex| & sample main file \\
    |cdocsch1.tex| & sample include file \\
    |cdocsch2.tex| & sample include file \\
    |cdocspt3.tex| & sample part file \\
    |cdocspt4.tex| & sample part file \\
    |cdocsdrf.tex| & sample redirection file \\
    |cdocsfn1.tex| & sample redirection file \\
    |cdocsfn2.tex| & sample redirection file \\
    |childdoc.pdf| & manual
\end{tabular}
\end{center}
%
The distribution consists of the files
|README.txt|, |childdoc.ins| and |childdoc.dtx|.
%
\begin{itemize}
\item
Run (pdf)\LaTeX{} on |childdoc.dtx|
to compile the manual |childdoc.pdf| (this file).
\item
Run \LaTeX{} on |childdoc.ins| to create the definitions file |childdoc.def|
and the sample |cdocsamp.tex| with include files
|cdocsch1.tex|, |cdocsch2.tex|, |cdocspt3.tex|, |cdocspt4.tex|,
|cdocsdrf.tex|, |cdocsfn1.tex|, |cdocsfn2.tex|.
Then copy the file |childdoc.def| to an appropriate directory of your \LaTeX{}
distribution, e.g.\ \textit{texmf-root}|/tex/latex/childdoc|.
\end{itemize}

%%%%%%%%%%%%%%%%%%%%%%%%%%%%%%%%%%%%%%%%%%%%%%%%%%%%%%%%%%%%%%%%%%%%%%%%%%%%%%%%
\subsection{Related CTAN Packages}

There are several other packages which offer a similar functionality:
%
\begin{itemize}
\item
The packages
\href{http://ctan.org/pkg/docmute}{\textsf{docmute}},
\href{http://ctan.org/pkg/includex}{\textsf{includex}} and
\href{http://ctan.org/pkg/standalone}{\textsf{standalone}}
provide commands to include only the document body of
a child file thus allowing both files to be compiled individually.
\item
The packages \href{http://ctan.org/pkg/subdocs}{\textsf{subdocs}}
and \href{http://ctan.org/pkg/subfiles}{\textsf{subfiles}}
provide structures in which the main and child documents can be
encapsulated and allowing them to be compiled individually.
The inclusion mechanism is different from the conventional |\include|.
\item
The package \href{http://ctan.org/pkg/combine}{\textsf{combine}}
is an elaborate solution to combine several documents into one.
\end{itemize}
%
See also the CTAN topic \href{http://ctan.org/topic/subdocs}{\textsf{subdocs}}
for further related packages.
The present package differs from the above solutions in that
a document structure constructed with the conventional |\include| mechanism
just needs two extra commands at the top of every file
such that all constituent files can be compiled individually.

%%%%%%%%%%%%%%%%%%%%%%%%%%%%%%%%%%%%%%%%%%%%%%%%%%%%%%%%%%%%%%%%%%%%%%%%%%%%%%%%
%\subsection{Feature Suggestions}
%
%The following is a list of features which may be useful for future
%versions of this package:
%%
%\begin{itemize}
%\item
%\ldots
%\end{itemize}

%%%%%%%%%%%%%%%%%%%%%%%%%%%%%%%%%%%%%%%%%%%%%%%%%%%%%%%%%%%%%%%%%%%%%%%%%%%%%%%%
\subsection{Revision History}

%%%%%%%%%%%%%%%%%%%%%%%%%%%%%%%%%%%%%%%%
\paragraph{v2.0:} 2018/12/30

\begin{itemize}
\item
immediate forward processing
\item
added |\childdocby| mechanism
\item
manual restructured
\end{itemize}

%%%%%%%%%%%%%%%%%%%%%%%%%%%%%%%%%%%%%%%%
\paragraph{v1.6:} 2018/01/17

\begin{itemize}
\item
application for development of include files
\item
corrections to manual
\end{itemize}

%%%%%%%%%%%%%%%%%%%%%%%%%%%%%%%%%%%%%%%%
\paragraph{v1.5:} 2017/05/21

\begin{itemize}
\item
more complete structuring introduced
\item
|\childdocof| introduced
\item
|\childdoc| renamed to |\childdocmain|
\item
|\childredirect| renamed to |\childdocforward| and |\childdocforwardprefix|
and functionality expanded
\end{itemize}

%%%%%%%%%%%%%%%%%%%%%%%%%%%%%%%%%%%%%%%%
\paragraph{v1.0:} 2017/04/27

\begin{itemize}
\item
manual and install package
\item
first version published on CTAN
\end{itemize}

%%%%%%%%%%%%%%%%%%%%%%%%%%%%%%%%%%%%%%%%
\paragraph{v0.6:} 2017/04/26

\begin{itemize}
\item
redirection mechanism added
\end{itemize}

%%%%%%%%%%%%%%%%%%%%%%%%%%%%%%%%%%%%%%%%
\paragraph{v0.5:} 2017/04/26

\begin{itemize}
\item
functionality in definition file
\end{itemize}


%%%%%%%%%%%%%%%%%%%%%%%%%%%%%%%%%%%%%%%%%%%%%%%%%%%%%%%%%%%%%%%%%%%%%%%%%%%%%%%%
%%%%%%%%%%%%%%%%%%%%%%%%%%%%%%%%%%%%%%%%%%%%%%%%%%%%%%%%%%%%%%%%%%%%%%%%%%%%%%%%
%%%%%%%%%%%%%%%%%%%%%%%%%%%%%%%%%%%%%%%%%%%%%%%%%%%%%%%%%%%%%%%%%%%%%%%%%%%%%%%%
\appendix

\settowidth\MacroIndent{\rmfamily\scriptsize 000\ }

 \DocInput{childdoc.dtx}

\end{document}
%</driver>
% \fi
%
% %%%%%%%%%%%%%%%%%%%%%%%%%%%%%%%%%%%%%%%%%%%%%%%%%%%%%%%%%%%%%%%%%%%%%%%%%%%%%%
% %%%%%%%%%%%%%%%%%%%%%%%%%%%%%%%%%%%%%%%%%%%%%%%%%%%%%%%%%%%%%%%%%%%%%%%%%%%%%%
% \section{Sample}
%\iffalse
%<*samplemain>
%\fi
%
% The following presents a sample document
% with two chapters, two parts, a title page,
% a compile flag as well as three forwarding files to set the flag.
% It consists of eight |.tex| files:
% \begin{center}
% \begin{tabular}{ll}
% |cdocsamp.tex|&main file\\
% |cdocsch1.tex|&include file for chapter 1\\
% |cdocsch2.tex|&include file for chapter 2\\
% |cdocspt3.tex|&include file for part 3\\
% |cdocspt4.tex|&include file for part 4\\
% |cdocsdrf.tex|&forwarding file for main file in draft mode\\
% |cdocsfi1.tex|&forwarding file for final version of chapter 1\\
% |cdocsfi2.tex|&forwarding file for final version of chapter 2\\
% \end{tabular}
% \end{center}
% Each of the eight files can be compiled directly by the \LaTeX{} compiler.
%
% %%%%%%%%%%%%%%%%%%%%%%%%%%%%%%%%%%%%%%
% \paragraph{Main File.}
%
% The main file is called |cdocsamp.tex|.
%
% Load the \textsf{childdoc} definitions and
% declare the filename for the main document:
%    \begin{macrocode}
\input{childdoc.def}
\childdocmain{}
%    \end{macrocode}

% Optional override for |\version| flag:
%    \begin{macrocode}
%%\ifchilddoc\else\providecommand{\version}{draft}\fi
%    \end{macrocode}

% Define the default values for the |\version| flag
% (|final| for the main file and |draft| for childs):
%    \begin{macrocode}
\ifchilddoc
\providecommand{\version}{draft}
\else
\providecommand{\version}{final}
\fi
%    \end{macrocode}

% Load the standard document class:
%    \begin{macrocode}
\documentclass[12pt]{article}
%    \end{macrocode}

% Start the document body:
%    \begin{macrocode}
\begin{document}
%    \end{macrocode}

% Declare a title page.
% Print title, part of document being processed and version flag:
%    \begin{macrocode}
\addtocounter{page}{-1}
\begin{center}
{\LARGE\bfseries{}childdoc example\par}
\vspace{1cm}
\ifchilddoc
\ifchilddocmanual part\else chapter\fi:
`\childdocname' of `\childdocjob'\par
\else
main document: `\childdocjob'\par
\fi
version: \version\par
\end{center}
\newpage
%    \end{macrocode}

% Manually include selected file,
% otherwise process as usual:
%    \begin{macrocode}
\ifchilddocmanual
\section*{part `\childdocname'}
\input{\childdocname}
\else
%    \end{macrocode}

% Include the two chapters:
%    \begin{macrocode}
\include{cdocsch1}
\include{cdocsch2}
%    \end{macrocode}

% Include the two parts unless only chapters should be displayed:
%    \begin{macrocode}
\ifchilddoc\else
\section{part three}
\input{cdocspt3}
\section{part four}
\input{cdocspt4}
\fi
%    \end{macrocode}

% Process as usual until here:
%    \begin{macrocode}
\fi
%    \end{macrocode}

% End of document body:
%    \begin{macrocode}
\end{document}
%    \end{macrocode}
%\iffalse
%</samplemain>
%\fi
%
% %%%%%%%%%%%%%%%%%%%%%%%%%%%%%%%%%%%%%%
% \paragraph{Chapter Include Files.}
%
% The include files are called |cdocsch1.tex| and |cdocsch2.tex|.
%
%\iffalse
%<*samplechap1|samplechap2>
%\fi

% Optional override for |\version| flag:
%    \begin{macrocode}
%%\providecommand{\version}{final}
%    \end{macrocode}

% Include the main document:
%    \begin{macrocode}
\input{childdoc.def}
\childdocof{cdocsamp}
%    \end{macrocode}

%\iffalse
%</samplechap1|samplechap2>
%\fi
%
%\iffalse
%<*samplechap1>
%\fi
% Some text for chapter 1:
%    \begin{macrocode}
\section{one}
some text in chapter one
%    \end{macrocode}

%\iffalse
%</samplechap1>
%\fi
% Some text for chapter 2:
%\iffalse
%<*samplechap2>
%\fi
%    \begin{macrocode}
\section{two}
more text in chapter two
%    \end{macrocode}

%\iffalse
%</samplechap2>
%\fi
%
% %%%%%%%%%%%%%%%%%%%%%%%%%%%%%%%%%%%%%%
% \paragraph{Part Include Files.}
%
% The include files are called |cdocspt3.tex| and |cdocspt4.tex|.
%
%\iffalse
%<*samplepart3|samplepart4>
%\fi

% Optional override for |\version| flag:
%    \begin{macrocode}
%%\providecommand{\version}{final}
%    \end{macrocode}

% Include the main document:
%    \begin{macrocode}
\input{childdoc.def}
\childdocby{cdocsamp}
%    \end{macrocode}

%\iffalse
%</samplepart3|samplepart4>
%\fi
%
%\iffalse
%<*samplepart3>
%\fi
% Some text for part 3:
%    \begin{macrocode}
some text in part three
%    \end{macrocode}

%\iffalse
%</samplepart3>
%\fi
% Some text for part 4:
%\iffalse
%<*samplepart4>
%\fi
%    \begin{macrocode}
more text in part four
%    \end{macrocode}

%\iffalse
%</samplepart4>
%\fi
%
% %%%%%%%%%%%%%%%%%%%%%%%%%%%%%%%%%%%%%%
% \paragraph{Forwarding for a Complete Draft.}
%
% The following forwarding file |cdocsdrf.tex|
% compiles the main document in draft mode:
%\iffalse
%<*sampledraft>
%\fi
%    \begin{macrocode}
\def\version{draft}
\input{childdoc.def}
\childdocforward{cdocsamp}
%    \end{macrocode}

%\iffalse
%</sampledraft>
%\fi
%
% %%%%%%%%%%%%%%%%%%%%%%%%%%%%%%%%%%%%%%
% \paragraph{Forwarding for Final Version of the Chapters.}
%
% The following forwarding files |cdocsfn1.tex| and |cdocsfn2.tex|
% (with identical content)
% compile the final versions of the child documents
% |cdocsch1.tex| and |cdocsch2.tex|, respectively:
%\iffalse
%<*samplefinal>
%\fi
%    \begin{macrocode}
\def\version{final}
\input{childdoc.def}
\childdocforwardprefix[cdocsamp]{cdocsfn}{cdocsch}
%    \end{macrocode}

%\iffalse
%</samplefinal>
%\fi
%
% %%%%%%%%%%%%%%%%%%%%%%%%%%%%%%%%%%%%%%
% \paragraph{Command Line Processing.}
%
% The following three command lines generate the output files
% |cdocscld|, |cdocscl1| and |cdocscl2|
% which should be identical to
% |cdocsdrf|, |cdocsch1| and |cdocsfn2|, respectively:
% \begin{center}
% \begin{tabular}{l}
% |latex -jobname cdocscld \|\\
% |  "\def\version{draft}\input{childdoc.def}\childdocforward{cdocsamp}"|\\
% |latex -jobname cdocscl1 \|\\
% |  "\input{childdoc.def}\childdocforward[cdocsamp]{cdocsch1}"|\\
% |latex -jobname cdocscl2 \|\\
% |  "\def\version{final}\input{childdoc.def}\childdocforward{cdocsch2}"|
% \end{tabular}
% \end{center}
% Note that the trailing backslash on each first line
% merely continues the input to the second line
% (for convenient cut ant paste).
% Furthermore, the command |latex| can be replaced by any
% of its alternative versions such as |pdflatex|.
%
% %%%%%%%%%%%%%%%%%%%%%%%%%%%%%%%%%%%%%%%%%%%%%%%%%%%%%%%%%%%%%%%%%%%%%%%%%%%%%%
% %%%%%%%%%%%%%%%%%%%%%%%%%%%%%%%%%%%%%%%%%%%%%%%%%%%%%%%%%%%%%%%%%%%%%%%%%%%%%%
% \section{Implementation}
%\iffalse
%<*package>
%\fi
%
% This section describes the definitions file |childdoc.def|.

% The definitions cannot be loaded using |\usepackage| or |\RequirePackage|
% which has a mechanism to prevent loading a style file more than once.
% When loading the definitions by means of |\input|
% multiple instances have to be prevented manually:
%\iffalse
%This code needs to be before the `\ProvidesFile' directive
%which is defined at the beginning of this file.
%Therefore it is also placed there and commented out here.
%</package>
%<*discard>
%\fi
%    \begin{macrocode}
\ifdefined\childdocmain\endinput\fi
%    \end{macrocode}
%\iffalse
%</discard>
%<*package>
%\fi
%
% \macro{\ifchilddoc}
% \macro{\ifchilddocmanual}
% The conditional |\ifchilddoc| tells whether a
% child (true) or main (false) document is being compiled.
% The conditional |\ifchilddocmanual| tells whether
% the |\includeonly| mechanism is used (false) or
% the selection of child files must be performed manually (true).
% The definitions initialise to false:
%    \begin{macrocode}
\newif\ifchilddoc
\newif\ifchilddocmanual
%    \end{macrocode}

% \macro{\childdocname}
% \macro{\childdocjob}
% The macro |\childdocname| stores the name of the main document
% to be compiled. The macro |\childdocjob| stores the name of
% the document on which the \LaTeX{} compiler was originally invoked.
% The content of |\jobname| cannot be compared
% to filenames specified in the source due to different catcodes.
% The following code rescans |\jobname|, stores the result
% in |\childdocname| and saves a copy in |\childdocjob|:
%    \begin{macrocode}
\edef\childdocname{\scantokens\expandafter{\jobname\noexpand}}
\let\childdocjob\childdocname
%    \end{macrocode}

% \macro{\childdocdisable}
% The macro |\childdocdisable| prevents the main file
% from being processed more than once.
% At this stage, the main document command |\childdocmain|
% is assumed to be called once again where it should do nothing.
% Any subsequent call to it should prevent
% a secondary processing of the main document
% It overwrites the forwarding commands
% |\childdocof| and |\childdocforward|
% with empty macros to prevent further inclusions of the main document:
%    \begin{macrocode}
\newcommand{\childdocdisable}
{
  \renewcommand{\childdocmain}[1]{\renewcommand{\childdocmain}[1]{\endinput}}
  \renewcommand{\childdocof}[1]{}
  \renewcommand{\childdocby}[2][]{}
  \renewcommand{\childdocforward}[2][]{}
  \renewcommand{\childdocdisable}{}
}
%    \end{macrocode}

% \macro{\childdocmain}
% The macro |\childdocmain| is to be called at the top of the main file
% with nothing or the main filename (without extension) as argument.
% First, it breaks loops.
% If the argument is not empty and does not match |\childdocname|
% (which is set by the first inclusion of |childdoc.def|),
% |\ifchilddoc| is set to true, |\includeonly| is applied to the child file
% and |\jobname| is set to the main file
% (for proper handling of |.aux| files):
%    \begin{macrocode}
\newcommand{\childdocmain}[1]
{
  \childdocdisable\childdocmain{}
  \if?#1?\else
    \begingroup
      \def\childdoctmp{#1}
      \ifx\childdoctmp\childdocname
        \def\childdoctmp{}
      \else
        \def\childdoctmp
        {
          \childdoctrue
          \includeonly{\childdocname}
          \def\childdocjob{#1}
          \def\jobname{#1}
        }
      \fi
      \expandafter
    \endgroup
    \childdoctmp
  \fi
}
%    \end{macrocode}

% \macro{\childdocof}
% The command |\childdocof| redirects
% compilation to the main file |#1|.
%    \begin{macrocode}
\newcommand{\childdocof}[1]
{
  \childdocdisable
  \childdoctrue
  \includeonly{\childdocname}
  \def\jobname{#1}
  \def\childdocjob{#1}
  \input{#1}
}
%    \end{macrocode}

% \macro{\childdocby}
% The command |\childdocby| ....
%    \begin{macrocode}
\newcommand{\childdocby}[2][]
{
  \childdocdisable
  \childdoctrue
  \childdocmanualtrue
  \if?#1?\else
    \def\jobname{#2}
  \fi
  \def\childdocjob{#2}
  \input{#2}
  \endinput
}
%    \end{macrocode}

% \macro{\childdocforward}
% The command |\childdocforward| redirects
% compilation to the main file or
% (if the optional argument is given) a child file.
% Parameters are set as if the main file
% or a child file starting with |\childdocof| was compiled.
% Then compilation is handed over to the main file:
%    \begin{macrocode}
\newcommand{\childdocforward}[2][]
{
  \begingroup
    \if?#1?
      \def\childdoctmp
      {
        \def\childdocname{#2}
        \def\childdocjob{#2}
        \def\jobname{#2}
        \input{#2}
        \endinput
      }
    \else
      \def\childdoctmp
      {
        \childdocdisable
        \def\childdocname{#2}
        \childdoctrue
        \includeonly{#2}
        \def\childdocjob{#1}
        \def\jobname{#1}
        \input{#1}
        \endinput
      }
    \fi
    \expandafter
  \endgroup
  \childdoctmp
}
%    \end{macrocode}

% \macro{\childdocforwardprefix}
% The command |\childdocforwardprefix| redirects
% compilation to the main or a child file by means of a pattern.
% The prefix |#1| in the current filename is replaced by |#2|
% and the suffix of the current filename is kept
% (it is assumed that the filename does not contain the substring `|~~~|'
% which is used as a delimiter).
% Compilation is handed over to the new file by |\childdocforward|:
%    \begin{macrocode}
\newcommand{\childdocforwardprefix}[3][]
{
  \begingroup
    \def\childdocextract #2##1~~~{\def\childdoctmp{\childdocforward[#1]{#3##1}}}
    \expandafter\childdocextract\childdocname~~~
    \expandafter
  \endgroup
  \childdoctmp
}
%    \end{macrocode}

% \macro{\childdoc}
% The deprecated macro |\childdoc| is a legacy version of |\childdocmain|:
%    \begin{macrocode}
\newcommand{\childdoc}{\childdocmain}
%    \end{macrocode}

% \macro{\childdocredirect}
% The deprecated macro |\childdocredirect| is a legacy version
% of |\childdocforward| and |\childdocforwardprefix|:
%    \begin{macrocode}
\newcommand{\childdocredirect}[2][]
{
  \begingroup
    \if?#1?
      \def\childdoctmp{\childdocforward{#2}}
    \else
      \def\childdoctmp{\childdocforwardprefix{#1}{#2}}
    \fi
    \expandafter
  \endgroup
  \childdoctmp
}
%    \end{macrocode}

%\iffalse
%</package>
%\fi
%
\endinput
|\\
|\childdocforward{|\textit{main}|}|
\end{tabular}
\end{center}
%
Likewise, the following files |final|\textit{nn}|.tex|
compile the final version of the child document
|child|\textit{nn}|.tex|:
%
\begin{center}
\begin{tabular}{l}
|\def\version{final}|\\
|% \iffalse
%
% childdoc.dtx Copyright (C) 2017-2018 Niklas Beisert
%
% This work may be distributed and/or modified under the
% conditions of the LaTeX Project Public License, either version 1.3
% of this license or (at your option) any later version.
% The latest version of this license is in
%   http://www.latex-project.org/lppl.txt
% and version 1.3 or later is part of all distributions of LaTeX
% version 2005/12/01 or later.
%
% This work has the LPPL maintenance status `maintained'.
%
% The Current Maintainer of this work is Niklas Beisert.
%
% This work consists of the files childdoc.dtx and childdoc.ins
% and the derived files childdoc.def and cdocsamp.tex with
% cdocsch1.tex, cdocsch2.tex, cdocsdrf.tex, cdocsfn1.tex, cdocsfn2.tex.
%
%<package>\ifdefined\childdocmain\endinput\fi
%<package>\ProvidesFile{childdoc.def}[2018/12/30 v2.0 child document driver]
%<samplemain>\ProvidesFile{cdocsamp.tex}[2018/12/30 v2.0 sample for childdoc]
%<*driver>
%\ProvidesFile{childdoc.drv}[2018/12/30 v2.0 childdoc reference manual file]
\PassOptionsToClass{10pt,a4paper}{article}
\documentclass{ltxdoc}

\usepackage[margin=35mm]{geometry}
\usepackage{hyperref}
\usepackage{hyperxmp}
\usepackage[usenames]{color}

\hypersetup{colorlinks=true}
\hypersetup{pdfstartview=FitH}
\hypersetup{pdfpagemode=UseNone}
\hypersetup{pdfsource={}}
\hypersetup{pdflang={en-UK}}
\hypersetup{pdfcopyright={Copyright 2017-2018 Niklas Beisert.
  This work may be distributed and/or modified under the
  conditions of the LaTeX Project Public License, either version 1.3
  of this license or (at your option) any later version.}}
\hypersetup{pdflicenseurl={http://www.latex-project.org/lppl.txt}}
\hypersetup{pdfcontactaddress={ETH Zurich, ITP, HIT K,
  Wolfgang-Pauli-Strasse 27}}
\hypersetup{pdfcontactpostcode={8093}}
\hypersetup{pdfcontactcity={Zurich}}
\hypersetup{pdfcontactcountry={Switzerland}}
\hypersetup{pdfcontactemail={nbeisert@itp.phys.ethz.ch}}
\hypersetup{pdfcontacturl={http://people.phys.ethz.ch/\xmptilde nbeisert/}}

\newcommand{\secref}[1]{\hyperref[#1]{section \ref*{#1}}}

\parskip1ex
\parindent0pt
\let\olditemize\itemize
\def\itemize{\olditemize\parskip0pt}

\begin{document}

\title{The \textsf{childdoc} Package}
\hypersetup{pdftitle={The childdoc Package}}
\author{Niklas Beisert\\[2ex]
  Institut f\"ur Theoretische Physik\\
  Eidgen\"ossische Technische Hochschule Z\"urich\\
  Wolfgang-Pauli-Strasse 27, 8093 Z\"urich, Switzerland\\[1ex]
  \href{mailto:nbeisert@itp.phys.ethz.ch}
  {\texttt{nbeisert@itp.phys.ethz.ch}}}
\hypersetup{pdfauthor={Niklas Beisert}}
\hypersetup{pdfsubject={Manual for the LaTeX2e Package childdoc}}
\date{30 December 2018, \textsf{v2.0}}
\maketitle

\begin{abstract}\noindent
\textsf{childdoc} is a \LaTeXe{} package
that enables the direct compilation
of document sections included by |\include|
to individual files.
\end{abstract}

\begingroup
\parskip0ex
\tableofcontents
\endgroup

%%%%%%%%%%%%%%%%%%%%%%%%%%%%%%%%%%%%%%%%%%%%%%%%%%%%%%%%%%%%%%%%%%%%%%%%%%%%%%%%
%%%%%%%%%%%%%%%%%%%%%%%%%%%%%%%%%%%%%%%%%%%%%%%%%%%%%%%%%%%%%%%%%%%%%%%%%%%%%%%%
\section{Introduction}

\LaTeX{} provides a mechanism to structure a large document (such as a book)
into a main file and several child files (containing the chapters)
using the |\include| command.
This mechanism is beneficial for documents
which span hundreds of pages in order to
make the source file(s) more manageable.
Moreover, compilation can be restricted to
selected child files by means of the |\includeonly| command.
The latter feature can be used to reduce the compilation time while editing
(this was significantly more useful in the earlier days of \LaTeX{})
or to generate a smaller document which is easier to navigate.
Another application of |\includeonly| is to generate
documents consisting of selected parts of the complete document.

However, there are a few drawbacks of the plain |\include| mechanism:
\begin{itemize}
\item
The child files cannot be compiled on their own,
they can only be compiled via the main file.
A naive editing environment
(such as a text editor with an option
to have the current file processed by \LaTeX)
may require one to switch to the main file before compiling;
attempting to compile the child file produces errors.
\item
The main file must be modified (each time)
to adjust the |\includeonly| command
to the present needs. This easily leaves the main file in a messy state.
\item
The generated document will always carry the filename
of the main document. This is inconvenient if
several child files are to be compiled and
to be kept for distribution.
\end{itemize}

The present package provides a simple interface
to make child files individually compilable by \LaTeX{}.
Compiling a child file then has the same effect as compiling
the main file with an |\includeonly| command
to select the appropriate child.
Moreover the generated document will carry the name of the child
rather than the main file.
This resolves all three above issues.

This feature is meant to make the editing of books,
thesis documents and lecture notes somewhat more convenient.
However, the package can also be used efficiently for
composing a series of documents (such as exercise sheets)
which are typically distributed individually.
It then assists the author in generating the individual documents
(potentially in different versions)
as well as a document containing the collected series.
Another application is in developing style files
or other kinds of included material
where compilation of the style file could redirect
to a sample or test file.

%%%%%%%%%%%%%%%%%%%%%%%%%%%%%%%%%%%%%%%%%%%%%%%%%%%%%%%%%%%%%%%%%%%%%%%%%%%%%%%%
%%%%%%%%%%%%%%%%%%%%%%%%%%%%%%%%%%%%%%%%%%%%%%%%%%%%%%%%%%%%%%%%%%%%%%%%%%%%%%%%
\section{Usage}

First of all, the package \textsf{childdoc} is \emph{not} a standard
\LaTeXe{} |.sty| style file! Therefore it needs to be invoked in
a non-standard way.

%%%%%%%%%%%%%%%%%%%%%%%%%%%%%%%%%%%%%%%%%%%%%%%%%%%%%%%%%%%%%%%%%%%%%%%%%%%%%%%%
\subsection{Included Files}
\label{sec:include}

%%%%%%%%%%%%%%%%%%%%%%%%%%%%%%%%%%%%%%%%
\DescribeMacro{\childdocmain}
To use the package, add the commands
\begin{center}
\begin{tabular}{l}
|\input{childdoc.def}|\\
|\childdocmain{}|\\
\end{tabular}
\end{center}
at the very top of the main \LaTeX{} file,
in particular \emph{before} the |\documentclass| statement!
The argument of |\childdocmain| should be left empty
(but it must be present).

%%%%%%%%%%%%%%%%%%%%%%%%%%%%%%%%%%%%%%%%
\DescribeMacro{\childdocof}
Furthermore, add the commands
\begin{center}
\begin{tabular}{l}
|\input{childdoc.def}|\\
|\childdocof{|\textit{main}|}|\\
\end{tabular}
\end{center}
at the top of every child file \textit{child}
which is included by |\include{|\textit{child}|}|
from within the main file
(or at least for those files to be compiled individually).
The argument \textit{main} must be the filename of the main file.

There are a couple of
considerations in setting up the main and child documents:

%%%%%%%%%%%%%%%%%%%%%%%%%%%%%%%%%%%%%%%%
\paragraph{Restrictions.}

Please note the following restrictions:
\begin{itemize}
\item
|\childdocmain| must be called with one argument \textit{main}
to ensure compatibility with earlier version of the package.
It must either be empty (|\childdocmain{}|)
or precisely match the filename of the main file in which it is specified.
See \secref{sec:detection} for further information.
\item
The filename \textit{main} must be specified without the |.tex| extension.
\item
The filename \textit{main} is case sensitive
(even in case-insensitive file systems)
due to internal string comparison.
\item
The argument \textit{main} should be fully expanded, it cannot be a macro.
\item
Subdirectories and special characters should be avoided in filenames.
\item
The command |\childdocmain{|\textit{main}|}| must be followed by a whitespace.
It should not be followed immediately by another command
or by a comment mark `|%|'.
This is because the \TeX{} parser reads the token immediately following
the argument of |\childdocmain| and puts it
at the beginning of every child section;
however, a white\-space is ignored.
\end{itemize}

%%%%%%%%%%%%%%%%%%%%%%%%%%%%%%%%%%%%%%%%
\paragraph{Content of Main File.}

It is advisable to place all content in the child files included by |\include|.
Any output contained in the main file will appear in all child documents
unless suppressed manually;
it cannot be suppressed automatically by the |\includeonly| directive
and thus should normally be avoided.
A method to include some content in the main file
by means of conditional processing is described in \secref{sec:conditional}.

%%%%%%%%%%%%%%%%%%%%%%%%%%%%%%%%%%%%%%%%
\paragraph{Page Numbering.}

When only a part of the document is compiled,
the appropriate numbering of pages
(as well as other status parameters)
is determined from the |.aux| files.
The latter contain information from previous passes.
However this information needs to propagate through
all intermediate child documents.
Therefore the page numbering in child documents may well
be inconsistent until the complete document is compiled at least once.

A useful (if unconventional) way to always ensure a consistent
page numbering is to restart the numbering in each child document
and denote the pages by `\textit{child}|.|\textit{page}'
where \textit{child} represents the chapter/section number of the child file.
This can be achieved by the command
|\numberwithin{page}{|\textit{child}|}|
of the \textsf{amsmath} package
where \textit{child} can be |chapter| or |section|
depending on the chosen structuring.
Alternatively, one can modify the macro |\thepage| appropriately
and reset the counter |page| at the start of each child file.

%%%%%%%%%%%%%%%%%%%%%%%%%%%%%%%%%%%%%%%%%%%%%%%%%%%%%%%%%%%%%%%%%%%%%%%%%%%%%%%%
\subsection{Conditional Processing}
\label{sec:conditional}

The package provides a mechanism to compile different versions
of a document. To customise the versions further some conditional processing
can come in handy to distinguish which version is being compiled.
The package provides two macros to describe the compilation context:

%%%%%%%%%%%%%%%%%%%%%%%%%%%%%%%%%%%%%%%%
\DescribeMacro{\ifchilddoc}
The conditional |\ifchilddoc| distinguishes between the compilation of
child documents and the main document:
%
\begin{center}
|\ifchilddoc |\textit{child-code}| |[|\||else |\textit{main-code}]| \||fi|
\end{center}

%%%%%%%%%%%%%%%%%%%%%%%%%%%%%%%%%%%%%%%%
\DescribeMacro{\childdocname}
\DescribeMacro{\childdocjob}
The macro |\childdocname| contains the filename (without extension)
of the main or child file being processed.
Note that |\childdocjob| will always contain the name of the main file.

%%%%%%%%%%%%%%%%%%%%%%%%%%%%%%%%%%%%%%%%
\paragraph{Title Page.}

Conditional processing can be used to include a title or banner page
in the main document when proper precautions are taken.
Importantly, the code in the main file should ensure that the page counter
(as well as other status parameters which are stored in the |.aux| files)
takes the same value after the conditional processing.
Otherwise the page numbers may take divergent values
depending on which part is compiled.

For example, a title page could be declared by:
%
\begin{center}
\begin{tabular}{l}
|\ifchilddoc\||else|\\
|\addtocounter{page}{-1}|\\
\textit{code for title page}\\
|\newpage|\\
|\||fi|
\end{tabular}
\end{center}
%
A banner page for the child documents can be generated by:
%
\begin{center}
\begin{tabular}{l}
|\ifchilddoc|\\
|\addtocounter{page}{-1}|\\
\textit{code for banner page}\\
|\newpage|\\
|\||fi|
\end{tabular}
\end{center}
%
Here one could write a message such as:
\begin{center}
|This is the part \childdocname{} of \childdocjob{}.|
\end{center}

%%%%%%%%%%%%%%%%%%%%%%%%%%%%%%%%%%%%%%%%%%%%%%%%%%%%%%%%%%%%%%%%%%%%%%%%%%%%%%%%
\subsection{Flags}
\label{sec:flags}

The package makes it easy to generate different versions
of the main or child documents.
To this end compilation flags can be defined
and assigned different default values.
They will be particularly useful in conjunction
with the forwarding mechanism described in \secref{sec:forward}.

For example, it may be useful to have a flag |\version|
which can be set to |draft| or |final|.
The document source will contain some conditional code
depending on the value of |\version|.
Suppose further, the flag should default to |final| for the main file
and to |draft| for child files
which is a natural assignment for editing the document.
This is achieved by placing the following code
in the preamble of the main document
(below the |\childdocmain| directive):
%
\begin{center}
\begin{tabular}{l}
|\ifchilddoc|\\
|\providecommand{\version}{draft}|\\
|\||else|\\
|\providecommand{\version}{final}|\\
|\||fi|
\end{tabular}
\end{center}
%
The definition by |\providecommand| makes sure
that previous definitions are not overwritten.
Further statements |\providecommand{\version}{...}|
can thus be added before the above code to override it.

For the main file, one might add a line
(between |\childdocmain| and the above block)
%
\begin{center}
|%\ifchilddoc\||else\providecommand{\version}{draft}\||fi|
\end{center}
%
which can be uncommented to produce a draft version.
Likewise one can add a line to the very top of a child file
(above the |\childdocof{|\textit{main}|}| directive)
%
\begin{center}
|%\providecommand{\version}{final}|
\end{center}
%
which can be uncommented to produce the final version of this child document.

%%%%%%%%%%%%%%%%%%%%%%%%%%%%%%%%%%%%%%%%%%%%%%%%%%%%%%%%%%%%%%%%%%%%%%%%%%%%%%%%
\subsection{Forwarding}
\label{sec:forward}

Different versions of the main or child documents
using compilation flags as described in \secref{sec:flags}
can be (permanently) stored in different files
for convenient compilation, viewing and distribution.
To this end, the package defines a command
to pass on compilation to a different file:

%%%%%%%%%%%%%%%%%%%%%%%%%%%%%%%%%%%%%%%%
\DescribeMacro{\childdocforward}
The command |\childdocforward| redirects processing to
another source file:
%
\begin{center}
\begin{tabular}{l}
|\input{childdoc.def}|\\
|\childdocforward[|\textit{main}|]{|\textit{dest}|}|\\
\end{tabular}
\end{center}
%
The argument \textit{dest} is the destination file
(without extension).
It should be the main file or one of the child files.
Note that further \textsf{childdoc} directives
such as |\childdocof| and |\childdocforward|
in the indicated file will be processed in this form.
The optional argument \textit{main}
passes on directly to the main file \textit{main}
while pretending to compile the child \textit{dest}.
This form behaves as if \textit{dest}
issues |\childdocof{|\textit{main}|}| right away,
and no further \textsf{childdoc} directives will be processed.

%%%%%%%%%%%%%%%%%%%%%%%%%%%%%%%%%%%%%%%%
\DescribeMacro{\...prefix}
In the alternative form |\childdocforwardprefix|,
%
\begin{center}
\begin{tabular}{l}
|\input{childdoc.def}|\\
|\childdocforwardprefix[|\textit{main}|]{|\textit{prefix}|}{|\textit{dest}|}|
\end{tabular}
\end{center}
%
the destination file is determined by a pattern
depending on the current file:
To make this work, the current file must be called
`{\textit{prefix}\hspace{0.2em}\textit{suffix}}'
with \textit{prefix} matching precisely the argument.
Processing is then passed on to the file
`{\textit{dest}\hspace{0.2em}\textit{suffix}}'.
Surely, the same effect is achieved by
directly specifying the
argument `{\textit{dest}\hspace{0.2em}\textit{suffix}}'
in the first form.
However, that requires to set up a different file
for each child. With the alternative form of the command
all these files can have exactly the same content
which simplifies setting them up and maintaining them.

For example, the following file |draft.tex|
with a compilation flag |\version| as described in \secref{sec:flags}
compiles the main document as a draft:
%
\begin{center}
\begin{tabular}{l}
|\def\version{draft}|\\
|\input{childdoc.def}|\\
|\childdocforward{|\textit{main}|}|
\end{tabular}
\end{center}
%
Likewise, the following files |final|\textit{nn}|.tex|
compile the final version of the child document
|child|\textit{nn}|.tex|:
%
\begin{center}
\begin{tabular}{l}
|\def\version{final}|\\
|\input{childdoc.def}|\\
|\childdocforwardprefix{final}{child}|
\end{tabular}
\end{center}
%

Note that when several versions of a main file and/or of each child file
are to be generated, it may be convenient to set up a |Makefile| or
shell script to automatise the process.

%%%%%%%%%%%%%%%%%%%%%%%%%%%%%%%%%%%%%%%%%%%%%%%%%%%%%%%%%%%%%%%%%%%%%%%%%%%%%%%%
\subsection{Command Line Processing}
\label{sec:commandline}

The effect of redirection files can also be achieved by invoking
the \LaTeX{} compiler with a more elaborate command line.
Most conveniently this should be done as part
of a shell script or a |Makefile|.

When using \textsf{childdoc} in the main file, the following
command lines effectively perform a redirection
(note that depending on the shell being used,
backslashes may have to be doubled: `|\|' $\to$ `|\\|'):
%
\begin{center}
|... -jobname "|\textit{target}|" |\\|"|[\textit{flags}]%
|\input{childdoc.def}\childdocforward[|\textit{main}|]{|\textit{dest}|}"|
\end{center}
%
Here \textit{target} is the name of the output file,
\textit{main} is the name of the main file
and \textit{dest} is the name of the main or child file to be processed
(all filenames without extensions).
The optional argument \textit{main} can be omitted
if \textit{main} matches \textit{dest}.
Optionally, compilation \textit{flags} can be defined via |\def| commands.
This command line makes the \TeX{} engine believe
it is compiling the file \textit{target}
whose content is specified as the latter parameter.
The provided code then forwards the processing to
\textit{main} or \textit{dest} as described in \secref{sec:forward}.

%%%%%%%%%%%%%%%%%%%%%%%%%%%%%%%%%%%%%%%%%%%%%%%%%%%%%%%%%%%%%%%%%%%%%%%%%%%%%%%%
\subsection{Include by Input}
\label{sec:input}

Including child documents by |\include| has some restrictions by design.
Most notably, the content of a child document always occupies
its own set of pages; pages cannot be shared between child documents.
Usually, this behaviour makes perfect sense
because each child document contain an essential part of the document.
However, in some situations it may be desirable to compose
a document from a collection of parts
without having mandatory page breaks between then.
For this case, the package
provides a mechanism to include parts
by |\input| which can also be processed individually.
However, by construction this mechanism
requires manual handling of the content to be output.

%%%%%%%%%%%%%%%%%%%%%%%%%%%%%%%%%%%%%%%%
\DescribeMacro{\ifchilddocmanual}
The main file should be prepared as usual, see \secref{sec:include}.
However, the document body must make a distinction
between processing of an individual part and of the main document, e.g.:
%
\begin{center}
\begin{tabular}{l}
|\ifchilddocmanual|\\
|\input{\childdocname}|\\
|\||else|\\
\textit{document body with }|\input{|\textit{part}|}|\\
|\||fi|
\end{tabular}
\end{center}
%
The conditional |\ifchilddocmanual| is true whenever
a part to be included by |\input| is being compiled,
and the name of the part is stored in |\childdocname|.

%%%%%%%%%%%%%%%%%%%%%%%%%%%%%%%%%%%%%%%%
\DescribeMacro{\childdocby}
Each part to be included by |\input| should start with:
%
\begin{center}
\begin{tabular}{l}
|\input{childdoc.def}|\\
|\childdocby{|\textit{main}|}|\\
\end{tabular}
\end{center}
%
The directive |\childdocby| is similar to |\childdocof|
described in \secref{sec:include},
but the subsequent selection of content must be done manually.
To that end, both |\ifchilddoc| and |\ifchilddocmanual|
will be true upon processing of a part,
and the name of the part is stored in |\childdocname|.
Note that |\jobname| will be set to the filename of the current part
so that each part receives an individual |.aux| file
that does not interfere with the |.aux| file(s) of the main document.
This behaviour can be altered by the alternative form
|\childdocby[*]{|\textit{main}|}| (with a non-empty optional argument)
which uses the |.aux| file of the main document
by setting |\jobname| to \textit{main}.

%%%%%%%%%%%%%%%%%%%%%%%%%%%%%%%%%%%%%%%%%%%%%%%%%%%%%%%%%%%%%%%%%%%%%%%%%%%%%%%%
\subsection{Driver Development}
\label{sec:driver}

The \textsf{childdoc} mechanism can also be use for the development
of definition files such as \LaTeX{} styles or classes.
This case differs from the above setup with multiple parts
included by |\include| in that no |\includeonly| should be invoked.
This can be achieved by starting the include file
(before |\ProvidesPackage|) with:
%
\begin{center}
\begin{tabular}{l}
|\input{childdoc.def}|\\
|\childdocforward{|\textit{main}|}|\\
\end{tabular}
\end{center}
%
or alternatively with:
%
\begin{center}
\begin{tabular}{l}
|\input{childdoc.def}|\\
|\childdocby{|\textit{main}|}|\\
\end{tabular}
\end{center}
%
Both forms have slightly different effects as described above.
The main file is prepared as usual, see \secref{sec:include}.

%%%%%%%%%%%%%%%%%%%%%%%%%%%%%%%%%%%%%%%%%%%%%%%%%%%%%%%%%%%%%%%%%%%%%%%%%%%%%%%%
\subsection{Legacy Detection}
\label{sec:detection}

The directive |\childdocmain| in the main file can detect
whether the complete document or merely a child is to be compiled
even without using the directive |\childdocof|.
This method is deprecated because it is less robust
and there is no compelling reason to use it;
it is merely provided for backward compatibility
and it may be removed in future versions.

If the detection mechanism is to be used,
it is mandatory to correctly specify
the filename of the main file as the argument of |\childdocmain|:
%
\begin{center}
\begin{tabular}{l}
|\input{childdoc.def}|\\
|\childdocmain{|\textit{main}|}|\\
\end{tabular}
\end{center}
%
If |\jobname| does not match the argument \textit{main} of |\childdocmain|,
it is assumed that |\jobname| points to the child file to be compiled.
When using |\childdocmain| with the main file specified as argument,
it suffices to start a child file
with just |\input{|\textit{main}|}|
without loading of the package and using |\childdocof|.
If instead all processing is done
with the appropriate \textsf{childdoc} directives,
the argument of \textit{main} of |\childdocmain| can be empty.

An alternative version of the command line processing described
in \secref{sec:commandline} using the detection mechanism reads:
%
\begin{center}
|... -jobname "|\textit{target}|" "|[\textit{flags}]%
[|\def\jobname{|\textit{dest}|}|]|\input{|\textit{main}|}"|
\end{center}

%%%%%%%%%%%%%%%%%%%%%%%%%%%%%%%%%%%%%%%%%%%%%%%%%%%%%%%%%%%%%%%%%%%%%%%%%%%%%%%%
\subsection{Manual Code}
\label{sec:manual}

In case one cannot be certain whether the definitions file |childdoc.def|
is installed on the target \TeX{} distribution
and one prefers not to ship it,
it is conceivable to paste a few relevant commands into the sources.

To that end, drop all statements |\input{childdoc.def}|
and perform the replacements as outlined below.
Instead of |\childdocmain{|\textit{main}|}| add the following code
to the top of the main file:
%
\begin{center}
\begin{tabular}{l}
|\||ifdefined\childdocname\endinput\||fi\newif\ifchilddoc|\\
|\edef\childdocname{\scantokens\expandafter{\jobname\noexpand}}|\\
|\def\childdocmain{|\textit{main}|}\||ifx\childdocmain\childdocname\||else|\\
|\childdoctrue\includeonly{\childdocname}\let\jobname\childdocmain\||fi|\\
\end{tabular}
\end{center}
%
Instead of |\childdocof{|\textit{main}|}| just include the main file
at the top of each child file:
%
\begin{center}
|\input{|\textit{main}|}|
\end{center}
%
A simple redirection |\childdocforward{|\textit{dest}|}| is achieved by:
%
\begin{center}
|\def\jobname{|\textit{dest}|}\input{\jobname}|
\end{center}
%
The redirection with prefix
|\childdocforwardprefix[|\textit{prefix}|]{|\textit{dest}|}|
is accomplished by:
%
\begin{center}
\begin{tabular}{l}
|{\edef\jobname{\scantokens\expandafter{\jobname\noexpand}}|\\
|\def\redirectjob |\textit{prefix}|#1~~~{\gdef\jobname{|\textit{dest}|#1}}|\\
|\expandafter\redirectjob\jobname~~~}\input{\jobname}|
\end{tabular}
\end{center}

In an alternative approach,
child documents can be compiled by a specific command line
without additional code or specific definitions:
%
\begin{center}
|... -jobname "|\textit{target}|" "|[\textit{flags}]%
|\includeonly{|\textit{dest}|}\input{|\textit{main}|}"|
\end{center}
%

%%%%%%%%%%%%%%%%%%%%%%%%%%%%%%%%%%%%%%%%%%%%%%%%%%%%%%%%%%%%%%%%%%%%%%%%%%%%%%%%
%%%%%%%%%%%%%%%%%%%%%%%%%%%%%%%%%%%%%%%%%%%%%%%%%%%%%%%%%%%%%%%%%%%%%%%%%%%%%%%%
\section{Information}

%%%%%%%%%%%%%%%%%%%%%%%%%%%%%%%%%%%%%%%%%%%%%%%%%%%%%%%%%%%%%%%%%%%%%%%%%%%%%%%%
\subsection{Copyright}

Copyright \copyright{} 2017--2018 Niklas Beisert

This work may be distributed and/or modified under the
conditions of the \LaTeX{} Project Public License, either version 1.3
of this license or (at your option) any later version.
The latest version of this license is in
  \url{http://www.latex-project.org/lppl.txt}
and version 1.3 or later is part of all distributions of \LaTeX{}
version 2005/12/01 or later.

This work has the LPPL maintenance status `maintained'.

The Current Maintainer of this work is Niklas Beisert.

This work consists of the files |README.txt|, |childdoc.ins| and |childdoc.dtx|
as well as the derived files |childdoc.def|, |cdocsamp.tex|
with |cdocsch1.tex|, |cdocsch2.tex|, |cdocspt3.tex|, |cdocspt4.tex|,
|cdocsdrf.tex|, |cdocsfn1.tex|, |cdocsfn2.tex|
as well as |childdoc.pdf|.

%%%%%%%%%%%%%%%%%%%%%%%%%%%%%%%%%%%%%%%%%%%%%%%%%%%%%%%%%%%%%%%%%%%%%%%%%%%%%%%%
\subsection{Files and Installation}

The package consists of the files:
%
\begin{center}
\begin{tabular}{ll}
    |README.txt|   & readme file \\
    |childdoc.ins| & installation file \\
    |childdoc.dtx| & source file \\
    |childdoc.def| & definition file \\
    |cdocsamp.tex| & sample main file \\
    |cdocsch1.tex| & sample include file \\
    |cdocsch2.tex| & sample include file \\
    |cdocspt3.tex| & sample part file \\
    |cdocspt4.tex| & sample part file \\
    |cdocsdrf.tex| & sample redirection file \\
    |cdocsfn1.tex| & sample redirection file \\
    |cdocsfn2.tex| & sample redirection file \\
    |childdoc.pdf| & manual
\end{tabular}
\end{center}
%
The distribution consists of the files
|README.txt|, |childdoc.ins| and |childdoc.dtx|.
%
\begin{itemize}
\item
Run (pdf)\LaTeX{} on |childdoc.dtx|
to compile the manual |childdoc.pdf| (this file).
\item
Run \LaTeX{} on |childdoc.ins| to create the definitions file |childdoc.def|
and the sample |cdocsamp.tex| with include files
|cdocsch1.tex|, |cdocsch2.tex|, |cdocspt3.tex|, |cdocspt4.tex|,
|cdocsdrf.tex|, |cdocsfn1.tex|, |cdocsfn2.tex|.
Then copy the file |childdoc.def| to an appropriate directory of your \LaTeX{}
distribution, e.g.\ \textit{texmf-root}|/tex/latex/childdoc|.
\end{itemize}

%%%%%%%%%%%%%%%%%%%%%%%%%%%%%%%%%%%%%%%%%%%%%%%%%%%%%%%%%%%%%%%%%%%%%%%%%%%%%%%%
\subsection{Related CTAN Packages}

There are several other packages which offer a similar functionality:
%
\begin{itemize}
\item
The packages
\href{http://ctan.org/pkg/docmute}{\textsf{docmute}},
\href{http://ctan.org/pkg/includex}{\textsf{includex}} and
\href{http://ctan.org/pkg/standalone}{\textsf{standalone}}
provide commands to include only the document body of
a child file thus allowing both files to be compiled individually.
\item
The packages \href{http://ctan.org/pkg/subdocs}{\textsf{subdocs}}
and \href{http://ctan.org/pkg/subfiles}{\textsf{subfiles}}
provide structures in which the main and child documents can be
encapsulated and allowing them to be compiled individually.
The inclusion mechanism is different from the conventional |\include|.
\item
The package \href{http://ctan.org/pkg/combine}{\textsf{combine}}
is an elaborate solution to combine several documents into one.
\end{itemize}
%
See also the CTAN topic \href{http://ctan.org/topic/subdocs}{\textsf{subdocs}}
for further related packages.
The present package differs from the above solutions in that
a document structure constructed with the conventional |\include| mechanism
just needs two extra commands at the top of every file
such that all constituent files can be compiled individually.

%%%%%%%%%%%%%%%%%%%%%%%%%%%%%%%%%%%%%%%%%%%%%%%%%%%%%%%%%%%%%%%%%%%%%%%%%%%%%%%%
%\subsection{Feature Suggestions}
%
%The following is a list of features which may be useful for future
%versions of this package:
%%
%\begin{itemize}
%\item
%\ldots
%\end{itemize}

%%%%%%%%%%%%%%%%%%%%%%%%%%%%%%%%%%%%%%%%%%%%%%%%%%%%%%%%%%%%%%%%%%%%%%%%%%%%%%%%
\subsection{Revision History}

%%%%%%%%%%%%%%%%%%%%%%%%%%%%%%%%%%%%%%%%
\paragraph{v2.0:} 2018/12/30

\begin{itemize}
\item
immediate forward processing
\item
added |\childdocby| mechanism
\item
manual restructured
\end{itemize}

%%%%%%%%%%%%%%%%%%%%%%%%%%%%%%%%%%%%%%%%
\paragraph{v1.6:} 2018/01/17

\begin{itemize}
\item
application for development of include files
\item
corrections to manual
\end{itemize}

%%%%%%%%%%%%%%%%%%%%%%%%%%%%%%%%%%%%%%%%
\paragraph{v1.5:} 2017/05/21

\begin{itemize}
\item
more complete structuring introduced
\item
|\childdocof| introduced
\item
|\childdoc| renamed to |\childdocmain|
\item
|\childredirect| renamed to |\childdocforward| and |\childdocforwardprefix|
and functionality expanded
\end{itemize}

%%%%%%%%%%%%%%%%%%%%%%%%%%%%%%%%%%%%%%%%
\paragraph{v1.0:} 2017/04/27

\begin{itemize}
\item
manual and install package
\item
first version published on CTAN
\end{itemize}

%%%%%%%%%%%%%%%%%%%%%%%%%%%%%%%%%%%%%%%%
\paragraph{v0.6:} 2017/04/26

\begin{itemize}
\item
redirection mechanism added
\end{itemize}

%%%%%%%%%%%%%%%%%%%%%%%%%%%%%%%%%%%%%%%%
\paragraph{v0.5:} 2017/04/26

\begin{itemize}
\item
functionality in definition file
\end{itemize}


%%%%%%%%%%%%%%%%%%%%%%%%%%%%%%%%%%%%%%%%%%%%%%%%%%%%%%%%%%%%%%%%%%%%%%%%%%%%%%%%
%%%%%%%%%%%%%%%%%%%%%%%%%%%%%%%%%%%%%%%%%%%%%%%%%%%%%%%%%%%%%%%%%%%%%%%%%%%%%%%%
%%%%%%%%%%%%%%%%%%%%%%%%%%%%%%%%%%%%%%%%%%%%%%%%%%%%%%%%%%%%%%%%%%%%%%%%%%%%%%%%
\appendix

\settowidth\MacroIndent{\rmfamily\scriptsize 000\ }

 \DocInput{childdoc.dtx}

\end{document}
%</driver>
% \fi
%
% %%%%%%%%%%%%%%%%%%%%%%%%%%%%%%%%%%%%%%%%%%%%%%%%%%%%%%%%%%%%%%%%%%%%%%%%%%%%%%
% %%%%%%%%%%%%%%%%%%%%%%%%%%%%%%%%%%%%%%%%%%%%%%%%%%%%%%%%%%%%%%%%%%%%%%%%%%%%%%
% \section{Sample}
%\iffalse
%<*samplemain>
%\fi
%
% The following presents a sample document
% with two chapters, two parts, a title page,
% a compile flag as well as three forwarding files to set the flag.
% It consists of eight |.tex| files:
% \begin{center}
% \begin{tabular}{ll}
% |cdocsamp.tex|&main file\\
% |cdocsch1.tex|&include file for chapter 1\\
% |cdocsch2.tex|&include file for chapter 2\\
% |cdocspt3.tex|&include file for part 3\\
% |cdocspt4.tex|&include file for part 4\\
% |cdocsdrf.tex|&forwarding file for main file in draft mode\\
% |cdocsfi1.tex|&forwarding file for final version of chapter 1\\
% |cdocsfi2.tex|&forwarding file for final version of chapter 2\\
% \end{tabular}
% \end{center}
% Each of the eight files can be compiled directly by the \LaTeX{} compiler.
%
% %%%%%%%%%%%%%%%%%%%%%%%%%%%%%%%%%%%%%%
% \paragraph{Main File.}
%
% The main file is called |cdocsamp.tex|.
%
% Load the \textsf{childdoc} definitions and
% declare the filename for the main document:
%    \begin{macrocode}
\input{childdoc.def}
\childdocmain{}
%    \end{macrocode}

% Optional override for |\version| flag:
%    \begin{macrocode}
%%\ifchilddoc\else\providecommand{\version}{draft}\fi
%    \end{macrocode}

% Define the default values for the |\version| flag
% (|final| for the main file and |draft| for childs):
%    \begin{macrocode}
\ifchilddoc
\providecommand{\version}{draft}
\else
\providecommand{\version}{final}
\fi
%    \end{macrocode}

% Load the standard document class:
%    \begin{macrocode}
\documentclass[12pt]{article}
%    \end{macrocode}

% Start the document body:
%    \begin{macrocode}
\begin{document}
%    \end{macrocode}

% Declare a title page.
% Print title, part of document being processed and version flag:
%    \begin{macrocode}
\addtocounter{page}{-1}
\begin{center}
{\LARGE\bfseries{}childdoc example\par}
\vspace{1cm}
\ifchilddoc
\ifchilddocmanual part\else chapter\fi:
`\childdocname' of `\childdocjob'\par
\else
main document: `\childdocjob'\par
\fi
version: \version\par
\end{center}
\newpage
%    \end{macrocode}

% Manually include selected file,
% otherwise process as usual:
%    \begin{macrocode}
\ifchilddocmanual
\section*{part `\childdocname'}
\input{\childdocname}
\else
%    \end{macrocode}

% Include the two chapters:
%    \begin{macrocode}
\include{cdocsch1}
\include{cdocsch2}
%    \end{macrocode}

% Include the two parts unless only chapters should be displayed:
%    \begin{macrocode}
\ifchilddoc\else
\section{part three}
\input{cdocspt3}
\section{part four}
\input{cdocspt4}
\fi
%    \end{macrocode}

% Process as usual until here:
%    \begin{macrocode}
\fi
%    \end{macrocode}

% End of document body:
%    \begin{macrocode}
\end{document}
%    \end{macrocode}
%\iffalse
%</samplemain>
%\fi
%
% %%%%%%%%%%%%%%%%%%%%%%%%%%%%%%%%%%%%%%
% \paragraph{Chapter Include Files.}
%
% The include files are called |cdocsch1.tex| and |cdocsch2.tex|.
%
%\iffalse
%<*samplechap1|samplechap2>
%\fi

% Optional override for |\version| flag:
%    \begin{macrocode}
%%\providecommand{\version}{final}
%    \end{macrocode}

% Include the main document:
%    \begin{macrocode}
\input{childdoc.def}
\childdocof{cdocsamp}
%    \end{macrocode}

%\iffalse
%</samplechap1|samplechap2>
%\fi
%
%\iffalse
%<*samplechap1>
%\fi
% Some text for chapter 1:
%    \begin{macrocode}
\section{one}
some text in chapter one
%    \end{macrocode}

%\iffalse
%</samplechap1>
%\fi
% Some text for chapter 2:
%\iffalse
%<*samplechap2>
%\fi
%    \begin{macrocode}
\section{two}
more text in chapter two
%    \end{macrocode}

%\iffalse
%</samplechap2>
%\fi
%
% %%%%%%%%%%%%%%%%%%%%%%%%%%%%%%%%%%%%%%
% \paragraph{Part Include Files.}
%
% The include files are called |cdocspt3.tex| and |cdocspt4.tex|.
%
%\iffalse
%<*samplepart3|samplepart4>
%\fi

% Optional override for |\version| flag:
%    \begin{macrocode}
%%\providecommand{\version}{final}
%    \end{macrocode}

% Include the main document:
%    \begin{macrocode}
\input{childdoc.def}
\childdocby{cdocsamp}
%    \end{macrocode}

%\iffalse
%</samplepart3|samplepart4>
%\fi
%
%\iffalse
%<*samplepart3>
%\fi
% Some text for part 3:
%    \begin{macrocode}
some text in part three
%    \end{macrocode}

%\iffalse
%</samplepart3>
%\fi
% Some text for part 4:
%\iffalse
%<*samplepart4>
%\fi
%    \begin{macrocode}
more text in part four
%    \end{macrocode}

%\iffalse
%</samplepart4>
%\fi
%
% %%%%%%%%%%%%%%%%%%%%%%%%%%%%%%%%%%%%%%
% \paragraph{Forwarding for a Complete Draft.}
%
% The following forwarding file |cdocsdrf.tex|
% compiles the main document in draft mode:
%\iffalse
%<*sampledraft>
%\fi
%    \begin{macrocode}
\def\version{draft}
\input{childdoc.def}
\childdocforward{cdocsamp}
%    \end{macrocode}

%\iffalse
%</sampledraft>
%\fi
%
% %%%%%%%%%%%%%%%%%%%%%%%%%%%%%%%%%%%%%%
% \paragraph{Forwarding for Final Version of the Chapters.}
%
% The following forwarding files |cdocsfn1.tex| and |cdocsfn2.tex|
% (with identical content)
% compile the final versions of the child documents
% |cdocsch1.tex| and |cdocsch2.tex|, respectively:
%\iffalse
%<*samplefinal>
%\fi
%    \begin{macrocode}
\def\version{final}
\input{childdoc.def}
\childdocforwardprefix[cdocsamp]{cdocsfn}{cdocsch}
%    \end{macrocode}

%\iffalse
%</samplefinal>
%\fi
%
% %%%%%%%%%%%%%%%%%%%%%%%%%%%%%%%%%%%%%%
% \paragraph{Command Line Processing.}
%
% The following three command lines generate the output files
% |cdocscld|, |cdocscl1| and |cdocscl2|
% which should be identical to
% |cdocsdrf|, |cdocsch1| and |cdocsfn2|, respectively:
% \begin{center}
% \begin{tabular}{l}
% |latex -jobname cdocscld \|\\
% |  "\def\version{draft}\input{childdoc.def}\childdocforward{cdocsamp}"|\\
% |latex -jobname cdocscl1 \|\\
% |  "\input{childdoc.def}\childdocforward[cdocsamp]{cdocsch1}"|\\
% |latex -jobname cdocscl2 \|\\
% |  "\def\version{final}\input{childdoc.def}\childdocforward{cdocsch2}"|
% \end{tabular}
% \end{center}
% Note that the trailing backslash on each first line
% merely continues the input to the second line
% (for convenient cut ant paste).
% Furthermore, the command |latex| can be replaced by any
% of its alternative versions such as |pdflatex|.
%
% %%%%%%%%%%%%%%%%%%%%%%%%%%%%%%%%%%%%%%%%%%%%%%%%%%%%%%%%%%%%%%%%%%%%%%%%%%%%%%
% %%%%%%%%%%%%%%%%%%%%%%%%%%%%%%%%%%%%%%%%%%%%%%%%%%%%%%%%%%%%%%%%%%%%%%%%%%%%%%
% \section{Implementation}
%\iffalse
%<*package>
%\fi
%
% This section describes the definitions file |childdoc.def|.

% The definitions cannot be loaded using |\usepackage| or |\RequirePackage|
% which has a mechanism to prevent loading a style file more than once.
% When loading the definitions by means of |\input|
% multiple instances have to be prevented manually:
%\iffalse
%This code needs to be before the `\ProvidesFile' directive
%which is defined at the beginning of this file.
%Therefore it is also placed there and commented out here.
%</package>
%<*discard>
%\fi
%    \begin{macrocode}
\ifdefined\childdocmain\endinput\fi
%    \end{macrocode}
%\iffalse
%</discard>
%<*package>
%\fi
%
% \macro{\ifchilddoc}
% \macro{\ifchilddocmanual}
% The conditional |\ifchilddoc| tells whether a
% child (true) or main (false) document is being compiled.
% The conditional |\ifchilddocmanual| tells whether
% the |\includeonly| mechanism is used (false) or
% the selection of child files must be performed manually (true).
% The definitions initialise to false:
%    \begin{macrocode}
\newif\ifchilddoc
\newif\ifchilddocmanual
%    \end{macrocode}

% \macro{\childdocname}
% \macro{\childdocjob}
% The macro |\childdocname| stores the name of the main document
% to be compiled. The macro |\childdocjob| stores the name of
% the document on which the \LaTeX{} compiler was originally invoked.
% The content of |\jobname| cannot be compared
% to filenames specified in the source due to different catcodes.
% The following code rescans |\jobname|, stores the result
% in |\childdocname| and saves a copy in |\childdocjob|:
%    \begin{macrocode}
\edef\childdocname{\scantokens\expandafter{\jobname\noexpand}}
\let\childdocjob\childdocname
%    \end{macrocode}

% \macro{\childdocdisable}
% The macro |\childdocdisable| prevents the main file
% from being processed more than once.
% At this stage, the main document command |\childdocmain|
% is assumed to be called once again where it should do nothing.
% Any subsequent call to it should prevent
% a secondary processing of the main document
% It overwrites the forwarding commands
% |\childdocof| and |\childdocforward|
% with empty macros to prevent further inclusions of the main document:
%    \begin{macrocode}
\newcommand{\childdocdisable}
{
  \renewcommand{\childdocmain}[1]{\renewcommand{\childdocmain}[1]{\endinput}}
  \renewcommand{\childdocof}[1]{}
  \renewcommand{\childdocby}[2][]{}
  \renewcommand{\childdocforward}[2][]{}
  \renewcommand{\childdocdisable}{}
}
%    \end{macrocode}

% \macro{\childdocmain}
% The macro |\childdocmain| is to be called at the top of the main file
% with nothing or the main filename (without extension) as argument.
% First, it breaks loops.
% If the argument is not empty and does not match |\childdocname|
% (which is set by the first inclusion of |childdoc.def|),
% |\ifchilddoc| is set to true, |\includeonly| is applied to the child file
% and |\jobname| is set to the main file
% (for proper handling of |.aux| files):
%    \begin{macrocode}
\newcommand{\childdocmain}[1]
{
  \childdocdisable\childdocmain{}
  \if?#1?\else
    \begingroup
      \def\childdoctmp{#1}
      \ifx\childdoctmp\childdocname
        \def\childdoctmp{}
      \else
        \def\childdoctmp
        {
          \childdoctrue
          \includeonly{\childdocname}
          \def\childdocjob{#1}
          \def\jobname{#1}
        }
      \fi
      \expandafter
    \endgroup
    \childdoctmp
  \fi
}
%    \end{macrocode}

% \macro{\childdocof}
% The command |\childdocof| redirects
% compilation to the main file |#1|.
%    \begin{macrocode}
\newcommand{\childdocof}[1]
{
  \childdocdisable
  \childdoctrue
  \includeonly{\childdocname}
  \def\jobname{#1}
  \def\childdocjob{#1}
  \input{#1}
}
%    \end{macrocode}

% \macro{\childdocby}
% The command |\childdocby| ....
%    \begin{macrocode}
\newcommand{\childdocby}[2][]
{
  \childdocdisable
  \childdoctrue
  \childdocmanualtrue
  \if?#1?\else
    \def\jobname{#2}
  \fi
  \def\childdocjob{#2}
  \input{#2}
  \endinput
}
%    \end{macrocode}

% \macro{\childdocforward}
% The command |\childdocforward| redirects
% compilation to the main file or
% (if the optional argument is given) a child file.
% Parameters are set as if the main file
% or a child file starting with |\childdocof| was compiled.
% Then compilation is handed over to the main file:
%    \begin{macrocode}
\newcommand{\childdocforward}[2][]
{
  \begingroup
    \if?#1?
      \def\childdoctmp
      {
        \def\childdocname{#2}
        \def\childdocjob{#2}
        \def\jobname{#2}
        \input{#2}
        \endinput
      }
    \else
      \def\childdoctmp
      {
        \childdocdisable
        \def\childdocname{#2}
        \childdoctrue
        \includeonly{#2}
        \def\childdocjob{#1}
        \def\jobname{#1}
        \input{#1}
        \endinput
      }
    \fi
    \expandafter
  \endgroup
  \childdoctmp
}
%    \end{macrocode}

% \macro{\childdocforwardprefix}
% The command |\childdocforwardprefix| redirects
% compilation to the main or a child file by means of a pattern.
% The prefix |#1| in the current filename is replaced by |#2|
% and the suffix of the current filename is kept
% (it is assumed that the filename does not contain the substring `|~~~|'
% which is used as a delimiter).
% Compilation is handed over to the new file by |\childdocforward|:
%    \begin{macrocode}
\newcommand{\childdocforwardprefix}[3][]
{
  \begingroup
    \def\childdocextract #2##1~~~{\def\childdoctmp{\childdocforward[#1]{#3##1}}}
    \expandafter\childdocextract\childdocname~~~
    \expandafter
  \endgroup
  \childdoctmp
}
%    \end{macrocode}

% \macro{\childdoc}
% The deprecated macro |\childdoc| is a legacy version of |\childdocmain|:
%    \begin{macrocode}
\newcommand{\childdoc}{\childdocmain}
%    \end{macrocode}

% \macro{\childdocredirect}
% The deprecated macro |\childdocredirect| is a legacy version
% of |\childdocforward| and |\childdocforwardprefix|:
%    \begin{macrocode}
\newcommand{\childdocredirect}[2][]
{
  \begingroup
    \if?#1?
      \def\childdoctmp{\childdocforward{#2}}
    \else
      \def\childdoctmp{\childdocforwardprefix{#1}{#2}}
    \fi
    \expandafter
  \endgroup
  \childdoctmp
}
%    \end{macrocode}

%\iffalse
%</package>
%\fi
%
\endinput
|\\
|\childdocforwardprefix{final}{child}|
\end{tabular}
\end{center}
%

Note that when several versions of a main file and/or of each child file
are to be generated, it may be convenient to set up a |Makefile| or
shell script to automatise the process.

%%%%%%%%%%%%%%%%%%%%%%%%%%%%%%%%%%%%%%%%%%%%%%%%%%%%%%%%%%%%%%%%%%%%%%%%%%%%%%%%
\subsection{Command Line Processing}
\label{sec:commandline}

The effect of redirection files can also be achieved by invoking
the \LaTeX{} compiler with a more elaborate command line.
Most conveniently this should be done as part
of a shell script or a |Makefile|.

When using \textsf{childdoc} in the main file, the following
command lines effectively perform a redirection
(note that depending on the shell being used,
backslashes may have to be doubled: `|\|' $\to$ `|\\|'):
%
\begin{center}
|... -jobname "|\textit{target}|" |\\|"|[\textit{flags}]%
|% \iffalse
%
% childdoc.dtx Copyright (C) 2017-2018 Niklas Beisert
%
% This work may be distributed and/or modified under the
% conditions of the LaTeX Project Public License, either version 1.3
% of this license or (at your option) any later version.
% The latest version of this license is in
%   http://www.latex-project.org/lppl.txt
% and version 1.3 or later is part of all distributions of LaTeX
% version 2005/12/01 or later.
%
% This work has the LPPL maintenance status `maintained'.
%
% The Current Maintainer of this work is Niklas Beisert.
%
% This work consists of the files childdoc.dtx and childdoc.ins
% and the derived files childdoc.def and cdocsamp.tex with
% cdocsch1.tex, cdocsch2.tex, cdocsdrf.tex, cdocsfn1.tex, cdocsfn2.tex.
%
%<package>\ifdefined\childdocmain\endinput\fi
%<package>\ProvidesFile{childdoc.def}[2018/12/30 v2.0 child document driver]
%<samplemain>\ProvidesFile{cdocsamp.tex}[2018/12/30 v2.0 sample for childdoc]
%<*driver>
%\ProvidesFile{childdoc.drv}[2018/12/30 v2.0 childdoc reference manual file]
\PassOptionsToClass{10pt,a4paper}{article}
\documentclass{ltxdoc}

\usepackage[margin=35mm]{geometry}
\usepackage{hyperref}
\usepackage{hyperxmp}
\usepackage[usenames]{color}

\hypersetup{colorlinks=true}
\hypersetup{pdfstartview=FitH}
\hypersetup{pdfpagemode=UseNone}
\hypersetup{pdfsource={}}
\hypersetup{pdflang={en-UK}}
\hypersetup{pdfcopyright={Copyright 2017-2018 Niklas Beisert.
  This work may be distributed and/or modified under the
  conditions of the LaTeX Project Public License, either version 1.3
  of this license or (at your option) any later version.}}
\hypersetup{pdflicenseurl={http://www.latex-project.org/lppl.txt}}
\hypersetup{pdfcontactaddress={ETH Zurich, ITP, HIT K,
  Wolfgang-Pauli-Strasse 27}}
\hypersetup{pdfcontactpostcode={8093}}
\hypersetup{pdfcontactcity={Zurich}}
\hypersetup{pdfcontactcountry={Switzerland}}
\hypersetup{pdfcontactemail={nbeisert@itp.phys.ethz.ch}}
\hypersetup{pdfcontacturl={http://people.phys.ethz.ch/\xmptilde nbeisert/}}

\newcommand{\secref}[1]{\hyperref[#1]{section \ref*{#1}}}

\parskip1ex
\parindent0pt
\let\olditemize\itemize
\def\itemize{\olditemize\parskip0pt}

\begin{document}

\title{The \textsf{childdoc} Package}
\hypersetup{pdftitle={The childdoc Package}}
\author{Niklas Beisert\\[2ex]
  Institut f\"ur Theoretische Physik\\
  Eidgen\"ossische Technische Hochschule Z\"urich\\
  Wolfgang-Pauli-Strasse 27, 8093 Z\"urich, Switzerland\\[1ex]
  \href{mailto:nbeisert@itp.phys.ethz.ch}
  {\texttt{nbeisert@itp.phys.ethz.ch}}}
\hypersetup{pdfauthor={Niklas Beisert}}
\hypersetup{pdfsubject={Manual for the LaTeX2e Package childdoc}}
\date{30 December 2018, \textsf{v2.0}}
\maketitle

\begin{abstract}\noindent
\textsf{childdoc} is a \LaTeXe{} package
that enables the direct compilation
of document sections included by |\include|
to individual files.
\end{abstract}

\begingroup
\parskip0ex
\tableofcontents
\endgroup

%%%%%%%%%%%%%%%%%%%%%%%%%%%%%%%%%%%%%%%%%%%%%%%%%%%%%%%%%%%%%%%%%%%%%%%%%%%%%%%%
%%%%%%%%%%%%%%%%%%%%%%%%%%%%%%%%%%%%%%%%%%%%%%%%%%%%%%%%%%%%%%%%%%%%%%%%%%%%%%%%
\section{Introduction}

\LaTeX{} provides a mechanism to structure a large document (such as a book)
into a main file and several child files (containing the chapters)
using the |\include| command.
This mechanism is beneficial for documents
which span hundreds of pages in order to
make the source file(s) more manageable.
Moreover, compilation can be restricted to
selected child files by means of the |\includeonly| command.
The latter feature can be used to reduce the compilation time while editing
(this was significantly more useful in the earlier days of \LaTeX{})
or to generate a smaller document which is easier to navigate.
Another application of |\includeonly| is to generate
documents consisting of selected parts of the complete document.

However, there are a few drawbacks of the plain |\include| mechanism:
\begin{itemize}
\item
The child files cannot be compiled on their own,
they can only be compiled via the main file.
A naive editing environment
(such as a text editor with an option
to have the current file processed by \LaTeX)
may require one to switch to the main file before compiling;
attempting to compile the child file produces errors.
\item
The main file must be modified (each time)
to adjust the |\includeonly| command
to the present needs. This easily leaves the main file in a messy state.
\item
The generated document will always carry the filename
of the main document. This is inconvenient if
several child files are to be compiled and
to be kept for distribution.
\end{itemize}

The present package provides a simple interface
to make child files individually compilable by \LaTeX{}.
Compiling a child file then has the same effect as compiling
the main file with an |\includeonly| command
to select the appropriate child.
Moreover the generated document will carry the name of the child
rather than the main file.
This resolves all three above issues.

This feature is meant to make the editing of books,
thesis documents and lecture notes somewhat more convenient.
However, the package can also be used efficiently for
composing a series of documents (such as exercise sheets)
which are typically distributed individually.
It then assists the author in generating the individual documents
(potentially in different versions)
as well as a document containing the collected series.
Another application is in developing style files
or other kinds of included material
where compilation of the style file could redirect
to a sample or test file.

%%%%%%%%%%%%%%%%%%%%%%%%%%%%%%%%%%%%%%%%%%%%%%%%%%%%%%%%%%%%%%%%%%%%%%%%%%%%%%%%
%%%%%%%%%%%%%%%%%%%%%%%%%%%%%%%%%%%%%%%%%%%%%%%%%%%%%%%%%%%%%%%%%%%%%%%%%%%%%%%%
\section{Usage}

First of all, the package \textsf{childdoc} is \emph{not} a standard
\LaTeXe{} |.sty| style file! Therefore it needs to be invoked in
a non-standard way.

%%%%%%%%%%%%%%%%%%%%%%%%%%%%%%%%%%%%%%%%%%%%%%%%%%%%%%%%%%%%%%%%%%%%%%%%%%%%%%%%
\subsection{Included Files}
\label{sec:include}

%%%%%%%%%%%%%%%%%%%%%%%%%%%%%%%%%%%%%%%%
\DescribeMacro{\childdocmain}
To use the package, add the commands
\begin{center}
\begin{tabular}{l}
|\input{childdoc.def}|\\
|\childdocmain{}|\\
\end{tabular}
\end{center}
at the very top of the main \LaTeX{} file,
in particular \emph{before} the |\documentclass| statement!
The argument of |\childdocmain| should be left empty
(but it must be present).

%%%%%%%%%%%%%%%%%%%%%%%%%%%%%%%%%%%%%%%%
\DescribeMacro{\childdocof}
Furthermore, add the commands
\begin{center}
\begin{tabular}{l}
|\input{childdoc.def}|\\
|\childdocof{|\textit{main}|}|\\
\end{tabular}
\end{center}
at the top of every child file \textit{child}
which is included by |\include{|\textit{child}|}|
from within the main file
(or at least for those files to be compiled individually).
The argument \textit{main} must be the filename of the main file.

There are a couple of
considerations in setting up the main and child documents:

%%%%%%%%%%%%%%%%%%%%%%%%%%%%%%%%%%%%%%%%
\paragraph{Restrictions.}

Please note the following restrictions:
\begin{itemize}
\item
|\childdocmain| must be called with one argument \textit{main}
to ensure compatibility with earlier version of the package.
It must either be empty (|\childdocmain{}|)
or precisely match the filename of the main file in which it is specified.
See \secref{sec:detection} for further information.
\item
The filename \textit{main} must be specified without the |.tex| extension.
\item
The filename \textit{main} is case sensitive
(even in case-insensitive file systems)
due to internal string comparison.
\item
The argument \textit{main} should be fully expanded, it cannot be a macro.
\item
Subdirectories and special characters should be avoided in filenames.
\item
The command |\childdocmain{|\textit{main}|}| must be followed by a whitespace.
It should not be followed immediately by another command
or by a comment mark `|%|'.
This is because the \TeX{} parser reads the token immediately following
the argument of |\childdocmain| and puts it
at the beginning of every child section;
however, a white\-space is ignored.
\end{itemize}

%%%%%%%%%%%%%%%%%%%%%%%%%%%%%%%%%%%%%%%%
\paragraph{Content of Main File.}

It is advisable to place all content in the child files included by |\include|.
Any output contained in the main file will appear in all child documents
unless suppressed manually;
it cannot be suppressed automatically by the |\includeonly| directive
and thus should normally be avoided.
A method to include some content in the main file
by means of conditional processing is described in \secref{sec:conditional}.

%%%%%%%%%%%%%%%%%%%%%%%%%%%%%%%%%%%%%%%%
\paragraph{Page Numbering.}

When only a part of the document is compiled,
the appropriate numbering of pages
(as well as other status parameters)
is determined from the |.aux| files.
The latter contain information from previous passes.
However this information needs to propagate through
all intermediate child documents.
Therefore the page numbering in child documents may well
be inconsistent until the complete document is compiled at least once.

A useful (if unconventional) way to always ensure a consistent
page numbering is to restart the numbering in each child document
and denote the pages by `\textit{child}|.|\textit{page}'
where \textit{child} represents the chapter/section number of the child file.
This can be achieved by the command
|\numberwithin{page}{|\textit{child}|}|
of the \textsf{amsmath} package
where \textit{child} can be |chapter| or |section|
depending on the chosen structuring.
Alternatively, one can modify the macro |\thepage| appropriately
and reset the counter |page| at the start of each child file.

%%%%%%%%%%%%%%%%%%%%%%%%%%%%%%%%%%%%%%%%%%%%%%%%%%%%%%%%%%%%%%%%%%%%%%%%%%%%%%%%
\subsection{Conditional Processing}
\label{sec:conditional}

The package provides a mechanism to compile different versions
of a document. To customise the versions further some conditional processing
can come in handy to distinguish which version is being compiled.
The package provides two macros to describe the compilation context:

%%%%%%%%%%%%%%%%%%%%%%%%%%%%%%%%%%%%%%%%
\DescribeMacro{\ifchilddoc}
The conditional |\ifchilddoc| distinguishes between the compilation of
child documents and the main document:
%
\begin{center}
|\ifchilddoc |\textit{child-code}| |[|\||else |\textit{main-code}]| \||fi|
\end{center}

%%%%%%%%%%%%%%%%%%%%%%%%%%%%%%%%%%%%%%%%
\DescribeMacro{\childdocname}
\DescribeMacro{\childdocjob}
The macro |\childdocname| contains the filename (without extension)
of the main or child file being processed.
Note that |\childdocjob| will always contain the name of the main file.

%%%%%%%%%%%%%%%%%%%%%%%%%%%%%%%%%%%%%%%%
\paragraph{Title Page.}

Conditional processing can be used to include a title or banner page
in the main document when proper precautions are taken.
Importantly, the code in the main file should ensure that the page counter
(as well as other status parameters which are stored in the |.aux| files)
takes the same value after the conditional processing.
Otherwise the page numbers may take divergent values
depending on which part is compiled.

For example, a title page could be declared by:
%
\begin{center}
\begin{tabular}{l}
|\ifchilddoc\||else|\\
|\addtocounter{page}{-1}|\\
\textit{code for title page}\\
|\newpage|\\
|\||fi|
\end{tabular}
\end{center}
%
A banner page for the child documents can be generated by:
%
\begin{center}
\begin{tabular}{l}
|\ifchilddoc|\\
|\addtocounter{page}{-1}|\\
\textit{code for banner page}\\
|\newpage|\\
|\||fi|
\end{tabular}
\end{center}
%
Here one could write a message such as:
\begin{center}
|This is the part \childdocname{} of \childdocjob{}.|
\end{center}

%%%%%%%%%%%%%%%%%%%%%%%%%%%%%%%%%%%%%%%%%%%%%%%%%%%%%%%%%%%%%%%%%%%%%%%%%%%%%%%%
\subsection{Flags}
\label{sec:flags}

The package makes it easy to generate different versions
of the main or child documents.
To this end compilation flags can be defined
and assigned different default values.
They will be particularly useful in conjunction
with the forwarding mechanism described in \secref{sec:forward}.

For example, it may be useful to have a flag |\version|
which can be set to |draft| or |final|.
The document source will contain some conditional code
depending on the value of |\version|.
Suppose further, the flag should default to |final| for the main file
and to |draft| for child files
which is a natural assignment for editing the document.
This is achieved by placing the following code
in the preamble of the main document
(below the |\childdocmain| directive):
%
\begin{center}
\begin{tabular}{l}
|\ifchilddoc|\\
|\providecommand{\version}{draft}|\\
|\||else|\\
|\providecommand{\version}{final}|\\
|\||fi|
\end{tabular}
\end{center}
%
The definition by |\providecommand| makes sure
that previous definitions are not overwritten.
Further statements |\providecommand{\version}{...}|
can thus be added before the above code to override it.

For the main file, one might add a line
(between |\childdocmain| and the above block)
%
\begin{center}
|%\ifchilddoc\||else\providecommand{\version}{draft}\||fi|
\end{center}
%
which can be uncommented to produce a draft version.
Likewise one can add a line to the very top of a child file
(above the |\childdocof{|\textit{main}|}| directive)
%
\begin{center}
|%\providecommand{\version}{final}|
\end{center}
%
which can be uncommented to produce the final version of this child document.

%%%%%%%%%%%%%%%%%%%%%%%%%%%%%%%%%%%%%%%%%%%%%%%%%%%%%%%%%%%%%%%%%%%%%%%%%%%%%%%%
\subsection{Forwarding}
\label{sec:forward}

Different versions of the main or child documents
using compilation flags as described in \secref{sec:flags}
can be (permanently) stored in different files
for convenient compilation, viewing and distribution.
To this end, the package defines a command
to pass on compilation to a different file:

%%%%%%%%%%%%%%%%%%%%%%%%%%%%%%%%%%%%%%%%
\DescribeMacro{\childdocforward}
The command |\childdocforward| redirects processing to
another source file:
%
\begin{center}
\begin{tabular}{l}
|\input{childdoc.def}|\\
|\childdocforward[|\textit{main}|]{|\textit{dest}|}|\\
\end{tabular}
\end{center}
%
The argument \textit{dest} is the destination file
(without extension).
It should be the main file or one of the child files.
Note that further \textsf{childdoc} directives
such as |\childdocof| and |\childdocforward|
in the indicated file will be processed in this form.
The optional argument \textit{main}
passes on directly to the main file \textit{main}
while pretending to compile the child \textit{dest}.
This form behaves as if \textit{dest}
issues |\childdocof{|\textit{main}|}| right away,
and no further \textsf{childdoc} directives will be processed.

%%%%%%%%%%%%%%%%%%%%%%%%%%%%%%%%%%%%%%%%
\DescribeMacro{\...prefix}
In the alternative form |\childdocforwardprefix|,
%
\begin{center}
\begin{tabular}{l}
|\input{childdoc.def}|\\
|\childdocforwardprefix[|\textit{main}|]{|\textit{prefix}|}{|\textit{dest}|}|
\end{tabular}
\end{center}
%
the destination file is determined by a pattern
depending on the current file:
To make this work, the current file must be called
`{\textit{prefix}\hspace{0.2em}\textit{suffix}}'
with \textit{prefix} matching precisely the argument.
Processing is then passed on to the file
`{\textit{dest}\hspace{0.2em}\textit{suffix}}'.
Surely, the same effect is achieved by
directly specifying the
argument `{\textit{dest}\hspace{0.2em}\textit{suffix}}'
in the first form.
However, that requires to set up a different file
for each child. With the alternative form of the command
all these files can have exactly the same content
which simplifies setting them up and maintaining them.

For example, the following file |draft.tex|
with a compilation flag |\version| as described in \secref{sec:flags}
compiles the main document as a draft:
%
\begin{center}
\begin{tabular}{l}
|\def\version{draft}|\\
|\input{childdoc.def}|\\
|\childdocforward{|\textit{main}|}|
\end{tabular}
\end{center}
%
Likewise, the following files |final|\textit{nn}|.tex|
compile the final version of the child document
|child|\textit{nn}|.tex|:
%
\begin{center}
\begin{tabular}{l}
|\def\version{final}|\\
|\input{childdoc.def}|\\
|\childdocforwardprefix{final}{child}|
\end{tabular}
\end{center}
%

Note that when several versions of a main file and/or of each child file
are to be generated, it may be convenient to set up a |Makefile| or
shell script to automatise the process.

%%%%%%%%%%%%%%%%%%%%%%%%%%%%%%%%%%%%%%%%%%%%%%%%%%%%%%%%%%%%%%%%%%%%%%%%%%%%%%%%
\subsection{Command Line Processing}
\label{sec:commandline}

The effect of redirection files can also be achieved by invoking
the \LaTeX{} compiler with a more elaborate command line.
Most conveniently this should be done as part
of a shell script or a |Makefile|.

When using \textsf{childdoc} in the main file, the following
command lines effectively perform a redirection
(note that depending on the shell being used,
backslashes may have to be doubled: `|\|' $\to$ `|\\|'):
%
\begin{center}
|... -jobname "|\textit{target}|" |\\|"|[\textit{flags}]%
|\input{childdoc.def}\childdocforward[|\textit{main}|]{|\textit{dest}|}"|
\end{center}
%
Here \textit{target} is the name of the output file,
\textit{main} is the name of the main file
and \textit{dest} is the name of the main or child file to be processed
(all filenames without extensions).
The optional argument \textit{main} can be omitted
if \textit{main} matches \textit{dest}.
Optionally, compilation \textit{flags} can be defined via |\def| commands.
This command line makes the \TeX{} engine believe
it is compiling the file \textit{target}
whose content is specified as the latter parameter.
The provided code then forwards the processing to
\textit{main} or \textit{dest} as described in \secref{sec:forward}.

%%%%%%%%%%%%%%%%%%%%%%%%%%%%%%%%%%%%%%%%%%%%%%%%%%%%%%%%%%%%%%%%%%%%%%%%%%%%%%%%
\subsection{Include by Input}
\label{sec:input}

Including child documents by |\include| has some restrictions by design.
Most notably, the content of a child document always occupies
its own set of pages; pages cannot be shared between child documents.
Usually, this behaviour makes perfect sense
because each child document contain an essential part of the document.
However, in some situations it may be desirable to compose
a document from a collection of parts
without having mandatory page breaks between then.
For this case, the package
provides a mechanism to include parts
by |\input| which can also be processed individually.
However, by construction this mechanism
requires manual handling of the content to be output.

%%%%%%%%%%%%%%%%%%%%%%%%%%%%%%%%%%%%%%%%
\DescribeMacro{\ifchilddocmanual}
The main file should be prepared as usual, see \secref{sec:include}.
However, the document body must make a distinction
between processing of an individual part and of the main document, e.g.:
%
\begin{center}
\begin{tabular}{l}
|\ifchilddocmanual|\\
|\input{\childdocname}|\\
|\||else|\\
\textit{document body with }|\input{|\textit{part}|}|\\
|\||fi|
\end{tabular}
\end{center}
%
The conditional |\ifchilddocmanual| is true whenever
a part to be included by |\input| is being compiled,
and the name of the part is stored in |\childdocname|.

%%%%%%%%%%%%%%%%%%%%%%%%%%%%%%%%%%%%%%%%
\DescribeMacro{\childdocby}
Each part to be included by |\input| should start with:
%
\begin{center}
\begin{tabular}{l}
|\input{childdoc.def}|\\
|\childdocby{|\textit{main}|}|\\
\end{tabular}
\end{center}
%
The directive |\childdocby| is similar to |\childdocof|
described in \secref{sec:include},
but the subsequent selection of content must be done manually.
To that end, both |\ifchilddoc| and |\ifchilddocmanual|
will be true upon processing of a part,
and the name of the part is stored in |\childdocname|.
Note that |\jobname| will be set to the filename of the current part
so that each part receives an individual |.aux| file
that does not interfere with the |.aux| file(s) of the main document.
This behaviour can be altered by the alternative form
|\childdocby[*]{|\textit{main}|}| (with a non-empty optional argument)
which uses the |.aux| file of the main document
by setting |\jobname| to \textit{main}.

%%%%%%%%%%%%%%%%%%%%%%%%%%%%%%%%%%%%%%%%%%%%%%%%%%%%%%%%%%%%%%%%%%%%%%%%%%%%%%%%
\subsection{Driver Development}
\label{sec:driver}

The \textsf{childdoc} mechanism can also be use for the development
of definition files such as \LaTeX{} styles or classes.
This case differs from the above setup with multiple parts
included by |\include| in that no |\includeonly| should be invoked.
This can be achieved by starting the include file
(before |\ProvidesPackage|) with:
%
\begin{center}
\begin{tabular}{l}
|\input{childdoc.def}|\\
|\childdocforward{|\textit{main}|}|\\
\end{tabular}
\end{center}
%
or alternatively with:
%
\begin{center}
\begin{tabular}{l}
|\input{childdoc.def}|\\
|\childdocby{|\textit{main}|}|\\
\end{tabular}
\end{center}
%
Both forms have slightly different effects as described above.
The main file is prepared as usual, see \secref{sec:include}.

%%%%%%%%%%%%%%%%%%%%%%%%%%%%%%%%%%%%%%%%%%%%%%%%%%%%%%%%%%%%%%%%%%%%%%%%%%%%%%%%
\subsection{Legacy Detection}
\label{sec:detection}

The directive |\childdocmain| in the main file can detect
whether the complete document or merely a child is to be compiled
even without using the directive |\childdocof|.
This method is deprecated because it is less robust
and there is no compelling reason to use it;
it is merely provided for backward compatibility
and it may be removed in future versions.

If the detection mechanism is to be used,
it is mandatory to correctly specify
the filename of the main file as the argument of |\childdocmain|:
%
\begin{center}
\begin{tabular}{l}
|\input{childdoc.def}|\\
|\childdocmain{|\textit{main}|}|\\
\end{tabular}
\end{center}
%
If |\jobname| does not match the argument \textit{main} of |\childdocmain|,
it is assumed that |\jobname| points to the child file to be compiled.
When using |\childdocmain| with the main file specified as argument,
it suffices to start a child file
with just |\input{|\textit{main}|}|
without loading of the package and using |\childdocof|.
If instead all processing is done
with the appropriate \textsf{childdoc} directives,
the argument of \textit{main} of |\childdocmain| can be empty.

An alternative version of the command line processing described
in \secref{sec:commandline} using the detection mechanism reads:
%
\begin{center}
|... -jobname "|\textit{target}|" "|[\textit{flags}]%
[|\def\jobname{|\textit{dest}|}|]|\input{|\textit{main}|}"|
\end{center}

%%%%%%%%%%%%%%%%%%%%%%%%%%%%%%%%%%%%%%%%%%%%%%%%%%%%%%%%%%%%%%%%%%%%%%%%%%%%%%%%
\subsection{Manual Code}
\label{sec:manual}

In case one cannot be certain whether the definitions file |childdoc.def|
is installed on the target \TeX{} distribution
and one prefers not to ship it,
it is conceivable to paste a few relevant commands into the sources.

To that end, drop all statements |\input{childdoc.def}|
and perform the replacements as outlined below.
Instead of |\childdocmain{|\textit{main}|}| add the following code
to the top of the main file:
%
\begin{center}
\begin{tabular}{l}
|\||ifdefined\childdocname\endinput\||fi\newif\ifchilddoc|\\
|\edef\childdocname{\scantokens\expandafter{\jobname\noexpand}}|\\
|\def\childdocmain{|\textit{main}|}\||ifx\childdocmain\childdocname\||else|\\
|\childdoctrue\includeonly{\childdocname}\let\jobname\childdocmain\||fi|\\
\end{tabular}
\end{center}
%
Instead of |\childdocof{|\textit{main}|}| just include the main file
at the top of each child file:
%
\begin{center}
|\input{|\textit{main}|}|
\end{center}
%
A simple redirection |\childdocforward{|\textit{dest}|}| is achieved by:
%
\begin{center}
|\def\jobname{|\textit{dest}|}\input{\jobname}|
\end{center}
%
The redirection with prefix
|\childdocforwardprefix[|\textit{prefix}|]{|\textit{dest}|}|
is accomplished by:
%
\begin{center}
\begin{tabular}{l}
|{\edef\jobname{\scantokens\expandafter{\jobname\noexpand}}|\\
|\def\redirectjob |\textit{prefix}|#1~~~{\gdef\jobname{|\textit{dest}|#1}}|\\
|\expandafter\redirectjob\jobname~~~}\input{\jobname}|
\end{tabular}
\end{center}

In an alternative approach,
child documents can be compiled by a specific command line
without additional code or specific definitions:
%
\begin{center}
|... -jobname "|\textit{target}|" "|[\textit{flags}]%
|\includeonly{|\textit{dest}|}\input{|\textit{main}|}"|
\end{center}
%

%%%%%%%%%%%%%%%%%%%%%%%%%%%%%%%%%%%%%%%%%%%%%%%%%%%%%%%%%%%%%%%%%%%%%%%%%%%%%%%%
%%%%%%%%%%%%%%%%%%%%%%%%%%%%%%%%%%%%%%%%%%%%%%%%%%%%%%%%%%%%%%%%%%%%%%%%%%%%%%%%
\section{Information}

%%%%%%%%%%%%%%%%%%%%%%%%%%%%%%%%%%%%%%%%%%%%%%%%%%%%%%%%%%%%%%%%%%%%%%%%%%%%%%%%
\subsection{Copyright}

Copyright \copyright{} 2017--2018 Niklas Beisert

This work may be distributed and/or modified under the
conditions of the \LaTeX{} Project Public License, either version 1.3
of this license or (at your option) any later version.
The latest version of this license is in
  \url{http://www.latex-project.org/lppl.txt}
and version 1.3 or later is part of all distributions of \LaTeX{}
version 2005/12/01 or later.

This work has the LPPL maintenance status `maintained'.

The Current Maintainer of this work is Niklas Beisert.

This work consists of the files |README.txt|, |childdoc.ins| and |childdoc.dtx|
as well as the derived files |childdoc.def|, |cdocsamp.tex|
with |cdocsch1.tex|, |cdocsch2.tex|, |cdocspt3.tex|, |cdocspt4.tex|,
|cdocsdrf.tex|, |cdocsfn1.tex|, |cdocsfn2.tex|
as well as |childdoc.pdf|.

%%%%%%%%%%%%%%%%%%%%%%%%%%%%%%%%%%%%%%%%%%%%%%%%%%%%%%%%%%%%%%%%%%%%%%%%%%%%%%%%
\subsection{Files and Installation}

The package consists of the files:
%
\begin{center}
\begin{tabular}{ll}
    |README.txt|   & readme file \\
    |childdoc.ins| & installation file \\
    |childdoc.dtx| & source file \\
    |childdoc.def| & definition file \\
    |cdocsamp.tex| & sample main file \\
    |cdocsch1.tex| & sample include file \\
    |cdocsch2.tex| & sample include file \\
    |cdocspt3.tex| & sample part file \\
    |cdocspt4.tex| & sample part file \\
    |cdocsdrf.tex| & sample redirection file \\
    |cdocsfn1.tex| & sample redirection file \\
    |cdocsfn2.tex| & sample redirection file \\
    |childdoc.pdf| & manual
\end{tabular}
\end{center}
%
The distribution consists of the files
|README.txt|, |childdoc.ins| and |childdoc.dtx|.
%
\begin{itemize}
\item
Run (pdf)\LaTeX{} on |childdoc.dtx|
to compile the manual |childdoc.pdf| (this file).
\item
Run \LaTeX{} on |childdoc.ins| to create the definitions file |childdoc.def|
and the sample |cdocsamp.tex| with include files
|cdocsch1.tex|, |cdocsch2.tex|, |cdocspt3.tex|, |cdocspt4.tex|,
|cdocsdrf.tex|, |cdocsfn1.tex|, |cdocsfn2.tex|.
Then copy the file |childdoc.def| to an appropriate directory of your \LaTeX{}
distribution, e.g.\ \textit{texmf-root}|/tex/latex/childdoc|.
\end{itemize}

%%%%%%%%%%%%%%%%%%%%%%%%%%%%%%%%%%%%%%%%%%%%%%%%%%%%%%%%%%%%%%%%%%%%%%%%%%%%%%%%
\subsection{Related CTAN Packages}

There are several other packages which offer a similar functionality:
%
\begin{itemize}
\item
The packages
\href{http://ctan.org/pkg/docmute}{\textsf{docmute}},
\href{http://ctan.org/pkg/includex}{\textsf{includex}} and
\href{http://ctan.org/pkg/standalone}{\textsf{standalone}}
provide commands to include only the document body of
a child file thus allowing both files to be compiled individually.
\item
The packages \href{http://ctan.org/pkg/subdocs}{\textsf{subdocs}}
and \href{http://ctan.org/pkg/subfiles}{\textsf{subfiles}}
provide structures in which the main and child documents can be
encapsulated and allowing them to be compiled individually.
The inclusion mechanism is different from the conventional |\include|.
\item
The package \href{http://ctan.org/pkg/combine}{\textsf{combine}}
is an elaborate solution to combine several documents into one.
\end{itemize}
%
See also the CTAN topic \href{http://ctan.org/topic/subdocs}{\textsf{subdocs}}
for further related packages.
The present package differs from the above solutions in that
a document structure constructed with the conventional |\include| mechanism
just needs two extra commands at the top of every file
such that all constituent files can be compiled individually.

%%%%%%%%%%%%%%%%%%%%%%%%%%%%%%%%%%%%%%%%%%%%%%%%%%%%%%%%%%%%%%%%%%%%%%%%%%%%%%%%
%\subsection{Feature Suggestions}
%
%The following is a list of features which may be useful for future
%versions of this package:
%%
%\begin{itemize}
%\item
%\ldots
%\end{itemize}

%%%%%%%%%%%%%%%%%%%%%%%%%%%%%%%%%%%%%%%%%%%%%%%%%%%%%%%%%%%%%%%%%%%%%%%%%%%%%%%%
\subsection{Revision History}

%%%%%%%%%%%%%%%%%%%%%%%%%%%%%%%%%%%%%%%%
\paragraph{v2.0:} 2018/12/30

\begin{itemize}
\item
immediate forward processing
\item
added |\childdocby| mechanism
\item
manual restructured
\end{itemize}

%%%%%%%%%%%%%%%%%%%%%%%%%%%%%%%%%%%%%%%%
\paragraph{v1.6:} 2018/01/17

\begin{itemize}
\item
application for development of include files
\item
corrections to manual
\end{itemize}

%%%%%%%%%%%%%%%%%%%%%%%%%%%%%%%%%%%%%%%%
\paragraph{v1.5:} 2017/05/21

\begin{itemize}
\item
more complete structuring introduced
\item
|\childdocof| introduced
\item
|\childdoc| renamed to |\childdocmain|
\item
|\childredirect| renamed to |\childdocforward| and |\childdocforwardprefix|
and functionality expanded
\end{itemize}

%%%%%%%%%%%%%%%%%%%%%%%%%%%%%%%%%%%%%%%%
\paragraph{v1.0:} 2017/04/27

\begin{itemize}
\item
manual and install package
\item
first version published on CTAN
\end{itemize}

%%%%%%%%%%%%%%%%%%%%%%%%%%%%%%%%%%%%%%%%
\paragraph{v0.6:} 2017/04/26

\begin{itemize}
\item
redirection mechanism added
\end{itemize}

%%%%%%%%%%%%%%%%%%%%%%%%%%%%%%%%%%%%%%%%
\paragraph{v0.5:} 2017/04/26

\begin{itemize}
\item
functionality in definition file
\end{itemize}


%%%%%%%%%%%%%%%%%%%%%%%%%%%%%%%%%%%%%%%%%%%%%%%%%%%%%%%%%%%%%%%%%%%%%%%%%%%%%%%%
%%%%%%%%%%%%%%%%%%%%%%%%%%%%%%%%%%%%%%%%%%%%%%%%%%%%%%%%%%%%%%%%%%%%%%%%%%%%%%%%
%%%%%%%%%%%%%%%%%%%%%%%%%%%%%%%%%%%%%%%%%%%%%%%%%%%%%%%%%%%%%%%%%%%%%%%%%%%%%%%%
\appendix

\settowidth\MacroIndent{\rmfamily\scriptsize 000\ }

 \DocInput{childdoc.dtx}

\end{document}
%</driver>
% \fi
%
% %%%%%%%%%%%%%%%%%%%%%%%%%%%%%%%%%%%%%%%%%%%%%%%%%%%%%%%%%%%%%%%%%%%%%%%%%%%%%%
% %%%%%%%%%%%%%%%%%%%%%%%%%%%%%%%%%%%%%%%%%%%%%%%%%%%%%%%%%%%%%%%%%%%%%%%%%%%%%%
% \section{Sample}
%\iffalse
%<*samplemain>
%\fi
%
% The following presents a sample document
% with two chapters, two parts, a title page,
% a compile flag as well as three forwarding files to set the flag.
% It consists of eight |.tex| files:
% \begin{center}
% \begin{tabular}{ll}
% |cdocsamp.tex|&main file\\
% |cdocsch1.tex|&include file for chapter 1\\
% |cdocsch2.tex|&include file for chapter 2\\
% |cdocspt3.tex|&include file for part 3\\
% |cdocspt4.tex|&include file for part 4\\
% |cdocsdrf.tex|&forwarding file for main file in draft mode\\
% |cdocsfi1.tex|&forwarding file for final version of chapter 1\\
% |cdocsfi2.tex|&forwarding file for final version of chapter 2\\
% \end{tabular}
% \end{center}
% Each of the eight files can be compiled directly by the \LaTeX{} compiler.
%
% %%%%%%%%%%%%%%%%%%%%%%%%%%%%%%%%%%%%%%
% \paragraph{Main File.}
%
% The main file is called |cdocsamp.tex|.
%
% Load the \textsf{childdoc} definitions and
% declare the filename for the main document:
%    \begin{macrocode}
\input{childdoc.def}
\childdocmain{}
%    \end{macrocode}

% Optional override for |\version| flag:
%    \begin{macrocode}
%%\ifchilddoc\else\providecommand{\version}{draft}\fi
%    \end{macrocode}

% Define the default values for the |\version| flag
% (|final| for the main file and |draft| for childs):
%    \begin{macrocode}
\ifchilddoc
\providecommand{\version}{draft}
\else
\providecommand{\version}{final}
\fi
%    \end{macrocode}

% Load the standard document class:
%    \begin{macrocode}
\documentclass[12pt]{article}
%    \end{macrocode}

% Start the document body:
%    \begin{macrocode}
\begin{document}
%    \end{macrocode}

% Declare a title page.
% Print title, part of document being processed and version flag:
%    \begin{macrocode}
\addtocounter{page}{-1}
\begin{center}
{\LARGE\bfseries{}childdoc example\par}
\vspace{1cm}
\ifchilddoc
\ifchilddocmanual part\else chapter\fi:
`\childdocname' of `\childdocjob'\par
\else
main document: `\childdocjob'\par
\fi
version: \version\par
\end{center}
\newpage
%    \end{macrocode}

% Manually include selected file,
% otherwise process as usual:
%    \begin{macrocode}
\ifchilddocmanual
\section*{part `\childdocname'}
\input{\childdocname}
\else
%    \end{macrocode}

% Include the two chapters:
%    \begin{macrocode}
\include{cdocsch1}
\include{cdocsch2}
%    \end{macrocode}

% Include the two parts unless only chapters should be displayed:
%    \begin{macrocode}
\ifchilddoc\else
\section{part three}
\input{cdocspt3}
\section{part four}
\input{cdocspt4}
\fi
%    \end{macrocode}

% Process as usual until here:
%    \begin{macrocode}
\fi
%    \end{macrocode}

% End of document body:
%    \begin{macrocode}
\end{document}
%    \end{macrocode}
%\iffalse
%</samplemain>
%\fi
%
% %%%%%%%%%%%%%%%%%%%%%%%%%%%%%%%%%%%%%%
% \paragraph{Chapter Include Files.}
%
% The include files are called |cdocsch1.tex| and |cdocsch2.tex|.
%
%\iffalse
%<*samplechap1|samplechap2>
%\fi

% Optional override for |\version| flag:
%    \begin{macrocode}
%%\providecommand{\version}{final}
%    \end{macrocode}

% Include the main document:
%    \begin{macrocode}
\input{childdoc.def}
\childdocof{cdocsamp}
%    \end{macrocode}

%\iffalse
%</samplechap1|samplechap2>
%\fi
%
%\iffalse
%<*samplechap1>
%\fi
% Some text for chapter 1:
%    \begin{macrocode}
\section{one}
some text in chapter one
%    \end{macrocode}

%\iffalse
%</samplechap1>
%\fi
% Some text for chapter 2:
%\iffalse
%<*samplechap2>
%\fi
%    \begin{macrocode}
\section{two}
more text in chapter two
%    \end{macrocode}

%\iffalse
%</samplechap2>
%\fi
%
% %%%%%%%%%%%%%%%%%%%%%%%%%%%%%%%%%%%%%%
% \paragraph{Part Include Files.}
%
% The include files are called |cdocspt3.tex| and |cdocspt4.tex|.
%
%\iffalse
%<*samplepart3|samplepart4>
%\fi

% Optional override for |\version| flag:
%    \begin{macrocode}
%%\providecommand{\version}{final}
%    \end{macrocode}

% Include the main document:
%    \begin{macrocode}
\input{childdoc.def}
\childdocby{cdocsamp}
%    \end{macrocode}

%\iffalse
%</samplepart3|samplepart4>
%\fi
%
%\iffalse
%<*samplepart3>
%\fi
% Some text for part 3:
%    \begin{macrocode}
some text in part three
%    \end{macrocode}

%\iffalse
%</samplepart3>
%\fi
% Some text for part 4:
%\iffalse
%<*samplepart4>
%\fi
%    \begin{macrocode}
more text in part four
%    \end{macrocode}

%\iffalse
%</samplepart4>
%\fi
%
% %%%%%%%%%%%%%%%%%%%%%%%%%%%%%%%%%%%%%%
% \paragraph{Forwarding for a Complete Draft.}
%
% The following forwarding file |cdocsdrf.tex|
% compiles the main document in draft mode:
%\iffalse
%<*sampledraft>
%\fi
%    \begin{macrocode}
\def\version{draft}
\input{childdoc.def}
\childdocforward{cdocsamp}
%    \end{macrocode}

%\iffalse
%</sampledraft>
%\fi
%
% %%%%%%%%%%%%%%%%%%%%%%%%%%%%%%%%%%%%%%
% \paragraph{Forwarding for Final Version of the Chapters.}
%
% The following forwarding files |cdocsfn1.tex| and |cdocsfn2.tex|
% (with identical content)
% compile the final versions of the child documents
% |cdocsch1.tex| and |cdocsch2.tex|, respectively:
%\iffalse
%<*samplefinal>
%\fi
%    \begin{macrocode}
\def\version{final}
\input{childdoc.def}
\childdocforwardprefix[cdocsamp]{cdocsfn}{cdocsch}
%    \end{macrocode}

%\iffalse
%</samplefinal>
%\fi
%
% %%%%%%%%%%%%%%%%%%%%%%%%%%%%%%%%%%%%%%
% \paragraph{Command Line Processing.}
%
% The following three command lines generate the output files
% |cdocscld|, |cdocscl1| and |cdocscl2|
% which should be identical to
% |cdocsdrf|, |cdocsch1| and |cdocsfn2|, respectively:
% \begin{center}
% \begin{tabular}{l}
% |latex -jobname cdocscld \|\\
% |  "\def\version{draft}\input{childdoc.def}\childdocforward{cdocsamp}"|\\
% |latex -jobname cdocscl1 \|\\
% |  "\input{childdoc.def}\childdocforward[cdocsamp]{cdocsch1}"|\\
% |latex -jobname cdocscl2 \|\\
% |  "\def\version{final}\input{childdoc.def}\childdocforward{cdocsch2}"|
% \end{tabular}
% \end{center}
% Note that the trailing backslash on each first line
% merely continues the input to the second line
% (for convenient cut ant paste).
% Furthermore, the command |latex| can be replaced by any
% of its alternative versions such as |pdflatex|.
%
% %%%%%%%%%%%%%%%%%%%%%%%%%%%%%%%%%%%%%%%%%%%%%%%%%%%%%%%%%%%%%%%%%%%%%%%%%%%%%%
% %%%%%%%%%%%%%%%%%%%%%%%%%%%%%%%%%%%%%%%%%%%%%%%%%%%%%%%%%%%%%%%%%%%%%%%%%%%%%%
% \section{Implementation}
%\iffalse
%<*package>
%\fi
%
% This section describes the definitions file |childdoc.def|.

% The definitions cannot be loaded using |\usepackage| or |\RequirePackage|
% which has a mechanism to prevent loading a style file more than once.
% When loading the definitions by means of |\input|
% multiple instances have to be prevented manually:
%\iffalse
%This code needs to be before the `\ProvidesFile' directive
%which is defined at the beginning of this file.
%Therefore it is also placed there and commented out here.
%</package>
%<*discard>
%\fi
%    \begin{macrocode}
\ifdefined\childdocmain\endinput\fi
%    \end{macrocode}
%\iffalse
%</discard>
%<*package>
%\fi
%
% \macro{\ifchilddoc}
% \macro{\ifchilddocmanual}
% The conditional |\ifchilddoc| tells whether a
% child (true) or main (false) document is being compiled.
% The conditional |\ifchilddocmanual| tells whether
% the |\includeonly| mechanism is used (false) or
% the selection of child files must be performed manually (true).
% The definitions initialise to false:
%    \begin{macrocode}
\newif\ifchilddoc
\newif\ifchilddocmanual
%    \end{macrocode}

% \macro{\childdocname}
% \macro{\childdocjob}
% The macro |\childdocname| stores the name of the main document
% to be compiled. The macro |\childdocjob| stores the name of
% the document on which the \LaTeX{} compiler was originally invoked.
% The content of |\jobname| cannot be compared
% to filenames specified in the source due to different catcodes.
% The following code rescans |\jobname|, stores the result
% in |\childdocname| and saves a copy in |\childdocjob|:
%    \begin{macrocode}
\edef\childdocname{\scantokens\expandafter{\jobname\noexpand}}
\let\childdocjob\childdocname
%    \end{macrocode}

% \macro{\childdocdisable}
% The macro |\childdocdisable| prevents the main file
% from being processed more than once.
% At this stage, the main document command |\childdocmain|
% is assumed to be called once again where it should do nothing.
% Any subsequent call to it should prevent
% a secondary processing of the main document
% It overwrites the forwarding commands
% |\childdocof| and |\childdocforward|
% with empty macros to prevent further inclusions of the main document:
%    \begin{macrocode}
\newcommand{\childdocdisable}
{
  \renewcommand{\childdocmain}[1]{\renewcommand{\childdocmain}[1]{\endinput}}
  \renewcommand{\childdocof}[1]{}
  \renewcommand{\childdocby}[2][]{}
  \renewcommand{\childdocforward}[2][]{}
  \renewcommand{\childdocdisable}{}
}
%    \end{macrocode}

% \macro{\childdocmain}
% The macro |\childdocmain| is to be called at the top of the main file
% with nothing or the main filename (without extension) as argument.
% First, it breaks loops.
% If the argument is not empty and does not match |\childdocname|
% (which is set by the first inclusion of |childdoc.def|),
% |\ifchilddoc| is set to true, |\includeonly| is applied to the child file
% and |\jobname| is set to the main file
% (for proper handling of |.aux| files):
%    \begin{macrocode}
\newcommand{\childdocmain}[1]
{
  \childdocdisable\childdocmain{}
  \if?#1?\else
    \begingroup
      \def\childdoctmp{#1}
      \ifx\childdoctmp\childdocname
        \def\childdoctmp{}
      \else
        \def\childdoctmp
        {
          \childdoctrue
          \includeonly{\childdocname}
          \def\childdocjob{#1}
          \def\jobname{#1}
        }
      \fi
      \expandafter
    \endgroup
    \childdoctmp
  \fi
}
%    \end{macrocode}

% \macro{\childdocof}
% The command |\childdocof| redirects
% compilation to the main file |#1|.
%    \begin{macrocode}
\newcommand{\childdocof}[1]
{
  \childdocdisable
  \childdoctrue
  \includeonly{\childdocname}
  \def\jobname{#1}
  \def\childdocjob{#1}
  \input{#1}
}
%    \end{macrocode}

% \macro{\childdocby}
% The command |\childdocby| ....
%    \begin{macrocode}
\newcommand{\childdocby}[2][]
{
  \childdocdisable
  \childdoctrue
  \childdocmanualtrue
  \if?#1?\else
    \def\jobname{#2}
  \fi
  \def\childdocjob{#2}
  \input{#2}
  \endinput
}
%    \end{macrocode}

% \macro{\childdocforward}
% The command |\childdocforward| redirects
% compilation to the main file or
% (if the optional argument is given) a child file.
% Parameters are set as if the main file
% or a child file starting with |\childdocof| was compiled.
% Then compilation is handed over to the main file:
%    \begin{macrocode}
\newcommand{\childdocforward}[2][]
{
  \begingroup
    \if?#1?
      \def\childdoctmp
      {
        \def\childdocname{#2}
        \def\childdocjob{#2}
        \def\jobname{#2}
        \input{#2}
        \endinput
      }
    \else
      \def\childdoctmp
      {
        \childdocdisable
        \def\childdocname{#2}
        \childdoctrue
        \includeonly{#2}
        \def\childdocjob{#1}
        \def\jobname{#1}
        \input{#1}
        \endinput
      }
    \fi
    \expandafter
  \endgroup
  \childdoctmp
}
%    \end{macrocode}

% \macro{\childdocforwardprefix}
% The command |\childdocforwardprefix| redirects
% compilation to the main or a child file by means of a pattern.
% The prefix |#1| in the current filename is replaced by |#2|
% and the suffix of the current filename is kept
% (it is assumed that the filename does not contain the substring `|~~~|'
% which is used as a delimiter).
% Compilation is handed over to the new file by |\childdocforward|:
%    \begin{macrocode}
\newcommand{\childdocforwardprefix}[3][]
{
  \begingroup
    \def\childdocextract #2##1~~~{\def\childdoctmp{\childdocforward[#1]{#3##1}}}
    \expandafter\childdocextract\childdocname~~~
    \expandafter
  \endgroup
  \childdoctmp
}
%    \end{macrocode}

% \macro{\childdoc}
% The deprecated macro |\childdoc| is a legacy version of |\childdocmain|:
%    \begin{macrocode}
\newcommand{\childdoc}{\childdocmain}
%    \end{macrocode}

% \macro{\childdocredirect}
% The deprecated macro |\childdocredirect| is a legacy version
% of |\childdocforward| and |\childdocforwardprefix|:
%    \begin{macrocode}
\newcommand{\childdocredirect}[2][]
{
  \begingroup
    \if?#1?
      \def\childdoctmp{\childdocforward{#2}}
    \else
      \def\childdoctmp{\childdocforwardprefix{#1}{#2}}
    \fi
    \expandafter
  \endgroup
  \childdoctmp
}
%    \end{macrocode}

%\iffalse
%</package>
%\fi
%
\endinput
\childdocforward[|\textit{main}|]{|\textit{dest}|}"|
\end{center}
%
Here \textit{target} is the name of the output file,
\textit{main} is the name of the main file
and \textit{dest} is the name of the main or child file to be processed
(all filenames without extensions).
The optional argument \textit{main} can be omitted
if \textit{main} matches \textit{dest}.
Optionally, compilation \textit{flags} can be defined via |\def| commands.
This command line makes the \TeX{} engine believe
it is compiling the file \textit{target}
whose content is specified as the latter parameter.
The provided code then forwards the processing to
\textit{main} or \textit{dest} as described in \secref{sec:forward}.

%%%%%%%%%%%%%%%%%%%%%%%%%%%%%%%%%%%%%%%%%%%%%%%%%%%%%%%%%%%%%%%%%%%%%%%%%%%%%%%%
\subsection{Include by Input}
\label{sec:input}

Including child documents by |\include| has some restrictions by design.
Most notably, the content of a child document always occupies
its own set of pages; pages cannot be shared between child documents.
Usually, this behaviour makes perfect sense
because each child document contain an essential part of the document.
However, in some situations it may be desirable to compose
a document from a collection of parts
without having mandatory page breaks between then.
For this case, the package
provides a mechanism to include parts
by |\input| which can also be processed individually.
However, by construction this mechanism
requires manual handling of the content to be output.

%%%%%%%%%%%%%%%%%%%%%%%%%%%%%%%%%%%%%%%%
\DescribeMacro{\ifchilddocmanual}
The main file should be prepared as usual, see \secref{sec:include}.
However, the document body must make a distinction
between processing of an individual part and of the main document, e.g.:
%
\begin{center}
\begin{tabular}{l}
|\ifchilddocmanual|\\
|\input{\childdocname}|\\
|\||else|\\
\textit{document body with }|\input{|\textit{part}|}|\\
|\||fi|
\end{tabular}
\end{center}
%
The conditional |\ifchilddocmanual| is true whenever
a part to be included by |\input| is being compiled,
and the name of the part is stored in |\childdocname|.

%%%%%%%%%%%%%%%%%%%%%%%%%%%%%%%%%%%%%%%%
\DescribeMacro{\childdocby}
Each part to be included by |\input| should start with:
%
\begin{center}
\begin{tabular}{l}
|% \iffalse
%
% childdoc.dtx Copyright (C) 2017-2018 Niklas Beisert
%
% This work may be distributed and/or modified under the
% conditions of the LaTeX Project Public License, either version 1.3
% of this license or (at your option) any later version.
% The latest version of this license is in
%   http://www.latex-project.org/lppl.txt
% and version 1.3 or later is part of all distributions of LaTeX
% version 2005/12/01 or later.
%
% This work has the LPPL maintenance status `maintained'.
%
% The Current Maintainer of this work is Niklas Beisert.
%
% This work consists of the files childdoc.dtx and childdoc.ins
% and the derived files childdoc.def and cdocsamp.tex with
% cdocsch1.tex, cdocsch2.tex, cdocsdrf.tex, cdocsfn1.tex, cdocsfn2.tex.
%
%<package>\ifdefined\childdocmain\endinput\fi
%<package>\ProvidesFile{childdoc.def}[2018/12/30 v2.0 child document driver]
%<samplemain>\ProvidesFile{cdocsamp.tex}[2018/12/30 v2.0 sample for childdoc]
%<*driver>
%\ProvidesFile{childdoc.drv}[2018/12/30 v2.0 childdoc reference manual file]
\PassOptionsToClass{10pt,a4paper}{article}
\documentclass{ltxdoc}

\usepackage[margin=35mm]{geometry}
\usepackage{hyperref}
\usepackage{hyperxmp}
\usepackage[usenames]{color}

\hypersetup{colorlinks=true}
\hypersetup{pdfstartview=FitH}
\hypersetup{pdfpagemode=UseNone}
\hypersetup{pdfsource={}}
\hypersetup{pdflang={en-UK}}
\hypersetup{pdfcopyright={Copyright 2017-2018 Niklas Beisert.
  This work may be distributed and/or modified under the
  conditions of the LaTeX Project Public License, either version 1.3
  of this license or (at your option) any later version.}}
\hypersetup{pdflicenseurl={http://www.latex-project.org/lppl.txt}}
\hypersetup{pdfcontactaddress={ETH Zurich, ITP, HIT K,
  Wolfgang-Pauli-Strasse 27}}
\hypersetup{pdfcontactpostcode={8093}}
\hypersetup{pdfcontactcity={Zurich}}
\hypersetup{pdfcontactcountry={Switzerland}}
\hypersetup{pdfcontactemail={nbeisert@itp.phys.ethz.ch}}
\hypersetup{pdfcontacturl={http://people.phys.ethz.ch/\xmptilde nbeisert/}}

\newcommand{\secref}[1]{\hyperref[#1]{section \ref*{#1}}}

\parskip1ex
\parindent0pt
\let\olditemize\itemize
\def\itemize{\olditemize\parskip0pt}

\begin{document}

\title{The \textsf{childdoc} Package}
\hypersetup{pdftitle={The childdoc Package}}
\author{Niklas Beisert\\[2ex]
  Institut f\"ur Theoretische Physik\\
  Eidgen\"ossische Technische Hochschule Z\"urich\\
  Wolfgang-Pauli-Strasse 27, 8093 Z\"urich, Switzerland\\[1ex]
  \href{mailto:nbeisert@itp.phys.ethz.ch}
  {\texttt{nbeisert@itp.phys.ethz.ch}}}
\hypersetup{pdfauthor={Niklas Beisert}}
\hypersetup{pdfsubject={Manual for the LaTeX2e Package childdoc}}
\date{30 December 2018, \textsf{v2.0}}
\maketitle

\begin{abstract}\noindent
\textsf{childdoc} is a \LaTeXe{} package
that enables the direct compilation
of document sections included by |\include|
to individual files.
\end{abstract}

\begingroup
\parskip0ex
\tableofcontents
\endgroup

%%%%%%%%%%%%%%%%%%%%%%%%%%%%%%%%%%%%%%%%%%%%%%%%%%%%%%%%%%%%%%%%%%%%%%%%%%%%%%%%
%%%%%%%%%%%%%%%%%%%%%%%%%%%%%%%%%%%%%%%%%%%%%%%%%%%%%%%%%%%%%%%%%%%%%%%%%%%%%%%%
\section{Introduction}

\LaTeX{} provides a mechanism to structure a large document (such as a book)
into a main file and several child files (containing the chapters)
using the |\include| command.
This mechanism is beneficial for documents
which span hundreds of pages in order to
make the source file(s) more manageable.
Moreover, compilation can be restricted to
selected child files by means of the |\includeonly| command.
The latter feature can be used to reduce the compilation time while editing
(this was significantly more useful in the earlier days of \LaTeX{})
or to generate a smaller document which is easier to navigate.
Another application of |\includeonly| is to generate
documents consisting of selected parts of the complete document.

However, there are a few drawbacks of the plain |\include| mechanism:
\begin{itemize}
\item
The child files cannot be compiled on their own,
they can only be compiled via the main file.
A naive editing environment
(such as a text editor with an option
to have the current file processed by \LaTeX)
may require one to switch to the main file before compiling;
attempting to compile the child file produces errors.
\item
The main file must be modified (each time)
to adjust the |\includeonly| command
to the present needs. This easily leaves the main file in a messy state.
\item
The generated document will always carry the filename
of the main document. This is inconvenient if
several child files are to be compiled and
to be kept for distribution.
\end{itemize}

The present package provides a simple interface
to make child files individually compilable by \LaTeX{}.
Compiling a child file then has the same effect as compiling
the main file with an |\includeonly| command
to select the appropriate child.
Moreover the generated document will carry the name of the child
rather than the main file.
This resolves all three above issues.

This feature is meant to make the editing of books,
thesis documents and lecture notes somewhat more convenient.
However, the package can also be used efficiently for
composing a series of documents (such as exercise sheets)
which are typically distributed individually.
It then assists the author in generating the individual documents
(potentially in different versions)
as well as a document containing the collected series.
Another application is in developing style files
or other kinds of included material
where compilation of the style file could redirect
to a sample or test file.

%%%%%%%%%%%%%%%%%%%%%%%%%%%%%%%%%%%%%%%%%%%%%%%%%%%%%%%%%%%%%%%%%%%%%%%%%%%%%%%%
%%%%%%%%%%%%%%%%%%%%%%%%%%%%%%%%%%%%%%%%%%%%%%%%%%%%%%%%%%%%%%%%%%%%%%%%%%%%%%%%
\section{Usage}

First of all, the package \textsf{childdoc} is \emph{not} a standard
\LaTeXe{} |.sty| style file! Therefore it needs to be invoked in
a non-standard way.

%%%%%%%%%%%%%%%%%%%%%%%%%%%%%%%%%%%%%%%%%%%%%%%%%%%%%%%%%%%%%%%%%%%%%%%%%%%%%%%%
\subsection{Included Files}
\label{sec:include}

%%%%%%%%%%%%%%%%%%%%%%%%%%%%%%%%%%%%%%%%
\DescribeMacro{\childdocmain}
To use the package, add the commands
\begin{center}
\begin{tabular}{l}
|\input{childdoc.def}|\\
|\childdocmain{}|\\
\end{tabular}
\end{center}
at the very top of the main \LaTeX{} file,
in particular \emph{before} the |\documentclass| statement!
The argument of |\childdocmain| should be left empty
(but it must be present).

%%%%%%%%%%%%%%%%%%%%%%%%%%%%%%%%%%%%%%%%
\DescribeMacro{\childdocof}
Furthermore, add the commands
\begin{center}
\begin{tabular}{l}
|\input{childdoc.def}|\\
|\childdocof{|\textit{main}|}|\\
\end{tabular}
\end{center}
at the top of every child file \textit{child}
which is included by |\include{|\textit{child}|}|
from within the main file
(or at least for those files to be compiled individually).
The argument \textit{main} must be the filename of the main file.

There are a couple of
considerations in setting up the main and child documents:

%%%%%%%%%%%%%%%%%%%%%%%%%%%%%%%%%%%%%%%%
\paragraph{Restrictions.}

Please note the following restrictions:
\begin{itemize}
\item
|\childdocmain| must be called with one argument \textit{main}
to ensure compatibility with earlier version of the package.
It must either be empty (|\childdocmain{}|)
or precisely match the filename of the main file in which it is specified.
See \secref{sec:detection} for further information.
\item
The filename \textit{main} must be specified without the |.tex| extension.
\item
The filename \textit{main} is case sensitive
(even in case-insensitive file systems)
due to internal string comparison.
\item
The argument \textit{main} should be fully expanded, it cannot be a macro.
\item
Subdirectories and special characters should be avoided in filenames.
\item
The command |\childdocmain{|\textit{main}|}| must be followed by a whitespace.
It should not be followed immediately by another command
or by a comment mark `|%|'.
This is because the \TeX{} parser reads the token immediately following
the argument of |\childdocmain| and puts it
at the beginning of every child section;
however, a white\-space is ignored.
\end{itemize}

%%%%%%%%%%%%%%%%%%%%%%%%%%%%%%%%%%%%%%%%
\paragraph{Content of Main File.}

It is advisable to place all content in the child files included by |\include|.
Any output contained in the main file will appear in all child documents
unless suppressed manually;
it cannot be suppressed automatically by the |\includeonly| directive
and thus should normally be avoided.
A method to include some content in the main file
by means of conditional processing is described in \secref{sec:conditional}.

%%%%%%%%%%%%%%%%%%%%%%%%%%%%%%%%%%%%%%%%
\paragraph{Page Numbering.}

When only a part of the document is compiled,
the appropriate numbering of pages
(as well as other status parameters)
is determined from the |.aux| files.
The latter contain information from previous passes.
However this information needs to propagate through
all intermediate child documents.
Therefore the page numbering in child documents may well
be inconsistent until the complete document is compiled at least once.

A useful (if unconventional) way to always ensure a consistent
page numbering is to restart the numbering in each child document
and denote the pages by `\textit{child}|.|\textit{page}'
where \textit{child} represents the chapter/section number of the child file.
This can be achieved by the command
|\numberwithin{page}{|\textit{child}|}|
of the \textsf{amsmath} package
where \textit{child} can be |chapter| or |section|
depending on the chosen structuring.
Alternatively, one can modify the macro |\thepage| appropriately
and reset the counter |page| at the start of each child file.

%%%%%%%%%%%%%%%%%%%%%%%%%%%%%%%%%%%%%%%%%%%%%%%%%%%%%%%%%%%%%%%%%%%%%%%%%%%%%%%%
\subsection{Conditional Processing}
\label{sec:conditional}

The package provides a mechanism to compile different versions
of a document. To customise the versions further some conditional processing
can come in handy to distinguish which version is being compiled.
The package provides two macros to describe the compilation context:

%%%%%%%%%%%%%%%%%%%%%%%%%%%%%%%%%%%%%%%%
\DescribeMacro{\ifchilddoc}
The conditional |\ifchilddoc| distinguishes between the compilation of
child documents and the main document:
%
\begin{center}
|\ifchilddoc |\textit{child-code}| |[|\||else |\textit{main-code}]| \||fi|
\end{center}

%%%%%%%%%%%%%%%%%%%%%%%%%%%%%%%%%%%%%%%%
\DescribeMacro{\childdocname}
\DescribeMacro{\childdocjob}
The macro |\childdocname| contains the filename (without extension)
of the main or child file being processed.
Note that |\childdocjob| will always contain the name of the main file.

%%%%%%%%%%%%%%%%%%%%%%%%%%%%%%%%%%%%%%%%
\paragraph{Title Page.}

Conditional processing can be used to include a title or banner page
in the main document when proper precautions are taken.
Importantly, the code in the main file should ensure that the page counter
(as well as other status parameters which are stored in the |.aux| files)
takes the same value after the conditional processing.
Otherwise the page numbers may take divergent values
depending on which part is compiled.

For example, a title page could be declared by:
%
\begin{center}
\begin{tabular}{l}
|\ifchilddoc\||else|\\
|\addtocounter{page}{-1}|\\
\textit{code for title page}\\
|\newpage|\\
|\||fi|
\end{tabular}
\end{center}
%
A banner page for the child documents can be generated by:
%
\begin{center}
\begin{tabular}{l}
|\ifchilddoc|\\
|\addtocounter{page}{-1}|\\
\textit{code for banner page}\\
|\newpage|\\
|\||fi|
\end{tabular}
\end{center}
%
Here one could write a message such as:
\begin{center}
|This is the part \childdocname{} of \childdocjob{}.|
\end{center}

%%%%%%%%%%%%%%%%%%%%%%%%%%%%%%%%%%%%%%%%%%%%%%%%%%%%%%%%%%%%%%%%%%%%%%%%%%%%%%%%
\subsection{Flags}
\label{sec:flags}

The package makes it easy to generate different versions
of the main or child documents.
To this end compilation flags can be defined
and assigned different default values.
They will be particularly useful in conjunction
with the forwarding mechanism described in \secref{sec:forward}.

For example, it may be useful to have a flag |\version|
which can be set to |draft| or |final|.
The document source will contain some conditional code
depending on the value of |\version|.
Suppose further, the flag should default to |final| for the main file
and to |draft| for child files
which is a natural assignment for editing the document.
This is achieved by placing the following code
in the preamble of the main document
(below the |\childdocmain| directive):
%
\begin{center}
\begin{tabular}{l}
|\ifchilddoc|\\
|\providecommand{\version}{draft}|\\
|\||else|\\
|\providecommand{\version}{final}|\\
|\||fi|
\end{tabular}
\end{center}
%
The definition by |\providecommand| makes sure
that previous definitions are not overwritten.
Further statements |\providecommand{\version}{...}|
can thus be added before the above code to override it.

For the main file, one might add a line
(between |\childdocmain| and the above block)
%
\begin{center}
|%\ifchilddoc\||else\providecommand{\version}{draft}\||fi|
\end{center}
%
which can be uncommented to produce a draft version.
Likewise one can add a line to the very top of a child file
(above the |\childdocof{|\textit{main}|}| directive)
%
\begin{center}
|%\providecommand{\version}{final}|
\end{center}
%
which can be uncommented to produce the final version of this child document.

%%%%%%%%%%%%%%%%%%%%%%%%%%%%%%%%%%%%%%%%%%%%%%%%%%%%%%%%%%%%%%%%%%%%%%%%%%%%%%%%
\subsection{Forwarding}
\label{sec:forward}

Different versions of the main or child documents
using compilation flags as described in \secref{sec:flags}
can be (permanently) stored in different files
for convenient compilation, viewing and distribution.
To this end, the package defines a command
to pass on compilation to a different file:

%%%%%%%%%%%%%%%%%%%%%%%%%%%%%%%%%%%%%%%%
\DescribeMacro{\childdocforward}
The command |\childdocforward| redirects processing to
another source file:
%
\begin{center}
\begin{tabular}{l}
|\input{childdoc.def}|\\
|\childdocforward[|\textit{main}|]{|\textit{dest}|}|\\
\end{tabular}
\end{center}
%
The argument \textit{dest} is the destination file
(without extension).
It should be the main file or one of the child files.
Note that further \textsf{childdoc} directives
such as |\childdocof| and |\childdocforward|
in the indicated file will be processed in this form.
The optional argument \textit{main}
passes on directly to the main file \textit{main}
while pretending to compile the child \textit{dest}.
This form behaves as if \textit{dest}
issues |\childdocof{|\textit{main}|}| right away,
and no further \textsf{childdoc} directives will be processed.

%%%%%%%%%%%%%%%%%%%%%%%%%%%%%%%%%%%%%%%%
\DescribeMacro{\...prefix}
In the alternative form |\childdocforwardprefix|,
%
\begin{center}
\begin{tabular}{l}
|\input{childdoc.def}|\\
|\childdocforwardprefix[|\textit{main}|]{|\textit{prefix}|}{|\textit{dest}|}|
\end{tabular}
\end{center}
%
the destination file is determined by a pattern
depending on the current file:
To make this work, the current file must be called
`{\textit{prefix}\hspace{0.2em}\textit{suffix}}'
with \textit{prefix} matching precisely the argument.
Processing is then passed on to the file
`{\textit{dest}\hspace{0.2em}\textit{suffix}}'.
Surely, the same effect is achieved by
directly specifying the
argument `{\textit{dest}\hspace{0.2em}\textit{suffix}}'
in the first form.
However, that requires to set up a different file
for each child. With the alternative form of the command
all these files can have exactly the same content
which simplifies setting them up and maintaining them.

For example, the following file |draft.tex|
with a compilation flag |\version| as described in \secref{sec:flags}
compiles the main document as a draft:
%
\begin{center}
\begin{tabular}{l}
|\def\version{draft}|\\
|\input{childdoc.def}|\\
|\childdocforward{|\textit{main}|}|
\end{tabular}
\end{center}
%
Likewise, the following files |final|\textit{nn}|.tex|
compile the final version of the child document
|child|\textit{nn}|.tex|:
%
\begin{center}
\begin{tabular}{l}
|\def\version{final}|\\
|\input{childdoc.def}|\\
|\childdocforwardprefix{final}{child}|
\end{tabular}
\end{center}
%

Note that when several versions of a main file and/or of each child file
are to be generated, it may be convenient to set up a |Makefile| or
shell script to automatise the process.

%%%%%%%%%%%%%%%%%%%%%%%%%%%%%%%%%%%%%%%%%%%%%%%%%%%%%%%%%%%%%%%%%%%%%%%%%%%%%%%%
\subsection{Command Line Processing}
\label{sec:commandline}

The effect of redirection files can also be achieved by invoking
the \LaTeX{} compiler with a more elaborate command line.
Most conveniently this should be done as part
of a shell script or a |Makefile|.

When using \textsf{childdoc} in the main file, the following
command lines effectively perform a redirection
(note that depending on the shell being used,
backslashes may have to be doubled: `|\|' $\to$ `|\\|'):
%
\begin{center}
|... -jobname "|\textit{target}|" |\\|"|[\textit{flags}]%
|\input{childdoc.def}\childdocforward[|\textit{main}|]{|\textit{dest}|}"|
\end{center}
%
Here \textit{target} is the name of the output file,
\textit{main} is the name of the main file
and \textit{dest} is the name of the main or child file to be processed
(all filenames without extensions).
The optional argument \textit{main} can be omitted
if \textit{main} matches \textit{dest}.
Optionally, compilation \textit{flags} can be defined via |\def| commands.
This command line makes the \TeX{} engine believe
it is compiling the file \textit{target}
whose content is specified as the latter parameter.
The provided code then forwards the processing to
\textit{main} or \textit{dest} as described in \secref{sec:forward}.

%%%%%%%%%%%%%%%%%%%%%%%%%%%%%%%%%%%%%%%%%%%%%%%%%%%%%%%%%%%%%%%%%%%%%%%%%%%%%%%%
\subsection{Include by Input}
\label{sec:input}

Including child documents by |\include| has some restrictions by design.
Most notably, the content of a child document always occupies
its own set of pages; pages cannot be shared between child documents.
Usually, this behaviour makes perfect sense
because each child document contain an essential part of the document.
However, in some situations it may be desirable to compose
a document from a collection of parts
without having mandatory page breaks between then.
For this case, the package
provides a mechanism to include parts
by |\input| which can also be processed individually.
However, by construction this mechanism
requires manual handling of the content to be output.

%%%%%%%%%%%%%%%%%%%%%%%%%%%%%%%%%%%%%%%%
\DescribeMacro{\ifchilddocmanual}
The main file should be prepared as usual, see \secref{sec:include}.
However, the document body must make a distinction
between processing of an individual part and of the main document, e.g.:
%
\begin{center}
\begin{tabular}{l}
|\ifchilddocmanual|\\
|\input{\childdocname}|\\
|\||else|\\
\textit{document body with }|\input{|\textit{part}|}|\\
|\||fi|
\end{tabular}
\end{center}
%
The conditional |\ifchilddocmanual| is true whenever
a part to be included by |\input| is being compiled,
and the name of the part is stored in |\childdocname|.

%%%%%%%%%%%%%%%%%%%%%%%%%%%%%%%%%%%%%%%%
\DescribeMacro{\childdocby}
Each part to be included by |\input| should start with:
%
\begin{center}
\begin{tabular}{l}
|\input{childdoc.def}|\\
|\childdocby{|\textit{main}|}|\\
\end{tabular}
\end{center}
%
The directive |\childdocby| is similar to |\childdocof|
described in \secref{sec:include},
but the subsequent selection of content must be done manually.
To that end, both |\ifchilddoc| and |\ifchilddocmanual|
will be true upon processing of a part,
and the name of the part is stored in |\childdocname|.
Note that |\jobname| will be set to the filename of the current part
so that each part receives an individual |.aux| file
that does not interfere with the |.aux| file(s) of the main document.
This behaviour can be altered by the alternative form
|\childdocby[*]{|\textit{main}|}| (with a non-empty optional argument)
which uses the |.aux| file of the main document
by setting |\jobname| to \textit{main}.

%%%%%%%%%%%%%%%%%%%%%%%%%%%%%%%%%%%%%%%%%%%%%%%%%%%%%%%%%%%%%%%%%%%%%%%%%%%%%%%%
\subsection{Driver Development}
\label{sec:driver}

The \textsf{childdoc} mechanism can also be use for the development
of definition files such as \LaTeX{} styles or classes.
This case differs from the above setup with multiple parts
included by |\include| in that no |\includeonly| should be invoked.
This can be achieved by starting the include file
(before |\ProvidesPackage|) with:
%
\begin{center}
\begin{tabular}{l}
|\input{childdoc.def}|\\
|\childdocforward{|\textit{main}|}|\\
\end{tabular}
\end{center}
%
or alternatively with:
%
\begin{center}
\begin{tabular}{l}
|\input{childdoc.def}|\\
|\childdocby{|\textit{main}|}|\\
\end{tabular}
\end{center}
%
Both forms have slightly different effects as described above.
The main file is prepared as usual, see \secref{sec:include}.

%%%%%%%%%%%%%%%%%%%%%%%%%%%%%%%%%%%%%%%%%%%%%%%%%%%%%%%%%%%%%%%%%%%%%%%%%%%%%%%%
\subsection{Legacy Detection}
\label{sec:detection}

The directive |\childdocmain| in the main file can detect
whether the complete document or merely a child is to be compiled
even without using the directive |\childdocof|.
This method is deprecated because it is less robust
and there is no compelling reason to use it;
it is merely provided for backward compatibility
and it may be removed in future versions.

If the detection mechanism is to be used,
it is mandatory to correctly specify
the filename of the main file as the argument of |\childdocmain|:
%
\begin{center}
\begin{tabular}{l}
|\input{childdoc.def}|\\
|\childdocmain{|\textit{main}|}|\\
\end{tabular}
\end{center}
%
If |\jobname| does not match the argument \textit{main} of |\childdocmain|,
it is assumed that |\jobname| points to the child file to be compiled.
When using |\childdocmain| with the main file specified as argument,
it suffices to start a child file
with just |\input{|\textit{main}|}|
without loading of the package and using |\childdocof|.
If instead all processing is done
with the appropriate \textsf{childdoc} directives,
the argument of \textit{main} of |\childdocmain| can be empty.

An alternative version of the command line processing described
in \secref{sec:commandline} using the detection mechanism reads:
%
\begin{center}
|... -jobname "|\textit{target}|" "|[\textit{flags}]%
[|\def\jobname{|\textit{dest}|}|]|\input{|\textit{main}|}"|
\end{center}

%%%%%%%%%%%%%%%%%%%%%%%%%%%%%%%%%%%%%%%%%%%%%%%%%%%%%%%%%%%%%%%%%%%%%%%%%%%%%%%%
\subsection{Manual Code}
\label{sec:manual}

In case one cannot be certain whether the definitions file |childdoc.def|
is installed on the target \TeX{} distribution
and one prefers not to ship it,
it is conceivable to paste a few relevant commands into the sources.

To that end, drop all statements |\input{childdoc.def}|
and perform the replacements as outlined below.
Instead of |\childdocmain{|\textit{main}|}| add the following code
to the top of the main file:
%
\begin{center}
\begin{tabular}{l}
|\||ifdefined\childdocname\endinput\||fi\newif\ifchilddoc|\\
|\edef\childdocname{\scantokens\expandafter{\jobname\noexpand}}|\\
|\def\childdocmain{|\textit{main}|}\||ifx\childdocmain\childdocname\||else|\\
|\childdoctrue\includeonly{\childdocname}\let\jobname\childdocmain\||fi|\\
\end{tabular}
\end{center}
%
Instead of |\childdocof{|\textit{main}|}| just include the main file
at the top of each child file:
%
\begin{center}
|\input{|\textit{main}|}|
\end{center}
%
A simple redirection |\childdocforward{|\textit{dest}|}| is achieved by:
%
\begin{center}
|\def\jobname{|\textit{dest}|}\input{\jobname}|
\end{center}
%
The redirection with prefix
|\childdocforwardprefix[|\textit{prefix}|]{|\textit{dest}|}|
is accomplished by:
%
\begin{center}
\begin{tabular}{l}
|{\edef\jobname{\scantokens\expandafter{\jobname\noexpand}}|\\
|\def\redirectjob |\textit{prefix}|#1~~~{\gdef\jobname{|\textit{dest}|#1}}|\\
|\expandafter\redirectjob\jobname~~~}\input{\jobname}|
\end{tabular}
\end{center}

In an alternative approach,
child documents can be compiled by a specific command line
without additional code or specific definitions:
%
\begin{center}
|... -jobname "|\textit{target}|" "|[\textit{flags}]%
|\includeonly{|\textit{dest}|}\input{|\textit{main}|}"|
\end{center}
%

%%%%%%%%%%%%%%%%%%%%%%%%%%%%%%%%%%%%%%%%%%%%%%%%%%%%%%%%%%%%%%%%%%%%%%%%%%%%%%%%
%%%%%%%%%%%%%%%%%%%%%%%%%%%%%%%%%%%%%%%%%%%%%%%%%%%%%%%%%%%%%%%%%%%%%%%%%%%%%%%%
\section{Information}

%%%%%%%%%%%%%%%%%%%%%%%%%%%%%%%%%%%%%%%%%%%%%%%%%%%%%%%%%%%%%%%%%%%%%%%%%%%%%%%%
\subsection{Copyright}

Copyright \copyright{} 2017--2018 Niklas Beisert

This work may be distributed and/or modified under the
conditions of the \LaTeX{} Project Public License, either version 1.3
of this license or (at your option) any later version.
The latest version of this license is in
  \url{http://www.latex-project.org/lppl.txt}
and version 1.3 or later is part of all distributions of \LaTeX{}
version 2005/12/01 or later.

This work has the LPPL maintenance status `maintained'.

The Current Maintainer of this work is Niklas Beisert.

This work consists of the files |README.txt|, |childdoc.ins| and |childdoc.dtx|
as well as the derived files |childdoc.def|, |cdocsamp.tex|
with |cdocsch1.tex|, |cdocsch2.tex|, |cdocspt3.tex|, |cdocspt4.tex|,
|cdocsdrf.tex|, |cdocsfn1.tex|, |cdocsfn2.tex|
as well as |childdoc.pdf|.

%%%%%%%%%%%%%%%%%%%%%%%%%%%%%%%%%%%%%%%%%%%%%%%%%%%%%%%%%%%%%%%%%%%%%%%%%%%%%%%%
\subsection{Files and Installation}

The package consists of the files:
%
\begin{center}
\begin{tabular}{ll}
    |README.txt|   & readme file \\
    |childdoc.ins| & installation file \\
    |childdoc.dtx| & source file \\
    |childdoc.def| & definition file \\
    |cdocsamp.tex| & sample main file \\
    |cdocsch1.tex| & sample include file \\
    |cdocsch2.tex| & sample include file \\
    |cdocspt3.tex| & sample part file \\
    |cdocspt4.tex| & sample part file \\
    |cdocsdrf.tex| & sample redirection file \\
    |cdocsfn1.tex| & sample redirection file \\
    |cdocsfn2.tex| & sample redirection file \\
    |childdoc.pdf| & manual
\end{tabular}
\end{center}
%
The distribution consists of the files
|README.txt|, |childdoc.ins| and |childdoc.dtx|.
%
\begin{itemize}
\item
Run (pdf)\LaTeX{} on |childdoc.dtx|
to compile the manual |childdoc.pdf| (this file).
\item
Run \LaTeX{} on |childdoc.ins| to create the definitions file |childdoc.def|
and the sample |cdocsamp.tex| with include files
|cdocsch1.tex|, |cdocsch2.tex|, |cdocspt3.tex|, |cdocspt4.tex|,
|cdocsdrf.tex|, |cdocsfn1.tex|, |cdocsfn2.tex|.
Then copy the file |childdoc.def| to an appropriate directory of your \LaTeX{}
distribution, e.g.\ \textit{texmf-root}|/tex/latex/childdoc|.
\end{itemize}

%%%%%%%%%%%%%%%%%%%%%%%%%%%%%%%%%%%%%%%%%%%%%%%%%%%%%%%%%%%%%%%%%%%%%%%%%%%%%%%%
\subsection{Related CTAN Packages}

There are several other packages which offer a similar functionality:
%
\begin{itemize}
\item
The packages
\href{http://ctan.org/pkg/docmute}{\textsf{docmute}},
\href{http://ctan.org/pkg/includex}{\textsf{includex}} and
\href{http://ctan.org/pkg/standalone}{\textsf{standalone}}
provide commands to include only the document body of
a child file thus allowing both files to be compiled individually.
\item
The packages \href{http://ctan.org/pkg/subdocs}{\textsf{subdocs}}
and \href{http://ctan.org/pkg/subfiles}{\textsf{subfiles}}
provide structures in which the main and child documents can be
encapsulated and allowing them to be compiled individually.
The inclusion mechanism is different from the conventional |\include|.
\item
The package \href{http://ctan.org/pkg/combine}{\textsf{combine}}
is an elaborate solution to combine several documents into one.
\end{itemize}
%
See also the CTAN topic \href{http://ctan.org/topic/subdocs}{\textsf{subdocs}}
for further related packages.
The present package differs from the above solutions in that
a document structure constructed with the conventional |\include| mechanism
just needs two extra commands at the top of every file
such that all constituent files can be compiled individually.

%%%%%%%%%%%%%%%%%%%%%%%%%%%%%%%%%%%%%%%%%%%%%%%%%%%%%%%%%%%%%%%%%%%%%%%%%%%%%%%%
%\subsection{Feature Suggestions}
%
%The following is a list of features which may be useful for future
%versions of this package:
%%
%\begin{itemize}
%\item
%\ldots
%\end{itemize}

%%%%%%%%%%%%%%%%%%%%%%%%%%%%%%%%%%%%%%%%%%%%%%%%%%%%%%%%%%%%%%%%%%%%%%%%%%%%%%%%
\subsection{Revision History}

%%%%%%%%%%%%%%%%%%%%%%%%%%%%%%%%%%%%%%%%
\paragraph{v2.0:} 2018/12/30

\begin{itemize}
\item
immediate forward processing
\item
added |\childdocby| mechanism
\item
manual restructured
\end{itemize}

%%%%%%%%%%%%%%%%%%%%%%%%%%%%%%%%%%%%%%%%
\paragraph{v1.6:} 2018/01/17

\begin{itemize}
\item
application for development of include files
\item
corrections to manual
\end{itemize}

%%%%%%%%%%%%%%%%%%%%%%%%%%%%%%%%%%%%%%%%
\paragraph{v1.5:} 2017/05/21

\begin{itemize}
\item
more complete structuring introduced
\item
|\childdocof| introduced
\item
|\childdoc| renamed to |\childdocmain|
\item
|\childredirect| renamed to |\childdocforward| and |\childdocforwardprefix|
and functionality expanded
\end{itemize}

%%%%%%%%%%%%%%%%%%%%%%%%%%%%%%%%%%%%%%%%
\paragraph{v1.0:} 2017/04/27

\begin{itemize}
\item
manual and install package
\item
first version published on CTAN
\end{itemize}

%%%%%%%%%%%%%%%%%%%%%%%%%%%%%%%%%%%%%%%%
\paragraph{v0.6:} 2017/04/26

\begin{itemize}
\item
redirection mechanism added
\end{itemize}

%%%%%%%%%%%%%%%%%%%%%%%%%%%%%%%%%%%%%%%%
\paragraph{v0.5:} 2017/04/26

\begin{itemize}
\item
functionality in definition file
\end{itemize}


%%%%%%%%%%%%%%%%%%%%%%%%%%%%%%%%%%%%%%%%%%%%%%%%%%%%%%%%%%%%%%%%%%%%%%%%%%%%%%%%
%%%%%%%%%%%%%%%%%%%%%%%%%%%%%%%%%%%%%%%%%%%%%%%%%%%%%%%%%%%%%%%%%%%%%%%%%%%%%%%%
%%%%%%%%%%%%%%%%%%%%%%%%%%%%%%%%%%%%%%%%%%%%%%%%%%%%%%%%%%%%%%%%%%%%%%%%%%%%%%%%
\appendix

\settowidth\MacroIndent{\rmfamily\scriptsize 000\ }

 \DocInput{childdoc.dtx}

\end{document}
%</driver>
% \fi
%
% %%%%%%%%%%%%%%%%%%%%%%%%%%%%%%%%%%%%%%%%%%%%%%%%%%%%%%%%%%%%%%%%%%%%%%%%%%%%%%
% %%%%%%%%%%%%%%%%%%%%%%%%%%%%%%%%%%%%%%%%%%%%%%%%%%%%%%%%%%%%%%%%%%%%%%%%%%%%%%
% \section{Sample}
%\iffalse
%<*samplemain>
%\fi
%
% The following presents a sample document
% with two chapters, two parts, a title page,
% a compile flag as well as three forwarding files to set the flag.
% It consists of eight |.tex| files:
% \begin{center}
% \begin{tabular}{ll}
% |cdocsamp.tex|&main file\\
% |cdocsch1.tex|&include file for chapter 1\\
% |cdocsch2.tex|&include file for chapter 2\\
% |cdocspt3.tex|&include file for part 3\\
% |cdocspt4.tex|&include file for part 4\\
% |cdocsdrf.tex|&forwarding file for main file in draft mode\\
% |cdocsfi1.tex|&forwarding file for final version of chapter 1\\
% |cdocsfi2.tex|&forwarding file for final version of chapter 2\\
% \end{tabular}
% \end{center}
% Each of the eight files can be compiled directly by the \LaTeX{} compiler.
%
% %%%%%%%%%%%%%%%%%%%%%%%%%%%%%%%%%%%%%%
% \paragraph{Main File.}
%
% The main file is called |cdocsamp.tex|.
%
% Load the \textsf{childdoc} definitions and
% declare the filename for the main document:
%    \begin{macrocode}
\input{childdoc.def}
\childdocmain{}
%    \end{macrocode}

% Optional override for |\version| flag:
%    \begin{macrocode}
%%\ifchilddoc\else\providecommand{\version}{draft}\fi
%    \end{macrocode}

% Define the default values for the |\version| flag
% (|final| for the main file and |draft| for childs):
%    \begin{macrocode}
\ifchilddoc
\providecommand{\version}{draft}
\else
\providecommand{\version}{final}
\fi
%    \end{macrocode}

% Load the standard document class:
%    \begin{macrocode}
\documentclass[12pt]{article}
%    \end{macrocode}

% Start the document body:
%    \begin{macrocode}
\begin{document}
%    \end{macrocode}

% Declare a title page.
% Print title, part of document being processed and version flag:
%    \begin{macrocode}
\addtocounter{page}{-1}
\begin{center}
{\LARGE\bfseries{}childdoc example\par}
\vspace{1cm}
\ifchilddoc
\ifchilddocmanual part\else chapter\fi:
`\childdocname' of `\childdocjob'\par
\else
main document: `\childdocjob'\par
\fi
version: \version\par
\end{center}
\newpage
%    \end{macrocode}

% Manually include selected file,
% otherwise process as usual:
%    \begin{macrocode}
\ifchilddocmanual
\section*{part `\childdocname'}
\input{\childdocname}
\else
%    \end{macrocode}

% Include the two chapters:
%    \begin{macrocode}
\include{cdocsch1}
\include{cdocsch2}
%    \end{macrocode}

% Include the two parts unless only chapters should be displayed:
%    \begin{macrocode}
\ifchilddoc\else
\section{part three}
\input{cdocspt3}
\section{part four}
\input{cdocspt4}
\fi
%    \end{macrocode}

% Process as usual until here:
%    \begin{macrocode}
\fi
%    \end{macrocode}

% End of document body:
%    \begin{macrocode}
\end{document}
%    \end{macrocode}
%\iffalse
%</samplemain>
%\fi
%
% %%%%%%%%%%%%%%%%%%%%%%%%%%%%%%%%%%%%%%
% \paragraph{Chapter Include Files.}
%
% The include files are called |cdocsch1.tex| and |cdocsch2.tex|.
%
%\iffalse
%<*samplechap1|samplechap2>
%\fi

% Optional override for |\version| flag:
%    \begin{macrocode}
%%\providecommand{\version}{final}
%    \end{macrocode}

% Include the main document:
%    \begin{macrocode}
\input{childdoc.def}
\childdocof{cdocsamp}
%    \end{macrocode}

%\iffalse
%</samplechap1|samplechap2>
%\fi
%
%\iffalse
%<*samplechap1>
%\fi
% Some text for chapter 1:
%    \begin{macrocode}
\section{one}
some text in chapter one
%    \end{macrocode}

%\iffalse
%</samplechap1>
%\fi
% Some text for chapter 2:
%\iffalse
%<*samplechap2>
%\fi
%    \begin{macrocode}
\section{two}
more text in chapter two
%    \end{macrocode}

%\iffalse
%</samplechap2>
%\fi
%
% %%%%%%%%%%%%%%%%%%%%%%%%%%%%%%%%%%%%%%
% \paragraph{Part Include Files.}
%
% The include files are called |cdocspt3.tex| and |cdocspt4.tex|.
%
%\iffalse
%<*samplepart3|samplepart4>
%\fi

% Optional override for |\version| flag:
%    \begin{macrocode}
%%\providecommand{\version}{final}
%    \end{macrocode}

% Include the main document:
%    \begin{macrocode}
\input{childdoc.def}
\childdocby{cdocsamp}
%    \end{macrocode}

%\iffalse
%</samplepart3|samplepart4>
%\fi
%
%\iffalse
%<*samplepart3>
%\fi
% Some text for part 3:
%    \begin{macrocode}
some text in part three
%    \end{macrocode}

%\iffalse
%</samplepart3>
%\fi
% Some text for part 4:
%\iffalse
%<*samplepart4>
%\fi
%    \begin{macrocode}
more text in part four
%    \end{macrocode}

%\iffalse
%</samplepart4>
%\fi
%
% %%%%%%%%%%%%%%%%%%%%%%%%%%%%%%%%%%%%%%
% \paragraph{Forwarding for a Complete Draft.}
%
% The following forwarding file |cdocsdrf.tex|
% compiles the main document in draft mode:
%\iffalse
%<*sampledraft>
%\fi
%    \begin{macrocode}
\def\version{draft}
\input{childdoc.def}
\childdocforward{cdocsamp}
%    \end{macrocode}

%\iffalse
%</sampledraft>
%\fi
%
% %%%%%%%%%%%%%%%%%%%%%%%%%%%%%%%%%%%%%%
% \paragraph{Forwarding for Final Version of the Chapters.}
%
% The following forwarding files |cdocsfn1.tex| and |cdocsfn2.tex|
% (with identical content)
% compile the final versions of the child documents
% |cdocsch1.tex| and |cdocsch2.tex|, respectively:
%\iffalse
%<*samplefinal>
%\fi
%    \begin{macrocode}
\def\version{final}
\input{childdoc.def}
\childdocforwardprefix[cdocsamp]{cdocsfn}{cdocsch}
%    \end{macrocode}

%\iffalse
%</samplefinal>
%\fi
%
% %%%%%%%%%%%%%%%%%%%%%%%%%%%%%%%%%%%%%%
% \paragraph{Command Line Processing.}
%
% The following three command lines generate the output files
% |cdocscld|, |cdocscl1| and |cdocscl2|
% which should be identical to
% |cdocsdrf|, |cdocsch1| and |cdocsfn2|, respectively:
% \begin{center}
% \begin{tabular}{l}
% |latex -jobname cdocscld \|\\
% |  "\def\version{draft}\input{childdoc.def}\childdocforward{cdocsamp}"|\\
% |latex -jobname cdocscl1 \|\\
% |  "\input{childdoc.def}\childdocforward[cdocsamp]{cdocsch1}"|\\
% |latex -jobname cdocscl2 \|\\
% |  "\def\version{final}\input{childdoc.def}\childdocforward{cdocsch2}"|
% \end{tabular}
% \end{center}
% Note that the trailing backslash on each first line
% merely continues the input to the second line
% (for convenient cut ant paste).
% Furthermore, the command |latex| can be replaced by any
% of its alternative versions such as |pdflatex|.
%
% %%%%%%%%%%%%%%%%%%%%%%%%%%%%%%%%%%%%%%%%%%%%%%%%%%%%%%%%%%%%%%%%%%%%%%%%%%%%%%
% %%%%%%%%%%%%%%%%%%%%%%%%%%%%%%%%%%%%%%%%%%%%%%%%%%%%%%%%%%%%%%%%%%%%%%%%%%%%%%
% \section{Implementation}
%\iffalse
%<*package>
%\fi
%
% This section describes the definitions file |childdoc.def|.

% The definitions cannot be loaded using |\usepackage| or |\RequirePackage|
% which has a mechanism to prevent loading a style file more than once.
% When loading the definitions by means of |\input|
% multiple instances have to be prevented manually:
%\iffalse
%This code needs to be before the `\ProvidesFile' directive
%which is defined at the beginning of this file.
%Therefore it is also placed there and commented out here.
%</package>
%<*discard>
%\fi
%    \begin{macrocode}
\ifdefined\childdocmain\endinput\fi
%    \end{macrocode}
%\iffalse
%</discard>
%<*package>
%\fi
%
% \macro{\ifchilddoc}
% \macro{\ifchilddocmanual}
% The conditional |\ifchilddoc| tells whether a
% child (true) or main (false) document is being compiled.
% The conditional |\ifchilddocmanual| tells whether
% the |\includeonly| mechanism is used (false) or
% the selection of child files must be performed manually (true).
% The definitions initialise to false:
%    \begin{macrocode}
\newif\ifchilddoc
\newif\ifchilddocmanual
%    \end{macrocode}

% \macro{\childdocname}
% \macro{\childdocjob}
% The macro |\childdocname| stores the name of the main document
% to be compiled. The macro |\childdocjob| stores the name of
% the document on which the \LaTeX{} compiler was originally invoked.
% The content of |\jobname| cannot be compared
% to filenames specified in the source due to different catcodes.
% The following code rescans |\jobname|, stores the result
% in |\childdocname| and saves a copy in |\childdocjob|:
%    \begin{macrocode}
\edef\childdocname{\scantokens\expandafter{\jobname\noexpand}}
\let\childdocjob\childdocname
%    \end{macrocode}

% \macro{\childdocdisable}
% The macro |\childdocdisable| prevents the main file
% from being processed more than once.
% At this stage, the main document command |\childdocmain|
% is assumed to be called once again where it should do nothing.
% Any subsequent call to it should prevent
% a secondary processing of the main document
% It overwrites the forwarding commands
% |\childdocof| and |\childdocforward|
% with empty macros to prevent further inclusions of the main document:
%    \begin{macrocode}
\newcommand{\childdocdisable}
{
  \renewcommand{\childdocmain}[1]{\renewcommand{\childdocmain}[1]{\endinput}}
  \renewcommand{\childdocof}[1]{}
  \renewcommand{\childdocby}[2][]{}
  \renewcommand{\childdocforward}[2][]{}
  \renewcommand{\childdocdisable}{}
}
%    \end{macrocode}

% \macro{\childdocmain}
% The macro |\childdocmain| is to be called at the top of the main file
% with nothing or the main filename (without extension) as argument.
% First, it breaks loops.
% If the argument is not empty and does not match |\childdocname|
% (which is set by the first inclusion of |childdoc.def|),
% |\ifchilddoc| is set to true, |\includeonly| is applied to the child file
% and |\jobname| is set to the main file
% (for proper handling of |.aux| files):
%    \begin{macrocode}
\newcommand{\childdocmain}[1]
{
  \childdocdisable\childdocmain{}
  \if?#1?\else
    \begingroup
      \def\childdoctmp{#1}
      \ifx\childdoctmp\childdocname
        \def\childdoctmp{}
      \else
        \def\childdoctmp
        {
          \childdoctrue
          \includeonly{\childdocname}
          \def\childdocjob{#1}
          \def\jobname{#1}
        }
      \fi
      \expandafter
    \endgroup
    \childdoctmp
  \fi
}
%    \end{macrocode}

% \macro{\childdocof}
% The command |\childdocof| redirects
% compilation to the main file |#1|.
%    \begin{macrocode}
\newcommand{\childdocof}[1]
{
  \childdocdisable
  \childdoctrue
  \includeonly{\childdocname}
  \def\jobname{#1}
  \def\childdocjob{#1}
  \input{#1}
}
%    \end{macrocode}

% \macro{\childdocby}
% The command |\childdocby| ....
%    \begin{macrocode}
\newcommand{\childdocby}[2][]
{
  \childdocdisable
  \childdoctrue
  \childdocmanualtrue
  \if?#1?\else
    \def\jobname{#2}
  \fi
  \def\childdocjob{#2}
  \input{#2}
  \endinput
}
%    \end{macrocode}

% \macro{\childdocforward}
% The command |\childdocforward| redirects
% compilation to the main file or
% (if the optional argument is given) a child file.
% Parameters are set as if the main file
% or a child file starting with |\childdocof| was compiled.
% Then compilation is handed over to the main file:
%    \begin{macrocode}
\newcommand{\childdocforward}[2][]
{
  \begingroup
    \if?#1?
      \def\childdoctmp
      {
        \def\childdocname{#2}
        \def\childdocjob{#2}
        \def\jobname{#2}
        \input{#2}
        \endinput
      }
    \else
      \def\childdoctmp
      {
        \childdocdisable
        \def\childdocname{#2}
        \childdoctrue
        \includeonly{#2}
        \def\childdocjob{#1}
        \def\jobname{#1}
        \input{#1}
        \endinput
      }
    \fi
    \expandafter
  \endgroup
  \childdoctmp
}
%    \end{macrocode}

% \macro{\childdocforwardprefix}
% The command |\childdocforwardprefix| redirects
% compilation to the main or a child file by means of a pattern.
% The prefix |#1| in the current filename is replaced by |#2|
% and the suffix of the current filename is kept
% (it is assumed that the filename does not contain the substring `|~~~|'
% which is used as a delimiter).
% Compilation is handed over to the new file by |\childdocforward|:
%    \begin{macrocode}
\newcommand{\childdocforwardprefix}[3][]
{
  \begingroup
    \def\childdocextract #2##1~~~{\def\childdoctmp{\childdocforward[#1]{#3##1}}}
    \expandafter\childdocextract\childdocname~~~
    \expandafter
  \endgroup
  \childdoctmp
}
%    \end{macrocode}

% \macro{\childdoc}
% The deprecated macro |\childdoc| is a legacy version of |\childdocmain|:
%    \begin{macrocode}
\newcommand{\childdoc}{\childdocmain}
%    \end{macrocode}

% \macro{\childdocredirect}
% The deprecated macro |\childdocredirect| is a legacy version
% of |\childdocforward| and |\childdocforwardprefix|:
%    \begin{macrocode}
\newcommand{\childdocredirect}[2][]
{
  \begingroup
    \if?#1?
      \def\childdoctmp{\childdocforward{#2}}
    \else
      \def\childdoctmp{\childdocforwardprefix{#1}{#2}}
    \fi
    \expandafter
  \endgroup
  \childdoctmp
}
%    \end{macrocode}

%\iffalse
%</package>
%\fi
%
\endinput
|\\
|\childdocby{|\textit{main}|}|\\
\end{tabular}
\end{center}
%
The directive |\childdocby| is similar to |\childdocof|
described in \secref{sec:include},
but the subsequent selection of content must be done manually.
To that end, both |\ifchilddoc| and |\ifchilddocmanual|
will be true upon processing of a part,
and the name of the part is stored in |\childdocname|.
Note that |\jobname| will be set to the filename of the current part
so that each part receives an individual |.aux| file
that does not interfere with the |.aux| file(s) of the main document.
This behaviour can be altered by the alternative form
|\childdocby[*]{|\textit{main}|}| (with a non-empty optional argument)
which uses the |.aux| file of the main document
by setting |\jobname| to \textit{main}.

%%%%%%%%%%%%%%%%%%%%%%%%%%%%%%%%%%%%%%%%%%%%%%%%%%%%%%%%%%%%%%%%%%%%%%%%%%%%%%%%
\subsection{Driver Development}
\label{sec:driver}

The \textsf{childdoc} mechanism can also be use for the development
of definition files such as \LaTeX{} styles or classes.
This case differs from the above setup with multiple parts
included by |\include| in that no |\includeonly| should be invoked.
This can be achieved by starting the include file
(before |\ProvidesPackage|) with:
%
\begin{center}
\begin{tabular}{l}
|% \iffalse
%
% childdoc.dtx Copyright (C) 2017-2018 Niklas Beisert
%
% This work may be distributed and/or modified under the
% conditions of the LaTeX Project Public License, either version 1.3
% of this license or (at your option) any later version.
% The latest version of this license is in
%   http://www.latex-project.org/lppl.txt
% and version 1.3 or later is part of all distributions of LaTeX
% version 2005/12/01 or later.
%
% This work has the LPPL maintenance status `maintained'.
%
% The Current Maintainer of this work is Niklas Beisert.
%
% This work consists of the files childdoc.dtx and childdoc.ins
% and the derived files childdoc.def and cdocsamp.tex with
% cdocsch1.tex, cdocsch2.tex, cdocsdrf.tex, cdocsfn1.tex, cdocsfn2.tex.
%
%<package>\ifdefined\childdocmain\endinput\fi
%<package>\ProvidesFile{childdoc.def}[2018/12/30 v2.0 child document driver]
%<samplemain>\ProvidesFile{cdocsamp.tex}[2018/12/30 v2.0 sample for childdoc]
%<*driver>
%\ProvidesFile{childdoc.drv}[2018/12/30 v2.0 childdoc reference manual file]
\PassOptionsToClass{10pt,a4paper}{article}
\documentclass{ltxdoc}

\usepackage[margin=35mm]{geometry}
\usepackage{hyperref}
\usepackage{hyperxmp}
\usepackage[usenames]{color}

\hypersetup{colorlinks=true}
\hypersetup{pdfstartview=FitH}
\hypersetup{pdfpagemode=UseNone}
\hypersetup{pdfsource={}}
\hypersetup{pdflang={en-UK}}
\hypersetup{pdfcopyright={Copyright 2017-2018 Niklas Beisert.
  This work may be distributed and/or modified under the
  conditions of the LaTeX Project Public License, either version 1.3
  of this license or (at your option) any later version.}}
\hypersetup{pdflicenseurl={http://www.latex-project.org/lppl.txt}}
\hypersetup{pdfcontactaddress={ETH Zurich, ITP, HIT K,
  Wolfgang-Pauli-Strasse 27}}
\hypersetup{pdfcontactpostcode={8093}}
\hypersetup{pdfcontactcity={Zurich}}
\hypersetup{pdfcontactcountry={Switzerland}}
\hypersetup{pdfcontactemail={nbeisert@itp.phys.ethz.ch}}
\hypersetup{pdfcontacturl={http://people.phys.ethz.ch/\xmptilde nbeisert/}}

\newcommand{\secref}[1]{\hyperref[#1]{section \ref*{#1}}}

\parskip1ex
\parindent0pt
\let\olditemize\itemize
\def\itemize{\olditemize\parskip0pt}

\begin{document}

\title{The \textsf{childdoc} Package}
\hypersetup{pdftitle={The childdoc Package}}
\author{Niklas Beisert\\[2ex]
  Institut f\"ur Theoretische Physik\\
  Eidgen\"ossische Technische Hochschule Z\"urich\\
  Wolfgang-Pauli-Strasse 27, 8093 Z\"urich, Switzerland\\[1ex]
  \href{mailto:nbeisert@itp.phys.ethz.ch}
  {\texttt{nbeisert@itp.phys.ethz.ch}}}
\hypersetup{pdfauthor={Niklas Beisert}}
\hypersetup{pdfsubject={Manual for the LaTeX2e Package childdoc}}
\date{30 December 2018, \textsf{v2.0}}
\maketitle

\begin{abstract}\noindent
\textsf{childdoc} is a \LaTeXe{} package
that enables the direct compilation
of document sections included by |\include|
to individual files.
\end{abstract}

\begingroup
\parskip0ex
\tableofcontents
\endgroup

%%%%%%%%%%%%%%%%%%%%%%%%%%%%%%%%%%%%%%%%%%%%%%%%%%%%%%%%%%%%%%%%%%%%%%%%%%%%%%%%
%%%%%%%%%%%%%%%%%%%%%%%%%%%%%%%%%%%%%%%%%%%%%%%%%%%%%%%%%%%%%%%%%%%%%%%%%%%%%%%%
\section{Introduction}

\LaTeX{} provides a mechanism to structure a large document (such as a book)
into a main file and several child files (containing the chapters)
using the |\include| command.
This mechanism is beneficial for documents
which span hundreds of pages in order to
make the source file(s) more manageable.
Moreover, compilation can be restricted to
selected child files by means of the |\includeonly| command.
The latter feature can be used to reduce the compilation time while editing
(this was significantly more useful in the earlier days of \LaTeX{})
or to generate a smaller document which is easier to navigate.
Another application of |\includeonly| is to generate
documents consisting of selected parts of the complete document.

However, there are a few drawbacks of the plain |\include| mechanism:
\begin{itemize}
\item
The child files cannot be compiled on their own,
they can only be compiled via the main file.
A naive editing environment
(such as a text editor with an option
to have the current file processed by \LaTeX)
may require one to switch to the main file before compiling;
attempting to compile the child file produces errors.
\item
The main file must be modified (each time)
to adjust the |\includeonly| command
to the present needs. This easily leaves the main file in a messy state.
\item
The generated document will always carry the filename
of the main document. This is inconvenient if
several child files are to be compiled and
to be kept for distribution.
\end{itemize}

The present package provides a simple interface
to make child files individually compilable by \LaTeX{}.
Compiling a child file then has the same effect as compiling
the main file with an |\includeonly| command
to select the appropriate child.
Moreover the generated document will carry the name of the child
rather than the main file.
This resolves all three above issues.

This feature is meant to make the editing of books,
thesis documents and lecture notes somewhat more convenient.
However, the package can also be used efficiently for
composing a series of documents (such as exercise sheets)
which are typically distributed individually.
It then assists the author in generating the individual documents
(potentially in different versions)
as well as a document containing the collected series.
Another application is in developing style files
or other kinds of included material
where compilation of the style file could redirect
to a sample or test file.

%%%%%%%%%%%%%%%%%%%%%%%%%%%%%%%%%%%%%%%%%%%%%%%%%%%%%%%%%%%%%%%%%%%%%%%%%%%%%%%%
%%%%%%%%%%%%%%%%%%%%%%%%%%%%%%%%%%%%%%%%%%%%%%%%%%%%%%%%%%%%%%%%%%%%%%%%%%%%%%%%
\section{Usage}

First of all, the package \textsf{childdoc} is \emph{not} a standard
\LaTeXe{} |.sty| style file! Therefore it needs to be invoked in
a non-standard way.

%%%%%%%%%%%%%%%%%%%%%%%%%%%%%%%%%%%%%%%%%%%%%%%%%%%%%%%%%%%%%%%%%%%%%%%%%%%%%%%%
\subsection{Included Files}
\label{sec:include}

%%%%%%%%%%%%%%%%%%%%%%%%%%%%%%%%%%%%%%%%
\DescribeMacro{\childdocmain}
To use the package, add the commands
\begin{center}
\begin{tabular}{l}
|\input{childdoc.def}|\\
|\childdocmain{}|\\
\end{tabular}
\end{center}
at the very top of the main \LaTeX{} file,
in particular \emph{before} the |\documentclass| statement!
The argument of |\childdocmain| should be left empty
(but it must be present).

%%%%%%%%%%%%%%%%%%%%%%%%%%%%%%%%%%%%%%%%
\DescribeMacro{\childdocof}
Furthermore, add the commands
\begin{center}
\begin{tabular}{l}
|\input{childdoc.def}|\\
|\childdocof{|\textit{main}|}|\\
\end{tabular}
\end{center}
at the top of every child file \textit{child}
which is included by |\include{|\textit{child}|}|
from within the main file
(or at least for those files to be compiled individually).
The argument \textit{main} must be the filename of the main file.

There are a couple of
considerations in setting up the main and child documents:

%%%%%%%%%%%%%%%%%%%%%%%%%%%%%%%%%%%%%%%%
\paragraph{Restrictions.}

Please note the following restrictions:
\begin{itemize}
\item
|\childdocmain| must be called with one argument \textit{main}
to ensure compatibility with earlier version of the package.
It must either be empty (|\childdocmain{}|)
or precisely match the filename of the main file in which it is specified.
See \secref{sec:detection} for further information.
\item
The filename \textit{main} must be specified without the |.tex| extension.
\item
The filename \textit{main} is case sensitive
(even in case-insensitive file systems)
due to internal string comparison.
\item
The argument \textit{main} should be fully expanded, it cannot be a macro.
\item
Subdirectories and special characters should be avoided in filenames.
\item
The command |\childdocmain{|\textit{main}|}| must be followed by a whitespace.
It should not be followed immediately by another command
or by a comment mark `|%|'.
This is because the \TeX{} parser reads the token immediately following
the argument of |\childdocmain| and puts it
at the beginning of every child section;
however, a white\-space is ignored.
\end{itemize}

%%%%%%%%%%%%%%%%%%%%%%%%%%%%%%%%%%%%%%%%
\paragraph{Content of Main File.}

It is advisable to place all content in the child files included by |\include|.
Any output contained in the main file will appear in all child documents
unless suppressed manually;
it cannot be suppressed automatically by the |\includeonly| directive
and thus should normally be avoided.
A method to include some content in the main file
by means of conditional processing is described in \secref{sec:conditional}.

%%%%%%%%%%%%%%%%%%%%%%%%%%%%%%%%%%%%%%%%
\paragraph{Page Numbering.}

When only a part of the document is compiled,
the appropriate numbering of pages
(as well as other status parameters)
is determined from the |.aux| files.
The latter contain information from previous passes.
However this information needs to propagate through
all intermediate child documents.
Therefore the page numbering in child documents may well
be inconsistent until the complete document is compiled at least once.

A useful (if unconventional) way to always ensure a consistent
page numbering is to restart the numbering in each child document
and denote the pages by `\textit{child}|.|\textit{page}'
where \textit{child} represents the chapter/section number of the child file.
This can be achieved by the command
|\numberwithin{page}{|\textit{child}|}|
of the \textsf{amsmath} package
where \textit{child} can be |chapter| or |section|
depending on the chosen structuring.
Alternatively, one can modify the macro |\thepage| appropriately
and reset the counter |page| at the start of each child file.

%%%%%%%%%%%%%%%%%%%%%%%%%%%%%%%%%%%%%%%%%%%%%%%%%%%%%%%%%%%%%%%%%%%%%%%%%%%%%%%%
\subsection{Conditional Processing}
\label{sec:conditional}

The package provides a mechanism to compile different versions
of a document. To customise the versions further some conditional processing
can come in handy to distinguish which version is being compiled.
The package provides two macros to describe the compilation context:

%%%%%%%%%%%%%%%%%%%%%%%%%%%%%%%%%%%%%%%%
\DescribeMacro{\ifchilddoc}
The conditional |\ifchilddoc| distinguishes between the compilation of
child documents and the main document:
%
\begin{center}
|\ifchilddoc |\textit{child-code}| |[|\||else |\textit{main-code}]| \||fi|
\end{center}

%%%%%%%%%%%%%%%%%%%%%%%%%%%%%%%%%%%%%%%%
\DescribeMacro{\childdocname}
\DescribeMacro{\childdocjob}
The macro |\childdocname| contains the filename (without extension)
of the main or child file being processed.
Note that |\childdocjob| will always contain the name of the main file.

%%%%%%%%%%%%%%%%%%%%%%%%%%%%%%%%%%%%%%%%
\paragraph{Title Page.}

Conditional processing can be used to include a title or banner page
in the main document when proper precautions are taken.
Importantly, the code in the main file should ensure that the page counter
(as well as other status parameters which are stored in the |.aux| files)
takes the same value after the conditional processing.
Otherwise the page numbers may take divergent values
depending on which part is compiled.

For example, a title page could be declared by:
%
\begin{center}
\begin{tabular}{l}
|\ifchilddoc\||else|\\
|\addtocounter{page}{-1}|\\
\textit{code for title page}\\
|\newpage|\\
|\||fi|
\end{tabular}
\end{center}
%
A banner page for the child documents can be generated by:
%
\begin{center}
\begin{tabular}{l}
|\ifchilddoc|\\
|\addtocounter{page}{-1}|\\
\textit{code for banner page}\\
|\newpage|\\
|\||fi|
\end{tabular}
\end{center}
%
Here one could write a message such as:
\begin{center}
|This is the part \childdocname{} of \childdocjob{}.|
\end{center}

%%%%%%%%%%%%%%%%%%%%%%%%%%%%%%%%%%%%%%%%%%%%%%%%%%%%%%%%%%%%%%%%%%%%%%%%%%%%%%%%
\subsection{Flags}
\label{sec:flags}

The package makes it easy to generate different versions
of the main or child documents.
To this end compilation flags can be defined
and assigned different default values.
They will be particularly useful in conjunction
with the forwarding mechanism described in \secref{sec:forward}.

For example, it may be useful to have a flag |\version|
which can be set to |draft| or |final|.
The document source will contain some conditional code
depending on the value of |\version|.
Suppose further, the flag should default to |final| for the main file
and to |draft| for child files
which is a natural assignment for editing the document.
This is achieved by placing the following code
in the preamble of the main document
(below the |\childdocmain| directive):
%
\begin{center}
\begin{tabular}{l}
|\ifchilddoc|\\
|\providecommand{\version}{draft}|\\
|\||else|\\
|\providecommand{\version}{final}|\\
|\||fi|
\end{tabular}
\end{center}
%
The definition by |\providecommand| makes sure
that previous definitions are not overwritten.
Further statements |\providecommand{\version}{...}|
can thus be added before the above code to override it.

For the main file, one might add a line
(between |\childdocmain| and the above block)
%
\begin{center}
|%\ifchilddoc\||else\providecommand{\version}{draft}\||fi|
\end{center}
%
which can be uncommented to produce a draft version.
Likewise one can add a line to the very top of a child file
(above the |\childdocof{|\textit{main}|}| directive)
%
\begin{center}
|%\providecommand{\version}{final}|
\end{center}
%
which can be uncommented to produce the final version of this child document.

%%%%%%%%%%%%%%%%%%%%%%%%%%%%%%%%%%%%%%%%%%%%%%%%%%%%%%%%%%%%%%%%%%%%%%%%%%%%%%%%
\subsection{Forwarding}
\label{sec:forward}

Different versions of the main or child documents
using compilation flags as described in \secref{sec:flags}
can be (permanently) stored in different files
for convenient compilation, viewing and distribution.
To this end, the package defines a command
to pass on compilation to a different file:

%%%%%%%%%%%%%%%%%%%%%%%%%%%%%%%%%%%%%%%%
\DescribeMacro{\childdocforward}
The command |\childdocforward| redirects processing to
another source file:
%
\begin{center}
\begin{tabular}{l}
|\input{childdoc.def}|\\
|\childdocforward[|\textit{main}|]{|\textit{dest}|}|\\
\end{tabular}
\end{center}
%
The argument \textit{dest} is the destination file
(without extension).
It should be the main file or one of the child files.
Note that further \textsf{childdoc} directives
such as |\childdocof| and |\childdocforward|
in the indicated file will be processed in this form.
The optional argument \textit{main}
passes on directly to the main file \textit{main}
while pretending to compile the child \textit{dest}.
This form behaves as if \textit{dest}
issues |\childdocof{|\textit{main}|}| right away,
and no further \textsf{childdoc} directives will be processed.

%%%%%%%%%%%%%%%%%%%%%%%%%%%%%%%%%%%%%%%%
\DescribeMacro{\...prefix}
In the alternative form |\childdocforwardprefix|,
%
\begin{center}
\begin{tabular}{l}
|\input{childdoc.def}|\\
|\childdocforwardprefix[|\textit{main}|]{|\textit{prefix}|}{|\textit{dest}|}|
\end{tabular}
\end{center}
%
the destination file is determined by a pattern
depending on the current file:
To make this work, the current file must be called
`{\textit{prefix}\hspace{0.2em}\textit{suffix}}'
with \textit{prefix} matching precisely the argument.
Processing is then passed on to the file
`{\textit{dest}\hspace{0.2em}\textit{suffix}}'.
Surely, the same effect is achieved by
directly specifying the
argument `{\textit{dest}\hspace{0.2em}\textit{suffix}}'
in the first form.
However, that requires to set up a different file
for each child. With the alternative form of the command
all these files can have exactly the same content
which simplifies setting them up and maintaining them.

For example, the following file |draft.tex|
with a compilation flag |\version| as described in \secref{sec:flags}
compiles the main document as a draft:
%
\begin{center}
\begin{tabular}{l}
|\def\version{draft}|\\
|\input{childdoc.def}|\\
|\childdocforward{|\textit{main}|}|
\end{tabular}
\end{center}
%
Likewise, the following files |final|\textit{nn}|.tex|
compile the final version of the child document
|child|\textit{nn}|.tex|:
%
\begin{center}
\begin{tabular}{l}
|\def\version{final}|\\
|\input{childdoc.def}|\\
|\childdocforwardprefix{final}{child}|
\end{tabular}
\end{center}
%

Note that when several versions of a main file and/or of each child file
are to be generated, it may be convenient to set up a |Makefile| or
shell script to automatise the process.

%%%%%%%%%%%%%%%%%%%%%%%%%%%%%%%%%%%%%%%%%%%%%%%%%%%%%%%%%%%%%%%%%%%%%%%%%%%%%%%%
\subsection{Command Line Processing}
\label{sec:commandline}

The effect of redirection files can also be achieved by invoking
the \LaTeX{} compiler with a more elaborate command line.
Most conveniently this should be done as part
of a shell script or a |Makefile|.

When using \textsf{childdoc} in the main file, the following
command lines effectively perform a redirection
(note that depending on the shell being used,
backslashes may have to be doubled: `|\|' $\to$ `|\\|'):
%
\begin{center}
|... -jobname "|\textit{target}|" |\\|"|[\textit{flags}]%
|\input{childdoc.def}\childdocforward[|\textit{main}|]{|\textit{dest}|}"|
\end{center}
%
Here \textit{target} is the name of the output file,
\textit{main} is the name of the main file
and \textit{dest} is the name of the main or child file to be processed
(all filenames without extensions).
The optional argument \textit{main} can be omitted
if \textit{main} matches \textit{dest}.
Optionally, compilation \textit{flags} can be defined via |\def| commands.
This command line makes the \TeX{} engine believe
it is compiling the file \textit{target}
whose content is specified as the latter parameter.
The provided code then forwards the processing to
\textit{main} or \textit{dest} as described in \secref{sec:forward}.

%%%%%%%%%%%%%%%%%%%%%%%%%%%%%%%%%%%%%%%%%%%%%%%%%%%%%%%%%%%%%%%%%%%%%%%%%%%%%%%%
\subsection{Include by Input}
\label{sec:input}

Including child documents by |\include| has some restrictions by design.
Most notably, the content of a child document always occupies
its own set of pages; pages cannot be shared between child documents.
Usually, this behaviour makes perfect sense
because each child document contain an essential part of the document.
However, in some situations it may be desirable to compose
a document from a collection of parts
without having mandatory page breaks between then.
For this case, the package
provides a mechanism to include parts
by |\input| which can also be processed individually.
However, by construction this mechanism
requires manual handling of the content to be output.

%%%%%%%%%%%%%%%%%%%%%%%%%%%%%%%%%%%%%%%%
\DescribeMacro{\ifchilddocmanual}
The main file should be prepared as usual, see \secref{sec:include}.
However, the document body must make a distinction
between processing of an individual part and of the main document, e.g.:
%
\begin{center}
\begin{tabular}{l}
|\ifchilddocmanual|\\
|\input{\childdocname}|\\
|\||else|\\
\textit{document body with }|\input{|\textit{part}|}|\\
|\||fi|
\end{tabular}
\end{center}
%
The conditional |\ifchilddocmanual| is true whenever
a part to be included by |\input| is being compiled,
and the name of the part is stored in |\childdocname|.

%%%%%%%%%%%%%%%%%%%%%%%%%%%%%%%%%%%%%%%%
\DescribeMacro{\childdocby}
Each part to be included by |\input| should start with:
%
\begin{center}
\begin{tabular}{l}
|\input{childdoc.def}|\\
|\childdocby{|\textit{main}|}|\\
\end{tabular}
\end{center}
%
The directive |\childdocby| is similar to |\childdocof|
described in \secref{sec:include},
but the subsequent selection of content must be done manually.
To that end, both |\ifchilddoc| and |\ifchilddocmanual|
will be true upon processing of a part,
and the name of the part is stored in |\childdocname|.
Note that |\jobname| will be set to the filename of the current part
so that each part receives an individual |.aux| file
that does not interfere with the |.aux| file(s) of the main document.
This behaviour can be altered by the alternative form
|\childdocby[*]{|\textit{main}|}| (with a non-empty optional argument)
which uses the |.aux| file of the main document
by setting |\jobname| to \textit{main}.

%%%%%%%%%%%%%%%%%%%%%%%%%%%%%%%%%%%%%%%%%%%%%%%%%%%%%%%%%%%%%%%%%%%%%%%%%%%%%%%%
\subsection{Driver Development}
\label{sec:driver}

The \textsf{childdoc} mechanism can also be use for the development
of definition files such as \LaTeX{} styles or classes.
This case differs from the above setup with multiple parts
included by |\include| in that no |\includeonly| should be invoked.
This can be achieved by starting the include file
(before |\ProvidesPackage|) with:
%
\begin{center}
\begin{tabular}{l}
|\input{childdoc.def}|\\
|\childdocforward{|\textit{main}|}|\\
\end{tabular}
\end{center}
%
or alternatively with:
%
\begin{center}
\begin{tabular}{l}
|\input{childdoc.def}|\\
|\childdocby{|\textit{main}|}|\\
\end{tabular}
\end{center}
%
Both forms have slightly different effects as described above.
The main file is prepared as usual, see \secref{sec:include}.

%%%%%%%%%%%%%%%%%%%%%%%%%%%%%%%%%%%%%%%%%%%%%%%%%%%%%%%%%%%%%%%%%%%%%%%%%%%%%%%%
\subsection{Legacy Detection}
\label{sec:detection}

The directive |\childdocmain| in the main file can detect
whether the complete document or merely a child is to be compiled
even without using the directive |\childdocof|.
This method is deprecated because it is less robust
and there is no compelling reason to use it;
it is merely provided for backward compatibility
and it may be removed in future versions.

If the detection mechanism is to be used,
it is mandatory to correctly specify
the filename of the main file as the argument of |\childdocmain|:
%
\begin{center}
\begin{tabular}{l}
|\input{childdoc.def}|\\
|\childdocmain{|\textit{main}|}|\\
\end{tabular}
\end{center}
%
If |\jobname| does not match the argument \textit{main} of |\childdocmain|,
it is assumed that |\jobname| points to the child file to be compiled.
When using |\childdocmain| with the main file specified as argument,
it suffices to start a child file
with just |\input{|\textit{main}|}|
without loading of the package and using |\childdocof|.
If instead all processing is done
with the appropriate \textsf{childdoc} directives,
the argument of \textit{main} of |\childdocmain| can be empty.

An alternative version of the command line processing described
in \secref{sec:commandline} using the detection mechanism reads:
%
\begin{center}
|... -jobname "|\textit{target}|" "|[\textit{flags}]%
[|\def\jobname{|\textit{dest}|}|]|\input{|\textit{main}|}"|
\end{center}

%%%%%%%%%%%%%%%%%%%%%%%%%%%%%%%%%%%%%%%%%%%%%%%%%%%%%%%%%%%%%%%%%%%%%%%%%%%%%%%%
\subsection{Manual Code}
\label{sec:manual}

In case one cannot be certain whether the definitions file |childdoc.def|
is installed on the target \TeX{} distribution
and one prefers not to ship it,
it is conceivable to paste a few relevant commands into the sources.

To that end, drop all statements |\input{childdoc.def}|
and perform the replacements as outlined below.
Instead of |\childdocmain{|\textit{main}|}| add the following code
to the top of the main file:
%
\begin{center}
\begin{tabular}{l}
|\||ifdefined\childdocname\endinput\||fi\newif\ifchilddoc|\\
|\edef\childdocname{\scantokens\expandafter{\jobname\noexpand}}|\\
|\def\childdocmain{|\textit{main}|}\||ifx\childdocmain\childdocname\||else|\\
|\childdoctrue\includeonly{\childdocname}\let\jobname\childdocmain\||fi|\\
\end{tabular}
\end{center}
%
Instead of |\childdocof{|\textit{main}|}| just include the main file
at the top of each child file:
%
\begin{center}
|\input{|\textit{main}|}|
\end{center}
%
A simple redirection |\childdocforward{|\textit{dest}|}| is achieved by:
%
\begin{center}
|\def\jobname{|\textit{dest}|}\input{\jobname}|
\end{center}
%
The redirection with prefix
|\childdocforwardprefix[|\textit{prefix}|]{|\textit{dest}|}|
is accomplished by:
%
\begin{center}
\begin{tabular}{l}
|{\edef\jobname{\scantokens\expandafter{\jobname\noexpand}}|\\
|\def\redirectjob |\textit{prefix}|#1~~~{\gdef\jobname{|\textit{dest}|#1}}|\\
|\expandafter\redirectjob\jobname~~~}\input{\jobname}|
\end{tabular}
\end{center}

In an alternative approach,
child documents can be compiled by a specific command line
without additional code or specific definitions:
%
\begin{center}
|... -jobname "|\textit{target}|" "|[\textit{flags}]%
|\includeonly{|\textit{dest}|}\input{|\textit{main}|}"|
\end{center}
%

%%%%%%%%%%%%%%%%%%%%%%%%%%%%%%%%%%%%%%%%%%%%%%%%%%%%%%%%%%%%%%%%%%%%%%%%%%%%%%%%
%%%%%%%%%%%%%%%%%%%%%%%%%%%%%%%%%%%%%%%%%%%%%%%%%%%%%%%%%%%%%%%%%%%%%%%%%%%%%%%%
\section{Information}

%%%%%%%%%%%%%%%%%%%%%%%%%%%%%%%%%%%%%%%%%%%%%%%%%%%%%%%%%%%%%%%%%%%%%%%%%%%%%%%%
\subsection{Copyright}

Copyright \copyright{} 2017--2018 Niklas Beisert

This work may be distributed and/or modified under the
conditions of the \LaTeX{} Project Public License, either version 1.3
of this license or (at your option) any later version.
The latest version of this license is in
  \url{http://www.latex-project.org/lppl.txt}
and version 1.3 or later is part of all distributions of \LaTeX{}
version 2005/12/01 or later.

This work has the LPPL maintenance status `maintained'.

The Current Maintainer of this work is Niklas Beisert.

This work consists of the files |README.txt|, |childdoc.ins| and |childdoc.dtx|
as well as the derived files |childdoc.def|, |cdocsamp.tex|
with |cdocsch1.tex|, |cdocsch2.tex|, |cdocspt3.tex|, |cdocspt4.tex|,
|cdocsdrf.tex|, |cdocsfn1.tex|, |cdocsfn2.tex|
as well as |childdoc.pdf|.

%%%%%%%%%%%%%%%%%%%%%%%%%%%%%%%%%%%%%%%%%%%%%%%%%%%%%%%%%%%%%%%%%%%%%%%%%%%%%%%%
\subsection{Files and Installation}

The package consists of the files:
%
\begin{center}
\begin{tabular}{ll}
    |README.txt|   & readme file \\
    |childdoc.ins| & installation file \\
    |childdoc.dtx| & source file \\
    |childdoc.def| & definition file \\
    |cdocsamp.tex| & sample main file \\
    |cdocsch1.tex| & sample include file \\
    |cdocsch2.tex| & sample include file \\
    |cdocspt3.tex| & sample part file \\
    |cdocspt4.tex| & sample part file \\
    |cdocsdrf.tex| & sample redirection file \\
    |cdocsfn1.tex| & sample redirection file \\
    |cdocsfn2.tex| & sample redirection file \\
    |childdoc.pdf| & manual
\end{tabular}
\end{center}
%
The distribution consists of the files
|README.txt|, |childdoc.ins| and |childdoc.dtx|.
%
\begin{itemize}
\item
Run (pdf)\LaTeX{} on |childdoc.dtx|
to compile the manual |childdoc.pdf| (this file).
\item
Run \LaTeX{} on |childdoc.ins| to create the definitions file |childdoc.def|
and the sample |cdocsamp.tex| with include files
|cdocsch1.tex|, |cdocsch2.tex|, |cdocspt3.tex|, |cdocspt4.tex|,
|cdocsdrf.tex|, |cdocsfn1.tex|, |cdocsfn2.tex|.
Then copy the file |childdoc.def| to an appropriate directory of your \LaTeX{}
distribution, e.g.\ \textit{texmf-root}|/tex/latex/childdoc|.
\end{itemize}

%%%%%%%%%%%%%%%%%%%%%%%%%%%%%%%%%%%%%%%%%%%%%%%%%%%%%%%%%%%%%%%%%%%%%%%%%%%%%%%%
\subsection{Related CTAN Packages}

There are several other packages which offer a similar functionality:
%
\begin{itemize}
\item
The packages
\href{http://ctan.org/pkg/docmute}{\textsf{docmute}},
\href{http://ctan.org/pkg/includex}{\textsf{includex}} and
\href{http://ctan.org/pkg/standalone}{\textsf{standalone}}
provide commands to include only the document body of
a child file thus allowing both files to be compiled individually.
\item
The packages \href{http://ctan.org/pkg/subdocs}{\textsf{subdocs}}
and \href{http://ctan.org/pkg/subfiles}{\textsf{subfiles}}
provide structures in which the main and child documents can be
encapsulated and allowing them to be compiled individually.
The inclusion mechanism is different from the conventional |\include|.
\item
The package \href{http://ctan.org/pkg/combine}{\textsf{combine}}
is an elaborate solution to combine several documents into one.
\end{itemize}
%
See also the CTAN topic \href{http://ctan.org/topic/subdocs}{\textsf{subdocs}}
for further related packages.
The present package differs from the above solutions in that
a document structure constructed with the conventional |\include| mechanism
just needs two extra commands at the top of every file
such that all constituent files can be compiled individually.

%%%%%%%%%%%%%%%%%%%%%%%%%%%%%%%%%%%%%%%%%%%%%%%%%%%%%%%%%%%%%%%%%%%%%%%%%%%%%%%%
%\subsection{Feature Suggestions}
%
%The following is a list of features which may be useful for future
%versions of this package:
%%
%\begin{itemize}
%\item
%\ldots
%\end{itemize}

%%%%%%%%%%%%%%%%%%%%%%%%%%%%%%%%%%%%%%%%%%%%%%%%%%%%%%%%%%%%%%%%%%%%%%%%%%%%%%%%
\subsection{Revision History}

%%%%%%%%%%%%%%%%%%%%%%%%%%%%%%%%%%%%%%%%
\paragraph{v2.0:} 2018/12/30

\begin{itemize}
\item
immediate forward processing
\item
added |\childdocby| mechanism
\item
manual restructured
\end{itemize}

%%%%%%%%%%%%%%%%%%%%%%%%%%%%%%%%%%%%%%%%
\paragraph{v1.6:} 2018/01/17

\begin{itemize}
\item
application for development of include files
\item
corrections to manual
\end{itemize}

%%%%%%%%%%%%%%%%%%%%%%%%%%%%%%%%%%%%%%%%
\paragraph{v1.5:} 2017/05/21

\begin{itemize}
\item
more complete structuring introduced
\item
|\childdocof| introduced
\item
|\childdoc| renamed to |\childdocmain|
\item
|\childredirect| renamed to |\childdocforward| and |\childdocforwardprefix|
and functionality expanded
\end{itemize}

%%%%%%%%%%%%%%%%%%%%%%%%%%%%%%%%%%%%%%%%
\paragraph{v1.0:} 2017/04/27

\begin{itemize}
\item
manual and install package
\item
first version published on CTAN
\end{itemize}

%%%%%%%%%%%%%%%%%%%%%%%%%%%%%%%%%%%%%%%%
\paragraph{v0.6:} 2017/04/26

\begin{itemize}
\item
redirection mechanism added
\end{itemize}

%%%%%%%%%%%%%%%%%%%%%%%%%%%%%%%%%%%%%%%%
\paragraph{v0.5:} 2017/04/26

\begin{itemize}
\item
functionality in definition file
\end{itemize}


%%%%%%%%%%%%%%%%%%%%%%%%%%%%%%%%%%%%%%%%%%%%%%%%%%%%%%%%%%%%%%%%%%%%%%%%%%%%%%%%
%%%%%%%%%%%%%%%%%%%%%%%%%%%%%%%%%%%%%%%%%%%%%%%%%%%%%%%%%%%%%%%%%%%%%%%%%%%%%%%%
%%%%%%%%%%%%%%%%%%%%%%%%%%%%%%%%%%%%%%%%%%%%%%%%%%%%%%%%%%%%%%%%%%%%%%%%%%%%%%%%
\appendix

\settowidth\MacroIndent{\rmfamily\scriptsize 000\ }

 \DocInput{childdoc.dtx}

\end{document}
%</driver>
% \fi
%
% %%%%%%%%%%%%%%%%%%%%%%%%%%%%%%%%%%%%%%%%%%%%%%%%%%%%%%%%%%%%%%%%%%%%%%%%%%%%%%
% %%%%%%%%%%%%%%%%%%%%%%%%%%%%%%%%%%%%%%%%%%%%%%%%%%%%%%%%%%%%%%%%%%%%%%%%%%%%%%
% \section{Sample}
%\iffalse
%<*samplemain>
%\fi
%
% The following presents a sample document
% with two chapters, two parts, a title page,
% a compile flag as well as three forwarding files to set the flag.
% It consists of eight |.tex| files:
% \begin{center}
% \begin{tabular}{ll}
% |cdocsamp.tex|&main file\\
% |cdocsch1.tex|&include file for chapter 1\\
% |cdocsch2.tex|&include file for chapter 2\\
% |cdocspt3.tex|&include file for part 3\\
% |cdocspt4.tex|&include file for part 4\\
% |cdocsdrf.tex|&forwarding file for main file in draft mode\\
% |cdocsfi1.tex|&forwarding file for final version of chapter 1\\
% |cdocsfi2.tex|&forwarding file for final version of chapter 2\\
% \end{tabular}
% \end{center}
% Each of the eight files can be compiled directly by the \LaTeX{} compiler.
%
% %%%%%%%%%%%%%%%%%%%%%%%%%%%%%%%%%%%%%%
% \paragraph{Main File.}
%
% The main file is called |cdocsamp.tex|.
%
% Load the \textsf{childdoc} definitions and
% declare the filename for the main document:
%    \begin{macrocode}
\input{childdoc.def}
\childdocmain{}
%    \end{macrocode}

% Optional override for |\version| flag:
%    \begin{macrocode}
%%\ifchilddoc\else\providecommand{\version}{draft}\fi
%    \end{macrocode}

% Define the default values for the |\version| flag
% (|final| for the main file and |draft| for childs):
%    \begin{macrocode}
\ifchilddoc
\providecommand{\version}{draft}
\else
\providecommand{\version}{final}
\fi
%    \end{macrocode}

% Load the standard document class:
%    \begin{macrocode}
\documentclass[12pt]{article}
%    \end{macrocode}

% Start the document body:
%    \begin{macrocode}
\begin{document}
%    \end{macrocode}

% Declare a title page.
% Print title, part of document being processed and version flag:
%    \begin{macrocode}
\addtocounter{page}{-1}
\begin{center}
{\LARGE\bfseries{}childdoc example\par}
\vspace{1cm}
\ifchilddoc
\ifchilddocmanual part\else chapter\fi:
`\childdocname' of `\childdocjob'\par
\else
main document: `\childdocjob'\par
\fi
version: \version\par
\end{center}
\newpage
%    \end{macrocode}

% Manually include selected file,
% otherwise process as usual:
%    \begin{macrocode}
\ifchilddocmanual
\section*{part `\childdocname'}
\input{\childdocname}
\else
%    \end{macrocode}

% Include the two chapters:
%    \begin{macrocode}
\include{cdocsch1}
\include{cdocsch2}
%    \end{macrocode}

% Include the two parts unless only chapters should be displayed:
%    \begin{macrocode}
\ifchilddoc\else
\section{part three}
\input{cdocspt3}
\section{part four}
\input{cdocspt4}
\fi
%    \end{macrocode}

% Process as usual until here:
%    \begin{macrocode}
\fi
%    \end{macrocode}

% End of document body:
%    \begin{macrocode}
\end{document}
%    \end{macrocode}
%\iffalse
%</samplemain>
%\fi
%
% %%%%%%%%%%%%%%%%%%%%%%%%%%%%%%%%%%%%%%
% \paragraph{Chapter Include Files.}
%
% The include files are called |cdocsch1.tex| and |cdocsch2.tex|.
%
%\iffalse
%<*samplechap1|samplechap2>
%\fi

% Optional override for |\version| flag:
%    \begin{macrocode}
%%\providecommand{\version}{final}
%    \end{macrocode}

% Include the main document:
%    \begin{macrocode}
\input{childdoc.def}
\childdocof{cdocsamp}
%    \end{macrocode}

%\iffalse
%</samplechap1|samplechap2>
%\fi
%
%\iffalse
%<*samplechap1>
%\fi
% Some text for chapter 1:
%    \begin{macrocode}
\section{one}
some text in chapter one
%    \end{macrocode}

%\iffalse
%</samplechap1>
%\fi
% Some text for chapter 2:
%\iffalse
%<*samplechap2>
%\fi
%    \begin{macrocode}
\section{two}
more text in chapter two
%    \end{macrocode}

%\iffalse
%</samplechap2>
%\fi
%
% %%%%%%%%%%%%%%%%%%%%%%%%%%%%%%%%%%%%%%
% \paragraph{Part Include Files.}
%
% The include files are called |cdocspt3.tex| and |cdocspt4.tex|.
%
%\iffalse
%<*samplepart3|samplepart4>
%\fi

% Optional override for |\version| flag:
%    \begin{macrocode}
%%\providecommand{\version}{final}
%    \end{macrocode}

% Include the main document:
%    \begin{macrocode}
\input{childdoc.def}
\childdocby{cdocsamp}
%    \end{macrocode}

%\iffalse
%</samplepart3|samplepart4>
%\fi
%
%\iffalse
%<*samplepart3>
%\fi
% Some text for part 3:
%    \begin{macrocode}
some text in part three
%    \end{macrocode}

%\iffalse
%</samplepart3>
%\fi
% Some text for part 4:
%\iffalse
%<*samplepart4>
%\fi
%    \begin{macrocode}
more text in part four
%    \end{macrocode}

%\iffalse
%</samplepart4>
%\fi
%
% %%%%%%%%%%%%%%%%%%%%%%%%%%%%%%%%%%%%%%
% \paragraph{Forwarding for a Complete Draft.}
%
% The following forwarding file |cdocsdrf.tex|
% compiles the main document in draft mode:
%\iffalse
%<*sampledraft>
%\fi
%    \begin{macrocode}
\def\version{draft}
\input{childdoc.def}
\childdocforward{cdocsamp}
%    \end{macrocode}

%\iffalse
%</sampledraft>
%\fi
%
% %%%%%%%%%%%%%%%%%%%%%%%%%%%%%%%%%%%%%%
% \paragraph{Forwarding for Final Version of the Chapters.}
%
% The following forwarding files |cdocsfn1.tex| and |cdocsfn2.tex|
% (with identical content)
% compile the final versions of the child documents
% |cdocsch1.tex| and |cdocsch2.tex|, respectively:
%\iffalse
%<*samplefinal>
%\fi
%    \begin{macrocode}
\def\version{final}
\input{childdoc.def}
\childdocforwardprefix[cdocsamp]{cdocsfn}{cdocsch}
%    \end{macrocode}

%\iffalse
%</samplefinal>
%\fi
%
% %%%%%%%%%%%%%%%%%%%%%%%%%%%%%%%%%%%%%%
% \paragraph{Command Line Processing.}
%
% The following three command lines generate the output files
% |cdocscld|, |cdocscl1| and |cdocscl2|
% which should be identical to
% |cdocsdrf|, |cdocsch1| and |cdocsfn2|, respectively:
% \begin{center}
% \begin{tabular}{l}
% |latex -jobname cdocscld \|\\
% |  "\def\version{draft}\input{childdoc.def}\childdocforward{cdocsamp}"|\\
% |latex -jobname cdocscl1 \|\\
% |  "\input{childdoc.def}\childdocforward[cdocsamp]{cdocsch1}"|\\
% |latex -jobname cdocscl2 \|\\
% |  "\def\version{final}\input{childdoc.def}\childdocforward{cdocsch2}"|
% \end{tabular}
% \end{center}
% Note that the trailing backslash on each first line
% merely continues the input to the second line
% (for convenient cut ant paste).
% Furthermore, the command |latex| can be replaced by any
% of its alternative versions such as |pdflatex|.
%
% %%%%%%%%%%%%%%%%%%%%%%%%%%%%%%%%%%%%%%%%%%%%%%%%%%%%%%%%%%%%%%%%%%%%%%%%%%%%%%
% %%%%%%%%%%%%%%%%%%%%%%%%%%%%%%%%%%%%%%%%%%%%%%%%%%%%%%%%%%%%%%%%%%%%%%%%%%%%%%
% \section{Implementation}
%\iffalse
%<*package>
%\fi
%
% This section describes the definitions file |childdoc.def|.

% The definitions cannot be loaded using |\usepackage| or |\RequirePackage|
% which has a mechanism to prevent loading a style file more than once.
% When loading the definitions by means of |\input|
% multiple instances have to be prevented manually:
%\iffalse
%This code needs to be before the `\ProvidesFile' directive
%which is defined at the beginning of this file.
%Therefore it is also placed there and commented out here.
%</package>
%<*discard>
%\fi
%    \begin{macrocode}
\ifdefined\childdocmain\endinput\fi
%    \end{macrocode}
%\iffalse
%</discard>
%<*package>
%\fi
%
% \macro{\ifchilddoc}
% \macro{\ifchilddocmanual}
% The conditional |\ifchilddoc| tells whether a
% child (true) or main (false) document is being compiled.
% The conditional |\ifchilddocmanual| tells whether
% the |\includeonly| mechanism is used (false) or
% the selection of child files must be performed manually (true).
% The definitions initialise to false:
%    \begin{macrocode}
\newif\ifchilddoc
\newif\ifchilddocmanual
%    \end{macrocode}

% \macro{\childdocname}
% \macro{\childdocjob}
% The macro |\childdocname| stores the name of the main document
% to be compiled. The macro |\childdocjob| stores the name of
% the document on which the \LaTeX{} compiler was originally invoked.
% The content of |\jobname| cannot be compared
% to filenames specified in the source due to different catcodes.
% The following code rescans |\jobname|, stores the result
% in |\childdocname| and saves a copy in |\childdocjob|:
%    \begin{macrocode}
\edef\childdocname{\scantokens\expandafter{\jobname\noexpand}}
\let\childdocjob\childdocname
%    \end{macrocode}

% \macro{\childdocdisable}
% The macro |\childdocdisable| prevents the main file
% from being processed more than once.
% At this stage, the main document command |\childdocmain|
% is assumed to be called once again where it should do nothing.
% Any subsequent call to it should prevent
% a secondary processing of the main document
% It overwrites the forwarding commands
% |\childdocof| and |\childdocforward|
% with empty macros to prevent further inclusions of the main document:
%    \begin{macrocode}
\newcommand{\childdocdisable}
{
  \renewcommand{\childdocmain}[1]{\renewcommand{\childdocmain}[1]{\endinput}}
  \renewcommand{\childdocof}[1]{}
  \renewcommand{\childdocby}[2][]{}
  \renewcommand{\childdocforward}[2][]{}
  \renewcommand{\childdocdisable}{}
}
%    \end{macrocode}

% \macro{\childdocmain}
% The macro |\childdocmain| is to be called at the top of the main file
% with nothing or the main filename (without extension) as argument.
% First, it breaks loops.
% If the argument is not empty and does not match |\childdocname|
% (which is set by the first inclusion of |childdoc.def|),
% |\ifchilddoc| is set to true, |\includeonly| is applied to the child file
% and |\jobname| is set to the main file
% (for proper handling of |.aux| files):
%    \begin{macrocode}
\newcommand{\childdocmain}[1]
{
  \childdocdisable\childdocmain{}
  \if?#1?\else
    \begingroup
      \def\childdoctmp{#1}
      \ifx\childdoctmp\childdocname
        \def\childdoctmp{}
      \else
        \def\childdoctmp
        {
          \childdoctrue
          \includeonly{\childdocname}
          \def\childdocjob{#1}
          \def\jobname{#1}
        }
      \fi
      \expandafter
    \endgroup
    \childdoctmp
  \fi
}
%    \end{macrocode}

% \macro{\childdocof}
% The command |\childdocof| redirects
% compilation to the main file |#1|.
%    \begin{macrocode}
\newcommand{\childdocof}[1]
{
  \childdocdisable
  \childdoctrue
  \includeonly{\childdocname}
  \def\jobname{#1}
  \def\childdocjob{#1}
  \input{#1}
}
%    \end{macrocode}

% \macro{\childdocby}
% The command |\childdocby| ....
%    \begin{macrocode}
\newcommand{\childdocby}[2][]
{
  \childdocdisable
  \childdoctrue
  \childdocmanualtrue
  \if?#1?\else
    \def\jobname{#2}
  \fi
  \def\childdocjob{#2}
  \input{#2}
  \endinput
}
%    \end{macrocode}

% \macro{\childdocforward}
% The command |\childdocforward| redirects
% compilation to the main file or
% (if the optional argument is given) a child file.
% Parameters are set as if the main file
% or a child file starting with |\childdocof| was compiled.
% Then compilation is handed over to the main file:
%    \begin{macrocode}
\newcommand{\childdocforward}[2][]
{
  \begingroup
    \if?#1?
      \def\childdoctmp
      {
        \def\childdocname{#2}
        \def\childdocjob{#2}
        \def\jobname{#2}
        \input{#2}
        \endinput
      }
    \else
      \def\childdoctmp
      {
        \childdocdisable
        \def\childdocname{#2}
        \childdoctrue
        \includeonly{#2}
        \def\childdocjob{#1}
        \def\jobname{#1}
        \input{#1}
        \endinput
      }
    \fi
    \expandafter
  \endgroup
  \childdoctmp
}
%    \end{macrocode}

% \macro{\childdocforwardprefix}
% The command |\childdocforwardprefix| redirects
% compilation to the main or a child file by means of a pattern.
% The prefix |#1| in the current filename is replaced by |#2|
% and the suffix of the current filename is kept
% (it is assumed that the filename does not contain the substring `|~~~|'
% which is used as a delimiter).
% Compilation is handed over to the new file by |\childdocforward|:
%    \begin{macrocode}
\newcommand{\childdocforwardprefix}[3][]
{
  \begingroup
    \def\childdocextract #2##1~~~{\def\childdoctmp{\childdocforward[#1]{#3##1}}}
    \expandafter\childdocextract\childdocname~~~
    \expandafter
  \endgroup
  \childdoctmp
}
%    \end{macrocode}

% \macro{\childdoc}
% The deprecated macro |\childdoc| is a legacy version of |\childdocmain|:
%    \begin{macrocode}
\newcommand{\childdoc}{\childdocmain}
%    \end{macrocode}

% \macro{\childdocredirect}
% The deprecated macro |\childdocredirect| is a legacy version
% of |\childdocforward| and |\childdocforwardprefix|:
%    \begin{macrocode}
\newcommand{\childdocredirect}[2][]
{
  \begingroup
    \if?#1?
      \def\childdoctmp{\childdocforward{#2}}
    \else
      \def\childdoctmp{\childdocforwardprefix{#1}{#2}}
    \fi
    \expandafter
  \endgroup
  \childdoctmp
}
%    \end{macrocode}

%\iffalse
%</package>
%\fi
%
\endinput
|\\
|\childdocforward{|\textit{main}|}|\\
\end{tabular}
\end{center}
%
or alternatively with:
%
\begin{center}
\begin{tabular}{l}
|% \iffalse
%
% childdoc.dtx Copyright (C) 2017-2018 Niklas Beisert
%
% This work may be distributed and/or modified under the
% conditions of the LaTeX Project Public License, either version 1.3
% of this license or (at your option) any later version.
% The latest version of this license is in
%   http://www.latex-project.org/lppl.txt
% and version 1.3 or later is part of all distributions of LaTeX
% version 2005/12/01 or later.
%
% This work has the LPPL maintenance status `maintained'.
%
% The Current Maintainer of this work is Niklas Beisert.
%
% This work consists of the files childdoc.dtx and childdoc.ins
% and the derived files childdoc.def and cdocsamp.tex with
% cdocsch1.tex, cdocsch2.tex, cdocsdrf.tex, cdocsfn1.tex, cdocsfn2.tex.
%
%<package>\ifdefined\childdocmain\endinput\fi
%<package>\ProvidesFile{childdoc.def}[2018/12/30 v2.0 child document driver]
%<samplemain>\ProvidesFile{cdocsamp.tex}[2018/12/30 v2.0 sample for childdoc]
%<*driver>
%\ProvidesFile{childdoc.drv}[2018/12/30 v2.0 childdoc reference manual file]
\PassOptionsToClass{10pt,a4paper}{article}
\documentclass{ltxdoc}

\usepackage[margin=35mm]{geometry}
\usepackage{hyperref}
\usepackage{hyperxmp}
\usepackage[usenames]{color}

\hypersetup{colorlinks=true}
\hypersetup{pdfstartview=FitH}
\hypersetup{pdfpagemode=UseNone}
\hypersetup{pdfsource={}}
\hypersetup{pdflang={en-UK}}
\hypersetup{pdfcopyright={Copyright 2017-2018 Niklas Beisert.
  This work may be distributed and/or modified under the
  conditions of the LaTeX Project Public License, either version 1.3
  of this license or (at your option) any later version.}}
\hypersetup{pdflicenseurl={http://www.latex-project.org/lppl.txt}}
\hypersetup{pdfcontactaddress={ETH Zurich, ITP, HIT K,
  Wolfgang-Pauli-Strasse 27}}
\hypersetup{pdfcontactpostcode={8093}}
\hypersetup{pdfcontactcity={Zurich}}
\hypersetup{pdfcontactcountry={Switzerland}}
\hypersetup{pdfcontactemail={nbeisert@itp.phys.ethz.ch}}
\hypersetup{pdfcontacturl={http://people.phys.ethz.ch/\xmptilde nbeisert/}}

\newcommand{\secref}[1]{\hyperref[#1]{section \ref*{#1}}}

\parskip1ex
\parindent0pt
\let\olditemize\itemize
\def\itemize{\olditemize\parskip0pt}

\begin{document}

\title{The \textsf{childdoc} Package}
\hypersetup{pdftitle={The childdoc Package}}
\author{Niklas Beisert\\[2ex]
  Institut f\"ur Theoretische Physik\\
  Eidgen\"ossische Technische Hochschule Z\"urich\\
  Wolfgang-Pauli-Strasse 27, 8093 Z\"urich, Switzerland\\[1ex]
  \href{mailto:nbeisert@itp.phys.ethz.ch}
  {\texttt{nbeisert@itp.phys.ethz.ch}}}
\hypersetup{pdfauthor={Niklas Beisert}}
\hypersetup{pdfsubject={Manual for the LaTeX2e Package childdoc}}
\date{30 December 2018, \textsf{v2.0}}
\maketitle

\begin{abstract}\noindent
\textsf{childdoc} is a \LaTeXe{} package
that enables the direct compilation
of document sections included by |\include|
to individual files.
\end{abstract}

\begingroup
\parskip0ex
\tableofcontents
\endgroup

%%%%%%%%%%%%%%%%%%%%%%%%%%%%%%%%%%%%%%%%%%%%%%%%%%%%%%%%%%%%%%%%%%%%%%%%%%%%%%%%
%%%%%%%%%%%%%%%%%%%%%%%%%%%%%%%%%%%%%%%%%%%%%%%%%%%%%%%%%%%%%%%%%%%%%%%%%%%%%%%%
\section{Introduction}

\LaTeX{} provides a mechanism to structure a large document (such as a book)
into a main file and several child files (containing the chapters)
using the |\include| command.
This mechanism is beneficial for documents
which span hundreds of pages in order to
make the source file(s) more manageable.
Moreover, compilation can be restricted to
selected child files by means of the |\includeonly| command.
The latter feature can be used to reduce the compilation time while editing
(this was significantly more useful in the earlier days of \LaTeX{})
or to generate a smaller document which is easier to navigate.
Another application of |\includeonly| is to generate
documents consisting of selected parts of the complete document.

However, there are a few drawbacks of the plain |\include| mechanism:
\begin{itemize}
\item
The child files cannot be compiled on their own,
they can only be compiled via the main file.
A naive editing environment
(such as a text editor with an option
to have the current file processed by \LaTeX)
may require one to switch to the main file before compiling;
attempting to compile the child file produces errors.
\item
The main file must be modified (each time)
to adjust the |\includeonly| command
to the present needs. This easily leaves the main file in a messy state.
\item
The generated document will always carry the filename
of the main document. This is inconvenient if
several child files are to be compiled and
to be kept for distribution.
\end{itemize}

The present package provides a simple interface
to make child files individually compilable by \LaTeX{}.
Compiling a child file then has the same effect as compiling
the main file with an |\includeonly| command
to select the appropriate child.
Moreover the generated document will carry the name of the child
rather than the main file.
This resolves all three above issues.

This feature is meant to make the editing of books,
thesis documents and lecture notes somewhat more convenient.
However, the package can also be used efficiently for
composing a series of documents (such as exercise sheets)
which are typically distributed individually.
It then assists the author in generating the individual documents
(potentially in different versions)
as well as a document containing the collected series.
Another application is in developing style files
or other kinds of included material
where compilation of the style file could redirect
to a sample or test file.

%%%%%%%%%%%%%%%%%%%%%%%%%%%%%%%%%%%%%%%%%%%%%%%%%%%%%%%%%%%%%%%%%%%%%%%%%%%%%%%%
%%%%%%%%%%%%%%%%%%%%%%%%%%%%%%%%%%%%%%%%%%%%%%%%%%%%%%%%%%%%%%%%%%%%%%%%%%%%%%%%
\section{Usage}

First of all, the package \textsf{childdoc} is \emph{not} a standard
\LaTeXe{} |.sty| style file! Therefore it needs to be invoked in
a non-standard way.

%%%%%%%%%%%%%%%%%%%%%%%%%%%%%%%%%%%%%%%%%%%%%%%%%%%%%%%%%%%%%%%%%%%%%%%%%%%%%%%%
\subsection{Included Files}
\label{sec:include}

%%%%%%%%%%%%%%%%%%%%%%%%%%%%%%%%%%%%%%%%
\DescribeMacro{\childdocmain}
To use the package, add the commands
\begin{center}
\begin{tabular}{l}
|\input{childdoc.def}|\\
|\childdocmain{}|\\
\end{tabular}
\end{center}
at the very top of the main \LaTeX{} file,
in particular \emph{before} the |\documentclass| statement!
The argument of |\childdocmain| should be left empty
(but it must be present).

%%%%%%%%%%%%%%%%%%%%%%%%%%%%%%%%%%%%%%%%
\DescribeMacro{\childdocof}
Furthermore, add the commands
\begin{center}
\begin{tabular}{l}
|\input{childdoc.def}|\\
|\childdocof{|\textit{main}|}|\\
\end{tabular}
\end{center}
at the top of every child file \textit{child}
which is included by |\include{|\textit{child}|}|
from within the main file
(or at least for those files to be compiled individually).
The argument \textit{main} must be the filename of the main file.

There are a couple of
considerations in setting up the main and child documents:

%%%%%%%%%%%%%%%%%%%%%%%%%%%%%%%%%%%%%%%%
\paragraph{Restrictions.}

Please note the following restrictions:
\begin{itemize}
\item
|\childdocmain| must be called with one argument \textit{main}
to ensure compatibility with earlier version of the package.
It must either be empty (|\childdocmain{}|)
or precisely match the filename of the main file in which it is specified.
See \secref{sec:detection} for further information.
\item
The filename \textit{main} must be specified without the |.tex| extension.
\item
The filename \textit{main} is case sensitive
(even in case-insensitive file systems)
due to internal string comparison.
\item
The argument \textit{main} should be fully expanded, it cannot be a macro.
\item
Subdirectories and special characters should be avoided in filenames.
\item
The command |\childdocmain{|\textit{main}|}| must be followed by a whitespace.
It should not be followed immediately by another command
or by a comment mark `|%|'.
This is because the \TeX{} parser reads the token immediately following
the argument of |\childdocmain| and puts it
at the beginning of every child section;
however, a white\-space is ignored.
\end{itemize}

%%%%%%%%%%%%%%%%%%%%%%%%%%%%%%%%%%%%%%%%
\paragraph{Content of Main File.}

It is advisable to place all content in the child files included by |\include|.
Any output contained in the main file will appear in all child documents
unless suppressed manually;
it cannot be suppressed automatically by the |\includeonly| directive
and thus should normally be avoided.
A method to include some content in the main file
by means of conditional processing is described in \secref{sec:conditional}.

%%%%%%%%%%%%%%%%%%%%%%%%%%%%%%%%%%%%%%%%
\paragraph{Page Numbering.}

When only a part of the document is compiled,
the appropriate numbering of pages
(as well as other status parameters)
is determined from the |.aux| files.
The latter contain information from previous passes.
However this information needs to propagate through
all intermediate child documents.
Therefore the page numbering in child documents may well
be inconsistent until the complete document is compiled at least once.

A useful (if unconventional) way to always ensure a consistent
page numbering is to restart the numbering in each child document
and denote the pages by `\textit{child}|.|\textit{page}'
where \textit{child} represents the chapter/section number of the child file.
This can be achieved by the command
|\numberwithin{page}{|\textit{child}|}|
of the \textsf{amsmath} package
where \textit{child} can be |chapter| or |section|
depending on the chosen structuring.
Alternatively, one can modify the macro |\thepage| appropriately
and reset the counter |page| at the start of each child file.

%%%%%%%%%%%%%%%%%%%%%%%%%%%%%%%%%%%%%%%%%%%%%%%%%%%%%%%%%%%%%%%%%%%%%%%%%%%%%%%%
\subsection{Conditional Processing}
\label{sec:conditional}

The package provides a mechanism to compile different versions
of a document. To customise the versions further some conditional processing
can come in handy to distinguish which version is being compiled.
The package provides two macros to describe the compilation context:

%%%%%%%%%%%%%%%%%%%%%%%%%%%%%%%%%%%%%%%%
\DescribeMacro{\ifchilddoc}
The conditional |\ifchilddoc| distinguishes between the compilation of
child documents and the main document:
%
\begin{center}
|\ifchilddoc |\textit{child-code}| |[|\||else |\textit{main-code}]| \||fi|
\end{center}

%%%%%%%%%%%%%%%%%%%%%%%%%%%%%%%%%%%%%%%%
\DescribeMacro{\childdocname}
\DescribeMacro{\childdocjob}
The macro |\childdocname| contains the filename (without extension)
of the main or child file being processed.
Note that |\childdocjob| will always contain the name of the main file.

%%%%%%%%%%%%%%%%%%%%%%%%%%%%%%%%%%%%%%%%
\paragraph{Title Page.}

Conditional processing can be used to include a title or banner page
in the main document when proper precautions are taken.
Importantly, the code in the main file should ensure that the page counter
(as well as other status parameters which are stored in the |.aux| files)
takes the same value after the conditional processing.
Otherwise the page numbers may take divergent values
depending on which part is compiled.

For example, a title page could be declared by:
%
\begin{center}
\begin{tabular}{l}
|\ifchilddoc\||else|\\
|\addtocounter{page}{-1}|\\
\textit{code for title page}\\
|\newpage|\\
|\||fi|
\end{tabular}
\end{center}
%
A banner page for the child documents can be generated by:
%
\begin{center}
\begin{tabular}{l}
|\ifchilddoc|\\
|\addtocounter{page}{-1}|\\
\textit{code for banner page}\\
|\newpage|\\
|\||fi|
\end{tabular}
\end{center}
%
Here one could write a message such as:
\begin{center}
|This is the part \childdocname{} of \childdocjob{}.|
\end{center}

%%%%%%%%%%%%%%%%%%%%%%%%%%%%%%%%%%%%%%%%%%%%%%%%%%%%%%%%%%%%%%%%%%%%%%%%%%%%%%%%
\subsection{Flags}
\label{sec:flags}

The package makes it easy to generate different versions
of the main or child documents.
To this end compilation flags can be defined
and assigned different default values.
They will be particularly useful in conjunction
with the forwarding mechanism described in \secref{sec:forward}.

For example, it may be useful to have a flag |\version|
which can be set to |draft| or |final|.
The document source will contain some conditional code
depending on the value of |\version|.
Suppose further, the flag should default to |final| for the main file
and to |draft| for child files
which is a natural assignment for editing the document.
This is achieved by placing the following code
in the preamble of the main document
(below the |\childdocmain| directive):
%
\begin{center}
\begin{tabular}{l}
|\ifchilddoc|\\
|\providecommand{\version}{draft}|\\
|\||else|\\
|\providecommand{\version}{final}|\\
|\||fi|
\end{tabular}
\end{center}
%
The definition by |\providecommand| makes sure
that previous definitions are not overwritten.
Further statements |\providecommand{\version}{...}|
can thus be added before the above code to override it.

For the main file, one might add a line
(between |\childdocmain| and the above block)
%
\begin{center}
|%\ifchilddoc\||else\providecommand{\version}{draft}\||fi|
\end{center}
%
which can be uncommented to produce a draft version.
Likewise one can add a line to the very top of a child file
(above the |\childdocof{|\textit{main}|}| directive)
%
\begin{center}
|%\providecommand{\version}{final}|
\end{center}
%
which can be uncommented to produce the final version of this child document.

%%%%%%%%%%%%%%%%%%%%%%%%%%%%%%%%%%%%%%%%%%%%%%%%%%%%%%%%%%%%%%%%%%%%%%%%%%%%%%%%
\subsection{Forwarding}
\label{sec:forward}

Different versions of the main or child documents
using compilation flags as described in \secref{sec:flags}
can be (permanently) stored in different files
for convenient compilation, viewing and distribution.
To this end, the package defines a command
to pass on compilation to a different file:

%%%%%%%%%%%%%%%%%%%%%%%%%%%%%%%%%%%%%%%%
\DescribeMacro{\childdocforward}
The command |\childdocforward| redirects processing to
another source file:
%
\begin{center}
\begin{tabular}{l}
|\input{childdoc.def}|\\
|\childdocforward[|\textit{main}|]{|\textit{dest}|}|\\
\end{tabular}
\end{center}
%
The argument \textit{dest} is the destination file
(without extension).
It should be the main file or one of the child files.
Note that further \textsf{childdoc} directives
such as |\childdocof| and |\childdocforward|
in the indicated file will be processed in this form.
The optional argument \textit{main}
passes on directly to the main file \textit{main}
while pretending to compile the child \textit{dest}.
This form behaves as if \textit{dest}
issues |\childdocof{|\textit{main}|}| right away,
and no further \textsf{childdoc} directives will be processed.

%%%%%%%%%%%%%%%%%%%%%%%%%%%%%%%%%%%%%%%%
\DescribeMacro{\...prefix}
In the alternative form |\childdocforwardprefix|,
%
\begin{center}
\begin{tabular}{l}
|\input{childdoc.def}|\\
|\childdocforwardprefix[|\textit{main}|]{|\textit{prefix}|}{|\textit{dest}|}|
\end{tabular}
\end{center}
%
the destination file is determined by a pattern
depending on the current file:
To make this work, the current file must be called
`{\textit{prefix}\hspace{0.2em}\textit{suffix}}'
with \textit{prefix} matching precisely the argument.
Processing is then passed on to the file
`{\textit{dest}\hspace{0.2em}\textit{suffix}}'.
Surely, the same effect is achieved by
directly specifying the
argument `{\textit{dest}\hspace{0.2em}\textit{suffix}}'
in the first form.
However, that requires to set up a different file
for each child. With the alternative form of the command
all these files can have exactly the same content
which simplifies setting them up and maintaining them.

For example, the following file |draft.tex|
with a compilation flag |\version| as described in \secref{sec:flags}
compiles the main document as a draft:
%
\begin{center}
\begin{tabular}{l}
|\def\version{draft}|\\
|\input{childdoc.def}|\\
|\childdocforward{|\textit{main}|}|
\end{tabular}
\end{center}
%
Likewise, the following files |final|\textit{nn}|.tex|
compile the final version of the child document
|child|\textit{nn}|.tex|:
%
\begin{center}
\begin{tabular}{l}
|\def\version{final}|\\
|\input{childdoc.def}|\\
|\childdocforwardprefix{final}{child}|
\end{tabular}
\end{center}
%

Note that when several versions of a main file and/or of each child file
are to be generated, it may be convenient to set up a |Makefile| or
shell script to automatise the process.

%%%%%%%%%%%%%%%%%%%%%%%%%%%%%%%%%%%%%%%%%%%%%%%%%%%%%%%%%%%%%%%%%%%%%%%%%%%%%%%%
\subsection{Command Line Processing}
\label{sec:commandline}

The effect of redirection files can also be achieved by invoking
the \LaTeX{} compiler with a more elaborate command line.
Most conveniently this should be done as part
of a shell script or a |Makefile|.

When using \textsf{childdoc} in the main file, the following
command lines effectively perform a redirection
(note that depending on the shell being used,
backslashes may have to be doubled: `|\|' $\to$ `|\\|'):
%
\begin{center}
|... -jobname "|\textit{target}|" |\\|"|[\textit{flags}]%
|\input{childdoc.def}\childdocforward[|\textit{main}|]{|\textit{dest}|}"|
\end{center}
%
Here \textit{target} is the name of the output file,
\textit{main} is the name of the main file
and \textit{dest} is the name of the main or child file to be processed
(all filenames without extensions).
The optional argument \textit{main} can be omitted
if \textit{main} matches \textit{dest}.
Optionally, compilation \textit{flags} can be defined via |\def| commands.
This command line makes the \TeX{} engine believe
it is compiling the file \textit{target}
whose content is specified as the latter parameter.
The provided code then forwards the processing to
\textit{main} or \textit{dest} as described in \secref{sec:forward}.

%%%%%%%%%%%%%%%%%%%%%%%%%%%%%%%%%%%%%%%%%%%%%%%%%%%%%%%%%%%%%%%%%%%%%%%%%%%%%%%%
\subsection{Include by Input}
\label{sec:input}

Including child documents by |\include| has some restrictions by design.
Most notably, the content of a child document always occupies
its own set of pages; pages cannot be shared between child documents.
Usually, this behaviour makes perfect sense
because each child document contain an essential part of the document.
However, in some situations it may be desirable to compose
a document from a collection of parts
without having mandatory page breaks between then.
For this case, the package
provides a mechanism to include parts
by |\input| which can also be processed individually.
However, by construction this mechanism
requires manual handling of the content to be output.

%%%%%%%%%%%%%%%%%%%%%%%%%%%%%%%%%%%%%%%%
\DescribeMacro{\ifchilddocmanual}
The main file should be prepared as usual, see \secref{sec:include}.
However, the document body must make a distinction
between processing of an individual part and of the main document, e.g.:
%
\begin{center}
\begin{tabular}{l}
|\ifchilddocmanual|\\
|\input{\childdocname}|\\
|\||else|\\
\textit{document body with }|\input{|\textit{part}|}|\\
|\||fi|
\end{tabular}
\end{center}
%
The conditional |\ifchilddocmanual| is true whenever
a part to be included by |\input| is being compiled,
and the name of the part is stored in |\childdocname|.

%%%%%%%%%%%%%%%%%%%%%%%%%%%%%%%%%%%%%%%%
\DescribeMacro{\childdocby}
Each part to be included by |\input| should start with:
%
\begin{center}
\begin{tabular}{l}
|\input{childdoc.def}|\\
|\childdocby{|\textit{main}|}|\\
\end{tabular}
\end{center}
%
The directive |\childdocby| is similar to |\childdocof|
described in \secref{sec:include},
but the subsequent selection of content must be done manually.
To that end, both |\ifchilddoc| and |\ifchilddocmanual|
will be true upon processing of a part,
and the name of the part is stored in |\childdocname|.
Note that |\jobname| will be set to the filename of the current part
so that each part receives an individual |.aux| file
that does not interfere with the |.aux| file(s) of the main document.
This behaviour can be altered by the alternative form
|\childdocby[*]{|\textit{main}|}| (with a non-empty optional argument)
which uses the |.aux| file of the main document
by setting |\jobname| to \textit{main}.

%%%%%%%%%%%%%%%%%%%%%%%%%%%%%%%%%%%%%%%%%%%%%%%%%%%%%%%%%%%%%%%%%%%%%%%%%%%%%%%%
\subsection{Driver Development}
\label{sec:driver}

The \textsf{childdoc} mechanism can also be use for the development
of definition files such as \LaTeX{} styles or classes.
This case differs from the above setup with multiple parts
included by |\include| in that no |\includeonly| should be invoked.
This can be achieved by starting the include file
(before |\ProvidesPackage|) with:
%
\begin{center}
\begin{tabular}{l}
|\input{childdoc.def}|\\
|\childdocforward{|\textit{main}|}|\\
\end{tabular}
\end{center}
%
or alternatively with:
%
\begin{center}
\begin{tabular}{l}
|\input{childdoc.def}|\\
|\childdocby{|\textit{main}|}|\\
\end{tabular}
\end{center}
%
Both forms have slightly different effects as described above.
The main file is prepared as usual, see \secref{sec:include}.

%%%%%%%%%%%%%%%%%%%%%%%%%%%%%%%%%%%%%%%%%%%%%%%%%%%%%%%%%%%%%%%%%%%%%%%%%%%%%%%%
\subsection{Legacy Detection}
\label{sec:detection}

The directive |\childdocmain| in the main file can detect
whether the complete document or merely a child is to be compiled
even without using the directive |\childdocof|.
This method is deprecated because it is less robust
and there is no compelling reason to use it;
it is merely provided for backward compatibility
and it may be removed in future versions.

If the detection mechanism is to be used,
it is mandatory to correctly specify
the filename of the main file as the argument of |\childdocmain|:
%
\begin{center}
\begin{tabular}{l}
|\input{childdoc.def}|\\
|\childdocmain{|\textit{main}|}|\\
\end{tabular}
\end{center}
%
If |\jobname| does not match the argument \textit{main} of |\childdocmain|,
it is assumed that |\jobname| points to the child file to be compiled.
When using |\childdocmain| with the main file specified as argument,
it suffices to start a child file
with just |\input{|\textit{main}|}|
without loading of the package and using |\childdocof|.
If instead all processing is done
with the appropriate \textsf{childdoc} directives,
the argument of \textit{main} of |\childdocmain| can be empty.

An alternative version of the command line processing described
in \secref{sec:commandline} using the detection mechanism reads:
%
\begin{center}
|... -jobname "|\textit{target}|" "|[\textit{flags}]%
[|\def\jobname{|\textit{dest}|}|]|\input{|\textit{main}|}"|
\end{center}

%%%%%%%%%%%%%%%%%%%%%%%%%%%%%%%%%%%%%%%%%%%%%%%%%%%%%%%%%%%%%%%%%%%%%%%%%%%%%%%%
\subsection{Manual Code}
\label{sec:manual}

In case one cannot be certain whether the definitions file |childdoc.def|
is installed on the target \TeX{} distribution
and one prefers not to ship it,
it is conceivable to paste a few relevant commands into the sources.

To that end, drop all statements |\input{childdoc.def}|
and perform the replacements as outlined below.
Instead of |\childdocmain{|\textit{main}|}| add the following code
to the top of the main file:
%
\begin{center}
\begin{tabular}{l}
|\||ifdefined\childdocname\endinput\||fi\newif\ifchilddoc|\\
|\edef\childdocname{\scantokens\expandafter{\jobname\noexpand}}|\\
|\def\childdocmain{|\textit{main}|}\||ifx\childdocmain\childdocname\||else|\\
|\childdoctrue\includeonly{\childdocname}\let\jobname\childdocmain\||fi|\\
\end{tabular}
\end{center}
%
Instead of |\childdocof{|\textit{main}|}| just include the main file
at the top of each child file:
%
\begin{center}
|\input{|\textit{main}|}|
\end{center}
%
A simple redirection |\childdocforward{|\textit{dest}|}| is achieved by:
%
\begin{center}
|\def\jobname{|\textit{dest}|}\input{\jobname}|
\end{center}
%
The redirection with prefix
|\childdocforwardprefix[|\textit{prefix}|]{|\textit{dest}|}|
is accomplished by:
%
\begin{center}
\begin{tabular}{l}
|{\edef\jobname{\scantokens\expandafter{\jobname\noexpand}}|\\
|\def\redirectjob |\textit{prefix}|#1~~~{\gdef\jobname{|\textit{dest}|#1}}|\\
|\expandafter\redirectjob\jobname~~~}\input{\jobname}|
\end{tabular}
\end{center}

In an alternative approach,
child documents can be compiled by a specific command line
without additional code or specific definitions:
%
\begin{center}
|... -jobname "|\textit{target}|" "|[\textit{flags}]%
|\includeonly{|\textit{dest}|}\input{|\textit{main}|}"|
\end{center}
%

%%%%%%%%%%%%%%%%%%%%%%%%%%%%%%%%%%%%%%%%%%%%%%%%%%%%%%%%%%%%%%%%%%%%%%%%%%%%%%%%
%%%%%%%%%%%%%%%%%%%%%%%%%%%%%%%%%%%%%%%%%%%%%%%%%%%%%%%%%%%%%%%%%%%%%%%%%%%%%%%%
\section{Information}

%%%%%%%%%%%%%%%%%%%%%%%%%%%%%%%%%%%%%%%%%%%%%%%%%%%%%%%%%%%%%%%%%%%%%%%%%%%%%%%%
\subsection{Copyright}

Copyright \copyright{} 2017--2018 Niklas Beisert

This work may be distributed and/or modified under the
conditions of the \LaTeX{} Project Public License, either version 1.3
of this license or (at your option) any later version.
The latest version of this license is in
  \url{http://www.latex-project.org/lppl.txt}
and version 1.3 or later is part of all distributions of \LaTeX{}
version 2005/12/01 or later.

This work has the LPPL maintenance status `maintained'.

The Current Maintainer of this work is Niklas Beisert.

This work consists of the files |README.txt|, |childdoc.ins| and |childdoc.dtx|
as well as the derived files |childdoc.def|, |cdocsamp.tex|
with |cdocsch1.tex|, |cdocsch2.tex|, |cdocspt3.tex|, |cdocspt4.tex|,
|cdocsdrf.tex|, |cdocsfn1.tex|, |cdocsfn2.tex|
as well as |childdoc.pdf|.

%%%%%%%%%%%%%%%%%%%%%%%%%%%%%%%%%%%%%%%%%%%%%%%%%%%%%%%%%%%%%%%%%%%%%%%%%%%%%%%%
\subsection{Files and Installation}

The package consists of the files:
%
\begin{center}
\begin{tabular}{ll}
    |README.txt|   & readme file \\
    |childdoc.ins| & installation file \\
    |childdoc.dtx| & source file \\
    |childdoc.def| & definition file \\
    |cdocsamp.tex| & sample main file \\
    |cdocsch1.tex| & sample include file \\
    |cdocsch2.tex| & sample include file \\
    |cdocspt3.tex| & sample part file \\
    |cdocspt4.tex| & sample part file \\
    |cdocsdrf.tex| & sample redirection file \\
    |cdocsfn1.tex| & sample redirection file \\
    |cdocsfn2.tex| & sample redirection file \\
    |childdoc.pdf| & manual
\end{tabular}
\end{center}
%
The distribution consists of the files
|README.txt|, |childdoc.ins| and |childdoc.dtx|.
%
\begin{itemize}
\item
Run (pdf)\LaTeX{} on |childdoc.dtx|
to compile the manual |childdoc.pdf| (this file).
\item
Run \LaTeX{} on |childdoc.ins| to create the definitions file |childdoc.def|
and the sample |cdocsamp.tex| with include files
|cdocsch1.tex|, |cdocsch2.tex|, |cdocspt3.tex|, |cdocspt4.tex|,
|cdocsdrf.tex|, |cdocsfn1.tex|, |cdocsfn2.tex|.
Then copy the file |childdoc.def| to an appropriate directory of your \LaTeX{}
distribution, e.g.\ \textit{texmf-root}|/tex/latex/childdoc|.
\end{itemize}

%%%%%%%%%%%%%%%%%%%%%%%%%%%%%%%%%%%%%%%%%%%%%%%%%%%%%%%%%%%%%%%%%%%%%%%%%%%%%%%%
\subsection{Related CTAN Packages}

There are several other packages which offer a similar functionality:
%
\begin{itemize}
\item
The packages
\href{http://ctan.org/pkg/docmute}{\textsf{docmute}},
\href{http://ctan.org/pkg/includex}{\textsf{includex}} and
\href{http://ctan.org/pkg/standalone}{\textsf{standalone}}
provide commands to include only the document body of
a child file thus allowing both files to be compiled individually.
\item
The packages \href{http://ctan.org/pkg/subdocs}{\textsf{subdocs}}
and \href{http://ctan.org/pkg/subfiles}{\textsf{subfiles}}
provide structures in which the main and child documents can be
encapsulated and allowing them to be compiled individually.
The inclusion mechanism is different from the conventional |\include|.
\item
The package \href{http://ctan.org/pkg/combine}{\textsf{combine}}
is an elaborate solution to combine several documents into one.
\end{itemize}
%
See also the CTAN topic \href{http://ctan.org/topic/subdocs}{\textsf{subdocs}}
for further related packages.
The present package differs from the above solutions in that
a document structure constructed with the conventional |\include| mechanism
just needs two extra commands at the top of every file
such that all constituent files can be compiled individually.

%%%%%%%%%%%%%%%%%%%%%%%%%%%%%%%%%%%%%%%%%%%%%%%%%%%%%%%%%%%%%%%%%%%%%%%%%%%%%%%%
%\subsection{Feature Suggestions}
%
%The following is a list of features which may be useful for future
%versions of this package:
%%
%\begin{itemize}
%\item
%\ldots
%\end{itemize}

%%%%%%%%%%%%%%%%%%%%%%%%%%%%%%%%%%%%%%%%%%%%%%%%%%%%%%%%%%%%%%%%%%%%%%%%%%%%%%%%
\subsection{Revision History}

%%%%%%%%%%%%%%%%%%%%%%%%%%%%%%%%%%%%%%%%
\paragraph{v2.0:} 2018/12/30

\begin{itemize}
\item
immediate forward processing
\item
added |\childdocby| mechanism
\item
manual restructured
\end{itemize}

%%%%%%%%%%%%%%%%%%%%%%%%%%%%%%%%%%%%%%%%
\paragraph{v1.6:} 2018/01/17

\begin{itemize}
\item
application for development of include files
\item
corrections to manual
\end{itemize}

%%%%%%%%%%%%%%%%%%%%%%%%%%%%%%%%%%%%%%%%
\paragraph{v1.5:} 2017/05/21

\begin{itemize}
\item
more complete structuring introduced
\item
|\childdocof| introduced
\item
|\childdoc| renamed to |\childdocmain|
\item
|\childredirect| renamed to |\childdocforward| and |\childdocforwardprefix|
and functionality expanded
\end{itemize}

%%%%%%%%%%%%%%%%%%%%%%%%%%%%%%%%%%%%%%%%
\paragraph{v1.0:} 2017/04/27

\begin{itemize}
\item
manual and install package
\item
first version published on CTAN
\end{itemize}

%%%%%%%%%%%%%%%%%%%%%%%%%%%%%%%%%%%%%%%%
\paragraph{v0.6:} 2017/04/26

\begin{itemize}
\item
redirection mechanism added
\end{itemize}

%%%%%%%%%%%%%%%%%%%%%%%%%%%%%%%%%%%%%%%%
\paragraph{v0.5:} 2017/04/26

\begin{itemize}
\item
functionality in definition file
\end{itemize}


%%%%%%%%%%%%%%%%%%%%%%%%%%%%%%%%%%%%%%%%%%%%%%%%%%%%%%%%%%%%%%%%%%%%%%%%%%%%%%%%
%%%%%%%%%%%%%%%%%%%%%%%%%%%%%%%%%%%%%%%%%%%%%%%%%%%%%%%%%%%%%%%%%%%%%%%%%%%%%%%%
%%%%%%%%%%%%%%%%%%%%%%%%%%%%%%%%%%%%%%%%%%%%%%%%%%%%%%%%%%%%%%%%%%%%%%%%%%%%%%%%
\appendix

\settowidth\MacroIndent{\rmfamily\scriptsize 000\ }

 \DocInput{childdoc.dtx}

\end{document}
%</driver>
% \fi
%
% %%%%%%%%%%%%%%%%%%%%%%%%%%%%%%%%%%%%%%%%%%%%%%%%%%%%%%%%%%%%%%%%%%%%%%%%%%%%%%
% %%%%%%%%%%%%%%%%%%%%%%%%%%%%%%%%%%%%%%%%%%%%%%%%%%%%%%%%%%%%%%%%%%%%%%%%%%%%%%
% \section{Sample}
%\iffalse
%<*samplemain>
%\fi
%
% The following presents a sample document
% with two chapters, two parts, a title page,
% a compile flag as well as three forwarding files to set the flag.
% It consists of eight |.tex| files:
% \begin{center}
% \begin{tabular}{ll}
% |cdocsamp.tex|&main file\\
% |cdocsch1.tex|&include file for chapter 1\\
% |cdocsch2.tex|&include file for chapter 2\\
% |cdocspt3.tex|&include file for part 3\\
% |cdocspt4.tex|&include file for part 4\\
% |cdocsdrf.tex|&forwarding file for main file in draft mode\\
% |cdocsfi1.tex|&forwarding file for final version of chapter 1\\
% |cdocsfi2.tex|&forwarding file for final version of chapter 2\\
% \end{tabular}
% \end{center}
% Each of the eight files can be compiled directly by the \LaTeX{} compiler.
%
% %%%%%%%%%%%%%%%%%%%%%%%%%%%%%%%%%%%%%%
% \paragraph{Main File.}
%
% The main file is called |cdocsamp.tex|.
%
% Load the \textsf{childdoc} definitions and
% declare the filename for the main document:
%    \begin{macrocode}
\input{childdoc.def}
\childdocmain{}
%    \end{macrocode}

% Optional override for |\version| flag:
%    \begin{macrocode}
%%\ifchilddoc\else\providecommand{\version}{draft}\fi
%    \end{macrocode}

% Define the default values for the |\version| flag
% (|final| for the main file and |draft| for childs):
%    \begin{macrocode}
\ifchilddoc
\providecommand{\version}{draft}
\else
\providecommand{\version}{final}
\fi
%    \end{macrocode}

% Load the standard document class:
%    \begin{macrocode}
\documentclass[12pt]{article}
%    \end{macrocode}

% Start the document body:
%    \begin{macrocode}
\begin{document}
%    \end{macrocode}

% Declare a title page.
% Print title, part of document being processed and version flag:
%    \begin{macrocode}
\addtocounter{page}{-1}
\begin{center}
{\LARGE\bfseries{}childdoc example\par}
\vspace{1cm}
\ifchilddoc
\ifchilddocmanual part\else chapter\fi:
`\childdocname' of `\childdocjob'\par
\else
main document: `\childdocjob'\par
\fi
version: \version\par
\end{center}
\newpage
%    \end{macrocode}

% Manually include selected file,
% otherwise process as usual:
%    \begin{macrocode}
\ifchilddocmanual
\section*{part `\childdocname'}
\input{\childdocname}
\else
%    \end{macrocode}

% Include the two chapters:
%    \begin{macrocode}
\include{cdocsch1}
\include{cdocsch2}
%    \end{macrocode}

% Include the two parts unless only chapters should be displayed:
%    \begin{macrocode}
\ifchilddoc\else
\section{part three}
\input{cdocspt3}
\section{part four}
\input{cdocspt4}
\fi
%    \end{macrocode}

% Process as usual until here:
%    \begin{macrocode}
\fi
%    \end{macrocode}

% End of document body:
%    \begin{macrocode}
\end{document}
%    \end{macrocode}
%\iffalse
%</samplemain>
%\fi
%
% %%%%%%%%%%%%%%%%%%%%%%%%%%%%%%%%%%%%%%
% \paragraph{Chapter Include Files.}
%
% The include files are called |cdocsch1.tex| and |cdocsch2.tex|.
%
%\iffalse
%<*samplechap1|samplechap2>
%\fi

% Optional override for |\version| flag:
%    \begin{macrocode}
%%\providecommand{\version}{final}
%    \end{macrocode}

% Include the main document:
%    \begin{macrocode}
\input{childdoc.def}
\childdocof{cdocsamp}
%    \end{macrocode}

%\iffalse
%</samplechap1|samplechap2>
%\fi
%
%\iffalse
%<*samplechap1>
%\fi
% Some text for chapter 1:
%    \begin{macrocode}
\section{one}
some text in chapter one
%    \end{macrocode}

%\iffalse
%</samplechap1>
%\fi
% Some text for chapter 2:
%\iffalse
%<*samplechap2>
%\fi
%    \begin{macrocode}
\section{two}
more text in chapter two
%    \end{macrocode}

%\iffalse
%</samplechap2>
%\fi
%
% %%%%%%%%%%%%%%%%%%%%%%%%%%%%%%%%%%%%%%
% \paragraph{Part Include Files.}
%
% The include files are called |cdocspt3.tex| and |cdocspt4.tex|.
%
%\iffalse
%<*samplepart3|samplepart4>
%\fi

% Optional override for |\version| flag:
%    \begin{macrocode}
%%\providecommand{\version}{final}
%    \end{macrocode}

% Include the main document:
%    \begin{macrocode}
\input{childdoc.def}
\childdocby{cdocsamp}
%    \end{macrocode}

%\iffalse
%</samplepart3|samplepart4>
%\fi
%
%\iffalse
%<*samplepart3>
%\fi
% Some text for part 3:
%    \begin{macrocode}
some text in part three
%    \end{macrocode}

%\iffalse
%</samplepart3>
%\fi
% Some text for part 4:
%\iffalse
%<*samplepart4>
%\fi
%    \begin{macrocode}
more text in part four
%    \end{macrocode}

%\iffalse
%</samplepart4>
%\fi
%
% %%%%%%%%%%%%%%%%%%%%%%%%%%%%%%%%%%%%%%
% \paragraph{Forwarding for a Complete Draft.}
%
% The following forwarding file |cdocsdrf.tex|
% compiles the main document in draft mode:
%\iffalse
%<*sampledraft>
%\fi
%    \begin{macrocode}
\def\version{draft}
\input{childdoc.def}
\childdocforward{cdocsamp}
%    \end{macrocode}

%\iffalse
%</sampledraft>
%\fi
%
% %%%%%%%%%%%%%%%%%%%%%%%%%%%%%%%%%%%%%%
% \paragraph{Forwarding for Final Version of the Chapters.}
%
% The following forwarding files |cdocsfn1.tex| and |cdocsfn2.tex|
% (with identical content)
% compile the final versions of the child documents
% |cdocsch1.tex| and |cdocsch2.tex|, respectively:
%\iffalse
%<*samplefinal>
%\fi
%    \begin{macrocode}
\def\version{final}
\input{childdoc.def}
\childdocforwardprefix[cdocsamp]{cdocsfn}{cdocsch}
%    \end{macrocode}

%\iffalse
%</samplefinal>
%\fi
%
% %%%%%%%%%%%%%%%%%%%%%%%%%%%%%%%%%%%%%%
% \paragraph{Command Line Processing.}
%
% The following three command lines generate the output files
% |cdocscld|, |cdocscl1| and |cdocscl2|
% which should be identical to
% |cdocsdrf|, |cdocsch1| and |cdocsfn2|, respectively:
% \begin{center}
% \begin{tabular}{l}
% |latex -jobname cdocscld \|\\
% |  "\def\version{draft}\input{childdoc.def}\childdocforward{cdocsamp}"|\\
% |latex -jobname cdocscl1 \|\\
% |  "\input{childdoc.def}\childdocforward[cdocsamp]{cdocsch1}"|\\
% |latex -jobname cdocscl2 \|\\
% |  "\def\version{final}\input{childdoc.def}\childdocforward{cdocsch2}"|
% \end{tabular}
% \end{center}
% Note that the trailing backslash on each first line
% merely continues the input to the second line
% (for convenient cut ant paste).
% Furthermore, the command |latex| can be replaced by any
% of its alternative versions such as |pdflatex|.
%
% %%%%%%%%%%%%%%%%%%%%%%%%%%%%%%%%%%%%%%%%%%%%%%%%%%%%%%%%%%%%%%%%%%%%%%%%%%%%%%
% %%%%%%%%%%%%%%%%%%%%%%%%%%%%%%%%%%%%%%%%%%%%%%%%%%%%%%%%%%%%%%%%%%%%%%%%%%%%%%
% \section{Implementation}
%\iffalse
%<*package>
%\fi
%
% This section describes the definitions file |childdoc.def|.

% The definitions cannot be loaded using |\usepackage| or |\RequirePackage|
% which has a mechanism to prevent loading a style file more than once.
% When loading the definitions by means of |\input|
% multiple instances have to be prevented manually:
%\iffalse
%This code needs to be before the `\ProvidesFile' directive
%which is defined at the beginning of this file.
%Therefore it is also placed there and commented out here.
%</package>
%<*discard>
%\fi
%    \begin{macrocode}
\ifdefined\childdocmain\endinput\fi
%    \end{macrocode}
%\iffalse
%</discard>
%<*package>
%\fi
%
% \macro{\ifchilddoc}
% \macro{\ifchilddocmanual}
% The conditional |\ifchilddoc| tells whether a
% child (true) or main (false) document is being compiled.
% The conditional |\ifchilddocmanual| tells whether
% the |\includeonly| mechanism is used (false) or
% the selection of child files must be performed manually (true).
% The definitions initialise to false:
%    \begin{macrocode}
\newif\ifchilddoc
\newif\ifchilddocmanual
%    \end{macrocode}

% \macro{\childdocname}
% \macro{\childdocjob}
% The macro |\childdocname| stores the name of the main document
% to be compiled. The macro |\childdocjob| stores the name of
% the document on which the \LaTeX{} compiler was originally invoked.
% The content of |\jobname| cannot be compared
% to filenames specified in the source due to different catcodes.
% The following code rescans |\jobname|, stores the result
% in |\childdocname| and saves a copy in |\childdocjob|:
%    \begin{macrocode}
\edef\childdocname{\scantokens\expandafter{\jobname\noexpand}}
\let\childdocjob\childdocname
%    \end{macrocode}

% \macro{\childdocdisable}
% The macro |\childdocdisable| prevents the main file
% from being processed more than once.
% At this stage, the main document command |\childdocmain|
% is assumed to be called once again where it should do nothing.
% Any subsequent call to it should prevent
% a secondary processing of the main document
% It overwrites the forwarding commands
% |\childdocof| and |\childdocforward|
% with empty macros to prevent further inclusions of the main document:
%    \begin{macrocode}
\newcommand{\childdocdisable}
{
  \renewcommand{\childdocmain}[1]{\renewcommand{\childdocmain}[1]{\endinput}}
  \renewcommand{\childdocof}[1]{}
  \renewcommand{\childdocby}[2][]{}
  \renewcommand{\childdocforward}[2][]{}
  \renewcommand{\childdocdisable}{}
}
%    \end{macrocode}

% \macro{\childdocmain}
% The macro |\childdocmain| is to be called at the top of the main file
% with nothing or the main filename (without extension) as argument.
% First, it breaks loops.
% If the argument is not empty and does not match |\childdocname|
% (which is set by the first inclusion of |childdoc.def|),
% |\ifchilddoc| is set to true, |\includeonly| is applied to the child file
% and |\jobname| is set to the main file
% (for proper handling of |.aux| files):
%    \begin{macrocode}
\newcommand{\childdocmain}[1]
{
  \childdocdisable\childdocmain{}
  \if?#1?\else
    \begingroup
      \def\childdoctmp{#1}
      \ifx\childdoctmp\childdocname
        \def\childdoctmp{}
      \else
        \def\childdoctmp
        {
          \childdoctrue
          \includeonly{\childdocname}
          \def\childdocjob{#1}
          \def\jobname{#1}
        }
      \fi
      \expandafter
    \endgroup
    \childdoctmp
  \fi
}
%    \end{macrocode}

% \macro{\childdocof}
% The command |\childdocof| redirects
% compilation to the main file |#1|.
%    \begin{macrocode}
\newcommand{\childdocof}[1]
{
  \childdocdisable
  \childdoctrue
  \includeonly{\childdocname}
  \def\jobname{#1}
  \def\childdocjob{#1}
  \input{#1}
}
%    \end{macrocode}

% \macro{\childdocby}
% The command |\childdocby| ....
%    \begin{macrocode}
\newcommand{\childdocby}[2][]
{
  \childdocdisable
  \childdoctrue
  \childdocmanualtrue
  \if?#1?\else
    \def\jobname{#2}
  \fi
  \def\childdocjob{#2}
  \input{#2}
  \endinput
}
%    \end{macrocode}

% \macro{\childdocforward}
% The command |\childdocforward| redirects
% compilation to the main file or
% (if the optional argument is given) a child file.
% Parameters are set as if the main file
% or a child file starting with |\childdocof| was compiled.
% Then compilation is handed over to the main file:
%    \begin{macrocode}
\newcommand{\childdocforward}[2][]
{
  \begingroup
    \if?#1?
      \def\childdoctmp
      {
        \def\childdocname{#2}
        \def\childdocjob{#2}
        \def\jobname{#2}
        \input{#2}
        \endinput
      }
    \else
      \def\childdoctmp
      {
        \childdocdisable
        \def\childdocname{#2}
        \childdoctrue
        \includeonly{#2}
        \def\childdocjob{#1}
        \def\jobname{#1}
        \input{#1}
        \endinput
      }
    \fi
    \expandafter
  \endgroup
  \childdoctmp
}
%    \end{macrocode}

% \macro{\childdocforwardprefix}
% The command |\childdocforwardprefix| redirects
% compilation to the main or a child file by means of a pattern.
% The prefix |#1| in the current filename is replaced by |#2|
% and the suffix of the current filename is kept
% (it is assumed that the filename does not contain the substring `|~~~|'
% which is used as a delimiter).
% Compilation is handed over to the new file by |\childdocforward|:
%    \begin{macrocode}
\newcommand{\childdocforwardprefix}[3][]
{
  \begingroup
    \def\childdocextract #2##1~~~{\def\childdoctmp{\childdocforward[#1]{#3##1}}}
    \expandafter\childdocextract\childdocname~~~
    \expandafter
  \endgroup
  \childdoctmp
}
%    \end{macrocode}

% \macro{\childdoc}
% The deprecated macro |\childdoc| is a legacy version of |\childdocmain|:
%    \begin{macrocode}
\newcommand{\childdoc}{\childdocmain}
%    \end{macrocode}

% \macro{\childdocredirect}
% The deprecated macro |\childdocredirect| is a legacy version
% of |\childdocforward| and |\childdocforwardprefix|:
%    \begin{macrocode}
\newcommand{\childdocredirect}[2][]
{
  \begingroup
    \if?#1?
      \def\childdoctmp{\childdocforward{#2}}
    \else
      \def\childdoctmp{\childdocforwardprefix{#1}{#2}}
    \fi
    \expandafter
  \endgroup
  \childdoctmp
}
%    \end{macrocode}

%\iffalse
%</package>
%\fi
%
\endinput
|\\
|\childdocby{|\textit{main}|}|\\
\end{tabular}
\end{center}
%
Both forms have slightly different effects as described above.
The main file is prepared as usual, see \secref{sec:include}.

%%%%%%%%%%%%%%%%%%%%%%%%%%%%%%%%%%%%%%%%%%%%%%%%%%%%%%%%%%%%%%%%%%%%%%%%%%%%%%%%
\subsection{Legacy Detection}
\label{sec:detection}

The directive |\childdocmain| in the main file can detect
whether the complete document or merely a child is to be compiled
even without using the directive |\childdocof|.
This method is deprecated because it is less robust
and there is no compelling reason to use it;
it is merely provided for backward compatibility
and it may be removed in future versions.

If the detection mechanism is to be used,
it is mandatory to correctly specify
the filename of the main file as the argument of |\childdocmain|:
%
\begin{center}
\begin{tabular}{l}
|% \iffalse
%
% childdoc.dtx Copyright (C) 2017-2018 Niklas Beisert
%
% This work may be distributed and/or modified under the
% conditions of the LaTeX Project Public License, either version 1.3
% of this license or (at your option) any later version.
% The latest version of this license is in
%   http://www.latex-project.org/lppl.txt
% and version 1.3 or later is part of all distributions of LaTeX
% version 2005/12/01 or later.
%
% This work has the LPPL maintenance status `maintained'.
%
% The Current Maintainer of this work is Niklas Beisert.
%
% This work consists of the files childdoc.dtx and childdoc.ins
% and the derived files childdoc.def and cdocsamp.tex with
% cdocsch1.tex, cdocsch2.tex, cdocsdrf.tex, cdocsfn1.tex, cdocsfn2.tex.
%
%<package>\ifdefined\childdocmain\endinput\fi
%<package>\ProvidesFile{childdoc.def}[2018/12/30 v2.0 child document driver]
%<samplemain>\ProvidesFile{cdocsamp.tex}[2018/12/30 v2.0 sample for childdoc]
%<*driver>
%\ProvidesFile{childdoc.drv}[2018/12/30 v2.0 childdoc reference manual file]
\PassOptionsToClass{10pt,a4paper}{article}
\documentclass{ltxdoc}

\usepackage[margin=35mm]{geometry}
\usepackage{hyperref}
\usepackage{hyperxmp}
\usepackage[usenames]{color}

\hypersetup{colorlinks=true}
\hypersetup{pdfstartview=FitH}
\hypersetup{pdfpagemode=UseNone}
\hypersetup{pdfsource={}}
\hypersetup{pdflang={en-UK}}
\hypersetup{pdfcopyright={Copyright 2017-2018 Niklas Beisert.
  This work may be distributed and/or modified under the
  conditions of the LaTeX Project Public License, either version 1.3
  of this license or (at your option) any later version.}}
\hypersetup{pdflicenseurl={http://www.latex-project.org/lppl.txt}}
\hypersetup{pdfcontactaddress={ETH Zurich, ITP, HIT K,
  Wolfgang-Pauli-Strasse 27}}
\hypersetup{pdfcontactpostcode={8093}}
\hypersetup{pdfcontactcity={Zurich}}
\hypersetup{pdfcontactcountry={Switzerland}}
\hypersetup{pdfcontactemail={nbeisert@itp.phys.ethz.ch}}
\hypersetup{pdfcontacturl={http://people.phys.ethz.ch/\xmptilde nbeisert/}}

\newcommand{\secref}[1]{\hyperref[#1]{section \ref*{#1}}}

\parskip1ex
\parindent0pt
\let\olditemize\itemize
\def\itemize{\olditemize\parskip0pt}

\begin{document}

\title{The \textsf{childdoc} Package}
\hypersetup{pdftitle={The childdoc Package}}
\author{Niklas Beisert\\[2ex]
  Institut f\"ur Theoretische Physik\\
  Eidgen\"ossische Technische Hochschule Z\"urich\\
  Wolfgang-Pauli-Strasse 27, 8093 Z\"urich, Switzerland\\[1ex]
  \href{mailto:nbeisert@itp.phys.ethz.ch}
  {\texttt{nbeisert@itp.phys.ethz.ch}}}
\hypersetup{pdfauthor={Niklas Beisert}}
\hypersetup{pdfsubject={Manual for the LaTeX2e Package childdoc}}
\date{30 December 2018, \textsf{v2.0}}
\maketitle

\begin{abstract}\noindent
\textsf{childdoc} is a \LaTeXe{} package
that enables the direct compilation
of document sections included by |\include|
to individual files.
\end{abstract}

\begingroup
\parskip0ex
\tableofcontents
\endgroup

%%%%%%%%%%%%%%%%%%%%%%%%%%%%%%%%%%%%%%%%%%%%%%%%%%%%%%%%%%%%%%%%%%%%%%%%%%%%%%%%
%%%%%%%%%%%%%%%%%%%%%%%%%%%%%%%%%%%%%%%%%%%%%%%%%%%%%%%%%%%%%%%%%%%%%%%%%%%%%%%%
\section{Introduction}

\LaTeX{} provides a mechanism to structure a large document (such as a book)
into a main file and several child files (containing the chapters)
using the |\include| command.
This mechanism is beneficial for documents
which span hundreds of pages in order to
make the source file(s) more manageable.
Moreover, compilation can be restricted to
selected child files by means of the |\includeonly| command.
The latter feature can be used to reduce the compilation time while editing
(this was significantly more useful in the earlier days of \LaTeX{})
or to generate a smaller document which is easier to navigate.
Another application of |\includeonly| is to generate
documents consisting of selected parts of the complete document.

However, there are a few drawbacks of the plain |\include| mechanism:
\begin{itemize}
\item
The child files cannot be compiled on their own,
they can only be compiled via the main file.
A naive editing environment
(such as a text editor with an option
to have the current file processed by \LaTeX)
may require one to switch to the main file before compiling;
attempting to compile the child file produces errors.
\item
The main file must be modified (each time)
to adjust the |\includeonly| command
to the present needs. This easily leaves the main file in a messy state.
\item
The generated document will always carry the filename
of the main document. This is inconvenient if
several child files are to be compiled and
to be kept for distribution.
\end{itemize}

The present package provides a simple interface
to make child files individually compilable by \LaTeX{}.
Compiling a child file then has the same effect as compiling
the main file with an |\includeonly| command
to select the appropriate child.
Moreover the generated document will carry the name of the child
rather than the main file.
This resolves all three above issues.

This feature is meant to make the editing of books,
thesis documents and lecture notes somewhat more convenient.
However, the package can also be used efficiently for
composing a series of documents (such as exercise sheets)
which are typically distributed individually.
It then assists the author in generating the individual documents
(potentially in different versions)
as well as a document containing the collected series.
Another application is in developing style files
or other kinds of included material
where compilation of the style file could redirect
to a sample or test file.

%%%%%%%%%%%%%%%%%%%%%%%%%%%%%%%%%%%%%%%%%%%%%%%%%%%%%%%%%%%%%%%%%%%%%%%%%%%%%%%%
%%%%%%%%%%%%%%%%%%%%%%%%%%%%%%%%%%%%%%%%%%%%%%%%%%%%%%%%%%%%%%%%%%%%%%%%%%%%%%%%
\section{Usage}

First of all, the package \textsf{childdoc} is \emph{not} a standard
\LaTeXe{} |.sty| style file! Therefore it needs to be invoked in
a non-standard way.

%%%%%%%%%%%%%%%%%%%%%%%%%%%%%%%%%%%%%%%%%%%%%%%%%%%%%%%%%%%%%%%%%%%%%%%%%%%%%%%%
\subsection{Included Files}
\label{sec:include}

%%%%%%%%%%%%%%%%%%%%%%%%%%%%%%%%%%%%%%%%
\DescribeMacro{\childdocmain}
To use the package, add the commands
\begin{center}
\begin{tabular}{l}
|\input{childdoc.def}|\\
|\childdocmain{}|\\
\end{tabular}
\end{center}
at the very top of the main \LaTeX{} file,
in particular \emph{before} the |\documentclass| statement!
The argument of |\childdocmain| should be left empty
(but it must be present).

%%%%%%%%%%%%%%%%%%%%%%%%%%%%%%%%%%%%%%%%
\DescribeMacro{\childdocof}
Furthermore, add the commands
\begin{center}
\begin{tabular}{l}
|\input{childdoc.def}|\\
|\childdocof{|\textit{main}|}|\\
\end{tabular}
\end{center}
at the top of every child file \textit{child}
which is included by |\include{|\textit{child}|}|
from within the main file
(or at least for those files to be compiled individually).
The argument \textit{main} must be the filename of the main file.

There are a couple of
considerations in setting up the main and child documents:

%%%%%%%%%%%%%%%%%%%%%%%%%%%%%%%%%%%%%%%%
\paragraph{Restrictions.}

Please note the following restrictions:
\begin{itemize}
\item
|\childdocmain| must be called with one argument \textit{main}
to ensure compatibility with earlier version of the package.
It must either be empty (|\childdocmain{}|)
or precisely match the filename of the main file in which it is specified.
See \secref{sec:detection} for further information.
\item
The filename \textit{main} must be specified without the |.tex| extension.
\item
The filename \textit{main} is case sensitive
(even in case-insensitive file systems)
due to internal string comparison.
\item
The argument \textit{main} should be fully expanded, it cannot be a macro.
\item
Subdirectories and special characters should be avoided in filenames.
\item
The command |\childdocmain{|\textit{main}|}| must be followed by a whitespace.
It should not be followed immediately by another command
or by a comment mark `|%|'.
This is because the \TeX{} parser reads the token immediately following
the argument of |\childdocmain| and puts it
at the beginning of every child section;
however, a white\-space is ignored.
\end{itemize}

%%%%%%%%%%%%%%%%%%%%%%%%%%%%%%%%%%%%%%%%
\paragraph{Content of Main File.}

It is advisable to place all content in the child files included by |\include|.
Any output contained in the main file will appear in all child documents
unless suppressed manually;
it cannot be suppressed automatically by the |\includeonly| directive
and thus should normally be avoided.
A method to include some content in the main file
by means of conditional processing is described in \secref{sec:conditional}.

%%%%%%%%%%%%%%%%%%%%%%%%%%%%%%%%%%%%%%%%
\paragraph{Page Numbering.}

When only a part of the document is compiled,
the appropriate numbering of pages
(as well as other status parameters)
is determined from the |.aux| files.
The latter contain information from previous passes.
However this information needs to propagate through
all intermediate child documents.
Therefore the page numbering in child documents may well
be inconsistent until the complete document is compiled at least once.

A useful (if unconventional) way to always ensure a consistent
page numbering is to restart the numbering in each child document
and denote the pages by `\textit{child}|.|\textit{page}'
where \textit{child} represents the chapter/section number of the child file.
This can be achieved by the command
|\numberwithin{page}{|\textit{child}|}|
of the \textsf{amsmath} package
where \textit{child} can be |chapter| or |section|
depending on the chosen structuring.
Alternatively, one can modify the macro |\thepage| appropriately
and reset the counter |page| at the start of each child file.

%%%%%%%%%%%%%%%%%%%%%%%%%%%%%%%%%%%%%%%%%%%%%%%%%%%%%%%%%%%%%%%%%%%%%%%%%%%%%%%%
\subsection{Conditional Processing}
\label{sec:conditional}

The package provides a mechanism to compile different versions
of a document. To customise the versions further some conditional processing
can come in handy to distinguish which version is being compiled.
The package provides two macros to describe the compilation context:

%%%%%%%%%%%%%%%%%%%%%%%%%%%%%%%%%%%%%%%%
\DescribeMacro{\ifchilddoc}
The conditional |\ifchilddoc| distinguishes between the compilation of
child documents and the main document:
%
\begin{center}
|\ifchilddoc |\textit{child-code}| |[|\||else |\textit{main-code}]| \||fi|
\end{center}

%%%%%%%%%%%%%%%%%%%%%%%%%%%%%%%%%%%%%%%%
\DescribeMacro{\childdocname}
\DescribeMacro{\childdocjob}
The macro |\childdocname| contains the filename (without extension)
of the main or child file being processed.
Note that |\childdocjob| will always contain the name of the main file.

%%%%%%%%%%%%%%%%%%%%%%%%%%%%%%%%%%%%%%%%
\paragraph{Title Page.}

Conditional processing can be used to include a title or banner page
in the main document when proper precautions are taken.
Importantly, the code in the main file should ensure that the page counter
(as well as other status parameters which are stored in the |.aux| files)
takes the same value after the conditional processing.
Otherwise the page numbers may take divergent values
depending on which part is compiled.

For example, a title page could be declared by:
%
\begin{center}
\begin{tabular}{l}
|\ifchilddoc\||else|\\
|\addtocounter{page}{-1}|\\
\textit{code for title page}\\
|\newpage|\\
|\||fi|
\end{tabular}
\end{center}
%
A banner page for the child documents can be generated by:
%
\begin{center}
\begin{tabular}{l}
|\ifchilddoc|\\
|\addtocounter{page}{-1}|\\
\textit{code for banner page}\\
|\newpage|\\
|\||fi|
\end{tabular}
\end{center}
%
Here one could write a message such as:
\begin{center}
|This is the part \childdocname{} of \childdocjob{}.|
\end{center}

%%%%%%%%%%%%%%%%%%%%%%%%%%%%%%%%%%%%%%%%%%%%%%%%%%%%%%%%%%%%%%%%%%%%%%%%%%%%%%%%
\subsection{Flags}
\label{sec:flags}

The package makes it easy to generate different versions
of the main or child documents.
To this end compilation flags can be defined
and assigned different default values.
They will be particularly useful in conjunction
with the forwarding mechanism described in \secref{sec:forward}.

For example, it may be useful to have a flag |\version|
which can be set to |draft| or |final|.
The document source will contain some conditional code
depending on the value of |\version|.
Suppose further, the flag should default to |final| for the main file
and to |draft| for child files
which is a natural assignment for editing the document.
This is achieved by placing the following code
in the preamble of the main document
(below the |\childdocmain| directive):
%
\begin{center}
\begin{tabular}{l}
|\ifchilddoc|\\
|\providecommand{\version}{draft}|\\
|\||else|\\
|\providecommand{\version}{final}|\\
|\||fi|
\end{tabular}
\end{center}
%
The definition by |\providecommand| makes sure
that previous definitions are not overwritten.
Further statements |\providecommand{\version}{...}|
can thus be added before the above code to override it.

For the main file, one might add a line
(between |\childdocmain| and the above block)
%
\begin{center}
|%\ifchilddoc\||else\providecommand{\version}{draft}\||fi|
\end{center}
%
which can be uncommented to produce a draft version.
Likewise one can add a line to the very top of a child file
(above the |\childdocof{|\textit{main}|}| directive)
%
\begin{center}
|%\providecommand{\version}{final}|
\end{center}
%
which can be uncommented to produce the final version of this child document.

%%%%%%%%%%%%%%%%%%%%%%%%%%%%%%%%%%%%%%%%%%%%%%%%%%%%%%%%%%%%%%%%%%%%%%%%%%%%%%%%
\subsection{Forwarding}
\label{sec:forward}

Different versions of the main or child documents
using compilation flags as described in \secref{sec:flags}
can be (permanently) stored in different files
for convenient compilation, viewing and distribution.
To this end, the package defines a command
to pass on compilation to a different file:

%%%%%%%%%%%%%%%%%%%%%%%%%%%%%%%%%%%%%%%%
\DescribeMacro{\childdocforward}
The command |\childdocforward| redirects processing to
another source file:
%
\begin{center}
\begin{tabular}{l}
|\input{childdoc.def}|\\
|\childdocforward[|\textit{main}|]{|\textit{dest}|}|\\
\end{tabular}
\end{center}
%
The argument \textit{dest} is the destination file
(without extension).
It should be the main file or one of the child files.
Note that further \textsf{childdoc} directives
such as |\childdocof| and |\childdocforward|
in the indicated file will be processed in this form.
The optional argument \textit{main}
passes on directly to the main file \textit{main}
while pretending to compile the child \textit{dest}.
This form behaves as if \textit{dest}
issues |\childdocof{|\textit{main}|}| right away,
and no further \textsf{childdoc} directives will be processed.

%%%%%%%%%%%%%%%%%%%%%%%%%%%%%%%%%%%%%%%%
\DescribeMacro{\...prefix}
In the alternative form |\childdocforwardprefix|,
%
\begin{center}
\begin{tabular}{l}
|\input{childdoc.def}|\\
|\childdocforwardprefix[|\textit{main}|]{|\textit{prefix}|}{|\textit{dest}|}|
\end{tabular}
\end{center}
%
the destination file is determined by a pattern
depending on the current file:
To make this work, the current file must be called
`{\textit{prefix}\hspace{0.2em}\textit{suffix}}'
with \textit{prefix} matching precisely the argument.
Processing is then passed on to the file
`{\textit{dest}\hspace{0.2em}\textit{suffix}}'.
Surely, the same effect is achieved by
directly specifying the
argument `{\textit{dest}\hspace{0.2em}\textit{suffix}}'
in the first form.
However, that requires to set up a different file
for each child. With the alternative form of the command
all these files can have exactly the same content
which simplifies setting them up and maintaining them.

For example, the following file |draft.tex|
with a compilation flag |\version| as described in \secref{sec:flags}
compiles the main document as a draft:
%
\begin{center}
\begin{tabular}{l}
|\def\version{draft}|\\
|\input{childdoc.def}|\\
|\childdocforward{|\textit{main}|}|
\end{tabular}
\end{center}
%
Likewise, the following files |final|\textit{nn}|.tex|
compile the final version of the child document
|child|\textit{nn}|.tex|:
%
\begin{center}
\begin{tabular}{l}
|\def\version{final}|\\
|\input{childdoc.def}|\\
|\childdocforwardprefix{final}{child}|
\end{tabular}
\end{center}
%

Note that when several versions of a main file and/or of each child file
are to be generated, it may be convenient to set up a |Makefile| or
shell script to automatise the process.

%%%%%%%%%%%%%%%%%%%%%%%%%%%%%%%%%%%%%%%%%%%%%%%%%%%%%%%%%%%%%%%%%%%%%%%%%%%%%%%%
\subsection{Command Line Processing}
\label{sec:commandline}

The effect of redirection files can also be achieved by invoking
the \LaTeX{} compiler with a more elaborate command line.
Most conveniently this should be done as part
of a shell script or a |Makefile|.

When using \textsf{childdoc} in the main file, the following
command lines effectively perform a redirection
(note that depending on the shell being used,
backslashes may have to be doubled: `|\|' $\to$ `|\\|'):
%
\begin{center}
|... -jobname "|\textit{target}|" |\\|"|[\textit{flags}]%
|\input{childdoc.def}\childdocforward[|\textit{main}|]{|\textit{dest}|}"|
\end{center}
%
Here \textit{target} is the name of the output file,
\textit{main} is the name of the main file
and \textit{dest} is the name of the main or child file to be processed
(all filenames without extensions).
The optional argument \textit{main} can be omitted
if \textit{main} matches \textit{dest}.
Optionally, compilation \textit{flags} can be defined via |\def| commands.
This command line makes the \TeX{} engine believe
it is compiling the file \textit{target}
whose content is specified as the latter parameter.
The provided code then forwards the processing to
\textit{main} or \textit{dest} as described in \secref{sec:forward}.

%%%%%%%%%%%%%%%%%%%%%%%%%%%%%%%%%%%%%%%%%%%%%%%%%%%%%%%%%%%%%%%%%%%%%%%%%%%%%%%%
\subsection{Include by Input}
\label{sec:input}

Including child documents by |\include| has some restrictions by design.
Most notably, the content of a child document always occupies
its own set of pages; pages cannot be shared between child documents.
Usually, this behaviour makes perfect sense
because each child document contain an essential part of the document.
However, in some situations it may be desirable to compose
a document from a collection of parts
without having mandatory page breaks between then.
For this case, the package
provides a mechanism to include parts
by |\input| which can also be processed individually.
However, by construction this mechanism
requires manual handling of the content to be output.

%%%%%%%%%%%%%%%%%%%%%%%%%%%%%%%%%%%%%%%%
\DescribeMacro{\ifchilddocmanual}
The main file should be prepared as usual, see \secref{sec:include}.
However, the document body must make a distinction
between processing of an individual part and of the main document, e.g.:
%
\begin{center}
\begin{tabular}{l}
|\ifchilddocmanual|\\
|\input{\childdocname}|\\
|\||else|\\
\textit{document body with }|\input{|\textit{part}|}|\\
|\||fi|
\end{tabular}
\end{center}
%
The conditional |\ifchilddocmanual| is true whenever
a part to be included by |\input| is being compiled,
and the name of the part is stored in |\childdocname|.

%%%%%%%%%%%%%%%%%%%%%%%%%%%%%%%%%%%%%%%%
\DescribeMacro{\childdocby}
Each part to be included by |\input| should start with:
%
\begin{center}
\begin{tabular}{l}
|\input{childdoc.def}|\\
|\childdocby{|\textit{main}|}|\\
\end{tabular}
\end{center}
%
The directive |\childdocby| is similar to |\childdocof|
described in \secref{sec:include},
but the subsequent selection of content must be done manually.
To that end, both |\ifchilddoc| and |\ifchilddocmanual|
will be true upon processing of a part,
and the name of the part is stored in |\childdocname|.
Note that |\jobname| will be set to the filename of the current part
so that each part receives an individual |.aux| file
that does not interfere with the |.aux| file(s) of the main document.
This behaviour can be altered by the alternative form
|\childdocby[*]{|\textit{main}|}| (with a non-empty optional argument)
which uses the |.aux| file of the main document
by setting |\jobname| to \textit{main}.

%%%%%%%%%%%%%%%%%%%%%%%%%%%%%%%%%%%%%%%%%%%%%%%%%%%%%%%%%%%%%%%%%%%%%%%%%%%%%%%%
\subsection{Driver Development}
\label{sec:driver}

The \textsf{childdoc} mechanism can also be use for the development
of definition files such as \LaTeX{} styles or classes.
This case differs from the above setup with multiple parts
included by |\include| in that no |\includeonly| should be invoked.
This can be achieved by starting the include file
(before |\ProvidesPackage|) with:
%
\begin{center}
\begin{tabular}{l}
|\input{childdoc.def}|\\
|\childdocforward{|\textit{main}|}|\\
\end{tabular}
\end{center}
%
or alternatively with:
%
\begin{center}
\begin{tabular}{l}
|\input{childdoc.def}|\\
|\childdocby{|\textit{main}|}|\\
\end{tabular}
\end{center}
%
Both forms have slightly different effects as described above.
The main file is prepared as usual, see \secref{sec:include}.

%%%%%%%%%%%%%%%%%%%%%%%%%%%%%%%%%%%%%%%%%%%%%%%%%%%%%%%%%%%%%%%%%%%%%%%%%%%%%%%%
\subsection{Legacy Detection}
\label{sec:detection}

The directive |\childdocmain| in the main file can detect
whether the complete document or merely a child is to be compiled
even without using the directive |\childdocof|.
This method is deprecated because it is less robust
and there is no compelling reason to use it;
it is merely provided for backward compatibility
and it may be removed in future versions.

If the detection mechanism is to be used,
it is mandatory to correctly specify
the filename of the main file as the argument of |\childdocmain|:
%
\begin{center}
\begin{tabular}{l}
|\input{childdoc.def}|\\
|\childdocmain{|\textit{main}|}|\\
\end{tabular}
\end{center}
%
If |\jobname| does not match the argument \textit{main} of |\childdocmain|,
it is assumed that |\jobname| points to the child file to be compiled.
When using |\childdocmain| with the main file specified as argument,
it suffices to start a child file
with just |\input{|\textit{main}|}|
without loading of the package and using |\childdocof|.
If instead all processing is done
with the appropriate \textsf{childdoc} directives,
the argument of \textit{main} of |\childdocmain| can be empty.

An alternative version of the command line processing described
in \secref{sec:commandline} using the detection mechanism reads:
%
\begin{center}
|... -jobname "|\textit{target}|" "|[\textit{flags}]%
[|\def\jobname{|\textit{dest}|}|]|\input{|\textit{main}|}"|
\end{center}

%%%%%%%%%%%%%%%%%%%%%%%%%%%%%%%%%%%%%%%%%%%%%%%%%%%%%%%%%%%%%%%%%%%%%%%%%%%%%%%%
\subsection{Manual Code}
\label{sec:manual}

In case one cannot be certain whether the definitions file |childdoc.def|
is installed on the target \TeX{} distribution
and one prefers not to ship it,
it is conceivable to paste a few relevant commands into the sources.

To that end, drop all statements |\input{childdoc.def}|
and perform the replacements as outlined below.
Instead of |\childdocmain{|\textit{main}|}| add the following code
to the top of the main file:
%
\begin{center}
\begin{tabular}{l}
|\||ifdefined\childdocname\endinput\||fi\newif\ifchilddoc|\\
|\edef\childdocname{\scantokens\expandafter{\jobname\noexpand}}|\\
|\def\childdocmain{|\textit{main}|}\||ifx\childdocmain\childdocname\||else|\\
|\childdoctrue\includeonly{\childdocname}\let\jobname\childdocmain\||fi|\\
\end{tabular}
\end{center}
%
Instead of |\childdocof{|\textit{main}|}| just include the main file
at the top of each child file:
%
\begin{center}
|\input{|\textit{main}|}|
\end{center}
%
A simple redirection |\childdocforward{|\textit{dest}|}| is achieved by:
%
\begin{center}
|\def\jobname{|\textit{dest}|}\input{\jobname}|
\end{center}
%
The redirection with prefix
|\childdocforwardprefix[|\textit{prefix}|]{|\textit{dest}|}|
is accomplished by:
%
\begin{center}
\begin{tabular}{l}
|{\edef\jobname{\scantokens\expandafter{\jobname\noexpand}}|\\
|\def\redirectjob |\textit{prefix}|#1~~~{\gdef\jobname{|\textit{dest}|#1}}|\\
|\expandafter\redirectjob\jobname~~~}\input{\jobname}|
\end{tabular}
\end{center}

In an alternative approach,
child documents can be compiled by a specific command line
without additional code or specific definitions:
%
\begin{center}
|... -jobname "|\textit{target}|" "|[\textit{flags}]%
|\includeonly{|\textit{dest}|}\input{|\textit{main}|}"|
\end{center}
%

%%%%%%%%%%%%%%%%%%%%%%%%%%%%%%%%%%%%%%%%%%%%%%%%%%%%%%%%%%%%%%%%%%%%%%%%%%%%%%%%
%%%%%%%%%%%%%%%%%%%%%%%%%%%%%%%%%%%%%%%%%%%%%%%%%%%%%%%%%%%%%%%%%%%%%%%%%%%%%%%%
\section{Information}

%%%%%%%%%%%%%%%%%%%%%%%%%%%%%%%%%%%%%%%%%%%%%%%%%%%%%%%%%%%%%%%%%%%%%%%%%%%%%%%%
\subsection{Copyright}

Copyright \copyright{} 2017--2018 Niklas Beisert

This work may be distributed and/or modified under the
conditions of the \LaTeX{} Project Public License, either version 1.3
of this license or (at your option) any later version.
The latest version of this license is in
  \url{http://www.latex-project.org/lppl.txt}
and version 1.3 or later is part of all distributions of \LaTeX{}
version 2005/12/01 or later.

This work has the LPPL maintenance status `maintained'.

The Current Maintainer of this work is Niklas Beisert.

This work consists of the files |README.txt|, |childdoc.ins| and |childdoc.dtx|
as well as the derived files |childdoc.def|, |cdocsamp.tex|
with |cdocsch1.tex|, |cdocsch2.tex|, |cdocspt3.tex|, |cdocspt4.tex|,
|cdocsdrf.tex|, |cdocsfn1.tex|, |cdocsfn2.tex|
as well as |childdoc.pdf|.

%%%%%%%%%%%%%%%%%%%%%%%%%%%%%%%%%%%%%%%%%%%%%%%%%%%%%%%%%%%%%%%%%%%%%%%%%%%%%%%%
\subsection{Files and Installation}

The package consists of the files:
%
\begin{center}
\begin{tabular}{ll}
    |README.txt|   & readme file \\
    |childdoc.ins| & installation file \\
    |childdoc.dtx| & source file \\
    |childdoc.def| & definition file \\
    |cdocsamp.tex| & sample main file \\
    |cdocsch1.tex| & sample include file \\
    |cdocsch2.tex| & sample include file \\
    |cdocspt3.tex| & sample part file \\
    |cdocspt4.tex| & sample part file \\
    |cdocsdrf.tex| & sample redirection file \\
    |cdocsfn1.tex| & sample redirection file \\
    |cdocsfn2.tex| & sample redirection file \\
    |childdoc.pdf| & manual
\end{tabular}
\end{center}
%
The distribution consists of the files
|README.txt|, |childdoc.ins| and |childdoc.dtx|.
%
\begin{itemize}
\item
Run (pdf)\LaTeX{} on |childdoc.dtx|
to compile the manual |childdoc.pdf| (this file).
\item
Run \LaTeX{} on |childdoc.ins| to create the definitions file |childdoc.def|
and the sample |cdocsamp.tex| with include files
|cdocsch1.tex|, |cdocsch2.tex|, |cdocspt3.tex|, |cdocspt4.tex|,
|cdocsdrf.tex|, |cdocsfn1.tex|, |cdocsfn2.tex|.
Then copy the file |childdoc.def| to an appropriate directory of your \LaTeX{}
distribution, e.g.\ \textit{texmf-root}|/tex/latex/childdoc|.
\end{itemize}

%%%%%%%%%%%%%%%%%%%%%%%%%%%%%%%%%%%%%%%%%%%%%%%%%%%%%%%%%%%%%%%%%%%%%%%%%%%%%%%%
\subsection{Related CTAN Packages}

There are several other packages which offer a similar functionality:
%
\begin{itemize}
\item
The packages
\href{http://ctan.org/pkg/docmute}{\textsf{docmute}},
\href{http://ctan.org/pkg/includex}{\textsf{includex}} and
\href{http://ctan.org/pkg/standalone}{\textsf{standalone}}
provide commands to include only the document body of
a child file thus allowing both files to be compiled individually.
\item
The packages \href{http://ctan.org/pkg/subdocs}{\textsf{subdocs}}
and \href{http://ctan.org/pkg/subfiles}{\textsf{subfiles}}
provide structures in which the main and child documents can be
encapsulated and allowing them to be compiled individually.
The inclusion mechanism is different from the conventional |\include|.
\item
The package \href{http://ctan.org/pkg/combine}{\textsf{combine}}
is an elaborate solution to combine several documents into one.
\end{itemize}
%
See also the CTAN topic \href{http://ctan.org/topic/subdocs}{\textsf{subdocs}}
for further related packages.
The present package differs from the above solutions in that
a document structure constructed with the conventional |\include| mechanism
just needs two extra commands at the top of every file
such that all constituent files can be compiled individually.

%%%%%%%%%%%%%%%%%%%%%%%%%%%%%%%%%%%%%%%%%%%%%%%%%%%%%%%%%%%%%%%%%%%%%%%%%%%%%%%%
%\subsection{Feature Suggestions}
%
%The following is a list of features which may be useful for future
%versions of this package:
%%
%\begin{itemize}
%\item
%\ldots
%\end{itemize}

%%%%%%%%%%%%%%%%%%%%%%%%%%%%%%%%%%%%%%%%%%%%%%%%%%%%%%%%%%%%%%%%%%%%%%%%%%%%%%%%
\subsection{Revision History}

%%%%%%%%%%%%%%%%%%%%%%%%%%%%%%%%%%%%%%%%
\paragraph{v2.0:} 2018/12/30

\begin{itemize}
\item
immediate forward processing
\item
added |\childdocby| mechanism
\item
manual restructured
\end{itemize}

%%%%%%%%%%%%%%%%%%%%%%%%%%%%%%%%%%%%%%%%
\paragraph{v1.6:} 2018/01/17

\begin{itemize}
\item
application for development of include files
\item
corrections to manual
\end{itemize}

%%%%%%%%%%%%%%%%%%%%%%%%%%%%%%%%%%%%%%%%
\paragraph{v1.5:} 2017/05/21

\begin{itemize}
\item
more complete structuring introduced
\item
|\childdocof| introduced
\item
|\childdoc| renamed to |\childdocmain|
\item
|\childredirect| renamed to |\childdocforward| and |\childdocforwardprefix|
and functionality expanded
\end{itemize}

%%%%%%%%%%%%%%%%%%%%%%%%%%%%%%%%%%%%%%%%
\paragraph{v1.0:} 2017/04/27

\begin{itemize}
\item
manual and install package
\item
first version published on CTAN
\end{itemize}

%%%%%%%%%%%%%%%%%%%%%%%%%%%%%%%%%%%%%%%%
\paragraph{v0.6:} 2017/04/26

\begin{itemize}
\item
redirection mechanism added
\end{itemize}

%%%%%%%%%%%%%%%%%%%%%%%%%%%%%%%%%%%%%%%%
\paragraph{v0.5:} 2017/04/26

\begin{itemize}
\item
functionality in definition file
\end{itemize}


%%%%%%%%%%%%%%%%%%%%%%%%%%%%%%%%%%%%%%%%%%%%%%%%%%%%%%%%%%%%%%%%%%%%%%%%%%%%%%%%
%%%%%%%%%%%%%%%%%%%%%%%%%%%%%%%%%%%%%%%%%%%%%%%%%%%%%%%%%%%%%%%%%%%%%%%%%%%%%%%%
%%%%%%%%%%%%%%%%%%%%%%%%%%%%%%%%%%%%%%%%%%%%%%%%%%%%%%%%%%%%%%%%%%%%%%%%%%%%%%%%
\appendix

\settowidth\MacroIndent{\rmfamily\scriptsize 000\ }

 \DocInput{childdoc.dtx}

\end{document}
%</driver>
% \fi
%
% %%%%%%%%%%%%%%%%%%%%%%%%%%%%%%%%%%%%%%%%%%%%%%%%%%%%%%%%%%%%%%%%%%%%%%%%%%%%%%
% %%%%%%%%%%%%%%%%%%%%%%%%%%%%%%%%%%%%%%%%%%%%%%%%%%%%%%%%%%%%%%%%%%%%%%%%%%%%%%
% \section{Sample}
%\iffalse
%<*samplemain>
%\fi
%
% The following presents a sample document
% with two chapters, two parts, a title page,
% a compile flag as well as three forwarding files to set the flag.
% It consists of eight |.tex| files:
% \begin{center}
% \begin{tabular}{ll}
% |cdocsamp.tex|&main file\\
% |cdocsch1.tex|&include file for chapter 1\\
% |cdocsch2.tex|&include file for chapter 2\\
% |cdocspt3.tex|&include file for part 3\\
% |cdocspt4.tex|&include file for part 4\\
% |cdocsdrf.tex|&forwarding file for main file in draft mode\\
% |cdocsfi1.tex|&forwarding file for final version of chapter 1\\
% |cdocsfi2.tex|&forwarding file for final version of chapter 2\\
% \end{tabular}
% \end{center}
% Each of the eight files can be compiled directly by the \LaTeX{} compiler.
%
% %%%%%%%%%%%%%%%%%%%%%%%%%%%%%%%%%%%%%%
% \paragraph{Main File.}
%
% The main file is called |cdocsamp.tex|.
%
% Load the \textsf{childdoc} definitions and
% declare the filename for the main document:
%    \begin{macrocode}
\input{childdoc.def}
\childdocmain{}
%    \end{macrocode}

% Optional override for |\version| flag:
%    \begin{macrocode}
%%\ifchilddoc\else\providecommand{\version}{draft}\fi
%    \end{macrocode}

% Define the default values for the |\version| flag
% (|final| for the main file and |draft| for childs):
%    \begin{macrocode}
\ifchilddoc
\providecommand{\version}{draft}
\else
\providecommand{\version}{final}
\fi
%    \end{macrocode}

% Load the standard document class:
%    \begin{macrocode}
\documentclass[12pt]{article}
%    \end{macrocode}

% Start the document body:
%    \begin{macrocode}
\begin{document}
%    \end{macrocode}

% Declare a title page.
% Print title, part of document being processed and version flag:
%    \begin{macrocode}
\addtocounter{page}{-1}
\begin{center}
{\LARGE\bfseries{}childdoc example\par}
\vspace{1cm}
\ifchilddoc
\ifchilddocmanual part\else chapter\fi:
`\childdocname' of `\childdocjob'\par
\else
main document: `\childdocjob'\par
\fi
version: \version\par
\end{center}
\newpage
%    \end{macrocode}

% Manually include selected file,
% otherwise process as usual:
%    \begin{macrocode}
\ifchilddocmanual
\section*{part `\childdocname'}
\input{\childdocname}
\else
%    \end{macrocode}

% Include the two chapters:
%    \begin{macrocode}
\include{cdocsch1}
\include{cdocsch2}
%    \end{macrocode}

% Include the two parts unless only chapters should be displayed:
%    \begin{macrocode}
\ifchilddoc\else
\section{part three}
\input{cdocspt3}
\section{part four}
\input{cdocspt4}
\fi
%    \end{macrocode}

% Process as usual until here:
%    \begin{macrocode}
\fi
%    \end{macrocode}

% End of document body:
%    \begin{macrocode}
\end{document}
%    \end{macrocode}
%\iffalse
%</samplemain>
%\fi
%
% %%%%%%%%%%%%%%%%%%%%%%%%%%%%%%%%%%%%%%
% \paragraph{Chapter Include Files.}
%
% The include files are called |cdocsch1.tex| and |cdocsch2.tex|.
%
%\iffalse
%<*samplechap1|samplechap2>
%\fi

% Optional override for |\version| flag:
%    \begin{macrocode}
%%\providecommand{\version}{final}
%    \end{macrocode}

% Include the main document:
%    \begin{macrocode}
\input{childdoc.def}
\childdocof{cdocsamp}
%    \end{macrocode}

%\iffalse
%</samplechap1|samplechap2>
%\fi
%
%\iffalse
%<*samplechap1>
%\fi
% Some text for chapter 1:
%    \begin{macrocode}
\section{one}
some text in chapter one
%    \end{macrocode}

%\iffalse
%</samplechap1>
%\fi
% Some text for chapter 2:
%\iffalse
%<*samplechap2>
%\fi
%    \begin{macrocode}
\section{two}
more text in chapter two
%    \end{macrocode}

%\iffalse
%</samplechap2>
%\fi
%
% %%%%%%%%%%%%%%%%%%%%%%%%%%%%%%%%%%%%%%
% \paragraph{Part Include Files.}
%
% The include files are called |cdocspt3.tex| and |cdocspt4.tex|.
%
%\iffalse
%<*samplepart3|samplepart4>
%\fi

% Optional override for |\version| flag:
%    \begin{macrocode}
%%\providecommand{\version}{final}
%    \end{macrocode}

% Include the main document:
%    \begin{macrocode}
\input{childdoc.def}
\childdocby{cdocsamp}
%    \end{macrocode}

%\iffalse
%</samplepart3|samplepart4>
%\fi
%
%\iffalse
%<*samplepart3>
%\fi
% Some text for part 3:
%    \begin{macrocode}
some text in part three
%    \end{macrocode}

%\iffalse
%</samplepart3>
%\fi
% Some text for part 4:
%\iffalse
%<*samplepart4>
%\fi
%    \begin{macrocode}
more text in part four
%    \end{macrocode}

%\iffalse
%</samplepart4>
%\fi
%
% %%%%%%%%%%%%%%%%%%%%%%%%%%%%%%%%%%%%%%
% \paragraph{Forwarding for a Complete Draft.}
%
% The following forwarding file |cdocsdrf.tex|
% compiles the main document in draft mode:
%\iffalse
%<*sampledraft>
%\fi
%    \begin{macrocode}
\def\version{draft}
\input{childdoc.def}
\childdocforward{cdocsamp}
%    \end{macrocode}

%\iffalse
%</sampledraft>
%\fi
%
% %%%%%%%%%%%%%%%%%%%%%%%%%%%%%%%%%%%%%%
% \paragraph{Forwarding for Final Version of the Chapters.}
%
% The following forwarding files |cdocsfn1.tex| and |cdocsfn2.tex|
% (with identical content)
% compile the final versions of the child documents
% |cdocsch1.tex| and |cdocsch2.tex|, respectively:
%\iffalse
%<*samplefinal>
%\fi
%    \begin{macrocode}
\def\version{final}
\input{childdoc.def}
\childdocforwardprefix[cdocsamp]{cdocsfn}{cdocsch}
%    \end{macrocode}

%\iffalse
%</samplefinal>
%\fi
%
% %%%%%%%%%%%%%%%%%%%%%%%%%%%%%%%%%%%%%%
% \paragraph{Command Line Processing.}
%
% The following three command lines generate the output files
% |cdocscld|, |cdocscl1| and |cdocscl2|
% which should be identical to
% |cdocsdrf|, |cdocsch1| and |cdocsfn2|, respectively:
% \begin{center}
% \begin{tabular}{l}
% |latex -jobname cdocscld \|\\
% |  "\def\version{draft}\input{childdoc.def}\childdocforward{cdocsamp}"|\\
% |latex -jobname cdocscl1 \|\\
% |  "\input{childdoc.def}\childdocforward[cdocsamp]{cdocsch1}"|\\
% |latex -jobname cdocscl2 \|\\
% |  "\def\version{final}\input{childdoc.def}\childdocforward{cdocsch2}"|
% \end{tabular}
% \end{center}
% Note that the trailing backslash on each first line
% merely continues the input to the second line
% (for convenient cut ant paste).
% Furthermore, the command |latex| can be replaced by any
% of its alternative versions such as |pdflatex|.
%
% %%%%%%%%%%%%%%%%%%%%%%%%%%%%%%%%%%%%%%%%%%%%%%%%%%%%%%%%%%%%%%%%%%%%%%%%%%%%%%
% %%%%%%%%%%%%%%%%%%%%%%%%%%%%%%%%%%%%%%%%%%%%%%%%%%%%%%%%%%%%%%%%%%%%%%%%%%%%%%
% \section{Implementation}
%\iffalse
%<*package>
%\fi
%
% This section describes the definitions file |childdoc.def|.

% The definitions cannot be loaded using |\usepackage| or |\RequirePackage|
% which has a mechanism to prevent loading a style file more than once.
% When loading the definitions by means of |\input|
% multiple instances have to be prevented manually:
%\iffalse
%This code needs to be before the `\ProvidesFile' directive
%which is defined at the beginning of this file.
%Therefore it is also placed there and commented out here.
%</package>
%<*discard>
%\fi
%    \begin{macrocode}
\ifdefined\childdocmain\endinput\fi
%    \end{macrocode}
%\iffalse
%</discard>
%<*package>
%\fi
%
% \macro{\ifchilddoc}
% \macro{\ifchilddocmanual}
% The conditional |\ifchilddoc| tells whether a
% child (true) or main (false) document is being compiled.
% The conditional |\ifchilddocmanual| tells whether
% the |\includeonly| mechanism is used (false) or
% the selection of child files must be performed manually (true).
% The definitions initialise to false:
%    \begin{macrocode}
\newif\ifchilddoc
\newif\ifchilddocmanual
%    \end{macrocode}

% \macro{\childdocname}
% \macro{\childdocjob}
% The macro |\childdocname| stores the name of the main document
% to be compiled. The macro |\childdocjob| stores the name of
% the document on which the \LaTeX{} compiler was originally invoked.
% The content of |\jobname| cannot be compared
% to filenames specified in the source due to different catcodes.
% The following code rescans |\jobname|, stores the result
% in |\childdocname| and saves a copy in |\childdocjob|:
%    \begin{macrocode}
\edef\childdocname{\scantokens\expandafter{\jobname\noexpand}}
\let\childdocjob\childdocname
%    \end{macrocode}

% \macro{\childdocdisable}
% The macro |\childdocdisable| prevents the main file
% from being processed more than once.
% At this stage, the main document command |\childdocmain|
% is assumed to be called once again where it should do nothing.
% Any subsequent call to it should prevent
% a secondary processing of the main document
% It overwrites the forwarding commands
% |\childdocof| and |\childdocforward|
% with empty macros to prevent further inclusions of the main document:
%    \begin{macrocode}
\newcommand{\childdocdisable}
{
  \renewcommand{\childdocmain}[1]{\renewcommand{\childdocmain}[1]{\endinput}}
  \renewcommand{\childdocof}[1]{}
  \renewcommand{\childdocby}[2][]{}
  \renewcommand{\childdocforward}[2][]{}
  \renewcommand{\childdocdisable}{}
}
%    \end{macrocode}

% \macro{\childdocmain}
% The macro |\childdocmain| is to be called at the top of the main file
% with nothing or the main filename (without extension) as argument.
% First, it breaks loops.
% If the argument is not empty and does not match |\childdocname|
% (which is set by the first inclusion of |childdoc.def|),
% |\ifchilddoc| is set to true, |\includeonly| is applied to the child file
% and |\jobname| is set to the main file
% (for proper handling of |.aux| files):
%    \begin{macrocode}
\newcommand{\childdocmain}[1]
{
  \childdocdisable\childdocmain{}
  \if?#1?\else
    \begingroup
      \def\childdoctmp{#1}
      \ifx\childdoctmp\childdocname
        \def\childdoctmp{}
      \else
        \def\childdoctmp
        {
          \childdoctrue
          \includeonly{\childdocname}
          \def\childdocjob{#1}
          \def\jobname{#1}
        }
      \fi
      \expandafter
    \endgroup
    \childdoctmp
  \fi
}
%    \end{macrocode}

% \macro{\childdocof}
% The command |\childdocof| redirects
% compilation to the main file |#1|.
%    \begin{macrocode}
\newcommand{\childdocof}[1]
{
  \childdocdisable
  \childdoctrue
  \includeonly{\childdocname}
  \def\jobname{#1}
  \def\childdocjob{#1}
  \input{#1}
}
%    \end{macrocode}

% \macro{\childdocby}
% The command |\childdocby| ....
%    \begin{macrocode}
\newcommand{\childdocby}[2][]
{
  \childdocdisable
  \childdoctrue
  \childdocmanualtrue
  \if?#1?\else
    \def\jobname{#2}
  \fi
  \def\childdocjob{#2}
  \input{#2}
  \endinput
}
%    \end{macrocode}

% \macro{\childdocforward}
% The command |\childdocforward| redirects
% compilation to the main file or
% (if the optional argument is given) a child file.
% Parameters are set as if the main file
% or a child file starting with |\childdocof| was compiled.
% Then compilation is handed over to the main file:
%    \begin{macrocode}
\newcommand{\childdocforward}[2][]
{
  \begingroup
    \if?#1?
      \def\childdoctmp
      {
        \def\childdocname{#2}
        \def\childdocjob{#2}
        \def\jobname{#2}
        \input{#2}
        \endinput
      }
    \else
      \def\childdoctmp
      {
        \childdocdisable
        \def\childdocname{#2}
        \childdoctrue
        \includeonly{#2}
        \def\childdocjob{#1}
        \def\jobname{#1}
        \input{#1}
        \endinput
      }
    \fi
    \expandafter
  \endgroup
  \childdoctmp
}
%    \end{macrocode}

% \macro{\childdocforwardprefix}
% The command |\childdocforwardprefix| redirects
% compilation to the main or a child file by means of a pattern.
% The prefix |#1| in the current filename is replaced by |#2|
% and the suffix of the current filename is kept
% (it is assumed that the filename does not contain the substring `|~~~|'
% which is used as a delimiter).
% Compilation is handed over to the new file by |\childdocforward|:
%    \begin{macrocode}
\newcommand{\childdocforwardprefix}[3][]
{
  \begingroup
    \def\childdocextract #2##1~~~{\def\childdoctmp{\childdocforward[#1]{#3##1}}}
    \expandafter\childdocextract\childdocname~~~
    \expandafter
  \endgroup
  \childdoctmp
}
%    \end{macrocode}

% \macro{\childdoc}
% The deprecated macro |\childdoc| is a legacy version of |\childdocmain|:
%    \begin{macrocode}
\newcommand{\childdoc}{\childdocmain}
%    \end{macrocode}

% \macro{\childdocredirect}
% The deprecated macro |\childdocredirect| is a legacy version
% of |\childdocforward| and |\childdocforwardprefix|:
%    \begin{macrocode}
\newcommand{\childdocredirect}[2][]
{
  \begingroup
    \if?#1?
      \def\childdoctmp{\childdocforward{#2}}
    \else
      \def\childdoctmp{\childdocforwardprefix{#1}{#2}}
    \fi
    \expandafter
  \endgroup
  \childdoctmp
}
%    \end{macrocode}

%\iffalse
%</package>
%\fi
%
\endinput
|\\
|\childdocmain{|\textit{main}|}|\\
\end{tabular}
\end{center}
%
If |\jobname| does not match the argument \textit{main} of |\childdocmain|,
it is assumed that |\jobname| points to the child file to be compiled.
When using |\childdocmain| with the main file specified as argument,
it suffices to start a child file
with just |\input{|\textit{main}|}|
without loading of the package and using |\childdocof|.
If instead all processing is done
with the appropriate \textsf{childdoc} directives,
the argument of \textit{main} of |\childdocmain| can be empty.

An alternative version of the command line processing described
in \secref{sec:commandline} using the detection mechanism reads:
%
\begin{center}
|... -jobname "|\textit{target}|" "|[\textit{flags}]%
[|\def\jobname{|\textit{dest}|}|]|\input{|\textit{main}|}"|
\end{center}

%%%%%%%%%%%%%%%%%%%%%%%%%%%%%%%%%%%%%%%%%%%%%%%%%%%%%%%%%%%%%%%%%%%%%%%%%%%%%%%%
\subsection{Manual Code}
\label{sec:manual}

In case one cannot be certain whether the definitions file |childdoc.def|
is installed on the target \TeX{} distribution
and one prefers not to ship it,
it is conceivable to paste a few relevant commands into the sources.

To that end, drop all statements |% \iffalse
%
% childdoc.dtx Copyright (C) 2017-2018 Niklas Beisert
%
% This work may be distributed and/or modified under the
% conditions of the LaTeX Project Public License, either version 1.3
% of this license or (at your option) any later version.
% The latest version of this license is in
%   http://www.latex-project.org/lppl.txt
% and version 1.3 or later is part of all distributions of LaTeX
% version 2005/12/01 or later.
%
% This work has the LPPL maintenance status `maintained'.
%
% The Current Maintainer of this work is Niklas Beisert.
%
% This work consists of the files childdoc.dtx and childdoc.ins
% and the derived files childdoc.def and cdocsamp.tex with
% cdocsch1.tex, cdocsch2.tex, cdocsdrf.tex, cdocsfn1.tex, cdocsfn2.tex.
%
%<package>\ifdefined\childdocmain\endinput\fi
%<package>\ProvidesFile{childdoc.def}[2018/12/30 v2.0 child document driver]
%<samplemain>\ProvidesFile{cdocsamp.tex}[2018/12/30 v2.0 sample for childdoc]
%<*driver>
%\ProvidesFile{childdoc.drv}[2018/12/30 v2.0 childdoc reference manual file]
\PassOptionsToClass{10pt,a4paper}{article}
\documentclass{ltxdoc}

\usepackage[margin=35mm]{geometry}
\usepackage{hyperref}
\usepackage{hyperxmp}
\usepackage[usenames]{color}

\hypersetup{colorlinks=true}
\hypersetup{pdfstartview=FitH}
\hypersetup{pdfpagemode=UseNone}
\hypersetup{pdfsource={}}
\hypersetup{pdflang={en-UK}}
\hypersetup{pdfcopyright={Copyright 2017-2018 Niklas Beisert.
  This work may be distributed and/or modified under the
  conditions of the LaTeX Project Public License, either version 1.3
  of this license or (at your option) any later version.}}
\hypersetup{pdflicenseurl={http://www.latex-project.org/lppl.txt}}
\hypersetup{pdfcontactaddress={ETH Zurich, ITP, HIT K,
  Wolfgang-Pauli-Strasse 27}}
\hypersetup{pdfcontactpostcode={8093}}
\hypersetup{pdfcontactcity={Zurich}}
\hypersetup{pdfcontactcountry={Switzerland}}
\hypersetup{pdfcontactemail={nbeisert@itp.phys.ethz.ch}}
\hypersetup{pdfcontacturl={http://people.phys.ethz.ch/\xmptilde nbeisert/}}

\newcommand{\secref}[1]{\hyperref[#1]{section \ref*{#1}}}

\parskip1ex
\parindent0pt
\let\olditemize\itemize
\def\itemize{\olditemize\parskip0pt}

\begin{document}

\title{The \textsf{childdoc} Package}
\hypersetup{pdftitle={The childdoc Package}}
\author{Niklas Beisert\\[2ex]
  Institut f\"ur Theoretische Physik\\
  Eidgen\"ossische Technische Hochschule Z\"urich\\
  Wolfgang-Pauli-Strasse 27, 8093 Z\"urich, Switzerland\\[1ex]
  \href{mailto:nbeisert@itp.phys.ethz.ch}
  {\texttt{nbeisert@itp.phys.ethz.ch}}}
\hypersetup{pdfauthor={Niklas Beisert}}
\hypersetup{pdfsubject={Manual for the LaTeX2e Package childdoc}}
\date{30 December 2018, \textsf{v2.0}}
\maketitle

\begin{abstract}\noindent
\textsf{childdoc} is a \LaTeXe{} package
that enables the direct compilation
of document sections included by |\include|
to individual files.
\end{abstract}

\begingroup
\parskip0ex
\tableofcontents
\endgroup

%%%%%%%%%%%%%%%%%%%%%%%%%%%%%%%%%%%%%%%%%%%%%%%%%%%%%%%%%%%%%%%%%%%%%%%%%%%%%%%%
%%%%%%%%%%%%%%%%%%%%%%%%%%%%%%%%%%%%%%%%%%%%%%%%%%%%%%%%%%%%%%%%%%%%%%%%%%%%%%%%
\section{Introduction}

\LaTeX{} provides a mechanism to structure a large document (such as a book)
into a main file and several child files (containing the chapters)
using the |\include| command.
This mechanism is beneficial for documents
which span hundreds of pages in order to
make the source file(s) more manageable.
Moreover, compilation can be restricted to
selected child files by means of the |\includeonly| command.
The latter feature can be used to reduce the compilation time while editing
(this was significantly more useful in the earlier days of \LaTeX{})
or to generate a smaller document which is easier to navigate.
Another application of |\includeonly| is to generate
documents consisting of selected parts of the complete document.

However, there are a few drawbacks of the plain |\include| mechanism:
\begin{itemize}
\item
The child files cannot be compiled on their own,
they can only be compiled via the main file.
A naive editing environment
(such as a text editor with an option
to have the current file processed by \LaTeX)
may require one to switch to the main file before compiling;
attempting to compile the child file produces errors.
\item
The main file must be modified (each time)
to adjust the |\includeonly| command
to the present needs. This easily leaves the main file in a messy state.
\item
The generated document will always carry the filename
of the main document. This is inconvenient if
several child files are to be compiled and
to be kept for distribution.
\end{itemize}

The present package provides a simple interface
to make child files individually compilable by \LaTeX{}.
Compiling a child file then has the same effect as compiling
the main file with an |\includeonly| command
to select the appropriate child.
Moreover the generated document will carry the name of the child
rather than the main file.
This resolves all three above issues.

This feature is meant to make the editing of books,
thesis documents and lecture notes somewhat more convenient.
However, the package can also be used efficiently for
composing a series of documents (such as exercise sheets)
which are typically distributed individually.
It then assists the author in generating the individual documents
(potentially in different versions)
as well as a document containing the collected series.
Another application is in developing style files
or other kinds of included material
where compilation of the style file could redirect
to a sample or test file.

%%%%%%%%%%%%%%%%%%%%%%%%%%%%%%%%%%%%%%%%%%%%%%%%%%%%%%%%%%%%%%%%%%%%%%%%%%%%%%%%
%%%%%%%%%%%%%%%%%%%%%%%%%%%%%%%%%%%%%%%%%%%%%%%%%%%%%%%%%%%%%%%%%%%%%%%%%%%%%%%%
\section{Usage}

First of all, the package \textsf{childdoc} is \emph{not} a standard
\LaTeXe{} |.sty| style file! Therefore it needs to be invoked in
a non-standard way.

%%%%%%%%%%%%%%%%%%%%%%%%%%%%%%%%%%%%%%%%%%%%%%%%%%%%%%%%%%%%%%%%%%%%%%%%%%%%%%%%
\subsection{Included Files}
\label{sec:include}

%%%%%%%%%%%%%%%%%%%%%%%%%%%%%%%%%%%%%%%%
\DescribeMacro{\childdocmain}
To use the package, add the commands
\begin{center}
\begin{tabular}{l}
|\input{childdoc.def}|\\
|\childdocmain{}|\\
\end{tabular}
\end{center}
at the very top of the main \LaTeX{} file,
in particular \emph{before} the |\documentclass| statement!
The argument of |\childdocmain| should be left empty
(but it must be present).

%%%%%%%%%%%%%%%%%%%%%%%%%%%%%%%%%%%%%%%%
\DescribeMacro{\childdocof}
Furthermore, add the commands
\begin{center}
\begin{tabular}{l}
|\input{childdoc.def}|\\
|\childdocof{|\textit{main}|}|\\
\end{tabular}
\end{center}
at the top of every child file \textit{child}
which is included by |\include{|\textit{child}|}|
from within the main file
(or at least for those files to be compiled individually).
The argument \textit{main} must be the filename of the main file.

There are a couple of
considerations in setting up the main and child documents:

%%%%%%%%%%%%%%%%%%%%%%%%%%%%%%%%%%%%%%%%
\paragraph{Restrictions.}

Please note the following restrictions:
\begin{itemize}
\item
|\childdocmain| must be called with one argument \textit{main}
to ensure compatibility with earlier version of the package.
It must either be empty (|\childdocmain{}|)
or precisely match the filename of the main file in which it is specified.
See \secref{sec:detection} for further information.
\item
The filename \textit{main} must be specified without the |.tex| extension.
\item
The filename \textit{main} is case sensitive
(even in case-insensitive file systems)
due to internal string comparison.
\item
The argument \textit{main} should be fully expanded, it cannot be a macro.
\item
Subdirectories and special characters should be avoided in filenames.
\item
The command |\childdocmain{|\textit{main}|}| must be followed by a whitespace.
It should not be followed immediately by another command
or by a comment mark `|%|'.
This is because the \TeX{} parser reads the token immediately following
the argument of |\childdocmain| and puts it
at the beginning of every child section;
however, a white\-space is ignored.
\end{itemize}

%%%%%%%%%%%%%%%%%%%%%%%%%%%%%%%%%%%%%%%%
\paragraph{Content of Main File.}

It is advisable to place all content in the child files included by |\include|.
Any output contained in the main file will appear in all child documents
unless suppressed manually;
it cannot be suppressed automatically by the |\includeonly| directive
and thus should normally be avoided.
A method to include some content in the main file
by means of conditional processing is described in \secref{sec:conditional}.

%%%%%%%%%%%%%%%%%%%%%%%%%%%%%%%%%%%%%%%%
\paragraph{Page Numbering.}

When only a part of the document is compiled,
the appropriate numbering of pages
(as well as other status parameters)
is determined from the |.aux| files.
The latter contain information from previous passes.
However this information needs to propagate through
all intermediate child documents.
Therefore the page numbering in child documents may well
be inconsistent until the complete document is compiled at least once.

A useful (if unconventional) way to always ensure a consistent
page numbering is to restart the numbering in each child document
and denote the pages by `\textit{child}|.|\textit{page}'
where \textit{child} represents the chapter/section number of the child file.
This can be achieved by the command
|\numberwithin{page}{|\textit{child}|}|
of the \textsf{amsmath} package
where \textit{child} can be |chapter| or |section|
depending on the chosen structuring.
Alternatively, one can modify the macro |\thepage| appropriately
and reset the counter |page| at the start of each child file.

%%%%%%%%%%%%%%%%%%%%%%%%%%%%%%%%%%%%%%%%%%%%%%%%%%%%%%%%%%%%%%%%%%%%%%%%%%%%%%%%
\subsection{Conditional Processing}
\label{sec:conditional}

The package provides a mechanism to compile different versions
of a document. To customise the versions further some conditional processing
can come in handy to distinguish which version is being compiled.
The package provides two macros to describe the compilation context:

%%%%%%%%%%%%%%%%%%%%%%%%%%%%%%%%%%%%%%%%
\DescribeMacro{\ifchilddoc}
The conditional |\ifchilddoc| distinguishes between the compilation of
child documents and the main document:
%
\begin{center}
|\ifchilddoc |\textit{child-code}| |[|\||else |\textit{main-code}]| \||fi|
\end{center}

%%%%%%%%%%%%%%%%%%%%%%%%%%%%%%%%%%%%%%%%
\DescribeMacro{\childdocname}
\DescribeMacro{\childdocjob}
The macro |\childdocname| contains the filename (without extension)
of the main or child file being processed.
Note that |\childdocjob| will always contain the name of the main file.

%%%%%%%%%%%%%%%%%%%%%%%%%%%%%%%%%%%%%%%%
\paragraph{Title Page.}

Conditional processing can be used to include a title or banner page
in the main document when proper precautions are taken.
Importantly, the code in the main file should ensure that the page counter
(as well as other status parameters which are stored in the |.aux| files)
takes the same value after the conditional processing.
Otherwise the page numbers may take divergent values
depending on which part is compiled.

For example, a title page could be declared by:
%
\begin{center}
\begin{tabular}{l}
|\ifchilddoc\||else|\\
|\addtocounter{page}{-1}|\\
\textit{code for title page}\\
|\newpage|\\
|\||fi|
\end{tabular}
\end{center}
%
A banner page for the child documents can be generated by:
%
\begin{center}
\begin{tabular}{l}
|\ifchilddoc|\\
|\addtocounter{page}{-1}|\\
\textit{code for banner page}\\
|\newpage|\\
|\||fi|
\end{tabular}
\end{center}
%
Here one could write a message such as:
\begin{center}
|This is the part \childdocname{} of \childdocjob{}.|
\end{center}

%%%%%%%%%%%%%%%%%%%%%%%%%%%%%%%%%%%%%%%%%%%%%%%%%%%%%%%%%%%%%%%%%%%%%%%%%%%%%%%%
\subsection{Flags}
\label{sec:flags}

The package makes it easy to generate different versions
of the main or child documents.
To this end compilation flags can be defined
and assigned different default values.
They will be particularly useful in conjunction
with the forwarding mechanism described in \secref{sec:forward}.

For example, it may be useful to have a flag |\version|
which can be set to |draft| or |final|.
The document source will contain some conditional code
depending on the value of |\version|.
Suppose further, the flag should default to |final| for the main file
and to |draft| for child files
which is a natural assignment for editing the document.
This is achieved by placing the following code
in the preamble of the main document
(below the |\childdocmain| directive):
%
\begin{center}
\begin{tabular}{l}
|\ifchilddoc|\\
|\providecommand{\version}{draft}|\\
|\||else|\\
|\providecommand{\version}{final}|\\
|\||fi|
\end{tabular}
\end{center}
%
The definition by |\providecommand| makes sure
that previous definitions are not overwritten.
Further statements |\providecommand{\version}{...}|
can thus be added before the above code to override it.

For the main file, one might add a line
(between |\childdocmain| and the above block)
%
\begin{center}
|%\ifchilddoc\||else\providecommand{\version}{draft}\||fi|
\end{center}
%
which can be uncommented to produce a draft version.
Likewise one can add a line to the very top of a child file
(above the |\childdocof{|\textit{main}|}| directive)
%
\begin{center}
|%\providecommand{\version}{final}|
\end{center}
%
which can be uncommented to produce the final version of this child document.

%%%%%%%%%%%%%%%%%%%%%%%%%%%%%%%%%%%%%%%%%%%%%%%%%%%%%%%%%%%%%%%%%%%%%%%%%%%%%%%%
\subsection{Forwarding}
\label{sec:forward}

Different versions of the main or child documents
using compilation flags as described in \secref{sec:flags}
can be (permanently) stored in different files
for convenient compilation, viewing and distribution.
To this end, the package defines a command
to pass on compilation to a different file:

%%%%%%%%%%%%%%%%%%%%%%%%%%%%%%%%%%%%%%%%
\DescribeMacro{\childdocforward}
The command |\childdocforward| redirects processing to
another source file:
%
\begin{center}
\begin{tabular}{l}
|\input{childdoc.def}|\\
|\childdocforward[|\textit{main}|]{|\textit{dest}|}|\\
\end{tabular}
\end{center}
%
The argument \textit{dest} is the destination file
(without extension).
It should be the main file or one of the child files.
Note that further \textsf{childdoc} directives
such as |\childdocof| and |\childdocforward|
in the indicated file will be processed in this form.
The optional argument \textit{main}
passes on directly to the main file \textit{main}
while pretending to compile the child \textit{dest}.
This form behaves as if \textit{dest}
issues |\childdocof{|\textit{main}|}| right away,
and no further \textsf{childdoc} directives will be processed.

%%%%%%%%%%%%%%%%%%%%%%%%%%%%%%%%%%%%%%%%
\DescribeMacro{\...prefix}
In the alternative form |\childdocforwardprefix|,
%
\begin{center}
\begin{tabular}{l}
|\input{childdoc.def}|\\
|\childdocforwardprefix[|\textit{main}|]{|\textit{prefix}|}{|\textit{dest}|}|
\end{tabular}
\end{center}
%
the destination file is determined by a pattern
depending on the current file:
To make this work, the current file must be called
`{\textit{prefix}\hspace{0.2em}\textit{suffix}}'
with \textit{prefix} matching precisely the argument.
Processing is then passed on to the file
`{\textit{dest}\hspace{0.2em}\textit{suffix}}'.
Surely, the same effect is achieved by
directly specifying the
argument `{\textit{dest}\hspace{0.2em}\textit{suffix}}'
in the first form.
However, that requires to set up a different file
for each child. With the alternative form of the command
all these files can have exactly the same content
which simplifies setting them up and maintaining them.

For example, the following file |draft.tex|
with a compilation flag |\version| as described in \secref{sec:flags}
compiles the main document as a draft:
%
\begin{center}
\begin{tabular}{l}
|\def\version{draft}|\\
|\input{childdoc.def}|\\
|\childdocforward{|\textit{main}|}|
\end{tabular}
\end{center}
%
Likewise, the following files |final|\textit{nn}|.tex|
compile the final version of the child document
|child|\textit{nn}|.tex|:
%
\begin{center}
\begin{tabular}{l}
|\def\version{final}|\\
|\input{childdoc.def}|\\
|\childdocforwardprefix{final}{child}|
\end{tabular}
\end{center}
%

Note that when several versions of a main file and/or of each child file
are to be generated, it may be convenient to set up a |Makefile| or
shell script to automatise the process.

%%%%%%%%%%%%%%%%%%%%%%%%%%%%%%%%%%%%%%%%%%%%%%%%%%%%%%%%%%%%%%%%%%%%%%%%%%%%%%%%
\subsection{Command Line Processing}
\label{sec:commandline}

The effect of redirection files can also be achieved by invoking
the \LaTeX{} compiler with a more elaborate command line.
Most conveniently this should be done as part
of a shell script or a |Makefile|.

When using \textsf{childdoc} in the main file, the following
command lines effectively perform a redirection
(note that depending on the shell being used,
backslashes may have to be doubled: `|\|' $\to$ `|\\|'):
%
\begin{center}
|... -jobname "|\textit{target}|" |\\|"|[\textit{flags}]%
|\input{childdoc.def}\childdocforward[|\textit{main}|]{|\textit{dest}|}"|
\end{center}
%
Here \textit{target} is the name of the output file,
\textit{main} is the name of the main file
and \textit{dest} is the name of the main or child file to be processed
(all filenames without extensions).
The optional argument \textit{main} can be omitted
if \textit{main} matches \textit{dest}.
Optionally, compilation \textit{flags} can be defined via |\def| commands.
This command line makes the \TeX{} engine believe
it is compiling the file \textit{target}
whose content is specified as the latter parameter.
The provided code then forwards the processing to
\textit{main} or \textit{dest} as described in \secref{sec:forward}.

%%%%%%%%%%%%%%%%%%%%%%%%%%%%%%%%%%%%%%%%%%%%%%%%%%%%%%%%%%%%%%%%%%%%%%%%%%%%%%%%
\subsection{Include by Input}
\label{sec:input}

Including child documents by |\include| has some restrictions by design.
Most notably, the content of a child document always occupies
its own set of pages; pages cannot be shared between child documents.
Usually, this behaviour makes perfect sense
because each child document contain an essential part of the document.
However, in some situations it may be desirable to compose
a document from a collection of parts
without having mandatory page breaks between then.
For this case, the package
provides a mechanism to include parts
by |\input| which can also be processed individually.
However, by construction this mechanism
requires manual handling of the content to be output.

%%%%%%%%%%%%%%%%%%%%%%%%%%%%%%%%%%%%%%%%
\DescribeMacro{\ifchilddocmanual}
The main file should be prepared as usual, see \secref{sec:include}.
However, the document body must make a distinction
between processing of an individual part and of the main document, e.g.:
%
\begin{center}
\begin{tabular}{l}
|\ifchilddocmanual|\\
|\input{\childdocname}|\\
|\||else|\\
\textit{document body with }|\input{|\textit{part}|}|\\
|\||fi|
\end{tabular}
\end{center}
%
The conditional |\ifchilddocmanual| is true whenever
a part to be included by |\input| is being compiled,
and the name of the part is stored in |\childdocname|.

%%%%%%%%%%%%%%%%%%%%%%%%%%%%%%%%%%%%%%%%
\DescribeMacro{\childdocby}
Each part to be included by |\input| should start with:
%
\begin{center}
\begin{tabular}{l}
|\input{childdoc.def}|\\
|\childdocby{|\textit{main}|}|\\
\end{tabular}
\end{center}
%
The directive |\childdocby| is similar to |\childdocof|
described in \secref{sec:include},
but the subsequent selection of content must be done manually.
To that end, both |\ifchilddoc| and |\ifchilddocmanual|
will be true upon processing of a part,
and the name of the part is stored in |\childdocname|.
Note that |\jobname| will be set to the filename of the current part
so that each part receives an individual |.aux| file
that does not interfere with the |.aux| file(s) of the main document.
This behaviour can be altered by the alternative form
|\childdocby[*]{|\textit{main}|}| (with a non-empty optional argument)
which uses the |.aux| file of the main document
by setting |\jobname| to \textit{main}.

%%%%%%%%%%%%%%%%%%%%%%%%%%%%%%%%%%%%%%%%%%%%%%%%%%%%%%%%%%%%%%%%%%%%%%%%%%%%%%%%
\subsection{Driver Development}
\label{sec:driver}

The \textsf{childdoc} mechanism can also be use for the development
of definition files such as \LaTeX{} styles or classes.
This case differs from the above setup with multiple parts
included by |\include| in that no |\includeonly| should be invoked.
This can be achieved by starting the include file
(before |\ProvidesPackage|) with:
%
\begin{center}
\begin{tabular}{l}
|\input{childdoc.def}|\\
|\childdocforward{|\textit{main}|}|\\
\end{tabular}
\end{center}
%
or alternatively with:
%
\begin{center}
\begin{tabular}{l}
|\input{childdoc.def}|\\
|\childdocby{|\textit{main}|}|\\
\end{tabular}
\end{center}
%
Both forms have slightly different effects as described above.
The main file is prepared as usual, see \secref{sec:include}.

%%%%%%%%%%%%%%%%%%%%%%%%%%%%%%%%%%%%%%%%%%%%%%%%%%%%%%%%%%%%%%%%%%%%%%%%%%%%%%%%
\subsection{Legacy Detection}
\label{sec:detection}

The directive |\childdocmain| in the main file can detect
whether the complete document or merely a child is to be compiled
even without using the directive |\childdocof|.
This method is deprecated because it is less robust
and there is no compelling reason to use it;
it is merely provided for backward compatibility
and it may be removed in future versions.

If the detection mechanism is to be used,
it is mandatory to correctly specify
the filename of the main file as the argument of |\childdocmain|:
%
\begin{center}
\begin{tabular}{l}
|\input{childdoc.def}|\\
|\childdocmain{|\textit{main}|}|\\
\end{tabular}
\end{center}
%
If |\jobname| does not match the argument \textit{main} of |\childdocmain|,
it is assumed that |\jobname| points to the child file to be compiled.
When using |\childdocmain| with the main file specified as argument,
it suffices to start a child file
with just |\input{|\textit{main}|}|
without loading of the package and using |\childdocof|.
If instead all processing is done
with the appropriate \textsf{childdoc} directives,
the argument of \textit{main} of |\childdocmain| can be empty.

An alternative version of the command line processing described
in \secref{sec:commandline} using the detection mechanism reads:
%
\begin{center}
|... -jobname "|\textit{target}|" "|[\textit{flags}]%
[|\def\jobname{|\textit{dest}|}|]|\input{|\textit{main}|}"|
\end{center}

%%%%%%%%%%%%%%%%%%%%%%%%%%%%%%%%%%%%%%%%%%%%%%%%%%%%%%%%%%%%%%%%%%%%%%%%%%%%%%%%
\subsection{Manual Code}
\label{sec:manual}

In case one cannot be certain whether the definitions file |childdoc.def|
is installed on the target \TeX{} distribution
and one prefers not to ship it,
it is conceivable to paste a few relevant commands into the sources.

To that end, drop all statements |\input{childdoc.def}|
and perform the replacements as outlined below.
Instead of |\childdocmain{|\textit{main}|}| add the following code
to the top of the main file:
%
\begin{center}
\begin{tabular}{l}
|\||ifdefined\childdocname\endinput\||fi\newif\ifchilddoc|\\
|\edef\childdocname{\scantokens\expandafter{\jobname\noexpand}}|\\
|\def\childdocmain{|\textit{main}|}\||ifx\childdocmain\childdocname\||else|\\
|\childdoctrue\includeonly{\childdocname}\let\jobname\childdocmain\||fi|\\
\end{tabular}
\end{center}
%
Instead of |\childdocof{|\textit{main}|}| just include the main file
at the top of each child file:
%
\begin{center}
|\input{|\textit{main}|}|
\end{center}
%
A simple redirection |\childdocforward{|\textit{dest}|}| is achieved by:
%
\begin{center}
|\def\jobname{|\textit{dest}|}\input{\jobname}|
\end{center}
%
The redirection with prefix
|\childdocforwardprefix[|\textit{prefix}|]{|\textit{dest}|}|
is accomplished by:
%
\begin{center}
\begin{tabular}{l}
|{\edef\jobname{\scantokens\expandafter{\jobname\noexpand}}|\\
|\def\redirectjob |\textit{prefix}|#1~~~{\gdef\jobname{|\textit{dest}|#1}}|\\
|\expandafter\redirectjob\jobname~~~}\input{\jobname}|
\end{tabular}
\end{center}

In an alternative approach,
child documents can be compiled by a specific command line
without additional code or specific definitions:
%
\begin{center}
|... -jobname "|\textit{target}|" "|[\textit{flags}]%
|\includeonly{|\textit{dest}|}\input{|\textit{main}|}"|
\end{center}
%

%%%%%%%%%%%%%%%%%%%%%%%%%%%%%%%%%%%%%%%%%%%%%%%%%%%%%%%%%%%%%%%%%%%%%%%%%%%%%%%%
%%%%%%%%%%%%%%%%%%%%%%%%%%%%%%%%%%%%%%%%%%%%%%%%%%%%%%%%%%%%%%%%%%%%%%%%%%%%%%%%
\section{Information}

%%%%%%%%%%%%%%%%%%%%%%%%%%%%%%%%%%%%%%%%%%%%%%%%%%%%%%%%%%%%%%%%%%%%%%%%%%%%%%%%
\subsection{Copyright}

Copyright \copyright{} 2017--2018 Niklas Beisert

This work may be distributed and/or modified under the
conditions of the \LaTeX{} Project Public License, either version 1.3
of this license or (at your option) any later version.
The latest version of this license is in
  \url{http://www.latex-project.org/lppl.txt}
and version 1.3 or later is part of all distributions of \LaTeX{}
version 2005/12/01 or later.

This work has the LPPL maintenance status `maintained'.

The Current Maintainer of this work is Niklas Beisert.

This work consists of the files |README.txt|, |childdoc.ins| and |childdoc.dtx|
as well as the derived files |childdoc.def|, |cdocsamp.tex|
with |cdocsch1.tex|, |cdocsch2.tex|, |cdocspt3.tex|, |cdocspt4.tex|,
|cdocsdrf.tex|, |cdocsfn1.tex|, |cdocsfn2.tex|
as well as |childdoc.pdf|.

%%%%%%%%%%%%%%%%%%%%%%%%%%%%%%%%%%%%%%%%%%%%%%%%%%%%%%%%%%%%%%%%%%%%%%%%%%%%%%%%
\subsection{Files and Installation}

The package consists of the files:
%
\begin{center}
\begin{tabular}{ll}
    |README.txt|   & readme file \\
    |childdoc.ins| & installation file \\
    |childdoc.dtx| & source file \\
    |childdoc.def| & definition file \\
    |cdocsamp.tex| & sample main file \\
    |cdocsch1.tex| & sample include file \\
    |cdocsch2.tex| & sample include file \\
    |cdocspt3.tex| & sample part file \\
    |cdocspt4.tex| & sample part file \\
    |cdocsdrf.tex| & sample redirection file \\
    |cdocsfn1.tex| & sample redirection file \\
    |cdocsfn2.tex| & sample redirection file \\
    |childdoc.pdf| & manual
\end{tabular}
\end{center}
%
The distribution consists of the files
|README.txt|, |childdoc.ins| and |childdoc.dtx|.
%
\begin{itemize}
\item
Run (pdf)\LaTeX{} on |childdoc.dtx|
to compile the manual |childdoc.pdf| (this file).
\item
Run \LaTeX{} on |childdoc.ins| to create the definitions file |childdoc.def|
and the sample |cdocsamp.tex| with include files
|cdocsch1.tex|, |cdocsch2.tex|, |cdocspt3.tex|, |cdocspt4.tex|,
|cdocsdrf.tex|, |cdocsfn1.tex|, |cdocsfn2.tex|.
Then copy the file |childdoc.def| to an appropriate directory of your \LaTeX{}
distribution, e.g.\ \textit{texmf-root}|/tex/latex/childdoc|.
\end{itemize}

%%%%%%%%%%%%%%%%%%%%%%%%%%%%%%%%%%%%%%%%%%%%%%%%%%%%%%%%%%%%%%%%%%%%%%%%%%%%%%%%
\subsection{Related CTAN Packages}

There are several other packages which offer a similar functionality:
%
\begin{itemize}
\item
The packages
\href{http://ctan.org/pkg/docmute}{\textsf{docmute}},
\href{http://ctan.org/pkg/includex}{\textsf{includex}} and
\href{http://ctan.org/pkg/standalone}{\textsf{standalone}}
provide commands to include only the document body of
a child file thus allowing both files to be compiled individually.
\item
The packages \href{http://ctan.org/pkg/subdocs}{\textsf{subdocs}}
and \href{http://ctan.org/pkg/subfiles}{\textsf{subfiles}}
provide structures in which the main and child documents can be
encapsulated and allowing them to be compiled individually.
The inclusion mechanism is different from the conventional |\include|.
\item
The package \href{http://ctan.org/pkg/combine}{\textsf{combine}}
is an elaborate solution to combine several documents into one.
\end{itemize}
%
See also the CTAN topic \href{http://ctan.org/topic/subdocs}{\textsf{subdocs}}
for further related packages.
The present package differs from the above solutions in that
a document structure constructed with the conventional |\include| mechanism
just needs two extra commands at the top of every file
such that all constituent files can be compiled individually.

%%%%%%%%%%%%%%%%%%%%%%%%%%%%%%%%%%%%%%%%%%%%%%%%%%%%%%%%%%%%%%%%%%%%%%%%%%%%%%%%
%\subsection{Feature Suggestions}
%
%The following is a list of features which may be useful for future
%versions of this package:
%%
%\begin{itemize}
%\item
%\ldots
%\end{itemize}

%%%%%%%%%%%%%%%%%%%%%%%%%%%%%%%%%%%%%%%%%%%%%%%%%%%%%%%%%%%%%%%%%%%%%%%%%%%%%%%%
\subsection{Revision History}

%%%%%%%%%%%%%%%%%%%%%%%%%%%%%%%%%%%%%%%%
\paragraph{v2.0:} 2018/12/30

\begin{itemize}
\item
immediate forward processing
\item
added |\childdocby| mechanism
\item
manual restructured
\end{itemize}

%%%%%%%%%%%%%%%%%%%%%%%%%%%%%%%%%%%%%%%%
\paragraph{v1.6:} 2018/01/17

\begin{itemize}
\item
application for development of include files
\item
corrections to manual
\end{itemize}

%%%%%%%%%%%%%%%%%%%%%%%%%%%%%%%%%%%%%%%%
\paragraph{v1.5:} 2017/05/21

\begin{itemize}
\item
more complete structuring introduced
\item
|\childdocof| introduced
\item
|\childdoc| renamed to |\childdocmain|
\item
|\childredirect| renamed to |\childdocforward| and |\childdocforwardprefix|
and functionality expanded
\end{itemize}

%%%%%%%%%%%%%%%%%%%%%%%%%%%%%%%%%%%%%%%%
\paragraph{v1.0:} 2017/04/27

\begin{itemize}
\item
manual and install package
\item
first version published on CTAN
\end{itemize}

%%%%%%%%%%%%%%%%%%%%%%%%%%%%%%%%%%%%%%%%
\paragraph{v0.6:} 2017/04/26

\begin{itemize}
\item
redirection mechanism added
\end{itemize}

%%%%%%%%%%%%%%%%%%%%%%%%%%%%%%%%%%%%%%%%
\paragraph{v0.5:} 2017/04/26

\begin{itemize}
\item
functionality in definition file
\end{itemize}


%%%%%%%%%%%%%%%%%%%%%%%%%%%%%%%%%%%%%%%%%%%%%%%%%%%%%%%%%%%%%%%%%%%%%%%%%%%%%%%%
%%%%%%%%%%%%%%%%%%%%%%%%%%%%%%%%%%%%%%%%%%%%%%%%%%%%%%%%%%%%%%%%%%%%%%%%%%%%%%%%
%%%%%%%%%%%%%%%%%%%%%%%%%%%%%%%%%%%%%%%%%%%%%%%%%%%%%%%%%%%%%%%%%%%%%%%%%%%%%%%%
\appendix

\settowidth\MacroIndent{\rmfamily\scriptsize 000\ }

 \DocInput{childdoc.dtx}

\end{document}
%</driver>
% \fi
%
% %%%%%%%%%%%%%%%%%%%%%%%%%%%%%%%%%%%%%%%%%%%%%%%%%%%%%%%%%%%%%%%%%%%%%%%%%%%%%%
% %%%%%%%%%%%%%%%%%%%%%%%%%%%%%%%%%%%%%%%%%%%%%%%%%%%%%%%%%%%%%%%%%%%%%%%%%%%%%%
% \section{Sample}
%\iffalse
%<*samplemain>
%\fi
%
% The following presents a sample document
% with two chapters, two parts, a title page,
% a compile flag as well as three forwarding files to set the flag.
% It consists of eight |.tex| files:
% \begin{center}
% \begin{tabular}{ll}
% |cdocsamp.tex|&main file\\
% |cdocsch1.tex|&include file for chapter 1\\
% |cdocsch2.tex|&include file for chapter 2\\
% |cdocspt3.tex|&include file for part 3\\
% |cdocspt4.tex|&include file for part 4\\
% |cdocsdrf.tex|&forwarding file for main file in draft mode\\
% |cdocsfi1.tex|&forwarding file for final version of chapter 1\\
% |cdocsfi2.tex|&forwarding file for final version of chapter 2\\
% \end{tabular}
% \end{center}
% Each of the eight files can be compiled directly by the \LaTeX{} compiler.
%
% %%%%%%%%%%%%%%%%%%%%%%%%%%%%%%%%%%%%%%
% \paragraph{Main File.}
%
% The main file is called |cdocsamp.tex|.
%
% Load the \textsf{childdoc} definitions and
% declare the filename for the main document:
%    \begin{macrocode}
\input{childdoc.def}
\childdocmain{}
%    \end{macrocode}

% Optional override for |\version| flag:
%    \begin{macrocode}
%%\ifchilddoc\else\providecommand{\version}{draft}\fi
%    \end{macrocode}

% Define the default values for the |\version| flag
% (|final| for the main file and |draft| for childs):
%    \begin{macrocode}
\ifchilddoc
\providecommand{\version}{draft}
\else
\providecommand{\version}{final}
\fi
%    \end{macrocode}

% Load the standard document class:
%    \begin{macrocode}
\documentclass[12pt]{article}
%    \end{macrocode}

% Start the document body:
%    \begin{macrocode}
\begin{document}
%    \end{macrocode}

% Declare a title page.
% Print title, part of document being processed and version flag:
%    \begin{macrocode}
\addtocounter{page}{-1}
\begin{center}
{\LARGE\bfseries{}childdoc example\par}
\vspace{1cm}
\ifchilddoc
\ifchilddocmanual part\else chapter\fi:
`\childdocname' of `\childdocjob'\par
\else
main document: `\childdocjob'\par
\fi
version: \version\par
\end{center}
\newpage
%    \end{macrocode}

% Manually include selected file,
% otherwise process as usual:
%    \begin{macrocode}
\ifchilddocmanual
\section*{part `\childdocname'}
\input{\childdocname}
\else
%    \end{macrocode}

% Include the two chapters:
%    \begin{macrocode}
\include{cdocsch1}
\include{cdocsch2}
%    \end{macrocode}

% Include the two parts unless only chapters should be displayed:
%    \begin{macrocode}
\ifchilddoc\else
\section{part three}
\input{cdocspt3}
\section{part four}
\input{cdocspt4}
\fi
%    \end{macrocode}

% Process as usual until here:
%    \begin{macrocode}
\fi
%    \end{macrocode}

% End of document body:
%    \begin{macrocode}
\end{document}
%    \end{macrocode}
%\iffalse
%</samplemain>
%\fi
%
% %%%%%%%%%%%%%%%%%%%%%%%%%%%%%%%%%%%%%%
% \paragraph{Chapter Include Files.}
%
% The include files are called |cdocsch1.tex| and |cdocsch2.tex|.
%
%\iffalse
%<*samplechap1|samplechap2>
%\fi

% Optional override for |\version| flag:
%    \begin{macrocode}
%%\providecommand{\version}{final}
%    \end{macrocode}

% Include the main document:
%    \begin{macrocode}
\input{childdoc.def}
\childdocof{cdocsamp}
%    \end{macrocode}

%\iffalse
%</samplechap1|samplechap2>
%\fi
%
%\iffalse
%<*samplechap1>
%\fi
% Some text for chapter 1:
%    \begin{macrocode}
\section{one}
some text in chapter one
%    \end{macrocode}

%\iffalse
%</samplechap1>
%\fi
% Some text for chapter 2:
%\iffalse
%<*samplechap2>
%\fi
%    \begin{macrocode}
\section{two}
more text in chapter two
%    \end{macrocode}

%\iffalse
%</samplechap2>
%\fi
%
% %%%%%%%%%%%%%%%%%%%%%%%%%%%%%%%%%%%%%%
% \paragraph{Part Include Files.}
%
% The include files are called |cdocspt3.tex| and |cdocspt4.tex|.
%
%\iffalse
%<*samplepart3|samplepart4>
%\fi

% Optional override for |\version| flag:
%    \begin{macrocode}
%%\providecommand{\version}{final}
%    \end{macrocode}

% Include the main document:
%    \begin{macrocode}
\input{childdoc.def}
\childdocby{cdocsamp}
%    \end{macrocode}

%\iffalse
%</samplepart3|samplepart4>
%\fi
%
%\iffalse
%<*samplepart3>
%\fi
% Some text for part 3:
%    \begin{macrocode}
some text in part three
%    \end{macrocode}

%\iffalse
%</samplepart3>
%\fi
% Some text for part 4:
%\iffalse
%<*samplepart4>
%\fi
%    \begin{macrocode}
more text in part four
%    \end{macrocode}

%\iffalse
%</samplepart4>
%\fi
%
% %%%%%%%%%%%%%%%%%%%%%%%%%%%%%%%%%%%%%%
% \paragraph{Forwarding for a Complete Draft.}
%
% The following forwarding file |cdocsdrf.tex|
% compiles the main document in draft mode:
%\iffalse
%<*sampledraft>
%\fi
%    \begin{macrocode}
\def\version{draft}
\input{childdoc.def}
\childdocforward{cdocsamp}
%    \end{macrocode}

%\iffalse
%</sampledraft>
%\fi
%
% %%%%%%%%%%%%%%%%%%%%%%%%%%%%%%%%%%%%%%
% \paragraph{Forwarding for Final Version of the Chapters.}
%
% The following forwarding files |cdocsfn1.tex| and |cdocsfn2.tex|
% (with identical content)
% compile the final versions of the child documents
% |cdocsch1.tex| and |cdocsch2.tex|, respectively:
%\iffalse
%<*samplefinal>
%\fi
%    \begin{macrocode}
\def\version{final}
\input{childdoc.def}
\childdocforwardprefix[cdocsamp]{cdocsfn}{cdocsch}
%    \end{macrocode}

%\iffalse
%</samplefinal>
%\fi
%
% %%%%%%%%%%%%%%%%%%%%%%%%%%%%%%%%%%%%%%
% \paragraph{Command Line Processing.}
%
% The following three command lines generate the output files
% |cdocscld|, |cdocscl1| and |cdocscl2|
% which should be identical to
% |cdocsdrf|, |cdocsch1| and |cdocsfn2|, respectively:
% \begin{center}
% \begin{tabular}{l}
% |latex -jobname cdocscld \|\\
% |  "\def\version{draft}\input{childdoc.def}\childdocforward{cdocsamp}"|\\
% |latex -jobname cdocscl1 \|\\
% |  "\input{childdoc.def}\childdocforward[cdocsamp]{cdocsch1}"|\\
% |latex -jobname cdocscl2 \|\\
% |  "\def\version{final}\input{childdoc.def}\childdocforward{cdocsch2}"|
% \end{tabular}
% \end{center}
% Note that the trailing backslash on each first line
% merely continues the input to the second line
% (for convenient cut ant paste).
% Furthermore, the command |latex| can be replaced by any
% of its alternative versions such as |pdflatex|.
%
% %%%%%%%%%%%%%%%%%%%%%%%%%%%%%%%%%%%%%%%%%%%%%%%%%%%%%%%%%%%%%%%%%%%%%%%%%%%%%%
% %%%%%%%%%%%%%%%%%%%%%%%%%%%%%%%%%%%%%%%%%%%%%%%%%%%%%%%%%%%%%%%%%%%%%%%%%%%%%%
% \section{Implementation}
%\iffalse
%<*package>
%\fi
%
% This section describes the definitions file |childdoc.def|.

% The definitions cannot be loaded using |\usepackage| or |\RequirePackage|
% which has a mechanism to prevent loading a style file more than once.
% When loading the definitions by means of |\input|
% multiple instances have to be prevented manually:
%\iffalse
%This code needs to be before the `\ProvidesFile' directive
%which is defined at the beginning of this file.
%Therefore it is also placed there and commented out here.
%</package>
%<*discard>
%\fi
%    \begin{macrocode}
\ifdefined\childdocmain\endinput\fi
%    \end{macrocode}
%\iffalse
%</discard>
%<*package>
%\fi
%
% \macro{\ifchilddoc}
% \macro{\ifchilddocmanual}
% The conditional |\ifchilddoc| tells whether a
% child (true) or main (false) document is being compiled.
% The conditional |\ifchilddocmanual| tells whether
% the |\includeonly| mechanism is used (false) or
% the selection of child files must be performed manually (true).
% The definitions initialise to false:
%    \begin{macrocode}
\newif\ifchilddoc
\newif\ifchilddocmanual
%    \end{macrocode}

% \macro{\childdocname}
% \macro{\childdocjob}
% The macro |\childdocname| stores the name of the main document
% to be compiled. The macro |\childdocjob| stores the name of
% the document on which the \LaTeX{} compiler was originally invoked.
% The content of |\jobname| cannot be compared
% to filenames specified in the source due to different catcodes.
% The following code rescans |\jobname|, stores the result
% in |\childdocname| and saves a copy in |\childdocjob|:
%    \begin{macrocode}
\edef\childdocname{\scantokens\expandafter{\jobname\noexpand}}
\let\childdocjob\childdocname
%    \end{macrocode}

% \macro{\childdocdisable}
% The macro |\childdocdisable| prevents the main file
% from being processed more than once.
% At this stage, the main document command |\childdocmain|
% is assumed to be called once again where it should do nothing.
% Any subsequent call to it should prevent
% a secondary processing of the main document
% It overwrites the forwarding commands
% |\childdocof| and |\childdocforward|
% with empty macros to prevent further inclusions of the main document:
%    \begin{macrocode}
\newcommand{\childdocdisable}
{
  \renewcommand{\childdocmain}[1]{\renewcommand{\childdocmain}[1]{\endinput}}
  \renewcommand{\childdocof}[1]{}
  \renewcommand{\childdocby}[2][]{}
  \renewcommand{\childdocforward}[2][]{}
  \renewcommand{\childdocdisable}{}
}
%    \end{macrocode}

% \macro{\childdocmain}
% The macro |\childdocmain| is to be called at the top of the main file
% with nothing or the main filename (without extension) as argument.
% First, it breaks loops.
% If the argument is not empty and does not match |\childdocname|
% (which is set by the first inclusion of |childdoc.def|),
% |\ifchilddoc| is set to true, |\includeonly| is applied to the child file
% and |\jobname| is set to the main file
% (for proper handling of |.aux| files):
%    \begin{macrocode}
\newcommand{\childdocmain}[1]
{
  \childdocdisable\childdocmain{}
  \if?#1?\else
    \begingroup
      \def\childdoctmp{#1}
      \ifx\childdoctmp\childdocname
        \def\childdoctmp{}
      \else
        \def\childdoctmp
        {
          \childdoctrue
          \includeonly{\childdocname}
          \def\childdocjob{#1}
          \def\jobname{#1}
        }
      \fi
      \expandafter
    \endgroup
    \childdoctmp
  \fi
}
%    \end{macrocode}

% \macro{\childdocof}
% The command |\childdocof| redirects
% compilation to the main file |#1|.
%    \begin{macrocode}
\newcommand{\childdocof}[1]
{
  \childdocdisable
  \childdoctrue
  \includeonly{\childdocname}
  \def\jobname{#1}
  \def\childdocjob{#1}
  \input{#1}
}
%    \end{macrocode}

% \macro{\childdocby}
% The command |\childdocby| ....
%    \begin{macrocode}
\newcommand{\childdocby}[2][]
{
  \childdocdisable
  \childdoctrue
  \childdocmanualtrue
  \if?#1?\else
    \def\jobname{#2}
  \fi
  \def\childdocjob{#2}
  \input{#2}
  \endinput
}
%    \end{macrocode}

% \macro{\childdocforward}
% The command |\childdocforward| redirects
% compilation to the main file or
% (if the optional argument is given) a child file.
% Parameters are set as if the main file
% or a child file starting with |\childdocof| was compiled.
% Then compilation is handed over to the main file:
%    \begin{macrocode}
\newcommand{\childdocforward}[2][]
{
  \begingroup
    \if?#1?
      \def\childdoctmp
      {
        \def\childdocname{#2}
        \def\childdocjob{#2}
        \def\jobname{#2}
        \input{#2}
        \endinput
      }
    \else
      \def\childdoctmp
      {
        \childdocdisable
        \def\childdocname{#2}
        \childdoctrue
        \includeonly{#2}
        \def\childdocjob{#1}
        \def\jobname{#1}
        \input{#1}
        \endinput
      }
    \fi
    \expandafter
  \endgroup
  \childdoctmp
}
%    \end{macrocode}

% \macro{\childdocforwardprefix}
% The command |\childdocforwardprefix| redirects
% compilation to the main or a child file by means of a pattern.
% The prefix |#1| in the current filename is replaced by |#2|
% and the suffix of the current filename is kept
% (it is assumed that the filename does not contain the substring `|~~~|'
% which is used as a delimiter).
% Compilation is handed over to the new file by |\childdocforward|:
%    \begin{macrocode}
\newcommand{\childdocforwardprefix}[3][]
{
  \begingroup
    \def\childdocextract #2##1~~~{\def\childdoctmp{\childdocforward[#1]{#3##1}}}
    \expandafter\childdocextract\childdocname~~~
    \expandafter
  \endgroup
  \childdoctmp
}
%    \end{macrocode}

% \macro{\childdoc}
% The deprecated macro |\childdoc| is a legacy version of |\childdocmain|:
%    \begin{macrocode}
\newcommand{\childdoc}{\childdocmain}
%    \end{macrocode}

% \macro{\childdocredirect}
% The deprecated macro |\childdocredirect| is a legacy version
% of |\childdocforward| and |\childdocforwardprefix|:
%    \begin{macrocode}
\newcommand{\childdocredirect}[2][]
{
  \begingroup
    \if?#1?
      \def\childdoctmp{\childdocforward{#2}}
    \else
      \def\childdoctmp{\childdocforwardprefix{#1}{#2}}
    \fi
    \expandafter
  \endgroup
  \childdoctmp
}
%    \end{macrocode}

%\iffalse
%</package>
%\fi
%
\endinput
|
and perform the replacements as outlined below.
Instead of |\childdocmain{|\textit{main}|}| add the following code
to the top of the main file:
%
\begin{center}
\begin{tabular}{l}
|\||ifdefined\childdocname\endinput\||fi\newif\ifchilddoc|\\
|\edef\childdocname{\scantokens\expandafter{\jobname\noexpand}}|\\
|\def\childdocmain{|\textit{main}|}\||ifx\childdocmain\childdocname\||else|\\
|\childdoctrue\includeonly{\childdocname}\let\jobname\childdocmain\||fi|\\
\end{tabular}
\end{center}
%
Instead of |\childdocof{|\textit{main}|}| just include the main file
at the top of each child file:
%
\begin{center}
|\input{|\textit{main}|}|
\end{center}
%
A simple redirection |\childdocforward{|\textit{dest}|}| is achieved by:
%
\begin{center}
|\def\jobname{|\textit{dest}|}\input{\jobname}|
\end{center}
%
The redirection with prefix
|\childdocforwardprefix[|\textit{prefix}|]{|\textit{dest}|}|
is accomplished by:
%
\begin{center}
\begin{tabular}{l}
|{\edef\jobname{\scantokens\expandafter{\jobname\noexpand}}|\\
|\def\redirectjob |\textit{prefix}|#1~~~{\gdef\jobname{|\textit{dest}|#1}}|\\
|\expandafter\redirectjob\jobname~~~}\input{\jobname}|
\end{tabular}
\end{center}

In an alternative approach,
child documents can be compiled by a specific command line
without additional code or specific definitions:
%
\begin{center}
|... -jobname "|\textit{target}|" "|[\textit{flags}]%
|\includeonly{|\textit{dest}|}\input{|\textit{main}|}"|
\end{center}
%

%%%%%%%%%%%%%%%%%%%%%%%%%%%%%%%%%%%%%%%%%%%%%%%%%%%%%%%%%%%%%%%%%%%%%%%%%%%%%%%%
%%%%%%%%%%%%%%%%%%%%%%%%%%%%%%%%%%%%%%%%%%%%%%%%%%%%%%%%%%%%%%%%%%%%%%%%%%%%%%%%
\section{Information}

%%%%%%%%%%%%%%%%%%%%%%%%%%%%%%%%%%%%%%%%%%%%%%%%%%%%%%%%%%%%%%%%%%%%%%%%%%%%%%%%
\subsection{Copyright}

Copyright \copyright{} 2017--2018 Niklas Beisert

This work may be distributed and/or modified under the
conditions of the \LaTeX{} Project Public License, either version 1.3
of this license or (at your option) any later version.
The latest version of this license is in
  \url{http://www.latex-project.org/lppl.txt}
and version 1.3 or later is part of all distributions of \LaTeX{}
version 2005/12/01 or later.

This work has the LPPL maintenance status `maintained'.

The Current Maintainer of this work is Niklas Beisert.

This work consists of the files |README.txt|, |childdoc.ins| and |childdoc.dtx|
as well as the derived files |childdoc.def|, |cdocsamp.tex|
with |cdocsch1.tex|, |cdocsch2.tex|, |cdocspt3.tex|, |cdocspt4.tex|,
|cdocsdrf.tex|, |cdocsfn1.tex|, |cdocsfn2.tex|
as well as |childdoc.pdf|.

%%%%%%%%%%%%%%%%%%%%%%%%%%%%%%%%%%%%%%%%%%%%%%%%%%%%%%%%%%%%%%%%%%%%%%%%%%%%%%%%
\subsection{Files and Installation}

The package consists of the files:
%
\begin{center}
\begin{tabular}{ll}
    |README.txt|   & readme file \\
    |childdoc.ins| & installation file \\
    |childdoc.dtx| & source file \\
    |childdoc.def| & definition file \\
    |cdocsamp.tex| & sample main file \\
    |cdocsch1.tex| & sample include file \\
    |cdocsch2.tex| & sample include file \\
    |cdocspt3.tex| & sample part file \\
    |cdocspt4.tex| & sample part file \\
    |cdocsdrf.tex| & sample redirection file \\
    |cdocsfn1.tex| & sample redirection file \\
    |cdocsfn2.tex| & sample redirection file \\
    |childdoc.pdf| & manual
\end{tabular}
\end{center}
%
The distribution consists of the files
|README.txt|, |childdoc.ins| and |childdoc.dtx|.
%
\begin{itemize}
\item
Run (pdf)\LaTeX{} on |childdoc.dtx|
to compile the manual |childdoc.pdf| (this file).
\item
Run \LaTeX{} on |childdoc.ins| to create the definitions file |childdoc.def|
and the sample |cdocsamp.tex| with include files
|cdocsch1.tex|, |cdocsch2.tex|, |cdocspt3.tex|, |cdocspt4.tex|,
|cdocsdrf.tex|, |cdocsfn1.tex|, |cdocsfn2.tex|.
Then copy the file |childdoc.def| to an appropriate directory of your \LaTeX{}
distribution, e.g.\ \textit{texmf-root}|/tex/latex/childdoc|.
\end{itemize}

%%%%%%%%%%%%%%%%%%%%%%%%%%%%%%%%%%%%%%%%%%%%%%%%%%%%%%%%%%%%%%%%%%%%%%%%%%%%%%%%
\subsection{Related CTAN Packages}

There are several other packages which offer a similar functionality:
%
\begin{itemize}
\item
The packages
\href{http://ctan.org/pkg/docmute}{\textsf{docmute}},
\href{http://ctan.org/pkg/includex}{\textsf{includex}} and
\href{http://ctan.org/pkg/standalone}{\textsf{standalone}}
provide commands to include only the document body of
a child file thus allowing both files to be compiled individually.
\item
The packages \href{http://ctan.org/pkg/subdocs}{\textsf{subdocs}}
and \href{http://ctan.org/pkg/subfiles}{\textsf{subfiles}}
provide structures in which the main and child documents can be
encapsulated and allowing them to be compiled individually.
The inclusion mechanism is different from the conventional |\include|.
\item
The package \href{http://ctan.org/pkg/combine}{\textsf{combine}}
is an elaborate solution to combine several documents into one.
\end{itemize}
%
See also the CTAN topic \href{http://ctan.org/topic/subdocs}{\textsf{subdocs}}
for further related packages.
The present package differs from the above solutions in that
a document structure constructed with the conventional |\include| mechanism
just needs two extra commands at the top of every file
such that all constituent files can be compiled individually.

%%%%%%%%%%%%%%%%%%%%%%%%%%%%%%%%%%%%%%%%%%%%%%%%%%%%%%%%%%%%%%%%%%%%%%%%%%%%%%%%
%\subsection{Feature Suggestions}
%
%The following is a list of features which may be useful for future
%versions of this package:
%%
%\begin{itemize}
%\item
%\ldots
%\end{itemize}

%%%%%%%%%%%%%%%%%%%%%%%%%%%%%%%%%%%%%%%%%%%%%%%%%%%%%%%%%%%%%%%%%%%%%%%%%%%%%%%%
\subsection{Revision History}

%%%%%%%%%%%%%%%%%%%%%%%%%%%%%%%%%%%%%%%%
\paragraph{v2.0:} 2018/12/30

\begin{itemize}
\item
immediate forward processing
\item
added |\childdocby| mechanism
\item
manual restructured
\end{itemize}

%%%%%%%%%%%%%%%%%%%%%%%%%%%%%%%%%%%%%%%%
\paragraph{v1.6:} 2018/01/17

\begin{itemize}
\item
application for development of include files
\item
corrections to manual
\end{itemize}

%%%%%%%%%%%%%%%%%%%%%%%%%%%%%%%%%%%%%%%%
\paragraph{v1.5:} 2017/05/21

\begin{itemize}
\item
more complete structuring introduced
\item
|\childdocof| introduced
\item
|\childdoc| renamed to |\childdocmain|
\item
|\childredirect| renamed to |\childdocforward| and |\childdocforwardprefix|
and functionality expanded
\end{itemize}

%%%%%%%%%%%%%%%%%%%%%%%%%%%%%%%%%%%%%%%%
\paragraph{v1.0:} 2017/04/27

\begin{itemize}
\item
manual and install package
\item
first version published on CTAN
\end{itemize}

%%%%%%%%%%%%%%%%%%%%%%%%%%%%%%%%%%%%%%%%
\paragraph{v0.6:} 2017/04/26

\begin{itemize}
\item
redirection mechanism added
\end{itemize}

%%%%%%%%%%%%%%%%%%%%%%%%%%%%%%%%%%%%%%%%
\paragraph{v0.5:} 2017/04/26

\begin{itemize}
\item
functionality in definition file
\end{itemize}


%%%%%%%%%%%%%%%%%%%%%%%%%%%%%%%%%%%%%%%%%%%%%%%%%%%%%%%%%%%%%%%%%%%%%%%%%%%%%%%%
%%%%%%%%%%%%%%%%%%%%%%%%%%%%%%%%%%%%%%%%%%%%%%%%%%%%%%%%%%%%%%%%%%%%%%%%%%%%%%%%
%%%%%%%%%%%%%%%%%%%%%%%%%%%%%%%%%%%%%%%%%%%%%%%%%%%%%%%%%%%%%%%%%%%%%%%%%%%%%%%%
\appendix

\settowidth\MacroIndent{\rmfamily\scriptsize 000\ }

 \DocInput{childdoc.dtx}

\end{document}
%</driver>
% \fi
%
% %%%%%%%%%%%%%%%%%%%%%%%%%%%%%%%%%%%%%%%%%%%%%%%%%%%%%%%%%%%%%%%%%%%%%%%%%%%%%%
% %%%%%%%%%%%%%%%%%%%%%%%%%%%%%%%%%%%%%%%%%%%%%%%%%%%%%%%%%%%%%%%%%%%%%%%%%%%%%%
% \section{Sample}
%\iffalse
%<*samplemain>
%\fi
%
% The following presents a sample document
% with two chapters, two parts, a title page,
% a compile flag as well as three forwarding files to set the flag.
% It consists of eight |.tex| files:
% \begin{center}
% \begin{tabular}{ll}
% |cdocsamp.tex|&main file\\
% |cdocsch1.tex|&include file for chapter 1\\
% |cdocsch2.tex|&include file for chapter 2\\
% |cdocspt3.tex|&include file for part 3\\
% |cdocspt4.tex|&include file for part 4\\
% |cdocsdrf.tex|&forwarding file for main file in draft mode\\
% |cdocsfi1.tex|&forwarding file for final version of chapter 1\\
% |cdocsfi2.tex|&forwarding file for final version of chapter 2\\
% \end{tabular}
% \end{center}
% Each of the eight files can be compiled directly by the \LaTeX{} compiler.
%
% %%%%%%%%%%%%%%%%%%%%%%%%%%%%%%%%%%%%%%
% \paragraph{Main File.}
%
% The main file is called |cdocsamp.tex|.
%
% Load the \textsf{childdoc} definitions and
% declare the filename for the main document:
%    \begin{macrocode}
% \iffalse
%
% childdoc.dtx Copyright (C) 2017-2018 Niklas Beisert
%
% This work may be distributed and/or modified under the
% conditions of the LaTeX Project Public License, either version 1.3
% of this license or (at your option) any later version.
% The latest version of this license is in
%   http://www.latex-project.org/lppl.txt
% and version 1.3 or later is part of all distributions of LaTeX
% version 2005/12/01 or later.
%
% This work has the LPPL maintenance status `maintained'.
%
% The Current Maintainer of this work is Niklas Beisert.
%
% This work consists of the files childdoc.dtx and childdoc.ins
% and the derived files childdoc.def and cdocsamp.tex with
% cdocsch1.tex, cdocsch2.tex, cdocsdrf.tex, cdocsfn1.tex, cdocsfn2.tex.
%
%<package>\ifdefined\childdocmain\endinput\fi
%<package>\ProvidesFile{childdoc.def}[2018/12/30 v2.0 child document driver]
%<samplemain>\ProvidesFile{cdocsamp.tex}[2018/12/30 v2.0 sample for childdoc]
%<*driver>
%\ProvidesFile{childdoc.drv}[2018/12/30 v2.0 childdoc reference manual file]
\PassOptionsToClass{10pt,a4paper}{article}
\documentclass{ltxdoc}

\usepackage[margin=35mm]{geometry}
\usepackage{hyperref}
\usepackage{hyperxmp}
\usepackage[usenames]{color}

\hypersetup{colorlinks=true}
\hypersetup{pdfstartview=FitH}
\hypersetup{pdfpagemode=UseNone}
\hypersetup{pdfsource={}}
\hypersetup{pdflang={en-UK}}
\hypersetup{pdfcopyright={Copyright 2017-2018 Niklas Beisert.
  This work may be distributed and/or modified under the
  conditions of the LaTeX Project Public License, either version 1.3
  of this license or (at your option) any later version.}}
\hypersetup{pdflicenseurl={http://www.latex-project.org/lppl.txt}}
\hypersetup{pdfcontactaddress={ETH Zurich, ITP, HIT K,
  Wolfgang-Pauli-Strasse 27}}
\hypersetup{pdfcontactpostcode={8093}}
\hypersetup{pdfcontactcity={Zurich}}
\hypersetup{pdfcontactcountry={Switzerland}}
\hypersetup{pdfcontactemail={nbeisert@itp.phys.ethz.ch}}
\hypersetup{pdfcontacturl={http://people.phys.ethz.ch/\xmptilde nbeisert/}}

\newcommand{\secref}[1]{\hyperref[#1]{section \ref*{#1}}}

\parskip1ex
\parindent0pt
\let\olditemize\itemize
\def\itemize{\olditemize\parskip0pt}

\begin{document}

\title{The \textsf{childdoc} Package}
\hypersetup{pdftitle={The childdoc Package}}
\author{Niklas Beisert\\[2ex]
  Institut f\"ur Theoretische Physik\\
  Eidgen\"ossische Technische Hochschule Z\"urich\\
  Wolfgang-Pauli-Strasse 27, 8093 Z\"urich, Switzerland\\[1ex]
  \href{mailto:nbeisert@itp.phys.ethz.ch}
  {\texttt{nbeisert@itp.phys.ethz.ch}}}
\hypersetup{pdfauthor={Niklas Beisert}}
\hypersetup{pdfsubject={Manual for the LaTeX2e Package childdoc}}
\date{30 December 2018, \textsf{v2.0}}
\maketitle

\begin{abstract}\noindent
\textsf{childdoc} is a \LaTeXe{} package
that enables the direct compilation
of document sections included by |\include|
to individual files.
\end{abstract}

\begingroup
\parskip0ex
\tableofcontents
\endgroup

%%%%%%%%%%%%%%%%%%%%%%%%%%%%%%%%%%%%%%%%%%%%%%%%%%%%%%%%%%%%%%%%%%%%%%%%%%%%%%%%
%%%%%%%%%%%%%%%%%%%%%%%%%%%%%%%%%%%%%%%%%%%%%%%%%%%%%%%%%%%%%%%%%%%%%%%%%%%%%%%%
\section{Introduction}

\LaTeX{} provides a mechanism to structure a large document (such as a book)
into a main file and several child files (containing the chapters)
using the |\include| command.
This mechanism is beneficial for documents
which span hundreds of pages in order to
make the source file(s) more manageable.
Moreover, compilation can be restricted to
selected child files by means of the |\includeonly| command.
The latter feature can be used to reduce the compilation time while editing
(this was significantly more useful in the earlier days of \LaTeX{})
or to generate a smaller document which is easier to navigate.
Another application of |\includeonly| is to generate
documents consisting of selected parts of the complete document.

However, there are a few drawbacks of the plain |\include| mechanism:
\begin{itemize}
\item
The child files cannot be compiled on their own,
they can only be compiled via the main file.
A naive editing environment
(such as a text editor with an option
to have the current file processed by \LaTeX)
may require one to switch to the main file before compiling;
attempting to compile the child file produces errors.
\item
The main file must be modified (each time)
to adjust the |\includeonly| command
to the present needs. This easily leaves the main file in a messy state.
\item
The generated document will always carry the filename
of the main document. This is inconvenient if
several child files are to be compiled and
to be kept for distribution.
\end{itemize}

The present package provides a simple interface
to make child files individually compilable by \LaTeX{}.
Compiling a child file then has the same effect as compiling
the main file with an |\includeonly| command
to select the appropriate child.
Moreover the generated document will carry the name of the child
rather than the main file.
This resolves all three above issues.

This feature is meant to make the editing of books,
thesis documents and lecture notes somewhat more convenient.
However, the package can also be used efficiently for
composing a series of documents (such as exercise sheets)
which are typically distributed individually.
It then assists the author in generating the individual documents
(potentially in different versions)
as well as a document containing the collected series.
Another application is in developing style files
or other kinds of included material
where compilation of the style file could redirect
to a sample or test file.

%%%%%%%%%%%%%%%%%%%%%%%%%%%%%%%%%%%%%%%%%%%%%%%%%%%%%%%%%%%%%%%%%%%%%%%%%%%%%%%%
%%%%%%%%%%%%%%%%%%%%%%%%%%%%%%%%%%%%%%%%%%%%%%%%%%%%%%%%%%%%%%%%%%%%%%%%%%%%%%%%
\section{Usage}

First of all, the package \textsf{childdoc} is \emph{not} a standard
\LaTeXe{} |.sty| style file! Therefore it needs to be invoked in
a non-standard way.

%%%%%%%%%%%%%%%%%%%%%%%%%%%%%%%%%%%%%%%%%%%%%%%%%%%%%%%%%%%%%%%%%%%%%%%%%%%%%%%%
\subsection{Included Files}
\label{sec:include}

%%%%%%%%%%%%%%%%%%%%%%%%%%%%%%%%%%%%%%%%
\DescribeMacro{\childdocmain}
To use the package, add the commands
\begin{center}
\begin{tabular}{l}
|\input{childdoc.def}|\\
|\childdocmain{}|\\
\end{tabular}
\end{center}
at the very top of the main \LaTeX{} file,
in particular \emph{before} the |\documentclass| statement!
The argument of |\childdocmain| should be left empty
(but it must be present).

%%%%%%%%%%%%%%%%%%%%%%%%%%%%%%%%%%%%%%%%
\DescribeMacro{\childdocof}
Furthermore, add the commands
\begin{center}
\begin{tabular}{l}
|\input{childdoc.def}|\\
|\childdocof{|\textit{main}|}|\\
\end{tabular}
\end{center}
at the top of every child file \textit{child}
which is included by |\include{|\textit{child}|}|
from within the main file
(or at least for those files to be compiled individually).
The argument \textit{main} must be the filename of the main file.

There are a couple of
considerations in setting up the main and child documents:

%%%%%%%%%%%%%%%%%%%%%%%%%%%%%%%%%%%%%%%%
\paragraph{Restrictions.}

Please note the following restrictions:
\begin{itemize}
\item
|\childdocmain| must be called with one argument \textit{main}
to ensure compatibility with earlier version of the package.
It must either be empty (|\childdocmain{}|)
or precisely match the filename of the main file in which it is specified.
See \secref{sec:detection} for further information.
\item
The filename \textit{main} must be specified without the |.tex| extension.
\item
The filename \textit{main} is case sensitive
(even in case-insensitive file systems)
due to internal string comparison.
\item
The argument \textit{main} should be fully expanded, it cannot be a macro.
\item
Subdirectories and special characters should be avoided in filenames.
\item
The command |\childdocmain{|\textit{main}|}| must be followed by a whitespace.
It should not be followed immediately by another command
or by a comment mark `|%|'.
This is because the \TeX{} parser reads the token immediately following
the argument of |\childdocmain| and puts it
at the beginning of every child section;
however, a white\-space is ignored.
\end{itemize}

%%%%%%%%%%%%%%%%%%%%%%%%%%%%%%%%%%%%%%%%
\paragraph{Content of Main File.}

It is advisable to place all content in the child files included by |\include|.
Any output contained in the main file will appear in all child documents
unless suppressed manually;
it cannot be suppressed automatically by the |\includeonly| directive
and thus should normally be avoided.
A method to include some content in the main file
by means of conditional processing is described in \secref{sec:conditional}.

%%%%%%%%%%%%%%%%%%%%%%%%%%%%%%%%%%%%%%%%
\paragraph{Page Numbering.}

When only a part of the document is compiled,
the appropriate numbering of pages
(as well as other status parameters)
is determined from the |.aux| files.
The latter contain information from previous passes.
However this information needs to propagate through
all intermediate child documents.
Therefore the page numbering in child documents may well
be inconsistent until the complete document is compiled at least once.

A useful (if unconventional) way to always ensure a consistent
page numbering is to restart the numbering in each child document
and denote the pages by `\textit{child}|.|\textit{page}'
where \textit{child} represents the chapter/section number of the child file.
This can be achieved by the command
|\numberwithin{page}{|\textit{child}|}|
of the \textsf{amsmath} package
where \textit{child} can be |chapter| or |section|
depending on the chosen structuring.
Alternatively, one can modify the macro |\thepage| appropriately
and reset the counter |page| at the start of each child file.

%%%%%%%%%%%%%%%%%%%%%%%%%%%%%%%%%%%%%%%%%%%%%%%%%%%%%%%%%%%%%%%%%%%%%%%%%%%%%%%%
\subsection{Conditional Processing}
\label{sec:conditional}

The package provides a mechanism to compile different versions
of a document. To customise the versions further some conditional processing
can come in handy to distinguish which version is being compiled.
The package provides two macros to describe the compilation context:

%%%%%%%%%%%%%%%%%%%%%%%%%%%%%%%%%%%%%%%%
\DescribeMacro{\ifchilddoc}
The conditional |\ifchilddoc| distinguishes between the compilation of
child documents and the main document:
%
\begin{center}
|\ifchilddoc |\textit{child-code}| |[|\||else |\textit{main-code}]| \||fi|
\end{center}

%%%%%%%%%%%%%%%%%%%%%%%%%%%%%%%%%%%%%%%%
\DescribeMacro{\childdocname}
\DescribeMacro{\childdocjob}
The macro |\childdocname| contains the filename (without extension)
of the main or child file being processed.
Note that |\childdocjob| will always contain the name of the main file.

%%%%%%%%%%%%%%%%%%%%%%%%%%%%%%%%%%%%%%%%
\paragraph{Title Page.}

Conditional processing can be used to include a title or banner page
in the main document when proper precautions are taken.
Importantly, the code in the main file should ensure that the page counter
(as well as other status parameters which are stored in the |.aux| files)
takes the same value after the conditional processing.
Otherwise the page numbers may take divergent values
depending on which part is compiled.

For example, a title page could be declared by:
%
\begin{center}
\begin{tabular}{l}
|\ifchilddoc\||else|\\
|\addtocounter{page}{-1}|\\
\textit{code for title page}\\
|\newpage|\\
|\||fi|
\end{tabular}
\end{center}
%
A banner page for the child documents can be generated by:
%
\begin{center}
\begin{tabular}{l}
|\ifchilddoc|\\
|\addtocounter{page}{-1}|\\
\textit{code for banner page}\\
|\newpage|\\
|\||fi|
\end{tabular}
\end{center}
%
Here one could write a message such as:
\begin{center}
|This is the part \childdocname{} of \childdocjob{}.|
\end{center}

%%%%%%%%%%%%%%%%%%%%%%%%%%%%%%%%%%%%%%%%%%%%%%%%%%%%%%%%%%%%%%%%%%%%%%%%%%%%%%%%
\subsection{Flags}
\label{sec:flags}

The package makes it easy to generate different versions
of the main or child documents.
To this end compilation flags can be defined
and assigned different default values.
They will be particularly useful in conjunction
with the forwarding mechanism described in \secref{sec:forward}.

For example, it may be useful to have a flag |\version|
which can be set to |draft| or |final|.
The document source will contain some conditional code
depending on the value of |\version|.
Suppose further, the flag should default to |final| for the main file
and to |draft| for child files
which is a natural assignment for editing the document.
This is achieved by placing the following code
in the preamble of the main document
(below the |\childdocmain| directive):
%
\begin{center}
\begin{tabular}{l}
|\ifchilddoc|\\
|\providecommand{\version}{draft}|\\
|\||else|\\
|\providecommand{\version}{final}|\\
|\||fi|
\end{tabular}
\end{center}
%
The definition by |\providecommand| makes sure
that previous definitions are not overwritten.
Further statements |\providecommand{\version}{...}|
can thus be added before the above code to override it.

For the main file, one might add a line
(between |\childdocmain| and the above block)
%
\begin{center}
|%\ifchilddoc\||else\providecommand{\version}{draft}\||fi|
\end{center}
%
which can be uncommented to produce a draft version.
Likewise one can add a line to the very top of a child file
(above the |\childdocof{|\textit{main}|}| directive)
%
\begin{center}
|%\providecommand{\version}{final}|
\end{center}
%
which can be uncommented to produce the final version of this child document.

%%%%%%%%%%%%%%%%%%%%%%%%%%%%%%%%%%%%%%%%%%%%%%%%%%%%%%%%%%%%%%%%%%%%%%%%%%%%%%%%
\subsection{Forwarding}
\label{sec:forward}

Different versions of the main or child documents
using compilation flags as described in \secref{sec:flags}
can be (permanently) stored in different files
for convenient compilation, viewing and distribution.
To this end, the package defines a command
to pass on compilation to a different file:

%%%%%%%%%%%%%%%%%%%%%%%%%%%%%%%%%%%%%%%%
\DescribeMacro{\childdocforward}
The command |\childdocforward| redirects processing to
another source file:
%
\begin{center}
\begin{tabular}{l}
|\input{childdoc.def}|\\
|\childdocforward[|\textit{main}|]{|\textit{dest}|}|\\
\end{tabular}
\end{center}
%
The argument \textit{dest} is the destination file
(without extension).
It should be the main file or one of the child files.
Note that further \textsf{childdoc} directives
such as |\childdocof| and |\childdocforward|
in the indicated file will be processed in this form.
The optional argument \textit{main}
passes on directly to the main file \textit{main}
while pretending to compile the child \textit{dest}.
This form behaves as if \textit{dest}
issues |\childdocof{|\textit{main}|}| right away,
and no further \textsf{childdoc} directives will be processed.

%%%%%%%%%%%%%%%%%%%%%%%%%%%%%%%%%%%%%%%%
\DescribeMacro{\...prefix}
In the alternative form |\childdocforwardprefix|,
%
\begin{center}
\begin{tabular}{l}
|\input{childdoc.def}|\\
|\childdocforwardprefix[|\textit{main}|]{|\textit{prefix}|}{|\textit{dest}|}|
\end{tabular}
\end{center}
%
the destination file is determined by a pattern
depending on the current file:
To make this work, the current file must be called
`{\textit{prefix}\hspace{0.2em}\textit{suffix}}'
with \textit{prefix} matching precisely the argument.
Processing is then passed on to the file
`{\textit{dest}\hspace{0.2em}\textit{suffix}}'.
Surely, the same effect is achieved by
directly specifying the
argument `{\textit{dest}\hspace{0.2em}\textit{suffix}}'
in the first form.
However, that requires to set up a different file
for each child. With the alternative form of the command
all these files can have exactly the same content
which simplifies setting them up and maintaining them.

For example, the following file |draft.tex|
with a compilation flag |\version| as described in \secref{sec:flags}
compiles the main document as a draft:
%
\begin{center}
\begin{tabular}{l}
|\def\version{draft}|\\
|\input{childdoc.def}|\\
|\childdocforward{|\textit{main}|}|
\end{tabular}
\end{center}
%
Likewise, the following files |final|\textit{nn}|.tex|
compile the final version of the child document
|child|\textit{nn}|.tex|:
%
\begin{center}
\begin{tabular}{l}
|\def\version{final}|\\
|\input{childdoc.def}|\\
|\childdocforwardprefix{final}{child}|
\end{tabular}
\end{center}
%

Note that when several versions of a main file and/or of each child file
are to be generated, it may be convenient to set up a |Makefile| or
shell script to automatise the process.

%%%%%%%%%%%%%%%%%%%%%%%%%%%%%%%%%%%%%%%%%%%%%%%%%%%%%%%%%%%%%%%%%%%%%%%%%%%%%%%%
\subsection{Command Line Processing}
\label{sec:commandline}

The effect of redirection files can also be achieved by invoking
the \LaTeX{} compiler with a more elaborate command line.
Most conveniently this should be done as part
of a shell script or a |Makefile|.

When using \textsf{childdoc} in the main file, the following
command lines effectively perform a redirection
(note that depending on the shell being used,
backslashes may have to be doubled: `|\|' $\to$ `|\\|'):
%
\begin{center}
|... -jobname "|\textit{target}|" |\\|"|[\textit{flags}]%
|\input{childdoc.def}\childdocforward[|\textit{main}|]{|\textit{dest}|}"|
\end{center}
%
Here \textit{target} is the name of the output file,
\textit{main} is the name of the main file
and \textit{dest} is the name of the main or child file to be processed
(all filenames without extensions).
The optional argument \textit{main} can be omitted
if \textit{main} matches \textit{dest}.
Optionally, compilation \textit{flags} can be defined via |\def| commands.
This command line makes the \TeX{} engine believe
it is compiling the file \textit{target}
whose content is specified as the latter parameter.
The provided code then forwards the processing to
\textit{main} or \textit{dest} as described in \secref{sec:forward}.

%%%%%%%%%%%%%%%%%%%%%%%%%%%%%%%%%%%%%%%%%%%%%%%%%%%%%%%%%%%%%%%%%%%%%%%%%%%%%%%%
\subsection{Include by Input}
\label{sec:input}

Including child documents by |\include| has some restrictions by design.
Most notably, the content of a child document always occupies
its own set of pages; pages cannot be shared between child documents.
Usually, this behaviour makes perfect sense
because each child document contain an essential part of the document.
However, in some situations it may be desirable to compose
a document from a collection of parts
without having mandatory page breaks between then.
For this case, the package
provides a mechanism to include parts
by |\input| which can also be processed individually.
However, by construction this mechanism
requires manual handling of the content to be output.

%%%%%%%%%%%%%%%%%%%%%%%%%%%%%%%%%%%%%%%%
\DescribeMacro{\ifchilddocmanual}
The main file should be prepared as usual, see \secref{sec:include}.
However, the document body must make a distinction
between processing of an individual part and of the main document, e.g.:
%
\begin{center}
\begin{tabular}{l}
|\ifchilddocmanual|\\
|\input{\childdocname}|\\
|\||else|\\
\textit{document body with }|\input{|\textit{part}|}|\\
|\||fi|
\end{tabular}
\end{center}
%
The conditional |\ifchilddocmanual| is true whenever
a part to be included by |\input| is being compiled,
and the name of the part is stored in |\childdocname|.

%%%%%%%%%%%%%%%%%%%%%%%%%%%%%%%%%%%%%%%%
\DescribeMacro{\childdocby}
Each part to be included by |\input| should start with:
%
\begin{center}
\begin{tabular}{l}
|\input{childdoc.def}|\\
|\childdocby{|\textit{main}|}|\\
\end{tabular}
\end{center}
%
The directive |\childdocby| is similar to |\childdocof|
described in \secref{sec:include},
but the subsequent selection of content must be done manually.
To that end, both |\ifchilddoc| and |\ifchilddocmanual|
will be true upon processing of a part,
and the name of the part is stored in |\childdocname|.
Note that |\jobname| will be set to the filename of the current part
so that each part receives an individual |.aux| file
that does not interfere with the |.aux| file(s) of the main document.
This behaviour can be altered by the alternative form
|\childdocby[*]{|\textit{main}|}| (with a non-empty optional argument)
which uses the |.aux| file of the main document
by setting |\jobname| to \textit{main}.

%%%%%%%%%%%%%%%%%%%%%%%%%%%%%%%%%%%%%%%%%%%%%%%%%%%%%%%%%%%%%%%%%%%%%%%%%%%%%%%%
\subsection{Driver Development}
\label{sec:driver}

The \textsf{childdoc} mechanism can also be use for the development
of definition files such as \LaTeX{} styles or classes.
This case differs from the above setup with multiple parts
included by |\include| in that no |\includeonly| should be invoked.
This can be achieved by starting the include file
(before |\ProvidesPackage|) with:
%
\begin{center}
\begin{tabular}{l}
|\input{childdoc.def}|\\
|\childdocforward{|\textit{main}|}|\\
\end{tabular}
\end{center}
%
or alternatively with:
%
\begin{center}
\begin{tabular}{l}
|\input{childdoc.def}|\\
|\childdocby{|\textit{main}|}|\\
\end{tabular}
\end{center}
%
Both forms have slightly different effects as described above.
The main file is prepared as usual, see \secref{sec:include}.

%%%%%%%%%%%%%%%%%%%%%%%%%%%%%%%%%%%%%%%%%%%%%%%%%%%%%%%%%%%%%%%%%%%%%%%%%%%%%%%%
\subsection{Legacy Detection}
\label{sec:detection}

The directive |\childdocmain| in the main file can detect
whether the complete document or merely a child is to be compiled
even without using the directive |\childdocof|.
This method is deprecated because it is less robust
and there is no compelling reason to use it;
it is merely provided for backward compatibility
and it may be removed in future versions.

If the detection mechanism is to be used,
it is mandatory to correctly specify
the filename of the main file as the argument of |\childdocmain|:
%
\begin{center}
\begin{tabular}{l}
|\input{childdoc.def}|\\
|\childdocmain{|\textit{main}|}|\\
\end{tabular}
\end{center}
%
If |\jobname| does not match the argument \textit{main} of |\childdocmain|,
it is assumed that |\jobname| points to the child file to be compiled.
When using |\childdocmain| with the main file specified as argument,
it suffices to start a child file
with just |\input{|\textit{main}|}|
without loading of the package and using |\childdocof|.
If instead all processing is done
with the appropriate \textsf{childdoc} directives,
the argument of \textit{main} of |\childdocmain| can be empty.

An alternative version of the command line processing described
in \secref{sec:commandline} using the detection mechanism reads:
%
\begin{center}
|... -jobname "|\textit{target}|" "|[\textit{flags}]%
[|\def\jobname{|\textit{dest}|}|]|\input{|\textit{main}|}"|
\end{center}

%%%%%%%%%%%%%%%%%%%%%%%%%%%%%%%%%%%%%%%%%%%%%%%%%%%%%%%%%%%%%%%%%%%%%%%%%%%%%%%%
\subsection{Manual Code}
\label{sec:manual}

In case one cannot be certain whether the definitions file |childdoc.def|
is installed on the target \TeX{} distribution
and one prefers not to ship it,
it is conceivable to paste a few relevant commands into the sources.

To that end, drop all statements |\input{childdoc.def}|
and perform the replacements as outlined below.
Instead of |\childdocmain{|\textit{main}|}| add the following code
to the top of the main file:
%
\begin{center}
\begin{tabular}{l}
|\||ifdefined\childdocname\endinput\||fi\newif\ifchilddoc|\\
|\edef\childdocname{\scantokens\expandafter{\jobname\noexpand}}|\\
|\def\childdocmain{|\textit{main}|}\||ifx\childdocmain\childdocname\||else|\\
|\childdoctrue\includeonly{\childdocname}\let\jobname\childdocmain\||fi|\\
\end{tabular}
\end{center}
%
Instead of |\childdocof{|\textit{main}|}| just include the main file
at the top of each child file:
%
\begin{center}
|\input{|\textit{main}|}|
\end{center}
%
A simple redirection |\childdocforward{|\textit{dest}|}| is achieved by:
%
\begin{center}
|\def\jobname{|\textit{dest}|}\input{\jobname}|
\end{center}
%
The redirection with prefix
|\childdocforwardprefix[|\textit{prefix}|]{|\textit{dest}|}|
is accomplished by:
%
\begin{center}
\begin{tabular}{l}
|{\edef\jobname{\scantokens\expandafter{\jobname\noexpand}}|\\
|\def\redirectjob |\textit{prefix}|#1~~~{\gdef\jobname{|\textit{dest}|#1}}|\\
|\expandafter\redirectjob\jobname~~~}\input{\jobname}|
\end{tabular}
\end{center}

In an alternative approach,
child documents can be compiled by a specific command line
without additional code or specific definitions:
%
\begin{center}
|... -jobname "|\textit{target}|" "|[\textit{flags}]%
|\includeonly{|\textit{dest}|}\input{|\textit{main}|}"|
\end{center}
%

%%%%%%%%%%%%%%%%%%%%%%%%%%%%%%%%%%%%%%%%%%%%%%%%%%%%%%%%%%%%%%%%%%%%%%%%%%%%%%%%
%%%%%%%%%%%%%%%%%%%%%%%%%%%%%%%%%%%%%%%%%%%%%%%%%%%%%%%%%%%%%%%%%%%%%%%%%%%%%%%%
\section{Information}

%%%%%%%%%%%%%%%%%%%%%%%%%%%%%%%%%%%%%%%%%%%%%%%%%%%%%%%%%%%%%%%%%%%%%%%%%%%%%%%%
\subsection{Copyright}

Copyright \copyright{} 2017--2018 Niklas Beisert

This work may be distributed and/or modified under the
conditions of the \LaTeX{} Project Public License, either version 1.3
of this license or (at your option) any later version.
The latest version of this license is in
  \url{http://www.latex-project.org/lppl.txt}
and version 1.3 or later is part of all distributions of \LaTeX{}
version 2005/12/01 or later.

This work has the LPPL maintenance status `maintained'.

The Current Maintainer of this work is Niklas Beisert.

This work consists of the files |README.txt|, |childdoc.ins| and |childdoc.dtx|
as well as the derived files |childdoc.def|, |cdocsamp.tex|
with |cdocsch1.tex|, |cdocsch2.tex|, |cdocspt3.tex|, |cdocspt4.tex|,
|cdocsdrf.tex|, |cdocsfn1.tex|, |cdocsfn2.tex|
as well as |childdoc.pdf|.

%%%%%%%%%%%%%%%%%%%%%%%%%%%%%%%%%%%%%%%%%%%%%%%%%%%%%%%%%%%%%%%%%%%%%%%%%%%%%%%%
\subsection{Files and Installation}

The package consists of the files:
%
\begin{center}
\begin{tabular}{ll}
    |README.txt|   & readme file \\
    |childdoc.ins| & installation file \\
    |childdoc.dtx| & source file \\
    |childdoc.def| & definition file \\
    |cdocsamp.tex| & sample main file \\
    |cdocsch1.tex| & sample include file \\
    |cdocsch2.tex| & sample include file \\
    |cdocspt3.tex| & sample part file \\
    |cdocspt4.tex| & sample part file \\
    |cdocsdrf.tex| & sample redirection file \\
    |cdocsfn1.tex| & sample redirection file \\
    |cdocsfn2.tex| & sample redirection file \\
    |childdoc.pdf| & manual
\end{tabular}
\end{center}
%
The distribution consists of the files
|README.txt|, |childdoc.ins| and |childdoc.dtx|.
%
\begin{itemize}
\item
Run (pdf)\LaTeX{} on |childdoc.dtx|
to compile the manual |childdoc.pdf| (this file).
\item
Run \LaTeX{} on |childdoc.ins| to create the definitions file |childdoc.def|
and the sample |cdocsamp.tex| with include files
|cdocsch1.tex|, |cdocsch2.tex|, |cdocspt3.tex|, |cdocspt4.tex|,
|cdocsdrf.tex|, |cdocsfn1.tex|, |cdocsfn2.tex|.
Then copy the file |childdoc.def| to an appropriate directory of your \LaTeX{}
distribution, e.g.\ \textit{texmf-root}|/tex/latex/childdoc|.
\end{itemize}

%%%%%%%%%%%%%%%%%%%%%%%%%%%%%%%%%%%%%%%%%%%%%%%%%%%%%%%%%%%%%%%%%%%%%%%%%%%%%%%%
\subsection{Related CTAN Packages}

There are several other packages which offer a similar functionality:
%
\begin{itemize}
\item
The packages
\href{http://ctan.org/pkg/docmute}{\textsf{docmute}},
\href{http://ctan.org/pkg/includex}{\textsf{includex}} and
\href{http://ctan.org/pkg/standalone}{\textsf{standalone}}
provide commands to include only the document body of
a child file thus allowing both files to be compiled individually.
\item
The packages \href{http://ctan.org/pkg/subdocs}{\textsf{subdocs}}
and \href{http://ctan.org/pkg/subfiles}{\textsf{subfiles}}
provide structures in which the main and child documents can be
encapsulated and allowing them to be compiled individually.
The inclusion mechanism is different from the conventional |\include|.
\item
The package \href{http://ctan.org/pkg/combine}{\textsf{combine}}
is an elaborate solution to combine several documents into one.
\end{itemize}
%
See also the CTAN topic \href{http://ctan.org/topic/subdocs}{\textsf{subdocs}}
for further related packages.
The present package differs from the above solutions in that
a document structure constructed with the conventional |\include| mechanism
just needs two extra commands at the top of every file
such that all constituent files can be compiled individually.

%%%%%%%%%%%%%%%%%%%%%%%%%%%%%%%%%%%%%%%%%%%%%%%%%%%%%%%%%%%%%%%%%%%%%%%%%%%%%%%%
%\subsection{Feature Suggestions}
%
%The following is a list of features which may be useful for future
%versions of this package:
%%
%\begin{itemize}
%\item
%\ldots
%\end{itemize}

%%%%%%%%%%%%%%%%%%%%%%%%%%%%%%%%%%%%%%%%%%%%%%%%%%%%%%%%%%%%%%%%%%%%%%%%%%%%%%%%
\subsection{Revision History}

%%%%%%%%%%%%%%%%%%%%%%%%%%%%%%%%%%%%%%%%
\paragraph{v2.0:} 2018/12/30

\begin{itemize}
\item
immediate forward processing
\item
added |\childdocby| mechanism
\item
manual restructured
\end{itemize}

%%%%%%%%%%%%%%%%%%%%%%%%%%%%%%%%%%%%%%%%
\paragraph{v1.6:} 2018/01/17

\begin{itemize}
\item
application for development of include files
\item
corrections to manual
\end{itemize}

%%%%%%%%%%%%%%%%%%%%%%%%%%%%%%%%%%%%%%%%
\paragraph{v1.5:} 2017/05/21

\begin{itemize}
\item
more complete structuring introduced
\item
|\childdocof| introduced
\item
|\childdoc| renamed to |\childdocmain|
\item
|\childredirect| renamed to |\childdocforward| and |\childdocforwardprefix|
and functionality expanded
\end{itemize}

%%%%%%%%%%%%%%%%%%%%%%%%%%%%%%%%%%%%%%%%
\paragraph{v1.0:} 2017/04/27

\begin{itemize}
\item
manual and install package
\item
first version published on CTAN
\end{itemize}

%%%%%%%%%%%%%%%%%%%%%%%%%%%%%%%%%%%%%%%%
\paragraph{v0.6:} 2017/04/26

\begin{itemize}
\item
redirection mechanism added
\end{itemize}

%%%%%%%%%%%%%%%%%%%%%%%%%%%%%%%%%%%%%%%%
\paragraph{v0.5:} 2017/04/26

\begin{itemize}
\item
functionality in definition file
\end{itemize}


%%%%%%%%%%%%%%%%%%%%%%%%%%%%%%%%%%%%%%%%%%%%%%%%%%%%%%%%%%%%%%%%%%%%%%%%%%%%%%%%
%%%%%%%%%%%%%%%%%%%%%%%%%%%%%%%%%%%%%%%%%%%%%%%%%%%%%%%%%%%%%%%%%%%%%%%%%%%%%%%%
%%%%%%%%%%%%%%%%%%%%%%%%%%%%%%%%%%%%%%%%%%%%%%%%%%%%%%%%%%%%%%%%%%%%%%%%%%%%%%%%
\appendix

\settowidth\MacroIndent{\rmfamily\scriptsize 000\ }

 \DocInput{childdoc.dtx}

\end{document}
%</driver>
% \fi
%
% %%%%%%%%%%%%%%%%%%%%%%%%%%%%%%%%%%%%%%%%%%%%%%%%%%%%%%%%%%%%%%%%%%%%%%%%%%%%%%
% %%%%%%%%%%%%%%%%%%%%%%%%%%%%%%%%%%%%%%%%%%%%%%%%%%%%%%%%%%%%%%%%%%%%%%%%%%%%%%
% \section{Sample}
%\iffalse
%<*samplemain>
%\fi
%
% The following presents a sample document
% with two chapters, two parts, a title page,
% a compile flag as well as three forwarding files to set the flag.
% It consists of eight |.tex| files:
% \begin{center}
% \begin{tabular}{ll}
% |cdocsamp.tex|&main file\\
% |cdocsch1.tex|&include file for chapter 1\\
% |cdocsch2.tex|&include file for chapter 2\\
% |cdocspt3.tex|&include file for part 3\\
% |cdocspt4.tex|&include file for part 4\\
% |cdocsdrf.tex|&forwarding file for main file in draft mode\\
% |cdocsfi1.tex|&forwarding file for final version of chapter 1\\
% |cdocsfi2.tex|&forwarding file for final version of chapter 2\\
% \end{tabular}
% \end{center}
% Each of the eight files can be compiled directly by the \LaTeX{} compiler.
%
% %%%%%%%%%%%%%%%%%%%%%%%%%%%%%%%%%%%%%%
% \paragraph{Main File.}
%
% The main file is called |cdocsamp.tex|.
%
% Load the \textsf{childdoc} definitions and
% declare the filename for the main document:
%    \begin{macrocode}
\input{childdoc.def}
\childdocmain{}
%    \end{macrocode}

% Optional override for |\version| flag:
%    \begin{macrocode}
%%\ifchilddoc\else\providecommand{\version}{draft}\fi
%    \end{macrocode}

% Define the default values for the |\version| flag
% (|final| for the main file and |draft| for childs):
%    \begin{macrocode}
\ifchilddoc
\providecommand{\version}{draft}
\else
\providecommand{\version}{final}
\fi
%    \end{macrocode}

% Load the standard document class:
%    \begin{macrocode}
\documentclass[12pt]{article}
%    \end{macrocode}

% Start the document body:
%    \begin{macrocode}
\begin{document}
%    \end{macrocode}

% Declare a title page.
% Print title, part of document being processed and version flag:
%    \begin{macrocode}
\addtocounter{page}{-1}
\begin{center}
{\LARGE\bfseries{}childdoc example\par}
\vspace{1cm}
\ifchilddoc
\ifchilddocmanual part\else chapter\fi:
`\childdocname' of `\childdocjob'\par
\else
main document: `\childdocjob'\par
\fi
version: \version\par
\end{center}
\newpage
%    \end{macrocode}

% Manually include selected file,
% otherwise process as usual:
%    \begin{macrocode}
\ifchilddocmanual
\section*{part `\childdocname'}
\input{\childdocname}
\else
%    \end{macrocode}

% Include the two chapters:
%    \begin{macrocode}
\include{cdocsch1}
\include{cdocsch2}
%    \end{macrocode}

% Include the two parts unless only chapters should be displayed:
%    \begin{macrocode}
\ifchilddoc\else
\section{part three}
\input{cdocspt3}
\section{part four}
\input{cdocspt4}
\fi
%    \end{macrocode}

% Process as usual until here:
%    \begin{macrocode}
\fi
%    \end{macrocode}

% End of document body:
%    \begin{macrocode}
\end{document}
%    \end{macrocode}
%\iffalse
%</samplemain>
%\fi
%
% %%%%%%%%%%%%%%%%%%%%%%%%%%%%%%%%%%%%%%
% \paragraph{Chapter Include Files.}
%
% The include files are called |cdocsch1.tex| and |cdocsch2.tex|.
%
%\iffalse
%<*samplechap1|samplechap2>
%\fi

% Optional override for |\version| flag:
%    \begin{macrocode}
%%\providecommand{\version}{final}
%    \end{macrocode}

% Include the main document:
%    \begin{macrocode}
\input{childdoc.def}
\childdocof{cdocsamp}
%    \end{macrocode}

%\iffalse
%</samplechap1|samplechap2>
%\fi
%
%\iffalse
%<*samplechap1>
%\fi
% Some text for chapter 1:
%    \begin{macrocode}
\section{one}
some text in chapter one
%    \end{macrocode}

%\iffalse
%</samplechap1>
%\fi
% Some text for chapter 2:
%\iffalse
%<*samplechap2>
%\fi
%    \begin{macrocode}
\section{two}
more text in chapter two
%    \end{macrocode}

%\iffalse
%</samplechap2>
%\fi
%
% %%%%%%%%%%%%%%%%%%%%%%%%%%%%%%%%%%%%%%
% \paragraph{Part Include Files.}
%
% The include files are called |cdocspt3.tex| and |cdocspt4.tex|.
%
%\iffalse
%<*samplepart3|samplepart4>
%\fi

% Optional override for |\version| flag:
%    \begin{macrocode}
%%\providecommand{\version}{final}
%    \end{macrocode}

% Include the main document:
%    \begin{macrocode}
\input{childdoc.def}
\childdocby{cdocsamp}
%    \end{macrocode}

%\iffalse
%</samplepart3|samplepart4>
%\fi
%
%\iffalse
%<*samplepart3>
%\fi
% Some text for part 3:
%    \begin{macrocode}
some text in part three
%    \end{macrocode}

%\iffalse
%</samplepart3>
%\fi
% Some text for part 4:
%\iffalse
%<*samplepart4>
%\fi
%    \begin{macrocode}
more text in part four
%    \end{macrocode}

%\iffalse
%</samplepart4>
%\fi
%
% %%%%%%%%%%%%%%%%%%%%%%%%%%%%%%%%%%%%%%
% \paragraph{Forwarding for a Complete Draft.}
%
% The following forwarding file |cdocsdrf.tex|
% compiles the main document in draft mode:
%\iffalse
%<*sampledraft>
%\fi
%    \begin{macrocode}
\def\version{draft}
\input{childdoc.def}
\childdocforward{cdocsamp}
%    \end{macrocode}

%\iffalse
%</sampledraft>
%\fi
%
% %%%%%%%%%%%%%%%%%%%%%%%%%%%%%%%%%%%%%%
% \paragraph{Forwarding for Final Version of the Chapters.}
%
% The following forwarding files |cdocsfn1.tex| and |cdocsfn2.tex|
% (with identical content)
% compile the final versions of the child documents
% |cdocsch1.tex| and |cdocsch2.tex|, respectively:
%\iffalse
%<*samplefinal>
%\fi
%    \begin{macrocode}
\def\version{final}
\input{childdoc.def}
\childdocforwardprefix[cdocsamp]{cdocsfn}{cdocsch}
%    \end{macrocode}

%\iffalse
%</samplefinal>
%\fi
%
% %%%%%%%%%%%%%%%%%%%%%%%%%%%%%%%%%%%%%%
% \paragraph{Command Line Processing.}
%
% The following three command lines generate the output files
% |cdocscld|, |cdocscl1| and |cdocscl2|
% which should be identical to
% |cdocsdrf|, |cdocsch1| and |cdocsfn2|, respectively:
% \begin{center}
% \begin{tabular}{l}
% |latex -jobname cdocscld \|\\
% |  "\def\version{draft}\input{childdoc.def}\childdocforward{cdocsamp}"|\\
% |latex -jobname cdocscl1 \|\\
% |  "\input{childdoc.def}\childdocforward[cdocsamp]{cdocsch1}"|\\
% |latex -jobname cdocscl2 \|\\
% |  "\def\version{final}\input{childdoc.def}\childdocforward{cdocsch2}"|
% \end{tabular}
% \end{center}
% Note that the trailing backslash on each first line
% merely continues the input to the second line
% (for convenient cut ant paste).
% Furthermore, the command |latex| can be replaced by any
% of its alternative versions such as |pdflatex|.
%
% %%%%%%%%%%%%%%%%%%%%%%%%%%%%%%%%%%%%%%%%%%%%%%%%%%%%%%%%%%%%%%%%%%%%%%%%%%%%%%
% %%%%%%%%%%%%%%%%%%%%%%%%%%%%%%%%%%%%%%%%%%%%%%%%%%%%%%%%%%%%%%%%%%%%%%%%%%%%%%
% \section{Implementation}
%\iffalse
%<*package>
%\fi
%
% This section describes the definitions file |childdoc.def|.

% The definitions cannot be loaded using |\usepackage| or |\RequirePackage|
% which has a mechanism to prevent loading a style file more than once.
% When loading the definitions by means of |\input|
% multiple instances have to be prevented manually:
%\iffalse
%This code needs to be before the `\ProvidesFile' directive
%which is defined at the beginning of this file.
%Therefore it is also placed there and commented out here.
%</package>
%<*discard>
%\fi
%    \begin{macrocode}
\ifdefined\childdocmain\endinput\fi
%    \end{macrocode}
%\iffalse
%</discard>
%<*package>
%\fi
%
% \macro{\ifchilddoc}
% \macro{\ifchilddocmanual}
% The conditional |\ifchilddoc| tells whether a
% child (true) or main (false) document is being compiled.
% The conditional |\ifchilddocmanual| tells whether
% the |\includeonly| mechanism is used (false) or
% the selection of child files must be performed manually (true).
% The definitions initialise to false:
%    \begin{macrocode}
\newif\ifchilddoc
\newif\ifchilddocmanual
%    \end{macrocode}

% \macro{\childdocname}
% \macro{\childdocjob}
% The macro |\childdocname| stores the name of the main document
% to be compiled. The macro |\childdocjob| stores the name of
% the document on which the \LaTeX{} compiler was originally invoked.
% The content of |\jobname| cannot be compared
% to filenames specified in the source due to different catcodes.
% The following code rescans |\jobname|, stores the result
% in |\childdocname| and saves a copy in |\childdocjob|:
%    \begin{macrocode}
\edef\childdocname{\scantokens\expandafter{\jobname\noexpand}}
\let\childdocjob\childdocname
%    \end{macrocode}

% \macro{\childdocdisable}
% The macro |\childdocdisable| prevents the main file
% from being processed more than once.
% At this stage, the main document command |\childdocmain|
% is assumed to be called once again where it should do nothing.
% Any subsequent call to it should prevent
% a secondary processing of the main document
% It overwrites the forwarding commands
% |\childdocof| and |\childdocforward|
% with empty macros to prevent further inclusions of the main document:
%    \begin{macrocode}
\newcommand{\childdocdisable}
{
  \renewcommand{\childdocmain}[1]{\renewcommand{\childdocmain}[1]{\endinput}}
  \renewcommand{\childdocof}[1]{}
  \renewcommand{\childdocby}[2][]{}
  \renewcommand{\childdocforward}[2][]{}
  \renewcommand{\childdocdisable}{}
}
%    \end{macrocode}

% \macro{\childdocmain}
% The macro |\childdocmain| is to be called at the top of the main file
% with nothing or the main filename (without extension) as argument.
% First, it breaks loops.
% If the argument is not empty and does not match |\childdocname|
% (which is set by the first inclusion of |childdoc.def|),
% |\ifchilddoc| is set to true, |\includeonly| is applied to the child file
% and |\jobname| is set to the main file
% (for proper handling of |.aux| files):
%    \begin{macrocode}
\newcommand{\childdocmain}[1]
{
  \childdocdisable\childdocmain{}
  \if?#1?\else
    \begingroup
      \def\childdoctmp{#1}
      \ifx\childdoctmp\childdocname
        \def\childdoctmp{}
      \else
        \def\childdoctmp
        {
          \childdoctrue
          \includeonly{\childdocname}
          \def\childdocjob{#1}
          \def\jobname{#1}
        }
      \fi
      \expandafter
    \endgroup
    \childdoctmp
  \fi
}
%    \end{macrocode}

% \macro{\childdocof}
% The command |\childdocof| redirects
% compilation to the main file |#1|.
%    \begin{macrocode}
\newcommand{\childdocof}[1]
{
  \childdocdisable
  \childdoctrue
  \includeonly{\childdocname}
  \def\jobname{#1}
  \def\childdocjob{#1}
  \input{#1}
}
%    \end{macrocode}

% \macro{\childdocby}
% The command |\childdocby| ....
%    \begin{macrocode}
\newcommand{\childdocby}[2][]
{
  \childdocdisable
  \childdoctrue
  \childdocmanualtrue
  \if?#1?\else
    \def\jobname{#2}
  \fi
  \def\childdocjob{#2}
  \input{#2}
  \endinput
}
%    \end{macrocode}

% \macro{\childdocforward}
% The command |\childdocforward| redirects
% compilation to the main file or
% (if the optional argument is given) a child file.
% Parameters are set as if the main file
% or a child file starting with |\childdocof| was compiled.
% Then compilation is handed over to the main file:
%    \begin{macrocode}
\newcommand{\childdocforward}[2][]
{
  \begingroup
    \if?#1?
      \def\childdoctmp
      {
        \def\childdocname{#2}
        \def\childdocjob{#2}
        \def\jobname{#2}
        \input{#2}
        \endinput
      }
    \else
      \def\childdoctmp
      {
        \childdocdisable
        \def\childdocname{#2}
        \childdoctrue
        \includeonly{#2}
        \def\childdocjob{#1}
        \def\jobname{#1}
        \input{#1}
        \endinput
      }
    \fi
    \expandafter
  \endgroup
  \childdoctmp
}
%    \end{macrocode}

% \macro{\childdocforwardprefix}
% The command |\childdocforwardprefix| redirects
% compilation to the main or a child file by means of a pattern.
% The prefix |#1| in the current filename is replaced by |#2|
% and the suffix of the current filename is kept
% (it is assumed that the filename does not contain the substring `|~~~|'
% which is used as a delimiter).
% Compilation is handed over to the new file by |\childdocforward|:
%    \begin{macrocode}
\newcommand{\childdocforwardprefix}[3][]
{
  \begingroup
    \def\childdocextract #2##1~~~{\def\childdoctmp{\childdocforward[#1]{#3##1}}}
    \expandafter\childdocextract\childdocname~~~
    \expandafter
  \endgroup
  \childdoctmp
}
%    \end{macrocode}

% \macro{\childdoc}
% The deprecated macro |\childdoc| is a legacy version of |\childdocmain|:
%    \begin{macrocode}
\newcommand{\childdoc}{\childdocmain}
%    \end{macrocode}

% \macro{\childdocredirect}
% The deprecated macro |\childdocredirect| is a legacy version
% of |\childdocforward| and |\childdocforwardprefix|:
%    \begin{macrocode}
\newcommand{\childdocredirect}[2][]
{
  \begingroup
    \if?#1?
      \def\childdoctmp{\childdocforward{#2}}
    \else
      \def\childdoctmp{\childdocforwardprefix{#1}{#2}}
    \fi
    \expandafter
  \endgroup
  \childdoctmp
}
%    \end{macrocode}

%\iffalse
%</package>
%\fi
%
\endinput

\childdocmain{}
%    \end{macrocode}

% Optional override for |\version| flag:
%    \begin{macrocode}
%%\ifchilddoc\else\providecommand{\version}{draft}\fi
%    \end{macrocode}

% Define the default values for the |\version| flag
% (|final| for the main file and |draft| for childs):
%    \begin{macrocode}
\ifchilddoc
\providecommand{\version}{draft}
\else
\providecommand{\version}{final}
\fi
%    \end{macrocode}

% Load the standard document class:
%    \begin{macrocode}
\documentclass[12pt]{article}
%    \end{macrocode}

% Start the document body:
%    \begin{macrocode}
\begin{document}
%    \end{macrocode}

% Declare a title page.
% Print title, part of document being processed and version flag:
%    \begin{macrocode}
\addtocounter{page}{-1}
\begin{center}
{\LARGE\bfseries{}childdoc example\par}
\vspace{1cm}
\ifchilddoc
\ifchilddocmanual part\else chapter\fi:
`\childdocname' of `\childdocjob'\par
\else
main document: `\childdocjob'\par
\fi
version: \version\par
\end{center}
\newpage
%    \end{macrocode}

% Manually include selected file,
% otherwise process as usual:
%    \begin{macrocode}
\ifchilddocmanual
\section*{part `\childdocname'}
\input{\childdocname}
\else
%    \end{macrocode}

% Include the two chapters:
%    \begin{macrocode}
\include{cdocsch1}
\include{cdocsch2}
%    \end{macrocode}

% Include the two parts unless only chapters should be displayed:
%    \begin{macrocode}
\ifchilddoc\else
\section{part three}
\input{cdocspt3}
\section{part four}
\input{cdocspt4}
\fi
%    \end{macrocode}

% Process as usual until here:
%    \begin{macrocode}
\fi
%    \end{macrocode}

% End of document body:
%    \begin{macrocode}
\end{document}
%    \end{macrocode}
%\iffalse
%</samplemain>
%\fi
%
% %%%%%%%%%%%%%%%%%%%%%%%%%%%%%%%%%%%%%%
% \paragraph{Chapter Include Files.}
%
% The include files are called |cdocsch1.tex| and |cdocsch2.tex|.
%
%\iffalse
%<*samplechap1|samplechap2>
%\fi

% Optional override for |\version| flag:
%    \begin{macrocode}
%%\providecommand{\version}{final}
%    \end{macrocode}

% Include the main document:
%    \begin{macrocode}
% \iffalse
%
% childdoc.dtx Copyright (C) 2017-2018 Niklas Beisert
%
% This work may be distributed and/or modified under the
% conditions of the LaTeX Project Public License, either version 1.3
% of this license or (at your option) any later version.
% The latest version of this license is in
%   http://www.latex-project.org/lppl.txt
% and version 1.3 or later is part of all distributions of LaTeX
% version 2005/12/01 or later.
%
% This work has the LPPL maintenance status `maintained'.
%
% The Current Maintainer of this work is Niklas Beisert.
%
% This work consists of the files childdoc.dtx and childdoc.ins
% and the derived files childdoc.def and cdocsamp.tex with
% cdocsch1.tex, cdocsch2.tex, cdocsdrf.tex, cdocsfn1.tex, cdocsfn2.tex.
%
%<package>\ifdefined\childdocmain\endinput\fi
%<package>\ProvidesFile{childdoc.def}[2018/12/30 v2.0 child document driver]
%<samplemain>\ProvidesFile{cdocsamp.tex}[2018/12/30 v2.0 sample for childdoc]
%<*driver>
%\ProvidesFile{childdoc.drv}[2018/12/30 v2.0 childdoc reference manual file]
\PassOptionsToClass{10pt,a4paper}{article}
\documentclass{ltxdoc}

\usepackage[margin=35mm]{geometry}
\usepackage{hyperref}
\usepackage{hyperxmp}
\usepackage[usenames]{color}

\hypersetup{colorlinks=true}
\hypersetup{pdfstartview=FitH}
\hypersetup{pdfpagemode=UseNone}
\hypersetup{pdfsource={}}
\hypersetup{pdflang={en-UK}}
\hypersetup{pdfcopyright={Copyright 2017-2018 Niklas Beisert.
  This work may be distributed and/or modified under the
  conditions of the LaTeX Project Public License, either version 1.3
  of this license or (at your option) any later version.}}
\hypersetup{pdflicenseurl={http://www.latex-project.org/lppl.txt}}
\hypersetup{pdfcontactaddress={ETH Zurich, ITP, HIT K,
  Wolfgang-Pauli-Strasse 27}}
\hypersetup{pdfcontactpostcode={8093}}
\hypersetup{pdfcontactcity={Zurich}}
\hypersetup{pdfcontactcountry={Switzerland}}
\hypersetup{pdfcontactemail={nbeisert@itp.phys.ethz.ch}}
\hypersetup{pdfcontacturl={http://people.phys.ethz.ch/\xmptilde nbeisert/}}

\newcommand{\secref}[1]{\hyperref[#1]{section \ref*{#1}}}

\parskip1ex
\parindent0pt
\let\olditemize\itemize
\def\itemize{\olditemize\parskip0pt}

\begin{document}

\title{The \textsf{childdoc} Package}
\hypersetup{pdftitle={The childdoc Package}}
\author{Niklas Beisert\\[2ex]
  Institut f\"ur Theoretische Physik\\
  Eidgen\"ossische Technische Hochschule Z\"urich\\
  Wolfgang-Pauli-Strasse 27, 8093 Z\"urich, Switzerland\\[1ex]
  \href{mailto:nbeisert@itp.phys.ethz.ch}
  {\texttt{nbeisert@itp.phys.ethz.ch}}}
\hypersetup{pdfauthor={Niklas Beisert}}
\hypersetup{pdfsubject={Manual for the LaTeX2e Package childdoc}}
\date{30 December 2018, \textsf{v2.0}}
\maketitle

\begin{abstract}\noindent
\textsf{childdoc} is a \LaTeXe{} package
that enables the direct compilation
of document sections included by |\include|
to individual files.
\end{abstract}

\begingroup
\parskip0ex
\tableofcontents
\endgroup

%%%%%%%%%%%%%%%%%%%%%%%%%%%%%%%%%%%%%%%%%%%%%%%%%%%%%%%%%%%%%%%%%%%%%%%%%%%%%%%%
%%%%%%%%%%%%%%%%%%%%%%%%%%%%%%%%%%%%%%%%%%%%%%%%%%%%%%%%%%%%%%%%%%%%%%%%%%%%%%%%
\section{Introduction}

\LaTeX{} provides a mechanism to structure a large document (such as a book)
into a main file and several child files (containing the chapters)
using the |\include| command.
This mechanism is beneficial for documents
which span hundreds of pages in order to
make the source file(s) more manageable.
Moreover, compilation can be restricted to
selected child files by means of the |\includeonly| command.
The latter feature can be used to reduce the compilation time while editing
(this was significantly more useful in the earlier days of \LaTeX{})
or to generate a smaller document which is easier to navigate.
Another application of |\includeonly| is to generate
documents consisting of selected parts of the complete document.

However, there are a few drawbacks of the plain |\include| mechanism:
\begin{itemize}
\item
The child files cannot be compiled on their own,
they can only be compiled via the main file.
A naive editing environment
(such as a text editor with an option
to have the current file processed by \LaTeX)
may require one to switch to the main file before compiling;
attempting to compile the child file produces errors.
\item
The main file must be modified (each time)
to adjust the |\includeonly| command
to the present needs. This easily leaves the main file in a messy state.
\item
The generated document will always carry the filename
of the main document. This is inconvenient if
several child files are to be compiled and
to be kept for distribution.
\end{itemize}

The present package provides a simple interface
to make child files individually compilable by \LaTeX{}.
Compiling a child file then has the same effect as compiling
the main file with an |\includeonly| command
to select the appropriate child.
Moreover the generated document will carry the name of the child
rather than the main file.
This resolves all three above issues.

This feature is meant to make the editing of books,
thesis documents and lecture notes somewhat more convenient.
However, the package can also be used efficiently for
composing a series of documents (such as exercise sheets)
which are typically distributed individually.
It then assists the author in generating the individual documents
(potentially in different versions)
as well as a document containing the collected series.
Another application is in developing style files
or other kinds of included material
where compilation of the style file could redirect
to a sample or test file.

%%%%%%%%%%%%%%%%%%%%%%%%%%%%%%%%%%%%%%%%%%%%%%%%%%%%%%%%%%%%%%%%%%%%%%%%%%%%%%%%
%%%%%%%%%%%%%%%%%%%%%%%%%%%%%%%%%%%%%%%%%%%%%%%%%%%%%%%%%%%%%%%%%%%%%%%%%%%%%%%%
\section{Usage}

First of all, the package \textsf{childdoc} is \emph{not} a standard
\LaTeXe{} |.sty| style file! Therefore it needs to be invoked in
a non-standard way.

%%%%%%%%%%%%%%%%%%%%%%%%%%%%%%%%%%%%%%%%%%%%%%%%%%%%%%%%%%%%%%%%%%%%%%%%%%%%%%%%
\subsection{Included Files}
\label{sec:include}

%%%%%%%%%%%%%%%%%%%%%%%%%%%%%%%%%%%%%%%%
\DescribeMacro{\childdocmain}
To use the package, add the commands
\begin{center}
\begin{tabular}{l}
|\input{childdoc.def}|\\
|\childdocmain{}|\\
\end{tabular}
\end{center}
at the very top of the main \LaTeX{} file,
in particular \emph{before} the |\documentclass| statement!
The argument of |\childdocmain| should be left empty
(but it must be present).

%%%%%%%%%%%%%%%%%%%%%%%%%%%%%%%%%%%%%%%%
\DescribeMacro{\childdocof}
Furthermore, add the commands
\begin{center}
\begin{tabular}{l}
|\input{childdoc.def}|\\
|\childdocof{|\textit{main}|}|\\
\end{tabular}
\end{center}
at the top of every child file \textit{child}
which is included by |\include{|\textit{child}|}|
from within the main file
(or at least for those files to be compiled individually).
The argument \textit{main} must be the filename of the main file.

There are a couple of
considerations in setting up the main and child documents:

%%%%%%%%%%%%%%%%%%%%%%%%%%%%%%%%%%%%%%%%
\paragraph{Restrictions.}

Please note the following restrictions:
\begin{itemize}
\item
|\childdocmain| must be called with one argument \textit{main}
to ensure compatibility with earlier version of the package.
It must either be empty (|\childdocmain{}|)
or precisely match the filename of the main file in which it is specified.
See \secref{sec:detection} for further information.
\item
The filename \textit{main} must be specified without the |.tex| extension.
\item
The filename \textit{main} is case sensitive
(even in case-insensitive file systems)
due to internal string comparison.
\item
The argument \textit{main} should be fully expanded, it cannot be a macro.
\item
Subdirectories and special characters should be avoided in filenames.
\item
The command |\childdocmain{|\textit{main}|}| must be followed by a whitespace.
It should not be followed immediately by another command
or by a comment mark `|%|'.
This is because the \TeX{} parser reads the token immediately following
the argument of |\childdocmain| and puts it
at the beginning of every child section;
however, a white\-space is ignored.
\end{itemize}

%%%%%%%%%%%%%%%%%%%%%%%%%%%%%%%%%%%%%%%%
\paragraph{Content of Main File.}

It is advisable to place all content in the child files included by |\include|.
Any output contained in the main file will appear in all child documents
unless suppressed manually;
it cannot be suppressed automatically by the |\includeonly| directive
and thus should normally be avoided.
A method to include some content in the main file
by means of conditional processing is described in \secref{sec:conditional}.

%%%%%%%%%%%%%%%%%%%%%%%%%%%%%%%%%%%%%%%%
\paragraph{Page Numbering.}

When only a part of the document is compiled,
the appropriate numbering of pages
(as well as other status parameters)
is determined from the |.aux| files.
The latter contain information from previous passes.
However this information needs to propagate through
all intermediate child documents.
Therefore the page numbering in child documents may well
be inconsistent until the complete document is compiled at least once.

A useful (if unconventional) way to always ensure a consistent
page numbering is to restart the numbering in each child document
and denote the pages by `\textit{child}|.|\textit{page}'
where \textit{child} represents the chapter/section number of the child file.
This can be achieved by the command
|\numberwithin{page}{|\textit{child}|}|
of the \textsf{amsmath} package
where \textit{child} can be |chapter| or |section|
depending on the chosen structuring.
Alternatively, one can modify the macro |\thepage| appropriately
and reset the counter |page| at the start of each child file.

%%%%%%%%%%%%%%%%%%%%%%%%%%%%%%%%%%%%%%%%%%%%%%%%%%%%%%%%%%%%%%%%%%%%%%%%%%%%%%%%
\subsection{Conditional Processing}
\label{sec:conditional}

The package provides a mechanism to compile different versions
of a document. To customise the versions further some conditional processing
can come in handy to distinguish which version is being compiled.
The package provides two macros to describe the compilation context:

%%%%%%%%%%%%%%%%%%%%%%%%%%%%%%%%%%%%%%%%
\DescribeMacro{\ifchilddoc}
The conditional |\ifchilddoc| distinguishes between the compilation of
child documents and the main document:
%
\begin{center}
|\ifchilddoc |\textit{child-code}| |[|\||else |\textit{main-code}]| \||fi|
\end{center}

%%%%%%%%%%%%%%%%%%%%%%%%%%%%%%%%%%%%%%%%
\DescribeMacro{\childdocname}
\DescribeMacro{\childdocjob}
The macro |\childdocname| contains the filename (without extension)
of the main or child file being processed.
Note that |\childdocjob| will always contain the name of the main file.

%%%%%%%%%%%%%%%%%%%%%%%%%%%%%%%%%%%%%%%%
\paragraph{Title Page.}

Conditional processing can be used to include a title or banner page
in the main document when proper precautions are taken.
Importantly, the code in the main file should ensure that the page counter
(as well as other status parameters which are stored in the |.aux| files)
takes the same value after the conditional processing.
Otherwise the page numbers may take divergent values
depending on which part is compiled.

For example, a title page could be declared by:
%
\begin{center}
\begin{tabular}{l}
|\ifchilddoc\||else|\\
|\addtocounter{page}{-1}|\\
\textit{code for title page}\\
|\newpage|\\
|\||fi|
\end{tabular}
\end{center}
%
A banner page for the child documents can be generated by:
%
\begin{center}
\begin{tabular}{l}
|\ifchilddoc|\\
|\addtocounter{page}{-1}|\\
\textit{code for banner page}\\
|\newpage|\\
|\||fi|
\end{tabular}
\end{center}
%
Here one could write a message such as:
\begin{center}
|This is the part \childdocname{} of \childdocjob{}.|
\end{center}

%%%%%%%%%%%%%%%%%%%%%%%%%%%%%%%%%%%%%%%%%%%%%%%%%%%%%%%%%%%%%%%%%%%%%%%%%%%%%%%%
\subsection{Flags}
\label{sec:flags}

The package makes it easy to generate different versions
of the main or child documents.
To this end compilation flags can be defined
and assigned different default values.
They will be particularly useful in conjunction
with the forwarding mechanism described in \secref{sec:forward}.

For example, it may be useful to have a flag |\version|
which can be set to |draft| or |final|.
The document source will contain some conditional code
depending on the value of |\version|.
Suppose further, the flag should default to |final| for the main file
and to |draft| for child files
which is a natural assignment for editing the document.
This is achieved by placing the following code
in the preamble of the main document
(below the |\childdocmain| directive):
%
\begin{center}
\begin{tabular}{l}
|\ifchilddoc|\\
|\providecommand{\version}{draft}|\\
|\||else|\\
|\providecommand{\version}{final}|\\
|\||fi|
\end{tabular}
\end{center}
%
The definition by |\providecommand| makes sure
that previous definitions are not overwritten.
Further statements |\providecommand{\version}{...}|
can thus be added before the above code to override it.

For the main file, one might add a line
(between |\childdocmain| and the above block)
%
\begin{center}
|%\ifchilddoc\||else\providecommand{\version}{draft}\||fi|
\end{center}
%
which can be uncommented to produce a draft version.
Likewise one can add a line to the very top of a child file
(above the |\childdocof{|\textit{main}|}| directive)
%
\begin{center}
|%\providecommand{\version}{final}|
\end{center}
%
which can be uncommented to produce the final version of this child document.

%%%%%%%%%%%%%%%%%%%%%%%%%%%%%%%%%%%%%%%%%%%%%%%%%%%%%%%%%%%%%%%%%%%%%%%%%%%%%%%%
\subsection{Forwarding}
\label{sec:forward}

Different versions of the main or child documents
using compilation flags as described in \secref{sec:flags}
can be (permanently) stored in different files
for convenient compilation, viewing and distribution.
To this end, the package defines a command
to pass on compilation to a different file:

%%%%%%%%%%%%%%%%%%%%%%%%%%%%%%%%%%%%%%%%
\DescribeMacro{\childdocforward}
The command |\childdocforward| redirects processing to
another source file:
%
\begin{center}
\begin{tabular}{l}
|\input{childdoc.def}|\\
|\childdocforward[|\textit{main}|]{|\textit{dest}|}|\\
\end{tabular}
\end{center}
%
The argument \textit{dest} is the destination file
(without extension).
It should be the main file or one of the child files.
Note that further \textsf{childdoc} directives
such as |\childdocof| and |\childdocforward|
in the indicated file will be processed in this form.
The optional argument \textit{main}
passes on directly to the main file \textit{main}
while pretending to compile the child \textit{dest}.
This form behaves as if \textit{dest}
issues |\childdocof{|\textit{main}|}| right away,
and no further \textsf{childdoc} directives will be processed.

%%%%%%%%%%%%%%%%%%%%%%%%%%%%%%%%%%%%%%%%
\DescribeMacro{\...prefix}
In the alternative form |\childdocforwardprefix|,
%
\begin{center}
\begin{tabular}{l}
|\input{childdoc.def}|\\
|\childdocforwardprefix[|\textit{main}|]{|\textit{prefix}|}{|\textit{dest}|}|
\end{tabular}
\end{center}
%
the destination file is determined by a pattern
depending on the current file:
To make this work, the current file must be called
`{\textit{prefix}\hspace{0.2em}\textit{suffix}}'
with \textit{prefix} matching precisely the argument.
Processing is then passed on to the file
`{\textit{dest}\hspace{0.2em}\textit{suffix}}'.
Surely, the same effect is achieved by
directly specifying the
argument `{\textit{dest}\hspace{0.2em}\textit{suffix}}'
in the first form.
However, that requires to set up a different file
for each child. With the alternative form of the command
all these files can have exactly the same content
which simplifies setting them up and maintaining them.

For example, the following file |draft.tex|
with a compilation flag |\version| as described in \secref{sec:flags}
compiles the main document as a draft:
%
\begin{center}
\begin{tabular}{l}
|\def\version{draft}|\\
|\input{childdoc.def}|\\
|\childdocforward{|\textit{main}|}|
\end{tabular}
\end{center}
%
Likewise, the following files |final|\textit{nn}|.tex|
compile the final version of the child document
|child|\textit{nn}|.tex|:
%
\begin{center}
\begin{tabular}{l}
|\def\version{final}|\\
|\input{childdoc.def}|\\
|\childdocforwardprefix{final}{child}|
\end{tabular}
\end{center}
%

Note that when several versions of a main file and/or of each child file
are to be generated, it may be convenient to set up a |Makefile| or
shell script to automatise the process.

%%%%%%%%%%%%%%%%%%%%%%%%%%%%%%%%%%%%%%%%%%%%%%%%%%%%%%%%%%%%%%%%%%%%%%%%%%%%%%%%
\subsection{Command Line Processing}
\label{sec:commandline}

The effect of redirection files can also be achieved by invoking
the \LaTeX{} compiler with a more elaborate command line.
Most conveniently this should be done as part
of a shell script or a |Makefile|.

When using \textsf{childdoc} in the main file, the following
command lines effectively perform a redirection
(note that depending on the shell being used,
backslashes may have to be doubled: `|\|' $\to$ `|\\|'):
%
\begin{center}
|... -jobname "|\textit{target}|" |\\|"|[\textit{flags}]%
|\input{childdoc.def}\childdocforward[|\textit{main}|]{|\textit{dest}|}"|
\end{center}
%
Here \textit{target} is the name of the output file,
\textit{main} is the name of the main file
and \textit{dest} is the name of the main or child file to be processed
(all filenames without extensions).
The optional argument \textit{main} can be omitted
if \textit{main} matches \textit{dest}.
Optionally, compilation \textit{flags} can be defined via |\def| commands.
This command line makes the \TeX{} engine believe
it is compiling the file \textit{target}
whose content is specified as the latter parameter.
The provided code then forwards the processing to
\textit{main} or \textit{dest} as described in \secref{sec:forward}.

%%%%%%%%%%%%%%%%%%%%%%%%%%%%%%%%%%%%%%%%%%%%%%%%%%%%%%%%%%%%%%%%%%%%%%%%%%%%%%%%
\subsection{Include by Input}
\label{sec:input}

Including child documents by |\include| has some restrictions by design.
Most notably, the content of a child document always occupies
its own set of pages; pages cannot be shared between child documents.
Usually, this behaviour makes perfect sense
because each child document contain an essential part of the document.
However, in some situations it may be desirable to compose
a document from a collection of parts
without having mandatory page breaks between then.
For this case, the package
provides a mechanism to include parts
by |\input| which can also be processed individually.
However, by construction this mechanism
requires manual handling of the content to be output.

%%%%%%%%%%%%%%%%%%%%%%%%%%%%%%%%%%%%%%%%
\DescribeMacro{\ifchilddocmanual}
The main file should be prepared as usual, see \secref{sec:include}.
However, the document body must make a distinction
between processing of an individual part and of the main document, e.g.:
%
\begin{center}
\begin{tabular}{l}
|\ifchilddocmanual|\\
|\input{\childdocname}|\\
|\||else|\\
\textit{document body with }|\input{|\textit{part}|}|\\
|\||fi|
\end{tabular}
\end{center}
%
The conditional |\ifchilddocmanual| is true whenever
a part to be included by |\input| is being compiled,
and the name of the part is stored in |\childdocname|.

%%%%%%%%%%%%%%%%%%%%%%%%%%%%%%%%%%%%%%%%
\DescribeMacro{\childdocby}
Each part to be included by |\input| should start with:
%
\begin{center}
\begin{tabular}{l}
|\input{childdoc.def}|\\
|\childdocby{|\textit{main}|}|\\
\end{tabular}
\end{center}
%
The directive |\childdocby| is similar to |\childdocof|
described in \secref{sec:include},
but the subsequent selection of content must be done manually.
To that end, both |\ifchilddoc| and |\ifchilddocmanual|
will be true upon processing of a part,
and the name of the part is stored in |\childdocname|.
Note that |\jobname| will be set to the filename of the current part
so that each part receives an individual |.aux| file
that does not interfere with the |.aux| file(s) of the main document.
This behaviour can be altered by the alternative form
|\childdocby[*]{|\textit{main}|}| (with a non-empty optional argument)
which uses the |.aux| file of the main document
by setting |\jobname| to \textit{main}.

%%%%%%%%%%%%%%%%%%%%%%%%%%%%%%%%%%%%%%%%%%%%%%%%%%%%%%%%%%%%%%%%%%%%%%%%%%%%%%%%
\subsection{Driver Development}
\label{sec:driver}

The \textsf{childdoc} mechanism can also be use for the development
of definition files such as \LaTeX{} styles or classes.
This case differs from the above setup with multiple parts
included by |\include| in that no |\includeonly| should be invoked.
This can be achieved by starting the include file
(before |\ProvidesPackage|) with:
%
\begin{center}
\begin{tabular}{l}
|\input{childdoc.def}|\\
|\childdocforward{|\textit{main}|}|\\
\end{tabular}
\end{center}
%
or alternatively with:
%
\begin{center}
\begin{tabular}{l}
|\input{childdoc.def}|\\
|\childdocby{|\textit{main}|}|\\
\end{tabular}
\end{center}
%
Both forms have slightly different effects as described above.
The main file is prepared as usual, see \secref{sec:include}.

%%%%%%%%%%%%%%%%%%%%%%%%%%%%%%%%%%%%%%%%%%%%%%%%%%%%%%%%%%%%%%%%%%%%%%%%%%%%%%%%
\subsection{Legacy Detection}
\label{sec:detection}

The directive |\childdocmain| in the main file can detect
whether the complete document or merely a child is to be compiled
even without using the directive |\childdocof|.
This method is deprecated because it is less robust
and there is no compelling reason to use it;
it is merely provided for backward compatibility
and it may be removed in future versions.

If the detection mechanism is to be used,
it is mandatory to correctly specify
the filename of the main file as the argument of |\childdocmain|:
%
\begin{center}
\begin{tabular}{l}
|\input{childdoc.def}|\\
|\childdocmain{|\textit{main}|}|\\
\end{tabular}
\end{center}
%
If |\jobname| does not match the argument \textit{main} of |\childdocmain|,
it is assumed that |\jobname| points to the child file to be compiled.
When using |\childdocmain| with the main file specified as argument,
it suffices to start a child file
with just |\input{|\textit{main}|}|
without loading of the package and using |\childdocof|.
If instead all processing is done
with the appropriate \textsf{childdoc} directives,
the argument of \textit{main} of |\childdocmain| can be empty.

An alternative version of the command line processing described
in \secref{sec:commandline} using the detection mechanism reads:
%
\begin{center}
|... -jobname "|\textit{target}|" "|[\textit{flags}]%
[|\def\jobname{|\textit{dest}|}|]|\input{|\textit{main}|}"|
\end{center}

%%%%%%%%%%%%%%%%%%%%%%%%%%%%%%%%%%%%%%%%%%%%%%%%%%%%%%%%%%%%%%%%%%%%%%%%%%%%%%%%
\subsection{Manual Code}
\label{sec:manual}

In case one cannot be certain whether the definitions file |childdoc.def|
is installed on the target \TeX{} distribution
and one prefers not to ship it,
it is conceivable to paste a few relevant commands into the sources.

To that end, drop all statements |\input{childdoc.def}|
and perform the replacements as outlined below.
Instead of |\childdocmain{|\textit{main}|}| add the following code
to the top of the main file:
%
\begin{center}
\begin{tabular}{l}
|\||ifdefined\childdocname\endinput\||fi\newif\ifchilddoc|\\
|\edef\childdocname{\scantokens\expandafter{\jobname\noexpand}}|\\
|\def\childdocmain{|\textit{main}|}\||ifx\childdocmain\childdocname\||else|\\
|\childdoctrue\includeonly{\childdocname}\let\jobname\childdocmain\||fi|\\
\end{tabular}
\end{center}
%
Instead of |\childdocof{|\textit{main}|}| just include the main file
at the top of each child file:
%
\begin{center}
|\input{|\textit{main}|}|
\end{center}
%
A simple redirection |\childdocforward{|\textit{dest}|}| is achieved by:
%
\begin{center}
|\def\jobname{|\textit{dest}|}\input{\jobname}|
\end{center}
%
The redirection with prefix
|\childdocforwardprefix[|\textit{prefix}|]{|\textit{dest}|}|
is accomplished by:
%
\begin{center}
\begin{tabular}{l}
|{\edef\jobname{\scantokens\expandafter{\jobname\noexpand}}|\\
|\def\redirectjob |\textit{prefix}|#1~~~{\gdef\jobname{|\textit{dest}|#1}}|\\
|\expandafter\redirectjob\jobname~~~}\input{\jobname}|
\end{tabular}
\end{center}

In an alternative approach,
child documents can be compiled by a specific command line
without additional code or specific definitions:
%
\begin{center}
|... -jobname "|\textit{target}|" "|[\textit{flags}]%
|\includeonly{|\textit{dest}|}\input{|\textit{main}|}"|
\end{center}
%

%%%%%%%%%%%%%%%%%%%%%%%%%%%%%%%%%%%%%%%%%%%%%%%%%%%%%%%%%%%%%%%%%%%%%%%%%%%%%%%%
%%%%%%%%%%%%%%%%%%%%%%%%%%%%%%%%%%%%%%%%%%%%%%%%%%%%%%%%%%%%%%%%%%%%%%%%%%%%%%%%
\section{Information}

%%%%%%%%%%%%%%%%%%%%%%%%%%%%%%%%%%%%%%%%%%%%%%%%%%%%%%%%%%%%%%%%%%%%%%%%%%%%%%%%
\subsection{Copyright}

Copyright \copyright{} 2017--2018 Niklas Beisert

This work may be distributed and/or modified under the
conditions of the \LaTeX{} Project Public License, either version 1.3
of this license or (at your option) any later version.
The latest version of this license is in
  \url{http://www.latex-project.org/lppl.txt}
and version 1.3 or later is part of all distributions of \LaTeX{}
version 2005/12/01 or later.

This work has the LPPL maintenance status `maintained'.

The Current Maintainer of this work is Niklas Beisert.

This work consists of the files |README.txt|, |childdoc.ins| and |childdoc.dtx|
as well as the derived files |childdoc.def|, |cdocsamp.tex|
with |cdocsch1.tex|, |cdocsch2.tex|, |cdocspt3.tex|, |cdocspt4.tex|,
|cdocsdrf.tex|, |cdocsfn1.tex|, |cdocsfn2.tex|
as well as |childdoc.pdf|.

%%%%%%%%%%%%%%%%%%%%%%%%%%%%%%%%%%%%%%%%%%%%%%%%%%%%%%%%%%%%%%%%%%%%%%%%%%%%%%%%
\subsection{Files and Installation}

The package consists of the files:
%
\begin{center}
\begin{tabular}{ll}
    |README.txt|   & readme file \\
    |childdoc.ins| & installation file \\
    |childdoc.dtx| & source file \\
    |childdoc.def| & definition file \\
    |cdocsamp.tex| & sample main file \\
    |cdocsch1.tex| & sample include file \\
    |cdocsch2.tex| & sample include file \\
    |cdocspt3.tex| & sample part file \\
    |cdocspt4.tex| & sample part file \\
    |cdocsdrf.tex| & sample redirection file \\
    |cdocsfn1.tex| & sample redirection file \\
    |cdocsfn2.tex| & sample redirection file \\
    |childdoc.pdf| & manual
\end{tabular}
\end{center}
%
The distribution consists of the files
|README.txt|, |childdoc.ins| and |childdoc.dtx|.
%
\begin{itemize}
\item
Run (pdf)\LaTeX{} on |childdoc.dtx|
to compile the manual |childdoc.pdf| (this file).
\item
Run \LaTeX{} on |childdoc.ins| to create the definitions file |childdoc.def|
and the sample |cdocsamp.tex| with include files
|cdocsch1.tex|, |cdocsch2.tex|, |cdocspt3.tex|, |cdocspt4.tex|,
|cdocsdrf.tex|, |cdocsfn1.tex|, |cdocsfn2.tex|.
Then copy the file |childdoc.def| to an appropriate directory of your \LaTeX{}
distribution, e.g.\ \textit{texmf-root}|/tex/latex/childdoc|.
\end{itemize}

%%%%%%%%%%%%%%%%%%%%%%%%%%%%%%%%%%%%%%%%%%%%%%%%%%%%%%%%%%%%%%%%%%%%%%%%%%%%%%%%
\subsection{Related CTAN Packages}

There are several other packages which offer a similar functionality:
%
\begin{itemize}
\item
The packages
\href{http://ctan.org/pkg/docmute}{\textsf{docmute}},
\href{http://ctan.org/pkg/includex}{\textsf{includex}} and
\href{http://ctan.org/pkg/standalone}{\textsf{standalone}}
provide commands to include only the document body of
a child file thus allowing both files to be compiled individually.
\item
The packages \href{http://ctan.org/pkg/subdocs}{\textsf{subdocs}}
and \href{http://ctan.org/pkg/subfiles}{\textsf{subfiles}}
provide structures in which the main and child documents can be
encapsulated and allowing them to be compiled individually.
The inclusion mechanism is different from the conventional |\include|.
\item
The package \href{http://ctan.org/pkg/combine}{\textsf{combine}}
is an elaborate solution to combine several documents into one.
\end{itemize}
%
See also the CTAN topic \href{http://ctan.org/topic/subdocs}{\textsf{subdocs}}
for further related packages.
The present package differs from the above solutions in that
a document structure constructed with the conventional |\include| mechanism
just needs two extra commands at the top of every file
such that all constituent files can be compiled individually.

%%%%%%%%%%%%%%%%%%%%%%%%%%%%%%%%%%%%%%%%%%%%%%%%%%%%%%%%%%%%%%%%%%%%%%%%%%%%%%%%
%\subsection{Feature Suggestions}
%
%The following is a list of features which may be useful for future
%versions of this package:
%%
%\begin{itemize}
%\item
%\ldots
%\end{itemize}

%%%%%%%%%%%%%%%%%%%%%%%%%%%%%%%%%%%%%%%%%%%%%%%%%%%%%%%%%%%%%%%%%%%%%%%%%%%%%%%%
\subsection{Revision History}

%%%%%%%%%%%%%%%%%%%%%%%%%%%%%%%%%%%%%%%%
\paragraph{v2.0:} 2018/12/30

\begin{itemize}
\item
immediate forward processing
\item
added |\childdocby| mechanism
\item
manual restructured
\end{itemize}

%%%%%%%%%%%%%%%%%%%%%%%%%%%%%%%%%%%%%%%%
\paragraph{v1.6:} 2018/01/17

\begin{itemize}
\item
application for development of include files
\item
corrections to manual
\end{itemize}

%%%%%%%%%%%%%%%%%%%%%%%%%%%%%%%%%%%%%%%%
\paragraph{v1.5:} 2017/05/21

\begin{itemize}
\item
more complete structuring introduced
\item
|\childdocof| introduced
\item
|\childdoc| renamed to |\childdocmain|
\item
|\childredirect| renamed to |\childdocforward| and |\childdocforwardprefix|
and functionality expanded
\end{itemize}

%%%%%%%%%%%%%%%%%%%%%%%%%%%%%%%%%%%%%%%%
\paragraph{v1.0:} 2017/04/27

\begin{itemize}
\item
manual and install package
\item
first version published on CTAN
\end{itemize}

%%%%%%%%%%%%%%%%%%%%%%%%%%%%%%%%%%%%%%%%
\paragraph{v0.6:} 2017/04/26

\begin{itemize}
\item
redirection mechanism added
\end{itemize}

%%%%%%%%%%%%%%%%%%%%%%%%%%%%%%%%%%%%%%%%
\paragraph{v0.5:} 2017/04/26

\begin{itemize}
\item
functionality in definition file
\end{itemize}


%%%%%%%%%%%%%%%%%%%%%%%%%%%%%%%%%%%%%%%%%%%%%%%%%%%%%%%%%%%%%%%%%%%%%%%%%%%%%%%%
%%%%%%%%%%%%%%%%%%%%%%%%%%%%%%%%%%%%%%%%%%%%%%%%%%%%%%%%%%%%%%%%%%%%%%%%%%%%%%%%
%%%%%%%%%%%%%%%%%%%%%%%%%%%%%%%%%%%%%%%%%%%%%%%%%%%%%%%%%%%%%%%%%%%%%%%%%%%%%%%%
\appendix

\settowidth\MacroIndent{\rmfamily\scriptsize 000\ }

 \DocInput{childdoc.dtx}

\end{document}
%</driver>
% \fi
%
% %%%%%%%%%%%%%%%%%%%%%%%%%%%%%%%%%%%%%%%%%%%%%%%%%%%%%%%%%%%%%%%%%%%%%%%%%%%%%%
% %%%%%%%%%%%%%%%%%%%%%%%%%%%%%%%%%%%%%%%%%%%%%%%%%%%%%%%%%%%%%%%%%%%%%%%%%%%%%%
% \section{Sample}
%\iffalse
%<*samplemain>
%\fi
%
% The following presents a sample document
% with two chapters, two parts, a title page,
% a compile flag as well as three forwarding files to set the flag.
% It consists of eight |.tex| files:
% \begin{center}
% \begin{tabular}{ll}
% |cdocsamp.tex|&main file\\
% |cdocsch1.tex|&include file for chapter 1\\
% |cdocsch2.tex|&include file for chapter 2\\
% |cdocspt3.tex|&include file for part 3\\
% |cdocspt4.tex|&include file for part 4\\
% |cdocsdrf.tex|&forwarding file for main file in draft mode\\
% |cdocsfi1.tex|&forwarding file for final version of chapter 1\\
% |cdocsfi2.tex|&forwarding file for final version of chapter 2\\
% \end{tabular}
% \end{center}
% Each of the eight files can be compiled directly by the \LaTeX{} compiler.
%
% %%%%%%%%%%%%%%%%%%%%%%%%%%%%%%%%%%%%%%
% \paragraph{Main File.}
%
% The main file is called |cdocsamp.tex|.
%
% Load the \textsf{childdoc} definitions and
% declare the filename for the main document:
%    \begin{macrocode}
\input{childdoc.def}
\childdocmain{}
%    \end{macrocode}

% Optional override for |\version| flag:
%    \begin{macrocode}
%%\ifchilddoc\else\providecommand{\version}{draft}\fi
%    \end{macrocode}

% Define the default values for the |\version| flag
% (|final| for the main file and |draft| for childs):
%    \begin{macrocode}
\ifchilddoc
\providecommand{\version}{draft}
\else
\providecommand{\version}{final}
\fi
%    \end{macrocode}

% Load the standard document class:
%    \begin{macrocode}
\documentclass[12pt]{article}
%    \end{macrocode}

% Start the document body:
%    \begin{macrocode}
\begin{document}
%    \end{macrocode}

% Declare a title page.
% Print title, part of document being processed and version flag:
%    \begin{macrocode}
\addtocounter{page}{-1}
\begin{center}
{\LARGE\bfseries{}childdoc example\par}
\vspace{1cm}
\ifchilddoc
\ifchilddocmanual part\else chapter\fi:
`\childdocname' of `\childdocjob'\par
\else
main document: `\childdocjob'\par
\fi
version: \version\par
\end{center}
\newpage
%    \end{macrocode}

% Manually include selected file,
% otherwise process as usual:
%    \begin{macrocode}
\ifchilddocmanual
\section*{part `\childdocname'}
\input{\childdocname}
\else
%    \end{macrocode}

% Include the two chapters:
%    \begin{macrocode}
\include{cdocsch1}
\include{cdocsch2}
%    \end{macrocode}

% Include the two parts unless only chapters should be displayed:
%    \begin{macrocode}
\ifchilddoc\else
\section{part three}
\input{cdocspt3}
\section{part four}
\input{cdocspt4}
\fi
%    \end{macrocode}

% Process as usual until here:
%    \begin{macrocode}
\fi
%    \end{macrocode}

% End of document body:
%    \begin{macrocode}
\end{document}
%    \end{macrocode}
%\iffalse
%</samplemain>
%\fi
%
% %%%%%%%%%%%%%%%%%%%%%%%%%%%%%%%%%%%%%%
% \paragraph{Chapter Include Files.}
%
% The include files are called |cdocsch1.tex| and |cdocsch2.tex|.
%
%\iffalse
%<*samplechap1|samplechap2>
%\fi

% Optional override for |\version| flag:
%    \begin{macrocode}
%%\providecommand{\version}{final}
%    \end{macrocode}

% Include the main document:
%    \begin{macrocode}
\input{childdoc.def}
\childdocof{cdocsamp}
%    \end{macrocode}

%\iffalse
%</samplechap1|samplechap2>
%\fi
%
%\iffalse
%<*samplechap1>
%\fi
% Some text for chapter 1:
%    \begin{macrocode}
\section{one}
some text in chapter one
%    \end{macrocode}

%\iffalse
%</samplechap1>
%\fi
% Some text for chapter 2:
%\iffalse
%<*samplechap2>
%\fi
%    \begin{macrocode}
\section{two}
more text in chapter two
%    \end{macrocode}

%\iffalse
%</samplechap2>
%\fi
%
% %%%%%%%%%%%%%%%%%%%%%%%%%%%%%%%%%%%%%%
% \paragraph{Part Include Files.}
%
% The include files are called |cdocspt3.tex| and |cdocspt4.tex|.
%
%\iffalse
%<*samplepart3|samplepart4>
%\fi

% Optional override for |\version| flag:
%    \begin{macrocode}
%%\providecommand{\version}{final}
%    \end{macrocode}

% Include the main document:
%    \begin{macrocode}
\input{childdoc.def}
\childdocby{cdocsamp}
%    \end{macrocode}

%\iffalse
%</samplepart3|samplepart4>
%\fi
%
%\iffalse
%<*samplepart3>
%\fi
% Some text for part 3:
%    \begin{macrocode}
some text in part three
%    \end{macrocode}

%\iffalse
%</samplepart3>
%\fi
% Some text for part 4:
%\iffalse
%<*samplepart4>
%\fi
%    \begin{macrocode}
more text in part four
%    \end{macrocode}

%\iffalse
%</samplepart4>
%\fi
%
% %%%%%%%%%%%%%%%%%%%%%%%%%%%%%%%%%%%%%%
% \paragraph{Forwarding for a Complete Draft.}
%
% The following forwarding file |cdocsdrf.tex|
% compiles the main document in draft mode:
%\iffalse
%<*sampledraft>
%\fi
%    \begin{macrocode}
\def\version{draft}
\input{childdoc.def}
\childdocforward{cdocsamp}
%    \end{macrocode}

%\iffalse
%</sampledraft>
%\fi
%
% %%%%%%%%%%%%%%%%%%%%%%%%%%%%%%%%%%%%%%
% \paragraph{Forwarding for Final Version of the Chapters.}
%
% The following forwarding files |cdocsfn1.tex| and |cdocsfn2.tex|
% (with identical content)
% compile the final versions of the child documents
% |cdocsch1.tex| and |cdocsch2.tex|, respectively:
%\iffalse
%<*samplefinal>
%\fi
%    \begin{macrocode}
\def\version{final}
\input{childdoc.def}
\childdocforwardprefix[cdocsamp]{cdocsfn}{cdocsch}
%    \end{macrocode}

%\iffalse
%</samplefinal>
%\fi
%
% %%%%%%%%%%%%%%%%%%%%%%%%%%%%%%%%%%%%%%
% \paragraph{Command Line Processing.}
%
% The following three command lines generate the output files
% |cdocscld|, |cdocscl1| and |cdocscl2|
% which should be identical to
% |cdocsdrf|, |cdocsch1| and |cdocsfn2|, respectively:
% \begin{center}
% \begin{tabular}{l}
% |latex -jobname cdocscld \|\\
% |  "\def\version{draft}\input{childdoc.def}\childdocforward{cdocsamp}"|\\
% |latex -jobname cdocscl1 \|\\
% |  "\input{childdoc.def}\childdocforward[cdocsamp]{cdocsch1}"|\\
% |latex -jobname cdocscl2 \|\\
% |  "\def\version{final}\input{childdoc.def}\childdocforward{cdocsch2}"|
% \end{tabular}
% \end{center}
% Note that the trailing backslash on each first line
% merely continues the input to the second line
% (for convenient cut ant paste).
% Furthermore, the command |latex| can be replaced by any
% of its alternative versions such as |pdflatex|.
%
% %%%%%%%%%%%%%%%%%%%%%%%%%%%%%%%%%%%%%%%%%%%%%%%%%%%%%%%%%%%%%%%%%%%%%%%%%%%%%%
% %%%%%%%%%%%%%%%%%%%%%%%%%%%%%%%%%%%%%%%%%%%%%%%%%%%%%%%%%%%%%%%%%%%%%%%%%%%%%%
% \section{Implementation}
%\iffalse
%<*package>
%\fi
%
% This section describes the definitions file |childdoc.def|.

% The definitions cannot be loaded using |\usepackage| or |\RequirePackage|
% which has a mechanism to prevent loading a style file more than once.
% When loading the definitions by means of |\input|
% multiple instances have to be prevented manually:
%\iffalse
%This code needs to be before the `\ProvidesFile' directive
%which is defined at the beginning of this file.
%Therefore it is also placed there and commented out here.
%</package>
%<*discard>
%\fi
%    \begin{macrocode}
\ifdefined\childdocmain\endinput\fi
%    \end{macrocode}
%\iffalse
%</discard>
%<*package>
%\fi
%
% \macro{\ifchilddoc}
% \macro{\ifchilddocmanual}
% The conditional |\ifchilddoc| tells whether a
% child (true) or main (false) document is being compiled.
% The conditional |\ifchilddocmanual| tells whether
% the |\includeonly| mechanism is used (false) or
% the selection of child files must be performed manually (true).
% The definitions initialise to false:
%    \begin{macrocode}
\newif\ifchilddoc
\newif\ifchilddocmanual
%    \end{macrocode}

% \macro{\childdocname}
% \macro{\childdocjob}
% The macro |\childdocname| stores the name of the main document
% to be compiled. The macro |\childdocjob| stores the name of
% the document on which the \LaTeX{} compiler was originally invoked.
% The content of |\jobname| cannot be compared
% to filenames specified in the source due to different catcodes.
% The following code rescans |\jobname|, stores the result
% in |\childdocname| and saves a copy in |\childdocjob|:
%    \begin{macrocode}
\edef\childdocname{\scantokens\expandafter{\jobname\noexpand}}
\let\childdocjob\childdocname
%    \end{macrocode}

% \macro{\childdocdisable}
% The macro |\childdocdisable| prevents the main file
% from being processed more than once.
% At this stage, the main document command |\childdocmain|
% is assumed to be called once again where it should do nothing.
% Any subsequent call to it should prevent
% a secondary processing of the main document
% It overwrites the forwarding commands
% |\childdocof| and |\childdocforward|
% with empty macros to prevent further inclusions of the main document:
%    \begin{macrocode}
\newcommand{\childdocdisable}
{
  \renewcommand{\childdocmain}[1]{\renewcommand{\childdocmain}[1]{\endinput}}
  \renewcommand{\childdocof}[1]{}
  \renewcommand{\childdocby}[2][]{}
  \renewcommand{\childdocforward}[2][]{}
  \renewcommand{\childdocdisable}{}
}
%    \end{macrocode}

% \macro{\childdocmain}
% The macro |\childdocmain| is to be called at the top of the main file
% with nothing or the main filename (without extension) as argument.
% First, it breaks loops.
% If the argument is not empty and does not match |\childdocname|
% (which is set by the first inclusion of |childdoc.def|),
% |\ifchilddoc| is set to true, |\includeonly| is applied to the child file
% and |\jobname| is set to the main file
% (for proper handling of |.aux| files):
%    \begin{macrocode}
\newcommand{\childdocmain}[1]
{
  \childdocdisable\childdocmain{}
  \if?#1?\else
    \begingroup
      \def\childdoctmp{#1}
      \ifx\childdoctmp\childdocname
        \def\childdoctmp{}
      \else
        \def\childdoctmp
        {
          \childdoctrue
          \includeonly{\childdocname}
          \def\childdocjob{#1}
          \def\jobname{#1}
        }
      \fi
      \expandafter
    \endgroup
    \childdoctmp
  \fi
}
%    \end{macrocode}

% \macro{\childdocof}
% The command |\childdocof| redirects
% compilation to the main file |#1|.
%    \begin{macrocode}
\newcommand{\childdocof}[1]
{
  \childdocdisable
  \childdoctrue
  \includeonly{\childdocname}
  \def\jobname{#1}
  \def\childdocjob{#1}
  \input{#1}
}
%    \end{macrocode}

% \macro{\childdocby}
% The command |\childdocby| ....
%    \begin{macrocode}
\newcommand{\childdocby}[2][]
{
  \childdocdisable
  \childdoctrue
  \childdocmanualtrue
  \if?#1?\else
    \def\jobname{#2}
  \fi
  \def\childdocjob{#2}
  \input{#2}
  \endinput
}
%    \end{macrocode}

% \macro{\childdocforward}
% The command |\childdocforward| redirects
% compilation to the main file or
% (if the optional argument is given) a child file.
% Parameters are set as if the main file
% or a child file starting with |\childdocof| was compiled.
% Then compilation is handed over to the main file:
%    \begin{macrocode}
\newcommand{\childdocforward}[2][]
{
  \begingroup
    \if?#1?
      \def\childdoctmp
      {
        \def\childdocname{#2}
        \def\childdocjob{#2}
        \def\jobname{#2}
        \input{#2}
        \endinput
      }
    \else
      \def\childdoctmp
      {
        \childdocdisable
        \def\childdocname{#2}
        \childdoctrue
        \includeonly{#2}
        \def\childdocjob{#1}
        \def\jobname{#1}
        \input{#1}
        \endinput
      }
    \fi
    \expandafter
  \endgroup
  \childdoctmp
}
%    \end{macrocode}

% \macro{\childdocforwardprefix}
% The command |\childdocforwardprefix| redirects
% compilation to the main or a child file by means of a pattern.
% The prefix |#1| in the current filename is replaced by |#2|
% and the suffix of the current filename is kept
% (it is assumed that the filename does not contain the substring `|~~~|'
% which is used as a delimiter).
% Compilation is handed over to the new file by |\childdocforward|:
%    \begin{macrocode}
\newcommand{\childdocforwardprefix}[3][]
{
  \begingroup
    \def\childdocextract #2##1~~~{\def\childdoctmp{\childdocforward[#1]{#3##1}}}
    \expandafter\childdocextract\childdocname~~~
    \expandafter
  \endgroup
  \childdoctmp
}
%    \end{macrocode}

% \macro{\childdoc}
% The deprecated macro |\childdoc| is a legacy version of |\childdocmain|:
%    \begin{macrocode}
\newcommand{\childdoc}{\childdocmain}
%    \end{macrocode}

% \macro{\childdocredirect}
% The deprecated macro |\childdocredirect| is a legacy version
% of |\childdocforward| and |\childdocforwardprefix|:
%    \begin{macrocode}
\newcommand{\childdocredirect}[2][]
{
  \begingroup
    \if?#1?
      \def\childdoctmp{\childdocforward{#2}}
    \else
      \def\childdoctmp{\childdocforwardprefix{#1}{#2}}
    \fi
    \expandafter
  \endgroup
  \childdoctmp
}
%    \end{macrocode}

%\iffalse
%</package>
%\fi
%
\endinput

\childdocof{cdocsamp}
%    \end{macrocode}

%\iffalse
%</samplechap1|samplechap2>
%\fi
%
%\iffalse
%<*samplechap1>
%\fi
% Some text for chapter 1:
%    \begin{macrocode}
\section{one}
some text in chapter one
%    \end{macrocode}

%\iffalse
%</samplechap1>
%\fi
% Some text for chapter 2:
%\iffalse
%<*samplechap2>
%\fi
%    \begin{macrocode}
\section{two}
more text in chapter two
%    \end{macrocode}

%\iffalse
%</samplechap2>
%\fi
%
% %%%%%%%%%%%%%%%%%%%%%%%%%%%%%%%%%%%%%%
% \paragraph{Part Include Files.}
%
% The include files are called |cdocspt3.tex| and |cdocspt4.tex|.
%
%\iffalse
%<*samplepart3|samplepart4>
%\fi

% Optional override for |\version| flag:
%    \begin{macrocode}
%%\providecommand{\version}{final}
%    \end{macrocode}

% Include the main document:
%    \begin{macrocode}
% \iffalse
%
% childdoc.dtx Copyright (C) 2017-2018 Niklas Beisert
%
% This work may be distributed and/or modified under the
% conditions of the LaTeX Project Public License, either version 1.3
% of this license or (at your option) any later version.
% The latest version of this license is in
%   http://www.latex-project.org/lppl.txt
% and version 1.3 or later is part of all distributions of LaTeX
% version 2005/12/01 or later.
%
% This work has the LPPL maintenance status `maintained'.
%
% The Current Maintainer of this work is Niklas Beisert.
%
% This work consists of the files childdoc.dtx and childdoc.ins
% and the derived files childdoc.def and cdocsamp.tex with
% cdocsch1.tex, cdocsch2.tex, cdocsdrf.tex, cdocsfn1.tex, cdocsfn2.tex.
%
%<package>\ifdefined\childdocmain\endinput\fi
%<package>\ProvidesFile{childdoc.def}[2018/12/30 v2.0 child document driver]
%<samplemain>\ProvidesFile{cdocsamp.tex}[2018/12/30 v2.0 sample for childdoc]
%<*driver>
%\ProvidesFile{childdoc.drv}[2018/12/30 v2.0 childdoc reference manual file]
\PassOptionsToClass{10pt,a4paper}{article}
\documentclass{ltxdoc}

\usepackage[margin=35mm]{geometry}
\usepackage{hyperref}
\usepackage{hyperxmp}
\usepackage[usenames]{color}

\hypersetup{colorlinks=true}
\hypersetup{pdfstartview=FitH}
\hypersetup{pdfpagemode=UseNone}
\hypersetup{pdfsource={}}
\hypersetup{pdflang={en-UK}}
\hypersetup{pdfcopyright={Copyright 2017-2018 Niklas Beisert.
  This work may be distributed and/or modified under the
  conditions of the LaTeX Project Public License, either version 1.3
  of this license or (at your option) any later version.}}
\hypersetup{pdflicenseurl={http://www.latex-project.org/lppl.txt}}
\hypersetup{pdfcontactaddress={ETH Zurich, ITP, HIT K,
  Wolfgang-Pauli-Strasse 27}}
\hypersetup{pdfcontactpostcode={8093}}
\hypersetup{pdfcontactcity={Zurich}}
\hypersetup{pdfcontactcountry={Switzerland}}
\hypersetup{pdfcontactemail={nbeisert@itp.phys.ethz.ch}}
\hypersetup{pdfcontacturl={http://people.phys.ethz.ch/\xmptilde nbeisert/}}

\newcommand{\secref}[1]{\hyperref[#1]{section \ref*{#1}}}

\parskip1ex
\parindent0pt
\let\olditemize\itemize
\def\itemize{\olditemize\parskip0pt}

\begin{document}

\title{The \textsf{childdoc} Package}
\hypersetup{pdftitle={The childdoc Package}}
\author{Niklas Beisert\\[2ex]
  Institut f\"ur Theoretische Physik\\
  Eidgen\"ossische Technische Hochschule Z\"urich\\
  Wolfgang-Pauli-Strasse 27, 8093 Z\"urich, Switzerland\\[1ex]
  \href{mailto:nbeisert@itp.phys.ethz.ch}
  {\texttt{nbeisert@itp.phys.ethz.ch}}}
\hypersetup{pdfauthor={Niklas Beisert}}
\hypersetup{pdfsubject={Manual for the LaTeX2e Package childdoc}}
\date{30 December 2018, \textsf{v2.0}}
\maketitle

\begin{abstract}\noindent
\textsf{childdoc} is a \LaTeXe{} package
that enables the direct compilation
of document sections included by |\include|
to individual files.
\end{abstract}

\begingroup
\parskip0ex
\tableofcontents
\endgroup

%%%%%%%%%%%%%%%%%%%%%%%%%%%%%%%%%%%%%%%%%%%%%%%%%%%%%%%%%%%%%%%%%%%%%%%%%%%%%%%%
%%%%%%%%%%%%%%%%%%%%%%%%%%%%%%%%%%%%%%%%%%%%%%%%%%%%%%%%%%%%%%%%%%%%%%%%%%%%%%%%
\section{Introduction}

\LaTeX{} provides a mechanism to structure a large document (such as a book)
into a main file and several child files (containing the chapters)
using the |\include| command.
This mechanism is beneficial for documents
which span hundreds of pages in order to
make the source file(s) more manageable.
Moreover, compilation can be restricted to
selected child files by means of the |\includeonly| command.
The latter feature can be used to reduce the compilation time while editing
(this was significantly more useful in the earlier days of \LaTeX{})
or to generate a smaller document which is easier to navigate.
Another application of |\includeonly| is to generate
documents consisting of selected parts of the complete document.

However, there are a few drawbacks of the plain |\include| mechanism:
\begin{itemize}
\item
The child files cannot be compiled on their own,
they can only be compiled via the main file.
A naive editing environment
(such as a text editor with an option
to have the current file processed by \LaTeX)
may require one to switch to the main file before compiling;
attempting to compile the child file produces errors.
\item
The main file must be modified (each time)
to adjust the |\includeonly| command
to the present needs. This easily leaves the main file in a messy state.
\item
The generated document will always carry the filename
of the main document. This is inconvenient if
several child files are to be compiled and
to be kept for distribution.
\end{itemize}

The present package provides a simple interface
to make child files individually compilable by \LaTeX{}.
Compiling a child file then has the same effect as compiling
the main file with an |\includeonly| command
to select the appropriate child.
Moreover the generated document will carry the name of the child
rather than the main file.
This resolves all three above issues.

This feature is meant to make the editing of books,
thesis documents and lecture notes somewhat more convenient.
However, the package can also be used efficiently for
composing a series of documents (such as exercise sheets)
which are typically distributed individually.
It then assists the author in generating the individual documents
(potentially in different versions)
as well as a document containing the collected series.
Another application is in developing style files
or other kinds of included material
where compilation of the style file could redirect
to a sample or test file.

%%%%%%%%%%%%%%%%%%%%%%%%%%%%%%%%%%%%%%%%%%%%%%%%%%%%%%%%%%%%%%%%%%%%%%%%%%%%%%%%
%%%%%%%%%%%%%%%%%%%%%%%%%%%%%%%%%%%%%%%%%%%%%%%%%%%%%%%%%%%%%%%%%%%%%%%%%%%%%%%%
\section{Usage}

First of all, the package \textsf{childdoc} is \emph{not} a standard
\LaTeXe{} |.sty| style file! Therefore it needs to be invoked in
a non-standard way.

%%%%%%%%%%%%%%%%%%%%%%%%%%%%%%%%%%%%%%%%%%%%%%%%%%%%%%%%%%%%%%%%%%%%%%%%%%%%%%%%
\subsection{Included Files}
\label{sec:include}

%%%%%%%%%%%%%%%%%%%%%%%%%%%%%%%%%%%%%%%%
\DescribeMacro{\childdocmain}
To use the package, add the commands
\begin{center}
\begin{tabular}{l}
|\input{childdoc.def}|\\
|\childdocmain{}|\\
\end{tabular}
\end{center}
at the very top of the main \LaTeX{} file,
in particular \emph{before} the |\documentclass| statement!
The argument of |\childdocmain| should be left empty
(but it must be present).

%%%%%%%%%%%%%%%%%%%%%%%%%%%%%%%%%%%%%%%%
\DescribeMacro{\childdocof}
Furthermore, add the commands
\begin{center}
\begin{tabular}{l}
|\input{childdoc.def}|\\
|\childdocof{|\textit{main}|}|\\
\end{tabular}
\end{center}
at the top of every child file \textit{child}
which is included by |\include{|\textit{child}|}|
from within the main file
(or at least for those files to be compiled individually).
The argument \textit{main} must be the filename of the main file.

There are a couple of
considerations in setting up the main and child documents:

%%%%%%%%%%%%%%%%%%%%%%%%%%%%%%%%%%%%%%%%
\paragraph{Restrictions.}

Please note the following restrictions:
\begin{itemize}
\item
|\childdocmain| must be called with one argument \textit{main}
to ensure compatibility with earlier version of the package.
It must either be empty (|\childdocmain{}|)
or precisely match the filename of the main file in which it is specified.
See \secref{sec:detection} for further information.
\item
The filename \textit{main} must be specified without the |.tex| extension.
\item
The filename \textit{main} is case sensitive
(even in case-insensitive file systems)
due to internal string comparison.
\item
The argument \textit{main} should be fully expanded, it cannot be a macro.
\item
Subdirectories and special characters should be avoided in filenames.
\item
The command |\childdocmain{|\textit{main}|}| must be followed by a whitespace.
It should not be followed immediately by another command
or by a comment mark `|%|'.
This is because the \TeX{} parser reads the token immediately following
the argument of |\childdocmain| and puts it
at the beginning of every child section;
however, a white\-space is ignored.
\end{itemize}

%%%%%%%%%%%%%%%%%%%%%%%%%%%%%%%%%%%%%%%%
\paragraph{Content of Main File.}

It is advisable to place all content in the child files included by |\include|.
Any output contained in the main file will appear in all child documents
unless suppressed manually;
it cannot be suppressed automatically by the |\includeonly| directive
and thus should normally be avoided.
A method to include some content in the main file
by means of conditional processing is described in \secref{sec:conditional}.

%%%%%%%%%%%%%%%%%%%%%%%%%%%%%%%%%%%%%%%%
\paragraph{Page Numbering.}

When only a part of the document is compiled,
the appropriate numbering of pages
(as well as other status parameters)
is determined from the |.aux| files.
The latter contain information from previous passes.
However this information needs to propagate through
all intermediate child documents.
Therefore the page numbering in child documents may well
be inconsistent until the complete document is compiled at least once.

A useful (if unconventional) way to always ensure a consistent
page numbering is to restart the numbering in each child document
and denote the pages by `\textit{child}|.|\textit{page}'
where \textit{child} represents the chapter/section number of the child file.
This can be achieved by the command
|\numberwithin{page}{|\textit{child}|}|
of the \textsf{amsmath} package
where \textit{child} can be |chapter| or |section|
depending on the chosen structuring.
Alternatively, one can modify the macro |\thepage| appropriately
and reset the counter |page| at the start of each child file.

%%%%%%%%%%%%%%%%%%%%%%%%%%%%%%%%%%%%%%%%%%%%%%%%%%%%%%%%%%%%%%%%%%%%%%%%%%%%%%%%
\subsection{Conditional Processing}
\label{sec:conditional}

The package provides a mechanism to compile different versions
of a document. To customise the versions further some conditional processing
can come in handy to distinguish which version is being compiled.
The package provides two macros to describe the compilation context:

%%%%%%%%%%%%%%%%%%%%%%%%%%%%%%%%%%%%%%%%
\DescribeMacro{\ifchilddoc}
The conditional |\ifchilddoc| distinguishes between the compilation of
child documents and the main document:
%
\begin{center}
|\ifchilddoc |\textit{child-code}| |[|\||else |\textit{main-code}]| \||fi|
\end{center}

%%%%%%%%%%%%%%%%%%%%%%%%%%%%%%%%%%%%%%%%
\DescribeMacro{\childdocname}
\DescribeMacro{\childdocjob}
The macro |\childdocname| contains the filename (without extension)
of the main or child file being processed.
Note that |\childdocjob| will always contain the name of the main file.

%%%%%%%%%%%%%%%%%%%%%%%%%%%%%%%%%%%%%%%%
\paragraph{Title Page.}

Conditional processing can be used to include a title or banner page
in the main document when proper precautions are taken.
Importantly, the code in the main file should ensure that the page counter
(as well as other status parameters which are stored in the |.aux| files)
takes the same value after the conditional processing.
Otherwise the page numbers may take divergent values
depending on which part is compiled.

For example, a title page could be declared by:
%
\begin{center}
\begin{tabular}{l}
|\ifchilddoc\||else|\\
|\addtocounter{page}{-1}|\\
\textit{code for title page}\\
|\newpage|\\
|\||fi|
\end{tabular}
\end{center}
%
A banner page for the child documents can be generated by:
%
\begin{center}
\begin{tabular}{l}
|\ifchilddoc|\\
|\addtocounter{page}{-1}|\\
\textit{code for banner page}\\
|\newpage|\\
|\||fi|
\end{tabular}
\end{center}
%
Here one could write a message such as:
\begin{center}
|This is the part \childdocname{} of \childdocjob{}.|
\end{center}

%%%%%%%%%%%%%%%%%%%%%%%%%%%%%%%%%%%%%%%%%%%%%%%%%%%%%%%%%%%%%%%%%%%%%%%%%%%%%%%%
\subsection{Flags}
\label{sec:flags}

The package makes it easy to generate different versions
of the main or child documents.
To this end compilation flags can be defined
and assigned different default values.
They will be particularly useful in conjunction
with the forwarding mechanism described in \secref{sec:forward}.

For example, it may be useful to have a flag |\version|
which can be set to |draft| or |final|.
The document source will contain some conditional code
depending on the value of |\version|.
Suppose further, the flag should default to |final| for the main file
and to |draft| for child files
which is a natural assignment for editing the document.
This is achieved by placing the following code
in the preamble of the main document
(below the |\childdocmain| directive):
%
\begin{center}
\begin{tabular}{l}
|\ifchilddoc|\\
|\providecommand{\version}{draft}|\\
|\||else|\\
|\providecommand{\version}{final}|\\
|\||fi|
\end{tabular}
\end{center}
%
The definition by |\providecommand| makes sure
that previous definitions are not overwritten.
Further statements |\providecommand{\version}{...}|
can thus be added before the above code to override it.

For the main file, one might add a line
(between |\childdocmain| and the above block)
%
\begin{center}
|%\ifchilddoc\||else\providecommand{\version}{draft}\||fi|
\end{center}
%
which can be uncommented to produce a draft version.
Likewise one can add a line to the very top of a child file
(above the |\childdocof{|\textit{main}|}| directive)
%
\begin{center}
|%\providecommand{\version}{final}|
\end{center}
%
which can be uncommented to produce the final version of this child document.

%%%%%%%%%%%%%%%%%%%%%%%%%%%%%%%%%%%%%%%%%%%%%%%%%%%%%%%%%%%%%%%%%%%%%%%%%%%%%%%%
\subsection{Forwarding}
\label{sec:forward}

Different versions of the main or child documents
using compilation flags as described in \secref{sec:flags}
can be (permanently) stored in different files
for convenient compilation, viewing and distribution.
To this end, the package defines a command
to pass on compilation to a different file:

%%%%%%%%%%%%%%%%%%%%%%%%%%%%%%%%%%%%%%%%
\DescribeMacro{\childdocforward}
The command |\childdocforward| redirects processing to
another source file:
%
\begin{center}
\begin{tabular}{l}
|\input{childdoc.def}|\\
|\childdocforward[|\textit{main}|]{|\textit{dest}|}|\\
\end{tabular}
\end{center}
%
The argument \textit{dest} is the destination file
(without extension).
It should be the main file or one of the child files.
Note that further \textsf{childdoc} directives
such as |\childdocof| and |\childdocforward|
in the indicated file will be processed in this form.
The optional argument \textit{main}
passes on directly to the main file \textit{main}
while pretending to compile the child \textit{dest}.
This form behaves as if \textit{dest}
issues |\childdocof{|\textit{main}|}| right away,
and no further \textsf{childdoc} directives will be processed.

%%%%%%%%%%%%%%%%%%%%%%%%%%%%%%%%%%%%%%%%
\DescribeMacro{\...prefix}
In the alternative form |\childdocforwardprefix|,
%
\begin{center}
\begin{tabular}{l}
|\input{childdoc.def}|\\
|\childdocforwardprefix[|\textit{main}|]{|\textit{prefix}|}{|\textit{dest}|}|
\end{tabular}
\end{center}
%
the destination file is determined by a pattern
depending on the current file:
To make this work, the current file must be called
`{\textit{prefix}\hspace{0.2em}\textit{suffix}}'
with \textit{prefix} matching precisely the argument.
Processing is then passed on to the file
`{\textit{dest}\hspace{0.2em}\textit{suffix}}'.
Surely, the same effect is achieved by
directly specifying the
argument `{\textit{dest}\hspace{0.2em}\textit{suffix}}'
in the first form.
However, that requires to set up a different file
for each child. With the alternative form of the command
all these files can have exactly the same content
which simplifies setting them up and maintaining them.

For example, the following file |draft.tex|
with a compilation flag |\version| as described in \secref{sec:flags}
compiles the main document as a draft:
%
\begin{center}
\begin{tabular}{l}
|\def\version{draft}|\\
|\input{childdoc.def}|\\
|\childdocforward{|\textit{main}|}|
\end{tabular}
\end{center}
%
Likewise, the following files |final|\textit{nn}|.tex|
compile the final version of the child document
|child|\textit{nn}|.tex|:
%
\begin{center}
\begin{tabular}{l}
|\def\version{final}|\\
|\input{childdoc.def}|\\
|\childdocforwardprefix{final}{child}|
\end{tabular}
\end{center}
%

Note that when several versions of a main file and/or of each child file
are to be generated, it may be convenient to set up a |Makefile| or
shell script to automatise the process.

%%%%%%%%%%%%%%%%%%%%%%%%%%%%%%%%%%%%%%%%%%%%%%%%%%%%%%%%%%%%%%%%%%%%%%%%%%%%%%%%
\subsection{Command Line Processing}
\label{sec:commandline}

The effect of redirection files can also be achieved by invoking
the \LaTeX{} compiler with a more elaborate command line.
Most conveniently this should be done as part
of a shell script or a |Makefile|.

When using \textsf{childdoc} in the main file, the following
command lines effectively perform a redirection
(note that depending on the shell being used,
backslashes may have to be doubled: `|\|' $\to$ `|\\|'):
%
\begin{center}
|... -jobname "|\textit{target}|" |\\|"|[\textit{flags}]%
|\input{childdoc.def}\childdocforward[|\textit{main}|]{|\textit{dest}|}"|
\end{center}
%
Here \textit{target} is the name of the output file,
\textit{main} is the name of the main file
and \textit{dest} is the name of the main or child file to be processed
(all filenames without extensions).
The optional argument \textit{main} can be omitted
if \textit{main} matches \textit{dest}.
Optionally, compilation \textit{flags} can be defined via |\def| commands.
This command line makes the \TeX{} engine believe
it is compiling the file \textit{target}
whose content is specified as the latter parameter.
The provided code then forwards the processing to
\textit{main} or \textit{dest} as described in \secref{sec:forward}.

%%%%%%%%%%%%%%%%%%%%%%%%%%%%%%%%%%%%%%%%%%%%%%%%%%%%%%%%%%%%%%%%%%%%%%%%%%%%%%%%
\subsection{Include by Input}
\label{sec:input}

Including child documents by |\include| has some restrictions by design.
Most notably, the content of a child document always occupies
its own set of pages; pages cannot be shared between child documents.
Usually, this behaviour makes perfect sense
because each child document contain an essential part of the document.
However, in some situations it may be desirable to compose
a document from a collection of parts
without having mandatory page breaks between then.
For this case, the package
provides a mechanism to include parts
by |\input| which can also be processed individually.
However, by construction this mechanism
requires manual handling of the content to be output.

%%%%%%%%%%%%%%%%%%%%%%%%%%%%%%%%%%%%%%%%
\DescribeMacro{\ifchilddocmanual}
The main file should be prepared as usual, see \secref{sec:include}.
However, the document body must make a distinction
between processing of an individual part and of the main document, e.g.:
%
\begin{center}
\begin{tabular}{l}
|\ifchilddocmanual|\\
|\input{\childdocname}|\\
|\||else|\\
\textit{document body with }|\input{|\textit{part}|}|\\
|\||fi|
\end{tabular}
\end{center}
%
The conditional |\ifchilddocmanual| is true whenever
a part to be included by |\input| is being compiled,
and the name of the part is stored in |\childdocname|.

%%%%%%%%%%%%%%%%%%%%%%%%%%%%%%%%%%%%%%%%
\DescribeMacro{\childdocby}
Each part to be included by |\input| should start with:
%
\begin{center}
\begin{tabular}{l}
|\input{childdoc.def}|\\
|\childdocby{|\textit{main}|}|\\
\end{tabular}
\end{center}
%
The directive |\childdocby| is similar to |\childdocof|
described in \secref{sec:include},
but the subsequent selection of content must be done manually.
To that end, both |\ifchilddoc| and |\ifchilddocmanual|
will be true upon processing of a part,
and the name of the part is stored in |\childdocname|.
Note that |\jobname| will be set to the filename of the current part
so that each part receives an individual |.aux| file
that does not interfere with the |.aux| file(s) of the main document.
This behaviour can be altered by the alternative form
|\childdocby[*]{|\textit{main}|}| (with a non-empty optional argument)
which uses the |.aux| file of the main document
by setting |\jobname| to \textit{main}.

%%%%%%%%%%%%%%%%%%%%%%%%%%%%%%%%%%%%%%%%%%%%%%%%%%%%%%%%%%%%%%%%%%%%%%%%%%%%%%%%
\subsection{Driver Development}
\label{sec:driver}

The \textsf{childdoc} mechanism can also be use for the development
of definition files such as \LaTeX{} styles or classes.
This case differs from the above setup with multiple parts
included by |\include| in that no |\includeonly| should be invoked.
This can be achieved by starting the include file
(before |\ProvidesPackage|) with:
%
\begin{center}
\begin{tabular}{l}
|\input{childdoc.def}|\\
|\childdocforward{|\textit{main}|}|\\
\end{tabular}
\end{center}
%
or alternatively with:
%
\begin{center}
\begin{tabular}{l}
|\input{childdoc.def}|\\
|\childdocby{|\textit{main}|}|\\
\end{tabular}
\end{center}
%
Both forms have slightly different effects as described above.
The main file is prepared as usual, see \secref{sec:include}.

%%%%%%%%%%%%%%%%%%%%%%%%%%%%%%%%%%%%%%%%%%%%%%%%%%%%%%%%%%%%%%%%%%%%%%%%%%%%%%%%
\subsection{Legacy Detection}
\label{sec:detection}

The directive |\childdocmain| in the main file can detect
whether the complete document or merely a child is to be compiled
even without using the directive |\childdocof|.
This method is deprecated because it is less robust
and there is no compelling reason to use it;
it is merely provided for backward compatibility
and it may be removed in future versions.

If the detection mechanism is to be used,
it is mandatory to correctly specify
the filename of the main file as the argument of |\childdocmain|:
%
\begin{center}
\begin{tabular}{l}
|\input{childdoc.def}|\\
|\childdocmain{|\textit{main}|}|\\
\end{tabular}
\end{center}
%
If |\jobname| does not match the argument \textit{main} of |\childdocmain|,
it is assumed that |\jobname| points to the child file to be compiled.
When using |\childdocmain| with the main file specified as argument,
it suffices to start a child file
with just |\input{|\textit{main}|}|
without loading of the package and using |\childdocof|.
If instead all processing is done
with the appropriate \textsf{childdoc} directives,
the argument of \textit{main} of |\childdocmain| can be empty.

An alternative version of the command line processing described
in \secref{sec:commandline} using the detection mechanism reads:
%
\begin{center}
|... -jobname "|\textit{target}|" "|[\textit{flags}]%
[|\def\jobname{|\textit{dest}|}|]|\input{|\textit{main}|}"|
\end{center}

%%%%%%%%%%%%%%%%%%%%%%%%%%%%%%%%%%%%%%%%%%%%%%%%%%%%%%%%%%%%%%%%%%%%%%%%%%%%%%%%
\subsection{Manual Code}
\label{sec:manual}

In case one cannot be certain whether the definitions file |childdoc.def|
is installed on the target \TeX{} distribution
and one prefers not to ship it,
it is conceivable to paste a few relevant commands into the sources.

To that end, drop all statements |\input{childdoc.def}|
and perform the replacements as outlined below.
Instead of |\childdocmain{|\textit{main}|}| add the following code
to the top of the main file:
%
\begin{center}
\begin{tabular}{l}
|\||ifdefined\childdocname\endinput\||fi\newif\ifchilddoc|\\
|\edef\childdocname{\scantokens\expandafter{\jobname\noexpand}}|\\
|\def\childdocmain{|\textit{main}|}\||ifx\childdocmain\childdocname\||else|\\
|\childdoctrue\includeonly{\childdocname}\let\jobname\childdocmain\||fi|\\
\end{tabular}
\end{center}
%
Instead of |\childdocof{|\textit{main}|}| just include the main file
at the top of each child file:
%
\begin{center}
|\input{|\textit{main}|}|
\end{center}
%
A simple redirection |\childdocforward{|\textit{dest}|}| is achieved by:
%
\begin{center}
|\def\jobname{|\textit{dest}|}\input{\jobname}|
\end{center}
%
The redirection with prefix
|\childdocforwardprefix[|\textit{prefix}|]{|\textit{dest}|}|
is accomplished by:
%
\begin{center}
\begin{tabular}{l}
|{\edef\jobname{\scantokens\expandafter{\jobname\noexpand}}|\\
|\def\redirectjob |\textit{prefix}|#1~~~{\gdef\jobname{|\textit{dest}|#1}}|\\
|\expandafter\redirectjob\jobname~~~}\input{\jobname}|
\end{tabular}
\end{center}

In an alternative approach,
child documents can be compiled by a specific command line
without additional code or specific definitions:
%
\begin{center}
|... -jobname "|\textit{target}|" "|[\textit{flags}]%
|\includeonly{|\textit{dest}|}\input{|\textit{main}|}"|
\end{center}
%

%%%%%%%%%%%%%%%%%%%%%%%%%%%%%%%%%%%%%%%%%%%%%%%%%%%%%%%%%%%%%%%%%%%%%%%%%%%%%%%%
%%%%%%%%%%%%%%%%%%%%%%%%%%%%%%%%%%%%%%%%%%%%%%%%%%%%%%%%%%%%%%%%%%%%%%%%%%%%%%%%
\section{Information}

%%%%%%%%%%%%%%%%%%%%%%%%%%%%%%%%%%%%%%%%%%%%%%%%%%%%%%%%%%%%%%%%%%%%%%%%%%%%%%%%
\subsection{Copyright}

Copyright \copyright{} 2017--2018 Niklas Beisert

This work may be distributed and/or modified under the
conditions of the \LaTeX{} Project Public License, either version 1.3
of this license or (at your option) any later version.
The latest version of this license is in
  \url{http://www.latex-project.org/lppl.txt}
and version 1.3 or later is part of all distributions of \LaTeX{}
version 2005/12/01 or later.

This work has the LPPL maintenance status `maintained'.

The Current Maintainer of this work is Niklas Beisert.

This work consists of the files |README.txt|, |childdoc.ins| and |childdoc.dtx|
as well as the derived files |childdoc.def|, |cdocsamp.tex|
with |cdocsch1.tex|, |cdocsch2.tex|, |cdocspt3.tex|, |cdocspt4.tex|,
|cdocsdrf.tex|, |cdocsfn1.tex|, |cdocsfn2.tex|
as well as |childdoc.pdf|.

%%%%%%%%%%%%%%%%%%%%%%%%%%%%%%%%%%%%%%%%%%%%%%%%%%%%%%%%%%%%%%%%%%%%%%%%%%%%%%%%
\subsection{Files and Installation}

The package consists of the files:
%
\begin{center}
\begin{tabular}{ll}
    |README.txt|   & readme file \\
    |childdoc.ins| & installation file \\
    |childdoc.dtx| & source file \\
    |childdoc.def| & definition file \\
    |cdocsamp.tex| & sample main file \\
    |cdocsch1.tex| & sample include file \\
    |cdocsch2.tex| & sample include file \\
    |cdocspt3.tex| & sample part file \\
    |cdocspt4.tex| & sample part file \\
    |cdocsdrf.tex| & sample redirection file \\
    |cdocsfn1.tex| & sample redirection file \\
    |cdocsfn2.tex| & sample redirection file \\
    |childdoc.pdf| & manual
\end{tabular}
\end{center}
%
The distribution consists of the files
|README.txt|, |childdoc.ins| and |childdoc.dtx|.
%
\begin{itemize}
\item
Run (pdf)\LaTeX{} on |childdoc.dtx|
to compile the manual |childdoc.pdf| (this file).
\item
Run \LaTeX{} on |childdoc.ins| to create the definitions file |childdoc.def|
and the sample |cdocsamp.tex| with include files
|cdocsch1.tex|, |cdocsch2.tex|, |cdocspt3.tex|, |cdocspt4.tex|,
|cdocsdrf.tex|, |cdocsfn1.tex|, |cdocsfn2.tex|.
Then copy the file |childdoc.def| to an appropriate directory of your \LaTeX{}
distribution, e.g.\ \textit{texmf-root}|/tex/latex/childdoc|.
\end{itemize}

%%%%%%%%%%%%%%%%%%%%%%%%%%%%%%%%%%%%%%%%%%%%%%%%%%%%%%%%%%%%%%%%%%%%%%%%%%%%%%%%
\subsection{Related CTAN Packages}

There are several other packages which offer a similar functionality:
%
\begin{itemize}
\item
The packages
\href{http://ctan.org/pkg/docmute}{\textsf{docmute}},
\href{http://ctan.org/pkg/includex}{\textsf{includex}} and
\href{http://ctan.org/pkg/standalone}{\textsf{standalone}}
provide commands to include only the document body of
a child file thus allowing both files to be compiled individually.
\item
The packages \href{http://ctan.org/pkg/subdocs}{\textsf{subdocs}}
and \href{http://ctan.org/pkg/subfiles}{\textsf{subfiles}}
provide structures in which the main and child documents can be
encapsulated and allowing them to be compiled individually.
The inclusion mechanism is different from the conventional |\include|.
\item
The package \href{http://ctan.org/pkg/combine}{\textsf{combine}}
is an elaborate solution to combine several documents into one.
\end{itemize}
%
See also the CTAN topic \href{http://ctan.org/topic/subdocs}{\textsf{subdocs}}
for further related packages.
The present package differs from the above solutions in that
a document structure constructed with the conventional |\include| mechanism
just needs two extra commands at the top of every file
such that all constituent files can be compiled individually.

%%%%%%%%%%%%%%%%%%%%%%%%%%%%%%%%%%%%%%%%%%%%%%%%%%%%%%%%%%%%%%%%%%%%%%%%%%%%%%%%
%\subsection{Feature Suggestions}
%
%The following is a list of features which may be useful for future
%versions of this package:
%%
%\begin{itemize}
%\item
%\ldots
%\end{itemize}

%%%%%%%%%%%%%%%%%%%%%%%%%%%%%%%%%%%%%%%%%%%%%%%%%%%%%%%%%%%%%%%%%%%%%%%%%%%%%%%%
\subsection{Revision History}

%%%%%%%%%%%%%%%%%%%%%%%%%%%%%%%%%%%%%%%%
\paragraph{v2.0:} 2018/12/30

\begin{itemize}
\item
immediate forward processing
\item
added |\childdocby| mechanism
\item
manual restructured
\end{itemize}

%%%%%%%%%%%%%%%%%%%%%%%%%%%%%%%%%%%%%%%%
\paragraph{v1.6:} 2018/01/17

\begin{itemize}
\item
application for development of include files
\item
corrections to manual
\end{itemize}

%%%%%%%%%%%%%%%%%%%%%%%%%%%%%%%%%%%%%%%%
\paragraph{v1.5:} 2017/05/21

\begin{itemize}
\item
more complete structuring introduced
\item
|\childdocof| introduced
\item
|\childdoc| renamed to |\childdocmain|
\item
|\childredirect| renamed to |\childdocforward| and |\childdocforwardprefix|
and functionality expanded
\end{itemize}

%%%%%%%%%%%%%%%%%%%%%%%%%%%%%%%%%%%%%%%%
\paragraph{v1.0:} 2017/04/27

\begin{itemize}
\item
manual and install package
\item
first version published on CTAN
\end{itemize}

%%%%%%%%%%%%%%%%%%%%%%%%%%%%%%%%%%%%%%%%
\paragraph{v0.6:} 2017/04/26

\begin{itemize}
\item
redirection mechanism added
\end{itemize}

%%%%%%%%%%%%%%%%%%%%%%%%%%%%%%%%%%%%%%%%
\paragraph{v0.5:} 2017/04/26

\begin{itemize}
\item
functionality in definition file
\end{itemize}


%%%%%%%%%%%%%%%%%%%%%%%%%%%%%%%%%%%%%%%%%%%%%%%%%%%%%%%%%%%%%%%%%%%%%%%%%%%%%%%%
%%%%%%%%%%%%%%%%%%%%%%%%%%%%%%%%%%%%%%%%%%%%%%%%%%%%%%%%%%%%%%%%%%%%%%%%%%%%%%%%
%%%%%%%%%%%%%%%%%%%%%%%%%%%%%%%%%%%%%%%%%%%%%%%%%%%%%%%%%%%%%%%%%%%%%%%%%%%%%%%%
\appendix

\settowidth\MacroIndent{\rmfamily\scriptsize 000\ }

 \DocInput{childdoc.dtx}

\end{document}
%</driver>
% \fi
%
% %%%%%%%%%%%%%%%%%%%%%%%%%%%%%%%%%%%%%%%%%%%%%%%%%%%%%%%%%%%%%%%%%%%%%%%%%%%%%%
% %%%%%%%%%%%%%%%%%%%%%%%%%%%%%%%%%%%%%%%%%%%%%%%%%%%%%%%%%%%%%%%%%%%%%%%%%%%%%%
% \section{Sample}
%\iffalse
%<*samplemain>
%\fi
%
% The following presents a sample document
% with two chapters, two parts, a title page,
% a compile flag as well as three forwarding files to set the flag.
% It consists of eight |.tex| files:
% \begin{center}
% \begin{tabular}{ll}
% |cdocsamp.tex|&main file\\
% |cdocsch1.tex|&include file for chapter 1\\
% |cdocsch2.tex|&include file for chapter 2\\
% |cdocspt3.tex|&include file for part 3\\
% |cdocspt4.tex|&include file for part 4\\
% |cdocsdrf.tex|&forwarding file for main file in draft mode\\
% |cdocsfi1.tex|&forwarding file for final version of chapter 1\\
% |cdocsfi2.tex|&forwarding file for final version of chapter 2\\
% \end{tabular}
% \end{center}
% Each of the eight files can be compiled directly by the \LaTeX{} compiler.
%
% %%%%%%%%%%%%%%%%%%%%%%%%%%%%%%%%%%%%%%
% \paragraph{Main File.}
%
% The main file is called |cdocsamp.tex|.
%
% Load the \textsf{childdoc} definitions and
% declare the filename for the main document:
%    \begin{macrocode}
\input{childdoc.def}
\childdocmain{}
%    \end{macrocode}

% Optional override for |\version| flag:
%    \begin{macrocode}
%%\ifchilddoc\else\providecommand{\version}{draft}\fi
%    \end{macrocode}

% Define the default values for the |\version| flag
% (|final| for the main file and |draft| for childs):
%    \begin{macrocode}
\ifchilddoc
\providecommand{\version}{draft}
\else
\providecommand{\version}{final}
\fi
%    \end{macrocode}

% Load the standard document class:
%    \begin{macrocode}
\documentclass[12pt]{article}
%    \end{macrocode}

% Start the document body:
%    \begin{macrocode}
\begin{document}
%    \end{macrocode}

% Declare a title page.
% Print title, part of document being processed and version flag:
%    \begin{macrocode}
\addtocounter{page}{-1}
\begin{center}
{\LARGE\bfseries{}childdoc example\par}
\vspace{1cm}
\ifchilddoc
\ifchilddocmanual part\else chapter\fi:
`\childdocname' of `\childdocjob'\par
\else
main document: `\childdocjob'\par
\fi
version: \version\par
\end{center}
\newpage
%    \end{macrocode}

% Manually include selected file,
% otherwise process as usual:
%    \begin{macrocode}
\ifchilddocmanual
\section*{part `\childdocname'}
\input{\childdocname}
\else
%    \end{macrocode}

% Include the two chapters:
%    \begin{macrocode}
\include{cdocsch1}
\include{cdocsch2}
%    \end{macrocode}

% Include the two parts unless only chapters should be displayed:
%    \begin{macrocode}
\ifchilddoc\else
\section{part three}
\input{cdocspt3}
\section{part four}
\input{cdocspt4}
\fi
%    \end{macrocode}

% Process as usual until here:
%    \begin{macrocode}
\fi
%    \end{macrocode}

% End of document body:
%    \begin{macrocode}
\end{document}
%    \end{macrocode}
%\iffalse
%</samplemain>
%\fi
%
% %%%%%%%%%%%%%%%%%%%%%%%%%%%%%%%%%%%%%%
% \paragraph{Chapter Include Files.}
%
% The include files are called |cdocsch1.tex| and |cdocsch2.tex|.
%
%\iffalse
%<*samplechap1|samplechap2>
%\fi

% Optional override for |\version| flag:
%    \begin{macrocode}
%%\providecommand{\version}{final}
%    \end{macrocode}

% Include the main document:
%    \begin{macrocode}
\input{childdoc.def}
\childdocof{cdocsamp}
%    \end{macrocode}

%\iffalse
%</samplechap1|samplechap2>
%\fi
%
%\iffalse
%<*samplechap1>
%\fi
% Some text for chapter 1:
%    \begin{macrocode}
\section{one}
some text in chapter one
%    \end{macrocode}

%\iffalse
%</samplechap1>
%\fi
% Some text for chapter 2:
%\iffalse
%<*samplechap2>
%\fi
%    \begin{macrocode}
\section{two}
more text in chapter two
%    \end{macrocode}

%\iffalse
%</samplechap2>
%\fi
%
% %%%%%%%%%%%%%%%%%%%%%%%%%%%%%%%%%%%%%%
% \paragraph{Part Include Files.}
%
% The include files are called |cdocspt3.tex| and |cdocspt4.tex|.
%
%\iffalse
%<*samplepart3|samplepart4>
%\fi

% Optional override for |\version| flag:
%    \begin{macrocode}
%%\providecommand{\version}{final}
%    \end{macrocode}

% Include the main document:
%    \begin{macrocode}
\input{childdoc.def}
\childdocby{cdocsamp}
%    \end{macrocode}

%\iffalse
%</samplepart3|samplepart4>
%\fi
%
%\iffalse
%<*samplepart3>
%\fi
% Some text for part 3:
%    \begin{macrocode}
some text in part three
%    \end{macrocode}

%\iffalse
%</samplepart3>
%\fi
% Some text for part 4:
%\iffalse
%<*samplepart4>
%\fi
%    \begin{macrocode}
more text in part four
%    \end{macrocode}

%\iffalse
%</samplepart4>
%\fi
%
% %%%%%%%%%%%%%%%%%%%%%%%%%%%%%%%%%%%%%%
% \paragraph{Forwarding for a Complete Draft.}
%
% The following forwarding file |cdocsdrf.tex|
% compiles the main document in draft mode:
%\iffalse
%<*sampledraft>
%\fi
%    \begin{macrocode}
\def\version{draft}
\input{childdoc.def}
\childdocforward{cdocsamp}
%    \end{macrocode}

%\iffalse
%</sampledraft>
%\fi
%
% %%%%%%%%%%%%%%%%%%%%%%%%%%%%%%%%%%%%%%
% \paragraph{Forwarding for Final Version of the Chapters.}
%
% The following forwarding files |cdocsfn1.tex| and |cdocsfn2.tex|
% (with identical content)
% compile the final versions of the child documents
% |cdocsch1.tex| and |cdocsch2.tex|, respectively:
%\iffalse
%<*samplefinal>
%\fi
%    \begin{macrocode}
\def\version{final}
\input{childdoc.def}
\childdocforwardprefix[cdocsamp]{cdocsfn}{cdocsch}
%    \end{macrocode}

%\iffalse
%</samplefinal>
%\fi
%
% %%%%%%%%%%%%%%%%%%%%%%%%%%%%%%%%%%%%%%
% \paragraph{Command Line Processing.}
%
% The following three command lines generate the output files
% |cdocscld|, |cdocscl1| and |cdocscl2|
% which should be identical to
% |cdocsdrf|, |cdocsch1| and |cdocsfn2|, respectively:
% \begin{center}
% \begin{tabular}{l}
% |latex -jobname cdocscld \|\\
% |  "\def\version{draft}\input{childdoc.def}\childdocforward{cdocsamp}"|\\
% |latex -jobname cdocscl1 \|\\
% |  "\input{childdoc.def}\childdocforward[cdocsamp]{cdocsch1}"|\\
% |latex -jobname cdocscl2 \|\\
% |  "\def\version{final}\input{childdoc.def}\childdocforward{cdocsch2}"|
% \end{tabular}
% \end{center}
% Note that the trailing backslash on each first line
% merely continues the input to the second line
% (for convenient cut ant paste).
% Furthermore, the command |latex| can be replaced by any
% of its alternative versions such as |pdflatex|.
%
% %%%%%%%%%%%%%%%%%%%%%%%%%%%%%%%%%%%%%%%%%%%%%%%%%%%%%%%%%%%%%%%%%%%%%%%%%%%%%%
% %%%%%%%%%%%%%%%%%%%%%%%%%%%%%%%%%%%%%%%%%%%%%%%%%%%%%%%%%%%%%%%%%%%%%%%%%%%%%%
% \section{Implementation}
%\iffalse
%<*package>
%\fi
%
% This section describes the definitions file |childdoc.def|.

% The definitions cannot be loaded using |\usepackage| or |\RequirePackage|
% which has a mechanism to prevent loading a style file more than once.
% When loading the definitions by means of |\input|
% multiple instances have to be prevented manually:
%\iffalse
%This code needs to be before the `\ProvidesFile' directive
%which is defined at the beginning of this file.
%Therefore it is also placed there and commented out here.
%</package>
%<*discard>
%\fi
%    \begin{macrocode}
\ifdefined\childdocmain\endinput\fi
%    \end{macrocode}
%\iffalse
%</discard>
%<*package>
%\fi
%
% \macro{\ifchilddoc}
% \macro{\ifchilddocmanual}
% The conditional |\ifchilddoc| tells whether a
% child (true) or main (false) document is being compiled.
% The conditional |\ifchilddocmanual| tells whether
% the |\includeonly| mechanism is used (false) or
% the selection of child files must be performed manually (true).
% The definitions initialise to false:
%    \begin{macrocode}
\newif\ifchilddoc
\newif\ifchilddocmanual
%    \end{macrocode}

% \macro{\childdocname}
% \macro{\childdocjob}
% The macro |\childdocname| stores the name of the main document
% to be compiled. The macro |\childdocjob| stores the name of
% the document on which the \LaTeX{} compiler was originally invoked.
% The content of |\jobname| cannot be compared
% to filenames specified in the source due to different catcodes.
% The following code rescans |\jobname|, stores the result
% in |\childdocname| and saves a copy in |\childdocjob|:
%    \begin{macrocode}
\edef\childdocname{\scantokens\expandafter{\jobname\noexpand}}
\let\childdocjob\childdocname
%    \end{macrocode}

% \macro{\childdocdisable}
% The macro |\childdocdisable| prevents the main file
% from being processed more than once.
% At this stage, the main document command |\childdocmain|
% is assumed to be called once again where it should do nothing.
% Any subsequent call to it should prevent
% a secondary processing of the main document
% It overwrites the forwarding commands
% |\childdocof| and |\childdocforward|
% with empty macros to prevent further inclusions of the main document:
%    \begin{macrocode}
\newcommand{\childdocdisable}
{
  \renewcommand{\childdocmain}[1]{\renewcommand{\childdocmain}[1]{\endinput}}
  \renewcommand{\childdocof}[1]{}
  \renewcommand{\childdocby}[2][]{}
  \renewcommand{\childdocforward}[2][]{}
  \renewcommand{\childdocdisable}{}
}
%    \end{macrocode}

% \macro{\childdocmain}
% The macro |\childdocmain| is to be called at the top of the main file
% with nothing or the main filename (without extension) as argument.
% First, it breaks loops.
% If the argument is not empty and does not match |\childdocname|
% (which is set by the first inclusion of |childdoc.def|),
% |\ifchilddoc| is set to true, |\includeonly| is applied to the child file
% and |\jobname| is set to the main file
% (for proper handling of |.aux| files):
%    \begin{macrocode}
\newcommand{\childdocmain}[1]
{
  \childdocdisable\childdocmain{}
  \if?#1?\else
    \begingroup
      \def\childdoctmp{#1}
      \ifx\childdoctmp\childdocname
        \def\childdoctmp{}
      \else
        \def\childdoctmp
        {
          \childdoctrue
          \includeonly{\childdocname}
          \def\childdocjob{#1}
          \def\jobname{#1}
        }
      \fi
      \expandafter
    \endgroup
    \childdoctmp
  \fi
}
%    \end{macrocode}

% \macro{\childdocof}
% The command |\childdocof| redirects
% compilation to the main file |#1|.
%    \begin{macrocode}
\newcommand{\childdocof}[1]
{
  \childdocdisable
  \childdoctrue
  \includeonly{\childdocname}
  \def\jobname{#1}
  \def\childdocjob{#1}
  \input{#1}
}
%    \end{macrocode}

% \macro{\childdocby}
% The command |\childdocby| ....
%    \begin{macrocode}
\newcommand{\childdocby}[2][]
{
  \childdocdisable
  \childdoctrue
  \childdocmanualtrue
  \if?#1?\else
    \def\jobname{#2}
  \fi
  \def\childdocjob{#2}
  \input{#2}
  \endinput
}
%    \end{macrocode}

% \macro{\childdocforward}
% The command |\childdocforward| redirects
% compilation to the main file or
% (if the optional argument is given) a child file.
% Parameters are set as if the main file
% or a child file starting with |\childdocof| was compiled.
% Then compilation is handed over to the main file:
%    \begin{macrocode}
\newcommand{\childdocforward}[2][]
{
  \begingroup
    \if?#1?
      \def\childdoctmp
      {
        \def\childdocname{#2}
        \def\childdocjob{#2}
        \def\jobname{#2}
        \input{#2}
        \endinput
      }
    \else
      \def\childdoctmp
      {
        \childdocdisable
        \def\childdocname{#2}
        \childdoctrue
        \includeonly{#2}
        \def\childdocjob{#1}
        \def\jobname{#1}
        \input{#1}
        \endinput
      }
    \fi
    \expandafter
  \endgroup
  \childdoctmp
}
%    \end{macrocode}

% \macro{\childdocforwardprefix}
% The command |\childdocforwardprefix| redirects
% compilation to the main or a child file by means of a pattern.
% The prefix |#1| in the current filename is replaced by |#2|
% and the suffix of the current filename is kept
% (it is assumed that the filename does not contain the substring `|~~~|'
% which is used as a delimiter).
% Compilation is handed over to the new file by |\childdocforward|:
%    \begin{macrocode}
\newcommand{\childdocforwardprefix}[3][]
{
  \begingroup
    \def\childdocextract #2##1~~~{\def\childdoctmp{\childdocforward[#1]{#3##1}}}
    \expandafter\childdocextract\childdocname~~~
    \expandafter
  \endgroup
  \childdoctmp
}
%    \end{macrocode}

% \macro{\childdoc}
% The deprecated macro |\childdoc| is a legacy version of |\childdocmain|:
%    \begin{macrocode}
\newcommand{\childdoc}{\childdocmain}
%    \end{macrocode}

% \macro{\childdocredirect}
% The deprecated macro |\childdocredirect| is a legacy version
% of |\childdocforward| and |\childdocforwardprefix|:
%    \begin{macrocode}
\newcommand{\childdocredirect}[2][]
{
  \begingroup
    \if?#1?
      \def\childdoctmp{\childdocforward{#2}}
    \else
      \def\childdoctmp{\childdocforwardprefix{#1}{#2}}
    \fi
    \expandafter
  \endgroup
  \childdoctmp
}
%    \end{macrocode}

%\iffalse
%</package>
%\fi
%
\endinput

\childdocby{cdocsamp}
%    \end{macrocode}

%\iffalse
%</samplepart3|samplepart4>
%\fi
%
%\iffalse
%<*samplepart3>
%\fi
% Some text for part 3:
%    \begin{macrocode}
some text in part three
%    \end{macrocode}

%\iffalse
%</samplepart3>
%\fi
% Some text for part 4:
%\iffalse
%<*samplepart4>
%\fi
%    \begin{macrocode}
more text in part four
%    \end{macrocode}

%\iffalse
%</samplepart4>
%\fi
%
% %%%%%%%%%%%%%%%%%%%%%%%%%%%%%%%%%%%%%%
% \paragraph{Forwarding for a Complete Draft.}
%
% The following forwarding file |cdocsdrf.tex|
% compiles the main document in draft mode:
%\iffalse
%<*sampledraft>
%\fi
%    \begin{macrocode}
\def\version{draft}
% \iffalse
%
% childdoc.dtx Copyright (C) 2017-2018 Niklas Beisert
%
% This work may be distributed and/or modified under the
% conditions of the LaTeX Project Public License, either version 1.3
% of this license or (at your option) any later version.
% The latest version of this license is in
%   http://www.latex-project.org/lppl.txt
% and version 1.3 or later is part of all distributions of LaTeX
% version 2005/12/01 or later.
%
% This work has the LPPL maintenance status `maintained'.
%
% The Current Maintainer of this work is Niklas Beisert.
%
% This work consists of the files childdoc.dtx and childdoc.ins
% and the derived files childdoc.def and cdocsamp.tex with
% cdocsch1.tex, cdocsch2.tex, cdocsdrf.tex, cdocsfn1.tex, cdocsfn2.tex.
%
%<package>\ifdefined\childdocmain\endinput\fi
%<package>\ProvidesFile{childdoc.def}[2018/12/30 v2.0 child document driver]
%<samplemain>\ProvidesFile{cdocsamp.tex}[2018/12/30 v2.0 sample for childdoc]
%<*driver>
%\ProvidesFile{childdoc.drv}[2018/12/30 v2.0 childdoc reference manual file]
\PassOptionsToClass{10pt,a4paper}{article}
\documentclass{ltxdoc}

\usepackage[margin=35mm]{geometry}
\usepackage{hyperref}
\usepackage{hyperxmp}
\usepackage[usenames]{color}

\hypersetup{colorlinks=true}
\hypersetup{pdfstartview=FitH}
\hypersetup{pdfpagemode=UseNone}
\hypersetup{pdfsource={}}
\hypersetup{pdflang={en-UK}}
\hypersetup{pdfcopyright={Copyright 2017-2018 Niklas Beisert.
  This work may be distributed and/or modified under the
  conditions of the LaTeX Project Public License, either version 1.3
  of this license or (at your option) any later version.}}
\hypersetup{pdflicenseurl={http://www.latex-project.org/lppl.txt}}
\hypersetup{pdfcontactaddress={ETH Zurich, ITP, HIT K,
  Wolfgang-Pauli-Strasse 27}}
\hypersetup{pdfcontactpostcode={8093}}
\hypersetup{pdfcontactcity={Zurich}}
\hypersetup{pdfcontactcountry={Switzerland}}
\hypersetup{pdfcontactemail={nbeisert@itp.phys.ethz.ch}}
\hypersetup{pdfcontacturl={http://people.phys.ethz.ch/\xmptilde nbeisert/}}

\newcommand{\secref}[1]{\hyperref[#1]{section \ref*{#1}}}

\parskip1ex
\parindent0pt
\let\olditemize\itemize
\def\itemize{\olditemize\parskip0pt}

\begin{document}

\title{The \textsf{childdoc} Package}
\hypersetup{pdftitle={The childdoc Package}}
\author{Niklas Beisert\\[2ex]
  Institut f\"ur Theoretische Physik\\
  Eidgen\"ossische Technische Hochschule Z\"urich\\
  Wolfgang-Pauli-Strasse 27, 8093 Z\"urich, Switzerland\\[1ex]
  \href{mailto:nbeisert@itp.phys.ethz.ch}
  {\texttt{nbeisert@itp.phys.ethz.ch}}}
\hypersetup{pdfauthor={Niklas Beisert}}
\hypersetup{pdfsubject={Manual for the LaTeX2e Package childdoc}}
\date{30 December 2018, \textsf{v2.0}}
\maketitle

\begin{abstract}\noindent
\textsf{childdoc} is a \LaTeXe{} package
that enables the direct compilation
of document sections included by |\include|
to individual files.
\end{abstract}

\begingroup
\parskip0ex
\tableofcontents
\endgroup

%%%%%%%%%%%%%%%%%%%%%%%%%%%%%%%%%%%%%%%%%%%%%%%%%%%%%%%%%%%%%%%%%%%%%%%%%%%%%%%%
%%%%%%%%%%%%%%%%%%%%%%%%%%%%%%%%%%%%%%%%%%%%%%%%%%%%%%%%%%%%%%%%%%%%%%%%%%%%%%%%
\section{Introduction}

\LaTeX{} provides a mechanism to structure a large document (such as a book)
into a main file and several child files (containing the chapters)
using the |\include| command.
This mechanism is beneficial for documents
which span hundreds of pages in order to
make the source file(s) more manageable.
Moreover, compilation can be restricted to
selected child files by means of the |\includeonly| command.
The latter feature can be used to reduce the compilation time while editing
(this was significantly more useful in the earlier days of \LaTeX{})
or to generate a smaller document which is easier to navigate.
Another application of |\includeonly| is to generate
documents consisting of selected parts of the complete document.

However, there are a few drawbacks of the plain |\include| mechanism:
\begin{itemize}
\item
The child files cannot be compiled on their own,
they can only be compiled via the main file.
A naive editing environment
(such as a text editor with an option
to have the current file processed by \LaTeX)
may require one to switch to the main file before compiling;
attempting to compile the child file produces errors.
\item
The main file must be modified (each time)
to adjust the |\includeonly| command
to the present needs. This easily leaves the main file in a messy state.
\item
The generated document will always carry the filename
of the main document. This is inconvenient if
several child files are to be compiled and
to be kept for distribution.
\end{itemize}

The present package provides a simple interface
to make child files individually compilable by \LaTeX{}.
Compiling a child file then has the same effect as compiling
the main file with an |\includeonly| command
to select the appropriate child.
Moreover the generated document will carry the name of the child
rather than the main file.
This resolves all three above issues.

This feature is meant to make the editing of books,
thesis documents and lecture notes somewhat more convenient.
However, the package can also be used efficiently for
composing a series of documents (such as exercise sheets)
which are typically distributed individually.
It then assists the author in generating the individual documents
(potentially in different versions)
as well as a document containing the collected series.
Another application is in developing style files
or other kinds of included material
where compilation of the style file could redirect
to a sample or test file.

%%%%%%%%%%%%%%%%%%%%%%%%%%%%%%%%%%%%%%%%%%%%%%%%%%%%%%%%%%%%%%%%%%%%%%%%%%%%%%%%
%%%%%%%%%%%%%%%%%%%%%%%%%%%%%%%%%%%%%%%%%%%%%%%%%%%%%%%%%%%%%%%%%%%%%%%%%%%%%%%%
\section{Usage}

First of all, the package \textsf{childdoc} is \emph{not} a standard
\LaTeXe{} |.sty| style file! Therefore it needs to be invoked in
a non-standard way.

%%%%%%%%%%%%%%%%%%%%%%%%%%%%%%%%%%%%%%%%%%%%%%%%%%%%%%%%%%%%%%%%%%%%%%%%%%%%%%%%
\subsection{Included Files}
\label{sec:include}

%%%%%%%%%%%%%%%%%%%%%%%%%%%%%%%%%%%%%%%%
\DescribeMacro{\childdocmain}
To use the package, add the commands
\begin{center}
\begin{tabular}{l}
|\input{childdoc.def}|\\
|\childdocmain{}|\\
\end{tabular}
\end{center}
at the very top of the main \LaTeX{} file,
in particular \emph{before} the |\documentclass| statement!
The argument of |\childdocmain| should be left empty
(but it must be present).

%%%%%%%%%%%%%%%%%%%%%%%%%%%%%%%%%%%%%%%%
\DescribeMacro{\childdocof}
Furthermore, add the commands
\begin{center}
\begin{tabular}{l}
|\input{childdoc.def}|\\
|\childdocof{|\textit{main}|}|\\
\end{tabular}
\end{center}
at the top of every child file \textit{child}
which is included by |\include{|\textit{child}|}|
from within the main file
(or at least for those files to be compiled individually).
The argument \textit{main} must be the filename of the main file.

There are a couple of
considerations in setting up the main and child documents:

%%%%%%%%%%%%%%%%%%%%%%%%%%%%%%%%%%%%%%%%
\paragraph{Restrictions.}

Please note the following restrictions:
\begin{itemize}
\item
|\childdocmain| must be called with one argument \textit{main}
to ensure compatibility with earlier version of the package.
It must either be empty (|\childdocmain{}|)
or precisely match the filename of the main file in which it is specified.
See \secref{sec:detection} for further information.
\item
The filename \textit{main} must be specified without the |.tex| extension.
\item
The filename \textit{main} is case sensitive
(even in case-insensitive file systems)
due to internal string comparison.
\item
The argument \textit{main} should be fully expanded, it cannot be a macro.
\item
Subdirectories and special characters should be avoided in filenames.
\item
The command |\childdocmain{|\textit{main}|}| must be followed by a whitespace.
It should not be followed immediately by another command
or by a comment mark `|%|'.
This is because the \TeX{} parser reads the token immediately following
the argument of |\childdocmain| and puts it
at the beginning of every child section;
however, a white\-space is ignored.
\end{itemize}

%%%%%%%%%%%%%%%%%%%%%%%%%%%%%%%%%%%%%%%%
\paragraph{Content of Main File.}

It is advisable to place all content in the child files included by |\include|.
Any output contained in the main file will appear in all child documents
unless suppressed manually;
it cannot be suppressed automatically by the |\includeonly| directive
and thus should normally be avoided.
A method to include some content in the main file
by means of conditional processing is described in \secref{sec:conditional}.

%%%%%%%%%%%%%%%%%%%%%%%%%%%%%%%%%%%%%%%%
\paragraph{Page Numbering.}

When only a part of the document is compiled,
the appropriate numbering of pages
(as well as other status parameters)
is determined from the |.aux| files.
The latter contain information from previous passes.
However this information needs to propagate through
all intermediate child documents.
Therefore the page numbering in child documents may well
be inconsistent until the complete document is compiled at least once.

A useful (if unconventional) way to always ensure a consistent
page numbering is to restart the numbering in each child document
and denote the pages by `\textit{child}|.|\textit{page}'
where \textit{child} represents the chapter/section number of the child file.
This can be achieved by the command
|\numberwithin{page}{|\textit{child}|}|
of the \textsf{amsmath} package
where \textit{child} can be |chapter| or |section|
depending on the chosen structuring.
Alternatively, one can modify the macro |\thepage| appropriately
and reset the counter |page| at the start of each child file.

%%%%%%%%%%%%%%%%%%%%%%%%%%%%%%%%%%%%%%%%%%%%%%%%%%%%%%%%%%%%%%%%%%%%%%%%%%%%%%%%
\subsection{Conditional Processing}
\label{sec:conditional}

The package provides a mechanism to compile different versions
of a document. To customise the versions further some conditional processing
can come in handy to distinguish which version is being compiled.
The package provides two macros to describe the compilation context:

%%%%%%%%%%%%%%%%%%%%%%%%%%%%%%%%%%%%%%%%
\DescribeMacro{\ifchilddoc}
The conditional |\ifchilddoc| distinguishes between the compilation of
child documents and the main document:
%
\begin{center}
|\ifchilddoc |\textit{child-code}| |[|\||else |\textit{main-code}]| \||fi|
\end{center}

%%%%%%%%%%%%%%%%%%%%%%%%%%%%%%%%%%%%%%%%
\DescribeMacro{\childdocname}
\DescribeMacro{\childdocjob}
The macro |\childdocname| contains the filename (without extension)
of the main or child file being processed.
Note that |\childdocjob| will always contain the name of the main file.

%%%%%%%%%%%%%%%%%%%%%%%%%%%%%%%%%%%%%%%%
\paragraph{Title Page.}

Conditional processing can be used to include a title or banner page
in the main document when proper precautions are taken.
Importantly, the code in the main file should ensure that the page counter
(as well as other status parameters which are stored in the |.aux| files)
takes the same value after the conditional processing.
Otherwise the page numbers may take divergent values
depending on which part is compiled.

For example, a title page could be declared by:
%
\begin{center}
\begin{tabular}{l}
|\ifchilddoc\||else|\\
|\addtocounter{page}{-1}|\\
\textit{code for title page}\\
|\newpage|\\
|\||fi|
\end{tabular}
\end{center}
%
A banner page for the child documents can be generated by:
%
\begin{center}
\begin{tabular}{l}
|\ifchilddoc|\\
|\addtocounter{page}{-1}|\\
\textit{code for banner page}\\
|\newpage|\\
|\||fi|
\end{tabular}
\end{center}
%
Here one could write a message such as:
\begin{center}
|This is the part \childdocname{} of \childdocjob{}.|
\end{center}

%%%%%%%%%%%%%%%%%%%%%%%%%%%%%%%%%%%%%%%%%%%%%%%%%%%%%%%%%%%%%%%%%%%%%%%%%%%%%%%%
\subsection{Flags}
\label{sec:flags}

The package makes it easy to generate different versions
of the main or child documents.
To this end compilation flags can be defined
and assigned different default values.
They will be particularly useful in conjunction
with the forwarding mechanism described in \secref{sec:forward}.

For example, it may be useful to have a flag |\version|
which can be set to |draft| or |final|.
The document source will contain some conditional code
depending on the value of |\version|.
Suppose further, the flag should default to |final| for the main file
and to |draft| for child files
which is a natural assignment for editing the document.
This is achieved by placing the following code
in the preamble of the main document
(below the |\childdocmain| directive):
%
\begin{center}
\begin{tabular}{l}
|\ifchilddoc|\\
|\providecommand{\version}{draft}|\\
|\||else|\\
|\providecommand{\version}{final}|\\
|\||fi|
\end{tabular}
\end{center}
%
The definition by |\providecommand| makes sure
that previous definitions are not overwritten.
Further statements |\providecommand{\version}{...}|
can thus be added before the above code to override it.

For the main file, one might add a line
(between |\childdocmain| and the above block)
%
\begin{center}
|%\ifchilddoc\||else\providecommand{\version}{draft}\||fi|
\end{center}
%
which can be uncommented to produce a draft version.
Likewise one can add a line to the very top of a child file
(above the |\childdocof{|\textit{main}|}| directive)
%
\begin{center}
|%\providecommand{\version}{final}|
\end{center}
%
which can be uncommented to produce the final version of this child document.

%%%%%%%%%%%%%%%%%%%%%%%%%%%%%%%%%%%%%%%%%%%%%%%%%%%%%%%%%%%%%%%%%%%%%%%%%%%%%%%%
\subsection{Forwarding}
\label{sec:forward}

Different versions of the main or child documents
using compilation flags as described in \secref{sec:flags}
can be (permanently) stored in different files
for convenient compilation, viewing and distribution.
To this end, the package defines a command
to pass on compilation to a different file:

%%%%%%%%%%%%%%%%%%%%%%%%%%%%%%%%%%%%%%%%
\DescribeMacro{\childdocforward}
The command |\childdocforward| redirects processing to
another source file:
%
\begin{center}
\begin{tabular}{l}
|\input{childdoc.def}|\\
|\childdocforward[|\textit{main}|]{|\textit{dest}|}|\\
\end{tabular}
\end{center}
%
The argument \textit{dest} is the destination file
(without extension).
It should be the main file or one of the child files.
Note that further \textsf{childdoc} directives
such as |\childdocof| and |\childdocforward|
in the indicated file will be processed in this form.
The optional argument \textit{main}
passes on directly to the main file \textit{main}
while pretending to compile the child \textit{dest}.
This form behaves as if \textit{dest}
issues |\childdocof{|\textit{main}|}| right away,
and no further \textsf{childdoc} directives will be processed.

%%%%%%%%%%%%%%%%%%%%%%%%%%%%%%%%%%%%%%%%
\DescribeMacro{\...prefix}
In the alternative form |\childdocforwardprefix|,
%
\begin{center}
\begin{tabular}{l}
|\input{childdoc.def}|\\
|\childdocforwardprefix[|\textit{main}|]{|\textit{prefix}|}{|\textit{dest}|}|
\end{tabular}
\end{center}
%
the destination file is determined by a pattern
depending on the current file:
To make this work, the current file must be called
`{\textit{prefix}\hspace{0.2em}\textit{suffix}}'
with \textit{prefix} matching precisely the argument.
Processing is then passed on to the file
`{\textit{dest}\hspace{0.2em}\textit{suffix}}'.
Surely, the same effect is achieved by
directly specifying the
argument `{\textit{dest}\hspace{0.2em}\textit{suffix}}'
in the first form.
However, that requires to set up a different file
for each child. With the alternative form of the command
all these files can have exactly the same content
which simplifies setting them up and maintaining them.

For example, the following file |draft.tex|
with a compilation flag |\version| as described in \secref{sec:flags}
compiles the main document as a draft:
%
\begin{center}
\begin{tabular}{l}
|\def\version{draft}|\\
|\input{childdoc.def}|\\
|\childdocforward{|\textit{main}|}|
\end{tabular}
\end{center}
%
Likewise, the following files |final|\textit{nn}|.tex|
compile the final version of the child document
|child|\textit{nn}|.tex|:
%
\begin{center}
\begin{tabular}{l}
|\def\version{final}|\\
|\input{childdoc.def}|\\
|\childdocforwardprefix{final}{child}|
\end{tabular}
\end{center}
%

Note that when several versions of a main file and/or of each child file
are to be generated, it may be convenient to set up a |Makefile| or
shell script to automatise the process.

%%%%%%%%%%%%%%%%%%%%%%%%%%%%%%%%%%%%%%%%%%%%%%%%%%%%%%%%%%%%%%%%%%%%%%%%%%%%%%%%
\subsection{Command Line Processing}
\label{sec:commandline}

The effect of redirection files can also be achieved by invoking
the \LaTeX{} compiler with a more elaborate command line.
Most conveniently this should be done as part
of a shell script or a |Makefile|.

When using \textsf{childdoc} in the main file, the following
command lines effectively perform a redirection
(note that depending on the shell being used,
backslashes may have to be doubled: `|\|' $\to$ `|\\|'):
%
\begin{center}
|... -jobname "|\textit{target}|" |\\|"|[\textit{flags}]%
|\input{childdoc.def}\childdocforward[|\textit{main}|]{|\textit{dest}|}"|
\end{center}
%
Here \textit{target} is the name of the output file,
\textit{main} is the name of the main file
and \textit{dest} is the name of the main or child file to be processed
(all filenames without extensions).
The optional argument \textit{main} can be omitted
if \textit{main} matches \textit{dest}.
Optionally, compilation \textit{flags} can be defined via |\def| commands.
This command line makes the \TeX{} engine believe
it is compiling the file \textit{target}
whose content is specified as the latter parameter.
The provided code then forwards the processing to
\textit{main} or \textit{dest} as described in \secref{sec:forward}.

%%%%%%%%%%%%%%%%%%%%%%%%%%%%%%%%%%%%%%%%%%%%%%%%%%%%%%%%%%%%%%%%%%%%%%%%%%%%%%%%
\subsection{Include by Input}
\label{sec:input}

Including child documents by |\include| has some restrictions by design.
Most notably, the content of a child document always occupies
its own set of pages; pages cannot be shared between child documents.
Usually, this behaviour makes perfect sense
because each child document contain an essential part of the document.
However, in some situations it may be desirable to compose
a document from a collection of parts
without having mandatory page breaks between then.
For this case, the package
provides a mechanism to include parts
by |\input| which can also be processed individually.
However, by construction this mechanism
requires manual handling of the content to be output.

%%%%%%%%%%%%%%%%%%%%%%%%%%%%%%%%%%%%%%%%
\DescribeMacro{\ifchilddocmanual}
The main file should be prepared as usual, see \secref{sec:include}.
However, the document body must make a distinction
between processing of an individual part and of the main document, e.g.:
%
\begin{center}
\begin{tabular}{l}
|\ifchilddocmanual|\\
|\input{\childdocname}|\\
|\||else|\\
\textit{document body with }|\input{|\textit{part}|}|\\
|\||fi|
\end{tabular}
\end{center}
%
The conditional |\ifchilddocmanual| is true whenever
a part to be included by |\input| is being compiled,
and the name of the part is stored in |\childdocname|.

%%%%%%%%%%%%%%%%%%%%%%%%%%%%%%%%%%%%%%%%
\DescribeMacro{\childdocby}
Each part to be included by |\input| should start with:
%
\begin{center}
\begin{tabular}{l}
|\input{childdoc.def}|\\
|\childdocby{|\textit{main}|}|\\
\end{tabular}
\end{center}
%
The directive |\childdocby| is similar to |\childdocof|
described in \secref{sec:include},
but the subsequent selection of content must be done manually.
To that end, both |\ifchilddoc| and |\ifchilddocmanual|
will be true upon processing of a part,
and the name of the part is stored in |\childdocname|.
Note that |\jobname| will be set to the filename of the current part
so that each part receives an individual |.aux| file
that does not interfere with the |.aux| file(s) of the main document.
This behaviour can be altered by the alternative form
|\childdocby[*]{|\textit{main}|}| (with a non-empty optional argument)
which uses the |.aux| file of the main document
by setting |\jobname| to \textit{main}.

%%%%%%%%%%%%%%%%%%%%%%%%%%%%%%%%%%%%%%%%%%%%%%%%%%%%%%%%%%%%%%%%%%%%%%%%%%%%%%%%
\subsection{Driver Development}
\label{sec:driver}

The \textsf{childdoc} mechanism can also be use for the development
of definition files such as \LaTeX{} styles or classes.
This case differs from the above setup with multiple parts
included by |\include| in that no |\includeonly| should be invoked.
This can be achieved by starting the include file
(before |\ProvidesPackage|) with:
%
\begin{center}
\begin{tabular}{l}
|\input{childdoc.def}|\\
|\childdocforward{|\textit{main}|}|\\
\end{tabular}
\end{center}
%
or alternatively with:
%
\begin{center}
\begin{tabular}{l}
|\input{childdoc.def}|\\
|\childdocby{|\textit{main}|}|\\
\end{tabular}
\end{center}
%
Both forms have slightly different effects as described above.
The main file is prepared as usual, see \secref{sec:include}.

%%%%%%%%%%%%%%%%%%%%%%%%%%%%%%%%%%%%%%%%%%%%%%%%%%%%%%%%%%%%%%%%%%%%%%%%%%%%%%%%
\subsection{Legacy Detection}
\label{sec:detection}

The directive |\childdocmain| in the main file can detect
whether the complete document or merely a child is to be compiled
even without using the directive |\childdocof|.
This method is deprecated because it is less robust
and there is no compelling reason to use it;
it is merely provided for backward compatibility
and it may be removed in future versions.

If the detection mechanism is to be used,
it is mandatory to correctly specify
the filename of the main file as the argument of |\childdocmain|:
%
\begin{center}
\begin{tabular}{l}
|\input{childdoc.def}|\\
|\childdocmain{|\textit{main}|}|\\
\end{tabular}
\end{center}
%
If |\jobname| does not match the argument \textit{main} of |\childdocmain|,
it is assumed that |\jobname| points to the child file to be compiled.
When using |\childdocmain| with the main file specified as argument,
it suffices to start a child file
with just |\input{|\textit{main}|}|
without loading of the package and using |\childdocof|.
If instead all processing is done
with the appropriate \textsf{childdoc} directives,
the argument of \textit{main} of |\childdocmain| can be empty.

An alternative version of the command line processing described
in \secref{sec:commandline} using the detection mechanism reads:
%
\begin{center}
|... -jobname "|\textit{target}|" "|[\textit{flags}]%
[|\def\jobname{|\textit{dest}|}|]|\input{|\textit{main}|}"|
\end{center}

%%%%%%%%%%%%%%%%%%%%%%%%%%%%%%%%%%%%%%%%%%%%%%%%%%%%%%%%%%%%%%%%%%%%%%%%%%%%%%%%
\subsection{Manual Code}
\label{sec:manual}

In case one cannot be certain whether the definitions file |childdoc.def|
is installed on the target \TeX{} distribution
and one prefers not to ship it,
it is conceivable to paste a few relevant commands into the sources.

To that end, drop all statements |\input{childdoc.def}|
and perform the replacements as outlined below.
Instead of |\childdocmain{|\textit{main}|}| add the following code
to the top of the main file:
%
\begin{center}
\begin{tabular}{l}
|\||ifdefined\childdocname\endinput\||fi\newif\ifchilddoc|\\
|\edef\childdocname{\scantokens\expandafter{\jobname\noexpand}}|\\
|\def\childdocmain{|\textit{main}|}\||ifx\childdocmain\childdocname\||else|\\
|\childdoctrue\includeonly{\childdocname}\let\jobname\childdocmain\||fi|\\
\end{tabular}
\end{center}
%
Instead of |\childdocof{|\textit{main}|}| just include the main file
at the top of each child file:
%
\begin{center}
|\input{|\textit{main}|}|
\end{center}
%
A simple redirection |\childdocforward{|\textit{dest}|}| is achieved by:
%
\begin{center}
|\def\jobname{|\textit{dest}|}\input{\jobname}|
\end{center}
%
The redirection with prefix
|\childdocforwardprefix[|\textit{prefix}|]{|\textit{dest}|}|
is accomplished by:
%
\begin{center}
\begin{tabular}{l}
|{\edef\jobname{\scantokens\expandafter{\jobname\noexpand}}|\\
|\def\redirectjob |\textit{prefix}|#1~~~{\gdef\jobname{|\textit{dest}|#1}}|\\
|\expandafter\redirectjob\jobname~~~}\input{\jobname}|
\end{tabular}
\end{center}

In an alternative approach,
child documents can be compiled by a specific command line
without additional code or specific definitions:
%
\begin{center}
|... -jobname "|\textit{target}|" "|[\textit{flags}]%
|\includeonly{|\textit{dest}|}\input{|\textit{main}|}"|
\end{center}
%

%%%%%%%%%%%%%%%%%%%%%%%%%%%%%%%%%%%%%%%%%%%%%%%%%%%%%%%%%%%%%%%%%%%%%%%%%%%%%%%%
%%%%%%%%%%%%%%%%%%%%%%%%%%%%%%%%%%%%%%%%%%%%%%%%%%%%%%%%%%%%%%%%%%%%%%%%%%%%%%%%
\section{Information}

%%%%%%%%%%%%%%%%%%%%%%%%%%%%%%%%%%%%%%%%%%%%%%%%%%%%%%%%%%%%%%%%%%%%%%%%%%%%%%%%
\subsection{Copyright}

Copyright \copyright{} 2017--2018 Niklas Beisert

This work may be distributed and/or modified under the
conditions of the \LaTeX{} Project Public License, either version 1.3
of this license or (at your option) any later version.
The latest version of this license is in
  \url{http://www.latex-project.org/lppl.txt}
and version 1.3 or later is part of all distributions of \LaTeX{}
version 2005/12/01 or later.

This work has the LPPL maintenance status `maintained'.

The Current Maintainer of this work is Niklas Beisert.

This work consists of the files |README.txt|, |childdoc.ins| and |childdoc.dtx|
as well as the derived files |childdoc.def|, |cdocsamp.tex|
with |cdocsch1.tex|, |cdocsch2.tex|, |cdocspt3.tex|, |cdocspt4.tex|,
|cdocsdrf.tex|, |cdocsfn1.tex|, |cdocsfn2.tex|
as well as |childdoc.pdf|.

%%%%%%%%%%%%%%%%%%%%%%%%%%%%%%%%%%%%%%%%%%%%%%%%%%%%%%%%%%%%%%%%%%%%%%%%%%%%%%%%
\subsection{Files and Installation}

The package consists of the files:
%
\begin{center}
\begin{tabular}{ll}
    |README.txt|   & readme file \\
    |childdoc.ins| & installation file \\
    |childdoc.dtx| & source file \\
    |childdoc.def| & definition file \\
    |cdocsamp.tex| & sample main file \\
    |cdocsch1.tex| & sample include file \\
    |cdocsch2.tex| & sample include file \\
    |cdocspt3.tex| & sample part file \\
    |cdocspt4.tex| & sample part file \\
    |cdocsdrf.tex| & sample redirection file \\
    |cdocsfn1.tex| & sample redirection file \\
    |cdocsfn2.tex| & sample redirection file \\
    |childdoc.pdf| & manual
\end{tabular}
\end{center}
%
The distribution consists of the files
|README.txt|, |childdoc.ins| and |childdoc.dtx|.
%
\begin{itemize}
\item
Run (pdf)\LaTeX{} on |childdoc.dtx|
to compile the manual |childdoc.pdf| (this file).
\item
Run \LaTeX{} on |childdoc.ins| to create the definitions file |childdoc.def|
and the sample |cdocsamp.tex| with include files
|cdocsch1.tex|, |cdocsch2.tex|, |cdocspt3.tex|, |cdocspt4.tex|,
|cdocsdrf.tex|, |cdocsfn1.tex|, |cdocsfn2.tex|.
Then copy the file |childdoc.def| to an appropriate directory of your \LaTeX{}
distribution, e.g.\ \textit{texmf-root}|/tex/latex/childdoc|.
\end{itemize}

%%%%%%%%%%%%%%%%%%%%%%%%%%%%%%%%%%%%%%%%%%%%%%%%%%%%%%%%%%%%%%%%%%%%%%%%%%%%%%%%
\subsection{Related CTAN Packages}

There are several other packages which offer a similar functionality:
%
\begin{itemize}
\item
The packages
\href{http://ctan.org/pkg/docmute}{\textsf{docmute}},
\href{http://ctan.org/pkg/includex}{\textsf{includex}} and
\href{http://ctan.org/pkg/standalone}{\textsf{standalone}}
provide commands to include only the document body of
a child file thus allowing both files to be compiled individually.
\item
The packages \href{http://ctan.org/pkg/subdocs}{\textsf{subdocs}}
and \href{http://ctan.org/pkg/subfiles}{\textsf{subfiles}}
provide structures in which the main and child documents can be
encapsulated and allowing them to be compiled individually.
The inclusion mechanism is different from the conventional |\include|.
\item
The package \href{http://ctan.org/pkg/combine}{\textsf{combine}}
is an elaborate solution to combine several documents into one.
\end{itemize}
%
See also the CTAN topic \href{http://ctan.org/topic/subdocs}{\textsf{subdocs}}
for further related packages.
The present package differs from the above solutions in that
a document structure constructed with the conventional |\include| mechanism
just needs two extra commands at the top of every file
such that all constituent files can be compiled individually.

%%%%%%%%%%%%%%%%%%%%%%%%%%%%%%%%%%%%%%%%%%%%%%%%%%%%%%%%%%%%%%%%%%%%%%%%%%%%%%%%
%\subsection{Feature Suggestions}
%
%The following is a list of features which may be useful for future
%versions of this package:
%%
%\begin{itemize}
%\item
%\ldots
%\end{itemize}

%%%%%%%%%%%%%%%%%%%%%%%%%%%%%%%%%%%%%%%%%%%%%%%%%%%%%%%%%%%%%%%%%%%%%%%%%%%%%%%%
\subsection{Revision History}

%%%%%%%%%%%%%%%%%%%%%%%%%%%%%%%%%%%%%%%%
\paragraph{v2.0:} 2018/12/30

\begin{itemize}
\item
immediate forward processing
\item
added |\childdocby| mechanism
\item
manual restructured
\end{itemize}

%%%%%%%%%%%%%%%%%%%%%%%%%%%%%%%%%%%%%%%%
\paragraph{v1.6:} 2018/01/17

\begin{itemize}
\item
application for development of include files
\item
corrections to manual
\end{itemize}

%%%%%%%%%%%%%%%%%%%%%%%%%%%%%%%%%%%%%%%%
\paragraph{v1.5:} 2017/05/21

\begin{itemize}
\item
more complete structuring introduced
\item
|\childdocof| introduced
\item
|\childdoc| renamed to |\childdocmain|
\item
|\childredirect| renamed to |\childdocforward| and |\childdocforwardprefix|
and functionality expanded
\end{itemize}

%%%%%%%%%%%%%%%%%%%%%%%%%%%%%%%%%%%%%%%%
\paragraph{v1.0:} 2017/04/27

\begin{itemize}
\item
manual and install package
\item
first version published on CTAN
\end{itemize}

%%%%%%%%%%%%%%%%%%%%%%%%%%%%%%%%%%%%%%%%
\paragraph{v0.6:} 2017/04/26

\begin{itemize}
\item
redirection mechanism added
\end{itemize}

%%%%%%%%%%%%%%%%%%%%%%%%%%%%%%%%%%%%%%%%
\paragraph{v0.5:} 2017/04/26

\begin{itemize}
\item
functionality in definition file
\end{itemize}


%%%%%%%%%%%%%%%%%%%%%%%%%%%%%%%%%%%%%%%%%%%%%%%%%%%%%%%%%%%%%%%%%%%%%%%%%%%%%%%%
%%%%%%%%%%%%%%%%%%%%%%%%%%%%%%%%%%%%%%%%%%%%%%%%%%%%%%%%%%%%%%%%%%%%%%%%%%%%%%%%
%%%%%%%%%%%%%%%%%%%%%%%%%%%%%%%%%%%%%%%%%%%%%%%%%%%%%%%%%%%%%%%%%%%%%%%%%%%%%%%%
\appendix

\settowidth\MacroIndent{\rmfamily\scriptsize 000\ }

 \DocInput{childdoc.dtx}

\end{document}
%</driver>
% \fi
%
% %%%%%%%%%%%%%%%%%%%%%%%%%%%%%%%%%%%%%%%%%%%%%%%%%%%%%%%%%%%%%%%%%%%%%%%%%%%%%%
% %%%%%%%%%%%%%%%%%%%%%%%%%%%%%%%%%%%%%%%%%%%%%%%%%%%%%%%%%%%%%%%%%%%%%%%%%%%%%%
% \section{Sample}
%\iffalse
%<*samplemain>
%\fi
%
% The following presents a sample document
% with two chapters, two parts, a title page,
% a compile flag as well as three forwarding files to set the flag.
% It consists of eight |.tex| files:
% \begin{center}
% \begin{tabular}{ll}
% |cdocsamp.tex|&main file\\
% |cdocsch1.tex|&include file for chapter 1\\
% |cdocsch2.tex|&include file for chapter 2\\
% |cdocspt3.tex|&include file for part 3\\
% |cdocspt4.tex|&include file for part 4\\
% |cdocsdrf.tex|&forwarding file for main file in draft mode\\
% |cdocsfi1.tex|&forwarding file for final version of chapter 1\\
% |cdocsfi2.tex|&forwarding file for final version of chapter 2\\
% \end{tabular}
% \end{center}
% Each of the eight files can be compiled directly by the \LaTeX{} compiler.
%
% %%%%%%%%%%%%%%%%%%%%%%%%%%%%%%%%%%%%%%
% \paragraph{Main File.}
%
% The main file is called |cdocsamp.tex|.
%
% Load the \textsf{childdoc} definitions and
% declare the filename for the main document:
%    \begin{macrocode}
\input{childdoc.def}
\childdocmain{}
%    \end{macrocode}

% Optional override for |\version| flag:
%    \begin{macrocode}
%%\ifchilddoc\else\providecommand{\version}{draft}\fi
%    \end{macrocode}

% Define the default values for the |\version| flag
% (|final| for the main file and |draft| for childs):
%    \begin{macrocode}
\ifchilddoc
\providecommand{\version}{draft}
\else
\providecommand{\version}{final}
\fi
%    \end{macrocode}

% Load the standard document class:
%    \begin{macrocode}
\documentclass[12pt]{article}
%    \end{macrocode}

% Start the document body:
%    \begin{macrocode}
\begin{document}
%    \end{macrocode}

% Declare a title page.
% Print title, part of document being processed and version flag:
%    \begin{macrocode}
\addtocounter{page}{-1}
\begin{center}
{\LARGE\bfseries{}childdoc example\par}
\vspace{1cm}
\ifchilddoc
\ifchilddocmanual part\else chapter\fi:
`\childdocname' of `\childdocjob'\par
\else
main document: `\childdocjob'\par
\fi
version: \version\par
\end{center}
\newpage
%    \end{macrocode}

% Manually include selected file,
% otherwise process as usual:
%    \begin{macrocode}
\ifchilddocmanual
\section*{part `\childdocname'}
\input{\childdocname}
\else
%    \end{macrocode}

% Include the two chapters:
%    \begin{macrocode}
\include{cdocsch1}
\include{cdocsch2}
%    \end{macrocode}

% Include the two parts unless only chapters should be displayed:
%    \begin{macrocode}
\ifchilddoc\else
\section{part three}
\input{cdocspt3}
\section{part four}
\input{cdocspt4}
\fi
%    \end{macrocode}

% Process as usual until here:
%    \begin{macrocode}
\fi
%    \end{macrocode}

% End of document body:
%    \begin{macrocode}
\end{document}
%    \end{macrocode}
%\iffalse
%</samplemain>
%\fi
%
% %%%%%%%%%%%%%%%%%%%%%%%%%%%%%%%%%%%%%%
% \paragraph{Chapter Include Files.}
%
% The include files are called |cdocsch1.tex| and |cdocsch2.tex|.
%
%\iffalse
%<*samplechap1|samplechap2>
%\fi

% Optional override for |\version| flag:
%    \begin{macrocode}
%%\providecommand{\version}{final}
%    \end{macrocode}

% Include the main document:
%    \begin{macrocode}
\input{childdoc.def}
\childdocof{cdocsamp}
%    \end{macrocode}

%\iffalse
%</samplechap1|samplechap2>
%\fi
%
%\iffalse
%<*samplechap1>
%\fi
% Some text for chapter 1:
%    \begin{macrocode}
\section{one}
some text in chapter one
%    \end{macrocode}

%\iffalse
%</samplechap1>
%\fi
% Some text for chapter 2:
%\iffalse
%<*samplechap2>
%\fi
%    \begin{macrocode}
\section{two}
more text in chapter two
%    \end{macrocode}

%\iffalse
%</samplechap2>
%\fi
%
% %%%%%%%%%%%%%%%%%%%%%%%%%%%%%%%%%%%%%%
% \paragraph{Part Include Files.}
%
% The include files are called |cdocspt3.tex| and |cdocspt4.tex|.
%
%\iffalse
%<*samplepart3|samplepart4>
%\fi

% Optional override for |\version| flag:
%    \begin{macrocode}
%%\providecommand{\version}{final}
%    \end{macrocode}

% Include the main document:
%    \begin{macrocode}
\input{childdoc.def}
\childdocby{cdocsamp}
%    \end{macrocode}

%\iffalse
%</samplepart3|samplepart4>
%\fi
%
%\iffalse
%<*samplepart3>
%\fi
% Some text for part 3:
%    \begin{macrocode}
some text in part three
%    \end{macrocode}

%\iffalse
%</samplepart3>
%\fi
% Some text for part 4:
%\iffalse
%<*samplepart4>
%\fi
%    \begin{macrocode}
more text in part four
%    \end{macrocode}

%\iffalse
%</samplepart4>
%\fi
%
% %%%%%%%%%%%%%%%%%%%%%%%%%%%%%%%%%%%%%%
% \paragraph{Forwarding for a Complete Draft.}
%
% The following forwarding file |cdocsdrf.tex|
% compiles the main document in draft mode:
%\iffalse
%<*sampledraft>
%\fi
%    \begin{macrocode}
\def\version{draft}
\input{childdoc.def}
\childdocforward{cdocsamp}
%    \end{macrocode}

%\iffalse
%</sampledraft>
%\fi
%
% %%%%%%%%%%%%%%%%%%%%%%%%%%%%%%%%%%%%%%
% \paragraph{Forwarding for Final Version of the Chapters.}
%
% The following forwarding files |cdocsfn1.tex| and |cdocsfn2.tex|
% (with identical content)
% compile the final versions of the child documents
% |cdocsch1.tex| and |cdocsch2.tex|, respectively:
%\iffalse
%<*samplefinal>
%\fi
%    \begin{macrocode}
\def\version{final}
\input{childdoc.def}
\childdocforwardprefix[cdocsamp]{cdocsfn}{cdocsch}
%    \end{macrocode}

%\iffalse
%</samplefinal>
%\fi
%
% %%%%%%%%%%%%%%%%%%%%%%%%%%%%%%%%%%%%%%
% \paragraph{Command Line Processing.}
%
% The following three command lines generate the output files
% |cdocscld|, |cdocscl1| and |cdocscl2|
% which should be identical to
% |cdocsdrf|, |cdocsch1| and |cdocsfn2|, respectively:
% \begin{center}
% \begin{tabular}{l}
% |latex -jobname cdocscld \|\\
% |  "\def\version{draft}\input{childdoc.def}\childdocforward{cdocsamp}"|\\
% |latex -jobname cdocscl1 \|\\
% |  "\input{childdoc.def}\childdocforward[cdocsamp]{cdocsch1}"|\\
% |latex -jobname cdocscl2 \|\\
% |  "\def\version{final}\input{childdoc.def}\childdocforward{cdocsch2}"|
% \end{tabular}
% \end{center}
% Note that the trailing backslash on each first line
% merely continues the input to the second line
% (for convenient cut ant paste).
% Furthermore, the command |latex| can be replaced by any
% of its alternative versions such as |pdflatex|.
%
% %%%%%%%%%%%%%%%%%%%%%%%%%%%%%%%%%%%%%%%%%%%%%%%%%%%%%%%%%%%%%%%%%%%%%%%%%%%%%%
% %%%%%%%%%%%%%%%%%%%%%%%%%%%%%%%%%%%%%%%%%%%%%%%%%%%%%%%%%%%%%%%%%%%%%%%%%%%%%%
% \section{Implementation}
%\iffalse
%<*package>
%\fi
%
% This section describes the definitions file |childdoc.def|.

% The definitions cannot be loaded using |\usepackage| or |\RequirePackage|
% which has a mechanism to prevent loading a style file more than once.
% When loading the definitions by means of |\input|
% multiple instances have to be prevented manually:
%\iffalse
%This code needs to be before the `\ProvidesFile' directive
%which is defined at the beginning of this file.
%Therefore it is also placed there and commented out here.
%</package>
%<*discard>
%\fi
%    \begin{macrocode}
\ifdefined\childdocmain\endinput\fi
%    \end{macrocode}
%\iffalse
%</discard>
%<*package>
%\fi
%
% \macro{\ifchilddoc}
% \macro{\ifchilddocmanual}
% The conditional |\ifchilddoc| tells whether a
% child (true) or main (false) document is being compiled.
% The conditional |\ifchilddocmanual| tells whether
% the |\includeonly| mechanism is used (false) or
% the selection of child files must be performed manually (true).
% The definitions initialise to false:
%    \begin{macrocode}
\newif\ifchilddoc
\newif\ifchilddocmanual
%    \end{macrocode}

% \macro{\childdocname}
% \macro{\childdocjob}
% The macro |\childdocname| stores the name of the main document
% to be compiled. The macro |\childdocjob| stores the name of
% the document on which the \LaTeX{} compiler was originally invoked.
% The content of |\jobname| cannot be compared
% to filenames specified in the source due to different catcodes.
% The following code rescans |\jobname|, stores the result
% in |\childdocname| and saves a copy in |\childdocjob|:
%    \begin{macrocode}
\edef\childdocname{\scantokens\expandafter{\jobname\noexpand}}
\let\childdocjob\childdocname
%    \end{macrocode}

% \macro{\childdocdisable}
% The macro |\childdocdisable| prevents the main file
% from being processed more than once.
% At this stage, the main document command |\childdocmain|
% is assumed to be called once again where it should do nothing.
% Any subsequent call to it should prevent
% a secondary processing of the main document
% It overwrites the forwarding commands
% |\childdocof| and |\childdocforward|
% with empty macros to prevent further inclusions of the main document:
%    \begin{macrocode}
\newcommand{\childdocdisable}
{
  \renewcommand{\childdocmain}[1]{\renewcommand{\childdocmain}[1]{\endinput}}
  \renewcommand{\childdocof}[1]{}
  \renewcommand{\childdocby}[2][]{}
  \renewcommand{\childdocforward}[2][]{}
  \renewcommand{\childdocdisable}{}
}
%    \end{macrocode}

% \macro{\childdocmain}
% The macro |\childdocmain| is to be called at the top of the main file
% with nothing or the main filename (without extension) as argument.
% First, it breaks loops.
% If the argument is not empty and does not match |\childdocname|
% (which is set by the first inclusion of |childdoc.def|),
% |\ifchilddoc| is set to true, |\includeonly| is applied to the child file
% and |\jobname| is set to the main file
% (for proper handling of |.aux| files):
%    \begin{macrocode}
\newcommand{\childdocmain}[1]
{
  \childdocdisable\childdocmain{}
  \if?#1?\else
    \begingroup
      \def\childdoctmp{#1}
      \ifx\childdoctmp\childdocname
        \def\childdoctmp{}
      \else
        \def\childdoctmp
        {
          \childdoctrue
          \includeonly{\childdocname}
          \def\childdocjob{#1}
          \def\jobname{#1}
        }
      \fi
      \expandafter
    \endgroup
    \childdoctmp
  \fi
}
%    \end{macrocode}

% \macro{\childdocof}
% The command |\childdocof| redirects
% compilation to the main file |#1|.
%    \begin{macrocode}
\newcommand{\childdocof}[1]
{
  \childdocdisable
  \childdoctrue
  \includeonly{\childdocname}
  \def\jobname{#1}
  \def\childdocjob{#1}
  \input{#1}
}
%    \end{macrocode}

% \macro{\childdocby}
% The command |\childdocby| ....
%    \begin{macrocode}
\newcommand{\childdocby}[2][]
{
  \childdocdisable
  \childdoctrue
  \childdocmanualtrue
  \if?#1?\else
    \def\jobname{#2}
  \fi
  \def\childdocjob{#2}
  \input{#2}
  \endinput
}
%    \end{macrocode}

% \macro{\childdocforward}
% The command |\childdocforward| redirects
% compilation to the main file or
% (if the optional argument is given) a child file.
% Parameters are set as if the main file
% or a child file starting with |\childdocof| was compiled.
% Then compilation is handed over to the main file:
%    \begin{macrocode}
\newcommand{\childdocforward}[2][]
{
  \begingroup
    \if?#1?
      \def\childdoctmp
      {
        \def\childdocname{#2}
        \def\childdocjob{#2}
        \def\jobname{#2}
        \input{#2}
        \endinput
      }
    \else
      \def\childdoctmp
      {
        \childdocdisable
        \def\childdocname{#2}
        \childdoctrue
        \includeonly{#2}
        \def\childdocjob{#1}
        \def\jobname{#1}
        \input{#1}
        \endinput
      }
    \fi
    \expandafter
  \endgroup
  \childdoctmp
}
%    \end{macrocode}

% \macro{\childdocforwardprefix}
% The command |\childdocforwardprefix| redirects
% compilation to the main or a child file by means of a pattern.
% The prefix |#1| in the current filename is replaced by |#2|
% and the suffix of the current filename is kept
% (it is assumed that the filename does not contain the substring `|~~~|'
% which is used as a delimiter).
% Compilation is handed over to the new file by |\childdocforward|:
%    \begin{macrocode}
\newcommand{\childdocforwardprefix}[3][]
{
  \begingroup
    \def\childdocextract #2##1~~~{\def\childdoctmp{\childdocforward[#1]{#3##1}}}
    \expandafter\childdocextract\childdocname~~~
    \expandafter
  \endgroup
  \childdoctmp
}
%    \end{macrocode}

% \macro{\childdoc}
% The deprecated macro |\childdoc| is a legacy version of |\childdocmain|:
%    \begin{macrocode}
\newcommand{\childdoc}{\childdocmain}
%    \end{macrocode}

% \macro{\childdocredirect}
% The deprecated macro |\childdocredirect| is a legacy version
% of |\childdocforward| and |\childdocforwardprefix|:
%    \begin{macrocode}
\newcommand{\childdocredirect}[2][]
{
  \begingroup
    \if?#1?
      \def\childdoctmp{\childdocforward{#2}}
    \else
      \def\childdoctmp{\childdocforwardprefix{#1}{#2}}
    \fi
    \expandafter
  \endgroup
  \childdoctmp
}
%    \end{macrocode}

%\iffalse
%</package>
%\fi
%
\endinput

\childdocforward{cdocsamp}
%    \end{macrocode}

%\iffalse
%</sampledraft>
%\fi
%
% %%%%%%%%%%%%%%%%%%%%%%%%%%%%%%%%%%%%%%
% \paragraph{Forwarding for Final Version of the Chapters.}
%
% The following forwarding files |cdocsfn1.tex| and |cdocsfn2.tex|
% (with identical content)
% compile the final versions of the child documents
% |cdocsch1.tex| and |cdocsch2.tex|, respectively:
%\iffalse
%<*samplefinal>
%\fi
%    \begin{macrocode}
\def\version{final}
% \iffalse
%
% childdoc.dtx Copyright (C) 2017-2018 Niklas Beisert
%
% This work may be distributed and/or modified under the
% conditions of the LaTeX Project Public License, either version 1.3
% of this license or (at your option) any later version.
% The latest version of this license is in
%   http://www.latex-project.org/lppl.txt
% and version 1.3 or later is part of all distributions of LaTeX
% version 2005/12/01 or later.
%
% This work has the LPPL maintenance status `maintained'.
%
% The Current Maintainer of this work is Niklas Beisert.
%
% This work consists of the files childdoc.dtx and childdoc.ins
% and the derived files childdoc.def and cdocsamp.tex with
% cdocsch1.tex, cdocsch2.tex, cdocsdrf.tex, cdocsfn1.tex, cdocsfn2.tex.
%
%<package>\ifdefined\childdocmain\endinput\fi
%<package>\ProvidesFile{childdoc.def}[2018/12/30 v2.0 child document driver]
%<samplemain>\ProvidesFile{cdocsamp.tex}[2018/12/30 v2.0 sample for childdoc]
%<*driver>
%\ProvidesFile{childdoc.drv}[2018/12/30 v2.0 childdoc reference manual file]
\PassOptionsToClass{10pt,a4paper}{article}
\documentclass{ltxdoc}

\usepackage[margin=35mm]{geometry}
\usepackage{hyperref}
\usepackage{hyperxmp}
\usepackage[usenames]{color}

\hypersetup{colorlinks=true}
\hypersetup{pdfstartview=FitH}
\hypersetup{pdfpagemode=UseNone}
\hypersetup{pdfsource={}}
\hypersetup{pdflang={en-UK}}
\hypersetup{pdfcopyright={Copyright 2017-2018 Niklas Beisert.
  This work may be distributed and/or modified under the
  conditions of the LaTeX Project Public License, either version 1.3
  of this license or (at your option) any later version.}}
\hypersetup{pdflicenseurl={http://www.latex-project.org/lppl.txt}}
\hypersetup{pdfcontactaddress={ETH Zurich, ITP, HIT K,
  Wolfgang-Pauli-Strasse 27}}
\hypersetup{pdfcontactpostcode={8093}}
\hypersetup{pdfcontactcity={Zurich}}
\hypersetup{pdfcontactcountry={Switzerland}}
\hypersetup{pdfcontactemail={nbeisert@itp.phys.ethz.ch}}
\hypersetup{pdfcontacturl={http://people.phys.ethz.ch/\xmptilde nbeisert/}}

\newcommand{\secref}[1]{\hyperref[#1]{section \ref*{#1}}}

\parskip1ex
\parindent0pt
\let\olditemize\itemize
\def\itemize{\olditemize\parskip0pt}

\begin{document}

\title{The \textsf{childdoc} Package}
\hypersetup{pdftitle={The childdoc Package}}
\author{Niklas Beisert\\[2ex]
  Institut f\"ur Theoretische Physik\\
  Eidgen\"ossische Technische Hochschule Z\"urich\\
  Wolfgang-Pauli-Strasse 27, 8093 Z\"urich, Switzerland\\[1ex]
  \href{mailto:nbeisert@itp.phys.ethz.ch}
  {\texttt{nbeisert@itp.phys.ethz.ch}}}
\hypersetup{pdfauthor={Niklas Beisert}}
\hypersetup{pdfsubject={Manual for the LaTeX2e Package childdoc}}
\date{30 December 2018, \textsf{v2.0}}
\maketitle

\begin{abstract}\noindent
\textsf{childdoc} is a \LaTeXe{} package
that enables the direct compilation
of document sections included by |\include|
to individual files.
\end{abstract}

\begingroup
\parskip0ex
\tableofcontents
\endgroup

%%%%%%%%%%%%%%%%%%%%%%%%%%%%%%%%%%%%%%%%%%%%%%%%%%%%%%%%%%%%%%%%%%%%%%%%%%%%%%%%
%%%%%%%%%%%%%%%%%%%%%%%%%%%%%%%%%%%%%%%%%%%%%%%%%%%%%%%%%%%%%%%%%%%%%%%%%%%%%%%%
\section{Introduction}

\LaTeX{} provides a mechanism to structure a large document (such as a book)
into a main file and several child files (containing the chapters)
using the |\include| command.
This mechanism is beneficial for documents
which span hundreds of pages in order to
make the source file(s) more manageable.
Moreover, compilation can be restricted to
selected child files by means of the |\includeonly| command.
The latter feature can be used to reduce the compilation time while editing
(this was significantly more useful in the earlier days of \LaTeX{})
or to generate a smaller document which is easier to navigate.
Another application of |\includeonly| is to generate
documents consisting of selected parts of the complete document.

However, there are a few drawbacks of the plain |\include| mechanism:
\begin{itemize}
\item
The child files cannot be compiled on their own,
they can only be compiled via the main file.
A naive editing environment
(such as a text editor with an option
to have the current file processed by \LaTeX)
may require one to switch to the main file before compiling;
attempting to compile the child file produces errors.
\item
The main file must be modified (each time)
to adjust the |\includeonly| command
to the present needs. This easily leaves the main file in a messy state.
\item
The generated document will always carry the filename
of the main document. This is inconvenient if
several child files are to be compiled and
to be kept for distribution.
\end{itemize}

The present package provides a simple interface
to make child files individually compilable by \LaTeX{}.
Compiling a child file then has the same effect as compiling
the main file with an |\includeonly| command
to select the appropriate child.
Moreover the generated document will carry the name of the child
rather than the main file.
This resolves all three above issues.

This feature is meant to make the editing of books,
thesis documents and lecture notes somewhat more convenient.
However, the package can also be used efficiently for
composing a series of documents (such as exercise sheets)
which are typically distributed individually.
It then assists the author in generating the individual documents
(potentially in different versions)
as well as a document containing the collected series.
Another application is in developing style files
or other kinds of included material
where compilation of the style file could redirect
to a sample or test file.

%%%%%%%%%%%%%%%%%%%%%%%%%%%%%%%%%%%%%%%%%%%%%%%%%%%%%%%%%%%%%%%%%%%%%%%%%%%%%%%%
%%%%%%%%%%%%%%%%%%%%%%%%%%%%%%%%%%%%%%%%%%%%%%%%%%%%%%%%%%%%%%%%%%%%%%%%%%%%%%%%
\section{Usage}

First of all, the package \textsf{childdoc} is \emph{not} a standard
\LaTeXe{} |.sty| style file! Therefore it needs to be invoked in
a non-standard way.

%%%%%%%%%%%%%%%%%%%%%%%%%%%%%%%%%%%%%%%%%%%%%%%%%%%%%%%%%%%%%%%%%%%%%%%%%%%%%%%%
\subsection{Included Files}
\label{sec:include}

%%%%%%%%%%%%%%%%%%%%%%%%%%%%%%%%%%%%%%%%
\DescribeMacro{\childdocmain}
To use the package, add the commands
\begin{center}
\begin{tabular}{l}
|\input{childdoc.def}|\\
|\childdocmain{}|\\
\end{tabular}
\end{center}
at the very top of the main \LaTeX{} file,
in particular \emph{before} the |\documentclass| statement!
The argument of |\childdocmain| should be left empty
(but it must be present).

%%%%%%%%%%%%%%%%%%%%%%%%%%%%%%%%%%%%%%%%
\DescribeMacro{\childdocof}
Furthermore, add the commands
\begin{center}
\begin{tabular}{l}
|\input{childdoc.def}|\\
|\childdocof{|\textit{main}|}|\\
\end{tabular}
\end{center}
at the top of every child file \textit{child}
which is included by |\include{|\textit{child}|}|
from within the main file
(or at least for those files to be compiled individually).
The argument \textit{main} must be the filename of the main file.

There are a couple of
considerations in setting up the main and child documents:

%%%%%%%%%%%%%%%%%%%%%%%%%%%%%%%%%%%%%%%%
\paragraph{Restrictions.}

Please note the following restrictions:
\begin{itemize}
\item
|\childdocmain| must be called with one argument \textit{main}
to ensure compatibility with earlier version of the package.
It must either be empty (|\childdocmain{}|)
or precisely match the filename of the main file in which it is specified.
See \secref{sec:detection} for further information.
\item
The filename \textit{main} must be specified without the |.tex| extension.
\item
The filename \textit{main} is case sensitive
(even in case-insensitive file systems)
due to internal string comparison.
\item
The argument \textit{main} should be fully expanded, it cannot be a macro.
\item
Subdirectories and special characters should be avoided in filenames.
\item
The command |\childdocmain{|\textit{main}|}| must be followed by a whitespace.
It should not be followed immediately by another command
or by a comment mark `|%|'.
This is because the \TeX{} parser reads the token immediately following
the argument of |\childdocmain| and puts it
at the beginning of every child section;
however, a white\-space is ignored.
\end{itemize}

%%%%%%%%%%%%%%%%%%%%%%%%%%%%%%%%%%%%%%%%
\paragraph{Content of Main File.}

It is advisable to place all content in the child files included by |\include|.
Any output contained in the main file will appear in all child documents
unless suppressed manually;
it cannot be suppressed automatically by the |\includeonly| directive
and thus should normally be avoided.
A method to include some content in the main file
by means of conditional processing is described in \secref{sec:conditional}.

%%%%%%%%%%%%%%%%%%%%%%%%%%%%%%%%%%%%%%%%
\paragraph{Page Numbering.}

When only a part of the document is compiled,
the appropriate numbering of pages
(as well as other status parameters)
is determined from the |.aux| files.
The latter contain information from previous passes.
However this information needs to propagate through
all intermediate child documents.
Therefore the page numbering in child documents may well
be inconsistent until the complete document is compiled at least once.

A useful (if unconventional) way to always ensure a consistent
page numbering is to restart the numbering in each child document
and denote the pages by `\textit{child}|.|\textit{page}'
where \textit{child} represents the chapter/section number of the child file.
This can be achieved by the command
|\numberwithin{page}{|\textit{child}|}|
of the \textsf{amsmath} package
where \textit{child} can be |chapter| or |section|
depending on the chosen structuring.
Alternatively, one can modify the macro |\thepage| appropriately
and reset the counter |page| at the start of each child file.

%%%%%%%%%%%%%%%%%%%%%%%%%%%%%%%%%%%%%%%%%%%%%%%%%%%%%%%%%%%%%%%%%%%%%%%%%%%%%%%%
\subsection{Conditional Processing}
\label{sec:conditional}

The package provides a mechanism to compile different versions
of a document. To customise the versions further some conditional processing
can come in handy to distinguish which version is being compiled.
The package provides two macros to describe the compilation context:

%%%%%%%%%%%%%%%%%%%%%%%%%%%%%%%%%%%%%%%%
\DescribeMacro{\ifchilddoc}
The conditional |\ifchilddoc| distinguishes between the compilation of
child documents and the main document:
%
\begin{center}
|\ifchilddoc |\textit{child-code}| |[|\||else |\textit{main-code}]| \||fi|
\end{center}

%%%%%%%%%%%%%%%%%%%%%%%%%%%%%%%%%%%%%%%%
\DescribeMacro{\childdocname}
\DescribeMacro{\childdocjob}
The macro |\childdocname| contains the filename (without extension)
of the main or child file being processed.
Note that |\childdocjob| will always contain the name of the main file.

%%%%%%%%%%%%%%%%%%%%%%%%%%%%%%%%%%%%%%%%
\paragraph{Title Page.}

Conditional processing can be used to include a title or banner page
in the main document when proper precautions are taken.
Importantly, the code in the main file should ensure that the page counter
(as well as other status parameters which are stored in the |.aux| files)
takes the same value after the conditional processing.
Otherwise the page numbers may take divergent values
depending on which part is compiled.

For example, a title page could be declared by:
%
\begin{center}
\begin{tabular}{l}
|\ifchilddoc\||else|\\
|\addtocounter{page}{-1}|\\
\textit{code for title page}\\
|\newpage|\\
|\||fi|
\end{tabular}
\end{center}
%
A banner page for the child documents can be generated by:
%
\begin{center}
\begin{tabular}{l}
|\ifchilddoc|\\
|\addtocounter{page}{-1}|\\
\textit{code for banner page}\\
|\newpage|\\
|\||fi|
\end{tabular}
\end{center}
%
Here one could write a message such as:
\begin{center}
|This is the part \childdocname{} of \childdocjob{}.|
\end{center}

%%%%%%%%%%%%%%%%%%%%%%%%%%%%%%%%%%%%%%%%%%%%%%%%%%%%%%%%%%%%%%%%%%%%%%%%%%%%%%%%
\subsection{Flags}
\label{sec:flags}

The package makes it easy to generate different versions
of the main or child documents.
To this end compilation flags can be defined
and assigned different default values.
They will be particularly useful in conjunction
with the forwarding mechanism described in \secref{sec:forward}.

For example, it may be useful to have a flag |\version|
which can be set to |draft| or |final|.
The document source will contain some conditional code
depending on the value of |\version|.
Suppose further, the flag should default to |final| for the main file
and to |draft| for child files
which is a natural assignment for editing the document.
This is achieved by placing the following code
in the preamble of the main document
(below the |\childdocmain| directive):
%
\begin{center}
\begin{tabular}{l}
|\ifchilddoc|\\
|\providecommand{\version}{draft}|\\
|\||else|\\
|\providecommand{\version}{final}|\\
|\||fi|
\end{tabular}
\end{center}
%
The definition by |\providecommand| makes sure
that previous definitions are not overwritten.
Further statements |\providecommand{\version}{...}|
can thus be added before the above code to override it.

For the main file, one might add a line
(between |\childdocmain| and the above block)
%
\begin{center}
|%\ifchilddoc\||else\providecommand{\version}{draft}\||fi|
\end{center}
%
which can be uncommented to produce a draft version.
Likewise one can add a line to the very top of a child file
(above the |\childdocof{|\textit{main}|}| directive)
%
\begin{center}
|%\providecommand{\version}{final}|
\end{center}
%
which can be uncommented to produce the final version of this child document.

%%%%%%%%%%%%%%%%%%%%%%%%%%%%%%%%%%%%%%%%%%%%%%%%%%%%%%%%%%%%%%%%%%%%%%%%%%%%%%%%
\subsection{Forwarding}
\label{sec:forward}

Different versions of the main or child documents
using compilation flags as described in \secref{sec:flags}
can be (permanently) stored in different files
for convenient compilation, viewing and distribution.
To this end, the package defines a command
to pass on compilation to a different file:

%%%%%%%%%%%%%%%%%%%%%%%%%%%%%%%%%%%%%%%%
\DescribeMacro{\childdocforward}
The command |\childdocforward| redirects processing to
another source file:
%
\begin{center}
\begin{tabular}{l}
|\input{childdoc.def}|\\
|\childdocforward[|\textit{main}|]{|\textit{dest}|}|\\
\end{tabular}
\end{center}
%
The argument \textit{dest} is the destination file
(without extension).
It should be the main file or one of the child files.
Note that further \textsf{childdoc} directives
such as |\childdocof| and |\childdocforward|
in the indicated file will be processed in this form.
The optional argument \textit{main}
passes on directly to the main file \textit{main}
while pretending to compile the child \textit{dest}.
This form behaves as if \textit{dest}
issues |\childdocof{|\textit{main}|}| right away,
and no further \textsf{childdoc} directives will be processed.

%%%%%%%%%%%%%%%%%%%%%%%%%%%%%%%%%%%%%%%%
\DescribeMacro{\...prefix}
In the alternative form |\childdocforwardprefix|,
%
\begin{center}
\begin{tabular}{l}
|\input{childdoc.def}|\\
|\childdocforwardprefix[|\textit{main}|]{|\textit{prefix}|}{|\textit{dest}|}|
\end{tabular}
\end{center}
%
the destination file is determined by a pattern
depending on the current file:
To make this work, the current file must be called
`{\textit{prefix}\hspace{0.2em}\textit{suffix}}'
with \textit{prefix} matching precisely the argument.
Processing is then passed on to the file
`{\textit{dest}\hspace{0.2em}\textit{suffix}}'.
Surely, the same effect is achieved by
directly specifying the
argument `{\textit{dest}\hspace{0.2em}\textit{suffix}}'
in the first form.
However, that requires to set up a different file
for each child. With the alternative form of the command
all these files can have exactly the same content
which simplifies setting them up and maintaining them.

For example, the following file |draft.tex|
with a compilation flag |\version| as described in \secref{sec:flags}
compiles the main document as a draft:
%
\begin{center}
\begin{tabular}{l}
|\def\version{draft}|\\
|\input{childdoc.def}|\\
|\childdocforward{|\textit{main}|}|
\end{tabular}
\end{center}
%
Likewise, the following files |final|\textit{nn}|.tex|
compile the final version of the child document
|child|\textit{nn}|.tex|:
%
\begin{center}
\begin{tabular}{l}
|\def\version{final}|\\
|\input{childdoc.def}|\\
|\childdocforwardprefix{final}{child}|
\end{tabular}
\end{center}
%

Note that when several versions of a main file and/or of each child file
are to be generated, it may be convenient to set up a |Makefile| or
shell script to automatise the process.

%%%%%%%%%%%%%%%%%%%%%%%%%%%%%%%%%%%%%%%%%%%%%%%%%%%%%%%%%%%%%%%%%%%%%%%%%%%%%%%%
\subsection{Command Line Processing}
\label{sec:commandline}

The effect of redirection files can also be achieved by invoking
the \LaTeX{} compiler with a more elaborate command line.
Most conveniently this should be done as part
of a shell script or a |Makefile|.

When using \textsf{childdoc} in the main file, the following
command lines effectively perform a redirection
(note that depending on the shell being used,
backslashes may have to be doubled: `|\|' $\to$ `|\\|'):
%
\begin{center}
|... -jobname "|\textit{target}|" |\\|"|[\textit{flags}]%
|\input{childdoc.def}\childdocforward[|\textit{main}|]{|\textit{dest}|}"|
\end{center}
%
Here \textit{target} is the name of the output file,
\textit{main} is the name of the main file
and \textit{dest} is the name of the main or child file to be processed
(all filenames without extensions).
The optional argument \textit{main} can be omitted
if \textit{main} matches \textit{dest}.
Optionally, compilation \textit{flags} can be defined via |\def| commands.
This command line makes the \TeX{} engine believe
it is compiling the file \textit{target}
whose content is specified as the latter parameter.
The provided code then forwards the processing to
\textit{main} or \textit{dest} as described in \secref{sec:forward}.

%%%%%%%%%%%%%%%%%%%%%%%%%%%%%%%%%%%%%%%%%%%%%%%%%%%%%%%%%%%%%%%%%%%%%%%%%%%%%%%%
\subsection{Include by Input}
\label{sec:input}

Including child documents by |\include| has some restrictions by design.
Most notably, the content of a child document always occupies
its own set of pages; pages cannot be shared between child documents.
Usually, this behaviour makes perfect sense
because each child document contain an essential part of the document.
However, in some situations it may be desirable to compose
a document from a collection of parts
without having mandatory page breaks between then.
For this case, the package
provides a mechanism to include parts
by |\input| which can also be processed individually.
However, by construction this mechanism
requires manual handling of the content to be output.

%%%%%%%%%%%%%%%%%%%%%%%%%%%%%%%%%%%%%%%%
\DescribeMacro{\ifchilddocmanual}
The main file should be prepared as usual, see \secref{sec:include}.
However, the document body must make a distinction
between processing of an individual part and of the main document, e.g.:
%
\begin{center}
\begin{tabular}{l}
|\ifchilddocmanual|\\
|\input{\childdocname}|\\
|\||else|\\
\textit{document body with }|\input{|\textit{part}|}|\\
|\||fi|
\end{tabular}
\end{center}
%
The conditional |\ifchilddocmanual| is true whenever
a part to be included by |\input| is being compiled,
and the name of the part is stored in |\childdocname|.

%%%%%%%%%%%%%%%%%%%%%%%%%%%%%%%%%%%%%%%%
\DescribeMacro{\childdocby}
Each part to be included by |\input| should start with:
%
\begin{center}
\begin{tabular}{l}
|\input{childdoc.def}|\\
|\childdocby{|\textit{main}|}|\\
\end{tabular}
\end{center}
%
The directive |\childdocby| is similar to |\childdocof|
described in \secref{sec:include},
but the subsequent selection of content must be done manually.
To that end, both |\ifchilddoc| and |\ifchilddocmanual|
will be true upon processing of a part,
and the name of the part is stored in |\childdocname|.
Note that |\jobname| will be set to the filename of the current part
so that each part receives an individual |.aux| file
that does not interfere with the |.aux| file(s) of the main document.
This behaviour can be altered by the alternative form
|\childdocby[*]{|\textit{main}|}| (with a non-empty optional argument)
which uses the |.aux| file of the main document
by setting |\jobname| to \textit{main}.

%%%%%%%%%%%%%%%%%%%%%%%%%%%%%%%%%%%%%%%%%%%%%%%%%%%%%%%%%%%%%%%%%%%%%%%%%%%%%%%%
\subsection{Driver Development}
\label{sec:driver}

The \textsf{childdoc} mechanism can also be use for the development
of definition files such as \LaTeX{} styles or classes.
This case differs from the above setup with multiple parts
included by |\include| in that no |\includeonly| should be invoked.
This can be achieved by starting the include file
(before |\ProvidesPackage|) with:
%
\begin{center}
\begin{tabular}{l}
|\input{childdoc.def}|\\
|\childdocforward{|\textit{main}|}|\\
\end{tabular}
\end{center}
%
or alternatively with:
%
\begin{center}
\begin{tabular}{l}
|\input{childdoc.def}|\\
|\childdocby{|\textit{main}|}|\\
\end{tabular}
\end{center}
%
Both forms have slightly different effects as described above.
The main file is prepared as usual, see \secref{sec:include}.

%%%%%%%%%%%%%%%%%%%%%%%%%%%%%%%%%%%%%%%%%%%%%%%%%%%%%%%%%%%%%%%%%%%%%%%%%%%%%%%%
\subsection{Legacy Detection}
\label{sec:detection}

The directive |\childdocmain| in the main file can detect
whether the complete document or merely a child is to be compiled
even without using the directive |\childdocof|.
This method is deprecated because it is less robust
and there is no compelling reason to use it;
it is merely provided for backward compatibility
and it may be removed in future versions.

If the detection mechanism is to be used,
it is mandatory to correctly specify
the filename of the main file as the argument of |\childdocmain|:
%
\begin{center}
\begin{tabular}{l}
|\input{childdoc.def}|\\
|\childdocmain{|\textit{main}|}|\\
\end{tabular}
\end{center}
%
If |\jobname| does not match the argument \textit{main} of |\childdocmain|,
it is assumed that |\jobname| points to the child file to be compiled.
When using |\childdocmain| with the main file specified as argument,
it suffices to start a child file
with just |\input{|\textit{main}|}|
without loading of the package and using |\childdocof|.
If instead all processing is done
with the appropriate \textsf{childdoc} directives,
the argument of \textit{main} of |\childdocmain| can be empty.

An alternative version of the command line processing described
in \secref{sec:commandline} using the detection mechanism reads:
%
\begin{center}
|... -jobname "|\textit{target}|" "|[\textit{flags}]%
[|\def\jobname{|\textit{dest}|}|]|\input{|\textit{main}|}"|
\end{center}

%%%%%%%%%%%%%%%%%%%%%%%%%%%%%%%%%%%%%%%%%%%%%%%%%%%%%%%%%%%%%%%%%%%%%%%%%%%%%%%%
\subsection{Manual Code}
\label{sec:manual}

In case one cannot be certain whether the definitions file |childdoc.def|
is installed on the target \TeX{} distribution
and one prefers not to ship it,
it is conceivable to paste a few relevant commands into the sources.

To that end, drop all statements |\input{childdoc.def}|
and perform the replacements as outlined below.
Instead of |\childdocmain{|\textit{main}|}| add the following code
to the top of the main file:
%
\begin{center}
\begin{tabular}{l}
|\||ifdefined\childdocname\endinput\||fi\newif\ifchilddoc|\\
|\edef\childdocname{\scantokens\expandafter{\jobname\noexpand}}|\\
|\def\childdocmain{|\textit{main}|}\||ifx\childdocmain\childdocname\||else|\\
|\childdoctrue\includeonly{\childdocname}\let\jobname\childdocmain\||fi|\\
\end{tabular}
\end{center}
%
Instead of |\childdocof{|\textit{main}|}| just include the main file
at the top of each child file:
%
\begin{center}
|\input{|\textit{main}|}|
\end{center}
%
A simple redirection |\childdocforward{|\textit{dest}|}| is achieved by:
%
\begin{center}
|\def\jobname{|\textit{dest}|}\input{\jobname}|
\end{center}
%
The redirection with prefix
|\childdocforwardprefix[|\textit{prefix}|]{|\textit{dest}|}|
is accomplished by:
%
\begin{center}
\begin{tabular}{l}
|{\edef\jobname{\scantokens\expandafter{\jobname\noexpand}}|\\
|\def\redirectjob |\textit{prefix}|#1~~~{\gdef\jobname{|\textit{dest}|#1}}|\\
|\expandafter\redirectjob\jobname~~~}\input{\jobname}|
\end{tabular}
\end{center}

In an alternative approach,
child documents can be compiled by a specific command line
without additional code or specific definitions:
%
\begin{center}
|... -jobname "|\textit{target}|" "|[\textit{flags}]%
|\includeonly{|\textit{dest}|}\input{|\textit{main}|}"|
\end{center}
%

%%%%%%%%%%%%%%%%%%%%%%%%%%%%%%%%%%%%%%%%%%%%%%%%%%%%%%%%%%%%%%%%%%%%%%%%%%%%%%%%
%%%%%%%%%%%%%%%%%%%%%%%%%%%%%%%%%%%%%%%%%%%%%%%%%%%%%%%%%%%%%%%%%%%%%%%%%%%%%%%%
\section{Information}

%%%%%%%%%%%%%%%%%%%%%%%%%%%%%%%%%%%%%%%%%%%%%%%%%%%%%%%%%%%%%%%%%%%%%%%%%%%%%%%%
\subsection{Copyright}

Copyright \copyright{} 2017--2018 Niklas Beisert

This work may be distributed and/or modified under the
conditions of the \LaTeX{} Project Public License, either version 1.3
of this license or (at your option) any later version.
The latest version of this license is in
  \url{http://www.latex-project.org/lppl.txt}
and version 1.3 or later is part of all distributions of \LaTeX{}
version 2005/12/01 or later.

This work has the LPPL maintenance status `maintained'.

The Current Maintainer of this work is Niklas Beisert.

This work consists of the files |README.txt|, |childdoc.ins| and |childdoc.dtx|
as well as the derived files |childdoc.def|, |cdocsamp.tex|
with |cdocsch1.tex|, |cdocsch2.tex|, |cdocspt3.tex|, |cdocspt4.tex|,
|cdocsdrf.tex|, |cdocsfn1.tex|, |cdocsfn2.tex|
as well as |childdoc.pdf|.

%%%%%%%%%%%%%%%%%%%%%%%%%%%%%%%%%%%%%%%%%%%%%%%%%%%%%%%%%%%%%%%%%%%%%%%%%%%%%%%%
\subsection{Files and Installation}

The package consists of the files:
%
\begin{center}
\begin{tabular}{ll}
    |README.txt|   & readme file \\
    |childdoc.ins| & installation file \\
    |childdoc.dtx| & source file \\
    |childdoc.def| & definition file \\
    |cdocsamp.tex| & sample main file \\
    |cdocsch1.tex| & sample include file \\
    |cdocsch2.tex| & sample include file \\
    |cdocspt3.tex| & sample part file \\
    |cdocspt4.tex| & sample part file \\
    |cdocsdrf.tex| & sample redirection file \\
    |cdocsfn1.tex| & sample redirection file \\
    |cdocsfn2.tex| & sample redirection file \\
    |childdoc.pdf| & manual
\end{tabular}
\end{center}
%
The distribution consists of the files
|README.txt|, |childdoc.ins| and |childdoc.dtx|.
%
\begin{itemize}
\item
Run (pdf)\LaTeX{} on |childdoc.dtx|
to compile the manual |childdoc.pdf| (this file).
\item
Run \LaTeX{} on |childdoc.ins| to create the definitions file |childdoc.def|
and the sample |cdocsamp.tex| with include files
|cdocsch1.tex|, |cdocsch2.tex|, |cdocspt3.tex|, |cdocspt4.tex|,
|cdocsdrf.tex|, |cdocsfn1.tex|, |cdocsfn2.tex|.
Then copy the file |childdoc.def| to an appropriate directory of your \LaTeX{}
distribution, e.g.\ \textit{texmf-root}|/tex/latex/childdoc|.
\end{itemize}

%%%%%%%%%%%%%%%%%%%%%%%%%%%%%%%%%%%%%%%%%%%%%%%%%%%%%%%%%%%%%%%%%%%%%%%%%%%%%%%%
\subsection{Related CTAN Packages}

There are several other packages which offer a similar functionality:
%
\begin{itemize}
\item
The packages
\href{http://ctan.org/pkg/docmute}{\textsf{docmute}},
\href{http://ctan.org/pkg/includex}{\textsf{includex}} and
\href{http://ctan.org/pkg/standalone}{\textsf{standalone}}
provide commands to include only the document body of
a child file thus allowing both files to be compiled individually.
\item
The packages \href{http://ctan.org/pkg/subdocs}{\textsf{subdocs}}
and \href{http://ctan.org/pkg/subfiles}{\textsf{subfiles}}
provide structures in which the main and child documents can be
encapsulated and allowing them to be compiled individually.
The inclusion mechanism is different from the conventional |\include|.
\item
The package \href{http://ctan.org/pkg/combine}{\textsf{combine}}
is an elaborate solution to combine several documents into one.
\end{itemize}
%
See also the CTAN topic \href{http://ctan.org/topic/subdocs}{\textsf{subdocs}}
for further related packages.
The present package differs from the above solutions in that
a document structure constructed with the conventional |\include| mechanism
just needs two extra commands at the top of every file
such that all constituent files can be compiled individually.

%%%%%%%%%%%%%%%%%%%%%%%%%%%%%%%%%%%%%%%%%%%%%%%%%%%%%%%%%%%%%%%%%%%%%%%%%%%%%%%%
%\subsection{Feature Suggestions}
%
%The following is a list of features which may be useful for future
%versions of this package:
%%
%\begin{itemize}
%\item
%\ldots
%\end{itemize}

%%%%%%%%%%%%%%%%%%%%%%%%%%%%%%%%%%%%%%%%%%%%%%%%%%%%%%%%%%%%%%%%%%%%%%%%%%%%%%%%
\subsection{Revision History}

%%%%%%%%%%%%%%%%%%%%%%%%%%%%%%%%%%%%%%%%
\paragraph{v2.0:} 2018/12/30

\begin{itemize}
\item
immediate forward processing
\item
added |\childdocby| mechanism
\item
manual restructured
\end{itemize}

%%%%%%%%%%%%%%%%%%%%%%%%%%%%%%%%%%%%%%%%
\paragraph{v1.6:} 2018/01/17

\begin{itemize}
\item
application for development of include files
\item
corrections to manual
\end{itemize}

%%%%%%%%%%%%%%%%%%%%%%%%%%%%%%%%%%%%%%%%
\paragraph{v1.5:} 2017/05/21

\begin{itemize}
\item
more complete structuring introduced
\item
|\childdocof| introduced
\item
|\childdoc| renamed to |\childdocmain|
\item
|\childredirect| renamed to |\childdocforward| and |\childdocforwardprefix|
and functionality expanded
\end{itemize}

%%%%%%%%%%%%%%%%%%%%%%%%%%%%%%%%%%%%%%%%
\paragraph{v1.0:} 2017/04/27

\begin{itemize}
\item
manual and install package
\item
first version published on CTAN
\end{itemize}

%%%%%%%%%%%%%%%%%%%%%%%%%%%%%%%%%%%%%%%%
\paragraph{v0.6:} 2017/04/26

\begin{itemize}
\item
redirection mechanism added
\end{itemize}

%%%%%%%%%%%%%%%%%%%%%%%%%%%%%%%%%%%%%%%%
\paragraph{v0.5:} 2017/04/26

\begin{itemize}
\item
functionality in definition file
\end{itemize}


%%%%%%%%%%%%%%%%%%%%%%%%%%%%%%%%%%%%%%%%%%%%%%%%%%%%%%%%%%%%%%%%%%%%%%%%%%%%%%%%
%%%%%%%%%%%%%%%%%%%%%%%%%%%%%%%%%%%%%%%%%%%%%%%%%%%%%%%%%%%%%%%%%%%%%%%%%%%%%%%%
%%%%%%%%%%%%%%%%%%%%%%%%%%%%%%%%%%%%%%%%%%%%%%%%%%%%%%%%%%%%%%%%%%%%%%%%%%%%%%%%
\appendix

\settowidth\MacroIndent{\rmfamily\scriptsize 000\ }

 \DocInput{childdoc.dtx}

\end{document}
%</driver>
% \fi
%
% %%%%%%%%%%%%%%%%%%%%%%%%%%%%%%%%%%%%%%%%%%%%%%%%%%%%%%%%%%%%%%%%%%%%%%%%%%%%%%
% %%%%%%%%%%%%%%%%%%%%%%%%%%%%%%%%%%%%%%%%%%%%%%%%%%%%%%%%%%%%%%%%%%%%%%%%%%%%%%
% \section{Sample}
%\iffalse
%<*samplemain>
%\fi
%
% The following presents a sample document
% with two chapters, two parts, a title page,
% a compile flag as well as three forwarding files to set the flag.
% It consists of eight |.tex| files:
% \begin{center}
% \begin{tabular}{ll}
% |cdocsamp.tex|&main file\\
% |cdocsch1.tex|&include file for chapter 1\\
% |cdocsch2.tex|&include file for chapter 2\\
% |cdocspt3.tex|&include file for part 3\\
% |cdocspt4.tex|&include file for part 4\\
% |cdocsdrf.tex|&forwarding file for main file in draft mode\\
% |cdocsfi1.tex|&forwarding file for final version of chapter 1\\
% |cdocsfi2.tex|&forwarding file for final version of chapter 2\\
% \end{tabular}
% \end{center}
% Each of the eight files can be compiled directly by the \LaTeX{} compiler.
%
% %%%%%%%%%%%%%%%%%%%%%%%%%%%%%%%%%%%%%%
% \paragraph{Main File.}
%
% The main file is called |cdocsamp.tex|.
%
% Load the \textsf{childdoc} definitions and
% declare the filename for the main document:
%    \begin{macrocode}
\input{childdoc.def}
\childdocmain{}
%    \end{macrocode}

% Optional override for |\version| flag:
%    \begin{macrocode}
%%\ifchilddoc\else\providecommand{\version}{draft}\fi
%    \end{macrocode}

% Define the default values for the |\version| flag
% (|final| for the main file and |draft| for childs):
%    \begin{macrocode}
\ifchilddoc
\providecommand{\version}{draft}
\else
\providecommand{\version}{final}
\fi
%    \end{macrocode}

% Load the standard document class:
%    \begin{macrocode}
\documentclass[12pt]{article}
%    \end{macrocode}

% Start the document body:
%    \begin{macrocode}
\begin{document}
%    \end{macrocode}

% Declare a title page.
% Print title, part of document being processed and version flag:
%    \begin{macrocode}
\addtocounter{page}{-1}
\begin{center}
{\LARGE\bfseries{}childdoc example\par}
\vspace{1cm}
\ifchilddoc
\ifchilddocmanual part\else chapter\fi:
`\childdocname' of `\childdocjob'\par
\else
main document: `\childdocjob'\par
\fi
version: \version\par
\end{center}
\newpage
%    \end{macrocode}

% Manually include selected file,
% otherwise process as usual:
%    \begin{macrocode}
\ifchilddocmanual
\section*{part `\childdocname'}
\input{\childdocname}
\else
%    \end{macrocode}

% Include the two chapters:
%    \begin{macrocode}
\include{cdocsch1}
\include{cdocsch2}
%    \end{macrocode}

% Include the two parts unless only chapters should be displayed:
%    \begin{macrocode}
\ifchilddoc\else
\section{part three}
\input{cdocspt3}
\section{part four}
\input{cdocspt4}
\fi
%    \end{macrocode}

% Process as usual until here:
%    \begin{macrocode}
\fi
%    \end{macrocode}

% End of document body:
%    \begin{macrocode}
\end{document}
%    \end{macrocode}
%\iffalse
%</samplemain>
%\fi
%
% %%%%%%%%%%%%%%%%%%%%%%%%%%%%%%%%%%%%%%
% \paragraph{Chapter Include Files.}
%
% The include files are called |cdocsch1.tex| and |cdocsch2.tex|.
%
%\iffalse
%<*samplechap1|samplechap2>
%\fi

% Optional override for |\version| flag:
%    \begin{macrocode}
%%\providecommand{\version}{final}
%    \end{macrocode}

% Include the main document:
%    \begin{macrocode}
\input{childdoc.def}
\childdocof{cdocsamp}
%    \end{macrocode}

%\iffalse
%</samplechap1|samplechap2>
%\fi
%
%\iffalse
%<*samplechap1>
%\fi
% Some text for chapter 1:
%    \begin{macrocode}
\section{one}
some text in chapter one
%    \end{macrocode}

%\iffalse
%</samplechap1>
%\fi
% Some text for chapter 2:
%\iffalse
%<*samplechap2>
%\fi
%    \begin{macrocode}
\section{two}
more text in chapter two
%    \end{macrocode}

%\iffalse
%</samplechap2>
%\fi
%
% %%%%%%%%%%%%%%%%%%%%%%%%%%%%%%%%%%%%%%
% \paragraph{Part Include Files.}
%
% The include files are called |cdocspt3.tex| and |cdocspt4.tex|.
%
%\iffalse
%<*samplepart3|samplepart4>
%\fi

% Optional override for |\version| flag:
%    \begin{macrocode}
%%\providecommand{\version}{final}
%    \end{macrocode}

% Include the main document:
%    \begin{macrocode}
\input{childdoc.def}
\childdocby{cdocsamp}
%    \end{macrocode}

%\iffalse
%</samplepart3|samplepart4>
%\fi
%
%\iffalse
%<*samplepart3>
%\fi
% Some text for part 3:
%    \begin{macrocode}
some text in part three
%    \end{macrocode}

%\iffalse
%</samplepart3>
%\fi
% Some text for part 4:
%\iffalse
%<*samplepart4>
%\fi
%    \begin{macrocode}
more text in part four
%    \end{macrocode}

%\iffalse
%</samplepart4>
%\fi
%
% %%%%%%%%%%%%%%%%%%%%%%%%%%%%%%%%%%%%%%
% \paragraph{Forwarding for a Complete Draft.}
%
% The following forwarding file |cdocsdrf.tex|
% compiles the main document in draft mode:
%\iffalse
%<*sampledraft>
%\fi
%    \begin{macrocode}
\def\version{draft}
\input{childdoc.def}
\childdocforward{cdocsamp}
%    \end{macrocode}

%\iffalse
%</sampledraft>
%\fi
%
% %%%%%%%%%%%%%%%%%%%%%%%%%%%%%%%%%%%%%%
% \paragraph{Forwarding for Final Version of the Chapters.}
%
% The following forwarding files |cdocsfn1.tex| and |cdocsfn2.tex|
% (with identical content)
% compile the final versions of the child documents
% |cdocsch1.tex| and |cdocsch2.tex|, respectively:
%\iffalse
%<*samplefinal>
%\fi
%    \begin{macrocode}
\def\version{final}
\input{childdoc.def}
\childdocforwardprefix[cdocsamp]{cdocsfn}{cdocsch}
%    \end{macrocode}

%\iffalse
%</samplefinal>
%\fi
%
% %%%%%%%%%%%%%%%%%%%%%%%%%%%%%%%%%%%%%%
% \paragraph{Command Line Processing.}
%
% The following three command lines generate the output files
% |cdocscld|, |cdocscl1| and |cdocscl2|
% which should be identical to
% |cdocsdrf|, |cdocsch1| and |cdocsfn2|, respectively:
% \begin{center}
% \begin{tabular}{l}
% |latex -jobname cdocscld \|\\
% |  "\def\version{draft}\input{childdoc.def}\childdocforward{cdocsamp}"|\\
% |latex -jobname cdocscl1 \|\\
% |  "\input{childdoc.def}\childdocforward[cdocsamp]{cdocsch1}"|\\
% |latex -jobname cdocscl2 \|\\
% |  "\def\version{final}\input{childdoc.def}\childdocforward{cdocsch2}"|
% \end{tabular}
% \end{center}
% Note that the trailing backslash on each first line
% merely continues the input to the second line
% (for convenient cut ant paste).
% Furthermore, the command |latex| can be replaced by any
% of its alternative versions such as |pdflatex|.
%
% %%%%%%%%%%%%%%%%%%%%%%%%%%%%%%%%%%%%%%%%%%%%%%%%%%%%%%%%%%%%%%%%%%%%%%%%%%%%%%
% %%%%%%%%%%%%%%%%%%%%%%%%%%%%%%%%%%%%%%%%%%%%%%%%%%%%%%%%%%%%%%%%%%%%%%%%%%%%%%
% \section{Implementation}
%\iffalse
%<*package>
%\fi
%
% This section describes the definitions file |childdoc.def|.

% The definitions cannot be loaded using |\usepackage| or |\RequirePackage|
% which has a mechanism to prevent loading a style file more than once.
% When loading the definitions by means of |\input|
% multiple instances have to be prevented manually:
%\iffalse
%This code needs to be before the `\ProvidesFile' directive
%which is defined at the beginning of this file.
%Therefore it is also placed there and commented out here.
%</package>
%<*discard>
%\fi
%    \begin{macrocode}
\ifdefined\childdocmain\endinput\fi
%    \end{macrocode}
%\iffalse
%</discard>
%<*package>
%\fi
%
% \macro{\ifchilddoc}
% \macro{\ifchilddocmanual}
% The conditional |\ifchilddoc| tells whether a
% child (true) or main (false) document is being compiled.
% The conditional |\ifchilddocmanual| tells whether
% the |\includeonly| mechanism is used (false) or
% the selection of child files must be performed manually (true).
% The definitions initialise to false:
%    \begin{macrocode}
\newif\ifchilddoc
\newif\ifchilddocmanual
%    \end{macrocode}

% \macro{\childdocname}
% \macro{\childdocjob}
% The macro |\childdocname| stores the name of the main document
% to be compiled. The macro |\childdocjob| stores the name of
% the document on which the \LaTeX{} compiler was originally invoked.
% The content of |\jobname| cannot be compared
% to filenames specified in the source due to different catcodes.
% The following code rescans |\jobname|, stores the result
% in |\childdocname| and saves a copy in |\childdocjob|:
%    \begin{macrocode}
\edef\childdocname{\scantokens\expandafter{\jobname\noexpand}}
\let\childdocjob\childdocname
%    \end{macrocode}

% \macro{\childdocdisable}
% The macro |\childdocdisable| prevents the main file
% from being processed more than once.
% At this stage, the main document command |\childdocmain|
% is assumed to be called once again where it should do nothing.
% Any subsequent call to it should prevent
% a secondary processing of the main document
% It overwrites the forwarding commands
% |\childdocof| and |\childdocforward|
% with empty macros to prevent further inclusions of the main document:
%    \begin{macrocode}
\newcommand{\childdocdisable}
{
  \renewcommand{\childdocmain}[1]{\renewcommand{\childdocmain}[1]{\endinput}}
  \renewcommand{\childdocof}[1]{}
  \renewcommand{\childdocby}[2][]{}
  \renewcommand{\childdocforward}[2][]{}
  \renewcommand{\childdocdisable}{}
}
%    \end{macrocode}

% \macro{\childdocmain}
% The macro |\childdocmain| is to be called at the top of the main file
% with nothing or the main filename (without extension) as argument.
% First, it breaks loops.
% If the argument is not empty and does not match |\childdocname|
% (which is set by the first inclusion of |childdoc.def|),
% |\ifchilddoc| is set to true, |\includeonly| is applied to the child file
% and |\jobname| is set to the main file
% (for proper handling of |.aux| files):
%    \begin{macrocode}
\newcommand{\childdocmain}[1]
{
  \childdocdisable\childdocmain{}
  \if?#1?\else
    \begingroup
      \def\childdoctmp{#1}
      \ifx\childdoctmp\childdocname
        \def\childdoctmp{}
      \else
        \def\childdoctmp
        {
          \childdoctrue
          \includeonly{\childdocname}
          \def\childdocjob{#1}
          \def\jobname{#1}
        }
      \fi
      \expandafter
    \endgroup
    \childdoctmp
  \fi
}
%    \end{macrocode}

% \macro{\childdocof}
% The command |\childdocof| redirects
% compilation to the main file |#1|.
%    \begin{macrocode}
\newcommand{\childdocof}[1]
{
  \childdocdisable
  \childdoctrue
  \includeonly{\childdocname}
  \def\jobname{#1}
  \def\childdocjob{#1}
  \input{#1}
}
%    \end{macrocode}

% \macro{\childdocby}
% The command |\childdocby| ....
%    \begin{macrocode}
\newcommand{\childdocby}[2][]
{
  \childdocdisable
  \childdoctrue
  \childdocmanualtrue
  \if?#1?\else
    \def\jobname{#2}
  \fi
  \def\childdocjob{#2}
  \input{#2}
  \endinput
}
%    \end{macrocode}

% \macro{\childdocforward}
% The command |\childdocforward| redirects
% compilation to the main file or
% (if the optional argument is given) a child file.
% Parameters are set as if the main file
% or a child file starting with |\childdocof| was compiled.
% Then compilation is handed over to the main file:
%    \begin{macrocode}
\newcommand{\childdocforward}[2][]
{
  \begingroup
    \if?#1?
      \def\childdoctmp
      {
        \def\childdocname{#2}
        \def\childdocjob{#2}
        \def\jobname{#2}
        \input{#2}
        \endinput
      }
    \else
      \def\childdoctmp
      {
        \childdocdisable
        \def\childdocname{#2}
        \childdoctrue
        \includeonly{#2}
        \def\childdocjob{#1}
        \def\jobname{#1}
        \input{#1}
        \endinput
      }
    \fi
    \expandafter
  \endgroup
  \childdoctmp
}
%    \end{macrocode}

% \macro{\childdocforwardprefix}
% The command |\childdocforwardprefix| redirects
% compilation to the main or a child file by means of a pattern.
% The prefix |#1| in the current filename is replaced by |#2|
% and the suffix of the current filename is kept
% (it is assumed that the filename does not contain the substring `|~~~|'
% which is used as a delimiter).
% Compilation is handed over to the new file by |\childdocforward|:
%    \begin{macrocode}
\newcommand{\childdocforwardprefix}[3][]
{
  \begingroup
    \def\childdocextract #2##1~~~{\def\childdoctmp{\childdocforward[#1]{#3##1}}}
    \expandafter\childdocextract\childdocname~~~
    \expandafter
  \endgroup
  \childdoctmp
}
%    \end{macrocode}

% \macro{\childdoc}
% The deprecated macro |\childdoc| is a legacy version of |\childdocmain|:
%    \begin{macrocode}
\newcommand{\childdoc}{\childdocmain}
%    \end{macrocode}

% \macro{\childdocredirect}
% The deprecated macro |\childdocredirect| is a legacy version
% of |\childdocforward| and |\childdocforwardprefix|:
%    \begin{macrocode}
\newcommand{\childdocredirect}[2][]
{
  \begingroup
    \if?#1?
      \def\childdoctmp{\childdocforward{#2}}
    \else
      \def\childdoctmp{\childdocforwardprefix{#1}{#2}}
    \fi
    \expandafter
  \endgroup
  \childdoctmp
}
%    \end{macrocode}

%\iffalse
%</package>
%\fi
%
\endinput

\childdocforwardprefix[cdocsamp]{cdocsfn}{cdocsch}
%    \end{macrocode}

%\iffalse
%</samplefinal>
%\fi
%
% %%%%%%%%%%%%%%%%%%%%%%%%%%%%%%%%%%%%%%
% \paragraph{Command Line Processing.}
%
% The following three command lines generate the output files
% |cdocscld|, |cdocscl1| and |cdocscl2|
% which should be identical to
% |cdocsdrf|, |cdocsch1| and |cdocsfn2|, respectively:
% \begin{center}
% \begin{tabular}{l}
% |latex -jobname cdocscld \|\\
% |  "\def\version{draft}% \iffalse
%
% childdoc.dtx Copyright (C) 2017-2018 Niklas Beisert
%
% This work may be distributed and/or modified under the
% conditions of the LaTeX Project Public License, either version 1.3
% of this license or (at your option) any later version.
% The latest version of this license is in
%   http://www.latex-project.org/lppl.txt
% and version 1.3 or later is part of all distributions of LaTeX
% version 2005/12/01 or later.
%
% This work has the LPPL maintenance status `maintained'.
%
% The Current Maintainer of this work is Niklas Beisert.
%
% This work consists of the files childdoc.dtx and childdoc.ins
% and the derived files childdoc.def and cdocsamp.tex with
% cdocsch1.tex, cdocsch2.tex, cdocsdrf.tex, cdocsfn1.tex, cdocsfn2.tex.
%
%<package>\ifdefined\childdocmain\endinput\fi
%<package>\ProvidesFile{childdoc.def}[2018/12/30 v2.0 child document driver]
%<samplemain>\ProvidesFile{cdocsamp.tex}[2018/12/30 v2.0 sample for childdoc]
%<*driver>
%\ProvidesFile{childdoc.drv}[2018/12/30 v2.0 childdoc reference manual file]
\PassOptionsToClass{10pt,a4paper}{article}
\documentclass{ltxdoc}

\usepackage[margin=35mm]{geometry}
\usepackage{hyperref}
\usepackage{hyperxmp}
\usepackage[usenames]{color}

\hypersetup{colorlinks=true}
\hypersetup{pdfstartview=FitH}
\hypersetup{pdfpagemode=UseNone}
\hypersetup{pdfsource={}}
\hypersetup{pdflang={en-UK}}
\hypersetup{pdfcopyright={Copyright 2017-2018 Niklas Beisert.
  This work may be distributed and/or modified under the
  conditions of the LaTeX Project Public License, either version 1.3
  of this license or (at your option) any later version.}}
\hypersetup{pdflicenseurl={http://www.latex-project.org/lppl.txt}}
\hypersetup{pdfcontactaddress={ETH Zurich, ITP, HIT K,
  Wolfgang-Pauli-Strasse 27}}
\hypersetup{pdfcontactpostcode={8093}}
\hypersetup{pdfcontactcity={Zurich}}
\hypersetup{pdfcontactcountry={Switzerland}}
\hypersetup{pdfcontactemail={nbeisert@itp.phys.ethz.ch}}
\hypersetup{pdfcontacturl={http://people.phys.ethz.ch/\xmptilde nbeisert/}}

\newcommand{\secref}[1]{\hyperref[#1]{section \ref*{#1}}}

\parskip1ex
\parindent0pt
\let\olditemize\itemize
\def\itemize{\olditemize\parskip0pt}

\begin{document}

\title{The \textsf{childdoc} Package}
\hypersetup{pdftitle={The childdoc Package}}
\author{Niklas Beisert\\[2ex]
  Institut f\"ur Theoretische Physik\\
  Eidgen\"ossische Technische Hochschule Z\"urich\\
  Wolfgang-Pauli-Strasse 27, 8093 Z\"urich, Switzerland\\[1ex]
  \href{mailto:nbeisert@itp.phys.ethz.ch}
  {\texttt{nbeisert@itp.phys.ethz.ch}}}
\hypersetup{pdfauthor={Niklas Beisert}}
\hypersetup{pdfsubject={Manual for the LaTeX2e Package childdoc}}
\date{30 December 2018, \textsf{v2.0}}
\maketitle

\begin{abstract}\noindent
\textsf{childdoc} is a \LaTeXe{} package
that enables the direct compilation
of document sections included by |\include|
to individual files.
\end{abstract}

\begingroup
\parskip0ex
\tableofcontents
\endgroup

%%%%%%%%%%%%%%%%%%%%%%%%%%%%%%%%%%%%%%%%%%%%%%%%%%%%%%%%%%%%%%%%%%%%%%%%%%%%%%%%
%%%%%%%%%%%%%%%%%%%%%%%%%%%%%%%%%%%%%%%%%%%%%%%%%%%%%%%%%%%%%%%%%%%%%%%%%%%%%%%%
\section{Introduction}

\LaTeX{} provides a mechanism to structure a large document (such as a book)
into a main file and several child files (containing the chapters)
using the |\include| command.
This mechanism is beneficial for documents
which span hundreds of pages in order to
make the source file(s) more manageable.
Moreover, compilation can be restricted to
selected child files by means of the |\includeonly| command.
The latter feature can be used to reduce the compilation time while editing
(this was significantly more useful in the earlier days of \LaTeX{})
or to generate a smaller document which is easier to navigate.
Another application of |\includeonly| is to generate
documents consisting of selected parts of the complete document.

However, there are a few drawbacks of the plain |\include| mechanism:
\begin{itemize}
\item
The child files cannot be compiled on their own,
they can only be compiled via the main file.
A naive editing environment
(such as a text editor with an option
to have the current file processed by \LaTeX)
may require one to switch to the main file before compiling;
attempting to compile the child file produces errors.
\item
The main file must be modified (each time)
to adjust the |\includeonly| command
to the present needs. This easily leaves the main file in a messy state.
\item
The generated document will always carry the filename
of the main document. This is inconvenient if
several child files are to be compiled and
to be kept for distribution.
\end{itemize}

The present package provides a simple interface
to make child files individually compilable by \LaTeX{}.
Compiling a child file then has the same effect as compiling
the main file with an |\includeonly| command
to select the appropriate child.
Moreover the generated document will carry the name of the child
rather than the main file.
This resolves all three above issues.

This feature is meant to make the editing of books,
thesis documents and lecture notes somewhat more convenient.
However, the package can also be used efficiently for
composing a series of documents (such as exercise sheets)
which are typically distributed individually.
It then assists the author in generating the individual documents
(potentially in different versions)
as well as a document containing the collected series.
Another application is in developing style files
or other kinds of included material
where compilation of the style file could redirect
to a sample or test file.

%%%%%%%%%%%%%%%%%%%%%%%%%%%%%%%%%%%%%%%%%%%%%%%%%%%%%%%%%%%%%%%%%%%%%%%%%%%%%%%%
%%%%%%%%%%%%%%%%%%%%%%%%%%%%%%%%%%%%%%%%%%%%%%%%%%%%%%%%%%%%%%%%%%%%%%%%%%%%%%%%
\section{Usage}

First of all, the package \textsf{childdoc} is \emph{not} a standard
\LaTeXe{} |.sty| style file! Therefore it needs to be invoked in
a non-standard way.

%%%%%%%%%%%%%%%%%%%%%%%%%%%%%%%%%%%%%%%%%%%%%%%%%%%%%%%%%%%%%%%%%%%%%%%%%%%%%%%%
\subsection{Included Files}
\label{sec:include}

%%%%%%%%%%%%%%%%%%%%%%%%%%%%%%%%%%%%%%%%
\DescribeMacro{\childdocmain}
To use the package, add the commands
\begin{center}
\begin{tabular}{l}
|\input{childdoc.def}|\\
|\childdocmain{}|\\
\end{tabular}
\end{center}
at the very top of the main \LaTeX{} file,
in particular \emph{before} the |\documentclass| statement!
The argument of |\childdocmain| should be left empty
(but it must be present).

%%%%%%%%%%%%%%%%%%%%%%%%%%%%%%%%%%%%%%%%
\DescribeMacro{\childdocof}
Furthermore, add the commands
\begin{center}
\begin{tabular}{l}
|\input{childdoc.def}|\\
|\childdocof{|\textit{main}|}|\\
\end{tabular}
\end{center}
at the top of every child file \textit{child}
which is included by |\include{|\textit{child}|}|
from within the main file
(or at least for those files to be compiled individually).
The argument \textit{main} must be the filename of the main file.

There are a couple of
considerations in setting up the main and child documents:

%%%%%%%%%%%%%%%%%%%%%%%%%%%%%%%%%%%%%%%%
\paragraph{Restrictions.}

Please note the following restrictions:
\begin{itemize}
\item
|\childdocmain| must be called with one argument \textit{main}
to ensure compatibility with earlier version of the package.
It must either be empty (|\childdocmain{}|)
or precisely match the filename of the main file in which it is specified.
See \secref{sec:detection} for further information.
\item
The filename \textit{main} must be specified without the |.tex| extension.
\item
The filename \textit{main} is case sensitive
(even in case-insensitive file systems)
due to internal string comparison.
\item
The argument \textit{main} should be fully expanded, it cannot be a macro.
\item
Subdirectories and special characters should be avoided in filenames.
\item
The command |\childdocmain{|\textit{main}|}| must be followed by a whitespace.
It should not be followed immediately by another command
or by a comment mark `|%|'.
This is because the \TeX{} parser reads the token immediately following
the argument of |\childdocmain| and puts it
at the beginning of every child section;
however, a white\-space is ignored.
\end{itemize}

%%%%%%%%%%%%%%%%%%%%%%%%%%%%%%%%%%%%%%%%
\paragraph{Content of Main File.}

It is advisable to place all content in the child files included by |\include|.
Any output contained in the main file will appear in all child documents
unless suppressed manually;
it cannot be suppressed automatically by the |\includeonly| directive
and thus should normally be avoided.
A method to include some content in the main file
by means of conditional processing is described in \secref{sec:conditional}.

%%%%%%%%%%%%%%%%%%%%%%%%%%%%%%%%%%%%%%%%
\paragraph{Page Numbering.}

When only a part of the document is compiled,
the appropriate numbering of pages
(as well as other status parameters)
is determined from the |.aux| files.
The latter contain information from previous passes.
However this information needs to propagate through
all intermediate child documents.
Therefore the page numbering in child documents may well
be inconsistent until the complete document is compiled at least once.

A useful (if unconventional) way to always ensure a consistent
page numbering is to restart the numbering in each child document
and denote the pages by `\textit{child}|.|\textit{page}'
where \textit{child} represents the chapter/section number of the child file.
This can be achieved by the command
|\numberwithin{page}{|\textit{child}|}|
of the \textsf{amsmath} package
where \textit{child} can be |chapter| or |section|
depending on the chosen structuring.
Alternatively, one can modify the macro |\thepage| appropriately
and reset the counter |page| at the start of each child file.

%%%%%%%%%%%%%%%%%%%%%%%%%%%%%%%%%%%%%%%%%%%%%%%%%%%%%%%%%%%%%%%%%%%%%%%%%%%%%%%%
\subsection{Conditional Processing}
\label{sec:conditional}

The package provides a mechanism to compile different versions
of a document. To customise the versions further some conditional processing
can come in handy to distinguish which version is being compiled.
The package provides two macros to describe the compilation context:

%%%%%%%%%%%%%%%%%%%%%%%%%%%%%%%%%%%%%%%%
\DescribeMacro{\ifchilddoc}
The conditional |\ifchilddoc| distinguishes between the compilation of
child documents and the main document:
%
\begin{center}
|\ifchilddoc |\textit{child-code}| |[|\||else |\textit{main-code}]| \||fi|
\end{center}

%%%%%%%%%%%%%%%%%%%%%%%%%%%%%%%%%%%%%%%%
\DescribeMacro{\childdocname}
\DescribeMacro{\childdocjob}
The macro |\childdocname| contains the filename (without extension)
of the main or child file being processed.
Note that |\childdocjob| will always contain the name of the main file.

%%%%%%%%%%%%%%%%%%%%%%%%%%%%%%%%%%%%%%%%
\paragraph{Title Page.}

Conditional processing can be used to include a title or banner page
in the main document when proper precautions are taken.
Importantly, the code in the main file should ensure that the page counter
(as well as other status parameters which are stored in the |.aux| files)
takes the same value after the conditional processing.
Otherwise the page numbers may take divergent values
depending on which part is compiled.

For example, a title page could be declared by:
%
\begin{center}
\begin{tabular}{l}
|\ifchilddoc\||else|\\
|\addtocounter{page}{-1}|\\
\textit{code for title page}\\
|\newpage|\\
|\||fi|
\end{tabular}
\end{center}
%
A banner page for the child documents can be generated by:
%
\begin{center}
\begin{tabular}{l}
|\ifchilddoc|\\
|\addtocounter{page}{-1}|\\
\textit{code for banner page}\\
|\newpage|\\
|\||fi|
\end{tabular}
\end{center}
%
Here one could write a message such as:
\begin{center}
|This is the part \childdocname{} of \childdocjob{}.|
\end{center}

%%%%%%%%%%%%%%%%%%%%%%%%%%%%%%%%%%%%%%%%%%%%%%%%%%%%%%%%%%%%%%%%%%%%%%%%%%%%%%%%
\subsection{Flags}
\label{sec:flags}

The package makes it easy to generate different versions
of the main or child documents.
To this end compilation flags can be defined
and assigned different default values.
They will be particularly useful in conjunction
with the forwarding mechanism described in \secref{sec:forward}.

For example, it may be useful to have a flag |\version|
which can be set to |draft| or |final|.
The document source will contain some conditional code
depending on the value of |\version|.
Suppose further, the flag should default to |final| for the main file
and to |draft| for child files
which is a natural assignment for editing the document.
This is achieved by placing the following code
in the preamble of the main document
(below the |\childdocmain| directive):
%
\begin{center}
\begin{tabular}{l}
|\ifchilddoc|\\
|\providecommand{\version}{draft}|\\
|\||else|\\
|\providecommand{\version}{final}|\\
|\||fi|
\end{tabular}
\end{center}
%
The definition by |\providecommand| makes sure
that previous definitions are not overwritten.
Further statements |\providecommand{\version}{...}|
can thus be added before the above code to override it.

For the main file, one might add a line
(between |\childdocmain| and the above block)
%
\begin{center}
|%\ifchilddoc\||else\providecommand{\version}{draft}\||fi|
\end{center}
%
which can be uncommented to produce a draft version.
Likewise one can add a line to the very top of a child file
(above the |\childdocof{|\textit{main}|}| directive)
%
\begin{center}
|%\providecommand{\version}{final}|
\end{center}
%
which can be uncommented to produce the final version of this child document.

%%%%%%%%%%%%%%%%%%%%%%%%%%%%%%%%%%%%%%%%%%%%%%%%%%%%%%%%%%%%%%%%%%%%%%%%%%%%%%%%
\subsection{Forwarding}
\label{sec:forward}

Different versions of the main or child documents
using compilation flags as described in \secref{sec:flags}
can be (permanently) stored in different files
for convenient compilation, viewing and distribution.
To this end, the package defines a command
to pass on compilation to a different file:

%%%%%%%%%%%%%%%%%%%%%%%%%%%%%%%%%%%%%%%%
\DescribeMacro{\childdocforward}
The command |\childdocforward| redirects processing to
another source file:
%
\begin{center}
\begin{tabular}{l}
|\input{childdoc.def}|\\
|\childdocforward[|\textit{main}|]{|\textit{dest}|}|\\
\end{tabular}
\end{center}
%
The argument \textit{dest} is the destination file
(without extension).
It should be the main file or one of the child files.
Note that further \textsf{childdoc} directives
such as |\childdocof| and |\childdocforward|
in the indicated file will be processed in this form.
The optional argument \textit{main}
passes on directly to the main file \textit{main}
while pretending to compile the child \textit{dest}.
This form behaves as if \textit{dest}
issues |\childdocof{|\textit{main}|}| right away,
and no further \textsf{childdoc} directives will be processed.

%%%%%%%%%%%%%%%%%%%%%%%%%%%%%%%%%%%%%%%%
\DescribeMacro{\...prefix}
In the alternative form |\childdocforwardprefix|,
%
\begin{center}
\begin{tabular}{l}
|\input{childdoc.def}|\\
|\childdocforwardprefix[|\textit{main}|]{|\textit{prefix}|}{|\textit{dest}|}|
\end{tabular}
\end{center}
%
the destination file is determined by a pattern
depending on the current file:
To make this work, the current file must be called
`{\textit{prefix}\hspace{0.2em}\textit{suffix}}'
with \textit{prefix} matching precisely the argument.
Processing is then passed on to the file
`{\textit{dest}\hspace{0.2em}\textit{suffix}}'.
Surely, the same effect is achieved by
directly specifying the
argument `{\textit{dest}\hspace{0.2em}\textit{suffix}}'
in the first form.
However, that requires to set up a different file
for each child. With the alternative form of the command
all these files can have exactly the same content
which simplifies setting them up and maintaining them.

For example, the following file |draft.tex|
with a compilation flag |\version| as described in \secref{sec:flags}
compiles the main document as a draft:
%
\begin{center}
\begin{tabular}{l}
|\def\version{draft}|\\
|\input{childdoc.def}|\\
|\childdocforward{|\textit{main}|}|
\end{tabular}
\end{center}
%
Likewise, the following files |final|\textit{nn}|.tex|
compile the final version of the child document
|child|\textit{nn}|.tex|:
%
\begin{center}
\begin{tabular}{l}
|\def\version{final}|\\
|\input{childdoc.def}|\\
|\childdocforwardprefix{final}{child}|
\end{tabular}
\end{center}
%

Note that when several versions of a main file and/or of each child file
are to be generated, it may be convenient to set up a |Makefile| or
shell script to automatise the process.

%%%%%%%%%%%%%%%%%%%%%%%%%%%%%%%%%%%%%%%%%%%%%%%%%%%%%%%%%%%%%%%%%%%%%%%%%%%%%%%%
\subsection{Command Line Processing}
\label{sec:commandline}

The effect of redirection files can also be achieved by invoking
the \LaTeX{} compiler with a more elaborate command line.
Most conveniently this should be done as part
of a shell script or a |Makefile|.

When using \textsf{childdoc} in the main file, the following
command lines effectively perform a redirection
(note that depending on the shell being used,
backslashes may have to be doubled: `|\|' $\to$ `|\\|'):
%
\begin{center}
|... -jobname "|\textit{target}|" |\\|"|[\textit{flags}]%
|\input{childdoc.def}\childdocforward[|\textit{main}|]{|\textit{dest}|}"|
\end{center}
%
Here \textit{target} is the name of the output file,
\textit{main} is the name of the main file
and \textit{dest} is the name of the main or child file to be processed
(all filenames without extensions).
The optional argument \textit{main} can be omitted
if \textit{main} matches \textit{dest}.
Optionally, compilation \textit{flags} can be defined via |\def| commands.
This command line makes the \TeX{} engine believe
it is compiling the file \textit{target}
whose content is specified as the latter parameter.
The provided code then forwards the processing to
\textit{main} or \textit{dest} as described in \secref{sec:forward}.

%%%%%%%%%%%%%%%%%%%%%%%%%%%%%%%%%%%%%%%%%%%%%%%%%%%%%%%%%%%%%%%%%%%%%%%%%%%%%%%%
\subsection{Include by Input}
\label{sec:input}

Including child documents by |\include| has some restrictions by design.
Most notably, the content of a child document always occupies
its own set of pages; pages cannot be shared between child documents.
Usually, this behaviour makes perfect sense
because each child document contain an essential part of the document.
However, in some situations it may be desirable to compose
a document from a collection of parts
without having mandatory page breaks between then.
For this case, the package
provides a mechanism to include parts
by |\input| which can also be processed individually.
However, by construction this mechanism
requires manual handling of the content to be output.

%%%%%%%%%%%%%%%%%%%%%%%%%%%%%%%%%%%%%%%%
\DescribeMacro{\ifchilddocmanual}
The main file should be prepared as usual, see \secref{sec:include}.
However, the document body must make a distinction
between processing of an individual part and of the main document, e.g.:
%
\begin{center}
\begin{tabular}{l}
|\ifchilddocmanual|\\
|\input{\childdocname}|\\
|\||else|\\
\textit{document body with }|\input{|\textit{part}|}|\\
|\||fi|
\end{tabular}
\end{center}
%
The conditional |\ifchilddocmanual| is true whenever
a part to be included by |\input| is being compiled,
and the name of the part is stored in |\childdocname|.

%%%%%%%%%%%%%%%%%%%%%%%%%%%%%%%%%%%%%%%%
\DescribeMacro{\childdocby}
Each part to be included by |\input| should start with:
%
\begin{center}
\begin{tabular}{l}
|\input{childdoc.def}|\\
|\childdocby{|\textit{main}|}|\\
\end{tabular}
\end{center}
%
The directive |\childdocby| is similar to |\childdocof|
described in \secref{sec:include},
but the subsequent selection of content must be done manually.
To that end, both |\ifchilddoc| and |\ifchilddocmanual|
will be true upon processing of a part,
and the name of the part is stored in |\childdocname|.
Note that |\jobname| will be set to the filename of the current part
so that each part receives an individual |.aux| file
that does not interfere with the |.aux| file(s) of the main document.
This behaviour can be altered by the alternative form
|\childdocby[*]{|\textit{main}|}| (with a non-empty optional argument)
which uses the |.aux| file of the main document
by setting |\jobname| to \textit{main}.

%%%%%%%%%%%%%%%%%%%%%%%%%%%%%%%%%%%%%%%%%%%%%%%%%%%%%%%%%%%%%%%%%%%%%%%%%%%%%%%%
\subsection{Driver Development}
\label{sec:driver}

The \textsf{childdoc} mechanism can also be use for the development
of definition files such as \LaTeX{} styles or classes.
This case differs from the above setup with multiple parts
included by |\include| in that no |\includeonly| should be invoked.
This can be achieved by starting the include file
(before |\ProvidesPackage|) with:
%
\begin{center}
\begin{tabular}{l}
|\input{childdoc.def}|\\
|\childdocforward{|\textit{main}|}|\\
\end{tabular}
\end{center}
%
or alternatively with:
%
\begin{center}
\begin{tabular}{l}
|\input{childdoc.def}|\\
|\childdocby{|\textit{main}|}|\\
\end{tabular}
\end{center}
%
Both forms have slightly different effects as described above.
The main file is prepared as usual, see \secref{sec:include}.

%%%%%%%%%%%%%%%%%%%%%%%%%%%%%%%%%%%%%%%%%%%%%%%%%%%%%%%%%%%%%%%%%%%%%%%%%%%%%%%%
\subsection{Legacy Detection}
\label{sec:detection}

The directive |\childdocmain| in the main file can detect
whether the complete document or merely a child is to be compiled
even without using the directive |\childdocof|.
This method is deprecated because it is less robust
and there is no compelling reason to use it;
it is merely provided for backward compatibility
and it may be removed in future versions.

If the detection mechanism is to be used,
it is mandatory to correctly specify
the filename of the main file as the argument of |\childdocmain|:
%
\begin{center}
\begin{tabular}{l}
|\input{childdoc.def}|\\
|\childdocmain{|\textit{main}|}|\\
\end{tabular}
\end{center}
%
If |\jobname| does not match the argument \textit{main} of |\childdocmain|,
it is assumed that |\jobname| points to the child file to be compiled.
When using |\childdocmain| with the main file specified as argument,
it suffices to start a child file
with just |\input{|\textit{main}|}|
without loading of the package and using |\childdocof|.
If instead all processing is done
with the appropriate \textsf{childdoc} directives,
the argument of \textit{main} of |\childdocmain| can be empty.

An alternative version of the command line processing described
in \secref{sec:commandline} using the detection mechanism reads:
%
\begin{center}
|... -jobname "|\textit{target}|" "|[\textit{flags}]%
[|\def\jobname{|\textit{dest}|}|]|\input{|\textit{main}|}"|
\end{center}

%%%%%%%%%%%%%%%%%%%%%%%%%%%%%%%%%%%%%%%%%%%%%%%%%%%%%%%%%%%%%%%%%%%%%%%%%%%%%%%%
\subsection{Manual Code}
\label{sec:manual}

In case one cannot be certain whether the definitions file |childdoc.def|
is installed on the target \TeX{} distribution
and one prefers not to ship it,
it is conceivable to paste a few relevant commands into the sources.

To that end, drop all statements |\input{childdoc.def}|
and perform the replacements as outlined below.
Instead of |\childdocmain{|\textit{main}|}| add the following code
to the top of the main file:
%
\begin{center}
\begin{tabular}{l}
|\||ifdefined\childdocname\endinput\||fi\newif\ifchilddoc|\\
|\edef\childdocname{\scantokens\expandafter{\jobname\noexpand}}|\\
|\def\childdocmain{|\textit{main}|}\||ifx\childdocmain\childdocname\||else|\\
|\childdoctrue\includeonly{\childdocname}\let\jobname\childdocmain\||fi|\\
\end{tabular}
\end{center}
%
Instead of |\childdocof{|\textit{main}|}| just include the main file
at the top of each child file:
%
\begin{center}
|\input{|\textit{main}|}|
\end{center}
%
A simple redirection |\childdocforward{|\textit{dest}|}| is achieved by:
%
\begin{center}
|\def\jobname{|\textit{dest}|}\input{\jobname}|
\end{center}
%
The redirection with prefix
|\childdocforwardprefix[|\textit{prefix}|]{|\textit{dest}|}|
is accomplished by:
%
\begin{center}
\begin{tabular}{l}
|{\edef\jobname{\scantokens\expandafter{\jobname\noexpand}}|\\
|\def\redirectjob |\textit{prefix}|#1~~~{\gdef\jobname{|\textit{dest}|#1}}|\\
|\expandafter\redirectjob\jobname~~~}\input{\jobname}|
\end{tabular}
\end{center}

In an alternative approach,
child documents can be compiled by a specific command line
without additional code or specific definitions:
%
\begin{center}
|... -jobname "|\textit{target}|" "|[\textit{flags}]%
|\includeonly{|\textit{dest}|}\input{|\textit{main}|}"|
\end{center}
%

%%%%%%%%%%%%%%%%%%%%%%%%%%%%%%%%%%%%%%%%%%%%%%%%%%%%%%%%%%%%%%%%%%%%%%%%%%%%%%%%
%%%%%%%%%%%%%%%%%%%%%%%%%%%%%%%%%%%%%%%%%%%%%%%%%%%%%%%%%%%%%%%%%%%%%%%%%%%%%%%%
\section{Information}

%%%%%%%%%%%%%%%%%%%%%%%%%%%%%%%%%%%%%%%%%%%%%%%%%%%%%%%%%%%%%%%%%%%%%%%%%%%%%%%%
\subsection{Copyright}

Copyright \copyright{} 2017--2018 Niklas Beisert

This work may be distributed and/or modified under the
conditions of the \LaTeX{} Project Public License, either version 1.3
of this license or (at your option) any later version.
The latest version of this license is in
  \url{http://www.latex-project.org/lppl.txt}
and version 1.3 or later is part of all distributions of \LaTeX{}
version 2005/12/01 or later.

This work has the LPPL maintenance status `maintained'.

The Current Maintainer of this work is Niklas Beisert.

This work consists of the files |README.txt|, |childdoc.ins| and |childdoc.dtx|
as well as the derived files |childdoc.def|, |cdocsamp.tex|
with |cdocsch1.tex|, |cdocsch2.tex|, |cdocspt3.tex|, |cdocspt4.tex|,
|cdocsdrf.tex|, |cdocsfn1.tex|, |cdocsfn2.tex|
as well as |childdoc.pdf|.

%%%%%%%%%%%%%%%%%%%%%%%%%%%%%%%%%%%%%%%%%%%%%%%%%%%%%%%%%%%%%%%%%%%%%%%%%%%%%%%%
\subsection{Files and Installation}

The package consists of the files:
%
\begin{center}
\begin{tabular}{ll}
    |README.txt|   & readme file \\
    |childdoc.ins| & installation file \\
    |childdoc.dtx| & source file \\
    |childdoc.def| & definition file \\
    |cdocsamp.tex| & sample main file \\
    |cdocsch1.tex| & sample include file \\
    |cdocsch2.tex| & sample include file \\
    |cdocspt3.tex| & sample part file \\
    |cdocspt4.tex| & sample part file \\
    |cdocsdrf.tex| & sample redirection file \\
    |cdocsfn1.tex| & sample redirection file \\
    |cdocsfn2.tex| & sample redirection file \\
    |childdoc.pdf| & manual
\end{tabular}
\end{center}
%
The distribution consists of the files
|README.txt|, |childdoc.ins| and |childdoc.dtx|.
%
\begin{itemize}
\item
Run (pdf)\LaTeX{} on |childdoc.dtx|
to compile the manual |childdoc.pdf| (this file).
\item
Run \LaTeX{} on |childdoc.ins| to create the definitions file |childdoc.def|
and the sample |cdocsamp.tex| with include files
|cdocsch1.tex|, |cdocsch2.tex|, |cdocspt3.tex|, |cdocspt4.tex|,
|cdocsdrf.tex|, |cdocsfn1.tex|, |cdocsfn2.tex|.
Then copy the file |childdoc.def| to an appropriate directory of your \LaTeX{}
distribution, e.g.\ \textit{texmf-root}|/tex/latex/childdoc|.
\end{itemize}

%%%%%%%%%%%%%%%%%%%%%%%%%%%%%%%%%%%%%%%%%%%%%%%%%%%%%%%%%%%%%%%%%%%%%%%%%%%%%%%%
\subsection{Related CTAN Packages}

There are several other packages which offer a similar functionality:
%
\begin{itemize}
\item
The packages
\href{http://ctan.org/pkg/docmute}{\textsf{docmute}},
\href{http://ctan.org/pkg/includex}{\textsf{includex}} and
\href{http://ctan.org/pkg/standalone}{\textsf{standalone}}
provide commands to include only the document body of
a child file thus allowing both files to be compiled individually.
\item
The packages \href{http://ctan.org/pkg/subdocs}{\textsf{subdocs}}
and \href{http://ctan.org/pkg/subfiles}{\textsf{subfiles}}
provide structures in which the main and child documents can be
encapsulated and allowing them to be compiled individually.
The inclusion mechanism is different from the conventional |\include|.
\item
The package \href{http://ctan.org/pkg/combine}{\textsf{combine}}
is an elaborate solution to combine several documents into one.
\end{itemize}
%
See also the CTAN topic \href{http://ctan.org/topic/subdocs}{\textsf{subdocs}}
for further related packages.
The present package differs from the above solutions in that
a document structure constructed with the conventional |\include| mechanism
just needs two extra commands at the top of every file
such that all constituent files can be compiled individually.

%%%%%%%%%%%%%%%%%%%%%%%%%%%%%%%%%%%%%%%%%%%%%%%%%%%%%%%%%%%%%%%%%%%%%%%%%%%%%%%%
%\subsection{Feature Suggestions}
%
%The following is a list of features which may be useful for future
%versions of this package:
%%
%\begin{itemize}
%\item
%\ldots
%\end{itemize}

%%%%%%%%%%%%%%%%%%%%%%%%%%%%%%%%%%%%%%%%%%%%%%%%%%%%%%%%%%%%%%%%%%%%%%%%%%%%%%%%
\subsection{Revision History}

%%%%%%%%%%%%%%%%%%%%%%%%%%%%%%%%%%%%%%%%
\paragraph{v2.0:} 2018/12/30

\begin{itemize}
\item
immediate forward processing
\item
added |\childdocby| mechanism
\item
manual restructured
\end{itemize}

%%%%%%%%%%%%%%%%%%%%%%%%%%%%%%%%%%%%%%%%
\paragraph{v1.6:} 2018/01/17

\begin{itemize}
\item
application for development of include files
\item
corrections to manual
\end{itemize}

%%%%%%%%%%%%%%%%%%%%%%%%%%%%%%%%%%%%%%%%
\paragraph{v1.5:} 2017/05/21

\begin{itemize}
\item
more complete structuring introduced
\item
|\childdocof| introduced
\item
|\childdoc| renamed to |\childdocmain|
\item
|\childredirect| renamed to |\childdocforward| and |\childdocforwardprefix|
and functionality expanded
\end{itemize}

%%%%%%%%%%%%%%%%%%%%%%%%%%%%%%%%%%%%%%%%
\paragraph{v1.0:} 2017/04/27

\begin{itemize}
\item
manual and install package
\item
first version published on CTAN
\end{itemize}

%%%%%%%%%%%%%%%%%%%%%%%%%%%%%%%%%%%%%%%%
\paragraph{v0.6:} 2017/04/26

\begin{itemize}
\item
redirection mechanism added
\end{itemize}

%%%%%%%%%%%%%%%%%%%%%%%%%%%%%%%%%%%%%%%%
\paragraph{v0.5:} 2017/04/26

\begin{itemize}
\item
functionality in definition file
\end{itemize}


%%%%%%%%%%%%%%%%%%%%%%%%%%%%%%%%%%%%%%%%%%%%%%%%%%%%%%%%%%%%%%%%%%%%%%%%%%%%%%%%
%%%%%%%%%%%%%%%%%%%%%%%%%%%%%%%%%%%%%%%%%%%%%%%%%%%%%%%%%%%%%%%%%%%%%%%%%%%%%%%%
%%%%%%%%%%%%%%%%%%%%%%%%%%%%%%%%%%%%%%%%%%%%%%%%%%%%%%%%%%%%%%%%%%%%%%%%%%%%%%%%
\appendix

\settowidth\MacroIndent{\rmfamily\scriptsize 000\ }

 \DocInput{childdoc.dtx}

\end{document}
%</driver>
% \fi
%
% %%%%%%%%%%%%%%%%%%%%%%%%%%%%%%%%%%%%%%%%%%%%%%%%%%%%%%%%%%%%%%%%%%%%%%%%%%%%%%
% %%%%%%%%%%%%%%%%%%%%%%%%%%%%%%%%%%%%%%%%%%%%%%%%%%%%%%%%%%%%%%%%%%%%%%%%%%%%%%
% \section{Sample}
%\iffalse
%<*samplemain>
%\fi
%
% The following presents a sample document
% with two chapters, two parts, a title page,
% a compile flag as well as three forwarding files to set the flag.
% It consists of eight |.tex| files:
% \begin{center}
% \begin{tabular}{ll}
% |cdocsamp.tex|&main file\\
% |cdocsch1.tex|&include file for chapter 1\\
% |cdocsch2.tex|&include file for chapter 2\\
% |cdocspt3.tex|&include file for part 3\\
% |cdocspt4.tex|&include file for part 4\\
% |cdocsdrf.tex|&forwarding file for main file in draft mode\\
% |cdocsfi1.tex|&forwarding file for final version of chapter 1\\
% |cdocsfi2.tex|&forwarding file for final version of chapter 2\\
% \end{tabular}
% \end{center}
% Each of the eight files can be compiled directly by the \LaTeX{} compiler.
%
% %%%%%%%%%%%%%%%%%%%%%%%%%%%%%%%%%%%%%%
% \paragraph{Main File.}
%
% The main file is called |cdocsamp.tex|.
%
% Load the \textsf{childdoc} definitions and
% declare the filename for the main document:
%    \begin{macrocode}
\input{childdoc.def}
\childdocmain{}
%    \end{macrocode}

% Optional override for |\version| flag:
%    \begin{macrocode}
%%\ifchilddoc\else\providecommand{\version}{draft}\fi
%    \end{macrocode}

% Define the default values for the |\version| flag
% (|final| for the main file and |draft| for childs):
%    \begin{macrocode}
\ifchilddoc
\providecommand{\version}{draft}
\else
\providecommand{\version}{final}
\fi
%    \end{macrocode}

% Load the standard document class:
%    \begin{macrocode}
\documentclass[12pt]{article}
%    \end{macrocode}

% Start the document body:
%    \begin{macrocode}
\begin{document}
%    \end{macrocode}

% Declare a title page.
% Print title, part of document being processed and version flag:
%    \begin{macrocode}
\addtocounter{page}{-1}
\begin{center}
{\LARGE\bfseries{}childdoc example\par}
\vspace{1cm}
\ifchilddoc
\ifchilddocmanual part\else chapter\fi:
`\childdocname' of `\childdocjob'\par
\else
main document: `\childdocjob'\par
\fi
version: \version\par
\end{center}
\newpage
%    \end{macrocode}

% Manually include selected file,
% otherwise process as usual:
%    \begin{macrocode}
\ifchilddocmanual
\section*{part `\childdocname'}
\input{\childdocname}
\else
%    \end{macrocode}

% Include the two chapters:
%    \begin{macrocode}
\include{cdocsch1}
\include{cdocsch2}
%    \end{macrocode}

% Include the two parts unless only chapters should be displayed:
%    \begin{macrocode}
\ifchilddoc\else
\section{part three}
\input{cdocspt3}
\section{part four}
\input{cdocspt4}
\fi
%    \end{macrocode}

% Process as usual until here:
%    \begin{macrocode}
\fi
%    \end{macrocode}

% End of document body:
%    \begin{macrocode}
\end{document}
%    \end{macrocode}
%\iffalse
%</samplemain>
%\fi
%
% %%%%%%%%%%%%%%%%%%%%%%%%%%%%%%%%%%%%%%
% \paragraph{Chapter Include Files.}
%
% The include files are called |cdocsch1.tex| and |cdocsch2.tex|.
%
%\iffalse
%<*samplechap1|samplechap2>
%\fi

% Optional override for |\version| flag:
%    \begin{macrocode}
%%\providecommand{\version}{final}
%    \end{macrocode}

% Include the main document:
%    \begin{macrocode}
\input{childdoc.def}
\childdocof{cdocsamp}
%    \end{macrocode}

%\iffalse
%</samplechap1|samplechap2>
%\fi
%
%\iffalse
%<*samplechap1>
%\fi
% Some text for chapter 1:
%    \begin{macrocode}
\section{one}
some text in chapter one
%    \end{macrocode}

%\iffalse
%</samplechap1>
%\fi
% Some text for chapter 2:
%\iffalse
%<*samplechap2>
%\fi
%    \begin{macrocode}
\section{two}
more text in chapter two
%    \end{macrocode}

%\iffalse
%</samplechap2>
%\fi
%
% %%%%%%%%%%%%%%%%%%%%%%%%%%%%%%%%%%%%%%
% \paragraph{Part Include Files.}
%
% The include files are called |cdocspt3.tex| and |cdocspt4.tex|.
%
%\iffalse
%<*samplepart3|samplepart4>
%\fi

% Optional override for |\version| flag:
%    \begin{macrocode}
%%\providecommand{\version}{final}
%    \end{macrocode}

% Include the main document:
%    \begin{macrocode}
\input{childdoc.def}
\childdocby{cdocsamp}
%    \end{macrocode}

%\iffalse
%</samplepart3|samplepart4>
%\fi
%
%\iffalse
%<*samplepart3>
%\fi
% Some text for part 3:
%    \begin{macrocode}
some text in part three
%    \end{macrocode}

%\iffalse
%</samplepart3>
%\fi
% Some text for part 4:
%\iffalse
%<*samplepart4>
%\fi
%    \begin{macrocode}
more text in part four
%    \end{macrocode}

%\iffalse
%</samplepart4>
%\fi
%
% %%%%%%%%%%%%%%%%%%%%%%%%%%%%%%%%%%%%%%
% \paragraph{Forwarding for a Complete Draft.}
%
% The following forwarding file |cdocsdrf.tex|
% compiles the main document in draft mode:
%\iffalse
%<*sampledraft>
%\fi
%    \begin{macrocode}
\def\version{draft}
\input{childdoc.def}
\childdocforward{cdocsamp}
%    \end{macrocode}

%\iffalse
%</sampledraft>
%\fi
%
% %%%%%%%%%%%%%%%%%%%%%%%%%%%%%%%%%%%%%%
% \paragraph{Forwarding for Final Version of the Chapters.}
%
% The following forwarding files |cdocsfn1.tex| and |cdocsfn2.tex|
% (with identical content)
% compile the final versions of the child documents
% |cdocsch1.tex| and |cdocsch2.tex|, respectively:
%\iffalse
%<*samplefinal>
%\fi
%    \begin{macrocode}
\def\version{final}
\input{childdoc.def}
\childdocforwardprefix[cdocsamp]{cdocsfn}{cdocsch}
%    \end{macrocode}

%\iffalse
%</samplefinal>
%\fi
%
% %%%%%%%%%%%%%%%%%%%%%%%%%%%%%%%%%%%%%%
% \paragraph{Command Line Processing.}
%
% The following three command lines generate the output files
% |cdocscld|, |cdocscl1| and |cdocscl2|
% which should be identical to
% |cdocsdrf|, |cdocsch1| and |cdocsfn2|, respectively:
% \begin{center}
% \begin{tabular}{l}
% |latex -jobname cdocscld \|\\
% |  "\def\version{draft}\input{childdoc.def}\childdocforward{cdocsamp}"|\\
% |latex -jobname cdocscl1 \|\\
% |  "\input{childdoc.def}\childdocforward[cdocsamp]{cdocsch1}"|\\
% |latex -jobname cdocscl2 \|\\
% |  "\def\version{final}\input{childdoc.def}\childdocforward{cdocsch2}"|
% \end{tabular}
% \end{center}
% Note that the trailing backslash on each first line
% merely continues the input to the second line
% (for convenient cut ant paste).
% Furthermore, the command |latex| can be replaced by any
% of its alternative versions such as |pdflatex|.
%
% %%%%%%%%%%%%%%%%%%%%%%%%%%%%%%%%%%%%%%%%%%%%%%%%%%%%%%%%%%%%%%%%%%%%%%%%%%%%%%
% %%%%%%%%%%%%%%%%%%%%%%%%%%%%%%%%%%%%%%%%%%%%%%%%%%%%%%%%%%%%%%%%%%%%%%%%%%%%%%
% \section{Implementation}
%\iffalse
%<*package>
%\fi
%
% This section describes the definitions file |childdoc.def|.

% The definitions cannot be loaded using |\usepackage| or |\RequirePackage|
% which has a mechanism to prevent loading a style file more than once.
% When loading the definitions by means of |\input|
% multiple instances have to be prevented manually:
%\iffalse
%This code needs to be before the `\ProvidesFile' directive
%which is defined at the beginning of this file.
%Therefore it is also placed there and commented out here.
%</package>
%<*discard>
%\fi
%    \begin{macrocode}
\ifdefined\childdocmain\endinput\fi
%    \end{macrocode}
%\iffalse
%</discard>
%<*package>
%\fi
%
% \macro{\ifchilddoc}
% \macro{\ifchilddocmanual}
% The conditional |\ifchilddoc| tells whether a
% child (true) or main (false) document is being compiled.
% The conditional |\ifchilddocmanual| tells whether
% the |\includeonly| mechanism is used (false) or
% the selection of child files must be performed manually (true).
% The definitions initialise to false:
%    \begin{macrocode}
\newif\ifchilddoc
\newif\ifchilddocmanual
%    \end{macrocode}

% \macro{\childdocname}
% \macro{\childdocjob}
% The macro |\childdocname| stores the name of the main document
% to be compiled. The macro |\childdocjob| stores the name of
% the document on which the \LaTeX{} compiler was originally invoked.
% The content of |\jobname| cannot be compared
% to filenames specified in the source due to different catcodes.
% The following code rescans |\jobname|, stores the result
% in |\childdocname| and saves a copy in |\childdocjob|:
%    \begin{macrocode}
\edef\childdocname{\scantokens\expandafter{\jobname\noexpand}}
\let\childdocjob\childdocname
%    \end{macrocode}

% \macro{\childdocdisable}
% The macro |\childdocdisable| prevents the main file
% from being processed more than once.
% At this stage, the main document command |\childdocmain|
% is assumed to be called once again where it should do nothing.
% Any subsequent call to it should prevent
% a secondary processing of the main document
% It overwrites the forwarding commands
% |\childdocof| and |\childdocforward|
% with empty macros to prevent further inclusions of the main document:
%    \begin{macrocode}
\newcommand{\childdocdisable}
{
  \renewcommand{\childdocmain}[1]{\renewcommand{\childdocmain}[1]{\endinput}}
  \renewcommand{\childdocof}[1]{}
  \renewcommand{\childdocby}[2][]{}
  \renewcommand{\childdocforward}[2][]{}
  \renewcommand{\childdocdisable}{}
}
%    \end{macrocode}

% \macro{\childdocmain}
% The macro |\childdocmain| is to be called at the top of the main file
% with nothing or the main filename (without extension) as argument.
% First, it breaks loops.
% If the argument is not empty and does not match |\childdocname|
% (which is set by the first inclusion of |childdoc.def|),
% |\ifchilddoc| is set to true, |\includeonly| is applied to the child file
% and |\jobname| is set to the main file
% (for proper handling of |.aux| files):
%    \begin{macrocode}
\newcommand{\childdocmain}[1]
{
  \childdocdisable\childdocmain{}
  \if?#1?\else
    \begingroup
      \def\childdoctmp{#1}
      \ifx\childdoctmp\childdocname
        \def\childdoctmp{}
      \else
        \def\childdoctmp
        {
          \childdoctrue
          \includeonly{\childdocname}
          \def\childdocjob{#1}
          \def\jobname{#1}
        }
      \fi
      \expandafter
    \endgroup
    \childdoctmp
  \fi
}
%    \end{macrocode}

% \macro{\childdocof}
% The command |\childdocof| redirects
% compilation to the main file |#1|.
%    \begin{macrocode}
\newcommand{\childdocof}[1]
{
  \childdocdisable
  \childdoctrue
  \includeonly{\childdocname}
  \def\jobname{#1}
  \def\childdocjob{#1}
  \input{#1}
}
%    \end{macrocode}

% \macro{\childdocby}
% The command |\childdocby| ....
%    \begin{macrocode}
\newcommand{\childdocby}[2][]
{
  \childdocdisable
  \childdoctrue
  \childdocmanualtrue
  \if?#1?\else
    \def\jobname{#2}
  \fi
  \def\childdocjob{#2}
  \input{#2}
  \endinput
}
%    \end{macrocode}

% \macro{\childdocforward}
% The command |\childdocforward| redirects
% compilation to the main file or
% (if the optional argument is given) a child file.
% Parameters are set as if the main file
% or a child file starting with |\childdocof| was compiled.
% Then compilation is handed over to the main file:
%    \begin{macrocode}
\newcommand{\childdocforward}[2][]
{
  \begingroup
    \if?#1?
      \def\childdoctmp
      {
        \def\childdocname{#2}
        \def\childdocjob{#2}
        \def\jobname{#2}
        \input{#2}
        \endinput
      }
    \else
      \def\childdoctmp
      {
        \childdocdisable
        \def\childdocname{#2}
        \childdoctrue
        \includeonly{#2}
        \def\childdocjob{#1}
        \def\jobname{#1}
        \input{#1}
        \endinput
      }
    \fi
    \expandafter
  \endgroup
  \childdoctmp
}
%    \end{macrocode}

% \macro{\childdocforwardprefix}
% The command |\childdocforwardprefix| redirects
% compilation to the main or a child file by means of a pattern.
% The prefix |#1| in the current filename is replaced by |#2|
% and the suffix of the current filename is kept
% (it is assumed that the filename does not contain the substring `|~~~|'
% which is used as a delimiter).
% Compilation is handed over to the new file by |\childdocforward|:
%    \begin{macrocode}
\newcommand{\childdocforwardprefix}[3][]
{
  \begingroup
    \def\childdocextract #2##1~~~{\def\childdoctmp{\childdocforward[#1]{#3##1}}}
    \expandafter\childdocextract\childdocname~~~
    \expandafter
  \endgroup
  \childdoctmp
}
%    \end{macrocode}

% \macro{\childdoc}
% The deprecated macro |\childdoc| is a legacy version of |\childdocmain|:
%    \begin{macrocode}
\newcommand{\childdoc}{\childdocmain}
%    \end{macrocode}

% \macro{\childdocredirect}
% The deprecated macro |\childdocredirect| is a legacy version
% of |\childdocforward| and |\childdocforwardprefix|:
%    \begin{macrocode}
\newcommand{\childdocredirect}[2][]
{
  \begingroup
    \if?#1?
      \def\childdoctmp{\childdocforward{#2}}
    \else
      \def\childdoctmp{\childdocforwardprefix{#1}{#2}}
    \fi
    \expandafter
  \endgroup
  \childdoctmp
}
%    \end{macrocode}

%\iffalse
%</package>
%\fi
%
\endinput
\childdocforward{cdocsamp}"|\\
% |latex -jobname cdocscl1 \|\\
% |  "% \iffalse
%
% childdoc.dtx Copyright (C) 2017-2018 Niklas Beisert
%
% This work may be distributed and/or modified under the
% conditions of the LaTeX Project Public License, either version 1.3
% of this license or (at your option) any later version.
% The latest version of this license is in
%   http://www.latex-project.org/lppl.txt
% and version 1.3 or later is part of all distributions of LaTeX
% version 2005/12/01 or later.
%
% This work has the LPPL maintenance status `maintained'.
%
% The Current Maintainer of this work is Niklas Beisert.
%
% This work consists of the files childdoc.dtx and childdoc.ins
% and the derived files childdoc.def and cdocsamp.tex with
% cdocsch1.tex, cdocsch2.tex, cdocsdrf.tex, cdocsfn1.tex, cdocsfn2.tex.
%
%<package>\ifdefined\childdocmain\endinput\fi
%<package>\ProvidesFile{childdoc.def}[2018/12/30 v2.0 child document driver]
%<samplemain>\ProvidesFile{cdocsamp.tex}[2018/12/30 v2.0 sample for childdoc]
%<*driver>
%\ProvidesFile{childdoc.drv}[2018/12/30 v2.0 childdoc reference manual file]
\PassOptionsToClass{10pt,a4paper}{article}
\documentclass{ltxdoc}

\usepackage[margin=35mm]{geometry}
\usepackage{hyperref}
\usepackage{hyperxmp}
\usepackage[usenames]{color}

\hypersetup{colorlinks=true}
\hypersetup{pdfstartview=FitH}
\hypersetup{pdfpagemode=UseNone}
\hypersetup{pdfsource={}}
\hypersetup{pdflang={en-UK}}
\hypersetup{pdfcopyright={Copyright 2017-2018 Niklas Beisert.
  This work may be distributed and/or modified under the
  conditions of the LaTeX Project Public License, either version 1.3
  of this license or (at your option) any later version.}}
\hypersetup{pdflicenseurl={http://www.latex-project.org/lppl.txt}}
\hypersetup{pdfcontactaddress={ETH Zurich, ITP, HIT K,
  Wolfgang-Pauli-Strasse 27}}
\hypersetup{pdfcontactpostcode={8093}}
\hypersetup{pdfcontactcity={Zurich}}
\hypersetup{pdfcontactcountry={Switzerland}}
\hypersetup{pdfcontactemail={nbeisert@itp.phys.ethz.ch}}
\hypersetup{pdfcontacturl={http://people.phys.ethz.ch/\xmptilde nbeisert/}}

\newcommand{\secref}[1]{\hyperref[#1]{section \ref*{#1}}}

\parskip1ex
\parindent0pt
\let\olditemize\itemize
\def\itemize{\olditemize\parskip0pt}

\begin{document}

\title{The \textsf{childdoc} Package}
\hypersetup{pdftitle={The childdoc Package}}
\author{Niklas Beisert\\[2ex]
  Institut f\"ur Theoretische Physik\\
  Eidgen\"ossische Technische Hochschule Z\"urich\\
  Wolfgang-Pauli-Strasse 27, 8093 Z\"urich, Switzerland\\[1ex]
  \href{mailto:nbeisert@itp.phys.ethz.ch}
  {\texttt{nbeisert@itp.phys.ethz.ch}}}
\hypersetup{pdfauthor={Niklas Beisert}}
\hypersetup{pdfsubject={Manual for the LaTeX2e Package childdoc}}
\date{30 December 2018, \textsf{v2.0}}
\maketitle

\begin{abstract}\noindent
\textsf{childdoc} is a \LaTeXe{} package
that enables the direct compilation
of document sections included by |\include|
to individual files.
\end{abstract}

\begingroup
\parskip0ex
\tableofcontents
\endgroup

%%%%%%%%%%%%%%%%%%%%%%%%%%%%%%%%%%%%%%%%%%%%%%%%%%%%%%%%%%%%%%%%%%%%%%%%%%%%%%%%
%%%%%%%%%%%%%%%%%%%%%%%%%%%%%%%%%%%%%%%%%%%%%%%%%%%%%%%%%%%%%%%%%%%%%%%%%%%%%%%%
\section{Introduction}

\LaTeX{} provides a mechanism to structure a large document (such as a book)
into a main file and several child files (containing the chapters)
using the |\include| command.
This mechanism is beneficial for documents
which span hundreds of pages in order to
make the source file(s) more manageable.
Moreover, compilation can be restricted to
selected child files by means of the |\includeonly| command.
The latter feature can be used to reduce the compilation time while editing
(this was significantly more useful in the earlier days of \LaTeX{})
or to generate a smaller document which is easier to navigate.
Another application of |\includeonly| is to generate
documents consisting of selected parts of the complete document.

However, there are a few drawbacks of the plain |\include| mechanism:
\begin{itemize}
\item
The child files cannot be compiled on their own,
they can only be compiled via the main file.
A naive editing environment
(such as a text editor with an option
to have the current file processed by \LaTeX)
may require one to switch to the main file before compiling;
attempting to compile the child file produces errors.
\item
The main file must be modified (each time)
to adjust the |\includeonly| command
to the present needs. This easily leaves the main file in a messy state.
\item
The generated document will always carry the filename
of the main document. This is inconvenient if
several child files are to be compiled and
to be kept for distribution.
\end{itemize}

The present package provides a simple interface
to make child files individually compilable by \LaTeX{}.
Compiling a child file then has the same effect as compiling
the main file with an |\includeonly| command
to select the appropriate child.
Moreover the generated document will carry the name of the child
rather than the main file.
This resolves all three above issues.

This feature is meant to make the editing of books,
thesis documents and lecture notes somewhat more convenient.
However, the package can also be used efficiently for
composing a series of documents (such as exercise sheets)
which are typically distributed individually.
It then assists the author in generating the individual documents
(potentially in different versions)
as well as a document containing the collected series.
Another application is in developing style files
or other kinds of included material
where compilation of the style file could redirect
to a sample or test file.

%%%%%%%%%%%%%%%%%%%%%%%%%%%%%%%%%%%%%%%%%%%%%%%%%%%%%%%%%%%%%%%%%%%%%%%%%%%%%%%%
%%%%%%%%%%%%%%%%%%%%%%%%%%%%%%%%%%%%%%%%%%%%%%%%%%%%%%%%%%%%%%%%%%%%%%%%%%%%%%%%
\section{Usage}

First of all, the package \textsf{childdoc} is \emph{not} a standard
\LaTeXe{} |.sty| style file! Therefore it needs to be invoked in
a non-standard way.

%%%%%%%%%%%%%%%%%%%%%%%%%%%%%%%%%%%%%%%%%%%%%%%%%%%%%%%%%%%%%%%%%%%%%%%%%%%%%%%%
\subsection{Included Files}
\label{sec:include}

%%%%%%%%%%%%%%%%%%%%%%%%%%%%%%%%%%%%%%%%
\DescribeMacro{\childdocmain}
To use the package, add the commands
\begin{center}
\begin{tabular}{l}
|\input{childdoc.def}|\\
|\childdocmain{}|\\
\end{tabular}
\end{center}
at the very top of the main \LaTeX{} file,
in particular \emph{before} the |\documentclass| statement!
The argument of |\childdocmain| should be left empty
(but it must be present).

%%%%%%%%%%%%%%%%%%%%%%%%%%%%%%%%%%%%%%%%
\DescribeMacro{\childdocof}
Furthermore, add the commands
\begin{center}
\begin{tabular}{l}
|\input{childdoc.def}|\\
|\childdocof{|\textit{main}|}|\\
\end{tabular}
\end{center}
at the top of every child file \textit{child}
which is included by |\include{|\textit{child}|}|
from within the main file
(or at least for those files to be compiled individually).
The argument \textit{main} must be the filename of the main file.

There are a couple of
considerations in setting up the main and child documents:

%%%%%%%%%%%%%%%%%%%%%%%%%%%%%%%%%%%%%%%%
\paragraph{Restrictions.}

Please note the following restrictions:
\begin{itemize}
\item
|\childdocmain| must be called with one argument \textit{main}
to ensure compatibility with earlier version of the package.
It must either be empty (|\childdocmain{}|)
or precisely match the filename of the main file in which it is specified.
See \secref{sec:detection} for further information.
\item
The filename \textit{main} must be specified without the |.tex| extension.
\item
The filename \textit{main} is case sensitive
(even in case-insensitive file systems)
due to internal string comparison.
\item
The argument \textit{main} should be fully expanded, it cannot be a macro.
\item
Subdirectories and special characters should be avoided in filenames.
\item
The command |\childdocmain{|\textit{main}|}| must be followed by a whitespace.
It should not be followed immediately by another command
or by a comment mark `|%|'.
This is because the \TeX{} parser reads the token immediately following
the argument of |\childdocmain| and puts it
at the beginning of every child section;
however, a white\-space is ignored.
\end{itemize}

%%%%%%%%%%%%%%%%%%%%%%%%%%%%%%%%%%%%%%%%
\paragraph{Content of Main File.}

It is advisable to place all content in the child files included by |\include|.
Any output contained in the main file will appear in all child documents
unless suppressed manually;
it cannot be suppressed automatically by the |\includeonly| directive
and thus should normally be avoided.
A method to include some content in the main file
by means of conditional processing is described in \secref{sec:conditional}.

%%%%%%%%%%%%%%%%%%%%%%%%%%%%%%%%%%%%%%%%
\paragraph{Page Numbering.}

When only a part of the document is compiled,
the appropriate numbering of pages
(as well as other status parameters)
is determined from the |.aux| files.
The latter contain information from previous passes.
However this information needs to propagate through
all intermediate child documents.
Therefore the page numbering in child documents may well
be inconsistent until the complete document is compiled at least once.

A useful (if unconventional) way to always ensure a consistent
page numbering is to restart the numbering in each child document
and denote the pages by `\textit{child}|.|\textit{page}'
where \textit{child} represents the chapter/section number of the child file.
This can be achieved by the command
|\numberwithin{page}{|\textit{child}|}|
of the \textsf{amsmath} package
where \textit{child} can be |chapter| or |section|
depending on the chosen structuring.
Alternatively, one can modify the macro |\thepage| appropriately
and reset the counter |page| at the start of each child file.

%%%%%%%%%%%%%%%%%%%%%%%%%%%%%%%%%%%%%%%%%%%%%%%%%%%%%%%%%%%%%%%%%%%%%%%%%%%%%%%%
\subsection{Conditional Processing}
\label{sec:conditional}

The package provides a mechanism to compile different versions
of a document. To customise the versions further some conditional processing
can come in handy to distinguish which version is being compiled.
The package provides two macros to describe the compilation context:

%%%%%%%%%%%%%%%%%%%%%%%%%%%%%%%%%%%%%%%%
\DescribeMacro{\ifchilddoc}
The conditional |\ifchilddoc| distinguishes between the compilation of
child documents and the main document:
%
\begin{center}
|\ifchilddoc |\textit{child-code}| |[|\||else |\textit{main-code}]| \||fi|
\end{center}

%%%%%%%%%%%%%%%%%%%%%%%%%%%%%%%%%%%%%%%%
\DescribeMacro{\childdocname}
\DescribeMacro{\childdocjob}
The macro |\childdocname| contains the filename (without extension)
of the main or child file being processed.
Note that |\childdocjob| will always contain the name of the main file.

%%%%%%%%%%%%%%%%%%%%%%%%%%%%%%%%%%%%%%%%
\paragraph{Title Page.}

Conditional processing can be used to include a title or banner page
in the main document when proper precautions are taken.
Importantly, the code in the main file should ensure that the page counter
(as well as other status parameters which are stored in the |.aux| files)
takes the same value after the conditional processing.
Otherwise the page numbers may take divergent values
depending on which part is compiled.

For example, a title page could be declared by:
%
\begin{center}
\begin{tabular}{l}
|\ifchilddoc\||else|\\
|\addtocounter{page}{-1}|\\
\textit{code for title page}\\
|\newpage|\\
|\||fi|
\end{tabular}
\end{center}
%
A banner page for the child documents can be generated by:
%
\begin{center}
\begin{tabular}{l}
|\ifchilddoc|\\
|\addtocounter{page}{-1}|\\
\textit{code for banner page}\\
|\newpage|\\
|\||fi|
\end{tabular}
\end{center}
%
Here one could write a message such as:
\begin{center}
|This is the part \childdocname{} of \childdocjob{}.|
\end{center}

%%%%%%%%%%%%%%%%%%%%%%%%%%%%%%%%%%%%%%%%%%%%%%%%%%%%%%%%%%%%%%%%%%%%%%%%%%%%%%%%
\subsection{Flags}
\label{sec:flags}

The package makes it easy to generate different versions
of the main or child documents.
To this end compilation flags can be defined
and assigned different default values.
They will be particularly useful in conjunction
with the forwarding mechanism described in \secref{sec:forward}.

For example, it may be useful to have a flag |\version|
which can be set to |draft| or |final|.
The document source will contain some conditional code
depending on the value of |\version|.
Suppose further, the flag should default to |final| for the main file
and to |draft| for child files
which is a natural assignment for editing the document.
This is achieved by placing the following code
in the preamble of the main document
(below the |\childdocmain| directive):
%
\begin{center}
\begin{tabular}{l}
|\ifchilddoc|\\
|\providecommand{\version}{draft}|\\
|\||else|\\
|\providecommand{\version}{final}|\\
|\||fi|
\end{tabular}
\end{center}
%
The definition by |\providecommand| makes sure
that previous definitions are not overwritten.
Further statements |\providecommand{\version}{...}|
can thus be added before the above code to override it.

For the main file, one might add a line
(between |\childdocmain| and the above block)
%
\begin{center}
|%\ifchilddoc\||else\providecommand{\version}{draft}\||fi|
\end{center}
%
which can be uncommented to produce a draft version.
Likewise one can add a line to the very top of a child file
(above the |\childdocof{|\textit{main}|}| directive)
%
\begin{center}
|%\providecommand{\version}{final}|
\end{center}
%
which can be uncommented to produce the final version of this child document.

%%%%%%%%%%%%%%%%%%%%%%%%%%%%%%%%%%%%%%%%%%%%%%%%%%%%%%%%%%%%%%%%%%%%%%%%%%%%%%%%
\subsection{Forwarding}
\label{sec:forward}

Different versions of the main or child documents
using compilation flags as described in \secref{sec:flags}
can be (permanently) stored in different files
for convenient compilation, viewing and distribution.
To this end, the package defines a command
to pass on compilation to a different file:

%%%%%%%%%%%%%%%%%%%%%%%%%%%%%%%%%%%%%%%%
\DescribeMacro{\childdocforward}
The command |\childdocforward| redirects processing to
another source file:
%
\begin{center}
\begin{tabular}{l}
|\input{childdoc.def}|\\
|\childdocforward[|\textit{main}|]{|\textit{dest}|}|\\
\end{tabular}
\end{center}
%
The argument \textit{dest} is the destination file
(without extension).
It should be the main file or one of the child files.
Note that further \textsf{childdoc} directives
such as |\childdocof| and |\childdocforward|
in the indicated file will be processed in this form.
The optional argument \textit{main}
passes on directly to the main file \textit{main}
while pretending to compile the child \textit{dest}.
This form behaves as if \textit{dest}
issues |\childdocof{|\textit{main}|}| right away,
and no further \textsf{childdoc} directives will be processed.

%%%%%%%%%%%%%%%%%%%%%%%%%%%%%%%%%%%%%%%%
\DescribeMacro{\...prefix}
In the alternative form |\childdocforwardprefix|,
%
\begin{center}
\begin{tabular}{l}
|\input{childdoc.def}|\\
|\childdocforwardprefix[|\textit{main}|]{|\textit{prefix}|}{|\textit{dest}|}|
\end{tabular}
\end{center}
%
the destination file is determined by a pattern
depending on the current file:
To make this work, the current file must be called
`{\textit{prefix}\hspace{0.2em}\textit{suffix}}'
with \textit{prefix} matching precisely the argument.
Processing is then passed on to the file
`{\textit{dest}\hspace{0.2em}\textit{suffix}}'.
Surely, the same effect is achieved by
directly specifying the
argument `{\textit{dest}\hspace{0.2em}\textit{suffix}}'
in the first form.
However, that requires to set up a different file
for each child. With the alternative form of the command
all these files can have exactly the same content
which simplifies setting them up and maintaining them.

For example, the following file |draft.tex|
with a compilation flag |\version| as described in \secref{sec:flags}
compiles the main document as a draft:
%
\begin{center}
\begin{tabular}{l}
|\def\version{draft}|\\
|\input{childdoc.def}|\\
|\childdocforward{|\textit{main}|}|
\end{tabular}
\end{center}
%
Likewise, the following files |final|\textit{nn}|.tex|
compile the final version of the child document
|child|\textit{nn}|.tex|:
%
\begin{center}
\begin{tabular}{l}
|\def\version{final}|\\
|\input{childdoc.def}|\\
|\childdocforwardprefix{final}{child}|
\end{tabular}
\end{center}
%

Note that when several versions of a main file and/or of each child file
are to be generated, it may be convenient to set up a |Makefile| or
shell script to automatise the process.

%%%%%%%%%%%%%%%%%%%%%%%%%%%%%%%%%%%%%%%%%%%%%%%%%%%%%%%%%%%%%%%%%%%%%%%%%%%%%%%%
\subsection{Command Line Processing}
\label{sec:commandline}

The effect of redirection files can also be achieved by invoking
the \LaTeX{} compiler with a more elaborate command line.
Most conveniently this should be done as part
of a shell script or a |Makefile|.

When using \textsf{childdoc} in the main file, the following
command lines effectively perform a redirection
(note that depending on the shell being used,
backslashes may have to be doubled: `|\|' $\to$ `|\\|'):
%
\begin{center}
|... -jobname "|\textit{target}|" |\\|"|[\textit{flags}]%
|\input{childdoc.def}\childdocforward[|\textit{main}|]{|\textit{dest}|}"|
\end{center}
%
Here \textit{target} is the name of the output file,
\textit{main} is the name of the main file
and \textit{dest} is the name of the main or child file to be processed
(all filenames without extensions).
The optional argument \textit{main} can be omitted
if \textit{main} matches \textit{dest}.
Optionally, compilation \textit{flags} can be defined via |\def| commands.
This command line makes the \TeX{} engine believe
it is compiling the file \textit{target}
whose content is specified as the latter parameter.
The provided code then forwards the processing to
\textit{main} or \textit{dest} as described in \secref{sec:forward}.

%%%%%%%%%%%%%%%%%%%%%%%%%%%%%%%%%%%%%%%%%%%%%%%%%%%%%%%%%%%%%%%%%%%%%%%%%%%%%%%%
\subsection{Include by Input}
\label{sec:input}

Including child documents by |\include| has some restrictions by design.
Most notably, the content of a child document always occupies
its own set of pages; pages cannot be shared between child documents.
Usually, this behaviour makes perfect sense
because each child document contain an essential part of the document.
However, in some situations it may be desirable to compose
a document from a collection of parts
without having mandatory page breaks between then.
For this case, the package
provides a mechanism to include parts
by |\input| which can also be processed individually.
However, by construction this mechanism
requires manual handling of the content to be output.

%%%%%%%%%%%%%%%%%%%%%%%%%%%%%%%%%%%%%%%%
\DescribeMacro{\ifchilddocmanual}
The main file should be prepared as usual, see \secref{sec:include}.
However, the document body must make a distinction
between processing of an individual part and of the main document, e.g.:
%
\begin{center}
\begin{tabular}{l}
|\ifchilddocmanual|\\
|\input{\childdocname}|\\
|\||else|\\
\textit{document body with }|\input{|\textit{part}|}|\\
|\||fi|
\end{tabular}
\end{center}
%
The conditional |\ifchilddocmanual| is true whenever
a part to be included by |\input| is being compiled,
and the name of the part is stored in |\childdocname|.

%%%%%%%%%%%%%%%%%%%%%%%%%%%%%%%%%%%%%%%%
\DescribeMacro{\childdocby}
Each part to be included by |\input| should start with:
%
\begin{center}
\begin{tabular}{l}
|\input{childdoc.def}|\\
|\childdocby{|\textit{main}|}|\\
\end{tabular}
\end{center}
%
The directive |\childdocby| is similar to |\childdocof|
described in \secref{sec:include},
but the subsequent selection of content must be done manually.
To that end, both |\ifchilddoc| and |\ifchilddocmanual|
will be true upon processing of a part,
and the name of the part is stored in |\childdocname|.
Note that |\jobname| will be set to the filename of the current part
so that each part receives an individual |.aux| file
that does not interfere with the |.aux| file(s) of the main document.
This behaviour can be altered by the alternative form
|\childdocby[*]{|\textit{main}|}| (with a non-empty optional argument)
which uses the |.aux| file of the main document
by setting |\jobname| to \textit{main}.

%%%%%%%%%%%%%%%%%%%%%%%%%%%%%%%%%%%%%%%%%%%%%%%%%%%%%%%%%%%%%%%%%%%%%%%%%%%%%%%%
\subsection{Driver Development}
\label{sec:driver}

The \textsf{childdoc} mechanism can also be use for the development
of definition files such as \LaTeX{} styles or classes.
This case differs from the above setup with multiple parts
included by |\include| in that no |\includeonly| should be invoked.
This can be achieved by starting the include file
(before |\ProvidesPackage|) with:
%
\begin{center}
\begin{tabular}{l}
|\input{childdoc.def}|\\
|\childdocforward{|\textit{main}|}|\\
\end{tabular}
\end{center}
%
or alternatively with:
%
\begin{center}
\begin{tabular}{l}
|\input{childdoc.def}|\\
|\childdocby{|\textit{main}|}|\\
\end{tabular}
\end{center}
%
Both forms have slightly different effects as described above.
The main file is prepared as usual, see \secref{sec:include}.

%%%%%%%%%%%%%%%%%%%%%%%%%%%%%%%%%%%%%%%%%%%%%%%%%%%%%%%%%%%%%%%%%%%%%%%%%%%%%%%%
\subsection{Legacy Detection}
\label{sec:detection}

The directive |\childdocmain| in the main file can detect
whether the complete document or merely a child is to be compiled
even without using the directive |\childdocof|.
This method is deprecated because it is less robust
and there is no compelling reason to use it;
it is merely provided for backward compatibility
and it may be removed in future versions.

If the detection mechanism is to be used,
it is mandatory to correctly specify
the filename of the main file as the argument of |\childdocmain|:
%
\begin{center}
\begin{tabular}{l}
|\input{childdoc.def}|\\
|\childdocmain{|\textit{main}|}|\\
\end{tabular}
\end{center}
%
If |\jobname| does not match the argument \textit{main} of |\childdocmain|,
it is assumed that |\jobname| points to the child file to be compiled.
When using |\childdocmain| with the main file specified as argument,
it suffices to start a child file
with just |\input{|\textit{main}|}|
without loading of the package and using |\childdocof|.
If instead all processing is done
with the appropriate \textsf{childdoc} directives,
the argument of \textit{main} of |\childdocmain| can be empty.

An alternative version of the command line processing described
in \secref{sec:commandline} using the detection mechanism reads:
%
\begin{center}
|... -jobname "|\textit{target}|" "|[\textit{flags}]%
[|\def\jobname{|\textit{dest}|}|]|\input{|\textit{main}|}"|
\end{center}

%%%%%%%%%%%%%%%%%%%%%%%%%%%%%%%%%%%%%%%%%%%%%%%%%%%%%%%%%%%%%%%%%%%%%%%%%%%%%%%%
\subsection{Manual Code}
\label{sec:manual}

In case one cannot be certain whether the definitions file |childdoc.def|
is installed on the target \TeX{} distribution
and one prefers not to ship it,
it is conceivable to paste a few relevant commands into the sources.

To that end, drop all statements |\input{childdoc.def}|
and perform the replacements as outlined below.
Instead of |\childdocmain{|\textit{main}|}| add the following code
to the top of the main file:
%
\begin{center}
\begin{tabular}{l}
|\||ifdefined\childdocname\endinput\||fi\newif\ifchilddoc|\\
|\edef\childdocname{\scantokens\expandafter{\jobname\noexpand}}|\\
|\def\childdocmain{|\textit{main}|}\||ifx\childdocmain\childdocname\||else|\\
|\childdoctrue\includeonly{\childdocname}\let\jobname\childdocmain\||fi|\\
\end{tabular}
\end{center}
%
Instead of |\childdocof{|\textit{main}|}| just include the main file
at the top of each child file:
%
\begin{center}
|\input{|\textit{main}|}|
\end{center}
%
A simple redirection |\childdocforward{|\textit{dest}|}| is achieved by:
%
\begin{center}
|\def\jobname{|\textit{dest}|}\input{\jobname}|
\end{center}
%
The redirection with prefix
|\childdocforwardprefix[|\textit{prefix}|]{|\textit{dest}|}|
is accomplished by:
%
\begin{center}
\begin{tabular}{l}
|{\edef\jobname{\scantokens\expandafter{\jobname\noexpand}}|\\
|\def\redirectjob |\textit{prefix}|#1~~~{\gdef\jobname{|\textit{dest}|#1}}|\\
|\expandafter\redirectjob\jobname~~~}\input{\jobname}|
\end{tabular}
\end{center}

In an alternative approach,
child documents can be compiled by a specific command line
without additional code or specific definitions:
%
\begin{center}
|... -jobname "|\textit{target}|" "|[\textit{flags}]%
|\includeonly{|\textit{dest}|}\input{|\textit{main}|}"|
\end{center}
%

%%%%%%%%%%%%%%%%%%%%%%%%%%%%%%%%%%%%%%%%%%%%%%%%%%%%%%%%%%%%%%%%%%%%%%%%%%%%%%%%
%%%%%%%%%%%%%%%%%%%%%%%%%%%%%%%%%%%%%%%%%%%%%%%%%%%%%%%%%%%%%%%%%%%%%%%%%%%%%%%%
\section{Information}

%%%%%%%%%%%%%%%%%%%%%%%%%%%%%%%%%%%%%%%%%%%%%%%%%%%%%%%%%%%%%%%%%%%%%%%%%%%%%%%%
\subsection{Copyright}

Copyright \copyright{} 2017--2018 Niklas Beisert

This work may be distributed and/or modified under the
conditions of the \LaTeX{} Project Public License, either version 1.3
of this license or (at your option) any later version.
The latest version of this license is in
  \url{http://www.latex-project.org/lppl.txt}
and version 1.3 or later is part of all distributions of \LaTeX{}
version 2005/12/01 or later.

This work has the LPPL maintenance status `maintained'.

The Current Maintainer of this work is Niklas Beisert.

This work consists of the files |README.txt|, |childdoc.ins| and |childdoc.dtx|
as well as the derived files |childdoc.def|, |cdocsamp.tex|
with |cdocsch1.tex|, |cdocsch2.tex|, |cdocspt3.tex|, |cdocspt4.tex|,
|cdocsdrf.tex|, |cdocsfn1.tex|, |cdocsfn2.tex|
as well as |childdoc.pdf|.

%%%%%%%%%%%%%%%%%%%%%%%%%%%%%%%%%%%%%%%%%%%%%%%%%%%%%%%%%%%%%%%%%%%%%%%%%%%%%%%%
\subsection{Files and Installation}

The package consists of the files:
%
\begin{center}
\begin{tabular}{ll}
    |README.txt|   & readme file \\
    |childdoc.ins| & installation file \\
    |childdoc.dtx| & source file \\
    |childdoc.def| & definition file \\
    |cdocsamp.tex| & sample main file \\
    |cdocsch1.tex| & sample include file \\
    |cdocsch2.tex| & sample include file \\
    |cdocspt3.tex| & sample part file \\
    |cdocspt4.tex| & sample part file \\
    |cdocsdrf.tex| & sample redirection file \\
    |cdocsfn1.tex| & sample redirection file \\
    |cdocsfn2.tex| & sample redirection file \\
    |childdoc.pdf| & manual
\end{tabular}
\end{center}
%
The distribution consists of the files
|README.txt|, |childdoc.ins| and |childdoc.dtx|.
%
\begin{itemize}
\item
Run (pdf)\LaTeX{} on |childdoc.dtx|
to compile the manual |childdoc.pdf| (this file).
\item
Run \LaTeX{} on |childdoc.ins| to create the definitions file |childdoc.def|
and the sample |cdocsamp.tex| with include files
|cdocsch1.tex|, |cdocsch2.tex|, |cdocspt3.tex|, |cdocspt4.tex|,
|cdocsdrf.tex|, |cdocsfn1.tex|, |cdocsfn2.tex|.
Then copy the file |childdoc.def| to an appropriate directory of your \LaTeX{}
distribution, e.g.\ \textit{texmf-root}|/tex/latex/childdoc|.
\end{itemize}

%%%%%%%%%%%%%%%%%%%%%%%%%%%%%%%%%%%%%%%%%%%%%%%%%%%%%%%%%%%%%%%%%%%%%%%%%%%%%%%%
\subsection{Related CTAN Packages}

There are several other packages which offer a similar functionality:
%
\begin{itemize}
\item
The packages
\href{http://ctan.org/pkg/docmute}{\textsf{docmute}},
\href{http://ctan.org/pkg/includex}{\textsf{includex}} and
\href{http://ctan.org/pkg/standalone}{\textsf{standalone}}
provide commands to include only the document body of
a child file thus allowing both files to be compiled individually.
\item
The packages \href{http://ctan.org/pkg/subdocs}{\textsf{subdocs}}
and \href{http://ctan.org/pkg/subfiles}{\textsf{subfiles}}
provide structures in which the main and child documents can be
encapsulated and allowing them to be compiled individually.
The inclusion mechanism is different from the conventional |\include|.
\item
The package \href{http://ctan.org/pkg/combine}{\textsf{combine}}
is an elaborate solution to combine several documents into one.
\end{itemize}
%
See also the CTAN topic \href{http://ctan.org/topic/subdocs}{\textsf{subdocs}}
for further related packages.
The present package differs from the above solutions in that
a document structure constructed with the conventional |\include| mechanism
just needs two extra commands at the top of every file
such that all constituent files can be compiled individually.

%%%%%%%%%%%%%%%%%%%%%%%%%%%%%%%%%%%%%%%%%%%%%%%%%%%%%%%%%%%%%%%%%%%%%%%%%%%%%%%%
%\subsection{Feature Suggestions}
%
%The following is a list of features which may be useful for future
%versions of this package:
%%
%\begin{itemize}
%\item
%\ldots
%\end{itemize}

%%%%%%%%%%%%%%%%%%%%%%%%%%%%%%%%%%%%%%%%%%%%%%%%%%%%%%%%%%%%%%%%%%%%%%%%%%%%%%%%
\subsection{Revision History}

%%%%%%%%%%%%%%%%%%%%%%%%%%%%%%%%%%%%%%%%
\paragraph{v2.0:} 2018/12/30

\begin{itemize}
\item
immediate forward processing
\item
added |\childdocby| mechanism
\item
manual restructured
\end{itemize}

%%%%%%%%%%%%%%%%%%%%%%%%%%%%%%%%%%%%%%%%
\paragraph{v1.6:} 2018/01/17

\begin{itemize}
\item
application for development of include files
\item
corrections to manual
\end{itemize}

%%%%%%%%%%%%%%%%%%%%%%%%%%%%%%%%%%%%%%%%
\paragraph{v1.5:} 2017/05/21

\begin{itemize}
\item
more complete structuring introduced
\item
|\childdocof| introduced
\item
|\childdoc| renamed to |\childdocmain|
\item
|\childredirect| renamed to |\childdocforward| and |\childdocforwardprefix|
and functionality expanded
\end{itemize}

%%%%%%%%%%%%%%%%%%%%%%%%%%%%%%%%%%%%%%%%
\paragraph{v1.0:} 2017/04/27

\begin{itemize}
\item
manual and install package
\item
first version published on CTAN
\end{itemize}

%%%%%%%%%%%%%%%%%%%%%%%%%%%%%%%%%%%%%%%%
\paragraph{v0.6:} 2017/04/26

\begin{itemize}
\item
redirection mechanism added
\end{itemize}

%%%%%%%%%%%%%%%%%%%%%%%%%%%%%%%%%%%%%%%%
\paragraph{v0.5:} 2017/04/26

\begin{itemize}
\item
functionality in definition file
\end{itemize}


%%%%%%%%%%%%%%%%%%%%%%%%%%%%%%%%%%%%%%%%%%%%%%%%%%%%%%%%%%%%%%%%%%%%%%%%%%%%%%%%
%%%%%%%%%%%%%%%%%%%%%%%%%%%%%%%%%%%%%%%%%%%%%%%%%%%%%%%%%%%%%%%%%%%%%%%%%%%%%%%%
%%%%%%%%%%%%%%%%%%%%%%%%%%%%%%%%%%%%%%%%%%%%%%%%%%%%%%%%%%%%%%%%%%%%%%%%%%%%%%%%
\appendix

\settowidth\MacroIndent{\rmfamily\scriptsize 000\ }

 \DocInput{childdoc.dtx}

\end{document}
%</driver>
% \fi
%
% %%%%%%%%%%%%%%%%%%%%%%%%%%%%%%%%%%%%%%%%%%%%%%%%%%%%%%%%%%%%%%%%%%%%%%%%%%%%%%
% %%%%%%%%%%%%%%%%%%%%%%%%%%%%%%%%%%%%%%%%%%%%%%%%%%%%%%%%%%%%%%%%%%%%%%%%%%%%%%
% \section{Sample}
%\iffalse
%<*samplemain>
%\fi
%
% The following presents a sample document
% with two chapters, two parts, a title page,
% a compile flag as well as three forwarding files to set the flag.
% It consists of eight |.tex| files:
% \begin{center}
% \begin{tabular}{ll}
% |cdocsamp.tex|&main file\\
% |cdocsch1.tex|&include file for chapter 1\\
% |cdocsch2.tex|&include file for chapter 2\\
% |cdocspt3.tex|&include file for part 3\\
% |cdocspt4.tex|&include file for part 4\\
% |cdocsdrf.tex|&forwarding file for main file in draft mode\\
% |cdocsfi1.tex|&forwarding file for final version of chapter 1\\
% |cdocsfi2.tex|&forwarding file for final version of chapter 2\\
% \end{tabular}
% \end{center}
% Each of the eight files can be compiled directly by the \LaTeX{} compiler.
%
% %%%%%%%%%%%%%%%%%%%%%%%%%%%%%%%%%%%%%%
% \paragraph{Main File.}
%
% The main file is called |cdocsamp.tex|.
%
% Load the \textsf{childdoc} definitions and
% declare the filename for the main document:
%    \begin{macrocode}
\input{childdoc.def}
\childdocmain{}
%    \end{macrocode}

% Optional override for |\version| flag:
%    \begin{macrocode}
%%\ifchilddoc\else\providecommand{\version}{draft}\fi
%    \end{macrocode}

% Define the default values for the |\version| flag
% (|final| for the main file and |draft| for childs):
%    \begin{macrocode}
\ifchilddoc
\providecommand{\version}{draft}
\else
\providecommand{\version}{final}
\fi
%    \end{macrocode}

% Load the standard document class:
%    \begin{macrocode}
\documentclass[12pt]{article}
%    \end{macrocode}

% Start the document body:
%    \begin{macrocode}
\begin{document}
%    \end{macrocode}

% Declare a title page.
% Print title, part of document being processed and version flag:
%    \begin{macrocode}
\addtocounter{page}{-1}
\begin{center}
{\LARGE\bfseries{}childdoc example\par}
\vspace{1cm}
\ifchilddoc
\ifchilddocmanual part\else chapter\fi:
`\childdocname' of `\childdocjob'\par
\else
main document: `\childdocjob'\par
\fi
version: \version\par
\end{center}
\newpage
%    \end{macrocode}

% Manually include selected file,
% otherwise process as usual:
%    \begin{macrocode}
\ifchilddocmanual
\section*{part `\childdocname'}
\input{\childdocname}
\else
%    \end{macrocode}

% Include the two chapters:
%    \begin{macrocode}
\include{cdocsch1}
\include{cdocsch2}
%    \end{macrocode}

% Include the two parts unless only chapters should be displayed:
%    \begin{macrocode}
\ifchilddoc\else
\section{part three}
\input{cdocspt3}
\section{part four}
\input{cdocspt4}
\fi
%    \end{macrocode}

% Process as usual until here:
%    \begin{macrocode}
\fi
%    \end{macrocode}

% End of document body:
%    \begin{macrocode}
\end{document}
%    \end{macrocode}
%\iffalse
%</samplemain>
%\fi
%
% %%%%%%%%%%%%%%%%%%%%%%%%%%%%%%%%%%%%%%
% \paragraph{Chapter Include Files.}
%
% The include files are called |cdocsch1.tex| and |cdocsch2.tex|.
%
%\iffalse
%<*samplechap1|samplechap2>
%\fi

% Optional override for |\version| flag:
%    \begin{macrocode}
%%\providecommand{\version}{final}
%    \end{macrocode}

% Include the main document:
%    \begin{macrocode}
\input{childdoc.def}
\childdocof{cdocsamp}
%    \end{macrocode}

%\iffalse
%</samplechap1|samplechap2>
%\fi
%
%\iffalse
%<*samplechap1>
%\fi
% Some text for chapter 1:
%    \begin{macrocode}
\section{one}
some text in chapter one
%    \end{macrocode}

%\iffalse
%</samplechap1>
%\fi
% Some text for chapter 2:
%\iffalse
%<*samplechap2>
%\fi
%    \begin{macrocode}
\section{two}
more text in chapter two
%    \end{macrocode}

%\iffalse
%</samplechap2>
%\fi
%
% %%%%%%%%%%%%%%%%%%%%%%%%%%%%%%%%%%%%%%
% \paragraph{Part Include Files.}
%
% The include files are called |cdocspt3.tex| and |cdocspt4.tex|.
%
%\iffalse
%<*samplepart3|samplepart4>
%\fi

% Optional override for |\version| flag:
%    \begin{macrocode}
%%\providecommand{\version}{final}
%    \end{macrocode}

% Include the main document:
%    \begin{macrocode}
\input{childdoc.def}
\childdocby{cdocsamp}
%    \end{macrocode}

%\iffalse
%</samplepart3|samplepart4>
%\fi
%
%\iffalse
%<*samplepart3>
%\fi
% Some text for part 3:
%    \begin{macrocode}
some text in part three
%    \end{macrocode}

%\iffalse
%</samplepart3>
%\fi
% Some text for part 4:
%\iffalse
%<*samplepart4>
%\fi
%    \begin{macrocode}
more text in part four
%    \end{macrocode}

%\iffalse
%</samplepart4>
%\fi
%
% %%%%%%%%%%%%%%%%%%%%%%%%%%%%%%%%%%%%%%
% \paragraph{Forwarding for a Complete Draft.}
%
% The following forwarding file |cdocsdrf.tex|
% compiles the main document in draft mode:
%\iffalse
%<*sampledraft>
%\fi
%    \begin{macrocode}
\def\version{draft}
\input{childdoc.def}
\childdocforward{cdocsamp}
%    \end{macrocode}

%\iffalse
%</sampledraft>
%\fi
%
% %%%%%%%%%%%%%%%%%%%%%%%%%%%%%%%%%%%%%%
% \paragraph{Forwarding for Final Version of the Chapters.}
%
% The following forwarding files |cdocsfn1.tex| and |cdocsfn2.tex|
% (with identical content)
% compile the final versions of the child documents
% |cdocsch1.tex| and |cdocsch2.tex|, respectively:
%\iffalse
%<*samplefinal>
%\fi
%    \begin{macrocode}
\def\version{final}
\input{childdoc.def}
\childdocforwardprefix[cdocsamp]{cdocsfn}{cdocsch}
%    \end{macrocode}

%\iffalse
%</samplefinal>
%\fi
%
% %%%%%%%%%%%%%%%%%%%%%%%%%%%%%%%%%%%%%%
% \paragraph{Command Line Processing.}
%
% The following three command lines generate the output files
% |cdocscld|, |cdocscl1| and |cdocscl2|
% which should be identical to
% |cdocsdrf|, |cdocsch1| and |cdocsfn2|, respectively:
% \begin{center}
% \begin{tabular}{l}
% |latex -jobname cdocscld \|\\
% |  "\def\version{draft}\input{childdoc.def}\childdocforward{cdocsamp}"|\\
% |latex -jobname cdocscl1 \|\\
% |  "\input{childdoc.def}\childdocforward[cdocsamp]{cdocsch1}"|\\
% |latex -jobname cdocscl2 \|\\
% |  "\def\version{final}\input{childdoc.def}\childdocforward{cdocsch2}"|
% \end{tabular}
% \end{center}
% Note that the trailing backslash on each first line
% merely continues the input to the second line
% (for convenient cut ant paste).
% Furthermore, the command |latex| can be replaced by any
% of its alternative versions such as |pdflatex|.
%
% %%%%%%%%%%%%%%%%%%%%%%%%%%%%%%%%%%%%%%%%%%%%%%%%%%%%%%%%%%%%%%%%%%%%%%%%%%%%%%
% %%%%%%%%%%%%%%%%%%%%%%%%%%%%%%%%%%%%%%%%%%%%%%%%%%%%%%%%%%%%%%%%%%%%%%%%%%%%%%
% \section{Implementation}
%\iffalse
%<*package>
%\fi
%
% This section describes the definitions file |childdoc.def|.

% The definitions cannot be loaded using |\usepackage| or |\RequirePackage|
% which has a mechanism to prevent loading a style file more than once.
% When loading the definitions by means of |\input|
% multiple instances have to be prevented manually:
%\iffalse
%This code needs to be before the `\ProvidesFile' directive
%which is defined at the beginning of this file.
%Therefore it is also placed there and commented out here.
%</package>
%<*discard>
%\fi
%    \begin{macrocode}
\ifdefined\childdocmain\endinput\fi
%    \end{macrocode}
%\iffalse
%</discard>
%<*package>
%\fi
%
% \macro{\ifchilddoc}
% \macro{\ifchilddocmanual}
% The conditional |\ifchilddoc| tells whether a
% child (true) or main (false) document is being compiled.
% The conditional |\ifchilddocmanual| tells whether
% the |\includeonly| mechanism is used (false) or
% the selection of child files must be performed manually (true).
% The definitions initialise to false:
%    \begin{macrocode}
\newif\ifchilddoc
\newif\ifchilddocmanual
%    \end{macrocode}

% \macro{\childdocname}
% \macro{\childdocjob}
% The macro |\childdocname| stores the name of the main document
% to be compiled. The macro |\childdocjob| stores the name of
% the document on which the \LaTeX{} compiler was originally invoked.
% The content of |\jobname| cannot be compared
% to filenames specified in the source due to different catcodes.
% The following code rescans |\jobname|, stores the result
% in |\childdocname| and saves a copy in |\childdocjob|:
%    \begin{macrocode}
\edef\childdocname{\scantokens\expandafter{\jobname\noexpand}}
\let\childdocjob\childdocname
%    \end{macrocode}

% \macro{\childdocdisable}
% The macro |\childdocdisable| prevents the main file
% from being processed more than once.
% At this stage, the main document command |\childdocmain|
% is assumed to be called once again where it should do nothing.
% Any subsequent call to it should prevent
% a secondary processing of the main document
% It overwrites the forwarding commands
% |\childdocof| and |\childdocforward|
% with empty macros to prevent further inclusions of the main document:
%    \begin{macrocode}
\newcommand{\childdocdisable}
{
  \renewcommand{\childdocmain}[1]{\renewcommand{\childdocmain}[1]{\endinput}}
  \renewcommand{\childdocof}[1]{}
  \renewcommand{\childdocby}[2][]{}
  \renewcommand{\childdocforward}[2][]{}
  \renewcommand{\childdocdisable}{}
}
%    \end{macrocode}

% \macro{\childdocmain}
% The macro |\childdocmain| is to be called at the top of the main file
% with nothing or the main filename (without extension) as argument.
% First, it breaks loops.
% If the argument is not empty and does not match |\childdocname|
% (which is set by the first inclusion of |childdoc.def|),
% |\ifchilddoc| is set to true, |\includeonly| is applied to the child file
% and |\jobname| is set to the main file
% (for proper handling of |.aux| files):
%    \begin{macrocode}
\newcommand{\childdocmain}[1]
{
  \childdocdisable\childdocmain{}
  \if?#1?\else
    \begingroup
      \def\childdoctmp{#1}
      \ifx\childdoctmp\childdocname
        \def\childdoctmp{}
      \else
        \def\childdoctmp
        {
          \childdoctrue
          \includeonly{\childdocname}
          \def\childdocjob{#1}
          \def\jobname{#1}
        }
      \fi
      \expandafter
    \endgroup
    \childdoctmp
  \fi
}
%    \end{macrocode}

% \macro{\childdocof}
% The command |\childdocof| redirects
% compilation to the main file |#1|.
%    \begin{macrocode}
\newcommand{\childdocof}[1]
{
  \childdocdisable
  \childdoctrue
  \includeonly{\childdocname}
  \def\jobname{#1}
  \def\childdocjob{#1}
  \input{#1}
}
%    \end{macrocode}

% \macro{\childdocby}
% The command |\childdocby| ....
%    \begin{macrocode}
\newcommand{\childdocby}[2][]
{
  \childdocdisable
  \childdoctrue
  \childdocmanualtrue
  \if?#1?\else
    \def\jobname{#2}
  \fi
  \def\childdocjob{#2}
  \input{#2}
  \endinput
}
%    \end{macrocode}

% \macro{\childdocforward}
% The command |\childdocforward| redirects
% compilation to the main file or
% (if the optional argument is given) a child file.
% Parameters are set as if the main file
% or a child file starting with |\childdocof| was compiled.
% Then compilation is handed over to the main file:
%    \begin{macrocode}
\newcommand{\childdocforward}[2][]
{
  \begingroup
    \if?#1?
      \def\childdoctmp
      {
        \def\childdocname{#2}
        \def\childdocjob{#2}
        \def\jobname{#2}
        \input{#2}
        \endinput
      }
    \else
      \def\childdoctmp
      {
        \childdocdisable
        \def\childdocname{#2}
        \childdoctrue
        \includeonly{#2}
        \def\childdocjob{#1}
        \def\jobname{#1}
        \input{#1}
        \endinput
      }
    \fi
    \expandafter
  \endgroup
  \childdoctmp
}
%    \end{macrocode}

% \macro{\childdocforwardprefix}
% The command |\childdocforwardprefix| redirects
% compilation to the main or a child file by means of a pattern.
% The prefix |#1| in the current filename is replaced by |#2|
% and the suffix of the current filename is kept
% (it is assumed that the filename does not contain the substring `|~~~|'
% which is used as a delimiter).
% Compilation is handed over to the new file by |\childdocforward|:
%    \begin{macrocode}
\newcommand{\childdocforwardprefix}[3][]
{
  \begingroup
    \def\childdocextract #2##1~~~{\def\childdoctmp{\childdocforward[#1]{#3##1}}}
    \expandafter\childdocextract\childdocname~~~
    \expandafter
  \endgroup
  \childdoctmp
}
%    \end{macrocode}

% \macro{\childdoc}
% The deprecated macro |\childdoc| is a legacy version of |\childdocmain|:
%    \begin{macrocode}
\newcommand{\childdoc}{\childdocmain}
%    \end{macrocode}

% \macro{\childdocredirect}
% The deprecated macro |\childdocredirect| is a legacy version
% of |\childdocforward| and |\childdocforwardprefix|:
%    \begin{macrocode}
\newcommand{\childdocredirect}[2][]
{
  \begingroup
    \if?#1?
      \def\childdoctmp{\childdocforward{#2}}
    \else
      \def\childdoctmp{\childdocforwardprefix{#1}{#2}}
    \fi
    \expandafter
  \endgroup
  \childdoctmp
}
%    \end{macrocode}

%\iffalse
%</package>
%\fi
%
\endinput
\childdocforward[cdocsamp]{cdocsch1}"|\\
% |latex -jobname cdocscl2 \|\\
% |  "\def\version{final}% \iffalse
%
% childdoc.dtx Copyright (C) 2017-2018 Niklas Beisert
%
% This work may be distributed and/or modified under the
% conditions of the LaTeX Project Public License, either version 1.3
% of this license or (at your option) any later version.
% The latest version of this license is in
%   http://www.latex-project.org/lppl.txt
% and version 1.3 or later is part of all distributions of LaTeX
% version 2005/12/01 or later.
%
% This work has the LPPL maintenance status `maintained'.
%
% The Current Maintainer of this work is Niklas Beisert.
%
% This work consists of the files childdoc.dtx and childdoc.ins
% and the derived files childdoc.def and cdocsamp.tex with
% cdocsch1.tex, cdocsch2.tex, cdocsdrf.tex, cdocsfn1.tex, cdocsfn2.tex.
%
%<package>\ifdefined\childdocmain\endinput\fi
%<package>\ProvidesFile{childdoc.def}[2018/12/30 v2.0 child document driver]
%<samplemain>\ProvidesFile{cdocsamp.tex}[2018/12/30 v2.0 sample for childdoc]
%<*driver>
%\ProvidesFile{childdoc.drv}[2018/12/30 v2.0 childdoc reference manual file]
\PassOptionsToClass{10pt,a4paper}{article}
\documentclass{ltxdoc}

\usepackage[margin=35mm]{geometry}
\usepackage{hyperref}
\usepackage{hyperxmp}
\usepackage[usenames]{color}

\hypersetup{colorlinks=true}
\hypersetup{pdfstartview=FitH}
\hypersetup{pdfpagemode=UseNone}
\hypersetup{pdfsource={}}
\hypersetup{pdflang={en-UK}}
\hypersetup{pdfcopyright={Copyright 2017-2018 Niklas Beisert.
  This work may be distributed and/or modified under the
  conditions of the LaTeX Project Public License, either version 1.3
  of this license or (at your option) any later version.}}
\hypersetup{pdflicenseurl={http://www.latex-project.org/lppl.txt}}
\hypersetup{pdfcontactaddress={ETH Zurich, ITP, HIT K,
  Wolfgang-Pauli-Strasse 27}}
\hypersetup{pdfcontactpostcode={8093}}
\hypersetup{pdfcontactcity={Zurich}}
\hypersetup{pdfcontactcountry={Switzerland}}
\hypersetup{pdfcontactemail={nbeisert@itp.phys.ethz.ch}}
\hypersetup{pdfcontacturl={http://people.phys.ethz.ch/\xmptilde nbeisert/}}

\newcommand{\secref}[1]{\hyperref[#1]{section \ref*{#1}}}

\parskip1ex
\parindent0pt
\let\olditemize\itemize
\def\itemize{\olditemize\parskip0pt}

\begin{document}

\title{The \textsf{childdoc} Package}
\hypersetup{pdftitle={The childdoc Package}}
\author{Niklas Beisert\\[2ex]
  Institut f\"ur Theoretische Physik\\
  Eidgen\"ossische Technische Hochschule Z\"urich\\
  Wolfgang-Pauli-Strasse 27, 8093 Z\"urich, Switzerland\\[1ex]
  \href{mailto:nbeisert@itp.phys.ethz.ch}
  {\texttt{nbeisert@itp.phys.ethz.ch}}}
\hypersetup{pdfauthor={Niklas Beisert}}
\hypersetup{pdfsubject={Manual for the LaTeX2e Package childdoc}}
\date{30 December 2018, \textsf{v2.0}}
\maketitle

\begin{abstract}\noindent
\textsf{childdoc} is a \LaTeXe{} package
that enables the direct compilation
of document sections included by |\include|
to individual files.
\end{abstract}

\begingroup
\parskip0ex
\tableofcontents
\endgroup

%%%%%%%%%%%%%%%%%%%%%%%%%%%%%%%%%%%%%%%%%%%%%%%%%%%%%%%%%%%%%%%%%%%%%%%%%%%%%%%%
%%%%%%%%%%%%%%%%%%%%%%%%%%%%%%%%%%%%%%%%%%%%%%%%%%%%%%%%%%%%%%%%%%%%%%%%%%%%%%%%
\section{Introduction}

\LaTeX{} provides a mechanism to structure a large document (such as a book)
into a main file and several child files (containing the chapters)
using the |\include| command.
This mechanism is beneficial for documents
which span hundreds of pages in order to
make the source file(s) more manageable.
Moreover, compilation can be restricted to
selected child files by means of the |\includeonly| command.
The latter feature can be used to reduce the compilation time while editing
(this was significantly more useful in the earlier days of \LaTeX{})
or to generate a smaller document which is easier to navigate.
Another application of |\includeonly| is to generate
documents consisting of selected parts of the complete document.

However, there are a few drawbacks of the plain |\include| mechanism:
\begin{itemize}
\item
The child files cannot be compiled on their own,
they can only be compiled via the main file.
A naive editing environment
(such as a text editor with an option
to have the current file processed by \LaTeX)
may require one to switch to the main file before compiling;
attempting to compile the child file produces errors.
\item
The main file must be modified (each time)
to adjust the |\includeonly| command
to the present needs. This easily leaves the main file in a messy state.
\item
The generated document will always carry the filename
of the main document. This is inconvenient if
several child files are to be compiled and
to be kept for distribution.
\end{itemize}

The present package provides a simple interface
to make child files individually compilable by \LaTeX{}.
Compiling a child file then has the same effect as compiling
the main file with an |\includeonly| command
to select the appropriate child.
Moreover the generated document will carry the name of the child
rather than the main file.
This resolves all three above issues.

This feature is meant to make the editing of books,
thesis documents and lecture notes somewhat more convenient.
However, the package can also be used efficiently for
composing a series of documents (such as exercise sheets)
which are typically distributed individually.
It then assists the author in generating the individual documents
(potentially in different versions)
as well as a document containing the collected series.
Another application is in developing style files
or other kinds of included material
where compilation of the style file could redirect
to a sample or test file.

%%%%%%%%%%%%%%%%%%%%%%%%%%%%%%%%%%%%%%%%%%%%%%%%%%%%%%%%%%%%%%%%%%%%%%%%%%%%%%%%
%%%%%%%%%%%%%%%%%%%%%%%%%%%%%%%%%%%%%%%%%%%%%%%%%%%%%%%%%%%%%%%%%%%%%%%%%%%%%%%%
\section{Usage}

First of all, the package \textsf{childdoc} is \emph{not} a standard
\LaTeXe{} |.sty| style file! Therefore it needs to be invoked in
a non-standard way.

%%%%%%%%%%%%%%%%%%%%%%%%%%%%%%%%%%%%%%%%%%%%%%%%%%%%%%%%%%%%%%%%%%%%%%%%%%%%%%%%
\subsection{Included Files}
\label{sec:include}

%%%%%%%%%%%%%%%%%%%%%%%%%%%%%%%%%%%%%%%%
\DescribeMacro{\childdocmain}
To use the package, add the commands
\begin{center}
\begin{tabular}{l}
|\input{childdoc.def}|\\
|\childdocmain{}|\\
\end{tabular}
\end{center}
at the very top of the main \LaTeX{} file,
in particular \emph{before} the |\documentclass| statement!
The argument of |\childdocmain| should be left empty
(but it must be present).

%%%%%%%%%%%%%%%%%%%%%%%%%%%%%%%%%%%%%%%%
\DescribeMacro{\childdocof}
Furthermore, add the commands
\begin{center}
\begin{tabular}{l}
|\input{childdoc.def}|\\
|\childdocof{|\textit{main}|}|\\
\end{tabular}
\end{center}
at the top of every child file \textit{child}
which is included by |\include{|\textit{child}|}|
from within the main file
(or at least for those files to be compiled individually).
The argument \textit{main} must be the filename of the main file.

There are a couple of
considerations in setting up the main and child documents:

%%%%%%%%%%%%%%%%%%%%%%%%%%%%%%%%%%%%%%%%
\paragraph{Restrictions.}

Please note the following restrictions:
\begin{itemize}
\item
|\childdocmain| must be called with one argument \textit{main}
to ensure compatibility with earlier version of the package.
It must either be empty (|\childdocmain{}|)
or precisely match the filename of the main file in which it is specified.
See \secref{sec:detection} for further information.
\item
The filename \textit{main} must be specified without the |.tex| extension.
\item
The filename \textit{main} is case sensitive
(even in case-insensitive file systems)
due to internal string comparison.
\item
The argument \textit{main} should be fully expanded, it cannot be a macro.
\item
Subdirectories and special characters should be avoided in filenames.
\item
The command |\childdocmain{|\textit{main}|}| must be followed by a whitespace.
It should not be followed immediately by another command
or by a comment mark `|%|'.
This is because the \TeX{} parser reads the token immediately following
the argument of |\childdocmain| and puts it
at the beginning of every child section;
however, a white\-space is ignored.
\end{itemize}

%%%%%%%%%%%%%%%%%%%%%%%%%%%%%%%%%%%%%%%%
\paragraph{Content of Main File.}

It is advisable to place all content in the child files included by |\include|.
Any output contained in the main file will appear in all child documents
unless suppressed manually;
it cannot be suppressed automatically by the |\includeonly| directive
and thus should normally be avoided.
A method to include some content in the main file
by means of conditional processing is described in \secref{sec:conditional}.

%%%%%%%%%%%%%%%%%%%%%%%%%%%%%%%%%%%%%%%%
\paragraph{Page Numbering.}

When only a part of the document is compiled,
the appropriate numbering of pages
(as well as other status parameters)
is determined from the |.aux| files.
The latter contain information from previous passes.
However this information needs to propagate through
all intermediate child documents.
Therefore the page numbering in child documents may well
be inconsistent until the complete document is compiled at least once.

A useful (if unconventional) way to always ensure a consistent
page numbering is to restart the numbering in each child document
and denote the pages by `\textit{child}|.|\textit{page}'
where \textit{child} represents the chapter/section number of the child file.
This can be achieved by the command
|\numberwithin{page}{|\textit{child}|}|
of the \textsf{amsmath} package
where \textit{child} can be |chapter| or |section|
depending on the chosen structuring.
Alternatively, one can modify the macro |\thepage| appropriately
and reset the counter |page| at the start of each child file.

%%%%%%%%%%%%%%%%%%%%%%%%%%%%%%%%%%%%%%%%%%%%%%%%%%%%%%%%%%%%%%%%%%%%%%%%%%%%%%%%
\subsection{Conditional Processing}
\label{sec:conditional}

The package provides a mechanism to compile different versions
of a document. To customise the versions further some conditional processing
can come in handy to distinguish which version is being compiled.
The package provides two macros to describe the compilation context:

%%%%%%%%%%%%%%%%%%%%%%%%%%%%%%%%%%%%%%%%
\DescribeMacro{\ifchilddoc}
The conditional |\ifchilddoc| distinguishes between the compilation of
child documents and the main document:
%
\begin{center}
|\ifchilddoc |\textit{child-code}| |[|\||else |\textit{main-code}]| \||fi|
\end{center}

%%%%%%%%%%%%%%%%%%%%%%%%%%%%%%%%%%%%%%%%
\DescribeMacro{\childdocname}
\DescribeMacro{\childdocjob}
The macro |\childdocname| contains the filename (without extension)
of the main or child file being processed.
Note that |\childdocjob| will always contain the name of the main file.

%%%%%%%%%%%%%%%%%%%%%%%%%%%%%%%%%%%%%%%%
\paragraph{Title Page.}

Conditional processing can be used to include a title or banner page
in the main document when proper precautions are taken.
Importantly, the code in the main file should ensure that the page counter
(as well as other status parameters which are stored in the |.aux| files)
takes the same value after the conditional processing.
Otherwise the page numbers may take divergent values
depending on which part is compiled.

For example, a title page could be declared by:
%
\begin{center}
\begin{tabular}{l}
|\ifchilddoc\||else|\\
|\addtocounter{page}{-1}|\\
\textit{code for title page}\\
|\newpage|\\
|\||fi|
\end{tabular}
\end{center}
%
A banner page for the child documents can be generated by:
%
\begin{center}
\begin{tabular}{l}
|\ifchilddoc|\\
|\addtocounter{page}{-1}|\\
\textit{code for banner page}\\
|\newpage|\\
|\||fi|
\end{tabular}
\end{center}
%
Here one could write a message such as:
\begin{center}
|This is the part \childdocname{} of \childdocjob{}.|
\end{center}

%%%%%%%%%%%%%%%%%%%%%%%%%%%%%%%%%%%%%%%%%%%%%%%%%%%%%%%%%%%%%%%%%%%%%%%%%%%%%%%%
\subsection{Flags}
\label{sec:flags}

The package makes it easy to generate different versions
of the main or child documents.
To this end compilation flags can be defined
and assigned different default values.
They will be particularly useful in conjunction
with the forwarding mechanism described in \secref{sec:forward}.

For example, it may be useful to have a flag |\version|
which can be set to |draft| or |final|.
The document source will contain some conditional code
depending on the value of |\version|.
Suppose further, the flag should default to |final| for the main file
and to |draft| for child files
which is a natural assignment for editing the document.
This is achieved by placing the following code
in the preamble of the main document
(below the |\childdocmain| directive):
%
\begin{center}
\begin{tabular}{l}
|\ifchilddoc|\\
|\providecommand{\version}{draft}|\\
|\||else|\\
|\providecommand{\version}{final}|\\
|\||fi|
\end{tabular}
\end{center}
%
The definition by |\providecommand| makes sure
that previous definitions are not overwritten.
Further statements |\providecommand{\version}{...}|
can thus be added before the above code to override it.

For the main file, one might add a line
(between |\childdocmain| and the above block)
%
\begin{center}
|%\ifchilddoc\||else\providecommand{\version}{draft}\||fi|
\end{center}
%
which can be uncommented to produce a draft version.
Likewise one can add a line to the very top of a child file
(above the |\childdocof{|\textit{main}|}| directive)
%
\begin{center}
|%\providecommand{\version}{final}|
\end{center}
%
which can be uncommented to produce the final version of this child document.

%%%%%%%%%%%%%%%%%%%%%%%%%%%%%%%%%%%%%%%%%%%%%%%%%%%%%%%%%%%%%%%%%%%%%%%%%%%%%%%%
\subsection{Forwarding}
\label{sec:forward}

Different versions of the main or child documents
using compilation flags as described in \secref{sec:flags}
can be (permanently) stored in different files
for convenient compilation, viewing and distribution.
To this end, the package defines a command
to pass on compilation to a different file:

%%%%%%%%%%%%%%%%%%%%%%%%%%%%%%%%%%%%%%%%
\DescribeMacro{\childdocforward}
The command |\childdocforward| redirects processing to
another source file:
%
\begin{center}
\begin{tabular}{l}
|\input{childdoc.def}|\\
|\childdocforward[|\textit{main}|]{|\textit{dest}|}|\\
\end{tabular}
\end{center}
%
The argument \textit{dest} is the destination file
(without extension).
It should be the main file or one of the child files.
Note that further \textsf{childdoc} directives
such as |\childdocof| and |\childdocforward|
in the indicated file will be processed in this form.
The optional argument \textit{main}
passes on directly to the main file \textit{main}
while pretending to compile the child \textit{dest}.
This form behaves as if \textit{dest}
issues |\childdocof{|\textit{main}|}| right away,
and no further \textsf{childdoc} directives will be processed.

%%%%%%%%%%%%%%%%%%%%%%%%%%%%%%%%%%%%%%%%
\DescribeMacro{\...prefix}
In the alternative form |\childdocforwardprefix|,
%
\begin{center}
\begin{tabular}{l}
|\input{childdoc.def}|\\
|\childdocforwardprefix[|\textit{main}|]{|\textit{prefix}|}{|\textit{dest}|}|
\end{tabular}
\end{center}
%
the destination file is determined by a pattern
depending on the current file:
To make this work, the current file must be called
`{\textit{prefix}\hspace{0.2em}\textit{suffix}}'
with \textit{prefix} matching precisely the argument.
Processing is then passed on to the file
`{\textit{dest}\hspace{0.2em}\textit{suffix}}'.
Surely, the same effect is achieved by
directly specifying the
argument `{\textit{dest}\hspace{0.2em}\textit{suffix}}'
in the first form.
However, that requires to set up a different file
for each child. With the alternative form of the command
all these files can have exactly the same content
which simplifies setting them up and maintaining them.

For example, the following file |draft.tex|
with a compilation flag |\version| as described in \secref{sec:flags}
compiles the main document as a draft:
%
\begin{center}
\begin{tabular}{l}
|\def\version{draft}|\\
|\input{childdoc.def}|\\
|\childdocforward{|\textit{main}|}|
\end{tabular}
\end{center}
%
Likewise, the following files |final|\textit{nn}|.tex|
compile the final version of the child document
|child|\textit{nn}|.tex|:
%
\begin{center}
\begin{tabular}{l}
|\def\version{final}|\\
|\input{childdoc.def}|\\
|\childdocforwardprefix{final}{child}|
\end{tabular}
\end{center}
%

Note that when several versions of a main file and/or of each child file
are to be generated, it may be convenient to set up a |Makefile| or
shell script to automatise the process.

%%%%%%%%%%%%%%%%%%%%%%%%%%%%%%%%%%%%%%%%%%%%%%%%%%%%%%%%%%%%%%%%%%%%%%%%%%%%%%%%
\subsection{Command Line Processing}
\label{sec:commandline}

The effect of redirection files can also be achieved by invoking
the \LaTeX{} compiler with a more elaborate command line.
Most conveniently this should be done as part
of a shell script or a |Makefile|.

When using \textsf{childdoc} in the main file, the following
command lines effectively perform a redirection
(note that depending on the shell being used,
backslashes may have to be doubled: `|\|' $\to$ `|\\|'):
%
\begin{center}
|... -jobname "|\textit{target}|" |\\|"|[\textit{flags}]%
|\input{childdoc.def}\childdocforward[|\textit{main}|]{|\textit{dest}|}"|
\end{center}
%
Here \textit{target} is the name of the output file,
\textit{main} is the name of the main file
and \textit{dest} is the name of the main or child file to be processed
(all filenames without extensions).
The optional argument \textit{main} can be omitted
if \textit{main} matches \textit{dest}.
Optionally, compilation \textit{flags} can be defined via |\def| commands.
This command line makes the \TeX{} engine believe
it is compiling the file \textit{target}
whose content is specified as the latter parameter.
The provided code then forwards the processing to
\textit{main} or \textit{dest} as described in \secref{sec:forward}.

%%%%%%%%%%%%%%%%%%%%%%%%%%%%%%%%%%%%%%%%%%%%%%%%%%%%%%%%%%%%%%%%%%%%%%%%%%%%%%%%
\subsection{Include by Input}
\label{sec:input}

Including child documents by |\include| has some restrictions by design.
Most notably, the content of a child document always occupies
its own set of pages; pages cannot be shared between child documents.
Usually, this behaviour makes perfect sense
because each child document contain an essential part of the document.
However, in some situations it may be desirable to compose
a document from a collection of parts
without having mandatory page breaks between then.
For this case, the package
provides a mechanism to include parts
by |\input| which can also be processed individually.
However, by construction this mechanism
requires manual handling of the content to be output.

%%%%%%%%%%%%%%%%%%%%%%%%%%%%%%%%%%%%%%%%
\DescribeMacro{\ifchilddocmanual}
The main file should be prepared as usual, see \secref{sec:include}.
However, the document body must make a distinction
between processing of an individual part and of the main document, e.g.:
%
\begin{center}
\begin{tabular}{l}
|\ifchilddocmanual|\\
|\input{\childdocname}|\\
|\||else|\\
\textit{document body with }|\input{|\textit{part}|}|\\
|\||fi|
\end{tabular}
\end{center}
%
The conditional |\ifchilddocmanual| is true whenever
a part to be included by |\input| is being compiled,
and the name of the part is stored in |\childdocname|.

%%%%%%%%%%%%%%%%%%%%%%%%%%%%%%%%%%%%%%%%
\DescribeMacro{\childdocby}
Each part to be included by |\input| should start with:
%
\begin{center}
\begin{tabular}{l}
|\input{childdoc.def}|\\
|\childdocby{|\textit{main}|}|\\
\end{tabular}
\end{center}
%
The directive |\childdocby| is similar to |\childdocof|
described in \secref{sec:include},
but the subsequent selection of content must be done manually.
To that end, both |\ifchilddoc| and |\ifchilddocmanual|
will be true upon processing of a part,
and the name of the part is stored in |\childdocname|.
Note that |\jobname| will be set to the filename of the current part
so that each part receives an individual |.aux| file
that does not interfere with the |.aux| file(s) of the main document.
This behaviour can be altered by the alternative form
|\childdocby[*]{|\textit{main}|}| (with a non-empty optional argument)
which uses the |.aux| file of the main document
by setting |\jobname| to \textit{main}.

%%%%%%%%%%%%%%%%%%%%%%%%%%%%%%%%%%%%%%%%%%%%%%%%%%%%%%%%%%%%%%%%%%%%%%%%%%%%%%%%
\subsection{Driver Development}
\label{sec:driver}

The \textsf{childdoc} mechanism can also be use for the development
of definition files such as \LaTeX{} styles or classes.
This case differs from the above setup with multiple parts
included by |\include| in that no |\includeonly| should be invoked.
This can be achieved by starting the include file
(before |\ProvidesPackage|) with:
%
\begin{center}
\begin{tabular}{l}
|\input{childdoc.def}|\\
|\childdocforward{|\textit{main}|}|\\
\end{tabular}
\end{center}
%
or alternatively with:
%
\begin{center}
\begin{tabular}{l}
|\input{childdoc.def}|\\
|\childdocby{|\textit{main}|}|\\
\end{tabular}
\end{center}
%
Both forms have slightly different effects as described above.
The main file is prepared as usual, see \secref{sec:include}.

%%%%%%%%%%%%%%%%%%%%%%%%%%%%%%%%%%%%%%%%%%%%%%%%%%%%%%%%%%%%%%%%%%%%%%%%%%%%%%%%
\subsection{Legacy Detection}
\label{sec:detection}

The directive |\childdocmain| in the main file can detect
whether the complete document or merely a child is to be compiled
even without using the directive |\childdocof|.
This method is deprecated because it is less robust
and there is no compelling reason to use it;
it is merely provided for backward compatibility
and it may be removed in future versions.

If the detection mechanism is to be used,
it is mandatory to correctly specify
the filename of the main file as the argument of |\childdocmain|:
%
\begin{center}
\begin{tabular}{l}
|\input{childdoc.def}|\\
|\childdocmain{|\textit{main}|}|\\
\end{tabular}
\end{center}
%
If |\jobname| does not match the argument \textit{main} of |\childdocmain|,
it is assumed that |\jobname| points to the child file to be compiled.
When using |\childdocmain| with the main file specified as argument,
it suffices to start a child file
with just |\input{|\textit{main}|}|
without loading of the package and using |\childdocof|.
If instead all processing is done
with the appropriate \textsf{childdoc} directives,
the argument of \textit{main} of |\childdocmain| can be empty.

An alternative version of the command line processing described
in \secref{sec:commandline} using the detection mechanism reads:
%
\begin{center}
|... -jobname "|\textit{target}|" "|[\textit{flags}]%
[|\def\jobname{|\textit{dest}|}|]|\input{|\textit{main}|}"|
\end{center}

%%%%%%%%%%%%%%%%%%%%%%%%%%%%%%%%%%%%%%%%%%%%%%%%%%%%%%%%%%%%%%%%%%%%%%%%%%%%%%%%
\subsection{Manual Code}
\label{sec:manual}

In case one cannot be certain whether the definitions file |childdoc.def|
is installed on the target \TeX{} distribution
and one prefers not to ship it,
it is conceivable to paste a few relevant commands into the sources.

To that end, drop all statements |\input{childdoc.def}|
and perform the replacements as outlined below.
Instead of |\childdocmain{|\textit{main}|}| add the following code
to the top of the main file:
%
\begin{center}
\begin{tabular}{l}
|\||ifdefined\childdocname\endinput\||fi\newif\ifchilddoc|\\
|\edef\childdocname{\scantokens\expandafter{\jobname\noexpand}}|\\
|\def\childdocmain{|\textit{main}|}\||ifx\childdocmain\childdocname\||else|\\
|\childdoctrue\includeonly{\childdocname}\let\jobname\childdocmain\||fi|\\
\end{tabular}
\end{center}
%
Instead of |\childdocof{|\textit{main}|}| just include the main file
at the top of each child file:
%
\begin{center}
|\input{|\textit{main}|}|
\end{center}
%
A simple redirection |\childdocforward{|\textit{dest}|}| is achieved by:
%
\begin{center}
|\def\jobname{|\textit{dest}|}\input{\jobname}|
\end{center}
%
The redirection with prefix
|\childdocforwardprefix[|\textit{prefix}|]{|\textit{dest}|}|
is accomplished by:
%
\begin{center}
\begin{tabular}{l}
|{\edef\jobname{\scantokens\expandafter{\jobname\noexpand}}|\\
|\def\redirectjob |\textit{prefix}|#1~~~{\gdef\jobname{|\textit{dest}|#1}}|\\
|\expandafter\redirectjob\jobname~~~}\input{\jobname}|
\end{tabular}
\end{center}

In an alternative approach,
child documents can be compiled by a specific command line
without additional code or specific definitions:
%
\begin{center}
|... -jobname "|\textit{target}|" "|[\textit{flags}]%
|\includeonly{|\textit{dest}|}\input{|\textit{main}|}"|
\end{center}
%

%%%%%%%%%%%%%%%%%%%%%%%%%%%%%%%%%%%%%%%%%%%%%%%%%%%%%%%%%%%%%%%%%%%%%%%%%%%%%%%%
%%%%%%%%%%%%%%%%%%%%%%%%%%%%%%%%%%%%%%%%%%%%%%%%%%%%%%%%%%%%%%%%%%%%%%%%%%%%%%%%
\section{Information}

%%%%%%%%%%%%%%%%%%%%%%%%%%%%%%%%%%%%%%%%%%%%%%%%%%%%%%%%%%%%%%%%%%%%%%%%%%%%%%%%
\subsection{Copyright}

Copyright \copyright{} 2017--2018 Niklas Beisert

This work may be distributed and/or modified under the
conditions of the \LaTeX{} Project Public License, either version 1.3
of this license or (at your option) any later version.
The latest version of this license is in
  \url{http://www.latex-project.org/lppl.txt}
and version 1.3 or later is part of all distributions of \LaTeX{}
version 2005/12/01 or later.

This work has the LPPL maintenance status `maintained'.

The Current Maintainer of this work is Niklas Beisert.

This work consists of the files |README.txt|, |childdoc.ins| and |childdoc.dtx|
as well as the derived files |childdoc.def|, |cdocsamp.tex|
with |cdocsch1.tex|, |cdocsch2.tex|, |cdocspt3.tex|, |cdocspt4.tex|,
|cdocsdrf.tex|, |cdocsfn1.tex|, |cdocsfn2.tex|
as well as |childdoc.pdf|.

%%%%%%%%%%%%%%%%%%%%%%%%%%%%%%%%%%%%%%%%%%%%%%%%%%%%%%%%%%%%%%%%%%%%%%%%%%%%%%%%
\subsection{Files and Installation}

The package consists of the files:
%
\begin{center}
\begin{tabular}{ll}
    |README.txt|   & readme file \\
    |childdoc.ins| & installation file \\
    |childdoc.dtx| & source file \\
    |childdoc.def| & definition file \\
    |cdocsamp.tex| & sample main file \\
    |cdocsch1.tex| & sample include file \\
    |cdocsch2.tex| & sample include file \\
    |cdocspt3.tex| & sample part file \\
    |cdocspt4.tex| & sample part file \\
    |cdocsdrf.tex| & sample redirection file \\
    |cdocsfn1.tex| & sample redirection file \\
    |cdocsfn2.tex| & sample redirection file \\
    |childdoc.pdf| & manual
\end{tabular}
\end{center}
%
The distribution consists of the files
|README.txt|, |childdoc.ins| and |childdoc.dtx|.
%
\begin{itemize}
\item
Run (pdf)\LaTeX{} on |childdoc.dtx|
to compile the manual |childdoc.pdf| (this file).
\item
Run \LaTeX{} on |childdoc.ins| to create the definitions file |childdoc.def|
and the sample |cdocsamp.tex| with include files
|cdocsch1.tex|, |cdocsch2.tex|, |cdocspt3.tex|, |cdocspt4.tex|,
|cdocsdrf.tex|, |cdocsfn1.tex|, |cdocsfn2.tex|.
Then copy the file |childdoc.def| to an appropriate directory of your \LaTeX{}
distribution, e.g.\ \textit{texmf-root}|/tex/latex/childdoc|.
\end{itemize}

%%%%%%%%%%%%%%%%%%%%%%%%%%%%%%%%%%%%%%%%%%%%%%%%%%%%%%%%%%%%%%%%%%%%%%%%%%%%%%%%
\subsection{Related CTAN Packages}

There are several other packages which offer a similar functionality:
%
\begin{itemize}
\item
The packages
\href{http://ctan.org/pkg/docmute}{\textsf{docmute}},
\href{http://ctan.org/pkg/includex}{\textsf{includex}} and
\href{http://ctan.org/pkg/standalone}{\textsf{standalone}}
provide commands to include only the document body of
a child file thus allowing both files to be compiled individually.
\item
The packages \href{http://ctan.org/pkg/subdocs}{\textsf{subdocs}}
and \href{http://ctan.org/pkg/subfiles}{\textsf{subfiles}}
provide structures in which the main and child documents can be
encapsulated and allowing them to be compiled individually.
The inclusion mechanism is different from the conventional |\include|.
\item
The package \href{http://ctan.org/pkg/combine}{\textsf{combine}}
is an elaborate solution to combine several documents into one.
\end{itemize}
%
See also the CTAN topic \href{http://ctan.org/topic/subdocs}{\textsf{subdocs}}
for further related packages.
The present package differs from the above solutions in that
a document structure constructed with the conventional |\include| mechanism
just needs two extra commands at the top of every file
such that all constituent files can be compiled individually.

%%%%%%%%%%%%%%%%%%%%%%%%%%%%%%%%%%%%%%%%%%%%%%%%%%%%%%%%%%%%%%%%%%%%%%%%%%%%%%%%
%\subsection{Feature Suggestions}
%
%The following is a list of features which may be useful for future
%versions of this package:
%%
%\begin{itemize}
%\item
%\ldots
%\end{itemize}

%%%%%%%%%%%%%%%%%%%%%%%%%%%%%%%%%%%%%%%%%%%%%%%%%%%%%%%%%%%%%%%%%%%%%%%%%%%%%%%%
\subsection{Revision History}

%%%%%%%%%%%%%%%%%%%%%%%%%%%%%%%%%%%%%%%%
\paragraph{v2.0:} 2018/12/30

\begin{itemize}
\item
immediate forward processing
\item
added |\childdocby| mechanism
\item
manual restructured
\end{itemize}

%%%%%%%%%%%%%%%%%%%%%%%%%%%%%%%%%%%%%%%%
\paragraph{v1.6:} 2018/01/17

\begin{itemize}
\item
application for development of include files
\item
corrections to manual
\end{itemize}

%%%%%%%%%%%%%%%%%%%%%%%%%%%%%%%%%%%%%%%%
\paragraph{v1.5:} 2017/05/21

\begin{itemize}
\item
more complete structuring introduced
\item
|\childdocof| introduced
\item
|\childdoc| renamed to |\childdocmain|
\item
|\childredirect| renamed to |\childdocforward| and |\childdocforwardprefix|
and functionality expanded
\end{itemize}

%%%%%%%%%%%%%%%%%%%%%%%%%%%%%%%%%%%%%%%%
\paragraph{v1.0:} 2017/04/27

\begin{itemize}
\item
manual and install package
\item
first version published on CTAN
\end{itemize}

%%%%%%%%%%%%%%%%%%%%%%%%%%%%%%%%%%%%%%%%
\paragraph{v0.6:} 2017/04/26

\begin{itemize}
\item
redirection mechanism added
\end{itemize}

%%%%%%%%%%%%%%%%%%%%%%%%%%%%%%%%%%%%%%%%
\paragraph{v0.5:} 2017/04/26

\begin{itemize}
\item
functionality in definition file
\end{itemize}


%%%%%%%%%%%%%%%%%%%%%%%%%%%%%%%%%%%%%%%%%%%%%%%%%%%%%%%%%%%%%%%%%%%%%%%%%%%%%%%%
%%%%%%%%%%%%%%%%%%%%%%%%%%%%%%%%%%%%%%%%%%%%%%%%%%%%%%%%%%%%%%%%%%%%%%%%%%%%%%%%
%%%%%%%%%%%%%%%%%%%%%%%%%%%%%%%%%%%%%%%%%%%%%%%%%%%%%%%%%%%%%%%%%%%%%%%%%%%%%%%%
\appendix

\settowidth\MacroIndent{\rmfamily\scriptsize 000\ }

 \DocInput{childdoc.dtx}

\end{document}
%</driver>
% \fi
%
% %%%%%%%%%%%%%%%%%%%%%%%%%%%%%%%%%%%%%%%%%%%%%%%%%%%%%%%%%%%%%%%%%%%%%%%%%%%%%%
% %%%%%%%%%%%%%%%%%%%%%%%%%%%%%%%%%%%%%%%%%%%%%%%%%%%%%%%%%%%%%%%%%%%%%%%%%%%%%%
% \section{Sample}
%\iffalse
%<*samplemain>
%\fi
%
% The following presents a sample document
% with two chapters, two parts, a title page,
% a compile flag as well as three forwarding files to set the flag.
% It consists of eight |.tex| files:
% \begin{center}
% \begin{tabular}{ll}
% |cdocsamp.tex|&main file\\
% |cdocsch1.tex|&include file for chapter 1\\
% |cdocsch2.tex|&include file for chapter 2\\
% |cdocspt3.tex|&include file for part 3\\
% |cdocspt4.tex|&include file for part 4\\
% |cdocsdrf.tex|&forwarding file for main file in draft mode\\
% |cdocsfi1.tex|&forwarding file for final version of chapter 1\\
% |cdocsfi2.tex|&forwarding file for final version of chapter 2\\
% \end{tabular}
% \end{center}
% Each of the eight files can be compiled directly by the \LaTeX{} compiler.
%
% %%%%%%%%%%%%%%%%%%%%%%%%%%%%%%%%%%%%%%
% \paragraph{Main File.}
%
% The main file is called |cdocsamp.tex|.
%
% Load the \textsf{childdoc} definitions and
% declare the filename for the main document:
%    \begin{macrocode}
\input{childdoc.def}
\childdocmain{}
%    \end{macrocode}

% Optional override for |\version| flag:
%    \begin{macrocode}
%%\ifchilddoc\else\providecommand{\version}{draft}\fi
%    \end{macrocode}

% Define the default values for the |\version| flag
% (|final| for the main file and |draft| for childs):
%    \begin{macrocode}
\ifchilddoc
\providecommand{\version}{draft}
\else
\providecommand{\version}{final}
\fi
%    \end{macrocode}

% Load the standard document class:
%    \begin{macrocode}
\documentclass[12pt]{article}
%    \end{macrocode}

% Start the document body:
%    \begin{macrocode}
\begin{document}
%    \end{macrocode}

% Declare a title page.
% Print title, part of document being processed and version flag:
%    \begin{macrocode}
\addtocounter{page}{-1}
\begin{center}
{\LARGE\bfseries{}childdoc example\par}
\vspace{1cm}
\ifchilddoc
\ifchilddocmanual part\else chapter\fi:
`\childdocname' of `\childdocjob'\par
\else
main document: `\childdocjob'\par
\fi
version: \version\par
\end{center}
\newpage
%    \end{macrocode}

% Manually include selected file,
% otherwise process as usual:
%    \begin{macrocode}
\ifchilddocmanual
\section*{part `\childdocname'}
\input{\childdocname}
\else
%    \end{macrocode}

% Include the two chapters:
%    \begin{macrocode}
\include{cdocsch1}
\include{cdocsch2}
%    \end{macrocode}

% Include the two parts unless only chapters should be displayed:
%    \begin{macrocode}
\ifchilddoc\else
\section{part three}
\input{cdocspt3}
\section{part four}
\input{cdocspt4}
\fi
%    \end{macrocode}

% Process as usual until here:
%    \begin{macrocode}
\fi
%    \end{macrocode}

% End of document body:
%    \begin{macrocode}
\end{document}
%    \end{macrocode}
%\iffalse
%</samplemain>
%\fi
%
% %%%%%%%%%%%%%%%%%%%%%%%%%%%%%%%%%%%%%%
% \paragraph{Chapter Include Files.}
%
% The include files are called |cdocsch1.tex| and |cdocsch2.tex|.
%
%\iffalse
%<*samplechap1|samplechap2>
%\fi

% Optional override for |\version| flag:
%    \begin{macrocode}
%%\providecommand{\version}{final}
%    \end{macrocode}

% Include the main document:
%    \begin{macrocode}
\input{childdoc.def}
\childdocof{cdocsamp}
%    \end{macrocode}

%\iffalse
%</samplechap1|samplechap2>
%\fi
%
%\iffalse
%<*samplechap1>
%\fi
% Some text for chapter 1:
%    \begin{macrocode}
\section{one}
some text in chapter one
%    \end{macrocode}

%\iffalse
%</samplechap1>
%\fi
% Some text for chapter 2:
%\iffalse
%<*samplechap2>
%\fi
%    \begin{macrocode}
\section{two}
more text in chapter two
%    \end{macrocode}

%\iffalse
%</samplechap2>
%\fi
%
% %%%%%%%%%%%%%%%%%%%%%%%%%%%%%%%%%%%%%%
% \paragraph{Part Include Files.}
%
% The include files are called |cdocspt3.tex| and |cdocspt4.tex|.
%
%\iffalse
%<*samplepart3|samplepart4>
%\fi

% Optional override for |\version| flag:
%    \begin{macrocode}
%%\providecommand{\version}{final}
%    \end{macrocode}

% Include the main document:
%    \begin{macrocode}
\input{childdoc.def}
\childdocby{cdocsamp}
%    \end{macrocode}

%\iffalse
%</samplepart3|samplepart4>
%\fi
%
%\iffalse
%<*samplepart3>
%\fi
% Some text for part 3:
%    \begin{macrocode}
some text in part three
%    \end{macrocode}

%\iffalse
%</samplepart3>
%\fi
% Some text for part 4:
%\iffalse
%<*samplepart4>
%\fi
%    \begin{macrocode}
more text in part four
%    \end{macrocode}

%\iffalse
%</samplepart4>
%\fi
%
% %%%%%%%%%%%%%%%%%%%%%%%%%%%%%%%%%%%%%%
% \paragraph{Forwarding for a Complete Draft.}
%
% The following forwarding file |cdocsdrf.tex|
% compiles the main document in draft mode:
%\iffalse
%<*sampledraft>
%\fi
%    \begin{macrocode}
\def\version{draft}
\input{childdoc.def}
\childdocforward{cdocsamp}
%    \end{macrocode}

%\iffalse
%</sampledraft>
%\fi
%
% %%%%%%%%%%%%%%%%%%%%%%%%%%%%%%%%%%%%%%
% \paragraph{Forwarding for Final Version of the Chapters.}
%
% The following forwarding files |cdocsfn1.tex| and |cdocsfn2.tex|
% (with identical content)
% compile the final versions of the child documents
% |cdocsch1.tex| and |cdocsch2.tex|, respectively:
%\iffalse
%<*samplefinal>
%\fi
%    \begin{macrocode}
\def\version{final}
\input{childdoc.def}
\childdocforwardprefix[cdocsamp]{cdocsfn}{cdocsch}
%    \end{macrocode}

%\iffalse
%</samplefinal>
%\fi
%
% %%%%%%%%%%%%%%%%%%%%%%%%%%%%%%%%%%%%%%
% \paragraph{Command Line Processing.}
%
% The following three command lines generate the output files
% |cdocscld|, |cdocscl1| and |cdocscl2|
% which should be identical to
% |cdocsdrf|, |cdocsch1| and |cdocsfn2|, respectively:
% \begin{center}
% \begin{tabular}{l}
% |latex -jobname cdocscld \|\\
% |  "\def\version{draft}\input{childdoc.def}\childdocforward{cdocsamp}"|\\
% |latex -jobname cdocscl1 \|\\
% |  "\input{childdoc.def}\childdocforward[cdocsamp]{cdocsch1}"|\\
% |latex -jobname cdocscl2 \|\\
% |  "\def\version{final}\input{childdoc.def}\childdocforward{cdocsch2}"|
% \end{tabular}
% \end{center}
% Note that the trailing backslash on each first line
% merely continues the input to the second line
% (for convenient cut ant paste).
% Furthermore, the command |latex| can be replaced by any
% of its alternative versions such as |pdflatex|.
%
% %%%%%%%%%%%%%%%%%%%%%%%%%%%%%%%%%%%%%%%%%%%%%%%%%%%%%%%%%%%%%%%%%%%%%%%%%%%%%%
% %%%%%%%%%%%%%%%%%%%%%%%%%%%%%%%%%%%%%%%%%%%%%%%%%%%%%%%%%%%%%%%%%%%%%%%%%%%%%%
% \section{Implementation}
%\iffalse
%<*package>
%\fi
%
% This section describes the definitions file |childdoc.def|.

% The definitions cannot be loaded using |\usepackage| or |\RequirePackage|
% which has a mechanism to prevent loading a style file more than once.
% When loading the definitions by means of |\input|
% multiple instances have to be prevented manually:
%\iffalse
%This code needs to be before the `\ProvidesFile' directive
%which is defined at the beginning of this file.
%Therefore it is also placed there and commented out here.
%</package>
%<*discard>
%\fi
%    \begin{macrocode}
\ifdefined\childdocmain\endinput\fi
%    \end{macrocode}
%\iffalse
%</discard>
%<*package>
%\fi
%
% \macro{\ifchilddoc}
% \macro{\ifchilddocmanual}
% The conditional |\ifchilddoc| tells whether a
% child (true) or main (false) document is being compiled.
% The conditional |\ifchilddocmanual| tells whether
% the |\includeonly| mechanism is used (false) or
% the selection of child files must be performed manually (true).
% The definitions initialise to false:
%    \begin{macrocode}
\newif\ifchilddoc
\newif\ifchilddocmanual
%    \end{macrocode}

% \macro{\childdocname}
% \macro{\childdocjob}
% The macro |\childdocname| stores the name of the main document
% to be compiled. The macro |\childdocjob| stores the name of
% the document on which the \LaTeX{} compiler was originally invoked.
% The content of |\jobname| cannot be compared
% to filenames specified in the source due to different catcodes.
% The following code rescans |\jobname|, stores the result
% in |\childdocname| and saves a copy in |\childdocjob|:
%    \begin{macrocode}
\edef\childdocname{\scantokens\expandafter{\jobname\noexpand}}
\let\childdocjob\childdocname
%    \end{macrocode}

% \macro{\childdocdisable}
% The macro |\childdocdisable| prevents the main file
% from being processed more than once.
% At this stage, the main document command |\childdocmain|
% is assumed to be called once again where it should do nothing.
% Any subsequent call to it should prevent
% a secondary processing of the main document
% It overwrites the forwarding commands
% |\childdocof| and |\childdocforward|
% with empty macros to prevent further inclusions of the main document:
%    \begin{macrocode}
\newcommand{\childdocdisable}
{
  \renewcommand{\childdocmain}[1]{\renewcommand{\childdocmain}[1]{\endinput}}
  \renewcommand{\childdocof}[1]{}
  \renewcommand{\childdocby}[2][]{}
  \renewcommand{\childdocforward}[2][]{}
  \renewcommand{\childdocdisable}{}
}
%    \end{macrocode}

% \macro{\childdocmain}
% The macro |\childdocmain| is to be called at the top of the main file
% with nothing or the main filename (without extension) as argument.
% First, it breaks loops.
% If the argument is not empty and does not match |\childdocname|
% (which is set by the first inclusion of |childdoc.def|),
% |\ifchilddoc| is set to true, |\includeonly| is applied to the child file
% and |\jobname| is set to the main file
% (for proper handling of |.aux| files):
%    \begin{macrocode}
\newcommand{\childdocmain}[1]
{
  \childdocdisable\childdocmain{}
  \if?#1?\else
    \begingroup
      \def\childdoctmp{#1}
      \ifx\childdoctmp\childdocname
        \def\childdoctmp{}
      \else
        \def\childdoctmp
        {
          \childdoctrue
          \includeonly{\childdocname}
          \def\childdocjob{#1}
          \def\jobname{#1}
        }
      \fi
      \expandafter
    \endgroup
    \childdoctmp
  \fi
}
%    \end{macrocode}

% \macro{\childdocof}
% The command |\childdocof| redirects
% compilation to the main file |#1|.
%    \begin{macrocode}
\newcommand{\childdocof}[1]
{
  \childdocdisable
  \childdoctrue
  \includeonly{\childdocname}
  \def\jobname{#1}
  \def\childdocjob{#1}
  \input{#1}
}
%    \end{macrocode}

% \macro{\childdocby}
% The command |\childdocby| ....
%    \begin{macrocode}
\newcommand{\childdocby}[2][]
{
  \childdocdisable
  \childdoctrue
  \childdocmanualtrue
  \if?#1?\else
    \def\jobname{#2}
  \fi
  \def\childdocjob{#2}
  \input{#2}
  \endinput
}
%    \end{macrocode}

% \macro{\childdocforward}
% The command |\childdocforward| redirects
% compilation to the main file or
% (if the optional argument is given) a child file.
% Parameters are set as if the main file
% or a child file starting with |\childdocof| was compiled.
% Then compilation is handed over to the main file:
%    \begin{macrocode}
\newcommand{\childdocforward}[2][]
{
  \begingroup
    \if?#1?
      \def\childdoctmp
      {
        \def\childdocname{#2}
        \def\childdocjob{#2}
        \def\jobname{#2}
        \input{#2}
        \endinput
      }
    \else
      \def\childdoctmp
      {
        \childdocdisable
        \def\childdocname{#2}
        \childdoctrue
        \includeonly{#2}
        \def\childdocjob{#1}
        \def\jobname{#1}
        \input{#1}
        \endinput
      }
    \fi
    \expandafter
  \endgroup
  \childdoctmp
}
%    \end{macrocode}

% \macro{\childdocforwardprefix}
% The command |\childdocforwardprefix| redirects
% compilation to the main or a child file by means of a pattern.
% The prefix |#1| in the current filename is replaced by |#2|
% and the suffix of the current filename is kept
% (it is assumed that the filename does not contain the substring `|~~~|'
% which is used as a delimiter).
% Compilation is handed over to the new file by |\childdocforward|:
%    \begin{macrocode}
\newcommand{\childdocforwardprefix}[3][]
{
  \begingroup
    \def\childdocextract #2##1~~~{\def\childdoctmp{\childdocforward[#1]{#3##1}}}
    \expandafter\childdocextract\childdocname~~~
    \expandafter
  \endgroup
  \childdoctmp
}
%    \end{macrocode}

% \macro{\childdoc}
% The deprecated macro |\childdoc| is a legacy version of |\childdocmain|:
%    \begin{macrocode}
\newcommand{\childdoc}{\childdocmain}
%    \end{macrocode}

% \macro{\childdocredirect}
% The deprecated macro |\childdocredirect| is a legacy version
% of |\childdocforward| and |\childdocforwardprefix|:
%    \begin{macrocode}
\newcommand{\childdocredirect}[2][]
{
  \begingroup
    \if?#1?
      \def\childdoctmp{\childdocforward{#2}}
    \else
      \def\childdoctmp{\childdocforwardprefix{#1}{#2}}
    \fi
    \expandafter
  \endgroup
  \childdoctmp
}
%    \end{macrocode}

%\iffalse
%</package>
%\fi
%
\endinput
\childdocforward{cdocsch2}"|
% \end{tabular}
% \end{center}
% Note that the trailing backslash on each first line
% merely continues the input to the second line
% (for convenient cut ant paste).
% Furthermore, the command |latex| can be replaced by any
% of its alternative versions such as |pdflatex|.
%
% %%%%%%%%%%%%%%%%%%%%%%%%%%%%%%%%%%%%%%%%%%%%%%%%%%%%%%%%%%%%%%%%%%%%%%%%%%%%%%
% %%%%%%%%%%%%%%%%%%%%%%%%%%%%%%%%%%%%%%%%%%%%%%%%%%%%%%%%%%%%%%%%%%%%%%%%%%%%%%
% \section{Implementation}
%\iffalse
%<*package>
%\fi
%
% This section describes the definitions file |childdoc.def|.

% The definitions cannot be loaded using |\usepackage| or |\RequirePackage|
% which has a mechanism to prevent loading a style file more than once.
% When loading the definitions by means of |\input|
% multiple instances have to be prevented manually:
%\iffalse
%This code needs to be before the `\ProvidesFile' directive
%which is defined at the beginning of this file.
%Therefore it is also placed there and commented out here.
%</package>
%<*discard>
%\fi
%    \begin{macrocode}
\ifdefined\childdocmain\endinput\fi
%    \end{macrocode}
%\iffalse
%</discard>
%<*package>
%\fi
%
% \macro{\ifchilddoc}
% \macro{\ifchilddocmanual}
% The conditional |\ifchilddoc| tells whether a
% child (true) or main (false) document is being compiled.
% The conditional |\ifchilddocmanual| tells whether
% the |\includeonly| mechanism is used (false) or
% the selection of child files must be performed manually (true).
% The definitions initialise to false:
%    \begin{macrocode}
\newif\ifchilddoc
\newif\ifchilddocmanual
%    \end{macrocode}

% \macro{\childdocname}
% \macro{\childdocjob}
% The macro |\childdocname| stores the name of the main document
% to be compiled. The macro |\childdocjob| stores the name of
% the document on which the \LaTeX{} compiler was originally invoked.
% The content of |\jobname| cannot be compared
% to filenames specified in the source due to different catcodes.
% The following code rescans |\jobname|, stores the result
% in |\childdocname| and saves a copy in |\childdocjob|:
%    \begin{macrocode}
\edef\childdocname{\scantokens\expandafter{\jobname\noexpand}}
\let\childdocjob\childdocname
%    \end{macrocode}

% \macro{\childdocdisable}
% The macro |\childdocdisable| prevents the main file
% from being processed more than once.
% At this stage, the main document command |\childdocmain|
% is assumed to be called once again where it should do nothing.
% Any subsequent call to it should prevent
% a secondary processing of the main document
% It overwrites the forwarding commands
% |\childdocof| and |\childdocforward|
% with empty macros to prevent further inclusions of the main document:
%    \begin{macrocode}
\newcommand{\childdocdisable}
{
  \renewcommand{\childdocmain}[1]{\renewcommand{\childdocmain}[1]{\endinput}}
  \renewcommand{\childdocof}[1]{}
  \renewcommand{\childdocby}[2][]{}
  \renewcommand{\childdocforward}[2][]{}
  \renewcommand{\childdocdisable}{}
}
%    \end{macrocode}

% \macro{\childdocmain}
% The macro |\childdocmain| is to be called at the top of the main file
% with nothing or the main filename (without extension) as argument.
% First, it breaks loops.
% If the argument is not empty and does not match |\childdocname|
% (which is set by the first inclusion of |childdoc.def|),
% |\ifchilddoc| is set to true, |\includeonly| is applied to the child file
% and |\jobname| is set to the main file
% (for proper handling of |.aux| files):
%    \begin{macrocode}
\newcommand{\childdocmain}[1]
{
  \childdocdisable\childdocmain{}
  \if?#1?\else
    \begingroup
      \def\childdoctmp{#1}
      \ifx\childdoctmp\childdocname
        \def\childdoctmp{}
      \else
        \def\childdoctmp
        {
          \childdoctrue
          \includeonly{\childdocname}
          \def\childdocjob{#1}
          \def\jobname{#1}
        }
      \fi
      \expandafter
    \endgroup
    \childdoctmp
  \fi
}
%    \end{macrocode}

% \macro{\childdocof}
% The command |\childdocof| redirects
% compilation to the main file |#1|.
%    \begin{macrocode}
\newcommand{\childdocof}[1]
{
  \childdocdisable
  \childdoctrue
  \includeonly{\childdocname}
  \def\jobname{#1}
  \def\childdocjob{#1}
  \input{#1}
}
%    \end{macrocode}

% \macro{\childdocby}
% The command |\childdocby| ....
%    \begin{macrocode}
\newcommand{\childdocby}[2][]
{
  \childdocdisable
  \childdoctrue
  \childdocmanualtrue
  \if?#1?\else
    \def\jobname{#2}
  \fi
  \def\childdocjob{#2}
  \input{#2}
  \endinput
}
%    \end{macrocode}

% \macro{\childdocforward}
% The command |\childdocforward| redirects
% compilation to the main file or
% (if the optional argument is given) a child file.
% Parameters are set as if the main file
% or a child file starting with |\childdocof| was compiled.
% Then compilation is handed over to the main file:
%    \begin{macrocode}
\newcommand{\childdocforward}[2][]
{
  \begingroup
    \if?#1?
      \def\childdoctmp
      {
        \def\childdocname{#2}
        \def\childdocjob{#2}
        \def\jobname{#2}
        \input{#2}
        \endinput
      }
    \else
      \def\childdoctmp
      {
        \childdocdisable
        \def\childdocname{#2}
        \childdoctrue
        \includeonly{#2}
        \def\childdocjob{#1}
        \def\jobname{#1}
        \input{#1}
        \endinput
      }
    \fi
    \expandafter
  \endgroup
  \childdoctmp
}
%    \end{macrocode}

% \macro{\childdocforwardprefix}
% The command |\childdocforwardprefix| redirects
% compilation to the main or a child file by means of a pattern.
% The prefix |#1| in the current filename is replaced by |#2|
% and the suffix of the current filename is kept
% (it is assumed that the filename does not contain the substring `|~~~|'
% which is used as a delimiter).
% Compilation is handed over to the new file by |\childdocforward|:
%    \begin{macrocode}
\newcommand{\childdocforwardprefix}[3][]
{
  \begingroup
    \def\childdocextract #2##1~~~{\def\childdoctmp{\childdocforward[#1]{#3##1}}}
    \expandafter\childdocextract\childdocname~~~
    \expandafter
  \endgroup
  \childdoctmp
}
%    \end{macrocode}

% \macro{\childdoc}
% The deprecated macro |\childdoc| is a legacy version of |\childdocmain|:
%    \begin{macrocode}
\newcommand{\childdoc}{\childdocmain}
%    \end{macrocode}

% \macro{\childdocredirect}
% The deprecated macro |\childdocredirect| is a legacy version
% of |\childdocforward| and |\childdocforwardprefix|:
%    \begin{macrocode}
\newcommand{\childdocredirect}[2][]
{
  \begingroup
    \if?#1?
      \def\childdoctmp{\childdocforward{#2}}
    \else
      \def\childdoctmp{\childdocforwardprefix{#1}{#2}}
    \fi
    \expandafter
  \endgroup
  \childdoctmp
}
%    \end{macrocode}

%\iffalse
%</package>
%\fi
%
\endinput
|\\
|\childdocforwardprefix{final}{child}|
\end{tabular}
\end{center}
%

Note that when several versions of a main file and/or of each child file
are to be generated, it may be convenient to set up a |Makefile| or
shell script to automatise the process.

%%%%%%%%%%%%%%%%%%%%%%%%%%%%%%%%%%%%%%%%%%%%%%%%%%%%%%%%%%%%%%%%%%%%%%%%%%%%%%%%
\subsection{Command Line Processing}
\label{sec:commandline}

The effect of redirection files can also be achieved by invoking
the \LaTeX{} compiler with a more elaborate command line.
Most conveniently this should be done as part
of a shell script or a |Makefile|.

When using \textsf{childdoc} in the main file, the following
command lines effectively perform a redirection
(note that depending on the shell being used,
backslashes may have to be doubled: `|\|' $\to$ `|\\|'):
%
\begin{center}
|... -jobname "|\textit{target}|" |\\|"|[\textit{flags}]%
|% \iffalse
%
% childdoc.dtx Copyright (C) 2017-2018 Niklas Beisert
%
% This work may be distributed and/or modified under the
% conditions of the LaTeX Project Public License, either version 1.3
% of this license or (at your option) any later version.
% The latest version of this license is in
%   http://www.latex-project.org/lppl.txt
% and version 1.3 or later is part of all distributions of LaTeX
% version 2005/12/01 or later.
%
% This work has the LPPL maintenance status `maintained'.
%
% The Current Maintainer of this work is Niklas Beisert.
%
% This work consists of the files childdoc.dtx and childdoc.ins
% and the derived files childdoc.def and cdocsamp.tex with
% cdocsch1.tex, cdocsch2.tex, cdocsdrf.tex, cdocsfn1.tex, cdocsfn2.tex.
%
%<package>\ifdefined\childdocmain\endinput\fi
%<package>\ProvidesFile{childdoc.def}[2018/12/30 v2.0 child document driver]
%<samplemain>\ProvidesFile{cdocsamp.tex}[2018/12/30 v2.0 sample for childdoc]
%<*driver>
%\ProvidesFile{childdoc.drv}[2018/12/30 v2.0 childdoc reference manual file]
\PassOptionsToClass{10pt,a4paper}{article}
\documentclass{ltxdoc}

\usepackage[margin=35mm]{geometry}
\usepackage{hyperref}
\usepackage{hyperxmp}
\usepackage[usenames]{color}

\hypersetup{colorlinks=true}
\hypersetup{pdfstartview=FitH}
\hypersetup{pdfpagemode=UseNone}
\hypersetup{pdfsource={}}
\hypersetup{pdflang={en-UK}}
\hypersetup{pdfcopyright={Copyright 2017-2018 Niklas Beisert.
  This work may be distributed and/or modified under the
  conditions of the LaTeX Project Public License, either version 1.3
  of this license or (at your option) any later version.}}
\hypersetup{pdflicenseurl={http://www.latex-project.org/lppl.txt}}
\hypersetup{pdfcontactaddress={ETH Zurich, ITP, HIT K,
  Wolfgang-Pauli-Strasse 27}}
\hypersetup{pdfcontactpostcode={8093}}
\hypersetup{pdfcontactcity={Zurich}}
\hypersetup{pdfcontactcountry={Switzerland}}
\hypersetup{pdfcontactemail={nbeisert@itp.phys.ethz.ch}}
\hypersetup{pdfcontacturl={http://people.phys.ethz.ch/\xmptilde nbeisert/}}

\newcommand{\secref}[1]{\hyperref[#1]{section \ref*{#1}}}

\parskip1ex
\parindent0pt
\let\olditemize\itemize
\def\itemize{\olditemize\parskip0pt}

\begin{document}

\title{The \textsf{childdoc} Package}
\hypersetup{pdftitle={The childdoc Package}}
\author{Niklas Beisert\\[2ex]
  Institut f\"ur Theoretische Physik\\
  Eidgen\"ossische Technische Hochschule Z\"urich\\
  Wolfgang-Pauli-Strasse 27, 8093 Z\"urich, Switzerland\\[1ex]
  \href{mailto:nbeisert@itp.phys.ethz.ch}
  {\texttt{nbeisert@itp.phys.ethz.ch}}}
\hypersetup{pdfauthor={Niklas Beisert}}
\hypersetup{pdfsubject={Manual for the LaTeX2e Package childdoc}}
\date{30 December 2018, \textsf{v2.0}}
\maketitle

\begin{abstract}\noindent
\textsf{childdoc} is a \LaTeXe{} package
that enables the direct compilation
of document sections included by |\include|
to individual files.
\end{abstract}

\begingroup
\parskip0ex
\tableofcontents
\endgroup

%%%%%%%%%%%%%%%%%%%%%%%%%%%%%%%%%%%%%%%%%%%%%%%%%%%%%%%%%%%%%%%%%%%%%%%%%%%%%%%%
%%%%%%%%%%%%%%%%%%%%%%%%%%%%%%%%%%%%%%%%%%%%%%%%%%%%%%%%%%%%%%%%%%%%%%%%%%%%%%%%
\section{Introduction}

\LaTeX{} provides a mechanism to structure a large document (such as a book)
into a main file and several child files (containing the chapters)
using the |\include| command.
This mechanism is beneficial for documents
which span hundreds of pages in order to
make the source file(s) more manageable.
Moreover, compilation can be restricted to
selected child files by means of the |\includeonly| command.
The latter feature can be used to reduce the compilation time while editing
(this was significantly more useful in the earlier days of \LaTeX{})
or to generate a smaller document which is easier to navigate.
Another application of |\includeonly| is to generate
documents consisting of selected parts of the complete document.

However, there are a few drawbacks of the plain |\include| mechanism:
\begin{itemize}
\item
The child files cannot be compiled on their own,
they can only be compiled via the main file.
A naive editing environment
(such as a text editor with an option
to have the current file processed by \LaTeX)
may require one to switch to the main file before compiling;
attempting to compile the child file produces errors.
\item
The main file must be modified (each time)
to adjust the |\includeonly| command
to the present needs. This easily leaves the main file in a messy state.
\item
The generated document will always carry the filename
of the main document. This is inconvenient if
several child files are to be compiled and
to be kept for distribution.
\end{itemize}

The present package provides a simple interface
to make child files individually compilable by \LaTeX{}.
Compiling a child file then has the same effect as compiling
the main file with an |\includeonly| command
to select the appropriate child.
Moreover the generated document will carry the name of the child
rather than the main file.
This resolves all three above issues.

This feature is meant to make the editing of books,
thesis documents and lecture notes somewhat more convenient.
However, the package can also be used efficiently for
composing a series of documents (such as exercise sheets)
which are typically distributed individually.
It then assists the author in generating the individual documents
(potentially in different versions)
as well as a document containing the collected series.
Another application is in developing style files
or other kinds of included material
where compilation of the style file could redirect
to a sample or test file.

%%%%%%%%%%%%%%%%%%%%%%%%%%%%%%%%%%%%%%%%%%%%%%%%%%%%%%%%%%%%%%%%%%%%%%%%%%%%%%%%
%%%%%%%%%%%%%%%%%%%%%%%%%%%%%%%%%%%%%%%%%%%%%%%%%%%%%%%%%%%%%%%%%%%%%%%%%%%%%%%%
\section{Usage}

First of all, the package \textsf{childdoc} is \emph{not} a standard
\LaTeXe{} |.sty| style file! Therefore it needs to be invoked in
a non-standard way.

%%%%%%%%%%%%%%%%%%%%%%%%%%%%%%%%%%%%%%%%%%%%%%%%%%%%%%%%%%%%%%%%%%%%%%%%%%%%%%%%
\subsection{Included Files}
\label{sec:include}

%%%%%%%%%%%%%%%%%%%%%%%%%%%%%%%%%%%%%%%%
\DescribeMacro{\childdocmain}
To use the package, add the commands
\begin{center}
\begin{tabular}{l}
|% \iffalse
%
% childdoc.dtx Copyright (C) 2017-2018 Niklas Beisert
%
% This work may be distributed and/or modified under the
% conditions of the LaTeX Project Public License, either version 1.3
% of this license or (at your option) any later version.
% The latest version of this license is in
%   http://www.latex-project.org/lppl.txt
% and version 1.3 or later is part of all distributions of LaTeX
% version 2005/12/01 or later.
%
% This work has the LPPL maintenance status `maintained'.
%
% The Current Maintainer of this work is Niklas Beisert.
%
% This work consists of the files childdoc.dtx and childdoc.ins
% and the derived files childdoc.def and cdocsamp.tex with
% cdocsch1.tex, cdocsch2.tex, cdocsdrf.tex, cdocsfn1.tex, cdocsfn2.tex.
%
%<package>\ifdefined\childdocmain\endinput\fi
%<package>\ProvidesFile{childdoc.def}[2018/12/30 v2.0 child document driver]
%<samplemain>\ProvidesFile{cdocsamp.tex}[2018/12/30 v2.0 sample for childdoc]
%<*driver>
%\ProvidesFile{childdoc.drv}[2018/12/30 v2.0 childdoc reference manual file]
\PassOptionsToClass{10pt,a4paper}{article}
\documentclass{ltxdoc}

\usepackage[margin=35mm]{geometry}
\usepackage{hyperref}
\usepackage{hyperxmp}
\usepackage[usenames]{color}

\hypersetup{colorlinks=true}
\hypersetup{pdfstartview=FitH}
\hypersetup{pdfpagemode=UseNone}
\hypersetup{pdfsource={}}
\hypersetup{pdflang={en-UK}}
\hypersetup{pdfcopyright={Copyright 2017-2018 Niklas Beisert.
  This work may be distributed and/or modified under the
  conditions of the LaTeX Project Public License, either version 1.3
  of this license or (at your option) any later version.}}
\hypersetup{pdflicenseurl={http://www.latex-project.org/lppl.txt}}
\hypersetup{pdfcontactaddress={ETH Zurich, ITP, HIT K,
  Wolfgang-Pauli-Strasse 27}}
\hypersetup{pdfcontactpostcode={8093}}
\hypersetup{pdfcontactcity={Zurich}}
\hypersetup{pdfcontactcountry={Switzerland}}
\hypersetup{pdfcontactemail={nbeisert@itp.phys.ethz.ch}}
\hypersetup{pdfcontacturl={http://people.phys.ethz.ch/\xmptilde nbeisert/}}

\newcommand{\secref}[1]{\hyperref[#1]{section \ref*{#1}}}

\parskip1ex
\parindent0pt
\let\olditemize\itemize
\def\itemize{\olditemize\parskip0pt}

\begin{document}

\title{The \textsf{childdoc} Package}
\hypersetup{pdftitle={The childdoc Package}}
\author{Niklas Beisert\\[2ex]
  Institut f\"ur Theoretische Physik\\
  Eidgen\"ossische Technische Hochschule Z\"urich\\
  Wolfgang-Pauli-Strasse 27, 8093 Z\"urich, Switzerland\\[1ex]
  \href{mailto:nbeisert@itp.phys.ethz.ch}
  {\texttt{nbeisert@itp.phys.ethz.ch}}}
\hypersetup{pdfauthor={Niklas Beisert}}
\hypersetup{pdfsubject={Manual for the LaTeX2e Package childdoc}}
\date{30 December 2018, \textsf{v2.0}}
\maketitle

\begin{abstract}\noindent
\textsf{childdoc} is a \LaTeXe{} package
that enables the direct compilation
of document sections included by |\include|
to individual files.
\end{abstract}

\begingroup
\parskip0ex
\tableofcontents
\endgroup

%%%%%%%%%%%%%%%%%%%%%%%%%%%%%%%%%%%%%%%%%%%%%%%%%%%%%%%%%%%%%%%%%%%%%%%%%%%%%%%%
%%%%%%%%%%%%%%%%%%%%%%%%%%%%%%%%%%%%%%%%%%%%%%%%%%%%%%%%%%%%%%%%%%%%%%%%%%%%%%%%
\section{Introduction}

\LaTeX{} provides a mechanism to structure a large document (such as a book)
into a main file and several child files (containing the chapters)
using the |\include| command.
This mechanism is beneficial for documents
which span hundreds of pages in order to
make the source file(s) more manageable.
Moreover, compilation can be restricted to
selected child files by means of the |\includeonly| command.
The latter feature can be used to reduce the compilation time while editing
(this was significantly more useful in the earlier days of \LaTeX{})
or to generate a smaller document which is easier to navigate.
Another application of |\includeonly| is to generate
documents consisting of selected parts of the complete document.

However, there are a few drawbacks of the plain |\include| mechanism:
\begin{itemize}
\item
The child files cannot be compiled on their own,
they can only be compiled via the main file.
A naive editing environment
(such as a text editor with an option
to have the current file processed by \LaTeX)
may require one to switch to the main file before compiling;
attempting to compile the child file produces errors.
\item
The main file must be modified (each time)
to adjust the |\includeonly| command
to the present needs. This easily leaves the main file in a messy state.
\item
The generated document will always carry the filename
of the main document. This is inconvenient if
several child files are to be compiled and
to be kept for distribution.
\end{itemize}

The present package provides a simple interface
to make child files individually compilable by \LaTeX{}.
Compiling a child file then has the same effect as compiling
the main file with an |\includeonly| command
to select the appropriate child.
Moreover the generated document will carry the name of the child
rather than the main file.
This resolves all three above issues.

This feature is meant to make the editing of books,
thesis documents and lecture notes somewhat more convenient.
However, the package can also be used efficiently for
composing a series of documents (such as exercise sheets)
which are typically distributed individually.
It then assists the author in generating the individual documents
(potentially in different versions)
as well as a document containing the collected series.
Another application is in developing style files
or other kinds of included material
where compilation of the style file could redirect
to a sample or test file.

%%%%%%%%%%%%%%%%%%%%%%%%%%%%%%%%%%%%%%%%%%%%%%%%%%%%%%%%%%%%%%%%%%%%%%%%%%%%%%%%
%%%%%%%%%%%%%%%%%%%%%%%%%%%%%%%%%%%%%%%%%%%%%%%%%%%%%%%%%%%%%%%%%%%%%%%%%%%%%%%%
\section{Usage}

First of all, the package \textsf{childdoc} is \emph{not} a standard
\LaTeXe{} |.sty| style file! Therefore it needs to be invoked in
a non-standard way.

%%%%%%%%%%%%%%%%%%%%%%%%%%%%%%%%%%%%%%%%%%%%%%%%%%%%%%%%%%%%%%%%%%%%%%%%%%%%%%%%
\subsection{Included Files}
\label{sec:include}

%%%%%%%%%%%%%%%%%%%%%%%%%%%%%%%%%%%%%%%%
\DescribeMacro{\childdocmain}
To use the package, add the commands
\begin{center}
\begin{tabular}{l}
|\input{childdoc.def}|\\
|\childdocmain{}|\\
\end{tabular}
\end{center}
at the very top of the main \LaTeX{} file,
in particular \emph{before} the |\documentclass| statement!
The argument of |\childdocmain| should be left empty
(but it must be present).

%%%%%%%%%%%%%%%%%%%%%%%%%%%%%%%%%%%%%%%%
\DescribeMacro{\childdocof}
Furthermore, add the commands
\begin{center}
\begin{tabular}{l}
|\input{childdoc.def}|\\
|\childdocof{|\textit{main}|}|\\
\end{tabular}
\end{center}
at the top of every child file \textit{child}
which is included by |\include{|\textit{child}|}|
from within the main file
(or at least for those files to be compiled individually).
The argument \textit{main} must be the filename of the main file.

There are a couple of
considerations in setting up the main and child documents:

%%%%%%%%%%%%%%%%%%%%%%%%%%%%%%%%%%%%%%%%
\paragraph{Restrictions.}

Please note the following restrictions:
\begin{itemize}
\item
|\childdocmain| must be called with one argument \textit{main}
to ensure compatibility with earlier version of the package.
It must either be empty (|\childdocmain{}|)
or precisely match the filename of the main file in which it is specified.
See \secref{sec:detection} for further information.
\item
The filename \textit{main} must be specified without the |.tex| extension.
\item
The filename \textit{main} is case sensitive
(even in case-insensitive file systems)
due to internal string comparison.
\item
The argument \textit{main} should be fully expanded, it cannot be a macro.
\item
Subdirectories and special characters should be avoided in filenames.
\item
The command |\childdocmain{|\textit{main}|}| must be followed by a whitespace.
It should not be followed immediately by another command
or by a comment mark `|%|'.
This is because the \TeX{} parser reads the token immediately following
the argument of |\childdocmain| and puts it
at the beginning of every child section;
however, a white\-space is ignored.
\end{itemize}

%%%%%%%%%%%%%%%%%%%%%%%%%%%%%%%%%%%%%%%%
\paragraph{Content of Main File.}

It is advisable to place all content in the child files included by |\include|.
Any output contained in the main file will appear in all child documents
unless suppressed manually;
it cannot be suppressed automatically by the |\includeonly| directive
and thus should normally be avoided.
A method to include some content in the main file
by means of conditional processing is described in \secref{sec:conditional}.

%%%%%%%%%%%%%%%%%%%%%%%%%%%%%%%%%%%%%%%%
\paragraph{Page Numbering.}

When only a part of the document is compiled,
the appropriate numbering of pages
(as well as other status parameters)
is determined from the |.aux| files.
The latter contain information from previous passes.
However this information needs to propagate through
all intermediate child documents.
Therefore the page numbering in child documents may well
be inconsistent until the complete document is compiled at least once.

A useful (if unconventional) way to always ensure a consistent
page numbering is to restart the numbering in each child document
and denote the pages by `\textit{child}|.|\textit{page}'
where \textit{child} represents the chapter/section number of the child file.
This can be achieved by the command
|\numberwithin{page}{|\textit{child}|}|
of the \textsf{amsmath} package
where \textit{child} can be |chapter| or |section|
depending on the chosen structuring.
Alternatively, one can modify the macro |\thepage| appropriately
and reset the counter |page| at the start of each child file.

%%%%%%%%%%%%%%%%%%%%%%%%%%%%%%%%%%%%%%%%%%%%%%%%%%%%%%%%%%%%%%%%%%%%%%%%%%%%%%%%
\subsection{Conditional Processing}
\label{sec:conditional}

The package provides a mechanism to compile different versions
of a document. To customise the versions further some conditional processing
can come in handy to distinguish which version is being compiled.
The package provides two macros to describe the compilation context:

%%%%%%%%%%%%%%%%%%%%%%%%%%%%%%%%%%%%%%%%
\DescribeMacro{\ifchilddoc}
The conditional |\ifchilddoc| distinguishes between the compilation of
child documents and the main document:
%
\begin{center}
|\ifchilddoc |\textit{child-code}| |[|\||else |\textit{main-code}]| \||fi|
\end{center}

%%%%%%%%%%%%%%%%%%%%%%%%%%%%%%%%%%%%%%%%
\DescribeMacro{\childdocname}
\DescribeMacro{\childdocjob}
The macro |\childdocname| contains the filename (without extension)
of the main or child file being processed.
Note that |\childdocjob| will always contain the name of the main file.

%%%%%%%%%%%%%%%%%%%%%%%%%%%%%%%%%%%%%%%%
\paragraph{Title Page.}

Conditional processing can be used to include a title or banner page
in the main document when proper precautions are taken.
Importantly, the code in the main file should ensure that the page counter
(as well as other status parameters which are stored in the |.aux| files)
takes the same value after the conditional processing.
Otherwise the page numbers may take divergent values
depending on which part is compiled.

For example, a title page could be declared by:
%
\begin{center}
\begin{tabular}{l}
|\ifchilddoc\||else|\\
|\addtocounter{page}{-1}|\\
\textit{code for title page}\\
|\newpage|\\
|\||fi|
\end{tabular}
\end{center}
%
A banner page for the child documents can be generated by:
%
\begin{center}
\begin{tabular}{l}
|\ifchilddoc|\\
|\addtocounter{page}{-1}|\\
\textit{code for banner page}\\
|\newpage|\\
|\||fi|
\end{tabular}
\end{center}
%
Here one could write a message such as:
\begin{center}
|This is the part \childdocname{} of \childdocjob{}.|
\end{center}

%%%%%%%%%%%%%%%%%%%%%%%%%%%%%%%%%%%%%%%%%%%%%%%%%%%%%%%%%%%%%%%%%%%%%%%%%%%%%%%%
\subsection{Flags}
\label{sec:flags}

The package makes it easy to generate different versions
of the main or child documents.
To this end compilation flags can be defined
and assigned different default values.
They will be particularly useful in conjunction
with the forwarding mechanism described in \secref{sec:forward}.

For example, it may be useful to have a flag |\version|
which can be set to |draft| or |final|.
The document source will contain some conditional code
depending on the value of |\version|.
Suppose further, the flag should default to |final| for the main file
and to |draft| for child files
which is a natural assignment for editing the document.
This is achieved by placing the following code
in the preamble of the main document
(below the |\childdocmain| directive):
%
\begin{center}
\begin{tabular}{l}
|\ifchilddoc|\\
|\providecommand{\version}{draft}|\\
|\||else|\\
|\providecommand{\version}{final}|\\
|\||fi|
\end{tabular}
\end{center}
%
The definition by |\providecommand| makes sure
that previous definitions are not overwritten.
Further statements |\providecommand{\version}{...}|
can thus be added before the above code to override it.

For the main file, one might add a line
(between |\childdocmain| and the above block)
%
\begin{center}
|%\ifchilddoc\||else\providecommand{\version}{draft}\||fi|
\end{center}
%
which can be uncommented to produce a draft version.
Likewise one can add a line to the very top of a child file
(above the |\childdocof{|\textit{main}|}| directive)
%
\begin{center}
|%\providecommand{\version}{final}|
\end{center}
%
which can be uncommented to produce the final version of this child document.

%%%%%%%%%%%%%%%%%%%%%%%%%%%%%%%%%%%%%%%%%%%%%%%%%%%%%%%%%%%%%%%%%%%%%%%%%%%%%%%%
\subsection{Forwarding}
\label{sec:forward}

Different versions of the main or child documents
using compilation flags as described in \secref{sec:flags}
can be (permanently) stored in different files
for convenient compilation, viewing and distribution.
To this end, the package defines a command
to pass on compilation to a different file:

%%%%%%%%%%%%%%%%%%%%%%%%%%%%%%%%%%%%%%%%
\DescribeMacro{\childdocforward}
The command |\childdocforward| redirects processing to
another source file:
%
\begin{center}
\begin{tabular}{l}
|\input{childdoc.def}|\\
|\childdocforward[|\textit{main}|]{|\textit{dest}|}|\\
\end{tabular}
\end{center}
%
The argument \textit{dest} is the destination file
(without extension).
It should be the main file or one of the child files.
Note that further \textsf{childdoc} directives
such as |\childdocof| and |\childdocforward|
in the indicated file will be processed in this form.
The optional argument \textit{main}
passes on directly to the main file \textit{main}
while pretending to compile the child \textit{dest}.
This form behaves as if \textit{dest}
issues |\childdocof{|\textit{main}|}| right away,
and no further \textsf{childdoc} directives will be processed.

%%%%%%%%%%%%%%%%%%%%%%%%%%%%%%%%%%%%%%%%
\DescribeMacro{\...prefix}
In the alternative form |\childdocforwardprefix|,
%
\begin{center}
\begin{tabular}{l}
|\input{childdoc.def}|\\
|\childdocforwardprefix[|\textit{main}|]{|\textit{prefix}|}{|\textit{dest}|}|
\end{tabular}
\end{center}
%
the destination file is determined by a pattern
depending on the current file:
To make this work, the current file must be called
`{\textit{prefix}\hspace{0.2em}\textit{suffix}}'
with \textit{prefix} matching precisely the argument.
Processing is then passed on to the file
`{\textit{dest}\hspace{0.2em}\textit{suffix}}'.
Surely, the same effect is achieved by
directly specifying the
argument `{\textit{dest}\hspace{0.2em}\textit{suffix}}'
in the first form.
However, that requires to set up a different file
for each child. With the alternative form of the command
all these files can have exactly the same content
which simplifies setting them up and maintaining them.

For example, the following file |draft.tex|
with a compilation flag |\version| as described in \secref{sec:flags}
compiles the main document as a draft:
%
\begin{center}
\begin{tabular}{l}
|\def\version{draft}|\\
|\input{childdoc.def}|\\
|\childdocforward{|\textit{main}|}|
\end{tabular}
\end{center}
%
Likewise, the following files |final|\textit{nn}|.tex|
compile the final version of the child document
|child|\textit{nn}|.tex|:
%
\begin{center}
\begin{tabular}{l}
|\def\version{final}|\\
|\input{childdoc.def}|\\
|\childdocforwardprefix{final}{child}|
\end{tabular}
\end{center}
%

Note that when several versions of a main file and/or of each child file
are to be generated, it may be convenient to set up a |Makefile| or
shell script to automatise the process.

%%%%%%%%%%%%%%%%%%%%%%%%%%%%%%%%%%%%%%%%%%%%%%%%%%%%%%%%%%%%%%%%%%%%%%%%%%%%%%%%
\subsection{Command Line Processing}
\label{sec:commandline}

The effect of redirection files can also be achieved by invoking
the \LaTeX{} compiler with a more elaborate command line.
Most conveniently this should be done as part
of a shell script or a |Makefile|.

When using \textsf{childdoc} in the main file, the following
command lines effectively perform a redirection
(note that depending on the shell being used,
backslashes may have to be doubled: `|\|' $\to$ `|\\|'):
%
\begin{center}
|... -jobname "|\textit{target}|" |\\|"|[\textit{flags}]%
|\input{childdoc.def}\childdocforward[|\textit{main}|]{|\textit{dest}|}"|
\end{center}
%
Here \textit{target} is the name of the output file,
\textit{main} is the name of the main file
and \textit{dest} is the name of the main or child file to be processed
(all filenames without extensions).
The optional argument \textit{main} can be omitted
if \textit{main} matches \textit{dest}.
Optionally, compilation \textit{flags} can be defined via |\def| commands.
This command line makes the \TeX{} engine believe
it is compiling the file \textit{target}
whose content is specified as the latter parameter.
The provided code then forwards the processing to
\textit{main} or \textit{dest} as described in \secref{sec:forward}.

%%%%%%%%%%%%%%%%%%%%%%%%%%%%%%%%%%%%%%%%%%%%%%%%%%%%%%%%%%%%%%%%%%%%%%%%%%%%%%%%
\subsection{Include by Input}
\label{sec:input}

Including child documents by |\include| has some restrictions by design.
Most notably, the content of a child document always occupies
its own set of pages; pages cannot be shared between child documents.
Usually, this behaviour makes perfect sense
because each child document contain an essential part of the document.
However, in some situations it may be desirable to compose
a document from a collection of parts
without having mandatory page breaks between then.
For this case, the package
provides a mechanism to include parts
by |\input| which can also be processed individually.
However, by construction this mechanism
requires manual handling of the content to be output.

%%%%%%%%%%%%%%%%%%%%%%%%%%%%%%%%%%%%%%%%
\DescribeMacro{\ifchilddocmanual}
The main file should be prepared as usual, see \secref{sec:include}.
However, the document body must make a distinction
between processing of an individual part and of the main document, e.g.:
%
\begin{center}
\begin{tabular}{l}
|\ifchilddocmanual|\\
|\input{\childdocname}|\\
|\||else|\\
\textit{document body with }|\input{|\textit{part}|}|\\
|\||fi|
\end{tabular}
\end{center}
%
The conditional |\ifchilddocmanual| is true whenever
a part to be included by |\input| is being compiled,
and the name of the part is stored in |\childdocname|.

%%%%%%%%%%%%%%%%%%%%%%%%%%%%%%%%%%%%%%%%
\DescribeMacro{\childdocby}
Each part to be included by |\input| should start with:
%
\begin{center}
\begin{tabular}{l}
|\input{childdoc.def}|\\
|\childdocby{|\textit{main}|}|\\
\end{tabular}
\end{center}
%
The directive |\childdocby| is similar to |\childdocof|
described in \secref{sec:include},
but the subsequent selection of content must be done manually.
To that end, both |\ifchilddoc| and |\ifchilddocmanual|
will be true upon processing of a part,
and the name of the part is stored in |\childdocname|.
Note that |\jobname| will be set to the filename of the current part
so that each part receives an individual |.aux| file
that does not interfere with the |.aux| file(s) of the main document.
This behaviour can be altered by the alternative form
|\childdocby[*]{|\textit{main}|}| (with a non-empty optional argument)
which uses the |.aux| file of the main document
by setting |\jobname| to \textit{main}.

%%%%%%%%%%%%%%%%%%%%%%%%%%%%%%%%%%%%%%%%%%%%%%%%%%%%%%%%%%%%%%%%%%%%%%%%%%%%%%%%
\subsection{Driver Development}
\label{sec:driver}

The \textsf{childdoc} mechanism can also be use for the development
of definition files such as \LaTeX{} styles or classes.
This case differs from the above setup with multiple parts
included by |\include| in that no |\includeonly| should be invoked.
This can be achieved by starting the include file
(before |\ProvidesPackage|) with:
%
\begin{center}
\begin{tabular}{l}
|\input{childdoc.def}|\\
|\childdocforward{|\textit{main}|}|\\
\end{tabular}
\end{center}
%
or alternatively with:
%
\begin{center}
\begin{tabular}{l}
|\input{childdoc.def}|\\
|\childdocby{|\textit{main}|}|\\
\end{tabular}
\end{center}
%
Both forms have slightly different effects as described above.
The main file is prepared as usual, see \secref{sec:include}.

%%%%%%%%%%%%%%%%%%%%%%%%%%%%%%%%%%%%%%%%%%%%%%%%%%%%%%%%%%%%%%%%%%%%%%%%%%%%%%%%
\subsection{Legacy Detection}
\label{sec:detection}

The directive |\childdocmain| in the main file can detect
whether the complete document or merely a child is to be compiled
even without using the directive |\childdocof|.
This method is deprecated because it is less robust
and there is no compelling reason to use it;
it is merely provided for backward compatibility
and it may be removed in future versions.

If the detection mechanism is to be used,
it is mandatory to correctly specify
the filename of the main file as the argument of |\childdocmain|:
%
\begin{center}
\begin{tabular}{l}
|\input{childdoc.def}|\\
|\childdocmain{|\textit{main}|}|\\
\end{tabular}
\end{center}
%
If |\jobname| does not match the argument \textit{main} of |\childdocmain|,
it is assumed that |\jobname| points to the child file to be compiled.
When using |\childdocmain| with the main file specified as argument,
it suffices to start a child file
with just |\input{|\textit{main}|}|
without loading of the package and using |\childdocof|.
If instead all processing is done
with the appropriate \textsf{childdoc} directives,
the argument of \textit{main} of |\childdocmain| can be empty.

An alternative version of the command line processing described
in \secref{sec:commandline} using the detection mechanism reads:
%
\begin{center}
|... -jobname "|\textit{target}|" "|[\textit{flags}]%
[|\def\jobname{|\textit{dest}|}|]|\input{|\textit{main}|}"|
\end{center}

%%%%%%%%%%%%%%%%%%%%%%%%%%%%%%%%%%%%%%%%%%%%%%%%%%%%%%%%%%%%%%%%%%%%%%%%%%%%%%%%
\subsection{Manual Code}
\label{sec:manual}

In case one cannot be certain whether the definitions file |childdoc.def|
is installed on the target \TeX{} distribution
and one prefers not to ship it,
it is conceivable to paste a few relevant commands into the sources.

To that end, drop all statements |\input{childdoc.def}|
and perform the replacements as outlined below.
Instead of |\childdocmain{|\textit{main}|}| add the following code
to the top of the main file:
%
\begin{center}
\begin{tabular}{l}
|\||ifdefined\childdocname\endinput\||fi\newif\ifchilddoc|\\
|\edef\childdocname{\scantokens\expandafter{\jobname\noexpand}}|\\
|\def\childdocmain{|\textit{main}|}\||ifx\childdocmain\childdocname\||else|\\
|\childdoctrue\includeonly{\childdocname}\let\jobname\childdocmain\||fi|\\
\end{tabular}
\end{center}
%
Instead of |\childdocof{|\textit{main}|}| just include the main file
at the top of each child file:
%
\begin{center}
|\input{|\textit{main}|}|
\end{center}
%
A simple redirection |\childdocforward{|\textit{dest}|}| is achieved by:
%
\begin{center}
|\def\jobname{|\textit{dest}|}\input{\jobname}|
\end{center}
%
The redirection with prefix
|\childdocforwardprefix[|\textit{prefix}|]{|\textit{dest}|}|
is accomplished by:
%
\begin{center}
\begin{tabular}{l}
|{\edef\jobname{\scantokens\expandafter{\jobname\noexpand}}|\\
|\def\redirectjob |\textit{prefix}|#1~~~{\gdef\jobname{|\textit{dest}|#1}}|\\
|\expandafter\redirectjob\jobname~~~}\input{\jobname}|
\end{tabular}
\end{center}

In an alternative approach,
child documents can be compiled by a specific command line
without additional code or specific definitions:
%
\begin{center}
|... -jobname "|\textit{target}|" "|[\textit{flags}]%
|\includeonly{|\textit{dest}|}\input{|\textit{main}|}"|
\end{center}
%

%%%%%%%%%%%%%%%%%%%%%%%%%%%%%%%%%%%%%%%%%%%%%%%%%%%%%%%%%%%%%%%%%%%%%%%%%%%%%%%%
%%%%%%%%%%%%%%%%%%%%%%%%%%%%%%%%%%%%%%%%%%%%%%%%%%%%%%%%%%%%%%%%%%%%%%%%%%%%%%%%
\section{Information}

%%%%%%%%%%%%%%%%%%%%%%%%%%%%%%%%%%%%%%%%%%%%%%%%%%%%%%%%%%%%%%%%%%%%%%%%%%%%%%%%
\subsection{Copyright}

Copyright \copyright{} 2017--2018 Niklas Beisert

This work may be distributed and/or modified under the
conditions of the \LaTeX{} Project Public License, either version 1.3
of this license or (at your option) any later version.
The latest version of this license is in
  \url{http://www.latex-project.org/lppl.txt}
and version 1.3 or later is part of all distributions of \LaTeX{}
version 2005/12/01 or later.

This work has the LPPL maintenance status `maintained'.

The Current Maintainer of this work is Niklas Beisert.

This work consists of the files |README.txt|, |childdoc.ins| and |childdoc.dtx|
as well as the derived files |childdoc.def|, |cdocsamp.tex|
with |cdocsch1.tex|, |cdocsch2.tex|, |cdocspt3.tex|, |cdocspt4.tex|,
|cdocsdrf.tex|, |cdocsfn1.tex|, |cdocsfn2.tex|
as well as |childdoc.pdf|.

%%%%%%%%%%%%%%%%%%%%%%%%%%%%%%%%%%%%%%%%%%%%%%%%%%%%%%%%%%%%%%%%%%%%%%%%%%%%%%%%
\subsection{Files and Installation}

The package consists of the files:
%
\begin{center}
\begin{tabular}{ll}
    |README.txt|   & readme file \\
    |childdoc.ins| & installation file \\
    |childdoc.dtx| & source file \\
    |childdoc.def| & definition file \\
    |cdocsamp.tex| & sample main file \\
    |cdocsch1.tex| & sample include file \\
    |cdocsch2.tex| & sample include file \\
    |cdocspt3.tex| & sample part file \\
    |cdocspt4.tex| & sample part file \\
    |cdocsdrf.tex| & sample redirection file \\
    |cdocsfn1.tex| & sample redirection file \\
    |cdocsfn2.tex| & sample redirection file \\
    |childdoc.pdf| & manual
\end{tabular}
\end{center}
%
The distribution consists of the files
|README.txt|, |childdoc.ins| and |childdoc.dtx|.
%
\begin{itemize}
\item
Run (pdf)\LaTeX{} on |childdoc.dtx|
to compile the manual |childdoc.pdf| (this file).
\item
Run \LaTeX{} on |childdoc.ins| to create the definitions file |childdoc.def|
and the sample |cdocsamp.tex| with include files
|cdocsch1.tex|, |cdocsch2.tex|, |cdocspt3.tex|, |cdocspt4.tex|,
|cdocsdrf.tex|, |cdocsfn1.tex|, |cdocsfn2.tex|.
Then copy the file |childdoc.def| to an appropriate directory of your \LaTeX{}
distribution, e.g.\ \textit{texmf-root}|/tex/latex/childdoc|.
\end{itemize}

%%%%%%%%%%%%%%%%%%%%%%%%%%%%%%%%%%%%%%%%%%%%%%%%%%%%%%%%%%%%%%%%%%%%%%%%%%%%%%%%
\subsection{Related CTAN Packages}

There are several other packages which offer a similar functionality:
%
\begin{itemize}
\item
The packages
\href{http://ctan.org/pkg/docmute}{\textsf{docmute}},
\href{http://ctan.org/pkg/includex}{\textsf{includex}} and
\href{http://ctan.org/pkg/standalone}{\textsf{standalone}}
provide commands to include only the document body of
a child file thus allowing both files to be compiled individually.
\item
The packages \href{http://ctan.org/pkg/subdocs}{\textsf{subdocs}}
and \href{http://ctan.org/pkg/subfiles}{\textsf{subfiles}}
provide structures in which the main and child documents can be
encapsulated and allowing them to be compiled individually.
The inclusion mechanism is different from the conventional |\include|.
\item
The package \href{http://ctan.org/pkg/combine}{\textsf{combine}}
is an elaborate solution to combine several documents into one.
\end{itemize}
%
See also the CTAN topic \href{http://ctan.org/topic/subdocs}{\textsf{subdocs}}
for further related packages.
The present package differs from the above solutions in that
a document structure constructed with the conventional |\include| mechanism
just needs two extra commands at the top of every file
such that all constituent files can be compiled individually.

%%%%%%%%%%%%%%%%%%%%%%%%%%%%%%%%%%%%%%%%%%%%%%%%%%%%%%%%%%%%%%%%%%%%%%%%%%%%%%%%
%\subsection{Feature Suggestions}
%
%The following is a list of features which may be useful for future
%versions of this package:
%%
%\begin{itemize}
%\item
%\ldots
%\end{itemize}

%%%%%%%%%%%%%%%%%%%%%%%%%%%%%%%%%%%%%%%%%%%%%%%%%%%%%%%%%%%%%%%%%%%%%%%%%%%%%%%%
\subsection{Revision History}

%%%%%%%%%%%%%%%%%%%%%%%%%%%%%%%%%%%%%%%%
\paragraph{v2.0:} 2018/12/30

\begin{itemize}
\item
immediate forward processing
\item
added |\childdocby| mechanism
\item
manual restructured
\end{itemize}

%%%%%%%%%%%%%%%%%%%%%%%%%%%%%%%%%%%%%%%%
\paragraph{v1.6:} 2018/01/17

\begin{itemize}
\item
application for development of include files
\item
corrections to manual
\end{itemize}

%%%%%%%%%%%%%%%%%%%%%%%%%%%%%%%%%%%%%%%%
\paragraph{v1.5:} 2017/05/21

\begin{itemize}
\item
more complete structuring introduced
\item
|\childdocof| introduced
\item
|\childdoc| renamed to |\childdocmain|
\item
|\childredirect| renamed to |\childdocforward| and |\childdocforwardprefix|
and functionality expanded
\end{itemize}

%%%%%%%%%%%%%%%%%%%%%%%%%%%%%%%%%%%%%%%%
\paragraph{v1.0:} 2017/04/27

\begin{itemize}
\item
manual and install package
\item
first version published on CTAN
\end{itemize}

%%%%%%%%%%%%%%%%%%%%%%%%%%%%%%%%%%%%%%%%
\paragraph{v0.6:} 2017/04/26

\begin{itemize}
\item
redirection mechanism added
\end{itemize}

%%%%%%%%%%%%%%%%%%%%%%%%%%%%%%%%%%%%%%%%
\paragraph{v0.5:} 2017/04/26

\begin{itemize}
\item
functionality in definition file
\end{itemize}


%%%%%%%%%%%%%%%%%%%%%%%%%%%%%%%%%%%%%%%%%%%%%%%%%%%%%%%%%%%%%%%%%%%%%%%%%%%%%%%%
%%%%%%%%%%%%%%%%%%%%%%%%%%%%%%%%%%%%%%%%%%%%%%%%%%%%%%%%%%%%%%%%%%%%%%%%%%%%%%%%
%%%%%%%%%%%%%%%%%%%%%%%%%%%%%%%%%%%%%%%%%%%%%%%%%%%%%%%%%%%%%%%%%%%%%%%%%%%%%%%%
\appendix

\settowidth\MacroIndent{\rmfamily\scriptsize 000\ }

 \DocInput{childdoc.dtx}

\end{document}
%</driver>
% \fi
%
% %%%%%%%%%%%%%%%%%%%%%%%%%%%%%%%%%%%%%%%%%%%%%%%%%%%%%%%%%%%%%%%%%%%%%%%%%%%%%%
% %%%%%%%%%%%%%%%%%%%%%%%%%%%%%%%%%%%%%%%%%%%%%%%%%%%%%%%%%%%%%%%%%%%%%%%%%%%%%%
% \section{Sample}
%\iffalse
%<*samplemain>
%\fi
%
% The following presents a sample document
% with two chapters, two parts, a title page,
% a compile flag as well as three forwarding files to set the flag.
% It consists of eight |.tex| files:
% \begin{center}
% \begin{tabular}{ll}
% |cdocsamp.tex|&main file\\
% |cdocsch1.tex|&include file for chapter 1\\
% |cdocsch2.tex|&include file for chapter 2\\
% |cdocspt3.tex|&include file for part 3\\
% |cdocspt4.tex|&include file for part 4\\
% |cdocsdrf.tex|&forwarding file for main file in draft mode\\
% |cdocsfi1.tex|&forwarding file for final version of chapter 1\\
% |cdocsfi2.tex|&forwarding file for final version of chapter 2\\
% \end{tabular}
% \end{center}
% Each of the eight files can be compiled directly by the \LaTeX{} compiler.
%
% %%%%%%%%%%%%%%%%%%%%%%%%%%%%%%%%%%%%%%
% \paragraph{Main File.}
%
% The main file is called |cdocsamp.tex|.
%
% Load the \textsf{childdoc} definitions and
% declare the filename for the main document:
%    \begin{macrocode}
\input{childdoc.def}
\childdocmain{}
%    \end{macrocode}

% Optional override for |\version| flag:
%    \begin{macrocode}
%%\ifchilddoc\else\providecommand{\version}{draft}\fi
%    \end{macrocode}

% Define the default values for the |\version| flag
% (|final| for the main file and |draft| for childs):
%    \begin{macrocode}
\ifchilddoc
\providecommand{\version}{draft}
\else
\providecommand{\version}{final}
\fi
%    \end{macrocode}

% Load the standard document class:
%    \begin{macrocode}
\documentclass[12pt]{article}
%    \end{macrocode}

% Start the document body:
%    \begin{macrocode}
\begin{document}
%    \end{macrocode}

% Declare a title page.
% Print title, part of document being processed and version flag:
%    \begin{macrocode}
\addtocounter{page}{-1}
\begin{center}
{\LARGE\bfseries{}childdoc example\par}
\vspace{1cm}
\ifchilddoc
\ifchilddocmanual part\else chapter\fi:
`\childdocname' of `\childdocjob'\par
\else
main document: `\childdocjob'\par
\fi
version: \version\par
\end{center}
\newpage
%    \end{macrocode}

% Manually include selected file,
% otherwise process as usual:
%    \begin{macrocode}
\ifchilddocmanual
\section*{part `\childdocname'}
\input{\childdocname}
\else
%    \end{macrocode}

% Include the two chapters:
%    \begin{macrocode}
\include{cdocsch1}
\include{cdocsch2}
%    \end{macrocode}

% Include the two parts unless only chapters should be displayed:
%    \begin{macrocode}
\ifchilddoc\else
\section{part three}
\input{cdocspt3}
\section{part four}
\input{cdocspt4}
\fi
%    \end{macrocode}

% Process as usual until here:
%    \begin{macrocode}
\fi
%    \end{macrocode}

% End of document body:
%    \begin{macrocode}
\end{document}
%    \end{macrocode}
%\iffalse
%</samplemain>
%\fi
%
% %%%%%%%%%%%%%%%%%%%%%%%%%%%%%%%%%%%%%%
% \paragraph{Chapter Include Files.}
%
% The include files are called |cdocsch1.tex| and |cdocsch2.tex|.
%
%\iffalse
%<*samplechap1|samplechap2>
%\fi

% Optional override for |\version| flag:
%    \begin{macrocode}
%%\providecommand{\version}{final}
%    \end{macrocode}

% Include the main document:
%    \begin{macrocode}
\input{childdoc.def}
\childdocof{cdocsamp}
%    \end{macrocode}

%\iffalse
%</samplechap1|samplechap2>
%\fi
%
%\iffalse
%<*samplechap1>
%\fi
% Some text for chapter 1:
%    \begin{macrocode}
\section{one}
some text in chapter one
%    \end{macrocode}

%\iffalse
%</samplechap1>
%\fi
% Some text for chapter 2:
%\iffalse
%<*samplechap2>
%\fi
%    \begin{macrocode}
\section{two}
more text in chapter two
%    \end{macrocode}

%\iffalse
%</samplechap2>
%\fi
%
% %%%%%%%%%%%%%%%%%%%%%%%%%%%%%%%%%%%%%%
% \paragraph{Part Include Files.}
%
% The include files are called |cdocspt3.tex| and |cdocspt4.tex|.
%
%\iffalse
%<*samplepart3|samplepart4>
%\fi

% Optional override for |\version| flag:
%    \begin{macrocode}
%%\providecommand{\version}{final}
%    \end{macrocode}

% Include the main document:
%    \begin{macrocode}
\input{childdoc.def}
\childdocby{cdocsamp}
%    \end{macrocode}

%\iffalse
%</samplepart3|samplepart4>
%\fi
%
%\iffalse
%<*samplepart3>
%\fi
% Some text for part 3:
%    \begin{macrocode}
some text in part three
%    \end{macrocode}

%\iffalse
%</samplepart3>
%\fi
% Some text for part 4:
%\iffalse
%<*samplepart4>
%\fi
%    \begin{macrocode}
more text in part four
%    \end{macrocode}

%\iffalse
%</samplepart4>
%\fi
%
% %%%%%%%%%%%%%%%%%%%%%%%%%%%%%%%%%%%%%%
% \paragraph{Forwarding for a Complete Draft.}
%
% The following forwarding file |cdocsdrf.tex|
% compiles the main document in draft mode:
%\iffalse
%<*sampledraft>
%\fi
%    \begin{macrocode}
\def\version{draft}
\input{childdoc.def}
\childdocforward{cdocsamp}
%    \end{macrocode}

%\iffalse
%</sampledraft>
%\fi
%
% %%%%%%%%%%%%%%%%%%%%%%%%%%%%%%%%%%%%%%
% \paragraph{Forwarding for Final Version of the Chapters.}
%
% The following forwarding files |cdocsfn1.tex| and |cdocsfn2.tex|
% (with identical content)
% compile the final versions of the child documents
% |cdocsch1.tex| and |cdocsch2.tex|, respectively:
%\iffalse
%<*samplefinal>
%\fi
%    \begin{macrocode}
\def\version{final}
\input{childdoc.def}
\childdocforwardprefix[cdocsamp]{cdocsfn}{cdocsch}
%    \end{macrocode}

%\iffalse
%</samplefinal>
%\fi
%
% %%%%%%%%%%%%%%%%%%%%%%%%%%%%%%%%%%%%%%
% \paragraph{Command Line Processing.}
%
% The following three command lines generate the output files
% |cdocscld|, |cdocscl1| and |cdocscl2|
% which should be identical to
% |cdocsdrf|, |cdocsch1| and |cdocsfn2|, respectively:
% \begin{center}
% \begin{tabular}{l}
% |latex -jobname cdocscld \|\\
% |  "\def\version{draft}\input{childdoc.def}\childdocforward{cdocsamp}"|\\
% |latex -jobname cdocscl1 \|\\
% |  "\input{childdoc.def}\childdocforward[cdocsamp]{cdocsch1}"|\\
% |latex -jobname cdocscl2 \|\\
% |  "\def\version{final}\input{childdoc.def}\childdocforward{cdocsch2}"|
% \end{tabular}
% \end{center}
% Note that the trailing backslash on each first line
% merely continues the input to the second line
% (for convenient cut ant paste).
% Furthermore, the command |latex| can be replaced by any
% of its alternative versions such as |pdflatex|.
%
% %%%%%%%%%%%%%%%%%%%%%%%%%%%%%%%%%%%%%%%%%%%%%%%%%%%%%%%%%%%%%%%%%%%%%%%%%%%%%%
% %%%%%%%%%%%%%%%%%%%%%%%%%%%%%%%%%%%%%%%%%%%%%%%%%%%%%%%%%%%%%%%%%%%%%%%%%%%%%%
% \section{Implementation}
%\iffalse
%<*package>
%\fi
%
% This section describes the definitions file |childdoc.def|.

% The definitions cannot be loaded using |\usepackage| or |\RequirePackage|
% which has a mechanism to prevent loading a style file more than once.
% When loading the definitions by means of |\input|
% multiple instances have to be prevented manually:
%\iffalse
%This code needs to be before the `\ProvidesFile' directive
%which is defined at the beginning of this file.
%Therefore it is also placed there and commented out here.
%</package>
%<*discard>
%\fi
%    \begin{macrocode}
\ifdefined\childdocmain\endinput\fi
%    \end{macrocode}
%\iffalse
%</discard>
%<*package>
%\fi
%
% \macro{\ifchilddoc}
% \macro{\ifchilddocmanual}
% The conditional |\ifchilddoc| tells whether a
% child (true) or main (false) document is being compiled.
% The conditional |\ifchilddocmanual| tells whether
% the |\includeonly| mechanism is used (false) or
% the selection of child files must be performed manually (true).
% The definitions initialise to false:
%    \begin{macrocode}
\newif\ifchilddoc
\newif\ifchilddocmanual
%    \end{macrocode}

% \macro{\childdocname}
% \macro{\childdocjob}
% The macro |\childdocname| stores the name of the main document
% to be compiled. The macro |\childdocjob| stores the name of
% the document on which the \LaTeX{} compiler was originally invoked.
% The content of |\jobname| cannot be compared
% to filenames specified in the source due to different catcodes.
% The following code rescans |\jobname|, stores the result
% in |\childdocname| and saves a copy in |\childdocjob|:
%    \begin{macrocode}
\edef\childdocname{\scantokens\expandafter{\jobname\noexpand}}
\let\childdocjob\childdocname
%    \end{macrocode}

% \macro{\childdocdisable}
% The macro |\childdocdisable| prevents the main file
% from being processed more than once.
% At this stage, the main document command |\childdocmain|
% is assumed to be called once again where it should do nothing.
% Any subsequent call to it should prevent
% a secondary processing of the main document
% It overwrites the forwarding commands
% |\childdocof| and |\childdocforward|
% with empty macros to prevent further inclusions of the main document:
%    \begin{macrocode}
\newcommand{\childdocdisable}
{
  \renewcommand{\childdocmain}[1]{\renewcommand{\childdocmain}[1]{\endinput}}
  \renewcommand{\childdocof}[1]{}
  \renewcommand{\childdocby}[2][]{}
  \renewcommand{\childdocforward}[2][]{}
  \renewcommand{\childdocdisable}{}
}
%    \end{macrocode}

% \macro{\childdocmain}
% The macro |\childdocmain| is to be called at the top of the main file
% with nothing or the main filename (without extension) as argument.
% First, it breaks loops.
% If the argument is not empty and does not match |\childdocname|
% (which is set by the first inclusion of |childdoc.def|),
% |\ifchilddoc| is set to true, |\includeonly| is applied to the child file
% and |\jobname| is set to the main file
% (for proper handling of |.aux| files):
%    \begin{macrocode}
\newcommand{\childdocmain}[1]
{
  \childdocdisable\childdocmain{}
  \if?#1?\else
    \begingroup
      \def\childdoctmp{#1}
      \ifx\childdoctmp\childdocname
        \def\childdoctmp{}
      \else
        \def\childdoctmp
        {
          \childdoctrue
          \includeonly{\childdocname}
          \def\childdocjob{#1}
          \def\jobname{#1}
        }
      \fi
      \expandafter
    \endgroup
    \childdoctmp
  \fi
}
%    \end{macrocode}

% \macro{\childdocof}
% The command |\childdocof| redirects
% compilation to the main file |#1|.
%    \begin{macrocode}
\newcommand{\childdocof}[1]
{
  \childdocdisable
  \childdoctrue
  \includeonly{\childdocname}
  \def\jobname{#1}
  \def\childdocjob{#1}
  \input{#1}
}
%    \end{macrocode}

% \macro{\childdocby}
% The command |\childdocby| ....
%    \begin{macrocode}
\newcommand{\childdocby}[2][]
{
  \childdocdisable
  \childdoctrue
  \childdocmanualtrue
  \if?#1?\else
    \def\jobname{#2}
  \fi
  \def\childdocjob{#2}
  \input{#2}
  \endinput
}
%    \end{macrocode}

% \macro{\childdocforward}
% The command |\childdocforward| redirects
% compilation to the main file or
% (if the optional argument is given) a child file.
% Parameters are set as if the main file
% or a child file starting with |\childdocof| was compiled.
% Then compilation is handed over to the main file:
%    \begin{macrocode}
\newcommand{\childdocforward}[2][]
{
  \begingroup
    \if?#1?
      \def\childdoctmp
      {
        \def\childdocname{#2}
        \def\childdocjob{#2}
        \def\jobname{#2}
        \input{#2}
        \endinput
      }
    \else
      \def\childdoctmp
      {
        \childdocdisable
        \def\childdocname{#2}
        \childdoctrue
        \includeonly{#2}
        \def\childdocjob{#1}
        \def\jobname{#1}
        \input{#1}
        \endinput
      }
    \fi
    \expandafter
  \endgroup
  \childdoctmp
}
%    \end{macrocode}

% \macro{\childdocforwardprefix}
% The command |\childdocforwardprefix| redirects
% compilation to the main or a child file by means of a pattern.
% The prefix |#1| in the current filename is replaced by |#2|
% and the suffix of the current filename is kept
% (it is assumed that the filename does not contain the substring `|~~~|'
% which is used as a delimiter).
% Compilation is handed over to the new file by |\childdocforward|:
%    \begin{macrocode}
\newcommand{\childdocforwardprefix}[3][]
{
  \begingroup
    \def\childdocextract #2##1~~~{\def\childdoctmp{\childdocforward[#1]{#3##1}}}
    \expandafter\childdocextract\childdocname~~~
    \expandafter
  \endgroup
  \childdoctmp
}
%    \end{macrocode}

% \macro{\childdoc}
% The deprecated macro |\childdoc| is a legacy version of |\childdocmain|:
%    \begin{macrocode}
\newcommand{\childdoc}{\childdocmain}
%    \end{macrocode}

% \macro{\childdocredirect}
% The deprecated macro |\childdocredirect| is a legacy version
% of |\childdocforward| and |\childdocforwardprefix|:
%    \begin{macrocode}
\newcommand{\childdocredirect}[2][]
{
  \begingroup
    \if?#1?
      \def\childdoctmp{\childdocforward{#2}}
    \else
      \def\childdoctmp{\childdocforwardprefix{#1}{#2}}
    \fi
    \expandafter
  \endgroup
  \childdoctmp
}
%    \end{macrocode}

%\iffalse
%</package>
%\fi
%
\endinput
|\\
|\childdocmain{}|\\
\end{tabular}
\end{center}
at the very top of the main \LaTeX{} file,
in particular \emph{before} the |\documentclass| statement!
The argument of |\childdocmain| should be left empty
(but it must be present).

%%%%%%%%%%%%%%%%%%%%%%%%%%%%%%%%%%%%%%%%
\DescribeMacro{\childdocof}
Furthermore, add the commands
\begin{center}
\begin{tabular}{l}
|% \iffalse
%
% childdoc.dtx Copyright (C) 2017-2018 Niklas Beisert
%
% This work may be distributed and/or modified under the
% conditions of the LaTeX Project Public License, either version 1.3
% of this license or (at your option) any later version.
% The latest version of this license is in
%   http://www.latex-project.org/lppl.txt
% and version 1.3 or later is part of all distributions of LaTeX
% version 2005/12/01 or later.
%
% This work has the LPPL maintenance status `maintained'.
%
% The Current Maintainer of this work is Niklas Beisert.
%
% This work consists of the files childdoc.dtx and childdoc.ins
% and the derived files childdoc.def and cdocsamp.tex with
% cdocsch1.tex, cdocsch2.tex, cdocsdrf.tex, cdocsfn1.tex, cdocsfn2.tex.
%
%<package>\ifdefined\childdocmain\endinput\fi
%<package>\ProvidesFile{childdoc.def}[2018/12/30 v2.0 child document driver]
%<samplemain>\ProvidesFile{cdocsamp.tex}[2018/12/30 v2.0 sample for childdoc]
%<*driver>
%\ProvidesFile{childdoc.drv}[2018/12/30 v2.0 childdoc reference manual file]
\PassOptionsToClass{10pt,a4paper}{article}
\documentclass{ltxdoc}

\usepackage[margin=35mm]{geometry}
\usepackage{hyperref}
\usepackage{hyperxmp}
\usepackage[usenames]{color}

\hypersetup{colorlinks=true}
\hypersetup{pdfstartview=FitH}
\hypersetup{pdfpagemode=UseNone}
\hypersetup{pdfsource={}}
\hypersetup{pdflang={en-UK}}
\hypersetup{pdfcopyright={Copyright 2017-2018 Niklas Beisert.
  This work may be distributed and/or modified under the
  conditions of the LaTeX Project Public License, either version 1.3
  of this license or (at your option) any later version.}}
\hypersetup{pdflicenseurl={http://www.latex-project.org/lppl.txt}}
\hypersetup{pdfcontactaddress={ETH Zurich, ITP, HIT K,
  Wolfgang-Pauli-Strasse 27}}
\hypersetup{pdfcontactpostcode={8093}}
\hypersetup{pdfcontactcity={Zurich}}
\hypersetup{pdfcontactcountry={Switzerland}}
\hypersetup{pdfcontactemail={nbeisert@itp.phys.ethz.ch}}
\hypersetup{pdfcontacturl={http://people.phys.ethz.ch/\xmptilde nbeisert/}}

\newcommand{\secref}[1]{\hyperref[#1]{section \ref*{#1}}}

\parskip1ex
\parindent0pt
\let\olditemize\itemize
\def\itemize{\olditemize\parskip0pt}

\begin{document}

\title{The \textsf{childdoc} Package}
\hypersetup{pdftitle={The childdoc Package}}
\author{Niklas Beisert\\[2ex]
  Institut f\"ur Theoretische Physik\\
  Eidgen\"ossische Technische Hochschule Z\"urich\\
  Wolfgang-Pauli-Strasse 27, 8093 Z\"urich, Switzerland\\[1ex]
  \href{mailto:nbeisert@itp.phys.ethz.ch}
  {\texttt{nbeisert@itp.phys.ethz.ch}}}
\hypersetup{pdfauthor={Niklas Beisert}}
\hypersetup{pdfsubject={Manual for the LaTeX2e Package childdoc}}
\date{30 December 2018, \textsf{v2.0}}
\maketitle

\begin{abstract}\noindent
\textsf{childdoc} is a \LaTeXe{} package
that enables the direct compilation
of document sections included by |\include|
to individual files.
\end{abstract}

\begingroup
\parskip0ex
\tableofcontents
\endgroup

%%%%%%%%%%%%%%%%%%%%%%%%%%%%%%%%%%%%%%%%%%%%%%%%%%%%%%%%%%%%%%%%%%%%%%%%%%%%%%%%
%%%%%%%%%%%%%%%%%%%%%%%%%%%%%%%%%%%%%%%%%%%%%%%%%%%%%%%%%%%%%%%%%%%%%%%%%%%%%%%%
\section{Introduction}

\LaTeX{} provides a mechanism to structure a large document (such as a book)
into a main file and several child files (containing the chapters)
using the |\include| command.
This mechanism is beneficial for documents
which span hundreds of pages in order to
make the source file(s) more manageable.
Moreover, compilation can be restricted to
selected child files by means of the |\includeonly| command.
The latter feature can be used to reduce the compilation time while editing
(this was significantly more useful in the earlier days of \LaTeX{})
or to generate a smaller document which is easier to navigate.
Another application of |\includeonly| is to generate
documents consisting of selected parts of the complete document.

However, there are a few drawbacks of the plain |\include| mechanism:
\begin{itemize}
\item
The child files cannot be compiled on their own,
they can only be compiled via the main file.
A naive editing environment
(such as a text editor with an option
to have the current file processed by \LaTeX)
may require one to switch to the main file before compiling;
attempting to compile the child file produces errors.
\item
The main file must be modified (each time)
to adjust the |\includeonly| command
to the present needs. This easily leaves the main file in a messy state.
\item
The generated document will always carry the filename
of the main document. This is inconvenient if
several child files are to be compiled and
to be kept for distribution.
\end{itemize}

The present package provides a simple interface
to make child files individually compilable by \LaTeX{}.
Compiling a child file then has the same effect as compiling
the main file with an |\includeonly| command
to select the appropriate child.
Moreover the generated document will carry the name of the child
rather than the main file.
This resolves all three above issues.

This feature is meant to make the editing of books,
thesis documents and lecture notes somewhat more convenient.
However, the package can also be used efficiently for
composing a series of documents (such as exercise sheets)
which are typically distributed individually.
It then assists the author in generating the individual documents
(potentially in different versions)
as well as a document containing the collected series.
Another application is in developing style files
or other kinds of included material
where compilation of the style file could redirect
to a sample or test file.

%%%%%%%%%%%%%%%%%%%%%%%%%%%%%%%%%%%%%%%%%%%%%%%%%%%%%%%%%%%%%%%%%%%%%%%%%%%%%%%%
%%%%%%%%%%%%%%%%%%%%%%%%%%%%%%%%%%%%%%%%%%%%%%%%%%%%%%%%%%%%%%%%%%%%%%%%%%%%%%%%
\section{Usage}

First of all, the package \textsf{childdoc} is \emph{not} a standard
\LaTeXe{} |.sty| style file! Therefore it needs to be invoked in
a non-standard way.

%%%%%%%%%%%%%%%%%%%%%%%%%%%%%%%%%%%%%%%%%%%%%%%%%%%%%%%%%%%%%%%%%%%%%%%%%%%%%%%%
\subsection{Included Files}
\label{sec:include}

%%%%%%%%%%%%%%%%%%%%%%%%%%%%%%%%%%%%%%%%
\DescribeMacro{\childdocmain}
To use the package, add the commands
\begin{center}
\begin{tabular}{l}
|\input{childdoc.def}|\\
|\childdocmain{}|\\
\end{tabular}
\end{center}
at the very top of the main \LaTeX{} file,
in particular \emph{before} the |\documentclass| statement!
The argument of |\childdocmain| should be left empty
(but it must be present).

%%%%%%%%%%%%%%%%%%%%%%%%%%%%%%%%%%%%%%%%
\DescribeMacro{\childdocof}
Furthermore, add the commands
\begin{center}
\begin{tabular}{l}
|\input{childdoc.def}|\\
|\childdocof{|\textit{main}|}|\\
\end{tabular}
\end{center}
at the top of every child file \textit{child}
which is included by |\include{|\textit{child}|}|
from within the main file
(or at least for those files to be compiled individually).
The argument \textit{main} must be the filename of the main file.

There are a couple of
considerations in setting up the main and child documents:

%%%%%%%%%%%%%%%%%%%%%%%%%%%%%%%%%%%%%%%%
\paragraph{Restrictions.}

Please note the following restrictions:
\begin{itemize}
\item
|\childdocmain| must be called with one argument \textit{main}
to ensure compatibility with earlier version of the package.
It must either be empty (|\childdocmain{}|)
or precisely match the filename of the main file in which it is specified.
See \secref{sec:detection} for further information.
\item
The filename \textit{main} must be specified without the |.tex| extension.
\item
The filename \textit{main} is case sensitive
(even in case-insensitive file systems)
due to internal string comparison.
\item
The argument \textit{main} should be fully expanded, it cannot be a macro.
\item
Subdirectories and special characters should be avoided in filenames.
\item
The command |\childdocmain{|\textit{main}|}| must be followed by a whitespace.
It should not be followed immediately by another command
or by a comment mark `|%|'.
This is because the \TeX{} parser reads the token immediately following
the argument of |\childdocmain| and puts it
at the beginning of every child section;
however, a white\-space is ignored.
\end{itemize}

%%%%%%%%%%%%%%%%%%%%%%%%%%%%%%%%%%%%%%%%
\paragraph{Content of Main File.}

It is advisable to place all content in the child files included by |\include|.
Any output contained in the main file will appear in all child documents
unless suppressed manually;
it cannot be suppressed automatically by the |\includeonly| directive
and thus should normally be avoided.
A method to include some content in the main file
by means of conditional processing is described in \secref{sec:conditional}.

%%%%%%%%%%%%%%%%%%%%%%%%%%%%%%%%%%%%%%%%
\paragraph{Page Numbering.}

When only a part of the document is compiled,
the appropriate numbering of pages
(as well as other status parameters)
is determined from the |.aux| files.
The latter contain information from previous passes.
However this information needs to propagate through
all intermediate child documents.
Therefore the page numbering in child documents may well
be inconsistent until the complete document is compiled at least once.

A useful (if unconventional) way to always ensure a consistent
page numbering is to restart the numbering in each child document
and denote the pages by `\textit{child}|.|\textit{page}'
where \textit{child} represents the chapter/section number of the child file.
This can be achieved by the command
|\numberwithin{page}{|\textit{child}|}|
of the \textsf{amsmath} package
where \textit{child} can be |chapter| or |section|
depending on the chosen structuring.
Alternatively, one can modify the macro |\thepage| appropriately
and reset the counter |page| at the start of each child file.

%%%%%%%%%%%%%%%%%%%%%%%%%%%%%%%%%%%%%%%%%%%%%%%%%%%%%%%%%%%%%%%%%%%%%%%%%%%%%%%%
\subsection{Conditional Processing}
\label{sec:conditional}

The package provides a mechanism to compile different versions
of a document. To customise the versions further some conditional processing
can come in handy to distinguish which version is being compiled.
The package provides two macros to describe the compilation context:

%%%%%%%%%%%%%%%%%%%%%%%%%%%%%%%%%%%%%%%%
\DescribeMacro{\ifchilddoc}
The conditional |\ifchilddoc| distinguishes between the compilation of
child documents and the main document:
%
\begin{center}
|\ifchilddoc |\textit{child-code}| |[|\||else |\textit{main-code}]| \||fi|
\end{center}

%%%%%%%%%%%%%%%%%%%%%%%%%%%%%%%%%%%%%%%%
\DescribeMacro{\childdocname}
\DescribeMacro{\childdocjob}
The macro |\childdocname| contains the filename (without extension)
of the main or child file being processed.
Note that |\childdocjob| will always contain the name of the main file.

%%%%%%%%%%%%%%%%%%%%%%%%%%%%%%%%%%%%%%%%
\paragraph{Title Page.}

Conditional processing can be used to include a title or banner page
in the main document when proper precautions are taken.
Importantly, the code in the main file should ensure that the page counter
(as well as other status parameters which are stored in the |.aux| files)
takes the same value after the conditional processing.
Otherwise the page numbers may take divergent values
depending on which part is compiled.

For example, a title page could be declared by:
%
\begin{center}
\begin{tabular}{l}
|\ifchilddoc\||else|\\
|\addtocounter{page}{-1}|\\
\textit{code for title page}\\
|\newpage|\\
|\||fi|
\end{tabular}
\end{center}
%
A banner page for the child documents can be generated by:
%
\begin{center}
\begin{tabular}{l}
|\ifchilddoc|\\
|\addtocounter{page}{-1}|\\
\textit{code for banner page}\\
|\newpage|\\
|\||fi|
\end{tabular}
\end{center}
%
Here one could write a message such as:
\begin{center}
|This is the part \childdocname{} of \childdocjob{}.|
\end{center}

%%%%%%%%%%%%%%%%%%%%%%%%%%%%%%%%%%%%%%%%%%%%%%%%%%%%%%%%%%%%%%%%%%%%%%%%%%%%%%%%
\subsection{Flags}
\label{sec:flags}

The package makes it easy to generate different versions
of the main or child documents.
To this end compilation flags can be defined
and assigned different default values.
They will be particularly useful in conjunction
with the forwarding mechanism described in \secref{sec:forward}.

For example, it may be useful to have a flag |\version|
which can be set to |draft| or |final|.
The document source will contain some conditional code
depending on the value of |\version|.
Suppose further, the flag should default to |final| for the main file
and to |draft| for child files
which is a natural assignment for editing the document.
This is achieved by placing the following code
in the preamble of the main document
(below the |\childdocmain| directive):
%
\begin{center}
\begin{tabular}{l}
|\ifchilddoc|\\
|\providecommand{\version}{draft}|\\
|\||else|\\
|\providecommand{\version}{final}|\\
|\||fi|
\end{tabular}
\end{center}
%
The definition by |\providecommand| makes sure
that previous definitions are not overwritten.
Further statements |\providecommand{\version}{...}|
can thus be added before the above code to override it.

For the main file, one might add a line
(between |\childdocmain| and the above block)
%
\begin{center}
|%\ifchilddoc\||else\providecommand{\version}{draft}\||fi|
\end{center}
%
which can be uncommented to produce a draft version.
Likewise one can add a line to the very top of a child file
(above the |\childdocof{|\textit{main}|}| directive)
%
\begin{center}
|%\providecommand{\version}{final}|
\end{center}
%
which can be uncommented to produce the final version of this child document.

%%%%%%%%%%%%%%%%%%%%%%%%%%%%%%%%%%%%%%%%%%%%%%%%%%%%%%%%%%%%%%%%%%%%%%%%%%%%%%%%
\subsection{Forwarding}
\label{sec:forward}

Different versions of the main or child documents
using compilation flags as described in \secref{sec:flags}
can be (permanently) stored in different files
for convenient compilation, viewing and distribution.
To this end, the package defines a command
to pass on compilation to a different file:

%%%%%%%%%%%%%%%%%%%%%%%%%%%%%%%%%%%%%%%%
\DescribeMacro{\childdocforward}
The command |\childdocforward| redirects processing to
another source file:
%
\begin{center}
\begin{tabular}{l}
|\input{childdoc.def}|\\
|\childdocforward[|\textit{main}|]{|\textit{dest}|}|\\
\end{tabular}
\end{center}
%
The argument \textit{dest} is the destination file
(without extension).
It should be the main file or one of the child files.
Note that further \textsf{childdoc} directives
such as |\childdocof| and |\childdocforward|
in the indicated file will be processed in this form.
The optional argument \textit{main}
passes on directly to the main file \textit{main}
while pretending to compile the child \textit{dest}.
This form behaves as if \textit{dest}
issues |\childdocof{|\textit{main}|}| right away,
and no further \textsf{childdoc} directives will be processed.

%%%%%%%%%%%%%%%%%%%%%%%%%%%%%%%%%%%%%%%%
\DescribeMacro{\...prefix}
In the alternative form |\childdocforwardprefix|,
%
\begin{center}
\begin{tabular}{l}
|\input{childdoc.def}|\\
|\childdocforwardprefix[|\textit{main}|]{|\textit{prefix}|}{|\textit{dest}|}|
\end{tabular}
\end{center}
%
the destination file is determined by a pattern
depending on the current file:
To make this work, the current file must be called
`{\textit{prefix}\hspace{0.2em}\textit{suffix}}'
with \textit{prefix} matching precisely the argument.
Processing is then passed on to the file
`{\textit{dest}\hspace{0.2em}\textit{suffix}}'.
Surely, the same effect is achieved by
directly specifying the
argument `{\textit{dest}\hspace{0.2em}\textit{suffix}}'
in the first form.
However, that requires to set up a different file
for each child. With the alternative form of the command
all these files can have exactly the same content
which simplifies setting them up and maintaining them.

For example, the following file |draft.tex|
with a compilation flag |\version| as described in \secref{sec:flags}
compiles the main document as a draft:
%
\begin{center}
\begin{tabular}{l}
|\def\version{draft}|\\
|\input{childdoc.def}|\\
|\childdocforward{|\textit{main}|}|
\end{tabular}
\end{center}
%
Likewise, the following files |final|\textit{nn}|.tex|
compile the final version of the child document
|child|\textit{nn}|.tex|:
%
\begin{center}
\begin{tabular}{l}
|\def\version{final}|\\
|\input{childdoc.def}|\\
|\childdocforwardprefix{final}{child}|
\end{tabular}
\end{center}
%

Note that when several versions of a main file and/or of each child file
are to be generated, it may be convenient to set up a |Makefile| or
shell script to automatise the process.

%%%%%%%%%%%%%%%%%%%%%%%%%%%%%%%%%%%%%%%%%%%%%%%%%%%%%%%%%%%%%%%%%%%%%%%%%%%%%%%%
\subsection{Command Line Processing}
\label{sec:commandline}

The effect of redirection files can also be achieved by invoking
the \LaTeX{} compiler with a more elaborate command line.
Most conveniently this should be done as part
of a shell script or a |Makefile|.

When using \textsf{childdoc} in the main file, the following
command lines effectively perform a redirection
(note that depending on the shell being used,
backslashes may have to be doubled: `|\|' $\to$ `|\\|'):
%
\begin{center}
|... -jobname "|\textit{target}|" |\\|"|[\textit{flags}]%
|\input{childdoc.def}\childdocforward[|\textit{main}|]{|\textit{dest}|}"|
\end{center}
%
Here \textit{target} is the name of the output file,
\textit{main} is the name of the main file
and \textit{dest} is the name of the main or child file to be processed
(all filenames without extensions).
The optional argument \textit{main} can be omitted
if \textit{main} matches \textit{dest}.
Optionally, compilation \textit{flags} can be defined via |\def| commands.
This command line makes the \TeX{} engine believe
it is compiling the file \textit{target}
whose content is specified as the latter parameter.
The provided code then forwards the processing to
\textit{main} or \textit{dest} as described in \secref{sec:forward}.

%%%%%%%%%%%%%%%%%%%%%%%%%%%%%%%%%%%%%%%%%%%%%%%%%%%%%%%%%%%%%%%%%%%%%%%%%%%%%%%%
\subsection{Include by Input}
\label{sec:input}

Including child documents by |\include| has some restrictions by design.
Most notably, the content of a child document always occupies
its own set of pages; pages cannot be shared between child documents.
Usually, this behaviour makes perfect sense
because each child document contain an essential part of the document.
However, in some situations it may be desirable to compose
a document from a collection of parts
without having mandatory page breaks between then.
For this case, the package
provides a mechanism to include parts
by |\input| which can also be processed individually.
However, by construction this mechanism
requires manual handling of the content to be output.

%%%%%%%%%%%%%%%%%%%%%%%%%%%%%%%%%%%%%%%%
\DescribeMacro{\ifchilddocmanual}
The main file should be prepared as usual, see \secref{sec:include}.
However, the document body must make a distinction
between processing of an individual part and of the main document, e.g.:
%
\begin{center}
\begin{tabular}{l}
|\ifchilddocmanual|\\
|\input{\childdocname}|\\
|\||else|\\
\textit{document body with }|\input{|\textit{part}|}|\\
|\||fi|
\end{tabular}
\end{center}
%
The conditional |\ifchilddocmanual| is true whenever
a part to be included by |\input| is being compiled,
and the name of the part is stored in |\childdocname|.

%%%%%%%%%%%%%%%%%%%%%%%%%%%%%%%%%%%%%%%%
\DescribeMacro{\childdocby}
Each part to be included by |\input| should start with:
%
\begin{center}
\begin{tabular}{l}
|\input{childdoc.def}|\\
|\childdocby{|\textit{main}|}|\\
\end{tabular}
\end{center}
%
The directive |\childdocby| is similar to |\childdocof|
described in \secref{sec:include},
but the subsequent selection of content must be done manually.
To that end, both |\ifchilddoc| and |\ifchilddocmanual|
will be true upon processing of a part,
and the name of the part is stored in |\childdocname|.
Note that |\jobname| will be set to the filename of the current part
so that each part receives an individual |.aux| file
that does not interfere with the |.aux| file(s) of the main document.
This behaviour can be altered by the alternative form
|\childdocby[*]{|\textit{main}|}| (with a non-empty optional argument)
which uses the |.aux| file of the main document
by setting |\jobname| to \textit{main}.

%%%%%%%%%%%%%%%%%%%%%%%%%%%%%%%%%%%%%%%%%%%%%%%%%%%%%%%%%%%%%%%%%%%%%%%%%%%%%%%%
\subsection{Driver Development}
\label{sec:driver}

The \textsf{childdoc} mechanism can also be use for the development
of definition files such as \LaTeX{} styles or classes.
This case differs from the above setup with multiple parts
included by |\include| in that no |\includeonly| should be invoked.
This can be achieved by starting the include file
(before |\ProvidesPackage|) with:
%
\begin{center}
\begin{tabular}{l}
|\input{childdoc.def}|\\
|\childdocforward{|\textit{main}|}|\\
\end{tabular}
\end{center}
%
or alternatively with:
%
\begin{center}
\begin{tabular}{l}
|\input{childdoc.def}|\\
|\childdocby{|\textit{main}|}|\\
\end{tabular}
\end{center}
%
Both forms have slightly different effects as described above.
The main file is prepared as usual, see \secref{sec:include}.

%%%%%%%%%%%%%%%%%%%%%%%%%%%%%%%%%%%%%%%%%%%%%%%%%%%%%%%%%%%%%%%%%%%%%%%%%%%%%%%%
\subsection{Legacy Detection}
\label{sec:detection}

The directive |\childdocmain| in the main file can detect
whether the complete document or merely a child is to be compiled
even without using the directive |\childdocof|.
This method is deprecated because it is less robust
and there is no compelling reason to use it;
it is merely provided for backward compatibility
and it may be removed in future versions.

If the detection mechanism is to be used,
it is mandatory to correctly specify
the filename of the main file as the argument of |\childdocmain|:
%
\begin{center}
\begin{tabular}{l}
|\input{childdoc.def}|\\
|\childdocmain{|\textit{main}|}|\\
\end{tabular}
\end{center}
%
If |\jobname| does not match the argument \textit{main} of |\childdocmain|,
it is assumed that |\jobname| points to the child file to be compiled.
When using |\childdocmain| with the main file specified as argument,
it suffices to start a child file
with just |\input{|\textit{main}|}|
without loading of the package and using |\childdocof|.
If instead all processing is done
with the appropriate \textsf{childdoc} directives,
the argument of \textit{main} of |\childdocmain| can be empty.

An alternative version of the command line processing described
in \secref{sec:commandline} using the detection mechanism reads:
%
\begin{center}
|... -jobname "|\textit{target}|" "|[\textit{flags}]%
[|\def\jobname{|\textit{dest}|}|]|\input{|\textit{main}|}"|
\end{center}

%%%%%%%%%%%%%%%%%%%%%%%%%%%%%%%%%%%%%%%%%%%%%%%%%%%%%%%%%%%%%%%%%%%%%%%%%%%%%%%%
\subsection{Manual Code}
\label{sec:manual}

In case one cannot be certain whether the definitions file |childdoc.def|
is installed on the target \TeX{} distribution
and one prefers not to ship it,
it is conceivable to paste a few relevant commands into the sources.

To that end, drop all statements |\input{childdoc.def}|
and perform the replacements as outlined below.
Instead of |\childdocmain{|\textit{main}|}| add the following code
to the top of the main file:
%
\begin{center}
\begin{tabular}{l}
|\||ifdefined\childdocname\endinput\||fi\newif\ifchilddoc|\\
|\edef\childdocname{\scantokens\expandafter{\jobname\noexpand}}|\\
|\def\childdocmain{|\textit{main}|}\||ifx\childdocmain\childdocname\||else|\\
|\childdoctrue\includeonly{\childdocname}\let\jobname\childdocmain\||fi|\\
\end{tabular}
\end{center}
%
Instead of |\childdocof{|\textit{main}|}| just include the main file
at the top of each child file:
%
\begin{center}
|\input{|\textit{main}|}|
\end{center}
%
A simple redirection |\childdocforward{|\textit{dest}|}| is achieved by:
%
\begin{center}
|\def\jobname{|\textit{dest}|}\input{\jobname}|
\end{center}
%
The redirection with prefix
|\childdocforwardprefix[|\textit{prefix}|]{|\textit{dest}|}|
is accomplished by:
%
\begin{center}
\begin{tabular}{l}
|{\edef\jobname{\scantokens\expandafter{\jobname\noexpand}}|\\
|\def\redirectjob |\textit{prefix}|#1~~~{\gdef\jobname{|\textit{dest}|#1}}|\\
|\expandafter\redirectjob\jobname~~~}\input{\jobname}|
\end{tabular}
\end{center}

In an alternative approach,
child documents can be compiled by a specific command line
without additional code or specific definitions:
%
\begin{center}
|... -jobname "|\textit{target}|" "|[\textit{flags}]%
|\includeonly{|\textit{dest}|}\input{|\textit{main}|}"|
\end{center}
%

%%%%%%%%%%%%%%%%%%%%%%%%%%%%%%%%%%%%%%%%%%%%%%%%%%%%%%%%%%%%%%%%%%%%%%%%%%%%%%%%
%%%%%%%%%%%%%%%%%%%%%%%%%%%%%%%%%%%%%%%%%%%%%%%%%%%%%%%%%%%%%%%%%%%%%%%%%%%%%%%%
\section{Information}

%%%%%%%%%%%%%%%%%%%%%%%%%%%%%%%%%%%%%%%%%%%%%%%%%%%%%%%%%%%%%%%%%%%%%%%%%%%%%%%%
\subsection{Copyright}

Copyright \copyright{} 2017--2018 Niklas Beisert

This work may be distributed and/or modified under the
conditions of the \LaTeX{} Project Public License, either version 1.3
of this license or (at your option) any later version.
The latest version of this license is in
  \url{http://www.latex-project.org/lppl.txt}
and version 1.3 or later is part of all distributions of \LaTeX{}
version 2005/12/01 or later.

This work has the LPPL maintenance status `maintained'.

The Current Maintainer of this work is Niklas Beisert.

This work consists of the files |README.txt|, |childdoc.ins| and |childdoc.dtx|
as well as the derived files |childdoc.def|, |cdocsamp.tex|
with |cdocsch1.tex|, |cdocsch2.tex|, |cdocspt3.tex|, |cdocspt4.tex|,
|cdocsdrf.tex|, |cdocsfn1.tex|, |cdocsfn2.tex|
as well as |childdoc.pdf|.

%%%%%%%%%%%%%%%%%%%%%%%%%%%%%%%%%%%%%%%%%%%%%%%%%%%%%%%%%%%%%%%%%%%%%%%%%%%%%%%%
\subsection{Files and Installation}

The package consists of the files:
%
\begin{center}
\begin{tabular}{ll}
    |README.txt|   & readme file \\
    |childdoc.ins| & installation file \\
    |childdoc.dtx| & source file \\
    |childdoc.def| & definition file \\
    |cdocsamp.tex| & sample main file \\
    |cdocsch1.tex| & sample include file \\
    |cdocsch2.tex| & sample include file \\
    |cdocspt3.tex| & sample part file \\
    |cdocspt4.tex| & sample part file \\
    |cdocsdrf.tex| & sample redirection file \\
    |cdocsfn1.tex| & sample redirection file \\
    |cdocsfn2.tex| & sample redirection file \\
    |childdoc.pdf| & manual
\end{tabular}
\end{center}
%
The distribution consists of the files
|README.txt|, |childdoc.ins| and |childdoc.dtx|.
%
\begin{itemize}
\item
Run (pdf)\LaTeX{} on |childdoc.dtx|
to compile the manual |childdoc.pdf| (this file).
\item
Run \LaTeX{} on |childdoc.ins| to create the definitions file |childdoc.def|
and the sample |cdocsamp.tex| with include files
|cdocsch1.tex|, |cdocsch2.tex|, |cdocspt3.tex|, |cdocspt4.tex|,
|cdocsdrf.tex|, |cdocsfn1.tex|, |cdocsfn2.tex|.
Then copy the file |childdoc.def| to an appropriate directory of your \LaTeX{}
distribution, e.g.\ \textit{texmf-root}|/tex/latex/childdoc|.
\end{itemize}

%%%%%%%%%%%%%%%%%%%%%%%%%%%%%%%%%%%%%%%%%%%%%%%%%%%%%%%%%%%%%%%%%%%%%%%%%%%%%%%%
\subsection{Related CTAN Packages}

There are several other packages which offer a similar functionality:
%
\begin{itemize}
\item
The packages
\href{http://ctan.org/pkg/docmute}{\textsf{docmute}},
\href{http://ctan.org/pkg/includex}{\textsf{includex}} and
\href{http://ctan.org/pkg/standalone}{\textsf{standalone}}
provide commands to include only the document body of
a child file thus allowing both files to be compiled individually.
\item
The packages \href{http://ctan.org/pkg/subdocs}{\textsf{subdocs}}
and \href{http://ctan.org/pkg/subfiles}{\textsf{subfiles}}
provide structures in which the main and child documents can be
encapsulated and allowing them to be compiled individually.
The inclusion mechanism is different from the conventional |\include|.
\item
The package \href{http://ctan.org/pkg/combine}{\textsf{combine}}
is an elaborate solution to combine several documents into one.
\end{itemize}
%
See also the CTAN topic \href{http://ctan.org/topic/subdocs}{\textsf{subdocs}}
for further related packages.
The present package differs from the above solutions in that
a document structure constructed with the conventional |\include| mechanism
just needs two extra commands at the top of every file
such that all constituent files can be compiled individually.

%%%%%%%%%%%%%%%%%%%%%%%%%%%%%%%%%%%%%%%%%%%%%%%%%%%%%%%%%%%%%%%%%%%%%%%%%%%%%%%%
%\subsection{Feature Suggestions}
%
%The following is a list of features which may be useful for future
%versions of this package:
%%
%\begin{itemize}
%\item
%\ldots
%\end{itemize}

%%%%%%%%%%%%%%%%%%%%%%%%%%%%%%%%%%%%%%%%%%%%%%%%%%%%%%%%%%%%%%%%%%%%%%%%%%%%%%%%
\subsection{Revision History}

%%%%%%%%%%%%%%%%%%%%%%%%%%%%%%%%%%%%%%%%
\paragraph{v2.0:} 2018/12/30

\begin{itemize}
\item
immediate forward processing
\item
added |\childdocby| mechanism
\item
manual restructured
\end{itemize}

%%%%%%%%%%%%%%%%%%%%%%%%%%%%%%%%%%%%%%%%
\paragraph{v1.6:} 2018/01/17

\begin{itemize}
\item
application for development of include files
\item
corrections to manual
\end{itemize}

%%%%%%%%%%%%%%%%%%%%%%%%%%%%%%%%%%%%%%%%
\paragraph{v1.5:} 2017/05/21

\begin{itemize}
\item
more complete structuring introduced
\item
|\childdocof| introduced
\item
|\childdoc| renamed to |\childdocmain|
\item
|\childredirect| renamed to |\childdocforward| and |\childdocforwardprefix|
and functionality expanded
\end{itemize}

%%%%%%%%%%%%%%%%%%%%%%%%%%%%%%%%%%%%%%%%
\paragraph{v1.0:} 2017/04/27

\begin{itemize}
\item
manual and install package
\item
first version published on CTAN
\end{itemize}

%%%%%%%%%%%%%%%%%%%%%%%%%%%%%%%%%%%%%%%%
\paragraph{v0.6:} 2017/04/26

\begin{itemize}
\item
redirection mechanism added
\end{itemize}

%%%%%%%%%%%%%%%%%%%%%%%%%%%%%%%%%%%%%%%%
\paragraph{v0.5:} 2017/04/26

\begin{itemize}
\item
functionality in definition file
\end{itemize}


%%%%%%%%%%%%%%%%%%%%%%%%%%%%%%%%%%%%%%%%%%%%%%%%%%%%%%%%%%%%%%%%%%%%%%%%%%%%%%%%
%%%%%%%%%%%%%%%%%%%%%%%%%%%%%%%%%%%%%%%%%%%%%%%%%%%%%%%%%%%%%%%%%%%%%%%%%%%%%%%%
%%%%%%%%%%%%%%%%%%%%%%%%%%%%%%%%%%%%%%%%%%%%%%%%%%%%%%%%%%%%%%%%%%%%%%%%%%%%%%%%
\appendix

\settowidth\MacroIndent{\rmfamily\scriptsize 000\ }

 \DocInput{childdoc.dtx}

\end{document}
%</driver>
% \fi
%
% %%%%%%%%%%%%%%%%%%%%%%%%%%%%%%%%%%%%%%%%%%%%%%%%%%%%%%%%%%%%%%%%%%%%%%%%%%%%%%
% %%%%%%%%%%%%%%%%%%%%%%%%%%%%%%%%%%%%%%%%%%%%%%%%%%%%%%%%%%%%%%%%%%%%%%%%%%%%%%
% \section{Sample}
%\iffalse
%<*samplemain>
%\fi
%
% The following presents a sample document
% with two chapters, two parts, a title page,
% a compile flag as well as three forwarding files to set the flag.
% It consists of eight |.tex| files:
% \begin{center}
% \begin{tabular}{ll}
% |cdocsamp.tex|&main file\\
% |cdocsch1.tex|&include file for chapter 1\\
% |cdocsch2.tex|&include file for chapter 2\\
% |cdocspt3.tex|&include file for part 3\\
% |cdocspt4.tex|&include file for part 4\\
% |cdocsdrf.tex|&forwarding file for main file in draft mode\\
% |cdocsfi1.tex|&forwarding file for final version of chapter 1\\
% |cdocsfi2.tex|&forwarding file for final version of chapter 2\\
% \end{tabular}
% \end{center}
% Each of the eight files can be compiled directly by the \LaTeX{} compiler.
%
% %%%%%%%%%%%%%%%%%%%%%%%%%%%%%%%%%%%%%%
% \paragraph{Main File.}
%
% The main file is called |cdocsamp.tex|.
%
% Load the \textsf{childdoc} definitions and
% declare the filename for the main document:
%    \begin{macrocode}
\input{childdoc.def}
\childdocmain{}
%    \end{macrocode}

% Optional override for |\version| flag:
%    \begin{macrocode}
%%\ifchilddoc\else\providecommand{\version}{draft}\fi
%    \end{macrocode}

% Define the default values for the |\version| flag
% (|final| for the main file and |draft| for childs):
%    \begin{macrocode}
\ifchilddoc
\providecommand{\version}{draft}
\else
\providecommand{\version}{final}
\fi
%    \end{macrocode}

% Load the standard document class:
%    \begin{macrocode}
\documentclass[12pt]{article}
%    \end{macrocode}

% Start the document body:
%    \begin{macrocode}
\begin{document}
%    \end{macrocode}

% Declare a title page.
% Print title, part of document being processed and version flag:
%    \begin{macrocode}
\addtocounter{page}{-1}
\begin{center}
{\LARGE\bfseries{}childdoc example\par}
\vspace{1cm}
\ifchilddoc
\ifchilddocmanual part\else chapter\fi:
`\childdocname' of `\childdocjob'\par
\else
main document: `\childdocjob'\par
\fi
version: \version\par
\end{center}
\newpage
%    \end{macrocode}

% Manually include selected file,
% otherwise process as usual:
%    \begin{macrocode}
\ifchilddocmanual
\section*{part `\childdocname'}
\input{\childdocname}
\else
%    \end{macrocode}

% Include the two chapters:
%    \begin{macrocode}
\include{cdocsch1}
\include{cdocsch2}
%    \end{macrocode}

% Include the two parts unless only chapters should be displayed:
%    \begin{macrocode}
\ifchilddoc\else
\section{part three}
\input{cdocspt3}
\section{part four}
\input{cdocspt4}
\fi
%    \end{macrocode}

% Process as usual until here:
%    \begin{macrocode}
\fi
%    \end{macrocode}

% End of document body:
%    \begin{macrocode}
\end{document}
%    \end{macrocode}
%\iffalse
%</samplemain>
%\fi
%
% %%%%%%%%%%%%%%%%%%%%%%%%%%%%%%%%%%%%%%
% \paragraph{Chapter Include Files.}
%
% The include files are called |cdocsch1.tex| and |cdocsch2.tex|.
%
%\iffalse
%<*samplechap1|samplechap2>
%\fi

% Optional override for |\version| flag:
%    \begin{macrocode}
%%\providecommand{\version}{final}
%    \end{macrocode}

% Include the main document:
%    \begin{macrocode}
\input{childdoc.def}
\childdocof{cdocsamp}
%    \end{macrocode}

%\iffalse
%</samplechap1|samplechap2>
%\fi
%
%\iffalse
%<*samplechap1>
%\fi
% Some text for chapter 1:
%    \begin{macrocode}
\section{one}
some text in chapter one
%    \end{macrocode}

%\iffalse
%</samplechap1>
%\fi
% Some text for chapter 2:
%\iffalse
%<*samplechap2>
%\fi
%    \begin{macrocode}
\section{two}
more text in chapter two
%    \end{macrocode}

%\iffalse
%</samplechap2>
%\fi
%
% %%%%%%%%%%%%%%%%%%%%%%%%%%%%%%%%%%%%%%
% \paragraph{Part Include Files.}
%
% The include files are called |cdocspt3.tex| and |cdocspt4.tex|.
%
%\iffalse
%<*samplepart3|samplepart4>
%\fi

% Optional override for |\version| flag:
%    \begin{macrocode}
%%\providecommand{\version}{final}
%    \end{macrocode}

% Include the main document:
%    \begin{macrocode}
\input{childdoc.def}
\childdocby{cdocsamp}
%    \end{macrocode}

%\iffalse
%</samplepart3|samplepart4>
%\fi
%
%\iffalse
%<*samplepart3>
%\fi
% Some text for part 3:
%    \begin{macrocode}
some text in part three
%    \end{macrocode}

%\iffalse
%</samplepart3>
%\fi
% Some text for part 4:
%\iffalse
%<*samplepart4>
%\fi
%    \begin{macrocode}
more text in part four
%    \end{macrocode}

%\iffalse
%</samplepart4>
%\fi
%
% %%%%%%%%%%%%%%%%%%%%%%%%%%%%%%%%%%%%%%
% \paragraph{Forwarding for a Complete Draft.}
%
% The following forwarding file |cdocsdrf.tex|
% compiles the main document in draft mode:
%\iffalse
%<*sampledraft>
%\fi
%    \begin{macrocode}
\def\version{draft}
\input{childdoc.def}
\childdocforward{cdocsamp}
%    \end{macrocode}

%\iffalse
%</sampledraft>
%\fi
%
% %%%%%%%%%%%%%%%%%%%%%%%%%%%%%%%%%%%%%%
% \paragraph{Forwarding for Final Version of the Chapters.}
%
% The following forwarding files |cdocsfn1.tex| and |cdocsfn2.tex|
% (with identical content)
% compile the final versions of the child documents
% |cdocsch1.tex| and |cdocsch2.tex|, respectively:
%\iffalse
%<*samplefinal>
%\fi
%    \begin{macrocode}
\def\version{final}
\input{childdoc.def}
\childdocforwardprefix[cdocsamp]{cdocsfn}{cdocsch}
%    \end{macrocode}

%\iffalse
%</samplefinal>
%\fi
%
% %%%%%%%%%%%%%%%%%%%%%%%%%%%%%%%%%%%%%%
% \paragraph{Command Line Processing.}
%
% The following three command lines generate the output files
% |cdocscld|, |cdocscl1| and |cdocscl2|
% which should be identical to
% |cdocsdrf|, |cdocsch1| and |cdocsfn2|, respectively:
% \begin{center}
% \begin{tabular}{l}
% |latex -jobname cdocscld \|\\
% |  "\def\version{draft}\input{childdoc.def}\childdocforward{cdocsamp}"|\\
% |latex -jobname cdocscl1 \|\\
% |  "\input{childdoc.def}\childdocforward[cdocsamp]{cdocsch1}"|\\
% |latex -jobname cdocscl2 \|\\
% |  "\def\version{final}\input{childdoc.def}\childdocforward{cdocsch2}"|
% \end{tabular}
% \end{center}
% Note that the trailing backslash on each first line
% merely continues the input to the second line
% (for convenient cut ant paste).
% Furthermore, the command |latex| can be replaced by any
% of its alternative versions such as |pdflatex|.
%
% %%%%%%%%%%%%%%%%%%%%%%%%%%%%%%%%%%%%%%%%%%%%%%%%%%%%%%%%%%%%%%%%%%%%%%%%%%%%%%
% %%%%%%%%%%%%%%%%%%%%%%%%%%%%%%%%%%%%%%%%%%%%%%%%%%%%%%%%%%%%%%%%%%%%%%%%%%%%%%
% \section{Implementation}
%\iffalse
%<*package>
%\fi
%
% This section describes the definitions file |childdoc.def|.

% The definitions cannot be loaded using |\usepackage| or |\RequirePackage|
% which has a mechanism to prevent loading a style file more than once.
% When loading the definitions by means of |\input|
% multiple instances have to be prevented manually:
%\iffalse
%This code needs to be before the `\ProvidesFile' directive
%which is defined at the beginning of this file.
%Therefore it is also placed there and commented out here.
%</package>
%<*discard>
%\fi
%    \begin{macrocode}
\ifdefined\childdocmain\endinput\fi
%    \end{macrocode}
%\iffalse
%</discard>
%<*package>
%\fi
%
% \macro{\ifchilddoc}
% \macro{\ifchilddocmanual}
% The conditional |\ifchilddoc| tells whether a
% child (true) or main (false) document is being compiled.
% The conditional |\ifchilddocmanual| tells whether
% the |\includeonly| mechanism is used (false) or
% the selection of child files must be performed manually (true).
% The definitions initialise to false:
%    \begin{macrocode}
\newif\ifchilddoc
\newif\ifchilddocmanual
%    \end{macrocode}

% \macro{\childdocname}
% \macro{\childdocjob}
% The macro |\childdocname| stores the name of the main document
% to be compiled. The macro |\childdocjob| stores the name of
% the document on which the \LaTeX{} compiler was originally invoked.
% The content of |\jobname| cannot be compared
% to filenames specified in the source due to different catcodes.
% The following code rescans |\jobname|, stores the result
% in |\childdocname| and saves a copy in |\childdocjob|:
%    \begin{macrocode}
\edef\childdocname{\scantokens\expandafter{\jobname\noexpand}}
\let\childdocjob\childdocname
%    \end{macrocode}

% \macro{\childdocdisable}
% The macro |\childdocdisable| prevents the main file
% from being processed more than once.
% At this stage, the main document command |\childdocmain|
% is assumed to be called once again where it should do nothing.
% Any subsequent call to it should prevent
% a secondary processing of the main document
% It overwrites the forwarding commands
% |\childdocof| and |\childdocforward|
% with empty macros to prevent further inclusions of the main document:
%    \begin{macrocode}
\newcommand{\childdocdisable}
{
  \renewcommand{\childdocmain}[1]{\renewcommand{\childdocmain}[1]{\endinput}}
  \renewcommand{\childdocof}[1]{}
  \renewcommand{\childdocby}[2][]{}
  \renewcommand{\childdocforward}[2][]{}
  \renewcommand{\childdocdisable}{}
}
%    \end{macrocode}

% \macro{\childdocmain}
% The macro |\childdocmain| is to be called at the top of the main file
% with nothing or the main filename (without extension) as argument.
% First, it breaks loops.
% If the argument is not empty and does not match |\childdocname|
% (which is set by the first inclusion of |childdoc.def|),
% |\ifchilddoc| is set to true, |\includeonly| is applied to the child file
% and |\jobname| is set to the main file
% (for proper handling of |.aux| files):
%    \begin{macrocode}
\newcommand{\childdocmain}[1]
{
  \childdocdisable\childdocmain{}
  \if?#1?\else
    \begingroup
      \def\childdoctmp{#1}
      \ifx\childdoctmp\childdocname
        \def\childdoctmp{}
      \else
        \def\childdoctmp
        {
          \childdoctrue
          \includeonly{\childdocname}
          \def\childdocjob{#1}
          \def\jobname{#1}
        }
      \fi
      \expandafter
    \endgroup
    \childdoctmp
  \fi
}
%    \end{macrocode}

% \macro{\childdocof}
% The command |\childdocof| redirects
% compilation to the main file |#1|.
%    \begin{macrocode}
\newcommand{\childdocof}[1]
{
  \childdocdisable
  \childdoctrue
  \includeonly{\childdocname}
  \def\jobname{#1}
  \def\childdocjob{#1}
  \input{#1}
}
%    \end{macrocode}

% \macro{\childdocby}
% The command |\childdocby| ....
%    \begin{macrocode}
\newcommand{\childdocby}[2][]
{
  \childdocdisable
  \childdoctrue
  \childdocmanualtrue
  \if?#1?\else
    \def\jobname{#2}
  \fi
  \def\childdocjob{#2}
  \input{#2}
  \endinput
}
%    \end{macrocode}

% \macro{\childdocforward}
% The command |\childdocforward| redirects
% compilation to the main file or
% (if the optional argument is given) a child file.
% Parameters are set as if the main file
% or a child file starting with |\childdocof| was compiled.
% Then compilation is handed over to the main file:
%    \begin{macrocode}
\newcommand{\childdocforward}[2][]
{
  \begingroup
    \if?#1?
      \def\childdoctmp
      {
        \def\childdocname{#2}
        \def\childdocjob{#2}
        \def\jobname{#2}
        \input{#2}
        \endinput
      }
    \else
      \def\childdoctmp
      {
        \childdocdisable
        \def\childdocname{#2}
        \childdoctrue
        \includeonly{#2}
        \def\childdocjob{#1}
        \def\jobname{#1}
        \input{#1}
        \endinput
      }
    \fi
    \expandafter
  \endgroup
  \childdoctmp
}
%    \end{macrocode}

% \macro{\childdocforwardprefix}
% The command |\childdocforwardprefix| redirects
% compilation to the main or a child file by means of a pattern.
% The prefix |#1| in the current filename is replaced by |#2|
% and the suffix of the current filename is kept
% (it is assumed that the filename does not contain the substring `|~~~|'
% which is used as a delimiter).
% Compilation is handed over to the new file by |\childdocforward|:
%    \begin{macrocode}
\newcommand{\childdocforwardprefix}[3][]
{
  \begingroup
    \def\childdocextract #2##1~~~{\def\childdoctmp{\childdocforward[#1]{#3##1}}}
    \expandafter\childdocextract\childdocname~~~
    \expandafter
  \endgroup
  \childdoctmp
}
%    \end{macrocode}

% \macro{\childdoc}
% The deprecated macro |\childdoc| is a legacy version of |\childdocmain|:
%    \begin{macrocode}
\newcommand{\childdoc}{\childdocmain}
%    \end{macrocode}

% \macro{\childdocredirect}
% The deprecated macro |\childdocredirect| is a legacy version
% of |\childdocforward| and |\childdocforwardprefix|:
%    \begin{macrocode}
\newcommand{\childdocredirect}[2][]
{
  \begingroup
    \if?#1?
      \def\childdoctmp{\childdocforward{#2}}
    \else
      \def\childdoctmp{\childdocforwardprefix{#1}{#2}}
    \fi
    \expandafter
  \endgroup
  \childdoctmp
}
%    \end{macrocode}

%\iffalse
%</package>
%\fi
%
\endinput
|\\
|\childdocof{|\textit{main}|}|\\
\end{tabular}
\end{center}
at the top of every child file \textit{child}
which is included by |\include{|\textit{child}|}|
from within the main file
(or at least for those files to be compiled individually).
The argument \textit{main} must be the filename of the main file.

There are a couple of
considerations in setting up the main and child documents:

%%%%%%%%%%%%%%%%%%%%%%%%%%%%%%%%%%%%%%%%
\paragraph{Restrictions.}

Please note the following restrictions:
\begin{itemize}
\item
|\childdocmain| must be called with one argument \textit{main}
to ensure compatibility with earlier version of the package.
It must either be empty (|\childdocmain{}|)
or precisely match the filename of the main file in which it is specified.
See \secref{sec:detection} for further information.
\item
The filename \textit{main} must be specified without the |.tex| extension.
\item
The filename \textit{main} is case sensitive
(even in case-insensitive file systems)
due to internal string comparison.
\item
The argument \textit{main} should be fully expanded, it cannot be a macro.
\item
Subdirectories and special characters should be avoided in filenames.
\item
The command |\childdocmain{|\textit{main}|}| must be followed by a whitespace.
It should not be followed immediately by another command
or by a comment mark `|%|'.
This is because the \TeX{} parser reads the token immediately following
the argument of |\childdocmain| and puts it
at the beginning of every child section;
however, a white\-space is ignored.
\end{itemize}

%%%%%%%%%%%%%%%%%%%%%%%%%%%%%%%%%%%%%%%%
\paragraph{Content of Main File.}

It is advisable to place all content in the child files included by |\include|.
Any output contained in the main file will appear in all child documents
unless suppressed manually;
it cannot be suppressed automatically by the |\includeonly| directive
and thus should normally be avoided.
A method to include some content in the main file
by means of conditional processing is described in \secref{sec:conditional}.

%%%%%%%%%%%%%%%%%%%%%%%%%%%%%%%%%%%%%%%%
\paragraph{Page Numbering.}

When only a part of the document is compiled,
the appropriate numbering of pages
(as well as other status parameters)
is determined from the |.aux| files.
The latter contain information from previous passes.
However this information needs to propagate through
all intermediate child documents.
Therefore the page numbering in child documents may well
be inconsistent until the complete document is compiled at least once.

A useful (if unconventional) way to always ensure a consistent
page numbering is to restart the numbering in each child document
and denote the pages by `\textit{child}|.|\textit{page}'
where \textit{child} represents the chapter/section number of the child file.
This can be achieved by the command
|\numberwithin{page}{|\textit{child}|}|
of the \textsf{amsmath} package
where \textit{child} can be |chapter| or |section|
depending on the chosen structuring.
Alternatively, one can modify the macro |\thepage| appropriately
and reset the counter |page| at the start of each child file.

%%%%%%%%%%%%%%%%%%%%%%%%%%%%%%%%%%%%%%%%%%%%%%%%%%%%%%%%%%%%%%%%%%%%%%%%%%%%%%%%
\subsection{Conditional Processing}
\label{sec:conditional}

The package provides a mechanism to compile different versions
of a document. To customise the versions further some conditional processing
can come in handy to distinguish which version is being compiled.
The package provides two macros to describe the compilation context:

%%%%%%%%%%%%%%%%%%%%%%%%%%%%%%%%%%%%%%%%
\DescribeMacro{\ifchilddoc}
The conditional |\ifchilddoc| distinguishes between the compilation of
child documents and the main document:
%
\begin{center}
|\ifchilddoc |\textit{child-code}| |[|\||else |\textit{main-code}]| \||fi|
\end{center}

%%%%%%%%%%%%%%%%%%%%%%%%%%%%%%%%%%%%%%%%
\DescribeMacro{\childdocname}
\DescribeMacro{\childdocjob}
The macro |\childdocname| contains the filename (without extension)
of the main or child file being processed.
Note that |\childdocjob| will always contain the name of the main file.

%%%%%%%%%%%%%%%%%%%%%%%%%%%%%%%%%%%%%%%%
\paragraph{Title Page.}

Conditional processing can be used to include a title or banner page
in the main document when proper precautions are taken.
Importantly, the code in the main file should ensure that the page counter
(as well as other status parameters which are stored in the |.aux| files)
takes the same value after the conditional processing.
Otherwise the page numbers may take divergent values
depending on which part is compiled.

For example, a title page could be declared by:
%
\begin{center}
\begin{tabular}{l}
|\ifchilddoc\||else|\\
|\addtocounter{page}{-1}|\\
\textit{code for title page}\\
|\newpage|\\
|\||fi|
\end{tabular}
\end{center}
%
A banner page for the child documents can be generated by:
%
\begin{center}
\begin{tabular}{l}
|\ifchilddoc|\\
|\addtocounter{page}{-1}|\\
\textit{code for banner page}\\
|\newpage|\\
|\||fi|
\end{tabular}
\end{center}
%
Here one could write a message such as:
\begin{center}
|This is the part \childdocname{} of \childdocjob{}.|
\end{center}

%%%%%%%%%%%%%%%%%%%%%%%%%%%%%%%%%%%%%%%%%%%%%%%%%%%%%%%%%%%%%%%%%%%%%%%%%%%%%%%%
\subsection{Flags}
\label{sec:flags}

The package makes it easy to generate different versions
of the main or child documents.
To this end compilation flags can be defined
and assigned different default values.
They will be particularly useful in conjunction
with the forwarding mechanism described in \secref{sec:forward}.

For example, it may be useful to have a flag |\version|
which can be set to |draft| or |final|.
The document source will contain some conditional code
depending on the value of |\version|.
Suppose further, the flag should default to |final| for the main file
and to |draft| for child files
which is a natural assignment for editing the document.
This is achieved by placing the following code
in the preamble of the main document
(below the |\childdocmain| directive):
%
\begin{center}
\begin{tabular}{l}
|\ifchilddoc|\\
|\providecommand{\version}{draft}|\\
|\||else|\\
|\providecommand{\version}{final}|\\
|\||fi|
\end{tabular}
\end{center}
%
The definition by |\providecommand| makes sure
that previous definitions are not overwritten.
Further statements |\providecommand{\version}{...}|
can thus be added before the above code to override it.

For the main file, one might add a line
(between |\childdocmain| and the above block)
%
\begin{center}
|%\ifchilddoc\||else\providecommand{\version}{draft}\||fi|
\end{center}
%
which can be uncommented to produce a draft version.
Likewise one can add a line to the very top of a child file
(above the |\childdocof{|\textit{main}|}| directive)
%
\begin{center}
|%\providecommand{\version}{final}|
\end{center}
%
which can be uncommented to produce the final version of this child document.

%%%%%%%%%%%%%%%%%%%%%%%%%%%%%%%%%%%%%%%%%%%%%%%%%%%%%%%%%%%%%%%%%%%%%%%%%%%%%%%%
\subsection{Forwarding}
\label{sec:forward}

Different versions of the main or child documents
using compilation flags as described in \secref{sec:flags}
can be (permanently) stored in different files
for convenient compilation, viewing and distribution.
To this end, the package defines a command
to pass on compilation to a different file:

%%%%%%%%%%%%%%%%%%%%%%%%%%%%%%%%%%%%%%%%
\DescribeMacro{\childdocforward}
The command |\childdocforward| redirects processing to
another source file:
%
\begin{center}
\begin{tabular}{l}
|% \iffalse
%
% childdoc.dtx Copyright (C) 2017-2018 Niklas Beisert
%
% This work may be distributed and/or modified under the
% conditions of the LaTeX Project Public License, either version 1.3
% of this license or (at your option) any later version.
% The latest version of this license is in
%   http://www.latex-project.org/lppl.txt
% and version 1.3 or later is part of all distributions of LaTeX
% version 2005/12/01 or later.
%
% This work has the LPPL maintenance status `maintained'.
%
% The Current Maintainer of this work is Niklas Beisert.
%
% This work consists of the files childdoc.dtx and childdoc.ins
% and the derived files childdoc.def and cdocsamp.tex with
% cdocsch1.tex, cdocsch2.tex, cdocsdrf.tex, cdocsfn1.tex, cdocsfn2.tex.
%
%<package>\ifdefined\childdocmain\endinput\fi
%<package>\ProvidesFile{childdoc.def}[2018/12/30 v2.0 child document driver]
%<samplemain>\ProvidesFile{cdocsamp.tex}[2018/12/30 v2.0 sample for childdoc]
%<*driver>
%\ProvidesFile{childdoc.drv}[2018/12/30 v2.0 childdoc reference manual file]
\PassOptionsToClass{10pt,a4paper}{article}
\documentclass{ltxdoc}

\usepackage[margin=35mm]{geometry}
\usepackage{hyperref}
\usepackage{hyperxmp}
\usepackage[usenames]{color}

\hypersetup{colorlinks=true}
\hypersetup{pdfstartview=FitH}
\hypersetup{pdfpagemode=UseNone}
\hypersetup{pdfsource={}}
\hypersetup{pdflang={en-UK}}
\hypersetup{pdfcopyright={Copyright 2017-2018 Niklas Beisert.
  This work may be distributed and/or modified under the
  conditions of the LaTeX Project Public License, either version 1.3
  of this license or (at your option) any later version.}}
\hypersetup{pdflicenseurl={http://www.latex-project.org/lppl.txt}}
\hypersetup{pdfcontactaddress={ETH Zurich, ITP, HIT K,
  Wolfgang-Pauli-Strasse 27}}
\hypersetup{pdfcontactpostcode={8093}}
\hypersetup{pdfcontactcity={Zurich}}
\hypersetup{pdfcontactcountry={Switzerland}}
\hypersetup{pdfcontactemail={nbeisert@itp.phys.ethz.ch}}
\hypersetup{pdfcontacturl={http://people.phys.ethz.ch/\xmptilde nbeisert/}}

\newcommand{\secref}[1]{\hyperref[#1]{section \ref*{#1}}}

\parskip1ex
\parindent0pt
\let\olditemize\itemize
\def\itemize{\olditemize\parskip0pt}

\begin{document}

\title{The \textsf{childdoc} Package}
\hypersetup{pdftitle={The childdoc Package}}
\author{Niklas Beisert\\[2ex]
  Institut f\"ur Theoretische Physik\\
  Eidgen\"ossische Technische Hochschule Z\"urich\\
  Wolfgang-Pauli-Strasse 27, 8093 Z\"urich, Switzerland\\[1ex]
  \href{mailto:nbeisert@itp.phys.ethz.ch}
  {\texttt{nbeisert@itp.phys.ethz.ch}}}
\hypersetup{pdfauthor={Niklas Beisert}}
\hypersetup{pdfsubject={Manual for the LaTeX2e Package childdoc}}
\date{30 December 2018, \textsf{v2.0}}
\maketitle

\begin{abstract}\noindent
\textsf{childdoc} is a \LaTeXe{} package
that enables the direct compilation
of document sections included by |\include|
to individual files.
\end{abstract}

\begingroup
\parskip0ex
\tableofcontents
\endgroup

%%%%%%%%%%%%%%%%%%%%%%%%%%%%%%%%%%%%%%%%%%%%%%%%%%%%%%%%%%%%%%%%%%%%%%%%%%%%%%%%
%%%%%%%%%%%%%%%%%%%%%%%%%%%%%%%%%%%%%%%%%%%%%%%%%%%%%%%%%%%%%%%%%%%%%%%%%%%%%%%%
\section{Introduction}

\LaTeX{} provides a mechanism to structure a large document (such as a book)
into a main file and several child files (containing the chapters)
using the |\include| command.
This mechanism is beneficial for documents
which span hundreds of pages in order to
make the source file(s) more manageable.
Moreover, compilation can be restricted to
selected child files by means of the |\includeonly| command.
The latter feature can be used to reduce the compilation time while editing
(this was significantly more useful in the earlier days of \LaTeX{})
or to generate a smaller document which is easier to navigate.
Another application of |\includeonly| is to generate
documents consisting of selected parts of the complete document.

However, there are a few drawbacks of the plain |\include| mechanism:
\begin{itemize}
\item
The child files cannot be compiled on their own,
they can only be compiled via the main file.
A naive editing environment
(such as a text editor with an option
to have the current file processed by \LaTeX)
may require one to switch to the main file before compiling;
attempting to compile the child file produces errors.
\item
The main file must be modified (each time)
to adjust the |\includeonly| command
to the present needs. This easily leaves the main file in a messy state.
\item
The generated document will always carry the filename
of the main document. This is inconvenient if
several child files are to be compiled and
to be kept for distribution.
\end{itemize}

The present package provides a simple interface
to make child files individually compilable by \LaTeX{}.
Compiling a child file then has the same effect as compiling
the main file with an |\includeonly| command
to select the appropriate child.
Moreover the generated document will carry the name of the child
rather than the main file.
This resolves all three above issues.

This feature is meant to make the editing of books,
thesis documents and lecture notes somewhat more convenient.
However, the package can also be used efficiently for
composing a series of documents (such as exercise sheets)
which are typically distributed individually.
It then assists the author in generating the individual documents
(potentially in different versions)
as well as a document containing the collected series.
Another application is in developing style files
or other kinds of included material
where compilation of the style file could redirect
to a sample or test file.

%%%%%%%%%%%%%%%%%%%%%%%%%%%%%%%%%%%%%%%%%%%%%%%%%%%%%%%%%%%%%%%%%%%%%%%%%%%%%%%%
%%%%%%%%%%%%%%%%%%%%%%%%%%%%%%%%%%%%%%%%%%%%%%%%%%%%%%%%%%%%%%%%%%%%%%%%%%%%%%%%
\section{Usage}

First of all, the package \textsf{childdoc} is \emph{not} a standard
\LaTeXe{} |.sty| style file! Therefore it needs to be invoked in
a non-standard way.

%%%%%%%%%%%%%%%%%%%%%%%%%%%%%%%%%%%%%%%%%%%%%%%%%%%%%%%%%%%%%%%%%%%%%%%%%%%%%%%%
\subsection{Included Files}
\label{sec:include}

%%%%%%%%%%%%%%%%%%%%%%%%%%%%%%%%%%%%%%%%
\DescribeMacro{\childdocmain}
To use the package, add the commands
\begin{center}
\begin{tabular}{l}
|\input{childdoc.def}|\\
|\childdocmain{}|\\
\end{tabular}
\end{center}
at the very top of the main \LaTeX{} file,
in particular \emph{before} the |\documentclass| statement!
The argument of |\childdocmain| should be left empty
(but it must be present).

%%%%%%%%%%%%%%%%%%%%%%%%%%%%%%%%%%%%%%%%
\DescribeMacro{\childdocof}
Furthermore, add the commands
\begin{center}
\begin{tabular}{l}
|\input{childdoc.def}|\\
|\childdocof{|\textit{main}|}|\\
\end{tabular}
\end{center}
at the top of every child file \textit{child}
which is included by |\include{|\textit{child}|}|
from within the main file
(or at least for those files to be compiled individually).
The argument \textit{main} must be the filename of the main file.

There are a couple of
considerations in setting up the main and child documents:

%%%%%%%%%%%%%%%%%%%%%%%%%%%%%%%%%%%%%%%%
\paragraph{Restrictions.}

Please note the following restrictions:
\begin{itemize}
\item
|\childdocmain| must be called with one argument \textit{main}
to ensure compatibility with earlier version of the package.
It must either be empty (|\childdocmain{}|)
or precisely match the filename of the main file in which it is specified.
See \secref{sec:detection} for further information.
\item
The filename \textit{main} must be specified without the |.tex| extension.
\item
The filename \textit{main} is case sensitive
(even in case-insensitive file systems)
due to internal string comparison.
\item
The argument \textit{main} should be fully expanded, it cannot be a macro.
\item
Subdirectories and special characters should be avoided in filenames.
\item
The command |\childdocmain{|\textit{main}|}| must be followed by a whitespace.
It should not be followed immediately by another command
or by a comment mark `|%|'.
This is because the \TeX{} parser reads the token immediately following
the argument of |\childdocmain| and puts it
at the beginning of every child section;
however, a white\-space is ignored.
\end{itemize}

%%%%%%%%%%%%%%%%%%%%%%%%%%%%%%%%%%%%%%%%
\paragraph{Content of Main File.}

It is advisable to place all content in the child files included by |\include|.
Any output contained in the main file will appear in all child documents
unless suppressed manually;
it cannot be suppressed automatically by the |\includeonly| directive
and thus should normally be avoided.
A method to include some content in the main file
by means of conditional processing is described in \secref{sec:conditional}.

%%%%%%%%%%%%%%%%%%%%%%%%%%%%%%%%%%%%%%%%
\paragraph{Page Numbering.}

When only a part of the document is compiled,
the appropriate numbering of pages
(as well as other status parameters)
is determined from the |.aux| files.
The latter contain information from previous passes.
However this information needs to propagate through
all intermediate child documents.
Therefore the page numbering in child documents may well
be inconsistent until the complete document is compiled at least once.

A useful (if unconventional) way to always ensure a consistent
page numbering is to restart the numbering in each child document
and denote the pages by `\textit{child}|.|\textit{page}'
where \textit{child} represents the chapter/section number of the child file.
This can be achieved by the command
|\numberwithin{page}{|\textit{child}|}|
of the \textsf{amsmath} package
where \textit{child} can be |chapter| or |section|
depending on the chosen structuring.
Alternatively, one can modify the macro |\thepage| appropriately
and reset the counter |page| at the start of each child file.

%%%%%%%%%%%%%%%%%%%%%%%%%%%%%%%%%%%%%%%%%%%%%%%%%%%%%%%%%%%%%%%%%%%%%%%%%%%%%%%%
\subsection{Conditional Processing}
\label{sec:conditional}

The package provides a mechanism to compile different versions
of a document. To customise the versions further some conditional processing
can come in handy to distinguish which version is being compiled.
The package provides two macros to describe the compilation context:

%%%%%%%%%%%%%%%%%%%%%%%%%%%%%%%%%%%%%%%%
\DescribeMacro{\ifchilddoc}
The conditional |\ifchilddoc| distinguishes between the compilation of
child documents and the main document:
%
\begin{center}
|\ifchilddoc |\textit{child-code}| |[|\||else |\textit{main-code}]| \||fi|
\end{center}

%%%%%%%%%%%%%%%%%%%%%%%%%%%%%%%%%%%%%%%%
\DescribeMacro{\childdocname}
\DescribeMacro{\childdocjob}
The macro |\childdocname| contains the filename (without extension)
of the main or child file being processed.
Note that |\childdocjob| will always contain the name of the main file.

%%%%%%%%%%%%%%%%%%%%%%%%%%%%%%%%%%%%%%%%
\paragraph{Title Page.}

Conditional processing can be used to include a title or banner page
in the main document when proper precautions are taken.
Importantly, the code in the main file should ensure that the page counter
(as well as other status parameters which are stored in the |.aux| files)
takes the same value after the conditional processing.
Otherwise the page numbers may take divergent values
depending on which part is compiled.

For example, a title page could be declared by:
%
\begin{center}
\begin{tabular}{l}
|\ifchilddoc\||else|\\
|\addtocounter{page}{-1}|\\
\textit{code for title page}\\
|\newpage|\\
|\||fi|
\end{tabular}
\end{center}
%
A banner page for the child documents can be generated by:
%
\begin{center}
\begin{tabular}{l}
|\ifchilddoc|\\
|\addtocounter{page}{-1}|\\
\textit{code for banner page}\\
|\newpage|\\
|\||fi|
\end{tabular}
\end{center}
%
Here one could write a message such as:
\begin{center}
|This is the part \childdocname{} of \childdocjob{}.|
\end{center}

%%%%%%%%%%%%%%%%%%%%%%%%%%%%%%%%%%%%%%%%%%%%%%%%%%%%%%%%%%%%%%%%%%%%%%%%%%%%%%%%
\subsection{Flags}
\label{sec:flags}

The package makes it easy to generate different versions
of the main or child documents.
To this end compilation flags can be defined
and assigned different default values.
They will be particularly useful in conjunction
with the forwarding mechanism described in \secref{sec:forward}.

For example, it may be useful to have a flag |\version|
which can be set to |draft| or |final|.
The document source will contain some conditional code
depending on the value of |\version|.
Suppose further, the flag should default to |final| for the main file
and to |draft| for child files
which is a natural assignment for editing the document.
This is achieved by placing the following code
in the preamble of the main document
(below the |\childdocmain| directive):
%
\begin{center}
\begin{tabular}{l}
|\ifchilddoc|\\
|\providecommand{\version}{draft}|\\
|\||else|\\
|\providecommand{\version}{final}|\\
|\||fi|
\end{tabular}
\end{center}
%
The definition by |\providecommand| makes sure
that previous definitions are not overwritten.
Further statements |\providecommand{\version}{...}|
can thus be added before the above code to override it.

For the main file, one might add a line
(between |\childdocmain| and the above block)
%
\begin{center}
|%\ifchilddoc\||else\providecommand{\version}{draft}\||fi|
\end{center}
%
which can be uncommented to produce a draft version.
Likewise one can add a line to the very top of a child file
(above the |\childdocof{|\textit{main}|}| directive)
%
\begin{center}
|%\providecommand{\version}{final}|
\end{center}
%
which can be uncommented to produce the final version of this child document.

%%%%%%%%%%%%%%%%%%%%%%%%%%%%%%%%%%%%%%%%%%%%%%%%%%%%%%%%%%%%%%%%%%%%%%%%%%%%%%%%
\subsection{Forwarding}
\label{sec:forward}

Different versions of the main or child documents
using compilation flags as described in \secref{sec:flags}
can be (permanently) stored in different files
for convenient compilation, viewing and distribution.
To this end, the package defines a command
to pass on compilation to a different file:

%%%%%%%%%%%%%%%%%%%%%%%%%%%%%%%%%%%%%%%%
\DescribeMacro{\childdocforward}
The command |\childdocforward| redirects processing to
another source file:
%
\begin{center}
\begin{tabular}{l}
|\input{childdoc.def}|\\
|\childdocforward[|\textit{main}|]{|\textit{dest}|}|\\
\end{tabular}
\end{center}
%
The argument \textit{dest} is the destination file
(without extension).
It should be the main file or one of the child files.
Note that further \textsf{childdoc} directives
such as |\childdocof| and |\childdocforward|
in the indicated file will be processed in this form.
The optional argument \textit{main}
passes on directly to the main file \textit{main}
while pretending to compile the child \textit{dest}.
This form behaves as if \textit{dest}
issues |\childdocof{|\textit{main}|}| right away,
and no further \textsf{childdoc} directives will be processed.

%%%%%%%%%%%%%%%%%%%%%%%%%%%%%%%%%%%%%%%%
\DescribeMacro{\...prefix}
In the alternative form |\childdocforwardprefix|,
%
\begin{center}
\begin{tabular}{l}
|\input{childdoc.def}|\\
|\childdocforwardprefix[|\textit{main}|]{|\textit{prefix}|}{|\textit{dest}|}|
\end{tabular}
\end{center}
%
the destination file is determined by a pattern
depending on the current file:
To make this work, the current file must be called
`{\textit{prefix}\hspace{0.2em}\textit{suffix}}'
with \textit{prefix} matching precisely the argument.
Processing is then passed on to the file
`{\textit{dest}\hspace{0.2em}\textit{suffix}}'.
Surely, the same effect is achieved by
directly specifying the
argument `{\textit{dest}\hspace{0.2em}\textit{suffix}}'
in the first form.
However, that requires to set up a different file
for each child. With the alternative form of the command
all these files can have exactly the same content
which simplifies setting them up and maintaining them.

For example, the following file |draft.tex|
with a compilation flag |\version| as described in \secref{sec:flags}
compiles the main document as a draft:
%
\begin{center}
\begin{tabular}{l}
|\def\version{draft}|\\
|\input{childdoc.def}|\\
|\childdocforward{|\textit{main}|}|
\end{tabular}
\end{center}
%
Likewise, the following files |final|\textit{nn}|.tex|
compile the final version of the child document
|child|\textit{nn}|.tex|:
%
\begin{center}
\begin{tabular}{l}
|\def\version{final}|\\
|\input{childdoc.def}|\\
|\childdocforwardprefix{final}{child}|
\end{tabular}
\end{center}
%

Note that when several versions of a main file and/or of each child file
are to be generated, it may be convenient to set up a |Makefile| or
shell script to automatise the process.

%%%%%%%%%%%%%%%%%%%%%%%%%%%%%%%%%%%%%%%%%%%%%%%%%%%%%%%%%%%%%%%%%%%%%%%%%%%%%%%%
\subsection{Command Line Processing}
\label{sec:commandline}

The effect of redirection files can also be achieved by invoking
the \LaTeX{} compiler with a more elaborate command line.
Most conveniently this should be done as part
of a shell script or a |Makefile|.

When using \textsf{childdoc} in the main file, the following
command lines effectively perform a redirection
(note that depending on the shell being used,
backslashes may have to be doubled: `|\|' $\to$ `|\\|'):
%
\begin{center}
|... -jobname "|\textit{target}|" |\\|"|[\textit{flags}]%
|\input{childdoc.def}\childdocforward[|\textit{main}|]{|\textit{dest}|}"|
\end{center}
%
Here \textit{target} is the name of the output file,
\textit{main} is the name of the main file
and \textit{dest} is the name of the main or child file to be processed
(all filenames without extensions).
The optional argument \textit{main} can be omitted
if \textit{main} matches \textit{dest}.
Optionally, compilation \textit{flags} can be defined via |\def| commands.
This command line makes the \TeX{} engine believe
it is compiling the file \textit{target}
whose content is specified as the latter parameter.
The provided code then forwards the processing to
\textit{main} or \textit{dest} as described in \secref{sec:forward}.

%%%%%%%%%%%%%%%%%%%%%%%%%%%%%%%%%%%%%%%%%%%%%%%%%%%%%%%%%%%%%%%%%%%%%%%%%%%%%%%%
\subsection{Include by Input}
\label{sec:input}

Including child documents by |\include| has some restrictions by design.
Most notably, the content of a child document always occupies
its own set of pages; pages cannot be shared between child documents.
Usually, this behaviour makes perfect sense
because each child document contain an essential part of the document.
However, in some situations it may be desirable to compose
a document from a collection of parts
without having mandatory page breaks between then.
For this case, the package
provides a mechanism to include parts
by |\input| which can also be processed individually.
However, by construction this mechanism
requires manual handling of the content to be output.

%%%%%%%%%%%%%%%%%%%%%%%%%%%%%%%%%%%%%%%%
\DescribeMacro{\ifchilddocmanual}
The main file should be prepared as usual, see \secref{sec:include}.
However, the document body must make a distinction
between processing of an individual part and of the main document, e.g.:
%
\begin{center}
\begin{tabular}{l}
|\ifchilddocmanual|\\
|\input{\childdocname}|\\
|\||else|\\
\textit{document body with }|\input{|\textit{part}|}|\\
|\||fi|
\end{tabular}
\end{center}
%
The conditional |\ifchilddocmanual| is true whenever
a part to be included by |\input| is being compiled,
and the name of the part is stored in |\childdocname|.

%%%%%%%%%%%%%%%%%%%%%%%%%%%%%%%%%%%%%%%%
\DescribeMacro{\childdocby}
Each part to be included by |\input| should start with:
%
\begin{center}
\begin{tabular}{l}
|\input{childdoc.def}|\\
|\childdocby{|\textit{main}|}|\\
\end{tabular}
\end{center}
%
The directive |\childdocby| is similar to |\childdocof|
described in \secref{sec:include},
but the subsequent selection of content must be done manually.
To that end, both |\ifchilddoc| and |\ifchilddocmanual|
will be true upon processing of a part,
and the name of the part is stored in |\childdocname|.
Note that |\jobname| will be set to the filename of the current part
so that each part receives an individual |.aux| file
that does not interfere with the |.aux| file(s) of the main document.
This behaviour can be altered by the alternative form
|\childdocby[*]{|\textit{main}|}| (with a non-empty optional argument)
which uses the |.aux| file of the main document
by setting |\jobname| to \textit{main}.

%%%%%%%%%%%%%%%%%%%%%%%%%%%%%%%%%%%%%%%%%%%%%%%%%%%%%%%%%%%%%%%%%%%%%%%%%%%%%%%%
\subsection{Driver Development}
\label{sec:driver}

The \textsf{childdoc} mechanism can also be use for the development
of definition files such as \LaTeX{} styles or classes.
This case differs from the above setup with multiple parts
included by |\include| in that no |\includeonly| should be invoked.
This can be achieved by starting the include file
(before |\ProvidesPackage|) with:
%
\begin{center}
\begin{tabular}{l}
|\input{childdoc.def}|\\
|\childdocforward{|\textit{main}|}|\\
\end{tabular}
\end{center}
%
or alternatively with:
%
\begin{center}
\begin{tabular}{l}
|\input{childdoc.def}|\\
|\childdocby{|\textit{main}|}|\\
\end{tabular}
\end{center}
%
Both forms have slightly different effects as described above.
The main file is prepared as usual, see \secref{sec:include}.

%%%%%%%%%%%%%%%%%%%%%%%%%%%%%%%%%%%%%%%%%%%%%%%%%%%%%%%%%%%%%%%%%%%%%%%%%%%%%%%%
\subsection{Legacy Detection}
\label{sec:detection}

The directive |\childdocmain| in the main file can detect
whether the complete document or merely a child is to be compiled
even without using the directive |\childdocof|.
This method is deprecated because it is less robust
and there is no compelling reason to use it;
it is merely provided for backward compatibility
and it may be removed in future versions.

If the detection mechanism is to be used,
it is mandatory to correctly specify
the filename of the main file as the argument of |\childdocmain|:
%
\begin{center}
\begin{tabular}{l}
|\input{childdoc.def}|\\
|\childdocmain{|\textit{main}|}|\\
\end{tabular}
\end{center}
%
If |\jobname| does not match the argument \textit{main} of |\childdocmain|,
it is assumed that |\jobname| points to the child file to be compiled.
When using |\childdocmain| with the main file specified as argument,
it suffices to start a child file
with just |\input{|\textit{main}|}|
without loading of the package and using |\childdocof|.
If instead all processing is done
with the appropriate \textsf{childdoc} directives,
the argument of \textit{main} of |\childdocmain| can be empty.

An alternative version of the command line processing described
in \secref{sec:commandline} using the detection mechanism reads:
%
\begin{center}
|... -jobname "|\textit{target}|" "|[\textit{flags}]%
[|\def\jobname{|\textit{dest}|}|]|\input{|\textit{main}|}"|
\end{center}

%%%%%%%%%%%%%%%%%%%%%%%%%%%%%%%%%%%%%%%%%%%%%%%%%%%%%%%%%%%%%%%%%%%%%%%%%%%%%%%%
\subsection{Manual Code}
\label{sec:manual}

In case one cannot be certain whether the definitions file |childdoc.def|
is installed on the target \TeX{} distribution
and one prefers not to ship it,
it is conceivable to paste a few relevant commands into the sources.

To that end, drop all statements |\input{childdoc.def}|
and perform the replacements as outlined below.
Instead of |\childdocmain{|\textit{main}|}| add the following code
to the top of the main file:
%
\begin{center}
\begin{tabular}{l}
|\||ifdefined\childdocname\endinput\||fi\newif\ifchilddoc|\\
|\edef\childdocname{\scantokens\expandafter{\jobname\noexpand}}|\\
|\def\childdocmain{|\textit{main}|}\||ifx\childdocmain\childdocname\||else|\\
|\childdoctrue\includeonly{\childdocname}\let\jobname\childdocmain\||fi|\\
\end{tabular}
\end{center}
%
Instead of |\childdocof{|\textit{main}|}| just include the main file
at the top of each child file:
%
\begin{center}
|\input{|\textit{main}|}|
\end{center}
%
A simple redirection |\childdocforward{|\textit{dest}|}| is achieved by:
%
\begin{center}
|\def\jobname{|\textit{dest}|}\input{\jobname}|
\end{center}
%
The redirection with prefix
|\childdocforwardprefix[|\textit{prefix}|]{|\textit{dest}|}|
is accomplished by:
%
\begin{center}
\begin{tabular}{l}
|{\edef\jobname{\scantokens\expandafter{\jobname\noexpand}}|\\
|\def\redirectjob |\textit{prefix}|#1~~~{\gdef\jobname{|\textit{dest}|#1}}|\\
|\expandafter\redirectjob\jobname~~~}\input{\jobname}|
\end{tabular}
\end{center}

In an alternative approach,
child documents can be compiled by a specific command line
without additional code or specific definitions:
%
\begin{center}
|... -jobname "|\textit{target}|" "|[\textit{flags}]%
|\includeonly{|\textit{dest}|}\input{|\textit{main}|}"|
\end{center}
%

%%%%%%%%%%%%%%%%%%%%%%%%%%%%%%%%%%%%%%%%%%%%%%%%%%%%%%%%%%%%%%%%%%%%%%%%%%%%%%%%
%%%%%%%%%%%%%%%%%%%%%%%%%%%%%%%%%%%%%%%%%%%%%%%%%%%%%%%%%%%%%%%%%%%%%%%%%%%%%%%%
\section{Information}

%%%%%%%%%%%%%%%%%%%%%%%%%%%%%%%%%%%%%%%%%%%%%%%%%%%%%%%%%%%%%%%%%%%%%%%%%%%%%%%%
\subsection{Copyright}

Copyright \copyright{} 2017--2018 Niklas Beisert

This work may be distributed and/or modified under the
conditions of the \LaTeX{} Project Public License, either version 1.3
of this license or (at your option) any later version.
The latest version of this license is in
  \url{http://www.latex-project.org/lppl.txt}
and version 1.3 or later is part of all distributions of \LaTeX{}
version 2005/12/01 or later.

This work has the LPPL maintenance status `maintained'.

The Current Maintainer of this work is Niklas Beisert.

This work consists of the files |README.txt|, |childdoc.ins| and |childdoc.dtx|
as well as the derived files |childdoc.def|, |cdocsamp.tex|
with |cdocsch1.tex|, |cdocsch2.tex|, |cdocspt3.tex|, |cdocspt4.tex|,
|cdocsdrf.tex|, |cdocsfn1.tex|, |cdocsfn2.tex|
as well as |childdoc.pdf|.

%%%%%%%%%%%%%%%%%%%%%%%%%%%%%%%%%%%%%%%%%%%%%%%%%%%%%%%%%%%%%%%%%%%%%%%%%%%%%%%%
\subsection{Files and Installation}

The package consists of the files:
%
\begin{center}
\begin{tabular}{ll}
    |README.txt|   & readme file \\
    |childdoc.ins| & installation file \\
    |childdoc.dtx| & source file \\
    |childdoc.def| & definition file \\
    |cdocsamp.tex| & sample main file \\
    |cdocsch1.tex| & sample include file \\
    |cdocsch2.tex| & sample include file \\
    |cdocspt3.tex| & sample part file \\
    |cdocspt4.tex| & sample part file \\
    |cdocsdrf.tex| & sample redirection file \\
    |cdocsfn1.tex| & sample redirection file \\
    |cdocsfn2.tex| & sample redirection file \\
    |childdoc.pdf| & manual
\end{tabular}
\end{center}
%
The distribution consists of the files
|README.txt|, |childdoc.ins| and |childdoc.dtx|.
%
\begin{itemize}
\item
Run (pdf)\LaTeX{} on |childdoc.dtx|
to compile the manual |childdoc.pdf| (this file).
\item
Run \LaTeX{} on |childdoc.ins| to create the definitions file |childdoc.def|
and the sample |cdocsamp.tex| with include files
|cdocsch1.tex|, |cdocsch2.tex|, |cdocspt3.tex|, |cdocspt4.tex|,
|cdocsdrf.tex|, |cdocsfn1.tex|, |cdocsfn2.tex|.
Then copy the file |childdoc.def| to an appropriate directory of your \LaTeX{}
distribution, e.g.\ \textit{texmf-root}|/tex/latex/childdoc|.
\end{itemize}

%%%%%%%%%%%%%%%%%%%%%%%%%%%%%%%%%%%%%%%%%%%%%%%%%%%%%%%%%%%%%%%%%%%%%%%%%%%%%%%%
\subsection{Related CTAN Packages}

There are several other packages which offer a similar functionality:
%
\begin{itemize}
\item
The packages
\href{http://ctan.org/pkg/docmute}{\textsf{docmute}},
\href{http://ctan.org/pkg/includex}{\textsf{includex}} and
\href{http://ctan.org/pkg/standalone}{\textsf{standalone}}
provide commands to include only the document body of
a child file thus allowing both files to be compiled individually.
\item
The packages \href{http://ctan.org/pkg/subdocs}{\textsf{subdocs}}
and \href{http://ctan.org/pkg/subfiles}{\textsf{subfiles}}
provide structures in which the main and child documents can be
encapsulated and allowing them to be compiled individually.
The inclusion mechanism is different from the conventional |\include|.
\item
The package \href{http://ctan.org/pkg/combine}{\textsf{combine}}
is an elaborate solution to combine several documents into one.
\end{itemize}
%
See also the CTAN topic \href{http://ctan.org/topic/subdocs}{\textsf{subdocs}}
for further related packages.
The present package differs from the above solutions in that
a document structure constructed with the conventional |\include| mechanism
just needs two extra commands at the top of every file
such that all constituent files can be compiled individually.

%%%%%%%%%%%%%%%%%%%%%%%%%%%%%%%%%%%%%%%%%%%%%%%%%%%%%%%%%%%%%%%%%%%%%%%%%%%%%%%%
%\subsection{Feature Suggestions}
%
%The following is a list of features which may be useful for future
%versions of this package:
%%
%\begin{itemize}
%\item
%\ldots
%\end{itemize}

%%%%%%%%%%%%%%%%%%%%%%%%%%%%%%%%%%%%%%%%%%%%%%%%%%%%%%%%%%%%%%%%%%%%%%%%%%%%%%%%
\subsection{Revision History}

%%%%%%%%%%%%%%%%%%%%%%%%%%%%%%%%%%%%%%%%
\paragraph{v2.0:} 2018/12/30

\begin{itemize}
\item
immediate forward processing
\item
added |\childdocby| mechanism
\item
manual restructured
\end{itemize}

%%%%%%%%%%%%%%%%%%%%%%%%%%%%%%%%%%%%%%%%
\paragraph{v1.6:} 2018/01/17

\begin{itemize}
\item
application for development of include files
\item
corrections to manual
\end{itemize}

%%%%%%%%%%%%%%%%%%%%%%%%%%%%%%%%%%%%%%%%
\paragraph{v1.5:} 2017/05/21

\begin{itemize}
\item
more complete structuring introduced
\item
|\childdocof| introduced
\item
|\childdoc| renamed to |\childdocmain|
\item
|\childredirect| renamed to |\childdocforward| and |\childdocforwardprefix|
and functionality expanded
\end{itemize}

%%%%%%%%%%%%%%%%%%%%%%%%%%%%%%%%%%%%%%%%
\paragraph{v1.0:} 2017/04/27

\begin{itemize}
\item
manual and install package
\item
first version published on CTAN
\end{itemize}

%%%%%%%%%%%%%%%%%%%%%%%%%%%%%%%%%%%%%%%%
\paragraph{v0.6:} 2017/04/26

\begin{itemize}
\item
redirection mechanism added
\end{itemize}

%%%%%%%%%%%%%%%%%%%%%%%%%%%%%%%%%%%%%%%%
\paragraph{v0.5:} 2017/04/26

\begin{itemize}
\item
functionality in definition file
\end{itemize}


%%%%%%%%%%%%%%%%%%%%%%%%%%%%%%%%%%%%%%%%%%%%%%%%%%%%%%%%%%%%%%%%%%%%%%%%%%%%%%%%
%%%%%%%%%%%%%%%%%%%%%%%%%%%%%%%%%%%%%%%%%%%%%%%%%%%%%%%%%%%%%%%%%%%%%%%%%%%%%%%%
%%%%%%%%%%%%%%%%%%%%%%%%%%%%%%%%%%%%%%%%%%%%%%%%%%%%%%%%%%%%%%%%%%%%%%%%%%%%%%%%
\appendix

\settowidth\MacroIndent{\rmfamily\scriptsize 000\ }

 \DocInput{childdoc.dtx}

\end{document}
%</driver>
% \fi
%
% %%%%%%%%%%%%%%%%%%%%%%%%%%%%%%%%%%%%%%%%%%%%%%%%%%%%%%%%%%%%%%%%%%%%%%%%%%%%%%
% %%%%%%%%%%%%%%%%%%%%%%%%%%%%%%%%%%%%%%%%%%%%%%%%%%%%%%%%%%%%%%%%%%%%%%%%%%%%%%
% \section{Sample}
%\iffalse
%<*samplemain>
%\fi
%
% The following presents a sample document
% with two chapters, two parts, a title page,
% a compile flag as well as three forwarding files to set the flag.
% It consists of eight |.tex| files:
% \begin{center}
% \begin{tabular}{ll}
% |cdocsamp.tex|&main file\\
% |cdocsch1.tex|&include file for chapter 1\\
% |cdocsch2.tex|&include file for chapter 2\\
% |cdocspt3.tex|&include file for part 3\\
% |cdocspt4.tex|&include file for part 4\\
% |cdocsdrf.tex|&forwarding file for main file in draft mode\\
% |cdocsfi1.tex|&forwarding file for final version of chapter 1\\
% |cdocsfi2.tex|&forwarding file for final version of chapter 2\\
% \end{tabular}
% \end{center}
% Each of the eight files can be compiled directly by the \LaTeX{} compiler.
%
% %%%%%%%%%%%%%%%%%%%%%%%%%%%%%%%%%%%%%%
% \paragraph{Main File.}
%
% The main file is called |cdocsamp.tex|.
%
% Load the \textsf{childdoc} definitions and
% declare the filename for the main document:
%    \begin{macrocode}
\input{childdoc.def}
\childdocmain{}
%    \end{macrocode}

% Optional override for |\version| flag:
%    \begin{macrocode}
%%\ifchilddoc\else\providecommand{\version}{draft}\fi
%    \end{macrocode}

% Define the default values for the |\version| flag
% (|final| for the main file and |draft| for childs):
%    \begin{macrocode}
\ifchilddoc
\providecommand{\version}{draft}
\else
\providecommand{\version}{final}
\fi
%    \end{macrocode}

% Load the standard document class:
%    \begin{macrocode}
\documentclass[12pt]{article}
%    \end{macrocode}

% Start the document body:
%    \begin{macrocode}
\begin{document}
%    \end{macrocode}

% Declare a title page.
% Print title, part of document being processed and version flag:
%    \begin{macrocode}
\addtocounter{page}{-1}
\begin{center}
{\LARGE\bfseries{}childdoc example\par}
\vspace{1cm}
\ifchilddoc
\ifchilddocmanual part\else chapter\fi:
`\childdocname' of `\childdocjob'\par
\else
main document: `\childdocjob'\par
\fi
version: \version\par
\end{center}
\newpage
%    \end{macrocode}

% Manually include selected file,
% otherwise process as usual:
%    \begin{macrocode}
\ifchilddocmanual
\section*{part `\childdocname'}
\input{\childdocname}
\else
%    \end{macrocode}

% Include the two chapters:
%    \begin{macrocode}
\include{cdocsch1}
\include{cdocsch2}
%    \end{macrocode}

% Include the two parts unless only chapters should be displayed:
%    \begin{macrocode}
\ifchilddoc\else
\section{part three}
\input{cdocspt3}
\section{part four}
\input{cdocspt4}
\fi
%    \end{macrocode}

% Process as usual until here:
%    \begin{macrocode}
\fi
%    \end{macrocode}

% End of document body:
%    \begin{macrocode}
\end{document}
%    \end{macrocode}
%\iffalse
%</samplemain>
%\fi
%
% %%%%%%%%%%%%%%%%%%%%%%%%%%%%%%%%%%%%%%
% \paragraph{Chapter Include Files.}
%
% The include files are called |cdocsch1.tex| and |cdocsch2.tex|.
%
%\iffalse
%<*samplechap1|samplechap2>
%\fi

% Optional override for |\version| flag:
%    \begin{macrocode}
%%\providecommand{\version}{final}
%    \end{macrocode}

% Include the main document:
%    \begin{macrocode}
\input{childdoc.def}
\childdocof{cdocsamp}
%    \end{macrocode}

%\iffalse
%</samplechap1|samplechap2>
%\fi
%
%\iffalse
%<*samplechap1>
%\fi
% Some text for chapter 1:
%    \begin{macrocode}
\section{one}
some text in chapter one
%    \end{macrocode}

%\iffalse
%</samplechap1>
%\fi
% Some text for chapter 2:
%\iffalse
%<*samplechap2>
%\fi
%    \begin{macrocode}
\section{two}
more text in chapter two
%    \end{macrocode}

%\iffalse
%</samplechap2>
%\fi
%
% %%%%%%%%%%%%%%%%%%%%%%%%%%%%%%%%%%%%%%
% \paragraph{Part Include Files.}
%
% The include files are called |cdocspt3.tex| and |cdocspt4.tex|.
%
%\iffalse
%<*samplepart3|samplepart4>
%\fi

% Optional override for |\version| flag:
%    \begin{macrocode}
%%\providecommand{\version}{final}
%    \end{macrocode}

% Include the main document:
%    \begin{macrocode}
\input{childdoc.def}
\childdocby{cdocsamp}
%    \end{macrocode}

%\iffalse
%</samplepart3|samplepart4>
%\fi
%
%\iffalse
%<*samplepart3>
%\fi
% Some text for part 3:
%    \begin{macrocode}
some text in part three
%    \end{macrocode}

%\iffalse
%</samplepart3>
%\fi
% Some text for part 4:
%\iffalse
%<*samplepart4>
%\fi
%    \begin{macrocode}
more text in part four
%    \end{macrocode}

%\iffalse
%</samplepart4>
%\fi
%
% %%%%%%%%%%%%%%%%%%%%%%%%%%%%%%%%%%%%%%
% \paragraph{Forwarding for a Complete Draft.}
%
% The following forwarding file |cdocsdrf.tex|
% compiles the main document in draft mode:
%\iffalse
%<*sampledraft>
%\fi
%    \begin{macrocode}
\def\version{draft}
\input{childdoc.def}
\childdocforward{cdocsamp}
%    \end{macrocode}

%\iffalse
%</sampledraft>
%\fi
%
% %%%%%%%%%%%%%%%%%%%%%%%%%%%%%%%%%%%%%%
% \paragraph{Forwarding for Final Version of the Chapters.}
%
% The following forwarding files |cdocsfn1.tex| and |cdocsfn2.tex|
% (with identical content)
% compile the final versions of the child documents
% |cdocsch1.tex| and |cdocsch2.tex|, respectively:
%\iffalse
%<*samplefinal>
%\fi
%    \begin{macrocode}
\def\version{final}
\input{childdoc.def}
\childdocforwardprefix[cdocsamp]{cdocsfn}{cdocsch}
%    \end{macrocode}

%\iffalse
%</samplefinal>
%\fi
%
% %%%%%%%%%%%%%%%%%%%%%%%%%%%%%%%%%%%%%%
% \paragraph{Command Line Processing.}
%
% The following three command lines generate the output files
% |cdocscld|, |cdocscl1| and |cdocscl2|
% which should be identical to
% |cdocsdrf|, |cdocsch1| and |cdocsfn2|, respectively:
% \begin{center}
% \begin{tabular}{l}
% |latex -jobname cdocscld \|\\
% |  "\def\version{draft}\input{childdoc.def}\childdocforward{cdocsamp}"|\\
% |latex -jobname cdocscl1 \|\\
% |  "\input{childdoc.def}\childdocforward[cdocsamp]{cdocsch1}"|\\
% |latex -jobname cdocscl2 \|\\
% |  "\def\version{final}\input{childdoc.def}\childdocforward{cdocsch2}"|
% \end{tabular}
% \end{center}
% Note that the trailing backslash on each first line
% merely continues the input to the second line
% (for convenient cut ant paste).
% Furthermore, the command |latex| can be replaced by any
% of its alternative versions such as |pdflatex|.
%
% %%%%%%%%%%%%%%%%%%%%%%%%%%%%%%%%%%%%%%%%%%%%%%%%%%%%%%%%%%%%%%%%%%%%%%%%%%%%%%
% %%%%%%%%%%%%%%%%%%%%%%%%%%%%%%%%%%%%%%%%%%%%%%%%%%%%%%%%%%%%%%%%%%%%%%%%%%%%%%
% \section{Implementation}
%\iffalse
%<*package>
%\fi
%
% This section describes the definitions file |childdoc.def|.

% The definitions cannot be loaded using |\usepackage| or |\RequirePackage|
% which has a mechanism to prevent loading a style file more than once.
% When loading the definitions by means of |\input|
% multiple instances have to be prevented manually:
%\iffalse
%This code needs to be before the `\ProvidesFile' directive
%which is defined at the beginning of this file.
%Therefore it is also placed there and commented out here.
%</package>
%<*discard>
%\fi
%    \begin{macrocode}
\ifdefined\childdocmain\endinput\fi
%    \end{macrocode}
%\iffalse
%</discard>
%<*package>
%\fi
%
% \macro{\ifchilddoc}
% \macro{\ifchilddocmanual}
% The conditional |\ifchilddoc| tells whether a
% child (true) or main (false) document is being compiled.
% The conditional |\ifchilddocmanual| tells whether
% the |\includeonly| mechanism is used (false) or
% the selection of child files must be performed manually (true).
% The definitions initialise to false:
%    \begin{macrocode}
\newif\ifchilddoc
\newif\ifchilddocmanual
%    \end{macrocode}

% \macro{\childdocname}
% \macro{\childdocjob}
% The macro |\childdocname| stores the name of the main document
% to be compiled. The macro |\childdocjob| stores the name of
% the document on which the \LaTeX{} compiler was originally invoked.
% The content of |\jobname| cannot be compared
% to filenames specified in the source due to different catcodes.
% The following code rescans |\jobname|, stores the result
% in |\childdocname| and saves a copy in |\childdocjob|:
%    \begin{macrocode}
\edef\childdocname{\scantokens\expandafter{\jobname\noexpand}}
\let\childdocjob\childdocname
%    \end{macrocode}

% \macro{\childdocdisable}
% The macro |\childdocdisable| prevents the main file
% from being processed more than once.
% At this stage, the main document command |\childdocmain|
% is assumed to be called once again where it should do nothing.
% Any subsequent call to it should prevent
% a secondary processing of the main document
% It overwrites the forwarding commands
% |\childdocof| and |\childdocforward|
% with empty macros to prevent further inclusions of the main document:
%    \begin{macrocode}
\newcommand{\childdocdisable}
{
  \renewcommand{\childdocmain}[1]{\renewcommand{\childdocmain}[1]{\endinput}}
  \renewcommand{\childdocof}[1]{}
  \renewcommand{\childdocby}[2][]{}
  \renewcommand{\childdocforward}[2][]{}
  \renewcommand{\childdocdisable}{}
}
%    \end{macrocode}

% \macro{\childdocmain}
% The macro |\childdocmain| is to be called at the top of the main file
% with nothing or the main filename (without extension) as argument.
% First, it breaks loops.
% If the argument is not empty and does not match |\childdocname|
% (which is set by the first inclusion of |childdoc.def|),
% |\ifchilddoc| is set to true, |\includeonly| is applied to the child file
% and |\jobname| is set to the main file
% (for proper handling of |.aux| files):
%    \begin{macrocode}
\newcommand{\childdocmain}[1]
{
  \childdocdisable\childdocmain{}
  \if?#1?\else
    \begingroup
      \def\childdoctmp{#1}
      \ifx\childdoctmp\childdocname
        \def\childdoctmp{}
      \else
        \def\childdoctmp
        {
          \childdoctrue
          \includeonly{\childdocname}
          \def\childdocjob{#1}
          \def\jobname{#1}
        }
      \fi
      \expandafter
    \endgroup
    \childdoctmp
  \fi
}
%    \end{macrocode}

% \macro{\childdocof}
% The command |\childdocof| redirects
% compilation to the main file |#1|.
%    \begin{macrocode}
\newcommand{\childdocof}[1]
{
  \childdocdisable
  \childdoctrue
  \includeonly{\childdocname}
  \def\jobname{#1}
  \def\childdocjob{#1}
  \input{#1}
}
%    \end{macrocode}

% \macro{\childdocby}
% The command |\childdocby| ....
%    \begin{macrocode}
\newcommand{\childdocby}[2][]
{
  \childdocdisable
  \childdoctrue
  \childdocmanualtrue
  \if?#1?\else
    \def\jobname{#2}
  \fi
  \def\childdocjob{#2}
  \input{#2}
  \endinput
}
%    \end{macrocode}

% \macro{\childdocforward}
% The command |\childdocforward| redirects
% compilation to the main file or
% (if the optional argument is given) a child file.
% Parameters are set as if the main file
% or a child file starting with |\childdocof| was compiled.
% Then compilation is handed over to the main file:
%    \begin{macrocode}
\newcommand{\childdocforward}[2][]
{
  \begingroup
    \if?#1?
      \def\childdoctmp
      {
        \def\childdocname{#2}
        \def\childdocjob{#2}
        \def\jobname{#2}
        \input{#2}
        \endinput
      }
    \else
      \def\childdoctmp
      {
        \childdocdisable
        \def\childdocname{#2}
        \childdoctrue
        \includeonly{#2}
        \def\childdocjob{#1}
        \def\jobname{#1}
        \input{#1}
        \endinput
      }
    \fi
    \expandafter
  \endgroup
  \childdoctmp
}
%    \end{macrocode}

% \macro{\childdocforwardprefix}
% The command |\childdocforwardprefix| redirects
% compilation to the main or a child file by means of a pattern.
% The prefix |#1| in the current filename is replaced by |#2|
% and the suffix of the current filename is kept
% (it is assumed that the filename does not contain the substring `|~~~|'
% which is used as a delimiter).
% Compilation is handed over to the new file by |\childdocforward|:
%    \begin{macrocode}
\newcommand{\childdocforwardprefix}[3][]
{
  \begingroup
    \def\childdocextract #2##1~~~{\def\childdoctmp{\childdocforward[#1]{#3##1}}}
    \expandafter\childdocextract\childdocname~~~
    \expandafter
  \endgroup
  \childdoctmp
}
%    \end{macrocode}

% \macro{\childdoc}
% The deprecated macro |\childdoc| is a legacy version of |\childdocmain|:
%    \begin{macrocode}
\newcommand{\childdoc}{\childdocmain}
%    \end{macrocode}

% \macro{\childdocredirect}
% The deprecated macro |\childdocredirect| is a legacy version
% of |\childdocforward| and |\childdocforwardprefix|:
%    \begin{macrocode}
\newcommand{\childdocredirect}[2][]
{
  \begingroup
    \if?#1?
      \def\childdoctmp{\childdocforward{#2}}
    \else
      \def\childdoctmp{\childdocforwardprefix{#1}{#2}}
    \fi
    \expandafter
  \endgroup
  \childdoctmp
}
%    \end{macrocode}

%\iffalse
%</package>
%\fi
%
\endinput
|\\
|\childdocforward[|\textit{main}|]{|\textit{dest}|}|\\
\end{tabular}
\end{center}
%
The argument \textit{dest} is the destination file
(without extension).
It should be the main file or one of the child files.
Note that further \textsf{childdoc} directives
such as |\childdocof| and |\childdocforward|
in the indicated file will be processed in this form.
The optional argument \textit{main}
passes on directly to the main file \textit{main}
while pretending to compile the child \textit{dest}.
This form behaves as if \textit{dest}
issues |\childdocof{|\textit{main}|}| right away,
and no further \textsf{childdoc} directives will be processed.

%%%%%%%%%%%%%%%%%%%%%%%%%%%%%%%%%%%%%%%%
\DescribeMacro{\...prefix}
In the alternative form |\childdocforwardprefix|,
%
\begin{center}
\begin{tabular}{l}
|% \iffalse
%
% childdoc.dtx Copyright (C) 2017-2018 Niklas Beisert
%
% This work may be distributed and/or modified under the
% conditions of the LaTeX Project Public License, either version 1.3
% of this license or (at your option) any later version.
% The latest version of this license is in
%   http://www.latex-project.org/lppl.txt
% and version 1.3 or later is part of all distributions of LaTeX
% version 2005/12/01 or later.
%
% This work has the LPPL maintenance status `maintained'.
%
% The Current Maintainer of this work is Niklas Beisert.
%
% This work consists of the files childdoc.dtx and childdoc.ins
% and the derived files childdoc.def and cdocsamp.tex with
% cdocsch1.tex, cdocsch2.tex, cdocsdrf.tex, cdocsfn1.tex, cdocsfn2.tex.
%
%<package>\ifdefined\childdocmain\endinput\fi
%<package>\ProvidesFile{childdoc.def}[2018/12/30 v2.0 child document driver]
%<samplemain>\ProvidesFile{cdocsamp.tex}[2018/12/30 v2.0 sample for childdoc]
%<*driver>
%\ProvidesFile{childdoc.drv}[2018/12/30 v2.0 childdoc reference manual file]
\PassOptionsToClass{10pt,a4paper}{article}
\documentclass{ltxdoc}

\usepackage[margin=35mm]{geometry}
\usepackage{hyperref}
\usepackage{hyperxmp}
\usepackage[usenames]{color}

\hypersetup{colorlinks=true}
\hypersetup{pdfstartview=FitH}
\hypersetup{pdfpagemode=UseNone}
\hypersetup{pdfsource={}}
\hypersetup{pdflang={en-UK}}
\hypersetup{pdfcopyright={Copyright 2017-2018 Niklas Beisert.
  This work may be distributed and/or modified under the
  conditions of the LaTeX Project Public License, either version 1.3
  of this license or (at your option) any later version.}}
\hypersetup{pdflicenseurl={http://www.latex-project.org/lppl.txt}}
\hypersetup{pdfcontactaddress={ETH Zurich, ITP, HIT K,
  Wolfgang-Pauli-Strasse 27}}
\hypersetup{pdfcontactpostcode={8093}}
\hypersetup{pdfcontactcity={Zurich}}
\hypersetup{pdfcontactcountry={Switzerland}}
\hypersetup{pdfcontactemail={nbeisert@itp.phys.ethz.ch}}
\hypersetup{pdfcontacturl={http://people.phys.ethz.ch/\xmptilde nbeisert/}}

\newcommand{\secref}[1]{\hyperref[#1]{section \ref*{#1}}}

\parskip1ex
\parindent0pt
\let\olditemize\itemize
\def\itemize{\olditemize\parskip0pt}

\begin{document}

\title{The \textsf{childdoc} Package}
\hypersetup{pdftitle={The childdoc Package}}
\author{Niklas Beisert\\[2ex]
  Institut f\"ur Theoretische Physik\\
  Eidgen\"ossische Technische Hochschule Z\"urich\\
  Wolfgang-Pauli-Strasse 27, 8093 Z\"urich, Switzerland\\[1ex]
  \href{mailto:nbeisert@itp.phys.ethz.ch}
  {\texttt{nbeisert@itp.phys.ethz.ch}}}
\hypersetup{pdfauthor={Niklas Beisert}}
\hypersetup{pdfsubject={Manual for the LaTeX2e Package childdoc}}
\date{30 December 2018, \textsf{v2.0}}
\maketitle

\begin{abstract}\noindent
\textsf{childdoc} is a \LaTeXe{} package
that enables the direct compilation
of document sections included by |\include|
to individual files.
\end{abstract}

\begingroup
\parskip0ex
\tableofcontents
\endgroup

%%%%%%%%%%%%%%%%%%%%%%%%%%%%%%%%%%%%%%%%%%%%%%%%%%%%%%%%%%%%%%%%%%%%%%%%%%%%%%%%
%%%%%%%%%%%%%%%%%%%%%%%%%%%%%%%%%%%%%%%%%%%%%%%%%%%%%%%%%%%%%%%%%%%%%%%%%%%%%%%%
\section{Introduction}

\LaTeX{} provides a mechanism to structure a large document (such as a book)
into a main file and several child files (containing the chapters)
using the |\include| command.
This mechanism is beneficial for documents
which span hundreds of pages in order to
make the source file(s) more manageable.
Moreover, compilation can be restricted to
selected child files by means of the |\includeonly| command.
The latter feature can be used to reduce the compilation time while editing
(this was significantly more useful in the earlier days of \LaTeX{})
or to generate a smaller document which is easier to navigate.
Another application of |\includeonly| is to generate
documents consisting of selected parts of the complete document.

However, there are a few drawbacks of the plain |\include| mechanism:
\begin{itemize}
\item
The child files cannot be compiled on their own,
they can only be compiled via the main file.
A naive editing environment
(such as a text editor with an option
to have the current file processed by \LaTeX)
may require one to switch to the main file before compiling;
attempting to compile the child file produces errors.
\item
The main file must be modified (each time)
to adjust the |\includeonly| command
to the present needs. This easily leaves the main file in a messy state.
\item
The generated document will always carry the filename
of the main document. This is inconvenient if
several child files are to be compiled and
to be kept for distribution.
\end{itemize}

The present package provides a simple interface
to make child files individually compilable by \LaTeX{}.
Compiling a child file then has the same effect as compiling
the main file with an |\includeonly| command
to select the appropriate child.
Moreover the generated document will carry the name of the child
rather than the main file.
This resolves all three above issues.

This feature is meant to make the editing of books,
thesis documents and lecture notes somewhat more convenient.
However, the package can also be used efficiently for
composing a series of documents (such as exercise sheets)
which are typically distributed individually.
It then assists the author in generating the individual documents
(potentially in different versions)
as well as a document containing the collected series.
Another application is in developing style files
or other kinds of included material
where compilation of the style file could redirect
to a sample or test file.

%%%%%%%%%%%%%%%%%%%%%%%%%%%%%%%%%%%%%%%%%%%%%%%%%%%%%%%%%%%%%%%%%%%%%%%%%%%%%%%%
%%%%%%%%%%%%%%%%%%%%%%%%%%%%%%%%%%%%%%%%%%%%%%%%%%%%%%%%%%%%%%%%%%%%%%%%%%%%%%%%
\section{Usage}

First of all, the package \textsf{childdoc} is \emph{not} a standard
\LaTeXe{} |.sty| style file! Therefore it needs to be invoked in
a non-standard way.

%%%%%%%%%%%%%%%%%%%%%%%%%%%%%%%%%%%%%%%%%%%%%%%%%%%%%%%%%%%%%%%%%%%%%%%%%%%%%%%%
\subsection{Included Files}
\label{sec:include}

%%%%%%%%%%%%%%%%%%%%%%%%%%%%%%%%%%%%%%%%
\DescribeMacro{\childdocmain}
To use the package, add the commands
\begin{center}
\begin{tabular}{l}
|\input{childdoc.def}|\\
|\childdocmain{}|\\
\end{tabular}
\end{center}
at the very top of the main \LaTeX{} file,
in particular \emph{before} the |\documentclass| statement!
The argument of |\childdocmain| should be left empty
(but it must be present).

%%%%%%%%%%%%%%%%%%%%%%%%%%%%%%%%%%%%%%%%
\DescribeMacro{\childdocof}
Furthermore, add the commands
\begin{center}
\begin{tabular}{l}
|\input{childdoc.def}|\\
|\childdocof{|\textit{main}|}|\\
\end{tabular}
\end{center}
at the top of every child file \textit{child}
which is included by |\include{|\textit{child}|}|
from within the main file
(or at least for those files to be compiled individually).
The argument \textit{main} must be the filename of the main file.

There are a couple of
considerations in setting up the main and child documents:

%%%%%%%%%%%%%%%%%%%%%%%%%%%%%%%%%%%%%%%%
\paragraph{Restrictions.}

Please note the following restrictions:
\begin{itemize}
\item
|\childdocmain| must be called with one argument \textit{main}
to ensure compatibility with earlier version of the package.
It must either be empty (|\childdocmain{}|)
or precisely match the filename of the main file in which it is specified.
See \secref{sec:detection} for further information.
\item
The filename \textit{main} must be specified without the |.tex| extension.
\item
The filename \textit{main} is case sensitive
(even in case-insensitive file systems)
due to internal string comparison.
\item
The argument \textit{main} should be fully expanded, it cannot be a macro.
\item
Subdirectories and special characters should be avoided in filenames.
\item
The command |\childdocmain{|\textit{main}|}| must be followed by a whitespace.
It should not be followed immediately by another command
or by a comment mark `|%|'.
This is because the \TeX{} parser reads the token immediately following
the argument of |\childdocmain| and puts it
at the beginning of every child section;
however, a white\-space is ignored.
\end{itemize}

%%%%%%%%%%%%%%%%%%%%%%%%%%%%%%%%%%%%%%%%
\paragraph{Content of Main File.}

It is advisable to place all content in the child files included by |\include|.
Any output contained in the main file will appear in all child documents
unless suppressed manually;
it cannot be suppressed automatically by the |\includeonly| directive
and thus should normally be avoided.
A method to include some content in the main file
by means of conditional processing is described in \secref{sec:conditional}.

%%%%%%%%%%%%%%%%%%%%%%%%%%%%%%%%%%%%%%%%
\paragraph{Page Numbering.}

When only a part of the document is compiled,
the appropriate numbering of pages
(as well as other status parameters)
is determined from the |.aux| files.
The latter contain information from previous passes.
However this information needs to propagate through
all intermediate child documents.
Therefore the page numbering in child documents may well
be inconsistent until the complete document is compiled at least once.

A useful (if unconventional) way to always ensure a consistent
page numbering is to restart the numbering in each child document
and denote the pages by `\textit{child}|.|\textit{page}'
where \textit{child} represents the chapter/section number of the child file.
This can be achieved by the command
|\numberwithin{page}{|\textit{child}|}|
of the \textsf{amsmath} package
where \textit{child} can be |chapter| or |section|
depending on the chosen structuring.
Alternatively, one can modify the macro |\thepage| appropriately
and reset the counter |page| at the start of each child file.

%%%%%%%%%%%%%%%%%%%%%%%%%%%%%%%%%%%%%%%%%%%%%%%%%%%%%%%%%%%%%%%%%%%%%%%%%%%%%%%%
\subsection{Conditional Processing}
\label{sec:conditional}

The package provides a mechanism to compile different versions
of a document. To customise the versions further some conditional processing
can come in handy to distinguish which version is being compiled.
The package provides two macros to describe the compilation context:

%%%%%%%%%%%%%%%%%%%%%%%%%%%%%%%%%%%%%%%%
\DescribeMacro{\ifchilddoc}
The conditional |\ifchilddoc| distinguishes between the compilation of
child documents and the main document:
%
\begin{center}
|\ifchilddoc |\textit{child-code}| |[|\||else |\textit{main-code}]| \||fi|
\end{center}

%%%%%%%%%%%%%%%%%%%%%%%%%%%%%%%%%%%%%%%%
\DescribeMacro{\childdocname}
\DescribeMacro{\childdocjob}
The macro |\childdocname| contains the filename (without extension)
of the main or child file being processed.
Note that |\childdocjob| will always contain the name of the main file.

%%%%%%%%%%%%%%%%%%%%%%%%%%%%%%%%%%%%%%%%
\paragraph{Title Page.}

Conditional processing can be used to include a title or banner page
in the main document when proper precautions are taken.
Importantly, the code in the main file should ensure that the page counter
(as well as other status parameters which are stored in the |.aux| files)
takes the same value after the conditional processing.
Otherwise the page numbers may take divergent values
depending on which part is compiled.

For example, a title page could be declared by:
%
\begin{center}
\begin{tabular}{l}
|\ifchilddoc\||else|\\
|\addtocounter{page}{-1}|\\
\textit{code for title page}\\
|\newpage|\\
|\||fi|
\end{tabular}
\end{center}
%
A banner page for the child documents can be generated by:
%
\begin{center}
\begin{tabular}{l}
|\ifchilddoc|\\
|\addtocounter{page}{-1}|\\
\textit{code for banner page}\\
|\newpage|\\
|\||fi|
\end{tabular}
\end{center}
%
Here one could write a message such as:
\begin{center}
|This is the part \childdocname{} of \childdocjob{}.|
\end{center}

%%%%%%%%%%%%%%%%%%%%%%%%%%%%%%%%%%%%%%%%%%%%%%%%%%%%%%%%%%%%%%%%%%%%%%%%%%%%%%%%
\subsection{Flags}
\label{sec:flags}

The package makes it easy to generate different versions
of the main or child documents.
To this end compilation flags can be defined
and assigned different default values.
They will be particularly useful in conjunction
with the forwarding mechanism described in \secref{sec:forward}.

For example, it may be useful to have a flag |\version|
which can be set to |draft| or |final|.
The document source will contain some conditional code
depending on the value of |\version|.
Suppose further, the flag should default to |final| for the main file
and to |draft| for child files
which is a natural assignment for editing the document.
This is achieved by placing the following code
in the preamble of the main document
(below the |\childdocmain| directive):
%
\begin{center}
\begin{tabular}{l}
|\ifchilddoc|\\
|\providecommand{\version}{draft}|\\
|\||else|\\
|\providecommand{\version}{final}|\\
|\||fi|
\end{tabular}
\end{center}
%
The definition by |\providecommand| makes sure
that previous definitions are not overwritten.
Further statements |\providecommand{\version}{...}|
can thus be added before the above code to override it.

For the main file, one might add a line
(between |\childdocmain| and the above block)
%
\begin{center}
|%\ifchilddoc\||else\providecommand{\version}{draft}\||fi|
\end{center}
%
which can be uncommented to produce a draft version.
Likewise one can add a line to the very top of a child file
(above the |\childdocof{|\textit{main}|}| directive)
%
\begin{center}
|%\providecommand{\version}{final}|
\end{center}
%
which can be uncommented to produce the final version of this child document.

%%%%%%%%%%%%%%%%%%%%%%%%%%%%%%%%%%%%%%%%%%%%%%%%%%%%%%%%%%%%%%%%%%%%%%%%%%%%%%%%
\subsection{Forwarding}
\label{sec:forward}

Different versions of the main or child documents
using compilation flags as described in \secref{sec:flags}
can be (permanently) stored in different files
for convenient compilation, viewing and distribution.
To this end, the package defines a command
to pass on compilation to a different file:

%%%%%%%%%%%%%%%%%%%%%%%%%%%%%%%%%%%%%%%%
\DescribeMacro{\childdocforward}
The command |\childdocforward| redirects processing to
another source file:
%
\begin{center}
\begin{tabular}{l}
|\input{childdoc.def}|\\
|\childdocforward[|\textit{main}|]{|\textit{dest}|}|\\
\end{tabular}
\end{center}
%
The argument \textit{dest} is the destination file
(without extension).
It should be the main file or one of the child files.
Note that further \textsf{childdoc} directives
such as |\childdocof| and |\childdocforward|
in the indicated file will be processed in this form.
The optional argument \textit{main}
passes on directly to the main file \textit{main}
while pretending to compile the child \textit{dest}.
This form behaves as if \textit{dest}
issues |\childdocof{|\textit{main}|}| right away,
and no further \textsf{childdoc} directives will be processed.

%%%%%%%%%%%%%%%%%%%%%%%%%%%%%%%%%%%%%%%%
\DescribeMacro{\...prefix}
In the alternative form |\childdocforwardprefix|,
%
\begin{center}
\begin{tabular}{l}
|\input{childdoc.def}|\\
|\childdocforwardprefix[|\textit{main}|]{|\textit{prefix}|}{|\textit{dest}|}|
\end{tabular}
\end{center}
%
the destination file is determined by a pattern
depending on the current file:
To make this work, the current file must be called
`{\textit{prefix}\hspace{0.2em}\textit{suffix}}'
with \textit{prefix} matching precisely the argument.
Processing is then passed on to the file
`{\textit{dest}\hspace{0.2em}\textit{suffix}}'.
Surely, the same effect is achieved by
directly specifying the
argument `{\textit{dest}\hspace{0.2em}\textit{suffix}}'
in the first form.
However, that requires to set up a different file
for each child. With the alternative form of the command
all these files can have exactly the same content
which simplifies setting them up and maintaining them.

For example, the following file |draft.tex|
with a compilation flag |\version| as described in \secref{sec:flags}
compiles the main document as a draft:
%
\begin{center}
\begin{tabular}{l}
|\def\version{draft}|\\
|\input{childdoc.def}|\\
|\childdocforward{|\textit{main}|}|
\end{tabular}
\end{center}
%
Likewise, the following files |final|\textit{nn}|.tex|
compile the final version of the child document
|child|\textit{nn}|.tex|:
%
\begin{center}
\begin{tabular}{l}
|\def\version{final}|\\
|\input{childdoc.def}|\\
|\childdocforwardprefix{final}{child}|
\end{tabular}
\end{center}
%

Note that when several versions of a main file and/or of each child file
are to be generated, it may be convenient to set up a |Makefile| or
shell script to automatise the process.

%%%%%%%%%%%%%%%%%%%%%%%%%%%%%%%%%%%%%%%%%%%%%%%%%%%%%%%%%%%%%%%%%%%%%%%%%%%%%%%%
\subsection{Command Line Processing}
\label{sec:commandline}

The effect of redirection files can also be achieved by invoking
the \LaTeX{} compiler with a more elaborate command line.
Most conveniently this should be done as part
of a shell script or a |Makefile|.

When using \textsf{childdoc} in the main file, the following
command lines effectively perform a redirection
(note that depending on the shell being used,
backslashes may have to be doubled: `|\|' $\to$ `|\\|'):
%
\begin{center}
|... -jobname "|\textit{target}|" |\\|"|[\textit{flags}]%
|\input{childdoc.def}\childdocforward[|\textit{main}|]{|\textit{dest}|}"|
\end{center}
%
Here \textit{target} is the name of the output file,
\textit{main} is the name of the main file
and \textit{dest} is the name of the main or child file to be processed
(all filenames without extensions).
The optional argument \textit{main} can be omitted
if \textit{main} matches \textit{dest}.
Optionally, compilation \textit{flags} can be defined via |\def| commands.
This command line makes the \TeX{} engine believe
it is compiling the file \textit{target}
whose content is specified as the latter parameter.
The provided code then forwards the processing to
\textit{main} or \textit{dest} as described in \secref{sec:forward}.

%%%%%%%%%%%%%%%%%%%%%%%%%%%%%%%%%%%%%%%%%%%%%%%%%%%%%%%%%%%%%%%%%%%%%%%%%%%%%%%%
\subsection{Include by Input}
\label{sec:input}

Including child documents by |\include| has some restrictions by design.
Most notably, the content of a child document always occupies
its own set of pages; pages cannot be shared between child documents.
Usually, this behaviour makes perfect sense
because each child document contain an essential part of the document.
However, in some situations it may be desirable to compose
a document from a collection of parts
without having mandatory page breaks between then.
For this case, the package
provides a mechanism to include parts
by |\input| which can also be processed individually.
However, by construction this mechanism
requires manual handling of the content to be output.

%%%%%%%%%%%%%%%%%%%%%%%%%%%%%%%%%%%%%%%%
\DescribeMacro{\ifchilddocmanual}
The main file should be prepared as usual, see \secref{sec:include}.
However, the document body must make a distinction
between processing of an individual part and of the main document, e.g.:
%
\begin{center}
\begin{tabular}{l}
|\ifchilddocmanual|\\
|\input{\childdocname}|\\
|\||else|\\
\textit{document body with }|\input{|\textit{part}|}|\\
|\||fi|
\end{tabular}
\end{center}
%
The conditional |\ifchilddocmanual| is true whenever
a part to be included by |\input| is being compiled,
and the name of the part is stored in |\childdocname|.

%%%%%%%%%%%%%%%%%%%%%%%%%%%%%%%%%%%%%%%%
\DescribeMacro{\childdocby}
Each part to be included by |\input| should start with:
%
\begin{center}
\begin{tabular}{l}
|\input{childdoc.def}|\\
|\childdocby{|\textit{main}|}|\\
\end{tabular}
\end{center}
%
The directive |\childdocby| is similar to |\childdocof|
described in \secref{sec:include},
but the subsequent selection of content must be done manually.
To that end, both |\ifchilddoc| and |\ifchilddocmanual|
will be true upon processing of a part,
and the name of the part is stored in |\childdocname|.
Note that |\jobname| will be set to the filename of the current part
so that each part receives an individual |.aux| file
that does not interfere with the |.aux| file(s) of the main document.
This behaviour can be altered by the alternative form
|\childdocby[*]{|\textit{main}|}| (with a non-empty optional argument)
which uses the |.aux| file of the main document
by setting |\jobname| to \textit{main}.

%%%%%%%%%%%%%%%%%%%%%%%%%%%%%%%%%%%%%%%%%%%%%%%%%%%%%%%%%%%%%%%%%%%%%%%%%%%%%%%%
\subsection{Driver Development}
\label{sec:driver}

The \textsf{childdoc} mechanism can also be use for the development
of definition files such as \LaTeX{} styles or classes.
This case differs from the above setup with multiple parts
included by |\include| in that no |\includeonly| should be invoked.
This can be achieved by starting the include file
(before |\ProvidesPackage|) with:
%
\begin{center}
\begin{tabular}{l}
|\input{childdoc.def}|\\
|\childdocforward{|\textit{main}|}|\\
\end{tabular}
\end{center}
%
or alternatively with:
%
\begin{center}
\begin{tabular}{l}
|\input{childdoc.def}|\\
|\childdocby{|\textit{main}|}|\\
\end{tabular}
\end{center}
%
Both forms have slightly different effects as described above.
The main file is prepared as usual, see \secref{sec:include}.

%%%%%%%%%%%%%%%%%%%%%%%%%%%%%%%%%%%%%%%%%%%%%%%%%%%%%%%%%%%%%%%%%%%%%%%%%%%%%%%%
\subsection{Legacy Detection}
\label{sec:detection}

The directive |\childdocmain| in the main file can detect
whether the complete document or merely a child is to be compiled
even without using the directive |\childdocof|.
This method is deprecated because it is less robust
and there is no compelling reason to use it;
it is merely provided for backward compatibility
and it may be removed in future versions.

If the detection mechanism is to be used,
it is mandatory to correctly specify
the filename of the main file as the argument of |\childdocmain|:
%
\begin{center}
\begin{tabular}{l}
|\input{childdoc.def}|\\
|\childdocmain{|\textit{main}|}|\\
\end{tabular}
\end{center}
%
If |\jobname| does not match the argument \textit{main} of |\childdocmain|,
it is assumed that |\jobname| points to the child file to be compiled.
When using |\childdocmain| with the main file specified as argument,
it suffices to start a child file
with just |\input{|\textit{main}|}|
without loading of the package and using |\childdocof|.
If instead all processing is done
with the appropriate \textsf{childdoc} directives,
the argument of \textit{main} of |\childdocmain| can be empty.

An alternative version of the command line processing described
in \secref{sec:commandline} using the detection mechanism reads:
%
\begin{center}
|... -jobname "|\textit{target}|" "|[\textit{flags}]%
[|\def\jobname{|\textit{dest}|}|]|\input{|\textit{main}|}"|
\end{center}

%%%%%%%%%%%%%%%%%%%%%%%%%%%%%%%%%%%%%%%%%%%%%%%%%%%%%%%%%%%%%%%%%%%%%%%%%%%%%%%%
\subsection{Manual Code}
\label{sec:manual}

In case one cannot be certain whether the definitions file |childdoc.def|
is installed on the target \TeX{} distribution
and one prefers not to ship it,
it is conceivable to paste a few relevant commands into the sources.

To that end, drop all statements |\input{childdoc.def}|
and perform the replacements as outlined below.
Instead of |\childdocmain{|\textit{main}|}| add the following code
to the top of the main file:
%
\begin{center}
\begin{tabular}{l}
|\||ifdefined\childdocname\endinput\||fi\newif\ifchilddoc|\\
|\edef\childdocname{\scantokens\expandafter{\jobname\noexpand}}|\\
|\def\childdocmain{|\textit{main}|}\||ifx\childdocmain\childdocname\||else|\\
|\childdoctrue\includeonly{\childdocname}\let\jobname\childdocmain\||fi|\\
\end{tabular}
\end{center}
%
Instead of |\childdocof{|\textit{main}|}| just include the main file
at the top of each child file:
%
\begin{center}
|\input{|\textit{main}|}|
\end{center}
%
A simple redirection |\childdocforward{|\textit{dest}|}| is achieved by:
%
\begin{center}
|\def\jobname{|\textit{dest}|}\input{\jobname}|
\end{center}
%
The redirection with prefix
|\childdocforwardprefix[|\textit{prefix}|]{|\textit{dest}|}|
is accomplished by:
%
\begin{center}
\begin{tabular}{l}
|{\edef\jobname{\scantokens\expandafter{\jobname\noexpand}}|\\
|\def\redirectjob |\textit{prefix}|#1~~~{\gdef\jobname{|\textit{dest}|#1}}|\\
|\expandafter\redirectjob\jobname~~~}\input{\jobname}|
\end{tabular}
\end{center}

In an alternative approach,
child documents can be compiled by a specific command line
without additional code or specific definitions:
%
\begin{center}
|... -jobname "|\textit{target}|" "|[\textit{flags}]%
|\includeonly{|\textit{dest}|}\input{|\textit{main}|}"|
\end{center}
%

%%%%%%%%%%%%%%%%%%%%%%%%%%%%%%%%%%%%%%%%%%%%%%%%%%%%%%%%%%%%%%%%%%%%%%%%%%%%%%%%
%%%%%%%%%%%%%%%%%%%%%%%%%%%%%%%%%%%%%%%%%%%%%%%%%%%%%%%%%%%%%%%%%%%%%%%%%%%%%%%%
\section{Information}

%%%%%%%%%%%%%%%%%%%%%%%%%%%%%%%%%%%%%%%%%%%%%%%%%%%%%%%%%%%%%%%%%%%%%%%%%%%%%%%%
\subsection{Copyright}

Copyright \copyright{} 2017--2018 Niklas Beisert

This work may be distributed and/or modified under the
conditions of the \LaTeX{} Project Public License, either version 1.3
of this license or (at your option) any later version.
The latest version of this license is in
  \url{http://www.latex-project.org/lppl.txt}
and version 1.3 or later is part of all distributions of \LaTeX{}
version 2005/12/01 or later.

This work has the LPPL maintenance status `maintained'.

The Current Maintainer of this work is Niklas Beisert.

This work consists of the files |README.txt|, |childdoc.ins| and |childdoc.dtx|
as well as the derived files |childdoc.def|, |cdocsamp.tex|
with |cdocsch1.tex|, |cdocsch2.tex|, |cdocspt3.tex|, |cdocspt4.tex|,
|cdocsdrf.tex|, |cdocsfn1.tex|, |cdocsfn2.tex|
as well as |childdoc.pdf|.

%%%%%%%%%%%%%%%%%%%%%%%%%%%%%%%%%%%%%%%%%%%%%%%%%%%%%%%%%%%%%%%%%%%%%%%%%%%%%%%%
\subsection{Files and Installation}

The package consists of the files:
%
\begin{center}
\begin{tabular}{ll}
    |README.txt|   & readme file \\
    |childdoc.ins| & installation file \\
    |childdoc.dtx| & source file \\
    |childdoc.def| & definition file \\
    |cdocsamp.tex| & sample main file \\
    |cdocsch1.tex| & sample include file \\
    |cdocsch2.tex| & sample include file \\
    |cdocspt3.tex| & sample part file \\
    |cdocspt4.tex| & sample part file \\
    |cdocsdrf.tex| & sample redirection file \\
    |cdocsfn1.tex| & sample redirection file \\
    |cdocsfn2.tex| & sample redirection file \\
    |childdoc.pdf| & manual
\end{tabular}
\end{center}
%
The distribution consists of the files
|README.txt|, |childdoc.ins| and |childdoc.dtx|.
%
\begin{itemize}
\item
Run (pdf)\LaTeX{} on |childdoc.dtx|
to compile the manual |childdoc.pdf| (this file).
\item
Run \LaTeX{} on |childdoc.ins| to create the definitions file |childdoc.def|
and the sample |cdocsamp.tex| with include files
|cdocsch1.tex|, |cdocsch2.tex|, |cdocspt3.tex|, |cdocspt4.tex|,
|cdocsdrf.tex|, |cdocsfn1.tex|, |cdocsfn2.tex|.
Then copy the file |childdoc.def| to an appropriate directory of your \LaTeX{}
distribution, e.g.\ \textit{texmf-root}|/tex/latex/childdoc|.
\end{itemize}

%%%%%%%%%%%%%%%%%%%%%%%%%%%%%%%%%%%%%%%%%%%%%%%%%%%%%%%%%%%%%%%%%%%%%%%%%%%%%%%%
\subsection{Related CTAN Packages}

There are several other packages which offer a similar functionality:
%
\begin{itemize}
\item
The packages
\href{http://ctan.org/pkg/docmute}{\textsf{docmute}},
\href{http://ctan.org/pkg/includex}{\textsf{includex}} and
\href{http://ctan.org/pkg/standalone}{\textsf{standalone}}
provide commands to include only the document body of
a child file thus allowing both files to be compiled individually.
\item
The packages \href{http://ctan.org/pkg/subdocs}{\textsf{subdocs}}
and \href{http://ctan.org/pkg/subfiles}{\textsf{subfiles}}
provide structures in which the main and child documents can be
encapsulated and allowing them to be compiled individually.
The inclusion mechanism is different from the conventional |\include|.
\item
The package \href{http://ctan.org/pkg/combine}{\textsf{combine}}
is an elaborate solution to combine several documents into one.
\end{itemize}
%
See also the CTAN topic \href{http://ctan.org/topic/subdocs}{\textsf{subdocs}}
for further related packages.
The present package differs from the above solutions in that
a document structure constructed with the conventional |\include| mechanism
just needs two extra commands at the top of every file
such that all constituent files can be compiled individually.

%%%%%%%%%%%%%%%%%%%%%%%%%%%%%%%%%%%%%%%%%%%%%%%%%%%%%%%%%%%%%%%%%%%%%%%%%%%%%%%%
%\subsection{Feature Suggestions}
%
%The following is a list of features which may be useful for future
%versions of this package:
%%
%\begin{itemize}
%\item
%\ldots
%\end{itemize}

%%%%%%%%%%%%%%%%%%%%%%%%%%%%%%%%%%%%%%%%%%%%%%%%%%%%%%%%%%%%%%%%%%%%%%%%%%%%%%%%
\subsection{Revision History}

%%%%%%%%%%%%%%%%%%%%%%%%%%%%%%%%%%%%%%%%
\paragraph{v2.0:} 2018/12/30

\begin{itemize}
\item
immediate forward processing
\item
added |\childdocby| mechanism
\item
manual restructured
\end{itemize}

%%%%%%%%%%%%%%%%%%%%%%%%%%%%%%%%%%%%%%%%
\paragraph{v1.6:} 2018/01/17

\begin{itemize}
\item
application for development of include files
\item
corrections to manual
\end{itemize}

%%%%%%%%%%%%%%%%%%%%%%%%%%%%%%%%%%%%%%%%
\paragraph{v1.5:} 2017/05/21

\begin{itemize}
\item
more complete structuring introduced
\item
|\childdocof| introduced
\item
|\childdoc| renamed to |\childdocmain|
\item
|\childredirect| renamed to |\childdocforward| and |\childdocforwardprefix|
and functionality expanded
\end{itemize}

%%%%%%%%%%%%%%%%%%%%%%%%%%%%%%%%%%%%%%%%
\paragraph{v1.0:} 2017/04/27

\begin{itemize}
\item
manual and install package
\item
first version published on CTAN
\end{itemize}

%%%%%%%%%%%%%%%%%%%%%%%%%%%%%%%%%%%%%%%%
\paragraph{v0.6:} 2017/04/26

\begin{itemize}
\item
redirection mechanism added
\end{itemize}

%%%%%%%%%%%%%%%%%%%%%%%%%%%%%%%%%%%%%%%%
\paragraph{v0.5:} 2017/04/26

\begin{itemize}
\item
functionality in definition file
\end{itemize}


%%%%%%%%%%%%%%%%%%%%%%%%%%%%%%%%%%%%%%%%%%%%%%%%%%%%%%%%%%%%%%%%%%%%%%%%%%%%%%%%
%%%%%%%%%%%%%%%%%%%%%%%%%%%%%%%%%%%%%%%%%%%%%%%%%%%%%%%%%%%%%%%%%%%%%%%%%%%%%%%%
%%%%%%%%%%%%%%%%%%%%%%%%%%%%%%%%%%%%%%%%%%%%%%%%%%%%%%%%%%%%%%%%%%%%%%%%%%%%%%%%
\appendix

\settowidth\MacroIndent{\rmfamily\scriptsize 000\ }

 \DocInput{childdoc.dtx}

\end{document}
%</driver>
% \fi
%
% %%%%%%%%%%%%%%%%%%%%%%%%%%%%%%%%%%%%%%%%%%%%%%%%%%%%%%%%%%%%%%%%%%%%%%%%%%%%%%
% %%%%%%%%%%%%%%%%%%%%%%%%%%%%%%%%%%%%%%%%%%%%%%%%%%%%%%%%%%%%%%%%%%%%%%%%%%%%%%
% \section{Sample}
%\iffalse
%<*samplemain>
%\fi
%
% The following presents a sample document
% with two chapters, two parts, a title page,
% a compile flag as well as three forwarding files to set the flag.
% It consists of eight |.tex| files:
% \begin{center}
% \begin{tabular}{ll}
% |cdocsamp.tex|&main file\\
% |cdocsch1.tex|&include file for chapter 1\\
% |cdocsch2.tex|&include file for chapter 2\\
% |cdocspt3.tex|&include file for part 3\\
% |cdocspt4.tex|&include file for part 4\\
% |cdocsdrf.tex|&forwarding file for main file in draft mode\\
% |cdocsfi1.tex|&forwarding file for final version of chapter 1\\
% |cdocsfi2.tex|&forwarding file for final version of chapter 2\\
% \end{tabular}
% \end{center}
% Each of the eight files can be compiled directly by the \LaTeX{} compiler.
%
% %%%%%%%%%%%%%%%%%%%%%%%%%%%%%%%%%%%%%%
% \paragraph{Main File.}
%
% The main file is called |cdocsamp.tex|.
%
% Load the \textsf{childdoc} definitions and
% declare the filename for the main document:
%    \begin{macrocode}
\input{childdoc.def}
\childdocmain{}
%    \end{macrocode}

% Optional override for |\version| flag:
%    \begin{macrocode}
%%\ifchilddoc\else\providecommand{\version}{draft}\fi
%    \end{macrocode}

% Define the default values for the |\version| flag
% (|final| for the main file and |draft| for childs):
%    \begin{macrocode}
\ifchilddoc
\providecommand{\version}{draft}
\else
\providecommand{\version}{final}
\fi
%    \end{macrocode}

% Load the standard document class:
%    \begin{macrocode}
\documentclass[12pt]{article}
%    \end{macrocode}

% Start the document body:
%    \begin{macrocode}
\begin{document}
%    \end{macrocode}

% Declare a title page.
% Print title, part of document being processed and version flag:
%    \begin{macrocode}
\addtocounter{page}{-1}
\begin{center}
{\LARGE\bfseries{}childdoc example\par}
\vspace{1cm}
\ifchilddoc
\ifchilddocmanual part\else chapter\fi:
`\childdocname' of `\childdocjob'\par
\else
main document: `\childdocjob'\par
\fi
version: \version\par
\end{center}
\newpage
%    \end{macrocode}

% Manually include selected file,
% otherwise process as usual:
%    \begin{macrocode}
\ifchilddocmanual
\section*{part `\childdocname'}
\input{\childdocname}
\else
%    \end{macrocode}

% Include the two chapters:
%    \begin{macrocode}
\include{cdocsch1}
\include{cdocsch2}
%    \end{macrocode}

% Include the two parts unless only chapters should be displayed:
%    \begin{macrocode}
\ifchilddoc\else
\section{part three}
\input{cdocspt3}
\section{part four}
\input{cdocspt4}
\fi
%    \end{macrocode}

% Process as usual until here:
%    \begin{macrocode}
\fi
%    \end{macrocode}

% End of document body:
%    \begin{macrocode}
\end{document}
%    \end{macrocode}
%\iffalse
%</samplemain>
%\fi
%
% %%%%%%%%%%%%%%%%%%%%%%%%%%%%%%%%%%%%%%
% \paragraph{Chapter Include Files.}
%
% The include files are called |cdocsch1.tex| and |cdocsch2.tex|.
%
%\iffalse
%<*samplechap1|samplechap2>
%\fi

% Optional override for |\version| flag:
%    \begin{macrocode}
%%\providecommand{\version}{final}
%    \end{macrocode}

% Include the main document:
%    \begin{macrocode}
\input{childdoc.def}
\childdocof{cdocsamp}
%    \end{macrocode}

%\iffalse
%</samplechap1|samplechap2>
%\fi
%
%\iffalse
%<*samplechap1>
%\fi
% Some text for chapter 1:
%    \begin{macrocode}
\section{one}
some text in chapter one
%    \end{macrocode}

%\iffalse
%</samplechap1>
%\fi
% Some text for chapter 2:
%\iffalse
%<*samplechap2>
%\fi
%    \begin{macrocode}
\section{two}
more text in chapter two
%    \end{macrocode}

%\iffalse
%</samplechap2>
%\fi
%
% %%%%%%%%%%%%%%%%%%%%%%%%%%%%%%%%%%%%%%
% \paragraph{Part Include Files.}
%
% The include files are called |cdocspt3.tex| and |cdocspt4.tex|.
%
%\iffalse
%<*samplepart3|samplepart4>
%\fi

% Optional override for |\version| flag:
%    \begin{macrocode}
%%\providecommand{\version}{final}
%    \end{macrocode}

% Include the main document:
%    \begin{macrocode}
\input{childdoc.def}
\childdocby{cdocsamp}
%    \end{macrocode}

%\iffalse
%</samplepart3|samplepart4>
%\fi
%
%\iffalse
%<*samplepart3>
%\fi
% Some text for part 3:
%    \begin{macrocode}
some text in part three
%    \end{macrocode}

%\iffalse
%</samplepart3>
%\fi
% Some text for part 4:
%\iffalse
%<*samplepart4>
%\fi
%    \begin{macrocode}
more text in part four
%    \end{macrocode}

%\iffalse
%</samplepart4>
%\fi
%
% %%%%%%%%%%%%%%%%%%%%%%%%%%%%%%%%%%%%%%
% \paragraph{Forwarding for a Complete Draft.}
%
% The following forwarding file |cdocsdrf.tex|
% compiles the main document in draft mode:
%\iffalse
%<*sampledraft>
%\fi
%    \begin{macrocode}
\def\version{draft}
\input{childdoc.def}
\childdocforward{cdocsamp}
%    \end{macrocode}

%\iffalse
%</sampledraft>
%\fi
%
% %%%%%%%%%%%%%%%%%%%%%%%%%%%%%%%%%%%%%%
% \paragraph{Forwarding for Final Version of the Chapters.}
%
% The following forwarding files |cdocsfn1.tex| and |cdocsfn2.tex|
% (with identical content)
% compile the final versions of the child documents
% |cdocsch1.tex| and |cdocsch2.tex|, respectively:
%\iffalse
%<*samplefinal>
%\fi
%    \begin{macrocode}
\def\version{final}
\input{childdoc.def}
\childdocforwardprefix[cdocsamp]{cdocsfn}{cdocsch}
%    \end{macrocode}

%\iffalse
%</samplefinal>
%\fi
%
% %%%%%%%%%%%%%%%%%%%%%%%%%%%%%%%%%%%%%%
% \paragraph{Command Line Processing.}
%
% The following three command lines generate the output files
% |cdocscld|, |cdocscl1| and |cdocscl2|
% which should be identical to
% |cdocsdrf|, |cdocsch1| and |cdocsfn2|, respectively:
% \begin{center}
% \begin{tabular}{l}
% |latex -jobname cdocscld \|\\
% |  "\def\version{draft}\input{childdoc.def}\childdocforward{cdocsamp}"|\\
% |latex -jobname cdocscl1 \|\\
% |  "\input{childdoc.def}\childdocforward[cdocsamp]{cdocsch1}"|\\
% |latex -jobname cdocscl2 \|\\
% |  "\def\version{final}\input{childdoc.def}\childdocforward{cdocsch2}"|
% \end{tabular}
% \end{center}
% Note that the trailing backslash on each first line
% merely continues the input to the second line
% (for convenient cut ant paste).
% Furthermore, the command |latex| can be replaced by any
% of its alternative versions such as |pdflatex|.
%
% %%%%%%%%%%%%%%%%%%%%%%%%%%%%%%%%%%%%%%%%%%%%%%%%%%%%%%%%%%%%%%%%%%%%%%%%%%%%%%
% %%%%%%%%%%%%%%%%%%%%%%%%%%%%%%%%%%%%%%%%%%%%%%%%%%%%%%%%%%%%%%%%%%%%%%%%%%%%%%
% \section{Implementation}
%\iffalse
%<*package>
%\fi
%
% This section describes the definitions file |childdoc.def|.

% The definitions cannot be loaded using |\usepackage| or |\RequirePackage|
% which has a mechanism to prevent loading a style file more than once.
% When loading the definitions by means of |\input|
% multiple instances have to be prevented manually:
%\iffalse
%This code needs to be before the `\ProvidesFile' directive
%which is defined at the beginning of this file.
%Therefore it is also placed there and commented out here.
%</package>
%<*discard>
%\fi
%    \begin{macrocode}
\ifdefined\childdocmain\endinput\fi
%    \end{macrocode}
%\iffalse
%</discard>
%<*package>
%\fi
%
% \macro{\ifchilddoc}
% \macro{\ifchilddocmanual}
% The conditional |\ifchilddoc| tells whether a
% child (true) or main (false) document is being compiled.
% The conditional |\ifchilddocmanual| tells whether
% the |\includeonly| mechanism is used (false) or
% the selection of child files must be performed manually (true).
% The definitions initialise to false:
%    \begin{macrocode}
\newif\ifchilddoc
\newif\ifchilddocmanual
%    \end{macrocode}

% \macro{\childdocname}
% \macro{\childdocjob}
% The macro |\childdocname| stores the name of the main document
% to be compiled. The macro |\childdocjob| stores the name of
% the document on which the \LaTeX{} compiler was originally invoked.
% The content of |\jobname| cannot be compared
% to filenames specified in the source due to different catcodes.
% The following code rescans |\jobname|, stores the result
% in |\childdocname| and saves a copy in |\childdocjob|:
%    \begin{macrocode}
\edef\childdocname{\scantokens\expandafter{\jobname\noexpand}}
\let\childdocjob\childdocname
%    \end{macrocode}

% \macro{\childdocdisable}
% The macro |\childdocdisable| prevents the main file
% from being processed more than once.
% At this stage, the main document command |\childdocmain|
% is assumed to be called once again where it should do nothing.
% Any subsequent call to it should prevent
% a secondary processing of the main document
% It overwrites the forwarding commands
% |\childdocof| and |\childdocforward|
% with empty macros to prevent further inclusions of the main document:
%    \begin{macrocode}
\newcommand{\childdocdisable}
{
  \renewcommand{\childdocmain}[1]{\renewcommand{\childdocmain}[1]{\endinput}}
  \renewcommand{\childdocof}[1]{}
  \renewcommand{\childdocby}[2][]{}
  \renewcommand{\childdocforward}[2][]{}
  \renewcommand{\childdocdisable}{}
}
%    \end{macrocode}

% \macro{\childdocmain}
% The macro |\childdocmain| is to be called at the top of the main file
% with nothing or the main filename (without extension) as argument.
% First, it breaks loops.
% If the argument is not empty and does not match |\childdocname|
% (which is set by the first inclusion of |childdoc.def|),
% |\ifchilddoc| is set to true, |\includeonly| is applied to the child file
% and |\jobname| is set to the main file
% (for proper handling of |.aux| files):
%    \begin{macrocode}
\newcommand{\childdocmain}[1]
{
  \childdocdisable\childdocmain{}
  \if?#1?\else
    \begingroup
      \def\childdoctmp{#1}
      \ifx\childdoctmp\childdocname
        \def\childdoctmp{}
      \else
        \def\childdoctmp
        {
          \childdoctrue
          \includeonly{\childdocname}
          \def\childdocjob{#1}
          \def\jobname{#1}
        }
      \fi
      \expandafter
    \endgroup
    \childdoctmp
  \fi
}
%    \end{macrocode}

% \macro{\childdocof}
% The command |\childdocof| redirects
% compilation to the main file |#1|.
%    \begin{macrocode}
\newcommand{\childdocof}[1]
{
  \childdocdisable
  \childdoctrue
  \includeonly{\childdocname}
  \def\jobname{#1}
  \def\childdocjob{#1}
  \input{#1}
}
%    \end{macrocode}

% \macro{\childdocby}
% The command |\childdocby| ....
%    \begin{macrocode}
\newcommand{\childdocby}[2][]
{
  \childdocdisable
  \childdoctrue
  \childdocmanualtrue
  \if?#1?\else
    \def\jobname{#2}
  \fi
  \def\childdocjob{#2}
  \input{#2}
  \endinput
}
%    \end{macrocode}

% \macro{\childdocforward}
% The command |\childdocforward| redirects
% compilation to the main file or
% (if the optional argument is given) a child file.
% Parameters are set as if the main file
% or a child file starting with |\childdocof| was compiled.
% Then compilation is handed over to the main file:
%    \begin{macrocode}
\newcommand{\childdocforward}[2][]
{
  \begingroup
    \if?#1?
      \def\childdoctmp
      {
        \def\childdocname{#2}
        \def\childdocjob{#2}
        \def\jobname{#2}
        \input{#2}
        \endinput
      }
    \else
      \def\childdoctmp
      {
        \childdocdisable
        \def\childdocname{#2}
        \childdoctrue
        \includeonly{#2}
        \def\childdocjob{#1}
        \def\jobname{#1}
        \input{#1}
        \endinput
      }
    \fi
    \expandafter
  \endgroup
  \childdoctmp
}
%    \end{macrocode}

% \macro{\childdocforwardprefix}
% The command |\childdocforwardprefix| redirects
% compilation to the main or a child file by means of a pattern.
% The prefix |#1| in the current filename is replaced by |#2|
% and the suffix of the current filename is kept
% (it is assumed that the filename does not contain the substring `|~~~|'
% which is used as a delimiter).
% Compilation is handed over to the new file by |\childdocforward|:
%    \begin{macrocode}
\newcommand{\childdocforwardprefix}[3][]
{
  \begingroup
    \def\childdocextract #2##1~~~{\def\childdoctmp{\childdocforward[#1]{#3##1}}}
    \expandafter\childdocextract\childdocname~~~
    \expandafter
  \endgroup
  \childdoctmp
}
%    \end{macrocode}

% \macro{\childdoc}
% The deprecated macro |\childdoc| is a legacy version of |\childdocmain|:
%    \begin{macrocode}
\newcommand{\childdoc}{\childdocmain}
%    \end{macrocode}

% \macro{\childdocredirect}
% The deprecated macro |\childdocredirect| is a legacy version
% of |\childdocforward| and |\childdocforwardprefix|:
%    \begin{macrocode}
\newcommand{\childdocredirect}[2][]
{
  \begingroup
    \if?#1?
      \def\childdoctmp{\childdocforward{#2}}
    \else
      \def\childdoctmp{\childdocforwardprefix{#1}{#2}}
    \fi
    \expandafter
  \endgroup
  \childdoctmp
}
%    \end{macrocode}

%\iffalse
%</package>
%\fi
%
\endinput
|\\
|\childdocforwardprefix[|\textit{main}|]{|\textit{prefix}|}{|\textit{dest}|}|
\end{tabular}
\end{center}
%
the destination file is determined by a pattern
depending on the current file:
To make this work, the current file must be called
`{\textit{prefix}\hspace{0.2em}\textit{suffix}}'
with \textit{prefix} matching precisely the argument.
Processing is then passed on to the file
`{\textit{dest}\hspace{0.2em}\textit{suffix}}'.
Surely, the same effect is achieved by
directly specifying the
argument `{\textit{dest}\hspace{0.2em}\textit{suffix}}'
in the first form.
However, that requires to set up a different file
for each child. With the alternative form of the command
all these files can have exactly the same content
which simplifies setting them up and maintaining them.

For example, the following file |draft.tex|
with a compilation flag |\version| as described in \secref{sec:flags}
compiles the main document as a draft:
%
\begin{center}
\begin{tabular}{l}
|\def\version{draft}|\\
|% \iffalse
%
% childdoc.dtx Copyright (C) 2017-2018 Niklas Beisert
%
% This work may be distributed and/or modified under the
% conditions of the LaTeX Project Public License, either version 1.3
% of this license or (at your option) any later version.
% The latest version of this license is in
%   http://www.latex-project.org/lppl.txt
% and version 1.3 or later is part of all distributions of LaTeX
% version 2005/12/01 or later.
%
% This work has the LPPL maintenance status `maintained'.
%
% The Current Maintainer of this work is Niklas Beisert.
%
% This work consists of the files childdoc.dtx and childdoc.ins
% and the derived files childdoc.def and cdocsamp.tex with
% cdocsch1.tex, cdocsch2.tex, cdocsdrf.tex, cdocsfn1.tex, cdocsfn2.tex.
%
%<package>\ifdefined\childdocmain\endinput\fi
%<package>\ProvidesFile{childdoc.def}[2018/12/30 v2.0 child document driver]
%<samplemain>\ProvidesFile{cdocsamp.tex}[2018/12/30 v2.0 sample for childdoc]
%<*driver>
%\ProvidesFile{childdoc.drv}[2018/12/30 v2.0 childdoc reference manual file]
\PassOptionsToClass{10pt,a4paper}{article}
\documentclass{ltxdoc}

\usepackage[margin=35mm]{geometry}
\usepackage{hyperref}
\usepackage{hyperxmp}
\usepackage[usenames]{color}

\hypersetup{colorlinks=true}
\hypersetup{pdfstartview=FitH}
\hypersetup{pdfpagemode=UseNone}
\hypersetup{pdfsource={}}
\hypersetup{pdflang={en-UK}}
\hypersetup{pdfcopyright={Copyright 2017-2018 Niklas Beisert.
  This work may be distributed and/or modified under the
  conditions of the LaTeX Project Public License, either version 1.3
  of this license or (at your option) any later version.}}
\hypersetup{pdflicenseurl={http://www.latex-project.org/lppl.txt}}
\hypersetup{pdfcontactaddress={ETH Zurich, ITP, HIT K,
  Wolfgang-Pauli-Strasse 27}}
\hypersetup{pdfcontactpostcode={8093}}
\hypersetup{pdfcontactcity={Zurich}}
\hypersetup{pdfcontactcountry={Switzerland}}
\hypersetup{pdfcontactemail={nbeisert@itp.phys.ethz.ch}}
\hypersetup{pdfcontacturl={http://people.phys.ethz.ch/\xmptilde nbeisert/}}

\newcommand{\secref}[1]{\hyperref[#1]{section \ref*{#1}}}

\parskip1ex
\parindent0pt
\let\olditemize\itemize
\def\itemize{\olditemize\parskip0pt}

\begin{document}

\title{The \textsf{childdoc} Package}
\hypersetup{pdftitle={The childdoc Package}}
\author{Niklas Beisert\\[2ex]
  Institut f\"ur Theoretische Physik\\
  Eidgen\"ossische Technische Hochschule Z\"urich\\
  Wolfgang-Pauli-Strasse 27, 8093 Z\"urich, Switzerland\\[1ex]
  \href{mailto:nbeisert@itp.phys.ethz.ch}
  {\texttt{nbeisert@itp.phys.ethz.ch}}}
\hypersetup{pdfauthor={Niklas Beisert}}
\hypersetup{pdfsubject={Manual for the LaTeX2e Package childdoc}}
\date{30 December 2018, \textsf{v2.0}}
\maketitle

\begin{abstract}\noindent
\textsf{childdoc} is a \LaTeXe{} package
that enables the direct compilation
of document sections included by |\include|
to individual files.
\end{abstract}

\begingroup
\parskip0ex
\tableofcontents
\endgroup

%%%%%%%%%%%%%%%%%%%%%%%%%%%%%%%%%%%%%%%%%%%%%%%%%%%%%%%%%%%%%%%%%%%%%%%%%%%%%%%%
%%%%%%%%%%%%%%%%%%%%%%%%%%%%%%%%%%%%%%%%%%%%%%%%%%%%%%%%%%%%%%%%%%%%%%%%%%%%%%%%
\section{Introduction}

\LaTeX{} provides a mechanism to structure a large document (such as a book)
into a main file and several child files (containing the chapters)
using the |\include| command.
This mechanism is beneficial for documents
which span hundreds of pages in order to
make the source file(s) more manageable.
Moreover, compilation can be restricted to
selected child files by means of the |\includeonly| command.
The latter feature can be used to reduce the compilation time while editing
(this was significantly more useful in the earlier days of \LaTeX{})
or to generate a smaller document which is easier to navigate.
Another application of |\includeonly| is to generate
documents consisting of selected parts of the complete document.

However, there are a few drawbacks of the plain |\include| mechanism:
\begin{itemize}
\item
The child files cannot be compiled on their own,
they can only be compiled via the main file.
A naive editing environment
(such as a text editor with an option
to have the current file processed by \LaTeX)
may require one to switch to the main file before compiling;
attempting to compile the child file produces errors.
\item
The main file must be modified (each time)
to adjust the |\includeonly| command
to the present needs. This easily leaves the main file in a messy state.
\item
The generated document will always carry the filename
of the main document. This is inconvenient if
several child files are to be compiled and
to be kept for distribution.
\end{itemize}

The present package provides a simple interface
to make child files individually compilable by \LaTeX{}.
Compiling a child file then has the same effect as compiling
the main file with an |\includeonly| command
to select the appropriate child.
Moreover the generated document will carry the name of the child
rather than the main file.
This resolves all three above issues.

This feature is meant to make the editing of books,
thesis documents and lecture notes somewhat more convenient.
However, the package can also be used efficiently for
composing a series of documents (such as exercise sheets)
which are typically distributed individually.
It then assists the author in generating the individual documents
(potentially in different versions)
as well as a document containing the collected series.
Another application is in developing style files
or other kinds of included material
where compilation of the style file could redirect
to a sample or test file.

%%%%%%%%%%%%%%%%%%%%%%%%%%%%%%%%%%%%%%%%%%%%%%%%%%%%%%%%%%%%%%%%%%%%%%%%%%%%%%%%
%%%%%%%%%%%%%%%%%%%%%%%%%%%%%%%%%%%%%%%%%%%%%%%%%%%%%%%%%%%%%%%%%%%%%%%%%%%%%%%%
\section{Usage}

First of all, the package \textsf{childdoc} is \emph{not} a standard
\LaTeXe{} |.sty| style file! Therefore it needs to be invoked in
a non-standard way.

%%%%%%%%%%%%%%%%%%%%%%%%%%%%%%%%%%%%%%%%%%%%%%%%%%%%%%%%%%%%%%%%%%%%%%%%%%%%%%%%
\subsection{Included Files}
\label{sec:include}

%%%%%%%%%%%%%%%%%%%%%%%%%%%%%%%%%%%%%%%%
\DescribeMacro{\childdocmain}
To use the package, add the commands
\begin{center}
\begin{tabular}{l}
|\input{childdoc.def}|\\
|\childdocmain{}|\\
\end{tabular}
\end{center}
at the very top of the main \LaTeX{} file,
in particular \emph{before} the |\documentclass| statement!
The argument of |\childdocmain| should be left empty
(but it must be present).

%%%%%%%%%%%%%%%%%%%%%%%%%%%%%%%%%%%%%%%%
\DescribeMacro{\childdocof}
Furthermore, add the commands
\begin{center}
\begin{tabular}{l}
|\input{childdoc.def}|\\
|\childdocof{|\textit{main}|}|\\
\end{tabular}
\end{center}
at the top of every child file \textit{child}
which is included by |\include{|\textit{child}|}|
from within the main file
(or at least for those files to be compiled individually).
The argument \textit{main} must be the filename of the main file.

There are a couple of
considerations in setting up the main and child documents:

%%%%%%%%%%%%%%%%%%%%%%%%%%%%%%%%%%%%%%%%
\paragraph{Restrictions.}

Please note the following restrictions:
\begin{itemize}
\item
|\childdocmain| must be called with one argument \textit{main}
to ensure compatibility with earlier version of the package.
It must either be empty (|\childdocmain{}|)
or precisely match the filename of the main file in which it is specified.
See \secref{sec:detection} for further information.
\item
The filename \textit{main} must be specified without the |.tex| extension.
\item
The filename \textit{main} is case sensitive
(even in case-insensitive file systems)
due to internal string comparison.
\item
The argument \textit{main} should be fully expanded, it cannot be a macro.
\item
Subdirectories and special characters should be avoided in filenames.
\item
The command |\childdocmain{|\textit{main}|}| must be followed by a whitespace.
It should not be followed immediately by another command
or by a comment mark `|%|'.
This is because the \TeX{} parser reads the token immediately following
the argument of |\childdocmain| and puts it
at the beginning of every child section;
however, a white\-space is ignored.
\end{itemize}

%%%%%%%%%%%%%%%%%%%%%%%%%%%%%%%%%%%%%%%%
\paragraph{Content of Main File.}

It is advisable to place all content in the child files included by |\include|.
Any output contained in the main file will appear in all child documents
unless suppressed manually;
it cannot be suppressed automatically by the |\includeonly| directive
and thus should normally be avoided.
A method to include some content in the main file
by means of conditional processing is described in \secref{sec:conditional}.

%%%%%%%%%%%%%%%%%%%%%%%%%%%%%%%%%%%%%%%%
\paragraph{Page Numbering.}

When only a part of the document is compiled,
the appropriate numbering of pages
(as well as other status parameters)
is determined from the |.aux| files.
The latter contain information from previous passes.
However this information needs to propagate through
all intermediate child documents.
Therefore the page numbering in child documents may well
be inconsistent until the complete document is compiled at least once.

A useful (if unconventional) way to always ensure a consistent
page numbering is to restart the numbering in each child document
and denote the pages by `\textit{child}|.|\textit{page}'
where \textit{child} represents the chapter/section number of the child file.
This can be achieved by the command
|\numberwithin{page}{|\textit{child}|}|
of the \textsf{amsmath} package
where \textit{child} can be |chapter| or |section|
depending on the chosen structuring.
Alternatively, one can modify the macro |\thepage| appropriately
and reset the counter |page| at the start of each child file.

%%%%%%%%%%%%%%%%%%%%%%%%%%%%%%%%%%%%%%%%%%%%%%%%%%%%%%%%%%%%%%%%%%%%%%%%%%%%%%%%
\subsection{Conditional Processing}
\label{sec:conditional}

The package provides a mechanism to compile different versions
of a document. To customise the versions further some conditional processing
can come in handy to distinguish which version is being compiled.
The package provides two macros to describe the compilation context:

%%%%%%%%%%%%%%%%%%%%%%%%%%%%%%%%%%%%%%%%
\DescribeMacro{\ifchilddoc}
The conditional |\ifchilddoc| distinguishes between the compilation of
child documents and the main document:
%
\begin{center}
|\ifchilddoc |\textit{child-code}| |[|\||else |\textit{main-code}]| \||fi|
\end{center}

%%%%%%%%%%%%%%%%%%%%%%%%%%%%%%%%%%%%%%%%
\DescribeMacro{\childdocname}
\DescribeMacro{\childdocjob}
The macro |\childdocname| contains the filename (without extension)
of the main or child file being processed.
Note that |\childdocjob| will always contain the name of the main file.

%%%%%%%%%%%%%%%%%%%%%%%%%%%%%%%%%%%%%%%%
\paragraph{Title Page.}

Conditional processing can be used to include a title or banner page
in the main document when proper precautions are taken.
Importantly, the code in the main file should ensure that the page counter
(as well as other status parameters which are stored in the |.aux| files)
takes the same value after the conditional processing.
Otherwise the page numbers may take divergent values
depending on which part is compiled.

For example, a title page could be declared by:
%
\begin{center}
\begin{tabular}{l}
|\ifchilddoc\||else|\\
|\addtocounter{page}{-1}|\\
\textit{code for title page}\\
|\newpage|\\
|\||fi|
\end{tabular}
\end{center}
%
A banner page for the child documents can be generated by:
%
\begin{center}
\begin{tabular}{l}
|\ifchilddoc|\\
|\addtocounter{page}{-1}|\\
\textit{code for banner page}\\
|\newpage|\\
|\||fi|
\end{tabular}
\end{center}
%
Here one could write a message such as:
\begin{center}
|This is the part \childdocname{} of \childdocjob{}.|
\end{center}

%%%%%%%%%%%%%%%%%%%%%%%%%%%%%%%%%%%%%%%%%%%%%%%%%%%%%%%%%%%%%%%%%%%%%%%%%%%%%%%%
\subsection{Flags}
\label{sec:flags}

The package makes it easy to generate different versions
of the main or child documents.
To this end compilation flags can be defined
and assigned different default values.
They will be particularly useful in conjunction
with the forwarding mechanism described in \secref{sec:forward}.

For example, it may be useful to have a flag |\version|
which can be set to |draft| or |final|.
The document source will contain some conditional code
depending on the value of |\version|.
Suppose further, the flag should default to |final| for the main file
and to |draft| for child files
which is a natural assignment for editing the document.
This is achieved by placing the following code
in the preamble of the main document
(below the |\childdocmain| directive):
%
\begin{center}
\begin{tabular}{l}
|\ifchilddoc|\\
|\providecommand{\version}{draft}|\\
|\||else|\\
|\providecommand{\version}{final}|\\
|\||fi|
\end{tabular}
\end{center}
%
The definition by |\providecommand| makes sure
that previous definitions are not overwritten.
Further statements |\providecommand{\version}{...}|
can thus be added before the above code to override it.

For the main file, one might add a line
(between |\childdocmain| and the above block)
%
\begin{center}
|%\ifchilddoc\||else\providecommand{\version}{draft}\||fi|
\end{center}
%
which can be uncommented to produce a draft version.
Likewise one can add a line to the very top of a child file
(above the |\childdocof{|\textit{main}|}| directive)
%
\begin{center}
|%\providecommand{\version}{final}|
\end{center}
%
which can be uncommented to produce the final version of this child document.

%%%%%%%%%%%%%%%%%%%%%%%%%%%%%%%%%%%%%%%%%%%%%%%%%%%%%%%%%%%%%%%%%%%%%%%%%%%%%%%%
\subsection{Forwarding}
\label{sec:forward}

Different versions of the main or child documents
using compilation flags as described in \secref{sec:flags}
can be (permanently) stored in different files
for convenient compilation, viewing and distribution.
To this end, the package defines a command
to pass on compilation to a different file:

%%%%%%%%%%%%%%%%%%%%%%%%%%%%%%%%%%%%%%%%
\DescribeMacro{\childdocforward}
The command |\childdocforward| redirects processing to
another source file:
%
\begin{center}
\begin{tabular}{l}
|\input{childdoc.def}|\\
|\childdocforward[|\textit{main}|]{|\textit{dest}|}|\\
\end{tabular}
\end{center}
%
The argument \textit{dest} is the destination file
(without extension).
It should be the main file or one of the child files.
Note that further \textsf{childdoc} directives
such as |\childdocof| and |\childdocforward|
in the indicated file will be processed in this form.
The optional argument \textit{main}
passes on directly to the main file \textit{main}
while pretending to compile the child \textit{dest}.
This form behaves as if \textit{dest}
issues |\childdocof{|\textit{main}|}| right away,
and no further \textsf{childdoc} directives will be processed.

%%%%%%%%%%%%%%%%%%%%%%%%%%%%%%%%%%%%%%%%
\DescribeMacro{\...prefix}
In the alternative form |\childdocforwardprefix|,
%
\begin{center}
\begin{tabular}{l}
|\input{childdoc.def}|\\
|\childdocforwardprefix[|\textit{main}|]{|\textit{prefix}|}{|\textit{dest}|}|
\end{tabular}
\end{center}
%
the destination file is determined by a pattern
depending on the current file:
To make this work, the current file must be called
`{\textit{prefix}\hspace{0.2em}\textit{suffix}}'
with \textit{prefix} matching precisely the argument.
Processing is then passed on to the file
`{\textit{dest}\hspace{0.2em}\textit{suffix}}'.
Surely, the same effect is achieved by
directly specifying the
argument `{\textit{dest}\hspace{0.2em}\textit{suffix}}'
in the first form.
However, that requires to set up a different file
for each child. With the alternative form of the command
all these files can have exactly the same content
which simplifies setting them up and maintaining them.

For example, the following file |draft.tex|
with a compilation flag |\version| as described in \secref{sec:flags}
compiles the main document as a draft:
%
\begin{center}
\begin{tabular}{l}
|\def\version{draft}|\\
|\input{childdoc.def}|\\
|\childdocforward{|\textit{main}|}|
\end{tabular}
\end{center}
%
Likewise, the following files |final|\textit{nn}|.tex|
compile the final version of the child document
|child|\textit{nn}|.tex|:
%
\begin{center}
\begin{tabular}{l}
|\def\version{final}|\\
|\input{childdoc.def}|\\
|\childdocforwardprefix{final}{child}|
\end{tabular}
\end{center}
%

Note that when several versions of a main file and/or of each child file
are to be generated, it may be convenient to set up a |Makefile| or
shell script to automatise the process.

%%%%%%%%%%%%%%%%%%%%%%%%%%%%%%%%%%%%%%%%%%%%%%%%%%%%%%%%%%%%%%%%%%%%%%%%%%%%%%%%
\subsection{Command Line Processing}
\label{sec:commandline}

The effect of redirection files can also be achieved by invoking
the \LaTeX{} compiler with a more elaborate command line.
Most conveniently this should be done as part
of a shell script or a |Makefile|.

When using \textsf{childdoc} in the main file, the following
command lines effectively perform a redirection
(note that depending on the shell being used,
backslashes may have to be doubled: `|\|' $\to$ `|\\|'):
%
\begin{center}
|... -jobname "|\textit{target}|" |\\|"|[\textit{flags}]%
|\input{childdoc.def}\childdocforward[|\textit{main}|]{|\textit{dest}|}"|
\end{center}
%
Here \textit{target} is the name of the output file,
\textit{main} is the name of the main file
and \textit{dest} is the name of the main or child file to be processed
(all filenames without extensions).
The optional argument \textit{main} can be omitted
if \textit{main} matches \textit{dest}.
Optionally, compilation \textit{flags} can be defined via |\def| commands.
This command line makes the \TeX{} engine believe
it is compiling the file \textit{target}
whose content is specified as the latter parameter.
The provided code then forwards the processing to
\textit{main} or \textit{dest} as described in \secref{sec:forward}.

%%%%%%%%%%%%%%%%%%%%%%%%%%%%%%%%%%%%%%%%%%%%%%%%%%%%%%%%%%%%%%%%%%%%%%%%%%%%%%%%
\subsection{Include by Input}
\label{sec:input}

Including child documents by |\include| has some restrictions by design.
Most notably, the content of a child document always occupies
its own set of pages; pages cannot be shared between child documents.
Usually, this behaviour makes perfect sense
because each child document contain an essential part of the document.
However, in some situations it may be desirable to compose
a document from a collection of parts
without having mandatory page breaks between then.
For this case, the package
provides a mechanism to include parts
by |\input| which can also be processed individually.
However, by construction this mechanism
requires manual handling of the content to be output.

%%%%%%%%%%%%%%%%%%%%%%%%%%%%%%%%%%%%%%%%
\DescribeMacro{\ifchilddocmanual}
The main file should be prepared as usual, see \secref{sec:include}.
However, the document body must make a distinction
between processing of an individual part and of the main document, e.g.:
%
\begin{center}
\begin{tabular}{l}
|\ifchilddocmanual|\\
|\input{\childdocname}|\\
|\||else|\\
\textit{document body with }|\input{|\textit{part}|}|\\
|\||fi|
\end{tabular}
\end{center}
%
The conditional |\ifchilddocmanual| is true whenever
a part to be included by |\input| is being compiled,
and the name of the part is stored in |\childdocname|.

%%%%%%%%%%%%%%%%%%%%%%%%%%%%%%%%%%%%%%%%
\DescribeMacro{\childdocby}
Each part to be included by |\input| should start with:
%
\begin{center}
\begin{tabular}{l}
|\input{childdoc.def}|\\
|\childdocby{|\textit{main}|}|\\
\end{tabular}
\end{center}
%
The directive |\childdocby| is similar to |\childdocof|
described in \secref{sec:include},
but the subsequent selection of content must be done manually.
To that end, both |\ifchilddoc| and |\ifchilddocmanual|
will be true upon processing of a part,
and the name of the part is stored in |\childdocname|.
Note that |\jobname| will be set to the filename of the current part
so that each part receives an individual |.aux| file
that does not interfere with the |.aux| file(s) of the main document.
This behaviour can be altered by the alternative form
|\childdocby[*]{|\textit{main}|}| (with a non-empty optional argument)
which uses the |.aux| file of the main document
by setting |\jobname| to \textit{main}.

%%%%%%%%%%%%%%%%%%%%%%%%%%%%%%%%%%%%%%%%%%%%%%%%%%%%%%%%%%%%%%%%%%%%%%%%%%%%%%%%
\subsection{Driver Development}
\label{sec:driver}

The \textsf{childdoc} mechanism can also be use for the development
of definition files such as \LaTeX{} styles or classes.
This case differs from the above setup with multiple parts
included by |\include| in that no |\includeonly| should be invoked.
This can be achieved by starting the include file
(before |\ProvidesPackage|) with:
%
\begin{center}
\begin{tabular}{l}
|\input{childdoc.def}|\\
|\childdocforward{|\textit{main}|}|\\
\end{tabular}
\end{center}
%
or alternatively with:
%
\begin{center}
\begin{tabular}{l}
|\input{childdoc.def}|\\
|\childdocby{|\textit{main}|}|\\
\end{tabular}
\end{center}
%
Both forms have slightly different effects as described above.
The main file is prepared as usual, see \secref{sec:include}.

%%%%%%%%%%%%%%%%%%%%%%%%%%%%%%%%%%%%%%%%%%%%%%%%%%%%%%%%%%%%%%%%%%%%%%%%%%%%%%%%
\subsection{Legacy Detection}
\label{sec:detection}

The directive |\childdocmain| in the main file can detect
whether the complete document or merely a child is to be compiled
even without using the directive |\childdocof|.
This method is deprecated because it is less robust
and there is no compelling reason to use it;
it is merely provided for backward compatibility
and it may be removed in future versions.

If the detection mechanism is to be used,
it is mandatory to correctly specify
the filename of the main file as the argument of |\childdocmain|:
%
\begin{center}
\begin{tabular}{l}
|\input{childdoc.def}|\\
|\childdocmain{|\textit{main}|}|\\
\end{tabular}
\end{center}
%
If |\jobname| does not match the argument \textit{main} of |\childdocmain|,
it is assumed that |\jobname| points to the child file to be compiled.
When using |\childdocmain| with the main file specified as argument,
it suffices to start a child file
with just |\input{|\textit{main}|}|
without loading of the package and using |\childdocof|.
If instead all processing is done
with the appropriate \textsf{childdoc} directives,
the argument of \textit{main} of |\childdocmain| can be empty.

An alternative version of the command line processing described
in \secref{sec:commandline} using the detection mechanism reads:
%
\begin{center}
|... -jobname "|\textit{target}|" "|[\textit{flags}]%
[|\def\jobname{|\textit{dest}|}|]|\input{|\textit{main}|}"|
\end{center}

%%%%%%%%%%%%%%%%%%%%%%%%%%%%%%%%%%%%%%%%%%%%%%%%%%%%%%%%%%%%%%%%%%%%%%%%%%%%%%%%
\subsection{Manual Code}
\label{sec:manual}

In case one cannot be certain whether the definitions file |childdoc.def|
is installed on the target \TeX{} distribution
and one prefers not to ship it,
it is conceivable to paste a few relevant commands into the sources.

To that end, drop all statements |\input{childdoc.def}|
and perform the replacements as outlined below.
Instead of |\childdocmain{|\textit{main}|}| add the following code
to the top of the main file:
%
\begin{center}
\begin{tabular}{l}
|\||ifdefined\childdocname\endinput\||fi\newif\ifchilddoc|\\
|\edef\childdocname{\scantokens\expandafter{\jobname\noexpand}}|\\
|\def\childdocmain{|\textit{main}|}\||ifx\childdocmain\childdocname\||else|\\
|\childdoctrue\includeonly{\childdocname}\let\jobname\childdocmain\||fi|\\
\end{tabular}
\end{center}
%
Instead of |\childdocof{|\textit{main}|}| just include the main file
at the top of each child file:
%
\begin{center}
|\input{|\textit{main}|}|
\end{center}
%
A simple redirection |\childdocforward{|\textit{dest}|}| is achieved by:
%
\begin{center}
|\def\jobname{|\textit{dest}|}\input{\jobname}|
\end{center}
%
The redirection with prefix
|\childdocforwardprefix[|\textit{prefix}|]{|\textit{dest}|}|
is accomplished by:
%
\begin{center}
\begin{tabular}{l}
|{\edef\jobname{\scantokens\expandafter{\jobname\noexpand}}|\\
|\def\redirectjob |\textit{prefix}|#1~~~{\gdef\jobname{|\textit{dest}|#1}}|\\
|\expandafter\redirectjob\jobname~~~}\input{\jobname}|
\end{tabular}
\end{center}

In an alternative approach,
child documents can be compiled by a specific command line
without additional code or specific definitions:
%
\begin{center}
|... -jobname "|\textit{target}|" "|[\textit{flags}]%
|\includeonly{|\textit{dest}|}\input{|\textit{main}|}"|
\end{center}
%

%%%%%%%%%%%%%%%%%%%%%%%%%%%%%%%%%%%%%%%%%%%%%%%%%%%%%%%%%%%%%%%%%%%%%%%%%%%%%%%%
%%%%%%%%%%%%%%%%%%%%%%%%%%%%%%%%%%%%%%%%%%%%%%%%%%%%%%%%%%%%%%%%%%%%%%%%%%%%%%%%
\section{Information}

%%%%%%%%%%%%%%%%%%%%%%%%%%%%%%%%%%%%%%%%%%%%%%%%%%%%%%%%%%%%%%%%%%%%%%%%%%%%%%%%
\subsection{Copyright}

Copyright \copyright{} 2017--2018 Niklas Beisert

This work may be distributed and/or modified under the
conditions of the \LaTeX{} Project Public License, either version 1.3
of this license or (at your option) any later version.
The latest version of this license is in
  \url{http://www.latex-project.org/lppl.txt}
and version 1.3 or later is part of all distributions of \LaTeX{}
version 2005/12/01 or later.

This work has the LPPL maintenance status `maintained'.

The Current Maintainer of this work is Niklas Beisert.

This work consists of the files |README.txt|, |childdoc.ins| and |childdoc.dtx|
as well as the derived files |childdoc.def|, |cdocsamp.tex|
with |cdocsch1.tex|, |cdocsch2.tex|, |cdocspt3.tex|, |cdocspt4.tex|,
|cdocsdrf.tex|, |cdocsfn1.tex|, |cdocsfn2.tex|
as well as |childdoc.pdf|.

%%%%%%%%%%%%%%%%%%%%%%%%%%%%%%%%%%%%%%%%%%%%%%%%%%%%%%%%%%%%%%%%%%%%%%%%%%%%%%%%
\subsection{Files and Installation}

The package consists of the files:
%
\begin{center}
\begin{tabular}{ll}
    |README.txt|   & readme file \\
    |childdoc.ins| & installation file \\
    |childdoc.dtx| & source file \\
    |childdoc.def| & definition file \\
    |cdocsamp.tex| & sample main file \\
    |cdocsch1.tex| & sample include file \\
    |cdocsch2.tex| & sample include file \\
    |cdocspt3.tex| & sample part file \\
    |cdocspt4.tex| & sample part file \\
    |cdocsdrf.tex| & sample redirection file \\
    |cdocsfn1.tex| & sample redirection file \\
    |cdocsfn2.tex| & sample redirection file \\
    |childdoc.pdf| & manual
\end{tabular}
\end{center}
%
The distribution consists of the files
|README.txt|, |childdoc.ins| and |childdoc.dtx|.
%
\begin{itemize}
\item
Run (pdf)\LaTeX{} on |childdoc.dtx|
to compile the manual |childdoc.pdf| (this file).
\item
Run \LaTeX{} on |childdoc.ins| to create the definitions file |childdoc.def|
and the sample |cdocsamp.tex| with include files
|cdocsch1.tex|, |cdocsch2.tex|, |cdocspt3.tex|, |cdocspt4.tex|,
|cdocsdrf.tex|, |cdocsfn1.tex|, |cdocsfn2.tex|.
Then copy the file |childdoc.def| to an appropriate directory of your \LaTeX{}
distribution, e.g.\ \textit{texmf-root}|/tex/latex/childdoc|.
\end{itemize}

%%%%%%%%%%%%%%%%%%%%%%%%%%%%%%%%%%%%%%%%%%%%%%%%%%%%%%%%%%%%%%%%%%%%%%%%%%%%%%%%
\subsection{Related CTAN Packages}

There are several other packages which offer a similar functionality:
%
\begin{itemize}
\item
The packages
\href{http://ctan.org/pkg/docmute}{\textsf{docmute}},
\href{http://ctan.org/pkg/includex}{\textsf{includex}} and
\href{http://ctan.org/pkg/standalone}{\textsf{standalone}}
provide commands to include only the document body of
a child file thus allowing both files to be compiled individually.
\item
The packages \href{http://ctan.org/pkg/subdocs}{\textsf{subdocs}}
and \href{http://ctan.org/pkg/subfiles}{\textsf{subfiles}}
provide structures in which the main and child documents can be
encapsulated and allowing them to be compiled individually.
The inclusion mechanism is different from the conventional |\include|.
\item
The package \href{http://ctan.org/pkg/combine}{\textsf{combine}}
is an elaborate solution to combine several documents into one.
\end{itemize}
%
See also the CTAN topic \href{http://ctan.org/topic/subdocs}{\textsf{subdocs}}
for further related packages.
The present package differs from the above solutions in that
a document structure constructed with the conventional |\include| mechanism
just needs two extra commands at the top of every file
such that all constituent files can be compiled individually.

%%%%%%%%%%%%%%%%%%%%%%%%%%%%%%%%%%%%%%%%%%%%%%%%%%%%%%%%%%%%%%%%%%%%%%%%%%%%%%%%
%\subsection{Feature Suggestions}
%
%The following is a list of features which may be useful for future
%versions of this package:
%%
%\begin{itemize}
%\item
%\ldots
%\end{itemize}

%%%%%%%%%%%%%%%%%%%%%%%%%%%%%%%%%%%%%%%%%%%%%%%%%%%%%%%%%%%%%%%%%%%%%%%%%%%%%%%%
\subsection{Revision History}

%%%%%%%%%%%%%%%%%%%%%%%%%%%%%%%%%%%%%%%%
\paragraph{v2.0:} 2018/12/30

\begin{itemize}
\item
immediate forward processing
\item
added |\childdocby| mechanism
\item
manual restructured
\end{itemize}

%%%%%%%%%%%%%%%%%%%%%%%%%%%%%%%%%%%%%%%%
\paragraph{v1.6:} 2018/01/17

\begin{itemize}
\item
application for development of include files
\item
corrections to manual
\end{itemize}

%%%%%%%%%%%%%%%%%%%%%%%%%%%%%%%%%%%%%%%%
\paragraph{v1.5:} 2017/05/21

\begin{itemize}
\item
more complete structuring introduced
\item
|\childdocof| introduced
\item
|\childdoc| renamed to |\childdocmain|
\item
|\childredirect| renamed to |\childdocforward| and |\childdocforwardprefix|
and functionality expanded
\end{itemize}

%%%%%%%%%%%%%%%%%%%%%%%%%%%%%%%%%%%%%%%%
\paragraph{v1.0:} 2017/04/27

\begin{itemize}
\item
manual and install package
\item
first version published on CTAN
\end{itemize}

%%%%%%%%%%%%%%%%%%%%%%%%%%%%%%%%%%%%%%%%
\paragraph{v0.6:} 2017/04/26

\begin{itemize}
\item
redirection mechanism added
\end{itemize}

%%%%%%%%%%%%%%%%%%%%%%%%%%%%%%%%%%%%%%%%
\paragraph{v0.5:} 2017/04/26

\begin{itemize}
\item
functionality in definition file
\end{itemize}


%%%%%%%%%%%%%%%%%%%%%%%%%%%%%%%%%%%%%%%%%%%%%%%%%%%%%%%%%%%%%%%%%%%%%%%%%%%%%%%%
%%%%%%%%%%%%%%%%%%%%%%%%%%%%%%%%%%%%%%%%%%%%%%%%%%%%%%%%%%%%%%%%%%%%%%%%%%%%%%%%
%%%%%%%%%%%%%%%%%%%%%%%%%%%%%%%%%%%%%%%%%%%%%%%%%%%%%%%%%%%%%%%%%%%%%%%%%%%%%%%%
\appendix

\settowidth\MacroIndent{\rmfamily\scriptsize 000\ }

 \DocInput{childdoc.dtx}

\end{document}
%</driver>
% \fi
%
% %%%%%%%%%%%%%%%%%%%%%%%%%%%%%%%%%%%%%%%%%%%%%%%%%%%%%%%%%%%%%%%%%%%%%%%%%%%%%%
% %%%%%%%%%%%%%%%%%%%%%%%%%%%%%%%%%%%%%%%%%%%%%%%%%%%%%%%%%%%%%%%%%%%%%%%%%%%%%%
% \section{Sample}
%\iffalse
%<*samplemain>
%\fi
%
% The following presents a sample document
% with two chapters, two parts, a title page,
% a compile flag as well as three forwarding files to set the flag.
% It consists of eight |.tex| files:
% \begin{center}
% \begin{tabular}{ll}
% |cdocsamp.tex|&main file\\
% |cdocsch1.tex|&include file for chapter 1\\
% |cdocsch2.tex|&include file for chapter 2\\
% |cdocspt3.tex|&include file for part 3\\
% |cdocspt4.tex|&include file for part 4\\
% |cdocsdrf.tex|&forwarding file for main file in draft mode\\
% |cdocsfi1.tex|&forwarding file for final version of chapter 1\\
% |cdocsfi2.tex|&forwarding file for final version of chapter 2\\
% \end{tabular}
% \end{center}
% Each of the eight files can be compiled directly by the \LaTeX{} compiler.
%
% %%%%%%%%%%%%%%%%%%%%%%%%%%%%%%%%%%%%%%
% \paragraph{Main File.}
%
% The main file is called |cdocsamp.tex|.
%
% Load the \textsf{childdoc} definitions and
% declare the filename for the main document:
%    \begin{macrocode}
\input{childdoc.def}
\childdocmain{}
%    \end{macrocode}

% Optional override for |\version| flag:
%    \begin{macrocode}
%%\ifchilddoc\else\providecommand{\version}{draft}\fi
%    \end{macrocode}

% Define the default values for the |\version| flag
% (|final| for the main file and |draft| for childs):
%    \begin{macrocode}
\ifchilddoc
\providecommand{\version}{draft}
\else
\providecommand{\version}{final}
\fi
%    \end{macrocode}

% Load the standard document class:
%    \begin{macrocode}
\documentclass[12pt]{article}
%    \end{macrocode}

% Start the document body:
%    \begin{macrocode}
\begin{document}
%    \end{macrocode}

% Declare a title page.
% Print title, part of document being processed and version flag:
%    \begin{macrocode}
\addtocounter{page}{-1}
\begin{center}
{\LARGE\bfseries{}childdoc example\par}
\vspace{1cm}
\ifchilddoc
\ifchilddocmanual part\else chapter\fi:
`\childdocname' of `\childdocjob'\par
\else
main document: `\childdocjob'\par
\fi
version: \version\par
\end{center}
\newpage
%    \end{macrocode}

% Manually include selected file,
% otherwise process as usual:
%    \begin{macrocode}
\ifchilddocmanual
\section*{part `\childdocname'}
\input{\childdocname}
\else
%    \end{macrocode}

% Include the two chapters:
%    \begin{macrocode}
\include{cdocsch1}
\include{cdocsch2}
%    \end{macrocode}

% Include the two parts unless only chapters should be displayed:
%    \begin{macrocode}
\ifchilddoc\else
\section{part three}
\input{cdocspt3}
\section{part four}
\input{cdocspt4}
\fi
%    \end{macrocode}

% Process as usual until here:
%    \begin{macrocode}
\fi
%    \end{macrocode}

% End of document body:
%    \begin{macrocode}
\end{document}
%    \end{macrocode}
%\iffalse
%</samplemain>
%\fi
%
% %%%%%%%%%%%%%%%%%%%%%%%%%%%%%%%%%%%%%%
% \paragraph{Chapter Include Files.}
%
% The include files are called |cdocsch1.tex| and |cdocsch2.tex|.
%
%\iffalse
%<*samplechap1|samplechap2>
%\fi

% Optional override for |\version| flag:
%    \begin{macrocode}
%%\providecommand{\version}{final}
%    \end{macrocode}

% Include the main document:
%    \begin{macrocode}
\input{childdoc.def}
\childdocof{cdocsamp}
%    \end{macrocode}

%\iffalse
%</samplechap1|samplechap2>
%\fi
%
%\iffalse
%<*samplechap1>
%\fi
% Some text for chapter 1:
%    \begin{macrocode}
\section{one}
some text in chapter one
%    \end{macrocode}

%\iffalse
%</samplechap1>
%\fi
% Some text for chapter 2:
%\iffalse
%<*samplechap2>
%\fi
%    \begin{macrocode}
\section{two}
more text in chapter two
%    \end{macrocode}

%\iffalse
%</samplechap2>
%\fi
%
% %%%%%%%%%%%%%%%%%%%%%%%%%%%%%%%%%%%%%%
% \paragraph{Part Include Files.}
%
% The include files are called |cdocspt3.tex| and |cdocspt4.tex|.
%
%\iffalse
%<*samplepart3|samplepart4>
%\fi

% Optional override for |\version| flag:
%    \begin{macrocode}
%%\providecommand{\version}{final}
%    \end{macrocode}

% Include the main document:
%    \begin{macrocode}
\input{childdoc.def}
\childdocby{cdocsamp}
%    \end{macrocode}

%\iffalse
%</samplepart3|samplepart4>
%\fi
%
%\iffalse
%<*samplepart3>
%\fi
% Some text for part 3:
%    \begin{macrocode}
some text in part three
%    \end{macrocode}

%\iffalse
%</samplepart3>
%\fi
% Some text for part 4:
%\iffalse
%<*samplepart4>
%\fi
%    \begin{macrocode}
more text in part four
%    \end{macrocode}

%\iffalse
%</samplepart4>
%\fi
%
% %%%%%%%%%%%%%%%%%%%%%%%%%%%%%%%%%%%%%%
% \paragraph{Forwarding for a Complete Draft.}
%
% The following forwarding file |cdocsdrf.tex|
% compiles the main document in draft mode:
%\iffalse
%<*sampledraft>
%\fi
%    \begin{macrocode}
\def\version{draft}
\input{childdoc.def}
\childdocforward{cdocsamp}
%    \end{macrocode}

%\iffalse
%</sampledraft>
%\fi
%
% %%%%%%%%%%%%%%%%%%%%%%%%%%%%%%%%%%%%%%
% \paragraph{Forwarding for Final Version of the Chapters.}
%
% The following forwarding files |cdocsfn1.tex| and |cdocsfn2.tex|
% (with identical content)
% compile the final versions of the child documents
% |cdocsch1.tex| and |cdocsch2.tex|, respectively:
%\iffalse
%<*samplefinal>
%\fi
%    \begin{macrocode}
\def\version{final}
\input{childdoc.def}
\childdocforwardprefix[cdocsamp]{cdocsfn}{cdocsch}
%    \end{macrocode}

%\iffalse
%</samplefinal>
%\fi
%
% %%%%%%%%%%%%%%%%%%%%%%%%%%%%%%%%%%%%%%
% \paragraph{Command Line Processing.}
%
% The following three command lines generate the output files
% |cdocscld|, |cdocscl1| and |cdocscl2|
% which should be identical to
% |cdocsdrf|, |cdocsch1| and |cdocsfn2|, respectively:
% \begin{center}
% \begin{tabular}{l}
% |latex -jobname cdocscld \|\\
% |  "\def\version{draft}\input{childdoc.def}\childdocforward{cdocsamp}"|\\
% |latex -jobname cdocscl1 \|\\
% |  "\input{childdoc.def}\childdocforward[cdocsamp]{cdocsch1}"|\\
% |latex -jobname cdocscl2 \|\\
% |  "\def\version{final}\input{childdoc.def}\childdocforward{cdocsch2}"|
% \end{tabular}
% \end{center}
% Note that the trailing backslash on each first line
% merely continues the input to the second line
% (for convenient cut ant paste).
% Furthermore, the command |latex| can be replaced by any
% of its alternative versions such as |pdflatex|.
%
% %%%%%%%%%%%%%%%%%%%%%%%%%%%%%%%%%%%%%%%%%%%%%%%%%%%%%%%%%%%%%%%%%%%%%%%%%%%%%%
% %%%%%%%%%%%%%%%%%%%%%%%%%%%%%%%%%%%%%%%%%%%%%%%%%%%%%%%%%%%%%%%%%%%%%%%%%%%%%%
% \section{Implementation}
%\iffalse
%<*package>
%\fi
%
% This section describes the definitions file |childdoc.def|.

% The definitions cannot be loaded using |\usepackage| or |\RequirePackage|
% which has a mechanism to prevent loading a style file more than once.
% When loading the definitions by means of |\input|
% multiple instances have to be prevented manually:
%\iffalse
%This code needs to be before the `\ProvidesFile' directive
%which is defined at the beginning of this file.
%Therefore it is also placed there and commented out here.
%</package>
%<*discard>
%\fi
%    \begin{macrocode}
\ifdefined\childdocmain\endinput\fi
%    \end{macrocode}
%\iffalse
%</discard>
%<*package>
%\fi
%
% \macro{\ifchilddoc}
% \macro{\ifchilddocmanual}
% The conditional |\ifchilddoc| tells whether a
% child (true) or main (false) document is being compiled.
% The conditional |\ifchilddocmanual| tells whether
% the |\includeonly| mechanism is used (false) or
% the selection of child files must be performed manually (true).
% The definitions initialise to false:
%    \begin{macrocode}
\newif\ifchilddoc
\newif\ifchilddocmanual
%    \end{macrocode}

% \macro{\childdocname}
% \macro{\childdocjob}
% The macro |\childdocname| stores the name of the main document
% to be compiled. The macro |\childdocjob| stores the name of
% the document on which the \LaTeX{} compiler was originally invoked.
% The content of |\jobname| cannot be compared
% to filenames specified in the source due to different catcodes.
% The following code rescans |\jobname|, stores the result
% in |\childdocname| and saves a copy in |\childdocjob|:
%    \begin{macrocode}
\edef\childdocname{\scantokens\expandafter{\jobname\noexpand}}
\let\childdocjob\childdocname
%    \end{macrocode}

% \macro{\childdocdisable}
% The macro |\childdocdisable| prevents the main file
% from being processed more than once.
% At this stage, the main document command |\childdocmain|
% is assumed to be called once again where it should do nothing.
% Any subsequent call to it should prevent
% a secondary processing of the main document
% It overwrites the forwarding commands
% |\childdocof| and |\childdocforward|
% with empty macros to prevent further inclusions of the main document:
%    \begin{macrocode}
\newcommand{\childdocdisable}
{
  \renewcommand{\childdocmain}[1]{\renewcommand{\childdocmain}[1]{\endinput}}
  \renewcommand{\childdocof}[1]{}
  \renewcommand{\childdocby}[2][]{}
  \renewcommand{\childdocforward}[2][]{}
  \renewcommand{\childdocdisable}{}
}
%    \end{macrocode}

% \macro{\childdocmain}
% The macro |\childdocmain| is to be called at the top of the main file
% with nothing or the main filename (without extension) as argument.
% First, it breaks loops.
% If the argument is not empty and does not match |\childdocname|
% (which is set by the first inclusion of |childdoc.def|),
% |\ifchilddoc| is set to true, |\includeonly| is applied to the child file
% and |\jobname| is set to the main file
% (for proper handling of |.aux| files):
%    \begin{macrocode}
\newcommand{\childdocmain}[1]
{
  \childdocdisable\childdocmain{}
  \if?#1?\else
    \begingroup
      \def\childdoctmp{#1}
      \ifx\childdoctmp\childdocname
        \def\childdoctmp{}
      \else
        \def\childdoctmp
        {
          \childdoctrue
          \includeonly{\childdocname}
          \def\childdocjob{#1}
          \def\jobname{#1}
        }
      \fi
      \expandafter
    \endgroup
    \childdoctmp
  \fi
}
%    \end{macrocode}

% \macro{\childdocof}
% The command |\childdocof| redirects
% compilation to the main file |#1|.
%    \begin{macrocode}
\newcommand{\childdocof}[1]
{
  \childdocdisable
  \childdoctrue
  \includeonly{\childdocname}
  \def\jobname{#1}
  \def\childdocjob{#1}
  \input{#1}
}
%    \end{macrocode}

% \macro{\childdocby}
% The command |\childdocby| ....
%    \begin{macrocode}
\newcommand{\childdocby}[2][]
{
  \childdocdisable
  \childdoctrue
  \childdocmanualtrue
  \if?#1?\else
    \def\jobname{#2}
  \fi
  \def\childdocjob{#2}
  \input{#2}
  \endinput
}
%    \end{macrocode}

% \macro{\childdocforward}
% The command |\childdocforward| redirects
% compilation to the main file or
% (if the optional argument is given) a child file.
% Parameters are set as if the main file
% or a child file starting with |\childdocof| was compiled.
% Then compilation is handed over to the main file:
%    \begin{macrocode}
\newcommand{\childdocforward}[2][]
{
  \begingroup
    \if?#1?
      \def\childdoctmp
      {
        \def\childdocname{#2}
        \def\childdocjob{#2}
        \def\jobname{#2}
        \input{#2}
        \endinput
      }
    \else
      \def\childdoctmp
      {
        \childdocdisable
        \def\childdocname{#2}
        \childdoctrue
        \includeonly{#2}
        \def\childdocjob{#1}
        \def\jobname{#1}
        \input{#1}
        \endinput
      }
    \fi
    \expandafter
  \endgroup
  \childdoctmp
}
%    \end{macrocode}

% \macro{\childdocforwardprefix}
% The command |\childdocforwardprefix| redirects
% compilation to the main or a child file by means of a pattern.
% The prefix |#1| in the current filename is replaced by |#2|
% and the suffix of the current filename is kept
% (it is assumed that the filename does not contain the substring `|~~~|'
% which is used as a delimiter).
% Compilation is handed over to the new file by |\childdocforward|:
%    \begin{macrocode}
\newcommand{\childdocforwardprefix}[3][]
{
  \begingroup
    \def\childdocextract #2##1~~~{\def\childdoctmp{\childdocforward[#1]{#3##1}}}
    \expandafter\childdocextract\childdocname~~~
    \expandafter
  \endgroup
  \childdoctmp
}
%    \end{macrocode}

% \macro{\childdoc}
% The deprecated macro |\childdoc| is a legacy version of |\childdocmain|:
%    \begin{macrocode}
\newcommand{\childdoc}{\childdocmain}
%    \end{macrocode}

% \macro{\childdocredirect}
% The deprecated macro |\childdocredirect| is a legacy version
% of |\childdocforward| and |\childdocforwardprefix|:
%    \begin{macrocode}
\newcommand{\childdocredirect}[2][]
{
  \begingroup
    \if?#1?
      \def\childdoctmp{\childdocforward{#2}}
    \else
      \def\childdoctmp{\childdocforwardprefix{#1}{#2}}
    \fi
    \expandafter
  \endgroup
  \childdoctmp
}
%    \end{macrocode}

%\iffalse
%</package>
%\fi
%
\endinput
|\\
|\childdocforward{|\textit{main}|}|
\end{tabular}
\end{center}
%
Likewise, the following files |final|\textit{nn}|.tex|
compile the final version of the child document
|child|\textit{nn}|.tex|:
%
\begin{center}
\begin{tabular}{l}
|\def\version{final}|\\
|% \iffalse
%
% childdoc.dtx Copyright (C) 2017-2018 Niklas Beisert
%
% This work may be distributed and/or modified under the
% conditions of the LaTeX Project Public License, either version 1.3
% of this license or (at your option) any later version.
% The latest version of this license is in
%   http://www.latex-project.org/lppl.txt
% and version 1.3 or later is part of all distributions of LaTeX
% version 2005/12/01 or later.
%
% This work has the LPPL maintenance status `maintained'.
%
% The Current Maintainer of this work is Niklas Beisert.
%
% This work consists of the files childdoc.dtx and childdoc.ins
% and the derived files childdoc.def and cdocsamp.tex with
% cdocsch1.tex, cdocsch2.tex, cdocsdrf.tex, cdocsfn1.tex, cdocsfn2.tex.
%
%<package>\ifdefined\childdocmain\endinput\fi
%<package>\ProvidesFile{childdoc.def}[2018/12/30 v2.0 child document driver]
%<samplemain>\ProvidesFile{cdocsamp.tex}[2018/12/30 v2.0 sample for childdoc]
%<*driver>
%\ProvidesFile{childdoc.drv}[2018/12/30 v2.0 childdoc reference manual file]
\PassOptionsToClass{10pt,a4paper}{article}
\documentclass{ltxdoc}

\usepackage[margin=35mm]{geometry}
\usepackage{hyperref}
\usepackage{hyperxmp}
\usepackage[usenames]{color}

\hypersetup{colorlinks=true}
\hypersetup{pdfstartview=FitH}
\hypersetup{pdfpagemode=UseNone}
\hypersetup{pdfsource={}}
\hypersetup{pdflang={en-UK}}
\hypersetup{pdfcopyright={Copyright 2017-2018 Niklas Beisert.
  This work may be distributed and/or modified under the
  conditions of the LaTeX Project Public License, either version 1.3
  of this license or (at your option) any later version.}}
\hypersetup{pdflicenseurl={http://www.latex-project.org/lppl.txt}}
\hypersetup{pdfcontactaddress={ETH Zurich, ITP, HIT K,
  Wolfgang-Pauli-Strasse 27}}
\hypersetup{pdfcontactpostcode={8093}}
\hypersetup{pdfcontactcity={Zurich}}
\hypersetup{pdfcontactcountry={Switzerland}}
\hypersetup{pdfcontactemail={nbeisert@itp.phys.ethz.ch}}
\hypersetup{pdfcontacturl={http://people.phys.ethz.ch/\xmptilde nbeisert/}}

\newcommand{\secref}[1]{\hyperref[#1]{section \ref*{#1}}}

\parskip1ex
\parindent0pt
\let\olditemize\itemize
\def\itemize{\olditemize\parskip0pt}

\begin{document}

\title{The \textsf{childdoc} Package}
\hypersetup{pdftitle={The childdoc Package}}
\author{Niklas Beisert\\[2ex]
  Institut f\"ur Theoretische Physik\\
  Eidgen\"ossische Technische Hochschule Z\"urich\\
  Wolfgang-Pauli-Strasse 27, 8093 Z\"urich, Switzerland\\[1ex]
  \href{mailto:nbeisert@itp.phys.ethz.ch}
  {\texttt{nbeisert@itp.phys.ethz.ch}}}
\hypersetup{pdfauthor={Niklas Beisert}}
\hypersetup{pdfsubject={Manual for the LaTeX2e Package childdoc}}
\date{30 December 2018, \textsf{v2.0}}
\maketitle

\begin{abstract}\noindent
\textsf{childdoc} is a \LaTeXe{} package
that enables the direct compilation
of document sections included by |\include|
to individual files.
\end{abstract}

\begingroup
\parskip0ex
\tableofcontents
\endgroup

%%%%%%%%%%%%%%%%%%%%%%%%%%%%%%%%%%%%%%%%%%%%%%%%%%%%%%%%%%%%%%%%%%%%%%%%%%%%%%%%
%%%%%%%%%%%%%%%%%%%%%%%%%%%%%%%%%%%%%%%%%%%%%%%%%%%%%%%%%%%%%%%%%%%%%%%%%%%%%%%%
\section{Introduction}

\LaTeX{} provides a mechanism to structure a large document (such as a book)
into a main file and several child files (containing the chapters)
using the |\include| command.
This mechanism is beneficial for documents
which span hundreds of pages in order to
make the source file(s) more manageable.
Moreover, compilation can be restricted to
selected child files by means of the |\includeonly| command.
The latter feature can be used to reduce the compilation time while editing
(this was significantly more useful in the earlier days of \LaTeX{})
or to generate a smaller document which is easier to navigate.
Another application of |\includeonly| is to generate
documents consisting of selected parts of the complete document.

However, there are a few drawbacks of the plain |\include| mechanism:
\begin{itemize}
\item
The child files cannot be compiled on their own,
they can only be compiled via the main file.
A naive editing environment
(such as a text editor with an option
to have the current file processed by \LaTeX)
may require one to switch to the main file before compiling;
attempting to compile the child file produces errors.
\item
The main file must be modified (each time)
to adjust the |\includeonly| command
to the present needs. This easily leaves the main file in a messy state.
\item
The generated document will always carry the filename
of the main document. This is inconvenient if
several child files are to be compiled and
to be kept for distribution.
\end{itemize}

The present package provides a simple interface
to make child files individually compilable by \LaTeX{}.
Compiling a child file then has the same effect as compiling
the main file with an |\includeonly| command
to select the appropriate child.
Moreover the generated document will carry the name of the child
rather than the main file.
This resolves all three above issues.

This feature is meant to make the editing of books,
thesis documents and lecture notes somewhat more convenient.
However, the package can also be used efficiently for
composing a series of documents (such as exercise sheets)
which are typically distributed individually.
It then assists the author in generating the individual documents
(potentially in different versions)
as well as a document containing the collected series.
Another application is in developing style files
or other kinds of included material
where compilation of the style file could redirect
to a sample or test file.

%%%%%%%%%%%%%%%%%%%%%%%%%%%%%%%%%%%%%%%%%%%%%%%%%%%%%%%%%%%%%%%%%%%%%%%%%%%%%%%%
%%%%%%%%%%%%%%%%%%%%%%%%%%%%%%%%%%%%%%%%%%%%%%%%%%%%%%%%%%%%%%%%%%%%%%%%%%%%%%%%
\section{Usage}

First of all, the package \textsf{childdoc} is \emph{not} a standard
\LaTeXe{} |.sty| style file! Therefore it needs to be invoked in
a non-standard way.

%%%%%%%%%%%%%%%%%%%%%%%%%%%%%%%%%%%%%%%%%%%%%%%%%%%%%%%%%%%%%%%%%%%%%%%%%%%%%%%%
\subsection{Included Files}
\label{sec:include}

%%%%%%%%%%%%%%%%%%%%%%%%%%%%%%%%%%%%%%%%
\DescribeMacro{\childdocmain}
To use the package, add the commands
\begin{center}
\begin{tabular}{l}
|\input{childdoc.def}|\\
|\childdocmain{}|\\
\end{tabular}
\end{center}
at the very top of the main \LaTeX{} file,
in particular \emph{before} the |\documentclass| statement!
The argument of |\childdocmain| should be left empty
(but it must be present).

%%%%%%%%%%%%%%%%%%%%%%%%%%%%%%%%%%%%%%%%
\DescribeMacro{\childdocof}
Furthermore, add the commands
\begin{center}
\begin{tabular}{l}
|\input{childdoc.def}|\\
|\childdocof{|\textit{main}|}|\\
\end{tabular}
\end{center}
at the top of every child file \textit{child}
which is included by |\include{|\textit{child}|}|
from within the main file
(or at least for those files to be compiled individually).
The argument \textit{main} must be the filename of the main file.

There are a couple of
considerations in setting up the main and child documents:

%%%%%%%%%%%%%%%%%%%%%%%%%%%%%%%%%%%%%%%%
\paragraph{Restrictions.}

Please note the following restrictions:
\begin{itemize}
\item
|\childdocmain| must be called with one argument \textit{main}
to ensure compatibility with earlier version of the package.
It must either be empty (|\childdocmain{}|)
or precisely match the filename of the main file in which it is specified.
See \secref{sec:detection} for further information.
\item
The filename \textit{main} must be specified without the |.tex| extension.
\item
The filename \textit{main} is case sensitive
(even in case-insensitive file systems)
due to internal string comparison.
\item
The argument \textit{main} should be fully expanded, it cannot be a macro.
\item
Subdirectories and special characters should be avoided in filenames.
\item
The command |\childdocmain{|\textit{main}|}| must be followed by a whitespace.
It should not be followed immediately by another command
or by a comment mark `|%|'.
This is because the \TeX{} parser reads the token immediately following
the argument of |\childdocmain| and puts it
at the beginning of every child section;
however, a white\-space is ignored.
\end{itemize}

%%%%%%%%%%%%%%%%%%%%%%%%%%%%%%%%%%%%%%%%
\paragraph{Content of Main File.}

It is advisable to place all content in the child files included by |\include|.
Any output contained in the main file will appear in all child documents
unless suppressed manually;
it cannot be suppressed automatically by the |\includeonly| directive
and thus should normally be avoided.
A method to include some content in the main file
by means of conditional processing is described in \secref{sec:conditional}.

%%%%%%%%%%%%%%%%%%%%%%%%%%%%%%%%%%%%%%%%
\paragraph{Page Numbering.}

When only a part of the document is compiled,
the appropriate numbering of pages
(as well as other status parameters)
is determined from the |.aux| files.
The latter contain information from previous passes.
However this information needs to propagate through
all intermediate child documents.
Therefore the page numbering in child documents may well
be inconsistent until the complete document is compiled at least once.

A useful (if unconventional) way to always ensure a consistent
page numbering is to restart the numbering in each child document
and denote the pages by `\textit{child}|.|\textit{page}'
where \textit{child} represents the chapter/section number of the child file.
This can be achieved by the command
|\numberwithin{page}{|\textit{child}|}|
of the \textsf{amsmath} package
where \textit{child} can be |chapter| or |section|
depending on the chosen structuring.
Alternatively, one can modify the macro |\thepage| appropriately
and reset the counter |page| at the start of each child file.

%%%%%%%%%%%%%%%%%%%%%%%%%%%%%%%%%%%%%%%%%%%%%%%%%%%%%%%%%%%%%%%%%%%%%%%%%%%%%%%%
\subsection{Conditional Processing}
\label{sec:conditional}

The package provides a mechanism to compile different versions
of a document. To customise the versions further some conditional processing
can come in handy to distinguish which version is being compiled.
The package provides two macros to describe the compilation context:

%%%%%%%%%%%%%%%%%%%%%%%%%%%%%%%%%%%%%%%%
\DescribeMacro{\ifchilddoc}
The conditional |\ifchilddoc| distinguishes between the compilation of
child documents and the main document:
%
\begin{center}
|\ifchilddoc |\textit{child-code}| |[|\||else |\textit{main-code}]| \||fi|
\end{center}

%%%%%%%%%%%%%%%%%%%%%%%%%%%%%%%%%%%%%%%%
\DescribeMacro{\childdocname}
\DescribeMacro{\childdocjob}
The macro |\childdocname| contains the filename (without extension)
of the main or child file being processed.
Note that |\childdocjob| will always contain the name of the main file.

%%%%%%%%%%%%%%%%%%%%%%%%%%%%%%%%%%%%%%%%
\paragraph{Title Page.}

Conditional processing can be used to include a title or banner page
in the main document when proper precautions are taken.
Importantly, the code in the main file should ensure that the page counter
(as well as other status parameters which are stored in the |.aux| files)
takes the same value after the conditional processing.
Otherwise the page numbers may take divergent values
depending on which part is compiled.

For example, a title page could be declared by:
%
\begin{center}
\begin{tabular}{l}
|\ifchilddoc\||else|\\
|\addtocounter{page}{-1}|\\
\textit{code for title page}\\
|\newpage|\\
|\||fi|
\end{tabular}
\end{center}
%
A banner page for the child documents can be generated by:
%
\begin{center}
\begin{tabular}{l}
|\ifchilddoc|\\
|\addtocounter{page}{-1}|\\
\textit{code for banner page}\\
|\newpage|\\
|\||fi|
\end{tabular}
\end{center}
%
Here one could write a message such as:
\begin{center}
|This is the part \childdocname{} of \childdocjob{}.|
\end{center}

%%%%%%%%%%%%%%%%%%%%%%%%%%%%%%%%%%%%%%%%%%%%%%%%%%%%%%%%%%%%%%%%%%%%%%%%%%%%%%%%
\subsection{Flags}
\label{sec:flags}

The package makes it easy to generate different versions
of the main or child documents.
To this end compilation flags can be defined
and assigned different default values.
They will be particularly useful in conjunction
with the forwarding mechanism described in \secref{sec:forward}.

For example, it may be useful to have a flag |\version|
which can be set to |draft| or |final|.
The document source will contain some conditional code
depending on the value of |\version|.
Suppose further, the flag should default to |final| for the main file
and to |draft| for child files
which is a natural assignment for editing the document.
This is achieved by placing the following code
in the preamble of the main document
(below the |\childdocmain| directive):
%
\begin{center}
\begin{tabular}{l}
|\ifchilddoc|\\
|\providecommand{\version}{draft}|\\
|\||else|\\
|\providecommand{\version}{final}|\\
|\||fi|
\end{tabular}
\end{center}
%
The definition by |\providecommand| makes sure
that previous definitions are not overwritten.
Further statements |\providecommand{\version}{...}|
can thus be added before the above code to override it.

For the main file, one might add a line
(between |\childdocmain| and the above block)
%
\begin{center}
|%\ifchilddoc\||else\providecommand{\version}{draft}\||fi|
\end{center}
%
which can be uncommented to produce a draft version.
Likewise one can add a line to the very top of a child file
(above the |\childdocof{|\textit{main}|}| directive)
%
\begin{center}
|%\providecommand{\version}{final}|
\end{center}
%
which can be uncommented to produce the final version of this child document.

%%%%%%%%%%%%%%%%%%%%%%%%%%%%%%%%%%%%%%%%%%%%%%%%%%%%%%%%%%%%%%%%%%%%%%%%%%%%%%%%
\subsection{Forwarding}
\label{sec:forward}

Different versions of the main or child documents
using compilation flags as described in \secref{sec:flags}
can be (permanently) stored in different files
for convenient compilation, viewing and distribution.
To this end, the package defines a command
to pass on compilation to a different file:

%%%%%%%%%%%%%%%%%%%%%%%%%%%%%%%%%%%%%%%%
\DescribeMacro{\childdocforward}
The command |\childdocforward| redirects processing to
another source file:
%
\begin{center}
\begin{tabular}{l}
|\input{childdoc.def}|\\
|\childdocforward[|\textit{main}|]{|\textit{dest}|}|\\
\end{tabular}
\end{center}
%
The argument \textit{dest} is the destination file
(without extension).
It should be the main file or one of the child files.
Note that further \textsf{childdoc} directives
such as |\childdocof| and |\childdocforward|
in the indicated file will be processed in this form.
The optional argument \textit{main}
passes on directly to the main file \textit{main}
while pretending to compile the child \textit{dest}.
This form behaves as if \textit{dest}
issues |\childdocof{|\textit{main}|}| right away,
and no further \textsf{childdoc} directives will be processed.

%%%%%%%%%%%%%%%%%%%%%%%%%%%%%%%%%%%%%%%%
\DescribeMacro{\...prefix}
In the alternative form |\childdocforwardprefix|,
%
\begin{center}
\begin{tabular}{l}
|\input{childdoc.def}|\\
|\childdocforwardprefix[|\textit{main}|]{|\textit{prefix}|}{|\textit{dest}|}|
\end{tabular}
\end{center}
%
the destination file is determined by a pattern
depending on the current file:
To make this work, the current file must be called
`{\textit{prefix}\hspace{0.2em}\textit{suffix}}'
with \textit{prefix} matching precisely the argument.
Processing is then passed on to the file
`{\textit{dest}\hspace{0.2em}\textit{suffix}}'.
Surely, the same effect is achieved by
directly specifying the
argument `{\textit{dest}\hspace{0.2em}\textit{suffix}}'
in the first form.
However, that requires to set up a different file
for each child. With the alternative form of the command
all these files can have exactly the same content
which simplifies setting them up and maintaining them.

For example, the following file |draft.tex|
with a compilation flag |\version| as described in \secref{sec:flags}
compiles the main document as a draft:
%
\begin{center}
\begin{tabular}{l}
|\def\version{draft}|\\
|\input{childdoc.def}|\\
|\childdocforward{|\textit{main}|}|
\end{tabular}
\end{center}
%
Likewise, the following files |final|\textit{nn}|.tex|
compile the final version of the child document
|child|\textit{nn}|.tex|:
%
\begin{center}
\begin{tabular}{l}
|\def\version{final}|\\
|\input{childdoc.def}|\\
|\childdocforwardprefix{final}{child}|
\end{tabular}
\end{center}
%

Note that when several versions of a main file and/or of each child file
are to be generated, it may be convenient to set up a |Makefile| or
shell script to automatise the process.

%%%%%%%%%%%%%%%%%%%%%%%%%%%%%%%%%%%%%%%%%%%%%%%%%%%%%%%%%%%%%%%%%%%%%%%%%%%%%%%%
\subsection{Command Line Processing}
\label{sec:commandline}

The effect of redirection files can also be achieved by invoking
the \LaTeX{} compiler with a more elaborate command line.
Most conveniently this should be done as part
of a shell script or a |Makefile|.

When using \textsf{childdoc} in the main file, the following
command lines effectively perform a redirection
(note that depending on the shell being used,
backslashes may have to be doubled: `|\|' $\to$ `|\\|'):
%
\begin{center}
|... -jobname "|\textit{target}|" |\\|"|[\textit{flags}]%
|\input{childdoc.def}\childdocforward[|\textit{main}|]{|\textit{dest}|}"|
\end{center}
%
Here \textit{target} is the name of the output file,
\textit{main} is the name of the main file
and \textit{dest} is the name of the main or child file to be processed
(all filenames without extensions).
The optional argument \textit{main} can be omitted
if \textit{main} matches \textit{dest}.
Optionally, compilation \textit{flags} can be defined via |\def| commands.
This command line makes the \TeX{} engine believe
it is compiling the file \textit{target}
whose content is specified as the latter parameter.
The provided code then forwards the processing to
\textit{main} or \textit{dest} as described in \secref{sec:forward}.

%%%%%%%%%%%%%%%%%%%%%%%%%%%%%%%%%%%%%%%%%%%%%%%%%%%%%%%%%%%%%%%%%%%%%%%%%%%%%%%%
\subsection{Include by Input}
\label{sec:input}

Including child documents by |\include| has some restrictions by design.
Most notably, the content of a child document always occupies
its own set of pages; pages cannot be shared between child documents.
Usually, this behaviour makes perfect sense
because each child document contain an essential part of the document.
However, in some situations it may be desirable to compose
a document from a collection of parts
without having mandatory page breaks between then.
For this case, the package
provides a mechanism to include parts
by |\input| which can also be processed individually.
However, by construction this mechanism
requires manual handling of the content to be output.

%%%%%%%%%%%%%%%%%%%%%%%%%%%%%%%%%%%%%%%%
\DescribeMacro{\ifchilddocmanual}
The main file should be prepared as usual, see \secref{sec:include}.
However, the document body must make a distinction
between processing of an individual part and of the main document, e.g.:
%
\begin{center}
\begin{tabular}{l}
|\ifchilddocmanual|\\
|\input{\childdocname}|\\
|\||else|\\
\textit{document body with }|\input{|\textit{part}|}|\\
|\||fi|
\end{tabular}
\end{center}
%
The conditional |\ifchilddocmanual| is true whenever
a part to be included by |\input| is being compiled,
and the name of the part is stored in |\childdocname|.

%%%%%%%%%%%%%%%%%%%%%%%%%%%%%%%%%%%%%%%%
\DescribeMacro{\childdocby}
Each part to be included by |\input| should start with:
%
\begin{center}
\begin{tabular}{l}
|\input{childdoc.def}|\\
|\childdocby{|\textit{main}|}|\\
\end{tabular}
\end{center}
%
The directive |\childdocby| is similar to |\childdocof|
described in \secref{sec:include},
but the subsequent selection of content must be done manually.
To that end, both |\ifchilddoc| and |\ifchilddocmanual|
will be true upon processing of a part,
and the name of the part is stored in |\childdocname|.
Note that |\jobname| will be set to the filename of the current part
so that each part receives an individual |.aux| file
that does not interfere with the |.aux| file(s) of the main document.
This behaviour can be altered by the alternative form
|\childdocby[*]{|\textit{main}|}| (with a non-empty optional argument)
which uses the |.aux| file of the main document
by setting |\jobname| to \textit{main}.

%%%%%%%%%%%%%%%%%%%%%%%%%%%%%%%%%%%%%%%%%%%%%%%%%%%%%%%%%%%%%%%%%%%%%%%%%%%%%%%%
\subsection{Driver Development}
\label{sec:driver}

The \textsf{childdoc} mechanism can also be use for the development
of definition files such as \LaTeX{} styles or classes.
This case differs from the above setup with multiple parts
included by |\include| in that no |\includeonly| should be invoked.
This can be achieved by starting the include file
(before |\ProvidesPackage|) with:
%
\begin{center}
\begin{tabular}{l}
|\input{childdoc.def}|\\
|\childdocforward{|\textit{main}|}|\\
\end{tabular}
\end{center}
%
or alternatively with:
%
\begin{center}
\begin{tabular}{l}
|\input{childdoc.def}|\\
|\childdocby{|\textit{main}|}|\\
\end{tabular}
\end{center}
%
Both forms have slightly different effects as described above.
The main file is prepared as usual, see \secref{sec:include}.

%%%%%%%%%%%%%%%%%%%%%%%%%%%%%%%%%%%%%%%%%%%%%%%%%%%%%%%%%%%%%%%%%%%%%%%%%%%%%%%%
\subsection{Legacy Detection}
\label{sec:detection}

The directive |\childdocmain| in the main file can detect
whether the complete document or merely a child is to be compiled
even without using the directive |\childdocof|.
This method is deprecated because it is less robust
and there is no compelling reason to use it;
it is merely provided for backward compatibility
and it may be removed in future versions.

If the detection mechanism is to be used,
it is mandatory to correctly specify
the filename of the main file as the argument of |\childdocmain|:
%
\begin{center}
\begin{tabular}{l}
|\input{childdoc.def}|\\
|\childdocmain{|\textit{main}|}|\\
\end{tabular}
\end{center}
%
If |\jobname| does not match the argument \textit{main} of |\childdocmain|,
it is assumed that |\jobname| points to the child file to be compiled.
When using |\childdocmain| with the main file specified as argument,
it suffices to start a child file
with just |\input{|\textit{main}|}|
without loading of the package and using |\childdocof|.
If instead all processing is done
with the appropriate \textsf{childdoc} directives,
the argument of \textit{main} of |\childdocmain| can be empty.

An alternative version of the command line processing described
in \secref{sec:commandline} using the detection mechanism reads:
%
\begin{center}
|... -jobname "|\textit{target}|" "|[\textit{flags}]%
[|\def\jobname{|\textit{dest}|}|]|\input{|\textit{main}|}"|
\end{center}

%%%%%%%%%%%%%%%%%%%%%%%%%%%%%%%%%%%%%%%%%%%%%%%%%%%%%%%%%%%%%%%%%%%%%%%%%%%%%%%%
\subsection{Manual Code}
\label{sec:manual}

In case one cannot be certain whether the definitions file |childdoc.def|
is installed on the target \TeX{} distribution
and one prefers not to ship it,
it is conceivable to paste a few relevant commands into the sources.

To that end, drop all statements |\input{childdoc.def}|
and perform the replacements as outlined below.
Instead of |\childdocmain{|\textit{main}|}| add the following code
to the top of the main file:
%
\begin{center}
\begin{tabular}{l}
|\||ifdefined\childdocname\endinput\||fi\newif\ifchilddoc|\\
|\edef\childdocname{\scantokens\expandafter{\jobname\noexpand}}|\\
|\def\childdocmain{|\textit{main}|}\||ifx\childdocmain\childdocname\||else|\\
|\childdoctrue\includeonly{\childdocname}\let\jobname\childdocmain\||fi|\\
\end{tabular}
\end{center}
%
Instead of |\childdocof{|\textit{main}|}| just include the main file
at the top of each child file:
%
\begin{center}
|\input{|\textit{main}|}|
\end{center}
%
A simple redirection |\childdocforward{|\textit{dest}|}| is achieved by:
%
\begin{center}
|\def\jobname{|\textit{dest}|}\input{\jobname}|
\end{center}
%
The redirection with prefix
|\childdocforwardprefix[|\textit{prefix}|]{|\textit{dest}|}|
is accomplished by:
%
\begin{center}
\begin{tabular}{l}
|{\edef\jobname{\scantokens\expandafter{\jobname\noexpand}}|\\
|\def\redirectjob |\textit{prefix}|#1~~~{\gdef\jobname{|\textit{dest}|#1}}|\\
|\expandafter\redirectjob\jobname~~~}\input{\jobname}|
\end{tabular}
\end{center}

In an alternative approach,
child documents can be compiled by a specific command line
without additional code or specific definitions:
%
\begin{center}
|... -jobname "|\textit{target}|" "|[\textit{flags}]%
|\includeonly{|\textit{dest}|}\input{|\textit{main}|}"|
\end{center}
%

%%%%%%%%%%%%%%%%%%%%%%%%%%%%%%%%%%%%%%%%%%%%%%%%%%%%%%%%%%%%%%%%%%%%%%%%%%%%%%%%
%%%%%%%%%%%%%%%%%%%%%%%%%%%%%%%%%%%%%%%%%%%%%%%%%%%%%%%%%%%%%%%%%%%%%%%%%%%%%%%%
\section{Information}

%%%%%%%%%%%%%%%%%%%%%%%%%%%%%%%%%%%%%%%%%%%%%%%%%%%%%%%%%%%%%%%%%%%%%%%%%%%%%%%%
\subsection{Copyright}

Copyright \copyright{} 2017--2018 Niklas Beisert

This work may be distributed and/or modified under the
conditions of the \LaTeX{} Project Public License, either version 1.3
of this license or (at your option) any later version.
The latest version of this license is in
  \url{http://www.latex-project.org/lppl.txt}
and version 1.3 or later is part of all distributions of \LaTeX{}
version 2005/12/01 or later.

This work has the LPPL maintenance status `maintained'.

The Current Maintainer of this work is Niklas Beisert.

This work consists of the files |README.txt|, |childdoc.ins| and |childdoc.dtx|
as well as the derived files |childdoc.def|, |cdocsamp.tex|
with |cdocsch1.tex|, |cdocsch2.tex|, |cdocspt3.tex|, |cdocspt4.tex|,
|cdocsdrf.tex|, |cdocsfn1.tex|, |cdocsfn2.tex|
as well as |childdoc.pdf|.

%%%%%%%%%%%%%%%%%%%%%%%%%%%%%%%%%%%%%%%%%%%%%%%%%%%%%%%%%%%%%%%%%%%%%%%%%%%%%%%%
\subsection{Files and Installation}

The package consists of the files:
%
\begin{center}
\begin{tabular}{ll}
    |README.txt|   & readme file \\
    |childdoc.ins| & installation file \\
    |childdoc.dtx| & source file \\
    |childdoc.def| & definition file \\
    |cdocsamp.tex| & sample main file \\
    |cdocsch1.tex| & sample include file \\
    |cdocsch2.tex| & sample include file \\
    |cdocspt3.tex| & sample part file \\
    |cdocspt4.tex| & sample part file \\
    |cdocsdrf.tex| & sample redirection file \\
    |cdocsfn1.tex| & sample redirection file \\
    |cdocsfn2.tex| & sample redirection file \\
    |childdoc.pdf| & manual
\end{tabular}
\end{center}
%
The distribution consists of the files
|README.txt|, |childdoc.ins| and |childdoc.dtx|.
%
\begin{itemize}
\item
Run (pdf)\LaTeX{} on |childdoc.dtx|
to compile the manual |childdoc.pdf| (this file).
\item
Run \LaTeX{} on |childdoc.ins| to create the definitions file |childdoc.def|
and the sample |cdocsamp.tex| with include files
|cdocsch1.tex|, |cdocsch2.tex|, |cdocspt3.tex|, |cdocspt4.tex|,
|cdocsdrf.tex|, |cdocsfn1.tex|, |cdocsfn2.tex|.
Then copy the file |childdoc.def| to an appropriate directory of your \LaTeX{}
distribution, e.g.\ \textit{texmf-root}|/tex/latex/childdoc|.
\end{itemize}

%%%%%%%%%%%%%%%%%%%%%%%%%%%%%%%%%%%%%%%%%%%%%%%%%%%%%%%%%%%%%%%%%%%%%%%%%%%%%%%%
\subsection{Related CTAN Packages}

There are several other packages which offer a similar functionality:
%
\begin{itemize}
\item
The packages
\href{http://ctan.org/pkg/docmute}{\textsf{docmute}},
\href{http://ctan.org/pkg/includex}{\textsf{includex}} and
\href{http://ctan.org/pkg/standalone}{\textsf{standalone}}
provide commands to include only the document body of
a child file thus allowing both files to be compiled individually.
\item
The packages \href{http://ctan.org/pkg/subdocs}{\textsf{subdocs}}
and \href{http://ctan.org/pkg/subfiles}{\textsf{subfiles}}
provide structures in which the main and child documents can be
encapsulated and allowing them to be compiled individually.
The inclusion mechanism is different from the conventional |\include|.
\item
The package \href{http://ctan.org/pkg/combine}{\textsf{combine}}
is an elaborate solution to combine several documents into one.
\end{itemize}
%
See also the CTAN topic \href{http://ctan.org/topic/subdocs}{\textsf{subdocs}}
for further related packages.
The present package differs from the above solutions in that
a document structure constructed with the conventional |\include| mechanism
just needs two extra commands at the top of every file
such that all constituent files can be compiled individually.

%%%%%%%%%%%%%%%%%%%%%%%%%%%%%%%%%%%%%%%%%%%%%%%%%%%%%%%%%%%%%%%%%%%%%%%%%%%%%%%%
%\subsection{Feature Suggestions}
%
%The following is a list of features which may be useful for future
%versions of this package:
%%
%\begin{itemize}
%\item
%\ldots
%\end{itemize}

%%%%%%%%%%%%%%%%%%%%%%%%%%%%%%%%%%%%%%%%%%%%%%%%%%%%%%%%%%%%%%%%%%%%%%%%%%%%%%%%
\subsection{Revision History}

%%%%%%%%%%%%%%%%%%%%%%%%%%%%%%%%%%%%%%%%
\paragraph{v2.0:} 2018/12/30

\begin{itemize}
\item
immediate forward processing
\item
added |\childdocby| mechanism
\item
manual restructured
\end{itemize}

%%%%%%%%%%%%%%%%%%%%%%%%%%%%%%%%%%%%%%%%
\paragraph{v1.6:} 2018/01/17

\begin{itemize}
\item
application for development of include files
\item
corrections to manual
\end{itemize}

%%%%%%%%%%%%%%%%%%%%%%%%%%%%%%%%%%%%%%%%
\paragraph{v1.5:} 2017/05/21

\begin{itemize}
\item
more complete structuring introduced
\item
|\childdocof| introduced
\item
|\childdoc| renamed to |\childdocmain|
\item
|\childredirect| renamed to |\childdocforward| and |\childdocforwardprefix|
and functionality expanded
\end{itemize}

%%%%%%%%%%%%%%%%%%%%%%%%%%%%%%%%%%%%%%%%
\paragraph{v1.0:} 2017/04/27

\begin{itemize}
\item
manual and install package
\item
first version published on CTAN
\end{itemize}

%%%%%%%%%%%%%%%%%%%%%%%%%%%%%%%%%%%%%%%%
\paragraph{v0.6:} 2017/04/26

\begin{itemize}
\item
redirection mechanism added
\end{itemize}

%%%%%%%%%%%%%%%%%%%%%%%%%%%%%%%%%%%%%%%%
\paragraph{v0.5:} 2017/04/26

\begin{itemize}
\item
functionality in definition file
\end{itemize}


%%%%%%%%%%%%%%%%%%%%%%%%%%%%%%%%%%%%%%%%%%%%%%%%%%%%%%%%%%%%%%%%%%%%%%%%%%%%%%%%
%%%%%%%%%%%%%%%%%%%%%%%%%%%%%%%%%%%%%%%%%%%%%%%%%%%%%%%%%%%%%%%%%%%%%%%%%%%%%%%%
%%%%%%%%%%%%%%%%%%%%%%%%%%%%%%%%%%%%%%%%%%%%%%%%%%%%%%%%%%%%%%%%%%%%%%%%%%%%%%%%
\appendix

\settowidth\MacroIndent{\rmfamily\scriptsize 000\ }

 \DocInput{childdoc.dtx}

\end{document}
%</driver>
% \fi
%
% %%%%%%%%%%%%%%%%%%%%%%%%%%%%%%%%%%%%%%%%%%%%%%%%%%%%%%%%%%%%%%%%%%%%%%%%%%%%%%
% %%%%%%%%%%%%%%%%%%%%%%%%%%%%%%%%%%%%%%%%%%%%%%%%%%%%%%%%%%%%%%%%%%%%%%%%%%%%%%
% \section{Sample}
%\iffalse
%<*samplemain>
%\fi
%
% The following presents a sample document
% with two chapters, two parts, a title page,
% a compile flag as well as three forwarding files to set the flag.
% It consists of eight |.tex| files:
% \begin{center}
% \begin{tabular}{ll}
% |cdocsamp.tex|&main file\\
% |cdocsch1.tex|&include file for chapter 1\\
% |cdocsch2.tex|&include file for chapter 2\\
% |cdocspt3.tex|&include file for part 3\\
% |cdocspt4.tex|&include file for part 4\\
% |cdocsdrf.tex|&forwarding file for main file in draft mode\\
% |cdocsfi1.tex|&forwarding file for final version of chapter 1\\
% |cdocsfi2.tex|&forwarding file for final version of chapter 2\\
% \end{tabular}
% \end{center}
% Each of the eight files can be compiled directly by the \LaTeX{} compiler.
%
% %%%%%%%%%%%%%%%%%%%%%%%%%%%%%%%%%%%%%%
% \paragraph{Main File.}
%
% The main file is called |cdocsamp.tex|.
%
% Load the \textsf{childdoc} definitions and
% declare the filename for the main document:
%    \begin{macrocode}
\input{childdoc.def}
\childdocmain{}
%    \end{macrocode}

% Optional override for |\version| flag:
%    \begin{macrocode}
%%\ifchilddoc\else\providecommand{\version}{draft}\fi
%    \end{macrocode}

% Define the default values for the |\version| flag
% (|final| for the main file and |draft| for childs):
%    \begin{macrocode}
\ifchilddoc
\providecommand{\version}{draft}
\else
\providecommand{\version}{final}
\fi
%    \end{macrocode}

% Load the standard document class:
%    \begin{macrocode}
\documentclass[12pt]{article}
%    \end{macrocode}

% Start the document body:
%    \begin{macrocode}
\begin{document}
%    \end{macrocode}

% Declare a title page.
% Print title, part of document being processed and version flag:
%    \begin{macrocode}
\addtocounter{page}{-1}
\begin{center}
{\LARGE\bfseries{}childdoc example\par}
\vspace{1cm}
\ifchilddoc
\ifchilddocmanual part\else chapter\fi:
`\childdocname' of `\childdocjob'\par
\else
main document: `\childdocjob'\par
\fi
version: \version\par
\end{center}
\newpage
%    \end{macrocode}

% Manually include selected file,
% otherwise process as usual:
%    \begin{macrocode}
\ifchilddocmanual
\section*{part `\childdocname'}
\input{\childdocname}
\else
%    \end{macrocode}

% Include the two chapters:
%    \begin{macrocode}
\include{cdocsch1}
\include{cdocsch2}
%    \end{macrocode}

% Include the two parts unless only chapters should be displayed:
%    \begin{macrocode}
\ifchilddoc\else
\section{part three}
\input{cdocspt3}
\section{part four}
\input{cdocspt4}
\fi
%    \end{macrocode}

% Process as usual until here:
%    \begin{macrocode}
\fi
%    \end{macrocode}

% End of document body:
%    \begin{macrocode}
\end{document}
%    \end{macrocode}
%\iffalse
%</samplemain>
%\fi
%
% %%%%%%%%%%%%%%%%%%%%%%%%%%%%%%%%%%%%%%
% \paragraph{Chapter Include Files.}
%
% The include files are called |cdocsch1.tex| and |cdocsch2.tex|.
%
%\iffalse
%<*samplechap1|samplechap2>
%\fi

% Optional override for |\version| flag:
%    \begin{macrocode}
%%\providecommand{\version}{final}
%    \end{macrocode}

% Include the main document:
%    \begin{macrocode}
\input{childdoc.def}
\childdocof{cdocsamp}
%    \end{macrocode}

%\iffalse
%</samplechap1|samplechap2>
%\fi
%
%\iffalse
%<*samplechap1>
%\fi
% Some text for chapter 1:
%    \begin{macrocode}
\section{one}
some text in chapter one
%    \end{macrocode}

%\iffalse
%</samplechap1>
%\fi
% Some text for chapter 2:
%\iffalse
%<*samplechap2>
%\fi
%    \begin{macrocode}
\section{two}
more text in chapter two
%    \end{macrocode}

%\iffalse
%</samplechap2>
%\fi
%
% %%%%%%%%%%%%%%%%%%%%%%%%%%%%%%%%%%%%%%
% \paragraph{Part Include Files.}
%
% The include files are called |cdocspt3.tex| and |cdocspt4.tex|.
%
%\iffalse
%<*samplepart3|samplepart4>
%\fi

% Optional override for |\version| flag:
%    \begin{macrocode}
%%\providecommand{\version}{final}
%    \end{macrocode}

% Include the main document:
%    \begin{macrocode}
\input{childdoc.def}
\childdocby{cdocsamp}
%    \end{macrocode}

%\iffalse
%</samplepart3|samplepart4>
%\fi
%
%\iffalse
%<*samplepart3>
%\fi
% Some text for part 3:
%    \begin{macrocode}
some text in part three
%    \end{macrocode}

%\iffalse
%</samplepart3>
%\fi
% Some text for part 4:
%\iffalse
%<*samplepart4>
%\fi
%    \begin{macrocode}
more text in part four
%    \end{macrocode}

%\iffalse
%</samplepart4>
%\fi
%
% %%%%%%%%%%%%%%%%%%%%%%%%%%%%%%%%%%%%%%
% \paragraph{Forwarding for a Complete Draft.}
%
% The following forwarding file |cdocsdrf.tex|
% compiles the main document in draft mode:
%\iffalse
%<*sampledraft>
%\fi
%    \begin{macrocode}
\def\version{draft}
\input{childdoc.def}
\childdocforward{cdocsamp}
%    \end{macrocode}

%\iffalse
%</sampledraft>
%\fi
%
% %%%%%%%%%%%%%%%%%%%%%%%%%%%%%%%%%%%%%%
% \paragraph{Forwarding for Final Version of the Chapters.}
%
% The following forwarding files |cdocsfn1.tex| and |cdocsfn2.tex|
% (with identical content)
% compile the final versions of the child documents
% |cdocsch1.tex| and |cdocsch2.tex|, respectively:
%\iffalse
%<*samplefinal>
%\fi
%    \begin{macrocode}
\def\version{final}
\input{childdoc.def}
\childdocforwardprefix[cdocsamp]{cdocsfn}{cdocsch}
%    \end{macrocode}

%\iffalse
%</samplefinal>
%\fi
%
% %%%%%%%%%%%%%%%%%%%%%%%%%%%%%%%%%%%%%%
% \paragraph{Command Line Processing.}
%
% The following three command lines generate the output files
% |cdocscld|, |cdocscl1| and |cdocscl2|
% which should be identical to
% |cdocsdrf|, |cdocsch1| and |cdocsfn2|, respectively:
% \begin{center}
% \begin{tabular}{l}
% |latex -jobname cdocscld \|\\
% |  "\def\version{draft}\input{childdoc.def}\childdocforward{cdocsamp}"|\\
% |latex -jobname cdocscl1 \|\\
% |  "\input{childdoc.def}\childdocforward[cdocsamp]{cdocsch1}"|\\
% |latex -jobname cdocscl2 \|\\
% |  "\def\version{final}\input{childdoc.def}\childdocforward{cdocsch2}"|
% \end{tabular}
% \end{center}
% Note that the trailing backslash on each first line
% merely continues the input to the second line
% (for convenient cut ant paste).
% Furthermore, the command |latex| can be replaced by any
% of its alternative versions such as |pdflatex|.
%
% %%%%%%%%%%%%%%%%%%%%%%%%%%%%%%%%%%%%%%%%%%%%%%%%%%%%%%%%%%%%%%%%%%%%%%%%%%%%%%
% %%%%%%%%%%%%%%%%%%%%%%%%%%%%%%%%%%%%%%%%%%%%%%%%%%%%%%%%%%%%%%%%%%%%%%%%%%%%%%
% \section{Implementation}
%\iffalse
%<*package>
%\fi
%
% This section describes the definitions file |childdoc.def|.

% The definitions cannot be loaded using |\usepackage| or |\RequirePackage|
% which has a mechanism to prevent loading a style file more than once.
% When loading the definitions by means of |\input|
% multiple instances have to be prevented manually:
%\iffalse
%This code needs to be before the `\ProvidesFile' directive
%which is defined at the beginning of this file.
%Therefore it is also placed there and commented out here.
%</package>
%<*discard>
%\fi
%    \begin{macrocode}
\ifdefined\childdocmain\endinput\fi
%    \end{macrocode}
%\iffalse
%</discard>
%<*package>
%\fi
%
% \macro{\ifchilddoc}
% \macro{\ifchilddocmanual}
% The conditional |\ifchilddoc| tells whether a
% child (true) or main (false) document is being compiled.
% The conditional |\ifchilddocmanual| tells whether
% the |\includeonly| mechanism is used (false) or
% the selection of child files must be performed manually (true).
% The definitions initialise to false:
%    \begin{macrocode}
\newif\ifchilddoc
\newif\ifchilddocmanual
%    \end{macrocode}

% \macro{\childdocname}
% \macro{\childdocjob}
% The macro |\childdocname| stores the name of the main document
% to be compiled. The macro |\childdocjob| stores the name of
% the document on which the \LaTeX{} compiler was originally invoked.
% The content of |\jobname| cannot be compared
% to filenames specified in the source due to different catcodes.
% The following code rescans |\jobname|, stores the result
% in |\childdocname| and saves a copy in |\childdocjob|:
%    \begin{macrocode}
\edef\childdocname{\scantokens\expandafter{\jobname\noexpand}}
\let\childdocjob\childdocname
%    \end{macrocode}

% \macro{\childdocdisable}
% The macro |\childdocdisable| prevents the main file
% from being processed more than once.
% At this stage, the main document command |\childdocmain|
% is assumed to be called once again where it should do nothing.
% Any subsequent call to it should prevent
% a secondary processing of the main document
% It overwrites the forwarding commands
% |\childdocof| and |\childdocforward|
% with empty macros to prevent further inclusions of the main document:
%    \begin{macrocode}
\newcommand{\childdocdisable}
{
  \renewcommand{\childdocmain}[1]{\renewcommand{\childdocmain}[1]{\endinput}}
  \renewcommand{\childdocof}[1]{}
  \renewcommand{\childdocby}[2][]{}
  \renewcommand{\childdocforward}[2][]{}
  \renewcommand{\childdocdisable}{}
}
%    \end{macrocode}

% \macro{\childdocmain}
% The macro |\childdocmain| is to be called at the top of the main file
% with nothing or the main filename (without extension) as argument.
% First, it breaks loops.
% If the argument is not empty and does not match |\childdocname|
% (which is set by the first inclusion of |childdoc.def|),
% |\ifchilddoc| is set to true, |\includeonly| is applied to the child file
% and |\jobname| is set to the main file
% (for proper handling of |.aux| files):
%    \begin{macrocode}
\newcommand{\childdocmain}[1]
{
  \childdocdisable\childdocmain{}
  \if?#1?\else
    \begingroup
      \def\childdoctmp{#1}
      \ifx\childdoctmp\childdocname
        \def\childdoctmp{}
      \else
        \def\childdoctmp
        {
          \childdoctrue
          \includeonly{\childdocname}
          \def\childdocjob{#1}
          \def\jobname{#1}
        }
      \fi
      \expandafter
    \endgroup
    \childdoctmp
  \fi
}
%    \end{macrocode}

% \macro{\childdocof}
% The command |\childdocof| redirects
% compilation to the main file |#1|.
%    \begin{macrocode}
\newcommand{\childdocof}[1]
{
  \childdocdisable
  \childdoctrue
  \includeonly{\childdocname}
  \def\jobname{#1}
  \def\childdocjob{#1}
  \input{#1}
}
%    \end{macrocode}

% \macro{\childdocby}
% The command |\childdocby| ....
%    \begin{macrocode}
\newcommand{\childdocby}[2][]
{
  \childdocdisable
  \childdoctrue
  \childdocmanualtrue
  \if?#1?\else
    \def\jobname{#2}
  \fi
  \def\childdocjob{#2}
  \input{#2}
  \endinput
}
%    \end{macrocode}

% \macro{\childdocforward}
% The command |\childdocforward| redirects
% compilation to the main file or
% (if the optional argument is given) a child file.
% Parameters are set as if the main file
% or a child file starting with |\childdocof| was compiled.
% Then compilation is handed over to the main file:
%    \begin{macrocode}
\newcommand{\childdocforward}[2][]
{
  \begingroup
    \if?#1?
      \def\childdoctmp
      {
        \def\childdocname{#2}
        \def\childdocjob{#2}
        \def\jobname{#2}
        \input{#2}
        \endinput
      }
    \else
      \def\childdoctmp
      {
        \childdocdisable
        \def\childdocname{#2}
        \childdoctrue
        \includeonly{#2}
        \def\childdocjob{#1}
        \def\jobname{#1}
        \input{#1}
        \endinput
      }
    \fi
    \expandafter
  \endgroup
  \childdoctmp
}
%    \end{macrocode}

% \macro{\childdocforwardprefix}
% The command |\childdocforwardprefix| redirects
% compilation to the main or a child file by means of a pattern.
% The prefix |#1| in the current filename is replaced by |#2|
% and the suffix of the current filename is kept
% (it is assumed that the filename does not contain the substring `|~~~|'
% which is used as a delimiter).
% Compilation is handed over to the new file by |\childdocforward|:
%    \begin{macrocode}
\newcommand{\childdocforwardprefix}[3][]
{
  \begingroup
    \def\childdocextract #2##1~~~{\def\childdoctmp{\childdocforward[#1]{#3##1}}}
    \expandafter\childdocextract\childdocname~~~
    \expandafter
  \endgroup
  \childdoctmp
}
%    \end{macrocode}

% \macro{\childdoc}
% The deprecated macro |\childdoc| is a legacy version of |\childdocmain|:
%    \begin{macrocode}
\newcommand{\childdoc}{\childdocmain}
%    \end{macrocode}

% \macro{\childdocredirect}
% The deprecated macro |\childdocredirect| is a legacy version
% of |\childdocforward| and |\childdocforwardprefix|:
%    \begin{macrocode}
\newcommand{\childdocredirect}[2][]
{
  \begingroup
    \if?#1?
      \def\childdoctmp{\childdocforward{#2}}
    \else
      \def\childdoctmp{\childdocforwardprefix{#1}{#2}}
    \fi
    \expandafter
  \endgroup
  \childdoctmp
}
%    \end{macrocode}

%\iffalse
%</package>
%\fi
%
\endinput
|\\
|\childdocforwardprefix{final}{child}|
\end{tabular}
\end{center}
%

Note that when several versions of a main file and/or of each child file
are to be generated, it may be convenient to set up a |Makefile| or
shell script to automatise the process.

%%%%%%%%%%%%%%%%%%%%%%%%%%%%%%%%%%%%%%%%%%%%%%%%%%%%%%%%%%%%%%%%%%%%%%%%%%%%%%%%
\subsection{Command Line Processing}
\label{sec:commandline}

The effect of redirection files can also be achieved by invoking
the \LaTeX{} compiler with a more elaborate command line.
Most conveniently this should be done as part
of a shell script or a |Makefile|.

When using \textsf{childdoc} in the main file, the following
command lines effectively perform a redirection
(note that depending on the shell being used,
backslashes may have to be doubled: `|\|' $\to$ `|\\|'):
%
\begin{center}
|... -jobname "|\textit{target}|" |\\|"|[\textit{flags}]%
|% \iffalse
%
% childdoc.dtx Copyright (C) 2017-2018 Niklas Beisert
%
% This work may be distributed and/or modified under the
% conditions of the LaTeX Project Public License, either version 1.3
% of this license or (at your option) any later version.
% The latest version of this license is in
%   http://www.latex-project.org/lppl.txt
% and version 1.3 or later is part of all distributions of LaTeX
% version 2005/12/01 or later.
%
% This work has the LPPL maintenance status `maintained'.
%
% The Current Maintainer of this work is Niklas Beisert.
%
% This work consists of the files childdoc.dtx and childdoc.ins
% and the derived files childdoc.def and cdocsamp.tex with
% cdocsch1.tex, cdocsch2.tex, cdocsdrf.tex, cdocsfn1.tex, cdocsfn2.tex.
%
%<package>\ifdefined\childdocmain\endinput\fi
%<package>\ProvidesFile{childdoc.def}[2018/12/30 v2.0 child document driver]
%<samplemain>\ProvidesFile{cdocsamp.tex}[2018/12/30 v2.0 sample for childdoc]
%<*driver>
%\ProvidesFile{childdoc.drv}[2018/12/30 v2.0 childdoc reference manual file]
\PassOptionsToClass{10pt,a4paper}{article}
\documentclass{ltxdoc}

\usepackage[margin=35mm]{geometry}
\usepackage{hyperref}
\usepackage{hyperxmp}
\usepackage[usenames]{color}

\hypersetup{colorlinks=true}
\hypersetup{pdfstartview=FitH}
\hypersetup{pdfpagemode=UseNone}
\hypersetup{pdfsource={}}
\hypersetup{pdflang={en-UK}}
\hypersetup{pdfcopyright={Copyright 2017-2018 Niklas Beisert.
  This work may be distributed and/or modified under the
  conditions of the LaTeX Project Public License, either version 1.3
  of this license or (at your option) any later version.}}
\hypersetup{pdflicenseurl={http://www.latex-project.org/lppl.txt}}
\hypersetup{pdfcontactaddress={ETH Zurich, ITP, HIT K,
  Wolfgang-Pauli-Strasse 27}}
\hypersetup{pdfcontactpostcode={8093}}
\hypersetup{pdfcontactcity={Zurich}}
\hypersetup{pdfcontactcountry={Switzerland}}
\hypersetup{pdfcontactemail={nbeisert@itp.phys.ethz.ch}}
\hypersetup{pdfcontacturl={http://people.phys.ethz.ch/\xmptilde nbeisert/}}

\newcommand{\secref}[1]{\hyperref[#1]{section \ref*{#1}}}

\parskip1ex
\parindent0pt
\let\olditemize\itemize
\def\itemize{\olditemize\parskip0pt}

\begin{document}

\title{The \textsf{childdoc} Package}
\hypersetup{pdftitle={The childdoc Package}}
\author{Niklas Beisert\\[2ex]
  Institut f\"ur Theoretische Physik\\
  Eidgen\"ossische Technische Hochschule Z\"urich\\
  Wolfgang-Pauli-Strasse 27, 8093 Z\"urich, Switzerland\\[1ex]
  \href{mailto:nbeisert@itp.phys.ethz.ch}
  {\texttt{nbeisert@itp.phys.ethz.ch}}}
\hypersetup{pdfauthor={Niklas Beisert}}
\hypersetup{pdfsubject={Manual for the LaTeX2e Package childdoc}}
\date{30 December 2018, \textsf{v2.0}}
\maketitle

\begin{abstract}\noindent
\textsf{childdoc} is a \LaTeXe{} package
that enables the direct compilation
of document sections included by |\include|
to individual files.
\end{abstract}

\begingroup
\parskip0ex
\tableofcontents
\endgroup

%%%%%%%%%%%%%%%%%%%%%%%%%%%%%%%%%%%%%%%%%%%%%%%%%%%%%%%%%%%%%%%%%%%%%%%%%%%%%%%%
%%%%%%%%%%%%%%%%%%%%%%%%%%%%%%%%%%%%%%%%%%%%%%%%%%%%%%%%%%%%%%%%%%%%%%%%%%%%%%%%
\section{Introduction}

\LaTeX{} provides a mechanism to structure a large document (such as a book)
into a main file and several child files (containing the chapters)
using the |\include| command.
This mechanism is beneficial for documents
which span hundreds of pages in order to
make the source file(s) more manageable.
Moreover, compilation can be restricted to
selected child files by means of the |\includeonly| command.
The latter feature can be used to reduce the compilation time while editing
(this was significantly more useful in the earlier days of \LaTeX{})
or to generate a smaller document which is easier to navigate.
Another application of |\includeonly| is to generate
documents consisting of selected parts of the complete document.

However, there are a few drawbacks of the plain |\include| mechanism:
\begin{itemize}
\item
The child files cannot be compiled on their own,
they can only be compiled via the main file.
A naive editing environment
(such as a text editor with an option
to have the current file processed by \LaTeX)
may require one to switch to the main file before compiling;
attempting to compile the child file produces errors.
\item
The main file must be modified (each time)
to adjust the |\includeonly| command
to the present needs. This easily leaves the main file in a messy state.
\item
The generated document will always carry the filename
of the main document. This is inconvenient if
several child files are to be compiled and
to be kept for distribution.
\end{itemize}

The present package provides a simple interface
to make child files individually compilable by \LaTeX{}.
Compiling a child file then has the same effect as compiling
the main file with an |\includeonly| command
to select the appropriate child.
Moreover the generated document will carry the name of the child
rather than the main file.
This resolves all three above issues.

This feature is meant to make the editing of books,
thesis documents and lecture notes somewhat more convenient.
However, the package can also be used efficiently for
composing a series of documents (such as exercise sheets)
which are typically distributed individually.
It then assists the author in generating the individual documents
(potentially in different versions)
as well as a document containing the collected series.
Another application is in developing style files
or other kinds of included material
where compilation of the style file could redirect
to a sample or test file.

%%%%%%%%%%%%%%%%%%%%%%%%%%%%%%%%%%%%%%%%%%%%%%%%%%%%%%%%%%%%%%%%%%%%%%%%%%%%%%%%
%%%%%%%%%%%%%%%%%%%%%%%%%%%%%%%%%%%%%%%%%%%%%%%%%%%%%%%%%%%%%%%%%%%%%%%%%%%%%%%%
\section{Usage}

First of all, the package \textsf{childdoc} is \emph{not} a standard
\LaTeXe{} |.sty| style file! Therefore it needs to be invoked in
a non-standard way.

%%%%%%%%%%%%%%%%%%%%%%%%%%%%%%%%%%%%%%%%%%%%%%%%%%%%%%%%%%%%%%%%%%%%%%%%%%%%%%%%
\subsection{Included Files}
\label{sec:include}

%%%%%%%%%%%%%%%%%%%%%%%%%%%%%%%%%%%%%%%%
\DescribeMacro{\childdocmain}
To use the package, add the commands
\begin{center}
\begin{tabular}{l}
|\input{childdoc.def}|\\
|\childdocmain{}|\\
\end{tabular}
\end{center}
at the very top of the main \LaTeX{} file,
in particular \emph{before} the |\documentclass| statement!
The argument of |\childdocmain| should be left empty
(but it must be present).

%%%%%%%%%%%%%%%%%%%%%%%%%%%%%%%%%%%%%%%%
\DescribeMacro{\childdocof}
Furthermore, add the commands
\begin{center}
\begin{tabular}{l}
|\input{childdoc.def}|\\
|\childdocof{|\textit{main}|}|\\
\end{tabular}
\end{center}
at the top of every child file \textit{child}
which is included by |\include{|\textit{child}|}|
from within the main file
(or at least for those files to be compiled individually).
The argument \textit{main} must be the filename of the main file.

There are a couple of
considerations in setting up the main and child documents:

%%%%%%%%%%%%%%%%%%%%%%%%%%%%%%%%%%%%%%%%
\paragraph{Restrictions.}

Please note the following restrictions:
\begin{itemize}
\item
|\childdocmain| must be called with one argument \textit{main}
to ensure compatibility with earlier version of the package.
It must either be empty (|\childdocmain{}|)
or precisely match the filename of the main file in which it is specified.
See \secref{sec:detection} for further information.
\item
The filename \textit{main} must be specified without the |.tex| extension.
\item
The filename \textit{main} is case sensitive
(even in case-insensitive file systems)
due to internal string comparison.
\item
The argument \textit{main} should be fully expanded, it cannot be a macro.
\item
Subdirectories and special characters should be avoided in filenames.
\item
The command |\childdocmain{|\textit{main}|}| must be followed by a whitespace.
It should not be followed immediately by another command
or by a comment mark `|%|'.
This is because the \TeX{} parser reads the token immediately following
the argument of |\childdocmain| and puts it
at the beginning of every child section;
however, a white\-space is ignored.
\end{itemize}

%%%%%%%%%%%%%%%%%%%%%%%%%%%%%%%%%%%%%%%%
\paragraph{Content of Main File.}

It is advisable to place all content in the child files included by |\include|.
Any output contained in the main file will appear in all child documents
unless suppressed manually;
it cannot be suppressed automatically by the |\includeonly| directive
and thus should normally be avoided.
A method to include some content in the main file
by means of conditional processing is described in \secref{sec:conditional}.

%%%%%%%%%%%%%%%%%%%%%%%%%%%%%%%%%%%%%%%%
\paragraph{Page Numbering.}

When only a part of the document is compiled,
the appropriate numbering of pages
(as well as other status parameters)
is determined from the |.aux| files.
The latter contain information from previous passes.
However this information needs to propagate through
all intermediate child documents.
Therefore the page numbering in child documents may well
be inconsistent until the complete document is compiled at least once.

A useful (if unconventional) way to always ensure a consistent
page numbering is to restart the numbering in each child document
and denote the pages by `\textit{child}|.|\textit{page}'
where \textit{child} represents the chapter/section number of the child file.
This can be achieved by the command
|\numberwithin{page}{|\textit{child}|}|
of the \textsf{amsmath} package
where \textit{child} can be |chapter| or |section|
depending on the chosen structuring.
Alternatively, one can modify the macro |\thepage| appropriately
and reset the counter |page| at the start of each child file.

%%%%%%%%%%%%%%%%%%%%%%%%%%%%%%%%%%%%%%%%%%%%%%%%%%%%%%%%%%%%%%%%%%%%%%%%%%%%%%%%
\subsection{Conditional Processing}
\label{sec:conditional}

The package provides a mechanism to compile different versions
of a document. To customise the versions further some conditional processing
can come in handy to distinguish which version is being compiled.
The package provides two macros to describe the compilation context:

%%%%%%%%%%%%%%%%%%%%%%%%%%%%%%%%%%%%%%%%
\DescribeMacro{\ifchilddoc}
The conditional |\ifchilddoc| distinguishes between the compilation of
child documents and the main document:
%
\begin{center}
|\ifchilddoc |\textit{child-code}| |[|\||else |\textit{main-code}]| \||fi|
\end{center}

%%%%%%%%%%%%%%%%%%%%%%%%%%%%%%%%%%%%%%%%
\DescribeMacro{\childdocname}
\DescribeMacro{\childdocjob}
The macro |\childdocname| contains the filename (without extension)
of the main or child file being processed.
Note that |\childdocjob| will always contain the name of the main file.

%%%%%%%%%%%%%%%%%%%%%%%%%%%%%%%%%%%%%%%%
\paragraph{Title Page.}

Conditional processing can be used to include a title or banner page
in the main document when proper precautions are taken.
Importantly, the code in the main file should ensure that the page counter
(as well as other status parameters which are stored in the |.aux| files)
takes the same value after the conditional processing.
Otherwise the page numbers may take divergent values
depending on which part is compiled.

For example, a title page could be declared by:
%
\begin{center}
\begin{tabular}{l}
|\ifchilddoc\||else|\\
|\addtocounter{page}{-1}|\\
\textit{code for title page}\\
|\newpage|\\
|\||fi|
\end{tabular}
\end{center}
%
A banner page for the child documents can be generated by:
%
\begin{center}
\begin{tabular}{l}
|\ifchilddoc|\\
|\addtocounter{page}{-1}|\\
\textit{code for banner page}\\
|\newpage|\\
|\||fi|
\end{tabular}
\end{center}
%
Here one could write a message such as:
\begin{center}
|This is the part \childdocname{} of \childdocjob{}.|
\end{center}

%%%%%%%%%%%%%%%%%%%%%%%%%%%%%%%%%%%%%%%%%%%%%%%%%%%%%%%%%%%%%%%%%%%%%%%%%%%%%%%%
\subsection{Flags}
\label{sec:flags}

The package makes it easy to generate different versions
of the main or child documents.
To this end compilation flags can be defined
and assigned different default values.
They will be particularly useful in conjunction
with the forwarding mechanism described in \secref{sec:forward}.

For example, it may be useful to have a flag |\version|
which can be set to |draft| or |final|.
The document source will contain some conditional code
depending on the value of |\version|.
Suppose further, the flag should default to |final| for the main file
and to |draft| for child files
which is a natural assignment for editing the document.
This is achieved by placing the following code
in the preamble of the main document
(below the |\childdocmain| directive):
%
\begin{center}
\begin{tabular}{l}
|\ifchilddoc|\\
|\providecommand{\version}{draft}|\\
|\||else|\\
|\providecommand{\version}{final}|\\
|\||fi|
\end{tabular}
\end{center}
%
The definition by |\providecommand| makes sure
that previous definitions are not overwritten.
Further statements |\providecommand{\version}{...}|
can thus be added before the above code to override it.

For the main file, one might add a line
(between |\childdocmain| and the above block)
%
\begin{center}
|%\ifchilddoc\||else\providecommand{\version}{draft}\||fi|
\end{center}
%
which can be uncommented to produce a draft version.
Likewise one can add a line to the very top of a child file
(above the |\childdocof{|\textit{main}|}| directive)
%
\begin{center}
|%\providecommand{\version}{final}|
\end{center}
%
which can be uncommented to produce the final version of this child document.

%%%%%%%%%%%%%%%%%%%%%%%%%%%%%%%%%%%%%%%%%%%%%%%%%%%%%%%%%%%%%%%%%%%%%%%%%%%%%%%%
\subsection{Forwarding}
\label{sec:forward}

Different versions of the main or child documents
using compilation flags as described in \secref{sec:flags}
can be (permanently) stored in different files
for convenient compilation, viewing and distribution.
To this end, the package defines a command
to pass on compilation to a different file:

%%%%%%%%%%%%%%%%%%%%%%%%%%%%%%%%%%%%%%%%
\DescribeMacro{\childdocforward}
The command |\childdocforward| redirects processing to
another source file:
%
\begin{center}
\begin{tabular}{l}
|\input{childdoc.def}|\\
|\childdocforward[|\textit{main}|]{|\textit{dest}|}|\\
\end{tabular}
\end{center}
%
The argument \textit{dest} is the destination file
(without extension).
It should be the main file or one of the child files.
Note that further \textsf{childdoc} directives
such as |\childdocof| and |\childdocforward|
in the indicated file will be processed in this form.
The optional argument \textit{main}
passes on directly to the main file \textit{main}
while pretending to compile the child \textit{dest}.
This form behaves as if \textit{dest}
issues |\childdocof{|\textit{main}|}| right away,
and no further \textsf{childdoc} directives will be processed.

%%%%%%%%%%%%%%%%%%%%%%%%%%%%%%%%%%%%%%%%
\DescribeMacro{\...prefix}
In the alternative form |\childdocforwardprefix|,
%
\begin{center}
\begin{tabular}{l}
|\input{childdoc.def}|\\
|\childdocforwardprefix[|\textit{main}|]{|\textit{prefix}|}{|\textit{dest}|}|
\end{tabular}
\end{center}
%
the destination file is determined by a pattern
depending on the current file:
To make this work, the current file must be called
`{\textit{prefix}\hspace{0.2em}\textit{suffix}}'
with \textit{prefix} matching precisely the argument.
Processing is then passed on to the file
`{\textit{dest}\hspace{0.2em}\textit{suffix}}'.
Surely, the same effect is achieved by
directly specifying the
argument `{\textit{dest}\hspace{0.2em}\textit{suffix}}'
in the first form.
However, that requires to set up a different file
for each child. With the alternative form of the command
all these files can have exactly the same content
which simplifies setting them up and maintaining them.

For example, the following file |draft.tex|
with a compilation flag |\version| as described in \secref{sec:flags}
compiles the main document as a draft:
%
\begin{center}
\begin{tabular}{l}
|\def\version{draft}|\\
|\input{childdoc.def}|\\
|\childdocforward{|\textit{main}|}|
\end{tabular}
\end{center}
%
Likewise, the following files |final|\textit{nn}|.tex|
compile the final version of the child document
|child|\textit{nn}|.tex|:
%
\begin{center}
\begin{tabular}{l}
|\def\version{final}|\\
|\input{childdoc.def}|\\
|\childdocforwardprefix{final}{child}|
\end{tabular}
\end{center}
%

Note that when several versions of a main file and/or of each child file
are to be generated, it may be convenient to set up a |Makefile| or
shell script to automatise the process.

%%%%%%%%%%%%%%%%%%%%%%%%%%%%%%%%%%%%%%%%%%%%%%%%%%%%%%%%%%%%%%%%%%%%%%%%%%%%%%%%
\subsection{Command Line Processing}
\label{sec:commandline}

The effect of redirection files can also be achieved by invoking
the \LaTeX{} compiler with a more elaborate command line.
Most conveniently this should be done as part
of a shell script or a |Makefile|.

When using \textsf{childdoc} in the main file, the following
command lines effectively perform a redirection
(note that depending on the shell being used,
backslashes may have to be doubled: `|\|' $\to$ `|\\|'):
%
\begin{center}
|... -jobname "|\textit{target}|" |\\|"|[\textit{flags}]%
|\input{childdoc.def}\childdocforward[|\textit{main}|]{|\textit{dest}|}"|
\end{center}
%
Here \textit{target} is the name of the output file,
\textit{main} is the name of the main file
and \textit{dest} is the name of the main or child file to be processed
(all filenames without extensions).
The optional argument \textit{main} can be omitted
if \textit{main} matches \textit{dest}.
Optionally, compilation \textit{flags} can be defined via |\def| commands.
This command line makes the \TeX{} engine believe
it is compiling the file \textit{target}
whose content is specified as the latter parameter.
The provided code then forwards the processing to
\textit{main} or \textit{dest} as described in \secref{sec:forward}.

%%%%%%%%%%%%%%%%%%%%%%%%%%%%%%%%%%%%%%%%%%%%%%%%%%%%%%%%%%%%%%%%%%%%%%%%%%%%%%%%
\subsection{Include by Input}
\label{sec:input}

Including child documents by |\include| has some restrictions by design.
Most notably, the content of a child document always occupies
its own set of pages; pages cannot be shared between child documents.
Usually, this behaviour makes perfect sense
because each child document contain an essential part of the document.
However, in some situations it may be desirable to compose
a document from a collection of parts
without having mandatory page breaks between then.
For this case, the package
provides a mechanism to include parts
by |\input| which can also be processed individually.
However, by construction this mechanism
requires manual handling of the content to be output.

%%%%%%%%%%%%%%%%%%%%%%%%%%%%%%%%%%%%%%%%
\DescribeMacro{\ifchilddocmanual}
The main file should be prepared as usual, see \secref{sec:include}.
However, the document body must make a distinction
between processing of an individual part and of the main document, e.g.:
%
\begin{center}
\begin{tabular}{l}
|\ifchilddocmanual|\\
|\input{\childdocname}|\\
|\||else|\\
\textit{document body with }|\input{|\textit{part}|}|\\
|\||fi|
\end{tabular}
\end{center}
%
The conditional |\ifchilddocmanual| is true whenever
a part to be included by |\input| is being compiled,
and the name of the part is stored in |\childdocname|.

%%%%%%%%%%%%%%%%%%%%%%%%%%%%%%%%%%%%%%%%
\DescribeMacro{\childdocby}
Each part to be included by |\input| should start with:
%
\begin{center}
\begin{tabular}{l}
|\input{childdoc.def}|\\
|\childdocby{|\textit{main}|}|\\
\end{tabular}
\end{center}
%
The directive |\childdocby| is similar to |\childdocof|
described in \secref{sec:include},
but the subsequent selection of content must be done manually.
To that end, both |\ifchilddoc| and |\ifchilddocmanual|
will be true upon processing of a part,
and the name of the part is stored in |\childdocname|.
Note that |\jobname| will be set to the filename of the current part
so that each part receives an individual |.aux| file
that does not interfere with the |.aux| file(s) of the main document.
This behaviour can be altered by the alternative form
|\childdocby[*]{|\textit{main}|}| (with a non-empty optional argument)
which uses the |.aux| file of the main document
by setting |\jobname| to \textit{main}.

%%%%%%%%%%%%%%%%%%%%%%%%%%%%%%%%%%%%%%%%%%%%%%%%%%%%%%%%%%%%%%%%%%%%%%%%%%%%%%%%
\subsection{Driver Development}
\label{sec:driver}

The \textsf{childdoc} mechanism can also be use for the development
of definition files such as \LaTeX{} styles or classes.
This case differs from the above setup with multiple parts
included by |\include| in that no |\includeonly| should be invoked.
This can be achieved by starting the include file
(before |\ProvidesPackage|) with:
%
\begin{center}
\begin{tabular}{l}
|\input{childdoc.def}|\\
|\childdocforward{|\textit{main}|}|\\
\end{tabular}
\end{center}
%
or alternatively with:
%
\begin{center}
\begin{tabular}{l}
|\input{childdoc.def}|\\
|\childdocby{|\textit{main}|}|\\
\end{tabular}
\end{center}
%
Both forms have slightly different effects as described above.
The main file is prepared as usual, see \secref{sec:include}.

%%%%%%%%%%%%%%%%%%%%%%%%%%%%%%%%%%%%%%%%%%%%%%%%%%%%%%%%%%%%%%%%%%%%%%%%%%%%%%%%
\subsection{Legacy Detection}
\label{sec:detection}

The directive |\childdocmain| in the main file can detect
whether the complete document or merely a child is to be compiled
even without using the directive |\childdocof|.
This method is deprecated because it is less robust
and there is no compelling reason to use it;
it is merely provided for backward compatibility
and it may be removed in future versions.

If the detection mechanism is to be used,
it is mandatory to correctly specify
the filename of the main file as the argument of |\childdocmain|:
%
\begin{center}
\begin{tabular}{l}
|\input{childdoc.def}|\\
|\childdocmain{|\textit{main}|}|\\
\end{tabular}
\end{center}
%
If |\jobname| does not match the argument \textit{main} of |\childdocmain|,
it is assumed that |\jobname| points to the child file to be compiled.
When using |\childdocmain| with the main file specified as argument,
it suffices to start a child file
with just |\input{|\textit{main}|}|
without loading of the package and using |\childdocof|.
If instead all processing is done
with the appropriate \textsf{childdoc} directives,
the argument of \textit{main} of |\childdocmain| can be empty.

An alternative version of the command line processing described
in \secref{sec:commandline} using the detection mechanism reads:
%
\begin{center}
|... -jobname "|\textit{target}|" "|[\textit{flags}]%
[|\def\jobname{|\textit{dest}|}|]|\input{|\textit{main}|}"|
\end{center}

%%%%%%%%%%%%%%%%%%%%%%%%%%%%%%%%%%%%%%%%%%%%%%%%%%%%%%%%%%%%%%%%%%%%%%%%%%%%%%%%
\subsection{Manual Code}
\label{sec:manual}

In case one cannot be certain whether the definitions file |childdoc.def|
is installed on the target \TeX{} distribution
and one prefers not to ship it,
it is conceivable to paste a few relevant commands into the sources.

To that end, drop all statements |\input{childdoc.def}|
and perform the replacements as outlined below.
Instead of |\childdocmain{|\textit{main}|}| add the following code
to the top of the main file:
%
\begin{center}
\begin{tabular}{l}
|\||ifdefined\childdocname\endinput\||fi\newif\ifchilddoc|\\
|\edef\childdocname{\scantokens\expandafter{\jobname\noexpand}}|\\
|\def\childdocmain{|\textit{main}|}\||ifx\childdocmain\childdocname\||else|\\
|\childdoctrue\includeonly{\childdocname}\let\jobname\childdocmain\||fi|\\
\end{tabular}
\end{center}
%
Instead of |\childdocof{|\textit{main}|}| just include the main file
at the top of each child file:
%
\begin{center}
|\input{|\textit{main}|}|
\end{center}
%
A simple redirection |\childdocforward{|\textit{dest}|}| is achieved by:
%
\begin{center}
|\def\jobname{|\textit{dest}|}\input{\jobname}|
\end{center}
%
The redirection with prefix
|\childdocforwardprefix[|\textit{prefix}|]{|\textit{dest}|}|
is accomplished by:
%
\begin{center}
\begin{tabular}{l}
|{\edef\jobname{\scantokens\expandafter{\jobname\noexpand}}|\\
|\def\redirectjob |\textit{prefix}|#1~~~{\gdef\jobname{|\textit{dest}|#1}}|\\
|\expandafter\redirectjob\jobname~~~}\input{\jobname}|
\end{tabular}
\end{center}

In an alternative approach,
child documents can be compiled by a specific command line
without additional code or specific definitions:
%
\begin{center}
|... -jobname "|\textit{target}|" "|[\textit{flags}]%
|\includeonly{|\textit{dest}|}\input{|\textit{main}|}"|
\end{center}
%

%%%%%%%%%%%%%%%%%%%%%%%%%%%%%%%%%%%%%%%%%%%%%%%%%%%%%%%%%%%%%%%%%%%%%%%%%%%%%%%%
%%%%%%%%%%%%%%%%%%%%%%%%%%%%%%%%%%%%%%%%%%%%%%%%%%%%%%%%%%%%%%%%%%%%%%%%%%%%%%%%
\section{Information}

%%%%%%%%%%%%%%%%%%%%%%%%%%%%%%%%%%%%%%%%%%%%%%%%%%%%%%%%%%%%%%%%%%%%%%%%%%%%%%%%
\subsection{Copyright}

Copyright \copyright{} 2017--2018 Niklas Beisert

This work may be distributed and/or modified under the
conditions of the \LaTeX{} Project Public License, either version 1.3
of this license or (at your option) any later version.
The latest version of this license is in
  \url{http://www.latex-project.org/lppl.txt}
and version 1.3 or later is part of all distributions of \LaTeX{}
version 2005/12/01 or later.

This work has the LPPL maintenance status `maintained'.

The Current Maintainer of this work is Niklas Beisert.

This work consists of the files |README.txt|, |childdoc.ins| and |childdoc.dtx|
as well as the derived files |childdoc.def|, |cdocsamp.tex|
with |cdocsch1.tex|, |cdocsch2.tex|, |cdocspt3.tex|, |cdocspt4.tex|,
|cdocsdrf.tex|, |cdocsfn1.tex|, |cdocsfn2.tex|
as well as |childdoc.pdf|.

%%%%%%%%%%%%%%%%%%%%%%%%%%%%%%%%%%%%%%%%%%%%%%%%%%%%%%%%%%%%%%%%%%%%%%%%%%%%%%%%
\subsection{Files and Installation}

The package consists of the files:
%
\begin{center}
\begin{tabular}{ll}
    |README.txt|   & readme file \\
    |childdoc.ins| & installation file \\
    |childdoc.dtx| & source file \\
    |childdoc.def| & definition file \\
    |cdocsamp.tex| & sample main file \\
    |cdocsch1.tex| & sample include file \\
    |cdocsch2.tex| & sample include file \\
    |cdocspt3.tex| & sample part file \\
    |cdocspt4.tex| & sample part file \\
    |cdocsdrf.tex| & sample redirection file \\
    |cdocsfn1.tex| & sample redirection file \\
    |cdocsfn2.tex| & sample redirection file \\
    |childdoc.pdf| & manual
\end{tabular}
\end{center}
%
The distribution consists of the files
|README.txt|, |childdoc.ins| and |childdoc.dtx|.
%
\begin{itemize}
\item
Run (pdf)\LaTeX{} on |childdoc.dtx|
to compile the manual |childdoc.pdf| (this file).
\item
Run \LaTeX{} on |childdoc.ins| to create the definitions file |childdoc.def|
and the sample |cdocsamp.tex| with include files
|cdocsch1.tex|, |cdocsch2.tex|, |cdocspt3.tex|, |cdocspt4.tex|,
|cdocsdrf.tex|, |cdocsfn1.tex|, |cdocsfn2.tex|.
Then copy the file |childdoc.def| to an appropriate directory of your \LaTeX{}
distribution, e.g.\ \textit{texmf-root}|/tex/latex/childdoc|.
\end{itemize}

%%%%%%%%%%%%%%%%%%%%%%%%%%%%%%%%%%%%%%%%%%%%%%%%%%%%%%%%%%%%%%%%%%%%%%%%%%%%%%%%
\subsection{Related CTAN Packages}

There are several other packages which offer a similar functionality:
%
\begin{itemize}
\item
The packages
\href{http://ctan.org/pkg/docmute}{\textsf{docmute}},
\href{http://ctan.org/pkg/includex}{\textsf{includex}} and
\href{http://ctan.org/pkg/standalone}{\textsf{standalone}}
provide commands to include only the document body of
a child file thus allowing both files to be compiled individually.
\item
The packages \href{http://ctan.org/pkg/subdocs}{\textsf{subdocs}}
and \href{http://ctan.org/pkg/subfiles}{\textsf{subfiles}}
provide structures in which the main and child documents can be
encapsulated and allowing them to be compiled individually.
The inclusion mechanism is different from the conventional |\include|.
\item
The package \href{http://ctan.org/pkg/combine}{\textsf{combine}}
is an elaborate solution to combine several documents into one.
\end{itemize}
%
See also the CTAN topic \href{http://ctan.org/topic/subdocs}{\textsf{subdocs}}
for further related packages.
The present package differs from the above solutions in that
a document structure constructed with the conventional |\include| mechanism
just needs two extra commands at the top of every file
such that all constituent files can be compiled individually.

%%%%%%%%%%%%%%%%%%%%%%%%%%%%%%%%%%%%%%%%%%%%%%%%%%%%%%%%%%%%%%%%%%%%%%%%%%%%%%%%
%\subsection{Feature Suggestions}
%
%The following is a list of features which may be useful for future
%versions of this package:
%%
%\begin{itemize}
%\item
%\ldots
%\end{itemize}

%%%%%%%%%%%%%%%%%%%%%%%%%%%%%%%%%%%%%%%%%%%%%%%%%%%%%%%%%%%%%%%%%%%%%%%%%%%%%%%%
\subsection{Revision History}

%%%%%%%%%%%%%%%%%%%%%%%%%%%%%%%%%%%%%%%%
\paragraph{v2.0:} 2018/12/30

\begin{itemize}
\item
immediate forward processing
\item
added |\childdocby| mechanism
\item
manual restructured
\end{itemize}

%%%%%%%%%%%%%%%%%%%%%%%%%%%%%%%%%%%%%%%%
\paragraph{v1.6:} 2018/01/17

\begin{itemize}
\item
application for development of include files
\item
corrections to manual
\end{itemize}

%%%%%%%%%%%%%%%%%%%%%%%%%%%%%%%%%%%%%%%%
\paragraph{v1.5:} 2017/05/21

\begin{itemize}
\item
more complete structuring introduced
\item
|\childdocof| introduced
\item
|\childdoc| renamed to |\childdocmain|
\item
|\childredirect| renamed to |\childdocforward| and |\childdocforwardprefix|
and functionality expanded
\end{itemize}

%%%%%%%%%%%%%%%%%%%%%%%%%%%%%%%%%%%%%%%%
\paragraph{v1.0:} 2017/04/27

\begin{itemize}
\item
manual and install package
\item
first version published on CTAN
\end{itemize}

%%%%%%%%%%%%%%%%%%%%%%%%%%%%%%%%%%%%%%%%
\paragraph{v0.6:} 2017/04/26

\begin{itemize}
\item
redirection mechanism added
\end{itemize}

%%%%%%%%%%%%%%%%%%%%%%%%%%%%%%%%%%%%%%%%
\paragraph{v0.5:} 2017/04/26

\begin{itemize}
\item
functionality in definition file
\end{itemize}


%%%%%%%%%%%%%%%%%%%%%%%%%%%%%%%%%%%%%%%%%%%%%%%%%%%%%%%%%%%%%%%%%%%%%%%%%%%%%%%%
%%%%%%%%%%%%%%%%%%%%%%%%%%%%%%%%%%%%%%%%%%%%%%%%%%%%%%%%%%%%%%%%%%%%%%%%%%%%%%%%
%%%%%%%%%%%%%%%%%%%%%%%%%%%%%%%%%%%%%%%%%%%%%%%%%%%%%%%%%%%%%%%%%%%%%%%%%%%%%%%%
\appendix

\settowidth\MacroIndent{\rmfamily\scriptsize 000\ }

 \DocInput{childdoc.dtx}

\end{document}
%</driver>
% \fi
%
% %%%%%%%%%%%%%%%%%%%%%%%%%%%%%%%%%%%%%%%%%%%%%%%%%%%%%%%%%%%%%%%%%%%%%%%%%%%%%%
% %%%%%%%%%%%%%%%%%%%%%%%%%%%%%%%%%%%%%%%%%%%%%%%%%%%%%%%%%%%%%%%%%%%%%%%%%%%%%%
% \section{Sample}
%\iffalse
%<*samplemain>
%\fi
%
% The following presents a sample document
% with two chapters, two parts, a title page,
% a compile flag as well as three forwarding files to set the flag.
% It consists of eight |.tex| files:
% \begin{center}
% \begin{tabular}{ll}
% |cdocsamp.tex|&main file\\
% |cdocsch1.tex|&include file for chapter 1\\
% |cdocsch2.tex|&include file for chapter 2\\
% |cdocspt3.tex|&include file for part 3\\
% |cdocspt4.tex|&include file for part 4\\
% |cdocsdrf.tex|&forwarding file for main file in draft mode\\
% |cdocsfi1.tex|&forwarding file for final version of chapter 1\\
% |cdocsfi2.tex|&forwarding file for final version of chapter 2\\
% \end{tabular}
% \end{center}
% Each of the eight files can be compiled directly by the \LaTeX{} compiler.
%
% %%%%%%%%%%%%%%%%%%%%%%%%%%%%%%%%%%%%%%
% \paragraph{Main File.}
%
% The main file is called |cdocsamp.tex|.
%
% Load the \textsf{childdoc} definitions and
% declare the filename for the main document:
%    \begin{macrocode}
\input{childdoc.def}
\childdocmain{}
%    \end{macrocode}

% Optional override for |\version| flag:
%    \begin{macrocode}
%%\ifchilddoc\else\providecommand{\version}{draft}\fi
%    \end{macrocode}

% Define the default values for the |\version| flag
% (|final| for the main file and |draft| for childs):
%    \begin{macrocode}
\ifchilddoc
\providecommand{\version}{draft}
\else
\providecommand{\version}{final}
\fi
%    \end{macrocode}

% Load the standard document class:
%    \begin{macrocode}
\documentclass[12pt]{article}
%    \end{macrocode}

% Start the document body:
%    \begin{macrocode}
\begin{document}
%    \end{macrocode}

% Declare a title page.
% Print title, part of document being processed and version flag:
%    \begin{macrocode}
\addtocounter{page}{-1}
\begin{center}
{\LARGE\bfseries{}childdoc example\par}
\vspace{1cm}
\ifchilddoc
\ifchilddocmanual part\else chapter\fi:
`\childdocname' of `\childdocjob'\par
\else
main document: `\childdocjob'\par
\fi
version: \version\par
\end{center}
\newpage
%    \end{macrocode}

% Manually include selected file,
% otherwise process as usual:
%    \begin{macrocode}
\ifchilddocmanual
\section*{part `\childdocname'}
\input{\childdocname}
\else
%    \end{macrocode}

% Include the two chapters:
%    \begin{macrocode}
\include{cdocsch1}
\include{cdocsch2}
%    \end{macrocode}

% Include the two parts unless only chapters should be displayed:
%    \begin{macrocode}
\ifchilddoc\else
\section{part three}
\input{cdocspt3}
\section{part four}
\input{cdocspt4}
\fi
%    \end{macrocode}

% Process as usual until here:
%    \begin{macrocode}
\fi
%    \end{macrocode}

% End of document body:
%    \begin{macrocode}
\end{document}
%    \end{macrocode}
%\iffalse
%</samplemain>
%\fi
%
% %%%%%%%%%%%%%%%%%%%%%%%%%%%%%%%%%%%%%%
% \paragraph{Chapter Include Files.}
%
% The include files are called |cdocsch1.tex| and |cdocsch2.tex|.
%
%\iffalse
%<*samplechap1|samplechap2>
%\fi

% Optional override for |\version| flag:
%    \begin{macrocode}
%%\providecommand{\version}{final}
%    \end{macrocode}

% Include the main document:
%    \begin{macrocode}
\input{childdoc.def}
\childdocof{cdocsamp}
%    \end{macrocode}

%\iffalse
%</samplechap1|samplechap2>
%\fi
%
%\iffalse
%<*samplechap1>
%\fi
% Some text for chapter 1:
%    \begin{macrocode}
\section{one}
some text in chapter one
%    \end{macrocode}

%\iffalse
%</samplechap1>
%\fi
% Some text for chapter 2:
%\iffalse
%<*samplechap2>
%\fi
%    \begin{macrocode}
\section{two}
more text in chapter two
%    \end{macrocode}

%\iffalse
%</samplechap2>
%\fi
%
% %%%%%%%%%%%%%%%%%%%%%%%%%%%%%%%%%%%%%%
% \paragraph{Part Include Files.}
%
% The include files are called |cdocspt3.tex| and |cdocspt4.tex|.
%
%\iffalse
%<*samplepart3|samplepart4>
%\fi

% Optional override for |\version| flag:
%    \begin{macrocode}
%%\providecommand{\version}{final}
%    \end{macrocode}

% Include the main document:
%    \begin{macrocode}
\input{childdoc.def}
\childdocby{cdocsamp}
%    \end{macrocode}

%\iffalse
%</samplepart3|samplepart4>
%\fi
%
%\iffalse
%<*samplepart3>
%\fi
% Some text for part 3:
%    \begin{macrocode}
some text in part three
%    \end{macrocode}

%\iffalse
%</samplepart3>
%\fi
% Some text for part 4:
%\iffalse
%<*samplepart4>
%\fi
%    \begin{macrocode}
more text in part four
%    \end{macrocode}

%\iffalse
%</samplepart4>
%\fi
%
% %%%%%%%%%%%%%%%%%%%%%%%%%%%%%%%%%%%%%%
% \paragraph{Forwarding for a Complete Draft.}
%
% The following forwarding file |cdocsdrf.tex|
% compiles the main document in draft mode:
%\iffalse
%<*sampledraft>
%\fi
%    \begin{macrocode}
\def\version{draft}
\input{childdoc.def}
\childdocforward{cdocsamp}
%    \end{macrocode}

%\iffalse
%</sampledraft>
%\fi
%
% %%%%%%%%%%%%%%%%%%%%%%%%%%%%%%%%%%%%%%
% \paragraph{Forwarding for Final Version of the Chapters.}
%
% The following forwarding files |cdocsfn1.tex| and |cdocsfn2.tex|
% (with identical content)
% compile the final versions of the child documents
% |cdocsch1.tex| and |cdocsch2.tex|, respectively:
%\iffalse
%<*samplefinal>
%\fi
%    \begin{macrocode}
\def\version{final}
\input{childdoc.def}
\childdocforwardprefix[cdocsamp]{cdocsfn}{cdocsch}
%    \end{macrocode}

%\iffalse
%</samplefinal>
%\fi
%
% %%%%%%%%%%%%%%%%%%%%%%%%%%%%%%%%%%%%%%
% \paragraph{Command Line Processing.}
%
% The following three command lines generate the output files
% |cdocscld|, |cdocscl1| and |cdocscl2|
% which should be identical to
% |cdocsdrf|, |cdocsch1| and |cdocsfn2|, respectively:
% \begin{center}
% \begin{tabular}{l}
% |latex -jobname cdocscld \|\\
% |  "\def\version{draft}\input{childdoc.def}\childdocforward{cdocsamp}"|\\
% |latex -jobname cdocscl1 \|\\
% |  "\input{childdoc.def}\childdocforward[cdocsamp]{cdocsch1}"|\\
% |latex -jobname cdocscl2 \|\\
% |  "\def\version{final}\input{childdoc.def}\childdocforward{cdocsch2}"|
% \end{tabular}
% \end{center}
% Note that the trailing backslash on each first line
% merely continues the input to the second line
% (for convenient cut ant paste).
% Furthermore, the command |latex| can be replaced by any
% of its alternative versions such as |pdflatex|.
%
% %%%%%%%%%%%%%%%%%%%%%%%%%%%%%%%%%%%%%%%%%%%%%%%%%%%%%%%%%%%%%%%%%%%%%%%%%%%%%%
% %%%%%%%%%%%%%%%%%%%%%%%%%%%%%%%%%%%%%%%%%%%%%%%%%%%%%%%%%%%%%%%%%%%%%%%%%%%%%%
% \section{Implementation}
%\iffalse
%<*package>
%\fi
%
% This section describes the definitions file |childdoc.def|.

% The definitions cannot be loaded using |\usepackage| or |\RequirePackage|
% which has a mechanism to prevent loading a style file more than once.
% When loading the definitions by means of |\input|
% multiple instances have to be prevented manually:
%\iffalse
%This code needs to be before the `\ProvidesFile' directive
%which is defined at the beginning of this file.
%Therefore it is also placed there and commented out here.
%</package>
%<*discard>
%\fi
%    \begin{macrocode}
\ifdefined\childdocmain\endinput\fi
%    \end{macrocode}
%\iffalse
%</discard>
%<*package>
%\fi
%
% \macro{\ifchilddoc}
% \macro{\ifchilddocmanual}
% The conditional |\ifchilddoc| tells whether a
% child (true) or main (false) document is being compiled.
% The conditional |\ifchilddocmanual| tells whether
% the |\includeonly| mechanism is used (false) or
% the selection of child files must be performed manually (true).
% The definitions initialise to false:
%    \begin{macrocode}
\newif\ifchilddoc
\newif\ifchilddocmanual
%    \end{macrocode}

% \macro{\childdocname}
% \macro{\childdocjob}
% The macro |\childdocname| stores the name of the main document
% to be compiled. The macro |\childdocjob| stores the name of
% the document on which the \LaTeX{} compiler was originally invoked.
% The content of |\jobname| cannot be compared
% to filenames specified in the source due to different catcodes.
% The following code rescans |\jobname|, stores the result
% in |\childdocname| and saves a copy in |\childdocjob|:
%    \begin{macrocode}
\edef\childdocname{\scantokens\expandafter{\jobname\noexpand}}
\let\childdocjob\childdocname
%    \end{macrocode}

% \macro{\childdocdisable}
% The macro |\childdocdisable| prevents the main file
% from being processed more than once.
% At this stage, the main document command |\childdocmain|
% is assumed to be called once again where it should do nothing.
% Any subsequent call to it should prevent
% a secondary processing of the main document
% It overwrites the forwarding commands
% |\childdocof| and |\childdocforward|
% with empty macros to prevent further inclusions of the main document:
%    \begin{macrocode}
\newcommand{\childdocdisable}
{
  \renewcommand{\childdocmain}[1]{\renewcommand{\childdocmain}[1]{\endinput}}
  \renewcommand{\childdocof}[1]{}
  \renewcommand{\childdocby}[2][]{}
  \renewcommand{\childdocforward}[2][]{}
  \renewcommand{\childdocdisable}{}
}
%    \end{macrocode}

% \macro{\childdocmain}
% The macro |\childdocmain| is to be called at the top of the main file
% with nothing or the main filename (without extension) as argument.
% First, it breaks loops.
% If the argument is not empty and does not match |\childdocname|
% (which is set by the first inclusion of |childdoc.def|),
% |\ifchilddoc| is set to true, |\includeonly| is applied to the child file
% and |\jobname| is set to the main file
% (for proper handling of |.aux| files):
%    \begin{macrocode}
\newcommand{\childdocmain}[1]
{
  \childdocdisable\childdocmain{}
  \if?#1?\else
    \begingroup
      \def\childdoctmp{#1}
      \ifx\childdoctmp\childdocname
        \def\childdoctmp{}
      \else
        \def\childdoctmp
        {
          \childdoctrue
          \includeonly{\childdocname}
          \def\childdocjob{#1}
          \def\jobname{#1}
        }
      \fi
      \expandafter
    \endgroup
    \childdoctmp
  \fi
}
%    \end{macrocode}

% \macro{\childdocof}
% The command |\childdocof| redirects
% compilation to the main file |#1|.
%    \begin{macrocode}
\newcommand{\childdocof}[1]
{
  \childdocdisable
  \childdoctrue
  \includeonly{\childdocname}
  \def\jobname{#1}
  \def\childdocjob{#1}
  \input{#1}
}
%    \end{macrocode}

% \macro{\childdocby}
% The command |\childdocby| ....
%    \begin{macrocode}
\newcommand{\childdocby}[2][]
{
  \childdocdisable
  \childdoctrue
  \childdocmanualtrue
  \if?#1?\else
    \def\jobname{#2}
  \fi
  \def\childdocjob{#2}
  \input{#2}
  \endinput
}
%    \end{macrocode}

% \macro{\childdocforward}
% The command |\childdocforward| redirects
% compilation to the main file or
% (if the optional argument is given) a child file.
% Parameters are set as if the main file
% or a child file starting with |\childdocof| was compiled.
% Then compilation is handed over to the main file:
%    \begin{macrocode}
\newcommand{\childdocforward}[2][]
{
  \begingroup
    \if?#1?
      \def\childdoctmp
      {
        \def\childdocname{#2}
        \def\childdocjob{#2}
        \def\jobname{#2}
        \input{#2}
        \endinput
      }
    \else
      \def\childdoctmp
      {
        \childdocdisable
        \def\childdocname{#2}
        \childdoctrue
        \includeonly{#2}
        \def\childdocjob{#1}
        \def\jobname{#1}
        \input{#1}
        \endinput
      }
    \fi
    \expandafter
  \endgroup
  \childdoctmp
}
%    \end{macrocode}

% \macro{\childdocforwardprefix}
% The command |\childdocforwardprefix| redirects
% compilation to the main or a child file by means of a pattern.
% The prefix |#1| in the current filename is replaced by |#2|
% and the suffix of the current filename is kept
% (it is assumed that the filename does not contain the substring `|~~~|'
% which is used as a delimiter).
% Compilation is handed over to the new file by |\childdocforward|:
%    \begin{macrocode}
\newcommand{\childdocforwardprefix}[3][]
{
  \begingroup
    \def\childdocextract #2##1~~~{\def\childdoctmp{\childdocforward[#1]{#3##1}}}
    \expandafter\childdocextract\childdocname~~~
    \expandafter
  \endgroup
  \childdoctmp
}
%    \end{macrocode}

% \macro{\childdoc}
% The deprecated macro |\childdoc| is a legacy version of |\childdocmain|:
%    \begin{macrocode}
\newcommand{\childdoc}{\childdocmain}
%    \end{macrocode}

% \macro{\childdocredirect}
% The deprecated macro |\childdocredirect| is a legacy version
% of |\childdocforward| and |\childdocforwardprefix|:
%    \begin{macrocode}
\newcommand{\childdocredirect}[2][]
{
  \begingroup
    \if?#1?
      \def\childdoctmp{\childdocforward{#2}}
    \else
      \def\childdoctmp{\childdocforwardprefix{#1}{#2}}
    \fi
    \expandafter
  \endgroup
  \childdoctmp
}
%    \end{macrocode}

%\iffalse
%</package>
%\fi
%
\endinput
\childdocforward[|\textit{main}|]{|\textit{dest}|}"|
\end{center}
%
Here \textit{target} is the name of the output file,
\textit{main} is the name of the main file
and \textit{dest} is the name of the main or child file to be processed
(all filenames without extensions).
The optional argument \textit{main} can be omitted
if \textit{main} matches \textit{dest}.
Optionally, compilation \textit{flags} can be defined via |\def| commands.
This command line makes the \TeX{} engine believe
it is compiling the file \textit{target}
whose content is specified as the latter parameter.
The provided code then forwards the processing to
\textit{main} or \textit{dest} as described in \secref{sec:forward}.

%%%%%%%%%%%%%%%%%%%%%%%%%%%%%%%%%%%%%%%%%%%%%%%%%%%%%%%%%%%%%%%%%%%%%%%%%%%%%%%%
\subsection{Include by Input}
\label{sec:input}

Including child documents by |\include| has some restrictions by design.
Most notably, the content of a child document always occupies
its own set of pages; pages cannot be shared between child documents.
Usually, this behaviour makes perfect sense
because each child document contain an essential part of the document.
However, in some situations it may be desirable to compose
a document from a collection of parts
without having mandatory page breaks between then.
For this case, the package
provides a mechanism to include parts
by |\input| which can also be processed individually.
However, by construction this mechanism
requires manual handling of the content to be output.

%%%%%%%%%%%%%%%%%%%%%%%%%%%%%%%%%%%%%%%%
\DescribeMacro{\ifchilddocmanual}
The main file should be prepared as usual, see \secref{sec:include}.
However, the document body must make a distinction
between processing of an individual part and of the main document, e.g.:
%
\begin{center}
\begin{tabular}{l}
|\ifchilddocmanual|\\
|\input{\childdocname}|\\
|\||else|\\
\textit{document body with }|\input{|\textit{part}|}|\\
|\||fi|
\end{tabular}
\end{center}
%
The conditional |\ifchilddocmanual| is true whenever
a part to be included by |\input| is being compiled,
and the name of the part is stored in |\childdocname|.

%%%%%%%%%%%%%%%%%%%%%%%%%%%%%%%%%%%%%%%%
\DescribeMacro{\childdocby}
Each part to be included by |\input| should start with:
%
\begin{center}
\begin{tabular}{l}
|% \iffalse
%
% childdoc.dtx Copyright (C) 2017-2018 Niklas Beisert
%
% This work may be distributed and/or modified under the
% conditions of the LaTeX Project Public License, either version 1.3
% of this license or (at your option) any later version.
% The latest version of this license is in
%   http://www.latex-project.org/lppl.txt
% and version 1.3 or later is part of all distributions of LaTeX
% version 2005/12/01 or later.
%
% This work has the LPPL maintenance status `maintained'.
%
% The Current Maintainer of this work is Niklas Beisert.
%
% This work consists of the files childdoc.dtx and childdoc.ins
% and the derived files childdoc.def and cdocsamp.tex with
% cdocsch1.tex, cdocsch2.tex, cdocsdrf.tex, cdocsfn1.tex, cdocsfn2.tex.
%
%<package>\ifdefined\childdocmain\endinput\fi
%<package>\ProvidesFile{childdoc.def}[2018/12/30 v2.0 child document driver]
%<samplemain>\ProvidesFile{cdocsamp.tex}[2018/12/30 v2.0 sample for childdoc]
%<*driver>
%\ProvidesFile{childdoc.drv}[2018/12/30 v2.0 childdoc reference manual file]
\PassOptionsToClass{10pt,a4paper}{article}
\documentclass{ltxdoc}

\usepackage[margin=35mm]{geometry}
\usepackage{hyperref}
\usepackage{hyperxmp}
\usepackage[usenames]{color}

\hypersetup{colorlinks=true}
\hypersetup{pdfstartview=FitH}
\hypersetup{pdfpagemode=UseNone}
\hypersetup{pdfsource={}}
\hypersetup{pdflang={en-UK}}
\hypersetup{pdfcopyright={Copyright 2017-2018 Niklas Beisert.
  This work may be distributed and/or modified under the
  conditions of the LaTeX Project Public License, either version 1.3
  of this license or (at your option) any later version.}}
\hypersetup{pdflicenseurl={http://www.latex-project.org/lppl.txt}}
\hypersetup{pdfcontactaddress={ETH Zurich, ITP, HIT K,
  Wolfgang-Pauli-Strasse 27}}
\hypersetup{pdfcontactpostcode={8093}}
\hypersetup{pdfcontactcity={Zurich}}
\hypersetup{pdfcontactcountry={Switzerland}}
\hypersetup{pdfcontactemail={nbeisert@itp.phys.ethz.ch}}
\hypersetup{pdfcontacturl={http://people.phys.ethz.ch/\xmptilde nbeisert/}}

\newcommand{\secref}[1]{\hyperref[#1]{section \ref*{#1}}}

\parskip1ex
\parindent0pt
\let\olditemize\itemize
\def\itemize{\olditemize\parskip0pt}

\begin{document}

\title{The \textsf{childdoc} Package}
\hypersetup{pdftitle={The childdoc Package}}
\author{Niklas Beisert\\[2ex]
  Institut f\"ur Theoretische Physik\\
  Eidgen\"ossische Technische Hochschule Z\"urich\\
  Wolfgang-Pauli-Strasse 27, 8093 Z\"urich, Switzerland\\[1ex]
  \href{mailto:nbeisert@itp.phys.ethz.ch}
  {\texttt{nbeisert@itp.phys.ethz.ch}}}
\hypersetup{pdfauthor={Niklas Beisert}}
\hypersetup{pdfsubject={Manual for the LaTeX2e Package childdoc}}
\date{30 December 2018, \textsf{v2.0}}
\maketitle

\begin{abstract}\noindent
\textsf{childdoc} is a \LaTeXe{} package
that enables the direct compilation
of document sections included by |\include|
to individual files.
\end{abstract}

\begingroup
\parskip0ex
\tableofcontents
\endgroup

%%%%%%%%%%%%%%%%%%%%%%%%%%%%%%%%%%%%%%%%%%%%%%%%%%%%%%%%%%%%%%%%%%%%%%%%%%%%%%%%
%%%%%%%%%%%%%%%%%%%%%%%%%%%%%%%%%%%%%%%%%%%%%%%%%%%%%%%%%%%%%%%%%%%%%%%%%%%%%%%%
\section{Introduction}

\LaTeX{} provides a mechanism to structure a large document (such as a book)
into a main file and several child files (containing the chapters)
using the |\include| command.
This mechanism is beneficial for documents
which span hundreds of pages in order to
make the source file(s) more manageable.
Moreover, compilation can be restricted to
selected child files by means of the |\includeonly| command.
The latter feature can be used to reduce the compilation time while editing
(this was significantly more useful in the earlier days of \LaTeX{})
or to generate a smaller document which is easier to navigate.
Another application of |\includeonly| is to generate
documents consisting of selected parts of the complete document.

However, there are a few drawbacks of the plain |\include| mechanism:
\begin{itemize}
\item
The child files cannot be compiled on their own,
they can only be compiled via the main file.
A naive editing environment
(such as a text editor with an option
to have the current file processed by \LaTeX)
may require one to switch to the main file before compiling;
attempting to compile the child file produces errors.
\item
The main file must be modified (each time)
to adjust the |\includeonly| command
to the present needs. This easily leaves the main file in a messy state.
\item
The generated document will always carry the filename
of the main document. This is inconvenient if
several child files are to be compiled and
to be kept for distribution.
\end{itemize}

The present package provides a simple interface
to make child files individually compilable by \LaTeX{}.
Compiling a child file then has the same effect as compiling
the main file with an |\includeonly| command
to select the appropriate child.
Moreover the generated document will carry the name of the child
rather than the main file.
This resolves all three above issues.

This feature is meant to make the editing of books,
thesis documents and lecture notes somewhat more convenient.
However, the package can also be used efficiently for
composing a series of documents (such as exercise sheets)
which are typically distributed individually.
It then assists the author in generating the individual documents
(potentially in different versions)
as well as a document containing the collected series.
Another application is in developing style files
or other kinds of included material
where compilation of the style file could redirect
to a sample or test file.

%%%%%%%%%%%%%%%%%%%%%%%%%%%%%%%%%%%%%%%%%%%%%%%%%%%%%%%%%%%%%%%%%%%%%%%%%%%%%%%%
%%%%%%%%%%%%%%%%%%%%%%%%%%%%%%%%%%%%%%%%%%%%%%%%%%%%%%%%%%%%%%%%%%%%%%%%%%%%%%%%
\section{Usage}

First of all, the package \textsf{childdoc} is \emph{not} a standard
\LaTeXe{} |.sty| style file! Therefore it needs to be invoked in
a non-standard way.

%%%%%%%%%%%%%%%%%%%%%%%%%%%%%%%%%%%%%%%%%%%%%%%%%%%%%%%%%%%%%%%%%%%%%%%%%%%%%%%%
\subsection{Included Files}
\label{sec:include}

%%%%%%%%%%%%%%%%%%%%%%%%%%%%%%%%%%%%%%%%
\DescribeMacro{\childdocmain}
To use the package, add the commands
\begin{center}
\begin{tabular}{l}
|\input{childdoc.def}|\\
|\childdocmain{}|\\
\end{tabular}
\end{center}
at the very top of the main \LaTeX{} file,
in particular \emph{before} the |\documentclass| statement!
The argument of |\childdocmain| should be left empty
(but it must be present).

%%%%%%%%%%%%%%%%%%%%%%%%%%%%%%%%%%%%%%%%
\DescribeMacro{\childdocof}
Furthermore, add the commands
\begin{center}
\begin{tabular}{l}
|\input{childdoc.def}|\\
|\childdocof{|\textit{main}|}|\\
\end{tabular}
\end{center}
at the top of every child file \textit{child}
which is included by |\include{|\textit{child}|}|
from within the main file
(or at least for those files to be compiled individually).
The argument \textit{main} must be the filename of the main file.

There are a couple of
considerations in setting up the main and child documents:

%%%%%%%%%%%%%%%%%%%%%%%%%%%%%%%%%%%%%%%%
\paragraph{Restrictions.}

Please note the following restrictions:
\begin{itemize}
\item
|\childdocmain| must be called with one argument \textit{main}
to ensure compatibility with earlier version of the package.
It must either be empty (|\childdocmain{}|)
or precisely match the filename of the main file in which it is specified.
See \secref{sec:detection} for further information.
\item
The filename \textit{main} must be specified without the |.tex| extension.
\item
The filename \textit{main} is case sensitive
(even in case-insensitive file systems)
due to internal string comparison.
\item
The argument \textit{main} should be fully expanded, it cannot be a macro.
\item
Subdirectories and special characters should be avoided in filenames.
\item
The command |\childdocmain{|\textit{main}|}| must be followed by a whitespace.
It should not be followed immediately by another command
or by a comment mark `|%|'.
This is because the \TeX{} parser reads the token immediately following
the argument of |\childdocmain| and puts it
at the beginning of every child section;
however, a white\-space is ignored.
\end{itemize}

%%%%%%%%%%%%%%%%%%%%%%%%%%%%%%%%%%%%%%%%
\paragraph{Content of Main File.}

It is advisable to place all content in the child files included by |\include|.
Any output contained in the main file will appear in all child documents
unless suppressed manually;
it cannot be suppressed automatically by the |\includeonly| directive
and thus should normally be avoided.
A method to include some content in the main file
by means of conditional processing is described in \secref{sec:conditional}.

%%%%%%%%%%%%%%%%%%%%%%%%%%%%%%%%%%%%%%%%
\paragraph{Page Numbering.}

When only a part of the document is compiled,
the appropriate numbering of pages
(as well as other status parameters)
is determined from the |.aux| files.
The latter contain information from previous passes.
However this information needs to propagate through
all intermediate child documents.
Therefore the page numbering in child documents may well
be inconsistent until the complete document is compiled at least once.

A useful (if unconventional) way to always ensure a consistent
page numbering is to restart the numbering in each child document
and denote the pages by `\textit{child}|.|\textit{page}'
where \textit{child} represents the chapter/section number of the child file.
This can be achieved by the command
|\numberwithin{page}{|\textit{child}|}|
of the \textsf{amsmath} package
where \textit{child} can be |chapter| or |section|
depending on the chosen structuring.
Alternatively, one can modify the macro |\thepage| appropriately
and reset the counter |page| at the start of each child file.

%%%%%%%%%%%%%%%%%%%%%%%%%%%%%%%%%%%%%%%%%%%%%%%%%%%%%%%%%%%%%%%%%%%%%%%%%%%%%%%%
\subsection{Conditional Processing}
\label{sec:conditional}

The package provides a mechanism to compile different versions
of a document. To customise the versions further some conditional processing
can come in handy to distinguish which version is being compiled.
The package provides two macros to describe the compilation context:

%%%%%%%%%%%%%%%%%%%%%%%%%%%%%%%%%%%%%%%%
\DescribeMacro{\ifchilddoc}
The conditional |\ifchilddoc| distinguishes between the compilation of
child documents and the main document:
%
\begin{center}
|\ifchilddoc |\textit{child-code}| |[|\||else |\textit{main-code}]| \||fi|
\end{center}

%%%%%%%%%%%%%%%%%%%%%%%%%%%%%%%%%%%%%%%%
\DescribeMacro{\childdocname}
\DescribeMacro{\childdocjob}
The macro |\childdocname| contains the filename (without extension)
of the main or child file being processed.
Note that |\childdocjob| will always contain the name of the main file.

%%%%%%%%%%%%%%%%%%%%%%%%%%%%%%%%%%%%%%%%
\paragraph{Title Page.}

Conditional processing can be used to include a title or banner page
in the main document when proper precautions are taken.
Importantly, the code in the main file should ensure that the page counter
(as well as other status parameters which are stored in the |.aux| files)
takes the same value after the conditional processing.
Otherwise the page numbers may take divergent values
depending on which part is compiled.

For example, a title page could be declared by:
%
\begin{center}
\begin{tabular}{l}
|\ifchilddoc\||else|\\
|\addtocounter{page}{-1}|\\
\textit{code for title page}\\
|\newpage|\\
|\||fi|
\end{tabular}
\end{center}
%
A banner page for the child documents can be generated by:
%
\begin{center}
\begin{tabular}{l}
|\ifchilddoc|\\
|\addtocounter{page}{-1}|\\
\textit{code for banner page}\\
|\newpage|\\
|\||fi|
\end{tabular}
\end{center}
%
Here one could write a message such as:
\begin{center}
|This is the part \childdocname{} of \childdocjob{}.|
\end{center}

%%%%%%%%%%%%%%%%%%%%%%%%%%%%%%%%%%%%%%%%%%%%%%%%%%%%%%%%%%%%%%%%%%%%%%%%%%%%%%%%
\subsection{Flags}
\label{sec:flags}

The package makes it easy to generate different versions
of the main or child documents.
To this end compilation flags can be defined
and assigned different default values.
They will be particularly useful in conjunction
with the forwarding mechanism described in \secref{sec:forward}.

For example, it may be useful to have a flag |\version|
which can be set to |draft| or |final|.
The document source will contain some conditional code
depending on the value of |\version|.
Suppose further, the flag should default to |final| for the main file
and to |draft| for child files
which is a natural assignment for editing the document.
This is achieved by placing the following code
in the preamble of the main document
(below the |\childdocmain| directive):
%
\begin{center}
\begin{tabular}{l}
|\ifchilddoc|\\
|\providecommand{\version}{draft}|\\
|\||else|\\
|\providecommand{\version}{final}|\\
|\||fi|
\end{tabular}
\end{center}
%
The definition by |\providecommand| makes sure
that previous definitions are not overwritten.
Further statements |\providecommand{\version}{...}|
can thus be added before the above code to override it.

For the main file, one might add a line
(between |\childdocmain| and the above block)
%
\begin{center}
|%\ifchilddoc\||else\providecommand{\version}{draft}\||fi|
\end{center}
%
which can be uncommented to produce a draft version.
Likewise one can add a line to the very top of a child file
(above the |\childdocof{|\textit{main}|}| directive)
%
\begin{center}
|%\providecommand{\version}{final}|
\end{center}
%
which can be uncommented to produce the final version of this child document.

%%%%%%%%%%%%%%%%%%%%%%%%%%%%%%%%%%%%%%%%%%%%%%%%%%%%%%%%%%%%%%%%%%%%%%%%%%%%%%%%
\subsection{Forwarding}
\label{sec:forward}

Different versions of the main or child documents
using compilation flags as described in \secref{sec:flags}
can be (permanently) stored in different files
for convenient compilation, viewing and distribution.
To this end, the package defines a command
to pass on compilation to a different file:

%%%%%%%%%%%%%%%%%%%%%%%%%%%%%%%%%%%%%%%%
\DescribeMacro{\childdocforward}
The command |\childdocforward| redirects processing to
another source file:
%
\begin{center}
\begin{tabular}{l}
|\input{childdoc.def}|\\
|\childdocforward[|\textit{main}|]{|\textit{dest}|}|\\
\end{tabular}
\end{center}
%
The argument \textit{dest} is the destination file
(without extension).
It should be the main file or one of the child files.
Note that further \textsf{childdoc} directives
such as |\childdocof| and |\childdocforward|
in the indicated file will be processed in this form.
The optional argument \textit{main}
passes on directly to the main file \textit{main}
while pretending to compile the child \textit{dest}.
This form behaves as if \textit{dest}
issues |\childdocof{|\textit{main}|}| right away,
and no further \textsf{childdoc} directives will be processed.

%%%%%%%%%%%%%%%%%%%%%%%%%%%%%%%%%%%%%%%%
\DescribeMacro{\...prefix}
In the alternative form |\childdocforwardprefix|,
%
\begin{center}
\begin{tabular}{l}
|\input{childdoc.def}|\\
|\childdocforwardprefix[|\textit{main}|]{|\textit{prefix}|}{|\textit{dest}|}|
\end{tabular}
\end{center}
%
the destination file is determined by a pattern
depending on the current file:
To make this work, the current file must be called
`{\textit{prefix}\hspace{0.2em}\textit{suffix}}'
with \textit{prefix} matching precisely the argument.
Processing is then passed on to the file
`{\textit{dest}\hspace{0.2em}\textit{suffix}}'.
Surely, the same effect is achieved by
directly specifying the
argument `{\textit{dest}\hspace{0.2em}\textit{suffix}}'
in the first form.
However, that requires to set up a different file
for each child. With the alternative form of the command
all these files can have exactly the same content
which simplifies setting them up and maintaining them.

For example, the following file |draft.tex|
with a compilation flag |\version| as described in \secref{sec:flags}
compiles the main document as a draft:
%
\begin{center}
\begin{tabular}{l}
|\def\version{draft}|\\
|\input{childdoc.def}|\\
|\childdocforward{|\textit{main}|}|
\end{tabular}
\end{center}
%
Likewise, the following files |final|\textit{nn}|.tex|
compile the final version of the child document
|child|\textit{nn}|.tex|:
%
\begin{center}
\begin{tabular}{l}
|\def\version{final}|\\
|\input{childdoc.def}|\\
|\childdocforwardprefix{final}{child}|
\end{tabular}
\end{center}
%

Note that when several versions of a main file and/or of each child file
are to be generated, it may be convenient to set up a |Makefile| or
shell script to automatise the process.

%%%%%%%%%%%%%%%%%%%%%%%%%%%%%%%%%%%%%%%%%%%%%%%%%%%%%%%%%%%%%%%%%%%%%%%%%%%%%%%%
\subsection{Command Line Processing}
\label{sec:commandline}

The effect of redirection files can also be achieved by invoking
the \LaTeX{} compiler with a more elaborate command line.
Most conveniently this should be done as part
of a shell script or a |Makefile|.

When using \textsf{childdoc} in the main file, the following
command lines effectively perform a redirection
(note that depending on the shell being used,
backslashes may have to be doubled: `|\|' $\to$ `|\\|'):
%
\begin{center}
|... -jobname "|\textit{target}|" |\\|"|[\textit{flags}]%
|\input{childdoc.def}\childdocforward[|\textit{main}|]{|\textit{dest}|}"|
\end{center}
%
Here \textit{target} is the name of the output file,
\textit{main} is the name of the main file
and \textit{dest} is the name of the main or child file to be processed
(all filenames without extensions).
The optional argument \textit{main} can be omitted
if \textit{main} matches \textit{dest}.
Optionally, compilation \textit{flags} can be defined via |\def| commands.
This command line makes the \TeX{} engine believe
it is compiling the file \textit{target}
whose content is specified as the latter parameter.
The provided code then forwards the processing to
\textit{main} or \textit{dest} as described in \secref{sec:forward}.

%%%%%%%%%%%%%%%%%%%%%%%%%%%%%%%%%%%%%%%%%%%%%%%%%%%%%%%%%%%%%%%%%%%%%%%%%%%%%%%%
\subsection{Include by Input}
\label{sec:input}

Including child documents by |\include| has some restrictions by design.
Most notably, the content of a child document always occupies
its own set of pages; pages cannot be shared between child documents.
Usually, this behaviour makes perfect sense
because each child document contain an essential part of the document.
However, in some situations it may be desirable to compose
a document from a collection of parts
without having mandatory page breaks between then.
For this case, the package
provides a mechanism to include parts
by |\input| which can also be processed individually.
However, by construction this mechanism
requires manual handling of the content to be output.

%%%%%%%%%%%%%%%%%%%%%%%%%%%%%%%%%%%%%%%%
\DescribeMacro{\ifchilddocmanual}
The main file should be prepared as usual, see \secref{sec:include}.
However, the document body must make a distinction
between processing of an individual part and of the main document, e.g.:
%
\begin{center}
\begin{tabular}{l}
|\ifchilddocmanual|\\
|\input{\childdocname}|\\
|\||else|\\
\textit{document body with }|\input{|\textit{part}|}|\\
|\||fi|
\end{tabular}
\end{center}
%
The conditional |\ifchilddocmanual| is true whenever
a part to be included by |\input| is being compiled,
and the name of the part is stored in |\childdocname|.

%%%%%%%%%%%%%%%%%%%%%%%%%%%%%%%%%%%%%%%%
\DescribeMacro{\childdocby}
Each part to be included by |\input| should start with:
%
\begin{center}
\begin{tabular}{l}
|\input{childdoc.def}|\\
|\childdocby{|\textit{main}|}|\\
\end{tabular}
\end{center}
%
The directive |\childdocby| is similar to |\childdocof|
described in \secref{sec:include},
but the subsequent selection of content must be done manually.
To that end, both |\ifchilddoc| and |\ifchilddocmanual|
will be true upon processing of a part,
and the name of the part is stored in |\childdocname|.
Note that |\jobname| will be set to the filename of the current part
so that each part receives an individual |.aux| file
that does not interfere with the |.aux| file(s) of the main document.
This behaviour can be altered by the alternative form
|\childdocby[*]{|\textit{main}|}| (with a non-empty optional argument)
which uses the |.aux| file of the main document
by setting |\jobname| to \textit{main}.

%%%%%%%%%%%%%%%%%%%%%%%%%%%%%%%%%%%%%%%%%%%%%%%%%%%%%%%%%%%%%%%%%%%%%%%%%%%%%%%%
\subsection{Driver Development}
\label{sec:driver}

The \textsf{childdoc} mechanism can also be use for the development
of definition files such as \LaTeX{} styles or classes.
This case differs from the above setup with multiple parts
included by |\include| in that no |\includeonly| should be invoked.
This can be achieved by starting the include file
(before |\ProvidesPackage|) with:
%
\begin{center}
\begin{tabular}{l}
|\input{childdoc.def}|\\
|\childdocforward{|\textit{main}|}|\\
\end{tabular}
\end{center}
%
or alternatively with:
%
\begin{center}
\begin{tabular}{l}
|\input{childdoc.def}|\\
|\childdocby{|\textit{main}|}|\\
\end{tabular}
\end{center}
%
Both forms have slightly different effects as described above.
The main file is prepared as usual, see \secref{sec:include}.

%%%%%%%%%%%%%%%%%%%%%%%%%%%%%%%%%%%%%%%%%%%%%%%%%%%%%%%%%%%%%%%%%%%%%%%%%%%%%%%%
\subsection{Legacy Detection}
\label{sec:detection}

The directive |\childdocmain| in the main file can detect
whether the complete document or merely a child is to be compiled
even without using the directive |\childdocof|.
This method is deprecated because it is less robust
and there is no compelling reason to use it;
it is merely provided for backward compatibility
and it may be removed in future versions.

If the detection mechanism is to be used,
it is mandatory to correctly specify
the filename of the main file as the argument of |\childdocmain|:
%
\begin{center}
\begin{tabular}{l}
|\input{childdoc.def}|\\
|\childdocmain{|\textit{main}|}|\\
\end{tabular}
\end{center}
%
If |\jobname| does not match the argument \textit{main} of |\childdocmain|,
it is assumed that |\jobname| points to the child file to be compiled.
When using |\childdocmain| with the main file specified as argument,
it suffices to start a child file
with just |\input{|\textit{main}|}|
without loading of the package and using |\childdocof|.
If instead all processing is done
with the appropriate \textsf{childdoc} directives,
the argument of \textit{main} of |\childdocmain| can be empty.

An alternative version of the command line processing described
in \secref{sec:commandline} using the detection mechanism reads:
%
\begin{center}
|... -jobname "|\textit{target}|" "|[\textit{flags}]%
[|\def\jobname{|\textit{dest}|}|]|\input{|\textit{main}|}"|
\end{center}

%%%%%%%%%%%%%%%%%%%%%%%%%%%%%%%%%%%%%%%%%%%%%%%%%%%%%%%%%%%%%%%%%%%%%%%%%%%%%%%%
\subsection{Manual Code}
\label{sec:manual}

In case one cannot be certain whether the definitions file |childdoc.def|
is installed on the target \TeX{} distribution
and one prefers not to ship it,
it is conceivable to paste a few relevant commands into the sources.

To that end, drop all statements |\input{childdoc.def}|
and perform the replacements as outlined below.
Instead of |\childdocmain{|\textit{main}|}| add the following code
to the top of the main file:
%
\begin{center}
\begin{tabular}{l}
|\||ifdefined\childdocname\endinput\||fi\newif\ifchilddoc|\\
|\edef\childdocname{\scantokens\expandafter{\jobname\noexpand}}|\\
|\def\childdocmain{|\textit{main}|}\||ifx\childdocmain\childdocname\||else|\\
|\childdoctrue\includeonly{\childdocname}\let\jobname\childdocmain\||fi|\\
\end{tabular}
\end{center}
%
Instead of |\childdocof{|\textit{main}|}| just include the main file
at the top of each child file:
%
\begin{center}
|\input{|\textit{main}|}|
\end{center}
%
A simple redirection |\childdocforward{|\textit{dest}|}| is achieved by:
%
\begin{center}
|\def\jobname{|\textit{dest}|}\input{\jobname}|
\end{center}
%
The redirection with prefix
|\childdocforwardprefix[|\textit{prefix}|]{|\textit{dest}|}|
is accomplished by:
%
\begin{center}
\begin{tabular}{l}
|{\edef\jobname{\scantokens\expandafter{\jobname\noexpand}}|\\
|\def\redirectjob |\textit{prefix}|#1~~~{\gdef\jobname{|\textit{dest}|#1}}|\\
|\expandafter\redirectjob\jobname~~~}\input{\jobname}|
\end{tabular}
\end{center}

In an alternative approach,
child documents can be compiled by a specific command line
without additional code or specific definitions:
%
\begin{center}
|... -jobname "|\textit{target}|" "|[\textit{flags}]%
|\includeonly{|\textit{dest}|}\input{|\textit{main}|}"|
\end{center}
%

%%%%%%%%%%%%%%%%%%%%%%%%%%%%%%%%%%%%%%%%%%%%%%%%%%%%%%%%%%%%%%%%%%%%%%%%%%%%%%%%
%%%%%%%%%%%%%%%%%%%%%%%%%%%%%%%%%%%%%%%%%%%%%%%%%%%%%%%%%%%%%%%%%%%%%%%%%%%%%%%%
\section{Information}

%%%%%%%%%%%%%%%%%%%%%%%%%%%%%%%%%%%%%%%%%%%%%%%%%%%%%%%%%%%%%%%%%%%%%%%%%%%%%%%%
\subsection{Copyright}

Copyright \copyright{} 2017--2018 Niklas Beisert

This work may be distributed and/or modified under the
conditions of the \LaTeX{} Project Public License, either version 1.3
of this license or (at your option) any later version.
The latest version of this license is in
  \url{http://www.latex-project.org/lppl.txt}
and version 1.3 or later is part of all distributions of \LaTeX{}
version 2005/12/01 or later.

This work has the LPPL maintenance status `maintained'.

The Current Maintainer of this work is Niklas Beisert.

This work consists of the files |README.txt|, |childdoc.ins| and |childdoc.dtx|
as well as the derived files |childdoc.def|, |cdocsamp.tex|
with |cdocsch1.tex|, |cdocsch2.tex|, |cdocspt3.tex|, |cdocspt4.tex|,
|cdocsdrf.tex|, |cdocsfn1.tex|, |cdocsfn2.tex|
as well as |childdoc.pdf|.

%%%%%%%%%%%%%%%%%%%%%%%%%%%%%%%%%%%%%%%%%%%%%%%%%%%%%%%%%%%%%%%%%%%%%%%%%%%%%%%%
\subsection{Files and Installation}

The package consists of the files:
%
\begin{center}
\begin{tabular}{ll}
    |README.txt|   & readme file \\
    |childdoc.ins| & installation file \\
    |childdoc.dtx| & source file \\
    |childdoc.def| & definition file \\
    |cdocsamp.tex| & sample main file \\
    |cdocsch1.tex| & sample include file \\
    |cdocsch2.tex| & sample include file \\
    |cdocspt3.tex| & sample part file \\
    |cdocspt4.tex| & sample part file \\
    |cdocsdrf.tex| & sample redirection file \\
    |cdocsfn1.tex| & sample redirection file \\
    |cdocsfn2.tex| & sample redirection file \\
    |childdoc.pdf| & manual
\end{tabular}
\end{center}
%
The distribution consists of the files
|README.txt|, |childdoc.ins| and |childdoc.dtx|.
%
\begin{itemize}
\item
Run (pdf)\LaTeX{} on |childdoc.dtx|
to compile the manual |childdoc.pdf| (this file).
\item
Run \LaTeX{} on |childdoc.ins| to create the definitions file |childdoc.def|
and the sample |cdocsamp.tex| with include files
|cdocsch1.tex|, |cdocsch2.tex|, |cdocspt3.tex|, |cdocspt4.tex|,
|cdocsdrf.tex|, |cdocsfn1.tex|, |cdocsfn2.tex|.
Then copy the file |childdoc.def| to an appropriate directory of your \LaTeX{}
distribution, e.g.\ \textit{texmf-root}|/tex/latex/childdoc|.
\end{itemize}

%%%%%%%%%%%%%%%%%%%%%%%%%%%%%%%%%%%%%%%%%%%%%%%%%%%%%%%%%%%%%%%%%%%%%%%%%%%%%%%%
\subsection{Related CTAN Packages}

There are several other packages which offer a similar functionality:
%
\begin{itemize}
\item
The packages
\href{http://ctan.org/pkg/docmute}{\textsf{docmute}},
\href{http://ctan.org/pkg/includex}{\textsf{includex}} and
\href{http://ctan.org/pkg/standalone}{\textsf{standalone}}
provide commands to include only the document body of
a child file thus allowing both files to be compiled individually.
\item
The packages \href{http://ctan.org/pkg/subdocs}{\textsf{subdocs}}
and \href{http://ctan.org/pkg/subfiles}{\textsf{subfiles}}
provide structures in which the main and child documents can be
encapsulated and allowing them to be compiled individually.
The inclusion mechanism is different from the conventional |\include|.
\item
The package \href{http://ctan.org/pkg/combine}{\textsf{combine}}
is an elaborate solution to combine several documents into one.
\end{itemize}
%
See also the CTAN topic \href{http://ctan.org/topic/subdocs}{\textsf{subdocs}}
for further related packages.
The present package differs from the above solutions in that
a document structure constructed with the conventional |\include| mechanism
just needs two extra commands at the top of every file
such that all constituent files can be compiled individually.

%%%%%%%%%%%%%%%%%%%%%%%%%%%%%%%%%%%%%%%%%%%%%%%%%%%%%%%%%%%%%%%%%%%%%%%%%%%%%%%%
%\subsection{Feature Suggestions}
%
%The following is a list of features which may be useful for future
%versions of this package:
%%
%\begin{itemize}
%\item
%\ldots
%\end{itemize}

%%%%%%%%%%%%%%%%%%%%%%%%%%%%%%%%%%%%%%%%%%%%%%%%%%%%%%%%%%%%%%%%%%%%%%%%%%%%%%%%
\subsection{Revision History}

%%%%%%%%%%%%%%%%%%%%%%%%%%%%%%%%%%%%%%%%
\paragraph{v2.0:} 2018/12/30

\begin{itemize}
\item
immediate forward processing
\item
added |\childdocby| mechanism
\item
manual restructured
\end{itemize}

%%%%%%%%%%%%%%%%%%%%%%%%%%%%%%%%%%%%%%%%
\paragraph{v1.6:} 2018/01/17

\begin{itemize}
\item
application for development of include files
\item
corrections to manual
\end{itemize}

%%%%%%%%%%%%%%%%%%%%%%%%%%%%%%%%%%%%%%%%
\paragraph{v1.5:} 2017/05/21

\begin{itemize}
\item
more complete structuring introduced
\item
|\childdocof| introduced
\item
|\childdoc| renamed to |\childdocmain|
\item
|\childredirect| renamed to |\childdocforward| and |\childdocforwardprefix|
and functionality expanded
\end{itemize}

%%%%%%%%%%%%%%%%%%%%%%%%%%%%%%%%%%%%%%%%
\paragraph{v1.0:} 2017/04/27

\begin{itemize}
\item
manual and install package
\item
first version published on CTAN
\end{itemize}

%%%%%%%%%%%%%%%%%%%%%%%%%%%%%%%%%%%%%%%%
\paragraph{v0.6:} 2017/04/26

\begin{itemize}
\item
redirection mechanism added
\end{itemize}

%%%%%%%%%%%%%%%%%%%%%%%%%%%%%%%%%%%%%%%%
\paragraph{v0.5:} 2017/04/26

\begin{itemize}
\item
functionality in definition file
\end{itemize}


%%%%%%%%%%%%%%%%%%%%%%%%%%%%%%%%%%%%%%%%%%%%%%%%%%%%%%%%%%%%%%%%%%%%%%%%%%%%%%%%
%%%%%%%%%%%%%%%%%%%%%%%%%%%%%%%%%%%%%%%%%%%%%%%%%%%%%%%%%%%%%%%%%%%%%%%%%%%%%%%%
%%%%%%%%%%%%%%%%%%%%%%%%%%%%%%%%%%%%%%%%%%%%%%%%%%%%%%%%%%%%%%%%%%%%%%%%%%%%%%%%
\appendix

\settowidth\MacroIndent{\rmfamily\scriptsize 000\ }

 \DocInput{childdoc.dtx}

\end{document}
%</driver>
% \fi
%
% %%%%%%%%%%%%%%%%%%%%%%%%%%%%%%%%%%%%%%%%%%%%%%%%%%%%%%%%%%%%%%%%%%%%%%%%%%%%%%
% %%%%%%%%%%%%%%%%%%%%%%%%%%%%%%%%%%%%%%%%%%%%%%%%%%%%%%%%%%%%%%%%%%%%%%%%%%%%%%
% \section{Sample}
%\iffalse
%<*samplemain>
%\fi
%
% The following presents a sample document
% with two chapters, two parts, a title page,
% a compile flag as well as three forwarding files to set the flag.
% It consists of eight |.tex| files:
% \begin{center}
% \begin{tabular}{ll}
% |cdocsamp.tex|&main file\\
% |cdocsch1.tex|&include file for chapter 1\\
% |cdocsch2.tex|&include file for chapter 2\\
% |cdocspt3.tex|&include file for part 3\\
% |cdocspt4.tex|&include file for part 4\\
% |cdocsdrf.tex|&forwarding file for main file in draft mode\\
% |cdocsfi1.tex|&forwarding file for final version of chapter 1\\
% |cdocsfi2.tex|&forwarding file for final version of chapter 2\\
% \end{tabular}
% \end{center}
% Each of the eight files can be compiled directly by the \LaTeX{} compiler.
%
% %%%%%%%%%%%%%%%%%%%%%%%%%%%%%%%%%%%%%%
% \paragraph{Main File.}
%
% The main file is called |cdocsamp.tex|.
%
% Load the \textsf{childdoc} definitions and
% declare the filename for the main document:
%    \begin{macrocode}
\input{childdoc.def}
\childdocmain{}
%    \end{macrocode}

% Optional override for |\version| flag:
%    \begin{macrocode}
%%\ifchilddoc\else\providecommand{\version}{draft}\fi
%    \end{macrocode}

% Define the default values for the |\version| flag
% (|final| for the main file and |draft| for childs):
%    \begin{macrocode}
\ifchilddoc
\providecommand{\version}{draft}
\else
\providecommand{\version}{final}
\fi
%    \end{macrocode}

% Load the standard document class:
%    \begin{macrocode}
\documentclass[12pt]{article}
%    \end{macrocode}

% Start the document body:
%    \begin{macrocode}
\begin{document}
%    \end{macrocode}

% Declare a title page.
% Print title, part of document being processed and version flag:
%    \begin{macrocode}
\addtocounter{page}{-1}
\begin{center}
{\LARGE\bfseries{}childdoc example\par}
\vspace{1cm}
\ifchilddoc
\ifchilddocmanual part\else chapter\fi:
`\childdocname' of `\childdocjob'\par
\else
main document: `\childdocjob'\par
\fi
version: \version\par
\end{center}
\newpage
%    \end{macrocode}

% Manually include selected file,
% otherwise process as usual:
%    \begin{macrocode}
\ifchilddocmanual
\section*{part `\childdocname'}
\input{\childdocname}
\else
%    \end{macrocode}

% Include the two chapters:
%    \begin{macrocode}
\include{cdocsch1}
\include{cdocsch2}
%    \end{macrocode}

% Include the two parts unless only chapters should be displayed:
%    \begin{macrocode}
\ifchilddoc\else
\section{part three}
\input{cdocspt3}
\section{part four}
\input{cdocspt4}
\fi
%    \end{macrocode}

% Process as usual until here:
%    \begin{macrocode}
\fi
%    \end{macrocode}

% End of document body:
%    \begin{macrocode}
\end{document}
%    \end{macrocode}
%\iffalse
%</samplemain>
%\fi
%
% %%%%%%%%%%%%%%%%%%%%%%%%%%%%%%%%%%%%%%
% \paragraph{Chapter Include Files.}
%
% The include files are called |cdocsch1.tex| and |cdocsch2.tex|.
%
%\iffalse
%<*samplechap1|samplechap2>
%\fi

% Optional override for |\version| flag:
%    \begin{macrocode}
%%\providecommand{\version}{final}
%    \end{macrocode}

% Include the main document:
%    \begin{macrocode}
\input{childdoc.def}
\childdocof{cdocsamp}
%    \end{macrocode}

%\iffalse
%</samplechap1|samplechap2>
%\fi
%
%\iffalse
%<*samplechap1>
%\fi
% Some text for chapter 1:
%    \begin{macrocode}
\section{one}
some text in chapter one
%    \end{macrocode}

%\iffalse
%</samplechap1>
%\fi
% Some text for chapter 2:
%\iffalse
%<*samplechap2>
%\fi
%    \begin{macrocode}
\section{two}
more text in chapter two
%    \end{macrocode}

%\iffalse
%</samplechap2>
%\fi
%
% %%%%%%%%%%%%%%%%%%%%%%%%%%%%%%%%%%%%%%
% \paragraph{Part Include Files.}
%
% The include files are called |cdocspt3.tex| and |cdocspt4.tex|.
%
%\iffalse
%<*samplepart3|samplepart4>
%\fi

% Optional override for |\version| flag:
%    \begin{macrocode}
%%\providecommand{\version}{final}
%    \end{macrocode}

% Include the main document:
%    \begin{macrocode}
\input{childdoc.def}
\childdocby{cdocsamp}
%    \end{macrocode}

%\iffalse
%</samplepart3|samplepart4>
%\fi
%
%\iffalse
%<*samplepart3>
%\fi
% Some text for part 3:
%    \begin{macrocode}
some text in part three
%    \end{macrocode}

%\iffalse
%</samplepart3>
%\fi
% Some text for part 4:
%\iffalse
%<*samplepart4>
%\fi
%    \begin{macrocode}
more text in part four
%    \end{macrocode}

%\iffalse
%</samplepart4>
%\fi
%
% %%%%%%%%%%%%%%%%%%%%%%%%%%%%%%%%%%%%%%
% \paragraph{Forwarding for a Complete Draft.}
%
% The following forwarding file |cdocsdrf.tex|
% compiles the main document in draft mode:
%\iffalse
%<*sampledraft>
%\fi
%    \begin{macrocode}
\def\version{draft}
\input{childdoc.def}
\childdocforward{cdocsamp}
%    \end{macrocode}

%\iffalse
%</sampledraft>
%\fi
%
% %%%%%%%%%%%%%%%%%%%%%%%%%%%%%%%%%%%%%%
% \paragraph{Forwarding for Final Version of the Chapters.}
%
% The following forwarding files |cdocsfn1.tex| and |cdocsfn2.tex|
% (with identical content)
% compile the final versions of the child documents
% |cdocsch1.tex| and |cdocsch2.tex|, respectively:
%\iffalse
%<*samplefinal>
%\fi
%    \begin{macrocode}
\def\version{final}
\input{childdoc.def}
\childdocforwardprefix[cdocsamp]{cdocsfn}{cdocsch}
%    \end{macrocode}

%\iffalse
%</samplefinal>
%\fi
%
% %%%%%%%%%%%%%%%%%%%%%%%%%%%%%%%%%%%%%%
% \paragraph{Command Line Processing.}
%
% The following three command lines generate the output files
% |cdocscld|, |cdocscl1| and |cdocscl2|
% which should be identical to
% |cdocsdrf|, |cdocsch1| and |cdocsfn2|, respectively:
% \begin{center}
% \begin{tabular}{l}
% |latex -jobname cdocscld \|\\
% |  "\def\version{draft}\input{childdoc.def}\childdocforward{cdocsamp}"|\\
% |latex -jobname cdocscl1 \|\\
% |  "\input{childdoc.def}\childdocforward[cdocsamp]{cdocsch1}"|\\
% |latex -jobname cdocscl2 \|\\
% |  "\def\version{final}\input{childdoc.def}\childdocforward{cdocsch2}"|
% \end{tabular}
% \end{center}
% Note that the trailing backslash on each first line
% merely continues the input to the second line
% (for convenient cut ant paste).
% Furthermore, the command |latex| can be replaced by any
% of its alternative versions such as |pdflatex|.
%
% %%%%%%%%%%%%%%%%%%%%%%%%%%%%%%%%%%%%%%%%%%%%%%%%%%%%%%%%%%%%%%%%%%%%%%%%%%%%%%
% %%%%%%%%%%%%%%%%%%%%%%%%%%%%%%%%%%%%%%%%%%%%%%%%%%%%%%%%%%%%%%%%%%%%%%%%%%%%%%
% \section{Implementation}
%\iffalse
%<*package>
%\fi
%
% This section describes the definitions file |childdoc.def|.

% The definitions cannot be loaded using |\usepackage| or |\RequirePackage|
% which has a mechanism to prevent loading a style file more than once.
% When loading the definitions by means of |\input|
% multiple instances have to be prevented manually:
%\iffalse
%This code needs to be before the `\ProvidesFile' directive
%which is defined at the beginning of this file.
%Therefore it is also placed there and commented out here.
%</package>
%<*discard>
%\fi
%    \begin{macrocode}
\ifdefined\childdocmain\endinput\fi
%    \end{macrocode}
%\iffalse
%</discard>
%<*package>
%\fi
%
% \macro{\ifchilddoc}
% \macro{\ifchilddocmanual}
% The conditional |\ifchilddoc| tells whether a
% child (true) or main (false) document is being compiled.
% The conditional |\ifchilddocmanual| tells whether
% the |\includeonly| mechanism is used (false) or
% the selection of child files must be performed manually (true).
% The definitions initialise to false:
%    \begin{macrocode}
\newif\ifchilddoc
\newif\ifchilddocmanual
%    \end{macrocode}

% \macro{\childdocname}
% \macro{\childdocjob}
% The macro |\childdocname| stores the name of the main document
% to be compiled. The macro |\childdocjob| stores the name of
% the document on which the \LaTeX{} compiler was originally invoked.
% The content of |\jobname| cannot be compared
% to filenames specified in the source due to different catcodes.
% The following code rescans |\jobname|, stores the result
% in |\childdocname| and saves a copy in |\childdocjob|:
%    \begin{macrocode}
\edef\childdocname{\scantokens\expandafter{\jobname\noexpand}}
\let\childdocjob\childdocname
%    \end{macrocode}

% \macro{\childdocdisable}
% The macro |\childdocdisable| prevents the main file
% from being processed more than once.
% At this stage, the main document command |\childdocmain|
% is assumed to be called once again where it should do nothing.
% Any subsequent call to it should prevent
% a secondary processing of the main document
% It overwrites the forwarding commands
% |\childdocof| and |\childdocforward|
% with empty macros to prevent further inclusions of the main document:
%    \begin{macrocode}
\newcommand{\childdocdisable}
{
  \renewcommand{\childdocmain}[1]{\renewcommand{\childdocmain}[1]{\endinput}}
  \renewcommand{\childdocof}[1]{}
  \renewcommand{\childdocby}[2][]{}
  \renewcommand{\childdocforward}[2][]{}
  \renewcommand{\childdocdisable}{}
}
%    \end{macrocode}

% \macro{\childdocmain}
% The macro |\childdocmain| is to be called at the top of the main file
% with nothing or the main filename (without extension) as argument.
% First, it breaks loops.
% If the argument is not empty and does not match |\childdocname|
% (which is set by the first inclusion of |childdoc.def|),
% |\ifchilddoc| is set to true, |\includeonly| is applied to the child file
% and |\jobname| is set to the main file
% (for proper handling of |.aux| files):
%    \begin{macrocode}
\newcommand{\childdocmain}[1]
{
  \childdocdisable\childdocmain{}
  \if?#1?\else
    \begingroup
      \def\childdoctmp{#1}
      \ifx\childdoctmp\childdocname
        \def\childdoctmp{}
      \else
        \def\childdoctmp
        {
          \childdoctrue
          \includeonly{\childdocname}
          \def\childdocjob{#1}
          \def\jobname{#1}
        }
      \fi
      \expandafter
    \endgroup
    \childdoctmp
  \fi
}
%    \end{macrocode}

% \macro{\childdocof}
% The command |\childdocof| redirects
% compilation to the main file |#1|.
%    \begin{macrocode}
\newcommand{\childdocof}[1]
{
  \childdocdisable
  \childdoctrue
  \includeonly{\childdocname}
  \def\jobname{#1}
  \def\childdocjob{#1}
  \input{#1}
}
%    \end{macrocode}

% \macro{\childdocby}
% The command |\childdocby| ....
%    \begin{macrocode}
\newcommand{\childdocby}[2][]
{
  \childdocdisable
  \childdoctrue
  \childdocmanualtrue
  \if?#1?\else
    \def\jobname{#2}
  \fi
  \def\childdocjob{#2}
  \input{#2}
  \endinput
}
%    \end{macrocode}

% \macro{\childdocforward}
% The command |\childdocforward| redirects
% compilation to the main file or
% (if the optional argument is given) a child file.
% Parameters are set as if the main file
% or a child file starting with |\childdocof| was compiled.
% Then compilation is handed over to the main file:
%    \begin{macrocode}
\newcommand{\childdocforward}[2][]
{
  \begingroup
    \if?#1?
      \def\childdoctmp
      {
        \def\childdocname{#2}
        \def\childdocjob{#2}
        \def\jobname{#2}
        \input{#2}
        \endinput
      }
    \else
      \def\childdoctmp
      {
        \childdocdisable
        \def\childdocname{#2}
        \childdoctrue
        \includeonly{#2}
        \def\childdocjob{#1}
        \def\jobname{#1}
        \input{#1}
        \endinput
      }
    \fi
    \expandafter
  \endgroup
  \childdoctmp
}
%    \end{macrocode}

% \macro{\childdocforwardprefix}
% The command |\childdocforwardprefix| redirects
% compilation to the main or a child file by means of a pattern.
% The prefix |#1| in the current filename is replaced by |#2|
% and the suffix of the current filename is kept
% (it is assumed that the filename does not contain the substring `|~~~|'
% which is used as a delimiter).
% Compilation is handed over to the new file by |\childdocforward|:
%    \begin{macrocode}
\newcommand{\childdocforwardprefix}[3][]
{
  \begingroup
    \def\childdocextract #2##1~~~{\def\childdoctmp{\childdocforward[#1]{#3##1}}}
    \expandafter\childdocextract\childdocname~~~
    \expandafter
  \endgroup
  \childdoctmp
}
%    \end{macrocode}

% \macro{\childdoc}
% The deprecated macro |\childdoc| is a legacy version of |\childdocmain|:
%    \begin{macrocode}
\newcommand{\childdoc}{\childdocmain}
%    \end{macrocode}

% \macro{\childdocredirect}
% The deprecated macro |\childdocredirect| is a legacy version
% of |\childdocforward| and |\childdocforwardprefix|:
%    \begin{macrocode}
\newcommand{\childdocredirect}[2][]
{
  \begingroup
    \if?#1?
      \def\childdoctmp{\childdocforward{#2}}
    \else
      \def\childdoctmp{\childdocforwardprefix{#1}{#2}}
    \fi
    \expandafter
  \endgroup
  \childdoctmp
}
%    \end{macrocode}

%\iffalse
%</package>
%\fi
%
\endinput
|\\
|\childdocby{|\textit{main}|}|\\
\end{tabular}
\end{center}
%
The directive |\childdocby| is similar to |\childdocof|
described in \secref{sec:include},
but the subsequent selection of content must be done manually.
To that end, both |\ifchilddoc| and |\ifchilddocmanual|
will be true upon processing of a part,
and the name of the part is stored in |\childdocname|.
Note that |\jobname| will be set to the filename of the current part
so that each part receives an individual |.aux| file
that does not interfere with the |.aux| file(s) of the main document.
This behaviour can be altered by the alternative form
|\childdocby[*]{|\textit{main}|}| (with a non-empty optional argument)
which uses the |.aux| file of the main document
by setting |\jobname| to \textit{main}.

%%%%%%%%%%%%%%%%%%%%%%%%%%%%%%%%%%%%%%%%%%%%%%%%%%%%%%%%%%%%%%%%%%%%%%%%%%%%%%%%
\subsection{Driver Development}
\label{sec:driver}

The \textsf{childdoc} mechanism can also be use for the development
of definition files such as \LaTeX{} styles or classes.
This case differs from the above setup with multiple parts
included by |\include| in that no |\includeonly| should be invoked.
This can be achieved by starting the include file
(before |\ProvidesPackage|) with:
%
\begin{center}
\begin{tabular}{l}
|% \iffalse
%
% childdoc.dtx Copyright (C) 2017-2018 Niklas Beisert
%
% This work may be distributed and/or modified under the
% conditions of the LaTeX Project Public License, either version 1.3
% of this license or (at your option) any later version.
% The latest version of this license is in
%   http://www.latex-project.org/lppl.txt
% and version 1.3 or later is part of all distributions of LaTeX
% version 2005/12/01 or later.
%
% This work has the LPPL maintenance status `maintained'.
%
% The Current Maintainer of this work is Niklas Beisert.
%
% This work consists of the files childdoc.dtx and childdoc.ins
% and the derived files childdoc.def and cdocsamp.tex with
% cdocsch1.tex, cdocsch2.tex, cdocsdrf.tex, cdocsfn1.tex, cdocsfn2.tex.
%
%<package>\ifdefined\childdocmain\endinput\fi
%<package>\ProvidesFile{childdoc.def}[2018/12/30 v2.0 child document driver]
%<samplemain>\ProvidesFile{cdocsamp.tex}[2018/12/30 v2.0 sample for childdoc]
%<*driver>
%\ProvidesFile{childdoc.drv}[2018/12/30 v2.0 childdoc reference manual file]
\PassOptionsToClass{10pt,a4paper}{article}
\documentclass{ltxdoc}

\usepackage[margin=35mm]{geometry}
\usepackage{hyperref}
\usepackage{hyperxmp}
\usepackage[usenames]{color}

\hypersetup{colorlinks=true}
\hypersetup{pdfstartview=FitH}
\hypersetup{pdfpagemode=UseNone}
\hypersetup{pdfsource={}}
\hypersetup{pdflang={en-UK}}
\hypersetup{pdfcopyright={Copyright 2017-2018 Niklas Beisert.
  This work may be distributed and/or modified under the
  conditions of the LaTeX Project Public License, either version 1.3
  of this license or (at your option) any later version.}}
\hypersetup{pdflicenseurl={http://www.latex-project.org/lppl.txt}}
\hypersetup{pdfcontactaddress={ETH Zurich, ITP, HIT K,
  Wolfgang-Pauli-Strasse 27}}
\hypersetup{pdfcontactpostcode={8093}}
\hypersetup{pdfcontactcity={Zurich}}
\hypersetup{pdfcontactcountry={Switzerland}}
\hypersetup{pdfcontactemail={nbeisert@itp.phys.ethz.ch}}
\hypersetup{pdfcontacturl={http://people.phys.ethz.ch/\xmptilde nbeisert/}}

\newcommand{\secref}[1]{\hyperref[#1]{section \ref*{#1}}}

\parskip1ex
\parindent0pt
\let\olditemize\itemize
\def\itemize{\olditemize\parskip0pt}

\begin{document}

\title{The \textsf{childdoc} Package}
\hypersetup{pdftitle={The childdoc Package}}
\author{Niklas Beisert\\[2ex]
  Institut f\"ur Theoretische Physik\\
  Eidgen\"ossische Technische Hochschule Z\"urich\\
  Wolfgang-Pauli-Strasse 27, 8093 Z\"urich, Switzerland\\[1ex]
  \href{mailto:nbeisert@itp.phys.ethz.ch}
  {\texttt{nbeisert@itp.phys.ethz.ch}}}
\hypersetup{pdfauthor={Niklas Beisert}}
\hypersetup{pdfsubject={Manual for the LaTeX2e Package childdoc}}
\date{30 December 2018, \textsf{v2.0}}
\maketitle

\begin{abstract}\noindent
\textsf{childdoc} is a \LaTeXe{} package
that enables the direct compilation
of document sections included by |\include|
to individual files.
\end{abstract}

\begingroup
\parskip0ex
\tableofcontents
\endgroup

%%%%%%%%%%%%%%%%%%%%%%%%%%%%%%%%%%%%%%%%%%%%%%%%%%%%%%%%%%%%%%%%%%%%%%%%%%%%%%%%
%%%%%%%%%%%%%%%%%%%%%%%%%%%%%%%%%%%%%%%%%%%%%%%%%%%%%%%%%%%%%%%%%%%%%%%%%%%%%%%%
\section{Introduction}

\LaTeX{} provides a mechanism to structure a large document (such as a book)
into a main file and several child files (containing the chapters)
using the |\include| command.
This mechanism is beneficial for documents
which span hundreds of pages in order to
make the source file(s) more manageable.
Moreover, compilation can be restricted to
selected child files by means of the |\includeonly| command.
The latter feature can be used to reduce the compilation time while editing
(this was significantly more useful in the earlier days of \LaTeX{})
or to generate a smaller document which is easier to navigate.
Another application of |\includeonly| is to generate
documents consisting of selected parts of the complete document.

However, there are a few drawbacks of the plain |\include| mechanism:
\begin{itemize}
\item
The child files cannot be compiled on their own,
they can only be compiled via the main file.
A naive editing environment
(such as a text editor with an option
to have the current file processed by \LaTeX)
may require one to switch to the main file before compiling;
attempting to compile the child file produces errors.
\item
The main file must be modified (each time)
to adjust the |\includeonly| command
to the present needs. This easily leaves the main file in a messy state.
\item
The generated document will always carry the filename
of the main document. This is inconvenient if
several child files are to be compiled and
to be kept for distribution.
\end{itemize}

The present package provides a simple interface
to make child files individually compilable by \LaTeX{}.
Compiling a child file then has the same effect as compiling
the main file with an |\includeonly| command
to select the appropriate child.
Moreover the generated document will carry the name of the child
rather than the main file.
This resolves all three above issues.

This feature is meant to make the editing of books,
thesis documents and lecture notes somewhat more convenient.
However, the package can also be used efficiently for
composing a series of documents (such as exercise sheets)
which are typically distributed individually.
It then assists the author in generating the individual documents
(potentially in different versions)
as well as a document containing the collected series.
Another application is in developing style files
or other kinds of included material
where compilation of the style file could redirect
to a sample or test file.

%%%%%%%%%%%%%%%%%%%%%%%%%%%%%%%%%%%%%%%%%%%%%%%%%%%%%%%%%%%%%%%%%%%%%%%%%%%%%%%%
%%%%%%%%%%%%%%%%%%%%%%%%%%%%%%%%%%%%%%%%%%%%%%%%%%%%%%%%%%%%%%%%%%%%%%%%%%%%%%%%
\section{Usage}

First of all, the package \textsf{childdoc} is \emph{not} a standard
\LaTeXe{} |.sty| style file! Therefore it needs to be invoked in
a non-standard way.

%%%%%%%%%%%%%%%%%%%%%%%%%%%%%%%%%%%%%%%%%%%%%%%%%%%%%%%%%%%%%%%%%%%%%%%%%%%%%%%%
\subsection{Included Files}
\label{sec:include}

%%%%%%%%%%%%%%%%%%%%%%%%%%%%%%%%%%%%%%%%
\DescribeMacro{\childdocmain}
To use the package, add the commands
\begin{center}
\begin{tabular}{l}
|\input{childdoc.def}|\\
|\childdocmain{}|\\
\end{tabular}
\end{center}
at the very top of the main \LaTeX{} file,
in particular \emph{before} the |\documentclass| statement!
The argument of |\childdocmain| should be left empty
(but it must be present).

%%%%%%%%%%%%%%%%%%%%%%%%%%%%%%%%%%%%%%%%
\DescribeMacro{\childdocof}
Furthermore, add the commands
\begin{center}
\begin{tabular}{l}
|\input{childdoc.def}|\\
|\childdocof{|\textit{main}|}|\\
\end{tabular}
\end{center}
at the top of every child file \textit{child}
which is included by |\include{|\textit{child}|}|
from within the main file
(or at least for those files to be compiled individually).
The argument \textit{main} must be the filename of the main file.

There are a couple of
considerations in setting up the main and child documents:

%%%%%%%%%%%%%%%%%%%%%%%%%%%%%%%%%%%%%%%%
\paragraph{Restrictions.}

Please note the following restrictions:
\begin{itemize}
\item
|\childdocmain| must be called with one argument \textit{main}
to ensure compatibility with earlier version of the package.
It must either be empty (|\childdocmain{}|)
or precisely match the filename of the main file in which it is specified.
See \secref{sec:detection} for further information.
\item
The filename \textit{main} must be specified without the |.tex| extension.
\item
The filename \textit{main} is case sensitive
(even in case-insensitive file systems)
due to internal string comparison.
\item
The argument \textit{main} should be fully expanded, it cannot be a macro.
\item
Subdirectories and special characters should be avoided in filenames.
\item
The command |\childdocmain{|\textit{main}|}| must be followed by a whitespace.
It should not be followed immediately by another command
or by a comment mark `|%|'.
This is because the \TeX{} parser reads the token immediately following
the argument of |\childdocmain| and puts it
at the beginning of every child section;
however, a white\-space is ignored.
\end{itemize}

%%%%%%%%%%%%%%%%%%%%%%%%%%%%%%%%%%%%%%%%
\paragraph{Content of Main File.}

It is advisable to place all content in the child files included by |\include|.
Any output contained in the main file will appear in all child documents
unless suppressed manually;
it cannot be suppressed automatically by the |\includeonly| directive
and thus should normally be avoided.
A method to include some content in the main file
by means of conditional processing is described in \secref{sec:conditional}.

%%%%%%%%%%%%%%%%%%%%%%%%%%%%%%%%%%%%%%%%
\paragraph{Page Numbering.}

When only a part of the document is compiled,
the appropriate numbering of pages
(as well as other status parameters)
is determined from the |.aux| files.
The latter contain information from previous passes.
However this information needs to propagate through
all intermediate child documents.
Therefore the page numbering in child documents may well
be inconsistent until the complete document is compiled at least once.

A useful (if unconventional) way to always ensure a consistent
page numbering is to restart the numbering in each child document
and denote the pages by `\textit{child}|.|\textit{page}'
where \textit{child} represents the chapter/section number of the child file.
This can be achieved by the command
|\numberwithin{page}{|\textit{child}|}|
of the \textsf{amsmath} package
where \textit{child} can be |chapter| or |section|
depending on the chosen structuring.
Alternatively, one can modify the macro |\thepage| appropriately
and reset the counter |page| at the start of each child file.

%%%%%%%%%%%%%%%%%%%%%%%%%%%%%%%%%%%%%%%%%%%%%%%%%%%%%%%%%%%%%%%%%%%%%%%%%%%%%%%%
\subsection{Conditional Processing}
\label{sec:conditional}

The package provides a mechanism to compile different versions
of a document. To customise the versions further some conditional processing
can come in handy to distinguish which version is being compiled.
The package provides two macros to describe the compilation context:

%%%%%%%%%%%%%%%%%%%%%%%%%%%%%%%%%%%%%%%%
\DescribeMacro{\ifchilddoc}
The conditional |\ifchilddoc| distinguishes between the compilation of
child documents and the main document:
%
\begin{center}
|\ifchilddoc |\textit{child-code}| |[|\||else |\textit{main-code}]| \||fi|
\end{center}

%%%%%%%%%%%%%%%%%%%%%%%%%%%%%%%%%%%%%%%%
\DescribeMacro{\childdocname}
\DescribeMacro{\childdocjob}
The macro |\childdocname| contains the filename (without extension)
of the main or child file being processed.
Note that |\childdocjob| will always contain the name of the main file.

%%%%%%%%%%%%%%%%%%%%%%%%%%%%%%%%%%%%%%%%
\paragraph{Title Page.}

Conditional processing can be used to include a title or banner page
in the main document when proper precautions are taken.
Importantly, the code in the main file should ensure that the page counter
(as well as other status parameters which are stored in the |.aux| files)
takes the same value after the conditional processing.
Otherwise the page numbers may take divergent values
depending on which part is compiled.

For example, a title page could be declared by:
%
\begin{center}
\begin{tabular}{l}
|\ifchilddoc\||else|\\
|\addtocounter{page}{-1}|\\
\textit{code for title page}\\
|\newpage|\\
|\||fi|
\end{tabular}
\end{center}
%
A banner page for the child documents can be generated by:
%
\begin{center}
\begin{tabular}{l}
|\ifchilddoc|\\
|\addtocounter{page}{-1}|\\
\textit{code for banner page}\\
|\newpage|\\
|\||fi|
\end{tabular}
\end{center}
%
Here one could write a message such as:
\begin{center}
|This is the part \childdocname{} of \childdocjob{}.|
\end{center}

%%%%%%%%%%%%%%%%%%%%%%%%%%%%%%%%%%%%%%%%%%%%%%%%%%%%%%%%%%%%%%%%%%%%%%%%%%%%%%%%
\subsection{Flags}
\label{sec:flags}

The package makes it easy to generate different versions
of the main or child documents.
To this end compilation flags can be defined
and assigned different default values.
They will be particularly useful in conjunction
with the forwarding mechanism described in \secref{sec:forward}.

For example, it may be useful to have a flag |\version|
which can be set to |draft| or |final|.
The document source will contain some conditional code
depending on the value of |\version|.
Suppose further, the flag should default to |final| for the main file
and to |draft| for child files
which is a natural assignment for editing the document.
This is achieved by placing the following code
in the preamble of the main document
(below the |\childdocmain| directive):
%
\begin{center}
\begin{tabular}{l}
|\ifchilddoc|\\
|\providecommand{\version}{draft}|\\
|\||else|\\
|\providecommand{\version}{final}|\\
|\||fi|
\end{tabular}
\end{center}
%
The definition by |\providecommand| makes sure
that previous definitions are not overwritten.
Further statements |\providecommand{\version}{...}|
can thus be added before the above code to override it.

For the main file, one might add a line
(between |\childdocmain| and the above block)
%
\begin{center}
|%\ifchilddoc\||else\providecommand{\version}{draft}\||fi|
\end{center}
%
which can be uncommented to produce a draft version.
Likewise one can add a line to the very top of a child file
(above the |\childdocof{|\textit{main}|}| directive)
%
\begin{center}
|%\providecommand{\version}{final}|
\end{center}
%
which can be uncommented to produce the final version of this child document.

%%%%%%%%%%%%%%%%%%%%%%%%%%%%%%%%%%%%%%%%%%%%%%%%%%%%%%%%%%%%%%%%%%%%%%%%%%%%%%%%
\subsection{Forwarding}
\label{sec:forward}

Different versions of the main or child documents
using compilation flags as described in \secref{sec:flags}
can be (permanently) stored in different files
for convenient compilation, viewing and distribution.
To this end, the package defines a command
to pass on compilation to a different file:

%%%%%%%%%%%%%%%%%%%%%%%%%%%%%%%%%%%%%%%%
\DescribeMacro{\childdocforward}
The command |\childdocforward| redirects processing to
another source file:
%
\begin{center}
\begin{tabular}{l}
|\input{childdoc.def}|\\
|\childdocforward[|\textit{main}|]{|\textit{dest}|}|\\
\end{tabular}
\end{center}
%
The argument \textit{dest} is the destination file
(without extension).
It should be the main file or one of the child files.
Note that further \textsf{childdoc} directives
such as |\childdocof| and |\childdocforward|
in the indicated file will be processed in this form.
The optional argument \textit{main}
passes on directly to the main file \textit{main}
while pretending to compile the child \textit{dest}.
This form behaves as if \textit{dest}
issues |\childdocof{|\textit{main}|}| right away,
and no further \textsf{childdoc} directives will be processed.

%%%%%%%%%%%%%%%%%%%%%%%%%%%%%%%%%%%%%%%%
\DescribeMacro{\...prefix}
In the alternative form |\childdocforwardprefix|,
%
\begin{center}
\begin{tabular}{l}
|\input{childdoc.def}|\\
|\childdocforwardprefix[|\textit{main}|]{|\textit{prefix}|}{|\textit{dest}|}|
\end{tabular}
\end{center}
%
the destination file is determined by a pattern
depending on the current file:
To make this work, the current file must be called
`{\textit{prefix}\hspace{0.2em}\textit{suffix}}'
with \textit{prefix} matching precisely the argument.
Processing is then passed on to the file
`{\textit{dest}\hspace{0.2em}\textit{suffix}}'.
Surely, the same effect is achieved by
directly specifying the
argument `{\textit{dest}\hspace{0.2em}\textit{suffix}}'
in the first form.
However, that requires to set up a different file
for each child. With the alternative form of the command
all these files can have exactly the same content
which simplifies setting them up and maintaining them.

For example, the following file |draft.tex|
with a compilation flag |\version| as described in \secref{sec:flags}
compiles the main document as a draft:
%
\begin{center}
\begin{tabular}{l}
|\def\version{draft}|\\
|\input{childdoc.def}|\\
|\childdocforward{|\textit{main}|}|
\end{tabular}
\end{center}
%
Likewise, the following files |final|\textit{nn}|.tex|
compile the final version of the child document
|child|\textit{nn}|.tex|:
%
\begin{center}
\begin{tabular}{l}
|\def\version{final}|\\
|\input{childdoc.def}|\\
|\childdocforwardprefix{final}{child}|
\end{tabular}
\end{center}
%

Note that when several versions of a main file and/or of each child file
are to be generated, it may be convenient to set up a |Makefile| or
shell script to automatise the process.

%%%%%%%%%%%%%%%%%%%%%%%%%%%%%%%%%%%%%%%%%%%%%%%%%%%%%%%%%%%%%%%%%%%%%%%%%%%%%%%%
\subsection{Command Line Processing}
\label{sec:commandline}

The effect of redirection files can also be achieved by invoking
the \LaTeX{} compiler with a more elaborate command line.
Most conveniently this should be done as part
of a shell script or a |Makefile|.

When using \textsf{childdoc} in the main file, the following
command lines effectively perform a redirection
(note that depending on the shell being used,
backslashes may have to be doubled: `|\|' $\to$ `|\\|'):
%
\begin{center}
|... -jobname "|\textit{target}|" |\\|"|[\textit{flags}]%
|\input{childdoc.def}\childdocforward[|\textit{main}|]{|\textit{dest}|}"|
\end{center}
%
Here \textit{target} is the name of the output file,
\textit{main} is the name of the main file
and \textit{dest} is the name of the main or child file to be processed
(all filenames without extensions).
The optional argument \textit{main} can be omitted
if \textit{main} matches \textit{dest}.
Optionally, compilation \textit{flags} can be defined via |\def| commands.
This command line makes the \TeX{} engine believe
it is compiling the file \textit{target}
whose content is specified as the latter parameter.
The provided code then forwards the processing to
\textit{main} or \textit{dest} as described in \secref{sec:forward}.

%%%%%%%%%%%%%%%%%%%%%%%%%%%%%%%%%%%%%%%%%%%%%%%%%%%%%%%%%%%%%%%%%%%%%%%%%%%%%%%%
\subsection{Include by Input}
\label{sec:input}

Including child documents by |\include| has some restrictions by design.
Most notably, the content of a child document always occupies
its own set of pages; pages cannot be shared between child documents.
Usually, this behaviour makes perfect sense
because each child document contain an essential part of the document.
However, in some situations it may be desirable to compose
a document from a collection of parts
without having mandatory page breaks between then.
For this case, the package
provides a mechanism to include parts
by |\input| which can also be processed individually.
However, by construction this mechanism
requires manual handling of the content to be output.

%%%%%%%%%%%%%%%%%%%%%%%%%%%%%%%%%%%%%%%%
\DescribeMacro{\ifchilddocmanual}
The main file should be prepared as usual, see \secref{sec:include}.
However, the document body must make a distinction
between processing of an individual part and of the main document, e.g.:
%
\begin{center}
\begin{tabular}{l}
|\ifchilddocmanual|\\
|\input{\childdocname}|\\
|\||else|\\
\textit{document body with }|\input{|\textit{part}|}|\\
|\||fi|
\end{tabular}
\end{center}
%
The conditional |\ifchilddocmanual| is true whenever
a part to be included by |\input| is being compiled,
and the name of the part is stored in |\childdocname|.

%%%%%%%%%%%%%%%%%%%%%%%%%%%%%%%%%%%%%%%%
\DescribeMacro{\childdocby}
Each part to be included by |\input| should start with:
%
\begin{center}
\begin{tabular}{l}
|\input{childdoc.def}|\\
|\childdocby{|\textit{main}|}|\\
\end{tabular}
\end{center}
%
The directive |\childdocby| is similar to |\childdocof|
described in \secref{sec:include},
but the subsequent selection of content must be done manually.
To that end, both |\ifchilddoc| and |\ifchilddocmanual|
will be true upon processing of a part,
and the name of the part is stored in |\childdocname|.
Note that |\jobname| will be set to the filename of the current part
so that each part receives an individual |.aux| file
that does not interfere with the |.aux| file(s) of the main document.
This behaviour can be altered by the alternative form
|\childdocby[*]{|\textit{main}|}| (with a non-empty optional argument)
which uses the |.aux| file of the main document
by setting |\jobname| to \textit{main}.

%%%%%%%%%%%%%%%%%%%%%%%%%%%%%%%%%%%%%%%%%%%%%%%%%%%%%%%%%%%%%%%%%%%%%%%%%%%%%%%%
\subsection{Driver Development}
\label{sec:driver}

The \textsf{childdoc} mechanism can also be use for the development
of definition files such as \LaTeX{} styles or classes.
This case differs from the above setup with multiple parts
included by |\include| in that no |\includeonly| should be invoked.
This can be achieved by starting the include file
(before |\ProvidesPackage|) with:
%
\begin{center}
\begin{tabular}{l}
|\input{childdoc.def}|\\
|\childdocforward{|\textit{main}|}|\\
\end{tabular}
\end{center}
%
or alternatively with:
%
\begin{center}
\begin{tabular}{l}
|\input{childdoc.def}|\\
|\childdocby{|\textit{main}|}|\\
\end{tabular}
\end{center}
%
Both forms have slightly different effects as described above.
The main file is prepared as usual, see \secref{sec:include}.

%%%%%%%%%%%%%%%%%%%%%%%%%%%%%%%%%%%%%%%%%%%%%%%%%%%%%%%%%%%%%%%%%%%%%%%%%%%%%%%%
\subsection{Legacy Detection}
\label{sec:detection}

The directive |\childdocmain| in the main file can detect
whether the complete document or merely a child is to be compiled
even without using the directive |\childdocof|.
This method is deprecated because it is less robust
and there is no compelling reason to use it;
it is merely provided for backward compatibility
and it may be removed in future versions.

If the detection mechanism is to be used,
it is mandatory to correctly specify
the filename of the main file as the argument of |\childdocmain|:
%
\begin{center}
\begin{tabular}{l}
|\input{childdoc.def}|\\
|\childdocmain{|\textit{main}|}|\\
\end{tabular}
\end{center}
%
If |\jobname| does not match the argument \textit{main} of |\childdocmain|,
it is assumed that |\jobname| points to the child file to be compiled.
When using |\childdocmain| with the main file specified as argument,
it suffices to start a child file
with just |\input{|\textit{main}|}|
without loading of the package and using |\childdocof|.
If instead all processing is done
with the appropriate \textsf{childdoc} directives,
the argument of \textit{main} of |\childdocmain| can be empty.

An alternative version of the command line processing described
in \secref{sec:commandline} using the detection mechanism reads:
%
\begin{center}
|... -jobname "|\textit{target}|" "|[\textit{flags}]%
[|\def\jobname{|\textit{dest}|}|]|\input{|\textit{main}|}"|
\end{center}

%%%%%%%%%%%%%%%%%%%%%%%%%%%%%%%%%%%%%%%%%%%%%%%%%%%%%%%%%%%%%%%%%%%%%%%%%%%%%%%%
\subsection{Manual Code}
\label{sec:manual}

In case one cannot be certain whether the definitions file |childdoc.def|
is installed on the target \TeX{} distribution
and one prefers not to ship it,
it is conceivable to paste a few relevant commands into the sources.

To that end, drop all statements |\input{childdoc.def}|
and perform the replacements as outlined below.
Instead of |\childdocmain{|\textit{main}|}| add the following code
to the top of the main file:
%
\begin{center}
\begin{tabular}{l}
|\||ifdefined\childdocname\endinput\||fi\newif\ifchilddoc|\\
|\edef\childdocname{\scantokens\expandafter{\jobname\noexpand}}|\\
|\def\childdocmain{|\textit{main}|}\||ifx\childdocmain\childdocname\||else|\\
|\childdoctrue\includeonly{\childdocname}\let\jobname\childdocmain\||fi|\\
\end{tabular}
\end{center}
%
Instead of |\childdocof{|\textit{main}|}| just include the main file
at the top of each child file:
%
\begin{center}
|\input{|\textit{main}|}|
\end{center}
%
A simple redirection |\childdocforward{|\textit{dest}|}| is achieved by:
%
\begin{center}
|\def\jobname{|\textit{dest}|}\input{\jobname}|
\end{center}
%
The redirection with prefix
|\childdocforwardprefix[|\textit{prefix}|]{|\textit{dest}|}|
is accomplished by:
%
\begin{center}
\begin{tabular}{l}
|{\edef\jobname{\scantokens\expandafter{\jobname\noexpand}}|\\
|\def\redirectjob |\textit{prefix}|#1~~~{\gdef\jobname{|\textit{dest}|#1}}|\\
|\expandafter\redirectjob\jobname~~~}\input{\jobname}|
\end{tabular}
\end{center}

In an alternative approach,
child documents can be compiled by a specific command line
without additional code or specific definitions:
%
\begin{center}
|... -jobname "|\textit{target}|" "|[\textit{flags}]%
|\includeonly{|\textit{dest}|}\input{|\textit{main}|}"|
\end{center}
%

%%%%%%%%%%%%%%%%%%%%%%%%%%%%%%%%%%%%%%%%%%%%%%%%%%%%%%%%%%%%%%%%%%%%%%%%%%%%%%%%
%%%%%%%%%%%%%%%%%%%%%%%%%%%%%%%%%%%%%%%%%%%%%%%%%%%%%%%%%%%%%%%%%%%%%%%%%%%%%%%%
\section{Information}

%%%%%%%%%%%%%%%%%%%%%%%%%%%%%%%%%%%%%%%%%%%%%%%%%%%%%%%%%%%%%%%%%%%%%%%%%%%%%%%%
\subsection{Copyright}

Copyright \copyright{} 2017--2018 Niklas Beisert

This work may be distributed and/or modified under the
conditions of the \LaTeX{} Project Public License, either version 1.3
of this license or (at your option) any later version.
The latest version of this license is in
  \url{http://www.latex-project.org/lppl.txt}
and version 1.3 or later is part of all distributions of \LaTeX{}
version 2005/12/01 or later.

This work has the LPPL maintenance status `maintained'.

The Current Maintainer of this work is Niklas Beisert.

This work consists of the files |README.txt|, |childdoc.ins| and |childdoc.dtx|
as well as the derived files |childdoc.def|, |cdocsamp.tex|
with |cdocsch1.tex|, |cdocsch2.tex|, |cdocspt3.tex|, |cdocspt4.tex|,
|cdocsdrf.tex|, |cdocsfn1.tex|, |cdocsfn2.tex|
as well as |childdoc.pdf|.

%%%%%%%%%%%%%%%%%%%%%%%%%%%%%%%%%%%%%%%%%%%%%%%%%%%%%%%%%%%%%%%%%%%%%%%%%%%%%%%%
\subsection{Files and Installation}

The package consists of the files:
%
\begin{center}
\begin{tabular}{ll}
    |README.txt|   & readme file \\
    |childdoc.ins| & installation file \\
    |childdoc.dtx| & source file \\
    |childdoc.def| & definition file \\
    |cdocsamp.tex| & sample main file \\
    |cdocsch1.tex| & sample include file \\
    |cdocsch2.tex| & sample include file \\
    |cdocspt3.tex| & sample part file \\
    |cdocspt4.tex| & sample part file \\
    |cdocsdrf.tex| & sample redirection file \\
    |cdocsfn1.tex| & sample redirection file \\
    |cdocsfn2.tex| & sample redirection file \\
    |childdoc.pdf| & manual
\end{tabular}
\end{center}
%
The distribution consists of the files
|README.txt|, |childdoc.ins| and |childdoc.dtx|.
%
\begin{itemize}
\item
Run (pdf)\LaTeX{} on |childdoc.dtx|
to compile the manual |childdoc.pdf| (this file).
\item
Run \LaTeX{} on |childdoc.ins| to create the definitions file |childdoc.def|
and the sample |cdocsamp.tex| with include files
|cdocsch1.tex|, |cdocsch2.tex|, |cdocspt3.tex|, |cdocspt4.tex|,
|cdocsdrf.tex|, |cdocsfn1.tex|, |cdocsfn2.tex|.
Then copy the file |childdoc.def| to an appropriate directory of your \LaTeX{}
distribution, e.g.\ \textit{texmf-root}|/tex/latex/childdoc|.
\end{itemize}

%%%%%%%%%%%%%%%%%%%%%%%%%%%%%%%%%%%%%%%%%%%%%%%%%%%%%%%%%%%%%%%%%%%%%%%%%%%%%%%%
\subsection{Related CTAN Packages}

There are several other packages which offer a similar functionality:
%
\begin{itemize}
\item
The packages
\href{http://ctan.org/pkg/docmute}{\textsf{docmute}},
\href{http://ctan.org/pkg/includex}{\textsf{includex}} and
\href{http://ctan.org/pkg/standalone}{\textsf{standalone}}
provide commands to include only the document body of
a child file thus allowing both files to be compiled individually.
\item
The packages \href{http://ctan.org/pkg/subdocs}{\textsf{subdocs}}
and \href{http://ctan.org/pkg/subfiles}{\textsf{subfiles}}
provide structures in which the main and child documents can be
encapsulated and allowing them to be compiled individually.
The inclusion mechanism is different from the conventional |\include|.
\item
The package \href{http://ctan.org/pkg/combine}{\textsf{combine}}
is an elaborate solution to combine several documents into one.
\end{itemize}
%
See also the CTAN topic \href{http://ctan.org/topic/subdocs}{\textsf{subdocs}}
for further related packages.
The present package differs from the above solutions in that
a document structure constructed with the conventional |\include| mechanism
just needs two extra commands at the top of every file
such that all constituent files can be compiled individually.

%%%%%%%%%%%%%%%%%%%%%%%%%%%%%%%%%%%%%%%%%%%%%%%%%%%%%%%%%%%%%%%%%%%%%%%%%%%%%%%%
%\subsection{Feature Suggestions}
%
%The following is a list of features which may be useful for future
%versions of this package:
%%
%\begin{itemize}
%\item
%\ldots
%\end{itemize}

%%%%%%%%%%%%%%%%%%%%%%%%%%%%%%%%%%%%%%%%%%%%%%%%%%%%%%%%%%%%%%%%%%%%%%%%%%%%%%%%
\subsection{Revision History}

%%%%%%%%%%%%%%%%%%%%%%%%%%%%%%%%%%%%%%%%
\paragraph{v2.0:} 2018/12/30

\begin{itemize}
\item
immediate forward processing
\item
added |\childdocby| mechanism
\item
manual restructured
\end{itemize}

%%%%%%%%%%%%%%%%%%%%%%%%%%%%%%%%%%%%%%%%
\paragraph{v1.6:} 2018/01/17

\begin{itemize}
\item
application for development of include files
\item
corrections to manual
\end{itemize}

%%%%%%%%%%%%%%%%%%%%%%%%%%%%%%%%%%%%%%%%
\paragraph{v1.5:} 2017/05/21

\begin{itemize}
\item
more complete structuring introduced
\item
|\childdocof| introduced
\item
|\childdoc| renamed to |\childdocmain|
\item
|\childredirect| renamed to |\childdocforward| and |\childdocforwardprefix|
and functionality expanded
\end{itemize}

%%%%%%%%%%%%%%%%%%%%%%%%%%%%%%%%%%%%%%%%
\paragraph{v1.0:} 2017/04/27

\begin{itemize}
\item
manual and install package
\item
first version published on CTAN
\end{itemize}

%%%%%%%%%%%%%%%%%%%%%%%%%%%%%%%%%%%%%%%%
\paragraph{v0.6:} 2017/04/26

\begin{itemize}
\item
redirection mechanism added
\end{itemize}

%%%%%%%%%%%%%%%%%%%%%%%%%%%%%%%%%%%%%%%%
\paragraph{v0.5:} 2017/04/26

\begin{itemize}
\item
functionality in definition file
\end{itemize}


%%%%%%%%%%%%%%%%%%%%%%%%%%%%%%%%%%%%%%%%%%%%%%%%%%%%%%%%%%%%%%%%%%%%%%%%%%%%%%%%
%%%%%%%%%%%%%%%%%%%%%%%%%%%%%%%%%%%%%%%%%%%%%%%%%%%%%%%%%%%%%%%%%%%%%%%%%%%%%%%%
%%%%%%%%%%%%%%%%%%%%%%%%%%%%%%%%%%%%%%%%%%%%%%%%%%%%%%%%%%%%%%%%%%%%%%%%%%%%%%%%
\appendix

\settowidth\MacroIndent{\rmfamily\scriptsize 000\ }

 \DocInput{childdoc.dtx}

\end{document}
%</driver>
% \fi
%
% %%%%%%%%%%%%%%%%%%%%%%%%%%%%%%%%%%%%%%%%%%%%%%%%%%%%%%%%%%%%%%%%%%%%%%%%%%%%%%
% %%%%%%%%%%%%%%%%%%%%%%%%%%%%%%%%%%%%%%%%%%%%%%%%%%%%%%%%%%%%%%%%%%%%%%%%%%%%%%
% \section{Sample}
%\iffalse
%<*samplemain>
%\fi
%
% The following presents a sample document
% with two chapters, two parts, a title page,
% a compile flag as well as three forwarding files to set the flag.
% It consists of eight |.tex| files:
% \begin{center}
% \begin{tabular}{ll}
% |cdocsamp.tex|&main file\\
% |cdocsch1.tex|&include file for chapter 1\\
% |cdocsch2.tex|&include file for chapter 2\\
% |cdocspt3.tex|&include file for part 3\\
% |cdocspt4.tex|&include file for part 4\\
% |cdocsdrf.tex|&forwarding file for main file in draft mode\\
% |cdocsfi1.tex|&forwarding file for final version of chapter 1\\
% |cdocsfi2.tex|&forwarding file for final version of chapter 2\\
% \end{tabular}
% \end{center}
% Each of the eight files can be compiled directly by the \LaTeX{} compiler.
%
% %%%%%%%%%%%%%%%%%%%%%%%%%%%%%%%%%%%%%%
% \paragraph{Main File.}
%
% The main file is called |cdocsamp.tex|.
%
% Load the \textsf{childdoc} definitions and
% declare the filename for the main document:
%    \begin{macrocode}
\input{childdoc.def}
\childdocmain{}
%    \end{macrocode}

% Optional override for |\version| flag:
%    \begin{macrocode}
%%\ifchilddoc\else\providecommand{\version}{draft}\fi
%    \end{macrocode}

% Define the default values for the |\version| flag
% (|final| for the main file and |draft| for childs):
%    \begin{macrocode}
\ifchilddoc
\providecommand{\version}{draft}
\else
\providecommand{\version}{final}
\fi
%    \end{macrocode}

% Load the standard document class:
%    \begin{macrocode}
\documentclass[12pt]{article}
%    \end{macrocode}

% Start the document body:
%    \begin{macrocode}
\begin{document}
%    \end{macrocode}

% Declare a title page.
% Print title, part of document being processed and version flag:
%    \begin{macrocode}
\addtocounter{page}{-1}
\begin{center}
{\LARGE\bfseries{}childdoc example\par}
\vspace{1cm}
\ifchilddoc
\ifchilddocmanual part\else chapter\fi:
`\childdocname' of `\childdocjob'\par
\else
main document: `\childdocjob'\par
\fi
version: \version\par
\end{center}
\newpage
%    \end{macrocode}

% Manually include selected file,
% otherwise process as usual:
%    \begin{macrocode}
\ifchilddocmanual
\section*{part `\childdocname'}
\input{\childdocname}
\else
%    \end{macrocode}

% Include the two chapters:
%    \begin{macrocode}
\include{cdocsch1}
\include{cdocsch2}
%    \end{macrocode}

% Include the two parts unless only chapters should be displayed:
%    \begin{macrocode}
\ifchilddoc\else
\section{part three}
\input{cdocspt3}
\section{part four}
\input{cdocspt4}
\fi
%    \end{macrocode}

% Process as usual until here:
%    \begin{macrocode}
\fi
%    \end{macrocode}

% End of document body:
%    \begin{macrocode}
\end{document}
%    \end{macrocode}
%\iffalse
%</samplemain>
%\fi
%
% %%%%%%%%%%%%%%%%%%%%%%%%%%%%%%%%%%%%%%
% \paragraph{Chapter Include Files.}
%
% The include files are called |cdocsch1.tex| and |cdocsch2.tex|.
%
%\iffalse
%<*samplechap1|samplechap2>
%\fi

% Optional override for |\version| flag:
%    \begin{macrocode}
%%\providecommand{\version}{final}
%    \end{macrocode}

% Include the main document:
%    \begin{macrocode}
\input{childdoc.def}
\childdocof{cdocsamp}
%    \end{macrocode}

%\iffalse
%</samplechap1|samplechap2>
%\fi
%
%\iffalse
%<*samplechap1>
%\fi
% Some text for chapter 1:
%    \begin{macrocode}
\section{one}
some text in chapter one
%    \end{macrocode}

%\iffalse
%</samplechap1>
%\fi
% Some text for chapter 2:
%\iffalse
%<*samplechap2>
%\fi
%    \begin{macrocode}
\section{two}
more text in chapter two
%    \end{macrocode}

%\iffalse
%</samplechap2>
%\fi
%
% %%%%%%%%%%%%%%%%%%%%%%%%%%%%%%%%%%%%%%
% \paragraph{Part Include Files.}
%
% The include files are called |cdocspt3.tex| and |cdocspt4.tex|.
%
%\iffalse
%<*samplepart3|samplepart4>
%\fi

% Optional override for |\version| flag:
%    \begin{macrocode}
%%\providecommand{\version}{final}
%    \end{macrocode}

% Include the main document:
%    \begin{macrocode}
\input{childdoc.def}
\childdocby{cdocsamp}
%    \end{macrocode}

%\iffalse
%</samplepart3|samplepart4>
%\fi
%
%\iffalse
%<*samplepart3>
%\fi
% Some text for part 3:
%    \begin{macrocode}
some text in part three
%    \end{macrocode}

%\iffalse
%</samplepart3>
%\fi
% Some text for part 4:
%\iffalse
%<*samplepart4>
%\fi
%    \begin{macrocode}
more text in part four
%    \end{macrocode}

%\iffalse
%</samplepart4>
%\fi
%
% %%%%%%%%%%%%%%%%%%%%%%%%%%%%%%%%%%%%%%
% \paragraph{Forwarding for a Complete Draft.}
%
% The following forwarding file |cdocsdrf.tex|
% compiles the main document in draft mode:
%\iffalse
%<*sampledraft>
%\fi
%    \begin{macrocode}
\def\version{draft}
\input{childdoc.def}
\childdocforward{cdocsamp}
%    \end{macrocode}

%\iffalse
%</sampledraft>
%\fi
%
% %%%%%%%%%%%%%%%%%%%%%%%%%%%%%%%%%%%%%%
% \paragraph{Forwarding for Final Version of the Chapters.}
%
% The following forwarding files |cdocsfn1.tex| and |cdocsfn2.tex|
% (with identical content)
% compile the final versions of the child documents
% |cdocsch1.tex| and |cdocsch2.tex|, respectively:
%\iffalse
%<*samplefinal>
%\fi
%    \begin{macrocode}
\def\version{final}
\input{childdoc.def}
\childdocforwardprefix[cdocsamp]{cdocsfn}{cdocsch}
%    \end{macrocode}

%\iffalse
%</samplefinal>
%\fi
%
% %%%%%%%%%%%%%%%%%%%%%%%%%%%%%%%%%%%%%%
% \paragraph{Command Line Processing.}
%
% The following three command lines generate the output files
% |cdocscld|, |cdocscl1| and |cdocscl2|
% which should be identical to
% |cdocsdrf|, |cdocsch1| and |cdocsfn2|, respectively:
% \begin{center}
% \begin{tabular}{l}
% |latex -jobname cdocscld \|\\
% |  "\def\version{draft}\input{childdoc.def}\childdocforward{cdocsamp}"|\\
% |latex -jobname cdocscl1 \|\\
% |  "\input{childdoc.def}\childdocforward[cdocsamp]{cdocsch1}"|\\
% |latex -jobname cdocscl2 \|\\
% |  "\def\version{final}\input{childdoc.def}\childdocforward{cdocsch2}"|
% \end{tabular}
% \end{center}
% Note that the trailing backslash on each first line
% merely continues the input to the second line
% (for convenient cut ant paste).
% Furthermore, the command |latex| can be replaced by any
% of its alternative versions such as |pdflatex|.
%
% %%%%%%%%%%%%%%%%%%%%%%%%%%%%%%%%%%%%%%%%%%%%%%%%%%%%%%%%%%%%%%%%%%%%%%%%%%%%%%
% %%%%%%%%%%%%%%%%%%%%%%%%%%%%%%%%%%%%%%%%%%%%%%%%%%%%%%%%%%%%%%%%%%%%%%%%%%%%%%
% \section{Implementation}
%\iffalse
%<*package>
%\fi
%
% This section describes the definitions file |childdoc.def|.

% The definitions cannot be loaded using |\usepackage| or |\RequirePackage|
% which has a mechanism to prevent loading a style file more than once.
% When loading the definitions by means of |\input|
% multiple instances have to be prevented manually:
%\iffalse
%This code needs to be before the `\ProvidesFile' directive
%which is defined at the beginning of this file.
%Therefore it is also placed there and commented out here.
%</package>
%<*discard>
%\fi
%    \begin{macrocode}
\ifdefined\childdocmain\endinput\fi
%    \end{macrocode}
%\iffalse
%</discard>
%<*package>
%\fi
%
% \macro{\ifchilddoc}
% \macro{\ifchilddocmanual}
% The conditional |\ifchilddoc| tells whether a
% child (true) or main (false) document is being compiled.
% The conditional |\ifchilddocmanual| tells whether
% the |\includeonly| mechanism is used (false) or
% the selection of child files must be performed manually (true).
% The definitions initialise to false:
%    \begin{macrocode}
\newif\ifchilddoc
\newif\ifchilddocmanual
%    \end{macrocode}

% \macro{\childdocname}
% \macro{\childdocjob}
% The macro |\childdocname| stores the name of the main document
% to be compiled. The macro |\childdocjob| stores the name of
% the document on which the \LaTeX{} compiler was originally invoked.
% The content of |\jobname| cannot be compared
% to filenames specified in the source due to different catcodes.
% The following code rescans |\jobname|, stores the result
% in |\childdocname| and saves a copy in |\childdocjob|:
%    \begin{macrocode}
\edef\childdocname{\scantokens\expandafter{\jobname\noexpand}}
\let\childdocjob\childdocname
%    \end{macrocode}

% \macro{\childdocdisable}
% The macro |\childdocdisable| prevents the main file
% from being processed more than once.
% At this stage, the main document command |\childdocmain|
% is assumed to be called once again where it should do nothing.
% Any subsequent call to it should prevent
% a secondary processing of the main document
% It overwrites the forwarding commands
% |\childdocof| and |\childdocforward|
% with empty macros to prevent further inclusions of the main document:
%    \begin{macrocode}
\newcommand{\childdocdisable}
{
  \renewcommand{\childdocmain}[1]{\renewcommand{\childdocmain}[1]{\endinput}}
  \renewcommand{\childdocof}[1]{}
  \renewcommand{\childdocby}[2][]{}
  \renewcommand{\childdocforward}[2][]{}
  \renewcommand{\childdocdisable}{}
}
%    \end{macrocode}

% \macro{\childdocmain}
% The macro |\childdocmain| is to be called at the top of the main file
% with nothing or the main filename (without extension) as argument.
% First, it breaks loops.
% If the argument is not empty and does not match |\childdocname|
% (which is set by the first inclusion of |childdoc.def|),
% |\ifchilddoc| is set to true, |\includeonly| is applied to the child file
% and |\jobname| is set to the main file
% (for proper handling of |.aux| files):
%    \begin{macrocode}
\newcommand{\childdocmain}[1]
{
  \childdocdisable\childdocmain{}
  \if?#1?\else
    \begingroup
      \def\childdoctmp{#1}
      \ifx\childdoctmp\childdocname
        \def\childdoctmp{}
      \else
        \def\childdoctmp
        {
          \childdoctrue
          \includeonly{\childdocname}
          \def\childdocjob{#1}
          \def\jobname{#1}
        }
      \fi
      \expandafter
    \endgroup
    \childdoctmp
  \fi
}
%    \end{macrocode}

% \macro{\childdocof}
% The command |\childdocof| redirects
% compilation to the main file |#1|.
%    \begin{macrocode}
\newcommand{\childdocof}[1]
{
  \childdocdisable
  \childdoctrue
  \includeonly{\childdocname}
  \def\jobname{#1}
  \def\childdocjob{#1}
  \input{#1}
}
%    \end{macrocode}

% \macro{\childdocby}
% The command |\childdocby| ....
%    \begin{macrocode}
\newcommand{\childdocby}[2][]
{
  \childdocdisable
  \childdoctrue
  \childdocmanualtrue
  \if?#1?\else
    \def\jobname{#2}
  \fi
  \def\childdocjob{#2}
  \input{#2}
  \endinput
}
%    \end{macrocode}

% \macro{\childdocforward}
% The command |\childdocforward| redirects
% compilation to the main file or
% (if the optional argument is given) a child file.
% Parameters are set as if the main file
% or a child file starting with |\childdocof| was compiled.
% Then compilation is handed over to the main file:
%    \begin{macrocode}
\newcommand{\childdocforward}[2][]
{
  \begingroup
    \if?#1?
      \def\childdoctmp
      {
        \def\childdocname{#2}
        \def\childdocjob{#2}
        \def\jobname{#2}
        \input{#2}
        \endinput
      }
    \else
      \def\childdoctmp
      {
        \childdocdisable
        \def\childdocname{#2}
        \childdoctrue
        \includeonly{#2}
        \def\childdocjob{#1}
        \def\jobname{#1}
        \input{#1}
        \endinput
      }
    \fi
    \expandafter
  \endgroup
  \childdoctmp
}
%    \end{macrocode}

% \macro{\childdocforwardprefix}
% The command |\childdocforwardprefix| redirects
% compilation to the main or a child file by means of a pattern.
% The prefix |#1| in the current filename is replaced by |#2|
% and the suffix of the current filename is kept
% (it is assumed that the filename does not contain the substring `|~~~|'
% which is used as a delimiter).
% Compilation is handed over to the new file by |\childdocforward|:
%    \begin{macrocode}
\newcommand{\childdocforwardprefix}[3][]
{
  \begingroup
    \def\childdocextract #2##1~~~{\def\childdoctmp{\childdocforward[#1]{#3##1}}}
    \expandafter\childdocextract\childdocname~~~
    \expandafter
  \endgroup
  \childdoctmp
}
%    \end{macrocode}

% \macro{\childdoc}
% The deprecated macro |\childdoc| is a legacy version of |\childdocmain|:
%    \begin{macrocode}
\newcommand{\childdoc}{\childdocmain}
%    \end{macrocode}

% \macro{\childdocredirect}
% The deprecated macro |\childdocredirect| is a legacy version
% of |\childdocforward| and |\childdocforwardprefix|:
%    \begin{macrocode}
\newcommand{\childdocredirect}[2][]
{
  \begingroup
    \if?#1?
      \def\childdoctmp{\childdocforward{#2}}
    \else
      \def\childdoctmp{\childdocforwardprefix{#1}{#2}}
    \fi
    \expandafter
  \endgroup
  \childdoctmp
}
%    \end{macrocode}

%\iffalse
%</package>
%\fi
%
\endinput
|\\
|\childdocforward{|\textit{main}|}|\\
\end{tabular}
\end{center}
%
or alternatively with:
%
\begin{center}
\begin{tabular}{l}
|% \iffalse
%
% childdoc.dtx Copyright (C) 2017-2018 Niklas Beisert
%
% This work may be distributed and/or modified under the
% conditions of the LaTeX Project Public License, either version 1.3
% of this license or (at your option) any later version.
% The latest version of this license is in
%   http://www.latex-project.org/lppl.txt
% and version 1.3 or later is part of all distributions of LaTeX
% version 2005/12/01 or later.
%
% This work has the LPPL maintenance status `maintained'.
%
% The Current Maintainer of this work is Niklas Beisert.
%
% This work consists of the files childdoc.dtx and childdoc.ins
% and the derived files childdoc.def and cdocsamp.tex with
% cdocsch1.tex, cdocsch2.tex, cdocsdrf.tex, cdocsfn1.tex, cdocsfn2.tex.
%
%<package>\ifdefined\childdocmain\endinput\fi
%<package>\ProvidesFile{childdoc.def}[2018/12/30 v2.0 child document driver]
%<samplemain>\ProvidesFile{cdocsamp.tex}[2018/12/30 v2.0 sample for childdoc]
%<*driver>
%\ProvidesFile{childdoc.drv}[2018/12/30 v2.0 childdoc reference manual file]
\PassOptionsToClass{10pt,a4paper}{article}
\documentclass{ltxdoc}

\usepackage[margin=35mm]{geometry}
\usepackage{hyperref}
\usepackage{hyperxmp}
\usepackage[usenames]{color}

\hypersetup{colorlinks=true}
\hypersetup{pdfstartview=FitH}
\hypersetup{pdfpagemode=UseNone}
\hypersetup{pdfsource={}}
\hypersetup{pdflang={en-UK}}
\hypersetup{pdfcopyright={Copyright 2017-2018 Niklas Beisert.
  This work may be distributed and/or modified under the
  conditions of the LaTeX Project Public License, either version 1.3
  of this license or (at your option) any later version.}}
\hypersetup{pdflicenseurl={http://www.latex-project.org/lppl.txt}}
\hypersetup{pdfcontactaddress={ETH Zurich, ITP, HIT K,
  Wolfgang-Pauli-Strasse 27}}
\hypersetup{pdfcontactpostcode={8093}}
\hypersetup{pdfcontactcity={Zurich}}
\hypersetup{pdfcontactcountry={Switzerland}}
\hypersetup{pdfcontactemail={nbeisert@itp.phys.ethz.ch}}
\hypersetup{pdfcontacturl={http://people.phys.ethz.ch/\xmptilde nbeisert/}}

\newcommand{\secref}[1]{\hyperref[#1]{section \ref*{#1}}}

\parskip1ex
\parindent0pt
\let\olditemize\itemize
\def\itemize{\olditemize\parskip0pt}

\begin{document}

\title{The \textsf{childdoc} Package}
\hypersetup{pdftitle={The childdoc Package}}
\author{Niklas Beisert\\[2ex]
  Institut f\"ur Theoretische Physik\\
  Eidgen\"ossische Technische Hochschule Z\"urich\\
  Wolfgang-Pauli-Strasse 27, 8093 Z\"urich, Switzerland\\[1ex]
  \href{mailto:nbeisert@itp.phys.ethz.ch}
  {\texttt{nbeisert@itp.phys.ethz.ch}}}
\hypersetup{pdfauthor={Niklas Beisert}}
\hypersetup{pdfsubject={Manual for the LaTeX2e Package childdoc}}
\date{30 December 2018, \textsf{v2.0}}
\maketitle

\begin{abstract}\noindent
\textsf{childdoc} is a \LaTeXe{} package
that enables the direct compilation
of document sections included by |\include|
to individual files.
\end{abstract}

\begingroup
\parskip0ex
\tableofcontents
\endgroup

%%%%%%%%%%%%%%%%%%%%%%%%%%%%%%%%%%%%%%%%%%%%%%%%%%%%%%%%%%%%%%%%%%%%%%%%%%%%%%%%
%%%%%%%%%%%%%%%%%%%%%%%%%%%%%%%%%%%%%%%%%%%%%%%%%%%%%%%%%%%%%%%%%%%%%%%%%%%%%%%%
\section{Introduction}

\LaTeX{} provides a mechanism to structure a large document (such as a book)
into a main file and several child files (containing the chapters)
using the |\include| command.
This mechanism is beneficial for documents
which span hundreds of pages in order to
make the source file(s) more manageable.
Moreover, compilation can be restricted to
selected child files by means of the |\includeonly| command.
The latter feature can be used to reduce the compilation time while editing
(this was significantly more useful in the earlier days of \LaTeX{})
or to generate a smaller document which is easier to navigate.
Another application of |\includeonly| is to generate
documents consisting of selected parts of the complete document.

However, there are a few drawbacks of the plain |\include| mechanism:
\begin{itemize}
\item
The child files cannot be compiled on their own,
they can only be compiled via the main file.
A naive editing environment
(such as a text editor with an option
to have the current file processed by \LaTeX)
may require one to switch to the main file before compiling;
attempting to compile the child file produces errors.
\item
The main file must be modified (each time)
to adjust the |\includeonly| command
to the present needs. This easily leaves the main file in a messy state.
\item
The generated document will always carry the filename
of the main document. This is inconvenient if
several child files are to be compiled and
to be kept for distribution.
\end{itemize}

The present package provides a simple interface
to make child files individually compilable by \LaTeX{}.
Compiling a child file then has the same effect as compiling
the main file with an |\includeonly| command
to select the appropriate child.
Moreover the generated document will carry the name of the child
rather than the main file.
This resolves all three above issues.

This feature is meant to make the editing of books,
thesis documents and lecture notes somewhat more convenient.
However, the package can also be used efficiently for
composing a series of documents (such as exercise sheets)
which are typically distributed individually.
It then assists the author in generating the individual documents
(potentially in different versions)
as well as a document containing the collected series.
Another application is in developing style files
or other kinds of included material
where compilation of the style file could redirect
to a sample or test file.

%%%%%%%%%%%%%%%%%%%%%%%%%%%%%%%%%%%%%%%%%%%%%%%%%%%%%%%%%%%%%%%%%%%%%%%%%%%%%%%%
%%%%%%%%%%%%%%%%%%%%%%%%%%%%%%%%%%%%%%%%%%%%%%%%%%%%%%%%%%%%%%%%%%%%%%%%%%%%%%%%
\section{Usage}

First of all, the package \textsf{childdoc} is \emph{not} a standard
\LaTeXe{} |.sty| style file! Therefore it needs to be invoked in
a non-standard way.

%%%%%%%%%%%%%%%%%%%%%%%%%%%%%%%%%%%%%%%%%%%%%%%%%%%%%%%%%%%%%%%%%%%%%%%%%%%%%%%%
\subsection{Included Files}
\label{sec:include}

%%%%%%%%%%%%%%%%%%%%%%%%%%%%%%%%%%%%%%%%
\DescribeMacro{\childdocmain}
To use the package, add the commands
\begin{center}
\begin{tabular}{l}
|\input{childdoc.def}|\\
|\childdocmain{}|\\
\end{tabular}
\end{center}
at the very top of the main \LaTeX{} file,
in particular \emph{before} the |\documentclass| statement!
The argument of |\childdocmain| should be left empty
(but it must be present).

%%%%%%%%%%%%%%%%%%%%%%%%%%%%%%%%%%%%%%%%
\DescribeMacro{\childdocof}
Furthermore, add the commands
\begin{center}
\begin{tabular}{l}
|\input{childdoc.def}|\\
|\childdocof{|\textit{main}|}|\\
\end{tabular}
\end{center}
at the top of every child file \textit{child}
which is included by |\include{|\textit{child}|}|
from within the main file
(or at least for those files to be compiled individually).
The argument \textit{main} must be the filename of the main file.

There are a couple of
considerations in setting up the main and child documents:

%%%%%%%%%%%%%%%%%%%%%%%%%%%%%%%%%%%%%%%%
\paragraph{Restrictions.}

Please note the following restrictions:
\begin{itemize}
\item
|\childdocmain| must be called with one argument \textit{main}
to ensure compatibility with earlier version of the package.
It must either be empty (|\childdocmain{}|)
or precisely match the filename of the main file in which it is specified.
See \secref{sec:detection} for further information.
\item
The filename \textit{main} must be specified without the |.tex| extension.
\item
The filename \textit{main} is case sensitive
(even in case-insensitive file systems)
due to internal string comparison.
\item
The argument \textit{main} should be fully expanded, it cannot be a macro.
\item
Subdirectories and special characters should be avoided in filenames.
\item
The command |\childdocmain{|\textit{main}|}| must be followed by a whitespace.
It should not be followed immediately by another command
or by a comment mark `|%|'.
This is because the \TeX{} parser reads the token immediately following
the argument of |\childdocmain| and puts it
at the beginning of every child section;
however, a white\-space is ignored.
\end{itemize}

%%%%%%%%%%%%%%%%%%%%%%%%%%%%%%%%%%%%%%%%
\paragraph{Content of Main File.}

It is advisable to place all content in the child files included by |\include|.
Any output contained in the main file will appear in all child documents
unless suppressed manually;
it cannot be suppressed automatically by the |\includeonly| directive
and thus should normally be avoided.
A method to include some content in the main file
by means of conditional processing is described in \secref{sec:conditional}.

%%%%%%%%%%%%%%%%%%%%%%%%%%%%%%%%%%%%%%%%
\paragraph{Page Numbering.}

When only a part of the document is compiled,
the appropriate numbering of pages
(as well as other status parameters)
is determined from the |.aux| files.
The latter contain information from previous passes.
However this information needs to propagate through
all intermediate child documents.
Therefore the page numbering in child documents may well
be inconsistent until the complete document is compiled at least once.

A useful (if unconventional) way to always ensure a consistent
page numbering is to restart the numbering in each child document
and denote the pages by `\textit{child}|.|\textit{page}'
where \textit{child} represents the chapter/section number of the child file.
This can be achieved by the command
|\numberwithin{page}{|\textit{child}|}|
of the \textsf{amsmath} package
where \textit{child} can be |chapter| or |section|
depending on the chosen structuring.
Alternatively, one can modify the macro |\thepage| appropriately
and reset the counter |page| at the start of each child file.

%%%%%%%%%%%%%%%%%%%%%%%%%%%%%%%%%%%%%%%%%%%%%%%%%%%%%%%%%%%%%%%%%%%%%%%%%%%%%%%%
\subsection{Conditional Processing}
\label{sec:conditional}

The package provides a mechanism to compile different versions
of a document. To customise the versions further some conditional processing
can come in handy to distinguish which version is being compiled.
The package provides two macros to describe the compilation context:

%%%%%%%%%%%%%%%%%%%%%%%%%%%%%%%%%%%%%%%%
\DescribeMacro{\ifchilddoc}
The conditional |\ifchilddoc| distinguishes between the compilation of
child documents and the main document:
%
\begin{center}
|\ifchilddoc |\textit{child-code}| |[|\||else |\textit{main-code}]| \||fi|
\end{center}

%%%%%%%%%%%%%%%%%%%%%%%%%%%%%%%%%%%%%%%%
\DescribeMacro{\childdocname}
\DescribeMacro{\childdocjob}
The macro |\childdocname| contains the filename (without extension)
of the main or child file being processed.
Note that |\childdocjob| will always contain the name of the main file.

%%%%%%%%%%%%%%%%%%%%%%%%%%%%%%%%%%%%%%%%
\paragraph{Title Page.}

Conditional processing can be used to include a title or banner page
in the main document when proper precautions are taken.
Importantly, the code in the main file should ensure that the page counter
(as well as other status parameters which are stored in the |.aux| files)
takes the same value after the conditional processing.
Otherwise the page numbers may take divergent values
depending on which part is compiled.

For example, a title page could be declared by:
%
\begin{center}
\begin{tabular}{l}
|\ifchilddoc\||else|\\
|\addtocounter{page}{-1}|\\
\textit{code for title page}\\
|\newpage|\\
|\||fi|
\end{tabular}
\end{center}
%
A banner page for the child documents can be generated by:
%
\begin{center}
\begin{tabular}{l}
|\ifchilddoc|\\
|\addtocounter{page}{-1}|\\
\textit{code for banner page}\\
|\newpage|\\
|\||fi|
\end{tabular}
\end{center}
%
Here one could write a message such as:
\begin{center}
|This is the part \childdocname{} of \childdocjob{}.|
\end{center}

%%%%%%%%%%%%%%%%%%%%%%%%%%%%%%%%%%%%%%%%%%%%%%%%%%%%%%%%%%%%%%%%%%%%%%%%%%%%%%%%
\subsection{Flags}
\label{sec:flags}

The package makes it easy to generate different versions
of the main or child documents.
To this end compilation flags can be defined
and assigned different default values.
They will be particularly useful in conjunction
with the forwarding mechanism described in \secref{sec:forward}.

For example, it may be useful to have a flag |\version|
which can be set to |draft| or |final|.
The document source will contain some conditional code
depending on the value of |\version|.
Suppose further, the flag should default to |final| for the main file
and to |draft| for child files
which is a natural assignment for editing the document.
This is achieved by placing the following code
in the preamble of the main document
(below the |\childdocmain| directive):
%
\begin{center}
\begin{tabular}{l}
|\ifchilddoc|\\
|\providecommand{\version}{draft}|\\
|\||else|\\
|\providecommand{\version}{final}|\\
|\||fi|
\end{tabular}
\end{center}
%
The definition by |\providecommand| makes sure
that previous definitions are not overwritten.
Further statements |\providecommand{\version}{...}|
can thus be added before the above code to override it.

For the main file, one might add a line
(between |\childdocmain| and the above block)
%
\begin{center}
|%\ifchilddoc\||else\providecommand{\version}{draft}\||fi|
\end{center}
%
which can be uncommented to produce a draft version.
Likewise one can add a line to the very top of a child file
(above the |\childdocof{|\textit{main}|}| directive)
%
\begin{center}
|%\providecommand{\version}{final}|
\end{center}
%
which can be uncommented to produce the final version of this child document.

%%%%%%%%%%%%%%%%%%%%%%%%%%%%%%%%%%%%%%%%%%%%%%%%%%%%%%%%%%%%%%%%%%%%%%%%%%%%%%%%
\subsection{Forwarding}
\label{sec:forward}

Different versions of the main or child documents
using compilation flags as described in \secref{sec:flags}
can be (permanently) stored in different files
for convenient compilation, viewing and distribution.
To this end, the package defines a command
to pass on compilation to a different file:

%%%%%%%%%%%%%%%%%%%%%%%%%%%%%%%%%%%%%%%%
\DescribeMacro{\childdocforward}
The command |\childdocforward| redirects processing to
another source file:
%
\begin{center}
\begin{tabular}{l}
|\input{childdoc.def}|\\
|\childdocforward[|\textit{main}|]{|\textit{dest}|}|\\
\end{tabular}
\end{center}
%
The argument \textit{dest} is the destination file
(without extension).
It should be the main file or one of the child files.
Note that further \textsf{childdoc} directives
such as |\childdocof| and |\childdocforward|
in the indicated file will be processed in this form.
The optional argument \textit{main}
passes on directly to the main file \textit{main}
while pretending to compile the child \textit{dest}.
This form behaves as if \textit{dest}
issues |\childdocof{|\textit{main}|}| right away,
and no further \textsf{childdoc} directives will be processed.

%%%%%%%%%%%%%%%%%%%%%%%%%%%%%%%%%%%%%%%%
\DescribeMacro{\...prefix}
In the alternative form |\childdocforwardprefix|,
%
\begin{center}
\begin{tabular}{l}
|\input{childdoc.def}|\\
|\childdocforwardprefix[|\textit{main}|]{|\textit{prefix}|}{|\textit{dest}|}|
\end{tabular}
\end{center}
%
the destination file is determined by a pattern
depending on the current file:
To make this work, the current file must be called
`{\textit{prefix}\hspace{0.2em}\textit{suffix}}'
with \textit{prefix} matching precisely the argument.
Processing is then passed on to the file
`{\textit{dest}\hspace{0.2em}\textit{suffix}}'.
Surely, the same effect is achieved by
directly specifying the
argument `{\textit{dest}\hspace{0.2em}\textit{suffix}}'
in the first form.
However, that requires to set up a different file
for each child. With the alternative form of the command
all these files can have exactly the same content
which simplifies setting them up and maintaining them.

For example, the following file |draft.tex|
with a compilation flag |\version| as described in \secref{sec:flags}
compiles the main document as a draft:
%
\begin{center}
\begin{tabular}{l}
|\def\version{draft}|\\
|\input{childdoc.def}|\\
|\childdocforward{|\textit{main}|}|
\end{tabular}
\end{center}
%
Likewise, the following files |final|\textit{nn}|.tex|
compile the final version of the child document
|child|\textit{nn}|.tex|:
%
\begin{center}
\begin{tabular}{l}
|\def\version{final}|\\
|\input{childdoc.def}|\\
|\childdocforwardprefix{final}{child}|
\end{tabular}
\end{center}
%

Note that when several versions of a main file and/or of each child file
are to be generated, it may be convenient to set up a |Makefile| or
shell script to automatise the process.

%%%%%%%%%%%%%%%%%%%%%%%%%%%%%%%%%%%%%%%%%%%%%%%%%%%%%%%%%%%%%%%%%%%%%%%%%%%%%%%%
\subsection{Command Line Processing}
\label{sec:commandline}

The effect of redirection files can also be achieved by invoking
the \LaTeX{} compiler with a more elaborate command line.
Most conveniently this should be done as part
of a shell script or a |Makefile|.

When using \textsf{childdoc} in the main file, the following
command lines effectively perform a redirection
(note that depending on the shell being used,
backslashes may have to be doubled: `|\|' $\to$ `|\\|'):
%
\begin{center}
|... -jobname "|\textit{target}|" |\\|"|[\textit{flags}]%
|\input{childdoc.def}\childdocforward[|\textit{main}|]{|\textit{dest}|}"|
\end{center}
%
Here \textit{target} is the name of the output file,
\textit{main} is the name of the main file
and \textit{dest} is the name of the main or child file to be processed
(all filenames without extensions).
The optional argument \textit{main} can be omitted
if \textit{main} matches \textit{dest}.
Optionally, compilation \textit{flags} can be defined via |\def| commands.
This command line makes the \TeX{} engine believe
it is compiling the file \textit{target}
whose content is specified as the latter parameter.
The provided code then forwards the processing to
\textit{main} or \textit{dest} as described in \secref{sec:forward}.

%%%%%%%%%%%%%%%%%%%%%%%%%%%%%%%%%%%%%%%%%%%%%%%%%%%%%%%%%%%%%%%%%%%%%%%%%%%%%%%%
\subsection{Include by Input}
\label{sec:input}

Including child documents by |\include| has some restrictions by design.
Most notably, the content of a child document always occupies
its own set of pages; pages cannot be shared between child documents.
Usually, this behaviour makes perfect sense
because each child document contain an essential part of the document.
However, in some situations it may be desirable to compose
a document from a collection of parts
without having mandatory page breaks between then.
For this case, the package
provides a mechanism to include parts
by |\input| which can also be processed individually.
However, by construction this mechanism
requires manual handling of the content to be output.

%%%%%%%%%%%%%%%%%%%%%%%%%%%%%%%%%%%%%%%%
\DescribeMacro{\ifchilddocmanual}
The main file should be prepared as usual, see \secref{sec:include}.
However, the document body must make a distinction
between processing of an individual part and of the main document, e.g.:
%
\begin{center}
\begin{tabular}{l}
|\ifchilddocmanual|\\
|\input{\childdocname}|\\
|\||else|\\
\textit{document body with }|\input{|\textit{part}|}|\\
|\||fi|
\end{tabular}
\end{center}
%
The conditional |\ifchilddocmanual| is true whenever
a part to be included by |\input| is being compiled,
and the name of the part is stored in |\childdocname|.

%%%%%%%%%%%%%%%%%%%%%%%%%%%%%%%%%%%%%%%%
\DescribeMacro{\childdocby}
Each part to be included by |\input| should start with:
%
\begin{center}
\begin{tabular}{l}
|\input{childdoc.def}|\\
|\childdocby{|\textit{main}|}|\\
\end{tabular}
\end{center}
%
The directive |\childdocby| is similar to |\childdocof|
described in \secref{sec:include},
but the subsequent selection of content must be done manually.
To that end, both |\ifchilddoc| and |\ifchilddocmanual|
will be true upon processing of a part,
and the name of the part is stored in |\childdocname|.
Note that |\jobname| will be set to the filename of the current part
so that each part receives an individual |.aux| file
that does not interfere with the |.aux| file(s) of the main document.
This behaviour can be altered by the alternative form
|\childdocby[*]{|\textit{main}|}| (with a non-empty optional argument)
which uses the |.aux| file of the main document
by setting |\jobname| to \textit{main}.

%%%%%%%%%%%%%%%%%%%%%%%%%%%%%%%%%%%%%%%%%%%%%%%%%%%%%%%%%%%%%%%%%%%%%%%%%%%%%%%%
\subsection{Driver Development}
\label{sec:driver}

The \textsf{childdoc} mechanism can also be use for the development
of definition files such as \LaTeX{} styles or classes.
This case differs from the above setup with multiple parts
included by |\include| in that no |\includeonly| should be invoked.
This can be achieved by starting the include file
(before |\ProvidesPackage|) with:
%
\begin{center}
\begin{tabular}{l}
|\input{childdoc.def}|\\
|\childdocforward{|\textit{main}|}|\\
\end{tabular}
\end{center}
%
or alternatively with:
%
\begin{center}
\begin{tabular}{l}
|\input{childdoc.def}|\\
|\childdocby{|\textit{main}|}|\\
\end{tabular}
\end{center}
%
Both forms have slightly different effects as described above.
The main file is prepared as usual, see \secref{sec:include}.

%%%%%%%%%%%%%%%%%%%%%%%%%%%%%%%%%%%%%%%%%%%%%%%%%%%%%%%%%%%%%%%%%%%%%%%%%%%%%%%%
\subsection{Legacy Detection}
\label{sec:detection}

The directive |\childdocmain| in the main file can detect
whether the complete document or merely a child is to be compiled
even without using the directive |\childdocof|.
This method is deprecated because it is less robust
and there is no compelling reason to use it;
it is merely provided for backward compatibility
and it may be removed in future versions.

If the detection mechanism is to be used,
it is mandatory to correctly specify
the filename of the main file as the argument of |\childdocmain|:
%
\begin{center}
\begin{tabular}{l}
|\input{childdoc.def}|\\
|\childdocmain{|\textit{main}|}|\\
\end{tabular}
\end{center}
%
If |\jobname| does not match the argument \textit{main} of |\childdocmain|,
it is assumed that |\jobname| points to the child file to be compiled.
When using |\childdocmain| with the main file specified as argument,
it suffices to start a child file
with just |\input{|\textit{main}|}|
without loading of the package and using |\childdocof|.
If instead all processing is done
with the appropriate \textsf{childdoc} directives,
the argument of \textit{main} of |\childdocmain| can be empty.

An alternative version of the command line processing described
in \secref{sec:commandline} using the detection mechanism reads:
%
\begin{center}
|... -jobname "|\textit{target}|" "|[\textit{flags}]%
[|\def\jobname{|\textit{dest}|}|]|\input{|\textit{main}|}"|
\end{center}

%%%%%%%%%%%%%%%%%%%%%%%%%%%%%%%%%%%%%%%%%%%%%%%%%%%%%%%%%%%%%%%%%%%%%%%%%%%%%%%%
\subsection{Manual Code}
\label{sec:manual}

In case one cannot be certain whether the definitions file |childdoc.def|
is installed on the target \TeX{} distribution
and one prefers not to ship it,
it is conceivable to paste a few relevant commands into the sources.

To that end, drop all statements |\input{childdoc.def}|
and perform the replacements as outlined below.
Instead of |\childdocmain{|\textit{main}|}| add the following code
to the top of the main file:
%
\begin{center}
\begin{tabular}{l}
|\||ifdefined\childdocname\endinput\||fi\newif\ifchilddoc|\\
|\edef\childdocname{\scantokens\expandafter{\jobname\noexpand}}|\\
|\def\childdocmain{|\textit{main}|}\||ifx\childdocmain\childdocname\||else|\\
|\childdoctrue\includeonly{\childdocname}\let\jobname\childdocmain\||fi|\\
\end{tabular}
\end{center}
%
Instead of |\childdocof{|\textit{main}|}| just include the main file
at the top of each child file:
%
\begin{center}
|\input{|\textit{main}|}|
\end{center}
%
A simple redirection |\childdocforward{|\textit{dest}|}| is achieved by:
%
\begin{center}
|\def\jobname{|\textit{dest}|}\input{\jobname}|
\end{center}
%
The redirection with prefix
|\childdocforwardprefix[|\textit{prefix}|]{|\textit{dest}|}|
is accomplished by:
%
\begin{center}
\begin{tabular}{l}
|{\edef\jobname{\scantokens\expandafter{\jobname\noexpand}}|\\
|\def\redirectjob |\textit{prefix}|#1~~~{\gdef\jobname{|\textit{dest}|#1}}|\\
|\expandafter\redirectjob\jobname~~~}\input{\jobname}|
\end{tabular}
\end{center}

In an alternative approach,
child documents can be compiled by a specific command line
without additional code or specific definitions:
%
\begin{center}
|... -jobname "|\textit{target}|" "|[\textit{flags}]%
|\includeonly{|\textit{dest}|}\input{|\textit{main}|}"|
\end{center}
%

%%%%%%%%%%%%%%%%%%%%%%%%%%%%%%%%%%%%%%%%%%%%%%%%%%%%%%%%%%%%%%%%%%%%%%%%%%%%%%%%
%%%%%%%%%%%%%%%%%%%%%%%%%%%%%%%%%%%%%%%%%%%%%%%%%%%%%%%%%%%%%%%%%%%%%%%%%%%%%%%%
\section{Information}

%%%%%%%%%%%%%%%%%%%%%%%%%%%%%%%%%%%%%%%%%%%%%%%%%%%%%%%%%%%%%%%%%%%%%%%%%%%%%%%%
\subsection{Copyright}

Copyright \copyright{} 2017--2018 Niklas Beisert

This work may be distributed and/or modified under the
conditions of the \LaTeX{} Project Public License, either version 1.3
of this license or (at your option) any later version.
The latest version of this license is in
  \url{http://www.latex-project.org/lppl.txt}
and version 1.3 or later is part of all distributions of \LaTeX{}
version 2005/12/01 or later.

This work has the LPPL maintenance status `maintained'.

The Current Maintainer of this work is Niklas Beisert.

This work consists of the files |README.txt|, |childdoc.ins| and |childdoc.dtx|
as well as the derived files |childdoc.def|, |cdocsamp.tex|
with |cdocsch1.tex|, |cdocsch2.tex|, |cdocspt3.tex|, |cdocspt4.tex|,
|cdocsdrf.tex|, |cdocsfn1.tex|, |cdocsfn2.tex|
as well as |childdoc.pdf|.

%%%%%%%%%%%%%%%%%%%%%%%%%%%%%%%%%%%%%%%%%%%%%%%%%%%%%%%%%%%%%%%%%%%%%%%%%%%%%%%%
\subsection{Files and Installation}

The package consists of the files:
%
\begin{center}
\begin{tabular}{ll}
    |README.txt|   & readme file \\
    |childdoc.ins| & installation file \\
    |childdoc.dtx| & source file \\
    |childdoc.def| & definition file \\
    |cdocsamp.tex| & sample main file \\
    |cdocsch1.tex| & sample include file \\
    |cdocsch2.tex| & sample include file \\
    |cdocspt3.tex| & sample part file \\
    |cdocspt4.tex| & sample part file \\
    |cdocsdrf.tex| & sample redirection file \\
    |cdocsfn1.tex| & sample redirection file \\
    |cdocsfn2.tex| & sample redirection file \\
    |childdoc.pdf| & manual
\end{tabular}
\end{center}
%
The distribution consists of the files
|README.txt|, |childdoc.ins| and |childdoc.dtx|.
%
\begin{itemize}
\item
Run (pdf)\LaTeX{} on |childdoc.dtx|
to compile the manual |childdoc.pdf| (this file).
\item
Run \LaTeX{} on |childdoc.ins| to create the definitions file |childdoc.def|
and the sample |cdocsamp.tex| with include files
|cdocsch1.tex|, |cdocsch2.tex|, |cdocspt3.tex|, |cdocspt4.tex|,
|cdocsdrf.tex|, |cdocsfn1.tex|, |cdocsfn2.tex|.
Then copy the file |childdoc.def| to an appropriate directory of your \LaTeX{}
distribution, e.g.\ \textit{texmf-root}|/tex/latex/childdoc|.
\end{itemize}

%%%%%%%%%%%%%%%%%%%%%%%%%%%%%%%%%%%%%%%%%%%%%%%%%%%%%%%%%%%%%%%%%%%%%%%%%%%%%%%%
\subsection{Related CTAN Packages}

There are several other packages which offer a similar functionality:
%
\begin{itemize}
\item
The packages
\href{http://ctan.org/pkg/docmute}{\textsf{docmute}},
\href{http://ctan.org/pkg/includex}{\textsf{includex}} and
\href{http://ctan.org/pkg/standalone}{\textsf{standalone}}
provide commands to include only the document body of
a child file thus allowing both files to be compiled individually.
\item
The packages \href{http://ctan.org/pkg/subdocs}{\textsf{subdocs}}
and \href{http://ctan.org/pkg/subfiles}{\textsf{subfiles}}
provide structures in which the main and child documents can be
encapsulated and allowing them to be compiled individually.
The inclusion mechanism is different from the conventional |\include|.
\item
The package \href{http://ctan.org/pkg/combine}{\textsf{combine}}
is an elaborate solution to combine several documents into one.
\end{itemize}
%
See also the CTAN topic \href{http://ctan.org/topic/subdocs}{\textsf{subdocs}}
for further related packages.
The present package differs from the above solutions in that
a document structure constructed with the conventional |\include| mechanism
just needs two extra commands at the top of every file
such that all constituent files can be compiled individually.

%%%%%%%%%%%%%%%%%%%%%%%%%%%%%%%%%%%%%%%%%%%%%%%%%%%%%%%%%%%%%%%%%%%%%%%%%%%%%%%%
%\subsection{Feature Suggestions}
%
%The following is a list of features which may be useful for future
%versions of this package:
%%
%\begin{itemize}
%\item
%\ldots
%\end{itemize}

%%%%%%%%%%%%%%%%%%%%%%%%%%%%%%%%%%%%%%%%%%%%%%%%%%%%%%%%%%%%%%%%%%%%%%%%%%%%%%%%
\subsection{Revision History}

%%%%%%%%%%%%%%%%%%%%%%%%%%%%%%%%%%%%%%%%
\paragraph{v2.0:} 2018/12/30

\begin{itemize}
\item
immediate forward processing
\item
added |\childdocby| mechanism
\item
manual restructured
\end{itemize}

%%%%%%%%%%%%%%%%%%%%%%%%%%%%%%%%%%%%%%%%
\paragraph{v1.6:} 2018/01/17

\begin{itemize}
\item
application for development of include files
\item
corrections to manual
\end{itemize}

%%%%%%%%%%%%%%%%%%%%%%%%%%%%%%%%%%%%%%%%
\paragraph{v1.5:} 2017/05/21

\begin{itemize}
\item
more complete structuring introduced
\item
|\childdocof| introduced
\item
|\childdoc| renamed to |\childdocmain|
\item
|\childredirect| renamed to |\childdocforward| and |\childdocforwardprefix|
and functionality expanded
\end{itemize}

%%%%%%%%%%%%%%%%%%%%%%%%%%%%%%%%%%%%%%%%
\paragraph{v1.0:} 2017/04/27

\begin{itemize}
\item
manual and install package
\item
first version published on CTAN
\end{itemize}

%%%%%%%%%%%%%%%%%%%%%%%%%%%%%%%%%%%%%%%%
\paragraph{v0.6:} 2017/04/26

\begin{itemize}
\item
redirection mechanism added
\end{itemize}

%%%%%%%%%%%%%%%%%%%%%%%%%%%%%%%%%%%%%%%%
\paragraph{v0.5:} 2017/04/26

\begin{itemize}
\item
functionality in definition file
\end{itemize}


%%%%%%%%%%%%%%%%%%%%%%%%%%%%%%%%%%%%%%%%%%%%%%%%%%%%%%%%%%%%%%%%%%%%%%%%%%%%%%%%
%%%%%%%%%%%%%%%%%%%%%%%%%%%%%%%%%%%%%%%%%%%%%%%%%%%%%%%%%%%%%%%%%%%%%%%%%%%%%%%%
%%%%%%%%%%%%%%%%%%%%%%%%%%%%%%%%%%%%%%%%%%%%%%%%%%%%%%%%%%%%%%%%%%%%%%%%%%%%%%%%
\appendix

\settowidth\MacroIndent{\rmfamily\scriptsize 000\ }

 \DocInput{childdoc.dtx}

\end{document}
%</driver>
% \fi
%
% %%%%%%%%%%%%%%%%%%%%%%%%%%%%%%%%%%%%%%%%%%%%%%%%%%%%%%%%%%%%%%%%%%%%%%%%%%%%%%
% %%%%%%%%%%%%%%%%%%%%%%%%%%%%%%%%%%%%%%%%%%%%%%%%%%%%%%%%%%%%%%%%%%%%%%%%%%%%%%
% \section{Sample}
%\iffalse
%<*samplemain>
%\fi
%
% The following presents a sample document
% with two chapters, two parts, a title page,
% a compile flag as well as three forwarding files to set the flag.
% It consists of eight |.tex| files:
% \begin{center}
% \begin{tabular}{ll}
% |cdocsamp.tex|&main file\\
% |cdocsch1.tex|&include file for chapter 1\\
% |cdocsch2.tex|&include file for chapter 2\\
% |cdocspt3.tex|&include file for part 3\\
% |cdocspt4.tex|&include file for part 4\\
% |cdocsdrf.tex|&forwarding file for main file in draft mode\\
% |cdocsfi1.tex|&forwarding file for final version of chapter 1\\
% |cdocsfi2.tex|&forwarding file for final version of chapter 2\\
% \end{tabular}
% \end{center}
% Each of the eight files can be compiled directly by the \LaTeX{} compiler.
%
% %%%%%%%%%%%%%%%%%%%%%%%%%%%%%%%%%%%%%%
% \paragraph{Main File.}
%
% The main file is called |cdocsamp.tex|.
%
% Load the \textsf{childdoc} definitions and
% declare the filename for the main document:
%    \begin{macrocode}
\input{childdoc.def}
\childdocmain{}
%    \end{macrocode}

% Optional override for |\version| flag:
%    \begin{macrocode}
%%\ifchilddoc\else\providecommand{\version}{draft}\fi
%    \end{macrocode}

% Define the default values for the |\version| flag
% (|final| for the main file and |draft| for childs):
%    \begin{macrocode}
\ifchilddoc
\providecommand{\version}{draft}
\else
\providecommand{\version}{final}
\fi
%    \end{macrocode}

% Load the standard document class:
%    \begin{macrocode}
\documentclass[12pt]{article}
%    \end{macrocode}

% Start the document body:
%    \begin{macrocode}
\begin{document}
%    \end{macrocode}

% Declare a title page.
% Print title, part of document being processed and version flag:
%    \begin{macrocode}
\addtocounter{page}{-1}
\begin{center}
{\LARGE\bfseries{}childdoc example\par}
\vspace{1cm}
\ifchilddoc
\ifchilddocmanual part\else chapter\fi:
`\childdocname' of `\childdocjob'\par
\else
main document: `\childdocjob'\par
\fi
version: \version\par
\end{center}
\newpage
%    \end{macrocode}

% Manually include selected file,
% otherwise process as usual:
%    \begin{macrocode}
\ifchilddocmanual
\section*{part `\childdocname'}
\input{\childdocname}
\else
%    \end{macrocode}

% Include the two chapters:
%    \begin{macrocode}
\include{cdocsch1}
\include{cdocsch2}
%    \end{macrocode}

% Include the two parts unless only chapters should be displayed:
%    \begin{macrocode}
\ifchilddoc\else
\section{part three}
\input{cdocspt3}
\section{part four}
\input{cdocspt4}
\fi
%    \end{macrocode}

% Process as usual until here:
%    \begin{macrocode}
\fi
%    \end{macrocode}

% End of document body:
%    \begin{macrocode}
\end{document}
%    \end{macrocode}
%\iffalse
%</samplemain>
%\fi
%
% %%%%%%%%%%%%%%%%%%%%%%%%%%%%%%%%%%%%%%
% \paragraph{Chapter Include Files.}
%
% The include files are called |cdocsch1.tex| and |cdocsch2.tex|.
%
%\iffalse
%<*samplechap1|samplechap2>
%\fi

% Optional override for |\version| flag:
%    \begin{macrocode}
%%\providecommand{\version}{final}
%    \end{macrocode}

% Include the main document:
%    \begin{macrocode}
\input{childdoc.def}
\childdocof{cdocsamp}
%    \end{macrocode}

%\iffalse
%</samplechap1|samplechap2>
%\fi
%
%\iffalse
%<*samplechap1>
%\fi
% Some text for chapter 1:
%    \begin{macrocode}
\section{one}
some text in chapter one
%    \end{macrocode}

%\iffalse
%</samplechap1>
%\fi
% Some text for chapter 2:
%\iffalse
%<*samplechap2>
%\fi
%    \begin{macrocode}
\section{two}
more text in chapter two
%    \end{macrocode}

%\iffalse
%</samplechap2>
%\fi
%
% %%%%%%%%%%%%%%%%%%%%%%%%%%%%%%%%%%%%%%
% \paragraph{Part Include Files.}
%
% The include files are called |cdocspt3.tex| and |cdocspt4.tex|.
%
%\iffalse
%<*samplepart3|samplepart4>
%\fi

% Optional override for |\version| flag:
%    \begin{macrocode}
%%\providecommand{\version}{final}
%    \end{macrocode}

% Include the main document:
%    \begin{macrocode}
\input{childdoc.def}
\childdocby{cdocsamp}
%    \end{macrocode}

%\iffalse
%</samplepart3|samplepart4>
%\fi
%
%\iffalse
%<*samplepart3>
%\fi
% Some text for part 3:
%    \begin{macrocode}
some text in part three
%    \end{macrocode}

%\iffalse
%</samplepart3>
%\fi
% Some text for part 4:
%\iffalse
%<*samplepart4>
%\fi
%    \begin{macrocode}
more text in part four
%    \end{macrocode}

%\iffalse
%</samplepart4>
%\fi
%
% %%%%%%%%%%%%%%%%%%%%%%%%%%%%%%%%%%%%%%
% \paragraph{Forwarding for a Complete Draft.}
%
% The following forwarding file |cdocsdrf.tex|
% compiles the main document in draft mode:
%\iffalse
%<*sampledraft>
%\fi
%    \begin{macrocode}
\def\version{draft}
\input{childdoc.def}
\childdocforward{cdocsamp}
%    \end{macrocode}

%\iffalse
%</sampledraft>
%\fi
%
% %%%%%%%%%%%%%%%%%%%%%%%%%%%%%%%%%%%%%%
% \paragraph{Forwarding for Final Version of the Chapters.}
%
% The following forwarding files |cdocsfn1.tex| and |cdocsfn2.tex|
% (with identical content)
% compile the final versions of the child documents
% |cdocsch1.tex| and |cdocsch2.tex|, respectively:
%\iffalse
%<*samplefinal>
%\fi
%    \begin{macrocode}
\def\version{final}
\input{childdoc.def}
\childdocforwardprefix[cdocsamp]{cdocsfn}{cdocsch}
%    \end{macrocode}

%\iffalse
%</samplefinal>
%\fi
%
% %%%%%%%%%%%%%%%%%%%%%%%%%%%%%%%%%%%%%%
% \paragraph{Command Line Processing.}
%
% The following three command lines generate the output files
% |cdocscld|, |cdocscl1| and |cdocscl2|
% which should be identical to
% |cdocsdrf|, |cdocsch1| and |cdocsfn2|, respectively:
% \begin{center}
% \begin{tabular}{l}
% |latex -jobname cdocscld \|\\
% |  "\def\version{draft}\input{childdoc.def}\childdocforward{cdocsamp}"|\\
% |latex -jobname cdocscl1 \|\\
% |  "\input{childdoc.def}\childdocforward[cdocsamp]{cdocsch1}"|\\
% |latex -jobname cdocscl2 \|\\
% |  "\def\version{final}\input{childdoc.def}\childdocforward{cdocsch2}"|
% \end{tabular}
% \end{center}
% Note that the trailing backslash on each first line
% merely continues the input to the second line
% (for convenient cut ant paste).
% Furthermore, the command |latex| can be replaced by any
% of its alternative versions such as |pdflatex|.
%
% %%%%%%%%%%%%%%%%%%%%%%%%%%%%%%%%%%%%%%%%%%%%%%%%%%%%%%%%%%%%%%%%%%%%%%%%%%%%%%
% %%%%%%%%%%%%%%%%%%%%%%%%%%%%%%%%%%%%%%%%%%%%%%%%%%%%%%%%%%%%%%%%%%%%%%%%%%%%%%
% \section{Implementation}
%\iffalse
%<*package>
%\fi
%
% This section describes the definitions file |childdoc.def|.

% The definitions cannot be loaded using |\usepackage| or |\RequirePackage|
% which has a mechanism to prevent loading a style file more than once.
% When loading the definitions by means of |\input|
% multiple instances have to be prevented manually:
%\iffalse
%This code needs to be before the `\ProvidesFile' directive
%which is defined at the beginning of this file.
%Therefore it is also placed there and commented out here.
%</package>
%<*discard>
%\fi
%    \begin{macrocode}
\ifdefined\childdocmain\endinput\fi
%    \end{macrocode}
%\iffalse
%</discard>
%<*package>
%\fi
%
% \macro{\ifchilddoc}
% \macro{\ifchilddocmanual}
% The conditional |\ifchilddoc| tells whether a
% child (true) or main (false) document is being compiled.
% The conditional |\ifchilddocmanual| tells whether
% the |\includeonly| mechanism is used (false) or
% the selection of child files must be performed manually (true).
% The definitions initialise to false:
%    \begin{macrocode}
\newif\ifchilddoc
\newif\ifchilddocmanual
%    \end{macrocode}

% \macro{\childdocname}
% \macro{\childdocjob}
% The macro |\childdocname| stores the name of the main document
% to be compiled. The macro |\childdocjob| stores the name of
% the document on which the \LaTeX{} compiler was originally invoked.
% The content of |\jobname| cannot be compared
% to filenames specified in the source due to different catcodes.
% The following code rescans |\jobname|, stores the result
% in |\childdocname| and saves a copy in |\childdocjob|:
%    \begin{macrocode}
\edef\childdocname{\scantokens\expandafter{\jobname\noexpand}}
\let\childdocjob\childdocname
%    \end{macrocode}

% \macro{\childdocdisable}
% The macro |\childdocdisable| prevents the main file
% from being processed more than once.
% At this stage, the main document command |\childdocmain|
% is assumed to be called once again where it should do nothing.
% Any subsequent call to it should prevent
% a secondary processing of the main document
% It overwrites the forwarding commands
% |\childdocof| and |\childdocforward|
% with empty macros to prevent further inclusions of the main document:
%    \begin{macrocode}
\newcommand{\childdocdisable}
{
  \renewcommand{\childdocmain}[1]{\renewcommand{\childdocmain}[1]{\endinput}}
  \renewcommand{\childdocof}[1]{}
  \renewcommand{\childdocby}[2][]{}
  \renewcommand{\childdocforward}[2][]{}
  \renewcommand{\childdocdisable}{}
}
%    \end{macrocode}

% \macro{\childdocmain}
% The macro |\childdocmain| is to be called at the top of the main file
% with nothing or the main filename (without extension) as argument.
% First, it breaks loops.
% If the argument is not empty and does not match |\childdocname|
% (which is set by the first inclusion of |childdoc.def|),
% |\ifchilddoc| is set to true, |\includeonly| is applied to the child file
% and |\jobname| is set to the main file
% (for proper handling of |.aux| files):
%    \begin{macrocode}
\newcommand{\childdocmain}[1]
{
  \childdocdisable\childdocmain{}
  \if?#1?\else
    \begingroup
      \def\childdoctmp{#1}
      \ifx\childdoctmp\childdocname
        \def\childdoctmp{}
      \else
        \def\childdoctmp
        {
          \childdoctrue
          \includeonly{\childdocname}
          \def\childdocjob{#1}
          \def\jobname{#1}
        }
      \fi
      \expandafter
    \endgroup
    \childdoctmp
  \fi
}
%    \end{macrocode}

% \macro{\childdocof}
% The command |\childdocof| redirects
% compilation to the main file |#1|.
%    \begin{macrocode}
\newcommand{\childdocof}[1]
{
  \childdocdisable
  \childdoctrue
  \includeonly{\childdocname}
  \def\jobname{#1}
  \def\childdocjob{#1}
  \input{#1}
}
%    \end{macrocode}

% \macro{\childdocby}
% The command |\childdocby| ....
%    \begin{macrocode}
\newcommand{\childdocby}[2][]
{
  \childdocdisable
  \childdoctrue
  \childdocmanualtrue
  \if?#1?\else
    \def\jobname{#2}
  \fi
  \def\childdocjob{#2}
  \input{#2}
  \endinput
}
%    \end{macrocode}

% \macro{\childdocforward}
% The command |\childdocforward| redirects
% compilation to the main file or
% (if the optional argument is given) a child file.
% Parameters are set as if the main file
% or a child file starting with |\childdocof| was compiled.
% Then compilation is handed over to the main file:
%    \begin{macrocode}
\newcommand{\childdocforward}[2][]
{
  \begingroup
    \if?#1?
      \def\childdoctmp
      {
        \def\childdocname{#2}
        \def\childdocjob{#2}
        \def\jobname{#2}
        \input{#2}
        \endinput
      }
    \else
      \def\childdoctmp
      {
        \childdocdisable
        \def\childdocname{#2}
        \childdoctrue
        \includeonly{#2}
        \def\childdocjob{#1}
        \def\jobname{#1}
        \input{#1}
        \endinput
      }
    \fi
    \expandafter
  \endgroup
  \childdoctmp
}
%    \end{macrocode}

% \macro{\childdocforwardprefix}
% The command |\childdocforwardprefix| redirects
% compilation to the main or a child file by means of a pattern.
% The prefix |#1| in the current filename is replaced by |#2|
% and the suffix of the current filename is kept
% (it is assumed that the filename does not contain the substring `|~~~|'
% which is used as a delimiter).
% Compilation is handed over to the new file by |\childdocforward|:
%    \begin{macrocode}
\newcommand{\childdocforwardprefix}[3][]
{
  \begingroup
    \def\childdocextract #2##1~~~{\def\childdoctmp{\childdocforward[#1]{#3##1}}}
    \expandafter\childdocextract\childdocname~~~
    \expandafter
  \endgroup
  \childdoctmp
}
%    \end{macrocode}

% \macro{\childdoc}
% The deprecated macro |\childdoc| is a legacy version of |\childdocmain|:
%    \begin{macrocode}
\newcommand{\childdoc}{\childdocmain}
%    \end{macrocode}

% \macro{\childdocredirect}
% The deprecated macro |\childdocredirect| is a legacy version
% of |\childdocforward| and |\childdocforwardprefix|:
%    \begin{macrocode}
\newcommand{\childdocredirect}[2][]
{
  \begingroup
    \if?#1?
      \def\childdoctmp{\childdocforward{#2}}
    \else
      \def\childdoctmp{\childdocforwardprefix{#1}{#2}}
    \fi
    \expandafter
  \endgroup
  \childdoctmp
}
%    \end{macrocode}

%\iffalse
%</package>
%\fi
%
\endinput
|\\
|\childdocby{|\textit{main}|}|\\
\end{tabular}
\end{center}
%
Both forms have slightly different effects as described above.
The main file is prepared as usual, see \secref{sec:include}.

%%%%%%%%%%%%%%%%%%%%%%%%%%%%%%%%%%%%%%%%%%%%%%%%%%%%%%%%%%%%%%%%%%%%%%%%%%%%%%%%
\subsection{Legacy Detection}
\label{sec:detection}

The directive |\childdocmain| in the main file can detect
whether the complete document or merely a child is to be compiled
even without using the directive |\childdocof|.
This method is deprecated because it is less robust
and there is no compelling reason to use it;
it is merely provided for backward compatibility
and it may be removed in future versions.

If the detection mechanism is to be used,
it is mandatory to correctly specify
the filename of the main file as the argument of |\childdocmain|:
%
\begin{center}
\begin{tabular}{l}
|% \iffalse
%
% childdoc.dtx Copyright (C) 2017-2018 Niklas Beisert
%
% This work may be distributed and/or modified under the
% conditions of the LaTeX Project Public License, either version 1.3
% of this license or (at your option) any later version.
% The latest version of this license is in
%   http://www.latex-project.org/lppl.txt
% and version 1.3 or later is part of all distributions of LaTeX
% version 2005/12/01 or later.
%
% This work has the LPPL maintenance status `maintained'.
%
% The Current Maintainer of this work is Niklas Beisert.
%
% This work consists of the files childdoc.dtx and childdoc.ins
% and the derived files childdoc.def and cdocsamp.tex with
% cdocsch1.tex, cdocsch2.tex, cdocsdrf.tex, cdocsfn1.tex, cdocsfn2.tex.
%
%<package>\ifdefined\childdocmain\endinput\fi
%<package>\ProvidesFile{childdoc.def}[2018/12/30 v2.0 child document driver]
%<samplemain>\ProvidesFile{cdocsamp.tex}[2018/12/30 v2.0 sample for childdoc]
%<*driver>
%\ProvidesFile{childdoc.drv}[2018/12/30 v2.0 childdoc reference manual file]
\PassOptionsToClass{10pt,a4paper}{article}
\documentclass{ltxdoc}

\usepackage[margin=35mm]{geometry}
\usepackage{hyperref}
\usepackage{hyperxmp}
\usepackage[usenames]{color}

\hypersetup{colorlinks=true}
\hypersetup{pdfstartview=FitH}
\hypersetup{pdfpagemode=UseNone}
\hypersetup{pdfsource={}}
\hypersetup{pdflang={en-UK}}
\hypersetup{pdfcopyright={Copyright 2017-2018 Niklas Beisert.
  This work may be distributed and/or modified under the
  conditions of the LaTeX Project Public License, either version 1.3
  of this license or (at your option) any later version.}}
\hypersetup{pdflicenseurl={http://www.latex-project.org/lppl.txt}}
\hypersetup{pdfcontactaddress={ETH Zurich, ITP, HIT K,
  Wolfgang-Pauli-Strasse 27}}
\hypersetup{pdfcontactpostcode={8093}}
\hypersetup{pdfcontactcity={Zurich}}
\hypersetup{pdfcontactcountry={Switzerland}}
\hypersetup{pdfcontactemail={nbeisert@itp.phys.ethz.ch}}
\hypersetup{pdfcontacturl={http://people.phys.ethz.ch/\xmptilde nbeisert/}}

\newcommand{\secref}[1]{\hyperref[#1]{section \ref*{#1}}}

\parskip1ex
\parindent0pt
\let\olditemize\itemize
\def\itemize{\olditemize\parskip0pt}

\begin{document}

\title{The \textsf{childdoc} Package}
\hypersetup{pdftitle={The childdoc Package}}
\author{Niklas Beisert\\[2ex]
  Institut f\"ur Theoretische Physik\\
  Eidgen\"ossische Technische Hochschule Z\"urich\\
  Wolfgang-Pauli-Strasse 27, 8093 Z\"urich, Switzerland\\[1ex]
  \href{mailto:nbeisert@itp.phys.ethz.ch}
  {\texttt{nbeisert@itp.phys.ethz.ch}}}
\hypersetup{pdfauthor={Niklas Beisert}}
\hypersetup{pdfsubject={Manual for the LaTeX2e Package childdoc}}
\date{30 December 2018, \textsf{v2.0}}
\maketitle

\begin{abstract}\noindent
\textsf{childdoc} is a \LaTeXe{} package
that enables the direct compilation
of document sections included by |\include|
to individual files.
\end{abstract}

\begingroup
\parskip0ex
\tableofcontents
\endgroup

%%%%%%%%%%%%%%%%%%%%%%%%%%%%%%%%%%%%%%%%%%%%%%%%%%%%%%%%%%%%%%%%%%%%%%%%%%%%%%%%
%%%%%%%%%%%%%%%%%%%%%%%%%%%%%%%%%%%%%%%%%%%%%%%%%%%%%%%%%%%%%%%%%%%%%%%%%%%%%%%%
\section{Introduction}

\LaTeX{} provides a mechanism to structure a large document (such as a book)
into a main file and several child files (containing the chapters)
using the |\include| command.
This mechanism is beneficial for documents
which span hundreds of pages in order to
make the source file(s) more manageable.
Moreover, compilation can be restricted to
selected child files by means of the |\includeonly| command.
The latter feature can be used to reduce the compilation time while editing
(this was significantly more useful in the earlier days of \LaTeX{})
or to generate a smaller document which is easier to navigate.
Another application of |\includeonly| is to generate
documents consisting of selected parts of the complete document.

However, there are a few drawbacks of the plain |\include| mechanism:
\begin{itemize}
\item
The child files cannot be compiled on their own,
they can only be compiled via the main file.
A naive editing environment
(such as a text editor with an option
to have the current file processed by \LaTeX)
may require one to switch to the main file before compiling;
attempting to compile the child file produces errors.
\item
The main file must be modified (each time)
to adjust the |\includeonly| command
to the present needs. This easily leaves the main file in a messy state.
\item
The generated document will always carry the filename
of the main document. This is inconvenient if
several child files are to be compiled and
to be kept for distribution.
\end{itemize}

The present package provides a simple interface
to make child files individually compilable by \LaTeX{}.
Compiling a child file then has the same effect as compiling
the main file with an |\includeonly| command
to select the appropriate child.
Moreover the generated document will carry the name of the child
rather than the main file.
This resolves all three above issues.

This feature is meant to make the editing of books,
thesis documents and lecture notes somewhat more convenient.
However, the package can also be used efficiently for
composing a series of documents (such as exercise sheets)
which are typically distributed individually.
It then assists the author in generating the individual documents
(potentially in different versions)
as well as a document containing the collected series.
Another application is in developing style files
or other kinds of included material
where compilation of the style file could redirect
to a sample or test file.

%%%%%%%%%%%%%%%%%%%%%%%%%%%%%%%%%%%%%%%%%%%%%%%%%%%%%%%%%%%%%%%%%%%%%%%%%%%%%%%%
%%%%%%%%%%%%%%%%%%%%%%%%%%%%%%%%%%%%%%%%%%%%%%%%%%%%%%%%%%%%%%%%%%%%%%%%%%%%%%%%
\section{Usage}

First of all, the package \textsf{childdoc} is \emph{not} a standard
\LaTeXe{} |.sty| style file! Therefore it needs to be invoked in
a non-standard way.

%%%%%%%%%%%%%%%%%%%%%%%%%%%%%%%%%%%%%%%%%%%%%%%%%%%%%%%%%%%%%%%%%%%%%%%%%%%%%%%%
\subsection{Included Files}
\label{sec:include}

%%%%%%%%%%%%%%%%%%%%%%%%%%%%%%%%%%%%%%%%
\DescribeMacro{\childdocmain}
To use the package, add the commands
\begin{center}
\begin{tabular}{l}
|\input{childdoc.def}|\\
|\childdocmain{}|\\
\end{tabular}
\end{center}
at the very top of the main \LaTeX{} file,
in particular \emph{before} the |\documentclass| statement!
The argument of |\childdocmain| should be left empty
(but it must be present).

%%%%%%%%%%%%%%%%%%%%%%%%%%%%%%%%%%%%%%%%
\DescribeMacro{\childdocof}
Furthermore, add the commands
\begin{center}
\begin{tabular}{l}
|\input{childdoc.def}|\\
|\childdocof{|\textit{main}|}|\\
\end{tabular}
\end{center}
at the top of every child file \textit{child}
which is included by |\include{|\textit{child}|}|
from within the main file
(or at least for those files to be compiled individually).
The argument \textit{main} must be the filename of the main file.

There are a couple of
considerations in setting up the main and child documents:

%%%%%%%%%%%%%%%%%%%%%%%%%%%%%%%%%%%%%%%%
\paragraph{Restrictions.}

Please note the following restrictions:
\begin{itemize}
\item
|\childdocmain| must be called with one argument \textit{main}
to ensure compatibility with earlier version of the package.
It must either be empty (|\childdocmain{}|)
or precisely match the filename of the main file in which it is specified.
See \secref{sec:detection} for further information.
\item
The filename \textit{main} must be specified without the |.tex| extension.
\item
The filename \textit{main} is case sensitive
(even in case-insensitive file systems)
due to internal string comparison.
\item
The argument \textit{main} should be fully expanded, it cannot be a macro.
\item
Subdirectories and special characters should be avoided in filenames.
\item
The command |\childdocmain{|\textit{main}|}| must be followed by a whitespace.
It should not be followed immediately by another command
or by a comment mark `|%|'.
This is because the \TeX{} parser reads the token immediately following
the argument of |\childdocmain| and puts it
at the beginning of every child section;
however, a white\-space is ignored.
\end{itemize}

%%%%%%%%%%%%%%%%%%%%%%%%%%%%%%%%%%%%%%%%
\paragraph{Content of Main File.}

It is advisable to place all content in the child files included by |\include|.
Any output contained in the main file will appear in all child documents
unless suppressed manually;
it cannot be suppressed automatically by the |\includeonly| directive
and thus should normally be avoided.
A method to include some content in the main file
by means of conditional processing is described in \secref{sec:conditional}.

%%%%%%%%%%%%%%%%%%%%%%%%%%%%%%%%%%%%%%%%
\paragraph{Page Numbering.}

When only a part of the document is compiled,
the appropriate numbering of pages
(as well as other status parameters)
is determined from the |.aux| files.
The latter contain information from previous passes.
However this information needs to propagate through
all intermediate child documents.
Therefore the page numbering in child documents may well
be inconsistent until the complete document is compiled at least once.

A useful (if unconventional) way to always ensure a consistent
page numbering is to restart the numbering in each child document
and denote the pages by `\textit{child}|.|\textit{page}'
where \textit{child} represents the chapter/section number of the child file.
This can be achieved by the command
|\numberwithin{page}{|\textit{child}|}|
of the \textsf{amsmath} package
where \textit{child} can be |chapter| or |section|
depending on the chosen structuring.
Alternatively, one can modify the macro |\thepage| appropriately
and reset the counter |page| at the start of each child file.

%%%%%%%%%%%%%%%%%%%%%%%%%%%%%%%%%%%%%%%%%%%%%%%%%%%%%%%%%%%%%%%%%%%%%%%%%%%%%%%%
\subsection{Conditional Processing}
\label{sec:conditional}

The package provides a mechanism to compile different versions
of a document. To customise the versions further some conditional processing
can come in handy to distinguish which version is being compiled.
The package provides two macros to describe the compilation context:

%%%%%%%%%%%%%%%%%%%%%%%%%%%%%%%%%%%%%%%%
\DescribeMacro{\ifchilddoc}
The conditional |\ifchilddoc| distinguishes between the compilation of
child documents and the main document:
%
\begin{center}
|\ifchilddoc |\textit{child-code}| |[|\||else |\textit{main-code}]| \||fi|
\end{center}

%%%%%%%%%%%%%%%%%%%%%%%%%%%%%%%%%%%%%%%%
\DescribeMacro{\childdocname}
\DescribeMacro{\childdocjob}
The macro |\childdocname| contains the filename (without extension)
of the main or child file being processed.
Note that |\childdocjob| will always contain the name of the main file.

%%%%%%%%%%%%%%%%%%%%%%%%%%%%%%%%%%%%%%%%
\paragraph{Title Page.}

Conditional processing can be used to include a title or banner page
in the main document when proper precautions are taken.
Importantly, the code in the main file should ensure that the page counter
(as well as other status parameters which are stored in the |.aux| files)
takes the same value after the conditional processing.
Otherwise the page numbers may take divergent values
depending on which part is compiled.

For example, a title page could be declared by:
%
\begin{center}
\begin{tabular}{l}
|\ifchilddoc\||else|\\
|\addtocounter{page}{-1}|\\
\textit{code for title page}\\
|\newpage|\\
|\||fi|
\end{tabular}
\end{center}
%
A banner page for the child documents can be generated by:
%
\begin{center}
\begin{tabular}{l}
|\ifchilddoc|\\
|\addtocounter{page}{-1}|\\
\textit{code for banner page}\\
|\newpage|\\
|\||fi|
\end{tabular}
\end{center}
%
Here one could write a message such as:
\begin{center}
|This is the part \childdocname{} of \childdocjob{}.|
\end{center}

%%%%%%%%%%%%%%%%%%%%%%%%%%%%%%%%%%%%%%%%%%%%%%%%%%%%%%%%%%%%%%%%%%%%%%%%%%%%%%%%
\subsection{Flags}
\label{sec:flags}

The package makes it easy to generate different versions
of the main or child documents.
To this end compilation flags can be defined
and assigned different default values.
They will be particularly useful in conjunction
with the forwarding mechanism described in \secref{sec:forward}.

For example, it may be useful to have a flag |\version|
which can be set to |draft| or |final|.
The document source will contain some conditional code
depending on the value of |\version|.
Suppose further, the flag should default to |final| for the main file
and to |draft| for child files
which is a natural assignment for editing the document.
This is achieved by placing the following code
in the preamble of the main document
(below the |\childdocmain| directive):
%
\begin{center}
\begin{tabular}{l}
|\ifchilddoc|\\
|\providecommand{\version}{draft}|\\
|\||else|\\
|\providecommand{\version}{final}|\\
|\||fi|
\end{tabular}
\end{center}
%
The definition by |\providecommand| makes sure
that previous definitions are not overwritten.
Further statements |\providecommand{\version}{...}|
can thus be added before the above code to override it.

For the main file, one might add a line
(between |\childdocmain| and the above block)
%
\begin{center}
|%\ifchilddoc\||else\providecommand{\version}{draft}\||fi|
\end{center}
%
which can be uncommented to produce a draft version.
Likewise one can add a line to the very top of a child file
(above the |\childdocof{|\textit{main}|}| directive)
%
\begin{center}
|%\providecommand{\version}{final}|
\end{center}
%
which can be uncommented to produce the final version of this child document.

%%%%%%%%%%%%%%%%%%%%%%%%%%%%%%%%%%%%%%%%%%%%%%%%%%%%%%%%%%%%%%%%%%%%%%%%%%%%%%%%
\subsection{Forwarding}
\label{sec:forward}

Different versions of the main or child documents
using compilation flags as described in \secref{sec:flags}
can be (permanently) stored in different files
for convenient compilation, viewing and distribution.
To this end, the package defines a command
to pass on compilation to a different file:

%%%%%%%%%%%%%%%%%%%%%%%%%%%%%%%%%%%%%%%%
\DescribeMacro{\childdocforward}
The command |\childdocforward| redirects processing to
another source file:
%
\begin{center}
\begin{tabular}{l}
|\input{childdoc.def}|\\
|\childdocforward[|\textit{main}|]{|\textit{dest}|}|\\
\end{tabular}
\end{center}
%
The argument \textit{dest} is the destination file
(without extension).
It should be the main file or one of the child files.
Note that further \textsf{childdoc} directives
such as |\childdocof| and |\childdocforward|
in the indicated file will be processed in this form.
The optional argument \textit{main}
passes on directly to the main file \textit{main}
while pretending to compile the child \textit{dest}.
This form behaves as if \textit{dest}
issues |\childdocof{|\textit{main}|}| right away,
and no further \textsf{childdoc} directives will be processed.

%%%%%%%%%%%%%%%%%%%%%%%%%%%%%%%%%%%%%%%%
\DescribeMacro{\...prefix}
In the alternative form |\childdocforwardprefix|,
%
\begin{center}
\begin{tabular}{l}
|\input{childdoc.def}|\\
|\childdocforwardprefix[|\textit{main}|]{|\textit{prefix}|}{|\textit{dest}|}|
\end{tabular}
\end{center}
%
the destination file is determined by a pattern
depending on the current file:
To make this work, the current file must be called
`{\textit{prefix}\hspace{0.2em}\textit{suffix}}'
with \textit{prefix} matching precisely the argument.
Processing is then passed on to the file
`{\textit{dest}\hspace{0.2em}\textit{suffix}}'.
Surely, the same effect is achieved by
directly specifying the
argument `{\textit{dest}\hspace{0.2em}\textit{suffix}}'
in the first form.
However, that requires to set up a different file
for each child. With the alternative form of the command
all these files can have exactly the same content
which simplifies setting them up and maintaining them.

For example, the following file |draft.tex|
with a compilation flag |\version| as described in \secref{sec:flags}
compiles the main document as a draft:
%
\begin{center}
\begin{tabular}{l}
|\def\version{draft}|\\
|\input{childdoc.def}|\\
|\childdocforward{|\textit{main}|}|
\end{tabular}
\end{center}
%
Likewise, the following files |final|\textit{nn}|.tex|
compile the final version of the child document
|child|\textit{nn}|.tex|:
%
\begin{center}
\begin{tabular}{l}
|\def\version{final}|\\
|\input{childdoc.def}|\\
|\childdocforwardprefix{final}{child}|
\end{tabular}
\end{center}
%

Note that when several versions of a main file and/or of each child file
are to be generated, it may be convenient to set up a |Makefile| or
shell script to automatise the process.

%%%%%%%%%%%%%%%%%%%%%%%%%%%%%%%%%%%%%%%%%%%%%%%%%%%%%%%%%%%%%%%%%%%%%%%%%%%%%%%%
\subsection{Command Line Processing}
\label{sec:commandline}

The effect of redirection files can also be achieved by invoking
the \LaTeX{} compiler with a more elaborate command line.
Most conveniently this should be done as part
of a shell script or a |Makefile|.

When using \textsf{childdoc} in the main file, the following
command lines effectively perform a redirection
(note that depending on the shell being used,
backslashes may have to be doubled: `|\|' $\to$ `|\\|'):
%
\begin{center}
|... -jobname "|\textit{target}|" |\\|"|[\textit{flags}]%
|\input{childdoc.def}\childdocforward[|\textit{main}|]{|\textit{dest}|}"|
\end{center}
%
Here \textit{target} is the name of the output file,
\textit{main} is the name of the main file
and \textit{dest} is the name of the main or child file to be processed
(all filenames without extensions).
The optional argument \textit{main} can be omitted
if \textit{main} matches \textit{dest}.
Optionally, compilation \textit{flags} can be defined via |\def| commands.
This command line makes the \TeX{} engine believe
it is compiling the file \textit{target}
whose content is specified as the latter parameter.
The provided code then forwards the processing to
\textit{main} or \textit{dest} as described in \secref{sec:forward}.

%%%%%%%%%%%%%%%%%%%%%%%%%%%%%%%%%%%%%%%%%%%%%%%%%%%%%%%%%%%%%%%%%%%%%%%%%%%%%%%%
\subsection{Include by Input}
\label{sec:input}

Including child documents by |\include| has some restrictions by design.
Most notably, the content of a child document always occupies
its own set of pages; pages cannot be shared between child documents.
Usually, this behaviour makes perfect sense
because each child document contain an essential part of the document.
However, in some situations it may be desirable to compose
a document from a collection of parts
without having mandatory page breaks between then.
For this case, the package
provides a mechanism to include parts
by |\input| which can also be processed individually.
However, by construction this mechanism
requires manual handling of the content to be output.

%%%%%%%%%%%%%%%%%%%%%%%%%%%%%%%%%%%%%%%%
\DescribeMacro{\ifchilddocmanual}
The main file should be prepared as usual, see \secref{sec:include}.
However, the document body must make a distinction
between processing of an individual part and of the main document, e.g.:
%
\begin{center}
\begin{tabular}{l}
|\ifchilddocmanual|\\
|\input{\childdocname}|\\
|\||else|\\
\textit{document body with }|\input{|\textit{part}|}|\\
|\||fi|
\end{tabular}
\end{center}
%
The conditional |\ifchilddocmanual| is true whenever
a part to be included by |\input| is being compiled,
and the name of the part is stored in |\childdocname|.

%%%%%%%%%%%%%%%%%%%%%%%%%%%%%%%%%%%%%%%%
\DescribeMacro{\childdocby}
Each part to be included by |\input| should start with:
%
\begin{center}
\begin{tabular}{l}
|\input{childdoc.def}|\\
|\childdocby{|\textit{main}|}|\\
\end{tabular}
\end{center}
%
The directive |\childdocby| is similar to |\childdocof|
described in \secref{sec:include},
but the subsequent selection of content must be done manually.
To that end, both |\ifchilddoc| and |\ifchilddocmanual|
will be true upon processing of a part,
and the name of the part is stored in |\childdocname|.
Note that |\jobname| will be set to the filename of the current part
so that each part receives an individual |.aux| file
that does not interfere with the |.aux| file(s) of the main document.
This behaviour can be altered by the alternative form
|\childdocby[*]{|\textit{main}|}| (with a non-empty optional argument)
which uses the |.aux| file of the main document
by setting |\jobname| to \textit{main}.

%%%%%%%%%%%%%%%%%%%%%%%%%%%%%%%%%%%%%%%%%%%%%%%%%%%%%%%%%%%%%%%%%%%%%%%%%%%%%%%%
\subsection{Driver Development}
\label{sec:driver}

The \textsf{childdoc} mechanism can also be use for the development
of definition files such as \LaTeX{} styles or classes.
This case differs from the above setup with multiple parts
included by |\include| in that no |\includeonly| should be invoked.
This can be achieved by starting the include file
(before |\ProvidesPackage|) with:
%
\begin{center}
\begin{tabular}{l}
|\input{childdoc.def}|\\
|\childdocforward{|\textit{main}|}|\\
\end{tabular}
\end{center}
%
or alternatively with:
%
\begin{center}
\begin{tabular}{l}
|\input{childdoc.def}|\\
|\childdocby{|\textit{main}|}|\\
\end{tabular}
\end{center}
%
Both forms have slightly different effects as described above.
The main file is prepared as usual, see \secref{sec:include}.

%%%%%%%%%%%%%%%%%%%%%%%%%%%%%%%%%%%%%%%%%%%%%%%%%%%%%%%%%%%%%%%%%%%%%%%%%%%%%%%%
\subsection{Legacy Detection}
\label{sec:detection}

The directive |\childdocmain| in the main file can detect
whether the complete document or merely a child is to be compiled
even without using the directive |\childdocof|.
This method is deprecated because it is less robust
and there is no compelling reason to use it;
it is merely provided for backward compatibility
and it may be removed in future versions.

If the detection mechanism is to be used,
it is mandatory to correctly specify
the filename of the main file as the argument of |\childdocmain|:
%
\begin{center}
\begin{tabular}{l}
|\input{childdoc.def}|\\
|\childdocmain{|\textit{main}|}|\\
\end{tabular}
\end{center}
%
If |\jobname| does not match the argument \textit{main} of |\childdocmain|,
it is assumed that |\jobname| points to the child file to be compiled.
When using |\childdocmain| with the main file specified as argument,
it suffices to start a child file
with just |\input{|\textit{main}|}|
without loading of the package and using |\childdocof|.
If instead all processing is done
with the appropriate \textsf{childdoc} directives,
the argument of \textit{main} of |\childdocmain| can be empty.

An alternative version of the command line processing described
in \secref{sec:commandline} using the detection mechanism reads:
%
\begin{center}
|... -jobname "|\textit{target}|" "|[\textit{flags}]%
[|\def\jobname{|\textit{dest}|}|]|\input{|\textit{main}|}"|
\end{center}

%%%%%%%%%%%%%%%%%%%%%%%%%%%%%%%%%%%%%%%%%%%%%%%%%%%%%%%%%%%%%%%%%%%%%%%%%%%%%%%%
\subsection{Manual Code}
\label{sec:manual}

In case one cannot be certain whether the definitions file |childdoc.def|
is installed on the target \TeX{} distribution
and one prefers not to ship it,
it is conceivable to paste a few relevant commands into the sources.

To that end, drop all statements |\input{childdoc.def}|
and perform the replacements as outlined below.
Instead of |\childdocmain{|\textit{main}|}| add the following code
to the top of the main file:
%
\begin{center}
\begin{tabular}{l}
|\||ifdefined\childdocname\endinput\||fi\newif\ifchilddoc|\\
|\edef\childdocname{\scantokens\expandafter{\jobname\noexpand}}|\\
|\def\childdocmain{|\textit{main}|}\||ifx\childdocmain\childdocname\||else|\\
|\childdoctrue\includeonly{\childdocname}\let\jobname\childdocmain\||fi|\\
\end{tabular}
\end{center}
%
Instead of |\childdocof{|\textit{main}|}| just include the main file
at the top of each child file:
%
\begin{center}
|\input{|\textit{main}|}|
\end{center}
%
A simple redirection |\childdocforward{|\textit{dest}|}| is achieved by:
%
\begin{center}
|\def\jobname{|\textit{dest}|}\input{\jobname}|
\end{center}
%
The redirection with prefix
|\childdocforwardprefix[|\textit{prefix}|]{|\textit{dest}|}|
is accomplished by:
%
\begin{center}
\begin{tabular}{l}
|{\edef\jobname{\scantokens\expandafter{\jobname\noexpand}}|\\
|\def\redirectjob |\textit{prefix}|#1~~~{\gdef\jobname{|\textit{dest}|#1}}|\\
|\expandafter\redirectjob\jobname~~~}\input{\jobname}|
\end{tabular}
\end{center}

In an alternative approach,
child documents can be compiled by a specific command line
without additional code or specific definitions:
%
\begin{center}
|... -jobname "|\textit{target}|" "|[\textit{flags}]%
|\includeonly{|\textit{dest}|}\input{|\textit{main}|}"|
\end{center}
%

%%%%%%%%%%%%%%%%%%%%%%%%%%%%%%%%%%%%%%%%%%%%%%%%%%%%%%%%%%%%%%%%%%%%%%%%%%%%%%%%
%%%%%%%%%%%%%%%%%%%%%%%%%%%%%%%%%%%%%%%%%%%%%%%%%%%%%%%%%%%%%%%%%%%%%%%%%%%%%%%%
\section{Information}

%%%%%%%%%%%%%%%%%%%%%%%%%%%%%%%%%%%%%%%%%%%%%%%%%%%%%%%%%%%%%%%%%%%%%%%%%%%%%%%%
\subsection{Copyright}

Copyright \copyright{} 2017--2018 Niklas Beisert

This work may be distributed and/or modified under the
conditions of the \LaTeX{} Project Public License, either version 1.3
of this license or (at your option) any later version.
The latest version of this license is in
  \url{http://www.latex-project.org/lppl.txt}
and version 1.3 or later is part of all distributions of \LaTeX{}
version 2005/12/01 or later.

This work has the LPPL maintenance status `maintained'.

The Current Maintainer of this work is Niklas Beisert.

This work consists of the files |README.txt|, |childdoc.ins| and |childdoc.dtx|
as well as the derived files |childdoc.def|, |cdocsamp.tex|
with |cdocsch1.tex|, |cdocsch2.tex|, |cdocspt3.tex|, |cdocspt4.tex|,
|cdocsdrf.tex|, |cdocsfn1.tex|, |cdocsfn2.tex|
as well as |childdoc.pdf|.

%%%%%%%%%%%%%%%%%%%%%%%%%%%%%%%%%%%%%%%%%%%%%%%%%%%%%%%%%%%%%%%%%%%%%%%%%%%%%%%%
\subsection{Files and Installation}

The package consists of the files:
%
\begin{center}
\begin{tabular}{ll}
    |README.txt|   & readme file \\
    |childdoc.ins| & installation file \\
    |childdoc.dtx| & source file \\
    |childdoc.def| & definition file \\
    |cdocsamp.tex| & sample main file \\
    |cdocsch1.tex| & sample include file \\
    |cdocsch2.tex| & sample include file \\
    |cdocspt3.tex| & sample part file \\
    |cdocspt4.tex| & sample part file \\
    |cdocsdrf.tex| & sample redirection file \\
    |cdocsfn1.tex| & sample redirection file \\
    |cdocsfn2.tex| & sample redirection file \\
    |childdoc.pdf| & manual
\end{tabular}
\end{center}
%
The distribution consists of the files
|README.txt|, |childdoc.ins| and |childdoc.dtx|.
%
\begin{itemize}
\item
Run (pdf)\LaTeX{} on |childdoc.dtx|
to compile the manual |childdoc.pdf| (this file).
\item
Run \LaTeX{} on |childdoc.ins| to create the definitions file |childdoc.def|
and the sample |cdocsamp.tex| with include files
|cdocsch1.tex|, |cdocsch2.tex|, |cdocspt3.tex|, |cdocspt4.tex|,
|cdocsdrf.tex|, |cdocsfn1.tex|, |cdocsfn2.tex|.
Then copy the file |childdoc.def| to an appropriate directory of your \LaTeX{}
distribution, e.g.\ \textit{texmf-root}|/tex/latex/childdoc|.
\end{itemize}

%%%%%%%%%%%%%%%%%%%%%%%%%%%%%%%%%%%%%%%%%%%%%%%%%%%%%%%%%%%%%%%%%%%%%%%%%%%%%%%%
\subsection{Related CTAN Packages}

There are several other packages which offer a similar functionality:
%
\begin{itemize}
\item
The packages
\href{http://ctan.org/pkg/docmute}{\textsf{docmute}},
\href{http://ctan.org/pkg/includex}{\textsf{includex}} and
\href{http://ctan.org/pkg/standalone}{\textsf{standalone}}
provide commands to include only the document body of
a child file thus allowing both files to be compiled individually.
\item
The packages \href{http://ctan.org/pkg/subdocs}{\textsf{subdocs}}
and \href{http://ctan.org/pkg/subfiles}{\textsf{subfiles}}
provide structures in which the main and child documents can be
encapsulated and allowing them to be compiled individually.
The inclusion mechanism is different from the conventional |\include|.
\item
The package \href{http://ctan.org/pkg/combine}{\textsf{combine}}
is an elaborate solution to combine several documents into one.
\end{itemize}
%
See also the CTAN topic \href{http://ctan.org/topic/subdocs}{\textsf{subdocs}}
for further related packages.
The present package differs from the above solutions in that
a document structure constructed with the conventional |\include| mechanism
just needs two extra commands at the top of every file
such that all constituent files can be compiled individually.

%%%%%%%%%%%%%%%%%%%%%%%%%%%%%%%%%%%%%%%%%%%%%%%%%%%%%%%%%%%%%%%%%%%%%%%%%%%%%%%%
%\subsection{Feature Suggestions}
%
%The following is a list of features which may be useful for future
%versions of this package:
%%
%\begin{itemize}
%\item
%\ldots
%\end{itemize}

%%%%%%%%%%%%%%%%%%%%%%%%%%%%%%%%%%%%%%%%%%%%%%%%%%%%%%%%%%%%%%%%%%%%%%%%%%%%%%%%
\subsection{Revision History}

%%%%%%%%%%%%%%%%%%%%%%%%%%%%%%%%%%%%%%%%
\paragraph{v2.0:} 2018/12/30

\begin{itemize}
\item
immediate forward processing
\item
added |\childdocby| mechanism
\item
manual restructured
\end{itemize}

%%%%%%%%%%%%%%%%%%%%%%%%%%%%%%%%%%%%%%%%
\paragraph{v1.6:} 2018/01/17

\begin{itemize}
\item
application for development of include files
\item
corrections to manual
\end{itemize}

%%%%%%%%%%%%%%%%%%%%%%%%%%%%%%%%%%%%%%%%
\paragraph{v1.5:} 2017/05/21

\begin{itemize}
\item
more complete structuring introduced
\item
|\childdocof| introduced
\item
|\childdoc| renamed to |\childdocmain|
\item
|\childredirect| renamed to |\childdocforward| and |\childdocforwardprefix|
and functionality expanded
\end{itemize}

%%%%%%%%%%%%%%%%%%%%%%%%%%%%%%%%%%%%%%%%
\paragraph{v1.0:} 2017/04/27

\begin{itemize}
\item
manual and install package
\item
first version published on CTAN
\end{itemize}

%%%%%%%%%%%%%%%%%%%%%%%%%%%%%%%%%%%%%%%%
\paragraph{v0.6:} 2017/04/26

\begin{itemize}
\item
redirection mechanism added
\end{itemize}

%%%%%%%%%%%%%%%%%%%%%%%%%%%%%%%%%%%%%%%%
\paragraph{v0.5:} 2017/04/26

\begin{itemize}
\item
functionality in definition file
\end{itemize}


%%%%%%%%%%%%%%%%%%%%%%%%%%%%%%%%%%%%%%%%%%%%%%%%%%%%%%%%%%%%%%%%%%%%%%%%%%%%%%%%
%%%%%%%%%%%%%%%%%%%%%%%%%%%%%%%%%%%%%%%%%%%%%%%%%%%%%%%%%%%%%%%%%%%%%%%%%%%%%%%%
%%%%%%%%%%%%%%%%%%%%%%%%%%%%%%%%%%%%%%%%%%%%%%%%%%%%%%%%%%%%%%%%%%%%%%%%%%%%%%%%
\appendix

\settowidth\MacroIndent{\rmfamily\scriptsize 000\ }

 \DocInput{childdoc.dtx}

\end{document}
%</driver>
% \fi
%
% %%%%%%%%%%%%%%%%%%%%%%%%%%%%%%%%%%%%%%%%%%%%%%%%%%%%%%%%%%%%%%%%%%%%%%%%%%%%%%
% %%%%%%%%%%%%%%%%%%%%%%%%%%%%%%%%%%%%%%%%%%%%%%%%%%%%%%%%%%%%%%%%%%%%%%%%%%%%%%
% \section{Sample}
%\iffalse
%<*samplemain>
%\fi
%
% The following presents a sample document
% with two chapters, two parts, a title page,
% a compile flag as well as three forwarding files to set the flag.
% It consists of eight |.tex| files:
% \begin{center}
% \begin{tabular}{ll}
% |cdocsamp.tex|&main file\\
% |cdocsch1.tex|&include file for chapter 1\\
% |cdocsch2.tex|&include file for chapter 2\\
% |cdocspt3.tex|&include file for part 3\\
% |cdocspt4.tex|&include file for part 4\\
% |cdocsdrf.tex|&forwarding file for main file in draft mode\\
% |cdocsfi1.tex|&forwarding file for final version of chapter 1\\
% |cdocsfi2.tex|&forwarding file for final version of chapter 2\\
% \end{tabular}
% \end{center}
% Each of the eight files can be compiled directly by the \LaTeX{} compiler.
%
% %%%%%%%%%%%%%%%%%%%%%%%%%%%%%%%%%%%%%%
% \paragraph{Main File.}
%
% The main file is called |cdocsamp.tex|.
%
% Load the \textsf{childdoc} definitions and
% declare the filename for the main document:
%    \begin{macrocode}
\input{childdoc.def}
\childdocmain{}
%    \end{macrocode}

% Optional override for |\version| flag:
%    \begin{macrocode}
%%\ifchilddoc\else\providecommand{\version}{draft}\fi
%    \end{macrocode}

% Define the default values for the |\version| flag
% (|final| for the main file and |draft| for childs):
%    \begin{macrocode}
\ifchilddoc
\providecommand{\version}{draft}
\else
\providecommand{\version}{final}
\fi
%    \end{macrocode}

% Load the standard document class:
%    \begin{macrocode}
\documentclass[12pt]{article}
%    \end{macrocode}

% Start the document body:
%    \begin{macrocode}
\begin{document}
%    \end{macrocode}

% Declare a title page.
% Print title, part of document being processed and version flag:
%    \begin{macrocode}
\addtocounter{page}{-1}
\begin{center}
{\LARGE\bfseries{}childdoc example\par}
\vspace{1cm}
\ifchilddoc
\ifchilddocmanual part\else chapter\fi:
`\childdocname' of `\childdocjob'\par
\else
main document: `\childdocjob'\par
\fi
version: \version\par
\end{center}
\newpage
%    \end{macrocode}

% Manually include selected file,
% otherwise process as usual:
%    \begin{macrocode}
\ifchilddocmanual
\section*{part `\childdocname'}
\input{\childdocname}
\else
%    \end{macrocode}

% Include the two chapters:
%    \begin{macrocode}
\include{cdocsch1}
\include{cdocsch2}
%    \end{macrocode}

% Include the two parts unless only chapters should be displayed:
%    \begin{macrocode}
\ifchilddoc\else
\section{part three}
\input{cdocspt3}
\section{part four}
\input{cdocspt4}
\fi
%    \end{macrocode}

% Process as usual until here:
%    \begin{macrocode}
\fi
%    \end{macrocode}

% End of document body:
%    \begin{macrocode}
\end{document}
%    \end{macrocode}
%\iffalse
%</samplemain>
%\fi
%
% %%%%%%%%%%%%%%%%%%%%%%%%%%%%%%%%%%%%%%
% \paragraph{Chapter Include Files.}
%
% The include files are called |cdocsch1.tex| and |cdocsch2.tex|.
%
%\iffalse
%<*samplechap1|samplechap2>
%\fi

% Optional override for |\version| flag:
%    \begin{macrocode}
%%\providecommand{\version}{final}
%    \end{macrocode}

% Include the main document:
%    \begin{macrocode}
\input{childdoc.def}
\childdocof{cdocsamp}
%    \end{macrocode}

%\iffalse
%</samplechap1|samplechap2>
%\fi
%
%\iffalse
%<*samplechap1>
%\fi
% Some text for chapter 1:
%    \begin{macrocode}
\section{one}
some text in chapter one
%    \end{macrocode}

%\iffalse
%</samplechap1>
%\fi
% Some text for chapter 2:
%\iffalse
%<*samplechap2>
%\fi
%    \begin{macrocode}
\section{two}
more text in chapter two
%    \end{macrocode}

%\iffalse
%</samplechap2>
%\fi
%
% %%%%%%%%%%%%%%%%%%%%%%%%%%%%%%%%%%%%%%
% \paragraph{Part Include Files.}
%
% The include files are called |cdocspt3.tex| and |cdocspt4.tex|.
%
%\iffalse
%<*samplepart3|samplepart4>
%\fi

% Optional override for |\version| flag:
%    \begin{macrocode}
%%\providecommand{\version}{final}
%    \end{macrocode}

% Include the main document:
%    \begin{macrocode}
\input{childdoc.def}
\childdocby{cdocsamp}
%    \end{macrocode}

%\iffalse
%</samplepart3|samplepart4>
%\fi
%
%\iffalse
%<*samplepart3>
%\fi
% Some text for part 3:
%    \begin{macrocode}
some text in part three
%    \end{macrocode}

%\iffalse
%</samplepart3>
%\fi
% Some text for part 4:
%\iffalse
%<*samplepart4>
%\fi
%    \begin{macrocode}
more text in part four
%    \end{macrocode}

%\iffalse
%</samplepart4>
%\fi
%
% %%%%%%%%%%%%%%%%%%%%%%%%%%%%%%%%%%%%%%
% \paragraph{Forwarding for a Complete Draft.}
%
% The following forwarding file |cdocsdrf.tex|
% compiles the main document in draft mode:
%\iffalse
%<*sampledraft>
%\fi
%    \begin{macrocode}
\def\version{draft}
\input{childdoc.def}
\childdocforward{cdocsamp}
%    \end{macrocode}

%\iffalse
%</sampledraft>
%\fi
%
% %%%%%%%%%%%%%%%%%%%%%%%%%%%%%%%%%%%%%%
% \paragraph{Forwarding for Final Version of the Chapters.}
%
% The following forwarding files |cdocsfn1.tex| and |cdocsfn2.tex|
% (with identical content)
% compile the final versions of the child documents
% |cdocsch1.tex| and |cdocsch2.tex|, respectively:
%\iffalse
%<*samplefinal>
%\fi
%    \begin{macrocode}
\def\version{final}
\input{childdoc.def}
\childdocforwardprefix[cdocsamp]{cdocsfn}{cdocsch}
%    \end{macrocode}

%\iffalse
%</samplefinal>
%\fi
%
% %%%%%%%%%%%%%%%%%%%%%%%%%%%%%%%%%%%%%%
% \paragraph{Command Line Processing.}
%
% The following three command lines generate the output files
% |cdocscld|, |cdocscl1| and |cdocscl2|
% which should be identical to
% |cdocsdrf|, |cdocsch1| and |cdocsfn2|, respectively:
% \begin{center}
% \begin{tabular}{l}
% |latex -jobname cdocscld \|\\
% |  "\def\version{draft}\input{childdoc.def}\childdocforward{cdocsamp}"|\\
% |latex -jobname cdocscl1 \|\\
% |  "\input{childdoc.def}\childdocforward[cdocsamp]{cdocsch1}"|\\
% |latex -jobname cdocscl2 \|\\
% |  "\def\version{final}\input{childdoc.def}\childdocforward{cdocsch2}"|
% \end{tabular}
% \end{center}
% Note that the trailing backslash on each first line
% merely continues the input to the second line
% (for convenient cut ant paste).
% Furthermore, the command |latex| can be replaced by any
% of its alternative versions such as |pdflatex|.
%
% %%%%%%%%%%%%%%%%%%%%%%%%%%%%%%%%%%%%%%%%%%%%%%%%%%%%%%%%%%%%%%%%%%%%%%%%%%%%%%
% %%%%%%%%%%%%%%%%%%%%%%%%%%%%%%%%%%%%%%%%%%%%%%%%%%%%%%%%%%%%%%%%%%%%%%%%%%%%%%
% \section{Implementation}
%\iffalse
%<*package>
%\fi
%
% This section describes the definitions file |childdoc.def|.

% The definitions cannot be loaded using |\usepackage| or |\RequirePackage|
% which has a mechanism to prevent loading a style file more than once.
% When loading the definitions by means of |\input|
% multiple instances have to be prevented manually:
%\iffalse
%This code needs to be before the `\ProvidesFile' directive
%which is defined at the beginning of this file.
%Therefore it is also placed there and commented out here.
%</package>
%<*discard>
%\fi
%    \begin{macrocode}
\ifdefined\childdocmain\endinput\fi
%    \end{macrocode}
%\iffalse
%</discard>
%<*package>
%\fi
%
% \macro{\ifchilddoc}
% \macro{\ifchilddocmanual}
% The conditional |\ifchilddoc| tells whether a
% child (true) or main (false) document is being compiled.
% The conditional |\ifchilddocmanual| tells whether
% the |\includeonly| mechanism is used (false) or
% the selection of child files must be performed manually (true).
% The definitions initialise to false:
%    \begin{macrocode}
\newif\ifchilddoc
\newif\ifchilddocmanual
%    \end{macrocode}

% \macro{\childdocname}
% \macro{\childdocjob}
% The macro |\childdocname| stores the name of the main document
% to be compiled. The macro |\childdocjob| stores the name of
% the document on which the \LaTeX{} compiler was originally invoked.
% The content of |\jobname| cannot be compared
% to filenames specified in the source due to different catcodes.
% The following code rescans |\jobname|, stores the result
% in |\childdocname| and saves a copy in |\childdocjob|:
%    \begin{macrocode}
\edef\childdocname{\scantokens\expandafter{\jobname\noexpand}}
\let\childdocjob\childdocname
%    \end{macrocode}

% \macro{\childdocdisable}
% The macro |\childdocdisable| prevents the main file
% from being processed more than once.
% At this stage, the main document command |\childdocmain|
% is assumed to be called once again where it should do nothing.
% Any subsequent call to it should prevent
% a secondary processing of the main document
% It overwrites the forwarding commands
% |\childdocof| and |\childdocforward|
% with empty macros to prevent further inclusions of the main document:
%    \begin{macrocode}
\newcommand{\childdocdisable}
{
  \renewcommand{\childdocmain}[1]{\renewcommand{\childdocmain}[1]{\endinput}}
  \renewcommand{\childdocof}[1]{}
  \renewcommand{\childdocby}[2][]{}
  \renewcommand{\childdocforward}[2][]{}
  \renewcommand{\childdocdisable}{}
}
%    \end{macrocode}

% \macro{\childdocmain}
% The macro |\childdocmain| is to be called at the top of the main file
% with nothing or the main filename (without extension) as argument.
% First, it breaks loops.
% If the argument is not empty and does not match |\childdocname|
% (which is set by the first inclusion of |childdoc.def|),
% |\ifchilddoc| is set to true, |\includeonly| is applied to the child file
% and |\jobname| is set to the main file
% (for proper handling of |.aux| files):
%    \begin{macrocode}
\newcommand{\childdocmain}[1]
{
  \childdocdisable\childdocmain{}
  \if?#1?\else
    \begingroup
      \def\childdoctmp{#1}
      \ifx\childdoctmp\childdocname
        \def\childdoctmp{}
      \else
        \def\childdoctmp
        {
          \childdoctrue
          \includeonly{\childdocname}
          \def\childdocjob{#1}
          \def\jobname{#1}
        }
      \fi
      \expandafter
    \endgroup
    \childdoctmp
  \fi
}
%    \end{macrocode}

% \macro{\childdocof}
% The command |\childdocof| redirects
% compilation to the main file |#1|.
%    \begin{macrocode}
\newcommand{\childdocof}[1]
{
  \childdocdisable
  \childdoctrue
  \includeonly{\childdocname}
  \def\jobname{#1}
  \def\childdocjob{#1}
  \input{#1}
}
%    \end{macrocode}

% \macro{\childdocby}
% The command |\childdocby| ....
%    \begin{macrocode}
\newcommand{\childdocby}[2][]
{
  \childdocdisable
  \childdoctrue
  \childdocmanualtrue
  \if?#1?\else
    \def\jobname{#2}
  \fi
  \def\childdocjob{#2}
  \input{#2}
  \endinput
}
%    \end{macrocode}

% \macro{\childdocforward}
% The command |\childdocforward| redirects
% compilation to the main file or
% (if the optional argument is given) a child file.
% Parameters are set as if the main file
% or a child file starting with |\childdocof| was compiled.
% Then compilation is handed over to the main file:
%    \begin{macrocode}
\newcommand{\childdocforward}[2][]
{
  \begingroup
    \if?#1?
      \def\childdoctmp
      {
        \def\childdocname{#2}
        \def\childdocjob{#2}
        \def\jobname{#2}
        \input{#2}
        \endinput
      }
    \else
      \def\childdoctmp
      {
        \childdocdisable
        \def\childdocname{#2}
        \childdoctrue
        \includeonly{#2}
        \def\childdocjob{#1}
        \def\jobname{#1}
        \input{#1}
        \endinput
      }
    \fi
    \expandafter
  \endgroup
  \childdoctmp
}
%    \end{macrocode}

% \macro{\childdocforwardprefix}
% The command |\childdocforwardprefix| redirects
% compilation to the main or a child file by means of a pattern.
% The prefix |#1| in the current filename is replaced by |#2|
% and the suffix of the current filename is kept
% (it is assumed that the filename does not contain the substring `|~~~|'
% which is used as a delimiter).
% Compilation is handed over to the new file by |\childdocforward|:
%    \begin{macrocode}
\newcommand{\childdocforwardprefix}[3][]
{
  \begingroup
    \def\childdocextract #2##1~~~{\def\childdoctmp{\childdocforward[#1]{#3##1}}}
    \expandafter\childdocextract\childdocname~~~
    \expandafter
  \endgroup
  \childdoctmp
}
%    \end{macrocode}

% \macro{\childdoc}
% The deprecated macro |\childdoc| is a legacy version of |\childdocmain|:
%    \begin{macrocode}
\newcommand{\childdoc}{\childdocmain}
%    \end{macrocode}

% \macro{\childdocredirect}
% The deprecated macro |\childdocredirect| is a legacy version
% of |\childdocforward| and |\childdocforwardprefix|:
%    \begin{macrocode}
\newcommand{\childdocredirect}[2][]
{
  \begingroup
    \if?#1?
      \def\childdoctmp{\childdocforward{#2}}
    \else
      \def\childdoctmp{\childdocforwardprefix{#1}{#2}}
    \fi
    \expandafter
  \endgroup
  \childdoctmp
}
%    \end{macrocode}

%\iffalse
%</package>
%\fi
%
\endinput
|\\
|\childdocmain{|\textit{main}|}|\\
\end{tabular}
\end{center}
%
If |\jobname| does not match the argument \textit{main} of |\childdocmain|,
it is assumed that |\jobname| points to the child file to be compiled.
When using |\childdocmain| with the main file specified as argument,
it suffices to start a child file
with just |\input{|\textit{main}|}|
without loading of the package and using |\childdocof|.
If instead all processing is done
with the appropriate \textsf{childdoc} directives,
the argument of \textit{main} of |\childdocmain| can be empty.

An alternative version of the command line processing described
in \secref{sec:commandline} using the detection mechanism reads:
%
\begin{center}
|... -jobname "|\textit{target}|" "|[\textit{flags}]%
[|\def\jobname{|\textit{dest}|}|]|\input{|\textit{main}|}"|
\end{center}

%%%%%%%%%%%%%%%%%%%%%%%%%%%%%%%%%%%%%%%%%%%%%%%%%%%%%%%%%%%%%%%%%%%%%%%%%%%%%%%%
\subsection{Manual Code}
\label{sec:manual}

In case one cannot be certain whether the definitions file |childdoc.def|
is installed on the target \TeX{} distribution
and one prefers not to ship it,
it is conceivable to paste a few relevant commands into the sources.

To that end, drop all statements |% \iffalse
%
% childdoc.dtx Copyright (C) 2017-2018 Niklas Beisert
%
% This work may be distributed and/or modified under the
% conditions of the LaTeX Project Public License, either version 1.3
% of this license or (at your option) any later version.
% The latest version of this license is in
%   http://www.latex-project.org/lppl.txt
% and version 1.3 or later is part of all distributions of LaTeX
% version 2005/12/01 or later.
%
% This work has the LPPL maintenance status `maintained'.
%
% The Current Maintainer of this work is Niklas Beisert.
%
% This work consists of the files childdoc.dtx and childdoc.ins
% and the derived files childdoc.def and cdocsamp.tex with
% cdocsch1.tex, cdocsch2.tex, cdocsdrf.tex, cdocsfn1.tex, cdocsfn2.tex.
%
%<package>\ifdefined\childdocmain\endinput\fi
%<package>\ProvidesFile{childdoc.def}[2018/12/30 v2.0 child document driver]
%<samplemain>\ProvidesFile{cdocsamp.tex}[2018/12/30 v2.0 sample for childdoc]
%<*driver>
%\ProvidesFile{childdoc.drv}[2018/12/30 v2.0 childdoc reference manual file]
\PassOptionsToClass{10pt,a4paper}{article}
\documentclass{ltxdoc}

\usepackage[margin=35mm]{geometry}
\usepackage{hyperref}
\usepackage{hyperxmp}
\usepackage[usenames]{color}

\hypersetup{colorlinks=true}
\hypersetup{pdfstartview=FitH}
\hypersetup{pdfpagemode=UseNone}
\hypersetup{pdfsource={}}
\hypersetup{pdflang={en-UK}}
\hypersetup{pdfcopyright={Copyright 2017-2018 Niklas Beisert.
  This work may be distributed and/or modified under the
  conditions of the LaTeX Project Public License, either version 1.3
  of this license or (at your option) any later version.}}
\hypersetup{pdflicenseurl={http://www.latex-project.org/lppl.txt}}
\hypersetup{pdfcontactaddress={ETH Zurich, ITP, HIT K,
  Wolfgang-Pauli-Strasse 27}}
\hypersetup{pdfcontactpostcode={8093}}
\hypersetup{pdfcontactcity={Zurich}}
\hypersetup{pdfcontactcountry={Switzerland}}
\hypersetup{pdfcontactemail={nbeisert@itp.phys.ethz.ch}}
\hypersetup{pdfcontacturl={http://people.phys.ethz.ch/\xmptilde nbeisert/}}

\newcommand{\secref}[1]{\hyperref[#1]{section \ref*{#1}}}

\parskip1ex
\parindent0pt
\let\olditemize\itemize
\def\itemize{\olditemize\parskip0pt}

\begin{document}

\title{The \textsf{childdoc} Package}
\hypersetup{pdftitle={The childdoc Package}}
\author{Niklas Beisert\\[2ex]
  Institut f\"ur Theoretische Physik\\
  Eidgen\"ossische Technische Hochschule Z\"urich\\
  Wolfgang-Pauli-Strasse 27, 8093 Z\"urich, Switzerland\\[1ex]
  \href{mailto:nbeisert@itp.phys.ethz.ch}
  {\texttt{nbeisert@itp.phys.ethz.ch}}}
\hypersetup{pdfauthor={Niklas Beisert}}
\hypersetup{pdfsubject={Manual for the LaTeX2e Package childdoc}}
\date{30 December 2018, \textsf{v2.0}}
\maketitle

\begin{abstract}\noindent
\textsf{childdoc} is a \LaTeXe{} package
that enables the direct compilation
of document sections included by |\include|
to individual files.
\end{abstract}

\begingroup
\parskip0ex
\tableofcontents
\endgroup

%%%%%%%%%%%%%%%%%%%%%%%%%%%%%%%%%%%%%%%%%%%%%%%%%%%%%%%%%%%%%%%%%%%%%%%%%%%%%%%%
%%%%%%%%%%%%%%%%%%%%%%%%%%%%%%%%%%%%%%%%%%%%%%%%%%%%%%%%%%%%%%%%%%%%%%%%%%%%%%%%
\section{Introduction}

\LaTeX{} provides a mechanism to structure a large document (such as a book)
into a main file and several child files (containing the chapters)
using the |\include| command.
This mechanism is beneficial for documents
which span hundreds of pages in order to
make the source file(s) more manageable.
Moreover, compilation can be restricted to
selected child files by means of the |\includeonly| command.
The latter feature can be used to reduce the compilation time while editing
(this was significantly more useful in the earlier days of \LaTeX{})
or to generate a smaller document which is easier to navigate.
Another application of |\includeonly| is to generate
documents consisting of selected parts of the complete document.

However, there are a few drawbacks of the plain |\include| mechanism:
\begin{itemize}
\item
The child files cannot be compiled on their own,
they can only be compiled via the main file.
A naive editing environment
(such as a text editor with an option
to have the current file processed by \LaTeX)
may require one to switch to the main file before compiling;
attempting to compile the child file produces errors.
\item
The main file must be modified (each time)
to adjust the |\includeonly| command
to the present needs. This easily leaves the main file in a messy state.
\item
The generated document will always carry the filename
of the main document. This is inconvenient if
several child files are to be compiled and
to be kept for distribution.
\end{itemize}

The present package provides a simple interface
to make child files individually compilable by \LaTeX{}.
Compiling a child file then has the same effect as compiling
the main file with an |\includeonly| command
to select the appropriate child.
Moreover the generated document will carry the name of the child
rather than the main file.
This resolves all three above issues.

This feature is meant to make the editing of books,
thesis documents and lecture notes somewhat more convenient.
However, the package can also be used efficiently for
composing a series of documents (such as exercise sheets)
which are typically distributed individually.
It then assists the author in generating the individual documents
(potentially in different versions)
as well as a document containing the collected series.
Another application is in developing style files
or other kinds of included material
where compilation of the style file could redirect
to a sample or test file.

%%%%%%%%%%%%%%%%%%%%%%%%%%%%%%%%%%%%%%%%%%%%%%%%%%%%%%%%%%%%%%%%%%%%%%%%%%%%%%%%
%%%%%%%%%%%%%%%%%%%%%%%%%%%%%%%%%%%%%%%%%%%%%%%%%%%%%%%%%%%%%%%%%%%%%%%%%%%%%%%%
\section{Usage}

First of all, the package \textsf{childdoc} is \emph{not} a standard
\LaTeXe{} |.sty| style file! Therefore it needs to be invoked in
a non-standard way.

%%%%%%%%%%%%%%%%%%%%%%%%%%%%%%%%%%%%%%%%%%%%%%%%%%%%%%%%%%%%%%%%%%%%%%%%%%%%%%%%
\subsection{Included Files}
\label{sec:include}

%%%%%%%%%%%%%%%%%%%%%%%%%%%%%%%%%%%%%%%%
\DescribeMacro{\childdocmain}
To use the package, add the commands
\begin{center}
\begin{tabular}{l}
|\input{childdoc.def}|\\
|\childdocmain{}|\\
\end{tabular}
\end{center}
at the very top of the main \LaTeX{} file,
in particular \emph{before} the |\documentclass| statement!
The argument of |\childdocmain| should be left empty
(but it must be present).

%%%%%%%%%%%%%%%%%%%%%%%%%%%%%%%%%%%%%%%%
\DescribeMacro{\childdocof}
Furthermore, add the commands
\begin{center}
\begin{tabular}{l}
|\input{childdoc.def}|\\
|\childdocof{|\textit{main}|}|\\
\end{tabular}
\end{center}
at the top of every child file \textit{child}
which is included by |\include{|\textit{child}|}|
from within the main file
(or at least for those files to be compiled individually).
The argument \textit{main} must be the filename of the main file.

There are a couple of
considerations in setting up the main and child documents:

%%%%%%%%%%%%%%%%%%%%%%%%%%%%%%%%%%%%%%%%
\paragraph{Restrictions.}

Please note the following restrictions:
\begin{itemize}
\item
|\childdocmain| must be called with one argument \textit{main}
to ensure compatibility with earlier version of the package.
It must either be empty (|\childdocmain{}|)
or precisely match the filename of the main file in which it is specified.
See \secref{sec:detection} for further information.
\item
The filename \textit{main} must be specified without the |.tex| extension.
\item
The filename \textit{main} is case sensitive
(even in case-insensitive file systems)
due to internal string comparison.
\item
The argument \textit{main} should be fully expanded, it cannot be a macro.
\item
Subdirectories and special characters should be avoided in filenames.
\item
The command |\childdocmain{|\textit{main}|}| must be followed by a whitespace.
It should not be followed immediately by another command
or by a comment mark `|%|'.
This is because the \TeX{} parser reads the token immediately following
the argument of |\childdocmain| and puts it
at the beginning of every child section;
however, a white\-space is ignored.
\end{itemize}

%%%%%%%%%%%%%%%%%%%%%%%%%%%%%%%%%%%%%%%%
\paragraph{Content of Main File.}

It is advisable to place all content in the child files included by |\include|.
Any output contained in the main file will appear in all child documents
unless suppressed manually;
it cannot be suppressed automatically by the |\includeonly| directive
and thus should normally be avoided.
A method to include some content in the main file
by means of conditional processing is described in \secref{sec:conditional}.

%%%%%%%%%%%%%%%%%%%%%%%%%%%%%%%%%%%%%%%%
\paragraph{Page Numbering.}

When only a part of the document is compiled,
the appropriate numbering of pages
(as well as other status parameters)
is determined from the |.aux| files.
The latter contain information from previous passes.
However this information needs to propagate through
all intermediate child documents.
Therefore the page numbering in child documents may well
be inconsistent until the complete document is compiled at least once.

A useful (if unconventional) way to always ensure a consistent
page numbering is to restart the numbering in each child document
and denote the pages by `\textit{child}|.|\textit{page}'
where \textit{child} represents the chapter/section number of the child file.
This can be achieved by the command
|\numberwithin{page}{|\textit{child}|}|
of the \textsf{amsmath} package
where \textit{child} can be |chapter| or |section|
depending on the chosen structuring.
Alternatively, one can modify the macro |\thepage| appropriately
and reset the counter |page| at the start of each child file.

%%%%%%%%%%%%%%%%%%%%%%%%%%%%%%%%%%%%%%%%%%%%%%%%%%%%%%%%%%%%%%%%%%%%%%%%%%%%%%%%
\subsection{Conditional Processing}
\label{sec:conditional}

The package provides a mechanism to compile different versions
of a document. To customise the versions further some conditional processing
can come in handy to distinguish which version is being compiled.
The package provides two macros to describe the compilation context:

%%%%%%%%%%%%%%%%%%%%%%%%%%%%%%%%%%%%%%%%
\DescribeMacro{\ifchilddoc}
The conditional |\ifchilddoc| distinguishes between the compilation of
child documents and the main document:
%
\begin{center}
|\ifchilddoc |\textit{child-code}| |[|\||else |\textit{main-code}]| \||fi|
\end{center}

%%%%%%%%%%%%%%%%%%%%%%%%%%%%%%%%%%%%%%%%
\DescribeMacro{\childdocname}
\DescribeMacro{\childdocjob}
The macro |\childdocname| contains the filename (without extension)
of the main or child file being processed.
Note that |\childdocjob| will always contain the name of the main file.

%%%%%%%%%%%%%%%%%%%%%%%%%%%%%%%%%%%%%%%%
\paragraph{Title Page.}

Conditional processing can be used to include a title or banner page
in the main document when proper precautions are taken.
Importantly, the code in the main file should ensure that the page counter
(as well as other status parameters which are stored in the |.aux| files)
takes the same value after the conditional processing.
Otherwise the page numbers may take divergent values
depending on which part is compiled.

For example, a title page could be declared by:
%
\begin{center}
\begin{tabular}{l}
|\ifchilddoc\||else|\\
|\addtocounter{page}{-1}|\\
\textit{code for title page}\\
|\newpage|\\
|\||fi|
\end{tabular}
\end{center}
%
A banner page for the child documents can be generated by:
%
\begin{center}
\begin{tabular}{l}
|\ifchilddoc|\\
|\addtocounter{page}{-1}|\\
\textit{code for banner page}\\
|\newpage|\\
|\||fi|
\end{tabular}
\end{center}
%
Here one could write a message such as:
\begin{center}
|This is the part \childdocname{} of \childdocjob{}.|
\end{center}

%%%%%%%%%%%%%%%%%%%%%%%%%%%%%%%%%%%%%%%%%%%%%%%%%%%%%%%%%%%%%%%%%%%%%%%%%%%%%%%%
\subsection{Flags}
\label{sec:flags}

The package makes it easy to generate different versions
of the main or child documents.
To this end compilation flags can be defined
and assigned different default values.
They will be particularly useful in conjunction
with the forwarding mechanism described in \secref{sec:forward}.

For example, it may be useful to have a flag |\version|
which can be set to |draft| or |final|.
The document source will contain some conditional code
depending on the value of |\version|.
Suppose further, the flag should default to |final| for the main file
and to |draft| for child files
which is a natural assignment for editing the document.
This is achieved by placing the following code
in the preamble of the main document
(below the |\childdocmain| directive):
%
\begin{center}
\begin{tabular}{l}
|\ifchilddoc|\\
|\providecommand{\version}{draft}|\\
|\||else|\\
|\providecommand{\version}{final}|\\
|\||fi|
\end{tabular}
\end{center}
%
The definition by |\providecommand| makes sure
that previous definitions are not overwritten.
Further statements |\providecommand{\version}{...}|
can thus be added before the above code to override it.

For the main file, one might add a line
(between |\childdocmain| and the above block)
%
\begin{center}
|%\ifchilddoc\||else\providecommand{\version}{draft}\||fi|
\end{center}
%
which can be uncommented to produce a draft version.
Likewise one can add a line to the very top of a child file
(above the |\childdocof{|\textit{main}|}| directive)
%
\begin{center}
|%\providecommand{\version}{final}|
\end{center}
%
which can be uncommented to produce the final version of this child document.

%%%%%%%%%%%%%%%%%%%%%%%%%%%%%%%%%%%%%%%%%%%%%%%%%%%%%%%%%%%%%%%%%%%%%%%%%%%%%%%%
\subsection{Forwarding}
\label{sec:forward}

Different versions of the main or child documents
using compilation flags as described in \secref{sec:flags}
can be (permanently) stored in different files
for convenient compilation, viewing and distribution.
To this end, the package defines a command
to pass on compilation to a different file:

%%%%%%%%%%%%%%%%%%%%%%%%%%%%%%%%%%%%%%%%
\DescribeMacro{\childdocforward}
The command |\childdocforward| redirects processing to
another source file:
%
\begin{center}
\begin{tabular}{l}
|\input{childdoc.def}|\\
|\childdocforward[|\textit{main}|]{|\textit{dest}|}|\\
\end{tabular}
\end{center}
%
The argument \textit{dest} is the destination file
(without extension).
It should be the main file or one of the child files.
Note that further \textsf{childdoc} directives
such as |\childdocof| and |\childdocforward|
in the indicated file will be processed in this form.
The optional argument \textit{main}
passes on directly to the main file \textit{main}
while pretending to compile the child \textit{dest}.
This form behaves as if \textit{dest}
issues |\childdocof{|\textit{main}|}| right away,
and no further \textsf{childdoc} directives will be processed.

%%%%%%%%%%%%%%%%%%%%%%%%%%%%%%%%%%%%%%%%
\DescribeMacro{\...prefix}
In the alternative form |\childdocforwardprefix|,
%
\begin{center}
\begin{tabular}{l}
|\input{childdoc.def}|\\
|\childdocforwardprefix[|\textit{main}|]{|\textit{prefix}|}{|\textit{dest}|}|
\end{tabular}
\end{center}
%
the destination file is determined by a pattern
depending on the current file:
To make this work, the current file must be called
`{\textit{prefix}\hspace{0.2em}\textit{suffix}}'
with \textit{prefix} matching precisely the argument.
Processing is then passed on to the file
`{\textit{dest}\hspace{0.2em}\textit{suffix}}'.
Surely, the same effect is achieved by
directly specifying the
argument `{\textit{dest}\hspace{0.2em}\textit{suffix}}'
in the first form.
However, that requires to set up a different file
for each child. With the alternative form of the command
all these files can have exactly the same content
which simplifies setting them up and maintaining them.

For example, the following file |draft.tex|
with a compilation flag |\version| as described in \secref{sec:flags}
compiles the main document as a draft:
%
\begin{center}
\begin{tabular}{l}
|\def\version{draft}|\\
|\input{childdoc.def}|\\
|\childdocforward{|\textit{main}|}|
\end{tabular}
\end{center}
%
Likewise, the following files |final|\textit{nn}|.tex|
compile the final version of the child document
|child|\textit{nn}|.tex|:
%
\begin{center}
\begin{tabular}{l}
|\def\version{final}|\\
|\input{childdoc.def}|\\
|\childdocforwardprefix{final}{child}|
\end{tabular}
\end{center}
%

Note that when several versions of a main file and/or of each child file
are to be generated, it may be convenient to set up a |Makefile| or
shell script to automatise the process.

%%%%%%%%%%%%%%%%%%%%%%%%%%%%%%%%%%%%%%%%%%%%%%%%%%%%%%%%%%%%%%%%%%%%%%%%%%%%%%%%
\subsection{Command Line Processing}
\label{sec:commandline}

The effect of redirection files can also be achieved by invoking
the \LaTeX{} compiler with a more elaborate command line.
Most conveniently this should be done as part
of a shell script or a |Makefile|.

When using \textsf{childdoc} in the main file, the following
command lines effectively perform a redirection
(note that depending on the shell being used,
backslashes may have to be doubled: `|\|' $\to$ `|\\|'):
%
\begin{center}
|... -jobname "|\textit{target}|" |\\|"|[\textit{flags}]%
|\input{childdoc.def}\childdocforward[|\textit{main}|]{|\textit{dest}|}"|
\end{center}
%
Here \textit{target} is the name of the output file,
\textit{main} is the name of the main file
and \textit{dest} is the name of the main or child file to be processed
(all filenames without extensions).
The optional argument \textit{main} can be omitted
if \textit{main} matches \textit{dest}.
Optionally, compilation \textit{flags} can be defined via |\def| commands.
This command line makes the \TeX{} engine believe
it is compiling the file \textit{target}
whose content is specified as the latter parameter.
The provided code then forwards the processing to
\textit{main} or \textit{dest} as described in \secref{sec:forward}.

%%%%%%%%%%%%%%%%%%%%%%%%%%%%%%%%%%%%%%%%%%%%%%%%%%%%%%%%%%%%%%%%%%%%%%%%%%%%%%%%
\subsection{Include by Input}
\label{sec:input}

Including child documents by |\include| has some restrictions by design.
Most notably, the content of a child document always occupies
its own set of pages; pages cannot be shared between child documents.
Usually, this behaviour makes perfect sense
because each child document contain an essential part of the document.
However, in some situations it may be desirable to compose
a document from a collection of parts
without having mandatory page breaks between then.
For this case, the package
provides a mechanism to include parts
by |\input| which can also be processed individually.
However, by construction this mechanism
requires manual handling of the content to be output.

%%%%%%%%%%%%%%%%%%%%%%%%%%%%%%%%%%%%%%%%
\DescribeMacro{\ifchilddocmanual}
The main file should be prepared as usual, see \secref{sec:include}.
However, the document body must make a distinction
between processing of an individual part and of the main document, e.g.:
%
\begin{center}
\begin{tabular}{l}
|\ifchilddocmanual|\\
|\input{\childdocname}|\\
|\||else|\\
\textit{document body with }|\input{|\textit{part}|}|\\
|\||fi|
\end{tabular}
\end{center}
%
The conditional |\ifchilddocmanual| is true whenever
a part to be included by |\input| is being compiled,
and the name of the part is stored in |\childdocname|.

%%%%%%%%%%%%%%%%%%%%%%%%%%%%%%%%%%%%%%%%
\DescribeMacro{\childdocby}
Each part to be included by |\input| should start with:
%
\begin{center}
\begin{tabular}{l}
|\input{childdoc.def}|\\
|\childdocby{|\textit{main}|}|\\
\end{tabular}
\end{center}
%
The directive |\childdocby| is similar to |\childdocof|
described in \secref{sec:include},
but the subsequent selection of content must be done manually.
To that end, both |\ifchilddoc| and |\ifchilddocmanual|
will be true upon processing of a part,
and the name of the part is stored in |\childdocname|.
Note that |\jobname| will be set to the filename of the current part
so that each part receives an individual |.aux| file
that does not interfere with the |.aux| file(s) of the main document.
This behaviour can be altered by the alternative form
|\childdocby[*]{|\textit{main}|}| (with a non-empty optional argument)
which uses the |.aux| file of the main document
by setting |\jobname| to \textit{main}.

%%%%%%%%%%%%%%%%%%%%%%%%%%%%%%%%%%%%%%%%%%%%%%%%%%%%%%%%%%%%%%%%%%%%%%%%%%%%%%%%
\subsection{Driver Development}
\label{sec:driver}

The \textsf{childdoc} mechanism can also be use for the development
of definition files such as \LaTeX{} styles or classes.
This case differs from the above setup with multiple parts
included by |\include| in that no |\includeonly| should be invoked.
This can be achieved by starting the include file
(before |\ProvidesPackage|) with:
%
\begin{center}
\begin{tabular}{l}
|\input{childdoc.def}|\\
|\childdocforward{|\textit{main}|}|\\
\end{tabular}
\end{center}
%
or alternatively with:
%
\begin{center}
\begin{tabular}{l}
|\input{childdoc.def}|\\
|\childdocby{|\textit{main}|}|\\
\end{tabular}
\end{center}
%
Both forms have slightly different effects as described above.
The main file is prepared as usual, see \secref{sec:include}.

%%%%%%%%%%%%%%%%%%%%%%%%%%%%%%%%%%%%%%%%%%%%%%%%%%%%%%%%%%%%%%%%%%%%%%%%%%%%%%%%
\subsection{Legacy Detection}
\label{sec:detection}

The directive |\childdocmain| in the main file can detect
whether the complete document or merely a child is to be compiled
even without using the directive |\childdocof|.
This method is deprecated because it is less robust
and there is no compelling reason to use it;
it is merely provided for backward compatibility
and it may be removed in future versions.

If the detection mechanism is to be used,
it is mandatory to correctly specify
the filename of the main file as the argument of |\childdocmain|:
%
\begin{center}
\begin{tabular}{l}
|\input{childdoc.def}|\\
|\childdocmain{|\textit{main}|}|\\
\end{tabular}
\end{center}
%
If |\jobname| does not match the argument \textit{main} of |\childdocmain|,
it is assumed that |\jobname| points to the child file to be compiled.
When using |\childdocmain| with the main file specified as argument,
it suffices to start a child file
with just |\input{|\textit{main}|}|
without loading of the package and using |\childdocof|.
If instead all processing is done
with the appropriate \textsf{childdoc} directives,
the argument of \textit{main} of |\childdocmain| can be empty.

An alternative version of the command line processing described
in \secref{sec:commandline} using the detection mechanism reads:
%
\begin{center}
|... -jobname "|\textit{target}|" "|[\textit{flags}]%
[|\def\jobname{|\textit{dest}|}|]|\input{|\textit{main}|}"|
\end{center}

%%%%%%%%%%%%%%%%%%%%%%%%%%%%%%%%%%%%%%%%%%%%%%%%%%%%%%%%%%%%%%%%%%%%%%%%%%%%%%%%
\subsection{Manual Code}
\label{sec:manual}

In case one cannot be certain whether the definitions file |childdoc.def|
is installed on the target \TeX{} distribution
and one prefers not to ship it,
it is conceivable to paste a few relevant commands into the sources.

To that end, drop all statements |\input{childdoc.def}|
and perform the replacements as outlined below.
Instead of |\childdocmain{|\textit{main}|}| add the following code
to the top of the main file:
%
\begin{center}
\begin{tabular}{l}
|\||ifdefined\childdocname\endinput\||fi\newif\ifchilddoc|\\
|\edef\childdocname{\scantokens\expandafter{\jobname\noexpand}}|\\
|\def\childdocmain{|\textit{main}|}\||ifx\childdocmain\childdocname\||else|\\
|\childdoctrue\includeonly{\childdocname}\let\jobname\childdocmain\||fi|\\
\end{tabular}
\end{center}
%
Instead of |\childdocof{|\textit{main}|}| just include the main file
at the top of each child file:
%
\begin{center}
|\input{|\textit{main}|}|
\end{center}
%
A simple redirection |\childdocforward{|\textit{dest}|}| is achieved by:
%
\begin{center}
|\def\jobname{|\textit{dest}|}\input{\jobname}|
\end{center}
%
The redirection with prefix
|\childdocforwardprefix[|\textit{prefix}|]{|\textit{dest}|}|
is accomplished by:
%
\begin{center}
\begin{tabular}{l}
|{\edef\jobname{\scantokens\expandafter{\jobname\noexpand}}|\\
|\def\redirectjob |\textit{prefix}|#1~~~{\gdef\jobname{|\textit{dest}|#1}}|\\
|\expandafter\redirectjob\jobname~~~}\input{\jobname}|
\end{tabular}
\end{center}

In an alternative approach,
child documents can be compiled by a specific command line
without additional code or specific definitions:
%
\begin{center}
|... -jobname "|\textit{target}|" "|[\textit{flags}]%
|\includeonly{|\textit{dest}|}\input{|\textit{main}|}"|
\end{center}
%

%%%%%%%%%%%%%%%%%%%%%%%%%%%%%%%%%%%%%%%%%%%%%%%%%%%%%%%%%%%%%%%%%%%%%%%%%%%%%%%%
%%%%%%%%%%%%%%%%%%%%%%%%%%%%%%%%%%%%%%%%%%%%%%%%%%%%%%%%%%%%%%%%%%%%%%%%%%%%%%%%
\section{Information}

%%%%%%%%%%%%%%%%%%%%%%%%%%%%%%%%%%%%%%%%%%%%%%%%%%%%%%%%%%%%%%%%%%%%%%%%%%%%%%%%
\subsection{Copyright}

Copyright \copyright{} 2017--2018 Niklas Beisert

This work may be distributed and/or modified under the
conditions of the \LaTeX{} Project Public License, either version 1.3
of this license or (at your option) any later version.
The latest version of this license is in
  \url{http://www.latex-project.org/lppl.txt}
and version 1.3 or later is part of all distributions of \LaTeX{}
version 2005/12/01 or later.

This work has the LPPL maintenance status `maintained'.

The Current Maintainer of this work is Niklas Beisert.

This work consists of the files |README.txt|, |childdoc.ins| and |childdoc.dtx|
as well as the derived files |childdoc.def|, |cdocsamp.tex|
with |cdocsch1.tex|, |cdocsch2.tex|, |cdocspt3.tex|, |cdocspt4.tex|,
|cdocsdrf.tex|, |cdocsfn1.tex|, |cdocsfn2.tex|
as well as |childdoc.pdf|.

%%%%%%%%%%%%%%%%%%%%%%%%%%%%%%%%%%%%%%%%%%%%%%%%%%%%%%%%%%%%%%%%%%%%%%%%%%%%%%%%
\subsection{Files and Installation}

The package consists of the files:
%
\begin{center}
\begin{tabular}{ll}
    |README.txt|   & readme file \\
    |childdoc.ins| & installation file \\
    |childdoc.dtx| & source file \\
    |childdoc.def| & definition file \\
    |cdocsamp.tex| & sample main file \\
    |cdocsch1.tex| & sample include file \\
    |cdocsch2.tex| & sample include file \\
    |cdocspt3.tex| & sample part file \\
    |cdocspt4.tex| & sample part file \\
    |cdocsdrf.tex| & sample redirection file \\
    |cdocsfn1.tex| & sample redirection file \\
    |cdocsfn2.tex| & sample redirection file \\
    |childdoc.pdf| & manual
\end{tabular}
\end{center}
%
The distribution consists of the files
|README.txt|, |childdoc.ins| and |childdoc.dtx|.
%
\begin{itemize}
\item
Run (pdf)\LaTeX{} on |childdoc.dtx|
to compile the manual |childdoc.pdf| (this file).
\item
Run \LaTeX{} on |childdoc.ins| to create the definitions file |childdoc.def|
and the sample |cdocsamp.tex| with include files
|cdocsch1.tex|, |cdocsch2.tex|, |cdocspt3.tex|, |cdocspt4.tex|,
|cdocsdrf.tex|, |cdocsfn1.tex|, |cdocsfn2.tex|.
Then copy the file |childdoc.def| to an appropriate directory of your \LaTeX{}
distribution, e.g.\ \textit{texmf-root}|/tex/latex/childdoc|.
\end{itemize}

%%%%%%%%%%%%%%%%%%%%%%%%%%%%%%%%%%%%%%%%%%%%%%%%%%%%%%%%%%%%%%%%%%%%%%%%%%%%%%%%
\subsection{Related CTAN Packages}

There are several other packages which offer a similar functionality:
%
\begin{itemize}
\item
The packages
\href{http://ctan.org/pkg/docmute}{\textsf{docmute}},
\href{http://ctan.org/pkg/includex}{\textsf{includex}} and
\href{http://ctan.org/pkg/standalone}{\textsf{standalone}}
provide commands to include only the document body of
a child file thus allowing both files to be compiled individually.
\item
The packages \href{http://ctan.org/pkg/subdocs}{\textsf{subdocs}}
and \href{http://ctan.org/pkg/subfiles}{\textsf{subfiles}}
provide structures in which the main and child documents can be
encapsulated and allowing them to be compiled individually.
The inclusion mechanism is different from the conventional |\include|.
\item
The package \href{http://ctan.org/pkg/combine}{\textsf{combine}}
is an elaborate solution to combine several documents into one.
\end{itemize}
%
See also the CTAN topic \href{http://ctan.org/topic/subdocs}{\textsf{subdocs}}
for further related packages.
The present package differs from the above solutions in that
a document structure constructed with the conventional |\include| mechanism
just needs two extra commands at the top of every file
such that all constituent files can be compiled individually.

%%%%%%%%%%%%%%%%%%%%%%%%%%%%%%%%%%%%%%%%%%%%%%%%%%%%%%%%%%%%%%%%%%%%%%%%%%%%%%%%
%\subsection{Feature Suggestions}
%
%The following is a list of features which may be useful for future
%versions of this package:
%%
%\begin{itemize}
%\item
%\ldots
%\end{itemize}

%%%%%%%%%%%%%%%%%%%%%%%%%%%%%%%%%%%%%%%%%%%%%%%%%%%%%%%%%%%%%%%%%%%%%%%%%%%%%%%%
\subsection{Revision History}

%%%%%%%%%%%%%%%%%%%%%%%%%%%%%%%%%%%%%%%%
\paragraph{v2.0:} 2018/12/30

\begin{itemize}
\item
immediate forward processing
\item
added |\childdocby| mechanism
\item
manual restructured
\end{itemize}

%%%%%%%%%%%%%%%%%%%%%%%%%%%%%%%%%%%%%%%%
\paragraph{v1.6:} 2018/01/17

\begin{itemize}
\item
application for development of include files
\item
corrections to manual
\end{itemize}

%%%%%%%%%%%%%%%%%%%%%%%%%%%%%%%%%%%%%%%%
\paragraph{v1.5:} 2017/05/21

\begin{itemize}
\item
more complete structuring introduced
\item
|\childdocof| introduced
\item
|\childdoc| renamed to |\childdocmain|
\item
|\childredirect| renamed to |\childdocforward| and |\childdocforwardprefix|
and functionality expanded
\end{itemize}

%%%%%%%%%%%%%%%%%%%%%%%%%%%%%%%%%%%%%%%%
\paragraph{v1.0:} 2017/04/27

\begin{itemize}
\item
manual and install package
\item
first version published on CTAN
\end{itemize}

%%%%%%%%%%%%%%%%%%%%%%%%%%%%%%%%%%%%%%%%
\paragraph{v0.6:} 2017/04/26

\begin{itemize}
\item
redirection mechanism added
\end{itemize}

%%%%%%%%%%%%%%%%%%%%%%%%%%%%%%%%%%%%%%%%
\paragraph{v0.5:} 2017/04/26

\begin{itemize}
\item
functionality in definition file
\end{itemize}


%%%%%%%%%%%%%%%%%%%%%%%%%%%%%%%%%%%%%%%%%%%%%%%%%%%%%%%%%%%%%%%%%%%%%%%%%%%%%%%%
%%%%%%%%%%%%%%%%%%%%%%%%%%%%%%%%%%%%%%%%%%%%%%%%%%%%%%%%%%%%%%%%%%%%%%%%%%%%%%%%
%%%%%%%%%%%%%%%%%%%%%%%%%%%%%%%%%%%%%%%%%%%%%%%%%%%%%%%%%%%%%%%%%%%%%%%%%%%%%%%%
\appendix

\settowidth\MacroIndent{\rmfamily\scriptsize 000\ }

 \DocInput{childdoc.dtx}

\end{document}
%</driver>
% \fi
%
% %%%%%%%%%%%%%%%%%%%%%%%%%%%%%%%%%%%%%%%%%%%%%%%%%%%%%%%%%%%%%%%%%%%%%%%%%%%%%%
% %%%%%%%%%%%%%%%%%%%%%%%%%%%%%%%%%%%%%%%%%%%%%%%%%%%%%%%%%%%%%%%%%%%%%%%%%%%%%%
% \section{Sample}
%\iffalse
%<*samplemain>
%\fi
%
% The following presents a sample document
% with two chapters, two parts, a title page,
% a compile flag as well as three forwarding files to set the flag.
% It consists of eight |.tex| files:
% \begin{center}
% \begin{tabular}{ll}
% |cdocsamp.tex|&main file\\
% |cdocsch1.tex|&include file for chapter 1\\
% |cdocsch2.tex|&include file for chapter 2\\
% |cdocspt3.tex|&include file for part 3\\
% |cdocspt4.tex|&include file for part 4\\
% |cdocsdrf.tex|&forwarding file for main file in draft mode\\
% |cdocsfi1.tex|&forwarding file for final version of chapter 1\\
% |cdocsfi2.tex|&forwarding file for final version of chapter 2\\
% \end{tabular}
% \end{center}
% Each of the eight files can be compiled directly by the \LaTeX{} compiler.
%
% %%%%%%%%%%%%%%%%%%%%%%%%%%%%%%%%%%%%%%
% \paragraph{Main File.}
%
% The main file is called |cdocsamp.tex|.
%
% Load the \textsf{childdoc} definitions and
% declare the filename for the main document:
%    \begin{macrocode}
\input{childdoc.def}
\childdocmain{}
%    \end{macrocode}

% Optional override for |\version| flag:
%    \begin{macrocode}
%%\ifchilddoc\else\providecommand{\version}{draft}\fi
%    \end{macrocode}

% Define the default values for the |\version| flag
% (|final| for the main file and |draft| for childs):
%    \begin{macrocode}
\ifchilddoc
\providecommand{\version}{draft}
\else
\providecommand{\version}{final}
\fi
%    \end{macrocode}

% Load the standard document class:
%    \begin{macrocode}
\documentclass[12pt]{article}
%    \end{macrocode}

% Start the document body:
%    \begin{macrocode}
\begin{document}
%    \end{macrocode}

% Declare a title page.
% Print title, part of document being processed and version flag:
%    \begin{macrocode}
\addtocounter{page}{-1}
\begin{center}
{\LARGE\bfseries{}childdoc example\par}
\vspace{1cm}
\ifchilddoc
\ifchilddocmanual part\else chapter\fi:
`\childdocname' of `\childdocjob'\par
\else
main document: `\childdocjob'\par
\fi
version: \version\par
\end{center}
\newpage
%    \end{macrocode}

% Manually include selected file,
% otherwise process as usual:
%    \begin{macrocode}
\ifchilddocmanual
\section*{part `\childdocname'}
\input{\childdocname}
\else
%    \end{macrocode}

% Include the two chapters:
%    \begin{macrocode}
\include{cdocsch1}
\include{cdocsch2}
%    \end{macrocode}

% Include the two parts unless only chapters should be displayed:
%    \begin{macrocode}
\ifchilddoc\else
\section{part three}
\input{cdocspt3}
\section{part four}
\input{cdocspt4}
\fi
%    \end{macrocode}

% Process as usual until here:
%    \begin{macrocode}
\fi
%    \end{macrocode}

% End of document body:
%    \begin{macrocode}
\end{document}
%    \end{macrocode}
%\iffalse
%</samplemain>
%\fi
%
% %%%%%%%%%%%%%%%%%%%%%%%%%%%%%%%%%%%%%%
% \paragraph{Chapter Include Files.}
%
% The include files are called |cdocsch1.tex| and |cdocsch2.tex|.
%
%\iffalse
%<*samplechap1|samplechap2>
%\fi

% Optional override for |\version| flag:
%    \begin{macrocode}
%%\providecommand{\version}{final}
%    \end{macrocode}

% Include the main document:
%    \begin{macrocode}
\input{childdoc.def}
\childdocof{cdocsamp}
%    \end{macrocode}

%\iffalse
%</samplechap1|samplechap2>
%\fi
%
%\iffalse
%<*samplechap1>
%\fi
% Some text for chapter 1:
%    \begin{macrocode}
\section{one}
some text in chapter one
%    \end{macrocode}

%\iffalse
%</samplechap1>
%\fi
% Some text for chapter 2:
%\iffalse
%<*samplechap2>
%\fi
%    \begin{macrocode}
\section{two}
more text in chapter two
%    \end{macrocode}

%\iffalse
%</samplechap2>
%\fi
%
% %%%%%%%%%%%%%%%%%%%%%%%%%%%%%%%%%%%%%%
% \paragraph{Part Include Files.}
%
% The include files are called |cdocspt3.tex| and |cdocspt4.tex|.
%
%\iffalse
%<*samplepart3|samplepart4>
%\fi

% Optional override for |\version| flag:
%    \begin{macrocode}
%%\providecommand{\version}{final}
%    \end{macrocode}

% Include the main document:
%    \begin{macrocode}
\input{childdoc.def}
\childdocby{cdocsamp}
%    \end{macrocode}

%\iffalse
%</samplepart3|samplepart4>
%\fi
%
%\iffalse
%<*samplepart3>
%\fi
% Some text for part 3:
%    \begin{macrocode}
some text in part three
%    \end{macrocode}

%\iffalse
%</samplepart3>
%\fi
% Some text for part 4:
%\iffalse
%<*samplepart4>
%\fi
%    \begin{macrocode}
more text in part four
%    \end{macrocode}

%\iffalse
%</samplepart4>
%\fi
%
% %%%%%%%%%%%%%%%%%%%%%%%%%%%%%%%%%%%%%%
% \paragraph{Forwarding for a Complete Draft.}
%
% The following forwarding file |cdocsdrf.tex|
% compiles the main document in draft mode:
%\iffalse
%<*sampledraft>
%\fi
%    \begin{macrocode}
\def\version{draft}
\input{childdoc.def}
\childdocforward{cdocsamp}
%    \end{macrocode}

%\iffalse
%</sampledraft>
%\fi
%
% %%%%%%%%%%%%%%%%%%%%%%%%%%%%%%%%%%%%%%
% \paragraph{Forwarding for Final Version of the Chapters.}
%
% The following forwarding files |cdocsfn1.tex| and |cdocsfn2.tex|
% (with identical content)
% compile the final versions of the child documents
% |cdocsch1.tex| and |cdocsch2.tex|, respectively:
%\iffalse
%<*samplefinal>
%\fi
%    \begin{macrocode}
\def\version{final}
\input{childdoc.def}
\childdocforwardprefix[cdocsamp]{cdocsfn}{cdocsch}
%    \end{macrocode}

%\iffalse
%</samplefinal>
%\fi
%
% %%%%%%%%%%%%%%%%%%%%%%%%%%%%%%%%%%%%%%
% \paragraph{Command Line Processing.}
%
% The following three command lines generate the output files
% |cdocscld|, |cdocscl1| and |cdocscl2|
% which should be identical to
% |cdocsdrf|, |cdocsch1| and |cdocsfn2|, respectively:
% \begin{center}
% \begin{tabular}{l}
% |latex -jobname cdocscld \|\\
% |  "\def\version{draft}\input{childdoc.def}\childdocforward{cdocsamp}"|\\
% |latex -jobname cdocscl1 \|\\
% |  "\input{childdoc.def}\childdocforward[cdocsamp]{cdocsch1}"|\\
% |latex -jobname cdocscl2 \|\\
% |  "\def\version{final}\input{childdoc.def}\childdocforward{cdocsch2}"|
% \end{tabular}
% \end{center}
% Note that the trailing backslash on each first line
% merely continues the input to the second line
% (for convenient cut ant paste).
% Furthermore, the command |latex| can be replaced by any
% of its alternative versions such as |pdflatex|.
%
% %%%%%%%%%%%%%%%%%%%%%%%%%%%%%%%%%%%%%%%%%%%%%%%%%%%%%%%%%%%%%%%%%%%%%%%%%%%%%%
% %%%%%%%%%%%%%%%%%%%%%%%%%%%%%%%%%%%%%%%%%%%%%%%%%%%%%%%%%%%%%%%%%%%%%%%%%%%%%%
% \section{Implementation}
%\iffalse
%<*package>
%\fi
%
% This section describes the definitions file |childdoc.def|.

% The definitions cannot be loaded using |\usepackage| or |\RequirePackage|
% which has a mechanism to prevent loading a style file more than once.
% When loading the definitions by means of |\input|
% multiple instances have to be prevented manually:
%\iffalse
%This code needs to be before the `\ProvidesFile' directive
%which is defined at the beginning of this file.
%Therefore it is also placed there and commented out here.
%</package>
%<*discard>
%\fi
%    \begin{macrocode}
\ifdefined\childdocmain\endinput\fi
%    \end{macrocode}
%\iffalse
%</discard>
%<*package>
%\fi
%
% \macro{\ifchilddoc}
% \macro{\ifchilddocmanual}
% The conditional |\ifchilddoc| tells whether a
% child (true) or main (false) document is being compiled.
% The conditional |\ifchilddocmanual| tells whether
% the |\includeonly| mechanism is used (false) or
% the selection of child files must be performed manually (true).
% The definitions initialise to false:
%    \begin{macrocode}
\newif\ifchilddoc
\newif\ifchilddocmanual
%    \end{macrocode}

% \macro{\childdocname}
% \macro{\childdocjob}
% The macro |\childdocname| stores the name of the main document
% to be compiled. The macro |\childdocjob| stores the name of
% the document on which the \LaTeX{} compiler was originally invoked.
% The content of |\jobname| cannot be compared
% to filenames specified in the source due to different catcodes.
% The following code rescans |\jobname|, stores the result
% in |\childdocname| and saves a copy in |\childdocjob|:
%    \begin{macrocode}
\edef\childdocname{\scantokens\expandafter{\jobname\noexpand}}
\let\childdocjob\childdocname
%    \end{macrocode}

% \macro{\childdocdisable}
% The macro |\childdocdisable| prevents the main file
% from being processed more than once.
% At this stage, the main document command |\childdocmain|
% is assumed to be called once again where it should do nothing.
% Any subsequent call to it should prevent
% a secondary processing of the main document
% It overwrites the forwarding commands
% |\childdocof| and |\childdocforward|
% with empty macros to prevent further inclusions of the main document:
%    \begin{macrocode}
\newcommand{\childdocdisable}
{
  \renewcommand{\childdocmain}[1]{\renewcommand{\childdocmain}[1]{\endinput}}
  \renewcommand{\childdocof}[1]{}
  \renewcommand{\childdocby}[2][]{}
  \renewcommand{\childdocforward}[2][]{}
  \renewcommand{\childdocdisable}{}
}
%    \end{macrocode}

% \macro{\childdocmain}
% The macro |\childdocmain| is to be called at the top of the main file
% with nothing or the main filename (without extension) as argument.
% First, it breaks loops.
% If the argument is not empty and does not match |\childdocname|
% (which is set by the first inclusion of |childdoc.def|),
% |\ifchilddoc| is set to true, |\includeonly| is applied to the child file
% and |\jobname| is set to the main file
% (for proper handling of |.aux| files):
%    \begin{macrocode}
\newcommand{\childdocmain}[1]
{
  \childdocdisable\childdocmain{}
  \if?#1?\else
    \begingroup
      \def\childdoctmp{#1}
      \ifx\childdoctmp\childdocname
        \def\childdoctmp{}
      \else
        \def\childdoctmp
        {
          \childdoctrue
          \includeonly{\childdocname}
          \def\childdocjob{#1}
          \def\jobname{#1}
        }
      \fi
      \expandafter
    \endgroup
    \childdoctmp
  \fi
}
%    \end{macrocode}

% \macro{\childdocof}
% The command |\childdocof| redirects
% compilation to the main file |#1|.
%    \begin{macrocode}
\newcommand{\childdocof}[1]
{
  \childdocdisable
  \childdoctrue
  \includeonly{\childdocname}
  \def\jobname{#1}
  \def\childdocjob{#1}
  \input{#1}
}
%    \end{macrocode}

% \macro{\childdocby}
% The command |\childdocby| ....
%    \begin{macrocode}
\newcommand{\childdocby}[2][]
{
  \childdocdisable
  \childdoctrue
  \childdocmanualtrue
  \if?#1?\else
    \def\jobname{#2}
  \fi
  \def\childdocjob{#2}
  \input{#2}
  \endinput
}
%    \end{macrocode}

% \macro{\childdocforward}
% The command |\childdocforward| redirects
% compilation to the main file or
% (if the optional argument is given) a child file.
% Parameters are set as if the main file
% or a child file starting with |\childdocof| was compiled.
% Then compilation is handed over to the main file:
%    \begin{macrocode}
\newcommand{\childdocforward}[2][]
{
  \begingroup
    \if?#1?
      \def\childdoctmp
      {
        \def\childdocname{#2}
        \def\childdocjob{#2}
        \def\jobname{#2}
        \input{#2}
        \endinput
      }
    \else
      \def\childdoctmp
      {
        \childdocdisable
        \def\childdocname{#2}
        \childdoctrue
        \includeonly{#2}
        \def\childdocjob{#1}
        \def\jobname{#1}
        \input{#1}
        \endinput
      }
    \fi
    \expandafter
  \endgroup
  \childdoctmp
}
%    \end{macrocode}

% \macro{\childdocforwardprefix}
% The command |\childdocforwardprefix| redirects
% compilation to the main or a child file by means of a pattern.
% The prefix |#1| in the current filename is replaced by |#2|
% and the suffix of the current filename is kept
% (it is assumed that the filename does not contain the substring `|~~~|'
% which is used as a delimiter).
% Compilation is handed over to the new file by |\childdocforward|:
%    \begin{macrocode}
\newcommand{\childdocforwardprefix}[3][]
{
  \begingroup
    \def\childdocextract #2##1~~~{\def\childdoctmp{\childdocforward[#1]{#3##1}}}
    \expandafter\childdocextract\childdocname~~~
    \expandafter
  \endgroup
  \childdoctmp
}
%    \end{macrocode}

% \macro{\childdoc}
% The deprecated macro |\childdoc| is a legacy version of |\childdocmain|:
%    \begin{macrocode}
\newcommand{\childdoc}{\childdocmain}
%    \end{macrocode}

% \macro{\childdocredirect}
% The deprecated macro |\childdocredirect| is a legacy version
% of |\childdocforward| and |\childdocforwardprefix|:
%    \begin{macrocode}
\newcommand{\childdocredirect}[2][]
{
  \begingroup
    \if?#1?
      \def\childdoctmp{\childdocforward{#2}}
    \else
      \def\childdoctmp{\childdocforwardprefix{#1}{#2}}
    \fi
    \expandafter
  \endgroup
  \childdoctmp
}
%    \end{macrocode}

%\iffalse
%</package>
%\fi
%
\endinput
|
and perform the replacements as outlined below.
Instead of |\childdocmain{|\textit{main}|}| add the following code
to the top of the main file:
%
\begin{center}
\begin{tabular}{l}
|\||ifdefined\childdocname\endinput\||fi\newif\ifchilddoc|\\
|\edef\childdocname{\scantokens\expandafter{\jobname\noexpand}}|\\
|\def\childdocmain{|\textit{main}|}\||ifx\childdocmain\childdocname\||else|\\
|\childdoctrue\includeonly{\childdocname}\let\jobname\childdocmain\||fi|\\
\end{tabular}
\end{center}
%
Instead of |\childdocof{|\textit{main}|}| just include the main file
at the top of each child file:
%
\begin{center}
|\input{|\textit{main}|}|
\end{center}
%
A simple redirection |\childdocforward{|\textit{dest}|}| is achieved by:
%
\begin{center}
|\def\jobname{|\textit{dest}|}\input{\jobname}|
\end{center}
%
The redirection with prefix
|\childdocforwardprefix[|\textit{prefix}|]{|\textit{dest}|}|
is accomplished by:
%
\begin{center}
\begin{tabular}{l}
|{\edef\jobname{\scantokens\expandafter{\jobname\noexpand}}|\\
|\def\redirectjob |\textit{prefix}|#1~~~{\gdef\jobname{|\textit{dest}|#1}}|\\
|\expandafter\redirectjob\jobname~~~}\input{\jobname}|
\end{tabular}
\end{center}

In an alternative approach,
child documents can be compiled by a specific command line
without additional code or specific definitions:
%
\begin{center}
|... -jobname "|\textit{target}|" "|[\textit{flags}]%
|\includeonly{|\textit{dest}|}\input{|\textit{main}|}"|
\end{center}
%

%%%%%%%%%%%%%%%%%%%%%%%%%%%%%%%%%%%%%%%%%%%%%%%%%%%%%%%%%%%%%%%%%%%%%%%%%%%%%%%%
%%%%%%%%%%%%%%%%%%%%%%%%%%%%%%%%%%%%%%%%%%%%%%%%%%%%%%%%%%%%%%%%%%%%%%%%%%%%%%%%
\section{Information}

%%%%%%%%%%%%%%%%%%%%%%%%%%%%%%%%%%%%%%%%%%%%%%%%%%%%%%%%%%%%%%%%%%%%%%%%%%%%%%%%
\subsection{Copyright}

Copyright \copyright{} 2017--2018 Niklas Beisert

This work may be distributed and/or modified under the
conditions of the \LaTeX{} Project Public License, either version 1.3
of this license or (at your option) any later version.
The latest version of this license is in
  \url{http://www.latex-project.org/lppl.txt}
and version 1.3 or later is part of all distributions of \LaTeX{}
version 2005/12/01 or later.

This work has the LPPL maintenance status `maintained'.

The Current Maintainer of this work is Niklas Beisert.

This work consists of the files |README.txt|, |childdoc.ins| and |childdoc.dtx|
as well as the derived files |childdoc.def|, |cdocsamp.tex|
with |cdocsch1.tex|, |cdocsch2.tex|, |cdocspt3.tex|, |cdocspt4.tex|,
|cdocsdrf.tex|, |cdocsfn1.tex|, |cdocsfn2.tex|
as well as |childdoc.pdf|.

%%%%%%%%%%%%%%%%%%%%%%%%%%%%%%%%%%%%%%%%%%%%%%%%%%%%%%%%%%%%%%%%%%%%%%%%%%%%%%%%
\subsection{Files and Installation}

The package consists of the files:
%
\begin{center}
\begin{tabular}{ll}
    |README.txt|   & readme file \\
    |childdoc.ins| & installation file \\
    |childdoc.dtx| & source file \\
    |childdoc.def| & definition file \\
    |cdocsamp.tex| & sample main file \\
    |cdocsch1.tex| & sample include file \\
    |cdocsch2.tex| & sample include file \\
    |cdocspt3.tex| & sample part file \\
    |cdocspt4.tex| & sample part file \\
    |cdocsdrf.tex| & sample redirection file \\
    |cdocsfn1.tex| & sample redirection file \\
    |cdocsfn2.tex| & sample redirection file \\
    |childdoc.pdf| & manual
\end{tabular}
\end{center}
%
The distribution consists of the files
|README.txt|, |childdoc.ins| and |childdoc.dtx|.
%
\begin{itemize}
\item
Run (pdf)\LaTeX{} on |childdoc.dtx|
to compile the manual |childdoc.pdf| (this file).
\item
Run \LaTeX{} on |childdoc.ins| to create the definitions file |childdoc.def|
and the sample |cdocsamp.tex| with include files
|cdocsch1.tex|, |cdocsch2.tex|, |cdocspt3.tex|, |cdocspt4.tex|,
|cdocsdrf.tex|, |cdocsfn1.tex|, |cdocsfn2.tex|.
Then copy the file |childdoc.def| to an appropriate directory of your \LaTeX{}
distribution, e.g.\ \textit{texmf-root}|/tex/latex/childdoc|.
\end{itemize}

%%%%%%%%%%%%%%%%%%%%%%%%%%%%%%%%%%%%%%%%%%%%%%%%%%%%%%%%%%%%%%%%%%%%%%%%%%%%%%%%
\subsection{Related CTAN Packages}

There are several other packages which offer a similar functionality:
%
\begin{itemize}
\item
The packages
\href{http://ctan.org/pkg/docmute}{\textsf{docmute}},
\href{http://ctan.org/pkg/includex}{\textsf{includex}} and
\href{http://ctan.org/pkg/standalone}{\textsf{standalone}}
provide commands to include only the document body of
a child file thus allowing both files to be compiled individually.
\item
The packages \href{http://ctan.org/pkg/subdocs}{\textsf{subdocs}}
and \href{http://ctan.org/pkg/subfiles}{\textsf{subfiles}}
provide structures in which the main and child documents can be
encapsulated and allowing them to be compiled individually.
The inclusion mechanism is different from the conventional |\include|.
\item
The package \href{http://ctan.org/pkg/combine}{\textsf{combine}}
is an elaborate solution to combine several documents into one.
\end{itemize}
%
See also the CTAN topic \href{http://ctan.org/topic/subdocs}{\textsf{subdocs}}
for further related packages.
The present package differs from the above solutions in that
a document structure constructed with the conventional |\include| mechanism
just needs two extra commands at the top of every file
such that all constituent files can be compiled individually.

%%%%%%%%%%%%%%%%%%%%%%%%%%%%%%%%%%%%%%%%%%%%%%%%%%%%%%%%%%%%%%%%%%%%%%%%%%%%%%%%
%\subsection{Feature Suggestions}
%
%The following is a list of features which may be useful for future
%versions of this package:
%%
%\begin{itemize}
%\item
%\ldots
%\end{itemize}

%%%%%%%%%%%%%%%%%%%%%%%%%%%%%%%%%%%%%%%%%%%%%%%%%%%%%%%%%%%%%%%%%%%%%%%%%%%%%%%%
\subsection{Revision History}

%%%%%%%%%%%%%%%%%%%%%%%%%%%%%%%%%%%%%%%%
\paragraph{v2.0:} 2018/12/30

\begin{itemize}
\item
immediate forward processing
\item
added |\childdocby| mechanism
\item
manual restructured
\end{itemize}

%%%%%%%%%%%%%%%%%%%%%%%%%%%%%%%%%%%%%%%%
\paragraph{v1.6:} 2018/01/17

\begin{itemize}
\item
application for development of include files
\item
corrections to manual
\end{itemize}

%%%%%%%%%%%%%%%%%%%%%%%%%%%%%%%%%%%%%%%%
\paragraph{v1.5:} 2017/05/21

\begin{itemize}
\item
more complete structuring introduced
\item
|\childdocof| introduced
\item
|\childdoc| renamed to |\childdocmain|
\item
|\childredirect| renamed to |\childdocforward| and |\childdocforwardprefix|
and functionality expanded
\end{itemize}

%%%%%%%%%%%%%%%%%%%%%%%%%%%%%%%%%%%%%%%%
\paragraph{v1.0:} 2017/04/27

\begin{itemize}
\item
manual and install package
\item
first version published on CTAN
\end{itemize}

%%%%%%%%%%%%%%%%%%%%%%%%%%%%%%%%%%%%%%%%
\paragraph{v0.6:} 2017/04/26

\begin{itemize}
\item
redirection mechanism added
\end{itemize}

%%%%%%%%%%%%%%%%%%%%%%%%%%%%%%%%%%%%%%%%
\paragraph{v0.5:} 2017/04/26

\begin{itemize}
\item
functionality in definition file
\end{itemize}


%%%%%%%%%%%%%%%%%%%%%%%%%%%%%%%%%%%%%%%%%%%%%%%%%%%%%%%%%%%%%%%%%%%%%%%%%%%%%%%%
%%%%%%%%%%%%%%%%%%%%%%%%%%%%%%%%%%%%%%%%%%%%%%%%%%%%%%%%%%%%%%%%%%%%%%%%%%%%%%%%
%%%%%%%%%%%%%%%%%%%%%%%%%%%%%%%%%%%%%%%%%%%%%%%%%%%%%%%%%%%%%%%%%%%%%%%%%%%%%%%%
\appendix

\settowidth\MacroIndent{\rmfamily\scriptsize 000\ }

 \DocInput{childdoc.dtx}

\end{document}
%</driver>
% \fi
%
% %%%%%%%%%%%%%%%%%%%%%%%%%%%%%%%%%%%%%%%%%%%%%%%%%%%%%%%%%%%%%%%%%%%%%%%%%%%%%%
% %%%%%%%%%%%%%%%%%%%%%%%%%%%%%%%%%%%%%%%%%%%%%%%%%%%%%%%%%%%%%%%%%%%%%%%%%%%%%%
% \section{Sample}
%\iffalse
%<*samplemain>
%\fi
%
% The following presents a sample document
% with two chapters, two parts, a title page,
% a compile flag as well as three forwarding files to set the flag.
% It consists of eight |.tex| files:
% \begin{center}
% \begin{tabular}{ll}
% |cdocsamp.tex|&main file\\
% |cdocsch1.tex|&include file for chapter 1\\
% |cdocsch2.tex|&include file for chapter 2\\
% |cdocspt3.tex|&include file for part 3\\
% |cdocspt4.tex|&include file for part 4\\
% |cdocsdrf.tex|&forwarding file for main file in draft mode\\
% |cdocsfi1.tex|&forwarding file for final version of chapter 1\\
% |cdocsfi2.tex|&forwarding file for final version of chapter 2\\
% \end{tabular}
% \end{center}
% Each of the eight files can be compiled directly by the \LaTeX{} compiler.
%
% %%%%%%%%%%%%%%%%%%%%%%%%%%%%%%%%%%%%%%
% \paragraph{Main File.}
%
% The main file is called |cdocsamp.tex|.
%
% Load the \textsf{childdoc} definitions and
% declare the filename for the main document:
%    \begin{macrocode}
% \iffalse
%
% childdoc.dtx Copyright (C) 2017-2018 Niklas Beisert
%
% This work may be distributed and/or modified under the
% conditions of the LaTeX Project Public License, either version 1.3
% of this license or (at your option) any later version.
% The latest version of this license is in
%   http://www.latex-project.org/lppl.txt
% and version 1.3 or later is part of all distributions of LaTeX
% version 2005/12/01 or later.
%
% This work has the LPPL maintenance status `maintained'.
%
% The Current Maintainer of this work is Niklas Beisert.
%
% This work consists of the files childdoc.dtx and childdoc.ins
% and the derived files childdoc.def and cdocsamp.tex with
% cdocsch1.tex, cdocsch2.tex, cdocsdrf.tex, cdocsfn1.tex, cdocsfn2.tex.
%
%<package>\ifdefined\childdocmain\endinput\fi
%<package>\ProvidesFile{childdoc.def}[2018/12/30 v2.0 child document driver]
%<samplemain>\ProvidesFile{cdocsamp.tex}[2018/12/30 v2.0 sample for childdoc]
%<*driver>
%\ProvidesFile{childdoc.drv}[2018/12/30 v2.0 childdoc reference manual file]
\PassOptionsToClass{10pt,a4paper}{article}
\documentclass{ltxdoc}

\usepackage[margin=35mm]{geometry}
\usepackage{hyperref}
\usepackage{hyperxmp}
\usepackage[usenames]{color}

\hypersetup{colorlinks=true}
\hypersetup{pdfstartview=FitH}
\hypersetup{pdfpagemode=UseNone}
\hypersetup{pdfsource={}}
\hypersetup{pdflang={en-UK}}
\hypersetup{pdfcopyright={Copyright 2017-2018 Niklas Beisert.
  This work may be distributed and/or modified under the
  conditions of the LaTeX Project Public License, either version 1.3
  of this license or (at your option) any later version.}}
\hypersetup{pdflicenseurl={http://www.latex-project.org/lppl.txt}}
\hypersetup{pdfcontactaddress={ETH Zurich, ITP, HIT K,
  Wolfgang-Pauli-Strasse 27}}
\hypersetup{pdfcontactpostcode={8093}}
\hypersetup{pdfcontactcity={Zurich}}
\hypersetup{pdfcontactcountry={Switzerland}}
\hypersetup{pdfcontactemail={nbeisert@itp.phys.ethz.ch}}
\hypersetup{pdfcontacturl={http://people.phys.ethz.ch/\xmptilde nbeisert/}}

\newcommand{\secref}[1]{\hyperref[#1]{section \ref*{#1}}}

\parskip1ex
\parindent0pt
\let\olditemize\itemize
\def\itemize{\olditemize\parskip0pt}

\begin{document}

\title{The \textsf{childdoc} Package}
\hypersetup{pdftitle={The childdoc Package}}
\author{Niklas Beisert\\[2ex]
  Institut f\"ur Theoretische Physik\\
  Eidgen\"ossische Technische Hochschule Z\"urich\\
  Wolfgang-Pauli-Strasse 27, 8093 Z\"urich, Switzerland\\[1ex]
  \href{mailto:nbeisert@itp.phys.ethz.ch}
  {\texttt{nbeisert@itp.phys.ethz.ch}}}
\hypersetup{pdfauthor={Niklas Beisert}}
\hypersetup{pdfsubject={Manual for the LaTeX2e Package childdoc}}
\date{30 December 2018, \textsf{v2.0}}
\maketitle

\begin{abstract}\noindent
\textsf{childdoc} is a \LaTeXe{} package
that enables the direct compilation
of document sections included by |\include|
to individual files.
\end{abstract}

\begingroup
\parskip0ex
\tableofcontents
\endgroup

%%%%%%%%%%%%%%%%%%%%%%%%%%%%%%%%%%%%%%%%%%%%%%%%%%%%%%%%%%%%%%%%%%%%%%%%%%%%%%%%
%%%%%%%%%%%%%%%%%%%%%%%%%%%%%%%%%%%%%%%%%%%%%%%%%%%%%%%%%%%%%%%%%%%%%%%%%%%%%%%%
\section{Introduction}

\LaTeX{} provides a mechanism to structure a large document (such as a book)
into a main file and several child files (containing the chapters)
using the |\include| command.
This mechanism is beneficial for documents
which span hundreds of pages in order to
make the source file(s) more manageable.
Moreover, compilation can be restricted to
selected child files by means of the |\includeonly| command.
The latter feature can be used to reduce the compilation time while editing
(this was significantly more useful in the earlier days of \LaTeX{})
or to generate a smaller document which is easier to navigate.
Another application of |\includeonly| is to generate
documents consisting of selected parts of the complete document.

However, there are a few drawbacks of the plain |\include| mechanism:
\begin{itemize}
\item
The child files cannot be compiled on their own,
they can only be compiled via the main file.
A naive editing environment
(such as a text editor with an option
to have the current file processed by \LaTeX)
may require one to switch to the main file before compiling;
attempting to compile the child file produces errors.
\item
The main file must be modified (each time)
to adjust the |\includeonly| command
to the present needs. This easily leaves the main file in a messy state.
\item
The generated document will always carry the filename
of the main document. This is inconvenient if
several child files are to be compiled and
to be kept for distribution.
\end{itemize}

The present package provides a simple interface
to make child files individually compilable by \LaTeX{}.
Compiling a child file then has the same effect as compiling
the main file with an |\includeonly| command
to select the appropriate child.
Moreover the generated document will carry the name of the child
rather than the main file.
This resolves all three above issues.

This feature is meant to make the editing of books,
thesis documents and lecture notes somewhat more convenient.
However, the package can also be used efficiently for
composing a series of documents (such as exercise sheets)
which are typically distributed individually.
It then assists the author in generating the individual documents
(potentially in different versions)
as well as a document containing the collected series.
Another application is in developing style files
or other kinds of included material
where compilation of the style file could redirect
to a sample or test file.

%%%%%%%%%%%%%%%%%%%%%%%%%%%%%%%%%%%%%%%%%%%%%%%%%%%%%%%%%%%%%%%%%%%%%%%%%%%%%%%%
%%%%%%%%%%%%%%%%%%%%%%%%%%%%%%%%%%%%%%%%%%%%%%%%%%%%%%%%%%%%%%%%%%%%%%%%%%%%%%%%
\section{Usage}

First of all, the package \textsf{childdoc} is \emph{not} a standard
\LaTeXe{} |.sty| style file! Therefore it needs to be invoked in
a non-standard way.

%%%%%%%%%%%%%%%%%%%%%%%%%%%%%%%%%%%%%%%%%%%%%%%%%%%%%%%%%%%%%%%%%%%%%%%%%%%%%%%%
\subsection{Included Files}
\label{sec:include}

%%%%%%%%%%%%%%%%%%%%%%%%%%%%%%%%%%%%%%%%
\DescribeMacro{\childdocmain}
To use the package, add the commands
\begin{center}
\begin{tabular}{l}
|\input{childdoc.def}|\\
|\childdocmain{}|\\
\end{tabular}
\end{center}
at the very top of the main \LaTeX{} file,
in particular \emph{before} the |\documentclass| statement!
The argument of |\childdocmain| should be left empty
(but it must be present).

%%%%%%%%%%%%%%%%%%%%%%%%%%%%%%%%%%%%%%%%
\DescribeMacro{\childdocof}
Furthermore, add the commands
\begin{center}
\begin{tabular}{l}
|\input{childdoc.def}|\\
|\childdocof{|\textit{main}|}|\\
\end{tabular}
\end{center}
at the top of every child file \textit{child}
which is included by |\include{|\textit{child}|}|
from within the main file
(or at least for those files to be compiled individually).
The argument \textit{main} must be the filename of the main file.

There are a couple of
considerations in setting up the main and child documents:

%%%%%%%%%%%%%%%%%%%%%%%%%%%%%%%%%%%%%%%%
\paragraph{Restrictions.}

Please note the following restrictions:
\begin{itemize}
\item
|\childdocmain| must be called with one argument \textit{main}
to ensure compatibility with earlier version of the package.
It must either be empty (|\childdocmain{}|)
or precisely match the filename of the main file in which it is specified.
See \secref{sec:detection} for further information.
\item
The filename \textit{main} must be specified without the |.tex| extension.
\item
The filename \textit{main} is case sensitive
(even in case-insensitive file systems)
due to internal string comparison.
\item
The argument \textit{main} should be fully expanded, it cannot be a macro.
\item
Subdirectories and special characters should be avoided in filenames.
\item
The command |\childdocmain{|\textit{main}|}| must be followed by a whitespace.
It should not be followed immediately by another command
or by a comment mark `|%|'.
This is because the \TeX{} parser reads the token immediately following
the argument of |\childdocmain| and puts it
at the beginning of every child section;
however, a white\-space is ignored.
\end{itemize}

%%%%%%%%%%%%%%%%%%%%%%%%%%%%%%%%%%%%%%%%
\paragraph{Content of Main File.}

It is advisable to place all content in the child files included by |\include|.
Any output contained in the main file will appear in all child documents
unless suppressed manually;
it cannot be suppressed automatically by the |\includeonly| directive
and thus should normally be avoided.
A method to include some content in the main file
by means of conditional processing is described in \secref{sec:conditional}.

%%%%%%%%%%%%%%%%%%%%%%%%%%%%%%%%%%%%%%%%
\paragraph{Page Numbering.}

When only a part of the document is compiled,
the appropriate numbering of pages
(as well as other status parameters)
is determined from the |.aux| files.
The latter contain information from previous passes.
However this information needs to propagate through
all intermediate child documents.
Therefore the page numbering in child documents may well
be inconsistent until the complete document is compiled at least once.

A useful (if unconventional) way to always ensure a consistent
page numbering is to restart the numbering in each child document
and denote the pages by `\textit{child}|.|\textit{page}'
where \textit{child} represents the chapter/section number of the child file.
This can be achieved by the command
|\numberwithin{page}{|\textit{child}|}|
of the \textsf{amsmath} package
where \textit{child} can be |chapter| or |section|
depending on the chosen structuring.
Alternatively, one can modify the macro |\thepage| appropriately
and reset the counter |page| at the start of each child file.

%%%%%%%%%%%%%%%%%%%%%%%%%%%%%%%%%%%%%%%%%%%%%%%%%%%%%%%%%%%%%%%%%%%%%%%%%%%%%%%%
\subsection{Conditional Processing}
\label{sec:conditional}

The package provides a mechanism to compile different versions
of a document. To customise the versions further some conditional processing
can come in handy to distinguish which version is being compiled.
The package provides two macros to describe the compilation context:

%%%%%%%%%%%%%%%%%%%%%%%%%%%%%%%%%%%%%%%%
\DescribeMacro{\ifchilddoc}
The conditional |\ifchilddoc| distinguishes between the compilation of
child documents and the main document:
%
\begin{center}
|\ifchilddoc |\textit{child-code}| |[|\||else |\textit{main-code}]| \||fi|
\end{center}

%%%%%%%%%%%%%%%%%%%%%%%%%%%%%%%%%%%%%%%%
\DescribeMacro{\childdocname}
\DescribeMacro{\childdocjob}
The macro |\childdocname| contains the filename (without extension)
of the main or child file being processed.
Note that |\childdocjob| will always contain the name of the main file.

%%%%%%%%%%%%%%%%%%%%%%%%%%%%%%%%%%%%%%%%
\paragraph{Title Page.}

Conditional processing can be used to include a title or banner page
in the main document when proper precautions are taken.
Importantly, the code in the main file should ensure that the page counter
(as well as other status parameters which are stored in the |.aux| files)
takes the same value after the conditional processing.
Otherwise the page numbers may take divergent values
depending on which part is compiled.

For example, a title page could be declared by:
%
\begin{center}
\begin{tabular}{l}
|\ifchilddoc\||else|\\
|\addtocounter{page}{-1}|\\
\textit{code for title page}\\
|\newpage|\\
|\||fi|
\end{tabular}
\end{center}
%
A banner page for the child documents can be generated by:
%
\begin{center}
\begin{tabular}{l}
|\ifchilddoc|\\
|\addtocounter{page}{-1}|\\
\textit{code for banner page}\\
|\newpage|\\
|\||fi|
\end{tabular}
\end{center}
%
Here one could write a message such as:
\begin{center}
|This is the part \childdocname{} of \childdocjob{}.|
\end{center}

%%%%%%%%%%%%%%%%%%%%%%%%%%%%%%%%%%%%%%%%%%%%%%%%%%%%%%%%%%%%%%%%%%%%%%%%%%%%%%%%
\subsection{Flags}
\label{sec:flags}

The package makes it easy to generate different versions
of the main or child documents.
To this end compilation flags can be defined
and assigned different default values.
They will be particularly useful in conjunction
with the forwarding mechanism described in \secref{sec:forward}.

For example, it may be useful to have a flag |\version|
which can be set to |draft| or |final|.
The document source will contain some conditional code
depending on the value of |\version|.
Suppose further, the flag should default to |final| for the main file
and to |draft| for child files
which is a natural assignment for editing the document.
This is achieved by placing the following code
in the preamble of the main document
(below the |\childdocmain| directive):
%
\begin{center}
\begin{tabular}{l}
|\ifchilddoc|\\
|\providecommand{\version}{draft}|\\
|\||else|\\
|\providecommand{\version}{final}|\\
|\||fi|
\end{tabular}
\end{center}
%
The definition by |\providecommand| makes sure
that previous definitions are not overwritten.
Further statements |\providecommand{\version}{...}|
can thus be added before the above code to override it.

For the main file, one might add a line
(between |\childdocmain| and the above block)
%
\begin{center}
|%\ifchilddoc\||else\providecommand{\version}{draft}\||fi|
\end{center}
%
which can be uncommented to produce a draft version.
Likewise one can add a line to the very top of a child file
(above the |\childdocof{|\textit{main}|}| directive)
%
\begin{center}
|%\providecommand{\version}{final}|
\end{center}
%
which can be uncommented to produce the final version of this child document.

%%%%%%%%%%%%%%%%%%%%%%%%%%%%%%%%%%%%%%%%%%%%%%%%%%%%%%%%%%%%%%%%%%%%%%%%%%%%%%%%
\subsection{Forwarding}
\label{sec:forward}

Different versions of the main or child documents
using compilation flags as described in \secref{sec:flags}
can be (permanently) stored in different files
for convenient compilation, viewing and distribution.
To this end, the package defines a command
to pass on compilation to a different file:

%%%%%%%%%%%%%%%%%%%%%%%%%%%%%%%%%%%%%%%%
\DescribeMacro{\childdocforward}
The command |\childdocforward| redirects processing to
another source file:
%
\begin{center}
\begin{tabular}{l}
|\input{childdoc.def}|\\
|\childdocforward[|\textit{main}|]{|\textit{dest}|}|\\
\end{tabular}
\end{center}
%
The argument \textit{dest} is the destination file
(without extension).
It should be the main file or one of the child files.
Note that further \textsf{childdoc} directives
such as |\childdocof| and |\childdocforward|
in the indicated file will be processed in this form.
The optional argument \textit{main}
passes on directly to the main file \textit{main}
while pretending to compile the child \textit{dest}.
This form behaves as if \textit{dest}
issues |\childdocof{|\textit{main}|}| right away,
and no further \textsf{childdoc} directives will be processed.

%%%%%%%%%%%%%%%%%%%%%%%%%%%%%%%%%%%%%%%%
\DescribeMacro{\...prefix}
In the alternative form |\childdocforwardprefix|,
%
\begin{center}
\begin{tabular}{l}
|\input{childdoc.def}|\\
|\childdocforwardprefix[|\textit{main}|]{|\textit{prefix}|}{|\textit{dest}|}|
\end{tabular}
\end{center}
%
the destination file is determined by a pattern
depending on the current file:
To make this work, the current file must be called
`{\textit{prefix}\hspace{0.2em}\textit{suffix}}'
with \textit{prefix} matching precisely the argument.
Processing is then passed on to the file
`{\textit{dest}\hspace{0.2em}\textit{suffix}}'.
Surely, the same effect is achieved by
directly specifying the
argument `{\textit{dest}\hspace{0.2em}\textit{suffix}}'
in the first form.
However, that requires to set up a different file
for each child. With the alternative form of the command
all these files can have exactly the same content
which simplifies setting them up and maintaining them.

For example, the following file |draft.tex|
with a compilation flag |\version| as described in \secref{sec:flags}
compiles the main document as a draft:
%
\begin{center}
\begin{tabular}{l}
|\def\version{draft}|\\
|\input{childdoc.def}|\\
|\childdocforward{|\textit{main}|}|
\end{tabular}
\end{center}
%
Likewise, the following files |final|\textit{nn}|.tex|
compile the final version of the child document
|child|\textit{nn}|.tex|:
%
\begin{center}
\begin{tabular}{l}
|\def\version{final}|\\
|\input{childdoc.def}|\\
|\childdocforwardprefix{final}{child}|
\end{tabular}
\end{center}
%

Note that when several versions of a main file and/or of each child file
are to be generated, it may be convenient to set up a |Makefile| or
shell script to automatise the process.

%%%%%%%%%%%%%%%%%%%%%%%%%%%%%%%%%%%%%%%%%%%%%%%%%%%%%%%%%%%%%%%%%%%%%%%%%%%%%%%%
\subsection{Command Line Processing}
\label{sec:commandline}

The effect of redirection files can also be achieved by invoking
the \LaTeX{} compiler with a more elaborate command line.
Most conveniently this should be done as part
of a shell script or a |Makefile|.

When using \textsf{childdoc} in the main file, the following
command lines effectively perform a redirection
(note that depending on the shell being used,
backslashes may have to be doubled: `|\|' $\to$ `|\\|'):
%
\begin{center}
|... -jobname "|\textit{target}|" |\\|"|[\textit{flags}]%
|\input{childdoc.def}\childdocforward[|\textit{main}|]{|\textit{dest}|}"|
\end{center}
%
Here \textit{target} is the name of the output file,
\textit{main} is the name of the main file
and \textit{dest} is the name of the main or child file to be processed
(all filenames without extensions).
The optional argument \textit{main} can be omitted
if \textit{main} matches \textit{dest}.
Optionally, compilation \textit{flags} can be defined via |\def| commands.
This command line makes the \TeX{} engine believe
it is compiling the file \textit{target}
whose content is specified as the latter parameter.
The provided code then forwards the processing to
\textit{main} or \textit{dest} as described in \secref{sec:forward}.

%%%%%%%%%%%%%%%%%%%%%%%%%%%%%%%%%%%%%%%%%%%%%%%%%%%%%%%%%%%%%%%%%%%%%%%%%%%%%%%%
\subsection{Include by Input}
\label{sec:input}

Including child documents by |\include| has some restrictions by design.
Most notably, the content of a child document always occupies
its own set of pages; pages cannot be shared between child documents.
Usually, this behaviour makes perfect sense
because each child document contain an essential part of the document.
However, in some situations it may be desirable to compose
a document from a collection of parts
without having mandatory page breaks between then.
For this case, the package
provides a mechanism to include parts
by |\input| which can also be processed individually.
However, by construction this mechanism
requires manual handling of the content to be output.

%%%%%%%%%%%%%%%%%%%%%%%%%%%%%%%%%%%%%%%%
\DescribeMacro{\ifchilddocmanual}
The main file should be prepared as usual, see \secref{sec:include}.
However, the document body must make a distinction
between processing of an individual part and of the main document, e.g.:
%
\begin{center}
\begin{tabular}{l}
|\ifchilddocmanual|\\
|\input{\childdocname}|\\
|\||else|\\
\textit{document body with }|\input{|\textit{part}|}|\\
|\||fi|
\end{tabular}
\end{center}
%
The conditional |\ifchilddocmanual| is true whenever
a part to be included by |\input| is being compiled,
and the name of the part is stored in |\childdocname|.

%%%%%%%%%%%%%%%%%%%%%%%%%%%%%%%%%%%%%%%%
\DescribeMacro{\childdocby}
Each part to be included by |\input| should start with:
%
\begin{center}
\begin{tabular}{l}
|\input{childdoc.def}|\\
|\childdocby{|\textit{main}|}|\\
\end{tabular}
\end{center}
%
The directive |\childdocby| is similar to |\childdocof|
described in \secref{sec:include},
but the subsequent selection of content must be done manually.
To that end, both |\ifchilddoc| and |\ifchilddocmanual|
will be true upon processing of a part,
and the name of the part is stored in |\childdocname|.
Note that |\jobname| will be set to the filename of the current part
so that each part receives an individual |.aux| file
that does not interfere with the |.aux| file(s) of the main document.
This behaviour can be altered by the alternative form
|\childdocby[*]{|\textit{main}|}| (with a non-empty optional argument)
which uses the |.aux| file of the main document
by setting |\jobname| to \textit{main}.

%%%%%%%%%%%%%%%%%%%%%%%%%%%%%%%%%%%%%%%%%%%%%%%%%%%%%%%%%%%%%%%%%%%%%%%%%%%%%%%%
\subsection{Driver Development}
\label{sec:driver}

The \textsf{childdoc} mechanism can also be use for the development
of definition files such as \LaTeX{} styles or classes.
This case differs from the above setup with multiple parts
included by |\include| in that no |\includeonly| should be invoked.
This can be achieved by starting the include file
(before |\ProvidesPackage|) with:
%
\begin{center}
\begin{tabular}{l}
|\input{childdoc.def}|\\
|\childdocforward{|\textit{main}|}|\\
\end{tabular}
\end{center}
%
or alternatively with:
%
\begin{center}
\begin{tabular}{l}
|\input{childdoc.def}|\\
|\childdocby{|\textit{main}|}|\\
\end{tabular}
\end{center}
%
Both forms have slightly different effects as described above.
The main file is prepared as usual, see \secref{sec:include}.

%%%%%%%%%%%%%%%%%%%%%%%%%%%%%%%%%%%%%%%%%%%%%%%%%%%%%%%%%%%%%%%%%%%%%%%%%%%%%%%%
\subsection{Legacy Detection}
\label{sec:detection}

The directive |\childdocmain| in the main file can detect
whether the complete document or merely a child is to be compiled
even without using the directive |\childdocof|.
This method is deprecated because it is less robust
and there is no compelling reason to use it;
it is merely provided for backward compatibility
and it may be removed in future versions.

If the detection mechanism is to be used,
it is mandatory to correctly specify
the filename of the main file as the argument of |\childdocmain|:
%
\begin{center}
\begin{tabular}{l}
|\input{childdoc.def}|\\
|\childdocmain{|\textit{main}|}|\\
\end{tabular}
\end{center}
%
If |\jobname| does not match the argument \textit{main} of |\childdocmain|,
it is assumed that |\jobname| points to the child file to be compiled.
When using |\childdocmain| with the main file specified as argument,
it suffices to start a child file
with just |\input{|\textit{main}|}|
without loading of the package and using |\childdocof|.
If instead all processing is done
with the appropriate \textsf{childdoc} directives,
the argument of \textit{main} of |\childdocmain| can be empty.

An alternative version of the command line processing described
in \secref{sec:commandline} using the detection mechanism reads:
%
\begin{center}
|... -jobname "|\textit{target}|" "|[\textit{flags}]%
[|\def\jobname{|\textit{dest}|}|]|\input{|\textit{main}|}"|
\end{center}

%%%%%%%%%%%%%%%%%%%%%%%%%%%%%%%%%%%%%%%%%%%%%%%%%%%%%%%%%%%%%%%%%%%%%%%%%%%%%%%%
\subsection{Manual Code}
\label{sec:manual}

In case one cannot be certain whether the definitions file |childdoc.def|
is installed on the target \TeX{} distribution
and one prefers not to ship it,
it is conceivable to paste a few relevant commands into the sources.

To that end, drop all statements |\input{childdoc.def}|
and perform the replacements as outlined below.
Instead of |\childdocmain{|\textit{main}|}| add the following code
to the top of the main file:
%
\begin{center}
\begin{tabular}{l}
|\||ifdefined\childdocname\endinput\||fi\newif\ifchilddoc|\\
|\edef\childdocname{\scantokens\expandafter{\jobname\noexpand}}|\\
|\def\childdocmain{|\textit{main}|}\||ifx\childdocmain\childdocname\||else|\\
|\childdoctrue\includeonly{\childdocname}\let\jobname\childdocmain\||fi|\\
\end{tabular}
\end{center}
%
Instead of |\childdocof{|\textit{main}|}| just include the main file
at the top of each child file:
%
\begin{center}
|\input{|\textit{main}|}|
\end{center}
%
A simple redirection |\childdocforward{|\textit{dest}|}| is achieved by:
%
\begin{center}
|\def\jobname{|\textit{dest}|}\input{\jobname}|
\end{center}
%
The redirection with prefix
|\childdocforwardprefix[|\textit{prefix}|]{|\textit{dest}|}|
is accomplished by:
%
\begin{center}
\begin{tabular}{l}
|{\edef\jobname{\scantokens\expandafter{\jobname\noexpand}}|\\
|\def\redirectjob |\textit{prefix}|#1~~~{\gdef\jobname{|\textit{dest}|#1}}|\\
|\expandafter\redirectjob\jobname~~~}\input{\jobname}|
\end{tabular}
\end{center}

In an alternative approach,
child documents can be compiled by a specific command line
without additional code or specific definitions:
%
\begin{center}
|... -jobname "|\textit{target}|" "|[\textit{flags}]%
|\includeonly{|\textit{dest}|}\input{|\textit{main}|}"|
\end{center}
%

%%%%%%%%%%%%%%%%%%%%%%%%%%%%%%%%%%%%%%%%%%%%%%%%%%%%%%%%%%%%%%%%%%%%%%%%%%%%%%%%
%%%%%%%%%%%%%%%%%%%%%%%%%%%%%%%%%%%%%%%%%%%%%%%%%%%%%%%%%%%%%%%%%%%%%%%%%%%%%%%%
\section{Information}

%%%%%%%%%%%%%%%%%%%%%%%%%%%%%%%%%%%%%%%%%%%%%%%%%%%%%%%%%%%%%%%%%%%%%%%%%%%%%%%%
\subsection{Copyright}

Copyright \copyright{} 2017--2018 Niklas Beisert

This work may be distributed and/or modified under the
conditions of the \LaTeX{} Project Public License, either version 1.3
of this license or (at your option) any later version.
The latest version of this license is in
  \url{http://www.latex-project.org/lppl.txt}
and version 1.3 or later is part of all distributions of \LaTeX{}
version 2005/12/01 or later.

This work has the LPPL maintenance status `maintained'.

The Current Maintainer of this work is Niklas Beisert.

This work consists of the files |README.txt|, |childdoc.ins| and |childdoc.dtx|
as well as the derived files |childdoc.def|, |cdocsamp.tex|
with |cdocsch1.tex|, |cdocsch2.tex|, |cdocspt3.tex|, |cdocspt4.tex|,
|cdocsdrf.tex|, |cdocsfn1.tex|, |cdocsfn2.tex|
as well as |childdoc.pdf|.

%%%%%%%%%%%%%%%%%%%%%%%%%%%%%%%%%%%%%%%%%%%%%%%%%%%%%%%%%%%%%%%%%%%%%%%%%%%%%%%%
\subsection{Files and Installation}

The package consists of the files:
%
\begin{center}
\begin{tabular}{ll}
    |README.txt|   & readme file \\
    |childdoc.ins| & installation file \\
    |childdoc.dtx| & source file \\
    |childdoc.def| & definition file \\
    |cdocsamp.tex| & sample main file \\
    |cdocsch1.tex| & sample include file \\
    |cdocsch2.tex| & sample include file \\
    |cdocspt3.tex| & sample part file \\
    |cdocspt4.tex| & sample part file \\
    |cdocsdrf.tex| & sample redirection file \\
    |cdocsfn1.tex| & sample redirection file \\
    |cdocsfn2.tex| & sample redirection file \\
    |childdoc.pdf| & manual
\end{tabular}
\end{center}
%
The distribution consists of the files
|README.txt|, |childdoc.ins| and |childdoc.dtx|.
%
\begin{itemize}
\item
Run (pdf)\LaTeX{} on |childdoc.dtx|
to compile the manual |childdoc.pdf| (this file).
\item
Run \LaTeX{} on |childdoc.ins| to create the definitions file |childdoc.def|
and the sample |cdocsamp.tex| with include files
|cdocsch1.tex|, |cdocsch2.tex|, |cdocspt3.tex|, |cdocspt4.tex|,
|cdocsdrf.tex|, |cdocsfn1.tex|, |cdocsfn2.tex|.
Then copy the file |childdoc.def| to an appropriate directory of your \LaTeX{}
distribution, e.g.\ \textit{texmf-root}|/tex/latex/childdoc|.
\end{itemize}

%%%%%%%%%%%%%%%%%%%%%%%%%%%%%%%%%%%%%%%%%%%%%%%%%%%%%%%%%%%%%%%%%%%%%%%%%%%%%%%%
\subsection{Related CTAN Packages}

There are several other packages which offer a similar functionality:
%
\begin{itemize}
\item
The packages
\href{http://ctan.org/pkg/docmute}{\textsf{docmute}},
\href{http://ctan.org/pkg/includex}{\textsf{includex}} and
\href{http://ctan.org/pkg/standalone}{\textsf{standalone}}
provide commands to include only the document body of
a child file thus allowing both files to be compiled individually.
\item
The packages \href{http://ctan.org/pkg/subdocs}{\textsf{subdocs}}
and \href{http://ctan.org/pkg/subfiles}{\textsf{subfiles}}
provide structures in which the main and child documents can be
encapsulated and allowing them to be compiled individually.
The inclusion mechanism is different from the conventional |\include|.
\item
The package \href{http://ctan.org/pkg/combine}{\textsf{combine}}
is an elaborate solution to combine several documents into one.
\end{itemize}
%
See also the CTAN topic \href{http://ctan.org/topic/subdocs}{\textsf{subdocs}}
for further related packages.
The present package differs from the above solutions in that
a document structure constructed with the conventional |\include| mechanism
just needs two extra commands at the top of every file
such that all constituent files can be compiled individually.

%%%%%%%%%%%%%%%%%%%%%%%%%%%%%%%%%%%%%%%%%%%%%%%%%%%%%%%%%%%%%%%%%%%%%%%%%%%%%%%%
%\subsection{Feature Suggestions}
%
%The following is a list of features which may be useful for future
%versions of this package:
%%
%\begin{itemize}
%\item
%\ldots
%\end{itemize}

%%%%%%%%%%%%%%%%%%%%%%%%%%%%%%%%%%%%%%%%%%%%%%%%%%%%%%%%%%%%%%%%%%%%%%%%%%%%%%%%
\subsection{Revision History}

%%%%%%%%%%%%%%%%%%%%%%%%%%%%%%%%%%%%%%%%
\paragraph{v2.0:} 2018/12/30

\begin{itemize}
\item
immediate forward processing
\item
added |\childdocby| mechanism
\item
manual restructured
\end{itemize}

%%%%%%%%%%%%%%%%%%%%%%%%%%%%%%%%%%%%%%%%
\paragraph{v1.6:} 2018/01/17

\begin{itemize}
\item
application for development of include files
\item
corrections to manual
\end{itemize}

%%%%%%%%%%%%%%%%%%%%%%%%%%%%%%%%%%%%%%%%
\paragraph{v1.5:} 2017/05/21

\begin{itemize}
\item
more complete structuring introduced
\item
|\childdocof| introduced
\item
|\childdoc| renamed to |\childdocmain|
\item
|\childredirect| renamed to |\childdocforward| and |\childdocforwardprefix|
and functionality expanded
\end{itemize}

%%%%%%%%%%%%%%%%%%%%%%%%%%%%%%%%%%%%%%%%
\paragraph{v1.0:} 2017/04/27

\begin{itemize}
\item
manual and install package
\item
first version published on CTAN
\end{itemize}

%%%%%%%%%%%%%%%%%%%%%%%%%%%%%%%%%%%%%%%%
\paragraph{v0.6:} 2017/04/26

\begin{itemize}
\item
redirection mechanism added
\end{itemize}

%%%%%%%%%%%%%%%%%%%%%%%%%%%%%%%%%%%%%%%%
\paragraph{v0.5:} 2017/04/26

\begin{itemize}
\item
functionality in definition file
\end{itemize}


%%%%%%%%%%%%%%%%%%%%%%%%%%%%%%%%%%%%%%%%%%%%%%%%%%%%%%%%%%%%%%%%%%%%%%%%%%%%%%%%
%%%%%%%%%%%%%%%%%%%%%%%%%%%%%%%%%%%%%%%%%%%%%%%%%%%%%%%%%%%%%%%%%%%%%%%%%%%%%%%%
%%%%%%%%%%%%%%%%%%%%%%%%%%%%%%%%%%%%%%%%%%%%%%%%%%%%%%%%%%%%%%%%%%%%%%%%%%%%%%%%
\appendix

\settowidth\MacroIndent{\rmfamily\scriptsize 000\ }

 \DocInput{childdoc.dtx}

\end{document}
%</driver>
% \fi
%
% %%%%%%%%%%%%%%%%%%%%%%%%%%%%%%%%%%%%%%%%%%%%%%%%%%%%%%%%%%%%%%%%%%%%%%%%%%%%%%
% %%%%%%%%%%%%%%%%%%%%%%%%%%%%%%%%%%%%%%%%%%%%%%%%%%%%%%%%%%%%%%%%%%%%%%%%%%%%%%
% \section{Sample}
%\iffalse
%<*samplemain>
%\fi
%
% The following presents a sample document
% with two chapters, two parts, a title page,
% a compile flag as well as three forwarding files to set the flag.
% It consists of eight |.tex| files:
% \begin{center}
% \begin{tabular}{ll}
% |cdocsamp.tex|&main file\\
% |cdocsch1.tex|&include file for chapter 1\\
% |cdocsch2.tex|&include file for chapter 2\\
% |cdocspt3.tex|&include file for part 3\\
% |cdocspt4.tex|&include file for part 4\\
% |cdocsdrf.tex|&forwarding file for main file in draft mode\\
% |cdocsfi1.tex|&forwarding file for final version of chapter 1\\
% |cdocsfi2.tex|&forwarding file for final version of chapter 2\\
% \end{tabular}
% \end{center}
% Each of the eight files can be compiled directly by the \LaTeX{} compiler.
%
% %%%%%%%%%%%%%%%%%%%%%%%%%%%%%%%%%%%%%%
% \paragraph{Main File.}
%
% The main file is called |cdocsamp.tex|.
%
% Load the \textsf{childdoc} definitions and
% declare the filename for the main document:
%    \begin{macrocode}
\input{childdoc.def}
\childdocmain{}
%    \end{macrocode}

% Optional override for |\version| flag:
%    \begin{macrocode}
%%\ifchilddoc\else\providecommand{\version}{draft}\fi
%    \end{macrocode}

% Define the default values for the |\version| flag
% (|final| for the main file and |draft| for childs):
%    \begin{macrocode}
\ifchilddoc
\providecommand{\version}{draft}
\else
\providecommand{\version}{final}
\fi
%    \end{macrocode}

% Load the standard document class:
%    \begin{macrocode}
\documentclass[12pt]{article}
%    \end{macrocode}

% Start the document body:
%    \begin{macrocode}
\begin{document}
%    \end{macrocode}

% Declare a title page.
% Print title, part of document being processed and version flag:
%    \begin{macrocode}
\addtocounter{page}{-1}
\begin{center}
{\LARGE\bfseries{}childdoc example\par}
\vspace{1cm}
\ifchilddoc
\ifchilddocmanual part\else chapter\fi:
`\childdocname' of `\childdocjob'\par
\else
main document: `\childdocjob'\par
\fi
version: \version\par
\end{center}
\newpage
%    \end{macrocode}

% Manually include selected file,
% otherwise process as usual:
%    \begin{macrocode}
\ifchilddocmanual
\section*{part `\childdocname'}
\input{\childdocname}
\else
%    \end{macrocode}

% Include the two chapters:
%    \begin{macrocode}
\include{cdocsch1}
\include{cdocsch2}
%    \end{macrocode}

% Include the two parts unless only chapters should be displayed:
%    \begin{macrocode}
\ifchilddoc\else
\section{part three}
\input{cdocspt3}
\section{part four}
\input{cdocspt4}
\fi
%    \end{macrocode}

% Process as usual until here:
%    \begin{macrocode}
\fi
%    \end{macrocode}

% End of document body:
%    \begin{macrocode}
\end{document}
%    \end{macrocode}
%\iffalse
%</samplemain>
%\fi
%
% %%%%%%%%%%%%%%%%%%%%%%%%%%%%%%%%%%%%%%
% \paragraph{Chapter Include Files.}
%
% The include files are called |cdocsch1.tex| and |cdocsch2.tex|.
%
%\iffalse
%<*samplechap1|samplechap2>
%\fi

% Optional override for |\version| flag:
%    \begin{macrocode}
%%\providecommand{\version}{final}
%    \end{macrocode}

% Include the main document:
%    \begin{macrocode}
\input{childdoc.def}
\childdocof{cdocsamp}
%    \end{macrocode}

%\iffalse
%</samplechap1|samplechap2>
%\fi
%
%\iffalse
%<*samplechap1>
%\fi
% Some text for chapter 1:
%    \begin{macrocode}
\section{one}
some text in chapter one
%    \end{macrocode}

%\iffalse
%</samplechap1>
%\fi
% Some text for chapter 2:
%\iffalse
%<*samplechap2>
%\fi
%    \begin{macrocode}
\section{two}
more text in chapter two
%    \end{macrocode}

%\iffalse
%</samplechap2>
%\fi
%
% %%%%%%%%%%%%%%%%%%%%%%%%%%%%%%%%%%%%%%
% \paragraph{Part Include Files.}
%
% The include files are called |cdocspt3.tex| and |cdocspt4.tex|.
%
%\iffalse
%<*samplepart3|samplepart4>
%\fi

% Optional override for |\version| flag:
%    \begin{macrocode}
%%\providecommand{\version}{final}
%    \end{macrocode}

% Include the main document:
%    \begin{macrocode}
\input{childdoc.def}
\childdocby{cdocsamp}
%    \end{macrocode}

%\iffalse
%</samplepart3|samplepart4>
%\fi
%
%\iffalse
%<*samplepart3>
%\fi
% Some text for part 3:
%    \begin{macrocode}
some text in part three
%    \end{macrocode}

%\iffalse
%</samplepart3>
%\fi
% Some text for part 4:
%\iffalse
%<*samplepart4>
%\fi
%    \begin{macrocode}
more text in part four
%    \end{macrocode}

%\iffalse
%</samplepart4>
%\fi
%
% %%%%%%%%%%%%%%%%%%%%%%%%%%%%%%%%%%%%%%
% \paragraph{Forwarding for a Complete Draft.}
%
% The following forwarding file |cdocsdrf.tex|
% compiles the main document in draft mode:
%\iffalse
%<*sampledraft>
%\fi
%    \begin{macrocode}
\def\version{draft}
\input{childdoc.def}
\childdocforward{cdocsamp}
%    \end{macrocode}

%\iffalse
%</sampledraft>
%\fi
%
% %%%%%%%%%%%%%%%%%%%%%%%%%%%%%%%%%%%%%%
% \paragraph{Forwarding for Final Version of the Chapters.}
%
% The following forwarding files |cdocsfn1.tex| and |cdocsfn2.tex|
% (with identical content)
% compile the final versions of the child documents
% |cdocsch1.tex| and |cdocsch2.tex|, respectively:
%\iffalse
%<*samplefinal>
%\fi
%    \begin{macrocode}
\def\version{final}
\input{childdoc.def}
\childdocforwardprefix[cdocsamp]{cdocsfn}{cdocsch}
%    \end{macrocode}

%\iffalse
%</samplefinal>
%\fi
%
% %%%%%%%%%%%%%%%%%%%%%%%%%%%%%%%%%%%%%%
% \paragraph{Command Line Processing.}
%
% The following three command lines generate the output files
% |cdocscld|, |cdocscl1| and |cdocscl2|
% which should be identical to
% |cdocsdrf|, |cdocsch1| and |cdocsfn2|, respectively:
% \begin{center}
% \begin{tabular}{l}
% |latex -jobname cdocscld \|\\
% |  "\def\version{draft}\input{childdoc.def}\childdocforward{cdocsamp}"|\\
% |latex -jobname cdocscl1 \|\\
% |  "\input{childdoc.def}\childdocforward[cdocsamp]{cdocsch1}"|\\
% |latex -jobname cdocscl2 \|\\
% |  "\def\version{final}\input{childdoc.def}\childdocforward{cdocsch2}"|
% \end{tabular}
% \end{center}
% Note that the trailing backslash on each first line
% merely continues the input to the second line
% (for convenient cut ant paste).
% Furthermore, the command |latex| can be replaced by any
% of its alternative versions such as |pdflatex|.
%
% %%%%%%%%%%%%%%%%%%%%%%%%%%%%%%%%%%%%%%%%%%%%%%%%%%%%%%%%%%%%%%%%%%%%%%%%%%%%%%
% %%%%%%%%%%%%%%%%%%%%%%%%%%%%%%%%%%%%%%%%%%%%%%%%%%%%%%%%%%%%%%%%%%%%%%%%%%%%%%
% \section{Implementation}
%\iffalse
%<*package>
%\fi
%
% This section describes the definitions file |childdoc.def|.

% The definitions cannot be loaded using |\usepackage| or |\RequirePackage|
% which has a mechanism to prevent loading a style file more than once.
% When loading the definitions by means of |\input|
% multiple instances have to be prevented manually:
%\iffalse
%This code needs to be before the `\ProvidesFile' directive
%which is defined at the beginning of this file.
%Therefore it is also placed there and commented out here.
%</package>
%<*discard>
%\fi
%    \begin{macrocode}
\ifdefined\childdocmain\endinput\fi
%    \end{macrocode}
%\iffalse
%</discard>
%<*package>
%\fi
%
% \macro{\ifchilddoc}
% \macro{\ifchilddocmanual}
% The conditional |\ifchilddoc| tells whether a
% child (true) or main (false) document is being compiled.
% The conditional |\ifchilddocmanual| tells whether
% the |\includeonly| mechanism is used (false) or
% the selection of child files must be performed manually (true).
% The definitions initialise to false:
%    \begin{macrocode}
\newif\ifchilddoc
\newif\ifchilddocmanual
%    \end{macrocode}

% \macro{\childdocname}
% \macro{\childdocjob}
% The macro |\childdocname| stores the name of the main document
% to be compiled. The macro |\childdocjob| stores the name of
% the document on which the \LaTeX{} compiler was originally invoked.
% The content of |\jobname| cannot be compared
% to filenames specified in the source due to different catcodes.
% The following code rescans |\jobname|, stores the result
% in |\childdocname| and saves a copy in |\childdocjob|:
%    \begin{macrocode}
\edef\childdocname{\scantokens\expandafter{\jobname\noexpand}}
\let\childdocjob\childdocname
%    \end{macrocode}

% \macro{\childdocdisable}
% The macro |\childdocdisable| prevents the main file
% from being processed more than once.
% At this stage, the main document command |\childdocmain|
% is assumed to be called once again where it should do nothing.
% Any subsequent call to it should prevent
% a secondary processing of the main document
% It overwrites the forwarding commands
% |\childdocof| and |\childdocforward|
% with empty macros to prevent further inclusions of the main document:
%    \begin{macrocode}
\newcommand{\childdocdisable}
{
  \renewcommand{\childdocmain}[1]{\renewcommand{\childdocmain}[1]{\endinput}}
  \renewcommand{\childdocof}[1]{}
  \renewcommand{\childdocby}[2][]{}
  \renewcommand{\childdocforward}[2][]{}
  \renewcommand{\childdocdisable}{}
}
%    \end{macrocode}

% \macro{\childdocmain}
% The macro |\childdocmain| is to be called at the top of the main file
% with nothing or the main filename (without extension) as argument.
% First, it breaks loops.
% If the argument is not empty and does not match |\childdocname|
% (which is set by the first inclusion of |childdoc.def|),
% |\ifchilddoc| is set to true, |\includeonly| is applied to the child file
% and |\jobname| is set to the main file
% (for proper handling of |.aux| files):
%    \begin{macrocode}
\newcommand{\childdocmain}[1]
{
  \childdocdisable\childdocmain{}
  \if?#1?\else
    \begingroup
      \def\childdoctmp{#1}
      \ifx\childdoctmp\childdocname
        \def\childdoctmp{}
      \else
        \def\childdoctmp
        {
          \childdoctrue
          \includeonly{\childdocname}
          \def\childdocjob{#1}
          \def\jobname{#1}
        }
      \fi
      \expandafter
    \endgroup
    \childdoctmp
  \fi
}
%    \end{macrocode}

% \macro{\childdocof}
% The command |\childdocof| redirects
% compilation to the main file |#1|.
%    \begin{macrocode}
\newcommand{\childdocof}[1]
{
  \childdocdisable
  \childdoctrue
  \includeonly{\childdocname}
  \def\jobname{#1}
  \def\childdocjob{#1}
  \input{#1}
}
%    \end{macrocode}

% \macro{\childdocby}
% The command |\childdocby| ....
%    \begin{macrocode}
\newcommand{\childdocby}[2][]
{
  \childdocdisable
  \childdoctrue
  \childdocmanualtrue
  \if?#1?\else
    \def\jobname{#2}
  \fi
  \def\childdocjob{#2}
  \input{#2}
  \endinput
}
%    \end{macrocode}

% \macro{\childdocforward}
% The command |\childdocforward| redirects
% compilation to the main file or
% (if the optional argument is given) a child file.
% Parameters are set as if the main file
% or a child file starting with |\childdocof| was compiled.
% Then compilation is handed over to the main file:
%    \begin{macrocode}
\newcommand{\childdocforward}[2][]
{
  \begingroup
    \if?#1?
      \def\childdoctmp
      {
        \def\childdocname{#2}
        \def\childdocjob{#2}
        \def\jobname{#2}
        \input{#2}
        \endinput
      }
    \else
      \def\childdoctmp
      {
        \childdocdisable
        \def\childdocname{#2}
        \childdoctrue
        \includeonly{#2}
        \def\childdocjob{#1}
        \def\jobname{#1}
        \input{#1}
        \endinput
      }
    \fi
    \expandafter
  \endgroup
  \childdoctmp
}
%    \end{macrocode}

% \macro{\childdocforwardprefix}
% The command |\childdocforwardprefix| redirects
% compilation to the main or a child file by means of a pattern.
% The prefix |#1| in the current filename is replaced by |#2|
% and the suffix of the current filename is kept
% (it is assumed that the filename does not contain the substring `|~~~|'
% which is used as a delimiter).
% Compilation is handed over to the new file by |\childdocforward|:
%    \begin{macrocode}
\newcommand{\childdocforwardprefix}[3][]
{
  \begingroup
    \def\childdocextract #2##1~~~{\def\childdoctmp{\childdocforward[#1]{#3##1}}}
    \expandafter\childdocextract\childdocname~~~
    \expandafter
  \endgroup
  \childdoctmp
}
%    \end{macrocode}

% \macro{\childdoc}
% The deprecated macro |\childdoc| is a legacy version of |\childdocmain|:
%    \begin{macrocode}
\newcommand{\childdoc}{\childdocmain}
%    \end{macrocode}

% \macro{\childdocredirect}
% The deprecated macro |\childdocredirect| is a legacy version
% of |\childdocforward| and |\childdocforwardprefix|:
%    \begin{macrocode}
\newcommand{\childdocredirect}[2][]
{
  \begingroup
    \if?#1?
      \def\childdoctmp{\childdocforward{#2}}
    \else
      \def\childdoctmp{\childdocforwardprefix{#1}{#2}}
    \fi
    \expandafter
  \endgroup
  \childdoctmp
}
%    \end{macrocode}

%\iffalse
%</package>
%\fi
%
\endinput

\childdocmain{}
%    \end{macrocode}

% Optional override for |\version| flag:
%    \begin{macrocode}
%%\ifchilddoc\else\providecommand{\version}{draft}\fi
%    \end{macrocode}

% Define the default values for the |\version| flag
% (|final| for the main file and |draft| for childs):
%    \begin{macrocode}
\ifchilddoc
\providecommand{\version}{draft}
\else
\providecommand{\version}{final}
\fi
%    \end{macrocode}

% Load the standard document class:
%    \begin{macrocode}
\documentclass[12pt]{article}
%    \end{macrocode}

% Start the document body:
%    \begin{macrocode}
\begin{document}
%    \end{macrocode}

% Declare a title page.
% Print title, part of document being processed and version flag:
%    \begin{macrocode}
\addtocounter{page}{-1}
\begin{center}
{\LARGE\bfseries{}childdoc example\par}
\vspace{1cm}
\ifchilddoc
\ifchilddocmanual part\else chapter\fi:
`\childdocname' of `\childdocjob'\par
\else
main document: `\childdocjob'\par
\fi
version: \version\par
\end{center}
\newpage
%    \end{macrocode}

% Manually include selected file,
% otherwise process as usual:
%    \begin{macrocode}
\ifchilddocmanual
\section*{part `\childdocname'}
\input{\childdocname}
\else
%    \end{macrocode}

% Include the two chapters:
%    \begin{macrocode}
\include{cdocsch1}
\include{cdocsch2}
%    \end{macrocode}

% Include the two parts unless only chapters should be displayed:
%    \begin{macrocode}
\ifchilddoc\else
\section{part three}
\input{cdocspt3}
\section{part four}
\input{cdocspt4}
\fi
%    \end{macrocode}

% Process as usual until here:
%    \begin{macrocode}
\fi
%    \end{macrocode}

% End of document body:
%    \begin{macrocode}
\end{document}
%    \end{macrocode}
%\iffalse
%</samplemain>
%\fi
%
% %%%%%%%%%%%%%%%%%%%%%%%%%%%%%%%%%%%%%%
% \paragraph{Chapter Include Files.}
%
% The include files are called |cdocsch1.tex| and |cdocsch2.tex|.
%
%\iffalse
%<*samplechap1|samplechap2>
%\fi

% Optional override for |\version| flag:
%    \begin{macrocode}
%%\providecommand{\version}{final}
%    \end{macrocode}

% Include the main document:
%    \begin{macrocode}
% \iffalse
%
% childdoc.dtx Copyright (C) 2017-2018 Niklas Beisert
%
% This work may be distributed and/or modified under the
% conditions of the LaTeX Project Public License, either version 1.3
% of this license or (at your option) any later version.
% The latest version of this license is in
%   http://www.latex-project.org/lppl.txt
% and version 1.3 or later is part of all distributions of LaTeX
% version 2005/12/01 or later.
%
% This work has the LPPL maintenance status `maintained'.
%
% The Current Maintainer of this work is Niklas Beisert.
%
% This work consists of the files childdoc.dtx and childdoc.ins
% and the derived files childdoc.def and cdocsamp.tex with
% cdocsch1.tex, cdocsch2.tex, cdocsdrf.tex, cdocsfn1.tex, cdocsfn2.tex.
%
%<package>\ifdefined\childdocmain\endinput\fi
%<package>\ProvidesFile{childdoc.def}[2018/12/30 v2.0 child document driver]
%<samplemain>\ProvidesFile{cdocsamp.tex}[2018/12/30 v2.0 sample for childdoc]
%<*driver>
%\ProvidesFile{childdoc.drv}[2018/12/30 v2.0 childdoc reference manual file]
\PassOptionsToClass{10pt,a4paper}{article}
\documentclass{ltxdoc}

\usepackage[margin=35mm]{geometry}
\usepackage{hyperref}
\usepackage{hyperxmp}
\usepackage[usenames]{color}

\hypersetup{colorlinks=true}
\hypersetup{pdfstartview=FitH}
\hypersetup{pdfpagemode=UseNone}
\hypersetup{pdfsource={}}
\hypersetup{pdflang={en-UK}}
\hypersetup{pdfcopyright={Copyright 2017-2018 Niklas Beisert.
  This work may be distributed and/or modified under the
  conditions of the LaTeX Project Public License, either version 1.3
  of this license or (at your option) any later version.}}
\hypersetup{pdflicenseurl={http://www.latex-project.org/lppl.txt}}
\hypersetup{pdfcontactaddress={ETH Zurich, ITP, HIT K,
  Wolfgang-Pauli-Strasse 27}}
\hypersetup{pdfcontactpostcode={8093}}
\hypersetup{pdfcontactcity={Zurich}}
\hypersetup{pdfcontactcountry={Switzerland}}
\hypersetup{pdfcontactemail={nbeisert@itp.phys.ethz.ch}}
\hypersetup{pdfcontacturl={http://people.phys.ethz.ch/\xmptilde nbeisert/}}

\newcommand{\secref}[1]{\hyperref[#1]{section \ref*{#1}}}

\parskip1ex
\parindent0pt
\let\olditemize\itemize
\def\itemize{\olditemize\parskip0pt}

\begin{document}

\title{The \textsf{childdoc} Package}
\hypersetup{pdftitle={The childdoc Package}}
\author{Niklas Beisert\\[2ex]
  Institut f\"ur Theoretische Physik\\
  Eidgen\"ossische Technische Hochschule Z\"urich\\
  Wolfgang-Pauli-Strasse 27, 8093 Z\"urich, Switzerland\\[1ex]
  \href{mailto:nbeisert@itp.phys.ethz.ch}
  {\texttt{nbeisert@itp.phys.ethz.ch}}}
\hypersetup{pdfauthor={Niklas Beisert}}
\hypersetup{pdfsubject={Manual for the LaTeX2e Package childdoc}}
\date{30 December 2018, \textsf{v2.0}}
\maketitle

\begin{abstract}\noindent
\textsf{childdoc} is a \LaTeXe{} package
that enables the direct compilation
of document sections included by |\include|
to individual files.
\end{abstract}

\begingroup
\parskip0ex
\tableofcontents
\endgroup

%%%%%%%%%%%%%%%%%%%%%%%%%%%%%%%%%%%%%%%%%%%%%%%%%%%%%%%%%%%%%%%%%%%%%%%%%%%%%%%%
%%%%%%%%%%%%%%%%%%%%%%%%%%%%%%%%%%%%%%%%%%%%%%%%%%%%%%%%%%%%%%%%%%%%%%%%%%%%%%%%
\section{Introduction}

\LaTeX{} provides a mechanism to structure a large document (such as a book)
into a main file and several child files (containing the chapters)
using the |\include| command.
This mechanism is beneficial for documents
which span hundreds of pages in order to
make the source file(s) more manageable.
Moreover, compilation can be restricted to
selected child files by means of the |\includeonly| command.
The latter feature can be used to reduce the compilation time while editing
(this was significantly more useful in the earlier days of \LaTeX{})
or to generate a smaller document which is easier to navigate.
Another application of |\includeonly| is to generate
documents consisting of selected parts of the complete document.

However, there are a few drawbacks of the plain |\include| mechanism:
\begin{itemize}
\item
The child files cannot be compiled on their own,
they can only be compiled via the main file.
A naive editing environment
(such as a text editor with an option
to have the current file processed by \LaTeX)
may require one to switch to the main file before compiling;
attempting to compile the child file produces errors.
\item
The main file must be modified (each time)
to adjust the |\includeonly| command
to the present needs. This easily leaves the main file in a messy state.
\item
The generated document will always carry the filename
of the main document. This is inconvenient if
several child files are to be compiled and
to be kept for distribution.
\end{itemize}

The present package provides a simple interface
to make child files individually compilable by \LaTeX{}.
Compiling a child file then has the same effect as compiling
the main file with an |\includeonly| command
to select the appropriate child.
Moreover the generated document will carry the name of the child
rather than the main file.
This resolves all three above issues.

This feature is meant to make the editing of books,
thesis documents and lecture notes somewhat more convenient.
However, the package can also be used efficiently for
composing a series of documents (such as exercise sheets)
which are typically distributed individually.
It then assists the author in generating the individual documents
(potentially in different versions)
as well as a document containing the collected series.
Another application is in developing style files
or other kinds of included material
where compilation of the style file could redirect
to a sample or test file.

%%%%%%%%%%%%%%%%%%%%%%%%%%%%%%%%%%%%%%%%%%%%%%%%%%%%%%%%%%%%%%%%%%%%%%%%%%%%%%%%
%%%%%%%%%%%%%%%%%%%%%%%%%%%%%%%%%%%%%%%%%%%%%%%%%%%%%%%%%%%%%%%%%%%%%%%%%%%%%%%%
\section{Usage}

First of all, the package \textsf{childdoc} is \emph{not} a standard
\LaTeXe{} |.sty| style file! Therefore it needs to be invoked in
a non-standard way.

%%%%%%%%%%%%%%%%%%%%%%%%%%%%%%%%%%%%%%%%%%%%%%%%%%%%%%%%%%%%%%%%%%%%%%%%%%%%%%%%
\subsection{Included Files}
\label{sec:include}

%%%%%%%%%%%%%%%%%%%%%%%%%%%%%%%%%%%%%%%%
\DescribeMacro{\childdocmain}
To use the package, add the commands
\begin{center}
\begin{tabular}{l}
|\input{childdoc.def}|\\
|\childdocmain{}|\\
\end{tabular}
\end{center}
at the very top of the main \LaTeX{} file,
in particular \emph{before} the |\documentclass| statement!
The argument of |\childdocmain| should be left empty
(but it must be present).

%%%%%%%%%%%%%%%%%%%%%%%%%%%%%%%%%%%%%%%%
\DescribeMacro{\childdocof}
Furthermore, add the commands
\begin{center}
\begin{tabular}{l}
|\input{childdoc.def}|\\
|\childdocof{|\textit{main}|}|\\
\end{tabular}
\end{center}
at the top of every child file \textit{child}
which is included by |\include{|\textit{child}|}|
from within the main file
(or at least for those files to be compiled individually).
The argument \textit{main} must be the filename of the main file.

There are a couple of
considerations in setting up the main and child documents:

%%%%%%%%%%%%%%%%%%%%%%%%%%%%%%%%%%%%%%%%
\paragraph{Restrictions.}

Please note the following restrictions:
\begin{itemize}
\item
|\childdocmain| must be called with one argument \textit{main}
to ensure compatibility with earlier version of the package.
It must either be empty (|\childdocmain{}|)
or precisely match the filename of the main file in which it is specified.
See \secref{sec:detection} for further information.
\item
The filename \textit{main} must be specified without the |.tex| extension.
\item
The filename \textit{main} is case sensitive
(even in case-insensitive file systems)
due to internal string comparison.
\item
The argument \textit{main} should be fully expanded, it cannot be a macro.
\item
Subdirectories and special characters should be avoided in filenames.
\item
The command |\childdocmain{|\textit{main}|}| must be followed by a whitespace.
It should not be followed immediately by another command
or by a comment mark `|%|'.
This is because the \TeX{} parser reads the token immediately following
the argument of |\childdocmain| and puts it
at the beginning of every child section;
however, a white\-space is ignored.
\end{itemize}

%%%%%%%%%%%%%%%%%%%%%%%%%%%%%%%%%%%%%%%%
\paragraph{Content of Main File.}

It is advisable to place all content in the child files included by |\include|.
Any output contained in the main file will appear in all child documents
unless suppressed manually;
it cannot be suppressed automatically by the |\includeonly| directive
and thus should normally be avoided.
A method to include some content in the main file
by means of conditional processing is described in \secref{sec:conditional}.

%%%%%%%%%%%%%%%%%%%%%%%%%%%%%%%%%%%%%%%%
\paragraph{Page Numbering.}

When only a part of the document is compiled,
the appropriate numbering of pages
(as well as other status parameters)
is determined from the |.aux| files.
The latter contain information from previous passes.
However this information needs to propagate through
all intermediate child documents.
Therefore the page numbering in child documents may well
be inconsistent until the complete document is compiled at least once.

A useful (if unconventional) way to always ensure a consistent
page numbering is to restart the numbering in each child document
and denote the pages by `\textit{child}|.|\textit{page}'
where \textit{child} represents the chapter/section number of the child file.
This can be achieved by the command
|\numberwithin{page}{|\textit{child}|}|
of the \textsf{amsmath} package
where \textit{child} can be |chapter| or |section|
depending on the chosen structuring.
Alternatively, one can modify the macro |\thepage| appropriately
and reset the counter |page| at the start of each child file.

%%%%%%%%%%%%%%%%%%%%%%%%%%%%%%%%%%%%%%%%%%%%%%%%%%%%%%%%%%%%%%%%%%%%%%%%%%%%%%%%
\subsection{Conditional Processing}
\label{sec:conditional}

The package provides a mechanism to compile different versions
of a document. To customise the versions further some conditional processing
can come in handy to distinguish which version is being compiled.
The package provides two macros to describe the compilation context:

%%%%%%%%%%%%%%%%%%%%%%%%%%%%%%%%%%%%%%%%
\DescribeMacro{\ifchilddoc}
The conditional |\ifchilddoc| distinguishes between the compilation of
child documents and the main document:
%
\begin{center}
|\ifchilddoc |\textit{child-code}| |[|\||else |\textit{main-code}]| \||fi|
\end{center}

%%%%%%%%%%%%%%%%%%%%%%%%%%%%%%%%%%%%%%%%
\DescribeMacro{\childdocname}
\DescribeMacro{\childdocjob}
The macro |\childdocname| contains the filename (without extension)
of the main or child file being processed.
Note that |\childdocjob| will always contain the name of the main file.

%%%%%%%%%%%%%%%%%%%%%%%%%%%%%%%%%%%%%%%%
\paragraph{Title Page.}

Conditional processing can be used to include a title or banner page
in the main document when proper precautions are taken.
Importantly, the code in the main file should ensure that the page counter
(as well as other status parameters which are stored in the |.aux| files)
takes the same value after the conditional processing.
Otherwise the page numbers may take divergent values
depending on which part is compiled.

For example, a title page could be declared by:
%
\begin{center}
\begin{tabular}{l}
|\ifchilddoc\||else|\\
|\addtocounter{page}{-1}|\\
\textit{code for title page}\\
|\newpage|\\
|\||fi|
\end{tabular}
\end{center}
%
A banner page for the child documents can be generated by:
%
\begin{center}
\begin{tabular}{l}
|\ifchilddoc|\\
|\addtocounter{page}{-1}|\\
\textit{code for banner page}\\
|\newpage|\\
|\||fi|
\end{tabular}
\end{center}
%
Here one could write a message such as:
\begin{center}
|This is the part \childdocname{} of \childdocjob{}.|
\end{center}

%%%%%%%%%%%%%%%%%%%%%%%%%%%%%%%%%%%%%%%%%%%%%%%%%%%%%%%%%%%%%%%%%%%%%%%%%%%%%%%%
\subsection{Flags}
\label{sec:flags}

The package makes it easy to generate different versions
of the main or child documents.
To this end compilation flags can be defined
and assigned different default values.
They will be particularly useful in conjunction
with the forwarding mechanism described in \secref{sec:forward}.

For example, it may be useful to have a flag |\version|
which can be set to |draft| or |final|.
The document source will contain some conditional code
depending on the value of |\version|.
Suppose further, the flag should default to |final| for the main file
and to |draft| for child files
which is a natural assignment for editing the document.
This is achieved by placing the following code
in the preamble of the main document
(below the |\childdocmain| directive):
%
\begin{center}
\begin{tabular}{l}
|\ifchilddoc|\\
|\providecommand{\version}{draft}|\\
|\||else|\\
|\providecommand{\version}{final}|\\
|\||fi|
\end{tabular}
\end{center}
%
The definition by |\providecommand| makes sure
that previous definitions are not overwritten.
Further statements |\providecommand{\version}{...}|
can thus be added before the above code to override it.

For the main file, one might add a line
(between |\childdocmain| and the above block)
%
\begin{center}
|%\ifchilddoc\||else\providecommand{\version}{draft}\||fi|
\end{center}
%
which can be uncommented to produce a draft version.
Likewise one can add a line to the very top of a child file
(above the |\childdocof{|\textit{main}|}| directive)
%
\begin{center}
|%\providecommand{\version}{final}|
\end{center}
%
which can be uncommented to produce the final version of this child document.

%%%%%%%%%%%%%%%%%%%%%%%%%%%%%%%%%%%%%%%%%%%%%%%%%%%%%%%%%%%%%%%%%%%%%%%%%%%%%%%%
\subsection{Forwarding}
\label{sec:forward}

Different versions of the main or child documents
using compilation flags as described in \secref{sec:flags}
can be (permanently) stored in different files
for convenient compilation, viewing and distribution.
To this end, the package defines a command
to pass on compilation to a different file:

%%%%%%%%%%%%%%%%%%%%%%%%%%%%%%%%%%%%%%%%
\DescribeMacro{\childdocforward}
The command |\childdocforward| redirects processing to
another source file:
%
\begin{center}
\begin{tabular}{l}
|\input{childdoc.def}|\\
|\childdocforward[|\textit{main}|]{|\textit{dest}|}|\\
\end{tabular}
\end{center}
%
The argument \textit{dest} is the destination file
(without extension).
It should be the main file or one of the child files.
Note that further \textsf{childdoc} directives
such as |\childdocof| and |\childdocforward|
in the indicated file will be processed in this form.
The optional argument \textit{main}
passes on directly to the main file \textit{main}
while pretending to compile the child \textit{dest}.
This form behaves as if \textit{dest}
issues |\childdocof{|\textit{main}|}| right away,
and no further \textsf{childdoc} directives will be processed.

%%%%%%%%%%%%%%%%%%%%%%%%%%%%%%%%%%%%%%%%
\DescribeMacro{\...prefix}
In the alternative form |\childdocforwardprefix|,
%
\begin{center}
\begin{tabular}{l}
|\input{childdoc.def}|\\
|\childdocforwardprefix[|\textit{main}|]{|\textit{prefix}|}{|\textit{dest}|}|
\end{tabular}
\end{center}
%
the destination file is determined by a pattern
depending on the current file:
To make this work, the current file must be called
`{\textit{prefix}\hspace{0.2em}\textit{suffix}}'
with \textit{prefix} matching precisely the argument.
Processing is then passed on to the file
`{\textit{dest}\hspace{0.2em}\textit{suffix}}'.
Surely, the same effect is achieved by
directly specifying the
argument `{\textit{dest}\hspace{0.2em}\textit{suffix}}'
in the first form.
However, that requires to set up a different file
for each child. With the alternative form of the command
all these files can have exactly the same content
which simplifies setting them up and maintaining them.

For example, the following file |draft.tex|
with a compilation flag |\version| as described in \secref{sec:flags}
compiles the main document as a draft:
%
\begin{center}
\begin{tabular}{l}
|\def\version{draft}|\\
|\input{childdoc.def}|\\
|\childdocforward{|\textit{main}|}|
\end{tabular}
\end{center}
%
Likewise, the following files |final|\textit{nn}|.tex|
compile the final version of the child document
|child|\textit{nn}|.tex|:
%
\begin{center}
\begin{tabular}{l}
|\def\version{final}|\\
|\input{childdoc.def}|\\
|\childdocforwardprefix{final}{child}|
\end{tabular}
\end{center}
%

Note that when several versions of a main file and/or of each child file
are to be generated, it may be convenient to set up a |Makefile| or
shell script to automatise the process.

%%%%%%%%%%%%%%%%%%%%%%%%%%%%%%%%%%%%%%%%%%%%%%%%%%%%%%%%%%%%%%%%%%%%%%%%%%%%%%%%
\subsection{Command Line Processing}
\label{sec:commandline}

The effect of redirection files can also be achieved by invoking
the \LaTeX{} compiler with a more elaborate command line.
Most conveniently this should be done as part
of a shell script or a |Makefile|.

When using \textsf{childdoc} in the main file, the following
command lines effectively perform a redirection
(note that depending on the shell being used,
backslashes may have to be doubled: `|\|' $\to$ `|\\|'):
%
\begin{center}
|... -jobname "|\textit{target}|" |\\|"|[\textit{flags}]%
|\input{childdoc.def}\childdocforward[|\textit{main}|]{|\textit{dest}|}"|
\end{center}
%
Here \textit{target} is the name of the output file,
\textit{main} is the name of the main file
and \textit{dest} is the name of the main or child file to be processed
(all filenames without extensions).
The optional argument \textit{main} can be omitted
if \textit{main} matches \textit{dest}.
Optionally, compilation \textit{flags} can be defined via |\def| commands.
This command line makes the \TeX{} engine believe
it is compiling the file \textit{target}
whose content is specified as the latter parameter.
The provided code then forwards the processing to
\textit{main} or \textit{dest} as described in \secref{sec:forward}.

%%%%%%%%%%%%%%%%%%%%%%%%%%%%%%%%%%%%%%%%%%%%%%%%%%%%%%%%%%%%%%%%%%%%%%%%%%%%%%%%
\subsection{Include by Input}
\label{sec:input}

Including child documents by |\include| has some restrictions by design.
Most notably, the content of a child document always occupies
its own set of pages; pages cannot be shared between child documents.
Usually, this behaviour makes perfect sense
because each child document contain an essential part of the document.
However, in some situations it may be desirable to compose
a document from a collection of parts
without having mandatory page breaks between then.
For this case, the package
provides a mechanism to include parts
by |\input| which can also be processed individually.
However, by construction this mechanism
requires manual handling of the content to be output.

%%%%%%%%%%%%%%%%%%%%%%%%%%%%%%%%%%%%%%%%
\DescribeMacro{\ifchilddocmanual}
The main file should be prepared as usual, see \secref{sec:include}.
However, the document body must make a distinction
between processing of an individual part and of the main document, e.g.:
%
\begin{center}
\begin{tabular}{l}
|\ifchilddocmanual|\\
|\input{\childdocname}|\\
|\||else|\\
\textit{document body with }|\input{|\textit{part}|}|\\
|\||fi|
\end{tabular}
\end{center}
%
The conditional |\ifchilddocmanual| is true whenever
a part to be included by |\input| is being compiled,
and the name of the part is stored in |\childdocname|.

%%%%%%%%%%%%%%%%%%%%%%%%%%%%%%%%%%%%%%%%
\DescribeMacro{\childdocby}
Each part to be included by |\input| should start with:
%
\begin{center}
\begin{tabular}{l}
|\input{childdoc.def}|\\
|\childdocby{|\textit{main}|}|\\
\end{tabular}
\end{center}
%
The directive |\childdocby| is similar to |\childdocof|
described in \secref{sec:include},
but the subsequent selection of content must be done manually.
To that end, both |\ifchilddoc| and |\ifchilddocmanual|
will be true upon processing of a part,
and the name of the part is stored in |\childdocname|.
Note that |\jobname| will be set to the filename of the current part
so that each part receives an individual |.aux| file
that does not interfere with the |.aux| file(s) of the main document.
This behaviour can be altered by the alternative form
|\childdocby[*]{|\textit{main}|}| (with a non-empty optional argument)
which uses the |.aux| file of the main document
by setting |\jobname| to \textit{main}.

%%%%%%%%%%%%%%%%%%%%%%%%%%%%%%%%%%%%%%%%%%%%%%%%%%%%%%%%%%%%%%%%%%%%%%%%%%%%%%%%
\subsection{Driver Development}
\label{sec:driver}

The \textsf{childdoc} mechanism can also be use for the development
of definition files such as \LaTeX{} styles or classes.
This case differs from the above setup with multiple parts
included by |\include| in that no |\includeonly| should be invoked.
This can be achieved by starting the include file
(before |\ProvidesPackage|) with:
%
\begin{center}
\begin{tabular}{l}
|\input{childdoc.def}|\\
|\childdocforward{|\textit{main}|}|\\
\end{tabular}
\end{center}
%
or alternatively with:
%
\begin{center}
\begin{tabular}{l}
|\input{childdoc.def}|\\
|\childdocby{|\textit{main}|}|\\
\end{tabular}
\end{center}
%
Both forms have slightly different effects as described above.
The main file is prepared as usual, see \secref{sec:include}.

%%%%%%%%%%%%%%%%%%%%%%%%%%%%%%%%%%%%%%%%%%%%%%%%%%%%%%%%%%%%%%%%%%%%%%%%%%%%%%%%
\subsection{Legacy Detection}
\label{sec:detection}

The directive |\childdocmain| in the main file can detect
whether the complete document or merely a child is to be compiled
even without using the directive |\childdocof|.
This method is deprecated because it is less robust
and there is no compelling reason to use it;
it is merely provided for backward compatibility
and it may be removed in future versions.

If the detection mechanism is to be used,
it is mandatory to correctly specify
the filename of the main file as the argument of |\childdocmain|:
%
\begin{center}
\begin{tabular}{l}
|\input{childdoc.def}|\\
|\childdocmain{|\textit{main}|}|\\
\end{tabular}
\end{center}
%
If |\jobname| does not match the argument \textit{main} of |\childdocmain|,
it is assumed that |\jobname| points to the child file to be compiled.
When using |\childdocmain| with the main file specified as argument,
it suffices to start a child file
with just |\input{|\textit{main}|}|
without loading of the package and using |\childdocof|.
If instead all processing is done
with the appropriate \textsf{childdoc} directives,
the argument of \textit{main} of |\childdocmain| can be empty.

An alternative version of the command line processing described
in \secref{sec:commandline} using the detection mechanism reads:
%
\begin{center}
|... -jobname "|\textit{target}|" "|[\textit{flags}]%
[|\def\jobname{|\textit{dest}|}|]|\input{|\textit{main}|}"|
\end{center}

%%%%%%%%%%%%%%%%%%%%%%%%%%%%%%%%%%%%%%%%%%%%%%%%%%%%%%%%%%%%%%%%%%%%%%%%%%%%%%%%
\subsection{Manual Code}
\label{sec:manual}

In case one cannot be certain whether the definitions file |childdoc.def|
is installed on the target \TeX{} distribution
and one prefers not to ship it,
it is conceivable to paste a few relevant commands into the sources.

To that end, drop all statements |\input{childdoc.def}|
and perform the replacements as outlined below.
Instead of |\childdocmain{|\textit{main}|}| add the following code
to the top of the main file:
%
\begin{center}
\begin{tabular}{l}
|\||ifdefined\childdocname\endinput\||fi\newif\ifchilddoc|\\
|\edef\childdocname{\scantokens\expandafter{\jobname\noexpand}}|\\
|\def\childdocmain{|\textit{main}|}\||ifx\childdocmain\childdocname\||else|\\
|\childdoctrue\includeonly{\childdocname}\let\jobname\childdocmain\||fi|\\
\end{tabular}
\end{center}
%
Instead of |\childdocof{|\textit{main}|}| just include the main file
at the top of each child file:
%
\begin{center}
|\input{|\textit{main}|}|
\end{center}
%
A simple redirection |\childdocforward{|\textit{dest}|}| is achieved by:
%
\begin{center}
|\def\jobname{|\textit{dest}|}\input{\jobname}|
\end{center}
%
The redirection with prefix
|\childdocforwardprefix[|\textit{prefix}|]{|\textit{dest}|}|
is accomplished by:
%
\begin{center}
\begin{tabular}{l}
|{\edef\jobname{\scantokens\expandafter{\jobname\noexpand}}|\\
|\def\redirectjob |\textit{prefix}|#1~~~{\gdef\jobname{|\textit{dest}|#1}}|\\
|\expandafter\redirectjob\jobname~~~}\input{\jobname}|
\end{tabular}
\end{center}

In an alternative approach,
child documents can be compiled by a specific command line
without additional code or specific definitions:
%
\begin{center}
|... -jobname "|\textit{target}|" "|[\textit{flags}]%
|\includeonly{|\textit{dest}|}\input{|\textit{main}|}"|
\end{center}
%

%%%%%%%%%%%%%%%%%%%%%%%%%%%%%%%%%%%%%%%%%%%%%%%%%%%%%%%%%%%%%%%%%%%%%%%%%%%%%%%%
%%%%%%%%%%%%%%%%%%%%%%%%%%%%%%%%%%%%%%%%%%%%%%%%%%%%%%%%%%%%%%%%%%%%%%%%%%%%%%%%
\section{Information}

%%%%%%%%%%%%%%%%%%%%%%%%%%%%%%%%%%%%%%%%%%%%%%%%%%%%%%%%%%%%%%%%%%%%%%%%%%%%%%%%
\subsection{Copyright}

Copyright \copyright{} 2017--2018 Niklas Beisert

This work may be distributed and/or modified under the
conditions of the \LaTeX{} Project Public License, either version 1.3
of this license or (at your option) any later version.
The latest version of this license is in
  \url{http://www.latex-project.org/lppl.txt}
and version 1.3 or later is part of all distributions of \LaTeX{}
version 2005/12/01 or later.

This work has the LPPL maintenance status `maintained'.

The Current Maintainer of this work is Niklas Beisert.

This work consists of the files |README.txt|, |childdoc.ins| and |childdoc.dtx|
as well as the derived files |childdoc.def|, |cdocsamp.tex|
with |cdocsch1.tex|, |cdocsch2.tex|, |cdocspt3.tex|, |cdocspt4.tex|,
|cdocsdrf.tex|, |cdocsfn1.tex|, |cdocsfn2.tex|
as well as |childdoc.pdf|.

%%%%%%%%%%%%%%%%%%%%%%%%%%%%%%%%%%%%%%%%%%%%%%%%%%%%%%%%%%%%%%%%%%%%%%%%%%%%%%%%
\subsection{Files and Installation}

The package consists of the files:
%
\begin{center}
\begin{tabular}{ll}
    |README.txt|   & readme file \\
    |childdoc.ins| & installation file \\
    |childdoc.dtx| & source file \\
    |childdoc.def| & definition file \\
    |cdocsamp.tex| & sample main file \\
    |cdocsch1.tex| & sample include file \\
    |cdocsch2.tex| & sample include file \\
    |cdocspt3.tex| & sample part file \\
    |cdocspt4.tex| & sample part file \\
    |cdocsdrf.tex| & sample redirection file \\
    |cdocsfn1.tex| & sample redirection file \\
    |cdocsfn2.tex| & sample redirection file \\
    |childdoc.pdf| & manual
\end{tabular}
\end{center}
%
The distribution consists of the files
|README.txt|, |childdoc.ins| and |childdoc.dtx|.
%
\begin{itemize}
\item
Run (pdf)\LaTeX{} on |childdoc.dtx|
to compile the manual |childdoc.pdf| (this file).
\item
Run \LaTeX{} on |childdoc.ins| to create the definitions file |childdoc.def|
and the sample |cdocsamp.tex| with include files
|cdocsch1.tex|, |cdocsch2.tex|, |cdocspt3.tex|, |cdocspt4.tex|,
|cdocsdrf.tex|, |cdocsfn1.tex|, |cdocsfn2.tex|.
Then copy the file |childdoc.def| to an appropriate directory of your \LaTeX{}
distribution, e.g.\ \textit{texmf-root}|/tex/latex/childdoc|.
\end{itemize}

%%%%%%%%%%%%%%%%%%%%%%%%%%%%%%%%%%%%%%%%%%%%%%%%%%%%%%%%%%%%%%%%%%%%%%%%%%%%%%%%
\subsection{Related CTAN Packages}

There are several other packages which offer a similar functionality:
%
\begin{itemize}
\item
The packages
\href{http://ctan.org/pkg/docmute}{\textsf{docmute}},
\href{http://ctan.org/pkg/includex}{\textsf{includex}} and
\href{http://ctan.org/pkg/standalone}{\textsf{standalone}}
provide commands to include only the document body of
a child file thus allowing both files to be compiled individually.
\item
The packages \href{http://ctan.org/pkg/subdocs}{\textsf{subdocs}}
and \href{http://ctan.org/pkg/subfiles}{\textsf{subfiles}}
provide structures in which the main and child documents can be
encapsulated and allowing them to be compiled individually.
The inclusion mechanism is different from the conventional |\include|.
\item
The package \href{http://ctan.org/pkg/combine}{\textsf{combine}}
is an elaborate solution to combine several documents into one.
\end{itemize}
%
See also the CTAN topic \href{http://ctan.org/topic/subdocs}{\textsf{subdocs}}
for further related packages.
The present package differs from the above solutions in that
a document structure constructed with the conventional |\include| mechanism
just needs two extra commands at the top of every file
such that all constituent files can be compiled individually.

%%%%%%%%%%%%%%%%%%%%%%%%%%%%%%%%%%%%%%%%%%%%%%%%%%%%%%%%%%%%%%%%%%%%%%%%%%%%%%%%
%\subsection{Feature Suggestions}
%
%The following is a list of features which may be useful for future
%versions of this package:
%%
%\begin{itemize}
%\item
%\ldots
%\end{itemize}

%%%%%%%%%%%%%%%%%%%%%%%%%%%%%%%%%%%%%%%%%%%%%%%%%%%%%%%%%%%%%%%%%%%%%%%%%%%%%%%%
\subsection{Revision History}

%%%%%%%%%%%%%%%%%%%%%%%%%%%%%%%%%%%%%%%%
\paragraph{v2.0:} 2018/12/30

\begin{itemize}
\item
immediate forward processing
\item
added |\childdocby| mechanism
\item
manual restructured
\end{itemize}

%%%%%%%%%%%%%%%%%%%%%%%%%%%%%%%%%%%%%%%%
\paragraph{v1.6:} 2018/01/17

\begin{itemize}
\item
application for development of include files
\item
corrections to manual
\end{itemize}

%%%%%%%%%%%%%%%%%%%%%%%%%%%%%%%%%%%%%%%%
\paragraph{v1.5:} 2017/05/21

\begin{itemize}
\item
more complete structuring introduced
\item
|\childdocof| introduced
\item
|\childdoc| renamed to |\childdocmain|
\item
|\childredirect| renamed to |\childdocforward| and |\childdocforwardprefix|
and functionality expanded
\end{itemize}

%%%%%%%%%%%%%%%%%%%%%%%%%%%%%%%%%%%%%%%%
\paragraph{v1.0:} 2017/04/27

\begin{itemize}
\item
manual and install package
\item
first version published on CTAN
\end{itemize}

%%%%%%%%%%%%%%%%%%%%%%%%%%%%%%%%%%%%%%%%
\paragraph{v0.6:} 2017/04/26

\begin{itemize}
\item
redirection mechanism added
\end{itemize}

%%%%%%%%%%%%%%%%%%%%%%%%%%%%%%%%%%%%%%%%
\paragraph{v0.5:} 2017/04/26

\begin{itemize}
\item
functionality in definition file
\end{itemize}


%%%%%%%%%%%%%%%%%%%%%%%%%%%%%%%%%%%%%%%%%%%%%%%%%%%%%%%%%%%%%%%%%%%%%%%%%%%%%%%%
%%%%%%%%%%%%%%%%%%%%%%%%%%%%%%%%%%%%%%%%%%%%%%%%%%%%%%%%%%%%%%%%%%%%%%%%%%%%%%%%
%%%%%%%%%%%%%%%%%%%%%%%%%%%%%%%%%%%%%%%%%%%%%%%%%%%%%%%%%%%%%%%%%%%%%%%%%%%%%%%%
\appendix

\settowidth\MacroIndent{\rmfamily\scriptsize 000\ }

 \DocInput{childdoc.dtx}

\end{document}
%</driver>
% \fi
%
% %%%%%%%%%%%%%%%%%%%%%%%%%%%%%%%%%%%%%%%%%%%%%%%%%%%%%%%%%%%%%%%%%%%%%%%%%%%%%%
% %%%%%%%%%%%%%%%%%%%%%%%%%%%%%%%%%%%%%%%%%%%%%%%%%%%%%%%%%%%%%%%%%%%%%%%%%%%%%%
% \section{Sample}
%\iffalse
%<*samplemain>
%\fi
%
% The following presents a sample document
% with two chapters, two parts, a title page,
% a compile flag as well as three forwarding files to set the flag.
% It consists of eight |.tex| files:
% \begin{center}
% \begin{tabular}{ll}
% |cdocsamp.tex|&main file\\
% |cdocsch1.tex|&include file for chapter 1\\
% |cdocsch2.tex|&include file for chapter 2\\
% |cdocspt3.tex|&include file for part 3\\
% |cdocspt4.tex|&include file for part 4\\
% |cdocsdrf.tex|&forwarding file for main file in draft mode\\
% |cdocsfi1.tex|&forwarding file for final version of chapter 1\\
% |cdocsfi2.tex|&forwarding file for final version of chapter 2\\
% \end{tabular}
% \end{center}
% Each of the eight files can be compiled directly by the \LaTeX{} compiler.
%
% %%%%%%%%%%%%%%%%%%%%%%%%%%%%%%%%%%%%%%
% \paragraph{Main File.}
%
% The main file is called |cdocsamp.tex|.
%
% Load the \textsf{childdoc} definitions and
% declare the filename for the main document:
%    \begin{macrocode}
\input{childdoc.def}
\childdocmain{}
%    \end{macrocode}

% Optional override for |\version| flag:
%    \begin{macrocode}
%%\ifchilddoc\else\providecommand{\version}{draft}\fi
%    \end{macrocode}

% Define the default values for the |\version| flag
% (|final| for the main file and |draft| for childs):
%    \begin{macrocode}
\ifchilddoc
\providecommand{\version}{draft}
\else
\providecommand{\version}{final}
\fi
%    \end{macrocode}

% Load the standard document class:
%    \begin{macrocode}
\documentclass[12pt]{article}
%    \end{macrocode}

% Start the document body:
%    \begin{macrocode}
\begin{document}
%    \end{macrocode}

% Declare a title page.
% Print title, part of document being processed and version flag:
%    \begin{macrocode}
\addtocounter{page}{-1}
\begin{center}
{\LARGE\bfseries{}childdoc example\par}
\vspace{1cm}
\ifchilddoc
\ifchilddocmanual part\else chapter\fi:
`\childdocname' of `\childdocjob'\par
\else
main document: `\childdocjob'\par
\fi
version: \version\par
\end{center}
\newpage
%    \end{macrocode}

% Manually include selected file,
% otherwise process as usual:
%    \begin{macrocode}
\ifchilddocmanual
\section*{part `\childdocname'}
\input{\childdocname}
\else
%    \end{macrocode}

% Include the two chapters:
%    \begin{macrocode}
\include{cdocsch1}
\include{cdocsch2}
%    \end{macrocode}

% Include the two parts unless only chapters should be displayed:
%    \begin{macrocode}
\ifchilddoc\else
\section{part three}
\input{cdocspt3}
\section{part four}
\input{cdocspt4}
\fi
%    \end{macrocode}

% Process as usual until here:
%    \begin{macrocode}
\fi
%    \end{macrocode}

% End of document body:
%    \begin{macrocode}
\end{document}
%    \end{macrocode}
%\iffalse
%</samplemain>
%\fi
%
% %%%%%%%%%%%%%%%%%%%%%%%%%%%%%%%%%%%%%%
% \paragraph{Chapter Include Files.}
%
% The include files are called |cdocsch1.tex| and |cdocsch2.tex|.
%
%\iffalse
%<*samplechap1|samplechap2>
%\fi

% Optional override for |\version| flag:
%    \begin{macrocode}
%%\providecommand{\version}{final}
%    \end{macrocode}

% Include the main document:
%    \begin{macrocode}
\input{childdoc.def}
\childdocof{cdocsamp}
%    \end{macrocode}

%\iffalse
%</samplechap1|samplechap2>
%\fi
%
%\iffalse
%<*samplechap1>
%\fi
% Some text for chapter 1:
%    \begin{macrocode}
\section{one}
some text in chapter one
%    \end{macrocode}

%\iffalse
%</samplechap1>
%\fi
% Some text for chapter 2:
%\iffalse
%<*samplechap2>
%\fi
%    \begin{macrocode}
\section{two}
more text in chapter two
%    \end{macrocode}

%\iffalse
%</samplechap2>
%\fi
%
% %%%%%%%%%%%%%%%%%%%%%%%%%%%%%%%%%%%%%%
% \paragraph{Part Include Files.}
%
% The include files are called |cdocspt3.tex| and |cdocspt4.tex|.
%
%\iffalse
%<*samplepart3|samplepart4>
%\fi

% Optional override for |\version| flag:
%    \begin{macrocode}
%%\providecommand{\version}{final}
%    \end{macrocode}

% Include the main document:
%    \begin{macrocode}
\input{childdoc.def}
\childdocby{cdocsamp}
%    \end{macrocode}

%\iffalse
%</samplepart3|samplepart4>
%\fi
%
%\iffalse
%<*samplepart3>
%\fi
% Some text for part 3:
%    \begin{macrocode}
some text in part three
%    \end{macrocode}

%\iffalse
%</samplepart3>
%\fi
% Some text for part 4:
%\iffalse
%<*samplepart4>
%\fi
%    \begin{macrocode}
more text in part four
%    \end{macrocode}

%\iffalse
%</samplepart4>
%\fi
%
% %%%%%%%%%%%%%%%%%%%%%%%%%%%%%%%%%%%%%%
% \paragraph{Forwarding for a Complete Draft.}
%
% The following forwarding file |cdocsdrf.tex|
% compiles the main document in draft mode:
%\iffalse
%<*sampledraft>
%\fi
%    \begin{macrocode}
\def\version{draft}
\input{childdoc.def}
\childdocforward{cdocsamp}
%    \end{macrocode}

%\iffalse
%</sampledraft>
%\fi
%
% %%%%%%%%%%%%%%%%%%%%%%%%%%%%%%%%%%%%%%
% \paragraph{Forwarding for Final Version of the Chapters.}
%
% The following forwarding files |cdocsfn1.tex| and |cdocsfn2.tex|
% (with identical content)
% compile the final versions of the child documents
% |cdocsch1.tex| and |cdocsch2.tex|, respectively:
%\iffalse
%<*samplefinal>
%\fi
%    \begin{macrocode}
\def\version{final}
\input{childdoc.def}
\childdocforwardprefix[cdocsamp]{cdocsfn}{cdocsch}
%    \end{macrocode}

%\iffalse
%</samplefinal>
%\fi
%
% %%%%%%%%%%%%%%%%%%%%%%%%%%%%%%%%%%%%%%
% \paragraph{Command Line Processing.}
%
% The following three command lines generate the output files
% |cdocscld|, |cdocscl1| and |cdocscl2|
% which should be identical to
% |cdocsdrf|, |cdocsch1| and |cdocsfn2|, respectively:
% \begin{center}
% \begin{tabular}{l}
% |latex -jobname cdocscld \|\\
% |  "\def\version{draft}\input{childdoc.def}\childdocforward{cdocsamp}"|\\
% |latex -jobname cdocscl1 \|\\
% |  "\input{childdoc.def}\childdocforward[cdocsamp]{cdocsch1}"|\\
% |latex -jobname cdocscl2 \|\\
% |  "\def\version{final}\input{childdoc.def}\childdocforward{cdocsch2}"|
% \end{tabular}
% \end{center}
% Note that the trailing backslash on each first line
% merely continues the input to the second line
% (for convenient cut ant paste).
% Furthermore, the command |latex| can be replaced by any
% of its alternative versions such as |pdflatex|.
%
% %%%%%%%%%%%%%%%%%%%%%%%%%%%%%%%%%%%%%%%%%%%%%%%%%%%%%%%%%%%%%%%%%%%%%%%%%%%%%%
% %%%%%%%%%%%%%%%%%%%%%%%%%%%%%%%%%%%%%%%%%%%%%%%%%%%%%%%%%%%%%%%%%%%%%%%%%%%%%%
% \section{Implementation}
%\iffalse
%<*package>
%\fi
%
% This section describes the definitions file |childdoc.def|.

% The definitions cannot be loaded using |\usepackage| or |\RequirePackage|
% which has a mechanism to prevent loading a style file more than once.
% When loading the definitions by means of |\input|
% multiple instances have to be prevented manually:
%\iffalse
%This code needs to be before the `\ProvidesFile' directive
%which is defined at the beginning of this file.
%Therefore it is also placed there and commented out here.
%</package>
%<*discard>
%\fi
%    \begin{macrocode}
\ifdefined\childdocmain\endinput\fi
%    \end{macrocode}
%\iffalse
%</discard>
%<*package>
%\fi
%
% \macro{\ifchilddoc}
% \macro{\ifchilddocmanual}
% The conditional |\ifchilddoc| tells whether a
% child (true) or main (false) document is being compiled.
% The conditional |\ifchilddocmanual| tells whether
% the |\includeonly| mechanism is used (false) or
% the selection of child files must be performed manually (true).
% The definitions initialise to false:
%    \begin{macrocode}
\newif\ifchilddoc
\newif\ifchilddocmanual
%    \end{macrocode}

% \macro{\childdocname}
% \macro{\childdocjob}
% The macro |\childdocname| stores the name of the main document
% to be compiled. The macro |\childdocjob| stores the name of
% the document on which the \LaTeX{} compiler was originally invoked.
% The content of |\jobname| cannot be compared
% to filenames specified in the source due to different catcodes.
% The following code rescans |\jobname|, stores the result
% in |\childdocname| and saves a copy in |\childdocjob|:
%    \begin{macrocode}
\edef\childdocname{\scantokens\expandafter{\jobname\noexpand}}
\let\childdocjob\childdocname
%    \end{macrocode}

% \macro{\childdocdisable}
% The macro |\childdocdisable| prevents the main file
% from being processed more than once.
% At this stage, the main document command |\childdocmain|
% is assumed to be called once again where it should do nothing.
% Any subsequent call to it should prevent
% a secondary processing of the main document
% It overwrites the forwarding commands
% |\childdocof| and |\childdocforward|
% with empty macros to prevent further inclusions of the main document:
%    \begin{macrocode}
\newcommand{\childdocdisable}
{
  \renewcommand{\childdocmain}[1]{\renewcommand{\childdocmain}[1]{\endinput}}
  \renewcommand{\childdocof}[1]{}
  \renewcommand{\childdocby}[2][]{}
  \renewcommand{\childdocforward}[2][]{}
  \renewcommand{\childdocdisable}{}
}
%    \end{macrocode}

% \macro{\childdocmain}
% The macro |\childdocmain| is to be called at the top of the main file
% with nothing or the main filename (without extension) as argument.
% First, it breaks loops.
% If the argument is not empty and does not match |\childdocname|
% (which is set by the first inclusion of |childdoc.def|),
% |\ifchilddoc| is set to true, |\includeonly| is applied to the child file
% and |\jobname| is set to the main file
% (for proper handling of |.aux| files):
%    \begin{macrocode}
\newcommand{\childdocmain}[1]
{
  \childdocdisable\childdocmain{}
  \if?#1?\else
    \begingroup
      \def\childdoctmp{#1}
      \ifx\childdoctmp\childdocname
        \def\childdoctmp{}
      \else
        \def\childdoctmp
        {
          \childdoctrue
          \includeonly{\childdocname}
          \def\childdocjob{#1}
          \def\jobname{#1}
        }
      \fi
      \expandafter
    \endgroup
    \childdoctmp
  \fi
}
%    \end{macrocode}

% \macro{\childdocof}
% The command |\childdocof| redirects
% compilation to the main file |#1|.
%    \begin{macrocode}
\newcommand{\childdocof}[1]
{
  \childdocdisable
  \childdoctrue
  \includeonly{\childdocname}
  \def\jobname{#1}
  \def\childdocjob{#1}
  \input{#1}
}
%    \end{macrocode}

% \macro{\childdocby}
% The command |\childdocby| ....
%    \begin{macrocode}
\newcommand{\childdocby}[2][]
{
  \childdocdisable
  \childdoctrue
  \childdocmanualtrue
  \if?#1?\else
    \def\jobname{#2}
  \fi
  \def\childdocjob{#2}
  \input{#2}
  \endinput
}
%    \end{macrocode}

% \macro{\childdocforward}
% The command |\childdocforward| redirects
% compilation to the main file or
% (if the optional argument is given) a child file.
% Parameters are set as if the main file
% or a child file starting with |\childdocof| was compiled.
% Then compilation is handed over to the main file:
%    \begin{macrocode}
\newcommand{\childdocforward}[2][]
{
  \begingroup
    \if?#1?
      \def\childdoctmp
      {
        \def\childdocname{#2}
        \def\childdocjob{#2}
        \def\jobname{#2}
        \input{#2}
        \endinput
      }
    \else
      \def\childdoctmp
      {
        \childdocdisable
        \def\childdocname{#2}
        \childdoctrue
        \includeonly{#2}
        \def\childdocjob{#1}
        \def\jobname{#1}
        \input{#1}
        \endinput
      }
    \fi
    \expandafter
  \endgroup
  \childdoctmp
}
%    \end{macrocode}

% \macro{\childdocforwardprefix}
% The command |\childdocforwardprefix| redirects
% compilation to the main or a child file by means of a pattern.
% The prefix |#1| in the current filename is replaced by |#2|
% and the suffix of the current filename is kept
% (it is assumed that the filename does not contain the substring `|~~~|'
% which is used as a delimiter).
% Compilation is handed over to the new file by |\childdocforward|:
%    \begin{macrocode}
\newcommand{\childdocforwardprefix}[3][]
{
  \begingroup
    \def\childdocextract #2##1~~~{\def\childdoctmp{\childdocforward[#1]{#3##1}}}
    \expandafter\childdocextract\childdocname~~~
    \expandafter
  \endgroup
  \childdoctmp
}
%    \end{macrocode}

% \macro{\childdoc}
% The deprecated macro |\childdoc| is a legacy version of |\childdocmain|:
%    \begin{macrocode}
\newcommand{\childdoc}{\childdocmain}
%    \end{macrocode}

% \macro{\childdocredirect}
% The deprecated macro |\childdocredirect| is a legacy version
% of |\childdocforward| and |\childdocforwardprefix|:
%    \begin{macrocode}
\newcommand{\childdocredirect}[2][]
{
  \begingroup
    \if?#1?
      \def\childdoctmp{\childdocforward{#2}}
    \else
      \def\childdoctmp{\childdocforwardprefix{#1}{#2}}
    \fi
    \expandafter
  \endgroup
  \childdoctmp
}
%    \end{macrocode}

%\iffalse
%</package>
%\fi
%
\endinput

\childdocof{cdocsamp}
%    \end{macrocode}

%\iffalse
%</samplechap1|samplechap2>
%\fi
%
%\iffalse
%<*samplechap1>
%\fi
% Some text for chapter 1:
%    \begin{macrocode}
\section{one}
some text in chapter one
%    \end{macrocode}

%\iffalse
%</samplechap1>
%\fi
% Some text for chapter 2:
%\iffalse
%<*samplechap2>
%\fi
%    \begin{macrocode}
\section{two}
more text in chapter two
%    \end{macrocode}

%\iffalse
%</samplechap2>
%\fi
%
% %%%%%%%%%%%%%%%%%%%%%%%%%%%%%%%%%%%%%%
% \paragraph{Part Include Files.}
%
% The include files are called |cdocspt3.tex| and |cdocspt4.tex|.
%
%\iffalse
%<*samplepart3|samplepart4>
%\fi

% Optional override for |\version| flag:
%    \begin{macrocode}
%%\providecommand{\version}{final}
%    \end{macrocode}

% Include the main document:
%    \begin{macrocode}
% \iffalse
%
% childdoc.dtx Copyright (C) 2017-2018 Niklas Beisert
%
% This work may be distributed and/or modified under the
% conditions of the LaTeX Project Public License, either version 1.3
% of this license or (at your option) any later version.
% The latest version of this license is in
%   http://www.latex-project.org/lppl.txt
% and version 1.3 or later is part of all distributions of LaTeX
% version 2005/12/01 or later.
%
% This work has the LPPL maintenance status `maintained'.
%
% The Current Maintainer of this work is Niklas Beisert.
%
% This work consists of the files childdoc.dtx and childdoc.ins
% and the derived files childdoc.def and cdocsamp.tex with
% cdocsch1.tex, cdocsch2.tex, cdocsdrf.tex, cdocsfn1.tex, cdocsfn2.tex.
%
%<package>\ifdefined\childdocmain\endinput\fi
%<package>\ProvidesFile{childdoc.def}[2018/12/30 v2.0 child document driver]
%<samplemain>\ProvidesFile{cdocsamp.tex}[2018/12/30 v2.0 sample for childdoc]
%<*driver>
%\ProvidesFile{childdoc.drv}[2018/12/30 v2.0 childdoc reference manual file]
\PassOptionsToClass{10pt,a4paper}{article}
\documentclass{ltxdoc}

\usepackage[margin=35mm]{geometry}
\usepackage{hyperref}
\usepackage{hyperxmp}
\usepackage[usenames]{color}

\hypersetup{colorlinks=true}
\hypersetup{pdfstartview=FitH}
\hypersetup{pdfpagemode=UseNone}
\hypersetup{pdfsource={}}
\hypersetup{pdflang={en-UK}}
\hypersetup{pdfcopyright={Copyright 2017-2018 Niklas Beisert.
  This work may be distributed and/or modified under the
  conditions of the LaTeX Project Public License, either version 1.3
  of this license or (at your option) any later version.}}
\hypersetup{pdflicenseurl={http://www.latex-project.org/lppl.txt}}
\hypersetup{pdfcontactaddress={ETH Zurich, ITP, HIT K,
  Wolfgang-Pauli-Strasse 27}}
\hypersetup{pdfcontactpostcode={8093}}
\hypersetup{pdfcontactcity={Zurich}}
\hypersetup{pdfcontactcountry={Switzerland}}
\hypersetup{pdfcontactemail={nbeisert@itp.phys.ethz.ch}}
\hypersetup{pdfcontacturl={http://people.phys.ethz.ch/\xmptilde nbeisert/}}

\newcommand{\secref}[1]{\hyperref[#1]{section \ref*{#1}}}

\parskip1ex
\parindent0pt
\let\olditemize\itemize
\def\itemize{\olditemize\parskip0pt}

\begin{document}

\title{The \textsf{childdoc} Package}
\hypersetup{pdftitle={The childdoc Package}}
\author{Niklas Beisert\\[2ex]
  Institut f\"ur Theoretische Physik\\
  Eidgen\"ossische Technische Hochschule Z\"urich\\
  Wolfgang-Pauli-Strasse 27, 8093 Z\"urich, Switzerland\\[1ex]
  \href{mailto:nbeisert@itp.phys.ethz.ch}
  {\texttt{nbeisert@itp.phys.ethz.ch}}}
\hypersetup{pdfauthor={Niklas Beisert}}
\hypersetup{pdfsubject={Manual for the LaTeX2e Package childdoc}}
\date{30 December 2018, \textsf{v2.0}}
\maketitle

\begin{abstract}\noindent
\textsf{childdoc} is a \LaTeXe{} package
that enables the direct compilation
of document sections included by |\include|
to individual files.
\end{abstract}

\begingroup
\parskip0ex
\tableofcontents
\endgroup

%%%%%%%%%%%%%%%%%%%%%%%%%%%%%%%%%%%%%%%%%%%%%%%%%%%%%%%%%%%%%%%%%%%%%%%%%%%%%%%%
%%%%%%%%%%%%%%%%%%%%%%%%%%%%%%%%%%%%%%%%%%%%%%%%%%%%%%%%%%%%%%%%%%%%%%%%%%%%%%%%
\section{Introduction}

\LaTeX{} provides a mechanism to structure a large document (such as a book)
into a main file and several child files (containing the chapters)
using the |\include| command.
This mechanism is beneficial for documents
which span hundreds of pages in order to
make the source file(s) more manageable.
Moreover, compilation can be restricted to
selected child files by means of the |\includeonly| command.
The latter feature can be used to reduce the compilation time while editing
(this was significantly more useful in the earlier days of \LaTeX{})
or to generate a smaller document which is easier to navigate.
Another application of |\includeonly| is to generate
documents consisting of selected parts of the complete document.

However, there are a few drawbacks of the plain |\include| mechanism:
\begin{itemize}
\item
The child files cannot be compiled on their own,
they can only be compiled via the main file.
A naive editing environment
(such as a text editor with an option
to have the current file processed by \LaTeX)
may require one to switch to the main file before compiling;
attempting to compile the child file produces errors.
\item
The main file must be modified (each time)
to adjust the |\includeonly| command
to the present needs. This easily leaves the main file in a messy state.
\item
The generated document will always carry the filename
of the main document. This is inconvenient if
several child files are to be compiled and
to be kept for distribution.
\end{itemize}

The present package provides a simple interface
to make child files individually compilable by \LaTeX{}.
Compiling a child file then has the same effect as compiling
the main file with an |\includeonly| command
to select the appropriate child.
Moreover the generated document will carry the name of the child
rather than the main file.
This resolves all three above issues.

This feature is meant to make the editing of books,
thesis documents and lecture notes somewhat more convenient.
However, the package can also be used efficiently for
composing a series of documents (such as exercise sheets)
which are typically distributed individually.
It then assists the author in generating the individual documents
(potentially in different versions)
as well as a document containing the collected series.
Another application is in developing style files
or other kinds of included material
where compilation of the style file could redirect
to a sample or test file.

%%%%%%%%%%%%%%%%%%%%%%%%%%%%%%%%%%%%%%%%%%%%%%%%%%%%%%%%%%%%%%%%%%%%%%%%%%%%%%%%
%%%%%%%%%%%%%%%%%%%%%%%%%%%%%%%%%%%%%%%%%%%%%%%%%%%%%%%%%%%%%%%%%%%%%%%%%%%%%%%%
\section{Usage}

First of all, the package \textsf{childdoc} is \emph{not} a standard
\LaTeXe{} |.sty| style file! Therefore it needs to be invoked in
a non-standard way.

%%%%%%%%%%%%%%%%%%%%%%%%%%%%%%%%%%%%%%%%%%%%%%%%%%%%%%%%%%%%%%%%%%%%%%%%%%%%%%%%
\subsection{Included Files}
\label{sec:include}

%%%%%%%%%%%%%%%%%%%%%%%%%%%%%%%%%%%%%%%%
\DescribeMacro{\childdocmain}
To use the package, add the commands
\begin{center}
\begin{tabular}{l}
|\input{childdoc.def}|\\
|\childdocmain{}|\\
\end{tabular}
\end{center}
at the very top of the main \LaTeX{} file,
in particular \emph{before} the |\documentclass| statement!
The argument of |\childdocmain| should be left empty
(but it must be present).

%%%%%%%%%%%%%%%%%%%%%%%%%%%%%%%%%%%%%%%%
\DescribeMacro{\childdocof}
Furthermore, add the commands
\begin{center}
\begin{tabular}{l}
|\input{childdoc.def}|\\
|\childdocof{|\textit{main}|}|\\
\end{tabular}
\end{center}
at the top of every child file \textit{child}
which is included by |\include{|\textit{child}|}|
from within the main file
(or at least for those files to be compiled individually).
The argument \textit{main} must be the filename of the main file.

There are a couple of
considerations in setting up the main and child documents:

%%%%%%%%%%%%%%%%%%%%%%%%%%%%%%%%%%%%%%%%
\paragraph{Restrictions.}

Please note the following restrictions:
\begin{itemize}
\item
|\childdocmain| must be called with one argument \textit{main}
to ensure compatibility with earlier version of the package.
It must either be empty (|\childdocmain{}|)
or precisely match the filename of the main file in which it is specified.
See \secref{sec:detection} for further information.
\item
The filename \textit{main} must be specified without the |.tex| extension.
\item
The filename \textit{main} is case sensitive
(even in case-insensitive file systems)
due to internal string comparison.
\item
The argument \textit{main} should be fully expanded, it cannot be a macro.
\item
Subdirectories and special characters should be avoided in filenames.
\item
The command |\childdocmain{|\textit{main}|}| must be followed by a whitespace.
It should not be followed immediately by another command
or by a comment mark `|%|'.
This is because the \TeX{} parser reads the token immediately following
the argument of |\childdocmain| and puts it
at the beginning of every child section;
however, a white\-space is ignored.
\end{itemize}

%%%%%%%%%%%%%%%%%%%%%%%%%%%%%%%%%%%%%%%%
\paragraph{Content of Main File.}

It is advisable to place all content in the child files included by |\include|.
Any output contained in the main file will appear in all child documents
unless suppressed manually;
it cannot be suppressed automatically by the |\includeonly| directive
and thus should normally be avoided.
A method to include some content in the main file
by means of conditional processing is described in \secref{sec:conditional}.

%%%%%%%%%%%%%%%%%%%%%%%%%%%%%%%%%%%%%%%%
\paragraph{Page Numbering.}

When only a part of the document is compiled,
the appropriate numbering of pages
(as well as other status parameters)
is determined from the |.aux| files.
The latter contain information from previous passes.
However this information needs to propagate through
all intermediate child documents.
Therefore the page numbering in child documents may well
be inconsistent until the complete document is compiled at least once.

A useful (if unconventional) way to always ensure a consistent
page numbering is to restart the numbering in each child document
and denote the pages by `\textit{child}|.|\textit{page}'
where \textit{child} represents the chapter/section number of the child file.
This can be achieved by the command
|\numberwithin{page}{|\textit{child}|}|
of the \textsf{amsmath} package
where \textit{child} can be |chapter| or |section|
depending on the chosen structuring.
Alternatively, one can modify the macro |\thepage| appropriately
and reset the counter |page| at the start of each child file.

%%%%%%%%%%%%%%%%%%%%%%%%%%%%%%%%%%%%%%%%%%%%%%%%%%%%%%%%%%%%%%%%%%%%%%%%%%%%%%%%
\subsection{Conditional Processing}
\label{sec:conditional}

The package provides a mechanism to compile different versions
of a document. To customise the versions further some conditional processing
can come in handy to distinguish which version is being compiled.
The package provides two macros to describe the compilation context:

%%%%%%%%%%%%%%%%%%%%%%%%%%%%%%%%%%%%%%%%
\DescribeMacro{\ifchilddoc}
The conditional |\ifchilddoc| distinguishes between the compilation of
child documents and the main document:
%
\begin{center}
|\ifchilddoc |\textit{child-code}| |[|\||else |\textit{main-code}]| \||fi|
\end{center}

%%%%%%%%%%%%%%%%%%%%%%%%%%%%%%%%%%%%%%%%
\DescribeMacro{\childdocname}
\DescribeMacro{\childdocjob}
The macro |\childdocname| contains the filename (without extension)
of the main or child file being processed.
Note that |\childdocjob| will always contain the name of the main file.

%%%%%%%%%%%%%%%%%%%%%%%%%%%%%%%%%%%%%%%%
\paragraph{Title Page.}

Conditional processing can be used to include a title or banner page
in the main document when proper precautions are taken.
Importantly, the code in the main file should ensure that the page counter
(as well as other status parameters which are stored in the |.aux| files)
takes the same value after the conditional processing.
Otherwise the page numbers may take divergent values
depending on which part is compiled.

For example, a title page could be declared by:
%
\begin{center}
\begin{tabular}{l}
|\ifchilddoc\||else|\\
|\addtocounter{page}{-1}|\\
\textit{code for title page}\\
|\newpage|\\
|\||fi|
\end{tabular}
\end{center}
%
A banner page for the child documents can be generated by:
%
\begin{center}
\begin{tabular}{l}
|\ifchilddoc|\\
|\addtocounter{page}{-1}|\\
\textit{code for banner page}\\
|\newpage|\\
|\||fi|
\end{tabular}
\end{center}
%
Here one could write a message such as:
\begin{center}
|This is the part \childdocname{} of \childdocjob{}.|
\end{center}

%%%%%%%%%%%%%%%%%%%%%%%%%%%%%%%%%%%%%%%%%%%%%%%%%%%%%%%%%%%%%%%%%%%%%%%%%%%%%%%%
\subsection{Flags}
\label{sec:flags}

The package makes it easy to generate different versions
of the main or child documents.
To this end compilation flags can be defined
and assigned different default values.
They will be particularly useful in conjunction
with the forwarding mechanism described in \secref{sec:forward}.

For example, it may be useful to have a flag |\version|
which can be set to |draft| or |final|.
The document source will contain some conditional code
depending on the value of |\version|.
Suppose further, the flag should default to |final| for the main file
and to |draft| for child files
which is a natural assignment for editing the document.
This is achieved by placing the following code
in the preamble of the main document
(below the |\childdocmain| directive):
%
\begin{center}
\begin{tabular}{l}
|\ifchilddoc|\\
|\providecommand{\version}{draft}|\\
|\||else|\\
|\providecommand{\version}{final}|\\
|\||fi|
\end{tabular}
\end{center}
%
The definition by |\providecommand| makes sure
that previous definitions are not overwritten.
Further statements |\providecommand{\version}{...}|
can thus be added before the above code to override it.

For the main file, one might add a line
(between |\childdocmain| and the above block)
%
\begin{center}
|%\ifchilddoc\||else\providecommand{\version}{draft}\||fi|
\end{center}
%
which can be uncommented to produce a draft version.
Likewise one can add a line to the very top of a child file
(above the |\childdocof{|\textit{main}|}| directive)
%
\begin{center}
|%\providecommand{\version}{final}|
\end{center}
%
which can be uncommented to produce the final version of this child document.

%%%%%%%%%%%%%%%%%%%%%%%%%%%%%%%%%%%%%%%%%%%%%%%%%%%%%%%%%%%%%%%%%%%%%%%%%%%%%%%%
\subsection{Forwarding}
\label{sec:forward}

Different versions of the main or child documents
using compilation flags as described in \secref{sec:flags}
can be (permanently) stored in different files
for convenient compilation, viewing and distribution.
To this end, the package defines a command
to pass on compilation to a different file:

%%%%%%%%%%%%%%%%%%%%%%%%%%%%%%%%%%%%%%%%
\DescribeMacro{\childdocforward}
The command |\childdocforward| redirects processing to
another source file:
%
\begin{center}
\begin{tabular}{l}
|\input{childdoc.def}|\\
|\childdocforward[|\textit{main}|]{|\textit{dest}|}|\\
\end{tabular}
\end{center}
%
The argument \textit{dest} is the destination file
(without extension).
It should be the main file or one of the child files.
Note that further \textsf{childdoc} directives
such as |\childdocof| and |\childdocforward|
in the indicated file will be processed in this form.
The optional argument \textit{main}
passes on directly to the main file \textit{main}
while pretending to compile the child \textit{dest}.
This form behaves as if \textit{dest}
issues |\childdocof{|\textit{main}|}| right away,
and no further \textsf{childdoc} directives will be processed.

%%%%%%%%%%%%%%%%%%%%%%%%%%%%%%%%%%%%%%%%
\DescribeMacro{\...prefix}
In the alternative form |\childdocforwardprefix|,
%
\begin{center}
\begin{tabular}{l}
|\input{childdoc.def}|\\
|\childdocforwardprefix[|\textit{main}|]{|\textit{prefix}|}{|\textit{dest}|}|
\end{tabular}
\end{center}
%
the destination file is determined by a pattern
depending on the current file:
To make this work, the current file must be called
`{\textit{prefix}\hspace{0.2em}\textit{suffix}}'
with \textit{prefix} matching precisely the argument.
Processing is then passed on to the file
`{\textit{dest}\hspace{0.2em}\textit{suffix}}'.
Surely, the same effect is achieved by
directly specifying the
argument `{\textit{dest}\hspace{0.2em}\textit{suffix}}'
in the first form.
However, that requires to set up a different file
for each child. With the alternative form of the command
all these files can have exactly the same content
which simplifies setting them up and maintaining them.

For example, the following file |draft.tex|
with a compilation flag |\version| as described in \secref{sec:flags}
compiles the main document as a draft:
%
\begin{center}
\begin{tabular}{l}
|\def\version{draft}|\\
|\input{childdoc.def}|\\
|\childdocforward{|\textit{main}|}|
\end{tabular}
\end{center}
%
Likewise, the following files |final|\textit{nn}|.tex|
compile the final version of the child document
|child|\textit{nn}|.tex|:
%
\begin{center}
\begin{tabular}{l}
|\def\version{final}|\\
|\input{childdoc.def}|\\
|\childdocforwardprefix{final}{child}|
\end{tabular}
\end{center}
%

Note that when several versions of a main file and/or of each child file
are to be generated, it may be convenient to set up a |Makefile| or
shell script to automatise the process.

%%%%%%%%%%%%%%%%%%%%%%%%%%%%%%%%%%%%%%%%%%%%%%%%%%%%%%%%%%%%%%%%%%%%%%%%%%%%%%%%
\subsection{Command Line Processing}
\label{sec:commandline}

The effect of redirection files can also be achieved by invoking
the \LaTeX{} compiler with a more elaborate command line.
Most conveniently this should be done as part
of a shell script or a |Makefile|.

When using \textsf{childdoc} in the main file, the following
command lines effectively perform a redirection
(note that depending on the shell being used,
backslashes may have to be doubled: `|\|' $\to$ `|\\|'):
%
\begin{center}
|... -jobname "|\textit{target}|" |\\|"|[\textit{flags}]%
|\input{childdoc.def}\childdocforward[|\textit{main}|]{|\textit{dest}|}"|
\end{center}
%
Here \textit{target} is the name of the output file,
\textit{main} is the name of the main file
and \textit{dest} is the name of the main or child file to be processed
(all filenames without extensions).
The optional argument \textit{main} can be omitted
if \textit{main} matches \textit{dest}.
Optionally, compilation \textit{flags} can be defined via |\def| commands.
This command line makes the \TeX{} engine believe
it is compiling the file \textit{target}
whose content is specified as the latter parameter.
The provided code then forwards the processing to
\textit{main} or \textit{dest} as described in \secref{sec:forward}.

%%%%%%%%%%%%%%%%%%%%%%%%%%%%%%%%%%%%%%%%%%%%%%%%%%%%%%%%%%%%%%%%%%%%%%%%%%%%%%%%
\subsection{Include by Input}
\label{sec:input}

Including child documents by |\include| has some restrictions by design.
Most notably, the content of a child document always occupies
its own set of pages; pages cannot be shared between child documents.
Usually, this behaviour makes perfect sense
because each child document contain an essential part of the document.
However, in some situations it may be desirable to compose
a document from a collection of parts
without having mandatory page breaks between then.
For this case, the package
provides a mechanism to include parts
by |\input| which can also be processed individually.
However, by construction this mechanism
requires manual handling of the content to be output.

%%%%%%%%%%%%%%%%%%%%%%%%%%%%%%%%%%%%%%%%
\DescribeMacro{\ifchilddocmanual}
The main file should be prepared as usual, see \secref{sec:include}.
However, the document body must make a distinction
between processing of an individual part and of the main document, e.g.:
%
\begin{center}
\begin{tabular}{l}
|\ifchilddocmanual|\\
|\input{\childdocname}|\\
|\||else|\\
\textit{document body with }|\input{|\textit{part}|}|\\
|\||fi|
\end{tabular}
\end{center}
%
The conditional |\ifchilddocmanual| is true whenever
a part to be included by |\input| is being compiled,
and the name of the part is stored in |\childdocname|.

%%%%%%%%%%%%%%%%%%%%%%%%%%%%%%%%%%%%%%%%
\DescribeMacro{\childdocby}
Each part to be included by |\input| should start with:
%
\begin{center}
\begin{tabular}{l}
|\input{childdoc.def}|\\
|\childdocby{|\textit{main}|}|\\
\end{tabular}
\end{center}
%
The directive |\childdocby| is similar to |\childdocof|
described in \secref{sec:include},
but the subsequent selection of content must be done manually.
To that end, both |\ifchilddoc| and |\ifchilddocmanual|
will be true upon processing of a part,
and the name of the part is stored in |\childdocname|.
Note that |\jobname| will be set to the filename of the current part
so that each part receives an individual |.aux| file
that does not interfere with the |.aux| file(s) of the main document.
This behaviour can be altered by the alternative form
|\childdocby[*]{|\textit{main}|}| (with a non-empty optional argument)
which uses the |.aux| file of the main document
by setting |\jobname| to \textit{main}.

%%%%%%%%%%%%%%%%%%%%%%%%%%%%%%%%%%%%%%%%%%%%%%%%%%%%%%%%%%%%%%%%%%%%%%%%%%%%%%%%
\subsection{Driver Development}
\label{sec:driver}

The \textsf{childdoc} mechanism can also be use for the development
of definition files such as \LaTeX{} styles or classes.
This case differs from the above setup with multiple parts
included by |\include| in that no |\includeonly| should be invoked.
This can be achieved by starting the include file
(before |\ProvidesPackage|) with:
%
\begin{center}
\begin{tabular}{l}
|\input{childdoc.def}|\\
|\childdocforward{|\textit{main}|}|\\
\end{tabular}
\end{center}
%
or alternatively with:
%
\begin{center}
\begin{tabular}{l}
|\input{childdoc.def}|\\
|\childdocby{|\textit{main}|}|\\
\end{tabular}
\end{center}
%
Both forms have slightly different effects as described above.
The main file is prepared as usual, see \secref{sec:include}.

%%%%%%%%%%%%%%%%%%%%%%%%%%%%%%%%%%%%%%%%%%%%%%%%%%%%%%%%%%%%%%%%%%%%%%%%%%%%%%%%
\subsection{Legacy Detection}
\label{sec:detection}

The directive |\childdocmain| in the main file can detect
whether the complete document or merely a child is to be compiled
even without using the directive |\childdocof|.
This method is deprecated because it is less robust
and there is no compelling reason to use it;
it is merely provided for backward compatibility
and it may be removed in future versions.

If the detection mechanism is to be used,
it is mandatory to correctly specify
the filename of the main file as the argument of |\childdocmain|:
%
\begin{center}
\begin{tabular}{l}
|\input{childdoc.def}|\\
|\childdocmain{|\textit{main}|}|\\
\end{tabular}
\end{center}
%
If |\jobname| does not match the argument \textit{main} of |\childdocmain|,
it is assumed that |\jobname| points to the child file to be compiled.
When using |\childdocmain| with the main file specified as argument,
it suffices to start a child file
with just |\input{|\textit{main}|}|
without loading of the package and using |\childdocof|.
If instead all processing is done
with the appropriate \textsf{childdoc} directives,
the argument of \textit{main} of |\childdocmain| can be empty.

An alternative version of the command line processing described
in \secref{sec:commandline} using the detection mechanism reads:
%
\begin{center}
|... -jobname "|\textit{target}|" "|[\textit{flags}]%
[|\def\jobname{|\textit{dest}|}|]|\input{|\textit{main}|}"|
\end{center}

%%%%%%%%%%%%%%%%%%%%%%%%%%%%%%%%%%%%%%%%%%%%%%%%%%%%%%%%%%%%%%%%%%%%%%%%%%%%%%%%
\subsection{Manual Code}
\label{sec:manual}

In case one cannot be certain whether the definitions file |childdoc.def|
is installed on the target \TeX{} distribution
and one prefers not to ship it,
it is conceivable to paste a few relevant commands into the sources.

To that end, drop all statements |\input{childdoc.def}|
and perform the replacements as outlined below.
Instead of |\childdocmain{|\textit{main}|}| add the following code
to the top of the main file:
%
\begin{center}
\begin{tabular}{l}
|\||ifdefined\childdocname\endinput\||fi\newif\ifchilddoc|\\
|\edef\childdocname{\scantokens\expandafter{\jobname\noexpand}}|\\
|\def\childdocmain{|\textit{main}|}\||ifx\childdocmain\childdocname\||else|\\
|\childdoctrue\includeonly{\childdocname}\let\jobname\childdocmain\||fi|\\
\end{tabular}
\end{center}
%
Instead of |\childdocof{|\textit{main}|}| just include the main file
at the top of each child file:
%
\begin{center}
|\input{|\textit{main}|}|
\end{center}
%
A simple redirection |\childdocforward{|\textit{dest}|}| is achieved by:
%
\begin{center}
|\def\jobname{|\textit{dest}|}\input{\jobname}|
\end{center}
%
The redirection with prefix
|\childdocforwardprefix[|\textit{prefix}|]{|\textit{dest}|}|
is accomplished by:
%
\begin{center}
\begin{tabular}{l}
|{\edef\jobname{\scantokens\expandafter{\jobname\noexpand}}|\\
|\def\redirectjob |\textit{prefix}|#1~~~{\gdef\jobname{|\textit{dest}|#1}}|\\
|\expandafter\redirectjob\jobname~~~}\input{\jobname}|
\end{tabular}
\end{center}

In an alternative approach,
child documents can be compiled by a specific command line
without additional code or specific definitions:
%
\begin{center}
|... -jobname "|\textit{target}|" "|[\textit{flags}]%
|\includeonly{|\textit{dest}|}\input{|\textit{main}|}"|
\end{center}
%

%%%%%%%%%%%%%%%%%%%%%%%%%%%%%%%%%%%%%%%%%%%%%%%%%%%%%%%%%%%%%%%%%%%%%%%%%%%%%%%%
%%%%%%%%%%%%%%%%%%%%%%%%%%%%%%%%%%%%%%%%%%%%%%%%%%%%%%%%%%%%%%%%%%%%%%%%%%%%%%%%
\section{Information}

%%%%%%%%%%%%%%%%%%%%%%%%%%%%%%%%%%%%%%%%%%%%%%%%%%%%%%%%%%%%%%%%%%%%%%%%%%%%%%%%
\subsection{Copyright}

Copyright \copyright{} 2017--2018 Niklas Beisert

This work may be distributed and/or modified under the
conditions of the \LaTeX{} Project Public License, either version 1.3
of this license or (at your option) any later version.
The latest version of this license is in
  \url{http://www.latex-project.org/lppl.txt}
and version 1.3 or later is part of all distributions of \LaTeX{}
version 2005/12/01 or later.

This work has the LPPL maintenance status `maintained'.

The Current Maintainer of this work is Niklas Beisert.

This work consists of the files |README.txt|, |childdoc.ins| and |childdoc.dtx|
as well as the derived files |childdoc.def|, |cdocsamp.tex|
with |cdocsch1.tex|, |cdocsch2.tex|, |cdocspt3.tex|, |cdocspt4.tex|,
|cdocsdrf.tex|, |cdocsfn1.tex|, |cdocsfn2.tex|
as well as |childdoc.pdf|.

%%%%%%%%%%%%%%%%%%%%%%%%%%%%%%%%%%%%%%%%%%%%%%%%%%%%%%%%%%%%%%%%%%%%%%%%%%%%%%%%
\subsection{Files and Installation}

The package consists of the files:
%
\begin{center}
\begin{tabular}{ll}
    |README.txt|   & readme file \\
    |childdoc.ins| & installation file \\
    |childdoc.dtx| & source file \\
    |childdoc.def| & definition file \\
    |cdocsamp.tex| & sample main file \\
    |cdocsch1.tex| & sample include file \\
    |cdocsch2.tex| & sample include file \\
    |cdocspt3.tex| & sample part file \\
    |cdocspt4.tex| & sample part file \\
    |cdocsdrf.tex| & sample redirection file \\
    |cdocsfn1.tex| & sample redirection file \\
    |cdocsfn2.tex| & sample redirection file \\
    |childdoc.pdf| & manual
\end{tabular}
\end{center}
%
The distribution consists of the files
|README.txt|, |childdoc.ins| and |childdoc.dtx|.
%
\begin{itemize}
\item
Run (pdf)\LaTeX{} on |childdoc.dtx|
to compile the manual |childdoc.pdf| (this file).
\item
Run \LaTeX{} on |childdoc.ins| to create the definitions file |childdoc.def|
and the sample |cdocsamp.tex| with include files
|cdocsch1.tex|, |cdocsch2.tex|, |cdocspt3.tex|, |cdocspt4.tex|,
|cdocsdrf.tex|, |cdocsfn1.tex|, |cdocsfn2.tex|.
Then copy the file |childdoc.def| to an appropriate directory of your \LaTeX{}
distribution, e.g.\ \textit{texmf-root}|/tex/latex/childdoc|.
\end{itemize}

%%%%%%%%%%%%%%%%%%%%%%%%%%%%%%%%%%%%%%%%%%%%%%%%%%%%%%%%%%%%%%%%%%%%%%%%%%%%%%%%
\subsection{Related CTAN Packages}

There are several other packages which offer a similar functionality:
%
\begin{itemize}
\item
The packages
\href{http://ctan.org/pkg/docmute}{\textsf{docmute}},
\href{http://ctan.org/pkg/includex}{\textsf{includex}} and
\href{http://ctan.org/pkg/standalone}{\textsf{standalone}}
provide commands to include only the document body of
a child file thus allowing both files to be compiled individually.
\item
The packages \href{http://ctan.org/pkg/subdocs}{\textsf{subdocs}}
and \href{http://ctan.org/pkg/subfiles}{\textsf{subfiles}}
provide structures in which the main and child documents can be
encapsulated and allowing them to be compiled individually.
The inclusion mechanism is different from the conventional |\include|.
\item
The package \href{http://ctan.org/pkg/combine}{\textsf{combine}}
is an elaborate solution to combine several documents into one.
\end{itemize}
%
See also the CTAN topic \href{http://ctan.org/topic/subdocs}{\textsf{subdocs}}
for further related packages.
The present package differs from the above solutions in that
a document structure constructed with the conventional |\include| mechanism
just needs two extra commands at the top of every file
such that all constituent files can be compiled individually.

%%%%%%%%%%%%%%%%%%%%%%%%%%%%%%%%%%%%%%%%%%%%%%%%%%%%%%%%%%%%%%%%%%%%%%%%%%%%%%%%
%\subsection{Feature Suggestions}
%
%The following is a list of features which may be useful for future
%versions of this package:
%%
%\begin{itemize}
%\item
%\ldots
%\end{itemize}

%%%%%%%%%%%%%%%%%%%%%%%%%%%%%%%%%%%%%%%%%%%%%%%%%%%%%%%%%%%%%%%%%%%%%%%%%%%%%%%%
\subsection{Revision History}

%%%%%%%%%%%%%%%%%%%%%%%%%%%%%%%%%%%%%%%%
\paragraph{v2.0:} 2018/12/30

\begin{itemize}
\item
immediate forward processing
\item
added |\childdocby| mechanism
\item
manual restructured
\end{itemize}

%%%%%%%%%%%%%%%%%%%%%%%%%%%%%%%%%%%%%%%%
\paragraph{v1.6:} 2018/01/17

\begin{itemize}
\item
application for development of include files
\item
corrections to manual
\end{itemize}

%%%%%%%%%%%%%%%%%%%%%%%%%%%%%%%%%%%%%%%%
\paragraph{v1.5:} 2017/05/21

\begin{itemize}
\item
more complete structuring introduced
\item
|\childdocof| introduced
\item
|\childdoc| renamed to |\childdocmain|
\item
|\childredirect| renamed to |\childdocforward| and |\childdocforwardprefix|
and functionality expanded
\end{itemize}

%%%%%%%%%%%%%%%%%%%%%%%%%%%%%%%%%%%%%%%%
\paragraph{v1.0:} 2017/04/27

\begin{itemize}
\item
manual and install package
\item
first version published on CTAN
\end{itemize}

%%%%%%%%%%%%%%%%%%%%%%%%%%%%%%%%%%%%%%%%
\paragraph{v0.6:} 2017/04/26

\begin{itemize}
\item
redirection mechanism added
\end{itemize}

%%%%%%%%%%%%%%%%%%%%%%%%%%%%%%%%%%%%%%%%
\paragraph{v0.5:} 2017/04/26

\begin{itemize}
\item
functionality in definition file
\end{itemize}


%%%%%%%%%%%%%%%%%%%%%%%%%%%%%%%%%%%%%%%%%%%%%%%%%%%%%%%%%%%%%%%%%%%%%%%%%%%%%%%%
%%%%%%%%%%%%%%%%%%%%%%%%%%%%%%%%%%%%%%%%%%%%%%%%%%%%%%%%%%%%%%%%%%%%%%%%%%%%%%%%
%%%%%%%%%%%%%%%%%%%%%%%%%%%%%%%%%%%%%%%%%%%%%%%%%%%%%%%%%%%%%%%%%%%%%%%%%%%%%%%%
\appendix

\settowidth\MacroIndent{\rmfamily\scriptsize 000\ }

 \DocInput{childdoc.dtx}

\end{document}
%</driver>
% \fi
%
% %%%%%%%%%%%%%%%%%%%%%%%%%%%%%%%%%%%%%%%%%%%%%%%%%%%%%%%%%%%%%%%%%%%%%%%%%%%%%%
% %%%%%%%%%%%%%%%%%%%%%%%%%%%%%%%%%%%%%%%%%%%%%%%%%%%%%%%%%%%%%%%%%%%%%%%%%%%%%%
% \section{Sample}
%\iffalse
%<*samplemain>
%\fi
%
% The following presents a sample document
% with two chapters, two parts, a title page,
% a compile flag as well as three forwarding files to set the flag.
% It consists of eight |.tex| files:
% \begin{center}
% \begin{tabular}{ll}
% |cdocsamp.tex|&main file\\
% |cdocsch1.tex|&include file for chapter 1\\
% |cdocsch2.tex|&include file for chapter 2\\
% |cdocspt3.tex|&include file for part 3\\
% |cdocspt4.tex|&include file for part 4\\
% |cdocsdrf.tex|&forwarding file for main file in draft mode\\
% |cdocsfi1.tex|&forwarding file for final version of chapter 1\\
% |cdocsfi2.tex|&forwarding file for final version of chapter 2\\
% \end{tabular}
% \end{center}
% Each of the eight files can be compiled directly by the \LaTeX{} compiler.
%
% %%%%%%%%%%%%%%%%%%%%%%%%%%%%%%%%%%%%%%
% \paragraph{Main File.}
%
% The main file is called |cdocsamp.tex|.
%
% Load the \textsf{childdoc} definitions and
% declare the filename for the main document:
%    \begin{macrocode}
\input{childdoc.def}
\childdocmain{}
%    \end{macrocode}

% Optional override for |\version| flag:
%    \begin{macrocode}
%%\ifchilddoc\else\providecommand{\version}{draft}\fi
%    \end{macrocode}

% Define the default values for the |\version| flag
% (|final| for the main file and |draft| for childs):
%    \begin{macrocode}
\ifchilddoc
\providecommand{\version}{draft}
\else
\providecommand{\version}{final}
\fi
%    \end{macrocode}

% Load the standard document class:
%    \begin{macrocode}
\documentclass[12pt]{article}
%    \end{macrocode}

% Start the document body:
%    \begin{macrocode}
\begin{document}
%    \end{macrocode}

% Declare a title page.
% Print title, part of document being processed and version flag:
%    \begin{macrocode}
\addtocounter{page}{-1}
\begin{center}
{\LARGE\bfseries{}childdoc example\par}
\vspace{1cm}
\ifchilddoc
\ifchilddocmanual part\else chapter\fi:
`\childdocname' of `\childdocjob'\par
\else
main document: `\childdocjob'\par
\fi
version: \version\par
\end{center}
\newpage
%    \end{macrocode}

% Manually include selected file,
% otherwise process as usual:
%    \begin{macrocode}
\ifchilddocmanual
\section*{part `\childdocname'}
\input{\childdocname}
\else
%    \end{macrocode}

% Include the two chapters:
%    \begin{macrocode}
\include{cdocsch1}
\include{cdocsch2}
%    \end{macrocode}

% Include the two parts unless only chapters should be displayed:
%    \begin{macrocode}
\ifchilddoc\else
\section{part three}
\input{cdocspt3}
\section{part four}
\input{cdocspt4}
\fi
%    \end{macrocode}

% Process as usual until here:
%    \begin{macrocode}
\fi
%    \end{macrocode}

% End of document body:
%    \begin{macrocode}
\end{document}
%    \end{macrocode}
%\iffalse
%</samplemain>
%\fi
%
% %%%%%%%%%%%%%%%%%%%%%%%%%%%%%%%%%%%%%%
% \paragraph{Chapter Include Files.}
%
% The include files are called |cdocsch1.tex| and |cdocsch2.tex|.
%
%\iffalse
%<*samplechap1|samplechap2>
%\fi

% Optional override for |\version| flag:
%    \begin{macrocode}
%%\providecommand{\version}{final}
%    \end{macrocode}

% Include the main document:
%    \begin{macrocode}
\input{childdoc.def}
\childdocof{cdocsamp}
%    \end{macrocode}

%\iffalse
%</samplechap1|samplechap2>
%\fi
%
%\iffalse
%<*samplechap1>
%\fi
% Some text for chapter 1:
%    \begin{macrocode}
\section{one}
some text in chapter one
%    \end{macrocode}

%\iffalse
%</samplechap1>
%\fi
% Some text for chapter 2:
%\iffalse
%<*samplechap2>
%\fi
%    \begin{macrocode}
\section{two}
more text in chapter two
%    \end{macrocode}

%\iffalse
%</samplechap2>
%\fi
%
% %%%%%%%%%%%%%%%%%%%%%%%%%%%%%%%%%%%%%%
% \paragraph{Part Include Files.}
%
% The include files are called |cdocspt3.tex| and |cdocspt4.tex|.
%
%\iffalse
%<*samplepart3|samplepart4>
%\fi

% Optional override for |\version| flag:
%    \begin{macrocode}
%%\providecommand{\version}{final}
%    \end{macrocode}

% Include the main document:
%    \begin{macrocode}
\input{childdoc.def}
\childdocby{cdocsamp}
%    \end{macrocode}

%\iffalse
%</samplepart3|samplepart4>
%\fi
%
%\iffalse
%<*samplepart3>
%\fi
% Some text for part 3:
%    \begin{macrocode}
some text in part three
%    \end{macrocode}

%\iffalse
%</samplepart3>
%\fi
% Some text for part 4:
%\iffalse
%<*samplepart4>
%\fi
%    \begin{macrocode}
more text in part four
%    \end{macrocode}

%\iffalse
%</samplepart4>
%\fi
%
% %%%%%%%%%%%%%%%%%%%%%%%%%%%%%%%%%%%%%%
% \paragraph{Forwarding for a Complete Draft.}
%
% The following forwarding file |cdocsdrf.tex|
% compiles the main document in draft mode:
%\iffalse
%<*sampledraft>
%\fi
%    \begin{macrocode}
\def\version{draft}
\input{childdoc.def}
\childdocforward{cdocsamp}
%    \end{macrocode}

%\iffalse
%</sampledraft>
%\fi
%
% %%%%%%%%%%%%%%%%%%%%%%%%%%%%%%%%%%%%%%
% \paragraph{Forwarding for Final Version of the Chapters.}
%
% The following forwarding files |cdocsfn1.tex| and |cdocsfn2.tex|
% (with identical content)
% compile the final versions of the child documents
% |cdocsch1.tex| and |cdocsch2.tex|, respectively:
%\iffalse
%<*samplefinal>
%\fi
%    \begin{macrocode}
\def\version{final}
\input{childdoc.def}
\childdocforwardprefix[cdocsamp]{cdocsfn}{cdocsch}
%    \end{macrocode}

%\iffalse
%</samplefinal>
%\fi
%
% %%%%%%%%%%%%%%%%%%%%%%%%%%%%%%%%%%%%%%
% \paragraph{Command Line Processing.}
%
% The following three command lines generate the output files
% |cdocscld|, |cdocscl1| and |cdocscl2|
% which should be identical to
% |cdocsdrf|, |cdocsch1| and |cdocsfn2|, respectively:
% \begin{center}
% \begin{tabular}{l}
% |latex -jobname cdocscld \|\\
% |  "\def\version{draft}\input{childdoc.def}\childdocforward{cdocsamp}"|\\
% |latex -jobname cdocscl1 \|\\
% |  "\input{childdoc.def}\childdocforward[cdocsamp]{cdocsch1}"|\\
% |latex -jobname cdocscl2 \|\\
% |  "\def\version{final}\input{childdoc.def}\childdocforward{cdocsch2}"|
% \end{tabular}
% \end{center}
% Note that the trailing backslash on each first line
% merely continues the input to the second line
% (for convenient cut ant paste).
% Furthermore, the command |latex| can be replaced by any
% of its alternative versions such as |pdflatex|.
%
% %%%%%%%%%%%%%%%%%%%%%%%%%%%%%%%%%%%%%%%%%%%%%%%%%%%%%%%%%%%%%%%%%%%%%%%%%%%%%%
% %%%%%%%%%%%%%%%%%%%%%%%%%%%%%%%%%%%%%%%%%%%%%%%%%%%%%%%%%%%%%%%%%%%%%%%%%%%%%%
% \section{Implementation}
%\iffalse
%<*package>
%\fi
%
% This section describes the definitions file |childdoc.def|.

% The definitions cannot be loaded using |\usepackage| or |\RequirePackage|
% which has a mechanism to prevent loading a style file more than once.
% When loading the definitions by means of |\input|
% multiple instances have to be prevented manually:
%\iffalse
%This code needs to be before the `\ProvidesFile' directive
%which is defined at the beginning of this file.
%Therefore it is also placed there and commented out here.
%</package>
%<*discard>
%\fi
%    \begin{macrocode}
\ifdefined\childdocmain\endinput\fi
%    \end{macrocode}
%\iffalse
%</discard>
%<*package>
%\fi
%
% \macro{\ifchilddoc}
% \macro{\ifchilddocmanual}
% The conditional |\ifchilddoc| tells whether a
% child (true) or main (false) document is being compiled.
% The conditional |\ifchilddocmanual| tells whether
% the |\includeonly| mechanism is used (false) or
% the selection of child files must be performed manually (true).
% The definitions initialise to false:
%    \begin{macrocode}
\newif\ifchilddoc
\newif\ifchilddocmanual
%    \end{macrocode}

% \macro{\childdocname}
% \macro{\childdocjob}
% The macro |\childdocname| stores the name of the main document
% to be compiled. The macro |\childdocjob| stores the name of
% the document on which the \LaTeX{} compiler was originally invoked.
% The content of |\jobname| cannot be compared
% to filenames specified in the source due to different catcodes.
% The following code rescans |\jobname|, stores the result
% in |\childdocname| and saves a copy in |\childdocjob|:
%    \begin{macrocode}
\edef\childdocname{\scantokens\expandafter{\jobname\noexpand}}
\let\childdocjob\childdocname
%    \end{macrocode}

% \macro{\childdocdisable}
% The macro |\childdocdisable| prevents the main file
% from being processed more than once.
% At this stage, the main document command |\childdocmain|
% is assumed to be called once again where it should do nothing.
% Any subsequent call to it should prevent
% a secondary processing of the main document
% It overwrites the forwarding commands
% |\childdocof| and |\childdocforward|
% with empty macros to prevent further inclusions of the main document:
%    \begin{macrocode}
\newcommand{\childdocdisable}
{
  \renewcommand{\childdocmain}[1]{\renewcommand{\childdocmain}[1]{\endinput}}
  \renewcommand{\childdocof}[1]{}
  \renewcommand{\childdocby}[2][]{}
  \renewcommand{\childdocforward}[2][]{}
  \renewcommand{\childdocdisable}{}
}
%    \end{macrocode}

% \macro{\childdocmain}
% The macro |\childdocmain| is to be called at the top of the main file
% with nothing or the main filename (without extension) as argument.
% First, it breaks loops.
% If the argument is not empty and does not match |\childdocname|
% (which is set by the first inclusion of |childdoc.def|),
% |\ifchilddoc| is set to true, |\includeonly| is applied to the child file
% and |\jobname| is set to the main file
% (for proper handling of |.aux| files):
%    \begin{macrocode}
\newcommand{\childdocmain}[1]
{
  \childdocdisable\childdocmain{}
  \if?#1?\else
    \begingroup
      \def\childdoctmp{#1}
      \ifx\childdoctmp\childdocname
        \def\childdoctmp{}
      \else
        \def\childdoctmp
        {
          \childdoctrue
          \includeonly{\childdocname}
          \def\childdocjob{#1}
          \def\jobname{#1}
        }
      \fi
      \expandafter
    \endgroup
    \childdoctmp
  \fi
}
%    \end{macrocode}

% \macro{\childdocof}
% The command |\childdocof| redirects
% compilation to the main file |#1|.
%    \begin{macrocode}
\newcommand{\childdocof}[1]
{
  \childdocdisable
  \childdoctrue
  \includeonly{\childdocname}
  \def\jobname{#1}
  \def\childdocjob{#1}
  \input{#1}
}
%    \end{macrocode}

% \macro{\childdocby}
% The command |\childdocby| ....
%    \begin{macrocode}
\newcommand{\childdocby}[2][]
{
  \childdocdisable
  \childdoctrue
  \childdocmanualtrue
  \if?#1?\else
    \def\jobname{#2}
  \fi
  \def\childdocjob{#2}
  \input{#2}
  \endinput
}
%    \end{macrocode}

% \macro{\childdocforward}
% The command |\childdocforward| redirects
% compilation to the main file or
% (if the optional argument is given) a child file.
% Parameters are set as if the main file
% or a child file starting with |\childdocof| was compiled.
% Then compilation is handed over to the main file:
%    \begin{macrocode}
\newcommand{\childdocforward}[2][]
{
  \begingroup
    \if?#1?
      \def\childdoctmp
      {
        \def\childdocname{#2}
        \def\childdocjob{#2}
        \def\jobname{#2}
        \input{#2}
        \endinput
      }
    \else
      \def\childdoctmp
      {
        \childdocdisable
        \def\childdocname{#2}
        \childdoctrue
        \includeonly{#2}
        \def\childdocjob{#1}
        \def\jobname{#1}
        \input{#1}
        \endinput
      }
    \fi
    \expandafter
  \endgroup
  \childdoctmp
}
%    \end{macrocode}

% \macro{\childdocforwardprefix}
% The command |\childdocforwardprefix| redirects
% compilation to the main or a child file by means of a pattern.
% The prefix |#1| in the current filename is replaced by |#2|
% and the suffix of the current filename is kept
% (it is assumed that the filename does not contain the substring `|~~~|'
% which is used as a delimiter).
% Compilation is handed over to the new file by |\childdocforward|:
%    \begin{macrocode}
\newcommand{\childdocforwardprefix}[3][]
{
  \begingroup
    \def\childdocextract #2##1~~~{\def\childdoctmp{\childdocforward[#1]{#3##1}}}
    \expandafter\childdocextract\childdocname~~~
    \expandafter
  \endgroup
  \childdoctmp
}
%    \end{macrocode}

% \macro{\childdoc}
% The deprecated macro |\childdoc| is a legacy version of |\childdocmain|:
%    \begin{macrocode}
\newcommand{\childdoc}{\childdocmain}
%    \end{macrocode}

% \macro{\childdocredirect}
% The deprecated macro |\childdocredirect| is a legacy version
% of |\childdocforward| and |\childdocforwardprefix|:
%    \begin{macrocode}
\newcommand{\childdocredirect}[2][]
{
  \begingroup
    \if?#1?
      \def\childdoctmp{\childdocforward{#2}}
    \else
      \def\childdoctmp{\childdocforwardprefix{#1}{#2}}
    \fi
    \expandafter
  \endgroup
  \childdoctmp
}
%    \end{macrocode}

%\iffalse
%</package>
%\fi
%
\endinput

\childdocby{cdocsamp}
%    \end{macrocode}

%\iffalse
%</samplepart3|samplepart4>
%\fi
%
%\iffalse
%<*samplepart3>
%\fi
% Some text for part 3:
%    \begin{macrocode}
some text in part three
%    \end{macrocode}

%\iffalse
%</samplepart3>
%\fi
% Some text for part 4:
%\iffalse
%<*samplepart4>
%\fi
%    \begin{macrocode}
more text in part four
%    \end{macrocode}

%\iffalse
%</samplepart4>
%\fi
%
% %%%%%%%%%%%%%%%%%%%%%%%%%%%%%%%%%%%%%%
% \paragraph{Forwarding for a Complete Draft.}
%
% The following forwarding file |cdocsdrf.tex|
% compiles the main document in draft mode:
%\iffalse
%<*sampledraft>
%\fi
%    \begin{macrocode}
\def\version{draft}
% \iffalse
%
% childdoc.dtx Copyright (C) 2017-2018 Niklas Beisert
%
% This work may be distributed and/or modified under the
% conditions of the LaTeX Project Public License, either version 1.3
% of this license or (at your option) any later version.
% The latest version of this license is in
%   http://www.latex-project.org/lppl.txt
% and version 1.3 or later is part of all distributions of LaTeX
% version 2005/12/01 or later.
%
% This work has the LPPL maintenance status `maintained'.
%
% The Current Maintainer of this work is Niklas Beisert.
%
% This work consists of the files childdoc.dtx and childdoc.ins
% and the derived files childdoc.def and cdocsamp.tex with
% cdocsch1.tex, cdocsch2.tex, cdocsdrf.tex, cdocsfn1.tex, cdocsfn2.tex.
%
%<package>\ifdefined\childdocmain\endinput\fi
%<package>\ProvidesFile{childdoc.def}[2018/12/30 v2.0 child document driver]
%<samplemain>\ProvidesFile{cdocsamp.tex}[2018/12/30 v2.0 sample for childdoc]
%<*driver>
%\ProvidesFile{childdoc.drv}[2018/12/30 v2.0 childdoc reference manual file]
\PassOptionsToClass{10pt,a4paper}{article}
\documentclass{ltxdoc}

\usepackage[margin=35mm]{geometry}
\usepackage{hyperref}
\usepackage{hyperxmp}
\usepackage[usenames]{color}

\hypersetup{colorlinks=true}
\hypersetup{pdfstartview=FitH}
\hypersetup{pdfpagemode=UseNone}
\hypersetup{pdfsource={}}
\hypersetup{pdflang={en-UK}}
\hypersetup{pdfcopyright={Copyright 2017-2018 Niklas Beisert.
  This work may be distributed and/or modified under the
  conditions of the LaTeX Project Public License, either version 1.3
  of this license or (at your option) any later version.}}
\hypersetup{pdflicenseurl={http://www.latex-project.org/lppl.txt}}
\hypersetup{pdfcontactaddress={ETH Zurich, ITP, HIT K,
  Wolfgang-Pauli-Strasse 27}}
\hypersetup{pdfcontactpostcode={8093}}
\hypersetup{pdfcontactcity={Zurich}}
\hypersetup{pdfcontactcountry={Switzerland}}
\hypersetup{pdfcontactemail={nbeisert@itp.phys.ethz.ch}}
\hypersetup{pdfcontacturl={http://people.phys.ethz.ch/\xmptilde nbeisert/}}

\newcommand{\secref}[1]{\hyperref[#1]{section \ref*{#1}}}

\parskip1ex
\parindent0pt
\let\olditemize\itemize
\def\itemize{\olditemize\parskip0pt}

\begin{document}

\title{The \textsf{childdoc} Package}
\hypersetup{pdftitle={The childdoc Package}}
\author{Niklas Beisert\\[2ex]
  Institut f\"ur Theoretische Physik\\
  Eidgen\"ossische Technische Hochschule Z\"urich\\
  Wolfgang-Pauli-Strasse 27, 8093 Z\"urich, Switzerland\\[1ex]
  \href{mailto:nbeisert@itp.phys.ethz.ch}
  {\texttt{nbeisert@itp.phys.ethz.ch}}}
\hypersetup{pdfauthor={Niklas Beisert}}
\hypersetup{pdfsubject={Manual for the LaTeX2e Package childdoc}}
\date{30 December 2018, \textsf{v2.0}}
\maketitle

\begin{abstract}\noindent
\textsf{childdoc} is a \LaTeXe{} package
that enables the direct compilation
of document sections included by |\include|
to individual files.
\end{abstract}

\begingroup
\parskip0ex
\tableofcontents
\endgroup

%%%%%%%%%%%%%%%%%%%%%%%%%%%%%%%%%%%%%%%%%%%%%%%%%%%%%%%%%%%%%%%%%%%%%%%%%%%%%%%%
%%%%%%%%%%%%%%%%%%%%%%%%%%%%%%%%%%%%%%%%%%%%%%%%%%%%%%%%%%%%%%%%%%%%%%%%%%%%%%%%
\section{Introduction}

\LaTeX{} provides a mechanism to structure a large document (such as a book)
into a main file and several child files (containing the chapters)
using the |\include| command.
This mechanism is beneficial for documents
which span hundreds of pages in order to
make the source file(s) more manageable.
Moreover, compilation can be restricted to
selected child files by means of the |\includeonly| command.
The latter feature can be used to reduce the compilation time while editing
(this was significantly more useful in the earlier days of \LaTeX{})
or to generate a smaller document which is easier to navigate.
Another application of |\includeonly| is to generate
documents consisting of selected parts of the complete document.

However, there are a few drawbacks of the plain |\include| mechanism:
\begin{itemize}
\item
The child files cannot be compiled on their own,
they can only be compiled via the main file.
A naive editing environment
(such as a text editor with an option
to have the current file processed by \LaTeX)
may require one to switch to the main file before compiling;
attempting to compile the child file produces errors.
\item
The main file must be modified (each time)
to adjust the |\includeonly| command
to the present needs. This easily leaves the main file in a messy state.
\item
The generated document will always carry the filename
of the main document. This is inconvenient if
several child files are to be compiled and
to be kept for distribution.
\end{itemize}

The present package provides a simple interface
to make child files individually compilable by \LaTeX{}.
Compiling a child file then has the same effect as compiling
the main file with an |\includeonly| command
to select the appropriate child.
Moreover the generated document will carry the name of the child
rather than the main file.
This resolves all three above issues.

This feature is meant to make the editing of books,
thesis documents and lecture notes somewhat more convenient.
However, the package can also be used efficiently for
composing a series of documents (such as exercise sheets)
which are typically distributed individually.
It then assists the author in generating the individual documents
(potentially in different versions)
as well as a document containing the collected series.
Another application is in developing style files
or other kinds of included material
where compilation of the style file could redirect
to a sample or test file.

%%%%%%%%%%%%%%%%%%%%%%%%%%%%%%%%%%%%%%%%%%%%%%%%%%%%%%%%%%%%%%%%%%%%%%%%%%%%%%%%
%%%%%%%%%%%%%%%%%%%%%%%%%%%%%%%%%%%%%%%%%%%%%%%%%%%%%%%%%%%%%%%%%%%%%%%%%%%%%%%%
\section{Usage}

First of all, the package \textsf{childdoc} is \emph{not} a standard
\LaTeXe{} |.sty| style file! Therefore it needs to be invoked in
a non-standard way.

%%%%%%%%%%%%%%%%%%%%%%%%%%%%%%%%%%%%%%%%%%%%%%%%%%%%%%%%%%%%%%%%%%%%%%%%%%%%%%%%
\subsection{Included Files}
\label{sec:include}

%%%%%%%%%%%%%%%%%%%%%%%%%%%%%%%%%%%%%%%%
\DescribeMacro{\childdocmain}
To use the package, add the commands
\begin{center}
\begin{tabular}{l}
|\input{childdoc.def}|\\
|\childdocmain{}|\\
\end{tabular}
\end{center}
at the very top of the main \LaTeX{} file,
in particular \emph{before} the |\documentclass| statement!
The argument of |\childdocmain| should be left empty
(but it must be present).

%%%%%%%%%%%%%%%%%%%%%%%%%%%%%%%%%%%%%%%%
\DescribeMacro{\childdocof}
Furthermore, add the commands
\begin{center}
\begin{tabular}{l}
|\input{childdoc.def}|\\
|\childdocof{|\textit{main}|}|\\
\end{tabular}
\end{center}
at the top of every child file \textit{child}
which is included by |\include{|\textit{child}|}|
from within the main file
(or at least for those files to be compiled individually).
The argument \textit{main} must be the filename of the main file.

There are a couple of
considerations in setting up the main and child documents:

%%%%%%%%%%%%%%%%%%%%%%%%%%%%%%%%%%%%%%%%
\paragraph{Restrictions.}

Please note the following restrictions:
\begin{itemize}
\item
|\childdocmain| must be called with one argument \textit{main}
to ensure compatibility with earlier version of the package.
It must either be empty (|\childdocmain{}|)
or precisely match the filename of the main file in which it is specified.
See \secref{sec:detection} for further information.
\item
The filename \textit{main} must be specified without the |.tex| extension.
\item
The filename \textit{main} is case sensitive
(even in case-insensitive file systems)
due to internal string comparison.
\item
The argument \textit{main} should be fully expanded, it cannot be a macro.
\item
Subdirectories and special characters should be avoided in filenames.
\item
The command |\childdocmain{|\textit{main}|}| must be followed by a whitespace.
It should not be followed immediately by another command
or by a comment mark `|%|'.
This is because the \TeX{} parser reads the token immediately following
the argument of |\childdocmain| and puts it
at the beginning of every child section;
however, a white\-space is ignored.
\end{itemize}

%%%%%%%%%%%%%%%%%%%%%%%%%%%%%%%%%%%%%%%%
\paragraph{Content of Main File.}

It is advisable to place all content in the child files included by |\include|.
Any output contained in the main file will appear in all child documents
unless suppressed manually;
it cannot be suppressed automatically by the |\includeonly| directive
and thus should normally be avoided.
A method to include some content in the main file
by means of conditional processing is described in \secref{sec:conditional}.

%%%%%%%%%%%%%%%%%%%%%%%%%%%%%%%%%%%%%%%%
\paragraph{Page Numbering.}

When only a part of the document is compiled,
the appropriate numbering of pages
(as well as other status parameters)
is determined from the |.aux| files.
The latter contain information from previous passes.
However this information needs to propagate through
all intermediate child documents.
Therefore the page numbering in child documents may well
be inconsistent until the complete document is compiled at least once.

A useful (if unconventional) way to always ensure a consistent
page numbering is to restart the numbering in each child document
and denote the pages by `\textit{child}|.|\textit{page}'
where \textit{child} represents the chapter/section number of the child file.
This can be achieved by the command
|\numberwithin{page}{|\textit{child}|}|
of the \textsf{amsmath} package
where \textit{child} can be |chapter| or |section|
depending on the chosen structuring.
Alternatively, one can modify the macro |\thepage| appropriately
and reset the counter |page| at the start of each child file.

%%%%%%%%%%%%%%%%%%%%%%%%%%%%%%%%%%%%%%%%%%%%%%%%%%%%%%%%%%%%%%%%%%%%%%%%%%%%%%%%
\subsection{Conditional Processing}
\label{sec:conditional}

The package provides a mechanism to compile different versions
of a document. To customise the versions further some conditional processing
can come in handy to distinguish which version is being compiled.
The package provides two macros to describe the compilation context:

%%%%%%%%%%%%%%%%%%%%%%%%%%%%%%%%%%%%%%%%
\DescribeMacro{\ifchilddoc}
The conditional |\ifchilddoc| distinguishes between the compilation of
child documents and the main document:
%
\begin{center}
|\ifchilddoc |\textit{child-code}| |[|\||else |\textit{main-code}]| \||fi|
\end{center}

%%%%%%%%%%%%%%%%%%%%%%%%%%%%%%%%%%%%%%%%
\DescribeMacro{\childdocname}
\DescribeMacro{\childdocjob}
The macro |\childdocname| contains the filename (without extension)
of the main or child file being processed.
Note that |\childdocjob| will always contain the name of the main file.

%%%%%%%%%%%%%%%%%%%%%%%%%%%%%%%%%%%%%%%%
\paragraph{Title Page.}

Conditional processing can be used to include a title or banner page
in the main document when proper precautions are taken.
Importantly, the code in the main file should ensure that the page counter
(as well as other status parameters which are stored in the |.aux| files)
takes the same value after the conditional processing.
Otherwise the page numbers may take divergent values
depending on which part is compiled.

For example, a title page could be declared by:
%
\begin{center}
\begin{tabular}{l}
|\ifchilddoc\||else|\\
|\addtocounter{page}{-1}|\\
\textit{code for title page}\\
|\newpage|\\
|\||fi|
\end{tabular}
\end{center}
%
A banner page for the child documents can be generated by:
%
\begin{center}
\begin{tabular}{l}
|\ifchilddoc|\\
|\addtocounter{page}{-1}|\\
\textit{code for banner page}\\
|\newpage|\\
|\||fi|
\end{tabular}
\end{center}
%
Here one could write a message such as:
\begin{center}
|This is the part \childdocname{} of \childdocjob{}.|
\end{center}

%%%%%%%%%%%%%%%%%%%%%%%%%%%%%%%%%%%%%%%%%%%%%%%%%%%%%%%%%%%%%%%%%%%%%%%%%%%%%%%%
\subsection{Flags}
\label{sec:flags}

The package makes it easy to generate different versions
of the main or child documents.
To this end compilation flags can be defined
and assigned different default values.
They will be particularly useful in conjunction
with the forwarding mechanism described in \secref{sec:forward}.

For example, it may be useful to have a flag |\version|
which can be set to |draft| or |final|.
The document source will contain some conditional code
depending on the value of |\version|.
Suppose further, the flag should default to |final| for the main file
and to |draft| for child files
which is a natural assignment for editing the document.
This is achieved by placing the following code
in the preamble of the main document
(below the |\childdocmain| directive):
%
\begin{center}
\begin{tabular}{l}
|\ifchilddoc|\\
|\providecommand{\version}{draft}|\\
|\||else|\\
|\providecommand{\version}{final}|\\
|\||fi|
\end{tabular}
\end{center}
%
The definition by |\providecommand| makes sure
that previous definitions are not overwritten.
Further statements |\providecommand{\version}{...}|
can thus be added before the above code to override it.

For the main file, one might add a line
(between |\childdocmain| and the above block)
%
\begin{center}
|%\ifchilddoc\||else\providecommand{\version}{draft}\||fi|
\end{center}
%
which can be uncommented to produce a draft version.
Likewise one can add a line to the very top of a child file
(above the |\childdocof{|\textit{main}|}| directive)
%
\begin{center}
|%\providecommand{\version}{final}|
\end{center}
%
which can be uncommented to produce the final version of this child document.

%%%%%%%%%%%%%%%%%%%%%%%%%%%%%%%%%%%%%%%%%%%%%%%%%%%%%%%%%%%%%%%%%%%%%%%%%%%%%%%%
\subsection{Forwarding}
\label{sec:forward}

Different versions of the main or child documents
using compilation flags as described in \secref{sec:flags}
can be (permanently) stored in different files
for convenient compilation, viewing and distribution.
To this end, the package defines a command
to pass on compilation to a different file:

%%%%%%%%%%%%%%%%%%%%%%%%%%%%%%%%%%%%%%%%
\DescribeMacro{\childdocforward}
The command |\childdocforward| redirects processing to
another source file:
%
\begin{center}
\begin{tabular}{l}
|\input{childdoc.def}|\\
|\childdocforward[|\textit{main}|]{|\textit{dest}|}|\\
\end{tabular}
\end{center}
%
The argument \textit{dest} is the destination file
(without extension).
It should be the main file or one of the child files.
Note that further \textsf{childdoc} directives
such as |\childdocof| and |\childdocforward|
in the indicated file will be processed in this form.
The optional argument \textit{main}
passes on directly to the main file \textit{main}
while pretending to compile the child \textit{dest}.
This form behaves as if \textit{dest}
issues |\childdocof{|\textit{main}|}| right away,
and no further \textsf{childdoc} directives will be processed.

%%%%%%%%%%%%%%%%%%%%%%%%%%%%%%%%%%%%%%%%
\DescribeMacro{\...prefix}
In the alternative form |\childdocforwardprefix|,
%
\begin{center}
\begin{tabular}{l}
|\input{childdoc.def}|\\
|\childdocforwardprefix[|\textit{main}|]{|\textit{prefix}|}{|\textit{dest}|}|
\end{tabular}
\end{center}
%
the destination file is determined by a pattern
depending on the current file:
To make this work, the current file must be called
`{\textit{prefix}\hspace{0.2em}\textit{suffix}}'
with \textit{prefix} matching precisely the argument.
Processing is then passed on to the file
`{\textit{dest}\hspace{0.2em}\textit{suffix}}'.
Surely, the same effect is achieved by
directly specifying the
argument `{\textit{dest}\hspace{0.2em}\textit{suffix}}'
in the first form.
However, that requires to set up a different file
for each child. With the alternative form of the command
all these files can have exactly the same content
which simplifies setting them up and maintaining them.

For example, the following file |draft.tex|
with a compilation flag |\version| as described in \secref{sec:flags}
compiles the main document as a draft:
%
\begin{center}
\begin{tabular}{l}
|\def\version{draft}|\\
|\input{childdoc.def}|\\
|\childdocforward{|\textit{main}|}|
\end{tabular}
\end{center}
%
Likewise, the following files |final|\textit{nn}|.tex|
compile the final version of the child document
|child|\textit{nn}|.tex|:
%
\begin{center}
\begin{tabular}{l}
|\def\version{final}|\\
|\input{childdoc.def}|\\
|\childdocforwardprefix{final}{child}|
\end{tabular}
\end{center}
%

Note that when several versions of a main file and/or of each child file
are to be generated, it may be convenient to set up a |Makefile| or
shell script to automatise the process.

%%%%%%%%%%%%%%%%%%%%%%%%%%%%%%%%%%%%%%%%%%%%%%%%%%%%%%%%%%%%%%%%%%%%%%%%%%%%%%%%
\subsection{Command Line Processing}
\label{sec:commandline}

The effect of redirection files can also be achieved by invoking
the \LaTeX{} compiler with a more elaborate command line.
Most conveniently this should be done as part
of a shell script or a |Makefile|.

When using \textsf{childdoc} in the main file, the following
command lines effectively perform a redirection
(note that depending on the shell being used,
backslashes may have to be doubled: `|\|' $\to$ `|\\|'):
%
\begin{center}
|... -jobname "|\textit{target}|" |\\|"|[\textit{flags}]%
|\input{childdoc.def}\childdocforward[|\textit{main}|]{|\textit{dest}|}"|
\end{center}
%
Here \textit{target} is the name of the output file,
\textit{main} is the name of the main file
and \textit{dest} is the name of the main or child file to be processed
(all filenames without extensions).
The optional argument \textit{main} can be omitted
if \textit{main} matches \textit{dest}.
Optionally, compilation \textit{flags} can be defined via |\def| commands.
This command line makes the \TeX{} engine believe
it is compiling the file \textit{target}
whose content is specified as the latter parameter.
The provided code then forwards the processing to
\textit{main} or \textit{dest} as described in \secref{sec:forward}.

%%%%%%%%%%%%%%%%%%%%%%%%%%%%%%%%%%%%%%%%%%%%%%%%%%%%%%%%%%%%%%%%%%%%%%%%%%%%%%%%
\subsection{Include by Input}
\label{sec:input}

Including child documents by |\include| has some restrictions by design.
Most notably, the content of a child document always occupies
its own set of pages; pages cannot be shared between child documents.
Usually, this behaviour makes perfect sense
because each child document contain an essential part of the document.
However, in some situations it may be desirable to compose
a document from a collection of parts
without having mandatory page breaks between then.
For this case, the package
provides a mechanism to include parts
by |\input| which can also be processed individually.
However, by construction this mechanism
requires manual handling of the content to be output.

%%%%%%%%%%%%%%%%%%%%%%%%%%%%%%%%%%%%%%%%
\DescribeMacro{\ifchilddocmanual}
The main file should be prepared as usual, see \secref{sec:include}.
However, the document body must make a distinction
between processing of an individual part and of the main document, e.g.:
%
\begin{center}
\begin{tabular}{l}
|\ifchilddocmanual|\\
|\input{\childdocname}|\\
|\||else|\\
\textit{document body with }|\input{|\textit{part}|}|\\
|\||fi|
\end{tabular}
\end{center}
%
The conditional |\ifchilddocmanual| is true whenever
a part to be included by |\input| is being compiled,
and the name of the part is stored in |\childdocname|.

%%%%%%%%%%%%%%%%%%%%%%%%%%%%%%%%%%%%%%%%
\DescribeMacro{\childdocby}
Each part to be included by |\input| should start with:
%
\begin{center}
\begin{tabular}{l}
|\input{childdoc.def}|\\
|\childdocby{|\textit{main}|}|\\
\end{tabular}
\end{center}
%
The directive |\childdocby| is similar to |\childdocof|
described in \secref{sec:include},
but the subsequent selection of content must be done manually.
To that end, both |\ifchilddoc| and |\ifchilddocmanual|
will be true upon processing of a part,
and the name of the part is stored in |\childdocname|.
Note that |\jobname| will be set to the filename of the current part
so that each part receives an individual |.aux| file
that does not interfere with the |.aux| file(s) of the main document.
This behaviour can be altered by the alternative form
|\childdocby[*]{|\textit{main}|}| (with a non-empty optional argument)
which uses the |.aux| file of the main document
by setting |\jobname| to \textit{main}.

%%%%%%%%%%%%%%%%%%%%%%%%%%%%%%%%%%%%%%%%%%%%%%%%%%%%%%%%%%%%%%%%%%%%%%%%%%%%%%%%
\subsection{Driver Development}
\label{sec:driver}

The \textsf{childdoc} mechanism can also be use for the development
of definition files such as \LaTeX{} styles or classes.
This case differs from the above setup with multiple parts
included by |\include| in that no |\includeonly| should be invoked.
This can be achieved by starting the include file
(before |\ProvidesPackage|) with:
%
\begin{center}
\begin{tabular}{l}
|\input{childdoc.def}|\\
|\childdocforward{|\textit{main}|}|\\
\end{tabular}
\end{center}
%
or alternatively with:
%
\begin{center}
\begin{tabular}{l}
|\input{childdoc.def}|\\
|\childdocby{|\textit{main}|}|\\
\end{tabular}
\end{center}
%
Both forms have slightly different effects as described above.
The main file is prepared as usual, see \secref{sec:include}.

%%%%%%%%%%%%%%%%%%%%%%%%%%%%%%%%%%%%%%%%%%%%%%%%%%%%%%%%%%%%%%%%%%%%%%%%%%%%%%%%
\subsection{Legacy Detection}
\label{sec:detection}

The directive |\childdocmain| in the main file can detect
whether the complete document or merely a child is to be compiled
even without using the directive |\childdocof|.
This method is deprecated because it is less robust
and there is no compelling reason to use it;
it is merely provided for backward compatibility
and it may be removed in future versions.

If the detection mechanism is to be used,
it is mandatory to correctly specify
the filename of the main file as the argument of |\childdocmain|:
%
\begin{center}
\begin{tabular}{l}
|\input{childdoc.def}|\\
|\childdocmain{|\textit{main}|}|\\
\end{tabular}
\end{center}
%
If |\jobname| does not match the argument \textit{main} of |\childdocmain|,
it is assumed that |\jobname| points to the child file to be compiled.
When using |\childdocmain| with the main file specified as argument,
it suffices to start a child file
with just |\input{|\textit{main}|}|
without loading of the package and using |\childdocof|.
If instead all processing is done
with the appropriate \textsf{childdoc} directives,
the argument of \textit{main} of |\childdocmain| can be empty.

An alternative version of the command line processing described
in \secref{sec:commandline} using the detection mechanism reads:
%
\begin{center}
|... -jobname "|\textit{target}|" "|[\textit{flags}]%
[|\def\jobname{|\textit{dest}|}|]|\input{|\textit{main}|}"|
\end{center}

%%%%%%%%%%%%%%%%%%%%%%%%%%%%%%%%%%%%%%%%%%%%%%%%%%%%%%%%%%%%%%%%%%%%%%%%%%%%%%%%
\subsection{Manual Code}
\label{sec:manual}

In case one cannot be certain whether the definitions file |childdoc.def|
is installed on the target \TeX{} distribution
and one prefers not to ship it,
it is conceivable to paste a few relevant commands into the sources.

To that end, drop all statements |\input{childdoc.def}|
and perform the replacements as outlined below.
Instead of |\childdocmain{|\textit{main}|}| add the following code
to the top of the main file:
%
\begin{center}
\begin{tabular}{l}
|\||ifdefined\childdocname\endinput\||fi\newif\ifchilddoc|\\
|\edef\childdocname{\scantokens\expandafter{\jobname\noexpand}}|\\
|\def\childdocmain{|\textit{main}|}\||ifx\childdocmain\childdocname\||else|\\
|\childdoctrue\includeonly{\childdocname}\let\jobname\childdocmain\||fi|\\
\end{tabular}
\end{center}
%
Instead of |\childdocof{|\textit{main}|}| just include the main file
at the top of each child file:
%
\begin{center}
|\input{|\textit{main}|}|
\end{center}
%
A simple redirection |\childdocforward{|\textit{dest}|}| is achieved by:
%
\begin{center}
|\def\jobname{|\textit{dest}|}\input{\jobname}|
\end{center}
%
The redirection with prefix
|\childdocforwardprefix[|\textit{prefix}|]{|\textit{dest}|}|
is accomplished by:
%
\begin{center}
\begin{tabular}{l}
|{\edef\jobname{\scantokens\expandafter{\jobname\noexpand}}|\\
|\def\redirectjob |\textit{prefix}|#1~~~{\gdef\jobname{|\textit{dest}|#1}}|\\
|\expandafter\redirectjob\jobname~~~}\input{\jobname}|
\end{tabular}
\end{center}

In an alternative approach,
child documents can be compiled by a specific command line
without additional code or specific definitions:
%
\begin{center}
|... -jobname "|\textit{target}|" "|[\textit{flags}]%
|\includeonly{|\textit{dest}|}\input{|\textit{main}|}"|
\end{center}
%

%%%%%%%%%%%%%%%%%%%%%%%%%%%%%%%%%%%%%%%%%%%%%%%%%%%%%%%%%%%%%%%%%%%%%%%%%%%%%%%%
%%%%%%%%%%%%%%%%%%%%%%%%%%%%%%%%%%%%%%%%%%%%%%%%%%%%%%%%%%%%%%%%%%%%%%%%%%%%%%%%
\section{Information}

%%%%%%%%%%%%%%%%%%%%%%%%%%%%%%%%%%%%%%%%%%%%%%%%%%%%%%%%%%%%%%%%%%%%%%%%%%%%%%%%
\subsection{Copyright}

Copyright \copyright{} 2017--2018 Niklas Beisert

This work may be distributed and/or modified under the
conditions of the \LaTeX{} Project Public License, either version 1.3
of this license or (at your option) any later version.
The latest version of this license is in
  \url{http://www.latex-project.org/lppl.txt}
and version 1.3 or later is part of all distributions of \LaTeX{}
version 2005/12/01 or later.

This work has the LPPL maintenance status `maintained'.

The Current Maintainer of this work is Niklas Beisert.

This work consists of the files |README.txt|, |childdoc.ins| and |childdoc.dtx|
as well as the derived files |childdoc.def|, |cdocsamp.tex|
with |cdocsch1.tex|, |cdocsch2.tex|, |cdocspt3.tex|, |cdocspt4.tex|,
|cdocsdrf.tex|, |cdocsfn1.tex|, |cdocsfn2.tex|
as well as |childdoc.pdf|.

%%%%%%%%%%%%%%%%%%%%%%%%%%%%%%%%%%%%%%%%%%%%%%%%%%%%%%%%%%%%%%%%%%%%%%%%%%%%%%%%
\subsection{Files and Installation}

The package consists of the files:
%
\begin{center}
\begin{tabular}{ll}
    |README.txt|   & readme file \\
    |childdoc.ins| & installation file \\
    |childdoc.dtx| & source file \\
    |childdoc.def| & definition file \\
    |cdocsamp.tex| & sample main file \\
    |cdocsch1.tex| & sample include file \\
    |cdocsch2.tex| & sample include file \\
    |cdocspt3.tex| & sample part file \\
    |cdocspt4.tex| & sample part file \\
    |cdocsdrf.tex| & sample redirection file \\
    |cdocsfn1.tex| & sample redirection file \\
    |cdocsfn2.tex| & sample redirection file \\
    |childdoc.pdf| & manual
\end{tabular}
\end{center}
%
The distribution consists of the files
|README.txt|, |childdoc.ins| and |childdoc.dtx|.
%
\begin{itemize}
\item
Run (pdf)\LaTeX{} on |childdoc.dtx|
to compile the manual |childdoc.pdf| (this file).
\item
Run \LaTeX{} on |childdoc.ins| to create the definitions file |childdoc.def|
and the sample |cdocsamp.tex| with include files
|cdocsch1.tex|, |cdocsch2.tex|, |cdocspt3.tex|, |cdocspt4.tex|,
|cdocsdrf.tex|, |cdocsfn1.tex|, |cdocsfn2.tex|.
Then copy the file |childdoc.def| to an appropriate directory of your \LaTeX{}
distribution, e.g.\ \textit{texmf-root}|/tex/latex/childdoc|.
\end{itemize}

%%%%%%%%%%%%%%%%%%%%%%%%%%%%%%%%%%%%%%%%%%%%%%%%%%%%%%%%%%%%%%%%%%%%%%%%%%%%%%%%
\subsection{Related CTAN Packages}

There are several other packages which offer a similar functionality:
%
\begin{itemize}
\item
The packages
\href{http://ctan.org/pkg/docmute}{\textsf{docmute}},
\href{http://ctan.org/pkg/includex}{\textsf{includex}} and
\href{http://ctan.org/pkg/standalone}{\textsf{standalone}}
provide commands to include only the document body of
a child file thus allowing both files to be compiled individually.
\item
The packages \href{http://ctan.org/pkg/subdocs}{\textsf{subdocs}}
and \href{http://ctan.org/pkg/subfiles}{\textsf{subfiles}}
provide structures in which the main and child documents can be
encapsulated and allowing them to be compiled individually.
The inclusion mechanism is different from the conventional |\include|.
\item
The package \href{http://ctan.org/pkg/combine}{\textsf{combine}}
is an elaborate solution to combine several documents into one.
\end{itemize}
%
See also the CTAN topic \href{http://ctan.org/topic/subdocs}{\textsf{subdocs}}
for further related packages.
The present package differs from the above solutions in that
a document structure constructed with the conventional |\include| mechanism
just needs two extra commands at the top of every file
such that all constituent files can be compiled individually.

%%%%%%%%%%%%%%%%%%%%%%%%%%%%%%%%%%%%%%%%%%%%%%%%%%%%%%%%%%%%%%%%%%%%%%%%%%%%%%%%
%\subsection{Feature Suggestions}
%
%The following is a list of features which may be useful for future
%versions of this package:
%%
%\begin{itemize}
%\item
%\ldots
%\end{itemize}

%%%%%%%%%%%%%%%%%%%%%%%%%%%%%%%%%%%%%%%%%%%%%%%%%%%%%%%%%%%%%%%%%%%%%%%%%%%%%%%%
\subsection{Revision History}

%%%%%%%%%%%%%%%%%%%%%%%%%%%%%%%%%%%%%%%%
\paragraph{v2.0:} 2018/12/30

\begin{itemize}
\item
immediate forward processing
\item
added |\childdocby| mechanism
\item
manual restructured
\end{itemize}

%%%%%%%%%%%%%%%%%%%%%%%%%%%%%%%%%%%%%%%%
\paragraph{v1.6:} 2018/01/17

\begin{itemize}
\item
application for development of include files
\item
corrections to manual
\end{itemize}

%%%%%%%%%%%%%%%%%%%%%%%%%%%%%%%%%%%%%%%%
\paragraph{v1.5:} 2017/05/21

\begin{itemize}
\item
more complete structuring introduced
\item
|\childdocof| introduced
\item
|\childdoc| renamed to |\childdocmain|
\item
|\childredirect| renamed to |\childdocforward| and |\childdocforwardprefix|
and functionality expanded
\end{itemize}

%%%%%%%%%%%%%%%%%%%%%%%%%%%%%%%%%%%%%%%%
\paragraph{v1.0:} 2017/04/27

\begin{itemize}
\item
manual and install package
\item
first version published on CTAN
\end{itemize}

%%%%%%%%%%%%%%%%%%%%%%%%%%%%%%%%%%%%%%%%
\paragraph{v0.6:} 2017/04/26

\begin{itemize}
\item
redirection mechanism added
\end{itemize}

%%%%%%%%%%%%%%%%%%%%%%%%%%%%%%%%%%%%%%%%
\paragraph{v0.5:} 2017/04/26

\begin{itemize}
\item
functionality in definition file
\end{itemize}


%%%%%%%%%%%%%%%%%%%%%%%%%%%%%%%%%%%%%%%%%%%%%%%%%%%%%%%%%%%%%%%%%%%%%%%%%%%%%%%%
%%%%%%%%%%%%%%%%%%%%%%%%%%%%%%%%%%%%%%%%%%%%%%%%%%%%%%%%%%%%%%%%%%%%%%%%%%%%%%%%
%%%%%%%%%%%%%%%%%%%%%%%%%%%%%%%%%%%%%%%%%%%%%%%%%%%%%%%%%%%%%%%%%%%%%%%%%%%%%%%%
\appendix

\settowidth\MacroIndent{\rmfamily\scriptsize 000\ }

 \DocInput{childdoc.dtx}

\end{document}
%</driver>
% \fi
%
% %%%%%%%%%%%%%%%%%%%%%%%%%%%%%%%%%%%%%%%%%%%%%%%%%%%%%%%%%%%%%%%%%%%%%%%%%%%%%%
% %%%%%%%%%%%%%%%%%%%%%%%%%%%%%%%%%%%%%%%%%%%%%%%%%%%%%%%%%%%%%%%%%%%%%%%%%%%%%%
% \section{Sample}
%\iffalse
%<*samplemain>
%\fi
%
% The following presents a sample document
% with two chapters, two parts, a title page,
% a compile flag as well as three forwarding files to set the flag.
% It consists of eight |.tex| files:
% \begin{center}
% \begin{tabular}{ll}
% |cdocsamp.tex|&main file\\
% |cdocsch1.tex|&include file for chapter 1\\
% |cdocsch2.tex|&include file for chapter 2\\
% |cdocspt3.tex|&include file for part 3\\
% |cdocspt4.tex|&include file for part 4\\
% |cdocsdrf.tex|&forwarding file for main file in draft mode\\
% |cdocsfi1.tex|&forwarding file for final version of chapter 1\\
% |cdocsfi2.tex|&forwarding file for final version of chapter 2\\
% \end{tabular}
% \end{center}
% Each of the eight files can be compiled directly by the \LaTeX{} compiler.
%
% %%%%%%%%%%%%%%%%%%%%%%%%%%%%%%%%%%%%%%
% \paragraph{Main File.}
%
% The main file is called |cdocsamp.tex|.
%
% Load the \textsf{childdoc} definitions and
% declare the filename for the main document:
%    \begin{macrocode}
\input{childdoc.def}
\childdocmain{}
%    \end{macrocode}

% Optional override for |\version| flag:
%    \begin{macrocode}
%%\ifchilddoc\else\providecommand{\version}{draft}\fi
%    \end{macrocode}

% Define the default values for the |\version| flag
% (|final| for the main file and |draft| for childs):
%    \begin{macrocode}
\ifchilddoc
\providecommand{\version}{draft}
\else
\providecommand{\version}{final}
\fi
%    \end{macrocode}

% Load the standard document class:
%    \begin{macrocode}
\documentclass[12pt]{article}
%    \end{macrocode}

% Start the document body:
%    \begin{macrocode}
\begin{document}
%    \end{macrocode}

% Declare a title page.
% Print title, part of document being processed and version flag:
%    \begin{macrocode}
\addtocounter{page}{-1}
\begin{center}
{\LARGE\bfseries{}childdoc example\par}
\vspace{1cm}
\ifchilddoc
\ifchilddocmanual part\else chapter\fi:
`\childdocname' of `\childdocjob'\par
\else
main document: `\childdocjob'\par
\fi
version: \version\par
\end{center}
\newpage
%    \end{macrocode}

% Manually include selected file,
% otherwise process as usual:
%    \begin{macrocode}
\ifchilddocmanual
\section*{part `\childdocname'}
\input{\childdocname}
\else
%    \end{macrocode}

% Include the two chapters:
%    \begin{macrocode}
\include{cdocsch1}
\include{cdocsch2}
%    \end{macrocode}

% Include the two parts unless only chapters should be displayed:
%    \begin{macrocode}
\ifchilddoc\else
\section{part three}
\input{cdocspt3}
\section{part four}
\input{cdocspt4}
\fi
%    \end{macrocode}

% Process as usual until here:
%    \begin{macrocode}
\fi
%    \end{macrocode}

% End of document body:
%    \begin{macrocode}
\end{document}
%    \end{macrocode}
%\iffalse
%</samplemain>
%\fi
%
% %%%%%%%%%%%%%%%%%%%%%%%%%%%%%%%%%%%%%%
% \paragraph{Chapter Include Files.}
%
% The include files are called |cdocsch1.tex| and |cdocsch2.tex|.
%
%\iffalse
%<*samplechap1|samplechap2>
%\fi

% Optional override for |\version| flag:
%    \begin{macrocode}
%%\providecommand{\version}{final}
%    \end{macrocode}

% Include the main document:
%    \begin{macrocode}
\input{childdoc.def}
\childdocof{cdocsamp}
%    \end{macrocode}

%\iffalse
%</samplechap1|samplechap2>
%\fi
%
%\iffalse
%<*samplechap1>
%\fi
% Some text for chapter 1:
%    \begin{macrocode}
\section{one}
some text in chapter one
%    \end{macrocode}

%\iffalse
%</samplechap1>
%\fi
% Some text for chapter 2:
%\iffalse
%<*samplechap2>
%\fi
%    \begin{macrocode}
\section{two}
more text in chapter two
%    \end{macrocode}

%\iffalse
%</samplechap2>
%\fi
%
% %%%%%%%%%%%%%%%%%%%%%%%%%%%%%%%%%%%%%%
% \paragraph{Part Include Files.}
%
% The include files are called |cdocspt3.tex| and |cdocspt4.tex|.
%
%\iffalse
%<*samplepart3|samplepart4>
%\fi

% Optional override for |\version| flag:
%    \begin{macrocode}
%%\providecommand{\version}{final}
%    \end{macrocode}

% Include the main document:
%    \begin{macrocode}
\input{childdoc.def}
\childdocby{cdocsamp}
%    \end{macrocode}

%\iffalse
%</samplepart3|samplepart4>
%\fi
%
%\iffalse
%<*samplepart3>
%\fi
% Some text for part 3:
%    \begin{macrocode}
some text in part three
%    \end{macrocode}

%\iffalse
%</samplepart3>
%\fi
% Some text for part 4:
%\iffalse
%<*samplepart4>
%\fi
%    \begin{macrocode}
more text in part four
%    \end{macrocode}

%\iffalse
%</samplepart4>
%\fi
%
% %%%%%%%%%%%%%%%%%%%%%%%%%%%%%%%%%%%%%%
% \paragraph{Forwarding for a Complete Draft.}
%
% The following forwarding file |cdocsdrf.tex|
% compiles the main document in draft mode:
%\iffalse
%<*sampledraft>
%\fi
%    \begin{macrocode}
\def\version{draft}
\input{childdoc.def}
\childdocforward{cdocsamp}
%    \end{macrocode}

%\iffalse
%</sampledraft>
%\fi
%
% %%%%%%%%%%%%%%%%%%%%%%%%%%%%%%%%%%%%%%
% \paragraph{Forwarding for Final Version of the Chapters.}
%
% The following forwarding files |cdocsfn1.tex| and |cdocsfn2.tex|
% (with identical content)
% compile the final versions of the child documents
% |cdocsch1.tex| and |cdocsch2.tex|, respectively:
%\iffalse
%<*samplefinal>
%\fi
%    \begin{macrocode}
\def\version{final}
\input{childdoc.def}
\childdocforwardprefix[cdocsamp]{cdocsfn}{cdocsch}
%    \end{macrocode}

%\iffalse
%</samplefinal>
%\fi
%
% %%%%%%%%%%%%%%%%%%%%%%%%%%%%%%%%%%%%%%
% \paragraph{Command Line Processing.}
%
% The following three command lines generate the output files
% |cdocscld|, |cdocscl1| and |cdocscl2|
% which should be identical to
% |cdocsdrf|, |cdocsch1| and |cdocsfn2|, respectively:
% \begin{center}
% \begin{tabular}{l}
% |latex -jobname cdocscld \|\\
% |  "\def\version{draft}\input{childdoc.def}\childdocforward{cdocsamp}"|\\
% |latex -jobname cdocscl1 \|\\
% |  "\input{childdoc.def}\childdocforward[cdocsamp]{cdocsch1}"|\\
% |latex -jobname cdocscl2 \|\\
% |  "\def\version{final}\input{childdoc.def}\childdocforward{cdocsch2}"|
% \end{tabular}
% \end{center}
% Note that the trailing backslash on each first line
% merely continues the input to the second line
% (for convenient cut ant paste).
% Furthermore, the command |latex| can be replaced by any
% of its alternative versions such as |pdflatex|.
%
% %%%%%%%%%%%%%%%%%%%%%%%%%%%%%%%%%%%%%%%%%%%%%%%%%%%%%%%%%%%%%%%%%%%%%%%%%%%%%%
% %%%%%%%%%%%%%%%%%%%%%%%%%%%%%%%%%%%%%%%%%%%%%%%%%%%%%%%%%%%%%%%%%%%%%%%%%%%%%%
% \section{Implementation}
%\iffalse
%<*package>
%\fi
%
% This section describes the definitions file |childdoc.def|.

% The definitions cannot be loaded using |\usepackage| or |\RequirePackage|
% which has a mechanism to prevent loading a style file more than once.
% When loading the definitions by means of |\input|
% multiple instances have to be prevented manually:
%\iffalse
%This code needs to be before the `\ProvidesFile' directive
%which is defined at the beginning of this file.
%Therefore it is also placed there and commented out here.
%</package>
%<*discard>
%\fi
%    \begin{macrocode}
\ifdefined\childdocmain\endinput\fi
%    \end{macrocode}
%\iffalse
%</discard>
%<*package>
%\fi
%
% \macro{\ifchilddoc}
% \macro{\ifchilddocmanual}
% The conditional |\ifchilddoc| tells whether a
% child (true) or main (false) document is being compiled.
% The conditional |\ifchilddocmanual| tells whether
% the |\includeonly| mechanism is used (false) or
% the selection of child files must be performed manually (true).
% The definitions initialise to false:
%    \begin{macrocode}
\newif\ifchilddoc
\newif\ifchilddocmanual
%    \end{macrocode}

% \macro{\childdocname}
% \macro{\childdocjob}
% The macro |\childdocname| stores the name of the main document
% to be compiled. The macro |\childdocjob| stores the name of
% the document on which the \LaTeX{} compiler was originally invoked.
% The content of |\jobname| cannot be compared
% to filenames specified in the source due to different catcodes.
% The following code rescans |\jobname|, stores the result
% in |\childdocname| and saves a copy in |\childdocjob|:
%    \begin{macrocode}
\edef\childdocname{\scantokens\expandafter{\jobname\noexpand}}
\let\childdocjob\childdocname
%    \end{macrocode}

% \macro{\childdocdisable}
% The macro |\childdocdisable| prevents the main file
% from being processed more than once.
% At this stage, the main document command |\childdocmain|
% is assumed to be called once again where it should do nothing.
% Any subsequent call to it should prevent
% a secondary processing of the main document
% It overwrites the forwarding commands
% |\childdocof| and |\childdocforward|
% with empty macros to prevent further inclusions of the main document:
%    \begin{macrocode}
\newcommand{\childdocdisable}
{
  \renewcommand{\childdocmain}[1]{\renewcommand{\childdocmain}[1]{\endinput}}
  \renewcommand{\childdocof}[1]{}
  \renewcommand{\childdocby}[2][]{}
  \renewcommand{\childdocforward}[2][]{}
  \renewcommand{\childdocdisable}{}
}
%    \end{macrocode}

% \macro{\childdocmain}
% The macro |\childdocmain| is to be called at the top of the main file
% with nothing or the main filename (without extension) as argument.
% First, it breaks loops.
% If the argument is not empty and does not match |\childdocname|
% (which is set by the first inclusion of |childdoc.def|),
% |\ifchilddoc| is set to true, |\includeonly| is applied to the child file
% and |\jobname| is set to the main file
% (for proper handling of |.aux| files):
%    \begin{macrocode}
\newcommand{\childdocmain}[1]
{
  \childdocdisable\childdocmain{}
  \if?#1?\else
    \begingroup
      \def\childdoctmp{#1}
      \ifx\childdoctmp\childdocname
        \def\childdoctmp{}
      \else
        \def\childdoctmp
        {
          \childdoctrue
          \includeonly{\childdocname}
          \def\childdocjob{#1}
          \def\jobname{#1}
        }
      \fi
      \expandafter
    \endgroup
    \childdoctmp
  \fi
}
%    \end{macrocode}

% \macro{\childdocof}
% The command |\childdocof| redirects
% compilation to the main file |#1|.
%    \begin{macrocode}
\newcommand{\childdocof}[1]
{
  \childdocdisable
  \childdoctrue
  \includeonly{\childdocname}
  \def\jobname{#1}
  \def\childdocjob{#1}
  \input{#1}
}
%    \end{macrocode}

% \macro{\childdocby}
% The command |\childdocby| ....
%    \begin{macrocode}
\newcommand{\childdocby}[2][]
{
  \childdocdisable
  \childdoctrue
  \childdocmanualtrue
  \if?#1?\else
    \def\jobname{#2}
  \fi
  \def\childdocjob{#2}
  \input{#2}
  \endinput
}
%    \end{macrocode}

% \macro{\childdocforward}
% The command |\childdocforward| redirects
% compilation to the main file or
% (if the optional argument is given) a child file.
% Parameters are set as if the main file
% or a child file starting with |\childdocof| was compiled.
% Then compilation is handed over to the main file:
%    \begin{macrocode}
\newcommand{\childdocforward}[2][]
{
  \begingroup
    \if?#1?
      \def\childdoctmp
      {
        \def\childdocname{#2}
        \def\childdocjob{#2}
        \def\jobname{#2}
        \input{#2}
        \endinput
      }
    \else
      \def\childdoctmp
      {
        \childdocdisable
        \def\childdocname{#2}
        \childdoctrue
        \includeonly{#2}
        \def\childdocjob{#1}
        \def\jobname{#1}
        \input{#1}
        \endinput
      }
    \fi
    \expandafter
  \endgroup
  \childdoctmp
}
%    \end{macrocode}

% \macro{\childdocforwardprefix}
% The command |\childdocforwardprefix| redirects
% compilation to the main or a child file by means of a pattern.
% The prefix |#1| in the current filename is replaced by |#2|
% and the suffix of the current filename is kept
% (it is assumed that the filename does not contain the substring `|~~~|'
% which is used as a delimiter).
% Compilation is handed over to the new file by |\childdocforward|:
%    \begin{macrocode}
\newcommand{\childdocforwardprefix}[3][]
{
  \begingroup
    \def\childdocextract #2##1~~~{\def\childdoctmp{\childdocforward[#1]{#3##1}}}
    \expandafter\childdocextract\childdocname~~~
    \expandafter
  \endgroup
  \childdoctmp
}
%    \end{macrocode}

% \macro{\childdoc}
% The deprecated macro |\childdoc| is a legacy version of |\childdocmain|:
%    \begin{macrocode}
\newcommand{\childdoc}{\childdocmain}
%    \end{macrocode}

% \macro{\childdocredirect}
% The deprecated macro |\childdocredirect| is a legacy version
% of |\childdocforward| and |\childdocforwardprefix|:
%    \begin{macrocode}
\newcommand{\childdocredirect}[2][]
{
  \begingroup
    \if?#1?
      \def\childdoctmp{\childdocforward{#2}}
    \else
      \def\childdoctmp{\childdocforwardprefix{#1}{#2}}
    \fi
    \expandafter
  \endgroup
  \childdoctmp
}
%    \end{macrocode}

%\iffalse
%</package>
%\fi
%
\endinput

\childdocforward{cdocsamp}
%    \end{macrocode}

%\iffalse
%</sampledraft>
%\fi
%
% %%%%%%%%%%%%%%%%%%%%%%%%%%%%%%%%%%%%%%
% \paragraph{Forwarding for Final Version of the Chapters.}
%
% The following forwarding files |cdocsfn1.tex| and |cdocsfn2.tex|
% (with identical content)
% compile the final versions of the child documents
% |cdocsch1.tex| and |cdocsch2.tex|, respectively:
%\iffalse
%<*samplefinal>
%\fi
%    \begin{macrocode}
\def\version{final}
% \iffalse
%
% childdoc.dtx Copyright (C) 2017-2018 Niklas Beisert
%
% This work may be distributed and/or modified under the
% conditions of the LaTeX Project Public License, either version 1.3
% of this license or (at your option) any later version.
% The latest version of this license is in
%   http://www.latex-project.org/lppl.txt
% and version 1.3 or later is part of all distributions of LaTeX
% version 2005/12/01 or later.
%
% This work has the LPPL maintenance status `maintained'.
%
% The Current Maintainer of this work is Niklas Beisert.
%
% This work consists of the files childdoc.dtx and childdoc.ins
% and the derived files childdoc.def and cdocsamp.tex with
% cdocsch1.tex, cdocsch2.tex, cdocsdrf.tex, cdocsfn1.tex, cdocsfn2.tex.
%
%<package>\ifdefined\childdocmain\endinput\fi
%<package>\ProvidesFile{childdoc.def}[2018/12/30 v2.0 child document driver]
%<samplemain>\ProvidesFile{cdocsamp.tex}[2018/12/30 v2.0 sample for childdoc]
%<*driver>
%\ProvidesFile{childdoc.drv}[2018/12/30 v2.0 childdoc reference manual file]
\PassOptionsToClass{10pt,a4paper}{article}
\documentclass{ltxdoc}

\usepackage[margin=35mm]{geometry}
\usepackage{hyperref}
\usepackage{hyperxmp}
\usepackage[usenames]{color}

\hypersetup{colorlinks=true}
\hypersetup{pdfstartview=FitH}
\hypersetup{pdfpagemode=UseNone}
\hypersetup{pdfsource={}}
\hypersetup{pdflang={en-UK}}
\hypersetup{pdfcopyright={Copyright 2017-2018 Niklas Beisert.
  This work may be distributed and/or modified under the
  conditions of the LaTeX Project Public License, either version 1.3
  of this license or (at your option) any later version.}}
\hypersetup{pdflicenseurl={http://www.latex-project.org/lppl.txt}}
\hypersetup{pdfcontactaddress={ETH Zurich, ITP, HIT K,
  Wolfgang-Pauli-Strasse 27}}
\hypersetup{pdfcontactpostcode={8093}}
\hypersetup{pdfcontactcity={Zurich}}
\hypersetup{pdfcontactcountry={Switzerland}}
\hypersetup{pdfcontactemail={nbeisert@itp.phys.ethz.ch}}
\hypersetup{pdfcontacturl={http://people.phys.ethz.ch/\xmptilde nbeisert/}}

\newcommand{\secref}[1]{\hyperref[#1]{section \ref*{#1}}}

\parskip1ex
\parindent0pt
\let\olditemize\itemize
\def\itemize{\olditemize\parskip0pt}

\begin{document}

\title{The \textsf{childdoc} Package}
\hypersetup{pdftitle={The childdoc Package}}
\author{Niklas Beisert\\[2ex]
  Institut f\"ur Theoretische Physik\\
  Eidgen\"ossische Technische Hochschule Z\"urich\\
  Wolfgang-Pauli-Strasse 27, 8093 Z\"urich, Switzerland\\[1ex]
  \href{mailto:nbeisert@itp.phys.ethz.ch}
  {\texttt{nbeisert@itp.phys.ethz.ch}}}
\hypersetup{pdfauthor={Niklas Beisert}}
\hypersetup{pdfsubject={Manual for the LaTeX2e Package childdoc}}
\date{30 December 2018, \textsf{v2.0}}
\maketitle

\begin{abstract}\noindent
\textsf{childdoc} is a \LaTeXe{} package
that enables the direct compilation
of document sections included by |\include|
to individual files.
\end{abstract}

\begingroup
\parskip0ex
\tableofcontents
\endgroup

%%%%%%%%%%%%%%%%%%%%%%%%%%%%%%%%%%%%%%%%%%%%%%%%%%%%%%%%%%%%%%%%%%%%%%%%%%%%%%%%
%%%%%%%%%%%%%%%%%%%%%%%%%%%%%%%%%%%%%%%%%%%%%%%%%%%%%%%%%%%%%%%%%%%%%%%%%%%%%%%%
\section{Introduction}

\LaTeX{} provides a mechanism to structure a large document (such as a book)
into a main file and several child files (containing the chapters)
using the |\include| command.
This mechanism is beneficial for documents
which span hundreds of pages in order to
make the source file(s) more manageable.
Moreover, compilation can be restricted to
selected child files by means of the |\includeonly| command.
The latter feature can be used to reduce the compilation time while editing
(this was significantly more useful in the earlier days of \LaTeX{})
or to generate a smaller document which is easier to navigate.
Another application of |\includeonly| is to generate
documents consisting of selected parts of the complete document.

However, there are a few drawbacks of the plain |\include| mechanism:
\begin{itemize}
\item
The child files cannot be compiled on their own,
they can only be compiled via the main file.
A naive editing environment
(such as a text editor with an option
to have the current file processed by \LaTeX)
may require one to switch to the main file before compiling;
attempting to compile the child file produces errors.
\item
The main file must be modified (each time)
to adjust the |\includeonly| command
to the present needs. This easily leaves the main file in a messy state.
\item
The generated document will always carry the filename
of the main document. This is inconvenient if
several child files are to be compiled and
to be kept for distribution.
\end{itemize}

The present package provides a simple interface
to make child files individually compilable by \LaTeX{}.
Compiling a child file then has the same effect as compiling
the main file with an |\includeonly| command
to select the appropriate child.
Moreover the generated document will carry the name of the child
rather than the main file.
This resolves all three above issues.

This feature is meant to make the editing of books,
thesis documents and lecture notes somewhat more convenient.
However, the package can also be used efficiently for
composing a series of documents (such as exercise sheets)
which are typically distributed individually.
It then assists the author in generating the individual documents
(potentially in different versions)
as well as a document containing the collected series.
Another application is in developing style files
or other kinds of included material
where compilation of the style file could redirect
to a sample or test file.

%%%%%%%%%%%%%%%%%%%%%%%%%%%%%%%%%%%%%%%%%%%%%%%%%%%%%%%%%%%%%%%%%%%%%%%%%%%%%%%%
%%%%%%%%%%%%%%%%%%%%%%%%%%%%%%%%%%%%%%%%%%%%%%%%%%%%%%%%%%%%%%%%%%%%%%%%%%%%%%%%
\section{Usage}

First of all, the package \textsf{childdoc} is \emph{not} a standard
\LaTeXe{} |.sty| style file! Therefore it needs to be invoked in
a non-standard way.

%%%%%%%%%%%%%%%%%%%%%%%%%%%%%%%%%%%%%%%%%%%%%%%%%%%%%%%%%%%%%%%%%%%%%%%%%%%%%%%%
\subsection{Included Files}
\label{sec:include}

%%%%%%%%%%%%%%%%%%%%%%%%%%%%%%%%%%%%%%%%
\DescribeMacro{\childdocmain}
To use the package, add the commands
\begin{center}
\begin{tabular}{l}
|\input{childdoc.def}|\\
|\childdocmain{}|\\
\end{tabular}
\end{center}
at the very top of the main \LaTeX{} file,
in particular \emph{before} the |\documentclass| statement!
The argument of |\childdocmain| should be left empty
(but it must be present).

%%%%%%%%%%%%%%%%%%%%%%%%%%%%%%%%%%%%%%%%
\DescribeMacro{\childdocof}
Furthermore, add the commands
\begin{center}
\begin{tabular}{l}
|\input{childdoc.def}|\\
|\childdocof{|\textit{main}|}|\\
\end{tabular}
\end{center}
at the top of every child file \textit{child}
which is included by |\include{|\textit{child}|}|
from within the main file
(or at least for those files to be compiled individually).
The argument \textit{main} must be the filename of the main file.

There are a couple of
considerations in setting up the main and child documents:

%%%%%%%%%%%%%%%%%%%%%%%%%%%%%%%%%%%%%%%%
\paragraph{Restrictions.}

Please note the following restrictions:
\begin{itemize}
\item
|\childdocmain| must be called with one argument \textit{main}
to ensure compatibility with earlier version of the package.
It must either be empty (|\childdocmain{}|)
or precisely match the filename of the main file in which it is specified.
See \secref{sec:detection} for further information.
\item
The filename \textit{main} must be specified without the |.tex| extension.
\item
The filename \textit{main} is case sensitive
(even in case-insensitive file systems)
due to internal string comparison.
\item
The argument \textit{main} should be fully expanded, it cannot be a macro.
\item
Subdirectories and special characters should be avoided in filenames.
\item
The command |\childdocmain{|\textit{main}|}| must be followed by a whitespace.
It should not be followed immediately by another command
or by a comment mark `|%|'.
This is because the \TeX{} parser reads the token immediately following
the argument of |\childdocmain| and puts it
at the beginning of every child section;
however, a white\-space is ignored.
\end{itemize}

%%%%%%%%%%%%%%%%%%%%%%%%%%%%%%%%%%%%%%%%
\paragraph{Content of Main File.}

It is advisable to place all content in the child files included by |\include|.
Any output contained in the main file will appear in all child documents
unless suppressed manually;
it cannot be suppressed automatically by the |\includeonly| directive
and thus should normally be avoided.
A method to include some content in the main file
by means of conditional processing is described in \secref{sec:conditional}.

%%%%%%%%%%%%%%%%%%%%%%%%%%%%%%%%%%%%%%%%
\paragraph{Page Numbering.}

When only a part of the document is compiled,
the appropriate numbering of pages
(as well as other status parameters)
is determined from the |.aux| files.
The latter contain information from previous passes.
However this information needs to propagate through
all intermediate child documents.
Therefore the page numbering in child documents may well
be inconsistent until the complete document is compiled at least once.

A useful (if unconventional) way to always ensure a consistent
page numbering is to restart the numbering in each child document
and denote the pages by `\textit{child}|.|\textit{page}'
where \textit{child} represents the chapter/section number of the child file.
This can be achieved by the command
|\numberwithin{page}{|\textit{child}|}|
of the \textsf{amsmath} package
where \textit{child} can be |chapter| or |section|
depending on the chosen structuring.
Alternatively, one can modify the macro |\thepage| appropriately
and reset the counter |page| at the start of each child file.

%%%%%%%%%%%%%%%%%%%%%%%%%%%%%%%%%%%%%%%%%%%%%%%%%%%%%%%%%%%%%%%%%%%%%%%%%%%%%%%%
\subsection{Conditional Processing}
\label{sec:conditional}

The package provides a mechanism to compile different versions
of a document. To customise the versions further some conditional processing
can come in handy to distinguish which version is being compiled.
The package provides two macros to describe the compilation context:

%%%%%%%%%%%%%%%%%%%%%%%%%%%%%%%%%%%%%%%%
\DescribeMacro{\ifchilddoc}
The conditional |\ifchilddoc| distinguishes between the compilation of
child documents and the main document:
%
\begin{center}
|\ifchilddoc |\textit{child-code}| |[|\||else |\textit{main-code}]| \||fi|
\end{center}

%%%%%%%%%%%%%%%%%%%%%%%%%%%%%%%%%%%%%%%%
\DescribeMacro{\childdocname}
\DescribeMacro{\childdocjob}
The macro |\childdocname| contains the filename (without extension)
of the main or child file being processed.
Note that |\childdocjob| will always contain the name of the main file.

%%%%%%%%%%%%%%%%%%%%%%%%%%%%%%%%%%%%%%%%
\paragraph{Title Page.}

Conditional processing can be used to include a title or banner page
in the main document when proper precautions are taken.
Importantly, the code in the main file should ensure that the page counter
(as well as other status parameters which are stored in the |.aux| files)
takes the same value after the conditional processing.
Otherwise the page numbers may take divergent values
depending on which part is compiled.

For example, a title page could be declared by:
%
\begin{center}
\begin{tabular}{l}
|\ifchilddoc\||else|\\
|\addtocounter{page}{-1}|\\
\textit{code for title page}\\
|\newpage|\\
|\||fi|
\end{tabular}
\end{center}
%
A banner page for the child documents can be generated by:
%
\begin{center}
\begin{tabular}{l}
|\ifchilddoc|\\
|\addtocounter{page}{-1}|\\
\textit{code for banner page}\\
|\newpage|\\
|\||fi|
\end{tabular}
\end{center}
%
Here one could write a message such as:
\begin{center}
|This is the part \childdocname{} of \childdocjob{}.|
\end{center}

%%%%%%%%%%%%%%%%%%%%%%%%%%%%%%%%%%%%%%%%%%%%%%%%%%%%%%%%%%%%%%%%%%%%%%%%%%%%%%%%
\subsection{Flags}
\label{sec:flags}

The package makes it easy to generate different versions
of the main or child documents.
To this end compilation flags can be defined
and assigned different default values.
They will be particularly useful in conjunction
with the forwarding mechanism described in \secref{sec:forward}.

For example, it may be useful to have a flag |\version|
which can be set to |draft| or |final|.
The document source will contain some conditional code
depending on the value of |\version|.
Suppose further, the flag should default to |final| for the main file
and to |draft| for child files
which is a natural assignment for editing the document.
This is achieved by placing the following code
in the preamble of the main document
(below the |\childdocmain| directive):
%
\begin{center}
\begin{tabular}{l}
|\ifchilddoc|\\
|\providecommand{\version}{draft}|\\
|\||else|\\
|\providecommand{\version}{final}|\\
|\||fi|
\end{tabular}
\end{center}
%
The definition by |\providecommand| makes sure
that previous definitions are not overwritten.
Further statements |\providecommand{\version}{...}|
can thus be added before the above code to override it.

For the main file, one might add a line
(between |\childdocmain| and the above block)
%
\begin{center}
|%\ifchilddoc\||else\providecommand{\version}{draft}\||fi|
\end{center}
%
which can be uncommented to produce a draft version.
Likewise one can add a line to the very top of a child file
(above the |\childdocof{|\textit{main}|}| directive)
%
\begin{center}
|%\providecommand{\version}{final}|
\end{center}
%
which can be uncommented to produce the final version of this child document.

%%%%%%%%%%%%%%%%%%%%%%%%%%%%%%%%%%%%%%%%%%%%%%%%%%%%%%%%%%%%%%%%%%%%%%%%%%%%%%%%
\subsection{Forwarding}
\label{sec:forward}

Different versions of the main or child documents
using compilation flags as described in \secref{sec:flags}
can be (permanently) stored in different files
for convenient compilation, viewing and distribution.
To this end, the package defines a command
to pass on compilation to a different file:

%%%%%%%%%%%%%%%%%%%%%%%%%%%%%%%%%%%%%%%%
\DescribeMacro{\childdocforward}
The command |\childdocforward| redirects processing to
another source file:
%
\begin{center}
\begin{tabular}{l}
|\input{childdoc.def}|\\
|\childdocforward[|\textit{main}|]{|\textit{dest}|}|\\
\end{tabular}
\end{center}
%
The argument \textit{dest} is the destination file
(without extension).
It should be the main file or one of the child files.
Note that further \textsf{childdoc} directives
such as |\childdocof| and |\childdocforward|
in the indicated file will be processed in this form.
The optional argument \textit{main}
passes on directly to the main file \textit{main}
while pretending to compile the child \textit{dest}.
This form behaves as if \textit{dest}
issues |\childdocof{|\textit{main}|}| right away,
and no further \textsf{childdoc} directives will be processed.

%%%%%%%%%%%%%%%%%%%%%%%%%%%%%%%%%%%%%%%%
\DescribeMacro{\...prefix}
In the alternative form |\childdocforwardprefix|,
%
\begin{center}
\begin{tabular}{l}
|\input{childdoc.def}|\\
|\childdocforwardprefix[|\textit{main}|]{|\textit{prefix}|}{|\textit{dest}|}|
\end{tabular}
\end{center}
%
the destination file is determined by a pattern
depending on the current file:
To make this work, the current file must be called
`{\textit{prefix}\hspace{0.2em}\textit{suffix}}'
with \textit{prefix} matching precisely the argument.
Processing is then passed on to the file
`{\textit{dest}\hspace{0.2em}\textit{suffix}}'.
Surely, the same effect is achieved by
directly specifying the
argument `{\textit{dest}\hspace{0.2em}\textit{suffix}}'
in the first form.
However, that requires to set up a different file
for each child. With the alternative form of the command
all these files can have exactly the same content
which simplifies setting them up and maintaining them.

For example, the following file |draft.tex|
with a compilation flag |\version| as described in \secref{sec:flags}
compiles the main document as a draft:
%
\begin{center}
\begin{tabular}{l}
|\def\version{draft}|\\
|\input{childdoc.def}|\\
|\childdocforward{|\textit{main}|}|
\end{tabular}
\end{center}
%
Likewise, the following files |final|\textit{nn}|.tex|
compile the final version of the child document
|child|\textit{nn}|.tex|:
%
\begin{center}
\begin{tabular}{l}
|\def\version{final}|\\
|\input{childdoc.def}|\\
|\childdocforwardprefix{final}{child}|
\end{tabular}
\end{center}
%

Note that when several versions of a main file and/or of each child file
are to be generated, it may be convenient to set up a |Makefile| or
shell script to automatise the process.

%%%%%%%%%%%%%%%%%%%%%%%%%%%%%%%%%%%%%%%%%%%%%%%%%%%%%%%%%%%%%%%%%%%%%%%%%%%%%%%%
\subsection{Command Line Processing}
\label{sec:commandline}

The effect of redirection files can also be achieved by invoking
the \LaTeX{} compiler with a more elaborate command line.
Most conveniently this should be done as part
of a shell script or a |Makefile|.

When using \textsf{childdoc} in the main file, the following
command lines effectively perform a redirection
(note that depending on the shell being used,
backslashes may have to be doubled: `|\|' $\to$ `|\\|'):
%
\begin{center}
|... -jobname "|\textit{target}|" |\\|"|[\textit{flags}]%
|\input{childdoc.def}\childdocforward[|\textit{main}|]{|\textit{dest}|}"|
\end{center}
%
Here \textit{target} is the name of the output file,
\textit{main} is the name of the main file
and \textit{dest} is the name of the main or child file to be processed
(all filenames without extensions).
The optional argument \textit{main} can be omitted
if \textit{main} matches \textit{dest}.
Optionally, compilation \textit{flags} can be defined via |\def| commands.
This command line makes the \TeX{} engine believe
it is compiling the file \textit{target}
whose content is specified as the latter parameter.
The provided code then forwards the processing to
\textit{main} or \textit{dest} as described in \secref{sec:forward}.

%%%%%%%%%%%%%%%%%%%%%%%%%%%%%%%%%%%%%%%%%%%%%%%%%%%%%%%%%%%%%%%%%%%%%%%%%%%%%%%%
\subsection{Include by Input}
\label{sec:input}

Including child documents by |\include| has some restrictions by design.
Most notably, the content of a child document always occupies
its own set of pages; pages cannot be shared between child documents.
Usually, this behaviour makes perfect sense
because each child document contain an essential part of the document.
However, in some situations it may be desirable to compose
a document from a collection of parts
without having mandatory page breaks between then.
For this case, the package
provides a mechanism to include parts
by |\input| which can also be processed individually.
However, by construction this mechanism
requires manual handling of the content to be output.

%%%%%%%%%%%%%%%%%%%%%%%%%%%%%%%%%%%%%%%%
\DescribeMacro{\ifchilddocmanual}
The main file should be prepared as usual, see \secref{sec:include}.
However, the document body must make a distinction
between processing of an individual part and of the main document, e.g.:
%
\begin{center}
\begin{tabular}{l}
|\ifchilddocmanual|\\
|\input{\childdocname}|\\
|\||else|\\
\textit{document body with }|\input{|\textit{part}|}|\\
|\||fi|
\end{tabular}
\end{center}
%
The conditional |\ifchilddocmanual| is true whenever
a part to be included by |\input| is being compiled,
and the name of the part is stored in |\childdocname|.

%%%%%%%%%%%%%%%%%%%%%%%%%%%%%%%%%%%%%%%%
\DescribeMacro{\childdocby}
Each part to be included by |\input| should start with:
%
\begin{center}
\begin{tabular}{l}
|\input{childdoc.def}|\\
|\childdocby{|\textit{main}|}|\\
\end{tabular}
\end{center}
%
The directive |\childdocby| is similar to |\childdocof|
described in \secref{sec:include},
but the subsequent selection of content must be done manually.
To that end, both |\ifchilddoc| and |\ifchilddocmanual|
will be true upon processing of a part,
and the name of the part is stored in |\childdocname|.
Note that |\jobname| will be set to the filename of the current part
so that each part receives an individual |.aux| file
that does not interfere with the |.aux| file(s) of the main document.
This behaviour can be altered by the alternative form
|\childdocby[*]{|\textit{main}|}| (with a non-empty optional argument)
which uses the |.aux| file of the main document
by setting |\jobname| to \textit{main}.

%%%%%%%%%%%%%%%%%%%%%%%%%%%%%%%%%%%%%%%%%%%%%%%%%%%%%%%%%%%%%%%%%%%%%%%%%%%%%%%%
\subsection{Driver Development}
\label{sec:driver}

The \textsf{childdoc} mechanism can also be use for the development
of definition files such as \LaTeX{} styles or classes.
This case differs from the above setup with multiple parts
included by |\include| in that no |\includeonly| should be invoked.
This can be achieved by starting the include file
(before |\ProvidesPackage|) with:
%
\begin{center}
\begin{tabular}{l}
|\input{childdoc.def}|\\
|\childdocforward{|\textit{main}|}|\\
\end{tabular}
\end{center}
%
or alternatively with:
%
\begin{center}
\begin{tabular}{l}
|\input{childdoc.def}|\\
|\childdocby{|\textit{main}|}|\\
\end{tabular}
\end{center}
%
Both forms have slightly different effects as described above.
The main file is prepared as usual, see \secref{sec:include}.

%%%%%%%%%%%%%%%%%%%%%%%%%%%%%%%%%%%%%%%%%%%%%%%%%%%%%%%%%%%%%%%%%%%%%%%%%%%%%%%%
\subsection{Legacy Detection}
\label{sec:detection}

The directive |\childdocmain| in the main file can detect
whether the complete document or merely a child is to be compiled
even without using the directive |\childdocof|.
This method is deprecated because it is less robust
and there is no compelling reason to use it;
it is merely provided for backward compatibility
and it may be removed in future versions.

If the detection mechanism is to be used,
it is mandatory to correctly specify
the filename of the main file as the argument of |\childdocmain|:
%
\begin{center}
\begin{tabular}{l}
|\input{childdoc.def}|\\
|\childdocmain{|\textit{main}|}|\\
\end{tabular}
\end{center}
%
If |\jobname| does not match the argument \textit{main} of |\childdocmain|,
it is assumed that |\jobname| points to the child file to be compiled.
When using |\childdocmain| with the main file specified as argument,
it suffices to start a child file
with just |\input{|\textit{main}|}|
without loading of the package and using |\childdocof|.
If instead all processing is done
with the appropriate \textsf{childdoc} directives,
the argument of \textit{main} of |\childdocmain| can be empty.

An alternative version of the command line processing described
in \secref{sec:commandline} using the detection mechanism reads:
%
\begin{center}
|... -jobname "|\textit{target}|" "|[\textit{flags}]%
[|\def\jobname{|\textit{dest}|}|]|\input{|\textit{main}|}"|
\end{center}

%%%%%%%%%%%%%%%%%%%%%%%%%%%%%%%%%%%%%%%%%%%%%%%%%%%%%%%%%%%%%%%%%%%%%%%%%%%%%%%%
\subsection{Manual Code}
\label{sec:manual}

In case one cannot be certain whether the definitions file |childdoc.def|
is installed on the target \TeX{} distribution
and one prefers not to ship it,
it is conceivable to paste a few relevant commands into the sources.

To that end, drop all statements |\input{childdoc.def}|
and perform the replacements as outlined below.
Instead of |\childdocmain{|\textit{main}|}| add the following code
to the top of the main file:
%
\begin{center}
\begin{tabular}{l}
|\||ifdefined\childdocname\endinput\||fi\newif\ifchilddoc|\\
|\edef\childdocname{\scantokens\expandafter{\jobname\noexpand}}|\\
|\def\childdocmain{|\textit{main}|}\||ifx\childdocmain\childdocname\||else|\\
|\childdoctrue\includeonly{\childdocname}\let\jobname\childdocmain\||fi|\\
\end{tabular}
\end{center}
%
Instead of |\childdocof{|\textit{main}|}| just include the main file
at the top of each child file:
%
\begin{center}
|\input{|\textit{main}|}|
\end{center}
%
A simple redirection |\childdocforward{|\textit{dest}|}| is achieved by:
%
\begin{center}
|\def\jobname{|\textit{dest}|}\input{\jobname}|
\end{center}
%
The redirection with prefix
|\childdocforwardprefix[|\textit{prefix}|]{|\textit{dest}|}|
is accomplished by:
%
\begin{center}
\begin{tabular}{l}
|{\edef\jobname{\scantokens\expandafter{\jobname\noexpand}}|\\
|\def\redirectjob |\textit{prefix}|#1~~~{\gdef\jobname{|\textit{dest}|#1}}|\\
|\expandafter\redirectjob\jobname~~~}\input{\jobname}|
\end{tabular}
\end{center}

In an alternative approach,
child documents can be compiled by a specific command line
without additional code or specific definitions:
%
\begin{center}
|... -jobname "|\textit{target}|" "|[\textit{flags}]%
|\includeonly{|\textit{dest}|}\input{|\textit{main}|}"|
\end{center}
%

%%%%%%%%%%%%%%%%%%%%%%%%%%%%%%%%%%%%%%%%%%%%%%%%%%%%%%%%%%%%%%%%%%%%%%%%%%%%%%%%
%%%%%%%%%%%%%%%%%%%%%%%%%%%%%%%%%%%%%%%%%%%%%%%%%%%%%%%%%%%%%%%%%%%%%%%%%%%%%%%%
\section{Information}

%%%%%%%%%%%%%%%%%%%%%%%%%%%%%%%%%%%%%%%%%%%%%%%%%%%%%%%%%%%%%%%%%%%%%%%%%%%%%%%%
\subsection{Copyright}

Copyright \copyright{} 2017--2018 Niklas Beisert

This work may be distributed and/or modified under the
conditions of the \LaTeX{} Project Public License, either version 1.3
of this license or (at your option) any later version.
The latest version of this license is in
  \url{http://www.latex-project.org/lppl.txt}
and version 1.3 or later is part of all distributions of \LaTeX{}
version 2005/12/01 or later.

This work has the LPPL maintenance status `maintained'.

The Current Maintainer of this work is Niklas Beisert.

This work consists of the files |README.txt|, |childdoc.ins| and |childdoc.dtx|
as well as the derived files |childdoc.def|, |cdocsamp.tex|
with |cdocsch1.tex|, |cdocsch2.tex|, |cdocspt3.tex|, |cdocspt4.tex|,
|cdocsdrf.tex|, |cdocsfn1.tex|, |cdocsfn2.tex|
as well as |childdoc.pdf|.

%%%%%%%%%%%%%%%%%%%%%%%%%%%%%%%%%%%%%%%%%%%%%%%%%%%%%%%%%%%%%%%%%%%%%%%%%%%%%%%%
\subsection{Files and Installation}

The package consists of the files:
%
\begin{center}
\begin{tabular}{ll}
    |README.txt|   & readme file \\
    |childdoc.ins| & installation file \\
    |childdoc.dtx| & source file \\
    |childdoc.def| & definition file \\
    |cdocsamp.tex| & sample main file \\
    |cdocsch1.tex| & sample include file \\
    |cdocsch2.tex| & sample include file \\
    |cdocspt3.tex| & sample part file \\
    |cdocspt4.tex| & sample part file \\
    |cdocsdrf.tex| & sample redirection file \\
    |cdocsfn1.tex| & sample redirection file \\
    |cdocsfn2.tex| & sample redirection file \\
    |childdoc.pdf| & manual
\end{tabular}
\end{center}
%
The distribution consists of the files
|README.txt|, |childdoc.ins| and |childdoc.dtx|.
%
\begin{itemize}
\item
Run (pdf)\LaTeX{} on |childdoc.dtx|
to compile the manual |childdoc.pdf| (this file).
\item
Run \LaTeX{} on |childdoc.ins| to create the definitions file |childdoc.def|
and the sample |cdocsamp.tex| with include files
|cdocsch1.tex|, |cdocsch2.tex|, |cdocspt3.tex|, |cdocspt4.tex|,
|cdocsdrf.tex|, |cdocsfn1.tex|, |cdocsfn2.tex|.
Then copy the file |childdoc.def| to an appropriate directory of your \LaTeX{}
distribution, e.g.\ \textit{texmf-root}|/tex/latex/childdoc|.
\end{itemize}

%%%%%%%%%%%%%%%%%%%%%%%%%%%%%%%%%%%%%%%%%%%%%%%%%%%%%%%%%%%%%%%%%%%%%%%%%%%%%%%%
\subsection{Related CTAN Packages}

There are several other packages which offer a similar functionality:
%
\begin{itemize}
\item
The packages
\href{http://ctan.org/pkg/docmute}{\textsf{docmute}},
\href{http://ctan.org/pkg/includex}{\textsf{includex}} and
\href{http://ctan.org/pkg/standalone}{\textsf{standalone}}
provide commands to include only the document body of
a child file thus allowing both files to be compiled individually.
\item
The packages \href{http://ctan.org/pkg/subdocs}{\textsf{subdocs}}
and \href{http://ctan.org/pkg/subfiles}{\textsf{subfiles}}
provide structures in which the main and child documents can be
encapsulated and allowing them to be compiled individually.
The inclusion mechanism is different from the conventional |\include|.
\item
The package \href{http://ctan.org/pkg/combine}{\textsf{combine}}
is an elaborate solution to combine several documents into one.
\end{itemize}
%
See also the CTAN topic \href{http://ctan.org/topic/subdocs}{\textsf{subdocs}}
for further related packages.
The present package differs from the above solutions in that
a document structure constructed with the conventional |\include| mechanism
just needs two extra commands at the top of every file
such that all constituent files can be compiled individually.

%%%%%%%%%%%%%%%%%%%%%%%%%%%%%%%%%%%%%%%%%%%%%%%%%%%%%%%%%%%%%%%%%%%%%%%%%%%%%%%%
%\subsection{Feature Suggestions}
%
%The following is a list of features which may be useful for future
%versions of this package:
%%
%\begin{itemize}
%\item
%\ldots
%\end{itemize}

%%%%%%%%%%%%%%%%%%%%%%%%%%%%%%%%%%%%%%%%%%%%%%%%%%%%%%%%%%%%%%%%%%%%%%%%%%%%%%%%
\subsection{Revision History}

%%%%%%%%%%%%%%%%%%%%%%%%%%%%%%%%%%%%%%%%
\paragraph{v2.0:} 2018/12/30

\begin{itemize}
\item
immediate forward processing
\item
added |\childdocby| mechanism
\item
manual restructured
\end{itemize}

%%%%%%%%%%%%%%%%%%%%%%%%%%%%%%%%%%%%%%%%
\paragraph{v1.6:} 2018/01/17

\begin{itemize}
\item
application for development of include files
\item
corrections to manual
\end{itemize}

%%%%%%%%%%%%%%%%%%%%%%%%%%%%%%%%%%%%%%%%
\paragraph{v1.5:} 2017/05/21

\begin{itemize}
\item
more complete structuring introduced
\item
|\childdocof| introduced
\item
|\childdoc| renamed to |\childdocmain|
\item
|\childredirect| renamed to |\childdocforward| and |\childdocforwardprefix|
and functionality expanded
\end{itemize}

%%%%%%%%%%%%%%%%%%%%%%%%%%%%%%%%%%%%%%%%
\paragraph{v1.0:} 2017/04/27

\begin{itemize}
\item
manual and install package
\item
first version published on CTAN
\end{itemize}

%%%%%%%%%%%%%%%%%%%%%%%%%%%%%%%%%%%%%%%%
\paragraph{v0.6:} 2017/04/26

\begin{itemize}
\item
redirection mechanism added
\end{itemize}

%%%%%%%%%%%%%%%%%%%%%%%%%%%%%%%%%%%%%%%%
\paragraph{v0.5:} 2017/04/26

\begin{itemize}
\item
functionality in definition file
\end{itemize}


%%%%%%%%%%%%%%%%%%%%%%%%%%%%%%%%%%%%%%%%%%%%%%%%%%%%%%%%%%%%%%%%%%%%%%%%%%%%%%%%
%%%%%%%%%%%%%%%%%%%%%%%%%%%%%%%%%%%%%%%%%%%%%%%%%%%%%%%%%%%%%%%%%%%%%%%%%%%%%%%%
%%%%%%%%%%%%%%%%%%%%%%%%%%%%%%%%%%%%%%%%%%%%%%%%%%%%%%%%%%%%%%%%%%%%%%%%%%%%%%%%
\appendix

\settowidth\MacroIndent{\rmfamily\scriptsize 000\ }

 \DocInput{childdoc.dtx}

\end{document}
%</driver>
% \fi
%
% %%%%%%%%%%%%%%%%%%%%%%%%%%%%%%%%%%%%%%%%%%%%%%%%%%%%%%%%%%%%%%%%%%%%%%%%%%%%%%
% %%%%%%%%%%%%%%%%%%%%%%%%%%%%%%%%%%%%%%%%%%%%%%%%%%%%%%%%%%%%%%%%%%%%%%%%%%%%%%
% \section{Sample}
%\iffalse
%<*samplemain>
%\fi
%
% The following presents a sample document
% with two chapters, two parts, a title page,
% a compile flag as well as three forwarding files to set the flag.
% It consists of eight |.tex| files:
% \begin{center}
% \begin{tabular}{ll}
% |cdocsamp.tex|&main file\\
% |cdocsch1.tex|&include file for chapter 1\\
% |cdocsch2.tex|&include file for chapter 2\\
% |cdocspt3.tex|&include file for part 3\\
% |cdocspt4.tex|&include file for part 4\\
% |cdocsdrf.tex|&forwarding file for main file in draft mode\\
% |cdocsfi1.tex|&forwarding file for final version of chapter 1\\
% |cdocsfi2.tex|&forwarding file for final version of chapter 2\\
% \end{tabular}
% \end{center}
% Each of the eight files can be compiled directly by the \LaTeX{} compiler.
%
% %%%%%%%%%%%%%%%%%%%%%%%%%%%%%%%%%%%%%%
% \paragraph{Main File.}
%
% The main file is called |cdocsamp.tex|.
%
% Load the \textsf{childdoc} definitions and
% declare the filename for the main document:
%    \begin{macrocode}
\input{childdoc.def}
\childdocmain{}
%    \end{macrocode}

% Optional override for |\version| flag:
%    \begin{macrocode}
%%\ifchilddoc\else\providecommand{\version}{draft}\fi
%    \end{macrocode}

% Define the default values for the |\version| flag
% (|final| for the main file and |draft| for childs):
%    \begin{macrocode}
\ifchilddoc
\providecommand{\version}{draft}
\else
\providecommand{\version}{final}
\fi
%    \end{macrocode}

% Load the standard document class:
%    \begin{macrocode}
\documentclass[12pt]{article}
%    \end{macrocode}

% Start the document body:
%    \begin{macrocode}
\begin{document}
%    \end{macrocode}

% Declare a title page.
% Print title, part of document being processed and version flag:
%    \begin{macrocode}
\addtocounter{page}{-1}
\begin{center}
{\LARGE\bfseries{}childdoc example\par}
\vspace{1cm}
\ifchilddoc
\ifchilddocmanual part\else chapter\fi:
`\childdocname' of `\childdocjob'\par
\else
main document: `\childdocjob'\par
\fi
version: \version\par
\end{center}
\newpage
%    \end{macrocode}

% Manually include selected file,
% otherwise process as usual:
%    \begin{macrocode}
\ifchilddocmanual
\section*{part `\childdocname'}
\input{\childdocname}
\else
%    \end{macrocode}

% Include the two chapters:
%    \begin{macrocode}
\include{cdocsch1}
\include{cdocsch2}
%    \end{macrocode}

% Include the two parts unless only chapters should be displayed:
%    \begin{macrocode}
\ifchilddoc\else
\section{part three}
\input{cdocspt3}
\section{part four}
\input{cdocspt4}
\fi
%    \end{macrocode}

% Process as usual until here:
%    \begin{macrocode}
\fi
%    \end{macrocode}

% End of document body:
%    \begin{macrocode}
\end{document}
%    \end{macrocode}
%\iffalse
%</samplemain>
%\fi
%
% %%%%%%%%%%%%%%%%%%%%%%%%%%%%%%%%%%%%%%
% \paragraph{Chapter Include Files.}
%
% The include files are called |cdocsch1.tex| and |cdocsch2.tex|.
%
%\iffalse
%<*samplechap1|samplechap2>
%\fi

% Optional override for |\version| flag:
%    \begin{macrocode}
%%\providecommand{\version}{final}
%    \end{macrocode}

% Include the main document:
%    \begin{macrocode}
\input{childdoc.def}
\childdocof{cdocsamp}
%    \end{macrocode}

%\iffalse
%</samplechap1|samplechap2>
%\fi
%
%\iffalse
%<*samplechap1>
%\fi
% Some text for chapter 1:
%    \begin{macrocode}
\section{one}
some text in chapter one
%    \end{macrocode}

%\iffalse
%</samplechap1>
%\fi
% Some text for chapter 2:
%\iffalse
%<*samplechap2>
%\fi
%    \begin{macrocode}
\section{two}
more text in chapter two
%    \end{macrocode}

%\iffalse
%</samplechap2>
%\fi
%
% %%%%%%%%%%%%%%%%%%%%%%%%%%%%%%%%%%%%%%
% \paragraph{Part Include Files.}
%
% The include files are called |cdocspt3.tex| and |cdocspt4.tex|.
%
%\iffalse
%<*samplepart3|samplepart4>
%\fi

% Optional override for |\version| flag:
%    \begin{macrocode}
%%\providecommand{\version}{final}
%    \end{macrocode}

% Include the main document:
%    \begin{macrocode}
\input{childdoc.def}
\childdocby{cdocsamp}
%    \end{macrocode}

%\iffalse
%</samplepart3|samplepart4>
%\fi
%
%\iffalse
%<*samplepart3>
%\fi
% Some text for part 3:
%    \begin{macrocode}
some text in part three
%    \end{macrocode}

%\iffalse
%</samplepart3>
%\fi
% Some text for part 4:
%\iffalse
%<*samplepart4>
%\fi
%    \begin{macrocode}
more text in part four
%    \end{macrocode}

%\iffalse
%</samplepart4>
%\fi
%
% %%%%%%%%%%%%%%%%%%%%%%%%%%%%%%%%%%%%%%
% \paragraph{Forwarding for a Complete Draft.}
%
% The following forwarding file |cdocsdrf.tex|
% compiles the main document in draft mode:
%\iffalse
%<*sampledraft>
%\fi
%    \begin{macrocode}
\def\version{draft}
\input{childdoc.def}
\childdocforward{cdocsamp}
%    \end{macrocode}

%\iffalse
%</sampledraft>
%\fi
%
% %%%%%%%%%%%%%%%%%%%%%%%%%%%%%%%%%%%%%%
% \paragraph{Forwarding for Final Version of the Chapters.}
%
% The following forwarding files |cdocsfn1.tex| and |cdocsfn2.tex|
% (with identical content)
% compile the final versions of the child documents
% |cdocsch1.tex| and |cdocsch2.tex|, respectively:
%\iffalse
%<*samplefinal>
%\fi
%    \begin{macrocode}
\def\version{final}
\input{childdoc.def}
\childdocforwardprefix[cdocsamp]{cdocsfn}{cdocsch}
%    \end{macrocode}

%\iffalse
%</samplefinal>
%\fi
%
% %%%%%%%%%%%%%%%%%%%%%%%%%%%%%%%%%%%%%%
% \paragraph{Command Line Processing.}
%
% The following three command lines generate the output files
% |cdocscld|, |cdocscl1| and |cdocscl2|
% which should be identical to
% |cdocsdrf|, |cdocsch1| and |cdocsfn2|, respectively:
% \begin{center}
% \begin{tabular}{l}
% |latex -jobname cdocscld \|\\
% |  "\def\version{draft}\input{childdoc.def}\childdocforward{cdocsamp}"|\\
% |latex -jobname cdocscl1 \|\\
% |  "\input{childdoc.def}\childdocforward[cdocsamp]{cdocsch1}"|\\
% |latex -jobname cdocscl2 \|\\
% |  "\def\version{final}\input{childdoc.def}\childdocforward{cdocsch2}"|
% \end{tabular}
% \end{center}
% Note that the trailing backslash on each first line
% merely continues the input to the second line
% (for convenient cut ant paste).
% Furthermore, the command |latex| can be replaced by any
% of its alternative versions such as |pdflatex|.
%
% %%%%%%%%%%%%%%%%%%%%%%%%%%%%%%%%%%%%%%%%%%%%%%%%%%%%%%%%%%%%%%%%%%%%%%%%%%%%%%
% %%%%%%%%%%%%%%%%%%%%%%%%%%%%%%%%%%%%%%%%%%%%%%%%%%%%%%%%%%%%%%%%%%%%%%%%%%%%%%
% \section{Implementation}
%\iffalse
%<*package>
%\fi
%
% This section describes the definitions file |childdoc.def|.

% The definitions cannot be loaded using |\usepackage| or |\RequirePackage|
% which has a mechanism to prevent loading a style file more than once.
% When loading the definitions by means of |\input|
% multiple instances have to be prevented manually:
%\iffalse
%This code needs to be before the `\ProvidesFile' directive
%which is defined at the beginning of this file.
%Therefore it is also placed there and commented out here.
%</package>
%<*discard>
%\fi
%    \begin{macrocode}
\ifdefined\childdocmain\endinput\fi
%    \end{macrocode}
%\iffalse
%</discard>
%<*package>
%\fi
%
% \macro{\ifchilddoc}
% \macro{\ifchilddocmanual}
% The conditional |\ifchilddoc| tells whether a
% child (true) or main (false) document is being compiled.
% The conditional |\ifchilddocmanual| tells whether
% the |\includeonly| mechanism is used (false) or
% the selection of child files must be performed manually (true).
% The definitions initialise to false:
%    \begin{macrocode}
\newif\ifchilddoc
\newif\ifchilddocmanual
%    \end{macrocode}

% \macro{\childdocname}
% \macro{\childdocjob}
% The macro |\childdocname| stores the name of the main document
% to be compiled. The macro |\childdocjob| stores the name of
% the document on which the \LaTeX{} compiler was originally invoked.
% The content of |\jobname| cannot be compared
% to filenames specified in the source due to different catcodes.
% The following code rescans |\jobname|, stores the result
% in |\childdocname| and saves a copy in |\childdocjob|:
%    \begin{macrocode}
\edef\childdocname{\scantokens\expandafter{\jobname\noexpand}}
\let\childdocjob\childdocname
%    \end{macrocode}

% \macro{\childdocdisable}
% The macro |\childdocdisable| prevents the main file
% from being processed more than once.
% At this stage, the main document command |\childdocmain|
% is assumed to be called once again where it should do nothing.
% Any subsequent call to it should prevent
% a secondary processing of the main document
% It overwrites the forwarding commands
% |\childdocof| and |\childdocforward|
% with empty macros to prevent further inclusions of the main document:
%    \begin{macrocode}
\newcommand{\childdocdisable}
{
  \renewcommand{\childdocmain}[1]{\renewcommand{\childdocmain}[1]{\endinput}}
  \renewcommand{\childdocof}[1]{}
  \renewcommand{\childdocby}[2][]{}
  \renewcommand{\childdocforward}[2][]{}
  \renewcommand{\childdocdisable}{}
}
%    \end{macrocode}

% \macro{\childdocmain}
% The macro |\childdocmain| is to be called at the top of the main file
% with nothing or the main filename (without extension) as argument.
% First, it breaks loops.
% If the argument is not empty and does not match |\childdocname|
% (which is set by the first inclusion of |childdoc.def|),
% |\ifchilddoc| is set to true, |\includeonly| is applied to the child file
% and |\jobname| is set to the main file
% (for proper handling of |.aux| files):
%    \begin{macrocode}
\newcommand{\childdocmain}[1]
{
  \childdocdisable\childdocmain{}
  \if?#1?\else
    \begingroup
      \def\childdoctmp{#1}
      \ifx\childdoctmp\childdocname
        \def\childdoctmp{}
      \else
        \def\childdoctmp
        {
          \childdoctrue
          \includeonly{\childdocname}
          \def\childdocjob{#1}
          \def\jobname{#1}
        }
      \fi
      \expandafter
    \endgroup
    \childdoctmp
  \fi
}
%    \end{macrocode}

% \macro{\childdocof}
% The command |\childdocof| redirects
% compilation to the main file |#1|.
%    \begin{macrocode}
\newcommand{\childdocof}[1]
{
  \childdocdisable
  \childdoctrue
  \includeonly{\childdocname}
  \def\jobname{#1}
  \def\childdocjob{#1}
  \input{#1}
}
%    \end{macrocode}

% \macro{\childdocby}
% The command |\childdocby| ....
%    \begin{macrocode}
\newcommand{\childdocby}[2][]
{
  \childdocdisable
  \childdoctrue
  \childdocmanualtrue
  \if?#1?\else
    \def\jobname{#2}
  \fi
  \def\childdocjob{#2}
  \input{#2}
  \endinput
}
%    \end{macrocode}

% \macro{\childdocforward}
% The command |\childdocforward| redirects
% compilation to the main file or
% (if the optional argument is given) a child file.
% Parameters are set as if the main file
% or a child file starting with |\childdocof| was compiled.
% Then compilation is handed over to the main file:
%    \begin{macrocode}
\newcommand{\childdocforward}[2][]
{
  \begingroup
    \if?#1?
      \def\childdoctmp
      {
        \def\childdocname{#2}
        \def\childdocjob{#2}
        \def\jobname{#2}
        \input{#2}
        \endinput
      }
    \else
      \def\childdoctmp
      {
        \childdocdisable
        \def\childdocname{#2}
        \childdoctrue
        \includeonly{#2}
        \def\childdocjob{#1}
        \def\jobname{#1}
        \input{#1}
        \endinput
      }
    \fi
    \expandafter
  \endgroup
  \childdoctmp
}
%    \end{macrocode}

% \macro{\childdocforwardprefix}
% The command |\childdocforwardprefix| redirects
% compilation to the main or a child file by means of a pattern.
% The prefix |#1| in the current filename is replaced by |#2|
% and the suffix of the current filename is kept
% (it is assumed that the filename does not contain the substring `|~~~|'
% which is used as a delimiter).
% Compilation is handed over to the new file by |\childdocforward|:
%    \begin{macrocode}
\newcommand{\childdocforwardprefix}[3][]
{
  \begingroup
    \def\childdocextract #2##1~~~{\def\childdoctmp{\childdocforward[#1]{#3##1}}}
    \expandafter\childdocextract\childdocname~~~
    \expandafter
  \endgroup
  \childdoctmp
}
%    \end{macrocode}

% \macro{\childdoc}
% The deprecated macro |\childdoc| is a legacy version of |\childdocmain|:
%    \begin{macrocode}
\newcommand{\childdoc}{\childdocmain}
%    \end{macrocode}

% \macro{\childdocredirect}
% The deprecated macro |\childdocredirect| is a legacy version
% of |\childdocforward| and |\childdocforwardprefix|:
%    \begin{macrocode}
\newcommand{\childdocredirect}[2][]
{
  \begingroup
    \if?#1?
      \def\childdoctmp{\childdocforward{#2}}
    \else
      \def\childdoctmp{\childdocforwardprefix{#1}{#2}}
    \fi
    \expandafter
  \endgroup
  \childdoctmp
}
%    \end{macrocode}

%\iffalse
%</package>
%\fi
%
\endinput

\childdocforwardprefix[cdocsamp]{cdocsfn}{cdocsch}
%    \end{macrocode}

%\iffalse
%</samplefinal>
%\fi
%
% %%%%%%%%%%%%%%%%%%%%%%%%%%%%%%%%%%%%%%
% \paragraph{Command Line Processing.}
%
% The following three command lines generate the output files
% |cdocscld|, |cdocscl1| and |cdocscl2|
% which should be identical to
% |cdocsdrf|, |cdocsch1| and |cdocsfn2|, respectively:
% \begin{center}
% \begin{tabular}{l}
% |latex -jobname cdocscld \|\\
% |  "\def\version{draft}% \iffalse
%
% childdoc.dtx Copyright (C) 2017-2018 Niklas Beisert
%
% This work may be distributed and/or modified under the
% conditions of the LaTeX Project Public License, either version 1.3
% of this license or (at your option) any later version.
% The latest version of this license is in
%   http://www.latex-project.org/lppl.txt
% and version 1.3 or later is part of all distributions of LaTeX
% version 2005/12/01 or later.
%
% This work has the LPPL maintenance status `maintained'.
%
% The Current Maintainer of this work is Niklas Beisert.
%
% This work consists of the files childdoc.dtx and childdoc.ins
% and the derived files childdoc.def and cdocsamp.tex with
% cdocsch1.tex, cdocsch2.tex, cdocsdrf.tex, cdocsfn1.tex, cdocsfn2.tex.
%
%<package>\ifdefined\childdocmain\endinput\fi
%<package>\ProvidesFile{childdoc.def}[2018/12/30 v2.0 child document driver]
%<samplemain>\ProvidesFile{cdocsamp.tex}[2018/12/30 v2.0 sample for childdoc]
%<*driver>
%\ProvidesFile{childdoc.drv}[2018/12/30 v2.0 childdoc reference manual file]
\PassOptionsToClass{10pt,a4paper}{article}
\documentclass{ltxdoc}

\usepackage[margin=35mm]{geometry}
\usepackage{hyperref}
\usepackage{hyperxmp}
\usepackage[usenames]{color}

\hypersetup{colorlinks=true}
\hypersetup{pdfstartview=FitH}
\hypersetup{pdfpagemode=UseNone}
\hypersetup{pdfsource={}}
\hypersetup{pdflang={en-UK}}
\hypersetup{pdfcopyright={Copyright 2017-2018 Niklas Beisert.
  This work may be distributed and/or modified under the
  conditions of the LaTeX Project Public License, either version 1.3
  of this license or (at your option) any later version.}}
\hypersetup{pdflicenseurl={http://www.latex-project.org/lppl.txt}}
\hypersetup{pdfcontactaddress={ETH Zurich, ITP, HIT K,
  Wolfgang-Pauli-Strasse 27}}
\hypersetup{pdfcontactpostcode={8093}}
\hypersetup{pdfcontactcity={Zurich}}
\hypersetup{pdfcontactcountry={Switzerland}}
\hypersetup{pdfcontactemail={nbeisert@itp.phys.ethz.ch}}
\hypersetup{pdfcontacturl={http://people.phys.ethz.ch/\xmptilde nbeisert/}}

\newcommand{\secref}[1]{\hyperref[#1]{section \ref*{#1}}}

\parskip1ex
\parindent0pt
\let\olditemize\itemize
\def\itemize{\olditemize\parskip0pt}

\begin{document}

\title{The \textsf{childdoc} Package}
\hypersetup{pdftitle={The childdoc Package}}
\author{Niklas Beisert\\[2ex]
  Institut f\"ur Theoretische Physik\\
  Eidgen\"ossische Technische Hochschule Z\"urich\\
  Wolfgang-Pauli-Strasse 27, 8093 Z\"urich, Switzerland\\[1ex]
  \href{mailto:nbeisert@itp.phys.ethz.ch}
  {\texttt{nbeisert@itp.phys.ethz.ch}}}
\hypersetup{pdfauthor={Niklas Beisert}}
\hypersetup{pdfsubject={Manual for the LaTeX2e Package childdoc}}
\date{30 December 2018, \textsf{v2.0}}
\maketitle

\begin{abstract}\noindent
\textsf{childdoc} is a \LaTeXe{} package
that enables the direct compilation
of document sections included by |\include|
to individual files.
\end{abstract}

\begingroup
\parskip0ex
\tableofcontents
\endgroup

%%%%%%%%%%%%%%%%%%%%%%%%%%%%%%%%%%%%%%%%%%%%%%%%%%%%%%%%%%%%%%%%%%%%%%%%%%%%%%%%
%%%%%%%%%%%%%%%%%%%%%%%%%%%%%%%%%%%%%%%%%%%%%%%%%%%%%%%%%%%%%%%%%%%%%%%%%%%%%%%%
\section{Introduction}

\LaTeX{} provides a mechanism to structure a large document (such as a book)
into a main file and several child files (containing the chapters)
using the |\include| command.
This mechanism is beneficial for documents
which span hundreds of pages in order to
make the source file(s) more manageable.
Moreover, compilation can be restricted to
selected child files by means of the |\includeonly| command.
The latter feature can be used to reduce the compilation time while editing
(this was significantly more useful in the earlier days of \LaTeX{})
or to generate a smaller document which is easier to navigate.
Another application of |\includeonly| is to generate
documents consisting of selected parts of the complete document.

However, there are a few drawbacks of the plain |\include| mechanism:
\begin{itemize}
\item
The child files cannot be compiled on their own,
they can only be compiled via the main file.
A naive editing environment
(such as a text editor with an option
to have the current file processed by \LaTeX)
may require one to switch to the main file before compiling;
attempting to compile the child file produces errors.
\item
The main file must be modified (each time)
to adjust the |\includeonly| command
to the present needs. This easily leaves the main file in a messy state.
\item
The generated document will always carry the filename
of the main document. This is inconvenient if
several child files are to be compiled and
to be kept for distribution.
\end{itemize}

The present package provides a simple interface
to make child files individually compilable by \LaTeX{}.
Compiling a child file then has the same effect as compiling
the main file with an |\includeonly| command
to select the appropriate child.
Moreover the generated document will carry the name of the child
rather than the main file.
This resolves all three above issues.

This feature is meant to make the editing of books,
thesis documents and lecture notes somewhat more convenient.
However, the package can also be used efficiently for
composing a series of documents (such as exercise sheets)
which are typically distributed individually.
It then assists the author in generating the individual documents
(potentially in different versions)
as well as a document containing the collected series.
Another application is in developing style files
or other kinds of included material
where compilation of the style file could redirect
to a sample or test file.

%%%%%%%%%%%%%%%%%%%%%%%%%%%%%%%%%%%%%%%%%%%%%%%%%%%%%%%%%%%%%%%%%%%%%%%%%%%%%%%%
%%%%%%%%%%%%%%%%%%%%%%%%%%%%%%%%%%%%%%%%%%%%%%%%%%%%%%%%%%%%%%%%%%%%%%%%%%%%%%%%
\section{Usage}

First of all, the package \textsf{childdoc} is \emph{not} a standard
\LaTeXe{} |.sty| style file! Therefore it needs to be invoked in
a non-standard way.

%%%%%%%%%%%%%%%%%%%%%%%%%%%%%%%%%%%%%%%%%%%%%%%%%%%%%%%%%%%%%%%%%%%%%%%%%%%%%%%%
\subsection{Included Files}
\label{sec:include}

%%%%%%%%%%%%%%%%%%%%%%%%%%%%%%%%%%%%%%%%
\DescribeMacro{\childdocmain}
To use the package, add the commands
\begin{center}
\begin{tabular}{l}
|\input{childdoc.def}|\\
|\childdocmain{}|\\
\end{tabular}
\end{center}
at the very top of the main \LaTeX{} file,
in particular \emph{before} the |\documentclass| statement!
The argument of |\childdocmain| should be left empty
(but it must be present).

%%%%%%%%%%%%%%%%%%%%%%%%%%%%%%%%%%%%%%%%
\DescribeMacro{\childdocof}
Furthermore, add the commands
\begin{center}
\begin{tabular}{l}
|\input{childdoc.def}|\\
|\childdocof{|\textit{main}|}|\\
\end{tabular}
\end{center}
at the top of every child file \textit{child}
which is included by |\include{|\textit{child}|}|
from within the main file
(or at least for those files to be compiled individually).
The argument \textit{main} must be the filename of the main file.

There are a couple of
considerations in setting up the main and child documents:

%%%%%%%%%%%%%%%%%%%%%%%%%%%%%%%%%%%%%%%%
\paragraph{Restrictions.}

Please note the following restrictions:
\begin{itemize}
\item
|\childdocmain| must be called with one argument \textit{main}
to ensure compatibility with earlier version of the package.
It must either be empty (|\childdocmain{}|)
or precisely match the filename of the main file in which it is specified.
See \secref{sec:detection} for further information.
\item
The filename \textit{main} must be specified without the |.tex| extension.
\item
The filename \textit{main} is case sensitive
(even in case-insensitive file systems)
due to internal string comparison.
\item
The argument \textit{main} should be fully expanded, it cannot be a macro.
\item
Subdirectories and special characters should be avoided in filenames.
\item
The command |\childdocmain{|\textit{main}|}| must be followed by a whitespace.
It should not be followed immediately by another command
or by a comment mark `|%|'.
This is because the \TeX{} parser reads the token immediately following
the argument of |\childdocmain| and puts it
at the beginning of every child section;
however, a white\-space is ignored.
\end{itemize}

%%%%%%%%%%%%%%%%%%%%%%%%%%%%%%%%%%%%%%%%
\paragraph{Content of Main File.}

It is advisable to place all content in the child files included by |\include|.
Any output contained in the main file will appear in all child documents
unless suppressed manually;
it cannot be suppressed automatically by the |\includeonly| directive
and thus should normally be avoided.
A method to include some content in the main file
by means of conditional processing is described in \secref{sec:conditional}.

%%%%%%%%%%%%%%%%%%%%%%%%%%%%%%%%%%%%%%%%
\paragraph{Page Numbering.}

When only a part of the document is compiled,
the appropriate numbering of pages
(as well as other status parameters)
is determined from the |.aux| files.
The latter contain information from previous passes.
However this information needs to propagate through
all intermediate child documents.
Therefore the page numbering in child documents may well
be inconsistent until the complete document is compiled at least once.

A useful (if unconventional) way to always ensure a consistent
page numbering is to restart the numbering in each child document
and denote the pages by `\textit{child}|.|\textit{page}'
where \textit{child} represents the chapter/section number of the child file.
This can be achieved by the command
|\numberwithin{page}{|\textit{child}|}|
of the \textsf{amsmath} package
where \textit{child} can be |chapter| or |section|
depending on the chosen structuring.
Alternatively, one can modify the macro |\thepage| appropriately
and reset the counter |page| at the start of each child file.

%%%%%%%%%%%%%%%%%%%%%%%%%%%%%%%%%%%%%%%%%%%%%%%%%%%%%%%%%%%%%%%%%%%%%%%%%%%%%%%%
\subsection{Conditional Processing}
\label{sec:conditional}

The package provides a mechanism to compile different versions
of a document. To customise the versions further some conditional processing
can come in handy to distinguish which version is being compiled.
The package provides two macros to describe the compilation context:

%%%%%%%%%%%%%%%%%%%%%%%%%%%%%%%%%%%%%%%%
\DescribeMacro{\ifchilddoc}
The conditional |\ifchilddoc| distinguishes between the compilation of
child documents and the main document:
%
\begin{center}
|\ifchilddoc |\textit{child-code}| |[|\||else |\textit{main-code}]| \||fi|
\end{center}

%%%%%%%%%%%%%%%%%%%%%%%%%%%%%%%%%%%%%%%%
\DescribeMacro{\childdocname}
\DescribeMacro{\childdocjob}
The macro |\childdocname| contains the filename (without extension)
of the main or child file being processed.
Note that |\childdocjob| will always contain the name of the main file.

%%%%%%%%%%%%%%%%%%%%%%%%%%%%%%%%%%%%%%%%
\paragraph{Title Page.}

Conditional processing can be used to include a title or banner page
in the main document when proper precautions are taken.
Importantly, the code in the main file should ensure that the page counter
(as well as other status parameters which are stored in the |.aux| files)
takes the same value after the conditional processing.
Otherwise the page numbers may take divergent values
depending on which part is compiled.

For example, a title page could be declared by:
%
\begin{center}
\begin{tabular}{l}
|\ifchilddoc\||else|\\
|\addtocounter{page}{-1}|\\
\textit{code for title page}\\
|\newpage|\\
|\||fi|
\end{tabular}
\end{center}
%
A banner page for the child documents can be generated by:
%
\begin{center}
\begin{tabular}{l}
|\ifchilddoc|\\
|\addtocounter{page}{-1}|\\
\textit{code for banner page}\\
|\newpage|\\
|\||fi|
\end{tabular}
\end{center}
%
Here one could write a message such as:
\begin{center}
|This is the part \childdocname{} of \childdocjob{}.|
\end{center}

%%%%%%%%%%%%%%%%%%%%%%%%%%%%%%%%%%%%%%%%%%%%%%%%%%%%%%%%%%%%%%%%%%%%%%%%%%%%%%%%
\subsection{Flags}
\label{sec:flags}

The package makes it easy to generate different versions
of the main or child documents.
To this end compilation flags can be defined
and assigned different default values.
They will be particularly useful in conjunction
with the forwarding mechanism described in \secref{sec:forward}.

For example, it may be useful to have a flag |\version|
which can be set to |draft| or |final|.
The document source will contain some conditional code
depending on the value of |\version|.
Suppose further, the flag should default to |final| for the main file
and to |draft| for child files
which is a natural assignment for editing the document.
This is achieved by placing the following code
in the preamble of the main document
(below the |\childdocmain| directive):
%
\begin{center}
\begin{tabular}{l}
|\ifchilddoc|\\
|\providecommand{\version}{draft}|\\
|\||else|\\
|\providecommand{\version}{final}|\\
|\||fi|
\end{tabular}
\end{center}
%
The definition by |\providecommand| makes sure
that previous definitions are not overwritten.
Further statements |\providecommand{\version}{...}|
can thus be added before the above code to override it.

For the main file, one might add a line
(between |\childdocmain| and the above block)
%
\begin{center}
|%\ifchilddoc\||else\providecommand{\version}{draft}\||fi|
\end{center}
%
which can be uncommented to produce a draft version.
Likewise one can add a line to the very top of a child file
(above the |\childdocof{|\textit{main}|}| directive)
%
\begin{center}
|%\providecommand{\version}{final}|
\end{center}
%
which can be uncommented to produce the final version of this child document.

%%%%%%%%%%%%%%%%%%%%%%%%%%%%%%%%%%%%%%%%%%%%%%%%%%%%%%%%%%%%%%%%%%%%%%%%%%%%%%%%
\subsection{Forwarding}
\label{sec:forward}

Different versions of the main or child documents
using compilation flags as described in \secref{sec:flags}
can be (permanently) stored in different files
for convenient compilation, viewing and distribution.
To this end, the package defines a command
to pass on compilation to a different file:

%%%%%%%%%%%%%%%%%%%%%%%%%%%%%%%%%%%%%%%%
\DescribeMacro{\childdocforward}
The command |\childdocforward| redirects processing to
another source file:
%
\begin{center}
\begin{tabular}{l}
|\input{childdoc.def}|\\
|\childdocforward[|\textit{main}|]{|\textit{dest}|}|\\
\end{tabular}
\end{center}
%
The argument \textit{dest} is the destination file
(without extension).
It should be the main file or one of the child files.
Note that further \textsf{childdoc} directives
such as |\childdocof| and |\childdocforward|
in the indicated file will be processed in this form.
The optional argument \textit{main}
passes on directly to the main file \textit{main}
while pretending to compile the child \textit{dest}.
This form behaves as if \textit{dest}
issues |\childdocof{|\textit{main}|}| right away,
and no further \textsf{childdoc} directives will be processed.

%%%%%%%%%%%%%%%%%%%%%%%%%%%%%%%%%%%%%%%%
\DescribeMacro{\...prefix}
In the alternative form |\childdocforwardprefix|,
%
\begin{center}
\begin{tabular}{l}
|\input{childdoc.def}|\\
|\childdocforwardprefix[|\textit{main}|]{|\textit{prefix}|}{|\textit{dest}|}|
\end{tabular}
\end{center}
%
the destination file is determined by a pattern
depending on the current file:
To make this work, the current file must be called
`{\textit{prefix}\hspace{0.2em}\textit{suffix}}'
with \textit{prefix} matching precisely the argument.
Processing is then passed on to the file
`{\textit{dest}\hspace{0.2em}\textit{suffix}}'.
Surely, the same effect is achieved by
directly specifying the
argument `{\textit{dest}\hspace{0.2em}\textit{suffix}}'
in the first form.
However, that requires to set up a different file
for each child. With the alternative form of the command
all these files can have exactly the same content
which simplifies setting them up and maintaining them.

For example, the following file |draft.tex|
with a compilation flag |\version| as described in \secref{sec:flags}
compiles the main document as a draft:
%
\begin{center}
\begin{tabular}{l}
|\def\version{draft}|\\
|\input{childdoc.def}|\\
|\childdocforward{|\textit{main}|}|
\end{tabular}
\end{center}
%
Likewise, the following files |final|\textit{nn}|.tex|
compile the final version of the child document
|child|\textit{nn}|.tex|:
%
\begin{center}
\begin{tabular}{l}
|\def\version{final}|\\
|\input{childdoc.def}|\\
|\childdocforwardprefix{final}{child}|
\end{tabular}
\end{center}
%

Note that when several versions of a main file and/or of each child file
are to be generated, it may be convenient to set up a |Makefile| or
shell script to automatise the process.

%%%%%%%%%%%%%%%%%%%%%%%%%%%%%%%%%%%%%%%%%%%%%%%%%%%%%%%%%%%%%%%%%%%%%%%%%%%%%%%%
\subsection{Command Line Processing}
\label{sec:commandline}

The effect of redirection files can also be achieved by invoking
the \LaTeX{} compiler with a more elaborate command line.
Most conveniently this should be done as part
of a shell script or a |Makefile|.

When using \textsf{childdoc} in the main file, the following
command lines effectively perform a redirection
(note that depending on the shell being used,
backslashes may have to be doubled: `|\|' $\to$ `|\\|'):
%
\begin{center}
|... -jobname "|\textit{target}|" |\\|"|[\textit{flags}]%
|\input{childdoc.def}\childdocforward[|\textit{main}|]{|\textit{dest}|}"|
\end{center}
%
Here \textit{target} is the name of the output file,
\textit{main} is the name of the main file
and \textit{dest} is the name of the main or child file to be processed
(all filenames without extensions).
The optional argument \textit{main} can be omitted
if \textit{main} matches \textit{dest}.
Optionally, compilation \textit{flags} can be defined via |\def| commands.
This command line makes the \TeX{} engine believe
it is compiling the file \textit{target}
whose content is specified as the latter parameter.
The provided code then forwards the processing to
\textit{main} or \textit{dest} as described in \secref{sec:forward}.

%%%%%%%%%%%%%%%%%%%%%%%%%%%%%%%%%%%%%%%%%%%%%%%%%%%%%%%%%%%%%%%%%%%%%%%%%%%%%%%%
\subsection{Include by Input}
\label{sec:input}

Including child documents by |\include| has some restrictions by design.
Most notably, the content of a child document always occupies
its own set of pages; pages cannot be shared between child documents.
Usually, this behaviour makes perfect sense
because each child document contain an essential part of the document.
However, in some situations it may be desirable to compose
a document from a collection of parts
without having mandatory page breaks between then.
For this case, the package
provides a mechanism to include parts
by |\input| which can also be processed individually.
However, by construction this mechanism
requires manual handling of the content to be output.

%%%%%%%%%%%%%%%%%%%%%%%%%%%%%%%%%%%%%%%%
\DescribeMacro{\ifchilddocmanual}
The main file should be prepared as usual, see \secref{sec:include}.
However, the document body must make a distinction
between processing of an individual part and of the main document, e.g.:
%
\begin{center}
\begin{tabular}{l}
|\ifchilddocmanual|\\
|\input{\childdocname}|\\
|\||else|\\
\textit{document body with }|\input{|\textit{part}|}|\\
|\||fi|
\end{tabular}
\end{center}
%
The conditional |\ifchilddocmanual| is true whenever
a part to be included by |\input| is being compiled,
and the name of the part is stored in |\childdocname|.

%%%%%%%%%%%%%%%%%%%%%%%%%%%%%%%%%%%%%%%%
\DescribeMacro{\childdocby}
Each part to be included by |\input| should start with:
%
\begin{center}
\begin{tabular}{l}
|\input{childdoc.def}|\\
|\childdocby{|\textit{main}|}|\\
\end{tabular}
\end{center}
%
The directive |\childdocby| is similar to |\childdocof|
described in \secref{sec:include},
but the subsequent selection of content must be done manually.
To that end, both |\ifchilddoc| and |\ifchilddocmanual|
will be true upon processing of a part,
and the name of the part is stored in |\childdocname|.
Note that |\jobname| will be set to the filename of the current part
so that each part receives an individual |.aux| file
that does not interfere with the |.aux| file(s) of the main document.
This behaviour can be altered by the alternative form
|\childdocby[*]{|\textit{main}|}| (with a non-empty optional argument)
which uses the |.aux| file of the main document
by setting |\jobname| to \textit{main}.

%%%%%%%%%%%%%%%%%%%%%%%%%%%%%%%%%%%%%%%%%%%%%%%%%%%%%%%%%%%%%%%%%%%%%%%%%%%%%%%%
\subsection{Driver Development}
\label{sec:driver}

The \textsf{childdoc} mechanism can also be use for the development
of definition files such as \LaTeX{} styles or classes.
This case differs from the above setup with multiple parts
included by |\include| in that no |\includeonly| should be invoked.
This can be achieved by starting the include file
(before |\ProvidesPackage|) with:
%
\begin{center}
\begin{tabular}{l}
|\input{childdoc.def}|\\
|\childdocforward{|\textit{main}|}|\\
\end{tabular}
\end{center}
%
or alternatively with:
%
\begin{center}
\begin{tabular}{l}
|\input{childdoc.def}|\\
|\childdocby{|\textit{main}|}|\\
\end{tabular}
\end{center}
%
Both forms have slightly different effects as described above.
The main file is prepared as usual, see \secref{sec:include}.

%%%%%%%%%%%%%%%%%%%%%%%%%%%%%%%%%%%%%%%%%%%%%%%%%%%%%%%%%%%%%%%%%%%%%%%%%%%%%%%%
\subsection{Legacy Detection}
\label{sec:detection}

The directive |\childdocmain| in the main file can detect
whether the complete document or merely a child is to be compiled
even without using the directive |\childdocof|.
This method is deprecated because it is less robust
and there is no compelling reason to use it;
it is merely provided for backward compatibility
and it may be removed in future versions.

If the detection mechanism is to be used,
it is mandatory to correctly specify
the filename of the main file as the argument of |\childdocmain|:
%
\begin{center}
\begin{tabular}{l}
|\input{childdoc.def}|\\
|\childdocmain{|\textit{main}|}|\\
\end{tabular}
\end{center}
%
If |\jobname| does not match the argument \textit{main} of |\childdocmain|,
it is assumed that |\jobname| points to the child file to be compiled.
When using |\childdocmain| with the main file specified as argument,
it suffices to start a child file
with just |\input{|\textit{main}|}|
without loading of the package and using |\childdocof|.
If instead all processing is done
with the appropriate \textsf{childdoc} directives,
the argument of \textit{main} of |\childdocmain| can be empty.

An alternative version of the command line processing described
in \secref{sec:commandline} using the detection mechanism reads:
%
\begin{center}
|... -jobname "|\textit{target}|" "|[\textit{flags}]%
[|\def\jobname{|\textit{dest}|}|]|\input{|\textit{main}|}"|
\end{center}

%%%%%%%%%%%%%%%%%%%%%%%%%%%%%%%%%%%%%%%%%%%%%%%%%%%%%%%%%%%%%%%%%%%%%%%%%%%%%%%%
\subsection{Manual Code}
\label{sec:manual}

In case one cannot be certain whether the definitions file |childdoc.def|
is installed on the target \TeX{} distribution
and one prefers not to ship it,
it is conceivable to paste a few relevant commands into the sources.

To that end, drop all statements |\input{childdoc.def}|
and perform the replacements as outlined below.
Instead of |\childdocmain{|\textit{main}|}| add the following code
to the top of the main file:
%
\begin{center}
\begin{tabular}{l}
|\||ifdefined\childdocname\endinput\||fi\newif\ifchilddoc|\\
|\edef\childdocname{\scantokens\expandafter{\jobname\noexpand}}|\\
|\def\childdocmain{|\textit{main}|}\||ifx\childdocmain\childdocname\||else|\\
|\childdoctrue\includeonly{\childdocname}\let\jobname\childdocmain\||fi|\\
\end{tabular}
\end{center}
%
Instead of |\childdocof{|\textit{main}|}| just include the main file
at the top of each child file:
%
\begin{center}
|\input{|\textit{main}|}|
\end{center}
%
A simple redirection |\childdocforward{|\textit{dest}|}| is achieved by:
%
\begin{center}
|\def\jobname{|\textit{dest}|}\input{\jobname}|
\end{center}
%
The redirection with prefix
|\childdocforwardprefix[|\textit{prefix}|]{|\textit{dest}|}|
is accomplished by:
%
\begin{center}
\begin{tabular}{l}
|{\edef\jobname{\scantokens\expandafter{\jobname\noexpand}}|\\
|\def\redirectjob |\textit{prefix}|#1~~~{\gdef\jobname{|\textit{dest}|#1}}|\\
|\expandafter\redirectjob\jobname~~~}\input{\jobname}|
\end{tabular}
\end{center}

In an alternative approach,
child documents can be compiled by a specific command line
without additional code or specific definitions:
%
\begin{center}
|... -jobname "|\textit{target}|" "|[\textit{flags}]%
|\includeonly{|\textit{dest}|}\input{|\textit{main}|}"|
\end{center}
%

%%%%%%%%%%%%%%%%%%%%%%%%%%%%%%%%%%%%%%%%%%%%%%%%%%%%%%%%%%%%%%%%%%%%%%%%%%%%%%%%
%%%%%%%%%%%%%%%%%%%%%%%%%%%%%%%%%%%%%%%%%%%%%%%%%%%%%%%%%%%%%%%%%%%%%%%%%%%%%%%%
\section{Information}

%%%%%%%%%%%%%%%%%%%%%%%%%%%%%%%%%%%%%%%%%%%%%%%%%%%%%%%%%%%%%%%%%%%%%%%%%%%%%%%%
\subsection{Copyright}

Copyright \copyright{} 2017--2018 Niklas Beisert

This work may be distributed and/or modified under the
conditions of the \LaTeX{} Project Public License, either version 1.3
of this license or (at your option) any later version.
The latest version of this license is in
  \url{http://www.latex-project.org/lppl.txt}
and version 1.3 or later is part of all distributions of \LaTeX{}
version 2005/12/01 or later.

This work has the LPPL maintenance status `maintained'.

The Current Maintainer of this work is Niklas Beisert.

This work consists of the files |README.txt|, |childdoc.ins| and |childdoc.dtx|
as well as the derived files |childdoc.def|, |cdocsamp.tex|
with |cdocsch1.tex|, |cdocsch2.tex|, |cdocspt3.tex|, |cdocspt4.tex|,
|cdocsdrf.tex|, |cdocsfn1.tex|, |cdocsfn2.tex|
as well as |childdoc.pdf|.

%%%%%%%%%%%%%%%%%%%%%%%%%%%%%%%%%%%%%%%%%%%%%%%%%%%%%%%%%%%%%%%%%%%%%%%%%%%%%%%%
\subsection{Files and Installation}

The package consists of the files:
%
\begin{center}
\begin{tabular}{ll}
    |README.txt|   & readme file \\
    |childdoc.ins| & installation file \\
    |childdoc.dtx| & source file \\
    |childdoc.def| & definition file \\
    |cdocsamp.tex| & sample main file \\
    |cdocsch1.tex| & sample include file \\
    |cdocsch2.tex| & sample include file \\
    |cdocspt3.tex| & sample part file \\
    |cdocspt4.tex| & sample part file \\
    |cdocsdrf.tex| & sample redirection file \\
    |cdocsfn1.tex| & sample redirection file \\
    |cdocsfn2.tex| & sample redirection file \\
    |childdoc.pdf| & manual
\end{tabular}
\end{center}
%
The distribution consists of the files
|README.txt|, |childdoc.ins| and |childdoc.dtx|.
%
\begin{itemize}
\item
Run (pdf)\LaTeX{} on |childdoc.dtx|
to compile the manual |childdoc.pdf| (this file).
\item
Run \LaTeX{} on |childdoc.ins| to create the definitions file |childdoc.def|
and the sample |cdocsamp.tex| with include files
|cdocsch1.tex|, |cdocsch2.tex|, |cdocspt3.tex|, |cdocspt4.tex|,
|cdocsdrf.tex|, |cdocsfn1.tex|, |cdocsfn2.tex|.
Then copy the file |childdoc.def| to an appropriate directory of your \LaTeX{}
distribution, e.g.\ \textit{texmf-root}|/tex/latex/childdoc|.
\end{itemize}

%%%%%%%%%%%%%%%%%%%%%%%%%%%%%%%%%%%%%%%%%%%%%%%%%%%%%%%%%%%%%%%%%%%%%%%%%%%%%%%%
\subsection{Related CTAN Packages}

There are several other packages which offer a similar functionality:
%
\begin{itemize}
\item
The packages
\href{http://ctan.org/pkg/docmute}{\textsf{docmute}},
\href{http://ctan.org/pkg/includex}{\textsf{includex}} and
\href{http://ctan.org/pkg/standalone}{\textsf{standalone}}
provide commands to include only the document body of
a child file thus allowing both files to be compiled individually.
\item
The packages \href{http://ctan.org/pkg/subdocs}{\textsf{subdocs}}
and \href{http://ctan.org/pkg/subfiles}{\textsf{subfiles}}
provide structures in which the main and child documents can be
encapsulated and allowing them to be compiled individually.
The inclusion mechanism is different from the conventional |\include|.
\item
The package \href{http://ctan.org/pkg/combine}{\textsf{combine}}
is an elaborate solution to combine several documents into one.
\end{itemize}
%
See also the CTAN topic \href{http://ctan.org/topic/subdocs}{\textsf{subdocs}}
for further related packages.
The present package differs from the above solutions in that
a document structure constructed with the conventional |\include| mechanism
just needs two extra commands at the top of every file
such that all constituent files can be compiled individually.

%%%%%%%%%%%%%%%%%%%%%%%%%%%%%%%%%%%%%%%%%%%%%%%%%%%%%%%%%%%%%%%%%%%%%%%%%%%%%%%%
%\subsection{Feature Suggestions}
%
%The following is a list of features which may be useful for future
%versions of this package:
%%
%\begin{itemize}
%\item
%\ldots
%\end{itemize}

%%%%%%%%%%%%%%%%%%%%%%%%%%%%%%%%%%%%%%%%%%%%%%%%%%%%%%%%%%%%%%%%%%%%%%%%%%%%%%%%
\subsection{Revision History}

%%%%%%%%%%%%%%%%%%%%%%%%%%%%%%%%%%%%%%%%
\paragraph{v2.0:} 2018/12/30

\begin{itemize}
\item
immediate forward processing
\item
added |\childdocby| mechanism
\item
manual restructured
\end{itemize}

%%%%%%%%%%%%%%%%%%%%%%%%%%%%%%%%%%%%%%%%
\paragraph{v1.6:} 2018/01/17

\begin{itemize}
\item
application for development of include files
\item
corrections to manual
\end{itemize}

%%%%%%%%%%%%%%%%%%%%%%%%%%%%%%%%%%%%%%%%
\paragraph{v1.5:} 2017/05/21

\begin{itemize}
\item
more complete structuring introduced
\item
|\childdocof| introduced
\item
|\childdoc| renamed to |\childdocmain|
\item
|\childredirect| renamed to |\childdocforward| and |\childdocforwardprefix|
and functionality expanded
\end{itemize}

%%%%%%%%%%%%%%%%%%%%%%%%%%%%%%%%%%%%%%%%
\paragraph{v1.0:} 2017/04/27

\begin{itemize}
\item
manual and install package
\item
first version published on CTAN
\end{itemize}

%%%%%%%%%%%%%%%%%%%%%%%%%%%%%%%%%%%%%%%%
\paragraph{v0.6:} 2017/04/26

\begin{itemize}
\item
redirection mechanism added
\end{itemize}

%%%%%%%%%%%%%%%%%%%%%%%%%%%%%%%%%%%%%%%%
\paragraph{v0.5:} 2017/04/26

\begin{itemize}
\item
functionality in definition file
\end{itemize}


%%%%%%%%%%%%%%%%%%%%%%%%%%%%%%%%%%%%%%%%%%%%%%%%%%%%%%%%%%%%%%%%%%%%%%%%%%%%%%%%
%%%%%%%%%%%%%%%%%%%%%%%%%%%%%%%%%%%%%%%%%%%%%%%%%%%%%%%%%%%%%%%%%%%%%%%%%%%%%%%%
%%%%%%%%%%%%%%%%%%%%%%%%%%%%%%%%%%%%%%%%%%%%%%%%%%%%%%%%%%%%%%%%%%%%%%%%%%%%%%%%
\appendix

\settowidth\MacroIndent{\rmfamily\scriptsize 000\ }

 \DocInput{childdoc.dtx}

\end{document}
%</driver>
% \fi
%
% %%%%%%%%%%%%%%%%%%%%%%%%%%%%%%%%%%%%%%%%%%%%%%%%%%%%%%%%%%%%%%%%%%%%%%%%%%%%%%
% %%%%%%%%%%%%%%%%%%%%%%%%%%%%%%%%%%%%%%%%%%%%%%%%%%%%%%%%%%%%%%%%%%%%%%%%%%%%%%
% \section{Sample}
%\iffalse
%<*samplemain>
%\fi
%
% The following presents a sample document
% with two chapters, two parts, a title page,
% a compile flag as well as three forwarding files to set the flag.
% It consists of eight |.tex| files:
% \begin{center}
% \begin{tabular}{ll}
% |cdocsamp.tex|&main file\\
% |cdocsch1.tex|&include file for chapter 1\\
% |cdocsch2.tex|&include file for chapter 2\\
% |cdocspt3.tex|&include file for part 3\\
% |cdocspt4.tex|&include file for part 4\\
% |cdocsdrf.tex|&forwarding file for main file in draft mode\\
% |cdocsfi1.tex|&forwarding file for final version of chapter 1\\
% |cdocsfi2.tex|&forwarding file for final version of chapter 2\\
% \end{tabular}
% \end{center}
% Each of the eight files can be compiled directly by the \LaTeX{} compiler.
%
% %%%%%%%%%%%%%%%%%%%%%%%%%%%%%%%%%%%%%%
% \paragraph{Main File.}
%
% The main file is called |cdocsamp.tex|.
%
% Load the \textsf{childdoc} definitions and
% declare the filename for the main document:
%    \begin{macrocode}
\input{childdoc.def}
\childdocmain{}
%    \end{macrocode}

% Optional override for |\version| flag:
%    \begin{macrocode}
%%\ifchilddoc\else\providecommand{\version}{draft}\fi
%    \end{macrocode}

% Define the default values for the |\version| flag
% (|final| for the main file and |draft| for childs):
%    \begin{macrocode}
\ifchilddoc
\providecommand{\version}{draft}
\else
\providecommand{\version}{final}
\fi
%    \end{macrocode}

% Load the standard document class:
%    \begin{macrocode}
\documentclass[12pt]{article}
%    \end{macrocode}

% Start the document body:
%    \begin{macrocode}
\begin{document}
%    \end{macrocode}

% Declare a title page.
% Print title, part of document being processed and version flag:
%    \begin{macrocode}
\addtocounter{page}{-1}
\begin{center}
{\LARGE\bfseries{}childdoc example\par}
\vspace{1cm}
\ifchilddoc
\ifchilddocmanual part\else chapter\fi:
`\childdocname' of `\childdocjob'\par
\else
main document: `\childdocjob'\par
\fi
version: \version\par
\end{center}
\newpage
%    \end{macrocode}

% Manually include selected file,
% otherwise process as usual:
%    \begin{macrocode}
\ifchilddocmanual
\section*{part `\childdocname'}
\input{\childdocname}
\else
%    \end{macrocode}

% Include the two chapters:
%    \begin{macrocode}
\include{cdocsch1}
\include{cdocsch2}
%    \end{macrocode}

% Include the two parts unless only chapters should be displayed:
%    \begin{macrocode}
\ifchilddoc\else
\section{part three}
\input{cdocspt3}
\section{part four}
\input{cdocspt4}
\fi
%    \end{macrocode}

% Process as usual until here:
%    \begin{macrocode}
\fi
%    \end{macrocode}

% End of document body:
%    \begin{macrocode}
\end{document}
%    \end{macrocode}
%\iffalse
%</samplemain>
%\fi
%
% %%%%%%%%%%%%%%%%%%%%%%%%%%%%%%%%%%%%%%
% \paragraph{Chapter Include Files.}
%
% The include files are called |cdocsch1.tex| and |cdocsch2.tex|.
%
%\iffalse
%<*samplechap1|samplechap2>
%\fi

% Optional override for |\version| flag:
%    \begin{macrocode}
%%\providecommand{\version}{final}
%    \end{macrocode}

% Include the main document:
%    \begin{macrocode}
\input{childdoc.def}
\childdocof{cdocsamp}
%    \end{macrocode}

%\iffalse
%</samplechap1|samplechap2>
%\fi
%
%\iffalse
%<*samplechap1>
%\fi
% Some text for chapter 1:
%    \begin{macrocode}
\section{one}
some text in chapter one
%    \end{macrocode}

%\iffalse
%</samplechap1>
%\fi
% Some text for chapter 2:
%\iffalse
%<*samplechap2>
%\fi
%    \begin{macrocode}
\section{two}
more text in chapter two
%    \end{macrocode}

%\iffalse
%</samplechap2>
%\fi
%
% %%%%%%%%%%%%%%%%%%%%%%%%%%%%%%%%%%%%%%
% \paragraph{Part Include Files.}
%
% The include files are called |cdocspt3.tex| and |cdocspt4.tex|.
%
%\iffalse
%<*samplepart3|samplepart4>
%\fi

% Optional override for |\version| flag:
%    \begin{macrocode}
%%\providecommand{\version}{final}
%    \end{macrocode}

% Include the main document:
%    \begin{macrocode}
\input{childdoc.def}
\childdocby{cdocsamp}
%    \end{macrocode}

%\iffalse
%</samplepart3|samplepart4>
%\fi
%
%\iffalse
%<*samplepart3>
%\fi
% Some text for part 3:
%    \begin{macrocode}
some text in part three
%    \end{macrocode}

%\iffalse
%</samplepart3>
%\fi
% Some text for part 4:
%\iffalse
%<*samplepart4>
%\fi
%    \begin{macrocode}
more text in part four
%    \end{macrocode}

%\iffalse
%</samplepart4>
%\fi
%
% %%%%%%%%%%%%%%%%%%%%%%%%%%%%%%%%%%%%%%
% \paragraph{Forwarding for a Complete Draft.}
%
% The following forwarding file |cdocsdrf.tex|
% compiles the main document in draft mode:
%\iffalse
%<*sampledraft>
%\fi
%    \begin{macrocode}
\def\version{draft}
\input{childdoc.def}
\childdocforward{cdocsamp}
%    \end{macrocode}

%\iffalse
%</sampledraft>
%\fi
%
% %%%%%%%%%%%%%%%%%%%%%%%%%%%%%%%%%%%%%%
% \paragraph{Forwarding for Final Version of the Chapters.}
%
% The following forwarding files |cdocsfn1.tex| and |cdocsfn2.tex|
% (with identical content)
% compile the final versions of the child documents
% |cdocsch1.tex| and |cdocsch2.tex|, respectively:
%\iffalse
%<*samplefinal>
%\fi
%    \begin{macrocode}
\def\version{final}
\input{childdoc.def}
\childdocforwardprefix[cdocsamp]{cdocsfn}{cdocsch}
%    \end{macrocode}

%\iffalse
%</samplefinal>
%\fi
%
% %%%%%%%%%%%%%%%%%%%%%%%%%%%%%%%%%%%%%%
% \paragraph{Command Line Processing.}
%
% The following three command lines generate the output files
% |cdocscld|, |cdocscl1| and |cdocscl2|
% which should be identical to
% |cdocsdrf|, |cdocsch1| and |cdocsfn2|, respectively:
% \begin{center}
% \begin{tabular}{l}
% |latex -jobname cdocscld \|\\
% |  "\def\version{draft}\input{childdoc.def}\childdocforward{cdocsamp}"|\\
% |latex -jobname cdocscl1 \|\\
% |  "\input{childdoc.def}\childdocforward[cdocsamp]{cdocsch1}"|\\
% |latex -jobname cdocscl2 \|\\
% |  "\def\version{final}\input{childdoc.def}\childdocforward{cdocsch2}"|
% \end{tabular}
% \end{center}
% Note that the trailing backslash on each first line
% merely continues the input to the second line
% (for convenient cut ant paste).
% Furthermore, the command |latex| can be replaced by any
% of its alternative versions such as |pdflatex|.
%
% %%%%%%%%%%%%%%%%%%%%%%%%%%%%%%%%%%%%%%%%%%%%%%%%%%%%%%%%%%%%%%%%%%%%%%%%%%%%%%
% %%%%%%%%%%%%%%%%%%%%%%%%%%%%%%%%%%%%%%%%%%%%%%%%%%%%%%%%%%%%%%%%%%%%%%%%%%%%%%
% \section{Implementation}
%\iffalse
%<*package>
%\fi
%
% This section describes the definitions file |childdoc.def|.

% The definitions cannot be loaded using |\usepackage| or |\RequirePackage|
% which has a mechanism to prevent loading a style file more than once.
% When loading the definitions by means of |\input|
% multiple instances have to be prevented manually:
%\iffalse
%This code needs to be before the `\ProvidesFile' directive
%which is defined at the beginning of this file.
%Therefore it is also placed there and commented out here.
%</package>
%<*discard>
%\fi
%    \begin{macrocode}
\ifdefined\childdocmain\endinput\fi
%    \end{macrocode}
%\iffalse
%</discard>
%<*package>
%\fi
%
% \macro{\ifchilddoc}
% \macro{\ifchilddocmanual}
% The conditional |\ifchilddoc| tells whether a
% child (true) or main (false) document is being compiled.
% The conditional |\ifchilddocmanual| tells whether
% the |\includeonly| mechanism is used (false) or
% the selection of child files must be performed manually (true).
% The definitions initialise to false:
%    \begin{macrocode}
\newif\ifchilddoc
\newif\ifchilddocmanual
%    \end{macrocode}

% \macro{\childdocname}
% \macro{\childdocjob}
% The macro |\childdocname| stores the name of the main document
% to be compiled. The macro |\childdocjob| stores the name of
% the document on which the \LaTeX{} compiler was originally invoked.
% The content of |\jobname| cannot be compared
% to filenames specified in the source due to different catcodes.
% The following code rescans |\jobname|, stores the result
% in |\childdocname| and saves a copy in |\childdocjob|:
%    \begin{macrocode}
\edef\childdocname{\scantokens\expandafter{\jobname\noexpand}}
\let\childdocjob\childdocname
%    \end{macrocode}

% \macro{\childdocdisable}
% The macro |\childdocdisable| prevents the main file
% from being processed more than once.
% At this stage, the main document command |\childdocmain|
% is assumed to be called once again where it should do nothing.
% Any subsequent call to it should prevent
% a secondary processing of the main document
% It overwrites the forwarding commands
% |\childdocof| and |\childdocforward|
% with empty macros to prevent further inclusions of the main document:
%    \begin{macrocode}
\newcommand{\childdocdisable}
{
  \renewcommand{\childdocmain}[1]{\renewcommand{\childdocmain}[1]{\endinput}}
  \renewcommand{\childdocof}[1]{}
  \renewcommand{\childdocby}[2][]{}
  \renewcommand{\childdocforward}[2][]{}
  \renewcommand{\childdocdisable}{}
}
%    \end{macrocode}

% \macro{\childdocmain}
% The macro |\childdocmain| is to be called at the top of the main file
% with nothing or the main filename (without extension) as argument.
% First, it breaks loops.
% If the argument is not empty and does not match |\childdocname|
% (which is set by the first inclusion of |childdoc.def|),
% |\ifchilddoc| is set to true, |\includeonly| is applied to the child file
% and |\jobname| is set to the main file
% (for proper handling of |.aux| files):
%    \begin{macrocode}
\newcommand{\childdocmain}[1]
{
  \childdocdisable\childdocmain{}
  \if?#1?\else
    \begingroup
      \def\childdoctmp{#1}
      \ifx\childdoctmp\childdocname
        \def\childdoctmp{}
      \else
        \def\childdoctmp
        {
          \childdoctrue
          \includeonly{\childdocname}
          \def\childdocjob{#1}
          \def\jobname{#1}
        }
      \fi
      \expandafter
    \endgroup
    \childdoctmp
  \fi
}
%    \end{macrocode}

% \macro{\childdocof}
% The command |\childdocof| redirects
% compilation to the main file |#1|.
%    \begin{macrocode}
\newcommand{\childdocof}[1]
{
  \childdocdisable
  \childdoctrue
  \includeonly{\childdocname}
  \def\jobname{#1}
  \def\childdocjob{#1}
  \input{#1}
}
%    \end{macrocode}

% \macro{\childdocby}
% The command |\childdocby| ....
%    \begin{macrocode}
\newcommand{\childdocby}[2][]
{
  \childdocdisable
  \childdoctrue
  \childdocmanualtrue
  \if?#1?\else
    \def\jobname{#2}
  \fi
  \def\childdocjob{#2}
  \input{#2}
  \endinput
}
%    \end{macrocode}

% \macro{\childdocforward}
% The command |\childdocforward| redirects
% compilation to the main file or
% (if the optional argument is given) a child file.
% Parameters are set as if the main file
% or a child file starting with |\childdocof| was compiled.
% Then compilation is handed over to the main file:
%    \begin{macrocode}
\newcommand{\childdocforward}[2][]
{
  \begingroup
    \if?#1?
      \def\childdoctmp
      {
        \def\childdocname{#2}
        \def\childdocjob{#2}
        \def\jobname{#2}
        \input{#2}
        \endinput
      }
    \else
      \def\childdoctmp
      {
        \childdocdisable
        \def\childdocname{#2}
        \childdoctrue
        \includeonly{#2}
        \def\childdocjob{#1}
        \def\jobname{#1}
        \input{#1}
        \endinput
      }
    \fi
    \expandafter
  \endgroup
  \childdoctmp
}
%    \end{macrocode}

% \macro{\childdocforwardprefix}
% The command |\childdocforwardprefix| redirects
% compilation to the main or a child file by means of a pattern.
% The prefix |#1| in the current filename is replaced by |#2|
% and the suffix of the current filename is kept
% (it is assumed that the filename does not contain the substring `|~~~|'
% which is used as a delimiter).
% Compilation is handed over to the new file by |\childdocforward|:
%    \begin{macrocode}
\newcommand{\childdocforwardprefix}[3][]
{
  \begingroup
    \def\childdocextract #2##1~~~{\def\childdoctmp{\childdocforward[#1]{#3##1}}}
    \expandafter\childdocextract\childdocname~~~
    \expandafter
  \endgroup
  \childdoctmp
}
%    \end{macrocode}

% \macro{\childdoc}
% The deprecated macro |\childdoc| is a legacy version of |\childdocmain|:
%    \begin{macrocode}
\newcommand{\childdoc}{\childdocmain}
%    \end{macrocode}

% \macro{\childdocredirect}
% The deprecated macro |\childdocredirect| is a legacy version
% of |\childdocforward| and |\childdocforwardprefix|:
%    \begin{macrocode}
\newcommand{\childdocredirect}[2][]
{
  \begingroup
    \if?#1?
      \def\childdoctmp{\childdocforward{#2}}
    \else
      \def\childdoctmp{\childdocforwardprefix{#1}{#2}}
    \fi
    \expandafter
  \endgroup
  \childdoctmp
}
%    \end{macrocode}

%\iffalse
%</package>
%\fi
%
\endinput
\childdocforward{cdocsamp}"|\\
% |latex -jobname cdocscl1 \|\\
% |  "% \iffalse
%
% childdoc.dtx Copyright (C) 2017-2018 Niklas Beisert
%
% This work may be distributed and/or modified under the
% conditions of the LaTeX Project Public License, either version 1.3
% of this license or (at your option) any later version.
% The latest version of this license is in
%   http://www.latex-project.org/lppl.txt
% and version 1.3 or later is part of all distributions of LaTeX
% version 2005/12/01 or later.
%
% This work has the LPPL maintenance status `maintained'.
%
% The Current Maintainer of this work is Niklas Beisert.
%
% This work consists of the files childdoc.dtx and childdoc.ins
% and the derived files childdoc.def and cdocsamp.tex with
% cdocsch1.tex, cdocsch2.tex, cdocsdrf.tex, cdocsfn1.tex, cdocsfn2.tex.
%
%<package>\ifdefined\childdocmain\endinput\fi
%<package>\ProvidesFile{childdoc.def}[2018/12/30 v2.0 child document driver]
%<samplemain>\ProvidesFile{cdocsamp.tex}[2018/12/30 v2.0 sample for childdoc]
%<*driver>
%\ProvidesFile{childdoc.drv}[2018/12/30 v2.0 childdoc reference manual file]
\PassOptionsToClass{10pt,a4paper}{article}
\documentclass{ltxdoc}

\usepackage[margin=35mm]{geometry}
\usepackage{hyperref}
\usepackage{hyperxmp}
\usepackage[usenames]{color}

\hypersetup{colorlinks=true}
\hypersetup{pdfstartview=FitH}
\hypersetup{pdfpagemode=UseNone}
\hypersetup{pdfsource={}}
\hypersetup{pdflang={en-UK}}
\hypersetup{pdfcopyright={Copyright 2017-2018 Niklas Beisert.
  This work may be distributed and/or modified under the
  conditions of the LaTeX Project Public License, either version 1.3
  of this license or (at your option) any later version.}}
\hypersetup{pdflicenseurl={http://www.latex-project.org/lppl.txt}}
\hypersetup{pdfcontactaddress={ETH Zurich, ITP, HIT K,
  Wolfgang-Pauli-Strasse 27}}
\hypersetup{pdfcontactpostcode={8093}}
\hypersetup{pdfcontactcity={Zurich}}
\hypersetup{pdfcontactcountry={Switzerland}}
\hypersetup{pdfcontactemail={nbeisert@itp.phys.ethz.ch}}
\hypersetup{pdfcontacturl={http://people.phys.ethz.ch/\xmptilde nbeisert/}}

\newcommand{\secref}[1]{\hyperref[#1]{section \ref*{#1}}}

\parskip1ex
\parindent0pt
\let\olditemize\itemize
\def\itemize{\olditemize\parskip0pt}

\begin{document}

\title{The \textsf{childdoc} Package}
\hypersetup{pdftitle={The childdoc Package}}
\author{Niklas Beisert\\[2ex]
  Institut f\"ur Theoretische Physik\\
  Eidgen\"ossische Technische Hochschule Z\"urich\\
  Wolfgang-Pauli-Strasse 27, 8093 Z\"urich, Switzerland\\[1ex]
  \href{mailto:nbeisert@itp.phys.ethz.ch}
  {\texttt{nbeisert@itp.phys.ethz.ch}}}
\hypersetup{pdfauthor={Niklas Beisert}}
\hypersetup{pdfsubject={Manual for the LaTeX2e Package childdoc}}
\date{30 December 2018, \textsf{v2.0}}
\maketitle

\begin{abstract}\noindent
\textsf{childdoc} is a \LaTeXe{} package
that enables the direct compilation
of document sections included by |\include|
to individual files.
\end{abstract}

\begingroup
\parskip0ex
\tableofcontents
\endgroup

%%%%%%%%%%%%%%%%%%%%%%%%%%%%%%%%%%%%%%%%%%%%%%%%%%%%%%%%%%%%%%%%%%%%%%%%%%%%%%%%
%%%%%%%%%%%%%%%%%%%%%%%%%%%%%%%%%%%%%%%%%%%%%%%%%%%%%%%%%%%%%%%%%%%%%%%%%%%%%%%%
\section{Introduction}

\LaTeX{} provides a mechanism to structure a large document (such as a book)
into a main file and several child files (containing the chapters)
using the |\include| command.
This mechanism is beneficial for documents
which span hundreds of pages in order to
make the source file(s) more manageable.
Moreover, compilation can be restricted to
selected child files by means of the |\includeonly| command.
The latter feature can be used to reduce the compilation time while editing
(this was significantly more useful in the earlier days of \LaTeX{})
or to generate a smaller document which is easier to navigate.
Another application of |\includeonly| is to generate
documents consisting of selected parts of the complete document.

However, there are a few drawbacks of the plain |\include| mechanism:
\begin{itemize}
\item
The child files cannot be compiled on their own,
they can only be compiled via the main file.
A naive editing environment
(such as a text editor with an option
to have the current file processed by \LaTeX)
may require one to switch to the main file before compiling;
attempting to compile the child file produces errors.
\item
The main file must be modified (each time)
to adjust the |\includeonly| command
to the present needs. This easily leaves the main file in a messy state.
\item
The generated document will always carry the filename
of the main document. This is inconvenient if
several child files are to be compiled and
to be kept for distribution.
\end{itemize}

The present package provides a simple interface
to make child files individually compilable by \LaTeX{}.
Compiling a child file then has the same effect as compiling
the main file with an |\includeonly| command
to select the appropriate child.
Moreover the generated document will carry the name of the child
rather than the main file.
This resolves all three above issues.

This feature is meant to make the editing of books,
thesis documents and lecture notes somewhat more convenient.
However, the package can also be used efficiently for
composing a series of documents (such as exercise sheets)
which are typically distributed individually.
It then assists the author in generating the individual documents
(potentially in different versions)
as well as a document containing the collected series.
Another application is in developing style files
or other kinds of included material
where compilation of the style file could redirect
to a sample or test file.

%%%%%%%%%%%%%%%%%%%%%%%%%%%%%%%%%%%%%%%%%%%%%%%%%%%%%%%%%%%%%%%%%%%%%%%%%%%%%%%%
%%%%%%%%%%%%%%%%%%%%%%%%%%%%%%%%%%%%%%%%%%%%%%%%%%%%%%%%%%%%%%%%%%%%%%%%%%%%%%%%
\section{Usage}

First of all, the package \textsf{childdoc} is \emph{not} a standard
\LaTeXe{} |.sty| style file! Therefore it needs to be invoked in
a non-standard way.

%%%%%%%%%%%%%%%%%%%%%%%%%%%%%%%%%%%%%%%%%%%%%%%%%%%%%%%%%%%%%%%%%%%%%%%%%%%%%%%%
\subsection{Included Files}
\label{sec:include}

%%%%%%%%%%%%%%%%%%%%%%%%%%%%%%%%%%%%%%%%
\DescribeMacro{\childdocmain}
To use the package, add the commands
\begin{center}
\begin{tabular}{l}
|\input{childdoc.def}|\\
|\childdocmain{}|\\
\end{tabular}
\end{center}
at the very top of the main \LaTeX{} file,
in particular \emph{before} the |\documentclass| statement!
The argument of |\childdocmain| should be left empty
(but it must be present).

%%%%%%%%%%%%%%%%%%%%%%%%%%%%%%%%%%%%%%%%
\DescribeMacro{\childdocof}
Furthermore, add the commands
\begin{center}
\begin{tabular}{l}
|\input{childdoc.def}|\\
|\childdocof{|\textit{main}|}|\\
\end{tabular}
\end{center}
at the top of every child file \textit{child}
which is included by |\include{|\textit{child}|}|
from within the main file
(or at least for those files to be compiled individually).
The argument \textit{main} must be the filename of the main file.

There are a couple of
considerations in setting up the main and child documents:

%%%%%%%%%%%%%%%%%%%%%%%%%%%%%%%%%%%%%%%%
\paragraph{Restrictions.}

Please note the following restrictions:
\begin{itemize}
\item
|\childdocmain| must be called with one argument \textit{main}
to ensure compatibility with earlier version of the package.
It must either be empty (|\childdocmain{}|)
or precisely match the filename of the main file in which it is specified.
See \secref{sec:detection} for further information.
\item
The filename \textit{main} must be specified without the |.tex| extension.
\item
The filename \textit{main} is case sensitive
(even in case-insensitive file systems)
due to internal string comparison.
\item
The argument \textit{main} should be fully expanded, it cannot be a macro.
\item
Subdirectories and special characters should be avoided in filenames.
\item
The command |\childdocmain{|\textit{main}|}| must be followed by a whitespace.
It should not be followed immediately by another command
or by a comment mark `|%|'.
This is because the \TeX{} parser reads the token immediately following
the argument of |\childdocmain| and puts it
at the beginning of every child section;
however, a white\-space is ignored.
\end{itemize}

%%%%%%%%%%%%%%%%%%%%%%%%%%%%%%%%%%%%%%%%
\paragraph{Content of Main File.}

It is advisable to place all content in the child files included by |\include|.
Any output contained in the main file will appear in all child documents
unless suppressed manually;
it cannot be suppressed automatically by the |\includeonly| directive
and thus should normally be avoided.
A method to include some content in the main file
by means of conditional processing is described in \secref{sec:conditional}.

%%%%%%%%%%%%%%%%%%%%%%%%%%%%%%%%%%%%%%%%
\paragraph{Page Numbering.}

When only a part of the document is compiled,
the appropriate numbering of pages
(as well as other status parameters)
is determined from the |.aux| files.
The latter contain information from previous passes.
However this information needs to propagate through
all intermediate child documents.
Therefore the page numbering in child documents may well
be inconsistent until the complete document is compiled at least once.

A useful (if unconventional) way to always ensure a consistent
page numbering is to restart the numbering in each child document
and denote the pages by `\textit{child}|.|\textit{page}'
where \textit{child} represents the chapter/section number of the child file.
This can be achieved by the command
|\numberwithin{page}{|\textit{child}|}|
of the \textsf{amsmath} package
where \textit{child} can be |chapter| or |section|
depending on the chosen structuring.
Alternatively, one can modify the macro |\thepage| appropriately
and reset the counter |page| at the start of each child file.

%%%%%%%%%%%%%%%%%%%%%%%%%%%%%%%%%%%%%%%%%%%%%%%%%%%%%%%%%%%%%%%%%%%%%%%%%%%%%%%%
\subsection{Conditional Processing}
\label{sec:conditional}

The package provides a mechanism to compile different versions
of a document. To customise the versions further some conditional processing
can come in handy to distinguish which version is being compiled.
The package provides two macros to describe the compilation context:

%%%%%%%%%%%%%%%%%%%%%%%%%%%%%%%%%%%%%%%%
\DescribeMacro{\ifchilddoc}
The conditional |\ifchilddoc| distinguishes between the compilation of
child documents and the main document:
%
\begin{center}
|\ifchilddoc |\textit{child-code}| |[|\||else |\textit{main-code}]| \||fi|
\end{center}

%%%%%%%%%%%%%%%%%%%%%%%%%%%%%%%%%%%%%%%%
\DescribeMacro{\childdocname}
\DescribeMacro{\childdocjob}
The macro |\childdocname| contains the filename (without extension)
of the main or child file being processed.
Note that |\childdocjob| will always contain the name of the main file.

%%%%%%%%%%%%%%%%%%%%%%%%%%%%%%%%%%%%%%%%
\paragraph{Title Page.}

Conditional processing can be used to include a title or banner page
in the main document when proper precautions are taken.
Importantly, the code in the main file should ensure that the page counter
(as well as other status parameters which are stored in the |.aux| files)
takes the same value after the conditional processing.
Otherwise the page numbers may take divergent values
depending on which part is compiled.

For example, a title page could be declared by:
%
\begin{center}
\begin{tabular}{l}
|\ifchilddoc\||else|\\
|\addtocounter{page}{-1}|\\
\textit{code for title page}\\
|\newpage|\\
|\||fi|
\end{tabular}
\end{center}
%
A banner page for the child documents can be generated by:
%
\begin{center}
\begin{tabular}{l}
|\ifchilddoc|\\
|\addtocounter{page}{-1}|\\
\textit{code for banner page}\\
|\newpage|\\
|\||fi|
\end{tabular}
\end{center}
%
Here one could write a message such as:
\begin{center}
|This is the part \childdocname{} of \childdocjob{}.|
\end{center}

%%%%%%%%%%%%%%%%%%%%%%%%%%%%%%%%%%%%%%%%%%%%%%%%%%%%%%%%%%%%%%%%%%%%%%%%%%%%%%%%
\subsection{Flags}
\label{sec:flags}

The package makes it easy to generate different versions
of the main or child documents.
To this end compilation flags can be defined
and assigned different default values.
They will be particularly useful in conjunction
with the forwarding mechanism described in \secref{sec:forward}.

For example, it may be useful to have a flag |\version|
which can be set to |draft| or |final|.
The document source will contain some conditional code
depending on the value of |\version|.
Suppose further, the flag should default to |final| for the main file
and to |draft| for child files
which is a natural assignment for editing the document.
This is achieved by placing the following code
in the preamble of the main document
(below the |\childdocmain| directive):
%
\begin{center}
\begin{tabular}{l}
|\ifchilddoc|\\
|\providecommand{\version}{draft}|\\
|\||else|\\
|\providecommand{\version}{final}|\\
|\||fi|
\end{tabular}
\end{center}
%
The definition by |\providecommand| makes sure
that previous definitions are not overwritten.
Further statements |\providecommand{\version}{...}|
can thus be added before the above code to override it.

For the main file, one might add a line
(between |\childdocmain| and the above block)
%
\begin{center}
|%\ifchilddoc\||else\providecommand{\version}{draft}\||fi|
\end{center}
%
which can be uncommented to produce a draft version.
Likewise one can add a line to the very top of a child file
(above the |\childdocof{|\textit{main}|}| directive)
%
\begin{center}
|%\providecommand{\version}{final}|
\end{center}
%
which can be uncommented to produce the final version of this child document.

%%%%%%%%%%%%%%%%%%%%%%%%%%%%%%%%%%%%%%%%%%%%%%%%%%%%%%%%%%%%%%%%%%%%%%%%%%%%%%%%
\subsection{Forwarding}
\label{sec:forward}

Different versions of the main or child documents
using compilation flags as described in \secref{sec:flags}
can be (permanently) stored in different files
for convenient compilation, viewing and distribution.
To this end, the package defines a command
to pass on compilation to a different file:

%%%%%%%%%%%%%%%%%%%%%%%%%%%%%%%%%%%%%%%%
\DescribeMacro{\childdocforward}
The command |\childdocforward| redirects processing to
another source file:
%
\begin{center}
\begin{tabular}{l}
|\input{childdoc.def}|\\
|\childdocforward[|\textit{main}|]{|\textit{dest}|}|\\
\end{tabular}
\end{center}
%
The argument \textit{dest} is the destination file
(without extension).
It should be the main file or one of the child files.
Note that further \textsf{childdoc} directives
such as |\childdocof| and |\childdocforward|
in the indicated file will be processed in this form.
The optional argument \textit{main}
passes on directly to the main file \textit{main}
while pretending to compile the child \textit{dest}.
This form behaves as if \textit{dest}
issues |\childdocof{|\textit{main}|}| right away,
and no further \textsf{childdoc} directives will be processed.

%%%%%%%%%%%%%%%%%%%%%%%%%%%%%%%%%%%%%%%%
\DescribeMacro{\...prefix}
In the alternative form |\childdocforwardprefix|,
%
\begin{center}
\begin{tabular}{l}
|\input{childdoc.def}|\\
|\childdocforwardprefix[|\textit{main}|]{|\textit{prefix}|}{|\textit{dest}|}|
\end{tabular}
\end{center}
%
the destination file is determined by a pattern
depending on the current file:
To make this work, the current file must be called
`{\textit{prefix}\hspace{0.2em}\textit{suffix}}'
with \textit{prefix} matching precisely the argument.
Processing is then passed on to the file
`{\textit{dest}\hspace{0.2em}\textit{suffix}}'.
Surely, the same effect is achieved by
directly specifying the
argument `{\textit{dest}\hspace{0.2em}\textit{suffix}}'
in the first form.
However, that requires to set up a different file
for each child. With the alternative form of the command
all these files can have exactly the same content
which simplifies setting them up and maintaining them.

For example, the following file |draft.tex|
with a compilation flag |\version| as described in \secref{sec:flags}
compiles the main document as a draft:
%
\begin{center}
\begin{tabular}{l}
|\def\version{draft}|\\
|\input{childdoc.def}|\\
|\childdocforward{|\textit{main}|}|
\end{tabular}
\end{center}
%
Likewise, the following files |final|\textit{nn}|.tex|
compile the final version of the child document
|child|\textit{nn}|.tex|:
%
\begin{center}
\begin{tabular}{l}
|\def\version{final}|\\
|\input{childdoc.def}|\\
|\childdocforwardprefix{final}{child}|
\end{tabular}
\end{center}
%

Note that when several versions of a main file and/or of each child file
are to be generated, it may be convenient to set up a |Makefile| or
shell script to automatise the process.

%%%%%%%%%%%%%%%%%%%%%%%%%%%%%%%%%%%%%%%%%%%%%%%%%%%%%%%%%%%%%%%%%%%%%%%%%%%%%%%%
\subsection{Command Line Processing}
\label{sec:commandline}

The effect of redirection files can also be achieved by invoking
the \LaTeX{} compiler with a more elaborate command line.
Most conveniently this should be done as part
of a shell script or a |Makefile|.

When using \textsf{childdoc} in the main file, the following
command lines effectively perform a redirection
(note that depending on the shell being used,
backslashes may have to be doubled: `|\|' $\to$ `|\\|'):
%
\begin{center}
|... -jobname "|\textit{target}|" |\\|"|[\textit{flags}]%
|\input{childdoc.def}\childdocforward[|\textit{main}|]{|\textit{dest}|}"|
\end{center}
%
Here \textit{target} is the name of the output file,
\textit{main} is the name of the main file
and \textit{dest} is the name of the main or child file to be processed
(all filenames without extensions).
The optional argument \textit{main} can be omitted
if \textit{main} matches \textit{dest}.
Optionally, compilation \textit{flags} can be defined via |\def| commands.
This command line makes the \TeX{} engine believe
it is compiling the file \textit{target}
whose content is specified as the latter parameter.
The provided code then forwards the processing to
\textit{main} or \textit{dest} as described in \secref{sec:forward}.

%%%%%%%%%%%%%%%%%%%%%%%%%%%%%%%%%%%%%%%%%%%%%%%%%%%%%%%%%%%%%%%%%%%%%%%%%%%%%%%%
\subsection{Include by Input}
\label{sec:input}

Including child documents by |\include| has some restrictions by design.
Most notably, the content of a child document always occupies
its own set of pages; pages cannot be shared between child documents.
Usually, this behaviour makes perfect sense
because each child document contain an essential part of the document.
However, in some situations it may be desirable to compose
a document from a collection of parts
without having mandatory page breaks between then.
For this case, the package
provides a mechanism to include parts
by |\input| which can also be processed individually.
However, by construction this mechanism
requires manual handling of the content to be output.

%%%%%%%%%%%%%%%%%%%%%%%%%%%%%%%%%%%%%%%%
\DescribeMacro{\ifchilddocmanual}
The main file should be prepared as usual, see \secref{sec:include}.
However, the document body must make a distinction
between processing of an individual part and of the main document, e.g.:
%
\begin{center}
\begin{tabular}{l}
|\ifchilddocmanual|\\
|\input{\childdocname}|\\
|\||else|\\
\textit{document body with }|\input{|\textit{part}|}|\\
|\||fi|
\end{tabular}
\end{center}
%
The conditional |\ifchilddocmanual| is true whenever
a part to be included by |\input| is being compiled,
and the name of the part is stored in |\childdocname|.

%%%%%%%%%%%%%%%%%%%%%%%%%%%%%%%%%%%%%%%%
\DescribeMacro{\childdocby}
Each part to be included by |\input| should start with:
%
\begin{center}
\begin{tabular}{l}
|\input{childdoc.def}|\\
|\childdocby{|\textit{main}|}|\\
\end{tabular}
\end{center}
%
The directive |\childdocby| is similar to |\childdocof|
described in \secref{sec:include},
but the subsequent selection of content must be done manually.
To that end, both |\ifchilddoc| and |\ifchilddocmanual|
will be true upon processing of a part,
and the name of the part is stored in |\childdocname|.
Note that |\jobname| will be set to the filename of the current part
so that each part receives an individual |.aux| file
that does not interfere with the |.aux| file(s) of the main document.
This behaviour can be altered by the alternative form
|\childdocby[*]{|\textit{main}|}| (with a non-empty optional argument)
which uses the |.aux| file of the main document
by setting |\jobname| to \textit{main}.

%%%%%%%%%%%%%%%%%%%%%%%%%%%%%%%%%%%%%%%%%%%%%%%%%%%%%%%%%%%%%%%%%%%%%%%%%%%%%%%%
\subsection{Driver Development}
\label{sec:driver}

The \textsf{childdoc} mechanism can also be use for the development
of definition files such as \LaTeX{} styles or classes.
This case differs from the above setup with multiple parts
included by |\include| in that no |\includeonly| should be invoked.
This can be achieved by starting the include file
(before |\ProvidesPackage|) with:
%
\begin{center}
\begin{tabular}{l}
|\input{childdoc.def}|\\
|\childdocforward{|\textit{main}|}|\\
\end{tabular}
\end{center}
%
or alternatively with:
%
\begin{center}
\begin{tabular}{l}
|\input{childdoc.def}|\\
|\childdocby{|\textit{main}|}|\\
\end{tabular}
\end{center}
%
Both forms have slightly different effects as described above.
The main file is prepared as usual, see \secref{sec:include}.

%%%%%%%%%%%%%%%%%%%%%%%%%%%%%%%%%%%%%%%%%%%%%%%%%%%%%%%%%%%%%%%%%%%%%%%%%%%%%%%%
\subsection{Legacy Detection}
\label{sec:detection}

The directive |\childdocmain| in the main file can detect
whether the complete document or merely a child is to be compiled
even without using the directive |\childdocof|.
This method is deprecated because it is less robust
and there is no compelling reason to use it;
it is merely provided for backward compatibility
and it may be removed in future versions.

If the detection mechanism is to be used,
it is mandatory to correctly specify
the filename of the main file as the argument of |\childdocmain|:
%
\begin{center}
\begin{tabular}{l}
|\input{childdoc.def}|\\
|\childdocmain{|\textit{main}|}|\\
\end{tabular}
\end{center}
%
If |\jobname| does not match the argument \textit{main} of |\childdocmain|,
it is assumed that |\jobname| points to the child file to be compiled.
When using |\childdocmain| with the main file specified as argument,
it suffices to start a child file
with just |\input{|\textit{main}|}|
without loading of the package and using |\childdocof|.
If instead all processing is done
with the appropriate \textsf{childdoc} directives,
the argument of \textit{main} of |\childdocmain| can be empty.

An alternative version of the command line processing described
in \secref{sec:commandline} using the detection mechanism reads:
%
\begin{center}
|... -jobname "|\textit{target}|" "|[\textit{flags}]%
[|\def\jobname{|\textit{dest}|}|]|\input{|\textit{main}|}"|
\end{center}

%%%%%%%%%%%%%%%%%%%%%%%%%%%%%%%%%%%%%%%%%%%%%%%%%%%%%%%%%%%%%%%%%%%%%%%%%%%%%%%%
\subsection{Manual Code}
\label{sec:manual}

In case one cannot be certain whether the definitions file |childdoc.def|
is installed on the target \TeX{} distribution
and one prefers not to ship it,
it is conceivable to paste a few relevant commands into the sources.

To that end, drop all statements |\input{childdoc.def}|
and perform the replacements as outlined below.
Instead of |\childdocmain{|\textit{main}|}| add the following code
to the top of the main file:
%
\begin{center}
\begin{tabular}{l}
|\||ifdefined\childdocname\endinput\||fi\newif\ifchilddoc|\\
|\edef\childdocname{\scantokens\expandafter{\jobname\noexpand}}|\\
|\def\childdocmain{|\textit{main}|}\||ifx\childdocmain\childdocname\||else|\\
|\childdoctrue\includeonly{\childdocname}\let\jobname\childdocmain\||fi|\\
\end{tabular}
\end{center}
%
Instead of |\childdocof{|\textit{main}|}| just include the main file
at the top of each child file:
%
\begin{center}
|\input{|\textit{main}|}|
\end{center}
%
A simple redirection |\childdocforward{|\textit{dest}|}| is achieved by:
%
\begin{center}
|\def\jobname{|\textit{dest}|}\input{\jobname}|
\end{center}
%
The redirection with prefix
|\childdocforwardprefix[|\textit{prefix}|]{|\textit{dest}|}|
is accomplished by:
%
\begin{center}
\begin{tabular}{l}
|{\edef\jobname{\scantokens\expandafter{\jobname\noexpand}}|\\
|\def\redirectjob |\textit{prefix}|#1~~~{\gdef\jobname{|\textit{dest}|#1}}|\\
|\expandafter\redirectjob\jobname~~~}\input{\jobname}|
\end{tabular}
\end{center}

In an alternative approach,
child documents can be compiled by a specific command line
without additional code or specific definitions:
%
\begin{center}
|... -jobname "|\textit{target}|" "|[\textit{flags}]%
|\includeonly{|\textit{dest}|}\input{|\textit{main}|}"|
\end{center}
%

%%%%%%%%%%%%%%%%%%%%%%%%%%%%%%%%%%%%%%%%%%%%%%%%%%%%%%%%%%%%%%%%%%%%%%%%%%%%%%%%
%%%%%%%%%%%%%%%%%%%%%%%%%%%%%%%%%%%%%%%%%%%%%%%%%%%%%%%%%%%%%%%%%%%%%%%%%%%%%%%%
\section{Information}

%%%%%%%%%%%%%%%%%%%%%%%%%%%%%%%%%%%%%%%%%%%%%%%%%%%%%%%%%%%%%%%%%%%%%%%%%%%%%%%%
\subsection{Copyright}

Copyright \copyright{} 2017--2018 Niklas Beisert

This work may be distributed and/or modified under the
conditions of the \LaTeX{} Project Public License, either version 1.3
of this license or (at your option) any later version.
The latest version of this license is in
  \url{http://www.latex-project.org/lppl.txt}
and version 1.3 or later is part of all distributions of \LaTeX{}
version 2005/12/01 or later.

This work has the LPPL maintenance status `maintained'.

The Current Maintainer of this work is Niklas Beisert.

This work consists of the files |README.txt|, |childdoc.ins| and |childdoc.dtx|
as well as the derived files |childdoc.def|, |cdocsamp.tex|
with |cdocsch1.tex|, |cdocsch2.tex|, |cdocspt3.tex|, |cdocspt4.tex|,
|cdocsdrf.tex|, |cdocsfn1.tex|, |cdocsfn2.tex|
as well as |childdoc.pdf|.

%%%%%%%%%%%%%%%%%%%%%%%%%%%%%%%%%%%%%%%%%%%%%%%%%%%%%%%%%%%%%%%%%%%%%%%%%%%%%%%%
\subsection{Files and Installation}

The package consists of the files:
%
\begin{center}
\begin{tabular}{ll}
    |README.txt|   & readme file \\
    |childdoc.ins| & installation file \\
    |childdoc.dtx| & source file \\
    |childdoc.def| & definition file \\
    |cdocsamp.tex| & sample main file \\
    |cdocsch1.tex| & sample include file \\
    |cdocsch2.tex| & sample include file \\
    |cdocspt3.tex| & sample part file \\
    |cdocspt4.tex| & sample part file \\
    |cdocsdrf.tex| & sample redirection file \\
    |cdocsfn1.tex| & sample redirection file \\
    |cdocsfn2.tex| & sample redirection file \\
    |childdoc.pdf| & manual
\end{tabular}
\end{center}
%
The distribution consists of the files
|README.txt|, |childdoc.ins| and |childdoc.dtx|.
%
\begin{itemize}
\item
Run (pdf)\LaTeX{} on |childdoc.dtx|
to compile the manual |childdoc.pdf| (this file).
\item
Run \LaTeX{} on |childdoc.ins| to create the definitions file |childdoc.def|
and the sample |cdocsamp.tex| with include files
|cdocsch1.tex|, |cdocsch2.tex|, |cdocspt3.tex|, |cdocspt4.tex|,
|cdocsdrf.tex|, |cdocsfn1.tex|, |cdocsfn2.tex|.
Then copy the file |childdoc.def| to an appropriate directory of your \LaTeX{}
distribution, e.g.\ \textit{texmf-root}|/tex/latex/childdoc|.
\end{itemize}

%%%%%%%%%%%%%%%%%%%%%%%%%%%%%%%%%%%%%%%%%%%%%%%%%%%%%%%%%%%%%%%%%%%%%%%%%%%%%%%%
\subsection{Related CTAN Packages}

There are several other packages which offer a similar functionality:
%
\begin{itemize}
\item
The packages
\href{http://ctan.org/pkg/docmute}{\textsf{docmute}},
\href{http://ctan.org/pkg/includex}{\textsf{includex}} and
\href{http://ctan.org/pkg/standalone}{\textsf{standalone}}
provide commands to include only the document body of
a child file thus allowing both files to be compiled individually.
\item
The packages \href{http://ctan.org/pkg/subdocs}{\textsf{subdocs}}
and \href{http://ctan.org/pkg/subfiles}{\textsf{subfiles}}
provide structures in which the main and child documents can be
encapsulated and allowing them to be compiled individually.
The inclusion mechanism is different from the conventional |\include|.
\item
The package \href{http://ctan.org/pkg/combine}{\textsf{combine}}
is an elaborate solution to combine several documents into one.
\end{itemize}
%
See also the CTAN topic \href{http://ctan.org/topic/subdocs}{\textsf{subdocs}}
for further related packages.
The present package differs from the above solutions in that
a document structure constructed with the conventional |\include| mechanism
just needs two extra commands at the top of every file
such that all constituent files can be compiled individually.

%%%%%%%%%%%%%%%%%%%%%%%%%%%%%%%%%%%%%%%%%%%%%%%%%%%%%%%%%%%%%%%%%%%%%%%%%%%%%%%%
%\subsection{Feature Suggestions}
%
%The following is a list of features which may be useful for future
%versions of this package:
%%
%\begin{itemize}
%\item
%\ldots
%\end{itemize}

%%%%%%%%%%%%%%%%%%%%%%%%%%%%%%%%%%%%%%%%%%%%%%%%%%%%%%%%%%%%%%%%%%%%%%%%%%%%%%%%
\subsection{Revision History}

%%%%%%%%%%%%%%%%%%%%%%%%%%%%%%%%%%%%%%%%
\paragraph{v2.0:} 2018/12/30

\begin{itemize}
\item
immediate forward processing
\item
added |\childdocby| mechanism
\item
manual restructured
\end{itemize}

%%%%%%%%%%%%%%%%%%%%%%%%%%%%%%%%%%%%%%%%
\paragraph{v1.6:} 2018/01/17

\begin{itemize}
\item
application for development of include files
\item
corrections to manual
\end{itemize}

%%%%%%%%%%%%%%%%%%%%%%%%%%%%%%%%%%%%%%%%
\paragraph{v1.5:} 2017/05/21

\begin{itemize}
\item
more complete structuring introduced
\item
|\childdocof| introduced
\item
|\childdoc| renamed to |\childdocmain|
\item
|\childredirect| renamed to |\childdocforward| and |\childdocforwardprefix|
and functionality expanded
\end{itemize}

%%%%%%%%%%%%%%%%%%%%%%%%%%%%%%%%%%%%%%%%
\paragraph{v1.0:} 2017/04/27

\begin{itemize}
\item
manual and install package
\item
first version published on CTAN
\end{itemize}

%%%%%%%%%%%%%%%%%%%%%%%%%%%%%%%%%%%%%%%%
\paragraph{v0.6:} 2017/04/26

\begin{itemize}
\item
redirection mechanism added
\end{itemize}

%%%%%%%%%%%%%%%%%%%%%%%%%%%%%%%%%%%%%%%%
\paragraph{v0.5:} 2017/04/26

\begin{itemize}
\item
functionality in definition file
\end{itemize}


%%%%%%%%%%%%%%%%%%%%%%%%%%%%%%%%%%%%%%%%%%%%%%%%%%%%%%%%%%%%%%%%%%%%%%%%%%%%%%%%
%%%%%%%%%%%%%%%%%%%%%%%%%%%%%%%%%%%%%%%%%%%%%%%%%%%%%%%%%%%%%%%%%%%%%%%%%%%%%%%%
%%%%%%%%%%%%%%%%%%%%%%%%%%%%%%%%%%%%%%%%%%%%%%%%%%%%%%%%%%%%%%%%%%%%%%%%%%%%%%%%
\appendix

\settowidth\MacroIndent{\rmfamily\scriptsize 000\ }

 \DocInput{childdoc.dtx}

\end{document}
%</driver>
% \fi
%
% %%%%%%%%%%%%%%%%%%%%%%%%%%%%%%%%%%%%%%%%%%%%%%%%%%%%%%%%%%%%%%%%%%%%%%%%%%%%%%
% %%%%%%%%%%%%%%%%%%%%%%%%%%%%%%%%%%%%%%%%%%%%%%%%%%%%%%%%%%%%%%%%%%%%%%%%%%%%%%
% \section{Sample}
%\iffalse
%<*samplemain>
%\fi
%
% The following presents a sample document
% with two chapters, two parts, a title page,
% a compile flag as well as three forwarding files to set the flag.
% It consists of eight |.tex| files:
% \begin{center}
% \begin{tabular}{ll}
% |cdocsamp.tex|&main file\\
% |cdocsch1.tex|&include file for chapter 1\\
% |cdocsch2.tex|&include file for chapter 2\\
% |cdocspt3.tex|&include file for part 3\\
% |cdocspt4.tex|&include file for part 4\\
% |cdocsdrf.tex|&forwarding file for main file in draft mode\\
% |cdocsfi1.tex|&forwarding file for final version of chapter 1\\
% |cdocsfi2.tex|&forwarding file for final version of chapter 2\\
% \end{tabular}
% \end{center}
% Each of the eight files can be compiled directly by the \LaTeX{} compiler.
%
% %%%%%%%%%%%%%%%%%%%%%%%%%%%%%%%%%%%%%%
% \paragraph{Main File.}
%
% The main file is called |cdocsamp.tex|.
%
% Load the \textsf{childdoc} definitions and
% declare the filename for the main document:
%    \begin{macrocode}
\input{childdoc.def}
\childdocmain{}
%    \end{macrocode}

% Optional override for |\version| flag:
%    \begin{macrocode}
%%\ifchilddoc\else\providecommand{\version}{draft}\fi
%    \end{macrocode}

% Define the default values for the |\version| flag
% (|final| for the main file and |draft| for childs):
%    \begin{macrocode}
\ifchilddoc
\providecommand{\version}{draft}
\else
\providecommand{\version}{final}
\fi
%    \end{macrocode}

% Load the standard document class:
%    \begin{macrocode}
\documentclass[12pt]{article}
%    \end{macrocode}

% Start the document body:
%    \begin{macrocode}
\begin{document}
%    \end{macrocode}

% Declare a title page.
% Print title, part of document being processed and version flag:
%    \begin{macrocode}
\addtocounter{page}{-1}
\begin{center}
{\LARGE\bfseries{}childdoc example\par}
\vspace{1cm}
\ifchilddoc
\ifchilddocmanual part\else chapter\fi:
`\childdocname' of `\childdocjob'\par
\else
main document: `\childdocjob'\par
\fi
version: \version\par
\end{center}
\newpage
%    \end{macrocode}

% Manually include selected file,
% otherwise process as usual:
%    \begin{macrocode}
\ifchilddocmanual
\section*{part `\childdocname'}
\input{\childdocname}
\else
%    \end{macrocode}

% Include the two chapters:
%    \begin{macrocode}
\include{cdocsch1}
\include{cdocsch2}
%    \end{macrocode}

% Include the two parts unless only chapters should be displayed:
%    \begin{macrocode}
\ifchilddoc\else
\section{part three}
\input{cdocspt3}
\section{part four}
\input{cdocspt4}
\fi
%    \end{macrocode}

% Process as usual until here:
%    \begin{macrocode}
\fi
%    \end{macrocode}

% End of document body:
%    \begin{macrocode}
\end{document}
%    \end{macrocode}
%\iffalse
%</samplemain>
%\fi
%
% %%%%%%%%%%%%%%%%%%%%%%%%%%%%%%%%%%%%%%
% \paragraph{Chapter Include Files.}
%
% The include files are called |cdocsch1.tex| and |cdocsch2.tex|.
%
%\iffalse
%<*samplechap1|samplechap2>
%\fi

% Optional override for |\version| flag:
%    \begin{macrocode}
%%\providecommand{\version}{final}
%    \end{macrocode}

% Include the main document:
%    \begin{macrocode}
\input{childdoc.def}
\childdocof{cdocsamp}
%    \end{macrocode}

%\iffalse
%</samplechap1|samplechap2>
%\fi
%
%\iffalse
%<*samplechap1>
%\fi
% Some text for chapter 1:
%    \begin{macrocode}
\section{one}
some text in chapter one
%    \end{macrocode}

%\iffalse
%</samplechap1>
%\fi
% Some text for chapter 2:
%\iffalse
%<*samplechap2>
%\fi
%    \begin{macrocode}
\section{two}
more text in chapter two
%    \end{macrocode}

%\iffalse
%</samplechap2>
%\fi
%
% %%%%%%%%%%%%%%%%%%%%%%%%%%%%%%%%%%%%%%
% \paragraph{Part Include Files.}
%
% The include files are called |cdocspt3.tex| and |cdocspt4.tex|.
%
%\iffalse
%<*samplepart3|samplepart4>
%\fi

% Optional override for |\version| flag:
%    \begin{macrocode}
%%\providecommand{\version}{final}
%    \end{macrocode}

% Include the main document:
%    \begin{macrocode}
\input{childdoc.def}
\childdocby{cdocsamp}
%    \end{macrocode}

%\iffalse
%</samplepart3|samplepart4>
%\fi
%
%\iffalse
%<*samplepart3>
%\fi
% Some text for part 3:
%    \begin{macrocode}
some text in part three
%    \end{macrocode}

%\iffalse
%</samplepart3>
%\fi
% Some text for part 4:
%\iffalse
%<*samplepart4>
%\fi
%    \begin{macrocode}
more text in part four
%    \end{macrocode}

%\iffalse
%</samplepart4>
%\fi
%
% %%%%%%%%%%%%%%%%%%%%%%%%%%%%%%%%%%%%%%
% \paragraph{Forwarding for a Complete Draft.}
%
% The following forwarding file |cdocsdrf.tex|
% compiles the main document in draft mode:
%\iffalse
%<*sampledraft>
%\fi
%    \begin{macrocode}
\def\version{draft}
\input{childdoc.def}
\childdocforward{cdocsamp}
%    \end{macrocode}

%\iffalse
%</sampledraft>
%\fi
%
% %%%%%%%%%%%%%%%%%%%%%%%%%%%%%%%%%%%%%%
% \paragraph{Forwarding for Final Version of the Chapters.}
%
% The following forwarding files |cdocsfn1.tex| and |cdocsfn2.tex|
% (with identical content)
% compile the final versions of the child documents
% |cdocsch1.tex| and |cdocsch2.tex|, respectively:
%\iffalse
%<*samplefinal>
%\fi
%    \begin{macrocode}
\def\version{final}
\input{childdoc.def}
\childdocforwardprefix[cdocsamp]{cdocsfn}{cdocsch}
%    \end{macrocode}

%\iffalse
%</samplefinal>
%\fi
%
% %%%%%%%%%%%%%%%%%%%%%%%%%%%%%%%%%%%%%%
% \paragraph{Command Line Processing.}
%
% The following three command lines generate the output files
% |cdocscld|, |cdocscl1| and |cdocscl2|
% which should be identical to
% |cdocsdrf|, |cdocsch1| and |cdocsfn2|, respectively:
% \begin{center}
% \begin{tabular}{l}
% |latex -jobname cdocscld \|\\
% |  "\def\version{draft}\input{childdoc.def}\childdocforward{cdocsamp}"|\\
% |latex -jobname cdocscl1 \|\\
% |  "\input{childdoc.def}\childdocforward[cdocsamp]{cdocsch1}"|\\
% |latex -jobname cdocscl2 \|\\
% |  "\def\version{final}\input{childdoc.def}\childdocforward{cdocsch2}"|
% \end{tabular}
% \end{center}
% Note that the trailing backslash on each first line
% merely continues the input to the second line
% (for convenient cut ant paste).
% Furthermore, the command |latex| can be replaced by any
% of its alternative versions such as |pdflatex|.
%
% %%%%%%%%%%%%%%%%%%%%%%%%%%%%%%%%%%%%%%%%%%%%%%%%%%%%%%%%%%%%%%%%%%%%%%%%%%%%%%
% %%%%%%%%%%%%%%%%%%%%%%%%%%%%%%%%%%%%%%%%%%%%%%%%%%%%%%%%%%%%%%%%%%%%%%%%%%%%%%
% \section{Implementation}
%\iffalse
%<*package>
%\fi
%
% This section describes the definitions file |childdoc.def|.

% The definitions cannot be loaded using |\usepackage| or |\RequirePackage|
% which has a mechanism to prevent loading a style file more than once.
% When loading the definitions by means of |\input|
% multiple instances have to be prevented manually:
%\iffalse
%This code needs to be before the `\ProvidesFile' directive
%which is defined at the beginning of this file.
%Therefore it is also placed there and commented out here.
%</package>
%<*discard>
%\fi
%    \begin{macrocode}
\ifdefined\childdocmain\endinput\fi
%    \end{macrocode}
%\iffalse
%</discard>
%<*package>
%\fi
%
% \macro{\ifchilddoc}
% \macro{\ifchilddocmanual}
% The conditional |\ifchilddoc| tells whether a
% child (true) or main (false) document is being compiled.
% The conditional |\ifchilddocmanual| tells whether
% the |\includeonly| mechanism is used (false) or
% the selection of child files must be performed manually (true).
% The definitions initialise to false:
%    \begin{macrocode}
\newif\ifchilddoc
\newif\ifchilddocmanual
%    \end{macrocode}

% \macro{\childdocname}
% \macro{\childdocjob}
% The macro |\childdocname| stores the name of the main document
% to be compiled. The macro |\childdocjob| stores the name of
% the document on which the \LaTeX{} compiler was originally invoked.
% The content of |\jobname| cannot be compared
% to filenames specified in the source due to different catcodes.
% The following code rescans |\jobname|, stores the result
% in |\childdocname| and saves a copy in |\childdocjob|:
%    \begin{macrocode}
\edef\childdocname{\scantokens\expandafter{\jobname\noexpand}}
\let\childdocjob\childdocname
%    \end{macrocode}

% \macro{\childdocdisable}
% The macro |\childdocdisable| prevents the main file
% from being processed more than once.
% At this stage, the main document command |\childdocmain|
% is assumed to be called once again where it should do nothing.
% Any subsequent call to it should prevent
% a secondary processing of the main document
% It overwrites the forwarding commands
% |\childdocof| and |\childdocforward|
% with empty macros to prevent further inclusions of the main document:
%    \begin{macrocode}
\newcommand{\childdocdisable}
{
  \renewcommand{\childdocmain}[1]{\renewcommand{\childdocmain}[1]{\endinput}}
  \renewcommand{\childdocof}[1]{}
  \renewcommand{\childdocby}[2][]{}
  \renewcommand{\childdocforward}[2][]{}
  \renewcommand{\childdocdisable}{}
}
%    \end{macrocode}

% \macro{\childdocmain}
% The macro |\childdocmain| is to be called at the top of the main file
% with nothing or the main filename (without extension) as argument.
% First, it breaks loops.
% If the argument is not empty and does not match |\childdocname|
% (which is set by the first inclusion of |childdoc.def|),
% |\ifchilddoc| is set to true, |\includeonly| is applied to the child file
% and |\jobname| is set to the main file
% (for proper handling of |.aux| files):
%    \begin{macrocode}
\newcommand{\childdocmain}[1]
{
  \childdocdisable\childdocmain{}
  \if?#1?\else
    \begingroup
      \def\childdoctmp{#1}
      \ifx\childdoctmp\childdocname
        \def\childdoctmp{}
      \else
        \def\childdoctmp
        {
          \childdoctrue
          \includeonly{\childdocname}
          \def\childdocjob{#1}
          \def\jobname{#1}
        }
      \fi
      \expandafter
    \endgroup
    \childdoctmp
  \fi
}
%    \end{macrocode}

% \macro{\childdocof}
% The command |\childdocof| redirects
% compilation to the main file |#1|.
%    \begin{macrocode}
\newcommand{\childdocof}[1]
{
  \childdocdisable
  \childdoctrue
  \includeonly{\childdocname}
  \def\jobname{#1}
  \def\childdocjob{#1}
  \input{#1}
}
%    \end{macrocode}

% \macro{\childdocby}
% The command |\childdocby| ....
%    \begin{macrocode}
\newcommand{\childdocby}[2][]
{
  \childdocdisable
  \childdoctrue
  \childdocmanualtrue
  \if?#1?\else
    \def\jobname{#2}
  \fi
  \def\childdocjob{#2}
  \input{#2}
  \endinput
}
%    \end{macrocode}

% \macro{\childdocforward}
% The command |\childdocforward| redirects
% compilation to the main file or
% (if the optional argument is given) a child file.
% Parameters are set as if the main file
% or a child file starting with |\childdocof| was compiled.
% Then compilation is handed over to the main file:
%    \begin{macrocode}
\newcommand{\childdocforward}[2][]
{
  \begingroup
    \if?#1?
      \def\childdoctmp
      {
        \def\childdocname{#2}
        \def\childdocjob{#2}
        \def\jobname{#2}
        \input{#2}
        \endinput
      }
    \else
      \def\childdoctmp
      {
        \childdocdisable
        \def\childdocname{#2}
        \childdoctrue
        \includeonly{#2}
        \def\childdocjob{#1}
        \def\jobname{#1}
        \input{#1}
        \endinput
      }
    \fi
    \expandafter
  \endgroup
  \childdoctmp
}
%    \end{macrocode}

% \macro{\childdocforwardprefix}
% The command |\childdocforwardprefix| redirects
% compilation to the main or a child file by means of a pattern.
% The prefix |#1| in the current filename is replaced by |#2|
% and the suffix of the current filename is kept
% (it is assumed that the filename does not contain the substring `|~~~|'
% which is used as a delimiter).
% Compilation is handed over to the new file by |\childdocforward|:
%    \begin{macrocode}
\newcommand{\childdocforwardprefix}[3][]
{
  \begingroup
    \def\childdocextract #2##1~~~{\def\childdoctmp{\childdocforward[#1]{#3##1}}}
    \expandafter\childdocextract\childdocname~~~
    \expandafter
  \endgroup
  \childdoctmp
}
%    \end{macrocode}

% \macro{\childdoc}
% The deprecated macro |\childdoc| is a legacy version of |\childdocmain|:
%    \begin{macrocode}
\newcommand{\childdoc}{\childdocmain}
%    \end{macrocode}

% \macro{\childdocredirect}
% The deprecated macro |\childdocredirect| is a legacy version
% of |\childdocforward| and |\childdocforwardprefix|:
%    \begin{macrocode}
\newcommand{\childdocredirect}[2][]
{
  \begingroup
    \if?#1?
      \def\childdoctmp{\childdocforward{#2}}
    \else
      \def\childdoctmp{\childdocforwardprefix{#1}{#2}}
    \fi
    \expandafter
  \endgroup
  \childdoctmp
}
%    \end{macrocode}

%\iffalse
%</package>
%\fi
%
\endinput
\childdocforward[cdocsamp]{cdocsch1}"|\\
% |latex -jobname cdocscl2 \|\\
% |  "\def\version{final}% \iffalse
%
% childdoc.dtx Copyright (C) 2017-2018 Niklas Beisert
%
% This work may be distributed and/or modified under the
% conditions of the LaTeX Project Public License, either version 1.3
% of this license or (at your option) any later version.
% The latest version of this license is in
%   http://www.latex-project.org/lppl.txt
% and version 1.3 or later is part of all distributions of LaTeX
% version 2005/12/01 or later.
%
% This work has the LPPL maintenance status `maintained'.
%
% The Current Maintainer of this work is Niklas Beisert.
%
% This work consists of the files childdoc.dtx and childdoc.ins
% and the derived files childdoc.def and cdocsamp.tex with
% cdocsch1.tex, cdocsch2.tex, cdocsdrf.tex, cdocsfn1.tex, cdocsfn2.tex.
%
%<package>\ifdefined\childdocmain\endinput\fi
%<package>\ProvidesFile{childdoc.def}[2018/12/30 v2.0 child document driver]
%<samplemain>\ProvidesFile{cdocsamp.tex}[2018/12/30 v2.0 sample for childdoc]
%<*driver>
%\ProvidesFile{childdoc.drv}[2018/12/30 v2.0 childdoc reference manual file]
\PassOptionsToClass{10pt,a4paper}{article}
\documentclass{ltxdoc}

\usepackage[margin=35mm]{geometry}
\usepackage{hyperref}
\usepackage{hyperxmp}
\usepackage[usenames]{color}

\hypersetup{colorlinks=true}
\hypersetup{pdfstartview=FitH}
\hypersetup{pdfpagemode=UseNone}
\hypersetup{pdfsource={}}
\hypersetup{pdflang={en-UK}}
\hypersetup{pdfcopyright={Copyright 2017-2018 Niklas Beisert.
  This work may be distributed and/or modified under the
  conditions of the LaTeX Project Public License, either version 1.3
  of this license or (at your option) any later version.}}
\hypersetup{pdflicenseurl={http://www.latex-project.org/lppl.txt}}
\hypersetup{pdfcontactaddress={ETH Zurich, ITP, HIT K,
  Wolfgang-Pauli-Strasse 27}}
\hypersetup{pdfcontactpostcode={8093}}
\hypersetup{pdfcontactcity={Zurich}}
\hypersetup{pdfcontactcountry={Switzerland}}
\hypersetup{pdfcontactemail={nbeisert@itp.phys.ethz.ch}}
\hypersetup{pdfcontacturl={http://people.phys.ethz.ch/\xmptilde nbeisert/}}

\newcommand{\secref}[1]{\hyperref[#1]{section \ref*{#1}}}

\parskip1ex
\parindent0pt
\let\olditemize\itemize
\def\itemize{\olditemize\parskip0pt}

\begin{document}

\title{The \textsf{childdoc} Package}
\hypersetup{pdftitle={The childdoc Package}}
\author{Niklas Beisert\\[2ex]
  Institut f\"ur Theoretische Physik\\
  Eidgen\"ossische Technische Hochschule Z\"urich\\
  Wolfgang-Pauli-Strasse 27, 8093 Z\"urich, Switzerland\\[1ex]
  \href{mailto:nbeisert@itp.phys.ethz.ch}
  {\texttt{nbeisert@itp.phys.ethz.ch}}}
\hypersetup{pdfauthor={Niklas Beisert}}
\hypersetup{pdfsubject={Manual for the LaTeX2e Package childdoc}}
\date{30 December 2018, \textsf{v2.0}}
\maketitle

\begin{abstract}\noindent
\textsf{childdoc} is a \LaTeXe{} package
that enables the direct compilation
of document sections included by |\include|
to individual files.
\end{abstract}

\begingroup
\parskip0ex
\tableofcontents
\endgroup

%%%%%%%%%%%%%%%%%%%%%%%%%%%%%%%%%%%%%%%%%%%%%%%%%%%%%%%%%%%%%%%%%%%%%%%%%%%%%%%%
%%%%%%%%%%%%%%%%%%%%%%%%%%%%%%%%%%%%%%%%%%%%%%%%%%%%%%%%%%%%%%%%%%%%%%%%%%%%%%%%
\section{Introduction}

\LaTeX{} provides a mechanism to structure a large document (such as a book)
into a main file and several child files (containing the chapters)
using the |\include| command.
This mechanism is beneficial for documents
which span hundreds of pages in order to
make the source file(s) more manageable.
Moreover, compilation can be restricted to
selected child files by means of the |\includeonly| command.
The latter feature can be used to reduce the compilation time while editing
(this was significantly more useful in the earlier days of \LaTeX{})
or to generate a smaller document which is easier to navigate.
Another application of |\includeonly| is to generate
documents consisting of selected parts of the complete document.

However, there are a few drawbacks of the plain |\include| mechanism:
\begin{itemize}
\item
The child files cannot be compiled on their own,
they can only be compiled via the main file.
A naive editing environment
(such as a text editor with an option
to have the current file processed by \LaTeX)
may require one to switch to the main file before compiling;
attempting to compile the child file produces errors.
\item
The main file must be modified (each time)
to adjust the |\includeonly| command
to the present needs. This easily leaves the main file in a messy state.
\item
The generated document will always carry the filename
of the main document. This is inconvenient if
several child files are to be compiled and
to be kept for distribution.
\end{itemize}

The present package provides a simple interface
to make child files individually compilable by \LaTeX{}.
Compiling a child file then has the same effect as compiling
the main file with an |\includeonly| command
to select the appropriate child.
Moreover the generated document will carry the name of the child
rather than the main file.
This resolves all three above issues.

This feature is meant to make the editing of books,
thesis documents and lecture notes somewhat more convenient.
However, the package can also be used efficiently for
composing a series of documents (such as exercise sheets)
which are typically distributed individually.
It then assists the author in generating the individual documents
(potentially in different versions)
as well as a document containing the collected series.
Another application is in developing style files
or other kinds of included material
where compilation of the style file could redirect
to a sample or test file.

%%%%%%%%%%%%%%%%%%%%%%%%%%%%%%%%%%%%%%%%%%%%%%%%%%%%%%%%%%%%%%%%%%%%%%%%%%%%%%%%
%%%%%%%%%%%%%%%%%%%%%%%%%%%%%%%%%%%%%%%%%%%%%%%%%%%%%%%%%%%%%%%%%%%%%%%%%%%%%%%%
\section{Usage}

First of all, the package \textsf{childdoc} is \emph{not} a standard
\LaTeXe{} |.sty| style file! Therefore it needs to be invoked in
a non-standard way.

%%%%%%%%%%%%%%%%%%%%%%%%%%%%%%%%%%%%%%%%%%%%%%%%%%%%%%%%%%%%%%%%%%%%%%%%%%%%%%%%
\subsection{Included Files}
\label{sec:include}

%%%%%%%%%%%%%%%%%%%%%%%%%%%%%%%%%%%%%%%%
\DescribeMacro{\childdocmain}
To use the package, add the commands
\begin{center}
\begin{tabular}{l}
|\input{childdoc.def}|\\
|\childdocmain{}|\\
\end{tabular}
\end{center}
at the very top of the main \LaTeX{} file,
in particular \emph{before} the |\documentclass| statement!
The argument of |\childdocmain| should be left empty
(but it must be present).

%%%%%%%%%%%%%%%%%%%%%%%%%%%%%%%%%%%%%%%%
\DescribeMacro{\childdocof}
Furthermore, add the commands
\begin{center}
\begin{tabular}{l}
|\input{childdoc.def}|\\
|\childdocof{|\textit{main}|}|\\
\end{tabular}
\end{center}
at the top of every child file \textit{child}
which is included by |\include{|\textit{child}|}|
from within the main file
(or at least for those files to be compiled individually).
The argument \textit{main} must be the filename of the main file.

There are a couple of
considerations in setting up the main and child documents:

%%%%%%%%%%%%%%%%%%%%%%%%%%%%%%%%%%%%%%%%
\paragraph{Restrictions.}

Please note the following restrictions:
\begin{itemize}
\item
|\childdocmain| must be called with one argument \textit{main}
to ensure compatibility with earlier version of the package.
It must either be empty (|\childdocmain{}|)
or precisely match the filename of the main file in which it is specified.
See \secref{sec:detection} for further information.
\item
The filename \textit{main} must be specified without the |.tex| extension.
\item
The filename \textit{main} is case sensitive
(even in case-insensitive file systems)
due to internal string comparison.
\item
The argument \textit{main} should be fully expanded, it cannot be a macro.
\item
Subdirectories and special characters should be avoided in filenames.
\item
The command |\childdocmain{|\textit{main}|}| must be followed by a whitespace.
It should not be followed immediately by another command
or by a comment mark `|%|'.
This is because the \TeX{} parser reads the token immediately following
the argument of |\childdocmain| and puts it
at the beginning of every child section;
however, a white\-space is ignored.
\end{itemize}

%%%%%%%%%%%%%%%%%%%%%%%%%%%%%%%%%%%%%%%%
\paragraph{Content of Main File.}

It is advisable to place all content in the child files included by |\include|.
Any output contained in the main file will appear in all child documents
unless suppressed manually;
it cannot be suppressed automatically by the |\includeonly| directive
and thus should normally be avoided.
A method to include some content in the main file
by means of conditional processing is described in \secref{sec:conditional}.

%%%%%%%%%%%%%%%%%%%%%%%%%%%%%%%%%%%%%%%%
\paragraph{Page Numbering.}

When only a part of the document is compiled,
the appropriate numbering of pages
(as well as other status parameters)
is determined from the |.aux| files.
The latter contain information from previous passes.
However this information needs to propagate through
all intermediate child documents.
Therefore the page numbering in child documents may well
be inconsistent until the complete document is compiled at least once.

A useful (if unconventional) way to always ensure a consistent
page numbering is to restart the numbering in each child document
and denote the pages by `\textit{child}|.|\textit{page}'
where \textit{child} represents the chapter/section number of the child file.
This can be achieved by the command
|\numberwithin{page}{|\textit{child}|}|
of the \textsf{amsmath} package
where \textit{child} can be |chapter| or |section|
depending on the chosen structuring.
Alternatively, one can modify the macro |\thepage| appropriately
and reset the counter |page| at the start of each child file.

%%%%%%%%%%%%%%%%%%%%%%%%%%%%%%%%%%%%%%%%%%%%%%%%%%%%%%%%%%%%%%%%%%%%%%%%%%%%%%%%
\subsection{Conditional Processing}
\label{sec:conditional}

The package provides a mechanism to compile different versions
of a document. To customise the versions further some conditional processing
can come in handy to distinguish which version is being compiled.
The package provides two macros to describe the compilation context:

%%%%%%%%%%%%%%%%%%%%%%%%%%%%%%%%%%%%%%%%
\DescribeMacro{\ifchilddoc}
The conditional |\ifchilddoc| distinguishes between the compilation of
child documents and the main document:
%
\begin{center}
|\ifchilddoc |\textit{child-code}| |[|\||else |\textit{main-code}]| \||fi|
\end{center}

%%%%%%%%%%%%%%%%%%%%%%%%%%%%%%%%%%%%%%%%
\DescribeMacro{\childdocname}
\DescribeMacro{\childdocjob}
The macro |\childdocname| contains the filename (without extension)
of the main or child file being processed.
Note that |\childdocjob| will always contain the name of the main file.

%%%%%%%%%%%%%%%%%%%%%%%%%%%%%%%%%%%%%%%%
\paragraph{Title Page.}

Conditional processing can be used to include a title or banner page
in the main document when proper precautions are taken.
Importantly, the code in the main file should ensure that the page counter
(as well as other status parameters which are stored in the |.aux| files)
takes the same value after the conditional processing.
Otherwise the page numbers may take divergent values
depending on which part is compiled.

For example, a title page could be declared by:
%
\begin{center}
\begin{tabular}{l}
|\ifchilddoc\||else|\\
|\addtocounter{page}{-1}|\\
\textit{code for title page}\\
|\newpage|\\
|\||fi|
\end{tabular}
\end{center}
%
A banner page for the child documents can be generated by:
%
\begin{center}
\begin{tabular}{l}
|\ifchilddoc|\\
|\addtocounter{page}{-1}|\\
\textit{code for banner page}\\
|\newpage|\\
|\||fi|
\end{tabular}
\end{center}
%
Here one could write a message such as:
\begin{center}
|This is the part \childdocname{} of \childdocjob{}.|
\end{center}

%%%%%%%%%%%%%%%%%%%%%%%%%%%%%%%%%%%%%%%%%%%%%%%%%%%%%%%%%%%%%%%%%%%%%%%%%%%%%%%%
\subsection{Flags}
\label{sec:flags}

The package makes it easy to generate different versions
of the main or child documents.
To this end compilation flags can be defined
and assigned different default values.
They will be particularly useful in conjunction
with the forwarding mechanism described in \secref{sec:forward}.

For example, it may be useful to have a flag |\version|
which can be set to |draft| or |final|.
The document source will contain some conditional code
depending on the value of |\version|.
Suppose further, the flag should default to |final| for the main file
and to |draft| for child files
which is a natural assignment for editing the document.
This is achieved by placing the following code
in the preamble of the main document
(below the |\childdocmain| directive):
%
\begin{center}
\begin{tabular}{l}
|\ifchilddoc|\\
|\providecommand{\version}{draft}|\\
|\||else|\\
|\providecommand{\version}{final}|\\
|\||fi|
\end{tabular}
\end{center}
%
The definition by |\providecommand| makes sure
that previous definitions are not overwritten.
Further statements |\providecommand{\version}{...}|
can thus be added before the above code to override it.

For the main file, one might add a line
(between |\childdocmain| and the above block)
%
\begin{center}
|%\ifchilddoc\||else\providecommand{\version}{draft}\||fi|
\end{center}
%
which can be uncommented to produce a draft version.
Likewise one can add a line to the very top of a child file
(above the |\childdocof{|\textit{main}|}| directive)
%
\begin{center}
|%\providecommand{\version}{final}|
\end{center}
%
which can be uncommented to produce the final version of this child document.

%%%%%%%%%%%%%%%%%%%%%%%%%%%%%%%%%%%%%%%%%%%%%%%%%%%%%%%%%%%%%%%%%%%%%%%%%%%%%%%%
\subsection{Forwarding}
\label{sec:forward}

Different versions of the main or child documents
using compilation flags as described in \secref{sec:flags}
can be (permanently) stored in different files
for convenient compilation, viewing and distribution.
To this end, the package defines a command
to pass on compilation to a different file:

%%%%%%%%%%%%%%%%%%%%%%%%%%%%%%%%%%%%%%%%
\DescribeMacro{\childdocforward}
The command |\childdocforward| redirects processing to
another source file:
%
\begin{center}
\begin{tabular}{l}
|\input{childdoc.def}|\\
|\childdocforward[|\textit{main}|]{|\textit{dest}|}|\\
\end{tabular}
\end{center}
%
The argument \textit{dest} is the destination file
(without extension).
It should be the main file or one of the child files.
Note that further \textsf{childdoc} directives
such as |\childdocof| and |\childdocforward|
in the indicated file will be processed in this form.
The optional argument \textit{main}
passes on directly to the main file \textit{main}
while pretending to compile the child \textit{dest}.
This form behaves as if \textit{dest}
issues |\childdocof{|\textit{main}|}| right away,
and no further \textsf{childdoc} directives will be processed.

%%%%%%%%%%%%%%%%%%%%%%%%%%%%%%%%%%%%%%%%
\DescribeMacro{\...prefix}
In the alternative form |\childdocforwardprefix|,
%
\begin{center}
\begin{tabular}{l}
|\input{childdoc.def}|\\
|\childdocforwardprefix[|\textit{main}|]{|\textit{prefix}|}{|\textit{dest}|}|
\end{tabular}
\end{center}
%
the destination file is determined by a pattern
depending on the current file:
To make this work, the current file must be called
`{\textit{prefix}\hspace{0.2em}\textit{suffix}}'
with \textit{prefix} matching precisely the argument.
Processing is then passed on to the file
`{\textit{dest}\hspace{0.2em}\textit{suffix}}'.
Surely, the same effect is achieved by
directly specifying the
argument `{\textit{dest}\hspace{0.2em}\textit{suffix}}'
in the first form.
However, that requires to set up a different file
for each child. With the alternative form of the command
all these files can have exactly the same content
which simplifies setting them up and maintaining them.

For example, the following file |draft.tex|
with a compilation flag |\version| as described in \secref{sec:flags}
compiles the main document as a draft:
%
\begin{center}
\begin{tabular}{l}
|\def\version{draft}|\\
|\input{childdoc.def}|\\
|\childdocforward{|\textit{main}|}|
\end{tabular}
\end{center}
%
Likewise, the following files |final|\textit{nn}|.tex|
compile the final version of the child document
|child|\textit{nn}|.tex|:
%
\begin{center}
\begin{tabular}{l}
|\def\version{final}|\\
|\input{childdoc.def}|\\
|\childdocforwardprefix{final}{child}|
\end{tabular}
\end{center}
%

Note that when several versions of a main file and/or of each child file
are to be generated, it may be convenient to set up a |Makefile| or
shell script to automatise the process.

%%%%%%%%%%%%%%%%%%%%%%%%%%%%%%%%%%%%%%%%%%%%%%%%%%%%%%%%%%%%%%%%%%%%%%%%%%%%%%%%
\subsection{Command Line Processing}
\label{sec:commandline}

The effect of redirection files can also be achieved by invoking
the \LaTeX{} compiler with a more elaborate command line.
Most conveniently this should be done as part
of a shell script or a |Makefile|.

When using \textsf{childdoc} in the main file, the following
command lines effectively perform a redirection
(note that depending on the shell being used,
backslashes may have to be doubled: `|\|' $\to$ `|\\|'):
%
\begin{center}
|... -jobname "|\textit{target}|" |\\|"|[\textit{flags}]%
|\input{childdoc.def}\childdocforward[|\textit{main}|]{|\textit{dest}|}"|
\end{center}
%
Here \textit{target} is the name of the output file,
\textit{main} is the name of the main file
and \textit{dest} is the name of the main or child file to be processed
(all filenames without extensions).
The optional argument \textit{main} can be omitted
if \textit{main} matches \textit{dest}.
Optionally, compilation \textit{flags} can be defined via |\def| commands.
This command line makes the \TeX{} engine believe
it is compiling the file \textit{target}
whose content is specified as the latter parameter.
The provided code then forwards the processing to
\textit{main} or \textit{dest} as described in \secref{sec:forward}.

%%%%%%%%%%%%%%%%%%%%%%%%%%%%%%%%%%%%%%%%%%%%%%%%%%%%%%%%%%%%%%%%%%%%%%%%%%%%%%%%
\subsection{Include by Input}
\label{sec:input}

Including child documents by |\include| has some restrictions by design.
Most notably, the content of a child document always occupies
its own set of pages; pages cannot be shared between child documents.
Usually, this behaviour makes perfect sense
because each child document contain an essential part of the document.
However, in some situations it may be desirable to compose
a document from a collection of parts
without having mandatory page breaks between then.
For this case, the package
provides a mechanism to include parts
by |\input| which can also be processed individually.
However, by construction this mechanism
requires manual handling of the content to be output.

%%%%%%%%%%%%%%%%%%%%%%%%%%%%%%%%%%%%%%%%
\DescribeMacro{\ifchilddocmanual}
The main file should be prepared as usual, see \secref{sec:include}.
However, the document body must make a distinction
between processing of an individual part and of the main document, e.g.:
%
\begin{center}
\begin{tabular}{l}
|\ifchilddocmanual|\\
|\input{\childdocname}|\\
|\||else|\\
\textit{document body with }|\input{|\textit{part}|}|\\
|\||fi|
\end{tabular}
\end{center}
%
The conditional |\ifchilddocmanual| is true whenever
a part to be included by |\input| is being compiled,
and the name of the part is stored in |\childdocname|.

%%%%%%%%%%%%%%%%%%%%%%%%%%%%%%%%%%%%%%%%
\DescribeMacro{\childdocby}
Each part to be included by |\input| should start with:
%
\begin{center}
\begin{tabular}{l}
|\input{childdoc.def}|\\
|\childdocby{|\textit{main}|}|\\
\end{tabular}
\end{center}
%
The directive |\childdocby| is similar to |\childdocof|
described in \secref{sec:include},
but the subsequent selection of content must be done manually.
To that end, both |\ifchilddoc| and |\ifchilddocmanual|
will be true upon processing of a part,
and the name of the part is stored in |\childdocname|.
Note that |\jobname| will be set to the filename of the current part
so that each part receives an individual |.aux| file
that does not interfere with the |.aux| file(s) of the main document.
This behaviour can be altered by the alternative form
|\childdocby[*]{|\textit{main}|}| (with a non-empty optional argument)
which uses the |.aux| file of the main document
by setting |\jobname| to \textit{main}.

%%%%%%%%%%%%%%%%%%%%%%%%%%%%%%%%%%%%%%%%%%%%%%%%%%%%%%%%%%%%%%%%%%%%%%%%%%%%%%%%
\subsection{Driver Development}
\label{sec:driver}

The \textsf{childdoc} mechanism can also be use for the development
of definition files such as \LaTeX{} styles or classes.
This case differs from the above setup with multiple parts
included by |\include| in that no |\includeonly| should be invoked.
This can be achieved by starting the include file
(before |\ProvidesPackage|) with:
%
\begin{center}
\begin{tabular}{l}
|\input{childdoc.def}|\\
|\childdocforward{|\textit{main}|}|\\
\end{tabular}
\end{center}
%
or alternatively with:
%
\begin{center}
\begin{tabular}{l}
|\input{childdoc.def}|\\
|\childdocby{|\textit{main}|}|\\
\end{tabular}
\end{center}
%
Both forms have slightly different effects as described above.
The main file is prepared as usual, see \secref{sec:include}.

%%%%%%%%%%%%%%%%%%%%%%%%%%%%%%%%%%%%%%%%%%%%%%%%%%%%%%%%%%%%%%%%%%%%%%%%%%%%%%%%
\subsection{Legacy Detection}
\label{sec:detection}

The directive |\childdocmain| in the main file can detect
whether the complete document or merely a child is to be compiled
even without using the directive |\childdocof|.
This method is deprecated because it is less robust
and there is no compelling reason to use it;
it is merely provided for backward compatibility
and it may be removed in future versions.

If the detection mechanism is to be used,
it is mandatory to correctly specify
the filename of the main file as the argument of |\childdocmain|:
%
\begin{center}
\begin{tabular}{l}
|\input{childdoc.def}|\\
|\childdocmain{|\textit{main}|}|\\
\end{tabular}
\end{center}
%
If |\jobname| does not match the argument \textit{main} of |\childdocmain|,
it is assumed that |\jobname| points to the child file to be compiled.
When using |\childdocmain| with the main file specified as argument,
it suffices to start a child file
with just |\input{|\textit{main}|}|
without loading of the package and using |\childdocof|.
If instead all processing is done
with the appropriate \textsf{childdoc} directives,
the argument of \textit{main} of |\childdocmain| can be empty.

An alternative version of the command line processing described
in \secref{sec:commandline} using the detection mechanism reads:
%
\begin{center}
|... -jobname "|\textit{target}|" "|[\textit{flags}]%
[|\def\jobname{|\textit{dest}|}|]|\input{|\textit{main}|}"|
\end{center}

%%%%%%%%%%%%%%%%%%%%%%%%%%%%%%%%%%%%%%%%%%%%%%%%%%%%%%%%%%%%%%%%%%%%%%%%%%%%%%%%
\subsection{Manual Code}
\label{sec:manual}

In case one cannot be certain whether the definitions file |childdoc.def|
is installed on the target \TeX{} distribution
and one prefers not to ship it,
it is conceivable to paste a few relevant commands into the sources.

To that end, drop all statements |\input{childdoc.def}|
and perform the replacements as outlined below.
Instead of |\childdocmain{|\textit{main}|}| add the following code
to the top of the main file:
%
\begin{center}
\begin{tabular}{l}
|\||ifdefined\childdocname\endinput\||fi\newif\ifchilddoc|\\
|\edef\childdocname{\scantokens\expandafter{\jobname\noexpand}}|\\
|\def\childdocmain{|\textit{main}|}\||ifx\childdocmain\childdocname\||else|\\
|\childdoctrue\includeonly{\childdocname}\let\jobname\childdocmain\||fi|\\
\end{tabular}
\end{center}
%
Instead of |\childdocof{|\textit{main}|}| just include the main file
at the top of each child file:
%
\begin{center}
|\input{|\textit{main}|}|
\end{center}
%
A simple redirection |\childdocforward{|\textit{dest}|}| is achieved by:
%
\begin{center}
|\def\jobname{|\textit{dest}|}\input{\jobname}|
\end{center}
%
The redirection with prefix
|\childdocforwardprefix[|\textit{prefix}|]{|\textit{dest}|}|
is accomplished by:
%
\begin{center}
\begin{tabular}{l}
|{\edef\jobname{\scantokens\expandafter{\jobname\noexpand}}|\\
|\def\redirectjob |\textit{prefix}|#1~~~{\gdef\jobname{|\textit{dest}|#1}}|\\
|\expandafter\redirectjob\jobname~~~}\input{\jobname}|
\end{tabular}
\end{center}

In an alternative approach,
child documents can be compiled by a specific command line
without additional code or specific definitions:
%
\begin{center}
|... -jobname "|\textit{target}|" "|[\textit{flags}]%
|\includeonly{|\textit{dest}|}\input{|\textit{main}|}"|
\end{center}
%

%%%%%%%%%%%%%%%%%%%%%%%%%%%%%%%%%%%%%%%%%%%%%%%%%%%%%%%%%%%%%%%%%%%%%%%%%%%%%%%%
%%%%%%%%%%%%%%%%%%%%%%%%%%%%%%%%%%%%%%%%%%%%%%%%%%%%%%%%%%%%%%%%%%%%%%%%%%%%%%%%
\section{Information}

%%%%%%%%%%%%%%%%%%%%%%%%%%%%%%%%%%%%%%%%%%%%%%%%%%%%%%%%%%%%%%%%%%%%%%%%%%%%%%%%
\subsection{Copyright}

Copyright \copyright{} 2017--2018 Niklas Beisert

This work may be distributed and/or modified under the
conditions of the \LaTeX{} Project Public License, either version 1.3
of this license or (at your option) any later version.
The latest version of this license is in
  \url{http://www.latex-project.org/lppl.txt}
and version 1.3 or later is part of all distributions of \LaTeX{}
version 2005/12/01 or later.

This work has the LPPL maintenance status `maintained'.

The Current Maintainer of this work is Niklas Beisert.

This work consists of the files |README.txt|, |childdoc.ins| and |childdoc.dtx|
as well as the derived files |childdoc.def|, |cdocsamp.tex|
with |cdocsch1.tex|, |cdocsch2.tex|, |cdocspt3.tex|, |cdocspt4.tex|,
|cdocsdrf.tex|, |cdocsfn1.tex|, |cdocsfn2.tex|
as well as |childdoc.pdf|.

%%%%%%%%%%%%%%%%%%%%%%%%%%%%%%%%%%%%%%%%%%%%%%%%%%%%%%%%%%%%%%%%%%%%%%%%%%%%%%%%
\subsection{Files and Installation}

The package consists of the files:
%
\begin{center}
\begin{tabular}{ll}
    |README.txt|   & readme file \\
    |childdoc.ins| & installation file \\
    |childdoc.dtx| & source file \\
    |childdoc.def| & definition file \\
    |cdocsamp.tex| & sample main file \\
    |cdocsch1.tex| & sample include file \\
    |cdocsch2.tex| & sample include file \\
    |cdocspt3.tex| & sample part file \\
    |cdocspt4.tex| & sample part file \\
    |cdocsdrf.tex| & sample redirection file \\
    |cdocsfn1.tex| & sample redirection file \\
    |cdocsfn2.tex| & sample redirection file \\
    |childdoc.pdf| & manual
\end{tabular}
\end{center}
%
The distribution consists of the files
|README.txt|, |childdoc.ins| and |childdoc.dtx|.
%
\begin{itemize}
\item
Run (pdf)\LaTeX{} on |childdoc.dtx|
to compile the manual |childdoc.pdf| (this file).
\item
Run \LaTeX{} on |childdoc.ins| to create the definitions file |childdoc.def|
and the sample |cdocsamp.tex| with include files
|cdocsch1.tex|, |cdocsch2.tex|, |cdocspt3.tex|, |cdocspt4.tex|,
|cdocsdrf.tex|, |cdocsfn1.tex|, |cdocsfn2.tex|.
Then copy the file |childdoc.def| to an appropriate directory of your \LaTeX{}
distribution, e.g.\ \textit{texmf-root}|/tex/latex/childdoc|.
\end{itemize}

%%%%%%%%%%%%%%%%%%%%%%%%%%%%%%%%%%%%%%%%%%%%%%%%%%%%%%%%%%%%%%%%%%%%%%%%%%%%%%%%
\subsection{Related CTAN Packages}

There are several other packages which offer a similar functionality:
%
\begin{itemize}
\item
The packages
\href{http://ctan.org/pkg/docmute}{\textsf{docmute}},
\href{http://ctan.org/pkg/includex}{\textsf{includex}} and
\href{http://ctan.org/pkg/standalone}{\textsf{standalone}}
provide commands to include only the document body of
a child file thus allowing both files to be compiled individually.
\item
The packages \href{http://ctan.org/pkg/subdocs}{\textsf{subdocs}}
and \href{http://ctan.org/pkg/subfiles}{\textsf{subfiles}}
provide structures in which the main and child documents can be
encapsulated and allowing them to be compiled individually.
The inclusion mechanism is different from the conventional |\include|.
\item
The package \href{http://ctan.org/pkg/combine}{\textsf{combine}}
is an elaborate solution to combine several documents into one.
\end{itemize}
%
See also the CTAN topic \href{http://ctan.org/topic/subdocs}{\textsf{subdocs}}
for further related packages.
The present package differs from the above solutions in that
a document structure constructed with the conventional |\include| mechanism
just needs two extra commands at the top of every file
such that all constituent files can be compiled individually.

%%%%%%%%%%%%%%%%%%%%%%%%%%%%%%%%%%%%%%%%%%%%%%%%%%%%%%%%%%%%%%%%%%%%%%%%%%%%%%%%
%\subsection{Feature Suggestions}
%
%The following is a list of features which may be useful for future
%versions of this package:
%%
%\begin{itemize}
%\item
%\ldots
%\end{itemize}

%%%%%%%%%%%%%%%%%%%%%%%%%%%%%%%%%%%%%%%%%%%%%%%%%%%%%%%%%%%%%%%%%%%%%%%%%%%%%%%%
\subsection{Revision History}

%%%%%%%%%%%%%%%%%%%%%%%%%%%%%%%%%%%%%%%%
\paragraph{v2.0:} 2018/12/30

\begin{itemize}
\item
immediate forward processing
\item
added |\childdocby| mechanism
\item
manual restructured
\end{itemize}

%%%%%%%%%%%%%%%%%%%%%%%%%%%%%%%%%%%%%%%%
\paragraph{v1.6:} 2018/01/17

\begin{itemize}
\item
application for development of include files
\item
corrections to manual
\end{itemize}

%%%%%%%%%%%%%%%%%%%%%%%%%%%%%%%%%%%%%%%%
\paragraph{v1.5:} 2017/05/21

\begin{itemize}
\item
more complete structuring introduced
\item
|\childdocof| introduced
\item
|\childdoc| renamed to |\childdocmain|
\item
|\childredirect| renamed to |\childdocforward| and |\childdocforwardprefix|
and functionality expanded
\end{itemize}

%%%%%%%%%%%%%%%%%%%%%%%%%%%%%%%%%%%%%%%%
\paragraph{v1.0:} 2017/04/27

\begin{itemize}
\item
manual and install package
\item
first version published on CTAN
\end{itemize}

%%%%%%%%%%%%%%%%%%%%%%%%%%%%%%%%%%%%%%%%
\paragraph{v0.6:} 2017/04/26

\begin{itemize}
\item
redirection mechanism added
\end{itemize}

%%%%%%%%%%%%%%%%%%%%%%%%%%%%%%%%%%%%%%%%
\paragraph{v0.5:} 2017/04/26

\begin{itemize}
\item
functionality in definition file
\end{itemize}


%%%%%%%%%%%%%%%%%%%%%%%%%%%%%%%%%%%%%%%%%%%%%%%%%%%%%%%%%%%%%%%%%%%%%%%%%%%%%%%%
%%%%%%%%%%%%%%%%%%%%%%%%%%%%%%%%%%%%%%%%%%%%%%%%%%%%%%%%%%%%%%%%%%%%%%%%%%%%%%%%
%%%%%%%%%%%%%%%%%%%%%%%%%%%%%%%%%%%%%%%%%%%%%%%%%%%%%%%%%%%%%%%%%%%%%%%%%%%%%%%%
\appendix

\settowidth\MacroIndent{\rmfamily\scriptsize 000\ }

 \DocInput{childdoc.dtx}

\end{document}
%</driver>
% \fi
%
% %%%%%%%%%%%%%%%%%%%%%%%%%%%%%%%%%%%%%%%%%%%%%%%%%%%%%%%%%%%%%%%%%%%%%%%%%%%%%%
% %%%%%%%%%%%%%%%%%%%%%%%%%%%%%%%%%%%%%%%%%%%%%%%%%%%%%%%%%%%%%%%%%%%%%%%%%%%%%%
% \section{Sample}
%\iffalse
%<*samplemain>
%\fi
%
% The following presents a sample document
% with two chapters, two parts, a title page,
% a compile flag as well as three forwarding files to set the flag.
% It consists of eight |.tex| files:
% \begin{center}
% \begin{tabular}{ll}
% |cdocsamp.tex|&main file\\
% |cdocsch1.tex|&include file for chapter 1\\
% |cdocsch2.tex|&include file for chapter 2\\
% |cdocspt3.tex|&include file for part 3\\
% |cdocspt4.tex|&include file for part 4\\
% |cdocsdrf.tex|&forwarding file for main file in draft mode\\
% |cdocsfi1.tex|&forwarding file for final version of chapter 1\\
% |cdocsfi2.tex|&forwarding file for final version of chapter 2\\
% \end{tabular}
% \end{center}
% Each of the eight files can be compiled directly by the \LaTeX{} compiler.
%
% %%%%%%%%%%%%%%%%%%%%%%%%%%%%%%%%%%%%%%
% \paragraph{Main File.}
%
% The main file is called |cdocsamp.tex|.
%
% Load the \textsf{childdoc} definitions and
% declare the filename for the main document:
%    \begin{macrocode}
\input{childdoc.def}
\childdocmain{}
%    \end{macrocode}

% Optional override for |\version| flag:
%    \begin{macrocode}
%%\ifchilddoc\else\providecommand{\version}{draft}\fi
%    \end{macrocode}

% Define the default values for the |\version| flag
% (|final| for the main file and |draft| for childs):
%    \begin{macrocode}
\ifchilddoc
\providecommand{\version}{draft}
\else
\providecommand{\version}{final}
\fi
%    \end{macrocode}

% Load the standard document class:
%    \begin{macrocode}
\documentclass[12pt]{article}
%    \end{macrocode}

% Start the document body:
%    \begin{macrocode}
\begin{document}
%    \end{macrocode}

% Declare a title page.
% Print title, part of document being processed and version flag:
%    \begin{macrocode}
\addtocounter{page}{-1}
\begin{center}
{\LARGE\bfseries{}childdoc example\par}
\vspace{1cm}
\ifchilddoc
\ifchilddocmanual part\else chapter\fi:
`\childdocname' of `\childdocjob'\par
\else
main document: `\childdocjob'\par
\fi
version: \version\par
\end{center}
\newpage
%    \end{macrocode}

% Manually include selected file,
% otherwise process as usual:
%    \begin{macrocode}
\ifchilddocmanual
\section*{part `\childdocname'}
\input{\childdocname}
\else
%    \end{macrocode}

% Include the two chapters:
%    \begin{macrocode}
\include{cdocsch1}
\include{cdocsch2}
%    \end{macrocode}

% Include the two parts unless only chapters should be displayed:
%    \begin{macrocode}
\ifchilddoc\else
\section{part three}
\input{cdocspt3}
\section{part four}
\input{cdocspt4}
\fi
%    \end{macrocode}

% Process as usual until here:
%    \begin{macrocode}
\fi
%    \end{macrocode}

% End of document body:
%    \begin{macrocode}
\end{document}
%    \end{macrocode}
%\iffalse
%</samplemain>
%\fi
%
% %%%%%%%%%%%%%%%%%%%%%%%%%%%%%%%%%%%%%%
% \paragraph{Chapter Include Files.}
%
% The include files are called |cdocsch1.tex| and |cdocsch2.tex|.
%
%\iffalse
%<*samplechap1|samplechap2>
%\fi

% Optional override for |\version| flag:
%    \begin{macrocode}
%%\providecommand{\version}{final}
%    \end{macrocode}

% Include the main document:
%    \begin{macrocode}
\input{childdoc.def}
\childdocof{cdocsamp}
%    \end{macrocode}

%\iffalse
%</samplechap1|samplechap2>
%\fi
%
%\iffalse
%<*samplechap1>
%\fi
% Some text for chapter 1:
%    \begin{macrocode}
\section{one}
some text in chapter one
%    \end{macrocode}

%\iffalse
%</samplechap1>
%\fi
% Some text for chapter 2:
%\iffalse
%<*samplechap2>
%\fi
%    \begin{macrocode}
\section{two}
more text in chapter two
%    \end{macrocode}

%\iffalse
%</samplechap2>
%\fi
%
% %%%%%%%%%%%%%%%%%%%%%%%%%%%%%%%%%%%%%%
% \paragraph{Part Include Files.}
%
% The include files are called |cdocspt3.tex| and |cdocspt4.tex|.
%
%\iffalse
%<*samplepart3|samplepart4>
%\fi

% Optional override for |\version| flag:
%    \begin{macrocode}
%%\providecommand{\version}{final}
%    \end{macrocode}

% Include the main document:
%    \begin{macrocode}
\input{childdoc.def}
\childdocby{cdocsamp}
%    \end{macrocode}

%\iffalse
%</samplepart3|samplepart4>
%\fi
%
%\iffalse
%<*samplepart3>
%\fi
% Some text for part 3:
%    \begin{macrocode}
some text in part three
%    \end{macrocode}

%\iffalse
%</samplepart3>
%\fi
% Some text for part 4:
%\iffalse
%<*samplepart4>
%\fi
%    \begin{macrocode}
more text in part four
%    \end{macrocode}

%\iffalse
%</samplepart4>
%\fi
%
% %%%%%%%%%%%%%%%%%%%%%%%%%%%%%%%%%%%%%%
% \paragraph{Forwarding for a Complete Draft.}
%
% The following forwarding file |cdocsdrf.tex|
% compiles the main document in draft mode:
%\iffalse
%<*sampledraft>
%\fi
%    \begin{macrocode}
\def\version{draft}
\input{childdoc.def}
\childdocforward{cdocsamp}
%    \end{macrocode}

%\iffalse
%</sampledraft>
%\fi
%
% %%%%%%%%%%%%%%%%%%%%%%%%%%%%%%%%%%%%%%
% \paragraph{Forwarding for Final Version of the Chapters.}
%
% The following forwarding files |cdocsfn1.tex| and |cdocsfn2.tex|
% (with identical content)
% compile the final versions of the child documents
% |cdocsch1.tex| and |cdocsch2.tex|, respectively:
%\iffalse
%<*samplefinal>
%\fi
%    \begin{macrocode}
\def\version{final}
\input{childdoc.def}
\childdocforwardprefix[cdocsamp]{cdocsfn}{cdocsch}
%    \end{macrocode}

%\iffalse
%</samplefinal>
%\fi
%
% %%%%%%%%%%%%%%%%%%%%%%%%%%%%%%%%%%%%%%
% \paragraph{Command Line Processing.}
%
% The following three command lines generate the output files
% |cdocscld|, |cdocscl1| and |cdocscl2|
% which should be identical to
% |cdocsdrf|, |cdocsch1| and |cdocsfn2|, respectively:
% \begin{center}
% \begin{tabular}{l}
% |latex -jobname cdocscld \|\\
% |  "\def\version{draft}\input{childdoc.def}\childdocforward{cdocsamp}"|\\
% |latex -jobname cdocscl1 \|\\
% |  "\input{childdoc.def}\childdocforward[cdocsamp]{cdocsch1}"|\\
% |latex -jobname cdocscl2 \|\\
% |  "\def\version{final}\input{childdoc.def}\childdocforward{cdocsch2}"|
% \end{tabular}
% \end{center}
% Note that the trailing backslash on each first line
% merely continues the input to the second line
% (for convenient cut ant paste).
% Furthermore, the command |latex| can be replaced by any
% of its alternative versions such as |pdflatex|.
%
% %%%%%%%%%%%%%%%%%%%%%%%%%%%%%%%%%%%%%%%%%%%%%%%%%%%%%%%%%%%%%%%%%%%%%%%%%%%%%%
% %%%%%%%%%%%%%%%%%%%%%%%%%%%%%%%%%%%%%%%%%%%%%%%%%%%%%%%%%%%%%%%%%%%%%%%%%%%%%%
% \section{Implementation}
%\iffalse
%<*package>
%\fi
%
% This section describes the definitions file |childdoc.def|.

% The definitions cannot be loaded using |\usepackage| or |\RequirePackage|
% which has a mechanism to prevent loading a style file more than once.
% When loading the definitions by means of |\input|
% multiple instances have to be prevented manually:
%\iffalse
%This code needs to be before the `\ProvidesFile' directive
%which is defined at the beginning of this file.
%Therefore it is also placed there and commented out here.
%</package>
%<*discard>
%\fi
%    \begin{macrocode}
\ifdefined\childdocmain\endinput\fi
%    \end{macrocode}
%\iffalse
%</discard>
%<*package>
%\fi
%
% \macro{\ifchilddoc}
% \macro{\ifchilddocmanual}
% The conditional |\ifchilddoc| tells whether a
% child (true) or main (false) document is being compiled.
% The conditional |\ifchilddocmanual| tells whether
% the |\includeonly| mechanism is used (false) or
% the selection of child files must be performed manually (true).
% The definitions initialise to false:
%    \begin{macrocode}
\newif\ifchilddoc
\newif\ifchilddocmanual
%    \end{macrocode}

% \macro{\childdocname}
% \macro{\childdocjob}
% The macro |\childdocname| stores the name of the main document
% to be compiled. The macro |\childdocjob| stores the name of
% the document on which the \LaTeX{} compiler was originally invoked.
% The content of |\jobname| cannot be compared
% to filenames specified in the source due to different catcodes.
% The following code rescans |\jobname|, stores the result
% in |\childdocname| and saves a copy in |\childdocjob|:
%    \begin{macrocode}
\edef\childdocname{\scantokens\expandafter{\jobname\noexpand}}
\let\childdocjob\childdocname
%    \end{macrocode}

% \macro{\childdocdisable}
% The macro |\childdocdisable| prevents the main file
% from being processed more than once.
% At this stage, the main document command |\childdocmain|
% is assumed to be called once again where it should do nothing.
% Any subsequent call to it should prevent
% a secondary processing of the main document
% It overwrites the forwarding commands
% |\childdocof| and |\childdocforward|
% with empty macros to prevent further inclusions of the main document:
%    \begin{macrocode}
\newcommand{\childdocdisable}
{
  \renewcommand{\childdocmain}[1]{\renewcommand{\childdocmain}[1]{\endinput}}
  \renewcommand{\childdocof}[1]{}
  \renewcommand{\childdocby}[2][]{}
  \renewcommand{\childdocforward}[2][]{}
  \renewcommand{\childdocdisable}{}
}
%    \end{macrocode}

% \macro{\childdocmain}
% The macro |\childdocmain| is to be called at the top of the main file
% with nothing or the main filename (without extension) as argument.
% First, it breaks loops.
% If the argument is not empty and does not match |\childdocname|
% (which is set by the first inclusion of |childdoc.def|),
% |\ifchilddoc| is set to true, |\includeonly| is applied to the child file
% and |\jobname| is set to the main file
% (for proper handling of |.aux| files):
%    \begin{macrocode}
\newcommand{\childdocmain}[1]
{
  \childdocdisable\childdocmain{}
  \if?#1?\else
    \begingroup
      \def\childdoctmp{#1}
      \ifx\childdoctmp\childdocname
        \def\childdoctmp{}
      \else
        \def\childdoctmp
        {
          \childdoctrue
          \includeonly{\childdocname}
          \def\childdocjob{#1}
          \def\jobname{#1}
        }
      \fi
      \expandafter
    \endgroup
    \childdoctmp
  \fi
}
%    \end{macrocode}

% \macro{\childdocof}
% The command |\childdocof| redirects
% compilation to the main file |#1|.
%    \begin{macrocode}
\newcommand{\childdocof}[1]
{
  \childdocdisable
  \childdoctrue
  \includeonly{\childdocname}
  \def\jobname{#1}
  \def\childdocjob{#1}
  \input{#1}
}
%    \end{macrocode}

% \macro{\childdocby}
% The command |\childdocby| ....
%    \begin{macrocode}
\newcommand{\childdocby}[2][]
{
  \childdocdisable
  \childdoctrue
  \childdocmanualtrue
  \if?#1?\else
    \def\jobname{#2}
  \fi
  \def\childdocjob{#2}
  \input{#2}
  \endinput
}
%    \end{macrocode}

% \macro{\childdocforward}
% The command |\childdocforward| redirects
% compilation to the main file or
% (if the optional argument is given) a child file.
% Parameters are set as if the main file
% or a child file starting with |\childdocof| was compiled.
% Then compilation is handed over to the main file:
%    \begin{macrocode}
\newcommand{\childdocforward}[2][]
{
  \begingroup
    \if?#1?
      \def\childdoctmp
      {
        \def\childdocname{#2}
        \def\childdocjob{#2}
        \def\jobname{#2}
        \input{#2}
        \endinput
      }
    \else
      \def\childdoctmp
      {
        \childdocdisable
        \def\childdocname{#2}
        \childdoctrue
        \includeonly{#2}
        \def\childdocjob{#1}
        \def\jobname{#1}
        \input{#1}
        \endinput
      }
    \fi
    \expandafter
  \endgroup
  \childdoctmp
}
%    \end{macrocode}

% \macro{\childdocforwardprefix}
% The command |\childdocforwardprefix| redirects
% compilation to the main or a child file by means of a pattern.
% The prefix |#1| in the current filename is replaced by |#2|
% and the suffix of the current filename is kept
% (it is assumed that the filename does not contain the substring `|~~~|'
% which is used as a delimiter).
% Compilation is handed over to the new file by |\childdocforward|:
%    \begin{macrocode}
\newcommand{\childdocforwardprefix}[3][]
{
  \begingroup
    \def\childdocextract #2##1~~~{\def\childdoctmp{\childdocforward[#1]{#3##1}}}
    \expandafter\childdocextract\childdocname~~~
    \expandafter
  \endgroup
  \childdoctmp
}
%    \end{macrocode}

% \macro{\childdoc}
% The deprecated macro |\childdoc| is a legacy version of |\childdocmain|:
%    \begin{macrocode}
\newcommand{\childdoc}{\childdocmain}
%    \end{macrocode}

% \macro{\childdocredirect}
% The deprecated macro |\childdocredirect| is a legacy version
% of |\childdocforward| and |\childdocforwardprefix|:
%    \begin{macrocode}
\newcommand{\childdocredirect}[2][]
{
  \begingroup
    \if?#1?
      \def\childdoctmp{\childdocforward{#2}}
    \else
      \def\childdoctmp{\childdocforwardprefix{#1}{#2}}
    \fi
    \expandafter
  \endgroup
  \childdoctmp
}
%    \end{macrocode}

%\iffalse
%</package>
%\fi
%
\endinput
\childdocforward{cdocsch2}"|
% \end{tabular}
% \end{center}
% Note that the trailing backslash on each first line
% merely continues the input to the second line
% (for convenient cut ant paste).
% Furthermore, the command |latex| can be replaced by any
% of its alternative versions such as |pdflatex|.
%
% %%%%%%%%%%%%%%%%%%%%%%%%%%%%%%%%%%%%%%%%%%%%%%%%%%%%%%%%%%%%%%%%%%%%%%%%%%%%%%
% %%%%%%%%%%%%%%%%%%%%%%%%%%%%%%%%%%%%%%%%%%%%%%%%%%%%%%%%%%%%%%%%%%%%%%%%%%%%%%
% \section{Implementation}
%\iffalse
%<*package>
%\fi
%
% This section describes the definitions file |childdoc.def|.

% The definitions cannot be loaded using |\usepackage| or |\RequirePackage|
% which has a mechanism to prevent loading a style file more than once.
% When loading the definitions by means of |\input|
% multiple instances have to be prevented manually:
%\iffalse
%This code needs to be before the `\ProvidesFile' directive
%which is defined at the beginning of this file.
%Therefore it is also placed there and commented out here.
%</package>
%<*discard>
%\fi
%    \begin{macrocode}
\ifdefined\childdocmain\endinput\fi
%    \end{macrocode}
%\iffalse
%</discard>
%<*package>
%\fi
%
% \macro{\ifchilddoc}
% \macro{\ifchilddocmanual}
% The conditional |\ifchilddoc| tells whether a
% child (true) or main (false) document is being compiled.
% The conditional |\ifchilddocmanual| tells whether
% the |\includeonly| mechanism is used (false) or
% the selection of child files must be performed manually (true).
% The definitions initialise to false:
%    \begin{macrocode}
\newif\ifchilddoc
\newif\ifchilddocmanual
%    \end{macrocode}

% \macro{\childdocname}
% \macro{\childdocjob}
% The macro |\childdocname| stores the name of the main document
% to be compiled. The macro |\childdocjob| stores the name of
% the document on which the \LaTeX{} compiler was originally invoked.
% The content of |\jobname| cannot be compared
% to filenames specified in the source due to different catcodes.
% The following code rescans |\jobname|, stores the result
% in |\childdocname| and saves a copy in |\childdocjob|:
%    \begin{macrocode}
\edef\childdocname{\scantokens\expandafter{\jobname\noexpand}}
\let\childdocjob\childdocname
%    \end{macrocode}

% \macro{\childdocdisable}
% The macro |\childdocdisable| prevents the main file
% from being processed more than once.
% At this stage, the main document command |\childdocmain|
% is assumed to be called once again where it should do nothing.
% Any subsequent call to it should prevent
% a secondary processing of the main document
% It overwrites the forwarding commands
% |\childdocof| and |\childdocforward|
% with empty macros to prevent further inclusions of the main document:
%    \begin{macrocode}
\newcommand{\childdocdisable}
{
  \renewcommand{\childdocmain}[1]{\renewcommand{\childdocmain}[1]{\endinput}}
  \renewcommand{\childdocof}[1]{}
  \renewcommand{\childdocby}[2][]{}
  \renewcommand{\childdocforward}[2][]{}
  \renewcommand{\childdocdisable}{}
}
%    \end{macrocode}

% \macro{\childdocmain}
% The macro |\childdocmain| is to be called at the top of the main file
% with nothing or the main filename (without extension) as argument.
% First, it breaks loops.
% If the argument is not empty and does not match |\childdocname|
% (which is set by the first inclusion of |childdoc.def|),
% |\ifchilddoc| is set to true, |\includeonly| is applied to the child file
% and |\jobname| is set to the main file
% (for proper handling of |.aux| files):
%    \begin{macrocode}
\newcommand{\childdocmain}[1]
{
  \childdocdisable\childdocmain{}
  \if?#1?\else
    \begingroup
      \def\childdoctmp{#1}
      \ifx\childdoctmp\childdocname
        \def\childdoctmp{}
      \else
        \def\childdoctmp
        {
          \childdoctrue
          \includeonly{\childdocname}
          \def\childdocjob{#1}
          \def\jobname{#1}
        }
      \fi
      \expandafter
    \endgroup
    \childdoctmp
  \fi
}
%    \end{macrocode}

% \macro{\childdocof}
% The command |\childdocof| redirects
% compilation to the main file |#1|.
%    \begin{macrocode}
\newcommand{\childdocof}[1]
{
  \childdocdisable
  \childdoctrue
  \includeonly{\childdocname}
  \def\jobname{#1}
  \def\childdocjob{#1}
  \input{#1}
}
%    \end{macrocode}

% \macro{\childdocby}
% The command |\childdocby| ....
%    \begin{macrocode}
\newcommand{\childdocby}[2][]
{
  \childdocdisable
  \childdoctrue
  \childdocmanualtrue
  \if?#1?\else
    \def\jobname{#2}
  \fi
  \def\childdocjob{#2}
  \input{#2}
  \endinput
}
%    \end{macrocode}

% \macro{\childdocforward}
% The command |\childdocforward| redirects
% compilation to the main file or
% (if the optional argument is given) a child file.
% Parameters are set as if the main file
% or a child file starting with |\childdocof| was compiled.
% Then compilation is handed over to the main file:
%    \begin{macrocode}
\newcommand{\childdocforward}[2][]
{
  \begingroup
    \if?#1?
      \def\childdoctmp
      {
        \def\childdocname{#2}
        \def\childdocjob{#2}
        \def\jobname{#2}
        \input{#2}
        \endinput
      }
    \else
      \def\childdoctmp
      {
        \childdocdisable
        \def\childdocname{#2}
        \childdoctrue
        \includeonly{#2}
        \def\childdocjob{#1}
        \def\jobname{#1}
        \input{#1}
        \endinput
      }
    \fi
    \expandafter
  \endgroup
  \childdoctmp
}
%    \end{macrocode}

% \macro{\childdocforwardprefix}
% The command |\childdocforwardprefix| redirects
% compilation to the main or a child file by means of a pattern.
% The prefix |#1| in the current filename is replaced by |#2|
% and the suffix of the current filename is kept
% (it is assumed that the filename does not contain the substring `|~~~|'
% which is used as a delimiter).
% Compilation is handed over to the new file by |\childdocforward|:
%    \begin{macrocode}
\newcommand{\childdocforwardprefix}[3][]
{
  \begingroup
    \def\childdocextract #2##1~~~{\def\childdoctmp{\childdocforward[#1]{#3##1}}}
    \expandafter\childdocextract\childdocname~~~
    \expandafter
  \endgroup
  \childdoctmp
}
%    \end{macrocode}

% \macro{\childdoc}
% The deprecated macro |\childdoc| is a legacy version of |\childdocmain|:
%    \begin{macrocode}
\newcommand{\childdoc}{\childdocmain}
%    \end{macrocode}

% \macro{\childdocredirect}
% The deprecated macro |\childdocredirect| is a legacy version
% of |\childdocforward| and |\childdocforwardprefix|:
%    \begin{macrocode}
\newcommand{\childdocredirect}[2][]
{
  \begingroup
    \if?#1?
      \def\childdoctmp{\childdocforward{#2}}
    \else
      \def\childdoctmp{\childdocforwardprefix{#1}{#2}}
    \fi
    \expandafter
  \endgroup
  \childdoctmp
}
%    \end{macrocode}

%\iffalse
%</package>
%\fi
%
\endinput
\childdocforward[|\textit{main}|]{|\textit{dest}|}"|
\end{center}
%
Here \textit{target} is the name of the output file,
\textit{main} is the name of the main file
and \textit{dest} is the name of the main or child file to be processed
(all filenames without extensions).
The optional argument \textit{main} can be omitted
if \textit{main} matches \textit{dest}.
Optionally, compilation \textit{flags} can be defined via |\def| commands.
This command line makes the \TeX{} engine believe
it is compiling the file \textit{target}
whose content is specified as the latter parameter.
The provided code then forwards the processing to
\textit{main} or \textit{dest} as described in \secref{sec:forward}.

%%%%%%%%%%%%%%%%%%%%%%%%%%%%%%%%%%%%%%%%%%%%%%%%%%%%%%%%%%%%%%%%%%%%%%%%%%%%%%%%
\subsection{Include by Input}
\label{sec:input}

Including child documents by |\include| has some restrictions by design.
Most notably, the content of a child document always occupies
its own set of pages; pages cannot be shared between child documents.
Usually, this behaviour makes perfect sense
because each child document contain an essential part of the document.
However, in some situations it may be desirable to compose
a document from a collection of parts
without having mandatory page breaks between then.
For this case, the package
provides a mechanism to include parts
by |\input| which can also be processed individually.
However, by construction this mechanism
requires manual handling of the content to be output.

%%%%%%%%%%%%%%%%%%%%%%%%%%%%%%%%%%%%%%%%
\DescribeMacro{\ifchilddocmanual}
The main file should be prepared as usual, see \secref{sec:include}.
However, the document body must make a distinction
between processing of an individual part and of the main document, e.g.:
%
\begin{center}
\begin{tabular}{l}
|\ifchilddocmanual|\\
|\input{\childdocname}|\\
|\||else|\\
\textit{document body with }|\input{|\textit{part}|}|\\
|\||fi|
\end{tabular}
\end{center}
%
The conditional |\ifchilddocmanual| is true whenever
a part to be included by |\input| is being compiled,
and the name of the part is stored in |\childdocname|.

%%%%%%%%%%%%%%%%%%%%%%%%%%%%%%%%%%%%%%%%
\DescribeMacro{\childdocby}
Each part to be included by |\input| should start with:
%
\begin{center}
\begin{tabular}{l}
|% \iffalse
%
% childdoc.dtx Copyright (C) 2017-2018 Niklas Beisert
%
% This work may be distributed and/or modified under the
% conditions of the LaTeX Project Public License, either version 1.3
% of this license or (at your option) any later version.
% The latest version of this license is in
%   http://www.latex-project.org/lppl.txt
% and version 1.3 or later is part of all distributions of LaTeX
% version 2005/12/01 or later.
%
% This work has the LPPL maintenance status `maintained'.
%
% The Current Maintainer of this work is Niklas Beisert.
%
% This work consists of the files childdoc.dtx and childdoc.ins
% and the derived files childdoc.def and cdocsamp.tex with
% cdocsch1.tex, cdocsch2.tex, cdocsdrf.tex, cdocsfn1.tex, cdocsfn2.tex.
%
%<package>\ifdefined\childdocmain\endinput\fi
%<package>\ProvidesFile{childdoc.def}[2018/12/30 v2.0 child document driver]
%<samplemain>\ProvidesFile{cdocsamp.tex}[2018/12/30 v2.0 sample for childdoc]
%<*driver>
%\ProvidesFile{childdoc.drv}[2018/12/30 v2.0 childdoc reference manual file]
\PassOptionsToClass{10pt,a4paper}{article}
\documentclass{ltxdoc}

\usepackage[margin=35mm]{geometry}
\usepackage{hyperref}
\usepackage{hyperxmp}
\usepackage[usenames]{color}

\hypersetup{colorlinks=true}
\hypersetup{pdfstartview=FitH}
\hypersetup{pdfpagemode=UseNone}
\hypersetup{pdfsource={}}
\hypersetup{pdflang={en-UK}}
\hypersetup{pdfcopyright={Copyright 2017-2018 Niklas Beisert.
  This work may be distributed and/or modified under the
  conditions of the LaTeX Project Public License, either version 1.3
  of this license or (at your option) any later version.}}
\hypersetup{pdflicenseurl={http://www.latex-project.org/lppl.txt}}
\hypersetup{pdfcontactaddress={ETH Zurich, ITP, HIT K,
  Wolfgang-Pauli-Strasse 27}}
\hypersetup{pdfcontactpostcode={8093}}
\hypersetup{pdfcontactcity={Zurich}}
\hypersetup{pdfcontactcountry={Switzerland}}
\hypersetup{pdfcontactemail={nbeisert@itp.phys.ethz.ch}}
\hypersetup{pdfcontacturl={http://people.phys.ethz.ch/\xmptilde nbeisert/}}

\newcommand{\secref}[1]{\hyperref[#1]{section \ref*{#1}}}

\parskip1ex
\parindent0pt
\let\olditemize\itemize
\def\itemize{\olditemize\parskip0pt}

\begin{document}

\title{The \textsf{childdoc} Package}
\hypersetup{pdftitle={The childdoc Package}}
\author{Niklas Beisert\\[2ex]
  Institut f\"ur Theoretische Physik\\
  Eidgen\"ossische Technische Hochschule Z\"urich\\
  Wolfgang-Pauli-Strasse 27, 8093 Z\"urich, Switzerland\\[1ex]
  \href{mailto:nbeisert@itp.phys.ethz.ch}
  {\texttt{nbeisert@itp.phys.ethz.ch}}}
\hypersetup{pdfauthor={Niklas Beisert}}
\hypersetup{pdfsubject={Manual for the LaTeX2e Package childdoc}}
\date{30 December 2018, \textsf{v2.0}}
\maketitle

\begin{abstract}\noindent
\textsf{childdoc} is a \LaTeXe{} package
that enables the direct compilation
of document sections included by |\include|
to individual files.
\end{abstract}

\begingroup
\parskip0ex
\tableofcontents
\endgroup

%%%%%%%%%%%%%%%%%%%%%%%%%%%%%%%%%%%%%%%%%%%%%%%%%%%%%%%%%%%%%%%%%%%%%%%%%%%%%%%%
%%%%%%%%%%%%%%%%%%%%%%%%%%%%%%%%%%%%%%%%%%%%%%%%%%%%%%%%%%%%%%%%%%%%%%%%%%%%%%%%
\section{Introduction}

\LaTeX{} provides a mechanism to structure a large document (such as a book)
into a main file and several child files (containing the chapters)
using the |\include| command.
This mechanism is beneficial for documents
which span hundreds of pages in order to
make the source file(s) more manageable.
Moreover, compilation can be restricted to
selected child files by means of the |\includeonly| command.
The latter feature can be used to reduce the compilation time while editing
(this was significantly more useful in the earlier days of \LaTeX{})
or to generate a smaller document which is easier to navigate.
Another application of |\includeonly| is to generate
documents consisting of selected parts of the complete document.

However, there are a few drawbacks of the plain |\include| mechanism:
\begin{itemize}
\item
The child files cannot be compiled on their own,
they can only be compiled via the main file.
A naive editing environment
(such as a text editor with an option
to have the current file processed by \LaTeX)
may require one to switch to the main file before compiling;
attempting to compile the child file produces errors.
\item
The main file must be modified (each time)
to adjust the |\includeonly| command
to the present needs. This easily leaves the main file in a messy state.
\item
The generated document will always carry the filename
of the main document. This is inconvenient if
several child files are to be compiled and
to be kept for distribution.
\end{itemize}

The present package provides a simple interface
to make child files individually compilable by \LaTeX{}.
Compiling a child file then has the same effect as compiling
the main file with an |\includeonly| command
to select the appropriate child.
Moreover the generated document will carry the name of the child
rather than the main file.
This resolves all three above issues.

This feature is meant to make the editing of books,
thesis documents and lecture notes somewhat more convenient.
However, the package can also be used efficiently for
composing a series of documents (such as exercise sheets)
which are typically distributed individually.
It then assists the author in generating the individual documents
(potentially in different versions)
as well as a document containing the collected series.
Another application is in developing style files
or other kinds of included material
where compilation of the style file could redirect
to a sample or test file.

%%%%%%%%%%%%%%%%%%%%%%%%%%%%%%%%%%%%%%%%%%%%%%%%%%%%%%%%%%%%%%%%%%%%%%%%%%%%%%%%
%%%%%%%%%%%%%%%%%%%%%%%%%%%%%%%%%%%%%%%%%%%%%%%%%%%%%%%%%%%%%%%%%%%%%%%%%%%%%%%%
\section{Usage}

First of all, the package \textsf{childdoc} is \emph{not} a standard
\LaTeXe{} |.sty| style file! Therefore it needs to be invoked in
a non-standard way.

%%%%%%%%%%%%%%%%%%%%%%%%%%%%%%%%%%%%%%%%%%%%%%%%%%%%%%%%%%%%%%%%%%%%%%%%%%%%%%%%
\subsection{Included Files}
\label{sec:include}

%%%%%%%%%%%%%%%%%%%%%%%%%%%%%%%%%%%%%%%%
\DescribeMacro{\childdocmain}
To use the package, add the commands
\begin{center}
\begin{tabular}{l}
|% \iffalse
%
% childdoc.dtx Copyright (C) 2017-2018 Niklas Beisert
%
% This work may be distributed and/or modified under the
% conditions of the LaTeX Project Public License, either version 1.3
% of this license or (at your option) any later version.
% The latest version of this license is in
%   http://www.latex-project.org/lppl.txt
% and version 1.3 or later is part of all distributions of LaTeX
% version 2005/12/01 or later.
%
% This work has the LPPL maintenance status `maintained'.
%
% The Current Maintainer of this work is Niklas Beisert.
%
% This work consists of the files childdoc.dtx and childdoc.ins
% and the derived files childdoc.def and cdocsamp.tex with
% cdocsch1.tex, cdocsch2.tex, cdocsdrf.tex, cdocsfn1.tex, cdocsfn2.tex.
%
%<package>\ifdefined\childdocmain\endinput\fi
%<package>\ProvidesFile{childdoc.def}[2018/12/30 v2.0 child document driver]
%<samplemain>\ProvidesFile{cdocsamp.tex}[2018/12/30 v2.0 sample for childdoc]
%<*driver>
%\ProvidesFile{childdoc.drv}[2018/12/30 v2.0 childdoc reference manual file]
\PassOptionsToClass{10pt,a4paper}{article}
\documentclass{ltxdoc}

\usepackage[margin=35mm]{geometry}
\usepackage{hyperref}
\usepackage{hyperxmp}
\usepackage[usenames]{color}

\hypersetup{colorlinks=true}
\hypersetup{pdfstartview=FitH}
\hypersetup{pdfpagemode=UseNone}
\hypersetup{pdfsource={}}
\hypersetup{pdflang={en-UK}}
\hypersetup{pdfcopyright={Copyright 2017-2018 Niklas Beisert.
  This work may be distributed and/or modified under the
  conditions of the LaTeX Project Public License, either version 1.3
  of this license or (at your option) any later version.}}
\hypersetup{pdflicenseurl={http://www.latex-project.org/lppl.txt}}
\hypersetup{pdfcontactaddress={ETH Zurich, ITP, HIT K,
  Wolfgang-Pauli-Strasse 27}}
\hypersetup{pdfcontactpostcode={8093}}
\hypersetup{pdfcontactcity={Zurich}}
\hypersetup{pdfcontactcountry={Switzerland}}
\hypersetup{pdfcontactemail={nbeisert@itp.phys.ethz.ch}}
\hypersetup{pdfcontacturl={http://people.phys.ethz.ch/\xmptilde nbeisert/}}

\newcommand{\secref}[1]{\hyperref[#1]{section \ref*{#1}}}

\parskip1ex
\parindent0pt
\let\olditemize\itemize
\def\itemize{\olditemize\parskip0pt}

\begin{document}

\title{The \textsf{childdoc} Package}
\hypersetup{pdftitle={The childdoc Package}}
\author{Niklas Beisert\\[2ex]
  Institut f\"ur Theoretische Physik\\
  Eidgen\"ossische Technische Hochschule Z\"urich\\
  Wolfgang-Pauli-Strasse 27, 8093 Z\"urich, Switzerland\\[1ex]
  \href{mailto:nbeisert@itp.phys.ethz.ch}
  {\texttt{nbeisert@itp.phys.ethz.ch}}}
\hypersetup{pdfauthor={Niklas Beisert}}
\hypersetup{pdfsubject={Manual for the LaTeX2e Package childdoc}}
\date{30 December 2018, \textsf{v2.0}}
\maketitle

\begin{abstract}\noindent
\textsf{childdoc} is a \LaTeXe{} package
that enables the direct compilation
of document sections included by |\include|
to individual files.
\end{abstract}

\begingroup
\parskip0ex
\tableofcontents
\endgroup

%%%%%%%%%%%%%%%%%%%%%%%%%%%%%%%%%%%%%%%%%%%%%%%%%%%%%%%%%%%%%%%%%%%%%%%%%%%%%%%%
%%%%%%%%%%%%%%%%%%%%%%%%%%%%%%%%%%%%%%%%%%%%%%%%%%%%%%%%%%%%%%%%%%%%%%%%%%%%%%%%
\section{Introduction}

\LaTeX{} provides a mechanism to structure a large document (such as a book)
into a main file and several child files (containing the chapters)
using the |\include| command.
This mechanism is beneficial for documents
which span hundreds of pages in order to
make the source file(s) more manageable.
Moreover, compilation can be restricted to
selected child files by means of the |\includeonly| command.
The latter feature can be used to reduce the compilation time while editing
(this was significantly more useful in the earlier days of \LaTeX{})
or to generate a smaller document which is easier to navigate.
Another application of |\includeonly| is to generate
documents consisting of selected parts of the complete document.

However, there are a few drawbacks of the plain |\include| mechanism:
\begin{itemize}
\item
The child files cannot be compiled on their own,
they can only be compiled via the main file.
A naive editing environment
(such as a text editor with an option
to have the current file processed by \LaTeX)
may require one to switch to the main file before compiling;
attempting to compile the child file produces errors.
\item
The main file must be modified (each time)
to adjust the |\includeonly| command
to the present needs. This easily leaves the main file in a messy state.
\item
The generated document will always carry the filename
of the main document. This is inconvenient if
several child files are to be compiled and
to be kept for distribution.
\end{itemize}

The present package provides a simple interface
to make child files individually compilable by \LaTeX{}.
Compiling a child file then has the same effect as compiling
the main file with an |\includeonly| command
to select the appropriate child.
Moreover the generated document will carry the name of the child
rather than the main file.
This resolves all three above issues.

This feature is meant to make the editing of books,
thesis documents and lecture notes somewhat more convenient.
However, the package can also be used efficiently for
composing a series of documents (such as exercise sheets)
which are typically distributed individually.
It then assists the author in generating the individual documents
(potentially in different versions)
as well as a document containing the collected series.
Another application is in developing style files
or other kinds of included material
where compilation of the style file could redirect
to a sample or test file.

%%%%%%%%%%%%%%%%%%%%%%%%%%%%%%%%%%%%%%%%%%%%%%%%%%%%%%%%%%%%%%%%%%%%%%%%%%%%%%%%
%%%%%%%%%%%%%%%%%%%%%%%%%%%%%%%%%%%%%%%%%%%%%%%%%%%%%%%%%%%%%%%%%%%%%%%%%%%%%%%%
\section{Usage}

First of all, the package \textsf{childdoc} is \emph{not} a standard
\LaTeXe{} |.sty| style file! Therefore it needs to be invoked in
a non-standard way.

%%%%%%%%%%%%%%%%%%%%%%%%%%%%%%%%%%%%%%%%%%%%%%%%%%%%%%%%%%%%%%%%%%%%%%%%%%%%%%%%
\subsection{Included Files}
\label{sec:include}

%%%%%%%%%%%%%%%%%%%%%%%%%%%%%%%%%%%%%%%%
\DescribeMacro{\childdocmain}
To use the package, add the commands
\begin{center}
\begin{tabular}{l}
|\input{childdoc.def}|\\
|\childdocmain{}|\\
\end{tabular}
\end{center}
at the very top of the main \LaTeX{} file,
in particular \emph{before} the |\documentclass| statement!
The argument of |\childdocmain| should be left empty
(but it must be present).

%%%%%%%%%%%%%%%%%%%%%%%%%%%%%%%%%%%%%%%%
\DescribeMacro{\childdocof}
Furthermore, add the commands
\begin{center}
\begin{tabular}{l}
|\input{childdoc.def}|\\
|\childdocof{|\textit{main}|}|\\
\end{tabular}
\end{center}
at the top of every child file \textit{child}
which is included by |\include{|\textit{child}|}|
from within the main file
(or at least for those files to be compiled individually).
The argument \textit{main} must be the filename of the main file.

There are a couple of
considerations in setting up the main and child documents:

%%%%%%%%%%%%%%%%%%%%%%%%%%%%%%%%%%%%%%%%
\paragraph{Restrictions.}

Please note the following restrictions:
\begin{itemize}
\item
|\childdocmain| must be called with one argument \textit{main}
to ensure compatibility with earlier version of the package.
It must either be empty (|\childdocmain{}|)
or precisely match the filename of the main file in which it is specified.
See \secref{sec:detection} for further information.
\item
The filename \textit{main} must be specified without the |.tex| extension.
\item
The filename \textit{main} is case sensitive
(even in case-insensitive file systems)
due to internal string comparison.
\item
The argument \textit{main} should be fully expanded, it cannot be a macro.
\item
Subdirectories and special characters should be avoided in filenames.
\item
The command |\childdocmain{|\textit{main}|}| must be followed by a whitespace.
It should not be followed immediately by another command
or by a comment mark `|%|'.
This is because the \TeX{} parser reads the token immediately following
the argument of |\childdocmain| and puts it
at the beginning of every child section;
however, a white\-space is ignored.
\end{itemize}

%%%%%%%%%%%%%%%%%%%%%%%%%%%%%%%%%%%%%%%%
\paragraph{Content of Main File.}

It is advisable to place all content in the child files included by |\include|.
Any output contained in the main file will appear in all child documents
unless suppressed manually;
it cannot be suppressed automatically by the |\includeonly| directive
and thus should normally be avoided.
A method to include some content in the main file
by means of conditional processing is described in \secref{sec:conditional}.

%%%%%%%%%%%%%%%%%%%%%%%%%%%%%%%%%%%%%%%%
\paragraph{Page Numbering.}

When only a part of the document is compiled,
the appropriate numbering of pages
(as well as other status parameters)
is determined from the |.aux| files.
The latter contain information from previous passes.
However this information needs to propagate through
all intermediate child documents.
Therefore the page numbering in child documents may well
be inconsistent until the complete document is compiled at least once.

A useful (if unconventional) way to always ensure a consistent
page numbering is to restart the numbering in each child document
and denote the pages by `\textit{child}|.|\textit{page}'
where \textit{child} represents the chapter/section number of the child file.
This can be achieved by the command
|\numberwithin{page}{|\textit{child}|}|
of the \textsf{amsmath} package
where \textit{child} can be |chapter| or |section|
depending on the chosen structuring.
Alternatively, one can modify the macro |\thepage| appropriately
and reset the counter |page| at the start of each child file.

%%%%%%%%%%%%%%%%%%%%%%%%%%%%%%%%%%%%%%%%%%%%%%%%%%%%%%%%%%%%%%%%%%%%%%%%%%%%%%%%
\subsection{Conditional Processing}
\label{sec:conditional}

The package provides a mechanism to compile different versions
of a document. To customise the versions further some conditional processing
can come in handy to distinguish which version is being compiled.
The package provides two macros to describe the compilation context:

%%%%%%%%%%%%%%%%%%%%%%%%%%%%%%%%%%%%%%%%
\DescribeMacro{\ifchilddoc}
The conditional |\ifchilddoc| distinguishes between the compilation of
child documents and the main document:
%
\begin{center}
|\ifchilddoc |\textit{child-code}| |[|\||else |\textit{main-code}]| \||fi|
\end{center}

%%%%%%%%%%%%%%%%%%%%%%%%%%%%%%%%%%%%%%%%
\DescribeMacro{\childdocname}
\DescribeMacro{\childdocjob}
The macro |\childdocname| contains the filename (without extension)
of the main or child file being processed.
Note that |\childdocjob| will always contain the name of the main file.

%%%%%%%%%%%%%%%%%%%%%%%%%%%%%%%%%%%%%%%%
\paragraph{Title Page.}

Conditional processing can be used to include a title or banner page
in the main document when proper precautions are taken.
Importantly, the code in the main file should ensure that the page counter
(as well as other status parameters which are stored in the |.aux| files)
takes the same value after the conditional processing.
Otherwise the page numbers may take divergent values
depending on which part is compiled.

For example, a title page could be declared by:
%
\begin{center}
\begin{tabular}{l}
|\ifchilddoc\||else|\\
|\addtocounter{page}{-1}|\\
\textit{code for title page}\\
|\newpage|\\
|\||fi|
\end{tabular}
\end{center}
%
A banner page for the child documents can be generated by:
%
\begin{center}
\begin{tabular}{l}
|\ifchilddoc|\\
|\addtocounter{page}{-1}|\\
\textit{code for banner page}\\
|\newpage|\\
|\||fi|
\end{tabular}
\end{center}
%
Here one could write a message such as:
\begin{center}
|This is the part \childdocname{} of \childdocjob{}.|
\end{center}

%%%%%%%%%%%%%%%%%%%%%%%%%%%%%%%%%%%%%%%%%%%%%%%%%%%%%%%%%%%%%%%%%%%%%%%%%%%%%%%%
\subsection{Flags}
\label{sec:flags}

The package makes it easy to generate different versions
of the main or child documents.
To this end compilation flags can be defined
and assigned different default values.
They will be particularly useful in conjunction
with the forwarding mechanism described in \secref{sec:forward}.

For example, it may be useful to have a flag |\version|
which can be set to |draft| or |final|.
The document source will contain some conditional code
depending on the value of |\version|.
Suppose further, the flag should default to |final| for the main file
and to |draft| for child files
which is a natural assignment for editing the document.
This is achieved by placing the following code
in the preamble of the main document
(below the |\childdocmain| directive):
%
\begin{center}
\begin{tabular}{l}
|\ifchilddoc|\\
|\providecommand{\version}{draft}|\\
|\||else|\\
|\providecommand{\version}{final}|\\
|\||fi|
\end{tabular}
\end{center}
%
The definition by |\providecommand| makes sure
that previous definitions are not overwritten.
Further statements |\providecommand{\version}{...}|
can thus be added before the above code to override it.

For the main file, one might add a line
(between |\childdocmain| and the above block)
%
\begin{center}
|%\ifchilddoc\||else\providecommand{\version}{draft}\||fi|
\end{center}
%
which can be uncommented to produce a draft version.
Likewise one can add a line to the very top of a child file
(above the |\childdocof{|\textit{main}|}| directive)
%
\begin{center}
|%\providecommand{\version}{final}|
\end{center}
%
which can be uncommented to produce the final version of this child document.

%%%%%%%%%%%%%%%%%%%%%%%%%%%%%%%%%%%%%%%%%%%%%%%%%%%%%%%%%%%%%%%%%%%%%%%%%%%%%%%%
\subsection{Forwarding}
\label{sec:forward}

Different versions of the main or child documents
using compilation flags as described in \secref{sec:flags}
can be (permanently) stored in different files
for convenient compilation, viewing and distribution.
To this end, the package defines a command
to pass on compilation to a different file:

%%%%%%%%%%%%%%%%%%%%%%%%%%%%%%%%%%%%%%%%
\DescribeMacro{\childdocforward}
The command |\childdocforward| redirects processing to
another source file:
%
\begin{center}
\begin{tabular}{l}
|\input{childdoc.def}|\\
|\childdocforward[|\textit{main}|]{|\textit{dest}|}|\\
\end{tabular}
\end{center}
%
The argument \textit{dest} is the destination file
(without extension).
It should be the main file or one of the child files.
Note that further \textsf{childdoc} directives
such as |\childdocof| and |\childdocforward|
in the indicated file will be processed in this form.
The optional argument \textit{main}
passes on directly to the main file \textit{main}
while pretending to compile the child \textit{dest}.
This form behaves as if \textit{dest}
issues |\childdocof{|\textit{main}|}| right away,
and no further \textsf{childdoc} directives will be processed.

%%%%%%%%%%%%%%%%%%%%%%%%%%%%%%%%%%%%%%%%
\DescribeMacro{\...prefix}
In the alternative form |\childdocforwardprefix|,
%
\begin{center}
\begin{tabular}{l}
|\input{childdoc.def}|\\
|\childdocforwardprefix[|\textit{main}|]{|\textit{prefix}|}{|\textit{dest}|}|
\end{tabular}
\end{center}
%
the destination file is determined by a pattern
depending on the current file:
To make this work, the current file must be called
`{\textit{prefix}\hspace{0.2em}\textit{suffix}}'
with \textit{prefix} matching precisely the argument.
Processing is then passed on to the file
`{\textit{dest}\hspace{0.2em}\textit{suffix}}'.
Surely, the same effect is achieved by
directly specifying the
argument `{\textit{dest}\hspace{0.2em}\textit{suffix}}'
in the first form.
However, that requires to set up a different file
for each child. With the alternative form of the command
all these files can have exactly the same content
which simplifies setting them up and maintaining them.

For example, the following file |draft.tex|
with a compilation flag |\version| as described in \secref{sec:flags}
compiles the main document as a draft:
%
\begin{center}
\begin{tabular}{l}
|\def\version{draft}|\\
|\input{childdoc.def}|\\
|\childdocforward{|\textit{main}|}|
\end{tabular}
\end{center}
%
Likewise, the following files |final|\textit{nn}|.tex|
compile the final version of the child document
|child|\textit{nn}|.tex|:
%
\begin{center}
\begin{tabular}{l}
|\def\version{final}|\\
|\input{childdoc.def}|\\
|\childdocforwardprefix{final}{child}|
\end{tabular}
\end{center}
%

Note that when several versions of a main file and/or of each child file
are to be generated, it may be convenient to set up a |Makefile| or
shell script to automatise the process.

%%%%%%%%%%%%%%%%%%%%%%%%%%%%%%%%%%%%%%%%%%%%%%%%%%%%%%%%%%%%%%%%%%%%%%%%%%%%%%%%
\subsection{Command Line Processing}
\label{sec:commandline}

The effect of redirection files can also be achieved by invoking
the \LaTeX{} compiler with a more elaborate command line.
Most conveniently this should be done as part
of a shell script or a |Makefile|.

When using \textsf{childdoc} in the main file, the following
command lines effectively perform a redirection
(note that depending on the shell being used,
backslashes may have to be doubled: `|\|' $\to$ `|\\|'):
%
\begin{center}
|... -jobname "|\textit{target}|" |\\|"|[\textit{flags}]%
|\input{childdoc.def}\childdocforward[|\textit{main}|]{|\textit{dest}|}"|
\end{center}
%
Here \textit{target} is the name of the output file,
\textit{main} is the name of the main file
and \textit{dest} is the name of the main or child file to be processed
(all filenames without extensions).
The optional argument \textit{main} can be omitted
if \textit{main} matches \textit{dest}.
Optionally, compilation \textit{flags} can be defined via |\def| commands.
This command line makes the \TeX{} engine believe
it is compiling the file \textit{target}
whose content is specified as the latter parameter.
The provided code then forwards the processing to
\textit{main} or \textit{dest} as described in \secref{sec:forward}.

%%%%%%%%%%%%%%%%%%%%%%%%%%%%%%%%%%%%%%%%%%%%%%%%%%%%%%%%%%%%%%%%%%%%%%%%%%%%%%%%
\subsection{Include by Input}
\label{sec:input}

Including child documents by |\include| has some restrictions by design.
Most notably, the content of a child document always occupies
its own set of pages; pages cannot be shared between child documents.
Usually, this behaviour makes perfect sense
because each child document contain an essential part of the document.
However, in some situations it may be desirable to compose
a document from a collection of parts
without having mandatory page breaks between then.
For this case, the package
provides a mechanism to include parts
by |\input| which can also be processed individually.
However, by construction this mechanism
requires manual handling of the content to be output.

%%%%%%%%%%%%%%%%%%%%%%%%%%%%%%%%%%%%%%%%
\DescribeMacro{\ifchilddocmanual}
The main file should be prepared as usual, see \secref{sec:include}.
However, the document body must make a distinction
between processing of an individual part and of the main document, e.g.:
%
\begin{center}
\begin{tabular}{l}
|\ifchilddocmanual|\\
|\input{\childdocname}|\\
|\||else|\\
\textit{document body with }|\input{|\textit{part}|}|\\
|\||fi|
\end{tabular}
\end{center}
%
The conditional |\ifchilddocmanual| is true whenever
a part to be included by |\input| is being compiled,
and the name of the part is stored in |\childdocname|.

%%%%%%%%%%%%%%%%%%%%%%%%%%%%%%%%%%%%%%%%
\DescribeMacro{\childdocby}
Each part to be included by |\input| should start with:
%
\begin{center}
\begin{tabular}{l}
|\input{childdoc.def}|\\
|\childdocby{|\textit{main}|}|\\
\end{tabular}
\end{center}
%
The directive |\childdocby| is similar to |\childdocof|
described in \secref{sec:include},
but the subsequent selection of content must be done manually.
To that end, both |\ifchilddoc| and |\ifchilddocmanual|
will be true upon processing of a part,
and the name of the part is stored in |\childdocname|.
Note that |\jobname| will be set to the filename of the current part
so that each part receives an individual |.aux| file
that does not interfere with the |.aux| file(s) of the main document.
This behaviour can be altered by the alternative form
|\childdocby[*]{|\textit{main}|}| (with a non-empty optional argument)
which uses the |.aux| file of the main document
by setting |\jobname| to \textit{main}.

%%%%%%%%%%%%%%%%%%%%%%%%%%%%%%%%%%%%%%%%%%%%%%%%%%%%%%%%%%%%%%%%%%%%%%%%%%%%%%%%
\subsection{Driver Development}
\label{sec:driver}

The \textsf{childdoc} mechanism can also be use for the development
of definition files such as \LaTeX{} styles or classes.
This case differs from the above setup with multiple parts
included by |\include| in that no |\includeonly| should be invoked.
This can be achieved by starting the include file
(before |\ProvidesPackage|) with:
%
\begin{center}
\begin{tabular}{l}
|\input{childdoc.def}|\\
|\childdocforward{|\textit{main}|}|\\
\end{tabular}
\end{center}
%
or alternatively with:
%
\begin{center}
\begin{tabular}{l}
|\input{childdoc.def}|\\
|\childdocby{|\textit{main}|}|\\
\end{tabular}
\end{center}
%
Both forms have slightly different effects as described above.
The main file is prepared as usual, see \secref{sec:include}.

%%%%%%%%%%%%%%%%%%%%%%%%%%%%%%%%%%%%%%%%%%%%%%%%%%%%%%%%%%%%%%%%%%%%%%%%%%%%%%%%
\subsection{Legacy Detection}
\label{sec:detection}

The directive |\childdocmain| in the main file can detect
whether the complete document or merely a child is to be compiled
even without using the directive |\childdocof|.
This method is deprecated because it is less robust
and there is no compelling reason to use it;
it is merely provided for backward compatibility
and it may be removed in future versions.

If the detection mechanism is to be used,
it is mandatory to correctly specify
the filename of the main file as the argument of |\childdocmain|:
%
\begin{center}
\begin{tabular}{l}
|\input{childdoc.def}|\\
|\childdocmain{|\textit{main}|}|\\
\end{tabular}
\end{center}
%
If |\jobname| does not match the argument \textit{main} of |\childdocmain|,
it is assumed that |\jobname| points to the child file to be compiled.
When using |\childdocmain| with the main file specified as argument,
it suffices to start a child file
with just |\input{|\textit{main}|}|
without loading of the package and using |\childdocof|.
If instead all processing is done
with the appropriate \textsf{childdoc} directives,
the argument of \textit{main} of |\childdocmain| can be empty.

An alternative version of the command line processing described
in \secref{sec:commandline} using the detection mechanism reads:
%
\begin{center}
|... -jobname "|\textit{target}|" "|[\textit{flags}]%
[|\def\jobname{|\textit{dest}|}|]|\input{|\textit{main}|}"|
\end{center}

%%%%%%%%%%%%%%%%%%%%%%%%%%%%%%%%%%%%%%%%%%%%%%%%%%%%%%%%%%%%%%%%%%%%%%%%%%%%%%%%
\subsection{Manual Code}
\label{sec:manual}

In case one cannot be certain whether the definitions file |childdoc.def|
is installed on the target \TeX{} distribution
and one prefers not to ship it,
it is conceivable to paste a few relevant commands into the sources.

To that end, drop all statements |\input{childdoc.def}|
and perform the replacements as outlined below.
Instead of |\childdocmain{|\textit{main}|}| add the following code
to the top of the main file:
%
\begin{center}
\begin{tabular}{l}
|\||ifdefined\childdocname\endinput\||fi\newif\ifchilddoc|\\
|\edef\childdocname{\scantokens\expandafter{\jobname\noexpand}}|\\
|\def\childdocmain{|\textit{main}|}\||ifx\childdocmain\childdocname\||else|\\
|\childdoctrue\includeonly{\childdocname}\let\jobname\childdocmain\||fi|\\
\end{tabular}
\end{center}
%
Instead of |\childdocof{|\textit{main}|}| just include the main file
at the top of each child file:
%
\begin{center}
|\input{|\textit{main}|}|
\end{center}
%
A simple redirection |\childdocforward{|\textit{dest}|}| is achieved by:
%
\begin{center}
|\def\jobname{|\textit{dest}|}\input{\jobname}|
\end{center}
%
The redirection with prefix
|\childdocforwardprefix[|\textit{prefix}|]{|\textit{dest}|}|
is accomplished by:
%
\begin{center}
\begin{tabular}{l}
|{\edef\jobname{\scantokens\expandafter{\jobname\noexpand}}|\\
|\def\redirectjob |\textit{prefix}|#1~~~{\gdef\jobname{|\textit{dest}|#1}}|\\
|\expandafter\redirectjob\jobname~~~}\input{\jobname}|
\end{tabular}
\end{center}

In an alternative approach,
child documents can be compiled by a specific command line
without additional code or specific definitions:
%
\begin{center}
|... -jobname "|\textit{target}|" "|[\textit{flags}]%
|\includeonly{|\textit{dest}|}\input{|\textit{main}|}"|
\end{center}
%

%%%%%%%%%%%%%%%%%%%%%%%%%%%%%%%%%%%%%%%%%%%%%%%%%%%%%%%%%%%%%%%%%%%%%%%%%%%%%%%%
%%%%%%%%%%%%%%%%%%%%%%%%%%%%%%%%%%%%%%%%%%%%%%%%%%%%%%%%%%%%%%%%%%%%%%%%%%%%%%%%
\section{Information}

%%%%%%%%%%%%%%%%%%%%%%%%%%%%%%%%%%%%%%%%%%%%%%%%%%%%%%%%%%%%%%%%%%%%%%%%%%%%%%%%
\subsection{Copyright}

Copyright \copyright{} 2017--2018 Niklas Beisert

This work may be distributed and/or modified under the
conditions of the \LaTeX{} Project Public License, either version 1.3
of this license or (at your option) any later version.
The latest version of this license is in
  \url{http://www.latex-project.org/lppl.txt}
and version 1.3 or later is part of all distributions of \LaTeX{}
version 2005/12/01 or later.

This work has the LPPL maintenance status `maintained'.

The Current Maintainer of this work is Niklas Beisert.

This work consists of the files |README.txt|, |childdoc.ins| and |childdoc.dtx|
as well as the derived files |childdoc.def|, |cdocsamp.tex|
with |cdocsch1.tex|, |cdocsch2.tex|, |cdocspt3.tex|, |cdocspt4.tex|,
|cdocsdrf.tex|, |cdocsfn1.tex|, |cdocsfn2.tex|
as well as |childdoc.pdf|.

%%%%%%%%%%%%%%%%%%%%%%%%%%%%%%%%%%%%%%%%%%%%%%%%%%%%%%%%%%%%%%%%%%%%%%%%%%%%%%%%
\subsection{Files and Installation}

The package consists of the files:
%
\begin{center}
\begin{tabular}{ll}
    |README.txt|   & readme file \\
    |childdoc.ins| & installation file \\
    |childdoc.dtx| & source file \\
    |childdoc.def| & definition file \\
    |cdocsamp.tex| & sample main file \\
    |cdocsch1.tex| & sample include file \\
    |cdocsch2.tex| & sample include file \\
    |cdocspt3.tex| & sample part file \\
    |cdocspt4.tex| & sample part file \\
    |cdocsdrf.tex| & sample redirection file \\
    |cdocsfn1.tex| & sample redirection file \\
    |cdocsfn2.tex| & sample redirection file \\
    |childdoc.pdf| & manual
\end{tabular}
\end{center}
%
The distribution consists of the files
|README.txt|, |childdoc.ins| and |childdoc.dtx|.
%
\begin{itemize}
\item
Run (pdf)\LaTeX{} on |childdoc.dtx|
to compile the manual |childdoc.pdf| (this file).
\item
Run \LaTeX{} on |childdoc.ins| to create the definitions file |childdoc.def|
and the sample |cdocsamp.tex| with include files
|cdocsch1.tex|, |cdocsch2.tex|, |cdocspt3.tex|, |cdocspt4.tex|,
|cdocsdrf.tex|, |cdocsfn1.tex|, |cdocsfn2.tex|.
Then copy the file |childdoc.def| to an appropriate directory of your \LaTeX{}
distribution, e.g.\ \textit{texmf-root}|/tex/latex/childdoc|.
\end{itemize}

%%%%%%%%%%%%%%%%%%%%%%%%%%%%%%%%%%%%%%%%%%%%%%%%%%%%%%%%%%%%%%%%%%%%%%%%%%%%%%%%
\subsection{Related CTAN Packages}

There are several other packages which offer a similar functionality:
%
\begin{itemize}
\item
The packages
\href{http://ctan.org/pkg/docmute}{\textsf{docmute}},
\href{http://ctan.org/pkg/includex}{\textsf{includex}} and
\href{http://ctan.org/pkg/standalone}{\textsf{standalone}}
provide commands to include only the document body of
a child file thus allowing both files to be compiled individually.
\item
The packages \href{http://ctan.org/pkg/subdocs}{\textsf{subdocs}}
and \href{http://ctan.org/pkg/subfiles}{\textsf{subfiles}}
provide structures in which the main and child documents can be
encapsulated and allowing them to be compiled individually.
The inclusion mechanism is different from the conventional |\include|.
\item
The package \href{http://ctan.org/pkg/combine}{\textsf{combine}}
is an elaborate solution to combine several documents into one.
\end{itemize}
%
See also the CTAN topic \href{http://ctan.org/topic/subdocs}{\textsf{subdocs}}
for further related packages.
The present package differs from the above solutions in that
a document structure constructed with the conventional |\include| mechanism
just needs two extra commands at the top of every file
such that all constituent files can be compiled individually.

%%%%%%%%%%%%%%%%%%%%%%%%%%%%%%%%%%%%%%%%%%%%%%%%%%%%%%%%%%%%%%%%%%%%%%%%%%%%%%%%
%\subsection{Feature Suggestions}
%
%The following is a list of features which may be useful for future
%versions of this package:
%%
%\begin{itemize}
%\item
%\ldots
%\end{itemize}

%%%%%%%%%%%%%%%%%%%%%%%%%%%%%%%%%%%%%%%%%%%%%%%%%%%%%%%%%%%%%%%%%%%%%%%%%%%%%%%%
\subsection{Revision History}

%%%%%%%%%%%%%%%%%%%%%%%%%%%%%%%%%%%%%%%%
\paragraph{v2.0:} 2018/12/30

\begin{itemize}
\item
immediate forward processing
\item
added |\childdocby| mechanism
\item
manual restructured
\end{itemize}

%%%%%%%%%%%%%%%%%%%%%%%%%%%%%%%%%%%%%%%%
\paragraph{v1.6:} 2018/01/17

\begin{itemize}
\item
application for development of include files
\item
corrections to manual
\end{itemize}

%%%%%%%%%%%%%%%%%%%%%%%%%%%%%%%%%%%%%%%%
\paragraph{v1.5:} 2017/05/21

\begin{itemize}
\item
more complete structuring introduced
\item
|\childdocof| introduced
\item
|\childdoc| renamed to |\childdocmain|
\item
|\childredirect| renamed to |\childdocforward| and |\childdocforwardprefix|
and functionality expanded
\end{itemize}

%%%%%%%%%%%%%%%%%%%%%%%%%%%%%%%%%%%%%%%%
\paragraph{v1.0:} 2017/04/27

\begin{itemize}
\item
manual and install package
\item
first version published on CTAN
\end{itemize}

%%%%%%%%%%%%%%%%%%%%%%%%%%%%%%%%%%%%%%%%
\paragraph{v0.6:} 2017/04/26

\begin{itemize}
\item
redirection mechanism added
\end{itemize}

%%%%%%%%%%%%%%%%%%%%%%%%%%%%%%%%%%%%%%%%
\paragraph{v0.5:} 2017/04/26

\begin{itemize}
\item
functionality in definition file
\end{itemize}


%%%%%%%%%%%%%%%%%%%%%%%%%%%%%%%%%%%%%%%%%%%%%%%%%%%%%%%%%%%%%%%%%%%%%%%%%%%%%%%%
%%%%%%%%%%%%%%%%%%%%%%%%%%%%%%%%%%%%%%%%%%%%%%%%%%%%%%%%%%%%%%%%%%%%%%%%%%%%%%%%
%%%%%%%%%%%%%%%%%%%%%%%%%%%%%%%%%%%%%%%%%%%%%%%%%%%%%%%%%%%%%%%%%%%%%%%%%%%%%%%%
\appendix

\settowidth\MacroIndent{\rmfamily\scriptsize 000\ }

 \DocInput{childdoc.dtx}

\end{document}
%</driver>
% \fi
%
% %%%%%%%%%%%%%%%%%%%%%%%%%%%%%%%%%%%%%%%%%%%%%%%%%%%%%%%%%%%%%%%%%%%%%%%%%%%%%%
% %%%%%%%%%%%%%%%%%%%%%%%%%%%%%%%%%%%%%%%%%%%%%%%%%%%%%%%%%%%%%%%%%%%%%%%%%%%%%%
% \section{Sample}
%\iffalse
%<*samplemain>
%\fi
%
% The following presents a sample document
% with two chapters, two parts, a title page,
% a compile flag as well as three forwarding files to set the flag.
% It consists of eight |.tex| files:
% \begin{center}
% \begin{tabular}{ll}
% |cdocsamp.tex|&main file\\
% |cdocsch1.tex|&include file for chapter 1\\
% |cdocsch2.tex|&include file for chapter 2\\
% |cdocspt3.tex|&include file for part 3\\
% |cdocspt4.tex|&include file for part 4\\
% |cdocsdrf.tex|&forwarding file for main file in draft mode\\
% |cdocsfi1.tex|&forwarding file for final version of chapter 1\\
% |cdocsfi2.tex|&forwarding file for final version of chapter 2\\
% \end{tabular}
% \end{center}
% Each of the eight files can be compiled directly by the \LaTeX{} compiler.
%
% %%%%%%%%%%%%%%%%%%%%%%%%%%%%%%%%%%%%%%
% \paragraph{Main File.}
%
% The main file is called |cdocsamp.tex|.
%
% Load the \textsf{childdoc} definitions and
% declare the filename for the main document:
%    \begin{macrocode}
\input{childdoc.def}
\childdocmain{}
%    \end{macrocode}

% Optional override for |\version| flag:
%    \begin{macrocode}
%%\ifchilddoc\else\providecommand{\version}{draft}\fi
%    \end{macrocode}

% Define the default values for the |\version| flag
% (|final| for the main file and |draft| for childs):
%    \begin{macrocode}
\ifchilddoc
\providecommand{\version}{draft}
\else
\providecommand{\version}{final}
\fi
%    \end{macrocode}

% Load the standard document class:
%    \begin{macrocode}
\documentclass[12pt]{article}
%    \end{macrocode}

% Start the document body:
%    \begin{macrocode}
\begin{document}
%    \end{macrocode}

% Declare a title page.
% Print title, part of document being processed and version flag:
%    \begin{macrocode}
\addtocounter{page}{-1}
\begin{center}
{\LARGE\bfseries{}childdoc example\par}
\vspace{1cm}
\ifchilddoc
\ifchilddocmanual part\else chapter\fi:
`\childdocname' of `\childdocjob'\par
\else
main document: `\childdocjob'\par
\fi
version: \version\par
\end{center}
\newpage
%    \end{macrocode}

% Manually include selected file,
% otherwise process as usual:
%    \begin{macrocode}
\ifchilddocmanual
\section*{part `\childdocname'}
\input{\childdocname}
\else
%    \end{macrocode}

% Include the two chapters:
%    \begin{macrocode}
\include{cdocsch1}
\include{cdocsch2}
%    \end{macrocode}

% Include the two parts unless only chapters should be displayed:
%    \begin{macrocode}
\ifchilddoc\else
\section{part three}
\input{cdocspt3}
\section{part four}
\input{cdocspt4}
\fi
%    \end{macrocode}

% Process as usual until here:
%    \begin{macrocode}
\fi
%    \end{macrocode}

% End of document body:
%    \begin{macrocode}
\end{document}
%    \end{macrocode}
%\iffalse
%</samplemain>
%\fi
%
% %%%%%%%%%%%%%%%%%%%%%%%%%%%%%%%%%%%%%%
% \paragraph{Chapter Include Files.}
%
% The include files are called |cdocsch1.tex| and |cdocsch2.tex|.
%
%\iffalse
%<*samplechap1|samplechap2>
%\fi

% Optional override for |\version| flag:
%    \begin{macrocode}
%%\providecommand{\version}{final}
%    \end{macrocode}

% Include the main document:
%    \begin{macrocode}
\input{childdoc.def}
\childdocof{cdocsamp}
%    \end{macrocode}

%\iffalse
%</samplechap1|samplechap2>
%\fi
%
%\iffalse
%<*samplechap1>
%\fi
% Some text for chapter 1:
%    \begin{macrocode}
\section{one}
some text in chapter one
%    \end{macrocode}

%\iffalse
%</samplechap1>
%\fi
% Some text for chapter 2:
%\iffalse
%<*samplechap2>
%\fi
%    \begin{macrocode}
\section{two}
more text in chapter two
%    \end{macrocode}

%\iffalse
%</samplechap2>
%\fi
%
% %%%%%%%%%%%%%%%%%%%%%%%%%%%%%%%%%%%%%%
% \paragraph{Part Include Files.}
%
% The include files are called |cdocspt3.tex| and |cdocspt4.tex|.
%
%\iffalse
%<*samplepart3|samplepart4>
%\fi

% Optional override for |\version| flag:
%    \begin{macrocode}
%%\providecommand{\version}{final}
%    \end{macrocode}

% Include the main document:
%    \begin{macrocode}
\input{childdoc.def}
\childdocby{cdocsamp}
%    \end{macrocode}

%\iffalse
%</samplepart3|samplepart4>
%\fi
%
%\iffalse
%<*samplepart3>
%\fi
% Some text for part 3:
%    \begin{macrocode}
some text in part three
%    \end{macrocode}

%\iffalse
%</samplepart3>
%\fi
% Some text for part 4:
%\iffalse
%<*samplepart4>
%\fi
%    \begin{macrocode}
more text in part four
%    \end{macrocode}

%\iffalse
%</samplepart4>
%\fi
%
% %%%%%%%%%%%%%%%%%%%%%%%%%%%%%%%%%%%%%%
% \paragraph{Forwarding for a Complete Draft.}
%
% The following forwarding file |cdocsdrf.tex|
% compiles the main document in draft mode:
%\iffalse
%<*sampledraft>
%\fi
%    \begin{macrocode}
\def\version{draft}
\input{childdoc.def}
\childdocforward{cdocsamp}
%    \end{macrocode}

%\iffalse
%</sampledraft>
%\fi
%
% %%%%%%%%%%%%%%%%%%%%%%%%%%%%%%%%%%%%%%
% \paragraph{Forwarding for Final Version of the Chapters.}
%
% The following forwarding files |cdocsfn1.tex| and |cdocsfn2.tex|
% (with identical content)
% compile the final versions of the child documents
% |cdocsch1.tex| and |cdocsch2.tex|, respectively:
%\iffalse
%<*samplefinal>
%\fi
%    \begin{macrocode}
\def\version{final}
\input{childdoc.def}
\childdocforwardprefix[cdocsamp]{cdocsfn}{cdocsch}
%    \end{macrocode}

%\iffalse
%</samplefinal>
%\fi
%
% %%%%%%%%%%%%%%%%%%%%%%%%%%%%%%%%%%%%%%
% \paragraph{Command Line Processing.}
%
% The following three command lines generate the output files
% |cdocscld|, |cdocscl1| and |cdocscl2|
% which should be identical to
% |cdocsdrf|, |cdocsch1| and |cdocsfn2|, respectively:
% \begin{center}
% \begin{tabular}{l}
% |latex -jobname cdocscld \|\\
% |  "\def\version{draft}\input{childdoc.def}\childdocforward{cdocsamp}"|\\
% |latex -jobname cdocscl1 \|\\
% |  "\input{childdoc.def}\childdocforward[cdocsamp]{cdocsch1}"|\\
% |latex -jobname cdocscl2 \|\\
% |  "\def\version{final}\input{childdoc.def}\childdocforward{cdocsch2}"|
% \end{tabular}
% \end{center}
% Note that the trailing backslash on each first line
% merely continues the input to the second line
% (for convenient cut ant paste).
% Furthermore, the command |latex| can be replaced by any
% of its alternative versions such as |pdflatex|.
%
% %%%%%%%%%%%%%%%%%%%%%%%%%%%%%%%%%%%%%%%%%%%%%%%%%%%%%%%%%%%%%%%%%%%%%%%%%%%%%%
% %%%%%%%%%%%%%%%%%%%%%%%%%%%%%%%%%%%%%%%%%%%%%%%%%%%%%%%%%%%%%%%%%%%%%%%%%%%%%%
% \section{Implementation}
%\iffalse
%<*package>
%\fi
%
% This section describes the definitions file |childdoc.def|.

% The definitions cannot be loaded using |\usepackage| or |\RequirePackage|
% which has a mechanism to prevent loading a style file more than once.
% When loading the definitions by means of |\input|
% multiple instances have to be prevented manually:
%\iffalse
%This code needs to be before the `\ProvidesFile' directive
%which is defined at the beginning of this file.
%Therefore it is also placed there and commented out here.
%</package>
%<*discard>
%\fi
%    \begin{macrocode}
\ifdefined\childdocmain\endinput\fi
%    \end{macrocode}
%\iffalse
%</discard>
%<*package>
%\fi
%
% \macro{\ifchilddoc}
% \macro{\ifchilddocmanual}
% The conditional |\ifchilddoc| tells whether a
% child (true) or main (false) document is being compiled.
% The conditional |\ifchilddocmanual| tells whether
% the |\includeonly| mechanism is used (false) or
% the selection of child files must be performed manually (true).
% The definitions initialise to false:
%    \begin{macrocode}
\newif\ifchilddoc
\newif\ifchilddocmanual
%    \end{macrocode}

% \macro{\childdocname}
% \macro{\childdocjob}
% The macro |\childdocname| stores the name of the main document
% to be compiled. The macro |\childdocjob| stores the name of
% the document on which the \LaTeX{} compiler was originally invoked.
% The content of |\jobname| cannot be compared
% to filenames specified in the source due to different catcodes.
% The following code rescans |\jobname|, stores the result
% in |\childdocname| and saves a copy in |\childdocjob|:
%    \begin{macrocode}
\edef\childdocname{\scantokens\expandafter{\jobname\noexpand}}
\let\childdocjob\childdocname
%    \end{macrocode}

% \macro{\childdocdisable}
% The macro |\childdocdisable| prevents the main file
% from being processed more than once.
% At this stage, the main document command |\childdocmain|
% is assumed to be called once again where it should do nothing.
% Any subsequent call to it should prevent
% a secondary processing of the main document
% It overwrites the forwarding commands
% |\childdocof| and |\childdocforward|
% with empty macros to prevent further inclusions of the main document:
%    \begin{macrocode}
\newcommand{\childdocdisable}
{
  \renewcommand{\childdocmain}[1]{\renewcommand{\childdocmain}[1]{\endinput}}
  \renewcommand{\childdocof}[1]{}
  \renewcommand{\childdocby}[2][]{}
  \renewcommand{\childdocforward}[2][]{}
  \renewcommand{\childdocdisable}{}
}
%    \end{macrocode}

% \macro{\childdocmain}
% The macro |\childdocmain| is to be called at the top of the main file
% with nothing or the main filename (without extension) as argument.
% First, it breaks loops.
% If the argument is not empty and does not match |\childdocname|
% (which is set by the first inclusion of |childdoc.def|),
% |\ifchilddoc| is set to true, |\includeonly| is applied to the child file
% and |\jobname| is set to the main file
% (for proper handling of |.aux| files):
%    \begin{macrocode}
\newcommand{\childdocmain}[1]
{
  \childdocdisable\childdocmain{}
  \if?#1?\else
    \begingroup
      \def\childdoctmp{#1}
      \ifx\childdoctmp\childdocname
        \def\childdoctmp{}
      \else
        \def\childdoctmp
        {
          \childdoctrue
          \includeonly{\childdocname}
          \def\childdocjob{#1}
          \def\jobname{#1}
        }
      \fi
      \expandafter
    \endgroup
    \childdoctmp
  \fi
}
%    \end{macrocode}

% \macro{\childdocof}
% The command |\childdocof| redirects
% compilation to the main file |#1|.
%    \begin{macrocode}
\newcommand{\childdocof}[1]
{
  \childdocdisable
  \childdoctrue
  \includeonly{\childdocname}
  \def\jobname{#1}
  \def\childdocjob{#1}
  \input{#1}
}
%    \end{macrocode}

% \macro{\childdocby}
% The command |\childdocby| ....
%    \begin{macrocode}
\newcommand{\childdocby}[2][]
{
  \childdocdisable
  \childdoctrue
  \childdocmanualtrue
  \if?#1?\else
    \def\jobname{#2}
  \fi
  \def\childdocjob{#2}
  \input{#2}
  \endinput
}
%    \end{macrocode}

% \macro{\childdocforward}
% The command |\childdocforward| redirects
% compilation to the main file or
% (if the optional argument is given) a child file.
% Parameters are set as if the main file
% or a child file starting with |\childdocof| was compiled.
% Then compilation is handed over to the main file:
%    \begin{macrocode}
\newcommand{\childdocforward}[2][]
{
  \begingroup
    \if?#1?
      \def\childdoctmp
      {
        \def\childdocname{#2}
        \def\childdocjob{#2}
        \def\jobname{#2}
        \input{#2}
        \endinput
      }
    \else
      \def\childdoctmp
      {
        \childdocdisable
        \def\childdocname{#2}
        \childdoctrue
        \includeonly{#2}
        \def\childdocjob{#1}
        \def\jobname{#1}
        \input{#1}
        \endinput
      }
    \fi
    \expandafter
  \endgroup
  \childdoctmp
}
%    \end{macrocode}

% \macro{\childdocforwardprefix}
% The command |\childdocforwardprefix| redirects
% compilation to the main or a child file by means of a pattern.
% The prefix |#1| in the current filename is replaced by |#2|
% and the suffix of the current filename is kept
% (it is assumed that the filename does not contain the substring `|~~~|'
% which is used as a delimiter).
% Compilation is handed over to the new file by |\childdocforward|:
%    \begin{macrocode}
\newcommand{\childdocforwardprefix}[3][]
{
  \begingroup
    \def\childdocextract #2##1~~~{\def\childdoctmp{\childdocforward[#1]{#3##1}}}
    \expandafter\childdocextract\childdocname~~~
    \expandafter
  \endgroup
  \childdoctmp
}
%    \end{macrocode}

% \macro{\childdoc}
% The deprecated macro |\childdoc| is a legacy version of |\childdocmain|:
%    \begin{macrocode}
\newcommand{\childdoc}{\childdocmain}
%    \end{macrocode}

% \macro{\childdocredirect}
% The deprecated macro |\childdocredirect| is a legacy version
% of |\childdocforward| and |\childdocforwardprefix|:
%    \begin{macrocode}
\newcommand{\childdocredirect}[2][]
{
  \begingroup
    \if?#1?
      \def\childdoctmp{\childdocforward{#2}}
    \else
      \def\childdoctmp{\childdocforwardprefix{#1}{#2}}
    \fi
    \expandafter
  \endgroup
  \childdoctmp
}
%    \end{macrocode}

%\iffalse
%</package>
%\fi
%
\endinput
|\\
|\childdocmain{}|\\
\end{tabular}
\end{center}
at the very top of the main \LaTeX{} file,
in particular \emph{before} the |\documentclass| statement!
The argument of |\childdocmain| should be left empty
(but it must be present).

%%%%%%%%%%%%%%%%%%%%%%%%%%%%%%%%%%%%%%%%
\DescribeMacro{\childdocof}
Furthermore, add the commands
\begin{center}
\begin{tabular}{l}
|% \iffalse
%
% childdoc.dtx Copyright (C) 2017-2018 Niklas Beisert
%
% This work may be distributed and/or modified under the
% conditions of the LaTeX Project Public License, either version 1.3
% of this license or (at your option) any later version.
% The latest version of this license is in
%   http://www.latex-project.org/lppl.txt
% and version 1.3 or later is part of all distributions of LaTeX
% version 2005/12/01 or later.
%
% This work has the LPPL maintenance status `maintained'.
%
% The Current Maintainer of this work is Niklas Beisert.
%
% This work consists of the files childdoc.dtx and childdoc.ins
% and the derived files childdoc.def and cdocsamp.tex with
% cdocsch1.tex, cdocsch2.tex, cdocsdrf.tex, cdocsfn1.tex, cdocsfn2.tex.
%
%<package>\ifdefined\childdocmain\endinput\fi
%<package>\ProvidesFile{childdoc.def}[2018/12/30 v2.0 child document driver]
%<samplemain>\ProvidesFile{cdocsamp.tex}[2018/12/30 v2.0 sample for childdoc]
%<*driver>
%\ProvidesFile{childdoc.drv}[2018/12/30 v2.0 childdoc reference manual file]
\PassOptionsToClass{10pt,a4paper}{article}
\documentclass{ltxdoc}

\usepackage[margin=35mm]{geometry}
\usepackage{hyperref}
\usepackage{hyperxmp}
\usepackage[usenames]{color}

\hypersetup{colorlinks=true}
\hypersetup{pdfstartview=FitH}
\hypersetup{pdfpagemode=UseNone}
\hypersetup{pdfsource={}}
\hypersetup{pdflang={en-UK}}
\hypersetup{pdfcopyright={Copyright 2017-2018 Niklas Beisert.
  This work may be distributed and/or modified under the
  conditions of the LaTeX Project Public License, either version 1.3
  of this license or (at your option) any later version.}}
\hypersetup{pdflicenseurl={http://www.latex-project.org/lppl.txt}}
\hypersetup{pdfcontactaddress={ETH Zurich, ITP, HIT K,
  Wolfgang-Pauli-Strasse 27}}
\hypersetup{pdfcontactpostcode={8093}}
\hypersetup{pdfcontactcity={Zurich}}
\hypersetup{pdfcontactcountry={Switzerland}}
\hypersetup{pdfcontactemail={nbeisert@itp.phys.ethz.ch}}
\hypersetup{pdfcontacturl={http://people.phys.ethz.ch/\xmptilde nbeisert/}}

\newcommand{\secref}[1]{\hyperref[#1]{section \ref*{#1}}}

\parskip1ex
\parindent0pt
\let\olditemize\itemize
\def\itemize{\olditemize\parskip0pt}

\begin{document}

\title{The \textsf{childdoc} Package}
\hypersetup{pdftitle={The childdoc Package}}
\author{Niklas Beisert\\[2ex]
  Institut f\"ur Theoretische Physik\\
  Eidgen\"ossische Technische Hochschule Z\"urich\\
  Wolfgang-Pauli-Strasse 27, 8093 Z\"urich, Switzerland\\[1ex]
  \href{mailto:nbeisert@itp.phys.ethz.ch}
  {\texttt{nbeisert@itp.phys.ethz.ch}}}
\hypersetup{pdfauthor={Niklas Beisert}}
\hypersetup{pdfsubject={Manual for the LaTeX2e Package childdoc}}
\date{30 December 2018, \textsf{v2.0}}
\maketitle

\begin{abstract}\noindent
\textsf{childdoc} is a \LaTeXe{} package
that enables the direct compilation
of document sections included by |\include|
to individual files.
\end{abstract}

\begingroup
\parskip0ex
\tableofcontents
\endgroup

%%%%%%%%%%%%%%%%%%%%%%%%%%%%%%%%%%%%%%%%%%%%%%%%%%%%%%%%%%%%%%%%%%%%%%%%%%%%%%%%
%%%%%%%%%%%%%%%%%%%%%%%%%%%%%%%%%%%%%%%%%%%%%%%%%%%%%%%%%%%%%%%%%%%%%%%%%%%%%%%%
\section{Introduction}

\LaTeX{} provides a mechanism to structure a large document (such as a book)
into a main file and several child files (containing the chapters)
using the |\include| command.
This mechanism is beneficial for documents
which span hundreds of pages in order to
make the source file(s) more manageable.
Moreover, compilation can be restricted to
selected child files by means of the |\includeonly| command.
The latter feature can be used to reduce the compilation time while editing
(this was significantly more useful in the earlier days of \LaTeX{})
or to generate a smaller document which is easier to navigate.
Another application of |\includeonly| is to generate
documents consisting of selected parts of the complete document.

However, there are a few drawbacks of the plain |\include| mechanism:
\begin{itemize}
\item
The child files cannot be compiled on their own,
they can only be compiled via the main file.
A naive editing environment
(such as a text editor with an option
to have the current file processed by \LaTeX)
may require one to switch to the main file before compiling;
attempting to compile the child file produces errors.
\item
The main file must be modified (each time)
to adjust the |\includeonly| command
to the present needs. This easily leaves the main file in a messy state.
\item
The generated document will always carry the filename
of the main document. This is inconvenient if
several child files are to be compiled and
to be kept for distribution.
\end{itemize}

The present package provides a simple interface
to make child files individually compilable by \LaTeX{}.
Compiling a child file then has the same effect as compiling
the main file with an |\includeonly| command
to select the appropriate child.
Moreover the generated document will carry the name of the child
rather than the main file.
This resolves all three above issues.

This feature is meant to make the editing of books,
thesis documents and lecture notes somewhat more convenient.
However, the package can also be used efficiently for
composing a series of documents (such as exercise sheets)
which are typically distributed individually.
It then assists the author in generating the individual documents
(potentially in different versions)
as well as a document containing the collected series.
Another application is in developing style files
or other kinds of included material
where compilation of the style file could redirect
to a sample or test file.

%%%%%%%%%%%%%%%%%%%%%%%%%%%%%%%%%%%%%%%%%%%%%%%%%%%%%%%%%%%%%%%%%%%%%%%%%%%%%%%%
%%%%%%%%%%%%%%%%%%%%%%%%%%%%%%%%%%%%%%%%%%%%%%%%%%%%%%%%%%%%%%%%%%%%%%%%%%%%%%%%
\section{Usage}

First of all, the package \textsf{childdoc} is \emph{not} a standard
\LaTeXe{} |.sty| style file! Therefore it needs to be invoked in
a non-standard way.

%%%%%%%%%%%%%%%%%%%%%%%%%%%%%%%%%%%%%%%%%%%%%%%%%%%%%%%%%%%%%%%%%%%%%%%%%%%%%%%%
\subsection{Included Files}
\label{sec:include}

%%%%%%%%%%%%%%%%%%%%%%%%%%%%%%%%%%%%%%%%
\DescribeMacro{\childdocmain}
To use the package, add the commands
\begin{center}
\begin{tabular}{l}
|\input{childdoc.def}|\\
|\childdocmain{}|\\
\end{tabular}
\end{center}
at the very top of the main \LaTeX{} file,
in particular \emph{before} the |\documentclass| statement!
The argument of |\childdocmain| should be left empty
(but it must be present).

%%%%%%%%%%%%%%%%%%%%%%%%%%%%%%%%%%%%%%%%
\DescribeMacro{\childdocof}
Furthermore, add the commands
\begin{center}
\begin{tabular}{l}
|\input{childdoc.def}|\\
|\childdocof{|\textit{main}|}|\\
\end{tabular}
\end{center}
at the top of every child file \textit{child}
which is included by |\include{|\textit{child}|}|
from within the main file
(or at least for those files to be compiled individually).
The argument \textit{main} must be the filename of the main file.

There are a couple of
considerations in setting up the main and child documents:

%%%%%%%%%%%%%%%%%%%%%%%%%%%%%%%%%%%%%%%%
\paragraph{Restrictions.}

Please note the following restrictions:
\begin{itemize}
\item
|\childdocmain| must be called with one argument \textit{main}
to ensure compatibility with earlier version of the package.
It must either be empty (|\childdocmain{}|)
or precisely match the filename of the main file in which it is specified.
See \secref{sec:detection} for further information.
\item
The filename \textit{main} must be specified without the |.tex| extension.
\item
The filename \textit{main} is case sensitive
(even in case-insensitive file systems)
due to internal string comparison.
\item
The argument \textit{main} should be fully expanded, it cannot be a macro.
\item
Subdirectories and special characters should be avoided in filenames.
\item
The command |\childdocmain{|\textit{main}|}| must be followed by a whitespace.
It should not be followed immediately by another command
or by a comment mark `|%|'.
This is because the \TeX{} parser reads the token immediately following
the argument of |\childdocmain| and puts it
at the beginning of every child section;
however, a white\-space is ignored.
\end{itemize}

%%%%%%%%%%%%%%%%%%%%%%%%%%%%%%%%%%%%%%%%
\paragraph{Content of Main File.}

It is advisable to place all content in the child files included by |\include|.
Any output contained in the main file will appear in all child documents
unless suppressed manually;
it cannot be suppressed automatically by the |\includeonly| directive
and thus should normally be avoided.
A method to include some content in the main file
by means of conditional processing is described in \secref{sec:conditional}.

%%%%%%%%%%%%%%%%%%%%%%%%%%%%%%%%%%%%%%%%
\paragraph{Page Numbering.}

When only a part of the document is compiled,
the appropriate numbering of pages
(as well as other status parameters)
is determined from the |.aux| files.
The latter contain information from previous passes.
However this information needs to propagate through
all intermediate child documents.
Therefore the page numbering in child documents may well
be inconsistent until the complete document is compiled at least once.

A useful (if unconventional) way to always ensure a consistent
page numbering is to restart the numbering in each child document
and denote the pages by `\textit{child}|.|\textit{page}'
where \textit{child} represents the chapter/section number of the child file.
This can be achieved by the command
|\numberwithin{page}{|\textit{child}|}|
of the \textsf{amsmath} package
where \textit{child} can be |chapter| or |section|
depending on the chosen structuring.
Alternatively, one can modify the macro |\thepage| appropriately
and reset the counter |page| at the start of each child file.

%%%%%%%%%%%%%%%%%%%%%%%%%%%%%%%%%%%%%%%%%%%%%%%%%%%%%%%%%%%%%%%%%%%%%%%%%%%%%%%%
\subsection{Conditional Processing}
\label{sec:conditional}

The package provides a mechanism to compile different versions
of a document. To customise the versions further some conditional processing
can come in handy to distinguish which version is being compiled.
The package provides two macros to describe the compilation context:

%%%%%%%%%%%%%%%%%%%%%%%%%%%%%%%%%%%%%%%%
\DescribeMacro{\ifchilddoc}
The conditional |\ifchilddoc| distinguishes between the compilation of
child documents and the main document:
%
\begin{center}
|\ifchilddoc |\textit{child-code}| |[|\||else |\textit{main-code}]| \||fi|
\end{center}

%%%%%%%%%%%%%%%%%%%%%%%%%%%%%%%%%%%%%%%%
\DescribeMacro{\childdocname}
\DescribeMacro{\childdocjob}
The macro |\childdocname| contains the filename (without extension)
of the main or child file being processed.
Note that |\childdocjob| will always contain the name of the main file.

%%%%%%%%%%%%%%%%%%%%%%%%%%%%%%%%%%%%%%%%
\paragraph{Title Page.}

Conditional processing can be used to include a title or banner page
in the main document when proper precautions are taken.
Importantly, the code in the main file should ensure that the page counter
(as well as other status parameters which are stored in the |.aux| files)
takes the same value after the conditional processing.
Otherwise the page numbers may take divergent values
depending on which part is compiled.

For example, a title page could be declared by:
%
\begin{center}
\begin{tabular}{l}
|\ifchilddoc\||else|\\
|\addtocounter{page}{-1}|\\
\textit{code for title page}\\
|\newpage|\\
|\||fi|
\end{tabular}
\end{center}
%
A banner page for the child documents can be generated by:
%
\begin{center}
\begin{tabular}{l}
|\ifchilddoc|\\
|\addtocounter{page}{-1}|\\
\textit{code for banner page}\\
|\newpage|\\
|\||fi|
\end{tabular}
\end{center}
%
Here one could write a message such as:
\begin{center}
|This is the part \childdocname{} of \childdocjob{}.|
\end{center}

%%%%%%%%%%%%%%%%%%%%%%%%%%%%%%%%%%%%%%%%%%%%%%%%%%%%%%%%%%%%%%%%%%%%%%%%%%%%%%%%
\subsection{Flags}
\label{sec:flags}

The package makes it easy to generate different versions
of the main or child documents.
To this end compilation flags can be defined
and assigned different default values.
They will be particularly useful in conjunction
with the forwarding mechanism described in \secref{sec:forward}.

For example, it may be useful to have a flag |\version|
which can be set to |draft| or |final|.
The document source will contain some conditional code
depending on the value of |\version|.
Suppose further, the flag should default to |final| for the main file
and to |draft| for child files
which is a natural assignment for editing the document.
This is achieved by placing the following code
in the preamble of the main document
(below the |\childdocmain| directive):
%
\begin{center}
\begin{tabular}{l}
|\ifchilddoc|\\
|\providecommand{\version}{draft}|\\
|\||else|\\
|\providecommand{\version}{final}|\\
|\||fi|
\end{tabular}
\end{center}
%
The definition by |\providecommand| makes sure
that previous definitions are not overwritten.
Further statements |\providecommand{\version}{...}|
can thus be added before the above code to override it.

For the main file, one might add a line
(between |\childdocmain| and the above block)
%
\begin{center}
|%\ifchilddoc\||else\providecommand{\version}{draft}\||fi|
\end{center}
%
which can be uncommented to produce a draft version.
Likewise one can add a line to the very top of a child file
(above the |\childdocof{|\textit{main}|}| directive)
%
\begin{center}
|%\providecommand{\version}{final}|
\end{center}
%
which can be uncommented to produce the final version of this child document.

%%%%%%%%%%%%%%%%%%%%%%%%%%%%%%%%%%%%%%%%%%%%%%%%%%%%%%%%%%%%%%%%%%%%%%%%%%%%%%%%
\subsection{Forwarding}
\label{sec:forward}

Different versions of the main or child documents
using compilation flags as described in \secref{sec:flags}
can be (permanently) stored in different files
for convenient compilation, viewing and distribution.
To this end, the package defines a command
to pass on compilation to a different file:

%%%%%%%%%%%%%%%%%%%%%%%%%%%%%%%%%%%%%%%%
\DescribeMacro{\childdocforward}
The command |\childdocforward| redirects processing to
another source file:
%
\begin{center}
\begin{tabular}{l}
|\input{childdoc.def}|\\
|\childdocforward[|\textit{main}|]{|\textit{dest}|}|\\
\end{tabular}
\end{center}
%
The argument \textit{dest} is the destination file
(without extension).
It should be the main file or one of the child files.
Note that further \textsf{childdoc} directives
such as |\childdocof| and |\childdocforward|
in the indicated file will be processed in this form.
The optional argument \textit{main}
passes on directly to the main file \textit{main}
while pretending to compile the child \textit{dest}.
This form behaves as if \textit{dest}
issues |\childdocof{|\textit{main}|}| right away,
and no further \textsf{childdoc} directives will be processed.

%%%%%%%%%%%%%%%%%%%%%%%%%%%%%%%%%%%%%%%%
\DescribeMacro{\...prefix}
In the alternative form |\childdocforwardprefix|,
%
\begin{center}
\begin{tabular}{l}
|\input{childdoc.def}|\\
|\childdocforwardprefix[|\textit{main}|]{|\textit{prefix}|}{|\textit{dest}|}|
\end{tabular}
\end{center}
%
the destination file is determined by a pattern
depending on the current file:
To make this work, the current file must be called
`{\textit{prefix}\hspace{0.2em}\textit{suffix}}'
with \textit{prefix} matching precisely the argument.
Processing is then passed on to the file
`{\textit{dest}\hspace{0.2em}\textit{suffix}}'.
Surely, the same effect is achieved by
directly specifying the
argument `{\textit{dest}\hspace{0.2em}\textit{suffix}}'
in the first form.
However, that requires to set up a different file
for each child. With the alternative form of the command
all these files can have exactly the same content
which simplifies setting them up and maintaining them.

For example, the following file |draft.tex|
with a compilation flag |\version| as described in \secref{sec:flags}
compiles the main document as a draft:
%
\begin{center}
\begin{tabular}{l}
|\def\version{draft}|\\
|\input{childdoc.def}|\\
|\childdocforward{|\textit{main}|}|
\end{tabular}
\end{center}
%
Likewise, the following files |final|\textit{nn}|.tex|
compile the final version of the child document
|child|\textit{nn}|.tex|:
%
\begin{center}
\begin{tabular}{l}
|\def\version{final}|\\
|\input{childdoc.def}|\\
|\childdocforwardprefix{final}{child}|
\end{tabular}
\end{center}
%

Note that when several versions of a main file and/or of each child file
are to be generated, it may be convenient to set up a |Makefile| or
shell script to automatise the process.

%%%%%%%%%%%%%%%%%%%%%%%%%%%%%%%%%%%%%%%%%%%%%%%%%%%%%%%%%%%%%%%%%%%%%%%%%%%%%%%%
\subsection{Command Line Processing}
\label{sec:commandline}

The effect of redirection files can also be achieved by invoking
the \LaTeX{} compiler with a more elaborate command line.
Most conveniently this should be done as part
of a shell script or a |Makefile|.

When using \textsf{childdoc} in the main file, the following
command lines effectively perform a redirection
(note that depending on the shell being used,
backslashes may have to be doubled: `|\|' $\to$ `|\\|'):
%
\begin{center}
|... -jobname "|\textit{target}|" |\\|"|[\textit{flags}]%
|\input{childdoc.def}\childdocforward[|\textit{main}|]{|\textit{dest}|}"|
\end{center}
%
Here \textit{target} is the name of the output file,
\textit{main} is the name of the main file
and \textit{dest} is the name of the main or child file to be processed
(all filenames without extensions).
The optional argument \textit{main} can be omitted
if \textit{main} matches \textit{dest}.
Optionally, compilation \textit{flags} can be defined via |\def| commands.
This command line makes the \TeX{} engine believe
it is compiling the file \textit{target}
whose content is specified as the latter parameter.
The provided code then forwards the processing to
\textit{main} or \textit{dest} as described in \secref{sec:forward}.

%%%%%%%%%%%%%%%%%%%%%%%%%%%%%%%%%%%%%%%%%%%%%%%%%%%%%%%%%%%%%%%%%%%%%%%%%%%%%%%%
\subsection{Include by Input}
\label{sec:input}

Including child documents by |\include| has some restrictions by design.
Most notably, the content of a child document always occupies
its own set of pages; pages cannot be shared between child documents.
Usually, this behaviour makes perfect sense
because each child document contain an essential part of the document.
However, in some situations it may be desirable to compose
a document from a collection of parts
without having mandatory page breaks between then.
For this case, the package
provides a mechanism to include parts
by |\input| which can also be processed individually.
However, by construction this mechanism
requires manual handling of the content to be output.

%%%%%%%%%%%%%%%%%%%%%%%%%%%%%%%%%%%%%%%%
\DescribeMacro{\ifchilddocmanual}
The main file should be prepared as usual, see \secref{sec:include}.
However, the document body must make a distinction
between processing of an individual part and of the main document, e.g.:
%
\begin{center}
\begin{tabular}{l}
|\ifchilddocmanual|\\
|\input{\childdocname}|\\
|\||else|\\
\textit{document body with }|\input{|\textit{part}|}|\\
|\||fi|
\end{tabular}
\end{center}
%
The conditional |\ifchilddocmanual| is true whenever
a part to be included by |\input| is being compiled,
and the name of the part is stored in |\childdocname|.

%%%%%%%%%%%%%%%%%%%%%%%%%%%%%%%%%%%%%%%%
\DescribeMacro{\childdocby}
Each part to be included by |\input| should start with:
%
\begin{center}
\begin{tabular}{l}
|\input{childdoc.def}|\\
|\childdocby{|\textit{main}|}|\\
\end{tabular}
\end{center}
%
The directive |\childdocby| is similar to |\childdocof|
described in \secref{sec:include},
but the subsequent selection of content must be done manually.
To that end, both |\ifchilddoc| and |\ifchilddocmanual|
will be true upon processing of a part,
and the name of the part is stored in |\childdocname|.
Note that |\jobname| will be set to the filename of the current part
so that each part receives an individual |.aux| file
that does not interfere with the |.aux| file(s) of the main document.
This behaviour can be altered by the alternative form
|\childdocby[*]{|\textit{main}|}| (with a non-empty optional argument)
which uses the |.aux| file of the main document
by setting |\jobname| to \textit{main}.

%%%%%%%%%%%%%%%%%%%%%%%%%%%%%%%%%%%%%%%%%%%%%%%%%%%%%%%%%%%%%%%%%%%%%%%%%%%%%%%%
\subsection{Driver Development}
\label{sec:driver}

The \textsf{childdoc} mechanism can also be use for the development
of definition files such as \LaTeX{} styles or classes.
This case differs from the above setup with multiple parts
included by |\include| in that no |\includeonly| should be invoked.
This can be achieved by starting the include file
(before |\ProvidesPackage|) with:
%
\begin{center}
\begin{tabular}{l}
|\input{childdoc.def}|\\
|\childdocforward{|\textit{main}|}|\\
\end{tabular}
\end{center}
%
or alternatively with:
%
\begin{center}
\begin{tabular}{l}
|\input{childdoc.def}|\\
|\childdocby{|\textit{main}|}|\\
\end{tabular}
\end{center}
%
Both forms have slightly different effects as described above.
The main file is prepared as usual, see \secref{sec:include}.

%%%%%%%%%%%%%%%%%%%%%%%%%%%%%%%%%%%%%%%%%%%%%%%%%%%%%%%%%%%%%%%%%%%%%%%%%%%%%%%%
\subsection{Legacy Detection}
\label{sec:detection}

The directive |\childdocmain| in the main file can detect
whether the complete document or merely a child is to be compiled
even without using the directive |\childdocof|.
This method is deprecated because it is less robust
and there is no compelling reason to use it;
it is merely provided for backward compatibility
and it may be removed in future versions.

If the detection mechanism is to be used,
it is mandatory to correctly specify
the filename of the main file as the argument of |\childdocmain|:
%
\begin{center}
\begin{tabular}{l}
|\input{childdoc.def}|\\
|\childdocmain{|\textit{main}|}|\\
\end{tabular}
\end{center}
%
If |\jobname| does not match the argument \textit{main} of |\childdocmain|,
it is assumed that |\jobname| points to the child file to be compiled.
When using |\childdocmain| with the main file specified as argument,
it suffices to start a child file
with just |\input{|\textit{main}|}|
without loading of the package and using |\childdocof|.
If instead all processing is done
with the appropriate \textsf{childdoc} directives,
the argument of \textit{main} of |\childdocmain| can be empty.

An alternative version of the command line processing described
in \secref{sec:commandline} using the detection mechanism reads:
%
\begin{center}
|... -jobname "|\textit{target}|" "|[\textit{flags}]%
[|\def\jobname{|\textit{dest}|}|]|\input{|\textit{main}|}"|
\end{center}

%%%%%%%%%%%%%%%%%%%%%%%%%%%%%%%%%%%%%%%%%%%%%%%%%%%%%%%%%%%%%%%%%%%%%%%%%%%%%%%%
\subsection{Manual Code}
\label{sec:manual}

In case one cannot be certain whether the definitions file |childdoc.def|
is installed on the target \TeX{} distribution
and one prefers not to ship it,
it is conceivable to paste a few relevant commands into the sources.

To that end, drop all statements |\input{childdoc.def}|
and perform the replacements as outlined below.
Instead of |\childdocmain{|\textit{main}|}| add the following code
to the top of the main file:
%
\begin{center}
\begin{tabular}{l}
|\||ifdefined\childdocname\endinput\||fi\newif\ifchilddoc|\\
|\edef\childdocname{\scantokens\expandafter{\jobname\noexpand}}|\\
|\def\childdocmain{|\textit{main}|}\||ifx\childdocmain\childdocname\||else|\\
|\childdoctrue\includeonly{\childdocname}\let\jobname\childdocmain\||fi|\\
\end{tabular}
\end{center}
%
Instead of |\childdocof{|\textit{main}|}| just include the main file
at the top of each child file:
%
\begin{center}
|\input{|\textit{main}|}|
\end{center}
%
A simple redirection |\childdocforward{|\textit{dest}|}| is achieved by:
%
\begin{center}
|\def\jobname{|\textit{dest}|}\input{\jobname}|
\end{center}
%
The redirection with prefix
|\childdocforwardprefix[|\textit{prefix}|]{|\textit{dest}|}|
is accomplished by:
%
\begin{center}
\begin{tabular}{l}
|{\edef\jobname{\scantokens\expandafter{\jobname\noexpand}}|\\
|\def\redirectjob |\textit{prefix}|#1~~~{\gdef\jobname{|\textit{dest}|#1}}|\\
|\expandafter\redirectjob\jobname~~~}\input{\jobname}|
\end{tabular}
\end{center}

In an alternative approach,
child documents can be compiled by a specific command line
without additional code or specific definitions:
%
\begin{center}
|... -jobname "|\textit{target}|" "|[\textit{flags}]%
|\includeonly{|\textit{dest}|}\input{|\textit{main}|}"|
\end{center}
%

%%%%%%%%%%%%%%%%%%%%%%%%%%%%%%%%%%%%%%%%%%%%%%%%%%%%%%%%%%%%%%%%%%%%%%%%%%%%%%%%
%%%%%%%%%%%%%%%%%%%%%%%%%%%%%%%%%%%%%%%%%%%%%%%%%%%%%%%%%%%%%%%%%%%%%%%%%%%%%%%%
\section{Information}

%%%%%%%%%%%%%%%%%%%%%%%%%%%%%%%%%%%%%%%%%%%%%%%%%%%%%%%%%%%%%%%%%%%%%%%%%%%%%%%%
\subsection{Copyright}

Copyright \copyright{} 2017--2018 Niklas Beisert

This work may be distributed and/or modified under the
conditions of the \LaTeX{} Project Public License, either version 1.3
of this license or (at your option) any later version.
The latest version of this license is in
  \url{http://www.latex-project.org/lppl.txt}
and version 1.3 or later is part of all distributions of \LaTeX{}
version 2005/12/01 or later.

This work has the LPPL maintenance status `maintained'.

The Current Maintainer of this work is Niklas Beisert.

This work consists of the files |README.txt|, |childdoc.ins| and |childdoc.dtx|
as well as the derived files |childdoc.def|, |cdocsamp.tex|
with |cdocsch1.tex|, |cdocsch2.tex|, |cdocspt3.tex|, |cdocspt4.tex|,
|cdocsdrf.tex|, |cdocsfn1.tex|, |cdocsfn2.tex|
as well as |childdoc.pdf|.

%%%%%%%%%%%%%%%%%%%%%%%%%%%%%%%%%%%%%%%%%%%%%%%%%%%%%%%%%%%%%%%%%%%%%%%%%%%%%%%%
\subsection{Files and Installation}

The package consists of the files:
%
\begin{center}
\begin{tabular}{ll}
    |README.txt|   & readme file \\
    |childdoc.ins| & installation file \\
    |childdoc.dtx| & source file \\
    |childdoc.def| & definition file \\
    |cdocsamp.tex| & sample main file \\
    |cdocsch1.tex| & sample include file \\
    |cdocsch2.tex| & sample include file \\
    |cdocspt3.tex| & sample part file \\
    |cdocspt4.tex| & sample part file \\
    |cdocsdrf.tex| & sample redirection file \\
    |cdocsfn1.tex| & sample redirection file \\
    |cdocsfn2.tex| & sample redirection file \\
    |childdoc.pdf| & manual
\end{tabular}
\end{center}
%
The distribution consists of the files
|README.txt|, |childdoc.ins| and |childdoc.dtx|.
%
\begin{itemize}
\item
Run (pdf)\LaTeX{} on |childdoc.dtx|
to compile the manual |childdoc.pdf| (this file).
\item
Run \LaTeX{} on |childdoc.ins| to create the definitions file |childdoc.def|
and the sample |cdocsamp.tex| with include files
|cdocsch1.tex|, |cdocsch2.tex|, |cdocspt3.tex|, |cdocspt4.tex|,
|cdocsdrf.tex|, |cdocsfn1.tex|, |cdocsfn2.tex|.
Then copy the file |childdoc.def| to an appropriate directory of your \LaTeX{}
distribution, e.g.\ \textit{texmf-root}|/tex/latex/childdoc|.
\end{itemize}

%%%%%%%%%%%%%%%%%%%%%%%%%%%%%%%%%%%%%%%%%%%%%%%%%%%%%%%%%%%%%%%%%%%%%%%%%%%%%%%%
\subsection{Related CTAN Packages}

There are several other packages which offer a similar functionality:
%
\begin{itemize}
\item
The packages
\href{http://ctan.org/pkg/docmute}{\textsf{docmute}},
\href{http://ctan.org/pkg/includex}{\textsf{includex}} and
\href{http://ctan.org/pkg/standalone}{\textsf{standalone}}
provide commands to include only the document body of
a child file thus allowing both files to be compiled individually.
\item
The packages \href{http://ctan.org/pkg/subdocs}{\textsf{subdocs}}
and \href{http://ctan.org/pkg/subfiles}{\textsf{subfiles}}
provide structures in which the main and child documents can be
encapsulated and allowing them to be compiled individually.
The inclusion mechanism is different from the conventional |\include|.
\item
The package \href{http://ctan.org/pkg/combine}{\textsf{combine}}
is an elaborate solution to combine several documents into one.
\end{itemize}
%
See also the CTAN topic \href{http://ctan.org/topic/subdocs}{\textsf{subdocs}}
for further related packages.
The present package differs from the above solutions in that
a document structure constructed with the conventional |\include| mechanism
just needs two extra commands at the top of every file
such that all constituent files can be compiled individually.

%%%%%%%%%%%%%%%%%%%%%%%%%%%%%%%%%%%%%%%%%%%%%%%%%%%%%%%%%%%%%%%%%%%%%%%%%%%%%%%%
%\subsection{Feature Suggestions}
%
%The following is a list of features which may be useful for future
%versions of this package:
%%
%\begin{itemize}
%\item
%\ldots
%\end{itemize}

%%%%%%%%%%%%%%%%%%%%%%%%%%%%%%%%%%%%%%%%%%%%%%%%%%%%%%%%%%%%%%%%%%%%%%%%%%%%%%%%
\subsection{Revision History}

%%%%%%%%%%%%%%%%%%%%%%%%%%%%%%%%%%%%%%%%
\paragraph{v2.0:} 2018/12/30

\begin{itemize}
\item
immediate forward processing
\item
added |\childdocby| mechanism
\item
manual restructured
\end{itemize}

%%%%%%%%%%%%%%%%%%%%%%%%%%%%%%%%%%%%%%%%
\paragraph{v1.6:} 2018/01/17

\begin{itemize}
\item
application for development of include files
\item
corrections to manual
\end{itemize}

%%%%%%%%%%%%%%%%%%%%%%%%%%%%%%%%%%%%%%%%
\paragraph{v1.5:} 2017/05/21

\begin{itemize}
\item
more complete structuring introduced
\item
|\childdocof| introduced
\item
|\childdoc| renamed to |\childdocmain|
\item
|\childredirect| renamed to |\childdocforward| and |\childdocforwardprefix|
and functionality expanded
\end{itemize}

%%%%%%%%%%%%%%%%%%%%%%%%%%%%%%%%%%%%%%%%
\paragraph{v1.0:} 2017/04/27

\begin{itemize}
\item
manual and install package
\item
first version published on CTAN
\end{itemize}

%%%%%%%%%%%%%%%%%%%%%%%%%%%%%%%%%%%%%%%%
\paragraph{v0.6:} 2017/04/26

\begin{itemize}
\item
redirection mechanism added
\end{itemize}

%%%%%%%%%%%%%%%%%%%%%%%%%%%%%%%%%%%%%%%%
\paragraph{v0.5:} 2017/04/26

\begin{itemize}
\item
functionality in definition file
\end{itemize}


%%%%%%%%%%%%%%%%%%%%%%%%%%%%%%%%%%%%%%%%%%%%%%%%%%%%%%%%%%%%%%%%%%%%%%%%%%%%%%%%
%%%%%%%%%%%%%%%%%%%%%%%%%%%%%%%%%%%%%%%%%%%%%%%%%%%%%%%%%%%%%%%%%%%%%%%%%%%%%%%%
%%%%%%%%%%%%%%%%%%%%%%%%%%%%%%%%%%%%%%%%%%%%%%%%%%%%%%%%%%%%%%%%%%%%%%%%%%%%%%%%
\appendix

\settowidth\MacroIndent{\rmfamily\scriptsize 000\ }

 \DocInput{childdoc.dtx}

\end{document}
%</driver>
% \fi
%
% %%%%%%%%%%%%%%%%%%%%%%%%%%%%%%%%%%%%%%%%%%%%%%%%%%%%%%%%%%%%%%%%%%%%%%%%%%%%%%
% %%%%%%%%%%%%%%%%%%%%%%%%%%%%%%%%%%%%%%%%%%%%%%%%%%%%%%%%%%%%%%%%%%%%%%%%%%%%%%
% \section{Sample}
%\iffalse
%<*samplemain>
%\fi
%
% The following presents a sample document
% with two chapters, two parts, a title page,
% a compile flag as well as three forwarding files to set the flag.
% It consists of eight |.tex| files:
% \begin{center}
% \begin{tabular}{ll}
% |cdocsamp.tex|&main file\\
% |cdocsch1.tex|&include file for chapter 1\\
% |cdocsch2.tex|&include file for chapter 2\\
% |cdocspt3.tex|&include file for part 3\\
% |cdocspt4.tex|&include file for part 4\\
% |cdocsdrf.tex|&forwarding file for main file in draft mode\\
% |cdocsfi1.tex|&forwarding file for final version of chapter 1\\
% |cdocsfi2.tex|&forwarding file for final version of chapter 2\\
% \end{tabular}
% \end{center}
% Each of the eight files can be compiled directly by the \LaTeX{} compiler.
%
% %%%%%%%%%%%%%%%%%%%%%%%%%%%%%%%%%%%%%%
% \paragraph{Main File.}
%
% The main file is called |cdocsamp.tex|.
%
% Load the \textsf{childdoc} definitions and
% declare the filename for the main document:
%    \begin{macrocode}
\input{childdoc.def}
\childdocmain{}
%    \end{macrocode}

% Optional override for |\version| flag:
%    \begin{macrocode}
%%\ifchilddoc\else\providecommand{\version}{draft}\fi
%    \end{macrocode}

% Define the default values for the |\version| flag
% (|final| for the main file and |draft| for childs):
%    \begin{macrocode}
\ifchilddoc
\providecommand{\version}{draft}
\else
\providecommand{\version}{final}
\fi
%    \end{macrocode}

% Load the standard document class:
%    \begin{macrocode}
\documentclass[12pt]{article}
%    \end{macrocode}

% Start the document body:
%    \begin{macrocode}
\begin{document}
%    \end{macrocode}

% Declare a title page.
% Print title, part of document being processed and version flag:
%    \begin{macrocode}
\addtocounter{page}{-1}
\begin{center}
{\LARGE\bfseries{}childdoc example\par}
\vspace{1cm}
\ifchilddoc
\ifchilddocmanual part\else chapter\fi:
`\childdocname' of `\childdocjob'\par
\else
main document: `\childdocjob'\par
\fi
version: \version\par
\end{center}
\newpage
%    \end{macrocode}

% Manually include selected file,
% otherwise process as usual:
%    \begin{macrocode}
\ifchilddocmanual
\section*{part `\childdocname'}
\input{\childdocname}
\else
%    \end{macrocode}

% Include the two chapters:
%    \begin{macrocode}
\include{cdocsch1}
\include{cdocsch2}
%    \end{macrocode}

% Include the two parts unless only chapters should be displayed:
%    \begin{macrocode}
\ifchilddoc\else
\section{part three}
\input{cdocspt3}
\section{part four}
\input{cdocspt4}
\fi
%    \end{macrocode}

% Process as usual until here:
%    \begin{macrocode}
\fi
%    \end{macrocode}

% End of document body:
%    \begin{macrocode}
\end{document}
%    \end{macrocode}
%\iffalse
%</samplemain>
%\fi
%
% %%%%%%%%%%%%%%%%%%%%%%%%%%%%%%%%%%%%%%
% \paragraph{Chapter Include Files.}
%
% The include files are called |cdocsch1.tex| and |cdocsch2.tex|.
%
%\iffalse
%<*samplechap1|samplechap2>
%\fi

% Optional override for |\version| flag:
%    \begin{macrocode}
%%\providecommand{\version}{final}
%    \end{macrocode}

% Include the main document:
%    \begin{macrocode}
\input{childdoc.def}
\childdocof{cdocsamp}
%    \end{macrocode}

%\iffalse
%</samplechap1|samplechap2>
%\fi
%
%\iffalse
%<*samplechap1>
%\fi
% Some text for chapter 1:
%    \begin{macrocode}
\section{one}
some text in chapter one
%    \end{macrocode}

%\iffalse
%</samplechap1>
%\fi
% Some text for chapter 2:
%\iffalse
%<*samplechap2>
%\fi
%    \begin{macrocode}
\section{two}
more text in chapter two
%    \end{macrocode}

%\iffalse
%</samplechap2>
%\fi
%
% %%%%%%%%%%%%%%%%%%%%%%%%%%%%%%%%%%%%%%
% \paragraph{Part Include Files.}
%
% The include files are called |cdocspt3.tex| and |cdocspt4.tex|.
%
%\iffalse
%<*samplepart3|samplepart4>
%\fi

% Optional override for |\version| flag:
%    \begin{macrocode}
%%\providecommand{\version}{final}
%    \end{macrocode}

% Include the main document:
%    \begin{macrocode}
\input{childdoc.def}
\childdocby{cdocsamp}
%    \end{macrocode}

%\iffalse
%</samplepart3|samplepart4>
%\fi
%
%\iffalse
%<*samplepart3>
%\fi
% Some text for part 3:
%    \begin{macrocode}
some text in part three
%    \end{macrocode}

%\iffalse
%</samplepart3>
%\fi
% Some text for part 4:
%\iffalse
%<*samplepart4>
%\fi
%    \begin{macrocode}
more text in part four
%    \end{macrocode}

%\iffalse
%</samplepart4>
%\fi
%
% %%%%%%%%%%%%%%%%%%%%%%%%%%%%%%%%%%%%%%
% \paragraph{Forwarding for a Complete Draft.}
%
% The following forwarding file |cdocsdrf.tex|
% compiles the main document in draft mode:
%\iffalse
%<*sampledraft>
%\fi
%    \begin{macrocode}
\def\version{draft}
\input{childdoc.def}
\childdocforward{cdocsamp}
%    \end{macrocode}

%\iffalse
%</sampledraft>
%\fi
%
% %%%%%%%%%%%%%%%%%%%%%%%%%%%%%%%%%%%%%%
% \paragraph{Forwarding for Final Version of the Chapters.}
%
% The following forwarding files |cdocsfn1.tex| and |cdocsfn2.tex|
% (with identical content)
% compile the final versions of the child documents
% |cdocsch1.tex| and |cdocsch2.tex|, respectively:
%\iffalse
%<*samplefinal>
%\fi
%    \begin{macrocode}
\def\version{final}
\input{childdoc.def}
\childdocforwardprefix[cdocsamp]{cdocsfn}{cdocsch}
%    \end{macrocode}

%\iffalse
%</samplefinal>
%\fi
%
% %%%%%%%%%%%%%%%%%%%%%%%%%%%%%%%%%%%%%%
% \paragraph{Command Line Processing.}
%
% The following three command lines generate the output files
% |cdocscld|, |cdocscl1| and |cdocscl2|
% which should be identical to
% |cdocsdrf|, |cdocsch1| and |cdocsfn2|, respectively:
% \begin{center}
% \begin{tabular}{l}
% |latex -jobname cdocscld \|\\
% |  "\def\version{draft}\input{childdoc.def}\childdocforward{cdocsamp}"|\\
% |latex -jobname cdocscl1 \|\\
% |  "\input{childdoc.def}\childdocforward[cdocsamp]{cdocsch1}"|\\
% |latex -jobname cdocscl2 \|\\
% |  "\def\version{final}\input{childdoc.def}\childdocforward{cdocsch2}"|
% \end{tabular}
% \end{center}
% Note that the trailing backslash on each first line
% merely continues the input to the second line
% (for convenient cut ant paste).
% Furthermore, the command |latex| can be replaced by any
% of its alternative versions such as |pdflatex|.
%
% %%%%%%%%%%%%%%%%%%%%%%%%%%%%%%%%%%%%%%%%%%%%%%%%%%%%%%%%%%%%%%%%%%%%%%%%%%%%%%
% %%%%%%%%%%%%%%%%%%%%%%%%%%%%%%%%%%%%%%%%%%%%%%%%%%%%%%%%%%%%%%%%%%%%%%%%%%%%%%
% \section{Implementation}
%\iffalse
%<*package>
%\fi
%
% This section describes the definitions file |childdoc.def|.

% The definitions cannot be loaded using |\usepackage| or |\RequirePackage|
% which has a mechanism to prevent loading a style file more than once.
% When loading the definitions by means of |\input|
% multiple instances have to be prevented manually:
%\iffalse
%This code needs to be before the `\ProvidesFile' directive
%which is defined at the beginning of this file.
%Therefore it is also placed there and commented out here.
%</package>
%<*discard>
%\fi
%    \begin{macrocode}
\ifdefined\childdocmain\endinput\fi
%    \end{macrocode}
%\iffalse
%</discard>
%<*package>
%\fi
%
% \macro{\ifchilddoc}
% \macro{\ifchilddocmanual}
% The conditional |\ifchilddoc| tells whether a
% child (true) or main (false) document is being compiled.
% The conditional |\ifchilddocmanual| tells whether
% the |\includeonly| mechanism is used (false) or
% the selection of child files must be performed manually (true).
% The definitions initialise to false:
%    \begin{macrocode}
\newif\ifchilddoc
\newif\ifchilddocmanual
%    \end{macrocode}

% \macro{\childdocname}
% \macro{\childdocjob}
% The macro |\childdocname| stores the name of the main document
% to be compiled. The macro |\childdocjob| stores the name of
% the document on which the \LaTeX{} compiler was originally invoked.
% The content of |\jobname| cannot be compared
% to filenames specified in the source due to different catcodes.
% The following code rescans |\jobname|, stores the result
% in |\childdocname| and saves a copy in |\childdocjob|:
%    \begin{macrocode}
\edef\childdocname{\scantokens\expandafter{\jobname\noexpand}}
\let\childdocjob\childdocname
%    \end{macrocode}

% \macro{\childdocdisable}
% The macro |\childdocdisable| prevents the main file
% from being processed more than once.
% At this stage, the main document command |\childdocmain|
% is assumed to be called once again where it should do nothing.
% Any subsequent call to it should prevent
% a secondary processing of the main document
% It overwrites the forwarding commands
% |\childdocof| and |\childdocforward|
% with empty macros to prevent further inclusions of the main document:
%    \begin{macrocode}
\newcommand{\childdocdisable}
{
  \renewcommand{\childdocmain}[1]{\renewcommand{\childdocmain}[1]{\endinput}}
  \renewcommand{\childdocof}[1]{}
  \renewcommand{\childdocby}[2][]{}
  \renewcommand{\childdocforward}[2][]{}
  \renewcommand{\childdocdisable}{}
}
%    \end{macrocode}

% \macro{\childdocmain}
% The macro |\childdocmain| is to be called at the top of the main file
% with nothing or the main filename (without extension) as argument.
% First, it breaks loops.
% If the argument is not empty and does not match |\childdocname|
% (which is set by the first inclusion of |childdoc.def|),
% |\ifchilddoc| is set to true, |\includeonly| is applied to the child file
% and |\jobname| is set to the main file
% (for proper handling of |.aux| files):
%    \begin{macrocode}
\newcommand{\childdocmain}[1]
{
  \childdocdisable\childdocmain{}
  \if?#1?\else
    \begingroup
      \def\childdoctmp{#1}
      \ifx\childdoctmp\childdocname
        \def\childdoctmp{}
      \else
        \def\childdoctmp
        {
          \childdoctrue
          \includeonly{\childdocname}
          \def\childdocjob{#1}
          \def\jobname{#1}
        }
      \fi
      \expandafter
    \endgroup
    \childdoctmp
  \fi
}
%    \end{macrocode}

% \macro{\childdocof}
% The command |\childdocof| redirects
% compilation to the main file |#1|.
%    \begin{macrocode}
\newcommand{\childdocof}[1]
{
  \childdocdisable
  \childdoctrue
  \includeonly{\childdocname}
  \def\jobname{#1}
  \def\childdocjob{#1}
  \input{#1}
}
%    \end{macrocode}

% \macro{\childdocby}
% The command |\childdocby| ....
%    \begin{macrocode}
\newcommand{\childdocby}[2][]
{
  \childdocdisable
  \childdoctrue
  \childdocmanualtrue
  \if?#1?\else
    \def\jobname{#2}
  \fi
  \def\childdocjob{#2}
  \input{#2}
  \endinput
}
%    \end{macrocode}

% \macro{\childdocforward}
% The command |\childdocforward| redirects
% compilation to the main file or
% (if the optional argument is given) a child file.
% Parameters are set as if the main file
% or a child file starting with |\childdocof| was compiled.
% Then compilation is handed over to the main file:
%    \begin{macrocode}
\newcommand{\childdocforward}[2][]
{
  \begingroup
    \if?#1?
      \def\childdoctmp
      {
        \def\childdocname{#2}
        \def\childdocjob{#2}
        \def\jobname{#2}
        \input{#2}
        \endinput
      }
    \else
      \def\childdoctmp
      {
        \childdocdisable
        \def\childdocname{#2}
        \childdoctrue
        \includeonly{#2}
        \def\childdocjob{#1}
        \def\jobname{#1}
        \input{#1}
        \endinput
      }
    \fi
    \expandafter
  \endgroup
  \childdoctmp
}
%    \end{macrocode}

% \macro{\childdocforwardprefix}
% The command |\childdocforwardprefix| redirects
% compilation to the main or a child file by means of a pattern.
% The prefix |#1| in the current filename is replaced by |#2|
% and the suffix of the current filename is kept
% (it is assumed that the filename does not contain the substring `|~~~|'
% which is used as a delimiter).
% Compilation is handed over to the new file by |\childdocforward|:
%    \begin{macrocode}
\newcommand{\childdocforwardprefix}[3][]
{
  \begingroup
    \def\childdocextract #2##1~~~{\def\childdoctmp{\childdocforward[#1]{#3##1}}}
    \expandafter\childdocextract\childdocname~~~
    \expandafter
  \endgroup
  \childdoctmp
}
%    \end{macrocode}

% \macro{\childdoc}
% The deprecated macro |\childdoc| is a legacy version of |\childdocmain|:
%    \begin{macrocode}
\newcommand{\childdoc}{\childdocmain}
%    \end{macrocode}

% \macro{\childdocredirect}
% The deprecated macro |\childdocredirect| is a legacy version
% of |\childdocforward| and |\childdocforwardprefix|:
%    \begin{macrocode}
\newcommand{\childdocredirect}[2][]
{
  \begingroup
    \if?#1?
      \def\childdoctmp{\childdocforward{#2}}
    \else
      \def\childdoctmp{\childdocforwardprefix{#1}{#2}}
    \fi
    \expandafter
  \endgroup
  \childdoctmp
}
%    \end{macrocode}

%\iffalse
%</package>
%\fi
%
\endinput
|\\
|\childdocof{|\textit{main}|}|\\
\end{tabular}
\end{center}
at the top of every child file \textit{child}
which is included by |\include{|\textit{child}|}|
from within the main file
(or at least for those files to be compiled individually).
The argument \textit{main} must be the filename of the main file.

There are a couple of
considerations in setting up the main and child documents:

%%%%%%%%%%%%%%%%%%%%%%%%%%%%%%%%%%%%%%%%
\paragraph{Restrictions.}

Please note the following restrictions:
\begin{itemize}
\item
|\childdocmain| must be called with one argument \textit{main}
to ensure compatibility with earlier version of the package.
It must either be empty (|\childdocmain{}|)
or precisely match the filename of the main file in which it is specified.
See \secref{sec:detection} for further information.
\item
The filename \textit{main} must be specified without the |.tex| extension.
\item
The filename \textit{main} is case sensitive
(even in case-insensitive file systems)
due to internal string comparison.
\item
The argument \textit{main} should be fully expanded, it cannot be a macro.
\item
Subdirectories and special characters should be avoided in filenames.
\item
The command |\childdocmain{|\textit{main}|}| must be followed by a whitespace.
It should not be followed immediately by another command
or by a comment mark `|%|'.
This is because the \TeX{} parser reads the token immediately following
the argument of |\childdocmain| and puts it
at the beginning of every child section;
however, a white\-space is ignored.
\end{itemize}

%%%%%%%%%%%%%%%%%%%%%%%%%%%%%%%%%%%%%%%%
\paragraph{Content of Main File.}

It is advisable to place all content in the child files included by |\include|.
Any output contained in the main file will appear in all child documents
unless suppressed manually;
it cannot be suppressed automatically by the |\includeonly| directive
and thus should normally be avoided.
A method to include some content in the main file
by means of conditional processing is described in \secref{sec:conditional}.

%%%%%%%%%%%%%%%%%%%%%%%%%%%%%%%%%%%%%%%%
\paragraph{Page Numbering.}

When only a part of the document is compiled,
the appropriate numbering of pages
(as well as other status parameters)
is determined from the |.aux| files.
The latter contain information from previous passes.
However this information needs to propagate through
all intermediate child documents.
Therefore the page numbering in child documents may well
be inconsistent until the complete document is compiled at least once.

A useful (if unconventional) way to always ensure a consistent
page numbering is to restart the numbering in each child document
and denote the pages by `\textit{child}|.|\textit{page}'
where \textit{child} represents the chapter/section number of the child file.
This can be achieved by the command
|\numberwithin{page}{|\textit{child}|}|
of the \textsf{amsmath} package
where \textit{child} can be |chapter| or |section|
depending on the chosen structuring.
Alternatively, one can modify the macro |\thepage| appropriately
and reset the counter |page| at the start of each child file.

%%%%%%%%%%%%%%%%%%%%%%%%%%%%%%%%%%%%%%%%%%%%%%%%%%%%%%%%%%%%%%%%%%%%%%%%%%%%%%%%
\subsection{Conditional Processing}
\label{sec:conditional}

The package provides a mechanism to compile different versions
of a document. To customise the versions further some conditional processing
can come in handy to distinguish which version is being compiled.
The package provides two macros to describe the compilation context:

%%%%%%%%%%%%%%%%%%%%%%%%%%%%%%%%%%%%%%%%
\DescribeMacro{\ifchilddoc}
The conditional |\ifchilddoc| distinguishes between the compilation of
child documents and the main document:
%
\begin{center}
|\ifchilddoc |\textit{child-code}| |[|\||else |\textit{main-code}]| \||fi|
\end{center}

%%%%%%%%%%%%%%%%%%%%%%%%%%%%%%%%%%%%%%%%
\DescribeMacro{\childdocname}
\DescribeMacro{\childdocjob}
The macro |\childdocname| contains the filename (without extension)
of the main or child file being processed.
Note that |\childdocjob| will always contain the name of the main file.

%%%%%%%%%%%%%%%%%%%%%%%%%%%%%%%%%%%%%%%%
\paragraph{Title Page.}

Conditional processing can be used to include a title or banner page
in the main document when proper precautions are taken.
Importantly, the code in the main file should ensure that the page counter
(as well as other status parameters which are stored in the |.aux| files)
takes the same value after the conditional processing.
Otherwise the page numbers may take divergent values
depending on which part is compiled.

For example, a title page could be declared by:
%
\begin{center}
\begin{tabular}{l}
|\ifchilddoc\||else|\\
|\addtocounter{page}{-1}|\\
\textit{code for title page}\\
|\newpage|\\
|\||fi|
\end{tabular}
\end{center}
%
A banner page for the child documents can be generated by:
%
\begin{center}
\begin{tabular}{l}
|\ifchilddoc|\\
|\addtocounter{page}{-1}|\\
\textit{code for banner page}\\
|\newpage|\\
|\||fi|
\end{tabular}
\end{center}
%
Here one could write a message such as:
\begin{center}
|This is the part \childdocname{} of \childdocjob{}.|
\end{center}

%%%%%%%%%%%%%%%%%%%%%%%%%%%%%%%%%%%%%%%%%%%%%%%%%%%%%%%%%%%%%%%%%%%%%%%%%%%%%%%%
\subsection{Flags}
\label{sec:flags}

The package makes it easy to generate different versions
of the main or child documents.
To this end compilation flags can be defined
and assigned different default values.
They will be particularly useful in conjunction
with the forwarding mechanism described in \secref{sec:forward}.

For example, it may be useful to have a flag |\version|
which can be set to |draft| or |final|.
The document source will contain some conditional code
depending on the value of |\version|.
Suppose further, the flag should default to |final| for the main file
and to |draft| for child files
which is a natural assignment for editing the document.
This is achieved by placing the following code
in the preamble of the main document
(below the |\childdocmain| directive):
%
\begin{center}
\begin{tabular}{l}
|\ifchilddoc|\\
|\providecommand{\version}{draft}|\\
|\||else|\\
|\providecommand{\version}{final}|\\
|\||fi|
\end{tabular}
\end{center}
%
The definition by |\providecommand| makes sure
that previous definitions are not overwritten.
Further statements |\providecommand{\version}{...}|
can thus be added before the above code to override it.

For the main file, one might add a line
(between |\childdocmain| and the above block)
%
\begin{center}
|%\ifchilddoc\||else\providecommand{\version}{draft}\||fi|
\end{center}
%
which can be uncommented to produce a draft version.
Likewise one can add a line to the very top of a child file
(above the |\childdocof{|\textit{main}|}| directive)
%
\begin{center}
|%\providecommand{\version}{final}|
\end{center}
%
which can be uncommented to produce the final version of this child document.

%%%%%%%%%%%%%%%%%%%%%%%%%%%%%%%%%%%%%%%%%%%%%%%%%%%%%%%%%%%%%%%%%%%%%%%%%%%%%%%%
\subsection{Forwarding}
\label{sec:forward}

Different versions of the main or child documents
using compilation flags as described in \secref{sec:flags}
can be (permanently) stored in different files
for convenient compilation, viewing and distribution.
To this end, the package defines a command
to pass on compilation to a different file:

%%%%%%%%%%%%%%%%%%%%%%%%%%%%%%%%%%%%%%%%
\DescribeMacro{\childdocforward}
The command |\childdocforward| redirects processing to
another source file:
%
\begin{center}
\begin{tabular}{l}
|% \iffalse
%
% childdoc.dtx Copyright (C) 2017-2018 Niklas Beisert
%
% This work may be distributed and/or modified under the
% conditions of the LaTeX Project Public License, either version 1.3
% of this license or (at your option) any later version.
% The latest version of this license is in
%   http://www.latex-project.org/lppl.txt
% and version 1.3 or later is part of all distributions of LaTeX
% version 2005/12/01 or later.
%
% This work has the LPPL maintenance status `maintained'.
%
% The Current Maintainer of this work is Niklas Beisert.
%
% This work consists of the files childdoc.dtx and childdoc.ins
% and the derived files childdoc.def and cdocsamp.tex with
% cdocsch1.tex, cdocsch2.tex, cdocsdrf.tex, cdocsfn1.tex, cdocsfn2.tex.
%
%<package>\ifdefined\childdocmain\endinput\fi
%<package>\ProvidesFile{childdoc.def}[2018/12/30 v2.0 child document driver]
%<samplemain>\ProvidesFile{cdocsamp.tex}[2018/12/30 v2.0 sample for childdoc]
%<*driver>
%\ProvidesFile{childdoc.drv}[2018/12/30 v2.0 childdoc reference manual file]
\PassOptionsToClass{10pt,a4paper}{article}
\documentclass{ltxdoc}

\usepackage[margin=35mm]{geometry}
\usepackage{hyperref}
\usepackage{hyperxmp}
\usepackage[usenames]{color}

\hypersetup{colorlinks=true}
\hypersetup{pdfstartview=FitH}
\hypersetup{pdfpagemode=UseNone}
\hypersetup{pdfsource={}}
\hypersetup{pdflang={en-UK}}
\hypersetup{pdfcopyright={Copyright 2017-2018 Niklas Beisert.
  This work may be distributed and/or modified under the
  conditions of the LaTeX Project Public License, either version 1.3
  of this license or (at your option) any later version.}}
\hypersetup{pdflicenseurl={http://www.latex-project.org/lppl.txt}}
\hypersetup{pdfcontactaddress={ETH Zurich, ITP, HIT K,
  Wolfgang-Pauli-Strasse 27}}
\hypersetup{pdfcontactpostcode={8093}}
\hypersetup{pdfcontactcity={Zurich}}
\hypersetup{pdfcontactcountry={Switzerland}}
\hypersetup{pdfcontactemail={nbeisert@itp.phys.ethz.ch}}
\hypersetup{pdfcontacturl={http://people.phys.ethz.ch/\xmptilde nbeisert/}}

\newcommand{\secref}[1]{\hyperref[#1]{section \ref*{#1}}}

\parskip1ex
\parindent0pt
\let\olditemize\itemize
\def\itemize{\olditemize\parskip0pt}

\begin{document}

\title{The \textsf{childdoc} Package}
\hypersetup{pdftitle={The childdoc Package}}
\author{Niklas Beisert\\[2ex]
  Institut f\"ur Theoretische Physik\\
  Eidgen\"ossische Technische Hochschule Z\"urich\\
  Wolfgang-Pauli-Strasse 27, 8093 Z\"urich, Switzerland\\[1ex]
  \href{mailto:nbeisert@itp.phys.ethz.ch}
  {\texttt{nbeisert@itp.phys.ethz.ch}}}
\hypersetup{pdfauthor={Niklas Beisert}}
\hypersetup{pdfsubject={Manual for the LaTeX2e Package childdoc}}
\date{30 December 2018, \textsf{v2.0}}
\maketitle

\begin{abstract}\noindent
\textsf{childdoc} is a \LaTeXe{} package
that enables the direct compilation
of document sections included by |\include|
to individual files.
\end{abstract}

\begingroup
\parskip0ex
\tableofcontents
\endgroup

%%%%%%%%%%%%%%%%%%%%%%%%%%%%%%%%%%%%%%%%%%%%%%%%%%%%%%%%%%%%%%%%%%%%%%%%%%%%%%%%
%%%%%%%%%%%%%%%%%%%%%%%%%%%%%%%%%%%%%%%%%%%%%%%%%%%%%%%%%%%%%%%%%%%%%%%%%%%%%%%%
\section{Introduction}

\LaTeX{} provides a mechanism to structure a large document (such as a book)
into a main file and several child files (containing the chapters)
using the |\include| command.
This mechanism is beneficial for documents
which span hundreds of pages in order to
make the source file(s) more manageable.
Moreover, compilation can be restricted to
selected child files by means of the |\includeonly| command.
The latter feature can be used to reduce the compilation time while editing
(this was significantly more useful in the earlier days of \LaTeX{})
or to generate a smaller document which is easier to navigate.
Another application of |\includeonly| is to generate
documents consisting of selected parts of the complete document.

However, there are a few drawbacks of the plain |\include| mechanism:
\begin{itemize}
\item
The child files cannot be compiled on their own,
they can only be compiled via the main file.
A naive editing environment
(such as a text editor with an option
to have the current file processed by \LaTeX)
may require one to switch to the main file before compiling;
attempting to compile the child file produces errors.
\item
The main file must be modified (each time)
to adjust the |\includeonly| command
to the present needs. This easily leaves the main file in a messy state.
\item
The generated document will always carry the filename
of the main document. This is inconvenient if
several child files are to be compiled and
to be kept for distribution.
\end{itemize}

The present package provides a simple interface
to make child files individually compilable by \LaTeX{}.
Compiling a child file then has the same effect as compiling
the main file with an |\includeonly| command
to select the appropriate child.
Moreover the generated document will carry the name of the child
rather than the main file.
This resolves all three above issues.

This feature is meant to make the editing of books,
thesis documents and lecture notes somewhat more convenient.
However, the package can also be used efficiently for
composing a series of documents (such as exercise sheets)
which are typically distributed individually.
It then assists the author in generating the individual documents
(potentially in different versions)
as well as a document containing the collected series.
Another application is in developing style files
or other kinds of included material
where compilation of the style file could redirect
to a sample or test file.

%%%%%%%%%%%%%%%%%%%%%%%%%%%%%%%%%%%%%%%%%%%%%%%%%%%%%%%%%%%%%%%%%%%%%%%%%%%%%%%%
%%%%%%%%%%%%%%%%%%%%%%%%%%%%%%%%%%%%%%%%%%%%%%%%%%%%%%%%%%%%%%%%%%%%%%%%%%%%%%%%
\section{Usage}

First of all, the package \textsf{childdoc} is \emph{not} a standard
\LaTeXe{} |.sty| style file! Therefore it needs to be invoked in
a non-standard way.

%%%%%%%%%%%%%%%%%%%%%%%%%%%%%%%%%%%%%%%%%%%%%%%%%%%%%%%%%%%%%%%%%%%%%%%%%%%%%%%%
\subsection{Included Files}
\label{sec:include}

%%%%%%%%%%%%%%%%%%%%%%%%%%%%%%%%%%%%%%%%
\DescribeMacro{\childdocmain}
To use the package, add the commands
\begin{center}
\begin{tabular}{l}
|\input{childdoc.def}|\\
|\childdocmain{}|\\
\end{tabular}
\end{center}
at the very top of the main \LaTeX{} file,
in particular \emph{before} the |\documentclass| statement!
The argument of |\childdocmain| should be left empty
(but it must be present).

%%%%%%%%%%%%%%%%%%%%%%%%%%%%%%%%%%%%%%%%
\DescribeMacro{\childdocof}
Furthermore, add the commands
\begin{center}
\begin{tabular}{l}
|\input{childdoc.def}|\\
|\childdocof{|\textit{main}|}|\\
\end{tabular}
\end{center}
at the top of every child file \textit{child}
which is included by |\include{|\textit{child}|}|
from within the main file
(or at least for those files to be compiled individually).
The argument \textit{main} must be the filename of the main file.

There are a couple of
considerations in setting up the main and child documents:

%%%%%%%%%%%%%%%%%%%%%%%%%%%%%%%%%%%%%%%%
\paragraph{Restrictions.}

Please note the following restrictions:
\begin{itemize}
\item
|\childdocmain| must be called with one argument \textit{main}
to ensure compatibility with earlier version of the package.
It must either be empty (|\childdocmain{}|)
or precisely match the filename of the main file in which it is specified.
See \secref{sec:detection} for further information.
\item
The filename \textit{main} must be specified without the |.tex| extension.
\item
The filename \textit{main} is case sensitive
(even in case-insensitive file systems)
due to internal string comparison.
\item
The argument \textit{main} should be fully expanded, it cannot be a macro.
\item
Subdirectories and special characters should be avoided in filenames.
\item
The command |\childdocmain{|\textit{main}|}| must be followed by a whitespace.
It should not be followed immediately by another command
or by a comment mark `|%|'.
This is because the \TeX{} parser reads the token immediately following
the argument of |\childdocmain| and puts it
at the beginning of every child section;
however, a white\-space is ignored.
\end{itemize}

%%%%%%%%%%%%%%%%%%%%%%%%%%%%%%%%%%%%%%%%
\paragraph{Content of Main File.}

It is advisable to place all content in the child files included by |\include|.
Any output contained in the main file will appear in all child documents
unless suppressed manually;
it cannot be suppressed automatically by the |\includeonly| directive
and thus should normally be avoided.
A method to include some content in the main file
by means of conditional processing is described in \secref{sec:conditional}.

%%%%%%%%%%%%%%%%%%%%%%%%%%%%%%%%%%%%%%%%
\paragraph{Page Numbering.}

When only a part of the document is compiled,
the appropriate numbering of pages
(as well as other status parameters)
is determined from the |.aux| files.
The latter contain information from previous passes.
However this information needs to propagate through
all intermediate child documents.
Therefore the page numbering in child documents may well
be inconsistent until the complete document is compiled at least once.

A useful (if unconventional) way to always ensure a consistent
page numbering is to restart the numbering in each child document
and denote the pages by `\textit{child}|.|\textit{page}'
where \textit{child} represents the chapter/section number of the child file.
This can be achieved by the command
|\numberwithin{page}{|\textit{child}|}|
of the \textsf{amsmath} package
where \textit{child} can be |chapter| or |section|
depending on the chosen structuring.
Alternatively, one can modify the macro |\thepage| appropriately
and reset the counter |page| at the start of each child file.

%%%%%%%%%%%%%%%%%%%%%%%%%%%%%%%%%%%%%%%%%%%%%%%%%%%%%%%%%%%%%%%%%%%%%%%%%%%%%%%%
\subsection{Conditional Processing}
\label{sec:conditional}

The package provides a mechanism to compile different versions
of a document. To customise the versions further some conditional processing
can come in handy to distinguish which version is being compiled.
The package provides two macros to describe the compilation context:

%%%%%%%%%%%%%%%%%%%%%%%%%%%%%%%%%%%%%%%%
\DescribeMacro{\ifchilddoc}
The conditional |\ifchilddoc| distinguishes between the compilation of
child documents and the main document:
%
\begin{center}
|\ifchilddoc |\textit{child-code}| |[|\||else |\textit{main-code}]| \||fi|
\end{center}

%%%%%%%%%%%%%%%%%%%%%%%%%%%%%%%%%%%%%%%%
\DescribeMacro{\childdocname}
\DescribeMacro{\childdocjob}
The macro |\childdocname| contains the filename (without extension)
of the main or child file being processed.
Note that |\childdocjob| will always contain the name of the main file.

%%%%%%%%%%%%%%%%%%%%%%%%%%%%%%%%%%%%%%%%
\paragraph{Title Page.}

Conditional processing can be used to include a title or banner page
in the main document when proper precautions are taken.
Importantly, the code in the main file should ensure that the page counter
(as well as other status parameters which are stored in the |.aux| files)
takes the same value after the conditional processing.
Otherwise the page numbers may take divergent values
depending on which part is compiled.

For example, a title page could be declared by:
%
\begin{center}
\begin{tabular}{l}
|\ifchilddoc\||else|\\
|\addtocounter{page}{-1}|\\
\textit{code for title page}\\
|\newpage|\\
|\||fi|
\end{tabular}
\end{center}
%
A banner page for the child documents can be generated by:
%
\begin{center}
\begin{tabular}{l}
|\ifchilddoc|\\
|\addtocounter{page}{-1}|\\
\textit{code for banner page}\\
|\newpage|\\
|\||fi|
\end{tabular}
\end{center}
%
Here one could write a message such as:
\begin{center}
|This is the part \childdocname{} of \childdocjob{}.|
\end{center}

%%%%%%%%%%%%%%%%%%%%%%%%%%%%%%%%%%%%%%%%%%%%%%%%%%%%%%%%%%%%%%%%%%%%%%%%%%%%%%%%
\subsection{Flags}
\label{sec:flags}

The package makes it easy to generate different versions
of the main or child documents.
To this end compilation flags can be defined
and assigned different default values.
They will be particularly useful in conjunction
with the forwarding mechanism described in \secref{sec:forward}.

For example, it may be useful to have a flag |\version|
which can be set to |draft| or |final|.
The document source will contain some conditional code
depending on the value of |\version|.
Suppose further, the flag should default to |final| for the main file
and to |draft| for child files
which is a natural assignment for editing the document.
This is achieved by placing the following code
in the preamble of the main document
(below the |\childdocmain| directive):
%
\begin{center}
\begin{tabular}{l}
|\ifchilddoc|\\
|\providecommand{\version}{draft}|\\
|\||else|\\
|\providecommand{\version}{final}|\\
|\||fi|
\end{tabular}
\end{center}
%
The definition by |\providecommand| makes sure
that previous definitions are not overwritten.
Further statements |\providecommand{\version}{...}|
can thus be added before the above code to override it.

For the main file, one might add a line
(between |\childdocmain| and the above block)
%
\begin{center}
|%\ifchilddoc\||else\providecommand{\version}{draft}\||fi|
\end{center}
%
which can be uncommented to produce a draft version.
Likewise one can add a line to the very top of a child file
(above the |\childdocof{|\textit{main}|}| directive)
%
\begin{center}
|%\providecommand{\version}{final}|
\end{center}
%
which can be uncommented to produce the final version of this child document.

%%%%%%%%%%%%%%%%%%%%%%%%%%%%%%%%%%%%%%%%%%%%%%%%%%%%%%%%%%%%%%%%%%%%%%%%%%%%%%%%
\subsection{Forwarding}
\label{sec:forward}

Different versions of the main or child documents
using compilation flags as described in \secref{sec:flags}
can be (permanently) stored in different files
for convenient compilation, viewing and distribution.
To this end, the package defines a command
to pass on compilation to a different file:

%%%%%%%%%%%%%%%%%%%%%%%%%%%%%%%%%%%%%%%%
\DescribeMacro{\childdocforward}
The command |\childdocforward| redirects processing to
another source file:
%
\begin{center}
\begin{tabular}{l}
|\input{childdoc.def}|\\
|\childdocforward[|\textit{main}|]{|\textit{dest}|}|\\
\end{tabular}
\end{center}
%
The argument \textit{dest} is the destination file
(without extension).
It should be the main file or one of the child files.
Note that further \textsf{childdoc} directives
such as |\childdocof| and |\childdocforward|
in the indicated file will be processed in this form.
The optional argument \textit{main}
passes on directly to the main file \textit{main}
while pretending to compile the child \textit{dest}.
This form behaves as if \textit{dest}
issues |\childdocof{|\textit{main}|}| right away,
and no further \textsf{childdoc} directives will be processed.

%%%%%%%%%%%%%%%%%%%%%%%%%%%%%%%%%%%%%%%%
\DescribeMacro{\...prefix}
In the alternative form |\childdocforwardprefix|,
%
\begin{center}
\begin{tabular}{l}
|\input{childdoc.def}|\\
|\childdocforwardprefix[|\textit{main}|]{|\textit{prefix}|}{|\textit{dest}|}|
\end{tabular}
\end{center}
%
the destination file is determined by a pattern
depending on the current file:
To make this work, the current file must be called
`{\textit{prefix}\hspace{0.2em}\textit{suffix}}'
with \textit{prefix} matching precisely the argument.
Processing is then passed on to the file
`{\textit{dest}\hspace{0.2em}\textit{suffix}}'.
Surely, the same effect is achieved by
directly specifying the
argument `{\textit{dest}\hspace{0.2em}\textit{suffix}}'
in the first form.
However, that requires to set up a different file
for each child. With the alternative form of the command
all these files can have exactly the same content
which simplifies setting them up and maintaining them.

For example, the following file |draft.tex|
with a compilation flag |\version| as described in \secref{sec:flags}
compiles the main document as a draft:
%
\begin{center}
\begin{tabular}{l}
|\def\version{draft}|\\
|\input{childdoc.def}|\\
|\childdocforward{|\textit{main}|}|
\end{tabular}
\end{center}
%
Likewise, the following files |final|\textit{nn}|.tex|
compile the final version of the child document
|child|\textit{nn}|.tex|:
%
\begin{center}
\begin{tabular}{l}
|\def\version{final}|\\
|\input{childdoc.def}|\\
|\childdocforwardprefix{final}{child}|
\end{tabular}
\end{center}
%

Note that when several versions of a main file and/or of each child file
are to be generated, it may be convenient to set up a |Makefile| or
shell script to automatise the process.

%%%%%%%%%%%%%%%%%%%%%%%%%%%%%%%%%%%%%%%%%%%%%%%%%%%%%%%%%%%%%%%%%%%%%%%%%%%%%%%%
\subsection{Command Line Processing}
\label{sec:commandline}

The effect of redirection files can also be achieved by invoking
the \LaTeX{} compiler with a more elaborate command line.
Most conveniently this should be done as part
of a shell script or a |Makefile|.

When using \textsf{childdoc} in the main file, the following
command lines effectively perform a redirection
(note that depending on the shell being used,
backslashes may have to be doubled: `|\|' $\to$ `|\\|'):
%
\begin{center}
|... -jobname "|\textit{target}|" |\\|"|[\textit{flags}]%
|\input{childdoc.def}\childdocforward[|\textit{main}|]{|\textit{dest}|}"|
\end{center}
%
Here \textit{target} is the name of the output file,
\textit{main} is the name of the main file
and \textit{dest} is the name of the main or child file to be processed
(all filenames without extensions).
The optional argument \textit{main} can be omitted
if \textit{main} matches \textit{dest}.
Optionally, compilation \textit{flags} can be defined via |\def| commands.
This command line makes the \TeX{} engine believe
it is compiling the file \textit{target}
whose content is specified as the latter parameter.
The provided code then forwards the processing to
\textit{main} or \textit{dest} as described in \secref{sec:forward}.

%%%%%%%%%%%%%%%%%%%%%%%%%%%%%%%%%%%%%%%%%%%%%%%%%%%%%%%%%%%%%%%%%%%%%%%%%%%%%%%%
\subsection{Include by Input}
\label{sec:input}

Including child documents by |\include| has some restrictions by design.
Most notably, the content of a child document always occupies
its own set of pages; pages cannot be shared between child documents.
Usually, this behaviour makes perfect sense
because each child document contain an essential part of the document.
However, in some situations it may be desirable to compose
a document from a collection of parts
without having mandatory page breaks between then.
For this case, the package
provides a mechanism to include parts
by |\input| which can also be processed individually.
However, by construction this mechanism
requires manual handling of the content to be output.

%%%%%%%%%%%%%%%%%%%%%%%%%%%%%%%%%%%%%%%%
\DescribeMacro{\ifchilddocmanual}
The main file should be prepared as usual, see \secref{sec:include}.
However, the document body must make a distinction
between processing of an individual part and of the main document, e.g.:
%
\begin{center}
\begin{tabular}{l}
|\ifchilddocmanual|\\
|\input{\childdocname}|\\
|\||else|\\
\textit{document body with }|\input{|\textit{part}|}|\\
|\||fi|
\end{tabular}
\end{center}
%
The conditional |\ifchilddocmanual| is true whenever
a part to be included by |\input| is being compiled,
and the name of the part is stored in |\childdocname|.

%%%%%%%%%%%%%%%%%%%%%%%%%%%%%%%%%%%%%%%%
\DescribeMacro{\childdocby}
Each part to be included by |\input| should start with:
%
\begin{center}
\begin{tabular}{l}
|\input{childdoc.def}|\\
|\childdocby{|\textit{main}|}|\\
\end{tabular}
\end{center}
%
The directive |\childdocby| is similar to |\childdocof|
described in \secref{sec:include},
but the subsequent selection of content must be done manually.
To that end, both |\ifchilddoc| and |\ifchilddocmanual|
will be true upon processing of a part,
and the name of the part is stored in |\childdocname|.
Note that |\jobname| will be set to the filename of the current part
so that each part receives an individual |.aux| file
that does not interfere with the |.aux| file(s) of the main document.
This behaviour can be altered by the alternative form
|\childdocby[*]{|\textit{main}|}| (with a non-empty optional argument)
which uses the |.aux| file of the main document
by setting |\jobname| to \textit{main}.

%%%%%%%%%%%%%%%%%%%%%%%%%%%%%%%%%%%%%%%%%%%%%%%%%%%%%%%%%%%%%%%%%%%%%%%%%%%%%%%%
\subsection{Driver Development}
\label{sec:driver}

The \textsf{childdoc} mechanism can also be use for the development
of definition files such as \LaTeX{} styles or classes.
This case differs from the above setup with multiple parts
included by |\include| in that no |\includeonly| should be invoked.
This can be achieved by starting the include file
(before |\ProvidesPackage|) with:
%
\begin{center}
\begin{tabular}{l}
|\input{childdoc.def}|\\
|\childdocforward{|\textit{main}|}|\\
\end{tabular}
\end{center}
%
or alternatively with:
%
\begin{center}
\begin{tabular}{l}
|\input{childdoc.def}|\\
|\childdocby{|\textit{main}|}|\\
\end{tabular}
\end{center}
%
Both forms have slightly different effects as described above.
The main file is prepared as usual, see \secref{sec:include}.

%%%%%%%%%%%%%%%%%%%%%%%%%%%%%%%%%%%%%%%%%%%%%%%%%%%%%%%%%%%%%%%%%%%%%%%%%%%%%%%%
\subsection{Legacy Detection}
\label{sec:detection}

The directive |\childdocmain| in the main file can detect
whether the complete document or merely a child is to be compiled
even without using the directive |\childdocof|.
This method is deprecated because it is less robust
and there is no compelling reason to use it;
it is merely provided for backward compatibility
and it may be removed in future versions.

If the detection mechanism is to be used,
it is mandatory to correctly specify
the filename of the main file as the argument of |\childdocmain|:
%
\begin{center}
\begin{tabular}{l}
|\input{childdoc.def}|\\
|\childdocmain{|\textit{main}|}|\\
\end{tabular}
\end{center}
%
If |\jobname| does not match the argument \textit{main} of |\childdocmain|,
it is assumed that |\jobname| points to the child file to be compiled.
When using |\childdocmain| with the main file specified as argument,
it suffices to start a child file
with just |\input{|\textit{main}|}|
without loading of the package and using |\childdocof|.
If instead all processing is done
with the appropriate \textsf{childdoc} directives,
the argument of \textit{main} of |\childdocmain| can be empty.

An alternative version of the command line processing described
in \secref{sec:commandline} using the detection mechanism reads:
%
\begin{center}
|... -jobname "|\textit{target}|" "|[\textit{flags}]%
[|\def\jobname{|\textit{dest}|}|]|\input{|\textit{main}|}"|
\end{center}

%%%%%%%%%%%%%%%%%%%%%%%%%%%%%%%%%%%%%%%%%%%%%%%%%%%%%%%%%%%%%%%%%%%%%%%%%%%%%%%%
\subsection{Manual Code}
\label{sec:manual}

In case one cannot be certain whether the definitions file |childdoc.def|
is installed on the target \TeX{} distribution
and one prefers not to ship it,
it is conceivable to paste a few relevant commands into the sources.

To that end, drop all statements |\input{childdoc.def}|
and perform the replacements as outlined below.
Instead of |\childdocmain{|\textit{main}|}| add the following code
to the top of the main file:
%
\begin{center}
\begin{tabular}{l}
|\||ifdefined\childdocname\endinput\||fi\newif\ifchilddoc|\\
|\edef\childdocname{\scantokens\expandafter{\jobname\noexpand}}|\\
|\def\childdocmain{|\textit{main}|}\||ifx\childdocmain\childdocname\||else|\\
|\childdoctrue\includeonly{\childdocname}\let\jobname\childdocmain\||fi|\\
\end{tabular}
\end{center}
%
Instead of |\childdocof{|\textit{main}|}| just include the main file
at the top of each child file:
%
\begin{center}
|\input{|\textit{main}|}|
\end{center}
%
A simple redirection |\childdocforward{|\textit{dest}|}| is achieved by:
%
\begin{center}
|\def\jobname{|\textit{dest}|}\input{\jobname}|
\end{center}
%
The redirection with prefix
|\childdocforwardprefix[|\textit{prefix}|]{|\textit{dest}|}|
is accomplished by:
%
\begin{center}
\begin{tabular}{l}
|{\edef\jobname{\scantokens\expandafter{\jobname\noexpand}}|\\
|\def\redirectjob |\textit{prefix}|#1~~~{\gdef\jobname{|\textit{dest}|#1}}|\\
|\expandafter\redirectjob\jobname~~~}\input{\jobname}|
\end{tabular}
\end{center}

In an alternative approach,
child documents can be compiled by a specific command line
without additional code or specific definitions:
%
\begin{center}
|... -jobname "|\textit{target}|" "|[\textit{flags}]%
|\includeonly{|\textit{dest}|}\input{|\textit{main}|}"|
\end{center}
%

%%%%%%%%%%%%%%%%%%%%%%%%%%%%%%%%%%%%%%%%%%%%%%%%%%%%%%%%%%%%%%%%%%%%%%%%%%%%%%%%
%%%%%%%%%%%%%%%%%%%%%%%%%%%%%%%%%%%%%%%%%%%%%%%%%%%%%%%%%%%%%%%%%%%%%%%%%%%%%%%%
\section{Information}

%%%%%%%%%%%%%%%%%%%%%%%%%%%%%%%%%%%%%%%%%%%%%%%%%%%%%%%%%%%%%%%%%%%%%%%%%%%%%%%%
\subsection{Copyright}

Copyright \copyright{} 2017--2018 Niklas Beisert

This work may be distributed and/or modified under the
conditions of the \LaTeX{} Project Public License, either version 1.3
of this license or (at your option) any later version.
The latest version of this license is in
  \url{http://www.latex-project.org/lppl.txt}
and version 1.3 or later is part of all distributions of \LaTeX{}
version 2005/12/01 or later.

This work has the LPPL maintenance status `maintained'.

The Current Maintainer of this work is Niklas Beisert.

This work consists of the files |README.txt|, |childdoc.ins| and |childdoc.dtx|
as well as the derived files |childdoc.def|, |cdocsamp.tex|
with |cdocsch1.tex|, |cdocsch2.tex|, |cdocspt3.tex|, |cdocspt4.tex|,
|cdocsdrf.tex|, |cdocsfn1.tex|, |cdocsfn2.tex|
as well as |childdoc.pdf|.

%%%%%%%%%%%%%%%%%%%%%%%%%%%%%%%%%%%%%%%%%%%%%%%%%%%%%%%%%%%%%%%%%%%%%%%%%%%%%%%%
\subsection{Files and Installation}

The package consists of the files:
%
\begin{center}
\begin{tabular}{ll}
    |README.txt|   & readme file \\
    |childdoc.ins| & installation file \\
    |childdoc.dtx| & source file \\
    |childdoc.def| & definition file \\
    |cdocsamp.tex| & sample main file \\
    |cdocsch1.tex| & sample include file \\
    |cdocsch2.tex| & sample include file \\
    |cdocspt3.tex| & sample part file \\
    |cdocspt4.tex| & sample part file \\
    |cdocsdrf.tex| & sample redirection file \\
    |cdocsfn1.tex| & sample redirection file \\
    |cdocsfn2.tex| & sample redirection file \\
    |childdoc.pdf| & manual
\end{tabular}
\end{center}
%
The distribution consists of the files
|README.txt|, |childdoc.ins| and |childdoc.dtx|.
%
\begin{itemize}
\item
Run (pdf)\LaTeX{} on |childdoc.dtx|
to compile the manual |childdoc.pdf| (this file).
\item
Run \LaTeX{} on |childdoc.ins| to create the definitions file |childdoc.def|
and the sample |cdocsamp.tex| with include files
|cdocsch1.tex|, |cdocsch2.tex|, |cdocspt3.tex|, |cdocspt4.tex|,
|cdocsdrf.tex|, |cdocsfn1.tex|, |cdocsfn2.tex|.
Then copy the file |childdoc.def| to an appropriate directory of your \LaTeX{}
distribution, e.g.\ \textit{texmf-root}|/tex/latex/childdoc|.
\end{itemize}

%%%%%%%%%%%%%%%%%%%%%%%%%%%%%%%%%%%%%%%%%%%%%%%%%%%%%%%%%%%%%%%%%%%%%%%%%%%%%%%%
\subsection{Related CTAN Packages}

There are several other packages which offer a similar functionality:
%
\begin{itemize}
\item
The packages
\href{http://ctan.org/pkg/docmute}{\textsf{docmute}},
\href{http://ctan.org/pkg/includex}{\textsf{includex}} and
\href{http://ctan.org/pkg/standalone}{\textsf{standalone}}
provide commands to include only the document body of
a child file thus allowing both files to be compiled individually.
\item
The packages \href{http://ctan.org/pkg/subdocs}{\textsf{subdocs}}
and \href{http://ctan.org/pkg/subfiles}{\textsf{subfiles}}
provide structures in which the main and child documents can be
encapsulated and allowing them to be compiled individually.
The inclusion mechanism is different from the conventional |\include|.
\item
The package \href{http://ctan.org/pkg/combine}{\textsf{combine}}
is an elaborate solution to combine several documents into one.
\end{itemize}
%
See also the CTAN topic \href{http://ctan.org/topic/subdocs}{\textsf{subdocs}}
for further related packages.
The present package differs from the above solutions in that
a document structure constructed with the conventional |\include| mechanism
just needs two extra commands at the top of every file
such that all constituent files can be compiled individually.

%%%%%%%%%%%%%%%%%%%%%%%%%%%%%%%%%%%%%%%%%%%%%%%%%%%%%%%%%%%%%%%%%%%%%%%%%%%%%%%%
%\subsection{Feature Suggestions}
%
%The following is a list of features which may be useful for future
%versions of this package:
%%
%\begin{itemize}
%\item
%\ldots
%\end{itemize}

%%%%%%%%%%%%%%%%%%%%%%%%%%%%%%%%%%%%%%%%%%%%%%%%%%%%%%%%%%%%%%%%%%%%%%%%%%%%%%%%
\subsection{Revision History}

%%%%%%%%%%%%%%%%%%%%%%%%%%%%%%%%%%%%%%%%
\paragraph{v2.0:} 2018/12/30

\begin{itemize}
\item
immediate forward processing
\item
added |\childdocby| mechanism
\item
manual restructured
\end{itemize}

%%%%%%%%%%%%%%%%%%%%%%%%%%%%%%%%%%%%%%%%
\paragraph{v1.6:} 2018/01/17

\begin{itemize}
\item
application for development of include files
\item
corrections to manual
\end{itemize}

%%%%%%%%%%%%%%%%%%%%%%%%%%%%%%%%%%%%%%%%
\paragraph{v1.5:} 2017/05/21

\begin{itemize}
\item
more complete structuring introduced
\item
|\childdocof| introduced
\item
|\childdoc| renamed to |\childdocmain|
\item
|\childredirect| renamed to |\childdocforward| and |\childdocforwardprefix|
and functionality expanded
\end{itemize}

%%%%%%%%%%%%%%%%%%%%%%%%%%%%%%%%%%%%%%%%
\paragraph{v1.0:} 2017/04/27

\begin{itemize}
\item
manual and install package
\item
first version published on CTAN
\end{itemize}

%%%%%%%%%%%%%%%%%%%%%%%%%%%%%%%%%%%%%%%%
\paragraph{v0.6:} 2017/04/26

\begin{itemize}
\item
redirection mechanism added
\end{itemize}

%%%%%%%%%%%%%%%%%%%%%%%%%%%%%%%%%%%%%%%%
\paragraph{v0.5:} 2017/04/26

\begin{itemize}
\item
functionality in definition file
\end{itemize}


%%%%%%%%%%%%%%%%%%%%%%%%%%%%%%%%%%%%%%%%%%%%%%%%%%%%%%%%%%%%%%%%%%%%%%%%%%%%%%%%
%%%%%%%%%%%%%%%%%%%%%%%%%%%%%%%%%%%%%%%%%%%%%%%%%%%%%%%%%%%%%%%%%%%%%%%%%%%%%%%%
%%%%%%%%%%%%%%%%%%%%%%%%%%%%%%%%%%%%%%%%%%%%%%%%%%%%%%%%%%%%%%%%%%%%%%%%%%%%%%%%
\appendix

\settowidth\MacroIndent{\rmfamily\scriptsize 000\ }

 \DocInput{childdoc.dtx}

\end{document}
%</driver>
% \fi
%
% %%%%%%%%%%%%%%%%%%%%%%%%%%%%%%%%%%%%%%%%%%%%%%%%%%%%%%%%%%%%%%%%%%%%%%%%%%%%%%
% %%%%%%%%%%%%%%%%%%%%%%%%%%%%%%%%%%%%%%%%%%%%%%%%%%%%%%%%%%%%%%%%%%%%%%%%%%%%%%
% \section{Sample}
%\iffalse
%<*samplemain>
%\fi
%
% The following presents a sample document
% with two chapters, two parts, a title page,
% a compile flag as well as three forwarding files to set the flag.
% It consists of eight |.tex| files:
% \begin{center}
% \begin{tabular}{ll}
% |cdocsamp.tex|&main file\\
% |cdocsch1.tex|&include file for chapter 1\\
% |cdocsch2.tex|&include file for chapter 2\\
% |cdocspt3.tex|&include file for part 3\\
% |cdocspt4.tex|&include file for part 4\\
% |cdocsdrf.tex|&forwarding file for main file in draft mode\\
% |cdocsfi1.tex|&forwarding file for final version of chapter 1\\
% |cdocsfi2.tex|&forwarding file for final version of chapter 2\\
% \end{tabular}
% \end{center}
% Each of the eight files can be compiled directly by the \LaTeX{} compiler.
%
% %%%%%%%%%%%%%%%%%%%%%%%%%%%%%%%%%%%%%%
% \paragraph{Main File.}
%
% The main file is called |cdocsamp.tex|.
%
% Load the \textsf{childdoc} definitions and
% declare the filename for the main document:
%    \begin{macrocode}
\input{childdoc.def}
\childdocmain{}
%    \end{macrocode}

% Optional override for |\version| flag:
%    \begin{macrocode}
%%\ifchilddoc\else\providecommand{\version}{draft}\fi
%    \end{macrocode}

% Define the default values for the |\version| flag
% (|final| for the main file and |draft| for childs):
%    \begin{macrocode}
\ifchilddoc
\providecommand{\version}{draft}
\else
\providecommand{\version}{final}
\fi
%    \end{macrocode}

% Load the standard document class:
%    \begin{macrocode}
\documentclass[12pt]{article}
%    \end{macrocode}

% Start the document body:
%    \begin{macrocode}
\begin{document}
%    \end{macrocode}

% Declare a title page.
% Print title, part of document being processed and version flag:
%    \begin{macrocode}
\addtocounter{page}{-1}
\begin{center}
{\LARGE\bfseries{}childdoc example\par}
\vspace{1cm}
\ifchilddoc
\ifchilddocmanual part\else chapter\fi:
`\childdocname' of `\childdocjob'\par
\else
main document: `\childdocjob'\par
\fi
version: \version\par
\end{center}
\newpage
%    \end{macrocode}

% Manually include selected file,
% otherwise process as usual:
%    \begin{macrocode}
\ifchilddocmanual
\section*{part `\childdocname'}
\input{\childdocname}
\else
%    \end{macrocode}

% Include the two chapters:
%    \begin{macrocode}
\include{cdocsch1}
\include{cdocsch2}
%    \end{macrocode}

% Include the two parts unless only chapters should be displayed:
%    \begin{macrocode}
\ifchilddoc\else
\section{part three}
\input{cdocspt3}
\section{part four}
\input{cdocspt4}
\fi
%    \end{macrocode}

% Process as usual until here:
%    \begin{macrocode}
\fi
%    \end{macrocode}

% End of document body:
%    \begin{macrocode}
\end{document}
%    \end{macrocode}
%\iffalse
%</samplemain>
%\fi
%
% %%%%%%%%%%%%%%%%%%%%%%%%%%%%%%%%%%%%%%
% \paragraph{Chapter Include Files.}
%
% The include files are called |cdocsch1.tex| and |cdocsch2.tex|.
%
%\iffalse
%<*samplechap1|samplechap2>
%\fi

% Optional override for |\version| flag:
%    \begin{macrocode}
%%\providecommand{\version}{final}
%    \end{macrocode}

% Include the main document:
%    \begin{macrocode}
\input{childdoc.def}
\childdocof{cdocsamp}
%    \end{macrocode}

%\iffalse
%</samplechap1|samplechap2>
%\fi
%
%\iffalse
%<*samplechap1>
%\fi
% Some text for chapter 1:
%    \begin{macrocode}
\section{one}
some text in chapter one
%    \end{macrocode}

%\iffalse
%</samplechap1>
%\fi
% Some text for chapter 2:
%\iffalse
%<*samplechap2>
%\fi
%    \begin{macrocode}
\section{two}
more text in chapter two
%    \end{macrocode}

%\iffalse
%</samplechap2>
%\fi
%
% %%%%%%%%%%%%%%%%%%%%%%%%%%%%%%%%%%%%%%
% \paragraph{Part Include Files.}
%
% The include files are called |cdocspt3.tex| and |cdocspt4.tex|.
%
%\iffalse
%<*samplepart3|samplepart4>
%\fi

% Optional override for |\version| flag:
%    \begin{macrocode}
%%\providecommand{\version}{final}
%    \end{macrocode}

% Include the main document:
%    \begin{macrocode}
\input{childdoc.def}
\childdocby{cdocsamp}
%    \end{macrocode}

%\iffalse
%</samplepart3|samplepart4>
%\fi
%
%\iffalse
%<*samplepart3>
%\fi
% Some text for part 3:
%    \begin{macrocode}
some text in part three
%    \end{macrocode}

%\iffalse
%</samplepart3>
%\fi
% Some text for part 4:
%\iffalse
%<*samplepart4>
%\fi
%    \begin{macrocode}
more text in part four
%    \end{macrocode}

%\iffalse
%</samplepart4>
%\fi
%
% %%%%%%%%%%%%%%%%%%%%%%%%%%%%%%%%%%%%%%
% \paragraph{Forwarding for a Complete Draft.}
%
% The following forwarding file |cdocsdrf.tex|
% compiles the main document in draft mode:
%\iffalse
%<*sampledraft>
%\fi
%    \begin{macrocode}
\def\version{draft}
\input{childdoc.def}
\childdocforward{cdocsamp}
%    \end{macrocode}

%\iffalse
%</sampledraft>
%\fi
%
% %%%%%%%%%%%%%%%%%%%%%%%%%%%%%%%%%%%%%%
% \paragraph{Forwarding for Final Version of the Chapters.}
%
% The following forwarding files |cdocsfn1.tex| and |cdocsfn2.tex|
% (with identical content)
% compile the final versions of the child documents
% |cdocsch1.tex| and |cdocsch2.tex|, respectively:
%\iffalse
%<*samplefinal>
%\fi
%    \begin{macrocode}
\def\version{final}
\input{childdoc.def}
\childdocforwardprefix[cdocsamp]{cdocsfn}{cdocsch}
%    \end{macrocode}

%\iffalse
%</samplefinal>
%\fi
%
% %%%%%%%%%%%%%%%%%%%%%%%%%%%%%%%%%%%%%%
% \paragraph{Command Line Processing.}
%
% The following three command lines generate the output files
% |cdocscld|, |cdocscl1| and |cdocscl2|
% which should be identical to
% |cdocsdrf|, |cdocsch1| and |cdocsfn2|, respectively:
% \begin{center}
% \begin{tabular}{l}
% |latex -jobname cdocscld \|\\
% |  "\def\version{draft}\input{childdoc.def}\childdocforward{cdocsamp}"|\\
% |latex -jobname cdocscl1 \|\\
% |  "\input{childdoc.def}\childdocforward[cdocsamp]{cdocsch1}"|\\
% |latex -jobname cdocscl2 \|\\
% |  "\def\version{final}\input{childdoc.def}\childdocforward{cdocsch2}"|
% \end{tabular}
% \end{center}
% Note that the trailing backslash on each first line
% merely continues the input to the second line
% (for convenient cut ant paste).
% Furthermore, the command |latex| can be replaced by any
% of its alternative versions such as |pdflatex|.
%
% %%%%%%%%%%%%%%%%%%%%%%%%%%%%%%%%%%%%%%%%%%%%%%%%%%%%%%%%%%%%%%%%%%%%%%%%%%%%%%
% %%%%%%%%%%%%%%%%%%%%%%%%%%%%%%%%%%%%%%%%%%%%%%%%%%%%%%%%%%%%%%%%%%%%%%%%%%%%%%
% \section{Implementation}
%\iffalse
%<*package>
%\fi
%
% This section describes the definitions file |childdoc.def|.

% The definitions cannot be loaded using |\usepackage| or |\RequirePackage|
% which has a mechanism to prevent loading a style file more than once.
% When loading the definitions by means of |\input|
% multiple instances have to be prevented manually:
%\iffalse
%This code needs to be before the `\ProvidesFile' directive
%which is defined at the beginning of this file.
%Therefore it is also placed there and commented out here.
%</package>
%<*discard>
%\fi
%    \begin{macrocode}
\ifdefined\childdocmain\endinput\fi
%    \end{macrocode}
%\iffalse
%</discard>
%<*package>
%\fi
%
% \macro{\ifchilddoc}
% \macro{\ifchilddocmanual}
% The conditional |\ifchilddoc| tells whether a
% child (true) or main (false) document is being compiled.
% The conditional |\ifchilddocmanual| tells whether
% the |\includeonly| mechanism is used (false) or
% the selection of child files must be performed manually (true).
% The definitions initialise to false:
%    \begin{macrocode}
\newif\ifchilddoc
\newif\ifchilddocmanual
%    \end{macrocode}

% \macro{\childdocname}
% \macro{\childdocjob}
% The macro |\childdocname| stores the name of the main document
% to be compiled. The macro |\childdocjob| stores the name of
% the document on which the \LaTeX{} compiler was originally invoked.
% The content of |\jobname| cannot be compared
% to filenames specified in the source due to different catcodes.
% The following code rescans |\jobname|, stores the result
% in |\childdocname| and saves a copy in |\childdocjob|:
%    \begin{macrocode}
\edef\childdocname{\scantokens\expandafter{\jobname\noexpand}}
\let\childdocjob\childdocname
%    \end{macrocode}

% \macro{\childdocdisable}
% The macro |\childdocdisable| prevents the main file
% from being processed more than once.
% At this stage, the main document command |\childdocmain|
% is assumed to be called once again where it should do nothing.
% Any subsequent call to it should prevent
% a secondary processing of the main document
% It overwrites the forwarding commands
% |\childdocof| and |\childdocforward|
% with empty macros to prevent further inclusions of the main document:
%    \begin{macrocode}
\newcommand{\childdocdisable}
{
  \renewcommand{\childdocmain}[1]{\renewcommand{\childdocmain}[1]{\endinput}}
  \renewcommand{\childdocof}[1]{}
  \renewcommand{\childdocby}[2][]{}
  \renewcommand{\childdocforward}[2][]{}
  \renewcommand{\childdocdisable}{}
}
%    \end{macrocode}

% \macro{\childdocmain}
% The macro |\childdocmain| is to be called at the top of the main file
% with nothing or the main filename (without extension) as argument.
% First, it breaks loops.
% If the argument is not empty and does not match |\childdocname|
% (which is set by the first inclusion of |childdoc.def|),
% |\ifchilddoc| is set to true, |\includeonly| is applied to the child file
% and |\jobname| is set to the main file
% (for proper handling of |.aux| files):
%    \begin{macrocode}
\newcommand{\childdocmain}[1]
{
  \childdocdisable\childdocmain{}
  \if?#1?\else
    \begingroup
      \def\childdoctmp{#1}
      \ifx\childdoctmp\childdocname
        \def\childdoctmp{}
      \else
        \def\childdoctmp
        {
          \childdoctrue
          \includeonly{\childdocname}
          \def\childdocjob{#1}
          \def\jobname{#1}
        }
      \fi
      \expandafter
    \endgroup
    \childdoctmp
  \fi
}
%    \end{macrocode}

% \macro{\childdocof}
% The command |\childdocof| redirects
% compilation to the main file |#1|.
%    \begin{macrocode}
\newcommand{\childdocof}[1]
{
  \childdocdisable
  \childdoctrue
  \includeonly{\childdocname}
  \def\jobname{#1}
  \def\childdocjob{#1}
  \input{#1}
}
%    \end{macrocode}

% \macro{\childdocby}
% The command |\childdocby| ....
%    \begin{macrocode}
\newcommand{\childdocby}[2][]
{
  \childdocdisable
  \childdoctrue
  \childdocmanualtrue
  \if?#1?\else
    \def\jobname{#2}
  \fi
  \def\childdocjob{#2}
  \input{#2}
  \endinput
}
%    \end{macrocode}

% \macro{\childdocforward}
% The command |\childdocforward| redirects
% compilation to the main file or
% (if the optional argument is given) a child file.
% Parameters are set as if the main file
% or a child file starting with |\childdocof| was compiled.
% Then compilation is handed over to the main file:
%    \begin{macrocode}
\newcommand{\childdocforward}[2][]
{
  \begingroup
    \if?#1?
      \def\childdoctmp
      {
        \def\childdocname{#2}
        \def\childdocjob{#2}
        \def\jobname{#2}
        \input{#2}
        \endinput
      }
    \else
      \def\childdoctmp
      {
        \childdocdisable
        \def\childdocname{#2}
        \childdoctrue
        \includeonly{#2}
        \def\childdocjob{#1}
        \def\jobname{#1}
        \input{#1}
        \endinput
      }
    \fi
    \expandafter
  \endgroup
  \childdoctmp
}
%    \end{macrocode}

% \macro{\childdocforwardprefix}
% The command |\childdocforwardprefix| redirects
% compilation to the main or a child file by means of a pattern.
% The prefix |#1| in the current filename is replaced by |#2|
% and the suffix of the current filename is kept
% (it is assumed that the filename does not contain the substring `|~~~|'
% which is used as a delimiter).
% Compilation is handed over to the new file by |\childdocforward|:
%    \begin{macrocode}
\newcommand{\childdocforwardprefix}[3][]
{
  \begingroup
    \def\childdocextract #2##1~~~{\def\childdoctmp{\childdocforward[#1]{#3##1}}}
    \expandafter\childdocextract\childdocname~~~
    \expandafter
  \endgroup
  \childdoctmp
}
%    \end{macrocode}

% \macro{\childdoc}
% The deprecated macro |\childdoc| is a legacy version of |\childdocmain|:
%    \begin{macrocode}
\newcommand{\childdoc}{\childdocmain}
%    \end{macrocode}

% \macro{\childdocredirect}
% The deprecated macro |\childdocredirect| is a legacy version
% of |\childdocforward| and |\childdocforwardprefix|:
%    \begin{macrocode}
\newcommand{\childdocredirect}[2][]
{
  \begingroup
    \if?#1?
      \def\childdoctmp{\childdocforward{#2}}
    \else
      \def\childdoctmp{\childdocforwardprefix{#1}{#2}}
    \fi
    \expandafter
  \endgroup
  \childdoctmp
}
%    \end{macrocode}

%\iffalse
%</package>
%\fi
%
\endinput
|\\
|\childdocforward[|\textit{main}|]{|\textit{dest}|}|\\
\end{tabular}
\end{center}
%
The argument \textit{dest} is the destination file
(without extension).
It should be the main file or one of the child files.
Note that further \textsf{childdoc} directives
such as |\childdocof| and |\childdocforward|
in the indicated file will be processed in this form.
The optional argument \textit{main}
passes on directly to the main file \textit{main}
while pretending to compile the child \textit{dest}.
This form behaves as if \textit{dest}
issues |\childdocof{|\textit{main}|}| right away,
and no further \textsf{childdoc} directives will be processed.

%%%%%%%%%%%%%%%%%%%%%%%%%%%%%%%%%%%%%%%%
\DescribeMacro{\...prefix}
In the alternative form |\childdocforwardprefix|,
%
\begin{center}
\begin{tabular}{l}
|% \iffalse
%
% childdoc.dtx Copyright (C) 2017-2018 Niklas Beisert
%
% This work may be distributed and/or modified under the
% conditions of the LaTeX Project Public License, either version 1.3
% of this license or (at your option) any later version.
% The latest version of this license is in
%   http://www.latex-project.org/lppl.txt
% and version 1.3 or later is part of all distributions of LaTeX
% version 2005/12/01 or later.
%
% This work has the LPPL maintenance status `maintained'.
%
% The Current Maintainer of this work is Niklas Beisert.
%
% This work consists of the files childdoc.dtx and childdoc.ins
% and the derived files childdoc.def and cdocsamp.tex with
% cdocsch1.tex, cdocsch2.tex, cdocsdrf.tex, cdocsfn1.tex, cdocsfn2.tex.
%
%<package>\ifdefined\childdocmain\endinput\fi
%<package>\ProvidesFile{childdoc.def}[2018/12/30 v2.0 child document driver]
%<samplemain>\ProvidesFile{cdocsamp.tex}[2018/12/30 v2.0 sample for childdoc]
%<*driver>
%\ProvidesFile{childdoc.drv}[2018/12/30 v2.0 childdoc reference manual file]
\PassOptionsToClass{10pt,a4paper}{article}
\documentclass{ltxdoc}

\usepackage[margin=35mm]{geometry}
\usepackage{hyperref}
\usepackage{hyperxmp}
\usepackage[usenames]{color}

\hypersetup{colorlinks=true}
\hypersetup{pdfstartview=FitH}
\hypersetup{pdfpagemode=UseNone}
\hypersetup{pdfsource={}}
\hypersetup{pdflang={en-UK}}
\hypersetup{pdfcopyright={Copyright 2017-2018 Niklas Beisert.
  This work may be distributed and/or modified under the
  conditions of the LaTeX Project Public License, either version 1.3
  of this license or (at your option) any later version.}}
\hypersetup{pdflicenseurl={http://www.latex-project.org/lppl.txt}}
\hypersetup{pdfcontactaddress={ETH Zurich, ITP, HIT K,
  Wolfgang-Pauli-Strasse 27}}
\hypersetup{pdfcontactpostcode={8093}}
\hypersetup{pdfcontactcity={Zurich}}
\hypersetup{pdfcontactcountry={Switzerland}}
\hypersetup{pdfcontactemail={nbeisert@itp.phys.ethz.ch}}
\hypersetup{pdfcontacturl={http://people.phys.ethz.ch/\xmptilde nbeisert/}}

\newcommand{\secref}[1]{\hyperref[#1]{section \ref*{#1}}}

\parskip1ex
\parindent0pt
\let\olditemize\itemize
\def\itemize{\olditemize\parskip0pt}

\begin{document}

\title{The \textsf{childdoc} Package}
\hypersetup{pdftitle={The childdoc Package}}
\author{Niklas Beisert\\[2ex]
  Institut f\"ur Theoretische Physik\\
  Eidgen\"ossische Technische Hochschule Z\"urich\\
  Wolfgang-Pauli-Strasse 27, 8093 Z\"urich, Switzerland\\[1ex]
  \href{mailto:nbeisert@itp.phys.ethz.ch}
  {\texttt{nbeisert@itp.phys.ethz.ch}}}
\hypersetup{pdfauthor={Niklas Beisert}}
\hypersetup{pdfsubject={Manual for the LaTeX2e Package childdoc}}
\date{30 December 2018, \textsf{v2.0}}
\maketitle

\begin{abstract}\noindent
\textsf{childdoc} is a \LaTeXe{} package
that enables the direct compilation
of document sections included by |\include|
to individual files.
\end{abstract}

\begingroup
\parskip0ex
\tableofcontents
\endgroup

%%%%%%%%%%%%%%%%%%%%%%%%%%%%%%%%%%%%%%%%%%%%%%%%%%%%%%%%%%%%%%%%%%%%%%%%%%%%%%%%
%%%%%%%%%%%%%%%%%%%%%%%%%%%%%%%%%%%%%%%%%%%%%%%%%%%%%%%%%%%%%%%%%%%%%%%%%%%%%%%%
\section{Introduction}

\LaTeX{} provides a mechanism to structure a large document (such as a book)
into a main file and several child files (containing the chapters)
using the |\include| command.
This mechanism is beneficial for documents
which span hundreds of pages in order to
make the source file(s) more manageable.
Moreover, compilation can be restricted to
selected child files by means of the |\includeonly| command.
The latter feature can be used to reduce the compilation time while editing
(this was significantly more useful in the earlier days of \LaTeX{})
or to generate a smaller document which is easier to navigate.
Another application of |\includeonly| is to generate
documents consisting of selected parts of the complete document.

However, there are a few drawbacks of the plain |\include| mechanism:
\begin{itemize}
\item
The child files cannot be compiled on their own,
they can only be compiled via the main file.
A naive editing environment
(such as a text editor with an option
to have the current file processed by \LaTeX)
may require one to switch to the main file before compiling;
attempting to compile the child file produces errors.
\item
The main file must be modified (each time)
to adjust the |\includeonly| command
to the present needs. This easily leaves the main file in a messy state.
\item
The generated document will always carry the filename
of the main document. This is inconvenient if
several child files are to be compiled and
to be kept for distribution.
\end{itemize}

The present package provides a simple interface
to make child files individually compilable by \LaTeX{}.
Compiling a child file then has the same effect as compiling
the main file with an |\includeonly| command
to select the appropriate child.
Moreover the generated document will carry the name of the child
rather than the main file.
This resolves all three above issues.

This feature is meant to make the editing of books,
thesis documents and lecture notes somewhat more convenient.
However, the package can also be used efficiently for
composing a series of documents (such as exercise sheets)
which are typically distributed individually.
It then assists the author in generating the individual documents
(potentially in different versions)
as well as a document containing the collected series.
Another application is in developing style files
or other kinds of included material
where compilation of the style file could redirect
to a sample or test file.

%%%%%%%%%%%%%%%%%%%%%%%%%%%%%%%%%%%%%%%%%%%%%%%%%%%%%%%%%%%%%%%%%%%%%%%%%%%%%%%%
%%%%%%%%%%%%%%%%%%%%%%%%%%%%%%%%%%%%%%%%%%%%%%%%%%%%%%%%%%%%%%%%%%%%%%%%%%%%%%%%
\section{Usage}

First of all, the package \textsf{childdoc} is \emph{not} a standard
\LaTeXe{} |.sty| style file! Therefore it needs to be invoked in
a non-standard way.

%%%%%%%%%%%%%%%%%%%%%%%%%%%%%%%%%%%%%%%%%%%%%%%%%%%%%%%%%%%%%%%%%%%%%%%%%%%%%%%%
\subsection{Included Files}
\label{sec:include}

%%%%%%%%%%%%%%%%%%%%%%%%%%%%%%%%%%%%%%%%
\DescribeMacro{\childdocmain}
To use the package, add the commands
\begin{center}
\begin{tabular}{l}
|\input{childdoc.def}|\\
|\childdocmain{}|\\
\end{tabular}
\end{center}
at the very top of the main \LaTeX{} file,
in particular \emph{before} the |\documentclass| statement!
The argument of |\childdocmain| should be left empty
(but it must be present).

%%%%%%%%%%%%%%%%%%%%%%%%%%%%%%%%%%%%%%%%
\DescribeMacro{\childdocof}
Furthermore, add the commands
\begin{center}
\begin{tabular}{l}
|\input{childdoc.def}|\\
|\childdocof{|\textit{main}|}|\\
\end{tabular}
\end{center}
at the top of every child file \textit{child}
which is included by |\include{|\textit{child}|}|
from within the main file
(or at least for those files to be compiled individually).
The argument \textit{main} must be the filename of the main file.

There are a couple of
considerations in setting up the main and child documents:

%%%%%%%%%%%%%%%%%%%%%%%%%%%%%%%%%%%%%%%%
\paragraph{Restrictions.}

Please note the following restrictions:
\begin{itemize}
\item
|\childdocmain| must be called with one argument \textit{main}
to ensure compatibility with earlier version of the package.
It must either be empty (|\childdocmain{}|)
or precisely match the filename of the main file in which it is specified.
See \secref{sec:detection} for further information.
\item
The filename \textit{main} must be specified without the |.tex| extension.
\item
The filename \textit{main} is case sensitive
(even in case-insensitive file systems)
due to internal string comparison.
\item
The argument \textit{main} should be fully expanded, it cannot be a macro.
\item
Subdirectories and special characters should be avoided in filenames.
\item
The command |\childdocmain{|\textit{main}|}| must be followed by a whitespace.
It should not be followed immediately by another command
or by a comment mark `|%|'.
This is because the \TeX{} parser reads the token immediately following
the argument of |\childdocmain| and puts it
at the beginning of every child section;
however, a white\-space is ignored.
\end{itemize}

%%%%%%%%%%%%%%%%%%%%%%%%%%%%%%%%%%%%%%%%
\paragraph{Content of Main File.}

It is advisable to place all content in the child files included by |\include|.
Any output contained in the main file will appear in all child documents
unless suppressed manually;
it cannot be suppressed automatically by the |\includeonly| directive
and thus should normally be avoided.
A method to include some content in the main file
by means of conditional processing is described in \secref{sec:conditional}.

%%%%%%%%%%%%%%%%%%%%%%%%%%%%%%%%%%%%%%%%
\paragraph{Page Numbering.}

When only a part of the document is compiled,
the appropriate numbering of pages
(as well as other status parameters)
is determined from the |.aux| files.
The latter contain information from previous passes.
However this information needs to propagate through
all intermediate child documents.
Therefore the page numbering in child documents may well
be inconsistent until the complete document is compiled at least once.

A useful (if unconventional) way to always ensure a consistent
page numbering is to restart the numbering in each child document
and denote the pages by `\textit{child}|.|\textit{page}'
where \textit{child} represents the chapter/section number of the child file.
This can be achieved by the command
|\numberwithin{page}{|\textit{child}|}|
of the \textsf{amsmath} package
where \textit{child} can be |chapter| or |section|
depending on the chosen structuring.
Alternatively, one can modify the macro |\thepage| appropriately
and reset the counter |page| at the start of each child file.

%%%%%%%%%%%%%%%%%%%%%%%%%%%%%%%%%%%%%%%%%%%%%%%%%%%%%%%%%%%%%%%%%%%%%%%%%%%%%%%%
\subsection{Conditional Processing}
\label{sec:conditional}

The package provides a mechanism to compile different versions
of a document. To customise the versions further some conditional processing
can come in handy to distinguish which version is being compiled.
The package provides two macros to describe the compilation context:

%%%%%%%%%%%%%%%%%%%%%%%%%%%%%%%%%%%%%%%%
\DescribeMacro{\ifchilddoc}
The conditional |\ifchilddoc| distinguishes between the compilation of
child documents and the main document:
%
\begin{center}
|\ifchilddoc |\textit{child-code}| |[|\||else |\textit{main-code}]| \||fi|
\end{center}

%%%%%%%%%%%%%%%%%%%%%%%%%%%%%%%%%%%%%%%%
\DescribeMacro{\childdocname}
\DescribeMacro{\childdocjob}
The macro |\childdocname| contains the filename (without extension)
of the main or child file being processed.
Note that |\childdocjob| will always contain the name of the main file.

%%%%%%%%%%%%%%%%%%%%%%%%%%%%%%%%%%%%%%%%
\paragraph{Title Page.}

Conditional processing can be used to include a title or banner page
in the main document when proper precautions are taken.
Importantly, the code in the main file should ensure that the page counter
(as well as other status parameters which are stored in the |.aux| files)
takes the same value after the conditional processing.
Otherwise the page numbers may take divergent values
depending on which part is compiled.

For example, a title page could be declared by:
%
\begin{center}
\begin{tabular}{l}
|\ifchilddoc\||else|\\
|\addtocounter{page}{-1}|\\
\textit{code for title page}\\
|\newpage|\\
|\||fi|
\end{tabular}
\end{center}
%
A banner page for the child documents can be generated by:
%
\begin{center}
\begin{tabular}{l}
|\ifchilddoc|\\
|\addtocounter{page}{-1}|\\
\textit{code for banner page}\\
|\newpage|\\
|\||fi|
\end{tabular}
\end{center}
%
Here one could write a message such as:
\begin{center}
|This is the part \childdocname{} of \childdocjob{}.|
\end{center}

%%%%%%%%%%%%%%%%%%%%%%%%%%%%%%%%%%%%%%%%%%%%%%%%%%%%%%%%%%%%%%%%%%%%%%%%%%%%%%%%
\subsection{Flags}
\label{sec:flags}

The package makes it easy to generate different versions
of the main or child documents.
To this end compilation flags can be defined
and assigned different default values.
They will be particularly useful in conjunction
with the forwarding mechanism described in \secref{sec:forward}.

For example, it may be useful to have a flag |\version|
which can be set to |draft| or |final|.
The document source will contain some conditional code
depending on the value of |\version|.
Suppose further, the flag should default to |final| for the main file
and to |draft| for child files
which is a natural assignment for editing the document.
This is achieved by placing the following code
in the preamble of the main document
(below the |\childdocmain| directive):
%
\begin{center}
\begin{tabular}{l}
|\ifchilddoc|\\
|\providecommand{\version}{draft}|\\
|\||else|\\
|\providecommand{\version}{final}|\\
|\||fi|
\end{tabular}
\end{center}
%
The definition by |\providecommand| makes sure
that previous definitions are not overwritten.
Further statements |\providecommand{\version}{...}|
can thus be added before the above code to override it.

For the main file, one might add a line
(between |\childdocmain| and the above block)
%
\begin{center}
|%\ifchilddoc\||else\providecommand{\version}{draft}\||fi|
\end{center}
%
which can be uncommented to produce a draft version.
Likewise one can add a line to the very top of a child file
(above the |\childdocof{|\textit{main}|}| directive)
%
\begin{center}
|%\providecommand{\version}{final}|
\end{center}
%
which can be uncommented to produce the final version of this child document.

%%%%%%%%%%%%%%%%%%%%%%%%%%%%%%%%%%%%%%%%%%%%%%%%%%%%%%%%%%%%%%%%%%%%%%%%%%%%%%%%
\subsection{Forwarding}
\label{sec:forward}

Different versions of the main or child documents
using compilation flags as described in \secref{sec:flags}
can be (permanently) stored in different files
for convenient compilation, viewing and distribution.
To this end, the package defines a command
to pass on compilation to a different file:

%%%%%%%%%%%%%%%%%%%%%%%%%%%%%%%%%%%%%%%%
\DescribeMacro{\childdocforward}
The command |\childdocforward| redirects processing to
another source file:
%
\begin{center}
\begin{tabular}{l}
|\input{childdoc.def}|\\
|\childdocforward[|\textit{main}|]{|\textit{dest}|}|\\
\end{tabular}
\end{center}
%
The argument \textit{dest} is the destination file
(without extension).
It should be the main file or one of the child files.
Note that further \textsf{childdoc} directives
such as |\childdocof| and |\childdocforward|
in the indicated file will be processed in this form.
The optional argument \textit{main}
passes on directly to the main file \textit{main}
while pretending to compile the child \textit{dest}.
This form behaves as if \textit{dest}
issues |\childdocof{|\textit{main}|}| right away,
and no further \textsf{childdoc} directives will be processed.

%%%%%%%%%%%%%%%%%%%%%%%%%%%%%%%%%%%%%%%%
\DescribeMacro{\...prefix}
In the alternative form |\childdocforwardprefix|,
%
\begin{center}
\begin{tabular}{l}
|\input{childdoc.def}|\\
|\childdocforwardprefix[|\textit{main}|]{|\textit{prefix}|}{|\textit{dest}|}|
\end{tabular}
\end{center}
%
the destination file is determined by a pattern
depending on the current file:
To make this work, the current file must be called
`{\textit{prefix}\hspace{0.2em}\textit{suffix}}'
with \textit{prefix} matching precisely the argument.
Processing is then passed on to the file
`{\textit{dest}\hspace{0.2em}\textit{suffix}}'.
Surely, the same effect is achieved by
directly specifying the
argument `{\textit{dest}\hspace{0.2em}\textit{suffix}}'
in the first form.
However, that requires to set up a different file
for each child. With the alternative form of the command
all these files can have exactly the same content
which simplifies setting them up and maintaining them.

For example, the following file |draft.tex|
with a compilation flag |\version| as described in \secref{sec:flags}
compiles the main document as a draft:
%
\begin{center}
\begin{tabular}{l}
|\def\version{draft}|\\
|\input{childdoc.def}|\\
|\childdocforward{|\textit{main}|}|
\end{tabular}
\end{center}
%
Likewise, the following files |final|\textit{nn}|.tex|
compile the final version of the child document
|child|\textit{nn}|.tex|:
%
\begin{center}
\begin{tabular}{l}
|\def\version{final}|\\
|\input{childdoc.def}|\\
|\childdocforwardprefix{final}{child}|
\end{tabular}
\end{center}
%

Note that when several versions of a main file and/or of each child file
are to be generated, it may be convenient to set up a |Makefile| or
shell script to automatise the process.

%%%%%%%%%%%%%%%%%%%%%%%%%%%%%%%%%%%%%%%%%%%%%%%%%%%%%%%%%%%%%%%%%%%%%%%%%%%%%%%%
\subsection{Command Line Processing}
\label{sec:commandline}

The effect of redirection files can also be achieved by invoking
the \LaTeX{} compiler with a more elaborate command line.
Most conveniently this should be done as part
of a shell script or a |Makefile|.

When using \textsf{childdoc} in the main file, the following
command lines effectively perform a redirection
(note that depending on the shell being used,
backslashes may have to be doubled: `|\|' $\to$ `|\\|'):
%
\begin{center}
|... -jobname "|\textit{target}|" |\\|"|[\textit{flags}]%
|\input{childdoc.def}\childdocforward[|\textit{main}|]{|\textit{dest}|}"|
\end{center}
%
Here \textit{target} is the name of the output file,
\textit{main} is the name of the main file
and \textit{dest} is the name of the main or child file to be processed
(all filenames without extensions).
The optional argument \textit{main} can be omitted
if \textit{main} matches \textit{dest}.
Optionally, compilation \textit{flags} can be defined via |\def| commands.
This command line makes the \TeX{} engine believe
it is compiling the file \textit{target}
whose content is specified as the latter parameter.
The provided code then forwards the processing to
\textit{main} or \textit{dest} as described in \secref{sec:forward}.

%%%%%%%%%%%%%%%%%%%%%%%%%%%%%%%%%%%%%%%%%%%%%%%%%%%%%%%%%%%%%%%%%%%%%%%%%%%%%%%%
\subsection{Include by Input}
\label{sec:input}

Including child documents by |\include| has some restrictions by design.
Most notably, the content of a child document always occupies
its own set of pages; pages cannot be shared between child documents.
Usually, this behaviour makes perfect sense
because each child document contain an essential part of the document.
However, in some situations it may be desirable to compose
a document from a collection of parts
without having mandatory page breaks between then.
For this case, the package
provides a mechanism to include parts
by |\input| which can also be processed individually.
However, by construction this mechanism
requires manual handling of the content to be output.

%%%%%%%%%%%%%%%%%%%%%%%%%%%%%%%%%%%%%%%%
\DescribeMacro{\ifchilddocmanual}
The main file should be prepared as usual, see \secref{sec:include}.
However, the document body must make a distinction
between processing of an individual part and of the main document, e.g.:
%
\begin{center}
\begin{tabular}{l}
|\ifchilddocmanual|\\
|\input{\childdocname}|\\
|\||else|\\
\textit{document body with }|\input{|\textit{part}|}|\\
|\||fi|
\end{tabular}
\end{center}
%
The conditional |\ifchilddocmanual| is true whenever
a part to be included by |\input| is being compiled,
and the name of the part is stored in |\childdocname|.

%%%%%%%%%%%%%%%%%%%%%%%%%%%%%%%%%%%%%%%%
\DescribeMacro{\childdocby}
Each part to be included by |\input| should start with:
%
\begin{center}
\begin{tabular}{l}
|\input{childdoc.def}|\\
|\childdocby{|\textit{main}|}|\\
\end{tabular}
\end{center}
%
The directive |\childdocby| is similar to |\childdocof|
described in \secref{sec:include},
but the subsequent selection of content must be done manually.
To that end, both |\ifchilddoc| and |\ifchilddocmanual|
will be true upon processing of a part,
and the name of the part is stored in |\childdocname|.
Note that |\jobname| will be set to the filename of the current part
so that each part receives an individual |.aux| file
that does not interfere with the |.aux| file(s) of the main document.
This behaviour can be altered by the alternative form
|\childdocby[*]{|\textit{main}|}| (with a non-empty optional argument)
which uses the |.aux| file of the main document
by setting |\jobname| to \textit{main}.

%%%%%%%%%%%%%%%%%%%%%%%%%%%%%%%%%%%%%%%%%%%%%%%%%%%%%%%%%%%%%%%%%%%%%%%%%%%%%%%%
\subsection{Driver Development}
\label{sec:driver}

The \textsf{childdoc} mechanism can also be use for the development
of definition files such as \LaTeX{} styles or classes.
This case differs from the above setup with multiple parts
included by |\include| in that no |\includeonly| should be invoked.
This can be achieved by starting the include file
(before |\ProvidesPackage|) with:
%
\begin{center}
\begin{tabular}{l}
|\input{childdoc.def}|\\
|\childdocforward{|\textit{main}|}|\\
\end{tabular}
\end{center}
%
or alternatively with:
%
\begin{center}
\begin{tabular}{l}
|\input{childdoc.def}|\\
|\childdocby{|\textit{main}|}|\\
\end{tabular}
\end{center}
%
Both forms have slightly different effects as described above.
The main file is prepared as usual, see \secref{sec:include}.

%%%%%%%%%%%%%%%%%%%%%%%%%%%%%%%%%%%%%%%%%%%%%%%%%%%%%%%%%%%%%%%%%%%%%%%%%%%%%%%%
\subsection{Legacy Detection}
\label{sec:detection}

The directive |\childdocmain| in the main file can detect
whether the complete document or merely a child is to be compiled
even without using the directive |\childdocof|.
This method is deprecated because it is less robust
and there is no compelling reason to use it;
it is merely provided for backward compatibility
and it may be removed in future versions.

If the detection mechanism is to be used,
it is mandatory to correctly specify
the filename of the main file as the argument of |\childdocmain|:
%
\begin{center}
\begin{tabular}{l}
|\input{childdoc.def}|\\
|\childdocmain{|\textit{main}|}|\\
\end{tabular}
\end{center}
%
If |\jobname| does not match the argument \textit{main} of |\childdocmain|,
it is assumed that |\jobname| points to the child file to be compiled.
When using |\childdocmain| with the main file specified as argument,
it suffices to start a child file
with just |\input{|\textit{main}|}|
without loading of the package and using |\childdocof|.
If instead all processing is done
with the appropriate \textsf{childdoc} directives,
the argument of \textit{main} of |\childdocmain| can be empty.

An alternative version of the command line processing described
in \secref{sec:commandline} using the detection mechanism reads:
%
\begin{center}
|... -jobname "|\textit{target}|" "|[\textit{flags}]%
[|\def\jobname{|\textit{dest}|}|]|\input{|\textit{main}|}"|
\end{center}

%%%%%%%%%%%%%%%%%%%%%%%%%%%%%%%%%%%%%%%%%%%%%%%%%%%%%%%%%%%%%%%%%%%%%%%%%%%%%%%%
\subsection{Manual Code}
\label{sec:manual}

In case one cannot be certain whether the definitions file |childdoc.def|
is installed on the target \TeX{} distribution
and one prefers not to ship it,
it is conceivable to paste a few relevant commands into the sources.

To that end, drop all statements |\input{childdoc.def}|
and perform the replacements as outlined below.
Instead of |\childdocmain{|\textit{main}|}| add the following code
to the top of the main file:
%
\begin{center}
\begin{tabular}{l}
|\||ifdefined\childdocname\endinput\||fi\newif\ifchilddoc|\\
|\edef\childdocname{\scantokens\expandafter{\jobname\noexpand}}|\\
|\def\childdocmain{|\textit{main}|}\||ifx\childdocmain\childdocname\||else|\\
|\childdoctrue\includeonly{\childdocname}\let\jobname\childdocmain\||fi|\\
\end{tabular}
\end{center}
%
Instead of |\childdocof{|\textit{main}|}| just include the main file
at the top of each child file:
%
\begin{center}
|\input{|\textit{main}|}|
\end{center}
%
A simple redirection |\childdocforward{|\textit{dest}|}| is achieved by:
%
\begin{center}
|\def\jobname{|\textit{dest}|}\input{\jobname}|
\end{center}
%
The redirection with prefix
|\childdocforwardprefix[|\textit{prefix}|]{|\textit{dest}|}|
is accomplished by:
%
\begin{center}
\begin{tabular}{l}
|{\edef\jobname{\scantokens\expandafter{\jobname\noexpand}}|\\
|\def\redirectjob |\textit{prefix}|#1~~~{\gdef\jobname{|\textit{dest}|#1}}|\\
|\expandafter\redirectjob\jobname~~~}\input{\jobname}|
\end{tabular}
\end{center}

In an alternative approach,
child documents can be compiled by a specific command line
without additional code or specific definitions:
%
\begin{center}
|... -jobname "|\textit{target}|" "|[\textit{flags}]%
|\includeonly{|\textit{dest}|}\input{|\textit{main}|}"|
\end{center}
%

%%%%%%%%%%%%%%%%%%%%%%%%%%%%%%%%%%%%%%%%%%%%%%%%%%%%%%%%%%%%%%%%%%%%%%%%%%%%%%%%
%%%%%%%%%%%%%%%%%%%%%%%%%%%%%%%%%%%%%%%%%%%%%%%%%%%%%%%%%%%%%%%%%%%%%%%%%%%%%%%%
\section{Information}

%%%%%%%%%%%%%%%%%%%%%%%%%%%%%%%%%%%%%%%%%%%%%%%%%%%%%%%%%%%%%%%%%%%%%%%%%%%%%%%%
\subsection{Copyright}

Copyright \copyright{} 2017--2018 Niklas Beisert

This work may be distributed and/or modified under the
conditions of the \LaTeX{} Project Public License, either version 1.3
of this license or (at your option) any later version.
The latest version of this license is in
  \url{http://www.latex-project.org/lppl.txt}
and version 1.3 or later is part of all distributions of \LaTeX{}
version 2005/12/01 or later.

This work has the LPPL maintenance status `maintained'.

The Current Maintainer of this work is Niklas Beisert.

This work consists of the files |README.txt|, |childdoc.ins| and |childdoc.dtx|
as well as the derived files |childdoc.def|, |cdocsamp.tex|
with |cdocsch1.tex|, |cdocsch2.tex|, |cdocspt3.tex|, |cdocspt4.tex|,
|cdocsdrf.tex|, |cdocsfn1.tex|, |cdocsfn2.tex|
as well as |childdoc.pdf|.

%%%%%%%%%%%%%%%%%%%%%%%%%%%%%%%%%%%%%%%%%%%%%%%%%%%%%%%%%%%%%%%%%%%%%%%%%%%%%%%%
\subsection{Files and Installation}

The package consists of the files:
%
\begin{center}
\begin{tabular}{ll}
    |README.txt|   & readme file \\
    |childdoc.ins| & installation file \\
    |childdoc.dtx| & source file \\
    |childdoc.def| & definition file \\
    |cdocsamp.tex| & sample main file \\
    |cdocsch1.tex| & sample include file \\
    |cdocsch2.tex| & sample include file \\
    |cdocspt3.tex| & sample part file \\
    |cdocspt4.tex| & sample part file \\
    |cdocsdrf.tex| & sample redirection file \\
    |cdocsfn1.tex| & sample redirection file \\
    |cdocsfn2.tex| & sample redirection file \\
    |childdoc.pdf| & manual
\end{tabular}
\end{center}
%
The distribution consists of the files
|README.txt|, |childdoc.ins| and |childdoc.dtx|.
%
\begin{itemize}
\item
Run (pdf)\LaTeX{} on |childdoc.dtx|
to compile the manual |childdoc.pdf| (this file).
\item
Run \LaTeX{} on |childdoc.ins| to create the definitions file |childdoc.def|
and the sample |cdocsamp.tex| with include files
|cdocsch1.tex|, |cdocsch2.tex|, |cdocspt3.tex|, |cdocspt4.tex|,
|cdocsdrf.tex|, |cdocsfn1.tex|, |cdocsfn2.tex|.
Then copy the file |childdoc.def| to an appropriate directory of your \LaTeX{}
distribution, e.g.\ \textit{texmf-root}|/tex/latex/childdoc|.
\end{itemize}

%%%%%%%%%%%%%%%%%%%%%%%%%%%%%%%%%%%%%%%%%%%%%%%%%%%%%%%%%%%%%%%%%%%%%%%%%%%%%%%%
\subsection{Related CTAN Packages}

There are several other packages which offer a similar functionality:
%
\begin{itemize}
\item
The packages
\href{http://ctan.org/pkg/docmute}{\textsf{docmute}},
\href{http://ctan.org/pkg/includex}{\textsf{includex}} and
\href{http://ctan.org/pkg/standalone}{\textsf{standalone}}
provide commands to include only the document body of
a child file thus allowing both files to be compiled individually.
\item
The packages \href{http://ctan.org/pkg/subdocs}{\textsf{subdocs}}
and \href{http://ctan.org/pkg/subfiles}{\textsf{subfiles}}
provide structures in which the main and child documents can be
encapsulated and allowing them to be compiled individually.
The inclusion mechanism is different from the conventional |\include|.
\item
The package \href{http://ctan.org/pkg/combine}{\textsf{combine}}
is an elaborate solution to combine several documents into one.
\end{itemize}
%
See also the CTAN topic \href{http://ctan.org/topic/subdocs}{\textsf{subdocs}}
for further related packages.
The present package differs from the above solutions in that
a document structure constructed with the conventional |\include| mechanism
just needs two extra commands at the top of every file
such that all constituent files can be compiled individually.

%%%%%%%%%%%%%%%%%%%%%%%%%%%%%%%%%%%%%%%%%%%%%%%%%%%%%%%%%%%%%%%%%%%%%%%%%%%%%%%%
%\subsection{Feature Suggestions}
%
%The following is a list of features which may be useful for future
%versions of this package:
%%
%\begin{itemize}
%\item
%\ldots
%\end{itemize}

%%%%%%%%%%%%%%%%%%%%%%%%%%%%%%%%%%%%%%%%%%%%%%%%%%%%%%%%%%%%%%%%%%%%%%%%%%%%%%%%
\subsection{Revision History}

%%%%%%%%%%%%%%%%%%%%%%%%%%%%%%%%%%%%%%%%
\paragraph{v2.0:} 2018/12/30

\begin{itemize}
\item
immediate forward processing
\item
added |\childdocby| mechanism
\item
manual restructured
\end{itemize}

%%%%%%%%%%%%%%%%%%%%%%%%%%%%%%%%%%%%%%%%
\paragraph{v1.6:} 2018/01/17

\begin{itemize}
\item
application for development of include files
\item
corrections to manual
\end{itemize}

%%%%%%%%%%%%%%%%%%%%%%%%%%%%%%%%%%%%%%%%
\paragraph{v1.5:} 2017/05/21

\begin{itemize}
\item
more complete structuring introduced
\item
|\childdocof| introduced
\item
|\childdoc| renamed to |\childdocmain|
\item
|\childredirect| renamed to |\childdocforward| and |\childdocforwardprefix|
and functionality expanded
\end{itemize}

%%%%%%%%%%%%%%%%%%%%%%%%%%%%%%%%%%%%%%%%
\paragraph{v1.0:} 2017/04/27

\begin{itemize}
\item
manual and install package
\item
first version published on CTAN
\end{itemize}

%%%%%%%%%%%%%%%%%%%%%%%%%%%%%%%%%%%%%%%%
\paragraph{v0.6:} 2017/04/26

\begin{itemize}
\item
redirection mechanism added
\end{itemize}

%%%%%%%%%%%%%%%%%%%%%%%%%%%%%%%%%%%%%%%%
\paragraph{v0.5:} 2017/04/26

\begin{itemize}
\item
functionality in definition file
\end{itemize}


%%%%%%%%%%%%%%%%%%%%%%%%%%%%%%%%%%%%%%%%%%%%%%%%%%%%%%%%%%%%%%%%%%%%%%%%%%%%%%%%
%%%%%%%%%%%%%%%%%%%%%%%%%%%%%%%%%%%%%%%%%%%%%%%%%%%%%%%%%%%%%%%%%%%%%%%%%%%%%%%%
%%%%%%%%%%%%%%%%%%%%%%%%%%%%%%%%%%%%%%%%%%%%%%%%%%%%%%%%%%%%%%%%%%%%%%%%%%%%%%%%
\appendix

\settowidth\MacroIndent{\rmfamily\scriptsize 000\ }

 \DocInput{childdoc.dtx}

\end{document}
%</driver>
% \fi
%
% %%%%%%%%%%%%%%%%%%%%%%%%%%%%%%%%%%%%%%%%%%%%%%%%%%%%%%%%%%%%%%%%%%%%%%%%%%%%%%
% %%%%%%%%%%%%%%%%%%%%%%%%%%%%%%%%%%%%%%%%%%%%%%%%%%%%%%%%%%%%%%%%%%%%%%%%%%%%%%
% \section{Sample}
%\iffalse
%<*samplemain>
%\fi
%
% The following presents a sample document
% with two chapters, two parts, a title page,
% a compile flag as well as three forwarding files to set the flag.
% It consists of eight |.tex| files:
% \begin{center}
% \begin{tabular}{ll}
% |cdocsamp.tex|&main file\\
% |cdocsch1.tex|&include file for chapter 1\\
% |cdocsch2.tex|&include file for chapter 2\\
% |cdocspt3.tex|&include file for part 3\\
% |cdocspt4.tex|&include file for part 4\\
% |cdocsdrf.tex|&forwarding file for main file in draft mode\\
% |cdocsfi1.tex|&forwarding file for final version of chapter 1\\
% |cdocsfi2.tex|&forwarding file for final version of chapter 2\\
% \end{tabular}
% \end{center}
% Each of the eight files can be compiled directly by the \LaTeX{} compiler.
%
% %%%%%%%%%%%%%%%%%%%%%%%%%%%%%%%%%%%%%%
% \paragraph{Main File.}
%
% The main file is called |cdocsamp.tex|.
%
% Load the \textsf{childdoc} definitions and
% declare the filename for the main document:
%    \begin{macrocode}
\input{childdoc.def}
\childdocmain{}
%    \end{macrocode}

% Optional override for |\version| flag:
%    \begin{macrocode}
%%\ifchilddoc\else\providecommand{\version}{draft}\fi
%    \end{macrocode}

% Define the default values for the |\version| flag
% (|final| for the main file and |draft| for childs):
%    \begin{macrocode}
\ifchilddoc
\providecommand{\version}{draft}
\else
\providecommand{\version}{final}
\fi
%    \end{macrocode}

% Load the standard document class:
%    \begin{macrocode}
\documentclass[12pt]{article}
%    \end{macrocode}

% Start the document body:
%    \begin{macrocode}
\begin{document}
%    \end{macrocode}

% Declare a title page.
% Print title, part of document being processed and version flag:
%    \begin{macrocode}
\addtocounter{page}{-1}
\begin{center}
{\LARGE\bfseries{}childdoc example\par}
\vspace{1cm}
\ifchilddoc
\ifchilddocmanual part\else chapter\fi:
`\childdocname' of `\childdocjob'\par
\else
main document: `\childdocjob'\par
\fi
version: \version\par
\end{center}
\newpage
%    \end{macrocode}

% Manually include selected file,
% otherwise process as usual:
%    \begin{macrocode}
\ifchilddocmanual
\section*{part `\childdocname'}
\input{\childdocname}
\else
%    \end{macrocode}

% Include the two chapters:
%    \begin{macrocode}
\include{cdocsch1}
\include{cdocsch2}
%    \end{macrocode}

% Include the two parts unless only chapters should be displayed:
%    \begin{macrocode}
\ifchilddoc\else
\section{part three}
\input{cdocspt3}
\section{part four}
\input{cdocspt4}
\fi
%    \end{macrocode}

% Process as usual until here:
%    \begin{macrocode}
\fi
%    \end{macrocode}

% End of document body:
%    \begin{macrocode}
\end{document}
%    \end{macrocode}
%\iffalse
%</samplemain>
%\fi
%
% %%%%%%%%%%%%%%%%%%%%%%%%%%%%%%%%%%%%%%
% \paragraph{Chapter Include Files.}
%
% The include files are called |cdocsch1.tex| and |cdocsch2.tex|.
%
%\iffalse
%<*samplechap1|samplechap2>
%\fi

% Optional override for |\version| flag:
%    \begin{macrocode}
%%\providecommand{\version}{final}
%    \end{macrocode}

% Include the main document:
%    \begin{macrocode}
\input{childdoc.def}
\childdocof{cdocsamp}
%    \end{macrocode}

%\iffalse
%</samplechap1|samplechap2>
%\fi
%
%\iffalse
%<*samplechap1>
%\fi
% Some text for chapter 1:
%    \begin{macrocode}
\section{one}
some text in chapter one
%    \end{macrocode}

%\iffalse
%</samplechap1>
%\fi
% Some text for chapter 2:
%\iffalse
%<*samplechap2>
%\fi
%    \begin{macrocode}
\section{two}
more text in chapter two
%    \end{macrocode}

%\iffalse
%</samplechap2>
%\fi
%
% %%%%%%%%%%%%%%%%%%%%%%%%%%%%%%%%%%%%%%
% \paragraph{Part Include Files.}
%
% The include files are called |cdocspt3.tex| and |cdocspt4.tex|.
%
%\iffalse
%<*samplepart3|samplepart4>
%\fi

% Optional override for |\version| flag:
%    \begin{macrocode}
%%\providecommand{\version}{final}
%    \end{macrocode}

% Include the main document:
%    \begin{macrocode}
\input{childdoc.def}
\childdocby{cdocsamp}
%    \end{macrocode}

%\iffalse
%</samplepart3|samplepart4>
%\fi
%
%\iffalse
%<*samplepart3>
%\fi
% Some text for part 3:
%    \begin{macrocode}
some text in part three
%    \end{macrocode}

%\iffalse
%</samplepart3>
%\fi
% Some text for part 4:
%\iffalse
%<*samplepart4>
%\fi
%    \begin{macrocode}
more text in part four
%    \end{macrocode}

%\iffalse
%</samplepart4>
%\fi
%
% %%%%%%%%%%%%%%%%%%%%%%%%%%%%%%%%%%%%%%
% \paragraph{Forwarding for a Complete Draft.}
%
% The following forwarding file |cdocsdrf.tex|
% compiles the main document in draft mode:
%\iffalse
%<*sampledraft>
%\fi
%    \begin{macrocode}
\def\version{draft}
\input{childdoc.def}
\childdocforward{cdocsamp}
%    \end{macrocode}

%\iffalse
%</sampledraft>
%\fi
%
% %%%%%%%%%%%%%%%%%%%%%%%%%%%%%%%%%%%%%%
% \paragraph{Forwarding for Final Version of the Chapters.}
%
% The following forwarding files |cdocsfn1.tex| and |cdocsfn2.tex|
% (with identical content)
% compile the final versions of the child documents
% |cdocsch1.tex| and |cdocsch2.tex|, respectively:
%\iffalse
%<*samplefinal>
%\fi
%    \begin{macrocode}
\def\version{final}
\input{childdoc.def}
\childdocforwardprefix[cdocsamp]{cdocsfn}{cdocsch}
%    \end{macrocode}

%\iffalse
%</samplefinal>
%\fi
%
% %%%%%%%%%%%%%%%%%%%%%%%%%%%%%%%%%%%%%%
% \paragraph{Command Line Processing.}
%
% The following three command lines generate the output files
% |cdocscld|, |cdocscl1| and |cdocscl2|
% which should be identical to
% |cdocsdrf|, |cdocsch1| and |cdocsfn2|, respectively:
% \begin{center}
% \begin{tabular}{l}
% |latex -jobname cdocscld \|\\
% |  "\def\version{draft}\input{childdoc.def}\childdocforward{cdocsamp}"|\\
% |latex -jobname cdocscl1 \|\\
% |  "\input{childdoc.def}\childdocforward[cdocsamp]{cdocsch1}"|\\
% |latex -jobname cdocscl2 \|\\
% |  "\def\version{final}\input{childdoc.def}\childdocforward{cdocsch2}"|
% \end{tabular}
% \end{center}
% Note that the trailing backslash on each first line
% merely continues the input to the second line
% (for convenient cut ant paste).
% Furthermore, the command |latex| can be replaced by any
% of its alternative versions such as |pdflatex|.
%
% %%%%%%%%%%%%%%%%%%%%%%%%%%%%%%%%%%%%%%%%%%%%%%%%%%%%%%%%%%%%%%%%%%%%%%%%%%%%%%
% %%%%%%%%%%%%%%%%%%%%%%%%%%%%%%%%%%%%%%%%%%%%%%%%%%%%%%%%%%%%%%%%%%%%%%%%%%%%%%
% \section{Implementation}
%\iffalse
%<*package>
%\fi
%
% This section describes the definitions file |childdoc.def|.

% The definitions cannot be loaded using |\usepackage| or |\RequirePackage|
% which has a mechanism to prevent loading a style file more than once.
% When loading the definitions by means of |\input|
% multiple instances have to be prevented manually:
%\iffalse
%This code needs to be before the `\ProvidesFile' directive
%which is defined at the beginning of this file.
%Therefore it is also placed there and commented out here.
%</package>
%<*discard>
%\fi
%    \begin{macrocode}
\ifdefined\childdocmain\endinput\fi
%    \end{macrocode}
%\iffalse
%</discard>
%<*package>
%\fi
%
% \macro{\ifchilddoc}
% \macro{\ifchilddocmanual}
% The conditional |\ifchilddoc| tells whether a
% child (true) or main (false) document is being compiled.
% The conditional |\ifchilddocmanual| tells whether
% the |\includeonly| mechanism is used (false) or
% the selection of child files must be performed manually (true).
% The definitions initialise to false:
%    \begin{macrocode}
\newif\ifchilddoc
\newif\ifchilddocmanual
%    \end{macrocode}

% \macro{\childdocname}
% \macro{\childdocjob}
% The macro |\childdocname| stores the name of the main document
% to be compiled. The macro |\childdocjob| stores the name of
% the document on which the \LaTeX{} compiler was originally invoked.
% The content of |\jobname| cannot be compared
% to filenames specified in the source due to different catcodes.
% The following code rescans |\jobname|, stores the result
% in |\childdocname| and saves a copy in |\childdocjob|:
%    \begin{macrocode}
\edef\childdocname{\scantokens\expandafter{\jobname\noexpand}}
\let\childdocjob\childdocname
%    \end{macrocode}

% \macro{\childdocdisable}
% The macro |\childdocdisable| prevents the main file
% from being processed more than once.
% At this stage, the main document command |\childdocmain|
% is assumed to be called once again where it should do nothing.
% Any subsequent call to it should prevent
% a secondary processing of the main document
% It overwrites the forwarding commands
% |\childdocof| and |\childdocforward|
% with empty macros to prevent further inclusions of the main document:
%    \begin{macrocode}
\newcommand{\childdocdisable}
{
  \renewcommand{\childdocmain}[1]{\renewcommand{\childdocmain}[1]{\endinput}}
  \renewcommand{\childdocof}[1]{}
  \renewcommand{\childdocby}[2][]{}
  \renewcommand{\childdocforward}[2][]{}
  \renewcommand{\childdocdisable}{}
}
%    \end{macrocode}

% \macro{\childdocmain}
% The macro |\childdocmain| is to be called at the top of the main file
% with nothing or the main filename (without extension) as argument.
% First, it breaks loops.
% If the argument is not empty and does not match |\childdocname|
% (which is set by the first inclusion of |childdoc.def|),
% |\ifchilddoc| is set to true, |\includeonly| is applied to the child file
% and |\jobname| is set to the main file
% (for proper handling of |.aux| files):
%    \begin{macrocode}
\newcommand{\childdocmain}[1]
{
  \childdocdisable\childdocmain{}
  \if?#1?\else
    \begingroup
      \def\childdoctmp{#1}
      \ifx\childdoctmp\childdocname
        \def\childdoctmp{}
      \else
        \def\childdoctmp
        {
          \childdoctrue
          \includeonly{\childdocname}
          \def\childdocjob{#1}
          \def\jobname{#1}
        }
      \fi
      \expandafter
    \endgroup
    \childdoctmp
  \fi
}
%    \end{macrocode}

% \macro{\childdocof}
% The command |\childdocof| redirects
% compilation to the main file |#1|.
%    \begin{macrocode}
\newcommand{\childdocof}[1]
{
  \childdocdisable
  \childdoctrue
  \includeonly{\childdocname}
  \def\jobname{#1}
  \def\childdocjob{#1}
  \input{#1}
}
%    \end{macrocode}

% \macro{\childdocby}
% The command |\childdocby| ....
%    \begin{macrocode}
\newcommand{\childdocby}[2][]
{
  \childdocdisable
  \childdoctrue
  \childdocmanualtrue
  \if?#1?\else
    \def\jobname{#2}
  \fi
  \def\childdocjob{#2}
  \input{#2}
  \endinput
}
%    \end{macrocode}

% \macro{\childdocforward}
% The command |\childdocforward| redirects
% compilation to the main file or
% (if the optional argument is given) a child file.
% Parameters are set as if the main file
% or a child file starting with |\childdocof| was compiled.
% Then compilation is handed over to the main file:
%    \begin{macrocode}
\newcommand{\childdocforward}[2][]
{
  \begingroup
    \if?#1?
      \def\childdoctmp
      {
        \def\childdocname{#2}
        \def\childdocjob{#2}
        \def\jobname{#2}
        \input{#2}
        \endinput
      }
    \else
      \def\childdoctmp
      {
        \childdocdisable
        \def\childdocname{#2}
        \childdoctrue
        \includeonly{#2}
        \def\childdocjob{#1}
        \def\jobname{#1}
        \input{#1}
        \endinput
      }
    \fi
    \expandafter
  \endgroup
  \childdoctmp
}
%    \end{macrocode}

% \macro{\childdocforwardprefix}
% The command |\childdocforwardprefix| redirects
% compilation to the main or a child file by means of a pattern.
% The prefix |#1| in the current filename is replaced by |#2|
% and the suffix of the current filename is kept
% (it is assumed that the filename does not contain the substring `|~~~|'
% which is used as a delimiter).
% Compilation is handed over to the new file by |\childdocforward|:
%    \begin{macrocode}
\newcommand{\childdocforwardprefix}[3][]
{
  \begingroup
    \def\childdocextract #2##1~~~{\def\childdoctmp{\childdocforward[#1]{#3##1}}}
    \expandafter\childdocextract\childdocname~~~
    \expandafter
  \endgroup
  \childdoctmp
}
%    \end{macrocode}

% \macro{\childdoc}
% The deprecated macro |\childdoc| is a legacy version of |\childdocmain|:
%    \begin{macrocode}
\newcommand{\childdoc}{\childdocmain}
%    \end{macrocode}

% \macro{\childdocredirect}
% The deprecated macro |\childdocredirect| is a legacy version
% of |\childdocforward| and |\childdocforwardprefix|:
%    \begin{macrocode}
\newcommand{\childdocredirect}[2][]
{
  \begingroup
    \if?#1?
      \def\childdoctmp{\childdocforward{#2}}
    \else
      \def\childdoctmp{\childdocforwardprefix{#1}{#2}}
    \fi
    \expandafter
  \endgroup
  \childdoctmp
}
%    \end{macrocode}

%\iffalse
%</package>
%\fi
%
\endinput
|\\
|\childdocforwardprefix[|\textit{main}|]{|\textit{prefix}|}{|\textit{dest}|}|
\end{tabular}
\end{center}
%
the destination file is determined by a pattern
depending on the current file:
To make this work, the current file must be called
`{\textit{prefix}\hspace{0.2em}\textit{suffix}}'
with \textit{prefix} matching precisely the argument.
Processing is then passed on to the file
`{\textit{dest}\hspace{0.2em}\textit{suffix}}'.
Surely, the same effect is achieved by
directly specifying the
argument `{\textit{dest}\hspace{0.2em}\textit{suffix}}'
in the first form.
However, that requires to set up a different file
for each child. With the alternative form of the command
all these files can have exactly the same content
which simplifies setting them up and maintaining them.

For example, the following file |draft.tex|
with a compilation flag |\version| as described in \secref{sec:flags}
compiles the main document as a draft:
%
\begin{center}
\begin{tabular}{l}
|\def\version{draft}|\\
|% \iffalse
%
% childdoc.dtx Copyright (C) 2017-2018 Niklas Beisert
%
% This work may be distributed and/or modified under the
% conditions of the LaTeX Project Public License, either version 1.3
% of this license or (at your option) any later version.
% The latest version of this license is in
%   http://www.latex-project.org/lppl.txt
% and version 1.3 or later is part of all distributions of LaTeX
% version 2005/12/01 or later.
%
% This work has the LPPL maintenance status `maintained'.
%
% The Current Maintainer of this work is Niklas Beisert.
%
% This work consists of the files childdoc.dtx and childdoc.ins
% and the derived files childdoc.def and cdocsamp.tex with
% cdocsch1.tex, cdocsch2.tex, cdocsdrf.tex, cdocsfn1.tex, cdocsfn2.tex.
%
%<package>\ifdefined\childdocmain\endinput\fi
%<package>\ProvidesFile{childdoc.def}[2018/12/30 v2.0 child document driver]
%<samplemain>\ProvidesFile{cdocsamp.tex}[2018/12/30 v2.0 sample for childdoc]
%<*driver>
%\ProvidesFile{childdoc.drv}[2018/12/30 v2.0 childdoc reference manual file]
\PassOptionsToClass{10pt,a4paper}{article}
\documentclass{ltxdoc}

\usepackage[margin=35mm]{geometry}
\usepackage{hyperref}
\usepackage{hyperxmp}
\usepackage[usenames]{color}

\hypersetup{colorlinks=true}
\hypersetup{pdfstartview=FitH}
\hypersetup{pdfpagemode=UseNone}
\hypersetup{pdfsource={}}
\hypersetup{pdflang={en-UK}}
\hypersetup{pdfcopyright={Copyright 2017-2018 Niklas Beisert.
  This work may be distributed and/or modified under the
  conditions of the LaTeX Project Public License, either version 1.3
  of this license or (at your option) any later version.}}
\hypersetup{pdflicenseurl={http://www.latex-project.org/lppl.txt}}
\hypersetup{pdfcontactaddress={ETH Zurich, ITP, HIT K,
  Wolfgang-Pauli-Strasse 27}}
\hypersetup{pdfcontactpostcode={8093}}
\hypersetup{pdfcontactcity={Zurich}}
\hypersetup{pdfcontactcountry={Switzerland}}
\hypersetup{pdfcontactemail={nbeisert@itp.phys.ethz.ch}}
\hypersetup{pdfcontacturl={http://people.phys.ethz.ch/\xmptilde nbeisert/}}

\newcommand{\secref}[1]{\hyperref[#1]{section \ref*{#1}}}

\parskip1ex
\parindent0pt
\let\olditemize\itemize
\def\itemize{\olditemize\parskip0pt}

\begin{document}

\title{The \textsf{childdoc} Package}
\hypersetup{pdftitle={The childdoc Package}}
\author{Niklas Beisert\\[2ex]
  Institut f\"ur Theoretische Physik\\
  Eidgen\"ossische Technische Hochschule Z\"urich\\
  Wolfgang-Pauli-Strasse 27, 8093 Z\"urich, Switzerland\\[1ex]
  \href{mailto:nbeisert@itp.phys.ethz.ch}
  {\texttt{nbeisert@itp.phys.ethz.ch}}}
\hypersetup{pdfauthor={Niklas Beisert}}
\hypersetup{pdfsubject={Manual for the LaTeX2e Package childdoc}}
\date{30 December 2018, \textsf{v2.0}}
\maketitle

\begin{abstract}\noindent
\textsf{childdoc} is a \LaTeXe{} package
that enables the direct compilation
of document sections included by |\include|
to individual files.
\end{abstract}

\begingroup
\parskip0ex
\tableofcontents
\endgroup

%%%%%%%%%%%%%%%%%%%%%%%%%%%%%%%%%%%%%%%%%%%%%%%%%%%%%%%%%%%%%%%%%%%%%%%%%%%%%%%%
%%%%%%%%%%%%%%%%%%%%%%%%%%%%%%%%%%%%%%%%%%%%%%%%%%%%%%%%%%%%%%%%%%%%%%%%%%%%%%%%
\section{Introduction}

\LaTeX{} provides a mechanism to structure a large document (such as a book)
into a main file and several child files (containing the chapters)
using the |\include| command.
This mechanism is beneficial for documents
which span hundreds of pages in order to
make the source file(s) more manageable.
Moreover, compilation can be restricted to
selected child files by means of the |\includeonly| command.
The latter feature can be used to reduce the compilation time while editing
(this was significantly more useful in the earlier days of \LaTeX{})
or to generate a smaller document which is easier to navigate.
Another application of |\includeonly| is to generate
documents consisting of selected parts of the complete document.

However, there are a few drawbacks of the plain |\include| mechanism:
\begin{itemize}
\item
The child files cannot be compiled on their own,
they can only be compiled via the main file.
A naive editing environment
(such as a text editor with an option
to have the current file processed by \LaTeX)
may require one to switch to the main file before compiling;
attempting to compile the child file produces errors.
\item
The main file must be modified (each time)
to adjust the |\includeonly| command
to the present needs. This easily leaves the main file in a messy state.
\item
The generated document will always carry the filename
of the main document. This is inconvenient if
several child files are to be compiled and
to be kept for distribution.
\end{itemize}

The present package provides a simple interface
to make child files individually compilable by \LaTeX{}.
Compiling a child file then has the same effect as compiling
the main file with an |\includeonly| command
to select the appropriate child.
Moreover the generated document will carry the name of the child
rather than the main file.
This resolves all three above issues.

This feature is meant to make the editing of books,
thesis documents and lecture notes somewhat more convenient.
However, the package can also be used efficiently for
composing a series of documents (such as exercise sheets)
which are typically distributed individually.
It then assists the author in generating the individual documents
(potentially in different versions)
as well as a document containing the collected series.
Another application is in developing style files
or other kinds of included material
where compilation of the style file could redirect
to a sample or test file.

%%%%%%%%%%%%%%%%%%%%%%%%%%%%%%%%%%%%%%%%%%%%%%%%%%%%%%%%%%%%%%%%%%%%%%%%%%%%%%%%
%%%%%%%%%%%%%%%%%%%%%%%%%%%%%%%%%%%%%%%%%%%%%%%%%%%%%%%%%%%%%%%%%%%%%%%%%%%%%%%%
\section{Usage}

First of all, the package \textsf{childdoc} is \emph{not} a standard
\LaTeXe{} |.sty| style file! Therefore it needs to be invoked in
a non-standard way.

%%%%%%%%%%%%%%%%%%%%%%%%%%%%%%%%%%%%%%%%%%%%%%%%%%%%%%%%%%%%%%%%%%%%%%%%%%%%%%%%
\subsection{Included Files}
\label{sec:include}

%%%%%%%%%%%%%%%%%%%%%%%%%%%%%%%%%%%%%%%%
\DescribeMacro{\childdocmain}
To use the package, add the commands
\begin{center}
\begin{tabular}{l}
|\input{childdoc.def}|\\
|\childdocmain{}|\\
\end{tabular}
\end{center}
at the very top of the main \LaTeX{} file,
in particular \emph{before} the |\documentclass| statement!
The argument of |\childdocmain| should be left empty
(but it must be present).

%%%%%%%%%%%%%%%%%%%%%%%%%%%%%%%%%%%%%%%%
\DescribeMacro{\childdocof}
Furthermore, add the commands
\begin{center}
\begin{tabular}{l}
|\input{childdoc.def}|\\
|\childdocof{|\textit{main}|}|\\
\end{tabular}
\end{center}
at the top of every child file \textit{child}
which is included by |\include{|\textit{child}|}|
from within the main file
(or at least for those files to be compiled individually).
The argument \textit{main} must be the filename of the main file.

There are a couple of
considerations in setting up the main and child documents:

%%%%%%%%%%%%%%%%%%%%%%%%%%%%%%%%%%%%%%%%
\paragraph{Restrictions.}

Please note the following restrictions:
\begin{itemize}
\item
|\childdocmain| must be called with one argument \textit{main}
to ensure compatibility with earlier version of the package.
It must either be empty (|\childdocmain{}|)
or precisely match the filename of the main file in which it is specified.
See \secref{sec:detection} for further information.
\item
The filename \textit{main} must be specified without the |.tex| extension.
\item
The filename \textit{main} is case sensitive
(even in case-insensitive file systems)
due to internal string comparison.
\item
The argument \textit{main} should be fully expanded, it cannot be a macro.
\item
Subdirectories and special characters should be avoided in filenames.
\item
The command |\childdocmain{|\textit{main}|}| must be followed by a whitespace.
It should not be followed immediately by another command
or by a comment mark `|%|'.
This is because the \TeX{} parser reads the token immediately following
the argument of |\childdocmain| and puts it
at the beginning of every child section;
however, a white\-space is ignored.
\end{itemize}

%%%%%%%%%%%%%%%%%%%%%%%%%%%%%%%%%%%%%%%%
\paragraph{Content of Main File.}

It is advisable to place all content in the child files included by |\include|.
Any output contained in the main file will appear in all child documents
unless suppressed manually;
it cannot be suppressed automatically by the |\includeonly| directive
and thus should normally be avoided.
A method to include some content in the main file
by means of conditional processing is described in \secref{sec:conditional}.

%%%%%%%%%%%%%%%%%%%%%%%%%%%%%%%%%%%%%%%%
\paragraph{Page Numbering.}

When only a part of the document is compiled,
the appropriate numbering of pages
(as well as other status parameters)
is determined from the |.aux| files.
The latter contain information from previous passes.
However this information needs to propagate through
all intermediate child documents.
Therefore the page numbering in child documents may well
be inconsistent until the complete document is compiled at least once.

A useful (if unconventional) way to always ensure a consistent
page numbering is to restart the numbering in each child document
and denote the pages by `\textit{child}|.|\textit{page}'
where \textit{child} represents the chapter/section number of the child file.
This can be achieved by the command
|\numberwithin{page}{|\textit{child}|}|
of the \textsf{amsmath} package
where \textit{child} can be |chapter| or |section|
depending on the chosen structuring.
Alternatively, one can modify the macro |\thepage| appropriately
and reset the counter |page| at the start of each child file.

%%%%%%%%%%%%%%%%%%%%%%%%%%%%%%%%%%%%%%%%%%%%%%%%%%%%%%%%%%%%%%%%%%%%%%%%%%%%%%%%
\subsection{Conditional Processing}
\label{sec:conditional}

The package provides a mechanism to compile different versions
of a document. To customise the versions further some conditional processing
can come in handy to distinguish which version is being compiled.
The package provides two macros to describe the compilation context:

%%%%%%%%%%%%%%%%%%%%%%%%%%%%%%%%%%%%%%%%
\DescribeMacro{\ifchilddoc}
The conditional |\ifchilddoc| distinguishes between the compilation of
child documents and the main document:
%
\begin{center}
|\ifchilddoc |\textit{child-code}| |[|\||else |\textit{main-code}]| \||fi|
\end{center}

%%%%%%%%%%%%%%%%%%%%%%%%%%%%%%%%%%%%%%%%
\DescribeMacro{\childdocname}
\DescribeMacro{\childdocjob}
The macro |\childdocname| contains the filename (without extension)
of the main or child file being processed.
Note that |\childdocjob| will always contain the name of the main file.

%%%%%%%%%%%%%%%%%%%%%%%%%%%%%%%%%%%%%%%%
\paragraph{Title Page.}

Conditional processing can be used to include a title or banner page
in the main document when proper precautions are taken.
Importantly, the code in the main file should ensure that the page counter
(as well as other status parameters which are stored in the |.aux| files)
takes the same value after the conditional processing.
Otherwise the page numbers may take divergent values
depending on which part is compiled.

For example, a title page could be declared by:
%
\begin{center}
\begin{tabular}{l}
|\ifchilddoc\||else|\\
|\addtocounter{page}{-1}|\\
\textit{code for title page}\\
|\newpage|\\
|\||fi|
\end{tabular}
\end{center}
%
A banner page for the child documents can be generated by:
%
\begin{center}
\begin{tabular}{l}
|\ifchilddoc|\\
|\addtocounter{page}{-1}|\\
\textit{code for banner page}\\
|\newpage|\\
|\||fi|
\end{tabular}
\end{center}
%
Here one could write a message such as:
\begin{center}
|This is the part \childdocname{} of \childdocjob{}.|
\end{center}

%%%%%%%%%%%%%%%%%%%%%%%%%%%%%%%%%%%%%%%%%%%%%%%%%%%%%%%%%%%%%%%%%%%%%%%%%%%%%%%%
\subsection{Flags}
\label{sec:flags}

The package makes it easy to generate different versions
of the main or child documents.
To this end compilation flags can be defined
and assigned different default values.
They will be particularly useful in conjunction
with the forwarding mechanism described in \secref{sec:forward}.

For example, it may be useful to have a flag |\version|
which can be set to |draft| or |final|.
The document source will contain some conditional code
depending on the value of |\version|.
Suppose further, the flag should default to |final| for the main file
and to |draft| for child files
which is a natural assignment for editing the document.
This is achieved by placing the following code
in the preamble of the main document
(below the |\childdocmain| directive):
%
\begin{center}
\begin{tabular}{l}
|\ifchilddoc|\\
|\providecommand{\version}{draft}|\\
|\||else|\\
|\providecommand{\version}{final}|\\
|\||fi|
\end{tabular}
\end{center}
%
The definition by |\providecommand| makes sure
that previous definitions are not overwritten.
Further statements |\providecommand{\version}{...}|
can thus be added before the above code to override it.

For the main file, one might add a line
(between |\childdocmain| and the above block)
%
\begin{center}
|%\ifchilddoc\||else\providecommand{\version}{draft}\||fi|
\end{center}
%
which can be uncommented to produce a draft version.
Likewise one can add a line to the very top of a child file
(above the |\childdocof{|\textit{main}|}| directive)
%
\begin{center}
|%\providecommand{\version}{final}|
\end{center}
%
which can be uncommented to produce the final version of this child document.

%%%%%%%%%%%%%%%%%%%%%%%%%%%%%%%%%%%%%%%%%%%%%%%%%%%%%%%%%%%%%%%%%%%%%%%%%%%%%%%%
\subsection{Forwarding}
\label{sec:forward}

Different versions of the main or child documents
using compilation flags as described in \secref{sec:flags}
can be (permanently) stored in different files
for convenient compilation, viewing and distribution.
To this end, the package defines a command
to pass on compilation to a different file:

%%%%%%%%%%%%%%%%%%%%%%%%%%%%%%%%%%%%%%%%
\DescribeMacro{\childdocforward}
The command |\childdocforward| redirects processing to
another source file:
%
\begin{center}
\begin{tabular}{l}
|\input{childdoc.def}|\\
|\childdocforward[|\textit{main}|]{|\textit{dest}|}|\\
\end{tabular}
\end{center}
%
The argument \textit{dest} is the destination file
(without extension).
It should be the main file or one of the child files.
Note that further \textsf{childdoc} directives
such as |\childdocof| and |\childdocforward|
in the indicated file will be processed in this form.
The optional argument \textit{main}
passes on directly to the main file \textit{main}
while pretending to compile the child \textit{dest}.
This form behaves as if \textit{dest}
issues |\childdocof{|\textit{main}|}| right away,
and no further \textsf{childdoc} directives will be processed.

%%%%%%%%%%%%%%%%%%%%%%%%%%%%%%%%%%%%%%%%
\DescribeMacro{\...prefix}
In the alternative form |\childdocforwardprefix|,
%
\begin{center}
\begin{tabular}{l}
|\input{childdoc.def}|\\
|\childdocforwardprefix[|\textit{main}|]{|\textit{prefix}|}{|\textit{dest}|}|
\end{tabular}
\end{center}
%
the destination file is determined by a pattern
depending on the current file:
To make this work, the current file must be called
`{\textit{prefix}\hspace{0.2em}\textit{suffix}}'
with \textit{prefix} matching precisely the argument.
Processing is then passed on to the file
`{\textit{dest}\hspace{0.2em}\textit{suffix}}'.
Surely, the same effect is achieved by
directly specifying the
argument `{\textit{dest}\hspace{0.2em}\textit{suffix}}'
in the first form.
However, that requires to set up a different file
for each child. With the alternative form of the command
all these files can have exactly the same content
which simplifies setting them up and maintaining them.

For example, the following file |draft.tex|
with a compilation flag |\version| as described in \secref{sec:flags}
compiles the main document as a draft:
%
\begin{center}
\begin{tabular}{l}
|\def\version{draft}|\\
|\input{childdoc.def}|\\
|\childdocforward{|\textit{main}|}|
\end{tabular}
\end{center}
%
Likewise, the following files |final|\textit{nn}|.tex|
compile the final version of the child document
|child|\textit{nn}|.tex|:
%
\begin{center}
\begin{tabular}{l}
|\def\version{final}|\\
|\input{childdoc.def}|\\
|\childdocforwardprefix{final}{child}|
\end{tabular}
\end{center}
%

Note that when several versions of a main file and/or of each child file
are to be generated, it may be convenient to set up a |Makefile| or
shell script to automatise the process.

%%%%%%%%%%%%%%%%%%%%%%%%%%%%%%%%%%%%%%%%%%%%%%%%%%%%%%%%%%%%%%%%%%%%%%%%%%%%%%%%
\subsection{Command Line Processing}
\label{sec:commandline}

The effect of redirection files can also be achieved by invoking
the \LaTeX{} compiler with a more elaborate command line.
Most conveniently this should be done as part
of a shell script or a |Makefile|.

When using \textsf{childdoc} in the main file, the following
command lines effectively perform a redirection
(note that depending on the shell being used,
backslashes may have to be doubled: `|\|' $\to$ `|\\|'):
%
\begin{center}
|... -jobname "|\textit{target}|" |\\|"|[\textit{flags}]%
|\input{childdoc.def}\childdocforward[|\textit{main}|]{|\textit{dest}|}"|
\end{center}
%
Here \textit{target} is the name of the output file,
\textit{main} is the name of the main file
and \textit{dest} is the name of the main or child file to be processed
(all filenames without extensions).
The optional argument \textit{main} can be omitted
if \textit{main} matches \textit{dest}.
Optionally, compilation \textit{flags} can be defined via |\def| commands.
This command line makes the \TeX{} engine believe
it is compiling the file \textit{target}
whose content is specified as the latter parameter.
The provided code then forwards the processing to
\textit{main} or \textit{dest} as described in \secref{sec:forward}.

%%%%%%%%%%%%%%%%%%%%%%%%%%%%%%%%%%%%%%%%%%%%%%%%%%%%%%%%%%%%%%%%%%%%%%%%%%%%%%%%
\subsection{Include by Input}
\label{sec:input}

Including child documents by |\include| has some restrictions by design.
Most notably, the content of a child document always occupies
its own set of pages; pages cannot be shared between child documents.
Usually, this behaviour makes perfect sense
because each child document contain an essential part of the document.
However, in some situations it may be desirable to compose
a document from a collection of parts
without having mandatory page breaks between then.
For this case, the package
provides a mechanism to include parts
by |\input| which can also be processed individually.
However, by construction this mechanism
requires manual handling of the content to be output.

%%%%%%%%%%%%%%%%%%%%%%%%%%%%%%%%%%%%%%%%
\DescribeMacro{\ifchilddocmanual}
The main file should be prepared as usual, see \secref{sec:include}.
However, the document body must make a distinction
between processing of an individual part and of the main document, e.g.:
%
\begin{center}
\begin{tabular}{l}
|\ifchilddocmanual|\\
|\input{\childdocname}|\\
|\||else|\\
\textit{document body with }|\input{|\textit{part}|}|\\
|\||fi|
\end{tabular}
\end{center}
%
The conditional |\ifchilddocmanual| is true whenever
a part to be included by |\input| is being compiled,
and the name of the part is stored in |\childdocname|.

%%%%%%%%%%%%%%%%%%%%%%%%%%%%%%%%%%%%%%%%
\DescribeMacro{\childdocby}
Each part to be included by |\input| should start with:
%
\begin{center}
\begin{tabular}{l}
|\input{childdoc.def}|\\
|\childdocby{|\textit{main}|}|\\
\end{tabular}
\end{center}
%
The directive |\childdocby| is similar to |\childdocof|
described in \secref{sec:include},
but the subsequent selection of content must be done manually.
To that end, both |\ifchilddoc| and |\ifchilddocmanual|
will be true upon processing of a part,
and the name of the part is stored in |\childdocname|.
Note that |\jobname| will be set to the filename of the current part
so that each part receives an individual |.aux| file
that does not interfere with the |.aux| file(s) of the main document.
This behaviour can be altered by the alternative form
|\childdocby[*]{|\textit{main}|}| (with a non-empty optional argument)
which uses the |.aux| file of the main document
by setting |\jobname| to \textit{main}.

%%%%%%%%%%%%%%%%%%%%%%%%%%%%%%%%%%%%%%%%%%%%%%%%%%%%%%%%%%%%%%%%%%%%%%%%%%%%%%%%
\subsection{Driver Development}
\label{sec:driver}

The \textsf{childdoc} mechanism can also be use for the development
of definition files such as \LaTeX{} styles or classes.
This case differs from the above setup with multiple parts
included by |\include| in that no |\includeonly| should be invoked.
This can be achieved by starting the include file
(before |\ProvidesPackage|) with:
%
\begin{center}
\begin{tabular}{l}
|\input{childdoc.def}|\\
|\childdocforward{|\textit{main}|}|\\
\end{tabular}
\end{center}
%
or alternatively with:
%
\begin{center}
\begin{tabular}{l}
|\input{childdoc.def}|\\
|\childdocby{|\textit{main}|}|\\
\end{tabular}
\end{center}
%
Both forms have slightly different effects as described above.
The main file is prepared as usual, see \secref{sec:include}.

%%%%%%%%%%%%%%%%%%%%%%%%%%%%%%%%%%%%%%%%%%%%%%%%%%%%%%%%%%%%%%%%%%%%%%%%%%%%%%%%
\subsection{Legacy Detection}
\label{sec:detection}

The directive |\childdocmain| in the main file can detect
whether the complete document or merely a child is to be compiled
even without using the directive |\childdocof|.
This method is deprecated because it is less robust
and there is no compelling reason to use it;
it is merely provided for backward compatibility
and it may be removed in future versions.

If the detection mechanism is to be used,
it is mandatory to correctly specify
the filename of the main file as the argument of |\childdocmain|:
%
\begin{center}
\begin{tabular}{l}
|\input{childdoc.def}|\\
|\childdocmain{|\textit{main}|}|\\
\end{tabular}
\end{center}
%
If |\jobname| does not match the argument \textit{main} of |\childdocmain|,
it is assumed that |\jobname| points to the child file to be compiled.
When using |\childdocmain| with the main file specified as argument,
it suffices to start a child file
with just |\input{|\textit{main}|}|
without loading of the package and using |\childdocof|.
If instead all processing is done
with the appropriate \textsf{childdoc} directives,
the argument of \textit{main} of |\childdocmain| can be empty.

An alternative version of the command line processing described
in \secref{sec:commandline} using the detection mechanism reads:
%
\begin{center}
|... -jobname "|\textit{target}|" "|[\textit{flags}]%
[|\def\jobname{|\textit{dest}|}|]|\input{|\textit{main}|}"|
\end{center}

%%%%%%%%%%%%%%%%%%%%%%%%%%%%%%%%%%%%%%%%%%%%%%%%%%%%%%%%%%%%%%%%%%%%%%%%%%%%%%%%
\subsection{Manual Code}
\label{sec:manual}

In case one cannot be certain whether the definitions file |childdoc.def|
is installed on the target \TeX{} distribution
and one prefers not to ship it,
it is conceivable to paste a few relevant commands into the sources.

To that end, drop all statements |\input{childdoc.def}|
and perform the replacements as outlined below.
Instead of |\childdocmain{|\textit{main}|}| add the following code
to the top of the main file:
%
\begin{center}
\begin{tabular}{l}
|\||ifdefined\childdocname\endinput\||fi\newif\ifchilddoc|\\
|\edef\childdocname{\scantokens\expandafter{\jobname\noexpand}}|\\
|\def\childdocmain{|\textit{main}|}\||ifx\childdocmain\childdocname\||else|\\
|\childdoctrue\includeonly{\childdocname}\let\jobname\childdocmain\||fi|\\
\end{tabular}
\end{center}
%
Instead of |\childdocof{|\textit{main}|}| just include the main file
at the top of each child file:
%
\begin{center}
|\input{|\textit{main}|}|
\end{center}
%
A simple redirection |\childdocforward{|\textit{dest}|}| is achieved by:
%
\begin{center}
|\def\jobname{|\textit{dest}|}\input{\jobname}|
\end{center}
%
The redirection with prefix
|\childdocforwardprefix[|\textit{prefix}|]{|\textit{dest}|}|
is accomplished by:
%
\begin{center}
\begin{tabular}{l}
|{\edef\jobname{\scantokens\expandafter{\jobname\noexpand}}|\\
|\def\redirectjob |\textit{prefix}|#1~~~{\gdef\jobname{|\textit{dest}|#1}}|\\
|\expandafter\redirectjob\jobname~~~}\input{\jobname}|
\end{tabular}
\end{center}

In an alternative approach,
child documents can be compiled by a specific command line
without additional code or specific definitions:
%
\begin{center}
|... -jobname "|\textit{target}|" "|[\textit{flags}]%
|\includeonly{|\textit{dest}|}\input{|\textit{main}|}"|
\end{center}
%

%%%%%%%%%%%%%%%%%%%%%%%%%%%%%%%%%%%%%%%%%%%%%%%%%%%%%%%%%%%%%%%%%%%%%%%%%%%%%%%%
%%%%%%%%%%%%%%%%%%%%%%%%%%%%%%%%%%%%%%%%%%%%%%%%%%%%%%%%%%%%%%%%%%%%%%%%%%%%%%%%
\section{Information}

%%%%%%%%%%%%%%%%%%%%%%%%%%%%%%%%%%%%%%%%%%%%%%%%%%%%%%%%%%%%%%%%%%%%%%%%%%%%%%%%
\subsection{Copyright}

Copyright \copyright{} 2017--2018 Niklas Beisert

This work may be distributed and/or modified under the
conditions of the \LaTeX{} Project Public License, either version 1.3
of this license or (at your option) any later version.
The latest version of this license is in
  \url{http://www.latex-project.org/lppl.txt}
and version 1.3 or later is part of all distributions of \LaTeX{}
version 2005/12/01 or later.

This work has the LPPL maintenance status `maintained'.

The Current Maintainer of this work is Niklas Beisert.

This work consists of the files |README.txt|, |childdoc.ins| and |childdoc.dtx|
as well as the derived files |childdoc.def|, |cdocsamp.tex|
with |cdocsch1.tex|, |cdocsch2.tex|, |cdocspt3.tex|, |cdocspt4.tex|,
|cdocsdrf.tex|, |cdocsfn1.tex|, |cdocsfn2.tex|
as well as |childdoc.pdf|.

%%%%%%%%%%%%%%%%%%%%%%%%%%%%%%%%%%%%%%%%%%%%%%%%%%%%%%%%%%%%%%%%%%%%%%%%%%%%%%%%
\subsection{Files and Installation}

The package consists of the files:
%
\begin{center}
\begin{tabular}{ll}
    |README.txt|   & readme file \\
    |childdoc.ins| & installation file \\
    |childdoc.dtx| & source file \\
    |childdoc.def| & definition file \\
    |cdocsamp.tex| & sample main file \\
    |cdocsch1.tex| & sample include file \\
    |cdocsch2.tex| & sample include file \\
    |cdocspt3.tex| & sample part file \\
    |cdocspt4.tex| & sample part file \\
    |cdocsdrf.tex| & sample redirection file \\
    |cdocsfn1.tex| & sample redirection file \\
    |cdocsfn2.tex| & sample redirection file \\
    |childdoc.pdf| & manual
\end{tabular}
\end{center}
%
The distribution consists of the files
|README.txt|, |childdoc.ins| and |childdoc.dtx|.
%
\begin{itemize}
\item
Run (pdf)\LaTeX{} on |childdoc.dtx|
to compile the manual |childdoc.pdf| (this file).
\item
Run \LaTeX{} on |childdoc.ins| to create the definitions file |childdoc.def|
and the sample |cdocsamp.tex| with include files
|cdocsch1.tex|, |cdocsch2.tex|, |cdocspt3.tex|, |cdocspt4.tex|,
|cdocsdrf.tex|, |cdocsfn1.tex|, |cdocsfn2.tex|.
Then copy the file |childdoc.def| to an appropriate directory of your \LaTeX{}
distribution, e.g.\ \textit{texmf-root}|/tex/latex/childdoc|.
\end{itemize}

%%%%%%%%%%%%%%%%%%%%%%%%%%%%%%%%%%%%%%%%%%%%%%%%%%%%%%%%%%%%%%%%%%%%%%%%%%%%%%%%
\subsection{Related CTAN Packages}

There are several other packages which offer a similar functionality:
%
\begin{itemize}
\item
The packages
\href{http://ctan.org/pkg/docmute}{\textsf{docmute}},
\href{http://ctan.org/pkg/includex}{\textsf{includex}} and
\href{http://ctan.org/pkg/standalone}{\textsf{standalone}}
provide commands to include only the document body of
a child file thus allowing both files to be compiled individually.
\item
The packages \href{http://ctan.org/pkg/subdocs}{\textsf{subdocs}}
and \href{http://ctan.org/pkg/subfiles}{\textsf{subfiles}}
provide structures in which the main and child documents can be
encapsulated and allowing them to be compiled individually.
The inclusion mechanism is different from the conventional |\include|.
\item
The package \href{http://ctan.org/pkg/combine}{\textsf{combine}}
is an elaborate solution to combine several documents into one.
\end{itemize}
%
See also the CTAN topic \href{http://ctan.org/topic/subdocs}{\textsf{subdocs}}
for further related packages.
The present package differs from the above solutions in that
a document structure constructed with the conventional |\include| mechanism
just needs two extra commands at the top of every file
such that all constituent files can be compiled individually.

%%%%%%%%%%%%%%%%%%%%%%%%%%%%%%%%%%%%%%%%%%%%%%%%%%%%%%%%%%%%%%%%%%%%%%%%%%%%%%%%
%\subsection{Feature Suggestions}
%
%The following is a list of features which may be useful for future
%versions of this package:
%%
%\begin{itemize}
%\item
%\ldots
%\end{itemize}

%%%%%%%%%%%%%%%%%%%%%%%%%%%%%%%%%%%%%%%%%%%%%%%%%%%%%%%%%%%%%%%%%%%%%%%%%%%%%%%%
\subsection{Revision History}

%%%%%%%%%%%%%%%%%%%%%%%%%%%%%%%%%%%%%%%%
\paragraph{v2.0:} 2018/12/30

\begin{itemize}
\item
immediate forward processing
\item
added |\childdocby| mechanism
\item
manual restructured
\end{itemize}

%%%%%%%%%%%%%%%%%%%%%%%%%%%%%%%%%%%%%%%%
\paragraph{v1.6:} 2018/01/17

\begin{itemize}
\item
application for development of include files
\item
corrections to manual
\end{itemize}

%%%%%%%%%%%%%%%%%%%%%%%%%%%%%%%%%%%%%%%%
\paragraph{v1.5:} 2017/05/21

\begin{itemize}
\item
more complete structuring introduced
\item
|\childdocof| introduced
\item
|\childdoc| renamed to |\childdocmain|
\item
|\childredirect| renamed to |\childdocforward| and |\childdocforwardprefix|
and functionality expanded
\end{itemize}

%%%%%%%%%%%%%%%%%%%%%%%%%%%%%%%%%%%%%%%%
\paragraph{v1.0:} 2017/04/27

\begin{itemize}
\item
manual and install package
\item
first version published on CTAN
\end{itemize}

%%%%%%%%%%%%%%%%%%%%%%%%%%%%%%%%%%%%%%%%
\paragraph{v0.6:} 2017/04/26

\begin{itemize}
\item
redirection mechanism added
\end{itemize}

%%%%%%%%%%%%%%%%%%%%%%%%%%%%%%%%%%%%%%%%
\paragraph{v0.5:} 2017/04/26

\begin{itemize}
\item
functionality in definition file
\end{itemize}


%%%%%%%%%%%%%%%%%%%%%%%%%%%%%%%%%%%%%%%%%%%%%%%%%%%%%%%%%%%%%%%%%%%%%%%%%%%%%%%%
%%%%%%%%%%%%%%%%%%%%%%%%%%%%%%%%%%%%%%%%%%%%%%%%%%%%%%%%%%%%%%%%%%%%%%%%%%%%%%%%
%%%%%%%%%%%%%%%%%%%%%%%%%%%%%%%%%%%%%%%%%%%%%%%%%%%%%%%%%%%%%%%%%%%%%%%%%%%%%%%%
\appendix

\settowidth\MacroIndent{\rmfamily\scriptsize 000\ }

 \DocInput{childdoc.dtx}

\end{document}
%</driver>
% \fi
%
% %%%%%%%%%%%%%%%%%%%%%%%%%%%%%%%%%%%%%%%%%%%%%%%%%%%%%%%%%%%%%%%%%%%%%%%%%%%%%%
% %%%%%%%%%%%%%%%%%%%%%%%%%%%%%%%%%%%%%%%%%%%%%%%%%%%%%%%%%%%%%%%%%%%%%%%%%%%%%%
% \section{Sample}
%\iffalse
%<*samplemain>
%\fi
%
% The following presents a sample document
% with two chapters, two parts, a title page,
% a compile flag as well as three forwarding files to set the flag.
% It consists of eight |.tex| files:
% \begin{center}
% \begin{tabular}{ll}
% |cdocsamp.tex|&main file\\
% |cdocsch1.tex|&include file for chapter 1\\
% |cdocsch2.tex|&include file for chapter 2\\
% |cdocspt3.tex|&include file for part 3\\
% |cdocspt4.tex|&include file for part 4\\
% |cdocsdrf.tex|&forwarding file for main file in draft mode\\
% |cdocsfi1.tex|&forwarding file for final version of chapter 1\\
% |cdocsfi2.tex|&forwarding file for final version of chapter 2\\
% \end{tabular}
% \end{center}
% Each of the eight files can be compiled directly by the \LaTeX{} compiler.
%
% %%%%%%%%%%%%%%%%%%%%%%%%%%%%%%%%%%%%%%
% \paragraph{Main File.}
%
% The main file is called |cdocsamp.tex|.
%
% Load the \textsf{childdoc} definitions and
% declare the filename for the main document:
%    \begin{macrocode}
\input{childdoc.def}
\childdocmain{}
%    \end{macrocode}

% Optional override for |\version| flag:
%    \begin{macrocode}
%%\ifchilddoc\else\providecommand{\version}{draft}\fi
%    \end{macrocode}

% Define the default values for the |\version| flag
% (|final| for the main file and |draft| for childs):
%    \begin{macrocode}
\ifchilddoc
\providecommand{\version}{draft}
\else
\providecommand{\version}{final}
\fi
%    \end{macrocode}

% Load the standard document class:
%    \begin{macrocode}
\documentclass[12pt]{article}
%    \end{macrocode}

% Start the document body:
%    \begin{macrocode}
\begin{document}
%    \end{macrocode}

% Declare a title page.
% Print title, part of document being processed and version flag:
%    \begin{macrocode}
\addtocounter{page}{-1}
\begin{center}
{\LARGE\bfseries{}childdoc example\par}
\vspace{1cm}
\ifchilddoc
\ifchilddocmanual part\else chapter\fi:
`\childdocname' of `\childdocjob'\par
\else
main document: `\childdocjob'\par
\fi
version: \version\par
\end{center}
\newpage
%    \end{macrocode}

% Manually include selected file,
% otherwise process as usual:
%    \begin{macrocode}
\ifchilddocmanual
\section*{part `\childdocname'}
\input{\childdocname}
\else
%    \end{macrocode}

% Include the two chapters:
%    \begin{macrocode}
\include{cdocsch1}
\include{cdocsch2}
%    \end{macrocode}

% Include the two parts unless only chapters should be displayed:
%    \begin{macrocode}
\ifchilddoc\else
\section{part three}
\input{cdocspt3}
\section{part four}
\input{cdocspt4}
\fi
%    \end{macrocode}

% Process as usual until here:
%    \begin{macrocode}
\fi
%    \end{macrocode}

% End of document body:
%    \begin{macrocode}
\end{document}
%    \end{macrocode}
%\iffalse
%</samplemain>
%\fi
%
% %%%%%%%%%%%%%%%%%%%%%%%%%%%%%%%%%%%%%%
% \paragraph{Chapter Include Files.}
%
% The include files are called |cdocsch1.tex| and |cdocsch2.tex|.
%
%\iffalse
%<*samplechap1|samplechap2>
%\fi

% Optional override for |\version| flag:
%    \begin{macrocode}
%%\providecommand{\version}{final}
%    \end{macrocode}

% Include the main document:
%    \begin{macrocode}
\input{childdoc.def}
\childdocof{cdocsamp}
%    \end{macrocode}

%\iffalse
%</samplechap1|samplechap2>
%\fi
%
%\iffalse
%<*samplechap1>
%\fi
% Some text for chapter 1:
%    \begin{macrocode}
\section{one}
some text in chapter one
%    \end{macrocode}

%\iffalse
%</samplechap1>
%\fi
% Some text for chapter 2:
%\iffalse
%<*samplechap2>
%\fi
%    \begin{macrocode}
\section{two}
more text in chapter two
%    \end{macrocode}

%\iffalse
%</samplechap2>
%\fi
%
% %%%%%%%%%%%%%%%%%%%%%%%%%%%%%%%%%%%%%%
% \paragraph{Part Include Files.}
%
% The include files are called |cdocspt3.tex| and |cdocspt4.tex|.
%
%\iffalse
%<*samplepart3|samplepart4>
%\fi

% Optional override for |\version| flag:
%    \begin{macrocode}
%%\providecommand{\version}{final}
%    \end{macrocode}

% Include the main document:
%    \begin{macrocode}
\input{childdoc.def}
\childdocby{cdocsamp}
%    \end{macrocode}

%\iffalse
%</samplepart3|samplepart4>
%\fi
%
%\iffalse
%<*samplepart3>
%\fi
% Some text for part 3:
%    \begin{macrocode}
some text in part three
%    \end{macrocode}

%\iffalse
%</samplepart3>
%\fi
% Some text for part 4:
%\iffalse
%<*samplepart4>
%\fi
%    \begin{macrocode}
more text in part four
%    \end{macrocode}

%\iffalse
%</samplepart4>
%\fi
%
% %%%%%%%%%%%%%%%%%%%%%%%%%%%%%%%%%%%%%%
% \paragraph{Forwarding for a Complete Draft.}
%
% The following forwarding file |cdocsdrf.tex|
% compiles the main document in draft mode:
%\iffalse
%<*sampledraft>
%\fi
%    \begin{macrocode}
\def\version{draft}
\input{childdoc.def}
\childdocforward{cdocsamp}
%    \end{macrocode}

%\iffalse
%</sampledraft>
%\fi
%
% %%%%%%%%%%%%%%%%%%%%%%%%%%%%%%%%%%%%%%
% \paragraph{Forwarding for Final Version of the Chapters.}
%
% The following forwarding files |cdocsfn1.tex| and |cdocsfn2.tex|
% (with identical content)
% compile the final versions of the child documents
% |cdocsch1.tex| and |cdocsch2.tex|, respectively:
%\iffalse
%<*samplefinal>
%\fi
%    \begin{macrocode}
\def\version{final}
\input{childdoc.def}
\childdocforwardprefix[cdocsamp]{cdocsfn}{cdocsch}
%    \end{macrocode}

%\iffalse
%</samplefinal>
%\fi
%
% %%%%%%%%%%%%%%%%%%%%%%%%%%%%%%%%%%%%%%
% \paragraph{Command Line Processing.}
%
% The following three command lines generate the output files
% |cdocscld|, |cdocscl1| and |cdocscl2|
% which should be identical to
% |cdocsdrf|, |cdocsch1| and |cdocsfn2|, respectively:
% \begin{center}
% \begin{tabular}{l}
% |latex -jobname cdocscld \|\\
% |  "\def\version{draft}\input{childdoc.def}\childdocforward{cdocsamp}"|\\
% |latex -jobname cdocscl1 \|\\
% |  "\input{childdoc.def}\childdocforward[cdocsamp]{cdocsch1}"|\\
% |latex -jobname cdocscl2 \|\\
% |  "\def\version{final}\input{childdoc.def}\childdocforward{cdocsch2}"|
% \end{tabular}
% \end{center}
% Note that the trailing backslash on each first line
% merely continues the input to the second line
% (for convenient cut ant paste).
% Furthermore, the command |latex| can be replaced by any
% of its alternative versions such as |pdflatex|.
%
% %%%%%%%%%%%%%%%%%%%%%%%%%%%%%%%%%%%%%%%%%%%%%%%%%%%%%%%%%%%%%%%%%%%%%%%%%%%%%%
% %%%%%%%%%%%%%%%%%%%%%%%%%%%%%%%%%%%%%%%%%%%%%%%%%%%%%%%%%%%%%%%%%%%%%%%%%%%%%%
% \section{Implementation}
%\iffalse
%<*package>
%\fi
%
% This section describes the definitions file |childdoc.def|.

% The definitions cannot be loaded using |\usepackage| or |\RequirePackage|
% which has a mechanism to prevent loading a style file more than once.
% When loading the definitions by means of |\input|
% multiple instances have to be prevented manually:
%\iffalse
%This code needs to be before the `\ProvidesFile' directive
%which is defined at the beginning of this file.
%Therefore it is also placed there and commented out here.
%</package>
%<*discard>
%\fi
%    \begin{macrocode}
\ifdefined\childdocmain\endinput\fi
%    \end{macrocode}
%\iffalse
%</discard>
%<*package>
%\fi
%
% \macro{\ifchilddoc}
% \macro{\ifchilddocmanual}
% The conditional |\ifchilddoc| tells whether a
% child (true) or main (false) document is being compiled.
% The conditional |\ifchilddocmanual| tells whether
% the |\includeonly| mechanism is used (false) or
% the selection of child files must be performed manually (true).
% The definitions initialise to false:
%    \begin{macrocode}
\newif\ifchilddoc
\newif\ifchilddocmanual
%    \end{macrocode}

% \macro{\childdocname}
% \macro{\childdocjob}
% The macro |\childdocname| stores the name of the main document
% to be compiled. The macro |\childdocjob| stores the name of
% the document on which the \LaTeX{} compiler was originally invoked.
% The content of |\jobname| cannot be compared
% to filenames specified in the source due to different catcodes.
% The following code rescans |\jobname|, stores the result
% in |\childdocname| and saves a copy in |\childdocjob|:
%    \begin{macrocode}
\edef\childdocname{\scantokens\expandafter{\jobname\noexpand}}
\let\childdocjob\childdocname
%    \end{macrocode}

% \macro{\childdocdisable}
% The macro |\childdocdisable| prevents the main file
% from being processed more than once.
% At this stage, the main document command |\childdocmain|
% is assumed to be called once again where it should do nothing.
% Any subsequent call to it should prevent
% a secondary processing of the main document
% It overwrites the forwarding commands
% |\childdocof| and |\childdocforward|
% with empty macros to prevent further inclusions of the main document:
%    \begin{macrocode}
\newcommand{\childdocdisable}
{
  \renewcommand{\childdocmain}[1]{\renewcommand{\childdocmain}[1]{\endinput}}
  \renewcommand{\childdocof}[1]{}
  \renewcommand{\childdocby}[2][]{}
  \renewcommand{\childdocforward}[2][]{}
  \renewcommand{\childdocdisable}{}
}
%    \end{macrocode}

% \macro{\childdocmain}
% The macro |\childdocmain| is to be called at the top of the main file
% with nothing or the main filename (without extension) as argument.
% First, it breaks loops.
% If the argument is not empty and does not match |\childdocname|
% (which is set by the first inclusion of |childdoc.def|),
% |\ifchilddoc| is set to true, |\includeonly| is applied to the child file
% and |\jobname| is set to the main file
% (for proper handling of |.aux| files):
%    \begin{macrocode}
\newcommand{\childdocmain}[1]
{
  \childdocdisable\childdocmain{}
  \if?#1?\else
    \begingroup
      \def\childdoctmp{#1}
      \ifx\childdoctmp\childdocname
        \def\childdoctmp{}
      \else
        \def\childdoctmp
        {
          \childdoctrue
          \includeonly{\childdocname}
          \def\childdocjob{#1}
          \def\jobname{#1}
        }
      \fi
      \expandafter
    \endgroup
    \childdoctmp
  \fi
}
%    \end{macrocode}

% \macro{\childdocof}
% The command |\childdocof| redirects
% compilation to the main file |#1|.
%    \begin{macrocode}
\newcommand{\childdocof}[1]
{
  \childdocdisable
  \childdoctrue
  \includeonly{\childdocname}
  \def\jobname{#1}
  \def\childdocjob{#1}
  \input{#1}
}
%    \end{macrocode}

% \macro{\childdocby}
% The command |\childdocby| ....
%    \begin{macrocode}
\newcommand{\childdocby}[2][]
{
  \childdocdisable
  \childdoctrue
  \childdocmanualtrue
  \if?#1?\else
    \def\jobname{#2}
  \fi
  \def\childdocjob{#2}
  \input{#2}
  \endinput
}
%    \end{macrocode}

% \macro{\childdocforward}
% The command |\childdocforward| redirects
% compilation to the main file or
% (if the optional argument is given) a child file.
% Parameters are set as if the main file
% or a child file starting with |\childdocof| was compiled.
% Then compilation is handed over to the main file:
%    \begin{macrocode}
\newcommand{\childdocforward}[2][]
{
  \begingroup
    \if?#1?
      \def\childdoctmp
      {
        \def\childdocname{#2}
        \def\childdocjob{#2}
        \def\jobname{#2}
        \input{#2}
        \endinput
      }
    \else
      \def\childdoctmp
      {
        \childdocdisable
        \def\childdocname{#2}
        \childdoctrue
        \includeonly{#2}
        \def\childdocjob{#1}
        \def\jobname{#1}
        \input{#1}
        \endinput
      }
    \fi
    \expandafter
  \endgroup
  \childdoctmp
}
%    \end{macrocode}

% \macro{\childdocforwardprefix}
% The command |\childdocforwardprefix| redirects
% compilation to the main or a child file by means of a pattern.
% The prefix |#1| in the current filename is replaced by |#2|
% and the suffix of the current filename is kept
% (it is assumed that the filename does not contain the substring `|~~~|'
% which is used as a delimiter).
% Compilation is handed over to the new file by |\childdocforward|:
%    \begin{macrocode}
\newcommand{\childdocforwardprefix}[3][]
{
  \begingroup
    \def\childdocextract #2##1~~~{\def\childdoctmp{\childdocforward[#1]{#3##1}}}
    \expandafter\childdocextract\childdocname~~~
    \expandafter
  \endgroup
  \childdoctmp
}
%    \end{macrocode}

% \macro{\childdoc}
% The deprecated macro |\childdoc| is a legacy version of |\childdocmain|:
%    \begin{macrocode}
\newcommand{\childdoc}{\childdocmain}
%    \end{macrocode}

% \macro{\childdocredirect}
% The deprecated macro |\childdocredirect| is a legacy version
% of |\childdocforward| and |\childdocforwardprefix|:
%    \begin{macrocode}
\newcommand{\childdocredirect}[2][]
{
  \begingroup
    \if?#1?
      \def\childdoctmp{\childdocforward{#2}}
    \else
      \def\childdoctmp{\childdocforwardprefix{#1}{#2}}
    \fi
    \expandafter
  \endgroup
  \childdoctmp
}
%    \end{macrocode}

%\iffalse
%</package>
%\fi
%
\endinput
|\\
|\childdocforward{|\textit{main}|}|
\end{tabular}
\end{center}
%
Likewise, the following files |final|\textit{nn}|.tex|
compile the final version of the child document
|child|\textit{nn}|.tex|:
%
\begin{center}
\begin{tabular}{l}
|\def\version{final}|\\
|% \iffalse
%
% childdoc.dtx Copyright (C) 2017-2018 Niklas Beisert
%
% This work may be distributed and/or modified under the
% conditions of the LaTeX Project Public License, either version 1.3
% of this license or (at your option) any later version.
% The latest version of this license is in
%   http://www.latex-project.org/lppl.txt
% and version 1.3 or later is part of all distributions of LaTeX
% version 2005/12/01 or later.
%
% This work has the LPPL maintenance status `maintained'.
%
% The Current Maintainer of this work is Niklas Beisert.
%
% This work consists of the files childdoc.dtx and childdoc.ins
% and the derived files childdoc.def and cdocsamp.tex with
% cdocsch1.tex, cdocsch2.tex, cdocsdrf.tex, cdocsfn1.tex, cdocsfn2.tex.
%
%<package>\ifdefined\childdocmain\endinput\fi
%<package>\ProvidesFile{childdoc.def}[2018/12/30 v2.0 child document driver]
%<samplemain>\ProvidesFile{cdocsamp.tex}[2018/12/30 v2.0 sample for childdoc]
%<*driver>
%\ProvidesFile{childdoc.drv}[2018/12/30 v2.0 childdoc reference manual file]
\PassOptionsToClass{10pt,a4paper}{article}
\documentclass{ltxdoc}

\usepackage[margin=35mm]{geometry}
\usepackage{hyperref}
\usepackage{hyperxmp}
\usepackage[usenames]{color}

\hypersetup{colorlinks=true}
\hypersetup{pdfstartview=FitH}
\hypersetup{pdfpagemode=UseNone}
\hypersetup{pdfsource={}}
\hypersetup{pdflang={en-UK}}
\hypersetup{pdfcopyright={Copyright 2017-2018 Niklas Beisert.
  This work may be distributed and/or modified under the
  conditions of the LaTeX Project Public License, either version 1.3
  of this license or (at your option) any later version.}}
\hypersetup{pdflicenseurl={http://www.latex-project.org/lppl.txt}}
\hypersetup{pdfcontactaddress={ETH Zurich, ITP, HIT K,
  Wolfgang-Pauli-Strasse 27}}
\hypersetup{pdfcontactpostcode={8093}}
\hypersetup{pdfcontactcity={Zurich}}
\hypersetup{pdfcontactcountry={Switzerland}}
\hypersetup{pdfcontactemail={nbeisert@itp.phys.ethz.ch}}
\hypersetup{pdfcontacturl={http://people.phys.ethz.ch/\xmptilde nbeisert/}}

\newcommand{\secref}[1]{\hyperref[#1]{section \ref*{#1}}}

\parskip1ex
\parindent0pt
\let\olditemize\itemize
\def\itemize{\olditemize\parskip0pt}

\begin{document}

\title{The \textsf{childdoc} Package}
\hypersetup{pdftitle={The childdoc Package}}
\author{Niklas Beisert\\[2ex]
  Institut f\"ur Theoretische Physik\\
  Eidgen\"ossische Technische Hochschule Z\"urich\\
  Wolfgang-Pauli-Strasse 27, 8093 Z\"urich, Switzerland\\[1ex]
  \href{mailto:nbeisert@itp.phys.ethz.ch}
  {\texttt{nbeisert@itp.phys.ethz.ch}}}
\hypersetup{pdfauthor={Niklas Beisert}}
\hypersetup{pdfsubject={Manual for the LaTeX2e Package childdoc}}
\date{30 December 2018, \textsf{v2.0}}
\maketitle

\begin{abstract}\noindent
\textsf{childdoc} is a \LaTeXe{} package
that enables the direct compilation
of document sections included by |\include|
to individual files.
\end{abstract}

\begingroup
\parskip0ex
\tableofcontents
\endgroup

%%%%%%%%%%%%%%%%%%%%%%%%%%%%%%%%%%%%%%%%%%%%%%%%%%%%%%%%%%%%%%%%%%%%%%%%%%%%%%%%
%%%%%%%%%%%%%%%%%%%%%%%%%%%%%%%%%%%%%%%%%%%%%%%%%%%%%%%%%%%%%%%%%%%%%%%%%%%%%%%%
\section{Introduction}

\LaTeX{} provides a mechanism to structure a large document (such as a book)
into a main file and several child files (containing the chapters)
using the |\include| command.
This mechanism is beneficial for documents
which span hundreds of pages in order to
make the source file(s) more manageable.
Moreover, compilation can be restricted to
selected child files by means of the |\includeonly| command.
The latter feature can be used to reduce the compilation time while editing
(this was significantly more useful in the earlier days of \LaTeX{})
or to generate a smaller document which is easier to navigate.
Another application of |\includeonly| is to generate
documents consisting of selected parts of the complete document.

However, there are a few drawbacks of the plain |\include| mechanism:
\begin{itemize}
\item
The child files cannot be compiled on their own,
they can only be compiled via the main file.
A naive editing environment
(such as a text editor with an option
to have the current file processed by \LaTeX)
may require one to switch to the main file before compiling;
attempting to compile the child file produces errors.
\item
The main file must be modified (each time)
to adjust the |\includeonly| command
to the present needs. This easily leaves the main file in a messy state.
\item
The generated document will always carry the filename
of the main document. This is inconvenient if
several child files are to be compiled and
to be kept for distribution.
\end{itemize}

The present package provides a simple interface
to make child files individually compilable by \LaTeX{}.
Compiling a child file then has the same effect as compiling
the main file with an |\includeonly| command
to select the appropriate child.
Moreover the generated document will carry the name of the child
rather than the main file.
This resolves all three above issues.

This feature is meant to make the editing of books,
thesis documents and lecture notes somewhat more convenient.
However, the package can also be used efficiently for
composing a series of documents (such as exercise sheets)
which are typically distributed individually.
It then assists the author in generating the individual documents
(potentially in different versions)
as well as a document containing the collected series.
Another application is in developing style files
or other kinds of included material
where compilation of the style file could redirect
to a sample or test file.

%%%%%%%%%%%%%%%%%%%%%%%%%%%%%%%%%%%%%%%%%%%%%%%%%%%%%%%%%%%%%%%%%%%%%%%%%%%%%%%%
%%%%%%%%%%%%%%%%%%%%%%%%%%%%%%%%%%%%%%%%%%%%%%%%%%%%%%%%%%%%%%%%%%%%%%%%%%%%%%%%
\section{Usage}

First of all, the package \textsf{childdoc} is \emph{not} a standard
\LaTeXe{} |.sty| style file! Therefore it needs to be invoked in
a non-standard way.

%%%%%%%%%%%%%%%%%%%%%%%%%%%%%%%%%%%%%%%%%%%%%%%%%%%%%%%%%%%%%%%%%%%%%%%%%%%%%%%%
\subsection{Included Files}
\label{sec:include}

%%%%%%%%%%%%%%%%%%%%%%%%%%%%%%%%%%%%%%%%
\DescribeMacro{\childdocmain}
To use the package, add the commands
\begin{center}
\begin{tabular}{l}
|\input{childdoc.def}|\\
|\childdocmain{}|\\
\end{tabular}
\end{center}
at the very top of the main \LaTeX{} file,
in particular \emph{before} the |\documentclass| statement!
The argument of |\childdocmain| should be left empty
(but it must be present).

%%%%%%%%%%%%%%%%%%%%%%%%%%%%%%%%%%%%%%%%
\DescribeMacro{\childdocof}
Furthermore, add the commands
\begin{center}
\begin{tabular}{l}
|\input{childdoc.def}|\\
|\childdocof{|\textit{main}|}|\\
\end{tabular}
\end{center}
at the top of every child file \textit{child}
which is included by |\include{|\textit{child}|}|
from within the main file
(or at least for those files to be compiled individually).
The argument \textit{main} must be the filename of the main file.

There are a couple of
considerations in setting up the main and child documents:

%%%%%%%%%%%%%%%%%%%%%%%%%%%%%%%%%%%%%%%%
\paragraph{Restrictions.}

Please note the following restrictions:
\begin{itemize}
\item
|\childdocmain| must be called with one argument \textit{main}
to ensure compatibility with earlier version of the package.
It must either be empty (|\childdocmain{}|)
or precisely match the filename of the main file in which it is specified.
See \secref{sec:detection} for further information.
\item
The filename \textit{main} must be specified without the |.tex| extension.
\item
The filename \textit{main} is case sensitive
(even in case-insensitive file systems)
due to internal string comparison.
\item
The argument \textit{main} should be fully expanded, it cannot be a macro.
\item
Subdirectories and special characters should be avoided in filenames.
\item
The command |\childdocmain{|\textit{main}|}| must be followed by a whitespace.
It should not be followed immediately by another command
or by a comment mark `|%|'.
This is because the \TeX{} parser reads the token immediately following
the argument of |\childdocmain| and puts it
at the beginning of every child section;
however, a white\-space is ignored.
\end{itemize}

%%%%%%%%%%%%%%%%%%%%%%%%%%%%%%%%%%%%%%%%
\paragraph{Content of Main File.}

It is advisable to place all content in the child files included by |\include|.
Any output contained in the main file will appear in all child documents
unless suppressed manually;
it cannot be suppressed automatically by the |\includeonly| directive
and thus should normally be avoided.
A method to include some content in the main file
by means of conditional processing is described in \secref{sec:conditional}.

%%%%%%%%%%%%%%%%%%%%%%%%%%%%%%%%%%%%%%%%
\paragraph{Page Numbering.}

When only a part of the document is compiled,
the appropriate numbering of pages
(as well as other status parameters)
is determined from the |.aux| files.
The latter contain information from previous passes.
However this information needs to propagate through
all intermediate child documents.
Therefore the page numbering in child documents may well
be inconsistent until the complete document is compiled at least once.

A useful (if unconventional) way to always ensure a consistent
page numbering is to restart the numbering in each child document
and denote the pages by `\textit{child}|.|\textit{page}'
where \textit{child} represents the chapter/section number of the child file.
This can be achieved by the command
|\numberwithin{page}{|\textit{child}|}|
of the \textsf{amsmath} package
where \textit{child} can be |chapter| or |section|
depending on the chosen structuring.
Alternatively, one can modify the macro |\thepage| appropriately
and reset the counter |page| at the start of each child file.

%%%%%%%%%%%%%%%%%%%%%%%%%%%%%%%%%%%%%%%%%%%%%%%%%%%%%%%%%%%%%%%%%%%%%%%%%%%%%%%%
\subsection{Conditional Processing}
\label{sec:conditional}

The package provides a mechanism to compile different versions
of a document. To customise the versions further some conditional processing
can come in handy to distinguish which version is being compiled.
The package provides two macros to describe the compilation context:

%%%%%%%%%%%%%%%%%%%%%%%%%%%%%%%%%%%%%%%%
\DescribeMacro{\ifchilddoc}
The conditional |\ifchilddoc| distinguishes between the compilation of
child documents and the main document:
%
\begin{center}
|\ifchilddoc |\textit{child-code}| |[|\||else |\textit{main-code}]| \||fi|
\end{center}

%%%%%%%%%%%%%%%%%%%%%%%%%%%%%%%%%%%%%%%%
\DescribeMacro{\childdocname}
\DescribeMacro{\childdocjob}
The macro |\childdocname| contains the filename (without extension)
of the main or child file being processed.
Note that |\childdocjob| will always contain the name of the main file.

%%%%%%%%%%%%%%%%%%%%%%%%%%%%%%%%%%%%%%%%
\paragraph{Title Page.}

Conditional processing can be used to include a title or banner page
in the main document when proper precautions are taken.
Importantly, the code in the main file should ensure that the page counter
(as well as other status parameters which are stored in the |.aux| files)
takes the same value after the conditional processing.
Otherwise the page numbers may take divergent values
depending on which part is compiled.

For example, a title page could be declared by:
%
\begin{center}
\begin{tabular}{l}
|\ifchilddoc\||else|\\
|\addtocounter{page}{-1}|\\
\textit{code for title page}\\
|\newpage|\\
|\||fi|
\end{tabular}
\end{center}
%
A banner page for the child documents can be generated by:
%
\begin{center}
\begin{tabular}{l}
|\ifchilddoc|\\
|\addtocounter{page}{-1}|\\
\textit{code for banner page}\\
|\newpage|\\
|\||fi|
\end{tabular}
\end{center}
%
Here one could write a message such as:
\begin{center}
|This is the part \childdocname{} of \childdocjob{}.|
\end{center}

%%%%%%%%%%%%%%%%%%%%%%%%%%%%%%%%%%%%%%%%%%%%%%%%%%%%%%%%%%%%%%%%%%%%%%%%%%%%%%%%
\subsection{Flags}
\label{sec:flags}

The package makes it easy to generate different versions
of the main or child documents.
To this end compilation flags can be defined
and assigned different default values.
They will be particularly useful in conjunction
with the forwarding mechanism described in \secref{sec:forward}.

For example, it may be useful to have a flag |\version|
which can be set to |draft| or |final|.
The document source will contain some conditional code
depending on the value of |\version|.
Suppose further, the flag should default to |final| for the main file
and to |draft| for child files
which is a natural assignment for editing the document.
This is achieved by placing the following code
in the preamble of the main document
(below the |\childdocmain| directive):
%
\begin{center}
\begin{tabular}{l}
|\ifchilddoc|\\
|\providecommand{\version}{draft}|\\
|\||else|\\
|\providecommand{\version}{final}|\\
|\||fi|
\end{tabular}
\end{center}
%
The definition by |\providecommand| makes sure
that previous definitions are not overwritten.
Further statements |\providecommand{\version}{...}|
can thus be added before the above code to override it.

For the main file, one might add a line
(between |\childdocmain| and the above block)
%
\begin{center}
|%\ifchilddoc\||else\providecommand{\version}{draft}\||fi|
\end{center}
%
which can be uncommented to produce a draft version.
Likewise one can add a line to the very top of a child file
(above the |\childdocof{|\textit{main}|}| directive)
%
\begin{center}
|%\providecommand{\version}{final}|
\end{center}
%
which can be uncommented to produce the final version of this child document.

%%%%%%%%%%%%%%%%%%%%%%%%%%%%%%%%%%%%%%%%%%%%%%%%%%%%%%%%%%%%%%%%%%%%%%%%%%%%%%%%
\subsection{Forwarding}
\label{sec:forward}

Different versions of the main or child documents
using compilation flags as described in \secref{sec:flags}
can be (permanently) stored in different files
for convenient compilation, viewing and distribution.
To this end, the package defines a command
to pass on compilation to a different file:

%%%%%%%%%%%%%%%%%%%%%%%%%%%%%%%%%%%%%%%%
\DescribeMacro{\childdocforward}
The command |\childdocforward| redirects processing to
another source file:
%
\begin{center}
\begin{tabular}{l}
|\input{childdoc.def}|\\
|\childdocforward[|\textit{main}|]{|\textit{dest}|}|\\
\end{tabular}
\end{center}
%
The argument \textit{dest} is the destination file
(without extension).
It should be the main file or one of the child files.
Note that further \textsf{childdoc} directives
such as |\childdocof| and |\childdocforward|
in the indicated file will be processed in this form.
The optional argument \textit{main}
passes on directly to the main file \textit{main}
while pretending to compile the child \textit{dest}.
This form behaves as if \textit{dest}
issues |\childdocof{|\textit{main}|}| right away,
and no further \textsf{childdoc} directives will be processed.

%%%%%%%%%%%%%%%%%%%%%%%%%%%%%%%%%%%%%%%%
\DescribeMacro{\...prefix}
In the alternative form |\childdocforwardprefix|,
%
\begin{center}
\begin{tabular}{l}
|\input{childdoc.def}|\\
|\childdocforwardprefix[|\textit{main}|]{|\textit{prefix}|}{|\textit{dest}|}|
\end{tabular}
\end{center}
%
the destination file is determined by a pattern
depending on the current file:
To make this work, the current file must be called
`{\textit{prefix}\hspace{0.2em}\textit{suffix}}'
with \textit{prefix} matching precisely the argument.
Processing is then passed on to the file
`{\textit{dest}\hspace{0.2em}\textit{suffix}}'.
Surely, the same effect is achieved by
directly specifying the
argument `{\textit{dest}\hspace{0.2em}\textit{suffix}}'
in the first form.
However, that requires to set up a different file
for each child. With the alternative form of the command
all these files can have exactly the same content
which simplifies setting them up and maintaining them.

For example, the following file |draft.tex|
with a compilation flag |\version| as described in \secref{sec:flags}
compiles the main document as a draft:
%
\begin{center}
\begin{tabular}{l}
|\def\version{draft}|\\
|\input{childdoc.def}|\\
|\childdocforward{|\textit{main}|}|
\end{tabular}
\end{center}
%
Likewise, the following files |final|\textit{nn}|.tex|
compile the final version of the child document
|child|\textit{nn}|.tex|:
%
\begin{center}
\begin{tabular}{l}
|\def\version{final}|\\
|\input{childdoc.def}|\\
|\childdocforwardprefix{final}{child}|
\end{tabular}
\end{center}
%

Note that when several versions of a main file and/or of each child file
are to be generated, it may be convenient to set up a |Makefile| or
shell script to automatise the process.

%%%%%%%%%%%%%%%%%%%%%%%%%%%%%%%%%%%%%%%%%%%%%%%%%%%%%%%%%%%%%%%%%%%%%%%%%%%%%%%%
\subsection{Command Line Processing}
\label{sec:commandline}

The effect of redirection files can also be achieved by invoking
the \LaTeX{} compiler with a more elaborate command line.
Most conveniently this should be done as part
of a shell script or a |Makefile|.

When using \textsf{childdoc} in the main file, the following
command lines effectively perform a redirection
(note that depending on the shell being used,
backslashes may have to be doubled: `|\|' $\to$ `|\\|'):
%
\begin{center}
|... -jobname "|\textit{target}|" |\\|"|[\textit{flags}]%
|\input{childdoc.def}\childdocforward[|\textit{main}|]{|\textit{dest}|}"|
\end{center}
%
Here \textit{target} is the name of the output file,
\textit{main} is the name of the main file
and \textit{dest} is the name of the main or child file to be processed
(all filenames without extensions).
The optional argument \textit{main} can be omitted
if \textit{main} matches \textit{dest}.
Optionally, compilation \textit{flags} can be defined via |\def| commands.
This command line makes the \TeX{} engine believe
it is compiling the file \textit{target}
whose content is specified as the latter parameter.
The provided code then forwards the processing to
\textit{main} or \textit{dest} as described in \secref{sec:forward}.

%%%%%%%%%%%%%%%%%%%%%%%%%%%%%%%%%%%%%%%%%%%%%%%%%%%%%%%%%%%%%%%%%%%%%%%%%%%%%%%%
\subsection{Include by Input}
\label{sec:input}

Including child documents by |\include| has some restrictions by design.
Most notably, the content of a child document always occupies
its own set of pages; pages cannot be shared between child documents.
Usually, this behaviour makes perfect sense
because each child document contain an essential part of the document.
However, in some situations it may be desirable to compose
a document from a collection of parts
without having mandatory page breaks between then.
For this case, the package
provides a mechanism to include parts
by |\input| which can also be processed individually.
However, by construction this mechanism
requires manual handling of the content to be output.

%%%%%%%%%%%%%%%%%%%%%%%%%%%%%%%%%%%%%%%%
\DescribeMacro{\ifchilddocmanual}
The main file should be prepared as usual, see \secref{sec:include}.
However, the document body must make a distinction
between processing of an individual part and of the main document, e.g.:
%
\begin{center}
\begin{tabular}{l}
|\ifchilddocmanual|\\
|\input{\childdocname}|\\
|\||else|\\
\textit{document body with }|\input{|\textit{part}|}|\\
|\||fi|
\end{tabular}
\end{center}
%
The conditional |\ifchilddocmanual| is true whenever
a part to be included by |\input| is being compiled,
and the name of the part is stored in |\childdocname|.

%%%%%%%%%%%%%%%%%%%%%%%%%%%%%%%%%%%%%%%%
\DescribeMacro{\childdocby}
Each part to be included by |\input| should start with:
%
\begin{center}
\begin{tabular}{l}
|\input{childdoc.def}|\\
|\childdocby{|\textit{main}|}|\\
\end{tabular}
\end{center}
%
The directive |\childdocby| is similar to |\childdocof|
described in \secref{sec:include},
but the subsequent selection of content must be done manually.
To that end, both |\ifchilddoc| and |\ifchilddocmanual|
will be true upon processing of a part,
and the name of the part is stored in |\childdocname|.
Note that |\jobname| will be set to the filename of the current part
so that each part receives an individual |.aux| file
that does not interfere with the |.aux| file(s) of the main document.
This behaviour can be altered by the alternative form
|\childdocby[*]{|\textit{main}|}| (with a non-empty optional argument)
which uses the |.aux| file of the main document
by setting |\jobname| to \textit{main}.

%%%%%%%%%%%%%%%%%%%%%%%%%%%%%%%%%%%%%%%%%%%%%%%%%%%%%%%%%%%%%%%%%%%%%%%%%%%%%%%%
\subsection{Driver Development}
\label{sec:driver}

The \textsf{childdoc} mechanism can also be use for the development
of definition files such as \LaTeX{} styles or classes.
This case differs from the above setup with multiple parts
included by |\include| in that no |\includeonly| should be invoked.
This can be achieved by starting the include file
(before |\ProvidesPackage|) with:
%
\begin{center}
\begin{tabular}{l}
|\input{childdoc.def}|\\
|\childdocforward{|\textit{main}|}|\\
\end{tabular}
\end{center}
%
or alternatively with:
%
\begin{center}
\begin{tabular}{l}
|\input{childdoc.def}|\\
|\childdocby{|\textit{main}|}|\\
\end{tabular}
\end{center}
%
Both forms have slightly different effects as described above.
The main file is prepared as usual, see \secref{sec:include}.

%%%%%%%%%%%%%%%%%%%%%%%%%%%%%%%%%%%%%%%%%%%%%%%%%%%%%%%%%%%%%%%%%%%%%%%%%%%%%%%%
\subsection{Legacy Detection}
\label{sec:detection}

The directive |\childdocmain| in the main file can detect
whether the complete document or merely a child is to be compiled
even without using the directive |\childdocof|.
This method is deprecated because it is less robust
and there is no compelling reason to use it;
it is merely provided for backward compatibility
and it may be removed in future versions.

If the detection mechanism is to be used,
it is mandatory to correctly specify
the filename of the main file as the argument of |\childdocmain|:
%
\begin{center}
\begin{tabular}{l}
|\input{childdoc.def}|\\
|\childdocmain{|\textit{main}|}|\\
\end{tabular}
\end{center}
%
If |\jobname| does not match the argument \textit{main} of |\childdocmain|,
it is assumed that |\jobname| points to the child file to be compiled.
When using |\childdocmain| with the main file specified as argument,
it suffices to start a child file
with just |\input{|\textit{main}|}|
without loading of the package and using |\childdocof|.
If instead all processing is done
with the appropriate \textsf{childdoc} directives,
the argument of \textit{main} of |\childdocmain| can be empty.

An alternative version of the command line processing described
in \secref{sec:commandline} using the detection mechanism reads:
%
\begin{center}
|... -jobname "|\textit{target}|" "|[\textit{flags}]%
[|\def\jobname{|\textit{dest}|}|]|\input{|\textit{main}|}"|
\end{center}

%%%%%%%%%%%%%%%%%%%%%%%%%%%%%%%%%%%%%%%%%%%%%%%%%%%%%%%%%%%%%%%%%%%%%%%%%%%%%%%%
\subsection{Manual Code}
\label{sec:manual}

In case one cannot be certain whether the definitions file |childdoc.def|
is installed on the target \TeX{} distribution
and one prefers not to ship it,
it is conceivable to paste a few relevant commands into the sources.

To that end, drop all statements |\input{childdoc.def}|
and perform the replacements as outlined below.
Instead of |\childdocmain{|\textit{main}|}| add the following code
to the top of the main file:
%
\begin{center}
\begin{tabular}{l}
|\||ifdefined\childdocname\endinput\||fi\newif\ifchilddoc|\\
|\edef\childdocname{\scantokens\expandafter{\jobname\noexpand}}|\\
|\def\childdocmain{|\textit{main}|}\||ifx\childdocmain\childdocname\||else|\\
|\childdoctrue\includeonly{\childdocname}\let\jobname\childdocmain\||fi|\\
\end{tabular}
\end{center}
%
Instead of |\childdocof{|\textit{main}|}| just include the main file
at the top of each child file:
%
\begin{center}
|\input{|\textit{main}|}|
\end{center}
%
A simple redirection |\childdocforward{|\textit{dest}|}| is achieved by:
%
\begin{center}
|\def\jobname{|\textit{dest}|}\input{\jobname}|
\end{center}
%
The redirection with prefix
|\childdocforwardprefix[|\textit{prefix}|]{|\textit{dest}|}|
is accomplished by:
%
\begin{center}
\begin{tabular}{l}
|{\edef\jobname{\scantokens\expandafter{\jobname\noexpand}}|\\
|\def\redirectjob |\textit{prefix}|#1~~~{\gdef\jobname{|\textit{dest}|#1}}|\\
|\expandafter\redirectjob\jobname~~~}\input{\jobname}|
\end{tabular}
\end{center}

In an alternative approach,
child documents can be compiled by a specific command line
without additional code or specific definitions:
%
\begin{center}
|... -jobname "|\textit{target}|" "|[\textit{flags}]%
|\includeonly{|\textit{dest}|}\input{|\textit{main}|}"|
\end{center}
%

%%%%%%%%%%%%%%%%%%%%%%%%%%%%%%%%%%%%%%%%%%%%%%%%%%%%%%%%%%%%%%%%%%%%%%%%%%%%%%%%
%%%%%%%%%%%%%%%%%%%%%%%%%%%%%%%%%%%%%%%%%%%%%%%%%%%%%%%%%%%%%%%%%%%%%%%%%%%%%%%%
\section{Information}

%%%%%%%%%%%%%%%%%%%%%%%%%%%%%%%%%%%%%%%%%%%%%%%%%%%%%%%%%%%%%%%%%%%%%%%%%%%%%%%%
\subsection{Copyright}

Copyright \copyright{} 2017--2018 Niklas Beisert

This work may be distributed and/or modified under the
conditions of the \LaTeX{} Project Public License, either version 1.3
of this license or (at your option) any later version.
The latest version of this license is in
  \url{http://www.latex-project.org/lppl.txt}
and version 1.3 or later is part of all distributions of \LaTeX{}
version 2005/12/01 or later.

This work has the LPPL maintenance status `maintained'.

The Current Maintainer of this work is Niklas Beisert.

This work consists of the files |README.txt|, |childdoc.ins| and |childdoc.dtx|
as well as the derived files |childdoc.def|, |cdocsamp.tex|
with |cdocsch1.tex|, |cdocsch2.tex|, |cdocspt3.tex|, |cdocspt4.tex|,
|cdocsdrf.tex|, |cdocsfn1.tex|, |cdocsfn2.tex|
as well as |childdoc.pdf|.

%%%%%%%%%%%%%%%%%%%%%%%%%%%%%%%%%%%%%%%%%%%%%%%%%%%%%%%%%%%%%%%%%%%%%%%%%%%%%%%%
\subsection{Files and Installation}

The package consists of the files:
%
\begin{center}
\begin{tabular}{ll}
    |README.txt|   & readme file \\
    |childdoc.ins| & installation file \\
    |childdoc.dtx| & source file \\
    |childdoc.def| & definition file \\
    |cdocsamp.tex| & sample main file \\
    |cdocsch1.tex| & sample include file \\
    |cdocsch2.tex| & sample include file \\
    |cdocspt3.tex| & sample part file \\
    |cdocspt4.tex| & sample part file \\
    |cdocsdrf.tex| & sample redirection file \\
    |cdocsfn1.tex| & sample redirection file \\
    |cdocsfn2.tex| & sample redirection file \\
    |childdoc.pdf| & manual
\end{tabular}
\end{center}
%
The distribution consists of the files
|README.txt|, |childdoc.ins| and |childdoc.dtx|.
%
\begin{itemize}
\item
Run (pdf)\LaTeX{} on |childdoc.dtx|
to compile the manual |childdoc.pdf| (this file).
\item
Run \LaTeX{} on |childdoc.ins| to create the definitions file |childdoc.def|
and the sample |cdocsamp.tex| with include files
|cdocsch1.tex|, |cdocsch2.tex|, |cdocspt3.tex|, |cdocspt4.tex|,
|cdocsdrf.tex|, |cdocsfn1.tex|, |cdocsfn2.tex|.
Then copy the file |childdoc.def| to an appropriate directory of your \LaTeX{}
distribution, e.g.\ \textit{texmf-root}|/tex/latex/childdoc|.
\end{itemize}

%%%%%%%%%%%%%%%%%%%%%%%%%%%%%%%%%%%%%%%%%%%%%%%%%%%%%%%%%%%%%%%%%%%%%%%%%%%%%%%%
\subsection{Related CTAN Packages}

There are several other packages which offer a similar functionality:
%
\begin{itemize}
\item
The packages
\href{http://ctan.org/pkg/docmute}{\textsf{docmute}},
\href{http://ctan.org/pkg/includex}{\textsf{includex}} and
\href{http://ctan.org/pkg/standalone}{\textsf{standalone}}
provide commands to include only the document body of
a child file thus allowing both files to be compiled individually.
\item
The packages \href{http://ctan.org/pkg/subdocs}{\textsf{subdocs}}
and \href{http://ctan.org/pkg/subfiles}{\textsf{subfiles}}
provide structures in which the main and child documents can be
encapsulated and allowing them to be compiled individually.
The inclusion mechanism is different from the conventional |\include|.
\item
The package \href{http://ctan.org/pkg/combine}{\textsf{combine}}
is an elaborate solution to combine several documents into one.
\end{itemize}
%
See also the CTAN topic \href{http://ctan.org/topic/subdocs}{\textsf{subdocs}}
for further related packages.
The present package differs from the above solutions in that
a document structure constructed with the conventional |\include| mechanism
just needs two extra commands at the top of every file
such that all constituent files can be compiled individually.

%%%%%%%%%%%%%%%%%%%%%%%%%%%%%%%%%%%%%%%%%%%%%%%%%%%%%%%%%%%%%%%%%%%%%%%%%%%%%%%%
%\subsection{Feature Suggestions}
%
%The following is a list of features which may be useful for future
%versions of this package:
%%
%\begin{itemize}
%\item
%\ldots
%\end{itemize}

%%%%%%%%%%%%%%%%%%%%%%%%%%%%%%%%%%%%%%%%%%%%%%%%%%%%%%%%%%%%%%%%%%%%%%%%%%%%%%%%
\subsection{Revision History}

%%%%%%%%%%%%%%%%%%%%%%%%%%%%%%%%%%%%%%%%
\paragraph{v2.0:} 2018/12/30

\begin{itemize}
\item
immediate forward processing
\item
added |\childdocby| mechanism
\item
manual restructured
\end{itemize}

%%%%%%%%%%%%%%%%%%%%%%%%%%%%%%%%%%%%%%%%
\paragraph{v1.6:} 2018/01/17

\begin{itemize}
\item
application for development of include files
\item
corrections to manual
\end{itemize}

%%%%%%%%%%%%%%%%%%%%%%%%%%%%%%%%%%%%%%%%
\paragraph{v1.5:} 2017/05/21

\begin{itemize}
\item
more complete structuring introduced
\item
|\childdocof| introduced
\item
|\childdoc| renamed to |\childdocmain|
\item
|\childredirect| renamed to |\childdocforward| and |\childdocforwardprefix|
and functionality expanded
\end{itemize}

%%%%%%%%%%%%%%%%%%%%%%%%%%%%%%%%%%%%%%%%
\paragraph{v1.0:} 2017/04/27

\begin{itemize}
\item
manual and install package
\item
first version published on CTAN
\end{itemize}

%%%%%%%%%%%%%%%%%%%%%%%%%%%%%%%%%%%%%%%%
\paragraph{v0.6:} 2017/04/26

\begin{itemize}
\item
redirection mechanism added
\end{itemize}

%%%%%%%%%%%%%%%%%%%%%%%%%%%%%%%%%%%%%%%%
\paragraph{v0.5:} 2017/04/26

\begin{itemize}
\item
functionality in definition file
\end{itemize}


%%%%%%%%%%%%%%%%%%%%%%%%%%%%%%%%%%%%%%%%%%%%%%%%%%%%%%%%%%%%%%%%%%%%%%%%%%%%%%%%
%%%%%%%%%%%%%%%%%%%%%%%%%%%%%%%%%%%%%%%%%%%%%%%%%%%%%%%%%%%%%%%%%%%%%%%%%%%%%%%%
%%%%%%%%%%%%%%%%%%%%%%%%%%%%%%%%%%%%%%%%%%%%%%%%%%%%%%%%%%%%%%%%%%%%%%%%%%%%%%%%
\appendix

\settowidth\MacroIndent{\rmfamily\scriptsize 000\ }

 \DocInput{childdoc.dtx}

\end{document}
%</driver>
% \fi
%
% %%%%%%%%%%%%%%%%%%%%%%%%%%%%%%%%%%%%%%%%%%%%%%%%%%%%%%%%%%%%%%%%%%%%%%%%%%%%%%
% %%%%%%%%%%%%%%%%%%%%%%%%%%%%%%%%%%%%%%%%%%%%%%%%%%%%%%%%%%%%%%%%%%%%%%%%%%%%%%
% \section{Sample}
%\iffalse
%<*samplemain>
%\fi
%
% The following presents a sample document
% with two chapters, two parts, a title page,
% a compile flag as well as three forwarding files to set the flag.
% It consists of eight |.tex| files:
% \begin{center}
% \begin{tabular}{ll}
% |cdocsamp.tex|&main file\\
% |cdocsch1.tex|&include file for chapter 1\\
% |cdocsch2.tex|&include file for chapter 2\\
% |cdocspt3.tex|&include file for part 3\\
% |cdocspt4.tex|&include file for part 4\\
% |cdocsdrf.tex|&forwarding file for main file in draft mode\\
% |cdocsfi1.tex|&forwarding file for final version of chapter 1\\
% |cdocsfi2.tex|&forwarding file for final version of chapter 2\\
% \end{tabular}
% \end{center}
% Each of the eight files can be compiled directly by the \LaTeX{} compiler.
%
% %%%%%%%%%%%%%%%%%%%%%%%%%%%%%%%%%%%%%%
% \paragraph{Main File.}
%
% The main file is called |cdocsamp.tex|.
%
% Load the \textsf{childdoc} definitions and
% declare the filename for the main document:
%    \begin{macrocode}
\input{childdoc.def}
\childdocmain{}
%    \end{macrocode}

% Optional override for |\version| flag:
%    \begin{macrocode}
%%\ifchilddoc\else\providecommand{\version}{draft}\fi
%    \end{macrocode}

% Define the default values for the |\version| flag
% (|final| for the main file and |draft| for childs):
%    \begin{macrocode}
\ifchilddoc
\providecommand{\version}{draft}
\else
\providecommand{\version}{final}
\fi
%    \end{macrocode}

% Load the standard document class:
%    \begin{macrocode}
\documentclass[12pt]{article}
%    \end{macrocode}

% Start the document body:
%    \begin{macrocode}
\begin{document}
%    \end{macrocode}

% Declare a title page.
% Print title, part of document being processed and version flag:
%    \begin{macrocode}
\addtocounter{page}{-1}
\begin{center}
{\LARGE\bfseries{}childdoc example\par}
\vspace{1cm}
\ifchilddoc
\ifchilddocmanual part\else chapter\fi:
`\childdocname' of `\childdocjob'\par
\else
main document: `\childdocjob'\par
\fi
version: \version\par
\end{center}
\newpage
%    \end{macrocode}

% Manually include selected file,
% otherwise process as usual:
%    \begin{macrocode}
\ifchilddocmanual
\section*{part `\childdocname'}
\input{\childdocname}
\else
%    \end{macrocode}

% Include the two chapters:
%    \begin{macrocode}
\include{cdocsch1}
\include{cdocsch2}
%    \end{macrocode}

% Include the two parts unless only chapters should be displayed:
%    \begin{macrocode}
\ifchilddoc\else
\section{part three}
\input{cdocspt3}
\section{part four}
\input{cdocspt4}
\fi
%    \end{macrocode}

% Process as usual until here:
%    \begin{macrocode}
\fi
%    \end{macrocode}

% End of document body:
%    \begin{macrocode}
\end{document}
%    \end{macrocode}
%\iffalse
%</samplemain>
%\fi
%
% %%%%%%%%%%%%%%%%%%%%%%%%%%%%%%%%%%%%%%
% \paragraph{Chapter Include Files.}
%
% The include files are called |cdocsch1.tex| and |cdocsch2.tex|.
%
%\iffalse
%<*samplechap1|samplechap2>
%\fi

% Optional override for |\version| flag:
%    \begin{macrocode}
%%\providecommand{\version}{final}
%    \end{macrocode}

% Include the main document:
%    \begin{macrocode}
\input{childdoc.def}
\childdocof{cdocsamp}
%    \end{macrocode}

%\iffalse
%</samplechap1|samplechap2>
%\fi
%
%\iffalse
%<*samplechap1>
%\fi
% Some text for chapter 1:
%    \begin{macrocode}
\section{one}
some text in chapter one
%    \end{macrocode}

%\iffalse
%</samplechap1>
%\fi
% Some text for chapter 2:
%\iffalse
%<*samplechap2>
%\fi
%    \begin{macrocode}
\section{two}
more text in chapter two
%    \end{macrocode}

%\iffalse
%</samplechap2>
%\fi
%
% %%%%%%%%%%%%%%%%%%%%%%%%%%%%%%%%%%%%%%
% \paragraph{Part Include Files.}
%
% The include files are called |cdocspt3.tex| and |cdocspt4.tex|.
%
%\iffalse
%<*samplepart3|samplepart4>
%\fi

% Optional override for |\version| flag:
%    \begin{macrocode}
%%\providecommand{\version}{final}
%    \end{macrocode}

% Include the main document:
%    \begin{macrocode}
\input{childdoc.def}
\childdocby{cdocsamp}
%    \end{macrocode}

%\iffalse
%</samplepart3|samplepart4>
%\fi
%
%\iffalse
%<*samplepart3>
%\fi
% Some text for part 3:
%    \begin{macrocode}
some text in part three
%    \end{macrocode}

%\iffalse
%</samplepart3>
%\fi
% Some text for part 4:
%\iffalse
%<*samplepart4>
%\fi
%    \begin{macrocode}
more text in part four
%    \end{macrocode}

%\iffalse
%</samplepart4>
%\fi
%
% %%%%%%%%%%%%%%%%%%%%%%%%%%%%%%%%%%%%%%
% \paragraph{Forwarding for a Complete Draft.}
%
% The following forwarding file |cdocsdrf.tex|
% compiles the main document in draft mode:
%\iffalse
%<*sampledraft>
%\fi
%    \begin{macrocode}
\def\version{draft}
\input{childdoc.def}
\childdocforward{cdocsamp}
%    \end{macrocode}

%\iffalse
%</sampledraft>
%\fi
%
% %%%%%%%%%%%%%%%%%%%%%%%%%%%%%%%%%%%%%%
% \paragraph{Forwarding for Final Version of the Chapters.}
%
% The following forwarding files |cdocsfn1.tex| and |cdocsfn2.tex|
% (with identical content)
% compile the final versions of the child documents
% |cdocsch1.tex| and |cdocsch2.tex|, respectively:
%\iffalse
%<*samplefinal>
%\fi
%    \begin{macrocode}
\def\version{final}
\input{childdoc.def}
\childdocforwardprefix[cdocsamp]{cdocsfn}{cdocsch}
%    \end{macrocode}

%\iffalse
%</samplefinal>
%\fi
%
% %%%%%%%%%%%%%%%%%%%%%%%%%%%%%%%%%%%%%%
% \paragraph{Command Line Processing.}
%
% The following three command lines generate the output files
% |cdocscld|, |cdocscl1| and |cdocscl2|
% which should be identical to
% |cdocsdrf|, |cdocsch1| and |cdocsfn2|, respectively:
% \begin{center}
% \begin{tabular}{l}
% |latex -jobname cdocscld \|\\
% |  "\def\version{draft}\input{childdoc.def}\childdocforward{cdocsamp}"|\\
% |latex -jobname cdocscl1 \|\\
% |  "\input{childdoc.def}\childdocforward[cdocsamp]{cdocsch1}"|\\
% |latex -jobname cdocscl2 \|\\
% |  "\def\version{final}\input{childdoc.def}\childdocforward{cdocsch2}"|
% \end{tabular}
% \end{center}
% Note that the trailing backslash on each first line
% merely continues the input to the second line
% (for convenient cut ant paste).
% Furthermore, the command |latex| can be replaced by any
% of its alternative versions such as |pdflatex|.
%
% %%%%%%%%%%%%%%%%%%%%%%%%%%%%%%%%%%%%%%%%%%%%%%%%%%%%%%%%%%%%%%%%%%%%%%%%%%%%%%
% %%%%%%%%%%%%%%%%%%%%%%%%%%%%%%%%%%%%%%%%%%%%%%%%%%%%%%%%%%%%%%%%%%%%%%%%%%%%%%
% \section{Implementation}
%\iffalse
%<*package>
%\fi
%
% This section describes the definitions file |childdoc.def|.

% The definitions cannot be loaded using |\usepackage| or |\RequirePackage|
% which has a mechanism to prevent loading a style file more than once.
% When loading the definitions by means of |\input|
% multiple instances have to be prevented manually:
%\iffalse
%This code needs to be before the `\ProvidesFile' directive
%which is defined at the beginning of this file.
%Therefore it is also placed there and commented out here.
%</package>
%<*discard>
%\fi
%    \begin{macrocode}
\ifdefined\childdocmain\endinput\fi
%    \end{macrocode}
%\iffalse
%</discard>
%<*package>
%\fi
%
% \macro{\ifchilddoc}
% \macro{\ifchilddocmanual}
% The conditional |\ifchilddoc| tells whether a
% child (true) or main (false) document is being compiled.
% The conditional |\ifchilddocmanual| tells whether
% the |\includeonly| mechanism is used (false) or
% the selection of child files must be performed manually (true).
% The definitions initialise to false:
%    \begin{macrocode}
\newif\ifchilddoc
\newif\ifchilddocmanual
%    \end{macrocode}

% \macro{\childdocname}
% \macro{\childdocjob}
% The macro |\childdocname| stores the name of the main document
% to be compiled. The macro |\childdocjob| stores the name of
% the document on which the \LaTeX{} compiler was originally invoked.
% The content of |\jobname| cannot be compared
% to filenames specified in the source due to different catcodes.
% The following code rescans |\jobname|, stores the result
% in |\childdocname| and saves a copy in |\childdocjob|:
%    \begin{macrocode}
\edef\childdocname{\scantokens\expandafter{\jobname\noexpand}}
\let\childdocjob\childdocname
%    \end{macrocode}

% \macro{\childdocdisable}
% The macro |\childdocdisable| prevents the main file
% from being processed more than once.
% At this stage, the main document command |\childdocmain|
% is assumed to be called once again where it should do nothing.
% Any subsequent call to it should prevent
% a secondary processing of the main document
% It overwrites the forwarding commands
% |\childdocof| and |\childdocforward|
% with empty macros to prevent further inclusions of the main document:
%    \begin{macrocode}
\newcommand{\childdocdisable}
{
  \renewcommand{\childdocmain}[1]{\renewcommand{\childdocmain}[1]{\endinput}}
  \renewcommand{\childdocof}[1]{}
  \renewcommand{\childdocby}[2][]{}
  \renewcommand{\childdocforward}[2][]{}
  \renewcommand{\childdocdisable}{}
}
%    \end{macrocode}

% \macro{\childdocmain}
% The macro |\childdocmain| is to be called at the top of the main file
% with nothing or the main filename (without extension) as argument.
% First, it breaks loops.
% If the argument is not empty and does not match |\childdocname|
% (which is set by the first inclusion of |childdoc.def|),
% |\ifchilddoc| is set to true, |\includeonly| is applied to the child file
% and |\jobname| is set to the main file
% (for proper handling of |.aux| files):
%    \begin{macrocode}
\newcommand{\childdocmain}[1]
{
  \childdocdisable\childdocmain{}
  \if?#1?\else
    \begingroup
      \def\childdoctmp{#1}
      \ifx\childdoctmp\childdocname
        \def\childdoctmp{}
      \else
        \def\childdoctmp
        {
          \childdoctrue
          \includeonly{\childdocname}
          \def\childdocjob{#1}
          \def\jobname{#1}
        }
      \fi
      \expandafter
    \endgroup
    \childdoctmp
  \fi
}
%    \end{macrocode}

% \macro{\childdocof}
% The command |\childdocof| redirects
% compilation to the main file |#1|.
%    \begin{macrocode}
\newcommand{\childdocof}[1]
{
  \childdocdisable
  \childdoctrue
  \includeonly{\childdocname}
  \def\jobname{#1}
  \def\childdocjob{#1}
  \input{#1}
}
%    \end{macrocode}

% \macro{\childdocby}
% The command |\childdocby| ....
%    \begin{macrocode}
\newcommand{\childdocby}[2][]
{
  \childdocdisable
  \childdoctrue
  \childdocmanualtrue
  \if?#1?\else
    \def\jobname{#2}
  \fi
  \def\childdocjob{#2}
  \input{#2}
  \endinput
}
%    \end{macrocode}

% \macro{\childdocforward}
% The command |\childdocforward| redirects
% compilation to the main file or
% (if the optional argument is given) a child file.
% Parameters are set as if the main file
% or a child file starting with |\childdocof| was compiled.
% Then compilation is handed over to the main file:
%    \begin{macrocode}
\newcommand{\childdocforward}[2][]
{
  \begingroup
    \if?#1?
      \def\childdoctmp
      {
        \def\childdocname{#2}
        \def\childdocjob{#2}
        \def\jobname{#2}
        \input{#2}
        \endinput
      }
    \else
      \def\childdoctmp
      {
        \childdocdisable
        \def\childdocname{#2}
        \childdoctrue
        \includeonly{#2}
        \def\childdocjob{#1}
        \def\jobname{#1}
        \input{#1}
        \endinput
      }
    \fi
    \expandafter
  \endgroup
  \childdoctmp
}
%    \end{macrocode}

% \macro{\childdocforwardprefix}
% The command |\childdocforwardprefix| redirects
% compilation to the main or a child file by means of a pattern.
% The prefix |#1| in the current filename is replaced by |#2|
% and the suffix of the current filename is kept
% (it is assumed that the filename does not contain the substring `|~~~|'
% which is used as a delimiter).
% Compilation is handed over to the new file by |\childdocforward|:
%    \begin{macrocode}
\newcommand{\childdocforwardprefix}[3][]
{
  \begingroup
    \def\childdocextract #2##1~~~{\def\childdoctmp{\childdocforward[#1]{#3##1}}}
    \expandafter\childdocextract\childdocname~~~
    \expandafter
  \endgroup
  \childdoctmp
}
%    \end{macrocode}

% \macro{\childdoc}
% The deprecated macro |\childdoc| is a legacy version of |\childdocmain|:
%    \begin{macrocode}
\newcommand{\childdoc}{\childdocmain}
%    \end{macrocode}

% \macro{\childdocredirect}
% The deprecated macro |\childdocredirect| is a legacy version
% of |\childdocforward| and |\childdocforwardprefix|:
%    \begin{macrocode}
\newcommand{\childdocredirect}[2][]
{
  \begingroup
    \if?#1?
      \def\childdoctmp{\childdocforward{#2}}
    \else
      \def\childdoctmp{\childdocforwardprefix{#1}{#2}}
    \fi
    \expandafter
  \endgroup
  \childdoctmp
}
%    \end{macrocode}

%\iffalse
%</package>
%\fi
%
\endinput
|\\
|\childdocforwardprefix{final}{child}|
\end{tabular}
\end{center}
%

Note that when several versions of a main file and/or of each child file
are to be generated, it may be convenient to set up a |Makefile| or
shell script to automatise the process.

%%%%%%%%%%%%%%%%%%%%%%%%%%%%%%%%%%%%%%%%%%%%%%%%%%%%%%%%%%%%%%%%%%%%%%%%%%%%%%%%
\subsection{Command Line Processing}
\label{sec:commandline}

The effect of redirection files can also be achieved by invoking
the \LaTeX{} compiler with a more elaborate command line.
Most conveniently this should be done as part
of a shell script or a |Makefile|.

When using \textsf{childdoc} in the main file, the following
command lines effectively perform a redirection
(note that depending on the shell being used,
backslashes may have to be doubled: `|\|' $\to$ `|\\|'):
%
\begin{center}
|... -jobname "|\textit{target}|" |\\|"|[\textit{flags}]%
|% \iffalse
%
% childdoc.dtx Copyright (C) 2017-2018 Niklas Beisert
%
% This work may be distributed and/or modified under the
% conditions of the LaTeX Project Public License, either version 1.3
% of this license or (at your option) any later version.
% The latest version of this license is in
%   http://www.latex-project.org/lppl.txt
% and version 1.3 or later is part of all distributions of LaTeX
% version 2005/12/01 or later.
%
% This work has the LPPL maintenance status `maintained'.
%
% The Current Maintainer of this work is Niklas Beisert.
%
% This work consists of the files childdoc.dtx and childdoc.ins
% and the derived files childdoc.def and cdocsamp.tex with
% cdocsch1.tex, cdocsch2.tex, cdocsdrf.tex, cdocsfn1.tex, cdocsfn2.tex.
%
%<package>\ifdefined\childdocmain\endinput\fi
%<package>\ProvidesFile{childdoc.def}[2018/12/30 v2.0 child document driver]
%<samplemain>\ProvidesFile{cdocsamp.tex}[2018/12/30 v2.0 sample for childdoc]
%<*driver>
%\ProvidesFile{childdoc.drv}[2018/12/30 v2.0 childdoc reference manual file]
\PassOptionsToClass{10pt,a4paper}{article}
\documentclass{ltxdoc}

\usepackage[margin=35mm]{geometry}
\usepackage{hyperref}
\usepackage{hyperxmp}
\usepackage[usenames]{color}

\hypersetup{colorlinks=true}
\hypersetup{pdfstartview=FitH}
\hypersetup{pdfpagemode=UseNone}
\hypersetup{pdfsource={}}
\hypersetup{pdflang={en-UK}}
\hypersetup{pdfcopyright={Copyright 2017-2018 Niklas Beisert.
  This work may be distributed and/or modified under the
  conditions of the LaTeX Project Public License, either version 1.3
  of this license or (at your option) any later version.}}
\hypersetup{pdflicenseurl={http://www.latex-project.org/lppl.txt}}
\hypersetup{pdfcontactaddress={ETH Zurich, ITP, HIT K,
  Wolfgang-Pauli-Strasse 27}}
\hypersetup{pdfcontactpostcode={8093}}
\hypersetup{pdfcontactcity={Zurich}}
\hypersetup{pdfcontactcountry={Switzerland}}
\hypersetup{pdfcontactemail={nbeisert@itp.phys.ethz.ch}}
\hypersetup{pdfcontacturl={http://people.phys.ethz.ch/\xmptilde nbeisert/}}

\newcommand{\secref}[1]{\hyperref[#1]{section \ref*{#1}}}

\parskip1ex
\parindent0pt
\let\olditemize\itemize
\def\itemize{\olditemize\parskip0pt}

\begin{document}

\title{The \textsf{childdoc} Package}
\hypersetup{pdftitle={The childdoc Package}}
\author{Niklas Beisert\\[2ex]
  Institut f\"ur Theoretische Physik\\
  Eidgen\"ossische Technische Hochschule Z\"urich\\
  Wolfgang-Pauli-Strasse 27, 8093 Z\"urich, Switzerland\\[1ex]
  \href{mailto:nbeisert@itp.phys.ethz.ch}
  {\texttt{nbeisert@itp.phys.ethz.ch}}}
\hypersetup{pdfauthor={Niklas Beisert}}
\hypersetup{pdfsubject={Manual for the LaTeX2e Package childdoc}}
\date{30 December 2018, \textsf{v2.0}}
\maketitle

\begin{abstract}\noindent
\textsf{childdoc} is a \LaTeXe{} package
that enables the direct compilation
of document sections included by |\include|
to individual files.
\end{abstract}

\begingroup
\parskip0ex
\tableofcontents
\endgroup

%%%%%%%%%%%%%%%%%%%%%%%%%%%%%%%%%%%%%%%%%%%%%%%%%%%%%%%%%%%%%%%%%%%%%%%%%%%%%%%%
%%%%%%%%%%%%%%%%%%%%%%%%%%%%%%%%%%%%%%%%%%%%%%%%%%%%%%%%%%%%%%%%%%%%%%%%%%%%%%%%
\section{Introduction}

\LaTeX{} provides a mechanism to structure a large document (such as a book)
into a main file and several child files (containing the chapters)
using the |\include| command.
This mechanism is beneficial for documents
which span hundreds of pages in order to
make the source file(s) more manageable.
Moreover, compilation can be restricted to
selected child files by means of the |\includeonly| command.
The latter feature can be used to reduce the compilation time while editing
(this was significantly more useful in the earlier days of \LaTeX{})
or to generate a smaller document which is easier to navigate.
Another application of |\includeonly| is to generate
documents consisting of selected parts of the complete document.

However, there are a few drawbacks of the plain |\include| mechanism:
\begin{itemize}
\item
The child files cannot be compiled on their own,
they can only be compiled via the main file.
A naive editing environment
(such as a text editor with an option
to have the current file processed by \LaTeX)
may require one to switch to the main file before compiling;
attempting to compile the child file produces errors.
\item
The main file must be modified (each time)
to adjust the |\includeonly| command
to the present needs. This easily leaves the main file in a messy state.
\item
The generated document will always carry the filename
of the main document. This is inconvenient if
several child files are to be compiled and
to be kept for distribution.
\end{itemize}

The present package provides a simple interface
to make child files individually compilable by \LaTeX{}.
Compiling a child file then has the same effect as compiling
the main file with an |\includeonly| command
to select the appropriate child.
Moreover the generated document will carry the name of the child
rather than the main file.
This resolves all three above issues.

This feature is meant to make the editing of books,
thesis documents and lecture notes somewhat more convenient.
However, the package can also be used efficiently for
composing a series of documents (such as exercise sheets)
which are typically distributed individually.
It then assists the author in generating the individual documents
(potentially in different versions)
as well as a document containing the collected series.
Another application is in developing style files
or other kinds of included material
where compilation of the style file could redirect
to a sample or test file.

%%%%%%%%%%%%%%%%%%%%%%%%%%%%%%%%%%%%%%%%%%%%%%%%%%%%%%%%%%%%%%%%%%%%%%%%%%%%%%%%
%%%%%%%%%%%%%%%%%%%%%%%%%%%%%%%%%%%%%%%%%%%%%%%%%%%%%%%%%%%%%%%%%%%%%%%%%%%%%%%%
\section{Usage}

First of all, the package \textsf{childdoc} is \emph{not} a standard
\LaTeXe{} |.sty| style file! Therefore it needs to be invoked in
a non-standard way.

%%%%%%%%%%%%%%%%%%%%%%%%%%%%%%%%%%%%%%%%%%%%%%%%%%%%%%%%%%%%%%%%%%%%%%%%%%%%%%%%
\subsection{Included Files}
\label{sec:include}

%%%%%%%%%%%%%%%%%%%%%%%%%%%%%%%%%%%%%%%%
\DescribeMacro{\childdocmain}
To use the package, add the commands
\begin{center}
\begin{tabular}{l}
|\input{childdoc.def}|\\
|\childdocmain{}|\\
\end{tabular}
\end{center}
at the very top of the main \LaTeX{} file,
in particular \emph{before} the |\documentclass| statement!
The argument of |\childdocmain| should be left empty
(but it must be present).

%%%%%%%%%%%%%%%%%%%%%%%%%%%%%%%%%%%%%%%%
\DescribeMacro{\childdocof}
Furthermore, add the commands
\begin{center}
\begin{tabular}{l}
|\input{childdoc.def}|\\
|\childdocof{|\textit{main}|}|\\
\end{tabular}
\end{center}
at the top of every child file \textit{child}
which is included by |\include{|\textit{child}|}|
from within the main file
(or at least for those files to be compiled individually).
The argument \textit{main} must be the filename of the main file.

There are a couple of
considerations in setting up the main and child documents:

%%%%%%%%%%%%%%%%%%%%%%%%%%%%%%%%%%%%%%%%
\paragraph{Restrictions.}

Please note the following restrictions:
\begin{itemize}
\item
|\childdocmain| must be called with one argument \textit{main}
to ensure compatibility with earlier version of the package.
It must either be empty (|\childdocmain{}|)
or precisely match the filename of the main file in which it is specified.
See \secref{sec:detection} for further information.
\item
The filename \textit{main} must be specified without the |.tex| extension.
\item
The filename \textit{main} is case sensitive
(even in case-insensitive file systems)
due to internal string comparison.
\item
The argument \textit{main} should be fully expanded, it cannot be a macro.
\item
Subdirectories and special characters should be avoided in filenames.
\item
The command |\childdocmain{|\textit{main}|}| must be followed by a whitespace.
It should not be followed immediately by another command
or by a comment mark `|%|'.
This is because the \TeX{} parser reads the token immediately following
the argument of |\childdocmain| and puts it
at the beginning of every child section;
however, a white\-space is ignored.
\end{itemize}

%%%%%%%%%%%%%%%%%%%%%%%%%%%%%%%%%%%%%%%%
\paragraph{Content of Main File.}

It is advisable to place all content in the child files included by |\include|.
Any output contained in the main file will appear in all child documents
unless suppressed manually;
it cannot be suppressed automatically by the |\includeonly| directive
and thus should normally be avoided.
A method to include some content in the main file
by means of conditional processing is described in \secref{sec:conditional}.

%%%%%%%%%%%%%%%%%%%%%%%%%%%%%%%%%%%%%%%%
\paragraph{Page Numbering.}

When only a part of the document is compiled,
the appropriate numbering of pages
(as well as other status parameters)
is determined from the |.aux| files.
The latter contain information from previous passes.
However this information needs to propagate through
all intermediate child documents.
Therefore the page numbering in child documents may well
be inconsistent until the complete document is compiled at least once.

A useful (if unconventional) way to always ensure a consistent
page numbering is to restart the numbering in each child document
and denote the pages by `\textit{child}|.|\textit{page}'
where \textit{child} represents the chapter/section number of the child file.
This can be achieved by the command
|\numberwithin{page}{|\textit{child}|}|
of the \textsf{amsmath} package
where \textit{child} can be |chapter| or |section|
depending on the chosen structuring.
Alternatively, one can modify the macro |\thepage| appropriately
and reset the counter |page| at the start of each child file.

%%%%%%%%%%%%%%%%%%%%%%%%%%%%%%%%%%%%%%%%%%%%%%%%%%%%%%%%%%%%%%%%%%%%%%%%%%%%%%%%
\subsection{Conditional Processing}
\label{sec:conditional}

The package provides a mechanism to compile different versions
of a document. To customise the versions further some conditional processing
can come in handy to distinguish which version is being compiled.
The package provides two macros to describe the compilation context:

%%%%%%%%%%%%%%%%%%%%%%%%%%%%%%%%%%%%%%%%
\DescribeMacro{\ifchilddoc}
The conditional |\ifchilddoc| distinguishes between the compilation of
child documents and the main document:
%
\begin{center}
|\ifchilddoc |\textit{child-code}| |[|\||else |\textit{main-code}]| \||fi|
\end{center}

%%%%%%%%%%%%%%%%%%%%%%%%%%%%%%%%%%%%%%%%
\DescribeMacro{\childdocname}
\DescribeMacro{\childdocjob}
The macro |\childdocname| contains the filename (without extension)
of the main or child file being processed.
Note that |\childdocjob| will always contain the name of the main file.

%%%%%%%%%%%%%%%%%%%%%%%%%%%%%%%%%%%%%%%%
\paragraph{Title Page.}

Conditional processing can be used to include a title or banner page
in the main document when proper precautions are taken.
Importantly, the code in the main file should ensure that the page counter
(as well as other status parameters which are stored in the |.aux| files)
takes the same value after the conditional processing.
Otherwise the page numbers may take divergent values
depending on which part is compiled.

For example, a title page could be declared by:
%
\begin{center}
\begin{tabular}{l}
|\ifchilddoc\||else|\\
|\addtocounter{page}{-1}|\\
\textit{code for title page}\\
|\newpage|\\
|\||fi|
\end{tabular}
\end{center}
%
A banner page for the child documents can be generated by:
%
\begin{center}
\begin{tabular}{l}
|\ifchilddoc|\\
|\addtocounter{page}{-1}|\\
\textit{code for banner page}\\
|\newpage|\\
|\||fi|
\end{tabular}
\end{center}
%
Here one could write a message such as:
\begin{center}
|This is the part \childdocname{} of \childdocjob{}.|
\end{center}

%%%%%%%%%%%%%%%%%%%%%%%%%%%%%%%%%%%%%%%%%%%%%%%%%%%%%%%%%%%%%%%%%%%%%%%%%%%%%%%%
\subsection{Flags}
\label{sec:flags}

The package makes it easy to generate different versions
of the main or child documents.
To this end compilation flags can be defined
and assigned different default values.
They will be particularly useful in conjunction
with the forwarding mechanism described in \secref{sec:forward}.

For example, it may be useful to have a flag |\version|
which can be set to |draft| or |final|.
The document source will contain some conditional code
depending on the value of |\version|.
Suppose further, the flag should default to |final| for the main file
and to |draft| for child files
which is a natural assignment for editing the document.
This is achieved by placing the following code
in the preamble of the main document
(below the |\childdocmain| directive):
%
\begin{center}
\begin{tabular}{l}
|\ifchilddoc|\\
|\providecommand{\version}{draft}|\\
|\||else|\\
|\providecommand{\version}{final}|\\
|\||fi|
\end{tabular}
\end{center}
%
The definition by |\providecommand| makes sure
that previous definitions are not overwritten.
Further statements |\providecommand{\version}{...}|
can thus be added before the above code to override it.

For the main file, one might add a line
(between |\childdocmain| and the above block)
%
\begin{center}
|%\ifchilddoc\||else\providecommand{\version}{draft}\||fi|
\end{center}
%
which can be uncommented to produce a draft version.
Likewise one can add a line to the very top of a child file
(above the |\childdocof{|\textit{main}|}| directive)
%
\begin{center}
|%\providecommand{\version}{final}|
\end{center}
%
which can be uncommented to produce the final version of this child document.

%%%%%%%%%%%%%%%%%%%%%%%%%%%%%%%%%%%%%%%%%%%%%%%%%%%%%%%%%%%%%%%%%%%%%%%%%%%%%%%%
\subsection{Forwarding}
\label{sec:forward}

Different versions of the main or child documents
using compilation flags as described in \secref{sec:flags}
can be (permanently) stored in different files
for convenient compilation, viewing and distribution.
To this end, the package defines a command
to pass on compilation to a different file:

%%%%%%%%%%%%%%%%%%%%%%%%%%%%%%%%%%%%%%%%
\DescribeMacro{\childdocforward}
The command |\childdocforward| redirects processing to
another source file:
%
\begin{center}
\begin{tabular}{l}
|\input{childdoc.def}|\\
|\childdocforward[|\textit{main}|]{|\textit{dest}|}|\\
\end{tabular}
\end{center}
%
The argument \textit{dest} is the destination file
(without extension).
It should be the main file or one of the child files.
Note that further \textsf{childdoc} directives
such as |\childdocof| and |\childdocforward|
in the indicated file will be processed in this form.
The optional argument \textit{main}
passes on directly to the main file \textit{main}
while pretending to compile the child \textit{dest}.
This form behaves as if \textit{dest}
issues |\childdocof{|\textit{main}|}| right away,
and no further \textsf{childdoc} directives will be processed.

%%%%%%%%%%%%%%%%%%%%%%%%%%%%%%%%%%%%%%%%
\DescribeMacro{\...prefix}
In the alternative form |\childdocforwardprefix|,
%
\begin{center}
\begin{tabular}{l}
|\input{childdoc.def}|\\
|\childdocforwardprefix[|\textit{main}|]{|\textit{prefix}|}{|\textit{dest}|}|
\end{tabular}
\end{center}
%
the destination file is determined by a pattern
depending on the current file:
To make this work, the current file must be called
`{\textit{prefix}\hspace{0.2em}\textit{suffix}}'
with \textit{prefix} matching precisely the argument.
Processing is then passed on to the file
`{\textit{dest}\hspace{0.2em}\textit{suffix}}'.
Surely, the same effect is achieved by
directly specifying the
argument `{\textit{dest}\hspace{0.2em}\textit{suffix}}'
in the first form.
However, that requires to set up a different file
for each child. With the alternative form of the command
all these files can have exactly the same content
which simplifies setting them up and maintaining them.

For example, the following file |draft.tex|
with a compilation flag |\version| as described in \secref{sec:flags}
compiles the main document as a draft:
%
\begin{center}
\begin{tabular}{l}
|\def\version{draft}|\\
|\input{childdoc.def}|\\
|\childdocforward{|\textit{main}|}|
\end{tabular}
\end{center}
%
Likewise, the following files |final|\textit{nn}|.tex|
compile the final version of the child document
|child|\textit{nn}|.tex|:
%
\begin{center}
\begin{tabular}{l}
|\def\version{final}|\\
|\input{childdoc.def}|\\
|\childdocforwardprefix{final}{child}|
\end{tabular}
\end{center}
%

Note that when several versions of a main file and/or of each child file
are to be generated, it may be convenient to set up a |Makefile| or
shell script to automatise the process.

%%%%%%%%%%%%%%%%%%%%%%%%%%%%%%%%%%%%%%%%%%%%%%%%%%%%%%%%%%%%%%%%%%%%%%%%%%%%%%%%
\subsection{Command Line Processing}
\label{sec:commandline}

The effect of redirection files can also be achieved by invoking
the \LaTeX{} compiler with a more elaborate command line.
Most conveniently this should be done as part
of a shell script or a |Makefile|.

When using \textsf{childdoc} in the main file, the following
command lines effectively perform a redirection
(note that depending on the shell being used,
backslashes may have to be doubled: `|\|' $\to$ `|\\|'):
%
\begin{center}
|... -jobname "|\textit{target}|" |\\|"|[\textit{flags}]%
|\input{childdoc.def}\childdocforward[|\textit{main}|]{|\textit{dest}|}"|
\end{center}
%
Here \textit{target} is the name of the output file,
\textit{main} is the name of the main file
and \textit{dest} is the name of the main or child file to be processed
(all filenames without extensions).
The optional argument \textit{main} can be omitted
if \textit{main} matches \textit{dest}.
Optionally, compilation \textit{flags} can be defined via |\def| commands.
This command line makes the \TeX{} engine believe
it is compiling the file \textit{target}
whose content is specified as the latter parameter.
The provided code then forwards the processing to
\textit{main} or \textit{dest} as described in \secref{sec:forward}.

%%%%%%%%%%%%%%%%%%%%%%%%%%%%%%%%%%%%%%%%%%%%%%%%%%%%%%%%%%%%%%%%%%%%%%%%%%%%%%%%
\subsection{Include by Input}
\label{sec:input}

Including child documents by |\include| has some restrictions by design.
Most notably, the content of a child document always occupies
its own set of pages; pages cannot be shared between child documents.
Usually, this behaviour makes perfect sense
because each child document contain an essential part of the document.
However, in some situations it may be desirable to compose
a document from a collection of parts
without having mandatory page breaks between then.
For this case, the package
provides a mechanism to include parts
by |\input| which can also be processed individually.
However, by construction this mechanism
requires manual handling of the content to be output.

%%%%%%%%%%%%%%%%%%%%%%%%%%%%%%%%%%%%%%%%
\DescribeMacro{\ifchilddocmanual}
The main file should be prepared as usual, see \secref{sec:include}.
However, the document body must make a distinction
between processing of an individual part and of the main document, e.g.:
%
\begin{center}
\begin{tabular}{l}
|\ifchilddocmanual|\\
|\input{\childdocname}|\\
|\||else|\\
\textit{document body with }|\input{|\textit{part}|}|\\
|\||fi|
\end{tabular}
\end{center}
%
The conditional |\ifchilddocmanual| is true whenever
a part to be included by |\input| is being compiled,
and the name of the part is stored in |\childdocname|.

%%%%%%%%%%%%%%%%%%%%%%%%%%%%%%%%%%%%%%%%
\DescribeMacro{\childdocby}
Each part to be included by |\input| should start with:
%
\begin{center}
\begin{tabular}{l}
|\input{childdoc.def}|\\
|\childdocby{|\textit{main}|}|\\
\end{tabular}
\end{center}
%
The directive |\childdocby| is similar to |\childdocof|
described in \secref{sec:include},
but the subsequent selection of content must be done manually.
To that end, both |\ifchilddoc| and |\ifchilddocmanual|
will be true upon processing of a part,
and the name of the part is stored in |\childdocname|.
Note that |\jobname| will be set to the filename of the current part
so that each part receives an individual |.aux| file
that does not interfere with the |.aux| file(s) of the main document.
This behaviour can be altered by the alternative form
|\childdocby[*]{|\textit{main}|}| (with a non-empty optional argument)
which uses the |.aux| file of the main document
by setting |\jobname| to \textit{main}.

%%%%%%%%%%%%%%%%%%%%%%%%%%%%%%%%%%%%%%%%%%%%%%%%%%%%%%%%%%%%%%%%%%%%%%%%%%%%%%%%
\subsection{Driver Development}
\label{sec:driver}

The \textsf{childdoc} mechanism can also be use for the development
of definition files such as \LaTeX{} styles or classes.
This case differs from the above setup with multiple parts
included by |\include| in that no |\includeonly| should be invoked.
This can be achieved by starting the include file
(before |\ProvidesPackage|) with:
%
\begin{center}
\begin{tabular}{l}
|\input{childdoc.def}|\\
|\childdocforward{|\textit{main}|}|\\
\end{tabular}
\end{center}
%
or alternatively with:
%
\begin{center}
\begin{tabular}{l}
|\input{childdoc.def}|\\
|\childdocby{|\textit{main}|}|\\
\end{tabular}
\end{center}
%
Both forms have slightly different effects as described above.
The main file is prepared as usual, see \secref{sec:include}.

%%%%%%%%%%%%%%%%%%%%%%%%%%%%%%%%%%%%%%%%%%%%%%%%%%%%%%%%%%%%%%%%%%%%%%%%%%%%%%%%
\subsection{Legacy Detection}
\label{sec:detection}

The directive |\childdocmain| in the main file can detect
whether the complete document or merely a child is to be compiled
even without using the directive |\childdocof|.
This method is deprecated because it is less robust
and there is no compelling reason to use it;
it is merely provided for backward compatibility
and it may be removed in future versions.

If the detection mechanism is to be used,
it is mandatory to correctly specify
the filename of the main file as the argument of |\childdocmain|:
%
\begin{center}
\begin{tabular}{l}
|\input{childdoc.def}|\\
|\childdocmain{|\textit{main}|}|\\
\end{tabular}
\end{center}
%
If |\jobname| does not match the argument \textit{main} of |\childdocmain|,
it is assumed that |\jobname| points to the child file to be compiled.
When using |\childdocmain| with the main file specified as argument,
it suffices to start a child file
with just |\input{|\textit{main}|}|
without loading of the package and using |\childdocof|.
If instead all processing is done
with the appropriate \textsf{childdoc} directives,
the argument of \textit{main} of |\childdocmain| can be empty.

An alternative version of the command line processing described
in \secref{sec:commandline} using the detection mechanism reads:
%
\begin{center}
|... -jobname "|\textit{target}|" "|[\textit{flags}]%
[|\def\jobname{|\textit{dest}|}|]|\input{|\textit{main}|}"|
\end{center}

%%%%%%%%%%%%%%%%%%%%%%%%%%%%%%%%%%%%%%%%%%%%%%%%%%%%%%%%%%%%%%%%%%%%%%%%%%%%%%%%
\subsection{Manual Code}
\label{sec:manual}

In case one cannot be certain whether the definitions file |childdoc.def|
is installed on the target \TeX{} distribution
and one prefers not to ship it,
it is conceivable to paste a few relevant commands into the sources.

To that end, drop all statements |\input{childdoc.def}|
and perform the replacements as outlined below.
Instead of |\childdocmain{|\textit{main}|}| add the following code
to the top of the main file:
%
\begin{center}
\begin{tabular}{l}
|\||ifdefined\childdocname\endinput\||fi\newif\ifchilddoc|\\
|\edef\childdocname{\scantokens\expandafter{\jobname\noexpand}}|\\
|\def\childdocmain{|\textit{main}|}\||ifx\childdocmain\childdocname\||else|\\
|\childdoctrue\includeonly{\childdocname}\let\jobname\childdocmain\||fi|\\
\end{tabular}
\end{center}
%
Instead of |\childdocof{|\textit{main}|}| just include the main file
at the top of each child file:
%
\begin{center}
|\input{|\textit{main}|}|
\end{center}
%
A simple redirection |\childdocforward{|\textit{dest}|}| is achieved by:
%
\begin{center}
|\def\jobname{|\textit{dest}|}\input{\jobname}|
\end{center}
%
The redirection with prefix
|\childdocforwardprefix[|\textit{prefix}|]{|\textit{dest}|}|
is accomplished by:
%
\begin{center}
\begin{tabular}{l}
|{\edef\jobname{\scantokens\expandafter{\jobname\noexpand}}|\\
|\def\redirectjob |\textit{prefix}|#1~~~{\gdef\jobname{|\textit{dest}|#1}}|\\
|\expandafter\redirectjob\jobname~~~}\input{\jobname}|
\end{tabular}
\end{center}

In an alternative approach,
child documents can be compiled by a specific command line
without additional code or specific definitions:
%
\begin{center}
|... -jobname "|\textit{target}|" "|[\textit{flags}]%
|\includeonly{|\textit{dest}|}\input{|\textit{main}|}"|
\end{center}
%

%%%%%%%%%%%%%%%%%%%%%%%%%%%%%%%%%%%%%%%%%%%%%%%%%%%%%%%%%%%%%%%%%%%%%%%%%%%%%%%%
%%%%%%%%%%%%%%%%%%%%%%%%%%%%%%%%%%%%%%%%%%%%%%%%%%%%%%%%%%%%%%%%%%%%%%%%%%%%%%%%
\section{Information}

%%%%%%%%%%%%%%%%%%%%%%%%%%%%%%%%%%%%%%%%%%%%%%%%%%%%%%%%%%%%%%%%%%%%%%%%%%%%%%%%
\subsection{Copyright}

Copyright \copyright{} 2017--2018 Niklas Beisert

This work may be distributed and/or modified under the
conditions of the \LaTeX{} Project Public License, either version 1.3
of this license or (at your option) any later version.
The latest version of this license is in
  \url{http://www.latex-project.org/lppl.txt}
and version 1.3 or later is part of all distributions of \LaTeX{}
version 2005/12/01 or later.

This work has the LPPL maintenance status `maintained'.

The Current Maintainer of this work is Niklas Beisert.

This work consists of the files |README.txt|, |childdoc.ins| and |childdoc.dtx|
as well as the derived files |childdoc.def|, |cdocsamp.tex|
with |cdocsch1.tex|, |cdocsch2.tex|, |cdocspt3.tex|, |cdocspt4.tex|,
|cdocsdrf.tex|, |cdocsfn1.tex|, |cdocsfn2.tex|
as well as |childdoc.pdf|.

%%%%%%%%%%%%%%%%%%%%%%%%%%%%%%%%%%%%%%%%%%%%%%%%%%%%%%%%%%%%%%%%%%%%%%%%%%%%%%%%
\subsection{Files and Installation}

The package consists of the files:
%
\begin{center}
\begin{tabular}{ll}
    |README.txt|   & readme file \\
    |childdoc.ins| & installation file \\
    |childdoc.dtx| & source file \\
    |childdoc.def| & definition file \\
    |cdocsamp.tex| & sample main file \\
    |cdocsch1.tex| & sample include file \\
    |cdocsch2.tex| & sample include file \\
    |cdocspt3.tex| & sample part file \\
    |cdocspt4.tex| & sample part file \\
    |cdocsdrf.tex| & sample redirection file \\
    |cdocsfn1.tex| & sample redirection file \\
    |cdocsfn2.tex| & sample redirection file \\
    |childdoc.pdf| & manual
\end{tabular}
\end{center}
%
The distribution consists of the files
|README.txt|, |childdoc.ins| and |childdoc.dtx|.
%
\begin{itemize}
\item
Run (pdf)\LaTeX{} on |childdoc.dtx|
to compile the manual |childdoc.pdf| (this file).
\item
Run \LaTeX{} on |childdoc.ins| to create the definitions file |childdoc.def|
and the sample |cdocsamp.tex| with include files
|cdocsch1.tex|, |cdocsch2.tex|, |cdocspt3.tex|, |cdocspt4.tex|,
|cdocsdrf.tex|, |cdocsfn1.tex|, |cdocsfn2.tex|.
Then copy the file |childdoc.def| to an appropriate directory of your \LaTeX{}
distribution, e.g.\ \textit{texmf-root}|/tex/latex/childdoc|.
\end{itemize}

%%%%%%%%%%%%%%%%%%%%%%%%%%%%%%%%%%%%%%%%%%%%%%%%%%%%%%%%%%%%%%%%%%%%%%%%%%%%%%%%
\subsection{Related CTAN Packages}

There are several other packages which offer a similar functionality:
%
\begin{itemize}
\item
The packages
\href{http://ctan.org/pkg/docmute}{\textsf{docmute}},
\href{http://ctan.org/pkg/includex}{\textsf{includex}} and
\href{http://ctan.org/pkg/standalone}{\textsf{standalone}}
provide commands to include only the document body of
a child file thus allowing both files to be compiled individually.
\item
The packages \href{http://ctan.org/pkg/subdocs}{\textsf{subdocs}}
and \href{http://ctan.org/pkg/subfiles}{\textsf{subfiles}}
provide structures in which the main and child documents can be
encapsulated and allowing them to be compiled individually.
The inclusion mechanism is different from the conventional |\include|.
\item
The package \href{http://ctan.org/pkg/combine}{\textsf{combine}}
is an elaborate solution to combine several documents into one.
\end{itemize}
%
See also the CTAN topic \href{http://ctan.org/topic/subdocs}{\textsf{subdocs}}
for further related packages.
The present package differs from the above solutions in that
a document structure constructed with the conventional |\include| mechanism
just needs two extra commands at the top of every file
such that all constituent files can be compiled individually.

%%%%%%%%%%%%%%%%%%%%%%%%%%%%%%%%%%%%%%%%%%%%%%%%%%%%%%%%%%%%%%%%%%%%%%%%%%%%%%%%
%\subsection{Feature Suggestions}
%
%The following is a list of features which may be useful for future
%versions of this package:
%%
%\begin{itemize}
%\item
%\ldots
%\end{itemize}

%%%%%%%%%%%%%%%%%%%%%%%%%%%%%%%%%%%%%%%%%%%%%%%%%%%%%%%%%%%%%%%%%%%%%%%%%%%%%%%%
\subsection{Revision History}

%%%%%%%%%%%%%%%%%%%%%%%%%%%%%%%%%%%%%%%%
\paragraph{v2.0:} 2018/12/30

\begin{itemize}
\item
immediate forward processing
\item
added |\childdocby| mechanism
\item
manual restructured
\end{itemize}

%%%%%%%%%%%%%%%%%%%%%%%%%%%%%%%%%%%%%%%%
\paragraph{v1.6:} 2018/01/17

\begin{itemize}
\item
application for development of include files
\item
corrections to manual
\end{itemize}

%%%%%%%%%%%%%%%%%%%%%%%%%%%%%%%%%%%%%%%%
\paragraph{v1.5:} 2017/05/21

\begin{itemize}
\item
more complete structuring introduced
\item
|\childdocof| introduced
\item
|\childdoc| renamed to |\childdocmain|
\item
|\childredirect| renamed to |\childdocforward| and |\childdocforwardprefix|
and functionality expanded
\end{itemize}

%%%%%%%%%%%%%%%%%%%%%%%%%%%%%%%%%%%%%%%%
\paragraph{v1.0:} 2017/04/27

\begin{itemize}
\item
manual and install package
\item
first version published on CTAN
\end{itemize}

%%%%%%%%%%%%%%%%%%%%%%%%%%%%%%%%%%%%%%%%
\paragraph{v0.6:} 2017/04/26

\begin{itemize}
\item
redirection mechanism added
\end{itemize}

%%%%%%%%%%%%%%%%%%%%%%%%%%%%%%%%%%%%%%%%
\paragraph{v0.5:} 2017/04/26

\begin{itemize}
\item
functionality in definition file
\end{itemize}


%%%%%%%%%%%%%%%%%%%%%%%%%%%%%%%%%%%%%%%%%%%%%%%%%%%%%%%%%%%%%%%%%%%%%%%%%%%%%%%%
%%%%%%%%%%%%%%%%%%%%%%%%%%%%%%%%%%%%%%%%%%%%%%%%%%%%%%%%%%%%%%%%%%%%%%%%%%%%%%%%
%%%%%%%%%%%%%%%%%%%%%%%%%%%%%%%%%%%%%%%%%%%%%%%%%%%%%%%%%%%%%%%%%%%%%%%%%%%%%%%%
\appendix

\settowidth\MacroIndent{\rmfamily\scriptsize 000\ }

 \DocInput{childdoc.dtx}

\end{document}
%</driver>
% \fi
%
% %%%%%%%%%%%%%%%%%%%%%%%%%%%%%%%%%%%%%%%%%%%%%%%%%%%%%%%%%%%%%%%%%%%%%%%%%%%%%%
% %%%%%%%%%%%%%%%%%%%%%%%%%%%%%%%%%%%%%%%%%%%%%%%%%%%%%%%%%%%%%%%%%%%%%%%%%%%%%%
% \section{Sample}
%\iffalse
%<*samplemain>
%\fi
%
% The following presents a sample document
% with two chapters, two parts, a title page,
% a compile flag as well as three forwarding files to set the flag.
% It consists of eight |.tex| files:
% \begin{center}
% \begin{tabular}{ll}
% |cdocsamp.tex|&main file\\
% |cdocsch1.tex|&include file for chapter 1\\
% |cdocsch2.tex|&include file for chapter 2\\
% |cdocspt3.tex|&include file for part 3\\
% |cdocspt4.tex|&include file for part 4\\
% |cdocsdrf.tex|&forwarding file for main file in draft mode\\
% |cdocsfi1.tex|&forwarding file for final version of chapter 1\\
% |cdocsfi2.tex|&forwarding file for final version of chapter 2\\
% \end{tabular}
% \end{center}
% Each of the eight files can be compiled directly by the \LaTeX{} compiler.
%
% %%%%%%%%%%%%%%%%%%%%%%%%%%%%%%%%%%%%%%
% \paragraph{Main File.}
%
% The main file is called |cdocsamp.tex|.
%
% Load the \textsf{childdoc} definitions and
% declare the filename for the main document:
%    \begin{macrocode}
\input{childdoc.def}
\childdocmain{}
%    \end{macrocode}

% Optional override for |\version| flag:
%    \begin{macrocode}
%%\ifchilddoc\else\providecommand{\version}{draft}\fi
%    \end{macrocode}

% Define the default values for the |\version| flag
% (|final| for the main file and |draft| for childs):
%    \begin{macrocode}
\ifchilddoc
\providecommand{\version}{draft}
\else
\providecommand{\version}{final}
\fi
%    \end{macrocode}

% Load the standard document class:
%    \begin{macrocode}
\documentclass[12pt]{article}
%    \end{macrocode}

% Start the document body:
%    \begin{macrocode}
\begin{document}
%    \end{macrocode}

% Declare a title page.
% Print title, part of document being processed and version flag:
%    \begin{macrocode}
\addtocounter{page}{-1}
\begin{center}
{\LARGE\bfseries{}childdoc example\par}
\vspace{1cm}
\ifchilddoc
\ifchilddocmanual part\else chapter\fi:
`\childdocname' of `\childdocjob'\par
\else
main document: `\childdocjob'\par
\fi
version: \version\par
\end{center}
\newpage
%    \end{macrocode}

% Manually include selected file,
% otherwise process as usual:
%    \begin{macrocode}
\ifchilddocmanual
\section*{part `\childdocname'}
\input{\childdocname}
\else
%    \end{macrocode}

% Include the two chapters:
%    \begin{macrocode}
\include{cdocsch1}
\include{cdocsch2}
%    \end{macrocode}

% Include the two parts unless only chapters should be displayed:
%    \begin{macrocode}
\ifchilddoc\else
\section{part three}
\input{cdocspt3}
\section{part four}
\input{cdocspt4}
\fi
%    \end{macrocode}

% Process as usual until here:
%    \begin{macrocode}
\fi
%    \end{macrocode}

% End of document body:
%    \begin{macrocode}
\end{document}
%    \end{macrocode}
%\iffalse
%</samplemain>
%\fi
%
% %%%%%%%%%%%%%%%%%%%%%%%%%%%%%%%%%%%%%%
% \paragraph{Chapter Include Files.}
%
% The include files are called |cdocsch1.tex| and |cdocsch2.tex|.
%
%\iffalse
%<*samplechap1|samplechap2>
%\fi

% Optional override for |\version| flag:
%    \begin{macrocode}
%%\providecommand{\version}{final}
%    \end{macrocode}

% Include the main document:
%    \begin{macrocode}
\input{childdoc.def}
\childdocof{cdocsamp}
%    \end{macrocode}

%\iffalse
%</samplechap1|samplechap2>
%\fi
%
%\iffalse
%<*samplechap1>
%\fi
% Some text for chapter 1:
%    \begin{macrocode}
\section{one}
some text in chapter one
%    \end{macrocode}

%\iffalse
%</samplechap1>
%\fi
% Some text for chapter 2:
%\iffalse
%<*samplechap2>
%\fi
%    \begin{macrocode}
\section{two}
more text in chapter two
%    \end{macrocode}

%\iffalse
%</samplechap2>
%\fi
%
% %%%%%%%%%%%%%%%%%%%%%%%%%%%%%%%%%%%%%%
% \paragraph{Part Include Files.}
%
% The include files are called |cdocspt3.tex| and |cdocspt4.tex|.
%
%\iffalse
%<*samplepart3|samplepart4>
%\fi

% Optional override for |\version| flag:
%    \begin{macrocode}
%%\providecommand{\version}{final}
%    \end{macrocode}

% Include the main document:
%    \begin{macrocode}
\input{childdoc.def}
\childdocby{cdocsamp}
%    \end{macrocode}

%\iffalse
%</samplepart3|samplepart4>
%\fi
%
%\iffalse
%<*samplepart3>
%\fi
% Some text for part 3:
%    \begin{macrocode}
some text in part three
%    \end{macrocode}

%\iffalse
%</samplepart3>
%\fi
% Some text for part 4:
%\iffalse
%<*samplepart4>
%\fi
%    \begin{macrocode}
more text in part four
%    \end{macrocode}

%\iffalse
%</samplepart4>
%\fi
%
% %%%%%%%%%%%%%%%%%%%%%%%%%%%%%%%%%%%%%%
% \paragraph{Forwarding for a Complete Draft.}
%
% The following forwarding file |cdocsdrf.tex|
% compiles the main document in draft mode:
%\iffalse
%<*sampledraft>
%\fi
%    \begin{macrocode}
\def\version{draft}
\input{childdoc.def}
\childdocforward{cdocsamp}
%    \end{macrocode}

%\iffalse
%</sampledraft>
%\fi
%
% %%%%%%%%%%%%%%%%%%%%%%%%%%%%%%%%%%%%%%
% \paragraph{Forwarding for Final Version of the Chapters.}
%
% The following forwarding files |cdocsfn1.tex| and |cdocsfn2.tex|
% (with identical content)
% compile the final versions of the child documents
% |cdocsch1.tex| and |cdocsch2.tex|, respectively:
%\iffalse
%<*samplefinal>
%\fi
%    \begin{macrocode}
\def\version{final}
\input{childdoc.def}
\childdocforwardprefix[cdocsamp]{cdocsfn}{cdocsch}
%    \end{macrocode}

%\iffalse
%</samplefinal>
%\fi
%
% %%%%%%%%%%%%%%%%%%%%%%%%%%%%%%%%%%%%%%
% \paragraph{Command Line Processing.}
%
% The following three command lines generate the output files
% |cdocscld|, |cdocscl1| and |cdocscl2|
% which should be identical to
% |cdocsdrf|, |cdocsch1| and |cdocsfn2|, respectively:
% \begin{center}
% \begin{tabular}{l}
% |latex -jobname cdocscld \|\\
% |  "\def\version{draft}\input{childdoc.def}\childdocforward{cdocsamp}"|\\
% |latex -jobname cdocscl1 \|\\
% |  "\input{childdoc.def}\childdocforward[cdocsamp]{cdocsch1}"|\\
% |latex -jobname cdocscl2 \|\\
% |  "\def\version{final}\input{childdoc.def}\childdocforward{cdocsch2}"|
% \end{tabular}
% \end{center}
% Note that the trailing backslash on each first line
% merely continues the input to the second line
% (for convenient cut ant paste).
% Furthermore, the command |latex| can be replaced by any
% of its alternative versions such as |pdflatex|.
%
% %%%%%%%%%%%%%%%%%%%%%%%%%%%%%%%%%%%%%%%%%%%%%%%%%%%%%%%%%%%%%%%%%%%%%%%%%%%%%%
% %%%%%%%%%%%%%%%%%%%%%%%%%%%%%%%%%%%%%%%%%%%%%%%%%%%%%%%%%%%%%%%%%%%%%%%%%%%%%%
% \section{Implementation}
%\iffalse
%<*package>
%\fi
%
% This section describes the definitions file |childdoc.def|.

% The definitions cannot be loaded using |\usepackage| or |\RequirePackage|
% which has a mechanism to prevent loading a style file more than once.
% When loading the definitions by means of |\input|
% multiple instances have to be prevented manually:
%\iffalse
%This code needs to be before the `\ProvidesFile' directive
%which is defined at the beginning of this file.
%Therefore it is also placed there and commented out here.
%</package>
%<*discard>
%\fi
%    \begin{macrocode}
\ifdefined\childdocmain\endinput\fi
%    \end{macrocode}
%\iffalse
%</discard>
%<*package>
%\fi
%
% \macro{\ifchilddoc}
% \macro{\ifchilddocmanual}
% The conditional |\ifchilddoc| tells whether a
% child (true) or main (false) document is being compiled.
% The conditional |\ifchilddocmanual| tells whether
% the |\includeonly| mechanism is used (false) or
% the selection of child files must be performed manually (true).
% The definitions initialise to false:
%    \begin{macrocode}
\newif\ifchilddoc
\newif\ifchilddocmanual
%    \end{macrocode}

% \macro{\childdocname}
% \macro{\childdocjob}
% The macro |\childdocname| stores the name of the main document
% to be compiled. The macro |\childdocjob| stores the name of
% the document on which the \LaTeX{} compiler was originally invoked.
% The content of |\jobname| cannot be compared
% to filenames specified in the source due to different catcodes.
% The following code rescans |\jobname|, stores the result
% in |\childdocname| and saves a copy in |\childdocjob|:
%    \begin{macrocode}
\edef\childdocname{\scantokens\expandafter{\jobname\noexpand}}
\let\childdocjob\childdocname
%    \end{macrocode}

% \macro{\childdocdisable}
% The macro |\childdocdisable| prevents the main file
% from being processed more than once.
% At this stage, the main document command |\childdocmain|
% is assumed to be called once again where it should do nothing.
% Any subsequent call to it should prevent
% a secondary processing of the main document
% It overwrites the forwarding commands
% |\childdocof| and |\childdocforward|
% with empty macros to prevent further inclusions of the main document:
%    \begin{macrocode}
\newcommand{\childdocdisable}
{
  \renewcommand{\childdocmain}[1]{\renewcommand{\childdocmain}[1]{\endinput}}
  \renewcommand{\childdocof}[1]{}
  \renewcommand{\childdocby}[2][]{}
  \renewcommand{\childdocforward}[2][]{}
  \renewcommand{\childdocdisable}{}
}
%    \end{macrocode}

% \macro{\childdocmain}
% The macro |\childdocmain| is to be called at the top of the main file
% with nothing or the main filename (without extension) as argument.
% First, it breaks loops.
% If the argument is not empty and does not match |\childdocname|
% (which is set by the first inclusion of |childdoc.def|),
% |\ifchilddoc| is set to true, |\includeonly| is applied to the child file
% and |\jobname| is set to the main file
% (for proper handling of |.aux| files):
%    \begin{macrocode}
\newcommand{\childdocmain}[1]
{
  \childdocdisable\childdocmain{}
  \if?#1?\else
    \begingroup
      \def\childdoctmp{#1}
      \ifx\childdoctmp\childdocname
        \def\childdoctmp{}
      \else
        \def\childdoctmp
        {
          \childdoctrue
          \includeonly{\childdocname}
          \def\childdocjob{#1}
          \def\jobname{#1}
        }
      \fi
      \expandafter
    \endgroup
    \childdoctmp
  \fi
}
%    \end{macrocode}

% \macro{\childdocof}
% The command |\childdocof| redirects
% compilation to the main file |#1|.
%    \begin{macrocode}
\newcommand{\childdocof}[1]
{
  \childdocdisable
  \childdoctrue
  \includeonly{\childdocname}
  \def\jobname{#1}
  \def\childdocjob{#1}
  \input{#1}
}
%    \end{macrocode}

% \macro{\childdocby}
% The command |\childdocby| ....
%    \begin{macrocode}
\newcommand{\childdocby}[2][]
{
  \childdocdisable
  \childdoctrue
  \childdocmanualtrue
  \if?#1?\else
    \def\jobname{#2}
  \fi
  \def\childdocjob{#2}
  \input{#2}
  \endinput
}
%    \end{macrocode}

% \macro{\childdocforward}
% The command |\childdocforward| redirects
% compilation to the main file or
% (if the optional argument is given) a child file.
% Parameters are set as if the main file
% or a child file starting with |\childdocof| was compiled.
% Then compilation is handed over to the main file:
%    \begin{macrocode}
\newcommand{\childdocforward}[2][]
{
  \begingroup
    \if?#1?
      \def\childdoctmp
      {
        \def\childdocname{#2}
        \def\childdocjob{#2}
        \def\jobname{#2}
        \input{#2}
        \endinput
      }
    \else
      \def\childdoctmp
      {
        \childdocdisable
        \def\childdocname{#2}
        \childdoctrue
        \includeonly{#2}
        \def\childdocjob{#1}
        \def\jobname{#1}
        \input{#1}
        \endinput
      }
    \fi
    \expandafter
  \endgroup
  \childdoctmp
}
%    \end{macrocode}

% \macro{\childdocforwardprefix}
% The command |\childdocforwardprefix| redirects
% compilation to the main or a child file by means of a pattern.
% The prefix |#1| in the current filename is replaced by |#2|
% and the suffix of the current filename is kept
% (it is assumed that the filename does not contain the substring `|~~~|'
% which is used as a delimiter).
% Compilation is handed over to the new file by |\childdocforward|:
%    \begin{macrocode}
\newcommand{\childdocforwardprefix}[3][]
{
  \begingroup
    \def\childdocextract #2##1~~~{\def\childdoctmp{\childdocforward[#1]{#3##1}}}
    \expandafter\childdocextract\childdocname~~~
    \expandafter
  \endgroup
  \childdoctmp
}
%    \end{macrocode}

% \macro{\childdoc}
% The deprecated macro |\childdoc| is a legacy version of |\childdocmain|:
%    \begin{macrocode}
\newcommand{\childdoc}{\childdocmain}
%    \end{macrocode}

% \macro{\childdocredirect}
% The deprecated macro |\childdocredirect| is a legacy version
% of |\childdocforward| and |\childdocforwardprefix|:
%    \begin{macrocode}
\newcommand{\childdocredirect}[2][]
{
  \begingroup
    \if?#1?
      \def\childdoctmp{\childdocforward{#2}}
    \else
      \def\childdoctmp{\childdocforwardprefix{#1}{#2}}
    \fi
    \expandafter
  \endgroup
  \childdoctmp
}
%    \end{macrocode}

%\iffalse
%</package>
%\fi
%
\endinput
\childdocforward[|\textit{main}|]{|\textit{dest}|}"|
\end{center}
%
Here \textit{target} is the name of the output file,
\textit{main} is the name of the main file
and \textit{dest} is the name of the main or child file to be processed
(all filenames without extensions).
The optional argument \textit{main} can be omitted
if \textit{main} matches \textit{dest}.
Optionally, compilation \textit{flags} can be defined via |\def| commands.
This command line makes the \TeX{} engine believe
it is compiling the file \textit{target}
whose content is specified as the latter parameter.
The provided code then forwards the processing to
\textit{main} or \textit{dest} as described in \secref{sec:forward}.

%%%%%%%%%%%%%%%%%%%%%%%%%%%%%%%%%%%%%%%%%%%%%%%%%%%%%%%%%%%%%%%%%%%%%%%%%%%%%%%%
\subsection{Include by Input}
\label{sec:input}

Including child documents by |\include| has some restrictions by design.
Most notably, the content of a child document always occupies
its own set of pages; pages cannot be shared between child documents.
Usually, this behaviour makes perfect sense
because each child document contain an essential part of the document.
However, in some situations it may be desirable to compose
a document from a collection of parts
without having mandatory page breaks between then.
For this case, the package
provides a mechanism to include parts
by |\input| which can also be processed individually.
However, by construction this mechanism
requires manual handling of the content to be output.

%%%%%%%%%%%%%%%%%%%%%%%%%%%%%%%%%%%%%%%%
\DescribeMacro{\ifchilddocmanual}
The main file should be prepared as usual, see \secref{sec:include}.
However, the document body must make a distinction
between processing of an individual part and of the main document, e.g.:
%
\begin{center}
\begin{tabular}{l}
|\ifchilddocmanual|\\
|\input{\childdocname}|\\
|\||else|\\
\textit{document body with }|\input{|\textit{part}|}|\\
|\||fi|
\end{tabular}
\end{center}
%
The conditional |\ifchilddocmanual| is true whenever
a part to be included by |\input| is being compiled,
and the name of the part is stored in |\childdocname|.

%%%%%%%%%%%%%%%%%%%%%%%%%%%%%%%%%%%%%%%%
\DescribeMacro{\childdocby}
Each part to be included by |\input| should start with:
%
\begin{center}
\begin{tabular}{l}
|% \iffalse
%
% childdoc.dtx Copyright (C) 2017-2018 Niklas Beisert
%
% This work may be distributed and/or modified under the
% conditions of the LaTeX Project Public License, either version 1.3
% of this license or (at your option) any later version.
% The latest version of this license is in
%   http://www.latex-project.org/lppl.txt
% and version 1.3 or later is part of all distributions of LaTeX
% version 2005/12/01 or later.
%
% This work has the LPPL maintenance status `maintained'.
%
% The Current Maintainer of this work is Niklas Beisert.
%
% This work consists of the files childdoc.dtx and childdoc.ins
% and the derived files childdoc.def and cdocsamp.tex with
% cdocsch1.tex, cdocsch2.tex, cdocsdrf.tex, cdocsfn1.tex, cdocsfn2.tex.
%
%<package>\ifdefined\childdocmain\endinput\fi
%<package>\ProvidesFile{childdoc.def}[2018/12/30 v2.0 child document driver]
%<samplemain>\ProvidesFile{cdocsamp.tex}[2018/12/30 v2.0 sample for childdoc]
%<*driver>
%\ProvidesFile{childdoc.drv}[2018/12/30 v2.0 childdoc reference manual file]
\PassOptionsToClass{10pt,a4paper}{article}
\documentclass{ltxdoc}

\usepackage[margin=35mm]{geometry}
\usepackage{hyperref}
\usepackage{hyperxmp}
\usepackage[usenames]{color}

\hypersetup{colorlinks=true}
\hypersetup{pdfstartview=FitH}
\hypersetup{pdfpagemode=UseNone}
\hypersetup{pdfsource={}}
\hypersetup{pdflang={en-UK}}
\hypersetup{pdfcopyright={Copyright 2017-2018 Niklas Beisert.
  This work may be distributed and/or modified under the
  conditions of the LaTeX Project Public License, either version 1.3
  of this license or (at your option) any later version.}}
\hypersetup{pdflicenseurl={http://www.latex-project.org/lppl.txt}}
\hypersetup{pdfcontactaddress={ETH Zurich, ITP, HIT K,
  Wolfgang-Pauli-Strasse 27}}
\hypersetup{pdfcontactpostcode={8093}}
\hypersetup{pdfcontactcity={Zurich}}
\hypersetup{pdfcontactcountry={Switzerland}}
\hypersetup{pdfcontactemail={nbeisert@itp.phys.ethz.ch}}
\hypersetup{pdfcontacturl={http://people.phys.ethz.ch/\xmptilde nbeisert/}}

\newcommand{\secref}[1]{\hyperref[#1]{section \ref*{#1}}}

\parskip1ex
\parindent0pt
\let\olditemize\itemize
\def\itemize{\olditemize\parskip0pt}

\begin{document}

\title{The \textsf{childdoc} Package}
\hypersetup{pdftitle={The childdoc Package}}
\author{Niklas Beisert\\[2ex]
  Institut f\"ur Theoretische Physik\\
  Eidgen\"ossische Technische Hochschule Z\"urich\\
  Wolfgang-Pauli-Strasse 27, 8093 Z\"urich, Switzerland\\[1ex]
  \href{mailto:nbeisert@itp.phys.ethz.ch}
  {\texttt{nbeisert@itp.phys.ethz.ch}}}
\hypersetup{pdfauthor={Niklas Beisert}}
\hypersetup{pdfsubject={Manual for the LaTeX2e Package childdoc}}
\date{30 December 2018, \textsf{v2.0}}
\maketitle

\begin{abstract}\noindent
\textsf{childdoc} is a \LaTeXe{} package
that enables the direct compilation
of document sections included by |\include|
to individual files.
\end{abstract}

\begingroup
\parskip0ex
\tableofcontents
\endgroup

%%%%%%%%%%%%%%%%%%%%%%%%%%%%%%%%%%%%%%%%%%%%%%%%%%%%%%%%%%%%%%%%%%%%%%%%%%%%%%%%
%%%%%%%%%%%%%%%%%%%%%%%%%%%%%%%%%%%%%%%%%%%%%%%%%%%%%%%%%%%%%%%%%%%%%%%%%%%%%%%%
\section{Introduction}

\LaTeX{} provides a mechanism to structure a large document (such as a book)
into a main file and several child files (containing the chapters)
using the |\include| command.
This mechanism is beneficial for documents
which span hundreds of pages in order to
make the source file(s) more manageable.
Moreover, compilation can be restricted to
selected child files by means of the |\includeonly| command.
The latter feature can be used to reduce the compilation time while editing
(this was significantly more useful in the earlier days of \LaTeX{})
or to generate a smaller document which is easier to navigate.
Another application of |\includeonly| is to generate
documents consisting of selected parts of the complete document.

However, there are a few drawbacks of the plain |\include| mechanism:
\begin{itemize}
\item
The child files cannot be compiled on their own,
they can only be compiled via the main file.
A naive editing environment
(such as a text editor with an option
to have the current file processed by \LaTeX)
may require one to switch to the main file before compiling;
attempting to compile the child file produces errors.
\item
The main file must be modified (each time)
to adjust the |\includeonly| command
to the present needs. This easily leaves the main file in a messy state.
\item
The generated document will always carry the filename
of the main document. This is inconvenient if
several child files are to be compiled and
to be kept for distribution.
\end{itemize}

The present package provides a simple interface
to make child files individually compilable by \LaTeX{}.
Compiling a child file then has the same effect as compiling
the main file with an |\includeonly| command
to select the appropriate child.
Moreover the generated document will carry the name of the child
rather than the main file.
This resolves all three above issues.

This feature is meant to make the editing of books,
thesis documents and lecture notes somewhat more convenient.
However, the package can also be used efficiently for
composing a series of documents (such as exercise sheets)
which are typically distributed individually.
It then assists the author in generating the individual documents
(potentially in different versions)
as well as a document containing the collected series.
Another application is in developing style files
or other kinds of included material
where compilation of the style file could redirect
to a sample or test file.

%%%%%%%%%%%%%%%%%%%%%%%%%%%%%%%%%%%%%%%%%%%%%%%%%%%%%%%%%%%%%%%%%%%%%%%%%%%%%%%%
%%%%%%%%%%%%%%%%%%%%%%%%%%%%%%%%%%%%%%%%%%%%%%%%%%%%%%%%%%%%%%%%%%%%%%%%%%%%%%%%
\section{Usage}

First of all, the package \textsf{childdoc} is \emph{not} a standard
\LaTeXe{} |.sty| style file! Therefore it needs to be invoked in
a non-standard way.

%%%%%%%%%%%%%%%%%%%%%%%%%%%%%%%%%%%%%%%%%%%%%%%%%%%%%%%%%%%%%%%%%%%%%%%%%%%%%%%%
\subsection{Included Files}
\label{sec:include}

%%%%%%%%%%%%%%%%%%%%%%%%%%%%%%%%%%%%%%%%
\DescribeMacro{\childdocmain}
To use the package, add the commands
\begin{center}
\begin{tabular}{l}
|\input{childdoc.def}|\\
|\childdocmain{}|\\
\end{tabular}
\end{center}
at the very top of the main \LaTeX{} file,
in particular \emph{before} the |\documentclass| statement!
The argument of |\childdocmain| should be left empty
(but it must be present).

%%%%%%%%%%%%%%%%%%%%%%%%%%%%%%%%%%%%%%%%
\DescribeMacro{\childdocof}
Furthermore, add the commands
\begin{center}
\begin{tabular}{l}
|\input{childdoc.def}|\\
|\childdocof{|\textit{main}|}|\\
\end{tabular}
\end{center}
at the top of every child file \textit{child}
which is included by |\include{|\textit{child}|}|
from within the main file
(or at least for those files to be compiled individually).
The argument \textit{main} must be the filename of the main file.

There are a couple of
considerations in setting up the main and child documents:

%%%%%%%%%%%%%%%%%%%%%%%%%%%%%%%%%%%%%%%%
\paragraph{Restrictions.}

Please note the following restrictions:
\begin{itemize}
\item
|\childdocmain| must be called with one argument \textit{main}
to ensure compatibility with earlier version of the package.
It must either be empty (|\childdocmain{}|)
or precisely match the filename of the main file in which it is specified.
See \secref{sec:detection} for further information.
\item
The filename \textit{main} must be specified without the |.tex| extension.
\item
The filename \textit{main} is case sensitive
(even in case-insensitive file systems)
due to internal string comparison.
\item
The argument \textit{main} should be fully expanded, it cannot be a macro.
\item
Subdirectories and special characters should be avoided in filenames.
\item
The command |\childdocmain{|\textit{main}|}| must be followed by a whitespace.
It should not be followed immediately by another command
or by a comment mark `|%|'.
This is because the \TeX{} parser reads the token immediately following
the argument of |\childdocmain| and puts it
at the beginning of every child section;
however, a white\-space is ignored.
\end{itemize}

%%%%%%%%%%%%%%%%%%%%%%%%%%%%%%%%%%%%%%%%
\paragraph{Content of Main File.}

It is advisable to place all content in the child files included by |\include|.
Any output contained in the main file will appear in all child documents
unless suppressed manually;
it cannot be suppressed automatically by the |\includeonly| directive
and thus should normally be avoided.
A method to include some content in the main file
by means of conditional processing is described in \secref{sec:conditional}.

%%%%%%%%%%%%%%%%%%%%%%%%%%%%%%%%%%%%%%%%
\paragraph{Page Numbering.}

When only a part of the document is compiled,
the appropriate numbering of pages
(as well as other status parameters)
is determined from the |.aux| files.
The latter contain information from previous passes.
However this information needs to propagate through
all intermediate child documents.
Therefore the page numbering in child documents may well
be inconsistent until the complete document is compiled at least once.

A useful (if unconventional) way to always ensure a consistent
page numbering is to restart the numbering in each child document
and denote the pages by `\textit{child}|.|\textit{page}'
where \textit{child} represents the chapter/section number of the child file.
This can be achieved by the command
|\numberwithin{page}{|\textit{child}|}|
of the \textsf{amsmath} package
where \textit{child} can be |chapter| or |section|
depending on the chosen structuring.
Alternatively, one can modify the macro |\thepage| appropriately
and reset the counter |page| at the start of each child file.

%%%%%%%%%%%%%%%%%%%%%%%%%%%%%%%%%%%%%%%%%%%%%%%%%%%%%%%%%%%%%%%%%%%%%%%%%%%%%%%%
\subsection{Conditional Processing}
\label{sec:conditional}

The package provides a mechanism to compile different versions
of a document. To customise the versions further some conditional processing
can come in handy to distinguish which version is being compiled.
The package provides two macros to describe the compilation context:

%%%%%%%%%%%%%%%%%%%%%%%%%%%%%%%%%%%%%%%%
\DescribeMacro{\ifchilddoc}
The conditional |\ifchilddoc| distinguishes between the compilation of
child documents and the main document:
%
\begin{center}
|\ifchilddoc |\textit{child-code}| |[|\||else |\textit{main-code}]| \||fi|
\end{center}

%%%%%%%%%%%%%%%%%%%%%%%%%%%%%%%%%%%%%%%%
\DescribeMacro{\childdocname}
\DescribeMacro{\childdocjob}
The macro |\childdocname| contains the filename (without extension)
of the main or child file being processed.
Note that |\childdocjob| will always contain the name of the main file.

%%%%%%%%%%%%%%%%%%%%%%%%%%%%%%%%%%%%%%%%
\paragraph{Title Page.}

Conditional processing can be used to include a title or banner page
in the main document when proper precautions are taken.
Importantly, the code in the main file should ensure that the page counter
(as well as other status parameters which are stored in the |.aux| files)
takes the same value after the conditional processing.
Otherwise the page numbers may take divergent values
depending on which part is compiled.

For example, a title page could be declared by:
%
\begin{center}
\begin{tabular}{l}
|\ifchilddoc\||else|\\
|\addtocounter{page}{-1}|\\
\textit{code for title page}\\
|\newpage|\\
|\||fi|
\end{tabular}
\end{center}
%
A banner page for the child documents can be generated by:
%
\begin{center}
\begin{tabular}{l}
|\ifchilddoc|\\
|\addtocounter{page}{-1}|\\
\textit{code for banner page}\\
|\newpage|\\
|\||fi|
\end{tabular}
\end{center}
%
Here one could write a message such as:
\begin{center}
|This is the part \childdocname{} of \childdocjob{}.|
\end{center}

%%%%%%%%%%%%%%%%%%%%%%%%%%%%%%%%%%%%%%%%%%%%%%%%%%%%%%%%%%%%%%%%%%%%%%%%%%%%%%%%
\subsection{Flags}
\label{sec:flags}

The package makes it easy to generate different versions
of the main or child documents.
To this end compilation flags can be defined
and assigned different default values.
They will be particularly useful in conjunction
with the forwarding mechanism described in \secref{sec:forward}.

For example, it may be useful to have a flag |\version|
which can be set to |draft| or |final|.
The document source will contain some conditional code
depending on the value of |\version|.
Suppose further, the flag should default to |final| for the main file
and to |draft| for child files
which is a natural assignment for editing the document.
This is achieved by placing the following code
in the preamble of the main document
(below the |\childdocmain| directive):
%
\begin{center}
\begin{tabular}{l}
|\ifchilddoc|\\
|\providecommand{\version}{draft}|\\
|\||else|\\
|\providecommand{\version}{final}|\\
|\||fi|
\end{tabular}
\end{center}
%
The definition by |\providecommand| makes sure
that previous definitions are not overwritten.
Further statements |\providecommand{\version}{...}|
can thus be added before the above code to override it.

For the main file, one might add a line
(between |\childdocmain| and the above block)
%
\begin{center}
|%\ifchilddoc\||else\providecommand{\version}{draft}\||fi|
\end{center}
%
which can be uncommented to produce a draft version.
Likewise one can add a line to the very top of a child file
(above the |\childdocof{|\textit{main}|}| directive)
%
\begin{center}
|%\providecommand{\version}{final}|
\end{center}
%
which can be uncommented to produce the final version of this child document.

%%%%%%%%%%%%%%%%%%%%%%%%%%%%%%%%%%%%%%%%%%%%%%%%%%%%%%%%%%%%%%%%%%%%%%%%%%%%%%%%
\subsection{Forwarding}
\label{sec:forward}

Different versions of the main or child documents
using compilation flags as described in \secref{sec:flags}
can be (permanently) stored in different files
for convenient compilation, viewing and distribution.
To this end, the package defines a command
to pass on compilation to a different file:

%%%%%%%%%%%%%%%%%%%%%%%%%%%%%%%%%%%%%%%%
\DescribeMacro{\childdocforward}
The command |\childdocforward| redirects processing to
another source file:
%
\begin{center}
\begin{tabular}{l}
|\input{childdoc.def}|\\
|\childdocforward[|\textit{main}|]{|\textit{dest}|}|\\
\end{tabular}
\end{center}
%
The argument \textit{dest} is the destination file
(without extension).
It should be the main file or one of the child files.
Note that further \textsf{childdoc} directives
such as |\childdocof| and |\childdocforward|
in the indicated file will be processed in this form.
The optional argument \textit{main}
passes on directly to the main file \textit{main}
while pretending to compile the child \textit{dest}.
This form behaves as if \textit{dest}
issues |\childdocof{|\textit{main}|}| right away,
and no further \textsf{childdoc} directives will be processed.

%%%%%%%%%%%%%%%%%%%%%%%%%%%%%%%%%%%%%%%%
\DescribeMacro{\...prefix}
In the alternative form |\childdocforwardprefix|,
%
\begin{center}
\begin{tabular}{l}
|\input{childdoc.def}|\\
|\childdocforwardprefix[|\textit{main}|]{|\textit{prefix}|}{|\textit{dest}|}|
\end{tabular}
\end{center}
%
the destination file is determined by a pattern
depending on the current file:
To make this work, the current file must be called
`{\textit{prefix}\hspace{0.2em}\textit{suffix}}'
with \textit{prefix} matching precisely the argument.
Processing is then passed on to the file
`{\textit{dest}\hspace{0.2em}\textit{suffix}}'.
Surely, the same effect is achieved by
directly specifying the
argument `{\textit{dest}\hspace{0.2em}\textit{suffix}}'
in the first form.
However, that requires to set up a different file
for each child. With the alternative form of the command
all these files can have exactly the same content
which simplifies setting them up and maintaining them.

For example, the following file |draft.tex|
with a compilation flag |\version| as described in \secref{sec:flags}
compiles the main document as a draft:
%
\begin{center}
\begin{tabular}{l}
|\def\version{draft}|\\
|\input{childdoc.def}|\\
|\childdocforward{|\textit{main}|}|
\end{tabular}
\end{center}
%
Likewise, the following files |final|\textit{nn}|.tex|
compile the final version of the child document
|child|\textit{nn}|.tex|:
%
\begin{center}
\begin{tabular}{l}
|\def\version{final}|\\
|\input{childdoc.def}|\\
|\childdocforwardprefix{final}{child}|
\end{tabular}
\end{center}
%

Note that when several versions of a main file and/or of each child file
are to be generated, it may be convenient to set up a |Makefile| or
shell script to automatise the process.

%%%%%%%%%%%%%%%%%%%%%%%%%%%%%%%%%%%%%%%%%%%%%%%%%%%%%%%%%%%%%%%%%%%%%%%%%%%%%%%%
\subsection{Command Line Processing}
\label{sec:commandline}

The effect of redirection files can also be achieved by invoking
the \LaTeX{} compiler with a more elaborate command line.
Most conveniently this should be done as part
of a shell script or a |Makefile|.

When using \textsf{childdoc} in the main file, the following
command lines effectively perform a redirection
(note that depending on the shell being used,
backslashes may have to be doubled: `|\|' $\to$ `|\\|'):
%
\begin{center}
|... -jobname "|\textit{target}|" |\\|"|[\textit{flags}]%
|\input{childdoc.def}\childdocforward[|\textit{main}|]{|\textit{dest}|}"|
\end{center}
%
Here \textit{target} is the name of the output file,
\textit{main} is the name of the main file
and \textit{dest} is the name of the main or child file to be processed
(all filenames without extensions).
The optional argument \textit{main} can be omitted
if \textit{main} matches \textit{dest}.
Optionally, compilation \textit{flags} can be defined via |\def| commands.
This command line makes the \TeX{} engine believe
it is compiling the file \textit{target}
whose content is specified as the latter parameter.
The provided code then forwards the processing to
\textit{main} or \textit{dest} as described in \secref{sec:forward}.

%%%%%%%%%%%%%%%%%%%%%%%%%%%%%%%%%%%%%%%%%%%%%%%%%%%%%%%%%%%%%%%%%%%%%%%%%%%%%%%%
\subsection{Include by Input}
\label{sec:input}

Including child documents by |\include| has some restrictions by design.
Most notably, the content of a child document always occupies
its own set of pages; pages cannot be shared between child documents.
Usually, this behaviour makes perfect sense
because each child document contain an essential part of the document.
However, in some situations it may be desirable to compose
a document from a collection of parts
without having mandatory page breaks between then.
For this case, the package
provides a mechanism to include parts
by |\input| which can also be processed individually.
However, by construction this mechanism
requires manual handling of the content to be output.

%%%%%%%%%%%%%%%%%%%%%%%%%%%%%%%%%%%%%%%%
\DescribeMacro{\ifchilddocmanual}
The main file should be prepared as usual, see \secref{sec:include}.
However, the document body must make a distinction
between processing of an individual part and of the main document, e.g.:
%
\begin{center}
\begin{tabular}{l}
|\ifchilddocmanual|\\
|\input{\childdocname}|\\
|\||else|\\
\textit{document body with }|\input{|\textit{part}|}|\\
|\||fi|
\end{tabular}
\end{center}
%
The conditional |\ifchilddocmanual| is true whenever
a part to be included by |\input| is being compiled,
and the name of the part is stored in |\childdocname|.

%%%%%%%%%%%%%%%%%%%%%%%%%%%%%%%%%%%%%%%%
\DescribeMacro{\childdocby}
Each part to be included by |\input| should start with:
%
\begin{center}
\begin{tabular}{l}
|\input{childdoc.def}|\\
|\childdocby{|\textit{main}|}|\\
\end{tabular}
\end{center}
%
The directive |\childdocby| is similar to |\childdocof|
described in \secref{sec:include},
but the subsequent selection of content must be done manually.
To that end, both |\ifchilddoc| and |\ifchilddocmanual|
will be true upon processing of a part,
and the name of the part is stored in |\childdocname|.
Note that |\jobname| will be set to the filename of the current part
so that each part receives an individual |.aux| file
that does not interfere with the |.aux| file(s) of the main document.
This behaviour can be altered by the alternative form
|\childdocby[*]{|\textit{main}|}| (with a non-empty optional argument)
which uses the |.aux| file of the main document
by setting |\jobname| to \textit{main}.

%%%%%%%%%%%%%%%%%%%%%%%%%%%%%%%%%%%%%%%%%%%%%%%%%%%%%%%%%%%%%%%%%%%%%%%%%%%%%%%%
\subsection{Driver Development}
\label{sec:driver}

The \textsf{childdoc} mechanism can also be use for the development
of definition files such as \LaTeX{} styles or classes.
This case differs from the above setup with multiple parts
included by |\include| in that no |\includeonly| should be invoked.
This can be achieved by starting the include file
(before |\ProvidesPackage|) with:
%
\begin{center}
\begin{tabular}{l}
|\input{childdoc.def}|\\
|\childdocforward{|\textit{main}|}|\\
\end{tabular}
\end{center}
%
or alternatively with:
%
\begin{center}
\begin{tabular}{l}
|\input{childdoc.def}|\\
|\childdocby{|\textit{main}|}|\\
\end{tabular}
\end{center}
%
Both forms have slightly different effects as described above.
The main file is prepared as usual, see \secref{sec:include}.

%%%%%%%%%%%%%%%%%%%%%%%%%%%%%%%%%%%%%%%%%%%%%%%%%%%%%%%%%%%%%%%%%%%%%%%%%%%%%%%%
\subsection{Legacy Detection}
\label{sec:detection}

The directive |\childdocmain| in the main file can detect
whether the complete document or merely a child is to be compiled
even without using the directive |\childdocof|.
This method is deprecated because it is less robust
and there is no compelling reason to use it;
it is merely provided for backward compatibility
and it may be removed in future versions.

If the detection mechanism is to be used,
it is mandatory to correctly specify
the filename of the main file as the argument of |\childdocmain|:
%
\begin{center}
\begin{tabular}{l}
|\input{childdoc.def}|\\
|\childdocmain{|\textit{main}|}|\\
\end{tabular}
\end{center}
%
If |\jobname| does not match the argument \textit{main} of |\childdocmain|,
it is assumed that |\jobname| points to the child file to be compiled.
When using |\childdocmain| with the main file specified as argument,
it suffices to start a child file
with just |\input{|\textit{main}|}|
without loading of the package and using |\childdocof|.
If instead all processing is done
with the appropriate \textsf{childdoc} directives,
the argument of \textit{main} of |\childdocmain| can be empty.

An alternative version of the command line processing described
in \secref{sec:commandline} using the detection mechanism reads:
%
\begin{center}
|... -jobname "|\textit{target}|" "|[\textit{flags}]%
[|\def\jobname{|\textit{dest}|}|]|\input{|\textit{main}|}"|
\end{center}

%%%%%%%%%%%%%%%%%%%%%%%%%%%%%%%%%%%%%%%%%%%%%%%%%%%%%%%%%%%%%%%%%%%%%%%%%%%%%%%%
\subsection{Manual Code}
\label{sec:manual}

In case one cannot be certain whether the definitions file |childdoc.def|
is installed on the target \TeX{} distribution
and one prefers not to ship it,
it is conceivable to paste a few relevant commands into the sources.

To that end, drop all statements |\input{childdoc.def}|
and perform the replacements as outlined below.
Instead of |\childdocmain{|\textit{main}|}| add the following code
to the top of the main file:
%
\begin{center}
\begin{tabular}{l}
|\||ifdefined\childdocname\endinput\||fi\newif\ifchilddoc|\\
|\edef\childdocname{\scantokens\expandafter{\jobname\noexpand}}|\\
|\def\childdocmain{|\textit{main}|}\||ifx\childdocmain\childdocname\||else|\\
|\childdoctrue\includeonly{\childdocname}\let\jobname\childdocmain\||fi|\\
\end{tabular}
\end{center}
%
Instead of |\childdocof{|\textit{main}|}| just include the main file
at the top of each child file:
%
\begin{center}
|\input{|\textit{main}|}|
\end{center}
%
A simple redirection |\childdocforward{|\textit{dest}|}| is achieved by:
%
\begin{center}
|\def\jobname{|\textit{dest}|}\input{\jobname}|
\end{center}
%
The redirection with prefix
|\childdocforwardprefix[|\textit{prefix}|]{|\textit{dest}|}|
is accomplished by:
%
\begin{center}
\begin{tabular}{l}
|{\edef\jobname{\scantokens\expandafter{\jobname\noexpand}}|\\
|\def\redirectjob |\textit{prefix}|#1~~~{\gdef\jobname{|\textit{dest}|#1}}|\\
|\expandafter\redirectjob\jobname~~~}\input{\jobname}|
\end{tabular}
\end{center}

In an alternative approach,
child documents can be compiled by a specific command line
without additional code or specific definitions:
%
\begin{center}
|... -jobname "|\textit{target}|" "|[\textit{flags}]%
|\includeonly{|\textit{dest}|}\input{|\textit{main}|}"|
\end{center}
%

%%%%%%%%%%%%%%%%%%%%%%%%%%%%%%%%%%%%%%%%%%%%%%%%%%%%%%%%%%%%%%%%%%%%%%%%%%%%%%%%
%%%%%%%%%%%%%%%%%%%%%%%%%%%%%%%%%%%%%%%%%%%%%%%%%%%%%%%%%%%%%%%%%%%%%%%%%%%%%%%%
\section{Information}

%%%%%%%%%%%%%%%%%%%%%%%%%%%%%%%%%%%%%%%%%%%%%%%%%%%%%%%%%%%%%%%%%%%%%%%%%%%%%%%%
\subsection{Copyright}

Copyright \copyright{} 2017--2018 Niklas Beisert

This work may be distributed and/or modified under the
conditions of the \LaTeX{} Project Public License, either version 1.3
of this license or (at your option) any later version.
The latest version of this license is in
  \url{http://www.latex-project.org/lppl.txt}
and version 1.3 or later is part of all distributions of \LaTeX{}
version 2005/12/01 or later.

This work has the LPPL maintenance status `maintained'.

The Current Maintainer of this work is Niklas Beisert.

This work consists of the files |README.txt|, |childdoc.ins| and |childdoc.dtx|
as well as the derived files |childdoc.def|, |cdocsamp.tex|
with |cdocsch1.tex|, |cdocsch2.tex|, |cdocspt3.tex|, |cdocspt4.tex|,
|cdocsdrf.tex|, |cdocsfn1.tex|, |cdocsfn2.tex|
as well as |childdoc.pdf|.

%%%%%%%%%%%%%%%%%%%%%%%%%%%%%%%%%%%%%%%%%%%%%%%%%%%%%%%%%%%%%%%%%%%%%%%%%%%%%%%%
\subsection{Files and Installation}

The package consists of the files:
%
\begin{center}
\begin{tabular}{ll}
    |README.txt|   & readme file \\
    |childdoc.ins| & installation file \\
    |childdoc.dtx| & source file \\
    |childdoc.def| & definition file \\
    |cdocsamp.tex| & sample main file \\
    |cdocsch1.tex| & sample include file \\
    |cdocsch2.tex| & sample include file \\
    |cdocspt3.tex| & sample part file \\
    |cdocspt4.tex| & sample part file \\
    |cdocsdrf.tex| & sample redirection file \\
    |cdocsfn1.tex| & sample redirection file \\
    |cdocsfn2.tex| & sample redirection file \\
    |childdoc.pdf| & manual
\end{tabular}
\end{center}
%
The distribution consists of the files
|README.txt|, |childdoc.ins| and |childdoc.dtx|.
%
\begin{itemize}
\item
Run (pdf)\LaTeX{} on |childdoc.dtx|
to compile the manual |childdoc.pdf| (this file).
\item
Run \LaTeX{} on |childdoc.ins| to create the definitions file |childdoc.def|
and the sample |cdocsamp.tex| with include files
|cdocsch1.tex|, |cdocsch2.tex|, |cdocspt3.tex|, |cdocspt4.tex|,
|cdocsdrf.tex|, |cdocsfn1.tex|, |cdocsfn2.tex|.
Then copy the file |childdoc.def| to an appropriate directory of your \LaTeX{}
distribution, e.g.\ \textit{texmf-root}|/tex/latex/childdoc|.
\end{itemize}

%%%%%%%%%%%%%%%%%%%%%%%%%%%%%%%%%%%%%%%%%%%%%%%%%%%%%%%%%%%%%%%%%%%%%%%%%%%%%%%%
\subsection{Related CTAN Packages}

There are several other packages which offer a similar functionality:
%
\begin{itemize}
\item
The packages
\href{http://ctan.org/pkg/docmute}{\textsf{docmute}},
\href{http://ctan.org/pkg/includex}{\textsf{includex}} and
\href{http://ctan.org/pkg/standalone}{\textsf{standalone}}
provide commands to include only the document body of
a child file thus allowing both files to be compiled individually.
\item
The packages \href{http://ctan.org/pkg/subdocs}{\textsf{subdocs}}
and \href{http://ctan.org/pkg/subfiles}{\textsf{subfiles}}
provide structures in which the main and child documents can be
encapsulated and allowing them to be compiled individually.
The inclusion mechanism is different from the conventional |\include|.
\item
The package \href{http://ctan.org/pkg/combine}{\textsf{combine}}
is an elaborate solution to combine several documents into one.
\end{itemize}
%
See also the CTAN topic \href{http://ctan.org/topic/subdocs}{\textsf{subdocs}}
for further related packages.
The present package differs from the above solutions in that
a document structure constructed with the conventional |\include| mechanism
just needs two extra commands at the top of every file
such that all constituent files can be compiled individually.

%%%%%%%%%%%%%%%%%%%%%%%%%%%%%%%%%%%%%%%%%%%%%%%%%%%%%%%%%%%%%%%%%%%%%%%%%%%%%%%%
%\subsection{Feature Suggestions}
%
%The following is a list of features which may be useful for future
%versions of this package:
%%
%\begin{itemize}
%\item
%\ldots
%\end{itemize}

%%%%%%%%%%%%%%%%%%%%%%%%%%%%%%%%%%%%%%%%%%%%%%%%%%%%%%%%%%%%%%%%%%%%%%%%%%%%%%%%
\subsection{Revision History}

%%%%%%%%%%%%%%%%%%%%%%%%%%%%%%%%%%%%%%%%
\paragraph{v2.0:} 2018/12/30

\begin{itemize}
\item
immediate forward processing
\item
added |\childdocby| mechanism
\item
manual restructured
\end{itemize}

%%%%%%%%%%%%%%%%%%%%%%%%%%%%%%%%%%%%%%%%
\paragraph{v1.6:} 2018/01/17

\begin{itemize}
\item
application for development of include files
\item
corrections to manual
\end{itemize}

%%%%%%%%%%%%%%%%%%%%%%%%%%%%%%%%%%%%%%%%
\paragraph{v1.5:} 2017/05/21

\begin{itemize}
\item
more complete structuring introduced
\item
|\childdocof| introduced
\item
|\childdoc| renamed to |\childdocmain|
\item
|\childredirect| renamed to |\childdocforward| and |\childdocforwardprefix|
and functionality expanded
\end{itemize}

%%%%%%%%%%%%%%%%%%%%%%%%%%%%%%%%%%%%%%%%
\paragraph{v1.0:} 2017/04/27

\begin{itemize}
\item
manual and install package
\item
first version published on CTAN
\end{itemize}

%%%%%%%%%%%%%%%%%%%%%%%%%%%%%%%%%%%%%%%%
\paragraph{v0.6:} 2017/04/26

\begin{itemize}
\item
redirection mechanism added
\end{itemize}

%%%%%%%%%%%%%%%%%%%%%%%%%%%%%%%%%%%%%%%%
\paragraph{v0.5:} 2017/04/26

\begin{itemize}
\item
functionality in definition file
\end{itemize}


%%%%%%%%%%%%%%%%%%%%%%%%%%%%%%%%%%%%%%%%%%%%%%%%%%%%%%%%%%%%%%%%%%%%%%%%%%%%%%%%
%%%%%%%%%%%%%%%%%%%%%%%%%%%%%%%%%%%%%%%%%%%%%%%%%%%%%%%%%%%%%%%%%%%%%%%%%%%%%%%%
%%%%%%%%%%%%%%%%%%%%%%%%%%%%%%%%%%%%%%%%%%%%%%%%%%%%%%%%%%%%%%%%%%%%%%%%%%%%%%%%
\appendix

\settowidth\MacroIndent{\rmfamily\scriptsize 000\ }

 \DocInput{childdoc.dtx}

\end{document}
%</driver>
% \fi
%
% %%%%%%%%%%%%%%%%%%%%%%%%%%%%%%%%%%%%%%%%%%%%%%%%%%%%%%%%%%%%%%%%%%%%%%%%%%%%%%
% %%%%%%%%%%%%%%%%%%%%%%%%%%%%%%%%%%%%%%%%%%%%%%%%%%%%%%%%%%%%%%%%%%%%%%%%%%%%%%
% \section{Sample}
%\iffalse
%<*samplemain>
%\fi
%
% The following presents a sample document
% with two chapters, two parts, a title page,
% a compile flag as well as three forwarding files to set the flag.
% It consists of eight |.tex| files:
% \begin{center}
% \begin{tabular}{ll}
% |cdocsamp.tex|&main file\\
% |cdocsch1.tex|&include file for chapter 1\\
% |cdocsch2.tex|&include file for chapter 2\\
% |cdocspt3.tex|&include file for part 3\\
% |cdocspt4.tex|&include file for part 4\\
% |cdocsdrf.tex|&forwarding file for main file in draft mode\\
% |cdocsfi1.tex|&forwarding file for final version of chapter 1\\
% |cdocsfi2.tex|&forwarding file for final version of chapter 2\\
% \end{tabular}
% \end{center}
% Each of the eight files can be compiled directly by the \LaTeX{} compiler.
%
% %%%%%%%%%%%%%%%%%%%%%%%%%%%%%%%%%%%%%%
% \paragraph{Main File.}
%
% The main file is called |cdocsamp.tex|.
%
% Load the \textsf{childdoc} definitions and
% declare the filename for the main document:
%    \begin{macrocode}
\input{childdoc.def}
\childdocmain{}
%    \end{macrocode}

% Optional override for |\version| flag:
%    \begin{macrocode}
%%\ifchilddoc\else\providecommand{\version}{draft}\fi
%    \end{macrocode}

% Define the default values for the |\version| flag
% (|final| for the main file and |draft| for childs):
%    \begin{macrocode}
\ifchilddoc
\providecommand{\version}{draft}
\else
\providecommand{\version}{final}
\fi
%    \end{macrocode}

% Load the standard document class:
%    \begin{macrocode}
\documentclass[12pt]{article}
%    \end{macrocode}

% Start the document body:
%    \begin{macrocode}
\begin{document}
%    \end{macrocode}

% Declare a title page.
% Print title, part of document being processed and version flag:
%    \begin{macrocode}
\addtocounter{page}{-1}
\begin{center}
{\LARGE\bfseries{}childdoc example\par}
\vspace{1cm}
\ifchilddoc
\ifchilddocmanual part\else chapter\fi:
`\childdocname' of `\childdocjob'\par
\else
main document: `\childdocjob'\par
\fi
version: \version\par
\end{center}
\newpage
%    \end{macrocode}

% Manually include selected file,
% otherwise process as usual:
%    \begin{macrocode}
\ifchilddocmanual
\section*{part `\childdocname'}
\input{\childdocname}
\else
%    \end{macrocode}

% Include the two chapters:
%    \begin{macrocode}
\include{cdocsch1}
\include{cdocsch2}
%    \end{macrocode}

% Include the two parts unless only chapters should be displayed:
%    \begin{macrocode}
\ifchilddoc\else
\section{part three}
\input{cdocspt3}
\section{part four}
\input{cdocspt4}
\fi
%    \end{macrocode}

% Process as usual until here:
%    \begin{macrocode}
\fi
%    \end{macrocode}

% End of document body:
%    \begin{macrocode}
\end{document}
%    \end{macrocode}
%\iffalse
%</samplemain>
%\fi
%
% %%%%%%%%%%%%%%%%%%%%%%%%%%%%%%%%%%%%%%
% \paragraph{Chapter Include Files.}
%
% The include files are called |cdocsch1.tex| and |cdocsch2.tex|.
%
%\iffalse
%<*samplechap1|samplechap2>
%\fi

% Optional override for |\version| flag:
%    \begin{macrocode}
%%\providecommand{\version}{final}
%    \end{macrocode}

% Include the main document:
%    \begin{macrocode}
\input{childdoc.def}
\childdocof{cdocsamp}
%    \end{macrocode}

%\iffalse
%</samplechap1|samplechap2>
%\fi
%
%\iffalse
%<*samplechap1>
%\fi
% Some text for chapter 1:
%    \begin{macrocode}
\section{one}
some text in chapter one
%    \end{macrocode}

%\iffalse
%</samplechap1>
%\fi
% Some text for chapter 2:
%\iffalse
%<*samplechap2>
%\fi
%    \begin{macrocode}
\section{two}
more text in chapter two
%    \end{macrocode}

%\iffalse
%</samplechap2>
%\fi
%
% %%%%%%%%%%%%%%%%%%%%%%%%%%%%%%%%%%%%%%
% \paragraph{Part Include Files.}
%
% The include files are called |cdocspt3.tex| and |cdocspt4.tex|.
%
%\iffalse
%<*samplepart3|samplepart4>
%\fi

% Optional override for |\version| flag:
%    \begin{macrocode}
%%\providecommand{\version}{final}
%    \end{macrocode}

% Include the main document:
%    \begin{macrocode}
\input{childdoc.def}
\childdocby{cdocsamp}
%    \end{macrocode}

%\iffalse
%</samplepart3|samplepart4>
%\fi
%
%\iffalse
%<*samplepart3>
%\fi
% Some text for part 3:
%    \begin{macrocode}
some text in part three
%    \end{macrocode}

%\iffalse
%</samplepart3>
%\fi
% Some text for part 4:
%\iffalse
%<*samplepart4>
%\fi
%    \begin{macrocode}
more text in part four
%    \end{macrocode}

%\iffalse
%</samplepart4>
%\fi
%
% %%%%%%%%%%%%%%%%%%%%%%%%%%%%%%%%%%%%%%
% \paragraph{Forwarding for a Complete Draft.}
%
% The following forwarding file |cdocsdrf.tex|
% compiles the main document in draft mode:
%\iffalse
%<*sampledraft>
%\fi
%    \begin{macrocode}
\def\version{draft}
\input{childdoc.def}
\childdocforward{cdocsamp}
%    \end{macrocode}

%\iffalse
%</sampledraft>
%\fi
%
% %%%%%%%%%%%%%%%%%%%%%%%%%%%%%%%%%%%%%%
% \paragraph{Forwarding for Final Version of the Chapters.}
%
% The following forwarding files |cdocsfn1.tex| and |cdocsfn2.tex|
% (with identical content)
% compile the final versions of the child documents
% |cdocsch1.tex| and |cdocsch2.tex|, respectively:
%\iffalse
%<*samplefinal>
%\fi
%    \begin{macrocode}
\def\version{final}
\input{childdoc.def}
\childdocforwardprefix[cdocsamp]{cdocsfn}{cdocsch}
%    \end{macrocode}

%\iffalse
%</samplefinal>
%\fi
%
% %%%%%%%%%%%%%%%%%%%%%%%%%%%%%%%%%%%%%%
% \paragraph{Command Line Processing.}
%
% The following three command lines generate the output files
% |cdocscld|, |cdocscl1| and |cdocscl2|
% which should be identical to
% |cdocsdrf|, |cdocsch1| and |cdocsfn2|, respectively:
% \begin{center}
% \begin{tabular}{l}
% |latex -jobname cdocscld \|\\
% |  "\def\version{draft}\input{childdoc.def}\childdocforward{cdocsamp}"|\\
% |latex -jobname cdocscl1 \|\\
% |  "\input{childdoc.def}\childdocforward[cdocsamp]{cdocsch1}"|\\
% |latex -jobname cdocscl2 \|\\
% |  "\def\version{final}\input{childdoc.def}\childdocforward{cdocsch2}"|
% \end{tabular}
% \end{center}
% Note that the trailing backslash on each first line
% merely continues the input to the second line
% (for convenient cut ant paste).
% Furthermore, the command |latex| can be replaced by any
% of its alternative versions such as |pdflatex|.
%
% %%%%%%%%%%%%%%%%%%%%%%%%%%%%%%%%%%%%%%%%%%%%%%%%%%%%%%%%%%%%%%%%%%%%%%%%%%%%%%
% %%%%%%%%%%%%%%%%%%%%%%%%%%%%%%%%%%%%%%%%%%%%%%%%%%%%%%%%%%%%%%%%%%%%%%%%%%%%%%
% \section{Implementation}
%\iffalse
%<*package>
%\fi
%
% This section describes the definitions file |childdoc.def|.

% The definitions cannot be loaded using |\usepackage| or |\RequirePackage|
% which has a mechanism to prevent loading a style file more than once.
% When loading the definitions by means of |\input|
% multiple instances have to be prevented manually:
%\iffalse
%This code needs to be before the `\ProvidesFile' directive
%which is defined at the beginning of this file.
%Therefore it is also placed there and commented out here.
%</package>
%<*discard>
%\fi
%    \begin{macrocode}
\ifdefined\childdocmain\endinput\fi
%    \end{macrocode}
%\iffalse
%</discard>
%<*package>
%\fi
%
% \macro{\ifchilddoc}
% \macro{\ifchilddocmanual}
% The conditional |\ifchilddoc| tells whether a
% child (true) or main (false) document is being compiled.
% The conditional |\ifchilddocmanual| tells whether
% the |\includeonly| mechanism is used (false) or
% the selection of child files must be performed manually (true).
% The definitions initialise to false:
%    \begin{macrocode}
\newif\ifchilddoc
\newif\ifchilddocmanual
%    \end{macrocode}

% \macro{\childdocname}
% \macro{\childdocjob}
% The macro |\childdocname| stores the name of the main document
% to be compiled. The macro |\childdocjob| stores the name of
% the document on which the \LaTeX{} compiler was originally invoked.
% The content of |\jobname| cannot be compared
% to filenames specified in the source due to different catcodes.
% The following code rescans |\jobname|, stores the result
% in |\childdocname| and saves a copy in |\childdocjob|:
%    \begin{macrocode}
\edef\childdocname{\scantokens\expandafter{\jobname\noexpand}}
\let\childdocjob\childdocname
%    \end{macrocode}

% \macro{\childdocdisable}
% The macro |\childdocdisable| prevents the main file
% from being processed more than once.
% At this stage, the main document command |\childdocmain|
% is assumed to be called once again where it should do nothing.
% Any subsequent call to it should prevent
% a secondary processing of the main document
% It overwrites the forwarding commands
% |\childdocof| and |\childdocforward|
% with empty macros to prevent further inclusions of the main document:
%    \begin{macrocode}
\newcommand{\childdocdisable}
{
  \renewcommand{\childdocmain}[1]{\renewcommand{\childdocmain}[1]{\endinput}}
  \renewcommand{\childdocof}[1]{}
  \renewcommand{\childdocby}[2][]{}
  \renewcommand{\childdocforward}[2][]{}
  \renewcommand{\childdocdisable}{}
}
%    \end{macrocode}

% \macro{\childdocmain}
% The macro |\childdocmain| is to be called at the top of the main file
% with nothing or the main filename (without extension) as argument.
% First, it breaks loops.
% If the argument is not empty and does not match |\childdocname|
% (which is set by the first inclusion of |childdoc.def|),
% |\ifchilddoc| is set to true, |\includeonly| is applied to the child file
% and |\jobname| is set to the main file
% (for proper handling of |.aux| files):
%    \begin{macrocode}
\newcommand{\childdocmain}[1]
{
  \childdocdisable\childdocmain{}
  \if?#1?\else
    \begingroup
      \def\childdoctmp{#1}
      \ifx\childdoctmp\childdocname
        \def\childdoctmp{}
      \else
        \def\childdoctmp
        {
          \childdoctrue
          \includeonly{\childdocname}
          \def\childdocjob{#1}
          \def\jobname{#1}
        }
      \fi
      \expandafter
    \endgroup
    \childdoctmp
  \fi
}
%    \end{macrocode}

% \macro{\childdocof}
% The command |\childdocof| redirects
% compilation to the main file |#1|.
%    \begin{macrocode}
\newcommand{\childdocof}[1]
{
  \childdocdisable
  \childdoctrue
  \includeonly{\childdocname}
  \def\jobname{#1}
  \def\childdocjob{#1}
  \input{#1}
}
%    \end{macrocode}

% \macro{\childdocby}
% The command |\childdocby| ....
%    \begin{macrocode}
\newcommand{\childdocby}[2][]
{
  \childdocdisable
  \childdoctrue
  \childdocmanualtrue
  \if?#1?\else
    \def\jobname{#2}
  \fi
  \def\childdocjob{#2}
  \input{#2}
  \endinput
}
%    \end{macrocode}

% \macro{\childdocforward}
% The command |\childdocforward| redirects
% compilation to the main file or
% (if the optional argument is given) a child file.
% Parameters are set as if the main file
% or a child file starting with |\childdocof| was compiled.
% Then compilation is handed over to the main file:
%    \begin{macrocode}
\newcommand{\childdocforward}[2][]
{
  \begingroup
    \if?#1?
      \def\childdoctmp
      {
        \def\childdocname{#2}
        \def\childdocjob{#2}
        \def\jobname{#2}
        \input{#2}
        \endinput
      }
    \else
      \def\childdoctmp
      {
        \childdocdisable
        \def\childdocname{#2}
        \childdoctrue
        \includeonly{#2}
        \def\childdocjob{#1}
        \def\jobname{#1}
        \input{#1}
        \endinput
      }
    \fi
    \expandafter
  \endgroup
  \childdoctmp
}
%    \end{macrocode}

% \macro{\childdocforwardprefix}
% The command |\childdocforwardprefix| redirects
% compilation to the main or a child file by means of a pattern.
% The prefix |#1| in the current filename is replaced by |#2|
% and the suffix of the current filename is kept
% (it is assumed that the filename does not contain the substring `|~~~|'
% which is used as a delimiter).
% Compilation is handed over to the new file by |\childdocforward|:
%    \begin{macrocode}
\newcommand{\childdocforwardprefix}[3][]
{
  \begingroup
    \def\childdocextract #2##1~~~{\def\childdoctmp{\childdocforward[#1]{#3##1}}}
    \expandafter\childdocextract\childdocname~~~
    \expandafter
  \endgroup
  \childdoctmp
}
%    \end{macrocode}

% \macro{\childdoc}
% The deprecated macro |\childdoc| is a legacy version of |\childdocmain|:
%    \begin{macrocode}
\newcommand{\childdoc}{\childdocmain}
%    \end{macrocode}

% \macro{\childdocredirect}
% The deprecated macro |\childdocredirect| is a legacy version
% of |\childdocforward| and |\childdocforwardprefix|:
%    \begin{macrocode}
\newcommand{\childdocredirect}[2][]
{
  \begingroup
    \if?#1?
      \def\childdoctmp{\childdocforward{#2}}
    \else
      \def\childdoctmp{\childdocforwardprefix{#1}{#2}}
    \fi
    \expandafter
  \endgroup
  \childdoctmp
}
%    \end{macrocode}

%\iffalse
%</package>
%\fi
%
\endinput
|\\
|\childdocby{|\textit{main}|}|\\
\end{tabular}
\end{center}
%
The directive |\childdocby| is similar to |\childdocof|
described in \secref{sec:include},
but the subsequent selection of content must be done manually.
To that end, both |\ifchilddoc| and |\ifchilddocmanual|
will be true upon processing of a part,
and the name of the part is stored in |\childdocname|.
Note that |\jobname| will be set to the filename of the current part
so that each part receives an individual |.aux| file
that does not interfere with the |.aux| file(s) of the main document.
This behaviour can be altered by the alternative form
|\childdocby[*]{|\textit{main}|}| (with a non-empty optional argument)
which uses the |.aux| file of the main document
by setting |\jobname| to \textit{main}.

%%%%%%%%%%%%%%%%%%%%%%%%%%%%%%%%%%%%%%%%%%%%%%%%%%%%%%%%%%%%%%%%%%%%%%%%%%%%%%%%
\subsection{Driver Development}
\label{sec:driver}

The \textsf{childdoc} mechanism can also be use for the development
of definition files such as \LaTeX{} styles or classes.
This case differs from the above setup with multiple parts
included by |\include| in that no |\includeonly| should be invoked.
This can be achieved by starting the include file
(before |\ProvidesPackage|) with:
%
\begin{center}
\begin{tabular}{l}
|% \iffalse
%
% childdoc.dtx Copyright (C) 2017-2018 Niklas Beisert
%
% This work may be distributed and/or modified under the
% conditions of the LaTeX Project Public License, either version 1.3
% of this license or (at your option) any later version.
% The latest version of this license is in
%   http://www.latex-project.org/lppl.txt
% and version 1.3 or later is part of all distributions of LaTeX
% version 2005/12/01 or later.
%
% This work has the LPPL maintenance status `maintained'.
%
% The Current Maintainer of this work is Niklas Beisert.
%
% This work consists of the files childdoc.dtx and childdoc.ins
% and the derived files childdoc.def and cdocsamp.tex with
% cdocsch1.tex, cdocsch2.tex, cdocsdrf.tex, cdocsfn1.tex, cdocsfn2.tex.
%
%<package>\ifdefined\childdocmain\endinput\fi
%<package>\ProvidesFile{childdoc.def}[2018/12/30 v2.0 child document driver]
%<samplemain>\ProvidesFile{cdocsamp.tex}[2018/12/30 v2.0 sample for childdoc]
%<*driver>
%\ProvidesFile{childdoc.drv}[2018/12/30 v2.0 childdoc reference manual file]
\PassOptionsToClass{10pt,a4paper}{article}
\documentclass{ltxdoc}

\usepackage[margin=35mm]{geometry}
\usepackage{hyperref}
\usepackage{hyperxmp}
\usepackage[usenames]{color}

\hypersetup{colorlinks=true}
\hypersetup{pdfstartview=FitH}
\hypersetup{pdfpagemode=UseNone}
\hypersetup{pdfsource={}}
\hypersetup{pdflang={en-UK}}
\hypersetup{pdfcopyright={Copyright 2017-2018 Niklas Beisert.
  This work may be distributed and/or modified under the
  conditions of the LaTeX Project Public License, either version 1.3
  of this license or (at your option) any later version.}}
\hypersetup{pdflicenseurl={http://www.latex-project.org/lppl.txt}}
\hypersetup{pdfcontactaddress={ETH Zurich, ITP, HIT K,
  Wolfgang-Pauli-Strasse 27}}
\hypersetup{pdfcontactpostcode={8093}}
\hypersetup{pdfcontactcity={Zurich}}
\hypersetup{pdfcontactcountry={Switzerland}}
\hypersetup{pdfcontactemail={nbeisert@itp.phys.ethz.ch}}
\hypersetup{pdfcontacturl={http://people.phys.ethz.ch/\xmptilde nbeisert/}}

\newcommand{\secref}[1]{\hyperref[#1]{section \ref*{#1}}}

\parskip1ex
\parindent0pt
\let\olditemize\itemize
\def\itemize{\olditemize\parskip0pt}

\begin{document}

\title{The \textsf{childdoc} Package}
\hypersetup{pdftitle={The childdoc Package}}
\author{Niklas Beisert\\[2ex]
  Institut f\"ur Theoretische Physik\\
  Eidgen\"ossische Technische Hochschule Z\"urich\\
  Wolfgang-Pauli-Strasse 27, 8093 Z\"urich, Switzerland\\[1ex]
  \href{mailto:nbeisert@itp.phys.ethz.ch}
  {\texttt{nbeisert@itp.phys.ethz.ch}}}
\hypersetup{pdfauthor={Niklas Beisert}}
\hypersetup{pdfsubject={Manual for the LaTeX2e Package childdoc}}
\date{30 December 2018, \textsf{v2.0}}
\maketitle

\begin{abstract}\noindent
\textsf{childdoc} is a \LaTeXe{} package
that enables the direct compilation
of document sections included by |\include|
to individual files.
\end{abstract}

\begingroup
\parskip0ex
\tableofcontents
\endgroup

%%%%%%%%%%%%%%%%%%%%%%%%%%%%%%%%%%%%%%%%%%%%%%%%%%%%%%%%%%%%%%%%%%%%%%%%%%%%%%%%
%%%%%%%%%%%%%%%%%%%%%%%%%%%%%%%%%%%%%%%%%%%%%%%%%%%%%%%%%%%%%%%%%%%%%%%%%%%%%%%%
\section{Introduction}

\LaTeX{} provides a mechanism to structure a large document (such as a book)
into a main file and several child files (containing the chapters)
using the |\include| command.
This mechanism is beneficial for documents
which span hundreds of pages in order to
make the source file(s) more manageable.
Moreover, compilation can be restricted to
selected child files by means of the |\includeonly| command.
The latter feature can be used to reduce the compilation time while editing
(this was significantly more useful in the earlier days of \LaTeX{})
or to generate a smaller document which is easier to navigate.
Another application of |\includeonly| is to generate
documents consisting of selected parts of the complete document.

However, there are a few drawbacks of the plain |\include| mechanism:
\begin{itemize}
\item
The child files cannot be compiled on their own,
they can only be compiled via the main file.
A naive editing environment
(such as a text editor with an option
to have the current file processed by \LaTeX)
may require one to switch to the main file before compiling;
attempting to compile the child file produces errors.
\item
The main file must be modified (each time)
to adjust the |\includeonly| command
to the present needs. This easily leaves the main file in a messy state.
\item
The generated document will always carry the filename
of the main document. This is inconvenient if
several child files are to be compiled and
to be kept for distribution.
\end{itemize}

The present package provides a simple interface
to make child files individually compilable by \LaTeX{}.
Compiling a child file then has the same effect as compiling
the main file with an |\includeonly| command
to select the appropriate child.
Moreover the generated document will carry the name of the child
rather than the main file.
This resolves all three above issues.

This feature is meant to make the editing of books,
thesis documents and lecture notes somewhat more convenient.
However, the package can also be used efficiently for
composing a series of documents (such as exercise sheets)
which are typically distributed individually.
It then assists the author in generating the individual documents
(potentially in different versions)
as well as a document containing the collected series.
Another application is in developing style files
or other kinds of included material
where compilation of the style file could redirect
to a sample or test file.

%%%%%%%%%%%%%%%%%%%%%%%%%%%%%%%%%%%%%%%%%%%%%%%%%%%%%%%%%%%%%%%%%%%%%%%%%%%%%%%%
%%%%%%%%%%%%%%%%%%%%%%%%%%%%%%%%%%%%%%%%%%%%%%%%%%%%%%%%%%%%%%%%%%%%%%%%%%%%%%%%
\section{Usage}

First of all, the package \textsf{childdoc} is \emph{not} a standard
\LaTeXe{} |.sty| style file! Therefore it needs to be invoked in
a non-standard way.

%%%%%%%%%%%%%%%%%%%%%%%%%%%%%%%%%%%%%%%%%%%%%%%%%%%%%%%%%%%%%%%%%%%%%%%%%%%%%%%%
\subsection{Included Files}
\label{sec:include}

%%%%%%%%%%%%%%%%%%%%%%%%%%%%%%%%%%%%%%%%
\DescribeMacro{\childdocmain}
To use the package, add the commands
\begin{center}
\begin{tabular}{l}
|\input{childdoc.def}|\\
|\childdocmain{}|\\
\end{tabular}
\end{center}
at the very top of the main \LaTeX{} file,
in particular \emph{before} the |\documentclass| statement!
The argument of |\childdocmain| should be left empty
(but it must be present).

%%%%%%%%%%%%%%%%%%%%%%%%%%%%%%%%%%%%%%%%
\DescribeMacro{\childdocof}
Furthermore, add the commands
\begin{center}
\begin{tabular}{l}
|\input{childdoc.def}|\\
|\childdocof{|\textit{main}|}|\\
\end{tabular}
\end{center}
at the top of every child file \textit{child}
which is included by |\include{|\textit{child}|}|
from within the main file
(or at least for those files to be compiled individually).
The argument \textit{main} must be the filename of the main file.

There are a couple of
considerations in setting up the main and child documents:

%%%%%%%%%%%%%%%%%%%%%%%%%%%%%%%%%%%%%%%%
\paragraph{Restrictions.}

Please note the following restrictions:
\begin{itemize}
\item
|\childdocmain| must be called with one argument \textit{main}
to ensure compatibility with earlier version of the package.
It must either be empty (|\childdocmain{}|)
or precisely match the filename of the main file in which it is specified.
See \secref{sec:detection} for further information.
\item
The filename \textit{main} must be specified without the |.tex| extension.
\item
The filename \textit{main} is case sensitive
(even in case-insensitive file systems)
due to internal string comparison.
\item
The argument \textit{main} should be fully expanded, it cannot be a macro.
\item
Subdirectories and special characters should be avoided in filenames.
\item
The command |\childdocmain{|\textit{main}|}| must be followed by a whitespace.
It should not be followed immediately by another command
or by a comment mark `|%|'.
This is because the \TeX{} parser reads the token immediately following
the argument of |\childdocmain| and puts it
at the beginning of every child section;
however, a white\-space is ignored.
\end{itemize}

%%%%%%%%%%%%%%%%%%%%%%%%%%%%%%%%%%%%%%%%
\paragraph{Content of Main File.}

It is advisable to place all content in the child files included by |\include|.
Any output contained in the main file will appear in all child documents
unless suppressed manually;
it cannot be suppressed automatically by the |\includeonly| directive
and thus should normally be avoided.
A method to include some content in the main file
by means of conditional processing is described in \secref{sec:conditional}.

%%%%%%%%%%%%%%%%%%%%%%%%%%%%%%%%%%%%%%%%
\paragraph{Page Numbering.}

When only a part of the document is compiled,
the appropriate numbering of pages
(as well as other status parameters)
is determined from the |.aux| files.
The latter contain information from previous passes.
However this information needs to propagate through
all intermediate child documents.
Therefore the page numbering in child documents may well
be inconsistent until the complete document is compiled at least once.

A useful (if unconventional) way to always ensure a consistent
page numbering is to restart the numbering in each child document
and denote the pages by `\textit{child}|.|\textit{page}'
where \textit{child} represents the chapter/section number of the child file.
This can be achieved by the command
|\numberwithin{page}{|\textit{child}|}|
of the \textsf{amsmath} package
where \textit{child} can be |chapter| or |section|
depending on the chosen structuring.
Alternatively, one can modify the macro |\thepage| appropriately
and reset the counter |page| at the start of each child file.

%%%%%%%%%%%%%%%%%%%%%%%%%%%%%%%%%%%%%%%%%%%%%%%%%%%%%%%%%%%%%%%%%%%%%%%%%%%%%%%%
\subsection{Conditional Processing}
\label{sec:conditional}

The package provides a mechanism to compile different versions
of a document. To customise the versions further some conditional processing
can come in handy to distinguish which version is being compiled.
The package provides two macros to describe the compilation context:

%%%%%%%%%%%%%%%%%%%%%%%%%%%%%%%%%%%%%%%%
\DescribeMacro{\ifchilddoc}
The conditional |\ifchilddoc| distinguishes between the compilation of
child documents and the main document:
%
\begin{center}
|\ifchilddoc |\textit{child-code}| |[|\||else |\textit{main-code}]| \||fi|
\end{center}

%%%%%%%%%%%%%%%%%%%%%%%%%%%%%%%%%%%%%%%%
\DescribeMacro{\childdocname}
\DescribeMacro{\childdocjob}
The macro |\childdocname| contains the filename (without extension)
of the main or child file being processed.
Note that |\childdocjob| will always contain the name of the main file.

%%%%%%%%%%%%%%%%%%%%%%%%%%%%%%%%%%%%%%%%
\paragraph{Title Page.}

Conditional processing can be used to include a title or banner page
in the main document when proper precautions are taken.
Importantly, the code in the main file should ensure that the page counter
(as well as other status parameters which are stored in the |.aux| files)
takes the same value after the conditional processing.
Otherwise the page numbers may take divergent values
depending on which part is compiled.

For example, a title page could be declared by:
%
\begin{center}
\begin{tabular}{l}
|\ifchilddoc\||else|\\
|\addtocounter{page}{-1}|\\
\textit{code for title page}\\
|\newpage|\\
|\||fi|
\end{tabular}
\end{center}
%
A banner page for the child documents can be generated by:
%
\begin{center}
\begin{tabular}{l}
|\ifchilddoc|\\
|\addtocounter{page}{-1}|\\
\textit{code for banner page}\\
|\newpage|\\
|\||fi|
\end{tabular}
\end{center}
%
Here one could write a message such as:
\begin{center}
|This is the part \childdocname{} of \childdocjob{}.|
\end{center}

%%%%%%%%%%%%%%%%%%%%%%%%%%%%%%%%%%%%%%%%%%%%%%%%%%%%%%%%%%%%%%%%%%%%%%%%%%%%%%%%
\subsection{Flags}
\label{sec:flags}

The package makes it easy to generate different versions
of the main or child documents.
To this end compilation flags can be defined
and assigned different default values.
They will be particularly useful in conjunction
with the forwarding mechanism described in \secref{sec:forward}.

For example, it may be useful to have a flag |\version|
which can be set to |draft| or |final|.
The document source will contain some conditional code
depending on the value of |\version|.
Suppose further, the flag should default to |final| for the main file
and to |draft| for child files
which is a natural assignment for editing the document.
This is achieved by placing the following code
in the preamble of the main document
(below the |\childdocmain| directive):
%
\begin{center}
\begin{tabular}{l}
|\ifchilddoc|\\
|\providecommand{\version}{draft}|\\
|\||else|\\
|\providecommand{\version}{final}|\\
|\||fi|
\end{tabular}
\end{center}
%
The definition by |\providecommand| makes sure
that previous definitions are not overwritten.
Further statements |\providecommand{\version}{...}|
can thus be added before the above code to override it.

For the main file, one might add a line
(between |\childdocmain| and the above block)
%
\begin{center}
|%\ifchilddoc\||else\providecommand{\version}{draft}\||fi|
\end{center}
%
which can be uncommented to produce a draft version.
Likewise one can add a line to the very top of a child file
(above the |\childdocof{|\textit{main}|}| directive)
%
\begin{center}
|%\providecommand{\version}{final}|
\end{center}
%
which can be uncommented to produce the final version of this child document.

%%%%%%%%%%%%%%%%%%%%%%%%%%%%%%%%%%%%%%%%%%%%%%%%%%%%%%%%%%%%%%%%%%%%%%%%%%%%%%%%
\subsection{Forwarding}
\label{sec:forward}

Different versions of the main or child documents
using compilation flags as described in \secref{sec:flags}
can be (permanently) stored in different files
for convenient compilation, viewing and distribution.
To this end, the package defines a command
to pass on compilation to a different file:

%%%%%%%%%%%%%%%%%%%%%%%%%%%%%%%%%%%%%%%%
\DescribeMacro{\childdocforward}
The command |\childdocforward| redirects processing to
another source file:
%
\begin{center}
\begin{tabular}{l}
|\input{childdoc.def}|\\
|\childdocforward[|\textit{main}|]{|\textit{dest}|}|\\
\end{tabular}
\end{center}
%
The argument \textit{dest} is the destination file
(without extension).
It should be the main file or one of the child files.
Note that further \textsf{childdoc} directives
such as |\childdocof| and |\childdocforward|
in the indicated file will be processed in this form.
The optional argument \textit{main}
passes on directly to the main file \textit{main}
while pretending to compile the child \textit{dest}.
This form behaves as if \textit{dest}
issues |\childdocof{|\textit{main}|}| right away,
and no further \textsf{childdoc} directives will be processed.

%%%%%%%%%%%%%%%%%%%%%%%%%%%%%%%%%%%%%%%%
\DescribeMacro{\...prefix}
In the alternative form |\childdocforwardprefix|,
%
\begin{center}
\begin{tabular}{l}
|\input{childdoc.def}|\\
|\childdocforwardprefix[|\textit{main}|]{|\textit{prefix}|}{|\textit{dest}|}|
\end{tabular}
\end{center}
%
the destination file is determined by a pattern
depending on the current file:
To make this work, the current file must be called
`{\textit{prefix}\hspace{0.2em}\textit{suffix}}'
with \textit{prefix} matching precisely the argument.
Processing is then passed on to the file
`{\textit{dest}\hspace{0.2em}\textit{suffix}}'.
Surely, the same effect is achieved by
directly specifying the
argument `{\textit{dest}\hspace{0.2em}\textit{suffix}}'
in the first form.
However, that requires to set up a different file
for each child. With the alternative form of the command
all these files can have exactly the same content
which simplifies setting them up and maintaining them.

For example, the following file |draft.tex|
with a compilation flag |\version| as described in \secref{sec:flags}
compiles the main document as a draft:
%
\begin{center}
\begin{tabular}{l}
|\def\version{draft}|\\
|\input{childdoc.def}|\\
|\childdocforward{|\textit{main}|}|
\end{tabular}
\end{center}
%
Likewise, the following files |final|\textit{nn}|.tex|
compile the final version of the child document
|child|\textit{nn}|.tex|:
%
\begin{center}
\begin{tabular}{l}
|\def\version{final}|\\
|\input{childdoc.def}|\\
|\childdocforwardprefix{final}{child}|
\end{tabular}
\end{center}
%

Note that when several versions of a main file and/or of each child file
are to be generated, it may be convenient to set up a |Makefile| or
shell script to automatise the process.

%%%%%%%%%%%%%%%%%%%%%%%%%%%%%%%%%%%%%%%%%%%%%%%%%%%%%%%%%%%%%%%%%%%%%%%%%%%%%%%%
\subsection{Command Line Processing}
\label{sec:commandline}

The effect of redirection files can also be achieved by invoking
the \LaTeX{} compiler with a more elaborate command line.
Most conveniently this should be done as part
of a shell script or a |Makefile|.

When using \textsf{childdoc} in the main file, the following
command lines effectively perform a redirection
(note that depending on the shell being used,
backslashes may have to be doubled: `|\|' $\to$ `|\\|'):
%
\begin{center}
|... -jobname "|\textit{target}|" |\\|"|[\textit{flags}]%
|\input{childdoc.def}\childdocforward[|\textit{main}|]{|\textit{dest}|}"|
\end{center}
%
Here \textit{target} is the name of the output file,
\textit{main} is the name of the main file
and \textit{dest} is the name of the main or child file to be processed
(all filenames without extensions).
The optional argument \textit{main} can be omitted
if \textit{main} matches \textit{dest}.
Optionally, compilation \textit{flags} can be defined via |\def| commands.
This command line makes the \TeX{} engine believe
it is compiling the file \textit{target}
whose content is specified as the latter parameter.
The provided code then forwards the processing to
\textit{main} or \textit{dest} as described in \secref{sec:forward}.

%%%%%%%%%%%%%%%%%%%%%%%%%%%%%%%%%%%%%%%%%%%%%%%%%%%%%%%%%%%%%%%%%%%%%%%%%%%%%%%%
\subsection{Include by Input}
\label{sec:input}

Including child documents by |\include| has some restrictions by design.
Most notably, the content of a child document always occupies
its own set of pages; pages cannot be shared between child documents.
Usually, this behaviour makes perfect sense
because each child document contain an essential part of the document.
However, in some situations it may be desirable to compose
a document from a collection of parts
without having mandatory page breaks between then.
For this case, the package
provides a mechanism to include parts
by |\input| which can also be processed individually.
However, by construction this mechanism
requires manual handling of the content to be output.

%%%%%%%%%%%%%%%%%%%%%%%%%%%%%%%%%%%%%%%%
\DescribeMacro{\ifchilddocmanual}
The main file should be prepared as usual, see \secref{sec:include}.
However, the document body must make a distinction
between processing of an individual part and of the main document, e.g.:
%
\begin{center}
\begin{tabular}{l}
|\ifchilddocmanual|\\
|\input{\childdocname}|\\
|\||else|\\
\textit{document body with }|\input{|\textit{part}|}|\\
|\||fi|
\end{tabular}
\end{center}
%
The conditional |\ifchilddocmanual| is true whenever
a part to be included by |\input| is being compiled,
and the name of the part is stored in |\childdocname|.

%%%%%%%%%%%%%%%%%%%%%%%%%%%%%%%%%%%%%%%%
\DescribeMacro{\childdocby}
Each part to be included by |\input| should start with:
%
\begin{center}
\begin{tabular}{l}
|\input{childdoc.def}|\\
|\childdocby{|\textit{main}|}|\\
\end{tabular}
\end{center}
%
The directive |\childdocby| is similar to |\childdocof|
described in \secref{sec:include},
but the subsequent selection of content must be done manually.
To that end, both |\ifchilddoc| and |\ifchilddocmanual|
will be true upon processing of a part,
and the name of the part is stored in |\childdocname|.
Note that |\jobname| will be set to the filename of the current part
so that each part receives an individual |.aux| file
that does not interfere with the |.aux| file(s) of the main document.
This behaviour can be altered by the alternative form
|\childdocby[*]{|\textit{main}|}| (with a non-empty optional argument)
which uses the |.aux| file of the main document
by setting |\jobname| to \textit{main}.

%%%%%%%%%%%%%%%%%%%%%%%%%%%%%%%%%%%%%%%%%%%%%%%%%%%%%%%%%%%%%%%%%%%%%%%%%%%%%%%%
\subsection{Driver Development}
\label{sec:driver}

The \textsf{childdoc} mechanism can also be use for the development
of definition files such as \LaTeX{} styles or classes.
This case differs from the above setup with multiple parts
included by |\include| in that no |\includeonly| should be invoked.
This can be achieved by starting the include file
(before |\ProvidesPackage|) with:
%
\begin{center}
\begin{tabular}{l}
|\input{childdoc.def}|\\
|\childdocforward{|\textit{main}|}|\\
\end{tabular}
\end{center}
%
or alternatively with:
%
\begin{center}
\begin{tabular}{l}
|\input{childdoc.def}|\\
|\childdocby{|\textit{main}|}|\\
\end{tabular}
\end{center}
%
Both forms have slightly different effects as described above.
The main file is prepared as usual, see \secref{sec:include}.

%%%%%%%%%%%%%%%%%%%%%%%%%%%%%%%%%%%%%%%%%%%%%%%%%%%%%%%%%%%%%%%%%%%%%%%%%%%%%%%%
\subsection{Legacy Detection}
\label{sec:detection}

The directive |\childdocmain| in the main file can detect
whether the complete document or merely a child is to be compiled
even without using the directive |\childdocof|.
This method is deprecated because it is less robust
and there is no compelling reason to use it;
it is merely provided for backward compatibility
and it may be removed in future versions.

If the detection mechanism is to be used,
it is mandatory to correctly specify
the filename of the main file as the argument of |\childdocmain|:
%
\begin{center}
\begin{tabular}{l}
|\input{childdoc.def}|\\
|\childdocmain{|\textit{main}|}|\\
\end{tabular}
\end{center}
%
If |\jobname| does not match the argument \textit{main} of |\childdocmain|,
it is assumed that |\jobname| points to the child file to be compiled.
When using |\childdocmain| with the main file specified as argument,
it suffices to start a child file
with just |\input{|\textit{main}|}|
without loading of the package and using |\childdocof|.
If instead all processing is done
with the appropriate \textsf{childdoc} directives,
the argument of \textit{main} of |\childdocmain| can be empty.

An alternative version of the command line processing described
in \secref{sec:commandline} using the detection mechanism reads:
%
\begin{center}
|... -jobname "|\textit{target}|" "|[\textit{flags}]%
[|\def\jobname{|\textit{dest}|}|]|\input{|\textit{main}|}"|
\end{center}

%%%%%%%%%%%%%%%%%%%%%%%%%%%%%%%%%%%%%%%%%%%%%%%%%%%%%%%%%%%%%%%%%%%%%%%%%%%%%%%%
\subsection{Manual Code}
\label{sec:manual}

In case one cannot be certain whether the definitions file |childdoc.def|
is installed on the target \TeX{} distribution
and one prefers not to ship it,
it is conceivable to paste a few relevant commands into the sources.

To that end, drop all statements |\input{childdoc.def}|
and perform the replacements as outlined below.
Instead of |\childdocmain{|\textit{main}|}| add the following code
to the top of the main file:
%
\begin{center}
\begin{tabular}{l}
|\||ifdefined\childdocname\endinput\||fi\newif\ifchilddoc|\\
|\edef\childdocname{\scantokens\expandafter{\jobname\noexpand}}|\\
|\def\childdocmain{|\textit{main}|}\||ifx\childdocmain\childdocname\||else|\\
|\childdoctrue\includeonly{\childdocname}\let\jobname\childdocmain\||fi|\\
\end{tabular}
\end{center}
%
Instead of |\childdocof{|\textit{main}|}| just include the main file
at the top of each child file:
%
\begin{center}
|\input{|\textit{main}|}|
\end{center}
%
A simple redirection |\childdocforward{|\textit{dest}|}| is achieved by:
%
\begin{center}
|\def\jobname{|\textit{dest}|}\input{\jobname}|
\end{center}
%
The redirection with prefix
|\childdocforwardprefix[|\textit{prefix}|]{|\textit{dest}|}|
is accomplished by:
%
\begin{center}
\begin{tabular}{l}
|{\edef\jobname{\scantokens\expandafter{\jobname\noexpand}}|\\
|\def\redirectjob |\textit{prefix}|#1~~~{\gdef\jobname{|\textit{dest}|#1}}|\\
|\expandafter\redirectjob\jobname~~~}\input{\jobname}|
\end{tabular}
\end{center}

In an alternative approach,
child documents can be compiled by a specific command line
without additional code or specific definitions:
%
\begin{center}
|... -jobname "|\textit{target}|" "|[\textit{flags}]%
|\includeonly{|\textit{dest}|}\input{|\textit{main}|}"|
\end{center}
%

%%%%%%%%%%%%%%%%%%%%%%%%%%%%%%%%%%%%%%%%%%%%%%%%%%%%%%%%%%%%%%%%%%%%%%%%%%%%%%%%
%%%%%%%%%%%%%%%%%%%%%%%%%%%%%%%%%%%%%%%%%%%%%%%%%%%%%%%%%%%%%%%%%%%%%%%%%%%%%%%%
\section{Information}

%%%%%%%%%%%%%%%%%%%%%%%%%%%%%%%%%%%%%%%%%%%%%%%%%%%%%%%%%%%%%%%%%%%%%%%%%%%%%%%%
\subsection{Copyright}

Copyright \copyright{} 2017--2018 Niklas Beisert

This work may be distributed and/or modified under the
conditions of the \LaTeX{} Project Public License, either version 1.3
of this license or (at your option) any later version.
The latest version of this license is in
  \url{http://www.latex-project.org/lppl.txt}
and version 1.3 or later is part of all distributions of \LaTeX{}
version 2005/12/01 or later.

This work has the LPPL maintenance status `maintained'.

The Current Maintainer of this work is Niklas Beisert.

This work consists of the files |README.txt|, |childdoc.ins| and |childdoc.dtx|
as well as the derived files |childdoc.def|, |cdocsamp.tex|
with |cdocsch1.tex|, |cdocsch2.tex|, |cdocspt3.tex|, |cdocspt4.tex|,
|cdocsdrf.tex|, |cdocsfn1.tex|, |cdocsfn2.tex|
as well as |childdoc.pdf|.

%%%%%%%%%%%%%%%%%%%%%%%%%%%%%%%%%%%%%%%%%%%%%%%%%%%%%%%%%%%%%%%%%%%%%%%%%%%%%%%%
\subsection{Files and Installation}

The package consists of the files:
%
\begin{center}
\begin{tabular}{ll}
    |README.txt|   & readme file \\
    |childdoc.ins| & installation file \\
    |childdoc.dtx| & source file \\
    |childdoc.def| & definition file \\
    |cdocsamp.tex| & sample main file \\
    |cdocsch1.tex| & sample include file \\
    |cdocsch2.tex| & sample include file \\
    |cdocspt3.tex| & sample part file \\
    |cdocspt4.tex| & sample part file \\
    |cdocsdrf.tex| & sample redirection file \\
    |cdocsfn1.tex| & sample redirection file \\
    |cdocsfn2.tex| & sample redirection file \\
    |childdoc.pdf| & manual
\end{tabular}
\end{center}
%
The distribution consists of the files
|README.txt|, |childdoc.ins| and |childdoc.dtx|.
%
\begin{itemize}
\item
Run (pdf)\LaTeX{} on |childdoc.dtx|
to compile the manual |childdoc.pdf| (this file).
\item
Run \LaTeX{} on |childdoc.ins| to create the definitions file |childdoc.def|
and the sample |cdocsamp.tex| with include files
|cdocsch1.tex|, |cdocsch2.tex|, |cdocspt3.tex|, |cdocspt4.tex|,
|cdocsdrf.tex|, |cdocsfn1.tex|, |cdocsfn2.tex|.
Then copy the file |childdoc.def| to an appropriate directory of your \LaTeX{}
distribution, e.g.\ \textit{texmf-root}|/tex/latex/childdoc|.
\end{itemize}

%%%%%%%%%%%%%%%%%%%%%%%%%%%%%%%%%%%%%%%%%%%%%%%%%%%%%%%%%%%%%%%%%%%%%%%%%%%%%%%%
\subsection{Related CTAN Packages}

There are several other packages which offer a similar functionality:
%
\begin{itemize}
\item
The packages
\href{http://ctan.org/pkg/docmute}{\textsf{docmute}},
\href{http://ctan.org/pkg/includex}{\textsf{includex}} and
\href{http://ctan.org/pkg/standalone}{\textsf{standalone}}
provide commands to include only the document body of
a child file thus allowing both files to be compiled individually.
\item
The packages \href{http://ctan.org/pkg/subdocs}{\textsf{subdocs}}
and \href{http://ctan.org/pkg/subfiles}{\textsf{subfiles}}
provide structures in which the main and child documents can be
encapsulated and allowing them to be compiled individually.
The inclusion mechanism is different from the conventional |\include|.
\item
The package \href{http://ctan.org/pkg/combine}{\textsf{combine}}
is an elaborate solution to combine several documents into one.
\end{itemize}
%
See also the CTAN topic \href{http://ctan.org/topic/subdocs}{\textsf{subdocs}}
for further related packages.
The present package differs from the above solutions in that
a document structure constructed with the conventional |\include| mechanism
just needs two extra commands at the top of every file
such that all constituent files can be compiled individually.

%%%%%%%%%%%%%%%%%%%%%%%%%%%%%%%%%%%%%%%%%%%%%%%%%%%%%%%%%%%%%%%%%%%%%%%%%%%%%%%%
%\subsection{Feature Suggestions}
%
%The following is a list of features which may be useful for future
%versions of this package:
%%
%\begin{itemize}
%\item
%\ldots
%\end{itemize}

%%%%%%%%%%%%%%%%%%%%%%%%%%%%%%%%%%%%%%%%%%%%%%%%%%%%%%%%%%%%%%%%%%%%%%%%%%%%%%%%
\subsection{Revision History}

%%%%%%%%%%%%%%%%%%%%%%%%%%%%%%%%%%%%%%%%
\paragraph{v2.0:} 2018/12/30

\begin{itemize}
\item
immediate forward processing
\item
added |\childdocby| mechanism
\item
manual restructured
\end{itemize}

%%%%%%%%%%%%%%%%%%%%%%%%%%%%%%%%%%%%%%%%
\paragraph{v1.6:} 2018/01/17

\begin{itemize}
\item
application for development of include files
\item
corrections to manual
\end{itemize}

%%%%%%%%%%%%%%%%%%%%%%%%%%%%%%%%%%%%%%%%
\paragraph{v1.5:} 2017/05/21

\begin{itemize}
\item
more complete structuring introduced
\item
|\childdocof| introduced
\item
|\childdoc| renamed to |\childdocmain|
\item
|\childredirect| renamed to |\childdocforward| and |\childdocforwardprefix|
and functionality expanded
\end{itemize}

%%%%%%%%%%%%%%%%%%%%%%%%%%%%%%%%%%%%%%%%
\paragraph{v1.0:} 2017/04/27

\begin{itemize}
\item
manual and install package
\item
first version published on CTAN
\end{itemize}

%%%%%%%%%%%%%%%%%%%%%%%%%%%%%%%%%%%%%%%%
\paragraph{v0.6:} 2017/04/26

\begin{itemize}
\item
redirection mechanism added
\end{itemize}

%%%%%%%%%%%%%%%%%%%%%%%%%%%%%%%%%%%%%%%%
\paragraph{v0.5:} 2017/04/26

\begin{itemize}
\item
functionality in definition file
\end{itemize}


%%%%%%%%%%%%%%%%%%%%%%%%%%%%%%%%%%%%%%%%%%%%%%%%%%%%%%%%%%%%%%%%%%%%%%%%%%%%%%%%
%%%%%%%%%%%%%%%%%%%%%%%%%%%%%%%%%%%%%%%%%%%%%%%%%%%%%%%%%%%%%%%%%%%%%%%%%%%%%%%%
%%%%%%%%%%%%%%%%%%%%%%%%%%%%%%%%%%%%%%%%%%%%%%%%%%%%%%%%%%%%%%%%%%%%%%%%%%%%%%%%
\appendix

\settowidth\MacroIndent{\rmfamily\scriptsize 000\ }

 \DocInput{childdoc.dtx}

\end{document}
%</driver>
% \fi
%
% %%%%%%%%%%%%%%%%%%%%%%%%%%%%%%%%%%%%%%%%%%%%%%%%%%%%%%%%%%%%%%%%%%%%%%%%%%%%%%
% %%%%%%%%%%%%%%%%%%%%%%%%%%%%%%%%%%%%%%%%%%%%%%%%%%%%%%%%%%%%%%%%%%%%%%%%%%%%%%
% \section{Sample}
%\iffalse
%<*samplemain>
%\fi
%
% The following presents a sample document
% with two chapters, two parts, a title page,
% a compile flag as well as three forwarding files to set the flag.
% It consists of eight |.tex| files:
% \begin{center}
% \begin{tabular}{ll}
% |cdocsamp.tex|&main file\\
% |cdocsch1.tex|&include file for chapter 1\\
% |cdocsch2.tex|&include file for chapter 2\\
% |cdocspt3.tex|&include file for part 3\\
% |cdocspt4.tex|&include file for part 4\\
% |cdocsdrf.tex|&forwarding file for main file in draft mode\\
% |cdocsfi1.tex|&forwarding file for final version of chapter 1\\
% |cdocsfi2.tex|&forwarding file for final version of chapter 2\\
% \end{tabular}
% \end{center}
% Each of the eight files can be compiled directly by the \LaTeX{} compiler.
%
% %%%%%%%%%%%%%%%%%%%%%%%%%%%%%%%%%%%%%%
% \paragraph{Main File.}
%
% The main file is called |cdocsamp.tex|.
%
% Load the \textsf{childdoc} definitions and
% declare the filename for the main document:
%    \begin{macrocode}
\input{childdoc.def}
\childdocmain{}
%    \end{macrocode}

% Optional override for |\version| flag:
%    \begin{macrocode}
%%\ifchilddoc\else\providecommand{\version}{draft}\fi
%    \end{macrocode}

% Define the default values for the |\version| flag
% (|final| for the main file and |draft| for childs):
%    \begin{macrocode}
\ifchilddoc
\providecommand{\version}{draft}
\else
\providecommand{\version}{final}
\fi
%    \end{macrocode}

% Load the standard document class:
%    \begin{macrocode}
\documentclass[12pt]{article}
%    \end{macrocode}

% Start the document body:
%    \begin{macrocode}
\begin{document}
%    \end{macrocode}

% Declare a title page.
% Print title, part of document being processed and version flag:
%    \begin{macrocode}
\addtocounter{page}{-1}
\begin{center}
{\LARGE\bfseries{}childdoc example\par}
\vspace{1cm}
\ifchilddoc
\ifchilddocmanual part\else chapter\fi:
`\childdocname' of `\childdocjob'\par
\else
main document: `\childdocjob'\par
\fi
version: \version\par
\end{center}
\newpage
%    \end{macrocode}

% Manually include selected file,
% otherwise process as usual:
%    \begin{macrocode}
\ifchilddocmanual
\section*{part `\childdocname'}
\input{\childdocname}
\else
%    \end{macrocode}

% Include the two chapters:
%    \begin{macrocode}
\include{cdocsch1}
\include{cdocsch2}
%    \end{macrocode}

% Include the two parts unless only chapters should be displayed:
%    \begin{macrocode}
\ifchilddoc\else
\section{part three}
\input{cdocspt3}
\section{part four}
\input{cdocspt4}
\fi
%    \end{macrocode}

% Process as usual until here:
%    \begin{macrocode}
\fi
%    \end{macrocode}

% End of document body:
%    \begin{macrocode}
\end{document}
%    \end{macrocode}
%\iffalse
%</samplemain>
%\fi
%
% %%%%%%%%%%%%%%%%%%%%%%%%%%%%%%%%%%%%%%
% \paragraph{Chapter Include Files.}
%
% The include files are called |cdocsch1.tex| and |cdocsch2.tex|.
%
%\iffalse
%<*samplechap1|samplechap2>
%\fi

% Optional override for |\version| flag:
%    \begin{macrocode}
%%\providecommand{\version}{final}
%    \end{macrocode}

% Include the main document:
%    \begin{macrocode}
\input{childdoc.def}
\childdocof{cdocsamp}
%    \end{macrocode}

%\iffalse
%</samplechap1|samplechap2>
%\fi
%
%\iffalse
%<*samplechap1>
%\fi
% Some text for chapter 1:
%    \begin{macrocode}
\section{one}
some text in chapter one
%    \end{macrocode}

%\iffalse
%</samplechap1>
%\fi
% Some text for chapter 2:
%\iffalse
%<*samplechap2>
%\fi
%    \begin{macrocode}
\section{two}
more text in chapter two
%    \end{macrocode}

%\iffalse
%</samplechap2>
%\fi
%
% %%%%%%%%%%%%%%%%%%%%%%%%%%%%%%%%%%%%%%
% \paragraph{Part Include Files.}
%
% The include files are called |cdocspt3.tex| and |cdocspt4.tex|.
%
%\iffalse
%<*samplepart3|samplepart4>
%\fi

% Optional override for |\version| flag:
%    \begin{macrocode}
%%\providecommand{\version}{final}
%    \end{macrocode}

% Include the main document:
%    \begin{macrocode}
\input{childdoc.def}
\childdocby{cdocsamp}
%    \end{macrocode}

%\iffalse
%</samplepart3|samplepart4>
%\fi
%
%\iffalse
%<*samplepart3>
%\fi
% Some text for part 3:
%    \begin{macrocode}
some text in part three
%    \end{macrocode}

%\iffalse
%</samplepart3>
%\fi
% Some text for part 4:
%\iffalse
%<*samplepart4>
%\fi
%    \begin{macrocode}
more text in part four
%    \end{macrocode}

%\iffalse
%</samplepart4>
%\fi
%
% %%%%%%%%%%%%%%%%%%%%%%%%%%%%%%%%%%%%%%
% \paragraph{Forwarding for a Complete Draft.}
%
% The following forwarding file |cdocsdrf.tex|
% compiles the main document in draft mode:
%\iffalse
%<*sampledraft>
%\fi
%    \begin{macrocode}
\def\version{draft}
\input{childdoc.def}
\childdocforward{cdocsamp}
%    \end{macrocode}

%\iffalse
%</sampledraft>
%\fi
%
% %%%%%%%%%%%%%%%%%%%%%%%%%%%%%%%%%%%%%%
% \paragraph{Forwarding for Final Version of the Chapters.}
%
% The following forwarding files |cdocsfn1.tex| and |cdocsfn2.tex|
% (with identical content)
% compile the final versions of the child documents
% |cdocsch1.tex| and |cdocsch2.tex|, respectively:
%\iffalse
%<*samplefinal>
%\fi
%    \begin{macrocode}
\def\version{final}
\input{childdoc.def}
\childdocforwardprefix[cdocsamp]{cdocsfn}{cdocsch}
%    \end{macrocode}

%\iffalse
%</samplefinal>
%\fi
%
% %%%%%%%%%%%%%%%%%%%%%%%%%%%%%%%%%%%%%%
% \paragraph{Command Line Processing.}
%
% The following three command lines generate the output files
% |cdocscld|, |cdocscl1| and |cdocscl2|
% which should be identical to
% |cdocsdrf|, |cdocsch1| and |cdocsfn2|, respectively:
% \begin{center}
% \begin{tabular}{l}
% |latex -jobname cdocscld \|\\
% |  "\def\version{draft}\input{childdoc.def}\childdocforward{cdocsamp}"|\\
% |latex -jobname cdocscl1 \|\\
% |  "\input{childdoc.def}\childdocforward[cdocsamp]{cdocsch1}"|\\
% |latex -jobname cdocscl2 \|\\
% |  "\def\version{final}\input{childdoc.def}\childdocforward{cdocsch2}"|
% \end{tabular}
% \end{center}
% Note that the trailing backslash on each first line
% merely continues the input to the second line
% (for convenient cut ant paste).
% Furthermore, the command |latex| can be replaced by any
% of its alternative versions such as |pdflatex|.
%
% %%%%%%%%%%%%%%%%%%%%%%%%%%%%%%%%%%%%%%%%%%%%%%%%%%%%%%%%%%%%%%%%%%%%%%%%%%%%%%
% %%%%%%%%%%%%%%%%%%%%%%%%%%%%%%%%%%%%%%%%%%%%%%%%%%%%%%%%%%%%%%%%%%%%%%%%%%%%%%
% \section{Implementation}
%\iffalse
%<*package>
%\fi
%
% This section describes the definitions file |childdoc.def|.

% The definitions cannot be loaded using |\usepackage| or |\RequirePackage|
% which has a mechanism to prevent loading a style file more than once.
% When loading the definitions by means of |\input|
% multiple instances have to be prevented manually:
%\iffalse
%This code needs to be before the `\ProvidesFile' directive
%which is defined at the beginning of this file.
%Therefore it is also placed there and commented out here.
%</package>
%<*discard>
%\fi
%    \begin{macrocode}
\ifdefined\childdocmain\endinput\fi
%    \end{macrocode}
%\iffalse
%</discard>
%<*package>
%\fi
%
% \macro{\ifchilddoc}
% \macro{\ifchilddocmanual}
% The conditional |\ifchilddoc| tells whether a
% child (true) or main (false) document is being compiled.
% The conditional |\ifchilddocmanual| tells whether
% the |\includeonly| mechanism is used (false) or
% the selection of child files must be performed manually (true).
% The definitions initialise to false:
%    \begin{macrocode}
\newif\ifchilddoc
\newif\ifchilddocmanual
%    \end{macrocode}

% \macro{\childdocname}
% \macro{\childdocjob}
% The macro |\childdocname| stores the name of the main document
% to be compiled. The macro |\childdocjob| stores the name of
% the document on which the \LaTeX{} compiler was originally invoked.
% The content of |\jobname| cannot be compared
% to filenames specified in the source due to different catcodes.
% The following code rescans |\jobname|, stores the result
% in |\childdocname| and saves a copy in |\childdocjob|:
%    \begin{macrocode}
\edef\childdocname{\scantokens\expandafter{\jobname\noexpand}}
\let\childdocjob\childdocname
%    \end{macrocode}

% \macro{\childdocdisable}
% The macro |\childdocdisable| prevents the main file
% from being processed more than once.
% At this stage, the main document command |\childdocmain|
% is assumed to be called once again where it should do nothing.
% Any subsequent call to it should prevent
% a secondary processing of the main document
% It overwrites the forwarding commands
% |\childdocof| and |\childdocforward|
% with empty macros to prevent further inclusions of the main document:
%    \begin{macrocode}
\newcommand{\childdocdisable}
{
  \renewcommand{\childdocmain}[1]{\renewcommand{\childdocmain}[1]{\endinput}}
  \renewcommand{\childdocof}[1]{}
  \renewcommand{\childdocby}[2][]{}
  \renewcommand{\childdocforward}[2][]{}
  \renewcommand{\childdocdisable}{}
}
%    \end{macrocode}

% \macro{\childdocmain}
% The macro |\childdocmain| is to be called at the top of the main file
% with nothing or the main filename (without extension) as argument.
% First, it breaks loops.
% If the argument is not empty and does not match |\childdocname|
% (which is set by the first inclusion of |childdoc.def|),
% |\ifchilddoc| is set to true, |\includeonly| is applied to the child file
% and |\jobname| is set to the main file
% (for proper handling of |.aux| files):
%    \begin{macrocode}
\newcommand{\childdocmain}[1]
{
  \childdocdisable\childdocmain{}
  \if?#1?\else
    \begingroup
      \def\childdoctmp{#1}
      \ifx\childdoctmp\childdocname
        \def\childdoctmp{}
      \else
        \def\childdoctmp
        {
          \childdoctrue
          \includeonly{\childdocname}
          \def\childdocjob{#1}
          \def\jobname{#1}
        }
      \fi
      \expandafter
    \endgroup
    \childdoctmp
  \fi
}
%    \end{macrocode}

% \macro{\childdocof}
% The command |\childdocof| redirects
% compilation to the main file |#1|.
%    \begin{macrocode}
\newcommand{\childdocof}[1]
{
  \childdocdisable
  \childdoctrue
  \includeonly{\childdocname}
  \def\jobname{#1}
  \def\childdocjob{#1}
  \input{#1}
}
%    \end{macrocode}

% \macro{\childdocby}
% The command |\childdocby| ....
%    \begin{macrocode}
\newcommand{\childdocby}[2][]
{
  \childdocdisable
  \childdoctrue
  \childdocmanualtrue
  \if?#1?\else
    \def\jobname{#2}
  \fi
  \def\childdocjob{#2}
  \input{#2}
  \endinput
}
%    \end{macrocode}

% \macro{\childdocforward}
% The command |\childdocforward| redirects
% compilation to the main file or
% (if the optional argument is given) a child file.
% Parameters are set as if the main file
% or a child file starting with |\childdocof| was compiled.
% Then compilation is handed over to the main file:
%    \begin{macrocode}
\newcommand{\childdocforward}[2][]
{
  \begingroup
    \if?#1?
      \def\childdoctmp
      {
        \def\childdocname{#2}
        \def\childdocjob{#2}
        \def\jobname{#2}
        \input{#2}
        \endinput
      }
    \else
      \def\childdoctmp
      {
        \childdocdisable
        \def\childdocname{#2}
        \childdoctrue
        \includeonly{#2}
        \def\childdocjob{#1}
        \def\jobname{#1}
        \input{#1}
        \endinput
      }
    \fi
    \expandafter
  \endgroup
  \childdoctmp
}
%    \end{macrocode}

% \macro{\childdocforwardprefix}
% The command |\childdocforwardprefix| redirects
% compilation to the main or a child file by means of a pattern.
% The prefix |#1| in the current filename is replaced by |#2|
% and the suffix of the current filename is kept
% (it is assumed that the filename does not contain the substring `|~~~|'
% which is used as a delimiter).
% Compilation is handed over to the new file by |\childdocforward|:
%    \begin{macrocode}
\newcommand{\childdocforwardprefix}[3][]
{
  \begingroup
    \def\childdocextract #2##1~~~{\def\childdoctmp{\childdocforward[#1]{#3##1}}}
    \expandafter\childdocextract\childdocname~~~
    \expandafter
  \endgroup
  \childdoctmp
}
%    \end{macrocode}

% \macro{\childdoc}
% The deprecated macro |\childdoc| is a legacy version of |\childdocmain|:
%    \begin{macrocode}
\newcommand{\childdoc}{\childdocmain}
%    \end{macrocode}

% \macro{\childdocredirect}
% The deprecated macro |\childdocredirect| is a legacy version
% of |\childdocforward| and |\childdocforwardprefix|:
%    \begin{macrocode}
\newcommand{\childdocredirect}[2][]
{
  \begingroup
    \if?#1?
      \def\childdoctmp{\childdocforward{#2}}
    \else
      \def\childdoctmp{\childdocforwardprefix{#1}{#2}}
    \fi
    \expandafter
  \endgroup
  \childdoctmp
}
%    \end{macrocode}

%\iffalse
%</package>
%\fi
%
\endinput
|\\
|\childdocforward{|\textit{main}|}|\\
\end{tabular}
\end{center}
%
or alternatively with:
%
\begin{center}
\begin{tabular}{l}
|% \iffalse
%
% childdoc.dtx Copyright (C) 2017-2018 Niklas Beisert
%
% This work may be distributed and/or modified under the
% conditions of the LaTeX Project Public License, either version 1.3
% of this license or (at your option) any later version.
% The latest version of this license is in
%   http://www.latex-project.org/lppl.txt
% and version 1.3 or later is part of all distributions of LaTeX
% version 2005/12/01 or later.
%
% This work has the LPPL maintenance status `maintained'.
%
% The Current Maintainer of this work is Niklas Beisert.
%
% This work consists of the files childdoc.dtx and childdoc.ins
% and the derived files childdoc.def and cdocsamp.tex with
% cdocsch1.tex, cdocsch2.tex, cdocsdrf.tex, cdocsfn1.tex, cdocsfn2.tex.
%
%<package>\ifdefined\childdocmain\endinput\fi
%<package>\ProvidesFile{childdoc.def}[2018/12/30 v2.0 child document driver]
%<samplemain>\ProvidesFile{cdocsamp.tex}[2018/12/30 v2.0 sample for childdoc]
%<*driver>
%\ProvidesFile{childdoc.drv}[2018/12/30 v2.0 childdoc reference manual file]
\PassOptionsToClass{10pt,a4paper}{article}
\documentclass{ltxdoc}

\usepackage[margin=35mm]{geometry}
\usepackage{hyperref}
\usepackage{hyperxmp}
\usepackage[usenames]{color}

\hypersetup{colorlinks=true}
\hypersetup{pdfstartview=FitH}
\hypersetup{pdfpagemode=UseNone}
\hypersetup{pdfsource={}}
\hypersetup{pdflang={en-UK}}
\hypersetup{pdfcopyright={Copyright 2017-2018 Niklas Beisert.
  This work may be distributed and/or modified under the
  conditions of the LaTeX Project Public License, either version 1.3
  of this license or (at your option) any later version.}}
\hypersetup{pdflicenseurl={http://www.latex-project.org/lppl.txt}}
\hypersetup{pdfcontactaddress={ETH Zurich, ITP, HIT K,
  Wolfgang-Pauli-Strasse 27}}
\hypersetup{pdfcontactpostcode={8093}}
\hypersetup{pdfcontactcity={Zurich}}
\hypersetup{pdfcontactcountry={Switzerland}}
\hypersetup{pdfcontactemail={nbeisert@itp.phys.ethz.ch}}
\hypersetup{pdfcontacturl={http://people.phys.ethz.ch/\xmptilde nbeisert/}}

\newcommand{\secref}[1]{\hyperref[#1]{section \ref*{#1}}}

\parskip1ex
\parindent0pt
\let\olditemize\itemize
\def\itemize{\olditemize\parskip0pt}

\begin{document}

\title{The \textsf{childdoc} Package}
\hypersetup{pdftitle={The childdoc Package}}
\author{Niklas Beisert\\[2ex]
  Institut f\"ur Theoretische Physik\\
  Eidgen\"ossische Technische Hochschule Z\"urich\\
  Wolfgang-Pauli-Strasse 27, 8093 Z\"urich, Switzerland\\[1ex]
  \href{mailto:nbeisert@itp.phys.ethz.ch}
  {\texttt{nbeisert@itp.phys.ethz.ch}}}
\hypersetup{pdfauthor={Niklas Beisert}}
\hypersetup{pdfsubject={Manual for the LaTeX2e Package childdoc}}
\date{30 December 2018, \textsf{v2.0}}
\maketitle

\begin{abstract}\noindent
\textsf{childdoc} is a \LaTeXe{} package
that enables the direct compilation
of document sections included by |\include|
to individual files.
\end{abstract}

\begingroup
\parskip0ex
\tableofcontents
\endgroup

%%%%%%%%%%%%%%%%%%%%%%%%%%%%%%%%%%%%%%%%%%%%%%%%%%%%%%%%%%%%%%%%%%%%%%%%%%%%%%%%
%%%%%%%%%%%%%%%%%%%%%%%%%%%%%%%%%%%%%%%%%%%%%%%%%%%%%%%%%%%%%%%%%%%%%%%%%%%%%%%%
\section{Introduction}

\LaTeX{} provides a mechanism to structure a large document (such as a book)
into a main file and several child files (containing the chapters)
using the |\include| command.
This mechanism is beneficial for documents
which span hundreds of pages in order to
make the source file(s) more manageable.
Moreover, compilation can be restricted to
selected child files by means of the |\includeonly| command.
The latter feature can be used to reduce the compilation time while editing
(this was significantly more useful in the earlier days of \LaTeX{})
or to generate a smaller document which is easier to navigate.
Another application of |\includeonly| is to generate
documents consisting of selected parts of the complete document.

However, there are a few drawbacks of the plain |\include| mechanism:
\begin{itemize}
\item
The child files cannot be compiled on their own,
they can only be compiled via the main file.
A naive editing environment
(such as a text editor with an option
to have the current file processed by \LaTeX)
may require one to switch to the main file before compiling;
attempting to compile the child file produces errors.
\item
The main file must be modified (each time)
to adjust the |\includeonly| command
to the present needs. This easily leaves the main file in a messy state.
\item
The generated document will always carry the filename
of the main document. This is inconvenient if
several child files are to be compiled and
to be kept for distribution.
\end{itemize}

The present package provides a simple interface
to make child files individually compilable by \LaTeX{}.
Compiling a child file then has the same effect as compiling
the main file with an |\includeonly| command
to select the appropriate child.
Moreover the generated document will carry the name of the child
rather than the main file.
This resolves all three above issues.

This feature is meant to make the editing of books,
thesis documents and lecture notes somewhat more convenient.
However, the package can also be used efficiently for
composing a series of documents (such as exercise sheets)
which are typically distributed individually.
It then assists the author in generating the individual documents
(potentially in different versions)
as well as a document containing the collected series.
Another application is in developing style files
or other kinds of included material
where compilation of the style file could redirect
to a sample or test file.

%%%%%%%%%%%%%%%%%%%%%%%%%%%%%%%%%%%%%%%%%%%%%%%%%%%%%%%%%%%%%%%%%%%%%%%%%%%%%%%%
%%%%%%%%%%%%%%%%%%%%%%%%%%%%%%%%%%%%%%%%%%%%%%%%%%%%%%%%%%%%%%%%%%%%%%%%%%%%%%%%
\section{Usage}

First of all, the package \textsf{childdoc} is \emph{not} a standard
\LaTeXe{} |.sty| style file! Therefore it needs to be invoked in
a non-standard way.

%%%%%%%%%%%%%%%%%%%%%%%%%%%%%%%%%%%%%%%%%%%%%%%%%%%%%%%%%%%%%%%%%%%%%%%%%%%%%%%%
\subsection{Included Files}
\label{sec:include}

%%%%%%%%%%%%%%%%%%%%%%%%%%%%%%%%%%%%%%%%
\DescribeMacro{\childdocmain}
To use the package, add the commands
\begin{center}
\begin{tabular}{l}
|\input{childdoc.def}|\\
|\childdocmain{}|\\
\end{tabular}
\end{center}
at the very top of the main \LaTeX{} file,
in particular \emph{before} the |\documentclass| statement!
The argument of |\childdocmain| should be left empty
(but it must be present).

%%%%%%%%%%%%%%%%%%%%%%%%%%%%%%%%%%%%%%%%
\DescribeMacro{\childdocof}
Furthermore, add the commands
\begin{center}
\begin{tabular}{l}
|\input{childdoc.def}|\\
|\childdocof{|\textit{main}|}|\\
\end{tabular}
\end{center}
at the top of every child file \textit{child}
which is included by |\include{|\textit{child}|}|
from within the main file
(or at least for those files to be compiled individually).
The argument \textit{main} must be the filename of the main file.

There are a couple of
considerations in setting up the main and child documents:

%%%%%%%%%%%%%%%%%%%%%%%%%%%%%%%%%%%%%%%%
\paragraph{Restrictions.}

Please note the following restrictions:
\begin{itemize}
\item
|\childdocmain| must be called with one argument \textit{main}
to ensure compatibility with earlier version of the package.
It must either be empty (|\childdocmain{}|)
or precisely match the filename of the main file in which it is specified.
See \secref{sec:detection} for further information.
\item
The filename \textit{main} must be specified without the |.tex| extension.
\item
The filename \textit{main} is case sensitive
(even in case-insensitive file systems)
due to internal string comparison.
\item
The argument \textit{main} should be fully expanded, it cannot be a macro.
\item
Subdirectories and special characters should be avoided in filenames.
\item
The command |\childdocmain{|\textit{main}|}| must be followed by a whitespace.
It should not be followed immediately by another command
or by a comment mark `|%|'.
This is because the \TeX{} parser reads the token immediately following
the argument of |\childdocmain| and puts it
at the beginning of every child section;
however, a white\-space is ignored.
\end{itemize}

%%%%%%%%%%%%%%%%%%%%%%%%%%%%%%%%%%%%%%%%
\paragraph{Content of Main File.}

It is advisable to place all content in the child files included by |\include|.
Any output contained in the main file will appear in all child documents
unless suppressed manually;
it cannot be suppressed automatically by the |\includeonly| directive
and thus should normally be avoided.
A method to include some content in the main file
by means of conditional processing is described in \secref{sec:conditional}.

%%%%%%%%%%%%%%%%%%%%%%%%%%%%%%%%%%%%%%%%
\paragraph{Page Numbering.}

When only a part of the document is compiled,
the appropriate numbering of pages
(as well as other status parameters)
is determined from the |.aux| files.
The latter contain information from previous passes.
However this information needs to propagate through
all intermediate child documents.
Therefore the page numbering in child documents may well
be inconsistent until the complete document is compiled at least once.

A useful (if unconventional) way to always ensure a consistent
page numbering is to restart the numbering in each child document
and denote the pages by `\textit{child}|.|\textit{page}'
where \textit{child} represents the chapter/section number of the child file.
This can be achieved by the command
|\numberwithin{page}{|\textit{child}|}|
of the \textsf{amsmath} package
where \textit{child} can be |chapter| or |section|
depending on the chosen structuring.
Alternatively, one can modify the macro |\thepage| appropriately
and reset the counter |page| at the start of each child file.

%%%%%%%%%%%%%%%%%%%%%%%%%%%%%%%%%%%%%%%%%%%%%%%%%%%%%%%%%%%%%%%%%%%%%%%%%%%%%%%%
\subsection{Conditional Processing}
\label{sec:conditional}

The package provides a mechanism to compile different versions
of a document. To customise the versions further some conditional processing
can come in handy to distinguish which version is being compiled.
The package provides two macros to describe the compilation context:

%%%%%%%%%%%%%%%%%%%%%%%%%%%%%%%%%%%%%%%%
\DescribeMacro{\ifchilddoc}
The conditional |\ifchilddoc| distinguishes between the compilation of
child documents and the main document:
%
\begin{center}
|\ifchilddoc |\textit{child-code}| |[|\||else |\textit{main-code}]| \||fi|
\end{center}

%%%%%%%%%%%%%%%%%%%%%%%%%%%%%%%%%%%%%%%%
\DescribeMacro{\childdocname}
\DescribeMacro{\childdocjob}
The macro |\childdocname| contains the filename (without extension)
of the main or child file being processed.
Note that |\childdocjob| will always contain the name of the main file.

%%%%%%%%%%%%%%%%%%%%%%%%%%%%%%%%%%%%%%%%
\paragraph{Title Page.}

Conditional processing can be used to include a title or banner page
in the main document when proper precautions are taken.
Importantly, the code in the main file should ensure that the page counter
(as well as other status parameters which are stored in the |.aux| files)
takes the same value after the conditional processing.
Otherwise the page numbers may take divergent values
depending on which part is compiled.

For example, a title page could be declared by:
%
\begin{center}
\begin{tabular}{l}
|\ifchilddoc\||else|\\
|\addtocounter{page}{-1}|\\
\textit{code for title page}\\
|\newpage|\\
|\||fi|
\end{tabular}
\end{center}
%
A banner page for the child documents can be generated by:
%
\begin{center}
\begin{tabular}{l}
|\ifchilddoc|\\
|\addtocounter{page}{-1}|\\
\textit{code for banner page}\\
|\newpage|\\
|\||fi|
\end{tabular}
\end{center}
%
Here one could write a message such as:
\begin{center}
|This is the part \childdocname{} of \childdocjob{}.|
\end{center}

%%%%%%%%%%%%%%%%%%%%%%%%%%%%%%%%%%%%%%%%%%%%%%%%%%%%%%%%%%%%%%%%%%%%%%%%%%%%%%%%
\subsection{Flags}
\label{sec:flags}

The package makes it easy to generate different versions
of the main or child documents.
To this end compilation flags can be defined
and assigned different default values.
They will be particularly useful in conjunction
with the forwarding mechanism described in \secref{sec:forward}.

For example, it may be useful to have a flag |\version|
which can be set to |draft| or |final|.
The document source will contain some conditional code
depending on the value of |\version|.
Suppose further, the flag should default to |final| for the main file
and to |draft| for child files
which is a natural assignment for editing the document.
This is achieved by placing the following code
in the preamble of the main document
(below the |\childdocmain| directive):
%
\begin{center}
\begin{tabular}{l}
|\ifchilddoc|\\
|\providecommand{\version}{draft}|\\
|\||else|\\
|\providecommand{\version}{final}|\\
|\||fi|
\end{tabular}
\end{center}
%
The definition by |\providecommand| makes sure
that previous definitions are not overwritten.
Further statements |\providecommand{\version}{...}|
can thus be added before the above code to override it.

For the main file, one might add a line
(between |\childdocmain| and the above block)
%
\begin{center}
|%\ifchilddoc\||else\providecommand{\version}{draft}\||fi|
\end{center}
%
which can be uncommented to produce a draft version.
Likewise one can add a line to the very top of a child file
(above the |\childdocof{|\textit{main}|}| directive)
%
\begin{center}
|%\providecommand{\version}{final}|
\end{center}
%
which can be uncommented to produce the final version of this child document.

%%%%%%%%%%%%%%%%%%%%%%%%%%%%%%%%%%%%%%%%%%%%%%%%%%%%%%%%%%%%%%%%%%%%%%%%%%%%%%%%
\subsection{Forwarding}
\label{sec:forward}

Different versions of the main or child documents
using compilation flags as described in \secref{sec:flags}
can be (permanently) stored in different files
for convenient compilation, viewing and distribution.
To this end, the package defines a command
to pass on compilation to a different file:

%%%%%%%%%%%%%%%%%%%%%%%%%%%%%%%%%%%%%%%%
\DescribeMacro{\childdocforward}
The command |\childdocforward| redirects processing to
another source file:
%
\begin{center}
\begin{tabular}{l}
|\input{childdoc.def}|\\
|\childdocforward[|\textit{main}|]{|\textit{dest}|}|\\
\end{tabular}
\end{center}
%
The argument \textit{dest} is the destination file
(without extension).
It should be the main file or one of the child files.
Note that further \textsf{childdoc} directives
such as |\childdocof| and |\childdocforward|
in the indicated file will be processed in this form.
The optional argument \textit{main}
passes on directly to the main file \textit{main}
while pretending to compile the child \textit{dest}.
This form behaves as if \textit{dest}
issues |\childdocof{|\textit{main}|}| right away,
and no further \textsf{childdoc} directives will be processed.

%%%%%%%%%%%%%%%%%%%%%%%%%%%%%%%%%%%%%%%%
\DescribeMacro{\...prefix}
In the alternative form |\childdocforwardprefix|,
%
\begin{center}
\begin{tabular}{l}
|\input{childdoc.def}|\\
|\childdocforwardprefix[|\textit{main}|]{|\textit{prefix}|}{|\textit{dest}|}|
\end{tabular}
\end{center}
%
the destination file is determined by a pattern
depending on the current file:
To make this work, the current file must be called
`{\textit{prefix}\hspace{0.2em}\textit{suffix}}'
with \textit{prefix} matching precisely the argument.
Processing is then passed on to the file
`{\textit{dest}\hspace{0.2em}\textit{suffix}}'.
Surely, the same effect is achieved by
directly specifying the
argument `{\textit{dest}\hspace{0.2em}\textit{suffix}}'
in the first form.
However, that requires to set up a different file
for each child. With the alternative form of the command
all these files can have exactly the same content
which simplifies setting them up and maintaining them.

For example, the following file |draft.tex|
with a compilation flag |\version| as described in \secref{sec:flags}
compiles the main document as a draft:
%
\begin{center}
\begin{tabular}{l}
|\def\version{draft}|\\
|\input{childdoc.def}|\\
|\childdocforward{|\textit{main}|}|
\end{tabular}
\end{center}
%
Likewise, the following files |final|\textit{nn}|.tex|
compile the final version of the child document
|child|\textit{nn}|.tex|:
%
\begin{center}
\begin{tabular}{l}
|\def\version{final}|\\
|\input{childdoc.def}|\\
|\childdocforwardprefix{final}{child}|
\end{tabular}
\end{center}
%

Note that when several versions of a main file and/or of each child file
are to be generated, it may be convenient to set up a |Makefile| or
shell script to automatise the process.

%%%%%%%%%%%%%%%%%%%%%%%%%%%%%%%%%%%%%%%%%%%%%%%%%%%%%%%%%%%%%%%%%%%%%%%%%%%%%%%%
\subsection{Command Line Processing}
\label{sec:commandline}

The effect of redirection files can also be achieved by invoking
the \LaTeX{} compiler with a more elaborate command line.
Most conveniently this should be done as part
of a shell script or a |Makefile|.

When using \textsf{childdoc} in the main file, the following
command lines effectively perform a redirection
(note that depending on the shell being used,
backslashes may have to be doubled: `|\|' $\to$ `|\\|'):
%
\begin{center}
|... -jobname "|\textit{target}|" |\\|"|[\textit{flags}]%
|\input{childdoc.def}\childdocforward[|\textit{main}|]{|\textit{dest}|}"|
\end{center}
%
Here \textit{target} is the name of the output file,
\textit{main} is the name of the main file
and \textit{dest} is the name of the main or child file to be processed
(all filenames without extensions).
The optional argument \textit{main} can be omitted
if \textit{main} matches \textit{dest}.
Optionally, compilation \textit{flags} can be defined via |\def| commands.
This command line makes the \TeX{} engine believe
it is compiling the file \textit{target}
whose content is specified as the latter parameter.
The provided code then forwards the processing to
\textit{main} or \textit{dest} as described in \secref{sec:forward}.

%%%%%%%%%%%%%%%%%%%%%%%%%%%%%%%%%%%%%%%%%%%%%%%%%%%%%%%%%%%%%%%%%%%%%%%%%%%%%%%%
\subsection{Include by Input}
\label{sec:input}

Including child documents by |\include| has some restrictions by design.
Most notably, the content of a child document always occupies
its own set of pages; pages cannot be shared between child documents.
Usually, this behaviour makes perfect sense
because each child document contain an essential part of the document.
However, in some situations it may be desirable to compose
a document from a collection of parts
without having mandatory page breaks between then.
For this case, the package
provides a mechanism to include parts
by |\input| which can also be processed individually.
However, by construction this mechanism
requires manual handling of the content to be output.

%%%%%%%%%%%%%%%%%%%%%%%%%%%%%%%%%%%%%%%%
\DescribeMacro{\ifchilddocmanual}
The main file should be prepared as usual, see \secref{sec:include}.
However, the document body must make a distinction
between processing of an individual part and of the main document, e.g.:
%
\begin{center}
\begin{tabular}{l}
|\ifchilddocmanual|\\
|\input{\childdocname}|\\
|\||else|\\
\textit{document body with }|\input{|\textit{part}|}|\\
|\||fi|
\end{tabular}
\end{center}
%
The conditional |\ifchilddocmanual| is true whenever
a part to be included by |\input| is being compiled,
and the name of the part is stored in |\childdocname|.

%%%%%%%%%%%%%%%%%%%%%%%%%%%%%%%%%%%%%%%%
\DescribeMacro{\childdocby}
Each part to be included by |\input| should start with:
%
\begin{center}
\begin{tabular}{l}
|\input{childdoc.def}|\\
|\childdocby{|\textit{main}|}|\\
\end{tabular}
\end{center}
%
The directive |\childdocby| is similar to |\childdocof|
described in \secref{sec:include},
but the subsequent selection of content must be done manually.
To that end, both |\ifchilddoc| and |\ifchilddocmanual|
will be true upon processing of a part,
and the name of the part is stored in |\childdocname|.
Note that |\jobname| will be set to the filename of the current part
so that each part receives an individual |.aux| file
that does not interfere with the |.aux| file(s) of the main document.
This behaviour can be altered by the alternative form
|\childdocby[*]{|\textit{main}|}| (with a non-empty optional argument)
which uses the |.aux| file of the main document
by setting |\jobname| to \textit{main}.

%%%%%%%%%%%%%%%%%%%%%%%%%%%%%%%%%%%%%%%%%%%%%%%%%%%%%%%%%%%%%%%%%%%%%%%%%%%%%%%%
\subsection{Driver Development}
\label{sec:driver}

The \textsf{childdoc} mechanism can also be use for the development
of definition files such as \LaTeX{} styles or classes.
This case differs from the above setup with multiple parts
included by |\include| in that no |\includeonly| should be invoked.
This can be achieved by starting the include file
(before |\ProvidesPackage|) with:
%
\begin{center}
\begin{tabular}{l}
|\input{childdoc.def}|\\
|\childdocforward{|\textit{main}|}|\\
\end{tabular}
\end{center}
%
or alternatively with:
%
\begin{center}
\begin{tabular}{l}
|\input{childdoc.def}|\\
|\childdocby{|\textit{main}|}|\\
\end{tabular}
\end{center}
%
Both forms have slightly different effects as described above.
The main file is prepared as usual, see \secref{sec:include}.

%%%%%%%%%%%%%%%%%%%%%%%%%%%%%%%%%%%%%%%%%%%%%%%%%%%%%%%%%%%%%%%%%%%%%%%%%%%%%%%%
\subsection{Legacy Detection}
\label{sec:detection}

The directive |\childdocmain| in the main file can detect
whether the complete document or merely a child is to be compiled
even without using the directive |\childdocof|.
This method is deprecated because it is less robust
and there is no compelling reason to use it;
it is merely provided for backward compatibility
and it may be removed in future versions.

If the detection mechanism is to be used,
it is mandatory to correctly specify
the filename of the main file as the argument of |\childdocmain|:
%
\begin{center}
\begin{tabular}{l}
|\input{childdoc.def}|\\
|\childdocmain{|\textit{main}|}|\\
\end{tabular}
\end{center}
%
If |\jobname| does not match the argument \textit{main} of |\childdocmain|,
it is assumed that |\jobname| points to the child file to be compiled.
When using |\childdocmain| with the main file specified as argument,
it suffices to start a child file
with just |\input{|\textit{main}|}|
without loading of the package and using |\childdocof|.
If instead all processing is done
with the appropriate \textsf{childdoc} directives,
the argument of \textit{main} of |\childdocmain| can be empty.

An alternative version of the command line processing described
in \secref{sec:commandline} using the detection mechanism reads:
%
\begin{center}
|... -jobname "|\textit{target}|" "|[\textit{flags}]%
[|\def\jobname{|\textit{dest}|}|]|\input{|\textit{main}|}"|
\end{center}

%%%%%%%%%%%%%%%%%%%%%%%%%%%%%%%%%%%%%%%%%%%%%%%%%%%%%%%%%%%%%%%%%%%%%%%%%%%%%%%%
\subsection{Manual Code}
\label{sec:manual}

In case one cannot be certain whether the definitions file |childdoc.def|
is installed on the target \TeX{} distribution
and one prefers not to ship it,
it is conceivable to paste a few relevant commands into the sources.

To that end, drop all statements |\input{childdoc.def}|
and perform the replacements as outlined below.
Instead of |\childdocmain{|\textit{main}|}| add the following code
to the top of the main file:
%
\begin{center}
\begin{tabular}{l}
|\||ifdefined\childdocname\endinput\||fi\newif\ifchilddoc|\\
|\edef\childdocname{\scantokens\expandafter{\jobname\noexpand}}|\\
|\def\childdocmain{|\textit{main}|}\||ifx\childdocmain\childdocname\||else|\\
|\childdoctrue\includeonly{\childdocname}\let\jobname\childdocmain\||fi|\\
\end{tabular}
\end{center}
%
Instead of |\childdocof{|\textit{main}|}| just include the main file
at the top of each child file:
%
\begin{center}
|\input{|\textit{main}|}|
\end{center}
%
A simple redirection |\childdocforward{|\textit{dest}|}| is achieved by:
%
\begin{center}
|\def\jobname{|\textit{dest}|}\input{\jobname}|
\end{center}
%
The redirection with prefix
|\childdocforwardprefix[|\textit{prefix}|]{|\textit{dest}|}|
is accomplished by:
%
\begin{center}
\begin{tabular}{l}
|{\edef\jobname{\scantokens\expandafter{\jobname\noexpand}}|\\
|\def\redirectjob |\textit{prefix}|#1~~~{\gdef\jobname{|\textit{dest}|#1}}|\\
|\expandafter\redirectjob\jobname~~~}\input{\jobname}|
\end{tabular}
\end{center}

In an alternative approach,
child documents can be compiled by a specific command line
without additional code or specific definitions:
%
\begin{center}
|... -jobname "|\textit{target}|" "|[\textit{flags}]%
|\includeonly{|\textit{dest}|}\input{|\textit{main}|}"|
\end{center}
%

%%%%%%%%%%%%%%%%%%%%%%%%%%%%%%%%%%%%%%%%%%%%%%%%%%%%%%%%%%%%%%%%%%%%%%%%%%%%%%%%
%%%%%%%%%%%%%%%%%%%%%%%%%%%%%%%%%%%%%%%%%%%%%%%%%%%%%%%%%%%%%%%%%%%%%%%%%%%%%%%%
\section{Information}

%%%%%%%%%%%%%%%%%%%%%%%%%%%%%%%%%%%%%%%%%%%%%%%%%%%%%%%%%%%%%%%%%%%%%%%%%%%%%%%%
\subsection{Copyright}

Copyright \copyright{} 2017--2018 Niklas Beisert

This work may be distributed and/or modified under the
conditions of the \LaTeX{} Project Public License, either version 1.3
of this license or (at your option) any later version.
The latest version of this license is in
  \url{http://www.latex-project.org/lppl.txt}
and version 1.3 or later is part of all distributions of \LaTeX{}
version 2005/12/01 or later.

This work has the LPPL maintenance status `maintained'.

The Current Maintainer of this work is Niklas Beisert.

This work consists of the files |README.txt|, |childdoc.ins| and |childdoc.dtx|
as well as the derived files |childdoc.def|, |cdocsamp.tex|
with |cdocsch1.tex|, |cdocsch2.tex|, |cdocspt3.tex|, |cdocspt4.tex|,
|cdocsdrf.tex|, |cdocsfn1.tex|, |cdocsfn2.tex|
as well as |childdoc.pdf|.

%%%%%%%%%%%%%%%%%%%%%%%%%%%%%%%%%%%%%%%%%%%%%%%%%%%%%%%%%%%%%%%%%%%%%%%%%%%%%%%%
\subsection{Files and Installation}

The package consists of the files:
%
\begin{center}
\begin{tabular}{ll}
    |README.txt|   & readme file \\
    |childdoc.ins| & installation file \\
    |childdoc.dtx| & source file \\
    |childdoc.def| & definition file \\
    |cdocsamp.tex| & sample main file \\
    |cdocsch1.tex| & sample include file \\
    |cdocsch2.tex| & sample include file \\
    |cdocspt3.tex| & sample part file \\
    |cdocspt4.tex| & sample part file \\
    |cdocsdrf.tex| & sample redirection file \\
    |cdocsfn1.tex| & sample redirection file \\
    |cdocsfn2.tex| & sample redirection file \\
    |childdoc.pdf| & manual
\end{tabular}
\end{center}
%
The distribution consists of the files
|README.txt|, |childdoc.ins| and |childdoc.dtx|.
%
\begin{itemize}
\item
Run (pdf)\LaTeX{} on |childdoc.dtx|
to compile the manual |childdoc.pdf| (this file).
\item
Run \LaTeX{} on |childdoc.ins| to create the definitions file |childdoc.def|
and the sample |cdocsamp.tex| with include files
|cdocsch1.tex|, |cdocsch2.tex|, |cdocspt3.tex|, |cdocspt4.tex|,
|cdocsdrf.tex|, |cdocsfn1.tex|, |cdocsfn2.tex|.
Then copy the file |childdoc.def| to an appropriate directory of your \LaTeX{}
distribution, e.g.\ \textit{texmf-root}|/tex/latex/childdoc|.
\end{itemize}

%%%%%%%%%%%%%%%%%%%%%%%%%%%%%%%%%%%%%%%%%%%%%%%%%%%%%%%%%%%%%%%%%%%%%%%%%%%%%%%%
\subsection{Related CTAN Packages}

There are several other packages which offer a similar functionality:
%
\begin{itemize}
\item
The packages
\href{http://ctan.org/pkg/docmute}{\textsf{docmute}},
\href{http://ctan.org/pkg/includex}{\textsf{includex}} and
\href{http://ctan.org/pkg/standalone}{\textsf{standalone}}
provide commands to include only the document body of
a child file thus allowing both files to be compiled individually.
\item
The packages \href{http://ctan.org/pkg/subdocs}{\textsf{subdocs}}
and \href{http://ctan.org/pkg/subfiles}{\textsf{subfiles}}
provide structures in which the main and child documents can be
encapsulated and allowing them to be compiled individually.
The inclusion mechanism is different from the conventional |\include|.
\item
The package \href{http://ctan.org/pkg/combine}{\textsf{combine}}
is an elaborate solution to combine several documents into one.
\end{itemize}
%
See also the CTAN topic \href{http://ctan.org/topic/subdocs}{\textsf{subdocs}}
for further related packages.
The present package differs from the above solutions in that
a document structure constructed with the conventional |\include| mechanism
just needs two extra commands at the top of every file
such that all constituent files can be compiled individually.

%%%%%%%%%%%%%%%%%%%%%%%%%%%%%%%%%%%%%%%%%%%%%%%%%%%%%%%%%%%%%%%%%%%%%%%%%%%%%%%%
%\subsection{Feature Suggestions}
%
%The following is a list of features which may be useful for future
%versions of this package:
%%
%\begin{itemize}
%\item
%\ldots
%\end{itemize}

%%%%%%%%%%%%%%%%%%%%%%%%%%%%%%%%%%%%%%%%%%%%%%%%%%%%%%%%%%%%%%%%%%%%%%%%%%%%%%%%
\subsection{Revision History}

%%%%%%%%%%%%%%%%%%%%%%%%%%%%%%%%%%%%%%%%
\paragraph{v2.0:} 2018/12/30

\begin{itemize}
\item
immediate forward processing
\item
added |\childdocby| mechanism
\item
manual restructured
\end{itemize}

%%%%%%%%%%%%%%%%%%%%%%%%%%%%%%%%%%%%%%%%
\paragraph{v1.6:} 2018/01/17

\begin{itemize}
\item
application for development of include files
\item
corrections to manual
\end{itemize}

%%%%%%%%%%%%%%%%%%%%%%%%%%%%%%%%%%%%%%%%
\paragraph{v1.5:} 2017/05/21

\begin{itemize}
\item
more complete structuring introduced
\item
|\childdocof| introduced
\item
|\childdoc| renamed to |\childdocmain|
\item
|\childredirect| renamed to |\childdocforward| and |\childdocforwardprefix|
and functionality expanded
\end{itemize}

%%%%%%%%%%%%%%%%%%%%%%%%%%%%%%%%%%%%%%%%
\paragraph{v1.0:} 2017/04/27

\begin{itemize}
\item
manual and install package
\item
first version published on CTAN
\end{itemize}

%%%%%%%%%%%%%%%%%%%%%%%%%%%%%%%%%%%%%%%%
\paragraph{v0.6:} 2017/04/26

\begin{itemize}
\item
redirection mechanism added
\end{itemize}

%%%%%%%%%%%%%%%%%%%%%%%%%%%%%%%%%%%%%%%%
\paragraph{v0.5:} 2017/04/26

\begin{itemize}
\item
functionality in definition file
\end{itemize}


%%%%%%%%%%%%%%%%%%%%%%%%%%%%%%%%%%%%%%%%%%%%%%%%%%%%%%%%%%%%%%%%%%%%%%%%%%%%%%%%
%%%%%%%%%%%%%%%%%%%%%%%%%%%%%%%%%%%%%%%%%%%%%%%%%%%%%%%%%%%%%%%%%%%%%%%%%%%%%%%%
%%%%%%%%%%%%%%%%%%%%%%%%%%%%%%%%%%%%%%%%%%%%%%%%%%%%%%%%%%%%%%%%%%%%%%%%%%%%%%%%
\appendix

\settowidth\MacroIndent{\rmfamily\scriptsize 000\ }

 \DocInput{childdoc.dtx}

\end{document}
%</driver>
% \fi
%
% %%%%%%%%%%%%%%%%%%%%%%%%%%%%%%%%%%%%%%%%%%%%%%%%%%%%%%%%%%%%%%%%%%%%%%%%%%%%%%
% %%%%%%%%%%%%%%%%%%%%%%%%%%%%%%%%%%%%%%%%%%%%%%%%%%%%%%%%%%%%%%%%%%%%%%%%%%%%%%
% \section{Sample}
%\iffalse
%<*samplemain>
%\fi
%
% The following presents a sample document
% with two chapters, two parts, a title page,
% a compile flag as well as three forwarding files to set the flag.
% It consists of eight |.tex| files:
% \begin{center}
% \begin{tabular}{ll}
% |cdocsamp.tex|&main file\\
% |cdocsch1.tex|&include file for chapter 1\\
% |cdocsch2.tex|&include file for chapter 2\\
% |cdocspt3.tex|&include file for part 3\\
% |cdocspt4.tex|&include file for part 4\\
% |cdocsdrf.tex|&forwarding file for main file in draft mode\\
% |cdocsfi1.tex|&forwarding file for final version of chapter 1\\
% |cdocsfi2.tex|&forwarding file for final version of chapter 2\\
% \end{tabular}
% \end{center}
% Each of the eight files can be compiled directly by the \LaTeX{} compiler.
%
% %%%%%%%%%%%%%%%%%%%%%%%%%%%%%%%%%%%%%%
% \paragraph{Main File.}
%
% The main file is called |cdocsamp.tex|.
%
% Load the \textsf{childdoc} definitions and
% declare the filename for the main document:
%    \begin{macrocode}
\input{childdoc.def}
\childdocmain{}
%    \end{macrocode}

% Optional override for |\version| flag:
%    \begin{macrocode}
%%\ifchilddoc\else\providecommand{\version}{draft}\fi
%    \end{macrocode}

% Define the default values for the |\version| flag
% (|final| for the main file and |draft| for childs):
%    \begin{macrocode}
\ifchilddoc
\providecommand{\version}{draft}
\else
\providecommand{\version}{final}
\fi
%    \end{macrocode}

% Load the standard document class:
%    \begin{macrocode}
\documentclass[12pt]{article}
%    \end{macrocode}

% Start the document body:
%    \begin{macrocode}
\begin{document}
%    \end{macrocode}

% Declare a title page.
% Print title, part of document being processed and version flag:
%    \begin{macrocode}
\addtocounter{page}{-1}
\begin{center}
{\LARGE\bfseries{}childdoc example\par}
\vspace{1cm}
\ifchilddoc
\ifchilddocmanual part\else chapter\fi:
`\childdocname' of `\childdocjob'\par
\else
main document: `\childdocjob'\par
\fi
version: \version\par
\end{center}
\newpage
%    \end{macrocode}

% Manually include selected file,
% otherwise process as usual:
%    \begin{macrocode}
\ifchilddocmanual
\section*{part `\childdocname'}
\input{\childdocname}
\else
%    \end{macrocode}

% Include the two chapters:
%    \begin{macrocode}
\include{cdocsch1}
\include{cdocsch2}
%    \end{macrocode}

% Include the two parts unless only chapters should be displayed:
%    \begin{macrocode}
\ifchilddoc\else
\section{part three}
\input{cdocspt3}
\section{part four}
\input{cdocspt4}
\fi
%    \end{macrocode}

% Process as usual until here:
%    \begin{macrocode}
\fi
%    \end{macrocode}

% End of document body:
%    \begin{macrocode}
\end{document}
%    \end{macrocode}
%\iffalse
%</samplemain>
%\fi
%
% %%%%%%%%%%%%%%%%%%%%%%%%%%%%%%%%%%%%%%
% \paragraph{Chapter Include Files.}
%
% The include files are called |cdocsch1.tex| and |cdocsch2.tex|.
%
%\iffalse
%<*samplechap1|samplechap2>
%\fi

% Optional override for |\version| flag:
%    \begin{macrocode}
%%\providecommand{\version}{final}
%    \end{macrocode}

% Include the main document:
%    \begin{macrocode}
\input{childdoc.def}
\childdocof{cdocsamp}
%    \end{macrocode}

%\iffalse
%</samplechap1|samplechap2>
%\fi
%
%\iffalse
%<*samplechap1>
%\fi
% Some text for chapter 1:
%    \begin{macrocode}
\section{one}
some text in chapter one
%    \end{macrocode}

%\iffalse
%</samplechap1>
%\fi
% Some text for chapter 2:
%\iffalse
%<*samplechap2>
%\fi
%    \begin{macrocode}
\section{two}
more text in chapter two
%    \end{macrocode}

%\iffalse
%</samplechap2>
%\fi
%
% %%%%%%%%%%%%%%%%%%%%%%%%%%%%%%%%%%%%%%
% \paragraph{Part Include Files.}
%
% The include files are called |cdocspt3.tex| and |cdocspt4.tex|.
%
%\iffalse
%<*samplepart3|samplepart4>
%\fi

% Optional override for |\version| flag:
%    \begin{macrocode}
%%\providecommand{\version}{final}
%    \end{macrocode}

% Include the main document:
%    \begin{macrocode}
\input{childdoc.def}
\childdocby{cdocsamp}
%    \end{macrocode}

%\iffalse
%</samplepart3|samplepart4>
%\fi
%
%\iffalse
%<*samplepart3>
%\fi
% Some text for part 3:
%    \begin{macrocode}
some text in part three
%    \end{macrocode}

%\iffalse
%</samplepart3>
%\fi
% Some text for part 4:
%\iffalse
%<*samplepart4>
%\fi
%    \begin{macrocode}
more text in part four
%    \end{macrocode}

%\iffalse
%</samplepart4>
%\fi
%
% %%%%%%%%%%%%%%%%%%%%%%%%%%%%%%%%%%%%%%
% \paragraph{Forwarding for a Complete Draft.}
%
% The following forwarding file |cdocsdrf.tex|
% compiles the main document in draft mode:
%\iffalse
%<*sampledraft>
%\fi
%    \begin{macrocode}
\def\version{draft}
\input{childdoc.def}
\childdocforward{cdocsamp}
%    \end{macrocode}

%\iffalse
%</sampledraft>
%\fi
%
% %%%%%%%%%%%%%%%%%%%%%%%%%%%%%%%%%%%%%%
% \paragraph{Forwarding for Final Version of the Chapters.}
%
% The following forwarding files |cdocsfn1.tex| and |cdocsfn2.tex|
% (with identical content)
% compile the final versions of the child documents
% |cdocsch1.tex| and |cdocsch2.tex|, respectively:
%\iffalse
%<*samplefinal>
%\fi
%    \begin{macrocode}
\def\version{final}
\input{childdoc.def}
\childdocforwardprefix[cdocsamp]{cdocsfn}{cdocsch}
%    \end{macrocode}

%\iffalse
%</samplefinal>
%\fi
%
% %%%%%%%%%%%%%%%%%%%%%%%%%%%%%%%%%%%%%%
% \paragraph{Command Line Processing.}
%
% The following three command lines generate the output files
% |cdocscld|, |cdocscl1| and |cdocscl2|
% which should be identical to
% |cdocsdrf|, |cdocsch1| and |cdocsfn2|, respectively:
% \begin{center}
% \begin{tabular}{l}
% |latex -jobname cdocscld \|\\
% |  "\def\version{draft}\input{childdoc.def}\childdocforward{cdocsamp}"|\\
% |latex -jobname cdocscl1 \|\\
% |  "\input{childdoc.def}\childdocforward[cdocsamp]{cdocsch1}"|\\
% |latex -jobname cdocscl2 \|\\
% |  "\def\version{final}\input{childdoc.def}\childdocforward{cdocsch2}"|
% \end{tabular}
% \end{center}
% Note that the trailing backslash on each first line
% merely continues the input to the second line
% (for convenient cut ant paste).
% Furthermore, the command |latex| can be replaced by any
% of its alternative versions such as |pdflatex|.
%
% %%%%%%%%%%%%%%%%%%%%%%%%%%%%%%%%%%%%%%%%%%%%%%%%%%%%%%%%%%%%%%%%%%%%%%%%%%%%%%
% %%%%%%%%%%%%%%%%%%%%%%%%%%%%%%%%%%%%%%%%%%%%%%%%%%%%%%%%%%%%%%%%%%%%%%%%%%%%%%
% \section{Implementation}
%\iffalse
%<*package>
%\fi
%
% This section describes the definitions file |childdoc.def|.

% The definitions cannot be loaded using |\usepackage| or |\RequirePackage|
% which has a mechanism to prevent loading a style file more than once.
% When loading the definitions by means of |\input|
% multiple instances have to be prevented manually:
%\iffalse
%This code needs to be before the `\ProvidesFile' directive
%which is defined at the beginning of this file.
%Therefore it is also placed there and commented out here.
%</package>
%<*discard>
%\fi
%    \begin{macrocode}
\ifdefined\childdocmain\endinput\fi
%    \end{macrocode}
%\iffalse
%</discard>
%<*package>
%\fi
%
% \macro{\ifchilddoc}
% \macro{\ifchilddocmanual}
% The conditional |\ifchilddoc| tells whether a
% child (true) or main (false) document is being compiled.
% The conditional |\ifchilddocmanual| tells whether
% the |\includeonly| mechanism is used (false) or
% the selection of child files must be performed manually (true).
% The definitions initialise to false:
%    \begin{macrocode}
\newif\ifchilddoc
\newif\ifchilddocmanual
%    \end{macrocode}

% \macro{\childdocname}
% \macro{\childdocjob}
% The macro |\childdocname| stores the name of the main document
% to be compiled. The macro |\childdocjob| stores the name of
% the document on which the \LaTeX{} compiler was originally invoked.
% The content of |\jobname| cannot be compared
% to filenames specified in the source due to different catcodes.
% The following code rescans |\jobname|, stores the result
% in |\childdocname| and saves a copy in |\childdocjob|:
%    \begin{macrocode}
\edef\childdocname{\scantokens\expandafter{\jobname\noexpand}}
\let\childdocjob\childdocname
%    \end{macrocode}

% \macro{\childdocdisable}
% The macro |\childdocdisable| prevents the main file
% from being processed more than once.
% At this stage, the main document command |\childdocmain|
% is assumed to be called once again where it should do nothing.
% Any subsequent call to it should prevent
% a secondary processing of the main document
% It overwrites the forwarding commands
% |\childdocof| and |\childdocforward|
% with empty macros to prevent further inclusions of the main document:
%    \begin{macrocode}
\newcommand{\childdocdisable}
{
  \renewcommand{\childdocmain}[1]{\renewcommand{\childdocmain}[1]{\endinput}}
  \renewcommand{\childdocof}[1]{}
  \renewcommand{\childdocby}[2][]{}
  \renewcommand{\childdocforward}[2][]{}
  \renewcommand{\childdocdisable}{}
}
%    \end{macrocode}

% \macro{\childdocmain}
% The macro |\childdocmain| is to be called at the top of the main file
% with nothing or the main filename (without extension) as argument.
% First, it breaks loops.
% If the argument is not empty and does not match |\childdocname|
% (which is set by the first inclusion of |childdoc.def|),
% |\ifchilddoc| is set to true, |\includeonly| is applied to the child file
% and |\jobname| is set to the main file
% (for proper handling of |.aux| files):
%    \begin{macrocode}
\newcommand{\childdocmain}[1]
{
  \childdocdisable\childdocmain{}
  \if?#1?\else
    \begingroup
      \def\childdoctmp{#1}
      \ifx\childdoctmp\childdocname
        \def\childdoctmp{}
      \else
        \def\childdoctmp
        {
          \childdoctrue
          \includeonly{\childdocname}
          \def\childdocjob{#1}
          \def\jobname{#1}
        }
      \fi
      \expandafter
    \endgroup
    \childdoctmp
  \fi
}
%    \end{macrocode}

% \macro{\childdocof}
% The command |\childdocof| redirects
% compilation to the main file |#1|.
%    \begin{macrocode}
\newcommand{\childdocof}[1]
{
  \childdocdisable
  \childdoctrue
  \includeonly{\childdocname}
  \def\jobname{#1}
  \def\childdocjob{#1}
  \input{#1}
}
%    \end{macrocode}

% \macro{\childdocby}
% The command |\childdocby| ....
%    \begin{macrocode}
\newcommand{\childdocby}[2][]
{
  \childdocdisable
  \childdoctrue
  \childdocmanualtrue
  \if?#1?\else
    \def\jobname{#2}
  \fi
  \def\childdocjob{#2}
  \input{#2}
  \endinput
}
%    \end{macrocode}

% \macro{\childdocforward}
% The command |\childdocforward| redirects
% compilation to the main file or
% (if the optional argument is given) a child file.
% Parameters are set as if the main file
% or a child file starting with |\childdocof| was compiled.
% Then compilation is handed over to the main file:
%    \begin{macrocode}
\newcommand{\childdocforward}[2][]
{
  \begingroup
    \if?#1?
      \def\childdoctmp
      {
        \def\childdocname{#2}
        \def\childdocjob{#2}
        \def\jobname{#2}
        \input{#2}
        \endinput
      }
    \else
      \def\childdoctmp
      {
        \childdocdisable
        \def\childdocname{#2}
        \childdoctrue
        \includeonly{#2}
        \def\childdocjob{#1}
        \def\jobname{#1}
        \input{#1}
        \endinput
      }
    \fi
    \expandafter
  \endgroup
  \childdoctmp
}
%    \end{macrocode}

% \macro{\childdocforwardprefix}
% The command |\childdocforwardprefix| redirects
% compilation to the main or a child file by means of a pattern.
% The prefix |#1| in the current filename is replaced by |#2|
% and the suffix of the current filename is kept
% (it is assumed that the filename does not contain the substring `|~~~|'
% which is used as a delimiter).
% Compilation is handed over to the new file by |\childdocforward|:
%    \begin{macrocode}
\newcommand{\childdocforwardprefix}[3][]
{
  \begingroup
    \def\childdocextract #2##1~~~{\def\childdoctmp{\childdocforward[#1]{#3##1}}}
    \expandafter\childdocextract\childdocname~~~
    \expandafter
  \endgroup
  \childdoctmp
}
%    \end{macrocode}

% \macro{\childdoc}
% The deprecated macro |\childdoc| is a legacy version of |\childdocmain|:
%    \begin{macrocode}
\newcommand{\childdoc}{\childdocmain}
%    \end{macrocode}

% \macro{\childdocredirect}
% The deprecated macro |\childdocredirect| is a legacy version
% of |\childdocforward| and |\childdocforwardprefix|:
%    \begin{macrocode}
\newcommand{\childdocredirect}[2][]
{
  \begingroup
    \if?#1?
      \def\childdoctmp{\childdocforward{#2}}
    \else
      \def\childdoctmp{\childdocforwardprefix{#1}{#2}}
    \fi
    \expandafter
  \endgroup
  \childdoctmp
}
%    \end{macrocode}

%\iffalse
%</package>
%\fi
%
\endinput
|\\
|\childdocby{|\textit{main}|}|\\
\end{tabular}
\end{center}
%
Both forms have slightly different effects as described above.
The main file is prepared as usual, see \secref{sec:include}.

%%%%%%%%%%%%%%%%%%%%%%%%%%%%%%%%%%%%%%%%%%%%%%%%%%%%%%%%%%%%%%%%%%%%%%%%%%%%%%%%
\subsection{Legacy Detection}
\label{sec:detection}

The directive |\childdocmain| in the main file can detect
whether the complete document or merely a child is to be compiled
even without using the directive |\childdocof|.
This method is deprecated because it is less robust
and there is no compelling reason to use it;
it is merely provided for backward compatibility
and it may be removed in future versions.

If the detection mechanism is to be used,
it is mandatory to correctly specify
the filename of the main file as the argument of |\childdocmain|:
%
\begin{center}
\begin{tabular}{l}
|% \iffalse
%
% childdoc.dtx Copyright (C) 2017-2018 Niklas Beisert
%
% This work may be distributed and/or modified under the
% conditions of the LaTeX Project Public License, either version 1.3
% of this license or (at your option) any later version.
% The latest version of this license is in
%   http://www.latex-project.org/lppl.txt
% and version 1.3 or later is part of all distributions of LaTeX
% version 2005/12/01 or later.
%
% This work has the LPPL maintenance status `maintained'.
%
% The Current Maintainer of this work is Niklas Beisert.
%
% This work consists of the files childdoc.dtx and childdoc.ins
% and the derived files childdoc.def and cdocsamp.tex with
% cdocsch1.tex, cdocsch2.tex, cdocsdrf.tex, cdocsfn1.tex, cdocsfn2.tex.
%
%<package>\ifdefined\childdocmain\endinput\fi
%<package>\ProvidesFile{childdoc.def}[2018/12/30 v2.0 child document driver]
%<samplemain>\ProvidesFile{cdocsamp.tex}[2018/12/30 v2.0 sample for childdoc]
%<*driver>
%\ProvidesFile{childdoc.drv}[2018/12/30 v2.0 childdoc reference manual file]
\PassOptionsToClass{10pt,a4paper}{article}
\documentclass{ltxdoc}

\usepackage[margin=35mm]{geometry}
\usepackage{hyperref}
\usepackage{hyperxmp}
\usepackage[usenames]{color}

\hypersetup{colorlinks=true}
\hypersetup{pdfstartview=FitH}
\hypersetup{pdfpagemode=UseNone}
\hypersetup{pdfsource={}}
\hypersetup{pdflang={en-UK}}
\hypersetup{pdfcopyright={Copyright 2017-2018 Niklas Beisert.
  This work may be distributed and/or modified under the
  conditions of the LaTeX Project Public License, either version 1.3
  of this license or (at your option) any later version.}}
\hypersetup{pdflicenseurl={http://www.latex-project.org/lppl.txt}}
\hypersetup{pdfcontactaddress={ETH Zurich, ITP, HIT K,
  Wolfgang-Pauli-Strasse 27}}
\hypersetup{pdfcontactpostcode={8093}}
\hypersetup{pdfcontactcity={Zurich}}
\hypersetup{pdfcontactcountry={Switzerland}}
\hypersetup{pdfcontactemail={nbeisert@itp.phys.ethz.ch}}
\hypersetup{pdfcontacturl={http://people.phys.ethz.ch/\xmptilde nbeisert/}}

\newcommand{\secref}[1]{\hyperref[#1]{section \ref*{#1}}}

\parskip1ex
\parindent0pt
\let\olditemize\itemize
\def\itemize{\olditemize\parskip0pt}

\begin{document}

\title{The \textsf{childdoc} Package}
\hypersetup{pdftitle={The childdoc Package}}
\author{Niklas Beisert\\[2ex]
  Institut f\"ur Theoretische Physik\\
  Eidgen\"ossische Technische Hochschule Z\"urich\\
  Wolfgang-Pauli-Strasse 27, 8093 Z\"urich, Switzerland\\[1ex]
  \href{mailto:nbeisert@itp.phys.ethz.ch}
  {\texttt{nbeisert@itp.phys.ethz.ch}}}
\hypersetup{pdfauthor={Niklas Beisert}}
\hypersetup{pdfsubject={Manual for the LaTeX2e Package childdoc}}
\date{30 December 2018, \textsf{v2.0}}
\maketitle

\begin{abstract}\noindent
\textsf{childdoc} is a \LaTeXe{} package
that enables the direct compilation
of document sections included by |\include|
to individual files.
\end{abstract}

\begingroup
\parskip0ex
\tableofcontents
\endgroup

%%%%%%%%%%%%%%%%%%%%%%%%%%%%%%%%%%%%%%%%%%%%%%%%%%%%%%%%%%%%%%%%%%%%%%%%%%%%%%%%
%%%%%%%%%%%%%%%%%%%%%%%%%%%%%%%%%%%%%%%%%%%%%%%%%%%%%%%%%%%%%%%%%%%%%%%%%%%%%%%%
\section{Introduction}

\LaTeX{} provides a mechanism to structure a large document (such as a book)
into a main file and several child files (containing the chapters)
using the |\include| command.
This mechanism is beneficial for documents
which span hundreds of pages in order to
make the source file(s) more manageable.
Moreover, compilation can be restricted to
selected child files by means of the |\includeonly| command.
The latter feature can be used to reduce the compilation time while editing
(this was significantly more useful in the earlier days of \LaTeX{})
or to generate a smaller document which is easier to navigate.
Another application of |\includeonly| is to generate
documents consisting of selected parts of the complete document.

However, there are a few drawbacks of the plain |\include| mechanism:
\begin{itemize}
\item
The child files cannot be compiled on their own,
they can only be compiled via the main file.
A naive editing environment
(such as a text editor with an option
to have the current file processed by \LaTeX)
may require one to switch to the main file before compiling;
attempting to compile the child file produces errors.
\item
The main file must be modified (each time)
to adjust the |\includeonly| command
to the present needs. This easily leaves the main file in a messy state.
\item
The generated document will always carry the filename
of the main document. This is inconvenient if
several child files are to be compiled and
to be kept for distribution.
\end{itemize}

The present package provides a simple interface
to make child files individually compilable by \LaTeX{}.
Compiling a child file then has the same effect as compiling
the main file with an |\includeonly| command
to select the appropriate child.
Moreover the generated document will carry the name of the child
rather than the main file.
This resolves all three above issues.

This feature is meant to make the editing of books,
thesis documents and lecture notes somewhat more convenient.
However, the package can also be used efficiently for
composing a series of documents (such as exercise sheets)
which are typically distributed individually.
It then assists the author in generating the individual documents
(potentially in different versions)
as well as a document containing the collected series.
Another application is in developing style files
or other kinds of included material
where compilation of the style file could redirect
to a sample or test file.

%%%%%%%%%%%%%%%%%%%%%%%%%%%%%%%%%%%%%%%%%%%%%%%%%%%%%%%%%%%%%%%%%%%%%%%%%%%%%%%%
%%%%%%%%%%%%%%%%%%%%%%%%%%%%%%%%%%%%%%%%%%%%%%%%%%%%%%%%%%%%%%%%%%%%%%%%%%%%%%%%
\section{Usage}

First of all, the package \textsf{childdoc} is \emph{not} a standard
\LaTeXe{} |.sty| style file! Therefore it needs to be invoked in
a non-standard way.

%%%%%%%%%%%%%%%%%%%%%%%%%%%%%%%%%%%%%%%%%%%%%%%%%%%%%%%%%%%%%%%%%%%%%%%%%%%%%%%%
\subsection{Included Files}
\label{sec:include}

%%%%%%%%%%%%%%%%%%%%%%%%%%%%%%%%%%%%%%%%
\DescribeMacro{\childdocmain}
To use the package, add the commands
\begin{center}
\begin{tabular}{l}
|\input{childdoc.def}|\\
|\childdocmain{}|\\
\end{tabular}
\end{center}
at the very top of the main \LaTeX{} file,
in particular \emph{before} the |\documentclass| statement!
The argument of |\childdocmain| should be left empty
(but it must be present).

%%%%%%%%%%%%%%%%%%%%%%%%%%%%%%%%%%%%%%%%
\DescribeMacro{\childdocof}
Furthermore, add the commands
\begin{center}
\begin{tabular}{l}
|\input{childdoc.def}|\\
|\childdocof{|\textit{main}|}|\\
\end{tabular}
\end{center}
at the top of every child file \textit{child}
which is included by |\include{|\textit{child}|}|
from within the main file
(or at least for those files to be compiled individually).
The argument \textit{main} must be the filename of the main file.

There are a couple of
considerations in setting up the main and child documents:

%%%%%%%%%%%%%%%%%%%%%%%%%%%%%%%%%%%%%%%%
\paragraph{Restrictions.}

Please note the following restrictions:
\begin{itemize}
\item
|\childdocmain| must be called with one argument \textit{main}
to ensure compatibility with earlier version of the package.
It must either be empty (|\childdocmain{}|)
or precisely match the filename of the main file in which it is specified.
See \secref{sec:detection} for further information.
\item
The filename \textit{main} must be specified without the |.tex| extension.
\item
The filename \textit{main} is case sensitive
(even in case-insensitive file systems)
due to internal string comparison.
\item
The argument \textit{main} should be fully expanded, it cannot be a macro.
\item
Subdirectories and special characters should be avoided in filenames.
\item
The command |\childdocmain{|\textit{main}|}| must be followed by a whitespace.
It should not be followed immediately by another command
or by a comment mark `|%|'.
This is because the \TeX{} parser reads the token immediately following
the argument of |\childdocmain| and puts it
at the beginning of every child section;
however, a white\-space is ignored.
\end{itemize}

%%%%%%%%%%%%%%%%%%%%%%%%%%%%%%%%%%%%%%%%
\paragraph{Content of Main File.}

It is advisable to place all content in the child files included by |\include|.
Any output contained in the main file will appear in all child documents
unless suppressed manually;
it cannot be suppressed automatically by the |\includeonly| directive
and thus should normally be avoided.
A method to include some content in the main file
by means of conditional processing is described in \secref{sec:conditional}.

%%%%%%%%%%%%%%%%%%%%%%%%%%%%%%%%%%%%%%%%
\paragraph{Page Numbering.}

When only a part of the document is compiled,
the appropriate numbering of pages
(as well as other status parameters)
is determined from the |.aux| files.
The latter contain information from previous passes.
However this information needs to propagate through
all intermediate child documents.
Therefore the page numbering in child documents may well
be inconsistent until the complete document is compiled at least once.

A useful (if unconventional) way to always ensure a consistent
page numbering is to restart the numbering in each child document
and denote the pages by `\textit{child}|.|\textit{page}'
where \textit{child} represents the chapter/section number of the child file.
This can be achieved by the command
|\numberwithin{page}{|\textit{child}|}|
of the \textsf{amsmath} package
where \textit{child} can be |chapter| or |section|
depending on the chosen structuring.
Alternatively, one can modify the macro |\thepage| appropriately
and reset the counter |page| at the start of each child file.

%%%%%%%%%%%%%%%%%%%%%%%%%%%%%%%%%%%%%%%%%%%%%%%%%%%%%%%%%%%%%%%%%%%%%%%%%%%%%%%%
\subsection{Conditional Processing}
\label{sec:conditional}

The package provides a mechanism to compile different versions
of a document. To customise the versions further some conditional processing
can come in handy to distinguish which version is being compiled.
The package provides two macros to describe the compilation context:

%%%%%%%%%%%%%%%%%%%%%%%%%%%%%%%%%%%%%%%%
\DescribeMacro{\ifchilddoc}
The conditional |\ifchilddoc| distinguishes between the compilation of
child documents and the main document:
%
\begin{center}
|\ifchilddoc |\textit{child-code}| |[|\||else |\textit{main-code}]| \||fi|
\end{center}

%%%%%%%%%%%%%%%%%%%%%%%%%%%%%%%%%%%%%%%%
\DescribeMacro{\childdocname}
\DescribeMacro{\childdocjob}
The macro |\childdocname| contains the filename (without extension)
of the main or child file being processed.
Note that |\childdocjob| will always contain the name of the main file.

%%%%%%%%%%%%%%%%%%%%%%%%%%%%%%%%%%%%%%%%
\paragraph{Title Page.}

Conditional processing can be used to include a title or banner page
in the main document when proper precautions are taken.
Importantly, the code in the main file should ensure that the page counter
(as well as other status parameters which are stored in the |.aux| files)
takes the same value after the conditional processing.
Otherwise the page numbers may take divergent values
depending on which part is compiled.

For example, a title page could be declared by:
%
\begin{center}
\begin{tabular}{l}
|\ifchilddoc\||else|\\
|\addtocounter{page}{-1}|\\
\textit{code for title page}\\
|\newpage|\\
|\||fi|
\end{tabular}
\end{center}
%
A banner page for the child documents can be generated by:
%
\begin{center}
\begin{tabular}{l}
|\ifchilddoc|\\
|\addtocounter{page}{-1}|\\
\textit{code for banner page}\\
|\newpage|\\
|\||fi|
\end{tabular}
\end{center}
%
Here one could write a message such as:
\begin{center}
|This is the part \childdocname{} of \childdocjob{}.|
\end{center}

%%%%%%%%%%%%%%%%%%%%%%%%%%%%%%%%%%%%%%%%%%%%%%%%%%%%%%%%%%%%%%%%%%%%%%%%%%%%%%%%
\subsection{Flags}
\label{sec:flags}

The package makes it easy to generate different versions
of the main or child documents.
To this end compilation flags can be defined
and assigned different default values.
They will be particularly useful in conjunction
with the forwarding mechanism described in \secref{sec:forward}.

For example, it may be useful to have a flag |\version|
which can be set to |draft| or |final|.
The document source will contain some conditional code
depending on the value of |\version|.
Suppose further, the flag should default to |final| for the main file
and to |draft| for child files
which is a natural assignment for editing the document.
This is achieved by placing the following code
in the preamble of the main document
(below the |\childdocmain| directive):
%
\begin{center}
\begin{tabular}{l}
|\ifchilddoc|\\
|\providecommand{\version}{draft}|\\
|\||else|\\
|\providecommand{\version}{final}|\\
|\||fi|
\end{tabular}
\end{center}
%
The definition by |\providecommand| makes sure
that previous definitions are not overwritten.
Further statements |\providecommand{\version}{...}|
can thus be added before the above code to override it.

For the main file, one might add a line
(between |\childdocmain| and the above block)
%
\begin{center}
|%\ifchilddoc\||else\providecommand{\version}{draft}\||fi|
\end{center}
%
which can be uncommented to produce a draft version.
Likewise one can add a line to the very top of a child file
(above the |\childdocof{|\textit{main}|}| directive)
%
\begin{center}
|%\providecommand{\version}{final}|
\end{center}
%
which can be uncommented to produce the final version of this child document.

%%%%%%%%%%%%%%%%%%%%%%%%%%%%%%%%%%%%%%%%%%%%%%%%%%%%%%%%%%%%%%%%%%%%%%%%%%%%%%%%
\subsection{Forwarding}
\label{sec:forward}

Different versions of the main or child documents
using compilation flags as described in \secref{sec:flags}
can be (permanently) stored in different files
for convenient compilation, viewing and distribution.
To this end, the package defines a command
to pass on compilation to a different file:

%%%%%%%%%%%%%%%%%%%%%%%%%%%%%%%%%%%%%%%%
\DescribeMacro{\childdocforward}
The command |\childdocforward| redirects processing to
another source file:
%
\begin{center}
\begin{tabular}{l}
|\input{childdoc.def}|\\
|\childdocforward[|\textit{main}|]{|\textit{dest}|}|\\
\end{tabular}
\end{center}
%
The argument \textit{dest} is the destination file
(without extension).
It should be the main file or one of the child files.
Note that further \textsf{childdoc} directives
such as |\childdocof| and |\childdocforward|
in the indicated file will be processed in this form.
The optional argument \textit{main}
passes on directly to the main file \textit{main}
while pretending to compile the child \textit{dest}.
This form behaves as if \textit{dest}
issues |\childdocof{|\textit{main}|}| right away,
and no further \textsf{childdoc} directives will be processed.

%%%%%%%%%%%%%%%%%%%%%%%%%%%%%%%%%%%%%%%%
\DescribeMacro{\...prefix}
In the alternative form |\childdocforwardprefix|,
%
\begin{center}
\begin{tabular}{l}
|\input{childdoc.def}|\\
|\childdocforwardprefix[|\textit{main}|]{|\textit{prefix}|}{|\textit{dest}|}|
\end{tabular}
\end{center}
%
the destination file is determined by a pattern
depending on the current file:
To make this work, the current file must be called
`{\textit{prefix}\hspace{0.2em}\textit{suffix}}'
with \textit{prefix} matching precisely the argument.
Processing is then passed on to the file
`{\textit{dest}\hspace{0.2em}\textit{suffix}}'.
Surely, the same effect is achieved by
directly specifying the
argument `{\textit{dest}\hspace{0.2em}\textit{suffix}}'
in the first form.
However, that requires to set up a different file
for each child. With the alternative form of the command
all these files can have exactly the same content
which simplifies setting them up and maintaining them.

For example, the following file |draft.tex|
with a compilation flag |\version| as described in \secref{sec:flags}
compiles the main document as a draft:
%
\begin{center}
\begin{tabular}{l}
|\def\version{draft}|\\
|\input{childdoc.def}|\\
|\childdocforward{|\textit{main}|}|
\end{tabular}
\end{center}
%
Likewise, the following files |final|\textit{nn}|.tex|
compile the final version of the child document
|child|\textit{nn}|.tex|:
%
\begin{center}
\begin{tabular}{l}
|\def\version{final}|\\
|\input{childdoc.def}|\\
|\childdocforwardprefix{final}{child}|
\end{tabular}
\end{center}
%

Note that when several versions of a main file and/or of each child file
are to be generated, it may be convenient to set up a |Makefile| or
shell script to automatise the process.

%%%%%%%%%%%%%%%%%%%%%%%%%%%%%%%%%%%%%%%%%%%%%%%%%%%%%%%%%%%%%%%%%%%%%%%%%%%%%%%%
\subsection{Command Line Processing}
\label{sec:commandline}

The effect of redirection files can also be achieved by invoking
the \LaTeX{} compiler with a more elaborate command line.
Most conveniently this should be done as part
of a shell script or a |Makefile|.

When using \textsf{childdoc} in the main file, the following
command lines effectively perform a redirection
(note that depending on the shell being used,
backslashes may have to be doubled: `|\|' $\to$ `|\\|'):
%
\begin{center}
|... -jobname "|\textit{target}|" |\\|"|[\textit{flags}]%
|\input{childdoc.def}\childdocforward[|\textit{main}|]{|\textit{dest}|}"|
\end{center}
%
Here \textit{target} is the name of the output file,
\textit{main} is the name of the main file
and \textit{dest} is the name of the main or child file to be processed
(all filenames without extensions).
The optional argument \textit{main} can be omitted
if \textit{main} matches \textit{dest}.
Optionally, compilation \textit{flags} can be defined via |\def| commands.
This command line makes the \TeX{} engine believe
it is compiling the file \textit{target}
whose content is specified as the latter parameter.
The provided code then forwards the processing to
\textit{main} or \textit{dest} as described in \secref{sec:forward}.

%%%%%%%%%%%%%%%%%%%%%%%%%%%%%%%%%%%%%%%%%%%%%%%%%%%%%%%%%%%%%%%%%%%%%%%%%%%%%%%%
\subsection{Include by Input}
\label{sec:input}

Including child documents by |\include| has some restrictions by design.
Most notably, the content of a child document always occupies
its own set of pages; pages cannot be shared between child documents.
Usually, this behaviour makes perfect sense
because each child document contain an essential part of the document.
However, in some situations it may be desirable to compose
a document from a collection of parts
without having mandatory page breaks between then.
For this case, the package
provides a mechanism to include parts
by |\input| which can also be processed individually.
However, by construction this mechanism
requires manual handling of the content to be output.

%%%%%%%%%%%%%%%%%%%%%%%%%%%%%%%%%%%%%%%%
\DescribeMacro{\ifchilddocmanual}
The main file should be prepared as usual, see \secref{sec:include}.
However, the document body must make a distinction
between processing of an individual part and of the main document, e.g.:
%
\begin{center}
\begin{tabular}{l}
|\ifchilddocmanual|\\
|\input{\childdocname}|\\
|\||else|\\
\textit{document body with }|\input{|\textit{part}|}|\\
|\||fi|
\end{tabular}
\end{center}
%
The conditional |\ifchilddocmanual| is true whenever
a part to be included by |\input| is being compiled,
and the name of the part is stored in |\childdocname|.

%%%%%%%%%%%%%%%%%%%%%%%%%%%%%%%%%%%%%%%%
\DescribeMacro{\childdocby}
Each part to be included by |\input| should start with:
%
\begin{center}
\begin{tabular}{l}
|\input{childdoc.def}|\\
|\childdocby{|\textit{main}|}|\\
\end{tabular}
\end{center}
%
The directive |\childdocby| is similar to |\childdocof|
described in \secref{sec:include},
but the subsequent selection of content must be done manually.
To that end, both |\ifchilddoc| and |\ifchilddocmanual|
will be true upon processing of a part,
and the name of the part is stored in |\childdocname|.
Note that |\jobname| will be set to the filename of the current part
so that each part receives an individual |.aux| file
that does not interfere with the |.aux| file(s) of the main document.
This behaviour can be altered by the alternative form
|\childdocby[*]{|\textit{main}|}| (with a non-empty optional argument)
which uses the |.aux| file of the main document
by setting |\jobname| to \textit{main}.

%%%%%%%%%%%%%%%%%%%%%%%%%%%%%%%%%%%%%%%%%%%%%%%%%%%%%%%%%%%%%%%%%%%%%%%%%%%%%%%%
\subsection{Driver Development}
\label{sec:driver}

The \textsf{childdoc} mechanism can also be use for the development
of definition files such as \LaTeX{} styles or classes.
This case differs from the above setup with multiple parts
included by |\include| in that no |\includeonly| should be invoked.
This can be achieved by starting the include file
(before |\ProvidesPackage|) with:
%
\begin{center}
\begin{tabular}{l}
|\input{childdoc.def}|\\
|\childdocforward{|\textit{main}|}|\\
\end{tabular}
\end{center}
%
or alternatively with:
%
\begin{center}
\begin{tabular}{l}
|\input{childdoc.def}|\\
|\childdocby{|\textit{main}|}|\\
\end{tabular}
\end{center}
%
Both forms have slightly different effects as described above.
The main file is prepared as usual, see \secref{sec:include}.

%%%%%%%%%%%%%%%%%%%%%%%%%%%%%%%%%%%%%%%%%%%%%%%%%%%%%%%%%%%%%%%%%%%%%%%%%%%%%%%%
\subsection{Legacy Detection}
\label{sec:detection}

The directive |\childdocmain| in the main file can detect
whether the complete document or merely a child is to be compiled
even without using the directive |\childdocof|.
This method is deprecated because it is less robust
and there is no compelling reason to use it;
it is merely provided for backward compatibility
and it may be removed in future versions.

If the detection mechanism is to be used,
it is mandatory to correctly specify
the filename of the main file as the argument of |\childdocmain|:
%
\begin{center}
\begin{tabular}{l}
|\input{childdoc.def}|\\
|\childdocmain{|\textit{main}|}|\\
\end{tabular}
\end{center}
%
If |\jobname| does not match the argument \textit{main} of |\childdocmain|,
it is assumed that |\jobname| points to the child file to be compiled.
When using |\childdocmain| with the main file specified as argument,
it suffices to start a child file
with just |\input{|\textit{main}|}|
without loading of the package and using |\childdocof|.
If instead all processing is done
with the appropriate \textsf{childdoc} directives,
the argument of \textit{main} of |\childdocmain| can be empty.

An alternative version of the command line processing described
in \secref{sec:commandline} using the detection mechanism reads:
%
\begin{center}
|... -jobname "|\textit{target}|" "|[\textit{flags}]%
[|\def\jobname{|\textit{dest}|}|]|\input{|\textit{main}|}"|
\end{center}

%%%%%%%%%%%%%%%%%%%%%%%%%%%%%%%%%%%%%%%%%%%%%%%%%%%%%%%%%%%%%%%%%%%%%%%%%%%%%%%%
\subsection{Manual Code}
\label{sec:manual}

In case one cannot be certain whether the definitions file |childdoc.def|
is installed on the target \TeX{} distribution
and one prefers not to ship it,
it is conceivable to paste a few relevant commands into the sources.

To that end, drop all statements |\input{childdoc.def}|
and perform the replacements as outlined below.
Instead of |\childdocmain{|\textit{main}|}| add the following code
to the top of the main file:
%
\begin{center}
\begin{tabular}{l}
|\||ifdefined\childdocname\endinput\||fi\newif\ifchilddoc|\\
|\edef\childdocname{\scantokens\expandafter{\jobname\noexpand}}|\\
|\def\childdocmain{|\textit{main}|}\||ifx\childdocmain\childdocname\||else|\\
|\childdoctrue\includeonly{\childdocname}\let\jobname\childdocmain\||fi|\\
\end{tabular}
\end{center}
%
Instead of |\childdocof{|\textit{main}|}| just include the main file
at the top of each child file:
%
\begin{center}
|\input{|\textit{main}|}|
\end{center}
%
A simple redirection |\childdocforward{|\textit{dest}|}| is achieved by:
%
\begin{center}
|\def\jobname{|\textit{dest}|}\input{\jobname}|
\end{center}
%
The redirection with prefix
|\childdocforwardprefix[|\textit{prefix}|]{|\textit{dest}|}|
is accomplished by:
%
\begin{center}
\begin{tabular}{l}
|{\edef\jobname{\scantokens\expandafter{\jobname\noexpand}}|\\
|\def\redirectjob |\textit{prefix}|#1~~~{\gdef\jobname{|\textit{dest}|#1}}|\\
|\expandafter\redirectjob\jobname~~~}\input{\jobname}|
\end{tabular}
\end{center}

In an alternative approach,
child documents can be compiled by a specific command line
without additional code or specific definitions:
%
\begin{center}
|... -jobname "|\textit{target}|" "|[\textit{flags}]%
|\includeonly{|\textit{dest}|}\input{|\textit{main}|}"|
\end{center}
%

%%%%%%%%%%%%%%%%%%%%%%%%%%%%%%%%%%%%%%%%%%%%%%%%%%%%%%%%%%%%%%%%%%%%%%%%%%%%%%%%
%%%%%%%%%%%%%%%%%%%%%%%%%%%%%%%%%%%%%%%%%%%%%%%%%%%%%%%%%%%%%%%%%%%%%%%%%%%%%%%%
\section{Information}

%%%%%%%%%%%%%%%%%%%%%%%%%%%%%%%%%%%%%%%%%%%%%%%%%%%%%%%%%%%%%%%%%%%%%%%%%%%%%%%%
\subsection{Copyright}

Copyright \copyright{} 2017--2018 Niklas Beisert

This work may be distributed and/or modified under the
conditions of the \LaTeX{} Project Public License, either version 1.3
of this license or (at your option) any later version.
The latest version of this license is in
  \url{http://www.latex-project.org/lppl.txt}
and version 1.3 or later is part of all distributions of \LaTeX{}
version 2005/12/01 or later.

This work has the LPPL maintenance status `maintained'.

The Current Maintainer of this work is Niklas Beisert.

This work consists of the files |README.txt|, |childdoc.ins| and |childdoc.dtx|
as well as the derived files |childdoc.def|, |cdocsamp.tex|
with |cdocsch1.tex|, |cdocsch2.tex|, |cdocspt3.tex|, |cdocspt4.tex|,
|cdocsdrf.tex|, |cdocsfn1.tex|, |cdocsfn2.tex|
as well as |childdoc.pdf|.

%%%%%%%%%%%%%%%%%%%%%%%%%%%%%%%%%%%%%%%%%%%%%%%%%%%%%%%%%%%%%%%%%%%%%%%%%%%%%%%%
\subsection{Files and Installation}

The package consists of the files:
%
\begin{center}
\begin{tabular}{ll}
    |README.txt|   & readme file \\
    |childdoc.ins| & installation file \\
    |childdoc.dtx| & source file \\
    |childdoc.def| & definition file \\
    |cdocsamp.tex| & sample main file \\
    |cdocsch1.tex| & sample include file \\
    |cdocsch2.tex| & sample include file \\
    |cdocspt3.tex| & sample part file \\
    |cdocspt4.tex| & sample part file \\
    |cdocsdrf.tex| & sample redirection file \\
    |cdocsfn1.tex| & sample redirection file \\
    |cdocsfn2.tex| & sample redirection file \\
    |childdoc.pdf| & manual
\end{tabular}
\end{center}
%
The distribution consists of the files
|README.txt|, |childdoc.ins| and |childdoc.dtx|.
%
\begin{itemize}
\item
Run (pdf)\LaTeX{} on |childdoc.dtx|
to compile the manual |childdoc.pdf| (this file).
\item
Run \LaTeX{} on |childdoc.ins| to create the definitions file |childdoc.def|
and the sample |cdocsamp.tex| with include files
|cdocsch1.tex|, |cdocsch2.tex|, |cdocspt3.tex|, |cdocspt4.tex|,
|cdocsdrf.tex|, |cdocsfn1.tex|, |cdocsfn2.tex|.
Then copy the file |childdoc.def| to an appropriate directory of your \LaTeX{}
distribution, e.g.\ \textit{texmf-root}|/tex/latex/childdoc|.
\end{itemize}

%%%%%%%%%%%%%%%%%%%%%%%%%%%%%%%%%%%%%%%%%%%%%%%%%%%%%%%%%%%%%%%%%%%%%%%%%%%%%%%%
\subsection{Related CTAN Packages}

There are several other packages which offer a similar functionality:
%
\begin{itemize}
\item
The packages
\href{http://ctan.org/pkg/docmute}{\textsf{docmute}},
\href{http://ctan.org/pkg/includex}{\textsf{includex}} and
\href{http://ctan.org/pkg/standalone}{\textsf{standalone}}
provide commands to include only the document body of
a child file thus allowing both files to be compiled individually.
\item
The packages \href{http://ctan.org/pkg/subdocs}{\textsf{subdocs}}
and \href{http://ctan.org/pkg/subfiles}{\textsf{subfiles}}
provide structures in which the main and child documents can be
encapsulated and allowing them to be compiled individually.
The inclusion mechanism is different from the conventional |\include|.
\item
The package \href{http://ctan.org/pkg/combine}{\textsf{combine}}
is an elaborate solution to combine several documents into one.
\end{itemize}
%
See also the CTAN topic \href{http://ctan.org/topic/subdocs}{\textsf{subdocs}}
for further related packages.
The present package differs from the above solutions in that
a document structure constructed with the conventional |\include| mechanism
just needs two extra commands at the top of every file
such that all constituent files can be compiled individually.

%%%%%%%%%%%%%%%%%%%%%%%%%%%%%%%%%%%%%%%%%%%%%%%%%%%%%%%%%%%%%%%%%%%%%%%%%%%%%%%%
%\subsection{Feature Suggestions}
%
%The following is a list of features which may be useful for future
%versions of this package:
%%
%\begin{itemize}
%\item
%\ldots
%\end{itemize}

%%%%%%%%%%%%%%%%%%%%%%%%%%%%%%%%%%%%%%%%%%%%%%%%%%%%%%%%%%%%%%%%%%%%%%%%%%%%%%%%
\subsection{Revision History}

%%%%%%%%%%%%%%%%%%%%%%%%%%%%%%%%%%%%%%%%
\paragraph{v2.0:} 2018/12/30

\begin{itemize}
\item
immediate forward processing
\item
added |\childdocby| mechanism
\item
manual restructured
\end{itemize}

%%%%%%%%%%%%%%%%%%%%%%%%%%%%%%%%%%%%%%%%
\paragraph{v1.6:} 2018/01/17

\begin{itemize}
\item
application for development of include files
\item
corrections to manual
\end{itemize}

%%%%%%%%%%%%%%%%%%%%%%%%%%%%%%%%%%%%%%%%
\paragraph{v1.5:} 2017/05/21

\begin{itemize}
\item
more complete structuring introduced
\item
|\childdocof| introduced
\item
|\childdoc| renamed to |\childdocmain|
\item
|\childredirect| renamed to |\childdocforward| and |\childdocforwardprefix|
and functionality expanded
\end{itemize}

%%%%%%%%%%%%%%%%%%%%%%%%%%%%%%%%%%%%%%%%
\paragraph{v1.0:} 2017/04/27

\begin{itemize}
\item
manual and install package
\item
first version published on CTAN
\end{itemize}

%%%%%%%%%%%%%%%%%%%%%%%%%%%%%%%%%%%%%%%%
\paragraph{v0.6:} 2017/04/26

\begin{itemize}
\item
redirection mechanism added
\end{itemize}

%%%%%%%%%%%%%%%%%%%%%%%%%%%%%%%%%%%%%%%%
\paragraph{v0.5:} 2017/04/26

\begin{itemize}
\item
functionality in definition file
\end{itemize}


%%%%%%%%%%%%%%%%%%%%%%%%%%%%%%%%%%%%%%%%%%%%%%%%%%%%%%%%%%%%%%%%%%%%%%%%%%%%%%%%
%%%%%%%%%%%%%%%%%%%%%%%%%%%%%%%%%%%%%%%%%%%%%%%%%%%%%%%%%%%%%%%%%%%%%%%%%%%%%%%%
%%%%%%%%%%%%%%%%%%%%%%%%%%%%%%%%%%%%%%%%%%%%%%%%%%%%%%%%%%%%%%%%%%%%%%%%%%%%%%%%
\appendix

\settowidth\MacroIndent{\rmfamily\scriptsize 000\ }

 \DocInput{childdoc.dtx}

\end{document}
%</driver>
% \fi
%
% %%%%%%%%%%%%%%%%%%%%%%%%%%%%%%%%%%%%%%%%%%%%%%%%%%%%%%%%%%%%%%%%%%%%%%%%%%%%%%
% %%%%%%%%%%%%%%%%%%%%%%%%%%%%%%%%%%%%%%%%%%%%%%%%%%%%%%%%%%%%%%%%%%%%%%%%%%%%%%
% \section{Sample}
%\iffalse
%<*samplemain>
%\fi
%
% The following presents a sample document
% with two chapters, two parts, a title page,
% a compile flag as well as three forwarding files to set the flag.
% It consists of eight |.tex| files:
% \begin{center}
% \begin{tabular}{ll}
% |cdocsamp.tex|&main file\\
% |cdocsch1.tex|&include file for chapter 1\\
% |cdocsch2.tex|&include file for chapter 2\\
% |cdocspt3.tex|&include file for part 3\\
% |cdocspt4.tex|&include file for part 4\\
% |cdocsdrf.tex|&forwarding file for main file in draft mode\\
% |cdocsfi1.tex|&forwarding file for final version of chapter 1\\
% |cdocsfi2.tex|&forwarding file for final version of chapter 2\\
% \end{tabular}
% \end{center}
% Each of the eight files can be compiled directly by the \LaTeX{} compiler.
%
% %%%%%%%%%%%%%%%%%%%%%%%%%%%%%%%%%%%%%%
% \paragraph{Main File.}
%
% The main file is called |cdocsamp.tex|.
%
% Load the \textsf{childdoc} definitions and
% declare the filename for the main document:
%    \begin{macrocode}
\input{childdoc.def}
\childdocmain{}
%    \end{macrocode}

% Optional override for |\version| flag:
%    \begin{macrocode}
%%\ifchilddoc\else\providecommand{\version}{draft}\fi
%    \end{macrocode}

% Define the default values for the |\version| flag
% (|final| for the main file and |draft| for childs):
%    \begin{macrocode}
\ifchilddoc
\providecommand{\version}{draft}
\else
\providecommand{\version}{final}
\fi
%    \end{macrocode}

% Load the standard document class:
%    \begin{macrocode}
\documentclass[12pt]{article}
%    \end{macrocode}

% Start the document body:
%    \begin{macrocode}
\begin{document}
%    \end{macrocode}

% Declare a title page.
% Print title, part of document being processed and version flag:
%    \begin{macrocode}
\addtocounter{page}{-1}
\begin{center}
{\LARGE\bfseries{}childdoc example\par}
\vspace{1cm}
\ifchilddoc
\ifchilddocmanual part\else chapter\fi:
`\childdocname' of `\childdocjob'\par
\else
main document: `\childdocjob'\par
\fi
version: \version\par
\end{center}
\newpage
%    \end{macrocode}

% Manually include selected file,
% otherwise process as usual:
%    \begin{macrocode}
\ifchilddocmanual
\section*{part `\childdocname'}
\input{\childdocname}
\else
%    \end{macrocode}

% Include the two chapters:
%    \begin{macrocode}
\include{cdocsch1}
\include{cdocsch2}
%    \end{macrocode}

% Include the two parts unless only chapters should be displayed:
%    \begin{macrocode}
\ifchilddoc\else
\section{part three}
\input{cdocspt3}
\section{part four}
\input{cdocspt4}
\fi
%    \end{macrocode}

% Process as usual until here:
%    \begin{macrocode}
\fi
%    \end{macrocode}

% End of document body:
%    \begin{macrocode}
\end{document}
%    \end{macrocode}
%\iffalse
%</samplemain>
%\fi
%
% %%%%%%%%%%%%%%%%%%%%%%%%%%%%%%%%%%%%%%
% \paragraph{Chapter Include Files.}
%
% The include files are called |cdocsch1.tex| and |cdocsch2.tex|.
%
%\iffalse
%<*samplechap1|samplechap2>
%\fi

% Optional override for |\version| flag:
%    \begin{macrocode}
%%\providecommand{\version}{final}
%    \end{macrocode}

% Include the main document:
%    \begin{macrocode}
\input{childdoc.def}
\childdocof{cdocsamp}
%    \end{macrocode}

%\iffalse
%</samplechap1|samplechap2>
%\fi
%
%\iffalse
%<*samplechap1>
%\fi
% Some text for chapter 1:
%    \begin{macrocode}
\section{one}
some text in chapter one
%    \end{macrocode}

%\iffalse
%</samplechap1>
%\fi
% Some text for chapter 2:
%\iffalse
%<*samplechap2>
%\fi
%    \begin{macrocode}
\section{two}
more text in chapter two
%    \end{macrocode}

%\iffalse
%</samplechap2>
%\fi
%
% %%%%%%%%%%%%%%%%%%%%%%%%%%%%%%%%%%%%%%
% \paragraph{Part Include Files.}
%
% The include files are called |cdocspt3.tex| and |cdocspt4.tex|.
%
%\iffalse
%<*samplepart3|samplepart4>
%\fi

% Optional override for |\version| flag:
%    \begin{macrocode}
%%\providecommand{\version}{final}
%    \end{macrocode}

% Include the main document:
%    \begin{macrocode}
\input{childdoc.def}
\childdocby{cdocsamp}
%    \end{macrocode}

%\iffalse
%</samplepart3|samplepart4>
%\fi
%
%\iffalse
%<*samplepart3>
%\fi
% Some text for part 3:
%    \begin{macrocode}
some text in part three
%    \end{macrocode}

%\iffalse
%</samplepart3>
%\fi
% Some text for part 4:
%\iffalse
%<*samplepart4>
%\fi
%    \begin{macrocode}
more text in part four
%    \end{macrocode}

%\iffalse
%</samplepart4>
%\fi
%
% %%%%%%%%%%%%%%%%%%%%%%%%%%%%%%%%%%%%%%
% \paragraph{Forwarding for a Complete Draft.}
%
% The following forwarding file |cdocsdrf.tex|
% compiles the main document in draft mode:
%\iffalse
%<*sampledraft>
%\fi
%    \begin{macrocode}
\def\version{draft}
\input{childdoc.def}
\childdocforward{cdocsamp}
%    \end{macrocode}

%\iffalse
%</sampledraft>
%\fi
%
% %%%%%%%%%%%%%%%%%%%%%%%%%%%%%%%%%%%%%%
% \paragraph{Forwarding for Final Version of the Chapters.}
%
% The following forwarding files |cdocsfn1.tex| and |cdocsfn2.tex|
% (with identical content)
% compile the final versions of the child documents
% |cdocsch1.tex| and |cdocsch2.tex|, respectively:
%\iffalse
%<*samplefinal>
%\fi
%    \begin{macrocode}
\def\version{final}
\input{childdoc.def}
\childdocforwardprefix[cdocsamp]{cdocsfn}{cdocsch}
%    \end{macrocode}

%\iffalse
%</samplefinal>
%\fi
%
% %%%%%%%%%%%%%%%%%%%%%%%%%%%%%%%%%%%%%%
% \paragraph{Command Line Processing.}
%
% The following three command lines generate the output files
% |cdocscld|, |cdocscl1| and |cdocscl2|
% which should be identical to
% |cdocsdrf|, |cdocsch1| and |cdocsfn2|, respectively:
% \begin{center}
% \begin{tabular}{l}
% |latex -jobname cdocscld \|\\
% |  "\def\version{draft}\input{childdoc.def}\childdocforward{cdocsamp}"|\\
% |latex -jobname cdocscl1 \|\\
% |  "\input{childdoc.def}\childdocforward[cdocsamp]{cdocsch1}"|\\
% |latex -jobname cdocscl2 \|\\
% |  "\def\version{final}\input{childdoc.def}\childdocforward{cdocsch2}"|
% \end{tabular}
% \end{center}
% Note that the trailing backslash on each first line
% merely continues the input to the second line
% (for convenient cut ant paste).
% Furthermore, the command |latex| can be replaced by any
% of its alternative versions such as |pdflatex|.
%
% %%%%%%%%%%%%%%%%%%%%%%%%%%%%%%%%%%%%%%%%%%%%%%%%%%%%%%%%%%%%%%%%%%%%%%%%%%%%%%
% %%%%%%%%%%%%%%%%%%%%%%%%%%%%%%%%%%%%%%%%%%%%%%%%%%%%%%%%%%%%%%%%%%%%%%%%%%%%%%
% \section{Implementation}
%\iffalse
%<*package>
%\fi
%
% This section describes the definitions file |childdoc.def|.

% The definitions cannot be loaded using |\usepackage| or |\RequirePackage|
% which has a mechanism to prevent loading a style file more than once.
% When loading the definitions by means of |\input|
% multiple instances have to be prevented manually:
%\iffalse
%This code needs to be before the `\ProvidesFile' directive
%which is defined at the beginning of this file.
%Therefore it is also placed there and commented out here.
%</package>
%<*discard>
%\fi
%    \begin{macrocode}
\ifdefined\childdocmain\endinput\fi
%    \end{macrocode}
%\iffalse
%</discard>
%<*package>
%\fi
%
% \macro{\ifchilddoc}
% \macro{\ifchilddocmanual}
% The conditional |\ifchilddoc| tells whether a
% child (true) or main (false) document is being compiled.
% The conditional |\ifchilddocmanual| tells whether
% the |\includeonly| mechanism is used (false) or
% the selection of child files must be performed manually (true).
% The definitions initialise to false:
%    \begin{macrocode}
\newif\ifchilddoc
\newif\ifchilddocmanual
%    \end{macrocode}

% \macro{\childdocname}
% \macro{\childdocjob}
% The macro |\childdocname| stores the name of the main document
% to be compiled. The macro |\childdocjob| stores the name of
% the document on which the \LaTeX{} compiler was originally invoked.
% The content of |\jobname| cannot be compared
% to filenames specified in the source due to different catcodes.
% The following code rescans |\jobname|, stores the result
% in |\childdocname| and saves a copy in |\childdocjob|:
%    \begin{macrocode}
\edef\childdocname{\scantokens\expandafter{\jobname\noexpand}}
\let\childdocjob\childdocname
%    \end{macrocode}

% \macro{\childdocdisable}
% The macro |\childdocdisable| prevents the main file
% from being processed more than once.
% At this stage, the main document command |\childdocmain|
% is assumed to be called once again where it should do nothing.
% Any subsequent call to it should prevent
% a secondary processing of the main document
% It overwrites the forwarding commands
% |\childdocof| and |\childdocforward|
% with empty macros to prevent further inclusions of the main document:
%    \begin{macrocode}
\newcommand{\childdocdisable}
{
  \renewcommand{\childdocmain}[1]{\renewcommand{\childdocmain}[1]{\endinput}}
  \renewcommand{\childdocof}[1]{}
  \renewcommand{\childdocby}[2][]{}
  \renewcommand{\childdocforward}[2][]{}
  \renewcommand{\childdocdisable}{}
}
%    \end{macrocode}

% \macro{\childdocmain}
% The macro |\childdocmain| is to be called at the top of the main file
% with nothing or the main filename (without extension) as argument.
% First, it breaks loops.
% If the argument is not empty and does not match |\childdocname|
% (which is set by the first inclusion of |childdoc.def|),
% |\ifchilddoc| is set to true, |\includeonly| is applied to the child file
% and |\jobname| is set to the main file
% (for proper handling of |.aux| files):
%    \begin{macrocode}
\newcommand{\childdocmain}[1]
{
  \childdocdisable\childdocmain{}
  \if?#1?\else
    \begingroup
      \def\childdoctmp{#1}
      \ifx\childdoctmp\childdocname
        \def\childdoctmp{}
      \else
        \def\childdoctmp
        {
          \childdoctrue
          \includeonly{\childdocname}
          \def\childdocjob{#1}
          \def\jobname{#1}
        }
      \fi
      \expandafter
    \endgroup
    \childdoctmp
  \fi
}
%    \end{macrocode}

% \macro{\childdocof}
% The command |\childdocof| redirects
% compilation to the main file |#1|.
%    \begin{macrocode}
\newcommand{\childdocof}[1]
{
  \childdocdisable
  \childdoctrue
  \includeonly{\childdocname}
  \def\jobname{#1}
  \def\childdocjob{#1}
  \input{#1}
}
%    \end{macrocode}

% \macro{\childdocby}
% The command |\childdocby| ....
%    \begin{macrocode}
\newcommand{\childdocby}[2][]
{
  \childdocdisable
  \childdoctrue
  \childdocmanualtrue
  \if?#1?\else
    \def\jobname{#2}
  \fi
  \def\childdocjob{#2}
  \input{#2}
  \endinput
}
%    \end{macrocode}

% \macro{\childdocforward}
% The command |\childdocforward| redirects
% compilation to the main file or
% (if the optional argument is given) a child file.
% Parameters are set as if the main file
% or a child file starting with |\childdocof| was compiled.
% Then compilation is handed over to the main file:
%    \begin{macrocode}
\newcommand{\childdocforward}[2][]
{
  \begingroup
    \if?#1?
      \def\childdoctmp
      {
        \def\childdocname{#2}
        \def\childdocjob{#2}
        \def\jobname{#2}
        \input{#2}
        \endinput
      }
    \else
      \def\childdoctmp
      {
        \childdocdisable
        \def\childdocname{#2}
        \childdoctrue
        \includeonly{#2}
        \def\childdocjob{#1}
        \def\jobname{#1}
        \input{#1}
        \endinput
      }
    \fi
    \expandafter
  \endgroup
  \childdoctmp
}
%    \end{macrocode}

% \macro{\childdocforwardprefix}
% The command |\childdocforwardprefix| redirects
% compilation to the main or a child file by means of a pattern.
% The prefix |#1| in the current filename is replaced by |#2|
% and the suffix of the current filename is kept
% (it is assumed that the filename does not contain the substring `|~~~|'
% which is used as a delimiter).
% Compilation is handed over to the new file by |\childdocforward|:
%    \begin{macrocode}
\newcommand{\childdocforwardprefix}[3][]
{
  \begingroup
    \def\childdocextract #2##1~~~{\def\childdoctmp{\childdocforward[#1]{#3##1}}}
    \expandafter\childdocextract\childdocname~~~
    \expandafter
  \endgroup
  \childdoctmp
}
%    \end{macrocode}

% \macro{\childdoc}
% The deprecated macro |\childdoc| is a legacy version of |\childdocmain|:
%    \begin{macrocode}
\newcommand{\childdoc}{\childdocmain}
%    \end{macrocode}

% \macro{\childdocredirect}
% The deprecated macro |\childdocredirect| is a legacy version
% of |\childdocforward| and |\childdocforwardprefix|:
%    \begin{macrocode}
\newcommand{\childdocredirect}[2][]
{
  \begingroup
    \if?#1?
      \def\childdoctmp{\childdocforward{#2}}
    \else
      \def\childdoctmp{\childdocforwardprefix{#1}{#2}}
    \fi
    \expandafter
  \endgroup
  \childdoctmp
}
%    \end{macrocode}

%\iffalse
%</package>
%\fi
%
\endinput
|\\
|\childdocmain{|\textit{main}|}|\\
\end{tabular}
\end{center}
%
If |\jobname| does not match the argument \textit{main} of |\childdocmain|,
it is assumed that |\jobname| points to the child file to be compiled.
When using |\childdocmain| with the main file specified as argument,
it suffices to start a child file
with just |\input{|\textit{main}|}|
without loading of the package and using |\childdocof|.
If instead all processing is done
with the appropriate \textsf{childdoc} directives,
the argument of \textit{main} of |\childdocmain| can be empty.

An alternative version of the command line processing described
in \secref{sec:commandline} using the detection mechanism reads:
%
\begin{center}
|... -jobname "|\textit{target}|" "|[\textit{flags}]%
[|\def\jobname{|\textit{dest}|}|]|\input{|\textit{main}|}"|
\end{center}

%%%%%%%%%%%%%%%%%%%%%%%%%%%%%%%%%%%%%%%%%%%%%%%%%%%%%%%%%%%%%%%%%%%%%%%%%%%%%%%%
\subsection{Manual Code}
\label{sec:manual}

In case one cannot be certain whether the definitions file |childdoc.def|
is installed on the target \TeX{} distribution
and one prefers not to ship it,
it is conceivable to paste a few relevant commands into the sources.

To that end, drop all statements |% \iffalse
%
% childdoc.dtx Copyright (C) 2017-2018 Niklas Beisert
%
% This work may be distributed and/or modified under the
% conditions of the LaTeX Project Public License, either version 1.3
% of this license or (at your option) any later version.
% The latest version of this license is in
%   http://www.latex-project.org/lppl.txt
% and version 1.3 or later is part of all distributions of LaTeX
% version 2005/12/01 or later.
%
% This work has the LPPL maintenance status `maintained'.
%
% The Current Maintainer of this work is Niklas Beisert.
%
% This work consists of the files childdoc.dtx and childdoc.ins
% and the derived files childdoc.def and cdocsamp.tex with
% cdocsch1.tex, cdocsch2.tex, cdocsdrf.tex, cdocsfn1.tex, cdocsfn2.tex.
%
%<package>\ifdefined\childdocmain\endinput\fi
%<package>\ProvidesFile{childdoc.def}[2018/12/30 v2.0 child document driver]
%<samplemain>\ProvidesFile{cdocsamp.tex}[2018/12/30 v2.0 sample for childdoc]
%<*driver>
%\ProvidesFile{childdoc.drv}[2018/12/30 v2.0 childdoc reference manual file]
\PassOptionsToClass{10pt,a4paper}{article}
\documentclass{ltxdoc}

\usepackage[margin=35mm]{geometry}
\usepackage{hyperref}
\usepackage{hyperxmp}
\usepackage[usenames]{color}

\hypersetup{colorlinks=true}
\hypersetup{pdfstartview=FitH}
\hypersetup{pdfpagemode=UseNone}
\hypersetup{pdfsource={}}
\hypersetup{pdflang={en-UK}}
\hypersetup{pdfcopyright={Copyright 2017-2018 Niklas Beisert.
  This work may be distributed and/or modified under the
  conditions of the LaTeX Project Public License, either version 1.3
  of this license or (at your option) any later version.}}
\hypersetup{pdflicenseurl={http://www.latex-project.org/lppl.txt}}
\hypersetup{pdfcontactaddress={ETH Zurich, ITP, HIT K,
  Wolfgang-Pauli-Strasse 27}}
\hypersetup{pdfcontactpostcode={8093}}
\hypersetup{pdfcontactcity={Zurich}}
\hypersetup{pdfcontactcountry={Switzerland}}
\hypersetup{pdfcontactemail={nbeisert@itp.phys.ethz.ch}}
\hypersetup{pdfcontacturl={http://people.phys.ethz.ch/\xmptilde nbeisert/}}

\newcommand{\secref}[1]{\hyperref[#1]{section \ref*{#1}}}

\parskip1ex
\parindent0pt
\let\olditemize\itemize
\def\itemize{\olditemize\parskip0pt}

\begin{document}

\title{The \textsf{childdoc} Package}
\hypersetup{pdftitle={The childdoc Package}}
\author{Niklas Beisert\\[2ex]
  Institut f\"ur Theoretische Physik\\
  Eidgen\"ossische Technische Hochschule Z\"urich\\
  Wolfgang-Pauli-Strasse 27, 8093 Z\"urich, Switzerland\\[1ex]
  \href{mailto:nbeisert@itp.phys.ethz.ch}
  {\texttt{nbeisert@itp.phys.ethz.ch}}}
\hypersetup{pdfauthor={Niklas Beisert}}
\hypersetup{pdfsubject={Manual for the LaTeX2e Package childdoc}}
\date{30 December 2018, \textsf{v2.0}}
\maketitle

\begin{abstract}\noindent
\textsf{childdoc} is a \LaTeXe{} package
that enables the direct compilation
of document sections included by |\include|
to individual files.
\end{abstract}

\begingroup
\parskip0ex
\tableofcontents
\endgroup

%%%%%%%%%%%%%%%%%%%%%%%%%%%%%%%%%%%%%%%%%%%%%%%%%%%%%%%%%%%%%%%%%%%%%%%%%%%%%%%%
%%%%%%%%%%%%%%%%%%%%%%%%%%%%%%%%%%%%%%%%%%%%%%%%%%%%%%%%%%%%%%%%%%%%%%%%%%%%%%%%
\section{Introduction}

\LaTeX{} provides a mechanism to structure a large document (such as a book)
into a main file and several child files (containing the chapters)
using the |\include| command.
This mechanism is beneficial for documents
which span hundreds of pages in order to
make the source file(s) more manageable.
Moreover, compilation can be restricted to
selected child files by means of the |\includeonly| command.
The latter feature can be used to reduce the compilation time while editing
(this was significantly more useful in the earlier days of \LaTeX{})
or to generate a smaller document which is easier to navigate.
Another application of |\includeonly| is to generate
documents consisting of selected parts of the complete document.

However, there are a few drawbacks of the plain |\include| mechanism:
\begin{itemize}
\item
The child files cannot be compiled on their own,
they can only be compiled via the main file.
A naive editing environment
(such as a text editor with an option
to have the current file processed by \LaTeX)
may require one to switch to the main file before compiling;
attempting to compile the child file produces errors.
\item
The main file must be modified (each time)
to adjust the |\includeonly| command
to the present needs. This easily leaves the main file in a messy state.
\item
The generated document will always carry the filename
of the main document. This is inconvenient if
several child files are to be compiled and
to be kept for distribution.
\end{itemize}

The present package provides a simple interface
to make child files individually compilable by \LaTeX{}.
Compiling a child file then has the same effect as compiling
the main file with an |\includeonly| command
to select the appropriate child.
Moreover the generated document will carry the name of the child
rather than the main file.
This resolves all three above issues.

This feature is meant to make the editing of books,
thesis documents and lecture notes somewhat more convenient.
However, the package can also be used efficiently for
composing a series of documents (such as exercise sheets)
which are typically distributed individually.
It then assists the author in generating the individual documents
(potentially in different versions)
as well as a document containing the collected series.
Another application is in developing style files
or other kinds of included material
where compilation of the style file could redirect
to a sample or test file.

%%%%%%%%%%%%%%%%%%%%%%%%%%%%%%%%%%%%%%%%%%%%%%%%%%%%%%%%%%%%%%%%%%%%%%%%%%%%%%%%
%%%%%%%%%%%%%%%%%%%%%%%%%%%%%%%%%%%%%%%%%%%%%%%%%%%%%%%%%%%%%%%%%%%%%%%%%%%%%%%%
\section{Usage}

First of all, the package \textsf{childdoc} is \emph{not} a standard
\LaTeXe{} |.sty| style file! Therefore it needs to be invoked in
a non-standard way.

%%%%%%%%%%%%%%%%%%%%%%%%%%%%%%%%%%%%%%%%%%%%%%%%%%%%%%%%%%%%%%%%%%%%%%%%%%%%%%%%
\subsection{Included Files}
\label{sec:include}

%%%%%%%%%%%%%%%%%%%%%%%%%%%%%%%%%%%%%%%%
\DescribeMacro{\childdocmain}
To use the package, add the commands
\begin{center}
\begin{tabular}{l}
|\input{childdoc.def}|\\
|\childdocmain{}|\\
\end{tabular}
\end{center}
at the very top of the main \LaTeX{} file,
in particular \emph{before} the |\documentclass| statement!
The argument of |\childdocmain| should be left empty
(but it must be present).

%%%%%%%%%%%%%%%%%%%%%%%%%%%%%%%%%%%%%%%%
\DescribeMacro{\childdocof}
Furthermore, add the commands
\begin{center}
\begin{tabular}{l}
|\input{childdoc.def}|\\
|\childdocof{|\textit{main}|}|\\
\end{tabular}
\end{center}
at the top of every child file \textit{child}
which is included by |\include{|\textit{child}|}|
from within the main file
(or at least for those files to be compiled individually).
The argument \textit{main} must be the filename of the main file.

There are a couple of
considerations in setting up the main and child documents:

%%%%%%%%%%%%%%%%%%%%%%%%%%%%%%%%%%%%%%%%
\paragraph{Restrictions.}

Please note the following restrictions:
\begin{itemize}
\item
|\childdocmain| must be called with one argument \textit{main}
to ensure compatibility with earlier version of the package.
It must either be empty (|\childdocmain{}|)
or precisely match the filename of the main file in which it is specified.
See \secref{sec:detection} for further information.
\item
The filename \textit{main} must be specified without the |.tex| extension.
\item
The filename \textit{main} is case sensitive
(even in case-insensitive file systems)
due to internal string comparison.
\item
The argument \textit{main} should be fully expanded, it cannot be a macro.
\item
Subdirectories and special characters should be avoided in filenames.
\item
The command |\childdocmain{|\textit{main}|}| must be followed by a whitespace.
It should not be followed immediately by another command
or by a comment mark `|%|'.
This is because the \TeX{} parser reads the token immediately following
the argument of |\childdocmain| and puts it
at the beginning of every child section;
however, a white\-space is ignored.
\end{itemize}

%%%%%%%%%%%%%%%%%%%%%%%%%%%%%%%%%%%%%%%%
\paragraph{Content of Main File.}

It is advisable to place all content in the child files included by |\include|.
Any output contained in the main file will appear in all child documents
unless suppressed manually;
it cannot be suppressed automatically by the |\includeonly| directive
and thus should normally be avoided.
A method to include some content in the main file
by means of conditional processing is described in \secref{sec:conditional}.

%%%%%%%%%%%%%%%%%%%%%%%%%%%%%%%%%%%%%%%%
\paragraph{Page Numbering.}

When only a part of the document is compiled,
the appropriate numbering of pages
(as well as other status parameters)
is determined from the |.aux| files.
The latter contain information from previous passes.
However this information needs to propagate through
all intermediate child documents.
Therefore the page numbering in child documents may well
be inconsistent until the complete document is compiled at least once.

A useful (if unconventional) way to always ensure a consistent
page numbering is to restart the numbering in each child document
and denote the pages by `\textit{child}|.|\textit{page}'
where \textit{child} represents the chapter/section number of the child file.
This can be achieved by the command
|\numberwithin{page}{|\textit{child}|}|
of the \textsf{amsmath} package
where \textit{child} can be |chapter| or |section|
depending on the chosen structuring.
Alternatively, one can modify the macro |\thepage| appropriately
and reset the counter |page| at the start of each child file.

%%%%%%%%%%%%%%%%%%%%%%%%%%%%%%%%%%%%%%%%%%%%%%%%%%%%%%%%%%%%%%%%%%%%%%%%%%%%%%%%
\subsection{Conditional Processing}
\label{sec:conditional}

The package provides a mechanism to compile different versions
of a document. To customise the versions further some conditional processing
can come in handy to distinguish which version is being compiled.
The package provides two macros to describe the compilation context:

%%%%%%%%%%%%%%%%%%%%%%%%%%%%%%%%%%%%%%%%
\DescribeMacro{\ifchilddoc}
The conditional |\ifchilddoc| distinguishes between the compilation of
child documents and the main document:
%
\begin{center}
|\ifchilddoc |\textit{child-code}| |[|\||else |\textit{main-code}]| \||fi|
\end{center}

%%%%%%%%%%%%%%%%%%%%%%%%%%%%%%%%%%%%%%%%
\DescribeMacro{\childdocname}
\DescribeMacro{\childdocjob}
The macro |\childdocname| contains the filename (without extension)
of the main or child file being processed.
Note that |\childdocjob| will always contain the name of the main file.

%%%%%%%%%%%%%%%%%%%%%%%%%%%%%%%%%%%%%%%%
\paragraph{Title Page.}

Conditional processing can be used to include a title or banner page
in the main document when proper precautions are taken.
Importantly, the code in the main file should ensure that the page counter
(as well as other status parameters which are stored in the |.aux| files)
takes the same value after the conditional processing.
Otherwise the page numbers may take divergent values
depending on which part is compiled.

For example, a title page could be declared by:
%
\begin{center}
\begin{tabular}{l}
|\ifchilddoc\||else|\\
|\addtocounter{page}{-1}|\\
\textit{code for title page}\\
|\newpage|\\
|\||fi|
\end{tabular}
\end{center}
%
A banner page for the child documents can be generated by:
%
\begin{center}
\begin{tabular}{l}
|\ifchilddoc|\\
|\addtocounter{page}{-1}|\\
\textit{code for banner page}\\
|\newpage|\\
|\||fi|
\end{tabular}
\end{center}
%
Here one could write a message such as:
\begin{center}
|This is the part \childdocname{} of \childdocjob{}.|
\end{center}

%%%%%%%%%%%%%%%%%%%%%%%%%%%%%%%%%%%%%%%%%%%%%%%%%%%%%%%%%%%%%%%%%%%%%%%%%%%%%%%%
\subsection{Flags}
\label{sec:flags}

The package makes it easy to generate different versions
of the main or child documents.
To this end compilation flags can be defined
and assigned different default values.
They will be particularly useful in conjunction
with the forwarding mechanism described in \secref{sec:forward}.

For example, it may be useful to have a flag |\version|
which can be set to |draft| or |final|.
The document source will contain some conditional code
depending on the value of |\version|.
Suppose further, the flag should default to |final| for the main file
and to |draft| for child files
which is a natural assignment for editing the document.
This is achieved by placing the following code
in the preamble of the main document
(below the |\childdocmain| directive):
%
\begin{center}
\begin{tabular}{l}
|\ifchilddoc|\\
|\providecommand{\version}{draft}|\\
|\||else|\\
|\providecommand{\version}{final}|\\
|\||fi|
\end{tabular}
\end{center}
%
The definition by |\providecommand| makes sure
that previous definitions are not overwritten.
Further statements |\providecommand{\version}{...}|
can thus be added before the above code to override it.

For the main file, one might add a line
(between |\childdocmain| and the above block)
%
\begin{center}
|%\ifchilddoc\||else\providecommand{\version}{draft}\||fi|
\end{center}
%
which can be uncommented to produce a draft version.
Likewise one can add a line to the very top of a child file
(above the |\childdocof{|\textit{main}|}| directive)
%
\begin{center}
|%\providecommand{\version}{final}|
\end{center}
%
which can be uncommented to produce the final version of this child document.

%%%%%%%%%%%%%%%%%%%%%%%%%%%%%%%%%%%%%%%%%%%%%%%%%%%%%%%%%%%%%%%%%%%%%%%%%%%%%%%%
\subsection{Forwarding}
\label{sec:forward}

Different versions of the main or child documents
using compilation flags as described in \secref{sec:flags}
can be (permanently) stored in different files
for convenient compilation, viewing and distribution.
To this end, the package defines a command
to pass on compilation to a different file:

%%%%%%%%%%%%%%%%%%%%%%%%%%%%%%%%%%%%%%%%
\DescribeMacro{\childdocforward}
The command |\childdocforward| redirects processing to
another source file:
%
\begin{center}
\begin{tabular}{l}
|\input{childdoc.def}|\\
|\childdocforward[|\textit{main}|]{|\textit{dest}|}|\\
\end{tabular}
\end{center}
%
The argument \textit{dest} is the destination file
(without extension).
It should be the main file or one of the child files.
Note that further \textsf{childdoc} directives
such as |\childdocof| and |\childdocforward|
in the indicated file will be processed in this form.
The optional argument \textit{main}
passes on directly to the main file \textit{main}
while pretending to compile the child \textit{dest}.
This form behaves as if \textit{dest}
issues |\childdocof{|\textit{main}|}| right away,
and no further \textsf{childdoc} directives will be processed.

%%%%%%%%%%%%%%%%%%%%%%%%%%%%%%%%%%%%%%%%
\DescribeMacro{\...prefix}
In the alternative form |\childdocforwardprefix|,
%
\begin{center}
\begin{tabular}{l}
|\input{childdoc.def}|\\
|\childdocforwardprefix[|\textit{main}|]{|\textit{prefix}|}{|\textit{dest}|}|
\end{tabular}
\end{center}
%
the destination file is determined by a pattern
depending on the current file:
To make this work, the current file must be called
`{\textit{prefix}\hspace{0.2em}\textit{suffix}}'
with \textit{prefix} matching precisely the argument.
Processing is then passed on to the file
`{\textit{dest}\hspace{0.2em}\textit{suffix}}'.
Surely, the same effect is achieved by
directly specifying the
argument `{\textit{dest}\hspace{0.2em}\textit{suffix}}'
in the first form.
However, that requires to set up a different file
for each child. With the alternative form of the command
all these files can have exactly the same content
which simplifies setting them up and maintaining them.

For example, the following file |draft.tex|
with a compilation flag |\version| as described in \secref{sec:flags}
compiles the main document as a draft:
%
\begin{center}
\begin{tabular}{l}
|\def\version{draft}|\\
|\input{childdoc.def}|\\
|\childdocforward{|\textit{main}|}|
\end{tabular}
\end{center}
%
Likewise, the following files |final|\textit{nn}|.tex|
compile the final version of the child document
|child|\textit{nn}|.tex|:
%
\begin{center}
\begin{tabular}{l}
|\def\version{final}|\\
|\input{childdoc.def}|\\
|\childdocforwardprefix{final}{child}|
\end{tabular}
\end{center}
%

Note that when several versions of a main file and/or of each child file
are to be generated, it may be convenient to set up a |Makefile| or
shell script to automatise the process.

%%%%%%%%%%%%%%%%%%%%%%%%%%%%%%%%%%%%%%%%%%%%%%%%%%%%%%%%%%%%%%%%%%%%%%%%%%%%%%%%
\subsection{Command Line Processing}
\label{sec:commandline}

The effect of redirection files can also be achieved by invoking
the \LaTeX{} compiler with a more elaborate command line.
Most conveniently this should be done as part
of a shell script or a |Makefile|.

When using \textsf{childdoc} in the main file, the following
command lines effectively perform a redirection
(note that depending on the shell being used,
backslashes may have to be doubled: `|\|' $\to$ `|\\|'):
%
\begin{center}
|... -jobname "|\textit{target}|" |\\|"|[\textit{flags}]%
|\input{childdoc.def}\childdocforward[|\textit{main}|]{|\textit{dest}|}"|
\end{center}
%
Here \textit{target} is the name of the output file,
\textit{main} is the name of the main file
and \textit{dest} is the name of the main or child file to be processed
(all filenames without extensions).
The optional argument \textit{main} can be omitted
if \textit{main} matches \textit{dest}.
Optionally, compilation \textit{flags} can be defined via |\def| commands.
This command line makes the \TeX{} engine believe
it is compiling the file \textit{target}
whose content is specified as the latter parameter.
The provided code then forwards the processing to
\textit{main} or \textit{dest} as described in \secref{sec:forward}.

%%%%%%%%%%%%%%%%%%%%%%%%%%%%%%%%%%%%%%%%%%%%%%%%%%%%%%%%%%%%%%%%%%%%%%%%%%%%%%%%
\subsection{Include by Input}
\label{sec:input}

Including child documents by |\include| has some restrictions by design.
Most notably, the content of a child document always occupies
its own set of pages; pages cannot be shared between child documents.
Usually, this behaviour makes perfect sense
because each child document contain an essential part of the document.
However, in some situations it may be desirable to compose
a document from a collection of parts
without having mandatory page breaks between then.
For this case, the package
provides a mechanism to include parts
by |\input| which can also be processed individually.
However, by construction this mechanism
requires manual handling of the content to be output.

%%%%%%%%%%%%%%%%%%%%%%%%%%%%%%%%%%%%%%%%
\DescribeMacro{\ifchilddocmanual}
The main file should be prepared as usual, see \secref{sec:include}.
However, the document body must make a distinction
between processing of an individual part and of the main document, e.g.:
%
\begin{center}
\begin{tabular}{l}
|\ifchilddocmanual|\\
|\input{\childdocname}|\\
|\||else|\\
\textit{document body with }|\input{|\textit{part}|}|\\
|\||fi|
\end{tabular}
\end{center}
%
The conditional |\ifchilddocmanual| is true whenever
a part to be included by |\input| is being compiled,
and the name of the part is stored in |\childdocname|.

%%%%%%%%%%%%%%%%%%%%%%%%%%%%%%%%%%%%%%%%
\DescribeMacro{\childdocby}
Each part to be included by |\input| should start with:
%
\begin{center}
\begin{tabular}{l}
|\input{childdoc.def}|\\
|\childdocby{|\textit{main}|}|\\
\end{tabular}
\end{center}
%
The directive |\childdocby| is similar to |\childdocof|
described in \secref{sec:include},
but the subsequent selection of content must be done manually.
To that end, both |\ifchilddoc| and |\ifchilddocmanual|
will be true upon processing of a part,
and the name of the part is stored in |\childdocname|.
Note that |\jobname| will be set to the filename of the current part
so that each part receives an individual |.aux| file
that does not interfere with the |.aux| file(s) of the main document.
This behaviour can be altered by the alternative form
|\childdocby[*]{|\textit{main}|}| (with a non-empty optional argument)
which uses the |.aux| file of the main document
by setting |\jobname| to \textit{main}.

%%%%%%%%%%%%%%%%%%%%%%%%%%%%%%%%%%%%%%%%%%%%%%%%%%%%%%%%%%%%%%%%%%%%%%%%%%%%%%%%
\subsection{Driver Development}
\label{sec:driver}

The \textsf{childdoc} mechanism can also be use for the development
of definition files such as \LaTeX{} styles or classes.
This case differs from the above setup with multiple parts
included by |\include| in that no |\includeonly| should be invoked.
This can be achieved by starting the include file
(before |\ProvidesPackage|) with:
%
\begin{center}
\begin{tabular}{l}
|\input{childdoc.def}|\\
|\childdocforward{|\textit{main}|}|\\
\end{tabular}
\end{center}
%
or alternatively with:
%
\begin{center}
\begin{tabular}{l}
|\input{childdoc.def}|\\
|\childdocby{|\textit{main}|}|\\
\end{tabular}
\end{center}
%
Both forms have slightly different effects as described above.
The main file is prepared as usual, see \secref{sec:include}.

%%%%%%%%%%%%%%%%%%%%%%%%%%%%%%%%%%%%%%%%%%%%%%%%%%%%%%%%%%%%%%%%%%%%%%%%%%%%%%%%
\subsection{Legacy Detection}
\label{sec:detection}

The directive |\childdocmain| in the main file can detect
whether the complete document or merely a child is to be compiled
even without using the directive |\childdocof|.
This method is deprecated because it is less robust
and there is no compelling reason to use it;
it is merely provided for backward compatibility
and it may be removed in future versions.

If the detection mechanism is to be used,
it is mandatory to correctly specify
the filename of the main file as the argument of |\childdocmain|:
%
\begin{center}
\begin{tabular}{l}
|\input{childdoc.def}|\\
|\childdocmain{|\textit{main}|}|\\
\end{tabular}
\end{center}
%
If |\jobname| does not match the argument \textit{main} of |\childdocmain|,
it is assumed that |\jobname| points to the child file to be compiled.
When using |\childdocmain| with the main file specified as argument,
it suffices to start a child file
with just |\input{|\textit{main}|}|
without loading of the package and using |\childdocof|.
If instead all processing is done
with the appropriate \textsf{childdoc} directives,
the argument of \textit{main} of |\childdocmain| can be empty.

An alternative version of the command line processing described
in \secref{sec:commandline} using the detection mechanism reads:
%
\begin{center}
|... -jobname "|\textit{target}|" "|[\textit{flags}]%
[|\def\jobname{|\textit{dest}|}|]|\input{|\textit{main}|}"|
\end{center}

%%%%%%%%%%%%%%%%%%%%%%%%%%%%%%%%%%%%%%%%%%%%%%%%%%%%%%%%%%%%%%%%%%%%%%%%%%%%%%%%
\subsection{Manual Code}
\label{sec:manual}

In case one cannot be certain whether the definitions file |childdoc.def|
is installed on the target \TeX{} distribution
and one prefers not to ship it,
it is conceivable to paste a few relevant commands into the sources.

To that end, drop all statements |\input{childdoc.def}|
and perform the replacements as outlined below.
Instead of |\childdocmain{|\textit{main}|}| add the following code
to the top of the main file:
%
\begin{center}
\begin{tabular}{l}
|\||ifdefined\childdocname\endinput\||fi\newif\ifchilddoc|\\
|\edef\childdocname{\scantokens\expandafter{\jobname\noexpand}}|\\
|\def\childdocmain{|\textit{main}|}\||ifx\childdocmain\childdocname\||else|\\
|\childdoctrue\includeonly{\childdocname}\let\jobname\childdocmain\||fi|\\
\end{tabular}
\end{center}
%
Instead of |\childdocof{|\textit{main}|}| just include the main file
at the top of each child file:
%
\begin{center}
|\input{|\textit{main}|}|
\end{center}
%
A simple redirection |\childdocforward{|\textit{dest}|}| is achieved by:
%
\begin{center}
|\def\jobname{|\textit{dest}|}\input{\jobname}|
\end{center}
%
The redirection with prefix
|\childdocforwardprefix[|\textit{prefix}|]{|\textit{dest}|}|
is accomplished by:
%
\begin{center}
\begin{tabular}{l}
|{\edef\jobname{\scantokens\expandafter{\jobname\noexpand}}|\\
|\def\redirectjob |\textit{prefix}|#1~~~{\gdef\jobname{|\textit{dest}|#1}}|\\
|\expandafter\redirectjob\jobname~~~}\input{\jobname}|
\end{tabular}
\end{center}

In an alternative approach,
child documents can be compiled by a specific command line
without additional code or specific definitions:
%
\begin{center}
|... -jobname "|\textit{target}|" "|[\textit{flags}]%
|\includeonly{|\textit{dest}|}\input{|\textit{main}|}"|
\end{center}
%

%%%%%%%%%%%%%%%%%%%%%%%%%%%%%%%%%%%%%%%%%%%%%%%%%%%%%%%%%%%%%%%%%%%%%%%%%%%%%%%%
%%%%%%%%%%%%%%%%%%%%%%%%%%%%%%%%%%%%%%%%%%%%%%%%%%%%%%%%%%%%%%%%%%%%%%%%%%%%%%%%
\section{Information}

%%%%%%%%%%%%%%%%%%%%%%%%%%%%%%%%%%%%%%%%%%%%%%%%%%%%%%%%%%%%%%%%%%%%%%%%%%%%%%%%
\subsection{Copyright}

Copyright \copyright{} 2017--2018 Niklas Beisert

This work may be distributed and/or modified under the
conditions of the \LaTeX{} Project Public License, either version 1.3
of this license or (at your option) any later version.
The latest version of this license is in
  \url{http://www.latex-project.org/lppl.txt}
and version 1.3 or later is part of all distributions of \LaTeX{}
version 2005/12/01 or later.

This work has the LPPL maintenance status `maintained'.

The Current Maintainer of this work is Niklas Beisert.

This work consists of the files |README.txt|, |childdoc.ins| and |childdoc.dtx|
as well as the derived files |childdoc.def|, |cdocsamp.tex|
with |cdocsch1.tex|, |cdocsch2.tex|, |cdocspt3.tex|, |cdocspt4.tex|,
|cdocsdrf.tex|, |cdocsfn1.tex|, |cdocsfn2.tex|
as well as |childdoc.pdf|.

%%%%%%%%%%%%%%%%%%%%%%%%%%%%%%%%%%%%%%%%%%%%%%%%%%%%%%%%%%%%%%%%%%%%%%%%%%%%%%%%
\subsection{Files and Installation}

The package consists of the files:
%
\begin{center}
\begin{tabular}{ll}
    |README.txt|   & readme file \\
    |childdoc.ins| & installation file \\
    |childdoc.dtx| & source file \\
    |childdoc.def| & definition file \\
    |cdocsamp.tex| & sample main file \\
    |cdocsch1.tex| & sample include file \\
    |cdocsch2.tex| & sample include file \\
    |cdocspt3.tex| & sample part file \\
    |cdocspt4.tex| & sample part file \\
    |cdocsdrf.tex| & sample redirection file \\
    |cdocsfn1.tex| & sample redirection file \\
    |cdocsfn2.tex| & sample redirection file \\
    |childdoc.pdf| & manual
\end{tabular}
\end{center}
%
The distribution consists of the files
|README.txt|, |childdoc.ins| and |childdoc.dtx|.
%
\begin{itemize}
\item
Run (pdf)\LaTeX{} on |childdoc.dtx|
to compile the manual |childdoc.pdf| (this file).
\item
Run \LaTeX{} on |childdoc.ins| to create the definitions file |childdoc.def|
and the sample |cdocsamp.tex| with include files
|cdocsch1.tex|, |cdocsch2.tex|, |cdocspt3.tex|, |cdocspt4.tex|,
|cdocsdrf.tex|, |cdocsfn1.tex|, |cdocsfn2.tex|.
Then copy the file |childdoc.def| to an appropriate directory of your \LaTeX{}
distribution, e.g.\ \textit{texmf-root}|/tex/latex/childdoc|.
\end{itemize}

%%%%%%%%%%%%%%%%%%%%%%%%%%%%%%%%%%%%%%%%%%%%%%%%%%%%%%%%%%%%%%%%%%%%%%%%%%%%%%%%
\subsection{Related CTAN Packages}

There are several other packages which offer a similar functionality:
%
\begin{itemize}
\item
The packages
\href{http://ctan.org/pkg/docmute}{\textsf{docmute}},
\href{http://ctan.org/pkg/includex}{\textsf{includex}} and
\href{http://ctan.org/pkg/standalone}{\textsf{standalone}}
provide commands to include only the document body of
a child file thus allowing both files to be compiled individually.
\item
The packages \href{http://ctan.org/pkg/subdocs}{\textsf{subdocs}}
and \href{http://ctan.org/pkg/subfiles}{\textsf{subfiles}}
provide structures in which the main and child documents can be
encapsulated and allowing them to be compiled individually.
The inclusion mechanism is different from the conventional |\include|.
\item
The package \href{http://ctan.org/pkg/combine}{\textsf{combine}}
is an elaborate solution to combine several documents into one.
\end{itemize}
%
See also the CTAN topic \href{http://ctan.org/topic/subdocs}{\textsf{subdocs}}
for further related packages.
The present package differs from the above solutions in that
a document structure constructed with the conventional |\include| mechanism
just needs two extra commands at the top of every file
such that all constituent files can be compiled individually.

%%%%%%%%%%%%%%%%%%%%%%%%%%%%%%%%%%%%%%%%%%%%%%%%%%%%%%%%%%%%%%%%%%%%%%%%%%%%%%%%
%\subsection{Feature Suggestions}
%
%The following is a list of features which may be useful for future
%versions of this package:
%%
%\begin{itemize}
%\item
%\ldots
%\end{itemize}

%%%%%%%%%%%%%%%%%%%%%%%%%%%%%%%%%%%%%%%%%%%%%%%%%%%%%%%%%%%%%%%%%%%%%%%%%%%%%%%%
\subsection{Revision History}

%%%%%%%%%%%%%%%%%%%%%%%%%%%%%%%%%%%%%%%%
\paragraph{v2.0:} 2018/12/30

\begin{itemize}
\item
immediate forward processing
\item
added |\childdocby| mechanism
\item
manual restructured
\end{itemize}

%%%%%%%%%%%%%%%%%%%%%%%%%%%%%%%%%%%%%%%%
\paragraph{v1.6:} 2018/01/17

\begin{itemize}
\item
application for development of include files
\item
corrections to manual
\end{itemize}

%%%%%%%%%%%%%%%%%%%%%%%%%%%%%%%%%%%%%%%%
\paragraph{v1.5:} 2017/05/21

\begin{itemize}
\item
more complete structuring introduced
\item
|\childdocof| introduced
\item
|\childdoc| renamed to |\childdocmain|
\item
|\childredirect| renamed to |\childdocforward| and |\childdocforwardprefix|
and functionality expanded
\end{itemize}

%%%%%%%%%%%%%%%%%%%%%%%%%%%%%%%%%%%%%%%%
\paragraph{v1.0:} 2017/04/27

\begin{itemize}
\item
manual and install package
\item
first version published on CTAN
\end{itemize}

%%%%%%%%%%%%%%%%%%%%%%%%%%%%%%%%%%%%%%%%
\paragraph{v0.6:} 2017/04/26

\begin{itemize}
\item
redirection mechanism added
\end{itemize}

%%%%%%%%%%%%%%%%%%%%%%%%%%%%%%%%%%%%%%%%
\paragraph{v0.5:} 2017/04/26

\begin{itemize}
\item
functionality in definition file
\end{itemize}


%%%%%%%%%%%%%%%%%%%%%%%%%%%%%%%%%%%%%%%%%%%%%%%%%%%%%%%%%%%%%%%%%%%%%%%%%%%%%%%%
%%%%%%%%%%%%%%%%%%%%%%%%%%%%%%%%%%%%%%%%%%%%%%%%%%%%%%%%%%%%%%%%%%%%%%%%%%%%%%%%
%%%%%%%%%%%%%%%%%%%%%%%%%%%%%%%%%%%%%%%%%%%%%%%%%%%%%%%%%%%%%%%%%%%%%%%%%%%%%%%%
\appendix

\settowidth\MacroIndent{\rmfamily\scriptsize 000\ }

 \DocInput{childdoc.dtx}

\end{document}
%</driver>
% \fi
%
% %%%%%%%%%%%%%%%%%%%%%%%%%%%%%%%%%%%%%%%%%%%%%%%%%%%%%%%%%%%%%%%%%%%%%%%%%%%%%%
% %%%%%%%%%%%%%%%%%%%%%%%%%%%%%%%%%%%%%%%%%%%%%%%%%%%%%%%%%%%%%%%%%%%%%%%%%%%%%%
% \section{Sample}
%\iffalse
%<*samplemain>
%\fi
%
% The following presents a sample document
% with two chapters, two parts, a title page,
% a compile flag as well as three forwarding files to set the flag.
% It consists of eight |.tex| files:
% \begin{center}
% \begin{tabular}{ll}
% |cdocsamp.tex|&main file\\
% |cdocsch1.tex|&include file for chapter 1\\
% |cdocsch2.tex|&include file for chapter 2\\
% |cdocspt3.tex|&include file for part 3\\
% |cdocspt4.tex|&include file for part 4\\
% |cdocsdrf.tex|&forwarding file for main file in draft mode\\
% |cdocsfi1.tex|&forwarding file for final version of chapter 1\\
% |cdocsfi2.tex|&forwarding file for final version of chapter 2\\
% \end{tabular}
% \end{center}
% Each of the eight files can be compiled directly by the \LaTeX{} compiler.
%
% %%%%%%%%%%%%%%%%%%%%%%%%%%%%%%%%%%%%%%
% \paragraph{Main File.}
%
% The main file is called |cdocsamp.tex|.
%
% Load the \textsf{childdoc} definitions and
% declare the filename for the main document:
%    \begin{macrocode}
\input{childdoc.def}
\childdocmain{}
%    \end{macrocode}

% Optional override for |\version| flag:
%    \begin{macrocode}
%%\ifchilddoc\else\providecommand{\version}{draft}\fi
%    \end{macrocode}

% Define the default values for the |\version| flag
% (|final| for the main file and |draft| for childs):
%    \begin{macrocode}
\ifchilddoc
\providecommand{\version}{draft}
\else
\providecommand{\version}{final}
\fi
%    \end{macrocode}

% Load the standard document class:
%    \begin{macrocode}
\documentclass[12pt]{article}
%    \end{macrocode}

% Start the document body:
%    \begin{macrocode}
\begin{document}
%    \end{macrocode}

% Declare a title page.
% Print title, part of document being processed and version flag:
%    \begin{macrocode}
\addtocounter{page}{-1}
\begin{center}
{\LARGE\bfseries{}childdoc example\par}
\vspace{1cm}
\ifchilddoc
\ifchilddocmanual part\else chapter\fi:
`\childdocname' of `\childdocjob'\par
\else
main document: `\childdocjob'\par
\fi
version: \version\par
\end{center}
\newpage
%    \end{macrocode}

% Manually include selected file,
% otherwise process as usual:
%    \begin{macrocode}
\ifchilddocmanual
\section*{part `\childdocname'}
\input{\childdocname}
\else
%    \end{macrocode}

% Include the two chapters:
%    \begin{macrocode}
\include{cdocsch1}
\include{cdocsch2}
%    \end{macrocode}

% Include the two parts unless only chapters should be displayed:
%    \begin{macrocode}
\ifchilddoc\else
\section{part three}
\input{cdocspt3}
\section{part four}
\input{cdocspt4}
\fi
%    \end{macrocode}

% Process as usual until here:
%    \begin{macrocode}
\fi
%    \end{macrocode}

% End of document body:
%    \begin{macrocode}
\end{document}
%    \end{macrocode}
%\iffalse
%</samplemain>
%\fi
%
% %%%%%%%%%%%%%%%%%%%%%%%%%%%%%%%%%%%%%%
% \paragraph{Chapter Include Files.}
%
% The include files are called |cdocsch1.tex| and |cdocsch2.tex|.
%
%\iffalse
%<*samplechap1|samplechap2>
%\fi

% Optional override for |\version| flag:
%    \begin{macrocode}
%%\providecommand{\version}{final}
%    \end{macrocode}

% Include the main document:
%    \begin{macrocode}
\input{childdoc.def}
\childdocof{cdocsamp}
%    \end{macrocode}

%\iffalse
%</samplechap1|samplechap2>
%\fi
%
%\iffalse
%<*samplechap1>
%\fi
% Some text for chapter 1:
%    \begin{macrocode}
\section{one}
some text in chapter one
%    \end{macrocode}

%\iffalse
%</samplechap1>
%\fi
% Some text for chapter 2:
%\iffalse
%<*samplechap2>
%\fi
%    \begin{macrocode}
\section{two}
more text in chapter two
%    \end{macrocode}

%\iffalse
%</samplechap2>
%\fi
%
% %%%%%%%%%%%%%%%%%%%%%%%%%%%%%%%%%%%%%%
% \paragraph{Part Include Files.}
%
% The include files are called |cdocspt3.tex| and |cdocspt4.tex|.
%
%\iffalse
%<*samplepart3|samplepart4>
%\fi

% Optional override for |\version| flag:
%    \begin{macrocode}
%%\providecommand{\version}{final}
%    \end{macrocode}

% Include the main document:
%    \begin{macrocode}
\input{childdoc.def}
\childdocby{cdocsamp}
%    \end{macrocode}

%\iffalse
%</samplepart3|samplepart4>
%\fi
%
%\iffalse
%<*samplepart3>
%\fi
% Some text for part 3:
%    \begin{macrocode}
some text in part three
%    \end{macrocode}

%\iffalse
%</samplepart3>
%\fi
% Some text for part 4:
%\iffalse
%<*samplepart4>
%\fi
%    \begin{macrocode}
more text in part four
%    \end{macrocode}

%\iffalse
%</samplepart4>
%\fi
%
% %%%%%%%%%%%%%%%%%%%%%%%%%%%%%%%%%%%%%%
% \paragraph{Forwarding for a Complete Draft.}
%
% The following forwarding file |cdocsdrf.tex|
% compiles the main document in draft mode:
%\iffalse
%<*sampledraft>
%\fi
%    \begin{macrocode}
\def\version{draft}
\input{childdoc.def}
\childdocforward{cdocsamp}
%    \end{macrocode}

%\iffalse
%</sampledraft>
%\fi
%
% %%%%%%%%%%%%%%%%%%%%%%%%%%%%%%%%%%%%%%
% \paragraph{Forwarding for Final Version of the Chapters.}
%
% The following forwarding files |cdocsfn1.tex| and |cdocsfn2.tex|
% (with identical content)
% compile the final versions of the child documents
% |cdocsch1.tex| and |cdocsch2.tex|, respectively:
%\iffalse
%<*samplefinal>
%\fi
%    \begin{macrocode}
\def\version{final}
\input{childdoc.def}
\childdocforwardprefix[cdocsamp]{cdocsfn}{cdocsch}
%    \end{macrocode}

%\iffalse
%</samplefinal>
%\fi
%
% %%%%%%%%%%%%%%%%%%%%%%%%%%%%%%%%%%%%%%
% \paragraph{Command Line Processing.}
%
% The following three command lines generate the output files
% |cdocscld|, |cdocscl1| and |cdocscl2|
% which should be identical to
% |cdocsdrf|, |cdocsch1| and |cdocsfn2|, respectively:
% \begin{center}
% \begin{tabular}{l}
% |latex -jobname cdocscld \|\\
% |  "\def\version{draft}\input{childdoc.def}\childdocforward{cdocsamp}"|\\
% |latex -jobname cdocscl1 \|\\
% |  "\input{childdoc.def}\childdocforward[cdocsamp]{cdocsch1}"|\\
% |latex -jobname cdocscl2 \|\\
% |  "\def\version{final}\input{childdoc.def}\childdocforward{cdocsch2}"|
% \end{tabular}
% \end{center}
% Note that the trailing backslash on each first line
% merely continues the input to the second line
% (for convenient cut ant paste).
% Furthermore, the command |latex| can be replaced by any
% of its alternative versions such as |pdflatex|.
%
% %%%%%%%%%%%%%%%%%%%%%%%%%%%%%%%%%%%%%%%%%%%%%%%%%%%%%%%%%%%%%%%%%%%%%%%%%%%%%%
% %%%%%%%%%%%%%%%%%%%%%%%%%%%%%%%%%%%%%%%%%%%%%%%%%%%%%%%%%%%%%%%%%%%%%%%%%%%%%%
% \section{Implementation}
%\iffalse
%<*package>
%\fi
%
% This section describes the definitions file |childdoc.def|.

% The definitions cannot be loaded using |\usepackage| or |\RequirePackage|
% which has a mechanism to prevent loading a style file more than once.
% When loading the definitions by means of |\input|
% multiple instances have to be prevented manually:
%\iffalse
%This code needs to be before the `\ProvidesFile' directive
%which is defined at the beginning of this file.
%Therefore it is also placed there and commented out here.
%</package>
%<*discard>
%\fi
%    \begin{macrocode}
\ifdefined\childdocmain\endinput\fi
%    \end{macrocode}
%\iffalse
%</discard>
%<*package>
%\fi
%
% \macro{\ifchilddoc}
% \macro{\ifchilddocmanual}
% The conditional |\ifchilddoc| tells whether a
% child (true) or main (false) document is being compiled.
% The conditional |\ifchilddocmanual| tells whether
% the |\includeonly| mechanism is used (false) or
% the selection of child files must be performed manually (true).
% The definitions initialise to false:
%    \begin{macrocode}
\newif\ifchilddoc
\newif\ifchilddocmanual
%    \end{macrocode}

% \macro{\childdocname}
% \macro{\childdocjob}
% The macro |\childdocname| stores the name of the main document
% to be compiled. The macro |\childdocjob| stores the name of
% the document on which the \LaTeX{} compiler was originally invoked.
% The content of |\jobname| cannot be compared
% to filenames specified in the source due to different catcodes.
% The following code rescans |\jobname|, stores the result
% in |\childdocname| and saves a copy in |\childdocjob|:
%    \begin{macrocode}
\edef\childdocname{\scantokens\expandafter{\jobname\noexpand}}
\let\childdocjob\childdocname
%    \end{macrocode}

% \macro{\childdocdisable}
% The macro |\childdocdisable| prevents the main file
% from being processed more than once.
% At this stage, the main document command |\childdocmain|
% is assumed to be called once again where it should do nothing.
% Any subsequent call to it should prevent
% a secondary processing of the main document
% It overwrites the forwarding commands
% |\childdocof| and |\childdocforward|
% with empty macros to prevent further inclusions of the main document:
%    \begin{macrocode}
\newcommand{\childdocdisable}
{
  \renewcommand{\childdocmain}[1]{\renewcommand{\childdocmain}[1]{\endinput}}
  \renewcommand{\childdocof}[1]{}
  \renewcommand{\childdocby}[2][]{}
  \renewcommand{\childdocforward}[2][]{}
  \renewcommand{\childdocdisable}{}
}
%    \end{macrocode}

% \macro{\childdocmain}
% The macro |\childdocmain| is to be called at the top of the main file
% with nothing or the main filename (without extension) as argument.
% First, it breaks loops.
% If the argument is not empty and does not match |\childdocname|
% (which is set by the first inclusion of |childdoc.def|),
% |\ifchilddoc| is set to true, |\includeonly| is applied to the child file
% and |\jobname| is set to the main file
% (for proper handling of |.aux| files):
%    \begin{macrocode}
\newcommand{\childdocmain}[1]
{
  \childdocdisable\childdocmain{}
  \if?#1?\else
    \begingroup
      \def\childdoctmp{#1}
      \ifx\childdoctmp\childdocname
        \def\childdoctmp{}
      \else
        \def\childdoctmp
        {
          \childdoctrue
          \includeonly{\childdocname}
          \def\childdocjob{#1}
          \def\jobname{#1}
        }
      \fi
      \expandafter
    \endgroup
    \childdoctmp
  \fi
}
%    \end{macrocode}

% \macro{\childdocof}
% The command |\childdocof| redirects
% compilation to the main file |#1|.
%    \begin{macrocode}
\newcommand{\childdocof}[1]
{
  \childdocdisable
  \childdoctrue
  \includeonly{\childdocname}
  \def\jobname{#1}
  \def\childdocjob{#1}
  \input{#1}
}
%    \end{macrocode}

% \macro{\childdocby}
% The command |\childdocby| ....
%    \begin{macrocode}
\newcommand{\childdocby}[2][]
{
  \childdocdisable
  \childdoctrue
  \childdocmanualtrue
  \if?#1?\else
    \def\jobname{#2}
  \fi
  \def\childdocjob{#2}
  \input{#2}
  \endinput
}
%    \end{macrocode}

% \macro{\childdocforward}
% The command |\childdocforward| redirects
% compilation to the main file or
% (if the optional argument is given) a child file.
% Parameters are set as if the main file
% or a child file starting with |\childdocof| was compiled.
% Then compilation is handed over to the main file:
%    \begin{macrocode}
\newcommand{\childdocforward}[2][]
{
  \begingroup
    \if?#1?
      \def\childdoctmp
      {
        \def\childdocname{#2}
        \def\childdocjob{#2}
        \def\jobname{#2}
        \input{#2}
        \endinput
      }
    \else
      \def\childdoctmp
      {
        \childdocdisable
        \def\childdocname{#2}
        \childdoctrue
        \includeonly{#2}
        \def\childdocjob{#1}
        \def\jobname{#1}
        \input{#1}
        \endinput
      }
    \fi
    \expandafter
  \endgroup
  \childdoctmp
}
%    \end{macrocode}

% \macro{\childdocforwardprefix}
% The command |\childdocforwardprefix| redirects
% compilation to the main or a child file by means of a pattern.
% The prefix |#1| in the current filename is replaced by |#2|
% and the suffix of the current filename is kept
% (it is assumed that the filename does not contain the substring `|~~~|'
% which is used as a delimiter).
% Compilation is handed over to the new file by |\childdocforward|:
%    \begin{macrocode}
\newcommand{\childdocforwardprefix}[3][]
{
  \begingroup
    \def\childdocextract #2##1~~~{\def\childdoctmp{\childdocforward[#1]{#3##1}}}
    \expandafter\childdocextract\childdocname~~~
    \expandafter
  \endgroup
  \childdoctmp
}
%    \end{macrocode}

% \macro{\childdoc}
% The deprecated macro |\childdoc| is a legacy version of |\childdocmain|:
%    \begin{macrocode}
\newcommand{\childdoc}{\childdocmain}
%    \end{macrocode}

% \macro{\childdocredirect}
% The deprecated macro |\childdocredirect| is a legacy version
% of |\childdocforward| and |\childdocforwardprefix|:
%    \begin{macrocode}
\newcommand{\childdocredirect}[2][]
{
  \begingroup
    \if?#1?
      \def\childdoctmp{\childdocforward{#2}}
    \else
      \def\childdoctmp{\childdocforwardprefix{#1}{#2}}
    \fi
    \expandafter
  \endgroup
  \childdoctmp
}
%    \end{macrocode}

%\iffalse
%</package>
%\fi
%
\endinput
|
and perform the replacements as outlined below.
Instead of |\childdocmain{|\textit{main}|}| add the following code
to the top of the main file:
%
\begin{center}
\begin{tabular}{l}
|\||ifdefined\childdocname\endinput\||fi\newif\ifchilddoc|\\
|\edef\childdocname{\scantokens\expandafter{\jobname\noexpand}}|\\
|\def\childdocmain{|\textit{main}|}\||ifx\childdocmain\childdocname\||else|\\
|\childdoctrue\includeonly{\childdocname}\let\jobname\childdocmain\||fi|\\
\end{tabular}
\end{center}
%
Instead of |\childdocof{|\textit{main}|}| just include the main file
at the top of each child file:
%
\begin{center}
|\input{|\textit{main}|}|
\end{center}
%
A simple redirection |\childdocforward{|\textit{dest}|}| is achieved by:
%
\begin{center}
|\def\jobname{|\textit{dest}|}\input{\jobname}|
\end{center}
%
The redirection with prefix
|\childdocforwardprefix[|\textit{prefix}|]{|\textit{dest}|}|
is accomplished by:
%
\begin{center}
\begin{tabular}{l}
|{\edef\jobname{\scantokens\expandafter{\jobname\noexpand}}|\\
|\def\redirectjob |\textit{prefix}|#1~~~{\gdef\jobname{|\textit{dest}|#1}}|\\
|\expandafter\redirectjob\jobname~~~}\input{\jobname}|
\end{tabular}
\end{center}

In an alternative approach,
child documents can be compiled by a specific command line
without additional code or specific definitions:
%
\begin{center}
|... -jobname "|\textit{target}|" "|[\textit{flags}]%
|\includeonly{|\textit{dest}|}\input{|\textit{main}|}"|
\end{center}
%

%%%%%%%%%%%%%%%%%%%%%%%%%%%%%%%%%%%%%%%%%%%%%%%%%%%%%%%%%%%%%%%%%%%%%%%%%%%%%%%%
%%%%%%%%%%%%%%%%%%%%%%%%%%%%%%%%%%%%%%%%%%%%%%%%%%%%%%%%%%%%%%%%%%%%%%%%%%%%%%%%
\section{Information}

%%%%%%%%%%%%%%%%%%%%%%%%%%%%%%%%%%%%%%%%%%%%%%%%%%%%%%%%%%%%%%%%%%%%%%%%%%%%%%%%
\subsection{Copyright}

Copyright \copyright{} 2017--2018 Niklas Beisert

This work may be distributed and/or modified under the
conditions of the \LaTeX{} Project Public License, either version 1.3
of this license or (at your option) any later version.
The latest version of this license is in
  \url{http://www.latex-project.org/lppl.txt}
and version 1.3 or later is part of all distributions of \LaTeX{}
version 2005/12/01 or later.

This work has the LPPL maintenance status `maintained'.

The Current Maintainer of this work is Niklas Beisert.

This work consists of the files |README.txt|, |childdoc.ins| and |childdoc.dtx|
as well as the derived files |childdoc.def|, |cdocsamp.tex|
with |cdocsch1.tex|, |cdocsch2.tex|, |cdocspt3.tex|, |cdocspt4.tex|,
|cdocsdrf.tex|, |cdocsfn1.tex|, |cdocsfn2.tex|
as well as |childdoc.pdf|.

%%%%%%%%%%%%%%%%%%%%%%%%%%%%%%%%%%%%%%%%%%%%%%%%%%%%%%%%%%%%%%%%%%%%%%%%%%%%%%%%
\subsection{Files and Installation}

The package consists of the files:
%
\begin{center}
\begin{tabular}{ll}
    |README.txt|   & readme file \\
    |childdoc.ins| & installation file \\
    |childdoc.dtx| & source file \\
    |childdoc.def| & definition file \\
    |cdocsamp.tex| & sample main file \\
    |cdocsch1.tex| & sample include file \\
    |cdocsch2.tex| & sample include file \\
    |cdocspt3.tex| & sample part file \\
    |cdocspt4.tex| & sample part file \\
    |cdocsdrf.tex| & sample redirection file \\
    |cdocsfn1.tex| & sample redirection file \\
    |cdocsfn2.tex| & sample redirection file \\
    |childdoc.pdf| & manual
\end{tabular}
\end{center}
%
The distribution consists of the files
|README.txt|, |childdoc.ins| and |childdoc.dtx|.
%
\begin{itemize}
\item
Run (pdf)\LaTeX{} on |childdoc.dtx|
to compile the manual |childdoc.pdf| (this file).
\item
Run \LaTeX{} on |childdoc.ins| to create the definitions file |childdoc.def|
and the sample |cdocsamp.tex| with include files
|cdocsch1.tex|, |cdocsch2.tex|, |cdocspt3.tex|, |cdocspt4.tex|,
|cdocsdrf.tex|, |cdocsfn1.tex|, |cdocsfn2.tex|.
Then copy the file |childdoc.def| to an appropriate directory of your \LaTeX{}
distribution, e.g.\ \textit{texmf-root}|/tex/latex/childdoc|.
\end{itemize}

%%%%%%%%%%%%%%%%%%%%%%%%%%%%%%%%%%%%%%%%%%%%%%%%%%%%%%%%%%%%%%%%%%%%%%%%%%%%%%%%
\subsection{Related CTAN Packages}

There are several other packages which offer a similar functionality:
%
\begin{itemize}
\item
The packages
\href{http://ctan.org/pkg/docmute}{\textsf{docmute}},
\href{http://ctan.org/pkg/includex}{\textsf{includex}} and
\href{http://ctan.org/pkg/standalone}{\textsf{standalone}}
provide commands to include only the document body of
a child file thus allowing both files to be compiled individually.
\item
The packages \href{http://ctan.org/pkg/subdocs}{\textsf{subdocs}}
and \href{http://ctan.org/pkg/subfiles}{\textsf{subfiles}}
provide structures in which the main and child documents can be
encapsulated and allowing them to be compiled individually.
The inclusion mechanism is different from the conventional |\include|.
\item
The package \href{http://ctan.org/pkg/combine}{\textsf{combine}}
is an elaborate solution to combine several documents into one.
\end{itemize}
%
See also the CTAN topic \href{http://ctan.org/topic/subdocs}{\textsf{subdocs}}
for further related packages.
The present package differs from the above solutions in that
a document structure constructed with the conventional |\include| mechanism
just needs two extra commands at the top of every file
such that all constituent files can be compiled individually.

%%%%%%%%%%%%%%%%%%%%%%%%%%%%%%%%%%%%%%%%%%%%%%%%%%%%%%%%%%%%%%%%%%%%%%%%%%%%%%%%
%\subsection{Feature Suggestions}
%
%The following is a list of features which may be useful for future
%versions of this package:
%%
%\begin{itemize}
%\item
%\ldots
%\end{itemize}

%%%%%%%%%%%%%%%%%%%%%%%%%%%%%%%%%%%%%%%%%%%%%%%%%%%%%%%%%%%%%%%%%%%%%%%%%%%%%%%%
\subsection{Revision History}

%%%%%%%%%%%%%%%%%%%%%%%%%%%%%%%%%%%%%%%%
\paragraph{v2.0:} 2018/12/30

\begin{itemize}
\item
immediate forward processing
\item
added |\childdocby| mechanism
\item
manual restructured
\end{itemize}

%%%%%%%%%%%%%%%%%%%%%%%%%%%%%%%%%%%%%%%%
\paragraph{v1.6:} 2018/01/17

\begin{itemize}
\item
application for development of include files
\item
corrections to manual
\end{itemize}

%%%%%%%%%%%%%%%%%%%%%%%%%%%%%%%%%%%%%%%%
\paragraph{v1.5:} 2017/05/21

\begin{itemize}
\item
more complete structuring introduced
\item
|\childdocof| introduced
\item
|\childdoc| renamed to |\childdocmain|
\item
|\childredirect| renamed to |\childdocforward| and |\childdocforwardprefix|
and functionality expanded
\end{itemize}

%%%%%%%%%%%%%%%%%%%%%%%%%%%%%%%%%%%%%%%%
\paragraph{v1.0:} 2017/04/27

\begin{itemize}
\item
manual and install package
\item
first version published on CTAN
\end{itemize}

%%%%%%%%%%%%%%%%%%%%%%%%%%%%%%%%%%%%%%%%
\paragraph{v0.6:} 2017/04/26

\begin{itemize}
\item
redirection mechanism added
\end{itemize}

%%%%%%%%%%%%%%%%%%%%%%%%%%%%%%%%%%%%%%%%
\paragraph{v0.5:} 2017/04/26

\begin{itemize}
\item
functionality in definition file
\end{itemize}


%%%%%%%%%%%%%%%%%%%%%%%%%%%%%%%%%%%%%%%%%%%%%%%%%%%%%%%%%%%%%%%%%%%%%%%%%%%%%%%%
%%%%%%%%%%%%%%%%%%%%%%%%%%%%%%%%%%%%%%%%%%%%%%%%%%%%%%%%%%%%%%%%%%%%%%%%%%%%%%%%
%%%%%%%%%%%%%%%%%%%%%%%%%%%%%%%%%%%%%%%%%%%%%%%%%%%%%%%%%%%%%%%%%%%%%%%%%%%%%%%%
\appendix

\settowidth\MacroIndent{\rmfamily\scriptsize 000\ }

 \DocInput{childdoc.dtx}

\end{document}
%</driver>
% \fi
%
% %%%%%%%%%%%%%%%%%%%%%%%%%%%%%%%%%%%%%%%%%%%%%%%%%%%%%%%%%%%%%%%%%%%%%%%%%%%%%%
% %%%%%%%%%%%%%%%%%%%%%%%%%%%%%%%%%%%%%%%%%%%%%%%%%%%%%%%%%%%%%%%%%%%%%%%%%%%%%%
% \section{Sample}
%\iffalse
%<*samplemain>
%\fi
%
% The following presents a sample document
% with two chapters, two parts, a title page,
% a compile flag as well as three forwarding files to set the flag.
% It consists of eight |.tex| files:
% \begin{center}
% \begin{tabular}{ll}
% |cdocsamp.tex|&main file\\
% |cdocsch1.tex|&include file for chapter 1\\
% |cdocsch2.tex|&include file for chapter 2\\
% |cdocspt3.tex|&include file for part 3\\
% |cdocspt4.tex|&include file for part 4\\
% |cdocsdrf.tex|&forwarding file for main file in draft mode\\
% |cdocsfi1.tex|&forwarding file for final version of chapter 1\\
% |cdocsfi2.tex|&forwarding file for final version of chapter 2\\
% \end{tabular}
% \end{center}
% Each of the eight files can be compiled directly by the \LaTeX{} compiler.
%
% %%%%%%%%%%%%%%%%%%%%%%%%%%%%%%%%%%%%%%
% \paragraph{Main File.}
%
% The main file is called |cdocsamp.tex|.
%
% Load the \textsf{childdoc} definitions and
% declare the filename for the main document:
%    \begin{macrocode}
% \iffalse
%
% childdoc.dtx Copyright (C) 2017-2018 Niklas Beisert
%
% This work may be distributed and/or modified under the
% conditions of the LaTeX Project Public License, either version 1.3
% of this license or (at your option) any later version.
% The latest version of this license is in
%   http://www.latex-project.org/lppl.txt
% and version 1.3 or later is part of all distributions of LaTeX
% version 2005/12/01 or later.
%
% This work has the LPPL maintenance status `maintained'.
%
% The Current Maintainer of this work is Niklas Beisert.
%
% This work consists of the files childdoc.dtx and childdoc.ins
% and the derived files childdoc.def and cdocsamp.tex with
% cdocsch1.tex, cdocsch2.tex, cdocsdrf.tex, cdocsfn1.tex, cdocsfn2.tex.
%
%<package>\ifdefined\childdocmain\endinput\fi
%<package>\ProvidesFile{childdoc.def}[2018/12/30 v2.0 child document driver]
%<samplemain>\ProvidesFile{cdocsamp.tex}[2018/12/30 v2.0 sample for childdoc]
%<*driver>
%\ProvidesFile{childdoc.drv}[2018/12/30 v2.0 childdoc reference manual file]
\PassOptionsToClass{10pt,a4paper}{article}
\documentclass{ltxdoc}

\usepackage[margin=35mm]{geometry}
\usepackage{hyperref}
\usepackage{hyperxmp}
\usepackage[usenames]{color}

\hypersetup{colorlinks=true}
\hypersetup{pdfstartview=FitH}
\hypersetup{pdfpagemode=UseNone}
\hypersetup{pdfsource={}}
\hypersetup{pdflang={en-UK}}
\hypersetup{pdfcopyright={Copyright 2017-2018 Niklas Beisert.
  This work may be distributed and/or modified under the
  conditions of the LaTeX Project Public License, either version 1.3
  of this license or (at your option) any later version.}}
\hypersetup{pdflicenseurl={http://www.latex-project.org/lppl.txt}}
\hypersetup{pdfcontactaddress={ETH Zurich, ITP, HIT K,
  Wolfgang-Pauli-Strasse 27}}
\hypersetup{pdfcontactpostcode={8093}}
\hypersetup{pdfcontactcity={Zurich}}
\hypersetup{pdfcontactcountry={Switzerland}}
\hypersetup{pdfcontactemail={nbeisert@itp.phys.ethz.ch}}
\hypersetup{pdfcontacturl={http://people.phys.ethz.ch/\xmptilde nbeisert/}}

\newcommand{\secref}[1]{\hyperref[#1]{section \ref*{#1}}}

\parskip1ex
\parindent0pt
\let\olditemize\itemize
\def\itemize{\olditemize\parskip0pt}

\begin{document}

\title{The \textsf{childdoc} Package}
\hypersetup{pdftitle={The childdoc Package}}
\author{Niklas Beisert\\[2ex]
  Institut f\"ur Theoretische Physik\\
  Eidgen\"ossische Technische Hochschule Z\"urich\\
  Wolfgang-Pauli-Strasse 27, 8093 Z\"urich, Switzerland\\[1ex]
  \href{mailto:nbeisert@itp.phys.ethz.ch}
  {\texttt{nbeisert@itp.phys.ethz.ch}}}
\hypersetup{pdfauthor={Niklas Beisert}}
\hypersetup{pdfsubject={Manual for the LaTeX2e Package childdoc}}
\date{30 December 2018, \textsf{v2.0}}
\maketitle

\begin{abstract}\noindent
\textsf{childdoc} is a \LaTeXe{} package
that enables the direct compilation
of document sections included by |\include|
to individual files.
\end{abstract}

\begingroup
\parskip0ex
\tableofcontents
\endgroup

%%%%%%%%%%%%%%%%%%%%%%%%%%%%%%%%%%%%%%%%%%%%%%%%%%%%%%%%%%%%%%%%%%%%%%%%%%%%%%%%
%%%%%%%%%%%%%%%%%%%%%%%%%%%%%%%%%%%%%%%%%%%%%%%%%%%%%%%%%%%%%%%%%%%%%%%%%%%%%%%%
\section{Introduction}

\LaTeX{} provides a mechanism to structure a large document (such as a book)
into a main file and several child files (containing the chapters)
using the |\include| command.
This mechanism is beneficial for documents
which span hundreds of pages in order to
make the source file(s) more manageable.
Moreover, compilation can be restricted to
selected child files by means of the |\includeonly| command.
The latter feature can be used to reduce the compilation time while editing
(this was significantly more useful in the earlier days of \LaTeX{})
or to generate a smaller document which is easier to navigate.
Another application of |\includeonly| is to generate
documents consisting of selected parts of the complete document.

However, there are a few drawbacks of the plain |\include| mechanism:
\begin{itemize}
\item
The child files cannot be compiled on their own,
they can only be compiled via the main file.
A naive editing environment
(such as a text editor with an option
to have the current file processed by \LaTeX)
may require one to switch to the main file before compiling;
attempting to compile the child file produces errors.
\item
The main file must be modified (each time)
to adjust the |\includeonly| command
to the present needs. This easily leaves the main file in a messy state.
\item
The generated document will always carry the filename
of the main document. This is inconvenient if
several child files are to be compiled and
to be kept for distribution.
\end{itemize}

The present package provides a simple interface
to make child files individually compilable by \LaTeX{}.
Compiling a child file then has the same effect as compiling
the main file with an |\includeonly| command
to select the appropriate child.
Moreover the generated document will carry the name of the child
rather than the main file.
This resolves all three above issues.

This feature is meant to make the editing of books,
thesis documents and lecture notes somewhat more convenient.
However, the package can also be used efficiently for
composing a series of documents (such as exercise sheets)
which are typically distributed individually.
It then assists the author in generating the individual documents
(potentially in different versions)
as well as a document containing the collected series.
Another application is in developing style files
or other kinds of included material
where compilation of the style file could redirect
to a sample or test file.

%%%%%%%%%%%%%%%%%%%%%%%%%%%%%%%%%%%%%%%%%%%%%%%%%%%%%%%%%%%%%%%%%%%%%%%%%%%%%%%%
%%%%%%%%%%%%%%%%%%%%%%%%%%%%%%%%%%%%%%%%%%%%%%%%%%%%%%%%%%%%%%%%%%%%%%%%%%%%%%%%
\section{Usage}

First of all, the package \textsf{childdoc} is \emph{not} a standard
\LaTeXe{} |.sty| style file! Therefore it needs to be invoked in
a non-standard way.

%%%%%%%%%%%%%%%%%%%%%%%%%%%%%%%%%%%%%%%%%%%%%%%%%%%%%%%%%%%%%%%%%%%%%%%%%%%%%%%%
\subsection{Included Files}
\label{sec:include}

%%%%%%%%%%%%%%%%%%%%%%%%%%%%%%%%%%%%%%%%
\DescribeMacro{\childdocmain}
To use the package, add the commands
\begin{center}
\begin{tabular}{l}
|\input{childdoc.def}|\\
|\childdocmain{}|\\
\end{tabular}
\end{center}
at the very top of the main \LaTeX{} file,
in particular \emph{before} the |\documentclass| statement!
The argument of |\childdocmain| should be left empty
(but it must be present).

%%%%%%%%%%%%%%%%%%%%%%%%%%%%%%%%%%%%%%%%
\DescribeMacro{\childdocof}
Furthermore, add the commands
\begin{center}
\begin{tabular}{l}
|\input{childdoc.def}|\\
|\childdocof{|\textit{main}|}|\\
\end{tabular}
\end{center}
at the top of every child file \textit{child}
which is included by |\include{|\textit{child}|}|
from within the main file
(or at least for those files to be compiled individually).
The argument \textit{main} must be the filename of the main file.

There are a couple of
considerations in setting up the main and child documents:

%%%%%%%%%%%%%%%%%%%%%%%%%%%%%%%%%%%%%%%%
\paragraph{Restrictions.}

Please note the following restrictions:
\begin{itemize}
\item
|\childdocmain| must be called with one argument \textit{main}
to ensure compatibility with earlier version of the package.
It must either be empty (|\childdocmain{}|)
or precisely match the filename of the main file in which it is specified.
See \secref{sec:detection} for further information.
\item
The filename \textit{main} must be specified without the |.tex| extension.
\item
The filename \textit{main} is case sensitive
(even in case-insensitive file systems)
due to internal string comparison.
\item
The argument \textit{main} should be fully expanded, it cannot be a macro.
\item
Subdirectories and special characters should be avoided in filenames.
\item
The command |\childdocmain{|\textit{main}|}| must be followed by a whitespace.
It should not be followed immediately by another command
or by a comment mark `|%|'.
This is because the \TeX{} parser reads the token immediately following
the argument of |\childdocmain| and puts it
at the beginning of every child section;
however, a white\-space is ignored.
\end{itemize}

%%%%%%%%%%%%%%%%%%%%%%%%%%%%%%%%%%%%%%%%
\paragraph{Content of Main File.}

It is advisable to place all content in the child files included by |\include|.
Any output contained in the main file will appear in all child documents
unless suppressed manually;
it cannot be suppressed automatically by the |\includeonly| directive
and thus should normally be avoided.
A method to include some content in the main file
by means of conditional processing is described in \secref{sec:conditional}.

%%%%%%%%%%%%%%%%%%%%%%%%%%%%%%%%%%%%%%%%
\paragraph{Page Numbering.}

When only a part of the document is compiled,
the appropriate numbering of pages
(as well as other status parameters)
is determined from the |.aux| files.
The latter contain information from previous passes.
However this information needs to propagate through
all intermediate child documents.
Therefore the page numbering in child documents may well
be inconsistent until the complete document is compiled at least once.

A useful (if unconventional) way to always ensure a consistent
page numbering is to restart the numbering in each child document
and denote the pages by `\textit{child}|.|\textit{page}'
where \textit{child} represents the chapter/section number of the child file.
This can be achieved by the command
|\numberwithin{page}{|\textit{child}|}|
of the \textsf{amsmath} package
where \textit{child} can be |chapter| or |section|
depending on the chosen structuring.
Alternatively, one can modify the macro |\thepage| appropriately
and reset the counter |page| at the start of each child file.

%%%%%%%%%%%%%%%%%%%%%%%%%%%%%%%%%%%%%%%%%%%%%%%%%%%%%%%%%%%%%%%%%%%%%%%%%%%%%%%%
\subsection{Conditional Processing}
\label{sec:conditional}

The package provides a mechanism to compile different versions
of a document. To customise the versions further some conditional processing
can come in handy to distinguish which version is being compiled.
The package provides two macros to describe the compilation context:

%%%%%%%%%%%%%%%%%%%%%%%%%%%%%%%%%%%%%%%%
\DescribeMacro{\ifchilddoc}
The conditional |\ifchilddoc| distinguishes between the compilation of
child documents and the main document:
%
\begin{center}
|\ifchilddoc |\textit{child-code}| |[|\||else |\textit{main-code}]| \||fi|
\end{center}

%%%%%%%%%%%%%%%%%%%%%%%%%%%%%%%%%%%%%%%%
\DescribeMacro{\childdocname}
\DescribeMacro{\childdocjob}
The macro |\childdocname| contains the filename (without extension)
of the main or child file being processed.
Note that |\childdocjob| will always contain the name of the main file.

%%%%%%%%%%%%%%%%%%%%%%%%%%%%%%%%%%%%%%%%
\paragraph{Title Page.}

Conditional processing can be used to include a title or banner page
in the main document when proper precautions are taken.
Importantly, the code in the main file should ensure that the page counter
(as well as other status parameters which are stored in the |.aux| files)
takes the same value after the conditional processing.
Otherwise the page numbers may take divergent values
depending on which part is compiled.

For example, a title page could be declared by:
%
\begin{center}
\begin{tabular}{l}
|\ifchilddoc\||else|\\
|\addtocounter{page}{-1}|\\
\textit{code for title page}\\
|\newpage|\\
|\||fi|
\end{tabular}
\end{center}
%
A banner page for the child documents can be generated by:
%
\begin{center}
\begin{tabular}{l}
|\ifchilddoc|\\
|\addtocounter{page}{-1}|\\
\textit{code for banner page}\\
|\newpage|\\
|\||fi|
\end{tabular}
\end{center}
%
Here one could write a message such as:
\begin{center}
|This is the part \childdocname{} of \childdocjob{}.|
\end{center}

%%%%%%%%%%%%%%%%%%%%%%%%%%%%%%%%%%%%%%%%%%%%%%%%%%%%%%%%%%%%%%%%%%%%%%%%%%%%%%%%
\subsection{Flags}
\label{sec:flags}

The package makes it easy to generate different versions
of the main or child documents.
To this end compilation flags can be defined
and assigned different default values.
They will be particularly useful in conjunction
with the forwarding mechanism described in \secref{sec:forward}.

For example, it may be useful to have a flag |\version|
which can be set to |draft| or |final|.
The document source will contain some conditional code
depending on the value of |\version|.
Suppose further, the flag should default to |final| for the main file
and to |draft| for child files
which is a natural assignment for editing the document.
This is achieved by placing the following code
in the preamble of the main document
(below the |\childdocmain| directive):
%
\begin{center}
\begin{tabular}{l}
|\ifchilddoc|\\
|\providecommand{\version}{draft}|\\
|\||else|\\
|\providecommand{\version}{final}|\\
|\||fi|
\end{tabular}
\end{center}
%
The definition by |\providecommand| makes sure
that previous definitions are not overwritten.
Further statements |\providecommand{\version}{...}|
can thus be added before the above code to override it.

For the main file, one might add a line
(between |\childdocmain| and the above block)
%
\begin{center}
|%\ifchilddoc\||else\providecommand{\version}{draft}\||fi|
\end{center}
%
which can be uncommented to produce a draft version.
Likewise one can add a line to the very top of a child file
(above the |\childdocof{|\textit{main}|}| directive)
%
\begin{center}
|%\providecommand{\version}{final}|
\end{center}
%
which can be uncommented to produce the final version of this child document.

%%%%%%%%%%%%%%%%%%%%%%%%%%%%%%%%%%%%%%%%%%%%%%%%%%%%%%%%%%%%%%%%%%%%%%%%%%%%%%%%
\subsection{Forwarding}
\label{sec:forward}

Different versions of the main or child documents
using compilation flags as described in \secref{sec:flags}
can be (permanently) stored in different files
for convenient compilation, viewing and distribution.
To this end, the package defines a command
to pass on compilation to a different file:

%%%%%%%%%%%%%%%%%%%%%%%%%%%%%%%%%%%%%%%%
\DescribeMacro{\childdocforward}
The command |\childdocforward| redirects processing to
another source file:
%
\begin{center}
\begin{tabular}{l}
|\input{childdoc.def}|\\
|\childdocforward[|\textit{main}|]{|\textit{dest}|}|\\
\end{tabular}
\end{center}
%
The argument \textit{dest} is the destination file
(without extension).
It should be the main file or one of the child files.
Note that further \textsf{childdoc} directives
such as |\childdocof| and |\childdocforward|
in the indicated file will be processed in this form.
The optional argument \textit{main}
passes on directly to the main file \textit{main}
while pretending to compile the child \textit{dest}.
This form behaves as if \textit{dest}
issues |\childdocof{|\textit{main}|}| right away,
and no further \textsf{childdoc} directives will be processed.

%%%%%%%%%%%%%%%%%%%%%%%%%%%%%%%%%%%%%%%%
\DescribeMacro{\...prefix}
In the alternative form |\childdocforwardprefix|,
%
\begin{center}
\begin{tabular}{l}
|\input{childdoc.def}|\\
|\childdocforwardprefix[|\textit{main}|]{|\textit{prefix}|}{|\textit{dest}|}|
\end{tabular}
\end{center}
%
the destination file is determined by a pattern
depending on the current file:
To make this work, the current file must be called
`{\textit{prefix}\hspace{0.2em}\textit{suffix}}'
with \textit{prefix} matching precisely the argument.
Processing is then passed on to the file
`{\textit{dest}\hspace{0.2em}\textit{suffix}}'.
Surely, the same effect is achieved by
directly specifying the
argument `{\textit{dest}\hspace{0.2em}\textit{suffix}}'
in the first form.
However, that requires to set up a different file
for each child. With the alternative form of the command
all these files can have exactly the same content
which simplifies setting them up and maintaining them.

For example, the following file |draft.tex|
with a compilation flag |\version| as described in \secref{sec:flags}
compiles the main document as a draft:
%
\begin{center}
\begin{tabular}{l}
|\def\version{draft}|\\
|\input{childdoc.def}|\\
|\childdocforward{|\textit{main}|}|
\end{tabular}
\end{center}
%
Likewise, the following files |final|\textit{nn}|.tex|
compile the final version of the child document
|child|\textit{nn}|.tex|:
%
\begin{center}
\begin{tabular}{l}
|\def\version{final}|\\
|\input{childdoc.def}|\\
|\childdocforwardprefix{final}{child}|
\end{tabular}
\end{center}
%

Note that when several versions of a main file and/or of each child file
are to be generated, it may be convenient to set up a |Makefile| or
shell script to automatise the process.

%%%%%%%%%%%%%%%%%%%%%%%%%%%%%%%%%%%%%%%%%%%%%%%%%%%%%%%%%%%%%%%%%%%%%%%%%%%%%%%%
\subsection{Command Line Processing}
\label{sec:commandline}

The effect of redirection files can also be achieved by invoking
the \LaTeX{} compiler with a more elaborate command line.
Most conveniently this should be done as part
of a shell script or a |Makefile|.

When using \textsf{childdoc} in the main file, the following
command lines effectively perform a redirection
(note that depending on the shell being used,
backslashes may have to be doubled: `|\|' $\to$ `|\\|'):
%
\begin{center}
|... -jobname "|\textit{target}|" |\\|"|[\textit{flags}]%
|\input{childdoc.def}\childdocforward[|\textit{main}|]{|\textit{dest}|}"|
\end{center}
%
Here \textit{target} is the name of the output file,
\textit{main} is the name of the main file
and \textit{dest} is the name of the main or child file to be processed
(all filenames without extensions).
The optional argument \textit{main} can be omitted
if \textit{main} matches \textit{dest}.
Optionally, compilation \textit{flags} can be defined via |\def| commands.
This command line makes the \TeX{} engine believe
it is compiling the file \textit{target}
whose content is specified as the latter parameter.
The provided code then forwards the processing to
\textit{main} or \textit{dest} as described in \secref{sec:forward}.

%%%%%%%%%%%%%%%%%%%%%%%%%%%%%%%%%%%%%%%%%%%%%%%%%%%%%%%%%%%%%%%%%%%%%%%%%%%%%%%%
\subsection{Include by Input}
\label{sec:input}

Including child documents by |\include| has some restrictions by design.
Most notably, the content of a child document always occupies
its own set of pages; pages cannot be shared between child documents.
Usually, this behaviour makes perfect sense
because each child document contain an essential part of the document.
However, in some situations it may be desirable to compose
a document from a collection of parts
without having mandatory page breaks between then.
For this case, the package
provides a mechanism to include parts
by |\input| which can also be processed individually.
However, by construction this mechanism
requires manual handling of the content to be output.

%%%%%%%%%%%%%%%%%%%%%%%%%%%%%%%%%%%%%%%%
\DescribeMacro{\ifchilddocmanual}
The main file should be prepared as usual, see \secref{sec:include}.
However, the document body must make a distinction
between processing of an individual part and of the main document, e.g.:
%
\begin{center}
\begin{tabular}{l}
|\ifchilddocmanual|\\
|\input{\childdocname}|\\
|\||else|\\
\textit{document body with }|\input{|\textit{part}|}|\\
|\||fi|
\end{tabular}
\end{center}
%
The conditional |\ifchilddocmanual| is true whenever
a part to be included by |\input| is being compiled,
and the name of the part is stored in |\childdocname|.

%%%%%%%%%%%%%%%%%%%%%%%%%%%%%%%%%%%%%%%%
\DescribeMacro{\childdocby}
Each part to be included by |\input| should start with:
%
\begin{center}
\begin{tabular}{l}
|\input{childdoc.def}|\\
|\childdocby{|\textit{main}|}|\\
\end{tabular}
\end{center}
%
The directive |\childdocby| is similar to |\childdocof|
described in \secref{sec:include},
but the subsequent selection of content must be done manually.
To that end, both |\ifchilddoc| and |\ifchilddocmanual|
will be true upon processing of a part,
and the name of the part is stored in |\childdocname|.
Note that |\jobname| will be set to the filename of the current part
so that each part receives an individual |.aux| file
that does not interfere with the |.aux| file(s) of the main document.
This behaviour can be altered by the alternative form
|\childdocby[*]{|\textit{main}|}| (with a non-empty optional argument)
which uses the |.aux| file of the main document
by setting |\jobname| to \textit{main}.

%%%%%%%%%%%%%%%%%%%%%%%%%%%%%%%%%%%%%%%%%%%%%%%%%%%%%%%%%%%%%%%%%%%%%%%%%%%%%%%%
\subsection{Driver Development}
\label{sec:driver}

The \textsf{childdoc} mechanism can also be use for the development
of definition files such as \LaTeX{} styles or classes.
This case differs from the above setup with multiple parts
included by |\include| in that no |\includeonly| should be invoked.
This can be achieved by starting the include file
(before |\ProvidesPackage|) with:
%
\begin{center}
\begin{tabular}{l}
|\input{childdoc.def}|\\
|\childdocforward{|\textit{main}|}|\\
\end{tabular}
\end{center}
%
or alternatively with:
%
\begin{center}
\begin{tabular}{l}
|\input{childdoc.def}|\\
|\childdocby{|\textit{main}|}|\\
\end{tabular}
\end{center}
%
Both forms have slightly different effects as described above.
The main file is prepared as usual, see \secref{sec:include}.

%%%%%%%%%%%%%%%%%%%%%%%%%%%%%%%%%%%%%%%%%%%%%%%%%%%%%%%%%%%%%%%%%%%%%%%%%%%%%%%%
\subsection{Legacy Detection}
\label{sec:detection}

The directive |\childdocmain| in the main file can detect
whether the complete document or merely a child is to be compiled
even without using the directive |\childdocof|.
This method is deprecated because it is less robust
and there is no compelling reason to use it;
it is merely provided for backward compatibility
and it may be removed in future versions.

If the detection mechanism is to be used,
it is mandatory to correctly specify
the filename of the main file as the argument of |\childdocmain|:
%
\begin{center}
\begin{tabular}{l}
|\input{childdoc.def}|\\
|\childdocmain{|\textit{main}|}|\\
\end{tabular}
\end{center}
%
If |\jobname| does not match the argument \textit{main} of |\childdocmain|,
it is assumed that |\jobname| points to the child file to be compiled.
When using |\childdocmain| with the main file specified as argument,
it suffices to start a child file
with just |\input{|\textit{main}|}|
without loading of the package and using |\childdocof|.
If instead all processing is done
with the appropriate \textsf{childdoc} directives,
the argument of \textit{main} of |\childdocmain| can be empty.

An alternative version of the command line processing described
in \secref{sec:commandline} using the detection mechanism reads:
%
\begin{center}
|... -jobname "|\textit{target}|" "|[\textit{flags}]%
[|\def\jobname{|\textit{dest}|}|]|\input{|\textit{main}|}"|
\end{center}

%%%%%%%%%%%%%%%%%%%%%%%%%%%%%%%%%%%%%%%%%%%%%%%%%%%%%%%%%%%%%%%%%%%%%%%%%%%%%%%%
\subsection{Manual Code}
\label{sec:manual}

In case one cannot be certain whether the definitions file |childdoc.def|
is installed on the target \TeX{} distribution
and one prefers not to ship it,
it is conceivable to paste a few relevant commands into the sources.

To that end, drop all statements |\input{childdoc.def}|
and perform the replacements as outlined below.
Instead of |\childdocmain{|\textit{main}|}| add the following code
to the top of the main file:
%
\begin{center}
\begin{tabular}{l}
|\||ifdefined\childdocname\endinput\||fi\newif\ifchilddoc|\\
|\edef\childdocname{\scantokens\expandafter{\jobname\noexpand}}|\\
|\def\childdocmain{|\textit{main}|}\||ifx\childdocmain\childdocname\||else|\\
|\childdoctrue\includeonly{\childdocname}\let\jobname\childdocmain\||fi|\\
\end{tabular}
\end{center}
%
Instead of |\childdocof{|\textit{main}|}| just include the main file
at the top of each child file:
%
\begin{center}
|\input{|\textit{main}|}|
\end{center}
%
A simple redirection |\childdocforward{|\textit{dest}|}| is achieved by:
%
\begin{center}
|\def\jobname{|\textit{dest}|}\input{\jobname}|
\end{center}
%
The redirection with prefix
|\childdocforwardprefix[|\textit{prefix}|]{|\textit{dest}|}|
is accomplished by:
%
\begin{center}
\begin{tabular}{l}
|{\edef\jobname{\scantokens\expandafter{\jobname\noexpand}}|\\
|\def\redirectjob |\textit{prefix}|#1~~~{\gdef\jobname{|\textit{dest}|#1}}|\\
|\expandafter\redirectjob\jobname~~~}\input{\jobname}|
\end{tabular}
\end{center}

In an alternative approach,
child documents can be compiled by a specific command line
without additional code or specific definitions:
%
\begin{center}
|... -jobname "|\textit{target}|" "|[\textit{flags}]%
|\includeonly{|\textit{dest}|}\input{|\textit{main}|}"|
\end{center}
%

%%%%%%%%%%%%%%%%%%%%%%%%%%%%%%%%%%%%%%%%%%%%%%%%%%%%%%%%%%%%%%%%%%%%%%%%%%%%%%%%
%%%%%%%%%%%%%%%%%%%%%%%%%%%%%%%%%%%%%%%%%%%%%%%%%%%%%%%%%%%%%%%%%%%%%%%%%%%%%%%%
\section{Information}

%%%%%%%%%%%%%%%%%%%%%%%%%%%%%%%%%%%%%%%%%%%%%%%%%%%%%%%%%%%%%%%%%%%%%%%%%%%%%%%%
\subsection{Copyright}

Copyright \copyright{} 2017--2018 Niklas Beisert

This work may be distributed and/or modified under the
conditions of the \LaTeX{} Project Public License, either version 1.3
of this license or (at your option) any later version.
The latest version of this license is in
  \url{http://www.latex-project.org/lppl.txt}
and version 1.3 or later is part of all distributions of \LaTeX{}
version 2005/12/01 or later.

This work has the LPPL maintenance status `maintained'.

The Current Maintainer of this work is Niklas Beisert.

This work consists of the files |README.txt|, |childdoc.ins| and |childdoc.dtx|
as well as the derived files |childdoc.def|, |cdocsamp.tex|
with |cdocsch1.tex|, |cdocsch2.tex|, |cdocspt3.tex|, |cdocspt4.tex|,
|cdocsdrf.tex|, |cdocsfn1.tex|, |cdocsfn2.tex|
as well as |childdoc.pdf|.

%%%%%%%%%%%%%%%%%%%%%%%%%%%%%%%%%%%%%%%%%%%%%%%%%%%%%%%%%%%%%%%%%%%%%%%%%%%%%%%%
\subsection{Files and Installation}

The package consists of the files:
%
\begin{center}
\begin{tabular}{ll}
    |README.txt|   & readme file \\
    |childdoc.ins| & installation file \\
    |childdoc.dtx| & source file \\
    |childdoc.def| & definition file \\
    |cdocsamp.tex| & sample main file \\
    |cdocsch1.tex| & sample include file \\
    |cdocsch2.tex| & sample include file \\
    |cdocspt3.tex| & sample part file \\
    |cdocspt4.tex| & sample part file \\
    |cdocsdrf.tex| & sample redirection file \\
    |cdocsfn1.tex| & sample redirection file \\
    |cdocsfn2.tex| & sample redirection file \\
    |childdoc.pdf| & manual
\end{tabular}
\end{center}
%
The distribution consists of the files
|README.txt|, |childdoc.ins| and |childdoc.dtx|.
%
\begin{itemize}
\item
Run (pdf)\LaTeX{} on |childdoc.dtx|
to compile the manual |childdoc.pdf| (this file).
\item
Run \LaTeX{} on |childdoc.ins| to create the definitions file |childdoc.def|
and the sample |cdocsamp.tex| with include files
|cdocsch1.tex|, |cdocsch2.tex|, |cdocspt3.tex|, |cdocspt4.tex|,
|cdocsdrf.tex|, |cdocsfn1.tex|, |cdocsfn2.tex|.
Then copy the file |childdoc.def| to an appropriate directory of your \LaTeX{}
distribution, e.g.\ \textit{texmf-root}|/tex/latex/childdoc|.
\end{itemize}

%%%%%%%%%%%%%%%%%%%%%%%%%%%%%%%%%%%%%%%%%%%%%%%%%%%%%%%%%%%%%%%%%%%%%%%%%%%%%%%%
\subsection{Related CTAN Packages}

There are several other packages which offer a similar functionality:
%
\begin{itemize}
\item
The packages
\href{http://ctan.org/pkg/docmute}{\textsf{docmute}},
\href{http://ctan.org/pkg/includex}{\textsf{includex}} and
\href{http://ctan.org/pkg/standalone}{\textsf{standalone}}
provide commands to include only the document body of
a child file thus allowing both files to be compiled individually.
\item
The packages \href{http://ctan.org/pkg/subdocs}{\textsf{subdocs}}
and \href{http://ctan.org/pkg/subfiles}{\textsf{subfiles}}
provide structures in which the main and child documents can be
encapsulated and allowing them to be compiled individually.
The inclusion mechanism is different from the conventional |\include|.
\item
The package \href{http://ctan.org/pkg/combine}{\textsf{combine}}
is an elaborate solution to combine several documents into one.
\end{itemize}
%
See also the CTAN topic \href{http://ctan.org/topic/subdocs}{\textsf{subdocs}}
for further related packages.
The present package differs from the above solutions in that
a document structure constructed with the conventional |\include| mechanism
just needs two extra commands at the top of every file
such that all constituent files can be compiled individually.

%%%%%%%%%%%%%%%%%%%%%%%%%%%%%%%%%%%%%%%%%%%%%%%%%%%%%%%%%%%%%%%%%%%%%%%%%%%%%%%%
%\subsection{Feature Suggestions}
%
%The following is a list of features which may be useful for future
%versions of this package:
%%
%\begin{itemize}
%\item
%\ldots
%\end{itemize}

%%%%%%%%%%%%%%%%%%%%%%%%%%%%%%%%%%%%%%%%%%%%%%%%%%%%%%%%%%%%%%%%%%%%%%%%%%%%%%%%
\subsection{Revision History}

%%%%%%%%%%%%%%%%%%%%%%%%%%%%%%%%%%%%%%%%
\paragraph{v2.0:} 2018/12/30

\begin{itemize}
\item
immediate forward processing
\item
added |\childdocby| mechanism
\item
manual restructured
\end{itemize}

%%%%%%%%%%%%%%%%%%%%%%%%%%%%%%%%%%%%%%%%
\paragraph{v1.6:} 2018/01/17

\begin{itemize}
\item
application for development of include files
\item
corrections to manual
\end{itemize}

%%%%%%%%%%%%%%%%%%%%%%%%%%%%%%%%%%%%%%%%
\paragraph{v1.5:} 2017/05/21

\begin{itemize}
\item
more complete structuring introduced
\item
|\childdocof| introduced
\item
|\childdoc| renamed to |\childdocmain|
\item
|\childredirect| renamed to |\childdocforward| and |\childdocforwardprefix|
and functionality expanded
\end{itemize}

%%%%%%%%%%%%%%%%%%%%%%%%%%%%%%%%%%%%%%%%
\paragraph{v1.0:} 2017/04/27

\begin{itemize}
\item
manual and install package
\item
first version published on CTAN
\end{itemize}

%%%%%%%%%%%%%%%%%%%%%%%%%%%%%%%%%%%%%%%%
\paragraph{v0.6:} 2017/04/26

\begin{itemize}
\item
redirection mechanism added
\end{itemize}

%%%%%%%%%%%%%%%%%%%%%%%%%%%%%%%%%%%%%%%%
\paragraph{v0.5:} 2017/04/26

\begin{itemize}
\item
functionality in definition file
\end{itemize}


%%%%%%%%%%%%%%%%%%%%%%%%%%%%%%%%%%%%%%%%%%%%%%%%%%%%%%%%%%%%%%%%%%%%%%%%%%%%%%%%
%%%%%%%%%%%%%%%%%%%%%%%%%%%%%%%%%%%%%%%%%%%%%%%%%%%%%%%%%%%%%%%%%%%%%%%%%%%%%%%%
%%%%%%%%%%%%%%%%%%%%%%%%%%%%%%%%%%%%%%%%%%%%%%%%%%%%%%%%%%%%%%%%%%%%%%%%%%%%%%%%
\appendix

\settowidth\MacroIndent{\rmfamily\scriptsize 000\ }

 \DocInput{childdoc.dtx}

\end{document}
%</driver>
% \fi
%
% %%%%%%%%%%%%%%%%%%%%%%%%%%%%%%%%%%%%%%%%%%%%%%%%%%%%%%%%%%%%%%%%%%%%%%%%%%%%%%
% %%%%%%%%%%%%%%%%%%%%%%%%%%%%%%%%%%%%%%%%%%%%%%%%%%%%%%%%%%%%%%%%%%%%%%%%%%%%%%
% \section{Sample}
%\iffalse
%<*samplemain>
%\fi
%
% The following presents a sample document
% with two chapters, two parts, a title page,
% a compile flag as well as three forwarding files to set the flag.
% It consists of eight |.tex| files:
% \begin{center}
% \begin{tabular}{ll}
% |cdocsamp.tex|&main file\\
% |cdocsch1.tex|&include file for chapter 1\\
% |cdocsch2.tex|&include file for chapter 2\\
% |cdocspt3.tex|&include file for part 3\\
% |cdocspt4.tex|&include file for part 4\\
% |cdocsdrf.tex|&forwarding file for main file in draft mode\\
% |cdocsfi1.tex|&forwarding file for final version of chapter 1\\
% |cdocsfi2.tex|&forwarding file for final version of chapter 2\\
% \end{tabular}
% \end{center}
% Each of the eight files can be compiled directly by the \LaTeX{} compiler.
%
% %%%%%%%%%%%%%%%%%%%%%%%%%%%%%%%%%%%%%%
% \paragraph{Main File.}
%
% The main file is called |cdocsamp.tex|.
%
% Load the \textsf{childdoc} definitions and
% declare the filename for the main document:
%    \begin{macrocode}
\input{childdoc.def}
\childdocmain{}
%    \end{macrocode}

% Optional override for |\version| flag:
%    \begin{macrocode}
%%\ifchilddoc\else\providecommand{\version}{draft}\fi
%    \end{macrocode}

% Define the default values for the |\version| flag
% (|final| for the main file and |draft| for childs):
%    \begin{macrocode}
\ifchilddoc
\providecommand{\version}{draft}
\else
\providecommand{\version}{final}
\fi
%    \end{macrocode}

% Load the standard document class:
%    \begin{macrocode}
\documentclass[12pt]{article}
%    \end{macrocode}

% Start the document body:
%    \begin{macrocode}
\begin{document}
%    \end{macrocode}

% Declare a title page.
% Print title, part of document being processed and version flag:
%    \begin{macrocode}
\addtocounter{page}{-1}
\begin{center}
{\LARGE\bfseries{}childdoc example\par}
\vspace{1cm}
\ifchilddoc
\ifchilddocmanual part\else chapter\fi:
`\childdocname' of `\childdocjob'\par
\else
main document: `\childdocjob'\par
\fi
version: \version\par
\end{center}
\newpage
%    \end{macrocode}

% Manually include selected file,
% otherwise process as usual:
%    \begin{macrocode}
\ifchilddocmanual
\section*{part `\childdocname'}
\input{\childdocname}
\else
%    \end{macrocode}

% Include the two chapters:
%    \begin{macrocode}
\include{cdocsch1}
\include{cdocsch2}
%    \end{macrocode}

% Include the two parts unless only chapters should be displayed:
%    \begin{macrocode}
\ifchilddoc\else
\section{part three}
\input{cdocspt3}
\section{part four}
\input{cdocspt4}
\fi
%    \end{macrocode}

% Process as usual until here:
%    \begin{macrocode}
\fi
%    \end{macrocode}

% End of document body:
%    \begin{macrocode}
\end{document}
%    \end{macrocode}
%\iffalse
%</samplemain>
%\fi
%
% %%%%%%%%%%%%%%%%%%%%%%%%%%%%%%%%%%%%%%
% \paragraph{Chapter Include Files.}
%
% The include files are called |cdocsch1.tex| and |cdocsch2.tex|.
%
%\iffalse
%<*samplechap1|samplechap2>
%\fi

% Optional override for |\version| flag:
%    \begin{macrocode}
%%\providecommand{\version}{final}
%    \end{macrocode}

% Include the main document:
%    \begin{macrocode}
\input{childdoc.def}
\childdocof{cdocsamp}
%    \end{macrocode}

%\iffalse
%</samplechap1|samplechap2>
%\fi
%
%\iffalse
%<*samplechap1>
%\fi
% Some text for chapter 1:
%    \begin{macrocode}
\section{one}
some text in chapter one
%    \end{macrocode}

%\iffalse
%</samplechap1>
%\fi
% Some text for chapter 2:
%\iffalse
%<*samplechap2>
%\fi
%    \begin{macrocode}
\section{two}
more text in chapter two
%    \end{macrocode}

%\iffalse
%</samplechap2>
%\fi
%
% %%%%%%%%%%%%%%%%%%%%%%%%%%%%%%%%%%%%%%
% \paragraph{Part Include Files.}
%
% The include files are called |cdocspt3.tex| and |cdocspt4.tex|.
%
%\iffalse
%<*samplepart3|samplepart4>
%\fi

% Optional override for |\version| flag:
%    \begin{macrocode}
%%\providecommand{\version}{final}
%    \end{macrocode}

% Include the main document:
%    \begin{macrocode}
\input{childdoc.def}
\childdocby{cdocsamp}
%    \end{macrocode}

%\iffalse
%</samplepart3|samplepart4>
%\fi
%
%\iffalse
%<*samplepart3>
%\fi
% Some text for part 3:
%    \begin{macrocode}
some text in part three
%    \end{macrocode}

%\iffalse
%</samplepart3>
%\fi
% Some text for part 4:
%\iffalse
%<*samplepart4>
%\fi
%    \begin{macrocode}
more text in part four
%    \end{macrocode}

%\iffalse
%</samplepart4>
%\fi
%
% %%%%%%%%%%%%%%%%%%%%%%%%%%%%%%%%%%%%%%
% \paragraph{Forwarding for a Complete Draft.}
%
% The following forwarding file |cdocsdrf.tex|
% compiles the main document in draft mode:
%\iffalse
%<*sampledraft>
%\fi
%    \begin{macrocode}
\def\version{draft}
\input{childdoc.def}
\childdocforward{cdocsamp}
%    \end{macrocode}

%\iffalse
%</sampledraft>
%\fi
%
% %%%%%%%%%%%%%%%%%%%%%%%%%%%%%%%%%%%%%%
% \paragraph{Forwarding for Final Version of the Chapters.}
%
% The following forwarding files |cdocsfn1.tex| and |cdocsfn2.tex|
% (with identical content)
% compile the final versions of the child documents
% |cdocsch1.tex| and |cdocsch2.tex|, respectively:
%\iffalse
%<*samplefinal>
%\fi
%    \begin{macrocode}
\def\version{final}
\input{childdoc.def}
\childdocforwardprefix[cdocsamp]{cdocsfn}{cdocsch}
%    \end{macrocode}

%\iffalse
%</samplefinal>
%\fi
%
% %%%%%%%%%%%%%%%%%%%%%%%%%%%%%%%%%%%%%%
% \paragraph{Command Line Processing.}
%
% The following three command lines generate the output files
% |cdocscld|, |cdocscl1| and |cdocscl2|
% which should be identical to
% |cdocsdrf|, |cdocsch1| and |cdocsfn2|, respectively:
% \begin{center}
% \begin{tabular}{l}
% |latex -jobname cdocscld \|\\
% |  "\def\version{draft}\input{childdoc.def}\childdocforward{cdocsamp}"|\\
% |latex -jobname cdocscl1 \|\\
% |  "\input{childdoc.def}\childdocforward[cdocsamp]{cdocsch1}"|\\
% |latex -jobname cdocscl2 \|\\
% |  "\def\version{final}\input{childdoc.def}\childdocforward{cdocsch2}"|
% \end{tabular}
% \end{center}
% Note that the trailing backslash on each first line
% merely continues the input to the second line
% (for convenient cut ant paste).
% Furthermore, the command |latex| can be replaced by any
% of its alternative versions such as |pdflatex|.
%
% %%%%%%%%%%%%%%%%%%%%%%%%%%%%%%%%%%%%%%%%%%%%%%%%%%%%%%%%%%%%%%%%%%%%%%%%%%%%%%
% %%%%%%%%%%%%%%%%%%%%%%%%%%%%%%%%%%%%%%%%%%%%%%%%%%%%%%%%%%%%%%%%%%%%%%%%%%%%%%
% \section{Implementation}
%\iffalse
%<*package>
%\fi
%
% This section describes the definitions file |childdoc.def|.

% The definitions cannot be loaded using |\usepackage| or |\RequirePackage|
% which has a mechanism to prevent loading a style file more than once.
% When loading the definitions by means of |\input|
% multiple instances have to be prevented manually:
%\iffalse
%This code needs to be before the `\ProvidesFile' directive
%which is defined at the beginning of this file.
%Therefore it is also placed there and commented out here.
%</package>
%<*discard>
%\fi
%    \begin{macrocode}
\ifdefined\childdocmain\endinput\fi
%    \end{macrocode}
%\iffalse
%</discard>
%<*package>
%\fi
%
% \macro{\ifchilddoc}
% \macro{\ifchilddocmanual}
% The conditional |\ifchilddoc| tells whether a
% child (true) or main (false) document is being compiled.
% The conditional |\ifchilddocmanual| tells whether
% the |\includeonly| mechanism is used (false) or
% the selection of child files must be performed manually (true).
% The definitions initialise to false:
%    \begin{macrocode}
\newif\ifchilddoc
\newif\ifchilddocmanual
%    \end{macrocode}

% \macro{\childdocname}
% \macro{\childdocjob}
% The macro |\childdocname| stores the name of the main document
% to be compiled. The macro |\childdocjob| stores the name of
% the document on which the \LaTeX{} compiler was originally invoked.
% The content of |\jobname| cannot be compared
% to filenames specified in the source due to different catcodes.
% The following code rescans |\jobname|, stores the result
% in |\childdocname| and saves a copy in |\childdocjob|:
%    \begin{macrocode}
\edef\childdocname{\scantokens\expandafter{\jobname\noexpand}}
\let\childdocjob\childdocname
%    \end{macrocode}

% \macro{\childdocdisable}
% The macro |\childdocdisable| prevents the main file
% from being processed more than once.
% At this stage, the main document command |\childdocmain|
% is assumed to be called once again where it should do nothing.
% Any subsequent call to it should prevent
% a secondary processing of the main document
% It overwrites the forwarding commands
% |\childdocof| and |\childdocforward|
% with empty macros to prevent further inclusions of the main document:
%    \begin{macrocode}
\newcommand{\childdocdisable}
{
  \renewcommand{\childdocmain}[1]{\renewcommand{\childdocmain}[1]{\endinput}}
  \renewcommand{\childdocof}[1]{}
  \renewcommand{\childdocby}[2][]{}
  \renewcommand{\childdocforward}[2][]{}
  \renewcommand{\childdocdisable}{}
}
%    \end{macrocode}

% \macro{\childdocmain}
% The macro |\childdocmain| is to be called at the top of the main file
% with nothing or the main filename (without extension) as argument.
% First, it breaks loops.
% If the argument is not empty and does not match |\childdocname|
% (which is set by the first inclusion of |childdoc.def|),
% |\ifchilddoc| is set to true, |\includeonly| is applied to the child file
% and |\jobname| is set to the main file
% (for proper handling of |.aux| files):
%    \begin{macrocode}
\newcommand{\childdocmain}[1]
{
  \childdocdisable\childdocmain{}
  \if?#1?\else
    \begingroup
      \def\childdoctmp{#1}
      \ifx\childdoctmp\childdocname
        \def\childdoctmp{}
      \else
        \def\childdoctmp
        {
          \childdoctrue
          \includeonly{\childdocname}
          \def\childdocjob{#1}
          \def\jobname{#1}
        }
      \fi
      \expandafter
    \endgroup
    \childdoctmp
  \fi
}
%    \end{macrocode}

% \macro{\childdocof}
% The command |\childdocof| redirects
% compilation to the main file |#1|.
%    \begin{macrocode}
\newcommand{\childdocof}[1]
{
  \childdocdisable
  \childdoctrue
  \includeonly{\childdocname}
  \def\jobname{#1}
  \def\childdocjob{#1}
  \input{#1}
}
%    \end{macrocode}

% \macro{\childdocby}
% The command |\childdocby| ....
%    \begin{macrocode}
\newcommand{\childdocby}[2][]
{
  \childdocdisable
  \childdoctrue
  \childdocmanualtrue
  \if?#1?\else
    \def\jobname{#2}
  \fi
  \def\childdocjob{#2}
  \input{#2}
  \endinput
}
%    \end{macrocode}

% \macro{\childdocforward}
% The command |\childdocforward| redirects
% compilation to the main file or
% (if the optional argument is given) a child file.
% Parameters are set as if the main file
% or a child file starting with |\childdocof| was compiled.
% Then compilation is handed over to the main file:
%    \begin{macrocode}
\newcommand{\childdocforward}[2][]
{
  \begingroup
    \if?#1?
      \def\childdoctmp
      {
        \def\childdocname{#2}
        \def\childdocjob{#2}
        \def\jobname{#2}
        \input{#2}
        \endinput
      }
    \else
      \def\childdoctmp
      {
        \childdocdisable
        \def\childdocname{#2}
        \childdoctrue
        \includeonly{#2}
        \def\childdocjob{#1}
        \def\jobname{#1}
        \input{#1}
        \endinput
      }
    \fi
    \expandafter
  \endgroup
  \childdoctmp
}
%    \end{macrocode}

% \macro{\childdocforwardprefix}
% The command |\childdocforwardprefix| redirects
% compilation to the main or a child file by means of a pattern.
% The prefix |#1| in the current filename is replaced by |#2|
% and the suffix of the current filename is kept
% (it is assumed that the filename does not contain the substring `|~~~|'
% which is used as a delimiter).
% Compilation is handed over to the new file by |\childdocforward|:
%    \begin{macrocode}
\newcommand{\childdocforwardprefix}[3][]
{
  \begingroup
    \def\childdocextract #2##1~~~{\def\childdoctmp{\childdocforward[#1]{#3##1}}}
    \expandafter\childdocextract\childdocname~~~
    \expandafter
  \endgroup
  \childdoctmp
}
%    \end{macrocode}

% \macro{\childdoc}
% The deprecated macro |\childdoc| is a legacy version of |\childdocmain|:
%    \begin{macrocode}
\newcommand{\childdoc}{\childdocmain}
%    \end{macrocode}

% \macro{\childdocredirect}
% The deprecated macro |\childdocredirect| is a legacy version
% of |\childdocforward| and |\childdocforwardprefix|:
%    \begin{macrocode}
\newcommand{\childdocredirect}[2][]
{
  \begingroup
    \if?#1?
      \def\childdoctmp{\childdocforward{#2}}
    \else
      \def\childdoctmp{\childdocforwardprefix{#1}{#2}}
    \fi
    \expandafter
  \endgroup
  \childdoctmp
}
%    \end{macrocode}

%\iffalse
%</package>
%\fi
%
\endinput

\childdocmain{}
%    \end{macrocode}

% Optional override for |\version| flag:
%    \begin{macrocode}
%%\ifchilddoc\else\providecommand{\version}{draft}\fi
%    \end{macrocode}

% Define the default values for the |\version| flag
% (|final| for the main file and |draft| for childs):
%    \begin{macrocode}
\ifchilddoc
\providecommand{\version}{draft}
\else
\providecommand{\version}{final}
\fi
%    \end{macrocode}

% Load the standard document class:
%    \begin{macrocode}
\documentclass[12pt]{article}
%    \end{macrocode}

% Start the document body:
%    \begin{macrocode}
\begin{document}
%    \end{macrocode}

% Declare a title page.
% Print title, part of document being processed and version flag:
%    \begin{macrocode}
\addtocounter{page}{-1}
\begin{center}
{\LARGE\bfseries{}childdoc example\par}
\vspace{1cm}
\ifchilddoc
\ifchilddocmanual part\else chapter\fi:
`\childdocname' of `\childdocjob'\par
\else
main document: `\childdocjob'\par
\fi
version: \version\par
\end{center}
\newpage
%    \end{macrocode}

% Manually include selected file,
% otherwise process as usual:
%    \begin{macrocode}
\ifchilddocmanual
\section*{part `\childdocname'}
\input{\childdocname}
\else
%    \end{macrocode}

% Include the two chapters:
%    \begin{macrocode}
\include{cdocsch1}
\include{cdocsch2}
%    \end{macrocode}

% Include the two parts unless only chapters should be displayed:
%    \begin{macrocode}
\ifchilddoc\else
\section{part three}
\input{cdocspt3}
\section{part four}
\input{cdocspt4}
\fi
%    \end{macrocode}

% Process as usual until here:
%    \begin{macrocode}
\fi
%    \end{macrocode}

% End of document body:
%    \begin{macrocode}
\end{document}
%    \end{macrocode}
%\iffalse
%</samplemain>
%\fi
%
% %%%%%%%%%%%%%%%%%%%%%%%%%%%%%%%%%%%%%%
% \paragraph{Chapter Include Files.}
%
% The include files are called |cdocsch1.tex| and |cdocsch2.tex|.
%
%\iffalse
%<*samplechap1|samplechap2>
%\fi

% Optional override for |\version| flag:
%    \begin{macrocode}
%%\providecommand{\version}{final}
%    \end{macrocode}

% Include the main document:
%    \begin{macrocode}
% \iffalse
%
% childdoc.dtx Copyright (C) 2017-2018 Niklas Beisert
%
% This work may be distributed and/or modified under the
% conditions of the LaTeX Project Public License, either version 1.3
% of this license or (at your option) any later version.
% The latest version of this license is in
%   http://www.latex-project.org/lppl.txt
% and version 1.3 or later is part of all distributions of LaTeX
% version 2005/12/01 or later.
%
% This work has the LPPL maintenance status `maintained'.
%
% The Current Maintainer of this work is Niklas Beisert.
%
% This work consists of the files childdoc.dtx and childdoc.ins
% and the derived files childdoc.def and cdocsamp.tex with
% cdocsch1.tex, cdocsch2.tex, cdocsdrf.tex, cdocsfn1.tex, cdocsfn2.tex.
%
%<package>\ifdefined\childdocmain\endinput\fi
%<package>\ProvidesFile{childdoc.def}[2018/12/30 v2.0 child document driver]
%<samplemain>\ProvidesFile{cdocsamp.tex}[2018/12/30 v2.0 sample for childdoc]
%<*driver>
%\ProvidesFile{childdoc.drv}[2018/12/30 v2.0 childdoc reference manual file]
\PassOptionsToClass{10pt,a4paper}{article}
\documentclass{ltxdoc}

\usepackage[margin=35mm]{geometry}
\usepackage{hyperref}
\usepackage{hyperxmp}
\usepackage[usenames]{color}

\hypersetup{colorlinks=true}
\hypersetup{pdfstartview=FitH}
\hypersetup{pdfpagemode=UseNone}
\hypersetup{pdfsource={}}
\hypersetup{pdflang={en-UK}}
\hypersetup{pdfcopyright={Copyright 2017-2018 Niklas Beisert.
  This work may be distributed and/or modified under the
  conditions of the LaTeX Project Public License, either version 1.3
  of this license or (at your option) any later version.}}
\hypersetup{pdflicenseurl={http://www.latex-project.org/lppl.txt}}
\hypersetup{pdfcontactaddress={ETH Zurich, ITP, HIT K,
  Wolfgang-Pauli-Strasse 27}}
\hypersetup{pdfcontactpostcode={8093}}
\hypersetup{pdfcontactcity={Zurich}}
\hypersetup{pdfcontactcountry={Switzerland}}
\hypersetup{pdfcontactemail={nbeisert@itp.phys.ethz.ch}}
\hypersetup{pdfcontacturl={http://people.phys.ethz.ch/\xmptilde nbeisert/}}

\newcommand{\secref}[1]{\hyperref[#1]{section \ref*{#1}}}

\parskip1ex
\parindent0pt
\let\olditemize\itemize
\def\itemize{\olditemize\parskip0pt}

\begin{document}

\title{The \textsf{childdoc} Package}
\hypersetup{pdftitle={The childdoc Package}}
\author{Niklas Beisert\\[2ex]
  Institut f\"ur Theoretische Physik\\
  Eidgen\"ossische Technische Hochschule Z\"urich\\
  Wolfgang-Pauli-Strasse 27, 8093 Z\"urich, Switzerland\\[1ex]
  \href{mailto:nbeisert@itp.phys.ethz.ch}
  {\texttt{nbeisert@itp.phys.ethz.ch}}}
\hypersetup{pdfauthor={Niklas Beisert}}
\hypersetup{pdfsubject={Manual for the LaTeX2e Package childdoc}}
\date{30 December 2018, \textsf{v2.0}}
\maketitle

\begin{abstract}\noindent
\textsf{childdoc} is a \LaTeXe{} package
that enables the direct compilation
of document sections included by |\include|
to individual files.
\end{abstract}

\begingroup
\parskip0ex
\tableofcontents
\endgroup

%%%%%%%%%%%%%%%%%%%%%%%%%%%%%%%%%%%%%%%%%%%%%%%%%%%%%%%%%%%%%%%%%%%%%%%%%%%%%%%%
%%%%%%%%%%%%%%%%%%%%%%%%%%%%%%%%%%%%%%%%%%%%%%%%%%%%%%%%%%%%%%%%%%%%%%%%%%%%%%%%
\section{Introduction}

\LaTeX{} provides a mechanism to structure a large document (such as a book)
into a main file and several child files (containing the chapters)
using the |\include| command.
This mechanism is beneficial for documents
which span hundreds of pages in order to
make the source file(s) more manageable.
Moreover, compilation can be restricted to
selected child files by means of the |\includeonly| command.
The latter feature can be used to reduce the compilation time while editing
(this was significantly more useful in the earlier days of \LaTeX{})
or to generate a smaller document which is easier to navigate.
Another application of |\includeonly| is to generate
documents consisting of selected parts of the complete document.

However, there are a few drawbacks of the plain |\include| mechanism:
\begin{itemize}
\item
The child files cannot be compiled on their own,
they can only be compiled via the main file.
A naive editing environment
(such as a text editor with an option
to have the current file processed by \LaTeX)
may require one to switch to the main file before compiling;
attempting to compile the child file produces errors.
\item
The main file must be modified (each time)
to adjust the |\includeonly| command
to the present needs. This easily leaves the main file in a messy state.
\item
The generated document will always carry the filename
of the main document. This is inconvenient if
several child files are to be compiled and
to be kept for distribution.
\end{itemize}

The present package provides a simple interface
to make child files individually compilable by \LaTeX{}.
Compiling a child file then has the same effect as compiling
the main file with an |\includeonly| command
to select the appropriate child.
Moreover the generated document will carry the name of the child
rather than the main file.
This resolves all three above issues.

This feature is meant to make the editing of books,
thesis documents and lecture notes somewhat more convenient.
However, the package can also be used efficiently for
composing a series of documents (such as exercise sheets)
which are typically distributed individually.
It then assists the author in generating the individual documents
(potentially in different versions)
as well as a document containing the collected series.
Another application is in developing style files
or other kinds of included material
where compilation of the style file could redirect
to a sample or test file.

%%%%%%%%%%%%%%%%%%%%%%%%%%%%%%%%%%%%%%%%%%%%%%%%%%%%%%%%%%%%%%%%%%%%%%%%%%%%%%%%
%%%%%%%%%%%%%%%%%%%%%%%%%%%%%%%%%%%%%%%%%%%%%%%%%%%%%%%%%%%%%%%%%%%%%%%%%%%%%%%%
\section{Usage}

First of all, the package \textsf{childdoc} is \emph{not} a standard
\LaTeXe{} |.sty| style file! Therefore it needs to be invoked in
a non-standard way.

%%%%%%%%%%%%%%%%%%%%%%%%%%%%%%%%%%%%%%%%%%%%%%%%%%%%%%%%%%%%%%%%%%%%%%%%%%%%%%%%
\subsection{Included Files}
\label{sec:include}

%%%%%%%%%%%%%%%%%%%%%%%%%%%%%%%%%%%%%%%%
\DescribeMacro{\childdocmain}
To use the package, add the commands
\begin{center}
\begin{tabular}{l}
|\input{childdoc.def}|\\
|\childdocmain{}|\\
\end{tabular}
\end{center}
at the very top of the main \LaTeX{} file,
in particular \emph{before} the |\documentclass| statement!
The argument of |\childdocmain| should be left empty
(but it must be present).

%%%%%%%%%%%%%%%%%%%%%%%%%%%%%%%%%%%%%%%%
\DescribeMacro{\childdocof}
Furthermore, add the commands
\begin{center}
\begin{tabular}{l}
|\input{childdoc.def}|\\
|\childdocof{|\textit{main}|}|\\
\end{tabular}
\end{center}
at the top of every child file \textit{child}
which is included by |\include{|\textit{child}|}|
from within the main file
(or at least for those files to be compiled individually).
The argument \textit{main} must be the filename of the main file.

There are a couple of
considerations in setting up the main and child documents:

%%%%%%%%%%%%%%%%%%%%%%%%%%%%%%%%%%%%%%%%
\paragraph{Restrictions.}

Please note the following restrictions:
\begin{itemize}
\item
|\childdocmain| must be called with one argument \textit{main}
to ensure compatibility with earlier version of the package.
It must either be empty (|\childdocmain{}|)
or precisely match the filename of the main file in which it is specified.
See \secref{sec:detection} for further information.
\item
The filename \textit{main} must be specified without the |.tex| extension.
\item
The filename \textit{main} is case sensitive
(even in case-insensitive file systems)
due to internal string comparison.
\item
The argument \textit{main} should be fully expanded, it cannot be a macro.
\item
Subdirectories and special characters should be avoided in filenames.
\item
The command |\childdocmain{|\textit{main}|}| must be followed by a whitespace.
It should not be followed immediately by another command
or by a comment mark `|%|'.
This is because the \TeX{} parser reads the token immediately following
the argument of |\childdocmain| and puts it
at the beginning of every child section;
however, a white\-space is ignored.
\end{itemize}

%%%%%%%%%%%%%%%%%%%%%%%%%%%%%%%%%%%%%%%%
\paragraph{Content of Main File.}

It is advisable to place all content in the child files included by |\include|.
Any output contained in the main file will appear in all child documents
unless suppressed manually;
it cannot be suppressed automatically by the |\includeonly| directive
and thus should normally be avoided.
A method to include some content in the main file
by means of conditional processing is described in \secref{sec:conditional}.

%%%%%%%%%%%%%%%%%%%%%%%%%%%%%%%%%%%%%%%%
\paragraph{Page Numbering.}

When only a part of the document is compiled,
the appropriate numbering of pages
(as well as other status parameters)
is determined from the |.aux| files.
The latter contain information from previous passes.
However this information needs to propagate through
all intermediate child documents.
Therefore the page numbering in child documents may well
be inconsistent until the complete document is compiled at least once.

A useful (if unconventional) way to always ensure a consistent
page numbering is to restart the numbering in each child document
and denote the pages by `\textit{child}|.|\textit{page}'
where \textit{child} represents the chapter/section number of the child file.
This can be achieved by the command
|\numberwithin{page}{|\textit{child}|}|
of the \textsf{amsmath} package
where \textit{child} can be |chapter| or |section|
depending on the chosen structuring.
Alternatively, one can modify the macro |\thepage| appropriately
and reset the counter |page| at the start of each child file.

%%%%%%%%%%%%%%%%%%%%%%%%%%%%%%%%%%%%%%%%%%%%%%%%%%%%%%%%%%%%%%%%%%%%%%%%%%%%%%%%
\subsection{Conditional Processing}
\label{sec:conditional}

The package provides a mechanism to compile different versions
of a document. To customise the versions further some conditional processing
can come in handy to distinguish which version is being compiled.
The package provides two macros to describe the compilation context:

%%%%%%%%%%%%%%%%%%%%%%%%%%%%%%%%%%%%%%%%
\DescribeMacro{\ifchilddoc}
The conditional |\ifchilddoc| distinguishes between the compilation of
child documents and the main document:
%
\begin{center}
|\ifchilddoc |\textit{child-code}| |[|\||else |\textit{main-code}]| \||fi|
\end{center}

%%%%%%%%%%%%%%%%%%%%%%%%%%%%%%%%%%%%%%%%
\DescribeMacro{\childdocname}
\DescribeMacro{\childdocjob}
The macro |\childdocname| contains the filename (without extension)
of the main or child file being processed.
Note that |\childdocjob| will always contain the name of the main file.

%%%%%%%%%%%%%%%%%%%%%%%%%%%%%%%%%%%%%%%%
\paragraph{Title Page.}

Conditional processing can be used to include a title or banner page
in the main document when proper precautions are taken.
Importantly, the code in the main file should ensure that the page counter
(as well as other status parameters which are stored in the |.aux| files)
takes the same value after the conditional processing.
Otherwise the page numbers may take divergent values
depending on which part is compiled.

For example, a title page could be declared by:
%
\begin{center}
\begin{tabular}{l}
|\ifchilddoc\||else|\\
|\addtocounter{page}{-1}|\\
\textit{code for title page}\\
|\newpage|\\
|\||fi|
\end{tabular}
\end{center}
%
A banner page for the child documents can be generated by:
%
\begin{center}
\begin{tabular}{l}
|\ifchilddoc|\\
|\addtocounter{page}{-1}|\\
\textit{code for banner page}\\
|\newpage|\\
|\||fi|
\end{tabular}
\end{center}
%
Here one could write a message such as:
\begin{center}
|This is the part \childdocname{} of \childdocjob{}.|
\end{center}

%%%%%%%%%%%%%%%%%%%%%%%%%%%%%%%%%%%%%%%%%%%%%%%%%%%%%%%%%%%%%%%%%%%%%%%%%%%%%%%%
\subsection{Flags}
\label{sec:flags}

The package makes it easy to generate different versions
of the main or child documents.
To this end compilation flags can be defined
and assigned different default values.
They will be particularly useful in conjunction
with the forwarding mechanism described in \secref{sec:forward}.

For example, it may be useful to have a flag |\version|
which can be set to |draft| or |final|.
The document source will contain some conditional code
depending on the value of |\version|.
Suppose further, the flag should default to |final| for the main file
and to |draft| for child files
which is a natural assignment for editing the document.
This is achieved by placing the following code
in the preamble of the main document
(below the |\childdocmain| directive):
%
\begin{center}
\begin{tabular}{l}
|\ifchilddoc|\\
|\providecommand{\version}{draft}|\\
|\||else|\\
|\providecommand{\version}{final}|\\
|\||fi|
\end{tabular}
\end{center}
%
The definition by |\providecommand| makes sure
that previous definitions are not overwritten.
Further statements |\providecommand{\version}{...}|
can thus be added before the above code to override it.

For the main file, one might add a line
(between |\childdocmain| and the above block)
%
\begin{center}
|%\ifchilddoc\||else\providecommand{\version}{draft}\||fi|
\end{center}
%
which can be uncommented to produce a draft version.
Likewise one can add a line to the very top of a child file
(above the |\childdocof{|\textit{main}|}| directive)
%
\begin{center}
|%\providecommand{\version}{final}|
\end{center}
%
which can be uncommented to produce the final version of this child document.

%%%%%%%%%%%%%%%%%%%%%%%%%%%%%%%%%%%%%%%%%%%%%%%%%%%%%%%%%%%%%%%%%%%%%%%%%%%%%%%%
\subsection{Forwarding}
\label{sec:forward}

Different versions of the main or child documents
using compilation flags as described in \secref{sec:flags}
can be (permanently) stored in different files
for convenient compilation, viewing and distribution.
To this end, the package defines a command
to pass on compilation to a different file:

%%%%%%%%%%%%%%%%%%%%%%%%%%%%%%%%%%%%%%%%
\DescribeMacro{\childdocforward}
The command |\childdocforward| redirects processing to
another source file:
%
\begin{center}
\begin{tabular}{l}
|\input{childdoc.def}|\\
|\childdocforward[|\textit{main}|]{|\textit{dest}|}|\\
\end{tabular}
\end{center}
%
The argument \textit{dest} is the destination file
(without extension).
It should be the main file or one of the child files.
Note that further \textsf{childdoc} directives
such as |\childdocof| and |\childdocforward|
in the indicated file will be processed in this form.
The optional argument \textit{main}
passes on directly to the main file \textit{main}
while pretending to compile the child \textit{dest}.
This form behaves as if \textit{dest}
issues |\childdocof{|\textit{main}|}| right away,
and no further \textsf{childdoc} directives will be processed.

%%%%%%%%%%%%%%%%%%%%%%%%%%%%%%%%%%%%%%%%
\DescribeMacro{\...prefix}
In the alternative form |\childdocforwardprefix|,
%
\begin{center}
\begin{tabular}{l}
|\input{childdoc.def}|\\
|\childdocforwardprefix[|\textit{main}|]{|\textit{prefix}|}{|\textit{dest}|}|
\end{tabular}
\end{center}
%
the destination file is determined by a pattern
depending on the current file:
To make this work, the current file must be called
`{\textit{prefix}\hspace{0.2em}\textit{suffix}}'
with \textit{prefix} matching precisely the argument.
Processing is then passed on to the file
`{\textit{dest}\hspace{0.2em}\textit{suffix}}'.
Surely, the same effect is achieved by
directly specifying the
argument `{\textit{dest}\hspace{0.2em}\textit{suffix}}'
in the first form.
However, that requires to set up a different file
for each child. With the alternative form of the command
all these files can have exactly the same content
which simplifies setting them up and maintaining them.

For example, the following file |draft.tex|
with a compilation flag |\version| as described in \secref{sec:flags}
compiles the main document as a draft:
%
\begin{center}
\begin{tabular}{l}
|\def\version{draft}|\\
|\input{childdoc.def}|\\
|\childdocforward{|\textit{main}|}|
\end{tabular}
\end{center}
%
Likewise, the following files |final|\textit{nn}|.tex|
compile the final version of the child document
|child|\textit{nn}|.tex|:
%
\begin{center}
\begin{tabular}{l}
|\def\version{final}|\\
|\input{childdoc.def}|\\
|\childdocforwardprefix{final}{child}|
\end{tabular}
\end{center}
%

Note that when several versions of a main file and/or of each child file
are to be generated, it may be convenient to set up a |Makefile| or
shell script to automatise the process.

%%%%%%%%%%%%%%%%%%%%%%%%%%%%%%%%%%%%%%%%%%%%%%%%%%%%%%%%%%%%%%%%%%%%%%%%%%%%%%%%
\subsection{Command Line Processing}
\label{sec:commandline}

The effect of redirection files can also be achieved by invoking
the \LaTeX{} compiler with a more elaborate command line.
Most conveniently this should be done as part
of a shell script or a |Makefile|.

When using \textsf{childdoc} in the main file, the following
command lines effectively perform a redirection
(note that depending on the shell being used,
backslashes may have to be doubled: `|\|' $\to$ `|\\|'):
%
\begin{center}
|... -jobname "|\textit{target}|" |\\|"|[\textit{flags}]%
|\input{childdoc.def}\childdocforward[|\textit{main}|]{|\textit{dest}|}"|
\end{center}
%
Here \textit{target} is the name of the output file,
\textit{main} is the name of the main file
and \textit{dest} is the name of the main or child file to be processed
(all filenames without extensions).
The optional argument \textit{main} can be omitted
if \textit{main} matches \textit{dest}.
Optionally, compilation \textit{flags} can be defined via |\def| commands.
This command line makes the \TeX{} engine believe
it is compiling the file \textit{target}
whose content is specified as the latter parameter.
The provided code then forwards the processing to
\textit{main} or \textit{dest} as described in \secref{sec:forward}.

%%%%%%%%%%%%%%%%%%%%%%%%%%%%%%%%%%%%%%%%%%%%%%%%%%%%%%%%%%%%%%%%%%%%%%%%%%%%%%%%
\subsection{Include by Input}
\label{sec:input}

Including child documents by |\include| has some restrictions by design.
Most notably, the content of a child document always occupies
its own set of pages; pages cannot be shared between child documents.
Usually, this behaviour makes perfect sense
because each child document contain an essential part of the document.
However, in some situations it may be desirable to compose
a document from a collection of parts
without having mandatory page breaks between then.
For this case, the package
provides a mechanism to include parts
by |\input| which can also be processed individually.
However, by construction this mechanism
requires manual handling of the content to be output.

%%%%%%%%%%%%%%%%%%%%%%%%%%%%%%%%%%%%%%%%
\DescribeMacro{\ifchilddocmanual}
The main file should be prepared as usual, see \secref{sec:include}.
However, the document body must make a distinction
between processing of an individual part and of the main document, e.g.:
%
\begin{center}
\begin{tabular}{l}
|\ifchilddocmanual|\\
|\input{\childdocname}|\\
|\||else|\\
\textit{document body with }|\input{|\textit{part}|}|\\
|\||fi|
\end{tabular}
\end{center}
%
The conditional |\ifchilddocmanual| is true whenever
a part to be included by |\input| is being compiled,
and the name of the part is stored in |\childdocname|.

%%%%%%%%%%%%%%%%%%%%%%%%%%%%%%%%%%%%%%%%
\DescribeMacro{\childdocby}
Each part to be included by |\input| should start with:
%
\begin{center}
\begin{tabular}{l}
|\input{childdoc.def}|\\
|\childdocby{|\textit{main}|}|\\
\end{tabular}
\end{center}
%
The directive |\childdocby| is similar to |\childdocof|
described in \secref{sec:include},
but the subsequent selection of content must be done manually.
To that end, both |\ifchilddoc| and |\ifchilddocmanual|
will be true upon processing of a part,
and the name of the part is stored in |\childdocname|.
Note that |\jobname| will be set to the filename of the current part
so that each part receives an individual |.aux| file
that does not interfere with the |.aux| file(s) of the main document.
This behaviour can be altered by the alternative form
|\childdocby[*]{|\textit{main}|}| (with a non-empty optional argument)
which uses the |.aux| file of the main document
by setting |\jobname| to \textit{main}.

%%%%%%%%%%%%%%%%%%%%%%%%%%%%%%%%%%%%%%%%%%%%%%%%%%%%%%%%%%%%%%%%%%%%%%%%%%%%%%%%
\subsection{Driver Development}
\label{sec:driver}

The \textsf{childdoc} mechanism can also be use for the development
of definition files such as \LaTeX{} styles or classes.
This case differs from the above setup with multiple parts
included by |\include| in that no |\includeonly| should be invoked.
This can be achieved by starting the include file
(before |\ProvidesPackage|) with:
%
\begin{center}
\begin{tabular}{l}
|\input{childdoc.def}|\\
|\childdocforward{|\textit{main}|}|\\
\end{tabular}
\end{center}
%
or alternatively with:
%
\begin{center}
\begin{tabular}{l}
|\input{childdoc.def}|\\
|\childdocby{|\textit{main}|}|\\
\end{tabular}
\end{center}
%
Both forms have slightly different effects as described above.
The main file is prepared as usual, see \secref{sec:include}.

%%%%%%%%%%%%%%%%%%%%%%%%%%%%%%%%%%%%%%%%%%%%%%%%%%%%%%%%%%%%%%%%%%%%%%%%%%%%%%%%
\subsection{Legacy Detection}
\label{sec:detection}

The directive |\childdocmain| in the main file can detect
whether the complete document or merely a child is to be compiled
even without using the directive |\childdocof|.
This method is deprecated because it is less robust
and there is no compelling reason to use it;
it is merely provided for backward compatibility
and it may be removed in future versions.

If the detection mechanism is to be used,
it is mandatory to correctly specify
the filename of the main file as the argument of |\childdocmain|:
%
\begin{center}
\begin{tabular}{l}
|\input{childdoc.def}|\\
|\childdocmain{|\textit{main}|}|\\
\end{tabular}
\end{center}
%
If |\jobname| does not match the argument \textit{main} of |\childdocmain|,
it is assumed that |\jobname| points to the child file to be compiled.
When using |\childdocmain| with the main file specified as argument,
it suffices to start a child file
with just |\input{|\textit{main}|}|
without loading of the package and using |\childdocof|.
If instead all processing is done
with the appropriate \textsf{childdoc} directives,
the argument of \textit{main} of |\childdocmain| can be empty.

An alternative version of the command line processing described
in \secref{sec:commandline} using the detection mechanism reads:
%
\begin{center}
|... -jobname "|\textit{target}|" "|[\textit{flags}]%
[|\def\jobname{|\textit{dest}|}|]|\input{|\textit{main}|}"|
\end{center}

%%%%%%%%%%%%%%%%%%%%%%%%%%%%%%%%%%%%%%%%%%%%%%%%%%%%%%%%%%%%%%%%%%%%%%%%%%%%%%%%
\subsection{Manual Code}
\label{sec:manual}

In case one cannot be certain whether the definitions file |childdoc.def|
is installed on the target \TeX{} distribution
and one prefers not to ship it,
it is conceivable to paste a few relevant commands into the sources.

To that end, drop all statements |\input{childdoc.def}|
and perform the replacements as outlined below.
Instead of |\childdocmain{|\textit{main}|}| add the following code
to the top of the main file:
%
\begin{center}
\begin{tabular}{l}
|\||ifdefined\childdocname\endinput\||fi\newif\ifchilddoc|\\
|\edef\childdocname{\scantokens\expandafter{\jobname\noexpand}}|\\
|\def\childdocmain{|\textit{main}|}\||ifx\childdocmain\childdocname\||else|\\
|\childdoctrue\includeonly{\childdocname}\let\jobname\childdocmain\||fi|\\
\end{tabular}
\end{center}
%
Instead of |\childdocof{|\textit{main}|}| just include the main file
at the top of each child file:
%
\begin{center}
|\input{|\textit{main}|}|
\end{center}
%
A simple redirection |\childdocforward{|\textit{dest}|}| is achieved by:
%
\begin{center}
|\def\jobname{|\textit{dest}|}\input{\jobname}|
\end{center}
%
The redirection with prefix
|\childdocforwardprefix[|\textit{prefix}|]{|\textit{dest}|}|
is accomplished by:
%
\begin{center}
\begin{tabular}{l}
|{\edef\jobname{\scantokens\expandafter{\jobname\noexpand}}|\\
|\def\redirectjob |\textit{prefix}|#1~~~{\gdef\jobname{|\textit{dest}|#1}}|\\
|\expandafter\redirectjob\jobname~~~}\input{\jobname}|
\end{tabular}
\end{center}

In an alternative approach,
child documents can be compiled by a specific command line
without additional code or specific definitions:
%
\begin{center}
|... -jobname "|\textit{target}|" "|[\textit{flags}]%
|\includeonly{|\textit{dest}|}\input{|\textit{main}|}"|
\end{center}
%

%%%%%%%%%%%%%%%%%%%%%%%%%%%%%%%%%%%%%%%%%%%%%%%%%%%%%%%%%%%%%%%%%%%%%%%%%%%%%%%%
%%%%%%%%%%%%%%%%%%%%%%%%%%%%%%%%%%%%%%%%%%%%%%%%%%%%%%%%%%%%%%%%%%%%%%%%%%%%%%%%
\section{Information}

%%%%%%%%%%%%%%%%%%%%%%%%%%%%%%%%%%%%%%%%%%%%%%%%%%%%%%%%%%%%%%%%%%%%%%%%%%%%%%%%
\subsection{Copyright}

Copyright \copyright{} 2017--2018 Niklas Beisert

This work may be distributed and/or modified under the
conditions of the \LaTeX{} Project Public License, either version 1.3
of this license or (at your option) any later version.
The latest version of this license is in
  \url{http://www.latex-project.org/lppl.txt}
and version 1.3 or later is part of all distributions of \LaTeX{}
version 2005/12/01 or later.

This work has the LPPL maintenance status `maintained'.

The Current Maintainer of this work is Niklas Beisert.

This work consists of the files |README.txt|, |childdoc.ins| and |childdoc.dtx|
as well as the derived files |childdoc.def|, |cdocsamp.tex|
with |cdocsch1.tex|, |cdocsch2.tex|, |cdocspt3.tex|, |cdocspt4.tex|,
|cdocsdrf.tex|, |cdocsfn1.tex|, |cdocsfn2.tex|
as well as |childdoc.pdf|.

%%%%%%%%%%%%%%%%%%%%%%%%%%%%%%%%%%%%%%%%%%%%%%%%%%%%%%%%%%%%%%%%%%%%%%%%%%%%%%%%
\subsection{Files and Installation}

The package consists of the files:
%
\begin{center}
\begin{tabular}{ll}
    |README.txt|   & readme file \\
    |childdoc.ins| & installation file \\
    |childdoc.dtx| & source file \\
    |childdoc.def| & definition file \\
    |cdocsamp.tex| & sample main file \\
    |cdocsch1.tex| & sample include file \\
    |cdocsch2.tex| & sample include file \\
    |cdocspt3.tex| & sample part file \\
    |cdocspt4.tex| & sample part file \\
    |cdocsdrf.tex| & sample redirection file \\
    |cdocsfn1.tex| & sample redirection file \\
    |cdocsfn2.tex| & sample redirection file \\
    |childdoc.pdf| & manual
\end{tabular}
\end{center}
%
The distribution consists of the files
|README.txt|, |childdoc.ins| and |childdoc.dtx|.
%
\begin{itemize}
\item
Run (pdf)\LaTeX{} on |childdoc.dtx|
to compile the manual |childdoc.pdf| (this file).
\item
Run \LaTeX{} on |childdoc.ins| to create the definitions file |childdoc.def|
and the sample |cdocsamp.tex| with include files
|cdocsch1.tex|, |cdocsch2.tex|, |cdocspt3.tex|, |cdocspt4.tex|,
|cdocsdrf.tex|, |cdocsfn1.tex|, |cdocsfn2.tex|.
Then copy the file |childdoc.def| to an appropriate directory of your \LaTeX{}
distribution, e.g.\ \textit{texmf-root}|/tex/latex/childdoc|.
\end{itemize}

%%%%%%%%%%%%%%%%%%%%%%%%%%%%%%%%%%%%%%%%%%%%%%%%%%%%%%%%%%%%%%%%%%%%%%%%%%%%%%%%
\subsection{Related CTAN Packages}

There are several other packages which offer a similar functionality:
%
\begin{itemize}
\item
The packages
\href{http://ctan.org/pkg/docmute}{\textsf{docmute}},
\href{http://ctan.org/pkg/includex}{\textsf{includex}} and
\href{http://ctan.org/pkg/standalone}{\textsf{standalone}}
provide commands to include only the document body of
a child file thus allowing both files to be compiled individually.
\item
The packages \href{http://ctan.org/pkg/subdocs}{\textsf{subdocs}}
and \href{http://ctan.org/pkg/subfiles}{\textsf{subfiles}}
provide structures in which the main and child documents can be
encapsulated and allowing them to be compiled individually.
The inclusion mechanism is different from the conventional |\include|.
\item
The package \href{http://ctan.org/pkg/combine}{\textsf{combine}}
is an elaborate solution to combine several documents into one.
\end{itemize}
%
See also the CTAN topic \href{http://ctan.org/topic/subdocs}{\textsf{subdocs}}
for further related packages.
The present package differs from the above solutions in that
a document structure constructed with the conventional |\include| mechanism
just needs two extra commands at the top of every file
such that all constituent files can be compiled individually.

%%%%%%%%%%%%%%%%%%%%%%%%%%%%%%%%%%%%%%%%%%%%%%%%%%%%%%%%%%%%%%%%%%%%%%%%%%%%%%%%
%\subsection{Feature Suggestions}
%
%The following is a list of features which may be useful for future
%versions of this package:
%%
%\begin{itemize}
%\item
%\ldots
%\end{itemize}

%%%%%%%%%%%%%%%%%%%%%%%%%%%%%%%%%%%%%%%%%%%%%%%%%%%%%%%%%%%%%%%%%%%%%%%%%%%%%%%%
\subsection{Revision History}

%%%%%%%%%%%%%%%%%%%%%%%%%%%%%%%%%%%%%%%%
\paragraph{v2.0:} 2018/12/30

\begin{itemize}
\item
immediate forward processing
\item
added |\childdocby| mechanism
\item
manual restructured
\end{itemize}

%%%%%%%%%%%%%%%%%%%%%%%%%%%%%%%%%%%%%%%%
\paragraph{v1.6:} 2018/01/17

\begin{itemize}
\item
application for development of include files
\item
corrections to manual
\end{itemize}

%%%%%%%%%%%%%%%%%%%%%%%%%%%%%%%%%%%%%%%%
\paragraph{v1.5:} 2017/05/21

\begin{itemize}
\item
more complete structuring introduced
\item
|\childdocof| introduced
\item
|\childdoc| renamed to |\childdocmain|
\item
|\childredirect| renamed to |\childdocforward| and |\childdocforwardprefix|
and functionality expanded
\end{itemize}

%%%%%%%%%%%%%%%%%%%%%%%%%%%%%%%%%%%%%%%%
\paragraph{v1.0:} 2017/04/27

\begin{itemize}
\item
manual and install package
\item
first version published on CTAN
\end{itemize}

%%%%%%%%%%%%%%%%%%%%%%%%%%%%%%%%%%%%%%%%
\paragraph{v0.6:} 2017/04/26

\begin{itemize}
\item
redirection mechanism added
\end{itemize}

%%%%%%%%%%%%%%%%%%%%%%%%%%%%%%%%%%%%%%%%
\paragraph{v0.5:} 2017/04/26

\begin{itemize}
\item
functionality in definition file
\end{itemize}


%%%%%%%%%%%%%%%%%%%%%%%%%%%%%%%%%%%%%%%%%%%%%%%%%%%%%%%%%%%%%%%%%%%%%%%%%%%%%%%%
%%%%%%%%%%%%%%%%%%%%%%%%%%%%%%%%%%%%%%%%%%%%%%%%%%%%%%%%%%%%%%%%%%%%%%%%%%%%%%%%
%%%%%%%%%%%%%%%%%%%%%%%%%%%%%%%%%%%%%%%%%%%%%%%%%%%%%%%%%%%%%%%%%%%%%%%%%%%%%%%%
\appendix

\settowidth\MacroIndent{\rmfamily\scriptsize 000\ }

 \DocInput{childdoc.dtx}

\end{document}
%</driver>
% \fi
%
% %%%%%%%%%%%%%%%%%%%%%%%%%%%%%%%%%%%%%%%%%%%%%%%%%%%%%%%%%%%%%%%%%%%%%%%%%%%%%%
% %%%%%%%%%%%%%%%%%%%%%%%%%%%%%%%%%%%%%%%%%%%%%%%%%%%%%%%%%%%%%%%%%%%%%%%%%%%%%%
% \section{Sample}
%\iffalse
%<*samplemain>
%\fi
%
% The following presents a sample document
% with two chapters, two parts, a title page,
% a compile flag as well as three forwarding files to set the flag.
% It consists of eight |.tex| files:
% \begin{center}
% \begin{tabular}{ll}
% |cdocsamp.tex|&main file\\
% |cdocsch1.tex|&include file for chapter 1\\
% |cdocsch2.tex|&include file for chapter 2\\
% |cdocspt3.tex|&include file for part 3\\
% |cdocspt4.tex|&include file for part 4\\
% |cdocsdrf.tex|&forwarding file for main file in draft mode\\
% |cdocsfi1.tex|&forwarding file for final version of chapter 1\\
% |cdocsfi2.tex|&forwarding file for final version of chapter 2\\
% \end{tabular}
% \end{center}
% Each of the eight files can be compiled directly by the \LaTeX{} compiler.
%
% %%%%%%%%%%%%%%%%%%%%%%%%%%%%%%%%%%%%%%
% \paragraph{Main File.}
%
% The main file is called |cdocsamp.tex|.
%
% Load the \textsf{childdoc} definitions and
% declare the filename for the main document:
%    \begin{macrocode}
\input{childdoc.def}
\childdocmain{}
%    \end{macrocode}

% Optional override for |\version| flag:
%    \begin{macrocode}
%%\ifchilddoc\else\providecommand{\version}{draft}\fi
%    \end{macrocode}

% Define the default values for the |\version| flag
% (|final| for the main file and |draft| for childs):
%    \begin{macrocode}
\ifchilddoc
\providecommand{\version}{draft}
\else
\providecommand{\version}{final}
\fi
%    \end{macrocode}

% Load the standard document class:
%    \begin{macrocode}
\documentclass[12pt]{article}
%    \end{macrocode}

% Start the document body:
%    \begin{macrocode}
\begin{document}
%    \end{macrocode}

% Declare a title page.
% Print title, part of document being processed and version flag:
%    \begin{macrocode}
\addtocounter{page}{-1}
\begin{center}
{\LARGE\bfseries{}childdoc example\par}
\vspace{1cm}
\ifchilddoc
\ifchilddocmanual part\else chapter\fi:
`\childdocname' of `\childdocjob'\par
\else
main document: `\childdocjob'\par
\fi
version: \version\par
\end{center}
\newpage
%    \end{macrocode}

% Manually include selected file,
% otherwise process as usual:
%    \begin{macrocode}
\ifchilddocmanual
\section*{part `\childdocname'}
\input{\childdocname}
\else
%    \end{macrocode}

% Include the two chapters:
%    \begin{macrocode}
\include{cdocsch1}
\include{cdocsch2}
%    \end{macrocode}

% Include the two parts unless only chapters should be displayed:
%    \begin{macrocode}
\ifchilddoc\else
\section{part three}
\input{cdocspt3}
\section{part four}
\input{cdocspt4}
\fi
%    \end{macrocode}

% Process as usual until here:
%    \begin{macrocode}
\fi
%    \end{macrocode}

% End of document body:
%    \begin{macrocode}
\end{document}
%    \end{macrocode}
%\iffalse
%</samplemain>
%\fi
%
% %%%%%%%%%%%%%%%%%%%%%%%%%%%%%%%%%%%%%%
% \paragraph{Chapter Include Files.}
%
% The include files are called |cdocsch1.tex| and |cdocsch2.tex|.
%
%\iffalse
%<*samplechap1|samplechap2>
%\fi

% Optional override for |\version| flag:
%    \begin{macrocode}
%%\providecommand{\version}{final}
%    \end{macrocode}

% Include the main document:
%    \begin{macrocode}
\input{childdoc.def}
\childdocof{cdocsamp}
%    \end{macrocode}

%\iffalse
%</samplechap1|samplechap2>
%\fi
%
%\iffalse
%<*samplechap1>
%\fi
% Some text for chapter 1:
%    \begin{macrocode}
\section{one}
some text in chapter one
%    \end{macrocode}

%\iffalse
%</samplechap1>
%\fi
% Some text for chapter 2:
%\iffalse
%<*samplechap2>
%\fi
%    \begin{macrocode}
\section{two}
more text in chapter two
%    \end{macrocode}

%\iffalse
%</samplechap2>
%\fi
%
% %%%%%%%%%%%%%%%%%%%%%%%%%%%%%%%%%%%%%%
% \paragraph{Part Include Files.}
%
% The include files are called |cdocspt3.tex| and |cdocspt4.tex|.
%
%\iffalse
%<*samplepart3|samplepart4>
%\fi

% Optional override for |\version| flag:
%    \begin{macrocode}
%%\providecommand{\version}{final}
%    \end{macrocode}

% Include the main document:
%    \begin{macrocode}
\input{childdoc.def}
\childdocby{cdocsamp}
%    \end{macrocode}

%\iffalse
%</samplepart3|samplepart4>
%\fi
%
%\iffalse
%<*samplepart3>
%\fi
% Some text for part 3:
%    \begin{macrocode}
some text in part three
%    \end{macrocode}

%\iffalse
%</samplepart3>
%\fi
% Some text for part 4:
%\iffalse
%<*samplepart4>
%\fi
%    \begin{macrocode}
more text in part four
%    \end{macrocode}

%\iffalse
%</samplepart4>
%\fi
%
% %%%%%%%%%%%%%%%%%%%%%%%%%%%%%%%%%%%%%%
% \paragraph{Forwarding for a Complete Draft.}
%
% The following forwarding file |cdocsdrf.tex|
% compiles the main document in draft mode:
%\iffalse
%<*sampledraft>
%\fi
%    \begin{macrocode}
\def\version{draft}
\input{childdoc.def}
\childdocforward{cdocsamp}
%    \end{macrocode}

%\iffalse
%</sampledraft>
%\fi
%
% %%%%%%%%%%%%%%%%%%%%%%%%%%%%%%%%%%%%%%
% \paragraph{Forwarding for Final Version of the Chapters.}
%
% The following forwarding files |cdocsfn1.tex| and |cdocsfn2.tex|
% (with identical content)
% compile the final versions of the child documents
% |cdocsch1.tex| and |cdocsch2.tex|, respectively:
%\iffalse
%<*samplefinal>
%\fi
%    \begin{macrocode}
\def\version{final}
\input{childdoc.def}
\childdocforwardprefix[cdocsamp]{cdocsfn}{cdocsch}
%    \end{macrocode}

%\iffalse
%</samplefinal>
%\fi
%
% %%%%%%%%%%%%%%%%%%%%%%%%%%%%%%%%%%%%%%
% \paragraph{Command Line Processing.}
%
% The following three command lines generate the output files
% |cdocscld|, |cdocscl1| and |cdocscl2|
% which should be identical to
% |cdocsdrf|, |cdocsch1| and |cdocsfn2|, respectively:
% \begin{center}
% \begin{tabular}{l}
% |latex -jobname cdocscld \|\\
% |  "\def\version{draft}\input{childdoc.def}\childdocforward{cdocsamp}"|\\
% |latex -jobname cdocscl1 \|\\
% |  "\input{childdoc.def}\childdocforward[cdocsamp]{cdocsch1}"|\\
% |latex -jobname cdocscl2 \|\\
% |  "\def\version{final}\input{childdoc.def}\childdocforward{cdocsch2}"|
% \end{tabular}
% \end{center}
% Note that the trailing backslash on each first line
% merely continues the input to the second line
% (for convenient cut ant paste).
% Furthermore, the command |latex| can be replaced by any
% of its alternative versions such as |pdflatex|.
%
% %%%%%%%%%%%%%%%%%%%%%%%%%%%%%%%%%%%%%%%%%%%%%%%%%%%%%%%%%%%%%%%%%%%%%%%%%%%%%%
% %%%%%%%%%%%%%%%%%%%%%%%%%%%%%%%%%%%%%%%%%%%%%%%%%%%%%%%%%%%%%%%%%%%%%%%%%%%%%%
% \section{Implementation}
%\iffalse
%<*package>
%\fi
%
% This section describes the definitions file |childdoc.def|.

% The definitions cannot be loaded using |\usepackage| or |\RequirePackage|
% which has a mechanism to prevent loading a style file more than once.
% When loading the definitions by means of |\input|
% multiple instances have to be prevented manually:
%\iffalse
%This code needs to be before the `\ProvidesFile' directive
%which is defined at the beginning of this file.
%Therefore it is also placed there and commented out here.
%</package>
%<*discard>
%\fi
%    \begin{macrocode}
\ifdefined\childdocmain\endinput\fi
%    \end{macrocode}
%\iffalse
%</discard>
%<*package>
%\fi
%
% \macro{\ifchilddoc}
% \macro{\ifchilddocmanual}
% The conditional |\ifchilddoc| tells whether a
% child (true) or main (false) document is being compiled.
% The conditional |\ifchilddocmanual| tells whether
% the |\includeonly| mechanism is used (false) or
% the selection of child files must be performed manually (true).
% The definitions initialise to false:
%    \begin{macrocode}
\newif\ifchilddoc
\newif\ifchilddocmanual
%    \end{macrocode}

% \macro{\childdocname}
% \macro{\childdocjob}
% The macro |\childdocname| stores the name of the main document
% to be compiled. The macro |\childdocjob| stores the name of
% the document on which the \LaTeX{} compiler was originally invoked.
% The content of |\jobname| cannot be compared
% to filenames specified in the source due to different catcodes.
% The following code rescans |\jobname|, stores the result
% in |\childdocname| and saves a copy in |\childdocjob|:
%    \begin{macrocode}
\edef\childdocname{\scantokens\expandafter{\jobname\noexpand}}
\let\childdocjob\childdocname
%    \end{macrocode}

% \macro{\childdocdisable}
% The macro |\childdocdisable| prevents the main file
% from being processed more than once.
% At this stage, the main document command |\childdocmain|
% is assumed to be called once again where it should do nothing.
% Any subsequent call to it should prevent
% a secondary processing of the main document
% It overwrites the forwarding commands
% |\childdocof| and |\childdocforward|
% with empty macros to prevent further inclusions of the main document:
%    \begin{macrocode}
\newcommand{\childdocdisable}
{
  \renewcommand{\childdocmain}[1]{\renewcommand{\childdocmain}[1]{\endinput}}
  \renewcommand{\childdocof}[1]{}
  \renewcommand{\childdocby}[2][]{}
  \renewcommand{\childdocforward}[2][]{}
  \renewcommand{\childdocdisable}{}
}
%    \end{macrocode}

% \macro{\childdocmain}
% The macro |\childdocmain| is to be called at the top of the main file
% with nothing or the main filename (without extension) as argument.
% First, it breaks loops.
% If the argument is not empty and does not match |\childdocname|
% (which is set by the first inclusion of |childdoc.def|),
% |\ifchilddoc| is set to true, |\includeonly| is applied to the child file
% and |\jobname| is set to the main file
% (for proper handling of |.aux| files):
%    \begin{macrocode}
\newcommand{\childdocmain}[1]
{
  \childdocdisable\childdocmain{}
  \if?#1?\else
    \begingroup
      \def\childdoctmp{#1}
      \ifx\childdoctmp\childdocname
        \def\childdoctmp{}
      \else
        \def\childdoctmp
        {
          \childdoctrue
          \includeonly{\childdocname}
          \def\childdocjob{#1}
          \def\jobname{#1}
        }
      \fi
      \expandafter
    \endgroup
    \childdoctmp
  \fi
}
%    \end{macrocode}

% \macro{\childdocof}
% The command |\childdocof| redirects
% compilation to the main file |#1|.
%    \begin{macrocode}
\newcommand{\childdocof}[1]
{
  \childdocdisable
  \childdoctrue
  \includeonly{\childdocname}
  \def\jobname{#1}
  \def\childdocjob{#1}
  \input{#1}
}
%    \end{macrocode}

% \macro{\childdocby}
% The command |\childdocby| ....
%    \begin{macrocode}
\newcommand{\childdocby}[2][]
{
  \childdocdisable
  \childdoctrue
  \childdocmanualtrue
  \if?#1?\else
    \def\jobname{#2}
  \fi
  \def\childdocjob{#2}
  \input{#2}
  \endinput
}
%    \end{macrocode}

% \macro{\childdocforward}
% The command |\childdocforward| redirects
% compilation to the main file or
% (if the optional argument is given) a child file.
% Parameters are set as if the main file
% or a child file starting with |\childdocof| was compiled.
% Then compilation is handed over to the main file:
%    \begin{macrocode}
\newcommand{\childdocforward}[2][]
{
  \begingroup
    \if?#1?
      \def\childdoctmp
      {
        \def\childdocname{#2}
        \def\childdocjob{#2}
        \def\jobname{#2}
        \input{#2}
        \endinput
      }
    \else
      \def\childdoctmp
      {
        \childdocdisable
        \def\childdocname{#2}
        \childdoctrue
        \includeonly{#2}
        \def\childdocjob{#1}
        \def\jobname{#1}
        \input{#1}
        \endinput
      }
    \fi
    \expandafter
  \endgroup
  \childdoctmp
}
%    \end{macrocode}

% \macro{\childdocforwardprefix}
% The command |\childdocforwardprefix| redirects
% compilation to the main or a child file by means of a pattern.
% The prefix |#1| in the current filename is replaced by |#2|
% and the suffix of the current filename is kept
% (it is assumed that the filename does not contain the substring `|~~~|'
% which is used as a delimiter).
% Compilation is handed over to the new file by |\childdocforward|:
%    \begin{macrocode}
\newcommand{\childdocforwardprefix}[3][]
{
  \begingroup
    \def\childdocextract #2##1~~~{\def\childdoctmp{\childdocforward[#1]{#3##1}}}
    \expandafter\childdocextract\childdocname~~~
    \expandafter
  \endgroup
  \childdoctmp
}
%    \end{macrocode}

% \macro{\childdoc}
% The deprecated macro |\childdoc| is a legacy version of |\childdocmain|:
%    \begin{macrocode}
\newcommand{\childdoc}{\childdocmain}
%    \end{macrocode}

% \macro{\childdocredirect}
% The deprecated macro |\childdocredirect| is a legacy version
% of |\childdocforward| and |\childdocforwardprefix|:
%    \begin{macrocode}
\newcommand{\childdocredirect}[2][]
{
  \begingroup
    \if?#1?
      \def\childdoctmp{\childdocforward{#2}}
    \else
      \def\childdoctmp{\childdocforwardprefix{#1}{#2}}
    \fi
    \expandafter
  \endgroup
  \childdoctmp
}
%    \end{macrocode}

%\iffalse
%</package>
%\fi
%
\endinput

\childdocof{cdocsamp}
%    \end{macrocode}

%\iffalse
%</samplechap1|samplechap2>
%\fi
%
%\iffalse
%<*samplechap1>
%\fi
% Some text for chapter 1:
%    \begin{macrocode}
\section{one}
some text in chapter one
%    \end{macrocode}

%\iffalse
%</samplechap1>
%\fi
% Some text for chapter 2:
%\iffalse
%<*samplechap2>
%\fi
%    \begin{macrocode}
\section{two}
more text in chapter two
%    \end{macrocode}

%\iffalse
%</samplechap2>
%\fi
%
% %%%%%%%%%%%%%%%%%%%%%%%%%%%%%%%%%%%%%%
% \paragraph{Part Include Files.}
%
% The include files are called |cdocspt3.tex| and |cdocspt4.tex|.
%
%\iffalse
%<*samplepart3|samplepart4>
%\fi

% Optional override for |\version| flag:
%    \begin{macrocode}
%%\providecommand{\version}{final}
%    \end{macrocode}

% Include the main document:
%    \begin{macrocode}
% \iffalse
%
% childdoc.dtx Copyright (C) 2017-2018 Niklas Beisert
%
% This work may be distributed and/or modified under the
% conditions of the LaTeX Project Public License, either version 1.3
% of this license or (at your option) any later version.
% The latest version of this license is in
%   http://www.latex-project.org/lppl.txt
% and version 1.3 or later is part of all distributions of LaTeX
% version 2005/12/01 or later.
%
% This work has the LPPL maintenance status `maintained'.
%
% The Current Maintainer of this work is Niklas Beisert.
%
% This work consists of the files childdoc.dtx and childdoc.ins
% and the derived files childdoc.def and cdocsamp.tex with
% cdocsch1.tex, cdocsch2.tex, cdocsdrf.tex, cdocsfn1.tex, cdocsfn2.tex.
%
%<package>\ifdefined\childdocmain\endinput\fi
%<package>\ProvidesFile{childdoc.def}[2018/12/30 v2.0 child document driver]
%<samplemain>\ProvidesFile{cdocsamp.tex}[2018/12/30 v2.0 sample for childdoc]
%<*driver>
%\ProvidesFile{childdoc.drv}[2018/12/30 v2.0 childdoc reference manual file]
\PassOptionsToClass{10pt,a4paper}{article}
\documentclass{ltxdoc}

\usepackage[margin=35mm]{geometry}
\usepackage{hyperref}
\usepackage{hyperxmp}
\usepackage[usenames]{color}

\hypersetup{colorlinks=true}
\hypersetup{pdfstartview=FitH}
\hypersetup{pdfpagemode=UseNone}
\hypersetup{pdfsource={}}
\hypersetup{pdflang={en-UK}}
\hypersetup{pdfcopyright={Copyright 2017-2018 Niklas Beisert.
  This work may be distributed and/or modified under the
  conditions of the LaTeX Project Public License, either version 1.3
  of this license or (at your option) any later version.}}
\hypersetup{pdflicenseurl={http://www.latex-project.org/lppl.txt}}
\hypersetup{pdfcontactaddress={ETH Zurich, ITP, HIT K,
  Wolfgang-Pauli-Strasse 27}}
\hypersetup{pdfcontactpostcode={8093}}
\hypersetup{pdfcontactcity={Zurich}}
\hypersetup{pdfcontactcountry={Switzerland}}
\hypersetup{pdfcontactemail={nbeisert@itp.phys.ethz.ch}}
\hypersetup{pdfcontacturl={http://people.phys.ethz.ch/\xmptilde nbeisert/}}

\newcommand{\secref}[1]{\hyperref[#1]{section \ref*{#1}}}

\parskip1ex
\parindent0pt
\let\olditemize\itemize
\def\itemize{\olditemize\parskip0pt}

\begin{document}

\title{The \textsf{childdoc} Package}
\hypersetup{pdftitle={The childdoc Package}}
\author{Niklas Beisert\\[2ex]
  Institut f\"ur Theoretische Physik\\
  Eidgen\"ossische Technische Hochschule Z\"urich\\
  Wolfgang-Pauli-Strasse 27, 8093 Z\"urich, Switzerland\\[1ex]
  \href{mailto:nbeisert@itp.phys.ethz.ch}
  {\texttt{nbeisert@itp.phys.ethz.ch}}}
\hypersetup{pdfauthor={Niklas Beisert}}
\hypersetup{pdfsubject={Manual for the LaTeX2e Package childdoc}}
\date{30 December 2018, \textsf{v2.0}}
\maketitle

\begin{abstract}\noindent
\textsf{childdoc} is a \LaTeXe{} package
that enables the direct compilation
of document sections included by |\include|
to individual files.
\end{abstract}

\begingroup
\parskip0ex
\tableofcontents
\endgroup

%%%%%%%%%%%%%%%%%%%%%%%%%%%%%%%%%%%%%%%%%%%%%%%%%%%%%%%%%%%%%%%%%%%%%%%%%%%%%%%%
%%%%%%%%%%%%%%%%%%%%%%%%%%%%%%%%%%%%%%%%%%%%%%%%%%%%%%%%%%%%%%%%%%%%%%%%%%%%%%%%
\section{Introduction}

\LaTeX{} provides a mechanism to structure a large document (such as a book)
into a main file and several child files (containing the chapters)
using the |\include| command.
This mechanism is beneficial for documents
which span hundreds of pages in order to
make the source file(s) more manageable.
Moreover, compilation can be restricted to
selected child files by means of the |\includeonly| command.
The latter feature can be used to reduce the compilation time while editing
(this was significantly more useful in the earlier days of \LaTeX{})
or to generate a smaller document which is easier to navigate.
Another application of |\includeonly| is to generate
documents consisting of selected parts of the complete document.

However, there are a few drawbacks of the plain |\include| mechanism:
\begin{itemize}
\item
The child files cannot be compiled on their own,
they can only be compiled via the main file.
A naive editing environment
(such as a text editor with an option
to have the current file processed by \LaTeX)
may require one to switch to the main file before compiling;
attempting to compile the child file produces errors.
\item
The main file must be modified (each time)
to adjust the |\includeonly| command
to the present needs. This easily leaves the main file in a messy state.
\item
The generated document will always carry the filename
of the main document. This is inconvenient if
several child files are to be compiled and
to be kept for distribution.
\end{itemize}

The present package provides a simple interface
to make child files individually compilable by \LaTeX{}.
Compiling a child file then has the same effect as compiling
the main file with an |\includeonly| command
to select the appropriate child.
Moreover the generated document will carry the name of the child
rather than the main file.
This resolves all three above issues.

This feature is meant to make the editing of books,
thesis documents and lecture notes somewhat more convenient.
However, the package can also be used efficiently for
composing a series of documents (such as exercise sheets)
which are typically distributed individually.
It then assists the author in generating the individual documents
(potentially in different versions)
as well as a document containing the collected series.
Another application is in developing style files
or other kinds of included material
where compilation of the style file could redirect
to a sample or test file.

%%%%%%%%%%%%%%%%%%%%%%%%%%%%%%%%%%%%%%%%%%%%%%%%%%%%%%%%%%%%%%%%%%%%%%%%%%%%%%%%
%%%%%%%%%%%%%%%%%%%%%%%%%%%%%%%%%%%%%%%%%%%%%%%%%%%%%%%%%%%%%%%%%%%%%%%%%%%%%%%%
\section{Usage}

First of all, the package \textsf{childdoc} is \emph{not} a standard
\LaTeXe{} |.sty| style file! Therefore it needs to be invoked in
a non-standard way.

%%%%%%%%%%%%%%%%%%%%%%%%%%%%%%%%%%%%%%%%%%%%%%%%%%%%%%%%%%%%%%%%%%%%%%%%%%%%%%%%
\subsection{Included Files}
\label{sec:include}

%%%%%%%%%%%%%%%%%%%%%%%%%%%%%%%%%%%%%%%%
\DescribeMacro{\childdocmain}
To use the package, add the commands
\begin{center}
\begin{tabular}{l}
|\input{childdoc.def}|\\
|\childdocmain{}|\\
\end{tabular}
\end{center}
at the very top of the main \LaTeX{} file,
in particular \emph{before} the |\documentclass| statement!
The argument of |\childdocmain| should be left empty
(but it must be present).

%%%%%%%%%%%%%%%%%%%%%%%%%%%%%%%%%%%%%%%%
\DescribeMacro{\childdocof}
Furthermore, add the commands
\begin{center}
\begin{tabular}{l}
|\input{childdoc.def}|\\
|\childdocof{|\textit{main}|}|\\
\end{tabular}
\end{center}
at the top of every child file \textit{child}
which is included by |\include{|\textit{child}|}|
from within the main file
(or at least for those files to be compiled individually).
The argument \textit{main} must be the filename of the main file.

There are a couple of
considerations in setting up the main and child documents:

%%%%%%%%%%%%%%%%%%%%%%%%%%%%%%%%%%%%%%%%
\paragraph{Restrictions.}

Please note the following restrictions:
\begin{itemize}
\item
|\childdocmain| must be called with one argument \textit{main}
to ensure compatibility with earlier version of the package.
It must either be empty (|\childdocmain{}|)
or precisely match the filename of the main file in which it is specified.
See \secref{sec:detection} for further information.
\item
The filename \textit{main} must be specified without the |.tex| extension.
\item
The filename \textit{main} is case sensitive
(even in case-insensitive file systems)
due to internal string comparison.
\item
The argument \textit{main} should be fully expanded, it cannot be a macro.
\item
Subdirectories and special characters should be avoided in filenames.
\item
The command |\childdocmain{|\textit{main}|}| must be followed by a whitespace.
It should not be followed immediately by another command
or by a comment mark `|%|'.
This is because the \TeX{} parser reads the token immediately following
the argument of |\childdocmain| and puts it
at the beginning of every child section;
however, a white\-space is ignored.
\end{itemize}

%%%%%%%%%%%%%%%%%%%%%%%%%%%%%%%%%%%%%%%%
\paragraph{Content of Main File.}

It is advisable to place all content in the child files included by |\include|.
Any output contained in the main file will appear in all child documents
unless suppressed manually;
it cannot be suppressed automatically by the |\includeonly| directive
and thus should normally be avoided.
A method to include some content in the main file
by means of conditional processing is described in \secref{sec:conditional}.

%%%%%%%%%%%%%%%%%%%%%%%%%%%%%%%%%%%%%%%%
\paragraph{Page Numbering.}

When only a part of the document is compiled,
the appropriate numbering of pages
(as well as other status parameters)
is determined from the |.aux| files.
The latter contain information from previous passes.
However this information needs to propagate through
all intermediate child documents.
Therefore the page numbering in child documents may well
be inconsistent until the complete document is compiled at least once.

A useful (if unconventional) way to always ensure a consistent
page numbering is to restart the numbering in each child document
and denote the pages by `\textit{child}|.|\textit{page}'
where \textit{child} represents the chapter/section number of the child file.
This can be achieved by the command
|\numberwithin{page}{|\textit{child}|}|
of the \textsf{amsmath} package
where \textit{child} can be |chapter| or |section|
depending on the chosen structuring.
Alternatively, one can modify the macro |\thepage| appropriately
and reset the counter |page| at the start of each child file.

%%%%%%%%%%%%%%%%%%%%%%%%%%%%%%%%%%%%%%%%%%%%%%%%%%%%%%%%%%%%%%%%%%%%%%%%%%%%%%%%
\subsection{Conditional Processing}
\label{sec:conditional}

The package provides a mechanism to compile different versions
of a document. To customise the versions further some conditional processing
can come in handy to distinguish which version is being compiled.
The package provides two macros to describe the compilation context:

%%%%%%%%%%%%%%%%%%%%%%%%%%%%%%%%%%%%%%%%
\DescribeMacro{\ifchilddoc}
The conditional |\ifchilddoc| distinguishes between the compilation of
child documents and the main document:
%
\begin{center}
|\ifchilddoc |\textit{child-code}| |[|\||else |\textit{main-code}]| \||fi|
\end{center}

%%%%%%%%%%%%%%%%%%%%%%%%%%%%%%%%%%%%%%%%
\DescribeMacro{\childdocname}
\DescribeMacro{\childdocjob}
The macro |\childdocname| contains the filename (without extension)
of the main or child file being processed.
Note that |\childdocjob| will always contain the name of the main file.

%%%%%%%%%%%%%%%%%%%%%%%%%%%%%%%%%%%%%%%%
\paragraph{Title Page.}

Conditional processing can be used to include a title or banner page
in the main document when proper precautions are taken.
Importantly, the code in the main file should ensure that the page counter
(as well as other status parameters which are stored in the |.aux| files)
takes the same value after the conditional processing.
Otherwise the page numbers may take divergent values
depending on which part is compiled.

For example, a title page could be declared by:
%
\begin{center}
\begin{tabular}{l}
|\ifchilddoc\||else|\\
|\addtocounter{page}{-1}|\\
\textit{code for title page}\\
|\newpage|\\
|\||fi|
\end{tabular}
\end{center}
%
A banner page for the child documents can be generated by:
%
\begin{center}
\begin{tabular}{l}
|\ifchilddoc|\\
|\addtocounter{page}{-1}|\\
\textit{code for banner page}\\
|\newpage|\\
|\||fi|
\end{tabular}
\end{center}
%
Here one could write a message such as:
\begin{center}
|This is the part \childdocname{} of \childdocjob{}.|
\end{center}

%%%%%%%%%%%%%%%%%%%%%%%%%%%%%%%%%%%%%%%%%%%%%%%%%%%%%%%%%%%%%%%%%%%%%%%%%%%%%%%%
\subsection{Flags}
\label{sec:flags}

The package makes it easy to generate different versions
of the main or child documents.
To this end compilation flags can be defined
and assigned different default values.
They will be particularly useful in conjunction
with the forwarding mechanism described in \secref{sec:forward}.

For example, it may be useful to have a flag |\version|
which can be set to |draft| or |final|.
The document source will contain some conditional code
depending on the value of |\version|.
Suppose further, the flag should default to |final| for the main file
and to |draft| for child files
which is a natural assignment for editing the document.
This is achieved by placing the following code
in the preamble of the main document
(below the |\childdocmain| directive):
%
\begin{center}
\begin{tabular}{l}
|\ifchilddoc|\\
|\providecommand{\version}{draft}|\\
|\||else|\\
|\providecommand{\version}{final}|\\
|\||fi|
\end{tabular}
\end{center}
%
The definition by |\providecommand| makes sure
that previous definitions are not overwritten.
Further statements |\providecommand{\version}{...}|
can thus be added before the above code to override it.

For the main file, one might add a line
(between |\childdocmain| and the above block)
%
\begin{center}
|%\ifchilddoc\||else\providecommand{\version}{draft}\||fi|
\end{center}
%
which can be uncommented to produce a draft version.
Likewise one can add a line to the very top of a child file
(above the |\childdocof{|\textit{main}|}| directive)
%
\begin{center}
|%\providecommand{\version}{final}|
\end{center}
%
which can be uncommented to produce the final version of this child document.

%%%%%%%%%%%%%%%%%%%%%%%%%%%%%%%%%%%%%%%%%%%%%%%%%%%%%%%%%%%%%%%%%%%%%%%%%%%%%%%%
\subsection{Forwarding}
\label{sec:forward}

Different versions of the main or child documents
using compilation flags as described in \secref{sec:flags}
can be (permanently) stored in different files
for convenient compilation, viewing and distribution.
To this end, the package defines a command
to pass on compilation to a different file:

%%%%%%%%%%%%%%%%%%%%%%%%%%%%%%%%%%%%%%%%
\DescribeMacro{\childdocforward}
The command |\childdocforward| redirects processing to
another source file:
%
\begin{center}
\begin{tabular}{l}
|\input{childdoc.def}|\\
|\childdocforward[|\textit{main}|]{|\textit{dest}|}|\\
\end{tabular}
\end{center}
%
The argument \textit{dest} is the destination file
(without extension).
It should be the main file or one of the child files.
Note that further \textsf{childdoc} directives
such as |\childdocof| and |\childdocforward|
in the indicated file will be processed in this form.
The optional argument \textit{main}
passes on directly to the main file \textit{main}
while pretending to compile the child \textit{dest}.
This form behaves as if \textit{dest}
issues |\childdocof{|\textit{main}|}| right away,
and no further \textsf{childdoc} directives will be processed.

%%%%%%%%%%%%%%%%%%%%%%%%%%%%%%%%%%%%%%%%
\DescribeMacro{\...prefix}
In the alternative form |\childdocforwardprefix|,
%
\begin{center}
\begin{tabular}{l}
|\input{childdoc.def}|\\
|\childdocforwardprefix[|\textit{main}|]{|\textit{prefix}|}{|\textit{dest}|}|
\end{tabular}
\end{center}
%
the destination file is determined by a pattern
depending on the current file:
To make this work, the current file must be called
`{\textit{prefix}\hspace{0.2em}\textit{suffix}}'
with \textit{prefix} matching precisely the argument.
Processing is then passed on to the file
`{\textit{dest}\hspace{0.2em}\textit{suffix}}'.
Surely, the same effect is achieved by
directly specifying the
argument `{\textit{dest}\hspace{0.2em}\textit{suffix}}'
in the first form.
However, that requires to set up a different file
for each child. With the alternative form of the command
all these files can have exactly the same content
which simplifies setting them up and maintaining them.

For example, the following file |draft.tex|
with a compilation flag |\version| as described in \secref{sec:flags}
compiles the main document as a draft:
%
\begin{center}
\begin{tabular}{l}
|\def\version{draft}|\\
|\input{childdoc.def}|\\
|\childdocforward{|\textit{main}|}|
\end{tabular}
\end{center}
%
Likewise, the following files |final|\textit{nn}|.tex|
compile the final version of the child document
|child|\textit{nn}|.tex|:
%
\begin{center}
\begin{tabular}{l}
|\def\version{final}|\\
|\input{childdoc.def}|\\
|\childdocforwardprefix{final}{child}|
\end{tabular}
\end{center}
%

Note that when several versions of a main file and/or of each child file
are to be generated, it may be convenient to set up a |Makefile| or
shell script to automatise the process.

%%%%%%%%%%%%%%%%%%%%%%%%%%%%%%%%%%%%%%%%%%%%%%%%%%%%%%%%%%%%%%%%%%%%%%%%%%%%%%%%
\subsection{Command Line Processing}
\label{sec:commandline}

The effect of redirection files can also be achieved by invoking
the \LaTeX{} compiler with a more elaborate command line.
Most conveniently this should be done as part
of a shell script or a |Makefile|.

When using \textsf{childdoc} in the main file, the following
command lines effectively perform a redirection
(note that depending on the shell being used,
backslashes may have to be doubled: `|\|' $\to$ `|\\|'):
%
\begin{center}
|... -jobname "|\textit{target}|" |\\|"|[\textit{flags}]%
|\input{childdoc.def}\childdocforward[|\textit{main}|]{|\textit{dest}|}"|
\end{center}
%
Here \textit{target} is the name of the output file,
\textit{main} is the name of the main file
and \textit{dest} is the name of the main or child file to be processed
(all filenames without extensions).
The optional argument \textit{main} can be omitted
if \textit{main} matches \textit{dest}.
Optionally, compilation \textit{flags} can be defined via |\def| commands.
This command line makes the \TeX{} engine believe
it is compiling the file \textit{target}
whose content is specified as the latter parameter.
The provided code then forwards the processing to
\textit{main} or \textit{dest} as described in \secref{sec:forward}.

%%%%%%%%%%%%%%%%%%%%%%%%%%%%%%%%%%%%%%%%%%%%%%%%%%%%%%%%%%%%%%%%%%%%%%%%%%%%%%%%
\subsection{Include by Input}
\label{sec:input}

Including child documents by |\include| has some restrictions by design.
Most notably, the content of a child document always occupies
its own set of pages; pages cannot be shared between child documents.
Usually, this behaviour makes perfect sense
because each child document contain an essential part of the document.
However, in some situations it may be desirable to compose
a document from a collection of parts
without having mandatory page breaks between then.
For this case, the package
provides a mechanism to include parts
by |\input| which can also be processed individually.
However, by construction this mechanism
requires manual handling of the content to be output.

%%%%%%%%%%%%%%%%%%%%%%%%%%%%%%%%%%%%%%%%
\DescribeMacro{\ifchilddocmanual}
The main file should be prepared as usual, see \secref{sec:include}.
However, the document body must make a distinction
between processing of an individual part and of the main document, e.g.:
%
\begin{center}
\begin{tabular}{l}
|\ifchilddocmanual|\\
|\input{\childdocname}|\\
|\||else|\\
\textit{document body with }|\input{|\textit{part}|}|\\
|\||fi|
\end{tabular}
\end{center}
%
The conditional |\ifchilddocmanual| is true whenever
a part to be included by |\input| is being compiled,
and the name of the part is stored in |\childdocname|.

%%%%%%%%%%%%%%%%%%%%%%%%%%%%%%%%%%%%%%%%
\DescribeMacro{\childdocby}
Each part to be included by |\input| should start with:
%
\begin{center}
\begin{tabular}{l}
|\input{childdoc.def}|\\
|\childdocby{|\textit{main}|}|\\
\end{tabular}
\end{center}
%
The directive |\childdocby| is similar to |\childdocof|
described in \secref{sec:include},
but the subsequent selection of content must be done manually.
To that end, both |\ifchilddoc| and |\ifchilddocmanual|
will be true upon processing of a part,
and the name of the part is stored in |\childdocname|.
Note that |\jobname| will be set to the filename of the current part
so that each part receives an individual |.aux| file
that does not interfere with the |.aux| file(s) of the main document.
This behaviour can be altered by the alternative form
|\childdocby[*]{|\textit{main}|}| (with a non-empty optional argument)
which uses the |.aux| file of the main document
by setting |\jobname| to \textit{main}.

%%%%%%%%%%%%%%%%%%%%%%%%%%%%%%%%%%%%%%%%%%%%%%%%%%%%%%%%%%%%%%%%%%%%%%%%%%%%%%%%
\subsection{Driver Development}
\label{sec:driver}

The \textsf{childdoc} mechanism can also be use for the development
of definition files such as \LaTeX{} styles or classes.
This case differs from the above setup with multiple parts
included by |\include| in that no |\includeonly| should be invoked.
This can be achieved by starting the include file
(before |\ProvidesPackage|) with:
%
\begin{center}
\begin{tabular}{l}
|\input{childdoc.def}|\\
|\childdocforward{|\textit{main}|}|\\
\end{tabular}
\end{center}
%
or alternatively with:
%
\begin{center}
\begin{tabular}{l}
|\input{childdoc.def}|\\
|\childdocby{|\textit{main}|}|\\
\end{tabular}
\end{center}
%
Both forms have slightly different effects as described above.
The main file is prepared as usual, see \secref{sec:include}.

%%%%%%%%%%%%%%%%%%%%%%%%%%%%%%%%%%%%%%%%%%%%%%%%%%%%%%%%%%%%%%%%%%%%%%%%%%%%%%%%
\subsection{Legacy Detection}
\label{sec:detection}

The directive |\childdocmain| in the main file can detect
whether the complete document or merely a child is to be compiled
even without using the directive |\childdocof|.
This method is deprecated because it is less robust
and there is no compelling reason to use it;
it is merely provided for backward compatibility
and it may be removed in future versions.

If the detection mechanism is to be used,
it is mandatory to correctly specify
the filename of the main file as the argument of |\childdocmain|:
%
\begin{center}
\begin{tabular}{l}
|\input{childdoc.def}|\\
|\childdocmain{|\textit{main}|}|\\
\end{tabular}
\end{center}
%
If |\jobname| does not match the argument \textit{main} of |\childdocmain|,
it is assumed that |\jobname| points to the child file to be compiled.
When using |\childdocmain| with the main file specified as argument,
it suffices to start a child file
with just |\input{|\textit{main}|}|
without loading of the package and using |\childdocof|.
If instead all processing is done
with the appropriate \textsf{childdoc} directives,
the argument of \textit{main} of |\childdocmain| can be empty.

An alternative version of the command line processing described
in \secref{sec:commandline} using the detection mechanism reads:
%
\begin{center}
|... -jobname "|\textit{target}|" "|[\textit{flags}]%
[|\def\jobname{|\textit{dest}|}|]|\input{|\textit{main}|}"|
\end{center}

%%%%%%%%%%%%%%%%%%%%%%%%%%%%%%%%%%%%%%%%%%%%%%%%%%%%%%%%%%%%%%%%%%%%%%%%%%%%%%%%
\subsection{Manual Code}
\label{sec:manual}

In case one cannot be certain whether the definitions file |childdoc.def|
is installed on the target \TeX{} distribution
and one prefers not to ship it,
it is conceivable to paste a few relevant commands into the sources.

To that end, drop all statements |\input{childdoc.def}|
and perform the replacements as outlined below.
Instead of |\childdocmain{|\textit{main}|}| add the following code
to the top of the main file:
%
\begin{center}
\begin{tabular}{l}
|\||ifdefined\childdocname\endinput\||fi\newif\ifchilddoc|\\
|\edef\childdocname{\scantokens\expandafter{\jobname\noexpand}}|\\
|\def\childdocmain{|\textit{main}|}\||ifx\childdocmain\childdocname\||else|\\
|\childdoctrue\includeonly{\childdocname}\let\jobname\childdocmain\||fi|\\
\end{tabular}
\end{center}
%
Instead of |\childdocof{|\textit{main}|}| just include the main file
at the top of each child file:
%
\begin{center}
|\input{|\textit{main}|}|
\end{center}
%
A simple redirection |\childdocforward{|\textit{dest}|}| is achieved by:
%
\begin{center}
|\def\jobname{|\textit{dest}|}\input{\jobname}|
\end{center}
%
The redirection with prefix
|\childdocforwardprefix[|\textit{prefix}|]{|\textit{dest}|}|
is accomplished by:
%
\begin{center}
\begin{tabular}{l}
|{\edef\jobname{\scantokens\expandafter{\jobname\noexpand}}|\\
|\def\redirectjob |\textit{prefix}|#1~~~{\gdef\jobname{|\textit{dest}|#1}}|\\
|\expandafter\redirectjob\jobname~~~}\input{\jobname}|
\end{tabular}
\end{center}

In an alternative approach,
child documents can be compiled by a specific command line
without additional code or specific definitions:
%
\begin{center}
|... -jobname "|\textit{target}|" "|[\textit{flags}]%
|\includeonly{|\textit{dest}|}\input{|\textit{main}|}"|
\end{center}
%

%%%%%%%%%%%%%%%%%%%%%%%%%%%%%%%%%%%%%%%%%%%%%%%%%%%%%%%%%%%%%%%%%%%%%%%%%%%%%%%%
%%%%%%%%%%%%%%%%%%%%%%%%%%%%%%%%%%%%%%%%%%%%%%%%%%%%%%%%%%%%%%%%%%%%%%%%%%%%%%%%
\section{Information}

%%%%%%%%%%%%%%%%%%%%%%%%%%%%%%%%%%%%%%%%%%%%%%%%%%%%%%%%%%%%%%%%%%%%%%%%%%%%%%%%
\subsection{Copyright}

Copyright \copyright{} 2017--2018 Niklas Beisert

This work may be distributed and/or modified under the
conditions of the \LaTeX{} Project Public License, either version 1.3
of this license or (at your option) any later version.
The latest version of this license is in
  \url{http://www.latex-project.org/lppl.txt}
and version 1.3 or later is part of all distributions of \LaTeX{}
version 2005/12/01 or later.

This work has the LPPL maintenance status `maintained'.

The Current Maintainer of this work is Niklas Beisert.

This work consists of the files |README.txt|, |childdoc.ins| and |childdoc.dtx|
as well as the derived files |childdoc.def|, |cdocsamp.tex|
with |cdocsch1.tex|, |cdocsch2.tex|, |cdocspt3.tex|, |cdocspt4.tex|,
|cdocsdrf.tex|, |cdocsfn1.tex|, |cdocsfn2.tex|
as well as |childdoc.pdf|.

%%%%%%%%%%%%%%%%%%%%%%%%%%%%%%%%%%%%%%%%%%%%%%%%%%%%%%%%%%%%%%%%%%%%%%%%%%%%%%%%
\subsection{Files and Installation}

The package consists of the files:
%
\begin{center}
\begin{tabular}{ll}
    |README.txt|   & readme file \\
    |childdoc.ins| & installation file \\
    |childdoc.dtx| & source file \\
    |childdoc.def| & definition file \\
    |cdocsamp.tex| & sample main file \\
    |cdocsch1.tex| & sample include file \\
    |cdocsch2.tex| & sample include file \\
    |cdocspt3.tex| & sample part file \\
    |cdocspt4.tex| & sample part file \\
    |cdocsdrf.tex| & sample redirection file \\
    |cdocsfn1.tex| & sample redirection file \\
    |cdocsfn2.tex| & sample redirection file \\
    |childdoc.pdf| & manual
\end{tabular}
\end{center}
%
The distribution consists of the files
|README.txt|, |childdoc.ins| and |childdoc.dtx|.
%
\begin{itemize}
\item
Run (pdf)\LaTeX{} on |childdoc.dtx|
to compile the manual |childdoc.pdf| (this file).
\item
Run \LaTeX{} on |childdoc.ins| to create the definitions file |childdoc.def|
and the sample |cdocsamp.tex| with include files
|cdocsch1.tex|, |cdocsch2.tex|, |cdocspt3.tex|, |cdocspt4.tex|,
|cdocsdrf.tex|, |cdocsfn1.tex|, |cdocsfn2.tex|.
Then copy the file |childdoc.def| to an appropriate directory of your \LaTeX{}
distribution, e.g.\ \textit{texmf-root}|/tex/latex/childdoc|.
\end{itemize}

%%%%%%%%%%%%%%%%%%%%%%%%%%%%%%%%%%%%%%%%%%%%%%%%%%%%%%%%%%%%%%%%%%%%%%%%%%%%%%%%
\subsection{Related CTAN Packages}

There are several other packages which offer a similar functionality:
%
\begin{itemize}
\item
The packages
\href{http://ctan.org/pkg/docmute}{\textsf{docmute}},
\href{http://ctan.org/pkg/includex}{\textsf{includex}} and
\href{http://ctan.org/pkg/standalone}{\textsf{standalone}}
provide commands to include only the document body of
a child file thus allowing both files to be compiled individually.
\item
The packages \href{http://ctan.org/pkg/subdocs}{\textsf{subdocs}}
and \href{http://ctan.org/pkg/subfiles}{\textsf{subfiles}}
provide structures in which the main and child documents can be
encapsulated and allowing them to be compiled individually.
The inclusion mechanism is different from the conventional |\include|.
\item
The package \href{http://ctan.org/pkg/combine}{\textsf{combine}}
is an elaborate solution to combine several documents into one.
\end{itemize}
%
See also the CTAN topic \href{http://ctan.org/topic/subdocs}{\textsf{subdocs}}
for further related packages.
The present package differs from the above solutions in that
a document structure constructed with the conventional |\include| mechanism
just needs two extra commands at the top of every file
such that all constituent files can be compiled individually.

%%%%%%%%%%%%%%%%%%%%%%%%%%%%%%%%%%%%%%%%%%%%%%%%%%%%%%%%%%%%%%%%%%%%%%%%%%%%%%%%
%\subsection{Feature Suggestions}
%
%The following is a list of features which may be useful for future
%versions of this package:
%%
%\begin{itemize}
%\item
%\ldots
%\end{itemize}

%%%%%%%%%%%%%%%%%%%%%%%%%%%%%%%%%%%%%%%%%%%%%%%%%%%%%%%%%%%%%%%%%%%%%%%%%%%%%%%%
\subsection{Revision History}

%%%%%%%%%%%%%%%%%%%%%%%%%%%%%%%%%%%%%%%%
\paragraph{v2.0:} 2018/12/30

\begin{itemize}
\item
immediate forward processing
\item
added |\childdocby| mechanism
\item
manual restructured
\end{itemize}

%%%%%%%%%%%%%%%%%%%%%%%%%%%%%%%%%%%%%%%%
\paragraph{v1.6:} 2018/01/17

\begin{itemize}
\item
application for development of include files
\item
corrections to manual
\end{itemize}

%%%%%%%%%%%%%%%%%%%%%%%%%%%%%%%%%%%%%%%%
\paragraph{v1.5:} 2017/05/21

\begin{itemize}
\item
more complete structuring introduced
\item
|\childdocof| introduced
\item
|\childdoc| renamed to |\childdocmain|
\item
|\childredirect| renamed to |\childdocforward| and |\childdocforwardprefix|
and functionality expanded
\end{itemize}

%%%%%%%%%%%%%%%%%%%%%%%%%%%%%%%%%%%%%%%%
\paragraph{v1.0:} 2017/04/27

\begin{itemize}
\item
manual and install package
\item
first version published on CTAN
\end{itemize}

%%%%%%%%%%%%%%%%%%%%%%%%%%%%%%%%%%%%%%%%
\paragraph{v0.6:} 2017/04/26

\begin{itemize}
\item
redirection mechanism added
\end{itemize}

%%%%%%%%%%%%%%%%%%%%%%%%%%%%%%%%%%%%%%%%
\paragraph{v0.5:} 2017/04/26

\begin{itemize}
\item
functionality in definition file
\end{itemize}


%%%%%%%%%%%%%%%%%%%%%%%%%%%%%%%%%%%%%%%%%%%%%%%%%%%%%%%%%%%%%%%%%%%%%%%%%%%%%%%%
%%%%%%%%%%%%%%%%%%%%%%%%%%%%%%%%%%%%%%%%%%%%%%%%%%%%%%%%%%%%%%%%%%%%%%%%%%%%%%%%
%%%%%%%%%%%%%%%%%%%%%%%%%%%%%%%%%%%%%%%%%%%%%%%%%%%%%%%%%%%%%%%%%%%%%%%%%%%%%%%%
\appendix

\settowidth\MacroIndent{\rmfamily\scriptsize 000\ }

 \DocInput{childdoc.dtx}

\end{document}
%</driver>
% \fi
%
% %%%%%%%%%%%%%%%%%%%%%%%%%%%%%%%%%%%%%%%%%%%%%%%%%%%%%%%%%%%%%%%%%%%%%%%%%%%%%%
% %%%%%%%%%%%%%%%%%%%%%%%%%%%%%%%%%%%%%%%%%%%%%%%%%%%%%%%%%%%%%%%%%%%%%%%%%%%%%%
% \section{Sample}
%\iffalse
%<*samplemain>
%\fi
%
% The following presents a sample document
% with two chapters, two parts, a title page,
% a compile flag as well as three forwarding files to set the flag.
% It consists of eight |.tex| files:
% \begin{center}
% \begin{tabular}{ll}
% |cdocsamp.tex|&main file\\
% |cdocsch1.tex|&include file for chapter 1\\
% |cdocsch2.tex|&include file for chapter 2\\
% |cdocspt3.tex|&include file for part 3\\
% |cdocspt4.tex|&include file for part 4\\
% |cdocsdrf.tex|&forwarding file for main file in draft mode\\
% |cdocsfi1.tex|&forwarding file for final version of chapter 1\\
% |cdocsfi2.tex|&forwarding file for final version of chapter 2\\
% \end{tabular}
% \end{center}
% Each of the eight files can be compiled directly by the \LaTeX{} compiler.
%
% %%%%%%%%%%%%%%%%%%%%%%%%%%%%%%%%%%%%%%
% \paragraph{Main File.}
%
% The main file is called |cdocsamp.tex|.
%
% Load the \textsf{childdoc} definitions and
% declare the filename for the main document:
%    \begin{macrocode}
\input{childdoc.def}
\childdocmain{}
%    \end{macrocode}

% Optional override for |\version| flag:
%    \begin{macrocode}
%%\ifchilddoc\else\providecommand{\version}{draft}\fi
%    \end{macrocode}

% Define the default values for the |\version| flag
% (|final| for the main file and |draft| for childs):
%    \begin{macrocode}
\ifchilddoc
\providecommand{\version}{draft}
\else
\providecommand{\version}{final}
\fi
%    \end{macrocode}

% Load the standard document class:
%    \begin{macrocode}
\documentclass[12pt]{article}
%    \end{macrocode}

% Start the document body:
%    \begin{macrocode}
\begin{document}
%    \end{macrocode}

% Declare a title page.
% Print title, part of document being processed and version flag:
%    \begin{macrocode}
\addtocounter{page}{-1}
\begin{center}
{\LARGE\bfseries{}childdoc example\par}
\vspace{1cm}
\ifchilddoc
\ifchilddocmanual part\else chapter\fi:
`\childdocname' of `\childdocjob'\par
\else
main document: `\childdocjob'\par
\fi
version: \version\par
\end{center}
\newpage
%    \end{macrocode}

% Manually include selected file,
% otherwise process as usual:
%    \begin{macrocode}
\ifchilddocmanual
\section*{part `\childdocname'}
\input{\childdocname}
\else
%    \end{macrocode}

% Include the two chapters:
%    \begin{macrocode}
\include{cdocsch1}
\include{cdocsch2}
%    \end{macrocode}

% Include the two parts unless only chapters should be displayed:
%    \begin{macrocode}
\ifchilddoc\else
\section{part three}
\input{cdocspt3}
\section{part four}
\input{cdocspt4}
\fi
%    \end{macrocode}

% Process as usual until here:
%    \begin{macrocode}
\fi
%    \end{macrocode}

% End of document body:
%    \begin{macrocode}
\end{document}
%    \end{macrocode}
%\iffalse
%</samplemain>
%\fi
%
% %%%%%%%%%%%%%%%%%%%%%%%%%%%%%%%%%%%%%%
% \paragraph{Chapter Include Files.}
%
% The include files are called |cdocsch1.tex| and |cdocsch2.tex|.
%
%\iffalse
%<*samplechap1|samplechap2>
%\fi

% Optional override for |\version| flag:
%    \begin{macrocode}
%%\providecommand{\version}{final}
%    \end{macrocode}

% Include the main document:
%    \begin{macrocode}
\input{childdoc.def}
\childdocof{cdocsamp}
%    \end{macrocode}

%\iffalse
%</samplechap1|samplechap2>
%\fi
%
%\iffalse
%<*samplechap1>
%\fi
% Some text for chapter 1:
%    \begin{macrocode}
\section{one}
some text in chapter one
%    \end{macrocode}

%\iffalse
%</samplechap1>
%\fi
% Some text for chapter 2:
%\iffalse
%<*samplechap2>
%\fi
%    \begin{macrocode}
\section{two}
more text in chapter two
%    \end{macrocode}

%\iffalse
%</samplechap2>
%\fi
%
% %%%%%%%%%%%%%%%%%%%%%%%%%%%%%%%%%%%%%%
% \paragraph{Part Include Files.}
%
% The include files are called |cdocspt3.tex| and |cdocspt4.tex|.
%
%\iffalse
%<*samplepart3|samplepart4>
%\fi

% Optional override for |\version| flag:
%    \begin{macrocode}
%%\providecommand{\version}{final}
%    \end{macrocode}

% Include the main document:
%    \begin{macrocode}
\input{childdoc.def}
\childdocby{cdocsamp}
%    \end{macrocode}

%\iffalse
%</samplepart3|samplepart4>
%\fi
%
%\iffalse
%<*samplepart3>
%\fi
% Some text for part 3:
%    \begin{macrocode}
some text in part three
%    \end{macrocode}

%\iffalse
%</samplepart3>
%\fi
% Some text for part 4:
%\iffalse
%<*samplepart4>
%\fi
%    \begin{macrocode}
more text in part four
%    \end{macrocode}

%\iffalse
%</samplepart4>
%\fi
%
% %%%%%%%%%%%%%%%%%%%%%%%%%%%%%%%%%%%%%%
% \paragraph{Forwarding for a Complete Draft.}
%
% The following forwarding file |cdocsdrf.tex|
% compiles the main document in draft mode:
%\iffalse
%<*sampledraft>
%\fi
%    \begin{macrocode}
\def\version{draft}
\input{childdoc.def}
\childdocforward{cdocsamp}
%    \end{macrocode}

%\iffalse
%</sampledraft>
%\fi
%
% %%%%%%%%%%%%%%%%%%%%%%%%%%%%%%%%%%%%%%
% \paragraph{Forwarding for Final Version of the Chapters.}
%
% The following forwarding files |cdocsfn1.tex| and |cdocsfn2.tex|
% (with identical content)
% compile the final versions of the child documents
% |cdocsch1.tex| and |cdocsch2.tex|, respectively:
%\iffalse
%<*samplefinal>
%\fi
%    \begin{macrocode}
\def\version{final}
\input{childdoc.def}
\childdocforwardprefix[cdocsamp]{cdocsfn}{cdocsch}
%    \end{macrocode}

%\iffalse
%</samplefinal>
%\fi
%
% %%%%%%%%%%%%%%%%%%%%%%%%%%%%%%%%%%%%%%
% \paragraph{Command Line Processing.}
%
% The following three command lines generate the output files
% |cdocscld|, |cdocscl1| and |cdocscl2|
% which should be identical to
% |cdocsdrf|, |cdocsch1| and |cdocsfn2|, respectively:
% \begin{center}
% \begin{tabular}{l}
% |latex -jobname cdocscld \|\\
% |  "\def\version{draft}\input{childdoc.def}\childdocforward{cdocsamp}"|\\
% |latex -jobname cdocscl1 \|\\
% |  "\input{childdoc.def}\childdocforward[cdocsamp]{cdocsch1}"|\\
% |latex -jobname cdocscl2 \|\\
% |  "\def\version{final}\input{childdoc.def}\childdocforward{cdocsch2}"|
% \end{tabular}
% \end{center}
% Note that the trailing backslash on each first line
% merely continues the input to the second line
% (for convenient cut ant paste).
% Furthermore, the command |latex| can be replaced by any
% of its alternative versions such as |pdflatex|.
%
% %%%%%%%%%%%%%%%%%%%%%%%%%%%%%%%%%%%%%%%%%%%%%%%%%%%%%%%%%%%%%%%%%%%%%%%%%%%%%%
% %%%%%%%%%%%%%%%%%%%%%%%%%%%%%%%%%%%%%%%%%%%%%%%%%%%%%%%%%%%%%%%%%%%%%%%%%%%%%%
% \section{Implementation}
%\iffalse
%<*package>
%\fi
%
% This section describes the definitions file |childdoc.def|.

% The definitions cannot be loaded using |\usepackage| or |\RequirePackage|
% which has a mechanism to prevent loading a style file more than once.
% When loading the definitions by means of |\input|
% multiple instances have to be prevented manually:
%\iffalse
%This code needs to be before the `\ProvidesFile' directive
%which is defined at the beginning of this file.
%Therefore it is also placed there and commented out here.
%</package>
%<*discard>
%\fi
%    \begin{macrocode}
\ifdefined\childdocmain\endinput\fi
%    \end{macrocode}
%\iffalse
%</discard>
%<*package>
%\fi
%
% \macro{\ifchilddoc}
% \macro{\ifchilddocmanual}
% The conditional |\ifchilddoc| tells whether a
% child (true) or main (false) document is being compiled.
% The conditional |\ifchilddocmanual| tells whether
% the |\includeonly| mechanism is used (false) or
% the selection of child files must be performed manually (true).
% The definitions initialise to false:
%    \begin{macrocode}
\newif\ifchilddoc
\newif\ifchilddocmanual
%    \end{macrocode}

% \macro{\childdocname}
% \macro{\childdocjob}
% The macro |\childdocname| stores the name of the main document
% to be compiled. The macro |\childdocjob| stores the name of
% the document on which the \LaTeX{} compiler was originally invoked.
% The content of |\jobname| cannot be compared
% to filenames specified in the source due to different catcodes.
% The following code rescans |\jobname|, stores the result
% in |\childdocname| and saves a copy in |\childdocjob|:
%    \begin{macrocode}
\edef\childdocname{\scantokens\expandafter{\jobname\noexpand}}
\let\childdocjob\childdocname
%    \end{macrocode}

% \macro{\childdocdisable}
% The macro |\childdocdisable| prevents the main file
% from being processed more than once.
% At this stage, the main document command |\childdocmain|
% is assumed to be called once again where it should do nothing.
% Any subsequent call to it should prevent
% a secondary processing of the main document
% It overwrites the forwarding commands
% |\childdocof| and |\childdocforward|
% with empty macros to prevent further inclusions of the main document:
%    \begin{macrocode}
\newcommand{\childdocdisable}
{
  \renewcommand{\childdocmain}[1]{\renewcommand{\childdocmain}[1]{\endinput}}
  \renewcommand{\childdocof}[1]{}
  \renewcommand{\childdocby}[2][]{}
  \renewcommand{\childdocforward}[2][]{}
  \renewcommand{\childdocdisable}{}
}
%    \end{macrocode}

% \macro{\childdocmain}
% The macro |\childdocmain| is to be called at the top of the main file
% with nothing or the main filename (without extension) as argument.
% First, it breaks loops.
% If the argument is not empty and does not match |\childdocname|
% (which is set by the first inclusion of |childdoc.def|),
% |\ifchilddoc| is set to true, |\includeonly| is applied to the child file
% and |\jobname| is set to the main file
% (for proper handling of |.aux| files):
%    \begin{macrocode}
\newcommand{\childdocmain}[1]
{
  \childdocdisable\childdocmain{}
  \if?#1?\else
    \begingroup
      \def\childdoctmp{#1}
      \ifx\childdoctmp\childdocname
        \def\childdoctmp{}
      \else
        \def\childdoctmp
        {
          \childdoctrue
          \includeonly{\childdocname}
          \def\childdocjob{#1}
          \def\jobname{#1}
        }
      \fi
      \expandafter
    \endgroup
    \childdoctmp
  \fi
}
%    \end{macrocode}

% \macro{\childdocof}
% The command |\childdocof| redirects
% compilation to the main file |#1|.
%    \begin{macrocode}
\newcommand{\childdocof}[1]
{
  \childdocdisable
  \childdoctrue
  \includeonly{\childdocname}
  \def\jobname{#1}
  \def\childdocjob{#1}
  \input{#1}
}
%    \end{macrocode}

% \macro{\childdocby}
% The command |\childdocby| ....
%    \begin{macrocode}
\newcommand{\childdocby}[2][]
{
  \childdocdisable
  \childdoctrue
  \childdocmanualtrue
  \if?#1?\else
    \def\jobname{#2}
  \fi
  \def\childdocjob{#2}
  \input{#2}
  \endinput
}
%    \end{macrocode}

% \macro{\childdocforward}
% The command |\childdocforward| redirects
% compilation to the main file or
% (if the optional argument is given) a child file.
% Parameters are set as if the main file
% or a child file starting with |\childdocof| was compiled.
% Then compilation is handed over to the main file:
%    \begin{macrocode}
\newcommand{\childdocforward}[2][]
{
  \begingroup
    \if?#1?
      \def\childdoctmp
      {
        \def\childdocname{#2}
        \def\childdocjob{#2}
        \def\jobname{#2}
        \input{#2}
        \endinput
      }
    \else
      \def\childdoctmp
      {
        \childdocdisable
        \def\childdocname{#2}
        \childdoctrue
        \includeonly{#2}
        \def\childdocjob{#1}
        \def\jobname{#1}
        \input{#1}
        \endinput
      }
    \fi
    \expandafter
  \endgroup
  \childdoctmp
}
%    \end{macrocode}

% \macro{\childdocforwardprefix}
% The command |\childdocforwardprefix| redirects
% compilation to the main or a child file by means of a pattern.
% The prefix |#1| in the current filename is replaced by |#2|
% and the suffix of the current filename is kept
% (it is assumed that the filename does not contain the substring `|~~~|'
% which is used as a delimiter).
% Compilation is handed over to the new file by |\childdocforward|:
%    \begin{macrocode}
\newcommand{\childdocforwardprefix}[3][]
{
  \begingroup
    \def\childdocextract #2##1~~~{\def\childdoctmp{\childdocforward[#1]{#3##1}}}
    \expandafter\childdocextract\childdocname~~~
    \expandafter
  \endgroup
  \childdoctmp
}
%    \end{macrocode}

% \macro{\childdoc}
% The deprecated macro |\childdoc| is a legacy version of |\childdocmain|:
%    \begin{macrocode}
\newcommand{\childdoc}{\childdocmain}
%    \end{macrocode}

% \macro{\childdocredirect}
% The deprecated macro |\childdocredirect| is a legacy version
% of |\childdocforward| and |\childdocforwardprefix|:
%    \begin{macrocode}
\newcommand{\childdocredirect}[2][]
{
  \begingroup
    \if?#1?
      \def\childdoctmp{\childdocforward{#2}}
    \else
      \def\childdoctmp{\childdocforwardprefix{#1}{#2}}
    \fi
    \expandafter
  \endgroup
  \childdoctmp
}
%    \end{macrocode}

%\iffalse
%</package>
%\fi
%
\endinput

\childdocby{cdocsamp}
%    \end{macrocode}

%\iffalse
%</samplepart3|samplepart4>
%\fi
%
%\iffalse
%<*samplepart3>
%\fi
% Some text for part 3:
%    \begin{macrocode}
some text in part three
%    \end{macrocode}

%\iffalse
%</samplepart3>
%\fi
% Some text for part 4:
%\iffalse
%<*samplepart4>
%\fi
%    \begin{macrocode}
more text in part four
%    \end{macrocode}

%\iffalse
%</samplepart4>
%\fi
%
% %%%%%%%%%%%%%%%%%%%%%%%%%%%%%%%%%%%%%%
% \paragraph{Forwarding for a Complete Draft.}
%
% The following forwarding file |cdocsdrf.tex|
% compiles the main document in draft mode:
%\iffalse
%<*sampledraft>
%\fi
%    \begin{macrocode}
\def\version{draft}
% \iffalse
%
% childdoc.dtx Copyright (C) 2017-2018 Niklas Beisert
%
% This work may be distributed and/or modified under the
% conditions of the LaTeX Project Public License, either version 1.3
% of this license or (at your option) any later version.
% The latest version of this license is in
%   http://www.latex-project.org/lppl.txt
% and version 1.3 or later is part of all distributions of LaTeX
% version 2005/12/01 or later.
%
% This work has the LPPL maintenance status `maintained'.
%
% The Current Maintainer of this work is Niklas Beisert.
%
% This work consists of the files childdoc.dtx and childdoc.ins
% and the derived files childdoc.def and cdocsamp.tex with
% cdocsch1.tex, cdocsch2.tex, cdocsdrf.tex, cdocsfn1.tex, cdocsfn2.tex.
%
%<package>\ifdefined\childdocmain\endinput\fi
%<package>\ProvidesFile{childdoc.def}[2018/12/30 v2.0 child document driver]
%<samplemain>\ProvidesFile{cdocsamp.tex}[2018/12/30 v2.0 sample for childdoc]
%<*driver>
%\ProvidesFile{childdoc.drv}[2018/12/30 v2.0 childdoc reference manual file]
\PassOptionsToClass{10pt,a4paper}{article}
\documentclass{ltxdoc}

\usepackage[margin=35mm]{geometry}
\usepackage{hyperref}
\usepackage{hyperxmp}
\usepackage[usenames]{color}

\hypersetup{colorlinks=true}
\hypersetup{pdfstartview=FitH}
\hypersetup{pdfpagemode=UseNone}
\hypersetup{pdfsource={}}
\hypersetup{pdflang={en-UK}}
\hypersetup{pdfcopyright={Copyright 2017-2018 Niklas Beisert.
  This work may be distributed and/or modified under the
  conditions of the LaTeX Project Public License, either version 1.3
  of this license or (at your option) any later version.}}
\hypersetup{pdflicenseurl={http://www.latex-project.org/lppl.txt}}
\hypersetup{pdfcontactaddress={ETH Zurich, ITP, HIT K,
  Wolfgang-Pauli-Strasse 27}}
\hypersetup{pdfcontactpostcode={8093}}
\hypersetup{pdfcontactcity={Zurich}}
\hypersetup{pdfcontactcountry={Switzerland}}
\hypersetup{pdfcontactemail={nbeisert@itp.phys.ethz.ch}}
\hypersetup{pdfcontacturl={http://people.phys.ethz.ch/\xmptilde nbeisert/}}

\newcommand{\secref}[1]{\hyperref[#1]{section \ref*{#1}}}

\parskip1ex
\parindent0pt
\let\olditemize\itemize
\def\itemize{\olditemize\parskip0pt}

\begin{document}

\title{The \textsf{childdoc} Package}
\hypersetup{pdftitle={The childdoc Package}}
\author{Niklas Beisert\\[2ex]
  Institut f\"ur Theoretische Physik\\
  Eidgen\"ossische Technische Hochschule Z\"urich\\
  Wolfgang-Pauli-Strasse 27, 8093 Z\"urich, Switzerland\\[1ex]
  \href{mailto:nbeisert@itp.phys.ethz.ch}
  {\texttt{nbeisert@itp.phys.ethz.ch}}}
\hypersetup{pdfauthor={Niklas Beisert}}
\hypersetup{pdfsubject={Manual for the LaTeX2e Package childdoc}}
\date{30 December 2018, \textsf{v2.0}}
\maketitle

\begin{abstract}\noindent
\textsf{childdoc} is a \LaTeXe{} package
that enables the direct compilation
of document sections included by |\include|
to individual files.
\end{abstract}

\begingroup
\parskip0ex
\tableofcontents
\endgroup

%%%%%%%%%%%%%%%%%%%%%%%%%%%%%%%%%%%%%%%%%%%%%%%%%%%%%%%%%%%%%%%%%%%%%%%%%%%%%%%%
%%%%%%%%%%%%%%%%%%%%%%%%%%%%%%%%%%%%%%%%%%%%%%%%%%%%%%%%%%%%%%%%%%%%%%%%%%%%%%%%
\section{Introduction}

\LaTeX{} provides a mechanism to structure a large document (such as a book)
into a main file and several child files (containing the chapters)
using the |\include| command.
This mechanism is beneficial for documents
which span hundreds of pages in order to
make the source file(s) more manageable.
Moreover, compilation can be restricted to
selected child files by means of the |\includeonly| command.
The latter feature can be used to reduce the compilation time while editing
(this was significantly more useful in the earlier days of \LaTeX{})
or to generate a smaller document which is easier to navigate.
Another application of |\includeonly| is to generate
documents consisting of selected parts of the complete document.

However, there are a few drawbacks of the plain |\include| mechanism:
\begin{itemize}
\item
The child files cannot be compiled on their own,
they can only be compiled via the main file.
A naive editing environment
(such as a text editor with an option
to have the current file processed by \LaTeX)
may require one to switch to the main file before compiling;
attempting to compile the child file produces errors.
\item
The main file must be modified (each time)
to adjust the |\includeonly| command
to the present needs. This easily leaves the main file in a messy state.
\item
The generated document will always carry the filename
of the main document. This is inconvenient if
several child files are to be compiled and
to be kept for distribution.
\end{itemize}

The present package provides a simple interface
to make child files individually compilable by \LaTeX{}.
Compiling a child file then has the same effect as compiling
the main file with an |\includeonly| command
to select the appropriate child.
Moreover the generated document will carry the name of the child
rather than the main file.
This resolves all three above issues.

This feature is meant to make the editing of books,
thesis documents and lecture notes somewhat more convenient.
However, the package can also be used efficiently for
composing a series of documents (such as exercise sheets)
which are typically distributed individually.
It then assists the author in generating the individual documents
(potentially in different versions)
as well as a document containing the collected series.
Another application is in developing style files
or other kinds of included material
where compilation of the style file could redirect
to a sample or test file.

%%%%%%%%%%%%%%%%%%%%%%%%%%%%%%%%%%%%%%%%%%%%%%%%%%%%%%%%%%%%%%%%%%%%%%%%%%%%%%%%
%%%%%%%%%%%%%%%%%%%%%%%%%%%%%%%%%%%%%%%%%%%%%%%%%%%%%%%%%%%%%%%%%%%%%%%%%%%%%%%%
\section{Usage}

First of all, the package \textsf{childdoc} is \emph{not} a standard
\LaTeXe{} |.sty| style file! Therefore it needs to be invoked in
a non-standard way.

%%%%%%%%%%%%%%%%%%%%%%%%%%%%%%%%%%%%%%%%%%%%%%%%%%%%%%%%%%%%%%%%%%%%%%%%%%%%%%%%
\subsection{Included Files}
\label{sec:include}

%%%%%%%%%%%%%%%%%%%%%%%%%%%%%%%%%%%%%%%%
\DescribeMacro{\childdocmain}
To use the package, add the commands
\begin{center}
\begin{tabular}{l}
|\input{childdoc.def}|\\
|\childdocmain{}|\\
\end{tabular}
\end{center}
at the very top of the main \LaTeX{} file,
in particular \emph{before} the |\documentclass| statement!
The argument of |\childdocmain| should be left empty
(but it must be present).

%%%%%%%%%%%%%%%%%%%%%%%%%%%%%%%%%%%%%%%%
\DescribeMacro{\childdocof}
Furthermore, add the commands
\begin{center}
\begin{tabular}{l}
|\input{childdoc.def}|\\
|\childdocof{|\textit{main}|}|\\
\end{tabular}
\end{center}
at the top of every child file \textit{child}
which is included by |\include{|\textit{child}|}|
from within the main file
(or at least for those files to be compiled individually).
The argument \textit{main} must be the filename of the main file.

There are a couple of
considerations in setting up the main and child documents:

%%%%%%%%%%%%%%%%%%%%%%%%%%%%%%%%%%%%%%%%
\paragraph{Restrictions.}

Please note the following restrictions:
\begin{itemize}
\item
|\childdocmain| must be called with one argument \textit{main}
to ensure compatibility with earlier version of the package.
It must either be empty (|\childdocmain{}|)
or precisely match the filename of the main file in which it is specified.
See \secref{sec:detection} for further information.
\item
The filename \textit{main} must be specified without the |.tex| extension.
\item
The filename \textit{main} is case sensitive
(even in case-insensitive file systems)
due to internal string comparison.
\item
The argument \textit{main} should be fully expanded, it cannot be a macro.
\item
Subdirectories and special characters should be avoided in filenames.
\item
The command |\childdocmain{|\textit{main}|}| must be followed by a whitespace.
It should not be followed immediately by another command
or by a comment mark `|%|'.
This is because the \TeX{} parser reads the token immediately following
the argument of |\childdocmain| and puts it
at the beginning of every child section;
however, a white\-space is ignored.
\end{itemize}

%%%%%%%%%%%%%%%%%%%%%%%%%%%%%%%%%%%%%%%%
\paragraph{Content of Main File.}

It is advisable to place all content in the child files included by |\include|.
Any output contained in the main file will appear in all child documents
unless suppressed manually;
it cannot be suppressed automatically by the |\includeonly| directive
and thus should normally be avoided.
A method to include some content in the main file
by means of conditional processing is described in \secref{sec:conditional}.

%%%%%%%%%%%%%%%%%%%%%%%%%%%%%%%%%%%%%%%%
\paragraph{Page Numbering.}

When only a part of the document is compiled,
the appropriate numbering of pages
(as well as other status parameters)
is determined from the |.aux| files.
The latter contain information from previous passes.
However this information needs to propagate through
all intermediate child documents.
Therefore the page numbering in child documents may well
be inconsistent until the complete document is compiled at least once.

A useful (if unconventional) way to always ensure a consistent
page numbering is to restart the numbering in each child document
and denote the pages by `\textit{child}|.|\textit{page}'
where \textit{child} represents the chapter/section number of the child file.
This can be achieved by the command
|\numberwithin{page}{|\textit{child}|}|
of the \textsf{amsmath} package
where \textit{child} can be |chapter| or |section|
depending on the chosen structuring.
Alternatively, one can modify the macro |\thepage| appropriately
and reset the counter |page| at the start of each child file.

%%%%%%%%%%%%%%%%%%%%%%%%%%%%%%%%%%%%%%%%%%%%%%%%%%%%%%%%%%%%%%%%%%%%%%%%%%%%%%%%
\subsection{Conditional Processing}
\label{sec:conditional}

The package provides a mechanism to compile different versions
of a document. To customise the versions further some conditional processing
can come in handy to distinguish which version is being compiled.
The package provides two macros to describe the compilation context:

%%%%%%%%%%%%%%%%%%%%%%%%%%%%%%%%%%%%%%%%
\DescribeMacro{\ifchilddoc}
The conditional |\ifchilddoc| distinguishes between the compilation of
child documents and the main document:
%
\begin{center}
|\ifchilddoc |\textit{child-code}| |[|\||else |\textit{main-code}]| \||fi|
\end{center}

%%%%%%%%%%%%%%%%%%%%%%%%%%%%%%%%%%%%%%%%
\DescribeMacro{\childdocname}
\DescribeMacro{\childdocjob}
The macro |\childdocname| contains the filename (without extension)
of the main or child file being processed.
Note that |\childdocjob| will always contain the name of the main file.

%%%%%%%%%%%%%%%%%%%%%%%%%%%%%%%%%%%%%%%%
\paragraph{Title Page.}

Conditional processing can be used to include a title or banner page
in the main document when proper precautions are taken.
Importantly, the code in the main file should ensure that the page counter
(as well as other status parameters which are stored in the |.aux| files)
takes the same value after the conditional processing.
Otherwise the page numbers may take divergent values
depending on which part is compiled.

For example, a title page could be declared by:
%
\begin{center}
\begin{tabular}{l}
|\ifchilddoc\||else|\\
|\addtocounter{page}{-1}|\\
\textit{code for title page}\\
|\newpage|\\
|\||fi|
\end{tabular}
\end{center}
%
A banner page for the child documents can be generated by:
%
\begin{center}
\begin{tabular}{l}
|\ifchilddoc|\\
|\addtocounter{page}{-1}|\\
\textit{code for banner page}\\
|\newpage|\\
|\||fi|
\end{tabular}
\end{center}
%
Here one could write a message such as:
\begin{center}
|This is the part \childdocname{} of \childdocjob{}.|
\end{center}

%%%%%%%%%%%%%%%%%%%%%%%%%%%%%%%%%%%%%%%%%%%%%%%%%%%%%%%%%%%%%%%%%%%%%%%%%%%%%%%%
\subsection{Flags}
\label{sec:flags}

The package makes it easy to generate different versions
of the main or child documents.
To this end compilation flags can be defined
and assigned different default values.
They will be particularly useful in conjunction
with the forwarding mechanism described in \secref{sec:forward}.

For example, it may be useful to have a flag |\version|
which can be set to |draft| or |final|.
The document source will contain some conditional code
depending on the value of |\version|.
Suppose further, the flag should default to |final| for the main file
and to |draft| for child files
which is a natural assignment for editing the document.
This is achieved by placing the following code
in the preamble of the main document
(below the |\childdocmain| directive):
%
\begin{center}
\begin{tabular}{l}
|\ifchilddoc|\\
|\providecommand{\version}{draft}|\\
|\||else|\\
|\providecommand{\version}{final}|\\
|\||fi|
\end{tabular}
\end{center}
%
The definition by |\providecommand| makes sure
that previous definitions are not overwritten.
Further statements |\providecommand{\version}{...}|
can thus be added before the above code to override it.

For the main file, one might add a line
(between |\childdocmain| and the above block)
%
\begin{center}
|%\ifchilddoc\||else\providecommand{\version}{draft}\||fi|
\end{center}
%
which can be uncommented to produce a draft version.
Likewise one can add a line to the very top of a child file
(above the |\childdocof{|\textit{main}|}| directive)
%
\begin{center}
|%\providecommand{\version}{final}|
\end{center}
%
which can be uncommented to produce the final version of this child document.

%%%%%%%%%%%%%%%%%%%%%%%%%%%%%%%%%%%%%%%%%%%%%%%%%%%%%%%%%%%%%%%%%%%%%%%%%%%%%%%%
\subsection{Forwarding}
\label{sec:forward}

Different versions of the main or child documents
using compilation flags as described in \secref{sec:flags}
can be (permanently) stored in different files
for convenient compilation, viewing and distribution.
To this end, the package defines a command
to pass on compilation to a different file:

%%%%%%%%%%%%%%%%%%%%%%%%%%%%%%%%%%%%%%%%
\DescribeMacro{\childdocforward}
The command |\childdocforward| redirects processing to
another source file:
%
\begin{center}
\begin{tabular}{l}
|\input{childdoc.def}|\\
|\childdocforward[|\textit{main}|]{|\textit{dest}|}|\\
\end{tabular}
\end{center}
%
The argument \textit{dest} is the destination file
(without extension).
It should be the main file or one of the child files.
Note that further \textsf{childdoc} directives
such as |\childdocof| and |\childdocforward|
in the indicated file will be processed in this form.
The optional argument \textit{main}
passes on directly to the main file \textit{main}
while pretending to compile the child \textit{dest}.
This form behaves as if \textit{dest}
issues |\childdocof{|\textit{main}|}| right away,
and no further \textsf{childdoc} directives will be processed.

%%%%%%%%%%%%%%%%%%%%%%%%%%%%%%%%%%%%%%%%
\DescribeMacro{\...prefix}
In the alternative form |\childdocforwardprefix|,
%
\begin{center}
\begin{tabular}{l}
|\input{childdoc.def}|\\
|\childdocforwardprefix[|\textit{main}|]{|\textit{prefix}|}{|\textit{dest}|}|
\end{tabular}
\end{center}
%
the destination file is determined by a pattern
depending on the current file:
To make this work, the current file must be called
`{\textit{prefix}\hspace{0.2em}\textit{suffix}}'
with \textit{prefix} matching precisely the argument.
Processing is then passed on to the file
`{\textit{dest}\hspace{0.2em}\textit{suffix}}'.
Surely, the same effect is achieved by
directly specifying the
argument `{\textit{dest}\hspace{0.2em}\textit{suffix}}'
in the first form.
However, that requires to set up a different file
for each child. With the alternative form of the command
all these files can have exactly the same content
which simplifies setting them up and maintaining them.

For example, the following file |draft.tex|
with a compilation flag |\version| as described in \secref{sec:flags}
compiles the main document as a draft:
%
\begin{center}
\begin{tabular}{l}
|\def\version{draft}|\\
|\input{childdoc.def}|\\
|\childdocforward{|\textit{main}|}|
\end{tabular}
\end{center}
%
Likewise, the following files |final|\textit{nn}|.tex|
compile the final version of the child document
|child|\textit{nn}|.tex|:
%
\begin{center}
\begin{tabular}{l}
|\def\version{final}|\\
|\input{childdoc.def}|\\
|\childdocforwardprefix{final}{child}|
\end{tabular}
\end{center}
%

Note that when several versions of a main file and/or of each child file
are to be generated, it may be convenient to set up a |Makefile| or
shell script to automatise the process.

%%%%%%%%%%%%%%%%%%%%%%%%%%%%%%%%%%%%%%%%%%%%%%%%%%%%%%%%%%%%%%%%%%%%%%%%%%%%%%%%
\subsection{Command Line Processing}
\label{sec:commandline}

The effect of redirection files can also be achieved by invoking
the \LaTeX{} compiler with a more elaborate command line.
Most conveniently this should be done as part
of a shell script or a |Makefile|.

When using \textsf{childdoc} in the main file, the following
command lines effectively perform a redirection
(note that depending on the shell being used,
backslashes may have to be doubled: `|\|' $\to$ `|\\|'):
%
\begin{center}
|... -jobname "|\textit{target}|" |\\|"|[\textit{flags}]%
|\input{childdoc.def}\childdocforward[|\textit{main}|]{|\textit{dest}|}"|
\end{center}
%
Here \textit{target} is the name of the output file,
\textit{main} is the name of the main file
and \textit{dest} is the name of the main or child file to be processed
(all filenames without extensions).
The optional argument \textit{main} can be omitted
if \textit{main} matches \textit{dest}.
Optionally, compilation \textit{flags} can be defined via |\def| commands.
This command line makes the \TeX{} engine believe
it is compiling the file \textit{target}
whose content is specified as the latter parameter.
The provided code then forwards the processing to
\textit{main} or \textit{dest} as described in \secref{sec:forward}.

%%%%%%%%%%%%%%%%%%%%%%%%%%%%%%%%%%%%%%%%%%%%%%%%%%%%%%%%%%%%%%%%%%%%%%%%%%%%%%%%
\subsection{Include by Input}
\label{sec:input}

Including child documents by |\include| has some restrictions by design.
Most notably, the content of a child document always occupies
its own set of pages; pages cannot be shared between child documents.
Usually, this behaviour makes perfect sense
because each child document contain an essential part of the document.
However, in some situations it may be desirable to compose
a document from a collection of parts
without having mandatory page breaks between then.
For this case, the package
provides a mechanism to include parts
by |\input| which can also be processed individually.
However, by construction this mechanism
requires manual handling of the content to be output.

%%%%%%%%%%%%%%%%%%%%%%%%%%%%%%%%%%%%%%%%
\DescribeMacro{\ifchilddocmanual}
The main file should be prepared as usual, see \secref{sec:include}.
However, the document body must make a distinction
between processing of an individual part and of the main document, e.g.:
%
\begin{center}
\begin{tabular}{l}
|\ifchilddocmanual|\\
|\input{\childdocname}|\\
|\||else|\\
\textit{document body with }|\input{|\textit{part}|}|\\
|\||fi|
\end{tabular}
\end{center}
%
The conditional |\ifchilddocmanual| is true whenever
a part to be included by |\input| is being compiled,
and the name of the part is stored in |\childdocname|.

%%%%%%%%%%%%%%%%%%%%%%%%%%%%%%%%%%%%%%%%
\DescribeMacro{\childdocby}
Each part to be included by |\input| should start with:
%
\begin{center}
\begin{tabular}{l}
|\input{childdoc.def}|\\
|\childdocby{|\textit{main}|}|\\
\end{tabular}
\end{center}
%
The directive |\childdocby| is similar to |\childdocof|
described in \secref{sec:include},
but the subsequent selection of content must be done manually.
To that end, both |\ifchilddoc| and |\ifchilddocmanual|
will be true upon processing of a part,
and the name of the part is stored in |\childdocname|.
Note that |\jobname| will be set to the filename of the current part
so that each part receives an individual |.aux| file
that does not interfere with the |.aux| file(s) of the main document.
This behaviour can be altered by the alternative form
|\childdocby[*]{|\textit{main}|}| (with a non-empty optional argument)
which uses the |.aux| file of the main document
by setting |\jobname| to \textit{main}.

%%%%%%%%%%%%%%%%%%%%%%%%%%%%%%%%%%%%%%%%%%%%%%%%%%%%%%%%%%%%%%%%%%%%%%%%%%%%%%%%
\subsection{Driver Development}
\label{sec:driver}

The \textsf{childdoc} mechanism can also be use for the development
of definition files such as \LaTeX{} styles or classes.
This case differs from the above setup with multiple parts
included by |\include| in that no |\includeonly| should be invoked.
This can be achieved by starting the include file
(before |\ProvidesPackage|) with:
%
\begin{center}
\begin{tabular}{l}
|\input{childdoc.def}|\\
|\childdocforward{|\textit{main}|}|\\
\end{tabular}
\end{center}
%
or alternatively with:
%
\begin{center}
\begin{tabular}{l}
|\input{childdoc.def}|\\
|\childdocby{|\textit{main}|}|\\
\end{tabular}
\end{center}
%
Both forms have slightly different effects as described above.
The main file is prepared as usual, see \secref{sec:include}.

%%%%%%%%%%%%%%%%%%%%%%%%%%%%%%%%%%%%%%%%%%%%%%%%%%%%%%%%%%%%%%%%%%%%%%%%%%%%%%%%
\subsection{Legacy Detection}
\label{sec:detection}

The directive |\childdocmain| in the main file can detect
whether the complete document or merely a child is to be compiled
even without using the directive |\childdocof|.
This method is deprecated because it is less robust
and there is no compelling reason to use it;
it is merely provided for backward compatibility
and it may be removed in future versions.

If the detection mechanism is to be used,
it is mandatory to correctly specify
the filename of the main file as the argument of |\childdocmain|:
%
\begin{center}
\begin{tabular}{l}
|\input{childdoc.def}|\\
|\childdocmain{|\textit{main}|}|\\
\end{tabular}
\end{center}
%
If |\jobname| does not match the argument \textit{main} of |\childdocmain|,
it is assumed that |\jobname| points to the child file to be compiled.
When using |\childdocmain| with the main file specified as argument,
it suffices to start a child file
with just |\input{|\textit{main}|}|
without loading of the package and using |\childdocof|.
If instead all processing is done
with the appropriate \textsf{childdoc} directives,
the argument of \textit{main} of |\childdocmain| can be empty.

An alternative version of the command line processing described
in \secref{sec:commandline} using the detection mechanism reads:
%
\begin{center}
|... -jobname "|\textit{target}|" "|[\textit{flags}]%
[|\def\jobname{|\textit{dest}|}|]|\input{|\textit{main}|}"|
\end{center}

%%%%%%%%%%%%%%%%%%%%%%%%%%%%%%%%%%%%%%%%%%%%%%%%%%%%%%%%%%%%%%%%%%%%%%%%%%%%%%%%
\subsection{Manual Code}
\label{sec:manual}

In case one cannot be certain whether the definitions file |childdoc.def|
is installed on the target \TeX{} distribution
and one prefers not to ship it,
it is conceivable to paste a few relevant commands into the sources.

To that end, drop all statements |\input{childdoc.def}|
and perform the replacements as outlined below.
Instead of |\childdocmain{|\textit{main}|}| add the following code
to the top of the main file:
%
\begin{center}
\begin{tabular}{l}
|\||ifdefined\childdocname\endinput\||fi\newif\ifchilddoc|\\
|\edef\childdocname{\scantokens\expandafter{\jobname\noexpand}}|\\
|\def\childdocmain{|\textit{main}|}\||ifx\childdocmain\childdocname\||else|\\
|\childdoctrue\includeonly{\childdocname}\let\jobname\childdocmain\||fi|\\
\end{tabular}
\end{center}
%
Instead of |\childdocof{|\textit{main}|}| just include the main file
at the top of each child file:
%
\begin{center}
|\input{|\textit{main}|}|
\end{center}
%
A simple redirection |\childdocforward{|\textit{dest}|}| is achieved by:
%
\begin{center}
|\def\jobname{|\textit{dest}|}\input{\jobname}|
\end{center}
%
The redirection with prefix
|\childdocforwardprefix[|\textit{prefix}|]{|\textit{dest}|}|
is accomplished by:
%
\begin{center}
\begin{tabular}{l}
|{\edef\jobname{\scantokens\expandafter{\jobname\noexpand}}|\\
|\def\redirectjob |\textit{prefix}|#1~~~{\gdef\jobname{|\textit{dest}|#1}}|\\
|\expandafter\redirectjob\jobname~~~}\input{\jobname}|
\end{tabular}
\end{center}

In an alternative approach,
child documents can be compiled by a specific command line
without additional code or specific definitions:
%
\begin{center}
|... -jobname "|\textit{target}|" "|[\textit{flags}]%
|\includeonly{|\textit{dest}|}\input{|\textit{main}|}"|
\end{center}
%

%%%%%%%%%%%%%%%%%%%%%%%%%%%%%%%%%%%%%%%%%%%%%%%%%%%%%%%%%%%%%%%%%%%%%%%%%%%%%%%%
%%%%%%%%%%%%%%%%%%%%%%%%%%%%%%%%%%%%%%%%%%%%%%%%%%%%%%%%%%%%%%%%%%%%%%%%%%%%%%%%
\section{Information}

%%%%%%%%%%%%%%%%%%%%%%%%%%%%%%%%%%%%%%%%%%%%%%%%%%%%%%%%%%%%%%%%%%%%%%%%%%%%%%%%
\subsection{Copyright}

Copyright \copyright{} 2017--2018 Niklas Beisert

This work may be distributed and/or modified under the
conditions of the \LaTeX{} Project Public License, either version 1.3
of this license or (at your option) any later version.
The latest version of this license is in
  \url{http://www.latex-project.org/lppl.txt}
and version 1.3 or later is part of all distributions of \LaTeX{}
version 2005/12/01 or later.

This work has the LPPL maintenance status `maintained'.

The Current Maintainer of this work is Niklas Beisert.

This work consists of the files |README.txt|, |childdoc.ins| and |childdoc.dtx|
as well as the derived files |childdoc.def|, |cdocsamp.tex|
with |cdocsch1.tex|, |cdocsch2.tex|, |cdocspt3.tex|, |cdocspt4.tex|,
|cdocsdrf.tex|, |cdocsfn1.tex|, |cdocsfn2.tex|
as well as |childdoc.pdf|.

%%%%%%%%%%%%%%%%%%%%%%%%%%%%%%%%%%%%%%%%%%%%%%%%%%%%%%%%%%%%%%%%%%%%%%%%%%%%%%%%
\subsection{Files and Installation}

The package consists of the files:
%
\begin{center}
\begin{tabular}{ll}
    |README.txt|   & readme file \\
    |childdoc.ins| & installation file \\
    |childdoc.dtx| & source file \\
    |childdoc.def| & definition file \\
    |cdocsamp.tex| & sample main file \\
    |cdocsch1.tex| & sample include file \\
    |cdocsch2.tex| & sample include file \\
    |cdocspt3.tex| & sample part file \\
    |cdocspt4.tex| & sample part file \\
    |cdocsdrf.tex| & sample redirection file \\
    |cdocsfn1.tex| & sample redirection file \\
    |cdocsfn2.tex| & sample redirection file \\
    |childdoc.pdf| & manual
\end{tabular}
\end{center}
%
The distribution consists of the files
|README.txt|, |childdoc.ins| and |childdoc.dtx|.
%
\begin{itemize}
\item
Run (pdf)\LaTeX{} on |childdoc.dtx|
to compile the manual |childdoc.pdf| (this file).
\item
Run \LaTeX{} on |childdoc.ins| to create the definitions file |childdoc.def|
and the sample |cdocsamp.tex| with include files
|cdocsch1.tex|, |cdocsch2.tex|, |cdocspt3.tex|, |cdocspt4.tex|,
|cdocsdrf.tex|, |cdocsfn1.tex|, |cdocsfn2.tex|.
Then copy the file |childdoc.def| to an appropriate directory of your \LaTeX{}
distribution, e.g.\ \textit{texmf-root}|/tex/latex/childdoc|.
\end{itemize}

%%%%%%%%%%%%%%%%%%%%%%%%%%%%%%%%%%%%%%%%%%%%%%%%%%%%%%%%%%%%%%%%%%%%%%%%%%%%%%%%
\subsection{Related CTAN Packages}

There are several other packages which offer a similar functionality:
%
\begin{itemize}
\item
The packages
\href{http://ctan.org/pkg/docmute}{\textsf{docmute}},
\href{http://ctan.org/pkg/includex}{\textsf{includex}} and
\href{http://ctan.org/pkg/standalone}{\textsf{standalone}}
provide commands to include only the document body of
a child file thus allowing both files to be compiled individually.
\item
The packages \href{http://ctan.org/pkg/subdocs}{\textsf{subdocs}}
and \href{http://ctan.org/pkg/subfiles}{\textsf{subfiles}}
provide structures in which the main and child documents can be
encapsulated and allowing them to be compiled individually.
The inclusion mechanism is different from the conventional |\include|.
\item
The package \href{http://ctan.org/pkg/combine}{\textsf{combine}}
is an elaborate solution to combine several documents into one.
\end{itemize}
%
See also the CTAN topic \href{http://ctan.org/topic/subdocs}{\textsf{subdocs}}
for further related packages.
The present package differs from the above solutions in that
a document structure constructed with the conventional |\include| mechanism
just needs two extra commands at the top of every file
such that all constituent files can be compiled individually.

%%%%%%%%%%%%%%%%%%%%%%%%%%%%%%%%%%%%%%%%%%%%%%%%%%%%%%%%%%%%%%%%%%%%%%%%%%%%%%%%
%\subsection{Feature Suggestions}
%
%The following is a list of features which may be useful for future
%versions of this package:
%%
%\begin{itemize}
%\item
%\ldots
%\end{itemize}

%%%%%%%%%%%%%%%%%%%%%%%%%%%%%%%%%%%%%%%%%%%%%%%%%%%%%%%%%%%%%%%%%%%%%%%%%%%%%%%%
\subsection{Revision History}

%%%%%%%%%%%%%%%%%%%%%%%%%%%%%%%%%%%%%%%%
\paragraph{v2.0:} 2018/12/30

\begin{itemize}
\item
immediate forward processing
\item
added |\childdocby| mechanism
\item
manual restructured
\end{itemize}

%%%%%%%%%%%%%%%%%%%%%%%%%%%%%%%%%%%%%%%%
\paragraph{v1.6:} 2018/01/17

\begin{itemize}
\item
application for development of include files
\item
corrections to manual
\end{itemize}

%%%%%%%%%%%%%%%%%%%%%%%%%%%%%%%%%%%%%%%%
\paragraph{v1.5:} 2017/05/21

\begin{itemize}
\item
more complete structuring introduced
\item
|\childdocof| introduced
\item
|\childdoc| renamed to |\childdocmain|
\item
|\childredirect| renamed to |\childdocforward| and |\childdocforwardprefix|
and functionality expanded
\end{itemize}

%%%%%%%%%%%%%%%%%%%%%%%%%%%%%%%%%%%%%%%%
\paragraph{v1.0:} 2017/04/27

\begin{itemize}
\item
manual and install package
\item
first version published on CTAN
\end{itemize}

%%%%%%%%%%%%%%%%%%%%%%%%%%%%%%%%%%%%%%%%
\paragraph{v0.6:} 2017/04/26

\begin{itemize}
\item
redirection mechanism added
\end{itemize}

%%%%%%%%%%%%%%%%%%%%%%%%%%%%%%%%%%%%%%%%
\paragraph{v0.5:} 2017/04/26

\begin{itemize}
\item
functionality in definition file
\end{itemize}


%%%%%%%%%%%%%%%%%%%%%%%%%%%%%%%%%%%%%%%%%%%%%%%%%%%%%%%%%%%%%%%%%%%%%%%%%%%%%%%%
%%%%%%%%%%%%%%%%%%%%%%%%%%%%%%%%%%%%%%%%%%%%%%%%%%%%%%%%%%%%%%%%%%%%%%%%%%%%%%%%
%%%%%%%%%%%%%%%%%%%%%%%%%%%%%%%%%%%%%%%%%%%%%%%%%%%%%%%%%%%%%%%%%%%%%%%%%%%%%%%%
\appendix

\settowidth\MacroIndent{\rmfamily\scriptsize 000\ }

 \DocInput{childdoc.dtx}

\end{document}
%</driver>
% \fi
%
% %%%%%%%%%%%%%%%%%%%%%%%%%%%%%%%%%%%%%%%%%%%%%%%%%%%%%%%%%%%%%%%%%%%%%%%%%%%%%%
% %%%%%%%%%%%%%%%%%%%%%%%%%%%%%%%%%%%%%%%%%%%%%%%%%%%%%%%%%%%%%%%%%%%%%%%%%%%%%%
% \section{Sample}
%\iffalse
%<*samplemain>
%\fi
%
% The following presents a sample document
% with two chapters, two parts, a title page,
% a compile flag as well as three forwarding files to set the flag.
% It consists of eight |.tex| files:
% \begin{center}
% \begin{tabular}{ll}
% |cdocsamp.tex|&main file\\
% |cdocsch1.tex|&include file for chapter 1\\
% |cdocsch2.tex|&include file for chapter 2\\
% |cdocspt3.tex|&include file for part 3\\
% |cdocspt4.tex|&include file for part 4\\
% |cdocsdrf.tex|&forwarding file for main file in draft mode\\
% |cdocsfi1.tex|&forwarding file for final version of chapter 1\\
% |cdocsfi2.tex|&forwarding file for final version of chapter 2\\
% \end{tabular}
% \end{center}
% Each of the eight files can be compiled directly by the \LaTeX{} compiler.
%
% %%%%%%%%%%%%%%%%%%%%%%%%%%%%%%%%%%%%%%
% \paragraph{Main File.}
%
% The main file is called |cdocsamp.tex|.
%
% Load the \textsf{childdoc} definitions and
% declare the filename for the main document:
%    \begin{macrocode}
\input{childdoc.def}
\childdocmain{}
%    \end{macrocode}

% Optional override for |\version| flag:
%    \begin{macrocode}
%%\ifchilddoc\else\providecommand{\version}{draft}\fi
%    \end{macrocode}

% Define the default values for the |\version| flag
% (|final| for the main file and |draft| for childs):
%    \begin{macrocode}
\ifchilddoc
\providecommand{\version}{draft}
\else
\providecommand{\version}{final}
\fi
%    \end{macrocode}

% Load the standard document class:
%    \begin{macrocode}
\documentclass[12pt]{article}
%    \end{macrocode}

% Start the document body:
%    \begin{macrocode}
\begin{document}
%    \end{macrocode}

% Declare a title page.
% Print title, part of document being processed and version flag:
%    \begin{macrocode}
\addtocounter{page}{-1}
\begin{center}
{\LARGE\bfseries{}childdoc example\par}
\vspace{1cm}
\ifchilddoc
\ifchilddocmanual part\else chapter\fi:
`\childdocname' of `\childdocjob'\par
\else
main document: `\childdocjob'\par
\fi
version: \version\par
\end{center}
\newpage
%    \end{macrocode}

% Manually include selected file,
% otherwise process as usual:
%    \begin{macrocode}
\ifchilddocmanual
\section*{part `\childdocname'}
\input{\childdocname}
\else
%    \end{macrocode}

% Include the two chapters:
%    \begin{macrocode}
\include{cdocsch1}
\include{cdocsch2}
%    \end{macrocode}

% Include the two parts unless only chapters should be displayed:
%    \begin{macrocode}
\ifchilddoc\else
\section{part three}
\input{cdocspt3}
\section{part four}
\input{cdocspt4}
\fi
%    \end{macrocode}

% Process as usual until here:
%    \begin{macrocode}
\fi
%    \end{macrocode}

% End of document body:
%    \begin{macrocode}
\end{document}
%    \end{macrocode}
%\iffalse
%</samplemain>
%\fi
%
% %%%%%%%%%%%%%%%%%%%%%%%%%%%%%%%%%%%%%%
% \paragraph{Chapter Include Files.}
%
% The include files are called |cdocsch1.tex| and |cdocsch2.tex|.
%
%\iffalse
%<*samplechap1|samplechap2>
%\fi

% Optional override for |\version| flag:
%    \begin{macrocode}
%%\providecommand{\version}{final}
%    \end{macrocode}

% Include the main document:
%    \begin{macrocode}
\input{childdoc.def}
\childdocof{cdocsamp}
%    \end{macrocode}

%\iffalse
%</samplechap1|samplechap2>
%\fi
%
%\iffalse
%<*samplechap1>
%\fi
% Some text for chapter 1:
%    \begin{macrocode}
\section{one}
some text in chapter one
%    \end{macrocode}

%\iffalse
%</samplechap1>
%\fi
% Some text for chapter 2:
%\iffalse
%<*samplechap2>
%\fi
%    \begin{macrocode}
\section{two}
more text in chapter two
%    \end{macrocode}

%\iffalse
%</samplechap2>
%\fi
%
% %%%%%%%%%%%%%%%%%%%%%%%%%%%%%%%%%%%%%%
% \paragraph{Part Include Files.}
%
% The include files are called |cdocspt3.tex| and |cdocspt4.tex|.
%
%\iffalse
%<*samplepart3|samplepart4>
%\fi

% Optional override for |\version| flag:
%    \begin{macrocode}
%%\providecommand{\version}{final}
%    \end{macrocode}

% Include the main document:
%    \begin{macrocode}
\input{childdoc.def}
\childdocby{cdocsamp}
%    \end{macrocode}

%\iffalse
%</samplepart3|samplepart4>
%\fi
%
%\iffalse
%<*samplepart3>
%\fi
% Some text for part 3:
%    \begin{macrocode}
some text in part three
%    \end{macrocode}

%\iffalse
%</samplepart3>
%\fi
% Some text for part 4:
%\iffalse
%<*samplepart4>
%\fi
%    \begin{macrocode}
more text in part four
%    \end{macrocode}

%\iffalse
%</samplepart4>
%\fi
%
% %%%%%%%%%%%%%%%%%%%%%%%%%%%%%%%%%%%%%%
% \paragraph{Forwarding for a Complete Draft.}
%
% The following forwarding file |cdocsdrf.tex|
% compiles the main document in draft mode:
%\iffalse
%<*sampledraft>
%\fi
%    \begin{macrocode}
\def\version{draft}
\input{childdoc.def}
\childdocforward{cdocsamp}
%    \end{macrocode}

%\iffalse
%</sampledraft>
%\fi
%
% %%%%%%%%%%%%%%%%%%%%%%%%%%%%%%%%%%%%%%
% \paragraph{Forwarding for Final Version of the Chapters.}
%
% The following forwarding files |cdocsfn1.tex| and |cdocsfn2.tex|
% (with identical content)
% compile the final versions of the child documents
% |cdocsch1.tex| and |cdocsch2.tex|, respectively:
%\iffalse
%<*samplefinal>
%\fi
%    \begin{macrocode}
\def\version{final}
\input{childdoc.def}
\childdocforwardprefix[cdocsamp]{cdocsfn}{cdocsch}
%    \end{macrocode}

%\iffalse
%</samplefinal>
%\fi
%
% %%%%%%%%%%%%%%%%%%%%%%%%%%%%%%%%%%%%%%
% \paragraph{Command Line Processing.}
%
% The following three command lines generate the output files
% |cdocscld|, |cdocscl1| and |cdocscl2|
% which should be identical to
% |cdocsdrf|, |cdocsch1| and |cdocsfn2|, respectively:
% \begin{center}
% \begin{tabular}{l}
% |latex -jobname cdocscld \|\\
% |  "\def\version{draft}\input{childdoc.def}\childdocforward{cdocsamp}"|\\
% |latex -jobname cdocscl1 \|\\
% |  "\input{childdoc.def}\childdocforward[cdocsamp]{cdocsch1}"|\\
% |latex -jobname cdocscl2 \|\\
% |  "\def\version{final}\input{childdoc.def}\childdocforward{cdocsch2}"|
% \end{tabular}
% \end{center}
% Note that the trailing backslash on each first line
% merely continues the input to the second line
% (for convenient cut ant paste).
% Furthermore, the command |latex| can be replaced by any
% of its alternative versions such as |pdflatex|.
%
% %%%%%%%%%%%%%%%%%%%%%%%%%%%%%%%%%%%%%%%%%%%%%%%%%%%%%%%%%%%%%%%%%%%%%%%%%%%%%%
% %%%%%%%%%%%%%%%%%%%%%%%%%%%%%%%%%%%%%%%%%%%%%%%%%%%%%%%%%%%%%%%%%%%%%%%%%%%%%%
% \section{Implementation}
%\iffalse
%<*package>
%\fi
%
% This section describes the definitions file |childdoc.def|.

% The definitions cannot be loaded using |\usepackage| or |\RequirePackage|
% which has a mechanism to prevent loading a style file more than once.
% When loading the definitions by means of |\input|
% multiple instances have to be prevented manually:
%\iffalse
%This code needs to be before the `\ProvidesFile' directive
%which is defined at the beginning of this file.
%Therefore it is also placed there and commented out here.
%</package>
%<*discard>
%\fi
%    \begin{macrocode}
\ifdefined\childdocmain\endinput\fi
%    \end{macrocode}
%\iffalse
%</discard>
%<*package>
%\fi
%
% \macro{\ifchilddoc}
% \macro{\ifchilddocmanual}
% The conditional |\ifchilddoc| tells whether a
% child (true) or main (false) document is being compiled.
% The conditional |\ifchilddocmanual| tells whether
% the |\includeonly| mechanism is used (false) or
% the selection of child files must be performed manually (true).
% The definitions initialise to false:
%    \begin{macrocode}
\newif\ifchilddoc
\newif\ifchilddocmanual
%    \end{macrocode}

% \macro{\childdocname}
% \macro{\childdocjob}
% The macro |\childdocname| stores the name of the main document
% to be compiled. The macro |\childdocjob| stores the name of
% the document on which the \LaTeX{} compiler was originally invoked.
% The content of |\jobname| cannot be compared
% to filenames specified in the source due to different catcodes.
% The following code rescans |\jobname|, stores the result
% in |\childdocname| and saves a copy in |\childdocjob|:
%    \begin{macrocode}
\edef\childdocname{\scantokens\expandafter{\jobname\noexpand}}
\let\childdocjob\childdocname
%    \end{macrocode}

% \macro{\childdocdisable}
% The macro |\childdocdisable| prevents the main file
% from being processed more than once.
% At this stage, the main document command |\childdocmain|
% is assumed to be called once again where it should do nothing.
% Any subsequent call to it should prevent
% a secondary processing of the main document
% It overwrites the forwarding commands
% |\childdocof| and |\childdocforward|
% with empty macros to prevent further inclusions of the main document:
%    \begin{macrocode}
\newcommand{\childdocdisable}
{
  \renewcommand{\childdocmain}[1]{\renewcommand{\childdocmain}[1]{\endinput}}
  \renewcommand{\childdocof}[1]{}
  \renewcommand{\childdocby}[2][]{}
  \renewcommand{\childdocforward}[2][]{}
  \renewcommand{\childdocdisable}{}
}
%    \end{macrocode}

% \macro{\childdocmain}
% The macro |\childdocmain| is to be called at the top of the main file
% with nothing or the main filename (without extension) as argument.
% First, it breaks loops.
% If the argument is not empty and does not match |\childdocname|
% (which is set by the first inclusion of |childdoc.def|),
% |\ifchilddoc| is set to true, |\includeonly| is applied to the child file
% and |\jobname| is set to the main file
% (for proper handling of |.aux| files):
%    \begin{macrocode}
\newcommand{\childdocmain}[1]
{
  \childdocdisable\childdocmain{}
  \if?#1?\else
    \begingroup
      \def\childdoctmp{#1}
      \ifx\childdoctmp\childdocname
        \def\childdoctmp{}
      \else
        \def\childdoctmp
        {
          \childdoctrue
          \includeonly{\childdocname}
          \def\childdocjob{#1}
          \def\jobname{#1}
        }
      \fi
      \expandafter
    \endgroup
    \childdoctmp
  \fi
}
%    \end{macrocode}

% \macro{\childdocof}
% The command |\childdocof| redirects
% compilation to the main file |#1|.
%    \begin{macrocode}
\newcommand{\childdocof}[1]
{
  \childdocdisable
  \childdoctrue
  \includeonly{\childdocname}
  \def\jobname{#1}
  \def\childdocjob{#1}
  \input{#1}
}
%    \end{macrocode}

% \macro{\childdocby}
% The command |\childdocby| ....
%    \begin{macrocode}
\newcommand{\childdocby}[2][]
{
  \childdocdisable
  \childdoctrue
  \childdocmanualtrue
  \if?#1?\else
    \def\jobname{#2}
  \fi
  \def\childdocjob{#2}
  \input{#2}
  \endinput
}
%    \end{macrocode}

% \macro{\childdocforward}
% The command |\childdocforward| redirects
% compilation to the main file or
% (if the optional argument is given) a child file.
% Parameters are set as if the main file
% or a child file starting with |\childdocof| was compiled.
% Then compilation is handed over to the main file:
%    \begin{macrocode}
\newcommand{\childdocforward}[2][]
{
  \begingroup
    \if?#1?
      \def\childdoctmp
      {
        \def\childdocname{#2}
        \def\childdocjob{#2}
        \def\jobname{#2}
        \input{#2}
        \endinput
      }
    \else
      \def\childdoctmp
      {
        \childdocdisable
        \def\childdocname{#2}
        \childdoctrue
        \includeonly{#2}
        \def\childdocjob{#1}
        \def\jobname{#1}
        \input{#1}
        \endinput
      }
    \fi
    \expandafter
  \endgroup
  \childdoctmp
}
%    \end{macrocode}

% \macro{\childdocforwardprefix}
% The command |\childdocforwardprefix| redirects
% compilation to the main or a child file by means of a pattern.
% The prefix |#1| in the current filename is replaced by |#2|
% and the suffix of the current filename is kept
% (it is assumed that the filename does not contain the substring `|~~~|'
% which is used as a delimiter).
% Compilation is handed over to the new file by |\childdocforward|:
%    \begin{macrocode}
\newcommand{\childdocforwardprefix}[3][]
{
  \begingroup
    \def\childdocextract #2##1~~~{\def\childdoctmp{\childdocforward[#1]{#3##1}}}
    \expandafter\childdocextract\childdocname~~~
    \expandafter
  \endgroup
  \childdoctmp
}
%    \end{macrocode}

% \macro{\childdoc}
% The deprecated macro |\childdoc| is a legacy version of |\childdocmain|:
%    \begin{macrocode}
\newcommand{\childdoc}{\childdocmain}
%    \end{macrocode}

% \macro{\childdocredirect}
% The deprecated macro |\childdocredirect| is a legacy version
% of |\childdocforward| and |\childdocforwardprefix|:
%    \begin{macrocode}
\newcommand{\childdocredirect}[2][]
{
  \begingroup
    \if?#1?
      \def\childdoctmp{\childdocforward{#2}}
    \else
      \def\childdoctmp{\childdocforwardprefix{#1}{#2}}
    \fi
    \expandafter
  \endgroup
  \childdoctmp
}
%    \end{macrocode}

%\iffalse
%</package>
%\fi
%
\endinput

\childdocforward{cdocsamp}
%    \end{macrocode}

%\iffalse
%</sampledraft>
%\fi
%
% %%%%%%%%%%%%%%%%%%%%%%%%%%%%%%%%%%%%%%
% \paragraph{Forwarding for Final Version of the Chapters.}
%
% The following forwarding files |cdocsfn1.tex| and |cdocsfn2.tex|
% (with identical content)
% compile the final versions of the child documents
% |cdocsch1.tex| and |cdocsch2.tex|, respectively:
%\iffalse
%<*samplefinal>
%\fi
%    \begin{macrocode}
\def\version{final}
% \iffalse
%
% childdoc.dtx Copyright (C) 2017-2018 Niklas Beisert
%
% This work may be distributed and/or modified under the
% conditions of the LaTeX Project Public License, either version 1.3
% of this license or (at your option) any later version.
% The latest version of this license is in
%   http://www.latex-project.org/lppl.txt
% and version 1.3 or later is part of all distributions of LaTeX
% version 2005/12/01 or later.
%
% This work has the LPPL maintenance status `maintained'.
%
% The Current Maintainer of this work is Niklas Beisert.
%
% This work consists of the files childdoc.dtx and childdoc.ins
% and the derived files childdoc.def and cdocsamp.tex with
% cdocsch1.tex, cdocsch2.tex, cdocsdrf.tex, cdocsfn1.tex, cdocsfn2.tex.
%
%<package>\ifdefined\childdocmain\endinput\fi
%<package>\ProvidesFile{childdoc.def}[2018/12/30 v2.0 child document driver]
%<samplemain>\ProvidesFile{cdocsamp.tex}[2018/12/30 v2.0 sample for childdoc]
%<*driver>
%\ProvidesFile{childdoc.drv}[2018/12/30 v2.0 childdoc reference manual file]
\PassOptionsToClass{10pt,a4paper}{article}
\documentclass{ltxdoc}

\usepackage[margin=35mm]{geometry}
\usepackage{hyperref}
\usepackage{hyperxmp}
\usepackage[usenames]{color}

\hypersetup{colorlinks=true}
\hypersetup{pdfstartview=FitH}
\hypersetup{pdfpagemode=UseNone}
\hypersetup{pdfsource={}}
\hypersetup{pdflang={en-UK}}
\hypersetup{pdfcopyright={Copyright 2017-2018 Niklas Beisert.
  This work may be distributed and/or modified under the
  conditions of the LaTeX Project Public License, either version 1.3
  of this license or (at your option) any later version.}}
\hypersetup{pdflicenseurl={http://www.latex-project.org/lppl.txt}}
\hypersetup{pdfcontactaddress={ETH Zurich, ITP, HIT K,
  Wolfgang-Pauli-Strasse 27}}
\hypersetup{pdfcontactpostcode={8093}}
\hypersetup{pdfcontactcity={Zurich}}
\hypersetup{pdfcontactcountry={Switzerland}}
\hypersetup{pdfcontactemail={nbeisert@itp.phys.ethz.ch}}
\hypersetup{pdfcontacturl={http://people.phys.ethz.ch/\xmptilde nbeisert/}}

\newcommand{\secref}[1]{\hyperref[#1]{section \ref*{#1}}}

\parskip1ex
\parindent0pt
\let\olditemize\itemize
\def\itemize{\olditemize\parskip0pt}

\begin{document}

\title{The \textsf{childdoc} Package}
\hypersetup{pdftitle={The childdoc Package}}
\author{Niklas Beisert\\[2ex]
  Institut f\"ur Theoretische Physik\\
  Eidgen\"ossische Technische Hochschule Z\"urich\\
  Wolfgang-Pauli-Strasse 27, 8093 Z\"urich, Switzerland\\[1ex]
  \href{mailto:nbeisert@itp.phys.ethz.ch}
  {\texttt{nbeisert@itp.phys.ethz.ch}}}
\hypersetup{pdfauthor={Niklas Beisert}}
\hypersetup{pdfsubject={Manual for the LaTeX2e Package childdoc}}
\date{30 December 2018, \textsf{v2.0}}
\maketitle

\begin{abstract}\noindent
\textsf{childdoc} is a \LaTeXe{} package
that enables the direct compilation
of document sections included by |\include|
to individual files.
\end{abstract}

\begingroup
\parskip0ex
\tableofcontents
\endgroup

%%%%%%%%%%%%%%%%%%%%%%%%%%%%%%%%%%%%%%%%%%%%%%%%%%%%%%%%%%%%%%%%%%%%%%%%%%%%%%%%
%%%%%%%%%%%%%%%%%%%%%%%%%%%%%%%%%%%%%%%%%%%%%%%%%%%%%%%%%%%%%%%%%%%%%%%%%%%%%%%%
\section{Introduction}

\LaTeX{} provides a mechanism to structure a large document (such as a book)
into a main file and several child files (containing the chapters)
using the |\include| command.
This mechanism is beneficial for documents
which span hundreds of pages in order to
make the source file(s) more manageable.
Moreover, compilation can be restricted to
selected child files by means of the |\includeonly| command.
The latter feature can be used to reduce the compilation time while editing
(this was significantly more useful in the earlier days of \LaTeX{})
or to generate a smaller document which is easier to navigate.
Another application of |\includeonly| is to generate
documents consisting of selected parts of the complete document.

However, there are a few drawbacks of the plain |\include| mechanism:
\begin{itemize}
\item
The child files cannot be compiled on their own,
they can only be compiled via the main file.
A naive editing environment
(such as a text editor with an option
to have the current file processed by \LaTeX)
may require one to switch to the main file before compiling;
attempting to compile the child file produces errors.
\item
The main file must be modified (each time)
to adjust the |\includeonly| command
to the present needs. This easily leaves the main file in a messy state.
\item
The generated document will always carry the filename
of the main document. This is inconvenient if
several child files are to be compiled and
to be kept for distribution.
\end{itemize}

The present package provides a simple interface
to make child files individually compilable by \LaTeX{}.
Compiling a child file then has the same effect as compiling
the main file with an |\includeonly| command
to select the appropriate child.
Moreover the generated document will carry the name of the child
rather than the main file.
This resolves all three above issues.

This feature is meant to make the editing of books,
thesis documents and lecture notes somewhat more convenient.
However, the package can also be used efficiently for
composing a series of documents (such as exercise sheets)
which are typically distributed individually.
It then assists the author in generating the individual documents
(potentially in different versions)
as well as a document containing the collected series.
Another application is in developing style files
or other kinds of included material
where compilation of the style file could redirect
to a sample or test file.

%%%%%%%%%%%%%%%%%%%%%%%%%%%%%%%%%%%%%%%%%%%%%%%%%%%%%%%%%%%%%%%%%%%%%%%%%%%%%%%%
%%%%%%%%%%%%%%%%%%%%%%%%%%%%%%%%%%%%%%%%%%%%%%%%%%%%%%%%%%%%%%%%%%%%%%%%%%%%%%%%
\section{Usage}

First of all, the package \textsf{childdoc} is \emph{not} a standard
\LaTeXe{} |.sty| style file! Therefore it needs to be invoked in
a non-standard way.

%%%%%%%%%%%%%%%%%%%%%%%%%%%%%%%%%%%%%%%%%%%%%%%%%%%%%%%%%%%%%%%%%%%%%%%%%%%%%%%%
\subsection{Included Files}
\label{sec:include}

%%%%%%%%%%%%%%%%%%%%%%%%%%%%%%%%%%%%%%%%
\DescribeMacro{\childdocmain}
To use the package, add the commands
\begin{center}
\begin{tabular}{l}
|\input{childdoc.def}|\\
|\childdocmain{}|\\
\end{tabular}
\end{center}
at the very top of the main \LaTeX{} file,
in particular \emph{before} the |\documentclass| statement!
The argument of |\childdocmain| should be left empty
(but it must be present).

%%%%%%%%%%%%%%%%%%%%%%%%%%%%%%%%%%%%%%%%
\DescribeMacro{\childdocof}
Furthermore, add the commands
\begin{center}
\begin{tabular}{l}
|\input{childdoc.def}|\\
|\childdocof{|\textit{main}|}|\\
\end{tabular}
\end{center}
at the top of every child file \textit{child}
which is included by |\include{|\textit{child}|}|
from within the main file
(or at least for those files to be compiled individually).
The argument \textit{main} must be the filename of the main file.

There are a couple of
considerations in setting up the main and child documents:

%%%%%%%%%%%%%%%%%%%%%%%%%%%%%%%%%%%%%%%%
\paragraph{Restrictions.}

Please note the following restrictions:
\begin{itemize}
\item
|\childdocmain| must be called with one argument \textit{main}
to ensure compatibility with earlier version of the package.
It must either be empty (|\childdocmain{}|)
or precisely match the filename of the main file in which it is specified.
See \secref{sec:detection} for further information.
\item
The filename \textit{main} must be specified without the |.tex| extension.
\item
The filename \textit{main} is case sensitive
(even in case-insensitive file systems)
due to internal string comparison.
\item
The argument \textit{main} should be fully expanded, it cannot be a macro.
\item
Subdirectories and special characters should be avoided in filenames.
\item
The command |\childdocmain{|\textit{main}|}| must be followed by a whitespace.
It should not be followed immediately by another command
or by a comment mark `|%|'.
This is because the \TeX{} parser reads the token immediately following
the argument of |\childdocmain| and puts it
at the beginning of every child section;
however, a white\-space is ignored.
\end{itemize}

%%%%%%%%%%%%%%%%%%%%%%%%%%%%%%%%%%%%%%%%
\paragraph{Content of Main File.}

It is advisable to place all content in the child files included by |\include|.
Any output contained in the main file will appear in all child documents
unless suppressed manually;
it cannot be suppressed automatically by the |\includeonly| directive
and thus should normally be avoided.
A method to include some content in the main file
by means of conditional processing is described in \secref{sec:conditional}.

%%%%%%%%%%%%%%%%%%%%%%%%%%%%%%%%%%%%%%%%
\paragraph{Page Numbering.}

When only a part of the document is compiled,
the appropriate numbering of pages
(as well as other status parameters)
is determined from the |.aux| files.
The latter contain information from previous passes.
However this information needs to propagate through
all intermediate child documents.
Therefore the page numbering in child documents may well
be inconsistent until the complete document is compiled at least once.

A useful (if unconventional) way to always ensure a consistent
page numbering is to restart the numbering in each child document
and denote the pages by `\textit{child}|.|\textit{page}'
where \textit{child} represents the chapter/section number of the child file.
This can be achieved by the command
|\numberwithin{page}{|\textit{child}|}|
of the \textsf{amsmath} package
where \textit{child} can be |chapter| or |section|
depending on the chosen structuring.
Alternatively, one can modify the macro |\thepage| appropriately
and reset the counter |page| at the start of each child file.

%%%%%%%%%%%%%%%%%%%%%%%%%%%%%%%%%%%%%%%%%%%%%%%%%%%%%%%%%%%%%%%%%%%%%%%%%%%%%%%%
\subsection{Conditional Processing}
\label{sec:conditional}

The package provides a mechanism to compile different versions
of a document. To customise the versions further some conditional processing
can come in handy to distinguish which version is being compiled.
The package provides two macros to describe the compilation context:

%%%%%%%%%%%%%%%%%%%%%%%%%%%%%%%%%%%%%%%%
\DescribeMacro{\ifchilddoc}
The conditional |\ifchilddoc| distinguishes between the compilation of
child documents and the main document:
%
\begin{center}
|\ifchilddoc |\textit{child-code}| |[|\||else |\textit{main-code}]| \||fi|
\end{center}

%%%%%%%%%%%%%%%%%%%%%%%%%%%%%%%%%%%%%%%%
\DescribeMacro{\childdocname}
\DescribeMacro{\childdocjob}
The macro |\childdocname| contains the filename (without extension)
of the main or child file being processed.
Note that |\childdocjob| will always contain the name of the main file.

%%%%%%%%%%%%%%%%%%%%%%%%%%%%%%%%%%%%%%%%
\paragraph{Title Page.}

Conditional processing can be used to include a title or banner page
in the main document when proper precautions are taken.
Importantly, the code in the main file should ensure that the page counter
(as well as other status parameters which are stored in the |.aux| files)
takes the same value after the conditional processing.
Otherwise the page numbers may take divergent values
depending on which part is compiled.

For example, a title page could be declared by:
%
\begin{center}
\begin{tabular}{l}
|\ifchilddoc\||else|\\
|\addtocounter{page}{-1}|\\
\textit{code for title page}\\
|\newpage|\\
|\||fi|
\end{tabular}
\end{center}
%
A banner page for the child documents can be generated by:
%
\begin{center}
\begin{tabular}{l}
|\ifchilddoc|\\
|\addtocounter{page}{-1}|\\
\textit{code for banner page}\\
|\newpage|\\
|\||fi|
\end{tabular}
\end{center}
%
Here one could write a message such as:
\begin{center}
|This is the part \childdocname{} of \childdocjob{}.|
\end{center}

%%%%%%%%%%%%%%%%%%%%%%%%%%%%%%%%%%%%%%%%%%%%%%%%%%%%%%%%%%%%%%%%%%%%%%%%%%%%%%%%
\subsection{Flags}
\label{sec:flags}

The package makes it easy to generate different versions
of the main or child documents.
To this end compilation flags can be defined
and assigned different default values.
They will be particularly useful in conjunction
with the forwarding mechanism described in \secref{sec:forward}.

For example, it may be useful to have a flag |\version|
which can be set to |draft| or |final|.
The document source will contain some conditional code
depending on the value of |\version|.
Suppose further, the flag should default to |final| for the main file
and to |draft| for child files
which is a natural assignment for editing the document.
This is achieved by placing the following code
in the preamble of the main document
(below the |\childdocmain| directive):
%
\begin{center}
\begin{tabular}{l}
|\ifchilddoc|\\
|\providecommand{\version}{draft}|\\
|\||else|\\
|\providecommand{\version}{final}|\\
|\||fi|
\end{tabular}
\end{center}
%
The definition by |\providecommand| makes sure
that previous definitions are not overwritten.
Further statements |\providecommand{\version}{...}|
can thus be added before the above code to override it.

For the main file, one might add a line
(between |\childdocmain| and the above block)
%
\begin{center}
|%\ifchilddoc\||else\providecommand{\version}{draft}\||fi|
\end{center}
%
which can be uncommented to produce a draft version.
Likewise one can add a line to the very top of a child file
(above the |\childdocof{|\textit{main}|}| directive)
%
\begin{center}
|%\providecommand{\version}{final}|
\end{center}
%
which can be uncommented to produce the final version of this child document.

%%%%%%%%%%%%%%%%%%%%%%%%%%%%%%%%%%%%%%%%%%%%%%%%%%%%%%%%%%%%%%%%%%%%%%%%%%%%%%%%
\subsection{Forwarding}
\label{sec:forward}

Different versions of the main or child documents
using compilation flags as described in \secref{sec:flags}
can be (permanently) stored in different files
for convenient compilation, viewing and distribution.
To this end, the package defines a command
to pass on compilation to a different file:

%%%%%%%%%%%%%%%%%%%%%%%%%%%%%%%%%%%%%%%%
\DescribeMacro{\childdocforward}
The command |\childdocforward| redirects processing to
another source file:
%
\begin{center}
\begin{tabular}{l}
|\input{childdoc.def}|\\
|\childdocforward[|\textit{main}|]{|\textit{dest}|}|\\
\end{tabular}
\end{center}
%
The argument \textit{dest} is the destination file
(without extension).
It should be the main file or one of the child files.
Note that further \textsf{childdoc} directives
such as |\childdocof| and |\childdocforward|
in the indicated file will be processed in this form.
The optional argument \textit{main}
passes on directly to the main file \textit{main}
while pretending to compile the child \textit{dest}.
This form behaves as if \textit{dest}
issues |\childdocof{|\textit{main}|}| right away,
and no further \textsf{childdoc} directives will be processed.

%%%%%%%%%%%%%%%%%%%%%%%%%%%%%%%%%%%%%%%%
\DescribeMacro{\...prefix}
In the alternative form |\childdocforwardprefix|,
%
\begin{center}
\begin{tabular}{l}
|\input{childdoc.def}|\\
|\childdocforwardprefix[|\textit{main}|]{|\textit{prefix}|}{|\textit{dest}|}|
\end{tabular}
\end{center}
%
the destination file is determined by a pattern
depending on the current file:
To make this work, the current file must be called
`{\textit{prefix}\hspace{0.2em}\textit{suffix}}'
with \textit{prefix} matching precisely the argument.
Processing is then passed on to the file
`{\textit{dest}\hspace{0.2em}\textit{suffix}}'.
Surely, the same effect is achieved by
directly specifying the
argument `{\textit{dest}\hspace{0.2em}\textit{suffix}}'
in the first form.
However, that requires to set up a different file
for each child. With the alternative form of the command
all these files can have exactly the same content
which simplifies setting them up and maintaining them.

For example, the following file |draft.tex|
with a compilation flag |\version| as described in \secref{sec:flags}
compiles the main document as a draft:
%
\begin{center}
\begin{tabular}{l}
|\def\version{draft}|\\
|\input{childdoc.def}|\\
|\childdocforward{|\textit{main}|}|
\end{tabular}
\end{center}
%
Likewise, the following files |final|\textit{nn}|.tex|
compile the final version of the child document
|child|\textit{nn}|.tex|:
%
\begin{center}
\begin{tabular}{l}
|\def\version{final}|\\
|\input{childdoc.def}|\\
|\childdocforwardprefix{final}{child}|
\end{tabular}
\end{center}
%

Note that when several versions of a main file and/or of each child file
are to be generated, it may be convenient to set up a |Makefile| or
shell script to automatise the process.

%%%%%%%%%%%%%%%%%%%%%%%%%%%%%%%%%%%%%%%%%%%%%%%%%%%%%%%%%%%%%%%%%%%%%%%%%%%%%%%%
\subsection{Command Line Processing}
\label{sec:commandline}

The effect of redirection files can also be achieved by invoking
the \LaTeX{} compiler with a more elaborate command line.
Most conveniently this should be done as part
of a shell script or a |Makefile|.

When using \textsf{childdoc} in the main file, the following
command lines effectively perform a redirection
(note that depending on the shell being used,
backslashes may have to be doubled: `|\|' $\to$ `|\\|'):
%
\begin{center}
|... -jobname "|\textit{target}|" |\\|"|[\textit{flags}]%
|\input{childdoc.def}\childdocforward[|\textit{main}|]{|\textit{dest}|}"|
\end{center}
%
Here \textit{target} is the name of the output file,
\textit{main} is the name of the main file
and \textit{dest} is the name of the main or child file to be processed
(all filenames without extensions).
The optional argument \textit{main} can be omitted
if \textit{main} matches \textit{dest}.
Optionally, compilation \textit{flags} can be defined via |\def| commands.
This command line makes the \TeX{} engine believe
it is compiling the file \textit{target}
whose content is specified as the latter parameter.
The provided code then forwards the processing to
\textit{main} or \textit{dest} as described in \secref{sec:forward}.

%%%%%%%%%%%%%%%%%%%%%%%%%%%%%%%%%%%%%%%%%%%%%%%%%%%%%%%%%%%%%%%%%%%%%%%%%%%%%%%%
\subsection{Include by Input}
\label{sec:input}

Including child documents by |\include| has some restrictions by design.
Most notably, the content of a child document always occupies
its own set of pages; pages cannot be shared between child documents.
Usually, this behaviour makes perfect sense
because each child document contain an essential part of the document.
However, in some situations it may be desirable to compose
a document from a collection of parts
without having mandatory page breaks between then.
For this case, the package
provides a mechanism to include parts
by |\input| which can also be processed individually.
However, by construction this mechanism
requires manual handling of the content to be output.

%%%%%%%%%%%%%%%%%%%%%%%%%%%%%%%%%%%%%%%%
\DescribeMacro{\ifchilddocmanual}
The main file should be prepared as usual, see \secref{sec:include}.
However, the document body must make a distinction
between processing of an individual part and of the main document, e.g.:
%
\begin{center}
\begin{tabular}{l}
|\ifchilddocmanual|\\
|\input{\childdocname}|\\
|\||else|\\
\textit{document body with }|\input{|\textit{part}|}|\\
|\||fi|
\end{tabular}
\end{center}
%
The conditional |\ifchilddocmanual| is true whenever
a part to be included by |\input| is being compiled,
and the name of the part is stored in |\childdocname|.

%%%%%%%%%%%%%%%%%%%%%%%%%%%%%%%%%%%%%%%%
\DescribeMacro{\childdocby}
Each part to be included by |\input| should start with:
%
\begin{center}
\begin{tabular}{l}
|\input{childdoc.def}|\\
|\childdocby{|\textit{main}|}|\\
\end{tabular}
\end{center}
%
The directive |\childdocby| is similar to |\childdocof|
described in \secref{sec:include},
but the subsequent selection of content must be done manually.
To that end, both |\ifchilddoc| and |\ifchilddocmanual|
will be true upon processing of a part,
and the name of the part is stored in |\childdocname|.
Note that |\jobname| will be set to the filename of the current part
so that each part receives an individual |.aux| file
that does not interfere with the |.aux| file(s) of the main document.
This behaviour can be altered by the alternative form
|\childdocby[*]{|\textit{main}|}| (with a non-empty optional argument)
which uses the |.aux| file of the main document
by setting |\jobname| to \textit{main}.

%%%%%%%%%%%%%%%%%%%%%%%%%%%%%%%%%%%%%%%%%%%%%%%%%%%%%%%%%%%%%%%%%%%%%%%%%%%%%%%%
\subsection{Driver Development}
\label{sec:driver}

The \textsf{childdoc} mechanism can also be use for the development
of definition files such as \LaTeX{} styles or classes.
This case differs from the above setup with multiple parts
included by |\include| in that no |\includeonly| should be invoked.
This can be achieved by starting the include file
(before |\ProvidesPackage|) with:
%
\begin{center}
\begin{tabular}{l}
|\input{childdoc.def}|\\
|\childdocforward{|\textit{main}|}|\\
\end{tabular}
\end{center}
%
or alternatively with:
%
\begin{center}
\begin{tabular}{l}
|\input{childdoc.def}|\\
|\childdocby{|\textit{main}|}|\\
\end{tabular}
\end{center}
%
Both forms have slightly different effects as described above.
The main file is prepared as usual, see \secref{sec:include}.

%%%%%%%%%%%%%%%%%%%%%%%%%%%%%%%%%%%%%%%%%%%%%%%%%%%%%%%%%%%%%%%%%%%%%%%%%%%%%%%%
\subsection{Legacy Detection}
\label{sec:detection}

The directive |\childdocmain| in the main file can detect
whether the complete document or merely a child is to be compiled
even without using the directive |\childdocof|.
This method is deprecated because it is less robust
and there is no compelling reason to use it;
it is merely provided for backward compatibility
and it may be removed in future versions.

If the detection mechanism is to be used,
it is mandatory to correctly specify
the filename of the main file as the argument of |\childdocmain|:
%
\begin{center}
\begin{tabular}{l}
|\input{childdoc.def}|\\
|\childdocmain{|\textit{main}|}|\\
\end{tabular}
\end{center}
%
If |\jobname| does not match the argument \textit{main} of |\childdocmain|,
it is assumed that |\jobname| points to the child file to be compiled.
When using |\childdocmain| with the main file specified as argument,
it suffices to start a child file
with just |\input{|\textit{main}|}|
without loading of the package and using |\childdocof|.
If instead all processing is done
with the appropriate \textsf{childdoc} directives,
the argument of \textit{main} of |\childdocmain| can be empty.

An alternative version of the command line processing described
in \secref{sec:commandline} using the detection mechanism reads:
%
\begin{center}
|... -jobname "|\textit{target}|" "|[\textit{flags}]%
[|\def\jobname{|\textit{dest}|}|]|\input{|\textit{main}|}"|
\end{center}

%%%%%%%%%%%%%%%%%%%%%%%%%%%%%%%%%%%%%%%%%%%%%%%%%%%%%%%%%%%%%%%%%%%%%%%%%%%%%%%%
\subsection{Manual Code}
\label{sec:manual}

In case one cannot be certain whether the definitions file |childdoc.def|
is installed on the target \TeX{} distribution
and one prefers not to ship it,
it is conceivable to paste a few relevant commands into the sources.

To that end, drop all statements |\input{childdoc.def}|
and perform the replacements as outlined below.
Instead of |\childdocmain{|\textit{main}|}| add the following code
to the top of the main file:
%
\begin{center}
\begin{tabular}{l}
|\||ifdefined\childdocname\endinput\||fi\newif\ifchilddoc|\\
|\edef\childdocname{\scantokens\expandafter{\jobname\noexpand}}|\\
|\def\childdocmain{|\textit{main}|}\||ifx\childdocmain\childdocname\||else|\\
|\childdoctrue\includeonly{\childdocname}\let\jobname\childdocmain\||fi|\\
\end{tabular}
\end{center}
%
Instead of |\childdocof{|\textit{main}|}| just include the main file
at the top of each child file:
%
\begin{center}
|\input{|\textit{main}|}|
\end{center}
%
A simple redirection |\childdocforward{|\textit{dest}|}| is achieved by:
%
\begin{center}
|\def\jobname{|\textit{dest}|}\input{\jobname}|
\end{center}
%
The redirection with prefix
|\childdocforwardprefix[|\textit{prefix}|]{|\textit{dest}|}|
is accomplished by:
%
\begin{center}
\begin{tabular}{l}
|{\edef\jobname{\scantokens\expandafter{\jobname\noexpand}}|\\
|\def\redirectjob |\textit{prefix}|#1~~~{\gdef\jobname{|\textit{dest}|#1}}|\\
|\expandafter\redirectjob\jobname~~~}\input{\jobname}|
\end{tabular}
\end{center}

In an alternative approach,
child documents can be compiled by a specific command line
without additional code or specific definitions:
%
\begin{center}
|... -jobname "|\textit{target}|" "|[\textit{flags}]%
|\includeonly{|\textit{dest}|}\input{|\textit{main}|}"|
\end{center}
%

%%%%%%%%%%%%%%%%%%%%%%%%%%%%%%%%%%%%%%%%%%%%%%%%%%%%%%%%%%%%%%%%%%%%%%%%%%%%%%%%
%%%%%%%%%%%%%%%%%%%%%%%%%%%%%%%%%%%%%%%%%%%%%%%%%%%%%%%%%%%%%%%%%%%%%%%%%%%%%%%%
\section{Information}

%%%%%%%%%%%%%%%%%%%%%%%%%%%%%%%%%%%%%%%%%%%%%%%%%%%%%%%%%%%%%%%%%%%%%%%%%%%%%%%%
\subsection{Copyright}

Copyright \copyright{} 2017--2018 Niklas Beisert

This work may be distributed and/or modified under the
conditions of the \LaTeX{} Project Public License, either version 1.3
of this license or (at your option) any later version.
The latest version of this license is in
  \url{http://www.latex-project.org/lppl.txt}
and version 1.3 or later is part of all distributions of \LaTeX{}
version 2005/12/01 or later.

This work has the LPPL maintenance status `maintained'.

The Current Maintainer of this work is Niklas Beisert.

This work consists of the files |README.txt|, |childdoc.ins| and |childdoc.dtx|
as well as the derived files |childdoc.def|, |cdocsamp.tex|
with |cdocsch1.tex|, |cdocsch2.tex|, |cdocspt3.tex|, |cdocspt4.tex|,
|cdocsdrf.tex|, |cdocsfn1.tex|, |cdocsfn2.tex|
as well as |childdoc.pdf|.

%%%%%%%%%%%%%%%%%%%%%%%%%%%%%%%%%%%%%%%%%%%%%%%%%%%%%%%%%%%%%%%%%%%%%%%%%%%%%%%%
\subsection{Files and Installation}

The package consists of the files:
%
\begin{center}
\begin{tabular}{ll}
    |README.txt|   & readme file \\
    |childdoc.ins| & installation file \\
    |childdoc.dtx| & source file \\
    |childdoc.def| & definition file \\
    |cdocsamp.tex| & sample main file \\
    |cdocsch1.tex| & sample include file \\
    |cdocsch2.tex| & sample include file \\
    |cdocspt3.tex| & sample part file \\
    |cdocspt4.tex| & sample part file \\
    |cdocsdrf.tex| & sample redirection file \\
    |cdocsfn1.tex| & sample redirection file \\
    |cdocsfn2.tex| & sample redirection file \\
    |childdoc.pdf| & manual
\end{tabular}
\end{center}
%
The distribution consists of the files
|README.txt|, |childdoc.ins| and |childdoc.dtx|.
%
\begin{itemize}
\item
Run (pdf)\LaTeX{} on |childdoc.dtx|
to compile the manual |childdoc.pdf| (this file).
\item
Run \LaTeX{} on |childdoc.ins| to create the definitions file |childdoc.def|
and the sample |cdocsamp.tex| with include files
|cdocsch1.tex|, |cdocsch2.tex|, |cdocspt3.tex|, |cdocspt4.tex|,
|cdocsdrf.tex|, |cdocsfn1.tex|, |cdocsfn2.tex|.
Then copy the file |childdoc.def| to an appropriate directory of your \LaTeX{}
distribution, e.g.\ \textit{texmf-root}|/tex/latex/childdoc|.
\end{itemize}

%%%%%%%%%%%%%%%%%%%%%%%%%%%%%%%%%%%%%%%%%%%%%%%%%%%%%%%%%%%%%%%%%%%%%%%%%%%%%%%%
\subsection{Related CTAN Packages}

There are several other packages which offer a similar functionality:
%
\begin{itemize}
\item
The packages
\href{http://ctan.org/pkg/docmute}{\textsf{docmute}},
\href{http://ctan.org/pkg/includex}{\textsf{includex}} and
\href{http://ctan.org/pkg/standalone}{\textsf{standalone}}
provide commands to include only the document body of
a child file thus allowing both files to be compiled individually.
\item
The packages \href{http://ctan.org/pkg/subdocs}{\textsf{subdocs}}
and \href{http://ctan.org/pkg/subfiles}{\textsf{subfiles}}
provide structures in which the main and child documents can be
encapsulated and allowing them to be compiled individually.
The inclusion mechanism is different from the conventional |\include|.
\item
The package \href{http://ctan.org/pkg/combine}{\textsf{combine}}
is an elaborate solution to combine several documents into one.
\end{itemize}
%
See also the CTAN topic \href{http://ctan.org/topic/subdocs}{\textsf{subdocs}}
for further related packages.
The present package differs from the above solutions in that
a document structure constructed with the conventional |\include| mechanism
just needs two extra commands at the top of every file
such that all constituent files can be compiled individually.

%%%%%%%%%%%%%%%%%%%%%%%%%%%%%%%%%%%%%%%%%%%%%%%%%%%%%%%%%%%%%%%%%%%%%%%%%%%%%%%%
%\subsection{Feature Suggestions}
%
%The following is a list of features which may be useful for future
%versions of this package:
%%
%\begin{itemize}
%\item
%\ldots
%\end{itemize}

%%%%%%%%%%%%%%%%%%%%%%%%%%%%%%%%%%%%%%%%%%%%%%%%%%%%%%%%%%%%%%%%%%%%%%%%%%%%%%%%
\subsection{Revision History}

%%%%%%%%%%%%%%%%%%%%%%%%%%%%%%%%%%%%%%%%
\paragraph{v2.0:} 2018/12/30

\begin{itemize}
\item
immediate forward processing
\item
added |\childdocby| mechanism
\item
manual restructured
\end{itemize}

%%%%%%%%%%%%%%%%%%%%%%%%%%%%%%%%%%%%%%%%
\paragraph{v1.6:} 2018/01/17

\begin{itemize}
\item
application for development of include files
\item
corrections to manual
\end{itemize}

%%%%%%%%%%%%%%%%%%%%%%%%%%%%%%%%%%%%%%%%
\paragraph{v1.5:} 2017/05/21

\begin{itemize}
\item
more complete structuring introduced
\item
|\childdocof| introduced
\item
|\childdoc| renamed to |\childdocmain|
\item
|\childredirect| renamed to |\childdocforward| and |\childdocforwardprefix|
and functionality expanded
\end{itemize}

%%%%%%%%%%%%%%%%%%%%%%%%%%%%%%%%%%%%%%%%
\paragraph{v1.0:} 2017/04/27

\begin{itemize}
\item
manual and install package
\item
first version published on CTAN
\end{itemize}

%%%%%%%%%%%%%%%%%%%%%%%%%%%%%%%%%%%%%%%%
\paragraph{v0.6:} 2017/04/26

\begin{itemize}
\item
redirection mechanism added
\end{itemize}

%%%%%%%%%%%%%%%%%%%%%%%%%%%%%%%%%%%%%%%%
\paragraph{v0.5:} 2017/04/26

\begin{itemize}
\item
functionality in definition file
\end{itemize}


%%%%%%%%%%%%%%%%%%%%%%%%%%%%%%%%%%%%%%%%%%%%%%%%%%%%%%%%%%%%%%%%%%%%%%%%%%%%%%%%
%%%%%%%%%%%%%%%%%%%%%%%%%%%%%%%%%%%%%%%%%%%%%%%%%%%%%%%%%%%%%%%%%%%%%%%%%%%%%%%%
%%%%%%%%%%%%%%%%%%%%%%%%%%%%%%%%%%%%%%%%%%%%%%%%%%%%%%%%%%%%%%%%%%%%%%%%%%%%%%%%
\appendix

\settowidth\MacroIndent{\rmfamily\scriptsize 000\ }

 \DocInput{childdoc.dtx}

\end{document}
%</driver>
% \fi
%
% %%%%%%%%%%%%%%%%%%%%%%%%%%%%%%%%%%%%%%%%%%%%%%%%%%%%%%%%%%%%%%%%%%%%%%%%%%%%%%
% %%%%%%%%%%%%%%%%%%%%%%%%%%%%%%%%%%%%%%%%%%%%%%%%%%%%%%%%%%%%%%%%%%%%%%%%%%%%%%
% \section{Sample}
%\iffalse
%<*samplemain>
%\fi
%
% The following presents a sample document
% with two chapters, two parts, a title page,
% a compile flag as well as three forwarding files to set the flag.
% It consists of eight |.tex| files:
% \begin{center}
% \begin{tabular}{ll}
% |cdocsamp.tex|&main file\\
% |cdocsch1.tex|&include file for chapter 1\\
% |cdocsch2.tex|&include file for chapter 2\\
% |cdocspt3.tex|&include file for part 3\\
% |cdocspt4.tex|&include file for part 4\\
% |cdocsdrf.tex|&forwarding file for main file in draft mode\\
% |cdocsfi1.tex|&forwarding file for final version of chapter 1\\
% |cdocsfi2.tex|&forwarding file for final version of chapter 2\\
% \end{tabular}
% \end{center}
% Each of the eight files can be compiled directly by the \LaTeX{} compiler.
%
% %%%%%%%%%%%%%%%%%%%%%%%%%%%%%%%%%%%%%%
% \paragraph{Main File.}
%
% The main file is called |cdocsamp.tex|.
%
% Load the \textsf{childdoc} definitions and
% declare the filename for the main document:
%    \begin{macrocode}
\input{childdoc.def}
\childdocmain{}
%    \end{macrocode}

% Optional override for |\version| flag:
%    \begin{macrocode}
%%\ifchilddoc\else\providecommand{\version}{draft}\fi
%    \end{macrocode}

% Define the default values for the |\version| flag
% (|final| for the main file and |draft| for childs):
%    \begin{macrocode}
\ifchilddoc
\providecommand{\version}{draft}
\else
\providecommand{\version}{final}
\fi
%    \end{macrocode}

% Load the standard document class:
%    \begin{macrocode}
\documentclass[12pt]{article}
%    \end{macrocode}

% Start the document body:
%    \begin{macrocode}
\begin{document}
%    \end{macrocode}

% Declare a title page.
% Print title, part of document being processed and version flag:
%    \begin{macrocode}
\addtocounter{page}{-1}
\begin{center}
{\LARGE\bfseries{}childdoc example\par}
\vspace{1cm}
\ifchilddoc
\ifchilddocmanual part\else chapter\fi:
`\childdocname' of `\childdocjob'\par
\else
main document: `\childdocjob'\par
\fi
version: \version\par
\end{center}
\newpage
%    \end{macrocode}

% Manually include selected file,
% otherwise process as usual:
%    \begin{macrocode}
\ifchilddocmanual
\section*{part `\childdocname'}
\input{\childdocname}
\else
%    \end{macrocode}

% Include the two chapters:
%    \begin{macrocode}
\include{cdocsch1}
\include{cdocsch2}
%    \end{macrocode}

% Include the two parts unless only chapters should be displayed:
%    \begin{macrocode}
\ifchilddoc\else
\section{part three}
\input{cdocspt3}
\section{part four}
\input{cdocspt4}
\fi
%    \end{macrocode}

% Process as usual until here:
%    \begin{macrocode}
\fi
%    \end{macrocode}

% End of document body:
%    \begin{macrocode}
\end{document}
%    \end{macrocode}
%\iffalse
%</samplemain>
%\fi
%
% %%%%%%%%%%%%%%%%%%%%%%%%%%%%%%%%%%%%%%
% \paragraph{Chapter Include Files.}
%
% The include files are called |cdocsch1.tex| and |cdocsch2.tex|.
%
%\iffalse
%<*samplechap1|samplechap2>
%\fi

% Optional override for |\version| flag:
%    \begin{macrocode}
%%\providecommand{\version}{final}
%    \end{macrocode}

% Include the main document:
%    \begin{macrocode}
\input{childdoc.def}
\childdocof{cdocsamp}
%    \end{macrocode}

%\iffalse
%</samplechap1|samplechap2>
%\fi
%
%\iffalse
%<*samplechap1>
%\fi
% Some text for chapter 1:
%    \begin{macrocode}
\section{one}
some text in chapter one
%    \end{macrocode}

%\iffalse
%</samplechap1>
%\fi
% Some text for chapter 2:
%\iffalse
%<*samplechap2>
%\fi
%    \begin{macrocode}
\section{two}
more text in chapter two
%    \end{macrocode}

%\iffalse
%</samplechap2>
%\fi
%
% %%%%%%%%%%%%%%%%%%%%%%%%%%%%%%%%%%%%%%
% \paragraph{Part Include Files.}
%
% The include files are called |cdocspt3.tex| and |cdocspt4.tex|.
%
%\iffalse
%<*samplepart3|samplepart4>
%\fi

% Optional override for |\version| flag:
%    \begin{macrocode}
%%\providecommand{\version}{final}
%    \end{macrocode}

% Include the main document:
%    \begin{macrocode}
\input{childdoc.def}
\childdocby{cdocsamp}
%    \end{macrocode}

%\iffalse
%</samplepart3|samplepart4>
%\fi
%
%\iffalse
%<*samplepart3>
%\fi
% Some text for part 3:
%    \begin{macrocode}
some text in part three
%    \end{macrocode}

%\iffalse
%</samplepart3>
%\fi
% Some text for part 4:
%\iffalse
%<*samplepart4>
%\fi
%    \begin{macrocode}
more text in part four
%    \end{macrocode}

%\iffalse
%</samplepart4>
%\fi
%
% %%%%%%%%%%%%%%%%%%%%%%%%%%%%%%%%%%%%%%
% \paragraph{Forwarding for a Complete Draft.}
%
% The following forwarding file |cdocsdrf.tex|
% compiles the main document in draft mode:
%\iffalse
%<*sampledraft>
%\fi
%    \begin{macrocode}
\def\version{draft}
\input{childdoc.def}
\childdocforward{cdocsamp}
%    \end{macrocode}

%\iffalse
%</sampledraft>
%\fi
%
% %%%%%%%%%%%%%%%%%%%%%%%%%%%%%%%%%%%%%%
% \paragraph{Forwarding for Final Version of the Chapters.}
%
% The following forwarding files |cdocsfn1.tex| and |cdocsfn2.tex|
% (with identical content)
% compile the final versions of the child documents
% |cdocsch1.tex| and |cdocsch2.tex|, respectively:
%\iffalse
%<*samplefinal>
%\fi
%    \begin{macrocode}
\def\version{final}
\input{childdoc.def}
\childdocforwardprefix[cdocsamp]{cdocsfn}{cdocsch}
%    \end{macrocode}

%\iffalse
%</samplefinal>
%\fi
%
% %%%%%%%%%%%%%%%%%%%%%%%%%%%%%%%%%%%%%%
% \paragraph{Command Line Processing.}
%
% The following three command lines generate the output files
% |cdocscld|, |cdocscl1| and |cdocscl2|
% which should be identical to
% |cdocsdrf|, |cdocsch1| and |cdocsfn2|, respectively:
% \begin{center}
% \begin{tabular}{l}
% |latex -jobname cdocscld \|\\
% |  "\def\version{draft}\input{childdoc.def}\childdocforward{cdocsamp}"|\\
% |latex -jobname cdocscl1 \|\\
% |  "\input{childdoc.def}\childdocforward[cdocsamp]{cdocsch1}"|\\
% |latex -jobname cdocscl2 \|\\
% |  "\def\version{final}\input{childdoc.def}\childdocforward{cdocsch2}"|
% \end{tabular}
% \end{center}
% Note that the trailing backslash on each first line
% merely continues the input to the second line
% (for convenient cut ant paste).
% Furthermore, the command |latex| can be replaced by any
% of its alternative versions such as |pdflatex|.
%
% %%%%%%%%%%%%%%%%%%%%%%%%%%%%%%%%%%%%%%%%%%%%%%%%%%%%%%%%%%%%%%%%%%%%%%%%%%%%%%
% %%%%%%%%%%%%%%%%%%%%%%%%%%%%%%%%%%%%%%%%%%%%%%%%%%%%%%%%%%%%%%%%%%%%%%%%%%%%%%
% \section{Implementation}
%\iffalse
%<*package>
%\fi
%
% This section describes the definitions file |childdoc.def|.

% The definitions cannot be loaded using |\usepackage| or |\RequirePackage|
% which has a mechanism to prevent loading a style file more than once.
% When loading the definitions by means of |\input|
% multiple instances have to be prevented manually:
%\iffalse
%This code needs to be before the `\ProvidesFile' directive
%which is defined at the beginning of this file.
%Therefore it is also placed there and commented out here.
%</package>
%<*discard>
%\fi
%    \begin{macrocode}
\ifdefined\childdocmain\endinput\fi
%    \end{macrocode}
%\iffalse
%</discard>
%<*package>
%\fi
%
% \macro{\ifchilddoc}
% \macro{\ifchilddocmanual}
% The conditional |\ifchilddoc| tells whether a
% child (true) or main (false) document is being compiled.
% The conditional |\ifchilddocmanual| tells whether
% the |\includeonly| mechanism is used (false) or
% the selection of child files must be performed manually (true).
% The definitions initialise to false:
%    \begin{macrocode}
\newif\ifchilddoc
\newif\ifchilddocmanual
%    \end{macrocode}

% \macro{\childdocname}
% \macro{\childdocjob}
% The macro |\childdocname| stores the name of the main document
% to be compiled. The macro |\childdocjob| stores the name of
% the document on which the \LaTeX{} compiler was originally invoked.
% The content of |\jobname| cannot be compared
% to filenames specified in the source due to different catcodes.
% The following code rescans |\jobname|, stores the result
% in |\childdocname| and saves a copy in |\childdocjob|:
%    \begin{macrocode}
\edef\childdocname{\scantokens\expandafter{\jobname\noexpand}}
\let\childdocjob\childdocname
%    \end{macrocode}

% \macro{\childdocdisable}
% The macro |\childdocdisable| prevents the main file
% from being processed more than once.
% At this stage, the main document command |\childdocmain|
% is assumed to be called once again where it should do nothing.
% Any subsequent call to it should prevent
% a secondary processing of the main document
% It overwrites the forwarding commands
% |\childdocof| and |\childdocforward|
% with empty macros to prevent further inclusions of the main document:
%    \begin{macrocode}
\newcommand{\childdocdisable}
{
  \renewcommand{\childdocmain}[1]{\renewcommand{\childdocmain}[1]{\endinput}}
  \renewcommand{\childdocof}[1]{}
  \renewcommand{\childdocby}[2][]{}
  \renewcommand{\childdocforward}[2][]{}
  \renewcommand{\childdocdisable}{}
}
%    \end{macrocode}

% \macro{\childdocmain}
% The macro |\childdocmain| is to be called at the top of the main file
% with nothing or the main filename (without extension) as argument.
% First, it breaks loops.
% If the argument is not empty and does not match |\childdocname|
% (which is set by the first inclusion of |childdoc.def|),
% |\ifchilddoc| is set to true, |\includeonly| is applied to the child file
% and |\jobname| is set to the main file
% (for proper handling of |.aux| files):
%    \begin{macrocode}
\newcommand{\childdocmain}[1]
{
  \childdocdisable\childdocmain{}
  \if?#1?\else
    \begingroup
      \def\childdoctmp{#1}
      \ifx\childdoctmp\childdocname
        \def\childdoctmp{}
      \else
        \def\childdoctmp
        {
          \childdoctrue
          \includeonly{\childdocname}
          \def\childdocjob{#1}
          \def\jobname{#1}
        }
      \fi
      \expandafter
    \endgroup
    \childdoctmp
  \fi
}
%    \end{macrocode}

% \macro{\childdocof}
% The command |\childdocof| redirects
% compilation to the main file |#1|.
%    \begin{macrocode}
\newcommand{\childdocof}[1]
{
  \childdocdisable
  \childdoctrue
  \includeonly{\childdocname}
  \def\jobname{#1}
  \def\childdocjob{#1}
  \input{#1}
}
%    \end{macrocode}

% \macro{\childdocby}
% The command |\childdocby| ....
%    \begin{macrocode}
\newcommand{\childdocby}[2][]
{
  \childdocdisable
  \childdoctrue
  \childdocmanualtrue
  \if?#1?\else
    \def\jobname{#2}
  \fi
  \def\childdocjob{#2}
  \input{#2}
  \endinput
}
%    \end{macrocode}

% \macro{\childdocforward}
% The command |\childdocforward| redirects
% compilation to the main file or
% (if the optional argument is given) a child file.
% Parameters are set as if the main file
% or a child file starting with |\childdocof| was compiled.
% Then compilation is handed over to the main file:
%    \begin{macrocode}
\newcommand{\childdocforward}[2][]
{
  \begingroup
    \if?#1?
      \def\childdoctmp
      {
        \def\childdocname{#2}
        \def\childdocjob{#2}
        \def\jobname{#2}
        \input{#2}
        \endinput
      }
    \else
      \def\childdoctmp
      {
        \childdocdisable
        \def\childdocname{#2}
        \childdoctrue
        \includeonly{#2}
        \def\childdocjob{#1}
        \def\jobname{#1}
        \input{#1}
        \endinput
      }
    \fi
    \expandafter
  \endgroup
  \childdoctmp
}
%    \end{macrocode}

% \macro{\childdocforwardprefix}
% The command |\childdocforwardprefix| redirects
% compilation to the main or a child file by means of a pattern.
% The prefix |#1| in the current filename is replaced by |#2|
% and the suffix of the current filename is kept
% (it is assumed that the filename does not contain the substring `|~~~|'
% which is used as a delimiter).
% Compilation is handed over to the new file by |\childdocforward|:
%    \begin{macrocode}
\newcommand{\childdocforwardprefix}[3][]
{
  \begingroup
    \def\childdocextract #2##1~~~{\def\childdoctmp{\childdocforward[#1]{#3##1}}}
    \expandafter\childdocextract\childdocname~~~
    \expandafter
  \endgroup
  \childdoctmp
}
%    \end{macrocode}

% \macro{\childdoc}
% The deprecated macro |\childdoc| is a legacy version of |\childdocmain|:
%    \begin{macrocode}
\newcommand{\childdoc}{\childdocmain}
%    \end{macrocode}

% \macro{\childdocredirect}
% The deprecated macro |\childdocredirect| is a legacy version
% of |\childdocforward| and |\childdocforwardprefix|:
%    \begin{macrocode}
\newcommand{\childdocredirect}[2][]
{
  \begingroup
    \if?#1?
      \def\childdoctmp{\childdocforward{#2}}
    \else
      \def\childdoctmp{\childdocforwardprefix{#1}{#2}}
    \fi
    \expandafter
  \endgroup
  \childdoctmp
}
%    \end{macrocode}

%\iffalse
%</package>
%\fi
%
\endinput

\childdocforwardprefix[cdocsamp]{cdocsfn}{cdocsch}
%    \end{macrocode}

%\iffalse
%</samplefinal>
%\fi
%
% %%%%%%%%%%%%%%%%%%%%%%%%%%%%%%%%%%%%%%
% \paragraph{Command Line Processing.}
%
% The following three command lines generate the output files
% |cdocscld|, |cdocscl1| and |cdocscl2|
% which should be identical to
% |cdocsdrf|, |cdocsch1| and |cdocsfn2|, respectively:
% \begin{center}
% \begin{tabular}{l}
% |latex -jobname cdocscld \|\\
% |  "\def\version{draft}% \iffalse
%
% childdoc.dtx Copyright (C) 2017-2018 Niklas Beisert
%
% This work may be distributed and/or modified under the
% conditions of the LaTeX Project Public License, either version 1.3
% of this license or (at your option) any later version.
% The latest version of this license is in
%   http://www.latex-project.org/lppl.txt
% and version 1.3 or later is part of all distributions of LaTeX
% version 2005/12/01 or later.
%
% This work has the LPPL maintenance status `maintained'.
%
% The Current Maintainer of this work is Niklas Beisert.
%
% This work consists of the files childdoc.dtx and childdoc.ins
% and the derived files childdoc.def and cdocsamp.tex with
% cdocsch1.tex, cdocsch2.tex, cdocsdrf.tex, cdocsfn1.tex, cdocsfn2.tex.
%
%<package>\ifdefined\childdocmain\endinput\fi
%<package>\ProvidesFile{childdoc.def}[2018/12/30 v2.0 child document driver]
%<samplemain>\ProvidesFile{cdocsamp.tex}[2018/12/30 v2.0 sample for childdoc]
%<*driver>
%\ProvidesFile{childdoc.drv}[2018/12/30 v2.0 childdoc reference manual file]
\PassOptionsToClass{10pt,a4paper}{article}
\documentclass{ltxdoc}

\usepackage[margin=35mm]{geometry}
\usepackage{hyperref}
\usepackage{hyperxmp}
\usepackage[usenames]{color}

\hypersetup{colorlinks=true}
\hypersetup{pdfstartview=FitH}
\hypersetup{pdfpagemode=UseNone}
\hypersetup{pdfsource={}}
\hypersetup{pdflang={en-UK}}
\hypersetup{pdfcopyright={Copyright 2017-2018 Niklas Beisert.
  This work may be distributed and/or modified under the
  conditions of the LaTeX Project Public License, either version 1.3
  of this license or (at your option) any later version.}}
\hypersetup{pdflicenseurl={http://www.latex-project.org/lppl.txt}}
\hypersetup{pdfcontactaddress={ETH Zurich, ITP, HIT K,
  Wolfgang-Pauli-Strasse 27}}
\hypersetup{pdfcontactpostcode={8093}}
\hypersetup{pdfcontactcity={Zurich}}
\hypersetup{pdfcontactcountry={Switzerland}}
\hypersetup{pdfcontactemail={nbeisert@itp.phys.ethz.ch}}
\hypersetup{pdfcontacturl={http://people.phys.ethz.ch/\xmptilde nbeisert/}}

\newcommand{\secref}[1]{\hyperref[#1]{section \ref*{#1}}}

\parskip1ex
\parindent0pt
\let\olditemize\itemize
\def\itemize{\olditemize\parskip0pt}

\begin{document}

\title{The \textsf{childdoc} Package}
\hypersetup{pdftitle={The childdoc Package}}
\author{Niklas Beisert\\[2ex]
  Institut f\"ur Theoretische Physik\\
  Eidgen\"ossische Technische Hochschule Z\"urich\\
  Wolfgang-Pauli-Strasse 27, 8093 Z\"urich, Switzerland\\[1ex]
  \href{mailto:nbeisert@itp.phys.ethz.ch}
  {\texttt{nbeisert@itp.phys.ethz.ch}}}
\hypersetup{pdfauthor={Niklas Beisert}}
\hypersetup{pdfsubject={Manual for the LaTeX2e Package childdoc}}
\date{30 December 2018, \textsf{v2.0}}
\maketitle

\begin{abstract}\noindent
\textsf{childdoc} is a \LaTeXe{} package
that enables the direct compilation
of document sections included by |\include|
to individual files.
\end{abstract}

\begingroup
\parskip0ex
\tableofcontents
\endgroup

%%%%%%%%%%%%%%%%%%%%%%%%%%%%%%%%%%%%%%%%%%%%%%%%%%%%%%%%%%%%%%%%%%%%%%%%%%%%%%%%
%%%%%%%%%%%%%%%%%%%%%%%%%%%%%%%%%%%%%%%%%%%%%%%%%%%%%%%%%%%%%%%%%%%%%%%%%%%%%%%%
\section{Introduction}

\LaTeX{} provides a mechanism to structure a large document (such as a book)
into a main file and several child files (containing the chapters)
using the |\include| command.
This mechanism is beneficial for documents
which span hundreds of pages in order to
make the source file(s) more manageable.
Moreover, compilation can be restricted to
selected child files by means of the |\includeonly| command.
The latter feature can be used to reduce the compilation time while editing
(this was significantly more useful in the earlier days of \LaTeX{})
or to generate a smaller document which is easier to navigate.
Another application of |\includeonly| is to generate
documents consisting of selected parts of the complete document.

However, there are a few drawbacks of the plain |\include| mechanism:
\begin{itemize}
\item
The child files cannot be compiled on their own,
they can only be compiled via the main file.
A naive editing environment
(such as a text editor with an option
to have the current file processed by \LaTeX)
may require one to switch to the main file before compiling;
attempting to compile the child file produces errors.
\item
The main file must be modified (each time)
to adjust the |\includeonly| command
to the present needs. This easily leaves the main file in a messy state.
\item
The generated document will always carry the filename
of the main document. This is inconvenient if
several child files are to be compiled and
to be kept for distribution.
\end{itemize}

The present package provides a simple interface
to make child files individually compilable by \LaTeX{}.
Compiling a child file then has the same effect as compiling
the main file with an |\includeonly| command
to select the appropriate child.
Moreover the generated document will carry the name of the child
rather than the main file.
This resolves all three above issues.

This feature is meant to make the editing of books,
thesis documents and lecture notes somewhat more convenient.
However, the package can also be used efficiently for
composing a series of documents (such as exercise sheets)
which are typically distributed individually.
It then assists the author in generating the individual documents
(potentially in different versions)
as well as a document containing the collected series.
Another application is in developing style files
or other kinds of included material
where compilation of the style file could redirect
to a sample or test file.

%%%%%%%%%%%%%%%%%%%%%%%%%%%%%%%%%%%%%%%%%%%%%%%%%%%%%%%%%%%%%%%%%%%%%%%%%%%%%%%%
%%%%%%%%%%%%%%%%%%%%%%%%%%%%%%%%%%%%%%%%%%%%%%%%%%%%%%%%%%%%%%%%%%%%%%%%%%%%%%%%
\section{Usage}

First of all, the package \textsf{childdoc} is \emph{not} a standard
\LaTeXe{} |.sty| style file! Therefore it needs to be invoked in
a non-standard way.

%%%%%%%%%%%%%%%%%%%%%%%%%%%%%%%%%%%%%%%%%%%%%%%%%%%%%%%%%%%%%%%%%%%%%%%%%%%%%%%%
\subsection{Included Files}
\label{sec:include}

%%%%%%%%%%%%%%%%%%%%%%%%%%%%%%%%%%%%%%%%
\DescribeMacro{\childdocmain}
To use the package, add the commands
\begin{center}
\begin{tabular}{l}
|\input{childdoc.def}|\\
|\childdocmain{}|\\
\end{tabular}
\end{center}
at the very top of the main \LaTeX{} file,
in particular \emph{before} the |\documentclass| statement!
The argument of |\childdocmain| should be left empty
(but it must be present).

%%%%%%%%%%%%%%%%%%%%%%%%%%%%%%%%%%%%%%%%
\DescribeMacro{\childdocof}
Furthermore, add the commands
\begin{center}
\begin{tabular}{l}
|\input{childdoc.def}|\\
|\childdocof{|\textit{main}|}|\\
\end{tabular}
\end{center}
at the top of every child file \textit{child}
which is included by |\include{|\textit{child}|}|
from within the main file
(or at least for those files to be compiled individually).
The argument \textit{main} must be the filename of the main file.

There are a couple of
considerations in setting up the main and child documents:

%%%%%%%%%%%%%%%%%%%%%%%%%%%%%%%%%%%%%%%%
\paragraph{Restrictions.}

Please note the following restrictions:
\begin{itemize}
\item
|\childdocmain| must be called with one argument \textit{main}
to ensure compatibility with earlier version of the package.
It must either be empty (|\childdocmain{}|)
or precisely match the filename of the main file in which it is specified.
See \secref{sec:detection} for further information.
\item
The filename \textit{main} must be specified without the |.tex| extension.
\item
The filename \textit{main} is case sensitive
(even in case-insensitive file systems)
due to internal string comparison.
\item
The argument \textit{main} should be fully expanded, it cannot be a macro.
\item
Subdirectories and special characters should be avoided in filenames.
\item
The command |\childdocmain{|\textit{main}|}| must be followed by a whitespace.
It should not be followed immediately by another command
or by a comment mark `|%|'.
This is because the \TeX{} parser reads the token immediately following
the argument of |\childdocmain| and puts it
at the beginning of every child section;
however, a white\-space is ignored.
\end{itemize}

%%%%%%%%%%%%%%%%%%%%%%%%%%%%%%%%%%%%%%%%
\paragraph{Content of Main File.}

It is advisable to place all content in the child files included by |\include|.
Any output contained in the main file will appear in all child documents
unless suppressed manually;
it cannot be suppressed automatically by the |\includeonly| directive
and thus should normally be avoided.
A method to include some content in the main file
by means of conditional processing is described in \secref{sec:conditional}.

%%%%%%%%%%%%%%%%%%%%%%%%%%%%%%%%%%%%%%%%
\paragraph{Page Numbering.}

When only a part of the document is compiled,
the appropriate numbering of pages
(as well as other status parameters)
is determined from the |.aux| files.
The latter contain information from previous passes.
However this information needs to propagate through
all intermediate child documents.
Therefore the page numbering in child documents may well
be inconsistent until the complete document is compiled at least once.

A useful (if unconventional) way to always ensure a consistent
page numbering is to restart the numbering in each child document
and denote the pages by `\textit{child}|.|\textit{page}'
where \textit{child} represents the chapter/section number of the child file.
This can be achieved by the command
|\numberwithin{page}{|\textit{child}|}|
of the \textsf{amsmath} package
where \textit{child} can be |chapter| or |section|
depending on the chosen structuring.
Alternatively, one can modify the macro |\thepage| appropriately
and reset the counter |page| at the start of each child file.

%%%%%%%%%%%%%%%%%%%%%%%%%%%%%%%%%%%%%%%%%%%%%%%%%%%%%%%%%%%%%%%%%%%%%%%%%%%%%%%%
\subsection{Conditional Processing}
\label{sec:conditional}

The package provides a mechanism to compile different versions
of a document. To customise the versions further some conditional processing
can come in handy to distinguish which version is being compiled.
The package provides two macros to describe the compilation context:

%%%%%%%%%%%%%%%%%%%%%%%%%%%%%%%%%%%%%%%%
\DescribeMacro{\ifchilddoc}
The conditional |\ifchilddoc| distinguishes between the compilation of
child documents and the main document:
%
\begin{center}
|\ifchilddoc |\textit{child-code}| |[|\||else |\textit{main-code}]| \||fi|
\end{center}

%%%%%%%%%%%%%%%%%%%%%%%%%%%%%%%%%%%%%%%%
\DescribeMacro{\childdocname}
\DescribeMacro{\childdocjob}
The macro |\childdocname| contains the filename (without extension)
of the main or child file being processed.
Note that |\childdocjob| will always contain the name of the main file.

%%%%%%%%%%%%%%%%%%%%%%%%%%%%%%%%%%%%%%%%
\paragraph{Title Page.}

Conditional processing can be used to include a title or banner page
in the main document when proper precautions are taken.
Importantly, the code in the main file should ensure that the page counter
(as well as other status parameters which are stored in the |.aux| files)
takes the same value after the conditional processing.
Otherwise the page numbers may take divergent values
depending on which part is compiled.

For example, a title page could be declared by:
%
\begin{center}
\begin{tabular}{l}
|\ifchilddoc\||else|\\
|\addtocounter{page}{-1}|\\
\textit{code for title page}\\
|\newpage|\\
|\||fi|
\end{tabular}
\end{center}
%
A banner page for the child documents can be generated by:
%
\begin{center}
\begin{tabular}{l}
|\ifchilddoc|\\
|\addtocounter{page}{-1}|\\
\textit{code for banner page}\\
|\newpage|\\
|\||fi|
\end{tabular}
\end{center}
%
Here one could write a message such as:
\begin{center}
|This is the part \childdocname{} of \childdocjob{}.|
\end{center}

%%%%%%%%%%%%%%%%%%%%%%%%%%%%%%%%%%%%%%%%%%%%%%%%%%%%%%%%%%%%%%%%%%%%%%%%%%%%%%%%
\subsection{Flags}
\label{sec:flags}

The package makes it easy to generate different versions
of the main or child documents.
To this end compilation flags can be defined
and assigned different default values.
They will be particularly useful in conjunction
with the forwarding mechanism described in \secref{sec:forward}.

For example, it may be useful to have a flag |\version|
which can be set to |draft| or |final|.
The document source will contain some conditional code
depending on the value of |\version|.
Suppose further, the flag should default to |final| for the main file
and to |draft| for child files
which is a natural assignment for editing the document.
This is achieved by placing the following code
in the preamble of the main document
(below the |\childdocmain| directive):
%
\begin{center}
\begin{tabular}{l}
|\ifchilddoc|\\
|\providecommand{\version}{draft}|\\
|\||else|\\
|\providecommand{\version}{final}|\\
|\||fi|
\end{tabular}
\end{center}
%
The definition by |\providecommand| makes sure
that previous definitions are not overwritten.
Further statements |\providecommand{\version}{...}|
can thus be added before the above code to override it.

For the main file, one might add a line
(between |\childdocmain| and the above block)
%
\begin{center}
|%\ifchilddoc\||else\providecommand{\version}{draft}\||fi|
\end{center}
%
which can be uncommented to produce a draft version.
Likewise one can add a line to the very top of a child file
(above the |\childdocof{|\textit{main}|}| directive)
%
\begin{center}
|%\providecommand{\version}{final}|
\end{center}
%
which can be uncommented to produce the final version of this child document.

%%%%%%%%%%%%%%%%%%%%%%%%%%%%%%%%%%%%%%%%%%%%%%%%%%%%%%%%%%%%%%%%%%%%%%%%%%%%%%%%
\subsection{Forwarding}
\label{sec:forward}

Different versions of the main or child documents
using compilation flags as described in \secref{sec:flags}
can be (permanently) stored in different files
for convenient compilation, viewing and distribution.
To this end, the package defines a command
to pass on compilation to a different file:

%%%%%%%%%%%%%%%%%%%%%%%%%%%%%%%%%%%%%%%%
\DescribeMacro{\childdocforward}
The command |\childdocforward| redirects processing to
another source file:
%
\begin{center}
\begin{tabular}{l}
|\input{childdoc.def}|\\
|\childdocforward[|\textit{main}|]{|\textit{dest}|}|\\
\end{tabular}
\end{center}
%
The argument \textit{dest} is the destination file
(without extension).
It should be the main file or one of the child files.
Note that further \textsf{childdoc} directives
such as |\childdocof| and |\childdocforward|
in the indicated file will be processed in this form.
The optional argument \textit{main}
passes on directly to the main file \textit{main}
while pretending to compile the child \textit{dest}.
This form behaves as if \textit{dest}
issues |\childdocof{|\textit{main}|}| right away,
and no further \textsf{childdoc} directives will be processed.

%%%%%%%%%%%%%%%%%%%%%%%%%%%%%%%%%%%%%%%%
\DescribeMacro{\...prefix}
In the alternative form |\childdocforwardprefix|,
%
\begin{center}
\begin{tabular}{l}
|\input{childdoc.def}|\\
|\childdocforwardprefix[|\textit{main}|]{|\textit{prefix}|}{|\textit{dest}|}|
\end{tabular}
\end{center}
%
the destination file is determined by a pattern
depending on the current file:
To make this work, the current file must be called
`{\textit{prefix}\hspace{0.2em}\textit{suffix}}'
with \textit{prefix} matching precisely the argument.
Processing is then passed on to the file
`{\textit{dest}\hspace{0.2em}\textit{suffix}}'.
Surely, the same effect is achieved by
directly specifying the
argument `{\textit{dest}\hspace{0.2em}\textit{suffix}}'
in the first form.
However, that requires to set up a different file
for each child. With the alternative form of the command
all these files can have exactly the same content
which simplifies setting them up and maintaining them.

For example, the following file |draft.tex|
with a compilation flag |\version| as described in \secref{sec:flags}
compiles the main document as a draft:
%
\begin{center}
\begin{tabular}{l}
|\def\version{draft}|\\
|\input{childdoc.def}|\\
|\childdocforward{|\textit{main}|}|
\end{tabular}
\end{center}
%
Likewise, the following files |final|\textit{nn}|.tex|
compile the final version of the child document
|child|\textit{nn}|.tex|:
%
\begin{center}
\begin{tabular}{l}
|\def\version{final}|\\
|\input{childdoc.def}|\\
|\childdocforwardprefix{final}{child}|
\end{tabular}
\end{center}
%

Note that when several versions of a main file and/or of each child file
are to be generated, it may be convenient to set up a |Makefile| or
shell script to automatise the process.

%%%%%%%%%%%%%%%%%%%%%%%%%%%%%%%%%%%%%%%%%%%%%%%%%%%%%%%%%%%%%%%%%%%%%%%%%%%%%%%%
\subsection{Command Line Processing}
\label{sec:commandline}

The effect of redirection files can also be achieved by invoking
the \LaTeX{} compiler with a more elaborate command line.
Most conveniently this should be done as part
of a shell script or a |Makefile|.

When using \textsf{childdoc} in the main file, the following
command lines effectively perform a redirection
(note that depending on the shell being used,
backslashes may have to be doubled: `|\|' $\to$ `|\\|'):
%
\begin{center}
|... -jobname "|\textit{target}|" |\\|"|[\textit{flags}]%
|\input{childdoc.def}\childdocforward[|\textit{main}|]{|\textit{dest}|}"|
\end{center}
%
Here \textit{target} is the name of the output file,
\textit{main} is the name of the main file
and \textit{dest} is the name of the main or child file to be processed
(all filenames without extensions).
The optional argument \textit{main} can be omitted
if \textit{main} matches \textit{dest}.
Optionally, compilation \textit{flags} can be defined via |\def| commands.
This command line makes the \TeX{} engine believe
it is compiling the file \textit{target}
whose content is specified as the latter parameter.
The provided code then forwards the processing to
\textit{main} or \textit{dest} as described in \secref{sec:forward}.

%%%%%%%%%%%%%%%%%%%%%%%%%%%%%%%%%%%%%%%%%%%%%%%%%%%%%%%%%%%%%%%%%%%%%%%%%%%%%%%%
\subsection{Include by Input}
\label{sec:input}

Including child documents by |\include| has some restrictions by design.
Most notably, the content of a child document always occupies
its own set of pages; pages cannot be shared between child documents.
Usually, this behaviour makes perfect sense
because each child document contain an essential part of the document.
However, in some situations it may be desirable to compose
a document from a collection of parts
without having mandatory page breaks between then.
For this case, the package
provides a mechanism to include parts
by |\input| which can also be processed individually.
However, by construction this mechanism
requires manual handling of the content to be output.

%%%%%%%%%%%%%%%%%%%%%%%%%%%%%%%%%%%%%%%%
\DescribeMacro{\ifchilddocmanual}
The main file should be prepared as usual, see \secref{sec:include}.
However, the document body must make a distinction
between processing of an individual part and of the main document, e.g.:
%
\begin{center}
\begin{tabular}{l}
|\ifchilddocmanual|\\
|\input{\childdocname}|\\
|\||else|\\
\textit{document body with }|\input{|\textit{part}|}|\\
|\||fi|
\end{tabular}
\end{center}
%
The conditional |\ifchilddocmanual| is true whenever
a part to be included by |\input| is being compiled,
and the name of the part is stored in |\childdocname|.

%%%%%%%%%%%%%%%%%%%%%%%%%%%%%%%%%%%%%%%%
\DescribeMacro{\childdocby}
Each part to be included by |\input| should start with:
%
\begin{center}
\begin{tabular}{l}
|\input{childdoc.def}|\\
|\childdocby{|\textit{main}|}|\\
\end{tabular}
\end{center}
%
The directive |\childdocby| is similar to |\childdocof|
described in \secref{sec:include},
but the subsequent selection of content must be done manually.
To that end, both |\ifchilddoc| and |\ifchilddocmanual|
will be true upon processing of a part,
and the name of the part is stored in |\childdocname|.
Note that |\jobname| will be set to the filename of the current part
so that each part receives an individual |.aux| file
that does not interfere with the |.aux| file(s) of the main document.
This behaviour can be altered by the alternative form
|\childdocby[*]{|\textit{main}|}| (with a non-empty optional argument)
which uses the |.aux| file of the main document
by setting |\jobname| to \textit{main}.

%%%%%%%%%%%%%%%%%%%%%%%%%%%%%%%%%%%%%%%%%%%%%%%%%%%%%%%%%%%%%%%%%%%%%%%%%%%%%%%%
\subsection{Driver Development}
\label{sec:driver}

The \textsf{childdoc} mechanism can also be use for the development
of definition files such as \LaTeX{} styles or classes.
This case differs from the above setup with multiple parts
included by |\include| in that no |\includeonly| should be invoked.
This can be achieved by starting the include file
(before |\ProvidesPackage|) with:
%
\begin{center}
\begin{tabular}{l}
|\input{childdoc.def}|\\
|\childdocforward{|\textit{main}|}|\\
\end{tabular}
\end{center}
%
or alternatively with:
%
\begin{center}
\begin{tabular}{l}
|\input{childdoc.def}|\\
|\childdocby{|\textit{main}|}|\\
\end{tabular}
\end{center}
%
Both forms have slightly different effects as described above.
The main file is prepared as usual, see \secref{sec:include}.

%%%%%%%%%%%%%%%%%%%%%%%%%%%%%%%%%%%%%%%%%%%%%%%%%%%%%%%%%%%%%%%%%%%%%%%%%%%%%%%%
\subsection{Legacy Detection}
\label{sec:detection}

The directive |\childdocmain| in the main file can detect
whether the complete document or merely a child is to be compiled
even without using the directive |\childdocof|.
This method is deprecated because it is less robust
and there is no compelling reason to use it;
it is merely provided for backward compatibility
and it may be removed in future versions.

If the detection mechanism is to be used,
it is mandatory to correctly specify
the filename of the main file as the argument of |\childdocmain|:
%
\begin{center}
\begin{tabular}{l}
|\input{childdoc.def}|\\
|\childdocmain{|\textit{main}|}|\\
\end{tabular}
\end{center}
%
If |\jobname| does not match the argument \textit{main} of |\childdocmain|,
it is assumed that |\jobname| points to the child file to be compiled.
When using |\childdocmain| with the main file specified as argument,
it suffices to start a child file
with just |\input{|\textit{main}|}|
without loading of the package and using |\childdocof|.
If instead all processing is done
with the appropriate \textsf{childdoc} directives,
the argument of \textit{main} of |\childdocmain| can be empty.

An alternative version of the command line processing described
in \secref{sec:commandline} using the detection mechanism reads:
%
\begin{center}
|... -jobname "|\textit{target}|" "|[\textit{flags}]%
[|\def\jobname{|\textit{dest}|}|]|\input{|\textit{main}|}"|
\end{center}

%%%%%%%%%%%%%%%%%%%%%%%%%%%%%%%%%%%%%%%%%%%%%%%%%%%%%%%%%%%%%%%%%%%%%%%%%%%%%%%%
\subsection{Manual Code}
\label{sec:manual}

In case one cannot be certain whether the definitions file |childdoc.def|
is installed on the target \TeX{} distribution
and one prefers not to ship it,
it is conceivable to paste a few relevant commands into the sources.

To that end, drop all statements |\input{childdoc.def}|
and perform the replacements as outlined below.
Instead of |\childdocmain{|\textit{main}|}| add the following code
to the top of the main file:
%
\begin{center}
\begin{tabular}{l}
|\||ifdefined\childdocname\endinput\||fi\newif\ifchilddoc|\\
|\edef\childdocname{\scantokens\expandafter{\jobname\noexpand}}|\\
|\def\childdocmain{|\textit{main}|}\||ifx\childdocmain\childdocname\||else|\\
|\childdoctrue\includeonly{\childdocname}\let\jobname\childdocmain\||fi|\\
\end{tabular}
\end{center}
%
Instead of |\childdocof{|\textit{main}|}| just include the main file
at the top of each child file:
%
\begin{center}
|\input{|\textit{main}|}|
\end{center}
%
A simple redirection |\childdocforward{|\textit{dest}|}| is achieved by:
%
\begin{center}
|\def\jobname{|\textit{dest}|}\input{\jobname}|
\end{center}
%
The redirection with prefix
|\childdocforwardprefix[|\textit{prefix}|]{|\textit{dest}|}|
is accomplished by:
%
\begin{center}
\begin{tabular}{l}
|{\edef\jobname{\scantokens\expandafter{\jobname\noexpand}}|\\
|\def\redirectjob |\textit{prefix}|#1~~~{\gdef\jobname{|\textit{dest}|#1}}|\\
|\expandafter\redirectjob\jobname~~~}\input{\jobname}|
\end{tabular}
\end{center}

In an alternative approach,
child documents can be compiled by a specific command line
without additional code or specific definitions:
%
\begin{center}
|... -jobname "|\textit{target}|" "|[\textit{flags}]%
|\includeonly{|\textit{dest}|}\input{|\textit{main}|}"|
\end{center}
%

%%%%%%%%%%%%%%%%%%%%%%%%%%%%%%%%%%%%%%%%%%%%%%%%%%%%%%%%%%%%%%%%%%%%%%%%%%%%%%%%
%%%%%%%%%%%%%%%%%%%%%%%%%%%%%%%%%%%%%%%%%%%%%%%%%%%%%%%%%%%%%%%%%%%%%%%%%%%%%%%%
\section{Information}

%%%%%%%%%%%%%%%%%%%%%%%%%%%%%%%%%%%%%%%%%%%%%%%%%%%%%%%%%%%%%%%%%%%%%%%%%%%%%%%%
\subsection{Copyright}

Copyright \copyright{} 2017--2018 Niklas Beisert

This work may be distributed and/or modified under the
conditions of the \LaTeX{} Project Public License, either version 1.3
of this license or (at your option) any later version.
The latest version of this license is in
  \url{http://www.latex-project.org/lppl.txt}
and version 1.3 or later is part of all distributions of \LaTeX{}
version 2005/12/01 or later.

This work has the LPPL maintenance status `maintained'.

The Current Maintainer of this work is Niklas Beisert.

This work consists of the files |README.txt|, |childdoc.ins| and |childdoc.dtx|
as well as the derived files |childdoc.def|, |cdocsamp.tex|
with |cdocsch1.tex|, |cdocsch2.tex|, |cdocspt3.tex|, |cdocspt4.tex|,
|cdocsdrf.tex|, |cdocsfn1.tex|, |cdocsfn2.tex|
as well as |childdoc.pdf|.

%%%%%%%%%%%%%%%%%%%%%%%%%%%%%%%%%%%%%%%%%%%%%%%%%%%%%%%%%%%%%%%%%%%%%%%%%%%%%%%%
\subsection{Files and Installation}

The package consists of the files:
%
\begin{center}
\begin{tabular}{ll}
    |README.txt|   & readme file \\
    |childdoc.ins| & installation file \\
    |childdoc.dtx| & source file \\
    |childdoc.def| & definition file \\
    |cdocsamp.tex| & sample main file \\
    |cdocsch1.tex| & sample include file \\
    |cdocsch2.tex| & sample include file \\
    |cdocspt3.tex| & sample part file \\
    |cdocspt4.tex| & sample part file \\
    |cdocsdrf.tex| & sample redirection file \\
    |cdocsfn1.tex| & sample redirection file \\
    |cdocsfn2.tex| & sample redirection file \\
    |childdoc.pdf| & manual
\end{tabular}
\end{center}
%
The distribution consists of the files
|README.txt|, |childdoc.ins| and |childdoc.dtx|.
%
\begin{itemize}
\item
Run (pdf)\LaTeX{} on |childdoc.dtx|
to compile the manual |childdoc.pdf| (this file).
\item
Run \LaTeX{} on |childdoc.ins| to create the definitions file |childdoc.def|
and the sample |cdocsamp.tex| with include files
|cdocsch1.tex|, |cdocsch2.tex|, |cdocspt3.tex|, |cdocspt4.tex|,
|cdocsdrf.tex|, |cdocsfn1.tex|, |cdocsfn2.tex|.
Then copy the file |childdoc.def| to an appropriate directory of your \LaTeX{}
distribution, e.g.\ \textit{texmf-root}|/tex/latex/childdoc|.
\end{itemize}

%%%%%%%%%%%%%%%%%%%%%%%%%%%%%%%%%%%%%%%%%%%%%%%%%%%%%%%%%%%%%%%%%%%%%%%%%%%%%%%%
\subsection{Related CTAN Packages}

There are several other packages which offer a similar functionality:
%
\begin{itemize}
\item
The packages
\href{http://ctan.org/pkg/docmute}{\textsf{docmute}},
\href{http://ctan.org/pkg/includex}{\textsf{includex}} and
\href{http://ctan.org/pkg/standalone}{\textsf{standalone}}
provide commands to include only the document body of
a child file thus allowing both files to be compiled individually.
\item
The packages \href{http://ctan.org/pkg/subdocs}{\textsf{subdocs}}
and \href{http://ctan.org/pkg/subfiles}{\textsf{subfiles}}
provide structures in which the main and child documents can be
encapsulated and allowing them to be compiled individually.
The inclusion mechanism is different from the conventional |\include|.
\item
The package \href{http://ctan.org/pkg/combine}{\textsf{combine}}
is an elaborate solution to combine several documents into one.
\end{itemize}
%
See also the CTAN topic \href{http://ctan.org/topic/subdocs}{\textsf{subdocs}}
for further related packages.
The present package differs from the above solutions in that
a document structure constructed with the conventional |\include| mechanism
just needs two extra commands at the top of every file
such that all constituent files can be compiled individually.

%%%%%%%%%%%%%%%%%%%%%%%%%%%%%%%%%%%%%%%%%%%%%%%%%%%%%%%%%%%%%%%%%%%%%%%%%%%%%%%%
%\subsection{Feature Suggestions}
%
%The following is a list of features which may be useful for future
%versions of this package:
%%
%\begin{itemize}
%\item
%\ldots
%\end{itemize}

%%%%%%%%%%%%%%%%%%%%%%%%%%%%%%%%%%%%%%%%%%%%%%%%%%%%%%%%%%%%%%%%%%%%%%%%%%%%%%%%
\subsection{Revision History}

%%%%%%%%%%%%%%%%%%%%%%%%%%%%%%%%%%%%%%%%
\paragraph{v2.0:} 2018/12/30

\begin{itemize}
\item
immediate forward processing
\item
added |\childdocby| mechanism
\item
manual restructured
\end{itemize}

%%%%%%%%%%%%%%%%%%%%%%%%%%%%%%%%%%%%%%%%
\paragraph{v1.6:} 2018/01/17

\begin{itemize}
\item
application for development of include files
\item
corrections to manual
\end{itemize}

%%%%%%%%%%%%%%%%%%%%%%%%%%%%%%%%%%%%%%%%
\paragraph{v1.5:} 2017/05/21

\begin{itemize}
\item
more complete structuring introduced
\item
|\childdocof| introduced
\item
|\childdoc| renamed to |\childdocmain|
\item
|\childredirect| renamed to |\childdocforward| and |\childdocforwardprefix|
and functionality expanded
\end{itemize}

%%%%%%%%%%%%%%%%%%%%%%%%%%%%%%%%%%%%%%%%
\paragraph{v1.0:} 2017/04/27

\begin{itemize}
\item
manual and install package
\item
first version published on CTAN
\end{itemize}

%%%%%%%%%%%%%%%%%%%%%%%%%%%%%%%%%%%%%%%%
\paragraph{v0.6:} 2017/04/26

\begin{itemize}
\item
redirection mechanism added
\end{itemize}

%%%%%%%%%%%%%%%%%%%%%%%%%%%%%%%%%%%%%%%%
\paragraph{v0.5:} 2017/04/26

\begin{itemize}
\item
functionality in definition file
\end{itemize}


%%%%%%%%%%%%%%%%%%%%%%%%%%%%%%%%%%%%%%%%%%%%%%%%%%%%%%%%%%%%%%%%%%%%%%%%%%%%%%%%
%%%%%%%%%%%%%%%%%%%%%%%%%%%%%%%%%%%%%%%%%%%%%%%%%%%%%%%%%%%%%%%%%%%%%%%%%%%%%%%%
%%%%%%%%%%%%%%%%%%%%%%%%%%%%%%%%%%%%%%%%%%%%%%%%%%%%%%%%%%%%%%%%%%%%%%%%%%%%%%%%
\appendix

\settowidth\MacroIndent{\rmfamily\scriptsize 000\ }

 \DocInput{childdoc.dtx}

\end{document}
%</driver>
% \fi
%
% %%%%%%%%%%%%%%%%%%%%%%%%%%%%%%%%%%%%%%%%%%%%%%%%%%%%%%%%%%%%%%%%%%%%%%%%%%%%%%
% %%%%%%%%%%%%%%%%%%%%%%%%%%%%%%%%%%%%%%%%%%%%%%%%%%%%%%%%%%%%%%%%%%%%%%%%%%%%%%
% \section{Sample}
%\iffalse
%<*samplemain>
%\fi
%
% The following presents a sample document
% with two chapters, two parts, a title page,
% a compile flag as well as three forwarding files to set the flag.
% It consists of eight |.tex| files:
% \begin{center}
% \begin{tabular}{ll}
% |cdocsamp.tex|&main file\\
% |cdocsch1.tex|&include file for chapter 1\\
% |cdocsch2.tex|&include file for chapter 2\\
% |cdocspt3.tex|&include file for part 3\\
% |cdocspt4.tex|&include file for part 4\\
% |cdocsdrf.tex|&forwarding file for main file in draft mode\\
% |cdocsfi1.tex|&forwarding file for final version of chapter 1\\
% |cdocsfi2.tex|&forwarding file for final version of chapter 2\\
% \end{tabular}
% \end{center}
% Each of the eight files can be compiled directly by the \LaTeX{} compiler.
%
% %%%%%%%%%%%%%%%%%%%%%%%%%%%%%%%%%%%%%%
% \paragraph{Main File.}
%
% The main file is called |cdocsamp.tex|.
%
% Load the \textsf{childdoc} definitions and
% declare the filename for the main document:
%    \begin{macrocode}
\input{childdoc.def}
\childdocmain{}
%    \end{macrocode}

% Optional override for |\version| flag:
%    \begin{macrocode}
%%\ifchilddoc\else\providecommand{\version}{draft}\fi
%    \end{macrocode}

% Define the default values for the |\version| flag
% (|final| for the main file and |draft| for childs):
%    \begin{macrocode}
\ifchilddoc
\providecommand{\version}{draft}
\else
\providecommand{\version}{final}
\fi
%    \end{macrocode}

% Load the standard document class:
%    \begin{macrocode}
\documentclass[12pt]{article}
%    \end{macrocode}

% Start the document body:
%    \begin{macrocode}
\begin{document}
%    \end{macrocode}

% Declare a title page.
% Print title, part of document being processed and version flag:
%    \begin{macrocode}
\addtocounter{page}{-1}
\begin{center}
{\LARGE\bfseries{}childdoc example\par}
\vspace{1cm}
\ifchilddoc
\ifchilddocmanual part\else chapter\fi:
`\childdocname' of `\childdocjob'\par
\else
main document: `\childdocjob'\par
\fi
version: \version\par
\end{center}
\newpage
%    \end{macrocode}

% Manually include selected file,
% otherwise process as usual:
%    \begin{macrocode}
\ifchilddocmanual
\section*{part `\childdocname'}
\input{\childdocname}
\else
%    \end{macrocode}

% Include the two chapters:
%    \begin{macrocode}
\include{cdocsch1}
\include{cdocsch2}
%    \end{macrocode}

% Include the two parts unless only chapters should be displayed:
%    \begin{macrocode}
\ifchilddoc\else
\section{part three}
\input{cdocspt3}
\section{part four}
\input{cdocspt4}
\fi
%    \end{macrocode}

% Process as usual until here:
%    \begin{macrocode}
\fi
%    \end{macrocode}

% End of document body:
%    \begin{macrocode}
\end{document}
%    \end{macrocode}
%\iffalse
%</samplemain>
%\fi
%
% %%%%%%%%%%%%%%%%%%%%%%%%%%%%%%%%%%%%%%
% \paragraph{Chapter Include Files.}
%
% The include files are called |cdocsch1.tex| and |cdocsch2.tex|.
%
%\iffalse
%<*samplechap1|samplechap2>
%\fi

% Optional override for |\version| flag:
%    \begin{macrocode}
%%\providecommand{\version}{final}
%    \end{macrocode}

% Include the main document:
%    \begin{macrocode}
\input{childdoc.def}
\childdocof{cdocsamp}
%    \end{macrocode}

%\iffalse
%</samplechap1|samplechap2>
%\fi
%
%\iffalse
%<*samplechap1>
%\fi
% Some text for chapter 1:
%    \begin{macrocode}
\section{one}
some text in chapter one
%    \end{macrocode}

%\iffalse
%</samplechap1>
%\fi
% Some text for chapter 2:
%\iffalse
%<*samplechap2>
%\fi
%    \begin{macrocode}
\section{two}
more text in chapter two
%    \end{macrocode}

%\iffalse
%</samplechap2>
%\fi
%
% %%%%%%%%%%%%%%%%%%%%%%%%%%%%%%%%%%%%%%
% \paragraph{Part Include Files.}
%
% The include files are called |cdocspt3.tex| and |cdocspt4.tex|.
%
%\iffalse
%<*samplepart3|samplepart4>
%\fi

% Optional override for |\version| flag:
%    \begin{macrocode}
%%\providecommand{\version}{final}
%    \end{macrocode}

% Include the main document:
%    \begin{macrocode}
\input{childdoc.def}
\childdocby{cdocsamp}
%    \end{macrocode}

%\iffalse
%</samplepart3|samplepart4>
%\fi
%
%\iffalse
%<*samplepart3>
%\fi
% Some text for part 3:
%    \begin{macrocode}
some text in part three
%    \end{macrocode}

%\iffalse
%</samplepart3>
%\fi
% Some text for part 4:
%\iffalse
%<*samplepart4>
%\fi
%    \begin{macrocode}
more text in part four
%    \end{macrocode}

%\iffalse
%</samplepart4>
%\fi
%
% %%%%%%%%%%%%%%%%%%%%%%%%%%%%%%%%%%%%%%
% \paragraph{Forwarding for a Complete Draft.}
%
% The following forwarding file |cdocsdrf.tex|
% compiles the main document in draft mode:
%\iffalse
%<*sampledraft>
%\fi
%    \begin{macrocode}
\def\version{draft}
\input{childdoc.def}
\childdocforward{cdocsamp}
%    \end{macrocode}

%\iffalse
%</sampledraft>
%\fi
%
% %%%%%%%%%%%%%%%%%%%%%%%%%%%%%%%%%%%%%%
% \paragraph{Forwarding for Final Version of the Chapters.}
%
% The following forwarding files |cdocsfn1.tex| and |cdocsfn2.tex|
% (with identical content)
% compile the final versions of the child documents
% |cdocsch1.tex| and |cdocsch2.tex|, respectively:
%\iffalse
%<*samplefinal>
%\fi
%    \begin{macrocode}
\def\version{final}
\input{childdoc.def}
\childdocforwardprefix[cdocsamp]{cdocsfn}{cdocsch}
%    \end{macrocode}

%\iffalse
%</samplefinal>
%\fi
%
% %%%%%%%%%%%%%%%%%%%%%%%%%%%%%%%%%%%%%%
% \paragraph{Command Line Processing.}
%
% The following three command lines generate the output files
% |cdocscld|, |cdocscl1| and |cdocscl2|
% which should be identical to
% |cdocsdrf|, |cdocsch1| and |cdocsfn2|, respectively:
% \begin{center}
% \begin{tabular}{l}
% |latex -jobname cdocscld \|\\
% |  "\def\version{draft}\input{childdoc.def}\childdocforward{cdocsamp}"|\\
% |latex -jobname cdocscl1 \|\\
% |  "\input{childdoc.def}\childdocforward[cdocsamp]{cdocsch1}"|\\
% |latex -jobname cdocscl2 \|\\
% |  "\def\version{final}\input{childdoc.def}\childdocforward{cdocsch2}"|
% \end{tabular}
% \end{center}
% Note that the trailing backslash on each first line
% merely continues the input to the second line
% (for convenient cut ant paste).
% Furthermore, the command |latex| can be replaced by any
% of its alternative versions such as |pdflatex|.
%
% %%%%%%%%%%%%%%%%%%%%%%%%%%%%%%%%%%%%%%%%%%%%%%%%%%%%%%%%%%%%%%%%%%%%%%%%%%%%%%
% %%%%%%%%%%%%%%%%%%%%%%%%%%%%%%%%%%%%%%%%%%%%%%%%%%%%%%%%%%%%%%%%%%%%%%%%%%%%%%
% \section{Implementation}
%\iffalse
%<*package>
%\fi
%
% This section describes the definitions file |childdoc.def|.

% The definitions cannot be loaded using |\usepackage| or |\RequirePackage|
% which has a mechanism to prevent loading a style file more than once.
% When loading the definitions by means of |\input|
% multiple instances have to be prevented manually:
%\iffalse
%This code needs to be before the `\ProvidesFile' directive
%which is defined at the beginning of this file.
%Therefore it is also placed there and commented out here.
%</package>
%<*discard>
%\fi
%    \begin{macrocode}
\ifdefined\childdocmain\endinput\fi
%    \end{macrocode}
%\iffalse
%</discard>
%<*package>
%\fi
%
% \macro{\ifchilddoc}
% \macro{\ifchilddocmanual}
% The conditional |\ifchilddoc| tells whether a
% child (true) or main (false) document is being compiled.
% The conditional |\ifchilddocmanual| tells whether
% the |\includeonly| mechanism is used (false) or
% the selection of child files must be performed manually (true).
% The definitions initialise to false:
%    \begin{macrocode}
\newif\ifchilddoc
\newif\ifchilddocmanual
%    \end{macrocode}

% \macro{\childdocname}
% \macro{\childdocjob}
% The macro |\childdocname| stores the name of the main document
% to be compiled. The macro |\childdocjob| stores the name of
% the document on which the \LaTeX{} compiler was originally invoked.
% The content of |\jobname| cannot be compared
% to filenames specified in the source due to different catcodes.
% The following code rescans |\jobname|, stores the result
% in |\childdocname| and saves a copy in |\childdocjob|:
%    \begin{macrocode}
\edef\childdocname{\scantokens\expandafter{\jobname\noexpand}}
\let\childdocjob\childdocname
%    \end{macrocode}

% \macro{\childdocdisable}
% The macro |\childdocdisable| prevents the main file
% from being processed more than once.
% At this stage, the main document command |\childdocmain|
% is assumed to be called once again where it should do nothing.
% Any subsequent call to it should prevent
% a secondary processing of the main document
% It overwrites the forwarding commands
% |\childdocof| and |\childdocforward|
% with empty macros to prevent further inclusions of the main document:
%    \begin{macrocode}
\newcommand{\childdocdisable}
{
  \renewcommand{\childdocmain}[1]{\renewcommand{\childdocmain}[1]{\endinput}}
  \renewcommand{\childdocof}[1]{}
  \renewcommand{\childdocby}[2][]{}
  \renewcommand{\childdocforward}[2][]{}
  \renewcommand{\childdocdisable}{}
}
%    \end{macrocode}

% \macro{\childdocmain}
% The macro |\childdocmain| is to be called at the top of the main file
% with nothing or the main filename (without extension) as argument.
% First, it breaks loops.
% If the argument is not empty and does not match |\childdocname|
% (which is set by the first inclusion of |childdoc.def|),
% |\ifchilddoc| is set to true, |\includeonly| is applied to the child file
% and |\jobname| is set to the main file
% (for proper handling of |.aux| files):
%    \begin{macrocode}
\newcommand{\childdocmain}[1]
{
  \childdocdisable\childdocmain{}
  \if?#1?\else
    \begingroup
      \def\childdoctmp{#1}
      \ifx\childdoctmp\childdocname
        \def\childdoctmp{}
      \else
        \def\childdoctmp
        {
          \childdoctrue
          \includeonly{\childdocname}
          \def\childdocjob{#1}
          \def\jobname{#1}
        }
      \fi
      \expandafter
    \endgroup
    \childdoctmp
  \fi
}
%    \end{macrocode}

% \macro{\childdocof}
% The command |\childdocof| redirects
% compilation to the main file |#1|.
%    \begin{macrocode}
\newcommand{\childdocof}[1]
{
  \childdocdisable
  \childdoctrue
  \includeonly{\childdocname}
  \def\jobname{#1}
  \def\childdocjob{#1}
  \input{#1}
}
%    \end{macrocode}

% \macro{\childdocby}
% The command |\childdocby| ....
%    \begin{macrocode}
\newcommand{\childdocby}[2][]
{
  \childdocdisable
  \childdoctrue
  \childdocmanualtrue
  \if?#1?\else
    \def\jobname{#2}
  \fi
  \def\childdocjob{#2}
  \input{#2}
  \endinput
}
%    \end{macrocode}

% \macro{\childdocforward}
% The command |\childdocforward| redirects
% compilation to the main file or
% (if the optional argument is given) a child file.
% Parameters are set as if the main file
% or a child file starting with |\childdocof| was compiled.
% Then compilation is handed over to the main file:
%    \begin{macrocode}
\newcommand{\childdocforward}[2][]
{
  \begingroup
    \if?#1?
      \def\childdoctmp
      {
        \def\childdocname{#2}
        \def\childdocjob{#2}
        \def\jobname{#2}
        \input{#2}
        \endinput
      }
    \else
      \def\childdoctmp
      {
        \childdocdisable
        \def\childdocname{#2}
        \childdoctrue
        \includeonly{#2}
        \def\childdocjob{#1}
        \def\jobname{#1}
        \input{#1}
        \endinput
      }
    \fi
    \expandafter
  \endgroup
  \childdoctmp
}
%    \end{macrocode}

% \macro{\childdocforwardprefix}
% The command |\childdocforwardprefix| redirects
% compilation to the main or a child file by means of a pattern.
% The prefix |#1| in the current filename is replaced by |#2|
% and the suffix of the current filename is kept
% (it is assumed that the filename does not contain the substring `|~~~|'
% which is used as a delimiter).
% Compilation is handed over to the new file by |\childdocforward|:
%    \begin{macrocode}
\newcommand{\childdocforwardprefix}[3][]
{
  \begingroup
    \def\childdocextract #2##1~~~{\def\childdoctmp{\childdocforward[#1]{#3##1}}}
    \expandafter\childdocextract\childdocname~~~
    \expandafter
  \endgroup
  \childdoctmp
}
%    \end{macrocode}

% \macro{\childdoc}
% The deprecated macro |\childdoc| is a legacy version of |\childdocmain|:
%    \begin{macrocode}
\newcommand{\childdoc}{\childdocmain}
%    \end{macrocode}

% \macro{\childdocredirect}
% The deprecated macro |\childdocredirect| is a legacy version
% of |\childdocforward| and |\childdocforwardprefix|:
%    \begin{macrocode}
\newcommand{\childdocredirect}[2][]
{
  \begingroup
    \if?#1?
      \def\childdoctmp{\childdocforward{#2}}
    \else
      \def\childdoctmp{\childdocforwardprefix{#1}{#2}}
    \fi
    \expandafter
  \endgroup
  \childdoctmp
}
%    \end{macrocode}

%\iffalse
%</package>
%\fi
%
\endinput
\childdocforward{cdocsamp}"|\\
% |latex -jobname cdocscl1 \|\\
% |  "% \iffalse
%
% childdoc.dtx Copyright (C) 2017-2018 Niklas Beisert
%
% This work may be distributed and/or modified under the
% conditions of the LaTeX Project Public License, either version 1.3
% of this license or (at your option) any later version.
% The latest version of this license is in
%   http://www.latex-project.org/lppl.txt
% and version 1.3 or later is part of all distributions of LaTeX
% version 2005/12/01 or later.
%
% This work has the LPPL maintenance status `maintained'.
%
% The Current Maintainer of this work is Niklas Beisert.
%
% This work consists of the files childdoc.dtx and childdoc.ins
% and the derived files childdoc.def and cdocsamp.tex with
% cdocsch1.tex, cdocsch2.tex, cdocsdrf.tex, cdocsfn1.tex, cdocsfn2.tex.
%
%<package>\ifdefined\childdocmain\endinput\fi
%<package>\ProvidesFile{childdoc.def}[2018/12/30 v2.0 child document driver]
%<samplemain>\ProvidesFile{cdocsamp.tex}[2018/12/30 v2.0 sample for childdoc]
%<*driver>
%\ProvidesFile{childdoc.drv}[2018/12/30 v2.0 childdoc reference manual file]
\PassOptionsToClass{10pt,a4paper}{article}
\documentclass{ltxdoc}

\usepackage[margin=35mm]{geometry}
\usepackage{hyperref}
\usepackage{hyperxmp}
\usepackage[usenames]{color}

\hypersetup{colorlinks=true}
\hypersetup{pdfstartview=FitH}
\hypersetup{pdfpagemode=UseNone}
\hypersetup{pdfsource={}}
\hypersetup{pdflang={en-UK}}
\hypersetup{pdfcopyright={Copyright 2017-2018 Niklas Beisert.
  This work may be distributed and/or modified under the
  conditions of the LaTeX Project Public License, either version 1.3
  of this license or (at your option) any later version.}}
\hypersetup{pdflicenseurl={http://www.latex-project.org/lppl.txt}}
\hypersetup{pdfcontactaddress={ETH Zurich, ITP, HIT K,
  Wolfgang-Pauli-Strasse 27}}
\hypersetup{pdfcontactpostcode={8093}}
\hypersetup{pdfcontactcity={Zurich}}
\hypersetup{pdfcontactcountry={Switzerland}}
\hypersetup{pdfcontactemail={nbeisert@itp.phys.ethz.ch}}
\hypersetup{pdfcontacturl={http://people.phys.ethz.ch/\xmptilde nbeisert/}}

\newcommand{\secref}[1]{\hyperref[#1]{section \ref*{#1}}}

\parskip1ex
\parindent0pt
\let\olditemize\itemize
\def\itemize{\olditemize\parskip0pt}

\begin{document}

\title{The \textsf{childdoc} Package}
\hypersetup{pdftitle={The childdoc Package}}
\author{Niklas Beisert\\[2ex]
  Institut f\"ur Theoretische Physik\\
  Eidgen\"ossische Technische Hochschule Z\"urich\\
  Wolfgang-Pauli-Strasse 27, 8093 Z\"urich, Switzerland\\[1ex]
  \href{mailto:nbeisert@itp.phys.ethz.ch}
  {\texttt{nbeisert@itp.phys.ethz.ch}}}
\hypersetup{pdfauthor={Niklas Beisert}}
\hypersetup{pdfsubject={Manual for the LaTeX2e Package childdoc}}
\date{30 December 2018, \textsf{v2.0}}
\maketitle

\begin{abstract}\noindent
\textsf{childdoc} is a \LaTeXe{} package
that enables the direct compilation
of document sections included by |\include|
to individual files.
\end{abstract}

\begingroup
\parskip0ex
\tableofcontents
\endgroup

%%%%%%%%%%%%%%%%%%%%%%%%%%%%%%%%%%%%%%%%%%%%%%%%%%%%%%%%%%%%%%%%%%%%%%%%%%%%%%%%
%%%%%%%%%%%%%%%%%%%%%%%%%%%%%%%%%%%%%%%%%%%%%%%%%%%%%%%%%%%%%%%%%%%%%%%%%%%%%%%%
\section{Introduction}

\LaTeX{} provides a mechanism to structure a large document (such as a book)
into a main file and several child files (containing the chapters)
using the |\include| command.
This mechanism is beneficial for documents
which span hundreds of pages in order to
make the source file(s) more manageable.
Moreover, compilation can be restricted to
selected child files by means of the |\includeonly| command.
The latter feature can be used to reduce the compilation time while editing
(this was significantly more useful in the earlier days of \LaTeX{})
or to generate a smaller document which is easier to navigate.
Another application of |\includeonly| is to generate
documents consisting of selected parts of the complete document.

However, there are a few drawbacks of the plain |\include| mechanism:
\begin{itemize}
\item
The child files cannot be compiled on their own,
they can only be compiled via the main file.
A naive editing environment
(such as a text editor with an option
to have the current file processed by \LaTeX)
may require one to switch to the main file before compiling;
attempting to compile the child file produces errors.
\item
The main file must be modified (each time)
to adjust the |\includeonly| command
to the present needs. This easily leaves the main file in a messy state.
\item
The generated document will always carry the filename
of the main document. This is inconvenient if
several child files are to be compiled and
to be kept for distribution.
\end{itemize}

The present package provides a simple interface
to make child files individually compilable by \LaTeX{}.
Compiling a child file then has the same effect as compiling
the main file with an |\includeonly| command
to select the appropriate child.
Moreover the generated document will carry the name of the child
rather than the main file.
This resolves all three above issues.

This feature is meant to make the editing of books,
thesis documents and lecture notes somewhat more convenient.
However, the package can also be used efficiently for
composing a series of documents (such as exercise sheets)
which are typically distributed individually.
It then assists the author in generating the individual documents
(potentially in different versions)
as well as a document containing the collected series.
Another application is in developing style files
or other kinds of included material
where compilation of the style file could redirect
to a sample or test file.

%%%%%%%%%%%%%%%%%%%%%%%%%%%%%%%%%%%%%%%%%%%%%%%%%%%%%%%%%%%%%%%%%%%%%%%%%%%%%%%%
%%%%%%%%%%%%%%%%%%%%%%%%%%%%%%%%%%%%%%%%%%%%%%%%%%%%%%%%%%%%%%%%%%%%%%%%%%%%%%%%
\section{Usage}

First of all, the package \textsf{childdoc} is \emph{not} a standard
\LaTeXe{} |.sty| style file! Therefore it needs to be invoked in
a non-standard way.

%%%%%%%%%%%%%%%%%%%%%%%%%%%%%%%%%%%%%%%%%%%%%%%%%%%%%%%%%%%%%%%%%%%%%%%%%%%%%%%%
\subsection{Included Files}
\label{sec:include}

%%%%%%%%%%%%%%%%%%%%%%%%%%%%%%%%%%%%%%%%
\DescribeMacro{\childdocmain}
To use the package, add the commands
\begin{center}
\begin{tabular}{l}
|\input{childdoc.def}|\\
|\childdocmain{}|\\
\end{tabular}
\end{center}
at the very top of the main \LaTeX{} file,
in particular \emph{before} the |\documentclass| statement!
The argument of |\childdocmain| should be left empty
(but it must be present).

%%%%%%%%%%%%%%%%%%%%%%%%%%%%%%%%%%%%%%%%
\DescribeMacro{\childdocof}
Furthermore, add the commands
\begin{center}
\begin{tabular}{l}
|\input{childdoc.def}|\\
|\childdocof{|\textit{main}|}|\\
\end{tabular}
\end{center}
at the top of every child file \textit{child}
which is included by |\include{|\textit{child}|}|
from within the main file
(or at least for those files to be compiled individually).
The argument \textit{main} must be the filename of the main file.

There are a couple of
considerations in setting up the main and child documents:

%%%%%%%%%%%%%%%%%%%%%%%%%%%%%%%%%%%%%%%%
\paragraph{Restrictions.}

Please note the following restrictions:
\begin{itemize}
\item
|\childdocmain| must be called with one argument \textit{main}
to ensure compatibility with earlier version of the package.
It must either be empty (|\childdocmain{}|)
or precisely match the filename of the main file in which it is specified.
See \secref{sec:detection} for further information.
\item
The filename \textit{main} must be specified without the |.tex| extension.
\item
The filename \textit{main} is case sensitive
(even in case-insensitive file systems)
due to internal string comparison.
\item
The argument \textit{main} should be fully expanded, it cannot be a macro.
\item
Subdirectories and special characters should be avoided in filenames.
\item
The command |\childdocmain{|\textit{main}|}| must be followed by a whitespace.
It should not be followed immediately by another command
or by a comment mark `|%|'.
This is because the \TeX{} parser reads the token immediately following
the argument of |\childdocmain| and puts it
at the beginning of every child section;
however, a white\-space is ignored.
\end{itemize}

%%%%%%%%%%%%%%%%%%%%%%%%%%%%%%%%%%%%%%%%
\paragraph{Content of Main File.}

It is advisable to place all content in the child files included by |\include|.
Any output contained in the main file will appear in all child documents
unless suppressed manually;
it cannot be suppressed automatically by the |\includeonly| directive
and thus should normally be avoided.
A method to include some content in the main file
by means of conditional processing is described in \secref{sec:conditional}.

%%%%%%%%%%%%%%%%%%%%%%%%%%%%%%%%%%%%%%%%
\paragraph{Page Numbering.}

When only a part of the document is compiled,
the appropriate numbering of pages
(as well as other status parameters)
is determined from the |.aux| files.
The latter contain information from previous passes.
However this information needs to propagate through
all intermediate child documents.
Therefore the page numbering in child documents may well
be inconsistent until the complete document is compiled at least once.

A useful (if unconventional) way to always ensure a consistent
page numbering is to restart the numbering in each child document
and denote the pages by `\textit{child}|.|\textit{page}'
where \textit{child} represents the chapter/section number of the child file.
This can be achieved by the command
|\numberwithin{page}{|\textit{child}|}|
of the \textsf{amsmath} package
where \textit{child} can be |chapter| or |section|
depending on the chosen structuring.
Alternatively, one can modify the macro |\thepage| appropriately
and reset the counter |page| at the start of each child file.

%%%%%%%%%%%%%%%%%%%%%%%%%%%%%%%%%%%%%%%%%%%%%%%%%%%%%%%%%%%%%%%%%%%%%%%%%%%%%%%%
\subsection{Conditional Processing}
\label{sec:conditional}

The package provides a mechanism to compile different versions
of a document. To customise the versions further some conditional processing
can come in handy to distinguish which version is being compiled.
The package provides two macros to describe the compilation context:

%%%%%%%%%%%%%%%%%%%%%%%%%%%%%%%%%%%%%%%%
\DescribeMacro{\ifchilddoc}
The conditional |\ifchilddoc| distinguishes between the compilation of
child documents and the main document:
%
\begin{center}
|\ifchilddoc |\textit{child-code}| |[|\||else |\textit{main-code}]| \||fi|
\end{center}

%%%%%%%%%%%%%%%%%%%%%%%%%%%%%%%%%%%%%%%%
\DescribeMacro{\childdocname}
\DescribeMacro{\childdocjob}
The macro |\childdocname| contains the filename (without extension)
of the main or child file being processed.
Note that |\childdocjob| will always contain the name of the main file.

%%%%%%%%%%%%%%%%%%%%%%%%%%%%%%%%%%%%%%%%
\paragraph{Title Page.}

Conditional processing can be used to include a title or banner page
in the main document when proper precautions are taken.
Importantly, the code in the main file should ensure that the page counter
(as well as other status parameters which are stored in the |.aux| files)
takes the same value after the conditional processing.
Otherwise the page numbers may take divergent values
depending on which part is compiled.

For example, a title page could be declared by:
%
\begin{center}
\begin{tabular}{l}
|\ifchilddoc\||else|\\
|\addtocounter{page}{-1}|\\
\textit{code for title page}\\
|\newpage|\\
|\||fi|
\end{tabular}
\end{center}
%
A banner page for the child documents can be generated by:
%
\begin{center}
\begin{tabular}{l}
|\ifchilddoc|\\
|\addtocounter{page}{-1}|\\
\textit{code for banner page}\\
|\newpage|\\
|\||fi|
\end{tabular}
\end{center}
%
Here one could write a message such as:
\begin{center}
|This is the part \childdocname{} of \childdocjob{}.|
\end{center}

%%%%%%%%%%%%%%%%%%%%%%%%%%%%%%%%%%%%%%%%%%%%%%%%%%%%%%%%%%%%%%%%%%%%%%%%%%%%%%%%
\subsection{Flags}
\label{sec:flags}

The package makes it easy to generate different versions
of the main or child documents.
To this end compilation flags can be defined
and assigned different default values.
They will be particularly useful in conjunction
with the forwarding mechanism described in \secref{sec:forward}.

For example, it may be useful to have a flag |\version|
which can be set to |draft| or |final|.
The document source will contain some conditional code
depending on the value of |\version|.
Suppose further, the flag should default to |final| for the main file
and to |draft| for child files
which is a natural assignment for editing the document.
This is achieved by placing the following code
in the preamble of the main document
(below the |\childdocmain| directive):
%
\begin{center}
\begin{tabular}{l}
|\ifchilddoc|\\
|\providecommand{\version}{draft}|\\
|\||else|\\
|\providecommand{\version}{final}|\\
|\||fi|
\end{tabular}
\end{center}
%
The definition by |\providecommand| makes sure
that previous definitions are not overwritten.
Further statements |\providecommand{\version}{...}|
can thus be added before the above code to override it.

For the main file, one might add a line
(between |\childdocmain| and the above block)
%
\begin{center}
|%\ifchilddoc\||else\providecommand{\version}{draft}\||fi|
\end{center}
%
which can be uncommented to produce a draft version.
Likewise one can add a line to the very top of a child file
(above the |\childdocof{|\textit{main}|}| directive)
%
\begin{center}
|%\providecommand{\version}{final}|
\end{center}
%
which can be uncommented to produce the final version of this child document.

%%%%%%%%%%%%%%%%%%%%%%%%%%%%%%%%%%%%%%%%%%%%%%%%%%%%%%%%%%%%%%%%%%%%%%%%%%%%%%%%
\subsection{Forwarding}
\label{sec:forward}

Different versions of the main or child documents
using compilation flags as described in \secref{sec:flags}
can be (permanently) stored in different files
for convenient compilation, viewing and distribution.
To this end, the package defines a command
to pass on compilation to a different file:

%%%%%%%%%%%%%%%%%%%%%%%%%%%%%%%%%%%%%%%%
\DescribeMacro{\childdocforward}
The command |\childdocforward| redirects processing to
another source file:
%
\begin{center}
\begin{tabular}{l}
|\input{childdoc.def}|\\
|\childdocforward[|\textit{main}|]{|\textit{dest}|}|\\
\end{tabular}
\end{center}
%
The argument \textit{dest} is the destination file
(without extension).
It should be the main file or one of the child files.
Note that further \textsf{childdoc} directives
such as |\childdocof| and |\childdocforward|
in the indicated file will be processed in this form.
The optional argument \textit{main}
passes on directly to the main file \textit{main}
while pretending to compile the child \textit{dest}.
This form behaves as if \textit{dest}
issues |\childdocof{|\textit{main}|}| right away,
and no further \textsf{childdoc} directives will be processed.

%%%%%%%%%%%%%%%%%%%%%%%%%%%%%%%%%%%%%%%%
\DescribeMacro{\...prefix}
In the alternative form |\childdocforwardprefix|,
%
\begin{center}
\begin{tabular}{l}
|\input{childdoc.def}|\\
|\childdocforwardprefix[|\textit{main}|]{|\textit{prefix}|}{|\textit{dest}|}|
\end{tabular}
\end{center}
%
the destination file is determined by a pattern
depending on the current file:
To make this work, the current file must be called
`{\textit{prefix}\hspace{0.2em}\textit{suffix}}'
with \textit{prefix} matching precisely the argument.
Processing is then passed on to the file
`{\textit{dest}\hspace{0.2em}\textit{suffix}}'.
Surely, the same effect is achieved by
directly specifying the
argument `{\textit{dest}\hspace{0.2em}\textit{suffix}}'
in the first form.
However, that requires to set up a different file
for each child. With the alternative form of the command
all these files can have exactly the same content
which simplifies setting them up and maintaining them.

For example, the following file |draft.tex|
with a compilation flag |\version| as described in \secref{sec:flags}
compiles the main document as a draft:
%
\begin{center}
\begin{tabular}{l}
|\def\version{draft}|\\
|\input{childdoc.def}|\\
|\childdocforward{|\textit{main}|}|
\end{tabular}
\end{center}
%
Likewise, the following files |final|\textit{nn}|.tex|
compile the final version of the child document
|child|\textit{nn}|.tex|:
%
\begin{center}
\begin{tabular}{l}
|\def\version{final}|\\
|\input{childdoc.def}|\\
|\childdocforwardprefix{final}{child}|
\end{tabular}
\end{center}
%

Note that when several versions of a main file and/or of each child file
are to be generated, it may be convenient to set up a |Makefile| or
shell script to automatise the process.

%%%%%%%%%%%%%%%%%%%%%%%%%%%%%%%%%%%%%%%%%%%%%%%%%%%%%%%%%%%%%%%%%%%%%%%%%%%%%%%%
\subsection{Command Line Processing}
\label{sec:commandline}

The effect of redirection files can also be achieved by invoking
the \LaTeX{} compiler with a more elaborate command line.
Most conveniently this should be done as part
of a shell script or a |Makefile|.

When using \textsf{childdoc} in the main file, the following
command lines effectively perform a redirection
(note that depending on the shell being used,
backslashes may have to be doubled: `|\|' $\to$ `|\\|'):
%
\begin{center}
|... -jobname "|\textit{target}|" |\\|"|[\textit{flags}]%
|\input{childdoc.def}\childdocforward[|\textit{main}|]{|\textit{dest}|}"|
\end{center}
%
Here \textit{target} is the name of the output file,
\textit{main} is the name of the main file
and \textit{dest} is the name of the main or child file to be processed
(all filenames without extensions).
The optional argument \textit{main} can be omitted
if \textit{main} matches \textit{dest}.
Optionally, compilation \textit{flags} can be defined via |\def| commands.
This command line makes the \TeX{} engine believe
it is compiling the file \textit{target}
whose content is specified as the latter parameter.
The provided code then forwards the processing to
\textit{main} or \textit{dest} as described in \secref{sec:forward}.

%%%%%%%%%%%%%%%%%%%%%%%%%%%%%%%%%%%%%%%%%%%%%%%%%%%%%%%%%%%%%%%%%%%%%%%%%%%%%%%%
\subsection{Include by Input}
\label{sec:input}

Including child documents by |\include| has some restrictions by design.
Most notably, the content of a child document always occupies
its own set of pages; pages cannot be shared between child documents.
Usually, this behaviour makes perfect sense
because each child document contain an essential part of the document.
However, in some situations it may be desirable to compose
a document from a collection of parts
without having mandatory page breaks between then.
For this case, the package
provides a mechanism to include parts
by |\input| which can also be processed individually.
However, by construction this mechanism
requires manual handling of the content to be output.

%%%%%%%%%%%%%%%%%%%%%%%%%%%%%%%%%%%%%%%%
\DescribeMacro{\ifchilddocmanual}
The main file should be prepared as usual, see \secref{sec:include}.
However, the document body must make a distinction
between processing of an individual part and of the main document, e.g.:
%
\begin{center}
\begin{tabular}{l}
|\ifchilddocmanual|\\
|\input{\childdocname}|\\
|\||else|\\
\textit{document body with }|\input{|\textit{part}|}|\\
|\||fi|
\end{tabular}
\end{center}
%
The conditional |\ifchilddocmanual| is true whenever
a part to be included by |\input| is being compiled,
and the name of the part is stored in |\childdocname|.

%%%%%%%%%%%%%%%%%%%%%%%%%%%%%%%%%%%%%%%%
\DescribeMacro{\childdocby}
Each part to be included by |\input| should start with:
%
\begin{center}
\begin{tabular}{l}
|\input{childdoc.def}|\\
|\childdocby{|\textit{main}|}|\\
\end{tabular}
\end{center}
%
The directive |\childdocby| is similar to |\childdocof|
described in \secref{sec:include},
but the subsequent selection of content must be done manually.
To that end, both |\ifchilddoc| and |\ifchilddocmanual|
will be true upon processing of a part,
and the name of the part is stored in |\childdocname|.
Note that |\jobname| will be set to the filename of the current part
so that each part receives an individual |.aux| file
that does not interfere with the |.aux| file(s) of the main document.
This behaviour can be altered by the alternative form
|\childdocby[*]{|\textit{main}|}| (with a non-empty optional argument)
which uses the |.aux| file of the main document
by setting |\jobname| to \textit{main}.

%%%%%%%%%%%%%%%%%%%%%%%%%%%%%%%%%%%%%%%%%%%%%%%%%%%%%%%%%%%%%%%%%%%%%%%%%%%%%%%%
\subsection{Driver Development}
\label{sec:driver}

The \textsf{childdoc} mechanism can also be use for the development
of definition files such as \LaTeX{} styles or classes.
This case differs from the above setup with multiple parts
included by |\include| in that no |\includeonly| should be invoked.
This can be achieved by starting the include file
(before |\ProvidesPackage|) with:
%
\begin{center}
\begin{tabular}{l}
|\input{childdoc.def}|\\
|\childdocforward{|\textit{main}|}|\\
\end{tabular}
\end{center}
%
or alternatively with:
%
\begin{center}
\begin{tabular}{l}
|\input{childdoc.def}|\\
|\childdocby{|\textit{main}|}|\\
\end{tabular}
\end{center}
%
Both forms have slightly different effects as described above.
The main file is prepared as usual, see \secref{sec:include}.

%%%%%%%%%%%%%%%%%%%%%%%%%%%%%%%%%%%%%%%%%%%%%%%%%%%%%%%%%%%%%%%%%%%%%%%%%%%%%%%%
\subsection{Legacy Detection}
\label{sec:detection}

The directive |\childdocmain| in the main file can detect
whether the complete document or merely a child is to be compiled
even without using the directive |\childdocof|.
This method is deprecated because it is less robust
and there is no compelling reason to use it;
it is merely provided for backward compatibility
and it may be removed in future versions.

If the detection mechanism is to be used,
it is mandatory to correctly specify
the filename of the main file as the argument of |\childdocmain|:
%
\begin{center}
\begin{tabular}{l}
|\input{childdoc.def}|\\
|\childdocmain{|\textit{main}|}|\\
\end{tabular}
\end{center}
%
If |\jobname| does not match the argument \textit{main} of |\childdocmain|,
it is assumed that |\jobname| points to the child file to be compiled.
When using |\childdocmain| with the main file specified as argument,
it suffices to start a child file
with just |\input{|\textit{main}|}|
without loading of the package and using |\childdocof|.
If instead all processing is done
with the appropriate \textsf{childdoc} directives,
the argument of \textit{main} of |\childdocmain| can be empty.

An alternative version of the command line processing described
in \secref{sec:commandline} using the detection mechanism reads:
%
\begin{center}
|... -jobname "|\textit{target}|" "|[\textit{flags}]%
[|\def\jobname{|\textit{dest}|}|]|\input{|\textit{main}|}"|
\end{center}

%%%%%%%%%%%%%%%%%%%%%%%%%%%%%%%%%%%%%%%%%%%%%%%%%%%%%%%%%%%%%%%%%%%%%%%%%%%%%%%%
\subsection{Manual Code}
\label{sec:manual}

In case one cannot be certain whether the definitions file |childdoc.def|
is installed on the target \TeX{} distribution
and one prefers not to ship it,
it is conceivable to paste a few relevant commands into the sources.

To that end, drop all statements |\input{childdoc.def}|
and perform the replacements as outlined below.
Instead of |\childdocmain{|\textit{main}|}| add the following code
to the top of the main file:
%
\begin{center}
\begin{tabular}{l}
|\||ifdefined\childdocname\endinput\||fi\newif\ifchilddoc|\\
|\edef\childdocname{\scantokens\expandafter{\jobname\noexpand}}|\\
|\def\childdocmain{|\textit{main}|}\||ifx\childdocmain\childdocname\||else|\\
|\childdoctrue\includeonly{\childdocname}\let\jobname\childdocmain\||fi|\\
\end{tabular}
\end{center}
%
Instead of |\childdocof{|\textit{main}|}| just include the main file
at the top of each child file:
%
\begin{center}
|\input{|\textit{main}|}|
\end{center}
%
A simple redirection |\childdocforward{|\textit{dest}|}| is achieved by:
%
\begin{center}
|\def\jobname{|\textit{dest}|}\input{\jobname}|
\end{center}
%
The redirection with prefix
|\childdocforwardprefix[|\textit{prefix}|]{|\textit{dest}|}|
is accomplished by:
%
\begin{center}
\begin{tabular}{l}
|{\edef\jobname{\scantokens\expandafter{\jobname\noexpand}}|\\
|\def\redirectjob |\textit{prefix}|#1~~~{\gdef\jobname{|\textit{dest}|#1}}|\\
|\expandafter\redirectjob\jobname~~~}\input{\jobname}|
\end{tabular}
\end{center}

In an alternative approach,
child documents can be compiled by a specific command line
without additional code or specific definitions:
%
\begin{center}
|... -jobname "|\textit{target}|" "|[\textit{flags}]%
|\includeonly{|\textit{dest}|}\input{|\textit{main}|}"|
\end{center}
%

%%%%%%%%%%%%%%%%%%%%%%%%%%%%%%%%%%%%%%%%%%%%%%%%%%%%%%%%%%%%%%%%%%%%%%%%%%%%%%%%
%%%%%%%%%%%%%%%%%%%%%%%%%%%%%%%%%%%%%%%%%%%%%%%%%%%%%%%%%%%%%%%%%%%%%%%%%%%%%%%%
\section{Information}

%%%%%%%%%%%%%%%%%%%%%%%%%%%%%%%%%%%%%%%%%%%%%%%%%%%%%%%%%%%%%%%%%%%%%%%%%%%%%%%%
\subsection{Copyright}

Copyright \copyright{} 2017--2018 Niklas Beisert

This work may be distributed and/or modified under the
conditions of the \LaTeX{} Project Public License, either version 1.3
of this license or (at your option) any later version.
The latest version of this license is in
  \url{http://www.latex-project.org/lppl.txt}
and version 1.3 or later is part of all distributions of \LaTeX{}
version 2005/12/01 or later.

This work has the LPPL maintenance status `maintained'.

The Current Maintainer of this work is Niklas Beisert.

This work consists of the files |README.txt|, |childdoc.ins| and |childdoc.dtx|
as well as the derived files |childdoc.def|, |cdocsamp.tex|
with |cdocsch1.tex|, |cdocsch2.tex|, |cdocspt3.tex|, |cdocspt4.tex|,
|cdocsdrf.tex|, |cdocsfn1.tex|, |cdocsfn2.tex|
as well as |childdoc.pdf|.

%%%%%%%%%%%%%%%%%%%%%%%%%%%%%%%%%%%%%%%%%%%%%%%%%%%%%%%%%%%%%%%%%%%%%%%%%%%%%%%%
\subsection{Files and Installation}

The package consists of the files:
%
\begin{center}
\begin{tabular}{ll}
    |README.txt|   & readme file \\
    |childdoc.ins| & installation file \\
    |childdoc.dtx| & source file \\
    |childdoc.def| & definition file \\
    |cdocsamp.tex| & sample main file \\
    |cdocsch1.tex| & sample include file \\
    |cdocsch2.tex| & sample include file \\
    |cdocspt3.tex| & sample part file \\
    |cdocspt4.tex| & sample part file \\
    |cdocsdrf.tex| & sample redirection file \\
    |cdocsfn1.tex| & sample redirection file \\
    |cdocsfn2.tex| & sample redirection file \\
    |childdoc.pdf| & manual
\end{tabular}
\end{center}
%
The distribution consists of the files
|README.txt|, |childdoc.ins| and |childdoc.dtx|.
%
\begin{itemize}
\item
Run (pdf)\LaTeX{} on |childdoc.dtx|
to compile the manual |childdoc.pdf| (this file).
\item
Run \LaTeX{} on |childdoc.ins| to create the definitions file |childdoc.def|
and the sample |cdocsamp.tex| with include files
|cdocsch1.tex|, |cdocsch2.tex|, |cdocspt3.tex|, |cdocspt4.tex|,
|cdocsdrf.tex|, |cdocsfn1.tex|, |cdocsfn2.tex|.
Then copy the file |childdoc.def| to an appropriate directory of your \LaTeX{}
distribution, e.g.\ \textit{texmf-root}|/tex/latex/childdoc|.
\end{itemize}

%%%%%%%%%%%%%%%%%%%%%%%%%%%%%%%%%%%%%%%%%%%%%%%%%%%%%%%%%%%%%%%%%%%%%%%%%%%%%%%%
\subsection{Related CTAN Packages}

There are several other packages which offer a similar functionality:
%
\begin{itemize}
\item
The packages
\href{http://ctan.org/pkg/docmute}{\textsf{docmute}},
\href{http://ctan.org/pkg/includex}{\textsf{includex}} and
\href{http://ctan.org/pkg/standalone}{\textsf{standalone}}
provide commands to include only the document body of
a child file thus allowing both files to be compiled individually.
\item
The packages \href{http://ctan.org/pkg/subdocs}{\textsf{subdocs}}
and \href{http://ctan.org/pkg/subfiles}{\textsf{subfiles}}
provide structures in which the main and child documents can be
encapsulated and allowing them to be compiled individually.
The inclusion mechanism is different from the conventional |\include|.
\item
The package \href{http://ctan.org/pkg/combine}{\textsf{combine}}
is an elaborate solution to combine several documents into one.
\end{itemize}
%
See also the CTAN topic \href{http://ctan.org/topic/subdocs}{\textsf{subdocs}}
for further related packages.
The present package differs from the above solutions in that
a document structure constructed with the conventional |\include| mechanism
just needs two extra commands at the top of every file
such that all constituent files can be compiled individually.

%%%%%%%%%%%%%%%%%%%%%%%%%%%%%%%%%%%%%%%%%%%%%%%%%%%%%%%%%%%%%%%%%%%%%%%%%%%%%%%%
%\subsection{Feature Suggestions}
%
%The following is a list of features which may be useful for future
%versions of this package:
%%
%\begin{itemize}
%\item
%\ldots
%\end{itemize}

%%%%%%%%%%%%%%%%%%%%%%%%%%%%%%%%%%%%%%%%%%%%%%%%%%%%%%%%%%%%%%%%%%%%%%%%%%%%%%%%
\subsection{Revision History}

%%%%%%%%%%%%%%%%%%%%%%%%%%%%%%%%%%%%%%%%
\paragraph{v2.0:} 2018/12/30

\begin{itemize}
\item
immediate forward processing
\item
added |\childdocby| mechanism
\item
manual restructured
\end{itemize}

%%%%%%%%%%%%%%%%%%%%%%%%%%%%%%%%%%%%%%%%
\paragraph{v1.6:} 2018/01/17

\begin{itemize}
\item
application for development of include files
\item
corrections to manual
\end{itemize}

%%%%%%%%%%%%%%%%%%%%%%%%%%%%%%%%%%%%%%%%
\paragraph{v1.5:} 2017/05/21

\begin{itemize}
\item
more complete structuring introduced
\item
|\childdocof| introduced
\item
|\childdoc| renamed to |\childdocmain|
\item
|\childredirect| renamed to |\childdocforward| and |\childdocforwardprefix|
and functionality expanded
\end{itemize}

%%%%%%%%%%%%%%%%%%%%%%%%%%%%%%%%%%%%%%%%
\paragraph{v1.0:} 2017/04/27

\begin{itemize}
\item
manual and install package
\item
first version published on CTAN
\end{itemize}

%%%%%%%%%%%%%%%%%%%%%%%%%%%%%%%%%%%%%%%%
\paragraph{v0.6:} 2017/04/26

\begin{itemize}
\item
redirection mechanism added
\end{itemize}

%%%%%%%%%%%%%%%%%%%%%%%%%%%%%%%%%%%%%%%%
\paragraph{v0.5:} 2017/04/26

\begin{itemize}
\item
functionality in definition file
\end{itemize}


%%%%%%%%%%%%%%%%%%%%%%%%%%%%%%%%%%%%%%%%%%%%%%%%%%%%%%%%%%%%%%%%%%%%%%%%%%%%%%%%
%%%%%%%%%%%%%%%%%%%%%%%%%%%%%%%%%%%%%%%%%%%%%%%%%%%%%%%%%%%%%%%%%%%%%%%%%%%%%%%%
%%%%%%%%%%%%%%%%%%%%%%%%%%%%%%%%%%%%%%%%%%%%%%%%%%%%%%%%%%%%%%%%%%%%%%%%%%%%%%%%
\appendix

\settowidth\MacroIndent{\rmfamily\scriptsize 000\ }

 \DocInput{childdoc.dtx}

\end{document}
%</driver>
% \fi
%
% %%%%%%%%%%%%%%%%%%%%%%%%%%%%%%%%%%%%%%%%%%%%%%%%%%%%%%%%%%%%%%%%%%%%%%%%%%%%%%
% %%%%%%%%%%%%%%%%%%%%%%%%%%%%%%%%%%%%%%%%%%%%%%%%%%%%%%%%%%%%%%%%%%%%%%%%%%%%%%
% \section{Sample}
%\iffalse
%<*samplemain>
%\fi
%
% The following presents a sample document
% with two chapters, two parts, a title page,
% a compile flag as well as three forwarding files to set the flag.
% It consists of eight |.tex| files:
% \begin{center}
% \begin{tabular}{ll}
% |cdocsamp.tex|&main file\\
% |cdocsch1.tex|&include file for chapter 1\\
% |cdocsch2.tex|&include file for chapter 2\\
% |cdocspt3.tex|&include file for part 3\\
% |cdocspt4.tex|&include file for part 4\\
% |cdocsdrf.tex|&forwarding file for main file in draft mode\\
% |cdocsfi1.tex|&forwarding file for final version of chapter 1\\
% |cdocsfi2.tex|&forwarding file for final version of chapter 2\\
% \end{tabular}
% \end{center}
% Each of the eight files can be compiled directly by the \LaTeX{} compiler.
%
% %%%%%%%%%%%%%%%%%%%%%%%%%%%%%%%%%%%%%%
% \paragraph{Main File.}
%
% The main file is called |cdocsamp.tex|.
%
% Load the \textsf{childdoc} definitions and
% declare the filename for the main document:
%    \begin{macrocode}
\input{childdoc.def}
\childdocmain{}
%    \end{macrocode}

% Optional override for |\version| flag:
%    \begin{macrocode}
%%\ifchilddoc\else\providecommand{\version}{draft}\fi
%    \end{macrocode}

% Define the default values for the |\version| flag
% (|final| for the main file and |draft| for childs):
%    \begin{macrocode}
\ifchilddoc
\providecommand{\version}{draft}
\else
\providecommand{\version}{final}
\fi
%    \end{macrocode}

% Load the standard document class:
%    \begin{macrocode}
\documentclass[12pt]{article}
%    \end{macrocode}

% Start the document body:
%    \begin{macrocode}
\begin{document}
%    \end{macrocode}

% Declare a title page.
% Print title, part of document being processed and version flag:
%    \begin{macrocode}
\addtocounter{page}{-1}
\begin{center}
{\LARGE\bfseries{}childdoc example\par}
\vspace{1cm}
\ifchilddoc
\ifchilddocmanual part\else chapter\fi:
`\childdocname' of `\childdocjob'\par
\else
main document: `\childdocjob'\par
\fi
version: \version\par
\end{center}
\newpage
%    \end{macrocode}

% Manually include selected file,
% otherwise process as usual:
%    \begin{macrocode}
\ifchilddocmanual
\section*{part `\childdocname'}
\input{\childdocname}
\else
%    \end{macrocode}

% Include the two chapters:
%    \begin{macrocode}
\include{cdocsch1}
\include{cdocsch2}
%    \end{macrocode}

% Include the two parts unless only chapters should be displayed:
%    \begin{macrocode}
\ifchilddoc\else
\section{part three}
\input{cdocspt3}
\section{part four}
\input{cdocspt4}
\fi
%    \end{macrocode}

% Process as usual until here:
%    \begin{macrocode}
\fi
%    \end{macrocode}

% End of document body:
%    \begin{macrocode}
\end{document}
%    \end{macrocode}
%\iffalse
%</samplemain>
%\fi
%
% %%%%%%%%%%%%%%%%%%%%%%%%%%%%%%%%%%%%%%
% \paragraph{Chapter Include Files.}
%
% The include files are called |cdocsch1.tex| and |cdocsch2.tex|.
%
%\iffalse
%<*samplechap1|samplechap2>
%\fi

% Optional override for |\version| flag:
%    \begin{macrocode}
%%\providecommand{\version}{final}
%    \end{macrocode}

% Include the main document:
%    \begin{macrocode}
\input{childdoc.def}
\childdocof{cdocsamp}
%    \end{macrocode}

%\iffalse
%</samplechap1|samplechap2>
%\fi
%
%\iffalse
%<*samplechap1>
%\fi
% Some text for chapter 1:
%    \begin{macrocode}
\section{one}
some text in chapter one
%    \end{macrocode}

%\iffalse
%</samplechap1>
%\fi
% Some text for chapter 2:
%\iffalse
%<*samplechap2>
%\fi
%    \begin{macrocode}
\section{two}
more text in chapter two
%    \end{macrocode}

%\iffalse
%</samplechap2>
%\fi
%
% %%%%%%%%%%%%%%%%%%%%%%%%%%%%%%%%%%%%%%
% \paragraph{Part Include Files.}
%
% The include files are called |cdocspt3.tex| and |cdocspt4.tex|.
%
%\iffalse
%<*samplepart3|samplepart4>
%\fi

% Optional override for |\version| flag:
%    \begin{macrocode}
%%\providecommand{\version}{final}
%    \end{macrocode}

% Include the main document:
%    \begin{macrocode}
\input{childdoc.def}
\childdocby{cdocsamp}
%    \end{macrocode}

%\iffalse
%</samplepart3|samplepart4>
%\fi
%
%\iffalse
%<*samplepart3>
%\fi
% Some text for part 3:
%    \begin{macrocode}
some text in part three
%    \end{macrocode}

%\iffalse
%</samplepart3>
%\fi
% Some text for part 4:
%\iffalse
%<*samplepart4>
%\fi
%    \begin{macrocode}
more text in part four
%    \end{macrocode}

%\iffalse
%</samplepart4>
%\fi
%
% %%%%%%%%%%%%%%%%%%%%%%%%%%%%%%%%%%%%%%
% \paragraph{Forwarding for a Complete Draft.}
%
% The following forwarding file |cdocsdrf.tex|
% compiles the main document in draft mode:
%\iffalse
%<*sampledraft>
%\fi
%    \begin{macrocode}
\def\version{draft}
\input{childdoc.def}
\childdocforward{cdocsamp}
%    \end{macrocode}

%\iffalse
%</sampledraft>
%\fi
%
% %%%%%%%%%%%%%%%%%%%%%%%%%%%%%%%%%%%%%%
% \paragraph{Forwarding for Final Version of the Chapters.}
%
% The following forwarding files |cdocsfn1.tex| and |cdocsfn2.tex|
% (with identical content)
% compile the final versions of the child documents
% |cdocsch1.tex| and |cdocsch2.tex|, respectively:
%\iffalse
%<*samplefinal>
%\fi
%    \begin{macrocode}
\def\version{final}
\input{childdoc.def}
\childdocforwardprefix[cdocsamp]{cdocsfn}{cdocsch}
%    \end{macrocode}

%\iffalse
%</samplefinal>
%\fi
%
% %%%%%%%%%%%%%%%%%%%%%%%%%%%%%%%%%%%%%%
% \paragraph{Command Line Processing.}
%
% The following three command lines generate the output files
% |cdocscld|, |cdocscl1| and |cdocscl2|
% which should be identical to
% |cdocsdrf|, |cdocsch1| and |cdocsfn2|, respectively:
% \begin{center}
% \begin{tabular}{l}
% |latex -jobname cdocscld \|\\
% |  "\def\version{draft}\input{childdoc.def}\childdocforward{cdocsamp}"|\\
% |latex -jobname cdocscl1 \|\\
% |  "\input{childdoc.def}\childdocforward[cdocsamp]{cdocsch1}"|\\
% |latex -jobname cdocscl2 \|\\
% |  "\def\version{final}\input{childdoc.def}\childdocforward{cdocsch2}"|
% \end{tabular}
% \end{center}
% Note that the trailing backslash on each first line
% merely continues the input to the second line
% (for convenient cut ant paste).
% Furthermore, the command |latex| can be replaced by any
% of its alternative versions such as |pdflatex|.
%
% %%%%%%%%%%%%%%%%%%%%%%%%%%%%%%%%%%%%%%%%%%%%%%%%%%%%%%%%%%%%%%%%%%%%%%%%%%%%%%
% %%%%%%%%%%%%%%%%%%%%%%%%%%%%%%%%%%%%%%%%%%%%%%%%%%%%%%%%%%%%%%%%%%%%%%%%%%%%%%
% \section{Implementation}
%\iffalse
%<*package>
%\fi
%
% This section describes the definitions file |childdoc.def|.

% The definitions cannot be loaded using |\usepackage| or |\RequirePackage|
% which has a mechanism to prevent loading a style file more than once.
% When loading the definitions by means of |\input|
% multiple instances have to be prevented manually:
%\iffalse
%This code needs to be before the `\ProvidesFile' directive
%which is defined at the beginning of this file.
%Therefore it is also placed there and commented out here.
%</package>
%<*discard>
%\fi
%    \begin{macrocode}
\ifdefined\childdocmain\endinput\fi
%    \end{macrocode}
%\iffalse
%</discard>
%<*package>
%\fi
%
% \macro{\ifchilddoc}
% \macro{\ifchilddocmanual}
% The conditional |\ifchilddoc| tells whether a
% child (true) or main (false) document is being compiled.
% The conditional |\ifchilddocmanual| tells whether
% the |\includeonly| mechanism is used (false) or
% the selection of child files must be performed manually (true).
% The definitions initialise to false:
%    \begin{macrocode}
\newif\ifchilddoc
\newif\ifchilddocmanual
%    \end{macrocode}

% \macro{\childdocname}
% \macro{\childdocjob}
% The macro |\childdocname| stores the name of the main document
% to be compiled. The macro |\childdocjob| stores the name of
% the document on which the \LaTeX{} compiler was originally invoked.
% The content of |\jobname| cannot be compared
% to filenames specified in the source due to different catcodes.
% The following code rescans |\jobname|, stores the result
% in |\childdocname| and saves a copy in |\childdocjob|:
%    \begin{macrocode}
\edef\childdocname{\scantokens\expandafter{\jobname\noexpand}}
\let\childdocjob\childdocname
%    \end{macrocode}

% \macro{\childdocdisable}
% The macro |\childdocdisable| prevents the main file
% from being processed more than once.
% At this stage, the main document command |\childdocmain|
% is assumed to be called once again where it should do nothing.
% Any subsequent call to it should prevent
% a secondary processing of the main document
% It overwrites the forwarding commands
% |\childdocof| and |\childdocforward|
% with empty macros to prevent further inclusions of the main document:
%    \begin{macrocode}
\newcommand{\childdocdisable}
{
  \renewcommand{\childdocmain}[1]{\renewcommand{\childdocmain}[1]{\endinput}}
  \renewcommand{\childdocof}[1]{}
  \renewcommand{\childdocby}[2][]{}
  \renewcommand{\childdocforward}[2][]{}
  \renewcommand{\childdocdisable}{}
}
%    \end{macrocode}

% \macro{\childdocmain}
% The macro |\childdocmain| is to be called at the top of the main file
% with nothing or the main filename (without extension) as argument.
% First, it breaks loops.
% If the argument is not empty and does not match |\childdocname|
% (which is set by the first inclusion of |childdoc.def|),
% |\ifchilddoc| is set to true, |\includeonly| is applied to the child file
% and |\jobname| is set to the main file
% (for proper handling of |.aux| files):
%    \begin{macrocode}
\newcommand{\childdocmain}[1]
{
  \childdocdisable\childdocmain{}
  \if?#1?\else
    \begingroup
      \def\childdoctmp{#1}
      \ifx\childdoctmp\childdocname
        \def\childdoctmp{}
      \else
        \def\childdoctmp
        {
          \childdoctrue
          \includeonly{\childdocname}
          \def\childdocjob{#1}
          \def\jobname{#1}
        }
      \fi
      \expandafter
    \endgroup
    \childdoctmp
  \fi
}
%    \end{macrocode}

% \macro{\childdocof}
% The command |\childdocof| redirects
% compilation to the main file |#1|.
%    \begin{macrocode}
\newcommand{\childdocof}[1]
{
  \childdocdisable
  \childdoctrue
  \includeonly{\childdocname}
  \def\jobname{#1}
  \def\childdocjob{#1}
  \input{#1}
}
%    \end{macrocode}

% \macro{\childdocby}
% The command |\childdocby| ....
%    \begin{macrocode}
\newcommand{\childdocby}[2][]
{
  \childdocdisable
  \childdoctrue
  \childdocmanualtrue
  \if?#1?\else
    \def\jobname{#2}
  \fi
  \def\childdocjob{#2}
  \input{#2}
  \endinput
}
%    \end{macrocode}

% \macro{\childdocforward}
% The command |\childdocforward| redirects
% compilation to the main file or
% (if the optional argument is given) a child file.
% Parameters are set as if the main file
% or a child file starting with |\childdocof| was compiled.
% Then compilation is handed over to the main file:
%    \begin{macrocode}
\newcommand{\childdocforward}[2][]
{
  \begingroup
    \if?#1?
      \def\childdoctmp
      {
        \def\childdocname{#2}
        \def\childdocjob{#2}
        \def\jobname{#2}
        \input{#2}
        \endinput
      }
    \else
      \def\childdoctmp
      {
        \childdocdisable
        \def\childdocname{#2}
        \childdoctrue
        \includeonly{#2}
        \def\childdocjob{#1}
        \def\jobname{#1}
        \input{#1}
        \endinput
      }
    \fi
    \expandafter
  \endgroup
  \childdoctmp
}
%    \end{macrocode}

% \macro{\childdocforwardprefix}
% The command |\childdocforwardprefix| redirects
% compilation to the main or a child file by means of a pattern.
% The prefix |#1| in the current filename is replaced by |#2|
% and the suffix of the current filename is kept
% (it is assumed that the filename does not contain the substring `|~~~|'
% which is used as a delimiter).
% Compilation is handed over to the new file by |\childdocforward|:
%    \begin{macrocode}
\newcommand{\childdocforwardprefix}[3][]
{
  \begingroup
    \def\childdocextract #2##1~~~{\def\childdoctmp{\childdocforward[#1]{#3##1}}}
    \expandafter\childdocextract\childdocname~~~
    \expandafter
  \endgroup
  \childdoctmp
}
%    \end{macrocode}

% \macro{\childdoc}
% The deprecated macro |\childdoc| is a legacy version of |\childdocmain|:
%    \begin{macrocode}
\newcommand{\childdoc}{\childdocmain}
%    \end{macrocode}

% \macro{\childdocredirect}
% The deprecated macro |\childdocredirect| is a legacy version
% of |\childdocforward| and |\childdocforwardprefix|:
%    \begin{macrocode}
\newcommand{\childdocredirect}[2][]
{
  \begingroup
    \if?#1?
      \def\childdoctmp{\childdocforward{#2}}
    \else
      \def\childdoctmp{\childdocforwardprefix{#1}{#2}}
    \fi
    \expandafter
  \endgroup
  \childdoctmp
}
%    \end{macrocode}

%\iffalse
%</package>
%\fi
%
\endinput
\childdocforward[cdocsamp]{cdocsch1}"|\\
% |latex -jobname cdocscl2 \|\\
% |  "\def\version{final}% \iffalse
%
% childdoc.dtx Copyright (C) 2017-2018 Niklas Beisert
%
% This work may be distributed and/or modified under the
% conditions of the LaTeX Project Public License, either version 1.3
% of this license or (at your option) any later version.
% The latest version of this license is in
%   http://www.latex-project.org/lppl.txt
% and version 1.3 or later is part of all distributions of LaTeX
% version 2005/12/01 or later.
%
% This work has the LPPL maintenance status `maintained'.
%
% The Current Maintainer of this work is Niklas Beisert.
%
% This work consists of the files childdoc.dtx and childdoc.ins
% and the derived files childdoc.def and cdocsamp.tex with
% cdocsch1.tex, cdocsch2.tex, cdocsdrf.tex, cdocsfn1.tex, cdocsfn2.tex.
%
%<package>\ifdefined\childdocmain\endinput\fi
%<package>\ProvidesFile{childdoc.def}[2018/12/30 v2.0 child document driver]
%<samplemain>\ProvidesFile{cdocsamp.tex}[2018/12/30 v2.0 sample for childdoc]
%<*driver>
%\ProvidesFile{childdoc.drv}[2018/12/30 v2.0 childdoc reference manual file]
\PassOptionsToClass{10pt,a4paper}{article}
\documentclass{ltxdoc}

\usepackage[margin=35mm]{geometry}
\usepackage{hyperref}
\usepackage{hyperxmp}
\usepackage[usenames]{color}

\hypersetup{colorlinks=true}
\hypersetup{pdfstartview=FitH}
\hypersetup{pdfpagemode=UseNone}
\hypersetup{pdfsource={}}
\hypersetup{pdflang={en-UK}}
\hypersetup{pdfcopyright={Copyright 2017-2018 Niklas Beisert.
  This work may be distributed and/or modified under the
  conditions of the LaTeX Project Public License, either version 1.3
  of this license or (at your option) any later version.}}
\hypersetup{pdflicenseurl={http://www.latex-project.org/lppl.txt}}
\hypersetup{pdfcontactaddress={ETH Zurich, ITP, HIT K,
  Wolfgang-Pauli-Strasse 27}}
\hypersetup{pdfcontactpostcode={8093}}
\hypersetup{pdfcontactcity={Zurich}}
\hypersetup{pdfcontactcountry={Switzerland}}
\hypersetup{pdfcontactemail={nbeisert@itp.phys.ethz.ch}}
\hypersetup{pdfcontacturl={http://people.phys.ethz.ch/\xmptilde nbeisert/}}

\newcommand{\secref}[1]{\hyperref[#1]{section \ref*{#1}}}

\parskip1ex
\parindent0pt
\let\olditemize\itemize
\def\itemize{\olditemize\parskip0pt}

\begin{document}

\title{The \textsf{childdoc} Package}
\hypersetup{pdftitle={The childdoc Package}}
\author{Niklas Beisert\\[2ex]
  Institut f\"ur Theoretische Physik\\
  Eidgen\"ossische Technische Hochschule Z\"urich\\
  Wolfgang-Pauli-Strasse 27, 8093 Z\"urich, Switzerland\\[1ex]
  \href{mailto:nbeisert@itp.phys.ethz.ch}
  {\texttt{nbeisert@itp.phys.ethz.ch}}}
\hypersetup{pdfauthor={Niklas Beisert}}
\hypersetup{pdfsubject={Manual for the LaTeX2e Package childdoc}}
\date{30 December 2018, \textsf{v2.0}}
\maketitle

\begin{abstract}\noindent
\textsf{childdoc} is a \LaTeXe{} package
that enables the direct compilation
of document sections included by |\include|
to individual files.
\end{abstract}

\begingroup
\parskip0ex
\tableofcontents
\endgroup

%%%%%%%%%%%%%%%%%%%%%%%%%%%%%%%%%%%%%%%%%%%%%%%%%%%%%%%%%%%%%%%%%%%%%%%%%%%%%%%%
%%%%%%%%%%%%%%%%%%%%%%%%%%%%%%%%%%%%%%%%%%%%%%%%%%%%%%%%%%%%%%%%%%%%%%%%%%%%%%%%
\section{Introduction}

\LaTeX{} provides a mechanism to structure a large document (such as a book)
into a main file and several child files (containing the chapters)
using the |\include| command.
This mechanism is beneficial for documents
which span hundreds of pages in order to
make the source file(s) more manageable.
Moreover, compilation can be restricted to
selected child files by means of the |\includeonly| command.
The latter feature can be used to reduce the compilation time while editing
(this was significantly more useful in the earlier days of \LaTeX{})
or to generate a smaller document which is easier to navigate.
Another application of |\includeonly| is to generate
documents consisting of selected parts of the complete document.

However, there are a few drawbacks of the plain |\include| mechanism:
\begin{itemize}
\item
The child files cannot be compiled on their own,
they can only be compiled via the main file.
A naive editing environment
(such as a text editor with an option
to have the current file processed by \LaTeX)
may require one to switch to the main file before compiling;
attempting to compile the child file produces errors.
\item
The main file must be modified (each time)
to adjust the |\includeonly| command
to the present needs. This easily leaves the main file in a messy state.
\item
The generated document will always carry the filename
of the main document. This is inconvenient if
several child files are to be compiled and
to be kept for distribution.
\end{itemize}

The present package provides a simple interface
to make child files individually compilable by \LaTeX{}.
Compiling a child file then has the same effect as compiling
the main file with an |\includeonly| command
to select the appropriate child.
Moreover the generated document will carry the name of the child
rather than the main file.
This resolves all three above issues.

This feature is meant to make the editing of books,
thesis documents and lecture notes somewhat more convenient.
However, the package can also be used efficiently for
composing a series of documents (such as exercise sheets)
which are typically distributed individually.
It then assists the author in generating the individual documents
(potentially in different versions)
as well as a document containing the collected series.
Another application is in developing style files
or other kinds of included material
where compilation of the style file could redirect
to a sample or test file.

%%%%%%%%%%%%%%%%%%%%%%%%%%%%%%%%%%%%%%%%%%%%%%%%%%%%%%%%%%%%%%%%%%%%%%%%%%%%%%%%
%%%%%%%%%%%%%%%%%%%%%%%%%%%%%%%%%%%%%%%%%%%%%%%%%%%%%%%%%%%%%%%%%%%%%%%%%%%%%%%%
\section{Usage}

First of all, the package \textsf{childdoc} is \emph{not} a standard
\LaTeXe{} |.sty| style file! Therefore it needs to be invoked in
a non-standard way.

%%%%%%%%%%%%%%%%%%%%%%%%%%%%%%%%%%%%%%%%%%%%%%%%%%%%%%%%%%%%%%%%%%%%%%%%%%%%%%%%
\subsection{Included Files}
\label{sec:include}

%%%%%%%%%%%%%%%%%%%%%%%%%%%%%%%%%%%%%%%%
\DescribeMacro{\childdocmain}
To use the package, add the commands
\begin{center}
\begin{tabular}{l}
|\input{childdoc.def}|\\
|\childdocmain{}|\\
\end{tabular}
\end{center}
at the very top of the main \LaTeX{} file,
in particular \emph{before} the |\documentclass| statement!
The argument of |\childdocmain| should be left empty
(but it must be present).

%%%%%%%%%%%%%%%%%%%%%%%%%%%%%%%%%%%%%%%%
\DescribeMacro{\childdocof}
Furthermore, add the commands
\begin{center}
\begin{tabular}{l}
|\input{childdoc.def}|\\
|\childdocof{|\textit{main}|}|\\
\end{tabular}
\end{center}
at the top of every child file \textit{child}
which is included by |\include{|\textit{child}|}|
from within the main file
(or at least for those files to be compiled individually).
The argument \textit{main} must be the filename of the main file.

There are a couple of
considerations in setting up the main and child documents:

%%%%%%%%%%%%%%%%%%%%%%%%%%%%%%%%%%%%%%%%
\paragraph{Restrictions.}

Please note the following restrictions:
\begin{itemize}
\item
|\childdocmain| must be called with one argument \textit{main}
to ensure compatibility with earlier version of the package.
It must either be empty (|\childdocmain{}|)
or precisely match the filename of the main file in which it is specified.
See \secref{sec:detection} for further information.
\item
The filename \textit{main} must be specified without the |.tex| extension.
\item
The filename \textit{main} is case sensitive
(even in case-insensitive file systems)
due to internal string comparison.
\item
The argument \textit{main} should be fully expanded, it cannot be a macro.
\item
Subdirectories and special characters should be avoided in filenames.
\item
The command |\childdocmain{|\textit{main}|}| must be followed by a whitespace.
It should not be followed immediately by another command
or by a comment mark `|%|'.
This is because the \TeX{} parser reads the token immediately following
the argument of |\childdocmain| and puts it
at the beginning of every child section;
however, a white\-space is ignored.
\end{itemize}

%%%%%%%%%%%%%%%%%%%%%%%%%%%%%%%%%%%%%%%%
\paragraph{Content of Main File.}

It is advisable to place all content in the child files included by |\include|.
Any output contained in the main file will appear in all child documents
unless suppressed manually;
it cannot be suppressed automatically by the |\includeonly| directive
and thus should normally be avoided.
A method to include some content in the main file
by means of conditional processing is described in \secref{sec:conditional}.

%%%%%%%%%%%%%%%%%%%%%%%%%%%%%%%%%%%%%%%%
\paragraph{Page Numbering.}

When only a part of the document is compiled,
the appropriate numbering of pages
(as well as other status parameters)
is determined from the |.aux| files.
The latter contain information from previous passes.
However this information needs to propagate through
all intermediate child documents.
Therefore the page numbering in child documents may well
be inconsistent until the complete document is compiled at least once.

A useful (if unconventional) way to always ensure a consistent
page numbering is to restart the numbering in each child document
and denote the pages by `\textit{child}|.|\textit{page}'
where \textit{child} represents the chapter/section number of the child file.
This can be achieved by the command
|\numberwithin{page}{|\textit{child}|}|
of the \textsf{amsmath} package
where \textit{child} can be |chapter| or |section|
depending on the chosen structuring.
Alternatively, one can modify the macro |\thepage| appropriately
and reset the counter |page| at the start of each child file.

%%%%%%%%%%%%%%%%%%%%%%%%%%%%%%%%%%%%%%%%%%%%%%%%%%%%%%%%%%%%%%%%%%%%%%%%%%%%%%%%
\subsection{Conditional Processing}
\label{sec:conditional}

The package provides a mechanism to compile different versions
of a document. To customise the versions further some conditional processing
can come in handy to distinguish which version is being compiled.
The package provides two macros to describe the compilation context:

%%%%%%%%%%%%%%%%%%%%%%%%%%%%%%%%%%%%%%%%
\DescribeMacro{\ifchilddoc}
The conditional |\ifchilddoc| distinguishes between the compilation of
child documents and the main document:
%
\begin{center}
|\ifchilddoc |\textit{child-code}| |[|\||else |\textit{main-code}]| \||fi|
\end{center}

%%%%%%%%%%%%%%%%%%%%%%%%%%%%%%%%%%%%%%%%
\DescribeMacro{\childdocname}
\DescribeMacro{\childdocjob}
The macro |\childdocname| contains the filename (without extension)
of the main or child file being processed.
Note that |\childdocjob| will always contain the name of the main file.

%%%%%%%%%%%%%%%%%%%%%%%%%%%%%%%%%%%%%%%%
\paragraph{Title Page.}

Conditional processing can be used to include a title or banner page
in the main document when proper precautions are taken.
Importantly, the code in the main file should ensure that the page counter
(as well as other status parameters which are stored in the |.aux| files)
takes the same value after the conditional processing.
Otherwise the page numbers may take divergent values
depending on which part is compiled.

For example, a title page could be declared by:
%
\begin{center}
\begin{tabular}{l}
|\ifchilddoc\||else|\\
|\addtocounter{page}{-1}|\\
\textit{code for title page}\\
|\newpage|\\
|\||fi|
\end{tabular}
\end{center}
%
A banner page for the child documents can be generated by:
%
\begin{center}
\begin{tabular}{l}
|\ifchilddoc|\\
|\addtocounter{page}{-1}|\\
\textit{code for banner page}\\
|\newpage|\\
|\||fi|
\end{tabular}
\end{center}
%
Here one could write a message such as:
\begin{center}
|This is the part \childdocname{} of \childdocjob{}.|
\end{center}

%%%%%%%%%%%%%%%%%%%%%%%%%%%%%%%%%%%%%%%%%%%%%%%%%%%%%%%%%%%%%%%%%%%%%%%%%%%%%%%%
\subsection{Flags}
\label{sec:flags}

The package makes it easy to generate different versions
of the main or child documents.
To this end compilation flags can be defined
and assigned different default values.
They will be particularly useful in conjunction
with the forwarding mechanism described in \secref{sec:forward}.

For example, it may be useful to have a flag |\version|
which can be set to |draft| or |final|.
The document source will contain some conditional code
depending on the value of |\version|.
Suppose further, the flag should default to |final| for the main file
and to |draft| for child files
which is a natural assignment for editing the document.
This is achieved by placing the following code
in the preamble of the main document
(below the |\childdocmain| directive):
%
\begin{center}
\begin{tabular}{l}
|\ifchilddoc|\\
|\providecommand{\version}{draft}|\\
|\||else|\\
|\providecommand{\version}{final}|\\
|\||fi|
\end{tabular}
\end{center}
%
The definition by |\providecommand| makes sure
that previous definitions are not overwritten.
Further statements |\providecommand{\version}{...}|
can thus be added before the above code to override it.

For the main file, one might add a line
(between |\childdocmain| and the above block)
%
\begin{center}
|%\ifchilddoc\||else\providecommand{\version}{draft}\||fi|
\end{center}
%
which can be uncommented to produce a draft version.
Likewise one can add a line to the very top of a child file
(above the |\childdocof{|\textit{main}|}| directive)
%
\begin{center}
|%\providecommand{\version}{final}|
\end{center}
%
which can be uncommented to produce the final version of this child document.

%%%%%%%%%%%%%%%%%%%%%%%%%%%%%%%%%%%%%%%%%%%%%%%%%%%%%%%%%%%%%%%%%%%%%%%%%%%%%%%%
\subsection{Forwarding}
\label{sec:forward}

Different versions of the main or child documents
using compilation flags as described in \secref{sec:flags}
can be (permanently) stored in different files
for convenient compilation, viewing and distribution.
To this end, the package defines a command
to pass on compilation to a different file:

%%%%%%%%%%%%%%%%%%%%%%%%%%%%%%%%%%%%%%%%
\DescribeMacro{\childdocforward}
The command |\childdocforward| redirects processing to
another source file:
%
\begin{center}
\begin{tabular}{l}
|\input{childdoc.def}|\\
|\childdocforward[|\textit{main}|]{|\textit{dest}|}|\\
\end{tabular}
\end{center}
%
The argument \textit{dest} is the destination file
(without extension).
It should be the main file or one of the child files.
Note that further \textsf{childdoc} directives
such as |\childdocof| and |\childdocforward|
in the indicated file will be processed in this form.
The optional argument \textit{main}
passes on directly to the main file \textit{main}
while pretending to compile the child \textit{dest}.
This form behaves as if \textit{dest}
issues |\childdocof{|\textit{main}|}| right away,
and no further \textsf{childdoc} directives will be processed.

%%%%%%%%%%%%%%%%%%%%%%%%%%%%%%%%%%%%%%%%
\DescribeMacro{\...prefix}
In the alternative form |\childdocforwardprefix|,
%
\begin{center}
\begin{tabular}{l}
|\input{childdoc.def}|\\
|\childdocforwardprefix[|\textit{main}|]{|\textit{prefix}|}{|\textit{dest}|}|
\end{tabular}
\end{center}
%
the destination file is determined by a pattern
depending on the current file:
To make this work, the current file must be called
`{\textit{prefix}\hspace{0.2em}\textit{suffix}}'
with \textit{prefix} matching precisely the argument.
Processing is then passed on to the file
`{\textit{dest}\hspace{0.2em}\textit{suffix}}'.
Surely, the same effect is achieved by
directly specifying the
argument `{\textit{dest}\hspace{0.2em}\textit{suffix}}'
in the first form.
However, that requires to set up a different file
for each child. With the alternative form of the command
all these files can have exactly the same content
which simplifies setting them up and maintaining them.

For example, the following file |draft.tex|
with a compilation flag |\version| as described in \secref{sec:flags}
compiles the main document as a draft:
%
\begin{center}
\begin{tabular}{l}
|\def\version{draft}|\\
|\input{childdoc.def}|\\
|\childdocforward{|\textit{main}|}|
\end{tabular}
\end{center}
%
Likewise, the following files |final|\textit{nn}|.tex|
compile the final version of the child document
|child|\textit{nn}|.tex|:
%
\begin{center}
\begin{tabular}{l}
|\def\version{final}|\\
|\input{childdoc.def}|\\
|\childdocforwardprefix{final}{child}|
\end{tabular}
\end{center}
%

Note that when several versions of a main file and/or of each child file
are to be generated, it may be convenient to set up a |Makefile| or
shell script to automatise the process.

%%%%%%%%%%%%%%%%%%%%%%%%%%%%%%%%%%%%%%%%%%%%%%%%%%%%%%%%%%%%%%%%%%%%%%%%%%%%%%%%
\subsection{Command Line Processing}
\label{sec:commandline}

The effect of redirection files can also be achieved by invoking
the \LaTeX{} compiler with a more elaborate command line.
Most conveniently this should be done as part
of a shell script or a |Makefile|.

When using \textsf{childdoc} in the main file, the following
command lines effectively perform a redirection
(note that depending on the shell being used,
backslashes may have to be doubled: `|\|' $\to$ `|\\|'):
%
\begin{center}
|... -jobname "|\textit{target}|" |\\|"|[\textit{flags}]%
|\input{childdoc.def}\childdocforward[|\textit{main}|]{|\textit{dest}|}"|
\end{center}
%
Here \textit{target} is the name of the output file,
\textit{main} is the name of the main file
and \textit{dest} is the name of the main or child file to be processed
(all filenames without extensions).
The optional argument \textit{main} can be omitted
if \textit{main} matches \textit{dest}.
Optionally, compilation \textit{flags} can be defined via |\def| commands.
This command line makes the \TeX{} engine believe
it is compiling the file \textit{target}
whose content is specified as the latter parameter.
The provided code then forwards the processing to
\textit{main} or \textit{dest} as described in \secref{sec:forward}.

%%%%%%%%%%%%%%%%%%%%%%%%%%%%%%%%%%%%%%%%%%%%%%%%%%%%%%%%%%%%%%%%%%%%%%%%%%%%%%%%
\subsection{Include by Input}
\label{sec:input}

Including child documents by |\include| has some restrictions by design.
Most notably, the content of a child document always occupies
its own set of pages; pages cannot be shared between child documents.
Usually, this behaviour makes perfect sense
because each child document contain an essential part of the document.
However, in some situations it may be desirable to compose
a document from a collection of parts
without having mandatory page breaks between then.
For this case, the package
provides a mechanism to include parts
by |\input| which can also be processed individually.
However, by construction this mechanism
requires manual handling of the content to be output.

%%%%%%%%%%%%%%%%%%%%%%%%%%%%%%%%%%%%%%%%
\DescribeMacro{\ifchilddocmanual}
The main file should be prepared as usual, see \secref{sec:include}.
However, the document body must make a distinction
between processing of an individual part and of the main document, e.g.:
%
\begin{center}
\begin{tabular}{l}
|\ifchilddocmanual|\\
|\input{\childdocname}|\\
|\||else|\\
\textit{document body with }|\input{|\textit{part}|}|\\
|\||fi|
\end{tabular}
\end{center}
%
The conditional |\ifchilddocmanual| is true whenever
a part to be included by |\input| is being compiled,
and the name of the part is stored in |\childdocname|.

%%%%%%%%%%%%%%%%%%%%%%%%%%%%%%%%%%%%%%%%
\DescribeMacro{\childdocby}
Each part to be included by |\input| should start with:
%
\begin{center}
\begin{tabular}{l}
|\input{childdoc.def}|\\
|\childdocby{|\textit{main}|}|\\
\end{tabular}
\end{center}
%
The directive |\childdocby| is similar to |\childdocof|
described in \secref{sec:include},
but the subsequent selection of content must be done manually.
To that end, both |\ifchilddoc| and |\ifchilddocmanual|
will be true upon processing of a part,
and the name of the part is stored in |\childdocname|.
Note that |\jobname| will be set to the filename of the current part
so that each part receives an individual |.aux| file
that does not interfere with the |.aux| file(s) of the main document.
This behaviour can be altered by the alternative form
|\childdocby[*]{|\textit{main}|}| (with a non-empty optional argument)
which uses the |.aux| file of the main document
by setting |\jobname| to \textit{main}.

%%%%%%%%%%%%%%%%%%%%%%%%%%%%%%%%%%%%%%%%%%%%%%%%%%%%%%%%%%%%%%%%%%%%%%%%%%%%%%%%
\subsection{Driver Development}
\label{sec:driver}

The \textsf{childdoc} mechanism can also be use for the development
of definition files such as \LaTeX{} styles or classes.
This case differs from the above setup with multiple parts
included by |\include| in that no |\includeonly| should be invoked.
This can be achieved by starting the include file
(before |\ProvidesPackage|) with:
%
\begin{center}
\begin{tabular}{l}
|\input{childdoc.def}|\\
|\childdocforward{|\textit{main}|}|\\
\end{tabular}
\end{center}
%
or alternatively with:
%
\begin{center}
\begin{tabular}{l}
|\input{childdoc.def}|\\
|\childdocby{|\textit{main}|}|\\
\end{tabular}
\end{center}
%
Both forms have slightly different effects as described above.
The main file is prepared as usual, see \secref{sec:include}.

%%%%%%%%%%%%%%%%%%%%%%%%%%%%%%%%%%%%%%%%%%%%%%%%%%%%%%%%%%%%%%%%%%%%%%%%%%%%%%%%
\subsection{Legacy Detection}
\label{sec:detection}

The directive |\childdocmain| in the main file can detect
whether the complete document or merely a child is to be compiled
even without using the directive |\childdocof|.
This method is deprecated because it is less robust
and there is no compelling reason to use it;
it is merely provided for backward compatibility
and it may be removed in future versions.

If the detection mechanism is to be used,
it is mandatory to correctly specify
the filename of the main file as the argument of |\childdocmain|:
%
\begin{center}
\begin{tabular}{l}
|\input{childdoc.def}|\\
|\childdocmain{|\textit{main}|}|\\
\end{tabular}
\end{center}
%
If |\jobname| does not match the argument \textit{main} of |\childdocmain|,
it is assumed that |\jobname| points to the child file to be compiled.
When using |\childdocmain| with the main file specified as argument,
it suffices to start a child file
with just |\input{|\textit{main}|}|
without loading of the package and using |\childdocof|.
If instead all processing is done
with the appropriate \textsf{childdoc} directives,
the argument of \textit{main} of |\childdocmain| can be empty.

An alternative version of the command line processing described
in \secref{sec:commandline} using the detection mechanism reads:
%
\begin{center}
|... -jobname "|\textit{target}|" "|[\textit{flags}]%
[|\def\jobname{|\textit{dest}|}|]|\input{|\textit{main}|}"|
\end{center}

%%%%%%%%%%%%%%%%%%%%%%%%%%%%%%%%%%%%%%%%%%%%%%%%%%%%%%%%%%%%%%%%%%%%%%%%%%%%%%%%
\subsection{Manual Code}
\label{sec:manual}

In case one cannot be certain whether the definitions file |childdoc.def|
is installed on the target \TeX{} distribution
and one prefers not to ship it,
it is conceivable to paste a few relevant commands into the sources.

To that end, drop all statements |\input{childdoc.def}|
and perform the replacements as outlined below.
Instead of |\childdocmain{|\textit{main}|}| add the following code
to the top of the main file:
%
\begin{center}
\begin{tabular}{l}
|\||ifdefined\childdocname\endinput\||fi\newif\ifchilddoc|\\
|\edef\childdocname{\scantokens\expandafter{\jobname\noexpand}}|\\
|\def\childdocmain{|\textit{main}|}\||ifx\childdocmain\childdocname\||else|\\
|\childdoctrue\includeonly{\childdocname}\let\jobname\childdocmain\||fi|\\
\end{tabular}
\end{center}
%
Instead of |\childdocof{|\textit{main}|}| just include the main file
at the top of each child file:
%
\begin{center}
|\input{|\textit{main}|}|
\end{center}
%
A simple redirection |\childdocforward{|\textit{dest}|}| is achieved by:
%
\begin{center}
|\def\jobname{|\textit{dest}|}\input{\jobname}|
\end{center}
%
The redirection with prefix
|\childdocforwardprefix[|\textit{prefix}|]{|\textit{dest}|}|
is accomplished by:
%
\begin{center}
\begin{tabular}{l}
|{\edef\jobname{\scantokens\expandafter{\jobname\noexpand}}|\\
|\def\redirectjob |\textit{prefix}|#1~~~{\gdef\jobname{|\textit{dest}|#1}}|\\
|\expandafter\redirectjob\jobname~~~}\input{\jobname}|
\end{tabular}
\end{center}

In an alternative approach,
child documents can be compiled by a specific command line
without additional code or specific definitions:
%
\begin{center}
|... -jobname "|\textit{target}|" "|[\textit{flags}]%
|\includeonly{|\textit{dest}|}\input{|\textit{main}|}"|
\end{center}
%

%%%%%%%%%%%%%%%%%%%%%%%%%%%%%%%%%%%%%%%%%%%%%%%%%%%%%%%%%%%%%%%%%%%%%%%%%%%%%%%%
%%%%%%%%%%%%%%%%%%%%%%%%%%%%%%%%%%%%%%%%%%%%%%%%%%%%%%%%%%%%%%%%%%%%%%%%%%%%%%%%
\section{Information}

%%%%%%%%%%%%%%%%%%%%%%%%%%%%%%%%%%%%%%%%%%%%%%%%%%%%%%%%%%%%%%%%%%%%%%%%%%%%%%%%
\subsection{Copyright}

Copyright \copyright{} 2017--2018 Niklas Beisert

This work may be distributed and/or modified under the
conditions of the \LaTeX{} Project Public License, either version 1.3
of this license or (at your option) any later version.
The latest version of this license is in
  \url{http://www.latex-project.org/lppl.txt}
and version 1.3 or later is part of all distributions of \LaTeX{}
version 2005/12/01 or later.

This work has the LPPL maintenance status `maintained'.

The Current Maintainer of this work is Niklas Beisert.

This work consists of the files |README.txt|, |childdoc.ins| and |childdoc.dtx|
as well as the derived files |childdoc.def|, |cdocsamp.tex|
with |cdocsch1.tex|, |cdocsch2.tex|, |cdocspt3.tex|, |cdocspt4.tex|,
|cdocsdrf.tex|, |cdocsfn1.tex|, |cdocsfn2.tex|
as well as |childdoc.pdf|.

%%%%%%%%%%%%%%%%%%%%%%%%%%%%%%%%%%%%%%%%%%%%%%%%%%%%%%%%%%%%%%%%%%%%%%%%%%%%%%%%
\subsection{Files and Installation}

The package consists of the files:
%
\begin{center}
\begin{tabular}{ll}
    |README.txt|   & readme file \\
    |childdoc.ins| & installation file \\
    |childdoc.dtx| & source file \\
    |childdoc.def| & definition file \\
    |cdocsamp.tex| & sample main file \\
    |cdocsch1.tex| & sample include file \\
    |cdocsch2.tex| & sample include file \\
    |cdocspt3.tex| & sample part file \\
    |cdocspt4.tex| & sample part file \\
    |cdocsdrf.tex| & sample redirection file \\
    |cdocsfn1.tex| & sample redirection file \\
    |cdocsfn2.tex| & sample redirection file \\
    |childdoc.pdf| & manual
\end{tabular}
\end{center}
%
The distribution consists of the files
|README.txt|, |childdoc.ins| and |childdoc.dtx|.
%
\begin{itemize}
\item
Run (pdf)\LaTeX{} on |childdoc.dtx|
to compile the manual |childdoc.pdf| (this file).
\item
Run \LaTeX{} on |childdoc.ins| to create the definitions file |childdoc.def|
and the sample |cdocsamp.tex| with include files
|cdocsch1.tex|, |cdocsch2.tex|, |cdocspt3.tex|, |cdocspt4.tex|,
|cdocsdrf.tex|, |cdocsfn1.tex|, |cdocsfn2.tex|.
Then copy the file |childdoc.def| to an appropriate directory of your \LaTeX{}
distribution, e.g.\ \textit{texmf-root}|/tex/latex/childdoc|.
\end{itemize}

%%%%%%%%%%%%%%%%%%%%%%%%%%%%%%%%%%%%%%%%%%%%%%%%%%%%%%%%%%%%%%%%%%%%%%%%%%%%%%%%
\subsection{Related CTAN Packages}

There are several other packages which offer a similar functionality:
%
\begin{itemize}
\item
The packages
\href{http://ctan.org/pkg/docmute}{\textsf{docmute}},
\href{http://ctan.org/pkg/includex}{\textsf{includex}} and
\href{http://ctan.org/pkg/standalone}{\textsf{standalone}}
provide commands to include only the document body of
a child file thus allowing both files to be compiled individually.
\item
The packages \href{http://ctan.org/pkg/subdocs}{\textsf{subdocs}}
and \href{http://ctan.org/pkg/subfiles}{\textsf{subfiles}}
provide structures in which the main and child documents can be
encapsulated and allowing them to be compiled individually.
The inclusion mechanism is different from the conventional |\include|.
\item
The package \href{http://ctan.org/pkg/combine}{\textsf{combine}}
is an elaborate solution to combine several documents into one.
\end{itemize}
%
See also the CTAN topic \href{http://ctan.org/topic/subdocs}{\textsf{subdocs}}
for further related packages.
The present package differs from the above solutions in that
a document structure constructed with the conventional |\include| mechanism
just needs two extra commands at the top of every file
such that all constituent files can be compiled individually.

%%%%%%%%%%%%%%%%%%%%%%%%%%%%%%%%%%%%%%%%%%%%%%%%%%%%%%%%%%%%%%%%%%%%%%%%%%%%%%%%
%\subsection{Feature Suggestions}
%
%The following is a list of features which may be useful for future
%versions of this package:
%%
%\begin{itemize}
%\item
%\ldots
%\end{itemize}

%%%%%%%%%%%%%%%%%%%%%%%%%%%%%%%%%%%%%%%%%%%%%%%%%%%%%%%%%%%%%%%%%%%%%%%%%%%%%%%%
\subsection{Revision History}

%%%%%%%%%%%%%%%%%%%%%%%%%%%%%%%%%%%%%%%%
\paragraph{v2.0:} 2018/12/30

\begin{itemize}
\item
immediate forward processing
\item
added |\childdocby| mechanism
\item
manual restructured
\end{itemize}

%%%%%%%%%%%%%%%%%%%%%%%%%%%%%%%%%%%%%%%%
\paragraph{v1.6:} 2018/01/17

\begin{itemize}
\item
application for development of include files
\item
corrections to manual
\end{itemize}

%%%%%%%%%%%%%%%%%%%%%%%%%%%%%%%%%%%%%%%%
\paragraph{v1.5:} 2017/05/21

\begin{itemize}
\item
more complete structuring introduced
\item
|\childdocof| introduced
\item
|\childdoc| renamed to |\childdocmain|
\item
|\childredirect| renamed to |\childdocforward| and |\childdocforwardprefix|
and functionality expanded
\end{itemize}

%%%%%%%%%%%%%%%%%%%%%%%%%%%%%%%%%%%%%%%%
\paragraph{v1.0:} 2017/04/27

\begin{itemize}
\item
manual and install package
\item
first version published on CTAN
\end{itemize}

%%%%%%%%%%%%%%%%%%%%%%%%%%%%%%%%%%%%%%%%
\paragraph{v0.6:} 2017/04/26

\begin{itemize}
\item
redirection mechanism added
\end{itemize}

%%%%%%%%%%%%%%%%%%%%%%%%%%%%%%%%%%%%%%%%
\paragraph{v0.5:} 2017/04/26

\begin{itemize}
\item
functionality in definition file
\end{itemize}


%%%%%%%%%%%%%%%%%%%%%%%%%%%%%%%%%%%%%%%%%%%%%%%%%%%%%%%%%%%%%%%%%%%%%%%%%%%%%%%%
%%%%%%%%%%%%%%%%%%%%%%%%%%%%%%%%%%%%%%%%%%%%%%%%%%%%%%%%%%%%%%%%%%%%%%%%%%%%%%%%
%%%%%%%%%%%%%%%%%%%%%%%%%%%%%%%%%%%%%%%%%%%%%%%%%%%%%%%%%%%%%%%%%%%%%%%%%%%%%%%%
\appendix

\settowidth\MacroIndent{\rmfamily\scriptsize 000\ }

 \DocInput{childdoc.dtx}

\end{document}
%</driver>
% \fi
%
% %%%%%%%%%%%%%%%%%%%%%%%%%%%%%%%%%%%%%%%%%%%%%%%%%%%%%%%%%%%%%%%%%%%%%%%%%%%%%%
% %%%%%%%%%%%%%%%%%%%%%%%%%%%%%%%%%%%%%%%%%%%%%%%%%%%%%%%%%%%%%%%%%%%%%%%%%%%%%%
% \section{Sample}
%\iffalse
%<*samplemain>
%\fi
%
% The following presents a sample document
% with two chapters, two parts, a title page,
% a compile flag as well as three forwarding files to set the flag.
% It consists of eight |.tex| files:
% \begin{center}
% \begin{tabular}{ll}
% |cdocsamp.tex|&main file\\
% |cdocsch1.tex|&include file for chapter 1\\
% |cdocsch2.tex|&include file for chapter 2\\
% |cdocspt3.tex|&include file for part 3\\
% |cdocspt4.tex|&include file for part 4\\
% |cdocsdrf.tex|&forwarding file for main file in draft mode\\
% |cdocsfi1.tex|&forwarding file for final version of chapter 1\\
% |cdocsfi2.tex|&forwarding file for final version of chapter 2\\
% \end{tabular}
% \end{center}
% Each of the eight files can be compiled directly by the \LaTeX{} compiler.
%
% %%%%%%%%%%%%%%%%%%%%%%%%%%%%%%%%%%%%%%
% \paragraph{Main File.}
%
% The main file is called |cdocsamp.tex|.
%
% Load the \textsf{childdoc} definitions and
% declare the filename for the main document:
%    \begin{macrocode}
\input{childdoc.def}
\childdocmain{}
%    \end{macrocode}

% Optional override for |\version| flag:
%    \begin{macrocode}
%%\ifchilddoc\else\providecommand{\version}{draft}\fi
%    \end{macrocode}

% Define the default values for the |\version| flag
% (|final| for the main file and |draft| for childs):
%    \begin{macrocode}
\ifchilddoc
\providecommand{\version}{draft}
\else
\providecommand{\version}{final}
\fi
%    \end{macrocode}

% Load the standard document class:
%    \begin{macrocode}
\documentclass[12pt]{article}
%    \end{macrocode}

% Start the document body:
%    \begin{macrocode}
\begin{document}
%    \end{macrocode}

% Declare a title page.
% Print title, part of document being processed and version flag:
%    \begin{macrocode}
\addtocounter{page}{-1}
\begin{center}
{\LARGE\bfseries{}childdoc example\par}
\vspace{1cm}
\ifchilddoc
\ifchilddocmanual part\else chapter\fi:
`\childdocname' of `\childdocjob'\par
\else
main document: `\childdocjob'\par
\fi
version: \version\par
\end{center}
\newpage
%    \end{macrocode}

% Manually include selected file,
% otherwise process as usual:
%    \begin{macrocode}
\ifchilddocmanual
\section*{part `\childdocname'}
\input{\childdocname}
\else
%    \end{macrocode}

% Include the two chapters:
%    \begin{macrocode}
\include{cdocsch1}
\include{cdocsch2}
%    \end{macrocode}

% Include the two parts unless only chapters should be displayed:
%    \begin{macrocode}
\ifchilddoc\else
\section{part three}
\input{cdocspt3}
\section{part four}
\input{cdocspt4}
\fi
%    \end{macrocode}

% Process as usual until here:
%    \begin{macrocode}
\fi
%    \end{macrocode}

% End of document body:
%    \begin{macrocode}
\end{document}
%    \end{macrocode}
%\iffalse
%</samplemain>
%\fi
%
% %%%%%%%%%%%%%%%%%%%%%%%%%%%%%%%%%%%%%%
% \paragraph{Chapter Include Files.}
%
% The include files are called |cdocsch1.tex| and |cdocsch2.tex|.
%
%\iffalse
%<*samplechap1|samplechap2>
%\fi

% Optional override for |\version| flag:
%    \begin{macrocode}
%%\providecommand{\version}{final}
%    \end{macrocode}

% Include the main document:
%    \begin{macrocode}
\input{childdoc.def}
\childdocof{cdocsamp}
%    \end{macrocode}

%\iffalse
%</samplechap1|samplechap2>
%\fi
%
%\iffalse
%<*samplechap1>
%\fi
% Some text for chapter 1:
%    \begin{macrocode}
\section{one}
some text in chapter one
%    \end{macrocode}

%\iffalse
%</samplechap1>
%\fi
% Some text for chapter 2:
%\iffalse
%<*samplechap2>
%\fi
%    \begin{macrocode}
\section{two}
more text in chapter two
%    \end{macrocode}

%\iffalse
%</samplechap2>
%\fi
%
% %%%%%%%%%%%%%%%%%%%%%%%%%%%%%%%%%%%%%%
% \paragraph{Part Include Files.}
%
% The include files are called |cdocspt3.tex| and |cdocspt4.tex|.
%
%\iffalse
%<*samplepart3|samplepart4>
%\fi

% Optional override for |\version| flag:
%    \begin{macrocode}
%%\providecommand{\version}{final}
%    \end{macrocode}

% Include the main document:
%    \begin{macrocode}
\input{childdoc.def}
\childdocby{cdocsamp}
%    \end{macrocode}

%\iffalse
%</samplepart3|samplepart4>
%\fi
%
%\iffalse
%<*samplepart3>
%\fi
% Some text for part 3:
%    \begin{macrocode}
some text in part three
%    \end{macrocode}

%\iffalse
%</samplepart3>
%\fi
% Some text for part 4:
%\iffalse
%<*samplepart4>
%\fi
%    \begin{macrocode}
more text in part four
%    \end{macrocode}

%\iffalse
%</samplepart4>
%\fi
%
% %%%%%%%%%%%%%%%%%%%%%%%%%%%%%%%%%%%%%%
% \paragraph{Forwarding for a Complete Draft.}
%
% The following forwarding file |cdocsdrf.tex|
% compiles the main document in draft mode:
%\iffalse
%<*sampledraft>
%\fi
%    \begin{macrocode}
\def\version{draft}
\input{childdoc.def}
\childdocforward{cdocsamp}
%    \end{macrocode}

%\iffalse
%</sampledraft>
%\fi
%
% %%%%%%%%%%%%%%%%%%%%%%%%%%%%%%%%%%%%%%
% \paragraph{Forwarding for Final Version of the Chapters.}
%
% The following forwarding files |cdocsfn1.tex| and |cdocsfn2.tex|
% (with identical content)
% compile the final versions of the child documents
% |cdocsch1.tex| and |cdocsch2.tex|, respectively:
%\iffalse
%<*samplefinal>
%\fi
%    \begin{macrocode}
\def\version{final}
\input{childdoc.def}
\childdocforwardprefix[cdocsamp]{cdocsfn}{cdocsch}
%    \end{macrocode}

%\iffalse
%</samplefinal>
%\fi
%
% %%%%%%%%%%%%%%%%%%%%%%%%%%%%%%%%%%%%%%
% \paragraph{Command Line Processing.}
%
% The following three command lines generate the output files
% |cdocscld|, |cdocscl1| and |cdocscl2|
% which should be identical to
% |cdocsdrf|, |cdocsch1| and |cdocsfn2|, respectively:
% \begin{center}
% \begin{tabular}{l}
% |latex -jobname cdocscld \|\\
% |  "\def\version{draft}\input{childdoc.def}\childdocforward{cdocsamp}"|\\
% |latex -jobname cdocscl1 \|\\
% |  "\input{childdoc.def}\childdocforward[cdocsamp]{cdocsch1}"|\\
% |latex -jobname cdocscl2 \|\\
% |  "\def\version{final}\input{childdoc.def}\childdocforward{cdocsch2}"|
% \end{tabular}
% \end{center}
% Note that the trailing backslash on each first line
% merely continues the input to the second line
% (for convenient cut ant paste).
% Furthermore, the command |latex| can be replaced by any
% of its alternative versions such as |pdflatex|.
%
% %%%%%%%%%%%%%%%%%%%%%%%%%%%%%%%%%%%%%%%%%%%%%%%%%%%%%%%%%%%%%%%%%%%%%%%%%%%%%%
% %%%%%%%%%%%%%%%%%%%%%%%%%%%%%%%%%%%%%%%%%%%%%%%%%%%%%%%%%%%%%%%%%%%%%%%%%%%%%%
% \section{Implementation}
%\iffalse
%<*package>
%\fi
%
% This section describes the definitions file |childdoc.def|.

% The definitions cannot be loaded using |\usepackage| or |\RequirePackage|
% which has a mechanism to prevent loading a style file more than once.
% When loading the definitions by means of |\input|
% multiple instances have to be prevented manually:
%\iffalse
%This code needs to be before the `\ProvidesFile' directive
%which is defined at the beginning of this file.
%Therefore it is also placed there and commented out here.
%</package>
%<*discard>
%\fi
%    \begin{macrocode}
\ifdefined\childdocmain\endinput\fi
%    \end{macrocode}
%\iffalse
%</discard>
%<*package>
%\fi
%
% \macro{\ifchilddoc}
% \macro{\ifchilddocmanual}
% The conditional |\ifchilddoc| tells whether a
% child (true) or main (false) document is being compiled.
% The conditional |\ifchilddocmanual| tells whether
% the |\includeonly| mechanism is used (false) or
% the selection of child files must be performed manually (true).
% The definitions initialise to false:
%    \begin{macrocode}
\newif\ifchilddoc
\newif\ifchilddocmanual
%    \end{macrocode}

% \macro{\childdocname}
% \macro{\childdocjob}
% The macro |\childdocname| stores the name of the main document
% to be compiled. The macro |\childdocjob| stores the name of
% the document on which the \LaTeX{} compiler was originally invoked.
% The content of |\jobname| cannot be compared
% to filenames specified in the source due to different catcodes.
% The following code rescans |\jobname|, stores the result
% in |\childdocname| and saves a copy in |\childdocjob|:
%    \begin{macrocode}
\edef\childdocname{\scantokens\expandafter{\jobname\noexpand}}
\let\childdocjob\childdocname
%    \end{macrocode}

% \macro{\childdocdisable}
% The macro |\childdocdisable| prevents the main file
% from being processed more than once.
% At this stage, the main document command |\childdocmain|
% is assumed to be called once again where it should do nothing.
% Any subsequent call to it should prevent
% a secondary processing of the main document
% It overwrites the forwarding commands
% |\childdocof| and |\childdocforward|
% with empty macros to prevent further inclusions of the main document:
%    \begin{macrocode}
\newcommand{\childdocdisable}
{
  \renewcommand{\childdocmain}[1]{\renewcommand{\childdocmain}[1]{\endinput}}
  \renewcommand{\childdocof}[1]{}
  \renewcommand{\childdocby}[2][]{}
  \renewcommand{\childdocforward}[2][]{}
  \renewcommand{\childdocdisable}{}
}
%    \end{macrocode}

% \macro{\childdocmain}
% The macro |\childdocmain| is to be called at the top of the main file
% with nothing or the main filename (without extension) as argument.
% First, it breaks loops.
% If the argument is not empty and does not match |\childdocname|
% (which is set by the first inclusion of |childdoc.def|),
% |\ifchilddoc| is set to true, |\includeonly| is applied to the child file
% and |\jobname| is set to the main file
% (for proper handling of |.aux| files):
%    \begin{macrocode}
\newcommand{\childdocmain}[1]
{
  \childdocdisable\childdocmain{}
  \if?#1?\else
    \begingroup
      \def\childdoctmp{#1}
      \ifx\childdoctmp\childdocname
        \def\childdoctmp{}
      \else
        \def\childdoctmp
        {
          \childdoctrue
          \includeonly{\childdocname}
          \def\childdocjob{#1}
          \def\jobname{#1}
        }
      \fi
      \expandafter
    \endgroup
    \childdoctmp
  \fi
}
%    \end{macrocode}

% \macro{\childdocof}
% The command |\childdocof| redirects
% compilation to the main file |#1|.
%    \begin{macrocode}
\newcommand{\childdocof}[1]
{
  \childdocdisable
  \childdoctrue
  \includeonly{\childdocname}
  \def\jobname{#1}
  \def\childdocjob{#1}
  \input{#1}
}
%    \end{macrocode}

% \macro{\childdocby}
% The command |\childdocby| ....
%    \begin{macrocode}
\newcommand{\childdocby}[2][]
{
  \childdocdisable
  \childdoctrue
  \childdocmanualtrue
  \if?#1?\else
    \def\jobname{#2}
  \fi
  \def\childdocjob{#2}
  \input{#2}
  \endinput
}
%    \end{macrocode}

% \macro{\childdocforward}
% The command |\childdocforward| redirects
% compilation to the main file or
% (if the optional argument is given) a child file.
% Parameters are set as if the main file
% or a child file starting with |\childdocof| was compiled.
% Then compilation is handed over to the main file:
%    \begin{macrocode}
\newcommand{\childdocforward}[2][]
{
  \begingroup
    \if?#1?
      \def\childdoctmp
      {
        \def\childdocname{#2}
        \def\childdocjob{#2}
        \def\jobname{#2}
        \input{#2}
        \endinput
      }
    \else
      \def\childdoctmp
      {
        \childdocdisable
        \def\childdocname{#2}
        \childdoctrue
        \includeonly{#2}
        \def\childdocjob{#1}
        \def\jobname{#1}
        \input{#1}
        \endinput
      }
    \fi
    \expandafter
  \endgroup
  \childdoctmp
}
%    \end{macrocode}

% \macro{\childdocforwardprefix}
% The command |\childdocforwardprefix| redirects
% compilation to the main or a child file by means of a pattern.
% The prefix |#1| in the current filename is replaced by |#2|
% and the suffix of the current filename is kept
% (it is assumed that the filename does not contain the substring `|~~~|'
% which is used as a delimiter).
% Compilation is handed over to the new file by |\childdocforward|:
%    \begin{macrocode}
\newcommand{\childdocforwardprefix}[3][]
{
  \begingroup
    \def\childdocextract #2##1~~~{\def\childdoctmp{\childdocforward[#1]{#3##1}}}
    \expandafter\childdocextract\childdocname~~~
    \expandafter
  \endgroup
  \childdoctmp
}
%    \end{macrocode}

% \macro{\childdoc}
% The deprecated macro |\childdoc| is a legacy version of |\childdocmain|:
%    \begin{macrocode}
\newcommand{\childdoc}{\childdocmain}
%    \end{macrocode}

% \macro{\childdocredirect}
% The deprecated macro |\childdocredirect| is a legacy version
% of |\childdocforward| and |\childdocforwardprefix|:
%    \begin{macrocode}
\newcommand{\childdocredirect}[2][]
{
  \begingroup
    \if?#1?
      \def\childdoctmp{\childdocforward{#2}}
    \else
      \def\childdoctmp{\childdocforwardprefix{#1}{#2}}
    \fi
    \expandafter
  \endgroup
  \childdoctmp
}
%    \end{macrocode}

%\iffalse
%</package>
%\fi
%
\endinput
\childdocforward{cdocsch2}"|
% \end{tabular}
% \end{center}
% Note that the trailing backslash on each first line
% merely continues the input to the second line
% (for convenient cut ant paste).
% Furthermore, the command |latex| can be replaced by any
% of its alternative versions such as |pdflatex|.
%
% %%%%%%%%%%%%%%%%%%%%%%%%%%%%%%%%%%%%%%%%%%%%%%%%%%%%%%%%%%%%%%%%%%%%%%%%%%%%%%
% %%%%%%%%%%%%%%%%%%%%%%%%%%%%%%%%%%%%%%%%%%%%%%%%%%%%%%%%%%%%%%%%%%%%%%%%%%%%%%
% \section{Implementation}
%\iffalse
%<*package>
%\fi
%
% This section describes the definitions file |childdoc.def|.

% The definitions cannot be loaded using |\usepackage| or |\RequirePackage|
% which has a mechanism to prevent loading a style file more than once.
% When loading the definitions by means of |\input|
% multiple instances have to be prevented manually:
%\iffalse
%This code needs to be before the `\ProvidesFile' directive
%which is defined at the beginning of this file.
%Therefore it is also placed there and commented out here.
%</package>
%<*discard>
%\fi
%    \begin{macrocode}
\ifdefined\childdocmain\endinput\fi
%    \end{macrocode}
%\iffalse
%</discard>
%<*package>
%\fi
%
% \macro{\ifchilddoc}
% \macro{\ifchilddocmanual}
% The conditional |\ifchilddoc| tells whether a
% child (true) or main (false) document is being compiled.
% The conditional |\ifchilddocmanual| tells whether
% the |\includeonly| mechanism is used (false) or
% the selection of child files must be performed manually (true).
% The definitions initialise to false:
%    \begin{macrocode}
\newif\ifchilddoc
\newif\ifchilddocmanual
%    \end{macrocode}

% \macro{\childdocname}
% \macro{\childdocjob}
% The macro |\childdocname| stores the name of the main document
% to be compiled. The macro |\childdocjob| stores the name of
% the document on which the \LaTeX{} compiler was originally invoked.
% The content of |\jobname| cannot be compared
% to filenames specified in the source due to different catcodes.
% The following code rescans |\jobname|, stores the result
% in |\childdocname| and saves a copy in |\childdocjob|:
%    \begin{macrocode}
\edef\childdocname{\scantokens\expandafter{\jobname\noexpand}}
\let\childdocjob\childdocname
%    \end{macrocode}

% \macro{\childdocdisable}
% The macro |\childdocdisable| prevents the main file
% from being processed more than once.
% At this stage, the main document command |\childdocmain|
% is assumed to be called once again where it should do nothing.
% Any subsequent call to it should prevent
% a secondary processing of the main document
% It overwrites the forwarding commands
% |\childdocof| and |\childdocforward|
% with empty macros to prevent further inclusions of the main document:
%    \begin{macrocode}
\newcommand{\childdocdisable}
{
  \renewcommand{\childdocmain}[1]{\renewcommand{\childdocmain}[1]{\endinput}}
  \renewcommand{\childdocof}[1]{}
  \renewcommand{\childdocby}[2][]{}
  \renewcommand{\childdocforward}[2][]{}
  \renewcommand{\childdocdisable}{}
}
%    \end{macrocode}

% \macro{\childdocmain}
% The macro |\childdocmain| is to be called at the top of the main file
% with nothing or the main filename (without extension) as argument.
% First, it breaks loops.
% If the argument is not empty and does not match |\childdocname|
% (which is set by the first inclusion of |childdoc.def|),
% |\ifchilddoc| is set to true, |\includeonly| is applied to the child file
% and |\jobname| is set to the main file
% (for proper handling of |.aux| files):
%    \begin{macrocode}
\newcommand{\childdocmain}[1]
{
  \childdocdisable\childdocmain{}
  \if?#1?\else
    \begingroup
      \def\childdoctmp{#1}
      \ifx\childdoctmp\childdocname
        \def\childdoctmp{}
      \else
        \def\childdoctmp
        {
          \childdoctrue
          \includeonly{\childdocname}
          \def\childdocjob{#1}
          \def\jobname{#1}
        }
      \fi
      \expandafter
    \endgroup
    \childdoctmp
  \fi
}
%    \end{macrocode}

% \macro{\childdocof}
% The command |\childdocof| redirects
% compilation to the main file |#1|.
%    \begin{macrocode}
\newcommand{\childdocof}[1]
{
  \childdocdisable
  \childdoctrue
  \includeonly{\childdocname}
  \def\jobname{#1}
  \def\childdocjob{#1}
  \input{#1}
}
%    \end{macrocode}

% \macro{\childdocby}
% The command |\childdocby| ....
%    \begin{macrocode}
\newcommand{\childdocby}[2][]
{
  \childdocdisable
  \childdoctrue
  \childdocmanualtrue
  \if?#1?\else
    \def\jobname{#2}
  \fi
  \def\childdocjob{#2}
  \input{#2}
  \endinput
}
%    \end{macrocode}

% \macro{\childdocforward}
% The command |\childdocforward| redirects
% compilation to the main file or
% (if the optional argument is given) a child file.
% Parameters are set as if the main file
% or a child file starting with |\childdocof| was compiled.
% Then compilation is handed over to the main file:
%    \begin{macrocode}
\newcommand{\childdocforward}[2][]
{
  \begingroup
    \if?#1?
      \def\childdoctmp
      {
        \def\childdocname{#2}
        \def\childdocjob{#2}
        \def\jobname{#2}
        \input{#2}
        \endinput
      }
    \else
      \def\childdoctmp
      {
        \childdocdisable
        \def\childdocname{#2}
        \childdoctrue
        \includeonly{#2}
        \def\childdocjob{#1}
        \def\jobname{#1}
        \input{#1}
        \endinput
      }
    \fi
    \expandafter
  \endgroup
  \childdoctmp
}
%    \end{macrocode}

% \macro{\childdocforwardprefix}
% The command |\childdocforwardprefix| redirects
% compilation to the main or a child file by means of a pattern.
% The prefix |#1| in the current filename is replaced by |#2|
% and the suffix of the current filename is kept
% (it is assumed that the filename does not contain the substring `|~~~|'
% which is used as a delimiter).
% Compilation is handed over to the new file by |\childdocforward|:
%    \begin{macrocode}
\newcommand{\childdocforwardprefix}[3][]
{
  \begingroup
    \def\childdocextract #2##1~~~{\def\childdoctmp{\childdocforward[#1]{#3##1}}}
    \expandafter\childdocextract\childdocname~~~
    \expandafter
  \endgroup
  \childdoctmp
}
%    \end{macrocode}

% \macro{\childdoc}
% The deprecated macro |\childdoc| is a legacy version of |\childdocmain|:
%    \begin{macrocode}
\newcommand{\childdoc}{\childdocmain}
%    \end{macrocode}

% \macro{\childdocredirect}
% The deprecated macro |\childdocredirect| is a legacy version
% of |\childdocforward| and |\childdocforwardprefix|:
%    \begin{macrocode}
\newcommand{\childdocredirect}[2][]
{
  \begingroup
    \if?#1?
      \def\childdoctmp{\childdocforward{#2}}
    \else
      \def\childdoctmp{\childdocforwardprefix{#1}{#2}}
    \fi
    \expandafter
  \endgroup
  \childdoctmp
}
%    \end{macrocode}

%\iffalse
%</package>
%\fi
%
\endinput
|\\
|\childdocby{|\textit{main}|}|\\
\end{tabular}
\end{center}
%
The directive |\childdocby| is similar to |\childdocof|
described in \secref{sec:include},
but the subsequent selection of content must be done manually.
To that end, both |\ifchilddoc| and |\ifchilddocmanual|
will be true upon processing of a part,
and the name of the part is stored in |\childdocname|.
Note that |\jobname| will be set to the filename of the current part
so that each part receives an individual |.aux| file
that does not interfere with the |.aux| file(s) of the main document.
This behaviour can be altered by the alternative form
|\childdocby[*]{|\textit{main}|}| (with a non-empty optional argument)
which uses the |.aux| file of the main document
by setting |\jobname| to \textit{main}.

%%%%%%%%%%%%%%%%%%%%%%%%%%%%%%%%%%%%%%%%%%%%%%%%%%%%%%%%%%%%%%%%%%%%%%%%%%%%%%%%
\subsection{Driver Development}
\label{sec:driver}

The \textsf{childdoc} mechanism can also be use for the development
of definition files such as \LaTeX{} styles or classes.
This case differs from the above setup with multiple parts
included by |\include| in that no |\includeonly| should be invoked.
This can be achieved by starting the include file
(before |\ProvidesPackage|) with:
%
\begin{center}
\begin{tabular}{l}
|% \iffalse
%
% childdoc.dtx Copyright (C) 2017-2018 Niklas Beisert
%
% This work may be distributed and/or modified under the
% conditions of the LaTeX Project Public License, either version 1.3
% of this license or (at your option) any later version.
% The latest version of this license is in
%   http://www.latex-project.org/lppl.txt
% and version 1.3 or later is part of all distributions of LaTeX
% version 2005/12/01 or later.
%
% This work has the LPPL maintenance status `maintained'.
%
% The Current Maintainer of this work is Niklas Beisert.
%
% This work consists of the files childdoc.dtx and childdoc.ins
% and the derived files childdoc.def and cdocsamp.tex with
% cdocsch1.tex, cdocsch2.tex, cdocsdrf.tex, cdocsfn1.tex, cdocsfn2.tex.
%
%<package>\ifdefined\childdocmain\endinput\fi
%<package>\ProvidesFile{childdoc.def}[2018/12/30 v2.0 child document driver]
%<samplemain>\ProvidesFile{cdocsamp.tex}[2018/12/30 v2.0 sample for childdoc]
%<*driver>
%\ProvidesFile{childdoc.drv}[2018/12/30 v2.0 childdoc reference manual file]
\PassOptionsToClass{10pt,a4paper}{article}
\documentclass{ltxdoc}

\usepackage[margin=35mm]{geometry}
\usepackage{hyperref}
\usepackage{hyperxmp}
\usepackage[usenames]{color}

\hypersetup{colorlinks=true}
\hypersetup{pdfstartview=FitH}
\hypersetup{pdfpagemode=UseNone}
\hypersetup{pdfsource={}}
\hypersetup{pdflang={en-UK}}
\hypersetup{pdfcopyright={Copyright 2017-2018 Niklas Beisert.
  This work may be distributed and/or modified under the
  conditions of the LaTeX Project Public License, either version 1.3
  of this license or (at your option) any later version.}}
\hypersetup{pdflicenseurl={http://www.latex-project.org/lppl.txt}}
\hypersetup{pdfcontactaddress={ETH Zurich, ITP, HIT K,
  Wolfgang-Pauli-Strasse 27}}
\hypersetup{pdfcontactpostcode={8093}}
\hypersetup{pdfcontactcity={Zurich}}
\hypersetup{pdfcontactcountry={Switzerland}}
\hypersetup{pdfcontactemail={nbeisert@itp.phys.ethz.ch}}
\hypersetup{pdfcontacturl={http://people.phys.ethz.ch/\xmptilde nbeisert/}}

\newcommand{\secref}[1]{\hyperref[#1]{section \ref*{#1}}}

\parskip1ex
\parindent0pt
\let\olditemize\itemize
\def\itemize{\olditemize\parskip0pt}

\begin{document}

\title{The \textsf{childdoc} Package}
\hypersetup{pdftitle={The childdoc Package}}
\author{Niklas Beisert\\[2ex]
  Institut f\"ur Theoretische Physik\\
  Eidgen\"ossische Technische Hochschule Z\"urich\\
  Wolfgang-Pauli-Strasse 27, 8093 Z\"urich, Switzerland\\[1ex]
  \href{mailto:nbeisert@itp.phys.ethz.ch}
  {\texttt{nbeisert@itp.phys.ethz.ch}}}
\hypersetup{pdfauthor={Niklas Beisert}}
\hypersetup{pdfsubject={Manual for the LaTeX2e Package childdoc}}
\date{30 December 2018, \textsf{v2.0}}
\maketitle

\begin{abstract}\noindent
\textsf{childdoc} is a \LaTeXe{} package
that enables the direct compilation
of document sections included by |\include|
to individual files.
\end{abstract}

\begingroup
\parskip0ex
\tableofcontents
\endgroup

%%%%%%%%%%%%%%%%%%%%%%%%%%%%%%%%%%%%%%%%%%%%%%%%%%%%%%%%%%%%%%%%%%%%%%%%%%%%%%%%
%%%%%%%%%%%%%%%%%%%%%%%%%%%%%%%%%%%%%%%%%%%%%%%%%%%%%%%%%%%%%%%%%%%%%%%%%%%%%%%%
\section{Introduction}

\LaTeX{} provides a mechanism to structure a large document (such as a book)
into a main file and several child files (containing the chapters)
using the |\include| command.
This mechanism is beneficial for documents
which span hundreds of pages in order to
make the source file(s) more manageable.
Moreover, compilation can be restricted to
selected child files by means of the |\includeonly| command.
The latter feature can be used to reduce the compilation time while editing
(this was significantly more useful in the earlier days of \LaTeX{})
or to generate a smaller document which is easier to navigate.
Another application of |\includeonly| is to generate
documents consisting of selected parts of the complete document.

However, there are a few drawbacks of the plain |\include| mechanism:
\begin{itemize}
\item
The child files cannot be compiled on their own,
they can only be compiled via the main file.
A naive editing environment
(such as a text editor with an option
to have the current file processed by \LaTeX)
may require one to switch to the main file before compiling;
attempting to compile the child file produces errors.
\item
The main file must be modified (each time)
to adjust the |\includeonly| command
to the present needs. This easily leaves the main file in a messy state.
\item
The generated document will always carry the filename
of the main document. This is inconvenient if
several child files are to be compiled and
to be kept for distribution.
\end{itemize}

The present package provides a simple interface
to make child files individually compilable by \LaTeX{}.
Compiling a child file then has the same effect as compiling
the main file with an |\includeonly| command
to select the appropriate child.
Moreover the generated document will carry the name of the child
rather than the main file.
This resolves all three above issues.

This feature is meant to make the editing of books,
thesis documents and lecture notes somewhat more convenient.
However, the package can also be used efficiently for
composing a series of documents (such as exercise sheets)
which are typically distributed individually.
It then assists the author in generating the individual documents
(potentially in different versions)
as well as a document containing the collected series.
Another application is in developing style files
or other kinds of included material
where compilation of the style file could redirect
to a sample or test file.

%%%%%%%%%%%%%%%%%%%%%%%%%%%%%%%%%%%%%%%%%%%%%%%%%%%%%%%%%%%%%%%%%%%%%%%%%%%%%%%%
%%%%%%%%%%%%%%%%%%%%%%%%%%%%%%%%%%%%%%%%%%%%%%%%%%%%%%%%%%%%%%%%%%%%%%%%%%%%%%%%
\section{Usage}

First of all, the package \textsf{childdoc} is \emph{not} a standard
\LaTeXe{} |.sty| style file! Therefore it needs to be invoked in
a non-standard way.

%%%%%%%%%%%%%%%%%%%%%%%%%%%%%%%%%%%%%%%%%%%%%%%%%%%%%%%%%%%%%%%%%%%%%%%%%%%%%%%%
\subsection{Included Files}
\label{sec:include}

%%%%%%%%%%%%%%%%%%%%%%%%%%%%%%%%%%%%%%%%
\DescribeMacro{\childdocmain}
To use the package, add the commands
\begin{center}
\begin{tabular}{l}
|% \iffalse
%
% childdoc.dtx Copyright (C) 2017-2018 Niklas Beisert
%
% This work may be distributed and/or modified under the
% conditions of the LaTeX Project Public License, either version 1.3
% of this license or (at your option) any later version.
% The latest version of this license is in
%   http://www.latex-project.org/lppl.txt
% and version 1.3 or later is part of all distributions of LaTeX
% version 2005/12/01 or later.
%
% This work has the LPPL maintenance status `maintained'.
%
% The Current Maintainer of this work is Niklas Beisert.
%
% This work consists of the files childdoc.dtx and childdoc.ins
% and the derived files childdoc.def and cdocsamp.tex with
% cdocsch1.tex, cdocsch2.tex, cdocsdrf.tex, cdocsfn1.tex, cdocsfn2.tex.
%
%<package>\ifdefined\childdocmain\endinput\fi
%<package>\ProvidesFile{childdoc.def}[2018/12/30 v2.0 child document driver]
%<samplemain>\ProvidesFile{cdocsamp.tex}[2018/12/30 v2.0 sample for childdoc]
%<*driver>
%\ProvidesFile{childdoc.drv}[2018/12/30 v2.0 childdoc reference manual file]
\PassOptionsToClass{10pt,a4paper}{article}
\documentclass{ltxdoc}

\usepackage[margin=35mm]{geometry}
\usepackage{hyperref}
\usepackage{hyperxmp}
\usepackage[usenames]{color}

\hypersetup{colorlinks=true}
\hypersetup{pdfstartview=FitH}
\hypersetup{pdfpagemode=UseNone}
\hypersetup{pdfsource={}}
\hypersetup{pdflang={en-UK}}
\hypersetup{pdfcopyright={Copyright 2017-2018 Niklas Beisert.
  This work may be distributed and/or modified under the
  conditions of the LaTeX Project Public License, either version 1.3
  of this license or (at your option) any later version.}}
\hypersetup{pdflicenseurl={http://www.latex-project.org/lppl.txt}}
\hypersetup{pdfcontactaddress={ETH Zurich, ITP, HIT K,
  Wolfgang-Pauli-Strasse 27}}
\hypersetup{pdfcontactpostcode={8093}}
\hypersetup{pdfcontactcity={Zurich}}
\hypersetup{pdfcontactcountry={Switzerland}}
\hypersetup{pdfcontactemail={nbeisert@itp.phys.ethz.ch}}
\hypersetup{pdfcontacturl={http://people.phys.ethz.ch/\xmptilde nbeisert/}}

\newcommand{\secref}[1]{\hyperref[#1]{section \ref*{#1}}}

\parskip1ex
\parindent0pt
\let\olditemize\itemize
\def\itemize{\olditemize\parskip0pt}

\begin{document}

\title{The \textsf{childdoc} Package}
\hypersetup{pdftitle={The childdoc Package}}
\author{Niklas Beisert\\[2ex]
  Institut f\"ur Theoretische Physik\\
  Eidgen\"ossische Technische Hochschule Z\"urich\\
  Wolfgang-Pauli-Strasse 27, 8093 Z\"urich, Switzerland\\[1ex]
  \href{mailto:nbeisert@itp.phys.ethz.ch}
  {\texttt{nbeisert@itp.phys.ethz.ch}}}
\hypersetup{pdfauthor={Niklas Beisert}}
\hypersetup{pdfsubject={Manual for the LaTeX2e Package childdoc}}
\date{30 December 2018, \textsf{v2.0}}
\maketitle

\begin{abstract}\noindent
\textsf{childdoc} is a \LaTeXe{} package
that enables the direct compilation
of document sections included by |\include|
to individual files.
\end{abstract}

\begingroup
\parskip0ex
\tableofcontents
\endgroup

%%%%%%%%%%%%%%%%%%%%%%%%%%%%%%%%%%%%%%%%%%%%%%%%%%%%%%%%%%%%%%%%%%%%%%%%%%%%%%%%
%%%%%%%%%%%%%%%%%%%%%%%%%%%%%%%%%%%%%%%%%%%%%%%%%%%%%%%%%%%%%%%%%%%%%%%%%%%%%%%%
\section{Introduction}

\LaTeX{} provides a mechanism to structure a large document (such as a book)
into a main file and several child files (containing the chapters)
using the |\include| command.
This mechanism is beneficial for documents
which span hundreds of pages in order to
make the source file(s) more manageable.
Moreover, compilation can be restricted to
selected child files by means of the |\includeonly| command.
The latter feature can be used to reduce the compilation time while editing
(this was significantly more useful in the earlier days of \LaTeX{})
or to generate a smaller document which is easier to navigate.
Another application of |\includeonly| is to generate
documents consisting of selected parts of the complete document.

However, there are a few drawbacks of the plain |\include| mechanism:
\begin{itemize}
\item
The child files cannot be compiled on their own,
they can only be compiled via the main file.
A naive editing environment
(such as a text editor with an option
to have the current file processed by \LaTeX)
may require one to switch to the main file before compiling;
attempting to compile the child file produces errors.
\item
The main file must be modified (each time)
to adjust the |\includeonly| command
to the present needs. This easily leaves the main file in a messy state.
\item
The generated document will always carry the filename
of the main document. This is inconvenient if
several child files are to be compiled and
to be kept for distribution.
\end{itemize}

The present package provides a simple interface
to make child files individually compilable by \LaTeX{}.
Compiling a child file then has the same effect as compiling
the main file with an |\includeonly| command
to select the appropriate child.
Moreover the generated document will carry the name of the child
rather than the main file.
This resolves all three above issues.

This feature is meant to make the editing of books,
thesis documents and lecture notes somewhat more convenient.
However, the package can also be used efficiently for
composing a series of documents (such as exercise sheets)
which are typically distributed individually.
It then assists the author in generating the individual documents
(potentially in different versions)
as well as a document containing the collected series.
Another application is in developing style files
or other kinds of included material
where compilation of the style file could redirect
to a sample or test file.

%%%%%%%%%%%%%%%%%%%%%%%%%%%%%%%%%%%%%%%%%%%%%%%%%%%%%%%%%%%%%%%%%%%%%%%%%%%%%%%%
%%%%%%%%%%%%%%%%%%%%%%%%%%%%%%%%%%%%%%%%%%%%%%%%%%%%%%%%%%%%%%%%%%%%%%%%%%%%%%%%
\section{Usage}

First of all, the package \textsf{childdoc} is \emph{not} a standard
\LaTeXe{} |.sty| style file! Therefore it needs to be invoked in
a non-standard way.

%%%%%%%%%%%%%%%%%%%%%%%%%%%%%%%%%%%%%%%%%%%%%%%%%%%%%%%%%%%%%%%%%%%%%%%%%%%%%%%%
\subsection{Included Files}
\label{sec:include}

%%%%%%%%%%%%%%%%%%%%%%%%%%%%%%%%%%%%%%%%
\DescribeMacro{\childdocmain}
To use the package, add the commands
\begin{center}
\begin{tabular}{l}
|\input{childdoc.def}|\\
|\childdocmain{}|\\
\end{tabular}
\end{center}
at the very top of the main \LaTeX{} file,
in particular \emph{before} the |\documentclass| statement!
The argument of |\childdocmain| should be left empty
(but it must be present).

%%%%%%%%%%%%%%%%%%%%%%%%%%%%%%%%%%%%%%%%
\DescribeMacro{\childdocof}
Furthermore, add the commands
\begin{center}
\begin{tabular}{l}
|\input{childdoc.def}|\\
|\childdocof{|\textit{main}|}|\\
\end{tabular}
\end{center}
at the top of every child file \textit{child}
which is included by |\include{|\textit{child}|}|
from within the main file
(or at least for those files to be compiled individually).
The argument \textit{main} must be the filename of the main file.

There are a couple of
considerations in setting up the main and child documents:

%%%%%%%%%%%%%%%%%%%%%%%%%%%%%%%%%%%%%%%%
\paragraph{Restrictions.}

Please note the following restrictions:
\begin{itemize}
\item
|\childdocmain| must be called with one argument \textit{main}
to ensure compatibility with earlier version of the package.
It must either be empty (|\childdocmain{}|)
or precisely match the filename of the main file in which it is specified.
See \secref{sec:detection} for further information.
\item
The filename \textit{main} must be specified without the |.tex| extension.
\item
The filename \textit{main} is case sensitive
(even in case-insensitive file systems)
due to internal string comparison.
\item
The argument \textit{main} should be fully expanded, it cannot be a macro.
\item
Subdirectories and special characters should be avoided in filenames.
\item
The command |\childdocmain{|\textit{main}|}| must be followed by a whitespace.
It should not be followed immediately by another command
or by a comment mark `|%|'.
This is because the \TeX{} parser reads the token immediately following
the argument of |\childdocmain| and puts it
at the beginning of every child section;
however, a white\-space is ignored.
\end{itemize}

%%%%%%%%%%%%%%%%%%%%%%%%%%%%%%%%%%%%%%%%
\paragraph{Content of Main File.}

It is advisable to place all content in the child files included by |\include|.
Any output contained in the main file will appear in all child documents
unless suppressed manually;
it cannot be suppressed automatically by the |\includeonly| directive
and thus should normally be avoided.
A method to include some content in the main file
by means of conditional processing is described in \secref{sec:conditional}.

%%%%%%%%%%%%%%%%%%%%%%%%%%%%%%%%%%%%%%%%
\paragraph{Page Numbering.}

When only a part of the document is compiled,
the appropriate numbering of pages
(as well as other status parameters)
is determined from the |.aux| files.
The latter contain information from previous passes.
However this information needs to propagate through
all intermediate child documents.
Therefore the page numbering in child documents may well
be inconsistent until the complete document is compiled at least once.

A useful (if unconventional) way to always ensure a consistent
page numbering is to restart the numbering in each child document
and denote the pages by `\textit{child}|.|\textit{page}'
where \textit{child} represents the chapter/section number of the child file.
This can be achieved by the command
|\numberwithin{page}{|\textit{child}|}|
of the \textsf{amsmath} package
where \textit{child} can be |chapter| or |section|
depending on the chosen structuring.
Alternatively, one can modify the macro |\thepage| appropriately
and reset the counter |page| at the start of each child file.

%%%%%%%%%%%%%%%%%%%%%%%%%%%%%%%%%%%%%%%%%%%%%%%%%%%%%%%%%%%%%%%%%%%%%%%%%%%%%%%%
\subsection{Conditional Processing}
\label{sec:conditional}

The package provides a mechanism to compile different versions
of a document. To customise the versions further some conditional processing
can come in handy to distinguish which version is being compiled.
The package provides two macros to describe the compilation context:

%%%%%%%%%%%%%%%%%%%%%%%%%%%%%%%%%%%%%%%%
\DescribeMacro{\ifchilddoc}
The conditional |\ifchilddoc| distinguishes between the compilation of
child documents and the main document:
%
\begin{center}
|\ifchilddoc |\textit{child-code}| |[|\||else |\textit{main-code}]| \||fi|
\end{center}

%%%%%%%%%%%%%%%%%%%%%%%%%%%%%%%%%%%%%%%%
\DescribeMacro{\childdocname}
\DescribeMacro{\childdocjob}
The macro |\childdocname| contains the filename (without extension)
of the main or child file being processed.
Note that |\childdocjob| will always contain the name of the main file.

%%%%%%%%%%%%%%%%%%%%%%%%%%%%%%%%%%%%%%%%
\paragraph{Title Page.}

Conditional processing can be used to include a title or banner page
in the main document when proper precautions are taken.
Importantly, the code in the main file should ensure that the page counter
(as well as other status parameters which are stored in the |.aux| files)
takes the same value after the conditional processing.
Otherwise the page numbers may take divergent values
depending on which part is compiled.

For example, a title page could be declared by:
%
\begin{center}
\begin{tabular}{l}
|\ifchilddoc\||else|\\
|\addtocounter{page}{-1}|\\
\textit{code for title page}\\
|\newpage|\\
|\||fi|
\end{tabular}
\end{center}
%
A banner page for the child documents can be generated by:
%
\begin{center}
\begin{tabular}{l}
|\ifchilddoc|\\
|\addtocounter{page}{-1}|\\
\textit{code for banner page}\\
|\newpage|\\
|\||fi|
\end{tabular}
\end{center}
%
Here one could write a message such as:
\begin{center}
|This is the part \childdocname{} of \childdocjob{}.|
\end{center}

%%%%%%%%%%%%%%%%%%%%%%%%%%%%%%%%%%%%%%%%%%%%%%%%%%%%%%%%%%%%%%%%%%%%%%%%%%%%%%%%
\subsection{Flags}
\label{sec:flags}

The package makes it easy to generate different versions
of the main or child documents.
To this end compilation flags can be defined
and assigned different default values.
They will be particularly useful in conjunction
with the forwarding mechanism described in \secref{sec:forward}.

For example, it may be useful to have a flag |\version|
which can be set to |draft| or |final|.
The document source will contain some conditional code
depending on the value of |\version|.
Suppose further, the flag should default to |final| for the main file
and to |draft| for child files
which is a natural assignment for editing the document.
This is achieved by placing the following code
in the preamble of the main document
(below the |\childdocmain| directive):
%
\begin{center}
\begin{tabular}{l}
|\ifchilddoc|\\
|\providecommand{\version}{draft}|\\
|\||else|\\
|\providecommand{\version}{final}|\\
|\||fi|
\end{tabular}
\end{center}
%
The definition by |\providecommand| makes sure
that previous definitions are not overwritten.
Further statements |\providecommand{\version}{...}|
can thus be added before the above code to override it.

For the main file, one might add a line
(between |\childdocmain| and the above block)
%
\begin{center}
|%\ifchilddoc\||else\providecommand{\version}{draft}\||fi|
\end{center}
%
which can be uncommented to produce a draft version.
Likewise one can add a line to the very top of a child file
(above the |\childdocof{|\textit{main}|}| directive)
%
\begin{center}
|%\providecommand{\version}{final}|
\end{center}
%
which can be uncommented to produce the final version of this child document.

%%%%%%%%%%%%%%%%%%%%%%%%%%%%%%%%%%%%%%%%%%%%%%%%%%%%%%%%%%%%%%%%%%%%%%%%%%%%%%%%
\subsection{Forwarding}
\label{sec:forward}

Different versions of the main or child documents
using compilation flags as described in \secref{sec:flags}
can be (permanently) stored in different files
for convenient compilation, viewing and distribution.
To this end, the package defines a command
to pass on compilation to a different file:

%%%%%%%%%%%%%%%%%%%%%%%%%%%%%%%%%%%%%%%%
\DescribeMacro{\childdocforward}
The command |\childdocforward| redirects processing to
another source file:
%
\begin{center}
\begin{tabular}{l}
|\input{childdoc.def}|\\
|\childdocforward[|\textit{main}|]{|\textit{dest}|}|\\
\end{tabular}
\end{center}
%
The argument \textit{dest} is the destination file
(without extension).
It should be the main file or one of the child files.
Note that further \textsf{childdoc} directives
such as |\childdocof| and |\childdocforward|
in the indicated file will be processed in this form.
The optional argument \textit{main}
passes on directly to the main file \textit{main}
while pretending to compile the child \textit{dest}.
This form behaves as if \textit{dest}
issues |\childdocof{|\textit{main}|}| right away,
and no further \textsf{childdoc} directives will be processed.

%%%%%%%%%%%%%%%%%%%%%%%%%%%%%%%%%%%%%%%%
\DescribeMacro{\...prefix}
In the alternative form |\childdocforwardprefix|,
%
\begin{center}
\begin{tabular}{l}
|\input{childdoc.def}|\\
|\childdocforwardprefix[|\textit{main}|]{|\textit{prefix}|}{|\textit{dest}|}|
\end{tabular}
\end{center}
%
the destination file is determined by a pattern
depending on the current file:
To make this work, the current file must be called
`{\textit{prefix}\hspace{0.2em}\textit{suffix}}'
with \textit{prefix} matching precisely the argument.
Processing is then passed on to the file
`{\textit{dest}\hspace{0.2em}\textit{suffix}}'.
Surely, the same effect is achieved by
directly specifying the
argument `{\textit{dest}\hspace{0.2em}\textit{suffix}}'
in the first form.
However, that requires to set up a different file
for each child. With the alternative form of the command
all these files can have exactly the same content
which simplifies setting them up and maintaining them.

For example, the following file |draft.tex|
with a compilation flag |\version| as described in \secref{sec:flags}
compiles the main document as a draft:
%
\begin{center}
\begin{tabular}{l}
|\def\version{draft}|\\
|\input{childdoc.def}|\\
|\childdocforward{|\textit{main}|}|
\end{tabular}
\end{center}
%
Likewise, the following files |final|\textit{nn}|.tex|
compile the final version of the child document
|child|\textit{nn}|.tex|:
%
\begin{center}
\begin{tabular}{l}
|\def\version{final}|\\
|\input{childdoc.def}|\\
|\childdocforwardprefix{final}{child}|
\end{tabular}
\end{center}
%

Note that when several versions of a main file and/or of each child file
are to be generated, it may be convenient to set up a |Makefile| or
shell script to automatise the process.

%%%%%%%%%%%%%%%%%%%%%%%%%%%%%%%%%%%%%%%%%%%%%%%%%%%%%%%%%%%%%%%%%%%%%%%%%%%%%%%%
\subsection{Command Line Processing}
\label{sec:commandline}

The effect of redirection files can also be achieved by invoking
the \LaTeX{} compiler with a more elaborate command line.
Most conveniently this should be done as part
of a shell script or a |Makefile|.

When using \textsf{childdoc} in the main file, the following
command lines effectively perform a redirection
(note that depending on the shell being used,
backslashes may have to be doubled: `|\|' $\to$ `|\\|'):
%
\begin{center}
|... -jobname "|\textit{target}|" |\\|"|[\textit{flags}]%
|\input{childdoc.def}\childdocforward[|\textit{main}|]{|\textit{dest}|}"|
\end{center}
%
Here \textit{target} is the name of the output file,
\textit{main} is the name of the main file
and \textit{dest} is the name of the main or child file to be processed
(all filenames without extensions).
The optional argument \textit{main} can be omitted
if \textit{main} matches \textit{dest}.
Optionally, compilation \textit{flags} can be defined via |\def| commands.
This command line makes the \TeX{} engine believe
it is compiling the file \textit{target}
whose content is specified as the latter parameter.
The provided code then forwards the processing to
\textit{main} or \textit{dest} as described in \secref{sec:forward}.

%%%%%%%%%%%%%%%%%%%%%%%%%%%%%%%%%%%%%%%%%%%%%%%%%%%%%%%%%%%%%%%%%%%%%%%%%%%%%%%%
\subsection{Include by Input}
\label{sec:input}

Including child documents by |\include| has some restrictions by design.
Most notably, the content of a child document always occupies
its own set of pages; pages cannot be shared between child documents.
Usually, this behaviour makes perfect sense
because each child document contain an essential part of the document.
However, in some situations it may be desirable to compose
a document from a collection of parts
without having mandatory page breaks between then.
For this case, the package
provides a mechanism to include parts
by |\input| which can also be processed individually.
However, by construction this mechanism
requires manual handling of the content to be output.

%%%%%%%%%%%%%%%%%%%%%%%%%%%%%%%%%%%%%%%%
\DescribeMacro{\ifchilddocmanual}
The main file should be prepared as usual, see \secref{sec:include}.
However, the document body must make a distinction
between processing of an individual part and of the main document, e.g.:
%
\begin{center}
\begin{tabular}{l}
|\ifchilddocmanual|\\
|\input{\childdocname}|\\
|\||else|\\
\textit{document body with }|\input{|\textit{part}|}|\\
|\||fi|
\end{tabular}
\end{center}
%
The conditional |\ifchilddocmanual| is true whenever
a part to be included by |\input| is being compiled,
and the name of the part is stored in |\childdocname|.

%%%%%%%%%%%%%%%%%%%%%%%%%%%%%%%%%%%%%%%%
\DescribeMacro{\childdocby}
Each part to be included by |\input| should start with:
%
\begin{center}
\begin{tabular}{l}
|\input{childdoc.def}|\\
|\childdocby{|\textit{main}|}|\\
\end{tabular}
\end{center}
%
The directive |\childdocby| is similar to |\childdocof|
described in \secref{sec:include},
but the subsequent selection of content must be done manually.
To that end, both |\ifchilddoc| and |\ifchilddocmanual|
will be true upon processing of a part,
and the name of the part is stored in |\childdocname|.
Note that |\jobname| will be set to the filename of the current part
so that each part receives an individual |.aux| file
that does not interfere with the |.aux| file(s) of the main document.
This behaviour can be altered by the alternative form
|\childdocby[*]{|\textit{main}|}| (with a non-empty optional argument)
which uses the |.aux| file of the main document
by setting |\jobname| to \textit{main}.

%%%%%%%%%%%%%%%%%%%%%%%%%%%%%%%%%%%%%%%%%%%%%%%%%%%%%%%%%%%%%%%%%%%%%%%%%%%%%%%%
\subsection{Driver Development}
\label{sec:driver}

The \textsf{childdoc} mechanism can also be use for the development
of definition files such as \LaTeX{} styles or classes.
This case differs from the above setup with multiple parts
included by |\include| in that no |\includeonly| should be invoked.
This can be achieved by starting the include file
(before |\ProvidesPackage|) with:
%
\begin{center}
\begin{tabular}{l}
|\input{childdoc.def}|\\
|\childdocforward{|\textit{main}|}|\\
\end{tabular}
\end{center}
%
or alternatively with:
%
\begin{center}
\begin{tabular}{l}
|\input{childdoc.def}|\\
|\childdocby{|\textit{main}|}|\\
\end{tabular}
\end{center}
%
Both forms have slightly different effects as described above.
The main file is prepared as usual, see \secref{sec:include}.

%%%%%%%%%%%%%%%%%%%%%%%%%%%%%%%%%%%%%%%%%%%%%%%%%%%%%%%%%%%%%%%%%%%%%%%%%%%%%%%%
\subsection{Legacy Detection}
\label{sec:detection}

The directive |\childdocmain| in the main file can detect
whether the complete document or merely a child is to be compiled
even without using the directive |\childdocof|.
This method is deprecated because it is less robust
and there is no compelling reason to use it;
it is merely provided for backward compatibility
and it may be removed in future versions.

If the detection mechanism is to be used,
it is mandatory to correctly specify
the filename of the main file as the argument of |\childdocmain|:
%
\begin{center}
\begin{tabular}{l}
|\input{childdoc.def}|\\
|\childdocmain{|\textit{main}|}|\\
\end{tabular}
\end{center}
%
If |\jobname| does not match the argument \textit{main} of |\childdocmain|,
it is assumed that |\jobname| points to the child file to be compiled.
When using |\childdocmain| with the main file specified as argument,
it suffices to start a child file
with just |\input{|\textit{main}|}|
without loading of the package and using |\childdocof|.
If instead all processing is done
with the appropriate \textsf{childdoc} directives,
the argument of \textit{main} of |\childdocmain| can be empty.

An alternative version of the command line processing described
in \secref{sec:commandline} using the detection mechanism reads:
%
\begin{center}
|... -jobname "|\textit{target}|" "|[\textit{flags}]%
[|\def\jobname{|\textit{dest}|}|]|\input{|\textit{main}|}"|
\end{center}

%%%%%%%%%%%%%%%%%%%%%%%%%%%%%%%%%%%%%%%%%%%%%%%%%%%%%%%%%%%%%%%%%%%%%%%%%%%%%%%%
\subsection{Manual Code}
\label{sec:manual}

In case one cannot be certain whether the definitions file |childdoc.def|
is installed on the target \TeX{} distribution
and one prefers not to ship it,
it is conceivable to paste a few relevant commands into the sources.

To that end, drop all statements |\input{childdoc.def}|
and perform the replacements as outlined below.
Instead of |\childdocmain{|\textit{main}|}| add the following code
to the top of the main file:
%
\begin{center}
\begin{tabular}{l}
|\||ifdefined\childdocname\endinput\||fi\newif\ifchilddoc|\\
|\edef\childdocname{\scantokens\expandafter{\jobname\noexpand}}|\\
|\def\childdocmain{|\textit{main}|}\||ifx\childdocmain\childdocname\||else|\\
|\childdoctrue\includeonly{\childdocname}\let\jobname\childdocmain\||fi|\\
\end{tabular}
\end{center}
%
Instead of |\childdocof{|\textit{main}|}| just include the main file
at the top of each child file:
%
\begin{center}
|\input{|\textit{main}|}|
\end{center}
%
A simple redirection |\childdocforward{|\textit{dest}|}| is achieved by:
%
\begin{center}
|\def\jobname{|\textit{dest}|}\input{\jobname}|
\end{center}
%
The redirection with prefix
|\childdocforwardprefix[|\textit{prefix}|]{|\textit{dest}|}|
is accomplished by:
%
\begin{center}
\begin{tabular}{l}
|{\edef\jobname{\scantokens\expandafter{\jobname\noexpand}}|\\
|\def\redirectjob |\textit{prefix}|#1~~~{\gdef\jobname{|\textit{dest}|#1}}|\\
|\expandafter\redirectjob\jobname~~~}\input{\jobname}|
\end{tabular}
\end{center}

In an alternative approach,
child documents can be compiled by a specific command line
without additional code or specific definitions:
%
\begin{center}
|... -jobname "|\textit{target}|" "|[\textit{flags}]%
|\includeonly{|\textit{dest}|}\input{|\textit{main}|}"|
\end{center}
%

%%%%%%%%%%%%%%%%%%%%%%%%%%%%%%%%%%%%%%%%%%%%%%%%%%%%%%%%%%%%%%%%%%%%%%%%%%%%%%%%
%%%%%%%%%%%%%%%%%%%%%%%%%%%%%%%%%%%%%%%%%%%%%%%%%%%%%%%%%%%%%%%%%%%%%%%%%%%%%%%%
\section{Information}

%%%%%%%%%%%%%%%%%%%%%%%%%%%%%%%%%%%%%%%%%%%%%%%%%%%%%%%%%%%%%%%%%%%%%%%%%%%%%%%%
\subsection{Copyright}

Copyright \copyright{} 2017--2018 Niklas Beisert

This work may be distributed and/or modified under the
conditions of the \LaTeX{} Project Public License, either version 1.3
of this license or (at your option) any later version.
The latest version of this license is in
  \url{http://www.latex-project.org/lppl.txt}
and version 1.3 or later is part of all distributions of \LaTeX{}
version 2005/12/01 or later.

This work has the LPPL maintenance status `maintained'.

The Current Maintainer of this work is Niklas Beisert.

This work consists of the files |README.txt|, |childdoc.ins| and |childdoc.dtx|
as well as the derived files |childdoc.def|, |cdocsamp.tex|
with |cdocsch1.tex|, |cdocsch2.tex|, |cdocspt3.tex|, |cdocspt4.tex|,
|cdocsdrf.tex|, |cdocsfn1.tex|, |cdocsfn2.tex|
as well as |childdoc.pdf|.

%%%%%%%%%%%%%%%%%%%%%%%%%%%%%%%%%%%%%%%%%%%%%%%%%%%%%%%%%%%%%%%%%%%%%%%%%%%%%%%%
\subsection{Files and Installation}

The package consists of the files:
%
\begin{center}
\begin{tabular}{ll}
    |README.txt|   & readme file \\
    |childdoc.ins| & installation file \\
    |childdoc.dtx| & source file \\
    |childdoc.def| & definition file \\
    |cdocsamp.tex| & sample main file \\
    |cdocsch1.tex| & sample include file \\
    |cdocsch2.tex| & sample include file \\
    |cdocspt3.tex| & sample part file \\
    |cdocspt4.tex| & sample part file \\
    |cdocsdrf.tex| & sample redirection file \\
    |cdocsfn1.tex| & sample redirection file \\
    |cdocsfn2.tex| & sample redirection file \\
    |childdoc.pdf| & manual
\end{tabular}
\end{center}
%
The distribution consists of the files
|README.txt|, |childdoc.ins| and |childdoc.dtx|.
%
\begin{itemize}
\item
Run (pdf)\LaTeX{} on |childdoc.dtx|
to compile the manual |childdoc.pdf| (this file).
\item
Run \LaTeX{} on |childdoc.ins| to create the definitions file |childdoc.def|
and the sample |cdocsamp.tex| with include files
|cdocsch1.tex|, |cdocsch2.tex|, |cdocspt3.tex|, |cdocspt4.tex|,
|cdocsdrf.tex|, |cdocsfn1.tex|, |cdocsfn2.tex|.
Then copy the file |childdoc.def| to an appropriate directory of your \LaTeX{}
distribution, e.g.\ \textit{texmf-root}|/tex/latex/childdoc|.
\end{itemize}

%%%%%%%%%%%%%%%%%%%%%%%%%%%%%%%%%%%%%%%%%%%%%%%%%%%%%%%%%%%%%%%%%%%%%%%%%%%%%%%%
\subsection{Related CTAN Packages}

There are several other packages which offer a similar functionality:
%
\begin{itemize}
\item
The packages
\href{http://ctan.org/pkg/docmute}{\textsf{docmute}},
\href{http://ctan.org/pkg/includex}{\textsf{includex}} and
\href{http://ctan.org/pkg/standalone}{\textsf{standalone}}
provide commands to include only the document body of
a child file thus allowing both files to be compiled individually.
\item
The packages \href{http://ctan.org/pkg/subdocs}{\textsf{subdocs}}
and \href{http://ctan.org/pkg/subfiles}{\textsf{subfiles}}
provide structures in which the main and child documents can be
encapsulated and allowing them to be compiled individually.
The inclusion mechanism is different from the conventional |\include|.
\item
The package \href{http://ctan.org/pkg/combine}{\textsf{combine}}
is an elaborate solution to combine several documents into one.
\end{itemize}
%
See also the CTAN topic \href{http://ctan.org/topic/subdocs}{\textsf{subdocs}}
for further related packages.
The present package differs from the above solutions in that
a document structure constructed with the conventional |\include| mechanism
just needs two extra commands at the top of every file
such that all constituent files can be compiled individually.

%%%%%%%%%%%%%%%%%%%%%%%%%%%%%%%%%%%%%%%%%%%%%%%%%%%%%%%%%%%%%%%%%%%%%%%%%%%%%%%%
%\subsection{Feature Suggestions}
%
%The following is a list of features which may be useful for future
%versions of this package:
%%
%\begin{itemize}
%\item
%\ldots
%\end{itemize}

%%%%%%%%%%%%%%%%%%%%%%%%%%%%%%%%%%%%%%%%%%%%%%%%%%%%%%%%%%%%%%%%%%%%%%%%%%%%%%%%
\subsection{Revision History}

%%%%%%%%%%%%%%%%%%%%%%%%%%%%%%%%%%%%%%%%
\paragraph{v2.0:} 2018/12/30

\begin{itemize}
\item
immediate forward processing
\item
added |\childdocby| mechanism
\item
manual restructured
\end{itemize}

%%%%%%%%%%%%%%%%%%%%%%%%%%%%%%%%%%%%%%%%
\paragraph{v1.6:} 2018/01/17

\begin{itemize}
\item
application for development of include files
\item
corrections to manual
\end{itemize}

%%%%%%%%%%%%%%%%%%%%%%%%%%%%%%%%%%%%%%%%
\paragraph{v1.5:} 2017/05/21

\begin{itemize}
\item
more complete structuring introduced
\item
|\childdocof| introduced
\item
|\childdoc| renamed to |\childdocmain|
\item
|\childredirect| renamed to |\childdocforward| and |\childdocforwardprefix|
and functionality expanded
\end{itemize}

%%%%%%%%%%%%%%%%%%%%%%%%%%%%%%%%%%%%%%%%
\paragraph{v1.0:} 2017/04/27

\begin{itemize}
\item
manual and install package
\item
first version published on CTAN
\end{itemize}

%%%%%%%%%%%%%%%%%%%%%%%%%%%%%%%%%%%%%%%%
\paragraph{v0.6:} 2017/04/26

\begin{itemize}
\item
redirection mechanism added
\end{itemize}

%%%%%%%%%%%%%%%%%%%%%%%%%%%%%%%%%%%%%%%%
\paragraph{v0.5:} 2017/04/26

\begin{itemize}
\item
functionality in definition file
\end{itemize}


%%%%%%%%%%%%%%%%%%%%%%%%%%%%%%%%%%%%%%%%%%%%%%%%%%%%%%%%%%%%%%%%%%%%%%%%%%%%%%%%
%%%%%%%%%%%%%%%%%%%%%%%%%%%%%%%%%%%%%%%%%%%%%%%%%%%%%%%%%%%%%%%%%%%%%%%%%%%%%%%%
%%%%%%%%%%%%%%%%%%%%%%%%%%%%%%%%%%%%%%%%%%%%%%%%%%%%%%%%%%%%%%%%%%%%%%%%%%%%%%%%
\appendix

\settowidth\MacroIndent{\rmfamily\scriptsize 000\ }

 \DocInput{childdoc.dtx}

\end{document}
%</driver>
% \fi
%
% %%%%%%%%%%%%%%%%%%%%%%%%%%%%%%%%%%%%%%%%%%%%%%%%%%%%%%%%%%%%%%%%%%%%%%%%%%%%%%
% %%%%%%%%%%%%%%%%%%%%%%%%%%%%%%%%%%%%%%%%%%%%%%%%%%%%%%%%%%%%%%%%%%%%%%%%%%%%%%
% \section{Sample}
%\iffalse
%<*samplemain>
%\fi
%
% The following presents a sample document
% with two chapters, two parts, a title page,
% a compile flag as well as three forwarding files to set the flag.
% It consists of eight |.tex| files:
% \begin{center}
% \begin{tabular}{ll}
% |cdocsamp.tex|&main file\\
% |cdocsch1.tex|&include file for chapter 1\\
% |cdocsch2.tex|&include file for chapter 2\\
% |cdocspt3.tex|&include file for part 3\\
% |cdocspt4.tex|&include file for part 4\\
% |cdocsdrf.tex|&forwarding file for main file in draft mode\\
% |cdocsfi1.tex|&forwarding file for final version of chapter 1\\
% |cdocsfi2.tex|&forwarding file for final version of chapter 2\\
% \end{tabular}
% \end{center}
% Each of the eight files can be compiled directly by the \LaTeX{} compiler.
%
% %%%%%%%%%%%%%%%%%%%%%%%%%%%%%%%%%%%%%%
% \paragraph{Main File.}
%
% The main file is called |cdocsamp.tex|.
%
% Load the \textsf{childdoc} definitions and
% declare the filename for the main document:
%    \begin{macrocode}
\input{childdoc.def}
\childdocmain{}
%    \end{macrocode}

% Optional override for |\version| flag:
%    \begin{macrocode}
%%\ifchilddoc\else\providecommand{\version}{draft}\fi
%    \end{macrocode}

% Define the default values for the |\version| flag
% (|final| for the main file and |draft| for childs):
%    \begin{macrocode}
\ifchilddoc
\providecommand{\version}{draft}
\else
\providecommand{\version}{final}
\fi
%    \end{macrocode}

% Load the standard document class:
%    \begin{macrocode}
\documentclass[12pt]{article}
%    \end{macrocode}

% Start the document body:
%    \begin{macrocode}
\begin{document}
%    \end{macrocode}

% Declare a title page.
% Print title, part of document being processed and version flag:
%    \begin{macrocode}
\addtocounter{page}{-1}
\begin{center}
{\LARGE\bfseries{}childdoc example\par}
\vspace{1cm}
\ifchilddoc
\ifchilddocmanual part\else chapter\fi:
`\childdocname' of `\childdocjob'\par
\else
main document: `\childdocjob'\par
\fi
version: \version\par
\end{center}
\newpage
%    \end{macrocode}

% Manually include selected file,
% otherwise process as usual:
%    \begin{macrocode}
\ifchilddocmanual
\section*{part `\childdocname'}
\input{\childdocname}
\else
%    \end{macrocode}

% Include the two chapters:
%    \begin{macrocode}
\include{cdocsch1}
\include{cdocsch2}
%    \end{macrocode}

% Include the two parts unless only chapters should be displayed:
%    \begin{macrocode}
\ifchilddoc\else
\section{part three}
\input{cdocspt3}
\section{part four}
\input{cdocspt4}
\fi
%    \end{macrocode}

% Process as usual until here:
%    \begin{macrocode}
\fi
%    \end{macrocode}

% End of document body:
%    \begin{macrocode}
\end{document}
%    \end{macrocode}
%\iffalse
%</samplemain>
%\fi
%
% %%%%%%%%%%%%%%%%%%%%%%%%%%%%%%%%%%%%%%
% \paragraph{Chapter Include Files.}
%
% The include files are called |cdocsch1.tex| and |cdocsch2.tex|.
%
%\iffalse
%<*samplechap1|samplechap2>
%\fi

% Optional override for |\version| flag:
%    \begin{macrocode}
%%\providecommand{\version}{final}
%    \end{macrocode}

% Include the main document:
%    \begin{macrocode}
\input{childdoc.def}
\childdocof{cdocsamp}
%    \end{macrocode}

%\iffalse
%</samplechap1|samplechap2>
%\fi
%
%\iffalse
%<*samplechap1>
%\fi
% Some text for chapter 1:
%    \begin{macrocode}
\section{one}
some text in chapter one
%    \end{macrocode}

%\iffalse
%</samplechap1>
%\fi
% Some text for chapter 2:
%\iffalse
%<*samplechap2>
%\fi
%    \begin{macrocode}
\section{two}
more text in chapter two
%    \end{macrocode}

%\iffalse
%</samplechap2>
%\fi
%
% %%%%%%%%%%%%%%%%%%%%%%%%%%%%%%%%%%%%%%
% \paragraph{Part Include Files.}
%
% The include files are called |cdocspt3.tex| and |cdocspt4.tex|.
%
%\iffalse
%<*samplepart3|samplepart4>
%\fi

% Optional override for |\version| flag:
%    \begin{macrocode}
%%\providecommand{\version}{final}
%    \end{macrocode}

% Include the main document:
%    \begin{macrocode}
\input{childdoc.def}
\childdocby{cdocsamp}
%    \end{macrocode}

%\iffalse
%</samplepart3|samplepart4>
%\fi
%
%\iffalse
%<*samplepart3>
%\fi
% Some text for part 3:
%    \begin{macrocode}
some text in part three
%    \end{macrocode}

%\iffalse
%</samplepart3>
%\fi
% Some text for part 4:
%\iffalse
%<*samplepart4>
%\fi
%    \begin{macrocode}
more text in part four
%    \end{macrocode}

%\iffalse
%</samplepart4>
%\fi
%
% %%%%%%%%%%%%%%%%%%%%%%%%%%%%%%%%%%%%%%
% \paragraph{Forwarding for a Complete Draft.}
%
% The following forwarding file |cdocsdrf.tex|
% compiles the main document in draft mode:
%\iffalse
%<*sampledraft>
%\fi
%    \begin{macrocode}
\def\version{draft}
\input{childdoc.def}
\childdocforward{cdocsamp}
%    \end{macrocode}

%\iffalse
%</sampledraft>
%\fi
%
% %%%%%%%%%%%%%%%%%%%%%%%%%%%%%%%%%%%%%%
% \paragraph{Forwarding for Final Version of the Chapters.}
%
% The following forwarding files |cdocsfn1.tex| and |cdocsfn2.tex|
% (with identical content)
% compile the final versions of the child documents
% |cdocsch1.tex| and |cdocsch2.tex|, respectively:
%\iffalse
%<*samplefinal>
%\fi
%    \begin{macrocode}
\def\version{final}
\input{childdoc.def}
\childdocforwardprefix[cdocsamp]{cdocsfn}{cdocsch}
%    \end{macrocode}

%\iffalse
%</samplefinal>
%\fi
%
% %%%%%%%%%%%%%%%%%%%%%%%%%%%%%%%%%%%%%%
% \paragraph{Command Line Processing.}
%
% The following three command lines generate the output files
% |cdocscld|, |cdocscl1| and |cdocscl2|
% which should be identical to
% |cdocsdrf|, |cdocsch1| and |cdocsfn2|, respectively:
% \begin{center}
% \begin{tabular}{l}
% |latex -jobname cdocscld \|\\
% |  "\def\version{draft}\input{childdoc.def}\childdocforward{cdocsamp}"|\\
% |latex -jobname cdocscl1 \|\\
% |  "\input{childdoc.def}\childdocforward[cdocsamp]{cdocsch1}"|\\
% |latex -jobname cdocscl2 \|\\
% |  "\def\version{final}\input{childdoc.def}\childdocforward{cdocsch2}"|
% \end{tabular}
% \end{center}
% Note that the trailing backslash on each first line
% merely continues the input to the second line
% (for convenient cut ant paste).
% Furthermore, the command |latex| can be replaced by any
% of its alternative versions such as |pdflatex|.
%
% %%%%%%%%%%%%%%%%%%%%%%%%%%%%%%%%%%%%%%%%%%%%%%%%%%%%%%%%%%%%%%%%%%%%%%%%%%%%%%
% %%%%%%%%%%%%%%%%%%%%%%%%%%%%%%%%%%%%%%%%%%%%%%%%%%%%%%%%%%%%%%%%%%%%%%%%%%%%%%
% \section{Implementation}
%\iffalse
%<*package>
%\fi
%
% This section describes the definitions file |childdoc.def|.

% The definitions cannot be loaded using |\usepackage| or |\RequirePackage|
% which has a mechanism to prevent loading a style file more than once.
% When loading the definitions by means of |\input|
% multiple instances have to be prevented manually:
%\iffalse
%This code needs to be before the `\ProvidesFile' directive
%which is defined at the beginning of this file.
%Therefore it is also placed there and commented out here.
%</package>
%<*discard>
%\fi
%    \begin{macrocode}
\ifdefined\childdocmain\endinput\fi
%    \end{macrocode}
%\iffalse
%</discard>
%<*package>
%\fi
%
% \macro{\ifchilddoc}
% \macro{\ifchilddocmanual}
% The conditional |\ifchilddoc| tells whether a
% child (true) or main (false) document is being compiled.
% The conditional |\ifchilddocmanual| tells whether
% the |\includeonly| mechanism is used (false) or
% the selection of child files must be performed manually (true).
% The definitions initialise to false:
%    \begin{macrocode}
\newif\ifchilddoc
\newif\ifchilddocmanual
%    \end{macrocode}

% \macro{\childdocname}
% \macro{\childdocjob}
% The macro |\childdocname| stores the name of the main document
% to be compiled. The macro |\childdocjob| stores the name of
% the document on which the \LaTeX{} compiler was originally invoked.
% The content of |\jobname| cannot be compared
% to filenames specified in the source due to different catcodes.
% The following code rescans |\jobname|, stores the result
% in |\childdocname| and saves a copy in |\childdocjob|:
%    \begin{macrocode}
\edef\childdocname{\scantokens\expandafter{\jobname\noexpand}}
\let\childdocjob\childdocname
%    \end{macrocode}

% \macro{\childdocdisable}
% The macro |\childdocdisable| prevents the main file
% from being processed more than once.
% At this stage, the main document command |\childdocmain|
% is assumed to be called once again where it should do nothing.
% Any subsequent call to it should prevent
% a secondary processing of the main document
% It overwrites the forwarding commands
% |\childdocof| and |\childdocforward|
% with empty macros to prevent further inclusions of the main document:
%    \begin{macrocode}
\newcommand{\childdocdisable}
{
  \renewcommand{\childdocmain}[1]{\renewcommand{\childdocmain}[1]{\endinput}}
  \renewcommand{\childdocof}[1]{}
  \renewcommand{\childdocby}[2][]{}
  \renewcommand{\childdocforward}[2][]{}
  \renewcommand{\childdocdisable}{}
}
%    \end{macrocode}

% \macro{\childdocmain}
% The macro |\childdocmain| is to be called at the top of the main file
% with nothing or the main filename (without extension) as argument.
% First, it breaks loops.
% If the argument is not empty and does not match |\childdocname|
% (which is set by the first inclusion of |childdoc.def|),
% |\ifchilddoc| is set to true, |\includeonly| is applied to the child file
% and |\jobname| is set to the main file
% (for proper handling of |.aux| files):
%    \begin{macrocode}
\newcommand{\childdocmain}[1]
{
  \childdocdisable\childdocmain{}
  \if?#1?\else
    \begingroup
      \def\childdoctmp{#1}
      \ifx\childdoctmp\childdocname
        \def\childdoctmp{}
      \else
        \def\childdoctmp
        {
          \childdoctrue
          \includeonly{\childdocname}
          \def\childdocjob{#1}
          \def\jobname{#1}
        }
      \fi
      \expandafter
    \endgroup
    \childdoctmp
  \fi
}
%    \end{macrocode}

% \macro{\childdocof}
% The command |\childdocof| redirects
% compilation to the main file |#1|.
%    \begin{macrocode}
\newcommand{\childdocof}[1]
{
  \childdocdisable
  \childdoctrue
  \includeonly{\childdocname}
  \def\jobname{#1}
  \def\childdocjob{#1}
  \input{#1}
}
%    \end{macrocode}

% \macro{\childdocby}
% The command |\childdocby| ....
%    \begin{macrocode}
\newcommand{\childdocby}[2][]
{
  \childdocdisable
  \childdoctrue
  \childdocmanualtrue
  \if?#1?\else
    \def\jobname{#2}
  \fi
  \def\childdocjob{#2}
  \input{#2}
  \endinput
}
%    \end{macrocode}

% \macro{\childdocforward}
% The command |\childdocforward| redirects
% compilation to the main file or
% (if the optional argument is given) a child file.
% Parameters are set as if the main file
% or a child file starting with |\childdocof| was compiled.
% Then compilation is handed over to the main file:
%    \begin{macrocode}
\newcommand{\childdocforward}[2][]
{
  \begingroup
    \if?#1?
      \def\childdoctmp
      {
        \def\childdocname{#2}
        \def\childdocjob{#2}
        \def\jobname{#2}
        \input{#2}
        \endinput
      }
    \else
      \def\childdoctmp
      {
        \childdocdisable
        \def\childdocname{#2}
        \childdoctrue
        \includeonly{#2}
        \def\childdocjob{#1}
        \def\jobname{#1}
        \input{#1}
        \endinput
      }
    \fi
    \expandafter
  \endgroup
  \childdoctmp
}
%    \end{macrocode}

% \macro{\childdocforwardprefix}
% The command |\childdocforwardprefix| redirects
% compilation to the main or a child file by means of a pattern.
% The prefix |#1| in the current filename is replaced by |#2|
% and the suffix of the current filename is kept
% (it is assumed that the filename does not contain the substring `|~~~|'
% which is used as a delimiter).
% Compilation is handed over to the new file by |\childdocforward|:
%    \begin{macrocode}
\newcommand{\childdocforwardprefix}[3][]
{
  \begingroup
    \def\childdocextract #2##1~~~{\def\childdoctmp{\childdocforward[#1]{#3##1}}}
    \expandafter\childdocextract\childdocname~~~
    \expandafter
  \endgroup
  \childdoctmp
}
%    \end{macrocode}

% \macro{\childdoc}
% The deprecated macro |\childdoc| is a legacy version of |\childdocmain|:
%    \begin{macrocode}
\newcommand{\childdoc}{\childdocmain}
%    \end{macrocode}

% \macro{\childdocredirect}
% The deprecated macro |\childdocredirect| is a legacy version
% of |\childdocforward| and |\childdocforwardprefix|:
%    \begin{macrocode}
\newcommand{\childdocredirect}[2][]
{
  \begingroup
    \if?#1?
      \def\childdoctmp{\childdocforward{#2}}
    \else
      \def\childdoctmp{\childdocforwardprefix{#1}{#2}}
    \fi
    \expandafter
  \endgroup
  \childdoctmp
}
%    \end{macrocode}

%\iffalse
%</package>
%\fi
%
\endinput
|\\
|\childdocmain{}|\\
\end{tabular}
\end{center}
at the very top of the main \LaTeX{} file,
in particular \emph{before} the |\documentclass| statement!
The argument of |\childdocmain| should be left empty
(but it must be present).

%%%%%%%%%%%%%%%%%%%%%%%%%%%%%%%%%%%%%%%%
\DescribeMacro{\childdocof}
Furthermore, add the commands
\begin{center}
\begin{tabular}{l}
|% \iffalse
%
% childdoc.dtx Copyright (C) 2017-2018 Niklas Beisert
%
% This work may be distributed and/or modified under the
% conditions of the LaTeX Project Public License, either version 1.3
% of this license or (at your option) any later version.
% The latest version of this license is in
%   http://www.latex-project.org/lppl.txt
% and version 1.3 or later is part of all distributions of LaTeX
% version 2005/12/01 or later.
%
% This work has the LPPL maintenance status `maintained'.
%
% The Current Maintainer of this work is Niklas Beisert.
%
% This work consists of the files childdoc.dtx and childdoc.ins
% and the derived files childdoc.def and cdocsamp.tex with
% cdocsch1.tex, cdocsch2.tex, cdocsdrf.tex, cdocsfn1.tex, cdocsfn2.tex.
%
%<package>\ifdefined\childdocmain\endinput\fi
%<package>\ProvidesFile{childdoc.def}[2018/12/30 v2.0 child document driver]
%<samplemain>\ProvidesFile{cdocsamp.tex}[2018/12/30 v2.0 sample for childdoc]
%<*driver>
%\ProvidesFile{childdoc.drv}[2018/12/30 v2.0 childdoc reference manual file]
\PassOptionsToClass{10pt,a4paper}{article}
\documentclass{ltxdoc}

\usepackage[margin=35mm]{geometry}
\usepackage{hyperref}
\usepackage{hyperxmp}
\usepackage[usenames]{color}

\hypersetup{colorlinks=true}
\hypersetup{pdfstartview=FitH}
\hypersetup{pdfpagemode=UseNone}
\hypersetup{pdfsource={}}
\hypersetup{pdflang={en-UK}}
\hypersetup{pdfcopyright={Copyright 2017-2018 Niklas Beisert.
  This work may be distributed and/or modified under the
  conditions of the LaTeX Project Public License, either version 1.3
  of this license or (at your option) any later version.}}
\hypersetup{pdflicenseurl={http://www.latex-project.org/lppl.txt}}
\hypersetup{pdfcontactaddress={ETH Zurich, ITP, HIT K,
  Wolfgang-Pauli-Strasse 27}}
\hypersetup{pdfcontactpostcode={8093}}
\hypersetup{pdfcontactcity={Zurich}}
\hypersetup{pdfcontactcountry={Switzerland}}
\hypersetup{pdfcontactemail={nbeisert@itp.phys.ethz.ch}}
\hypersetup{pdfcontacturl={http://people.phys.ethz.ch/\xmptilde nbeisert/}}

\newcommand{\secref}[1]{\hyperref[#1]{section \ref*{#1}}}

\parskip1ex
\parindent0pt
\let\olditemize\itemize
\def\itemize{\olditemize\parskip0pt}

\begin{document}

\title{The \textsf{childdoc} Package}
\hypersetup{pdftitle={The childdoc Package}}
\author{Niklas Beisert\\[2ex]
  Institut f\"ur Theoretische Physik\\
  Eidgen\"ossische Technische Hochschule Z\"urich\\
  Wolfgang-Pauli-Strasse 27, 8093 Z\"urich, Switzerland\\[1ex]
  \href{mailto:nbeisert@itp.phys.ethz.ch}
  {\texttt{nbeisert@itp.phys.ethz.ch}}}
\hypersetup{pdfauthor={Niklas Beisert}}
\hypersetup{pdfsubject={Manual for the LaTeX2e Package childdoc}}
\date{30 December 2018, \textsf{v2.0}}
\maketitle

\begin{abstract}\noindent
\textsf{childdoc} is a \LaTeXe{} package
that enables the direct compilation
of document sections included by |\include|
to individual files.
\end{abstract}

\begingroup
\parskip0ex
\tableofcontents
\endgroup

%%%%%%%%%%%%%%%%%%%%%%%%%%%%%%%%%%%%%%%%%%%%%%%%%%%%%%%%%%%%%%%%%%%%%%%%%%%%%%%%
%%%%%%%%%%%%%%%%%%%%%%%%%%%%%%%%%%%%%%%%%%%%%%%%%%%%%%%%%%%%%%%%%%%%%%%%%%%%%%%%
\section{Introduction}

\LaTeX{} provides a mechanism to structure a large document (such as a book)
into a main file and several child files (containing the chapters)
using the |\include| command.
This mechanism is beneficial for documents
which span hundreds of pages in order to
make the source file(s) more manageable.
Moreover, compilation can be restricted to
selected child files by means of the |\includeonly| command.
The latter feature can be used to reduce the compilation time while editing
(this was significantly more useful in the earlier days of \LaTeX{})
or to generate a smaller document which is easier to navigate.
Another application of |\includeonly| is to generate
documents consisting of selected parts of the complete document.

However, there are a few drawbacks of the plain |\include| mechanism:
\begin{itemize}
\item
The child files cannot be compiled on their own,
they can only be compiled via the main file.
A naive editing environment
(such as a text editor with an option
to have the current file processed by \LaTeX)
may require one to switch to the main file before compiling;
attempting to compile the child file produces errors.
\item
The main file must be modified (each time)
to adjust the |\includeonly| command
to the present needs. This easily leaves the main file in a messy state.
\item
The generated document will always carry the filename
of the main document. This is inconvenient if
several child files are to be compiled and
to be kept for distribution.
\end{itemize}

The present package provides a simple interface
to make child files individually compilable by \LaTeX{}.
Compiling a child file then has the same effect as compiling
the main file with an |\includeonly| command
to select the appropriate child.
Moreover the generated document will carry the name of the child
rather than the main file.
This resolves all three above issues.

This feature is meant to make the editing of books,
thesis documents and lecture notes somewhat more convenient.
However, the package can also be used efficiently for
composing a series of documents (such as exercise sheets)
which are typically distributed individually.
It then assists the author in generating the individual documents
(potentially in different versions)
as well as a document containing the collected series.
Another application is in developing style files
or other kinds of included material
where compilation of the style file could redirect
to a sample or test file.

%%%%%%%%%%%%%%%%%%%%%%%%%%%%%%%%%%%%%%%%%%%%%%%%%%%%%%%%%%%%%%%%%%%%%%%%%%%%%%%%
%%%%%%%%%%%%%%%%%%%%%%%%%%%%%%%%%%%%%%%%%%%%%%%%%%%%%%%%%%%%%%%%%%%%%%%%%%%%%%%%
\section{Usage}

First of all, the package \textsf{childdoc} is \emph{not} a standard
\LaTeXe{} |.sty| style file! Therefore it needs to be invoked in
a non-standard way.

%%%%%%%%%%%%%%%%%%%%%%%%%%%%%%%%%%%%%%%%%%%%%%%%%%%%%%%%%%%%%%%%%%%%%%%%%%%%%%%%
\subsection{Included Files}
\label{sec:include}

%%%%%%%%%%%%%%%%%%%%%%%%%%%%%%%%%%%%%%%%
\DescribeMacro{\childdocmain}
To use the package, add the commands
\begin{center}
\begin{tabular}{l}
|\input{childdoc.def}|\\
|\childdocmain{}|\\
\end{tabular}
\end{center}
at the very top of the main \LaTeX{} file,
in particular \emph{before} the |\documentclass| statement!
The argument of |\childdocmain| should be left empty
(but it must be present).

%%%%%%%%%%%%%%%%%%%%%%%%%%%%%%%%%%%%%%%%
\DescribeMacro{\childdocof}
Furthermore, add the commands
\begin{center}
\begin{tabular}{l}
|\input{childdoc.def}|\\
|\childdocof{|\textit{main}|}|\\
\end{tabular}
\end{center}
at the top of every child file \textit{child}
which is included by |\include{|\textit{child}|}|
from within the main file
(or at least for those files to be compiled individually).
The argument \textit{main} must be the filename of the main file.

There are a couple of
considerations in setting up the main and child documents:

%%%%%%%%%%%%%%%%%%%%%%%%%%%%%%%%%%%%%%%%
\paragraph{Restrictions.}

Please note the following restrictions:
\begin{itemize}
\item
|\childdocmain| must be called with one argument \textit{main}
to ensure compatibility with earlier version of the package.
It must either be empty (|\childdocmain{}|)
or precisely match the filename of the main file in which it is specified.
See \secref{sec:detection} for further information.
\item
The filename \textit{main} must be specified without the |.tex| extension.
\item
The filename \textit{main} is case sensitive
(even in case-insensitive file systems)
due to internal string comparison.
\item
The argument \textit{main} should be fully expanded, it cannot be a macro.
\item
Subdirectories and special characters should be avoided in filenames.
\item
The command |\childdocmain{|\textit{main}|}| must be followed by a whitespace.
It should not be followed immediately by another command
or by a comment mark `|%|'.
This is because the \TeX{} parser reads the token immediately following
the argument of |\childdocmain| and puts it
at the beginning of every child section;
however, a white\-space is ignored.
\end{itemize}

%%%%%%%%%%%%%%%%%%%%%%%%%%%%%%%%%%%%%%%%
\paragraph{Content of Main File.}

It is advisable to place all content in the child files included by |\include|.
Any output contained in the main file will appear in all child documents
unless suppressed manually;
it cannot be suppressed automatically by the |\includeonly| directive
and thus should normally be avoided.
A method to include some content in the main file
by means of conditional processing is described in \secref{sec:conditional}.

%%%%%%%%%%%%%%%%%%%%%%%%%%%%%%%%%%%%%%%%
\paragraph{Page Numbering.}

When only a part of the document is compiled,
the appropriate numbering of pages
(as well as other status parameters)
is determined from the |.aux| files.
The latter contain information from previous passes.
However this information needs to propagate through
all intermediate child documents.
Therefore the page numbering in child documents may well
be inconsistent until the complete document is compiled at least once.

A useful (if unconventional) way to always ensure a consistent
page numbering is to restart the numbering in each child document
and denote the pages by `\textit{child}|.|\textit{page}'
where \textit{child} represents the chapter/section number of the child file.
This can be achieved by the command
|\numberwithin{page}{|\textit{child}|}|
of the \textsf{amsmath} package
where \textit{child} can be |chapter| or |section|
depending on the chosen structuring.
Alternatively, one can modify the macro |\thepage| appropriately
and reset the counter |page| at the start of each child file.

%%%%%%%%%%%%%%%%%%%%%%%%%%%%%%%%%%%%%%%%%%%%%%%%%%%%%%%%%%%%%%%%%%%%%%%%%%%%%%%%
\subsection{Conditional Processing}
\label{sec:conditional}

The package provides a mechanism to compile different versions
of a document. To customise the versions further some conditional processing
can come in handy to distinguish which version is being compiled.
The package provides two macros to describe the compilation context:

%%%%%%%%%%%%%%%%%%%%%%%%%%%%%%%%%%%%%%%%
\DescribeMacro{\ifchilddoc}
The conditional |\ifchilddoc| distinguishes between the compilation of
child documents and the main document:
%
\begin{center}
|\ifchilddoc |\textit{child-code}| |[|\||else |\textit{main-code}]| \||fi|
\end{center}

%%%%%%%%%%%%%%%%%%%%%%%%%%%%%%%%%%%%%%%%
\DescribeMacro{\childdocname}
\DescribeMacro{\childdocjob}
The macro |\childdocname| contains the filename (without extension)
of the main or child file being processed.
Note that |\childdocjob| will always contain the name of the main file.

%%%%%%%%%%%%%%%%%%%%%%%%%%%%%%%%%%%%%%%%
\paragraph{Title Page.}

Conditional processing can be used to include a title or banner page
in the main document when proper precautions are taken.
Importantly, the code in the main file should ensure that the page counter
(as well as other status parameters which are stored in the |.aux| files)
takes the same value after the conditional processing.
Otherwise the page numbers may take divergent values
depending on which part is compiled.

For example, a title page could be declared by:
%
\begin{center}
\begin{tabular}{l}
|\ifchilddoc\||else|\\
|\addtocounter{page}{-1}|\\
\textit{code for title page}\\
|\newpage|\\
|\||fi|
\end{tabular}
\end{center}
%
A banner page for the child documents can be generated by:
%
\begin{center}
\begin{tabular}{l}
|\ifchilddoc|\\
|\addtocounter{page}{-1}|\\
\textit{code for banner page}\\
|\newpage|\\
|\||fi|
\end{tabular}
\end{center}
%
Here one could write a message such as:
\begin{center}
|This is the part \childdocname{} of \childdocjob{}.|
\end{center}

%%%%%%%%%%%%%%%%%%%%%%%%%%%%%%%%%%%%%%%%%%%%%%%%%%%%%%%%%%%%%%%%%%%%%%%%%%%%%%%%
\subsection{Flags}
\label{sec:flags}

The package makes it easy to generate different versions
of the main or child documents.
To this end compilation flags can be defined
and assigned different default values.
They will be particularly useful in conjunction
with the forwarding mechanism described in \secref{sec:forward}.

For example, it may be useful to have a flag |\version|
which can be set to |draft| or |final|.
The document source will contain some conditional code
depending on the value of |\version|.
Suppose further, the flag should default to |final| for the main file
and to |draft| for child files
which is a natural assignment for editing the document.
This is achieved by placing the following code
in the preamble of the main document
(below the |\childdocmain| directive):
%
\begin{center}
\begin{tabular}{l}
|\ifchilddoc|\\
|\providecommand{\version}{draft}|\\
|\||else|\\
|\providecommand{\version}{final}|\\
|\||fi|
\end{tabular}
\end{center}
%
The definition by |\providecommand| makes sure
that previous definitions are not overwritten.
Further statements |\providecommand{\version}{...}|
can thus be added before the above code to override it.

For the main file, one might add a line
(between |\childdocmain| and the above block)
%
\begin{center}
|%\ifchilddoc\||else\providecommand{\version}{draft}\||fi|
\end{center}
%
which can be uncommented to produce a draft version.
Likewise one can add a line to the very top of a child file
(above the |\childdocof{|\textit{main}|}| directive)
%
\begin{center}
|%\providecommand{\version}{final}|
\end{center}
%
which can be uncommented to produce the final version of this child document.

%%%%%%%%%%%%%%%%%%%%%%%%%%%%%%%%%%%%%%%%%%%%%%%%%%%%%%%%%%%%%%%%%%%%%%%%%%%%%%%%
\subsection{Forwarding}
\label{sec:forward}

Different versions of the main or child documents
using compilation flags as described in \secref{sec:flags}
can be (permanently) stored in different files
for convenient compilation, viewing and distribution.
To this end, the package defines a command
to pass on compilation to a different file:

%%%%%%%%%%%%%%%%%%%%%%%%%%%%%%%%%%%%%%%%
\DescribeMacro{\childdocforward}
The command |\childdocforward| redirects processing to
another source file:
%
\begin{center}
\begin{tabular}{l}
|\input{childdoc.def}|\\
|\childdocforward[|\textit{main}|]{|\textit{dest}|}|\\
\end{tabular}
\end{center}
%
The argument \textit{dest} is the destination file
(without extension).
It should be the main file or one of the child files.
Note that further \textsf{childdoc} directives
such as |\childdocof| and |\childdocforward|
in the indicated file will be processed in this form.
The optional argument \textit{main}
passes on directly to the main file \textit{main}
while pretending to compile the child \textit{dest}.
This form behaves as if \textit{dest}
issues |\childdocof{|\textit{main}|}| right away,
and no further \textsf{childdoc} directives will be processed.

%%%%%%%%%%%%%%%%%%%%%%%%%%%%%%%%%%%%%%%%
\DescribeMacro{\...prefix}
In the alternative form |\childdocforwardprefix|,
%
\begin{center}
\begin{tabular}{l}
|\input{childdoc.def}|\\
|\childdocforwardprefix[|\textit{main}|]{|\textit{prefix}|}{|\textit{dest}|}|
\end{tabular}
\end{center}
%
the destination file is determined by a pattern
depending on the current file:
To make this work, the current file must be called
`{\textit{prefix}\hspace{0.2em}\textit{suffix}}'
with \textit{prefix} matching precisely the argument.
Processing is then passed on to the file
`{\textit{dest}\hspace{0.2em}\textit{suffix}}'.
Surely, the same effect is achieved by
directly specifying the
argument `{\textit{dest}\hspace{0.2em}\textit{suffix}}'
in the first form.
However, that requires to set up a different file
for each child. With the alternative form of the command
all these files can have exactly the same content
which simplifies setting them up and maintaining them.

For example, the following file |draft.tex|
with a compilation flag |\version| as described in \secref{sec:flags}
compiles the main document as a draft:
%
\begin{center}
\begin{tabular}{l}
|\def\version{draft}|\\
|\input{childdoc.def}|\\
|\childdocforward{|\textit{main}|}|
\end{tabular}
\end{center}
%
Likewise, the following files |final|\textit{nn}|.tex|
compile the final version of the child document
|child|\textit{nn}|.tex|:
%
\begin{center}
\begin{tabular}{l}
|\def\version{final}|\\
|\input{childdoc.def}|\\
|\childdocforwardprefix{final}{child}|
\end{tabular}
\end{center}
%

Note that when several versions of a main file and/or of each child file
are to be generated, it may be convenient to set up a |Makefile| or
shell script to automatise the process.

%%%%%%%%%%%%%%%%%%%%%%%%%%%%%%%%%%%%%%%%%%%%%%%%%%%%%%%%%%%%%%%%%%%%%%%%%%%%%%%%
\subsection{Command Line Processing}
\label{sec:commandline}

The effect of redirection files can also be achieved by invoking
the \LaTeX{} compiler with a more elaborate command line.
Most conveniently this should be done as part
of a shell script or a |Makefile|.

When using \textsf{childdoc} in the main file, the following
command lines effectively perform a redirection
(note that depending on the shell being used,
backslashes may have to be doubled: `|\|' $\to$ `|\\|'):
%
\begin{center}
|... -jobname "|\textit{target}|" |\\|"|[\textit{flags}]%
|\input{childdoc.def}\childdocforward[|\textit{main}|]{|\textit{dest}|}"|
\end{center}
%
Here \textit{target} is the name of the output file,
\textit{main} is the name of the main file
and \textit{dest} is the name of the main or child file to be processed
(all filenames without extensions).
The optional argument \textit{main} can be omitted
if \textit{main} matches \textit{dest}.
Optionally, compilation \textit{flags} can be defined via |\def| commands.
This command line makes the \TeX{} engine believe
it is compiling the file \textit{target}
whose content is specified as the latter parameter.
The provided code then forwards the processing to
\textit{main} or \textit{dest} as described in \secref{sec:forward}.

%%%%%%%%%%%%%%%%%%%%%%%%%%%%%%%%%%%%%%%%%%%%%%%%%%%%%%%%%%%%%%%%%%%%%%%%%%%%%%%%
\subsection{Include by Input}
\label{sec:input}

Including child documents by |\include| has some restrictions by design.
Most notably, the content of a child document always occupies
its own set of pages; pages cannot be shared between child documents.
Usually, this behaviour makes perfect sense
because each child document contain an essential part of the document.
However, in some situations it may be desirable to compose
a document from a collection of parts
without having mandatory page breaks between then.
For this case, the package
provides a mechanism to include parts
by |\input| which can also be processed individually.
However, by construction this mechanism
requires manual handling of the content to be output.

%%%%%%%%%%%%%%%%%%%%%%%%%%%%%%%%%%%%%%%%
\DescribeMacro{\ifchilddocmanual}
The main file should be prepared as usual, see \secref{sec:include}.
However, the document body must make a distinction
between processing of an individual part and of the main document, e.g.:
%
\begin{center}
\begin{tabular}{l}
|\ifchilddocmanual|\\
|\input{\childdocname}|\\
|\||else|\\
\textit{document body with }|\input{|\textit{part}|}|\\
|\||fi|
\end{tabular}
\end{center}
%
The conditional |\ifchilddocmanual| is true whenever
a part to be included by |\input| is being compiled,
and the name of the part is stored in |\childdocname|.

%%%%%%%%%%%%%%%%%%%%%%%%%%%%%%%%%%%%%%%%
\DescribeMacro{\childdocby}
Each part to be included by |\input| should start with:
%
\begin{center}
\begin{tabular}{l}
|\input{childdoc.def}|\\
|\childdocby{|\textit{main}|}|\\
\end{tabular}
\end{center}
%
The directive |\childdocby| is similar to |\childdocof|
described in \secref{sec:include},
but the subsequent selection of content must be done manually.
To that end, both |\ifchilddoc| and |\ifchilddocmanual|
will be true upon processing of a part,
and the name of the part is stored in |\childdocname|.
Note that |\jobname| will be set to the filename of the current part
so that each part receives an individual |.aux| file
that does not interfere with the |.aux| file(s) of the main document.
This behaviour can be altered by the alternative form
|\childdocby[*]{|\textit{main}|}| (with a non-empty optional argument)
which uses the |.aux| file of the main document
by setting |\jobname| to \textit{main}.

%%%%%%%%%%%%%%%%%%%%%%%%%%%%%%%%%%%%%%%%%%%%%%%%%%%%%%%%%%%%%%%%%%%%%%%%%%%%%%%%
\subsection{Driver Development}
\label{sec:driver}

The \textsf{childdoc} mechanism can also be use for the development
of definition files such as \LaTeX{} styles or classes.
This case differs from the above setup with multiple parts
included by |\include| in that no |\includeonly| should be invoked.
This can be achieved by starting the include file
(before |\ProvidesPackage|) with:
%
\begin{center}
\begin{tabular}{l}
|\input{childdoc.def}|\\
|\childdocforward{|\textit{main}|}|\\
\end{tabular}
\end{center}
%
or alternatively with:
%
\begin{center}
\begin{tabular}{l}
|\input{childdoc.def}|\\
|\childdocby{|\textit{main}|}|\\
\end{tabular}
\end{center}
%
Both forms have slightly different effects as described above.
The main file is prepared as usual, see \secref{sec:include}.

%%%%%%%%%%%%%%%%%%%%%%%%%%%%%%%%%%%%%%%%%%%%%%%%%%%%%%%%%%%%%%%%%%%%%%%%%%%%%%%%
\subsection{Legacy Detection}
\label{sec:detection}

The directive |\childdocmain| in the main file can detect
whether the complete document or merely a child is to be compiled
even without using the directive |\childdocof|.
This method is deprecated because it is less robust
and there is no compelling reason to use it;
it is merely provided for backward compatibility
and it may be removed in future versions.

If the detection mechanism is to be used,
it is mandatory to correctly specify
the filename of the main file as the argument of |\childdocmain|:
%
\begin{center}
\begin{tabular}{l}
|\input{childdoc.def}|\\
|\childdocmain{|\textit{main}|}|\\
\end{tabular}
\end{center}
%
If |\jobname| does not match the argument \textit{main} of |\childdocmain|,
it is assumed that |\jobname| points to the child file to be compiled.
When using |\childdocmain| with the main file specified as argument,
it suffices to start a child file
with just |\input{|\textit{main}|}|
without loading of the package and using |\childdocof|.
If instead all processing is done
with the appropriate \textsf{childdoc} directives,
the argument of \textit{main} of |\childdocmain| can be empty.

An alternative version of the command line processing described
in \secref{sec:commandline} using the detection mechanism reads:
%
\begin{center}
|... -jobname "|\textit{target}|" "|[\textit{flags}]%
[|\def\jobname{|\textit{dest}|}|]|\input{|\textit{main}|}"|
\end{center}

%%%%%%%%%%%%%%%%%%%%%%%%%%%%%%%%%%%%%%%%%%%%%%%%%%%%%%%%%%%%%%%%%%%%%%%%%%%%%%%%
\subsection{Manual Code}
\label{sec:manual}

In case one cannot be certain whether the definitions file |childdoc.def|
is installed on the target \TeX{} distribution
and one prefers not to ship it,
it is conceivable to paste a few relevant commands into the sources.

To that end, drop all statements |\input{childdoc.def}|
and perform the replacements as outlined below.
Instead of |\childdocmain{|\textit{main}|}| add the following code
to the top of the main file:
%
\begin{center}
\begin{tabular}{l}
|\||ifdefined\childdocname\endinput\||fi\newif\ifchilddoc|\\
|\edef\childdocname{\scantokens\expandafter{\jobname\noexpand}}|\\
|\def\childdocmain{|\textit{main}|}\||ifx\childdocmain\childdocname\||else|\\
|\childdoctrue\includeonly{\childdocname}\let\jobname\childdocmain\||fi|\\
\end{tabular}
\end{center}
%
Instead of |\childdocof{|\textit{main}|}| just include the main file
at the top of each child file:
%
\begin{center}
|\input{|\textit{main}|}|
\end{center}
%
A simple redirection |\childdocforward{|\textit{dest}|}| is achieved by:
%
\begin{center}
|\def\jobname{|\textit{dest}|}\input{\jobname}|
\end{center}
%
The redirection with prefix
|\childdocforwardprefix[|\textit{prefix}|]{|\textit{dest}|}|
is accomplished by:
%
\begin{center}
\begin{tabular}{l}
|{\edef\jobname{\scantokens\expandafter{\jobname\noexpand}}|\\
|\def\redirectjob |\textit{prefix}|#1~~~{\gdef\jobname{|\textit{dest}|#1}}|\\
|\expandafter\redirectjob\jobname~~~}\input{\jobname}|
\end{tabular}
\end{center}

In an alternative approach,
child documents can be compiled by a specific command line
without additional code or specific definitions:
%
\begin{center}
|... -jobname "|\textit{target}|" "|[\textit{flags}]%
|\includeonly{|\textit{dest}|}\input{|\textit{main}|}"|
\end{center}
%

%%%%%%%%%%%%%%%%%%%%%%%%%%%%%%%%%%%%%%%%%%%%%%%%%%%%%%%%%%%%%%%%%%%%%%%%%%%%%%%%
%%%%%%%%%%%%%%%%%%%%%%%%%%%%%%%%%%%%%%%%%%%%%%%%%%%%%%%%%%%%%%%%%%%%%%%%%%%%%%%%
\section{Information}

%%%%%%%%%%%%%%%%%%%%%%%%%%%%%%%%%%%%%%%%%%%%%%%%%%%%%%%%%%%%%%%%%%%%%%%%%%%%%%%%
\subsection{Copyright}

Copyright \copyright{} 2017--2018 Niklas Beisert

This work may be distributed and/or modified under the
conditions of the \LaTeX{} Project Public License, either version 1.3
of this license or (at your option) any later version.
The latest version of this license is in
  \url{http://www.latex-project.org/lppl.txt}
and version 1.3 or later is part of all distributions of \LaTeX{}
version 2005/12/01 or later.

This work has the LPPL maintenance status `maintained'.

The Current Maintainer of this work is Niklas Beisert.

This work consists of the files |README.txt|, |childdoc.ins| and |childdoc.dtx|
as well as the derived files |childdoc.def|, |cdocsamp.tex|
with |cdocsch1.tex|, |cdocsch2.tex|, |cdocspt3.tex|, |cdocspt4.tex|,
|cdocsdrf.tex|, |cdocsfn1.tex|, |cdocsfn2.tex|
as well as |childdoc.pdf|.

%%%%%%%%%%%%%%%%%%%%%%%%%%%%%%%%%%%%%%%%%%%%%%%%%%%%%%%%%%%%%%%%%%%%%%%%%%%%%%%%
\subsection{Files and Installation}

The package consists of the files:
%
\begin{center}
\begin{tabular}{ll}
    |README.txt|   & readme file \\
    |childdoc.ins| & installation file \\
    |childdoc.dtx| & source file \\
    |childdoc.def| & definition file \\
    |cdocsamp.tex| & sample main file \\
    |cdocsch1.tex| & sample include file \\
    |cdocsch2.tex| & sample include file \\
    |cdocspt3.tex| & sample part file \\
    |cdocspt4.tex| & sample part file \\
    |cdocsdrf.tex| & sample redirection file \\
    |cdocsfn1.tex| & sample redirection file \\
    |cdocsfn2.tex| & sample redirection file \\
    |childdoc.pdf| & manual
\end{tabular}
\end{center}
%
The distribution consists of the files
|README.txt|, |childdoc.ins| and |childdoc.dtx|.
%
\begin{itemize}
\item
Run (pdf)\LaTeX{} on |childdoc.dtx|
to compile the manual |childdoc.pdf| (this file).
\item
Run \LaTeX{} on |childdoc.ins| to create the definitions file |childdoc.def|
and the sample |cdocsamp.tex| with include files
|cdocsch1.tex|, |cdocsch2.tex|, |cdocspt3.tex|, |cdocspt4.tex|,
|cdocsdrf.tex|, |cdocsfn1.tex|, |cdocsfn2.tex|.
Then copy the file |childdoc.def| to an appropriate directory of your \LaTeX{}
distribution, e.g.\ \textit{texmf-root}|/tex/latex/childdoc|.
\end{itemize}

%%%%%%%%%%%%%%%%%%%%%%%%%%%%%%%%%%%%%%%%%%%%%%%%%%%%%%%%%%%%%%%%%%%%%%%%%%%%%%%%
\subsection{Related CTAN Packages}

There are several other packages which offer a similar functionality:
%
\begin{itemize}
\item
The packages
\href{http://ctan.org/pkg/docmute}{\textsf{docmute}},
\href{http://ctan.org/pkg/includex}{\textsf{includex}} and
\href{http://ctan.org/pkg/standalone}{\textsf{standalone}}
provide commands to include only the document body of
a child file thus allowing both files to be compiled individually.
\item
The packages \href{http://ctan.org/pkg/subdocs}{\textsf{subdocs}}
and \href{http://ctan.org/pkg/subfiles}{\textsf{subfiles}}
provide structures in which the main and child documents can be
encapsulated and allowing them to be compiled individually.
The inclusion mechanism is different from the conventional |\include|.
\item
The package \href{http://ctan.org/pkg/combine}{\textsf{combine}}
is an elaborate solution to combine several documents into one.
\end{itemize}
%
See also the CTAN topic \href{http://ctan.org/topic/subdocs}{\textsf{subdocs}}
for further related packages.
The present package differs from the above solutions in that
a document structure constructed with the conventional |\include| mechanism
just needs two extra commands at the top of every file
such that all constituent files can be compiled individually.

%%%%%%%%%%%%%%%%%%%%%%%%%%%%%%%%%%%%%%%%%%%%%%%%%%%%%%%%%%%%%%%%%%%%%%%%%%%%%%%%
%\subsection{Feature Suggestions}
%
%The following is a list of features which may be useful for future
%versions of this package:
%%
%\begin{itemize}
%\item
%\ldots
%\end{itemize}

%%%%%%%%%%%%%%%%%%%%%%%%%%%%%%%%%%%%%%%%%%%%%%%%%%%%%%%%%%%%%%%%%%%%%%%%%%%%%%%%
\subsection{Revision History}

%%%%%%%%%%%%%%%%%%%%%%%%%%%%%%%%%%%%%%%%
\paragraph{v2.0:} 2018/12/30

\begin{itemize}
\item
immediate forward processing
\item
added |\childdocby| mechanism
\item
manual restructured
\end{itemize}

%%%%%%%%%%%%%%%%%%%%%%%%%%%%%%%%%%%%%%%%
\paragraph{v1.6:} 2018/01/17

\begin{itemize}
\item
application for development of include files
\item
corrections to manual
\end{itemize}

%%%%%%%%%%%%%%%%%%%%%%%%%%%%%%%%%%%%%%%%
\paragraph{v1.5:} 2017/05/21

\begin{itemize}
\item
more complete structuring introduced
\item
|\childdocof| introduced
\item
|\childdoc| renamed to |\childdocmain|
\item
|\childredirect| renamed to |\childdocforward| and |\childdocforwardprefix|
and functionality expanded
\end{itemize}

%%%%%%%%%%%%%%%%%%%%%%%%%%%%%%%%%%%%%%%%
\paragraph{v1.0:} 2017/04/27

\begin{itemize}
\item
manual and install package
\item
first version published on CTAN
\end{itemize}

%%%%%%%%%%%%%%%%%%%%%%%%%%%%%%%%%%%%%%%%
\paragraph{v0.6:} 2017/04/26

\begin{itemize}
\item
redirection mechanism added
\end{itemize}

%%%%%%%%%%%%%%%%%%%%%%%%%%%%%%%%%%%%%%%%
\paragraph{v0.5:} 2017/04/26

\begin{itemize}
\item
functionality in definition file
\end{itemize}


%%%%%%%%%%%%%%%%%%%%%%%%%%%%%%%%%%%%%%%%%%%%%%%%%%%%%%%%%%%%%%%%%%%%%%%%%%%%%%%%
%%%%%%%%%%%%%%%%%%%%%%%%%%%%%%%%%%%%%%%%%%%%%%%%%%%%%%%%%%%%%%%%%%%%%%%%%%%%%%%%
%%%%%%%%%%%%%%%%%%%%%%%%%%%%%%%%%%%%%%%%%%%%%%%%%%%%%%%%%%%%%%%%%%%%%%%%%%%%%%%%
\appendix

\settowidth\MacroIndent{\rmfamily\scriptsize 000\ }

 \DocInput{childdoc.dtx}

\end{document}
%</driver>
% \fi
%
% %%%%%%%%%%%%%%%%%%%%%%%%%%%%%%%%%%%%%%%%%%%%%%%%%%%%%%%%%%%%%%%%%%%%%%%%%%%%%%
% %%%%%%%%%%%%%%%%%%%%%%%%%%%%%%%%%%%%%%%%%%%%%%%%%%%%%%%%%%%%%%%%%%%%%%%%%%%%%%
% \section{Sample}
%\iffalse
%<*samplemain>
%\fi
%
% The following presents a sample document
% with two chapters, two parts, a title page,
% a compile flag as well as three forwarding files to set the flag.
% It consists of eight |.tex| files:
% \begin{center}
% \begin{tabular}{ll}
% |cdocsamp.tex|&main file\\
% |cdocsch1.tex|&include file for chapter 1\\
% |cdocsch2.tex|&include file for chapter 2\\
% |cdocspt3.tex|&include file for part 3\\
% |cdocspt4.tex|&include file for part 4\\
% |cdocsdrf.tex|&forwarding file for main file in draft mode\\
% |cdocsfi1.tex|&forwarding file for final version of chapter 1\\
% |cdocsfi2.tex|&forwarding file for final version of chapter 2\\
% \end{tabular}
% \end{center}
% Each of the eight files can be compiled directly by the \LaTeX{} compiler.
%
% %%%%%%%%%%%%%%%%%%%%%%%%%%%%%%%%%%%%%%
% \paragraph{Main File.}
%
% The main file is called |cdocsamp.tex|.
%
% Load the \textsf{childdoc} definitions and
% declare the filename for the main document:
%    \begin{macrocode}
\input{childdoc.def}
\childdocmain{}
%    \end{macrocode}

% Optional override for |\version| flag:
%    \begin{macrocode}
%%\ifchilddoc\else\providecommand{\version}{draft}\fi
%    \end{macrocode}

% Define the default values for the |\version| flag
% (|final| for the main file and |draft| for childs):
%    \begin{macrocode}
\ifchilddoc
\providecommand{\version}{draft}
\else
\providecommand{\version}{final}
\fi
%    \end{macrocode}

% Load the standard document class:
%    \begin{macrocode}
\documentclass[12pt]{article}
%    \end{macrocode}

% Start the document body:
%    \begin{macrocode}
\begin{document}
%    \end{macrocode}

% Declare a title page.
% Print title, part of document being processed and version flag:
%    \begin{macrocode}
\addtocounter{page}{-1}
\begin{center}
{\LARGE\bfseries{}childdoc example\par}
\vspace{1cm}
\ifchilddoc
\ifchilddocmanual part\else chapter\fi:
`\childdocname' of `\childdocjob'\par
\else
main document: `\childdocjob'\par
\fi
version: \version\par
\end{center}
\newpage
%    \end{macrocode}

% Manually include selected file,
% otherwise process as usual:
%    \begin{macrocode}
\ifchilddocmanual
\section*{part `\childdocname'}
\input{\childdocname}
\else
%    \end{macrocode}

% Include the two chapters:
%    \begin{macrocode}
\include{cdocsch1}
\include{cdocsch2}
%    \end{macrocode}

% Include the two parts unless only chapters should be displayed:
%    \begin{macrocode}
\ifchilddoc\else
\section{part three}
\input{cdocspt3}
\section{part four}
\input{cdocspt4}
\fi
%    \end{macrocode}

% Process as usual until here:
%    \begin{macrocode}
\fi
%    \end{macrocode}

% End of document body:
%    \begin{macrocode}
\end{document}
%    \end{macrocode}
%\iffalse
%</samplemain>
%\fi
%
% %%%%%%%%%%%%%%%%%%%%%%%%%%%%%%%%%%%%%%
% \paragraph{Chapter Include Files.}
%
% The include files are called |cdocsch1.tex| and |cdocsch2.tex|.
%
%\iffalse
%<*samplechap1|samplechap2>
%\fi

% Optional override for |\version| flag:
%    \begin{macrocode}
%%\providecommand{\version}{final}
%    \end{macrocode}

% Include the main document:
%    \begin{macrocode}
\input{childdoc.def}
\childdocof{cdocsamp}
%    \end{macrocode}

%\iffalse
%</samplechap1|samplechap2>
%\fi
%
%\iffalse
%<*samplechap1>
%\fi
% Some text for chapter 1:
%    \begin{macrocode}
\section{one}
some text in chapter one
%    \end{macrocode}

%\iffalse
%</samplechap1>
%\fi
% Some text for chapter 2:
%\iffalse
%<*samplechap2>
%\fi
%    \begin{macrocode}
\section{two}
more text in chapter two
%    \end{macrocode}

%\iffalse
%</samplechap2>
%\fi
%
% %%%%%%%%%%%%%%%%%%%%%%%%%%%%%%%%%%%%%%
% \paragraph{Part Include Files.}
%
% The include files are called |cdocspt3.tex| and |cdocspt4.tex|.
%
%\iffalse
%<*samplepart3|samplepart4>
%\fi

% Optional override for |\version| flag:
%    \begin{macrocode}
%%\providecommand{\version}{final}
%    \end{macrocode}

% Include the main document:
%    \begin{macrocode}
\input{childdoc.def}
\childdocby{cdocsamp}
%    \end{macrocode}

%\iffalse
%</samplepart3|samplepart4>
%\fi
%
%\iffalse
%<*samplepart3>
%\fi
% Some text for part 3:
%    \begin{macrocode}
some text in part three
%    \end{macrocode}

%\iffalse
%</samplepart3>
%\fi
% Some text for part 4:
%\iffalse
%<*samplepart4>
%\fi
%    \begin{macrocode}
more text in part four
%    \end{macrocode}

%\iffalse
%</samplepart4>
%\fi
%
% %%%%%%%%%%%%%%%%%%%%%%%%%%%%%%%%%%%%%%
% \paragraph{Forwarding for a Complete Draft.}
%
% The following forwarding file |cdocsdrf.tex|
% compiles the main document in draft mode:
%\iffalse
%<*sampledraft>
%\fi
%    \begin{macrocode}
\def\version{draft}
\input{childdoc.def}
\childdocforward{cdocsamp}
%    \end{macrocode}

%\iffalse
%</sampledraft>
%\fi
%
% %%%%%%%%%%%%%%%%%%%%%%%%%%%%%%%%%%%%%%
% \paragraph{Forwarding for Final Version of the Chapters.}
%
% The following forwarding files |cdocsfn1.tex| and |cdocsfn2.tex|
% (with identical content)
% compile the final versions of the child documents
% |cdocsch1.tex| and |cdocsch2.tex|, respectively:
%\iffalse
%<*samplefinal>
%\fi
%    \begin{macrocode}
\def\version{final}
\input{childdoc.def}
\childdocforwardprefix[cdocsamp]{cdocsfn}{cdocsch}
%    \end{macrocode}

%\iffalse
%</samplefinal>
%\fi
%
% %%%%%%%%%%%%%%%%%%%%%%%%%%%%%%%%%%%%%%
% \paragraph{Command Line Processing.}
%
% The following three command lines generate the output files
% |cdocscld|, |cdocscl1| and |cdocscl2|
% which should be identical to
% |cdocsdrf|, |cdocsch1| and |cdocsfn2|, respectively:
% \begin{center}
% \begin{tabular}{l}
% |latex -jobname cdocscld \|\\
% |  "\def\version{draft}\input{childdoc.def}\childdocforward{cdocsamp}"|\\
% |latex -jobname cdocscl1 \|\\
% |  "\input{childdoc.def}\childdocforward[cdocsamp]{cdocsch1}"|\\
% |latex -jobname cdocscl2 \|\\
% |  "\def\version{final}\input{childdoc.def}\childdocforward{cdocsch2}"|
% \end{tabular}
% \end{center}
% Note that the trailing backslash on each first line
% merely continues the input to the second line
% (for convenient cut ant paste).
% Furthermore, the command |latex| can be replaced by any
% of its alternative versions such as |pdflatex|.
%
% %%%%%%%%%%%%%%%%%%%%%%%%%%%%%%%%%%%%%%%%%%%%%%%%%%%%%%%%%%%%%%%%%%%%%%%%%%%%%%
% %%%%%%%%%%%%%%%%%%%%%%%%%%%%%%%%%%%%%%%%%%%%%%%%%%%%%%%%%%%%%%%%%%%%%%%%%%%%%%
% \section{Implementation}
%\iffalse
%<*package>
%\fi
%
% This section describes the definitions file |childdoc.def|.

% The definitions cannot be loaded using |\usepackage| or |\RequirePackage|
% which has a mechanism to prevent loading a style file more than once.
% When loading the definitions by means of |\input|
% multiple instances have to be prevented manually:
%\iffalse
%This code needs to be before the `\ProvidesFile' directive
%which is defined at the beginning of this file.
%Therefore it is also placed there and commented out here.
%</package>
%<*discard>
%\fi
%    \begin{macrocode}
\ifdefined\childdocmain\endinput\fi
%    \end{macrocode}
%\iffalse
%</discard>
%<*package>
%\fi
%
% \macro{\ifchilddoc}
% \macro{\ifchilddocmanual}
% The conditional |\ifchilddoc| tells whether a
% child (true) or main (false) document is being compiled.
% The conditional |\ifchilddocmanual| tells whether
% the |\includeonly| mechanism is used (false) or
% the selection of child files must be performed manually (true).
% The definitions initialise to false:
%    \begin{macrocode}
\newif\ifchilddoc
\newif\ifchilddocmanual
%    \end{macrocode}

% \macro{\childdocname}
% \macro{\childdocjob}
% The macro |\childdocname| stores the name of the main document
% to be compiled. The macro |\childdocjob| stores the name of
% the document on which the \LaTeX{} compiler was originally invoked.
% The content of |\jobname| cannot be compared
% to filenames specified in the source due to different catcodes.
% The following code rescans |\jobname|, stores the result
% in |\childdocname| and saves a copy in |\childdocjob|:
%    \begin{macrocode}
\edef\childdocname{\scantokens\expandafter{\jobname\noexpand}}
\let\childdocjob\childdocname
%    \end{macrocode}

% \macro{\childdocdisable}
% The macro |\childdocdisable| prevents the main file
% from being processed more than once.
% At this stage, the main document command |\childdocmain|
% is assumed to be called once again where it should do nothing.
% Any subsequent call to it should prevent
% a secondary processing of the main document
% It overwrites the forwarding commands
% |\childdocof| and |\childdocforward|
% with empty macros to prevent further inclusions of the main document:
%    \begin{macrocode}
\newcommand{\childdocdisable}
{
  \renewcommand{\childdocmain}[1]{\renewcommand{\childdocmain}[1]{\endinput}}
  \renewcommand{\childdocof}[1]{}
  \renewcommand{\childdocby}[2][]{}
  \renewcommand{\childdocforward}[2][]{}
  \renewcommand{\childdocdisable}{}
}
%    \end{macrocode}

% \macro{\childdocmain}
% The macro |\childdocmain| is to be called at the top of the main file
% with nothing or the main filename (without extension) as argument.
% First, it breaks loops.
% If the argument is not empty and does not match |\childdocname|
% (which is set by the first inclusion of |childdoc.def|),
% |\ifchilddoc| is set to true, |\includeonly| is applied to the child file
% and |\jobname| is set to the main file
% (for proper handling of |.aux| files):
%    \begin{macrocode}
\newcommand{\childdocmain}[1]
{
  \childdocdisable\childdocmain{}
  \if?#1?\else
    \begingroup
      \def\childdoctmp{#1}
      \ifx\childdoctmp\childdocname
        \def\childdoctmp{}
      \else
        \def\childdoctmp
        {
          \childdoctrue
          \includeonly{\childdocname}
          \def\childdocjob{#1}
          \def\jobname{#1}
        }
      \fi
      \expandafter
    \endgroup
    \childdoctmp
  \fi
}
%    \end{macrocode}

% \macro{\childdocof}
% The command |\childdocof| redirects
% compilation to the main file |#1|.
%    \begin{macrocode}
\newcommand{\childdocof}[1]
{
  \childdocdisable
  \childdoctrue
  \includeonly{\childdocname}
  \def\jobname{#1}
  \def\childdocjob{#1}
  \input{#1}
}
%    \end{macrocode}

% \macro{\childdocby}
% The command |\childdocby| ....
%    \begin{macrocode}
\newcommand{\childdocby}[2][]
{
  \childdocdisable
  \childdoctrue
  \childdocmanualtrue
  \if?#1?\else
    \def\jobname{#2}
  \fi
  \def\childdocjob{#2}
  \input{#2}
  \endinput
}
%    \end{macrocode}

% \macro{\childdocforward}
% The command |\childdocforward| redirects
% compilation to the main file or
% (if the optional argument is given) a child file.
% Parameters are set as if the main file
% or a child file starting with |\childdocof| was compiled.
% Then compilation is handed over to the main file:
%    \begin{macrocode}
\newcommand{\childdocforward}[2][]
{
  \begingroup
    \if?#1?
      \def\childdoctmp
      {
        \def\childdocname{#2}
        \def\childdocjob{#2}
        \def\jobname{#2}
        \input{#2}
        \endinput
      }
    \else
      \def\childdoctmp
      {
        \childdocdisable
        \def\childdocname{#2}
        \childdoctrue
        \includeonly{#2}
        \def\childdocjob{#1}
        \def\jobname{#1}
        \input{#1}
        \endinput
      }
    \fi
    \expandafter
  \endgroup
  \childdoctmp
}
%    \end{macrocode}

% \macro{\childdocforwardprefix}
% The command |\childdocforwardprefix| redirects
% compilation to the main or a child file by means of a pattern.
% The prefix |#1| in the current filename is replaced by |#2|
% and the suffix of the current filename is kept
% (it is assumed that the filename does not contain the substring `|~~~|'
% which is used as a delimiter).
% Compilation is handed over to the new file by |\childdocforward|:
%    \begin{macrocode}
\newcommand{\childdocforwardprefix}[3][]
{
  \begingroup
    \def\childdocextract #2##1~~~{\def\childdoctmp{\childdocforward[#1]{#3##1}}}
    \expandafter\childdocextract\childdocname~~~
    \expandafter
  \endgroup
  \childdoctmp
}
%    \end{macrocode}

% \macro{\childdoc}
% The deprecated macro |\childdoc| is a legacy version of |\childdocmain|:
%    \begin{macrocode}
\newcommand{\childdoc}{\childdocmain}
%    \end{macrocode}

% \macro{\childdocredirect}
% The deprecated macro |\childdocredirect| is a legacy version
% of |\childdocforward| and |\childdocforwardprefix|:
%    \begin{macrocode}
\newcommand{\childdocredirect}[2][]
{
  \begingroup
    \if?#1?
      \def\childdoctmp{\childdocforward{#2}}
    \else
      \def\childdoctmp{\childdocforwardprefix{#1}{#2}}
    \fi
    \expandafter
  \endgroup
  \childdoctmp
}
%    \end{macrocode}

%\iffalse
%</package>
%\fi
%
\endinput
|\\
|\childdocof{|\textit{main}|}|\\
\end{tabular}
\end{center}
at the top of every child file \textit{child}
which is included by |\include{|\textit{child}|}|
from within the main file
(or at least for those files to be compiled individually).
The argument \textit{main} must be the filename of the main file.

There are a couple of
considerations in setting up the main and child documents:

%%%%%%%%%%%%%%%%%%%%%%%%%%%%%%%%%%%%%%%%
\paragraph{Restrictions.}

Please note the following restrictions:
\begin{itemize}
\item
|\childdocmain| must be called with one argument \textit{main}
to ensure compatibility with earlier version of the package.
It must either be empty (|\childdocmain{}|)
or precisely match the filename of the main file in which it is specified.
See \secref{sec:detection} for further information.
\item
The filename \textit{main} must be specified without the |.tex| extension.
\item
The filename \textit{main} is case sensitive
(even in case-insensitive file systems)
due to internal string comparison.
\item
The argument \textit{main} should be fully expanded, it cannot be a macro.
\item
Subdirectories and special characters should be avoided in filenames.
\item
The command |\childdocmain{|\textit{main}|}| must be followed by a whitespace.
It should not be followed immediately by another command
or by a comment mark `|%|'.
This is because the \TeX{} parser reads the token immediately following
the argument of |\childdocmain| and puts it
at the beginning of every child section;
however, a white\-space is ignored.
\end{itemize}

%%%%%%%%%%%%%%%%%%%%%%%%%%%%%%%%%%%%%%%%
\paragraph{Content of Main File.}

It is advisable to place all content in the child files included by |\include|.
Any output contained in the main file will appear in all child documents
unless suppressed manually;
it cannot be suppressed automatically by the |\includeonly| directive
and thus should normally be avoided.
A method to include some content in the main file
by means of conditional processing is described in \secref{sec:conditional}.

%%%%%%%%%%%%%%%%%%%%%%%%%%%%%%%%%%%%%%%%
\paragraph{Page Numbering.}

When only a part of the document is compiled,
the appropriate numbering of pages
(as well as other status parameters)
is determined from the |.aux| files.
The latter contain information from previous passes.
However this information needs to propagate through
all intermediate child documents.
Therefore the page numbering in child documents may well
be inconsistent until the complete document is compiled at least once.

A useful (if unconventional) way to always ensure a consistent
page numbering is to restart the numbering in each child document
and denote the pages by `\textit{child}|.|\textit{page}'
where \textit{child} represents the chapter/section number of the child file.
This can be achieved by the command
|\numberwithin{page}{|\textit{child}|}|
of the \textsf{amsmath} package
where \textit{child} can be |chapter| or |section|
depending on the chosen structuring.
Alternatively, one can modify the macro |\thepage| appropriately
and reset the counter |page| at the start of each child file.

%%%%%%%%%%%%%%%%%%%%%%%%%%%%%%%%%%%%%%%%%%%%%%%%%%%%%%%%%%%%%%%%%%%%%%%%%%%%%%%%
\subsection{Conditional Processing}
\label{sec:conditional}

The package provides a mechanism to compile different versions
of a document. To customise the versions further some conditional processing
can come in handy to distinguish which version is being compiled.
The package provides two macros to describe the compilation context:

%%%%%%%%%%%%%%%%%%%%%%%%%%%%%%%%%%%%%%%%
\DescribeMacro{\ifchilddoc}
The conditional |\ifchilddoc| distinguishes between the compilation of
child documents and the main document:
%
\begin{center}
|\ifchilddoc |\textit{child-code}| |[|\||else |\textit{main-code}]| \||fi|
\end{center}

%%%%%%%%%%%%%%%%%%%%%%%%%%%%%%%%%%%%%%%%
\DescribeMacro{\childdocname}
\DescribeMacro{\childdocjob}
The macro |\childdocname| contains the filename (without extension)
of the main or child file being processed.
Note that |\childdocjob| will always contain the name of the main file.

%%%%%%%%%%%%%%%%%%%%%%%%%%%%%%%%%%%%%%%%
\paragraph{Title Page.}

Conditional processing can be used to include a title or banner page
in the main document when proper precautions are taken.
Importantly, the code in the main file should ensure that the page counter
(as well as other status parameters which are stored in the |.aux| files)
takes the same value after the conditional processing.
Otherwise the page numbers may take divergent values
depending on which part is compiled.

For example, a title page could be declared by:
%
\begin{center}
\begin{tabular}{l}
|\ifchilddoc\||else|\\
|\addtocounter{page}{-1}|\\
\textit{code for title page}\\
|\newpage|\\
|\||fi|
\end{tabular}
\end{center}
%
A banner page for the child documents can be generated by:
%
\begin{center}
\begin{tabular}{l}
|\ifchilddoc|\\
|\addtocounter{page}{-1}|\\
\textit{code for banner page}\\
|\newpage|\\
|\||fi|
\end{tabular}
\end{center}
%
Here one could write a message such as:
\begin{center}
|This is the part \childdocname{} of \childdocjob{}.|
\end{center}

%%%%%%%%%%%%%%%%%%%%%%%%%%%%%%%%%%%%%%%%%%%%%%%%%%%%%%%%%%%%%%%%%%%%%%%%%%%%%%%%
\subsection{Flags}
\label{sec:flags}

The package makes it easy to generate different versions
of the main or child documents.
To this end compilation flags can be defined
and assigned different default values.
They will be particularly useful in conjunction
with the forwarding mechanism described in \secref{sec:forward}.

For example, it may be useful to have a flag |\version|
which can be set to |draft| or |final|.
The document source will contain some conditional code
depending on the value of |\version|.
Suppose further, the flag should default to |final| for the main file
and to |draft| for child files
which is a natural assignment for editing the document.
This is achieved by placing the following code
in the preamble of the main document
(below the |\childdocmain| directive):
%
\begin{center}
\begin{tabular}{l}
|\ifchilddoc|\\
|\providecommand{\version}{draft}|\\
|\||else|\\
|\providecommand{\version}{final}|\\
|\||fi|
\end{tabular}
\end{center}
%
The definition by |\providecommand| makes sure
that previous definitions are not overwritten.
Further statements |\providecommand{\version}{...}|
can thus be added before the above code to override it.

For the main file, one might add a line
(between |\childdocmain| and the above block)
%
\begin{center}
|%\ifchilddoc\||else\providecommand{\version}{draft}\||fi|
\end{center}
%
which can be uncommented to produce a draft version.
Likewise one can add a line to the very top of a child file
(above the |\childdocof{|\textit{main}|}| directive)
%
\begin{center}
|%\providecommand{\version}{final}|
\end{center}
%
which can be uncommented to produce the final version of this child document.

%%%%%%%%%%%%%%%%%%%%%%%%%%%%%%%%%%%%%%%%%%%%%%%%%%%%%%%%%%%%%%%%%%%%%%%%%%%%%%%%
\subsection{Forwarding}
\label{sec:forward}

Different versions of the main or child documents
using compilation flags as described in \secref{sec:flags}
can be (permanently) stored in different files
for convenient compilation, viewing and distribution.
To this end, the package defines a command
to pass on compilation to a different file:

%%%%%%%%%%%%%%%%%%%%%%%%%%%%%%%%%%%%%%%%
\DescribeMacro{\childdocforward}
The command |\childdocforward| redirects processing to
another source file:
%
\begin{center}
\begin{tabular}{l}
|% \iffalse
%
% childdoc.dtx Copyright (C) 2017-2018 Niklas Beisert
%
% This work may be distributed and/or modified under the
% conditions of the LaTeX Project Public License, either version 1.3
% of this license or (at your option) any later version.
% The latest version of this license is in
%   http://www.latex-project.org/lppl.txt
% and version 1.3 or later is part of all distributions of LaTeX
% version 2005/12/01 or later.
%
% This work has the LPPL maintenance status `maintained'.
%
% The Current Maintainer of this work is Niklas Beisert.
%
% This work consists of the files childdoc.dtx and childdoc.ins
% and the derived files childdoc.def and cdocsamp.tex with
% cdocsch1.tex, cdocsch2.tex, cdocsdrf.tex, cdocsfn1.tex, cdocsfn2.tex.
%
%<package>\ifdefined\childdocmain\endinput\fi
%<package>\ProvidesFile{childdoc.def}[2018/12/30 v2.0 child document driver]
%<samplemain>\ProvidesFile{cdocsamp.tex}[2018/12/30 v2.0 sample for childdoc]
%<*driver>
%\ProvidesFile{childdoc.drv}[2018/12/30 v2.0 childdoc reference manual file]
\PassOptionsToClass{10pt,a4paper}{article}
\documentclass{ltxdoc}

\usepackage[margin=35mm]{geometry}
\usepackage{hyperref}
\usepackage{hyperxmp}
\usepackage[usenames]{color}

\hypersetup{colorlinks=true}
\hypersetup{pdfstartview=FitH}
\hypersetup{pdfpagemode=UseNone}
\hypersetup{pdfsource={}}
\hypersetup{pdflang={en-UK}}
\hypersetup{pdfcopyright={Copyright 2017-2018 Niklas Beisert.
  This work may be distributed and/or modified under the
  conditions of the LaTeX Project Public License, either version 1.3
  of this license or (at your option) any later version.}}
\hypersetup{pdflicenseurl={http://www.latex-project.org/lppl.txt}}
\hypersetup{pdfcontactaddress={ETH Zurich, ITP, HIT K,
  Wolfgang-Pauli-Strasse 27}}
\hypersetup{pdfcontactpostcode={8093}}
\hypersetup{pdfcontactcity={Zurich}}
\hypersetup{pdfcontactcountry={Switzerland}}
\hypersetup{pdfcontactemail={nbeisert@itp.phys.ethz.ch}}
\hypersetup{pdfcontacturl={http://people.phys.ethz.ch/\xmptilde nbeisert/}}

\newcommand{\secref}[1]{\hyperref[#1]{section \ref*{#1}}}

\parskip1ex
\parindent0pt
\let\olditemize\itemize
\def\itemize{\olditemize\parskip0pt}

\begin{document}

\title{The \textsf{childdoc} Package}
\hypersetup{pdftitle={The childdoc Package}}
\author{Niklas Beisert\\[2ex]
  Institut f\"ur Theoretische Physik\\
  Eidgen\"ossische Technische Hochschule Z\"urich\\
  Wolfgang-Pauli-Strasse 27, 8093 Z\"urich, Switzerland\\[1ex]
  \href{mailto:nbeisert@itp.phys.ethz.ch}
  {\texttt{nbeisert@itp.phys.ethz.ch}}}
\hypersetup{pdfauthor={Niklas Beisert}}
\hypersetup{pdfsubject={Manual for the LaTeX2e Package childdoc}}
\date{30 December 2018, \textsf{v2.0}}
\maketitle

\begin{abstract}\noindent
\textsf{childdoc} is a \LaTeXe{} package
that enables the direct compilation
of document sections included by |\include|
to individual files.
\end{abstract}

\begingroup
\parskip0ex
\tableofcontents
\endgroup

%%%%%%%%%%%%%%%%%%%%%%%%%%%%%%%%%%%%%%%%%%%%%%%%%%%%%%%%%%%%%%%%%%%%%%%%%%%%%%%%
%%%%%%%%%%%%%%%%%%%%%%%%%%%%%%%%%%%%%%%%%%%%%%%%%%%%%%%%%%%%%%%%%%%%%%%%%%%%%%%%
\section{Introduction}

\LaTeX{} provides a mechanism to structure a large document (such as a book)
into a main file and several child files (containing the chapters)
using the |\include| command.
This mechanism is beneficial for documents
which span hundreds of pages in order to
make the source file(s) more manageable.
Moreover, compilation can be restricted to
selected child files by means of the |\includeonly| command.
The latter feature can be used to reduce the compilation time while editing
(this was significantly more useful in the earlier days of \LaTeX{})
or to generate a smaller document which is easier to navigate.
Another application of |\includeonly| is to generate
documents consisting of selected parts of the complete document.

However, there are a few drawbacks of the plain |\include| mechanism:
\begin{itemize}
\item
The child files cannot be compiled on their own,
they can only be compiled via the main file.
A naive editing environment
(such as a text editor with an option
to have the current file processed by \LaTeX)
may require one to switch to the main file before compiling;
attempting to compile the child file produces errors.
\item
The main file must be modified (each time)
to adjust the |\includeonly| command
to the present needs. This easily leaves the main file in a messy state.
\item
The generated document will always carry the filename
of the main document. This is inconvenient if
several child files are to be compiled and
to be kept for distribution.
\end{itemize}

The present package provides a simple interface
to make child files individually compilable by \LaTeX{}.
Compiling a child file then has the same effect as compiling
the main file with an |\includeonly| command
to select the appropriate child.
Moreover the generated document will carry the name of the child
rather than the main file.
This resolves all three above issues.

This feature is meant to make the editing of books,
thesis documents and lecture notes somewhat more convenient.
However, the package can also be used efficiently for
composing a series of documents (such as exercise sheets)
which are typically distributed individually.
It then assists the author in generating the individual documents
(potentially in different versions)
as well as a document containing the collected series.
Another application is in developing style files
or other kinds of included material
where compilation of the style file could redirect
to a sample or test file.

%%%%%%%%%%%%%%%%%%%%%%%%%%%%%%%%%%%%%%%%%%%%%%%%%%%%%%%%%%%%%%%%%%%%%%%%%%%%%%%%
%%%%%%%%%%%%%%%%%%%%%%%%%%%%%%%%%%%%%%%%%%%%%%%%%%%%%%%%%%%%%%%%%%%%%%%%%%%%%%%%
\section{Usage}

First of all, the package \textsf{childdoc} is \emph{not} a standard
\LaTeXe{} |.sty| style file! Therefore it needs to be invoked in
a non-standard way.

%%%%%%%%%%%%%%%%%%%%%%%%%%%%%%%%%%%%%%%%%%%%%%%%%%%%%%%%%%%%%%%%%%%%%%%%%%%%%%%%
\subsection{Included Files}
\label{sec:include}

%%%%%%%%%%%%%%%%%%%%%%%%%%%%%%%%%%%%%%%%
\DescribeMacro{\childdocmain}
To use the package, add the commands
\begin{center}
\begin{tabular}{l}
|\input{childdoc.def}|\\
|\childdocmain{}|\\
\end{tabular}
\end{center}
at the very top of the main \LaTeX{} file,
in particular \emph{before} the |\documentclass| statement!
The argument of |\childdocmain| should be left empty
(but it must be present).

%%%%%%%%%%%%%%%%%%%%%%%%%%%%%%%%%%%%%%%%
\DescribeMacro{\childdocof}
Furthermore, add the commands
\begin{center}
\begin{tabular}{l}
|\input{childdoc.def}|\\
|\childdocof{|\textit{main}|}|\\
\end{tabular}
\end{center}
at the top of every child file \textit{child}
which is included by |\include{|\textit{child}|}|
from within the main file
(or at least for those files to be compiled individually).
The argument \textit{main} must be the filename of the main file.

There are a couple of
considerations in setting up the main and child documents:

%%%%%%%%%%%%%%%%%%%%%%%%%%%%%%%%%%%%%%%%
\paragraph{Restrictions.}

Please note the following restrictions:
\begin{itemize}
\item
|\childdocmain| must be called with one argument \textit{main}
to ensure compatibility with earlier version of the package.
It must either be empty (|\childdocmain{}|)
or precisely match the filename of the main file in which it is specified.
See \secref{sec:detection} for further information.
\item
The filename \textit{main} must be specified without the |.tex| extension.
\item
The filename \textit{main} is case sensitive
(even in case-insensitive file systems)
due to internal string comparison.
\item
The argument \textit{main} should be fully expanded, it cannot be a macro.
\item
Subdirectories and special characters should be avoided in filenames.
\item
The command |\childdocmain{|\textit{main}|}| must be followed by a whitespace.
It should not be followed immediately by another command
or by a comment mark `|%|'.
This is because the \TeX{} parser reads the token immediately following
the argument of |\childdocmain| and puts it
at the beginning of every child section;
however, a white\-space is ignored.
\end{itemize}

%%%%%%%%%%%%%%%%%%%%%%%%%%%%%%%%%%%%%%%%
\paragraph{Content of Main File.}

It is advisable to place all content in the child files included by |\include|.
Any output contained in the main file will appear in all child documents
unless suppressed manually;
it cannot be suppressed automatically by the |\includeonly| directive
and thus should normally be avoided.
A method to include some content in the main file
by means of conditional processing is described in \secref{sec:conditional}.

%%%%%%%%%%%%%%%%%%%%%%%%%%%%%%%%%%%%%%%%
\paragraph{Page Numbering.}

When only a part of the document is compiled,
the appropriate numbering of pages
(as well as other status parameters)
is determined from the |.aux| files.
The latter contain information from previous passes.
However this information needs to propagate through
all intermediate child documents.
Therefore the page numbering in child documents may well
be inconsistent until the complete document is compiled at least once.

A useful (if unconventional) way to always ensure a consistent
page numbering is to restart the numbering in each child document
and denote the pages by `\textit{child}|.|\textit{page}'
where \textit{child} represents the chapter/section number of the child file.
This can be achieved by the command
|\numberwithin{page}{|\textit{child}|}|
of the \textsf{amsmath} package
where \textit{child} can be |chapter| or |section|
depending on the chosen structuring.
Alternatively, one can modify the macro |\thepage| appropriately
and reset the counter |page| at the start of each child file.

%%%%%%%%%%%%%%%%%%%%%%%%%%%%%%%%%%%%%%%%%%%%%%%%%%%%%%%%%%%%%%%%%%%%%%%%%%%%%%%%
\subsection{Conditional Processing}
\label{sec:conditional}

The package provides a mechanism to compile different versions
of a document. To customise the versions further some conditional processing
can come in handy to distinguish which version is being compiled.
The package provides two macros to describe the compilation context:

%%%%%%%%%%%%%%%%%%%%%%%%%%%%%%%%%%%%%%%%
\DescribeMacro{\ifchilddoc}
The conditional |\ifchilddoc| distinguishes between the compilation of
child documents and the main document:
%
\begin{center}
|\ifchilddoc |\textit{child-code}| |[|\||else |\textit{main-code}]| \||fi|
\end{center}

%%%%%%%%%%%%%%%%%%%%%%%%%%%%%%%%%%%%%%%%
\DescribeMacro{\childdocname}
\DescribeMacro{\childdocjob}
The macro |\childdocname| contains the filename (without extension)
of the main or child file being processed.
Note that |\childdocjob| will always contain the name of the main file.

%%%%%%%%%%%%%%%%%%%%%%%%%%%%%%%%%%%%%%%%
\paragraph{Title Page.}

Conditional processing can be used to include a title or banner page
in the main document when proper precautions are taken.
Importantly, the code in the main file should ensure that the page counter
(as well as other status parameters which are stored in the |.aux| files)
takes the same value after the conditional processing.
Otherwise the page numbers may take divergent values
depending on which part is compiled.

For example, a title page could be declared by:
%
\begin{center}
\begin{tabular}{l}
|\ifchilddoc\||else|\\
|\addtocounter{page}{-1}|\\
\textit{code for title page}\\
|\newpage|\\
|\||fi|
\end{tabular}
\end{center}
%
A banner page for the child documents can be generated by:
%
\begin{center}
\begin{tabular}{l}
|\ifchilddoc|\\
|\addtocounter{page}{-1}|\\
\textit{code for banner page}\\
|\newpage|\\
|\||fi|
\end{tabular}
\end{center}
%
Here one could write a message such as:
\begin{center}
|This is the part \childdocname{} of \childdocjob{}.|
\end{center}

%%%%%%%%%%%%%%%%%%%%%%%%%%%%%%%%%%%%%%%%%%%%%%%%%%%%%%%%%%%%%%%%%%%%%%%%%%%%%%%%
\subsection{Flags}
\label{sec:flags}

The package makes it easy to generate different versions
of the main or child documents.
To this end compilation flags can be defined
and assigned different default values.
They will be particularly useful in conjunction
with the forwarding mechanism described in \secref{sec:forward}.

For example, it may be useful to have a flag |\version|
which can be set to |draft| or |final|.
The document source will contain some conditional code
depending on the value of |\version|.
Suppose further, the flag should default to |final| for the main file
and to |draft| for child files
which is a natural assignment for editing the document.
This is achieved by placing the following code
in the preamble of the main document
(below the |\childdocmain| directive):
%
\begin{center}
\begin{tabular}{l}
|\ifchilddoc|\\
|\providecommand{\version}{draft}|\\
|\||else|\\
|\providecommand{\version}{final}|\\
|\||fi|
\end{tabular}
\end{center}
%
The definition by |\providecommand| makes sure
that previous definitions are not overwritten.
Further statements |\providecommand{\version}{...}|
can thus be added before the above code to override it.

For the main file, one might add a line
(between |\childdocmain| and the above block)
%
\begin{center}
|%\ifchilddoc\||else\providecommand{\version}{draft}\||fi|
\end{center}
%
which can be uncommented to produce a draft version.
Likewise one can add a line to the very top of a child file
(above the |\childdocof{|\textit{main}|}| directive)
%
\begin{center}
|%\providecommand{\version}{final}|
\end{center}
%
which can be uncommented to produce the final version of this child document.

%%%%%%%%%%%%%%%%%%%%%%%%%%%%%%%%%%%%%%%%%%%%%%%%%%%%%%%%%%%%%%%%%%%%%%%%%%%%%%%%
\subsection{Forwarding}
\label{sec:forward}

Different versions of the main or child documents
using compilation flags as described in \secref{sec:flags}
can be (permanently) stored in different files
for convenient compilation, viewing and distribution.
To this end, the package defines a command
to pass on compilation to a different file:

%%%%%%%%%%%%%%%%%%%%%%%%%%%%%%%%%%%%%%%%
\DescribeMacro{\childdocforward}
The command |\childdocforward| redirects processing to
another source file:
%
\begin{center}
\begin{tabular}{l}
|\input{childdoc.def}|\\
|\childdocforward[|\textit{main}|]{|\textit{dest}|}|\\
\end{tabular}
\end{center}
%
The argument \textit{dest} is the destination file
(without extension).
It should be the main file or one of the child files.
Note that further \textsf{childdoc} directives
such as |\childdocof| and |\childdocforward|
in the indicated file will be processed in this form.
The optional argument \textit{main}
passes on directly to the main file \textit{main}
while pretending to compile the child \textit{dest}.
This form behaves as if \textit{dest}
issues |\childdocof{|\textit{main}|}| right away,
and no further \textsf{childdoc} directives will be processed.

%%%%%%%%%%%%%%%%%%%%%%%%%%%%%%%%%%%%%%%%
\DescribeMacro{\...prefix}
In the alternative form |\childdocforwardprefix|,
%
\begin{center}
\begin{tabular}{l}
|\input{childdoc.def}|\\
|\childdocforwardprefix[|\textit{main}|]{|\textit{prefix}|}{|\textit{dest}|}|
\end{tabular}
\end{center}
%
the destination file is determined by a pattern
depending on the current file:
To make this work, the current file must be called
`{\textit{prefix}\hspace{0.2em}\textit{suffix}}'
with \textit{prefix} matching precisely the argument.
Processing is then passed on to the file
`{\textit{dest}\hspace{0.2em}\textit{suffix}}'.
Surely, the same effect is achieved by
directly specifying the
argument `{\textit{dest}\hspace{0.2em}\textit{suffix}}'
in the first form.
However, that requires to set up a different file
for each child. With the alternative form of the command
all these files can have exactly the same content
which simplifies setting them up and maintaining them.

For example, the following file |draft.tex|
with a compilation flag |\version| as described in \secref{sec:flags}
compiles the main document as a draft:
%
\begin{center}
\begin{tabular}{l}
|\def\version{draft}|\\
|\input{childdoc.def}|\\
|\childdocforward{|\textit{main}|}|
\end{tabular}
\end{center}
%
Likewise, the following files |final|\textit{nn}|.tex|
compile the final version of the child document
|child|\textit{nn}|.tex|:
%
\begin{center}
\begin{tabular}{l}
|\def\version{final}|\\
|\input{childdoc.def}|\\
|\childdocforwardprefix{final}{child}|
\end{tabular}
\end{center}
%

Note that when several versions of a main file and/or of each child file
are to be generated, it may be convenient to set up a |Makefile| or
shell script to automatise the process.

%%%%%%%%%%%%%%%%%%%%%%%%%%%%%%%%%%%%%%%%%%%%%%%%%%%%%%%%%%%%%%%%%%%%%%%%%%%%%%%%
\subsection{Command Line Processing}
\label{sec:commandline}

The effect of redirection files can also be achieved by invoking
the \LaTeX{} compiler with a more elaborate command line.
Most conveniently this should be done as part
of a shell script or a |Makefile|.

When using \textsf{childdoc} in the main file, the following
command lines effectively perform a redirection
(note that depending on the shell being used,
backslashes may have to be doubled: `|\|' $\to$ `|\\|'):
%
\begin{center}
|... -jobname "|\textit{target}|" |\\|"|[\textit{flags}]%
|\input{childdoc.def}\childdocforward[|\textit{main}|]{|\textit{dest}|}"|
\end{center}
%
Here \textit{target} is the name of the output file,
\textit{main} is the name of the main file
and \textit{dest} is the name of the main or child file to be processed
(all filenames without extensions).
The optional argument \textit{main} can be omitted
if \textit{main} matches \textit{dest}.
Optionally, compilation \textit{flags} can be defined via |\def| commands.
This command line makes the \TeX{} engine believe
it is compiling the file \textit{target}
whose content is specified as the latter parameter.
The provided code then forwards the processing to
\textit{main} or \textit{dest} as described in \secref{sec:forward}.

%%%%%%%%%%%%%%%%%%%%%%%%%%%%%%%%%%%%%%%%%%%%%%%%%%%%%%%%%%%%%%%%%%%%%%%%%%%%%%%%
\subsection{Include by Input}
\label{sec:input}

Including child documents by |\include| has some restrictions by design.
Most notably, the content of a child document always occupies
its own set of pages; pages cannot be shared between child documents.
Usually, this behaviour makes perfect sense
because each child document contain an essential part of the document.
However, in some situations it may be desirable to compose
a document from a collection of parts
without having mandatory page breaks between then.
For this case, the package
provides a mechanism to include parts
by |\input| which can also be processed individually.
However, by construction this mechanism
requires manual handling of the content to be output.

%%%%%%%%%%%%%%%%%%%%%%%%%%%%%%%%%%%%%%%%
\DescribeMacro{\ifchilddocmanual}
The main file should be prepared as usual, see \secref{sec:include}.
However, the document body must make a distinction
between processing of an individual part and of the main document, e.g.:
%
\begin{center}
\begin{tabular}{l}
|\ifchilddocmanual|\\
|\input{\childdocname}|\\
|\||else|\\
\textit{document body with }|\input{|\textit{part}|}|\\
|\||fi|
\end{tabular}
\end{center}
%
The conditional |\ifchilddocmanual| is true whenever
a part to be included by |\input| is being compiled,
and the name of the part is stored in |\childdocname|.

%%%%%%%%%%%%%%%%%%%%%%%%%%%%%%%%%%%%%%%%
\DescribeMacro{\childdocby}
Each part to be included by |\input| should start with:
%
\begin{center}
\begin{tabular}{l}
|\input{childdoc.def}|\\
|\childdocby{|\textit{main}|}|\\
\end{tabular}
\end{center}
%
The directive |\childdocby| is similar to |\childdocof|
described in \secref{sec:include},
but the subsequent selection of content must be done manually.
To that end, both |\ifchilddoc| and |\ifchilddocmanual|
will be true upon processing of a part,
and the name of the part is stored in |\childdocname|.
Note that |\jobname| will be set to the filename of the current part
so that each part receives an individual |.aux| file
that does not interfere with the |.aux| file(s) of the main document.
This behaviour can be altered by the alternative form
|\childdocby[*]{|\textit{main}|}| (with a non-empty optional argument)
which uses the |.aux| file of the main document
by setting |\jobname| to \textit{main}.

%%%%%%%%%%%%%%%%%%%%%%%%%%%%%%%%%%%%%%%%%%%%%%%%%%%%%%%%%%%%%%%%%%%%%%%%%%%%%%%%
\subsection{Driver Development}
\label{sec:driver}

The \textsf{childdoc} mechanism can also be use for the development
of definition files such as \LaTeX{} styles or classes.
This case differs from the above setup with multiple parts
included by |\include| in that no |\includeonly| should be invoked.
This can be achieved by starting the include file
(before |\ProvidesPackage|) with:
%
\begin{center}
\begin{tabular}{l}
|\input{childdoc.def}|\\
|\childdocforward{|\textit{main}|}|\\
\end{tabular}
\end{center}
%
or alternatively with:
%
\begin{center}
\begin{tabular}{l}
|\input{childdoc.def}|\\
|\childdocby{|\textit{main}|}|\\
\end{tabular}
\end{center}
%
Both forms have slightly different effects as described above.
The main file is prepared as usual, see \secref{sec:include}.

%%%%%%%%%%%%%%%%%%%%%%%%%%%%%%%%%%%%%%%%%%%%%%%%%%%%%%%%%%%%%%%%%%%%%%%%%%%%%%%%
\subsection{Legacy Detection}
\label{sec:detection}

The directive |\childdocmain| in the main file can detect
whether the complete document or merely a child is to be compiled
even without using the directive |\childdocof|.
This method is deprecated because it is less robust
and there is no compelling reason to use it;
it is merely provided for backward compatibility
and it may be removed in future versions.

If the detection mechanism is to be used,
it is mandatory to correctly specify
the filename of the main file as the argument of |\childdocmain|:
%
\begin{center}
\begin{tabular}{l}
|\input{childdoc.def}|\\
|\childdocmain{|\textit{main}|}|\\
\end{tabular}
\end{center}
%
If |\jobname| does not match the argument \textit{main} of |\childdocmain|,
it is assumed that |\jobname| points to the child file to be compiled.
When using |\childdocmain| with the main file specified as argument,
it suffices to start a child file
with just |\input{|\textit{main}|}|
without loading of the package and using |\childdocof|.
If instead all processing is done
with the appropriate \textsf{childdoc} directives,
the argument of \textit{main} of |\childdocmain| can be empty.

An alternative version of the command line processing described
in \secref{sec:commandline} using the detection mechanism reads:
%
\begin{center}
|... -jobname "|\textit{target}|" "|[\textit{flags}]%
[|\def\jobname{|\textit{dest}|}|]|\input{|\textit{main}|}"|
\end{center}

%%%%%%%%%%%%%%%%%%%%%%%%%%%%%%%%%%%%%%%%%%%%%%%%%%%%%%%%%%%%%%%%%%%%%%%%%%%%%%%%
\subsection{Manual Code}
\label{sec:manual}

In case one cannot be certain whether the definitions file |childdoc.def|
is installed on the target \TeX{} distribution
and one prefers not to ship it,
it is conceivable to paste a few relevant commands into the sources.

To that end, drop all statements |\input{childdoc.def}|
and perform the replacements as outlined below.
Instead of |\childdocmain{|\textit{main}|}| add the following code
to the top of the main file:
%
\begin{center}
\begin{tabular}{l}
|\||ifdefined\childdocname\endinput\||fi\newif\ifchilddoc|\\
|\edef\childdocname{\scantokens\expandafter{\jobname\noexpand}}|\\
|\def\childdocmain{|\textit{main}|}\||ifx\childdocmain\childdocname\||else|\\
|\childdoctrue\includeonly{\childdocname}\let\jobname\childdocmain\||fi|\\
\end{tabular}
\end{center}
%
Instead of |\childdocof{|\textit{main}|}| just include the main file
at the top of each child file:
%
\begin{center}
|\input{|\textit{main}|}|
\end{center}
%
A simple redirection |\childdocforward{|\textit{dest}|}| is achieved by:
%
\begin{center}
|\def\jobname{|\textit{dest}|}\input{\jobname}|
\end{center}
%
The redirection with prefix
|\childdocforwardprefix[|\textit{prefix}|]{|\textit{dest}|}|
is accomplished by:
%
\begin{center}
\begin{tabular}{l}
|{\edef\jobname{\scantokens\expandafter{\jobname\noexpand}}|\\
|\def\redirectjob |\textit{prefix}|#1~~~{\gdef\jobname{|\textit{dest}|#1}}|\\
|\expandafter\redirectjob\jobname~~~}\input{\jobname}|
\end{tabular}
\end{center}

In an alternative approach,
child documents can be compiled by a specific command line
without additional code or specific definitions:
%
\begin{center}
|... -jobname "|\textit{target}|" "|[\textit{flags}]%
|\includeonly{|\textit{dest}|}\input{|\textit{main}|}"|
\end{center}
%

%%%%%%%%%%%%%%%%%%%%%%%%%%%%%%%%%%%%%%%%%%%%%%%%%%%%%%%%%%%%%%%%%%%%%%%%%%%%%%%%
%%%%%%%%%%%%%%%%%%%%%%%%%%%%%%%%%%%%%%%%%%%%%%%%%%%%%%%%%%%%%%%%%%%%%%%%%%%%%%%%
\section{Information}

%%%%%%%%%%%%%%%%%%%%%%%%%%%%%%%%%%%%%%%%%%%%%%%%%%%%%%%%%%%%%%%%%%%%%%%%%%%%%%%%
\subsection{Copyright}

Copyright \copyright{} 2017--2018 Niklas Beisert

This work may be distributed and/or modified under the
conditions of the \LaTeX{} Project Public License, either version 1.3
of this license or (at your option) any later version.
The latest version of this license is in
  \url{http://www.latex-project.org/lppl.txt}
and version 1.3 or later is part of all distributions of \LaTeX{}
version 2005/12/01 or later.

This work has the LPPL maintenance status `maintained'.

The Current Maintainer of this work is Niklas Beisert.

This work consists of the files |README.txt|, |childdoc.ins| and |childdoc.dtx|
as well as the derived files |childdoc.def|, |cdocsamp.tex|
with |cdocsch1.tex|, |cdocsch2.tex|, |cdocspt3.tex|, |cdocspt4.tex|,
|cdocsdrf.tex|, |cdocsfn1.tex|, |cdocsfn2.tex|
as well as |childdoc.pdf|.

%%%%%%%%%%%%%%%%%%%%%%%%%%%%%%%%%%%%%%%%%%%%%%%%%%%%%%%%%%%%%%%%%%%%%%%%%%%%%%%%
\subsection{Files and Installation}

The package consists of the files:
%
\begin{center}
\begin{tabular}{ll}
    |README.txt|   & readme file \\
    |childdoc.ins| & installation file \\
    |childdoc.dtx| & source file \\
    |childdoc.def| & definition file \\
    |cdocsamp.tex| & sample main file \\
    |cdocsch1.tex| & sample include file \\
    |cdocsch2.tex| & sample include file \\
    |cdocspt3.tex| & sample part file \\
    |cdocspt4.tex| & sample part file \\
    |cdocsdrf.tex| & sample redirection file \\
    |cdocsfn1.tex| & sample redirection file \\
    |cdocsfn2.tex| & sample redirection file \\
    |childdoc.pdf| & manual
\end{tabular}
\end{center}
%
The distribution consists of the files
|README.txt|, |childdoc.ins| and |childdoc.dtx|.
%
\begin{itemize}
\item
Run (pdf)\LaTeX{} on |childdoc.dtx|
to compile the manual |childdoc.pdf| (this file).
\item
Run \LaTeX{} on |childdoc.ins| to create the definitions file |childdoc.def|
and the sample |cdocsamp.tex| with include files
|cdocsch1.tex|, |cdocsch2.tex|, |cdocspt3.tex|, |cdocspt4.tex|,
|cdocsdrf.tex|, |cdocsfn1.tex|, |cdocsfn2.tex|.
Then copy the file |childdoc.def| to an appropriate directory of your \LaTeX{}
distribution, e.g.\ \textit{texmf-root}|/tex/latex/childdoc|.
\end{itemize}

%%%%%%%%%%%%%%%%%%%%%%%%%%%%%%%%%%%%%%%%%%%%%%%%%%%%%%%%%%%%%%%%%%%%%%%%%%%%%%%%
\subsection{Related CTAN Packages}

There are several other packages which offer a similar functionality:
%
\begin{itemize}
\item
The packages
\href{http://ctan.org/pkg/docmute}{\textsf{docmute}},
\href{http://ctan.org/pkg/includex}{\textsf{includex}} and
\href{http://ctan.org/pkg/standalone}{\textsf{standalone}}
provide commands to include only the document body of
a child file thus allowing both files to be compiled individually.
\item
The packages \href{http://ctan.org/pkg/subdocs}{\textsf{subdocs}}
and \href{http://ctan.org/pkg/subfiles}{\textsf{subfiles}}
provide structures in which the main and child documents can be
encapsulated and allowing them to be compiled individually.
The inclusion mechanism is different from the conventional |\include|.
\item
The package \href{http://ctan.org/pkg/combine}{\textsf{combine}}
is an elaborate solution to combine several documents into one.
\end{itemize}
%
See also the CTAN topic \href{http://ctan.org/topic/subdocs}{\textsf{subdocs}}
for further related packages.
The present package differs from the above solutions in that
a document structure constructed with the conventional |\include| mechanism
just needs two extra commands at the top of every file
such that all constituent files can be compiled individually.

%%%%%%%%%%%%%%%%%%%%%%%%%%%%%%%%%%%%%%%%%%%%%%%%%%%%%%%%%%%%%%%%%%%%%%%%%%%%%%%%
%\subsection{Feature Suggestions}
%
%The following is a list of features which may be useful for future
%versions of this package:
%%
%\begin{itemize}
%\item
%\ldots
%\end{itemize}

%%%%%%%%%%%%%%%%%%%%%%%%%%%%%%%%%%%%%%%%%%%%%%%%%%%%%%%%%%%%%%%%%%%%%%%%%%%%%%%%
\subsection{Revision History}

%%%%%%%%%%%%%%%%%%%%%%%%%%%%%%%%%%%%%%%%
\paragraph{v2.0:} 2018/12/30

\begin{itemize}
\item
immediate forward processing
\item
added |\childdocby| mechanism
\item
manual restructured
\end{itemize}

%%%%%%%%%%%%%%%%%%%%%%%%%%%%%%%%%%%%%%%%
\paragraph{v1.6:} 2018/01/17

\begin{itemize}
\item
application for development of include files
\item
corrections to manual
\end{itemize}

%%%%%%%%%%%%%%%%%%%%%%%%%%%%%%%%%%%%%%%%
\paragraph{v1.5:} 2017/05/21

\begin{itemize}
\item
more complete structuring introduced
\item
|\childdocof| introduced
\item
|\childdoc| renamed to |\childdocmain|
\item
|\childredirect| renamed to |\childdocforward| and |\childdocforwardprefix|
and functionality expanded
\end{itemize}

%%%%%%%%%%%%%%%%%%%%%%%%%%%%%%%%%%%%%%%%
\paragraph{v1.0:} 2017/04/27

\begin{itemize}
\item
manual and install package
\item
first version published on CTAN
\end{itemize}

%%%%%%%%%%%%%%%%%%%%%%%%%%%%%%%%%%%%%%%%
\paragraph{v0.6:} 2017/04/26

\begin{itemize}
\item
redirection mechanism added
\end{itemize}

%%%%%%%%%%%%%%%%%%%%%%%%%%%%%%%%%%%%%%%%
\paragraph{v0.5:} 2017/04/26

\begin{itemize}
\item
functionality in definition file
\end{itemize}


%%%%%%%%%%%%%%%%%%%%%%%%%%%%%%%%%%%%%%%%%%%%%%%%%%%%%%%%%%%%%%%%%%%%%%%%%%%%%%%%
%%%%%%%%%%%%%%%%%%%%%%%%%%%%%%%%%%%%%%%%%%%%%%%%%%%%%%%%%%%%%%%%%%%%%%%%%%%%%%%%
%%%%%%%%%%%%%%%%%%%%%%%%%%%%%%%%%%%%%%%%%%%%%%%%%%%%%%%%%%%%%%%%%%%%%%%%%%%%%%%%
\appendix

\settowidth\MacroIndent{\rmfamily\scriptsize 000\ }

 \DocInput{childdoc.dtx}

\end{document}
%</driver>
% \fi
%
% %%%%%%%%%%%%%%%%%%%%%%%%%%%%%%%%%%%%%%%%%%%%%%%%%%%%%%%%%%%%%%%%%%%%%%%%%%%%%%
% %%%%%%%%%%%%%%%%%%%%%%%%%%%%%%%%%%%%%%%%%%%%%%%%%%%%%%%%%%%%%%%%%%%%%%%%%%%%%%
% \section{Sample}
%\iffalse
%<*samplemain>
%\fi
%
% The following presents a sample document
% with two chapters, two parts, a title page,
% a compile flag as well as three forwarding files to set the flag.
% It consists of eight |.tex| files:
% \begin{center}
% \begin{tabular}{ll}
% |cdocsamp.tex|&main file\\
% |cdocsch1.tex|&include file for chapter 1\\
% |cdocsch2.tex|&include file for chapter 2\\
% |cdocspt3.tex|&include file for part 3\\
% |cdocspt4.tex|&include file for part 4\\
% |cdocsdrf.tex|&forwarding file for main file in draft mode\\
% |cdocsfi1.tex|&forwarding file for final version of chapter 1\\
% |cdocsfi2.tex|&forwarding file for final version of chapter 2\\
% \end{tabular}
% \end{center}
% Each of the eight files can be compiled directly by the \LaTeX{} compiler.
%
% %%%%%%%%%%%%%%%%%%%%%%%%%%%%%%%%%%%%%%
% \paragraph{Main File.}
%
% The main file is called |cdocsamp.tex|.
%
% Load the \textsf{childdoc} definitions and
% declare the filename for the main document:
%    \begin{macrocode}
\input{childdoc.def}
\childdocmain{}
%    \end{macrocode}

% Optional override for |\version| flag:
%    \begin{macrocode}
%%\ifchilddoc\else\providecommand{\version}{draft}\fi
%    \end{macrocode}

% Define the default values for the |\version| flag
% (|final| for the main file and |draft| for childs):
%    \begin{macrocode}
\ifchilddoc
\providecommand{\version}{draft}
\else
\providecommand{\version}{final}
\fi
%    \end{macrocode}

% Load the standard document class:
%    \begin{macrocode}
\documentclass[12pt]{article}
%    \end{macrocode}

% Start the document body:
%    \begin{macrocode}
\begin{document}
%    \end{macrocode}

% Declare a title page.
% Print title, part of document being processed and version flag:
%    \begin{macrocode}
\addtocounter{page}{-1}
\begin{center}
{\LARGE\bfseries{}childdoc example\par}
\vspace{1cm}
\ifchilddoc
\ifchilddocmanual part\else chapter\fi:
`\childdocname' of `\childdocjob'\par
\else
main document: `\childdocjob'\par
\fi
version: \version\par
\end{center}
\newpage
%    \end{macrocode}

% Manually include selected file,
% otherwise process as usual:
%    \begin{macrocode}
\ifchilddocmanual
\section*{part `\childdocname'}
\input{\childdocname}
\else
%    \end{macrocode}

% Include the two chapters:
%    \begin{macrocode}
\include{cdocsch1}
\include{cdocsch2}
%    \end{macrocode}

% Include the two parts unless only chapters should be displayed:
%    \begin{macrocode}
\ifchilddoc\else
\section{part three}
\input{cdocspt3}
\section{part four}
\input{cdocspt4}
\fi
%    \end{macrocode}

% Process as usual until here:
%    \begin{macrocode}
\fi
%    \end{macrocode}

% End of document body:
%    \begin{macrocode}
\end{document}
%    \end{macrocode}
%\iffalse
%</samplemain>
%\fi
%
% %%%%%%%%%%%%%%%%%%%%%%%%%%%%%%%%%%%%%%
% \paragraph{Chapter Include Files.}
%
% The include files are called |cdocsch1.tex| and |cdocsch2.tex|.
%
%\iffalse
%<*samplechap1|samplechap2>
%\fi

% Optional override for |\version| flag:
%    \begin{macrocode}
%%\providecommand{\version}{final}
%    \end{macrocode}

% Include the main document:
%    \begin{macrocode}
\input{childdoc.def}
\childdocof{cdocsamp}
%    \end{macrocode}

%\iffalse
%</samplechap1|samplechap2>
%\fi
%
%\iffalse
%<*samplechap1>
%\fi
% Some text for chapter 1:
%    \begin{macrocode}
\section{one}
some text in chapter one
%    \end{macrocode}

%\iffalse
%</samplechap1>
%\fi
% Some text for chapter 2:
%\iffalse
%<*samplechap2>
%\fi
%    \begin{macrocode}
\section{two}
more text in chapter two
%    \end{macrocode}

%\iffalse
%</samplechap2>
%\fi
%
% %%%%%%%%%%%%%%%%%%%%%%%%%%%%%%%%%%%%%%
% \paragraph{Part Include Files.}
%
% The include files are called |cdocspt3.tex| and |cdocspt4.tex|.
%
%\iffalse
%<*samplepart3|samplepart4>
%\fi

% Optional override for |\version| flag:
%    \begin{macrocode}
%%\providecommand{\version}{final}
%    \end{macrocode}

% Include the main document:
%    \begin{macrocode}
\input{childdoc.def}
\childdocby{cdocsamp}
%    \end{macrocode}

%\iffalse
%</samplepart3|samplepart4>
%\fi
%
%\iffalse
%<*samplepart3>
%\fi
% Some text for part 3:
%    \begin{macrocode}
some text in part three
%    \end{macrocode}

%\iffalse
%</samplepart3>
%\fi
% Some text for part 4:
%\iffalse
%<*samplepart4>
%\fi
%    \begin{macrocode}
more text in part four
%    \end{macrocode}

%\iffalse
%</samplepart4>
%\fi
%
% %%%%%%%%%%%%%%%%%%%%%%%%%%%%%%%%%%%%%%
% \paragraph{Forwarding for a Complete Draft.}
%
% The following forwarding file |cdocsdrf.tex|
% compiles the main document in draft mode:
%\iffalse
%<*sampledraft>
%\fi
%    \begin{macrocode}
\def\version{draft}
\input{childdoc.def}
\childdocforward{cdocsamp}
%    \end{macrocode}

%\iffalse
%</sampledraft>
%\fi
%
% %%%%%%%%%%%%%%%%%%%%%%%%%%%%%%%%%%%%%%
% \paragraph{Forwarding for Final Version of the Chapters.}
%
% The following forwarding files |cdocsfn1.tex| and |cdocsfn2.tex|
% (with identical content)
% compile the final versions of the child documents
% |cdocsch1.tex| and |cdocsch2.tex|, respectively:
%\iffalse
%<*samplefinal>
%\fi
%    \begin{macrocode}
\def\version{final}
\input{childdoc.def}
\childdocforwardprefix[cdocsamp]{cdocsfn}{cdocsch}
%    \end{macrocode}

%\iffalse
%</samplefinal>
%\fi
%
% %%%%%%%%%%%%%%%%%%%%%%%%%%%%%%%%%%%%%%
% \paragraph{Command Line Processing.}
%
% The following three command lines generate the output files
% |cdocscld|, |cdocscl1| and |cdocscl2|
% which should be identical to
% |cdocsdrf|, |cdocsch1| and |cdocsfn2|, respectively:
% \begin{center}
% \begin{tabular}{l}
% |latex -jobname cdocscld \|\\
% |  "\def\version{draft}\input{childdoc.def}\childdocforward{cdocsamp}"|\\
% |latex -jobname cdocscl1 \|\\
% |  "\input{childdoc.def}\childdocforward[cdocsamp]{cdocsch1}"|\\
% |latex -jobname cdocscl2 \|\\
% |  "\def\version{final}\input{childdoc.def}\childdocforward{cdocsch2}"|
% \end{tabular}
% \end{center}
% Note that the trailing backslash on each first line
% merely continues the input to the second line
% (for convenient cut ant paste).
% Furthermore, the command |latex| can be replaced by any
% of its alternative versions such as |pdflatex|.
%
% %%%%%%%%%%%%%%%%%%%%%%%%%%%%%%%%%%%%%%%%%%%%%%%%%%%%%%%%%%%%%%%%%%%%%%%%%%%%%%
% %%%%%%%%%%%%%%%%%%%%%%%%%%%%%%%%%%%%%%%%%%%%%%%%%%%%%%%%%%%%%%%%%%%%%%%%%%%%%%
% \section{Implementation}
%\iffalse
%<*package>
%\fi
%
% This section describes the definitions file |childdoc.def|.

% The definitions cannot be loaded using |\usepackage| or |\RequirePackage|
% which has a mechanism to prevent loading a style file more than once.
% When loading the definitions by means of |\input|
% multiple instances have to be prevented manually:
%\iffalse
%This code needs to be before the `\ProvidesFile' directive
%which is defined at the beginning of this file.
%Therefore it is also placed there and commented out here.
%</package>
%<*discard>
%\fi
%    \begin{macrocode}
\ifdefined\childdocmain\endinput\fi
%    \end{macrocode}
%\iffalse
%</discard>
%<*package>
%\fi
%
% \macro{\ifchilddoc}
% \macro{\ifchilddocmanual}
% The conditional |\ifchilddoc| tells whether a
% child (true) or main (false) document is being compiled.
% The conditional |\ifchilddocmanual| tells whether
% the |\includeonly| mechanism is used (false) or
% the selection of child files must be performed manually (true).
% The definitions initialise to false:
%    \begin{macrocode}
\newif\ifchilddoc
\newif\ifchilddocmanual
%    \end{macrocode}

% \macro{\childdocname}
% \macro{\childdocjob}
% The macro |\childdocname| stores the name of the main document
% to be compiled. The macro |\childdocjob| stores the name of
% the document on which the \LaTeX{} compiler was originally invoked.
% The content of |\jobname| cannot be compared
% to filenames specified in the source due to different catcodes.
% The following code rescans |\jobname|, stores the result
% in |\childdocname| and saves a copy in |\childdocjob|:
%    \begin{macrocode}
\edef\childdocname{\scantokens\expandafter{\jobname\noexpand}}
\let\childdocjob\childdocname
%    \end{macrocode}

% \macro{\childdocdisable}
% The macro |\childdocdisable| prevents the main file
% from being processed more than once.
% At this stage, the main document command |\childdocmain|
% is assumed to be called once again where it should do nothing.
% Any subsequent call to it should prevent
% a secondary processing of the main document
% It overwrites the forwarding commands
% |\childdocof| and |\childdocforward|
% with empty macros to prevent further inclusions of the main document:
%    \begin{macrocode}
\newcommand{\childdocdisable}
{
  \renewcommand{\childdocmain}[1]{\renewcommand{\childdocmain}[1]{\endinput}}
  \renewcommand{\childdocof}[1]{}
  \renewcommand{\childdocby}[2][]{}
  \renewcommand{\childdocforward}[2][]{}
  \renewcommand{\childdocdisable}{}
}
%    \end{macrocode}

% \macro{\childdocmain}
% The macro |\childdocmain| is to be called at the top of the main file
% with nothing or the main filename (without extension) as argument.
% First, it breaks loops.
% If the argument is not empty and does not match |\childdocname|
% (which is set by the first inclusion of |childdoc.def|),
% |\ifchilddoc| is set to true, |\includeonly| is applied to the child file
% and |\jobname| is set to the main file
% (for proper handling of |.aux| files):
%    \begin{macrocode}
\newcommand{\childdocmain}[1]
{
  \childdocdisable\childdocmain{}
  \if?#1?\else
    \begingroup
      \def\childdoctmp{#1}
      \ifx\childdoctmp\childdocname
        \def\childdoctmp{}
      \else
        \def\childdoctmp
        {
          \childdoctrue
          \includeonly{\childdocname}
          \def\childdocjob{#1}
          \def\jobname{#1}
        }
      \fi
      \expandafter
    \endgroup
    \childdoctmp
  \fi
}
%    \end{macrocode}

% \macro{\childdocof}
% The command |\childdocof| redirects
% compilation to the main file |#1|.
%    \begin{macrocode}
\newcommand{\childdocof}[1]
{
  \childdocdisable
  \childdoctrue
  \includeonly{\childdocname}
  \def\jobname{#1}
  \def\childdocjob{#1}
  \input{#1}
}
%    \end{macrocode}

% \macro{\childdocby}
% The command |\childdocby| ....
%    \begin{macrocode}
\newcommand{\childdocby}[2][]
{
  \childdocdisable
  \childdoctrue
  \childdocmanualtrue
  \if?#1?\else
    \def\jobname{#2}
  \fi
  \def\childdocjob{#2}
  \input{#2}
  \endinput
}
%    \end{macrocode}

% \macro{\childdocforward}
% The command |\childdocforward| redirects
% compilation to the main file or
% (if the optional argument is given) a child file.
% Parameters are set as if the main file
% or a child file starting with |\childdocof| was compiled.
% Then compilation is handed over to the main file:
%    \begin{macrocode}
\newcommand{\childdocforward}[2][]
{
  \begingroup
    \if?#1?
      \def\childdoctmp
      {
        \def\childdocname{#2}
        \def\childdocjob{#2}
        \def\jobname{#2}
        \input{#2}
        \endinput
      }
    \else
      \def\childdoctmp
      {
        \childdocdisable
        \def\childdocname{#2}
        \childdoctrue
        \includeonly{#2}
        \def\childdocjob{#1}
        \def\jobname{#1}
        \input{#1}
        \endinput
      }
    \fi
    \expandafter
  \endgroup
  \childdoctmp
}
%    \end{macrocode}

% \macro{\childdocforwardprefix}
% The command |\childdocforwardprefix| redirects
% compilation to the main or a child file by means of a pattern.
% The prefix |#1| in the current filename is replaced by |#2|
% and the suffix of the current filename is kept
% (it is assumed that the filename does not contain the substring `|~~~|'
% which is used as a delimiter).
% Compilation is handed over to the new file by |\childdocforward|:
%    \begin{macrocode}
\newcommand{\childdocforwardprefix}[3][]
{
  \begingroup
    \def\childdocextract #2##1~~~{\def\childdoctmp{\childdocforward[#1]{#3##1}}}
    \expandafter\childdocextract\childdocname~~~
    \expandafter
  \endgroup
  \childdoctmp
}
%    \end{macrocode}

% \macro{\childdoc}
% The deprecated macro |\childdoc| is a legacy version of |\childdocmain|:
%    \begin{macrocode}
\newcommand{\childdoc}{\childdocmain}
%    \end{macrocode}

% \macro{\childdocredirect}
% The deprecated macro |\childdocredirect| is a legacy version
% of |\childdocforward| and |\childdocforwardprefix|:
%    \begin{macrocode}
\newcommand{\childdocredirect}[2][]
{
  \begingroup
    \if?#1?
      \def\childdoctmp{\childdocforward{#2}}
    \else
      \def\childdoctmp{\childdocforwardprefix{#1}{#2}}
    \fi
    \expandafter
  \endgroup
  \childdoctmp
}
%    \end{macrocode}

%\iffalse
%</package>
%\fi
%
\endinput
|\\
|\childdocforward[|\textit{main}|]{|\textit{dest}|}|\\
\end{tabular}
\end{center}
%
The argument \textit{dest} is the destination file
(without extension).
It should be the main file or one of the child files.
Note that further \textsf{childdoc} directives
such as |\childdocof| and |\childdocforward|
in the indicated file will be processed in this form.
The optional argument \textit{main}
passes on directly to the main file \textit{main}
while pretending to compile the child \textit{dest}.
This form behaves as if \textit{dest}
issues |\childdocof{|\textit{main}|}| right away,
and no further \textsf{childdoc} directives will be processed.

%%%%%%%%%%%%%%%%%%%%%%%%%%%%%%%%%%%%%%%%
\DescribeMacro{\...prefix}
In the alternative form |\childdocforwardprefix|,
%
\begin{center}
\begin{tabular}{l}
|% \iffalse
%
% childdoc.dtx Copyright (C) 2017-2018 Niklas Beisert
%
% This work may be distributed and/or modified under the
% conditions of the LaTeX Project Public License, either version 1.3
% of this license or (at your option) any later version.
% The latest version of this license is in
%   http://www.latex-project.org/lppl.txt
% and version 1.3 or later is part of all distributions of LaTeX
% version 2005/12/01 or later.
%
% This work has the LPPL maintenance status `maintained'.
%
% The Current Maintainer of this work is Niklas Beisert.
%
% This work consists of the files childdoc.dtx and childdoc.ins
% and the derived files childdoc.def and cdocsamp.tex with
% cdocsch1.tex, cdocsch2.tex, cdocsdrf.tex, cdocsfn1.tex, cdocsfn2.tex.
%
%<package>\ifdefined\childdocmain\endinput\fi
%<package>\ProvidesFile{childdoc.def}[2018/12/30 v2.0 child document driver]
%<samplemain>\ProvidesFile{cdocsamp.tex}[2018/12/30 v2.0 sample for childdoc]
%<*driver>
%\ProvidesFile{childdoc.drv}[2018/12/30 v2.0 childdoc reference manual file]
\PassOptionsToClass{10pt,a4paper}{article}
\documentclass{ltxdoc}

\usepackage[margin=35mm]{geometry}
\usepackage{hyperref}
\usepackage{hyperxmp}
\usepackage[usenames]{color}

\hypersetup{colorlinks=true}
\hypersetup{pdfstartview=FitH}
\hypersetup{pdfpagemode=UseNone}
\hypersetup{pdfsource={}}
\hypersetup{pdflang={en-UK}}
\hypersetup{pdfcopyright={Copyright 2017-2018 Niklas Beisert.
  This work may be distributed and/or modified under the
  conditions of the LaTeX Project Public License, either version 1.3
  of this license or (at your option) any later version.}}
\hypersetup{pdflicenseurl={http://www.latex-project.org/lppl.txt}}
\hypersetup{pdfcontactaddress={ETH Zurich, ITP, HIT K,
  Wolfgang-Pauli-Strasse 27}}
\hypersetup{pdfcontactpostcode={8093}}
\hypersetup{pdfcontactcity={Zurich}}
\hypersetup{pdfcontactcountry={Switzerland}}
\hypersetup{pdfcontactemail={nbeisert@itp.phys.ethz.ch}}
\hypersetup{pdfcontacturl={http://people.phys.ethz.ch/\xmptilde nbeisert/}}

\newcommand{\secref}[1]{\hyperref[#1]{section \ref*{#1}}}

\parskip1ex
\parindent0pt
\let\olditemize\itemize
\def\itemize{\olditemize\parskip0pt}

\begin{document}

\title{The \textsf{childdoc} Package}
\hypersetup{pdftitle={The childdoc Package}}
\author{Niklas Beisert\\[2ex]
  Institut f\"ur Theoretische Physik\\
  Eidgen\"ossische Technische Hochschule Z\"urich\\
  Wolfgang-Pauli-Strasse 27, 8093 Z\"urich, Switzerland\\[1ex]
  \href{mailto:nbeisert@itp.phys.ethz.ch}
  {\texttt{nbeisert@itp.phys.ethz.ch}}}
\hypersetup{pdfauthor={Niklas Beisert}}
\hypersetup{pdfsubject={Manual for the LaTeX2e Package childdoc}}
\date{30 December 2018, \textsf{v2.0}}
\maketitle

\begin{abstract}\noindent
\textsf{childdoc} is a \LaTeXe{} package
that enables the direct compilation
of document sections included by |\include|
to individual files.
\end{abstract}

\begingroup
\parskip0ex
\tableofcontents
\endgroup

%%%%%%%%%%%%%%%%%%%%%%%%%%%%%%%%%%%%%%%%%%%%%%%%%%%%%%%%%%%%%%%%%%%%%%%%%%%%%%%%
%%%%%%%%%%%%%%%%%%%%%%%%%%%%%%%%%%%%%%%%%%%%%%%%%%%%%%%%%%%%%%%%%%%%%%%%%%%%%%%%
\section{Introduction}

\LaTeX{} provides a mechanism to structure a large document (such as a book)
into a main file and several child files (containing the chapters)
using the |\include| command.
This mechanism is beneficial for documents
which span hundreds of pages in order to
make the source file(s) more manageable.
Moreover, compilation can be restricted to
selected child files by means of the |\includeonly| command.
The latter feature can be used to reduce the compilation time while editing
(this was significantly more useful in the earlier days of \LaTeX{})
or to generate a smaller document which is easier to navigate.
Another application of |\includeonly| is to generate
documents consisting of selected parts of the complete document.

However, there are a few drawbacks of the plain |\include| mechanism:
\begin{itemize}
\item
The child files cannot be compiled on their own,
they can only be compiled via the main file.
A naive editing environment
(such as a text editor with an option
to have the current file processed by \LaTeX)
may require one to switch to the main file before compiling;
attempting to compile the child file produces errors.
\item
The main file must be modified (each time)
to adjust the |\includeonly| command
to the present needs. This easily leaves the main file in a messy state.
\item
The generated document will always carry the filename
of the main document. This is inconvenient if
several child files are to be compiled and
to be kept for distribution.
\end{itemize}

The present package provides a simple interface
to make child files individually compilable by \LaTeX{}.
Compiling a child file then has the same effect as compiling
the main file with an |\includeonly| command
to select the appropriate child.
Moreover the generated document will carry the name of the child
rather than the main file.
This resolves all three above issues.

This feature is meant to make the editing of books,
thesis documents and lecture notes somewhat more convenient.
However, the package can also be used efficiently for
composing a series of documents (such as exercise sheets)
which are typically distributed individually.
It then assists the author in generating the individual documents
(potentially in different versions)
as well as a document containing the collected series.
Another application is in developing style files
or other kinds of included material
where compilation of the style file could redirect
to a sample or test file.

%%%%%%%%%%%%%%%%%%%%%%%%%%%%%%%%%%%%%%%%%%%%%%%%%%%%%%%%%%%%%%%%%%%%%%%%%%%%%%%%
%%%%%%%%%%%%%%%%%%%%%%%%%%%%%%%%%%%%%%%%%%%%%%%%%%%%%%%%%%%%%%%%%%%%%%%%%%%%%%%%
\section{Usage}

First of all, the package \textsf{childdoc} is \emph{not} a standard
\LaTeXe{} |.sty| style file! Therefore it needs to be invoked in
a non-standard way.

%%%%%%%%%%%%%%%%%%%%%%%%%%%%%%%%%%%%%%%%%%%%%%%%%%%%%%%%%%%%%%%%%%%%%%%%%%%%%%%%
\subsection{Included Files}
\label{sec:include}

%%%%%%%%%%%%%%%%%%%%%%%%%%%%%%%%%%%%%%%%
\DescribeMacro{\childdocmain}
To use the package, add the commands
\begin{center}
\begin{tabular}{l}
|\input{childdoc.def}|\\
|\childdocmain{}|\\
\end{tabular}
\end{center}
at the very top of the main \LaTeX{} file,
in particular \emph{before} the |\documentclass| statement!
The argument of |\childdocmain| should be left empty
(but it must be present).

%%%%%%%%%%%%%%%%%%%%%%%%%%%%%%%%%%%%%%%%
\DescribeMacro{\childdocof}
Furthermore, add the commands
\begin{center}
\begin{tabular}{l}
|\input{childdoc.def}|\\
|\childdocof{|\textit{main}|}|\\
\end{tabular}
\end{center}
at the top of every child file \textit{child}
which is included by |\include{|\textit{child}|}|
from within the main file
(or at least for those files to be compiled individually).
The argument \textit{main} must be the filename of the main file.

There are a couple of
considerations in setting up the main and child documents:

%%%%%%%%%%%%%%%%%%%%%%%%%%%%%%%%%%%%%%%%
\paragraph{Restrictions.}

Please note the following restrictions:
\begin{itemize}
\item
|\childdocmain| must be called with one argument \textit{main}
to ensure compatibility with earlier version of the package.
It must either be empty (|\childdocmain{}|)
or precisely match the filename of the main file in which it is specified.
See \secref{sec:detection} for further information.
\item
The filename \textit{main} must be specified without the |.tex| extension.
\item
The filename \textit{main} is case sensitive
(even in case-insensitive file systems)
due to internal string comparison.
\item
The argument \textit{main} should be fully expanded, it cannot be a macro.
\item
Subdirectories and special characters should be avoided in filenames.
\item
The command |\childdocmain{|\textit{main}|}| must be followed by a whitespace.
It should not be followed immediately by another command
or by a comment mark `|%|'.
This is because the \TeX{} parser reads the token immediately following
the argument of |\childdocmain| and puts it
at the beginning of every child section;
however, a white\-space is ignored.
\end{itemize}

%%%%%%%%%%%%%%%%%%%%%%%%%%%%%%%%%%%%%%%%
\paragraph{Content of Main File.}

It is advisable to place all content in the child files included by |\include|.
Any output contained in the main file will appear in all child documents
unless suppressed manually;
it cannot be suppressed automatically by the |\includeonly| directive
and thus should normally be avoided.
A method to include some content in the main file
by means of conditional processing is described in \secref{sec:conditional}.

%%%%%%%%%%%%%%%%%%%%%%%%%%%%%%%%%%%%%%%%
\paragraph{Page Numbering.}

When only a part of the document is compiled,
the appropriate numbering of pages
(as well as other status parameters)
is determined from the |.aux| files.
The latter contain information from previous passes.
However this information needs to propagate through
all intermediate child documents.
Therefore the page numbering in child documents may well
be inconsistent until the complete document is compiled at least once.

A useful (if unconventional) way to always ensure a consistent
page numbering is to restart the numbering in each child document
and denote the pages by `\textit{child}|.|\textit{page}'
where \textit{child} represents the chapter/section number of the child file.
This can be achieved by the command
|\numberwithin{page}{|\textit{child}|}|
of the \textsf{amsmath} package
where \textit{child} can be |chapter| or |section|
depending on the chosen structuring.
Alternatively, one can modify the macro |\thepage| appropriately
and reset the counter |page| at the start of each child file.

%%%%%%%%%%%%%%%%%%%%%%%%%%%%%%%%%%%%%%%%%%%%%%%%%%%%%%%%%%%%%%%%%%%%%%%%%%%%%%%%
\subsection{Conditional Processing}
\label{sec:conditional}

The package provides a mechanism to compile different versions
of a document. To customise the versions further some conditional processing
can come in handy to distinguish which version is being compiled.
The package provides two macros to describe the compilation context:

%%%%%%%%%%%%%%%%%%%%%%%%%%%%%%%%%%%%%%%%
\DescribeMacro{\ifchilddoc}
The conditional |\ifchilddoc| distinguishes between the compilation of
child documents and the main document:
%
\begin{center}
|\ifchilddoc |\textit{child-code}| |[|\||else |\textit{main-code}]| \||fi|
\end{center}

%%%%%%%%%%%%%%%%%%%%%%%%%%%%%%%%%%%%%%%%
\DescribeMacro{\childdocname}
\DescribeMacro{\childdocjob}
The macro |\childdocname| contains the filename (without extension)
of the main or child file being processed.
Note that |\childdocjob| will always contain the name of the main file.

%%%%%%%%%%%%%%%%%%%%%%%%%%%%%%%%%%%%%%%%
\paragraph{Title Page.}

Conditional processing can be used to include a title or banner page
in the main document when proper precautions are taken.
Importantly, the code in the main file should ensure that the page counter
(as well as other status parameters which are stored in the |.aux| files)
takes the same value after the conditional processing.
Otherwise the page numbers may take divergent values
depending on which part is compiled.

For example, a title page could be declared by:
%
\begin{center}
\begin{tabular}{l}
|\ifchilddoc\||else|\\
|\addtocounter{page}{-1}|\\
\textit{code for title page}\\
|\newpage|\\
|\||fi|
\end{tabular}
\end{center}
%
A banner page for the child documents can be generated by:
%
\begin{center}
\begin{tabular}{l}
|\ifchilddoc|\\
|\addtocounter{page}{-1}|\\
\textit{code for banner page}\\
|\newpage|\\
|\||fi|
\end{tabular}
\end{center}
%
Here one could write a message such as:
\begin{center}
|This is the part \childdocname{} of \childdocjob{}.|
\end{center}

%%%%%%%%%%%%%%%%%%%%%%%%%%%%%%%%%%%%%%%%%%%%%%%%%%%%%%%%%%%%%%%%%%%%%%%%%%%%%%%%
\subsection{Flags}
\label{sec:flags}

The package makes it easy to generate different versions
of the main or child documents.
To this end compilation flags can be defined
and assigned different default values.
They will be particularly useful in conjunction
with the forwarding mechanism described in \secref{sec:forward}.

For example, it may be useful to have a flag |\version|
which can be set to |draft| or |final|.
The document source will contain some conditional code
depending on the value of |\version|.
Suppose further, the flag should default to |final| for the main file
and to |draft| for child files
which is a natural assignment for editing the document.
This is achieved by placing the following code
in the preamble of the main document
(below the |\childdocmain| directive):
%
\begin{center}
\begin{tabular}{l}
|\ifchilddoc|\\
|\providecommand{\version}{draft}|\\
|\||else|\\
|\providecommand{\version}{final}|\\
|\||fi|
\end{tabular}
\end{center}
%
The definition by |\providecommand| makes sure
that previous definitions are not overwritten.
Further statements |\providecommand{\version}{...}|
can thus be added before the above code to override it.

For the main file, one might add a line
(between |\childdocmain| and the above block)
%
\begin{center}
|%\ifchilddoc\||else\providecommand{\version}{draft}\||fi|
\end{center}
%
which can be uncommented to produce a draft version.
Likewise one can add a line to the very top of a child file
(above the |\childdocof{|\textit{main}|}| directive)
%
\begin{center}
|%\providecommand{\version}{final}|
\end{center}
%
which can be uncommented to produce the final version of this child document.

%%%%%%%%%%%%%%%%%%%%%%%%%%%%%%%%%%%%%%%%%%%%%%%%%%%%%%%%%%%%%%%%%%%%%%%%%%%%%%%%
\subsection{Forwarding}
\label{sec:forward}

Different versions of the main or child documents
using compilation flags as described in \secref{sec:flags}
can be (permanently) stored in different files
for convenient compilation, viewing and distribution.
To this end, the package defines a command
to pass on compilation to a different file:

%%%%%%%%%%%%%%%%%%%%%%%%%%%%%%%%%%%%%%%%
\DescribeMacro{\childdocforward}
The command |\childdocforward| redirects processing to
another source file:
%
\begin{center}
\begin{tabular}{l}
|\input{childdoc.def}|\\
|\childdocforward[|\textit{main}|]{|\textit{dest}|}|\\
\end{tabular}
\end{center}
%
The argument \textit{dest} is the destination file
(without extension).
It should be the main file or one of the child files.
Note that further \textsf{childdoc} directives
such as |\childdocof| and |\childdocforward|
in the indicated file will be processed in this form.
The optional argument \textit{main}
passes on directly to the main file \textit{main}
while pretending to compile the child \textit{dest}.
This form behaves as if \textit{dest}
issues |\childdocof{|\textit{main}|}| right away,
and no further \textsf{childdoc} directives will be processed.

%%%%%%%%%%%%%%%%%%%%%%%%%%%%%%%%%%%%%%%%
\DescribeMacro{\...prefix}
In the alternative form |\childdocforwardprefix|,
%
\begin{center}
\begin{tabular}{l}
|\input{childdoc.def}|\\
|\childdocforwardprefix[|\textit{main}|]{|\textit{prefix}|}{|\textit{dest}|}|
\end{tabular}
\end{center}
%
the destination file is determined by a pattern
depending on the current file:
To make this work, the current file must be called
`{\textit{prefix}\hspace{0.2em}\textit{suffix}}'
with \textit{prefix} matching precisely the argument.
Processing is then passed on to the file
`{\textit{dest}\hspace{0.2em}\textit{suffix}}'.
Surely, the same effect is achieved by
directly specifying the
argument `{\textit{dest}\hspace{0.2em}\textit{suffix}}'
in the first form.
However, that requires to set up a different file
for each child. With the alternative form of the command
all these files can have exactly the same content
which simplifies setting them up and maintaining them.

For example, the following file |draft.tex|
with a compilation flag |\version| as described in \secref{sec:flags}
compiles the main document as a draft:
%
\begin{center}
\begin{tabular}{l}
|\def\version{draft}|\\
|\input{childdoc.def}|\\
|\childdocforward{|\textit{main}|}|
\end{tabular}
\end{center}
%
Likewise, the following files |final|\textit{nn}|.tex|
compile the final version of the child document
|child|\textit{nn}|.tex|:
%
\begin{center}
\begin{tabular}{l}
|\def\version{final}|\\
|\input{childdoc.def}|\\
|\childdocforwardprefix{final}{child}|
\end{tabular}
\end{center}
%

Note that when several versions of a main file and/or of each child file
are to be generated, it may be convenient to set up a |Makefile| or
shell script to automatise the process.

%%%%%%%%%%%%%%%%%%%%%%%%%%%%%%%%%%%%%%%%%%%%%%%%%%%%%%%%%%%%%%%%%%%%%%%%%%%%%%%%
\subsection{Command Line Processing}
\label{sec:commandline}

The effect of redirection files can also be achieved by invoking
the \LaTeX{} compiler with a more elaborate command line.
Most conveniently this should be done as part
of a shell script or a |Makefile|.

When using \textsf{childdoc} in the main file, the following
command lines effectively perform a redirection
(note that depending on the shell being used,
backslashes may have to be doubled: `|\|' $\to$ `|\\|'):
%
\begin{center}
|... -jobname "|\textit{target}|" |\\|"|[\textit{flags}]%
|\input{childdoc.def}\childdocforward[|\textit{main}|]{|\textit{dest}|}"|
\end{center}
%
Here \textit{target} is the name of the output file,
\textit{main} is the name of the main file
and \textit{dest} is the name of the main or child file to be processed
(all filenames without extensions).
The optional argument \textit{main} can be omitted
if \textit{main} matches \textit{dest}.
Optionally, compilation \textit{flags} can be defined via |\def| commands.
This command line makes the \TeX{} engine believe
it is compiling the file \textit{target}
whose content is specified as the latter parameter.
The provided code then forwards the processing to
\textit{main} or \textit{dest} as described in \secref{sec:forward}.

%%%%%%%%%%%%%%%%%%%%%%%%%%%%%%%%%%%%%%%%%%%%%%%%%%%%%%%%%%%%%%%%%%%%%%%%%%%%%%%%
\subsection{Include by Input}
\label{sec:input}

Including child documents by |\include| has some restrictions by design.
Most notably, the content of a child document always occupies
its own set of pages; pages cannot be shared between child documents.
Usually, this behaviour makes perfect sense
because each child document contain an essential part of the document.
However, in some situations it may be desirable to compose
a document from a collection of parts
without having mandatory page breaks between then.
For this case, the package
provides a mechanism to include parts
by |\input| which can also be processed individually.
However, by construction this mechanism
requires manual handling of the content to be output.

%%%%%%%%%%%%%%%%%%%%%%%%%%%%%%%%%%%%%%%%
\DescribeMacro{\ifchilddocmanual}
The main file should be prepared as usual, see \secref{sec:include}.
However, the document body must make a distinction
between processing of an individual part and of the main document, e.g.:
%
\begin{center}
\begin{tabular}{l}
|\ifchilddocmanual|\\
|\input{\childdocname}|\\
|\||else|\\
\textit{document body with }|\input{|\textit{part}|}|\\
|\||fi|
\end{tabular}
\end{center}
%
The conditional |\ifchilddocmanual| is true whenever
a part to be included by |\input| is being compiled,
and the name of the part is stored in |\childdocname|.

%%%%%%%%%%%%%%%%%%%%%%%%%%%%%%%%%%%%%%%%
\DescribeMacro{\childdocby}
Each part to be included by |\input| should start with:
%
\begin{center}
\begin{tabular}{l}
|\input{childdoc.def}|\\
|\childdocby{|\textit{main}|}|\\
\end{tabular}
\end{center}
%
The directive |\childdocby| is similar to |\childdocof|
described in \secref{sec:include},
but the subsequent selection of content must be done manually.
To that end, both |\ifchilddoc| and |\ifchilddocmanual|
will be true upon processing of a part,
and the name of the part is stored in |\childdocname|.
Note that |\jobname| will be set to the filename of the current part
so that each part receives an individual |.aux| file
that does not interfere with the |.aux| file(s) of the main document.
This behaviour can be altered by the alternative form
|\childdocby[*]{|\textit{main}|}| (with a non-empty optional argument)
which uses the |.aux| file of the main document
by setting |\jobname| to \textit{main}.

%%%%%%%%%%%%%%%%%%%%%%%%%%%%%%%%%%%%%%%%%%%%%%%%%%%%%%%%%%%%%%%%%%%%%%%%%%%%%%%%
\subsection{Driver Development}
\label{sec:driver}

The \textsf{childdoc} mechanism can also be use for the development
of definition files such as \LaTeX{} styles or classes.
This case differs from the above setup with multiple parts
included by |\include| in that no |\includeonly| should be invoked.
This can be achieved by starting the include file
(before |\ProvidesPackage|) with:
%
\begin{center}
\begin{tabular}{l}
|\input{childdoc.def}|\\
|\childdocforward{|\textit{main}|}|\\
\end{tabular}
\end{center}
%
or alternatively with:
%
\begin{center}
\begin{tabular}{l}
|\input{childdoc.def}|\\
|\childdocby{|\textit{main}|}|\\
\end{tabular}
\end{center}
%
Both forms have slightly different effects as described above.
The main file is prepared as usual, see \secref{sec:include}.

%%%%%%%%%%%%%%%%%%%%%%%%%%%%%%%%%%%%%%%%%%%%%%%%%%%%%%%%%%%%%%%%%%%%%%%%%%%%%%%%
\subsection{Legacy Detection}
\label{sec:detection}

The directive |\childdocmain| in the main file can detect
whether the complete document or merely a child is to be compiled
even without using the directive |\childdocof|.
This method is deprecated because it is less robust
and there is no compelling reason to use it;
it is merely provided for backward compatibility
and it may be removed in future versions.

If the detection mechanism is to be used,
it is mandatory to correctly specify
the filename of the main file as the argument of |\childdocmain|:
%
\begin{center}
\begin{tabular}{l}
|\input{childdoc.def}|\\
|\childdocmain{|\textit{main}|}|\\
\end{tabular}
\end{center}
%
If |\jobname| does not match the argument \textit{main} of |\childdocmain|,
it is assumed that |\jobname| points to the child file to be compiled.
When using |\childdocmain| with the main file specified as argument,
it suffices to start a child file
with just |\input{|\textit{main}|}|
without loading of the package and using |\childdocof|.
If instead all processing is done
with the appropriate \textsf{childdoc} directives,
the argument of \textit{main} of |\childdocmain| can be empty.

An alternative version of the command line processing described
in \secref{sec:commandline} using the detection mechanism reads:
%
\begin{center}
|... -jobname "|\textit{target}|" "|[\textit{flags}]%
[|\def\jobname{|\textit{dest}|}|]|\input{|\textit{main}|}"|
\end{center}

%%%%%%%%%%%%%%%%%%%%%%%%%%%%%%%%%%%%%%%%%%%%%%%%%%%%%%%%%%%%%%%%%%%%%%%%%%%%%%%%
\subsection{Manual Code}
\label{sec:manual}

In case one cannot be certain whether the definitions file |childdoc.def|
is installed on the target \TeX{} distribution
and one prefers not to ship it,
it is conceivable to paste a few relevant commands into the sources.

To that end, drop all statements |\input{childdoc.def}|
and perform the replacements as outlined below.
Instead of |\childdocmain{|\textit{main}|}| add the following code
to the top of the main file:
%
\begin{center}
\begin{tabular}{l}
|\||ifdefined\childdocname\endinput\||fi\newif\ifchilddoc|\\
|\edef\childdocname{\scantokens\expandafter{\jobname\noexpand}}|\\
|\def\childdocmain{|\textit{main}|}\||ifx\childdocmain\childdocname\||else|\\
|\childdoctrue\includeonly{\childdocname}\let\jobname\childdocmain\||fi|\\
\end{tabular}
\end{center}
%
Instead of |\childdocof{|\textit{main}|}| just include the main file
at the top of each child file:
%
\begin{center}
|\input{|\textit{main}|}|
\end{center}
%
A simple redirection |\childdocforward{|\textit{dest}|}| is achieved by:
%
\begin{center}
|\def\jobname{|\textit{dest}|}\input{\jobname}|
\end{center}
%
The redirection with prefix
|\childdocforwardprefix[|\textit{prefix}|]{|\textit{dest}|}|
is accomplished by:
%
\begin{center}
\begin{tabular}{l}
|{\edef\jobname{\scantokens\expandafter{\jobname\noexpand}}|\\
|\def\redirectjob |\textit{prefix}|#1~~~{\gdef\jobname{|\textit{dest}|#1}}|\\
|\expandafter\redirectjob\jobname~~~}\input{\jobname}|
\end{tabular}
\end{center}

In an alternative approach,
child documents can be compiled by a specific command line
without additional code or specific definitions:
%
\begin{center}
|... -jobname "|\textit{target}|" "|[\textit{flags}]%
|\includeonly{|\textit{dest}|}\input{|\textit{main}|}"|
\end{center}
%

%%%%%%%%%%%%%%%%%%%%%%%%%%%%%%%%%%%%%%%%%%%%%%%%%%%%%%%%%%%%%%%%%%%%%%%%%%%%%%%%
%%%%%%%%%%%%%%%%%%%%%%%%%%%%%%%%%%%%%%%%%%%%%%%%%%%%%%%%%%%%%%%%%%%%%%%%%%%%%%%%
\section{Information}

%%%%%%%%%%%%%%%%%%%%%%%%%%%%%%%%%%%%%%%%%%%%%%%%%%%%%%%%%%%%%%%%%%%%%%%%%%%%%%%%
\subsection{Copyright}

Copyright \copyright{} 2017--2018 Niklas Beisert

This work may be distributed and/or modified under the
conditions of the \LaTeX{} Project Public License, either version 1.3
of this license or (at your option) any later version.
The latest version of this license is in
  \url{http://www.latex-project.org/lppl.txt}
and version 1.3 or later is part of all distributions of \LaTeX{}
version 2005/12/01 or later.

This work has the LPPL maintenance status `maintained'.

The Current Maintainer of this work is Niklas Beisert.

This work consists of the files |README.txt|, |childdoc.ins| and |childdoc.dtx|
as well as the derived files |childdoc.def|, |cdocsamp.tex|
with |cdocsch1.tex|, |cdocsch2.tex|, |cdocspt3.tex|, |cdocspt4.tex|,
|cdocsdrf.tex|, |cdocsfn1.tex|, |cdocsfn2.tex|
as well as |childdoc.pdf|.

%%%%%%%%%%%%%%%%%%%%%%%%%%%%%%%%%%%%%%%%%%%%%%%%%%%%%%%%%%%%%%%%%%%%%%%%%%%%%%%%
\subsection{Files and Installation}

The package consists of the files:
%
\begin{center}
\begin{tabular}{ll}
    |README.txt|   & readme file \\
    |childdoc.ins| & installation file \\
    |childdoc.dtx| & source file \\
    |childdoc.def| & definition file \\
    |cdocsamp.tex| & sample main file \\
    |cdocsch1.tex| & sample include file \\
    |cdocsch2.tex| & sample include file \\
    |cdocspt3.tex| & sample part file \\
    |cdocspt4.tex| & sample part file \\
    |cdocsdrf.tex| & sample redirection file \\
    |cdocsfn1.tex| & sample redirection file \\
    |cdocsfn2.tex| & sample redirection file \\
    |childdoc.pdf| & manual
\end{tabular}
\end{center}
%
The distribution consists of the files
|README.txt|, |childdoc.ins| and |childdoc.dtx|.
%
\begin{itemize}
\item
Run (pdf)\LaTeX{} on |childdoc.dtx|
to compile the manual |childdoc.pdf| (this file).
\item
Run \LaTeX{} on |childdoc.ins| to create the definitions file |childdoc.def|
and the sample |cdocsamp.tex| with include files
|cdocsch1.tex|, |cdocsch2.tex|, |cdocspt3.tex|, |cdocspt4.tex|,
|cdocsdrf.tex|, |cdocsfn1.tex|, |cdocsfn2.tex|.
Then copy the file |childdoc.def| to an appropriate directory of your \LaTeX{}
distribution, e.g.\ \textit{texmf-root}|/tex/latex/childdoc|.
\end{itemize}

%%%%%%%%%%%%%%%%%%%%%%%%%%%%%%%%%%%%%%%%%%%%%%%%%%%%%%%%%%%%%%%%%%%%%%%%%%%%%%%%
\subsection{Related CTAN Packages}

There are several other packages which offer a similar functionality:
%
\begin{itemize}
\item
The packages
\href{http://ctan.org/pkg/docmute}{\textsf{docmute}},
\href{http://ctan.org/pkg/includex}{\textsf{includex}} and
\href{http://ctan.org/pkg/standalone}{\textsf{standalone}}
provide commands to include only the document body of
a child file thus allowing both files to be compiled individually.
\item
The packages \href{http://ctan.org/pkg/subdocs}{\textsf{subdocs}}
and \href{http://ctan.org/pkg/subfiles}{\textsf{subfiles}}
provide structures in which the main and child documents can be
encapsulated and allowing them to be compiled individually.
The inclusion mechanism is different from the conventional |\include|.
\item
The package \href{http://ctan.org/pkg/combine}{\textsf{combine}}
is an elaborate solution to combine several documents into one.
\end{itemize}
%
See also the CTAN topic \href{http://ctan.org/topic/subdocs}{\textsf{subdocs}}
for further related packages.
The present package differs from the above solutions in that
a document structure constructed with the conventional |\include| mechanism
just needs two extra commands at the top of every file
such that all constituent files can be compiled individually.

%%%%%%%%%%%%%%%%%%%%%%%%%%%%%%%%%%%%%%%%%%%%%%%%%%%%%%%%%%%%%%%%%%%%%%%%%%%%%%%%
%\subsection{Feature Suggestions}
%
%The following is a list of features which may be useful for future
%versions of this package:
%%
%\begin{itemize}
%\item
%\ldots
%\end{itemize}

%%%%%%%%%%%%%%%%%%%%%%%%%%%%%%%%%%%%%%%%%%%%%%%%%%%%%%%%%%%%%%%%%%%%%%%%%%%%%%%%
\subsection{Revision History}

%%%%%%%%%%%%%%%%%%%%%%%%%%%%%%%%%%%%%%%%
\paragraph{v2.0:} 2018/12/30

\begin{itemize}
\item
immediate forward processing
\item
added |\childdocby| mechanism
\item
manual restructured
\end{itemize}

%%%%%%%%%%%%%%%%%%%%%%%%%%%%%%%%%%%%%%%%
\paragraph{v1.6:} 2018/01/17

\begin{itemize}
\item
application for development of include files
\item
corrections to manual
\end{itemize}

%%%%%%%%%%%%%%%%%%%%%%%%%%%%%%%%%%%%%%%%
\paragraph{v1.5:} 2017/05/21

\begin{itemize}
\item
more complete structuring introduced
\item
|\childdocof| introduced
\item
|\childdoc| renamed to |\childdocmain|
\item
|\childredirect| renamed to |\childdocforward| and |\childdocforwardprefix|
and functionality expanded
\end{itemize}

%%%%%%%%%%%%%%%%%%%%%%%%%%%%%%%%%%%%%%%%
\paragraph{v1.0:} 2017/04/27

\begin{itemize}
\item
manual and install package
\item
first version published on CTAN
\end{itemize}

%%%%%%%%%%%%%%%%%%%%%%%%%%%%%%%%%%%%%%%%
\paragraph{v0.6:} 2017/04/26

\begin{itemize}
\item
redirection mechanism added
\end{itemize}

%%%%%%%%%%%%%%%%%%%%%%%%%%%%%%%%%%%%%%%%
\paragraph{v0.5:} 2017/04/26

\begin{itemize}
\item
functionality in definition file
\end{itemize}


%%%%%%%%%%%%%%%%%%%%%%%%%%%%%%%%%%%%%%%%%%%%%%%%%%%%%%%%%%%%%%%%%%%%%%%%%%%%%%%%
%%%%%%%%%%%%%%%%%%%%%%%%%%%%%%%%%%%%%%%%%%%%%%%%%%%%%%%%%%%%%%%%%%%%%%%%%%%%%%%%
%%%%%%%%%%%%%%%%%%%%%%%%%%%%%%%%%%%%%%%%%%%%%%%%%%%%%%%%%%%%%%%%%%%%%%%%%%%%%%%%
\appendix

\settowidth\MacroIndent{\rmfamily\scriptsize 000\ }

 \DocInput{childdoc.dtx}

\end{document}
%</driver>
% \fi
%
% %%%%%%%%%%%%%%%%%%%%%%%%%%%%%%%%%%%%%%%%%%%%%%%%%%%%%%%%%%%%%%%%%%%%%%%%%%%%%%
% %%%%%%%%%%%%%%%%%%%%%%%%%%%%%%%%%%%%%%%%%%%%%%%%%%%%%%%%%%%%%%%%%%%%%%%%%%%%%%
% \section{Sample}
%\iffalse
%<*samplemain>
%\fi
%
% The following presents a sample document
% with two chapters, two parts, a title page,
% a compile flag as well as three forwarding files to set the flag.
% It consists of eight |.tex| files:
% \begin{center}
% \begin{tabular}{ll}
% |cdocsamp.tex|&main file\\
% |cdocsch1.tex|&include file for chapter 1\\
% |cdocsch2.tex|&include file for chapter 2\\
% |cdocspt3.tex|&include file for part 3\\
% |cdocspt4.tex|&include file for part 4\\
% |cdocsdrf.tex|&forwarding file for main file in draft mode\\
% |cdocsfi1.tex|&forwarding file for final version of chapter 1\\
% |cdocsfi2.tex|&forwarding file for final version of chapter 2\\
% \end{tabular}
% \end{center}
% Each of the eight files can be compiled directly by the \LaTeX{} compiler.
%
% %%%%%%%%%%%%%%%%%%%%%%%%%%%%%%%%%%%%%%
% \paragraph{Main File.}
%
% The main file is called |cdocsamp.tex|.
%
% Load the \textsf{childdoc} definitions and
% declare the filename for the main document:
%    \begin{macrocode}
\input{childdoc.def}
\childdocmain{}
%    \end{macrocode}

% Optional override for |\version| flag:
%    \begin{macrocode}
%%\ifchilddoc\else\providecommand{\version}{draft}\fi
%    \end{macrocode}

% Define the default values for the |\version| flag
% (|final| for the main file and |draft| for childs):
%    \begin{macrocode}
\ifchilddoc
\providecommand{\version}{draft}
\else
\providecommand{\version}{final}
\fi
%    \end{macrocode}

% Load the standard document class:
%    \begin{macrocode}
\documentclass[12pt]{article}
%    \end{macrocode}

% Start the document body:
%    \begin{macrocode}
\begin{document}
%    \end{macrocode}

% Declare a title page.
% Print title, part of document being processed and version flag:
%    \begin{macrocode}
\addtocounter{page}{-1}
\begin{center}
{\LARGE\bfseries{}childdoc example\par}
\vspace{1cm}
\ifchilddoc
\ifchilddocmanual part\else chapter\fi:
`\childdocname' of `\childdocjob'\par
\else
main document: `\childdocjob'\par
\fi
version: \version\par
\end{center}
\newpage
%    \end{macrocode}

% Manually include selected file,
% otherwise process as usual:
%    \begin{macrocode}
\ifchilddocmanual
\section*{part `\childdocname'}
\input{\childdocname}
\else
%    \end{macrocode}

% Include the two chapters:
%    \begin{macrocode}
\include{cdocsch1}
\include{cdocsch2}
%    \end{macrocode}

% Include the two parts unless only chapters should be displayed:
%    \begin{macrocode}
\ifchilddoc\else
\section{part three}
\input{cdocspt3}
\section{part four}
\input{cdocspt4}
\fi
%    \end{macrocode}

% Process as usual until here:
%    \begin{macrocode}
\fi
%    \end{macrocode}

% End of document body:
%    \begin{macrocode}
\end{document}
%    \end{macrocode}
%\iffalse
%</samplemain>
%\fi
%
% %%%%%%%%%%%%%%%%%%%%%%%%%%%%%%%%%%%%%%
% \paragraph{Chapter Include Files.}
%
% The include files are called |cdocsch1.tex| and |cdocsch2.tex|.
%
%\iffalse
%<*samplechap1|samplechap2>
%\fi

% Optional override for |\version| flag:
%    \begin{macrocode}
%%\providecommand{\version}{final}
%    \end{macrocode}

% Include the main document:
%    \begin{macrocode}
\input{childdoc.def}
\childdocof{cdocsamp}
%    \end{macrocode}

%\iffalse
%</samplechap1|samplechap2>
%\fi
%
%\iffalse
%<*samplechap1>
%\fi
% Some text for chapter 1:
%    \begin{macrocode}
\section{one}
some text in chapter one
%    \end{macrocode}

%\iffalse
%</samplechap1>
%\fi
% Some text for chapter 2:
%\iffalse
%<*samplechap2>
%\fi
%    \begin{macrocode}
\section{two}
more text in chapter two
%    \end{macrocode}

%\iffalse
%</samplechap2>
%\fi
%
% %%%%%%%%%%%%%%%%%%%%%%%%%%%%%%%%%%%%%%
% \paragraph{Part Include Files.}
%
% The include files are called |cdocspt3.tex| and |cdocspt4.tex|.
%
%\iffalse
%<*samplepart3|samplepart4>
%\fi

% Optional override for |\version| flag:
%    \begin{macrocode}
%%\providecommand{\version}{final}
%    \end{macrocode}

% Include the main document:
%    \begin{macrocode}
\input{childdoc.def}
\childdocby{cdocsamp}
%    \end{macrocode}

%\iffalse
%</samplepart3|samplepart4>
%\fi
%
%\iffalse
%<*samplepart3>
%\fi
% Some text for part 3:
%    \begin{macrocode}
some text in part three
%    \end{macrocode}

%\iffalse
%</samplepart3>
%\fi
% Some text for part 4:
%\iffalse
%<*samplepart4>
%\fi
%    \begin{macrocode}
more text in part four
%    \end{macrocode}

%\iffalse
%</samplepart4>
%\fi
%
% %%%%%%%%%%%%%%%%%%%%%%%%%%%%%%%%%%%%%%
% \paragraph{Forwarding for a Complete Draft.}
%
% The following forwarding file |cdocsdrf.tex|
% compiles the main document in draft mode:
%\iffalse
%<*sampledraft>
%\fi
%    \begin{macrocode}
\def\version{draft}
\input{childdoc.def}
\childdocforward{cdocsamp}
%    \end{macrocode}

%\iffalse
%</sampledraft>
%\fi
%
% %%%%%%%%%%%%%%%%%%%%%%%%%%%%%%%%%%%%%%
% \paragraph{Forwarding for Final Version of the Chapters.}
%
% The following forwarding files |cdocsfn1.tex| and |cdocsfn2.tex|
% (with identical content)
% compile the final versions of the child documents
% |cdocsch1.tex| and |cdocsch2.tex|, respectively:
%\iffalse
%<*samplefinal>
%\fi
%    \begin{macrocode}
\def\version{final}
\input{childdoc.def}
\childdocforwardprefix[cdocsamp]{cdocsfn}{cdocsch}
%    \end{macrocode}

%\iffalse
%</samplefinal>
%\fi
%
% %%%%%%%%%%%%%%%%%%%%%%%%%%%%%%%%%%%%%%
% \paragraph{Command Line Processing.}
%
% The following three command lines generate the output files
% |cdocscld|, |cdocscl1| and |cdocscl2|
% which should be identical to
% |cdocsdrf|, |cdocsch1| and |cdocsfn2|, respectively:
% \begin{center}
% \begin{tabular}{l}
% |latex -jobname cdocscld \|\\
% |  "\def\version{draft}\input{childdoc.def}\childdocforward{cdocsamp}"|\\
% |latex -jobname cdocscl1 \|\\
% |  "\input{childdoc.def}\childdocforward[cdocsamp]{cdocsch1}"|\\
% |latex -jobname cdocscl2 \|\\
% |  "\def\version{final}\input{childdoc.def}\childdocforward{cdocsch2}"|
% \end{tabular}
% \end{center}
% Note that the trailing backslash on each first line
% merely continues the input to the second line
% (for convenient cut ant paste).
% Furthermore, the command |latex| can be replaced by any
% of its alternative versions such as |pdflatex|.
%
% %%%%%%%%%%%%%%%%%%%%%%%%%%%%%%%%%%%%%%%%%%%%%%%%%%%%%%%%%%%%%%%%%%%%%%%%%%%%%%
% %%%%%%%%%%%%%%%%%%%%%%%%%%%%%%%%%%%%%%%%%%%%%%%%%%%%%%%%%%%%%%%%%%%%%%%%%%%%%%
% \section{Implementation}
%\iffalse
%<*package>
%\fi
%
% This section describes the definitions file |childdoc.def|.

% The definitions cannot be loaded using |\usepackage| or |\RequirePackage|
% which has a mechanism to prevent loading a style file more than once.
% When loading the definitions by means of |\input|
% multiple instances have to be prevented manually:
%\iffalse
%This code needs to be before the `\ProvidesFile' directive
%which is defined at the beginning of this file.
%Therefore it is also placed there and commented out here.
%</package>
%<*discard>
%\fi
%    \begin{macrocode}
\ifdefined\childdocmain\endinput\fi
%    \end{macrocode}
%\iffalse
%</discard>
%<*package>
%\fi
%
% \macro{\ifchilddoc}
% \macro{\ifchilddocmanual}
% The conditional |\ifchilddoc| tells whether a
% child (true) or main (false) document is being compiled.
% The conditional |\ifchilddocmanual| tells whether
% the |\includeonly| mechanism is used (false) or
% the selection of child files must be performed manually (true).
% The definitions initialise to false:
%    \begin{macrocode}
\newif\ifchilddoc
\newif\ifchilddocmanual
%    \end{macrocode}

% \macro{\childdocname}
% \macro{\childdocjob}
% The macro |\childdocname| stores the name of the main document
% to be compiled. The macro |\childdocjob| stores the name of
% the document on which the \LaTeX{} compiler was originally invoked.
% The content of |\jobname| cannot be compared
% to filenames specified in the source due to different catcodes.
% The following code rescans |\jobname|, stores the result
% in |\childdocname| and saves a copy in |\childdocjob|:
%    \begin{macrocode}
\edef\childdocname{\scantokens\expandafter{\jobname\noexpand}}
\let\childdocjob\childdocname
%    \end{macrocode}

% \macro{\childdocdisable}
% The macro |\childdocdisable| prevents the main file
% from being processed more than once.
% At this stage, the main document command |\childdocmain|
% is assumed to be called once again where it should do nothing.
% Any subsequent call to it should prevent
% a secondary processing of the main document
% It overwrites the forwarding commands
% |\childdocof| and |\childdocforward|
% with empty macros to prevent further inclusions of the main document:
%    \begin{macrocode}
\newcommand{\childdocdisable}
{
  \renewcommand{\childdocmain}[1]{\renewcommand{\childdocmain}[1]{\endinput}}
  \renewcommand{\childdocof}[1]{}
  \renewcommand{\childdocby}[2][]{}
  \renewcommand{\childdocforward}[2][]{}
  \renewcommand{\childdocdisable}{}
}
%    \end{macrocode}

% \macro{\childdocmain}
% The macro |\childdocmain| is to be called at the top of the main file
% with nothing or the main filename (without extension) as argument.
% First, it breaks loops.
% If the argument is not empty and does not match |\childdocname|
% (which is set by the first inclusion of |childdoc.def|),
% |\ifchilddoc| is set to true, |\includeonly| is applied to the child file
% and |\jobname| is set to the main file
% (for proper handling of |.aux| files):
%    \begin{macrocode}
\newcommand{\childdocmain}[1]
{
  \childdocdisable\childdocmain{}
  \if?#1?\else
    \begingroup
      \def\childdoctmp{#1}
      \ifx\childdoctmp\childdocname
        \def\childdoctmp{}
      \else
        \def\childdoctmp
        {
          \childdoctrue
          \includeonly{\childdocname}
          \def\childdocjob{#1}
          \def\jobname{#1}
        }
      \fi
      \expandafter
    \endgroup
    \childdoctmp
  \fi
}
%    \end{macrocode}

% \macro{\childdocof}
% The command |\childdocof| redirects
% compilation to the main file |#1|.
%    \begin{macrocode}
\newcommand{\childdocof}[1]
{
  \childdocdisable
  \childdoctrue
  \includeonly{\childdocname}
  \def\jobname{#1}
  \def\childdocjob{#1}
  \input{#1}
}
%    \end{macrocode}

% \macro{\childdocby}
% The command |\childdocby| ....
%    \begin{macrocode}
\newcommand{\childdocby}[2][]
{
  \childdocdisable
  \childdoctrue
  \childdocmanualtrue
  \if?#1?\else
    \def\jobname{#2}
  \fi
  \def\childdocjob{#2}
  \input{#2}
  \endinput
}
%    \end{macrocode}

% \macro{\childdocforward}
% The command |\childdocforward| redirects
% compilation to the main file or
% (if the optional argument is given) a child file.
% Parameters are set as if the main file
% or a child file starting with |\childdocof| was compiled.
% Then compilation is handed over to the main file:
%    \begin{macrocode}
\newcommand{\childdocforward}[2][]
{
  \begingroup
    \if?#1?
      \def\childdoctmp
      {
        \def\childdocname{#2}
        \def\childdocjob{#2}
        \def\jobname{#2}
        \input{#2}
        \endinput
      }
    \else
      \def\childdoctmp
      {
        \childdocdisable
        \def\childdocname{#2}
        \childdoctrue
        \includeonly{#2}
        \def\childdocjob{#1}
        \def\jobname{#1}
        \input{#1}
        \endinput
      }
    \fi
    \expandafter
  \endgroup
  \childdoctmp
}
%    \end{macrocode}

% \macro{\childdocforwardprefix}
% The command |\childdocforwardprefix| redirects
% compilation to the main or a child file by means of a pattern.
% The prefix |#1| in the current filename is replaced by |#2|
% and the suffix of the current filename is kept
% (it is assumed that the filename does not contain the substring `|~~~|'
% which is used as a delimiter).
% Compilation is handed over to the new file by |\childdocforward|:
%    \begin{macrocode}
\newcommand{\childdocforwardprefix}[3][]
{
  \begingroup
    \def\childdocextract #2##1~~~{\def\childdoctmp{\childdocforward[#1]{#3##1}}}
    \expandafter\childdocextract\childdocname~~~
    \expandafter
  \endgroup
  \childdoctmp
}
%    \end{macrocode}

% \macro{\childdoc}
% The deprecated macro |\childdoc| is a legacy version of |\childdocmain|:
%    \begin{macrocode}
\newcommand{\childdoc}{\childdocmain}
%    \end{macrocode}

% \macro{\childdocredirect}
% The deprecated macro |\childdocredirect| is a legacy version
% of |\childdocforward| and |\childdocforwardprefix|:
%    \begin{macrocode}
\newcommand{\childdocredirect}[2][]
{
  \begingroup
    \if?#1?
      \def\childdoctmp{\childdocforward{#2}}
    \else
      \def\childdoctmp{\childdocforwardprefix{#1}{#2}}
    \fi
    \expandafter
  \endgroup
  \childdoctmp
}
%    \end{macrocode}

%\iffalse
%</package>
%\fi
%
\endinput
|\\
|\childdocforwardprefix[|\textit{main}|]{|\textit{prefix}|}{|\textit{dest}|}|
\end{tabular}
\end{center}
%
the destination file is determined by a pattern
depending on the current file:
To make this work, the current file must be called
`{\textit{prefix}\hspace{0.2em}\textit{suffix}}'
with \textit{prefix} matching precisely the argument.
Processing is then passed on to the file
`{\textit{dest}\hspace{0.2em}\textit{suffix}}'.
Surely, the same effect is achieved by
directly specifying the
argument `{\textit{dest}\hspace{0.2em}\textit{suffix}}'
in the first form.
However, that requires to set up a different file
for each child. With the alternative form of the command
all these files can have exactly the same content
which simplifies setting them up and maintaining them.

For example, the following file |draft.tex|
with a compilation flag |\version| as described in \secref{sec:flags}
compiles the main document as a draft:
%
\begin{center}
\begin{tabular}{l}
|\def\version{draft}|\\
|% \iffalse
%
% childdoc.dtx Copyright (C) 2017-2018 Niklas Beisert
%
% This work may be distributed and/or modified under the
% conditions of the LaTeX Project Public License, either version 1.3
% of this license or (at your option) any later version.
% The latest version of this license is in
%   http://www.latex-project.org/lppl.txt
% and version 1.3 or later is part of all distributions of LaTeX
% version 2005/12/01 or later.
%
% This work has the LPPL maintenance status `maintained'.
%
% The Current Maintainer of this work is Niklas Beisert.
%
% This work consists of the files childdoc.dtx and childdoc.ins
% and the derived files childdoc.def and cdocsamp.tex with
% cdocsch1.tex, cdocsch2.tex, cdocsdrf.tex, cdocsfn1.tex, cdocsfn2.tex.
%
%<package>\ifdefined\childdocmain\endinput\fi
%<package>\ProvidesFile{childdoc.def}[2018/12/30 v2.0 child document driver]
%<samplemain>\ProvidesFile{cdocsamp.tex}[2018/12/30 v2.0 sample for childdoc]
%<*driver>
%\ProvidesFile{childdoc.drv}[2018/12/30 v2.0 childdoc reference manual file]
\PassOptionsToClass{10pt,a4paper}{article}
\documentclass{ltxdoc}

\usepackage[margin=35mm]{geometry}
\usepackage{hyperref}
\usepackage{hyperxmp}
\usepackage[usenames]{color}

\hypersetup{colorlinks=true}
\hypersetup{pdfstartview=FitH}
\hypersetup{pdfpagemode=UseNone}
\hypersetup{pdfsource={}}
\hypersetup{pdflang={en-UK}}
\hypersetup{pdfcopyright={Copyright 2017-2018 Niklas Beisert.
  This work may be distributed and/or modified under the
  conditions of the LaTeX Project Public License, either version 1.3
  of this license or (at your option) any later version.}}
\hypersetup{pdflicenseurl={http://www.latex-project.org/lppl.txt}}
\hypersetup{pdfcontactaddress={ETH Zurich, ITP, HIT K,
  Wolfgang-Pauli-Strasse 27}}
\hypersetup{pdfcontactpostcode={8093}}
\hypersetup{pdfcontactcity={Zurich}}
\hypersetup{pdfcontactcountry={Switzerland}}
\hypersetup{pdfcontactemail={nbeisert@itp.phys.ethz.ch}}
\hypersetup{pdfcontacturl={http://people.phys.ethz.ch/\xmptilde nbeisert/}}

\newcommand{\secref}[1]{\hyperref[#1]{section \ref*{#1}}}

\parskip1ex
\parindent0pt
\let\olditemize\itemize
\def\itemize{\olditemize\parskip0pt}

\begin{document}

\title{The \textsf{childdoc} Package}
\hypersetup{pdftitle={The childdoc Package}}
\author{Niklas Beisert\\[2ex]
  Institut f\"ur Theoretische Physik\\
  Eidgen\"ossische Technische Hochschule Z\"urich\\
  Wolfgang-Pauli-Strasse 27, 8093 Z\"urich, Switzerland\\[1ex]
  \href{mailto:nbeisert@itp.phys.ethz.ch}
  {\texttt{nbeisert@itp.phys.ethz.ch}}}
\hypersetup{pdfauthor={Niklas Beisert}}
\hypersetup{pdfsubject={Manual for the LaTeX2e Package childdoc}}
\date{30 December 2018, \textsf{v2.0}}
\maketitle

\begin{abstract}\noindent
\textsf{childdoc} is a \LaTeXe{} package
that enables the direct compilation
of document sections included by |\include|
to individual files.
\end{abstract}

\begingroup
\parskip0ex
\tableofcontents
\endgroup

%%%%%%%%%%%%%%%%%%%%%%%%%%%%%%%%%%%%%%%%%%%%%%%%%%%%%%%%%%%%%%%%%%%%%%%%%%%%%%%%
%%%%%%%%%%%%%%%%%%%%%%%%%%%%%%%%%%%%%%%%%%%%%%%%%%%%%%%%%%%%%%%%%%%%%%%%%%%%%%%%
\section{Introduction}

\LaTeX{} provides a mechanism to structure a large document (such as a book)
into a main file and several child files (containing the chapters)
using the |\include| command.
This mechanism is beneficial for documents
which span hundreds of pages in order to
make the source file(s) more manageable.
Moreover, compilation can be restricted to
selected child files by means of the |\includeonly| command.
The latter feature can be used to reduce the compilation time while editing
(this was significantly more useful in the earlier days of \LaTeX{})
or to generate a smaller document which is easier to navigate.
Another application of |\includeonly| is to generate
documents consisting of selected parts of the complete document.

However, there are a few drawbacks of the plain |\include| mechanism:
\begin{itemize}
\item
The child files cannot be compiled on their own,
they can only be compiled via the main file.
A naive editing environment
(such as a text editor with an option
to have the current file processed by \LaTeX)
may require one to switch to the main file before compiling;
attempting to compile the child file produces errors.
\item
The main file must be modified (each time)
to adjust the |\includeonly| command
to the present needs. This easily leaves the main file in a messy state.
\item
The generated document will always carry the filename
of the main document. This is inconvenient if
several child files are to be compiled and
to be kept for distribution.
\end{itemize}

The present package provides a simple interface
to make child files individually compilable by \LaTeX{}.
Compiling a child file then has the same effect as compiling
the main file with an |\includeonly| command
to select the appropriate child.
Moreover the generated document will carry the name of the child
rather than the main file.
This resolves all three above issues.

This feature is meant to make the editing of books,
thesis documents and lecture notes somewhat more convenient.
However, the package can also be used efficiently for
composing a series of documents (such as exercise sheets)
which are typically distributed individually.
It then assists the author in generating the individual documents
(potentially in different versions)
as well as a document containing the collected series.
Another application is in developing style files
or other kinds of included material
where compilation of the style file could redirect
to a sample or test file.

%%%%%%%%%%%%%%%%%%%%%%%%%%%%%%%%%%%%%%%%%%%%%%%%%%%%%%%%%%%%%%%%%%%%%%%%%%%%%%%%
%%%%%%%%%%%%%%%%%%%%%%%%%%%%%%%%%%%%%%%%%%%%%%%%%%%%%%%%%%%%%%%%%%%%%%%%%%%%%%%%
\section{Usage}

First of all, the package \textsf{childdoc} is \emph{not} a standard
\LaTeXe{} |.sty| style file! Therefore it needs to be invoked in
a non-standard way.

%%%%%%%%%%%%%%%%%%%%%%%%%%%%%%%%%%%%%%%%%%%%%%%%%%%%%%%%%%%%%%%%%%%%%%%%%%%%%%%%
\subsection{Included Files}
\label{sec:include}

%%%%%%%%%%%%%%%%%%%%%%%%%%%%%%%%%%%%%%%%
\DescribeMacro{\childdocmain}
To use the package, add the commands
\begin{center}
\begin{tabular}{l}
|\input{childdoc.def}|\\
|\childdocmain{}|\\
\end{tabular}
\end{center}
at the very top of the main \LaTeX{} file,
in particular \emph{before} the |\documentclass| statement!
The argument of |\childdocmain| should be left empty
(but it must be present).

%%%%%%%%%%%%%%%%%%%%%%%%%%%%%%%%%%%%%%%%
\DescribeMacro{\childdocof}
Furthermore, add the commands
\begin{center}
\begin{tabular}{l}
|\input{childdoc.def}|\\
|\childdocof{|\textit{main}|}|\\
\end{tabular}
\end{center}
at the top of every child file \textit{child}
which is included by |\include{|\textit{child}|}|
from within the main file
(or at least for those files to be compiled individually).
The argument \textit{main} must be the filename of the main file.

There are a couple of
considerations in setting up the main and child documents:

%%%%%%%%%%%%%%%%%%%%%%%%%%%%%%%%%%%%%%%%
\paragraph{Restrictions.}

Please note the following restrictions:
\begin{itemize}
\item
|\childdocmain| must be called with one argument \textit{main}
to ensure compatibility with earlier version of the package.
It must either be empty (|\childdocmain{}|)
or precisely match the filename of the main file in which it is specified.
See \secref{sec:detection} for further information.
\item
The filename \textit{main} must be specified without the |.tex| extension.
\item
The filename \textit{main} is case sensitive
(even in case-insensitive file systems)
due to internal string comparison.
\item
The argument \textit{main} should be fully expanded, it cannot be a macro.
\item
Subdirectories and special characters should be avoided in filenames.
\item
The command |\childdocmain{|\textit{main}|}| must be followed by a whitespace.
It should not be followed immediately by another command
or by a comment mark `|%|'.
This is because the \TeX{} parser reads the token immediately following
the argument of |\childdocmain| and puts it
at the beginning of every child section;
however, a white\-space is ignored.
\end{itemize}

%%%%%%%%%%%%%%%%%%%%%%%%%%%%%%%%%%%%%%%%
\paragraph{Content of Main File.}

It is advisable to place all content in the child files included by |\include|.
Any output contained in the main file will appear in all child documents
unless suppressed manually;
it cannot be suppressed automatically by the |\includeonly| directive
and thus should normally be avoided.
A method to include some content in the main file
by means of conditional processing is described in \secref{sec:conditional}.

%%%%%%%%%%%%%%%%%%%%%%%%%%%%%%%%%%%%%%%%
\paragraph{Page Numbering.}

When only a part of the document is compiled,
the appropriate numbering of pages
(as well as other status parameters)
is determined from the |.aux| files.
The latter contain information from previous passes.
However this information needs to propagate through
all intermediate child documents.
Therefore the page numbering in child documents may well
be inconsistent until the complete document is compiled at least once.

A useful (if unconventional) way to always ensure a consistent
page numbering is to restart the numbering in each child document
and denote the pages by `\textit{child}|.|\textit{page}'
where \textit{child} represents the chapter/section number of the child file.
This can be achieved by the command
|\numberwithin{page}{|\textit{child}|}|
of the \textsf{amsmath} package
where \textit{child} can be |chapter| or |section|
depending on the chosen structuring.
Alternatively, one can modify the macro |\thepage| appropriately
and reset the counter |page| at the start of each child file.

%%%%%%%%%%%%%%%%%%%%%%%%%%%%%%%%%%%%%%%%%%%%%%%%%%%%%%%%%%%%%%%%%%%%%%%%%%%%%%%%
\subsection{Conditional Processing}
\label{sec:conditional}

The package provides a mechanism to compile different versions
of a document. To customise the versions further some conditional processing
can come in handy to distinguish which version is being compiled.
The package provides two macros to describe the compilation context:

%%%%%%%%%%%%%%%%%%%%%%%%%%%%%%%%%%%%%%%%
\DescribeMacro{\ifchilddoc}
The conditional |\ifchilddoc| distinguishes between the compilation of
child documents and the main document:
%
\begin{center}
|\ifchilddoc |\textit{child-code}| |[|\||else |\textit{main-code}]| \||fi|
\end{center}

%%%%%%%%%%%%%%%%%%%%%%%%%%%%%%%%%%%%%%%%
\DescribeMacro{\childdocname}
\DescribeMacro{\childdocjob}
The macro |\childdocname| contains the filename (without extension)
of the main or child file being processed.
Note that |\childdocjob| will always contain the name of the main file.

%%%%%%%%%%%%%%%%%%%%%%%%%%%%%%%%%%%%%%%%
\paragraph{Title Page.}

Conditional processing can be used to include a title or banner page
in the main document when proper precautions are taken.
Importantly, the code in the main file should ensure that the page counter
(as well as other status parameters which are stored in the |.aux| files)
takes the same value after the conditional processing.
Otherwise the page numbers may take divergent values
depending on which part is compiled.

For example, a title page could be declared by:
%
\begin{center}
\begin{tabular}{l}
|\ifchilddoc\||else|\\
|\addtocounter{page}{-1}|\\
\textit{code for title page}\\
|\newpage|\\
|\||fi|
\end{tabular}
\end{center}
%
A banner page for the child documents can be generated by:
%
\begin{center}
\begin{tabular}{l}
|\ifchilddoc|\\
|\addtocounter{page}{-1}|\\
\textit{code for banner page}\\
|\newpage|\\
|\||fi|
\end{tabular}
\end{center}
%
Here one could write a message such as:
\begin{center}
|This is the part \childdocname{} of \childdocjob{}.|
\end{center}

%%%%%%%%%%%%%%%%%%%%%%%%%%%%%%%%%%%%%%%%%%%%%%%%%%%%%%%%%%%%%%%%%%%%%%%%%%%%%%%%
\subsection{Flags}
\label{sec:flags}

The package makes it easy to generate different versions
of the main or child documents.
To this end compilation flags can be defined
and assigned different default values.
They will be particularly useful in conjunction
with the forwarding mechanism described in \secref{sec:forward}.

For example, it may be useful to have a flag |\version|
which can be set to |draft| or |final|.
The document source will contain some conditional code
depending on the value of |\version|.
Suppose further, the flag should default to |final| for the main file
and to |draft| for child files
which is a natural assignment for editing the document.
This is achieved by placing the following code
in the preamble of the main document
(below the |\childdocmain| directive):
%
\begin{center}
\begin{tabular}{l}
|\ifchilddoc|\\
|\providecommand{\version}{draft}|\\
|\||else|\\
|\providecommand{\version}{final}|\\
|\||fi|
\end{tabular}
\end{center}
%
The definition by |\providecommand| makes sure
that previous definitions are not overwritten.
Further statements |\providecommand{\version}{...}|
can thus be added before the above code to override it.

For the main file, one might add a line
(between |\childdocmain| and the above block)
%
\begin{center}
|%\ifchilddoc\||else\providecommand{\version}{draft}\||fi|
\end{center}
%
which can be uncommented to produce a draft version.
Likewise one can add a line to the very top of a child file
(above the |\childdocof{|\textit{main}|}| directive)
%
\begin{center}
|%\providecommand{\version}{final}|
\end{center}
%
which can be uncommented to produce the final version of this child document.

%%%%%%%%%%%%%%%%%%%%%%%%%%%%%%%%%%%%%%%%%%%%%%%%%%%%%%%%%%%%%%%%%%%%%%%%%%%%%%%%
\subsection{Forwarding}
\label{sec:forward}

Different versions of the main or child documents
using compilation flags as described in \secref{sec:flags}
can be (permanently) stored in different files
for convenient compilation, viewing and distribution.
To this end, the package defines a command
to pass on compilation to a different file:

%%%%%%%%%%%%%%%%%%%%%%%%%%%%%%%%%%%%%%%%
\DescribeMacro{\childdocforward}
The command |\childdocforward| redirects processing to
another source file:
%
\begin{center}
\begin{tabular}{l}
|\input{childdoc.def}|\\
|\childdocforward[|\textit{main}|]{|\textit{dest}|}|\\
\end{tabular}
\end{center}
%
The argument \textit{dest} is the destination file
(without extension).
It should be the main file or one of the child files.
Note that further \textsf{childdoc} directives
such as |\childdocof| and |\childdocforward|
in the indicated file will be processed in this form.
The optional argument \textit{main}
passes on directly to the main file \textit{main}
while pretending to compile the child \textit{dest}.
This form behaves as if \textit{dest}
issues |\childdocof{|\textit{main}|}| right away,
and no further \textsf{childdoc} directives will be processed.

%%%%%%%%%%%%%%%%%%%%%%%%%%%%%%%%%%%%%%%%
\DescribeMacro{\...prefix}
In the alternative form |\childdocforwardprefix|,
%
\begin{center}
\begin{tabular}{l}
|\input{childdoc.def}|\\
|\childdocforwardprefix[|\textit{main}|]{|\textit{prefix}|}{|\textit{dest}|}|
\end{tabular}
\end{center}
%
the destination file is determined by a pattern
depending on the current file:
To make this work, the current file must be called
`{\textit{prefix}\hspace{0.2em}\textit{suffix}}'
with \textit{prefix} matching precisely the argument.
Processing is then passed on to the file
`{\textit{dest}\hspace{0.2em}\textit{suffix}}'.
Surely, the same effect is achieved by
directly specifying the
argument `{\textit{dest}\hspace{0.2em}\textit{suffix}}'
in the first form.
However, that requires to set up a different file
for each child. With the alternative form of the command
all these files can have exactly the same content
which simplifies setting them up and maintaining them.

For example, the following file |draft.tex|
with a compilation flag |\version| as described in \secref{sec:flags}
compiles the main document as a draft:
%
\begin{center}
\begin{tabular}{l}
|\def\version{draft}|\\
|\input{childdoc.def}|\\
|\childdocforward{|\textit{main}|}|
\end{tabular}
\end{center}
%
Likewise, the following files |final|\textit{nn}|.tex|
compile the final version of the child document
|child|\textit{nn}|.tex|:
%
\begin{center}
\begin{tabular}{l}
|\def\version{final}|\\
|\input{childdoc.def}|\\
|\childdocforwardprefix{final}{child}|
\end{tabular}
\end{center}
%

Note that when several versions of a main file and/or of each child file
are to be generated, it may be convenient to set up a |Makefile| or
shell script to automatise the process.

%%%%%%%%%%%%%%%%%%%%%%%%%%%%%%%%%%%%%%%%%%%%%%%%%%%%%%%%%%%%%%%%%%%%%%%%%%%%%%%%
\subsection{Command Line Processing}
\label{sec:commandline}

The effect of redirection files can also be achieved by invoking
the \LaTeX{} compiler with a more elaborate command line.
Most conveniently this should be done as part
of a shell script or a |Makefile|.

When using \textsf{childdoc} in the main file, the following
command lines effectively perform a redirection
(note that depending on the shell being used,
backslashes may have to be doubled: `|\|' $\to$ `|\\|'):
%
\begin{center}
|... -jobname "|\textit{target}|" |\\|"|[\textit{flags}]%
|\input{childdoc.def}\childdocforward[|\textit{main}|]{|\textit{dest}|}"|
\end{center}
%
Here \textit{target} is the name of the output file,
\textit{main} is the name of the main file
and \textit{dest} is the name of the main or child file to be processed
(all filenames without extensions).
The optional argument \textit{main} can be omitted
if \textit{main} matches \textit{dest}.
Optionally, compilation \textit{flags} can be defined via |\def| commands.
This command line makes the \TeX{} engine believe
it is compiling the file \textit{target}
whose content is specified as the latter parameter.
The provided code then forwards the processing to
\textit{main} or \textit{dest} as described in \secref{sec:forward}.

%%%%%%%%%%%%%%%%%%%%%%%%%%%%%%%%%%%%%%%%%%%%%%%%%%%%%%%%%%%%%%%%%%%%%%%%%%%%%%%%
\subsection{Include by Input}
\label{sec:input}

Including child documents by |\include| has some restrictions by design.
Most notably, the content of a child document always occupies
its own set of pages; pages cannot be shared between child documents.
Usually, this behaviour makes perfect sense
because each child document contain an essential part of the document.
However, in some situations it may be desirable to compose
a document from a collection of parts
without having mandatory page breaks between then.
For this case, the package
provides a mechanism to include parts
by |\input| which can also be processed individually.
However, by construction this mechanism
requires manual handling of the content to be output.

%%%%%%%%%%%%%%%%%%%%%%%%%%%%%%%%%%%%%%%%
\DescribeMacro{\ifchilddocmanual}
The main file should be prepared as usual, see \secref{sec:include}.
However, the document body must make a distinction
between processing of an individual part and of the main document, e.g.:
%
\begin{center}
\begin{tabular}{l}
|\ifchilddocmanual|\\
|\input{\childdocname}|\\
|\||else|\\
\textit{document body with }|\input{|\textit{part}|}|\\
|\||fi|
\end{tabular}
\end{center}
%
The conditional |\ifchilddocmanual| is true whenever
a part to be included by |\input| is being compiled,
and the name of the part is stored in |\childdocname|.

%%%%%%%%%%%%%%%%%%%%%%%%%%%%%%%%%%%%%%%%
\DescribeMacro{\childdocby}
Each part to be included by |\input| should start with:
%
\begin{center}
\begin{tabular}{l}
|\input{childdoc.def}|\\
|\childdocby{|\textit{main}|}|\\
\end{tabular}
\end{center}
%
The directive |\childdocby| is similar to |\childdocof|
described in \secref{sec:include},
but the subsequent selection of content must be done manually.
To that end, both |\ifchilddoc| and |\ifchilddocmanual|
will be true upon processing of a part,
and the name of the part is stored in |\childdocname|.
Note that |\jobname| will be set to the filename of the current part
so that each part receives an individual |.aux| file
that does not interfere with the |.aux| file(s) of the main document.
This behaviour can be altered by the alternative form
|\childdocby[*]{|\textit{main}|}| (with a non-empty optional argument)
which uses the |.aux| file of the main document
by setting |\jobname| to \textit{main}.

%%%%%%%%%%%%%%%%%%%%%%%%%%%%%%%%%%%%%%%%%%%%%%%%%%%%%%%%%%%%%%%%%%%%%%%%%%%%%%%%
\subsection{Driver Development}
\label{sec:driver}

The \textsf{childdoc} mechanism can also be use for the development
of definition files such as \LaTeX{} styles or classes.
This case differs from the above setup with multiple parts
included by |\include| in that no |\includeonly| should be invoked.
This can be achieved by starting the include file
(before |\ProvidesPackage|) with:
%
\begin{center}
\begin{tabular}{l}
|\input{childdoc.def}|\\
|\childdocforward{|\textit{main}|}|\\
\end{tabular}
\end{center}
%
or alternatively with:
%
\begin{center}
\begin{tabular}{l}
|\input{childdoc.def}|\\
|\childdocby{|\textit{main}|}|\\
\end{tabular}
\end{center}
%
Both forms have slightly different effects as described above.
The main file is prepared as usual, see \secref{sec:include}.

%%%%%%%%%%%%%%%%%%%%%%%%%%%%%%%%%%%%%%%%%%%%%%%%%%%%%%%%%%%%%%%%%%%%%%%%%%%%%%%%
\subsection{Legacy Detection}
\label{sec:detection}

The directive |\childdocmain| in the main file can detect
whether the complete document or merely a child is to be compiled
even without using the directive |\childdocof|.
This method is deprecated because it is less robust
and there is no compelling reason to use it;
it is merely provided for backward compatibility
and it may be removed in future versions.

If the detection mechanism is to be used,
it is mandatory to correctly specify
the filename of the main file as the argument of |\childdocmain|:
%
\begin{center}
\begin{tabular}{l}
|\input{childdoc.def}|\\
|\childdocmain{|\textit{main}|}|\\
\end{tabular}
\end{center}
%
If |\jobname| does not match the argument \textit{main} of |\childdocmain|,
it is assumed that |\jobname| points to the child file to be compiled.
When using |\childdocmain| with the main file specified as argument,
it suffices to start a child file
with just |\input{|\textit{main}|}|
without loading of the package and using |\childdocof|.
If instead all processing is done
with the appropriate \textsf{childdoc} directives,
the argument of \textit{main} of |\childdocmain| can be empty.

An alternative version of the command line processing described
in \secref{sec:commandline} using the detection mechanism reads:
%
\begin{center}
|... -jobname "|\textit{target}|" "|[\textit{flags}]%
[|\def\jobname{|\textit{dest}|}|]|\input{|\textit{main}|}"|
\end{center}

%%%%%%%%%%%%%%%%%%%%%%%%%%%%%%%%%%%%%%%%%%%%%%%%%%%%%%%%%%%%%%%%%%%%%%%%%%%%%%%%
\subsection{Manual Code}
\label{sec:manual}

In case one cannot be certain whether the definitions file |childdoc.def|
is installed on the target \TeX{} distribution
and one prefers not to ship it,
it is conceivable to paste a few relevant commands into the sources.

To that end, drop all statements |\input{childdoc.def}|
and perform the replacements as outlined below.
Instead of |\childdocmain{|\textit{main}|}| add the following code
to the top of the main file:
%
\begin{center}
\begin{tabular}{l}
|\||ifdefined\childdocname\endinput\||fi\newif\ifchilddoc|\\
|\edef\childdocname{\scantokens\expandafter{\jobname\noexpand}}|\\
|\def\childdocmain{|\textit{main}|}\||ifx\childdocmain\childdocname\||else|\\
|\childdoctrue\includeonly{\childdocname}\let\jobname\childdocmain\||fi|\\
\end{tabular}
\end{center}
%
Instead of |\childdocof{|\textit{main}|}| just include the main file
at the top of each child file:
%
\begin{center}
|\input{|\textit{main}|}|
\end{center}
%
A simple redirection |\childdocforward{|\textit{dest}|}| is achieved by:
%
\begin{center}
|\def\jobname{|\textit{dest}|}\input{\jobname}|
\end{center}
%
The redirection with prefix
|\childdocforwardprefix[|\textit{prefix}|]{|\textit{dest}|}|
is accomplished by:
%
\begin{center}
\begin{tabular}{l}
|{\edef\jobname{\scantokens\expandafter{\jobname\noexpand}}|\\
|\def\redirectjob |\textit{prefix}|#1~~~{\gdef\jobname{|\textit{dest}|#1}}|\\
|\expandafter\redirectjob\jobname~~~}\input{\jobname}|
\end{tabular}
\end{center}

In an alternative approach,
child documents can be compiled by a specific command line
without additional code or specific definitions:
%
\begin{center}
|... -jobname "|\textit{target}|" "|[\textit{flags}]%
|\includeonly{|\textit{dest}|}\input{|\textit{main}|}"|
\end{center}
%

%%%%%%%%%%%%%%%%%%%%%%%%%%%%%%%%%%%%%%%%%%%%%%%%%%%%%%%%%%%%%%%%%%%%%%%%%%%%%%%%
%%%%%%%%%%%%%%%%%%%%%%%%%%%%%%%%%%%%%%%%%%%%%%%%%%%%%%%%%%%%%%%%%%%%%%%%%%%%%%%%
\section{Information}

%%%%%%%%%%%%%%%%%%%%%%%%%%%%%%%%%%%%%%%%%%%%%%%%%%%%%%%%%%%%%%%%%%%%%%%%%%%%%%%%
\subsection{Copyright}

Copyright \copyright{} 2017--2018 Niklas Beisert

This work may be distributed and/or modified under the
conditions of the \LaTeX{} Project Public License, either version 1.3
of this license or (at your option) any later version.
The latest version of this license is in
  \url{http://www.latex-project.org/lppl.txt}
and version 1.3 or later is part of all distributions of \LaTeX{}
version 2005/12/01 or later.

This work has the LPPL maintenance status `maintained'.

The Current Maintainer of this work is Niklas Beisert.

This work consists of the files |README.txt|, |childdoc.ins| and |childdoc.dtx|
as well as the derived files |childdoc.def|, |cdocsamp.tex|
with |cdocsch1.tex|, |cdocsch2.tex|, |cdocspt3.tex|, |cdocspt4.tex|,
|cdocsdrf.tex|, |cdocsfn1.tex|, |cdocsfn2.tex|
as well as |childdoc.pdf|.

%%%%%%%%%%%%%%%%%%%%%%%%%%%%%%%%%%%%%%%%%%%%%%%%%%%%%%%%%%%%%%%%%%%%%%%%%%%%%%%%
\subsection{Files and Installation}

The package consists of the files:
%
\begin{center}
\begin{tabular}{ll}
    |README.txt|   & readme file \\
    |childdoc.ins| & installation file \\
    |childdoc.dtx| & source file \\
    |childdoc.def| & definition file \\
    |cdocsamp.tex| & sample main file \\
    |cdocsch1.tex| & sample include file \\
    |cdocsch2.tex| & sample include file \\
    |cdocspt3.tex| & sample part file \\
    |cdocspt4.tex| & sample part file \\
    |cdocsdrf.tex| & sample redirection file \\
    |cdocsfn1.tex| & sample redirection file \\
    |cdocsfn2.tex| & sample redirection file \\
    |childdoc.pdf| & manual
\end{tabular}
\end{center}
%
The distribution consists of the files
|README.txt|, |childdoc.ins| and |childdoc.dtx|.
%
\begin{itemize}
\item
Run (pdf)\LaTeX{} on |childdoc.dtx|
to compile the manual |childdoc.pdf| (this file).
\item
Run \LaTeX{} on |childdoc.ins| to create the definitions file |childdoc.def|
and the sample |cdocsamp.tex| with include files
|cdocsch1.tex|, |cdocsch2.tex|, |cdocspt3.tex|, |cdocspt4.tex|,
|cdocsdrf.tex|, |cdocsfn1.tex|, |cdocsfn2.tex|.
Then copy the file |childdoc.def| to an appropriate directory of your \LaTeX{}
distribution, e.g.\ \textit{texmf-root}|/tex/latex/childdoc|.
\end{itemize}

%%%%%%%%%%%%%%%%%%%%%%%%%%%%%%%%%%%%%%%%%%%%%%%%%%%%%%%%%%%%%%%%%%%%%%%%%%%%%%%%
\subsection{Related CTAN Packages}

There are several other packages which offer a similar functionality:
%
\begin{itemize}
\item
The packages
\href{http://ctan.org/pkg/docmute}{\textsf{docmute}},
\href{http://ctan.org/pkg/includex}{\textsf{includex}} and
\href{http://ctan.org/pkg/standalone}{\textsf{standalone}}
provide commands to include only the document body of
a child file thus allowing both files to be compiled individually.
\item
The packages \href{http://ctan.org/pkg/subdocs}{\textsf{subdocs}}
and \href{http://ctan.org/pkg/subfiles}{\textsf{subfiles}}
provide structures in which the main and child documents can be
encapsulated and allowing them to be compiled individually.
The inclusion mechanism is different from the conventional |\include|.
\item
The package \href{http://ctan.org/pkg/combine}{\textsf{combine}}
is an elaborate solution to combine several documents into one.
\end{itemize}
%
See also the CTAN topic \href{http://ctan.org/topic/subdocs}{\textsf{subdocs}}
for further related packages.
The present package differs from the above solutions in that
a document structure constructed with the conventional |\include| mechanism
just needs two extra commands at the top of every file
such that all constituent files can be compiled individually.

%%%%%%%%%%%%%%%%%%%%%%%%%%%%%%%%%%%%%%%%%%%%%%%%%%%%%%%%%%%%%%%%%%%%%%%%%%%%%%%%
%\subsection{Feature Suggestions}
%
%The following is a list of features which may be useful for future
%versions of this package:
%%
%\begin{itemize}
%\item
%\ldots
%\end{itemize}

%%%%%%%%%%%%%%%%%%%%%%%%%%%%%%%%%%%%%%%%%%%%%%%%%%%%%%%%%%%%%%%%%%%%%%%%%%%%%%%%
\subsection{Revision History}

%%%%%%%%%%%%%%%%%%%%%%%%%%%%%%%%%%%%%%%%
\paragraph{v2.0:} 2018/12/30

\begin{itemize}
\item
immediate forward processing
\item
added |\childdocby| mechanism
\item
manual restructured
\end{itemize}

%%%%%%%%%%%%%%%%%%%%%%%%%%%%%%%%%%%%%%%%
\paragraph{v1.6:} 2018/01/17

\begin{itemize}
\item
application for development of include files
\item
corrections to manual
\end{itemize}

%%%%%%%%%%%%%%%%%%%%%%%%%%%%%%%%%%%%%%%%
\paragraph{v1.5:} 2017/05/21

\begin{itemize}
\item
more complete structuring introduced
\item
|\childdocof| introduced
\item
|\childdoc| renamed to |\childdocmain|
\item
|\childredirect| renamed to |\childdocforward| and |\childdocforwardprefix|
and functionality expanded
\end{itemize}

%%%%%%%%%%%%%%%%%%%%%%%%%%%%%%%%%%%%%%%%
\paragraph{v1.0:} 2017/04/27

\begin{itemize}
\item
manual and install package
\item
first version published on CTAN
\end{itemize}

%%%%%%%%%%%%%%%%%%%%%%%%%%%%%%%%%%%%%%%%
\paragraph{v0.6:} 2017/04/26

\begin{itemize}
\item
redirection mechanism added
\end{itemize}

%%%%%%%%%%%%%%%%%%%%%%%%%%%%%%%%%%%%%%%%
\paragraph{v0.5:} 2017/04/26

\begin{itemize}
\item
functionality in definition file
\end{itemize}


%%%%%%%%%%%%%%%%%%%%%%%%%%%%%%%%%%%%%%%%%%%%%%%%%%%%%%%%%%%%%%%%%%%%%%%%%%%%%%%%
%%%%%%%%%%%%%%%%%%%%%%%%%%%%%%%%%%%%%%%%%%%%%%%%%%%%%%%%%%%%%%%%%%%%%%%%%%%%%%%%
%%%%%%%%%%%%%%%%%%%%%%%%%%%%%%%%%%%%%%%%%%%%%%%%%%%%%%%%%%%%%%%%%%%%%%%%%%%%%%%%
\appendix

\settowidth\MacroIndent{\rmfamily\scriptsize 000\ }

 \DocInput{childdoc.dtx}

\end{document}
%</driver>
% \fi
%
% %%%%%%%%%%%%%%%%%%%%%%%%%%%%%%%%%%%%%%%%%%%%%%%%%%%%%%%%%%%%%%%%%%%%%%%%%%%%%%
% %%%%%%%%%%%%%%%%%%%%%%%%%%%%%%%%%%%%%%%%%%%%%%%%%%%%%%%%%%%%%%%%%%%%%%%%%%%%%%
% \section{Sample}
%\iffalse
%<*samplemain>
%\fi
%
% The following presents a sample document
% with two chapters, two parts, a title page,
% a compile flag as well as three forwarding files to set the flag.
% It consists of eight |.tex| files:
% \begin{center}
% \begin{tabular}{ll}
% |cdocsamp.tex|&main file\\
% |cdocsch1.tex|&include file for chapter 1\\
% |cdocsch2.tex|&include file for chapter 2\\
% |cdocspt3.tex|&include file for part 3\\
% |cdocspt4.tex|&include file for part 4\\
% |cdocsdrf.tex|&forwarding file for main file in draft mode\\
% |cdocsfi1.tex|&forwarding file for final version of chapter 1\\
% |cdocsfi2.tex|&forwarding file for final version of chapter 2\\
% \end{tabular}
% \end{center}
% Each of the eight files can be compiled directly by the \LaTeX{} compiler.
%
% %%%%%%%%%%%%%%%%%%%%%%%%%%%%%%%%%%%%%%
% \paragraph{Main File.}
%
% The main file is called |cdocsamp.tex|.
%
% Load the \textsf{childdoc} definitions and
% declare the filename for the main document:
%    \begin{macrocode}
\input{childdoc.def}
\childdocmain{}
%    \end{macrocode}

% Optional override for |\version| flag:
%    \begin{macrocode}
%%\ifchilddoc\else\providecommand{\version}{draft}\fi
%    \end{macrocode}

% Define the default values for the |\version| flag
% (|final| for the main file and |draft| for childs):
%    \begin{macrocode}
\ifchilddoc
\providecommand{\version}{draft}
\else
\providecommand{\version}{final}
\fi
%    \end{macrocode}

% Load the standard document class:
%    \begin{macrocode}
\documentclass[12pt]{article}
%    \end{macrocode}

% Start the document body:
%    \begin{macrocode}
\begin{document}
%    \end{macrocode}

% Declare a title page.
% Print title, part of document being processed and version flag:
%    \begin{macrocode}
\addtocounter{page}{-1}
\begin{center}
{\LARGE\bfseries{}childdoc example\par}
\vspace{1cm}
\ifchilddoc
\ifchilddocmanual part\else chapter\fi:
`\childdocname' of `\childdocjob'\par
\else
main document: `\childdocjob'\par
\fi
version: \version\par
\end{center}
\newpage
%    \end{macrocode}

% Manually include selected file,
% otherwise process as usual:
%    \begin{macrocode}
\ifchilddocmanual
\section*{part `\childdocname'}
\input{\childdocname}
\else
%    \end{macrocode}

% Include the two chapters:
%    \begin{macrocode}
\include{cdocsch1}
\include{cdocsch2}
%    \end{macrocode}

% Include the two parts unless only chapters should be displayed:
%    \begin{macrocode}
\ifchilddoc\else
\section{part three}
\input{cdocspt3}
\section{part four}
\input{cdocspt4}
\fi
%    \end{macrocode}

% Process as usual until here:
%    \begin{macrocode}
\fi
%    \end{macrocode}

% End of document body:
%    \begin{macrocode}
\end{document}
%    \end{macrocode}
%\iffalse
%</samplemain>
%\fi
%
% %%%%%%%%%%%%%%%%%%%%%%%%%%%%%%%%%%%%%%
% \paragraph{Chapter Include Files.}
%
% The include files are called |cdocsch1.tex| and |cdocsch2.tex|.
%
%\iffalse
%<*samplechap1|samplechap2>
%\fi

% Optional override for |\version| flag:
%    \begin{macrocode}
%%\providecommand{\version}{final}
%    \end{macrocode}

% Include the main document:
%    \begin{macrocode}
\input{childdoc.def}
\childdocof{cdocsamp}
%    \end{macrocode}

%\iffalse
%</samplechap1|samplechap2>
%\fi
%
%\iffalse
%<*samplechap1>
%\fi
% Some text for chapter 1:
%    \begin{macrocode}
\section{one}
some text in chapter one
%    \end{macrocode}

%\iffalse
%</samplechap1>
%\fi
% Some text for chapter 2:
%\iffalse
%<*samplechap2>
%\fi
%    \begin{macrocode}
\section{two}
more text in chapter two
%    \end{macrocode}

%\iffalse
%</samplechap2>
%\fi
%
% %%%%%%%%%%%%%%%%%%%%%%%%%%%%%%%%%%%%%%
% \paragraph{Part Include Files.}
%
% The include files are called |cdocspt3.tex| and |cdocspt4.tex|.
%
%\iffalse
%<*samplepart3|samplepart4>
%\fi

% Optional override for |\version| flag:
%    \begin{macrocode}
%%\providecommand{\version}{final}
%    \end{macrocode}

% Include the main document:
%    \begin{macrocode}
\input{childdoc.def}
\childdocby{cdocsamp}
%    \end{macrocode}

%\iffalse
%</samplepart3|samplepart4>
%\fi
%
%\iffalse
%<*samplepart3>
%\fi
% Some text for part 3:
%    \begin{macrocode}
some text in part three
%    \end{macrocode}

%\iffalse
%</samplepart3>
%\fi
% Some text for part 4:
%\iffalse
%<*samplepart4>
%\fi
%    \begin{macrocode}
more text in part four
%    \end{macrocode}

%\iffalse
%</samplepart4>
%\fi
%
% %%%%%%%%%%%%%%%%%%%%%%%%%%%%%%%%%%%%%%
% \paragraph{Forwarding for a Complete Draft.}
%
% The following forwarding file |cdocsdrf.tex|
% compiles the main document in draft mode:
%\iffalse
%<*sampledraft>
%\fi
%    \begin{macrocode}
\def\version{draft}
\input{childdoc.def}
\childdocforward{cdocsamp}
%    \end{macrocode}

%\iffalse
%</sampledraft>
%\fi
%
% %%%%%%%%%%%%%%%%%%%%%%%%%%%%%%%%%%%%%%
% \paragraph{Forwarding for Final Version of the Chapters.}
%
% The following forwarding files |cdocsfn1.tex| and |cdocsfn2.tex|
% (with identical content)
% compile the final versions of the child documents
% |cdocsch1.tex| and |cdocsch2.tex|, respectively:
%\iffalse
%<*samplefinal>
%\fi
%    \begin{macrocode}
\def\version{final}
\input{childdoc.def}
\childdocforwardprefix[cdocsamp]{cdocsfn}{cdocsch}
%    \end{macrocode}

%\iffalse
%</samplefinal>
%\fi
%
% %%%%%%%%%%%%%%%%%%%%%%%%%%%%%%%%%%%%%%
% \paragraph{Command Line Processing.}
%
% The following three command lines generate the output files
% |cdocscld|, |cdocscl1| and |cdocscl2|
% which should be identical to
% |cdocsdrf|, |cdocsch1| and |cdocsfn2|, respectively:
% \begin{center}
% \begin{tabular}{l}
% |latex -jobname cdocscld \|\\
% |  "\def\version{draft}\input{childdoc.def}\childdocforward{cdocsamp}"|\\
% |latex -jobname cdocscl1 \|\\
% |  "\input{childdoc.def}\childdocforward[cdocsamp]{cdocsch1}"|\\
% |latex -jobname cdocscl2 \|\\
% |  "\def\version{final}\input{childdoc.def}\childdocforward{cdocsch2}"|
% \end{tabular}
% \end{center}
% Note that the trailing backslash on each first line
% merely continues the input to the second line
% (for convenient cut ant paste).
% Furthermore, the command |latex| can be replaced by any
% of its alternative versions such as |pdflatex|.
%
% %%%%%%%%%%%%%%%%%%%%%%%%%%%%%%%%%%%%%%%%%%%%%%%%%%%%%%%%%%%%%%%%%%%%%%%%%%%%%%
% %%%%%%%%%%%%%%%%%%%%%%%%%%%%%%%%%%%%%%%%%%%%%%%%%%%%%%%%%%%%%%%%%%%%%%%%%%%%%%
% \section{Implementation}
%\iffalse
%<*package>
%\fi
%
% This section describes the definitions file |childdoc.def|.

% The definitions cannot be loaded using |\usepackage| or |\RequirePackage|
% which has a mechanism to prevent loading a style file more than once.
% When loading the definitions by means of |\input|
% multiple instances have to be prevented manually:
%\iffalse
%This code needs to be before the `\ProvidesFile' directive
%which is defined at the beginning of this file.
%Therefore it is also placed there and commented out here.
%</package>
%<*discard>
%\fi
%    \begin{macrocode}
\ifdefined\childdocmain\endinput\fi
%    \end{macrocode}
%\iffalse
%</discard>
%<*package>
%\fi
%
% \macro{\ifchilddoc}
% \macro{\ifchilddocmanual}
% The conditional |\ifchilddoc| tells whether a
% child (true) or main (false) document is being compiled.
% The conditional |\ifchilddocmanual| tells whether
% the |\includeonly| mechanism is used (false) or
% the selection of child files must be performed manually (true).
% The definitions initialise to false:
%    \begin{macrocode}
\newif\ifchilddoc
\newif\ifchilddocmanual
%    \end{macrocode}

% \macro{\childdocname}
% \macro{\childdocjob}
% The macro |\childdocname| stores the name of the main document
% to be compiled. The macro |\childdocjob| stores the name of
% the document on which the \LaTeX{} compiler was originally invoked.
% The content of |\jobname| cannot be compared
% to filenames specified in the source due to different catcodes.
% The following code rescans |\jobname|, stores the result
% in |\childdocname| and saves a copy in |\childdocjob|:
%    \begin{macrocode}
\edef\childdocname{\scantokens\expandafter{\jobname\noexpand}}
\let\childdocjob\childdocname
%    \end{macrocode}

% \macro{\childdocdisable}
% The macro |\childdocdisable| prevents the main file
% from being processed more than once.
% At this stage, the main document command |\childdocmain|
% is assumed to be called once again where it should do nothing.
% Any subsequent call to it should prevent
% a secondary processing of the main document
% It overwrites the forwarding commands
% |\childdocof| and |\childdocforward|
% with empty macros to prevent further inclusions of the main document:
%    \begin{macrocode}
\newcommand{\childdocdisable}
{
  \renewcommand{\childdocmain}[1]{\renewcommand{\childdocmain}[1]{\endinput}}
  \renewcommand{\childdocof}[1]{}
  \renewcommand{\childdocby}[2][]{}
  \renewcommand{\childdocforward}[2][]{}
  \renewcommand{\childdocdisable}{}
}
%    \end{macrocode}

% \macro{\childdocmain}
% The macro |\childdocmain| is to be called at the top of the main file
% with nothing or the main filename (without extension) as argument.
% First, it breaks loops.
% If the argument is not empty and does not match |\childdocname|
% (which is set by the first inclusion of |childdoc.def|),
% |\ifchilddoc| is set to true, |\includeonly| is applied to the child file
% and |\jobname| is set to the main file
% (for proper handling of |.aux| files):
%    \begin{macrocode}
\newcommand{\childdocmain}[1]
{
  \childdocdisable\childdocmain{}
  \if?#1?\else
    \begingroup
      \def\childdoctmp{#1}
      \ifx\childdoctmp\childdocname
        \def\childdoctmp{}
      \else
        \def\childdoctmp
        {
          \childdoctrue
          \includeonly{\childdocname}
          \def\childdocjob{#1}
          \def\jobname{#1}
        }
      \fi
      \expandafter
    \endgroup
    \childdoctmp
  \fi
}
%    \end{macrocode}

% \macro{\childdocof}
% The command |\childdocof| redirects
% compilation to the main file |#1|.
%    \begin{macrocode}
\newcommand{\childdocof}[1]
{
  \childdocdisable
  \childdoctrue
  \includeonly{\childdocname}
  \def\jobname{#1}
  \def\childdocjob{#1}
  \input{#1}
}
%    \end{macrocode}

% \macro{\childdocby}
% The command |\childdocby| ....
%    \begin{macrocode}
\newcommand{\childdocby}[2][]
{
  \childdocdisable
  \childdoctrue
  \childdocmanualtrue
  \if?#1?\else
    \def\jobname{#2}
  \fi
  \def\childdocjob{#2}
  \input{#2}
  \endinput
}
%    \end{macrocode}

% \macro{\childdocforward}
% The command |\childdocforward| redirects
% compilation to the main file or
% (if the optional argument is given) a child file.
% Parameters are set as if the main file
% or a child file starting with |\childdocof| was compiled.
% Then compilation is handed over to the main file:
%    \begin{macrocode}
\newcommand{\childdocforward}[2][]
{
  \begingroup
    \if?#1?
      \def\childdoctmp
      {
        \def\childdocname{#2}
        \def\childdocjob{#2}
        \def\jobname{#2}
        \input{#2}
        \endinput
      }
    \else
      \def\childdoctmp
      {
        \childdocdisable
        \def\childdocname{#2}
        \childdoctrue
        \includeonly{#2}
        \def\childdocjob{#1}
        \def\jobname{#1}
        \input{#1}
        \endinput
      }
    \fi
    \expandafter
  \endgroup
  \childdoctmp
}
%    \end{macrocode}

% \macro{\childdocforwardprefix}
% The command |\childdocforwardprefix| redirects
% compilation to the main or a child file by means of a pattern.
% The prefix |#1| in the current filename is replaced by |#2|
% and the suffix of the current filename is kept
% (it is assumed that the filename does not contain the substring `|~~~|'
% which is used as a delimiter).
% Compilation is handed over to the new file by |\childdocforward|:
%    \begin{macrocode}
\newcommand{\childdocforwardprefix}[3][]
{
  \begingroup
    \def\childdocextract #2##1~~~{\def\childdoctmp{\childdocforward[#1]{#3##1}}}
    \expandafter\childdocextract\childdocname~~~
    \expandafter
  \endgroup
  \childdoctmp
}
%    \end{macrocode}

% \macro{\childdoc}
% The deprecated macro |\childdoc| is a legacy version of |\childdocmain|:
%    \begin{macrocode}
\newcommand{\childdoc}{\childdocmain}
%    \end{macrocode}

% \macro{\childdocredirect}
% The deprecated macro |\childdocredirect| is a legacy version
% of |\childdocforward| and |\childdocforwardprefix|:
%    \begin{macrocode}
\newcommand{\childdocredirect}[2][]
{
  \begingroup
    \if?#1?
      \def\childdoctmp{\childdocforward{#2}}
    \else
      \def\childdoctmp{\childdocforwardprefix{#1}{#2}}
    \fi
    \expandafter
  \endgroup
  \childdoctmp
}
%    \end{macrocode}

%\iffalse
%</package>
%\fi
%
\endinput
|\\
|\childdocforward{|\textit{main}|}|
\end{tabular}
\end{center}
%
Likewise, the following files |final|\textit{nn}|.tex|
compile the final version of the child document
|child|\textit{nn}|.tex|:
%
\begin{center}
\begin{tabular}{l}
|\def\version{final}|\\
|% \iffalse
%
% childdoc.dtx Copyright (C) 2017-2018 Niklas Beisert
%
% This work may be distributed and/or modified under the
% conditions of the LaTeX Project Public License, either version 1.3
% of this license or (at your option) any later version.
% The latest version of this license is in
%   http://www.latex-project.org/lppl.txt
% and version 1.3 or later is part of all distributions of LaTeX
% version 2005/12/01 or later.
%
% This work has the LPPL maintenance status `maintained'.
%
% The Current Maintainer of this work is Niklas Beisert.
%
% This work consists of the files childdoc.dtx and childdoc.ins
% and the derived files childdoc.def and cdocsamp.tex with
% cdocsch1.tex, cdocsch2.tex, cdocsdrf.tex, cdocsfn1.tex, cdocsfn2.tex.
%
%<package>\ifdefined\childdocmain\endinput\fi
%<package>\ProvidesFile{childdoc.def}[2018/12/30 v2.0 child document driver]
%<samplemain>\ProvidesFile{cdocsamp.tex}[2018/12/30 v2.0 sample for childdoc]
%<*driver>
%\ProvidesFile{childdoc.drv}[2018/12/30 v2.0 childdoc reference manual file]
\PassOptionsToClass{10pt,a4paper}{article}
\documentclass{ltxdoc}

\usepackage[margin=35mm]{geometry}
\usepackage{hyperref}
\usepackage{hyperxmp}
\usepackage[usenames]{color}

\hypersetup{colorlinks=true}
\hypersetup{pdfstartview=FitH}
\hypersetup{pdfpagemode=UseNone}
\hypersetup{pdfsource={}}
\hypersetup{pdflang={en-UK}}
\hypersetup{pdfcopyright={Copyright 2017-2018 Niklas Beisert.
  This work may be distributed and/or modified under the
  conditions of the LaTeX Project Public License, either version 1.3
  of this license or (at your option) any later version.}}
\hypersetup{pdflicenseurl={http://www.latex-project.org/lppl.txt}}
\hypersetup{pdfcontactaddress={ETH Zurich, ITP, HIT K,
  Wolfgang-Pauli-Strasse 27}}
\hypersetup{pdfcontactpostcode={8093}}
\hypersetup{pdfcontactcity={Zurich}}
\hypersetup{pdfcontactcountry={Switzerland}}
\hypersetup{pdfcontactemail={nbeisert@itp.phys.ethz.ch}}
\hypersetup{pdfcontacturl={http://people.phys.ethz.ch/\xmptilde nbeisert/}}

\newcommand{\secref}[1]{\hyperref[#1]{section \ref*{#1}}}

\parskip1ex
\parindent0pt
\let\olditemize\itemize
\def\itemize{\olditemize\parskip0pt}

\begin{document}

\title{The \textsf{childdoc} Package}
\hypersetup{pdftitle={The childdoc Package}}
\author{Niklas Beisert\\[2ex]
  Institut f\"ur Theoretische Physik\\
  Eidgen\"ossische Technische Hochschule Z\"urich\\
  Wolfgang-Pauli-Strasse 27, 8093 Z\"urich, Switzerland\\[1ex]
  \href{mailto:nbeisert@itp.phys.ethz.ch}
  {\texttt{nbeisert@itp.phys.ethz.ch}}}
\hypersetup{pdfauthor={Niklas Beisert}}
\hypersetup{pdfsubject={Manual for the LaTeX2e Package childdoc}}
\date{30 December 2018, \textsf{v2.0}}
\maketitle

\begin{abstract}\noindent
\textsf{childdoc} is a \LaTeXe{} package
that enables the direct compilation
of document sections included by |\include|
to individual files.
\end{abstract}

\begingroup
\parskip0ex
\tableofcontents
\endgroup

%%%%%%%%%%%%%%%%%%%%%%%%%%%%%%%%%%%%%%%%%%%%%%%%%%%%%%%%%%%%%%%%%%%%%%%%%%%%%%%%
%%%%%%%%%%%%%%%%%%%%%%%%%%%%%%%%%%%%%%%%%%%%%%%%%%%%%%%%%%%%%%%%%%%%%%%%%%%%%%%%
\section{Introduction}

\LaTeX{} provides a mechanism to structure a large document (such as a book)
into a main file and several child files (containing the chapters)
using the |\include| command.
This mechanism is beneficial for documents
which span hundreds of pages in order to
make the source file(s) more manageable.
Moreover, compilation can be restricted to
selected child files by means of the |\includeonly| command.
The latter feature can be used to reduce the compilation time while editing
(this was significantly more useful in the earlier days of \LaTeX{})
or to generate a smaller document which is easier to navigate.
Another application of |\includeonly| is to generate
documents consisting of selected parts of the complete document.

However, there are a few drawbacks of the plain |\include| mechanism:
\begin{itemize}
\item
The child files cannot be compiled on their own,
they can only be compiled via the main file.
A naive editing environment
(such as a text editor with an option
to have the current file processed by \LaTeX)
may require one to switch to the main file before compiling;
attempting to compile the child file produces errors.
\item
The main file must be modified (each time)
to adjust the |\includeonly| command
to the present needs. This easily leaves the main file in a messy state.
\item
The generated document will always carry the filename
of the main document. This is inconvenient if
several child files are to be compiled and
to be kept for distribution.
\end{itemize}

The present package provides a simple interface
to make child files individually compilable by \LaTeX{}.
Compiling a child file then has the same effect as compiling
the main file with an |\includeonly| command
to select the appropriate child.
Moreover the generated document will carry the name of the child
rather than the main file.
This resolves all three above issues.

This feature is meant to make the editing of books,
thesis documents and lecture notes somewhat more convenient.
However, the package can also be used efficiently for
composing a series of documents (such as exercise sheets)
which are typically distributed individually.
It then assists the author in generating the individual documents
(potentially in different versions)
as well as a document containing the collected series.
Another application is in developing style files
or other kinds of included material
where compilation of the style file could redirect
to a sample or test file.

%%%%%%%%%%%%%%%%%%%%%%%%%%%%%%%%%%%%%%%%%%%%%%%%%%%%%%%%%%%%%%%%%%%%%%%%%%%%%%%%
%%%%%%%%%%%%%%%%%%%%%%%%%%%%%%%%%%%%%%%%%%%%%%%%%%%%%%%%%%%%%%%%%%%%%%%%%%%%%%%%
\section{Usage}

First of all, the package \textsf{childdoc} is \emph{not} a standard
\LaTeXe{} |.sty| style file! Therefore it needs to be invoked in
a non-standard way.

%%%%%%%%%%%%%%%%%%%%%%%%%%%%%%%%%%%%%%%%%%%%%%%%%%%%%%%%%%%%%%%%%%%%%%%%%%%%%%%%
\subsection{Included Files}
\label{sec:include}

%%%%%%%%%%%%%%%%%%%%%%%%%%%%%%%%%%%%%%%%
\DescribeMacro{\childdocmain}
To use the package, add the commands
\begin{center}
\begin{tabular}{l}
|\input{childdoc.def}|\\
|\childdocmain{}|\\
\end{tabular}
\end{center}
at the very top of the main \LaTeX{} file,
in particular \emph{before} the |\documentclass| statement!
The argument of |\childdocmain| should be left empty
(but it must be present).

%%%%%%%%%%%%%%%%%%%%%%%%%%%%%%%%%%%%%%%%
\DescribeMacro{\childdocof}
Furthermore, add the commands
\begin{center}
\begin{tabular}{l}
|\input{childdoc.def}|\\
|\childdocof{|\textit{main}|}|\\
\end{tabular}
\end{center}
at the top of every child file \textit{child}
which is included by |\include{|\textit{child}|}|
from within the main file
(or at least for those files to be compiled individually).
The argument \textit{main} must be the filename of the main file.

There are a couple of
considerations in setting up the main and child documents:

%%%%%%%%%%%%%%%%%%%%%%%%%%%%%%%%%%%%%%%%
\paragraph{Restrictions.}

Please note the following restrictions:
\begin{itemize}
\item
|\childdocmain| must be called with one argument \textit{main}
to ensure compatibility with earlier version of the package.
It must either be empty (|\childdocmain{}|)
or precisely match the filename of the main file in which it is specified.
See \secref{sec:detection} for further information.
\item
The filename \textit{main} must be specified without the |.tex| extension.
\item
The filename \textit{main} is case sensitive
(even in case-insensitive file systems)
due to internal string comparison.
\item
The argument \textit{main} should be fully expanded, it cannot be a macro.
\item
Subdirectories and special characters should be avoided in filenames.
\item
The command |\childdocmain{|\textit{main}|}| must be followed by a whitespace.
It should not be followed immediately by another command
or by a comment mark `|%|'.
This is because the \TeX{} parser reads the token immediately following
the argument of |\childdocmain| and puts it
at the beginning of every child section;
however, a white\-space is ignored.
\end{itemize}

%%%%%%%%%%%%%%%%%%%%%%%%%%%%%%%%%%%%%%%%
\paragraph{Content of Main File.}

It is advisable to place all content in the child files included by |\include|.
Any output contained in the main file will appear in all child documents
unless suppressed manually;
it cannot be suppressed automatically by the |\includeonly| directive
and thus should normally be avoided.
A method to include some content in the main file
by means of conditional processing is described in \secref{sec:conditional}.

%%%%%%%%%%%%%%%%%%%%%%%%%%%%%%%%%%%%%%%%
\paragraph{Page Numbering.}

When only a part of the document is compiled,
the appropriate numbering of pages
(as well as other status parameters)
is determined from the |.aux| files.
The latter contain information from previous passes.
However this information needs to propagate through
all intermediate child documents.
Therefore the page numbering in child documents may well
be inconsistent until the complete document is compiled at least once.

A useful (if unconventional) way to always ensure a consistent
page numbering is to restart the numbering in each child document
and denote the pages by `\textit{child}|.|\textit{page}'
where \textit{child} represents the chapter/section number of the child file.
This can be achieved by the command
|\numberwithin{page}{|\textit{child}|}|
of the \textsf{amsmath} package
where \textit{child} can be |chapter| or |section|
depending on the chosen structuring.
Alternatively, one can modify the macro |\thepage| appropriately
and reset the counter |page| at the start of each child file.

%%%%%%%%%%%%%%%%%%%%%%%%%%%%%%%%%%%%%%%%%%%%%%%%%%%%%%%%%%%%%%%%%%%%%%%%%%%%%%%%
\subsection{Conditional Processing}
\label{sec:conditional}

The package provides a mechanism to compile different versions
of a document. To customise the versions further some conditional processing
can come in handy to distinguish which version is being compiled.
The package provides two macros to describe the compilation context:

%%%%%%%%%%%%%%%%%%%%%%%%%%%%%%%%%%%%%%%%
\DescribeMacro{\ifchilddoc}
The conditional |\ifchilddoc| distinguishes between the compilation of
child documents and the main document:
%
\begin{center}
|\ifchilddoc |\textit{child-code}| |[|\||else |\textit{main-code}]| \||fi|
\end{center}

%%%%%%%%%%%%%%%%%%%%%%%%%%%%%%%%%%%%%%%%
\DescribeMacro{\childdocname}
\DescribeMacro{\childdocjob}
The macro |\childdocname| contains the filename (without extension)
of the main or child file being processed.
Note that |\childdocjob| will always contain the name of the main file.

%%%%%%%%%%%%%%%%%%%%%%%%%%%%%%%%%%%%%%%%
\paragraph{Title Page.}

Conditional processing can be used to include a title or banner page
in the main document when proper precautions are taken.
Importantly, the code in the main file should ensure that the page counter
(as well as other status parameters which are stored in the |.aux| files)
takes the same value after the conditional processing.
Otherwise the page numbers may take divergent values
depending on which part is compiled.

For example, a title page could be declared by:
%
\begin{center}
\begin{tabular}{l}
|\ifchilddoc\||else|\\
|\addtocounter{page}{-1}|\\
\textit{code for title page}\\
|\newpage|\\
|\||fi|
\end{tabular}
\end{center}
%
A banner page for the child documents can be generated by:
%
\begin{center}
\begin{tabular}{l}
|\ifchilddoc|\\
|\addtocounter{page}{-1}|\\
\textit{code for banner page}\\
|\newpage|\\
|\||fi|
\end{tabular}
\end{center}
%
Here one could write a message such as:
\begin{center}
|This is the part \childdocname{} of \childdocjob{}.|
\end{center}

%%%%%%%%%%%%%%%%%%%%%%%%%%%%%%%%%%%%%%%%%%%%%%%%%%%%%%%%%%%%%%%%%%%%%%%%%%%%%%%%
\subsection{Flags}
\label{sec:flags}

The package makes it easy to generate different versions
of the main or child documents.
To this end compilation flags can be defined
and assigned different default values.
They will be particularly useful in conjunction
with the forwarding mechanism described in \secref{sec:forward}.

For example, it may be useful to have a flag |\version|
which can be set to |draft| or |final|.
The document source will contain some conditional code
depending on the value of |\version|.
Suppose further, the flag should default to |final| for the main file
and to |draft| for child files
which is a natural assignment for editing the document.
This is achieved by placing the following code
in the preamble of the main document
(below the |\childdocmain| directive):
%
\begin{center}
\begin{tabular}{l}
|\ifchilddoc|\\
|\providecommand{\version}{draft}|\\
|\||else|\\
|\providecommand{\version}{final}|\\
|\||fi|
\end{tabular}
\end{center}
%
The definition by |\providecommand| makes sure
that previous definitions are not overwritten.
Further statements |\providecommand{\version}{...}|
can thus be added before the above code to override it.

For the main file, one might add a line
(between |\childdocmain| and the above block)
%
\begin{center}
|%\ifchilddoc\||else\providecommand{\version}{draft}\||fi|
\end{center}
%
which can be uncommented to produce a draft version.
Likewise one can add a line to the very top of a child file
(above the |\childdocof{|\textit{main}|}| directive)
%
\begin{center}
|%\providecommand{\version}{final}|
\end{center}
%
which can be uncommented to produce the final version of this child document.

%%%%%%%%%%%%%%%%%%%%%%%%%%%%%%%%%%%%%%%%%%%%%%%%%%%%%%%%%%%%%%%%%%%%%%%%%%%%%%%%
\subsection{Forwarding}
\label{sec:forward}

Different versions of the main or child documents
using compilation flags as described in \secref{sec:flags}
can be (permanently) stored in different files
for convenient compilation, viewing and distribution.
To this end, the package defines a command
to pass on compilation to a different file:

%%%%%%%%%%%%%%%%%%%%%%%%%%%%%%%%%%%%%%%%
\DescribeMacro{\childdocforward}
The command |\childdocforward| redirects processing to
another source file:
%
\begin{center}
\begin{tabular}{l}
|\input{childdoc.def}|\\
|\childdocforward[|\textit{main}|]{|\textit{dest}|}|\\
\end{tabular}
\end{center}
%
The argument \textit{dest} is the destination file
(without extension).
It should be the main file or one of the child files.
Note that further \textsf{childdoc} directives
such as |\childdocof| and |\childdocforward|
in the indicated file will be processed in this form.
The optional argument \textit{main}
passes on directly to the main file \textit{main}
while pretending to compile the child \textit{dest}.
This form behaves as if \textit{dest}
issues |\childdocof{|\textit{main}|}| right away,
and no further \textsf{childdoc} directives will be processed.

%%%%%%%%%%%%%%%%%%%%%%%%%%%%%%%%%%%%%%%%
\DescribeMacro{\...prefix}
In the alternative form |\childdocforwardprefix|,
%
\begin{center}
\begin{tabular}{l}
|\input{childdoc.def}|\\
|\childdocforwardprefix[|\textit{main}|]{|\textit{prefix}|}{|\textit{dest}|}|
\end{tabular}
\end{center}
%
the destination file is determined by a pattern
depending on the current file:
To make this work, the current file must be called
`{\textit{prefix}\hspace{0.2em}\textit{suffix}}'
with \textit{prefix} matching precisely the argument.
Processing is then passed on to the file
`{\textit{dest}\hspace{0.2em}\textit{suffix}}'.
Surely, the same effect is achieved by
directly specifying the
argument `{\textit{dest}\hspace{0.2em}\textit{suffix}}'
in the first form.
However, that requires to set up a different file
for each child. With the alternative form of the command
all these files can have exactly the same content
which simplifies setting them up and maintaining them.

For example, the following file |draft.tex|
with a compilation flag |\version| as described in \secref{sec:flags}
compiles the main document as a draft:
%
\begin{center}
\begin{tabular}{l}
|\def\version{draft}|\\
|\input{childdoc.def}|\\
|\childdocforward{|\textit{main}|}|
\end{tabular}
\end{center}
%
Likewise, the following files |final|\textit{nn}|.tex|
compile the final version of the child document
|child|\textit{nn}|.tex|:
%
\begin{center}
\begin{tabular}{l}
|\def\version{final}|\\
|\input{childdoc.def}|\\
|\childdocforwardprefix{final}{child}|
\end{tabular}
\end{center}
%

Note that when several versions of a main file and/or of each child file
are to be generated, it may be convenient to set up a |Makefile| or
shell script to automatise the process.

%%%%%%%%%%%%%%%%%%%%%%%%%%%%%%%%%%%%%%%%%%%%%%%%%%%%%%%%%%%%%%%%%%%%%%%%%%%%%%%%
\subsection{Command Line Processing}
\label{sec:commandline}

The effect of redirection files can also be achieved by invoking
the \LaTeX{} compiler with a more elaborate command line.
Most conveniently this should be done as part
of a shell script or a |Makefile|.

When using \textsf{childdoc} in the main file, the following
command lines effectively perform a redirection
(note that depending on the shell being used,
backslashes may have to be doubled: `|\|' $\to$ `|\\|'):
%
\begin{center}
|... -jobname "|\textit{target}|" |\\|"|[\textit{flags}]%
|\input{childdoc.def}\childdocforward[|\textit{main}|]{|\textit{dest}|}"|
\end{center}
%
Here \textit{target} is the name of the output file,
\textit{main} is the name of the main file
and \textit{dest} is the name of the main or child file to be processed
(all filenames without extensions).
The optional argument \textit{main} can be omitted
if \textit{main} matches \textit{dest}.
Optionally, compilation \textit{flags} can be defined via |\def| commands.
This command line makes the \TeX{} engine believe
it is compiling the file \textit{target}
whose content is specified as the latter parameter.
The provided code then forwards the processing to
\textit{main} or \textit{dest} as described in \secref{sec:forward}.

%%%%%%%%%%%%%%%%%%%%%%%%%%%%%%%%%%%%%%%%%%%%%%%%%%%%%%%%%%%%%%%%%%%%%%%%%%%%%%%%
\subsection{Include by Input}
\label{sec:input}

Including child documents by |\include| has some restrictions by design.
Most notably, the content of a child document always occupies
its own set of pages; pages cannot be shared between child documents.
Usually, this behaviour makes perfect sense
because each child document contain an essential part of the document.
However, in some situations it may be desirable to compose
a document from a collection of parts
without having mandatory page breaks between then.
For this case, the package
provides a mechanism to include parts
by |\input| which can also be processed individually.
However, by construction this mechanism
requires manual handling of the content to be output.

%%%%%%%%%%%%%%%%%%%%%%%%%%%%%%%%%%%%%%%%
\DescribeMacro{\ifchilddocmanual}
The main file should be prepared as usual, see \secref{sec:include}.
However, the document body must make a distinction
between processing of an individual part and of the main document, e.g.:
%
\begin{center}
\begin{tabular}{l}
|\ifchilddocmanual|\\
|\input{\childdocname}|\\
|\||else|\\
\textit{document body with }|\input{|\textit{part}|}|\\
|\||fi|
\end{tabular}
\end{center}
%
The conditional |\ifchilddocmanual| is true whenever
a part to be included by |\input| is being compiled,
and the name of the part is stored in |\childdocname|.

%%%%%%%%%%%%%%%%%%%%%%%%%%%%%%%%%%%%%%%%
\DescribeMacro{\childdocby}
Each part to be included by |\input| should start with:
%
\begin{center}
\begin{tabular}{l}
|\input{childdoc.def}|\\
|\childdocby{|\textit{main}|}|\\
\end{tabular}
\end{center}
%
The directive |\childdocby| is similar to |\childdocof|
described in \secref{sec:include},
but the subsequent selection of content must be done manually.
To that end, both |\ifchilddoc| and |\ifchilddocmanual|
will be true upon processing of a part,
and the name of the part is stored in |\childdocname|.
Note that |\jobname| will be set to the filename of the current part
so that each part receives an individual |.aux| file
that does not interfere with the |.aux| file(s) of the main document.
This behaviour can be altered by the alternative form
|\childdocby[*]{|\textit{main}|}| (with a non-empty optional argument)
which uses the |.aux| file of the main document
by setting |\jobname| to \textit{main}.

%%%%%%%%%%%%%%%%%%%%%%%%%%%%%%%%%%%%%%%%%%%%%%%%%%%%%%%%%%%%%%%%%%%%%%%%%%%%%%%%
\subsection{Driver Development}
\label{sec:driver}

The \textsf{childdoc} mechanism can also be use for the development
of definition files such as \LaTeX{} styles or classes.
This case differs from the above setup with multiple parts
included by |\include| in that no |\includeonly| should be invoked.
This can be achieved by starting the include file
(before |\ProvidesPackage|) with:
%
\begin{center}
\begin{tabular}{l}
|\input{childdoc.def}|\\
|\childdocforward{|\textit{main}|}|\\
\end{tabular}
\end{center}
%
or alternatively with:
%
\begin{center}
\begin{tabular}{l}
|\input{childdoc.def}|\\
|\childdocby{|\textit{main}|}|\\
\end{tabular}
\end{center}
%
Both forms have slightly different effects as described above.
The main file is prepared as usual, see \secref{sec:include}.

%%%%%%%%%%%%%%%%%%%%%%%%%%%%%%%%%%%%%%%%%%%%%%%%%%%%%%%%%%%%%%%%%%%%%%%%%%%%%%%%
\subsection{Legacy Detection}
\label{sec:detection}

The directive |\childdocmain| in the main file can detect
whether the complete document or merely a child is to be compiled
even without using the directive |\childdocof|.
This method is deprecated because it is less robust
and there is no compelling reason to use it;
it is merely provided for backward compatibility
and it may be removed in future versions.

If the detection mechanism is to be used,
it is mandatory to correctly specify
the filename of the main file as the argument of |\childdocmain|:
%
\begin{center}
\begin{tabular}{l}
|\input{childdoc.def}|\\
|\childdocmain{|\textit{main}|}|\\
\end{tabular}
\end{center}
%
If |\jobname| does not match the argument \textit{main} of |\childdocmain|,
it is assumed that |\jobname| points to the child file to be compiled.
When using |\childdocmain| with the main file specified as argument,
it suffices to start a child file
with just |\input{|\textit{main}|}|
without loading of the package and using |\childdocof|.
If instead all processing is done
with the appropriate \textsf{childdoc} directives,
the argument of \textit{main} of |\childdocmain| can be empty.

An alternative version of the command line processing described
in \secref{sec:commandline} using the detection mechanism reads:
%
\begin{center}
|... -jobname "|\textit{target}|" "|[\textit{flags}]%
[|\def\jobname{|\textit{dest}|}|]|\input{|\textit{main}|}"|
\end{center}

%%%%%%%%%%%%%%%%%%%%%%%%%%%%%%%%%%%%%%%%%%%%%%%%%%%%%%%%%%%%%%%%%%%%%%%%%%%%%%%%
\subsection{Manual Code}
\label{sec:manual}

In case one cannot be certain whether the definitions file |childdoc.def|
is installed on the target \TeX{} distribution
and one prefers not to ship it,
it is conceivable to paste a few relevant commands into the sources.

To that end, drop all statements |\input{childdoc.def}|
and perform the replacements as outlined below.
Instead of |\childdocmain{|\textit{main}|}| add the following code
to the top of the main file:
%
\begin{center}
\begin{tabular}{l}
|\||ifdefined\childdocname\endinput\||fi\newif\ifchilddoc|\\
|\edef\childdocname{\scantokens\expandafter{\jobname\noexpand}}|\\
|\def\childdocmain{|\textit{main}|}\||ifx\childdocmain\childdocname\||else|\\
|\childdoctrue\includeonly{\childdocname}\let\jobname\childdocmain\||fi|\\
\end{tabular}
\end{center}
%
Instead of |\childdocof{|\textit{main}|}| just include the main file
at the top of each child file:
%
\begin{center}
|\input{|\textit{main}|}|
\end{center}
%
A simple redirection |\childdocforward{|\textit{dest}|}| is achieved by:
%
\begin{center}
|\def\jobname{|\textit{dest}|}\input{\jobname}|
\end{center}
%
The redirection with prefix
|\childdocforwardprefix[|\textit{prefix}|]{|\textit{dest}|}|
is accomplished by:
%
\begin{center}
\begin{tabular}{l}
|{\edef\jobname{\scantokens\expandafter{\jobname\noexpand}}|\\
|\def\redirectjob |\textit{prefix}|#1~~~{\gdef\jobname{|\textit{dest}|#1}}|\\
|\expandafter\redirectjob\jobname~~~}\input{\jobname}|
\end{tabular}
\end{center}

In an alternative approach,
child documents can be compiled by a specific command line
without additional code or specific definitions:
%
\begin{center}
|... -jobname "|\textit{target}|" "|[\textit{flags}]%
|\includeonly{|\textit{dest}|}\input{|\textit{main}|}"|
\end{center}
%

%%%%%%%%%%%%%%%%%%%%%%%%%%%%%%%%%%%%%%%%%%%%%%%%%%%%%%%%%%%%%%%%%%%%%%%%%%%%%%%%
%%%%%%%%%%%%%%%%%%%%%%%%%%%%%%%%%%%%%%%%%%%%%%%%%%%%%%%%%%%%%%%%%%%%%%%%%%%%%%%%
\section{Information}

%%%%%%%%%%%%%%%%%%%%%%%%%%%%%%%%%%%%%%%%%%%%%%%%%%%%%%%%%%%%%%%%%%%%%%%%%%%%%%%%
\subsection{Copyright}

Copyright \copyright{} 2017--2018 Niklas Beisert

This work may be distributed and/or modified under the
conditions of the \LaTeX{} Project Public License, either version 1.3
of this license or (at your option) any later version.
The latest version of this license is in
  \url{http://www.latex-project.org/lppl.txt}
and version 1.3 or later is part of all distributions of \LaTeX{}
version 2005/12/01 or later.

This work has the LPPL maintenance status `maintained'.

The Current Maintainer of this work is Niklas Beisert.

This work consists of the files |README.txt|, |childdoc.ins| and |childdoc.dtx|
as well as the derived files |childdoc.def|, |cdocsamp.tex|
with |cdocsch1.tex|, |cdocsch2.tex|, |cdocspt3.tex|, |cdocspt4.tex|,
|cdocsdrf.tex|, |cdocsfn1.tex|, |cdocsfn2.tex|
as well as |childdoc.pdf|.

%%%%%%%%%%%%%%%%%%%%%%%%%%%%%%%%%%%%%%%%%%%%%%%%%%%%%%%%%%%%%%%%%%%%%%%%%%%%%%%%
\subsection{Files and Installation}

The package consists of the files:
%
\begin{center}
\begin{tabular}{ll}
    |README.txt|   & readme file \\
    |childdoc.ins| & installation file \\
    |childdoc.dtx| & source file \\
    |childdoc.def| & definition file \\
    |cdocsamp.tex| & sample main file \\
    |cdocsch1.tex| & sample include file \\
    |cdocsch2.tex| & sample include file \\
    |cdocspt3.tex| & sample part file \\
    |cdocspt4.tex| & sample part file \\
    |cdocsdrf.tex| & sample redirection file \\
    |cdocsfn1.tex| & sample redirection file \\
    |cdocsfn2.tex| & sample redirection file \\
    |childdoc.pdf| & manual
\end{tabular}
\end{center}
%
The distribution consists of the files
|README.txt|, |childdoc.ins| and |childdoc.dtx|.
%
\begin{itemize}
\item
Run (pdf)\LaTeX{} on |childdoc.dtx|
to compile the manual |childdoc.pdf| (this file).
\item
Run \LaTeX{} on |childdoc.ins| to create the definitions file |childdoc.def|
and the sample |cdocsamp.tex| with include files
|cdocsch1.tex|, |cdocsch2.tex|, |cdocspt3.tex|, |cdocspt4.tex|,
|cdocsdrf.tex|, |cdocsfn1.tex|, |cdocsfn2.tex|.
Then copy the file |childdoc.def| to an appropriate directory of your \LaTeX{}
distribution, e.g.\ \textit{texmf-root}|/tex/latex/childdoc|.
\end{itemize}

%%%%%%%%%%%%%%%%%%%%%%%%%%%%%%%%%%%%%%%%%%%%%%%%%%%%%%%%%%%%%%%%%%%%%%%%%%%%%%%%
\subsection{Related CTAN Packages}

There are several other packages which offer a similar functionality:
%
\begin{itemize}
\item
The packages
\href{http://ctan.org/pkg/docmute}{\textsf{docmute}},
\href{http://ctan.org/pkg/includex}{\textsf{includex}} and
\href{http://ctan.org/pkg/standalone}{\textsf{standalone}}
provide commands to include only the document body of
a child file thus allowing both files to be compiled individually.
\item
The packages \href{http://ctan.org/pkg/subdocs}{\textsf{subdocs}}
and \href{http://ctan.org/pkg/subfiles}{\textsf{subfiles}}
provide structures in which the main and child documents can be
encapsulated and allowing them to be compiled individually.
The inclusion mechanism is different from the conventional |\include|.
\item
The package \href{http://ctan.org/pkg/combine}{\textsf{combine}}
is an elaborate solution to combine several documents into one.
\end{itemize}
%
See also the CTAN topic \href{http://ctan.org/topic/subdocs}{\textsf{subdocs}}
for further related packages.
The present package differs from the above solutions in that
a document structure constructed with the conventional |\include| mechanism
just needs two extra commands at the top of every file
such that all constituent files can be compiled individually.

%%%%%%%%%%%%%%%%%%%%%%%%%%%%%%%%%%%%%%%%%%%%%%%%%%%%%%%%%%%%%%%%%%%%%%%%%%%%%%%%
%\subsection{Feature Suggestions}
%
%The following is a list of features which may be useful for future
%versions of this package:
%%
%\begin{itemize}
%\item
%\ldots
%\end{itemize}

%%%%%%%%%%%%%%%%%%%%%%%%%%%%%%%%%%%%%%%%%%%%%%%%%%%%%%%%%%%%%%%%%%%%%%%%%%%%%%%%
\subsection{Revision History}

%%%%%%%%%%%%%%%%%%%%%%%%%%%%%%%%%%%%%%%%
\paragraph{v2.0:} 2018/12/30

\begin{itemize}
\item
immediate forward processing
\item
added |\childdocby| mechanism
\item
manual restructured
\end{itemize}

%%%%%%%%%%%%%%%%%%%%%%%%%%%%%%%%%%%%%%%%
\paragraph{v1.6:} 2018/01/17

\begin{itemize}
\item
application for development of include files
\item
corrections to manual
\end{itemize}

%%%%%%%%%%%%%%%%%%%%%%%%%%%%%%%%%%%%%%%%
\paragraph{v1.5:} 2017/05/21

\begin{itemize}
\item
more complete structuring introduced
\item
|\childdocof| introduced
\item
|\childdoc| renamed to |\childdocmain|
\item
|\childredirect| renamed to |\childdocforward| and |\childdocforwardprefix|
and functionality expanded
\end{itemize}

%%%%%%%%%%%%%%%%%%%%%%%%%%%%%%%%%%%%%%%%
\paragraph{v1.0:} 2017/04/27

\begin{itemize}
\item
manual and install package
\item
first version published on CTAN
\end{itemize}

%%%%%%%%%%%%%%%%%%%%%%%%%%%%%%%%%%%%%%%%
\paragraph{v0.6:} 2017/04/26

\begin{itemize}
\item
redirection mechanism added
\end{itemize}

%%%%%%%%%%%%%%%%%%%%%%%%%%%%%%%%%%%%%%%%
\paragraph{v0.5:} 2017/04/26

\begin{itemize}
\item
functionality in definition file
\end{itemize}


%%%%%%%%%%%%%%%%%%%%%%%%%%%%%%%%%%%%%%%%%%%%%%%%%%%%%%%%%%%%%%%%%%%%%%%%%%%%%%%%
%%%%%%%%%%%%%%%%%%%%%%%%%%%%%%%%%%%%%%%%%%%%%%%%%%%%%%%%%%%%%%%%%%%%%%%%%%%%%%%%
%%%%%%%%%%%%%%%%%%%%%%%%%%%%%%%%%%%%%%%%%%%%%%%%%%%%%%%%%%%%%%%%%%%%%%%%%%%%%%%%
\appendix

\settowidth\MacroIndent{\rmfamily\scriptsize 000\ }

 \DocInput{childdoc.dtx}

\end{document}
%</driver>
% \fi
%
% %%%%%%%%%%%%%%%%%%%%%%%%%%%%%%%%%%%%%%%%%%%%%%%%%%%%%%%%%%%%%%%%%%%%%%%%%%%%%%
% %%%%%%%%%%%%%%%%%%%%%%%%%%%%%%%%%%%%%%%%%%%%%%%%%%%%%%%%%%%%%%%%%%%%%%%%%%%%%%
% \section{Sample}
%\iffalse
%<*samplemain>
%\fi
%
% The following presents a sample document
% with two chapters, two parts, a title page,
% a compile flag as well as three forwarding files to set the flag.
% It consists of eight |.tex| files:
% \begin{center}
% \begin{tabular}{ll}
% |cdocsamp.tex|&main file\\
% |cdocsch1.tex|&include file for chapter 1\\
% |cdocsch2.tex|&include file for chapter 2\\
% |cdocspt3.tex|&include file for part 3\\
% |cdocspt4.tex|&include file for part 4\\
% |cdocsdrf.tex|&forwarding file for main file in draft mode\\
% |cdocsfi1.tex|&forwarding file for final version of chapter 1\\
% |cdocsfi2.tex|&forwarding file for final version of chapter 2\\
% \end{tabular}
% \end{center}
% Each of the eight files can be compiled directly by the \LaTeX{} compiler.
%
% %%%%%%%%%%%%%%%%%%%%%%%%%%%%%%%%%%%%%%
% \paragraph{Main File.}
%
% The main file is called |cdocsamp.tex|.
%
% Load the \textsf{childdoc} definitions and
% declare the filename for the main document:
%    \begin{macrocode}
\input{childdoc.def}
\childdocmain{}
%    \end{macrocode}

% Optional override for |\version| flag:
%    \begin{macrocode}
%%\ifchilddoc\else\providecommand{\version}{draft}\fi
%    \end{macrocode}

% Define the default values for the |\version| flag
% (|final| for the main file and |draft| for childs):
%    \begin{macrocode}
\ifchilddoc
\providecommand{\version}{draft}
\else
\providecommand{\version}{final}
\fi
%    \end{macrocode}

% Load the standard document class:
%    \begin{macrocode}
\documentclass[12pt]{article}
%    \end{macrocode}

% Start the document body:
%    \begin{macrocode}
\begin{document}
%    \end{macrocode}

% Declare a title page.
% Print title, part of document being processed and version flag:
%    \begin{macrocode}
\addtocounter{page}{-1}
\begin{center}
{\LARGE\bfseries{}childdoc example\par}
\vspace{1cm}
\ifchilddoc
\ifchilddocmanual part\else chapter\fi:
`\childdocname' of `\childdocjob'\par
\else
main document: `\childdocjob'\par
\fi
version: \version\par
\end{center}
\newpage
%    \end{macrocode}

% Manually include selected file,
% otherwise process as usual:
%    \begin{macrocode}
\ifchilddocmanual
\section*{part `\childdocname'}
\input{\childdocname}
\else
%    \end{macrocode}

% Include the two chapters:
%    \begin{macrocode}
\include{cdocsch1}
\include{cdocsch2}
%    \end{macrocode}

% Include the two parts unless only chapters should be displayed:
%    \begin{macrocode}
\ifchilddoc\else
\section{part three}
\input{cdocspt3}
\section{part four}
\input{cdocspt4}
\fi
%    \end{macrocode}

% Process as usual until here:
%    \begin{macrocode}
\fi
%    \end{macrocode}

% End of document body:
%    \begin{macrocode}
\end{document}
%    \end{macrocode}
%\iffalse
%</samplemain>
%\fi
%
% %%%%%%%%%%%%%%%%%%%%%%%%%%%%%%%%%%%%%%
% \paragraph{Chapter Include Files.}
%
% The include files are called |cdocsch1.tex| and |cdocsch2.tex|.
%
%\iffalse
%<*samplechap1|samplechap2>
%\fi

% Optional override for |\version| flag:
%    \begin{macrocode}
%%\providecommand{\version}{final}
%    \end{macrocode}

% Include the main document:
%    \begin{macrocode}
\input{childdoc.def}
\childdocof{cdocsamp}
%    \end{macrocode}

%\iffalse
%</samplechap1|samplechap2>
%\fi
%
%\iffalse
%<*samplechap1>
%\fi
% Some text for chapter 1:
%    \begin{macrocode}
\section{one}
some text in chapter one
%    \end{macrocode}

%\iffalse
%</samplechap1>
%\fi
% Some text for chapter 2:
%\iffalse
%<*samplechap2>
%\fi
%    \begin{macrocode}
\section{two}
more text in chapter two
%    \end{macrocode}

%\iffalse
%</samplechap2>
%\fi
%
% %%%%%%%%%%%%%%%%%%%%%%%%%%%%%%%%%%%%%%
% \paragraph{Part Include Files.}
%
% The include files are called |cdocspt3.tex| and |cdocspt4.tex|.
%
%\iffalse
%<*samplepart3|samplepart4>
%\fi

% Optional override for |\version| flag:
%    \begin{macrocode}
%%\providecommand{\version}{final}
%    \end{macrocode}

% Include the main document:
%    \begin{macrocode}
\input{childdoc.def}
\childdocby{cdocsamp}
%    \end{macrocode}

%\iffalse
%</samplepart3|samplepart4>
%\fi
%
%\iffalse
%<*samplepart3>
%\fi
% Some text for part 3:
%    \begin{macrocode}
some text in part three
%    \end{macrocode}

%\iffalse
%</samplepart3>
%\fi
% Some text for part 4:
%\iffalse
%<*samplepart4>
%\fi
%    \begin{macrocode}
more text in part four
%    \end{macrocode}

%\iffalse
%</samplepart4>
%\fi
%
% %%%%%%%%%%%%%%%%%%%%%%%%%%%%%%%%%%%%%%
% \paragraph{Forwarding for a Complete Draft.}
%
% The following forwarding file |cdocsdrf.tex|
% compiles the main document in draft mode:
%\iffalse
%<*sampledraft>
%\fi
%    \begin{macrocode}
\def\version{draft}
\input{childdoc.def}
\childdocforward{cdocsamp}
%    \end{macrocode}

%\iffalse
%</sampledraft>
%\fi
%
% %%%%%%%%%%%%%%%%%%%%%%%%%%%%%%%%%%%%%%
% \paragraph{Forwarding for Final Version of the Chapters.}
%
% The following forwarding files |cdocsfn1.tex| and |cdocsfn2.tex|
% (with identical content)
% compile the final versions of the child documents
% |cdocsch1.tex| and |cdocsch2.tex|, respectively:
%\iffalse
%<*samplefinal>
%\fi
%    \begin{macrocode}
\def\version{final}
\input{childdoc.def}
\childdocforwardprefix[cdocsamp]{cdocsfn}{cdocsch}
%    \end{macrocode}

%\iffalse
%</samplefinal>
%\fi
%
% %%%%%%%%%%%%%%%%%%%%%%%%%%%%%%%%%%%%%%
% \paragraph{Command Line Processing.}
%
% The following three command lines generate the output files
% |cdocscld|, |cdocscl1| and |cdocscl2|
% which should be identical to
% |cdocsdrf|, |cdocsch1| and |cdocsfn2|, respectively:
% \begin{center}
% \begin{tabular}{l}
% |latex -jobname cdocscld \|\\
% |  "\def\version{draft}\input{childdoc.def}\childdocforward{cdocsamp}"|\\
% |latex -jobname cdocscl1 \|\\
% |  "\input{childdoc.def}\childdocforward[cdocsamp]{cdocsch1}"|\\
% |latex -jobname cdocscl2 \|\\
% |  "\def\version{final}\input{childdoc.def}\childdocforward{cdocsch2}"|
% \end{tabular}
% \end{center}
% Note that the trailing backslash on each first line
% merely continues the input to the second line
% (for convenient cut ant paste).
% Furthermore, the command |latex| can be replaced by any
% of its alternative versions such as |pdflatex|.
%
% %%%%%%%%%%%%%%%%%%%%%%%%%%%%%%%%%%%%%%%%%%%%%%%%%%%%%%%%%%%%%%%%%%%%%%%%%%%%%%
% %%%%%%%%%%%%%%%%%%%%%%%%%%%%%%%%%%%%%%%%%%%%%%%%%%%%%%%%%%%%%%%%%%%%%%%%%%%%%%
% \section{Implementation}
%\iffalse
%<*package>
%\fi
%
% This section describes the definitions file |childdoc.def|.

% The definitions cannot be loaded using |\usepackage| or |\RequirePackage|
% which has a mechanism to prevent loading a style file more than once.
% When loading the definitions by means of |\input|
% multiple instances have to be prevented manually:
%\iffalse
%This code needs to be before the `\ProvidesFile' directive
%which is defined at the beginning of this file.
%Therefore it is also placed there and commented out here.
%</package>
%<*discard>
%\fi
%    \begin{macrocode}
\ifdefined\childdocmain\endinput\fi
%    \end{macrocode}
%\iffalse
%</discard>
%<*package>
%\fi
%
% \macro{\ifchilddoc}
% \macro{\ifchilddocmanual}
% The conditional |\ifchilddoc| tells whether a
% child (true) or main (false) document is being compiled.
% The conditional |\ifchilddocmanual| tells whether
% the |\includeonly| mechanism is used (false) or
% the selection of child files must be performed manually (true).
% The definitions initialise to false:
%    \begin{macrocode}
\newif\ifchilddoc
\newif\ifchilddocmanual
%    \end{macrocode}

% \macro{\childdocname}
% \macro{\childdocjob}
% The macro |\childdocname| stores the name of the main document
% to be compiled. The macro |\childdocjob| stores the name of
% the document on which the \LaTeX{} compiler was originally invoked.
% The content of |\jobname| cannot be compared
% to filenames specified in the source due to different catcodes.
% The following code rescans |\jobname|, stores the result
% in |\childdocname| and saves a copy in |\childdocjob|:
%    \begin{macrocode}
\edef\childdocname{\scantokens\expandafter{\jobname\noexpand}}
\let\childdocjob\childdocname
%    \end{macrocode}

% \macro{\childdocdisable}
% The macro |\childdocdisable| prevents the main file
% from being processed more than once.
% At this stage, the main document command |\childdocmain|
% is assumed to be called once again where it should do nothing.
% Any subsequent call to it should prevent
% a secondary processing of the main document
% It overwrites the forwarding commands
% |\childdocof| and |\childdocforward|
% with empty macros to prevent further inclusions of the main document:
%    \begin{macrocode}
\newcommand{\childdocdisable}
{
  \renewcommand{\childdocmain}[1]{\renewcommand{\childdocmain}[1]{\endinput}}
  \renewcommand{\childdocof}[1]{}
  \renewcommand{\childdocby}[2][]{}
  \renewcommand{\childdocforward}[2][]{}
  \renewcommand{\childdocdisable}{}
}
%    \end{macrocode}

% \macro{\childdocmain}
% The macro |\childdocmain| is to be called at the top of the main file
% with nothing or the main filename (without extension) as argument.
% First, it breaks loops.
% If the argument is not empty and does not match |\childdocname|
% (which is set by the first inclusion of |childdoc.def|),
% |\ifchilddoc| is set to true, |\includeonly| is applied to the child file
% and |\jobname| is set to the main file
% (for proper handling of |.aux| files):
%    \begin{macrocode}
\newcommand{\childdocmain}[1]
{
  \childdocdisable\childdocmain{}
  \if?#1?\else
    \begingroup
      \def\childdoctmp{#1}
      \ifx\childdoctmp\childdocname
        \def\childdoctmp{}
      \else
        \def\childdoctmp
        {
          \childdoctrue
          \includeonly{\childdocname}
          \def\childdocjob{#1}
          \def\jobname{#1}
        }
      \fi
      \expandafter
    \endgroup
    \childdoctmp
  \fi
}
%    \end{macrocode}

% \macro{\childdocof}
% The command |\childdocof| redirects
% compilation to the main file |#1|.
%    \begin{macrocode}
\newcommand{\childdocof}[1]
{
  \childdocdisable
  \childdoctrue
  \includeonly{\childdocname}
  \def\jobname{#1}
  \def\childdocjob{#1}
  \input{#1}
}
%    \end{macrocode}

% \macro{\childdocby}
% The command |\childdocby| ....
%    \begin{macrocode}
\newcommand{\childdocby}[2][]
{
  \childdocdisable
  \childdoctrue
  \childdocmanualtrue
  \if?#1?\else
    \def\jobname{#2}
  \fi
  \def\childdocjob{#2}
  \input{#2}
  \endinput
}
%    \end{macrocode}

% \macro{\childdocforward}
% The command |\childdocforward| redirects
% compilation to the main file or
% (if the optional argument is given) a child file.
% Parameters are set as if the main file
% or a child file starting with |\childdocof| was compiled.
% Then compilation is handed over to the main file:
%    \begin{macrocode}
\newcommand{\childdocforward}[2][]
{
  \begingroup
    \if?#1?
      \def\childdoctmp
      {
        \def\childdocname{#2}
        \def\childdocjob{#2}
        \def\jobname{#2}
        \input{#2}
        \endinput
      }
    \else
      \def\childdoctmp
      {
        \childdocdisable
        \def\childdocname{#2}
        \childdoctrue
        \includeonly{#2}
        \def\childdocjob{#1}
        \def\jobname{#1}
        \input{#1}
        \endinput
      }
    \fi
    \expandafter
  \endgroup
  \childdoctmp
}
%    \end{macrocode}

% \macro{\childdocforwardprefix}
% The command |\childdocforwardprefix| redirects
% compilation to the main or a child file by means of a pattern.
% The prefix |#1| in the current filename is replaced by |#2|
% and the suffix of the current filename is kept
% (it is assumed that the filename does not contain the substring `|~~~|'
% which is used as a delimiter).
% Compilation is handed over to the new file by |\childdocforward|:
%    \begin{macrocode}
\newcommand{\childdocforwardprefix}[3][]
{
  \begingroup
    \def\childdocextract #2##1~~~{\def\childdoctmp{\childdocforward[#1]{#3##1}}}
    \expandafter\childdocextract\childdocname~~~
    \expandafter
  \endgroup
  \childdoctmp
}
%    \end{macrocode}

% \macro{\childdoc}
% The deprecated macro |\childdoc| is a legacy version of |\childdocmain|:
%    \begin{macrocode}
\newcommand{\childdoc}{\childdocmain}
%    \end{macrocode}

% \macro{\childdocredirect}
% The deprecated macro |\childdocredirect| is a legacy version
% of |\childdocforward| and |\childdocforwardprefix|:
%    \begin{macrocode}
\newcommand{\childdocredirect}[2][]
{
  \begingroup
    \if?#1?
      \def\childdoctmp{\childdocforward{#2}}
    \else
      \def\childdoctmp{\childdocforwardprefix{#1}{#2}}
    \fi
    \expandafter
  \endgroup
  \childdoctmp
}
%    \end{macrocode}

%\iffalse
%</package>
%\fi
%
\endinput
|\\
|\childdocforwardprefix{final}{child}|
\end{tabular}
\end{center}
%

Note that when several versions of a main file and/or of each child file
are to be generated, it may be convenient to set up a |Makefile| or
shell script to automatise the process.

%%%%%%%%%%%%%%%%%%%%%%%%%%%%%%%%%%%%%%%%%%%%%%%%%%%%%%%%%%%%%%%%%%%%%%%%%%%%%%%%
\subsection{Command Line Processing}
\label{sec:commandline}

The effect of redirection files can also be achieved by invoking
the \LaTeX{} compiler with a more elaborate command line.
Most conveniently this should be done as part
of a shell script or a |Makefile|.

When using \textsf{childdoc} in the main file, the following
command lines effectively perform a redirection
(note that depending on the shell being used,
backslashes may have to be doubled: `|\|' $\to$ `|\\|'):
%
\begin{center}
|... -jobname "|\textit{target}|" |\\|"|[\textit{flags}]%
|% \iffalse
%
% childdoc.dtx Copyright (C) 2017-2018 Niklas Beisert
%
% This work may be distributed and/or modified under the
% conditions of the LaTeX Project Public License, either version 1.3
% of this license or (at your option) any later version.
% The latest version of this license is in
%   http://www.latex-project.org/lppl.txt
% and version 1.3 or later is part of all distributions of LaTeX
% version 2005/12/01 or later.
%
% This work has the LPPL maintenance status `maintained'.
%
% The Current Maintainer of this work is Niklas Beisert.
%
% This work consists of the files childdoc.dtx and childdoc.ins
% and the derived files childdoc.def and cdocsamp.tex with
% cdocsch1.tex, cdocsch2.tex, cdocsdrf.tex, cdocsfn1.tex, cdocsfn2.tex.
%
%<package>\ifdefined\childdocmain\endinput\fi
%<package>\ProvidesFile{childdoc.def}[2018/12/30 v2.0 child document driver]
%<samplemain>\ProvidesFile{cdocsamp.tex}[2018/12/30 v2.0 sample for childdoc]
%<*driver>
%\ProvidesFile{childdoc.drv}[2018/12/30 v2.0 childdoc reference manual file]
\PassOptionsToClass{10pt,a4paper}{article}
\documentclass{ltxdoc}

\usepackage[margin=35mm]{geometry}
\usepackage{hyperref}
\usepackage{hyperxmp}
\usepackage[usenames]{color}

\hypersetup{colorlinks=true}
\hypersetup{pdfstartview=FitH}
\hypersetup{pdfpagemode=UseNone}
\hypersetup{pdfsource={}}
\hypersetup{pdflang={en-UK}}
\hypersetup{pdfcopyright={Copyright 2017-2018 Niklas Beisert.
  This work may be distributed and/or modified under the
  conditions of the LaTeX Project Public License, either version 1.3
  of this license or (at your option) any later version.}}
\hypersetup{pdflicenseurl={http://www.latex-project.org/lppl.txt}}
\hypersetup{pdfcontactaddress={ETH Zurich, ITP, HIT K,
  Wolfgang-Pauli-Strasse 27}}
\hypersetup{pdfcontactpostcode={8093}}
\hypersetup{pdfcontactcity={Zurich}}
\hypersetup{pdfcontactcountry={Switzerland}}
\hypersetup{pdfcontactemail={nbeisert@itp.phys.ethz.ch}}
\hypersetup{pdfcontacturl={http://people.phys.ethz.ch/\xmptilde nbeisert/}}

\newcommand{\secref}[1]{\hyperref[#1]{section \ref*{#1}}}

\parskip1ex
\parindent0pt
\let\olditemize\itemize
\def\itemize{\olditemize\parskip0pt}

\begin{document}

\title{The \textsf{childdoc} Package}
\hypersetup{pdftitle={The childdoc Package}}
\author{Niklas Beisert\\[2ex]
  Institut f\"ur Theoretische Physik\\
  Eidgen\"ossische Technische Hochschule Z\"urich\\
  Wolfgang-Pauli-Strasse 27, 8093 Z\"urich, Switzerland\\[1ex]
  \href{mailto:nbeisert@itp.phys.ethz.ch}
  {\texttt{nbeisert@itp.phys.ethz.ch}}}
\hypersetup{pdfauthor={Niklas Beisert}}
\hypersetup{pdfsubject={Manual for the LaTeX2e Package childdoc}}
\date{30 December 2018, \textsf{v2.0}}
\maketitle

\begin{abstract}\noindent
\textsf{childdoc} is a \LaTeXe{} package
that enables the direct compilation
of document sections included by |\include|
to individual files.
\end{abstract}

\begingroup
\parskip0ex
\tableofcontents
\endgroup

%%%%%%%%%%%%%%%%%%%%%%%%%%%%%%%%%%%%%%%%%%%%%%%%%%%%%%%%%%%%%%%%%%%%%%%%%%%%%%%%
%%%%%%%%%%%%%%%%%%%%%%%%%%%%%%%%%%%%%%%%%%%%%%%%%%%%%%%%%%%%%%%%%%%%%%%%%%%%%%%%
\section{Introduction}

\LaTeX{} provides a mechanism to structure a large document (such as a book)
into a main file and several child files (containing the chapters)
using the |\include| command.
This mechanism is beneficial for documents
which span hundreds of pages in order to
make the source file(s) more manageable.
Moreover, compilation can be restricted to
selected child files by means of the |\includeonly| command.
The latter feature can be used to reduce the compilation time while editing
(this was significantly more useful in the earlier days of \LaTeX{})
or to generate a smaller document which is easier to navigate.
Another application of |\includeonly| is to generate
documents consisting of selected parts of the complete document.

However, there are a few drawbacks of the plain |\include| mechanism:
\begin{itemize}
\item
The child files cannot be compiled on their own,
they can only be compiled via the main file.
A naive editing environment
(such as a text editor with an option
to have the current file processed by \LaTeX)
may require one to switch to the main file before compiling;
attempting to compile the child file produces errors.
\item
The main file must be modified (each time)
to adjust the |\includeonly| command
to the present needs. This easily leaves the main file in a messy state.
\item
The generated document will always carry the filename
of the main document. This is inconvenient if
several child files are to be compiled and
to be kept for distribution.
\end{itemize}

The present package provides a simple interface
to make child files individually compilable by \LaTeX{}.
Compiling a child file then has the same effect as compiling
the main file with an |\includeonly| command
to select the appropriate child.
Moreover the generated document will carry the name of the child
rather than the main file.
This resolves all three above issues.

This feature is meant to make the editing of books,
thesis documents and lecture notes somewhat more convenient.
However, the package can also be used efficiently for
composing a series of documents (such as exercise sheets)
which are typically distributed individually.
It then assists the author in generating the individual documents
(potentially in different versions)
as well as a document containing the collected series.
Another application is in developing style files
or other kinds of included material
where compilation of the style file could redirect
to a sample or test file.

%%%%%%%%%%%%%%%%%%%%%%%%%%%%%%%%%%%%%%%%%%%%%%%%%%%%%%%%%%%%%%%%%%%%%%%%%%%%%%%%
%%%%%%%%%%%%%%%%%%%%%%%%%%%%%%%%%%%%%%%%%%%%%%%%%%%%%%%%%%%%%%%%%%%%%%%%%%%%%%%%
\section{Usage}

First of all, the package \textsf{childdoc} is \emph{not} a standard
\LaTeXe{} |.sty| style file! Therefore it needs to be invoked in
a non-standard way.

%%%%%%%%%%%%%%%%%%%%%%%%%%%%%%%%%%%%%%%%%%%%%%%%%%%%%%%%%%%%%%%%%%%%%%%%%%%%%%%%
\subsection{Included Files}
\label{sec:include}

%%%%%%%%%%%%%%%%%%%%%%%%%%%%%%%%%%%%%%%%
\DescribeMacro{\childdocmain}
To use the package, add the commands
\begin{center}
\begin{tabular}{l}
|\input{childdoc.def}|\\
|\childdocmain{}|\\
\end{tabular}
\end{center}
at the very top of the main \LaTeX{} file,
in particular \emph{before} the |\documentclass| statement!
The argument of |\childdocmain| should be left empty
(but it must be present).

%%%%%%%%%%%%%%%%%%%%%%%%%%%%%%%%%%%%%%%%
\DescribeMacro{\childdocof}
Furthermore, add the commands
\begin{center}
\begin{tabular}{l}
|\input{childdoc.def}|\\
|\childdocof{|\textit{main}|}|\\
\end{tabular}
\end{center}
at the top of every child file \textit{child}
which is included by |\include{|\textit{child}|}|
from within the main file
(or at least for those files to be compiled individually).
The argument \textit{main} must be the filename of the main file.

There are a couple of
considerations in setting up the main and child documents:

%%%%%%%%%%%%%%%%%%%%%%%%%%%%%%%%%%%%%%%%
\paragraph{Restrictions.}

Please note the following restrictions:
\begin{itemize}
\item
|\childdocmain| must be called with one argument \textit{main}
to ensure compatibility with earlier version of the package.
It must either be empty (|\childdocmain{}|)
or precisely match the filename of the main file in which it is specified.
See \secref{sec:detection} for further information.
\item
The filename \textit{main} must be specified without the |.tex| extension.
\item
The filename \textit{main} is case sensitive
(even in case-insensitive file systems)
due to internal string comparison.
\item
The argument \textit{main} should be fully expanded, it cannot be a macro.
\item
Subdirectories and special characters should be avoided in filenames.
\item
The command |\childdocmain{|\textit{main}|}| must be followed by a whitespace.
It should not be followed immediately by another command
or by a comment mark `|%|'.
This is because the \TeX{} parser reads the token immediately following
the argument of |\childdocmain| and puts it
at the beginning of every child section;
however, a white\-space is ignored.
\end{itemize}

%%%%%%%%%%%%%%%%%%%%%%%%%%%%%%%%%%%%%%%%
\paragraph{Content of Main File.}

It is advisable to place all content in the child files included by |\include|.
Any output contained in the main file will appear in all child documents
unless suppressed manually;
it cannot be suppressed automatically by the |\includeonly| directive
and thus should normally be avoided.
A method to include some content in the main file
by means of conditional processing is described in \secref{sec:conditional}.

%%%%%%%%%%%%%%%%%%%%%%%%%%%%%%%%%%%%%%%%
\paragraph{Page Numbering.}

When only a part of the document is compiled,
the appropriate numbering of pages
(as well as other status parameters)
is determined from the |.aux| files.
The latter contain information from previous passes.
However this information needs to propagate through
all intermediate child documents.
Therefore the page numbering in child documents may well
be inconsistent until the complete document is compiled at least once.

A useful (if unconventional) way to always ensure a consistent
page numbering is to restart the numbering in each child document
and denote the pages by `\textit{child}|.|\textit{page}'
where \textit{child} represents the chapter/section number of the child file.
This can be achieved by the command
|\numberwithin{page}{|\textit{child}|}|
of the \textsf{amsmath} package
where \textit{child} can be |chapter| or |section|
depending on the chosen structuring.
Alternatively, one can modify the macro |\thepage| appropriately
and reset the counter |page| at the start of each child file.

%%%%%%%%%%%%%%%%%%%%%%%%%%%%%%%%%%%%%%%%%%%%%%%%%%%%%%%%%%%%%%%%%%%%%%%%%%%%%%%%
\subsection{Conditional Processing}
\label{sec:conditional}

The package provides a mechanism to compile different versions
of a document. To customise the versions further some conditional processing
can come in handy to distinguish which version is being compiled.
The package provides two macros to describe the compilation context:

%%%%%%%%%%%%%%%%%%%%%%%%%%%%%%%%%%%%%%%%
\DescribeMacro{\ifchilddoc}
The conditional |\ifchilddoc| distinguishes between the compilation of
child documents and the main document:
%
\begin{center}
|\ifchilddoc |\textit{child-code}| |[|\||else |\textit{main-code}]| \||fi|
\end{center}

%%%%%%%%%%%%%%%%%%%%%%%%%%%%%%%%%%%%%%%%
\DescribeMacro{\childdocname}
\DescribeMacro{\childdocjob}
The macro |\childdocname| contains the filename (without extension)
of the main or child file being processed.
Note that |\childdocjob| will always contain the name of the main file.

%%%%%%%%%%%%%%%%%%%%%%%%%%%%%%%%%%%%%%%%
\paragraph{Title Page.}

Conditional processing can be used to include a title or banner page
in the main document when proper precautions are taken.
Importantly, the code in the main file should ensure that the page counter
(as well as other status parameters which are stored in the |.aux| files)
takes the same value after the conditional processing.
Otherwise the page numbers may take divergent values
depending on which part is compiled.

For example, a title page could be declared by:
%
\begin{center}
\begin{tabular}{l}
|\ifchilddoc\||else|\\
|\addtocounter{page}{-1}|\\
\textit{code for title page}\\
|\newpage|\\
|\||fi|
\end{tabular}
\end{center}
%
A banner page for the child documents can be generated by:
%
\begin{center}
\begin{tabular}{l}
|\ifchilddoc|\\
|\addtocounter{page}{-1}|\\
\textit{code for banner page}\\
|\newpage|\\
|\||fi|
\end{tabular}
\end{center}
%
Here one could write a message such as:
\begin{center}
|This is the part \childdocname{} of \childdocjob{}.|
\end{center}

%%%%%%%%%%%%%%%%%%%%%%%%%%%%%%%%%%%%%%%%%%%%%%%%%%%%%%%%%%%%%%%%%%%%%%%%%%%%%%%%
\subsection{Flags}
\label{sec:flags}

The package makes it easy to generate different versions
of the main or child documents.
To this end compilation flags can be defined
and assigned different default values.
They will be particularly useful in conjunction
with the forwarding mechanism described in \secref{sec:forward}.

For example, it may be useful to have a flag |\version|
which can be set to |draft| or |final|.
The document source will contain some conditional code
depending on the value of |\version|.
Suppose further, the flag should default to |final| for the main file
and to |draft| for child files
which is a natural assignment for editing the document.
This is achieved by placing the following code
in the preamble of the main document
(below the |\childdocmain| directive):
%
\begin{center}
\begin{tabular}{l}
|\ifchilddoc|\\
|\providecommand{\version}{draft}|\\
|\||else|\\
|\providecommand{\version}{final}|\\
|\||fi|
\end{tabular}
\end{center}
%
The definition by |\providecommand| makes sure
that previous definitions are not overwritten.
Further statements |\providecommand{\version}{...}|
can thus be added before the above code to override it.

For the main file, one might add a line
(between |\childdocmain| and the above block)
%
\begin{center}
|%\ifchilddoc\||else\providecommand{\version}{draft}\||fi|
\end{center}
%
which can be uncommented to produce a draft version.
Likewise one can add a line to the very top of a child file
(above the |\childdocof{|\textit{main}|}| directive)
%
\begin{center}
|%\providecommand{\version}{final}|
\end{center}
%
which can be uncommented to produce the final version of this child document.

%%%%%%%%%%%%%%%%%%%%%%%%%%%%%%%%%%%%%%%%%%%%%%%%%%%%%%%%%%%%%%%%%%%%%%%%%%%%%%%%
\subsection{Forwarding}
\label{sec:forward}

Different versions of the main or child documents
using compilation flags as described in \secref{sec:flags}
can be (permanently) stored in different files
for convenient compilation, viewing and distribution.
To this end, the package defines a command
to pass on compilation to a different file:

%%%%%%%%%%%%%%%%%%%%%%%%%%%%%%%%%%%%%%%%
\DescribeMacro{\childdocforward}
The command |\childdocforward| redirects processing to
another source file:
%
\begin{center}
\begin{tabular}{l}
|\input{childdoc.def}|\\
|\childdocforward[|\textit{main}|]{|\textit{dest}|}|\\
\end{tabular}
\end{center}
%
The argument \textit{dest} is the destination file
(without extension).
It should be the main file or one of the child files.
Note that further \textsf{childdoc} directives
such as |\childdocof| and |\childdocforward|
in the indicated file will be processed in this form.
The optional argument \textit{main}
passes on directly to the main file \textit{main}
while pretending to compile the child \textit{dest}.
This form behaves as if \textit{dest}
issues |\childdocof{|\textit{main}|}| right away,
and no further \textsf{childdoc} directives will be processed.

%%%%%%%%%%%%%%%%%%%%%%%%%%%%%%%%%%%%%%%%
\DescribeMacro{\...prefix}
In the alternative form |\childdocforwardprefix|,
%
\begin{center}
\begin{tabular}{l}
|\input{childdoc.def}|\\
|\childdocforwardprefix[|\textit{main}|]{|\textit{prefix}|}{|\textit{dest}|}|
\end{tabular}
\end{center}
%
the destination file is determined by a pattern
depending on the current file:
To make this work, the current file must be called
`{\textit{prefix}\hspace{0.2em}\textit{suffix}}'
with \textit{prefix} matching precisely the argument.
Processing is then passed on to the file
`{\textit{dest}\hspace{0.2em}\textit{suffix}}'.
Surely, the same effect is achieved by
directly specifying the
argument `{\textit{dest}\hspace{0.2em}\textit{suffix}}'
in the first form.
However, that requires to set up a different file
for each child. With the alternative form of the command
all these files can have exactly the same content
which simplifies setting them up and maintaining them.

For example, the following file |draft.tex|
with a compilation flag |\version| as described in \secref{sec:flags}
compiles the main document as a draft:
%
\begin{center}
\begin{tabular}{l}
|\def\version{draft}|\\
|\input{childdoc.def}|\\
|\childdocforward{|\textit{main}|}|
\end{tabular}
\end{center}
%
Likewise, the following files |final|\textit{nn}|.tex|
compile the final version of the child document
|child|\textit{nn}|.tex|:
%
\begin{center}
\begin{tabular}{l}
|\def\version{final}|\\
|\input{childdoc.def}|\\
|\childdocforwardprefix{final}{child}|
\end{tabular}
\end{center}
%

Note that when several versions of a main file and/or of each child file
are to be generated, it may be convenient to set up a |Makefile| or
shell script to automatise the process.

%%%%%%%%%%%%%%%%%%%%%%%%%%%%%%%%%%%%%%%%%%%%%%%%%%%%%%%%%%%%%%%%%%%%%%%%%%%%%%%%
\subsection{Command Line Processing}
\label{sec:commandline}

The effect of redirection files can also be achieved by invoking
the \LaTeX{} compiler with a more elaborate command line.
Most conveniently this should be done as part
of a shell script or a |Makefile|.

When using \textsf{childdoc} in the main file, the following
command lines effectively perform a redirection
(note that depending on the shell being used,
backslashes may have to be doubled: `|\|' $\to$ `|\\|'):
%
\begin{center}
|... -jobname "|\textit{target}|" |\\|"|[\textit{flags}]%
|\input{childdoc.def}\childdocforward[|\textit{main}|]{|\textit{dest}|}"|
\end{center}
%
Here \textit{target} is the name of the output file,
\textit{main} is the name of the main file
and \textit{dest} is the name of the main or child file to be processed
(all filenames without extensions).
The optional argument \textit{main} can be omitted
if \textit{main} matches \textit{dest}.
Optionally, compilation \textit{flags} can be defined via |\def| commands.
This command line makes the \TeX{} engine believe
it is compiling the file \textit{target}
whose content is specified as the latter parameter.
The provided code then forwards the processing to
\textit{main} or \textit{dest} as described in \secref{sec:forward}.

%%%%%%%%%%%%%%%%%%%%%%%%%%%%%%%%%%%%%%%%%%%%%%%%%%%%%%%%%%%%%%%%%%%%%%%%%%%%%%%%
\subsection{Include by Input}
\label{sec:input}

Including child documents by |\include| has some restrictions by design.
Most notably, the content of a child document always occupies
its own set of pages; pages cannot be shared between child documents.
Usually, this behaviour makes perfect sense
because each child document contain an essential part of the document.
However, in some situations it may be desirable to compose
a document from a collection of parts
without having mandatory page breaks between then.
For this case, the package
provides a mechanism to include parts
by |\input| which can also be processed individually.
However, by construction this mechanism
requires manual handling of the content to be output.

%%%%%%%%%%%%%%%%%%%%%%%%%%%%%%%%%%%%%%%%
\DescribeMacro{\ifchilddocmanual}
The main file should be prepared as usual, see \secref{sec:include}.
However, the document body must make a distinction
between processing of an individual part and of the main document, e.g.:
%
\begin{center}
\begin{tabular}{l}
|\ifchilddocmanual|\\
|\input{\childdocname}|\\
|\||else|\\
\textit{document body with }|\input{|\textit{part}|}|\\
|\||fi|
\end{tabular}
\end{center}
%
The conditional |\ifchilddocmanual| is true whenever
a part to be included by |\input| is being compiled,
and the name of the part is stored in |\childdocname|.

%%%%%%%%%%%%%%%%%%%%%%%%%%%%%%%%%%%%%%%%
\DescribeMacro{\childdocby}
Each part to be included by |\input| should start with:
%
\begin{center}
\begin{tabular}{l}
|\input{childdoc.def}|\\
|\childdocby{|\textit{main}|}|\\
\end{tabular}
\end{center}
%
The directive |\childdocby| is similar to |\childdocof|
described in \secref{sec:include},
but the subsequent selection of content must be done manually.
To that end, both |\ifchilddoc| and |\ifchilddocmanual|
will be true upon processing of a part,
and the name of the part is stored in |\childdocname|.
Note that |\jobname| will be set to the filename of the current part
so that each part receives an individual |.aux| file
that does not interfere with the |.aux| file(s) of the main document.
This behaviour can be altered by the alternative form
|\childdocby[*]{|\textit{main}|}| (with a non-empty optional argument)
which uses the |.aux| file of the main document
by setting |\jobname| to \textit{main}.

%%%%%%%%%%%%%%%%%%%%%%%%%%%%%%%%%%%%%%%%%%%%%%%%%%%%%%%%%%%%%%%%%%%%%%%%%%%%%%%%
\subsection{Driver Development}
\label{sec:driver}

The \textsf{childdoc} mechanism can also be use for the development
of definition files such as \LaTeX{} styles or classes.
This case differs from the above setup with multiple parts
included by |\include| in that no |\includeonly| should be invoked.
This can be achieved by starting the include file
(before |\ProvidesPackage|) with:
%
\begin{center}
\begin{tabular}{l}
|\input{childdoc.def}|\\
|\childdocforward{|\textit{main}|}|\\
\end{tabular}
\end{center}
%
or alternatively with:
%
\begin{center}
\begin{tabular}{l}
|\input{childdoc.def}|\\
|\childdocby{|\textit{main}|}|\\
\end{tabular}
\end{center}
%
Both forms have slightly different effects as described above.
The main file is prepared as usual, see \secref{sec:include}.

%%%%%%%%%%%%%%%%%%%%%%%%%%%%%%%%%%%%%%%%%%%%%%%%%%%%%%%%%%%%%%%%%%%%%%%%%%%%%%%%
\subsection{Legacy Detection}
\label{sec:detection}

The directive |\childdocmain| in the main file can detect
whether the complete document or merely a child is to be compiled
even without using the directive |\childdocof|.
This method is deprecated because it is less robust
and there is no compelling reason to use it;
it is merely provided for backward compatibility
and it may be removed in future versions.

If the detection mechanism is to be used,
it is mandatory to correctly specify
the filename of the main file as the argument of |\childdocmain|:
%
\begin{center}
\begin{tabular}{l}
|\input{childdoc.def}|\\
|\childdocmain{|\textit{main}|}|\\
\end{tabular}
\end{center}
%
If |\jobname| does not match the argument \textit{main} of |\childdocmain|,
it is assumed that |\jobname| points to the child file to be compiled.
When using |\childdocmain| with the main file specified as argument,
it suffices to start a child file
with just |\input{|\textit{main}|}|
without loading of the package and using |\childdocof|.
If instead all processing is done
with the appropriate \textsf{childdoc} directives,
the argument of \textit{main} of |\childdocmain| can be empty.

An alternative version of the command line processing described
in \secref{sec:commandline} using the detection mechanism reads:
%
\begin{center}
|... -jobname "|\textit{target}|" "|[\textit{flags}]%
[|\def\jobname{|\textit{dest}|}|]|\input{|\textit{main}|}"|
\end{center}

%%%%%%%%%%%%%%%%%%%%%%%%%%%%%%%%%%%%%%%%%%%%%%%%%%%%%%%%%%%%%%%%%%%%%%%%%%%%%%%%
\subsection{Manual Code}
\label{sec:manual}

In case one cannot be certain whether the definitions file |childdoc.def|
is installed on the target \TeX{} distribution
and one prefers not to ship it,
it is conceivable to paste a few relevant commands into the sources.

To that end, drop all statements |\input{childdoc.def}|
and perform the replacements as outlined below.
Instead of |\childdocmain{|\textit{main}|}| add the following code
to the top of the main file:
%
\begin{center}
\begin{tabular}{l}
|\||ifdefined\childdocname\endinput\||fi\newif\ifchilddoc|\\
|\edef\childdocname{\scantokens\expandafter{\jobname\noexpand}}|\\
|\def\childdocmain{|\textit{main}|}\||ifx\childdocmain\childdocname\||else|\\
|\childdoctrue\includeonly{\childdocname}\let\jobname\childdocmain\||fi|\\
\end{tabular}
\end{center}
%
Instead of |\childdocof{|\textit{main}|}| just include the main file
at the top of each child file:
%
\begin{center}
|\input{|\textit{main}|}|
\end{center}
%
A simple redirection |\childdocforward{|\textit{dest}|}| is achieved by:
%
\begin{center}
|\def\jobname{|\textit{dest}|}\input{\jobname}|
\end{center}
%
The redirection with prefix
|\childdocforwardprefix[|\textit{prefix}|]{|\textit{dest}|}|
is accomplished by:
%
\begin{center}
\begin{tabular}{l}
|{\edef\jobname{\scantokens\expandafter{\jobname\noexpand}}|\\
|\def\redirectjob |\textit{prefix}|#1~~~{\gdef\jobname{|\textit{dest}|#1}}|\\
|\expandafter\redirectjob\jobname~~~}\input{\jobname}|
\end{tabular}
\end{center}

In an alternative approach,
child documents can be compiled by a specific command line
without additional code or specific definitions:
%
\begin{center}
|... -jobname "|\textit{target}|" "|[\textit{flags}]%
|\includeonly{|\textit{dest}|}\input{|\textit{main}|}"|
\end{center}
%

%%%%%%%%%%%%%%%%%%%%%%%%%%%%%%%%%%%%%%%%%%%%%%%%%%%%%%%%%%%%%%%%%%%%%%%%%%%%%%%%
%%%%%%%%%%%%%%%%%%%%%%%%%%%%%%%%%%%%%%%%%%%%%%%%%%%%%%%%%%%%%%%%%%%%%%%%%%%%%%%%
\section{Information}

%%%%%%%%%%%%%%%%%%%%%%%%%%%%%%%%%%%%%%%%%%%%%%%%%%%%%%%%%%%%%%%%%%%%%%%%%%%%%%%%
\subsection{Copyright}

Copyright \copyright{} 2017--2018 Niklas Beisert

This work may be distributed and/or modified under the
conditions of the \LaTeX{} Project Public License, either version 1.3
of this license or (at your option) any later version.
The latest version of this license is in
  \url{http://www.latex-project.org/lppl.txt}
and version 1.3 or later is part of all distributions of \LaTeX{}
version 2005/12/01 or later.

This work has the LPPL maintenance status `maintained'.

The Current Maintainer of this work is Niklas Beisert.

This work consists of the files |README.txt|, |childdoc.ins| and |childdoc.dtx|
as well as the derived files |childdoc.def|, |cdocsamp.tex|
with |cdocsch1.tex|, |cdocsch2.tex|, |cdocspt3.tex|, |cdocspt4.tex|,
|cdocsdrf.tex|, |cdocsfn1.tex|, |cdocsfn2.tex|
as well as |childdoc.pdf|.

%%%%%%%%%%%%%%%%%%%%%%%%%%%%%%%%%%%%%%%%%%%%%%%%%%%%%%%%%%%%%%%%%%%%%%%%%%%%%%%%
\subsection{Files and Installation}

The package consists of the files:
%
\begin{center}
\begin{tabular}{ll}
    |README.txt|   & readme file \\
    |childdoc.ins| & installation file \\
    |childdoc.dtx| & source file \\
    |childdoc.def| & definition file \\
    |cdocsamp.tex| & sample main file \\
    |cdocsch1.tex| & sample include file \\
    |cdocsch2.tex| & sample include file \\
    |cdocspt3.tex| & sample part file \\
    |cdocspt4.tex| & sample part file \\
    |cdocsdrf.tex| & sample redirection file \\
    |cdocsfn1.tex| & sample redirection file \\
    |cdocsfn2.tex| & sample redirection file \\
    |childdoc.pdf| & manual
\end{tabular}
\end{center}
%
The distribution consists of the files
|README.txt|, |childdoc.ins| and |childdoc.dtx|.
%
\begin{itemize}
\item
Run (pdf)\LaTeX{} on |childdoc.dtx|
to compile the manual |childdoc.pdf| (this file).
\item
Run \LaTeX{} on |childdoc.ins| to create the definitions file |childdoc.def|
and the sample |cdocsamp.tex| with include files
|cdocsch1.tex|, |cdocsch2.tex|, |cdocspt3.tex|, |cdocspt4.tex|,
|cdocsdrf.tex|, |cdocsfn1.tex|, |cdocsfn2.tex|.
Then copy the file |childdoc.def| to an appropriate directory of your \LaTeX{}
distribution, e.g.\ \textit{texmf-root}|/tex/latex/childdoc|.
\end{itemize}

%%%%%%%%%%%%%%%%%%%%%%%%%%%%%%%%%%%%%%%%%%%%%%%%%%%%%%%%%%%%%%%%%%%%%%%%%%%%%%%%
\subsection{Related CTAN Packages}

There are several other packages which offer a similar functionality:
%
\begin{itemize}
\item
The packages
\href{http://ctan.org/pkg/docmute}{\textsf{docmute}},
\href{http://ctan.org/pkg/includex}{\textsf{includex}} and
\href{http://ctan.org/pkg/standalone}{\textsf{standalone}}
provide commands to include only the document body of
a child file thus allowing both files to be compiled individually.
\item
The packages \href{http://ctan.org/pkg/subdocs}{\textsf{subdocs}}
and \href{http://ctan.org/pkg/subfiles}{\textsf{subfiles}}
provide structures in which the main and child documents can be
encapsulated and allowing them to be compiled individually.
The inclusion mechanism is different from the conventional |\include|.
\item
The package \href{http://ctan.org/pkg/combine}{\textsf{combine}}
is an elaborate solution to combine several documents into one.
\end{itemize}
%
See also the CTAN topic \href{http://ctan.org/topic/subdocs}{\textsf{subdocs}}
for further related packages.
The present package differs from the above solutions in that
a document structure constructed with the conventional |\include| mechanism
just needs two extra commands at the top of every file
such that all constituent files can be compiled individually.

%%%%%%%%%%%%%%%%%%%%%%%%%%%%%%%%%%%%%%%%%%%%%%%%%%%%%%%%%%%%%%%%%%%%%%%%%%%%%%%%
%\subsection{Feature Suggestions}
%
%The following is a list of features which may be useful for future
%versions of this package:
%%
%\begin{itemize}
%\item
%\ldots
%\end{itemize}

%%%%%%%%%%%%%%%%%%%%%%%%%%%%%%%%%%%%%%%%%%%%%%%%%%%%%%%%%%%%%%%%%%%%%%%%%%%%%%%%
\subsection{Revision History}

%%%%%%%%%%%%%%%%%%%%%%%%%%%%%%%%%%%%%%%%
\paragraph{v2.0:} 2018/12/30

\begin{itemize}
\item
immediate forward processing
\item
added |\childdocby| mechanism
\item
manual restructured
\end{itemize}

%%%%%%%%%%%%%%%%%%%%%%%%%%%%%%%%%%%%%%%%
\paragraph{v1.6:} 2018/01/17

\begin{itemize}
\item
application for development of include files
\item
corrections to manual
\end{itemize}

%%%%%%%%%%%%%%%%%%%%%%%%%%%%%%%%%%%%%%%%
\paragraph{v1.5:} 2017/05/21

\begin{itemize}
\item
more complete structuring introduced
\item
|\childdocof| introduced
\item
|\childdoc| renamed to |\childdocmain|
\item
|\childredirect| renamed to |\childdocforward| and |\childdocforwardprefix|
and functionality expanded
\end{itemize}

%%%%%%%%%%%%%%%%%%%%%%%%%%%%%%%%%%%%%%%%
\paragraph{v1.0:} 2017/04/27

\begin{itemize}
\item
manual and install package
\item
first version published on CTAN
\end{itemize}

%%%%%%%%%%%%%%%%%%%%%%%%%%%%%%%%%%%%%%%%
\paragraph{v0.6:} 2017/04/26

\begin{itemize}
\item
redirection mechanism added
\end{itemize}

%%%%%%%%%%%%%%%%%%%%%%%%%%%%%%%%%%%%%%%%
\paragraph{v0.5:} 2017/04/26

\begin{itemize}
\item
functionality in definition file
\end{itemize}


%%%%%%%%%%%%%%%%%%%%%%%%%%%%%%%%%%%%%%%%%%%%%%%%%%%%%%%%%%%%%%%%%%%%%%%%%%%%%%%%
%%%%%%%%%%%%%%%%%%%%%%%%%%%%%%%%%%%%%%%%%%%%%%%%%%%%%%%%%%%%%%%%%%%%%%%%%%%%%%%%
%%%%%%%%%%%%%%%%%%%%%%%%%%%%%%%%%%%%%%%%%%%%%%%%%%%%%%%%%%%%%%%%%%%%%%%%%%%%%%%%
\appendix

\settowidth\MacroIndent{\rmfamily\scriptsize 000\ }

 \DocInput{childdoc.dtx}

\end{document}
%</driver>
% \fi
%
% %%%%%%%%%%%%%%%%%%%%%%%%%%%%%%%%%%%%%%%%%%%%%%%%%%%%%%%%%%%%%%%%%%%%%%%%%%%%%%
% %%%%%%%%%%%%%%%%%%%%%%%%%%%%%%%%%%%%%%%%%%%%%%%%%%%%%%%%%%%%%%%%%%%%%%%%%%%%%%
% \section{Sample}
%\iffalse
%<*samplemain>
%\fi
%
% The following presents a sample document
% with two chapters, two parts, a title page,
% a compile flag as well as three forwarding files to set the flag.
% It consists of eight |.tex| files:
% \begin{center}
% \begin{tabular}{ll}
% |cdocsamp.tex|&main file\\
% |cdocsch1.tex|&include file for chapter 1\\
% |cdocsch2.tex|&include file for chapter 2\\
% |cdocspt3.tex|&include file for part 3\\
% |cdocspt4.tex|&include file for part 4\\
% |cdocsdrf.tex|&forwarding file for main file in draft mode\\
% |cdocsfi1.tex|&forwarding file for final version of chapter 1\\
% |cdocsfi2.tex|&forwarding file for final version of chapter 2\\
% \end{tabular}
% \end{center}
% Each of the eight files can be compiled directly by the \LaTeX{} compiler.
%
% %%%%%%%%%%%%%%%%%%%%%%%%%%%%%%%%%%%%%%
% \paragraph{Main File.}
%
% The main file is called |cdocsamp.tex|.
%
% Load the \textsf{childdoc} definitions and
% declare the filename for the main document:
%    \begin{macrocode}
\input{childdoc.def}
\childdocmain{}
%    \end{macrocode}

% Optional override for |\version| flag:
%    \begin{macrocode}
%%\ifchilddoc\else\providecommand{\version}{draft}\fi
%    \end{macrocode}

% Define the default values for the |\version| flag
% (|final| for the main file and |draft| for childs):
%    \begin{macrocode}
\ifchilddoc
\providecommand{\version}{draft}
\else
\providecommand{\version}{final}
\fi
%    \end{macrocode}

% Load the standard document class:
%    \begin{macrocode}
\documentclass[12pt]{article}
%    \end{macrocode}

% Start the document body:
%    \begin{macrocode}
\begin{document}
%    \end{macrocode}

% Declare a title page.
% Print title, part of document being processed and version flag:
%    \begin{macrocode}
\addtocounter{page}{-1}
\begin{center}
{\LARGE\bfseries{}childdoc example\par}
\vspace{1cm}
\ifchilddoc
\ifchilddocmanual part\else chapter\fi:
`\childdocname' of `\childdocjob'\par
\else
main document: `\childdocjob'\par
\fi
version: \version\par
\end{center}
\newpage
%    \end{macrocode}

% Manually include selected file,
% otherwise process as usual:
%    \begin{macrocode}
\ifchilddocmanual
\section*{part `\childdocname'}
\input{\childdocname}
\else
%    \end{macrocode}

% Include the two chapters:
%    \begin{macrocode}
\include{cdocsch1}
\include{cdocsch2}
%    \end{macrocode}

% Include the two parts unless only chapters should be displayed:
%    \begin{macrocode}
\ifchilddoc\else
\section{part three}
\input{cdocspt3}
\section{part four}
\input{cdocspt4}
\fi
%    \end{macrocode}

% Process as usual until here:
%    \begin{macrocode}
\fi
%    \end{macrocode}

% End of document body:
%    \begin{macrocode}
\end{document}
%    \end{macrocode}
%\iffalse
%</samplemain>
%\fi
%
% %%%%%%%%%%%%%%%%%%%%%%%%%%%%%%%%%%%%%%
% \paragraph{Chapter Include Files.}
%
% The include files are called |cdocsch1.tex| and |cdocsch2.tex|.
%
%\iffalse
%<*samplechap1|samplechap2>
%\fi

% Optional override for |\version| flag:
%    \begin{macrocode}
%%\providecommand{\version}{final}
%    \end{macrocode}

% Include the main document:
%    \begin{macrocode}
\input{childdoc.def}
\childdocof{cdocsamp}
%    \end{macrocode}

%\iffalse
%</samplechap1|samplechap2>
%\fi
%
%\iffalse
%<*samplechap1>
%\fi
% Some text for chapter 1:
%    \begin{macrocode}
\section{one}
some text in chapter one
%    \end{macrocode}

%\iffalse
%</samplechap1>
%\fi
% Some text for chapter 2:
%\iffalse
%<*samplechap2>
%\fi
%    \begin{macrocode}
\section{two}
more text in chapter two
%    \end{macrocode}

%\iffalse
%</samplechap2>
%\fi
%
% %%%%%%%%%%%%%%%%%%%%%%%%%%%%%%%%%%%%%%
% \paragraph{Part Include Files.}
%
% The include files are called |cdocspt3.tex| and |cdocspt4.tex|.
%
%\iffalse
%<*samplepart3|samplepart4>
%\fi

% Optional override for |\version| flag:
%    \begin{macrocode}
%%\providecommand{\version}{final}
%    \end{macrocode}

% Include the main document:
%    \begin{macrocode}
\input{childdoc.def}
\childdocby{cdocsamp}
%    \end{macrocode}

%\iffalse
%</samplepart3|samplepart4>
%\fi
%
%\iffalse
%<*samplepart3>
%\fi
% Some text for part 3:
%    \begin{macrocode}
some text in part three
%    \end{macrocode}

%\iffalse
%</samplepart3>
%\fi
% Some text for part 4:
%\iffalse
%<*samplepart4>
%\fi
%    \begin{macrocode}
more text in part four
%    \end{macrocode}

%\iffalse
%</samplepart4>
%\fi
%
% %%%%%%%%%%%%%%%%%%%%%%%%%%%%%%%%%%%%%%
% \paragraph{Forwarding for a Complete Draft.}
%
% The following forwarding file |cdocsdrf.tex|
% compiles the main document in draft mode:
%\iffalse
%<*sampledraft>
%\fi
%    \begin{macrocode}
\def\version{draft}
\input{childdoc.def}
\childdocforward{cdocsamp}
%    \end{macrocode}

%\iffalse
%</sampledraft>
%\fi
%
% %%%%%%%%%%%%%%%%%%%%%%%%%%%%%%%%%%%%%%
% \paragraph{Forwarding for Final Version of the Chapters.}
%
% The following forwarding files |cdocsfn1.tex| and |cdocsfn2.tex|
% (with identical content)
% compile the final versions of the child documents
% |cdocsch1.tex| and |cdocsch2.tex|, respectively:
%\iffalse
%<*samplefinal>
%\fi
%    \begin{macrocode}
\def\version{final}
\input{childdoc.def}
\childdocforwardprefix[cdocsamp]{cdocsfn}{cdocsch}
%    \end{macrocode}

%\iffalse
%</samplefinal>
%\fi
%
% %%%%%%%%%%%%%%%%%%%%%%%%%%%%%%%%%%%%%%
% \paragraph{Command Line Processing.}
%
% The following three command lines generate the output files
% |cdocscld|, |cdocscl1| and |cdocscl2|
% which should be identical to
% |cdocsdrf|, |cdocsch1| and |cdocsfn2|, respectively:
% \begin{center}
% \begin{tabular}{l}
% |latex -jobname cdocscld \|\\
% |  "\def\version{draft}\input{childdoc.def}\childdocforward{cdocsamp}"|\\
% |latex -jobname cdocscl1 \|\\
% |  "\input{childdoc.def}\childdocforward[cdocsamp]{cdocsch1}"|\\
% |latex -jobname cdocscl2 \|\\
% |  "\def\version{final}\input{childdoc.def}\childdocforward{cdocsch2}"|
% \end{tabular}
% \end{center}
% Note that the trailing backslash on each first line
% merely continues the input to the second line
% (for convenient cut ant paste).
% Furthermore, the command |latex| can be replaced by any
% of its alternative versions such as |pdflatex|.
%
% %%%%%%%%%%%%%%%%%%%%%%%%%%%%%%%%%%%%%%%%%%%%%%%%%%%%%%%%%%%%%%%%%%%%%%%%%%%%%%
% %%%%%%%%%%%%%%%%%%%%%%%%%%%%%%%%%%%%%%%%%%%%%%%%%%%%%%%%%%%%%%%%%%%%%%%%%%%%%%
% \section{Implementation}
%\iffalse
%<*package>
%\fi
%
% This section describes the definitions file |childdoc.def|.

% The definitions cannot be loaded using |\usepackage| or |\RequirePackage|
% which has a mechanism to prevent loading a style file more than once.
% When loading the definitions by means of |\input|
% multiple instances have to be prevented manually:
%\iffalse
%This code needs to be before the `\ProvidesFile' directive
%which is defined at the beginning of this file.
%Therefore it is also placed there and commented out here.
%</package>
%<*discard>
%\fi
%    \begin{macrocode}
\ifdefined\childdocmain\endinput\fi
%    \end{macrocode}
%\iffalse
%</discard>
%<*package>
%\fi
%
% \macro{\ifchilddoc}
% \macro{\ifchilddocmanual}
% The conditional |\ifchilddoc| tells whether a
% child (true) or main (false) document is being compiled.
% The conditional |\ifchilddocmanual| tells whether
% the |\includeonly| mechanism is used (false) or
% the selection of child files must be performed manually (true).
% The definitions initialise to false:
%    \begin{macrocode}
\newif\ifchilddoc
\newif\ifchilddocmanual
%    \end{macrocode}

% \macro{\childdocname}
% \macro{\childdocjob}
% The macro |\childdocname| stores the name of the main document
% to be compiled. The macro |\childdocjob| stores the name of
% the document on which the \LaTeX{} compiler was originally invoked.
% The content of |\jobname| cannot be compared
% to filenames specified in the source due to different catcodes.
% The following code rescans |\jobname|, stores the result
% in |\childdocname| and saves a copy in |\childdocjob|:
%    \begin{macrocode}
\edef\childdocname{\scantokens\expandafter{\jobname\noexpand}}
\let\childdocjob\childdocname
%    \end{macrocode}

% \macro{\childdocdisable}
% The macro |\childdocdisable| prevents the main file
% from being processed more than once.
% At this stage, the main document command |\childdocmain|
% is assumed to be called once again where it should do nothing.
% Any subsequent call to it should prevent
% a secondary processing of the main document
% It overwrites the forwarding commands
% |\childdocof| and |\childdocforward|
% with empty macros to prevent further inclusions of the main document:
%    \begin{macrocode}
\newcommand{\childdocdisable}
{
  \renewcommand{\childdocmain}[1]{\renewcommand{\childdocmain}[1]{\endinput}}
  \renewcommand{\childdocof}[1]{}
  \renewcommand{\childdocby}[2][]{}
  \renewcommand{\childdocforward}[2][]{}
  \renewcommand{\childdocdisable}{}
}
%    \end{macrocode}

% \macro{\childdocmain}
% The macro |\childdocmain| is to be called at the top of the main file
% with nothing or the main filename (without extension) as argument.
% First, it breaks loops.
% If the argument is not empty and does not match |\childdocname|
% (which is set by the first inclusion of |childdoc.def|),
% |\ifchilddoc| is set to true, |\includeonly| is applied to the child file
% and |\jobname| is set to the main file
% (for proper handling of |.aux| files):
%    \begin{macrocode}
\newcommand{\childdocmain}[1]
{
  \childdocdisable\childdocmain{}
  \if?#1?\else
    \begingroup
      \def\childdoctmp{#1}
      \ifx\childdoctmp\childdocname
        \def\childdoctmp{}
      \else
        \def\childdoctmp
        {
          \childdoctrue
          \includeonly{\childdocname}
          \def\childdocjob{#1}
          \def\jobname{#1}
        }
      \fi
      \expandafter
    \endgroup
    \childdoctmp
  \fi
}
%    \end{macrocode}

% \macro{\childdocof}
% The command |\childdocof| redirects
% compilation to the main file |#1|.
%    \begin{macrocode}
\newcommand{\childdocof}[1]
{
  \childdocdisable
  \childdoctrue
  \includeonly{\childdocname}
  \def\jobname{#1}
  \def\childdocjob{#1}
  \input{#1}
}
%    \end{macrocode}

% \macro{\childdocby}
% The command |\childdocby| ....
%    \begin{macrocode}
\newcommand{\childdocby}[2][]
{
  \childdocdisable
  \childdoctrue
  \childdocmanualtrue
  \if?#1?\else
    \def\jobname{#2}
  \fi
  \def\childdocjob{#2}
  \input{#2}
  \endinput
}
%    \end{macrocode}

% \macro{\childdocforward}
% The command |\childdocforward| redirects
% compilation to the main file or
% (if the optional argument is given) a child file.
% Parameters are set as if the main file
% or a child file starting with |\childdocof| was compiled.
% Then compilation is handed over to the main file:
%    \begin{macrocode}
\newcommand{\childdocforward}[2][]
{
  \begingroup
    \if?#1?
      \def\childdoctmp
      {
        \def\childdocname{#2}
        \def\childdocjob{#2}
        \def\jobname{#2}
        \input{#2}
        \endinput
      }
    \else
      \def\childdoctmp
      {
        \childdocdisable
        \def\childdocname{#2}
        \childdoctrue
        \includeonly{#2}
        \def\childdocjob{#1}
        \def\jobname{#1}
        \input{#1}
        \endinput
      }
    \fi
    \expandafter
  \endgroup
  \childdoctmp
}
%    \end{macrocode}

% \macro{\childdocforwardprefix}
% The command |\childdocforwardprefix| redirects
% compilation to the main or a child file by means of a pattern.
% The prefix |#1| in the current filename is replaced by |#2|
% and the suffix of the current filename is kept
% (it is assumed that the filename does not contain the substring `|~~~|'
% which is used as a delimiter).
% Compilation is handed over to the new file by |\childdocforward|:
%    \begin{macrocode}
\newcommand{\childdocforwardprefix}[3][]
{
  \begingroup
    \def\childdocextract #2##1~~~{\def\childdoctmp{\childdocforward[#1]{#3##1}}}
    \expandafter\childdocextract\childdocname~~~
    \expandafter
  \endgroup
  \childdoctmp
}
%    \end{macrocode}

% \macro{\childdoc}
% The deprecated macro |\childdoc| is a legacy version of |\childdocmain|:
%    \begin{macrocode}
\newcommand{\childdoc}{\childdocmain}
%    \end{macrocode}

% \macro{\childdocredirect}
% The deprecated macro |\childdocredirect| is a legacy version
% of |\childdocforward| and |\childdocforwardprefix|:
%    \begin{macrocode}
\newcommand{\childdocredirect}[2][]
{
  \begingroup
    \if?#1?
      \def\childdoctmp{\childdocforward{#2}}
    \else
      \def\childdoctmp{\childdocforwardprefix{#1}{#2}}
    \fi
    \expandafter
  \endgroup
  \childdoctmp
}
%    \end{macrocode}

%\iffalse
%</package>
%\fi
%
\endinput
\childdocforward[|\textit{main}|]{|\textit{dest}|}"|
\end{center}
%
Here \textit{target} is the name of the output file,
\textit{main} is the name of the main file
and \textit{dest} is the name of the main or child file to be processed
(all filenames without extensions).
The optional argument \textit{main} can be omitted
if \textit{main} matches \textit{dest}.
Optionally, compilation \textit{flags} can be defined via |\def| commands.
This command line makes the \TeX{} engine believe
it is compiling the file \textit{target}
whose content is specified as the latter parameter.
The provided code then forwards the processing to
\textit{main} or \textit{dest} as described in \secref{sec:forward}.

%%%%%%%%%%%%%%%%%%%%%%%%%%%%%%%%%%%%%%%%%%%%%%%%%%%%%%%%%%%%%%%%%%%%%%%%%%%%%%%%
\subsection{Include by Input}
\label{sec:input}

Including child documents by |\include| has some restrictions by design.
Most notably, the content of a child document always occupies
its own set of pages; pages cannot be shared between child documents.
Usually, this behaviour makes perfect sense
because each child document contain an essential part of the document.
However, in some situations it may be desirable to compose
a document from a collection of parts
without having mandatory page breaks between then.
For this case, the package
provides a mechanism to include parts
by |\input| which can also be processed individually.
However, by construction this mechanism
requires manual handling of the content to be output.

%%%%%%%%%%%%%%%%%%%%%%%%%%%%%%%%%%%%%%%%
\DescribeMacro{\ifchilddocmanual}
The main file should be prepared as usual, see \secref{sec:include}.
However, the document body must make a distinction
between processing of an individual part and of the main document, e.g.:
%
\begin{center}
\begin{tabular}{l}
|\ifchilddocmanual|\\
|\input{\childdocname}|\\
|\||else|\\
\textit{document body with }|\input{|\textit{part}|}|\\
|\||fi|
\end{tabular}
\end{center}
%
The conditional |\ifchilddocmanual| is true whenever
a part to be included by |\input| is being compiled,
and the name of the part is stored in |\childdocname|.

%%%%%%%%%%%%%%%%%%%%%%%%%%%%%%%%%%%%%%%%
\DescribeMacro{\childdocby}
Each part to be included by |\input| should start with:
%
\begin{center}
\begin{tabular}{l}
|% \iffalse
%
% childdoc.dtx Copyright (C) 2017-2018 Niklas Beisert
%
% This work may be distributed and/or modified under the
% conditions of the LaTeX Project Public License, either version 1.3
% of this license or (at your option) any later version.
% The latest version of this license is in
%   http://www.latex-project.org/lppl.txt
% and version 1.3 or later is part of all distributions of LaTeX
% version 2005/12/01 or later.
%
% This work has the LPPL maintenance status `maintained'.
%
% The Current Maintainer of this work is Niklas Beisert.
%
% This work consists of the files childdoc.dtx and childdoc.ins
% and the derived files childdoc.def and cdocsamp.tex with
% cdocsch1.tex, cdocsch2.tex, cdocsdrf.tex, cdocsfn1.tex, cdocsfn2.tex.
%
%<package>\ifdefined\childdocmain\endinput\fi
%<package>\ProvidesFile{childdoc.def}[2018/12/30 v2.0 child document driver]
%<samplemain>\ProvidesFile{cdocsamp.tex}[2018/12/30 v2.0 sample for childdoc]
%<*driver>
%\ProvidesFile{childdoc.drv}[2018/12/30 v2.0 childdoc reference manual file]
\PassOptionsToClass{10pt,a4paper}{article}
\documentclass{ltxdoc}

\usepackage[margin=35mm]{geometry}
\usepackage{hyperref}
\usepackage{hyperxmp}
\usepackage[usenames]{color}

\hypersetup{colorlinks=true}
\hypersetup{pdfstartview=FitH}
\hypersetup{pdfpagemode=UseNone}
\hypersetup{pdfsource={}}
\hypersetup{pdflang={en-UK}}
\hypersetup{pdfcopyright={Copyright 2017-2018 Niklas Beisert.
  This work may be distributed and/or modified under the
  conditions of the LaTeX Project Public License, either version 1.3
  of this license or (at your option) any later version.}}
\hypersetup{pdflicenseurl={http://www.latex-project.org/lppl.txt}}
\hypersetup{pdfcontactaddress={ETH Zurich, ITP, HIT K,
  Wolfgang-Pauli-Strasse 27}}
\hypersetup{pdfcontactpostcode={8093}}
\hypersetup{pdfcontactcity={Zurich}}
\hypersetup{pdfcontactcountry={Switzerland}}
\hypersetup{pdfcontactemail={nbeisert@itp.phys.ethz.ch}}
\hypersetup{pdfcontacturl={http://people.phys.ethz.ch/\xmptilde nbeisert/}}

\newcommand{\secref}[1]{\hyperref[#1]{section \ref*{#1}}}

\parskip1ex
\parindent0pt
\let\olditemize\itemize
\def\itemize{\olditemize\parskip0pt}

\begin{document}

\title{The \textsf{childdoc} Package}
\hypersetup{pdftitle={The childdoc Package}}
\author{Niklas Beisert\\[2ex]
  Institut f\"ur Theoretische Physik\\
  Eidgen\"ossische Technische Hochschule Z\"urich\\
  Wolfgang-Pauli-Strasse 27, 8093 Z\"urich, Switzerland\\[1ex]
  \href{mailto:nbeisert@itp.phys.ethz.ch}
  {\texttt{nbeisert@itp.phys.ethz.ch}}}
\hypersetup{pdfauthor={Niklas Beisert}}
\hypersetup{pdfsubject={Manual for the LaTeX2e Package childdoc}}
\date{30 December 2018, \textsf{v2.0}}
\maketitle

\begin{abstract}\noindent
\textsf{childdoc} is a \LaTeXe{} package
that enables the direct compilation
of document sections included by |\include|
to individual files.
\end{abstract}

\begingroup
\parskip0ex
\tableofcontents
\endgroup

%%%%%%%%%%%%%%%%%%%%%%%%%%%%%%%%%%%%%%%%%%%%%%%%%%%%%%%%%%%%%%%%%%%%%%%%%%%%%%%%
%%%%%%%%%%%%%%%%%%%%%%%%%%%%%%%%%%%%%%%%%%%%%%%%%%%%%%%%%%%%%%%%%%%%%%%%%%%%%%%%
\section{Introduction}

\LaTeX{} provides a mechanism to structure a large document (such as a book)
into a main file and several child files (containing the chapters)
using the |\include| command.
This mechanism is beneficial for documents
which span hundreds of pages in order to
make the source file(s) more manageable.
Moreover, compilation can be restricted to
selected child files by means of the |\includeonly| command.
The latter feature can be used to reduce the compilation time while editing
(this was significantly more useful in the earlier days of \LaTeX{})
or to generate a smaller document which is easier to navigate.
Another application of |\includeonly| is to generate
documents consisting of selected parts of the complete document.

However, there are a few drawbacks of the plain |\include| mechanism:
\begin{itemize}
\item
The child files cannot be compiled on their own,
they can only be compiled via the main file.
A naive editing environment
(such as a text editor with an option
to have the current file processed by \LaTeX)
may require one to switch to the main file before compiling;
attempting to compile the child file produces errors.
\item
The main file must be modified (each time)
to adjust the |\includeonly| command
to the present needs. This easily leaves the main file in a messy state.
\item
The generated document will always carry the filename
of the main document. This is inconvenient if
several child files are to be compiled and
to be kept for distribution.
\end{itemize}

The present package provides a simple interface
to make child files individually compilable by \LaTeX{}.
Compiling a child file then has the same effect as compiling
the main file with an |\includeonly| command
to select the appropriate child.
Moreover the generated document will carry the name of the child
rather than the main file.
This resolves all three above issues.

This feature is meant to make the editing of books,
thesis documents and lecture notes somewhat more convenient.
However, the package can also be used efficiently for
composing a series of documents (such as exercise sheets)
which are typically distributed individually.
It then assists the author in generating the individual documents
(potentially in different versions)
as well as a document containing the collected series.
Another application is in developing style files
or other kinds of included material
where compilation of the style file could redirect
to a sample or test file.

%%%%%%%%%%%%%%%%%%%%%%%%%%%%%%%%%%%%%%%%%%%%%%%%%%%%%%%%%%%%%%%%%%%%%%%%%%%%%%%%
%%%%%%%%%%%%%%%%%%%%%%%%%%%%%%%%%%%%%%%%%%%%%%%%%%%%%%%%%%%%%%%%%%%%%%%%%%%%%%%%
\section{Usage}

First of all, the package \textsf{childdoc} is \emph{not} a standard
\LaTeXe{} |.sty| style file! Therefore it needs to be invoked in
a non-standard way.

%%%%%%%%%%%%%%%%%%%%%%%%%%%%%%%%%%%%%%%%%%%%%%%%%%%%%%%%%%%%%%%%%%%%%%%%%%%%%%%%
\subsection{Included Files}
\label{sec:include}

%%%%%%%%%%%%%%%%%%%%%%%%%%%%%%%%%%%%%%%%
\DescribeMacro{\childdocmain}
To use the package, add the commands
\begin{center}
\begin{tabular}{l}
|\input{childdoc.def}|\\
|\childdocmain{}|\\
\end{tabular}
\end{center}
at the very top of the main \LaTeX{} file,
in particular \emph{before} the |\documentclass| statement!
The argument of |\childdocmain| should be left empty
(but it must be present).

%%%%%%%%%%%%%%%%%%%%%%%%%%%%%%%%%%%%%%%%
\DescribeMacro{\childdocof}
Furthermore, add the commands
\begin{center}
\begin{tabular}{l}
|\input{childdoc.def}|\\
|\childdocof{|\textit{main}|}|\\
\end{tabular}
\end{center}
at the top of every child file \textit{child}
which is included by |\include{|\textit{child}|}|
from within the main file
(or at least for those files to be compiled individually).
The argument \textit{main} must be the filename of the main file.

There are a couple of
considerations in setting up the main and child documents:

%%%%%%%%%%%%%%%%%%%%%%%%%%%%%%%%%%%%%%%%
\paragraph{Restrictions.}

Please note the following restrictions:
\begin{itemize}
\item
|\childdocmain| must be called with one argument \textit{main}
to ensure compatibility with earlier version of the package.
It must either be empty (|\childdocmain{}|)
or precisely match the filename of the main file in which it is specified.
See \secref{sec:detection} for further information.
\item
The filename \textit{main} must be specified without the |.tex| extension.
\item
The filename \textit{main} is case sensitive
(even in case-insensitive file systems)
due to internal string comparison.
\item
The argument \textit{main} should be fully expanded, it cannot be a macro.
\item
Subdirectories and special characters should be avoided in filenames.
\item
The command |\childdocmain{|\textit{main}|}| must be followed by a whitespace.
It should not be followed immediately by another command
or by a comment mark `|%|'.
This is because the \TeX{} parser reads the token immediately following
the argument of |\childdocmain| and puts it
at the beginning of every child section;
however, a white\-space is ignored.
\end{itemize}

%%%%%%%%%%%%%%%%%%%%%%%%%%%%%%%%%%%%%%%%
\paragraph{Content of Main File.}

It is advisable to place all content in the child files included by |\include|.
Any output contained in the main file will appear in all child documents
unless suppressed manually;
it cannot be suppressed automatically by the |\includeonly| directive
and thus should normally be avoided.
A method to include some content in the main file
by means of conditional processing is described in \secref{sec:conditional}.

%%%%%%%%%%%%%%%%%%%%%%%%%%%%%%%%%%%%%%%%
\paragraph{Page Numbering.}

When only a part of the document is compiled,
the appropriate numbering of pages
(as well as other status parameters)
is determined from the |.aux| files.
The latter contain information from previous passes.
However this information needs to propagate through
all intermediate child documents.
Therefore the page numbering in child documents may well
be inconsistent until the complete document is compiled at least once.

A useful (if unconventional) way to always ensure a consistent
page numbering is to restart the numbering in each child document
and denote the pages by `\textit{child}|.|\textit{page}'
where \textit{child} represents the chapter/section number of the child file.
This can be achieved by the command
|\numberwithin{page}{|\textit{child}|}|
of the \textsf{amsmath} package
where \textit{child} can be |chapter| or |section|
depending on the chosen structuring.
Alternatively, one can modify the macro |\thepage| appropriately
and reset the counter |page| at the start of each child file.

%%%%%%%%%%%%%%%%%%%%%%%%%%%%%%%%%%%%%%%%%%%%%%%%%%%%%%%%%%%%%%%%%%%%%%%%%%%%%%%%
\subsection{Conditional Processing}
\label{sec:conditional}

The package provides a mechanism to compile different versions
of a document. To customise the versions further some conditional processing
can come in handy to distinguish which version is being compiled.
The package provides two macros to describe the compilation context:

%%%%%%%%%%%%%%%%%%%%%%%%%%%%%%%%%%%%%%%%
\DescribeMacro{\ifchilddoc}
The conditional |\ifchilddoc| distinguishes between the compilation of
child documents and the main document:
%
\begin{center}
|\ifchilddoc |\textit{child-code}| |[|\||else |\textit{main-code}]| \||fi|
\end{center}

%%%%%%%%%%%%%%%%%%%%%%%%%%%%%%%%%%%%%%%%
\DescribeMacro{\childdocname}
\DescribeMacro{\childdocjob}
The macro |\childdocname| contains the filename (without extension)
of the main or child file being processed.
Note that |\childdocjob| will always contain the name of the main file.

%%%%%%%%%%%%%%%%%%%%%%%%%%%%%%%%%%%%%%%%
\paragraph{Title Page.}

Conditional processing can be used to include a title or banner page
in the main document when proper precautions are taken.
Importantly, the code in the main file should ensure that the page counter
(as well as other status parameters which are stored in the |.aux| files)
takes the same value after the conditional processing.
Otherwise the page numbers may take divergent values
depending on which part is compiled.

For example, a title page could be declared by:
%
\begin{center}
\begin{tabular}{l}
|\ifchilddoc\||else|\\
|\addtocounter{page}{-1}|\\
\textit{code for title page}\\
|\newpage|\\
|\||fi|
\end{tabular}
\end{center}
%
A banner page for the child documents can be generated by:
%
\begin{center}
\begin{tabular}{l}
|\ifchilddoc|\\
|\addtocounter{page}{-1}|\\
\textit{code for banner page}\\
|\newpage|\\
|\||fi|
\end{tabular}
\end{center}
%
Here one could write a message such as:
\begin{center}
|This is the part \childdocname{} of \childdocjob{}.|
\end{center}

%%%%%%%%%%%%%%%%%%%%%%%%%%%%%%%%%%%%%%%%%%%%%%%%%%%%%%%%%%%%%%%%%%%%%%%%%%%%%%%%
\subsection{Flags}
\label{sec:flags}

The package makes it easy to generate different versions
of the main or child documents.
To this end compilation flags can be defined
and assigned different default values.
They will be particularly useful in conjunction
with the forwarding mechanism described in \secref{sec:forward}.

For example, it may be useful to have a flag |\version|
which can be set to |draft| or |final|.
The document source will contain some conditional code
depending on the value of |\version|.
Suppose further, the flag should default to |final| for the main file
and to |draft| for child files
which is a natural assignment for editing the document.
This is achieved by placing the following code
in the preamble of the main document
(below the |\childdocmain| directive):
%
\begin{center}
\begin{tabular}{l}
|\ifchilddoc|\\
|\providecommand{\version}{draft}|\\
|\||else|\\
|\providecommand{\version}{final}|\\
|\||fi|
\end{tabular}
\end{center}
%
The definition by |\providecommand| makes sure
that previous definitions are not overwritten.
Further statements |\providecommand{\version}{...}|
can thus be added before the above code to override it.

For the main file, one might add a line
(between |\childdocmain| and the above block)
%
\begin{center}
|%\ifchilddoc\||else\providecommand{\version}{draft}\||fi|
\end{center}
%
which can be uncommented to produce a draft version.
Likewise one can add a line to the very top of a child file
(above the |\childdocof{|\textit{main}|}| directive)
%
\begin{center}
|%\providecommand{\version}{final}|
\end{center}
%
which can be uncommented to produce the final version of this child document.

%%%%%%%%%%%%%%%%%%%%%%%%%%%%%%%%%%%%%%%%%%%%%%%%%%%%%%%%%%%%%%%%%%%%%%%%%%%%%%%%
\subsection{Forwarding}
\label{sec:forward}

Different versions of the main or child documents
using compilation flags as described in \secref{sec:flags}
can be (permanently) stored in different files
for convenient compilation, viewing and distribution.
To this end, the package defines a command
to pass on compilation to a different file:

%%%%%%%%%%%%%%%%%%%%%%%%%%%%%%%%%%%%%%%%
\DescribeMacro{\childdocforward}
The command |\childdocforward| redirects processing to
another source file:
%
\begin{center}
\begin{tabular}{l}
|\input{childdoc.def}|\\
|\childdocforward[|\textit{main}|]{|\textit{dest}|}|\\
\end{tabular}
\end{center}
%
The argument \textit{dest} is the destination file
(without extension).
It should be the main file or one of the child files.
Note that further \textsf{childdoc} directives
such as |\childdocof| and |\childdocforward|
in the indicated file will be processed in this form.
The optional argument \textit{main}
passes on directly to the main file \textit{main}
while pretending to compile the child \textit{dest}.
This form behaves as if \textit{dest}
issues |\childdocof{|\textit{main}|}| right away,
and no further \textsf{childdoc} directives will be processed.

%%%%%%%%%%%%%%%%%%%%%%%%%%%%%%%%%%%%%%%%
\DescribeMacro{\...prefix}
In the alternative form |\childdocforwardprefix|,
%
\begin{center}
\begin{tabular}{l}
|\input{childdoc.def}|\\
|\childdocforwardprefix[|\textit{main}|]{|\textit{prefix}|}{|\textit{dest}|}|
\end{tabular}
\end{center}
%
the destination file is determined by a pattern
depending on the current file:
To make this work, the current file must be called
`{\textit{prefix}\hspace{0.2em}\textit{suffix}}'
with \textit{prefix} matching precisely the argument.
Processing is then passed on to the file
`{\textit{dest}\hspace{0.2em}\textit{suffix}}'.
Surely, the same effect is achieved by
directly specifying the
argument `{\textit{dest}\hspace{0.2em}\textit{suffix}}'
in the first form.
However, that requires to set up a different file
for each child. With the alternative form of the command
all these files can have exactly the same content
which simplifies setting them up and maintaining them.

For example, the following file |draft.tex|
with a compilation flag |\version| as described in \secref{sec:flags}
compiles the main document as a draft:
%
\begin{center}
\begin{tabular}{l}
|\def\version{draft}|\\
|\input{childdoc.def}|\\
|\childdocforward{|\textit{main}|}|
\end{tabular}
\end{center}
%
Likewise, the following files |final|\textit{nn}|.tex|
compile the final version of the child document
|child|\textit{nn}|.tex|:
%
\begin{center}
\begin{tabular}{l}
|\def\version{final}|\\
|\input{childdoc.def}|\\
|\childdocforwardprefix{final}{child}|
\end{tabular}
\end{center}
%

Note that when several versions of a main file and/or of each child file
are to be generated, it may be convenient to set up a |Makefile| or
shell script to automatise the process.

%%%%%%%%%%%%%%%%%%%%%%%%%%%%%%%%%%%%%%%%%%%%%%%%%%%%%%%%%%%%%%%%%%%%%%%%%%%%%%%%
\subsection{Command Line Processing}
\label{sec:commandline}

The effect of redirection files can also be achieved by invoking
the \LaTeX{} compiler with a more elaborate command line.
Most conveniently this should be done as part
of a shell script or a |Makefile|.

When using \textsf{childdoc} in the main file, the following
command lines effectively perform a redirection
(note that depending on the shell being used,
backslashes may have to be doubled: `|\|' $\to$ `|\\|'):
%
\begin{center}
|... -jobname "|\textit{target}|" |\\|"|[\textit{flags}]%
|\input{childdoc.def}\childdocforward[|\textit{main}|]{|\textit{dest}|}"|
\end{center}
%
Here \textit{target} is the name of the output file,
\textit{main} is the name of the main file
and \textit{dest} is the name of the main or child file to be processed
(all filenames without extensions).
The optional argument \textit{main} can be omitted
if \textit{main} matches \textit{dest}.
Optionally, compilation \textit{flags} can be defined via |\def| commands.
This command line makes the \TeX{} engine believe
it is compiling the file \textit{target}
whose content is specified as the latter parameter.
The provided code then forwards the processing to
\textit{main} or \textit{dest} as described in \secref{sec:forward}.

%%%%%%%%%%%%%%%%%%%%%%%%%%%%%%%%%%%%%%%%%%%%%%%%%%%%%%%%%%%%%%%%%%%%%%%%%%%%%%%%
\subsection{Include by Input}
\label{sec:input}

Including child documents by |\include| has some restrictions by design.
Most notably, the content of a child document always occupies
its own set of pages; pages cannot be shared between child documents.
Usually, this behaviour makes perfect sense
because each child document contain an essential part of the document.
However, in some situations it may be desirable to compose
a document from a collection of parts
without having mandatory page breaks between then.
For this case, the package
provides a mechanism to include parts
by |\input| which can also be processed individually.
However, by construction this mechanism
requires manual handling of the content to be output.

%%%%%%%%%%%%%%%%%%%%%%%%%%%%%%%%%%%%%%%%
\DescribeMacro{\ifchilddocmanual}
The main file should be prepared as usual, see \secref{sec:include}.
However, the document body must make a distinction
between processing of an individual part and of the main document, e.g.:
%
\begin{center}
\begin{tabular}{l}
|\ifchilddocmanual|\\
|\input{\childdocname}|\\
|\||else|\\
\textit{document body with }|\input{|\textit{part}|}|\\
|\||fi|
\end{tabular}
\end{center}
%
The conditional |\ifchilddocmanual| is true whenever
a part to be included by |\input| is being compiled,
and the name of the part is stored in |\childdocname|.

%%%%%%%%%%%%%%%%%%%%%%%%%%%%%%%%%%%%%%%%
\DescribeMacro{\childdocby}
Each part to be included by |\input| should start with:
%
\begin{center}
\begin{tabular}{l}
|\input{childdoc.def}|\\
|\childdocby{|\textit{main}|}|\\
\end{tabular}
\end{center}
%
The directive |\childdocby| is similar to |\childdocof|
described in \secref{sec:include},
but the subsequent selection of content must be done manually.
To that end, both |\ifchilddoc| and |\ifchilddocmanual|
will be true upon processing of a part,
and the name of the part is stored in |\childdocname|.
Note that |\jobname| will be set to the filename of the current part
so that each part receives an individual |.aux| file
that does not interfere with the |.aux| file(s) of the main document.
This behaviour can be altered by the alternative form
|\childdocby[*]{|\textit{main}|}| (with a non-empty optional argument)
which uses the |.aux| file of the main document
by setting |\jobname| to \textit{main}.

%%%%%%%%%%%%%%%%%%%%%%%%%%%%%%%%%%%%%%%%%%%%%%%%%%%%%%%%%%%%%%%%%%%%%%%%%%%%%%%%
\subsection{Driver Development}
\label{sec:driver}

The \textsf{childdoc} mechanism can also be use for the development
of definition files such as \LaTeX{} styles or classes.
This case differs from the above setup with multiple parts
included by |\include| in that no |\includeonly| should be invoked.
This can be achieved by starting the include file
(before |\ProvidesPackage|) with:
%
\begin{center}
\begin{tabular}{l}
|\input{childdoc.def}|\\
|\childdocforward{|\textit{main}|}|\\
\end{tabular}
\end{center}
%
or alternatively with:
%
\begin{center}
\begin{tabular}{l}
|\input{childdoc.def}|\\
|\childdocby{|\textit{main}|}|\\
\end{tabular}
\end{center}
%
Both forms have slightly different effects as described above.
The main file is prepared as usual, see \secref{sec:include}.

%%%%%%%%%%%%%%%%%%%%%%%%%%%%%%%%%%%%%%%%%%%%%%%%%%%%%%%%%%%%%%%%%%%%%%%%%%%%%%%%
\subsection{Legacy Detection}
\label{sec:detection}

The directive |\childdocmain| in the main file can detect
whether the complete document or merely a child is to be compiled
even without using the directive |\childdocof|.
This method is deprecated because it is less robust
and there is no compelling reason to use it;
it is merely provided for backward compatibility
and it may be removed in future versions.

If the detection mechanism is to be used,
it is mandatory to correctly specify
the filename of the main file as the argument of |\childdocmain|:
%
\begin{center}
\begin{tabular}{l}
|\input{childdoc.def}|\\
|\childdocmain{|\textit{main}|}|\\
\end{tabular}
\end{center}
%
If |\jobname| does not match the argument \textit{main} of |\childdocmain|,
it is assumed that |\jobname| points to the child file to be compiled.
When using |\childdocmain| with the main file specified as argument,
it suffices to start a child file
with just |\input{|\textit{main}|}|
without loading of the package and using |\childdocof|.
If instead all processing is done
with the appropriate \textsf{childdoc} directives,
the argument of \textit{main} of |\childdocmain| can be empty.

An alternative version of the command line processing described
in \secref{sec:commandline} using the detection mechanism reads:
%
\begin{center}
|... -jobname "|\textit{target}|" "|[\textit{flags}]%
[|\def\jobname{|\textit{dest}|}|]|\input{|\textit{main}|}"|
\end{center}

%%%%%%%%%%%%%%%%%%%%%%%%%%%%%%%%%%%%%%%%%%%%%%%%%%%%%%%%%%%%%%%%%%%%%%%%%%%%%%%%
\subsection{Manual Code}
\label{sec:manual}

In case one cannot be certain whether the definitions file |childdoc.def|
is installed on the target \TeX{} distribution
and one prefers not to ship it,
it is conceivable to paste a few relevant commands into the sources.

To that end, drop all statements |\input{childdoc.def}|
and perform the replacements as outlined below.
Instead of |\childdocmain{|\textit{main}|}| add the following code
to the top of the main file:
%
\begin{center}
\begin{tabular}{l}
|\||ifdefined\childdocname\endinput\||fi\newif\ifchilddoc|\\
|\edef\childdocname{\scantokens\expandafter{\jobname\noexpand}}|\\
|\def\childdocmain{|\textit{main}|}\||ifx\childdocmain\childdocname\||else|\\
|\childdoctrue\includeonly{\childdocname}\let\jobname\childdocmain\||fi|\\
\end{tabular}
\end{center}
%
Instead of |\childdocof{|\textit{main}|}| just include the main file
at the top of each child file:
%
\begin{center}
|\input{|\textit{main}|}|
\end{center}
%
A simple redirection |\childdocforward{|\textit{dest}|}| is achieved by:
%
\begin{center}
|\def\jobname{|\textit{dest}|}\input{\jobname}|
\end{center}
%
The redirection with prefix
|\childdocforwardprefix[|\textit{prefix}|]{|\textit{dest}|}|
is accomplished by:
%
\begin{center}
\begin{tabular}{l}
|{\edef\jobname{\scantokens\expandafter{\jobname\noexpand}}|\\
|\def\redirectjob |\textit{prefix}|#1~~~{\gdef\jobname{|\textit{dest}|#1}}|\\
|\expandafter\redirectjob\jobname~~~}\input{\jobname}|
\end{tabular}
\end{center}

In an alternative approach,
child documents can be compiled by a specific command line
without additional code or specific definitions:
%
\begin{center}
|... -jobname "|\textit{target}|" "|[\textit{flags}]%
|\includeonly{|\textit{dest}|}\input{|\textit{main}|}"|
\end{center}
%

%%%%%%%%%%%%%%%%%%%%%%%%%%%%%%%%%%%%%%%%%%%%%%%%%%%%%%%%%%%%%%%%%%%%%%%%%%%%%%%%
%%%%%%%%%%%%%%%%%%%%%%%%%%%%%%%%%%%%%%%%%%%%%%%%%%%%%%%%%%%%%%%%%%%%%%%%%%%%%%%%
\section{Information}

%%%%%%%%%%%%%%%%%%%%%%%%%%%%%%%%%%%%%%%%%%%%%%%%%%%%%%%%%%%%%%%%%%%%%%%%%%%%%%%%
\subsection{Copyright}

Copyright \copyright{} 2017--2018 Niklas Beisert

This work may be distributed and/or modified under the
conditions of the \LaTeX{} Project Public License, either version 1.3
of this license or (at your option) any later version.
The latest version of this license is in
  \url{http://www.latex-project.org/lppl.txt}
and version 1.3 or later is part of all distributions of \LaTeX{}
version 2005/12/01 or later.

This work has the LPPL maintenance status `maintained'.

The Current Maintainer of this work is Niklas Beisert.

This work consists of the files |README.txt|, |childdoc.ins| and |childdoc.dtx|
as well as the derived files |childdoc.def|, |cdocsamp.tex|
with |cdocsch1.tex|, |cdocsch2.tex|, |cdocspt3.tex|, |cdocspt4.tex|,
|cdocsdrf.tex|, |cdocsfn1.tex|, |cdocsfn2.tex|
as well as |childdoc.pdf|.

%%%%%%%%%%%%%%%%%%%%%%%%%%%%%%%%%%%%%%%%%%%%%%%%%%%%%%%%%%%%%%%%%%%%%%%%%%%%%%%%
\subsection{Files and Installation}

The package consists of the files:
%
\begin{center}
\begin{tabular}{ll}
    |README.txt|   & readme file \\
    |childdoc.ins| & installation file \\
    |childdoc.dtx| & source file \\
    |childdoc.def| & definition file \\
    |cdocsamp.tex| & sample main file \\
    |cdocsch1.tex| & sample include file \\
    |cdocsch2.tex| & sample include file \\
    |cdocspt3.tex| & sample part file \\
    |cdocspt4.tex| & sample part file \\
    |cdocsdrf.tex| & sample redirection file \\
    |cdocsfn1.tex| & sample redirection file \\
    |cdocsfn2.tex| & sample redirection file \\
    |childdoc.pdf| & manual
\end{tabular}
\end{center}
%
The distribution consists of the files
|README.txt|, |childdoc.ins| and |childdoc.dtx|.
%
\begin{itemize}
\item
Run (pdf)\LaTeX{} on |childdoc.dtx|
to compile the manual |childdoc.pdf| (this file).
\item
Run \LaTeX{} on |childdoc.ins| to create the definitions file |childdoc.def|
and the sample |cdocsamp.tex| with include files
|cdocsch1.tex|, |cdocsch2.tex|, |cdocspt3.tex|, |cdocspt4.tex|,
|cdocsdrf.tex|, |cdocsfn1.tex|, |cdocsfn2.tex|.
Then copy the file |childdoc.def| to an appropriate directory of your \LaTeX{}
distribution, e.g.\ \textit{texmf-root}|/tex/latex/childdoc|.
\end{itemize}

%%%%%%%%%%%%%%%%%%%%%%%%%%%%%%%%%%%%%%%%%%%%%%%%%%%%%%%%%%%%%%%%%%%%%%%%%%%%%%%%
\subsection{Related CTAN Packages}

There are several other packages which offer a similar functionality:
%
\begin{itemize}
\item
The packages
\href{http://ctan.org/pkg/docmute}{\textsf{docmute}},
\href{http://ctan.org/pkg/includex}{\textsf{includex}} and
\href{http://ctan.org/pkg/standalone}{\textsf{standalone}}
provide commands to include only the document body of
a child file thus allowing both files to be compiled individually.
\item
The packages \href{http://ctan.org/pkg/subdocs}{\textsf{subdocs}}
and \href{http://ctan.org/pkg/subfiles}{\textsf{subfiles}}
provide structures in which the main and child documents can be
encapsulated and allowing them to be compiled individually.
The inclusion mechanism is different from the conventional |\include|.
\item
The package \href{http://ctan.org/pkg/combine}{\textsf{combine}}
is an elaborate solution to combine several documents into one.
\end{itemize}
%
See also the CTAN topic \href{http://ctan.org/topic/subdocs}{\textsf{subdocs}}
for further related packages.
The present package differs from the above solutions in that
a document structure constructed with the conventional |\include| mechanism
just needs two extra commands at the top of every file
such that all constituent files can be compiled individually.

%%%%%%%%%%%%%%%%%%%%%%%%%%%%%%%%%%%%%%%%%%%%%%%%%%%%%%%%%%%%%%%%%%%%%%%%%%%%%%%%
%\subsection{Feature Suggestions}
%
%The following is a list of features which may be useful for future
%versions of this package:
%%
%\begin{itemize}
%\item
%\ldots
%\end{itemize}

%%%%%%%%%%%%%%%%%%%%%%%%%%%%%%%%%%%%%%%%%%%%%%%%%%%%%%%%%%%%%%%%%%%%%%%%%%%%%%%%
\subsection{Revision History}

%%%%%%%%%%%%%%%%%%%%%%%%%%%%%%%%%%%%%%%%
\paragraph{v2.0:} 2018/12/30

\begin{itemize}
\item
immediate forward processing
\item
added |\childdocby| mechanism
\item
manual restructured
\end{itemize}

%%%%%%%%%%%%%%%%%%%%%%%%%%%%%%%%%%%%%%%%
\paragraph{v1.6:} 2018/01/17

\begin{itemize}
\item
application for development of include files
\item
corrections to manual
\end{itemize}

%%%%%%%%%%%%%%%%%%%%%%%%%%%%%%%%%%%%%%%%
\paragraph{v1.5:} 2017/05/21

\begin{itemize}
\item
more complete structuring introduced
\item
|\childdocof| introduced
\item
|\childdoc| renamed to |\childdocmain|
\item
|\childredirect| renamed to |\childdocforward| and |\childdocforwardprefix|
and functionality expanded
\end{itemize}

%%%%%%%%%%%%%%%%%%%%%%%%%%%%%%%%%%%%%%%%
\paragraph{v1.0:} 2017/04/27

\begin{itemize}
\item
manual and install package
\item
first version published on CTAN
\end{itemize}

%%%%%%%%%%%%%%%%%%%%%%%%%%%%%%%%%%%%%%%%
\paragraph{v0.6:} 2017/04/26

\begin{itemize}
\item
redirection mechanism added
\end{itemize}

%%%%%%%%%%%%%%%%%%%%%%%%%%%%%%%%%%%%%%%%
\paragraph{v0.5:} 2017/04/26

\begin{itemize}
\item
functionality in definition file
\end{itemize}


%%%%%%%%%%%%%%%%%%%%%%%%%%%%%%%%%%%%%%%%%%%%%%%%%%%%%%%%%%%%%%%%%%%%%%%%%%%%%%%%
%%%%%%%%%%%%%%%%%%%%%%%%%%%%%%%%%%%%%%%%%%%%%%%%%%%%%%%%%%%%%%%%%%%%%%%%%%%%%%%%
%%%%%%%%%%%%%%%%%%%%%%%%%%%%%%%%%%%%%%%%%%%%%%%%%%%%%%%%%%%%%%%%%%%%%%%%%%%%%%%%
\appendix

\settowidth\MacroIndent{\rmfamily\scriptsize 000\ }

 \DocInput{childdoc.dtx}

\end{document}
%</driver>
% \fi
%
% %%%%%%%%%%%%%%%%%%%%%%%%%%%%%%%%%%%%%%%%%%%%%%%%%%%%%%%%%%%%%%%%%%%%%%%%%%%%%%
% %%%%%%%%%%%%%%%%%%%%%%%%%%%%%%%%%%%%%%%%%%%%%%%%%%%%%%%%%%%%%%%%%%%%%%%%%%%%%%
% \section{Sample}
%\iffalse
%<*samplemain>
%\fi
%
% The following presents a sample document
% with two chapters, two parts, a title page,
% a compile flag as well as three forwarding files to set the flag.
% It consists of eight |.tex| files:
% \begin{center}
% \begin{tabular}{ll}
% |cdocsamp.tex|&main file\\
% |cdocsch1.tex|&include file for chapter 1\\
% |cdocsch2.tex|&include file for chapter 2\\
% |cdocspt3.tex|&include file for part 3\\
% |cdocspt4.tex|&include file for part 4\\
% |cdocsdrf.tex|&forwarding file for main file in draft mode\\
% |cdocsfi1.tex|&forwarding file for final version of chapter 1\\
% |cdocsfi2.tex|&forwarding file for final version of chapter 2\\
% \end{tabular}
% \end{center}
% Each of the eight files can be compiled directly by the \LaTeX{} compiler.
%
% %%%%%%%%%%%%%%%%%%%%%%%%%%%%%%%%%%%%%%
% \paragraph{Main File.}
%
% The main file is called |cdocsamp.tex|.
%
% Load the \textsf{childdoc} definitions and
% declare the filename for the main document:
%    \begin{macrocode}
\input{childdoc.def}
\childdocmain{}
%    \end{macrocode}

% Optional override for |\version| flag:
%    \begin{macrocode}
%%\ifchilddoc\else\providecommand{\version}{draft}\fi
%    \end{macrocode}

% Define the default values for the |\version| flag
% (|final| for the main file and |draft| for childs):
%    \begin{macrocode}
\ifchilddoc
\providecommand{\version}{draft}
\else
\providecommand{\version}{final}
\fi
%    \end{macrocode}

% Load the standard document class:
%    \begin{macrocode}
\documentclass[12pt]{article}
%    \end{macrocode}

% Start the document body:
%    \begin{macrocode}
\begin{document}
%    \end{macrocode}

% Declare a title page.
% Print title, part of document being processed and version flag:
%    \begin{macrocode}
\addtocounter{page}{-1}
\begin{center}
{\LARGE\bfseries{}childdoc example\par}
\vspace{1cm}
\ifchilddoc
\ifchilddocmanual part\else chapter\fi:
`\childdocname' of `\childdocjob'\par
\else
main document: `\childdocjob'\par
\fi
version: \version\par
\end{center}
\newpage
%    \end{macrocode}

% Manually include selected file,
% otherwise process as usual:
%    \begin{macrocode}
\ifchilddocmanual
\section*{part `\childdocname'}
\input{\childdocname}
\else
%    \end{macrocode}

% Include the two chapters:
%    \begin{macrocode}
\include{cdocsch1}
\include{cdocsch2}
%    \end{macrocode}

% Include the two parts unless only chapters should be displayed:
%    \begin{macrocode}
\ifchilddoc\else
\section{part three}
\input{cdocspt3}
\section{part four}
\input{cdocspt4}
\fi
%    \end{macrocode}

% Process as usual until here:
%    \begin{macrocode}
\fi
%    \end{macrocode}

% End of document body:
%    \begin{macrocode}
\end{document}
%    \end{macrocode}
%\iffalse
%</samplemain>
%\fi
%
% %%%%%%%%%%%%%%%%%%%%%%%%%%%%%%%%%%%%%%
% \paragraph{Chapter Include Files.}
%
% The include files are called |cdocsch1.tex| and |cdocsch2.tex|.
%
%\iffalse
%<*samplechap1|samplechap2>
%\fi

% Optional override for |\version| flag:
%    \begin{macrocode}
%%\providecommand{\version}{final}
%    \end{macrocode}

% Include the main document:
%    \begin{macrocode}
\input{childdoc.def}
\childdocof{cdocsamp}
%    \end{macrocode}

%\iffalse
%</samplechap1|samplechap2>
%\fi
%
%\iffalse
%<*samplechap1>
%\fi
% Some text for chapter 1:
%    \begin{macrocode}
\section{one}
some text in chapter one
%    \end{macrocode}

%\iffalse
%</samplechap1>
%\fi
% Some text for chapter 2:
%\iffalse
%<*samplechap2>
%\fi
%    \begin{macrocode}
\section{two}
more text in chapter two
%    \end{macrocode}

%\iffalse
%</samplechap2>
%\fi
%
% %%%%%%%%%%%%%%%%%%%%%%%%%%%%%%%%%%%%%%
% \paragraph{Part Include Files.}
%
% The include files are called |cdocspt3.tex| and |cdocspt4.tex|.
%
%\iffalse
%<*samplepart3|samplepart4>
%\fi

% Optional override for |\version| flag:
%    \begin{macrocode}
%%\providecommand{\version}{final}
%    \end{macrocode}

% Include the main document:
%    \begin{macrocode}
\input{childdoc.def}
\childdocby{cdocsamp}
%    \end{macrocode}

%\iffalse
%</samplepart3|samplepart4>
%\fi
%
%\iffalse
%<*samplepart3>
%\fi
% Some text for part 3:
%    \begin{macrocode}
some text in part three
%    \end{macrocode}

%\iffalse
%</samplepart3>
%\fi
% Some text for part 4:
%\iffalse
%<*samplepart4>
%\fi
%    \begin{macrocode}
more text in part four
%    \end{macrocode}

%\iffalse
%</samplepart4>
%\fi
%
% %%%%%%%%%%%%%%%%%%%%%%%%%%%%%%%%%%%%%%
% \paragraph{Forwarding for a Complete Draft.}
%
% The following forwarding file |cdocsdrf.tex|
% compiles the main document in draft mode:
%\iffalse
%<*sampledraft>
%\fi
%    \begin{macrocode}
\def\version{draft}
\input{childdoc.def}
\childdocforward{cdocsamp}
%    \end{macrocode}

%\iffalse
%</sampledraft>
%\fi
%
% %%%%%%%%%%%%%%%%%%%%%%%%%%%%%%%%%%%%%%
% \paragraph{Forwarding for Final Version of the Chapters.}
%
% The following forwarding files |cdocsfn1.tex| and |cdocsfn2.tex|
% (with identical content)
% compile the final versions of the child documents
% |cdocsch1.tex| and |cdocsch2.tex|, respectively:
%\iffalse
%<*samplefinal>
%\fi
%    \begin{macrocode}
\def\version{final}
\input{childdoc.def}
\childdocforwardprefix[cdocsamp]{cdocsfn}{cdocsch}
%    \end{macrocode}

%\iffalse
%</samplefinal>
%\fi
%
% %%%%%%%%%%%%%%%%%%%%%%%%%%%%%%%%%%%%%%
% \paragraph{Command Line Processing.}
%
% The following three command lines generate the output files
% |cdocscld|, |cdocscl1| and |cdocscl2|
% which should be identical to
% |cdocsdrf|, |cdocsch1| and |cdocsfn2|, respectively:
% \begin{center}
% \begin{tabular}{l}
% |latex -jobname cdocscld \|\\
% |  "\def\version{draft}\input{childdoc.def}\childdocforward{cdocsamp}"|\\
% |latex -jobname cdocscl1 \|\\
% |  "\input{childdoc.def}\childdocforward[cdocsamp]{cdocsch1}"|\\
% |latex -jobname cdocscl2 \|\\
% |  "\def\version{final}\input{childdoc.def}\childdocforward{cdocsch2}"|
% \end{tabular}
% \end{center}
% Note that the trailing backslash on each first line
% merely continues the input to the second line
% (for convenient cut ant paste).
% Furthermore, the command |latex| can be replaced by any
% of its alternative versions such as |pdflatex|.
%
% %%%%%%%%%%%%%%%%%%%%%%%%%%%%%%%%%%%%%%%%%%%%%%%%%%%%%%%%%%%%%%%%%%%%%%%%%%%%%%
% %%%%%%%%%%%%%%%%%%%%%%%%%%%%%%%%%%%%%%%%%%%%%%%%%%%%%%%%%%%%%%%%%%%%%%%%%%%%%%
% \section{Implementation}
%\iffalse
%<*package>
%\fi
%
% This section describes the definitions file |childdoc.def|.

% The definitions cannot be loaded using |\usepackage| or |\RequirePackage|
% which has a mechanism to prevent loading a style file more than once.
% When loading the definitions by means of |\input|
% multiple instances have to be prevented manually:
%\iffalse
%This code needs to be before the `\ProvidesFile' directive
%which is defined at the beginning of this file.
%Therefore it is also placed there and commented out here.
%</package>
%<*discard>
%\fi
%    \begin{macrocode}
\ifdefined\childdocmain\endinput\fi
%    \end{macrocode}
%\iffalse
%</discard>
%<*package>
%\fi
%
% \macro{\ifchilddoc}
% \macro{\ifchilddocmanual}
% The conditional |\ifchilddoc| tells whether a
% child (true) or main (false) document is being compiled.
% The conditional |\ifchilddocmanual| tells whether
% the |\includeonly| mechanism is used (false) or
% the selection of child files must be performed manually (true).
% The definitions initialise to false:
%    \begin{macrocode}
\newif\ifchilddoc
\newif\ifchilddocmanual
%    \end{macrocode}

% \macro{\childdocname}
% \macro{\childdocjob}
% The macro |\childdocname| stores the name of the main document
% to be compiled. The macro |\childdocjob| stores the name of
% the document on which the \LaTeX{} compiler was originally invoked.
% The content of |\jobname| cannot be compared
% to filenames specified in the source due to different catcodes.
% The following code rescans |\jobname|, stores the result
% in |\childdocname| and saves a copy in |\childdocjob|:
%    \begin{macrocode}
\edef\childdocname{\scantokens\expandafter{\jobname\noexpand}}
\let\childdocjob\childdocname
%    \end{macrocode}

% \macro{\childdocdisable}
% The macro |\childdocdisable| prevents the main file
% from being processed more than once.
% At this stage, the main document command |\childdocmain|
% is assumed to be called once again where it should do nothing.
% Any subsequent call to it should prevent
% a secondary processing of the main document
% It overwrites the forwarding commands
% |\childdocof| and |\childdocforward|
% with empty macros to prevent further inclusions of the main document:
%    \begin{macrocode}
\newcommand{\childdocdisable}
{
  \renewcommand{\childdocmain}[1]{\renewcommand{\childdocmain}[1]{\endinput}}
  \renewcommand{\childdocof}[1]{}
  \renewcommand{\childdocby}[2][]{}
  \renewcommand{\childdocforward}[2][]{}
  \renewcommand{\childdocdisable}{}
}
%    \end{macrocode}

% \macro{\childdocmain}
% The macro |\childdocmain| is to be called at the top of the main file
% with nothing or the main filename (without extension) as argument.
% First, it breaks loops.
% If the argument is not empty and does not match |\childdocname|
% (which is set by the first inclusion of |childdoc.def|),
% |\ifchilddoc| is set to true, |\includeonly| is applied to the child file
% and |\jobname| is set to the main file
% (for proper handling of |.aux| files):
%    \begin{macrocode}
\newcommand{\childdocmain}[1]
{
  \childdocdisable\childdocmain{}
  \if?#1?\else
    \begingroup
      \def\childdoctmp{#1}
      \ifx\childdoctmp\childdocname
        \def\childdoctmp{}
      \else
        \def\childdoctmp
        {
          \childdoctrue
          \includeonly{\childdocname}
          \def\childdocjob{#1}
          \def\jobname{#1}
        }
      \fi
      \expandafter
    \endgroup
    \childdoctmp
  \fi
}
%    \end{macrocode}

% \macro{\childdocof}
% The command |\childdocof| redirects
% compilation to the main file |#1|.
%    \begin{macrocode}
\newcommand{\childdocof}[1]
{
  \childdocdisable
  \childdoctrue
  \includeonly{\childdocname}
  \def\jobname{#1}
  \def\childdocjob{#1}
  \input{#1}
}
%    \end{macrocode}

% \macro{\childdocby}
% The command |\childdocby| ....
%    \begin{macrocode}
\newcommand{\childdocby}[2][]
{
  \childdocdisable
  \childdoctrue
  \childdocmanualtrue
  \if?#1?\else
    \def\jobname{#2}
  \fi
  \def\childdocjob{#2}
  \input{#2}
  \endinput
}
%    \end{macrocode}

% \macro{\childdocforward}
% The command |\childdocforward| redirects
% compilation to the main file or
% (if the optional argument is given) a child file.
% Parameters are set as if the main file
% or a child file starting with |\childdocof| was compiled.
% Then compilation is handed over to the main file:
%    \begin{macrocode}
\newcommand{\childdocforward}[2][]
{
  \begingroup
    \if?#1?
      \def\childdoctmp
      {
        \def\childdocname{#2}
        \def\childdocjob{#2}
        \def\jobname{#2}
        \input{#2}
        \endinput
      }
    \else
      \def\childdoctmp
      {
        \childdocdisable
        \def\childdocname{#2}
        \childdoctrue
        \includeonly{#2}
        \def\childdocjob{#1}
        \def\jobname{#1}
        \input{#1}
        \endinput
      }
    \fi
    \expandafter
  \endgroup
  \childdoctmp
}
%    \end{macrocode}

% \macro{\childdocforwardprefix}
% The command |\childdocforwardprefix| redirects
% compilation to the main or a child file by means of a pattern.
% The prefix |#1| in the current filename is replaced by |#2|
% and the suffix of the current filename is kept
% (it is assumed that the filename does not contain the substring `|~~~|'
% which is used as a delimiter).
% Compilation is handed over to the new file by |\childdocforward|:
%    \begin{macrocode}
\newcommand{\childdocforwardprefix}[3][]
{
  \begingroup
    \def\childdocextract #2##1~~~{\def\childdoctmp{\childdocforward[#1]{#3##1}}}
    \expandafter\childdocextract\childdocname~~~
    \expandafter
  \endgroup
  \childdoctmp
}
%    \end{macrocode}

% \macro{\childdoc}
% The deprecated macro |\childdoc| is a legacy version of |\childdocmain|:
%    \begin{macrocode}
\newcommand{\childdoc}{\childdocmain}
%    \end{macrocode}

% \macro{\childdocredirect}
% The deprecated macro |\childdocredirect| is a legacy version
% of |\childdocforward| and |\childdocforwardprefix|:
%    \begin{macrocode}
\newcommand{\childdocredirect}[2][]
{
  \begingroup
    \if?#1?
      \def\childdoctmp{\childdocforward{#2}}
    \else
      \def\childdoctmp{\childdocforwardprefix{#1}{#2}}
    \fi
    \expandafter
  \endgroup
  \childdoctmp
}
%    \end{macrocode}

%\iffalse
%</package>
%\fi
%
\endinput
|\\
|\childdocby{|\textit{main}|}|\\
\end{tabular}
\end{center}
%
The directive |\childdocby| is similar to |\childdocof|
described in \secref{sec:include},
but the subsequent selection of content must be done manually.
To that end, both |\ifchilddoc| and |\ifchilddocmanual|
will be true upon processing of a part,
and the name of the part is stored in |\childdocname|.
Note that |\jobname| will be set to the filename of the current part
so that each part receives an individual |.aux| file
that does not interfere with the |.aux| file(s) of the main document.
This behaviour can be altered by the alternative form
|\childdocby[*]{|\textit{main}|}| (with a non-empty optional argument)
which uses the |.aux| file of the main document
by setting |\jobname| to \textit{main}.

%%%%%%%%%%%%%%%%%%%%%%%%%%%%%%%%%%%%%%%%%%%%%%%%%%%%%%%%%%%%%%%%%%%%%%%%%%%%%%%%
\subsection{Driver Development}
\label{sec:driver}

The \textsf{childdoc} mechanism can also be use for the development
of definition files such as \LaTeX{} styles or classes.
This case differs from the above setup with multiple parts
included by |\include| in that no |\includeonly| should be invoked.
This can be achieved by starting the include file
(before |\ProvidesPackage|) with:
%
\begin{center}
\begin{tabular}{l}
|% \iffalse
%
% childdoc.dtx Copyright (C) 2017-2018 Niklas Beisert
%
% This work may be distributed and/or modified under the
% conditions of the LaTeX Project Public License, either version 1.3
% of this license or (at your option) any later version.
% The latest version of this license is in
%   http://www.latex-project.org/lppl.txt
% and version 1.3 or later is part of all distributions of LaTeX
% version 2005/12/01 or later.
%
% This work has the LPPL maintenance status `maintained'.
%
% The Current Maintainer of this work is Niklas Beisert.
%
% This work consists of the files childdoc.dtx and childdoc.ins
% and the derived files childdoc.def and cdocsamp.tex with
% cdocsch1.tex, cdocsch2.tex, cdocsdrf.tex, cdocsfn1.tex, cdocsfn2.tex.
%
%<package>\ifdefined\childdocmain\endinput\fi
%<package>\ProvidesFile{childdoc.def}[2018/12/30 v2.0 child document driver]
%<samplemain>\ProvidesFile{cdocsamp.tex}[2018/12/30 v2.0 sample for childdoc]
%<*driver>
%\ProvidesFile{childdoc.drv}[2018/12/30 v2.0 childdoc reference manual file]
\PassOptionsToClass{10pt,a4paper}{article}
\documentclass{ltxdoc}

\usepackage[margin=35mm]{geometry}
\usepackage{hyperref}
\usepackage{hyperxmp}
\usepackage[usenames]{color}

\hypersetup{colorlinks=true}
\hypersetup{pdfstartview=FitH}
\hypersetup{pdfpagemode=UseNone}
\hypersetup{pdfsource={}}
\hypersetup{pdflang={en-UK}}
\hypersetup{pdfcopyright={Copyright 2017-2018 Niklas Beisert.
  This work may be distributed and/or modified under the
  conditions of the LaTeX Project Public License, either version 1.3
  of this license or (at your option) any later version.}}
\hypersetup{pdflicenseurl={http://www.latex-project.org/lppl.txt}}
\hypersetup{pdfcontactaddress={ETH Zurich, ITP, HIT K,
  Wolfgang-Pauli-Strasse 27}}
\hypersetup{pdfcontactpostcode={8093}}
\hypersetup{pdfcontactcity={Zurich}}
\hypersetup{pdfcontactcountry={Switzerland}}
\hypersetup{pdfcontactemail={nbeisert@itp.phys.ethz.ch}}
\hypersetup{pdfcontacturl={http://people.phys.ethz.ch/\xmptilde nbeisert/}}

\newcommand{\secref}[1]{\hyperref[#1]{section \ref*{#1}}}

\parskip1ex
\parindent0pt
\let\olditemize\itemize
\def\itemize{\olditemize\parskip0pt}

\begin{document}

\title{The \textsf{childdoc} Package}
\hypersetup{pdftitle={The childdoc Package}}
\author{Niklas Beisert\\[2ex]
  Institut f\"ur Theoretische Physik\\
  Eidgen\"ossische Technische Hochschule Z\"urich\\
  Wolfgang-Pauli-Strasse 27, 8093 Z\"urich, Switzerland\\[1ex]
  \href{mailto:nbeisert@itp.phys.ethz.ch}
  {\texttt{nbeisert@itp.phys.ethz.ch}}}
\hypersetup{pdfauthor={Niklas Beisert}}
\hypersetup{pdfsubject={Manual for the LaTeX2e Package childdoc}}
\date{30 December 2018, \textsf{v2.0}}
\maketitle

\begin{abstract}\noindent
\textsf{childdoc} is a \LaTeXe{} package
that enables the direct compilation
of document sections included by |\include|
to individual files.
\end{abstract}

\begingroup
\parskip0ex
\tableofcontents
\endgroup

%%%%%%%%%%%%%%%%%%%%%%%%%%%%%%%%%%%%%%%%%%%%%%%%%%%%%%%%%%%%%%%%%%%%%%%%%%%%%%%%
%%%%%%%%%%%%%%%%%%%%%%%%%%%%%%%%%%%%%%%%%%%%%%%%%%%%%%%%%%%%%%%%%%%%%%%%%%%%%%%%
\section{Introduction}

\LaTeX{} provides a mechanism to structure a large document (such as a book)
into a main file and several child files (containing the chapters)
using the |\include| command.
This mechanism is beneficial for documents
which span hundreds of pages in order to
make the source file(s) more manageable.
Moreover, compilation can be restricted to
selected child files by means of the |\includeonly| command.
The latter feature can be used to reduce the compilation time while editing
(this was significantly more useful in the earlier days of \LaTeX{})
or to generate a smaller document which is easier to navigate.
Another application of |\includeonly| is to generate
documents consisting of selected parts of the complete document.

However, there are a few drawbacks of the plain |\include| mechanism:
\begin{itemize}
\item
The child files cannot be compiled on their own,
they can only be compiled via the main file.
A naive editing environment
(such as a text editor with an option
to have the current file processed by \LaTeX)
may require one to switch to the main file before compiling;
attempting to compile the child file produces errors.
\item
The main file must be modified (each time)
to adjust the |\includeonly| command
to the present needs. This easily leaves the main file in a messy state.
\item
The generated document will always carry the filename
of the main document. This is inconvenient if
several child files are to be compiled and
to be kept for distribution.
\end{itemize}

The present package provides a simple interface
to make child files individually compilable by \LaTeX{}.
Compiling a child file then has the same effect as compiling
the main file with an |\includeonly| command
to select the appropriate child.
Moreover the generated document will carry the name of the child
rather than the main file.
This resolves all three above issues.

This feature is meant to make the editing of books,
thesis documents and lecture notes somewhat more convenient.
However, the package can also be used efficiently for
composing a series of documents (such as exercise sheets)
which are typically distributed individually.
It then assists the author in generating the individual documents
(potentially in different versions)
as well as a document containing the collected series.
Another application is in developing style files
or other kinds of included material
where compilation of the style file could redirect
to a sample or test file.

%%%%%%%%%%%%%%%%%%%%%%%%%%%%%%%%%%%%%%%%%%%%%%%%%%%%%%%%%%%%%%%%%%%%%%%%%%%%%%%%
%%%%%%%%%%%%%%%%%%%%%%%%%%%%%%%%%%%%%%%%%%%%%%%%%%%%%%%%%%%%%%%%%%%%%%%%%%%%%%%%
\section{Usage}

First of all, the package \textsf{childdoc} is \emph{not} a standard
\LaTeXe{} |.sty| style file! Therefore it needs to be invoked in
a non-standard way.

%%%%%%%%%%%%%%%%%%%%%%%%%%%%%%%%%%%%%%%%%%%%%%%%%%%%%%%%%%%%%%%%%%%%%%%%%%%%%%%%
\subsection{Included Files}
\label{sec:include}

%%%%%%%%%%%%%%%%%%%%%%%%%%%%%%%%%%%%%%%%
\DescribeMacro{\childdocmain}
To use the package, add the commands
\begin{center}
\begin{tabular}{l}
|\input{childdoc.def}|\\
|\childdocmain{}|\\
\end{tabular}
\end{center}
at the very top of the main \LaTeX{} file,
in particular \emph{before} the |\documentclass| statement!
The argument of |\childdocmain| should be left empty
(but it must be present).

%%%%%%%%%%%%%%%%%%%%%%%%%%%%%%%%%%%%%%%%
\DescribeMacro{\childdocof}
Furthermore, add the commands
\begin{center}
\begin{tabular}{l}
|\input{childdoc.def}|\\
|\childdocof{|\textit{main}|}|\\
\end{tabular}
\end{center}
at the top of every child file \textit{child}
which is included by |\include{|\textit{child}|}|
from within the main file
(or at least for those files to be compiled individually).
The argument \textit{main} must be the filename of the main file.

There are a couple of
considerations in setting up the main and child documents:

%%%%%%%%%%%%%%%%%%%%%%%%%%%%%%%%%%%%%%%%
\paragraph{Restrictions.}

Please note the following restrictions:
\begin{itemize}
\item
|\childdocmain| must be called with one argument \textit{main}
to ensure compatibility with earlier version of the package.
It must either be empty (|\childdocmain{}|)
or precisely match the filename of the main file in which it is specified.
See \secref{sec:detection} for further information.
\item
The filename \textit{main} must be specified without the |.tex| extension.
\item
The filename \textit{main} is case sensitive
(even in case-insensitive file systems)
due to internal string comparison.
\item
The argument \textit{main} should be fully expanded, it cannot be a macro.
\item
Subdirectories and special characters should be avoided in filenames.
\item
The command |\childdocmain{|\textit{main}|}| must be followed by a whitespace.
It should not be followed immediately by another command
or by a comment mark `|%|'.
This is because the \TeX{} parser reads the token immediately following
the argument of |\childdocmain| and puts it
at the beginning of every child section;
however, a white\-space is ignored.
\end{itemize}

%%%%%%%%%%%%%%%%%%%%%%%%%%%%%%%%%%%%%%%%
\paragraph{Content of Main File.}

It is advisable to place all content in the child files included by |\include|.
Any output contained in the main file will appear in all child documents
unless suppressed manually;
it cannot be suppressed automatically by the |\includeonly| directive
and thus should normally be avoided.
A method to include some content in the main file
by means of conditional processing is described in \secref{sec:conditional}.

%%%%%%%%%%%%%%%%%%%%%%%%%%%%%%%%%%%%%%%%
\paragraph{Page Numbering.}

When only a part of the document is compiled,
the appropriate numbering of pages
(as well as other status parameters)
is determined from the |.aux| files.
The latter contain information from previous passes.
However this information needs to propagate through
all intermediate child documents.
Therefore the page numbering in child documents may well
be inconsistent until the complete document is compiled at least once.

A useful (if unconventional) way to always ensure a consistent
page numbering is to restart the numbering in each child document
and denote the pages by `\textit{child}|.|\textit{page}'
where \textit{child} represents the chapter/section number of the child file.
This can be achieved by the command
|\numberwithin{page}{|\textit{child}|}|
of the \textsf{amsmath} package
where \textit{child} can be |chapter| or |section|
depending on the chosen structuring.
Alternatively, one can modify the macro |\thepage| appropriately
and reset the counter |page| at the start of each child file.

%%%%%%%%%%%%%%%%%%%%%%%%%%%%%%%%%%%%%%%%%%%%%%%%%%%%%%%%%%%%%%%%%%%%%%%%%%%%%%%%
\subsection{Conditional Processing}
\label{sec:conditional}

The package provides a mechanism to compile different versions
of a document. To customise the versions further some conditional processing
can come in handy to distinguish which version is being compiled.
The package provides two macros to describe the compilation context:

%%%%%%%%%%%%%%%%%%%%%%%%%%%%%%%%%%%%%%%%
\DescribeMacro{\ifchilddoc}
The conditional |\ifchilddoc| distinguishes between the compilation of
child documents and the main document:
%
\begin{center}
|\ifchilddoc |\textit{child-code}| |[|\||else |\textit{main-code}]| \||fi|
\end{center}

%%%%%%%%%%%%%%%%%%%%%%%%%%%%%%%%%%%%%%%%
\DescribeMacro{\childdocname}
\DescribeMacro{\childdocjob}
The macro |\childdocname| contains the filename (without extension)
of the main or child file being processed.
Note that |\childdocjob| will always contain the name of the main file.

%%%%%%%%%%%%%%%%%%%%%%%%%%%%%%%%%%%%%%%%
\paragraph{Title Page.}

Conditional processing can be used to include a title or banner page
in the main document when proper precautions are taken.
Importantly, the code in the main file should ensure that the page counter
(as well as other status parameters which are stored in the |.aux| files)
takes the same value after the conditional processing.
Otherwise the page numbers may take divergent values
depending on which part is compiled.

For example, a title page could be declared by:
%
\begin{center}
\begin{tabular}{l}
|\ifchilddoc\||else|\\
|\addtocounter{page}{-1}|\\
\textit{code for title page}\\
|\newpage|\\
|\||fi|
\end{tabular}
\end{center}
%
A banner page for the child documents can be generated by:
%
\begin{center}
\begin{tabular}{l}
|\ifchilddoc|\\
|\addtocounter{page}{-1}|\\
\textit{code for banner page}\\
|\newpage|\\
|\||fi|
\end{tabular}
\end{center}
%
Here one could write a message such as:
\begin{center}
|This is the part \childdocname{} of \childdocjob{}.|
\end{center}

%%%%%%%%%%%%%%%%%%%%%%%%%%%%%%%%%%%%%%%%%%%%%%%%%%%%%%%%%%%%%%%%%%%%%%%%%%%%%%%%
\subsection{Flags}
\label{sec:flags}

The package makes it easy to generate different versions
of the main or child documents.
To this end compilation flags can be defined
and assigned different default values.
They will be particularly useful in conjunction
with the forwarding mechanism described in \secref{sec:forward}.

For example, it may be useful to have a flag |\version|
which can be set to |draft| or |final|.
The document source will contain some conditional code
depending on the value of |\version|.
Suppose further, the flag should default to |final| for the main file
and to |draft| for child files
which is a natural assignment for editing the document.
This is achieved by placing the following code
in the preamble of the main document
(below the |\childdocmain| directive):
%
\begin{center}
\begin{tabular}{l}
|\ifchilddoc|\\
|\providecommand{\version}{draft}|\\
|\||else|\\
|\providecommand{\version}{final}|\\
|\||fi|
\end{tabular}
\end{center}
%
The definition by |\providecommand| makes sure
that previous definitions are not overwritten.
Further statements |\providecommand{\version}{...}|
can thus be added before the above code to override it.

For the main file, one might add a line
(between |\childdocmain| and the above block)
%
\begin{center}
|%\ifchilddoc\||else\providecommand{\version}{draft}\||fi|
\end{center}
%
which can be uncommented to produce a draft version.
Likewise one can add a line to the very top of a child file
(above the |\childdocof{|\textit{main}|}| directive)
%
\begin{center}
|%\providecommand{\version}{final}|
\end{center}
%
which can be uncommented to produce the final version of this child document.

%%%%%%%%%%%%%%%%%%%%%%%%%%%%%%%%%%%%%%%%%%%%%%%%%%%%%%%%%%%%%%%%%%%%%%%%%%%%%%%%
\subsection{Forwarding}
\label{sec:forward}

Different versions of the main or child documents
using compilation flags as described in \secref{sec:flags}
can be (permanently) stored in different files
for convenient compilation, viewing and distribution.
To this end, the package defines a command
to pass on compilation to a different file:

%%%%%%%%%%%%%%%%%%%%%%%%%%%%%%%%%%%%%%%%
\DescribeMacro{\childdocforward}
The command |\childdocforward| redirects processing to
another source file:
%
\begin{center}
\begin{tabular}{l}
|\input{childdoc.def}|\\
|\childdocforward[|\textit{main}|]{|\textit{dest}|}|\\
\end{tabular}
\end{center}
%
The argument \textit{dest} is the destination file
(without extension).
It should be the main file or one of the child files.
Note that further \textsf{childdoc} directives
such as |\childdocof| and |\childdocforward|
in the indicated file will be processed in this form.
The optional argument \textit{main}
passes on directly to the main file \textit{main}
while pretending to compile the child \textit{dest}.
This form behaves as if \textit{dest}
issues |\childdocof{|\textit{main}|}| right away,
and no further \textsf{childdoc} directives will be processed.

%%%%%%%%%%%%%%%%%%%%%%%%%%%%%%%%%%%%%%%%
\DescribeMacro{\...prefix}
In the alternative form |\childdocforwardprefix|,
%
\begin{center}
\begin{tabular}{l}
|\input{childdoc.def}|\\
|\childdocforwardprefix[|\textit{main}|]{|\textit{prefix}|}{|\textit{dest}|}|
\end{tabular}
\end{center}
%
the destination file is determined by a pattern
depending on the current file:
To make this work, the current file must be called
`{\textit{prefix}\hspace{0.2em}\textit{suffix}}'
with \textit{prefix} matching precisely the argument.
Processing is then passed on to the file
`{\textit{dest}\hspace{0.2em}\textit{suffix}}'.
Surely, the same effect is achieved by
directly specifying the
argument `{\textit{dest}\hspace{0.2em}\textit{suffix}}'
in the first form.
However, that requires to set up a different file
for each child. With the alternative form of the command
all these files can have exactly the same content
which simplifies setting them up and maintaining them.

For example, the following file |draft.tex|
with a compilation flag |\version| as described in \secref{sec:flags}
compiles the main document as a draft:
%
\begin{center}
\begin{tabular}{l}
|\def\version{draft}|\\
|\input{childdoc.def}|\\
|\childdocforward{|\textit{main}|}|
\end{tabular}
\end{center}
%
Likewise, the following files |final|\textit{nn}|.tex|
compile the final version of the child document
|child|\textit{nn}|.tex|:
%
\begin{center}
\begin{tabular}{l}
|\def\version{final}|\\
|\input{childdoc.def}|\\
|\childdocforwardprefix{final}{child}|
\end{tabular}
\end{center}
%

Note that when several versions of a main file and/or of each child file
are to be generated, it may be convenient to set up a |Makefile| or
shell script to automatise the process.

%%%%%%%%%%%%%%%%%%%%%%%%%%%%%%%%%%%%%%%%%%%%%%%%%%%%%%%%%%%%%%%%%%%%%%%%%%%%%%%%
\subsection{Command Line Processing}
\label{sec:commandline}

The effect of redirection files can also be achieved by invoking
the \LaTeX{} compiler with a more elaborate command line.
Most conveniently this should be done as part
of a shell script or a |Makefile|.

When using \textsf{childdoc} in the main file, the following
command lines effectively perform a redirection
(note that depending on the shell being used,
backslashes may have to be doubled: `|\|' $\to$ `|\\|'):
%
\begin{center}
|... -jobname "|\textit{target}|" |\\|"|[\textit{flags}]%
|\input{childdoc.def}\childdocforward[|\textit{main}|]{|\textit{dest}|}"|
\end{center}
%
Here \textit{target} is the name of the output file,
\textit{main} is the name of the main file
and \textit{dest} is the name of the main or child file to be processed
(all filenames without extensions).
The optional argument \textit{main} can be omitted
if \textit{main} matches \textit{dest}.
Optionally, compilation \textit{flags} can be defined via |\def| commands.
This command line makes the \TeX{} engine believe
it is compiling the file \textit{target}
whose content is specified as the latter parameter.
The provided code then forwards the processing to
\textit{main} or \textit{dest} as described in \secref{sec:forward}.

%%%%%%%%%%%%%%%%%%%%%%%%%%%%%%%%%%%%%%%%%%%%%%%%%%%%%%%%%%%%%%%%%%%%%%%%%%%%%%%%
\subsection{Include by Input}
\label{sec:input}

Including child documents by |\include| has some restrictions by design.
Most notably, the content of a child document always occupies
its own set of pages; pages cannot be shared between child documents.
Usually, this behaviour makes perfect sense
because each child document contain an essential part of the document.
However, in some situations it may be desirable to compose
a document from a collection of parts
without having mandatory page breaks between then.
For this case, the package
provides a mechanism to include parts
by |\input| which can also be processed individually.
However, by construction this mechanism
requires manual handling of the content to be output.

%%%%%%%%%%%%%%%%%%%%%%%%%%%%%%%%%%%%%%%%
\DescribeMacro{\ifchilddocmanual}
The main file should be prepared as usual, see \secref{sec:include}.
However, the document body must make a distinction
between processing of an individual part and of the main document, e.g.:
%
\begin{center}
\begin{tabular}{l}
|\ifchilddocmanual|\\
|\input{\childdocname}|\\
|\||else|\\
\textit{document body with }|\input{|\textit{part}|}|\\
|\||fi|
\end{tabular}
\end{center}
%
The conditional |\ifchilddocmanual| is true whenever
a part to be included by |\input| is being compiled,
and the name of the part is stored in |\childdocname|.

%%%%%%%%%%%%%%%%%%%%%%%%%%%%%%%%%%%%%%%%
\DescribeMacro{\childdocby}
Each part to be included by |\input| should start with:
%
\begin{center}
\begin{tabular}{l}
|\input{childdoc.def}|\\
|\childdocby{|\textit{main}|}|\\
\end{tabular}
\end{center}
%
The directive |\childdocby| is similar to |\childdocof|
described in \secref{sec:include},
but the subsequent selection of content must be done manually.
To that end, both |\ifchilddoc| and |\ifchilddocmanual|
will be true upon processing of a part,
and the name of the part is stored in |\childdocname|.
Note that |\jobname| will be set to the filename of the current part
so that each part receives an individual |.aux| file
that does not interfere with the |.aux| file(s) of the main document.
This behaviour can be altered by the alternative form
|\childdocby[*]{|\textit{main}|}| (with a non-empty optional argument)
which uses the |.aux| file of the main document
by setting |\jobname| to \textit{main}.

%%%%%%%%%%%%%%%%%%%%%%%%%%%%%%%%%%%%%%%%%%%%%%%%%%%%%%%%%%%%%%%%%%%%%%%%%%%%%%%%
\subsection{Driver Development}
\label{sec:driver}

The \textsf{childdoc} mechanism can also be use for the development
of definition files such as \LaTeX{} styles or classes.
This case differs from the above setup with multiple parts
included by |\include| in that no |\includeonly| should be invoked.
This can be achieved by starting the include file
(before |\ProvidesPackage|) with:
%
\begin{center}
\begin{tabular}{l}
|\input{childdoc.def}|\\
|\childdocforward{|\textit{main}|}|\\
\end{tabular}
\end{center}
%
or alternatively with:
%
\begin{center}
\begin{tabular}{l}
|\input{childdoc.def}|\\
|\childdocby{|\textit{main}|}|\\
\end{tabular}
\end{center}
%
Both forms have slightly different effects as described above.
The main file is prepared as usual, see \secref{sec:include}.

%%%%%%%%%%%%%%%%%%%%%%%%%%%%%%%%%%%%%%%%%%%%%%%%%%%%%%%%%%%%%%%%%%%%%%%%%%%%%%%%
\subsection{Legacy Detection}
\label{sec:detection}

The directive |\childdocmain| in the main file can detect
whether the complete document or merely a child is to be compiled
even without using the directive |\childdocof|.
This method is deprecated because it is less robust
and there is no compelling reason to use it;
it is merely provided for backward compatibility
and it may be removed in future versions.

If the detection mechanism is to be used,
it is mandatory to correctly specify
the filename of the main file as the argument of |\childdocmain|:
%
\begin{center}
\begin{tabular}{l}
|\input{childdoc.def}|\\
|\childdocmain{|\textit{main}|}|\\
\end{tabular}
\end{center}
%
If |\jobname| does not match the argument \textit{main} of |\childdocmain|,
it is assumed that |\jobname| points to the child file to be compiled.
When using |\childdocmain| with the main file specified as argument,
it suffices to start a child file
with just |\input{|\textit{main}|}|
without loading of the package and using |\childdocof|.
If instead all processing is done
with the appropriate \textsf{childdoc} directives,
the argument of \textit{main} of |\childdocmain| can be empty.

An alternative version of the command line processing described
in \secref{sec:commandline} using the detection mechanism reads:
%
\begin{center}
|... -jobname "|\textit{target}|" "|[\textit{flags}]%
[|\def\jobname{|\textit{dest}|}|]|\input{|\textit{main}|}"|
\end{center}

%%%%%%%%%%%%%%%%%%%%%%%%%%%%%%%%%%%%%%%%%%%%%%%%%%%%%%%%%%%%%%%%%%%%%%%%%%%%%%%%
\subsection{Manual Code}
\label{sec:manual}

In case one cannot be certain whether the definitions file |childdoc.def|
is installed on the target \TeX{} distribution
and one prefers not to ship it,
it is conceivable to paste a few relevant commands into the sources.

To that end, drop all statements |\input{childdoc.def}|
and perform the replacements as outlined below.
Instead of |\childdocmain{|\textit{main}|}| add the following code
to the top of the main file:
%
\begin{center}
\begin{tabular}{l}
|\||ifdefined\childdocname\endinput\||fi\newif\ifchilddoc|\\
|\edef\childdocname{\scantokens\expandafter{\jobname\noexpand}}|\\
|\def\childdocmain{|\textit{main}|}\||ifx\childdocmain\childdocname\||else|\\
|\childdoctrue\includeonly{\childdocname}\let\jobname\childdocmain\||fi|\\
\end{tabular}
\end{center}
%
Instead of |\childdocof{|\textit{main}|}| just include the main file
at the top of each child file:
%
\begin{center}
|\input{|\textit{main}|}|
\end{center}
%
A simple redirection |\childdocforward{|\textit{dest}|}| is achieved by:
%
\begin{center}
|\def\jobname{|\textit{dest}|}\input{\jobname}|
\end{center}
%
The redirection with prefix
|\childdocforwardprefix[|\textit{prefix}|]{|\textit{dest}|}|
is accomplished by:
%
\begin{center}
\begin{tabular}{l}
|{\edef\jobname{\scantokens\expandafter{\jobname\noexpand}}|\\
|\def\redirectjob |\textit{prefix}|#1~~~{\gdef\jobname{|\textit{dest}|#1}}|\\
|\expandafter\redirectjob\jobname~~~}\input{\jobname}|
\end{tabular}
\end{center}

In an alternative approach,
child documents can be compiled by a specific command line
without additional code or specific definitions:
%
\begin{center}
|... -jobname "|\textit{target}|" "|[\textit{flags}]%
|\includeonly{|\textit{dest}|}\input{|\textit{main}|}"|
\end{center}
%

%%%%%%%%%%%%%%%%%%%%%%%%%%%%%%%%%%%%%%%%%%%%%%%%%%%%%%%%%%%%%%%%%%%%%%%%%%%%%%%%
%%%%%%%%%%%%%%%%%%%%%%%%%%%%%%%%%%%%%%%%%%%%%%%%%%%%%%%%%%%%%%%%%%%%%%%%%%%%%%%%
\section{Information}

%%%%%%%%%%%%%%%%%%%%%%%%%%%%%%%%%%%%%%%%%%%%%%%%%%%%%%%%%%%%%%%%%%%%%%%%%%%%%%%%
\subsection{Copyright}

Copyright \copyright{} 2017--2018 Niklas Beisert

This work may be distributed and/or modified under the
conditions of the \LaTeX{} Project Public License, either version 1.3
of this license or (at your option) any later version.
The latest version of this license is in
  \url{http://www.latex-project.org/lppl.txt}
and version 1.3 or later is part of all distributions of \LaTeX{}
version 2005/12/01 or later.

This work has the LPPL maintenance status `maintained'.

The Current Maintainer of this work is Niklas Beisert.

This work consists of the files |README.txt|, |childdoc.ins| and |childdoc.dtx|
as well as the derived files |childdoc.def|, |cdocsamp.tex|
with |cdocsch1.tex|, |cdocsch2.tex|, |cdocspt3.tex|, |cdocspt4.tex|,
|cdocsdrf.tex|, |cdocsfn1.tex|, |cdocsfn2.tex|
as well as |childdoc.pdf|.

%%%%%%%%%%%%%%%%%%%%%%%%%%%%%%%%%%%%%%%%%%%%%%%%%%%%%%%%%%%%%%%%%%%%%%%%%%%%%%%%
\subsection{Files and Installation}

The package consists of the files:
%
\begin{center}
\begin{tabular}{ll}
    |README.txt|   & readme file \\
    |childdoc.ins| & installation file \\
    |childdoc.dtx| & source file \\
    |childdoc.def| & definition file \\
    |cdocsamp.tex| & sample main file \\
    |cdocsch1.tex| & sample include file \\
    |cdocsch2.tex| & sample include file \\
    |cdocspt3.tex| & sample part file \\
    |cdocspt4.tex| & sample part file \\
    |cdocsdrf.tex| & sample redirection file \\
    |cdocsfn1.tex| & sample redirection file \\
    |cdocsfn2.tex| & sample redirection file \\
    |childdoc.pdf| & manual
\end{tabular}
\end{center}
%
The distribution consists of the files
|README.txt|, |childdoc.ins| and |childdoc.dtx|.
%
\begin{itemize}
\item
Run (pdf)\LaTeX{} on |childdoc.dtx|
to compile the manual |childdoc.pdf| (this file).
\item
Run \LaTeX{} on |childdoc.ins| to create the definitions file |childdoc.def|
and the sample |cdocsamp.tex| with include files
|cdocsch1.tex|, |cdocsch2.tex|, |cdocspt3.tex|, |cdocspt4.tex|,
|cdocsdrf.tex|, |cdocsfn1.tex|, |cdocsfn2.tex|.
Then copy the file |childdoc.def| to an appropriate directory of your \LaTeX{}
distribution, e.g.\ \textit{texmf-root}|/tex/latex/childdoc|.
\end{itemize}

%%%%%%%%%%%%%%%%%%%%%%%%%%%%%%%%%%%%%%%%%%%%%%%%%%%%%%%%%%%%%%%%%%%%%%%%%%%%%%%%
\subsection{Related CTAN Packages}

There are several other packages which offer a similar functionality:
%
\begin{itemize}
\item
The packages
\href{http://ctan.org/pkg/docmute}{\textsf{docmute}},
\href{http://ctan.org/pkg/includex}{\textsf{includex}} and
\href{http://ctan.org/pkg/standalone}{\textsf{standalone}}
provide commands to include only the document body of
a child file thus allowing both files to be compiled individually.
\item
The packages \href{http://ctan.org/pkg/subdocs}{\textsf{subdocs}}
and \href{http://ctan.org/pkg/subfiles}{\textsf{subfiles}}
provide structures in which the main and child documents can be
encapsulated and allowing them to be compiled individually.
The inclusion mechanism is different from the conventional |\include|.
\item
The package \href{http://ctan.org/pkg/combine}{\textsf{combine}}
is an elaborate solution to combine several documents into one.
\end{itemize}
%
See also the CTAN topic \href{http://ctan.org/topic/subdocs}{\textsf{subdocs}}
for further related packages.
The present package differs from the above solutions in that
a document structure constructed with the conventional |\include| mechanism
just needs two extra commands at the top of every file
such that all constituent files can be compiled individually.

%%%%%%%%%%%%%%%%%%%%%%%%%%%%%%%%%%%%%%%%%%%%%%%%%%%%%%%%%%%%%%%%%%%%%%%%%%%%%%%%
%\subsection{Feature Suggestions}
%
%The following is a list of features which may be useful for future
%versions of this package:
%%
%\begin{itemize}
%\item
%\ldots
%\end{itemize}

%%%%%%%%%%%%%%%%%%%%%%%%%%%%%%%%%%%%%%%%%%%%%%%%%%%%%%%%%%%%%%%%%%%%%%%%%%%%%%%%
\subsection{Revision History}

%%%%%%%%%%%%%%%%%%%%%%%%%%%%%%%%%%%%%%%%
\paragraph{v2.0:} 2018/12/30

\begin{itemize}
\item
immediate forward processing
\item
added |\childdocby| mechanism
\item
manual restructured
\end{itemize}

%%%%%%%%%%%%%%%%%%%%%%%%%%%%%%%%%%%%%%%%
\paragraph{v1.6:} 2018/01/17

\begin{itemize}
\item
application for development of include files
\item
corrections to manual
\end{itemize}

%%%%%%%%%%%%%%%%%%%%%%%%%%%%%%%%%%%%%%%%
\paragraph{v1.5:} 2017/05/21

\begin{itemize}
\item
more complete structuring introduced
\item
|\childdocof| introduced
\item
|\childdoc| renamed to |\childdocmain|
\item
|\childredirect| renamed to |\childdocforward| and |\childdocforwardprefix|
and functionality expanded
\end{itemize}

%%%%%%%%%%%%%%%%%%%%%%%%%%%%%%%%%%%%%%%%
\paragraph{v1.0:} 2017/04/27

\begin{itemize}
\item
manual and install package
\item
first version published on CTAN
\end{itemize}

%%%%%%%%%%%%%%%%%%%%%%%%%%%%%%%%%%%%%%%%
\paragraph{v0.6:} 2017/04/26

\begin{itemize}
\item
redirection mechanism added
\end{itemize}

%%%%%%%%%%%%%%%%%%%%%%%%%%%%%%%%%%%%%%%%
\paragraph{v0.5:} 2017/04/26

\begin{itemize}
\item
functionality in definition file
\end{itemize}


%%%%%%%%%%%%%%%%%%%%%%%%%%%%%%%%%%%%%%%%%%%%%%%%%%%%%%%%%%%%%%%%%%%%%%%%%%%%%%%%
%%%%%%%%%%%%%%%%%%%%%%%%%%%%%%%%%%%%%%%%%%%%%%%%%%%%%%%%%%%%%%%%%%%%%%%%%%%%%%%%
%%%%%%%%%%%%%%%%%%%%%%%%%%%%%%%%%%%%%%%%%%%%%%%%%%%%%%%%%%%%%%%%%%%%%%%%%%%%%%%%
\appendix

\settowidth\MacroIndent{\rmfamily\scriptsize 000\ }

 \DocInput{childdoc.dtx}

\end{document}
%</driver>
% \fi
%
% %%%%%%%%%%%%%%%%%%%%%%%%%%%%%%%%%%%%%%%%%%%%%%%%%%%%%%%%%%%%%%%%%%%%%%%%%%%%%%
% %%%%%%%%%%%%%%%%%%%%%%%%%%%%%%%%%%%%%%%%%%%%%%%%%%%%%%%%%%%%%%%%%%%%%%%%%%%%%%
% \section{Sample}
%\iffalse
%<*samplemain>
%\fi
%
% The following presents a sample document
% with two chapters, two parts, a title page,
% a compile flag as well as three forwarding files to set the flag.
% It consists of eight |.tex| files:
% \begin{center}
% \begin{tabular}{ll}
% |cdocsamp.tex|&main file\\
% |cdocsch1.tex|&include file for chapter 1\\
% |cdocsch2.tex|&include file for chapter 2\\
% |cdocspt3.tex|&include file for part 3\\
% |cdocspt4.tex|&include file for part 4\\
% |cdocsdrf.tex|&forwarding file for main file in draft mode\\
% |cdocsfi1.tex|&forwarding file for final version of chapter 1\\
% |cdocsfi2.tex|&forwarding file for final version of chapter 2\\
% \end{tabular}
% \end{center}
% Each of the eight files can be compiled directly by the \LaTeX{} compiler.
%
% %%%%%%%%%%%%%%%%%%%%%%%%%%%%%%%%%%%%%%
% \paragraph{Main File.}
%
% The main file is called |cdocsamp.tex|.
%
% Load the \textsf{childdoc} definitions and
% declare the filename for the main document:
%    \begin{macrocode}
\input{childdoc.def}
\childdocmain{}
%    \end{macrocode}

% Optional override for |\version| flag:
%    \begin{macrocode}
%%\ifchilddoc\else\providecommand{\version}{draft}\fi
%    \end{macrocode}

% Define the default values for the |\version| flag
% (|final| for the main file and |draft| for childs):
%    \begin{macrocode}
\ifchilddoc
\providecommand{\version}{draft}
\else
\providecommand{\version}{final}
\fi
%    \end{macrocode}

% Load the standard document class:
%    \begin{macrocode}
\documentclass[12pt]{article}
%    \end{macrocode}

% Start the document body:
%    \begin{macrocode}
\begin{document}
%    \end{macrocode}

% Declare a title page.
% Print title, part of document being processed and version flag:
%    \begin{macrocode}
\addtocounter{page}{-1}
\begin{center}
{\LARGE\bfseries{}childdoc example\par}
\vspace{1cm}
\ifchilddoc
\ifchilddocmanual part\else chapter\fi:
`\childdocname' of `\childdocjob'\par
\else
main document: `\childdocjob'\par
\fi
version: \version\par
\end{center}
\newpage
%    \end{macrocode}

% Manually include selected file,
% otherwise process as usual:
%    \begin{macrocode}
\ifchilddocmanual
\section*{part `\childdocname'}
\input{\childdocname}
\else
%    \end{macrocode}

% Include the two chapters:
%    \begin{macrocode}
\include{cdocsch1}
\include{cdocsch2}
%    \end{macrocode}

% Include the two parts unless only chapters should be displayed:
%    \begin{macrocode}
\ifchilddoc\else
\section{part three}
\input{cdocspt3}
\section{part four}
\input{cdocspt4}
\fi
%    \end{macrocode}

% Process as usual until here:
%    \begin{macrocode}
\fi
%    \end{macrocode}

% End of document body:
%    \begin{macrocode}
\end{document}
%    \end{macrocode}
%\iffalse
%</samplemain>
%\fi
%
% %%%%%%%%%%%%%%%%%%%%%%%%%%%%%%%%%%%%%%
% \paragraph{Chapter Include Files.}
%
% The include files are called |cdocsch1.tex| and |cdocsch2.tex|.
%
%\iffalse
%<*samplechap1|samplechap2>
%\fi

% Optional override for |\version| flag:
%    \begin{macrocode}
%%\providecommand{\version}{final}
%    \end{macrocode}

% Include the main document:
%    \begin{macrocode}
\input{childdoc.def}
\childdocof{cdocsamp}
%    \end{macrocode}

%\iffalse
%</samplechap1|samplechap2>
%\fi
%
%\iffalse
%<*samplechap1>
%\fi
% Some text for chapter 1:
%    \begin{macrocode}
\section{one}
some text in chapter one
%    \end{macrocode}

%\iffalse
%</samplechap1>
%\fi
% Some text for chapter 2:
%\iffalse
%<*samplechap2>
%\fi
%    \begin{macrocode}
\section{two}
more text in chapter two
%    \end{macrocode}

%\iffalse
%</samplechap2>
%\fi
%
% %%%%%%%%%%%%%%%%%%%%%%%%%%%%%%%%%%%%%%
% \paragraph{Part Include Files.}
%
% The include files are called |cdocspt3.tex| and |cdocspt4.tex|.
%
%\iffalse
%<*samplepart3|samplepart4>
%\fi

% Optional override for |\version| flag:
%    \begin{macrocode}
%%\providecommand{\version}{final}
%    \end{macrocode}

% Include the main document:
%    \begin{macrocode}
\input{childdoc.def}
\childdocby{cdocsamp}
%    \end{macrocode}

%\iffalse
%</samplepart3|samplepart4>
%\fi
%
%\iffalse
%<*samplepart3>
%\fi
% Some text for part 3:
%    \begin{macrocode}
some text in part three
%    \end{macrocode}

%\iffalse
%</samplepart3>
%\fi
% Some text for part 4:
%\iffalse
%<*samplepart4>
%\fi
%    \begin{macrocode}
more text in part four
%    \end{macrocode}

%\iffalse
%</samplepart4>
%\fi
%
% %%%%%%%%%%%%%%%%%%%%%%%%%%%%%%%%%%%%%%
% \paragraph{Forwarding for a Complete Draft.}
%
% The following forwarding file |cdocsdrf.tex|
% compiles the main document in draft mode:
%\iffalse
%<*sampledraft>
%\fi
%    \begin{macrocode}
\def\version{draft}
\input{childdoc.def}
\childdocforward{cdocsamp}
%    \end{macrocode}

%\iffalse
%</sampledraft>
%\fi
%
% %%%%%%%%%%%%%%%%%%%%%%%%%%%%%%%%%%%%%%
% \paragraph{Forwarding for Final Version of the Chapters.}
%
% The following forwarding files |cdocsfn1.tex| and |cdocsfn2.tex|
% (with identical content)
% compile the final versions of the child documents
% |cdocsch1.tex| and |cdocsch2.tex|, respectively:
%\iffalse
%<*samplefinal>
%\fi
%    \begin{macrocode}
\def\version{final}
\input{childdoc.def}
\childdocforwardprefix[cdocsamp]{cdocsfn}{cdocsch}
%    \end{macrocode}

%\iffalse
%</samplefinal>
%\fi
%
% %%%%%%%%%%%%%%%%%%%%%%%%%%%%%%%%%%%%%%
% \paragraph{Command Line Processing.}
%
% The following three command lines generate the output files
% |cdocscld|, |cdocscl1| and |cdocscl2|
% which should be identical to
% |cdocsdrf|, |cdocsch1| and |cdocsfn2|, respectively:
% \begin{center}
% \begin{tabular}{l}
% |latex -jobname cdocscld \|\\
% |  "\def\version{draft}\input{childdoc.def}\childdocforward{cdocsamp}"|\\
% |latex -jobname cdocscl1 \|\\
% |  "\input{childdoc.def}\childdocforward[cdocsamp]{cdocsch1}"|\\
% |latex -jobname cdocscl2 \|\\
% |  "\def\version{final}\input{childdoc.def}\childdocforward{cdocsch2}"|
% \end{tabular}
% \end{center}
% Note that the trailing backslash on each first line
% merely continues the input to the second line
% (for convenient cut ant paste).
% Furthermore, the command |latex| can be replaced by any
% of its alternative versions such as |pdflatex|.
%
% %%%%%%%%%%%%%%%%%%%%%%%%%%%%%%%%%%%%%%%%%%%%%%%%%%%%%%%%%%%%%%%%%%%%%%%%%%%%%%
% %%%%%%%%%%%%%%%%%%%%%%%%%%%%%%%%%%%%%%%%%%%%%%%%%%%%%%%%%%%%%%%%%%%%%%%%%%%%%%
% \section{Implementation}
%\iffalse
%<*package>
%\fi
%
% This section describes the definitions file |childdoc.def|.

% The definitions cannot be loaded using |\usepackage| or |\RequirePackage|
% which has a mechanism to prevent loading a style file more than once.
% When loading the definitions by means of |\input|
% multiple instances have to be prevented manually:
%\iffalse
%This code needs to be before the `\ProvidesFile' directive
%which is defined at the beginning of this file.
%Therefore it is also placed there and commented out here.
%</package>
%<*discard>
%\fi
%    \begin{macrocode}
\ifdefined\childdocmain\endinput\fi
%    \end{macrocode}
%\iffalse
%</discard>
%<*package>
%\fi
%
% \macro{\ifchilddoc}
% \macro{\ifchilddocmanual}
% The conditional |\ifchilddoc| tells whether a
% child (true) or main (false) document is being compiled.
% The conditional |\ifchilddocmanual| tells whether
% the |\includeonly| mechanism is used (false) or
% the selection of child files must be performed manually (true).
% The definitions initialise to false:
%    \begin{macrocode}
\newif\ifchilddoc
\newif\ifchilddocmanual
%    \end{macrocode}

% \macro{\childdocname}
% \macro{\childdocjob}
% The macro |\childdocname| stores the name of the main document
% to be compiled. The macro |\childdocjob| stores the name of
% the document on which the \LaTeX{} compiler was originally invoked.
% The content of |\jobname| cannot be compared
% to filenames specified in the source due to different catcodes.
% The following code rescans |\jobname|, stores the result
% in |\childdocname| and saves a copy in |\childdocjob|:
%    \begin{macrocode}
\edef\childdocname{\scantokens\expandafter{\jobname\noexpand}}
\let\childdocjob\childdocname
%    \end{macrocode}

% \macro{\childdocdisable}
% The macro |\childdocdisable| prevents the main file
% from being processed more than once.
% At this stage, the main document command |\childdocmain|
% is assumed to be called once again where it should do nothing.
% Any subsequent call to it should prevent
% a secondary processing of the main document
% It overwrites the forwarding commands
% |\childdocof| and |\childdocforward|
% with empty macros to prevent further inclusions of the main document:
%    \begin{macrocode}
\newcommand{\childdocdisable}
{
  \renewcommand{\childdocmain}[1]{\renewcommand{\childdocmain}[1]{\endinput}}
  \renewcommand{\childdocof}[1]{}
  \renewcommand{\childdocby}[2][]{}
  \renewcommand{\childdocforward}[2][]{}
  \renewcommand{\childdocdisable}{}
}
%    \end{macrocode}

% \macro{\childdocmain}
% The macro |\childdocmain| is to be called at the top of the main file
% with nothing or the main filename (without extension) as argument.
% First, it breaks loops.
% If the argument is not empty and does not match |\childdocname|
% (which is set by the first inclusion of |childdoc.def|),
% |\ifchilddoc| is set to true, |\includeonly| is applied to the child file
% and |\jobname| is set to the main file
% (for proper handling of |.aux| files):
%    \begin{macrocode}
\newcommand{\childdocmain}[1]
{
  \childdocdisable\childdocmain{}
  \if?#1?\else
    \begingroup
      \def\childdoctmp{#1}
      \ifx\childdoctmp\childdocname
        \def\childdoctmp{}
      \else
        \def\childdoctmp
        {
          \childdoctrue
          \includeonly{\childdocname}
          \def\childdocjob{#1}
          \def\jobname{#1}
        }
      \fi
      \expandafter
    \endgroup
    \childdoctmp
  \fi
}
%    \end{macrocode}

% \macro{\childdocof}
% The command |\childdocof| redirects
% compilation to the main file |#1|.
%    \begin{macrocode}
\newcommand{\childdocof}[1]
{
  \childdocdisable
  \childdoctrue
  \includeonly{\childdocname}
  \def\jobname{#1}
  \def\childdocjob{#1}
  \input{#1}
}
%    \end{macrocode}

% \macro{\childdocby}
% The command |\childdocby| ....
%    \begin{macrocode}
\newcommand{\childdocby}[2][]
{
  \childdocdisable
  \childdoctrue
  \childdocmanualtrue
  \if?#1?\else
    \def\jobname{#2}
  \fi
  \def\childdocjob{#2}
  \input{#2}
  \endinput
}
%    \end{macrocode}

% \macro{\childdocforward}
% The command |\childdocforward| redirects
% compilation to the main file or
% (if the optional argument is given) a child file.
% Parameters are set as if the main file
% or a child file starting with |\childdocof| was compiled.
% Then compilation is handed over to the main file:
%    \begin{macrocode}
\newcommand{\childdocforward}[2][]
{
  \begingroup
    \if?#1?
      \def\childdoctmp
      {
        \def\childdocname{#2}
        \def\childdocjob{#2}
        \def\jobname{#2}
        \input{#2}
        \endinput
      }
    \else
      \def\childdoctmp
      {
        \childdocdisable
        \def\childdocname{#2}
        \childdoctrue
        \includeonly{#2}
        \def\childdocjob{#1}
        \def\jobname{#1}
        \input{#1}
        \endinput
      }
    \fi
    \expandafter
  \endgroup
  \childdoctmp
}
%    \end{macrocode}

% \macro{\childdocforwardprefix}
% The command |\childdocforwardprefix| redirects
% compilation to the main or a child file by means of a pattern.
% The prefix |#1| in the current filename is replaced by |#2|
% and the suffix of the current filename is kept
% (it is assumed that the filename does not contain the substring `|~~~|'
% which is used as a delimiter).
% Compilation is handed over to the new file by |\childdocforward|:
%    \begin{macrocode}
\newcommand{\childdocforwardprefix}[3][]
{
  \begingroup
    \def\childdocextract #2##1~~~{\def\childdoctmp{\childdocforward[#1]{#3##1}}}
    \expandafter\childdocextract\childdocname~~~
    \expandafter
  \endgroup
  \childdoctmp
}
%    \end{macrocode}

% \macro{\childdoc}
% The deprecated macro |\childdoc| is a legacy version of |\childdocmain|:
%    \begin{macrocode}
\newcommand{\childdoc}{\childdocmain}
%    \end{macrocode}

% \macro{\childdocredirect}
% The deprecated macro |\childdocredirect| is a legacy version
% of |\childdocforward| and |\childdocforwardprefix|:
%    \begin{macrocode}
\newcommand{\childdocredirect}[2][]
{
  \begingroup
    \if?#1?
      \def\childdoctmp{\childdocforward{#2}}
    \else
      \def\childdoctmp{\childdocforwardprefix{#1}{#2}}
    \fi
    \expandafter
  \endgroup
  \childdoctmp
}
%    \end{macrocode}

%\iffalse
%</package>
%\fi
%
\endinput
|\\
|\childdocforward{|\textit{main}|}|\\
\end{tabular}
\end{center}
%
or alternatively with:
%
\begin{center}
\begin{tabular}{l}
|% \iffalse
%
% childdoc.dtx Copyright (C) 2017-2018 Niklas Beisert
%
% This work may be distributed and/or modified under the
% conditions of the LaTeX Project Public License, either version 1.3
% of this license or (at your option) any later version.
% The latest version of this license is in
%   http://www.latex-project.org/lppl.txt
% and version 1.3 or later is part of all distributions of LaTeX
% version 2005/12/01 or later.
%
% This work has the LPPL maintenance status `maintained'.
%
% The Current Maintainer of this work is Niklas Beisert.
%
% This work consists of the files childdoc.dtx and childdoc.ins
% and the derived files childdoc.def and cdocsamp.tex with
% cdocsch1.tex, cdocsch2.tex, cdocsdrf.tex, cdocsfn1.tex, cdocsfn2.tex.
%
%<package>\ifdefined\childdocmain\endinput\fi
%<package>\ProvidesFile{childdoc.def}[2018/12/30 v2.0 child document driver]
%<samplemain>\ProvidesFile{cdocsamp.tex}[2018/12/30 v2.0 sample for childdoc]
%<*driver>
%\ProvidesFile{childdoc.drv}[2018/12/30 v2.0 childdoc reference manual file]
\PassOptionsToClass{10pt,a4paper}{article}
\documentclass{ltxdoc}

\usepackage[margin=35mm]{geometry}
\usepackage{hyperref}
\usepackage{hyperxmp}
\usepackage[usenames]{color}

\hypersetup{colorlinks=true}
\hypersetup{pdfstartview=FitH}
\hypersetup{pdfpagemode=UseNone}
\hypersetup{pdfsource={}}
\hypersetup{pdflang={en-UK}}
\hypersetup{pdfcopyright={Copyright 2017-2018 Niklas Beisert.
  This work may be distributed and/or modified under the
  conditions of the LaTeX Project Public License, either version 1.3
  of this license or (at your option) any later version.}}
\hypersetup{pdflicenseurl={http://www.latex-project.org/lppl.txt}}
\hypersetup{pdfcontactaddress={ETH Zurich, ITP, HIT K,
  Wolfgang-Pauli-Strasse 27}}
\hypersetup{pdfcontactpostcode={8093}}
\hypersetup{pdfcontactcity={Zurich}}
\hypersetup{pdfcontactcountry={Switzerland}}
\hypersetup{pdfcontactemail={nbeisert@itp.phys.ethz.ch}}
\hypersetup{pdfcontacturl={http://people.phys.ethz.ch/\xmptilde nbeisert/}}

\newcommand{\secref}[1]{\hyperref[#1]{section \ref*{#1}}}

\parskip1ex
\parindent0pt
\let\olditemize\itemize
\def\itemize{\olditemize\parskip0pt}

\begin{document}

\title{The \textsf{childdoc} Package}
\hypersetup{pdftitle={The childdoc Package}}
\author{Niklas Beisert\\[2ex]
  Institut f\"ur Theoretische Physik\\
  Eidgen\"ossische Technische Hochschule Z\"urich\\
  Wolfgang-Pauli-Strasse 27, 8093 Z\"urich, Switzerland\\[1ex]
  \href{mailto:nbeisert@itp.phys.ethz.ch}
  {\texttt{nbeisert@itp.phys.ethz.ch}}}
\hypersetup{pdfauthor={Niklas Beisert}}
\hypersetup{pdfsubject={Manual for the LaTeX2e Package childdoc}}
\date{30 December 2018, \textsf{v2.0}}
\maketitle

\begin{abstract}\noindent
\textsf{childdoc} is a \LaTeXe{} package
that enables the direct compilation
of document sections included by |\include|
to individual files.
\end{abstract}

\begingroup
\parskip0ex
\tableofcontents
\endgroup

%%%%%%%%%%%%%%%%%%%%%%%%%%%%%%%%%%%%%%%%%%%%%%%%%%%%%%%%%%%%%%%%%%%%%%%%%%%%%%%%
%%%%%%%%%%%%%%%%%%%%%%%%%%%%%%%%%%%%%%%%%%%%%%%%%%%%%%%%%%%%%%%%%%%%%%%%%%%%%%%%
\section{Introduction}

\LaTeX{} provides a mechanism to structure a large document (such as a book)
into a main file and several child files (containing the chapters)
using the |\include| command.
This mechanism is beneficial for documents
which span hundreds of pages in order to
make the source file(s) more manageable.
Moreover, compilation can be restricted to
selected child files by means of the |\includeonly| command.
The latter feature can be used to reduce the compilation time while editing
(this was significantly more useful in the earlier days of \LaTeX{})
or to generate a smaller document which is easier to navigate.
Another application of |\includeonly| is to generate
documents consisting of selected parts of the complete document.

However, there are a few drawbacks of the plain |\include| mechanism:
\begin{itemize}
\item
The child files cannot be compiled on their own,
they can only be compiled via the main file.
A naive editing environment
(such as a text editor with an option
to have the current file processed by \LaTeX)
may require one to switch to the main file before compiling;
attempting to compile the child file produces errors.
\item
The main file must be modified (each time)
to adjust the |\includeonly| command
to the present needs. This easily leaves the main file in a messy state.
\item
The generated document will always carry the filename
of the main document. This is inconvenient if
several child files are to be compiled and
to be kept for distribution.
\end{itemize}

The present package provides a simple interface
to make child files individually compilable by \LaTeX{}.
Compiling a child file then has the same effect as compiling
the main file with an |\includeonly| command
to select the appropriate child.
Moreover the generated document will carry the name of the child
rather than the main file.
This resolves all three above issues.

This feature is meant to make the editing of books,
thesis documents and lecture notes somewhat more convenient.
However, the package can also be used efficiently for
composing a series of documents (such as exercise sheets)
which are typically distributed individually.
It then assists the author in generating the individual documents
(potentially in different versions)
as well as a document containing the collected series.
Another application is in developing style files
or other kinds of included material
where compilation of the style file could redirect
to a sample or test file.

%%%%%%%%%%%%%%%%%%%%%%%%%%%%%%%%%%%%%%%%%%%%%%%%%%%%%%%%%%%%%%%%%%%%%%%%%%%%%%%%
%%%%%%%%%%%%%%%%%%%%%%%%%%%%%%%%%%%%%%%%%%%%%%%%%%%%%%%%%%%%%%%%%%%%%%%%%%%%%%%%
\section{Usage}

First of all, the package \textsf{childdoc} is \emph{not} a standard
\LaTeXe{} |.sty| style file! Therefore it needs to be invoked in
a non-standard way.

%%%%%%%%%%%%%%%%%%%%%%%%%%%%%%%%%%%%%%%%%%%%%%%%%%%%%%%%%%%%%%%%%%%%%%%%%%%%%%%%
\subsection{Included Files}
\label{sec:include}

%%%%%%%%%%%%%%%%%%%%%%%%%%%%%%%%%%%%%%%%
\DescribeMacro{\childdocmain}
To use the package, add the commands
\begin{center}
\begin{tabular}{l}
|\input{childdoc.def}|\\
|\childdocmain{}|\\
\end{tabular}
\end{center}
at the very top of the main \LaTeX{} file,
in particular \emph{before} the |\documentclass| statement!
The argument of |\childdocmain| should be left empty
(but it must be present).

%%%%%%%%%%%%%%%%%%%%%%%%%%%%%%%%%%%%%%%%
\DescribeMacro{\childdocof}
Furthermore, add the commands
\begin{center}
\begin{tabular}{l}
|\input{childdoc.def}|\\
|\childdocof{|\textit{main}|}|\\
\end{tabular}
\end{center}
at the top of every child file \textit{child}
which is included by |\include{|\textit{child}|}|
from within the main file
(or at least for those files to be compiled individually).
The argument \textit{main} must be the filename of the main file.

There are a couple of
considerations in setting up the main and child documents:

%%%%%%%%%%%%%%%%%%%%%%%%%%%%%%%%%%%%%%%%
\paragraph{Restrictions.}

Please note the following restrictions:
\begin{itemize}
\item
|\childdocmain| must be called with one argument \textit{main}
to ensure compatibility with earlier version of the package.
It must either be empty (|\childdocmain{}|)
or precisely match the filename of the main file in which it is specified.
See \secref{sec:detection} for further information.
\item
The filename \textit{main} must be specified without the |.tex| extension.
\item
The filename \textit{main} is case sensitive
(even in case-insensitive file systems)
due to internal string comparison.
\item
The argument \textit{main} should be fully expanded, it cannot be a macro.
\item
Subdirectories and special characters should be avoided in filenames.
\item
The command |\childdocmain{|\textit{main}|}| must be followed by a whitespace.
It should not be followed immediately by another command
or by a comment mark `|%|'.
This is because the \TeX{} parser reads the token immediately following
the argument of |\childdocmain| and puts it
at the beginning of every child section;
however, a white\-space is ignored.
\end{itemize}

%%%%%%%%%%%%%%%%%%%%%%%%%%%%%%%%%%%%%%%%
\paragraph{Content of Main File.}

It is advisable to place all content in the child files included by |\include|.
Any output contained in the main file will appear in all child documents
unless suppressed manually;
it cannot be suppressed automatically by the |\includeonly| directive
and thus should normally be avoided.
A method to include some content in the main file
by means of conditional processing is described in \secref{sec:conditional}.

%%%%%%%%%%%%%%%%%%%%%%%%%%%%%%%%%%%%%%%%
\paragraph{Page Numbering.}

When only a part of the document is compiled,
the appropriate numbering of pages
(as well as other status parameters)
is determined from the |.aux| files.
The latter contain information from previous passes.
However this information needs to propagate through
all intermediate child documents.
Therefore the page numbering in child documents may well
be inconsistent until the complete document is compiled at least once.

A useful (if unconventional) way to always ensure a consistent
page numbering is to restart the numbering in each child document
and denote the pages by `\textit{child}|.|\textit{page}'
where \textit{child} represents the chapter/section number of the child file.
This can be achieved by the command
|\numberwithin{page}{|\textit{child}|}|
of the \textsf{amsmath} package
where \textit{child} can be |chapter| or |section|
depending on the chosen structuring.
Alternatively, one can modify the macro |\thepage| appropriately
and reset the counter |page| at the start of each child file.

%%%%%%%%%%%%%%%%%%%%%%%%%%%%%%%%%%%%%%%%%%%%%%%%%%%%%%%%%%%%%%%%%%%%%%%%%%%%%%%%
\subsection{Conditional Processing}
\label{sec:conditional}

The package provides a mechanism to compile different versions
of a document. To customise the versions further some conditional processing
can come in handy to distinguish which version is being compiled.
The package provides two macros to describe the compilation context:

%%%%%%%%%%%%%%%%%%%%%%%%%%%%%%%%%%%%%%%%
\DescribeMacro{\ifchilddoc}
The conditional |\ifchilddoc| distinguishes between the compilation of
child documents and the main document:
%
\begin{center}
|\ifchilddoc |\textit{child-code}| |[|\||else |\textit{main-code}]| \||fi|
\end{center}

%%%%%%%%%%%%%%%%%%%%%%%%%%%%%%%%%%%%%%%%
\DescribeMacro{\childdocname}
\DescribeMacro{\childdocjob}
The macro |\childdocname| contains the filename (without extension)
of the main or child file being processed.
Note that |\childdocjob| will always contain the name of the main file.

%%%%%%%%%%%%%%%%%%%%%%%%%%%%%%%%%%%%%%%%
\paragraph{Title Page.}

Conditional processing can be used to include a title or banner page
in the main document when proper precautions are taken.
Importantly, the code in the main file should ensure that the page counter
(as well as other status parameters which are stored in the |.aux| files)
takes the same value after the conditional processing.
Otherwise the page numbers may take divergent values
depending on which part is compiled.

For example, a title page could be declared by:
%
\begin{center}
\begin{tabular}{l}
|\ifchilddoc\||else|\\
|\addtocounter{page}{-1}|\\
\textit{code for title page}\\
|\newpage|\\
|\||fi|
\end{tabular}
\end{center}
%
A banner page for the child documents can be generated by:
%
\begin{center}
\begin{tabular}{l}
|\ifchilddoc|\\
|\addtocounter{page}{-1}|\\
\textit{code for banner page}\\
|\newpage|\\
|\||fi|
\end{tabular}
\end{center}
%
Here one could write a message such as:
\begin{center}
|This is the part \childdocname{} of \childdocjob{}.|
\end{center}

%%%%%%%%%%%%%%%%%%%%%%%%%%%%%%%%%%%%%%%%%%%%%%%%%%%%%%%%%%%%%%%%%%%%%%%%%%%%%%%%
\subsection{Flags}
\label{sec:flags}

The package makes it easy to generate different versions
of the main or child documents.
To this end compilation flags can be defined
and assigned different default values.
They will be particularly useful in conjunction
with the forwarding mechanism described in \secref{sec:forward}.

For example, it may be useful to have a flag |\version|
which can be set to |draft| or |final|.
The document source will contain some conditional code
depending on the value of |\version|.
Suppose further, the flag should default to |final| for the main file
and to |draft| for child files
which is a natural assignment for editing the document.
This is achieved by placing the following code
in the preamble of the main document
(below the |\childdocmain| directive):
%
\begin{center}
\begin{tabular}{l}
|\ifchilddoc|\\
|\providecommand{\version}{draft}|\\
|\||else|\\
|\providecommand{\version}{final}|\\
|\||fi|
\end{tabular}
\end{center}
%
The definition by |\providecommand| makes sure
that previous definitions are not overwritten.
Further statements |\providecommand{\version}{...}|
can thus be added before the above code to override it.

For the main file, one might add a line
(between |\childdocmain| and the above block)
%
\begin{center}
|%\ifchilddoc\||else\providecommand{\version}{draft}\||fi|
\end{center}
%
which can be uncommented to produce a draft version.
Likewise one can add a line to the very top of a child file
(above the |\childdocof{|\textit{main}|}| directive)
%
\begin{center}
|%\providecommand{\version}{final}|
\end{center}
%
which can be uncommented to produce the final version of this child document.

%%%%%%%%%%%%%%%%%%%%%%%%%%%%%%%%%%%%%%%%%%%%%%%%%%%%%%%%%%%%%%%%%%%%%%%%%%%%%%%%
\subsection{Forwarding}
\label{sec:forward}

Different versions of the main or child documents
using compilation flags as described in \secref{sec:flags}
can be (permanently) stored in different files
for convenient compilation, viewing and distribution.
To this end, the package defines a command
to pass on compilation to a different file:

%%%%%%%%%%%%%%%%%%%%%%%%%%%%%%%%%%%%%%%%
\DescribeMacro{\childdocforward}
The command |\childdocforward| redirects processing to
another source file:
%
\begin{center}
\begin{tabular}{l}
|\input{childdoc.def}|\\
|\childdocforward[|\textit{main}|]{|\textit{dest}|}|\\
\end{tabular}
\end{center}
%
The argument \textit{dest} is the destination file
(without extension).
It should be the main file or one of the child files.
Note that further \textsf{childdoc} directives
such as |\childdocof| and |\childdocforward|
in the indicated file will be processed in this form.
The optional argument \textit{main}
passes on directly to the main file \textit{main}
while pretending to compile the child \textit{dest}.
This form behaves as if \textit{dest}
issues |\childdocof{|\textit{main}|}| right away,
and no further \textsf{childdoc} directives will be processed.

%%%%%%%%%%%%%%%%%%%%%%%%%%%%%%%%%%%%%%%%
\DescribeMacro{\...prefix}
In the alternative form |\childdocforwardprefix|,
%
\begin{center}
\begin{tabular}{l}
|\input{childdoc.def}|\\
|\childdocforwardprefix[|\textit{main}|]{|\textit{prefix}|}{|\textit{dest}|}|
\end{tabular}
\end{center}
%
the destination file is determined by a pattern
depending on the current file:
To make this work, the current file must be called
`{\textit{prefix}\hspace{0.2em}\textit{suffix}}'
with \textit{prefix} matching precisely the argument.
Processing is then passed on to the file
`{\textit{dest}\hspace{0.2em}\textit{suffix}}'.
Surely, the same effect is achieved by
directly specifying the
argument `{\textit{dest}\hspace{0.2em}\textit{suffix}}'
in the first form.
However, that requires to set up a different file
for each child. With the alternative form of the command
all these files can have exactly the same content
which simplifies setting them up and maintaining them.

For example, the following file |draft.tex|
with a compilation flag |\version| as described in \secref{sec:flags}
compiles the main document as a draft:
%
\begin{center}
\begin{tabular}{l}
|\def\version{draft}|\\
|\input{childdoc.def}|\\
|\childdocforward{|\textit{main}|}|
\end{tabular}
\end{center}
%
Likewise, the following files |final|\textit{nn}|.tex|
compile the final version of the child document
|child|\textit{nn}|.tex|:
%
\begin{center}
\begin{tabular}{l}
|\def\version{final}|\\
|\input{childdoc.def}|\\
|\childdocforwardprefix{final}{child}|
\end{tabular}
\end{center}
%

Note that when several versions of a main file and/or of each child file
are to be generated, it may be convenient to set up a |Makefile| or
shell script to automatise the process.

%%%%%%%%%%%%%%%%%%%%%%%%%%%%%%%%%%%%%%%%%%%%%%%%%%%%%%%%%%%%%%%%%%%%%%%%%%%%%%%%
\subsection{Command Line Processing}
\label{sec:commandline}

The effect of redirection files can also be achieved by invoking
the \LaTeX{} compiler with a more elaborate command line.
Most conveniently this should be done as part
of a shell script or a |Makefile|.

When using \textsf{childdoc} in the main file, the following
command lines effectively perform a redirection
(note that depending on the shell being used,
backslashes may have to be doubled: `|\|' $\to$ `|\\|'):
%
\begin{center}
|... -jobname "|\textit{target}|" |\\|"|[\textit{flags}]%
|\input{childdoc.def}\childdocforward[|\textit{main}|]{|\textit{dest}|}"|
\end{center}
%
Here \textit{target} is the name of the output file,
\textit{main} is the name of the main file
and \textit{dest} is the name of the main or child file to be processed
(all filenames without extensions).
The optional argument \textit{main} can be omitted
if \textit{main} matches \textit{dest}.
Optionally, compilation \textit{flags} can be defined via |\def| commands.
This command line makes the \TeX{} engine believe
it is compiling the file \textit{target}
whose content is specified as the latter parameter.
The provided code then forwards the processing to
\textit{main} or \textit{dest} as described in \secref{sec:forward}.

%%%%%%%%%%%%%%%%%%%%%%%%%%%%%%%%%%%%%%%%%%%%%%%%%%%%%%%%%%%%%%%%%%%%%%%%%%%%%%%%
\subsection{Include by Input}
\label{sec:input}

Including child documents by |\include| has some restrictions by design.
Most notably, the content of a child document always occupies
its own set of pages; pages cannot be shared between child documents.
Usually, this behaviour makes perfect sense
because each child document contain an essential part of the document.
However, in some situations it may be desirable to compose
a document from a collection of parts
without having mandatory page breaks between then.
For this case, the package
provides a mechanism to include parts
by |\input| which can also be processed individually.
However, by construction this mechanism
requires manual handling of the content to be output.

%%%%%%%%%%%%%%%%%%%%%%%%%%%%%%%%%%%%%%%%
\DescribeMacro{\ifchilddocmanual}
The main file should be prepared as usual, see \secref{sec:include}.
However, the document body must make a distinction
between processing of an individual part and of the main document, e.g.:
%
\begin{center}
\begin{tabular}{l}
|\ifchilddocmanual|\\
|\input{\childdocname}|\\
|\||else|\\
\textit{document body with }|\input{|\textit{part}|}|\\
|\||fi|
\end{tabular}
\end{center}
%
The conditional |\ifchilddocmanual| is true whenever
a part to be included by |\input| is being compiled,
and the name of the part is stored in |\childdocname|.

%%%%%%%%%%%%%%%%%%%%%%%%%%%%%%%%%%%%%%%%
\DescribeMacro{\childdocby}
Each part to be included by |\input| should start with:
%
\begin{center}
\begin{tabular}{l}
|\input{childdoc.def}|\\
|\childdocby{|\textit{main}|}|\\
\end{tabular}
\end{center}
%
The directive |\childdocby| is similar to |\childdocof|
described in \secref{sec:include},
but the subsequent selection of content must be done manually.
To that end, both |\ifchilddoc| and |\ifchilddocmanual|
will be true upon processing of a part,
and the name of the part is stored in |\childdocname|.
Note that |\jobname| will be set to the filename of the current part
so that each part receives an individual |.aux| file
that does not interfere with the |.aux| file(s) of the main document.
This behaviour can be altered by the alternative form
|\childdocby[*]{|\textit{main}|}| (with a non-empty optional argument)
which uses the |.aux| file of the main document
by setting |\jobname| to \textit{main}.

%%%%%%%%%%%%%%%%%%%%%%%%%%%%%%%%%%%%%%%%%%%%%%%%%%%%%%%%%%%%%%%%%%%%%%%%%%%%%%%%
\subsection{Driver Development}
\label{sec:driver}

The \textsf{childdoc} mechanism can also be use for the development
of definition files such as \LaTeX{} styles or classes.
This case differs from the above setup with multiple parts
included by |\include| in that no |\includeonly| should be invoked.
This can be achieved by starting the include file
(before |\ProvidesPackage|) with:
%
\begin{center}
\begin{tabular}{l}
|\input{childdoc.def}|\\
|\childdocforward{|\textit{main}|}|\\
\end{tabular}
\end{center}
%
or alternatively with:
%
\begin{center}
\begin{tabular}{l}
|\input{childdoc.def}|\\
|\childdocby{|\textit{main}|}|\\
\end{tabular}
\end{center}
%
Both forms have slightly different effects as described above.
The main file is prepared as usual, see \secref{sec:include}.

%%%%%%%%%%%%%%%%%%%%%%%%%%%%%%%%%%%%%%%%%%%%%%%%%%%%%%%%%%%%%%%%%%%%%%%%%%%%%%%%
\subsection{Legacy Detection}
\label{sec:detection}

The directive |\childdocmain| in the main file can detect
whether the complete document or merely a child is to be compiled
even without using the directive |\childdocof|.
This method is deprecated because it is less robust
and there is no compelling reason to use it;
it is merely provided for backward compatibility
and it may be removed in future versions.

If the detection mechanism is to be used,
it is mandatory to correctly specify
the filename of the main file as the argument of |\childdocmain|:
%
\begin{center}
\begin{tabular}{l}
|\input{childdoc.def}|\\
|\childdocmain{|\textit{main}|}|\\
\end{tabular}
\end{center}
%
If |\jobname| does not match the argument \textit{main} of |\childdocmain|,
it is assumed that |\jobname| points to the child file to be compiled.
When using |\childdocmain| with the main file specified as argument,
it suffices to start a child file
with just |\input{|\textit{main}|}|
without loading of the package and using |\childdocof|.
If instead all processing is done
with the appropriate \textsf{childdoc} directives,
the argument of \textit{main} of |\childdocmain| can be empty.

An alternative version of the command line processing described
in \secref{sec:commandline} using the detection mechanism reads:
%
\begin{center}
|... -jobname "|\textit{target}|" "|[\textit{flags}]%
[|\def\jobname{|\textit{dest}|}|]|\input{|\textit{main}|}"|
\end{center}

%%%%%%%%%%%%%%%%%%%%%%%%%%%%%%%%%%%%%%%%%%%%%%%%%%%%%%%%%%%%%%%%%%%%%%%%%%%%%%%%
\subsection{Manual Code}
\label{sec:manual}

In case one cannot be certain whether the definitions file |childdoc.def|
is installed on the target \TeX{} distribution
and one prefers not to ship it,
it is conceivable to paste a few relevant commands into the sources.

To that end, drop all statements |\input{childdoc.def}|
and perform the replacements as outlined below.
Instead of |\childdocmain{|\textit{main}|}| add the following code
to the top of the main file:
%
\begin{center}
\begin{tabular}{l}
|\||ifdefined\childdocname\endinput\||fi\newif\ifchilddoc|\\
|\edef\childdocname{\scantokens\expandafter{\jobname\noexpand}}|\\
|\def\childdocmain{|\textit{main}|}\||ifx\childdocmain\childdocname\||else|\\
|\childdoctrue\includeonly{\childdocname}\let\jobname\childdocmain\||fi|\\
\end{tabular}
\end{center}
%
Instead of |\childdocof{|\textit{main}|}| just include the main file
at the top of each child file:
%
\begin{center}
|\input{|\textit{main}|}|
\end{center}
%
A simple redirection |\childdocforward{|\textit{dest}|}| is achieved by:
%
\begin{center}
|\def\jobname{|\textit{dest}|}\input{\jobname}|
\end{center}
%
The redirection with prefix
|\childdocforwardprefix[|\textit{prefix}|]{|\textit{dest}|}|
is accomplished by:
%
\begin{center}
\begin{tabular}{l}
|{\edef\jobname{\scantokens\expandafter{\jobname\noexpand}}|\\
|\def\redirectjob |\textit{prefix}|#1~~~{\gdef\jobname{|\textit{dest}|#1}}|\\
|\expandafter\redirectjob\jobname~~~}\input{\jobname}|
\end{tabular}
\end{center}

In an alternative approach,
child documents can be compiled by a specific command line
without additional code or specific definitions:
%
\begin{center}
|... -jobname "|\textit{target}|" "|[\textit{flags}]%
|\includeonly{|\textit{dest}|}\input{|\textit{main}|}"|
\end{center}
%

%%%%%%%%%%%%%%%%%%%%%%%%%%%%%%%%%%%%%%%%%%%%%%%%%%%%%%%%%%%%%%%%%%%%%%%%%%%%%%%%
%%%%%%%%%%%%%%%%%%%%%%%%%%%%%%%%%%%%%%%%%%%%%%%%%%%%%%%%%%%%%%%%%%%%%%%%%%%%%%%%
\section{Information}

%%%%%%%%%%%%%%%%%%%%%%%%%%%%%%%%%%%%%%%%%%%%%%%%%%%%%%%%%%%%%%%%%%%%%%%%%%%%%%%%
\subsection{Copyright}

Copyright \copyright{} 2017--2018 Niklas Beisert

This work may be distributed and/or modified under the
conditions of the \LaTeX{} Project Public License, either version 1.3
of this license or (at your option) any later version.
The latest version of this license is in
  \url{http://www.latex-project.org/lppl.txt}
and version 1.3 or later is part of all distributions of \LaTeX{}
version 2005/12/01 or later.

This work has the LPPL maintenance status `maintained'.

The Current Maintainer of this work is Niklas Beisert.

This work consists of the files |README.txt|, |childdoc.ins| and |childdoc.dtx|
as well as the derived files |childdoc.def|, |cdocsamp.tex|
with |cdocsch1.tex|, |cdocsch2.tex|, |cdocspt3.tex|, |cdocspt4.tex|,
|cdocsdrf.tex|, |cdocsfn1.tex|, |cdocsfn2.tex|
as well as |childdoc.pdf|.

%%%%%%%%%%%%%%%%%%%%%%%%%%%%%%%%%%%%%%%%%%%%%%%%%%%%%%%%%%%%%%%%%%%%%%%%%%%%%%%%
\subsection{Files and Installation}

The package consists of the files:
%
\begin{center}
\begin{tabular}{ll}
    |README.txt|   & readme file \\
    |childdoc.ins| & installation file \\
    |childdoc.dtx| & source file \\
    |childdoc.def| & definition file \\
    |cdocsamp.tex| & sample main file \\
    |cdocsch1.tex| & sample include file \\
    |cdocsch2.tex| & sample include file \\
    |cdocspt3.tex| & sample part file \\
    |cdocspt4.tex| & sample part file \\
    |cdocsdrf.tex| & sample redirection file \\
    |cdocsfn1.tex| & sample redirection file \\
    |cdocsfn2.tex| & sample redirection file \\
    |childdoc.pdf| & manual
\end{tabular}
\end{center}
%
The distribution consists of the files
|README.txt|, |childdoc.ins| and |childdoc.dtx|.
%
\begin{itemize}
\item
Run (pdf)\LaTeX{} on |childdoc.dtx|
to compile the manual |childdoc.pdf| (this file).
\item
Run \LaTeX{} on |childdoc.ins| to create the definitions file |childdoc.def|
and the sample |cdocsamp.tex| with include files
|cdocsch1.tex|, |cdocsch2.tex|, |cdocspt3.tex|, |cdocspt4.tex|,
|cdocsdrf.tex|, |cdocsfn1.tex|, |cdocsfn2.tex|.
Then copy the file |childdoc.def| to an appropriate directory of your \LaTeX{}
distribution, e.g.\ \textit{texmf-root}|/tex/latex/childdoc|.
\end{itemize}

%%%%%%%%%%%%%%%%%%%%%%%%%%%%%%%%%%%%%%%%%%%%%%%%%%%%%%%%%%%%%%%%%%%%%%%%%%%%%%%%
\subsection{Related CTAN Packages}

There are several other packages which offer a similar functionality:
%
\begin{itemize}
\item
The packages
\href{http://ctan.org/pkg/docmute}{\textsf{docmute}},
\href{http://ctan.org/pkg/includex}{\textsf{includex}} and
\href{http://ctan.org/pkg/standalone}{\textsf{standalone}}
provide commands to include only the document body of
a child file thus allowing both files to be compiled individually.
\item
The packages \href{http://ctan.org/pkg/subdocs}{\textsf{subdocs}}
and \href{http://ctan.org/pkg/subfiles}{\textsf{subfiles}}
provide structures in which the main and child documents can be
encapsulated and allowing them to be compiled individually.
The inclusion mechanism is different from the conventional |\include|.
\item
The package \href{http://ctan.org/pkg/combine}{\textsf{combine}}
is an elaborate solution to combine several documents into one.
\end{itemize}
%
See also the CTAN topic \href{http://ctan.org/topic/subdocs}{\textsf{subdocs}}
for further related packages.
The present package differs from the above solutions in that
a document structure constructed with the conventional |\include| mechanism
just needs two extra commands at the top of every file
such that all constituent files can be compiled individually.

%%%%%%%%%%%%%%%%%%%%%%%%%%%%%%%%%%%%%%%%%%%%%%%%%%%%%%%%%%%%%%%%%%%%%%%%%%%%%%%%
%\subsection{Feature Suggestions}
%
%The following is a list of features which may be useful for future
%versions of this package:
%%
%\begin{itemize}
%\item
%\ldots
%\end{itemize}

%%%%%%%%%%%%%%%%%%%%%%%%%%%%%%%%%%%%%%%%%%%%%%%%%%%%%%%%%%%%%%%%%%%%%%%%%%%%%%%%
\subsection{Revision History}

%%%%%%%%%%%%%%%%%%%%%%%%%%%%%%%%%%%%%%%%
\paragraph{v2.0:} 2018/12/30

\begin{itemize}
\item
immediate forward processing
\item
added |\childdocby| mechanism
\item
manual restructured
\end{itemize}

%%%%%%%%%%%%%%%%%%%%%%%%%%%%%%%%%%%%%%%%
\paragraph{v1.6:} 2018/01/17

\begin{itemize}
\item
application for development of include files
\item
corrections to manual
\end{itemize}

%%%%%%%%%%%%%%%%%%%%%%%%%%%%%%%%%%%%%%%%
\paragraph{v1.5:} 2017/05/21

\begin{itemize}
\item
more complete structuring introduced
\item
|\childdocof| introduced
\item
|\childdoc| renamed to |\childdocmain|
\item
|\childredirect| renamed to |\childdocforward| and |\childdocforwardprefix|
and functionality expanded
\end{itemize}

%%%%%%%%%%%%%%%%%%%%%%%%%%%%%%%%%%%%%%%%
\paragraph{v1.0:} 2017/04/27

\begin{itemize}
\item
manual and install package
\item
first version published on CTAN
\end{itemize}

%%%%%%%%%%%%%%%%%%%%%%%%%%%%%%%%%%%%%%%%
\paragraph{v0.6:} 2017/04/26

\begin{itemize}
\item
redirection mechanism added
\end{itemize}

%%%%%%%%%%%%%%%%%%%%%%%%%%%%%%%%%%%%%%%%
\paragraph{v0.5:} 2017/04/26

\begin{itemize}
\item
functionality in definition file
\end{itemize}


%%%%%%%%%%%%%%%%%%%%%%%%%%%%%%%%%%%%%%%%%%%%%%%%%%%%%%%%%%%%%%%%%%%%%%%%%%%%%%%%
%%%%%%%%%%%%%%%%%%%%%%%%%%%%%%%%%%%%%%%%%%%%%%%%%%%%%%%%%%%%%%%%%%%%%%%%%%%%%%%%
%%%%%%%%%%%%%%%%%%%%%%%%%%%%%%%%%%%%%%%%%%%%%%%%%%%%%%%%%%%%%%%%%%%%%%%%%%%%%%%%
\appendix

\settowidth\MacroIndent{\rmfamily\scriptsize 000\ }

 \DocInput{childdoc.dtx}

\end{document}
%</driver>
% \fi
%
% %%%%%%%%%%%%%%%%%%%%%%%%%%%%%%%%%%%%%%%%%%%%%%%%%%%%%%%%%%%%%%%%%%%%%%%%%%%%%%
% %%%%%%%%%%%%%%%%%%%%%%%%%%%%%%%%%%%%%%%%%%%%%%%%%%%%%%%%%%%%%%%%%%%%%%%%%%%%%%
% \section{Sample}
%\iffalse
%<*samplemain>
%\fi
%
% The following presents a sample document
% with two chapters, two parts, a title page,
% a compile flag as well as three forwarding files to set the flag.
% It consists of eight |.tex| files:
% \begin{center}
% \begin{tabular}{ll}
% |cdocsamp.tex|&main file\\
% |cdocsch1.tex|&include file for chapter 1\\
% |cdocsch2.tex|&include file for chapter 2\\
% |cdocspt3.tex|&include file for part 3\\
% |cdocspt4.tex|&include file for part 4\\
% |cdocsdrf.tex|&forwarding file for main file in draft mode\\
% |cdocsfi1.tex|&forwarding file for final version of chapter 1\\
% |cdocsfi2.tex|&forwarding file for final version of chapter 2\\
% \end{tabular}
% \end{center}
% Each of the eight files can be compiled directly by the \LaTeX{} compiler.
%
% %%%%%%%%%%%%%%%%%%%%%%%%%%%%%%%%%%%%%%
% \paragraph{Main File.}
%
% The main file is called |cdocsamp.tex|.
%
% Load the \textsf{childdoc} definitions and
% declare the filename for the main document:
%    \begin{macrocode}
\input{childdoc.def}
\childdocmain{}
%    \end{macrocode}

% Optional override for |\version| flag:
%    \begin{macrocode}
%%\ifchilddoc\else\providecommand{\version}{draft}\fi
%    \end{macrocode}

% Define the default values for the |\version| flag
% (|final| for the main file and |draft| for childs):
%    \begin{macrocode}
\ifchilddoc
\providecommand{\version}{draft}
\else
\providecommand{\version}{final}
\fi
%    \end{macrocode}

% Load the standard document class:
%    \begin{macrocode}
\documentclass[12pt]{article}
%    \end{macrocode}

% Start the document body:
%    \begin{macrocode}
\begin{document}
%    \end{macrocode}

% Declare a title page.
% Print title, part of document being processed and version flag:
%    \begin{macrocode}
\addtocounter{page}{-1}
\begin{center}
{\LARGE\bfseries{}childdoc example\par}
\vspace{1cm}
\ifchilddoc
\ifchilddocmanual part\else chapter\fi:
`\childdocname' of `\childdocjob'\par
\else
main document: `\childdocjob'\par
\fi
version: \version\par
\end{center}
\newpage
%    \end{macrocode}

% Manually include selected file,
% otherwise process as usual:
%    \begin{macrocode}
\ifchilddocmanual
\section*{part `\childdocname'}
\input{\childdocname}
\else
%    \end{macrocode}

% Include the two chapters:
%    \begin{macrocode}
\include{cdocsch1}
\include{cdocsch2}
%    \end{macrocode}

% Include the two parts unless only chapters should be displayed:
%    \begin{macrocode}
\ifchilddoc\else
\section{part three}
\input{cdocspt3}
\section{part four}
\input{cdocspt4}
\fi
%    \end{macrocode}

% Process as usual until here:
%    \begin{macrocode}
\fi
%    \end{macrocode}

% End of document body:
%    \begin{macrocode}
\end{document}
%    \end{macrocode}
%\iffalse
%</samplemain>
%\fi
%
% %%%%%%%%%%%%%%%%%%%%%%%%%%%%%%%%%%%%%%
% \paragraph{Chapter Include Files.}
%
% The include files are called |cdocsch1.tex| and |cdocsch2.tex|.
%
%\iffalse
%<*samplechap1|samplechap2>
%\fi

% Optional override for |\version| flag:
%    \begin{macrocode}
%%\providecommand{\version}{final}
%    \end{macrocode}

% Include the main document:
%    \begin{macrocode}
\input{childdoc.def}
\childdocof{cdocsamp}
%    \end{macrocode}

%\iffalse
%</samplechap1|samplechap2>
%\fi
%
%\iffalse
%<*samplechap1>
%\fi
% Some text for chapter 1:
%    \begin{macrocode}
\section{one}
some text in chapter one
%    \end{macrocode}

%\iffalse
%</samplechap1>
%\fi
% Some text for chapter 2:
%\iffalse
%<*samplechap2>
%\fi
%    \begin{macrocode}
\section{two}
more text in chapter two
%    \end{macrocode}

%\iffalse
%</samplechap2>
%\fi
%
% %%%%%%%%%%%%%%%%%%%%%%%%%%%%%%%%%%%%%%
% \paragraph{Part Include Files.}
%
% The include files are called |cdocspt3.tex| and |cdocspt4.tex|.
%
%\iffalse
%<*samplepart3|samplepart4>
%\fi

% Optional override for |\version| flag:
%    \begin{macrocode}
%%\providecommand{\version}{final}
%    \end{macrocode}

% Include the main document:
%    \begin{macrocode}
\input{childdoc.def}
\childdocby{cdocsamp}
%    \end{macrocode}

%\iffalse
%</samplepart3|samplepart4>
%\fi
%
%\iffalse
%<*samplepart3>
%\fi
% Some text for part 3:
%    \begin{macrocode}
some text in part three
%    \end{macrocode}

%\iffalse
%</samplepart3>
%\fi
% Some text for part 4:
%\iffalse
%<*samplepart4>
%\fi
%    \begin{macrocode}
more text in part four
%    \end{macrocode}

%\iffalse
%</samplepart4>
%\fi
%
% %%%%%%%%%%%%%%%%%%%%%%%%%%%%%%%%%%%%%%
% \paragraph{Forwarding for a Complete Draft.}
%
% The following forwarding file |cdocsdrf.tex|
% compiles the main document in draft mode:
%\iffalse
%<*sampledraft>
%\fi
%    \begin{macrocode}
\def\version{draft}
\input{childdoc.def}
\childdocforward{cdocsamp}
%    \end{macrocode}

%\iffalse
%</sampledraft>
%\fi
%
% %%%%%%%%%%%%%%%%%%%%%%%%%%%%%%%%%%%%%%
% \paragraph{Forwarding for Final Version of the Chapters.}
%
% The following forwarding files |cdocsfn1.tex| and |cdocsfn2.tex|
% (with identical content)
% compile the final versions of the child documents
% |cdocsch1.tex| and |cdocsch2.tex|, respectively:
%\iffalse
%<*samplefinal>
%\fi
%    \begin{macrocode}
\def\version{final}
\input{childdoc.def}
\childdocforwardprefix[cdocsamp]{cdocsfn}{cdocsch}
%    \end{macrocode}

%\iffalse
%</samplefinal>
%\fi
%
% %%%%%%%%%%%%%%%%%%%%%%%%%%%%%%%%%%%%%%
% \paragraph{Command Line Processing.}
%
% The following three command lines generate the output files
% |cdocscld|, |cdocscl1| and |cdocscl2|
% which should be identical to
% |cdocsdrf|, |cdocsch1| and |cdocsfn2|, respectively:
% \begin{center}
% \begin{tabular}{l}
% |latex -jobname cdocscld \|\\
% |  "\def\version{draft}\input{childdoc.def}\childdocforward{cdocsamp}"|\\
% |latex -jobname cdocscl1 \|\\
% |  "\input{childdoc.def}\childdocforward[cdocsamp]{cdocsch1}"|\\
% |latex -jobname cdocscl2 \|\\
% |  "\def\version{final}\input{childdoc.def}\childdocforward{cdocsch2}"|
% \end{tabular}
% \end{center}
% Note that the trailing backslash on each first line
% merely continues the input to the second line
% (for convenient cut ant paste).
% Furthermore, the command |latex| can be replaced by any
% of its alternative versions such as |pdflatex|.
%
% %%%%%%%%%%%%%%%%%%%%%%%%%%%%%%%%%%%%%%%%%%%%%%%%%%%%%%%%%%%%%%%%%%%%%%%%%%%%%%
% %%%%%%%%%%%%%%%%%%%%%%%%%%%%%%%%%%%%%%%%%%%%%%%%%%%%%%%%%%%%%%%%%%%%%%%%%%%%%%
% \section{Implementation}
%\iffalse
%<*package>
%\fi
%
% This section describes the definitions file |childdoc.def|.

% The definitions cannot be loaded using |\usepackage| or |\RequirePackage|
% which has a mechanism to prevent loading a style file more than once.
% When loading the definitions by means of |\input|
% multiple instances have to be prevented manually:
%\iffalse
%This code needs to be before the `\ProvidesFile' directive
%which is defined at the beginning of this file.
%Therefore it is also placed there and commented out here.
%</package>
%<*discard>
%\fi
%    \begin{macrocode}
\ifdefined\childdocmain\endinput\fi
%    \end{macrocode}
%\iffalse
%</discard>
%<*package>
%\fi
%
% \macro{\ifchilddoc}
% \macro{\ifchilddocmanual}
% The conditional |\ifchilddoc| tells whether a
% child (true) or main (false) document is being compiled.
% The conditional |\ifchilddocmanual| tells whether
% the |\includeonly| mechanism is used (false) or
% the selection of child files must be performed manually (true).
% The definitions initialise to false:
%    \begin{macrocode}
\newif\ifchilddoc
\newif\ifchilddocmanual
%    \end{macrocode}

% \macro{\childdocname}
% \macro{\childdocjob}
% The macro |\childdocname| stores the name of the main document
% to be compiled. The macro |\childdocjob| stores the name of
% the document on which the \LaTeX{} compiler was originally invoked.
% The content of |\jobname| cannot be compared
% to filenames specified in the source due to different catcodes.
% The following code rescans |\jobname|, stores the result
% in |\childdocname| and saves a copy in |\childdocjob|:
%    \begin{macrocode}
\edef\childdocname{\scantokens\expandafter{\jobname\noexpand}}
\let\childdocjob\childdocname
%    \end{macrocode}

% \macro{\childdocdisable}
% The macro |\childdocdisable| prevents the main file
% from being processed more than once.
% At this stage, the main document command |\childdocmain|
% is assumed to be called once again where it should do nothing.
% Any subsequent call to it should prevent
% a secondary processing of the main document
% It overwrites the forwarding commands
% |\childdocof| and |\childdocforward|
% with empty macros to prevent further inclusions of the main document:
%    \begin{macrocode}
\newcommand{\childdocdisable}
{
  \renewcommand{\childdocmain}[1]{\renewcommand{\childdocmain}[1]{\endinput}}
  \renewcommand{\childdocof}[1]{}
  \renewcommand{\childdocby}[2][]{}
  \renewcommand{\childdocforward}[2][]{}
  \renewcommand{\childdocdisable}{}
}
%    \end{macrocode}

% \macro{\childdocmain}
% The macro |\childdocmain| is to be called at the top of the main file
% with nothing or the main filename (without extension) as argument.
% First, it breaks loops.
% If the argument is not empty and does not match |\childdocname|
% (which is set by the first inclusion of |childdoc.def|),
% |\ifchilddoc| is set to true, |\includeonly| is applied to the child file
% and |\jobname| is set to the main file
% (for proper handling of |.aux| files):
%    \begin{macrocode}
\newcommand{\childdocmain}[1]
{
  \childdocdisable\childdocmain{}
  \if?#1?\else
    \begingroup
      \def\childdoctmp{#1}
      \ifx\childdoctmp\childdocname
        \def\childdoctmp{}
      \else
        \def\childdoctmp
        {
          \childdoctrue
          \includeonly{\childdocname}
          \def\childdocjob{#1}
          \def\jobname{#1}
        }
      \fi
      \expandafter
    \endgroup
    \childdoctmp
  \fi
}
%    \end{macrocode}

% \macro{\childdocof}
% The command |\childdocof| redirects
% compilation to the main file |#1|.
%    \begin{macrocode}
\newcommand{\childdocof}[1]
{
  \childdocdisable
  \childdoctrue
  \includeonly{\childdocname}
  \def\jobname{#1}
  \def\childdocjob{#1}
  \input{#1}
}
%    \end{macrocode}

% \macro{\childdocby}
% The command |\childdocby| ....
%    \begin{macrocode}
\newcommand{\childdocby}[2][]
{
  \childdocdisable
  \childdoctrue
  \childdocmanualtrue
  \if?#1?\else
    \def\jobname{#2}
  \fi
  \def\childdocjob{#2}
  \input{#2}
  \endinput
}
%    \end{macrocode}

% \macro{\childdocforward}
% The command |\childdocforward| redirects
% compilation to the main file or
% (if the optional argument is given) a child file.
% Parameters are set as if the main file
% or a child file starting with |\childdocof| was compiled.
% Then compilation is handed over to the main file:
%    \begin{macrocode}
\newcommand{\childdocforward}[2][]
{
  \begingroup
    \if?#1?
      \def\childdoctmp
      {
        \def\childdocname{#2}
        \def\childdocjob{#2}
        \def\jobname{#2}
        \input{#2}
        \endinput
      }
    \else
      \def\childdoctmp
      {
        \childdocdisable
        \def\childdocname{#2}
        \childdoctrue
        \includeonly{#2}
        \def\childdocjob{#1}
        \def\jobname{#1}
        \input{#1}
        \endinput
      }
    \fi
    \expandafter
  \endgroup
  \childdoctmp
}
%    \end{macrocode}

% \macro{\childdocforwardprefix}
% The command |\childdocforwardprefix| redirects
% compilation to the main or a child file by means of a pattern.
% The prefix |#1| in the current filename is replaced by |#2|
% and the suffix of the current filename is kept
% (it is assumed that the filename does not contain the substring `|~~~|'
% which is used as a delimiter).
% Compilation is handed over to the new file by |\childdocforward|:
%    \begin{macrocode}
\newcommand{\childdocforwardprefix}[3][]
{
  \begingroup
    \def\childdocextract #2##1~~~{\def\childdoctmp{\childdocforward[#1]{#3##1}}}
    \expandafter\childdocextract\childdocname~~~
    \expandafter
  \endgroup
  \childdoctmp
}
%    \end{macrocode}

% \macro{\childdoc}
% The deprecated macro |\childdoc| is a legacy version of |\childdocmain|:
%    \begin{macrocode}
\newcommand{\childdoc}{\childdocmain}
%    \end{macrocode}

% \macro{\childdocredirect}
% The deprecated macro |\childdocredirect| is a legacy version
% of |\childdocforward| and |\childdocforwardprefix|:
%    \begin{macrocode}
\newcommand{\childdocredirect}[2][]
{
  \begingroup
    \if?#1?
      \def\childdoctmp{\childdocforward{#2}}
    \else
      \def\childdoctmp{\childdocforwardprefix{#1}{#2}}
    \fi
    \expandafter
  \endgroup
  \childdoctmp
}
%    \end{macrocode}

%\iffalse
%</package>
%\fi
%
\endinput
|\\
|\childdocby{|\textit{main}|}|\\
\end{tabular}
\end{center}
%
Both forms have slightly different effects as described above.
The main file is prepared as usual, see \secref{sec:include}.

%%%%%%%%%%%%%%%%%%%%%%%%%%%%%%%%%%%%%%%%%%%%%%%%%%%%%%%%%%%%%%%%%%%%%%%%%%%%%%%%
\subsection{Legacy Detection}
\label{sec:detection}

The directive |\childdocmain| in the main file can detect
whether the complete document or merely a child is to be compiled
even without using the directive |\childdocof|.
This method is deprecated because it is less robust
and there is no compelling reason to use it;
it is merely provided for backward compatibility
and it may be removed in future versions.

If the detection mechanism is to be used,
it is mandatory to correctly specify
the filename of the main file as the argument of |\childdocmain|:
%
\begin{center}
\begin{tabular}{l}
|% \iffalse
%
% childdoc.dtx Copyright (C) 2017-2018 Niklas Beisert
%
% This work may be distributed and/or modified under the
% conditions of the LaTeX Project Public License, either version 1.3
% of this license or (at your option) any later version.
% The latest version of this license is in
%   http://www.latex-project.org/lppl.txt
% and version 1.3 or later is part of all distributions of LaTeX
% version 2005/12/01 or later.
%
% This work has the LPPL maintenance status `maintained'.
%
% The Current Maintainer of this work is Niklas Beisert.
%
% This work consists of the files childdoc.dtx and childdoc.ins
% and the derived files childdoc.def and cdocsamp.tex with
% cdocsch1.tex, cdocsch2.tex, cdocsdrf.tex, cdocsfn1.tex, cdocsfn2.tex.
%
%<package>\ifdefined\childdocmain\endinput\fi
%<package>\ProvidesFile{childdoc.def}[2018/12/30 v2.0 child document driver]
%<samplemain>\ProvidesFile{cdocsamp.tex}[2018/12/30 v2.0 sample for childdoc]
%<*driver>
%\ProvidesFile{childdoc.drv}[2018/12/30 v2.0 childdoc reference manual file]
\PassOptionsToClass{10pt,a4paper}{article}
\documentclass{ltxdoc}

\usepackage[margin=35mm]{geometry}
\usepackage{hyperref}
\usepackage{hyperxmp}
\usepackage[usenames]{color}

\hypersetup{colorlinks=true}
\hypersetup{pdfstartview=FitH}
\hypersetup{pdfpagemode=UseNone}
\hypersetup{pdfsource={}}
\hypersetup{pdflang={en-UK}}
\hypersetup{pdfcopyright={Copyright 2017-2018 Niklas Beisert.
  This work may be distributed and/or modified under the
  conditions of the LaTeX Project Public License, either version 1.3
  of this license or (at your option) any later version.}}
\hypersetup{pdflicenseurl={http://www.latex-project.org/lppl.txt}}
\hypersetup{pdfcontactaddress={ETH Zurich, ITP, HIT K,
  Wolfgang-Pauli-Strasse 27}}
\hypersetup{pdfcontactpostcode={8093}}
\hypersetup{pdfcontactcity={Zurich}}
\hypersetup{pdfcontactcountry={Switzerland}}
\hypersetup{pdfcontactemail={nbeisert@itp.phys.ethz.ch}}
\hypersetup{pdfcontacturl={http://people.phys.ethz.ch/\xmptilde nbeisert/}}

\newcommand{\secref}[1]{\hyperref[#1]{section \ref*{#1}}}

\parskip1ex
\parindent0pt
\let\olditemize\itemize
\def\itemize{\olditemize\parskip0pt}

\begin{document}

\title{The \textsf{childdoc} Package}
\hypersetup{pdftitle={The childdoc Package}}
\author{Niklas Beisert\\[2ex]
  Institut f\"ur Theoretische Physik\\
  Eidgen\"ossische Technische Hochschule Z\"urich\\
  Wolfgang-Pauli-Strasse 27, 8093 Z\"urich, Switzerland\\[1ex]
  \href{mailto:nbeisert@itp.phys.ethz.ch}
  {\texttt{nbeisert@itp.phys.ethz.ch}}}
\hypersetup{pdfauthor={Niklas Beisert}}
\hypersetup{pdfsubject={Manual for the LaTeX2e Package childdoc}}
\date{30 December 2018, \textsf{v2.0}}
\maketitle

\begin{abstract}\noindent
\textsf{childdoc} is a \LaTeXe{} package
that enables the direct compilation
of document sections included by |\include|
to individual files.
\end{abstract}

\begingroup
\parskip0ex
\tableofcontents
\endgroup

%%%%%%%%%%%%%%%%%%%%%%%%%%%%%%%%%%%%%%%%%%%%%%%%%%%%%%%%%%%%%%%%%%%%%%%%%%%%%%%%
%%%%%%%%%%%%%%%%%%%%%%%%%%%%%%%%%%%%%%%%%%%%%%%%%%%%%%%%%%%%%%%%%%%%%%%%%%%%%%%%
\section{Introduction}

\LaTeX{} provides a mechanism to structure a large document (such as a book)
into a main file and several child files (containing the chapters)
using the |\include| command.
This mechanism is beneficial for documents
which span hundreds of pages in order to
make the source file(s) more manageable.
Moreover, compilation can be restricted to
selected child files by means of the |\includeonly| command.
The latter feature can be used to reduce the compilation time while editing
(this was significantly more useful in the earlier days of \LaTeX{})
or to generate a smaller document which is easier to navigate.
Another application of |\includeonly| is to generate
documents consisting of selected parts of the complete document.

However, there are a few drawbacks of the plain |\include| mechanism:
\begin{itemize}
\item
The child files cannot be compiled on their own,
they can only be compiled via the main file.
A naive editing environment
(such as a text editor with an option
to have the current file processed by \LaTeX)
may require one to switch to the main file before compiling;
attempting to compile the child file produces errors.
\item
The main file must be modified (each time)
to adjust the |\includeonly| command
to the present needs. This easily leaves the main file in a messy state.
\item
The generated document will always carry the filename
of the main document. This is inconvenient if
several child files are to be compiled and
to be kept for distribution.
\end{itemize}

The present package provides a simple interface
to make child files individually compilable by \LaTeX{}.
Compiling a child file then has the same effect as compiling
the main file with an |\includeonly| command
to select the appropriate child.
Moreover the generated document will carry the name of the child
rather than the main file.
This resolves all three above issues.

This feature is meant to make the editing of books,
thesis documents and lecture notes somewhat more convenient.
However, the package can also be used efficiently for
composing a series of documents (such as exercise sheets)
which are typically distributed individually.
It then assists the author in generating the individual documents
(potentially in different versions)
as well as a document containing the collected series.
Another application is in developing style files
or other kinds of included material
where compilation of the style file could redirect
to a sample or test file.

%%%%%%%%%%%%%%%%%%%%%%%%%%%%%%%%%%%%%%%%%%%%%%%%%%%%%%%%%%%%%%%%%%%%%%%%%%%%%%%%
%%%%%%%%%%%%%%%%%%%%%%%%%%%%%%%%%%%%%%%%%%%%%%%%%%%%%%%%%%%%%%%%%%%%%%%%%%%%%%%%
\section{Usage}

First of all, the package \textsf{childdoc} is \emph{not} a standard
\LaTeXe{} |.sty| style file! Therefore it needs to be invoked in
a non-standard way.

%%%%%%%%%%%%%%%%%%%%%%%%%%%%%%%%%%%%%%%%%%%%%%%%%%%%%%%%%%%%%%%%%%%%%%%%%%%%%%%%
\subsection{Included Files}
\label{sec:include}

%%%%%%%%%%%%%%%%%%%%%%%%%%%%%%%%%%%%%%%%
\DescribeMacro{\childdocmain}
To use the package, add the commands
\begin{center}
\begin{tabular}{l}
|\input{childdoc.def}|\\
|\childdocmain{}|\\
\end{tabular}
\end{center}
at the very top of the main \LaTeX{} file,
in particular \emph{before} the |\documentclass| statement!
The argument of |\childdocmain| should be left empty
(but it must be present).

%%%%%%%%%%%%%%%%%%%%%%%%%%%%%%%%%%%%%%%%
\DescribeMacro{\childdocof}
Furthermore, add the commands
\begin{center}
\begin{tabular}{l}
|\input{childdoc.def}|\\
|\childdocof{|\textit{main}|}|\\
\end{tabular}
\end{center}
at the top of every child file \textit{child}
which is included by |\include{|\textit{child}|}|
from within the main file
(or at least for those files to be compiled individually).
The argument \textit{main} must be the filename of the main file.

There are a couple of
considerations in setting up the main and child documents:

%%%%%%%%%%%%%%%%%%%%%%%%%%%%%%%%%%%%%%%%
\paragraph{Restrictions.}

Please note the following restrictions:
\begin{itemize}
\item
|\childdocmain| must be called with one argument \textit{main}
to ensure compatibility with earlier version of the package.
It must either be empty (|\childdocmain{}|)
or precisely match the filename of the main file in which it is specified.
See \secref{sec:detection} for further information.
\item
The filename \textit{main} must be specified without the |.tex| extension.
\item
The filename \textit{main} is case sensitive
(even in case-insensitive file systems)
due to internal string comparison.
\item
The argument \textit{main} should be fully expanded, it cannot be a macro.
\item
Subdirectories and special characters should be avoided in filenames.
\item
The command |\childdocmain{|\textit{main}|}| must be followed by a whitespace.
It should not be followed immediately by another command
or by a comment mark `|%|'.
This is because the \TeX{} parser reads the token immediately following
the argument of |\childdocmain| and puts it
at the beginning of every child section;
however, a white\-space is ignored.
\end{itemize}

%%%%%%%%%%%%%%%%%%%%%%%%%%%%%%%%%%%%%%%%
\paragraph{Content of Main File.}

It is advisable to place all content in the child files included by |\include|.
Any output contained in the main file will appear in all child documents
unless suppressed manually;
it cannot be suppressed automatically by the |\includeonly| directive
and thus should normally be avoided.
A method to include some content in the main file
by means of conditional processing is described in \secref{sec:conditional}.

%%%%%%%%%%%%%%%%%%%%%%%%%%%%%%%%%%%%%%%%
\paragraph{Page Numbering.}

When only a part of the document is compiled,
the appropriate numbering of pages
(as well as other status parameters)
is determined from the |.aux| files.
The latter contain information from previous passes.
However this information needs to propagate through
all intermediate child documents.
Therefore the page numbering in child documents may well
be inconsistent until the complete document is compiled at least once.

A useful (if unconventional) way to always ensure a consistent
page numbering is to restart the numbering in each child document
and denote the pages by `\textit{child}|.|\textit{page}'
where \textit{child} represents the chapter/section number of the child file.
This can be achieved by the command
|\numberwithin{page}{|\textit{child}|}|
of the \textsf{amsmath} package
where \textit{child} can be |chapter| or |section|
depending on the chosen structuring.
Alternatively, one can modify the macro |\thepage| appropriately
and reset the counter |page| at the start of each child file.

%%%%%%%%%%%%%%%%%%%%%%%%%%%%%%%%%%%%%%%%%%%%%%%%%%%%%%%%%%%%%%%%%%%%%%%%%%%%%%%%
\subsection{Conditional Processing}
\label{sec:conditional}

The package provides a mechanism to compile different versions
of a document. To customise the versions further some conditional processing
can come in handy to distinguish which version is being compiled.
The package provides two macros to describe the compilation context:

%%%%%%%%%%%%%%%%%%%%%%%%%%%%%%%%%%%%%%%%
\DescribeMacro{\ifchilddoc}
The conditional |\ifchilddoc| distinguishes between the compilation of
child documents and the main document:
%
\begin{center}
|\ifchilddoc |\textit{child-code}| |[|\||else |\textit{main-code}]| \||fi|
\end{center}

%%%%%%%%%%%%%%%%%%%%%%%%%%%%%%%%%%%%%%%%
\DescribeMacro{\childdocname}
\DescribeMacro{\childdocjob}
The macro |\childdocname| contains the filename (without extension)
of the main or child file being processed.
Note that |\childdocjob| will always contain the name of the main file.

%%%%%%%%%%%%%%%%%%%%%%%%%%%%%%%%%%%%%%%%
\paragraph{Title Page.}

Conditional processing can be used to include a title or banner page
in the main document when proper precautions are taken.
Importantly, the code in the main file should ensure that the page counter
(as well as other status parameters which are stored in the |.aux| files)
takes the same value after the conditional processing.
Otherwise the page numbers may take divergent values
depending on which part is compiled.

For example, a title page could be declared by:
%
\begin{center}
\begin{tabular}{l}
|\ifchilddoc\||else|\\
|\addtocounter{page}{-1}|\\
\textit{code for title page}\\
|\newpage|\\
|\||fi|
\end{tabular}
\end{center}
%
A banner page for the child documents can be generated by:
%
\begin{center}
\begin{tabular}{l}
|\ifchilddoc|\\
|\addtocounter{page}{-1}|\\
\textit{code for banner page}\\
|\newpage|\\
|\||fi|
\end{tabular}
\end{center}
%
Here one could write a message such as:
\begin{center}
|This is the part \childdocname{} of \childdocjob{}.|
\end{center}

%%%%%%%%%%%%%%%%%%%%%%%%%%%%%%%%%%%%%%%%%%%%%%%%%%%%%%%%%%%%%%%%%%%%%%%%%%%%%%%%
\subsection{Flags}
\label{sec:flags}

The package makes it easy to generate different versions
of the main or child documents.
To this end compilation flags can be defined
and assigned different default values.
They will be particularly useful in conjunction
with the forwarding mechanism described in \secref{sec:forward}.

For example, it may be useful to have a flag |\version|
which can be set to |draft| or |final|.
The document source will contain some conditional code
depending on the value of |\version|.
Suppose further, the flag should default to |final| for the main file
and to |draft| for child files
which is a natural assignment for editing the document.
This is achieved by placing the following code
in the preamble of the main document
(below the |\childdocmain| directive):
%
\begin{center}
\begin{tabular}{l}
|\ifchilddoc|\\
|\providecommand{\version}{draft}|\\
|\||else|\\
|\providecommand{\version}{final}|\\
|\||fi|
\end{tabular}
\end{center}
%
The definition by |\providecommand| makes sure
that previous definitions are not overwritten.
Further statements |\providecommand{\version}{...}|
can thus be added before the above code to override it.

For the main file, one might add a line
(between |\childdocmain| and the above block)
%
\begin{center}
|%\ifchilddoc\||else\providecommand{\version}{draft}\||fi|
\end{center}
%
which can be uncommented to produce a draft version.
Likewise one can add a line to the very top of a child file
(above the |\childdocof{|\textit{main}|}| directive)
%
\begin{center}
|%\providecommand{\version}{final}|
\end{center}
%
which can be uncommented to produce the final version of this child document.

%%%%%%%%%%%%%%%%%%%%%%%%%%%%%%%%%%%%%%%%%%%%%%%%%%%%%%%%%%%%%%%%%%%%%%%%%%%%%%%%
\subsection{Forwarding}
\label{sec:forward}

Different versions of the main or child documents
using compilation flags as described in \secref{sec:flags}
can be (permanently) stored in different files
for convenient compilation, viewing and distribution.
To this end, the package defines a command
to pass on compilation to a different file:

%%%%%%%%%%%%%%%%%%%%%%%%%%%%%%%%%%%%%%%%
\DescribeMacro{\childdocforward}
The command |\childdocforward| redirects processing to
another source file:
%
\begin{center}
\begin{tabular}{l}
|\input{childdoc.def}|\\
|\childdocforward[|\textit{main}|]{|\textit{dest}|}|\\
\end{tabular}
\end{center}
%
The argument \textit{dest} is the destination file
(without extension).
It should be the main file or one of the child files.
Note that further \textsf{childdoc} directives
such as |\childdocof| and |\childdocforward|
in the indicated file will be processed in this form.
The optional argument \textit{main}
passes on directly to the main file \textit{main}
while pretending to compile the child \textit{dest}.
This form behaves as if \textit{dest}
issues |\childdocof{|\textit{main}|}| right away,
and no further \textsf{childdoc} directives will be processed.

%%%%%%%%%%%%%%%%%%%%%%%%%%%%%%%%%%%%%%%%
\DescribeMacro{\...prefix}
In the alternative form |\childdocforwardprefix|,
%
\begin{center}
\begin{tabular}{l}
|\input{childdoc.def}|\\
|\childdocforwardprefix[|\textit{main}|]{|\textit{prefix}|}{|\textit{dest}|}|
\end{tabular}
\end{center}
%
the destination file is determined by a pattern
depending on the current file:
To make this work, the current file must be called
`{\textit{prefix}\hspace{0.2em}\textit{suffix}}'
with \textit{prefix} matching precisely the argument.
Processing is then passed on to the file
`{\textit{dest}\hspace{0.2em}\textit{suffix}}'.
Surely, the same effect is achieved by
directly specifying the
argument `{\textit{dest}\hspace{0.2em}\textit{suffix}}'
in the first form.
However, that requires to set up a different file
for each child. With the alternative form of the command
all these files can have exactly the same content
which simplifies setting them up and maintaining them.

For example, the following file |draft.tex|
with a compilation flag |\version| as described in \secref{sec:flags}
compiles the main document as a draft:
%
\begin{center}
\begin{tabular}{l}
|\def\version{draft}|\\
|\input{childdoc.def}|\\
|\childdocforward{|\textit{main}|}|
\end{tabular}
\end{center}
%
Likewise, the following files |final|\textit{nn}|.tex|
compile the final version of the child document
|child|\textit{nn}|.tex|:
%
\begin{center}
\begin{tabular}{l}
|\def\version{final}|\\
|\input{childdoc.def}|\\
|\childdocforwardprefix{final}{child}|
\end{tabular}
\end{center}
%

Note that when several versions of a main file and/or of each child file
are to be generated, it may be convenient to set up a |Makefile| or
shell script to automatise the process.

%%%%%%%%%%%%%%%%%%%%%%%%%%%%%%%%%%%%%%%%%%%%%%%%%%%%%%%%%%%%%%%%%%%%%%%%%%%%%%%%
\subsection{Command Line Processing}
\label{sec:commandline}

The effect of redirection files can also be achieved by invoking
the \LaTeX{} compiler with a more elaborate command line.
Most conveniently this should be done as part
of a shell script or a |Makefile|.

When using \textsf{childdoc} in the main file, the following
command lines effectively perform a redirection
(note that depending on the shell being used,
backslashes may have to be doubled: `|\|' $\to$ `|\\|'):
%
\begin{center}
|... -jobname "|\textit{target}|" |\\|"|[\textit{flags}]%
|\input{childdoc.def}\childdocforward[|\textit{main}|]{|\textit{dest}|}"|
\end{center}
%
Here \textit{target} is the name of the output file,
\textit{main} is the name of the main file
and \textit{dest} is the name of the main or child file to be processed
(all filenames without extensions).
The optional argument \textit{main} can be omitted
if \textit{main} matches \textit{dest}.
Optionally, compilation \textit{flags} can be defined via |\def| commands.
This command line makes the \TeX{} engine believe
it is compiling the file \textit{target}
whose content is specified as the latter parameter.
The provided code then forwards the processing to
\textit{main} or \textit{dest} as described in \secref{sec:forward}.

%%%%%%%%%%%%%%%%%%%%%%%%%%%%%%%%%%%%%%%%%%%%%%%%%%%%%%%%%%%%%%%%%%%%%%%%%%%%%%%%
\subsection{Include by Input}
\label{sec:input}

Including child documents by |\include| has some restrictions by design.
Most notably, the content of a child document always occupies
its own set of pages; pages cannot be shared between child documents.
Usually, this behaviour makes perfect sense
because each child document contain an essential part of the document.
However, in some situations it may be desirable to compose
a document from a collection of parts
without having mandatory page breaks between then.
For this case, the package
provides a mechanism to include parts
by |\input| which can also be processed individually.
However, by construction this mechanism
requires manual handling of the content to be output.

%%%%%%%%%%%%%%%%%%%%%%%%%%%%%%%%%%%%%%%%
\DescribeMacro{\ifchilddocmanual}
The main file should be prepared as usual, see \secref{sec:include}.
However, the document body must make a distinction
between processing of an individual part and of the main document, e.g.:
%
\begin{center}
\begin{tabular}{l}
|\ifchilddocmanual|\\
|\input{\childdocname}|\\
|\||else|\\
\textit{document body with }|\input{|\textit{part}|}|\\
|\||fi|
\end{tabular}
\end{center}
%
The conditional |\ifchilddocmanual| is true whenever
a part to be included by |\input| is being compiled,
and the name of the part is stored in |\childdocname|.

%%%%%%%%%%%%%%%%%%%%%%%%%%%%%%%%%%%%%%%%
\DescribeMacro{\childdocby}
Each part to be included by |\input| should start with:
%
\begin{center}
\begin{tabular}{l}
|\input{childdoc.def}|\\
|\childdocby{|\textit{main}|}|\\
\end{tabular}
\end{center}
%
The directive |\childdocby| is similar to |\childdocof|
described in \secref{sec:include},
but the subsequent selection of content must be done manually.
To that end, both |\ifchilddoc| and |\ifchilddocmanual|
will be true upon processing of a part,
and the name of the part is stored in |\childdocname|.
Note that |\jobname| will be set to the filename of the current part
so that each part receives an individual |.aux| file
that does not interfere with the |.aux| file(s) of the main document.
This behaviour can be altered by the alternative form
|\childdocby[*]{|\textit{main}|}| (with a non-empty optional argument)
which uses the |.aux| file of the main document
by setting |\jobname| to \textit{main}.

%%%%%%%%%%%%%%%%%%%%%%%%%%%%%%%%%%%%%%%%%%%%%%%%%%%%%%%%%%%%%%%%%%%%%%%%%%%%%%%%
\subsection{Driver Development}
\label{sec:driver}

The \textsf{childdoc} mechanism can also be use for the development
of definition files such as \LaTeX{} styles or classes.
This case differs from the above setup with multiple parts
included by |\include| in that no |\includeonly| should be invoked.
This can be achieved by starting the include file
(before |\ProvidesPackage|) with:
%
\begin{center}
\begin{tabular}{l}
|\input{childdoc.def}|\\
|\childdocforward{|\textit{main}|}|\\
\end{tabular}
\end{center}
%
or alternatively with:
%
\begin{center}
\begin{tabular}{l}
|\input{childdoc.def}|\\
|\childdocby{|\textit{main}|}|\\
\end{tabular}
\end{center}
%
Both forms have slightly different effects as described above.
The main file is prepared as usual, see \secref{sec:include}.

%%%%%%%%%%%%%%%%%%%%%%%%%%%%%%%%%%%%%%%%%%%%%%%%%%%%%%%%%%%%%%%%%%%%%%%%%%%%%%%%
\subsection{Legacy Detection}
\label{sec:detection}

The directive |\childdocmain| in the main file can detect
whether the complete document or merely a child is to be compiled
even without using the directive |\childdocof|.
This method is deprecated because it is less robust
and there is no compelling reason to use it;
it is merely provided for backward compatibility
and it may be removed in future versions.

If the detection mechanism is to be used,
it is mandatory to correctly specify
the filename of the main file as the argument of |\childdocmain|:
%
\begin{center}
\begin{tabular}{l}
|\input{childdoc.def}|\\
|\childdocmain{|\textit{main}|}|\\
\end{tabular}
\end{center}
%
If |\jobname| does not match the argument \textit{main} of |\childdocmain|,
it is assumed that |\jobname| points to the child file to be compiled.
When using |\childdocmain| with the main file specified as argument,
it suffices to start a child file
with just |\input{|\textit{main}|}|
without loading of the package and using |\childdocof|.
If instead all processing is done
with the appropriate \textsf{childdoc} directives,
the argument of \textit{main} of |\childdocmain| can be empty.

An alternative version of the command line processing described
in \secref{sec:commandline} using the detection mechanism reads:
%
\begin{center}
|... -jobname "|\textit{target}|" "|[\textit{flags}]%
[|\def\jobname{|\textit{dest}|}|]|\input{|\textit{main}|}"|
\end{center}

%%%%%%%%%%%%%%%%%%%%%%%%%%%%%%%%%%%%%%%%%%%%%%%%%%%%%%%%%%%%%%%%%%%%%%%%%%%%%%%%
\subsection{Manual Code}
\label{sec:manual}

In case one cannot be certain whether the definitions file |childdoc.def|
is installed on the target \TeX{} distribution
and one prefers not to ship it,
it is conceivable to paste a few relevant commands into the sources.

To that end, drop all statements |\input{childdoc.def}|
and perform the replacements as outlined below.
Instead of |\childdocmain{|\textit{main}|}| add the following code
to the top of the main file:
%
\begin{center}
\begin{tabular}{l}
|\||ifdefined\childdocname\endinput\||fi\newif\ifchilddoc|\\
|\edef\childdocname{\scantokens\expandafter{\jobname\noexpand}}|\\
|\def\childdocmain{|\textit{main}|}\||ifx\childdocmain\childdocname\||else|\\
|\childdoctrue\includeonly{\childdocname}\let\jobname\childdocmain\||fi|\\
\end{tabular}
\end{center}
%
Instead of |\childdocof{|\textit{main}|}| just include the main file
at the top of each child file:
%
\begin{center}
|\input{|\textit{main}|}|
\end{center}
%
A simple redirection |\childdocforward{|\textit{dest}|}| is achieved by:
%
\begin{center}
|\def\jobname{|\textit{dest}|}\input{\jobname}|
\end{center}
%
The redirection with prefix
|\childdocforwardprefix[|\textit{prefix}|]{|\textit{dest}|}|
is accomplished by:
%
\begin{center}
\begin{tabular}{l}
|{\edef\jobname{\scantokens\expandafter{\jobname\noexpand}}|\\
|\def\redirectjob |\textit{prefix}|#1~~~{\gdef\jobname{|\textit{dest}|#1}}|\\
|\expandafter\redirectjob\jobname~~~}\input{\jobname}|
\end{tabular}
\end{center}

In an alternative approach,
child documents can be compiled by a specific command line
without additional code or specific definitions:
%
\begin{center}
|... -jobname "|\textit{target}|" "|[\textit{flags}]%
|\includeonly{|\textit{dest}|}\input{|\textit{main}|}"|
\end{center}
%

%%%%%%%%%%%%%%%%%%%%%%%%%%%%%%%%%%%%%%%%%%%%%%%%%%%%%%%%%%%%%%%%%%%%%%%%%%%%%%%%
%%%%%%%%%%%%%%%%%%%%%%%%%%%%%%%%%%%%%%%%%%%%%%%%%%%%%%%%%%%%%%%%%%%%%%%%%%%%%%%%
\section{Information}

%%%%%%%%%%%%%%%%%%%%%%%%%%%%%%%%%%%%%%%%%%%%%%%%%%%%%%%%%%%%%%%%%%%%%%%%%%%%%%%%
\subsection{Copyright}

Copyright \copyright{} 2017--2018 Niklas Beisert

This work may be distributed and/or modified under the
conditions of the \LaTeX{} Project Public License, either version 1.3
of this license or (at your option) any later version.
The latest version of this license is in
  \url{http://www.latex-project.org/lppl.txt}
and version 1.3 or later is part of all distributions of \LaTeX{}
version 2005/12/01 or later.

This work has the LPPL maintenance status `maintained'.

The Current Maintainer of this work is Niklas Beisert.

This work consists of the files |README.txt|, |childdoc.ins| and |childdoc.dtx|
as well as the derived files |childdoc.def|, |cdocsamp.tex|
with |cdocsch1.tex|, |cdocsch2.tex|, |cdocspt3.tex|, |cdocspt4.tex|,
|cdocsdrf.tex|, |cdocsfn1.tex|, |cdocsfn2.tex|
as well as |childdoc.pdf|.

%%%%%%%%%%%%%%%%%%%%%%%%%%%%%%%%%%%%%%%%%%%%%%%%%%%%%%%%%%%%%%%%%%%%%%%%%%%%%%%%
\subsection{Files and Installation}

The package consists of the files:
%
\begin{center}
\begin{tabular}{ll}
    |README.txt|   & readme file \\
    |childdoc.ins| & installation file \\
    |childdoc.dtx| & source file \\
    |childdoc.def| & definition file \\
    |cdocsamp.tex| & sample main file \\
    |cdocsch1.tex| & sample include file \\
    |cdocsch2.tex| & sample include file \\
    |cdocspt3.tex| & sample part file \\
    |cdocspt4.tex| & sample part file \\
    |cdocsdrf.tex| & sample redirection file \\
    |cdocsfn1.tex| & sample redirection file \\
    |cdocsfn2.tex| & sample redirection file \\
    |childdoc.pdf| & manual
\end{tabular}
\end{center}
%
The distribution consists of the files
|README.txt|, |childdoc.ins| and |childdoc.dtx|.
%
\begin{itemize}
\item
Run (pdf)\LaTeX{} on |childdoc.dtx|
to compile the manual |childdoc.pdf| (this file).
\item
Run \LaTeX{} on |childdoc.ins| to create the definitions file |childdoc.def|
and the sample |cdocsamp.tex| with include files
|cdocsch1.tex|, |cdocsch2.tex|, |cdocspt3.tex|, |cdocspt4.tex|,
|cdocsdrf.tex|, |cdocsfn1.tex|, |cdocsfn2.tex|.
Then copy the file |childdoc.def| to an appropriate directory of your \LaTeX{}
distribution, e.g.\ \textit{texmf-root}|/tex/latex/childdoc|.
\end{itemize}

%%%%%%%%%%%%%%%%%%%%%%%%%%%%%%%%%%%%%%%%%%%%%%%%%%%%%%%%%%%%%%%%%%%%%%%%%%%%%%%%
\subsection{Related CTAN Packages}

There are several other packages which offer a similar functionality:
%
\begin{itemize}
\item
The packages
\href{http://ctan.org/pkg/docmute}{\textsf{docmute}},
\href{http://ctan.org/pkg/includex}{\textsf{includex}} and
\href{http://ctan.org/pkg/standalone}{\textsf{standalone}}
provide commands to include only the document body of
a child file thus allowing both files to be compiled individually.
\item
The packages \href{http://ctan.org/pkg/subdocs}{\textsf{subdocs}}
and \href{http://ctan.org/pkg/subfiles}{\textsf{subfiles}}
provide structures in which the main and child documents can be
encapsulated and allowing them to be compiled individually.
The inclusion mechanism is different from the conventional |\include|.
\item
The package \href{http://ctan.org/pkg/combine}{\textsf{combine}}
is an elaborate solution to combine several documents into one.
\end{itemize}
%
See also the CTAN topic \href{http://ctan.org/topic/subdocs}{\textsf{subdocs}}
for further related packages.
The present package differs from the above solutions in that
a document structure constructed with the conventional |\include| mechanism
just needs two extra commands at the top of every file
such that all constituent files can be compiled individually.

%%%%%%%%%%%%%%%%%%%%%%%%%%%%%%%%%%%%%%%%%%%%%%%%%%%%%%%%%%%%%%%%%%%%%%%%%%%%%%%%
%\subsection{Feature Suggestions}
%
%The following is a list of features which may be useful for future
%versions of this package:
%%
%\begin{itemize}
%\item
%\ldots
%\end{itemize}

%%%%%%%%%%%%%%%%%%%%%%%%%%%%%%%%%%%%%%%%%%%%%%%%%%%%%%%%%%%%%%%%%%%%%%%%%%%%%%%%
\subsection{Revision History}

%%%%%%%%%%%%%%%%%%%%%%%%%%%%%%%%%%%%%%%%
\paragraph{v2.0:} 2018/12/30

\begin{itemize}
\item
immediate forward processing
\item
added |\childdocby| mechanism
\item
manual restructured
\end{itemize}

%%%%%%%%%%%%%%%%%%%%%%%%%%%%%%%%%%%%%%%%
\paragraph{v1.6:} 2018/01/17

\begin{itemize}
\item
application for development of include files
\item
corrections to manual
\end{itemize}

%%%%%%%%%%%%%%%%%%%%%%%%%%%%%%%%%%%%%%%%
\paragraph{v1.5:} 2017/05/21

\begin{itemize}
\item
more complete structuring introduced
\item
|\childdocof| introduced
\item
|\childdoc| renamed to |\childdocmain|
\item
|\childredirect| renamed to |\childdocforward| and |\childdocforwardprefix|
and functionality expanded
\end{itemize}

%%%%%%%%%%%%%%%%%%%%%%%%%%%%%%%%%%%%%%%%
\paragraph{v1.0:} 2017/04/27

\begin{itemize}
\item
manual and install package
\item
first version published on CTAN
\end{itemize}

%%%%%%%%%%%%%%%%%%%%%%%%%%%%%%%%%%%%%%%%
\paragraph{v0.6:} 2017/04/26

\begin{itemize}
\item
redirection mechanism added
\end{itemize}

%%%%%%%%%%%%%%%%%%%%%%%%%%%%%%%%%%%%%%%%
\paragraph{v0.5:} 2017/04/26

\begin{itemize}
\item
functionality in definition file
\end{itemize}


%%%%%%%%%%%%%%%%%%%%%%%%%%%%%%%%%%%%%%%%%%%%%%%%%%%%%%%%%%%%%%%%%%%%%%%%%%%%%%%%
%%%%%%%%%%%%%%%%%%%%%%%%%%%%%%%%%%%%%%%%%%%%%%%%%%%%%%%%%%%%%%%%%%%%%%%%%%%%%%%%
%%%%%%%%%%%%%%%%%%%%%%%%%%%%%%%%%%%%%%%%%%%%%%%%%%%%%%%%%%%%%%%%%%%%%%%%%%%%%%%%
\appendix

\settowidth\MacroIndent{\rmfamily\scriptsize 000\ }

 \DocInput{childdoc.dtx}

\end{document}
%</driver>
% \fi
%
% %%%%%%%%%%%%%%%%%%%%%%%%%%%%%%%%%%%%%%%%%%%%%%%%%%%%%%%%%%%%%%%%%%%%%%%%%%%%%%
% %%%%%%%%%%%%%%%%%%%%%%%%%%%%%%%%%%%%%%%%%%%%%%%%%%%%%%%%%%%%%%%%%%%%%%%%%%%%%%
% \section{Sample}
%\iffalse
%<*samplemain>
%\fi
%
% The following presents a sample document
% with two chapters, two parts, a title page,
% a compile flag as well as three forwarding files to set the flag.
% It consists of eight |.tex| files:
% \begin{center}
% \begin{tabular}{ll}
% |cdocsamp.tex|&main file\\
% |cdocsch1.tex|&include file for chapter 1\\
% |cdocsch2.tex|&include file for chapter 2\\
% |cdocspt3.tex|&include file for part 3\\
% |cdocspt4.tex|&include file for part 4\\
% |cdocsdrf.tex|&forwarding file for main file in draft mode\\
% |cdocsfi1.tex|&forwarding file for final version of chapter 1\\
% |cdocsfi2.tex|&forwarding file for final version of chapter 2\\
% \end{tabular}
% \end{center}
% Each of the eight files can be compiled directly by the \LaTeX{} compiler.
%
% %%%%%%%%%%%%%%%%%%%%%%%%%%%%%%%%%%%%%%
% \paragraph{Main File.}
%
% The main file is called |cdocsamp.tex|.
%
% Load the \textsf{childdoc} definitions and
% declare the filename for the main document:
%    \begin{macrocode}
\input{childdoc.def}
\childdocmain{}
%    \end{macrocode}

% Optional override for |\version| flag:
%    \begin{macrocode}
%%\ifchilddoc\else\providecommand{\version}{draft}\fi
%    \end{macrocode}

% Define the default values for the |\version| flag
% (|final| for the main file and |draft| for childs):
%    \begin{macrocode}
\ifchilddoc
\providecommand{\version}{draft}
\else
\providecommand{\version}{final}
\fi
%    \end{macrocode}

% Load the standard document class:
%    \begin{macrocode}
\documentclass[12pt]{article}
%    \end{macrocode}

% Start the document body:
%    \begin{macrocode}
\begin{document}
%    \end{macrocode}

% Declare a title page.
% Print title, part of document being processed and version flag:
%    \begin{macrocode}
\addtocounter{page}{-1}
\begin{center}
{\LARGE\bfseries{}childdoc example\par}
\vspace{1cm}
\ifchilddoc
\ifchilddocmanual part\else chapter\fi:
`\childdocname' of `\childdocjob'\par
\else
main document: `\childdocjob'\par
\fi
version: \version\par
\end{center}
\newpage
%    \end{macrocode}

% Manually include selected file,
% otherwise process as usual:
%    \begin{macrocode}
\ifchilddocmanual
\section*{part `\childdocname'}
\input{\childdocname}
\else
%    \end{macrocode}

% Include the two chapters:
%    \begin{macrocode}
\include{cdocsch1}
\include{cdocsch2}
%    \end{macrocode}

% Include the two parts unless only chapters should be displayed:
%    \begin{macrocode}
\ifchilddoc\else
\section{part three}
\input{cdocspt3}
\section{part four}
\input{cdocspt4}
\fi
%    \end{macrocode}

% Process as usual until here:
%    \begin{macrocode}
\fi
%    \end{macrocode}

% End of document body:
%    \begin{macrocode}
\end{document}
%    \end{macrocode}
%\iffalse
%</samplemain>
%\fi
%
% %%%%%%%%%%%%%%%%%%%%%%%%%%%%%%%%%%%%%%
% \paragraph{Chapter Include Files.}
%
% The include files are called |cdocsch1.tex| and |cdocsch2.tex|.
%
%\iffalse
%<*samplechap1|samplechap2>
%\fi

% Optional override for |\version| flag:
%    \begin{macrocode}
%%\providecommand{\version}{final}
%    \end{macrocode}

% Include the main document:
%    \begin{macrocode}
\input{childdoc.def}
\childdocof{cdocsamp}
%    \end{macrocode}

%\iffalse
%</samplechap1|samplechap2>
%\fi
%
%\iffalse
%<*samplechap1>
%\fi
% Some text for chapter 1:
%    \begin{macrocode}
\section{one}
some text in chapter one
%    \end{macrocode}

%\iffalse
%</samplechap1>
%\fi
% Some text for chapter 2:
%\iffalse
%<*samplechap2>
%\fi
%    \begin{macrocode}
\section{two}
more text in chapter two
%    \end{macrocode}

%\iffalse
%</samplechap2>
%\fi
%
% %%%%%%%%%%%%%%%%%%%%%%%%%%%%%%%%%%%%%%
% \paragraph{Part Include Files.}
%
% The include files are called |cdocspt3.tex| and |cdocspt4.tex|.
%
%\iffalse
%<*samplepart3|samplepart4>
%\fi

% Optional override for |\version| flag:
%    \begin{macrocode}
%%\providecommand{\version}{final}
%    \end{macrocode}

% Include the main document:
%    \begin{macrocode}
\input{childdoc.def}
\childdocby{cdocsamp}
%    \end{macrocode}

%\iffalse
%</samplepart3|samplepart4>
%\fi
%
%\iffalse
%<*samplepart3>
%\fi
% Some text for part 3:
%    \begin{macrocode}
some text in part three
%    \end{macrocode}

%\iffalse
%</samplepart3>
%\fi
% Some text for part 4:
%\iffalse
%<*samplepart4>
%\fi
%    \begin{macrocode}
more text in part four
%    \end{macrocode}

%\iffalse
%</samplepart4>
%\fi
%
% %%%%%%%%%%%%%%%%%%%%%%%%%%%%%%%%%%%%%%
% \paragraph{Forwarding for a Complete Draft.}
%
% The following forwarding file |cdocsdrf.tex|
% compiles the main document in draft mode:
%\iffalse
%<*sampledraft>
%\fi
%    \begin{macrocode}
\def\version{draft}
\input{childdoc.def}
\childdocforward{cdocsamp}
%    \end{macrocode}

%\iffalse
%</sampledraft>
%\fi
%
% %%%%%%%%%%%%%%%%%%%%%%%%%%%%%%%%%%%%%%
% \paragraph{Forwarding for Final Version of the Chapters.}
%
% The following forwarding files |cdocsfn1.tex| and |cdocsfn2.tex|
% (with identical content)
% compile the final versions of the child documents
% |cdocsch1.tex| and |cdocsch2.tex|, respectively:
%\iffalse
%<*samplefinal>
%\fi
%    \begin{macrocode}
\def\version{final}
\input{childdoc.def}
\childdocforwardprefix[cdocsamp]{cdocsfn}{cdocsch}
%    \end{macrocode}

%\iffalse
%</samplefinal>
%\fi
%
% %%%%%%%%%%%%%%%%%%%%%%%%%%%%%%%%%%%%%%
% \paragraph{Command Line Processing.}
%
% The following three command lines generate the output files
% |cdocscld|, |cdocscl1| and |cdocscl2|
% which should be identical to
% |cdocsdrf|, |cdocsch1| and |cdocsfn2|, respectively:
% \begin{center}
% \begin{tabular}{l}
% |latex -jobname cdocscld \|\\
% |  "\def\version{draft}\input{childdoc.def}\childdocforward{cdocsamp}"|\\
% |latex -jobname cdocscl1 \|\\
% |  "\input{childdoc.def}\childdocforward[cdocsamp]{cdocsch1}"|\\
% |latex -jobname cdocscl2 \|\\
% |  "\def\version{final}\input{childdoc.def}\childdocforward{cdocsch2}"|
% \end{tabular}
% \end{center}
% Note that the trailing backslash on each first line
% merely continues the input to the second line
% (for convenient cut ant paste).
% Furthermore, the command |latex| can be replaced by any
% of its alternative versions such as |pdflatex|.
%
% %%%%%%%%%%%%%%%%%%%%%%%%%%%%%%%%%%%%%%%%%%%%%%%%%%%%%%%%%%%%%%%%%%%%%%%%%%%%%%
% %%%%%%%%%%%%%%%%%%%%%%%%%%%%%%%%%%%%%%%%%%%%%%%%%%%%%%%%%%%%%%%%%%%%%%%%%%%%%%
% \section{Implementation}
%\iffalse
%<*package>
%\fi
%
% This section describes the definitions file |childdoc.def|.

% The definitions cannot be loaded using |\usepackage| or |\RequirePackage|
% which has a mechanism to prevent loading a style file more than once.
% When loading the definitions by means of |\input|
% multiple instances have to be prevented manually:
%\iffalse
%This code needs to be before the `\ProvidesFile' directive
%which is defined at the beginning of this file.
%Therefore it is also placed there and commented out here.
%</package>
%<*discard>
%\fi
%    \begin{macrocode}
\ifdefined\childdocmain\endinput\fi
%    \end{macrocode}
%\iffalse
%</discard>
%<*package>
%\fi
%
% \macro{\ifchilddoc}
% \macro{\ifchilddocmanual}
% The conditional |\ifchilddoc| tells whether a
% child (true) or main (false) document is being compiled.
% The conditional |\ifchilddocmanual| tells whether
% the |\includeonly| mechanism is used (false) or
% the selection of child files must be performed manually (true).
% The definitions initialise to false:
%    \begin{macrocode}
\newif\ifchilddoc
\newif\ifchilddocmanual
%    \end{macrocode}

% \macro{\childdocname}
% \macro{\childdocjob}
% The macro |\childdocname| stores the name of the main document
% to be compiled. The macro |\childdocjob| stores the name of
% the document on which the \LaTeX{} compiler was originally invoked.
% The content of |\jobname| cannot be compared
% to filenames specified in the source due to different catcodes.
% The following code rescans |\jobname|, stores the result
% in |\childdocname| and saves a copy in |\childdocjob|:
%    \begin{macrocode}
\edef\childdocname{\scantokens\expandafter{\jobname\noexpand}}
\let\childdocjob\childdocname
%    \end{macrocode}

% \macro{\childdocdisable}
% The macro |\childdocdisable| prevents the main file
% from being processed more than once.
% At this stage, the main document command |\childdocmain|
% is assumed to be called once again where it should do nothing.
% Any subsequent call to it should prevent
% a secondary processing of the main document
% It overwrites the forwarding commands
% |\childdocof| and |\childdocforward|
% with empty macros to prevent further inclusions of the main document:
%    \begin{macrocode}
\newcommand{\childdocdisable}
{
  \renewcommand{\childdocmain}[1]{\renewcommand{\childdocmain}[1]{\endinput}}
  \renewcommand{\childdocof}[1]{}
  \renewcommand{\childdocby}[2][]{}
  \renewcommand{\childdocforward}[2][]{}
  \renewcommand{\childdocdisable}{}
}
%    \end{macrocode}

% \macro{\childdocmain}
% The macro |\childdocmain| is to be called at the top of the main file
% with nothing or the main filename (without extension) as argument.
% First, it breaks loops.
% If the argument is not empty and does not match |\childdocname|
% (which is set by the first inclusion of |childdoc.def|),
% |\ifchilddoc| is set to true, |\includeonly| is applied to the child file
% and |\jobname| is set to the main file
% (for proper handling of |.aux| files):
%    \begin{macrocode}
\newcommand{\childdocmain}[1]
{
  \childdocdisable\childdocmain{}
  \if?#1?\else
    \begingroup
      \def\childdoctmp{#1}
      \ifx\childdoctmp\childdocname
        \def\childdoctmp{}
      \else
        \def\childdoctmp
        {
          \childdoctrue
          \includeonly{\childdocname}
          \def\childdocjob{#1}
          \def\jobname{#1}
        }
      \fi
      \expandafter
    \endgroup
    \childdoctmp
  \fi
}
%    \end{macrocode}

% \macro{\childdocof}
% The command |\childdocof| redirects
% compilation to the main file |#1|.
%    \begin{macrocode}
\newcommand{\childdocof}[1]
{
  \childdocdisable
  \childdoctrue
  \includeonly{\childdocname}
  \def\jobname{#1}
  \def\childdocjob{#1}
  \input{#1}
}
%    \end{macrocode}

% \macro{\childdocby}
% The command |\childdocby| ....
%    \begin{macrocode}
\newcommand{\childdocby}[2][]
{
  \childdocdisable
  \childdoctrue
  \childdocmanualtrue
  \if?#1?\else
    \def\jobname{#2}
  \fi
  \def\childdocjob{#2}
  \input{#2}
  \endinput
}
%    \end{macrocode}

% \macro{\childdocforward}
% The command |\childdocforward| redirects
% compilation to the main file or
% (if the optional argument is given) a child file.
% Parameters are set as if the main file
% or a child file starting with |\childdocof| was compiled.
% Then compilation is handed over to the main file:
%    \begin{macrocode}
\newcommand{\childdocforward}[2][]
{
  \begingroup
    \if?#1?
      \def\childdoctmp
      {
        \def\childdocname{#2}
        \def\childdocjob{#2}
        \def\jobname{#2}
        \input{#2}
        \endinput
      }
    \else
      \def\childdoctmp
      {
        \childdocdisable
        \def\childdocname{#2}
        \childdoctrue
        \includeonly{#2}
        \def\childdocjob{#1}
        \def\jobname{#1}
        \input{#1}
        \endinput
      }
    \fi
    \expandafter
  \endgroup
  \childdoctmp
}
%    \end{macrocode}

% \macro{\childdocforwardprefix}
% The command |\childdocforwardprefix| redirects
% compilation to the main or a child file by means of a pattern.
% The prefix |#1| in the current filename is replaced by |#2|
% and the suffix of the current filename is kept
% (it is assumed that the filename does not contain the substring `|~~~|'
% which is used as a delimiter).
% Compilation is handed over to the new file by |\childdocforward|:
%    \begin{macrocode}
\newcommand{\childdocforwardprefix}[3][]
{
  \begingroup
    \def\childdocextract #2##1~~~{\def\childdoctmp{\childdocforward[#1]{#3##1}}}
    \expandafter\childdocextract\childdocname~~~
    \expandafter
  \endgroup
  \childdoctmp
}
%    \end{macrocode}

% \macro{\childdoc}
% The deprecated macro |\childdoc| is a legacy version of |\childdocmain|:
%    \begin{macrocode}
\newcommand{\childdoc}{\childdocmain}
%    \end{macrocode}

% \macro{\childdocredirect}
% The deprecated macro |\childdocredirect| is a legacy version
% of |\childdocforward| and |\childdocforwardprefix|:
%    \begin{macrocode}
\newcommand{\childdocredirect}[2][]
{
  \begingroup
    \if?#1?
      \def\childdoctmp{\childdocforward{#2}}
    \else
      \def\childdoctmp{\childdocforwardprefix{#1}{#2}}
    \fi
    \expandafter
  \endgroup
  \childdoctmp
}
%    \end{macrocode}

%\iffalse
%</package>
%\fi
%
\endinput
|\\
|\childdocmain{|\textit{main}|}|\\
\end{tabular}
\end{center}
%
If |\jobname| does not match the argument \textit{main} of |\childdocmain|,
it is assumed that |\jobname| points to the child file to be compiled.
When using |\childdocmain| with the main file specified as argument,
it suffices to start a child file
with just |\input{|\textit{main}|}|
without loading of the package and using |\childdocof|.
If instead all processing is done
with the appropriate \textsf{childdoc} directives,
the argument of \textit{main} of |\childdocmain| can be empty.

An alternative version of the command line processing described
in \secref{sec:commandline} using the detection mechanism reads:
%
\begin{center}
|... -jobname "|\textit{target}|" "|[\textit{flags}]%
[|\def\jobname{|\textit{dest}|}|]|\input{|\textit{main}|}"|
\end{center}

%%%%%%%%%%%%%%%%%%%%%%%%%%%%%%%%%%%%%%%%%%%%%%%%%%%%%%%%%%%%%%%%%%%%%%%%%%%%%%%%
\subsection{Manual Code}
\label{sec:manual}

In case one cannot be certain whether the definitions file |childdoc.def|
is installed on the target \TeX{} distribution
and one prefers not to ship it,
it is conceivable to paste a few relevant commands into the sources.

To that end, drop all statements |% \iffalse
%
% childdoc.dtx Copyright (C) 2017-2018 Niklas Beisert
%
% This work may be distributed and/or modified under the
% conditions of the LaTeX Project Public License, either version 1.3
% of this license or (at your option) any later version.
% The latest version of this license is in
%   http://www.latex-project.org/lppl.txt
% and version 1.3 or later is part of all distributions of LaTeX
% version 2005/12/01 or later.
%
% This work has the LPPL maintenance status `maintained'.
%
% The Current Maintainer of this work is Niklas Beisert.
%
% This work consists of the files childdoc.dtx and childdoc.ins
% and the derived files childdoc.def and cdocsamp.tex with
% cdocsch1.tex, cdocsch2.tex, cdocsdrf.tex, cdocsfn1.tex, cdocsfn2.tex.
%
%<package>\ifdefined\childdocmain\endinput\fi
%<package>\ProvidesFile{childdoc.def}[2018/12/30 v2.0 child document driver]
%<samplemain>\ProvidesFile{cdocsamp.tex}[2018/12/30 v2.0 sample for childdoc]
%<*driver>
%\ProvidesFile{childdoc.drv}[2018/12/30 v2.0 childdoc reference manual file]
\PassOptionsToClass{10pt,a4paper}{article}
\documentclass{ltxdoc}

\usepackage[margin=35mm]{geometry}
\usepackage{hyperref}
\usepackage{hyperxmp}
\usepackage[usenames]{color}

\hypersetup{colorlinks=true}
\hypersetup{pdfstartview=FitH}
\hypersetup{pdfpagemode=UseNone}
\hypersetup{pdfsource={}}
\hypersetup{pdflang={en-UK}}
\hypersetup{pdfcopyright={Copyright 2017-2018 Niklas Beisert.
  This work may be distributed and/or modified under the
  conditions of the LaTeX Project Public License, either version 1.3
  of this license or (at your option) any later version.}}
\hypersetup{pdflicenseurl={http://www.latex-project.org/lppl.txt}}
\hypersetup{pdfcontactaddress={ETH Zurich, ITP, HIT K,
  Wolfgang-Pauli-Strasse 27}}
\hypersetup{pdfcontactpostcode={8093}}
\hypersetup{pdfcontactcity={Zurich}}
\hypersetup{pdfcontactcountry={Switzerland}}
\hypersetup{pdfcontactemail={nbeisert@itp.phys.ethz.ch}}
\hypersetup{pdfcontacturl={http://people.phys.ethz.ch/\xmptilde nbeisert/}}

\newcommand{\secref}[1]{\hyperref[#1]{section \ref*{#1}}}

\parskip1ex
\parindent0pt
\let\olditemize\itemize
\def\itemize{\olditemize\parskip0pt}

\begin{document}

\title{The \textsf{childdoc} Package}
\hypersetup{pdftitle={The childdoc Package}}
\author{Niklas Beisert\\[2ex]
  Institut f\"ur Theoretische Physik\\
  Eidgen\"ossische Technische Hochschule Z\"urich\\
  Wolfgang-Pauli-Strasse 27, 8093 Z\"urich, Switzerland\\[1ex]
  \href{mailto:nbeisert@itp.phys.ethz.ch}
  {\texttt{nbeisert@itp.phys.ethz.ch}}}
\hypersetup{pdfauthor={Niklas Beisert}}
\hypersetup{pdfsubject={Manual for the LaTeX2e Package childdoc}}
\date{30 December 2018, \textsf{v2.0}}
\maketitle

\begin{abstract}\noindent
\textsf{childdoc} is a \LaTeXe{} package
that enables the direct compilation
of document sections included by |\include|
to individual files.
\end{abstract}

\begingroup
\parskip0ex
\tableofcontents
\endgroup

%%%%%%%%%%%%%%%%%%%%%%%%%%%%%%%%%%%%%%%%%%%%%%%%%%%%%%%%%%%%%%%%%%%%%%%%%%%%%%%%
%%%%%%%%%%%%%%%%%%%%%%%%%%%%%%%%%%%%%%%%%%%%%%%%%%%%%%%%%%%%%%%%%%%%%%%%%%%%%%%%
\section{Introduction}

\LaTeX{} provides a mechanism to structure a large document (such as a book)
into a main file and several child files (containing the chapters)
using the |\include| command.
This mechanism is beneficial for documents
which span hundreds of pages in order to
make the source file(s) more manageable.
Moreover, compilation can be restricted to
selected child files by means of the |\includeonly| command.
The latter feature can be used to reduce the compilation time while editing
(this was significantly more useful in the earlier days of \LaTeX{})
or to generate a smaller document which is easier to navigate.
Another application of |\includeonly| is to generate
documents consisting of selected parts of the complete document.

However, there are a few drawbacks of the plain |\include| mechanism:
\begin{itemize}
\item
The child files cannot be compiled on their own,
they can only be compiled via the main file.
A naive editing environment
(such as a text editor with an option
to have the current file processed by \LaTeX)
may require one to switch to the main file before compiling;
attempting to compile the child file produces errors.
\item
The main file must be modified (each time)
to adjust the |\includeonly| command
to the present needs. This easily leaves the main file in a messy state.
\item
The generated document will always carry the filename
of the main document. This is inconvenient if
several child files are to be compiled and
to be kept for distribution.
\end{itemize}

The present package provides a simple interface
to make child files individually compilable by \LaTeX{}.
Compiling a child file then has the same effect as compiling
the main file with an |\includeonly| command
to select the appropriate child.
Moreover the generated document will carry the name of the child
rather than the main file.
This resolves all three above issues.

This feature is meant to make the editing of books,
thesis documents and lecture notes somewhat more convenient.
However, the package can also be used efficiently for
composing a series of documents (such as exercise sheets)
which are typically distributed individually.
It then assists the author in generating the individual documents
(potentially in different versions)
as well as a document containing the collected series.
Another application is in developing style files
or other kinds of included material
where compilation of the style file could redirect
to a sample or test file.

%%%%%%%%%%%%%%%%%%%%%%%%%%%%%%%%%%%%%%%%%%%%%%%%%%%%%%%%%%%%%%%%%%%%%%%%%%%%%%%%
%%%%%%%%%%%%%%%%%%%%%%%%%%%%%%%%%%%%%%%%%%%%%%%%%%%%%%%%%%%%%%%%%%%%%%%%%%%%%%%%
\section{Usage}

First of all, the package \textsf{childdoc} is \emph{not} a standard
\LaTeXe{} |.sty| style file! Therefore it needs to be invoked in
a non-standard way.

%%%%%%%%%%%%%%%%%%%%%%%%%%%%%%%%%%%%%%%%%%%%%%%%%%%%%%%%%%%%%%%%%%%%%%%%%%%%%%%%
\subsection{Included Files}
\label{sec:include}

%%%%%%%%%%%%%%%%%%%%%%%%%%%%%%%%%%%%%%%%
\DescribeMacro{\childdocmain}
To use the package, add the commands
\begin{center}
\begin{tabular}{l}
|\input{childdoc.def}|\\
|\childdocmain{}|\\
\end{tabular}
\end{center}
at the very top of the main \LaTeX{} file,
in particular \emph{before} the |\documentclass| statement!
The argument of |\childdocmain| should be left empty
(but it must be present).

%%%%%%%%%%%%%%%%%%%%%%%%%%%%%%%%%%%%%%%%
\DescribeMacro{\childdocof}
Furthermore, add the commands
\begin{center}
\begin{tabular}{l}
|\input{childdoc.def}|\\
|\childdocof{|\textit{main}|}|\\
\end{tabular}
\end{center}
at the top of every child file \textit{child}
which is included by |\include{|\textit{child}|}|
from within the main file
(or at least for those files to be compiled individually).
The argument \textit{main} must be the filename of the main file.

There are a couple of
considerations in setting up the main and child documents:

%%%%%%%%%%%%%%%%%%%%%%%%%%%%%%%%%%%%%%%%
\paragraph{Restrictions.}

Please note the following restrictions:
\begin{itemize}
\item
|\childdocmain| must be called with one argument \textit{main}
to ensure compatibility with earlier version of the package.
It must either be empty (|\childdocmain{}|)
or precisely match the filename of the main file in which it is specified.
See \secref{sec:detection} for further information.
\item
The filename \textit{main} must be specified without the |.tex| extension.
\item
The filename \textit{main} is case sensitive
(even in case-insensitive file systems)
due to internal string comparison.
\item
The argument \textit{main} should be fully expanded, it cannot be a macro.
\item
Subdirectories and special characters should be avoided in filenames.
\item
The command |\childdocmain{|\textit{main}|}| must be followed by a whitespace.
It should not be followed immediately by another command
or by a comment mark `|%|'.
This is because the \TeX{} parser reads the token immediately following
the argument of |\childdocmain| and puts it
at the beginning of every child section;
however, a white\-space is ignored.
\end{itemize}

%%%%%%%%%%%%%%%%%%%%%%%%%%%%%%%%%%%%%%%%
\paragraph{Content of Main File.}

It is advisable to place all content in the child files included by |\include|.
Any output contained in the main file will appear in all child documents
unless suppressed manually;
it cannot be suppressed automatically by the |\includeonly| directive
and thus should normally be avoided.
A method to include some content in the main file
by means of conditional processing is described in \secref{sec:conditional}.

%%%%%%%%%%%%%%%%%%%%%%%%%%%%%%%%%%%%%%%%
\paragraph{Page Numbering.}

When only a part of the document is compiled,
the appropriate numbering of pages
(as well as other status parameters)
is determined from the |.aux| files.
The latter contain information from previous passes.
However this information needs to propagate through
all intermediate child documents.
Therefore the page numbering in child documents may well
be inconsistent until the complete document is compiled at least once.

A useful (if unconventional) way to always ensure a consistent
page numbering is to restart the numbering in each child document
and denote the pages by `\textit{child}|.|\textit{page}'
where \textit{child} represents the chapter/section number of the child file.
This can be achieved by the command
|\numberwithin{page}{|\textit{child}|}|
of the \textsf{amsmath} package
where \textit{child} can be |chapter| or |section|
depending on the chosen structuring.
Alternatively, one can modify the macro |\thepage| appropriately
and reset the counter |page| at the start of each child file.

%%%%%%%%%%%%%%%%%%%%%%%%%%%%%%%%%%%%%%%%%%%%%%%%%%%%%%%%%%%%%%%%%%%%%%%%%%%%%%%%
\subsection{Conditional Processing}
\label{sec:conditional}

The package provides a mechanism to compile different versions
of a document. To customise the versions further some conditional processing
can come in handy to distinguish which version is being compiled.
The package provides two macros to describe the compilation context:

%%%%%%%%%%%%%%%%%%%%%%%%%%%%%%%%%%%%%%%%
\DescribeMacro{\ifchilddoc}
The conditional |\ifchilddoc| distinguishes between the compilation of
child documents and the main document:
%
\begin{center}
|\ifchilddoc |\textit{child-code}| |[|\||else |\textit{main-code}]| \||fi|
\end{center}

%%%%%%%%%%%%%%%%%%%%%%%%%%%%%%%%%%%%%%%%
\DescribeMacro{\childdocname}
\DescribeMacro{\childdocjob}
The macro |\childdocname| contains the filename (without extension)
of the main or child file being processed.
Note that |\childdocjob| will always contain the name of the main file.

%%%%%%%%%%%%%%%%%%%%%%%%%%%%%%%%%%%%%%%%
\paragraph{Title Page.}

Conditional processing can be used to include a title or banner page
in the main document when proper precautions are taken.
Importantly, the code in the main file should ensure that the page counter
(as well as other status parameters which are stored in the |.aux| files)
takes the same value after the conditional processing.
Otherwise the page numbers may take divergent values
depending on which part is compiled.

For example, a title page could be declared by:
%
\begin{center}
\begin{tabular}{l}
|\ifchilddoc\||else|\\
|\addtocounter{page}{-1}|\\
\textit{code for title page}\\
|\newpage|\\
|\||fi|
\end{tabular}
\end{center}
%
A banner page for the child documents can be generated by:
%
\begin{center}
\begin{tabular}{l}
|\ifchilddoc|\\
|\addtocounter{page}{-1}|\\
\textit{code for banner page}\\
|\newpage|\\
|\||fi|
\end{tabular}
\end{center}
%
Here one could write a message such as:
\begin{center}
|This is the part \childdocname{} of \childdocjob{}.|
\end{center}

%%%%%%%%%%%%%%%%%%%%%%%%%%%%%%%%%%%%%%%%%%%%%%%%%%%%%%%%%%%%%%%%%%%%%%%%%%%%%%%%
\subsection{Flags}
\label{sec:flags}

The package makes it easy to generate different versions
of the main or child documents.
To this end compilation flags can be defined
and assigned different default values.
They will be particularly useful in conjunction
with the forwarding mechanism described in \secref{sec:forward}.

For example, it may be useful to have a flag |\version|
which can be set to |draft| or |final|.
The document source will contain some conditional code
depending on the value of |\version|.
Suppose further, the flag should default to |final| for the main file
and to |draft| for child files
which is a natural assignment for editing the document.
This is achieved by placing the following code
in the preamble of the main document
(below the |\childdocmain| directive):
%
\begin{center}
\begin{tabular}{l}
|\ifchilddoc|\\
|\providecommand{\version}{draft}|\\
|\||else|\\
|\providecommand{\version}{final}|\\
|\||fi|
\end{tabular}
\end{center}
%
The definition by |\providecommand| makes sure
that previous definitions are not overwritten.
Further statements |\providecommand{\version}{...}|
can thus be added before the above code to override it.

For the main file, one might add a line
(between |\childdocmain| and the above block)
%
\begin{center}
|%\ifchilddoc\||else\providecommand{\version}{draft}\||fi|
\end{center}
%
which can be uncommented to produce a draft version.
Likewise one can add a line to the very top of a child file
(above the |\childdocof{|\textit{main}|}| directive)
%
\begin{center}
|%\providecommand{\version}{final}|
\end{center}
%
which can be uncommented to produce the final version of this child document.

%%%%%%%%%%%%%%%%%%%%%%%%%%%%%%%%%%%%%%%%%%%%%%%%%%%%%%%%%%%%%%%%%%%%%%%%%%%%%%%%
\subsection{Forwarding}
\label{sec:forward}

Different versions of the main or child documents
using compilation flags as described in \secref{sec:flags}
can be (permanently) stored in different files
for convenient compilation, viewing and distribution.
To this end, the package defines a command
to pass on compilation to a different file:

%%%%%%%%%%%%%%%%%%%%%%%%%%%%%%%%%%%%%%%%
\DescribeMacro{\childdocforward}
The command |\childdocforward| redirects processing to
another source file:
%
\begin{center}
\begin{tabular}{l}
|\input{childdoc.def}|\\
|\childdocforward[|\textit{main}|]{|\textit{dest}|}|\\
\end{tabular}
\end{center}
%
The argument \textit{dest} is the destination file
(without extension).
It should be the main file or one of the child files.
Note that further \textsf{childdoc} directives
such as |\childdocof| and |\childdocforward|
in the indicated file will be processed in this form.
The optional argument \textit{main}
passes on directly to the main file \textit{main}
while pretending to compile the child \textit{dest}.
This form behaves as if \textit{dest}
issues |\childdocof{|\textit{main}|}| right away,
and no further \textsf{childdoc} directives will be processed.

%%%%%%%%%%%%%%%%%%%%%%%%%%%%%%%%%%%%%%%%
\DescribeMacro{\...prefix}
In the alternative form |\childdocforwardprefix|,
%
\begin{center}
\begin{tabular}{l}
|\input{childdoc.def}|\\
|\childdocforwardprefix[|\textit{main}|]{|\textit{prefix}|}{|\textit{dest}|}|
\end{tabular}
\end{center}
%
the destination file is determined by a pattern
depending on the current file:
To make this work, the current file must be called
`{\textit{prefix}\hspace{0.2em}\textit{suffix}}'
with \textit{prefix} matching precisely the argument.
Processing is then passed on to the file
`{\textit{dest}\hspace{0.2em}\textit{suffix}}'.
Surely, the same effect is achieved by
directly specifying the
argument `{\textit{dest}\hspace{0.2em}\textit{suffix}}'
in the first form.
However, that requires to set up a different file
for each child. With the alternative form of the command
all these files can have exactly the same content
which simplifies setting them up and maintaining them.

For example, the following file |draft.tex|
with a compilation flag |\version| as described in \secref{sec:flags}
compiles the main document as a draft:
%
\begin{center}
\begin{tabular}{l}
|\def\version{draft}|\\
|\input{childdoc.def}|\\
|\childdocforward{|\textit{main}|}|
\end{tabular}
\end{center}
%
Likewise, the following files |final|\textit{nn}|.tex|
compile the final version of the child document
|child|\textit{nn}|.tex|:
%
\begin{center}
\begin{tabular}{l}
|\def\version{final}|\\
|\input{childdoc.def}|\\
|\childdocforwardprefix{final}{child}|
\end{tabular}
\end{center}
%

Note that when several versions of a main file and/or of each child file
are to be generated, it may be convenient to set up a |Makefile| or
shell script to automatise the process.

%%%%%%%%%%%%%%%%%%%%%%%%%%%%%%%%%%%%%%%%%%%%%%%%%%%%%%%%%%%%%%%%%%%%%%%%%%%%%%%%
\subsection{Command Line Processing}
\label{sec:commandline}

The effect of redirection files can also be achieved by invoking
the \LaTeX{} compiler with a more elaborate command line.
Most conveniently this should be done as part
of a shell script or a |Makefile|.

When using \textsf{childdoc} in the main file, the following
command lines effectively perform a redirection
(note that depending on the shell being used,
backslashes may have to be doubled: `|\|' $\to$ `|\\|'):
%
\begin{center}
|... -jobname "|\textit{target}|" |\\|"|[\textit{flags}]%
|\input{childdoc.def}\childdocforward[|\textit{main}|]{|\textit{dest}|}"|
\end{center}
%
Here \textit{target} is the name of the output file,
\textit{main} is the name of the main file
and \textit{dest} is the name of the main or child file to be processed
(all filenames without extensions).
The optional argument \textit{main} can be omitted
if \textit{main} matches \textit{dest}.
Optionally, compilation \textit{flags} can be defined via |\def| commands.
This command line makes the \TeX{} engine believe
it is compiling the file \textit{target}
whose content is specified as the latter parameter.
The provided code then forwards the processing to
\textit{main} or \textit{dest} as described in \secref{sec:forward}.

%%%%%%%%%%%%%%%%%%%%%%%%%%%%%%%%%%%%%%%%%%%%%%%%%%%%%%%%%%%%%%%%%%%%%%%%%%%%%%%%
\subsection{Include by Input}
\label{sec:input}

Including child documents by |\include| has some restrictions by design.
Most notably, the content of a child document always occupies
its own set of pages; pages cannot be shared between child documents.
Usually, this behaviour makes perfect sense
because each child document contain an essential part of the document.
However, in some situations it may be desirable to compose
a document from a collection of parts
without having mandatory page breaks between then.
For this case, the package
provides a mechanism to include parts
by |\input| which can also be processed individually.
However, by construction this mechanism
requires manual handling of the content to be output.

%%%%%%%%%%%%%%%%%%%%%%%%%%%%%%%%%%%%%%%%
\DescribeMacro{\ifchilddocmanual}
The main file should be prepared as usual, see \secref{sec:include}.
However, the document body must make a distinction
between processing of an individual part and of the main document, e.g.:
%
\begin{center}
\begin{tabular}{l}
|\ifchilddocmanual|\\
|\input{\childdocname}|\\
|\||else|\\
\textit{document body with }|\input{|\textit{part}|}|\\
|\||fi|
\end{tabular}
\end{center}
%
The conditional |\ifchilddocmanual| is true whenever
a part to be included by |\input| is being compiled,
and the name of the part is stored in |\childdocname|.

%%%%%%%%%%%%%%%%%%%%%%%%%%%%%%%%%%%%%%%%
\DescribeMacro{\childdocby}
Each part to be included by |\input| should start with:
%
\begin{center}
\begin{tabular}{l}
|\input{childdoc.def}|\\
|\childdocby{|\textit{main}|}|\\
\end{tabular}
\end{center}
%
The directive |\childdocby| is similar to |\childdocof|
described in \secref{sec:include},
but the subsequent selection of content must be done manually.
To that end, both |\ifchilddoc| and |\ifchilddocmanual|
will be true upon processing of a part,
and the name of the part is stored in |\childdocname|.
Note that |\jobname| will be set to the filename of the current part
so that each part receives an individual |.aux| file
that does not interfere with the |.aux| file(s) of the main document.
This behaviour can be altered by the alternative form
|\childdocby[*]{|\textit{main}|}| (with a non-empty optional argument)
which uses the |.aux| file of the main document
by setting |\jobname| to \textit{main}.

%%%%%%%%%%%%%%%%%%%%%%%%%%%%%%%%%%%%%%%%%%%%%%%%%%%%%%%%%%%%%%%%%%%%%%%%%%%%%%%%
\subsection{Driver Development}
\label{sec:driver}

The \textsf{childdoc} mechanism can also be use for the development
of definition files such as \LaTeX{} styles or classes.
This case differs from the above setup with multiple parts
included by |\include| in that no |\includeonly| should be invoked.
This can be achieved by starting the include file
(before |\ProvidesPackage|) with:
%
\begin{center}
\begin{tabular}{l}
|\input{childdoc.def}|\\
|\childdocforward{|\textit{main}|}|\\
\end{tabular}
\end{center}
%
or alternatively with:
%
\begin{center}
\begin{tabular}{l}
|\input{childdoc.def}|\\
|\childdocby{|\textit{main}|}|\\
\end{tabular}
\end{center}
%
Both forms have slightly different effects as described above.
The main file is prepared as usual, see \secref{sec:include}.

%%%%%%%%%%%%%%%%%%%%%%%%%%%%%%%%%%%%%%%%%%%%%%%%%%%%%%%%%%%%%%%%%%%%%%%%%%%%%%%%
\subsection{Legacy Detection}
\label{sec:detection}

The directive |\childdocmain| in the main file can detect
whether the complete document or merely a child is to be compiled
even without using the directive |\childdocof|.
This method is deprecated because it is less robust
and there is no compelling reason to use it;
it is merely provided for backward compatibility
and it may be removed in future versions.

If the detection mechanism is to be used,
it is mandatory to correctly specify
the filename of the main file as the argument of |\childdocmain|:
%
\begin{center}
\begin{tabular}{l}
|\input{childdoc.def}|\\
|\childdocmain{|\textit{main}|}|\\
\end{tabular}
\end{center}
%
If |\jobname| does not match the argument \textit{main} of |\childdocmain|,
it is assumed that |\jobname| points to the child file to be compiled.
When using |\childdocmain| with the main file specified as argument,
it suffices to start a child file
with just |\input{|\textit{main}|}|
without loading of the package and using |\childdocof|.
If instead all processing is done
with the appropriate \textsf{childdoc} directives,
the argument of \textit{main} of |\childdocmain| can be empty.

An alternative version of the command line processing described
in \secref{sec:commandline} using the detection mechanism reads:
%
\begin{center}
|... -jobname "|\textit{target}|" "|[\textit{flags}]%
[|\def\jobname{|\textit{dest}|}|]|\input{|\textit{main}|}"|
\end{center}

%%%%%%%%%%%%%%%%%%%%%%%%%%%%%%%%%%%%%%%%%%%%%%%%%%%%%%%%%%%%%%%%%%%%%%%%%%%%%%%%
\subsection{Manual Code}
\label{sec:manual}

In case one cannot be certain whether the definitions file |childdoc.def|
is installed on the target \TeX{} distribution
and one prefers not to ship it,
it is conceivable to paste a few relevant commands into the sources.

To that end, drop all statements |\input{childdoc.def}|
and perform the replacements as outlined below.
Instead of |\childdocmain{|\textit{main}|}| add the following code
to the top of the main file:
%
\begin{center}
\begin{tabular}{l}
|\||ifdefined\childdocname\endinput\||fi\newif\ifchilddoc|\\
|\edef\childdocname{\scantokens\expandafter{\jobname\noexpand}}|\\
|\def\childdocmain{|\textit{main}|}\||ifx\childdocmain\childdocname\||else|\\
|\childdoctrue\includeonly{\childdocname}\let\jobname\childdocmain\||fi|\\
\end{tabular}
\end{center}
%
Instead of |\childdocof{|\textit{main}|}| just include the main file
at the top of each child file:
%
\begin{center}
|\input{|\textit{main}|}|
\end{center}
%
A simple redirection |\childdocforward{|\textit{dest}|}| is achieved by:
%
\begin{center}
|\def\jobname{|\textit{dest}|}\input{\jobname}|
\end{center}
%
The redirection with prefix
|\childdocforwardprefix[|\textit{prefix}|]{|\textit{dest}|}|
is accomplished by:
%
\begin{center}
\begin{tabular}{l}
|{\edef\jobname{\scantokens\expandafter{\jobname\noexpand}}|\\
|\def\redirectjob |\textit{prefix}|#1~~~{\gdef\jobname{|\textit{dest}|#1}}|\\
|\expandafter\redirectjob\jobname~~~}\input{\jobname}|
\end{tabular}
\end{center}

In an alternative approach,
child documents can be compiled by a specific command line
without additional code or specific definitions:
%
\begin{center}
|... -jobname "|\textit{target}|" "|[\textit{flags}]%
|\includeonly{|\textit{dest}|}\input{|\textit{main}|}"|
\end{center}
%

%%%%%%%%%%%%%%%%%%%%%%%%%%%%%%%%%%%%%%%%%%%%%%%%%%%%%%%%%%%%%%%%%%%%%%%%%%%%%%%%
%%%%%%%%%%%%%%%%%%%%%%%%%%%%%%%%%%%%%%%%%%%%%%%%%%%%%%%%%%%%%%%%%%%%%%%%%%%%%%%%
\section{Information}

%%%%%%%%%%%%%%%%%%%%%%%%%%%%%%%%%%%%%%%%%%%%%%%%%%%%%%%%%%%%%%%%%%%%%%%%%%%%%%%%
\subsection{Copyright}

Copyright \copyright{} 2017--2018 Niklas Beisert

This work may be distributed and/or modified under the
conditions of the \LaTeX{} Project Public License, either version 1.3
of this license or (at your option) any later version.
The latest version of this license is in
  \url{http://www.latex-project.org/lppl.txt}
and version 1.3 or later is part of all distributions of \LaTeX{}
version 2005/12/01 or later.

This work has the LPPL maintenance status `maintained'.

The Current Maintainer of this work is Niklas Beisert.

This work consists of the files |README.txt|, |childdoc.ins| and |childdoc.dtx|
as well as the derived files |childdoc.def|, |cdocsamp.tex|
with |cdocsch1.tex|, |cdocsch2.tex|, |cdocspt3.tex|, |cdocspt4.tex|,
|cdocsdrf.tex|, |cdocsfn1.tex|, |cdocsfn2.tex|
as well as |childdoc.pdf|.

%%%%%%%%%%%%%%%%%%%%%%%%%%%%%%%%%%%%%%%%%%%%%%%%%%%%%%%%%%%%%%%%%%%%%%%%%%%%%%%%
\subsection{Files and Installation}

The package consists of the files:
%
\begin{center}
\begin{tabular}{ll}
    |README.txt|   & readme file \\
    |childdoc.ins| & installation file \\
    |childdoc.dtx| & source file \\
    |childdoc.def| & definition file \\
    |cdocsamp.tex| & sample main file \\
    |cdocsch1.tex| & sample include file \\
    |cdocsch2.tex| & sample include file \\
    |cdocspt3.tex| & sample part file \\
    |cdocspt4.tex| & sample part file \\
    |cdocsdrf.tex| & sample redirection file \\
    |cdocsfn1.tex| & sample redirection file \\
    |cdocsfn2.tex| & sample redirection file \\
    |childdoc.pdf| & manual
\end{tabular}
\end{center}
%
The distribution consists of the files
|README.txt|, |childdoc.ins| and |childdoc.dtx|.
%
\begin{itemize}
\item
Run (pdf)\LaTeX{} on |childdoc.dtx|
to compile the manual |childdoc.pdf| (this file).
\item
Run \LaTeX{} on |childdoc.ins| to create the definitions file |childdoc.def|
and the sample |cdocsamp.tex| with include files
|cdocsch1.tex|, |cdocsch2.tex|, |cdocspt3.tex|, |cdocspt4.tex|,
|cdocsdrf.tex|, |cdocsfn1.tex|, |cdocsfn2.tex|.
Then copy the file |childdoc.def| to an appropriate directory of your \LaTeX{}
distribution, e.g.\ \textit{texmf-root}|/tex/latex/childdoc|.
\end{itemize}

%%%%%%%%%%%%%%%%%%%%%%%%%%%%%%%%%%%%%%%%%%%%%%%%%%%%%%%%%%%%%%%%%%%%%%%%%%%%%%%%
\subsection{Related CTAN Packages}

There are several other packages which offer a similar functionality:
%
\begin{itemize}
\item
The packages
\href{http://ctan.org/pkg/docmute}{\textsf{docmute}},
\href{http://ctan.org/pkg/includex}{\textsf{includex}} and
\href{http://ctan.org/pkg/standalone}{\textsf{standalone}}
provide commands to include only the document body of
a child file thus allowing both files to be compiled individually.
\item
The packages \href{http://ctan.org/pkg/subdocs}{\textsf{subdocs}}
and \href{http://ctan.org/pkg/subfiles}{\textsf{subfiles}}
provide structures in which the main and child documents can be
encapsulated and allowing them to be compiled individually.
The inclusion mechanism is different from the conventional |\include|.
\item
The package \href{http://ctan.org/pkg/combine}{\textsf{combine}}
is an elaborate solution to combine several documents into one.
\end{itemize}
%
See also the CTAN topic \href{http://ctan.org/topic/subdocs}{\textsf{subdocs}}
for further related packages.
The present package differs from the above solutions in that
a document structure constructed with the conventional |\include| mechanism
just needs two extra commands at the top of every file
such that all constituent files can be compiled individually.

%%%%%%%%%%%%%%%%%%%%%%%%%%%%%%%%%%%%%%%%%%%%%%%%%%%%%%%%%%%%%%%%%%%%%%%%%%%%%%%%
%\subsection{Feature Suggestions}
%
%The following is a list of features which may be useful for future
%versions of this package:
%%
%\begin{itemize}
%\item
%\ldots
%\end{itemize}

%%%%%%%%%%%%%%%%%%%%%%%%%%%%%%%%%%%%%%%%%%%%%%%%%%%%%%%%%%%%%%%%%%%%%%%%%%%%%%%%
\subsection{Revision History}

%%%%%%%%%%%%%%%%%%%%%%%%%%%%%%%%%%%%%%%%
\paragraph{v2.0:} 2018/12/30

\begin{itemize}
\item
immediate forward processing
\item
added |\childdocby| mechanism
\item
manual restructured
\end{itemize}

%%%%%%%%%%%%%%%%%%%%%%%%%%%%%%%%%%%%%%%%
\paragraph{v1.6:} 2018/01/17

\begin{itemize}
\item
application for development of include files
\item
corrections to manual
\end{itemize}

%%%%%%%%%%%%%%%%%%%%%%%%%%%%%%%%%%%%%%%%
\paragraph{v1.5:} 2017/05/21

\begin{itemize}
\item
more complete structuring introduced
\item
|\childdocof| introduced
\item
|\childdoc| renamed to |\childdocmain|
\item
|\childredirect| renamed to |\childdocforward| and |\childdocforwardprefix|
and functionality expanded
\end{itemize}

%%%%%%%%%%%%%%%%%%%%%%%%%%%%%%%%%%%%%%%%
\paragraph{v1.0:} 2017/04/27

\begin{itemize}
\item
manual and install package
\item
first version published on CTAN
\end{itemize}

%%%%%%%%%%%%%%%%%%%%%%%%%%%%%%%%%%%%%%%%
\paragraph{v0.6:} 2017/04/26

\begin{itemize}
\item
redirection mechanism added
\end{itemize}

%%%%%%%%%%%%%%%%%%%%%%%%%%%%%%%%%%%%%%%%
\paragraph{v0.5:} 2017/04/26

\begin{itemize}
\item
functionality in definition file
\end{itemize}


%%%%%%%%%%%%%%%%%%%%%%%%%%%%%%%%%%%%%%%%%%%%%%%%%%%%%%%%%%%%%%%%%%%%%%%%%%%%%%%%
%%%%%%%%%%%%%%%%%%%%%%%%%%%%%%%%%%%%%%%%%%%%%%%%%%%%%%%%%%%%%%%%%%%%%%%%%%%%%%%%
%%%%%%%%%%%%%%%%%%%%%%%%%%%%%%%%%%%%%%%%%%%%%%%%%%%%%%%%%%%%%%%%%%%%%%%%%%%%%%%%
\appendix

\settowidth\MacroIndent{\rmfamily\scriptsize 000\ }

 \DocInput{childdoc.dtx}

\end{document}
%</driver>
% \fi
%
% %%%%%%%%%%%%%%%%%%%%%%%%%%%%%%%%%%%%%%%%%%%%%%%%%%%%%%%%%%%%%%%%%%%%%%%%%%%%%%
% %%%%%%%%%%%%%%%%%%%%%%%%%%%%%%%%%%%%%%%%%%%%%%%%%%%%%%%%%%%%%%%%%%%%%%%%%%%%%%
% \section{Sample}
%\iffalse
%<*samplemain>
%\fi
%
% The following presents a sample document
% with two chapters, two parts, a title page,
% a compile flag as well as three forwarding files to set the flag.
% It consists of eight |.tex| files:
% \begin{center}
% \begin{tabular}{ll}
% |cdocsamp.tex|&main file\\
% |cdocsch1.tex|&include file for chapter 1\\
% |cdocsch2.tex|&include file for chapter 2\\
% |cdocspt3.tex|&include file for part 3\\
% |cdocspt4.tex|&include file for part 4\\
% |cdocsdrf.tex|&forwarding file for main file in draft mode\\
% |cdocsfi1.tex|&forwarding file for final version of chapter 1\\
% |cdocsfi2.tex|&forwarding file for final version of chapter 2\\
% \end{tabular}
% \end{center}
% Each of the eight files can be compiled directly by the \LaTeX{} compiler.
%
% %%%%%%%%%%%%%%%%%%%%%%%%%%%%%%%%%%%%%%
% \paragraph{Main File.}
%
% The main file is called |cdocsamp.tex|.
%
% Load the \textsf{childdoc} definitions and
% declare the filename for the main document:
%    \begin{macrocode}
\input{childdoc.def}
\childdocmain{}
%    \end{macrocode}

% Optional override for |\version| flag:
%    \begin{macrocode}
%%\ifchilddoc\else\providecommand{\version}{draft}\fi
%    \end{macrocode}

% Define the default values for the |\version| flag
% (|final| for the main file and |draft| for childs):
%    \begin{macrocode}
\ifchilddoc
\providecommand{\version}{draft}
\else
\providecommand{\version}{final}
\fi
%    \end{macrocode}

% Load the standard document class:
%    \begin{macrocode}
\documentclass[12pt]{article}
%    \end{macrocode}

% Start the document body:
%    \begin{macrocode}
\begin{document}
%    \end{macrocode}

% Declare a title page.
% Print title, part of document being processed and version flag:
%    \begin{macrocode}
\addtocounter{page}{-1}
\begin{center}
{\LARGE\bfseries{}childdoc example\par}
\vspace{1cm}
\ifchilddoc
\ifchilddocmanual part\else chapter\fi:
`\childdocname' of `\childdocjob'\par
\else
main document: `\childdocjob'\par
\fi
version: \version\par
\end{center}
\newpage
%    \end{macrocode}

% Manually include selected file,
% otherwise process as usual:
%    \begin{macrocode}
\ifchilddocmanual
\section*{part `\childdocname'}
\input{\childdocname}
\else
%    \end{macrocode}

% Include the two chapters:
%    \begin{macrocode}
\include{cdocsch1}
\include{cdocsch2}
%    \end{macrocode}

% Include the two parts unless only chapters should be displayed:
%    \begin{macrocode}
\ifchilddoc\else
\section{part three}
\input{cdocspt3}
\section{part four}
\input{cdocspt4}
\fi
%    \end{macrocode}

% Process as usual until here:
%    \begin{macrocode}
\fi
%    \end{macrocode}

% End of document body:
%    \begin{macrocode}
\end{document}
%    \end{macrocode}
%\iffalse
%</samplemain>
%\fi
%
% %%%%%%%%%%%%%%%%%%%%%%%%%%%%%%%%%%%%%%
% \paragraph{Chapter Include Files.}
%
% The include files are called |cdocsch1.tex| and |cdocsch2.tex|.
%
%\iffalse
%<*samplechap1|samplechap2>
%\fi

% Optional override for |\version| flag:
%    \begin{macrocode}
%%\providecommand{\version}{final}
%    \end{macrocode}

% Include the main document:
%    \begin{macrocode}
\input{childdoc.def}
\childdocof{cdocsamp}
%    \end{macrocode}

%\iffalse
%</samplechap1|samplechap2>
%\fi
%
%\iffalse
%<*samplechap1>
%\fi
% Some text for chapter 1:
%    \begin{macrocode}
\section{one}
some text in chapter one
%    \end{macrocode}

%\iffalse
%</samplechap1>
%\fi
% Some text for chapter 2:
%\iffalse
%<*samplechap2>
%\fi
%    \begin{macrocode}
\section{two}
more text in chapter two
%    \end{macrocode}

%\iffalse
%</samplechap2>
%\fi
%
% %%%%%%%%%%%%%%%%%%%%%%%%%%%%%%%%%%%%%%
% \paragraph{Part Include Files.}
%
% The include files are called |cdocspt3.tex| and |cdocspt4.tex|.
%
%\iffalse
%<*samplepart3|samplepart4>
%\fi

% Optional override for |\version| flag:
%    \begin{macrocode}
%%\providecommand{\version}{final}
%    \end{macrocode}

% Include the main document:
%    \begin{macrocode}
\input{childdoc.def}
\childdocby{cdocsamp}
%    \end{macrocode}

%\iffalse
%</samplepart3|samplepart4>
%\fi
%
%\iffalse
%<*samplepart3>
%\fi
% Some text for part 3:
%    \begin{macrocode}
some text in part three
%    \end{macrocode}

%\iffalse
%</samplepart3>
%\fi
% Some text for part 4:
%\iffalse
%<*samplepart4>
%\fi
%    \begin{macrocode}
more text in part four
%    \end{macrocode}

%\iffalse
%</samplepart4>
%\fi
%
% %%%%%%%%%%%%%%%%%%%%%%%%%%%%%%%%%%%%%%
% \paragraph{Forwarding for a Complete Draft.}
%
% The following forwarding file |cdocsdrf.tex|
% compiles the main document in draft mode:
%\iffalse
%<*sampledraft>
%\fi
%    \begin{macrocode}
\def\version{draft}
\input{childdoc.def}
\childdocforward{cdocsamp}
%    \end{macrocode}

%\iffalse
%</sampledraft>
%\fi
%
% %%%%%%%%%%%%%%%%%%%%%%%%%%%%%%%%%%%%%%
% \paragraph{Forwarding for Final Version of the Chapters.}
%
% The following forwarding files |cdocsfn1.tex| and |cdocsfn2.tex|
% (with identical content)
% compile the final versions of the child documents
% |cdocsch1.tex| and |cdocsch2.tex|, respectively:
%\iffalse
%<*samplefinal>
%\fi
%    \begin{macrocode}
\def\version{final}
\input{childdoc.def}
\childdocforwardprefix[cdocsamp]{cdocsfn}{cdocsch}
%    \end{macrocode}

%\iffalse
%</samplefinal>
%\fi
%
% %%%%%%%%%%%%%%%%%%%%%%%%%%%%%%%%%%%%%%
% \paragraph{Command Line Processing.}
%
% The following three command lines generate the output files
% |cdocscld|, |cdocscl1| and |cdocscl2|
% which should be identical to
% |cdocsdrf|, |cdocsch1| and |cdocsfn2|, respectively:
% \begin{center}
% \begin{tabular}{l}
% |latex -jobname cdocscld \|\\
% |  "\def\version{draft}\input{childdoc.def}\childdocforward{cdocsamp}"|\\
% |latex -jobname cdocscl1 \|\\
% |  "\input{childdoc.def}\childdocforward[cdocsamp]{cdocsch1}"|\\
% |latex -jobname cdocscl2 \|\\
% |  "\def\version{final}\input{childdoc.def}\childdocforward{cdocsch2}"|
% \end{tabular}
% \end{center}
% Note that the trailing backslash on each first line
% merely continues the input to the second line
% (for convenient cut ant paste).
% Furthermore, the command |latex| can be replaced by any
% of its alternative versions such as |pdflatex|.
%
% %%%%%%%%%%%%%%%%%%%%%%%%%%%%%%%%%%%%%%%%%%%%%%%%%%%%%%%%%%%%%%%%%%%%%%%%%%%%%%
% %%%%%%%%%%%%%%%%%%%%%%%%%%%%%%%%%%%%%%%%%%%%%%%%%%%%%%%%%%%%%%%%%%%%%%%%%%%%%%
% \section{Implementation}
%\iffalse
%<*package>
%\fi
%
% This section describes the definitions file |childdoc.def|.

% The definitions cannot be loaded using |\usepackage| or |\RequirePackage|
% which has a mechanism to prevent loading a style file more than once.
% When loading the definitions by means of |\input|
% multiple instances have to be prevented manually:
%\iffalse
%This code needs to be before the `\ProvidesFile' directive
%which is defined at the beginning of this file.
%Therefore it is also placed there and commented out here.
%</package>
%<*discard>
%\fi
%    \begin{macrocode}
\ifdefined\childdocmain\endinput\fi
%    \end{macrocode}
%\iffalse
%</discard>
%<*package>
%\fi
%
% \macro{\ifchilddoc}
% \macro{\ifchilddocmanual}
% The conditional |\ifchilddoc| tells whether a
% child (true) or main (false) document is being compiled.
% The conditional |\ifchilddocmanual| tells whether
% the |\includeonly| mechanism is used (false) or
% the selection of child files must be performed manually (true).
% The definitions initialise to false:
%    \begin{macrocode}
\newif\ifchilddoc
\newif\ifchilddocmanual
%    \end{macrocode}

% \macro{\childdocname}
% \macro{\childdocjob}
% The macro |\childdocname| stores the name of the main document
% to be compiled. The macro |\childdocjob| stores the name of
% the document on which the \LaTeX{} compiler was originally invoked.
% The content of |\jobname| cannot be compared
% to filenames specified in the source due to different catcodes.
% The following code rescans |\jobname|, stores the result
% in |\childdocname| and saves a copy in |\childdocjob|:
%    \begin{macrocode}
\edef\childdocname{\scantokens\expandafter{\jobname\noexpand}}
\let\childdocjob\childdocname
%    \end{macrocode}

% \macro{\childdocdisable}
% The macro |\childdocdisable| prevents the main file
% from being processed more than once.
% At this stage, the main document command |\childdocmain|
% is assumed to be called once again where it should do nothing.
% Any subsequent call to it should prevent
% a secondary processing of the main document
% It overwrites the forwarding commands
% |\childdocof| and |\childdocforward|
% with empty macros to prevent further inclusions of the main document:
%    \begin{macrocode}
\newcommand{\childdocdisable}
{
  \renewcommand{\childdocmain}[1]{\renewcommand{\childdocmain}[1]{\endinput}}
  \renewcommand{\childdocof}[1]{}
  \renewcommand{\childdocby}[2][]{}
  \renewcommand{\childdocforward}[2][]{}
  \renewcommand{\childdocdisable}{}
}
%    \end{macrocode}

% \macro{\childdocmain}
% The macro |\childdocmain| is to be called at the top of the main file
% with nothing or the main filename (without extension) as argument.
% First, it breaks loops.
% If the argument is not empty and does not match |\childdocname|
% (which is set by the first inclusion of |childdoc.def|),
% |\ifchilddoc| is set to true, |\includeonly| is applied to the child file
% and |\jobname| is set to the main file
% (for proper handling of |.aux| files):
%    \begin{macrocode}
\newcommand{\childdocmain}[1]
{
  \childdocdisable\childdocmain{}
  \if?#1?\else
    \begingroup
      \def\childdoctmp{#1}
      \ifx\childdoctmp\childdocname
        \def\childdoctmp{}
      \else
        \def\childdoctmp
        {
          \childdoctrue
          \includeonly{\childdocname}
          \def\childdocjob{#1}
          \def\jobname{#1}
        }
      \fi
      \expandafter
    \endgroup
    \childdoctmp
  \fi
}
%    \end{macrocode}

% \macro{\childdocof}
% The command |\childdocof| redirects
% compilation to the main file |#1|.
%    \begin{macrocode}
\newcommand{\childdocof}[1]
{
  \childdocdisable
  \childdoctrue
  \includeonly{\childdocname}
  \def\jobname{#1}
  \def\childdocjob{#1}
  \input{#1}
}
%    \end{macrocode}

% \macro{\childdocby}
% The command |\childdocby| ....
%    \begin{macrocode}
\newcommand{\childdocby}[2][]
{
  \childdocdisable
  \childdoctrue
  \childdocmanualtrue
  \if?#1?\else
    \def\jobname{#2}
  \fi
  \def\childdocjob{#2}
  \input{#2}
  \endinput
}
%    \end{macrocode}

% \macro{\childdocforward}
% The command |\childdocforward| redirects
% compilation to the main file or
% (if the optional argument is given) a child file.
% Parameters are set as if the main file
% or a child file starting with |\childdocof| was compiled.
% Then compilation is handed over to the main file:
%    \begin{macrocode}
\newcommand{\childdocforward}[2][]
{
  \begingroup
    \if?#1?
      \def\childdoctmp
      {
        \def\childdocname{#2}
        \def\childdocjob{#2}
        \def\jobname{#2}
        \input{#2}
        \endinput
      }
    \else
      \def\childdoctmp
      {
        \childdocdisable
        \def\childdocname{#2}
        \childdoctrue
        \includeonly{#2}
        \def\childdocjob{#1}
        \def\jobname{#1}
        \input{#1}
        \endinput
      }
    \fi
    \expandafter
  \endgroup
  \childdoctmp
}
%    \end{macrocode}

% \macro{\childdocforwardprefix}
% The command |\childdocforwardprefix| redirects
% compilation to the main or a child file by means of a pattern.
% The prefix |#1| in the current filename is replaced by |#2|
% and the suffix of the current filename is kept
% (it is assumed that the filename does not contain the substring `|~~~|'
% which is used as a delimiter).
% Compilation is handed over to the new file by |\childdocforward|:
%    \begin{macrocode}
\newcommand{\childdocforwardprefix}[3][]
{
  \begingroup
    \def\childdocextract #2##1~~~{\def\childdoctmp{\childdocforward[#1]{#3##1}}}
    \expandafter\childdocextract\childdocname~~~
    \expandafter
  \endgroup
  \childdoctmp
}
%    \end{macrocode}

% \macro{\childdoc}
% The deprecated macro |\childdoc| is a legacy version of |\childdocmain|:
%    \begin{macrocode}
\newcommand{\childdoc}{\childdocmain}
%    \end{macrocode}

% \macro{\childdocredirect}
% The deprecated macro |\childdocredirect| is a legacy version
% of |\childdocforward| and |\childdocforwardprefix|:
%    \begin{macrocode}
\newcommand{\childdocredirect}[2][]
{
  \begingroup
    \if?#1?
      \def\childdoctmp{\childdocforward{#2}}
    \else
      \def\childdoctmp{\childdocforwardprefix{#1}{#2}}
    \fi
    \expandafter
  \endgroup
  \childdoctmp
}
%    \end{macrocode}

%\iffalse
%</package>
%\fi
%
\endinput
|
and perform the replacements as outlined below.
Instead of |\childdocmain{|\textit{main}|}| add the following code
to the top of the main file:
%
\begin{center}
\begin{tabular}{l}
|\||ifdefined\childdocname\endinput\||fi\newif\ifchilddoc|\\
|\edef\childdocname{\scantokens\expandafter{\jobname\noexpand}}|\\
|\def\childdocmain{|\textit{main}|}\||ifx\childdocmain\childdocname\||else|\\
|\childdoctrue\includeonly{\childdocname}\let\jobname\childdocmain\||fi|\\
\end{tabular}
\end{center}
%
Instead of |\childdocof{|\textit{main}|}| just include the main file
at the top of each child file:
%
\begin{center}
|\input{|\textit{main}|}|
\end{center}
%
A simple redirection |\childdocforward{|\textit{dest}|}| is achieved by:
%
\begin{center}
|\def\jobname{|\textit{dest}|}\input{\jobname}|
\end{center}
%
The redirection with prefix
|\childdocforwardprefix[|\textit{prefix}|]{|\textit{dest}|}|
is accomplished by:
%
\begin{center}
\begin{tabular}{l}
|{\edef\jobname{\scantokens\expandafter{\jobname\noexpand}}|\\
|\def\redirectjob |\textit{prefix}|#1~~~{\gdef\jobname{|\textit{dest}|#1}}|\\
|\expandafter\redirectjob\jobname~~~}\input{\jobname}|
\end{tabular}
\end{center}

In an alternative approach,
child documents can be compiled by a specific command line
without additional code or specific definitions:
%
\begin{center}
|... -jobname "|\textit{target}|" "|[\textit{flags}]%
|\includeonly{|\textit{dest}|}\input{|\textit{main}|}"|
\end{center}
%

%%%%%%%%%%%%%%%%%%%%%%%%%%%%%%%%%%%%%%%%%%%%%%%%%%%%%%%%%%%%%%%%%%%%%%%%%%%%%%%%
%%%%%%%%%%%%%%%%%%%%%%%%%%%%%%%%%%%%%%%%%%%%%%%%%%%%%%%%%%%%%%%%%%%%%%%%%%%%%%%%
\section{Information}

%%%%%%%%%%%%%%%%%%%%%%%%%%%%%%%%%%%%%%%%%%%%%%%%%%%%%%%%%%%%%%%%%%%%%%%%%%%%%%%%
\subsection{Copyright}

Copyright \copyright{} 2017--2018 Niklas Beisert

This work may be distributed and/or modified under the
conditions of the \LaTeX{} Project Public License, either version 1.3
of this license or (at your option) any later version.
The latest version of this license is in
  \url{http://www.latex-project.org/lppl.txt}
and version 1.3 or later is part of all distributions of \LaTeX{}
version 2005/12/01 or later.

This work has the LPPL maintenance status `maintained'.

The Current Maintainer of this work is Niklas Beisert.

This work consists of the files |README.txt|, |childdoc.ins| and |childdoc.dtx|
as well as the derived files |childdoc.def|, |cdocsamp.tex|
with |cdocsch1.tex|, |cdocsch2.tex|, |cdocspt3.tex|, |cdocspt4.tex|,
|cdocsdrf.tex|, |cdocsfn1.tex|, |cdocsfn2.tex|
as well as |childdoc.pdf|.

%%%%%%%%%%%%%%%%%%%%%%%%%%%%%%%%%%%%%%%%%%%%%%%%%%%%%%%%%%%%%%%%%%%%%%%%%%%%%%%%
\subsection{Files and Installation}

The package consists of the files:
%
\begin{center}
\begin{tabular}{ll}
    |README.txt|   & readme file \\
    |childdoc.ins| & installation file \\
    |childdoc.dtx| & source file \\
    |childdoc.def| & definition file \\
    |cdocsamp.tex| & sample main file \\
    |cdocsch1.tex| & sample include file \\
    |cdocsch2.tex| & sample include file \\
    |cdocspt3.tex| & sample part file \\
    |cdocspt4.tex| & sample part file \\
    |cdocsdrf.tex| & sample redirection file \\
    |cdocsfn1.tex| & sample redirection file \\
    |cdocsfn2.tex| & sample redirection file \\
    |childdoc.pdf| & manual
\end{tabular}
\end{center}
%
The distribution consists of the files
|README.txt|, |childdoc.ins| and |childdoc.dtx|.
%
\begin{itemize}
\item
Run (pdf)\LaTeX{} on |childdoc.dtx|
to compile the manual |childdoc.pdf| (this file).
\item
Run \LaTeX{} on |childdoc.ins| to create the definitions file |childdoc.def|
and the sample |cdocsamp.tex| with include files
|cdocsch1.tex|, |cdocsch2.tex|, |cdocspt3.tex|, |cdocspt4.tex|,
|cdocsdrf.tex|, |cdocsfn1.tex|, |cdocsfn2.tex|.
Then copy the file |childdoc.def| to an appropriate directory of your \LaTeX{}
distribution, e.g.\ \textit{texmf-root}|/tex/latex/childdoc|.
\end{itemize}

%%%%%%%%%%%%%%%%%%%%%%%%%%%%%%%%%%%%%%%%%%%%%%%%%%%%%%%%%%%%%%%%%%%%%%%%%%%%%%%%
\subsection{Related CTAN Packages}

There are several other packages which offer a similar functionality:
%
\begin{itemize}
\item
The packages
\href{http://ctan.org/pkg/docmute}{\textsf{docmute}},
\href{http://ctan.org/pkg/includex}{\textsf{includex}} and
\href{http://ctan.org/pkg/standalone}{\textsf{standalone}}
provide commands to include only the document body of
a child file thus allowing both files to be compiled individually.
\item
The packages \href{http://ctan.org/pkg/subdocs}{\textsf{subdocs}}
and \href{http://ctan.org/pkg/subfiles}{\textsf{subfiles}}
provide structures in which the main and child documents can be
encapsulated and allowing them to be compiled individually.
The inclusion mechanism is different from the conventional |\include|.
\item
The package \href{http://ctan.org/pkg/combine}{\textsf{combine}}
is an elaborate solution to combine several documents into one.
\end{itemize}
%
See also the CTAN topic \href{http://ctan.org/topic/subdocs}{\textsf{subdocs}}
for further related packages.
The present package differs from the above solutions in that
a document structure constructed with the conventional |\include| mechanism
just needs two extra commands at the top of every file
such that all constituent files can be compiled individually.

%%%%%%%%%%%%%%%%%%%%%%%%%%%%%%%%%%%%%%%%%%%%%%%%%%%%%%%%%%%%%%%%%%%%%%%%%%%%%%%%
%\subsection{Feature Suggestions}
%
%The following is a list of features which may be useful for future
%versions of this package:
%%
%\begin{itemize}
%\item
%\ldots
%\end{itemize}

%%%%%%%%%%%%%%%%%%%%%%%%%%%%%%%%%%%%%%%%%%%%%%%%%%%%%%%%%%%%%%%%%%%%%%%%%%%%%%%%
\subsection{Revision History}

%%%%%%%%%%%%%%%%%%%%%%%%%%%%%%%%%%%%%%%%
\paragraph{v2.0:} 2018/12/30

\begin{itemize}
\item
immediate forward processing
\item
added |\childdocby| mechanism
\item
manual restructured
\end{itemize}

%%%%%%%%%%%%%%%%%%%%%%%%%%%%%%%%%%%%%%%%
\paragraph{v1.6:} 2018/01/17

\begin{itemize}
\item
application for development of include files
\item
corrections to manual
\end{itemize}

%%%%%%%%%%%%%%%%%%%%%%%%%%%%%%%%%%%%%%%%
\paragraph{v1.5:} 2017/05/21

\begin{itemize}
\item
more complete structuring introduced
\item
|\childdocof| introduced
\item
|\childdoc| renamed to |\childdocmain|
\item
|\childredirect| renamed to |\childdocforward| and |\childdocforwardprefix|
and functionality expanded
\end{itemize}

%%%%%%%%%%%%%%%%%%%%%%%%%%%%%%%%%%%%%%%%
\paragraph{v1.0:} 2017/04/27

\begin{itemize}
\item
manual and install package
\item
first version published on CTAN
\end{itemize}

%%%%%%%%%%%%%%%%%%%%%%%%%%%%%%%%%%%%%%%%
\paragraph{v0.6:} 2017/04/26

\begin{itemize}
\item
redirection mechanism added
\end{itemize}

%%%%%%%%%%%%%%%%%%%%%%%%%%%%%%%%%%%%%%%%
\paragraph{v0.5:} 2017/04/26

\begin{itemize}
\item
functionality in definition file
\end{itemize}


%%%%%%%%%%%%%%%%%%%%%%%%%%%%%%%%%%%%%%%%%%%%%%%%%%%%%%%%%%%%%%%%%%%%%%%%%%%%%%%%
%%%%%%%%%%%%%%%%%%%%%%%%%%%%%%%%%%%%%%%%%%%%%%%%%%%%%%%%%%%%%%%%%%%%%%%%%%%%%%%%
%%%%%%%%%%%%%%%%%%%%%%%%%%%%%%%%%%%%%%%%%%%%%%%%%%%%%%%%%%%%%%%%%%%%%%%%%%%%%%%%
\appendix

\settowidth\MacroIndent{\rmfamily\scriptsize 000\ }

 \DocInput{childdoc.dtx}

\end{document}
%</driver>
% \fi
%
% %%%%%%%%%%%%%%%%%%%%%%%%%%%%%%%%%%%%%%%%%%%%%%%%%%%%%%%%%%%%%%%%%%%%%%%%%%%%%%
% %%%%%%%%%%%%%%%%%%%%%%%%%%%%%%%%%%%%%%%%%%%%%%%%%%%%%%%%%%%%%%%%%%%%%%%%%%%%%%
% \section{Sample}
%\iffalse
%<*samplemain>
%\fi
%
% The following presents a sample document
% with two chapters, two parts, a title page,
% a compile flag as well as three forwarding files to set the flag.
% It consists of eight |.tex| files:
% \begin{center}
% \begin{tabular}{ll}
% |cdocsamp.tex|&main file\\
% |cdocsch1.tex|&include file for chapter 1\\
% |cdocsch2.tex|&include file for chapter 2\\
% |cdocspt3.tex|&include file for part 3\\
% |cdocspt4.tex|&include file for part 4\\
% |cdocsdrf.tex|&forwarding file for main file in draft mode\\
% |cdocsfi1.tex|&forwarding file for final version of chapter 1\\
% |cdocsfi2.tex|&forwarding file for final version of chapter 2\\
% \end{tabular}
% \end{center}
% Each of the eight files can be compiled directly by the \LaTeX{} compiler.
%
% %%%%%%%%%%%%%%%%%%%%%%%%%%%%%%%%%%%%%%
% \paragraph{Main File.}
%
% The main file is called |cdocsamp.tex|.
%
% Load the \textsf{childdoc} definitions and
% declare the filename for the main document:
%    \begin{macrocode}
% \iffalse
%
% childdoc.dtx Copyright (C) 2017-2018 Niklas Beisert
%
% This work may be distributed and/or modified under the
% conditions of the LaTeX Project Public License, either version 1.3
% of this license or (at your option) any later version.
% The latest version of this license is in
%   http://www.latex-project.org/lppl.txt
% and version 1.3 or later is part of all distributions of LaTeX
% version 2005/12/01 or later.
%
% This work has the LPPL maintenance status `maintained'.
%
% The Current Maintainer of this work is Niklas Beisert.
%
% This work consists of the files childdoc.dtx and childdoc.ins
% and the derived files childdoc.def and cdocsamp.tex with
% cdocsch1.tex, cdocsch2.tex, cdocsdrf.tex, cdocsfn1.tex, cdocsfn2.tex.
%
%<package>\ifdefined\childdocmain\endinput\fi
%<package>\ProvidesFile{childdoc.def}[2018/12/30 v2.0 child document driver]
%<samplemain>\ProvidesFile{cdocsamp.tex}[2018/12/30 v2.0 sample for childdoc]
%<*driver>
%\ProvidesFile{childdoc.drv}[2018/12/30 v2.0 childdoc reference manual file]
\PassOptionsToClass{10pt,a4paper}{article}
\documentclass{ltxdoc}

\usepackage[margin=35mm]{geometry}
\usepackage{hyperref}
\usepackage{hyperxmp}
\usepackage[usenames]{color}

\hypersetup{colorlinks=true}
\hypersetup{pdfstartview=FitH}
\hypersetup{pdfpagemode=UseNone}
\hypersetup{pdfsource={}}
\hypersetup{pdflang={en-UK}}
\hypersetup{pdfcopyright={Copyright 2017-2018 Niklas Beisert.
  This work may be distributed and/or modified under the
  conditions of the LaTeX Project Public License, either version 1.3
  of this license or (at your option) any later version.}}
\hypersetup{pdflicenseurl={http://www.latex-project.org/lppl.txt}}
\hypersetup{pdfcontactaddress={ETH Zurich, ITP, HIT K,
  Wolfgang-Pauli-Strasse 27}}
\hypersetup{pdfcontactpostcode={8093}}
\hypersetup{pdfcontactcity={Zurich}}
\hypersetup{pdfcontactcountry={Switzerland}}
\hypersetup{pdfcontactemail={nbeisert@itp.phys.ethz.ch}}
\hypersetup{pdfcontacturl={http://people.phys.ethz.ch/\xmptilde nbeisert/}}

\newcommand{\secref}[1]{\hyperref[#1]{section \ref*{#1}}}

\parskip1ex
\parindent0pt
\let\olditemize\itemize
\def\itemize{\olditemize\parskip0pt}

\begin{document}

\title{The \textsf{childdoc} Package}
\hypersetup{pdftitle={The childdoc Package}}
\author{Niklas Beisert\\[2ex]
  Institut f\"ur Theoretische Physik\\
  Eidgen\"ossische Technische Hochschule Z\"urich\\
  Wolfgang-Pauli-Strasse 27, 8093 Z\"urich, Switzerland\\[1ex]
  \href{mailto:nbeisert@itp.phys.ethz.ch}
  {\texttt{nbeisert@itp.phys.ethz.ch}}}
\hypersetup{pdfauthor={Niklas Beisert}}
\hypersetup{pdfsubject={Manual for the LaTeX2e Package childdoc}}
\date{30 December 2018, \textsf{v2.0}}
\maketitle

\begin{abstract}\noindent
\textsf{childdoc} is a \LaTeXe{} package
that enables the direct compilation
of document sections included by |\include|
to individual files.
\end{abstract}

\begingroup
\parskip0ex
\tableofcontents
\endgroup

%%%%%%%%%%%%%%%%%%%%%%%%%%%%%%%%%%%%%%%%%%%%%%%%%%%%%%%%%%%%%%%%%%%%%%%%%%%%%%%%
%%%%%%%%%%%%%%%%%%%%%%%%%%%%%%%%%%%%%%%%%%%%%%%%%%%%%%%%%%%%%%%%%%%%%%%%%%%%%%%%
\section{Introduction}

\LaTeX{} provides a mechanism to structure a large document (such as a book)
into a main file and several child files (containing the chapters)
using the |\include| command.
This mechanism is beneficial for documents
which span hundreds of pages in order to
make the source file(s) more manageable.
Moreover, compilation can be restricted to
selected child files by means of the |\includeonly| command.
The latter feature can be used to reduce the compilation time while editing
(this was significantly more useful in the earlier days of \LaTeX{})
or to generate a smaller document which is easier to navigate.
Another application of |\includeonly| is to generate
documents consisting of selected parts of the complete document.

However, there are a few drawbacks of the plain |\include| mechanism:
\begin{itemize}
\item
The child files cannot be compiled on their own,
they can only be compiled via the main file.
A naive editing environment
(such as a text editor with an option
to have the current file processed by \LaTeX)
may require one to switch to the main file before compiling;
attempting to compile the child file produces errors.
\item
The main file must be modified (each time)
to adjust the |\includeonly| command
to the present needs. This easily leaves the main file in a messy state.
\item
The generated document will always carry the filename
of the main document. This is inconvenient if
several child files are to be compiled and
to be kept for distribution.
\end{itemize}

The present package provides a simple interface
to make child files individually compilable by \LaTeX{}.
Compiling a child file then has the same effect as compiling
the main file with an |\includeonly| command
to select the appropriate child.
Moreover the generated document will carry the name of the child
rather than the main file.
This resolves all three above issues.

This feature is meant to make the editing of books,
thesis documents and lecture notes somewhat more convenient.
However, the package can also be used efficiently for
composing a series of documents (such as exercise sheets)
which are typically distributed individually.
It then assists the author in generating the individual documents
(potentially in different versions)
as well as a document containing the collected series.
Another application is in developing style files
or other kinds of included material
where compilation of the style file could redirect
to a sample or test file.

%%%%%%%%%%%%%%%%%%%%%%%%%%%%%%%%%%%%%%%%%%%%%%%%%%%%%%%%%%%%%%%%%%%%%%%%%%%%%%%%
%%%%%%%%%%%%%%%%%%%%%%%%%%%%%%%%%%%%%%%%%%%%%%%%%%%%%%%%%%%%%%%%%%%%%%%%%%%%%%%%
\section{Usage}

First of all, the package \textsf{childdoc} is \emph{not} a standard
\LaTeXe{} |.sty| style file! Therefore it needs to be invoked in
a non-standard way.

%%%%%%%%%%%%%%%%%%%%%%%%%%%%%%%%%%%%%%%%%%%%%%%%%%%%%%%%%%%%%%%%%%%%%%%%%%%%%%%%
\subsection{Included Files}
\label{sec:include}

%%%%%%%%%%%%%%%%%%%%%%%%%%%%%%%%%%%%%%%%
\DescribeMacro{\childdocmain}
To use the package, add the commands
\begin{center}
\begin{tabular}{l}
|\input{childdoc.def}|\\
|\childdocmain{}|\\
\end{tabular}
\end{center}
at the very top of the main \LaTeX{} file,
in particular \emph{before} the |\documentclass| statement!
The argument of |\childdocmain| should be left empty
(but it must be present).

%%%%%%%%%%%%%%%%%%%%%%%%%%%%%%%%%%%%%%%%
\DescribeMacro{\childdocof}
Furthermore, add the commands
\begin{center}
\begin{tabular}{l}
|\input{childdoc.def}|\\
|\childdocof{|\textit{main}|}|\\
\end{tabular}
\end{center}
at the top of every child file \textit{child}
which is included by |\include{|\textit{child}|}|
from within the main file
(or at least for those files to be compiled individually).
The argument \textit{main} must be the filename of the main file.

There are a couple of
considerations in setting up the main and child documents:

%%%%%%%%%%%%%%%%%%%%%%%%%%%%%%%%%%%%%%%%
\paragraph{Restrictions.}

Please note the following restrictions:
\begin{itemize}
\item
|\childdocmain| must be called with one argument \textit{main}
to ensure compatibility with earlier version of the package.
It must either be empty (|\childdocmain{}|)
or precisely match the filename of the main file in which it is specified.
See \secref{sec:detection} for further information.
\item
The filename \textit{main} must be specified without the |.tex| extension.
\item
The filename \textit{main} is case sensitive
(even in case-insensitive file systems)
due to internal string comparison.
\item
The argument \textit{main} should be fully expanded, it cannot be a macro.
\item
Subdirectories and special characters should be avoided in filenames.
\item
The command |\childdocmain{|\textit{main}|}| must be followed by a whitespace.
It should not be followed immediately by another command
or by a comment mark `|%|'.
This is because the \TeX{} parser reads the token immediately following
the argument of |\childdocmain| and puts it
at the beginning of every child section;
however, a white\-space is ignored.
\end{itemize}

%%%%%%%%%%%%%%%%%%%%%%%%%%%%%%%%%%%%%%%%
\paragraph{Content of Main File.}

It is advisable to place all content in the child files included by |\include|.
Any output contained in the main file will appear in all child documents
unless suppressed manually;
it cannot be suppressed automatically by the |\includeonly| directive
and thus should normally be avoided.
A method to include some content in the main file
by means of conditional processing is described in \secref{sec:conditional}.

%%%%%%%%%%%%%%%%%%%%%%%%%%%%%%%%%%%%%%%%
\paragraph{Page Numbering.}

When only a part of the document is compiled,
the appropriate numbering of pages
(as well as other status parameters)
is determined from the |.aux| files.
The latter contain information from previous passes.
However this information needs to propagate through
all intermediate child documents.
Therefore the page numbering in child documents may well
be inconsistent until the complete document is compiled at least once.

A useful (if unconventional) way to always ensure a consistent
page numbering is to restart the numbering in each child document
and denote the pages by `\textit{child}|.|\textit{page}'
where \textit{child} represents the chapter/section number of the child file.
This can be achieved by the command
|\numberwithin{page}{|\textit{child}|}|
of the \textsf{amsmath} package
where \textit{child} can be |chapter| or |section|
depending on the chosen structuring.
Alternatively, one can modify the macro |\thepage| appropriately
and reset the counter |page| at the start of each child file.

%%%%%%%%%%%%%%%%%%%%%%%%%%%%%%%%%%%%%%%%%%%%%%%%%%%%%%%%%%%%%%%%%%%%%%%%%%%%%%%%
\subsection{Conditional Processing}
\label{sec:conditional}

The package provides a mechanism to compile different versions
of a document. To customise the versions further some conditional processing
can come in handy to distinguish which version is being compiled.
The package provides two macros to describe the compilation context:

%%%%%%%%%%%%%%%%%%%%%%%%%%%%%%%%%%%%%%%%
\DescribeMacro{\ifchilddoc}
The conditional |\ifchilddoc| distinguishes between the compilation of
child documents and the main document:
%
\begin{center}
|\ifchilddoc |\textit{child-code}| |[|\||else |\textit{main-code}]| \||fi|
\end{center}

%%%%%%%%%%%%%%%%%%%%%%%%%%%%%%%%%%%%%%%%
\DescribeMacro{\childdocname}
\DescribeMacro{\childdocjob}
The macro |\childdocname| contains the filename (without extension)
of the main or child file being processed.
Note that |\childdocjob| will always contain the name of the main file.

%%%%%%%%%%%%%%%%%%%%%%%%%%%%%%%%%%%%%%%%
\paragraph{Title Page.}

Conditional processing can be used to include a title or banner page
in the main document when proper precautions are taken.
Importantly, the code in the main file should ensure that the page counter
(as well as other status parameters which are stored in the |.aux| files)
takes the same value after the conditional processing.
Otherwise the page numbers may take divergent values
depending on which part is compiled.

For example, a title page could be declared by:
%
\begin{center}
\begin{tabular}{l}
|\ifchilddoc\||else|\\
|\addtocounter{page}{-1}|\\
\textit{code for title page}\\
|\newpage|\\
|\||fi|
\end{tabular}
\end{center}
%
A banner page for the child documents can be generated by:
%
\begin{center}
\begin{tabular}{l}
|\ifchilddoc|\\
|\addtocounter{page}{-1}|\\
\textit{code for banner page}\\
|\newpage|\\
|\||fi|
\end{tabular}
\end{center}
%
Here one could write a message such as:
\begin{center}
|This is the part \childdocname{} of \childdocjob{}.|
\end{center}

%%%%%%%%%%%%%%%%%%%%%%%%%%%%%%%%%%%%%%%%%%%%%%%%%%%%%%%%%%%%%%%%%%%%%%%%%%%%%%%%
\subsection{Flags}
\label{sec:flags}

The package makes it easy to generate different versions
of the main or child documents.
To this end compilation flags can be defined
and assigned different default values.
They will be particularly useful in conjunction
with the forwarding mechanism described in \secref{sec:forward}.

For example, it may be useful to have a flag |\version|
which can be set to |draft| or |final|.
The document source will contain some conditional code
depending on the value of |\version|.
Suppose further, the flag should default to |final| for the main file
and to |draft| for child files
which is a natural assignment for editing the document.
This is achieved by placing the following code
in the preamble of the main document
(below the |\childdocmain| directive):
%
\begin{center}
\begin{tabular}{l}
|\ifchilddoc|\\
|\providecommand{\version}{draft}|\\
|\||else|\\
|\providecommand{\version}{final}|\\
|\||fi|
\end{tabular}
\end{center}
%
The definition by |\providecommand| makes sure
that previous definitions are not overwritten.
Further statements |\providecommand{\version}{...}|
can thus be added before the above code to override it.

For the main file, one might add a line
(between |\childdocmain| and the above block)
%
\begin{center}
|%\ifchilddoc\||else\providecommand{\version}{draft}\||fi|
\end{center}
%
which can be uncommented to produce a draft version.
Likewise one can add a line to the very top of a child file
(above the |\childdocof{|\textit{main}|}| directive)
%
\begin{center}
|%\providecommand{\version}{final}|
\end{center}
%
which can be uncommented to produce the final version of this child document.

%%%%%%%%%%%%%%%%%%%%%%%%%%%%%%%%%%%%%%%%%%%%%%%%%%%%%%%%%%%%%%%%%%%%%%%%%%%%%%%%
\subsection{Forwarding}
\label{sec:forward}

Different versions of the main or child documents
using compilation flags as described in \secref{sec:flags}
can be (permanently) stored in different files
for convenient compilation, viewing and distribution.
To this end, the package defines a command
to pass on compilation to a different file:

%%%%%%%%%%%%%%%%%%%%%%%%%%%%%%%%%%%%%%%%
\DescribeMacro{\childdocforward}
The command |\childdocforward| redirects processing to
another source file:
%
\begin{center}
\begin{tabular}{l}
|\input{childdoc.def}|\\
|\childdocforward[|\textit{main}|]{|\textit{dest}|}|\\
\end{tabular}
\end{center}
%
The argument \textit{dest} is the destination file
(without extension).
It should be the main file or one of the child files.
Note that further \textsf{childdoc} directives
such as |\childdocof| and |\childdocforward|
in the indicated file will be processed in this form.
The optional argument \textit{main}
passes on directly to the main file \textit{main}
while pretending to compile the child \textit{dest}.
This form behaves as if \textit{dest}
issues |\childdocof{|\textit{main}|}| right away,
and no further \textsf{childdoc} directives will be processed.

%%%%%%%%%%%%%%%%%%%%%%%%%%%%%%%%%%%%%%%%
\DescribeMacro{\...prefix}
In the alternative form |\childdocforwardprefix|,
%
\begin{center}
\begin{tabular}{l}
|\input{childdoc.def}|\\
|\childdocforwardprefix[|\textit{main}|]{|\textit{prefix}|}{|\textit{dest}|}|
\end{tabular}
\end{center}
%
the destination file is determined by a pattern
depending on the current file:
To make this work, the current file must be called
`{\textit{prefix}\hspace{0.2em}\textit{suffix}}'
with \textit{prefix} matching precisely the argument.
Processing is then passed on to the file
`{\textit{dest}\hspace{0.2em}\textit{suffix}}'.
Surely, the same effect is achieved by
directly specifying the
argument `{\textit{dest}\hspace{0.2em}\textit{suffix}}'
in the first form.
However, that requires to set up a different file
for each child. With the alternative form of the command
all these files can have exactly the same content
which simplifies setting them up and maintaining them.

For example, the following file |draft.tex|
with a compilation flag |\version| as described in \secref{sec:flags}
compiles the main document as a draft:
%
\begin{center}
\begin{tabular}{l}
|\def\version{draft}|\\
|\input{childdoc.def}|\\
|\childdocforward{|\textit{main}|}|
\end{tabular}
\end{center}
%
Likewise, the following files |final|\textit{nn}|.tex|
compile the final version of the child document
|child|\textit{nn}|.tex|:
%
\begin{center}
\begin{tabular}{l}
|\def\version{final}|\\
|\input{childdoc.def}|\\
|\childdocforwardprefix{final}{child}|
\end{tabular}
\end{center}
%

Note that when several versions of a main file and/or of each child file
are to be generated, it may be convenient to set up a |Makefile| or
shell script to automatise the process.

%%%%%%%%%%%%%%%%%%%%%%%%%%%%%%%%%%%%%%%%%%%%%%%%%%%%%%%%%%%%%%%%%%%%%%%%%%%%%%%%
\subsection{Command Line Processing}
\label{sec:commandline}

The effect of redirection files can also be achieved by invoking
the \LaTeX{} compiler with a more elaborate command line.
Most conveniently this should be done as part
of a shell script or a |Makefile|.

When using \textsf{childdoc} in the main file, the following
command lines effectively perform a redirection
(note that depending on the shell being used,
backslashes may have to be doubled: `|\|' $\to$ `|\\|'):
%
\begin{center}
|... -jobname "|\textit{target}|" |\\|"|[\textit{flags}]%
|\input{childdoc.def}\childdocforward[|\textit{main}|]{|\textit{dest}|}"|
\end{center}
%
Here \textit{target} is the name of the output file,
\textit{main} is the name of the main file
and \textit{dest} is the name of the main or child file to be processed
(all filenames without extensions).
The optional argument \textit{main} can be omitted
if \textit{main} matches \textit{dest}.
Optionally, compilation \textit{flags} can be defined via |\def| commands.
This command line makes the \TeX{} engine believe
it is compiling the file \textit{target}
whose content is specified as the latter parameter.
The provided code then forwards the processing to
\textit{main} or \textit{dest} as described in \secref{sec:forward}.

%%%%%%%%%%%%%%%%%%%%%%%%%%%%%%%%%%%%%%%%%%%%%%%%%%%%%%%%%%%%%%%%%%%%%%%%%%%%%%%%
\subsection{Include by Input}
\label{sec:input}

Including child documents by |\include| has some restrictions by design.
Most notably, the content of a child document always occupies
its own set of pages; pages cannot be shared between child documents.
Usually, this behaviour makes perfect sense
because each child document contain an essential part of the document.
However, in some situations it may be desirable to compose
a document from a collection of parts
without having mandatory page breaks between then.
For this case, the package
provides a mechanism to include parts
by |\input| which can also be processed individually.
However, by construction this mechanism
requires manual handling of the content to be output.

%%%%%%%%%%%%%%%%%%%%%%%%%%%%%%%%%%%%%%%%
\DescribeMacro{\ifchilddocmanual}
The main file should be prepared as usual, see \secref{sec:include}.
However, the document body must make a distinction
between processing of an individual part and of the main document, e.g.:
%
\begin{center}
\begin{tabular}{l}
|\ifchilddocmanual|\\
|\input{\childdocname}|\\
|\||else|\\
\textit{document body with }|\input{|\textit{part}|}|\\
|\||fi|
\end{tabular}
\end{center}
%
The conditional |\ifchilddocmanual| is true whenever
a part to be included by |\input| is being compiled,
and the name of the part is stored in |\childdocname|.

%%%%%%%%%%%%%%%%%%%%%%%%%%%%%%%%%%%%%%%%
\DescribeMacro{\childdocby}
Each part to be included by |\input| should start with:
%
\begin{center}
\begin{tabular}{l}
|\input{childdoc.def}|\\
|\childdocby{|\textit{main}|}|\\
\end{tabular}
\end{center}
%
The directive |\childdocby| is similar to |\childdocof|
described in \secref{sec:include},
but the subsequent selection of content must be done manually.
To that end, both |\ifchilddoc| and |\ifchilddocmanual|
will be true upon processing of a part,
and the name of the part is stored in |\childdocname|.
Note that |\jobname| will be set to the filename of the current part
so that each part receives an individual |.aux| file
that does not interfere with the |.aux| file(s) of the main document.
This behaviour can be altered by the alternative form
|\childdocby[*]{|\textit{main}|}| (with a non-empty optional argument)
which uses the |.aux| file of the main document
by setting |\jobname| to \textit{main}.

%%%%%%%%%%%%%%%%%%%%%%%%%%%%%%%%%%%%%%%%%%%%%%%%%%%%%%%%%%%%%%%%%%%%%%%%%%%%%%%%
\subsection{Driver Development}
\label{sec:driver}

The \textsf{childdoc} mechanism can also be use for the development
of definition files such as \LaTeX{} styles or classes.
This case differs from the above setup with multiple parts
included by |\include| in that no |\includeonly| should be invoked.
This can be achieved by starting the include file
(before |\ProvidesPackage|) with:
%
\begin{center}
\begin{tabular}{l}
|\input{childdoc.def}|\\
|\childdocforward{|\textit{main}|}|\\
\end{tabular}
\end{center}
%
or alternatively with:
%
\begin{center}
\begin{tabular}{l}
|\input{childdoc.def}|\\
|\childdocby{|\textit{main}|}|\\
\end{tabular}
\end{center}
%
Both forms have slightly different effects as described above.
The main file is prepared as usual, see \secref{sec:include}.

%%%%%%%%%%%%%%%%%%%%%%%%%%%%%%%%%%%%%%%%%%%%%%%%%%%%%%%%%%%%%%%%%%%%%%%%%%%%%%%%
\subsection{Legacy Detection}
\label{sec:detection}

The directive |\childdocmain| in the main file can detect
whether the complete document or merely a child is to be compiled
even without using the directive |\childdocof|.
This method is deprecated because it is less robust
and there is no compelling reason to use it;
it is merely provided for backward compatibility
and it may be removed in future versions.

If the detection mechanism is to be used,
it is mandatory to correctly specify
the filename of the main file as the argument of |\childdocmain|:
%
\begin{center}
\begin{tabular}{l}
|\input{childdoc.def}|\\
|\childdocmain{|\textit{main}|}|\\
\end{tabular}
\end{center}
%
If |\jobname| does not match the argument \textit{main} of |\childdocmain|,
it is assumed that |\jobname| points to the child file to be compiled.
When using |\childdocmain| with the main file specified as argument,
it suffices to start a child file
with just |\input{|\textit{main}|}|
without loading of the package and using |\childdocof|.
If instead all processing is done
with the appropriate \textsf{childdoc} directives,
the argument of \textit{main} of |\childdocmain| can be empty.

An alternative version of the command line processing described
in \secref{sec:commandline} using the detection mechanism reads:
%
\begin{center}
|... -jobname "|\textit{target}|" "|[\textit{flags}]%
[|\def\jobname{|\textit{dest}|}|]|\input{|\textit{main}|}"|
\end{center}

%%%%%%%%%%%%%%%%%%%%%%%%%%%%%%%%%%%%%%%%%%%%%%%%%%%%%%%%%%%%%%%%%%%%%%%%%%%%%%%%
\subsection{Manual Code}
\label{sec:manual}

In case one cannot be certain whether the definitions file |childdoc.def|
is installed on the target \TeX{} distribution
and one prefers not to ship it,
it is conceivable to paste a few relevant commands into the sources.

To that end, drop all statements |\input{childdoc.def}|
and perform the replacements as outlined below.
Instead of |\childdocmain{|\textit{main}|}| add the following code
to the top of the main file:
%
\begin{center}
\begin{tabular}{l}
|\||ifdefined\childdocname\endinput\||fi\newif\ifchilddoc|\\
|\edef\childdocname{\scantokens\expandafter{\jobname\noexpand}}|\\
|\def\childdocmain{|\textit{main}|}\||ifx\childdocmain\childdocname\||else|\\
|\childdoctrue\includeonly{\childdocname}\let\jobname\childdocmain\||fi|\\
\end{tabular}
\end{center}
%
Instead of |\childdocof{|\textit{main}|}| just include the main file
at the top of each child file:
%
\begin{center}
|\input{|\textit{main}|}|
\end{center}
%
A simple redirection |\childdocforward{|\textit{dest}|}| is achieved by:
%
\begin{center}
|\def\jobname{|\textit{dest}|}\input{\jobname}|
\end{center}
%
The redirection with prefix
|\childdocforwardprefix[|\textit{prefix}|]{|\textit{dest}|}|
is accomplished by:
%
\begin{center}
\begin{tabular}{l}
|{\edef\jobname{\scantokens\expandafter{\jobname\noexpand}}|\\
|\def\redirectjob |\textit{prefix}|#1~~~{\gdef\jobname{|\textit{dest}|#1}}|\\
|\expandafter\redirectjob\jobname~~~}\input{\jobname}|
\end{tabular}
\end{center}

In an alternative approach,
child documents can be compiled by a specific command line
without additional code or specific definitions:
%
\begin{center}
|... -jobname "|\textit{target}|" "|[\textit{flags}]%
|\includeonly{|\textit{dest}|}\input{|\textit{main}|}"|
\end{center}
%

%%%%%%%%%%%%%%%%%%%%%%%%%%%%%%%%%%%%%%%%%%%%%%%%%%%%%%%%%%%%%%%%%%%%%%%%%%%%%%%%
%%%%%%%%%%%%%%%%%%%%%%%%%%%%%%%%%%%%%%%%%%%%%%%%%%%%%%%%%%%%%%%%%%%%%%%%%%%%%%%%
\section{Information}

%%%%%%%%%%%%%%%%%%%%%%%%%%%%%%%%%%%%%%%%%%%%%%%%%%%%%%%%%%%%%%%%%%%%%%%%%%%%%%%%
\subsection{Copyright}

Copyright \copyright{} 2017--2018 Niklas Beisert

This work may be distributed and/or modified under the
conditions of the \LaTeX{} Project Public License, either version 1.3
of this license or (at your option) any later version.
The latest version of this license is in
  \url{http://www.latex-project.org/lppl.txt}
and version 1.3 or later is part of all distributions of \LaTeX{}
version 2005/12/01 or later.

This work has the LPPL maintenance status `maintained'.

The Current Maintainer of this work is Niklas Beisert.

This work consists of the files |README.txt|, |childdoc.ins| and |childdoc.dtx|
as well as the derived files |childdoc.def|, |cdocsamp.tex|
with |cdocsch1.tex|, |cdocsch2.tex|, |cdocspt3.tex|, |cdocspt4.tex|,
|cdocsdrf.tex|, |cdocsfn1.tex|, |cdocsfn2.tex|
as well as |childdoc.pdf|.

%%%%%%%%%%%%%%%%%%%%%%%%%%%%%%%%%%%%%%%%%%%%%%%%%%%%%%%%%%%%%%%%%%%%%%%%%%%%%%%%
\subsection{Files and Installation}

The package consists of the files:
%
\begin{center}
\begin{tabular}{ll}
    |README.txt|   & readme file \\
    |childdoc.ins| & installation file \\
    |childdoc.dtx| & source file \\
    |childdoc.def| & definition file \\
    |cdocsamp.tex| & sample main file \\
    |cdocsch1.tex| & sample include file \\
    |cdocsch2.tex| & sample include file \\
    |cdocspt3.tex| & sample part file \\
    |cdocspt4.tex| & sample part file \\
    |cdocsdrf.tex| & sample redirection file \\
    |cdocsfn1.tex| & sample redirection file \\
    |cdocsfn2.tex| & sample redirection file \\
    |childdoc.pdf| & manual
\end{tabular}
\end{center}
%
The distribution consists of the files
|README.txt|, |childdoc.ins| and |childdoc.dtx|.
%
\begin{itemize}
\item
Run (pdf)\LaTeX{} on |childdoc.dtx|
to compile the manual |childdoc.pdf| (this file).
\item
Run \LaTeX{} on |childdoc.ins| to create the definitions file |childdoc.def|
and the sample |cdocsamp.tex| with include files
|cdocsch1.tex|, |cdocsch2.tex|, |cdocspt3.tex|, |cdocspt4.tex|,
|cdocsdrf.tex|, |cdocsfn1.tex|, |cdocsfn2.tex|.
Then copy the file |childdoc.def| to an appropriate directory of your \LaTeX{}
distribution, e.g.\ \textit{texmf-root}|/tex/latex/childdoc|.
\end{itemize}

%%%%%%%%%%%%%%%%%%%%%%%%%%%%%%%%%%%%%%%%%%%%%%%%%%%%%%%%%%%%%%%%%%%%%%%%%%%%%%%%
\subsection{Related CTAN Packages}

There are several other packages which offer a similar functionality:
%
\begin{itemize}
\item
The packages
\href{http://ctan.org/pkg/docmute}{\textsf{docmute}},
\href{http://ctan.org/pkg/includex}{\textsf{includex}} and
\href{http://ctan.org/pkg/standalone}{\textsf{standalone}}
provide commands to include only the document body of
a child file thus allowing both files to be compiled individually.
\item
The packages \href{http://ctan.org/pkg/subdocs}{\textsf{subdocs}}
and \href{http://ctan.org/pkg/subfiles}{\textsf{subfiles}}
provide structures in which the main and child documents can be
encapsulated and allowing them to be compiled individually.
The inclusion mechanism is different from the conventional |\include|.
\item
The package \href{http://ctan.org/pkg/combine}{\textsf{combine}}
is an elaborate solution to combine several documents into one.
\end{itemize}
%
See also the CTAN topic \href{http://ctan.org/topic/subdocs}{\textsf{subdocs}}
for further related packages.
The present package differs from the above solutions in that
a document structure constructed with the conventional |\include| mechanism
just needs two extra commands at the top of every file
such that all constituent files can be compiled individually.

%%%%%%%%%%%%%%%%%%%%%%%%%%%%%%%%%%%%%%%%%%%%%%%%%%%%%%%%%%%%%%%%%%%%%%%%%%%%%%%%
%\subsection{Feature Suggestions}
%
%The following is a list of features which may be useful for future
%versions of this package:
%%
%\begin{itemize}
%\item
%\ldots
%\end{itemize}

%%%%%%%%%%%%%%%%%%%%%%%%%%%%%%%%%%%%%%%%%%%%%%%%%%%%%%%%%%%%%%%%%%%%%%%%%%%%%%%%
\subsection{Revision History}

%%%%%%%%%%%%%%%%%%%%%%%%%%%%%%%%%%%%%%%%
\paragraph{v2.0:} 2018/12/30

\begin{itemize}
\item
immediate forward processing
\item
added |\childdocby| mechanism
\item
manual restructured
\end{itemize}

%%%%%%%%%%%%%%%%%%%%%%%%%%%%%%%%%%%%%%%%
\paragraph{v1.6:} 2018/01/17

\begin{itemize}
\item
application for development of include files
\item
corrections to manual
\end{itemize}

%%%%%%%%%%%%%%%%%%%%%%%%%%%%%%%%%%%%%%%%
\paragraph{v1.5:} 2017/05/21

\begin{itemize}
\item
more complete structuring introduced
\item
|\childdocof| introduced
\item
|\childdoc| renamed to |\childdocmain|
\item
|\childredirect| renamed to |\childdocforward| and |\childdocforwardprefix|
and functionality expanded
\end{itemize}

%%%%%%%%%%%%%%%%%%%%%%%%%%%%%%%%%%%%%%%%
\paragraph{v1.0:} 2017/04/27

\begin{itemize}
\item
manual and install package
\item
first version published on CTAN
\end{itemize}

%%%%%%%%%%%%%%%%%%%%%%%%%%%%%%%%%%%%%%%%
\paragraph{v0.6:} 2017/04/26

\begin{itemize}
\item
redirection mechanism added
\end{itemize}

%%%%%%%%%%%%%%%%%%%%%%%%%%%%%%%%%%%%%%%%
\paragraph{v0.5:} 2017/04/26

\begin{itemize}
\item
functionality in definition file
\end{itemize}


%%%%%%%%%%%%%%%%%%%%%%%%%%%%%%%%%%%%%%%%%%%%%%%%%%%%%%%%%%%%%%%%%%%%%%%%%%%%%%%%
%%%%%%%%%%%%%%%%%%%%%%%%%%%%%%%%%%%%%%%%%%%%%%%%%%%%%%%%%%%%%%%%%%%%%%%%%%%%%%%%
%%%%%%%%%%%%%%%%%%%%%%%%%%%%%%%%%%%%%%%%%%%%%%%%%%%%%%%%%%%%%%%%%%%%%%%%%%%%%%%%
\appendix

\settowidth\MacroIndent{\rmfamily\scriptsize 000\ }

 \DocInput{childdoc.dtx}

\end{document}
%</driver>
% \fi
%
% %%%%%%%%%%%%%%%%%%%%%%%%%%%%%%%%%%%%%%%%%%%%%%%%%%%%%%%%%%%%%%%%%%%%%%%%%%%%%%
% %%%%%%%%%%%%%%%%%%%%%%%%%%%%%%%%%%%%%%%%%%%%%%%%%%%%%%%%%%%%%%%%%%%%%%%%%%%%%%
% \section{Sample}
%\iffalse
%<*samplemain>
%\fi
%
% The following presents a sample document
% with two chapters, two parts, a title page,
% a compile flag as well as three forwarding files to set the flag.
% It consists of eight |.tex| files:
% \begin{center}
% \begin{tabular}{ll}
% |cdocsamp.tex|&main file\\
% |cdocsch1.tex|&include file for chapter 1\\
% |cdocsch2.tex|&include file for chapter 2\\
% |cdocspt3.tex|&include file for part 3\\
% |cdocspt4.tex|&include file for part 4\\
% |cdocsdrf.tex|&forwarding file for main file in draft mode\\
% |cdocsfi1.tex|&forwarding file for final version of chapter 1\\
% |cdocsfi2.tex|&forwarding file for final version of chapter 2\\
% \end{tabular}
% \end{center}
% Each of the eight files can be compiled directly by the \LaTeX{} compiler.
%
% %%%%%%%%%%%%%%%%%%%%%%%%%%%%%%%%%%%%%%
% \paragraph{Main File.}
%
% The main file is called |cdocsamp.tex|.
%
% Load the \textsf{childdoc} definitions and
% declare the filename for the main document:
%    \begin{macrocode}
\input{childdoc.def}
\childdocmain{}
%    \end{macrocode}

% Optional override for |\version| flag:
%    \begin{macrocode}
%%\ifchilddoc\else\providecommand{\version}{draft}\fi
%    \end{macrocode}

% Define the default values for the |\version| flag
% (|final| for the main file and |draft| for childs):
%    \begin{macrocode}
\ifchilddoc
\providecommand{\version}{draft}
\else
\providecommand{\version}{final}
\fi
%    \end{macrocode}

% Load the standard document class:
%    \begin{macrocode}
\documentclass[12pt]{article}
%    \end{macrocode}

% Start the document body:
%    \begin{macrocode}
\begin{document}
%    \end{macrocode}

% Declare a title page.
% Print title, part of document being processed and version flag:
%    \begin{macrocode}
\addtocounter{page}{-1}
\begin{center}
{\LARGE\bfseries{}childdoc example\par}
\vspace{1cm}
\ifchilddoc
\ifchilddocmanual part\else chapter\fi:
`\childdocname' of `\childdocjob'\par
\else
main document: `\childdocjob'\par
\fi
version: \version\par
\end{center}
\newpage
%    \end{macrocode}

% Manually include selected file,
% otherwise process as usual:
%    \begin{macrocode}
\ifchilddocmanual
\section*{part `\childdocname'}
\input{\childdocname}
\else
%    \end{macrocode}

% Include the two chapters:
%    \begin{macrocode}
\include{cdocsch1}
\include{cdocsch2}
%    \end{macrocode}

% Include the two parts unless only chapters should be displayed:
%    \begin{macrocode}
\ifchilddoc\else
\section{part three}
\input{cdocspt3}
\section{part four}
\input{cdocspt4}
\fi
%    \end{macrocode}

% Process as usual until here:
%    \begin{macrocode}
\fi
%    \end{macrocode}

% End of document body:
%    \begin{macrocode}
\end{document}
%    \end{macrocode}
%\iffalse
%</samplemain>
%\fi
%
% %%%%%%%%%%%%%%%%%%%%%%%%%%%%%%%%%%%%%%
% \paragraph{Chapter Include Files.}
%
% The include files are called |cdocsch1.tex| and |cdocsch2.tex|.
%
%\iffalse
%<*samplechap1|samplechap2>
%\fi

% Optional override for |\version| flag:
%    \begin{macrocode}
%%\providecommand{\version}{final}
%    \end{macrocode}

% Include the main document:
%    \begin{macrocode}
\input{childdoc.def}
\childdocof{cdocsamp}
%    \end{macrocode}

%\iffalse
%</samplechap1|samplechap2>
%\fi
%
%\iffalse
%<*samplechap1>
%\fi
% Some text for chapter 1:
%    \begin{macrocode}
\section{one}
some text in chapter one
%    \end{macrocode}

%\iffalse
%</samplechap1>
%\fi
% Some text for chapter 2:
%\iffalse
%<*samplechap2>
%\fi
%    \begin{macrocode}
\section{two}
more text in chapter two
%    \end{macrocode}

%\iffalse
%</samplechap2>
%\fi
%
% %%%%%%%%%%%%%%%%%%%%%%%%%%%%%%%%%%%%%%
% \paragraph{Part Include Files.}
%
% The include files are called |cdocspt3.tex| and |cdocspt4.tex|.
%
%\iffalse
%<*samplepart3|samplepart4>
%\fi

% Optional override for |\version| flag:
%    \begin{macrocode}
%%\providecommand{\version}{final}
%    \end{macrocode}

% Include the main document:
%    \begin{macrocode}
\input{childdoc.def}
\childdocby{cdocsamp}
%    \end{macrocode}

%\iffalse
%</samplepart3|samplepart4>
%\fi
%
%\iffalse
%<*samplepart3>
%\fi
% Some text for part 3:
%    \begin{macrocode}
some text in part three
%    \end{macrocode}

%\iffalse
%</samplepart3>
%\fi
% Some text for part 4:
%\iffalse
%<*samplepart4>
%\fi
%    \begin{macrocode}
more text in part four
%    \end{macrocode}

%\iffalse
%</samplepart4>
%\fi
%
% %%%%%%%%%%%%%%%%%%%%%%%%%%%%%%%%%%%%%%
% \paragraph{Forwarding for a Complete Draft.}
%
% The following forwarding file |cdocsdrf.tex|
% compiles the main document in draft mode:
%\iffalse
%<*sampledraft>
%\fi
%    \begin{macrocode}
\def\version{draft}
\input{childdoc.def}
\childdocforward{cdocsamp}
%    \end{macrocode}

%\iffalse
%</sampledraft>
%\fi
%
% %%%%%%%%%%%%%%%%%%%%%%%%%%%%%%%%%%%%%%
% \paragraph{Forwarding for Final Version of the Chapters.}
%
% The following forwarding files |cdocsfn1.tex| and |cdocsfn2.tex|
% (with identical content)
% compile the final versions of the child documents
% |cdocsch1.tex| and |cdocsch2.tex|, respectively:
%\iffalse
%<*samplefinal>
%\fi
%    \begin{macrocode}
\def\version{final}
\input{childdoc.def}
\childdocforwardprefix[cdocsamp]{cdocsfn}{cdocsch}
%    \end{macrocode}

%\iffalse
%</samplefinal>
%\fi
%
% %%%%%%%%%%%%%%%%%%%%%%%%%%%%%%%%%%%%%%
% \paragraph{Command Line Processing.}
%
% The following three command lines generate the output files
% |cdocscld|, |cdocscl1| and |cdocscl2|
% which should be identical to
% |cdocsdrf|, |cdocsch1| and |cdocsfn2|, respectively:
% \begin{center}
% \begin{tabular}{l}
% |latex -jobname cdocscld \|\\
% |  "\def\version{draft}\input{childdoc.def}\childdocforward{cdocsamp}"|\\
% |latex -jobname cdocscl1 \|\\
% |  "\input{childdoc.def}\childdocforward[cdocsamp]{cdocsch1}"|\\
% |latex -jobname cdocscl2 \|\\
% |  "\def\version{final}\input{childdoc.def}\childdocforward{cdocsch2}"|
% \end{tabular}
% \end{center}
% Note that the trailing backslash on each first line
% merely continues the input to the second line
% (for convenient cut ant paste).
% Furthermore, the command |latex| can be replaced by any
% of its alternative versions such as |pdflatex|.
%
% %%%%%%%%%%%%%%%%%%%%%%%%%%%%%%%%%%%%%%%%%%%%%%%%%%%%%%%%%%%%%%%%%%%%%%%%%%%%%%
% %%%%%%%%%%%%%%%%%%%%%%%%%%%%%%%%%%%%%%%%%%%%%%%%%%%%%%%%%%%%%%%%%%%%%%%%%%%%%%
% \section{Implementation}
%\iffalse
%<*package>
%\fi
%
% This section describes the definitions file |childdoc.def|.

% The definitions cannot be loaded using |\usepackage| or |\RequirePackage|
% which has a mechanism to prevent loading a style file more than once.
% When loading the definitions by means of |\input|
% multiple instances have to be prevented manually:
%\iffalse
%This code needs to be before the `\ProvidesFile' directive
%which is defined at the beginning of this file.
%Therefore it is also placed there and commented out here.
%</package>
%<*discard>
%\fi
%    \begin{macrocode}
\ifdefined\childdocmain\endinput\fi
%    \end{macrocode}
%\iffalse
%</discard>
%<*package>
%\fi
%
% \macro{\ifchilddoc}
% \macro{\ifchilddocmanual}
% The conditional |\ifchilddoc| tells whether a
% child (true) or main (false) document is being compiled.
% The conditional |\ifchilddocmanual| tells whether
% the |\includeonly| mechanism is used (false) or
% the selection of child files must be performed manually (true).
% The definitions initialise to false:
%    \begin{macrocode}
\newif\ifchilddoc
\newif\ifchilddocmanual
%    \end{macrocode}

% \macro{\childdocname}
% \macro{\childdocjob}
% The macro |\childdocname| stores the name of the main document
% to be compiled. The macro |\childdocjob| stores the name of
% the document on which the \LaTeX{} compiler was originally invoked.
% The content of |\jobname| cannot be compared
% to filenames specified in the source due to different catcodes.
% The following code rescans |\jobname|, stores the result
% in |\childdocname| and saves a copy in |\childdocjob|:
%    \begin{macrocode}
\edef\childdocname{\scantokens\expandafter{\jobname\noexpand}}
\let\childdocjob\childdocname
%    \end{macrocode}

% \macro{\childdocdisable}
% The macro |\childdocdisable| prevents the main file
% from being processed more than once.
% At this stage, the main document command |\childdocmain|
% is assumed to be called once again where it should do nothing.
% Any subsequent call to it should prevent
% a secondary processing of the main document
% It overwrites the forwarding commands
% |\childdocof| and |\childdocforward|
% with empty macros to prevent further inclusions of the main document:
%    \begin{macrocode}
\newcommand{\childdocdisable}
{
  \renewcommand{\childdocmain}[1]{\renewcommand{\childdocmain}[1]{\endinput}}
  \renewcommand{\childdocof}[1]{}
  \renewcommand{\childdocby}[2][]{}
  \renewcommand{\childdocforward}[2][]{}
  \renewcommand{\childdocdisable}{}
}
%    \end{macrocode}

% \macro{\childdocmain}
% The macro |\childdocmain| is to be called at the top of the main file
% with nothing or the main filename (without extension) as argument.
% First, it breaks loops.
% If the argument is not empty and does not match |\childdocname|
% (which is set by the first inclusion of |childdoc.def|),
% |\ifchilddoc| is set to true, |\includeonly| is applied to the child file
% and |\jobname| is set to the main file
% (for proper handling of |.aux| files):
%    \begin{macrocode}
\newcommand{\childdocmain}[1]
{
  \childdocdisable\childdocmain{}
  \if?#1?\else
    \begingroup
      \def\childdoctmp{#1}
      \ifx\childdoctmp\childdocname
        \def\childdoctmp{}
      \else
        \def\childdoctmp
        {
          \childdoctrue
          \includeonly{\childdocname}
          \def\childdocjob{#1}
          \def\jobname{#1}
        }
      \fi
      \expandafter
    \endgroup
    \childdoctmp
  \fi
}
%    \end{macrocode}

% \macro{\childdocof}
% The command |\childdocof| redirects
% compilation to the main file |#1|.
%    \begin{macrocode}
\newcommand{\childdocof}[1]
{
  \childdocdisable
  \childdoctrue
  \includeonly{\childdocname}
  \def\jobname{#1}
  \def\childdocjob{#1}
  \input{#1}
}
%    \end{macrocode}

% \macro{\childdocby}
% The command |\childdocby| ....
%    \begin{macrocode}
\newcommand{\childdocby}[2][]
{
  \childdocdisable
  \childdoctrue
  \childdocmanualtrue
  \if?#1?\else
    \def\jobname{#2}
  \fi
  \def\childdocjob{#2}
  \input{#2}
  \endinput
}
%    \end{macrocode}

% \macro{\childdocforward}
% The command |\childdocforward| redirects
% compilation to the main file or
% (if the optional argument is given) a child file.
% Parameters are set as if the main file
% or a child file starting with |\childdocof| was compiled.
% Then compilation is handed over to the main file:
%    \begin{macrocode}
\newcommand{\childdocforward}[2][]
{
  \begingroup
    \if?#1?
      \def\childdoctmp
      {
        \def\childdocname{#2}
        \def\childdocjob{#2}
        \def\jobname{#2}
        \input{#2}
        \endinput
      }
    \else
      \def\childdoctmp
      {
        \childdocdisable
        \def\childdocname{#2}
        \childdoctrue
        \includeonly{#2}
        \def\childdocjob{#1}
        \def\jobname{#1}
        \input{#1}
        \endinput
      }
    \fi
    \expandafter
  \endgroup
  \childdoctmp
}
%    \end{macrocode}

% \macro{\childdocforwardprefix}
% The command |\childdocforwardprefix| redirects
% compilation to the main or a child file by means of a pattern.
% The prefix |#1| in the current filename is replaced by |#2|
% and the suffix of the current filename is kept
% (it is assumed that the filename does not contain the substring `|~~~|'
% which is used as a delimiter).
% Compilation is handed over to the new file by |\childdocforward|:
%    \begin{macrocode}
\newcommand{\childdocforwardprefix}[3][]
{
  \begingroup
    \def\childdocextract #2##1~~~{\def\childdoctmp{\childdocforward[#1]{#3##1}}}
    \expandafter\childdocextract\childdocname~~~
    \expandafter
  \endgroup
  \childdoctmp
}
%    \end{macrocode}

% \macro{\childdoc}
% The deprecated macro |\childdoc| is a legacy version of |\childdocmain|:
%    \begin{macrocode}
\newcommand{\childdoc}{\childdocmain}
%    \end{macrocode}

% \macro{\childdocredirect}
% The deprecated macro |\childdocredirect| is a legacy version
% of |\childdocforward| and |\childdocforwardprefix|:
%    \begin{macrocode}
\newcommand{\childdocredirect}[2][]
{
  \begingroup
    \if?#1?
      \def\childdoctmp{\childdocforward{#2}}
    \else
      \def\childdoctmp{\childdocforwardprefix{#1}{#2}}
    \fi
    \expandafter
  \endgroup
  \childdoctmp
}
%    \end{macrocode}

%\iffalse
%</package>
%\fi
%
\endinput

\childdocmain{}
%    \end{macrocode}

% Optional override for |\version| flag:
%    \begin{macrocode}
%%\ifchilddoc\else\providecommand{\version}{draft}\fi
%    \end{macrocode}

% Define the default values for the |\version| flag
% (|final| for the main file and |draft| for childs):
%    \begin{macrocode}
\ifchilddoc
\providecommand{\version}{draft}
\else
\providecommand{\version}{final}
\fi
%    \end{macrocode}

% Load the standard document class:
%    \begin{macrocode}
\documentclass[12pt]{article}
%    \end{macrocode}

% Start the document body:
%    \begin{macrocode}
\begin{document}
%    \end{macrocode}

% Declare a title page.
% Print title, part of document being processed and version flag:
%    \begin{macrocode}
\addtocounter{page}{-1}
\begin{center}
{\LARGE\bfseries{}childdoc example\par}
\vspace{1cm}
\ifchilddoc
\ifchilddocmanual part\else chapter\fi:
`\childdocname' of `\childdocjob'\par
\else
main document: `\childdocjob'\par
\fi
version: \version\par
\end{center}
\newpage
%    \end{macrocode}

% Manually include selected file,
% otherwise process as usual:
%    \begin{macrocode}
\ifchilddocmanual
\section*{part `\childdocname'}
\input{\childdocname}
\else
%    \end{macrocode}

% Include the two chapters:
%    \begin{macrocode}
\include{cdocsch1}
\include{cdocsch2}
%    \end{macrocode}

% Include the two parts unless only chapters should be displayed:
%    \begin{macrocode}
\ifchilddoc\else
\section{part three}
\input{cdocspt3}
\section{part four}
\input{cdocspt4}
\fi
%    \end{macrocode}

% Process as usual until here:
%    \begin{macrocode}
\fi
%    \end{macrocode}

% End of document body:
%    \begin{macrocode}
\end{document}
%    \end{macrocode}
%\iffalse
%</samplemain>
%\fi
%
% %%%%%%%%%%%%%%%%%%%%%%%%%%%%%%%%%%%%%%
% \paragraph{Chapter Include Files.}
%
% The include files are called |cdocsch1.tex| and |cdocsch2.tex|.
%
%\iffalse
%<*samplechap1|samplechap2>
%\fi

% Optional override for |\version| flag:
%    \begin{macrocode}
%%\providecommand{\version}{final}
%    \end{macrocode}

% Include the main document:
%    \begin{macrocode}
% \iffalse
%
% childdoc.dtx Copyright (C) 2017-2018 Niklas Beisert
%
% This work may be distributed and/or modified under the
% conditions of the LaTeX Project Public License, either version 1.3
% of this license or (at your option) any later version.
% The latest version of this license is in
%   http://www.latex-project.org/lppl.txt
% and version 1.3 or later is part of all distributions of LaTeX
% version 2005/12/01 or later.
%
% This work has the LPPL maintenance status `maintained'.
%
% The Current Maintainer of this work is Niklas Beisert.
%
% This work consists of the files childdoc.dtx and childdoc.ins
% and the derived files childdoc.def and cdocsamp.tex with
% cdocsch1.tex, cdocsch2.tex, cdocsdrf.tex, cdocsfn1.tex, cdocsfn2.tex.
%
%<package>\ifdefined\childdocmain\endinput\fi
%<package>\ProvidesFile{childdoc.def}[2018/12/30 v2.0 child document driver]
%<samplemain>\ProvidesFile{cdocsamp.tex}[2018/12/30 v2.0 sample for childdoc]
%<*driver>
%\ProvidesFile{childdoc.drv}[2018/12/30 v2.0 childdoc reference manual file]
\PassOptionsToClass{10pt,a4paper}{article}
\documentclass{ltxdoc}

\usepackage[margin=35mm]{geometry}
\usepackage{hyperref}
\usepackage{hyperxmp}
\usepackage[usenames]{color}

\hypersetup{colorlinks=true}
\hypersetup{pdfstartview=FitH}
\hypersetup{pdfpagemode=UseNone}
\hypersetup{pdfsource={}}
\hypersetup{pdflang={en-UK}}
\hypersetup{pdfcopyright={Copyright 2017-2018 Niklas Beisert.
  This work may be distributed and/or modified under the
  conditions of the LaTeX Project Public License, either version 1.3
  of this license or (at your option) any later version.}}
\hypersetup{pdflicenseurl={http://www.latex-project.org/lppl.txt}}
\hypersetup{pdfcontactaddress={ETH Zurich, ITP, HIT K,
  Wolfgang-Pauli-Strasse 27}}
\hypersetup{pdfcontactpostcode={8093}}
\hypersetup{pdfcontactcity={Zurich}}
\hypersetup{pdfcontactcountry={Switzerland}}
\hypersetup{pdfcontactemail={nbeisert@itp.phys.ethz.ch}}
\hypersetup{pdfcontacturl={http://people.phys.ethz.ch/\xmptilde nbeisert/}}

\newcommand{\secref}[1]{\hyperref[#1]{section \ref*{#1}}}

\parskip1ex
\parindent0pt
\let\olditemize\itemize
\def\itemize{\olditemize\parskip0pt}

\begin{document}

\title{The \textsf{childdoc} Package}
\hypersetup{pdftitle={The childdoc Package}}
\author{Niklas Beisert\\[2ex]
  Institut f\"ur Theoretische Physik\\
  Eidgen\"ossische Technische Hochschule Z\"urich\\
  Wolfgang-Pauli-Strasse 27, 8093 Z\"urich, Switzerland\\[1ex]
  \href{mailto:nbeisert@itp.phys.ethz.ch}
  {\texttt{nbeisert@itp.phys.ethz.ch}}}
\hypersetup{pdfauthor={Niklas Beisert}}
\hypersetup{pdfsubject={Manual for the LaTeX2e Package childdoc}}
\date{30 December 2018, \textsf{v2.0}}
\maketitle

\begin{abstract}\noindent
\textsf{childdoc} is a \LaTeXe{} package
that enables the direct compilation
of document sections included by |\include|
to individual files.
\end{abstract}

\begingroup
\parskip0ex
\tableofcontents
\endgroup

%%%%%%%%%%%%%%%%%%%%%%%%%%%%%%%%%%%%%%%%%%%%%%%%%%%%%%%%%%%%%%%%%%%%%%%%%%%%%%%%
%%%%%%%%%%%%%%%%%%%%%%%%%%%%%%%%%%%%%%%%%%%%%%%%%%%%%%%%%%%%%%%%%%%%%%%%%%%%%%%%
\section{Introduction}

\LaTeX{} provides a mechanism to structure a large document (such as a book)
into a main file and several child files (containing the chapters)
using the |\include| command.
This mechanism is beneficial for documents
which span hundreds of pages in order to
make the source file(s) more manageable.
Moreover, compilation can be restricted to
selected child files by means of the |\includeonly| command.
The latter feature can be used to reduce the compilation time while editing
(this was significantly more useful in the earlier days of \LaTeX{})
or to generate a smaller document which is easier to navigate.
Another application of |\includeonly| is to generate
documents consisting of selected parts of the complete document.

However, there are a few drawbacks of the plain |\include| mechanism:
\begin{itemize}
\item
The child files cannot be compiled on their own,
they can only be compiled via the main file.
A naive editing environment
(such as a text editor with an option
to have the current file processed by \LaTeX)
may require one to switch to the main file before compiling;
attempting to compile the child file produces errors.
\item
The main file must be modified (each time)
to adjust the |\includeonly| command
to the present needs. This easily leaves the main file in a messy state.
\item
The generated document will always carry the filename
of the main document. This is inconvenient if
several child files are to be compiled and
to be kept for distribution.
\end{itemize}

The present package provides a simple interface
to make child files individually compilable by \LaTeX{}.
Compiling a child file then has the same effect as compiling
the main file with an |\includeonly| command
to select the appropriate child.
Moreover the generated document will carry the name of the child
rather than the main file.
This resolves all three above issues.

This feature is meant to make the editing of books,
thesis documents and lecture notes somewhat more convenient.
However, the package can also be used efficiently for
composing a series of documents (such as exercise sheets)
which are typically distributed individually.
It then assists the author in generating the individual documents
(potentially in different versions)
as well as a document containing the collected series.
Another application is in developing style files
or other kinds of included material
where compilation of the style file could redirect
to a sample or test file.

%%%%%%%%%%%%%%%%%%%%%%%%%%%%%%%%%%%%%%%%%%%%%%%%%%%%%%%%%%%%%%%%%%%%%%%%%%%%%%%%
%%%%%%%%%%%%%%%%%%%%%%%%%%%%%%%%%%%%%%%%%%%%%%%%%%%%%%%%%%%%%%%%%%%%%%%%%%%%%%%%
\section{Usage}

First of all, the package \textsf{childdoc} is \emph{not} a standard
\LaTeXe{} |.sty| style file! Therefore it needs to be invoked in
a non-standard way.

%%%%%%%%%%%%%%%%%%%%%%%%%%%%%%%%%%%%%%%%%%%%%%%%%%%%%%%%%%%%%%%%%%%%%%%%%%%%%%%%
\subsection{Included Files}
\label{sec:include}

%%%%%%%%%%%%%%%%%%%%%%%%%%%%%%%%%%%%%%%%
\DescribeMacro{\childdocmain}
To use the package, add the commands
\begin{center}
\begin{tabular}{l}
|\input{childdoc.def}|\\
|\childdocmain{}|\\
\end{tabular}
\end{center}
at the very top of the main \LaTeX{} file,
in particular \emph{before} the |\documentclass| statement!
The argument of |\childdocmain| should be left empty
(but it must be present).

%%%%%%%%%%%%%%%%%%%%%%%%%%%%%%%%%%%%%%%%
\DescribeMacro{\childdocof}
Furthermore, add the commands
\begin{center}
\begin{tabular}{l}
|\input{childdoc.def}|\\
|\childdocof{|\textit{main}|}|\\
\end{tabular}
\end{center}
at the top of every child file \textit{child}
which is included by |\include{|\textit{child}|}|
from within the main file
(or at least for those files to be compiled individually).
The argument \textit{main} must be the filename of the main file.

There are a couple of
considerations in setting up the main and child documents:

%%%%%%%%%%%%%%%%%%%%%%%%%%%%%%%%%%%%%%%%
\paragraph{Restrictions.}

Please note the following restrictions:
\begin{itemize}
\item
|\childdocmain| must be called with one argument \textit{main}
to ensure compatibility with earlier version of the package.
It must either be empty (|\childdocmain{}|)
or precisely match the filename of the main file in which it is specified.
See \secref{sec:detection} for further information.
\item
The filename \textit{main} must be specified without the |.tex| extension.
\item
The filename \textit{main} is case sensitive
(even in case-insensitive file systems)
due to internal string comparison.
\item
The argument \textit{main} should be fully expanded, it cannot be a macro.
\item
Subdirectories and special characters should be avoided in filenames.
\item
The command |\childdocmain{|\textit{main}|}| must be followed by a whitespace.
It should not be followed immediately by another command
or by a comment mark `|%|'.
This is because the \TeX{} parser reads the token immediately following
the argument of |\childdocmain| and puts it
at the beginning of every child section;
however, a white\-space is ignored.
\end{itemize}

%%%%%%%%%%%%%%%%%%%%%%%%%%%%%%%%%%%%%%%%
\paragraph{Content of Main File.}

It is advisable to place all content in the child files included by |\include|.
Any output contained in the main file will appear in all child documents
unless suppressed manually;
it cannot be suppressed automatically by the |\includeonly| directive
and thus should normally be avoided.
A method to include some content in the main file
by means of conditional processing is described in \secref{sec:conditional}.

%%%%%%%%%%%%%%%%%%%%%%%%%%%%%%%%%%%%%%%%
\paragraph{Page Numbering.}

When only a part of the document is compiled,
the appropriate numbering of pages
(as well as other status parameters)
is determined from the |.aux| files.
The latter contain information from previous passes.
However this information needs to propagate through
all intermediate child documents.
Therefore the page numbering in child documents may well
be inconsistent until the complete document is compiled at least once.

A useful (if unconventional) way to always ensure a consistent
page numbering is to restart the numbering in each child document
and denote the pages by `\textit{child}|.|\textit{page}'
where \textit{child} represents the chapter/section number of the child file.
This can be achieved by the command
|\numberwithin{page}{|\textit{child}|}|
of the \textsf{amsmath} package
where \textit{child} can be |chapter| or |section|
depending on the chosen structuring.
Alternatively, one can modify the macro |\thepage| appropriately
and reset the counter |page| at the start of each child file.

%%%%%%%%%%%%%%%%%%%%%%%%%%%%%%%%%%%%%%%%%%%%%%%%%%%%%%%%%%%%%%%%%%%%%%%%%%%%%%%%
\subsection{Conditional Processing}
\label{sec:conditional}

The package provides a mechanism to compile different versions
of a document. To customise the versions further some conditional processing
can come in handy to distinguish which version is being compiled.
The package provides two macros to describe the compilation context:

%%%%%%%%%%%%%%%%%%%%%%%%%%%%%%%%%%%%%%%%
\DescribeMacro{\ifchilddoc}
The conditional |\ifchilddoc| distinguishes between the compilation of
child documents and the main document:
%
\begin{center}
|\ifchilddoc |\textit{child-code}| |[|\||else |\textit{main-code}]| \||fi|
\end{center}

%%%%%%%%%%%%%%%%%%%%%%%%%%%%%%%%%%%%%%%%
\DescribeMacro{\childdocname}
\DescribeMacro{\childdocjob}
The macro |\childdocname| contains the filename (without extension)
of the main or child file being processed.
Note that |\childdocjob| will always contain the name of the main file.

%%%%%%%%%%%%%%%%%%%%%%%%%%%%%%%%%%%%%%%%
\paragraph{Title Page.}

Conditional processing can be used to include a title or banner page
in the main document when proper precautions are taken.
Importantly, the code in the main file should ensure that the page counter
(as well as other status parameters which are stored in the |.aux| files)
takes the same value after the conditional processing.
Otherwise the page numbers may take divergent values
depending on which part is compiled.

For example, a title page could be declared by:
%
\begin{center}
\begin{tabular}{l}
|\ifchilddoc\||else|\\
|\addtocounter{page}{-1}|\\
\textit{code for title page}\\
|\newpage|\\
|\||fi|
\end{tabular}
\end{center}
%
A banner page for the child documents can be generated by:
%
\begin{center}
\begin{tabular}{l}
|\ifchilddoc|\\
|\addtocounter{page}{-1}|\\
\textit{code for banner page}\\
|\newpage|\\
|\||fi|
\end{tabular}
\end{center}
%
Here one could write a message such as:
\begin{center}
|This is the part \childdocname{} of \childdocjob{}.|
\end{center}

%%%%%%%%%%%%%%%%%%%%%%%%%%%%%%%%%%%%%%%%%%%%%%%%%%%%%%%%%%%%%%%%%%%%%%%%%%%%%%%%
\subsection{Flags}
\label{sec:flags}

The package makes it easy to generate different versions
of the main or child documents.
To this end compilation flags can be defined
and assigned different default values.
They will be particularly useful in conjunction
with the forwarding mechanism described in \secref{sec:forward}.

For example, it may be useful to have a flag |\version|
which can be set to |draft| or |final|.
The document source will contain some conditional code
depending on the value of |\version|.
Suppose further, the flag should default to |final| for the main file
and to |draft| for child files
which is a natural assignment for editing the document.
This is achieved by placing the following code
in the preamble of the main document
(below the |\childdocmain| directive):
%
\begin{center}
\begin{tabular}{l}
|\ifchilddoc|\\
|\providecommand{\version}{draft}|\\
|\||else|\\
|\providecommand{\version}{final}|\\
|\||fi|
\end{tabular}
\end{center}
%
The definition by |\providecommand| makes sure
that previous definitions are not overwritten.
Further statements |\providecommand{\version}{...}|
can thus be added before the above code to override it.

For the main file, one might add a line
(between |\childdocmain| and the above block)
%
\begin{center}
|%\ifchilddoc\||else\providecommand{\version}{draft}\||fi|
\end{center}
%
which can be uncommented to produce a draft version.
Likewise one can add a line to the very top of a child file
(above the |\childdocof{|\textit{main}|}| directive)
%
\begin{center}
|%\providecommand{\version}{final}|
\end{center}
%
which can be uncommented to produce the final version of this child document.

%%%%%%%%%%%%%%%%%%%%%%%%%%%%%%%%%%%%%%%%%%%%%%%%%%%%%%%%%%%%%%%%%%%%%%%%%%%%%%%%
\subsection{Forwarding}
\label{sec:forward}

Different versions of the main or child documents
using compilation flags as described in \secref{sec:flags}
can be (permanently) stored in different files
for convenient compilation, viewing and distribution.
To this end, the package defines a command
to pass on compilation to a different file:

%%%%%%%%%%%%%%%%%%%%%%%%%%%%%%%%%%%%%%%%
\DescribeMacro{\childdocforward}
The command |\childdocforward| redirects processing to
another source file:
%
\begin{center}
\begin{tabular}{l}
|\input{childdoc.def}|\\
|\childdocforward[|\textit{main}|]{|\textit{dest}|}|\\
\end{tabular}
\end{center}
%
The argument \textit{dest} is the destination file
(without extension).
It should be the main file or one of the child files.
Note that further \textsf{childdoc} directives
such as |\childdocof| and |\childdocforward|
in the indicated file will be processed in this form.
The optional argument \textit{main}
passes on directly to the main file \textit{main}
while pretending to compile the child \textit{dest}.
This form behaves as if \textit{dest}
issues |\childdocof{|\textit{main}|}| right away,
and no further \textsf{childdoc} directives will be processed.

%%%%%%%%%%%%%%%%%%%%%%%%%%%%%%%%%%%%%%%%
\DescribeMacro{\...prefix}
In the alternative form |\childdocforwardprefix|,
%
\begin{center}
\begin{tabular}{l}
|\input{childdoc.def}|\\
|\childdocforwardprefix[|\textit{main}|]{|\textit{prefix}|}{|\textit{dest}|}|
\end{tabular}
\end{center}
%
the destination file is determined by a pattern
depending on the current file:
To make this work, the current file must be called
`{\textit{prefix}\hspace{0.2em}\textit{suffix}}'
with \textit{prefix} matching precisely the argument.
Processing is then passed on to the file
`{\textit{dest}\hspace{0.2em}\textit{suffix}}'.
Surely, the same effect is achieved by
directly specifying the
argument `{\textit{dest}\hspace{0.2em}\textit{suffix}}'
in the first form.
However, that requires to set up a different file
for each child. With the alternative form of the command
all these files can have exactly the same content
which simplifies setting them up and maintaining them.

For example, the following file |draft.tex|
with a compilation flag |\version| as described in \secref{sec:flags}
compiles the main document as a draft:
%
\begin{center}
\begin{tabular}{l}
|\def\version{draft}|\\
|\input{childdoc.def}|\\
|\childdocforward{|\textit{main}|}|
\end{tabular}
\end{center}
%
Likewise, the following files |final|\textit{nn}|.tex|
compile the final version of the child document
|child|\textit{nn}|.tex|:
%
\begin{center}
\begin{tabular}{l}
|\def\version{final}|\\
|\input{childdoc.def}|\\
|\childdocforwardprefix{final}{child}|
\end{tabular}
\end{center}
%

Note that when several versions of a main file and/or of each child file
are to be generated, it may be convenient to set up a |Makefile| or
shell script to automatise the process.

%%%%%%%%%%%%%%%%%%%%%%%%%%%%%%%%%%%%%%%%%%%%%%%%%%%%%%%%%%%%%%%%%%%%%%%%%%%%%%%%
\subsection{Command Line Processing}
\label{sec:commandline}

The effect of redirection files can also be achieved by invoking
the \LaTeX{} compiler with a more elaborate command line.
Most conveniently this should be done as part
of a shell script or a |Makefile|.

When using \textsf{childdoc} in the main file, the following
command lines effectively perform a redirection
(note that depending on the shell being used,
backslashes may have to be doubled: `|\|' $\to$ `|\\|'):
%
\begin{center}
|... -jobname "|\textit{target}|" |\\|"|[\textit{flags}]%
|\input{childdoc.def}\childdocforward[|\textit{main}|]{|\textit{dest}|}"|
\end{center}
%
Here \textit{target} is the name of the output file,
\textit{main} is the name of the main file
and \textit{dest} is the name of the main or child file to be processed
(all filenames without extensions).
The optional argument \textit{main} can be omitted
if \textit{main} matches \textit{dest}.
Optionally, compilation \textit{flags} can be defined via |\def| commands.
This command line makes the \TeX{} engine believe
it is compiling the file \textit{target}
whose content is specified as the latter parameter.
The provided code then forwards the processing to
\textit{main} or \textit{dest} as described in \secref{sec:forward}.

%%%%%%%%%%%%%%%%%%%%%%%%%%%%%%%%%%%%%%%%%%%%%%%%%%%%%%%%%%%%%%%%%%%%%%%%%%%%%%%%
\subsection{Include by Input}
\label{sec:input}

Including child documents by |\include| has some restrictions by design.
Most notably, the content of a child document always occupies
its own set of pages; pages cannot be shared between child documents.
Usually, this behaviour makes perfect sense
because each child document contain an essential part of the document.
However, in some situations it may be desirable to compose
a document from a collection of parts
without having mandatory page breaks between then.
For this case, the package
provides a mechanism to include parts
by |\input| which can also be processed individually.
However, by construction this mechanism
requires manual handling of the content to be output.

%%%%%%%%%%%%%%%%%%%%%%%%%%%%%%%%%%%%%%%%
\DescribeMacro{\ifchilddocmanual}
The main file should be prepared as usual, see \secref{sec:include}.
However, the document body must make a distinction
between processing of an individual part and of the main document, e.g.:
%
\begin{center}
\begin{tabular}{l}
|\ifchilddocmanual|\\
|\input{\childdocname}|\\
|\||else|\\
\textit{document body with }|\input{|\textit{part}|}|\\
|\||fi|
\end{tabular}
\end{center}
%
The conditional |\ifchilddocmanual| is true whenever
a part to be included by |\input| is being compiled,
and the name of the part is stored in |\childdocname|.

%%%%%%%%%%%%%%%%%%%%%%%%%%%%%%%%%%%%%%%%
\DescribeMacro{\childdocby}
Each part to be included by |\input| should start with:
%
\begin{center}
\begin{tabular}{l}
|\input{childdoc.def}|\\
|\childdocby{|\textit{main}|}|\\
\end{tabular}
\end{center}
%
The directive |\childdocby| is similar to |\childdocof|
described in \secref{sec:include},
but the subsequent selection of content must be done manually.
To that end, both |\ifchilddoc| and |\ifchilddocmanual|
will be true upon processing of a part,
and the name of the part is stored in |\childdocname|.
Note that |\jobname| will be set to the filename of the current part
so that each part receives an individual |.aux| file
that does not interfere with the |.aux| file(s) of the main document.
This behaviour can be altered by the alternative form
|\childdocby[*]{|\textit{main}|}| (with a non-empty optional argument)
which uses the |.aux| file of the main document
by setting |\jobname| to \textit{main}.

%%%%%%%%%%%%%%%%%%%%%%%%%%%%%%%%%%%%%%%%%%%%%%%%%%%%%%%%%%%%%%%%%%%%%%%%%%%%%%%%
\subsection{Driver Development}
\label{sec:driver}

The \textsf{childdoc} mechanism can also be use for the development
of definition files such as \LaTeX{} styles or classes.
This case differs from the above setup with multiple parts
included by |\include| in that no |\includeonly| should be invoked.
This can be achieved by starting the include file
(before |\ProvidesPackage|) with:
%
\begin{center}
\begin{tabular}{l}
|\input{childdoc.def}|\\
|\childdocforward{|\textit{main}|}|\\
\end{tabular}
\end{center}
%
or alternatively with:
%
\begin{center}
\begin{tabular}{l}
|\input{childdoc.def}|\\
|\childdocby{|\textit{main}|}|\\
\end{tabular}
\end{center}
%
Both forms have slightly different effects as described above.
The main file is prepared as usual, see \secref{sec:include}.

%%%%%%%%%%%%%%%%%%%%%%%%%%%%%%%%%%%%%%%%%%%%%%%%%%%%%%%%%%%%%%%%%%%%%%%%%%%%%%%%
\subsection{Legacy Detection}
\label{sec:detection}

The directive |\childdocmain| in the main file can detect
whether the complete document or merely a child is to be compiled
even without using the directive |\childdocof|.
This method is deprecated because it is less robust
and there is no compelling reason to use it;
it is merely provided for backward compatibility
and it may be removed in future versions.

If the detection mechanism is to be used,
it is mandatory to correctly specify
the filename of the main file as the argument of |\childdocmain|:
%
\begin{center}
\begin{tabular}{l}
|\input{childdoc.def}|\\
|\childdocmain{|\textit{main}|}|\\
\end{tabular}
\end{center}
%
If |\jobname| does not match the argument \textit{main} of |\childdocmain|,
it is assumed that |\jobname| points to the child file to be compiled.
When using |\childdocmain| with the main file specified as argument,
it suffices to start a child file
with just |\input{|\textit{main}|}|
without loading of the package and using |\childdocof|.
If instead all processing is done
with the appropriate \textsf{childdoc} directives,
the argument of \textit{main} of |\childdocmain| can be empty.

An alternative version of the command line processing described
in \secref{sec:commandline} using the detection mechanism reads:
%
\begin{center}
|... -jobname "|\textit{target}|" "|[\textit{flags}]%
[|\def\jobname{|\textit{dest}|}|]|\input{|\textit{main}|}"|
\end{center}

%%%%%%%%%%%%%%%%%%%%%%%%%%%%%%%%%%%%%%%%%%%%%%%%%%%%%%%%%%%%%%%%%%%%%%%%%%%%%%%%
\subsection{Manual Code}
\label{sec:manual}

In case one cannot be certain whether the definitions file |childdoc.def|
is installed on the target \TeX{} distribution
and one prefers not to ship it,
it is conceivable to paste a few relevant commands into the sources.

To that end, drop all statements |\input{childdoc.def}|
and perform the replacements as outlined below.
Instead of |\childdocmain{|\textit{main}|}| add the following code
to the top of the main file:
%
\begin{center}
\begin{tabular}{l}
|\||ifdefined\childdocname\endinput\||fi\newif\ifchilddoc|\\
|\edef\childdocname{\scantokens\expandafter{\jobname\noexpand}}|\\
|\def\childdocmain{|\textit{main}|}\||ifx\childdocmain\childdocname\||else|\\
|\childdoctrue\includeonly{\childdocname}\let\jobname\childdocmain\||fi|\\
\end{tabular}
\end{center}
%
Instead of |\childdocof{|\textit{main}|}| just include the main file
at the top of each child file:
%
\begin{center}
|\input{|\textit{main}|}|
\end{center}
%
A simple redirection |\childdocforward{|\textit{dest}|}| is achieved by:
%
\begin{center}
|\def\jobname{|\textit{dest}|}\input{\jobname}|
\end{center}
%
The redirection with prefix
|\childdocforwardprefix[|\textit{prefix}|]{|\textit{dest}|}|
is accomplished by:
%
\begin{center}
\begin{tabular}{l}
|{\edef\jobname{\scantokens\expandafter{\jobname\noexpand}}|\\
|\def\redirectjob |\textit{prefix}|#1~~~{\gdef\jobname{|\textit{dest}|#1}}|\\
|\expandafter\redirectjob\jobname~~~}\input{\jobname}|
\end{tabular}
\end{center}

In an alternative approach,
child documents can be compiled by a specific command line
without additional code or specific definitions:
%
\begin{center}
|... -jobname "|\textit{target}|" "|[\textit{flags}]%
|\includeonly{|\textit{dest}|}\input{|\textit{main}|}"|
\end{center}
%

%%%%%%%%%%%%%%%%%%%%%%%%%%%%%%%%%%%%%%%%%%%%%%%%%%%%%%%%%%%%%%%%%%%%%%%%%%%%%%%%
%%%%%%%%%%%%%%%%%%%%%%%%%%%%%%%%%%%%%%%%%%%%%%%%%%%%%%%%%%%%%%%%%%%%%%%%%%%%%%%%
\section{Information}

%%%%%%%%%%%%%%%%%%%%%%%%%%%%%%%%%%%%%%%%%%%%%%%%%%%%%%%%%%%%%%%%%%%%%%%%%%%%%%%%
\subsection{Copyright}

Copyright \copyright{} 2017--2018 Niklas Beisert

This work may be distributed and/or modified under the
conditions of the \LaTeX{} Project Public License, either version 1.3
of this license or (at your option) any later version.
The latest version of this license is in
  \url{http://www.latex-project.org/lppl.txt}
and version 1.3 or later is part of all distributions of \LaTeX{}
version 2005/12/01 or later.

This work has the LPPL maintenance status `maintained'.

The Current Maintainer of this work is Niklas Beisert.

This work consists of the files |README.txt|, |childdoc.ins| and |childdoc.dtx|
as well as the derived files |childdoc.def|, |cdocsamp.tex|
with |cdocsch1.tex|, |cdocsch2.tex|, |cdocspt3.tex|, |cdocspt4.tex|,
|cdocsdrf.tex|, |cdocsfn1.tex|, |cdocsfn2.tex|
as well as |childdoc.pdf|.

%%%%%%%%%%%%%%%%%%%%%%%%%%%%%%%%%%%%%%%%%%%%%%%%%%%%%%%%%%%%%%%%%%%%%%%%%%%%%%%%
\subsection{Files and Installation}

The package consists of the files:
%
\begin{center}
\begin{tabular}{ll}
    |README.txt|   & readme file \\
    |childdoc.ins| & installation file \\
    |childdoc.dtx| & source file \\
    |childdoc.def| & definition file \\
    |cdocsamp.tex| & sample main file \\
    |cdocsch1.tex| & sample include file \\
    |cdocsch2.tex| & sample include file \\
    |cdocspt3.tex| & sample part file \\
    |cdocspt4.tex| & sample part file \\
    |cdocsdrf.tex| & sample redirection file \\
    |cdocsfn1.tex| & sample redirection file \\
    |cdocsfn2.tex| & sample redirection file \\
    |childdoc.pdf| & manual
\end{tabular}
\end{center}
%
The distribution consists of the files
|README.txt|, |childdoc.ins| and |childdoc.dtx|.
%
\begin{itemize}
\item
Run (pdf)\LaTeX{} on |childdoc.dtx|
to compile the manual |childdoc.pdf| (this file).
\item
Run \LaTeX{} on |childdoc.ins| to create the definitions file |childdoc.def|
and the sample |cdocsamp.tex| with include files
|cdocsch1.tex|, |cdocsch2.tex|, |cdocspt3.tex|, |cdocspt4.tex|,
|cdocsdrf.tex|, |cdocsfn1.tex|, |cdocsfn2.tex|.
Then copy the file |childdoc.def| to an appropriate directory of your \LaTeX{}
distribution, e.g.\ \textit{texmf-root}|/tex/latex/childdoc|.
\end{itemize}

%%%%%%%%%%%%%%%%%%%%%%%%%%%%%%%%%%%%%%%%%%%%%%%%%%%%%%%%%%%%%%%%%%%%%%%%%%%%%%%%
\subsection{Related CTAN Packages}

There are several other packages which offer a similar functionality:
%
\begin{itemize}
\item
The packages
\href{http://ctan.org/pkg/docmute}{\textsf{docmute}},
\href{http://ctan.org/pkg/includex}{\textsf{includex}} and
\href{http://ctan.org/pkg/standalone}{\textsf{standalone}}
provide commands to include only the document body of
a child file thus allowing both files to be compiled individually.
\item
The packages \href{http://ctan.org/pkg/subdocs}{\textsf{subdocs}}
and \href{http://ctan.org/pkg/subfiles}{\textsf{subfiles}}
provide structures in which the main and child documents can be
encapsulated and allowing them to be compiled individually.
The inclusion mechanism is different from the conventional |\include|.
\item
The package \href{http://ctan.org/pkg/combine}{\textsf{combine}}
is an elaborate solution to combine several documents into one.
\end{itemize}
%
See also the CTAN topic \href{http://ctan.org/topic/subdocs}{\textsf{subdocs}}
for further related packages.
The present package differs from the above solutions in that
a document structure constructed with the conventional |\include| mechanism
just needs two extra commands at the top of every file
such that all constituent files can be compiled individually.

%%%%%%%%%%%%%%%%%%%%%%%%%%%%%%%%%%%%%%%%%%%%%%%%%%%%%%%%%%%%%%%%%%%%%%%%%%%%%%%%
%\subsection{Feature Suggestions}
%
%The following is a list of features which may be useful for future
%versions of this package:
%%
%\begin{itemize}
%\item
%\ldots
%\end{itemize}

%%%%%%%%%%%%%%%%%%%%%%%%%%%%%%%%%%%%%%%%%%%%%%%%%%%%%%%%%%%%%%%%%%%%%%%%%%%%%%%%
\subsection{Revision History}

%%%%%%%%%%%%%%%%%%%%%%%%%%%%%%%%%%%%%%%%
\paragraph{v2.0:} 2018/12/30

\begin{itemize}
\item
immediate forward processing
\item
added |\childdocby| mechanism
\item
manual restructured
\end{itemize}

%%%%%%%%%%%%%%%%%%%%%%%%%%%%%%%%%%%%%%%%
\paragraph{v1.6:} 2018/01/17

\begin{itemize}
\item
application for development of include files
\item
corrections to manual
\end{itemize}

%%%%%%%%%%%%%%%%%%%%%%%%%%%%%%%%%%%%%%%%
\paragraph{v1.5:} 2017/05/21

\begin{itemize}
\item
more complete structuring introduced
\item
|\childdocof| introduced
\item
|\childdoc| renamed to |\childdocmain|
\item
|\childredirect| renamed to |\childdocforward| and |\childdocforwardprefix|
and functionality expanded
\end{itemize}

%%%%%%%%%%%%%%%%%%%%%%%%%%%%%%%%%%%%%%%%
\paragraph{v1.0:} 2017/04/27

\begin{itemize}
\item
manual and install package
\item
first version published on CTAN
\end{itemize}

%%%%%%%%%%%%%%%%%%%%%%%%%%%%%%%%%%%%%%%%
\paragraph{v0.6:} 2017/04/26

\begin{itemize}
\item
redirection mechanism added
\end{itemize}

%%%%%%%%%%%%%%%%%%%%%%%%%%%%%%%%%%%%%%%%
\paragraph{v0.5:} 2017/04/26

\begin{itemize}
\item
functionality in definition file
\end{itemize}


%%%%%%%%%%%%%%%%%%%%%%%%%%%%%%%%%%%%%%%%%%%%%%%%%%%%%%%%%%%%%%%%%%%%%%%%%%%%%%%%
%%%%%%%%%%%%%%%%%%%%%%%%%%%%%%%%%%%%%%%%%%%%%%%%%%%%%%%%%%%%%%%%%%%%%%%%%%%%%%%%
%%%%%%%%%%%%%%%%%%%%%%%%%%%%%%%%%%%%%%%%%%%%%%%%%%%%%%%%%%%%%%%%%%%%%%%%%%%%%%%%
\appendix

\settowidth\MacroIndent{\rmfamily\scriptsize 000\ }

 \DocInput{childdoc.dtx}

\end{document}
%</driver>
% \fi
%
% %%%%%%%%%%%%%%%%%%%%%%%%%%%%%%%%%%%%%%%%%%%%%%%%%%%%%%%%%%%%%%%%%%%%%%%%%%%%%%
% %%%%%%%%%%%%%%%%%%%%%%%%%%%%%%%%%%%%%%%%%%%%%%%%%%%%%%%%%%%%%%%%%%%%%%%%%%%%%%
% \section{Sample}
%\iffalse
%<*samplemain>
%\fi
%
% The following presents a sample document
% with two chapters, two parts, a title page,
% a compile flag as well as three forwarding files to set the flag.
% It consists of eight |.tex| files:
% \begin{center}
% \begin{tabular}{ll}
% |cdocsamp.tex|&main file\\
% |cdocsch1.tex|&include file for chapter 1\\
% |cdocsch2.tex|&include file for chapter 2\\
% |cdocspt3.tex|&include file for part 3\\
% |cdocspt4.tex|&include file for part 4\\
% |cdocsdrf.tex|&forwarding file for main file in draft mode\\
% |cdocsfi1.tex|&forwarding file for final version of chapter 1\\
% |cdocsfi2.tex|&forwarding file for final version of chapter 2\\
% \end{tabular}
% \end{center}
% Each of the eight files can be compiled directly by the \LaTeX{} compiler.
%
% %%%%%%%%%%%%%%%%%%%%%%%%%%%%%%%%%%%%%%
% \paragraph{Main File.}
%
% The main file is called |cdocsamp.tex|.
%
% Load the \textsf{childdoc} definitions and
% declare the filename for the main document:
%    \begin{macrocode}
\input{childdoc.def}
\childdocmain{}
%    \end{macrocode}

% Optional override for |\version| flag:
%    \begin{macrocode}
%%\ifchilddoc\else\providecommand{\version}{draft}\fi
%    \end{macrocode}

% Define the default values for the |\version| flag
% (|final| for the main file and |draft| for childs):
%    \begin{macrocode}
\ifchilddoc
\providecommand{\version}{draft}
\else
\providecommand{\version}{final}
\fi
%    \end{macrocode}

% Load the standard document class:
%    \begin{macrocode}
\documentclass[12pt]{article}
%    \end{macrocode}

% Start the document body:
%    \begin{macrocode}
\begin{document}
%    \end{macrocode}

% Declare a title page.
% Print title, part of document being processed and version flag:
%    \begin{macrocode}
\addtocounter{page}{-1}
\begin{center}
{\LARGE\bfseries{}childdoc example\par}
\vspace{1cm}
\ifchilddoc
\ifchilddocmanual part\else chapter\fi:
`\childdocname' of `\childdocjob'\par
\else
main document: `\childdocjob'\par
\fi
version: \version\par
\end{center}
\newpage
%    \end{macrocode}

% Manually include selected file,
% otherwise process as usual:
%    \begin{macrocode}
\ifchilddocmanual
\section*{part `\childdocname'}
\input{\childdocname}
\else
%    \end{macrocode}

% Include the two chapters:
%    \begin{macrocode}
\include{cdocsch1}
\include{cdocsch2}
%    \end{macrocode}

% Include the two parts unless only chapters should be displayed:
%    \begin{macrocode}
\ifchilddoc\else
\section{part three}
\input{cdocspt3}
\section{part four}
\input{cdocspt4}
\fi
%    \end{macrocode}

% Process as usual until here:
%    \begin{macrocode}
\fi
%    \end{macrocode}

% End of document body:
%    \begin{macrocode}
\end{document}
%    \end{macrocode}
%\iffalse
%</samplemain>
%\fi
%
% %%%%%%%%%%%%%%%%%%%%%%%%%%%%%%%%%%%%%%
% \paragraph{Chapter Include Files.}
%
% The include files are called |cdocsch1.tex| and |cdocsch2.tex|.
%
%\iffalse
%<*samplechap1|samplechap2>
%\fi

% Optional override for |\version| flag:
%    \begin{macrocode}
%%\providecommand{\version}{final}
%    \end{macrocode}

% Include the main document:
%    \begin{macrocode}
\input{childdoc.def}
\childdocof{cdocsamp}
%    \end{macrocode}

%\iffalse
%</samplechap1|samplechap2>
%\fi
%
%\iffalse
%<*samplechap1>
%\fi
% Some text for chapter 1:
%    \begin{macrocode}
\section{one}
some text in chapter one
%    \end{macrocode}

%\iffalse
%</samplechap1>
%\fi
% Some text for chapter 2:
%\iffalse
%<*samplechap2>
%\fi
%    \begin{macrocode}
\section{two}
more text in chapter two
%    \end{macrocode}

%\iffalse
%</samplechap2>
%\fi
%
% %%%%%%%%%%%%%%%%%%%%%%%%%%%%%%%%%%%%%%
% \paragraph{Part Include Files.}
%
% The include files are called |cdocspt3.tex| and |cdocspt4.tex|.
%
%\iffalse
%<*samplepart3|samplepart4>
%\fi

% Optional override for |\version| flag:
%    \begin{macrocode}
%%\providecommand{\version}{final}
%    \end{macrocode}

% Include the main document:
%    \begin{macrocode}
\input{childdoc.def}
\childdocby{cdocsamp}
%    \end{macrocode}

%\iffalse
%</samplepart3|samplepart4>
%\fi
%
%\iffalse
%<*samplepart3>
%\fi
% Some text for part 3:
%    \begin{macrocode}
some text in part three
%    \end{macrocode}

%\iffalse
%</samplepart3>
%\fi
% Some text for part 4:
%\iffalse
%<*samplepart4>
%\fi
%    \begin{macrocode}
more text in part four
%    \end{macrocode}

%\iffalse
%</samplepart4>
%\fi
%
% %%%%%%%%%%%%%%%%%%%%%%%%%%%%%%%%%%%%%%
% \paragraph{Forwarding for a Complete Draft.}
%
% The following forwarding file |cdocsdrf.tex|
% compiles the main document in draft mode:
%\iffalse
%<*sampledraft>
%\fi
%    \begin{macrocode}
\def\version{draft}
\input{childdoc.def}
\childdocforward{cdocsamp}
%    \end{macrocode}

%\iffalse
%</sampledraft>
%\fi
%
% %%%%%%%%%%%%%%%%%%%%%%%%%%%%%%%%%%%%%%
% \paragraph{Forwarding for Final Version of the Chapters.}
%
% The following forwarding files |cdocsfn1.tex| and |cdocsfn2.tex|
% (with identical content)
% compile the final versions of the child documents
% |cdocsch1.tex| and |cdocsch2.tex|, respectively:
%\iffalse
%<*samplefinal>
%\fi
%    \begin{macrocode}
\def\version{final}
\input{childdoc.def}
\childdocforwardprefix[cdocsamp]{cdocsfn}{cdocsch}
%    \end{macrocode}

%\iffalse
%</samplefinal>
%\fi
%
% %%%%%%%%%%%%%%%%%%%%%%%%%%%%%%%%%%%%%%
% \paragraph{Command Line Processing.}
%
% The following three command lines generate the output files
% |cdocscld|, |cdocscl1| and |cdocscl2|
% which should be identical to
% |cdocsdrf|, |cdocsch1| and |cdocsfn2|, respectively:
% \begin{center}
% \begin{tabular}{l}
% |latex -jobname cdocscld \|\\
% |  "\def\version{draft}\input{childdoc.def}\childdocforward{cdocsamp}"|\\
% |latex -jobname cdocscl1 \|\\
% |  "\input{childdoc.def}\childdocforward[cdocsamp]{cdocsch1}"|\\
% |latex -jobname cdocscl2 \|\\
% |  "\def\version{final}\input{childdoc.def}\childdocforward{cdocsch2}"|
% \end{tabular}
% \end{center}
% Note that the trailing backslash on each first line
% merely continues the input to the second line
% (for convenient cut ant paste).
% Furthermore, the command |latex| can be replaced by any
% of its alternative versions such as |pdflatex|.
%
% %%%%%%%%%%%%%%%%%%%%%%%%%%%%%%%%%%%%%%%%%%%%%%%%%%%%%%%%%%%%%%%%%%%%%%%%%%%%%%
% %%%%%%%%%%%%%%%%%%%%%%%%%%%%%%%%%%%%%%%%%%%%%%%%%%%%%%%%%%%%%%%%%%%%%%%%%%%%%%
% \section{Implementation}
%\iffalse
%<*package>
%\fi
%
% This section describes the definitions file |childdoc.def|.

% The definitions cannot be loaded using |\usepackage| or |\RequirePackage|
% which has a mechanism to prevent loading a style file more than once.
% When loading the definitions by means of |\input|
% multiple instances have to be prevented manually:
%\iffalse
%This code needs to be before the `\ProvidesFile' directive
%which is defined at the beginning of this file.
%Therefore it is also placed there and commented out here.
%</package>
%<*discard>
%\fi
%    \begin{macrocode}
\ifdefined\childdocmain\endinput\fi
%    \end{macrocode}
%\iffalse
%</discard>
%<*package>
%\fi
%
% \macro{\ifchilddoc}
% \macro{\ifchilddocmanual}
% The conditional |\ifchilddoc| tells whether a
% child (true) or main (false) document is being compiled.
% The conditional |\ifchilddocmanual| tells whether
% the |\includeonly| mechanism is used (false) or
% the selection of child files must be performed manually (true).
% The definitions initialise to false:
%    \begin{macrocode}
\newif\ifchilddoc
\newif\ifchilddocmanual
%    \end{macrocode}

% \macro{\childdocname}
% \macro{\childdocjob}
% The macro |\childdocname| stores the name of the main document
% to be compiled. The macro |\childdocjob| stores the name of
% the document on which the \LaTeX{} compiler was originally invoked.
% The content of |\jobname| cannot be compared
% to filenames specified in the source due to different catcodes.
% The following code rescans |\jobname|, stores the result
% in |\childdocname| and saves a copy in |\childdocjob|:
%    \begin{macrocode}
\edef\childdocname{\scantokens\expandafter{\jobname\noexpand}}
\let\childdocjob\childdocname
%    \end{macrocode}

% \macro{\childdocdisable}
% The macro |\childdocdisable| prevents the main file
% from being processed more than once.
% At this stage, the main document command |\childdocmain|
% is assumed to be called once again where it should do nothing.
% Any subsequent call to it should prevent
% a secondary processing of the main document
% It overwrites the forwarding commands
% |\childdocof| and |\childdocforward|
% with empty macros to prevent further inclusions of the main document:
%    \begin{macrocode}
\newcommand{\childdocdisable}
{
  \renewcommand{\childdocmain}[1]{\renewcommand{\childdocmain}[1]{\endinput}}
  \renewcommand{\childdocof}[1]{}
  \renewcommand{\childdocby}[2][]{}
  \renewcommand{\childdocforward}[2][]{}
  \renewcommand{\childdocdisable}{}
}
%    \end{macrocode}

% \macro{\childdocmain}
% The macro |\childdocmain| is to be called at the top of the main file
% with nothing or the main filename (without extension) as argument.
% First, it breaks loops.
% If the argument is not empty and does not match |\childdocname|
% (which is set by the first inclusion of |childdoc.def|),
% |\ifchilddoc| is set to true, |\includeonly| is applied to the child file
% and |\jobname| is set to the main file
% (for proper handling of |.aux| files):
%    \begin{macrocode}
\newcommand{\childdocmain}[1]
{
  \childdocdisable\childdocmain{}
  \if?#1?\else
    \begingroup
      \def\childdoctmp{#1}
      \ifx\childdoctmp\childdocname
        \def\childdoctmp{}
      \else
        \def\childdoctmp
        {
          \childdoctrue
          \includeonly{\childdocname}
          \def\childdocjob{#1}
          \def\jobname{#1}
        }
      \fi
      \expandafter
    \endgroup
    \childdoctmp
  \fi
}
%    \end{macrocode}

% \macro{\childdocof}
% The command |\childdocof| redirects
% compilation to the main file |#1|.
%    \begin{macrocode}
\newcommand{\childdocof}[1]
{
  \childdocdisable
  \childdoctrue
  \includeonly{\childdocname}
  \def\jobname{#1}
  \def\childdocjob{#1}
  \input{#1}
}
%    \end{macrocode}

% \macro{\childdocby}
% The command |\childdocby| ....
%    \begin{macrocode}
\newcommand{\childdocby}[2][]
{
  \childdocdisable
  \childdoctrue
  \childdocmanualtrue
  \if?#1?\else
    \def\jobname{#2}
  \fi
  \def\childdocjob{#2}
  \input{#2}
  \endinput
}
%    \end{macrocode}

% \macro{\childdocforward}
% The command |\childdocforward| redirects
% compilation to the main file or
% (if the optional argument is given) a child file.
% Parameters are set as if the main file
% or a child file starting with |\childdocof| was compiled.
% Then compilation is handed over to the main file:
%    \begin{macrocode}
\newcommand{\childdocforward}[2][]
{
  \begingroup
    \if?#1?
      \def\childdoctmp
      {
        \def\childdocname{#2}
        \def\childdocjob{#2}
        \def\jobname{#2}
        \input{#2}
        \endinput
      }
    \else
      \def\childdoctmp
      {
        \childdocdisable
        \def\childdocname{#2}
        \childdoctrue
        \includeonly{#2}
        \def\childdocjob{#1}
        \def\jobname{#1}
        \input{#1}
        \endinput
      }
    \fi
    \expandafter
  \endgroup
  \childdoctmp
}
%    \end{macrocode}

% \macro{\childdocforwardprefix}
% The command |\childdocforwardprefix| redirects
% compilation to the main or a child file by means of a pattern.
% The prefix |#1| in the current filename is replaced by |#2|
% and the suffix of the current filename is kept
% (it is assumed that the filename does not contain the substring `|~~~|'
% which is used as a delimiter).
% Compilation is handed over to the new file by |\childdocforward|:
%    \begin{macrocode}
\newcommand{\childdocforwardprefix}[3][]
{
  \begingroup
    \def\childdocextract #2##1~~~{\def\childdoctmp{\childdocforward[#1]{#3##1}}}
    \expandafter\childdocextract\childdocname~~~
    \expandafter
  \endgroup
  \childdoctmp
}
%    \end{macrocode}

% \macro{\childdoc}
% The deprecated macro |\childdoc| is a legacy version of |\childdocmain|:
%    \begin{macrocode}
\newcommand{\childdoc}{\childdocmain}
%    \end{macrocode}

% \macro{\childdocredirect}
% The deprecated macro |\childdocredirect| is a legacy version
% of |\childdocforward| and |\childdocforwardprefix|:
%    \begin{macrocode}
\newcommand{\childdocredirect}[2][]
{
  \begingroup
    \if?#1?
      \def\childdoctmp{\childdocforward{#2}}
    \else
      \def\childdoctmp{\childdocforwardprefix{#1}{#2}}
    \fi
    \expandafter
  \endgroup
  \childdoctmp
}
%    \end{macrocode}

%\iffalse
%</package>
%\fi
%
\endinput

\childdocof{cdocsamp}
%    \end{macrocode}

%\iffalse
%</samplechap1|samplechap2>
%\fi
%
%\iffalse
%<*samplechap1>
%\fi
% Some text for chapter 1:
%    \begin{macrocode}
\section{one}
some text in chapter one
%    \end{macrocode}

%\iffalse
%</samplechap1>
%\fi
% Some text for chapter 2:
%\iffalse
%<*samplechap2>
%\fi
%    \begin{macrocode}
\section{two}
more text in chapter two
%    \end{macrocode}

%\iffalse
%</samplechap2>
%\fi
%
% %%%%%%%%%%%%%%%%%%%%%%%%%%%%%%%%%%%%%%
% \paragraph{Part Include Files.}
%
% The include files are called |cdocspt3.tex| and |cdocspt4.tex|.
%
%\iffalse
%<*samplepart3|samplepart4>
%\fi

% Optional override for |\version| flag:
%    \begin{macrocode}
%%\providecommand{\version}{final}
%    \end{macrocode}

% Include the main document:
%    \begin{macrocode}
% \iffalse
%
% childdoc.dtx Copyright (C) 2017-2018 Niklas Beisert
%
% This work may be distributed and/or modified under the
% conditions of the LaTeX Project Public License, either version 1.3
% of this license or (at your option) any later version.
% The latest version of this license is in
%   http://www.latex-project.org/lppl.txt
% and version 1.3 or later is part of all distributions of LaTeX
% version 2005/12/01 or later.
%
% This work has the LPPL maintenance status `maintained'.
%
% The Current Maintainer of this work is Niklas Beisert.
%
% This work consists of the files childdoc.dtx and childdoc.ins
% and the derived files childdoc.def and cdocsamp.tex with
% cdocsch1.tex, cdocsch2.tex, cdocsdrf.tex, cdocsfn1.tex, cdocsfn2.tex.
%
%<package>\ifdefined\childdocmain\endinput\fi
%<package>\ProvidesFile{childdoc.def}[2018/12/30 v2.0 child document driver]
%<samplemain>\ProvidesFile{cdocsamp.tex}[2018/12/30 v2.0 sample for childdoc]
%<*driver>
%\ProvidesFile{childdoc.drv}[2018/12/30 v2.0 childdoc reference manual file]
\PassOptionsToClass{10pt,a4paper}{article}
\documentclass{ltxdoc}

\usepackage[margin=35mm]{geometry}
\usepackage{hyperref}
\usepackage{hyperxmp}
\usepackage[usenames]{color}

\hypersetup{colorlinks=true}
\hypersetup{pdfstartview=FitH}
\hypersetup{pdfpagemode=UseNone}
\hypersetup{pdfsource={}}
\hypersetup{pdflang={en-UK}}
\hypersetup{pdfcopyright={Copyright 2017-2018 Niklas Beisert.
  This work may be distributed and/or modified under the
  conditions of the LaTeX Project Public License, either version 1.3
  of this license or (at your option) any later version.}}
\hypersetup{pdflicenseurl={http://www.latex-project.org/lppl.txt}}
\hypersetup{pdfcontactaddress={ETH Zurich, ITP, HIT K,
  Wolfgang-Pauli-Strasse 27}}
\hypersetup{pdfcontactpostcode={8093}}
\hypersetup{pdfcontactcity={Zurich}}
\hypersetup{pdfcontactcountry={Switzerland}}
\hypersetup{pdfcontactemail={nbeisert@itp.phys.ethz.ch}}
\hypersetup{pdfcontacturl={http://people.phys.ethz.ch/\xmptilde nbeisert/}}

\newcommand{\secref}[1]{\hyperref[#1]{section \ref*{#1}}}

\parskip1ex
\parindent0pt
\let\olditemize\itemize
\def\itemize{\olditemize\parskip0pt}

\begin{document}

\title{The \textsf{childdoc} Package}
\hypersetup{pdftitle={The childdoc Package}}
\author{Niklas Beisert\\[2ex]
  Institut f\"ur Theoretische Physik\\
  Eidgen\"ossische Technische Hochschule Z\"urich\\
  Wolfgang-Pauli-Strasse 27, 8093 Z\"urich, Switzerland\\[1ex]
  \href{mailto:nbeisert@itp.phys.ethz.ch}
  {\texttt{nbeisert@itp.phys.ethz.ch}}}
\hypersetup{pdfauthor={Niklas Beisert}}
\hypersetup{pdfsubject={Manual for the LaTeX2e Package childdoc}}
\date{30 December 2018, \textsf{v2.0}}
\maketitle

\begin{abstract}\noindent
\textsf{childdoc} is a \LaTeXe{} package
that enables the direct compilation
of document sections included by |\include|
to individual files.
\end{abstract}

\begingroup
\parskip0ex
\tableofcontents
\endgroup

%%%%%%%%%%%%%%%%%%%%%%%%%%%%%%%%%%%%%%%%%%%%%%%%%%%%%%%%%%%%%%%%%%%%%%%%%%%%%%%%
%%%%%%%%%%%%%%%%%%%%%%%%%%%%%%%%%%%%%%%%%%%%%%%%%%%%%%%%%%%%%%%%%%%%%%%%%%%%%%%%
\section{Introduction}

\LaTeX{} provides a mechanism to structure a large document (such as a book)
into a main file and several child files (containing the chapters)
using the |\include| command.
This mechanism is beneficial for documents
which span hundreds of pages in order to
make the source file(s) more manageable.
Moreover, compilation can be restricted to
selected child files by means of the |\includeonly| command.
The latter feature can be used to reduce the compilation time while editing
(this was significantly more useful in the earlier days of \LaTeX{})
or to generate a smaller document which is easier to navigate.
Another application of |\includeonly| is to generate
documents consisting of selected parts of the complete document.

However, there are a few drawbacks of the plain |\include| mechanism:
\begin{itemize}
\item
The child files cannot be compiled on their own,
they can only be compiled via the main file.
A naive editing environment
(such as a text editor with an option
to have the current file processed by \LaTeX)
may require one to switch to the main file before compiling;
attempting to compile the child file produces errors.
\item
The main file must be modified (each time)
to adjust the |\includeonly| command
to the present needs. This easily leaves the main file in a messy state.
\item
The generated document will always carry the filename
of the main document. This is inconvenient if
several child files are to be compiled and
to be kept for distribution.
\end{itemize}

The present package provides a simple interface
to make child files individually compilable by \LaTeX{}.
Compiling a child file then has the same effect as compiling
the main file with an |\includeonly| command
to select the appropriate child.
Moreover the generated document will carry the name of the child
rather than the main file.
This resolves all three above issues.

This feature is meant to make the editing of books,
thesis documents and lecture notes somewhat more convenient.
However, the package can also be used efficiently for
composing a series of documents (such as exercise sheets)
which are typically distributed individually.
It then assists the author in generating the individual documents
(potentially in different versions)
as well as a document containing the collected series.
Another application is in developing style files
or other kinds of included material
where compilation of the style file could redirect
to a sample or test file.

%%%%%%%%%%%%%%%%%%%%%%%%%%%%%%%%%%%%%%%%%%%%%%%%%%%%%%%%%%%%%%%%%%%%%%%%%%%%%%%%
%%%%%%%%%%%%%%%%%%%%%%%%%%%%%%%%%%%%%%%%%%%%%%%%%%%%%%%%%%%%%%%%%%%%%%%%%%%%%%%%
\section{Usage}

First of all, the package \textsf{childdoc} is \emph{not} a standard
\LaTeXe{} |.sty| style file! Therefore it needs to be invoked in
a non-standard way.

%%%%%%%%%%%%%%%%%%%%%%%%%%%%%%%%%%%%%%%%%%%%%%%%%%%%%%%%%%%%%%%%%%%%%%%%%%%%%%%%
\subsection{Included Files}
\label{sec:include}

%%%%%%%%%%%%%%%%%%%%%%%%%%%%%%%%%%%%%%%%
\DescribeMacro{\childdocmain}
To use the package, add the commands
\begin{center}
\begin{tabular}{l}
|\input{childdoc.def}|\\
|\childdocmain{}|\\
\end{tabular}
\end{center}
at the very top of the main \LaTeX{} file,
in particular \emph{before} the |\documentclass| statement!
The argument of |\childdocmain| should be left empty
(but it must be present).

%%%%%%%%%%%%%%%%%%%%%%%%%%%%%%%%%%%%%%%%
\DescribeMacro{\childdocof}
Furthermore, add the commands
\begin{center}
\begin{tabular}{l}
|\input{childdoc.def}|\\
|\childdocof{|\textit{main}|}|\\
\end{tabular}
\end{center}
at the top of every child file \textit{child}
which is included by |\include{|\textit{child}|}|
from within the main file
(or at least for those files to be compiled individually).
The argument \textit{main} must be the filename of the main file.

There are a couple of
considerations in setting up the main and child documents:

%%%%%%%%%%%%%%%%%%%%%%%%%%%%%%%%%%%%%%%%
\paragraph{Restrictions.}

Please note the following restrictions:
\begin{itemize}
\item
|\childdocmain| must be called with one argument \textit{main}
to ensure compatibility with earlier version of the package.
It must either be empty (|\childdocmain{}|)
or precisely match the filename of the main file in which it is specified.
See \secref{sec:detection} for further information.
\item
The filename \textit{main} must be specified without the |.tex| extension.
\item
The filename \textit{main} is case sensitive
(even in case-insensitive file systems)
due to internal string comparison.
\item
The argument \textit{main} should be fully expanded, it cannot be a macro.
\item
Subdirectories and special characters should be avoided in filenames.
\item
The command |\childdocmain{|\textit{main}|}| must be followed by a whitespace.
It should not be followed immediately by another command
or by a comment mark `|%|'.
This is because the \TeX{} parser reads the token immediately following
the argument of |\childdocmain| and puts it
at the beginning of every child section;
however, a white\-space is ignored.
\end{itemize}

%%%%%%%%%%%%%%%%%%%%%%%%%%%%%%%%%%%%%%%%
\paragraph{Content of Main File.}

It is advisable to place all content in the child files included by |\include|.
Any output contained in the main file will appear in all child documents
unless suppressed manually;
it cannot be suppressed automatically by the |\includeonly| directive
and thus should normally be avoided.
A method to include some content in the main file
by means of conditional processing is described in \secref{sec:conditional}.

%%%%%%%%%%%%%%%%%%%%%%%%%%%%%%%%%%%%%%%%
\paragraph{Page Numbering.}

When only a part of the document is compiled,
the appropriate numbering of pages
(as well as other status parameters)
is determined from the |.aux| files.
The latter contain information from previous passes.
However this information needs to propagate through
all intermediate child documents.
Therefore the page numbering in child documents may well
be inconsistent until the complete document is compiled at least once.

A useful (if unconventional) way to always ensure a consistent
page numbering is to restart the numbering in each child document
and denote the pages by `\textit{child}|.|\textit{page}'
where \textit{child} represents the chapter/section number of the child file.
This can be achieved by the command
|\numberwithin{page}{|\textit{child}|}|
of the \textsf{amsmath} package
where \textit{child} can be |chapter| or |section|
depending on the chosen structuring.
Alternatively, one can modify the macro |\thepage| appropriately
and reset the counter |page| at the start of each child file.

%%%%%%%%%%%%%%%%%%%%%%%%%%%%%%%%%%%%%%%%%%%%%%%%%%%%%%%%%%%%%%%%%%%%%%%%%%%%%%%%
\subsection{Conditional Processing}
\label{sec:conditional}

The package provides a mechanism to compile different versions
of a document. To customise the versions further some conditional processing
can come in handy to distinguish which version is being compiled.
The package provides two macros to describe the compilation context:

%%%%%%%%%%%%%%%%%%%%%%%%%%%%%%%%%%%%%%%%
\DescribeMacro{\ifchilddoc}
The conditional |\ifchilddoc| distinguishes between the compilation of
child documents and the main document:
%
\begin{center}
|\ifchilddoc |\textit{child-code}| |[|\||else |\textit{main-code}]| \||fi|
\end{center}

%%%%%%%%%%%%%%%%%%%%%%%%%%%%%%%%%%%%%%%%
\DescribeMacro{\childdocname}
\DescribeMacro{\childdocjob}
The macro |\childdocname| contains the filename (without extension)
of the main or child file being processed.
Note that |\childdocjob| will always contain the name of the main file.

%%%%%%%%%%%%%%%%%%%%%%%%%%%%%%%%%%%%%%%%
\paragraph{Title Page.}

Conditional processing can be used to include a title or banner page
in the main document when proper precautions are taken.
Importantly, the code in the main file should ensure that the page counter
(as well as other status parameters which are stored in the |.aux| files)
takes the same value after the conditional processing.
Otherwise the page numbers may take divergent values
depending on which part is compiled.

For example, a title page could be declared by:
%
\begin{center}
\begin{tabular}{l}
|\ifchilddoc\||else|\\
|\addtocounter{page}{-1}|\\
\textit{code for title page}\\
|\newpage|\\
|\||fi|
\end{tabular}
\end{center}
%
A banner page for the child documents can be generated by:
%
\begin{center}
\begin{tabular}{l}
|\ifchilddoc|\\
|\addtocounter{page}{-1}|\\
\textit{code for banner page}\\
|\newpage|\\
|\||fi|
\end{tabular}
\end{center}
%
Here one could write a message such as:
\begin{center}
|This is the part \childdocname{} of \childdocjob{}.|
\end{center}

%%%%%%%%%%%%%%%%%%%%%%%%%%%%%%%%%%%%%%%%%%%%%%%%%%%%%%%%%%%%%%%%%%%%%%%%%%%%%%%%
\subsection{Flags}
\label{sec:flags}

The package makes it easy to generate different versions
of the main or child documents.
To this end compilation flags can be defined
and assigned different default values.
They will be particularly useful in conjunction
with the forwarding mechanism described in \secref{sec:forward}.

For example, it may be useful to have a flag |\version|
which can be set to |draft| or |final|.
The document source will contain some conditional code
depending on the value of |\version|.
Suppose further, the flag should default to |final| for the main file
and to |draft| for child files
which is a natural assignment for editing the document.
This is achieved by placing the following code
in the preamble of the main document
(below the |\childdocmain| directive):
%
\begin{center}
\begin{tabular}{l}
|\ifchilddoc|\\
|\providecommand{\version}{draft}|\\
|\||else|\\
|\providecommand{\version}{final}|\\
|\||fi|
\end{tabular}
\end{center}
%
The definition by |\providecommand| makes sure
that previous definitions are not overwritten.
Further statements |\providecommand{\version}{...}|
can thus be added before the above code to override it.

For the main file, one might add a line
(between |\childdocmain| and the above block)
%
\begin{center}
|%\ifchilddoc\||else\providecommand{\version}{draft}\||fi|
\end{center}
%
which can be uncommented to produce a draft version.
Likewise one can add a line to the very top of a child file
(above the |\childdocof{|\textit{main}|}| directive)
%
\begin{center}
|%\providecommand{\version}{final}|
\end{center}
%
which can be uncommented to produce the final version of this child document.

%%%%%%%%%%%%%%%%%%%%%%%%%%%%%%%%%%%%%%%%%%%%%%%%%%%%%%%%%%%%%%%%%%%%%%%%%%%%%%%%
\subsection{Forwarding}
\label{sec:forward}

Different versions of the main or child documents
using compilation flags as described in \secref{sec:flags}
can be (permanently) stored in different files
for convenient compilation, viewing and distribution.
To this end, the package defines a command
to pass on compilation to a different file:

%%%%%%%%%%%%%%%%%%%%%%%%%%%%%%%%%%%%%%%%
\DescribeMacro{\childdocforward}
The command |\childdocforward| redirects processing to
another source file:
%
\begin{center}
\begin{tabular}{l}
|\input{childdoc.def}|\\
|\childdocforward[|\textit{main}|]{|\textit{dest}|}|\\
\end{tabular}
\end{center}
%
The argument \textit{dest} is the destination file
(without extension).
It should be the main file or one of the child files.
Note that further \textsf{childdoc} directives
such as |\childdocof| and |\childdocforward|
in the indicated file will be processed in this form.
The optional argument \textit{main}
passes on directly to the main file \textit{main}
while pretending to compile the child \textit{dest}.
This form behaves as if \textit{dest}
issues |\childdocof{|\textit{main}|}| right away,
and no further \textsf{childdoc} directives will be processed.

%%%%%%%%%%%%%%%%%%%%%%%%%%%%%%%%%%%%%%%%
\DescribeMacro{\...prefix}
In the alternative form |\childdocforwardprefix|,
%
\begin{center}
\begin{tabular}{l}
|\input{childdoc.def}|\\
|\childdocforwardprefix[|\textit{main}|]{|\textit{prefix}|}{|\textit{dest}|}|
\end{tabular}
\end{center}
%
the destination file is determined by a pattern
depending on the current file:
To make this work, the current file must be called
`{\textit{prefix}\hspace{0.2em}\textit{suffix}}'
with \textit{prefix} matching precisely the argument.
Processing is then passed on to the file
`{\textit{dest}\hspace{0.2em}\textit{suffix}}'.
Surely, the same effect is achieved by
directly specifying the
argument `{\textit{dest}\hspace{0.2em}\textit{suffix}}'
in the first form.
However, that requires to set up a different file
for each child. With the alternative form of the command
all these files can have exactly the same content
which simplifies setting them up and maintaining them.

For example, the following file |draft.tex|
with a compilation flag |\version| as described in \secref{sec:flags}
compiles the main document as a draft:
%
\begin{center}
\begin{tabular}{l}
|\def\version{draft}|\\
|\input{childdoc.def}|\\
|\childdocforward{|\textit{main}|}|
\end{tabular}
\end{center}
%
Likewise, the following files |final|\textit{nn}|.tex|
compile the final version of the child document
|child|\textit{nn}|.tex|:
%
\begin{center}
\begin{tabular}{l}
|\def\version{final}|\\
|\input{childdoc.def}|\\
|\childdocforwardprefix{final}{child}|
\end{tabular}
\end{center}
%

Note that when several versions of a main file and/or of each child file
are to be generated, it may be convenient to set up a |Makefile| or
shell script to automatise the process.

%%%%%%%%%%%%%%%%%%%%%%%%%%%%%%%%%%%%%%%%%%%%%%%%%%%%%%%%%%%%%%%%%%%%%%%%%%%%%%%%
\subsection{Command Line Processing}
\label{sec:commandline}

The effect of redirection files can also be achieved by invoking
the \LaTeX{} compiler with a more elaborate command line.
Most conveniently this should be done as part
of a shell script or a |Makefile|.

When using \textsf{childdoc} in the main file, the following
command lines effectively perform a redirection
(note that depending on the shell being used,
backslashes may have to be doubled: `|\|' $\to$ `|\\|'):
%
\begin{center}
|... -jobname "|\textit{target}|" |\\|"|[\textit{flags}]%
|\input{childdoc.def}\childdocforward[|\textit{main}|]{|\textit{dest}|}"|
\end{center}
%
Here \textit{target} is the name of the output file,
\textit{main} is the name of the main file
and \textit{dest} is the name of the main or child file to be processed
(all filenames without extensions).
The optional argument \textit{main} can be omitted
if \textit{main} matches \textit{dest}.
Optionally, compilation \textit{flags} can be defined via |\def| commands.
This command line makes the \TeX{} engine believe
it is compiling the file \textit{target}
whose content is specified as the latter parameter.
The provided code then forwards the processing to
\textit{main} or \textit{dest} as described in \secref{sec:forward}.

%%%%%%%%%%%%%%%%%%%%%%%%%%%%%%%%%%%%%%%%%%%%%%%%%%%%%%%%%%%%%%%%%%%%%%%%%%%%%%%%
\subsection{Include by Input}
\label{sec:input}

Including child documents by |\include| has some restrictions by design.
Most notably, the content of a child document always occupies
its own set of pages; pages cannot be shared between child documents.
Usually, this behaviour makes perfect sense
because each child document contain an essential part of the document.
However, in some situations it may be desirable to compose
a document from a collection of parts
without having mandatory page breaks between then.
For this case, the package
provides a mechanism to include parts
by |\input| which can also be processed individually.
However, by construction this mechanism
requires manual handling of the content to be output.

%%%%%%%%%%%%%%%%%%%%%%%%%%%%%%%%%%%%%%%%
\DescribeMacro{\ifchilddocmanual}
The main file should be prepared as usual, see \secref{sec:include}.
However, the document body must make a distinction
between processing of an individual part and of the main document, e.g.:
%
\begin{center}
\begin{tabular}{l}
|\ifchilddocmanual|\\
|\input{\childdocname}|\\
|\||else|\\
\textit{document body with }|\input{|\textit{part}|}|\\
|\||fi|
\end{tabular}
\end{center}
%
The conditional |\ifchilddocmanual| is true whenever
a part to be included by |\input| is being compiled,
and the name of the part is stored in |\childdocname|.

%%%%%%%%%%%%%%%%%%%%%%%%%%%%%%%%%%%%%%%%
\DescribeMacro{\childdocby}
Each part to be included by |\input| should start with:
%
\begin{center}
\begin{tabular}{l}
|\input{childdoc.def}|\\
|\childdocby{|\textit{main}|}|\\
\end{tabular}
\end{center}
%
The directive |\childdocby| is similar to |\childdocof|
described in \secref{sec:include},
but the subsequent selection of content must be done manually.
To that end, both |\ifchilddoc| and |\ifchilddocmanual|
will be true upon processing of a part,
and the name of the part is stored in |\childdocname|.
Note that |\jobname| will be set to the filename of the current part
so that each part receives an individual |.aux| file
that does not interfere with the |.aux| file(s) of the main document.
This behaviour can be altered by the alternative form
|\childdocby[*]{|\textit{main}|}| (with a non-empty optional argument)
which uses the |.aux| file of the main document
by setting |\jobname| to \textit{main}.

%%%%%%%%%%%%%%%%%%%%%%%%%%%%%%%%%%%%%%%%%%%%%%%%%%%%%%%%%%%%%%%%%%%%%%%%%%%%%%%%
\subsection{Driver Development}
\label{sec:driver}

The \textsf{childdoc} mechanism can also be use for the development
of definition files such as \LaTeX{} styles or classes.
This case differs from the above setup with multiple parts
included by |\include| in that no |\includeonly| should be invoked.
This can be achieved by starting the include file
(before |\ProvidesPackage|) with:
%
\begin{center}
\begin{tabular}{l}
|\input{childdoc.def}|\\
|\childdocforward{|\textit{main}|}|\\
\end{tabular}
\end{center}
%
or alternatively with:
%
\begin{center}
\begin{tabular}{l}
|\input{childdoc.def}|\\
|\childdocby{|\textit{main}|}|\\
\end{tabular}
\end{center}
%
Both forms have slightly different effects as described above.
The main file is prepared as usual, see \secref{sec:include}.

%%%%%%%%%%%%%%%%%%%%%%%%%%%%%%%%%%%%%%%%%%%%%%%%%%%%%%%%%%%%%%%%%%%%%%%%%%%%%%%%
\subsection{Legacy Detection}
\label{sec:detection}

The directive |\childdocmain| in the main file can detect
whether the complete document or merely a child is to be compiled
even without using the directive |\childdocof|.
This method is deprecated because it is less robust
and there is no compelling reason to use it;
it is merely provided for backward compatibility
and it may be removed in future versions.

If the detection mechanism is to be used,
it is mandatory to correctly specify
the filename of the main file as the argument of |\childdocmain|:
%
\begin{center}
\begin{tabular}{l}
|\input{childdoc.def}|\\
|\childdocmain{|\textit{main}|}|\\
\end{tabular}
\end{center}
%
If |\jobname| does not match the argument \textit{main} of |\childdocmain|,
it is assumed that |\jobname| points to the child file to be compiled.
When using |\childdocmain| with the main file specified as argument,
it suffices to start a child file
with just |\input{|\textit{main}|}|
without loading of the package and using |\childdocof|.
If instead all processing is done
with the appropriate \textsf{childdoc} directives,
the argument of \textit{main} of |\childdocmain| can be empty.

An alternative version of the command line processing described
in \secref{sec:commandline} using the detection mechanism reads:
%
\begin{center}
|... -jobname "|\textit{target}|" "|[\textit{flags}]%
[|\def\jobname{|\textit{dest}|}|]|\input{|\textit{main}|}"|
\end{center}

%%%%%%%%%%%%%%%%%%%%%%%%%%%%%%%%%%%%%%%%%%%%%%%%%%%%%%%%%%%%%%%%%%%%%%%%%%%%%%%%
\subsection{Manual Code}
\label{sec:manual}

In case one cannot be certain whether the definitions file |childdoc.def|
is installed on the target \TeX{} distribution
and one prefers not to ship it,
it is conceivable to paste a few relevant commands into the sources.

To that end, drop all statements |\input{childdoc.def}|
and perform the replacements as outlined below.
Instead of |\childdocmain{|\textit{main}|}| add the following code
to the top of the main file:
%
\begin{center}
\begin{tabular}{l}
|\||ifdefined\childdocname\endinput\||fi\newif\ifchilddoc|\\
|\edef\childdocname{\scantokens\expandafter{\jobname\noexpand}}|\\
|\def\childdocmain{|\textit{main}|}\||ifx\childdocmain\childdocname\||else|\\
|\childdoctrue\includeonly{\childdocname}\let\jobname\childdocmain\||fi|\\
\end{tabular}
\end{center}
%
Instead of |\childdocof{|\textit{main}|}| just include the main file
at the top of each child file:
%
\begin{center}
|\input{|\textit{main}|}|
\end{center}
%
A simple redirection |\childdocforward{|\textit{dest}|}| is achieved by:
%
\begin{center}
|\def\jobname{|\textit{dest}|}\input{\jobname}|
\end{center}
%
The redirection with prefix
|\childdocforwardprefix[|\textit{prefix}|]{|\textit{dest}|}|
is accomplished by:
%
\begin{center}
\begin{tabular}{l}
|{\edef\jobname{\scantokens\expandafter{\jobname\noexpand}}|\\
|\def\redirectjob |\textit{prefix}|#1~~~{\gdef\jobname{|\textit{dest}|#1}}|\\
|\expandafter\redirectjob\jobname~~~}\input{\jobname}|
\end{tabular}
\end{center}

In an alternative approach,
child documents can be compiled by a specific command line
without additional code or specific definitions:
%
\begin{center}
|... -jobname "|\textit{target}|" "|[\textit{flags}]%
|\includeonly{|\textit{dest}|}\input{|\textit{main}|}"|
\end{center}
%

%%%%%%%%%%%%%%%%%%%%%%%%%%%%%%%%%%%%%%%%%%%%%%%%%%%%%%%%%%%%%%%%%%%%%%%%%%%%%%%%
%%%%%%%%%%%%%%%%%%%%%%%%%%%%%%%%%%%%%%%%%%%%%%%%%%%%%%%%%%%%%%%%%%%%%%%%%%%%%%%%
\section{Information}

%%%%%%%%%%%%%%%%%%%%%%%%%%%%%%%%%%%%%%%%%%%%%%%%%%%%%%%%%%%%%%%%%%%%%%%%%%%%%%%%
\subsection{Copyright}

Copyright \copyright{} 2017--2018 Niklas Beisert

This work may be distributed and/or modified under the
conditions of the \LaTeX{} Project Public License, either version 1.3
of this license or (at your option) any later version.
The latest version of this license is in
  \url{http://www.latex-project.org/lppl.txt}
and version 1.3 or later is part of all distributions of \LaTeX{}
version 2005/12/01 or later.

This work has the LPPL maintenance status `maintained'.

The Current Maintainer of this work is Niklas Beisert.

This work consists of the files |README.txt|, |childdoc.ins| and |childdoc.dtx|
as well as the derived files |childdoc.def|, |cdocsamp.tex|
with |cdocsch1.tex|, |cdocsch2.tex|, |cdocspt3.tex|, |cdocspt4.tex|,
|cdocsdrf.tex|, |cdocsfn1.tex|, |cdocsfn2.tex|
as well as |childdoc.pdf|.

%%%%%%%%%%%%%%%%%%%%%%%%%%%%%%%%%%%%%%%%%%%%%%%%%%%%%%%%%%%%%%%%%%%%%%%%%%%%%%%%
\subsection{Files and Installation}

The package consists of the files:
%
\begin{center}
\begin{tabular}{ll}
    |README.txt|   & readme file \\
    |childdoc.ins| & installation file \\
    |childdoc.dtx| & source file \\
    |childdoc.def| & definition file \\
    |cdocsamp.tex| & sample main file \\
    |cdocsch1.tex| & sample include file \\
    |cdocsch2.tex| & sample include file \\
    |cdocspt3.tex| & sample part file \\
    |cdocspt4.tex| & sample part file \\
    |cdocsdrf.tex| & sample redirection file \\
    |cdocsfn1.tex| & sample redirection file \\
    |cdocsfn2.tex| & sample redirection file \\
    |childdoc.pdf| & manual
\end{tabular}
\end{center}
%
The distribution consists of the files
|README.txt|, |childdoc.ins| and |childdoc.dtx|.
%
\begin{itemize}
\item
Run (pdf)\LaTeX{} on |childdoc.dtx|
to compile the manual |childdoc.pdf| (this file).
\item
Run \LaTeX{} on |childdoc.ins| to create the definitions file |childdoc.def|
and the sample |cdocsamp.tex| with include files
|cdocsch1.tex|, |cdocsch2.tex|, |cdocspt3.tex|, |cdocspt4.tex|,
|cdocsdrf.tex|, |cdocsfn1.tex|, |cdocsfn2.tex|.
Then copy the file |childdoc.def| to an appropriate directory of your \LaTeX{}
distribution, e.g.\ \textit{texmf-root}|/tex/latex/childdoc|.
\end{itemize}

%%%%%%%%%%%%%%%%%%%%%%%%%%%%%%%%%%%%%%%%%%%%%%%%%%%%%%%%%%%%%%%%%%%%%%%%%%%%%%%%
\subsection{Related CTAN Packages}

There are several other packages which offer a similar functionality:
%
\begin{itemize}
\item
The packages
\href{http://ctan.org/pkg/docmute}{\textsf{docmute}},
\href{http://ctan.org/pkg/includex}{\textsf{includex}} and
\href{http://ctan.org/pkg/standalone}{\textsf{standalone}}
provide commands to include only the document body of
a child file thus allowing both files to be compiled individually.
\item
The packages \href{http://ctan.org/pkg/subdocs}{\textsf{subdocs}}
and \href{http://ctan.org/pkg/subfiles}{\textsf{subfiles}}
provide structures in which the main and child documents can be
encapsulated and allowing them to be compiled individually.
The inclusion mechanism is different from the conventional |\include|.
\item
The package \href{http://ctan.org/pkg/combine}{\textsf{combine}}
is an elaborate solution to combine several documents into one.
\end{itemize}
%
See also the CTAN topic \href{http://ctan.org/topic/subdocs}{\textsf{subdocs}}
for further related packages.
The present package differs from the above solutions in that
a document structure constructed with the conventional |\include| mechanism
just needs two extra commands at the top of every file
such that all constituent files can be compiled individually.

%%%%%%%%%%%%%%%%%%%%%%%%%%%%%%%%%%%%%%%%%%%%%%%%%%%%%%%%%%%%%%%%%%%%%%%%%%%%%%%%
%\subsection{Feature Suggestions}
%
%The following is a list of features which may be useful for future
%versions of this package:
%%
%\begin{itemize}
%\item
%\ldots
%\end{itemize}

%%%%%%%%%%%%%%%%%%%%%%%%%%%%%%%%%%%%%%%%%%%%%%%%%%%%%%%%%%%%%%%%%%%%%%%%%%%%%%%%
\subsection{Revision History}

%%%%%%%%%%%%%%%%%%%%%%%%%%%%%%%%%%%%%%%%
\paragraph{v2.0:} 2018/12/30

\begin{itemize}
\item
immediate forward processing
\item
added |\childdocby| mechanism
\item
manual restructured
\end{itemize}

%%%%%%%%%%%%%%%%%%%%%%%%%%%%%%%%%%%%%%%%
\paragraph{v1.6:} 2018/01/17

\begin{itemize}
\item
application for development of include files
\item
corrections to manual
\end{itemize}

%%%%%%%%%%%%%%%%%%%%%%%%%%%%%%%%%%%%%%%%
\paragraph{v1.5:} 2017/05/21

\begin{itemize}
\item
more complete structuring introduced
\item
|\childdocof| introduced
\item
|\childdoc| renamed to |\childdocmain|
\item
|\childredirect| renamed to |\childdocforward| and |\childdocforwardprefix|
and functionality expanded
\end{itemize}

%%%%%%%%%%%%%%%%%%%%%%%%%%%%%%%%%%%%%%%%
\paragraph{v1.0:} 2017/04/27

\begin{itemize}
\item
manual and install package
\item
first version published on CTAN
\end{itemize}

%%%%%%%%%%%%%%%%%%%%%%%%%%%%%%%%%%%%%%%%
\paragraph{v0.6:} 2017/04/26

\begin{itemize}
\item
redirection mechanism added
\end{itemize}

%%%%%%%%%%%%%%%%%%%%%%%%%%%%%%%%%%%%%%%%
\paragraph{v0.5:} 2017/04/26

\begin{itemize}
\item
functionality in definition file
\end{itemize}


%%%%%%%%%%%%%%%%%%%%%%%%%%%%%%%%%%%%%%%%%%%%%%%%%%%%%%%%%%%%%%%%%%%%%%%%%%%%%%%%
%%%%%%%%%%%%%%%%%%%%%%%%%%%%%%%%%%%%%%%%%%%%%%%%%%%%%%%%%%%%%%%%%%%%%%%%%%%%%%%%
%%%%%%%%%%%%%%%%%%%%%%%%%%%%%%%%%%%%%%%%%%%%%%%%%%%%%%%%%%%%%%%%%%%%%%%%%%%%%%%%
\appendix

\settowidth\MacroIndent{\rmfamily\scriptsize 000\ }

 \DocInput{childdoc.dtx}

\end{document}
%</driver>
% \fi
%
% %%%%%%%%%%%%%%%%%%%%%%%%%%%%%%%%%%%%%%%%%%%%%%%%%%%%%%%%%%%%%%%%%%%%%%%%%%%%%%
% %%%%%%%%%%%%%%%%%%%%%%%%%%%%%%%%%%%%%%%%%%%%%%%%%%%%%%%%%%%%%%%%%%%%%%%%%%%%%%
% \section{Sample}
%\iffalse
%<*samplemain>
%\fi
%
% The following presents a sample document
% with two chapters, two parts, a title page,
% a compile flag as well as three forwarding files to set the flag.
% It consists of eight |.tex| files:
% \begin{center}
% \begin{tabular}{ll}
% |cdocsamp.tex|&main file\\
% |cdocsch1.tex|&include file for chapter 1\\
% |cdocsch2.tex|&include file for chapter 2\\
% |cdocspt3.tex|&include file for part 3\\
% |cdocspt4.tex|&include file for part 4\\
% |cdocsdrf.tex|&forwarding file for main file in draft mode\\
% |cdocsfi1.tex|&forwarding file for final version of chapter 1\\
% |cdocsfi2.tex|&forwarding file for final version of chapter 2\\
% \end{tabular}
% \end{center}
% Each of the eight files can be compiled directly by the \LaTeX{} compiler.
%
% %%%%%%%%%%%%%%%%%%%%%%%%%%%%%%%%%%%%%%
% \paragraph{Main File.}
%
% The main file is called |cdocsamp.tex|.
%
% Load the \textsf{childdoc} definitions and
% declare the filename for the main document:
%    \begin{macrocode}
\input{childdoc.def}
\childdocmain{}
%    \end{macrocode}

% Optional override for |\version| flag:
%    \begin{macrocode}
%%\ifchilddoc\else\providecommand{\version}{draft}\fi
%    \end{macrocode}

% Define the default values for the |\version| flag
% (|final| for the main file and |draft| for childs):
%    \begin{macrocode}
\ifchilddoc
\providecommand{\version}{draft}
\else
\providecommand{\version}{final}
\fi
%    \end{macrocode}

% Load the standard document class:
%    \begin{macrocode}
\documentclass[12pt]{article}
%    \end{macrocode}

% Start the document body:
%    \begin{macrocode}
\begin{document}
%    \end{macrocode}

% Declare a title page.
% Print title, part of document being processed and version flag:
%    \begin{macrocode}
\addtocounter{page}{-1}
\begin{center}
{\LARGE\bfseries{}childdoc example\par}
\vspace{1cm}
\ifchilddoc
\ifchilddocmanual part\else chapter\fi:
`\childdocname' of `\childdocjob'\par
\else
main document: `\childdocjob'\par
\fi
version: \version\par
\end{center}
\newpage
%    \end{macrocode}

% Manually include selected file,
% otherwise process as usual:
%    \begin{macrocode}
\ifchilddocmanual
\section*{part `\childdocname'}
\input{\childdocname}
\else
%    \end{macrocode}

% Include the two chapters:
%    \begin{macrocode}
\include{cdocsch1}
\include{cdocsch2}
%    \end{macrocode}

% Include the two parts unless only chapters should be displayed:
%    \begin{macrocode}
\ifchilddoc\else
\section{part three}
\input{cdocspt3}
\section{part four}
\input{cdocspt4}
\fi
%    \end{macrocode}

% Process as usual until here:
%    \begin{macrocode}
\fi
%    \end{macrocode}

% End of document body:
%    \begin{macrocode}
\end{document}
%    \end{macrocode}
%\iffalse
%</samplemain>
%\fi
%
% %%%%%%%%%%%%%%%%%%%%%%%%%%%%%%%%%%%%%%
% \paragraph{Chapter Include Files.}
%
% The include files are called |cdocsch1.tex| and |cdocsch2.tex|.
%
%\iffalse
%<*samplechap1|samplechap2>
%\fi

% Optional override for |\version| flag:
%    \begin{macrocode}
%%\providecommand{\version}{final}
%    \end{macrocode}

% Include the main document:
%    \begin{macrocode}
\input{childdoc.def}
\childdocof{cdocsamp}
%    \end{macrocode}

%\iffalse
%</samplechap1|samplechap2>
%\fi
%
%\iffalse
%<*samplechap1>
%\fi
% Some text for chapter 1:
%    \begin{macrocode}
\section{one}
some text in chapter one
%    \end{macrocode}

%\iffalse
%</samplechap1>
%\fi
% Some text for chapter 2:
%\iffalse
%<*samplechap2>
%\fi
%    \begin{macrocode}
\section{two}
more text in chapter two
%    \end{macrocode}

%\iffalse
%</samplechap2>
%\fi
%
% %%%%%%%%%%%%%%%%%%%%%%%%%%%%%%%%%%%%%%
% \paragraph{Part Include Files.}
%
% The include files are called |cdocspt3.tex| and |cdocspt4.tex|.
%
%\iffalse
%<*samplepart3|samplepart4>
%\fi

% Optional override for |\version| flag:
%    \begin{macrocode}
%%\providecommand{\version}{final}
%    \end{macrocode}

% Include the main document:
%    \begin{macrocode}
\input{childdoc.def}
\childdocby{cdocsamp}
%    \end{macrocode}

%\iffalse
%</samplepart3|samplepart4>
%\fi
%
%\iffalse
%<*samplepart3>
%\fi
% Some text for part 3:
%    \begin{macrocode}
some text in part three
%    \end{macrocode}

%\iffalse
%</samplepart3>
%\fi
% Some text for part 4:
%\iffalse
%<*samplepart4>
%\fi
%    \begin{macrocode}
more text in part four
%    \end{macrocode}

%\iffalse
%</samplepart4>
%\fi
%
% %%%%%%%%%%%%%%%%%%%%%%%%%%%%%%%%%%%%%%
% \paragraph{Forwarding for a Complete Draft.}
%
% The following forwarding file |cdocsdrf.tex|
% compiles the main document in draft mode:
%\iffalse
%<*sampledraft>
%\fi
%    \begin{macrocode}
\def\version{draft}
\input{childdoc.def}
\childdocforward{cdocsamp}
%    \end{macrocode}

%\iffalse
%</sampledraft>
%\fi
%
% %%%%%%%%%%%%%%%%%%%%%%%%%%%%%%%%%%%%%%
% \paragraph{Forwarding for Final Version of the Chapters.}
%
% The following forwarding files |cdocsfn1.tex| and |cdocsfn2.tex|
% (with identical content)
% compile the final versions of the child documents
% |cdocsch1.tex| and |cdocsch2.tex|, respectively:
%\iffalse
%<*samplefinal>
%\fi
%    \begin{macrocode}
\def\version{final}
\input{childdoc.def}
\childdocforwardprefix[cdocsamp]{cdocsfn}{cdocsch}
%    \end{macrocode}

%\iffalse
%</samplefinal>
%\fi
%
% %%%%%%%%%%%%%%%%%%%%%%%%%%%%%%%%%%%%%%
% \paragraph{Command Line Processing.}
%
% The following three command lines generate the output files
% |cdocscld|, |cdocscl1| and |cdocscl2|
% which should be identical to
% |cdocsdrf|, |cdocsch1| and |cdocsfn2|, respectively:
% \begin{center}
% \begin{tabular}{l}
% |latex -jobname cdocscld \|\\
% |  "\def\version{draft}\input{childdoc.def}\childdocforward{cdocsamp}"|\\
% |latex -jobname cdocscl1 \|\\
% |  "\input{childdoc.def}\childdocforward[cdocsamp]{cdocsch1}"|\\
% |latex -jobname cdocscl2 \|\\
% |  "\def\version{final}\input{childdoc.def}\childdocforward{cdocsch2}"|
% \end{tabular}
% \end{center}
% Note that the trailing backslash on each first line
% merely continues the input to the second line
% (for convenient cut ant paste).
% Furthermore, the command |latex| can be replaced by any
% of its alternative versions such as |pdflatex|.
%
% %%%%%%%%%%%%%%%%%%%%%%%%%%%%%%%%%%%%%%%%%%%%%%%%%%%%%%%%%%%%%%%%%%%%%%%%%%%%%%
% %%%%%%%%%%%%%%%%%%%%%%%%%%%%%%%%%%%%%%%%%%%%%%%%%%%%%%%%%%%%%%%%%%%%%%%%%%%%%%
% \section{Implementation}
%\iffalse
%<*package>
%\fi
%
% This section describes the definitions file |childdoc.def|.

% The definitions cannot be loaded using |\usepackage| or |\RequirePackage|
% which has a mechanism to prevent loading a style file more than once.
% When loading the definitions by means of |\input|
% multiple instances have to be prevented manually:
%\iffalse
%This code needs to be before the `\ProvidesFile' directive
%which is defined at the beginning of this file.
%Therefore it is also placed there and commented out here.
%</package>
%<*discard>
%\fi
%    \begin{macrocode}
\ifdefined\childdocmain\endinput\fi
%    \end{macrocode}
%\iffalse
%</discard>
%<*package>
%\fi
%
% \macro{\ifchilddoc}
% \macro{\ifchilddocmanual}
% The conditional |\ifchilddoc| tells whether a
% child (true) or main (false) document is being compiled.
% The conditional |\ifchilddocmanual| tells whether
% the |\includeonly| mechanism is used (false) or
% the selection of child files must be performed manually (true).
% The definitions initialise to false:
%    \begin{macrocode}
\newif\ifchilddoc
\newif\ifchilddocmanual
%    \end{macrocode}

% \macro{\childdocname}
% \macro{\childdocjob}
% The macro |\childdocname| stores the name of the main document
% to be compiled. The macro |\childdocjob| stores the name of
% the document on which the \LaTeX{} compiler was originally invoked.
% The content of |\jobname| cannot be compared
% to filenames specified in the source due to different catcodes.
% The following code rescans |\jobname|, stores the result
% in |\childdocname| and saves a copy in |\childdocjob|:
%    \begin{macrocode}
\edef\childdocname{\scantokens\expandafter{\jobname\noexpand}}
\let\childdocjob\childdocname
%    \end{macrocode}

% \macro{\childdocdisable}
% The macro |\childdocdisable| prevents the main file
% from being processed more than once.
% At this stage, the main document command |\childdocmain|
% is assumed to be called once again where it should do nothing.
% Any subsequent call to it should prevent
% a secondary processing of the main document
% It overwrites the forwarding commands
% |\childdocof| and |\childdocforward|
% with empty macros to prevent further inclusions of the main document:
%    \begin{macrocode}
\newcommand{\childdocdisable}
{
  \renewcommand{\childdocmain}[1]{\renewcommand{\childdocmain}[1]{\endinput}}
  \renewcommand{\childdocof}[1]{}
  \renewcommand{\childdocby}[2][]{}
  \renewcommand{\childdocforward}[2][]{}
  \renewcommand{\childdocdisable}{}
}
%    \end{macrocode}

% \macro{\childdocmain}
% The macro |\childdocmain| is to be called at the top of the main file
% with nothing or the main filename (without extension) as argument.
% First, it breaks loops.
% If the argument is not empty and does not match |\childdocname|
% (which is set by the first inclusion of |childdoc.def|),
% |\ifchilddoc| is set to true, |\includeonly| is applied to the child file
% and |\jobname| is set to the main file
% (for proper handling of |.aux| files):
%    \begin{macrocode}
\newcommand{\childdocmain}[1]
{
  \childdocdisable\childdocmain{}
  \if?#1?\else
    \begingroup
      \def\childdoctmp{#1}
      \ifx\childdoctmp\childdocname
        \def\childdoctmp{}
      \else
        \def\childdoctmp
        {
          \childdoctrue
          \includeonly{\childdocname}
          \def\childdocjob{#1}
          \def\jobname{#1}
        }
      \fi
      \expandafter
    \endgroup
    \childdoctmp
  \fi
}
%    \end{macrocode}

% \macro{\childdocof}
% The command |\childdocof| redirects
% compilation to the main file |#1|.
%    \begin{macrocode}
\newcommand{\childdocof}[1]
{
  \childdocdisable
  \childdoctrue
  \includeonly{\childdocname}
  \def\jobname{#1}
  \def\childdocjob{#1}
  \input{#1}
}
%    \end{macrocode}

% \macro{\childdocby}
% The command |\childdocby| ....
%    \begin{macrocode}
\newcommand{\childdocby}[2][]
{
  \childdocdisable
  \childdoctrue
  \childdocmanualtrue
  \if?#1?\else
    \def\jobname{#2}
  \fi
  \def\childdocjob{#2}
  \input{#2}
  \endinput
}
%    \end{macrocode}

% \macro{\childdocforward}
% The command |\childdocforward| redirects
% compilation to the main file or
% (if the optional argument is given) a child file.
% Parameters are set as if the main file
% or a child file starting with |\childdocof| was compiled.
% Then compilation is handed over to the main file:
%    \begin{macrocode}
\newcommand{\childdocforward}[2][]
{
  \begingroup
    \if?#1?
      \def\childdoctmp
      {
        \def\childdocname{#2}
        \def\childdocjob{#2}
        \def\jobname{#2}
        \input{#2}
        \endinput
      }
    \else
      \def\childdoctmp
      {
        \childdocdisable
        \def\childdocname{#2}
        \childdoctrue
        \includeonly{#2}
        \def\childdocjob{#1}
        \def\jobname{#1}
        \input{#1}
        \endinput
      }
    \fi
    \expandafter
  \endgroup
  \childdoctmp
}
%    \end{macrocode}

% \macro{\childdocforwardprefix}
% The command |\childdocforwardprefix| redirects
% compilation to the main or a child file by means of a pattern.
% The prefix |#1| in the current filename is replaced by |#2|
% and the suffix of the current filename is kept
% (it is assumed that the filename does not contain the substring `|~~~|'
% which is used as a delimiter).
% Compilation is handed over to the new file by |\childdocforward|:
%    \begin{macrocode}
\newcommand{\childdocforwardprefix}[3][]
{
  \begingroup
    \def\childdocextract #2##1~~~{\def\childdoctmp{\childdocforward[#1]{#3##1}}}
    \expandafter\childdocextract\childdocname~~~
    \expandafter
  \endgroup
  \childdoctmp
}
%    \end{macrocode}

% \macro{\childdoc}
% The deprecated macro |\childdoc| is a legacy version of |\childdocmain|:
%    \begin{macrocode}
\newcommand{\childdoc}{\childdocmain}
%    \end{macrocode}

% \macro{\childdocredirect}
% The deprecated macro |\childdocredirect| is a legacy version
% of |\childdocforward| and |\childdocforwardprefix|:
%    \begin{macrocode}
\newcommand{\childdocredirect}[2][]
{
  \begingroup
    \if?#1?
      \def\childdoctmp{\childdocforward{#2}}
    \else
      \def\childdoctmp{\childdocforwardprefix{#1}{#2}}
    \fi
    \expandafter
  \endgroup
  \childdoctmp
}
%    \end{macrocode}

%\iffalse
%</package>
%\fi
%
\endinput

\childdocby{cdocsamp}
%    \end{macrocode}

%\iffalse
%</samplepart3|samplepart4>
%\fi
%
%\iffalse
%<*samplepart3>
%\fi
% Some text for part 3:
%    \begin{macrocode}
some text in part three
%    \end{macrocode}

%\iffalse
%</samplepart3>
%\fi
% Some text for part 4:
%\iffalse
%<*samplepart4>
%\fi
%    \begin{macrocode}
more text in part four
%    \end{macrocode}

%\iffalse
%</samplepart4>
%\fi
%
% %%%%%%%%%%%%%%%%%%%%%%%%%%%%%%%%%%%%%%
% \paragraph{Forwarding for a Complete Draft.}
%
% The following forwarding file |cdocsdrf.tex|
% compiles the main document in draft mode:
%\iffalse
%<*sampledraft>
%\fi
%    \begin{macrocode}
\def\version{draft}
% \iffalse
%
% childdoc.dtx Copyright (C) 2017-2018 Niklas Beisert
%
% This work may be distributed and/or modified under the
% conditions of the LaTeX Project Public License, either version 1.3
% of this license or (at your option) any later version.
% The latest version of this license is in
%   http://www.latex-project.org/lppl.txt
% and version 1.3 or later is part of all distributions of LaTeX
% version 2005/12/01 or later.
%
% This work has the LPPL maintenance status `maintained'.
%
% The Current Maintainer of this work is Niklas Beisert.
%
% This work consists of the files childdoc.dtx and childdoc.ins
% and the derived files childdoc.def and cdocsamp.tex with
% cdocsch1.tex, cdocsch2.tex, cdocsdrf.tex, cdocsfn1.tex, cdocsfn2.tex.
%
%<package>\ifdefined\childdocmain\endinput\fi
%<package>\ProvidesFile{childdoc.def}[2018/12/30 v2.0 child document driver]
%<samplemain>\ProvidesFile{cdocsamp.tex}[2018/12/30 v2.0 sample for childdoc]
%<*driver>
%\ProvidesFile{childdoc.drv}[2018/12/30 v2.0 childdoc reference manual file]
\PassOptionsToClass{10pt,a4paper}{article}
\documentclass{ltxdoc}

\usepackage[margin=35mm]{geometry}
\usepackage{hyperref}
\usepackage{hyperxmp}
\usepackage[usenames]{color}

\hypersetup{colorlinks=true}
\hypersetup{pdfstartview=FitH}
\hypersetup{pdfpagemode=UseNone}
\hypersetup{pdfsource={}}
\hypersetup{pdflang={en-UK}}
\hypersetup{pdfcopyright={Copyright 2017-2018 Niklas Beisert.
  This work may be distributed and/or modified under the
  conditions of the LaTeX Project Public License, either version 1.3
  of this license or (at your option) any later version.}}
\hypersetup{pdflicenseurl={http://www.latex-project.org/lppl.txt}}
\hypersetup{pdfcontactaddress={ETH Zurich, ITP, HIT K,
  Wolfgang-Pauli-Strasse 27}}
\hypersetup{pdfcontactpostcode={8093}}
\hypersetup{pdfcontactcity={Zurich}}
\hypersetup{pdfcontactcountry={Switzerland}}
\hypersetup{pdfcontactemail={nbeisert@itp.phys.ethz.ch}}
\hypersetup{pdfcontacturl={http://people.phys.ethz.ch/\xmptilde nbeisert/}}

\newcommand{\secref}[1]{\hyperref[#1]{section \ref*{#1}}}

\parskip1ex
\parindent0pt
\let\olditemize\itemize
\def\itemize{\olditemize\parskip0pt}

\begin{document}

\title{The \textsf{childdoc} Package}
\hypersetup{pdftitle={The childdoc Package}}
\author{Niklas Beisert\\[2ex]
  Institut f\"ur Theoretische Physik\\
  Eidgen\"ossische Technische Hochschule Z\"urich\\
  Wolfgang-Pauli-Strasse 27, 8093 Z\"urich, Switzerland\\[1ex]
  \href{mailto:nbeisert@itp.phys.ethz.ch}
  {\texttt{nbeisert@itp.phys.ethz.ch}}}
\hypersetup{pdfauthor={Niklas Beisert}}
\hypersetup{pdfsubject={Manual for the LaTeX2e Package childdoc}}
\date{30 December 2018, \textsf{v2.0}}
\maketitle

\begin{abstract}\noindent
\textsf{childdoc} is a \LaTeXe{} package
that enables the direct compilation
of document sections included by |\include|
to individual files.
\end{abstract}

\begingroup
\parskip0ex
\tableofcontents
\endgroup

%%%%%%%%%%%%%%%%%%%%%%%%%%%%%%%%%%%%%%%%%%%%%%%%%%%%%%%%%%%%%%%%%%%%%%%%%%%%%%%%
%%%%%%%%%%%%%%%%%%%%%%%%%%%%%%%%%%%%%%%%%%%%%%%%%%%%%%%%%%%%%%%%%%%%%%%%%%%%%%%%
\section{Introduction}

\LaTeX{} provides a mechanism to structure a large document (such as a book)
into a main file and several child files (containing the chapters)
using the |\include| command.
This mechanism is beneficial for documents
which span hundreds of pages in order to
make the source file(s) more manageable.
Moreover, compilation can be restricted to
selected child files by means of the |\includeonly| command.
The latter feature can be used to reduce the compilation time while editing
(this was significantly more useful in the earlier days of \LaTeX{})
or to generate a smaller document which is easier to navigate.
Another application of |\includeonly| is to generate
documents consisting of selected parts of the complete document.

However, there are a few drawbacks of the plain |\include| mechanism:
\begin{itemize}
\item
The child files cannot be compiled on their own,
they can only be compiled via the main file.
A naive editing environment
(such as a text editor with an option
to have the current file processed by \LaTeX)
may require one to switch to the main file before compiling;
attempting to compile the child file produces errors.
\item
The main file must be modified (each time)
to adjust the |\includeonly| command
to the present needs. This easily leaves the main file in a messy state.
\item
The generated document will always carry the filename
of the main document. This is inconvenient if
several child files are to be compiled and
to be kept for distribution.
\end{itemize}

The present package provides a simple interface
to make child files individually compilable by \LaTeX{}.
Compiling a child file then has the same effect as compiling
the main file with an |\includeonly| command
to select the appropriate child.
Moreover the generated document will carry the name of the child
rather than the main file.
This resolves all three above issues.

This feature is meant to make the editing of books,
thesis documents and lecture notes somewhat more convenient.
However, the package can also be used efficiently for
composing a series of documents (such as exercise sheets)
which are typically distributed individually.
It then assists the author in generating the individual documents
(potentially in different versions)
as well as a document containing the collected series.
Another application is in developing style files
or other kinds of included material
where compilation of the style file could redirect
to a sample or test file.

%%%%%%%%%%%%%%%%%%%%%%%%%%%%%%%%%%%%%%%%%%%%%%%%%%%%%%%%%%%%%%%%%%%%%%%%%%%%%%%%
%%%%%%%%%%%%%%%%%%%%%%%%%%%%%%%%%%%%%%%%%%%%%%%%%%%%%%%%%%%%%%%%%%%%%%%%%%%%%%%%
\section{Usage}

First of all, the package \textsf{childdoc} is \emph{not} a standard
\LaTeXe{} |.sty| style file! Therefore it needs to be invoked in
a non-standard way.

%%%%%%%%%%%%%%%%%%%%%%%%%%%%%%%%%%%%%%%%%%%%%%%%%%%%%%%%%%%%%%%%%%%%%%%%%%%%%%%%
\subsection{Included Files}
\label{sec:include}

%%%%%%%%%%%%%%%%%%%%%%%%%%%%%%%%%%%%%%%%
\DescribeMacro{\childdocmain}
To use the package, add the commands
\begin{center}
\begin{tabular}{l}
|\input{childdoc.def}|\\
|\childdocmain{}|\\
\end{tabular}
\end{center}
at the very top of the main \LaTeX{} file,
in particular \emph{before} the |\documentclass| statement!
The argument of |\childdocmain| should be left empty
(but it must be present).

%%%%%%%%%%%%%%%%%%%%%%%%%%%%%%%%%%%%%%%%
\DescribeMacro{\childdocof}
Furthermore, add the commands
\begin{center}
\begin{tabular}{l}
|\input{childdoc.def}|\\
|\childdocof{|\textit{main}|}|\\
\end{tabular}
\end{center}
at the top of every child file \textit{child}
which is included by |\include{|\textit{child}|}|
from within the main file
(or at least for those files to be compiled individually).
The argument \textit{main} must be the filename of the main file.

There are a couple of
considerations in setting up the main and child documents:

%%%%%%%%%%%%%%%%%%%%%%%%%%%%%%%%%%%%%%%%
\paragraph{Restrictions.}

Please note the following restrictions:
\begin{itemize}
\item
|\childdocmain| must be called with one argument \textit{main}
to ensure compatibility with earlier version of the package.
It must either be empty (|\childdocmain{}|)
or precisely match the filename of the main file in which it is specified.
See \secref{sec:detection} for further information.
\item
The filename \textit{main} must be specified without the |.tex| extension.
\item
The filename \textit{main} is case sensitive
(even in case-insensitive file systems)
due to internal string comparison.
\item
The argument \textit{main} should be fully expanded, it cannot be a macro.
\item
Subdirectories and special characters should be avoided in filenames.
\item
The command |\childdocmain{|\textit{main}|}| must be followed by a whitespace.
It should not be followed immediately by another command
or by a comment mark `|%|'.
This is because the \TeX{} parser reads the token immediately following
the argument of |\childdocmain| and puts it
at the beginning of every child section;
however, a white\-space is ignored.
\end{itemize}

%%%%%%%%%%%%%%%%%%%%%%%%%%%%%%%%%%%%%%%%
\paragraph{Content of Main File.}

It is advisable to place all content in the child files included by |\include|.
Any output contained in the main file will appear in all child documents
unless suppressed manually;
it cannot be suppressed automatically by the |\includeonly| directive
and thus should normally be avoided.
A method to include some content in the main file
by means of conditional processing is described in \secref{sec:conditional}.

%%%%%%%%%%%%%%%%%%%%%%%%%%%%%%%%%%%%%%%%
\paragraph{Page Numbering.}

When only a part of the document is compiled,
the appropriate numbering of pages
(as well as other status parameters)
is determined from the |.aux| files.
The latter contain information from previous passes.
However this information needs to propagate through
all intermediate child documents.
Therefore the page numbering in child documents may well
be inconsistent until the complete document is compiled at least once.

A useful (if unconventional) way to always ensure a consistent
page numbering is to restart the numbering in each child document
and denote the pages by `\textit{child}|.|\textit{page}'
where \textit{child} represents the chapter/section number of the child file.
This can be achieved by the command
|\numberwithin{page}{|\textit{child}|}|
of the \textsf{amsmath} package
where \textit{child} can be |chapter| or |section|
depending on the chosen structuring.
Alternatively, one can modify the macro |\thepage| appropriately
and reset the counter |page| at the start of each child file.

%%%%%%%%%%%%%%%%%%%%%%%%%%%%%%%%%%%%%%%%%%%%%%%%%%%%%%%%%%%%%%%%%%%%%%%%%%%%%%%%
\subsection{Conditional Processing}
\label{sec:conditional}

The package provides a mechanism to compile different versions
of a document. To customise the versions further some conditional processing
can come in handy to distinguish which version is being compiled.
The package provides two macros to describe the compilation context:

%%%%%%%%%%%%%%%%%%%%%%%%%%%%%%%%%%%%%%%%
\DescribeMacro{\ifchilddoc}
The conditional |\ifchilddoc| distinguishes between the compilation of
child documents and the main document:
%
\begin{center}
|\ifchilddoc |\textit{child-code}| |[|\||else |\textit{main-code}]| \||fi|
\end{center}

%%%%%%%%%%%%%%%%%%%%%%%%%%%%%%%%%%%%%%%%
\DescribeMacro{\childdocname}
\DescribeMacro{\childdocjob}
The macro |\childdocname| contains the filename (without extension)
of the main or child file being processed.
Note that |\childdocjob| will always contain the name of the main file.

%%%%%%%%%%%%%%%%%%%%%%%%%%%%%%%%%%%%%%%%
\paragraph{Title Page.}

Conditional processing can be used to include a title or banner page
in the main document when proper precautions are taken.
Importantly, the code in the main file should ensure that the page counter
(as well as other status parameters which are stored in the |.aux| files)
takes the same value after the conditional processing.
Otherwise the page numbers may take divergent values
depending on which part is compiled.

For example, a title page could be declared by:
%
\begin{center}
\begin{tabular}{l}
|\ifchilddoc\||else|\\
|\addtocounter{page}{-1}|\\
\textit{code for title page}\\
|\newpage|\\
|\||fi|
\end{tabular}
\end{center}
%
A banner page for the child documents can be generated by:
%
\begin{center}
\begin{tabular}{l}
|\ifchilddoc|\\
|\addtocounter{page}{-1}|\\
\textit{code for banner page}\\
|\newpage|\\
|\||fi|
\end{tabular}
\end{center}
%
Here one could write a message such as:
\begin{center}
|This is the part \childdocname{} of \childdocjob{}.|
\end{center}

%%%%%%%%%%%%%%%%%%%%%%%%%%%%%%%%%%%%%%%%%%%%%%%%%%%%%%%%%%%%%%%%%%%%%%%%%%%%%%%%
\subsection{Flags}
\label{sec:flags}

The package makes it easy to generate different versions
of the main or child documents.
To this end compilation flags can be defined
and assigned different default values.
They will be particularly useful in conjunction
with the forwarding mechanism described in \secref{sec:forward}.

For example, it may be useful to have a flag |\version|
which can be set to |draft| or |final|.
The document source will contain some conditional code
depending on the value of |\version|.
Suppose further, the flag should default to |final| for the main file
and to |draft| for child files
which is a natural assignment for editing the document.
This is achieved by placing the following code
in the preamble of the main document
(below the |\childdocmain| directive):
%
\begin{center}
\begin{tabular}{l}
|\ifchilddoc|\\
|\providecommand{\version}{draft}|\\
|\||else|\\
|\providecommand{\version}{final}|\\
|\||fi|
\end{tabular}
\end{center}
%
The definition by |\providecommand| makes sure
that previous definitions are not overwritten.
Further statements |\providecommand{\version}{...}|
can thus be added before the above code to override it.

For the main file, one might add a line
(between |\childdocmain| and the above block)
%
\begin{center}
|%\ifchilddoc\||else\providecommand{\version}{draft}\||fi|
\end{center}
%
which can be uncommented to produce a draft version.
Likewise one can add a line to the very top of a child file
(above the |\childdocof{|\textit{main}|}| directive)
%
\begin{center}
|%\providecommand{\version}{final}|
\end{center}
%
which can be uncommented to produce the final version of this child document.

%%%%%%%%%%%%%%%%%%%%%%%%%%%%%%%%%%%%%%%%%%%%%%%%%%%%%%%%%%%%%%%%%%%%%%%%%%%%%%%%
\subsection{Forwarding}
\label{sec:forward}

Different versions of the main or child documents
using compilation flags as described in \secref{sec:flags}
can be (permanently) stored in different files
for convenient compilation, viewing and distribution.
To this end, the package defines a command
to pass on compilation to a different file:

%%%%%%%%%%%%%%%%%%%%%%%%%%%%%%%%%%%%%%%%
\DescribeMacro{\childdocforward}
The command |\childdocforward| redirects processing to
another source file:
%
\begin{center}
\begin{tabular}{l}
|\input{childdoc.def}|\\
|\childdocforward[|\textit{main}|]{|\textit{dest}|}|\\
\end{tabular}
\end{center}
%
The argument \textit{dest} is the destination file
(without extension).
It should be the main file or one of the child files.
Note that further \textsf{childdoc} directives
such as |\childdocof| and |\childdocforward|
in the indicated file will be processed in this form.
The optional argument \textit{main}
passes on directly to the main file \textit{main}
while pretending to compile the child \textit{dest}.
This form behaves as if \textit{dest}
issues |\childdocof{|\textit{main}|}| right away,
and no further \textsf{childdoc} directives will be processed.

%%%%%%%%%%%%%%%%%%%%%%%%%%%%%%%%%%%%%%%%
\DescribeMacro{\...prefix}
In the alternative form |\childdocforwardprefix|,
%
\begin{center}
\begin{tabular}{l}
|\input{childdoc.def}|\\
|\childdocforwardprefix[|\textit{main}|]{|\textit{prefix}|}{|\textit{dest}|}|
\end{tabular}
\end{center}
%
the destination file is determined by a pattern
depending on the current file:
To make this work, the current file must be called
`{\textit{prefix}\hspace{0.2em}\textit{suffix}}'
with \textit{prefix} matching precisely the argument.
Processing is then passed on to the file
`{\textit{dest}\hspace{0.2em}\textit{suffix}}'.
Surely, the same effect is achieved by
directly specifying the
argument `{\textit{dest}\hspace{0.2em}\textit{suffix}}'
in the first form.
However, that requires to set up a different file
for each child. With the alternative form of the command
all these files can have exactly the same content
which simplifies setting them up and maintaining them.

For example, the following file |draft.tex|
with a compilation flag |\version| as described in \secref{sec:flags}
compiles the main document as a draft:
%
\begin{center}
\begin{tabular}{l}
|\def\version{draft}|\\
|\input{childdoc.def}|\\
|\childdocforward{|\textit{main}|}|
\end{tabular}
\end{center}
%
Likewise, the following files |final|\textit{nn}|.tex|
compile the final version of the child document
|child|\textit{nn}|.tex|:
%
\begin{center}
\begin{tabular}{l}
|\def\version{final}|\\
|\input{childdoc.def}|\\
|\childdocforwardprefix{final}{child}|
\end{tabular}
\end{center}
%

Note that when several versions of a main file and/or of each child file
are to be generated, it may be convenient to set up a |Makefile| or
shell script to automatise the process.

%%%%%%%%%%%%%%%%%%%%%%%%%%%%%%%%%%%%%%%%%%%%%%%%%%%%%%%%%%%%%%%%%%%%%%%%%%%%%%%%
\subsection{Command Line Processing}
\label{sec:commandline}

The effect of redirection files can also be achieved by invoking
the \LaTeX{} compiler with a more elaborate command line.
Most conveniently this should be done as part
of a shell script or a |Makefile|.

When using \textsf{childdoc} in the main file, the following
command lines effectively perform a redirection
(note that depending on the shell being used,
backslashes may have to be doubled: `|\|' $\to$ `|\\|'):
%
\begin{center}
|... -jobname "|\textit{target}|" |\\|"|[\textit{flags}]%
|\input{childdoc.def}\childdocforward[|\textit{main}|]{|\textit{dest}|}"|
\end{center}
%
Here \textit{target} is the name of the output file,
\textit{main} is the name of the main file
and \textit{dest} is the name of the main or child file to be processed
(all filenames without extensions).
The optional argument \textit{main} can be omitted
if \textit{main} matches \textit{dest}.
Optionally, compilation \textit{flags} can be defined via |\def| commands.
This command line makes the \TeX{} engine believe
it is compiling the file \textit{target}
whose content is specified as the latter parameter.
The provided code then forwards the processing to
\textit{main} or \textit{dest} as described in \secref{sec:forward}.

%%%%%%%%%%%%%%%%%%%%%%%%%%%%%%%%%%%%%%%%%%%%%%%%%%%%%%%%%%%%%%%%%%%%%%%%%%%%%%%%
\subsection{Include by Input}
\label{sec:input}

Including child documents by |\include| has some restrictions by design.
Most notably, the content of a child document always occupies
its own set of pages; pages cannot be shared between child documents.
Usually, this behaviour makes perfect sense
because each child document contain an essential part of the document.
However, in some situations it may be desirable to compose
a document from a collection of parts
without having mandatory page breaks between then.
For this case, the package
provides a mechanism to include parts
by |\input| which can also be processed individually.
However, by construction this mechanism
requires manual handling of the content to be output.

%%%%%%%%%%%%%%%%%%%%%%%%%%%%%%%%%%%%%%%%
\DescribeMacro{\ifchilddocmanual}
The main file should be prepared as usual, see \secref{sec:include}.
However, the document body must make a distinction
between processing of an individual part and of the main document, e.g.:
%
\begin{center}
\begin{tabular}{l}
|\ifchilddocmanual|\\
|\input{\childdocname}|\\
|\||else|\\
\textit{document body with }|\input{|\textit{part}|}|\\
|\||fi|
\end{tabular}
\end{center}
%
The conditional |\ifchilddocmanual| is true whenever
a part to be included by |\input| is being compiled,
and the name of the part is stored in |\childdocname|.

%%%%%%%%%%%%%%%%%%%%%%%%%%%%%%%%%%%%%%%%
\DescribeMacro{\childdocby}
Each part to be included by |\input| should start with:
%
\begin{center}
\begin{tabular}{l}
|\input{childdoc.def}|\\
|\childdocby{|\textit{main}|}|\\
\end{tabular}
\end{center}
%
The directive |\childdocby| is similar to |\childdocof|
described in \secref{sec:include},
but the subsequent selection of content must be done manually.
To that end, both |\ifchilddoc| and |\ifchilddocmanual|
will be true upon processing of a part,
and the name of the part is stored in |\childdocname|.
Note that |\jobname| will be set to the filename of the current part
so that each part receives an individual |.aux| file
that does not interfere with the |.aux| file(s) of the main document.
This behaviour can be altered by the alternative form
|\childdocby[*]{|\textit{main}|}| (with a non-empty optional argument)
which uses the |.aux| file of the main document
by setting |\jobname| to \textit{main}.

%%%%%%%%%%%%%%%%%%%%%%%%%%%%%%%%%%%%%%%%%%%%%%%%%%%%%%%%%%%%%%%%%%%%%%%%%%%%%%%%
\subsection{Driver Development}
\label{sec:driver}

The \textsf{childdoc} mechanism can also be use for the development
of definition files such as \LaTeX{} styles or classes.
This case differs from the above setup with multiple parts
included by |\include| in that no |\includeonly| should be invoked.
This can be achieved by starting the include file
(before |\ProvidesPackage|) with:
%
\begin{center}
\begin{tabular}{l}
|\input{childdoc.def}|\\
|\childdocforward{|\textit{main}|}|\\
\end{tabular}
\end{center}
%
or alternatively with:
%
\begin{center}
\begin{tabular}{l}
|\input{childdoc.def}|\\
|\childdocby{|\textit{main}|}|\\
\end{tabular}
\end{center}
%
Both forms have slightly different effects as described above.
The main file is prepared as usual, see \secref{sec:include}.

%%%%%%%%%%%%%%%%%%%%%%%%%%%%%%%%%%%%%%%%%%%%%%%%%%%%%%%%%%%%%%%%%%%%%%%%%%%%%%%%
\subsection{Legacy Detection}
\label{sec:detection}

The directive |\childdocmain| in the main file can detect
whether the complete document or merely a child is to be compiled
even without using the directive |\childdocof|.
This method is deprecated because it is less robust
and there is no compelling reason to use it;
it is merely provided for backward compatibility
and it may be removed in future versions.

If the detection mechanism is to be used,
it is mandatory to correctly specify
the filename of the main file as the argument of |\childdocmain|:
%
\begin{center}
\begin{tabular}{l}
|\input{childdoc.def}|\\
|\childdocmain{|\textit{main}|}|\\
\end{tabular}
\end{center}
%
If |\jobname| does not match the argument \textit{main} of |\childdocmain|,
it is assumed that |\jobname| points to the child file to be compiled.
When using |\childdocmain| with the main file specified as argument,
it suffices to start a child file
with just |\input{|\textit{main}|}|
without loading of the package and using |\childdocof|.
If instead all processing is done
with the appropriate \textsf{childdoc} directives,
the argument of \textit{main} of |\childdocmain| can be empty.

An alternative version of the command line processing described
in \secref{sec:commandline} using the detection mechanism reads:
%
\begin{center}
|... -jobname "|\textit{target}|" "|[\textit{flags}]%
[|\def\jobname{|\textit{dest}|}|]|\input{|\textit{main}|}"|
\end{center}

%%%%%%%%%%%%%%%%%%%%%%%%%%%%%%%%%%%%%%%%%%%%%%%%%%%%%%%%%%%%%%%%%%%%%%%%%%%%%%%%
\subsection{Manual Code}
\label{sec:manual}

In case one cannot be certain whether the definitions file |childdoc.def|
is installed on the target \TeX{} distribution
and one prefers not to ship it,
it is conceivable to paste a few relevant commands into the sources.

To that end, drop all statements |\input{childdoc.def}|
and perform the replacements as outlined below.
Instead of |\childdocmain{|\textit{main}|}| add the following code
to the top of the main file:
%
\begin{center}
\begin{tabular}{l}
|\||ifdefined\childdocname\endinput\||fi\newif\ifchilddoc|\\
|\edef\childdocname{\scantokens\expandafter{\jobname\noexpand}}|\\
|\def\childdocmain{|\textit{main}|}\||ifx\childdocmain\childdocname\||else|\\
|\childdoctrue\includeonly{\childdocname}\let\jobname\childdocmain\||fi|\\
\end{tabular}
\end{center}
%
Instead of |\childdocof{|\textit{main}|}| just include the main file
at the top of each child file:
%
\begin{center}
|\input{|\textit{main}|}|
\end{center}
%
A simple redirection |\childdocforward{|\textit{dest}|}| is achieved by:
%
\begin{center}
|\def\jobname{|\textit{dest}|}\input{\jobname}|
\end{center}
%
The redirection with prefix
|\childdocforwardprefix[|\textit{prefix}|]{|\textit{dest}|}|
is accomplished by:
%
\begin{center}
\begin{tabular}{l}
|{\edef\jobname{\scantokens\expandafter{\jobname\noexpand}}|\\
|\def\redirectjob |\textit{prefix}|#1~~~{\gdef\jobname{|\textit{dest}|#1}}|\\
|\expandafter\redirectjob\jobname~~~}\input{\jobname}|
\end{tabular}
\end{center}

In an alternative approach,
child documents can be compiled by a specific command line
without additional code or specific definitions:
%
\begin{center}
|... -jobname "|\textit{target}|" "|[\textit{flags}]%
|\includeonly{|\textit{dest}|}\input{|\textit{main}|}"|
\end{center}
%

%%%%%%%%%%%%%%%%%%%%%%%%%%%%%%%%%%%%%%%%%%%%%%%%%%%%%%%%%%%%%%%%%%%%%%%%%%%%%%%%
%%%%%%%%%%%%%%%%%%%%%%%%%%%%%%%%%%%%%%%%%%%%%%%%%%%%%%%%%%%%%%%%%%%%%%%%%%%%%%%%
\section{Information}

%%%%%%%%%%%%%%%%%%%%%%%%%%%%%%%%%%%%%%%%%%%%%%%%%%%%%%%%%%%%%%%%%%%%%%%%%%%%%%%%
\subsection{Copyright}

Copyright \copyright{} 2017--2018 Niklas Beisert

This work may be distributed and/or modified under the
conditions of the \LaTeX{} Project Public License, either version 1.3
of this license or (at your option) any later version.
The latest version of this license is in
  \url{http://www.latex-project.org/lppl.txt}
and version 1.3 or later is part of all distributions of \LaTeX{}
version 2005/12/01 or later.

This work has the LPPL maintenance status `maintained'.

The Current Maintainer of this work is Niklas Beisert.

This work consists of the files |README.txt|, |childdoc.ins| and |childdoc.dtx|
as well as the derived files |childdoc.def|, |cdocsamp.tex|
with |cdocsch1.tex|, |cdocsch2.tex|, |cdocspt3.tex|, |cdocspt4.tex|,
|cdocsdrf.tex|, |cdocsfn1.tex|, |cdocsfn2.tex|
as well as |childdoc.pdf|.

%%%%%%%%%%%%%%%%%%%%%%%%%%%%%%%%%%%%%%%%%%%%%%%%%%%%%%%%%%%%%%%%%%%%%%%%%%%%%%%%
\subsection{Files and Installation}

The package consists of the files:
%
\begin{center}
\begin{tabular}{ll}
    |README.txt|   & readme file \\
    |childdoc.ins| & installation file \\
    |childdoc.dtx| & source file \\
    |childdoc.def| & definition file \\
    |cdocsamp.tex| & sample main file \\
    |cdocsch1.tex| & sample include file \\
    |cdocsch2.tex| & sample include file \\
    |cdocspt3.tex| & sample part file \\
    |cdocspt4.tex| & sample part file \\
    |cdocsdrf.tex| & sample redirection file \\
    |cdocsfn1.tex| & sample redirection file \\
    |cdocsfn2.tex| & sample redirection file \\
    |childdoc.pdf| & manual
\end{tabular}
\end{center}
%
The distribution consists of the files
|README.txt|, |childdoc.ins| and |childdoc.dtx|.
%
\begin{itemize}
\item
Run (pdf)\LaTeX{} on |childdoc.dtx|
to compile the manual |childdoc.pdf| (this file).
\item
Run \LaTeX{} on |childdoc.ins| to create the definitions file |childdoc.def|
and the sample |cdocsamp.tex| with include files
|cdocsch1.tex|, |cdocsch2.tex|, |cdocspt3.tex|, |cdocspt4.tex|,
|cdocsdrf.tex|, |cdocsfn1.tex|, |cdocsfn2.tex|.
Then copy the file |childdoc.def| to an appropriate directory of your \LaTeX{}
distribution, e.g.\ \textit{texmf-root}|/tex/latex/childdoc|.
\end{itemize}

%%%%%%%%%%%%%%%%%%%%%%%%%%%%%%%%%%%%%%%%%%%%%%%%%%%%%%%%%%%%%%%%%%%%%%%%%%%%%%%%
\subsection{Related CTAN Packages}

There are several other packages which offer a similar functionality:
%
\begin{itemize}
\item
The packages
\href{http://ctan.org/pkg/docmute}{\textsf{docmute}},
\href{http://ctan.org/pkg/includex}{\textsf{includex}} and
\href{http://ctan.org/pkg/standalone}{\textsf{standalone}}
provide commands to include only the document body of
a child file thus allowing both files to be compiled individually.
\item
The packages \href{http://ctan.org/pkg/subdocs}{\textsf{subdocs}}
and \href{http://ctan.org/pkg/subfiles}{\textsf{subfiles}}
provide structures in which the main and child documents can be
encapsulated and allowing them to be compiled individually.
The inclusion mechanism is different from the conventional |\include|.
\item
The package \href{http://ctan.org/pkg/combine}{\textsf{combine}}
is an elaborate solution to combine several documents into one.
\end{itemize}
%
See also the CTAN topic \href{http://ctan.org/topic/subdocs}{\textsf{subdocs}}
for further related packages.
The present package differs from the above solutions in that
a document structure constructed with the conventional |\include| mechanism
just needs two extra commands at the top of every file
such that all constituent files can be compiled individually.

%%%%%%%%%%%%%%%%%%%%%%%%%%%%%%%%%%%%%%%%%%%%%%%%%%%%%%%%%%%%%%%%%%%%%%%%%%%%%%%%
%\subsection{Feature Suggestions}
%
%The following is a list of features which may be useful for future
%versions of this package:
%%
%\begin{itemize}
%\item
%\ldots
%\end{itemize}

%%%%%%%%%%%%%%%%%%%%%%%%%%%%%%%%%%%%%%%%%%%%%%%%%%%%%%%%%%%%%%%%%%%%%%%%%%%%%%%%
\subsection{Revision History}

%%%%%%%%%%%%%%%%%%%%%%%%%%%%%%%%%%%%%%%%
\paragraph{v2.0:} 2018/12/30

\begin{itemize}
\item
immediate forward processing
\item
added |\childdocby| mechanism
\item
manual restructured
\end{itemize}

%%%%%%%%%%%%%%%%%%%%%%%%%%%%%%%%%%%%%%%%
\paragraph{v1.6:} 2018/01/17

\begin{itemize}
\item
application for development of include files
\item
corrections to manual
\end{itemize}

%%%%%%%%%%%%%%%%%%%%%%%%%%%%%%%%%%%%%%%%
\paragraph{v1.5:} 2017/05/21

\begin{itemize}
\item
more complete structuring introduced
\item
|\childdocof| introduced
\item
|\childdoc| renamed to |\childdocmain|
\item
|\childredirect| renamed to |\childdocforward| and |\childdocforwardprefix|
and functionality expanded
\end{itemize}

%%%%%%%%%%%%%%%%%%%%%%%%%%%%%%%%%%%%%%%%
\paragraph{v1.0:} 2017/04/27

\begin{itemize}
\item
manual and install package
\item
first version published on CTAN
\end{itemize}

%%%%%%%%%%%%%%%%%%%%%%%%%%%%%%%%%%%%%%%%
\paragraph{v0.6:} 2017/04/26

\begin{itemize}
\item
redirection mechanism added
\end{itemize}

%%%%%%%%%%%%%%%%%%%%%%%%%%%%%%%%%%%%%%%%
\paragraph{v0.5:} 2017/04/26

\begin{itemize}
\item
functionality in definition file
\end{itemize}


%%%%%%%%%%%%%%%%%%%%%%%%%%%%%%%%%%%%%%%%%%%%%%%%%%%%%%%%%%%%%%%%%%%%%%%%%%%%%%%%
%%%%%%%%%%%%%%%%%%%%%%%%%%%%%%%%%%%%%%%%%%%%%%%%%%%%%%%%%%%%%%%%%%%%%%%%%%%%%%%%
%%%%%%%%%%%%%%%%%%%%%%%%%%%%%%%%%%%%%%%%%%%%%%%%%%%%%%%%%%%%%%%%%%%%%%%%%%%%%%%%
\appendix

\settowidth\MacroIndent{\rmfamily\scriptsize 000\ }

 \DocInput{childdoc.dtx}

\end{document}
%</driver>
% \fi
%
% %%%%%%%%%%%%%%%%%%%%%%%%%%%%%%%%%%%%%%%%%%%%%%%%%%%%%%%%%%%%%%%%%%%%%%%%%%%%%%
% %%%%%%%%%%%%%%%%%%%%%%%%%%%%%%%%%%%%%%%%%%%%%%%%%%%%%%%%%%%%%%%%%%%%%%%%%%%%%%
% \section{Sample}
%\iffalse
%<*samplemain>
%\fi
%
% The following presents a sample document
% with two chapters, two parts, a title page,
% a compile flag as well as three forwarding files to set the flag.
% It consists of eight |.tex| files:
% \begin{center}
% \begin{tabular}{ll}
% |cdocsamp.tex|&main file\\
% |cdocsch1.tex|&include file for chapter 1\\
% |cdocsch2.tex|&include file for chapter 2\\
% |cdocspt3.tex|&include file for part 3\\
% |cdocspt4.tex|&include file for part 4\\
% |cdocsdrf.tex|&forwarding file for main file in draft mode\\
% |cdocsfi1.tex|&forwarding file for final version of chapter 1\\
% |cdocsfi2.tex|&forwarding file for final version of chapter 2\\
% \end{tabular}
% \end{center}
% Each of the eight files can be compiled directly by the \LaTeX{} compiler.
%
% %%%%%%%%%%%%%%%%%%%%%%%%%%%%%%%%%%%%%%
% \paragraph{Main File.}
%
% The main file is called |cdocsamp.tex|.
%
% Load the \textsf{childdoc} definitions and
% declare the filename for the main document:
%    \begin{macrocode}
\input{childdoc.def}
\childdocmain{}
%    \end{macrocode}

% Optional override for |\version| flag:
%    \begin{macrocode}
%%\ifchilddoc\else\providecommand{\version}{draft}\fi
%    \end{macrocode}

% Define the default values for the |\version| flag
% (|final| for the main file and |draft| for childs):
%    \begin{macrocode}
\ifchilddoc
\providecommand{\version}{draft}
\else
\providecommand{\version}{final}
\fi
%    \end{macrocode}

% Load the standard document class:
%    \begin{macrocode}
\documentclass[12pt]{article}
%    \end{macrocode}

% Start the document body:
%    \begin{macrocode}
\begin{document}
%    \end{macrocode}

% Declare a title page.
% Print title, part of document being processed and version flag:
%    \begin{macrocode}
\addtocounter{page}{-1}
\begin{center}
{\LARGE\bfseries{}childdoc example\par}
\vspace{1cm}
\ifchilddoc
\ifchilddocmanual part\else chapter\fi:
`\childdocname' of `\childdocjob'\par
\else
main document: `\childdocjob'\par
\fi
version: \version\par
\end{center}
\newpage
%    \end{macrocode}

% Manually include selected file,
% otherwise process as usual:
%    \begin{macrocode}
\ifchilddocmanual
\section*{part `\childdocname'}
\input{\childdocname}
\else
%    \end{macrocode}

% Include the two chapters:
%    \begin{macrocode}
\include{cdocsch1}
\include{cdocsch2}
%    \end{macrocode}

% Include the two parts unless only chapters should be displayed:
%    \begin{macrocode}
\ifchilddoc\else
\section{part three}
\input{cdocspt3}
\section{part four}
\input{cdocspt4}
\fi
%    \end{macrocode}

% Process as usual until here:
%    \begin{macrocode}
\fi
%    \end{macrocode}

% End of document body:
%    \begin{macrocode}
\end{document}
%    \end{macrocode}
%\iffalse
%</samplemain>
%\fi
%
% %%%%%%%%%%%%%%%%%%%%%%%%%%%%%%%%%%%%%%
% \paragraph{Chapter Include Files.}
%
% The include files are called |cdocsch1.tex| and |cdocsch2.tex|.
%
%\iffalse
%<*samplechap1|samplechap2>
%\fi

% Optional override for |\version| flag:
%    \begin{macrocode}
%%\providecommand{\version}{final}
%    \end{macrocode}

% Include the main document:
%    \begin{macrocode}
\input{childdoc.def}
\childdocof{cdocsamp}
%    \end{macrocode}

%\iffalse
%</samplechap1|samplechap2>
%\fi
%
%\iffalse
%<*samplechap1>
%\fi
% Some text for chapter 1:
%    \begin{macrocode}
\section{one}
some text in chapter one
%    \end{macrocode}

%\iffalse
%</samplechap1>
%\fi
% Some text for chapter 2:
%\iffalse
%<*samplechap2>
%\fi
%    \begin{macrocode}
\section{two}
more text in chapter two
%    \end{macrocode}

%\iffalse
%</samplechap2>
%\fi
%
% %%%%%%%%%%%%%%%%%%%%%%%%%%%%%%%%%%%%%%
% \paragraph{Part Include Files.}
%
% The include files are called |cdocspt3.tex| and |cdocspt4.tex|.
%
%\iffalse
%<*samplepart3|samplepart4>
%\fi

% Optional override for |\version| flag:
%    \begin{macrocode}
%%\providecommand{\version}{final}
%    \end{macrocode}

% Include the main document:
%    \begin{macrocode}
\input{childdoc.def}
\childdocby{cdocsamp}
%    \end{macrocode}

%\iffalse
%</samplepart3|samplepart4>
%\fi
%
%\iffalse
%<*samplepart3>
%\fi
% Some text for part 3:
%    \begin{macrocode}
some text in part three
%    \end{macrocode}

%\iffalse
%</samplepart3>
%\fi
% Some text for part 4:
%\iffalse
%<*samplepart4>
%\fi
%    \begin{macrocode}
more text in part four
%    \end{macrocode}

%\iffalse
%</samplepart4>
%\fi
%
% %%%%%%%%%%%%%%%%%%%%%%%%%%%%%%%%%%%%%%
% \paragraph{Forwarding for a Complete Draft.}
%
% The following forwarding file |cdocsdrf.tex|
% compiles the main document in draft mode:
%\iffalse
%<*sampledraft>
%\fi
%    \begin{macrocode}
\def\version{draft}
\input{childdoc.def}
\childdocforward{cdocsamp}
%    \end{macrocode}

%\iffalse
%</sampledraft>
%\fi
%
% %%%%%%%%%%%%%%%%%%%%%%%%%%%%%%%%%%%%%%
% \paragraph{Forwarding for Final Version of the Chapters.}
%
% The following forwarding files |cdocsfn1.tex| and |cdocsfn2.tex|
% (with identical content)
% compile the final versions of the child documents
% |cdocsch1.tex| and |cdocsch2.tex|, respectively:
%\iffalse
%<*samplefinal>
%\fi
%    \begin{macrocode}
\def\version{final}
\input{childdoc.def}
\childdocforwardprefix[cdocsamp]{cdocsfn}{cdocsch}
%    \end{macrocode}

%\iffalse
%</samplefinal>
%\fi
%
% %%%%%%%%%%%%%%%%%%%%%%%%%%%%%%%%%%%%%%
% \paragraph{Command Line Processing.}
%
% The following three command lines generate the output files
% |cdocscld|, |cdocscl1| and |cdocscl2|
% which should be identical to
% |cdocsdrf|, |cdocsch1| and |cdocsfn2|, respectively:
% \begin{center}
% \begin{tabular}{l}
% |latex -jobname cdocscld \|\\
% |  "\def\version{draft}\input{childdoc.def}\childdocforward{cdocsamp}"|\\
% |latex -jobname cdocscl1 \|\\
% |  "\input{childdoc.def}\childdocforward[cdocsamp]{cdocsch1}"|\\
% |latex -jobname cdocscl2 \|\\
% |  "\def\version{final}\input{childdoc.def}\childdocforward{cdocsch2}"|
% \end{tabular}
% \end{center}
% Note that the trailing backslash on each first line
% merely continues the input to the second line
% (for convenient cut ant paste).
% Furthermore, the command |latex| can be replaced by any
% of its alternative versions such as |pdflatex|.
%
% %%%%%%%%%%%%%%%%%%%%%%%%%%%%%%%%%%%%%%%%%%%%%%%%%%%%%%%%%%%%%%%%%%%%%%%%%%%%%%
% %%%%%%%%%%%%%%%%%%%%%%%%%%%%%%%%%%%%%%%%%%%%%%%%%%%%%%%%%%%%%%%%%%%%%%%%%%%%%%
% \section{Implementation}
%\iffalse
%<*package>
%\fi
%
% This section describes the definitions file |childdoc.def|.

% The definitions cannot be loaded using |\usepackage| or |\RequirePackage|
% which has a mechanism to prevent loading a style file more than once.
% When loading the definitions by means of |\input|
% multiple instances have to be prevented manually:
%\iffalse
%This code needs to be before the `\ProvidesFile' directive
%which is defined at the beginning of this file.
%Therefore it is also placed there and commented out here.
%</package>
%<*discard>
%\fi
%    \begin{macrocode}
\ifdefined\childdocmain\endinput\fi
%    \end{macrocode}
%\iffalse
%</discard>
%<*package>
%\fi
%
% \macro{\ifchilddoc}
% \macro{\ifchilddocmanual}
% The conditional |\ifchilddoc| tells whether a
% child (true) or main (false) document is being compiled.
% The conditional |\ifchilddocmanual| tells whether
% the |\includeonly| mechanism is used (false) or
% the selection of child files must be performed manually (true).
% The definitions initialise to false:
%    \begin{macrocode}
\newif\ifchilddoc
\newif\ifchilddocmanual
%    \end{macrocode}

% \macro{\childdocname}
% \macro{\childdocjob}
% The macro |\childdocname| stores the name of the main document
% to be compiled. The macro |\childdocjob| stores the name of
% the document on which the \LaTeX{} compiler was originally invoked.
% The content of |\jobname| cannot be compared
% to filenames specified in the source due to different catcodes.
% The following code rescans |\jobname|, stores the result
% in |\childdocname| and saves a copy in |\childdocjob|:
%    \begin{macrocode}
\edef\childdocname{\scantokens\expandafter{\jobname\noexpand}}
\let\childdocjob\childdocname
%    \end{macrocode}

% \macro{\childdocdisable}
% The macro |\childdocdisable| prevents the main file
% from being processed more than once.
% At this stage, the main document command |\childdocmain|
% is assumed to be called once again where it should do nothing.
% Any subsequent call to it should prevent
% a secondary processing of the main document
% It overwrites the forwarding commands
% |\childdocof| and |\childdocforward|
% with empty macros to prevent further inclusions of the main document:
%    \begin{macrocode}
\newcommand{\childdocdisable}
{
  \renewcommand{\childdocmain}[1]{\renewcommand{\childdocmain}[1]{\endinput}}
  \renewcommand{\childdocof}[1]{}
  \renewcommand{\childdocby}[2][]{}
  \renewcommand{\childdocforward}[2][]{}
  \renewcommand{\childdocdisable}{}
}
%    \end{macrocode}

% \macro{\childdocmain}
% The macro |\childdocmain| is to be called at the top of the main file
% with nothing or the main filename (without extension) as argument.
% First, it breaks loops.
% If the argument is not empty and does not match |\childdocname|
% (which is set by the first inclusion of |childdoc.def|),
% |\ifchilddoc| is set to true, |\includeonly| is applied to the child file
% and |\jobname| is set to the main file
% (for proper handling of |.aux| files):
%    \begin{macrocode}
\newcommand{\childdocmain}[1]
{
  \childdocdisable\childdocmain{}
  \if?#1?\else
    \begingroup
      \def\childdoctmp{#1}
      \ifx\childdoctmp\childdocname
        \def\childdoctmp{}
      \else
        \def\childdoctmp
        {
          \childdoctrue
          \includeonly{\childdocname}
          \def\childdocjob{#1}
          \def\jobname{#1}
        }
      \fi
      \expandafter
    \endgroup
    \childdoctmp
  \fi
}
%    \end{macrocode}

% \macro{\childdocof}
% The command |\childdocof| redirects
% compilation to the main file |#1|.
%    \begin{macrocode}
\newcommand{\childdocof}[1]
{
  \childdocdisable
  \childdoctrue
  \includeonly{\childdocname}
  \def\jobname{#1}
  \def\childdocjob{#1}
  \input{#1}
}
%    \end{macrocode}

% \macro{\childdocby}
% The command |\childdocby| ....
%    \begin{macrocode}
\newcommand{\childdocby}[2][]
{
  \childdocdisable
  \childdoctrue
  \childdocmanualtrue
  \if?#1?\else
    \def\jobname{#2}
  \fi
  \def\childdocjob{#2}
  \input{#2}
  \endinput
}
%    \end{macrocode}

% \macro{\childdocforward}
% The command |\childdocforward| redirects
% compilation to the main file or
% (if the optional argument is given) a child file.
% Parameters are set as if the main file
% or a child file starting with |\childdocof| was compiled.
% Then compilation is handed over to the main file:
%    \begin{macrocode}
\newcommand{\childdocforward}[2][]
{
  \begingroup
    \if?#1?
      \def\childdoctmp
      {
        \def\childdocname{#2}
        \def\childdocjob{#2}
        \def\jobname{#2}
        \input{#2}
        \endinput
      }
    \else
      \def\childdoctmp
      {
        \childdocdisable
        \def\childdocname{#2}
        \childdoctrue
        \includeonly{#2}
        \def\childdocjob{#1}
        \def\jobname{#1}
        \input{#1}
        \endinput
      }
    \fi
    \expandafter
  \endgroup
  \childdoctmp
}
%    \end{macrocode}

% \macro{\childdocforwardprefix}
% The command |\childdocforwardprefix| redirects
% compilation to the main or a child file by means of a pattern.
% The prefix |#1| in the current filename is replaced by |#2|
% and the suffix of the current filename is kept
% (it is assumed that the filename does not contain the substring `|~~~|'
% which is used as a delimiter).
% Compilation is handed over to the new file by |\childdocforward|:
%    \begin{macrocode}
\newcommand{\childdocforwardprefix}[3][]
{
  \begingroup
    \def\childdocextract #2##1~~~{\def\childdoctmp{\childdocforward[#1]{#3##1}}}
    \expandafter\childdocextract\childdocname~~~
    \expandafter
  \endgroup
  \childdoctmp
}
%    \end{macrocode}

% \macro{\childdoc}
% The deprecated macro |\childdoc| is a legacy version of |\childdocmain|:
%    \begin{macrocode}
\newcommand{\childdoc}{\childdocmain}
%    \end{macrocode}

% \macro{\childdocredirect}
% The deprecated macro |\childdocredirect| is a legacy version
% of |\childdocforward| and |\childdocforwardprefix|:
%    \begin{macrocode}
\newcommand{\childdocredirect}[2][]
{
  \begingroup
    \if?#1?
      \def\childdoctmp{\childdocforward{#2}}
    \else
      \def\childdoctmp{\childdocforwardprefix{#1}{#2}}
    \fi
    \expandafter
  \endgroup
  \childdoctmp
}
%    \end{macrocode}

%\iffalse
%</package>
%\fi
%
\endinput

\childdocforward{cdocsamp}
%    \end{macrocode}

%\iffalse
%</sampledraft>
%\fi
%
% %%%%%%%%%%%%%%%%%%%%%%%%%%%%%%%%%%%%%%
% \paragraph{Forwarding for Final Version of the Chapters.}
%
% The following forwarding files |cdocsfn1.tex| and |cdocsfn2.tex|
% (with identical content)
% compile the final versions of the child documents
% |cdocsch1.tex| and |cdocsch2.tex|, respectively:
%\iffalse
%<*samplefinal>
%\fi
%    \begin{macrocode}
\def\version{final}
% \iffalse
%
% childdoc.dtx Copyright (C) 2017-2018 Niklas Beisert
%
% This work may be distributed and/or modified under the
% conditions of the LaTeX Project Public License, either version 1.3
% of this license or (at your option) any later version.
% The latest version of this license is in
%   http://www.latex-project.org/lppl.txt
% and version 1.3 or later is part of all distributions of LaTeX
% version 2005/12/01 or later.
%
% This work has the LPPL maintenance status `maintained'.
%
% The Current Maintainer of this work is Niklas Beisert.
%
% This work consists of the files childdoc.dtx and childdoc.ins
% and the derived files childdoc.def and cdocsamp.tex with
% cdocsch1.tex, cdocsch2.tex, cdocsdrf.tex, cdocsfn1.tex, cdocsfn2.tex.
%
%<package>\ifdefined\childdocmain\endinput\fi
%<package>\ProvidesFile{childdoc.def}[2018/12/30 v2.0 child document driver]
%<samplemain>\ProvidesFile{cdocsamp.tex}[2018/12/30 v2.0 sample for childdoc]
%<*driver>
%\ProvidesFile{childdoc.drv}[2018/12/30 v2.0 childdoc reference manual file]
\PassOptionsToClass{10pt,a4paper}{article}
\documentclass{ltxdoc}

\usepackage[margin=35mm]{geometry}
\usepackage{hyperref}
\usepackage{hyperxmp}
\usepackage[usenames]{color}

\hypersetup{colorlinks=true}
\hypersetup{pdfstartview=FitH}
\hypersetup{pdfpagemode=UseNone}
\hypersetup{pdfsource={}}
\hypersetup{pdflang={en-UK}}
\hypersetup{pdfcopyright={Copyright 2017-2018 Niklas Beisert.
  This work may be distributed and/or modified under the
  conditions of the LaTeX Project Public License, either version 1.3
  of this license or (at your option) any later version.}}
\hypersetup{pdflicenseurl={http://www.latex-project.org/lppl.txt}}
\hypersetup{pdfcontactaddress={ETH Zurich, ITP, HIT K,
  Wolfgang-Pauli-Strasse 27}}
\hypersetup{pdfcontactpostcode={8093}}
\hypersetup{pdfcontactcity={Zurich}}
\hypersetup{pdfcontactcountry={Switzerland}}
\hypersetup{pdfcontactemail={nbeisert@itp.phys.ethz.ch}}
\hypersetup{pdfcontacturl={http://people.phys.ethz.ch/\xmptilde nbeisert/}}

\newcommand{\secref}[1]{\hyperref[#1]{section \ref*{#1}}}

\parskip1ex
\parindent0pt
\let\olditemize\itemize
\def\itemize{\olditemize\parskip0pt}

\begin{document}

\title{The \textsf{childdoc} Package}
\hypersetup{pdftitle={The childdoc Package}}
\author{Niklas Beisert\\[2ex]
  Institut f\"ur Theoretische Physik\\
  Eidgen\"ossische Technische Hochschule Z\"urich\\
  Wolfgang-Pauli-Strasse 27, 8093 Z\"urich, Switzerland\\[1ex]
  \href{mailto:nbeisert@itp.phys.ethz.ch}
  {\texttt{nbeisert@itp.phys.ethz.ch}}}
\hypersetup{pdfauthor={Niklas Beisert}}
\hypersetup{pdfsubject={Manual for the LaTeX2e Package childdoc}}
\date{30 December 2018, \textsf{v2.0}}
\maketitle

\begin{abstract}\noindent
\textsf{childdoc} is a \LaTeXe{} package
that enables the direct compilation
of document sections included by |\include|
to individual files.
\end{abstract}

\begingroup
\parskip0ex
\tableofcontents
\endgroup

%%%%%%%%%%%%%%%%%%%%%%%%%%%%%%%%%%%%%%%%%%%%%%%%%%%%%%%%%%%%%%%%%%%%%%%%%%%%%%%%
%%%%%%%%%%%%%%%%%%%%%%%%%%%%%%%%%%%%%%%%%%%%%%%%%%%%%%%%%%%%%%%%%%%%%%%%%%%%%%%%
\section{Introduction}

\LaTeX{} provides a mechanism to structure a large document (such as a book)
into a main file and several child files (containing the chapters)
using the |\include| command.
This mechanism is beneficial for documents
which span hundreds of pages in order to
make the source file(s) more manageable.
Moreover, compilation can be restricted to
selected child files by means of the |\includeonly| command.
The latter feature can be used to reduce the compilation time while editing
(this was significantly more useful in the earlier days of \LaTeX{})
or to generate a smaller document which is easier to navigate.
Another application of |\includeonly| is to generate
documents consisting of selected parts of the complete document.

However, there are a few drawbacks of the plain |\include| mechanism:
\begin{itemize}
\item
The child files cannot be compiled on their own,
they can only be compiled via the main file.
A naive editing environment
(such as a text editor with an option
to have the current file processed by \LaTeX)
may require one to switch to the main file before compiling;
attempting to compile the child file produces errors.
\item
The main file must be modified (each time)
to adjust the |\includeonly| command
to the present needs. This easily leaves the main file in a messy state.
\item
The generated document will always carry the filename
of the main document. This is inconvenient if
several child files are to be compiled and
to be kept for distribution.
\end{itemize}

The present package provides a simple interface
to make child files individually compilable by \LaTeX{}.
Compiling a child file then has the same effect as compiling
the main file with an |\includeonly| command
to select the appropriate child.
Moreover the generated document will carry the name of the child
rather than the main file.
This resolves all three above issues.

This feature is meant to make the editing of books,
thesis documents and lecture notes somewhat more convenient.
However, the package can also be used efficiently for
composing a series of documents (such as exercise sheets)
which are typically distributed individually.
It then assists the author in generating the individual documents
(potentially in different versions)
as well as a document containing the collected series.
Another application is in developing style files
or other kinds of included material
where compilation of the style file could redirect
to a sample or test file.

%%%%%%%%%%%%%%%%%%%%%%%%%%%%%%%%%%%%%%%%%%%%%%%%%%%%%%%%%%%%%%%%%%%%%%%%%%%%%%%%
%%%%%%%%%%%%%%%%%%%%%%%%%%%%%%%%%%%%%%%%%%%%%%%%%%%%%%%%%%%%%%%%%%%%%%%%%%%%%%%%
\section{Usage}

First of all, the package \textsf{childdoc} is \emph{not} a standard
\LaTeXe{} |.sty| style file! Therefore it needs to be invoked in
a non-standard way.

%%%%%%%%%%%%%%%%%%%%%%%%%%%%%%%%%%%%%%%%%%%%%%%%%%%%%%%%%%%%%%%%%%%%%%%%%%%%%%%%
\subsection{Included Files}
\label{sec:include}

%%%%%%%%%%%%%%%%%%%%%%%%%%%%%%%%%%%%%%%%
\DescribeMacro{\childdocmain}
To use the package, add the commands
\begin{center}
\begin{tabular}{l}
|\input{childdoc.def}|\\
|\childdocmain{}|\\
\end{tabular}
\end{center}
at the very top of the main \LaTeX{} file,
in particular \emph{before} the |\documentclass| statement!
The argument of |\childdocmain| should be left empty
(but it must be present).

%%%%%%%%%%%%%%%%%%%%%%%%%%%%%%%%%%%%%%%%
\DescribeMacro{\childdocof}
Furthermore, add the commands
\begin{center}
\begin{tabular}{l}
|\input{childdoc.def}|\\
|\childdocof{|\textit{main}|}|\\
\end{tabular}
\end{center}
at the top of every child file \textit{child}
which is included by |\include{|\textit{child}|}|
from within the main file
(or at least for those files to be compiled individually).
The argument \textit{main} must be the filename of the main file.

There are a couple of
considerations in setting up the main and child documents:

%%%%%%%%%%%%%%%%%%%%%%%%%%%%%%%%%%%%%%%%
\paragraph{Restrictions.}

Please note the following restrictions:
\begin{itemize}
\item
|\childdocmain| must be called with one argument \textit{main}
to ensure compatibility with earlier version of the package.
It must either be empty (|\childdocmain{}|)
or precisely match the filename of the main file in which it is specified.
See \secref{sec:detection} for further information.
\item
The filename \textit{main} must be specified without the |.tex| extension.
\item
The filename \textit{main} is case sensitive
(even in case-insensitive file systems)
due to internal string comparison.
\item
The argument \textit{main} should be fully expanded, it cannot be a macro.
\item
Subdirectories and special characters should be avoided in filenames.
\item
The command |\childdocmain{|\textit{main}|}| must be followed by a whitespace.
It should not be followed immediately by another command
or by a comment mark `|%|'.
This is because the \TeX{} parser reads the token immediately following
the argument of |\childdocmain| and puts it
at the beginning of every child section;
however, a white\-space is ignored.
\end{itemize}

%%%%%%%%%%%%%%%%%%%%%%%%%%%%%%%%%%%%%%%%
\paragraph{Content of Main File.}

It is advisable to place all content in the child files included by |\include|.
Any output contained in the main file will appear in all child documents
unless suppressed manually;
it cannot be suppressed automatically by the |\includeonly| directive
and thus should normally be avoided.
A method to include some content in the main file
by means of conditional processing is described in \secref{sec:conditional}.

%%%%%%%%%%%%%%%%%%%%%%%%%%%%%%%%%%%%%%%%
\paragraph{Page Numbering.}

When only a part of the document is compiled,
the appropriate numbering of pages
(as well as other status parameters)
is determined from the |.aux| files.
The latter contain information from previous passes.
However this information needs to propagate through
all intermediate child documents.
Therefore the page numbering in child documents may well
be inconsistent until the complete document is compiled at least once.

A useful (if unconventional) way to always ensure a consistent
page numbering is to restart the numbering in each child document
and denote the pages by `\textit{child}|.|\textit{page}'
where \textit{child} represents the chapter/section number of the child file.
This can be achieved by the command
|\numberwithin{page}{|\textit{child}|}|
of the \textsf{amsmath} package
where \textit{child} can be |chapter| or |section|
depending on the chosen structuring.
Alternatively, one can modify the macro |\thepage| appropriately
and reset the counter |page| at the start of each child file.

%%%%%%%%%%%%%%%%%%%%%%%%%%%%%%%%%%%%%%%%%%%%%%%%%%%%%%%%%%%%%%%%%%%%%%%%%%%%%%%%
\subsection{Conditional Processing}
\label{sec:conditional}

The package provides a mechanism to compile different versions
of a document. To customise the versions further some conditional processing
can come in handy to distinguish which version is being compiled.
The package provides two macros to describe the compilation context:

%%%%%%%%%%%%%%%%%%%%%%%%%%%%%%%%%%%%%%%%
\DescribeMacro{\ifchilddoc}
The conditional |\ifchilddoc| distinguishes between the compilation of
child documents and the main document:
%
\begin{center}
|\ifchilddoc |\textit{child-code}| |[|\||else |\textit{main-code}]| \||fi|
\end{center}

%%%%%%%%%%%%%%%%%%%%%%%%%%%%%%%%%%%%%%%%
\DescribeMacro{\childdocname}
\DescribeMacro{\childdocjob}
The macro |\childdocname| contains the filename (without extension)
of the main or child file being processed.
Note that |\childdocjob| will always contain the name of the main file.

%%%%%%%%%%%%%%%%%%%%%%%%%%%%%%%%%%%%%%%%
\paragraph{Title Page.}

Conditional processing can be used to include a title or banner page
in the main document when proper precautions are taken.
Importantly, the code in the main file should ensure that the page counter
(as well as other status parameters which are stored in the |.aux| files)
takes the same value after the conditional processing.
Otherwise the page numbers may take divergent values
depending on which part is compiled.

For example, a title page could be declared by:
%
\begin{center}
\begin{tabular}{l}
|\ifchilddoc\||else|\\
|\addtocounter{page}{-1}|\\
\textit{code for title page}\\
|\newpage|\\
|\||fi|
\end{tabular}
\end{center}
%
A banner page for the child documents can be generated by:
%
\begin{center}
\begin{tabular}{l}
|\ifchilddoc|\\
|\addtocounter{page}{-1}|\\
\textit{code for banner page}\\
|\newpage|\\
|\||fi|
\end{tabular}
\end{center}
%
Here one could write a message such as:
\begin{center}
|This is the part \childdocname{} of \childdocjob{}.|
\end{center}

%%%%%%%%%%%%%%%%%%%%%%%%%%%%%%%%%%%%%%%%%%%%%%%%%%%%%%%%%%%%%%%%%%%%%%%%%%%%%%%%
\subsection{Flags}
\label{sec:flags}

The package makes it easy to generate different versions
of the main or child documents.
To this end compilation flags can be defined
and assigned different default values.
They will be particularly useful in conjunction
with the forwarding mechanism described in \secref{sec:forward}.

For example, it may be useful to have a flag |\version|
which can be set to |draft| or |final|.
The document source will contain some conditional code
depending on the value of |\version|.
Suppose further, the flag should default to |final| for the main file
and to |draft| for child files
which is a natural assignment for editing the document.
This is achieved by placing the following code
in the preamble of the main document
(below the |\childdocmain| directive):
%
\begin{center}
\begin{tabular}{l}
|\ifchilddoc|\\
|\providecommand{\version}{draft}|\\
|\||else|\\
|\providecommand{\version}{final}|\\
|\||fi|
\end{tabular}
\end{center}
%
The definition by |\providecommand| makes sure
that previous definitions are not overwritten.
Further statements |\providecommand{\version}{...}|
can thus be added before the above code to override it.

For the main file, one might add a line
(between |\childdocmain| and the above block)
%
\begin{center}
|%\ifchilddoc\||else\providecommand{\version}{draft}\||fi|
\end{center}
%
which can be uncommented to produce a draft version.
Likewise one can add a line to the very top of a child file
(above the |\childdocof{|\textit{main}|}| directive)
%
\begin{center}
|%\providecommand{\version}{final}|
\end{center}
%
which can be uncommented to produce the final version of this child document.

%%%%%%%%%%%%%%%%%%%%%%%%%%%%%%%%%%%%%%%%%%%%%%%%%%%%%%%%%%%%%%%%%%%%%%%%%%%%%%%%
\subsection{Forwarding}
\label{sec:forward}

Different versions of the main or child documents
using compilation flags as described in \secref{sec:flags}
can be (permanently) stored in different files
for convenient compilation, viewing and distribution.
To this end, the package defines a command
to pass on compilation to a different file:

%%%%%%%%%%%%%%%%%%%%%%%%%%%%%%%%%%%%%%%%
\DescribeMacro{\childdocforward}
The command |\childdocforward| redirects processing to
another source file:
%
\begin{center}
\begin{tabular}{l}
|\input{childdoc.def}|\\
|\childdocforward[|\textit{main}|]{|\textit{dest}|}|\\
\end{tabular}
\end{center}
%
The argument \textit{dest} is the destination file
(without extension).
It should be the main file or one of the child files.
Note that further \textsf{childdoc} directives
such as |\childdocof| and |\childdocforward|
in the indicated file will be processed in this form.
The optional argument \textit{main}
passes on directly to the main file \textit{main}
while pretending to compile the child \textit{dest}.
This form behaves as if \textit{dest}
issues |\childdocof{|\textit{main}|}| right away,
and no further \textsf{childdoc} directives will be processed.

%%%%%%%%%%%%%%%%%%%%%%%%%%%%%%%%%%%%%%%%
\DescribeMacro{\...prefix}
In the alternative form |\childdocforwardprefix|,
%
\begin{center}
\begin{tabular}{l}
|\input{childdoc.def}|\\
|\childdocforwardprefix[|\textit{main}|]{|\textit{prefix}|}{|\textit{dest}|}|
\end{tabular}
\end{center}
%
the destination file is determined by a pattern
depending on the current file:
To make this work, the current file must be called
`{\textit{prefix}\hspace{0.2em}\textit{suffix}}'
with \textit{prefix} matching precisely the argument.
Processing is then passed on to the file
`{\textit{dest}\hspace{0.2em}\textit{suffix}}'.
Surely, the same effect is achieved by
directly specifying the
argument `{\textit{dest}\hspace{0.2em}\textit{suffix}}'
in the first form.
However, that requires to set up a different file
for each child. With the alternative form of the command
all these files can have exactly the same content
which simplifies setting them up and maintaining them.

For example, the following file |draft.tex|
with a compilation flag |\version| as described in \secref{sec:flags}
compiles the main document as a draft:
%
\begin{center}
\begin{tabular}{l}
|\def\version{draft}|\\
|\input{childdoc.def}|\\
|\childdocforward{|\textit{main}|}|
\end{tabular}
\end{center}
%
Likewise, the following files |final|\textit{nn}|.tex|
compile the final version of the child document
|child|\textit{nn}|.tex|:
%
\begin{center}
\begin{tabular}{l}
|\def\version{final}|\\
|\input{childdoc.def}|\\
|\childdocforwardprefix{final}{child}|
\end{tabular}
\end{center}
%

Note that when several versions of a main file and/or of each child file
are to be generated, it may be convenient to set up a |Makefile| or
shell script to automatise the process.

%%%%%%%%%%%%%%%%%%%%%%%%%%%%%%%%%%%%%%%%%%%%%%%%%%%%%%%%%%%%%%%%%%%%%%%%%%%%%%%%
\subsection{Command Line Processing}
\label{sec:commandline}

The effect of redirection files can also be achieved by invoking
the \LaTeX{} compiler with a more elaborate command line.
Most conveniently this should be done as part
of a shell script or a |Makefile|.

When using \textsf{childdoc} in the main file, the following
command lines effectively perform a redirection
(note that depending on the shell being used,
backslashes may have to be doubled: `|\|' $\to$ `|\\|'):
%
\begin{center}
|... -jobname "|\textit{target}|" |\\|"|[\textit{flags}]%
|\input{childdoc.def}\childdocforward[|\textit{main}|]{|\textit{dest}|}"|
\end{center}
%
Here \textit{target} is the name of the output file,
\textit{main} is the name of the main file
and \textit{dest} is the name of the main or child file to be processed
(all filenames without extensions).
The optional argument \textit{main} can be omitted
if \textit{main} matches \textit{dest}.
Optionally, compilation \textit{flags} can be defined via |\def| commands.
This command line makes the \TeX{} engine believe
it is compiling the file \textit{target}
whose content is specified as the latter parameter.
The provided code then forwards the processing to
\textit{main} or \textit{dest} as described in \secref{sec:forward}.

%%%%%%%%%%%%%%%%%%%%%%%%%%%%%%%%%%%%%%%%%%%%%%%%%%%%%%%%%%%%%%%%%%%%%%%%%%%%%%%%
\subsection{Include by Input}
\label{sec:input}

Including child documents by |\include| has some restrictions by design.
Most notably, the content of a child document always occupies
its own set of pages; pages cannot be shared between child documents.
Usually, this behaviour makes perfect sense
because each child document contain an essential part of the document.
However, in some situations it may be desirable to compose
a document from a collection of parts
without having mandatory page breaks between then.
For this case, the package
provides a mechanism to include parts
by |\input| which can also be processed individually.
However, by construction this mechanism
requires manual handling of the content to be output.

%%%%%%%%%%%%%%%%%%%%%%%%%%%%%%%%%%%%%%%%
\DescribeMacro{\ifchilddocmanual}
The main file should be prepared as usual, see \secref{sec:include}.
However, the document body must make a distinction
between processing of an individual part and of the main document, e.g.:
%
\begin{center}
\begin{tabular}{l}
|\ifchilddocmanual|\\
|\input{\childdocname}|\\
|\||else|\\
\textit{document body with }|\input{|\textit{part}|}|\\
|\||fi|
\end{tabular}
\end{center}
%
The conditional |\ifchilddocmanual| is true whenever
a part to be included by |\input| is being compiled,
and the name of the part is stored in |\childdocname|.

%%%%%%%%%%%%%%%%%%%%%%%%%%%%%%%%%%%%%%%%
\DescribeMacro{\childdocby}
Each part to be included by |\input| should start with:
%
\begin{center}
\begin{tabular}{l}
|\input{childdoc.def}|\\
|\childdocby{|\textit{main}|}|\\
\end{tabular}
\end{center}
%
The directive |\childdocby| is similar to |\childdocof|
described in \secref{sec:include},
but the subsequent selection of content must be done manually.
To that end, both |\ifchilddoc| and |\ifchilddocmanual|
will be true upon processing of a part,
and the name of the part is stored in |\childdocname|.
Note that |\jobname| will be set to the filename of the current part
so that each part receives an individual |.aux| file
that does not interfere with the |.aux| file(s) of the main document.
This behaviour can be altered by the alternative form
|\childdocby[*]{|\textit{main}|}| (with a non-empty optional argument)
which uses the |.aux| file of the main document
by setting |\jobname| to \textit{main}.

%%%%%%%%%%%%%%%%%%%%%%%%%%%%%%%%%%%%%%%%%%%%%%%%%%%%%%%%%%%%%%%%%%%%%%%%%%%%%%%%
\subsection{Driver Development}
\label{sec:driver}

The \textsf{childdoc} mechanism can also be use for the development
of definition files such as \LaTeX{} styles or classes.
This case differs from the above setup with multiple parts
included by |\include| in that no |\includeonly| should be invoked.
This can be achieved by starting the include file
(before |\ProvidesPackage|) with:
%
\begin{center}
\begin{tabular}{l}
|\input{childdoc.def}|\\
|\childdocforward{|\textit{main}|}|\\
\end{tabular}
\end{center}
%
or alternatively with:
%
\begin{center}
\begin{tabular}{l}
|\input{childdoc.def}|\\
|\childdocby{|\textit{main}|}|\\
\end{tabular}
\end{center}
%
Both forms have slightly different effects as described above.
The main file is prepared as usual, see \secref{sec:include}.

%%%%%%%%%%%%%%%%%%%%%%%%%%%%%%%%%%%%%%%%%%%%%%%%%%%%%%%%%%%%%%%%%%%%%%%%%%%%%%%%
\subsection{Legacy Detection}
\label{sec:detection}

The directive |\childdocmain| in the main file can detect
whether the complete document or merely a child is to be compiled
even without using the directive |\childdocof|.
This method is deprecated because it is less robust
and there is no compelling reason to use it;
it is merely provided for backward compatibility
and it may be removed in future versions.

If the detection mechanism is to be used,
it is mandatory to correctly specify
the filename of the main file as the argument of |\childdocmain|:
%
\begin{center}
\begin{tabular}{l}
|\input{childdoc.def}|\\
|\childdocmain{|\textit{main}|}|\\
\end{tabular}
\end{center}
%
If |\jobname| does not match the argument \textit{main} of |\childdocmain|,
it is assumed that |\jobname| points to the child file to be compiled.
When using |\childdocmain| with the main file specified as argument,
it suffices to start a child file
with just |\input{|\textit{main}|}|
without loading of the package and using |\childdocof|.
If instead all processing is done
with the appropriate \textsf{childdoc} directives,
the argument of \textit{main} of |\childdocmain| can be empty.

An alternative version of the command line processing described
in \secref{sec:commandline} using the detection mechanism reads:
%
\begin{center}
|... -jobname "|\textit{target}|" "|[\textit{flags}]%
[|\def\jobname{|\textit{dest}|}|]|\input{|\textit{main}|}"|
\end{center}

%%%%%%%%%%%%%%%%%%%%%%%%%%%%%%%%%%%%%%%%%%%%%%%%%%%%%%%%%%%%%%%%%%%%%%%%%%%%%%%%
\subsection{Manual Code}
\label{sec:manual}

In case one cannot be certain whether the definitions file |childdoc.def|
is installed on the target \TeX{} distribution
and one prefers not to ship it,
it is conceivable to paste a few relevant commands into the sources.

To that end, drop all statements |\input{childdoc.def}|
and perform the replacements as outlined below.
Instead of |\childdocmain{|\textit{main}|}| add the following code
to the top of the main file:
%
\begin{center}
\begin{tabular}{l}
|\||ifdefined\childdocname\endinput\||fi\newif\ifchilddoc|\\
|\edef\childdocname{\scantokens\expandafter{\jobname\noexpand}}|\\
|\def\childdocmain{|\textit{main}|}\||ifx\childdocmain\childdocname\||else|\\
|\childdoctrue\includeonly{\childdocname}\let\jobname\childdocmain\||fi|\\
\end{tabular}
\end{center}
%
Instead of |\childdocof{|\textit{main}|}| just include the main file
at the top of each child file:
%
\begin{center}
|\input{|\textit{main}|}|
\end{center}
%
A simple redirection |\childdocforward{|\textit{dest}|}| is achieved by:
%
\begin{center}
|\def\jobname{|\textit{dest}|}\input{\jobname}|
\end{center}
%
The redirection with prefix
|\childdocforwardprefix[|\textit{prefix}|]{|\textit{dest}|}|
is accomplished by:
%
\begin{center}
\begin{tabular}{l}
|{\edef\jobname{\scantokens\expandafter{\jobname\noexpand}}|\\
|\def\redirectjob |\textit{prefix}|#1~~~{\gdef\jobname{|\textit{dest}|#1}}|\\
|\expandafter\redirectjob\jobname~~~}\input{\jobname}|
\end{tabular}
\end{center}

In an alternative approach,
child documents can be compiled by a specific command line
without additional code or specific definitions:
%
\begin{center}
|... -jobname "|\textit{target}|" "|[\textit{flags}]%
|\includeonly{|\textit{dest}|}\input{|\textit{main}|}"|
\end{center}
%

%%%%%%%%%%%%%%%%%%%%%%%%%%%%%%%%%%%%%%%%%%%%%%%%%%%%%%%%%%%%%%%%%%%%%%%%%%%%%%%%
%%%%%%%%%%%%%%%%%%%%%%%%%%%%%%%%%%%%%%%%%%%%%%%%%%%%%%%%%%%%%%%%%%%%%%%%%%%%%%%%
\section{Information}

%%%%%%%%%%%%%%%%%%%%%%%%%%%%%%%%%%%%%%%%%%%%%%%%%%%%%%%%%%%%%%%%%%%%%%%%%%%%%%%%
\subsection{Copyright}

Copyright \copyright{} 2017--2018 Niklas Beisert

This work may be distributed and/or modified under the
conditions of the \LaTeX{} Project Public License, either version 1.3
of this license or (at your option) any later version.
The latest version of this license is in
  \url{http://www.latex-project.org/lppl.txt}
and version 1.3 or later is part of all distributions of \LaTeX{}
version 2005/12/01 or later.

This work has the LPPL maintenance status `maintained'.

The Current Maintainer of this work is Niklas Beisert.

This work consists of the files |README.txt|, |childdoc.ins| and |childdoc.dtx|
as well as the derived files |childdoc.def|, |cdocsamp.tex|
with |cdocsch1.tex|, |cdocsch2.tex|, |cdocspt3.tex|, |cdocspt4.tex|,
|cdocsdrf.tex|, |cdocsfn1.tex|, |cdocsfn2.tex|
as well as |childdoc.pdf|.

%%%%%%%%%%%%%%%%%%%%%%%%%%%%%%%%%%%%%%%%%%%%%%%%%%%%%%%%%%%%%%%%%%%%%%%%%%%%%%%%
\subsection{Files and Installation}

The package consists of the files:
%
\begin{center}
\begin{tabular}{ll}
    |README.txt|   & readme file \\
    |childdoc.ins| & installation file \\
    |childdoc.dtx| & source file \\
    |childdoc.def| & definition file \\
    |cdocsamp.tex| & sample main file \\
    |cdocsch1.tex| & sample include file \\
    |cdocsch2.tex| & sample include file \\
    |cdocspt3.tex| & sample part file \\
    |cdocspt4.tex| & sample part file \\
    |cdocsdrf.tex| & sample redirection file \\
    |cdocsfn1.tex| & sample redirection file \\
    |cdocsfn2.tex| & sample redirection file \\
    |childdoc.pdf| & manual
\end{tabular}
\end{center}
%
The distribution consists of the files
|README.txt|, |childdoc.ins| and |childdoc.dtx|.
%
\begin{itemize}
\item
Run (pdf)\LaTeX{} on |childdoc.dtx|
to compile the manual |childdoc.pdf| (this file).
\item
Run \LaTeX{} on |childdoc.ins| to create the definitions file |childdoc.def|
and the sample |cdocsamp.tex| with include files
|cdocsch1.tex|, |cdocsch2.tex|, |cdocspt3.tex|, |cdocspt4.tex|,
|cdocsdrf.tex|, |cdocsfn1.tex|, |cdocsfn2.tex|.
Then copy the file |childdoc.def| to an appropriate directory of your \LaTeX{}
distribution, e.g.\ \textit{texmf-root}|/tex/latex/childdoc|.
\end{itemize}

%%%%%%%%%%%%%%%%%%%%%%%%%%%%%%%%%%%%%%%%%%%%%%%%%%%%%%%%%%%%%%%%%%%%%%%%%%%%%%%%
\subsection{Related CTAN Packages}

There are several other packages which offer a similar functionality:
%
\begin{itemize}
\item
The packages
\href{http://ctan.org/pkg/docmute}{\textsf{docmute}},
\href{http://ctan.org/pkg/includex}{\textsf{includex}} and
\href{http://ctan.org/pkg/standalone}{\textsf{standalone}}
provide commands to include only the document body of
a child file thus allowing both files to be compiled individually.
\item
The packages \href{http://ctan.org/pkg/subdocs}{\textsf{subdocs}}
and \href{http://ctan.org/pkg/subfiles}{\textsf{subfiles}}
provide structures in which the main and child documents can be
encapsulated and allowing them to be compiled individually.
The inclusion mechanism is different from the conventional |\include|.
\item
The package \href{http://ctan.org/pkg/combine}{\textsf{combine}}
is an elaborate solution to combine several documents into one.
\end{itemize}
%
See also the CTAN topic \href{http://ctan.org/topic/subdocs}{\textsf{subdocs}}
for further related packages.
The present package differs from the above solutions in that
a document structure constructed with the conventional |\include| mechanism
just needs two extra commands at the top of every file
such that all constituent files can be compiled individually.

%%%%%%%%%%%%%%%%%%%%%%%%%%%%%%%%%%%%%%%%%%%%%%%%%%%%%%%%%%%%%%%%%%%%%%%%%%%%%%%%
%\subsection{Feature Suggestions}
%
%The following is a list of features which may be useful for future
%versions of this package:
%%
%\begin{itemize}
%\item
%\ldots
%\end{itemize}

%%%%%%%%%%%%%%%%%%%%%%%%%%%%%%%%%%%%%%%%%%%%%%%%%%%%%%%%%%%%%%%%%%%%%%%%%%%%%%%%
\subsection{Revision History}

%%%%%%%%%%%%%%%%%%%%%%%%%%%%%%%%%%%%%%%%
\paragraph{v2.0:} 2018/12/30

\begin{itemize}
\item
immediate forward processing
\item
added |\childdocby| mechanism
\item
manual restructured
\end{itemize}

%%%%%%%%%%%%%%%%%%%%%%%%%%%%%%%%%%%%%%%%
\paragraph{v1.6:} 2018/01/17

\begin{itemize}
\item
application for development of include files
\item
corrections to manual
\end{itemize}

%%%%%%%%%%%%%%%%%%%%%%%%%%%%%%%%%%%%%%%%
\paragraph{v1.5:} 2017/05/21

\begin{itemize}
\item
more complete structuring introduced
\item
|\childdocof| introduced
\item
|\childdoc| renamed to |\childdocmain|
\item
|\childredirect| renamed to |\childdocforward| and |\childdocforwardprefix|
and functionality expanded
\end{itemize}

%%%%%%%%%%%%%%%%%%%%%%%%%%%%%%%%%%%%%%%%
\paragraph{v1.0:} 2017/04/27

\begin{itemize}
\item
manual and install package
\item
first version published on CTAN
\end{itemize}

%%%%%%%%%%%%%%%%%%%%%%%%%%%%%%%%%%%%%%%%
\paragraph{v0.6:} 2017/04/26

\begin{itemize}
\item
redirection mechanism added
\end{itemize}

%%%%%%%%%%%%%%%%%%%%%%%%%%%%%%%%%%%%%%%%
\paragraph{v0.5:} 2017/04/26

\begin{itemize}
\item
functionality in definition file
\end{itemize}


%%%%%%%%%%%%%%%%%%%%%%%%%%%%%%%%%%%%%%%%%%%%%%%%%%%%%%%%%%%%%%%%%%%%%%%%%%%%%%%%
%%%%%%%%%%%%%%%%%%%%%%%%%%%%%%%%%%%%%%%%%%%%%%%%%%%%%%%%%%%%%%%%%%%%%%%%%%%%%%%%
%%%%%%%%%%%%%%%%%%%%%%%%%%%%%%%%%%%%%%%%%%%%%%%%%%%%%%%%%%%%%%%%%%%%%%%%%%%%%%%%
\appendix

\settowidth\MacroIndent{\rmfamily\scriptsize 000\ }

 \DocInput{childdoc.dtx}

\end{document}
%</driver>
% \fi
%
% %%%%%%%%%%%%%%%%%%%%%%%%%%%%%%%%%%%%%%%%%%%%%%%%%%%%%%%%%%%%%%%%%%%%%%%%%%%%%%
% %%%%%%%%%%%%%%%%%%%%%%%%%%%%%%%%%%%%%%%%%%%%%%%%%%%%%%%%%%%%%%%%%%%%%%%%%%%%%%
% \section{Sample}
%\iffalse
%<*samplemain>
%\fi
%
% The following presents a sample document
% with two chapters, two parts, a title page,
% a compile flag as well as three forwarding files to set the flag.
% It consists of eight |.tex| files:
% \begin{center}
% \begin{tabular}{ll}
% |cdocsamp.tex|&main file\\
% |cdocsch1.tex|&include file for chapter 1\\
% |cdocsch2.tex|&include file for chapter 2\\
% |cdocspt3.tex|&include file for part 3\\
% |cdocspt4.tex|&include file for part 4\\
% |cdocsdrf.tex|&forwarding file for main file in draft mode\\
% |cdocsfi1.tex|&forwarding file for final version of chapter 1\\
% |cdocsfi2.tex|&forwarding file for final version of chapter 2\\
% \end{tabular}
% \end{center}
% Each of the eight files can be compiled directly by the \LaTeX{} compiler.
%
% %%%%%%%%%%%%%%%%%%%%%%%%%%%%%%%%%%%%%%
% \paragraph{Main File.}
%
% The main file is called |cdocsamp.tex|.
%
% Load the \textsf{childdoc} definitions and
% declare the filename for the main document:
%    \begin{macrocode}
\input{childdoc.def}
\childdocmain{}
%    \end{macrocode}

% Optional override for |\version| flag:
%    \begin{macrocode}
%%\ifchilddoc\else\providecommand{\version}{draft}\fi
%    \end{macrocode}

% Define the default values for the |\version| flag
% (|final| for the main file and |draft| for childs):
%    \begin{macrocode}
\ifchilddoc
\providecommand{\version}{draft}
\else
\providecommand{\version}{final}
\fi
%    \end{macrocode}

% Load the standard document class:
%    \begin{macrocode}
\documentclass[12pt]{article}
%    \end{macrocode}

% Start the document body:
%    \begin{macrocode}
\begin{document}
%    \end{macrocode}

% Declare a title page.
% Print title, part of document being processed and version flag:
%    \begin{macrocode}
\addtocounter{page}{-1}
\begin{center}
{\LARGE\bfseries{}childdoc example\par}
\vspace{1cm}
\ifchilddoc
\ifchilddocmanual part\else chapter\fi:
`\childdocname' of `\childdocjob'\par
\else
main document: `\childdocjob'\par
\fi
version: \version\par
\end{center}
\newpage
%    \end{macrocode}

% Manually include selected file,
% otherwise process as usual:
%    \begin{macrocode}
\ifchilddocmanual
\section*{part `\childdocname'}
\input{\childdocname}
\else
%    \end{macrocode}

% Include the two chapters:
%    \begin{macrocode}
\include{cdocsch1}
\include{cdocsch2}
%    \end{macrocode}

% Include the two parts unless only chapters should be displayed:
%    \begin{macrocode}
\ifchilddoc\else
\section{part three}
\input{cdocspt3}
\section{part four}
\input{cdocspt4}
\fi
%    \end{macrocode}

% Process as usual until here:
%    \begin{macrocode}
\fi
%    \end{macrocode}

% End of document body:
%    \begin{macrocode}
\end{document}
%    \end{macrocode}
%\iffalse
%</samplemain>
%\fi
%
% %%%%%%%%%%%%%%%%%%%%%%%%%%%%%%%%%%%%%%
% \paragraph{Chapter Include Files.}
%
% The include files are called |cdocsch1.tex| and |cdocsch2.tex|.
%
%\iffalse
%<*samplechap1|samplechap2>
%\fi

% Optional override for |\version| flag:
%    \begin{macrocode}
%%\providecommand{\version}{final}
%    \end{macrocode}

% Include the main document:
%    \begin{macrocode}
\input{childdoc.def}
\childdocof{cdocsamp}
%    \end{macrocode}

%\iffalse
%</samplechap1|samplechap2>
%\fi
%
%\iffalse
%<*samplechap1>
%\fi
% Some text for chapter 1:
%    \begin{macrocode}
\section{one}
some text in chapter one
%    \end{macrocode}

%\iffalse
%</samplechap1>
%\fi
% Some text for chapter 2:
%\iffalse
%<*samplechap2>
%\fi
%    \begin{macrocode}
\section{two}
more text in chapter two
%    \end{macrocode}

%\iffalse
%</samplechap2>
%\fi
%
% %%%%%%%%%%%%%%%%%%%%%%%%%%%%%%%%%%%%%%
% \paragraph{Part Include Files.}
%
% The include files are called |cdocspt3.tex| and |cdocspt4.tex|.
%
%\iffalse
%<*samplepart3|samplepart4>
%\fi

% Optional override for |\version| flag:
%    \begin{macrocode}
%%\providecommand{\version}{final}
%    \end{macrocode}

% Include the main document:
%    \begin{macrocode}
\input{childdoc.def}
\childdocby{cdocsamp}
%    \end{macrocode}

%\iffalse
%</samplepart3|samplepart4>
%\fi
%
%\iffalse
%<*samplepart3>
%\fi
% Some text for part 3:
%    \begin{macrocode}
some text in part three
%    \end{macrocode}

%\iffalse
%</samplepart3>
%\fi
% Some text for part 4:
%\iffalse
%<*samplepart4>
%\fi
%    \begin{macrocode}
more text in part four
%    \end{macrocode}

%\iffalse
%</samplepart4>
%\fi
%
% %%%%%%%%%%%%%%%%%%%%%%%%%%%%%%%%%%%%%%
% \paragraph{Forwarding for a Complete Draft.}
%
% The following forwarding file |cdocsdrf.tex|
% compiles the main document in draft mode:
%\iffalse
%<*sampledraft>
%\fi
%    \begin{macrocode}
\def\version{draft}
\input{childdoc.def}
\childdocforward{cdocsamp}
%    \end{macrocode}

%\iffalse
%</sampledraft>
%\fi
%
% %%%%%%%%%%%%%%%%%%%%%%%%%%%%%%%%%%%%%%
% \paragraph{Forwarding for Final Version of the Chapters.}
%
% The following forwarding files |cdocsfn1.tex| and |cdocsfn2.tex|
% (with identical content)
% compile the final versions of the child documents
% |cdocsch1.tex| and |cdocsch2.tex|, respectively:
%\iffalse
%<*samplefinal>
%\fi
%    \begin{macrocode}
\def\version{final}
\input{childdoc.def}
\childdocforwardprefix[cdocsamp]{cdocsfn}{cdocsch}
%    \end{macrocode}

%\iffalse
%</samplefinal>
%\fi
%
% %%%%%%%%%%%%%%%%%%%%%%%%%%%%%%%%%%%%%%
% \paragraph{Command Line Processing.}
%
% The following three command lines generate the output files
% |cdocscld|, |cdocscl1| and |cdocscl2|
% which should be identical to
% |cdocsdrf|, |cdocsch1| and |cdocsfn2|, respectively:
% \begin{center}
% \begin{tabular}{l}
% |latex -jobname cdocscld \|\\
% |  "\def\version{draft}\input{childdoc.def}\childdocforward{cdocsamp}"|\\
% |latex -jobname cdocscl1 \|\\
% |  "\input{childdoc.def}\childdocforward[cdocsamp]{cdocsch1}"|\\
% |latex -jobname cdocscl2 \|\\
% |  "\def\version{final}\input{childdoc.def}\childdocforward{cdocsch2}"|
% \end{tabular}
% \end{center}
% Note that the trailing backslash on each first line
% merely continues the input to the second line
% (for convenient cut ant paste).
% Furthermore, the command |latex| can be replaced by any
% of its alternative versions such as |pdflatex|.
%
% %%%%%%%%%%%%%%%%%%%%%%%%%%%%%%%%%%%%%%%%%%%%%%%%%%%%%%%%%%%%%%%%%%%%%%%%%%%%%%
% %%%%%%%%%%%%%%%%%%%%%%%%%%%%%%%%%%%%%%%%%%%%%%%%%%%%%%%%%%%%%%%%%%%%%%%%%%%%%%
% \section{Implementation}
%\iffalse
%<*package>
%\fi
%
% This section describes the definitions file |childdoc.def|.

% The definitions cannot be loaded using |\usepackage| or |\RequirePackage|
% which has a mechanism to prevent loading a style file more than once.
% When loading the definitions by means of |\input|
% multiple instances have to be prevented manually:
%\iffalse
%This code needs to be before the `\ProvidesFile' directive
%which is defined at the beginning of this file.
%Therefore it is also placed there and commented out here.
%</package>
%<*discard>
%\fi
%    \begin{macrocode}
\ifdefined\childdocmain\endinput\fi
%    \end{macrocode}
%\iffalse
%</discard>
%<*package>
%\fi
%
% \macro{\ifchilddoc}
% \macro{\ifchilddocmanual}
% The conditional |\ifchilddoc| tells whether a
% child (true) or main (false) document is being compiled.
% The conditional |\ifchilddocmanual| tells whether
% the |\includeonly| mechanism is used (false) or
% the selection of child files must be performed manually (true).
% The definitions initialise to false:
%    \begin{macrocode}
\newif\ifchilddoc
\newif\ifchilddocmanual
%    \end{macrocode}

% \macro{\childdocname}
% \macro{\childdocjob}
% The macro |\childdocname| stores the name of the main document
% to be compiled. The macro |\childdocjob| stores the name of
% the document on which the \LaTeX{} compiler was originally invoked.
% The content of |\jobname| cannot be compared
% to filenames specified in the source due to different catcodes.
% The following code rescans |\jobname|, stores the result
% in |\childdocname| and saves a copy in |\childdocjob|:
%    \begin{macrocode}
\edef\childdocname{\scantokens\expandafter{\jobname\noexpand}}
\let\childdocjob\childdocname
%    \end{macrocode}

% \macro{\childdocdisable}
% The macro |\childdocdisable| prevents the main file
% from being processed more than once.
% At this stage, the main document command |\childdocmain|
% is assumed to be called once again where it should do nothing.
% Any subsequent call to it should prevent
% a secondary processing of the main document
% It overwrites the forwarding commands
% |\childdocof| and |\childdocforward|
% with empty macros to prevent further inclusions of the main document:
%    \begin{macrocode}
\newcommand{\childdocdisable}
{
  \renewcommand{\childdocmain}[1]{\renewcommand{\childdocmain}[1]{\endinput}}
  \renewcommand{\childdocof}[1]{}
  \renewcommand{\childdocby}[2][]{}
  \renewcommand{\childdocforward}[2][]{}
  \renewcommand{\childdocdisable}{}
}
%    \end{macrocode}

% \macro{\childdocmain}
% The macro |\childdocmain| is to be called at the top of the main file
% with nothing or the main filename (without extension) as argument.
% First, it breaks loops.
% If the argument is not empty and does not match |\childdocname|
% (which is set by the first inclusion of |childdoc.def|),
% |\ifchilddoc| is set to true, |\includeonly| is applied to the child file
% and |\jobname| is set to the main file
% (for proper handling of |.aux| files):
%    \begin{macrocode}
\newcommand{\childdocmain}[1]
{
  \childdocdisable\childdocmain{}
  \if?#1?\else
    \begingroup
      \def\childdoctmp{#1}
      \ifx\childdoctmp\childdocname
        \def\childdoctmp{}
      \else
        \def\childdoctmp
        {
          \childdoctrue
          \includeonly{\childdocname}
          \def\childdocjob{#1}
          \def\jobname{#1}
        }
      \fi
      \expandafter
    \endgroup
    \childdoctmp
  \fi
}
%    \end{macrocode}

% \macro{\childdocof}
% The command |\childdocof| redirects
% compilation to the main file |#1|.
%    \begin{macrocode}
\newcommand{\childdocof}[1]
{
  \childdocdisable
  \childdoctrue
  \includeonly{\childdocname}
  \def\jobname{#1}
  \def\childdocjob{#1}
  \input{#1}
}
%    \end{macrocode}

% \macro{\childdocby}
% The command |\childdocby| ....
%    \begin{macrocode}
\newcommand{\childdocby}[2][]
{
  \childdocdisable
  \childdoctrue
  \childdocmanualtrue
  \if?#1?\else
    \def\jobname{#2}
  \fi
  \def\childdocjob{#2}
  \input{#2}
  \endinput
}
%    \end{macrocode}

% \macro{\childdocforward}
% The command |\childdocforward| redirects
% compilation to the main file or
% (if the optional argument is given) a child file.
% Parameters are set as if the main file
% or a child file starting with |\childdocof| was compiled.
% Then compilation is handed over to the main file:
%    \begin{macrocode}
\newcommand{\childdocforward}[2][]
{
  \begingroup
    \if?#1?
      \def\childdoctmp
      {
        \def\childdocname{#2}
        \def\childdocjob{#2}
        \def\jobname{#2}
        \input{#2}
        \endinput
      }
    \else
      \def\childdoctmp
      {
        \childdocdisable
        \def\childdocname{#2}
        \childdoctrue
        \includeonly{#2}
        \def\childdocjob{#1}
        \def\jobname{#1}
        \input{#1}
        \endinput
      }
    \fi
    \expandafter
  \endgroup
  \childdoctmp
}
%    \end{macrocode}

% \macro{\childdocforwardprefix}
% The command |\childdocforwardprefix| redirects
% compilation to the main or a child file by means of a pattern.
% The prefix |#1| in the current filename is replaced by |#2|
% and the suffix of the current filename is kept
% (it is assumed that the filename does not contain the substring `|~~~|'
% which is used as a delimiter).
% Compilation is handed over to the new file by |\childdocforward|:
%    \begin{macrocode}
\newcommand{\childdocforwardprefix}[3][]
{
  \begingroup
    \def\childdocextract #2##1~~~{\def\childdoctmp{\childdocforward[#1]{#3##1}}}
    \expandafter\childdocextract\childdocname~~~
    \expandafter
  \endgroup
  \childdoctmp
}
%    \end{macrocode}

% \macro{\childdoc}
% The deprecated macro |\childdoc| is a legacy version of |\childdocmain|:
%    \begin{macrocode}
\newcommand{\childdoc}{\childdocmain}
%    \end{macrocode}

% \macro{\childdocredirect}
% The deprecated macro |\childdocredirect| is a legacy version
% of |\childdocforward| and |\childdocforwardprefix|:
%    \begin{macrocode}
\newcommand{\childdocredirect}[2][]
{
  \begingroup
    \if?#1?
      \def\childdoctmp{\childdocforward{#2}}
    \else
      \def\childdoctmp{\childdocforwardprefix{#1}{#2}}
    \fi
    \expandafter
  \endgroup
  \childdoctmp
}
%    \end{macrocode}

%\iffalse
%</package>
%\fi
%
\endinput

\childdocforwardprefix[cdocsamp]{cdocsfn}{cdocsch}
%    \end{macrocode}

%\iffalse
%</samplefinal>
%\fi
%
% %%%%%%%%%%%%%%%%%%%%%%%%%%%%%%%%%%%%%%
% \paragraph{Command Line Processing.}
%
% The following three command lines generate the output files
% |cdocscld|, |cdocscl1| and |cdocscl2|
% which should be identical to
% |cdocsdrf|, |cdocsch1| and |cdocsfn2|, respectively:
% \begin{center}
% \begin{tabular}{l}
% |latex -jobname cdocscld \|\\
% |  "\def\version{draft}% \iffalse
%
% childdoc.dtx Copyright (C) 2017-2018 Niklas Beisert
%
% This work may be distributed and/or modified under the
% conditions of the LaTeX Project Public License, either version 1.3
% of this license or (at your option) any later version.
% The latest version of this license is in
%   http://www.latex-project.org/lppl.txt
% and version 1.3 or later is part of all distributions of LaTeX
% version 2005/12/01 or later.
%
% This work has the LPPL maintenance status `maintained'.
%
% The Current Maintainer of this work is Niklas Beisert.
%
% This work consists of the files childdoc.dtx and childdoc.ins
% and the derived files childdoc.def and cdocsamp.tex with
% cdocsch1.tex, cdocsch2.tex, cdocsdrf.tex, cdocsfn1.tex, cdocsfn2.tex.
%
%<package>\ifdefined\childdocmain\endinput\fi
%<package>\ProvidesFile{childdoc.def}[2018/12/30 v2.0 child document driver]
%<samplemain>\ProvidesFile{cdocsamp.tex}[2018/12/30 v2.0 sample for childdoc]
%<*driver>
%\ProvidesFile{childdoc.drv}[2018/12/30 v2.0 childdoc reference manual file]
\PassOptionsToClass{10pt,a4paper}{article}
\documentclass{ltxdoc}

\usepackage[margin=35mm]{geometry}
\usepackage{hyperref}
\usepackage{hyperxmp}
\usepackage[usenames]{color}

\hypersetup{colorlinks=true}
\hypersetup{pdfstartview=FitH}
\hypersetup{pdfpagemode=UseNone}
\hypersetup{pdfsource={}}
\hypersetup{pdflang={en-UK}}
\hypersetup{pdfcopyright={Copyright 2017-2018 Niklas Beisert.
  This work may be distributed and/or modified under the
  conditions of the LaTeX Project Public License, either version 1.3
  of this license or (at your option) any later version.}}
\hypersetup{pdflicenseurl={http://www.latex-project.org/lppl.txt}}
\hypersetup{pdfcontactaddress={ETH Zurich, ITP, HIT K,
  Wolfgang-Pauli-Strasse 27}}
\hypersetup{pdfcontactpostcode={8093}}
\hypersetup{pdfcontactcity={Zurich}}
\hypersetup{pdfcontactcountry={Switzerland}}
\hypersetup{pdfcontactemail={nbeisert@itp.phys.ethz.ch}}
\hypersetup{pdfcontacturl={http://people.phys.ethz.ch/\xmptilde nbeisert/}}

\newcommand{\secref}[1]{\hyperref[#1]{section \ref*{#1}}}

\parskip1ex
\parindent0pt
\let\olditemize\itemize
\def\itemize{\olditemize\parskip0pt}

\begin{document}

\title{The \textsf{childdoc} Package}
\hypersetup{pdftitle={The childdoc Package}}
\author{Niklas Beisert\\[2ex]
  Institut f\"ur Theoretische Physik\\
  Eidgen\"ossische Technische Hochschule Z\"urich\\
  Wolfgang-Pauli-Strasse 27, 8093 Z\"urich, Switzerland\\[1ex]
  \href{mailto:nbeisert@itp.phys.ethz.ch}
  {\texttt{nbeisert@itp.phys.ethz.ch}}}
\hypersetup{pdfauthor={Niklas Beisert}}
\hypersetup{pdfsubject={Manual for the LaTeX2e Package childdoc}}
\date{30 December 2018, \textsf{v2.0}}
\maketitle

\begin{abstract}\noindent
\textsf{childdoc} is a \LaTeXe{} package
that enables the direct compilation
of document sections included by |\include|
to individual files.
\end{abstract}

\begingroup
\parskip0ex
\tableofcontents
\endgroup

%%%%%%%%%%%%%%%%%%%%%%%%%%%%%%%%%%%%%%%%%%%%%%%%%%%%%%%%%%%%%%%%%%%%%%%%%%%%%%%%
%%%%%%%%%%%%%%%%%%%%%%%%%%%%%%%%%%%%%%%%%%%%%%%%%%%%%%%%%%%%%%%%%%%%%%%%%%%%%%%%
\section{Introduction}

\LaTeX{} provides a mechanism to structure a large document (such as a book)
into a main file and several child files (containing the chapters)
using the |\include| command.
This mechanism is beneficial for documents
which span hundreds of pages in order to
make the source file(s) more manageable.
Moreover, compilation can be restricted to
selected child files by means of the |\includeonly| command.
The latter feature can be used to reduce the compilation time while editing
(this was significantly more useful in the earlier days of \LaTeX{})
or to generate a smaller document which is easier to navigate.
Another application of |\includeonly| is to generate
documents consisting of selected parts of the complete document.

However, there are a few drawbacks of the plain |\include| mechanism:
\begin{itemize}
\item
The child files cannot be compiled on their own,
they can only be compiled via the main file.
A naive editing environment
(such as a text editor with an option
to have the current file processed by \LaTeX)
may require one to switch to the main file before compiling;
attempting to compile the child file produces errors.
\item
The main file must be modified (each time)
to adjust the |\includeonly| command
to the present needs. This easily leaves the main file in a messy state.
\item
The generated document will always carry the filename
of the main document. This is inconvenient if
several child files are to be compiled and
to be kept for distribution.
\end{itemize}

The present package provides a simple interface
to make child files individually compilable by \LaTeX{}.
Compiling a child file then has the same effect as compiling
the main file with an |\includeonly| command
to select the appropriate child.
Moreover the generated document will carry the name of the child
rather than the main file.
This resolves all three above issues.

This feature is meant to make the editing of books,
thesis documents and lecture notes somewhat more convenient.
However, the package can also be used efficiently for
composing a series of documents (such as exercise sheets)
which are typically distributed individually.
It then assists the author in generating the individual documents
(potentially in different versions)
as well as a document containing the collected series.
Another application is in developing style files
or other kinds of included material
where compilation of the style file could redirect
to a sample or test file.

%%%%%%%%%%%%%%%%%%%%%%%%%%%%%%%%%%%%%%%%%%%%%%%%%%%%%%%%%%%%%%%%%%%%%%%%%%%%%%%%
%%%%%%%%%%%%%%%%%%%%%%%%%%%%%%%%%%%%%%%%%%%%%%%%%%%%%%%%%%%%%%%%%%%%%%%%%%%%%%%%
\section{Usage}

First of all, the package \textsf{childdoc} is \emph{not} a standard
\LaTeXe{} |.sty| style file! Therefore it needs to be invoked in
a non-standard way.

%%%%%%%%%%%%%%%%%%%%%%%%%%%%%%%%%%%%%%%%%%%%%%%%%%%%%%%%%%%%%%%%%%%%%%%%%%%%%%%%
\subsection{Included Files}
\label{sec:include}

%%%%%%%%%%%%%%%%%%%%%%%%%%%%%%%%%%%%%%%%
\DescribeMacro{\childdocmain}
To use the package, add the commands
\begin{center}
\begin{tabular}{l}
|\input{childdoc.def}|\\
|\childdocmain{}|\\
\end{tabular}
\end{center}
at the very top of the main \LaTeX{} file,
in particular \emph{before} the |\documentclass| statement!
The argument of |\childdocmain| should be left empty
(but it must be present).

%%%%%%%%%%%%%%%%%%%%%%%%%%%%%%%%%%%%%%%%
\DescribeMacro{\childdocof}
Furthermore, add the commands
\begin{center}
\begin{tabular}{l}
|\input{childdoc.def}|\\
|\childdocof{|\textit{main}|}|\\
\end{tabular}
\end{center}
at the top of every child file \textit{child}
which is included by |\include{|\textit{child}|}|
from within the main file
(or at least for those files to be compiled individually).
The argument \textit{main} must be the filename of the main file.

There are a couple of
considerations in setting up the main and child documents:

%%%%%%%%%%%%%%%%%%%%%%%%%%%%%%%%%%%%%%%%
\paragraph{Restrictions.}

Please note the following restrictions:
\begin{itemize}
\item
|\childdocmain| must be called with one argument \textit{main}
to ensure compatibility with earlier version of the package.
It must either be empty (|\childdocmain{}|)
or precisely match the filename of the main file in which it is specified.
See \secref{sec:detection} for further information.
\item
The filename \textit{main} must be specified without the |.tex| extension.
\item
The filename \textit{main} is case sensitive
(even in case-insensitive file systems)
due to internal string comparison.
\item
The argument \textit{main} should be fully expanded, it cannot be a macro.
\item
Subdirectories and special characters should be avoided in filenames.
\item
The command |\childdocmain{|\textit{main}|}| must be followed by a whitespace.
It should not be followed immediately by another command
or by a comment mark `|%|'.
This is because the \TeX{} parser reads the token immediately following
the argument of |\childdocmain| and puts it
at the beginning of every child section;
however, a white\-space is ignored.
\end{itemize}

%%%%%%%%%%%%%%%%%%%%%%%%%%%%%%%%%%%%%%%%
\paragraph{Content of Main File.}

It is advisable to place all content in the child files included by |\include|.
Any output contained in the main file will appear in all child documents
unless suppressed manually;
it cannot be suppressed automatically by the |\includeonly| directive
and thus should normally be avoided.
A method to include some content in the main file
by means of conditional processing is described in \secref{sec:conditional}.

%%%%%%%%%%%%%%%%%%%%%%%%%%%%%%%%%%%%%%%%
\paragraph{Page Numbering.}

When only a part of the document is compiled,
the appropriate numbering of pages
(as well as other status parameters)
is determined from the |.aux| files.
The latter contain information from previous passes.
However this information needs to propagate through
all intermediate child documents.
Therefore the page numbering in child documents may well
be inconsistent until the complete document is compiled at least once.

A useful (if unconventional) way to always ensure a consistent
page numbering is to restart the numbering in each child document
and denote the pages by `\textit{child}|.|\textit{page}'
where \textit{child} represents the chapter/section number of the child file.
This can be achieved by the command
|\numberwithin{page}{|\textit{child}|}|
of the \textsf{amsmath} package
where \textit{child} can be |chapter| or |section|
depending on the chosen structuring.
Alternatively, one can modify the macro |\thepage| appropriately
and reset the counter |page| at the start of each child file.

%%%%%%%%%%%%%%%%%%%%%%%%%%%%%%%%%%%%%%%%%%%%%%%%%%%%%%%%%%%%%%%%%%%%%%%%%%%%%%%%
\subsection{Conditional Processing}
\label{sec:conditional}

The package provides a mechanism to compile different versions
of a document. To customise the versions further some conditional processing
can come in handy to distinguish which version is being compiled.
The package provides two macros to describe the compilation context:

%%%%%%%%%%%%%%%%%%%%%%%%%%%%%%%%%%%%%%%%
\DescribeMacro{\ifchilddoc}
The conditional |\ifchilddoc| distinguishes between the compilation of
child documents and the main document:
%
\begin{center}
|\ifchilddoc |\textit{child-code}| |[|\||else |\textit{main-code}]| \||fi|
\end{center}

%%%%%%%%%%%%%%%%%%%%%%%%%%%%%%%%%%%%%%%%
\DescribeMacro{\childdocname}
\DescribeMacro{\childdocjob}
The macro |\childdocname| contains the filename (without extension)
of the main or child file being processed.
Note that |\childdocjob| will always contain the name of the main file.

%%%%%%%%%%%%%%%%%%%%%%%%%%%%%%%%%%%%%%%%
\paragraph{Title Page.}

Conditional processing can be used to include a title or banner page
in the main document when proper precautions are taken.
Importantly, the code in the main file should ensure that the page counter
(as well as other status parameters which are stored in the |.aux| files)
takes the same value after the conditional processing.
Otherwise the page numbers may take divergent values
depending on which part is compiled.

For example, a title page could be declared by:
%
\begin{center}
\begin{tabular}{l}
|\ifchilddoc\||else|\\
|\addtocounter{page}{-1}|\\
\textit{code for title page}\\
|\newpage|\\
|\||fi|
\end{tabular}
\end{center}
%
A banner page for the child documents can be generated by:
%
\begin{center}
\begin{tabular}{l}
|\ifchilddoc|\\
|\addtocounter{page}{-1}|\\
\textit{code for banner page}\\
|\newpage|\\
|\||fi|
\end{tabular}
\end{center}
%
Here one could write a message such as:
\begin{center}
|This is the part \childdocname{} of \childdocjob{}.|
\end{center}

%%%%%%%%%%%%%%%%%%%%%%%%%%%%%%%%%%%%%%%%%%%%%%%%%%%%%%%%%%%%%%%%%%%%%%%%%%%%%%%%
\subsection{Flags}
\label{sec:flags}

The package makes it easy to generate different versions
of the main or child documents.
To this end compilation flags can be defined
and assigned different default values.
They will be particularly useful in conjunction
with the forwarding mechanism described in \secref{sec:forward}.

For example, it may be useful to have a flag |\version|
which can be set to |draft| or |final|.
The document source will contain some conditional code
depending on the value of |\version|.
Suppose further, the flag should default to |final| for the main file
and to |draft| for child files
which is a natural assignment for editing the document.
This is achieved by placing the following code
in the preamble of the main document
(below the |\childdocmain| directive):
%
\begin{center}
\begin{tabular}{l}
|\ifchilddoc|\\
|\providecommand{\version}{draft}|\\
|\||else|\\
|\providecommand{\version}{final}|\\
|\||fi|
\end{tabular}
\end{center}
%
The definition by |\providecommand| makes sure
that previous definitions are not overwritten.
Further statements |\providecommand{\version}{...}|
can thus be added before the above code to override it.

For the main file, one might add a line
(between |\childdocmain| and the above block)
%
\begin{center}
|%\ifchilddoc\||else\providecommand{\version}{draft}\||fi|
\end{center}
%
which can be uncommented to produce a draft version.
Likewise one can add a line to the very top of a child file
(above the |\childdocof{|\textit{main}|}| directive)
%
\begin{center}
|%\providecommand{\version}{final}|
\end{center}
%
which can be uncommented to produce the final version of this child document.

%%%%%%%%%%%%%%%%%%%%%%%%%%%%%%%%%%%%%%%%%%%%%%%%%%%%%%%%%%%%%%%%%%%%%%%%%%%%%%%%
\subsection{Forwarding}
\label{sec:forward}

Different versions of the main or child documents
using compilation flags as described in \secref{sec:flags}
can be (permanently) stored in different files
for convenient compilation, viewing and distribution.
To this end, the package defines a command
to pass on compilation to a different file:

%%%%%%%%%%%%%%%%%%%%%%%%%%%%%%%%%%%%%%%%
\DescribeMacro{\childdocforward}
The command |\childdocforward| redirects processing to
another source file:
%
\begin{center}
\begin{tabular}{l}
|\input{childdoc.def}|\\
|\childdocforward[|\textit{main}|]{|\textit{dest}|}|\\
\end{tabular}
\end{center}
%
The argument \textit{dest} is the destination file
(without extension).
It should be the main file or one of the child files.
Note that further \textsf{childdoc} directives
such as |\childdocof| and |\childdocforward|
in the indicated file will be processed in this form.
The optional argument \textit{main}
passes on directly to the main file \textit{main}
while pretending to compile the child \textit{dest}.
This form behaves as if \textit{dest}
issues |\childdocof{|\textit{main}|}| right away,
and no further \textsf{childdoc} directives will be processed.

%%%%%%%%%%%%%%%%%%%%%%%%%%%%%%%%%%%%%%%%
\DescribeMacro{\...prefix}
In the alternative form |\childdocforwardprefix|,
%
\begin{center}
\begin{tabular}{l}
|\input{childdoc.def}|\\
|\childdocforwardprefix[|\textit{main}|]{|\textit{prefix}|}{|\textit{dest}|}|
\end{tabular}
\end{center}
%
the destination file is determined by a pattern
depending on the current file:
To make this work, the current file must be called
`{\textit{prefix}\hspace{0.2em}\textit{suffix}}'
with \textit{prefix} matching precisely the argument.
Processing is then passed on to the file
`{\textit{dest}\hspace{0.2em}\textit{suffix}}'.
Surely, the same effect is achieved by
directly specifying the
argument `{\textit{dest}\hspace{0.2em}\textit{suffix}}'
in the first form.
However, that requires to set up a different file
for each child. With the alternative form of the command
all these files can have exactly the same content
which simplifies setting them up and maintaining them.

For example, the following file |draft.tex|
with a compilation flag |\version| as described in \secref{sec:flags}
compiles the main document as a draft:
%
\begin{center}
\begin{tabular}{l}
|\def\version{draft}|\\
|\input{childdoc.def}|\\
|\childdocforward{|\textit{main}|}|
\end{tabular}
\end{center}
%
Likewise, the following files |final|\textit{nn}|.tex|
compile the final version of the child document
|child|\textit{nn}|.tex|:
%
\begin{center}
\begin{tabular}{l}
|\def\version{final}|\\
|\input{childdoc.def}|\\
|\childdocforwardprefix{final}{child}|
\end{tabular}
\end{center}
%

Note that when several versions of a main file and/or of each child file
are to be generated, it may be convenient to set up a |Makefile| or
shell script to automatise the process.

%%%%%%%%%%%%%%%%%%%%%%%%%%%%%%%%%%%%%%%%%%%%%%%%%%%%%%%%%%%%%%%%%%%%%%%%%%%%%%%%
\subsection{Command Line Processing}
\label{sec:commandline}

The effect of redirection files can also be achieved by invoking
the \LaTeX{} compiler with a more elaborate command line.
Most conveniently this should be done as part
of a shell script or a |Makefile|.

When using \textsf{childdoc} in the main file, the following
command lines effectively perform a redirection
(note that depending on the shell being used,
backslashes may have to be doubled: `|\|' $\to$ `|\\|'):
%
\begin{center}
|... -jobname "|\textit{target}|" |\\|"|[\textit{flags}]%
|\input{childdoc.def}\childdocforward[|\textit{main}|]{|\textit{dest}|}"|
\end{center}
%
Here \textit{target} is the name of the output file,
\textit{main} is the name of the main file
and \textit{dest} is the name of the main or child file to be processed
(all filenames without extensions).
The optional argument \textit{main} can be omitted
if \textit{main} matches \textit{dest}.
Optionally, compilation \textit{flags} can be defined via |\def| commands.
This command line makes the \TeX{} engine believe
it is compiling the file \textit{target}
whose content is specified as the latter parameter.
The provided code then forwards the processing to
\textit{main} or \textit{dest} as described in \secref{sec:forward}.

%%%%%%%%%%%%%%%%%%%%%%%%%%%%%%%%%%%%%%%%%%%%%%%%%%%%%%%%%%%%%%%%%%%%%%%%%%%%%%%%
\subsection{Include by Input}
\label{sec:input}

Including child documents by |\include| has some restrictions by design.
Most notably, the content of a child document always occupies
its own set of pages; pages cannot be shared between child documents.
Usually, this behaviour makes perfect sense
because each child document contain an essential part of the document.
However, in some situations it may be desirable to compose
a document from a collection of parts
without having mandatory page breaks between then.
For this case, the package
provides a mechanism to include parts
by |\input| which can also be processed individually.
However, by construction this mechanism
requires manual handling of the content to be output.

%%%%%%%%%%%%%%%%%%%%%%%%%%%%%%%%%%%%%%%%
\DescribeMacro{\ifchilddocmanual}
The main file should be prepared as usual, see \secref{sec:include}.
However, the document body must make a distinction
between processing of an individual part and of the main document, e.g.:
%
\begin{center}
\begin{tabular}{l}
|\ifchilddocmanual|\\
|\input{\childdocname}|\\
|\||else|\\
\textit{document body with }|\input{|\textit{part}|}|\\
|\||fi|
\end{tabular}
\end{center}
%
The conditional |\ifchilddocmanual| is true whenever
a part to be included by |\input| is being compiled,
and the name of the part is stored in |\childdocname|.

%%%%%%%%%%%%%%%%%%%%%%%%%%%%%%%%%%%%%%%%
\DescribeMacro{\childdocby}
Each part to be included by |\input| should start with:
%
\begin{center}
\begin{tabular}{l}
|\input{childdoc.def}|\\
|\childdocby{|\textit{main}|}|\\
\end{tabular}
\end{center}
%
The directive |\childdocby| is similar to |\childdocof|
described in \secref{sec:include},
but the subsequent selection of content must be done manually.
To that end, both |\ifchilddoc| and |\ifchilddocmanual|
will be true upon processing of a part,
and the name of the part is stored in |\childdocname|.
Note that |\jobname| will be set to the filename of the current part
so that each part receives an individual |.aux| file
that does not interfere with the |.aux| file(s) of the main document.
This behaviour can be altered by the alternative form
|\childdocby[*]{|\textit{main}|}| (with a non-empty optional argument)
which uses the |.aux| file of the main document
by setting |\jobname| to \textit{main}.

%%%%%%%%%%%%%%%%%%%%%%%%%%%%%%%%%%%%%%%%%%%%%%%%%%%%%%%%%%%%%%%%%%%%%%%%%%%%%%%%
\subsection{Driver Development}
\label{sec:driver}

The \textsf{childdoc} mechanism can also be use for the development
of definition files such as \LaTeX{} styles or classes.
This case differs from the above setup with multiple parts
included by |\include| in that no |\includeonly| should be invoked.
This can be achieved by starting the include file
(before |\ProvidesPackage|) with:
%
\begin{center}
\begin{tabular}{l}
|\input{childdoc.def}|\\
|\childdocforward{|\textit{main}|}|\\
\end{tabular}
\end{center}
%
or alternatively with:
%
\begin{center}
\begin{tabular}{l}
|\input{childdoc.def}|\\
|\childdocby{|\textit{main}|}|\\
\end{tabular}
\end{center}
%
Both forms have slightly different effects as described above.
The main file is prepared as usual, see \secref{sec:include}.

%%%%%%%%%%%%%%%%%%%%%%%%%%%%%%%%%%%%%%%%%%%%%%%%%%%%%%%%%%%%%%%%%%%%%%%%%%%%%%%%
\subsection{Legacy Detection}
\label{sec:detection}

The directive |\childdocmain| in the main file can detect
whether the complete document or merely a child is to be compiled
even without using the directive |\childdocof|.
This method is deprecated because it is less robust
and there is no compelling reason to use it;
it is merely provided for backward compatibility
and it may be removed in future versions.

If the detection mechanism is to be used,
it is mandatory to correctly specify
the filename of the main file as the argument of |\childdocmain|:
%
\begin{center}
\begin{tabular}{l}
|\input{childdoc.def}|\\
|\childdocmain{|\textit{main}|}|\\
\end{tabular}
\end{center}
%
If |\jobname| does not match the argument \textit{main} of |\childdocmain|,
it is assumed that |\jobname| points to the child file to be compiled.
When using |\childdocmain| with the main file specified as argument,
it suffices to start a child file
with just |\input{|\textit{main}|}|
without loading of the package and using |\childdocof|.
If instead all processing is done
with the appropriate \textsf{childdoc} directives,
the argument of \textit{main} of |\childdocmain| can be empty.

An alternative version of the command line processing described
in \secref{sec:commandline} using the detection mechanism reads:
%
\begin{center}
|... -jobname "|\textit{target}|" "|[\textit{flags}]%
[|\def\jobname{|\textit{dest}|}|]|\input{|\textit{main}|}"|
\end{center}

%%%%%%%%%%%%%%%%%%%%%%%%%%%%%%%%%%%%%%%%%%%%%%%%%%%%%%%%%%%%%%%%%%%%%%%%%%%%%%%%
\subsection{Manual Code}
\label{sec:manual}

In case one cannot be certain whether the definitions file |childdoc.def|
is installed on the target \TeX{} distribution
and one prefers not to ship it,
it is conceivable to paste a few relevant commands into the sources.

To that end, drop all statements |\input{childdoc.def}|
and perform the replacements as outlined below.
Instead of |\childdocmain{|\textit{main}|}| add the following code
to the top of the main file:
%
\begin{center}
\begin{tabular}{l}
|\||ifdefined\childdocname\endinput\||fi\newif\ifchilddoc|\\
|\edef\childdocname{\scantokens\expandafter{\jobname\noexpand}}|\\
|\def\childdocmain{|\textit{main}|}\||ifx\childdocmain\childdocname\||else|\\
|\childdoctrue\includeonly{\childdocname}\let\jobname\childdocmain\||fi|\\
\end{tabular}
\end{center}
%
Instead of |\childdocof{|\textit{main}|}| just include the main file
at the top of each child file:
%
\begin{center}
|\input{|\textit{main}|}|
\end{center}
%
A simple redirection |\childdocforward{|\textit{dest}|}| is achieved by:
%
\begin{center}
|\def\jobname{|\textit{dest}|}\input{\jobname}|
\end{center}
%
The redirection with prefix
|\childdocforwardprefix[|\textit{prefix}|]{|\textit{dest}|}|
is accomplished by:
%
\begin{center}
\begin{tabular}{l}
|{\edef\jobname{\scantokens\expandafter{\jobname\noexpand}}|\\
|\def\redirectjob |\textit{prefix}|#1~~~{\gdef\jobname{|\textit{dest}|#1}}|\\
|\expandafter\redirectjob\jobname~~~}\input{\jobname}|
\end{tabular}
\end{center}

In an alternative approach,
child documents can be compiled by a specific command line
without additional code or specific definitions:
%
\begin{center}
|... -jobname "|\textit{target}|" "|[\textit{flags}]%
|\includeonly{|\textit{dest}|}\input{|\textit{main}|}"|
\end{center}
%

%%%%%%%%%%%%%%%%%%%%%%%%%%%%%%%%%%%%%%%%%%%%%%%%%%%%%%%%%%%%%%%%%%%%%%%%%%%%%%%%
%%%%%%%%%%%%%%%%%%%%%%%%%%%%%%%%%%%%%%%%%%%%%%%%%%%%%%%%%%%%%%%%%%%%%%%%%%%%%%%%
\section{Information}

%%%%%%%%%%%%%%%%%%%%%%%%%%%%%%%%%%%%%%%%%%%%%%%%%%%%%%%%%%%%%%%%%%%%%%%%%%%%%%%%
\subsection{Copyright}

Copyright \copyright{} 2017--2018 Niklas Beisert

This work may be distributed and/or modified under the
conditions of the \LaTeX{} Project Public License, either version 1.3
of this license or (at your option) any later version.
The latest version of this license is in
  \url{http://www.latex-project.org/lppl.txt}
and version 1.3 or later is part of all distributions of \LaTeX{}
version 2005/12/01 or later.

This work has the LPPL maintenance status `maintained'.

The Current Maintainer of this work is Niklas Beisert.

This work consists of the files |README.txt|, |childdoc.ins| and |childdoc.dtx|
as well as the derived files |childdoc.def|, |cdocsamp.tex|
with |cdocsch1.tex|, |cdocsch2.tex|, |cdocspt3.tex|, |cdocspt4.tex|,
|cdocsdrf.tex|, |cdocsfn1.tex|, |cdocsfn2.tex|
as well as |childdoc.pdf|.

%%%%%%%%%%%%%%%%%%%%%%%%%%%%%%%%%%%%%%%%%%%%%%%%%%%%%%%%%%%%%%%%%%%%%%%%%%%%%%%%
\subsection{Files and Installation}

The package consists of the files:
%
\begin{center}
\begin{tabular}{ll}
    |README.txt|   & readme file \\
    |childdoc.ins| & installation file \\
    |childdoc.dtx| & source file \\
    |childdoc.def| & definition file \\
    |cdocsamp.tex| & sample main file \\
    |cdocsch1.tex| & sample include file \\
    |cdocsch2.tex| & sample include file \\
    |cdocspt3.tex| & sample part file \\
    |cdocspt4.tex| & sample part file \\
    |cdocsdrf.tex| & sample redirection file \\
    |cdocsfn1.tex| & sample redirection file \\
    |cdocsfn2.tex| & sample redirection file \\
    |childdoc.pdf| & manual
\end{tabular}
\end{center}
%
The distribution consists of the files
|README.txt|, |childdoc.ins| and |childdoc.dtx|.
%
\begin{itemize}
\item
Run (pdf)\LaTeX{} on |childdoc.dtx|
to compile the manual |childdoc.pdf| (this file).
\item
Run \LaTeX{} on |childdoc.ins| to create the definitions file |childdoc.def|
and the sample |cdocsamp.tex| with include files
|cdocsch1.tex|, |cdocsch2.tex|, |cdocspt3.tex|, |cdocspt4.tex|,
|cdocsdrf.tex|, |cdocsfn1.tex|, |cdocsfn2.tex|.
Then copy the file |childdoc.def| to an appropriate directory of your \LaTeX{}
distribution, e.g.\ \textit{texmf-root}|/tex/latex/childdoc|.
\end{itemize}

%%%%%%%%%%%%%%%%%%%%%%%%%%%%%%%%%%%%%%%%%%%%%%%%%%%%%%%%%%%%%%%%%%%%%%%%%%%%%%%%
\subsection{Related CTAN Packages}

There are several other packages which offer a similar functionality:
%
\begin{itemize}
\item
The packages
\href{http://ctan.org/pkg/docmute}{\textsf{docmute}},
\href{http://ctan.org/pkg/includex}{\textsf{includex}} and
\href{http://ctan.org/pkg/standalone}{\textsf{standalone}}
provide commands to include only the document body of
a child file thus allowing both files to be compiled individually.
\item
The packages \href{http://ctan.org/pkg/subdocs}{\textsf{subdocs}}
and \href{http://ctan.org/pkg/subfiles}{\textsf{subfiles}}
provide structures in which the main and child documents can be
encapsulated and allowing them to be compiled individually.
The inclusion mechanism is different from the conventional |\include|.
\item
The package \href{http://ctan.org/pkg/combine}{\textsf{combine}}
is an elaborate solution to combine several documents into one.
\end{itemize}
%
See also the CTAN topic \href{http://ctan.org/topic/subdocs}{\textsf{subdocs}}
for further related packages.
The present package differs from the above solutions in that
a document structure constructed with the conventional |\include| mechanism
just needs two extra commands at the top of every file
such that all constituent files can be compiled individually.

%%%%%%%%%%%%%%%%%%%%%%%%%%%%%%%%%%%%%%%%%%%%%%%%%%%%%%%%%%%%%%%%%%%%%%%%%%%%%%%%
%\subsection{Feature Suggestions}
%
%The following is a list of features which may be useful for future
%versions of this package:
%%
%\begin{itemize}
%\item
%\ldots
%\end{itemize}

%%%%%%%%%%%%%%%%%%%%%%%%%%%%%%%%%%%%%%%%%%%%%%%%%%%%%%%%%%%%%%%%%%%%%%%%%%%%%%%%
\subsection{Revision History}

%%%%%%%%%%%%%%%%%%%%%%%%%%%%%%%%%%%%%%%%
\paragraph{v2.0:} 2018/12/30

\begin{itemize}
\item
immediate forward processing
\item
added |\childdocby| mechanism
\item
manual restructured
\end{itemize}

%%%%%%%%%%%%%%%%%%%%%%%%%%%%%%%%%%%%%%%%
\paragraph{v1.6:} 2018/01/17

\begin{itemize}
\item
application for development of include files
\item
corrections to manual
\end{itemize}

%%%%%%%%%%%%%%%%%%%%%%%%%%%%%%%%%%%%%%%%
\paragraph{v1.5:} 2017/05/21

\begin{itemize}
\item
more complete structuring introduced
\item
|\childdocof| introduced
\item
|\childdoc| renamed to |\childdocmain|
\item
|\childredirect| renamed to |\childdocforward| and |\childdocforwardprefix|
and functionality expanded
\end{itemize}

%%%%%%%%%%%%%%%%%%%%%%%%%%%%%%%%%%%%%%%%
\paragraph{v1.0:} 2017/04/27

\begin{itemize}
\item
manual and install package
\item
first version published on CTAN
\end{itemize}

%%%%%%%%%%%%%%%%%%%%%%%%%%%%%%%%%%%%%%%%
\paragraph{v0.6:} 2017/04/26

\begin{itemize}
\item
redirection mechanism added
\end{itemize}

%%%%%%%%%%%%%%%%%%%%%%%%%%%%%%%%%%%%%%%%
\paragraph{v0.5:} 2017/04/26

\begin{itemize}
\item
functionality in definition file
\end{itemize}


%%%%%%%%%%%%%%%%%%%%%%%%%%%%%%%%%%%%%%%%%%%%%%%%%%%%%%%%%%%%%%%%%%%%%%%%%%%%%%%%
%%%%%%%%%%%%%%%%%%%%%%%%%%%%%%%%%%%%%%%%%%%%%%%%%%%%%%%%%%%%%%%%%%%%%%%%%%%%%%%%
%%%%%%%%%%%%%%%%%%%%%%%%%%%%%%%%%%%%%%%%%%%%%%%%%%%%%%%%%%%%%%%%%%%%%%%%%%%%%%%%
\appendix

\settowidth\MacroIndent{\rmfamily\scriptsize 000\ }

 \DocInput{childdoc.dtx}

\end{document}
%</driver>
% \fi
%
% %%%%%%%%%%%%%%%%%%%%%%%%%%%%%%%%%%%%%%%%%%%%%%%%%%%%%%%%%%%%%%%%%%%%%%%%%%%%%%
% %%%%%%%%%%%%%%%%%%%%%%%%%%%%%%%%%%%%%%%%%%%%%%%%%%%%%%%%%%%%%%%%%%%%%%%%%%%%%%
% \section{Sample}
%\iffalse
%<*samplemain>
%\fi
%
% The following presents a sample document
% with two chapters, two parts, a title page,
% a compile flag as well as three forwarding files to set the flag.
% It consists of eight |.tex| files:
% \begin{center}
% \begin{tabular}{ll}
% |cdocsamp.tex|&main file\\
% |cdocsch1.tex|&include file for chapter 1\\
% |cdocsch2.tex|&include file for chapter 2\\
% |cdocspt3.tex|&include file for part 3\\
% |cdocspt4.tex|&include file for part 4\\
% |cdocsdrf.tex|&forwarding file for main file in draft mode\\
% |cdocsfi1.tex|&forwarding file for final version of chapter 1\\
% |cdocsfi2.tex|&forwarding file for final version of chapter 2\\
% \end{tabular}
% \end{center}
% Each of the eight files can be compiled directly by the \LaTeX{} compiler.
%
% %%%%%%%%%%%%%%%%%%%%%%%%%%%%%%%%%%%%%%
% \paragraph{Main File.}
%
% The main file is called |cdocsamp.tex|.
%
% Load the \textsf{childdoc} definitions and
% declare the filename for the main document:
%    \begin{macrocode}
\input{childdoc.def}
\childdocmain{}
%    \end{macrocode}

% Optional override for |\version| flag:
%    \begin{macrocode}
%%\ifchilddoc\else\providecommand{\version}{draft}\fi
%    \end{macrocode}

% Define the default values for the |\version| flag
% (|final| for the main file and |draft| for childs):
%    \begin{macrocode}
\ifchilddoc
\providecommand{\version}{draft}
\else
\providecommand{\version}{final}
\fi
%    \end{macrocode}

% Load the standard document class:
%    \begin{macrocode}
\documentclass[12pt]{article}
%    \end{macrocode}

% Start the document body:
%    \begin{macrocode}
\begin{document}
%    \end{macrocode}

% Declare a title page.
% Print title, part of document being processed and version flag:
%    \begin{macrocode}
\addtocounter{page}{-1}
\begin{center}
{\LARGE\bfseries{}childdoc example\par}
\vspace{1cm}
\ifchilddoc
\ifchilddocmanual part\else chapter\fi:
`\childdocname' of `\childdocjob'\par
\else
main document: `\childdocjob'\par
\fi
version: \version\par
\end{center}
\newpage
%    \end{macrocode}

% Manually include selected file,
% otherwise process as usual:
%    \begin{macrocode}
\ifchilddocmanual
\section*{part `\childdocname'}
\input{\childdocname}
\else
%    \end{macrocode}

% Include the two chapters:
%    \begin{macrocode}
\include{cdocsch1}
\include{cdocsch2}
%    \end{macrocode}

% Include the two parts unless only chapters should be displayed:
%    \begin{macrocode}
\ifchilddoc\else
\section{part three}
\input{cdocspt3}
\section{part four}
\input{cdocspt4}
\fi
%    \end{macrocode}

% Process as usual until here:
%    \begin{macrocode}
\fi
%    \end{macrocode}

% End of document body:
%    \begin{macrocode}
\end{document}
%    \end{macrocode}
%\iffalse
%</samplemain>
%\fi
%
% %%%%%%%%%%%%%%%%%%%%%%%%%%%%%%%%%%%%%%
% \paragraph{Chapter Include Files.}
%
% The include files are called |cdocsch1.tex| and |cdocsch2.tex|.
%
%\iffalse
%<*samplechap1|samplechap2>
%\fi

% Optional override for |\version| flag:
%    \begin{macrocode}
%%\providecommand{\version}{final}
%    \end{macrocode}

% Include the main document:
%    \begin{macrocode}
\input{childdoc.def}
\childdocof{cdocsamp}
%    \end{macrocode}

%\iffalse
%</samplechap1|samplechap2>
%\fi
%
%\iffalse
%<*samplechap1>
%\fi
% Some text for chapter 1:
%    \begin{macrocode}
\section{one}
some text in chapter one
%    \end{macrocode}

%\iffalse
%</samplechap1>
%\fi
% Some text for chapter 2:
%\iffalse
%<*samplechap2>
%\fi
%    \begin{macrocode}
\section{two}
more text in chapter two
%    \end{macrocode}

%\iffalse
%</samplechap2>
%\fi
%
% %%%%%%%%%%%%%%%%%%%%%%%%%%%%%%%%%%%%%%
% \paragraph{Part Include Files.}
%
% The include files are called |cdocspt3.tex| and |cdocspt4.tex|.
%
%\iffalse
%<*samplepart3|samplepart4>
%\fi

% Optional override for |\version| flag:
%    \begin{macrocode}
%%\providecommand{\version}{final}
%    \end{macrocode}

% Include the main document:
%    \begin{macrocode}
\input{childdoc.def}
\childdocby{cdocsamp}
%    \end{macrocode}

%\iffalse
%</samplepart3|samplepart4>
%\fi
%
%\iffalse
%<*samplepart3>
%\fi
% Some text for part 3:
%    \begin{macrocode}
some text in part three
%    \end{macrocode}

%\iffalse
%</samplepart3>
%\fi
% Some text for part 4:
%\iffalse
%<*samplepart4>
%\fi
%    \begin{macrocode}
more text in part four
%    \end{macrocode}

%\iffalse
%</samplepart4>
%\fi
%
% %%%%%%%%%%%%%%%%%%%%%%%%%%%%%%%%%%%%%%
% \paragraph{Forwarding for a Complete Draft.}
%
% The following forwarding file |cdocsdrf.tex|
% compiles the main document in draft mode:
%\iffalse
%<*sampledraft>
%\fi
%    \begin{macrocode}
\def\version{draft}
\input{childdoc.def}
\childdocforward{cdocsamp}
%    \end{macrocode}

%\iffalse
%</sampledraft>
%\fi
%
% %%%%%%%%%%%%%%%%%%%%%%%%%%%%%%%%%%%%%%
% \paragraph{Forwarding for Final Version of the Chapters.}
%
% The following forwarding files |cdocsfn1.tex| and |cdocsfn2.tex|
% (with identical content)
% compile the final versions of the child documents
% |cdocsch1.tex| and |cdocsch2.tex|, respectively:
%\iffalse
%<*samplefinal>
%\fi
%    \begin{macrocode}
\def\version{final}
\input{childdoc.def}
\childdocforwardprefix[cdocsamp]{cdocsfn}{cdocsch}
%    \end{macrocode}

%\iffalse
%</samplefinal>
%\fi
%
% %%%%%%%%%%%%%%%%%%%%%%%%%%%%%%%%%%%%%%
% \paragraph{Command Line Processing.}
%
% The following three command lines generate the output files
% |cdocscld|, |cdocscl1| and |cdocscl2|
% which should be identical to
% |cdocsdrf|, |cdocsch1| and |cdocsfn2|, respectively:
% \begin{center}
% \begin{tabular}{l}
% |latex -jobname cdocscld \|\\
% |  "\def\version{draft}\input{childdoc.def}\childdocforward{cdocsamp}"|\\
% |latex -jobname cdocscl1 \|\\
% |  "\input{childdoc.def}\childdocforward[cdocsamp]{cdocsch1}"|\\
% |latex -jobname cdocscl2 \|\\
% |  "\def\version{final}\input{childdoc.def}\childdocforward{cdocsch2}"|
% \end{tabular}
% \end{center}
% Note that the trailing backslash on each first line
% merely continues the input to the second line
% (for convenient cut ant paste).
% Furthermore, the command |latex| can be replaced by any
% of its alternative versions such as |pdflatex|.
%
% %%%%%%%%%%%%%%%%%%%%%%%%%%%%%%%%%%%%%%%%%%%%%%%%%%%%%%%%%%%%%%%%%%%%%%%%%%%%%%
% %%%%%%%%%%%%%%%%%%%%%%%%%%%%%%%%%%%%%%%%%%%%%%%%%%%%%%%%%%%%%%%%%%%%%%%%%%%%%%
% \section{Implementation}
%\iffalse
%<*package>
%\fi
%
% This section describes the definitions file |childdoc.def|.

% The definitions cannot be loaded using |\usepackage| or |\RequirePackage|
% which has a mechanism to prevent loading a style file more than once.
% When loading the definitions by means of |\input|
% multiple instances have to be prevented manually:
%\iffalse
%This code needs to be before the `\ProvidesFile' directive
%which is defined at the beginning of this file.
%Therefore it is also placed there and commented out here.
%</package>
%<*discard>
%\fi
%    \begin{macrocode}
\ifdefined\childdocmain\endinput\fi
%    \end{macrocode}
%\iffalse
%</discard>
%<*package>
%\fi
%
% \macro{\ifchilddoc}
% \macro{\ifchilddocmanual}
% The conditional |\ifchilddoc| tells whether a
% child (true) or main (false) document is being compiled.
% The conditional |\ifchilddocmanual| tells whether
% the |\includeonly| mechanism is used (false) or
% the selection of child files must be performed manually (true).
% The definitions initialise to false:
%    \begin{macrocode}
\newif\ifchilddoc
\newif\ifchilddocmanual
%    \end{macrocode}

% \macro{\childdocname}
% \macro{\childdocjob}
% The macro |\childdocname| stores the name of the main document
% to be compiled. The macro |\childdocjob| stores the name of
% the document on which the \LaTeX{} compiler was originally invoked.
% The content of |\jobname| cannot be compared
% to filenames specified in the source due to different catcodes.
% The following code rescans |\jobname|, stores the result
% in |\childdocname| and saves a copy in |\childdocjob|:
%    \begin{macrocode}
\edef\childdocname{\scantokens\expandafter{\jobname\noexpand}}
\let\childdocjob\childdocname
%    \end{macrocode}

% \macro{\childdocdisable}
% The macro |\childdocdisable| prevents the main file
% from being processed more than once.
% At this stage, the main document command |\childdocmain|
% is assumed to be called once again where it should do nothing.
% Any subsequent call to it should prevent
% a secondary processing of the main document
% It overwrites the forwarding commands
% |\childdocof| and |\childdocforward|
% with empty macros to prevent further inclusions of the main document:
%    \begin{macrocode}
\newcommand{\childdocdisable}
{
  \renewcommand{\childdocmain}[1]{\renewcommand{\childdocmain}[1]{\endinput}}
  \renewcommand{\childdocof}[1]{}
  \renewcommand{\childdocby}[2][]{}
  \renewcommand{\childdocforward}[2][]{}
  \renewcommand{\childdocdisable}{}
}
%    \end{macrocode}

% \macro{\childdocmain}
% The macro |\childdocmain| is to be called at the top of the main file
% with nothing or the main filename (without extension) as argument.
% First, it breaks loops.
% If the argument is not empty and does not match |\childdocname|
% (which is set by the first inclusion of |childdoc.def|),
% |\ifchilddoc| is set to true, |\includeonly| is applied to the child file
% and |\jobname| is set to the main file
% (for proper handling of |.aux| files):
%    \begin{macrocode}
\newcommand{\childdocmain}[1]
{
  \childdocdisable\childdocmain{}
  \if?#1?\else
    \begingroup
      \def\childdoctmp{#1}
      \ifx\childdoctmp\childdocname
        \def\childdoctmp{}
      \else
        \def\childdoctmp
        {
          \childdoctrue
          \includeonly{\childdocname}
          \def\childdocjob{#1}
          \def\jobname{#1}
        }
      \fi
      \expandafter
    \endgroup
    \childdoctmp
  \fi
}
%    \end{macrocode}

% \macro{\childdocof}
% The command |\childdocof| redirects
% compilation to the main file |#1|.
%    \begin{macrocode}
\newcommand{\childdocof}[1]
{
  \childdocdisable
  \childdoctrue
  \includeonly{\childdocname}
  \def\jobname{#1}
  \def\childdocjob{#1}
  \input{#1}
}
%    \end{macrocode}

% \macro{\childdocby}
% The command |\childdocby| ....
%    \begin{macrocode}
\newcommand{\childdocby}[2][]
{
  \childdocdisable
  \childdoctrue
  \childdocmanualtrue
  \if?#1?\else
    \def\jobname{#2}
  \fi
  \def\childdocjob{#2}
  \input{#2}
  \endinput
}
%    \end{macrocode}

% \macro{\childdocforward}
% The command |\childdocforward| redirects
% compilation to the main file or
% (if the optional argument is given) a child file.
% Parameters are set as if the main file
% or a child file starting with |\childdocof| was compiled.
% Then compilation is handed over to the main file:
%    \begin{macrocode}
\newcommand{\childdocforward}[2][]
{
  \begingroup
    \if?#1?
      \def\childdoctmp
      {
        \def\childdocname{#2}
        \def\childdocjob{#2}
        \def\jobname{#2}
        \input{#2}
        \endinput
      }
    \else
      \def\childdoctmp
      {
        \childdocdisable
        \def\childdocname{#2}
        \childdoctrue
        \includeonly{#2}
        \def\childdocjob{#1}
        \def\jobname{#1}
        \input{#1}
        \endinput
      }
    \fi
    \expandafter
  \endgroup
  \childdoctmp
}
%    \end{macrocode}

% \macro{\childdocforwardprefix}
% The command |\childdocforwardprefix| redirects
% compilation to the main or a child file by means of a pattern.
% The prefix |#1| in the current filename is replaced by |#2|
% and the suffix of the current filename is kept
% (it is assumed that the filename does not contain the substring `|~~~|'
% which is used as a delimiter).
% Compilation is handed over to the new file by |\childdocforward|:
%    \begin{macrocode}
\newcommand{\childdocforwardprefix}[3][]
{
  \begingroup
    \def\childdocextract #2##1~~~{\def\childdoctmp{\childdocforward[#1]{#3##1}}}
    \expandafter\childdocextract\childdocname~~~
    \expandafter
  \endgroup
  \childdoctmp
}
%    \end{macrocode}

% \macro{\childdoc}
% The deprecated macro |\childdoc| is a legacy version of |\childdocmain|:
%    \begin{macrocode}
\newcommand{\childdoc}{\childdocmain}
%    \end{macrocode}

% \macro{\childdocredirect}
% The deprecated macro |\childdocredirect| is a legacy version
% of |\childdocforward| and |\childdocforwardprefix|:
%    \begin{macrocode}
\newcommand{\childdocredirect}[2][]
{
  \begingroup
    \if?#1?
      \def\childdoctmp{\childdocforward{#2}}
    \else
      \def\childdoctmp{\childdocforwardprefix{#1}{#2}}
    \fi
    \expandafter
  \endgroup
  \childdoctmp
}
%    \end{macrocode}

%\iffalse
%</package>
%\fi
%
\endinput
\childdocforward{cdocsamp}"|\\
% |latex -jobname cdocscl1 \|\\
% |  "% \iffalse
%
% childdoc.dtx Copyright (C) 2017-2018 Niklas Beisert
%
% This work may be distributed and/or modified under the
% conditions of the LaTeX Project Public License, either version 1.3
% of this license or (at your option) any later version.
% The latest version of this license is in
%   http://www.latex-project.org/lppl.txt
% and version 1.3 or later is part of all distributions of LaTeX
% version 2005/12/01 or later.
%
% This work has the LPPL maintenance status `maintained'.
%
% The Current Maintainer of this work is Niklas Beisert.
%
% This work consists of the files childdoc.dtx and childdoc.ins
% and the derived files childdoc.def and cdocsamp.tex with
% cdocsch1.tex, cdocsch2.tex, cdocsdrf.tex, cdocsfn1.tex, cdocsfn2.tex.
%
%<package>\ifdefined\childdocmain\endinput\fi
%<package>\ProvidesFile{childdoc.def}[2018/12/30 v2.0 child document driver]
%<samplemain>\ProvidesFile{cdocsamp.tex}[2018/12/30 v2.0 sample for childdoc]
%<*driver>
%\ProvidesFile{childdoc.drv}[2018/12/30 v2.0 childdoc reference manual file]
\PassOptionsToClass{10pt,a4paper}{article}
\documentclass{ltxdoc}

\usepackage[margin=35mm]{geometry}
\usepackage{hyperref}
\usepackage{hyperxmp}
\usepackage[usenames]{color}

\hypersetup{colorlinks=true}
\hypersetup{pdfstartview=FitH}
\hypersetup{pdfpagemode=UseNone}
\hypersetup{pdfsource={}}
\hypersetup{pdflang={en-UK}}
\hypersetup{pdfcopyright={Copyright 2017-2018 Niklas Beisert.
  This work may be distributed and/or modified under the
  conditions of the LaTeX Project Public License, either version 1.3
  of this license or (at your option) any later version.}}
\hypersetup{pdflicenseurl={http://www.latex-project.org/lppl.txt}}
\hypersetup{pdfcontactaddress={ETH Zurich, ITP, HIT K,
  Wolfgang-Pauli-Strasse 27}}
\hypersetup{pdfcontactpostcode={8093}}
\hypersetup{pdfcontactcity={Zurich}}
\hypersetup{pdfcontactcountry={Switzerland}}
\hypersetup{pdfcontactemail={nbeisert@itp.phys.ethz.ch}}
\hypersetup{pdfcontacturl={http://people.phys.ethz.ch/\xmptilde nbeisert/}}

\newcommand{\secref}[1]{\hyperref[#1]{section \ref*{#1}}}

\parskip1ex
\parindent0pt
\let\olditemize\itemize
\def\itemize{\olditemize\parskip0pt}

\begin{document}

\title{The \textsf{childdoc} Package}
\hypersetup{pdftitle={The childdoc Package}}
\author{Niklas Beisert\\[2ex]
  Institut f\"ur Theoretische Physik\\
  Eidgen\"ossische Technische Hochschule Z\"urich\\
  Wolfgang-Pauli-Strasse 27, 8093 Z\"urich, Switzerland\\[1ex]
  \href{mailto:nbeisert@itp.phys.ethz.ch}
  {\texttt{nbeisert@itp.phys.ethz.ch}}}
\hypersetup{pdfauthor={Niklas Beisert}}
\hypersetup{pdfsubject={Manual for the LaTeX2e Package childdoc}}
\date{30 December 2018, \textsf{v2.0}}
\maketitle

\begin{abstract}\noindent
\textsf{childdoc} is a \LaTeXe{} package
that enables the direct compilation
of document sections included by |\include|
to individual files.
\end{abstract}

\begingroup
\parskip0ex
\tableofcontents
\endgroup

%%%%%%%%%%%%%%%%%%%%%%%%%%%%%%%%%%%%%%%%%%%%%%%%%%%%%%%%%%%%%%%%%%%%%%%%%%%%%%%%
%%%%%%%%%%%%%%%%%%%%%%%%%%%%%%%%%%%%%%%%%%%%%%%%%%%%%%%%%%%%%%%%%%%%%%%%%%%%%%%%
\section{Introduction}

\LaTeX{} provides a mechanism to structure a large document (such as a book)
into a main file and several child files (containing the chapters)
using the |\include| command.
This mechanism is beneficial for documents
which span hundreds of pages in order to
make the source file(s) more manageable.
Moreover, compilation can be restricted to
selected child files by means of the |\includeonly| command.
The latter feature can be used to reduce the compilation time while editing
(this was significantly more useful in the earlier days of \LaTeX{})
or to generate a smaller document which is easier to navigate.
Another application of |\includeonly| is to generate
documents consisting of selected parts of the complete document.

However, there are a few drawbacks of the plain |\include| mechanism:
\begin{itemize}
\item
The child files cannot be compiled on their own,
they can only be compiled via the main file.
A naive editing environment
(such as a text editor with an option
to have the current file processed by \LaTeX)
may require one to switch to the main file before compiling;
attempting to compile the child file produces errors.
\item
The main file must be modified (each time)
to adjust the |\includeonly| command
to the present needs. This easily leaves the main file in a messy state.
\item
The generated document will always carry the filename
of the main document. This is inconvenient if
several child files are to be compiled and
to be kept for distribution.
\end{itemize}

The present package provides a simple interface
to make child files individually compilable by \LaTeX{}.
Compiling a child file then has the same effect as compiling
the main file with an |\includeonly| command
to select the appropriate child.
Moreover the generated document will carry the name of the child
rather than the main file.
This resolves all three above issues.

This feature is meant to make the editing of books,
thesis documents and lecture notes somewhat more convenient.
However, the package can also be used efficiently for
composing a series of documents (such as exercise sheets)
which are typically distributed individually.
It then assists the author in generating the individual documents
(potentially in different versions)
as well as a document containing the collected series.
Another application is in developing style files
or other kinds of included material
where compilation of the style file could redirect
to a sample or test file.

%%%%%%%%%%%%%%%%%%%%%%%%%%%%%%%%%%%%%%%%%%%%%%%%%%%%%%%%%%%%%%%%%%%%%%%%%%%%%%%%
%%%%%%%%%%%%%%%%%%%%%%%%%%%%%%%%%%%%%%%%%%%%%%%%%%%%%%%%%%%%%%%%%%%%%%%%%%%%%%%%
\section{Usage}

First of all, the package \textsf{childdoc} is \emph{not} a standard
\LaTeXe{} |.sty| style file! Therefore it needs to be invoked in
a non-standard way.

%%%%%%%%%%%%%%%%%%%%%%%%%%%%%%%%%%%%%%%%%%%%%%%%%%%%%%%%%%%%%%%%%%%%%%%%%%%%%%%%
\subsection{Included Files}
\label{sec:include}

%%%%%%%%%%%%%%%%%%%%%%%%%%%%%%%%%%%%%%%%
\DescribeMacro{\childdocmain}
To use the package, add the commands
\begin{center}
\begin{tabular}{l}
|\input{childdoc.def}|\\
|\childdocmain{}|\\
\end{tabular}
\end{center}
at the very top of the main \LaTeX{} file,
in particular \emph{before} the |\documentclass| statement!
The argument of |\childdocmain| should be left empty
(but it must be present).

%%%%%%%%%%%%%%%%%%%%%%%%%%%%%%%%%%%%%%%%
\DescribeMacro{\childdocof}
Furthermore, add the commands
\begin{center}
\begin{tabular}{l}
|\input{childdoc.def}|\\
|\childdocof{|\textit{main}|}|\\
\end{tabular}
\end{center}
at the top of every child file \textit{child}
which is included by |\include{|\textit{child}|}|
from within the main file
(or at least for those files to be compiled individually).
The argument \textit{main} must be the filename of the main file.

There are a couple of
considerations in setting up the main and child documents:

%%%%%%%%%%%%%%%%%%%%%%%%%%%%%%%%%%%%%%%%
\paragraph{Restrictions.}

Please note the following restrictions:
\begin{itemize}
\item
|\childdocmain| must be called with one argument \textit{main}
to ensure compatibility with earlier version of the package.
It must either be empty (|\childdocmain{}|)
or precisely match the filename of the main file in which it is specified.
See \secref{sec:detection} for further information.
\item
The filename \textit{main} must be specified without the |.tex| extension.
\item
The filename \textit{main} is case sensitive
(even in case-insensitive file systems)
due to internal string comparison.
\item
The argument \textit{main} should be fully expanded, it cannot be a macro.
\item
Subdirectories and special characters should be avoided in filenames.
\item
The command |\childdocmain{|\textit{main}|}| must be followed by a whitespace.
It should not be followed immediately by another command
or by a comment mark `|%|'.
This is because the \TeX{} parser reads the token immediately following
the argument of |\childdocmain| and puts it
at the beginning of every child section;
however, a white\-space is ignored.
\end{itemize}

%%%%%%%%%%%%%%%%%%%%%%%%%%%%%%%%%%%%%%%%
\paragraph{Content of Main File.}

It is advisable to place all content in the child files included by |\include|.
Any output contained in the main file will appear in all child documents
unless suppressed manually;
it cannot be suppressed automatically by the |\includeonly| directive
and thus should normally be avoided.
A method to include some content in the main file
by means of conditional processing is described in \secref{sec:conditional}.

%%%%%%%%%%%%%%%%%%%%%%%%%%%%%%%%%%%%%%%%
\paragraph{Page Numbering.}

When only a part of the document is compiled,
the appropriate numbering of pages
(as well as other status parameters)
is determined from the |.aux| files.
The latter contain information from previous passes.
However this information needs to propagate through
all intermediate child documents.
Therefore the page numbering in child documents may well
be inconsistent until the complete document is compiled at least once.

A useful (if unconventional) way to always ensure a consistent
page numbering is to restart the numbering in each child document
and denote the pages by `\textit{child}|.|\textit{page}'
where \textit{child} represents the chapter/section number of the child file.
This can be achieved by the command
|\numberwithin{page}{|\textit{child}|}|
of the \textsf{amsmath} package
where \textit{child} can be |chapter| or |section|
depending on the chosen structuring.
Alternatively, one can modify the macro |\thepage| appropriately
and reset the counter |page| at the start of each child file.

%%%%%%%%%%%%%%%%%%%%%%%%%%%%%%%%%%%%%%%%%%%%%%%%%%%%%%%%%%%%%%%%%%%%%%%%%%%%%%%%
\subsection{Conditional Processing}
\label{sec:conditional}

The package provides a mechanism to compile different versions
of a document. To customise the versions further some conditional processing
can come in handy to distinguish which version is being compiled.
The package provides two macros to describe the compilation context:

%%%%%%%%%%%%%%%%%%%%%%%%%%%%%%%%%%%%%%%%
\DescribeMacro{\ifchilddoc}
The conditional |\ifchilddoc| distinguishes between the compilation of
child documents and the main document:
%
\begin{center}
|\ifchilddoc |\textit{child-code}| |[|\||else |\textit{main-code}]| \||fi|
\end{center}

%%%%%%%%%%%%%%%%%%%%%%%%%%%%%%%%%%%%%%%%
\DescribeMacro{\childdocname}
\DescribeMacro{\childdocjob}
The macro |\childdocname| contains the filename (without extension)
of the main or child file being processed.
Note that |\childdocjob| will always contain the name of the main file.

%%%%%%%%%%%%%%%%%%%%%%%%%%%%%%%%%%%%%%%%
\paragraph{Title Page.}

Conditional processing can be used to include a title or banner page
in the main document when proper precautions are taken.
Importantly, the code in the main file should ensure that the page counter
(as well as other status parameters which are stored in the |.aux| files)
takes the same value after the conditional processing.
Otherwise the page numbers may take divergent values
depending on which part is compiled.

For example, a title page could be declared by:
%
\begin{center}
\begin{tabular}{l}
|\ifchilddoc\||else|\\
|\addtocounter{page}{-1}|\\
\textit{code for title page}\\
|\newpage|\\
|\||fi|
\end{tabular}
\end{center}
%
A banner page for the child documents can be generated by:
%
\begin{center}
\begin{tabular}{l}
|\ifchilddoc|\\
|\addtocounter{page}{-1}|\\
\textit{code for banner page}\\
|\newpage|\\
|\||fi|
\end{tabular}
\end{center}
%
Here one could write a message such as:
\begin{center}
|This is the part \childdocname{} of \childdocjob{}.|
\end{center}

%%%%%%%%%%%%%%%%%%%%%%%%%%%%%%%%%%%%%%%%%%%%%%%%%%%%%%%%%%%%%%%%%%%%%%%%%%%%%%%%
\subsection{Flags}
\label{sec:flags}

The package makes it easy to generate different versions
of the main or child documents.
To this end compilation flags can be defined
and assigned different default values.
They will be particularly useful in conjunction
with the forwarding mechanism described in \secref{sec:forward}.

For example, it may be useful to have a flag |\version|
which can be set to |draft| or |final|.
The document source will contain some conditional code
depending on the value of |\version|.
Suppose further, the flag should default to |final| for the main file
and to |draft| for child files
which is a natural assignment for editing the document.
This is achieved by placing the following code
in the preamble of the main document
(below the |\childdocmain| directive):
%
\begin{center}
\begin{tabular}{l}
|\ifchilddoc|\\
|\providecommand{\version}{draft}|\\
|\||else|\\
|\providecommand{\version}{final}|\\
|\||fi|
\end{tabular}
\end{center}
%
The definition by |\providecommand| makes sure
that previous definitions are not overwritten.
Further statements |\providecommand{\version}{...}|
can thus be added before the above code to override it.

For the main file, one might add a line
(between |\childdocmain| and the above block)
%
\begin{center}
|%\ifchilddoc\||else\providecommand{\version}{draft}\||fi|
\end{center}
%
which can be uncommented to produce a draft version.
Likewise one can add a line to the very top of a child file
(above the |\childdocof{|\textit{main}|}| directive)
%
\begin{center}
|%\providecommand{\version}{final}|
\end{center}
%
which can be uncommented to produce the final version of this child document.

%%%%%%%%%%%%%%%%%%%%%%%%%%%%%%%%%%%%%%%%%%%%%%%%%%%%%%%%%%%%%%%%%%%%%%%%%%%%%%%%
\subsection{Forwarding}
\label{sec:forward}

Different versions of the main or child documents
using compilation flags as described in \secref{sec:flags}
can be (permanently) stored in different files
for convenient compilation, viewing and distribution.
To this end, the package defines a command
to pass on compilation to a different file:

%%%%%%%%%%%%%%%%%%%%%%%%%%%%%%%%%%%%%%%%
\DescribeMacro{\childdocforward}
The command |\childdocforward| redirects processing to
another source file:
%
\begin{center}
\begin{tabular}{l}
|\input{childdoc.def}|\\
|\childdocforward[|\textit{main}|]{|\textit{dest}|}|\\
\end{tabular}
\end{center}
%
The argument \textit{dest} is the destination file
(without extension).
It should be the main file or one of the child files.
Note that further \textsf{childdoc} directives
such as |\childdocof| and |\childdocforward|
in the indicated file will be processed in this form.
The optional argument \textit{main}
passes on directly to the main file \textit{main}
while pretending to compile the child \textit{dest}.
This form behaves as if \textit{dest}
issues |\childdocof{|\textit{main}|}| right away,
and no further \textsf{childdoc} directives will be processed.

%%%%%%%%%%%%%%%%%%%%%%%%%%%%%%%%%%%%%%%%
\DescribeMacro{\...prefix}
In the alternative form |\childdocforwardprefix|,
%
\begin{center}
\begin{tabular}{l}
|\input{childdoc.def}|\\
|\childdocforwardprefix[|\textit{main}|]{|\textit{prefix}|}{|\textit{dest}|}|
\end{tabular}
\end{center}
%
the destination file is determined by a pattern
depending on the current file:
To make this work, the current file must be called
`{\textit{prefix}\hspace{0.2em}\textit{suffix}}'
with \textit{prefix} matching precisely the argument.
Processing is then passed on to the file
`{\textit{dest}\hspace{0.2em}\textit{suffix}}'.
Surely, the same effect is achieved by
directly specifying the
argument `{\textit{dest}\hspace{0.2em}\textit{suffix}}'
in the first form.
However, that requires to set up a different file
for each child. With the alternative form of the command
all these files can have exactly the same content
which simplifies setting them up and maintaining them.

For example, the following file |draft.tex|
with a compilation flag |\version| as described in \secref{sec:flags}
compiles the main document as a draft:
%
\begin{center}
\begin{tabular}{l}
|\def\version{draft}|\\
|\input{childdoc.def}|\\
|\childdocforward{|\textit{main}|}|
\end{tabular}
\end{center}
%
Likewise, the following files |final|\textit{nn}|.tex|
compile the final version of the child document
|child|\textit{nn}|.tex|:
%
\begin{center}
\begin{tabular}{l}
|\def\version{final}|\\
|\input{childdoc.def}|\\
|\childdocforwardprefix{final}{child}|
\end{tabular}
\end{center}
%

Note that when several versions of a main file and/or of each child file
are to be generated, it may be convenient to set up a |Makefile| or
shell script to automatise the process.

%%%%%%%%%%%%%%%%%%%%%%%%%%%%%%%%%%%%%%%%%%%%%%%%%%%%%%%%%%%%%%%%%%%%%%%%%%%%%%%%
\subsection{Command Line Processing}
\label{sec:commandline}

The effect of redirection files can also be achieved by invoking
the \LaTeX{} compiler with a more elaborate command line.
Most conveniently this should be done as part
of a shell script or a |Makefile|.

When using \textsf{childdoc} in the main file, the following
command lines effectively perform a redirection
(note that depending on the shell being used,
backslashes may have to be doubled: `|\|' $\to$ `|\\|'):
%
\begin{center}
|... -jobname "|\textit{target}|" |\\|"|[\textit{flags}]%
|\input{childdoc.def}\childdocforward[|\textit{main}|]{|\textit{dest}|}"|
\end{center}
%
Here \textit{target} is the name of the output file,
\textit{main} is the name of the main file
and \textit{dest} is the name of the main or child file to be processed
(all filenames without extensions).
The optional argument \textit{main} can be omitted
if \textit{main} matches \textit{dest}.
Optionally, compilation \textit{flags} can be defined via |\def| commands.
This command line makes the \TeX{} engine believe
it is compiling the file \textit{target}
whose content is specified as the latter parameter.
The provided code then forwards the processing to
\textit{main} or \textit{dest} as described in \secref{sec:forward}.

%%%%%%%%%%%%%%%%%%%%%%%%%%%%%%%%%%%%%%%%%%%%%%%%%%%%%%%%%%%%%%%%%%%%%%%%%%%%%%%%
\subsection{Include by Input}
\label{sec:input}

Including child documents by |\include| has some restrictions by design.
Most notably, the content of a child document always occupies
its own set of pages; pages cannot be shared between child documents.
Usually, this behaviour makes perfect sense
because each child document contain an essential part of the document.
However, in some situations it may be desirable to compose
a document from a collection of parts
without having mandatory page breaks between then.
For this case, the package
provides a mechanism to include parts
by |\input| which can also be processed individually.
However, by construction this mechanism
requires manual handling of the content to be output.

%%%%%%%%%%%%%%%%%%%%%%%%%%%%%%%%%%%%%%%%
\DescribeMacro{\ifchilddocmanual}
The main file should be prepared as usual, see \secref{sec:include}.
However, the document body must make a distinction
between processing of an individual part and of the main document, e.g.:
%
\begin{center}
\begin{tabular}{l}
|\ifchilddocmanual|\\
|\input{\childdocname}|\\
|\||else|\\
\textit{document body with }|\input{|\textit{part}|}|\\
|\||fi|
\end{tabular}
\end{center}
%
The conditional |\ifchilddocmanual| is true whenever
a part to be included by |\input| is being compiled,
and the name of the part is stored in |\childdocname|.

%%%%%%%%%%%%%%%%%%%%%%%%%%%%%%%%%%%%%%%%
\DescribeMacro{\childdocby}
Each part to be included by |\input| should start with:
%
\begin{center}
\begin{tabular}{l}
|\input{childdoc.def}|\\
|\childdocby{|\textit{main}|}|\\
\end{tabular}
\end{center}
%
The directive |\childdocby| is similar to |\childdocof|
described in \secref{sec:include},
but the subsequent selection of content must be done manually.
To that end, both |\ifchilddoc| and |\ifchilddocmanual|
will be true upon processing of a part,
and the name of the part is stored in |\childdocname|.
Note that |\jobname| will be set to the filename of the current part
so that each part receives an individual |.aux| file
that does not interfere with the |.aux| file(s) of the main document.
This behaviour can be altered by the alternative form
|\childdocby[*]{|\textit{main}|}| (with a non-empty optional argument)
which uses the |.aux| file of the main document
by setting |\jobname| to \textit{main}.

%%%%%%%%%%%%%%%%%%%%%%%%%%%%%%%%%%%%%%%%%%%%%%%%%%%%%%%%%%%%%%%%%%%%%%%%%%%%%%%%
\subsection{Driver Development}
\label{sec:driver}

The \textsf{childdoc} mechanism can also be use for the development
of definition files such as \LaTeX{} styles or classes.
This case differs from the above setup with multiple parts
included by |\include| in that no |\includeonly| should be invoked.
This can be achieved by starting the include file
(before |\ProvidesPackage|) with:
%
\begin{center}
\begin{tabular}{l}
|\input{childdoc.def}|\\
|\childdocforward{|\textit{main}|}|\\
\end{tabular}
\end{center}
%
or alternatively with:
%
\begin{center}
\begin{tabular}{l}
|\input{childdoc.def}|\\
|\childdocby{|\textit{main}|}|\\
\end{tabular}
\end{center}
%
Both forms have slightly different effects as described above.
The main file is prepared as usual, see \secref{sec:include}.

%%%%%%%%%%%%%%%%%%%%%%%%%%%%%%%%%%%%%%%%%%%%%%%%%%%%%%%%%%%%%%%%%%%%%%%%%%%%%%%%
\subsection{Legacy Detection}
\label{sec:detection}

The directive |\childdocmain| in the main file can detect
whether the complete document or merely a child is to be compiled
even without using the directive |\childdocof|.
This method is deprecated because it is less robust
and there is no compelling reason to use it;
it is merely provided for backward compatibility
and it may be removed in future versions.

If the detection mechanism is to be used,
it is mandatory to correctly specify
the filename of the main file as the argument of |\childdocmain|:
%
\begin{center}
\begin{tabular}{l}
|\input{childdoc.def}|\\
|\childdocmain{|\textit{main}|}|\\
\end{tabular}
\end{center}
%
If |\jobname| does not match the argument \textit{main} of |\childdocmain|,
it is assumed that |\jobname| points to the child file to be compiled.
When using |\childdocmain| with the main file specified as argument,
it suffices to start a child file
with just |\input{|\textit{main}|}|
without loading of the package and using |\childdocof|.
If instead all processing is done
with the appropriate \textsf{childdoc} directives,
the argument of \textit{main} of |\childdocmain| can be empty.

An alternative version of the command line processing described
in \secref{sec:commandline} using the detection mechanism reads:
%
\begin{center}
|... -jobname "|\textit{target}|" "|[\textit{flags}]%
[|\def\jobname{|\textit{dest}|}|]|\input{|\textit{main}|}"|
\end{center}

%%%%%%%%%%%%%%%%%%%%%%%%%%%%%%%%%%%%%%%%%%%%%%%%%%%%%%%%%%%%%%%%%%%%%%%%%%%%%%%%
\subsection{Manual Code}
\label{sec:manual}

In case one cannot be certain whether the definitions file |childdoc.def|
is installed on the target \TeX{} distribution
and one prefers not to ship it,
it is conceivable to paste a few relevant commands into the sources.

To that end, drop all statements |\input{childdoc.def}|
and perform the replacements as outlined below.
Instead of |\childdocmain{|\textit{main}|}| add the following code
to the top of the main file:
%
\begin{center}
\begin{tabular}{l}
|\||ifdefined\childdocname\endinput\||fi\newif\ifchilddoc|\\
|\edef\childdocname{\scantokens\expandafter{\jobname\noexpand}}|\\
|\def\childdocmain{|\textit{main}|}\||ifx\childdocmain\childdocname\||else|\\
|\childdoctrue\includeonly{\childdocname}\let\jobname\childdocmain\||fi|\\
\end{tabular}
\end{center}
%
Instead of |\childdocof{|\textit{main}|}| just include the main file
at the top of each child file:
%
\begin{center}
|\input{|\textit{main}|}|
\end{center}
%
A simple redirection |\childdocforward{|\textit{dest}|}| is achieved by:
%
\begin{center}
|\def\jobname{|\textit{dest}|}\input{\jobname}|
\end{center}
%
The redirection with prefix
|\childdocforwardprefix[|\textit{prefix}|]{|\textit{dest}|}|
is accomplished by:
%
\begin{center}
\begin{tabular}{l}
|{\edef\jobname{\scantokens\expandafter{\jobname\noexpand}}|\\
|\def\redirectjob |\textit{prefix}|#1~~~{\gdef\jobname{|\textit{dest}|#1}}|\\
|\expandafter\redirectjob\jobname~~~}\input{\jobname}|
\end{tabular}
\end{center}

In an alternative approach,
child documents can be compiled by a specific command line
without additional code or specific definitions:
%
\begin{center}
|... -jobname "|\textit{target}|" "|[\textit{flags}]%
|\includeonly{|\textit{dest}|}\input{|\textit{main}|}"|
\end{center}
%

%%%%%%%%%%%%%%%%%%%%%%%%%%%%%%%%%%%%%%%%%%%%%%%%%%%%%%%%%%%%%%%%%%%%%%%%%%%%%%%%
%%%%%%%%%%%%%%%%%%%%%%%%%%%%%%%%%%%%%%%%%%%%%%%%%%%%%%%%%%%%%%%%%%%%%%%%%%%%%%%%
\section{Information}

%%%%%%%%%%%%%%%%%%%%%%%%%%%%%%%%%%%%%%%%%%%%%%%%%%%%%%%%%%%%%%%%%%%%%%%%%%%%%%%%
\subsection{Copyright}

Copyright \copyright{} 2017--2018 Niklas Beisert

This work may be distributed and/or modified under the
conditions of the \LaTeX{} Project Public License, either version 1.3
of this license or (at your option) any later version.
The latest version of this license is in
  \url{http://www.latex-project.org/lppl.txt}
and version 1.3 or later is part of all distributions of \LaTeX{}
version 2005/12/01 or later.

This work has the LPPL maintenance status `maintained'.

The Current Maintainer of this work is Niklas Beisert.

This work consists of the files |README.txt|, |childdoc.ins| and |childdoc.dtx|
as well as the derived files |childdoc.def|, |cdocsamp.tex|
with |cdocsch1.tex|, |cdocsch2.tex|, |cdocspt3.tex|, |cdocspt4.tex|,
|cdocsdrf.tex|, |cdocsfn1.tex|, |cdocsfn2.tex|
as well as |childdoc.pdf|.

%%%%%%%%%%%%%%%%%%%%%%%%%%%%%%%%%%%%%%%%%%%%%%%%%%%%%%%%%%%%%%%%%%%%%%%%%%%%%%%%
\subsection{Files and Installation}

The package consists of the files:
%
\begin{center}
\begin{tabular}{ll}
    |README.txt|   & readme file \\
    |childdoc.ins| & installation file \\
    |childdoc.dtx| & source file \\
    |childdoc.def| & definition file \\
    |cdocsamp.tex| & sample main file \\
    |cdocsch1.tex| & sample include file \\
    |cdocsch2.tex| & sample include file \\
    |cdocspt3.tex| & sample part file \\
    |cdocspt4.tex| & sample part file \\
    |cdocsdrf.tex| & sample redirection file \\
    |cdocsfn1.tex| & sample redirection file \\
    |cdocsfn2.tex| & sample redirection file \\
    |childdoc.pdf| & manual
\end{tabular}
\end{center}
%
The distribution consists of the files
|README.txt|, |childdoc.ins| and |childdoc.dtx|.
%
\begin{itemize}
\item
Run (pdf)\LaTeX{} on |childdoc.dtx|
to compile the manual |childdoc.pdf| (this file).
\item
Run \LaTeX{} on |childdoc.ins| to create the definitions file |childdoc.def|
and the sample |cdocsamp.tex| with include files
|cdocsch1.tex|, |cdocsch2.tex|, |cdocspt3.tex|, |cdocspt4.tex|,
|cdocsdrf.tex|, |cdocsfn1.tex|, |cdocsfn2.tex|.
Then copy the file |childdoc.def| to an appropriate directory of your \LaTeX{}
distribution, e.g.\ \textit{texmf-root}|/tex/latex/childdoc|.
\end{itemize}

%%%%%%%%%%%%%%%%%%%%%%%%%%%%%%%%%%%%%%%%%%%%%%%%%%%%%%%%%%%%%%%%%%%%%%%%%%%%%%%%
\subsection{Related CTAN Packages}

There are several other packages which offer a similar functionality:
%
\begin{itemize}
\item
The packages
\href{http://ctan.org/pkg/docmute}{\textsf{docmute}},
\href{http://ctan.org/pkg/includex}{\textsf{includex}} and
\href{http://ctan.org/pkg/standalone}{\textsf{standalone}}
provide commands to include only the document body of
a child file thus allowing both files to be compiled individually.
\item
The packages \href{http://ctan.org/pkg/subdocs}{\textsf{subdocs}}
and \href{http://ctan.org/pkg/subfiles}{\textsf{subfiles}}
provide structures in which the main and child documents can be
encapsulated and allowing them to be compiled individually.
The inclusion mechanism is different from the conventional |\include|.
\item
The package \href{http://ctan.org/pkg/combine}{\textsf{combine}}
is an elaborate solution to combine several documents into one.
\end{itemize}
%
See also the CTAN topic \href{http://ctan.org/topic/subdocs}{\textsf{subdocs}}
for further related packages.
The present package differs from the above solutions in that
a document structure constructed with the conventional |\include| mechanism
just needs two extra commands at the top of every file
such that all constituent files can be compiled individually.

%%%%%%%%%%%%%%%%%%%%%%%%%%%%%%%%%%%%%%%%%%%%%%%%%%%%%%%%%%%%%%%%%%%%%%%%%%%%%%%%
%\subsection{Feature Suggestions}
%
%The following is a list of features which may be useful for future
%versions of this package:
%%
%\begin{itemize}
%\item
%\ldots
%\end{itemize}

%%%%%%%%%%%%%%%%%%%%%%%%%%%%%%%%%%%%%%%%%%%%%%%%%%%%%%%%%%%%%%%%%%%%%%%%%%%%%%%%
\subsection{Revision History}

%%%%%%%%%%%%%%%%%%%%%%%%%%%%%%%%%%%%%%%%
\paragraph{v2.0:} 2018/12/30

\begin{itemize}
\item
immediate forward processing
\item
added |\childdocby| mechanism
\item
manual restructured
\end{itemize}

%%%%%%%%%%%%%%%%%%%%%%%%%%%%%%%%%%%%%%%%
\paragraph{v1.6:} 2018/01/17

\begin{itemize}
\item
application for development of include files
\item
corrections to manual
\end{itemize}

%%%%%%%%%%%%%%%%%%%%%%%%%%%%%%%%%%%%%%%%
\paragraph{v1.5:} 2017/05/21

\begin{itemize}
\item
more complete structuring introduced
\item
|\childdocof| introduced
\item
|\childdoc| renamed to |\childdocmain|
\item
|\childredirect| renamed to |\childdocforward| and |\childdocforwardprefix|
and functionality expanded
\end{itemize}

%%%%%%%%%%%%%%%%%%%%%%%%%%%%%%%%%%%%%%%%
\paragraph{v1.0:} 2017/04/27

\begin{itemize}
\item
manual and install package
\item
first version published on CTAN
\end{itemize}

%%%%%%%%%%%%%%%%%%%%%%%%%%%%%%%%%%%%%%%%
\paragraph{v0.6:} 2017/04/26

\begin{itemize}
\item
redirection mechanism added
\end{itemize}

%%%%%%%%%%%%%%%%%%%%%%%%%%%%%%%%%%%%%%%%
\paragraph{v0.5:} 2017/04/26

\begin{itemize}
\item
functionality in definition file
\end{itemize}


%%%%%%%%%%%%%%%%%%%%%%%%%%%%%%%%%%%%%%%%%%%%%%%%%%%%%%%%%%%%%%%%%%%%%%%%%%%%%%%%
%%%%%%%%%%%%%%%%%%%%%%%%%%%%%%%%%%%%%%%%%%%%%%%%%%%%%%%%%%%%%%%%%%%%%%%%%%%%%%%%
%%%%%%%%%%%%%%%%%%%%%%%%%%%%%%%%%%%%%%%%%%%%%%%%%%%%%%%%%%%%%%%%%%%%%%%%%%%%%%%%
\appendix

\settowidth\MacroIndent{\rmfamily\scriptsize 000\ }

 \DocInput{childdoc.dtx}

\end{document}
%</driver>
% \fi
%
% %%%%%%%%%%%%%%%%%%%%%%%%%%%%%%%%%%%%%%%%%%%%%%%%%%%%%%%%%%%%%%%%%%%%%%%%%%%%%%
% %%%%%%%%%%%%%%%%%%%%%%%%%%%%%%%%%%%%%%%%%%%%%%%%%%%%%%%%%%%%%%%%%%%%%%%%%%%%%%
% \section{Sample}
%\iffalse
%<*samplemain>
%\fi
%
% The following presents a sample document
% with two chapters, two parts, a title page,
% a compile flag as well as three forwarding files to set the flag.
% It consists of eight |.tex| files:
% \begin{center}
% \begin{tabular}{ll}
% |cdocsamp.tex|&main file\\
% |cdocsch1.tex|&include file for chapter 1\\
% |cdocsch2.tex|&include file for chapter 2\\
% |cdocspt3.tex|&include file for part 3\\
% |cdocspt4.tex|&include file for part 4\\
% |cdocsdrf.tex|&forwarding file for main file in draft mode\\
% |cdocsfi1.tex|&forwarding file for final version of chapter 1\\
% |cdocsfi2.tex|&forwarding file for final version of chapter 2\\
% \end{tabular}
% \end{center}
% Each of the eight files can be compiled directly by the \LaTeX{} compiler.
%
% %%%%%%%%%%%%%%%%%%%%%%%%%%%%%%%%%%%%%%
% \paragraph{Main File.}
%
% The main file is called |cdocsamp.tex|.
%
% Load the \textsf{childdoc} definitions and
% declare the filename for the main document:
%    \begin{macrocode}
\input{childdoc.def}
\childdocmain{}
%    \end{macrocode}

% Optional override for |\version| flag:
%    \begin{macrocode}
%%\ifchilddoc\else\providecommand{\version}{draft}\fi
%    \end{macrocode}

% Define the default values for the |\version| flag
% (|final| for the main file and |draft| for childs):
%    \begin{macrocode}
\ifchilddoc
\providecommand{\version}{draft}
\else
\providecommand{\version}{final}
\fi
%    \end{macrocode}

% Load the standard document class:
%    \begin{macrocode}
\documentclass[12pt]{article}
%    \end{macrocode}

% Start the document body:
%    \begin{macrocode}
\begin{document}
%    \end{macrocode}

% Declare a title page.
% Print title, part of document being processed and version flag:
%    \begin{macrocode}
\addtocounter{page}{-1}
\begin{center}
{\LARGE\bfseries{}childdoc example\par}
\vspace{1cm}
\ifchilddoc
\ifchilddocmanual part\else chapter\fi:
`\childdocname' of `\childdocjob'\par
\else
main document: `\childdocjob'\par
\fi
version: \version\par
\end{center}
\newpage
%    \end{macrocode}

% Manually include selected file,
% otherwise process as usual:
%    \begin{macrocode}
\ifchilddocmanual
\section*{part `\childdocname'}
\input{\childdocname}
\else
%    \end{macrocode}

% Include the two chapters:
%    \begin{macrocode}
\include{cdocsch1}
\include{cdocsch2}
%    \end{macrocode}

% Include the two parts unless only chapters should be displayed:
%    \begin{macrocode}
\ifchilddoc\else
\section{part three}
\input{cdocspt3}
\section{part four}
\input{cdocspt4}
\fi
%    \end{macrocode}

% Process as usual until here:
%    \begin{macrocode}
\fi
%    \end{macrocode}

% End of document body:
%    \begin{macrocode}
\end{document}
%    \end{macrocode}
%\iffalse
%</samplemain>
%\fi
%
% %%%%%%%%%%%%%%%%%%%%%%%%%%%%%%%%%%%%%%
% \paragraph{Chapter Include Files.}
%
% The include files are called |cdocsch1.tex| and |cdocsch2.tex|.
%
%\iffalse
%<*samplechap1|samplechap2>
%\fi

% Optional override for |\version| flag:
%    \begin{macrocode}
%%\providecommand{\version}{final}
%    \end{macrocode}

% Include the main document:
%    \begin{macrocode}
\input{childdoc.def}
\childdocof{cdocsamp}
%    \end{macrocode}

%\iffalse
%</samplechap1|samplechap2>
%\fi
%
%\iffalse
%<*samplechap1>
%\fi
% Some text for chapter 1:
%    \begin{macrocode}
\section{one}
some text in chapter one
%    \end{macrocode}

%\iffalse
%</samplechap1>
%\fi
% Some text for chapter 2:
%\iffalse
%<*samplechap2>
%\fi
%    \begin{macrocode}
\section{two}
more text in chapter two
%    \end{macrocode}

%\iffalse
%</samplechap2>
%\fi
%
% %%%%%%%%%%%%%%%%%%%%%%%%%%%%%%%%%%%%%%
% \paragraph{Part Include Files.}
%
% The include files are called |cdocspt3.tex| and |cdocspt4.tex|.
%
%\iffalse
%<*samplepart3|samplepart4>
%\fi

% Optional override for |\version| flag:
%    \begin{macrocode}
%%\providecommand{\version}{final}
%    \end{macrocode}

% Include the main document:
%    \begin{macrocode}
\input{childdoc.def}
\childdocby{cdocsamp}
%    \end{macrocode}

%\iffalse
%</samplepart3|samplepart4>
%\fi
%
%\iffalse
%<*samplepart3>
%\fi
% Some text for part 3:
%    \begin{macrocode}
some text in part three
%    \end{macrocode}

%\iffalse
%</samplepart3>
%\fi
% Some text for part 4:
%\iffalse
%<*samplepart4>
%\fi
%    \begin{macrocode}
more text in part four
%    \end{macrocode}

%\iffalse
%</samplepart4>
%\fi
%
% %%%%%%%%%%%%%%%%%%%%%%%%%%%%%%%%%%%%%%
% \paragraph{Forwarding for a Complete Draft.}
%
% The following forwarding file |cdocsdrf.tex|
% compiles the main document in draft mode:
%\iffalse
%<*sampledraft>
%\fi
%    \begin{macrocode}
\def\version{draft}
\input{childdoc.def}
\childdocforward{cdocsamp}
%    \end{macrocode}

%\iffalse
%</sampledraft>
%\fi
%
% %%%%%%%%%%%%%%%%%%%%%%%%%%%%%%%%%%%%%%
% \paragraph{Forwarding for Final Version of the Chapters.}
%
% The following forwarding files |cdocsfn1.tex| and |cdocsfn2.tex|
% (with identical content)
% compile the final versions of the child documents
% |cdocsch1.tex| and |cdocsch2.tex|, respectively:
%\iffalse
%<*samplefinal>
%\fi
%    \begin{macrocode}
\def\version{final}
\input{childdoc.def}
\childdocforwardprefix[cdocsamp]{cdocsfn}{cdocsch}
%    \end{macrocode}

%\iffalse
%</samplefinal>
%\fi
%
% %%%%%%%%%%%%%%%%%%%%%%%%%%%%%%%%%%%%%%
% \paragraph{Command Line Processing.}
%
% The following three command lines generate the output files
% |cdocscld|, |cdocscl1| and |cdocscl2|
% which should be identical to
% |cdocsdrf|, |cdocsch1| and |cdocsfn2|, respectively:
% \begin{center}
% \begin{tabular}{l}
% |latex -jobname cdocscld \|\\
% |  "\def\version{draft}\input{childdoc.def}\childdocforward{cdocsamp}"|\\
% |latex -jobname cdocscl1 \|\\
% |  "\input{childdoc.def}\childdocforward[cdocsamp]{cdocsch1}"|\\
% |latex -jobname cdocscl2 \|\\
% |  "\def\version{final}\input{childdoc.def}\childdocforward{cdocsch2}"|
% \end{tabular}
% \end{center}
% Note that the trailing backslash on each first line
% merely continues the input to the second line
% (for convenient cut ant paste).
% Furthermore, the command |latex| can be replaced by any
% of its alternative versions such as |pdflatex|.
%
% %%%%%%%%%%%%%%%%%%%%%%%%%%%%%%%%%%%%%%%%%%%%%%%%%%%%%%%%%%%%%%%%%%%%%%%%%%%%%%
% %%%%%%%%%%%%%%%%%%%%%%%%%%%%%%%%%%%%%%%%%%%%%%%%%%%%%%%%%%%%%%%%%%%%%%%%%%%%%%
% \section{Implementation}
%\iffalse
%<*package>
%\fi
%
% This section describes the definitions file |childdoc.def|.

% The definitions cannot be loaded using |\usepackage| or |\RequirePackage|
% which has a mechanism to prevent loading a style file more than once.
% When loading the definitions by means of |\input|
% multiple instances have to be prevented manually:
%\iffalse
%This code needs to be before the `\ProvidesFile' directive
%which is defined at the beginning of this file.
%Therefore it is also placed there and commented out here.
%</package>
%<*discard>
%\fi
%    \begin{macrocode}
\ifdefined\childdocmain\endinput\fi
%    \end{macrocode}
%\iffalse
%</discard>
%<*package>
%\fi
%
% \macro{\ifchilddoc}
% \macro{\ifchilddocmanual}
% The conditional |\ifchilddoc| tells whether a
% child (true) or main (false) document is being compiled.
% The conditional |\ifchilddocmanual| tells whether
% the |\includeonly| mechanism is used (false) or
% the selection of child files must be performed manually (true).
% The definitions initialise to false:
%    \begin{macrocode}
\newif\ifchilddoc
\newif\ifchilddocmanual
%    \end{macrocode}

% \macro{\childdocname}
% \macro{\childdocjob}
% The macro |\childdocname| stores the name of the main document
% to be compiled. The macro |\childdocjob| stores the name of
% the document on which the \LaTeX{} compiler was originally invoked.
% The content of |\jobname| cannot be compared
% to filenames specified in the source due to different catcodes.
% The following code rescans |\jobname|, stores the result
% in |\childdocname| and saves a copy in |\childdocjob|:
%    \begin{macrocode}
\edef\childdocname{\scantokens\expandafter{\jobname\noexpand}}
\let\childdocjob\childdocname
%    \end{macrocode}

% \macro{\childdocdisable}
% The macro |\childdocdisable| prevents the main file
% from being processed more than once.
% At this stage, the main document command |\childdocmain|
% is assumed to be called once again where it should do nothing.
% Any subsequent call to it should prevent
% a secondary processing of the main document
% It overwrites the forwarding commands
% |\childdocof| and |\childdocforward|
% with empty macros to prevent further inclusions of the main document:
%    \begin{macrocode}
\newcommand{\childdocdisable}
{
  \renewcommand{\childdocmain}[1]{\renewcommand{\childdocmain}[1]{\endinput}}
  \renewcommand{\childdocof}[1]{}
  \renewcommand{\childdocby}[2][]{}
  \renewcommand{\childdocforward}[2][]{}
  \renewcommand{\childdocdisable}{}
}
%    \end{macrocode}

% \macro{\childdocmain}
% The macro |\childdocmain| is to be called at the top of the main file
% with nothing or the main filename (without extension) as argument.
% First, it breaks loops.
% If the argument is not empty and does not match |\childdocname|
% (which is set by the first inclusion of |childdoc.def|),
% |\ifchilddoc| is set to true, |\includeonly| is applied to the child file
% and |\jobname| is set to the main file
% (for proper handling of |.aux| files):
%    \begin{macrocode}
\newcommand{\childdocmain}[1]
{
  \childdocdisable\childdocmain{}
  \if?#1?\else
    \begingroup
      \def\childdoctmp{#1}
      \ifx\childdoctmp\childdocname
        \def\childdoctmp{}
      \else
        \def\childdoctmp
        {
          \childdoctrue
          \includeonly{\childdocname}
          \def\childdocjob{#1}
          \def\jobname{#1}
        }
      \fi
      \expandafter
    \endgroup
    \childdoctmp
  \fi
}
%    \end{macrocode}

% \macro{\childdocof}
% The command |\childdocof| redirects
% compilation to the main file |#1|.
%    \begin{macrocode}
\newcommand{\childdocof}[1]
{
  \childdocdisable
  \childdoctrue
  \includeonly{\childdocname}
  \def\jobname{#1}
  \def\childdocjob{#1}
  \input{#1}
}
%    \end{macrocode}

% \macro{\childdocby}
% The command |\childdocby| ....
%    \begin{macrocode}
\newcommand{\childdocby}[2][]
{
  \childdocdisable
  \childdoctrue
  \childdocmanualtrue
  \if?#1?\else
    \def\jobname{#2}
  \fi
  \def\childdocjob{#2}
  \input{#2}
  \endinput
}
%    \end{macrocode}

% \macro{\childdocforward}
% The command |\childdocforward| redirects
% compilation to the main file or
% (if the optional argument is given) a child file.
% Parameters are set as if the main file
% or a child file starting with |\childdocof| was compiled.
% Then compilation is handed over to the main file:
%    \begin{macrocode}
\newcommand{\childdocforward}[2][]
{
  \begingroup
    \if?#1?
      \def\childdoctmp
      {
        \def\childdocname{#2}
        \def\childdocjob{#2}
        \def\jobname{#2}
        \input{#2}
        \endinput
      }
    \else
      \def\childdoctmp
      {
        \childdocdisable
        \def\childdocname{#2}
        \childdoctrue
        \includeonly{#2}
        \def\childdocjob{#1}
        \def\jobname{#1}
        \input{#1}
        \endinput
      }
    \fi
    \expandafter
  \endgroup
  \childdoctmp
}
%    \end{macrocode}

% \macro{\childdocforwardprefix}
% The command |\childdocforwardprefix| redirects
% compilation to the main or a child file by means of a pattern.
% The prefix |#1| in the current filename is replaced by |#2|
% and the suffix of the current filename is kept
% (it is assumed that the filename does not contain the substring `|~~~|'
% which is used as a delimiter).
% Compilation is handed over to the new file by |\childdocforward|:
%    \begin{macrocode}
\newcommand{\childdocforwardprefix}[3][]
{
  \begingroup
    \def\childdocextract #2##1~~~{\def\childdoctmp{\childdocforward[#1]{#3##1}}}
    \expandafter\childdocextract\childdocname~~~
    \expandafter
  \endgroup
  \childdoctmp
}
%    \end{macrocode}

% \macro{\childdoc}
% The deprecated macro |\childdoc| is a legacy version of |\childdocmain|:
%    \begin{macrocode}
\newcommand{\childdoc}{\childdocmain}
%    \end{macrocode}

% \macro{\childdocredirect}
% The deprecated macro |\childdocredirect| is a legacy version
% of |\childdocforward| and |\childdocforwardprefix|:
%    \begin{macrocode}
\newcommand{\childdocredirect}[2][]
{
  \begingroup
    \if?#1?
      \def\childdoctmp{\childdocforward{#2}}
    \else
      \def\childdoctmp{\childdocforwardprefix{#1}{#2}}
    \fi
    \expandafter
  \endgroup
  \childdoctmp
}
%    \end{macrocode}

%\iffalse
%</package>
%\fi
%
\endinput
\childdocforward[cdocsamp]{cdocsch1}"|\\
% |latex -jobname cdocscl2 \|\\
% |  "\def\version{final}% \iffalse
%
% childdoc.dtx Copyright (C) 2017-2018 Niklas Beisert
%
% This work may be distributed and/or modified under the
% conditions of the LaTeX Project Public License, either version 1.3
% of this license or (at your option) any later version.
% The latest version of this license is in
%   http://www.latex-project.org/lppl.txt
% and version 1.3 or later is part of all distributions of LaTeX
% version 2005/12/01 or later.
%
% This work has the LPPL maintenance status `maintained'.
%
% The Current Maintainer of this work is Niklas Beisert.
%
% This work consists of the files childdoc.dtx and childdoc.ins
% and the derived files childdoc.def and cdocsamp.tex with
% cdocsch1.tex, cdocsch2.tex, cdocsdrf.tex, cdocsfn1.tex, cdocsfn2.tex.
%
%<package>\ifdefined\childdocmain\endinput\fi
%<package>\ProvidesFile{childdoc.def}[2018/12/30 v2.0 child document driver]
%<samplemain>\ProvidesFile{cdocsamp.tex}[2018/12/30 v2.0 sample for childdoc]
%<*driver>
%\ProvidesFile{childdoc.drv}[2018/12/30 v2.0 childdoc reference manual file]
\PassOptionsToClass{10pt,a4paper}{article}
\documentclass{ltxdoc}

\usepackage[margin=35mm]{geometry}
\usepackage{hyperref}
\usepackage{hyperxmp}
\usepackage[usenames]{color}

\hypersetup{colorlinks=true}
\hypersetup{pdfstartview=FitH}
\hypersetup{pdfpagemode=UseNone}
\hypersetup{pdfsource={}}
\hypersetup{pdflang={en-UK}}
\hypersetup{pdfcopyright={Copyright 2017-2018 Niklas Beisert.
  This work may be distributed and/or modified under the
  conditions of the LaTeX Project Public License, either version 1.3
  of this license or (at your option) any later version.}}
\hypersetup{pdflicenseurl={http://www.latex-project.org/lppl.txt}}
\hypersetup{pdfcontactaddress={ETH Zurich, ITP, HIT K,
  Wolfgang-Pauli-Strasse 27}}
\hypersetup{pdfcontactpostcode={8093}}
\hypersetup{pdfcontactcity={Zurich}}
\hypersetup{pdfcontactcountry={Switzerland}}
\hypersetup{pdfcontactemail={nbeisert@itp.phys.ethz.ch}}
\hypersetup{pdfcontacturl={http://people.phys.ethz.ch/\xmptilde nbeisert/}}

\newcommand{\secref}[1]{\hyperref[#1]{section \ref*{#1}}}

\parskip1ex
\parindent0pt
\let\olditemize\itemize
\def\itemize{\olditemize\parskip0pt}

\begin{document}

\title{The \textsf{childdoc} Package}
\hypersetup{pdftitle={The childdoc Package}}
\author{Niklas Beisert\\[2ex]
  Institut f\"ur Theoretische Physik\\
  Eidgen\"ossische Technische Hochschule Z\"urich\\
  Wolfgang-Pauli-Strasse 27, 8093 Z\"urich, Switzerland\\[1ex]
  \href{mailto:nbeisert@itp.phys.ethz.ch}
  {\texttt{nbeisert@itp.phys.ethz.ch}}}
\hypersetup{pdfauthor={Niklas Beisert}}
\hypersetup{pdfsubject={Manual for the LaTeX2e Package childdoc}}
\date{30 December 2018, \textsf{v2.0}}
\maketitle

\begin{abstract}\noindent
\textsf{childdoc} is a \LaTeXe{} package
that enables the direct compilation
of document sections included by |\include|
to individual files.
\end{abstract}

\begingroup
\parskip0ex
\tableofcontents
\endgroup

%%%%%%%%%%%%%%%%%%%%%%%%%%%%%%%%%%%%%%%%%%%%%%%%%%%%%%%%%%%%%%%%%%%%%%%%%%%%%%%%
%%%%%%%%%%%%%%%%%%%%%%%%%%%%%%%%%%%%%%%%%%%%%%%%%%%%%%%%%%%%%%%%%%%%%%%%%%%%%%%%
\section{Introduction}

\LaTeX{} provides a mechanism to structure a large document (such as a book)
into a main file and several child files (containing the chapters)
using the |\include| command.
This mechanism is beneficial for documents
which span hundreds of pages in order to
make the source file(s) more manageable.
Moreover, compilation can be restricted to
selected child files by means of the |\includeonly| command.
The latter feature can be used to reduce the compilation time while editing
(this was significantly more useful in the earlier days of \LaTeX{})
or to generate a smaller document which is easier to navigate.
Another application of |\includeonly| is to generate
documents consisting of selected parts of the complete document.

However, there are a few drawbacks of the plain |\include| mechanism:
\begin{itemize}
\item
The child files cannot be compiled on their own,
they can only be compiled via the main file.
A naive editing environment
(such as a text editor with an option
to have the current file processed by \LaTeX)
may require one to switch to the main file before compiling;
attempting to compile the child file produces errors.
\item
The main file must be modified (each time)
to adjust the |\includeonly| command
to the present needs. This easily leaves the main file in a messy state.
\item
The generated document will always carry the filename
of the main document. This is inconvenient if
several child files are to be compiled and
to be kept for distribution.
\end{itemize}

The present package provides a simple interface
to make child files individually compilable by \LaTeX{}.
Compiling a child file then has the same effect as compiling
the main file with an |\includeonly| command
to select the appropriate child.
Moreover the generated document will carry the name of the child
rather than the main file.
This resolves all three above issues.

This feature is meant to make the editing of books,
thesis documents and lecture notes somewhat more convenient.
However, the package can also be used efficiently for
composing a series of documents (such as exercise sheets)
which are typically distributed individually.
It then assists the author in generating the individual documents
(potentially in different versions)
as well as a document containing the collected series.
Another application is in developing style files
or other kinds of included material
where compilation of the style file could redirect
to a sample or test file.

%%%%%%%%%%%%%%%%%%%%%%%%%%%%%%%%%%%%%%%%%%%%%%%%%%%%%%%%%%%%%%%%%%%%%%%%%%%%%%%%
%%%%%%%%%%%%%%%%%%%%%%%%%%%%%%%%%%%%%%%%%%%%%%%%%%%%%%%%%%%%%%%%%%%%%%%%%%%%%%%%
\section{Usage}

First of all, the package \textsf{childdoc} is \emph{not} a standard
\LaTeXe{} |.sty| style file! Therefore it needs to be invoked in
a non-standard way.

%%%%%%%%%%%%%%%%%%%%%%%%%%%%%%%%%%%%%%%%%%%%%%%%%%%%%%%%%%%%%%%%%%%%%%%%%%%%%%%%
\subsection{Included Files}
\label{sec:include}

%%%%%%%%%%%%%%%%%%%%%%%%%%%%%%%%%%%%%%%%
\DescribeMacro{\childdocmain}
To use the package, add the commands
\begin{center}
\begin{tabular}{l}
|\input{childdoc.def}|\\
|\childdocmain{}|\\
\end{tabular}
\end{center}
at the very top of the main \LaTeX{} file,
in particular \emph{before} the |\documentclass| statement!
The argument of |\childdocmain| should be left empty
(but it must be present).

%%%%%%%%%%%%%%%%%%%%%%%%%%%%%%%%%%%%%%%%
\DescribeMacro{\childdocof}
Furthermore, add the commands
\begin{center}
\begin{tabular}{l}
|\input{childdoc.def}|\\
|\childdocof{|\textit{main}|}|\\
\end{tabular}
\end{center}
at the top of every child file \textit{child}
which is included by |\include{|\textit{child}|}|
from within the main file
(or at least for those files to be compiled individually).
The argument \textit{main} must be the filename of the main file.

There are a couple of
considerations in setting up the main and child documents:

%%%%%%%%%%%%%%%%%%%%%%%%%%%%%%%%%%%%%%%%
\paragraph{Restrictions.}

Please note the following restrictions:
\begin{itemize}
\item
|\childdocmain| must be called with one argument \textit{main}
to ensure compatibility with earlier version of the package.
It must either be empty (|\childdocmain{}|)
or precisely match the filename of the main file in which it is specified.
See \secref{sec:detection} for further information.
\item
The filename \textit{main} must be specified without the |.tex| extension.
\item
The filename \textit{main} is case sensitive
(even in case-insensitive file systems)
due to internal string comparison.
\item
The argument \textit{main} should be fully expanded, it cannot be a macro.
\item
Subdirectories and special characters should be avoided in filenames.
\item
The command |\childdocmain{|\textit{main}|}| must be followed by a whitespace.
It should not be followed immediately by another command
or by a comment mark `|%|'.
This is because the \TeX{} parser reads the token immediately following
the argument of |\childdocmain| and puts it
at the beginning of every child section;
however, a white\-space is ignored.
\end{itemize}

%%%%%%%%%%%%%%%%%%%%%%%%%%%%%%%%%%%%%%%%
\paragraph{Content of Main File.}

It is advisable to place all content in the child files included by |\include|.
Any output contained in the main file will appear in all child documents
unless suppressed manually;
it cannot be suppressed automatically by the |\includeonly| directive
and thus should normally be avoided.
A method to include some content in the main file
by means of conditional processing is described in \secref{sec:conditional}.

%%%%%%%%%%%%%%%%%%%%%%%%%%%%%%%%%%%%%%%%
\paragraph{Page Numbering.}

When only a part of the document is compiled,
the appropriate numbering of pages
(as well as other status parameters)
is determined from the |.aux| files.
The latter contain information from previous passes.
However this information needs to propagate through
all intermediate child documents.
Therefore the page numbering in child documents may well
be inconsistent until the complete document is compiled at least once.

A useful (if unconventional) way to always ensure a consistent
page numbering is to restart the numbering in each child document
and denote the pages by `\textit{child}|.|\textit{page}'
where \textit{child} represents the chapter/section number of the child file.
This can be achieved by the command
|\numberwithin{page}{|\textit{child}|}|
of the \textsf{amsmath} package
where \textit{child} can be |chapter| or |section|
depending on the chosen structuring.
Alternatively, one can modify the macro |\thepage| appropriately
and reset the counter |page| at the start of each child file.

%%%%%%%%%%%%%%%%%%%%%%%%%%%%%%%%%%%%%%%%%%%%%%%%%%%%%%%%%%%%%%%%%%%%%%%%%%%%%%%%
\subsection{Conditional Processing}
\label{sec:conditional}

The package provides a mechanism to compile different versions
of a document. To customise the versions further some conditional processing
can come in handy to distinguish which version is being compiled.
The package provides two macros to describe the compilation context:

%%%%%%%%%%%%%%%%%%%%%%%%%%%%%%%%%%%%%%%%
\DescribeMacro{\ifchilddoc}
The conditional |\ifchilddoc| distinguishes between the compilation of
child documents and the main document:
%
\begin{center}
|\ifchilddoc |\textit{child-code}| |[|\||else |\textit{main-code}]| \||fi|
\end{center}

%%%%%%%%%%%%%%%%%%%%%%%%%%%%%%%%%%%%%%%%
\DescribeMacro{\childdocname}
\DescribeMacro{\childdocjob}
The macro |\childdocname| contains the filename (without extension)
of the main or child file being processed.
Note that |\childdocjob| will always contain the name of the main file.

%%%%%%%%%%%%%%%%%%%%%%%%%%%%%%%%%%%%%%%%
\paragraph{Title Page.}

Conditional processing can be used to include a title or banner page
in the main document when proper precautions are taken.
Importantly, the code in the main file should ensure that the page counter
(as well as other status parameters which are stored in the |.aux| files)
takes the same value after the conditional processing.
Otherwise the page numbers may take divergent values
depending on which part is compiled.

For example, a title page could be declared by:
%
\begin{center}
\begin{tabular}{l}
|\ifchilddoc\||else|\\
|\addtocounter{page}{-1}|\\
\textit{code for title page}\\
|\newpage|\\
|\||fi|
\end{tabular}
\end{center}
%
A banner page for the child documents can be generated by:
%
\begin{center}
\begin{tabular}{l}
|\ifchilddoc|\\
|\addtocounter{page}{-1}|\\
\textit{code for banner page}\\
|\newpage|\\
|\||fi|
\end{tabular}
\end{center}
%
Here one could write a message such as:
\begin{center}
|This is the part \childdocname{} of \childdocjob{}.|
\end{center}

%%%%%%%%%%%%%%%%%%%%%%%%%%%%%%%%%%%%%%%%%%%%%%%%%%%%%%%%%%%%%%%%%%%%%%%%%%%%%%%%
\subsection{Flags}
\label{sec:flags}

The package makes it easy to generate different versions
of the main or child documents.
To this end compilation flags can be defined
and assigned different default values.
They will be particularly useful in conjunction
with the forwarding mechanism described in \secref{sec:forward}.

For example, it may be useful to have a flag |\version|
which can be set to |draft| or |final|.
The document source will contain some conditional code
depending on the value of |\version|.
Suppose further, the flag should default to |final| for the main file
and to |draft| for child files
which is a natural assignment for editing the document.
This is achieved by placing the following code
in the preamble of the main document
(below the |\childdocmain| directive):
%
\begin{center}
\begin{tabular}{l}
|\ifchilddoc|\\
|\providecommand{\version}{draft}|\\
|\||else|\\
|\providecommand{\version}{final}|\\
|\||fi|
\end{tabular}
\end{center}
%
The definition by |\providecommand| makes sure
that previous definitions are not overwritten.
Further statements |\providecommand{\version}{...}|
can thus be added before the above code to override it.

For the main file, one might add a line
(between |\childdocmain| and the above block)
%
\begin{center}
|%\ifchilddoc\||else\providecommand{\version}{draft}\||fi|
\end{center}
%
which can be uncommented to produce a draft version.
Likewise one can add a line to the very top of a child file
(above the |\childdocof{|\textit{main}|}| directive)
%
\begin{center}
|%\providecommand{\version}{final}|
\end{center}
%
which can be uncommented to produce the final version of this child document.

%%%%%%%%%%%%%%%%%%%%%%%%%%%%%%%%%%%%%%%%%%%%%%%%%%%%%%%%%%%%%%%%%%%%%%%%%%%%%%%%
\subsection{Forwarding}
\label{sec:forward}

Different versions of the main or child documents
using compilation flags as described in \secref{sec:flags}
can be (permanently) stored in different files
for convenient compilation, viewing and distribution.
To this end, the package defines a command
to pass on compilation to a different file:

%%%%%%%%%%%%%%%%%%%%%%%%%%%%%%%%%%%%%%%%
\DescribeMacro{\childdocforward}
The command |\childdocforward| redirects processing to
another source file:
%
\begin{center}
\begin{tabular}{l}
|\input{childdoc.def}|\\
|\childdocforward[|\textit{main}|]{|\textit{dest}|}|\\
\end{tabular}
\end{center}
%
The argument \textit{dest} is the destination file
(without extension).
It should be the main file or one of the child files.
Note that further \textsf{childdoc} directives
such as |\childdocof| and |\childdocforward|
in the indicated file will be processed in this form.
The optional argument \textit{main}
passes on directly to the main file \textit{main}
while pretending to compile the child \textit{dest}.
This form behaves as if \textit{dest}
issues |\childdocof{|\textit{main}|}| right away,
and no further \textsf{childdoc} directives will be processed.

%%%%%%%%%%%%%%%%%%%%%%%%%%%%%%%%%%%%%%%%
\DescribeMacro{\...prefix}
In the alternative form |\childdocforwardprefix|,
%
\begin{center}
\begin{tabular}{l}
|\input{childdoc.def}|\\
|\childdocforwardprefix[|\textit{main}|]{|\textit{prefix}|}{|\textit{dest}|}|
\end{tabular}
\end{center}
%
the destination file is determined by a pattern
depending on the current file:
To make this work, the current file must be called
`{\textit{prefix}\hspace{0.2em}\textit{suffix}}'
with \textit{prefix} matching precisely the argument.
Processing is then passed on to the file
`{\textit{dest}\hspace{0.2em}\textit{suffix}}'.
Surely, the same effect is achieved by
directly specifying the
argument `{\textit{dest}\hspace{0.2em}\textit{suffix}}'
in the first form.
However, that requires to set up a different file
for each child. With the alternative form of the command
all these files can have exactly the same content
which simplifies setting them up and maintaining them.

For example, the following file |draft.tex|
with a compilation flag |\version| as described in \secref{sec:flags}
compiles the main document as a draft:
%
\begin{center}
\begin{tabular}{l}
|\def\version{draft}|\\
|\input{childdoc.def}|\\
|\childdocforward{|\textit{main}|}|
\end{tabular}
\end{center}
%
Likewise, the following files |final|\textit{nn}|.tex|
compile the final version of the child document
|child|\textit{nn}|.tex|:
%
\begin{center}
\begin{tabular}{l}
|\def\version{final}|\\
|\input{childdoc.def}|\\
|\childdocforwardprefix{final}{child}|
\end{tabular}
\end{center}
%

Note that when several versions of a main file and/or of each child file
are to be generated, it may be convenient to set up a |Makefile| or
shell script to automatise the process.

%%%%%%%%%%%%%%%%%%%%%%%%%%%%%%%%%%%%%%%%%%%%%%%%%%%%%%%%%%%%%%%%%%%%%%%%%%%%%%%%
\subsection{Command Line Processing}
\label{sec:commandline}

The effect of redirection files can also be achieved by invoking
the \LaTeX{} compiler with a more elaborate command line.
Most conveniently this should be done as part
of a shell script or a |Makefile|.

When using \textsf{childdoc} in the main file, the following
command lines effectively perform a redirection
(note that depending on the shell being used,
backslashes may have to be doubled: `|\|' $\to$ `|\\|'):
%
\begin{center}
|... -jobname "|\textit{target}|" |\\|"|[\textit{flags}]%
|\input{childdoc.def}\childdocforward[|\textit{main}|]{|\textit{dest}|}"|
\end{center}
%
Here \textit{target} is the name of the output file,
\textit{main} is the name of the main file
and \textit{dest} is the name of the main or child file to be processed
(all filenames without extensions).
The optional argument \textit{main} can be omitted
if \textit{main} matches \textit{dest}.
Optionally, compilation \textit{flags} can be defined via |\def| commands.
This command line makes the \TeX{} engine believe
it is compiling the file \textit{target}
whose content is specified as the latter parameter.
The provided code then forwards the processing to
\textit{main} or \textit{dest} as described in \secref{sec:forward}.

%%%%%%%%%%%%%%%%%%%%%%%%%%%%%%%%%%%%%%%%%%%%%%%%%%%%%%%%%%%%%%%%%%%%%%%%%%%%%%%%
\subsection{Include by Input}
\label{sec:input}

Including child documents by |\include| has some restrictions by design.
Most notably, the content of a child document always occupies
its own set of pages; pages cannot be shared between child documents.
Usually, this behaviour makes perfect sense
because each child document contain an essential part of the document.
However, in some situations it may be desirable to compose
a document from a collection of parts
without having mandatory page breaks between then.
For this case, the package
provides a mechanism to include parts
by |\input| which can also be processed individually.
However, by construction this mechanism
requires manual handling of the content to be output.

%%%%%%%%%%%%%%%%%%%%%%%%%%%%%%%%%%%%%%%%
\DescribeMacro{\ifchilddocmanual}
The main file should be prepared as usual, see \secref{sec:include}.
However, the document body must make a distinction
between processing of an individual part and of the main document, e.g.:
%
\begin{center}
\begin{tabular}{l}
|\ifchilddocmanual|\\
|\input{\childdocname}|\\
|\||else|\\
\textit{document body with }|\input{|\textit{part}|}|\\
|\||fi|
\end{tabular}
\end{center}
%
The conditional |\ifchilddocmanual| is true whenever
a part to be included by |\input| is being compiled,
and the name of the part is stored in |\childdocname|.

%%%%%%%%%%%%%%%%%%%%%%%%%%%%%%%%%%%%%%%%
\DescribeMacro{\childdocby}
Each part to be included by |\input| should start with:
%
\begin{center}
\begin{tabular}{l}
|\input{childdoc.def}|\\
|\childdocby{|\textit{main}|}|\\
\end{tabular}
\end{center}
%
The directive |\childdocby| is similar to |\childdocof|
described in \secref{sec:include},
but the subsequent selection of content must be done manually.
To that end, both |\ifchilddoc| and |\ifchilddocmanual|
will be true upon processing of a part,
and the name of the part is stored in |\childdocname|.
Note that |\jobname| will be set to the filename of the current part
so that each part receives an individual |.aux| file
that does not interfere with the |.aux| file(s) of the main document.
This behaviour can be altered by the alternative form
|\childdocby[*]{|\textit{main}|}| (with a non-empty optional argument)
which uses the |.aux| file of the main document
by setting |\jobname| to \textit{main}.

%%%%%%%%%%%%%%%%%%%%%%%%%%%%%%%%%%%%%%%%%%%%%%%%%%%%%%%%%%%%%%%%%%%%%%%%%%%%%%%%
\subsection{Driver Development}
\label{sec:driver}

The \textsf{childdoc} mechanism can also be use for the development
of definition files such as \LaTeX{} styles or classes.
This case differs from the above setup with multiple parts
included by |\include| in that no |\includeonly| should be invoked.
This can be achieved by starting the include file
(before |\ProvidesPackage|) with:
%
\begin{center}
\begin{tabular}{l}
|\input{childdoc.def}|\\
|\childdocforward{|\textit{main}|}|\\
\end{tabular}
\end{center}
%
or alternatively with:
%
\begin{center}
\begin{tabular}{l}
|\input{childdoc.def}|\\
|\childdocby{|\textit{main}|}|\\
\end{tabular}
\end{center}
%
Both forms have slightly different effects as described above.
The main file is prepared as usual, see \secref{sec:include}.

%%%%%%%%%%%%%%%%%%%%%%%%%%%%%%%%%%%%%%%%%%%%%%%%%%%%%%%%%%%%%%%%%%%%%%%%%%%%%%%%
\subsection{Legacy Detection}
\label{sec:detection}

The directive |\childdocmain| in the main file can detect
whether the complete document or merely a child is to be compiled
even without using the directive |\childdocof|.
This method is deprecated because it is less robust
and there is no compelling reason to use it;
it is merely provided for backward compatibility
and it may be removed in future versions.

If the detection mechanism is to be used,
it is mandatory to correctly specify
the filename of the main file as the argument of |\childdocmain|:
%
\begin{center}
\begin{tabular}{l}
|\input{childdoc.def}|\\
|\childdocmain{|\textit{main}|}|\\
\end{tabular}
\end{center}
%
If |\jobname| does not match the argument \textit{main} of |\childdocmain|,
it is assumed that |\jobname| points to the child file to be compiled.
When using |\childdocmain| with the main file specified as argument,
it suffices to start a child file
with just |\input{|\textit{main}|}|
without loading of the package and using |\childdocof|.
If instead all processing is done
with the appropriate \textsf{childdoc} directives,
the argument of \textit{main} of |\childdocmain| can be empty.

An alternative version of the command line processing described
in \secref{sec:commandline} using the detection mechanism reads:
%
\begin{center}
|... -jobname "|\textit{target}|" "|[\textit{flags}]%
[|\def\jobname{|\textit{dest}|}|]|\input{|\textit{main}|}"|
\end{center}

%%%%%%%%%%%%%%%%%%%%%%%%%%%%%%%%%%%%%%%%%%%%%%%%%%%%%%%%%%%%%%%%%%%%%%%%%%%%%%%%
\subsection{Manual Code}
\label{sec:manual}

In case one cannot be certain whether the definitions file |childdoc.def|
is installed on the target \TeX{} distribution
and one prefers not to ship it,
it is conceivable to paste a few relevant commands into the sources.

To that end, drop all statements |\input{childdoc.def}|
and perform the replacements as outlined below.
Instead of |\childdocmain{|\textit{main}|}| add the following code
to the top of the main file:
%
\begin{center}
\begin{tabular}{l}
|\||ifdefined\childdocname\endinput\||fi\newif\ifchilddoc|\\
|\edef\childdocname{\scantokens\expandafter{\jobname\noexpand}}|\\
|\def\childdocmain{|\textit{main}|}\||ifx\childdocmain\childdocname\||else|\\
|\childdoctrue\includeonly{\childdocname}\let\jobname\childdocmain\||fi|\\
\end{tabular}
\end{center}
%
Instead of |\childdocof{|\textit{main}|}| just include the main file
at the top of each child file:
%
\begin{center}
|\input{|\textit{main}|}|
\end{center}
%
A simple redirection |\childdocforward{|\textit{dest}|}| is achieved by:
%
\begin{center}
|\def\jobname{|\textit{dest}|}\input{\jobname}|
\end{center}
%
The redirection with prefix
|\childdocforwardprefix[|\textit{prefix}|]{|\textit{dest}|}|
is accomplished by:
%
\begin{center}
\begin{tabular}{l}
|{\edef\jobname{\scantokens\expandafter{\jobname\noexpand}}|\\
|\def\redirectjob |\textit{prefix}|#1~~~{\gdef\jobname{|\textit{dest}|#1}}|\\
|\expandafter\redirectjob\jobname~~~}\input{\jobname}|
\end{tabular}
\end{center}

In an alternative approach,
child documents can be compiled by a specific command line
without additional code or specific definitions:
%
\begin{center}
|... -jobname "|\textit{target}|" "|[\textit{flags}]%
|\includeonly{|\textit{dest}|}\input{|\textit{main}|}"|
\end{center}
%

%%%%%%%%%%%%%%%%%%%%%%%%%%%%%%%%%%%%%%%%%%%%%%%%%%%%%%%%%%%%%%%%%%%%%%%%%%%%%%%%
%%%%%%%%%%%%%%%%%%%%%%%%%%%%%%%%%%%%%%%%%%%%%%%%%%%%%%%%%%%%%%%%%%%%%%%%%%%%%%%%
\section{Information}

%%%%%%%%%%%%%%%%%%%%%%%%%%%%%%%%%%%%%%%%%%%%%%%%%%%%%%%%%%%%%%%%%%%%%%%%%%%%%%%%
\subsection{Copyright}

Copyright \copyright{} 2017--2018 Niklas Beisert

This work may be distributed and/or modified under the
conditions of the \LaTeX{} Project Public License, either version 1.3
of this license or (at your option) any later version.
The latest version of this license is in
  \url{http://www.latex-project.org/lppl.txt}
and version 1.3 or later is part of all distributions of \LaTeX{}
version 2005/12/01 or later.

This work has the LPPL maintenance status `maintained'.

The Current Maintainer of this work is Niklas Beisert.

This work consists of the files |README.txt|, |childdoc.ins| and |childdoc.dtx|
as well as the derived files |childdoc.def|, |cdocsamp.tex|
with |cdocsch1.tex|, |cdocsch2.tex|, |cdocspt3.tex|, |cdocspt4.tex|,
|cdocsdrf.tex|, |cdocsfn1.tex|, |cdocsfn2.tex|
as well as |childdoc.pdf|.

%%%%%%%%%%%%%%%%%%%%%%%%%%%%%%%%%%%%%%%%%%%%%%%%%%%%%%%%%%%%%%%%%%%%%%%%%%%%%%%%
\subsection{Files and Installation}

The package consists of the files:
%
\begin{center}
\begin{tabular}{ll}
    |README.txt|   & readme file \\
    |childdoc.ins| & installation file \\
    |childdoc.dtx| & source file \\
    |childdoc.def| & definition file \\
    |cdocsamp.tex| & sample main file \\
    |cdocsch1.tex| & sample include file \\
    |cdocsch2.tex| & sample include file \\
    |cdocspt3.tex| & sample part file \\
    |cdocspt4.tex| & sample part file \\
    |cdocsdrf.tex| & sample redirection file \\
    |cdocsfn1.tex| & sample redirection file \\
    |cdocsfn2.tex| & sample redirection file \\
    |childdoc.pdf| & manual
\end{tabular}
\end{center}
%
The distribution consists of the files
|README.txt|, |childdoc.ins| and |childdoc.dtx|.
%
\begin{itemize}
\item
Run (pdf)\LaTeX{} on |childdoc.dtx|
to compile the manual |childdoc.pdf| (this file).
\item
Run \LaTeX{} on |childdoc.ins| to create the definitions file |childdoc.def|
and the sample |cdocsamp.tex| with include files
|cdocsch1.tex|, |cdocsch2.tex|, |cdocspt3.tex|, |cdocspt4.tex|,
|cdocsdrf.tex|, |cdocsfn1.tex|, |cdocsfn2.tex|.
Then copy the file |childdoc.def| to an appropriate directory of your \LaTeX{}
distribution, e.g.\ \textit{texmf-root}|/tex/latex/childdoc|.
\end{itemize}

%%%%%%%%%%%%%%%%%%%%%%%%%%%%%%%%%%%%%%%%%%%%%%%%%%%%%%%%%%%%%%%%%%%%%%%%%%%%%%%%
\subsection{Related CTAN Packages}

There are several other packages which offer a similar functionality:
%
\begin{itemize}
\item
The packages
\href{http://ctan.org/pkg/docmute}{\textsf{docmute}},
\href{http://ctan.org/pkg/includex}{\textsf{includex}} and
\href{http://ctan.org/pkg/standalone}{\textsf{standalone}}
provide commands to include only the document body of
a child file thus allowing both files to be compiled individually.
\item
The packages \href{http://ctan.org/pkg/subdocs}{\textsf{subdocs}}
and \href{http://ctan.org/pkg/subfiles}{\textsf{subfiles}}
provide structures in which the main and child documents can be
encapsulated and allowing them to be compiled individually.
The inclusion mechanism is different from the conventional |\include|.
\item
The package \href{http://ctan.org/pkg/combine}{\textsf{combine}}
is an elaborate solution to combine several documents into one.
\end{itemize}
%
See also the CTAN topic \href{http://ctan.org/topic/subdocs}{\textsf{subdocs}}
for further related packages.
The present package differs from the above solutions in that
a document structure constructed with the conventional |\include| mechanism
just needs two extra commands at the top of every file
such that all constituent files can be compiled individually.

%%%%%%%%%%%%%%%%%%%%%%%%%%%%%%%%%%%%%%%%%%%%%%%%%%%%%%%%%%%%%%%%%%%%%%%%%%%%%%%%
%\subsection{Feature Suggestions}
%
%The following is a list of features which may be useful for future
%versions of this package:
%%
%\begin{itemize}
%\item
%\ldots
%\end{itemize}

%%%%%%%%%%%%%%%%%%%%%%%%%%%%%%%%%%%%%%%%%%%%%%%%%%%%%%%%%%%%%%%%%%%%%%%%%%%%%%%%
\subsection{Revision History}

%%%%%%%%%%%%%%%%%%%%%%%%%%%%%%%%%%%%%%%%
\paragraph{v2.0:} 2018/12/30

\begin{itemize}
\item
immediate forward processing
\item
added |\childdocby| mechanism
\item
manual restructured
\end{itemize}

%%%%%%%%%%%%%%%%%%%%%%%%%%%%%%%%%%%%%%%%
\paragraph{v1.6:} 2018/01/17

\begin{itemize}
\item
application for development of include files
\item
corrections to manual
\end{itemize}

%%%%%%%%%%%%%%%%%%%%%%%%%%%%%%%%%%%%%%%%
\paragraph{v1.5:} 2017/05/21

\begin{itemize}
\item
more complete structuring introduced
\item
|\childdocof| introduced
\item
|\childdoc| renamed to |\childdocmain|
\item
|\childredirect| renamed to |\childdocforward| and |\childdocforwardprefix|
and functionality expanded
\end{itemize}

%%%%%%%%%%%%%%%%%%%%%%%%%%%%%%%%%%%%%%%%
\paragraph{v1.0:} 2017/04/27

\begin{itemize}
\item
manual and install package
\item
first version published on CTAN
\end{itemize}

%%%%%%%%%%%%%%%%%%%%%%%%%%%%%%%%%%%%%%%%
\paragraph{v0.6:} 2017/04/26

\begin{itemize}
\item
redirection mechanism added
\end{itemize}

%%%%%%%%%%%%%%%%%%%%%%%%%%%%%%%%%%%%%%%%
\paragraph{v0.5:} 2017/04/26

\begin{itemize}
\item
functionality in definition file
\end{itemize}


%%%%%%%%%%%%%%%%%%%%%%%%%%%%%%%%%%%%%%%%%%%%%%%%%%%%%%%%%%%%%%%%%%%%%%%%%%%%%%%%
%%%%%%%%%%%%%%%%%%%%%%%%%%%%%%%%%%%%%%%%%%%%%%%%%%%%%%%%%%%%%%%%%%%%%%%%%%%%%%%%
%%%%%%%%%%%%%%%%%%%%%%%%%%%%%%%%%%%%%%%%%%%%%%%%%%%%%%%%%%%%%%%%%%%%%%%%%%%%%%%%
\appendix

\settowidth\MacroIndent{\rmfamily\scriptsize 000\ }

 \DocInput{childdoc.dtx}

\end{document}
%</driver>
% \fi
%
% %%%%%%%%%%%%%%%%%%%%%%%%%%%%%%%%%%%%%%%%%%%%%%%%%%%%%%%%%%%%%%%%%%%%%%%%%%%%%%
% %%%%%%%%%%%%%%%%%%%%%%%%%%%%%%%%%%%%%%%%%%%%%%%%%%%%%%%%%%%%%%%%%%%%%%%%%%%%%%
% \section{Sample}
%\iffalse
%<*samplemain>
%\fi
%
% The following presents a sample document
% with two chapters, two parts, a title page,
% a compile flag as well as three forwarding files to set the flag.
% It consists of eight |.tex| files:
% \begin{center}
% \begin{tabular}{ll}
% |cdocsamp.tex|&main file\\
% |cdocsch1.tex|&include file for chapter 1\\
% |cdocsch2.tex|&include file for chapter 2\\
% |cdocspt3.tex|&include file for part 3\\
% |cdocspt4.tex|&include file for part 4\\
% |cdocsdrf.tex|&forwarding file for main file in draft mode\\
% |cdocsfi1.tex|&forwarding file for final version of chapter 1\\
% |cdocsfi2.tex|&forwarding file for final version of chapter 2\\
% \end{tabular}
% \end{center}
% Each of the eight files can be compiled directly by the \LaTeX{} compiler.
%
% %%%%%%%%%%%%%%%%%%%%%%%%%%%%%%%%%%%%%%
% \paragraph{Main File.}
%
% The main file is called |cdocsamp.tex|.
%
% Load the \textsf{childdoc} definitions and
% declare the filename for the main document:
%    \begin{macrocode}
\input{childdoc.def}
\childdocmain{}
%    \end{macrocode}

% Optional override for |\version| flag:
%    \begin{macrocode}
%%\ifchilddoc\else\providecommand{\version}{draft}\fi
%    \end{macrocode}

% Define the default values for the |\version| flag
% (|final| for the main file and |draft| for childs):
%    \begin{macrocode}
\ifchilddoc
\providecommand{\version}{draft}
\else
\providecommand{\version}{final}
\fi
%    \end{macrocode}

% Load the standard document class:
%    \begin{macrocode}
\documentclass[12pt]{article}
%    \end{macrocode}

% Start the document body:
%    \begin{macrocode}
\begin{document}
%    \end{macrocode}

% Declare a title page.
% Print title, part of document being processed and version flag:
%    \begin{macrocode}
\addtocounter{page}{-1}
\begin{center}
{\LARGE\bfseries{}childdoc example\par}
\vspace{1cm}
\ifchilddoc
\ifchilddocmanual part\else chapter\fi:
`\childdocname' of `\childdocjob'\par
\else
main document: `\childdocjob'\par
\fi
version: \version\par
\end{center}
\newpage
%    \end{macrocode}

% Manually include selected file,
% otherwise process as usual:
%    \begin{macrocode}
\ifchilddocmanual
\section*{part `\childdocname'}
\input{\childdocname}
\else
%    \end{macrocode}

% Include the two chapters:
%    \begin{macrocode}
\include{cdocsch1}
\include{cdocsch2}
%    \end{macrocode}

% Include the two parts unless only chapters should be displayed:
%    \begin{macrocode}
\ifchilddoc\else
\section{part three}
\input{cdocspt3}
\section{part four}
\input{cdocspt4}
\fi
%    \end{macrocode}

% Process as usual until here:
%    \begin{macrocode}
\fi
%    \end{macrocode}

% End of document body:
%    \begin{macrocode}
\end{document}
%    \end{macrocode}
%\iffalse
%</samplemain>
%\fi
%
% %%%%%%%%%%%%%%%%%%%%%%%%%%%%%%%%%%%%%%
% \paragraph{Chapter Include Files.}
%
% The include files are called |cdocsch1.tex| and |cdocsch2.tex|.
%
%\iffalse
%<*samplechap1|samplechap2>
%\fi

% Optional override for |\version| flag:
%    \begin{macrocode}
%%\providecommand{\version}{final}
%    \end{macrocode}

% Include the main document:
%    \begin{macrocode}
\input{childdoc.def}
\childdocof{cdocsamp}
%    \end{macrocode}

%\iffalse
%</samplechap1|samplechap2>
%\fi
%
%\iffalse
%<*samplechap1>
%\fi
% Some text for chapter 1:
%    \begin{macrocode}
\section{one}
some text in chapter one
%    \end{macrocode}

%\iffalse
%</samplechap1>
%\fi
% Some text for chapter 2:
%\iffalse
%<*samplechap2>
%\fi
%    \begin{macrocode}
\section{two}
more text in chapter two
%    \end{macrocode}

%\iffalse
%</samplechap2>
%\fi
%
% %%%%%%%%%%%%%%%%%%%%%%%%%%%%%%%%%%%%%%
% \paragraph{Part Include Files.}
%
% The include files are called |cdocspt3.tex| and |cdocspt4.tex|.
%
%\iffalse
%<*samplepart3|samplepart4>
%\fi

% Optional override for |\version| flag:
%    \begin{macrocode}
%%\providecommand{\version}{final}
%    \end{macrocode}

% Include the main document:
%    \begin{macrocode}
\input{childdoc.def}
\childdocby{cdocsamp}
%    \end{macrocode}

%\iffalse
%</samplepart3|samplepart4>
%\fi
%
%\iffalse
%<*samplepart3>
%\fi
% Some text for part 3:
%    \begin{macrocode}
some text in part three
%    \end{macrocode}

%\iffalse
%</samplepart3>
%\fi
% Some text for part 4:
%\iffalse
%<*samplepart4>
%\fi
%    \begin{macrocode}
more text in part four
%    \end{macrocode}

%\iffalse
%</samplepart4>
%\fi
%
% %%%%%%%%%%%%%%%%%%%%%%%%%%%%%%%%%%%%%%
% \paragraph{Forwarding for a Complete Draft.}
%
% The following forwarding file |cdocsdrf.tex|
% compiles the main document in draft mode:
%\iffalse
%<*sampledraft>
%\fi
%    \begin{macrocode}
\def\version{draft}
\input{childdoc.def}
\childdocforward{cdocsamp}
%    \end{macrocode}

%\iffalse
%</sampledraft>
%\fi
%
% %%%%%%%%%%%%%%%%%%%%%%%%%%%%%%%%%%%%%%
% \paragraph{Forwarding for Final Version of the Chapters.}
%
% The following forwarding files |cdocsfn1.tex| and |cdocsfn2.tex|
% (with identical content)
% compile the final versions of the child documents
% |cdocsch1.tex| and |cdocsch2.tex|, respectively:
%\iffalse
%<*samplefinal>
%\fi
%    \begin{macrocode}
\def\version{final}
\input{childdoc.def}
\childdocforwardprefix[cdocsamp]{cdocsfn}{cdocsch}
%    \end{macrocode}

%\iffalse
%</samplefinal>
%\fi
%
% %%%%%%%%%%%%%%%%%%%%%%%%%%%%%%%%%%%%%%
% \paragraph{Command Line Processing.}
%
% The following three command lines generate the output files
% |cdocscld|, |cdocscl1| and |cdocscl2|
% which should be identical to
% |cdocsdrf|, |cdocsch1| and |cdocsfn2|, respectively:
% \begin{center}
% \begin{tabular}{l}
% |latex -jobname cdocscld \|\\
% |  "\def\version{draft}\input{childdoc.def}\childdocforward{cdocsamp}"|\\
% |latex -jobname cdocscl1 \|\\
% |  "\input{childdoc.def}\childdocforward[cdocsamp]{cdocsch1}"|\\
% |latex -jobname cdocscl2 \|\\
% |  "\def\version{final}\input{childdoc.def}\childdocforward{cdocsch2}"|
% \end{tabular}
% \end{center}
% Note that the trailing backslash on each first line
% merely continues the input to the second line
% (for convenient cut ant paste).
% Furthermore, the command |latex| can be replaced by any
% of its alternative versions such as |pdflatex|.
%
% %%%%%%%%%%%%%%%%%%%%%%%%%%%%%%%%%%%%%%%%%%%%%%%%%%%%%%%%%%%%%%%%%%%%%%%%%%%%%%
% %%%%%%%%%%%%%%%%%%%%%%%%%%%%%%%%%%%%%%%%%%%%%%%%%%%%%%%%%%%%%%%%%%%%%%%%%%%%%%
% \section{Implementation}
%\iffalse
%<*package>
%\fi
%
% This section describes the definitions file |childdoc.def|.

% The definitions cannot be loaded using |\usepackage| or |\RequirePackage|
% which has a mechanism to prevent loading a style file more than once.
% When loading the definitions by means of |\input|
% multiple instances have to be prevented manually:
%\iffalse
%This code needs to be before the `\ProvidesFile' directive
%which is defined at the beginning of this file.
%Therefore it is also placed there and commented out here.
%</package>
%<*discard>
%\fi
%    \begin{macrocode}
\ifdefined\childdocmain\endinput\fi
%    \end{macrocode}
%\iffalse
%</discard>
%<*package>
%\fi
%
% \macro{\ifchilddoc}
% \macro{\ifchilddocmanual}
% The conditional |\ifchilddoc| tells whether a
% child (true) or main (false) document is being compiled.
% The conditional |\ifchilddocmanual| tells whether
% the |\includeonly| mechanism is used (false) or
% the selection of child files must be performed manually (true).
% The definitions initialise to false:
%    \begin{macrocode}
\newif\ifchilddoc
\newif\ifchilddocmanual
%    \end{macrocode}

% \macro{\childdocname}
% \macro{\childdocjob}
% The macro |\childdocname| stores the name of the main document
% to be compiled. The macro |\childdocjob| stores the name of
% the document on which the \LaTeX{} compiler was originally invoked.
% The content of |\jobname| cannot be compared
% to filenames specified in the source due to different catcodes.
% The following code rescans |\jobname|, stores the result
% in |\childdocname| and saves a copy in |\childdocjob|:
%    \begin{macrocode}
\edef\childdocname{\scantokens\expandafter{\jobname\noexpand}}
\let\childdocjob\childdocname
%    \end{macrocode}

% \macro{\childdocdisable}
% The macro |\childdocdisable| prevents the main file
% from being processed more than once.
% At this stage, the main document command |\childdocmain|
% is assumed to be called once again where it should do nothing.
% Any subsequent call to it should prevent
% a secondary processing of the main document
% It overwrites the forwarding commands
% |\childdocof| and |\childdocforward|
% with empty macros to prevent further inclusions of the main document:
%    \begin{macrocode}
\newcommand{\childdocdisable}
{
  \renewcommand{\childdocmain}[1]{\renewcommand{\childdocmain}[1]{\endinput}}
  \renewcommand{\childdocof}[1]{}
  \renewcommand{\childdocby}[2][]{}
  \renewcommand{\childdocforward}[2][]{}
  \renewcommand{\childdocdisable}{}
}
%    \end{macrocode}

% \macro{\childdocmain}
% The macro |\childdocmain| is to be called at the top of the main file
% with nothing or the main filename (without extension) as argument.
% First, it breaks loops.
% If the argument is not empty and does not match |\childdocname|
% (which is set by the first inclusion of |childdoc.def|),
% |\ifchilddoc| is set to true, |\includeonly| is applied to the child file
% and |\jobname| is set to the main file
% (for proper handling of |.aux| files):
%    \begin{macrocode}
\newcommand{\childdocmain}[1]
{
  \childdocdisable\childdocmain{}
  \if?#1?\else
    \begingroup
      \def\childdoctmp{#1}
      \ifx\childdoctmp\childdocname
        \def\childdoctmp{}
      \else
        \def\childdoctmp
        {
          \childdoctrue
          \includeonly{\childdocname}
          \def\childdocjob{#1}
          \def\jobname{#1}
        }
      \fi
      \expandafter
    \endgroup
    \childdoctmp
  \fi
}
%    \end{macrocode}

% \macro{\childdocof}
% The command |\childdocof| redirects
% compilation to the main file |#1|.
%    \begin{macrocode}
\newcommand{\childdocof}[1]
{
  \childdocdisable
  \childdoctrue
  \includeonly{\childdocname}
  \def\jobname{#1}
  \def\childdocjob{#1}
  \input{#1}
}
%    \end{macrocode}

% \macro{\childdocby}
% The command |\childdocby| ....
%    \begin{macrocode}
\newcommand{\childdocby}[2][]
{
  \childdocdisable
  \childdoctrue
  \childdocmanualtrue
  \if?#1?\else
    \def\jobname{#2}
  \fi
  \def\childdocjob{#2}
  \input{#2}
  \endinput
}
%    \end{macrocode}

% \macro{\childdocforward}
% The command |\childdocforward| redirects
% compilation to the main file or
% (if the optional argument is given) a child file.
% Parameters are set as if the main file
% or a child file starting with |\childdocof| was compiled.
% Then compilation is handed over to the main file:
%    \begin{macrocode}
\newcommand{\childdocforward}[2][]
{
  \begingroup
    \if?#1?
      \def\childdoctmp
      {
        \def\childdocname{#2}
        \def\childdocjob{#2}
        \def\jobname{#2}
        \input{#2}
        \endinput
      }
    \else
      \def\childdoctmp
      {
        \childdocdisable
        \def\childdocname{#2}
        \childdoctrue
        \includeonly{#2}
        \def\childdocjob{#1}
        \def\jobname{#1}
        \input{#1}
        \endinput
      }
    \fi
    \expandafter
  \endgroup
  \childdoctmp
}
%    \end{macrocode}

% \macro{\childdocforwardprefix}
% The command |\childdocforwardprefix| redirects
% compilation to the main or a child file by means of a pattern.
% The prefix |#1| in the current filename is replaced by |#2|
% and the suffix of the current filename is kept
% (it is assumed that the filename does not contain the substring `|~~~|'
% which is used as a delimiter).
% Compilation is handed over to the new file by |\childdocforward|:
%    \begin{macrocode}
\newcommand{\childdocforwardprefix}[3][]
{
  \begingroup
    \def\childdocextract #2##1~~~{\def\childdoctmp{\childdocforward[#1]{#3##1}}}
    \expandafter\childdocextract\childdocname~~~
    \expandafter
  \endgroup
  \childdoctmp
}
%    \end{macrocode}

% \macro{\childdoc}
% The deprecated macro |\childdoc| is a legacy version of |\childdocmain|:
%    \begin{macrocode}
\newcommand{\childdoc}{\childdocmain}
%    \end{macrocode}

% \macro{\childdocredirect}
% The deprecated macro |\childdocredirect| is a legacy version
% of |\childdocforward| and |\childdocforwardprefix|:
%    \begin{macrocode}
\newcommand{\childdocredirect}[2][]
{
  \begingroup
    \if?#1?
      \def\childdoctmp{\childdocforward{#2}}
    \else
      \def\childdoctmp{\childdocforwardprefix{#1}{#2}}
    \fi
    \expandafter
  \endgroup
  \childdoctmp
}
%    \end{macrocode}

%\iffalse
%</package>
%\fi
%
\endinput
\childdocforward{cdocsch2}"|
% \end{tabular}
% \end{center}
% Note that the trailing backslash on each first line
% merely continues the input to the second line
% (for convenient cut ant paste).
% Furthermore, the command |latex| can be replaced by any
% of its alternative versions such as |pdflatex|.
%
% %%%%%%%%%%%%%%%%%%%%%%%%%%%%%%%%%%%%%%%%%%%%%%%%%%%%%%%%%%%%%%%%%%%%%%%%%%%%%%
% %%%%%%%%%%%%%%%%%%%%%%%%%%%%%%%%%%%%%%%%%%%%%%%%%%%%%%%%%%%%%%%%%%%%%%%%%%%%%%
% \section{Implementation}
%\iffalse
%<*package>
%\fi
%
% This section describes the definitions file |childdoc.def|.

% The definitions cannot be loaded using |\usepackage| or |\RequirePackage|
% which has a mechanism to prevent loading a style file more than once.
% When loading the definitions by means of |\input|
% multiple instances have to be prevented manually:
%\iffalse
%This code needs to be before the `\ProvidesFile' directive
%which is defined at the beginning of this file.
%Therefore it is also placed there and commented out here.
%</package>
%<*discard>
%\fi
%    \begin{macrocode}
\ifdefined\childdocmain\endinput\fi
%    \end{macrocode}
%\iffalse
%</discard>
%<*package>
%\fi
%
% \macro{\ifchilddoc}
% \macro{\ifchilddocmanual}
% The conditional |\ifchilddoc| tells whether a
% child (true) or main (false) document is being compiled.
% The conditional |\ifchilddocmanual| tells whether
% the |\includeonly| mechanism is used (false) or
% the selection of child files must be performed manually (true).
% The definitions initialise to false:
%    \begin{macrocode}
\newif\ifchilddoc
\newif\ifchilddocmanual
%    \end{macrocode}

% \macro{\childdocname}
% \macro{\childdocjob}
% The macro |\childdocname| stores the name of the main document
% to be compiled. The macro |\childdocjob| stores the name of
% the document on which the \LaTeX{} compiler was originally invoked.
% The content of |\jobname| cannot be compared
% to filenames specified in the source due to different catcodes.
% The following code rescans |\jobname|, stores the result
% in |\childdocname| and saves a copy in |\childdocjob|:
%    \begin{macrocode}
\edef\childdocname{\scantokens\expandafter{\jobname\noexpand}}
\let\childdocjob\childdocname
%    \end{macrocode}

% \macro{\childdocdisable}
% The macro |\childdocdisable| prevents the main file
% from being processed more than once.
% At this stage, the main document command |\childdocmain|
% is assumed to be called once again where it should do nothing.
% Any subsequent call to it should prevent
% a secondary processing of the main document
% It overwrites the forwarding commands
% |\childdocof| and |\childdocforward|
% with empty macros to prevent further inclusions of the main document:
%    \begin{macrocode}
\newcommand{\childdocdisable}
{
  \renewcommand{\childdocmain}[1]{\renewcommand{\childdocmain}[1]{\endinput}}
  \renewcommand{\childdocof}[1]{}
  \renewcommand{\childdocby}[2][]{}
  \renewcommand{\childdocforward}[2][]{}
  \renewcommand{\childdocdisable}{}
}
%    \end{macrocode}

% \macro{\childdocmain}
% The macro |\childdocmain| is to be called at the top of the main file
% with nothing or the main filename (without extension) as argument.
% First, it breaks loops.
% If the argument is not empty and does not match |\childdocname|
% (which is set by the first inclusion of |childdoc.def|),
% |\ifchilddoc| is set to true, |\includeonly| is applied to the child file
% and |\jobname| is set to the main file
% (for proper handling of |.aux| files):
%    \begin{macrocode}
\newcommand{\childdocmain}[1]
{
  \childdocdisable\childdocmain{}
  \if?#1?\else
    \begingroup
      \def\childdoctmp{#1}
      \ifx\childdoctmp\childdocname
        \def\childdoctmp{}
      \else
        \def\childdoctmp
        {
          \childdoctrue
          \includeonly{\childdocname}
          \def\childdocjob{#1}
          \def\jobname{#1}
        }
      \fi
      \expandafter
    \endgroup
    \childdoctmp
  \fi
}
%    \end{macrocode}

% \macro{\childdocof}
% The command |\childdocof| redirects
% compilation to the main file |#1|.
%    \begin{macrocode}
\newcommand{\childdocof}[1]
{
  \childdocdisable
  \childdoctrue
  \includeonly{\childdocname}
  \def\jobname{#1}
  \def\childdocjob{#1}
  \input{#1}
}
%    \end{macrocode}

% \macro{\childdocby}
% The command |\childdocby| ....
%    \begin{macrocode}
\newcommand{\childdocby}[2][]
{
  \childdocdisable
  \childdoctrue
  \childdocmanualtrue
  \if?#1?\else
    \def\jobname{#2}
  \fi
  \def\childdocjob{#2}
  \input{#2}
  \endinput
}
%    \end{macrocode}

% \macro{\childdocforward}
% The command |\childdocforward| redirects
% compilation to the main file or
% (if the optional argument is given) a child file.
% Parameters are set as if the main file
% or a child file starting with |\childdocof| was compiled.
% Then compilation is handed over to the main file:
%    \begin{macrocode}
\newcommand{\childdocforward}[2][]
{
  \begingroup
    \if?#1?
      \def\childdoctmp
      {
        \def\childdocname{#2}
        \def\childdocjob{#2}
        \def\jobname{#2}
        \input{#2}
        \endinput
      }
    \else
      \def\childdoctmp
      {
        \childdocdisable
        \def\childdocname{#2}
        \childdoctrue
        \includeonly{#2}
        \def\childdocjob{#1}
        \def\jobname{#1}
        \input{#1}
        \endinput
      }
    \fi
    \expandafter
  \endgroup
  \childdoctmp
}
%    \end{macrocode}

% \macro{\childdocforwardprefix}
% The command |\childdocforwardprefix| redirects
% compilation to the main or a child file by means of a pattern.
% The prefix |#1| in the current filename is replaced by |#2|
% and the suffix of the current filename is kept
% (it is assumed that the filename does not contain the substring `|~~~|'
% which is used as a delimiter).
% Compilation is handed over to the new file by |\childdocforward|:
%    \begin{macrocode}
\newcommand{\childdocforwardprefix}[3][]
{
  \begingroup
    \def\childdocextract #2##1~~~{\def\childdoctmp{\childdocforward[#1]{#3##1}}}
    \expandafter\childdocextract\childdocname~~~
    \expandafter
  \endgroup
  \childdoctmp
}
%    \end{macrocode}

% \macro{\childdoc}
% The deprecated macro |\childdoc| is a legacy version of |\childdocmain|:
%    \begin{macrocode}
\newcommand{\childdoc}{\childdocmain}
%    \end{macrocode}

% \macro{\childdocredirect}
% The deprecated macro |\childdocredirect| is a legacy version
% of |\childdocforward| and |\childdocforwardprefix|:
%    \begin{macrocode}
\newcommand{\childdocredirect}[2][]
{
  \begingroup
    \if?#1?
      \def\childdoctmp{\childdocforward{#2}}
    \else
      \def\childdoctmp{\childdocforwardprefix{#1}{#2}}
    \fi
    \expandafter
  \endgroup
  \childdoctmp
}
%    \end{macrocode}

%\iffalse
%</package>
%\fi
%
\endinput
|\\
|\childdocforward{|\textit{main}|}|\\
\end{tabular}
\end{center}
%
or alternatively with:
%
\begin{center}
\begin{tabular}{l}
|% \iffalse
%
% childdoc.dtx Copyright (C) 2017-2018 Niklas Beisert
%
% This work may be distributed and/or modified under the
% conditions of the LaTeX Project Public License, either version 1.3
% of this license or (at your option) any later version.
% The latest version of this license is in
%   http://www.latex-project.org/lppl.txt
% and version 1.3 or later is part of all distributions of LaTeX
% version 2005/12/01 or later.
%
% This work has the LPPL maintenance status `maintained'.
%
% The Current Maintainer of this work is Niklas Beisert.
%
% This work consists of the files childdoc.dtx and childdoc.ins
% and the derived files childdoc.def and cdocsamp.tex with
% cdocsch1.tex, cdocsch2.tex, cdocsdrf.tex, cdocsfn1.tex, cdocsfn2.tex.
%
%<package>\ifdefined\childdocmain\endinput\fi
%<package>\ProvidesFile{childdoc.def}[2018/12/30 v2.0 child document driver]
%<samplemain>\ProvidesFile{cdocsamp.tex}[2018/12/30 v2.0 sample for childdoc]
%<*driver>
%\ProvidesFile{childdoc.drv}[2018/12/30 v2.0 childdoc reference manual file]
\PassOptionsToClass{10pt,a4paper}{article}
\documentclass{ltxdoc}

\usepackage[margin=35mm]{geometry}
\usepackage{hyperref}
\usepackage{hyperxmp}
\usepackage[usenames]{color}

\hypersetup{colorlinks=true}
\hypersetup{pdfstartview=FitH}
\hypersetup{pdfpagemode=UseNone}
\hypersetup{pdfsource={}}
\hypersetup{pdflang={en-UK}}
\hypersetup{pdfcopyright={Copyright 2017-2018 Niklas Beisert.
  This work may be distributed and/or modified under the
  conditions of the LaTeX Project Public License, either version 1.3
  of this license or (at your option) any later version.}}
\hypersetup{pdflicenseurl={http://www.latex-project.org/lppl.txt}}
\hypersetup{pdfcontactaddress={ETH Zurich, ITP, HIT K,
  Wolfgang-Pauli-Strasse 27}}
\hypersetup{pdfcontactpostcode={8093}}
\hypersetup{pdfcontactcity={Zurich}}
\hypersetup{pdfcontactcountry={Switzerland}}
\hypersetup{pdfcontactemail={nbeisert@itp.phys.ethz.ch}}
\hypersetup{pdfcontacturl={http://people.phys.ethz.ch/\xmptilde nbeisert/}}

\newcommand{\secref}[1]{\hyperref[#1]{section \ref*{#1}}}

\parskip1ex
\parindent0pt
\let\olditemize\itemize
\def\itemize{\olditemize\parskip0pt}

\begin{document}

\title{The \textsf{childdoc} Package}
\hypersetup{pdftitle={The childdoc Package}}
\author{Niklas Beisert\\[2ex]
  Institut f\"ur Theoretische Physik\\
  Eidgen\"ossische Technische Hochschule Z\"urich\\
  Wolfgang-Pauli-Strasse 27, 8093 Z\"urich, Switzerland\\[1ex]
  \href{mailto:nbeisert@itp.phys.ethz.ch}
  {\texttt{nbeisert@itp.phys.ethz.ch}}}
\hypersetup{pdfauthor={Niklas Beisert}}
\hypersetup{pdfsubject={Manual for the LaTeX2e Package childdoc}}
\date{30 December 2018, \textsf{v2.0}}
\maketitle

\begin{abstract}\noindent
\textsf{childdoc} is a \LaTeXe{} package
that enables the direct compilation
of document sections included by |\include|
to individual files.
\end{abstract}

\begingroup
\parskip0ex
\tableofcontents
\endgroup

%%%%%%%%%%%%%%%%%%%%%%%%%%%%%%%%%%%%%%%%%%%%%%%%%%%%%%%%%%%%%%%%%%%%%%%%%%%%%%%%
%%%%%%%%%%%%%%%%%%%%%%%%%%%%%%%%%%%%%%%%%%%%%%%%%%%%%%%%%%%%%%%%%%%%%%%%%%%%%%%%
\section{Introduction}

\LaTeX{} provides a mechanism to structure a large document (such as a book)
into a main file and several child files (containing the chapters)
using the |\include| command.
This mechanism is beneficial for documents
which span hundreds of pages in order to
make the source file(s) more manageable.
Moreover, compilation can be restricted to
selected child files by means of the |\includeonly| command.
The latter feature can be used to reduce the compilation time while editing
(this was significantly more useful in the earlier days of \LaTeX{})
or to generate a smaller document which is easier to navigate.
Another application of |\includeonly| is to generate
documents consisting of selected parts of the complete document.

However, there are a few drawbacks of the plain |\include| mechanism:
\begin{itemize}
\item
The child files cannot be compiled on their own,
they can only be compiled via the main file.
A naive editing environment
(such as a text editor with an option
to have the current file processed by \LaTeX)
may require one to switch to the main file before compiling;
attempting to compile the child file produces errors.
\item
The main file must be modified (each time)
to adjust the |\includeonly| command
to the present needs. This easily leaves the main file in a messy state.
\item
The generated document will always carry the filename
of the main document. This is inconvenient if
several child files are to be compiled and
to be kept for distribution.
\end{itemize}

The present package provides a simple interface
to make child files individually compilable by \LaTeX{}.
Compiling a child file then has the same effect as compiling
the main file with an |\includeonly| command
to select the appropriate child.
Moreover the generated document will carry the name of the child
rather than the main file.
This resolves all three above issues.

This feature is meant to make the editing of books,
thesis documents and lecture notes somewhat more convenient.
However, the package can also be used efficiently for
composing a series of documents (such as exercise sheets)
which are typically distributed individually.
It then assists the author in generating the individual documents
(potentially in different versions)
as well as a document containing the collected series.
Another application is in developing style files
or other kinds of included material
where compilation of the style file could redirect
to a sample or test file.

%%%%%%%%%%%%%%%%%%%%%%%%%%%%%%%%%%%%%%%%%%%%%%%%%%%%%%%%%%%%%%%%%%%%%%%%%%%%%%%%
%%%%%%%%%%%%%%%%%%%%%%%%%%%%%%%%%%%%%%%%%%%%%%%%%%%%%%%%%%%%%%%%%%%%%%%%%%%%%%%%
\section{Usage}

First of all, the package \textsf{childdoc} is \emph{not} a standard
\LaTeXe{} |.sty| style file! Therefore it needs to be invoked in
a non-standard way.

%%%%%%%%%%%%%%%%%%%%%%%%%%%%%%%%%%%%%%%%%%%%%%%%%%%%%%%%%%%%%%%%%%%%%%%%%%%%%%%%
\subsection{Included Files}
\label{sec:include}

%%%%%%%%%%%%%%%%%%%%%%%%%%%%%%%%%%%%%%%%
\DescribeMacro{\childdocmain}
To use the package, add the commands
\begin{center}
\begin{tabular}{l}
|% \iffalse
%
% childdoc.dtx Copyright (C) 2017-2018 Niklas Beisert
%
% This work may be distributed and/or modified under the
% conditions of the LaTeX Project Public License, either version 1.3
% of this license or (at your option) any later version.
% The latest version of this license is in
%   http://www.latex-project.org/lppl.txt
% and version 1.3 or later is part of all distributions of LaTeX
% version 2005/12/01 or later.
%
% This work has the LPPL maintenance status `maintained'.
%
% The Current Maintainer of this work is Niklas Beisert.
%
% This work consists of the files childdoc.dtx and childdoc.ins
% and the derived files childdoc.def and cdocsamp.tex with
% cdocsch1.tex, cdocsch2.tex, cdocsdrf.tex, cdocsfn1.tex, cdocsfn2.tex.
%
%<package>\ifdefined\childdocmain\endinput\fi
%<package>\ProvidesFile{childdoc.def}[2018/12/30 v2.0 child document driver]
%<samplemain>\ProvidesFile{cdocsamp.tex}[2018/12/30 v2.0 sample for childdoc]
%<*driver>
%\ProvidesFile{childdoc.drv}[2018/12/30 v2.0 childdoc reference manual file]
\PassOptionsToClass{10pt,a4paper}{article}
\documentclass{ltxdoc}

\usepackage[margin=35mm]{geometry}
\usepackage{hyperref}
\usepackage{hyperxmp}
\usepackage[usenames]{color}

\hypersetup{colorlinks=true}
\hypersetup{pdfstartview=FitH}
\hypersetup{pdfpagemode=UseNone}
\hypersetup{pdfsource={}}
\hypersetup{pdflang={en-UK}}
\hypersetup{pdfcopyright={Copyright 2017-2018 Niklas Beisert.
  This work may be distributed and/or modified under the
  conditions of the LaTeX Project Public License, either version 1.3
  of this license or (at your option) any later version.}}
\hypersetup{pdflicenseurl={http://www.latex-project.org/lppl.txt}}
\hypersetup{pdfcontactaddress={ETH Zurich, ITP, HIT K,
  Wolfgang-Pauli-Strasse 27}}
\hypersetup{pdfcontactpostcode={8093}}
\hypersetup{pdfcontactcity={Zurich}}
\hypersetup{pdfcontactcountry={Switzerland}}
\hypersetup{pdfcontactemail={nbeisert@itp.phys.ethz.ch}}
\hypersetup{pdfcontacturl={http://people.phys.ethz.ch/\xmptilde nbeisert/}}

\newcommand{\secref}[1]{\hyperref[#1]{section \ref*{#1}}}

\parskip1ex
\parindent0pt
\let\olditemize\itemize
\def\itemize{\olditemize\parskip0pt}

\begin{document}

\title{The \textsf{childdoc} Package}
\hypersetup{pdftitle={The childdoc Package}}
\author{Niklas Beisert\\[2ex]
  Institut f\"ur Theoretische Physik\\
  Eidgen\"ossische Technische Hochschule Z\"urich\\
  Wolfgang-Pauli-Strasse 27, 8093 Z\"urich, Switzerland\\[1ex]
  \href{mailto:nbeisert@itp.phys.ethz.ch}
  {\texttt{nbeisert@itp.phys.ethz.ch}}}
\hypersetup{pdfauthor={Niklas Beisert}}
\hypersetup{pdfsubject={Manual for the LaTeX2e Package childdoc}}
\date{30 December 2018, \textsf{v2.0}}
\maketitle

\begin{abstract}\noindent
\textsf{childdoc} is a \LaTeXe{} package
that enables the direct compilation
of document sections included by |\include|
to individual files.
\end{abstract}

\begingroup
\parskip0ex
\tableofcontents
\endgroup

%%%%%%%%%%%%%%%%%%%%%%%%%%%%%%%%%%%%%%%%%%%%%%%%%%%%%%%%%%%%%%%%%%%%%%%%%%%%%%%%
%%%%%%%%%%%%%%%%%%%%%%%%%%%%%%%%%%%%%%%%%%%%%%%%%%%%%%%%%%%%%%%%%%%%%%%%%%%%%%%%
\section{Introduction}

\LaTeX{} provides a mechanism to structure a large document (such as a book)
into a main file and several child files (containing the chapters)
using the |\include| command.
This mechanism is beneficial for documents
which span hundreds of pages in order to
make the source file(s) more manageable.
Moreover, compilation can be restricted to
selected child files by means of the |\includeonly| command.
The latter feature can be used to reduce the compilation time while editing
(this was significantly more useful in the earlier days of \LaTeX{})
or to generate a smaller document which is easier to navigate.
Another application of |\includeonly| is to generate
documents consisting of selected parts of the complete document.

However, there are a few drawbacks of the plain |\include| mechanism:
\begin{itemize}
\item
The child files cannot be compiled on their own,
they can only be compiled via the main file.
A naive editing environment
(such as a text editor with an option
to have the current file processed by \LaTeX)
may require one to switch to the main file before compiling;
attempting to compile the child file produces errors.
\item
The main file must be modified (each time)
to adjust the |\includeonly| command
to the present needs. This easily leaves the main file in a messy state.
\item
The generated document will always carry the filename
of the main document. This is inconvenient if
several child files are to be compiled and
to be kept for distribution.
\end{itemize}

The present package provides a simple interface
to make child files individually compilable by \LaTeX{}.
Compiling a child file then has the same effect as compiling
the main file with an |\includeonly| command
to select the appropriate child.
Moreover the generated document will carry the name of the child
rather than the main file.
This resolves all three above issues.

This feature is meant to make the editing of books,
thesis documents and lecture notes somewhat more convenient.
However, the package can also be used efficiently for
composing a series of documents (such as exercise sheets)
which are typically distributed individually.
It then assists the author in generating the individual documents
(potentially in different versions)
as well as a document containing the collected series.
Another application is in developing style files
or other kinds of included material
where compilation of the style file could redirect
to a sample or test file.

%%%%%%%%%%%%%%%%%%%%%%%%%%%%%%%%%%%%%%%%%%%%%%%%%%%%%%%%%%%%%%%%%%%%%%%%%%%%%%%%
%%%%%%%%%%%%%%%%%%%%%%%%%%%%%%%%%%%%%%%%%%%%%%%%%%%%%%%%%%%%%%%%%%%%%%%%%%%%%%%%
\section{Usage}

First of all, the package \textsf{childdoc} is \emph{not} a standard
\LaTeXe{} |.sty| style file! Therefore it needs to be invoked in
a non-standard way.

%%%%%%%%%%%%%%%%%%%%%%%%%%%%%%%%%%%%%%%%%%%%%%%%%%%%%%%%%%%%%%%%%%%%%%%%%%%%%%%%
\subsection{Included Files}
\label{sec:include}

%%%%%%%%%%%%%%%%%%%%%%%%%%%%%%%%%%%%%%%%
\DescribeMacro{\childdocmain}
To use the package, add the commands
\begin{center}
\begin{tabular}{l}
|\input{childdoc.def}|\\
|\childdocmain{}|\\
\end{tabular}
\end{center}
at the very top of the main \LaTeX{} file,
in particular \emph{before} the |\documentclass| statement!
The argument of |\childdocmain| should be left empty
(but it must be present).

%%%%%%%%%%%%%%%%%%%%%%%%%%%%%%%%%%%%%%%%
\DescribeMacro{\childdocof}
Furthermore, add the commands
\begin{center}
\begin{tabular}{l}
|\input{childdoc.def}|\\
|\childdocof{|\textit{main}|}|\\
\end{tabular}
\end{center}
at the top of every child file \textit{child}
which is included by |\include{|\textit{child}|}|
from within the main file
(or at least for those files to be compiled individually).
The argument \textit{main} must be the filename of the main file.

There are a couple of
considerations in setting up the main and child documents:

%%%%%%%%%%%%%%%%%%%%%%%%%%%%%%%%%%%%%%%%
\paragraph{Restrictions.}

Please note the following restrictions:
\begin{itemize}
\item
|\childdocmain| must be called with one argument \textit{main}
to ensure compatibility with earlier version of the package.
It must either be empty (|\childdocmain{}|)
or precisely match the filename of the main file in which it is specified.
See \secref{sec:detection} for further information.
\item
The filename \textit{main} must be specified without the |.tex| extension.
\item
The filename \textit{main} is case sensitive
(even in case-insensitive file systems)
due to internal string comparison.
\item
The argument \textit{main} should be fully expanded, it cannot be a macro.
\item
Subdirectories and special characters should be avoided in filenames.
\item
The command |\childdocmain{|\textit{main}|}| must be followed by a whitespace.
It should not be followed immediately by another command
or by a comment mark `|%|'.
This is because the \TeX{} parser reads the token immediately following
the argument of |\childdocmain| and puts it
at the beginning of every child section;
however, a white\-space is ignored.
\end{itemize}

%%%%%%%%%%%%%%%%%%%%%%%%%%%%%%%%%%%%%%%%
\paragraph{Content of Main File.}

It is advisable to place all content in the child files included by |\include|.
Any output contained in the main file will appear in all child documents
unless suppressed manually;
it cannot be suppressed automatically by the |\includeonly| directive
and thus should normally be avoided.
A method to include some content in the main file
by means of conditional processing is described in \secref{sec:conditional}.

%%%%%%%%%%%%%%%%%%%%%%%%%%%%%%%%%%%%%%%%
\paragraph{Page Numbering.}

When only a part of the document is compiled,
the appropriate numbering of pages
(as well as other status parameters)
is determined from the |.aux| files.
The latter contain information from previous passes.
However this information needs to propagate through
all intermediate child documents.
Therefore the page numbering in child documents may well
be inconsistent until the complete document is compiled at least once.

A useful (if unconventional) way to always ensure a consistent
page numbering is to restart the numbering in each child document
and denote the pages by `\textit{child}|.|\textit{page}'
where \textit{child} represents the chapter/section number of the child file.
This can be achieved by the command
|\numberwithin{page}{|\textit{child}|}|
of the \textsf{amsmath} package
where \textit{child} can be |chapter| or |section|
depending on the chosen structuring.
Alternatively, one can modify the macro |\thepage| appropriately
and reset the counter |page| at the start of each child file.

%%%%%%%%%%%%%%%%%%%%%%%%%%%%%%%%%%%%%%%%%%%%%%%%%%%%%%%%%%%%%%%%%%%%%%%%%%%%%%%%
\subsection{Conditional Processing}
\label{sec:conditional}

The package provides a mechanism to compile different versions
of a document. To customise the versions further some conditional processing
can come in handy to distinguish which version is being compiled.
The package provides two macros to describe the compilation context:

%%%%%%%%%%%%%%%%%%%%%%%%%%%%%%%%%%%%%%%%
\DescribeMacro{\ifchilddoc}
The conditional |\ifchilddoc| distinguishes between the compilation of
child documents and the main document:
%
\begin{center}
|\ifchilddoc |\textit{child-code}| |[|\||else |\textit{main-code}]| \||fi|
\end{center}

%%%%%%%%%%%%%%%%%%%%%%%%%%%%%%%%%%%%%%%%
\DescribeMacro{\childdocname}
\DescribeMacro{\childdocjob}
The macro |\childdocname| contains the filename (without extension)
of the main or child file being processed.
Note that |\childdocjob| will always contain the name of the main file.

%%%%%%%%%%%%%%%%%%%%%%%%%%%%%%%%%%%%%%%%
\paragraph{Title Page.}

Conditional processing can be used to include a title or banner page
in the main document when proper precautions are taken.
Importantly, the code in the main file should ensure that the page counter
(as well as other status parameters which are stored in the |.aux| files)
takes the same value after the conditional processing.
Otherwise the page numbers may take divergent values
depending on which part is compiled.

For example, a title page could be declared by:
%
\begin{center}
\begin{tabular}{l}
|\ifchilddoc\||else|\\
|\addtocounter{page}{-1}|\\
\textit{code for title page}\\
|\newpage|\\
|\||fi|
\end{tabular}
\end{center}
%
A banner page for the child documents can be generated by:
%
\begin{center}
\begin{tabular}{l}
|\ifchilddoc|\\
|\addtocounter{page}{-1}|\\
\textit{code for banner page}\\
|\newpage|\\
|\||fi|
\end{tabular}
\end{center}
%
Here one could write a message such as:
\begin{center}
|This is the part \childdocname{} of \childdocjob{}.|
\end{center}

%%%%%%%%%%%%%%%%%%%%%%%%%%%%%%%%%%%%%%%%%%%%%%%%%%%%%%%%%%%%%%%%%%%%%%%%%%%%%%%%
\subsection{Flags}
\label{sec:flags}

The package makes it easy to generate different versions
of the main or child documents.
To this end compilation flags can be defined
and assigned different default values.
They will be particularly useful in conjunction
with the forwarding mechanism described in \secref{sec:forward}.

For example, it may be useful to have a flag |\version|
which can be set to |draft| or |final|.
The document source will contain some conditional code
depending on the value of |\version|.
Suppose further, the flag should default to |final| for the main file
and to |draft| for child files
which is a natural assignment for editing the document.
This is achieved by placing the following code
in the preamble of the main document
(below the |\childdocmain| directive):
%
\begin{center}
\begin{tabular}{l}
|\ifchilddoc|\\
|\providecommand{\version}{draft}|\\
|\||else|\\
|\providecommand{\version}{final}|\\
|\||fi|
\end{tabular}
\end{center}
%
The definition by |\providecommand| makes sure
that previous definitions are not overwritten.
Further statements |\providecommand{\version}{...}|
can thus be added before the above code to override it.

For the main file, one might add a line
(between |\childdocmain| and the above block)
%
\begin{center}
|%\ifchilddoc\||else\providecommand{\version}{draft}\||fi|
\end{center}
%
which can be uncommented to produce a draft version.
Likewise one can add a line to the very top of a child file
(above the |\childdocof{|\textit{main}|}| directive)
%
\begin{center}
|%\providecommand{\version}{final}|
\end{center}
%
which can be uncommented to produce the final version of this child document.

%%%%%%%%%%%%%%%%%%%%%%%%%%%%%%%%%%%%%%%%%%%%%%%%%%%%%%%%%%%%%%%%%%%%%%%%%%%%%%%%
\subsection{Forwarding}
\label{sec:forward}

Different versions of the main or child documents
using compilation flags as described in \secref{sec:flags}
can be (permanently) stored in different files
for convenient compilation, viewing and distribution.
To this end, the package defines a command
to pass on compilation to a different file:

%%%%%%%%%%%%%%%%%%%%%%%%%%%%%%%%%%%%%%%%
\DescribeMacro{\childdocforward}
The command |\childdocforward| redirects processing to
another source file:
%
\begin{center}
\begin{tabular}{l}
|\input{childdoc.def}|\\
|\childdocforward[|\textit{main}|]{|\textit{dest}|}|\\
\end{tabular}
\end{center}
%
The argument \textit{dest} is the destination file
(without extension).
It should be the main file or one of the child files.
Note that further \textsf{childdoc} directives
such as |\childdocof| and |\childdocforward|
in the indicated file will be processed in this form.
The optional argument \textit{main}
passes on directly to the main file \textit{main}
while pretending to compile the child \textit{dest}.
This form behaves as if \textit{dest}
issues |\childdocof{|\textit{main}|}| right away,
and no further \textsf{childdoc} directives will be processed.

%%%%%%%%%%%%%%%%%%%%%%%%%%%%%%%%%%%%%%%%
\DescribeMacro{\...prefix}
In the alternative form |\childdocforwardprefix|,
%
\begin{center}
\begin{tabular}{l}
|\input{childdoc.def}|\\
|\childdocforwardprefix[|\textit{main}|]{|\textit{prefix}|}{|\textit{dest}|}|
\end{tabular}
\end{center}
%
the destination file is determined by a pattern
depending on the current file:
To make this work, the current file must be called
`{\textit{prefix}\hspace{0.2em}\textit{suffix}}'
with \textit{prefix} matching precisely the argument.
Processing is then passed on to the file
`{\textit{dest}\hspace{0.2em}\textit{suffix}}'.
Surely, the same effect is achieved by
directly specifying the
argument `{\textit{dest}\hspace{0.2em}\textit{suffix}}'
in the first form.
However, that requires to set up a different file
for each child. With the alternative form of the command
all these files can have exactly the same content
which simplifies setting them up and maintaining them.

For example, the following file |draft.tex|
with a compilation flag |\version| as described in \secref{sec:flags}
compiles the main document as a draft:
%
\begin{center}
\begin{tabular}{l}
|\def\version{draft}|\\
|\input{childdoc.def}|\\
|\childdocforward{|\textit{main}|}|
\end{tabular}
\end{center}
%
Likewise, the following files |final|\textit{nn}|.tex|
compile the final version of the child document
|child|\textit{nn}|.tex|:
%
\begin{center}
\begin{tabular}{l}
|\def\version{final}|\\
|\input{childdoc.def}|\\
|\childdocforwardprefix{final}{child}|
\end{tabular}
\end{center}
%

Note that when several versions of a main file and/or of each child file
are to be generated, it may be convenient to set up a |Makefile| or
shell script to automatise the process.

%%%%%%%%%%%%%%%%%%%%%%%%%%%%%%%%%%%%%%%%%%%%%%%%%%%%%%%%%%%%%%%%%%%%%%%%%%%%%%%%
\subsection{Command Line Processing}
\label{sec:commandline}

The effect of redirection files can also be achieved by invoking
the \LaTeX{} compiler with a more elaborate command line.
Most conveniently this should be done as part
of a shell script or a |Makefile|.

When using \textsf{childdoc} in the main file, the following
command lines effectively perform a redirection
(note that depending on the shell being used,
backslashes may have to be doubled: `|\|' $\to$ `|\\|'):
%
\begin{center}
|... -jobname "|\textit{target}|" |\\|"|[\textit{flags}]%
|\input{childdoc.def}\childdocforward[|\textit{main}|]{|\textit{dest}|}"|
\end{center}
%
Here \textit{target} is the name of the output file,
\textit{main} is the name of the main file
and \textit{dest} is the name of the main or child file to be processed
(all filenames without extensions).
The optional argument \textit{main} can be omitted
if \textit{main} matches \textit{dest}.
Optionally, compilation \textit{flags} can be defined via |\def| commands.
This command line makes the \TeX{} engine believe
it is compiling the file \textit{target}
whose content is specified as the latter parameter.
The provided code then forwards the processing to
\textit{main} or \textit{dest} as described in \secref{sec:forward}.

%%%%%%%%%%%%%%%%%%%%%%%%%%%%%%%%%%%%%%%%%%%%%%%%%%%%%%%%%%%%%%%%%%%%%%%%%%%%%%%%
\subsection{Include by Input}
\label{sec:input}

Including child documents by |\include| has some restrictions by design.
Most notably, the content of a child document always occupies
its own set of pages; pages cannot be shared between child documents.
Usually, this behaviour makes perfect sense
because each child document contain an essential part of the document.
However, in some situations it may be desirable to compose
a document from a collection of parts
without having mandatory page breaks between then.
For this case, the package
provides a mechanism to include parts
by |\input| which can also be processed individually.
However, by construction this mechanism
requires manual handling of the content to be output.

%%%%%%%%%%%%%%%%%%%%%%%%%%%%%%%%%%%%%%%%
\DescribeMacro{\ifchilddocmanual}
The main file should be prepared as usual, see \secref{sec:include}.
However, the document body must make a distinction
between processing of an individual part and of the main document, e.g.:
%
\begin{center}
\begin{tabular}{l}
|\ifchilddocmanual|\\
|\input{\childdocname}|\\
|\||else|\\
\textit{document body with }|\input{|\textit{part}|}|\\
|\||fi|
\end{tabular}
\end{center}
%
The conditional |\ifchilddocmanual| is true whenever
a part to be included by |\input| is being compiled,
and the name of the part is stored in |\childdocname|.

%%%%%%%%%%%%%%%%%%%%%%%%%%%%%%%%%%%%%%%%
\DescribeMacro{\childdocby}
Each part to be included by |\input| should start with:
%
\begin{center}
\begin{tabular}{l}
|\input{childdoc.def}|\\
|\childdocby{|\textit{main}|}|\\
\end{tabular}
\end{center}
%
The directive |\childdocby| is similar to |\childdocof|
described in \secref{sec:include},
but the subsequent selection of content must be done manually.
To that end, both |\ifchilddoc| and |\ifchilddocmanual|
will be true upon processing of a part,
and the name of the part is stored in |\childdocname|.
Note that |\jobname| will be set to the filename of the current part
so that each part receives an individual |.aux| file
that does not interfere with the |.aux| file(s) of the main document.
This behaviour can be altered by the alternative form
|\childdocby[*]{|\textit{main}|}| (with a non-empty optional argument)
which uses the |.aux| file of the main document
by setting |\jobname| to \textit{main}.

%%%%%%%%%%%%%%%%%%%%%%%%%%%%%%%%%%%%%%%%%%%%%%%%%%%%%%%%%%%%%%%%%%%%%%%%%%%%%%%%
\subsection{Driver Development}
\label{sec:driver}

The \textsf{childdoc} mechanism can also be use for the development
of definition files such as \LaTeX{} styles or classes.
This case differs from the above setup with multiple parts
included by |\include| in that no |\includeonly| should be invoked.
This can be achieved by starting the include file
(before |\ProvidesPackage|) with:
%
\begin{center}
\begin{tabular}{l}
|\input{childdoc.def}|\\
|\childdocforward{|\textit{main}|}|\\
\end{tabular}
\end{center}
%
or alternatively with:
%
\begin{center}
\begin{tabular}{l}
|\input{childdoc.def}|\\
|\childdocby{|\textit{main}|}|\\
\end{tabular}
\end{center}
%
Both forms have slightly different effects as described above.
The main file is prepared as usual, see \secref{sec:include}.

%%%%%%%%%%%%%%%%%%%%%%%%%%%%%%%%%%%%%%%%%%%%%%%%%%%%%%%%%%%%%%%%%%%%%%%%%%%%%%%%
\subsection{Legacy Detection}
\label{sec:detection}

The directive |\childdocmain| in the main file can detect
whether the complete document or merely a child is to be compiled
even without using the directive |\childdocof|.
This method is deprecated because it is less robust
and there is no compelling reason to use it;
it is merely provided for backward compatibility
and it may be removed in future versions.

If the detection mechanism is to be used,
it is mandatory to correctly specify
the filename of the main file as the argument of |\childdocmain|:
%
\begin{center}
\begin{tabular}{l}
|\input{childdoc.def}|\\
|\childdocmain{|\textit{main}|}|\\
\end{tabular}
\end{center}
%
If |\jobname| does not match the argument \textit{main} of |\childdocmain|,
it is assumed that |\jobname| points to the child file to be compiled.
When using |\childdocmain| with the main file specified as argument,
it suffices to start a child file
with just |\input{|\textit{main}|}|
without loading of the package and using |\childdocof|.
If instead all processing is done
with the appropriate \textsf{childdoc} directives,
the argument of \textit{main} of |\childdocmain| can be empty.

An alternative version of the command line processing described
in \secref{sec:commandline} using the detection mechanism reads:
%
\begin{center}
|... -jobname "|\textit{target}|" "|[\textit{flags}]%
[|\def\jobname{|\textit{dest}|}|]|\input{|\textit{main}|}"|
\end{center}

%%%%%%%%%%%%%%%%%%%%%%%%%%%%%%%%%%%%%%%%%%%%%%%%%%%%%%%%%%%%%%%%%%%%%%%%%%%%%%%%
\subsection{Manual Code}
\label{sec:manual}

In case one cannot be certain whether the definitions file |childdoc.def|
is installed on the target \TeX{} distribution
and one prefers not to ship it,
it is conceivable to paste a few relevant commands into the sources.

To that end, drop all statements |\input{childdoc.def}|
and perform the replacements as outlined below.
Instead of |\childdocmain{|\textit{main}|}| add the following code
to the top of the main file:
%
\begin{center}
\begin{tabular}{l}
|\||ifdefined\childdocname\endinput\||fi\newif\ifchilddoc|\\
|\edef\childdocname{\scantokens\expandafter{\jobname\noexpand}}|\\
|\def\childdocmain{|\textit{main}|}\||ifx\childdocmain\childdocname\||else|\\
|\childdoctrue\includeonly{\childdocname}\let\jobname\childdocmain\||fi|\\
\end{tabular}
\end{center}
%
Instead of |\childdocof{|\textit{main}|}| just include the main file
at the top of each child file:
%
\begin{center}
|\input{|\textit{main}|}|
\end{center}
%
A simple redirection |\childdocforward{|\textit{dest}|}| is achieved by:
%
\begin{center}
|\def\jobname{|\textit{dest}|}\input{\jobname}|
\end{center}
%
The redirection with prefix
|\childdocforwardprefix[|\textit{prefix}|]{|\textit{dest}|}|
is accomplished by:
%
\begin{center}
\begin{tabular}{l}
|{\edef\jobname{\scantokens\expandafter{\jobname\noexpand}}|\\
|\def\redirectjob |\textit{prefix}|#1~~~{\gdef\jobname{|\textit{dest}|#1}}|\\
|\expandafter\redirectjob\jobname~~~}\input{\jobname}|
\end{tabular}
\end{center}

In an alternative approach,
child documents can be compiled by a specific command line
without additional code or specific definitions:
%
\begin{center}
|... -jobname "|\textit{target}|" "|[\textit{flags}]%
|\includeonly{|\textit{dest}|}\input{|\textit{main}|}"|
\end{center}
%

%%%%%%%%%%%%%%%%%%%%%%%%%%%%%%%%%%%%%%%%%%%%%%%%%%%%%%%%%%%%%%%%%%%%%%%%%%%%%%%%
%%%%%%%%%%%%%%%%%%%%%%%%%%%%%%%%%%%%%%%%%%%%%%%%%%%%%%%%%%%%%%%%%%%%%%%%%%%%%%%%
\section{Information}

%%%%%%%%%%%%%%%%%%%%%%%%%%%%%%%%%%%%%%%%%%%%%%%%%%%%%%%%%%%%%%%%%%%%%%%%%%%%%%%%
\subsection{Copyright}

Copyright \copyright{} 2017--2018 Niklas Beisert

This work may be distributed and/or modified under the
conditions of the \LaTeX{} Project Public License, either version 1.3
of this license or (at your option) any later version.
The latest version of this license is in
  \url{http://www.latex-project.org/lppl.txt}
and version 1.3 or later is part of all distributions of \LaTeX{}
version 2005/12/01 or later.

This work has the LPPL maintenance status `maintained'.

The Current Maintainer of this work is Niklas Beisert.

This work consists of the files |README.txt|, |childdoc.ins| and |childdoc.dtx|
as well as the derived files |childdoc.def|, |cdocsamp.tex|
with |cdocsch1.tex|, |cdocsch2.tex|, |cdocspt3.tex|, |cdocspt4.tex|,
|cdocsdrf.tex|, |cdocsfn1.tex|, |cdocsfn2.tex|
as well as |childdoc.pdf|.

%%%%%%%%%%%%%%%%%%%%%%%%%%%%%%%%%%%%%%%%%%%%%%%%%%%%%%%%%%%%%%%%%%%%%%%%%%%%%%%%
\subsection{Files and Installation}

The package consists of the files:
%
\begin{center}
\begin{tabular}{ll}
    |README.txt|   & readme file \\
    |childdoc.ins| & installation file \\
    |childdoc.dtx| & source file \\
    |childdoc.def| & definition file \\
    |cdocsamp.tex| & sample main file \\
    |cdocsch1.tex| & sample include file \\
    |cdocsch2.tex| & sample include file \\
    |cdocspt3.tex| & sample part file \\
    |cdocspt4.tex| & sample part file \\
    |cdocsdrf.tex| & sample redirection file \\
    |cdocsfn1.tex| & sample redirection file \\
    |cdocsfn2.tex| & sample redirection file \\
    |childdoc.pdf| & manual
\end{tabular}
\end{center}
%
The distribution consists of the files
|README.txt|, |childdoc.ins| and |childdoc.dtx|.
%
\begin{itemize}
\item
Run (pdf)\LaTeX{} on |childdoc.dtx|
to compile the manual |childdoc.pdf| (this file).
\item
Run \LaTeX{} on |childdoc.ins| to create the definitions file |childdoc.def|
and the sample |cdocsamp.tex| with include files
|cdocsch1.tex|, |cdocsch2.tex|, |cdocspt3.tex|, |cdocspt4.tex|,
|cdocsdrf.tex|, |cdocsfn1.tex|, |cdocsfn2.tex|.
Then copy the file |childdoc.def| to an appropriate directory of your \LaTeX{}
distribution, e.g.\ \textit{texmf-root}|/tex/latex/childdoc|.
\end{itemize}

%%%%%%%%%%%%%%%%%%%%%%%%%%%%%%%%%%%%%%%%%%%%%%%%%%%%%%%%%%%%%%%%%%%%%%%%%%%%%%%%
\subsection{Related CTAN Packages}

There are several other packages which offer a similar functionality:
%
\begin{itemize}
\item
The packages
\href{http://ctan.org/pkg/docmute}{\textsf{docmute}},
\href{http://ctan.org/pkg/includex}{\textsf{includex}} and
\href{http://ctan.org/pkg/standalone}{\textsf{standalone}}
provide commands to include only the document body of
a child file thus allowing both files to be compiled individually.
\item
The packages \href{http://ctan.org/pkg/subdocs}{\textsf{subdocs}}
and \href{http://ctan.org/pkg/subfiles}{\textsf{subfiles}}
provide structures in which the main and child documents can be
encapsulated and allowing them to be compiled individually.
The inclusion mechanism is different from the conventional |\include|.
\item
The package \href{http://ctan.org/pkg/combine}{\textsf{combine}}
is an elaborate solution to combine several documents into one.
\end{itemize}
%
See also the CTAN topic \href{http://ctan.org/topic/subdocs}{\textsf{subdocs}}
for further related packages.
The present package differs from the above solutions in that
a document structure constructed with the conventional |\include| mechanism
just needs two extra commands at the top of every file
such that all constituent files can be compiled individually.

%%%%%%%%%%%%%%%%%%%%%%%%%%%%%%%%%%%%%%%%%%%%%%%%%%%%%%%%%%%%%%%%%%%%%%%%%%%%%%%%
%\subsection{Feature Suggestions}
%
%The following is a list of features which may be useful for future
%versions of this package:
%%
%\begin{itemize}
%\item
%\ldots
%\end{itemize}

%%%%%%%%%%%%%%%%%%%%%%%%%%%%%%%%%%%%%%%%%%%%%%%%%%%%%%%%%%%%%%%%%%%%%%%%%%%%%%%%
\subsection{Revision History}

%%%%%%%%%%%%%%%%%%%%%%%%%%%%%%%%%%%%%%%%
\paragraph{v2.0:} 2018/12/30

\begin{itemize}
\item
immediate forward processing
\item
added |\childdocby| mechanism
\item
manual restructured
\end{itemize}

%%%%%%%%%%%%%%%%%%%%%%%%%%%%%%%%%%%%%%%%
\paragraph{v1.6:} 2018/01/17

\begin{itemize}
\item
application for development of include files
\item
corrections to manual
\end{itemize}

%%%%%%%%%%%%%%%%%%%%%%%%%%%%%%%%%%%%%%%%
\paragraph{v1.5:} 2017/05/21

\begin{itemize}
\item
more complete structuring introduced
\item
|\childdocof| introduced
\item
|\childdoc| renamed to |\childdocmain|
\item
|\childredirect| renamed to |\childdocforward| and |\childdocforwardprefix|
and functionality expanded
\end{itemize}

%%%%%%%%%%%%%%%%%%%%%%%%%%%%%%%%%%%%%%%%
\paragraph{v1.0:} 2017/04/27

\begin{itemize}
\item
manual and install package
\item
first version published on CTAN
\end{itemize}

%%%%%%%%%%%%%%%%%%%%%%%%%%%%%%%%%%%%%%%%
\paragraph{v0.6:} 2017/04/26

\begin{itemize}
\item
redirection mechanism added
\end{itemize}

%%%%%%%%%%%%%%%%%%%%%%%%%%%%%%%%%%%%%%%%
\paragraph{v0.5:} 2017/04/26

\begin{itemize}
\item
functionality in definition file
\end{itemize}


%%%%%%%%%%%%%%%%%%%%%%%%%%%%%%%%%%%%%%%%%%%%%%%%%%%%%%%%%%%%%%%%%%%%%%%%%%%%%%%%
%%%%%%%%%%%%%%%%%%%%%%%%%%%%%%%%%%%%%%%%%%%%%%%%%%%%%%%%%%%%%%%%%%%%%%%%%%%%%%%%
%%%%%%%%%%%%%%%%%%%%%%%%%%%%%%%%%%%%%%%%%%%%%%%%%%%%%%%%%%%%%%%%%%%%%%%%%%%%%%%%
\appendix

\settowidth\MacroIndent{\rmfamily\scriptsize 000\ }

 \DocInput{childdoc.dtx}

\end{document}
%</driver>
% \fi
%
% %%%%%%%%%%%%%%%%%%%%%%%%%%%%%%%%%%%%%%%%%%%%%%%%%%%%%%%%%%%%%%%%%%%%%%%%%%%%%%
% %%%%%%%%%%%%%%%%%%%%%%%%%%%%%%%%%%%%%%%%%%%%%%%%%%%%%%%%%%%%%%%%%%%%%%%%%%%%%%
% \section{Sample}
%\iffalse
%<*samplemain>
%\fi
%
% The following presents a sample document
% with two chapters, two parts, a title page,
% a compile flag as well as three forwarding files to set the flag.
% It consists of eight |.tex| files:
% \begin{center}
% \begin{tabular}{ll}
% |cdocsamp.tex|&main file\\
% |cdocsch1.tex|&include file for chapter 1\\
% |cdocsch2.tex|&include file for chapter 2\\
% |cdocspt3.tex|&include file for part 3\\
% |cdocspt4.tex|&include file for part 4\\
% |cdocsdrf.tex|&forwarding file for main file in draft mode\\
% |cdocsfi1.tex|&forwarding file for final version of chapter 1\\
% |cdocsfi2.tex|&forwarding file for final version of chapter 2\\
% \end{tabular}
% \end{center}
% Each of the eight files can be compiled directly by the \LaTeX{} compiler.
%
% %%%%%%%%%%%%%%%%%%%%%%%%%%%%%%%%%%%%%%
% \paragraph{Main File.}
%
% The main file is called |cdocsamp.tex|.
%
% Load the \textsf{childdoc} definitions and
% declare the filename for the main document:
%    \begin{macrocode}
\input{childdoc.def}
\childdocmain{}
%    \end{macrocode}

% Optional override for |\version| flag:
%    \begin{macrocode}
%%\ifchilddoc\else\providecommand{\version}{draft}\fi
%    \end{macrocode}

% Define the default values for the |\version| flag
% (|final| for the main file and |draft| for childs):
%    \begin{macrocode}
\ifchilddoc
\providecommand{\version}{draft}
\else
\providecommand{\version}{final}
\fi
%    \end{macrocode}

% Load the standard document class:
%    \begin{macrocode}
\documentclass[12pt]{article}
%    \end{macrocode}

% Start the document body:
%    \begin{macrocode}
\begin{document}
%    \end{macrocode}

% Declare a title page.
% Print title, part of document being processed and version flag:
%    \begin{macrocode}
\addtocounter{page}{-1}
\begin{center}
{\LARGE\bfseries{}childdoc example\par}
\vspace{1cm}
\ifchilddoc
\ifchilddocmanual part\else chapter\fi:
`\childdocname' of `\childdocjob'\par
\else
main document: `\childdocjob'\par
\fi
version: \version\par
\end{center}
\newpage
%    \end{macrocode}

% Manually include selected file,
% otherwise process as usual:
%    \begin{macrocode}
\ifchilddocmanual
\section*{part `\childdocname'}
\input{\childdocname}
\else
%    \end{macrocode}

% Include the two chapters:
%    \begin{macrocode}
\include{cdocsch1}
\include{cdocsch2}
%    \end{macrocode}

% Include the two parts unless only chapters should be displayed:
%    \begin{macrocode}
\ifchilddoc\else
\section{part three}
\input{cdocspt3}
\section{part four}
\input{cdocspt4}
\fi
%    \end{macrocode}

% Process as usual until here:
%    \begin{macrocode}
\fi
%    \end{macrocode}

% End of document body:
%    \begin{macrocode}
\end{document}
%    \end{macrocode}
%\iffalse
%</samplemain>
%\fi
%
% %%%%%%%%%%%%%%%%%%%%%%%%%%%%%%%%%%%%%%
% \paragraph{Chapter Include Files.}
%
% The include files are called |cdocsch1.tex| and |cdocsch2.tex|.
%
%\iffalse
%<*samplechap1|samplechap2>
%\fi

% Optional override for |\version| flag:
%    \begin{macrocode}
%%\providecommand{\version}{final}
%    \end{macrocode}

% Include the main document:
%    \begin{macrocode}
\input{childdoc.def}
\childdocof{cdocsamp}
%    \end{macrocode}

%\iffalse
%</samplechap1|samplechap2>
%\fi
%
%\iffalse
%<*samplechap1>
%\fi
% Some text for chapter 1:
%    \begin{macrocode}
\section{one}
some text in chapter one
%    \end{macrocode}

%\iffalse
%</samplechap1>
%\fi
% Some text for chapter 2:
%\iffalse
%<*samplechap2>
%\fi
%    \begin{macrocode}
\section{two}
more text in chapter two
%    \end{macrocode}

%\iffalse
%</samplechap2>
%\fi
%
% %%%%%%%%%%%%%%%%%%%%%%%%%%%%%%%%%%%%%%
% \paragraph{Part Include Files.}
%
% The include files are called |cdocspt3.tex| and |cdocspt4.tex|.
%
%\iffalse
%<*samplepart3|samplepart4>
%\fi

% Optional override for |\version| flag:
%    \begin{macrocode}
%%\providecommand{\version}{final}
%    \end{macrocode}

% Include the main document:
%    \begin{macrocode}
\input{childdoc.def}
\childdocby{cdocsamp}
%    \end{macrocode}

%\iffalse
%</samplepart3|samplepart4>
%\fi
%
%\iffalse
%<*samplepart3>
%\fi
% Some text for part 3:
%    \begin{macrocode}
some text in part three
%    \end{macrocode}

%\iffalse
%</samplepart3>
%\fi
% Some text for part 4:
%\iffalse
%<*samplepart4>
%\fi
%    \begin{macrocode}
more text in part four
%    \end{macrocode}

%\iffalse
%</samplepart4>
%\fi
%
% %%%%%%%%%%%%%%%%%%%%%%%%%%%%%%%%%%%%%%
% \paragraph{Forwarding for a Complete Draft.}
%
% The following forwarding file |cdocsdrf.tex|
% compiles the main document in draft mode:
%\iffalse
%<*sampledraft>
%\fi
%    \begin{macrocode}
\def\version{draft}
\input{childdoc.def}
\childdocforward{cdocsamp}
%    \end{macrocode}

%\iffalse
%</sampledraft>
%\fi
%
% %%%%%%%%%%%%%%%%%%%%%%%%%%%%%%%%%%%%%%
% \paragraph{Forwarding for Final Version of the Chapters.}
%
% The following forwarding files |cdocsfn1.tex| and |cdocsfn2.tex|
% (with identical content)
% compile the final versions of the child documents
% |cdocsch1.tex| and |cdocsch2.tex|, respectively:
%\iffalse
%<*samplefinal>
%\fi
%    \begin{macrocode}
\def\version{final}
\input{childdoc.def}
\childdocforwardprefix[cdocsamp]{cdocsfn}{cdocsch}
%    \end{macrocode}

%\iffalse
%</samplefinal>
%\fi
%
% %%%%%%%%%%%%%%%%%%%%%%%%%%%%%%%%%%%%%%
% \paragraph{Command Line Processing.}
%
% The following three command lines generate the output files
% |cdocscld|, |cdocscl1| and |cdocscl2|
% which should be identical to
% |cdocsdrf|, |cdocsch1| and |cdocsfn2|, respectively:
% \begin{center}
% \begin{tabular}{l}
% |latex -jobname cdocscld \|\\
% |  "\def\version{draft}\input{childdoc.def}\childdocforward{cdocsamp}"|\\
% |latex -jobname cdocscl1 \|\\
% |  "\input{childdoc.def}\childdocforward[cdocsamp]{cdocsch1}"|\\
% |latex -jobname cdocscl2 \|\\
% |  "\def\version{final}\input{childdoc.def}\childdocforward{cdocsch2}"|
% \end{tabular}
% \end{center}
% Note that the trailing backslash on each first line
% merely continues the input to the second line
% (for convenient cut ant paste).
% Furthermore, the command |latex| can be replaced by any
% of its alternative versions such as |pdflatex|.
%
% %%%%%%%%%%%%%%%%%%%%%%%%%%%%%%%%%%%%%%%%%%%%%%%%%%%%%%%%%%%%%%%%%%%%%%%%%%%%%%
% %%%%%%%%%%%%%%%%%%%%%%%%%%%%%%%%%%%%%%%%%%%%%%%%%%%%%%%%%%%%%%%%%%%%%%%%%%%%%%
% \section{Implementation}
%\iffalse
%<*package>
%\fi
%
% This section describes the definitions file |childdoc.def|.

% The definitions cannot be loaded using |\usepackage| or |\RequirePackage|
% which has a mechanism to prevent loading a style file more than once.
% When loading the definitions by means of |\input|
% multiple instances have to be prevented manually:
%\iffalse
%This code needs to be before the `\ProvidesFile' directive
%which is defined at the beginning of this file.
%Therefore it is also placed there and commented out here.
%</package>
%<*discard>
%\fi
%    \begin{macrocode}
\ifdefined\childdocmain\endinput\fi
%    \end{macrocode}
%\iffalse
%</discard>
%<*package>
%\fi
%
% \macro{\ifchilddoc}
% \macro{\ifchilddocmanual}
% The conditional |\ifchilddoc| tells whether a
% child (true) or main (false) document is being compiled.
% The conditional |\ifchilddocmanual| tells whether
% the |\includeonly| mechanism is used (false) or
% the selection of child files must be performed manually (true).
% The definitions initialise to false:
%    \begin{macrocode}
\newif\ifchilddoc
\newif\ifchilddocmanual
%    \end{macrocode}

% \macro{\childdocname}
% \macro{\childdocjob}
% The macro |\childdocname| stores the name of the main document
% to be compiled. The macro |\childdocjob| stores the name of
% the document on which the \LaTeX{} compiler was originally invoked.
% The content of |\jobname| cannot be compared
% to filenames specified in the source due to different catcodes.
% The following code rescans |\jobname|, stores the result
% in |\childdocname| and saves a copy in |\childdocjob|:
%    \begin{macrocode}
\edef\childdocname{\scantokens\expandafter{\jobname\noexpand}}
\let\childdocjob\childdocname
%    \end{macrocode}

% \macro{\childdocdisable}
% The macro |\childdocdisable| prevents the main file
% from being processed more than once.
% At this stage, the main document command |\childdocmain|
% is assumed to be called once again where it should do nothing.
% Any subsequent call to it should prevent
% a secondary processing of the main document
% It overwrites the forwarding commands
% |\childdocof| and |\childdocforward|
% with empty macros to prevent further inclusions of the main document:
%    \begin{macrocode}
\newcommand{\childdocdisable}
{
  \renewcommand{\childdocmain}[1]{\renewcommand{\childdocmain}[1]{\endinput}}
  \renewcommand{\childdocof}[1]{}
  \renewcommand{\childdocby}[2][]{}
  \renewcommand{\childdocforward}[2][]{}
  \renewcommand{\childdocdisable}{}
}
%    \end{macrocode}

% \macro{\childdocmain}
% The macro |\childdocmain| is to be called at the top of the main file
% with nothing or the main filename (without extension) as argument.
% First, it breaks loops.
% If the argument is not empty and does not match |\childdocname|
% (which is set by the first inclusion of |childdoc.def|),
% |\ifchilddoc| is set to true, |\includeonly| is applied to the child file
% and |\jobname| is set to the main file
% (for proper handling of |.aux| files):
%    \begin{macrocode}
\newcommand{\childdocmain}[1]
{
  \childdocdisable\childdocmain{}
  \if?#1?\else
    \begingroup
      \def\childdoctmp{#1}
      \ifx\childdoctmp\childdocname
        \def\childdoctmp{}
      \else
        \def\childdoctmp
        {
          \childdoctrue
          \includeonly{\childdocname}
          \def\childdocjob{#1}
          \def\jobname{#1}
        }
      \fi
      \expandafter
    \endgroup
    \childdoctmp
  \fi
}
%    \end{macrocode}

% \macro{\childdocof}
% The command |\childdocof| redirects
% compilation to the main file |#1|.
%    \begin{macrocode}
\newcommand{\childdocof}[1]
{
  \childdocdisable
  \childdoctrue
  \includeonly{\childdocname}
  \def\jobname{#1}
  \def\childdocjob{#1}
  \input{#1}
}
%    \end{macrocode}

% \macro{\childdocby}
% The command |\childdocby| ....
%    \begin{macrocode}
\newcommand{\childdocby}[2][]
{
  \childdocdisable
  \childdoctrue
  \childdocmanualtrue
  \if?#1?\else
    \def\jobname{#2}
  \fi
  \def\childdocjob{#2}
  \input{#2}
  \endinput
}
%    \end{macrocode}

% \macro{\childdocforward}
% The command |\childdocforward| redirects
% compilation to the main file or
% (if the optional argument is given) a child file.
% Parameters are set as if the main file
% or a child file starting with |\childdocof| was compiled.
% Then compilation is handed over to the main file:
%    \begin{macrocode}
\newcommand{\childdocforward}[2][]
{
  \begingroup
    \if?#1?
      \def\childdoctmp
      {
        \def\childdocname{#2}
        \def\childdocjob{#2}
        \def\jobname{#2}
        \input{#2}
        \endinput
      }
    \else
      \def\childdoctmp
      {
        \childdocdisable
        \def\childdocname{#2}
        \childdoctrue
        \includeonly{#2}
        \def\childdocjob{#1}
        \def\jobname{#1}
        \input{#1}
        \endinput
      }
    \fi
    \expandafter
  \endgroup
  \childdoctmp
}
%    \end{macrocode}

% \macro{\childdocforwardprefix}
% The command |\childdocforwardprefix| redirects
% compilation to the main or a child file by means of a pattern.
% The prefix |#1| in the current filename is replaced by |#2|
% and the suffix of the current filename is kept
% (it is assumed that the filename does not contain the substring `|~~~|'
% which is used as a delimiter).
% Compilation is handed over to the new file by |\childdocforward|:
%    \begin{macrocode}
\newcommand{\childdocforwardprefix}[3][]
{
  \begingroup
    \def\childdocextract #2##1~~~{\def\childdoctmp{\childdocforward[#1]{#3##1}}}
    \expandafter\childdocextract\childdocname~~~
    \expandafter
  \endgroup
  \childdoctmp
}
%    \end{macrocode}

% \macro{\childdoc}
% The deprecated macro |\childdoc| is a legacy version of |\childdocmain|:
%    \begin{macrocode}
\newcommand{\childdoc}{\childdocmain}
%    \end{macrocode}

% \macro{\childdocredirect}
% The deprecated macro |\childdocredirect| is a legacy version
% of |\childdocforward| and |\childdocforwardprefix|:
%    \begin{macrocode}
\newcommand{\childdocredirect}[2][]
{
  \begingroup
    \if?#1?
      \def\childdoctmp{\childdocforward{#2}}
    \else
      \def\childdoctmp{\childdocforwardprefix{#1}{#2}}
    \fi
    \expandafter
  \endgroup
  \childdoctmp
}
%    \end{macrocode}

%\iffalse
%</package>
%\fi
%
\endinput
|\\
|\childdocmain{}|\\
\end{tabular}
\end{center}
at the very top of the main \LaTeX{} file,
in particular \emph{before} the |\documentclass| statement!
The argument of |\childdocmain| should be left empty
(but it must be present).

%%%%%%%%%%%%%%%%%%%%%%%%%%%%%%%%%%%%%%%%
\DescribeMacro{\childdocof}
Furthermore, add the commands
\begin{center}
\begin{tabular}{l}
|% \iffalse
%
% childdoc.dtx Copyright (C) 2017-2018 Niklas Beisert
%
% This work may be distributed and/or modified under the
% conditions of the LaTeX Project Public License, either version 1.3
% of this license or (at your option) any later version.
% The latest version of this license is in
%   http://www.latex-project.org/lppl.txt
% and version 1.3 or later is part of all distributions of LaTeX
% version 2005/12/01 or later.
%
% This work has the LPPL maintenance status `maintained'.
%
% The Current Maintainer of this work is Niklas Beisert.
%
% This work consists of the files childdoc.dtx and childdoc.ins
% and the derived files childdoc.def and cdocsamp.tex with
% cdocsch1.tex, cdocsch2.tex, cdocsdrf.tex, cdocsfn1.tex, cdocsfn2.tex.
%
%<package>\ifdefined\childdocmain\endinput\fi
%<package>\ProvidesFile{childdoc.def}[2018/12/30 v2.0 child document driver]
%<samplemain>\ProvidesFile{cdocsamp.tex}[2018/12/30 v2.0 sample for childdoc]
%<*driver>
%\ProvidesFile{childdoc.drv}[2018/12/30 v2.0 childdoc reference manual file]
\PassOptionsToClass{10pt,a4paper}{article}
\documentclass{ltxdoc}

\usepackage[margin=35mm]{geometry}
\usepackage{hyperref}
\usepackage{hyperxmp}
\usepackage[usenames]{color}

\hypersetup{colorlinks=true}
\hypersetup{pdfstartview=FitH}
\hypersetup{pdfpagemode=UseNone}
\hypersetup{pdfsource={}}
\hypersetup{pdflang={en-UK}}
\hypersetup{pdfcopyright={Copyright 2017-2018 Niklas Beisert.
  This work may be distributed and/or modified under the
  conditions of the LaTeX Project Public License, either version 1.3
  of this license or (at your option) any later version.}}
\hypersetup{pdflicenseurl={http://www.latex-project.org/lppl.txt}}
\hypersetup{pdfcontactaddress={ETH Zurich, ITP, HIT K,
  Wolfgang-Pauli-Strasse 27}}
\hypersetup{pdfcontactpostcode={8093}}
\hypersetup{pdfcontactcity={Zurich}}
\hypersetup{pdfcontactcountry={Switzerland}}
\hypersetup{pdfcontactemail={nbeisert@itp.phys.ethz.ch}}
\hypersetup{pdfcontacturl={http://people.phys.ethz.ch/\xmptilde nbeisert/}}

\newcommand{\secref}[1]{\hyperref[#1]{section \ref*{#1}}}

\parskip1ex
\parindent0pt
\let\olditemize\itemize
\def\itemize{\olditemize\parskip0pt}

\begin{document}

\title{The \textsf{childdoc} Package}
\hypersetup{pdftitle={The childdoc Package}}
\author{Niklas Beisert\\[2ex]
  Institut f\"ur Theoretische Physik\\
  Eidgen\"ossische Technische Hochschule Z\"urich\\
  Wolfgang-Pauli-Strasse 27, 8093 Z\"urich, Switzerland\\[1ex]
  \href{mailto:nbeisert@itp.phys.ethz.ch}
  {\texttt{nbeisert@itp.phys.ethz.ch}}}
\hypersetup{pdfauthor={Niklas Beisert}}
\hypersetup{pdfsubject={Manual for the LaTeX2e Package childdoc}}
\date{30 December 2018, \textsf{v2.0}}
\maketitle

\begin{abstract}\noindent
\textsf{childdoc} is a \LaTeXe{} package
that enables the direct compilation
of document sections included by |\include|
to individual files.
\end{abstract}

\begingroup
\parskip0ex
\tableofcontents
\endgroup

%%%%%%%%%%%%%%%%%%%%%%%%%%%%%%%%%%%%%%%%%%%%%%%%%%%%%%%%%%%%%%%%%%%%%%%%%%%%%%%%
%%%%%%%%%%%%%%%%%%%%%%%%%%%%%%%%%%%%%%%%%%%%%%%%%%%%%%%%%%%%%%%%%%%%%%%%%%%%%%%%
\section{Introduction}

\LaTeX{} provides a mechanism to structure a large document (such as a book)
into a main file and several child files (containing the chapters)
using the |\include| command.
This mechanism is beneficial for documents
which span hundreds of pages in order to
make the source file(s) more manageable.
Moreover, compilation can be restricted to
selected child files by means of the |\includeonly| command.
The latter feature can be used to reduce the compilation time while editing
(this was significantly more useful in the earlier days of \LaTeX{})
or to generate a smaller document which is easier to navigate.
Another application of |\includeonly| is to generate
documents consisting of selected parts of the complete document.

However, there are a few drawbacks of the plain |\include| mechanism:
\begin{itemize}
\item
The child files cannot be compiled on their own,
they can only be compiled via the main file.
A naive editing environment
(such as a text editor with an option
to have the current file processed by \LaTeX)
may require one to switch to the main file before compiling;
attempting to compile the child file produces errors.
\item
The main file must be modified (each time)
to adjust the |\includeonly| command
to the present needs. This easily leaves the main file in a messy state.
\item
The generated document will always carry the filename
of the main document. This is inconvenient if
several child files are to be compiled and
to be kept for distribution.
\end{itemize}

The present package provides a simple interface
to make child files individually compilable by \LaTeX{}.
Compiling a child file then has the same effect as compiling
the main file with an |\includeonly| command
to select the appropriate child.
Moreover the generated document will carry the name of the child
rather than the main file.
This resolves all three above issues.

This feature is meant to make the editing of books,
thesis documents and lecture notes somewhat more convenient.
However, the package can also be used efficiently for
composing a series of documents (such as exercise sheets)
which are typically distributed individually.
It then assists the author in generating the individual documents
(potentially in different versions)
as well as a document containing the collected series.
Another application is in developing style files
or other kinds of included material
where compilation of the style file could redirect
to a sample or test file.

%%%%%%%%%%%%%%%%%%%%%%%%%%%%%%%%%%%%%%%%%%%%%%%%%%%%%%%%%%%%%%%%%%%%%%%%%%%%%%%%
%%%%%%%%%%%%%%%%%%%%%%%%%%%%%%%%%%%%%%%%%%%%%%%%%%%%%%%%%%%%%%%%%%%%%%%%%%%%%%%%
\section{Usage}

First of all, the package \textsf{childdoc} is \emph{not} a standard
\LaTeXe{} |.sty| style file! Therefore it needs to be invoked in
a non-standard way.

%%%%%%%%%%%%%%%%%%%%%%%%%%%%%%%%%%%%%%%%%%%%%%%%%%%%%%%%%%%%%%%%%%%%%%%%%%%%%%%%
\subsection{Included Files}
\label{sec:include}

%%%%%%%%%%%%%%%%%%%%%%%%%%%%%%%%%%%%%%%%
\DescribeMacro{\childdocmain}
To use the package, add the commands
\begin{center}
\begin{tabular}{l}
|\input{childdoc.def}|\\
|\childdocmain{}|\\
\end{tabular}
\end{center}
at the very top of the main \LaTeX{} file,
in particular \emph{before} the |\documentclass| statement!
The argument of |\childdocmain| should be left empty
(but it must be present).

%%%%%%%%%%%%%%%%%%%%%%%%%%%%%%%%%%%%%%%%
\DescribeMacro{\childdocof}
Furthermore, add the commands
\begin{center}
\begin{tabular}{l}
|\input{childdoc.def}|\\
|\childdocof{|\textit{main}|}|\\
\end{tabular}
\end{center}
at the top of every child file \textit{child}
which is included by |\include{|\textit{child}|}|
from within the main file
(or at least for those files to be compiled individually).
The argument \textit{main} must be the filename of the main file.

There are a couple of
considerations in setting up the main and child documents:

%%%%%%%%%%%%%%%%%%%%%%%%%%%%%%%%%%%%%%%%
\paragraph{Restrictions.}

Please note the following restrictions:
\begin{itemize}
\item
|\childdocmain| must be called with one argument \textit{main}
to ensure compatibility with earlier version of the package.
It must either be empty (|\childdocmain{}|)
or precisely match the filename of the main file in which it is specified.
See \secref{sec:detection} for further information.
\item
The filename \textit{main} must be specified without the |.tex| extension.
\item
The filename \textit{main} is case sensitive
(even in case-insensitive file systems)
due to internal string comparison.
\item
The argument \textit{main} should be fully expanded, it cannot be a macro.
\item
Subdirectories and special characters should be avoided in filenames.
\item
The command |\childdocmain{|\textit{main}|}| must be followed by a whitespace.
It should not be followed immediately by another command
or by a comment mark `|%|'.
This is because the \TeX{} parser reads the token immediately following
the argument of |\childdocmain| and puts it
at the beginning of every child section;
however, a white\-space is ignored.
\end{itemize}

%%%%%%%%%%%%%%%%%%%%%%%%%%%%%%%%%%%%%%%%
\paragraph{Content of Main File.}

It is advisable to place all content in the child files included by |\include|.
Any output contained in the main file will appear in all child documents
unless suppressed manually;
it cannot be suppressed automatically by the |\includeonly| directive
and thus should normally be avoided.
A method to include some content in the main file
by means of conditional processing is described in \secref{sec:conditional}.

%%%%%%%%%%%%%%%%%%%%%%%%%%%%%%%%%%%%%%%%
\paragraph{Page Numbering.}

When only a part of the document is compiled,
the appropriate numbering of pages
(as well as other status parameters)
is determined from the |.aux| files.
The latter contain information from previous passes.
However this information needs to propagate through
all intermediate child documents.
Therefore the page numbering in child documents may well
be inconsistent until the complete document is compiled at least once.

A useful (if unconventional) way to always ensure a consistent
page numbering is to restart the numbering in each child document
and denote the pages by `\textit{child}|.|\textit{page}'
where \textit{child} represents the chapter/section number of the child file.
This can be achieved by the command
|\numberwithin{page}{|\textit{child}|}|
of the \textsf{amsmath} package
where \textit{child} can be |chapter| or |section|
depending on the chosen structuring.
Alternatively, one can modify the macro |\thepage| appropriately
and reset the counter |page| at the start of each child file.

%%%%%%%%%%%%%%%%%%%%%%%%%%%%%%%%%%%%%%%%%%%%%%%%%%%%%%%%%%%%%%%%%%%%%%%%%%%%%%%%
\subsection{Conditional Processing}
\label{sec:conditional}

The package provides a mechanism to compile different versions
of a document. To customise the versions further some conditional processing
can come in handy to distinguish which version is being compiled.
The package provides two macros to describe the compilation context:

%%%%%%%%%%%%%%%%%%%%%%%%%%%%%%%%%%%%%%%%
\DescribeMacro{\ifchilddoc}
The conditional |\ifchilddoc| distinguishes between the compilation of
child documents and the main document:
%
\begin{center}
|\ifchilddoc |\textit{child-code}| |[|\||else |\textit{main-code}]| \||fi|
\end{center}

%%%%%%%%%%%%%%%%%%%%%%%%%%%%%%%%%%%%%%%%
\DescribeMacro{\childdocname}
\DescribeMacro{\childdocjob}
The macro |\childdocname| contains the filename (without extension)
of the main or child file being processed.
Note that |\childdocjob| will always contain the name of the main file.

%%%%%%%%%%%%%%%%%%%%%%%%%%%%%%%%%%%%%%%%
\paragraph{Title Page.}

Conditional processing can be used to include a title or banner page
in the main document when proper precautions are taken.
Importantly, the code in the main file should ensure that the page counter
(as well as other status parameters which are stored in the |.aux| files)
takes the same value after the conditional processing.
Otherwise the page numbers may take divergent values
depending on which part is compiled.

For example, a title page could be declared by:
%
\begin{center}
\begin{tabular}{l}
|\ifchilddoc\||else|\\
|\addtocounter{page}{-1}|\\
\textit{code for title page}\\
|\newpage|\\
|\||fi|
\end{tabular}
\end{center}
%
A banner page for the child documents can be generated by:
%
\begin{center}
\begin{tabular}{l}
|\ifchilddoc|\\
|\addtocounter{page}{-1}|\\
\textit{code for banner page}\\
|\newpage|\\
|\||fi|
\end{tabular}
\end{center}
%
Here one could write a message such as:
\begin{center}
|This is the part \childdocname{} of \childdocjob{}.|
\end{center}

%%%%%%%%%%%%%%%%%%%%%%%%%%%%%%%%%%%%%%%%%%%%%%%%%%%%%%%%%%%%%%%%%%%%%%%%%%%%%%%%
\subsection{Flags}
\label{sec:flags}

The package makes it easy to generate different versions
of the main or child documents.
To this end compilation flags can be defined
and assigned different default values.
They will be particularly useful in conjunction
with the forwarding mechanism described in \secref{sec:forward}.

For example, it may be useful to have a flag |\version|
which can be set to |draft| or |final|.
The document source will contain some conditional code
depending on the value of |\version|.
Suppose further, the flag should default to |final| for the main file
and to |draft| for child files
which is a natural assignment for editing the document.
This is achieved by placing the following code
in the preamble of the main document
(below the |\childdocmain| directive):
%
\begin{center}
\begin{tabular}{l}
|\ifchilddoc|\\
|\providecommand{\version}{draft}|\\
|\||else|\\
|\providecommand{\version}{final}|\\
|\||fi|
\end{tabular}
\end{center}
%
The definition by |\providecommand| makes sure
that previous definitions are not overwritten.
Further statements |\providecommand{\version}{...}|
can thus be added before the above code to override it.

For the main file, one might add a line
(between |\childdocmain| and the above block)
%
\begin{center}
|%\ifchilddoc\||else\providecommand{\version}{draft}\||fi|
\end{center}
%
which can be uncommented to produce a draft version.
Likewise one can add a line to the very top of a child file
(above the |\childdocof{|\textit{main}|}| directive)
%
\begin{center}
|%\providecommand{\version}{final}|
\end{center}
%
which can be uncommented to produce the final version of this child document.

%%%%%%%%%%%%%%%%%%%%%%%%%%%%%%%%%%%%%%%%%%%%%%%%%%%%%%%%%%%%%%%%%%%%%%%%%%%%%%%%
\subsection{Forwarding}
\label{sec:forward}

Different versions of the main or child documents
using compilation flags as described in \secref{sec:flags}
can be (permanently) stored in different files
for convenient compilation, viewing and distribution.
To this end, the package defines a command
to pass on compilation to a different file:

%%%%%%%%%%%%%%%%%%%%%%%%%%%%%%%%%%%%%%%%
\DescribeMacro{\childdocforward}
The command |\childdocforward| redirects processing to
another source file:
%
\begin{center}
\begin{tabular}{l}
|\input{childdoc.def}|\\
|\childdocforward[|\textit{main}|]{|\textit{dest}|}|\\
\end{tabular}
\end{center}
%
The argument \textit{dest} is the destination file
(without extension).
It should be the main file or one of the child files.
Note that further \textsf{childdoc} directives
such as |\childdocof| and |\childdocforward|
in the indicated file will be processed in this form.
The optional argument \textit{main}
passes on directly to the main file \textit{main}
while pretending to compile the child \textit{dest}.
This form behaves as if \textit{dest}
issues |\childdocof{|\textit{main}|}| right away,
and no further \textsf{childdoc} directives will be processed.

%%%%%%%%%%%%%%%%%%%%%%%%%%%%%%%%%%%%%%%%
\DescribeMacro{\...prefix}
In the alternative form |\childdocforwardprefix|,
%
\begin{center}
\begin{tabular}{l}
|\input{childdoc.def}|\\
|\childdocforwardprefix[|\textit{main}|]{|\textit{prefix}|}{|\textit{dest}|}|
\end{tabular}
\end{center}
%
the destination file is determined by a pattern
depending on the current file:
To make this work, the current file must be called
`{\textit{prefix}\hspace{0.2em}\textit{suffix}}'
with \textit{prefix} matching precisely the argument.
Processing is then passed on to the file
`{\textit{dest}\hspace{0.2em}\textit{suffix}}'.
Surely, the same effect is achieved by
directly specifying the
argument `{\textit{dest}\hspace{0.2em}\textit{suffix}}'
in the first form.
However, that requires to set up a different file
for each child. With the alternative form of the command
all these files can have exactly the same content
which simplifies setting them up and maintaining them.

For example, the following file |draft.tex|
with a compilation flag |\version| as described in \secref{sec:flags}
compiles the main document as a draft:
%
\begin{center}
\begin{tabular}{l}
|\def\version{draft}|\\
|\input{childdoc.def}|\\
|\childdocforward{|\textit{main}|}|
\end{tabular}
\end{center}
%
Likewise, the following files |final|\textit{nn}|.tex|
compile the final version of the child document
|child|\textit{nn}|.tex|:
%
\begin{center}
\begin{tabular}{l}
|\def\version{final}|\\
|\input{childdoc.def}|\\
|\childdocforwardprefix{final}{child}|
\end{tabular}
\end{center}
%

Note that when several versions of a main file and/or of each child file
are to be generated, it may be convenient to set up a |Makefile| or
shell script to automatise the process.

%%%%%%%%%%%%%%%%%%%%%%%%%%%%%%%%%%%%%%%%%%%%%%%%%%%%%%%%%%%%%%%%%%%%%%%%%%%%%%%%
\subsection{Command Line Processing}
\label{sec:commandline}

The effect of redirection files can also be achieved by invoking
the \LaTeX{} compiler with a more elaborate command line.
Most conveniently this should be done as part
of a shell script or a |Makefile|.

When using \textsf{childdoc} in the main file, the following
command lines effectively perform a redirection
(note that depending on the shell being used,
backslashes may have to be doubled: `|\|' $\to$ `|\\|'):
%
\begin{center}
|... -jobname "|\textit{target}|" |\\|"|[\textit{flags}]%
|\input{childdoc.def}\childdocforward[|\textit{main}|]{|\textit{dest}|}"|
\end{center}
%
Here \textit{target} is the name of the output file,
\textit{main} is the name of the main file
and \textit{dest} is the name of the main or child file to be processed
(all filenames without extensions).
The optional argument \textit{main} can be omitted
if \textit{main} matches \textit{dest}.
Optionally, compilation \textit{flags} can be defined via |\def| commands.
This command line makes the \TeX{} engine believe
it is compiling the file \textit{target}
whose content is specified as the latter parameter.
The provided code then forwards the processing to
\textit{main} or \textit{dest} as described in \secref{sec:forward}.

%%%%%%%%%%%%%%%%%%%%%%%%%%%%%%%%%%%%%%%%%%%%%%%%%%%%%%%%%%%%%%%%%%%%%%%%%%%%%%%%
\subsection{Include by Input}
\label{sec:input}

Including child documents by |\include| has some restrictions by design.
Most notably, the content of a child document always occupies
its own set of pages; pages cannot be shared between child documents.
Usually, this behaviour makes perfect sense
because each child document contain an essential part of the document.
However, in some situations it may be desirable to compose
a document from a collection of parts
without having mandatory page breaks between then.
For this case, the package
provides a mechanism to include parts
by |\input| which can also be processed individually.
However, by construction this mechanism
requires manual handling of the content to be output.

%%%%%%%%%%%%%%%%%%%%%%%%%%%%%%%%%%%%%%%%
\DescribeMacro{\ifchilddocmanual}
The main file should be prepared as usual, see \secref{sec:include}.
However, the document body must make a distinction
between processing of an individual part and of the main document, e.g.:
%
\begin{center}
\begin{tabular}{l}
|\ifchilddocmanual|\\
|\input{\childdocname}|\\
|\||else|\\
\textit{document body with }|\input{|\textit{part}|}|\\
|\||fi|
\end{tabular}
\end{center}
%
The conditional |\ifchilddocmanual| is true whenever
a part to be included by |\input| is being compiled,
and the name of the part is stored in |\childdocname|.

%%%%%%%%%%%%%%%%%%%%%%%%%%%%%%%%%%%%%%%%
\DescribeMacro{\childdocby}
Each part to be included by |\input| should start with:
%
\begin{center}
\begin{tabular}{l}
|\input{childdoc.def}|\\
|\childdocby{|\textit{main}|}|\\
\end{tabular}
\end{center}
%
The directive |\childdocby| is similar to |\childdocof|
described in \secref{sec:include},
but the subsequent selection of content must be done manually.
To that end, both |\ifchilddoc| and |\ifchilddocmanual|
will be true upon processing of a part,
and the name of the part is stored in |\childdocname|.
Note that |\jobname| will be set to the filename of the current part
so that each part receives an individual |.aux| file
that does not interfere with the |.aux| file(s) of the main document.
This behaviour can be altered by the alternative form
|\childdocby[*]{|\textit{main}|}| (with a non-empty optional argument)
which uses the |.aux| file of the main document
by setting |\jobname| to \textit{main}.

%%%%%%%%%%%%%%%%%%%%%%%%%%%%%%%%%%%%%%%%%%%%%%%%%%%%%%%%%%%%%%%%%%%%%%%%%%%%%%%%
\subsection{Driver Development}
\label{sec:driver}

The \textsf{childdoc} mechanism can also be use for the development
of definition files such as \LaTeX{} styles or classes.
This case differs from the above setup with multiple parts
included by |\include| in that no |\includeonly| should be invoked.
This can be achieved by starting the include file
(before |\ProvidesPackage|) with:
%
\begin{center}
\begin{tabular}{l}
|\input{childdoc.def}|\\
|\childdocforward{|\textit{main}|}|\\
\end{tabular}
\end{center}
%
or alternatively with:
%
\begin{center}
\begin{tabular}{l}
|\input{childdoc.def}|\\
|\childdocby{|\textit{main}|}|\\
\end{tabular}
\end{center}
%
Both forms have slightly different effects as described above.
The main file is prepared as usual, see \secref{sec:include}.

%%%%%%%%%%%%%%%%%%%%%%%%%%%%%%%%%%%%%%%%%%%%%%%%%%%%%%%%%%%%%%%%%%%%%%%%%%%%%%%%
\subsection{Legacy Detection}
\label{sec:detection}

The directive |\childdocmain| in the main file can detect
whether the complete document or merely a child is to be compiled
even without using the directive |\childdocof|.
This method is deprecated because it is less robust
and there is no compelling reason to use it;
it is merely provided for backward compatibility
and it may be removed in future versions.

If the detection mechanism is to be used,
it is mandatory to correctly specify
the filename of the main file as the argument of |\childdocmain|:
%
\begin{center}
\begin{tabular}{l}
|\input{childdoc.def}|\\
|\childdocmain{|\textit{main}|}|\\
\end{tabular}
\end{center}
%
If |\jobname| does not match the argument \textit{main} of |\childdocmain|,
it is assumed that |\jobname| points to the child file to be compiled.
When using |\childdocmain| with the main file specified as argument,
it suffices to start a child file
with just |\input{|\textit{main}|}|
without loading of the package and using |\childdocof|.
If instead all processing is done
with the appropriate \textsf{childdoc} directives,
the argument of \textit{main} of |\childdocmain| can be empty.

An alternative version of the command line processing described
in \secref{sec:commandline} using the detection mechanism reads:
%
\begin{center}
|... -jobname "|\textit{target}|" "|[\textit{flags}]%
[|\def\jobname{|\textit{dest}|}|]|\input{|\textit{main}|}"|
\end{center}

%%%%%%%%%%%%%%%%%%%%%%%%%%%%%%%%%%%%%%%%%%%%%%%%%%%%%%%%%%%%%%%%%%%%%%%%%%%%%%%%
\subsection{Manual Code}
\label{sec:manual}

In case one cannot be certain whether the definitions file |childdoc.def|
is installed on the target \TeX{} distribution
and one prefers not to ship it,
it is conceivable to paste a few relevant commands into the sources.

To that end, drop all statements |\input{childdoc.def}|
and perform the replacements as outlined below.
Instead of |\childdocmain{|\textit{main}|}| add the following code
to the top of the main file:
%
\begin{center}
\begin{tabular}{l}
|\||ifdefined\childdocname\endinput\||fi\newif\ifchilddoc|\\
|\edef\childdocname{\scantokens\expandafter{\jobname\noexpand}}|\\
|\def\childdocmain{|\textit{main}|}\||ifx\childdocmain\childdocname\||else|\\
|\childdoctrue\includeonly{\childdocname}\let\jobname\childdocmain\||fi|\\
\end{tabular}
\end{center}
%
Instead of |\childdocof{|\textit{main}|}| just include the main file
at the top of each child file:
%
\begin{center}
|\input{|\textit{main}|}|
\end{center}
%
A simple redirection |\childdocforward{|\textit{dest}|}| is achieved by:
%
\begin{center}
|\def\jobname{|\textit{dest}|}\input{\jobname}|
\end{center}
%
The redirection with prefix
|\childdocforwardprefix[|\textit{prefix}|]{|\textit{dest}|}|
is accomplished by:
%
\begin{center}
\begin{tabular}{l}
|{\edef\jobname{\scantokens\expandafter{\jobname\noexpand}}|\\
|\def\redirectjob |\textit{prefix}|#1~~~{\gdef\jobname{|\textit{dest}|#1}}|\\
|\expandafter\redirectjob\jobname~~~}\input{\jobname}|
\end{tabular}
\end{center}

In an alternative approach,
child documents can be compiled by a specific command line
without additional code or specific definitions:
%
\begin{center}
|... -jobname "|\textit{target}|" "|[\textit{flags}]%
|\includeonly{|\textit{dest}|}\input{|\textit{main}|}"|
\end{center}
%

%%%%%%%%%%%%%%%%%%%%%%%%%%%%%%%%%%%%%%%%%%%%%%%%%%%%%%%%%%%%%%%%%%%%%%%%%%%%%%%%
%%%%%%%%%%%%%%%%%%%%%%%%%%%%%%%%%%%%%%%%%%%%%%%%%%%%%%%%%%%%%%%%%%%%%%%%%%%%%%%%
\section{Information}

%%%%%%%%%%%%%%%%%%%%%%%%%%%%%%%%%%%%%%%%%%%%%%%%%%%%%%%%%%%%%%%%%%%%%%%%%%%%%%%%
\subsection{Copyright}

Copyright \copyright{} 2017--2018 Niklas Beisert

This work may be distributed and/or modified under the
conditions of the \LaTeX{} Project Public License, either version 1.3
of this license or (at your option) any later version.
The latest version of this license is in
  \url{http://www.latex-project.org/lppl.txt}
and version 1.3 or later is part of all distributions of \LaTeX{}
version 2005/12/01 or later.

This work has the LPPL maintenance status `maintained'.

The Current Maintainer of this work is Niklas Beisert.

This work consists of the files |README.txt|, |childdoc.ins| and |childdoc.dtx|
as well as the derived files |childdoc.def|, |cdocsamp.tex|
with |cdocsch1.tex|, |cdocsch2.tex|, |cdocspt3.tex|, |cdocspt4.tex|,
|cdocsdrf.tex|, |cdocsfn1.tex|, |cdocsfn2.tex|
as well as |childdoc.pdf|.

%%%%%%%%%%%%%%%%%%%%%%%%%%%%%%%%%%%%%%%%%%%%%%%%%%%%%%%%%%%%%%%%%%%%%%%%%%%%%%%%
\subsection{Files and Installation}

The package consists of the files:
%
\begin{center}
\begin{tabular}{ll}
    |README.txt|   & readme file \\
    |childdoc.ins| & installation file \\
    |childdoc.dtx| & source file \\
    |childdoc.def| & definition file \\
    |cdocsamp.tex| & sample main file \\
    |cdocsch1.tex| & sample include file \\
    |cdocsch2.tex| & sample include file \\
    |cdocspt3.tex| & sample part file \\
    |cdocspt4.tex| & sample part file \\
    |cdocsdrf.tex| & sample redirection file \\
    |cdocsfn1.tex| & sample redirection file \\
    |cdocsfn2.tex| & sample redirection file \\
    |childdoc.pdf| & manual
\end{tabular}
\end{center}
%
The distribution consists of the files
|README.txt|, |childdoc.ins| and |childdoc.dtx|.
%
\begin{itemize}
\item
Run (pdf)\LaTeX{} on |childdoc.dtx|
to compile the manual |childdoc.pdf| (this file).
\item
Run \LaTeX{} on |childdoc.ins| to create the definitions file |childdoc.def|
and the sample |cdocsamp.tex| with include files
|cdocsch1.tex|, |cdocsch2.tex|, |cdocspt3.tex|, |cdocspt4.tex|,
|cdocsdrf.tex|, |cdocsfn1.tex|, |cdocsfn2.tex|.
Then copy the file |childdoc.def| to an appropriate directory of your \LaTeX{}
distribution, e.g.\ \textit{texmf-root}|/tex/latex/childdoc|.
\end{itemize}

%%%%%%%%%%%%%%%%%%%%%%%%%%%%%%%%%%%%%%%%%%%%%%%%%%%%%%%%%%%%%%%%%%%%%%%%%%%%%%%%
\subsection{Related CTAN Packages}

There are several other packages which offer a similar functionality:
%
\begin{itemize}
\item
The packages
\href{http://ctan.org/pkg/docmute}{\textsf{docmute}},
\href{http://ctan.org/pkg/includex}{\textsf{includex}} and
\href{http://ctan.org/pkg/standalone}{\textsf{standalone}}
provide commands to include only the document body of
a child file thus allowing both files to be compiled individually.
\item
The packages \href{http://ctan.org/pkg/subdocs}{\textsf{subdocs}}
and \href{http://ctan.org/pkg/subfiles}{\textsf{subfiles}}
provide structures in which the main and child documents can be
encapsulated and allowing them to be compiled individually.
The inclusion mechanism is different from the conventional |\include|.
\item
The package \href{http://ctan.org/pkg/combine}{\textsf{combine}}
is an elaborate solution to combine several documents into one.
\end{itemize}
%
See also the CTAN topic \href{http://ctan.org/topic/subdocs}{\textsf{subdocs}}
for further related packages.
The present package differs from the above solutions in that
a document structure constructed with the conventional |\include| mechanism
just needs two extra commands at the top of every file
such that all constituent files can be compiled individually.

%%%%%%%%%%%%%%%%%%%%%%%%%%%%%%%%%%%%%%%%%%%%%%%%%%%%%%%%%%%%%%%%%%%%%%%%%%%%%%%%
%\subsection{Feature Suggestions}
%
%The following is a list of features which may be useful for future
%versions of this package:
%%
%\begin{itemize}
%\item
%\ldots
%\end{itemize}

%%%%%%%%%%%%%%%%%%%%%%%%%%%%%%%%%%%%%%%%%%%%%%%%%%%%%%%%%%%%%%%%%%%%%%%%%%%%%%%%
\subsection{Revision History}

%%%%%%%%%%%%%%%%%%%%%%%%%%%%%%%%%%%%%%%%
\paragraph{v2.0:} 2018/12/30

\begin{itemize}
\item
immediate forward processing
\item
added |\childdocby| mechanism
\item
manual restructured
\end{itemize}

%%%%%%%%%%%%%%%%%%%%%%%%%%%%%%%%%%%%%%%%
\paragraph{v1.6:} 2018/01/17

\begin{itemize}
\item
application for development of include files
\item
corrections to manual
\end{itemize}

%%%%%%%%%%%%%%%%%%%%%%%%%%%%%%%%%%%%%%%%
\paragraph{v1.5:} 2017/05/21

\begin{itemize}
\item
more complete structuring introduced
\item
|\childdocof| introduced
\item
|\childdoc| renamed to |\childdocmain|
\item
|\childredirect| renamed to |\childdocforward| and |\childdocforwardprefix|
and functionality expanded
\end{itemize}

%%%%%%%%%%%%%%%%%%%%%%%%%%%%%%%%%%%%%%%%
\paragraph{v1.0:} 2017/04/27

\begin{itemize}
\item
manual and install package
\item
first version published on CTAN
\end{itemize}

%%%%%%%%%%%%%%%%%%%%%%%%%%%%%%%%%%%%%%%%
\paragraph{v0.6:} 2017/04/26

\begin{itemize}
\item
redirection mechanism added
\end{itemize}

%%%%%%%%%%%%%%%%%%%%%%%%%%%%%%%%%%%%%%%%
\paragraph{v0.5:} 2017/04/26

\begin{itemize}
\item
functionality in definition file
\end{itemize}


%%%%%%%%%%%%%%%%%%%%%%%%%%%%%%%%%%%%%%%%%%%%%%%%%%%%%%%%%%%%%%%%%%%%%%%%%%%%%%%%
%%%%%%%%%%%%%%%%%%%%%%%%%%%%%%%%%%%%%%%%%%%%%%%%%%%%%%%%%%%%%%%%%%%%%%%%%%%%%%%%
%%%%%%%%%%%%%%%%%%%%%%%%%%%%%%%%%%%%%%%%%%%%%%%%%%%%%%%%%%%%%%%%%%%%%%%%%%%%%%%%
\appendix

\settowidth\MacroIndent{\rmfamily\scriptsize 000\ }

 \DocInput{childdoc.dtx}

\end{document}
%</driver>
% \fi
%
% %%%%%%%%%%%%%%%%%%%%%%%%%%%%%%%%%%%%%%%%%%%%%%%%%%%%%%%%%%%%%%%%%%%%%%%%%%%%%%
% %%%%%%%%%%%%%%%%%%%%%%%%%%%%%%%%%%%%%%%%%%%%%%%%%%%%%%%%%%%%%%%%%%%%%%%%%%%%%%
% \section{Sample}
%\iffalse
%<*samplemain>
%\fi
%
% The following presents a sample document
% with two chapters, two parts, a title page,
% a compile flag as well as three forwarding files to set the flag.
% It consists of eight |.tex| files:
% \begin{center}
% \begin{tabular}{ll}
% |cdocsamp.tex|&main file\\
% |cdocsch1.tex|&include file for chapter 1\\
% |cdocsch2.tex|&include file for chapter 2\\
% |cdocspt3.tex|&include file for part 3\\
% |cdocspt4.tex|&include file for part 4\\
% |cdocsdrf.tex|&forwarding file for main file in draft mode\\
% |cdocsfi1.tex|&forwarding file for final version of chapter 1\\
% |cdocsfi2.tex|&forwarding file for final version of chapter 2\\
% \end{tabular}
% \end{center}
% Each of the eight files can be compiled directly by the \LaTeX{} compiler.
%
% %%%%%%%%%%%%%%%%%%%%%%%%%%%%%%%%%%%%%%
% \paragraph{Main File.}
%
% The main file is called |cdocsamp.tex|.
%
% Load the \textsf{childdoc} definitions and
% declare the filename for the main document:
%    \begin{macrocode}
\input{childdoc.def}
\childdocmain{}
%    \end{macrocode}

% Optional override for |\version| flag:
%    \begin{macrocode}
%%\ifchilddoc\else\providecommand{\version}{draft}\fi
%    \end{macrocode}

% Define the default values for the |\version| flag
% (|final| for the main file and |draft| for childs):
%    \begin{macrocode}
\ifchilddoc
\providecommand{\version}{draft}
\else
\providecommand{\version}{final}
\fi
%    \end{macrocode}

% Load the standard document class:
%    \begin{macrocode}
\documentclass[12pt]{article}
%    \end{macrocode}

% Start the document body:
%    \begin{macrocode}
\begin{document}
%    \end{macrocode}

% Declare a title page.
% Print title, part of document being processed and version flag:
%    \begin{macrocode}
\addtocounter{page}{-1}
\begin{center}
{\LARGE\bfseries{}childdoc example\par}
\vspace{1cm}
\ifchilddoc
\ifchilddocmanual part\else chapter\fi:
`\childdocname' of `\childdocjob'\par
\else
main document: `\childdocjob'\par
\fi
version: \version\par
\end{center}
\newpage
%    \end{macrocode}

% Manually include selected file,
% otherwise process as usual:
%    \begin{macrocode}
\ifchilddocmanual
\section*{part `\childdocname'}
\input{\childdocname}
\else
%    \end{macrocode}

% Include the two chapters:
%    \begin{macrocode}
\include{cdocsch1}
\include{cdocsch2}
%    \end{macrocode}

% Include the two parts unless only chapters should be displayed:
%    \begin{macrocode}
\ifchilddoc\else
\section{part three}
\input{cdocspt3}
\section{part four}
\input{cdocspt4}
\fi
%    \end{macrocode}

% Process as usual until here:
%    \begin{macrocode}
\fi
%    \end{macrocode}

% End of document body:
%    \begin{macrocode}
\end{document}
%    \end{macrocode}
%\iffalse
%</samplemain>
%\fi
%
% %%%%%%%%%%%%%%%%%%%%%%%%%%%%%%%%%%%%%%
% \paragraph{Chapter Include Files.}
%
% The include files are called |cdocsch1.tex| and |cdocsch2.tex|.
%
%\iffalse
%<*samplechap1|samplechap2>
%\fi

% Optional override for |\version| flag:
%    \begin{macrocode}
%%\providecommand{\version}{final}
%    \end{macrocode}

% Include the main document:
%    \begin{macrocode}
\input{childdoc.def}
\childdocof{cdocsamp}
%    \end{macrocode}

%\iffalse
%</samplechap1|samplechap2>
%\fi
%
%\iffalse
%<*samplechap1>
%\fi
% Some text for chapter 1:
%    \begin{macrocode}
\section{one}
some text in chapter one
%    \end{macrocode}

%\iffalse
%</samplechap1>
%\fi
% Some text for chapter 2:
%\iffalse
%<*samplechap2>
%\fi
%    \begin{macrocode}
\section{two}
more text in chapter two
%    \end{macrocode}

%\iffalse
%</samplechap2>
%\fi
%
% %%%%%%%%%%%%%%%%%%%%%%%%%%%%%%%%%%%%%%
% \paragraph{Part Include Files.}
%
% The include files are called |cdocspt3.tex| and |cdocspt4.tex|.
%
%\iffalse
%<*samplepart3|samplepart4>
%\fi

% Optional override for |\version| flag:
%    \begin{macrocode}
%%\providecommand{\version}{final}
%    \end{macrocode}

% Include the main document:
%    \begin{macrocode}
\input{childdoc.def}
\childdocby{cdocsamp}
%    \end{macrocode}

%\iffalse
%</samplepart3|samplepart4>
%\fi
%
%\iffalse
%<*samplepart3>
%\fi
% Some text for part 3:
%    \begin{macrocode}
some text in part three
%    \end{macrocode}

%\iffalse
%</samplepart3>
%\fi
% Some text for part 4:
%\iffalse
%<*samplepart4>
%\fi
%    \begin{macrocode}
more text in part four
%    \end{macrocode}

%\iffalse
%</samplepart4>
%\fi
%
% %%%%%%%%%%%%%%%%%%%%%%%%%%%%%%%%%%%%%%
% \paragraph{Forwarding for a Complete Draft.}
%
% The following forwarding file |cdocsdrf.tex|
% compiles the main document in draft mode:
%\iffalse
%<*sampledraft>
%\fi
%    \begin{macrocode}
\def\version{draft}
\input{childdoc.def}
\childdocforward{cdocsamp}
%    \end{macrocode}

%\iffalse
%</sampledraft>
%\fi
%
% %%%%%%%%%%%%%%%%%%%%%%%%%%%%%%%%%%%%%%
% \paragraph{Forwarding for Final Version of the Chapters.}
%
% The following forwarding files |cdocsfn1.tex| and |cdocsfn2.tex|
% (with identical content)
% compile the final versions of the child documents
% |cdocsch1.tex| and |cdocsch2.tex|, respectively:
%\iffalse
%<*samplefinal>
%\fi
%    \begin{macrocode}
\def\version{final}
\input{childdoc.def}
\childdocforwardprefix[cdocsamp]{cdocsfn}{cdocsch}
%    \end{macrocode}

%\iffalse
%</samplefinal>
%\fi
%
% %%%%%%%%%%%%%%%%%%%%%%%%%%%%%%%%%%%%%%
% \paragraph{Command Line Processing.}
%
% The following three command lines generate the output files
% |cdocscld|, |cdocscl1| and |cdocscl2|
% which should be identical to
% |cdocsdrf|, |cdocsch1| and |cdocsfn2|, respectively:
% \begin{center}
% \begin{tabular}{l}
% |latex -jobname cdocscld \|\\
% |  "\def\version{draft}\input{childdoc.def}\childdocforward{cdocsamp}"|\\
% |latex -jobname cdocscl1 \|\\
% |  "\input{childdoc.def}\childdocforward[cdocsamp]{cdocsch1}"|\\
% |latex -jobname cdocscl2 \|\\
% |  "\def\version{final}\input{childdoc.def}\childdocforward{cdocsch2}"|
% \end{tabular}
% \end{center}
% Note that the trailing backslash on each first line
% merely continues the input to the second line
% (for convenient cut ant paste).
% Furthermore, the command |latex| can be replaced by any
% of its alternative versions such as |pdflatex|.
%
% %%%%%%%%%%%%%%%%%%%%%%%%%%%%%%%%%%%%%%%%%%%%%%%%%%%%%%%%%%%%%%%%%%%%%%%%%%%%%%
% %%%%%%%%%%%%%%%%%%%%%%%%%%%%%%%%%%%%%%%%%%%%%%%%%%%%%%%%%%%%%%%%%%%%%%%%%%%%%%
% \section{Implementation}
%\iffalse
%<*package>
%\fi
%
% This section describes the definitions file |childdoc.def|.

% The definitions cannot be loaded using |\usepackage| or |\RequirePackage|
% which has a mechanism to prevent loading a style file more than once.
% When loading the definitions by means of |\input|
% multiple instances have to be prevented manually:
%\iffalse
%This code needs to be before the `\ProvidesFile' directive
%which is defined at the beginning of this file.
%Therefore it is also placed there and commented out here.
%</package>
%<*discard>
%\fi
%    \begin{macrocode}
\ifdefined\childdocmain\endinput\fi
%    \end{macrocode}
%\iffalse
%</discard>
%<*package>
%\fi
%
% \macro{\ifchilddoc}
% \macro{\ifchilddocmanual}
% The conditional |\ifchilddoc| tells whether a
% child (true) or main (false) document is being compiled.
% The conditional |\ifchilddocmanual| tells whether
% the |\includeonly| mechanism is used (false) or
% the selection of child files must be performed manually (true).
% The definitions initialise to false:
%    \begin{macrocode}
\newif\ifchilddoc
\newif\ifchilddocmanual
%    \end{macrocode}

% \macro{\childdocname}
% \macro{\childdocjob}
% The macro |\childdocname| stores the name of the main document
% to be compiled. The macro |\childdocjob| stores the name of
% the document on which the \LaTeX{} compiler was originally invoked.
% The content of |\jobname| cannot be compared
% to filenames specified in the source due to different catcodes.
% The following code rescans |\jobname|, stores the result
% in |\childdocname| and saves a copy in |\childdocjob|:
%    \begin{macrocode}
\edef\childdocname{\scantokens\expandafter{\jobname\noexpand}}
\let\childdocjob\childdocname
%    \end{macrocode}

% \macro{\childdocdisable}
% The macro |\childdocdisable| prevents the main file
% from being processed more than once.
% At this stage, the main document command |\childdocmain|
% is assumed to be called once again where it should do nothing.
% Any subsequent call to it should prevent
% a secondary processing of the main document
% It overwrites the forwarding commands
% |\childdocof| and |\childdocforward|
% with empty macros to prevent further inclusions of the main document:
%    \begin{macrocode}
\newcommand{\childdocdisable}
{
  \renewcommand{\childdocmain}[1]{\renewcommand{\childdocmain}[1]{\endinput}}
  \renewcommand{\childdocof}[1]{}
  \renewcommand{\childdocby}[2][]{}
  \renewcommand{\childdocforward}[2][]{}
  \renewcommand{\childdocdisable}{}
}
%    \end{macrocode}

% \macro{\childdocmain}
% The macro |\childdocmain| is to be called at the top of the main file
% with nothing or the main filename (without extension) as argument.
% First, it breaks loops.
% If the argument is not empty and does not match |\childdocname|
% (which is set by the first inclusion of |childdoc.def|),
% |\ifchilddoc| is set to true, |\includeonly| is applied to the child file
% and |\jobname| is set to the main file
% (for proper handling of |.aux| files):
%    \begin{macrocode}
\newcommand{\childdocmain}[1]
{
  \childdocdisable\childdocmain{}
  \if?#1?\else
    \begingroup
      \def\childdoctmp{#1}
      \ifx\childdoctmp\childdocname
        \def\childdoctmp{}
      \else
        \def\childdoctmp
        {
          \childdoctrue
          \includeonly{\childdocname}
          \def\childdocjob{#1}
          \def\jobname{#1}
        }
      \fi
      \expandafter
    \endgroup
    \childdoctmp
  \fi
}
%    \end{macrocode}

% \macro{\childdocof}
% The command |\childdocof| redirects
% compilation to the main file |#1|.
%    \begin{macrocode}
\newcommand{\childdocof}[1]
{
  \childdocdisable
  \childdoctrue
  \includeonly{\childdocname}
  \def\jobname{#1}
  \def\childdocjob{#1}
  \input{#1}
}
%    \end{macrocode}

% \macro{\childdocby}
% The command |\childdocby| ....
%    \begin{macrocode}
\newcommand{\childdocby}[2][]
{
  \childdocdisable
  \childdoctrue
  \childdocmanualtrue
  \if?#1?\else
    \def\jobname{#2}
  \fi
  \def\childdocjob{#2}
  \input{#2}
  \endinput
}
%    \end{macrocode}

% \macro{\childdocforward}
% The command |\childdocforward| redirects
% compilation to the main file or
% (if the optional argument is given) a child file.
% Parameters are set as if the main file
% or a child file starting with |\childdocof| was compiled.
% Then compilation is handed over to the main file:
%    \begin{macrocode}
\newcommand{\childdocforward}[2][]
{
  \begingroup
    \if?#1?
      \def\childdoctmp
      {
        \def\childdocname{#2}
        \def\childdocjob{#2}
        \def\jobname{#2}
        \input{#2}
        \endinput
      }
    \else
      \def\childdoctmp
      {
        \childdocdisable
        \def\childdocname{#2}
        \childdoctrue
        \includeonly{#2}
        \def\childdocjob{#1}
        \def\jobname{#1}
        \input{#1}
        \endinput
      }
    \fi
    \expandafter
  \endgroup
  \childdoctmp
}
%    \end{macrocode}

% \macro{\childdocforwardprefix}
% The command |\childdocforwardprefix| redirects
% compilation to the main or a child file by means of a pattern.
% The prefix |#1| in the current filename is replaced by |#2|
% and the suffix of the current filename is kept
% (it is assumed that the filename does not contain the substring `|~~~|'
% which is used as a delimiter).
% Compilation is handed over to the new file by |\childdocforward|:
%    \begin{macrocode}
\newcommand{\childdocforwardprefix}[3][]
{
  \begingroup
    \def\childdocextract #2##1~~~{\def\childdoctmp{\childdocforward[#1]{#3##1}}}
    \expandafter\childdocextract\childdocname~~~
    \expandafter
  \endgroup
  \childdoctmp
}
%    \end{macrocode}

% \macro{\childdoc}
% The deprecated macro |\childdoc| is a legacy version of |\childdocmain|:
%    \begin{macrocode}
\newcommand{\childdoc}{\childdocmain}
%    \end{macrocode}

% \macro{\childdocredirect}
% The deprecated macro |\childdocredirect| is a legacy version
% of |\childdocforward| and |\childdocforwardprefix|:
%    \begin{macrocode}
\newcommand{\childdocredirect}[2][]
{
  \begingroup
    \if?#1?
      \def\childdoctmp{\childdocforward{#2}}
    \else
      \def\childdoctmp{\childdocforwardprefix{#1}{#2}}
    \fi
    \expandafter
  \endgroup
  \childdoctmp
}
%    \end{macrocode}

%\iffalse
%</package>
%\fi
%
\endinput
|\\
|\childdocof{|\textit{main}|}|\\
\end{tabular}
\end{center}
at the top of every child file \textit{child}
which is included by |\include{|\textit{child}|}|
from within the main file
(or at least for those files to be compiled individually).
The argument \textit{main} must be the filename of the main file.

There are a couple of
considerations in setting up the main and child documents:

%%%%%%%%%%%%%%%%%%%%%%%%%%%%%%%%%%%%%%%%
\paragraph{Restrictions.}

Please note the following restrictions:
\begin{itemize}
\item
|\childdocmain| must be called with one argument \textit{main}
to ensure compatibility with earlier version of the package.
It must either be empty (|\childdocmain{}|)
or precisely match the filename of the main file in which it is specified.
See \secref{sec:detection} for further information.
\item
The filename \textit{main} must be specified without the |.tex| extension.
\item
The filename \textit{main} is case sensitive
(even in case-insensitive file systems)
due to internal string comparison.
\item
The argument \textit{main} should be fully expanded, it cannot be a macro.
\item
Subdirectories and special characters should be avoided in filenames.
\item
The command |\childdocmain{|\textit{main}|}| must be followed by a whitespace.
It should not be followed immediately by another command
or by a comment mark `|%|'.
This is because the \TeX{} parser reads the token immediately following
the argument of |\childdocmain| and puts it
at the beginning of every child section;
however, a white\-space is ignored.
\end{itemize}

%%%%%%%%%%%%%%%%%%%%%%%%%%%%%%%%%%%%%%%%
\paragraph{Content of Main File.}

It is advisable to place all content in the child files included by |\include|.
Any output contained in the main file will appear in all child documents
unless suppressed manually;
it cannot be suppressed automatically by the |\includeonly| directive
and thus should normally be avoided.
A method to include some content in the main file
by means of conditional processing is described in \secref{sec:conditional}.

%%%%%%%%%%%%%%%%%%%%%%%%%%%%%%%%%%%%%%%%
\paragraph{Page Numbering.}

When only a part of the document is compiled,
the appropriate numbering of pages
(as well as other status parameters)
is determined from the |.aux| files.
The latter contain information from previous passes.
However this information needs to propagate through
all intermediate child documents.
Therefore the page numbering in child documents may well
be inconsistent until the complete document is compiled at least once.

A useful (if unconventional) way to always ensure a consistent
page numbering is to restart the numbering in each child document
and denote the pages by `\textit{child}|.|\textit{page}'
where \textit{child} represents the chapter/section number of the child file.
This can be achieved by the command
|\numberwithin{page}{|\textit{child}|}|
of the \textsf{amsmath} package
where \textit{child} can be |chapter| or |section|
depending on the chosen structuring.
Alternatively, one can modify the macro |\thepage| appropriately
and reset the counter |page| at the start of each child file.

%%%%%%%%%%%%%%%%%%%%%%%%%%%%%%%%%%%%%%%%%%%%%%%%%%%%%%%%%%%%%%%%%%%%%%%%%%%%%%%%
\subsection{Conditional Processing}
\label{sec:conditional}

The package provides a mechanism to compile different versions
of a document. To customise the versions further some conditional processing
can come in handy to distinguish which version is being compiled.
The package provides two macros to describe the compilation context:

%%%%%%%%%%%%%%%%%%%%%%%%%%%%%%%%%%%%%%%%
\DescribeMacro{\ifchilddoc}
The conditional |\ifchilddoc| distinguishes between the compilation of
child documents and the main document:
%
\begin{center}
|\ifchilddoc |\textit{child-code}| |[|\||else |\textit{main-code}]| \||fi|
\end{center}

%%%%%%%%%%%%%%%%%%%%%%%%%%%%%%%%%%%%%%%%
\DescribeMacro{\childdocname}
\DescribeMacro{\childdocjob}
The macro |\childdocname| contains the filename (without extension)
of the main or child file being processed.
Note that |\childdocjob| will always contain the name of the main file.

%%%%%%%%%%%%%%%%%%%%%%%%%%%%%%%%%%%%%%%%
\paragraph{Title Page.}

Conditional processing can be used to include a title or banner page
in the main document when proper precautions are taken.
Importantly, the code in the main file should ensure that the page counter
(as well as other status parameters which are stored in the |.aux| files)
takes the same value after the conditional processing.
Otherwise the page numbers may take divergent values
depending on which part is compiled.

For example, a title page could be declared by:
%
\begin{center}
\begin{tabular}{l}
|\ifchilddoc\||else|\\
|\addtocounter{page}{-1}|\\
\textit{code for title page}\\
|\newpage|\\
|\||fi|
\end{tabular}
\end{center}
%
A banner page for the child documents can be generated by:
%
\begin{center}
\begin{tabular}{l}
|\ifchilddoc|\\
|\addtocounter{page}{-1}|\\
\textit{code for banner page}\\
|\newpage|\\
|\||fi|
\end{tabular}
\end{center}
%
Here one could write a message such as:
\begin{center}
|This is the part \childdocname{} of \childdocjob{}.|
\end{center}

%%%%%%%%%%%%%%%%%%%%%%%%%%%%%%%%%%%%%%%%%%%%%%%%%%%%%%%%%%%%%%%%%%%%%%%%%%%%%%%%
\subsection{Flags}
\label{sec:flags}

The package makes it easy to generate different versions
of the main or child documents.
To this end compilation flags can be defined
and assigned different default values.
They will be particularly useful in conjunction
with the forwarding mechanism described in \secref{sec:forward}.

For example, it may be useful to have a flag |\version|
which can be set to |draft| or |final|.
The document source will contain some conditional code
depending on the value of |\version|.
Suppose further, the flag should default to |final| for the main file
and to |draft| for child files
which is a natural assignment for editing the document.
This is achieved by placing the following code
in the preamble of the main document
(below the |\childdocmain| directive):
%
\begin{center}
\begin{tabular}{l}
|\ifchilddoc|\\
|\providecommand{\version}{draft}|\\
|\||else|\\
|\providecommand{\version}{final}|\\
|\||fi|
\end{tabular}
\end{center}
%
The definition by |\providecommand| makes sure
that previous definitions are not overwritten.
Further statements |\providecommand{\version}{...}|
can thus be added before the above code to override it.

For the main file, one might add a line
(between |\childdocmain| and the above block)
%
\begin{center}
|%\ifchilddoc\||else\providecommand{\version}{draft}\||fi|
\end{center}
%
which can be uncommented to produce a draft version.
Likewise one can add a line to the very top of a child file
(above the |\childdocof{|\textit{main}|}| directive)
%
\begin{center}
|%\providecommand{\version}{final}|
\end{center}
%
which can be uncommented to produce the final version of this child document.

%%%%%%%%%%%%%%%%%%%%%%%%%%%%%%%%%%%%%%%%%%%%%%%%%%%%%%%%%%%%%%%%%%%%%%%%%%%%%%%%
\subsection{Forwarding}
\label{sec:forward}

Different versions of the main or child documents
using compilation flags as described in \secref{sec:flags}
can be (permanently) stored in different files
for convenient compilation, viewing and distribution.
To this end, the package defines a command
to pass on compilation to a different file:

%%%%%%%%%%%%%%%%%%%%%%%%%%%%%%%%%%%%%%%%
\DescribeMacro{\childdocforward}
The command |\childdocforward| redirects processing to
another source file:
%
\begin{center}
\begin{tabular}{l}
|% \iffalse
%
% childdoc.dtx Copyright (C) 2017-2018 Niklas Beisert
%
% This work may be distributed and/or modified under the
% conditions of the LaTeX Project Public License, either version 1.3
% of this license or (at your option) any later version.
% The latest version of this license is in
%   http://www.latex-project.org/lppl.txt
% and version 1.3 or later is part of all distributions of LaTeX
% version 2005/12/01 or later.
%
% This work has the LPPL maintenance status `maintained'.
%
% The Current Maintainer of this work is Niklas Beisert.
%
% This work consists of the files childdoc.dtx and childdoc.ins
% and the derived files childdoc.def and cdocsamp.tex with
% cdocsch1.tex, cdocsch2.tex, cdocsdrf.tex, cdocsfn1.tex, cdocsfn2.tex.
%
%<package>\ifdefined\childdocmain\endinput\fi
%<package>\ProvidesFile{childdoc.def}[2018/12/30 v2.0 child document driver]
%<samplemain>\ProvidesFile{cdocsamp.tex}[2018/12/30 v2.0 sample for childdoc]
%<*driver>
%\ProvidesFile{childdoc.drv}[2018/12/30 v2.0 childdoc reference manual file]
\PassOptionsToClass{10pt,a4paper}{article}
\documentclass{ltxdoc}

\usepackage[margin=35mm]{geometry}
\usepackage{hyperref}
\usepackage{hyperxmp}
\usepackage[usenames]{color}

\hypersetup{colorlinks=true}
\hypersetup{pdfstartview=FitH}
\hypersetup{pdfpagemode=UseNone}
\hypersetup{pdfsource={}}
\hypersetup{pdflang={en-UK}}
\hypersetup{pdfcopyright={Copyright 2017-2018 Niklas Beisert.
  This work may be distributed and/or modified under the
  conditions of the LaTeX Project Public License, either version 1.3
  of this license or (at your option) any later version.}}
\hypersetup{pdflicenseurl={http://www.latex-project.org/lppl.txt}}
\hypersetup{pdfcontactaddress={ETH Zurich, ITP, HIT K,
  Wolfgang-Pauli-Strasse 27}}
\hypersetup{pdfcontactpostcode={8093}}
\hypersetup{pdfcontactcity={Zurich}}
\hypersetup{pdfcontactcountry={Switzerland}}
\hypersetup{pdfcontactemail={nbeisert@itp.phys.ethz.ch}}
\hypersetup{pdfcontacturl={http://people.phys.ethz.ch/\xmptilde nbeisert/}}

\newcommand{\secref}[1]{\hyperref[#1]{section \ref*{#1}}}

\parskip1ex
\parindent0pt
\let\olditemize\itemize
\def\itemize{\olditemize\parskip0pt}

\begin{document}

\title{The \textsf{childdoc} Package}
\hypersetup{pdftitle={The childdoc Package}}
\author{Niklas Beisert\\[2ex]
  Institut f\"ur Theoretische Physik\\
  Eidgen\"ossische Technische Hochschule Z\"urich\\
  Wolfgang-Pauli-Strasse 27, 8093 Z\"urich, Switzerland\\[1ex]
  \href{mailto:nbeisert@itp.phys.ethz.ch}
  {\texttt{nbeisert@itp.phys.ethz.ch}}}
\hypersetup{pdfauthor={Niklas Beisert}}
\hypersetup{pdfsubject={Manual for the LaTeX2e Package childdoc}}
\date{30 December 2018, \textsf{v2.0}}
\maketitle

\begin{abstract}\noindent
\textsf{childdoc} is a \LaTeXe{} package
that enables the direct compilation
of document sections included by |\include|
to individual files.
\end{abstract}

\begingroup
\parskip0ex
\tableofcontents
\endgroup

%%%%%%%%%%%%%%%%%%%%%%%%%%%%%%%%%%%%%%%%%%%%%%%%%%%%%%%%%%%%%%%%%%%%%%%%%%%%%%%%
%%%%%%%%%%%%%%%%%%%%%%%%%%%%%%%%%%%%%%%%%%%%%%%%%%%%%%%%%%%%%%%%%%%%%%%%%%%%%%%%
\section{Introduction}

\LaTeX{} provides a mechanism to structure a large document (such as a book)
into a main file and several child files (containing the chapters)
using the |\include| command.
This mechanism is beneficial for documents
which span hundreds of pages in order to
make the source file(s) more manageable.
Moreover, compilation can be restricted to
selected child files by means of the |\includeonly| command.
The latter feature can be used to reduce the compilation time while editing
(this was significantly more useful in the earlier days of \LaTeX{})
or to generate a smaller document which is easier to navigate.
Another application of |\includeonly| is to generate
documents consisting of selected parts of the complete document.

However, there are a few drawbacks of the plain |\include| mechanism:
\begin{itemize}
\item
The child files cannot be compiled on their own,
they can only be compiled via the main file.
A naive editing environment
(such as a text editor with an option
to have the current file processed by \LaTeX)
may require one to switch to the main file before compiling;
attempting to compile the child file produces errors.
\item
The main file must be modified (each time)
to adjust the |\includeonly| command
to the present needs. This easily leaves the main file in a messy state.
\item
The generated document will always carry the filename
of the main document. This is inconvenient if
several child files are to be compiled and
to be kept for distribution.
\end{itemize}

The present package provides a simple interface
to make child files individually compilable by \LaTeX{}.
Compiling a child file then has the same effect as compiling
the main file with an |\includeonly| command
to select the appropriate child.
Moreover the generated document will carry the name of the child
rather than the main file.
This resolves all three above issues.

This feature is meant to make the editing of books,
thesis documents and lecture notes somewhat more convenient.
However, the package can also be used efficiently for
composing a series of documents (such as exercise sheets)
which are typically distributed individually.
It then assists the author in generating the individual documents
(potentially in different versions)
as well as a document containing the collected series.
Another application is in developing style files
or other kinds of included material
where compilation of the style file could redirect
to a sample or test file.

%%%%%%%%%%%%%%%%%%%%%%%%%%%%%%%%%%%%%%%%%%%%%%%%%%%%%%%%%%%%%%%%%%%%%%%%%%%%%%%%
%%%%%%%%%%%%%%%%%%%%%%%%%%%%%%%%%%%%%%%%%%%%%%%%%%%%%%%%%%%%%%%%%%%%%%%%%%%%%%%%
\section{Usage}

First of all, the package \textsf{childdoc} is \emph{not} a standard
\LaTeXe{} |.sty| style file! Therefore it needs to be invoked in
a non-standard way.

%%%%%%%%%%%%%%%%%%%%%%%%%%%%%%%%%%%%%%%%%%%%%%%%%%%%%%%%%%%%%%%%%%%%%%%%%%%%%%%%
\subsection{Included Files}
\label{sec:include}

%%%%%%%%%%%%%%%%%%%%%%%%%%%%%%%%%%%%%%%%
\DescribeMacro{\childdocmain}
To use the package, add the commands
\begin{center}
\begin{tabular}{l}
|\input{childdoc.def}|\\
|\childdocmain{}|\\
\end{tabular}
\end{center}
at the very top of the main \LaTeX{} file,
in particular \emph{before} the |\documentclass| statement!
The argument of |\childdocmain| should be left empty
(but it must be present).

%%%%%%%%%%%%%%%%%%%%%%%%%%%%%%%%%%%%%%%%
\DescribeMacro{\childdocof}
Furthermore, add the commands
\begin{center}
\begin{tabular}{l}
|\input{childdoc.def}|\\
|\childdocof{|\textit{main}|}|\\
\end{tabular}
\end{center}
at the top of every child file \textit{child}
which is included by |\include{|\textit{child}|}|
from within the main file
(or at least for those files to be compiled individually).
The argument \textit{main} must be the filename of the main file.

There are a couple of
considerations in setting up the main and child documents:

%%%%%%%%%%%%%%%%%%%%%%%%%%%%%%%%%%%%%%%%
\paragraph{Restrictions.}

Please note the following restrictions:
\begin{itemize}
\item
|\childdocmain| must be called with one argument \textit{main}
to ensure compatibility with earlier version of the package.
It must either be empty (|\childdocmain{}|)
or precisely match the filename of the main file in which it is specified.
See \secref{sec:detection} for further information.
\item
The filename \textit{main} must be specified without the |.tex| extension.
\item
The filename \textit{main} is case sensitive
(even in case-insensitive file systems)
due to internal string comparison.
\item
The argument \textit{main} should be fully expanded, it cannot be a macro.
\item
Subdirectories and special characters should be avoided in filenames.
\item
The command |\childdocmain{|\textit{main}|}| must be followed by a whitespace.
It should not be followed immediately by another command
or by a comment mark `|%|'.
This is because the \TeX{} parser reads the token immediately following
the argument of |\childdocmain| and puts it
at the beginning of every child section;
however, a white\-space is ignored.
\end{itemize}

%%%%%%%%%%%%%%%%%%%%%%%%%%%%%%%%%%%%%%%%
\paragraph{Content of Main File.}

It is advisable to place all content in the child files included by |\include|.
Any output contained in the main file will appear in all child documents
unless suppressed manually;
it cannot be suppressed automatically by the |\includeonly| directive
and thus should normally be avoided.
A method to include some content in the main file
by means of conditional processing is described in \secref{sec:conditional}.

%%%%%%%%%%%%%%%%%%%%%%%%%%%%%%%%%%%%%%%%
\paragraph{Page Numbering.}

When only a part of the document is compiled,
the appropriate numbering of pages
(as well as other status parameters)
is determined from the |.aux| files.
The latter contain information from previous passes.
However this information needs to propagate through
all intermediate child documents.
Therefore the page numbering in child documents may well
be inconsistent until the complete document is compiled at least once.

A useful (if unconventional) way to always ensure a consistent
page numbering is to restart the numbering in each child document
and denote the pages by `\textit{child}|.|\textit{page}'
where \textit{child} represents the chapter/section number of the child file.
This can be achieved by the command
|\numberwithin{page}{|\textit{child}|}|
of the \textsf{amsmath} package
where \textit{child} can be |chapter| or |section|
depending on the chosen structuring.
Alternatively, one can modify the macro |\thepage| appropriately
and reset the counter |page| at the start of each child file.

%%%%%%%%%%%%%%%%%%%%%%%%%%%%%%%%%%%%%%%%%%%%%%%%%%%%%%%%%%%%%%%%%%%%%%%%%%%%%%%%
\subsection{Conditional Processing}
\label{sec:conditional}

The package provides a mechanism to compile different versions
of a document. To customise the versions further some conditional processing
can come in handy to distinguish which version is being compiled.
The package provides two macros to describe the compilation context:

%%%%%%%%%%%%%%%%%%%%%%%%%%%%%%%%%%%%%%%%
\DescribeMacro{\ifchilddoc}
The conditional |\ifchilddoc| distinguishes between the compilation of
child documents and the main document:
%
\begin{center}
|\ifchilddoc |\textit{child-code}| |[|\||else |\textit{main-code}]| \||fi|
\end{center}

%%%%%%%%%%%%%%%%%%%%%%%%%%%%%%%%%%%%%%%%
\DescribeMacro{\childdocname}
\DescribeMacro{\childdocjob}
The macro |\childdocname| contains the filename (without extension)
of the main or child file being processed.
Note that |\childdocjob| will always contain the name of the main file.

%%%%%%%%%%%%%%%%%%%%%%%%%%%%%%%%%%%%%%%%
\paragraph{Title Page.}

Conditional processing can be used to include a title or banner page
in the main document when proper precautions are taken.
Importantly, the code in the main file should ensure that the page counter
(as well as other status parameters which are stored in the |.aux| files)
takes the same value after the conditional processing.
Otherwise the page numbers may take divergent values
depending on which part is compiled.

For example, a title page could be declared by:
%
\begin{center}
\begin{tabular}{l}
|\ifchilddoc\||else|\\
|\addtocounter{page}{-1}|\\
\textit{code for title page}\\
|\newpage|\\
|\||fi|
\end{tabular}
\end{center}
%
A banner page for the child documents can be generated by:
%
\begin{center}
\begin{tabular}{l}
|\ifchilddoc|\\
|\addtocounter{page}{-1}|\\
\textit{code for banner page}\\
|\newpage|\\
|\||fi|
\end{tabular}
\end{center}
%
Here one could write a message such as:
\begin{center}
|This is the part \childdocname{} of \childdocjob{}.|
\end{center}

%%%%%%%%%%%%%%%%%%%%%%%%%%%%%%%%%%%%%%%%%%%%%%%%%%%%%%%%%%%%%%%%%%%%%%%%%%%%%%%%
\subsection{Flags}
\label{sec:flags}

The package makes it easy to generate different versions
of the main or child documents.
To this end compilation flags can be defined
and assigned different default values.
They will be particularly useful in conjunction
with the forwarding mechanism described in \secref{sec:forward}.

For example, it may be useful to have a flag |\version|
which can be set to |draft| or |final|.
The document source will contain some conditional code
depending on the value of |\version|.
Suppose further, the flag should default to |final| for the main file
and to |draft| for child files
which is a natural assignment for editing the document.
This is achieved by placing the following code
in the preamble of the main document
(below the |\childdocmain| directive):
%
\begin{center}
\begin{tabular}{l}
|\ifchilddoc|\\
|\providecommand{\version}{draft}|\\
|\||else|\\
|\providecommand{\version}{final}|\\
|\||fi|
\end{tabular}
\end{center}
%
The definition by |\providecommand| makes sure
that previous definitions are not overwritten.
Further statements |\providecommand{\version}{...}|
can thus be added before the above code to override it.

For the main file, one might add a line
(between |\childdocmain| and the above block)
%
\begin{center}
|%\ifchilddoc\||else\providecommand{\version}{draft}\||fi|
\end{center}
%
which can be uncommented to produce a draft version.
Likewise one can add a line to the very top of a child file
(above the |\childdocof{|\textit{main}|}| directive)
%
\begin{center}
|%\providecommand{\version}{final}|
\end{center}
%
which can be uncommented to produce the final version of this child document.

%%%%%%%%%%%%%%%%%%%%%%%%%%%%%%%%%%%%%%%%%%%%%%%%%%%%%%%%%%%%%%%%%%%%%%%%%%%%%%%%
\subsection{Forwarding}
\label{sec:forward}

Different versions of the main or child documents
using compilation flags as described in \secref{sec:flags}
can be (permanently) stored in different files
for convenient compilation, viewing and distribution.
To this end, the package defines a command
to pass on compilation to a different file:

%%%%%%%%%%%%%%%%%%%%%%%%%%%%%%%%%%%%%%%%
\DescribeMacro{\childdocforward}
The command |\childdocforward| redirects processing to
another source file:
%
\begin{center}
\begin{tabular}{l}
|\input{childdoc.def}|\\
|\childdocforward[|\textit{main}|]{|\textit{dest}|}|\\
\end{tabular}
\end{center}
%
The argument \textit{dest} is the destination file
(without extension).
It should be the main file or one of the child files.
Note that further \textsf{childdoc} directives
such as |\childdocof| and |\childdocforward|
in the indicated file will be processed in this form.
The optional argument \textit{main}
passes on directly to the main file \textit{main}
while pretending to compile the child \textit{dest}.
This form behaves as if \textit{dest}
issues |\childdocof{|\textit{main}|}| right away,
and no further \textsf{childdoc} directives will be processed.

%%%%%%%%%%%%%%%%%%%%%%%%%%%%%%%%%%%%%%%%
\DescribeMacro{\...prefix}
In the alternative form |\childdocforwardprefix|,
%
\begin{center}
\begin{tabular}{l}
|\input{childdoc.def}|\\
|\childdocforwardprefix[|\textit{main}|]{|\textit{prefix}|}{|\textit{dest}|}|
\end{tabular}
\end{center}
%
the destination file is determined by a pattern
depending on the current file:
To make this work, the current file must be called
`{\textit{prefix}\hspace{0.2em}\textit{suffix}}'
with \textit{prefix} matching precisely the argument.
Processing is then passed on to the file
`{\textit{dest}\hspace{0.2em}\textit{suffix}}'.
Surely, the same effect is achieved by
directly specifying the
argument `{\textit{dest}\hspace{0.2em}\textit{suffix}}'
in the first form.
However, that requires to set up a different file
for each child. With the alternative form of the command
all these files can have exactly the same content
which simplifies setting them up and maintaining them.

For example, the following file |draft.tex|
with a compilation flag |\version| as described in \secref{sec:flags}
compiles the main document as a draft:
%
\begin{center}
\begin{tabular}{l}
|\def\version{draft}|\\
|\input{childdoc.def}|\\
|\childdocforward{|\textit{main}|}|
\end{tabular}
\end{center}
%
Likewise, the following files |final|\textit{nn}|.tex|
compile the final version of the child document
|child|\textit{nn}|.tex|:
%
\begin{center}
\begin{tabular}{l}
|\def\version{final}|\\
|\input{childdoc.def}|\\
|\childdocforwardprefix{final}{child}|
\end{tabular}
\end{center}
%

Note that when several versions of a main file and/or of each child file
are to be generated, it may be convenient to set up a |Makefile| or
shell script to automatise the process.

%%%%%%%%%%%%%%%%%%%%%%%%%%%%%%%%%%%%%%%%%%%%%%%%%%%%%%%%%%%%%%%%%%%%%%%%%%%%%%%%
\subsection{Command Line Processing}
\label{sec:commandline}

The effect of redirection files can also be achieved by invoking
the \LaTeX{} compiler with a more elaborate command line.
Most conveniently this should be done as part
of a shell script or a |Makefile|.

When using \textsf{childdoc} in the main file, the following
command lines effectively perform a redirection
(note that depending on the shell being used,
backslashes may have to be doubled: `|\|' $\to$ `|\\|'):
%
\begin{center}
|... -jobname "|\textit{target}|" |\\|"|[\textit{flags}]%
|\input{childdoc.def}\childdocforward[|\textit{main}|]{|\textit{dest}|}"|
\end{center}
%
Here \textit{target} is the name of the output file,
\textit{main} is the name of the main file
and \textit{dest} is the name of the main or child file to be processed
(all filenames without extensions).
The optional argument \textit{main} can be omitted
if \textit{main} matches \textit{dest}.
Optionally, compilation \textit{flags} can be defined via |\def| commands.
This command line makes the \TeX{} engine believe
it is compiling the file \textit{target}
whose content is specified as the latter parameter.
The provided code then forwards the processing to
\textit{main} or \textit{dest} as described in \secref{sec:forward}.

%%%%%%%%%%%%%%%%%%%%%%%%%%%%%%%%%%%%%%%%%%%%%%%%%%%%%%%%%%%%%%%%%%%%%%%%%%%%%%%%
\subsection{Include by Input}
\label{sec:input}

Including child documents by |\include| has some restrictions by design.
Most notably, the content of a child document always occupies
its own set of pages; pages cannot be shared between child documents.
Usually, this behaviour makes perfect sense
because each child document contain an essential part of the document.
However, in some situations it may be desirable to compose
a document from a collection of parts
without having mandatory page breaks between then.
For this case, the package
provides a mechanism to include parts
by |\input| which can also be processed individually.
However, by construction this mechanism
requires manual handling of the content to be output.

%%%%%%%%%%%%%%%%%%%%%%%%%%%%%%%%%%%%%%%%
\DescribeMacro{\ifchilddocmanual}
The main file should be prepared as usual, see \secref{sec:include}.
However, the document body must make a distinction
between processing of an individual part and of the main document, e.g.:
%
\begin{center}
\begin{tabular}{l}
|\ifchilddocmanual|\\
|\input{\childdocname}|\\
|\||else|\\
\textit{document body with }|\input{|\textit{part}|}|\\
|\||fi|
\end{tabular}
\end{center}
%
The conditional |\ifchilddocmanual| is true whenever
a part to be included by |\input| is being compiled,
and the name of the part is stored in |\childdocname|.

%%%%%%%%%%%%%%%%%%%%%%%%%%%%%%%%%%%%%%%%
\DescribeMacro{\childdocby}
Each part to be included by |\input| should start with:
%
\begin{center}
\begin{tabular}{l}
|\input{childdoc.def}|\\
|\childdocby{|\textit{main}|}|\\
\end{tabular}
\end{center}
%
The directive |\childdocby| is similar to |\childdocof|
described in \secref{sec:include},
but the subsequent selection of content must be done manually.
To that end, both |\ifchilddoc| and |\ifchilddocmanual|
will be true upon processing of a part,
and the name of the part is stored in |\childdocname|.
Note that |\jobname| will be set to the filename of the current part
so that each part receives an individual |.aux| file
that does not interfere with the |.aux| file(s) of the main document.
This behaviour can be altered by the alternative form
|\childdocby[*]{|\textit{main}|}| (with a non-empty optional argument)
which uses the |.aux| file of the main document
by setting |\jobname| to \textit{main}.

%%%%%%%%%%%%%%%%%%%%%%%%%%%%%%%%%%%%%%%%%%%%%%%%%%%%%%%%%%%%%%%%%%%%%%%%%%%%%%%%
\subsection{Driver Development}
\label{sec:driver}

The \textsf{childdoc} mechanism can also be use for the development
of definition files such as \LaTeX{} styles or classes.
This case differs from the above setup with multiple parts
included by |\include| in that no |\includeonly| should be invoked.
This can be achieved by starting the include file
(before |\ProvidesPackage|) with:
%
\begin{center}
\begin{tabular}{l}
|\input{childdoc.def}|\\
|\childdocforward{|\textit{main}|}|\\
\end{tabular}
\end{center}
%
or alternatively with:
%
\begin{center}
\begin{tabular}{l}
|\input{childdoc.def}|\\
|\childdocby{|\textit{main}|}|\\
\end{tabular}
\end{center}
%
Both forms have slightly different effects as described above.
The main file is prepared as usual, see \secref{sec:include}.

%%%%%%%%%%%%%%%%%%%%%%%%%%%%%%%%%%%%%%%%%%%%%%%%%%%%%%%%%%%%%%%%%%%%%%%%%%%%%%%%
\subsection{Legacy Detection}
\label{sec:detection}

The directive |\childdocmain| in the main file can detect
whether the complete document or merely a child is to be compiled
even without using the directive |\childdocof|.
This method is deprecated because it is less robust
and there is no compelling reason to use it;
it is merely provided for backward compatibility
and it may be removed in future versions.

If the detection mechanism is to be used,
it is mandatory to correctly specify
the filename of the main file as the argument of |\childdocmain|:
%
\begin{center}
\begin{tabular}{l}
|\input{childdoc.def}|\\
|\childdocmain{|\textit{main}|}|\\
\end{tabular}
\end{center}
%
If |\jobname| does not match the argument \textit{main} of |\childdocmain|,
it is assumed that |\jobname| points to the child file to be compiled.
When using |\childdocmain| with the main file specified as argument,
it suffices to start a child file
with just |\input{|\textit{main}|}|
without loading of the package and using |\childdocof|.
If instead all processing is done
with the appropriate \textsf{childdoc} directives,
the argument of \textit{main} of |\childdocmain| can be empty.

An alternative version of the command line processing described
in \secref{sec:commandline} using the detection mechanism reads:
%
\begin{center}
|... -jobname "|\textit{target}|" "|[\textit{flags}]%
[|\def\jobname{|\textit{dest}|}|]|\input{|\textit{main}|}"|
\end{center}

%%%%%%%%%%%%%%%%%%%%%%%%%%%%%%%%%%%%%%%%%%%%%%%%%%%%%%%%%%%%%%%%%%%%%%%%%%%%%%%%
\subsection{Manual Code}
\label{sec:manual}

In case one cannot be certain whether the definitions file |childdoc.def|
is installed on the target \TeX{} distribution
and one prefers not to ship it,
it is conceivable to paste a few relevant commands into the sources.

To that end, drop all statements |\input{childdoc.def}|
and perform the replacements as outlined below.
Instead of |\childdocmain{|\textit{main}|}| add the following code
to the top of the main file:
%
\begin{center}
\begin{tabular}{l}
|\||ifdefined\childdocname\endinput\||fi\newif\ifchilddoc|\\
|\edef\childdocname{\scantokens\expandafter{\jobname\noexpand}}|\\
|\def\childdocmain{|\textit{main}|}\||ifx\childdocmain\childdocname\||else|\\
|\childdoctrue\includeonly{\childdocname}\let\jobname\childdocmain\||fi|\\
\end{tabular}
\end{center}
%
Instead of |\childdocof{|\textit{main}|}| just include the main file
at the top of each child file:
%
\begin{center}
|\input{|\textit{main}|}|
\end{center}
%
A simple redirection |\childdocforward{|\textit{dest}|}| is achieved by:
%
\begin{center}
|\def\jobname{|\textit{dest}|}\input{\jobname}|
\end{center}
%
The redirection with prefix
|\childdocforwardprefix[|\textit{prefix}|]{|\textit{dest}|}|
is accomplished by:
%
\begin{center}
\begin{tabular}{l}
|{\edef\jobname{\scantokens\expandafter{\jobname\noexpand}}|\\
|\def\redirectjob |\textit{prefix}|#1~~~{\gdef\jobname{|\textit{dest}|#1}}|\\
|\expandafter\redirectjob\jobname~~~}\input{\jobname}|
\end{tabular}
\end{center}

In an alternative approach,
child documents can be compiled by a specific command line
without additional code or specific definitions:
%
\begin{center}
|... -jobname "|\textit{target}|" "|[\textit{flags}]%
|\includeonly{|\textit{dest}|}\input{|\textit{main}|}"|
\end{center}
%

%%%%%%%%%%%%%%%%%%%%%%%%%%%%%%%%%%%%%%%%%%%%%%%%%%%%%%%%%%%%%%%%%%%%%%%%%%%%%%%%
%%%%%%%%%%%%%%%%%%%%%%%%%%%%%%%%%%%%%%%%%%%%%%%%%%%%%%%%%%%%%%%%%%%%%%%%%%%%%%%%
\section{Information}

%%%%%%%%%%%%%%%%%%%%%%%%%%%%%%%%%%%%%%%%%%%%%%%%%%%%%%%%%%%%%%%%%%%%%%%%%%%%%%%%
\subsection{Copyright}

Copyright \copyright{} 2017--2018 Niklas Beisert

This work may be distributed and/or modified under the
conditions of the \LaTeX{} Project Public License, either version 1.3
of this license or (at your option) any later version.
The latest version of this license is in
  \url{http://www.latex-project.org/lppl.txt}
and version 1.3 or later is part of all distributions of \LaTeX{}
version 2005/12/01 or later.

This work has the LPPL maintenance status `maintained'.

The Current Maintainer of this work is Niklas Beisert.

This work consists of the files |README.txt|, |childdoc.ins| and |childdoc.dtx|
as well as the derived files |childdoc.def|, |cdocsamp.tex|
with |cdocsch1.tex|, |cdocsch2.tex|, |cdocspt3.tex|, |cdocspt4.tex|,
|cdocsdrf.tex|, |cdocsfn1.tex|, |cdocsfn2.tex|
as well as |childdoc.pdf|.

%%%%%%%%%%%%%%%%%%%%%%%%%%%%%%%%%%%%%%%%%%%%%%%%%%%%%%%%%%%%%%%%%%%%%%%%%%%%%%%%
\subsection{Files and Installation}

The package consists of the files:
%
\begin{center}
\begin{tabular}{ll}
    |README.txt|   & readme file \\
    |childdoc.ins| & installation file \\
    |childdoc.dtx| & source file \\
    |childdoc.def| & definition file \\
    |cdocsamp.tex| & sample main file \\
    |cdocsch1.tex| & sample include file \\
    |cdocsch2.tex| & sample include file \\
    |cdocspt3.tex| & sample part file \\
    |cdocspt4.tex| & sample part file \\
    |cdocsdrf.tex| & sample redirection file \\
    |cdocsfn1.tex| & sample redirection file \\
    |cdocsfn2.tex| & sample redirection file \\
    |childdoc.pdf| & manual
\end{tabular}
\end{center}
%
The distribution consists of the files
|README.txt|, |childdoc.ins| and |childdoc.dtx|.
%
\begin{itemize}
\item
Run (pdf)\LaTeX{} on |childdoc.dtx|
to compile the manual |childdoc.pdf| (this file).
\item
Run \LaTeX{} on |childdoc.ins| to create the definitions file |childdoc.def|
and the sample |cdocsamp.tex| with include files
|cdocsch1.tex|, |cdocsch2.tex|, |cdocspt3.tex|, |cdocspt4.tex|,
|cdocsdrf.tex|, |cdocsfn1.tex|, |cdocsfn2.tex|.
Then copy the file |childdoc.def| to an appropriate directory of your \LaTeX{}
distribution, e.g.\ \textit{texmf-root}|/tex/latex/childdoc|.
\end{itemize}

%%%%%%%%%%%%%%%%%%%%%%%%%%%%%%%%%%%%%%%%%%%%%%%%%%%%%%%%%%%%%%%%%%%%%%%%%%%%%%%%
\subsection{Related CTAN Packages}

There are several other packages which offer a similar functionality:
%
\begin{itemize}
\item
The packages
\href{http://ctan.org/pkg/docmute}{\textsf{docmute}},
\href{http://ctan.org/pkg/includex}{\textsf{includex}} and
\href{http://ctan.org/pkg/standalone}{\textsf{standalone}}
provide commands to include only the document body of
a child file thus allowing both files to be compiled individually.
\item
The packages \href{http://ctan.org/pkg/subdocs}{\textsf{subdocs}}
and \href{http://ctan.org/pkg/subfiles}{\textsf{subfiles}}
provide structures in which the main and child documents can be
encapsulated and allowing them to be compiled individually.
The inclusion mechanism is different from the conventional |\include|.
\item
The package \href{http://ctan.org/pkg/combine}{\textsf{combine}}
is an elaborate solution to combine several documents into one.
\end{itemize}
%
See also the CTAN topic \href{http://ctan.org/topic/subdocs}{\textsf{subdocs}}
for further related packages.
The present package differs from the above solutions in that
a document structure constructed with the conventional |\include| mechanism
just needs two extra commands at the top of every file
such that all constituent files can be compiled individually.

%%%%%%%%%%%%%%%%%%%%%%%%%%%%%%%%%%%%%%%%%%%%%%%%%%%%%%%%%%%%%%%%%%%%%%%%%%%%%%%%
%\subsection{Feature Suggestions}
%
%The following is a list of features which may be useful for future
%versions of this package:
%%
%\begin{itemize}
%\item
%\ldots
%\end{itemize}

%%%%%%%%%%%%%%%%%%%%%%%%%%%%%%%%%%%%%%%%%%%%%%%%%%%%%%%%%%%%%%%%%%%%%%%%%%%%%%%%
\subsection{Revision History}

%%%%%%%%%%%%%%%%%%%%%%%%%%%%%%%%%%%%%%%%
\paragraph{v2.0:} 2018/12/30

\begin{itemize}
\item
immediate forward processing
\item
added |\childdocby| mechanism
\item
manual restructured
\end{itemize}

%%%%%%%%%%%%%%%%%%%%%%%%%%%%%%%%%%%%%%%%
\paragraph{v1.6:} 2018/01/17

\begin{itemize}
\item
application for development of include files
\item
corrections to manual
\end{itemize}

%%%%%%%%%%%%%%%%%%%%%%%%%%%%%%%%%%%%%%%%
\paragraph{v1.5:} 2017/05/21

\begin{itemize}
\item
more complete structuring introduced
\item
|\childdocof| introduced
\item
|\childdoc| renamed to |\childdocmain|
\item
|\childredirect| renamed to |\childdocforward| and |\childdocforwardprefix|
and functionality expanded
\end{itemize}

%%%%%%%%%%%%%%%%%%%%%%%%%%%%%%%%%%%%%%%%
\paragraph{v1.0:} 2017/04/27

\begin{itemize}
\item
manual and install package
\item
first version published on CTAN
\end{itemize}

%%%%%%%%%%%%%%%%%%%%%%%%%%%%%%%%%%%%%%%%
\paragraph{v0.6:} 2017/04/26

\begin{itemize}
\item
redirection mechanism added
\end{itemize}

%%%%%%%%%%%%%%%%%%%%%%%%%%%%%%%%%%%%%%%%
\paragraph{v0.5:} 2017/04/26

\begin{itemize}
\item
functionality in definition file
\end{itemize}


%%%%%%%%%%%%%%%%%%%%%%%%%%%%%%%%%%%%%%%%%%%%%%%%%%%%%%%%%%%%%%%%%%%%%%%%%%%%%%%%
%%%%%%%%%%%%%%%%%%%%%%%%%%%%%%%%%%%%%%%%%%%%%%%%%%%%%%%%%%%%%%%%%%%%%%%%%%%%%%%%
%%%%%%%%%%%%%%%%%%%%%%%%%%%%%%%%%%%%%%%%%%%%%%%%%%%%%%%%%%%%%%%%%%%%%%%%%%%%%%%%
\appendix

\settowidth\MacroIndent{\rmfamily\scriptsize 000\ }

 \DocInput{childdoc.dtx}

\end{document}
%</driver>
% \fi
%
% %%%%%%%%%%%%%%%%%%%%%%%%%%%%%%%%%%%%%%%%%%%%%%%%%%%%%%%%%%%%%%%%%%%%%%%%%%%%%%
% %%%%%%%%%%%%%%%%%%%%%%%%%%%%%%%%%%%%%%%%%%%%%%%%%%%%%%%%%%%%%%%%%%%%%%%%%%%%%%
% \section{Sample}
%\iffalse
%<*samplemain>
%\fi
%
% The following presents a sample document
% with two chapters, two parts, a title page,
% a compile flag as well as three forwarding files to set the flag.
% It consists of eight |.tex| files:
% \begin{center}
% \begin{tabular}{ll}
% |cdocsamp.tex|&main file\\
% |cdocsch1.tex|&include file for chapter 1\\
% |cdocsch2.tex|&include file for chapter 2\\
% |cdocspt3.tex|&include file for part 3\\
% |cdocspt4.tex|&include file for part 4\\
% |cdocsdrf.tex|&forwarding file for main file in draft mode\\
% |cdocsfi1.tex|&forwarding file for final version of chapter 1\\
% |cdocsfi2.tex|&forwarding file for final version of chapter 2\\
% \end{tabular}
% \end{center}
% Each of the eight files can be compiled directly by the \LaTeX{} compiler.
%
% %%%%%%%%%%%%%%%%%%%%%%%%%%%%%%%%%%%%%%
% \paragraph{Main File.}
%
% The main file is called |cdocsamp.tex|.
%
% Load the \textsf{childdoc} definitions and
% declare the filename for the main document:
%    \begin{macrocode}
\input{childdoc.def}
\childdocmain{}
%    \end{macrocode}

% Optional override for |\version| flag:
%    \begin{macrocode}
%%\ifchilddoc\else\providecommand{\version}{draft}\fi
%    \end{macrocode}

% Define the default values for the |\version| flag
% (|final| for the main file and |draft| for childs):
%    \begin{macrocode}
\ifchilddoc
\providecommand{\version}{draft}
\else
\providecommand{\version}{final}
\fi
%    \end{macrocode}

% Load the standard document class:
%    \begin{macrocode}
\documentclass[12pt]{article}
%    \end{macrocode}

% Start the document body:
%    \begin{macrocode}
\begin{document}
%    \end{macrocode}

% Declare a title page.
% Print title, part of document being processed and version flag:
%    \begin{macrocode}
\addtocounter{page}{-1}
\begin{center}
{\LARGE\bfseries{}childdoc example\par}
\vspace{1cm}
\ifchilddoc
\ifchilddocmanual part\else chapter\fi:
`\childdocname' of `\childdocjob'\par
\else
main document: `\childdocjob'\par
\fi
version: \version\par
\end{center}
\newpage
%    \end{macrocode}

% Manually include selected file,
% otherwise process as usual:
%    \begin{macrocode}
\ifchilddocmanual
\section*{part `\childdocname'}
\input{\childdocname}
\else
%    \end{macrocode}

% Include the two chapters:
%    \begin{macrocode}
\include{cdocsch1}
\include{cdocsch2}
%    \end{macrocode}

% Include the two parts unless only chapters should be displayed:
%    \begin{macrocode}
\ifchilddoc\else
\section{part three}
\input{cdocspt3}
\section{part four}
\input{cdocspt4}
\fi
%    \end{macrocode}

% Process as usual until here:
%    \begin{macrocode}
\fi
%    \end{macrocode}

% End of document body:
%    \begin{macrocode}
\end{document}
%    \end{macrocode}
%\iffalse
%</samplemain>
%\fi
%
% %%%%%%%%%%%%%%%%%%%%%%%%%%%%%%%%%%%%%%
% \paragraph{Chapter Include Files.}
%
% The include files are called |cdocsch1.tex| and |cdocsch2.tex|.
%
%\iffalse
%<*samplechap1|samplechap2>
%\fi

% Optional override for |\version| flag:
%    \begin{macrocode}
%%\providecommand{\version}{final}
%    \end{macrocode}

% Include the main document:
%    \begin{macrocode}
\input{childdoc.def}
\childdocof{cdocsamp}
%    \end{macrocode}

%\iffalse
%</samplechap1|samplechap2>
%\fi
%
%\iffalse
%<*samplechap1>
%\fi
% Some text for chapter 1:
%    \begin{macrocode}
\section{one}
some text in chapter one
%    \end{macrocode}

%\iffalse
%</samplechap1>
%\fi
% Some text for chapter 2:
%\iffalse
%<*samplechap2>
%\fi
%    \begin{macrocode}
\section{two}
more text in chapter two
%    \end{macrocode}

%\iffalse
%</samplechap2>
%\fi
%
% %%%%%%%%%%%%%%%%%%%%%%%%%%%%%%%%%%%%%%
% \paragraph{Part Include Files.}
%
% The include files are called |cdocspt3.tex| and |cdocspt4.tex|.
%
%\iffalse
%<*samplepart3|samplepart4>
%\fi

% Optional override for |\version| flag:
%    \begin{macrocode}
%%\providecommand{\version}{final}
%    \end{macrocode}

% Include the main document:
%    \begin{macrocode}
\input{childdoc.def}
\childdocby{cdocsamp}
%    \end{macrocode}

%\iffalse
%</samplepart3|samplepart4>
%\fi
%
%\iffalse
%<*samplepart3>
%\fi
% Some text for part 3:
%    \begin{macrocode}
some text in part three
%    \end{macrocode}

%\iffalse
%</samplepart3>
%\fi
% Some text for part 4:
%\iffalse
%<*samplepart4>
%\fi
%    \begin{macrocode}
more text in part four
%    \end{macrocode}

%\iffalse
%</samplepart4>
%\fi
%
% %%%%%%%%%%%%%%%%%%%%%%%%%%%%%%%%%%%%%%
% \paragraph{Forwarding for a Complete Draft.}
%
% The following forwarding file |cdocsdrf.tex|
% compiles the main document in draft mode:
%\iffalse
%<*sampledraft>
%\fi
%    \begin{macrocode}
\def\version{draft}
\input{childdoc.def}
\childdocforward{cdocsamp}
%    \end{macrocode}

%\iffalse
%</sampledraft>
%\fi
%
% %%%%%%%%%%%%%%%%%%%%%%%%%%%%%%%%%%%%%%
% \paragraph{Forwarding for Final Version of the Chapters.}
%
% The following forwarding files |cdocsfn1.tex| and |cdocsfn2.tex|
% (with identical content)
% compile the final versions of the child documents
% |cdocsch1.tex| and |cdocsch2.tex|, respectively:
%\iffalse
%<*samplefinal>
%\fi
%    \begin{macrocode}
\def\version{final}
\input{childdoc.def}
\childdocforwardprefix[cdocsamp]{cdocsfn}{cdocsch}
%    \end{macrocode}

%\iffalse
%</samplefinal>
%\fi
%
% %%%%%%%%%%%%%%%%%%%%%%%%%%%%%%%%%%%%%%
% \paragraph{Command Line Processing.}
%
% The following three command lines generate the output files
% |cdocscld|, |cdocscl1| and |cdocscl2|
% which should be identical to
% |cdocsdrf|, |cdocsch1| and |cdocsfn2|, respectively:
% \begin{center}
% \begin{tabular}{l}
% |latex -jobname cdocscld \|\\
% |  "\def\version{draft}\input{childdoc.def}\childdocforward{cdocsamp}"|\\
% |latex -jobname cdocscl1 \|\\
% |  "\input{childdoc.def}\childdocforward[cdocsamp]{cdocsch1}"|\\
% |latex -jobname cdocscl2 \|\\
% |  "\def\version{final}\input{childdoc.def}\childdocforward{cdocsch2}"|
% \end{tabular}
% \end{center}
% Note that the trailing backslash on each first line
% merely continues the input to the second line
% (for convenient cut ant paste).
% Furthermore, the command |latex| can be replaced by any
% of its alternative versions such as |pdflatex|.
%
% %%%%%%%%%%%%%%%%%%%%%%%%%%%%%%%%%%%%%%%%%%%%%%%%%%%%%%%%%%%%%%%%%%%%%%%%%%%%%%
% %%%%%%%%%%%%%%%%%%%%%%%%%%%%%%%%%%%%%%%%%%%%%%%%%%%%%%%%%%%%%%%%%%%%%%%%%%%%%%
% \section{Implementation}
%\iffalse
%<*package>
%\fi
%
% This section describes the definitions file |childdoc.def|.

% The definitions cannot be loaded using |\usepackage| or |\RequirePackage|
% which has a mechanism to prevent loading a style file more than once.
% When loading the definitions by means of |\input|
% multiple instances have to be prevented manually:
%\iffalse
%This code needs to be before the `\ProvidesFile' directive
%which is defined at the beginning of this file.
%Therefore it is also placed there and commented out here.
%</package>
%<*discard>
%\fi
%    \begin{macrocode}
\ifdefined\childdocmain\endinput\fi
%    \end{macrocode}
%\iffalse
%</discard>
%<*package>
%\fi
%
% \macro{\ifchilddoc}
% \macro{\ifchilddocmanual}
% The conditional |\ifchilddoc| tells whether a
% child (true) or main (false) document is being compiled.
% The conditional |\ifchilddocmanual| tells whether
% the |\includeonly| mechanism is used (false) or
% the selection of child files must be performed manually (true).
% The definitions initialise to false:
%    \begin{macrocode}
\newif\ifchilddoc
\newif\ifchilddocmanual
%    \end{macrocode}

% \macro{\childdocname}
% \macro{\childdocjob}
% The macro |\childdocname| stores the name of the main document
% to be compiled. The macro |\childdocjob| stores the name of
% the document on which the \LaTeX{} compiler was originally invoked.
% The content of |\jobname| cannot be compared
% to filenames specified in the source due to different catcodes.
% The following code rescans |\jobname|, stores the result
% in |\childdocname| and saves a copy in |\childdocjob|:
%    \begin{macrocode}
\edef\childdocname{\scantokens\expandafter{\jobname\noexpand}}
\let\childdocjob\childdocname
%    \end{macrocode}

% \macro{\childdocdisable}
% The macro |\childdocdisable| prevents the main file
% from being processed more than once.
% At this stage, the main document command |\childdocmain|
% is assumed to be called once again where it should do nothing.
% Any subsequent call to it should prevent
% a secondary processing of the main document
% It overwrites the forwarding commands
% |\childdocof| and |\childdocforward|
% with empty macros to prevent further inclusions of the main document:
%    \begin{macrocode}
\newcommand{\childdocdisable}
{
  \renewcommand{\childdocmain}[1]{\renewcommand{\childdocmain}[1]{\endinput}}
  \renewcommand{\childdocof}[1]{}
  \renewcommand{\childdocby}[2][]{}
  \renewcommand{\childdocforward}[2][]{}
  \renewcommand{\childdocdisable}{}
}
%    \end{macrocode}

% \macro{\childdocmain}
% The macro |\childdocmain| is to be called at the top of the main file
% with nothing or the main filename (without extension) as argument.
% First, it breaks loops.
% If the argument is not empty and does not match |\childdocname|
% (which is set by the first inclusion of |childdoc.def|),
% |\ifchilddoc| is set to true, |\includeonly| is applied to the child file
% and |\jobname| is set to the main file
% (for proper handling of |.aux| files):
%    \begin{macrocode}
\newcommand{\childdocmain}[1]
{
  \childdocdisable\childdocmain{}
  \if?#1?\else
    \begingroup
      \def\childdoctmp{#1}
      \ifx\childdoctmp\childdocname
        \def\childdoctmp{}
      \else
        \def\childdoctmp
        {
          \childdoctrue
          \includeonly{\childdocname}
          \def\childdocjob{#1}
          \def\jobname{#1}
        }
      \fi
      \expandafter
    \endgroup
    \childdoctmp
  \fi
}
%    \end{macrocode}

% \macro{\childdocof}
% The command |\childdocof| redirects
% compilation to the main file |#1|.
%    \begin{macrocode}
\newcommand{\childdocof}[1]
{
  \childdocdisable
  \childdoctrue
  \includeonly{\childdocname}
  \def\jobname{#1}
  \def\childdocjob{#1}
  \input{#1}
}
%    \end{macrocode}

% \macro{\childdocby}
% The command |\childdocby| ....
%    \begin{macrocode}
\newcommand{\childdocby}[2][]
{
  \childdocdisable
  \childdoctrue
  \childdocmanualtrue
  \if?#1?\else
    \def\jobname{#2}
  \fi
  \def\childdocjob{#2}
  \input{#2}
  \endinput
}
%    \end{macrocode}

% \macro{\childdocforward}
% The command |\childdocforward| redirects
% compilation to the main file or
% (if the optional argument is given) a child file.
% Parameters are set as if the main file
% or a child file starting with |\childdocof| was compiled.
% Then compilation is handed over to the main file:
%    \begin{macrocode}
\newcommand{\childdocforward}[2][]
{
  \begingroup
    \if?#1?
      \def\childdoctmp
      {
        \def\childdocname{#2}
        \def\childdocjob{#2}
        \def\jobname{#2}
        \input{#2}
        \endinput
      }
    \else
      \def\childdoctmp
      {
        \childdocdisable
        \def\childdocname{#2}
        \childdoctrue
        \includeonly{#2}
        \def\childdocjob{#1}
        \def\jobname{#1}
        \input{#1}
        \endinput
      }
    \fi
    \expandafter
  \endgroup
  \childdoctmp
}
%    \end{macrocode}

% \macro{\childdocforwardprefix}
% The command |\childdocforwardprefix| redirects
% compilation to the main or a child file by means of a pattern.
% The prefix |#1| in the current filename is replaced by |#2|
% and the suffix of the current filename is kept
% (it is assumed that the filename does not contain the substring `|~~~|'
% which is used as a delimiter).
% Compilation is handed over to the new file by |\childdocforward|:
%    \begin{macrocode}
\newcommand{\childdocforwardprefix}[3][]
{
  \begingroup
    \def\childdocextract #2##1~~~{\def\childdoctmp{\childdocforward[#1]{#3##1}}}
    \expandafter\childdocextract\childdocname~~~
    \expandafter
  \endgroup
  \childdoctmp
}
%    \end{macrocode}

% \macro{\childdoc}
% The deprecated macro |\childdoc| is a legacy version of |\childdocmain|:
%    \begin{macrocode}
\newcommand{\childdoc}{\childdocmain}
%    \end{macrocode}

% \macro{\childdocredirect}
% The deprecated macro |\childdocredirect| is a legacy version
% of |\childdocforward| and |\childdocforwardprefix|:
%    \begin{macrocode}
\newcommand{\childdocredirect}[2][]
{
  \begingroup
    \if?#1?
      \def\childdoctmp{\childdocforward{#2}}
    \else
      \def\childdoctmp{\childdocforwardprefix{#1}{#2}}
    \fi
    \expandafter
  \endgroup
  \childdoctmp
}
%    \end{macrocode}

%\iffalse
%</package>
%\fi
%
\endinput
|\\
|\childdocforward[|\textit{main}|]{|\textit{dest}|}|\\
\end{tabular}
\end{center}
%
The argument \textit{dest} is the destination file
(without extension).
It should be the main file or one of the child files.
Note that further \textsf{childdoc} directives
such as |\childdocof| and |\childdocforward|
in the indicated file will be processed in this form.
The optional argument \textit{main}
passes on directly to the main file \textit{main}
while pretending to compile the child \textit{dest}.
This form behaves as if \textit{dest}
issues |\childdocof{|\textit{main}|}| right away,
and no further \textsf{childdoc} directives will be processed.

%%%%%%%%%%%%%%%%%%%%%%%%%%%%%%%%%%%%%%%%
\DescribeMacro{\...prefix}
In the alternative form |\childdocforwardprefix|,
%
\begin{center}
\begin{tabular}{l}
|% \iffalse
%
% childdoc.dtx Copyright (C) 2017-2018 Niklas Beisert
%
% This work may be distributed and/or modified under the
% conditions of the LaTeX Project Public License, either version 1.3
% of this license or (at your option) any later version.
% The latest version of this license is in
%   http://www.latex-project.org/lppl.txt
% and version 1.3 or later is part of all distributions of LaTeX
% version 2005/12/01 or later.
%
% This work has the LPPL maintenance status `maintained'.
%
% The Current Maintainer of this work is Niklas Beisert.
%
% This work consists of the files childdoc.dtx and childdoc.ins
% and the derived files childdoc.def and cdocsamp.tex with
% cdocsch1.tex, cdocsch2.tex, cdocsdrf.tex, cdocsfn1.tex, cdocsfn2.tex.
%
%<package>\ifdefined\childdocmain\endinput\fi
%<package>\ProvidesFile{childdoc.def}[2018/12/30 v2.0 child document driver]
%<samplemain>\ProvidesFile{cdocsamp.tex}[2018/12/30 v2.0 sample for childdoc]
%<*driver>
%\ProvidesFile{childdoc.drv}[2018/12/30 v2.0 childdoc reference manual file]
\PassOptionsToClass{10pt,a4paper}{article}
\documentclass{ltxdoc}

\usepackage[margin=35mm]{geometry}
\usepackage{hyperref}
\usepackage{hyperxmp}
\usepackage[usenames]{color}

\hypersetup{colorlinks=true}
\hypersetup{pdfstartview=FitH}
\hypersetup{pdfpagemode=UseNone}
\hypersetup{pdfsource={}}
\hypersetup{pdflang={en-UK}}
\hypersetup{pdfcopyright={Copyright 2017-2018 Niklas Beisert.
  This work may be distributed and/or modified under the
  conditions of the LaTeX Project Public License, either version 1.3
  of this license or (at your option) any later version.}}
\hypersetup{pdflicenseurl={http://www.latex-project.org/lppl.txt}}
\hypersetup{pdfcontactaddress={ETH Zurich, ITP, HIT K,
  Wolfgang-Pauli-Strasse 27}}
\hypersetup{pdfcontactpostcode={8093}}
\hypersetup{pdfcontactcity={Zurich}}
\hypersetup{pdfcontactcountry={Switzerland}}
\hypersetup{pdfcontactemail={nbeisert@itp.phys.ethz.ch}}
\hypersetup{pdfcontacturl={http://people.phys.ethz.ch/\xmptilde nbeisert/}}

\newcommand{\secref}[1]{\hyperref[#1]{section \ref*{#1}}}

\parskip1ex
\parindent0pt
\let\olditemize\itemize
\def\itemize{\olditemize\parskip0pt}

\begin{document}

\title{The \textsf{childdoc} Package}
\hypersetup{pdftitle={The childdoc Package}}
\author{Niklas Beisert\\[2ex]
  Institut f\"ur Theoretische Physik\\
  Eidgen\"ossische Technische Hochschule Z\"urich\\
  Wolfgang-Pauli-Strasse 27, 8093 Z\"urich, Switzerland\\[1ex]
  \href{mailto:nbeisert@itp.phys.ethz.ch}
  {\texttt{nbeisert@itp.phys.ethz.ch}}}
\hypersetup{pdfauthor={Niklas Beisert}}
\hypersetup{pdfsubject={Manual for the LaTeX2e Package childdoc}}
\date{30 December 2018, \textsf{v2.0}}
\maketitle

\begin{abstract}\noindent
\textsf{childdoc} is a \LaTeXe{} package
that enables the direct compilation
of document sections included by |\include|
to individual files.
\end{abstract}

\begingroup
\parskip0ex
\tableofcontents
\endgroup

%%%%%%%%%%%%%%%%%%%%%%%%%%%%%%%%%%%%%%%%%%%%%%%%%%%%%%%%%%%%%%%%%%%%%%%%%%%%%%%%
%%%%%%%%%%%%%%%%%%%%%%%%%%%%%%%%%%%%%%%%%%%%%%%%%%%%%%%%%%%%%%%%%%%%%%%%%%%%%%%%
\section{Introduction}

\LaTeX{} provides a mechanism to structure a large document (such as a book)
into a main file and several child files (containing the chapters)
using the |\include| command.
This mechanism is beneficial for documents
which span hundreds of pages in order to
make the source file(s) more manageable.
Moreover, compilation can be restricted to
selected child files by means of the |\includeonly| command.
The latter feature can be used to reduce the compilation time while editing
(this was significantly more useful in the earlier days of \LaTeX{})
or to generate a smaller document which is easier to navigate.
Another application of |\includeonly| is to generate
documents consisting of selected parts of the complete document.

However, there are a few drawbacks of the plain |\include| mechanism:
\begin{itemize}
\item
The child files cannot be compiled on their own,
they can only be compiled via the main file.
A naive editing environment
(such as a text editor with an option
to have the current file processed by \LaTeX)
may require one to switch to the main file before compiling;
attempting to compile the child file produces errors.
\item
The main file must be modified (each time)
to adjust the |\includeonly| command
to the present needs. This easily leaves the main file in a messy state.
\item
The generated document will always carry the filename
of the main document. This is inconvenient if
several child files are to be compiled and
to be kept for distribution.
\end{itemize}

The present package provides a simple interface
to make child files individually compilable by \LaTeX{}.
Compiling a child file then has the same effect as compiling
the main file with an |\includeonly| command
to select the appropriate child.
Moreover the generated document will carry the name of the child
rather than the main file.
This resolves all three above issues.

This feature is meant to make the editing of books,
thesis documents and lecture notes somewhat more convenient.
However, the package can also be used efficiently for
composing a series of documents (such as exercise sheets)
which are typically distributed individually.
It then assists the author in generating the individual documents
(potentially in different versions)
as well as a document containing the collected series.
Another application is in developing style files
or other kinds of included material
where compilation of the style file could redirect
to a sample or test file.

%%%%%%%%%%%%%%%%%%%%%%%%%%%%%%%%%%%%%%%%%%%%%%%%%%%%%%%%%%%%%%%%%%%%%%%%%%%%%%%%
%%%%%%%%%%%%%%%%%%%%%%%%%%%%%%%%%%%%%%%%%%%%%%%%%%%%%%%%%%%%%%%%%%%%%%%%%%%%%%%%
\section{Usage}

First of all, the package \textsf{childdoc} is \emph{not} a standard
\LaTeXe{} |.sty| style file! Therefore it needs to be invoked in
a non-standard way.

%%%%%%%%%%%%%%%%%%%%%%%%%%%%%%%%%%%%%%%%%%%%%%%%%%%%%%%%%%%%%%%%%%%%%%%%%%%%%%%%
\subsection{Included Files}
\label{sec:include}

%%%%%%%%%%%%%%%%%%%%%%%%%%%%%%%%%%%%%%%%
\DescribeMacro{\childdocmain}
To use the package, add the commands
\begin{center}
\begin{tabular}{l}
|\input{childdoc.def}|\\
|\childdocmain{}|\\
\end{tabular}
\end{center}
at the very top of the main \LaTeX{} file,
in particular \emph{before} the |\documentclass| statement!
The argument of |\childdocmain| should be left empty
(but it must be present).

%%%%%%%%%%%%%%%%%%%%%%%%%%%%%%%%%%%%%%%%
\DescribeMacro{\childdocof}
Furthermore, add the commands
\begin{center}
\begin{tabular}{l}
|\input{childdoc.def}|\\
|\childdocof{|\textit{main}|}|\\
\end{tabular}
\end{center}
at the top of every child file \textit{child}
which is included by |\include{|\textit{child}|}|
from within the main file
(or at least for those files to be compiled individually).
The argument \textit{main} must be the filename of the main file.

There are a couple of
considerations in setting up the main and child documents:

%%%%%%%%%%%%%%%%%%%%%%%%%%%%%%%%%%%%%%%%
\paragraph{Restrictions.}

Please note the following restrictions:
\begin{itemize}
\item
|\childdocmain| must be called with one argument \textit{main}
to ensure compatibility with earlier version of the package.
It must either be empty (|\childdocmain{}|)
or precisely match the filename of the main file in which it is specified.
See \secref{sec:detection} for further information.
\item
The filename \textit{main} must be specified without the |.tex| extension.
\item
The filename \textit{main} is case sensitive
(even in case-insensitive file systems)
due to internal string comparison.
\item
The argument \textit{main} should be fully expanded, it cannot be a macro.
\item
Subdirectories and special characters should be avoided in filenames.
\item
The command |\childdocmain{|\textit{main}|}| must be followed by a whitespace.
It should not be followed immediately by another command
or by a comment mark `|%|'.
This is because the \TeX{} parser reads the token immediately following
the argument of |\childdocmain| and puts it
at the beginning of every child section;
however, a white\-space is ignored.
\end{itemize}

%%%%%%%%%%%%%%%%%%%%%%%%%%%%%%%%%%%%%%%%
\paragraph{Content of Main File.}

It is advisable to place all content in the child files included by |\include|.
Any output contained in the main file will appear in all child documents
unless suppressed manually;
it cannot be suppressed automatically by the |\includeonly| directive
and thus should normally be avoided.
A method to include some content in the main file
by means of conditional processing is described in \secref{sec:conditional}.

%%%%%%%%%%%%%%%%%%%%%%%%%%%%%%%%%%%%%%%%
\paragraph{Page Numbering.}

When only a part of the document is compiled,
the appropriate numbering of pages
(as well as other status parameters)
is determined from the |.aux| files.
The latter contain information from previous passes.
However this information needs to propagate through
all intermediate child documents.
Therefore the page numbering in child documents may well
be inconsistent until the complete document is compiled at least once.

A useful (if unconventional) way to always ensure a consistent
page numbering is to restart the numbering in each child document
and denote the pages by `\textit{child}|.|\textit{page}'
where \textit{child} represents the chapter/section number of the child file.
This can be achieved by the command
|\numberwithin{page}{|\textit{child}|}|
of the \textsf{amsmath} package
where \textit{child} can be |chapter| or |section|
depending on the chosen structuring.
Alternatively, one can modify the macro |\thepage| appropriately
and reset the counter |page| at the start of each child file.

%%%%%%%%%%%%%%%%%%%%%%%%%%%%%%%%%%%%%%%%%%%%%%%%%%%%%%%%%%%%%%%%%%%%%%%%%%%%%%%%
\subsection{Conditional Processing}
\label{sec:conditional}

The package provides a mechanism to compile different versions
of a document. To customise the versions further some conditional processing
can come in handy to distinguish which version is being compiled.
The package provides two macros to describe the compilation context:

%%%%%%%%%%%%%%%%%%%%%%%%%%%%%%%%%%%%%%%%
\DescribeMacro{\ifchilddoc}
The conditional |\ifchilddoc| distinguishes between the compilation of
child documents and the main document:
%
\begin{center}
|\ifchilddoc |\textit{child-code}| |[|\||else |\textit{main-code}]| \||fi|
\end{center}

%%%%%%%%%%%%%%%%%%%%%%%%%%%%%%%%%%%%%%%%
\DescribeMacro{\childdocname}
\DescribeMacro{\childdocjob}
The macro |\childdocname| contains the filename (without extension)
of the main or child file being processed.
Note that |\childdocjob| will always contain the name of the main file.

%%%%%%%%%%%%%%%%%%%%%%%%%%%%%%%%%%%%%%%%
\paragraph{Title Page.}

Conditional processing can be used to include a title or banner page
in the main document when proper precautions are taken.
Importantly, the code in the main file should ensure that the page counter
(as well as other status parameters which are stored in the |.aux| files)
takes the same value after the conditional processing.
Otherwise the page numbers may take divergent values
depending on which part is compiled.

For example, a title page could be declared by:
%
\begin{center}
\begin{tabular}{l}
|\ifchilddoc\||else|\\
|\addtocounter{page}{-1}|\\
\textit{code for title page}\\
|\newpage|\\
|\||fi|
\end{tabular}
\end{center}
%
A banner page for the child documents can be generated by:
%
\begin{center}
\begin{tabular}{l}
|\ifchilddoc|\\
|\addtocounter{page}{-1}|\\
\textit{code for banner page}\\
|\newpage|\\
|\||fi|
\end{tabular}
\end{center}
%
Here one could write a message such as:
\begin{center}
|This is the part \childdocname{} of \childdocjob{}.|
\end{center}

%%%%%%%%%%%%%%%%%%%%%%%%%%%%%%%%%%%%%%%%%%%%%%%%%%%%%%%%%%%%%%%%%%%%%%%%%%%%%%%%
\subsection{Flags}
\label{sec:flags}

The package makes it easy to generate different versions
of the main or child documents.
To this end compilation flags can be defined
and assigned different default values.
They will be particularly useful in conjunction
with the forwarding mechanism described in \secref{sec:forward}.

For example, it may be useful to have a flag |\version|
which can be set to |draft| or |final|.
The document source will contain some conditional code
depending on the value of |\version|.
Suppose further, the flag should default to |final| for the main file
and to |draft| for child files
which is a natural assignment for editing the document.
This is achieved by placing the following code
in the preamble of the main document
(below the |\childdocmain| directive):
%
\begin{center}
\begin{tabular}{l}
|\ifchilddoc|\\
|\providecommand{\version}{draft}|\\
|\||else|\\
|\providecommand{\version}{final}|\\
|\||fi|
\end{tabular}
\end{center}
%
The definition by |\providecommand| makes sure
that previous definitions are not overwritten.
Further statements |\providecommand{\version}{...}|
can thus be added before the above code to override it.

For the main file, one might add a line
(between |\childdocmain| and the above block)
%
\begin{center}
|%\ifchilddoc\||else\providecommand{\version}{draft}\||fi|
\end{center}
%
which can be uncommented to produce a draft version.
Likewise one can add a line to the very top of a child file
(above the |\childdocof{|\textit{main}|}| directive)
%
\begin{center}
|%\providecommand{\version}{final}|
\end{center}
%
which can be uncommented to produce the final version of this child document.

%%%%%%%%%%%%%%%%%%%%%%%%%%%%%%%%%%%%%%%%%%%%%%%%%%%%%%%%%%%%%%%%%%%%%%%%%%%%%%%%
\subsection{Forwarding}
\label{sec:forward}

Different versions of the main or child documents
using compilation flags as described in \secref{sec:flags}
can be (permanently) stored in different files
for convenient compilation, viewing and distribution.
To this end, the package defines a command
to pass on compilation to a different file:

%%%%%%%%%%%%%%%%%%%%%%%%%%%%%%%%%%%%%%%%
\DescribeMacro{\childdocforward}
The command |\childdocforward| redirects processing to
another source file:
%
\begin{center}
\begin{tabular}{l}
|\input{childdoc.def}|\\
|\childdocforward[|\textit{main}|]{|\textit{dest}|}|\\
\end{tabular}
\end{center}
%
The argument \textit{dest} is the destination file
(without extension).
It should be the main file or one of the child files.
Note that further \textsf{childdoc} directives
such as |\childdocof| and |\childdocforward|
in the indicated file will be processed in this form.
The optional argument \textit{main}
passes on directly to the main file \textit{main}
while pretending to compile the child \textit{dest}.
This form behaves as if \textit{dest}
issues |\childdocof{|\textit{main}|}| right away,
and no further \textsf{childdoc} directives will be processed.

%%%%%%%%%%%%%%%%%%%%%%%%%%%%%%%%%%%%%%%%
\DescribeMacro{\...prefix}
In the alternative form |\childdocforwardprefix|,
%
\begin{center}
\begin{tabular}{l}
|\input{childdoc.def}|\\
|\childdocforwardprefix[|\textit{main}|]{|\textit{prefix}|}{|\textit{dest}|}|
\end{tabular}
\end{center}
%
the destination file is determined by a pattern
depending on the current file:
To make this work, the current file must be called
`{\textit{prefix}\hspace{0.2em}\textit{suffix}}'
with \textit{prefix} matching precisely the argument.
Processing is then passed on to the file
`{\textit{dest}\hspace{0.2em}\textit{suffix}}'.
Surely, the same effect is achieved by
directly specifying the
argument `{\textit{dest}\hspace{0.2em}\textit{suffix}}'
in the first form.
However, that requires to set up a different file
for each child. With the alternative form of the command
all these files can have exactly the same content
which simplifies setting them up and maintaining them.

For example, the following file |draft.tex|
with a compilation flag |\version| as described in \secref{sec:flags}
compiles the main document as a draft:
%
\begin{center}
\begin{tabular}{l}
|\def\version{draft}|\\
|\input{childdoc.def}|\\
|\childdocforward{|\textit{main}|}|
\end{tabular}
\end{center}
%
Likewise, the following files |final|\textit{nn}|.tex|
compile the final version of the child document
|child|\textit{nn}|.tex|:
%
\begin{center}
\begin{tabular}{l}
|\def\version{final}|\\
|\input{childdoc.def}|\\
|\childdocforwardprefix{final}{child}|
\end{tabular}
\end{center}
%

Note that when several versions of a main file and/or of each child file
are to be generated, it may be convenient to set up a |Makefile| or
shell script to automatise the process.

%%%%%%%%%%%%%%%%%%%%%%%%%%%%%%%%%%%%%%%%%%%%%%%%%%%%%%%%%%%%%%%%%%%%%%%%%%%%%%%%
\subsection{Command Line Processing}
\label{sec:commandline}

The effect of redirection files can also be achieved by invoking
the \LaTeX{} compiler with a more elaborate command line.
Most conveniently this should be done as part
of a shell script or a |Makefile|.

When using \textsf{childdoc} in the main file, the following
command lines effectively perform a redirection
(note that depending on the shell being used,
backslashes may have to be doubled: `|\|' $\to$ `|\\|'):
%
\begin{center}
|... -jobname "|\textit{target}|" |\\|"|[\textit{flags}]%
|\input{childdoc.def}\childdocforward[|\textit{main}|]{|\textit{dest}|}"|
\end{center}
%
Here \textit{target} is the name of the output file,
\textit{main} is the name of the main file
and \textit{dest} is the name of the main or child file to be processed
(all filenames without extensions).
The optional argument \textit{main} can be omitted
if \textit{main} matches \textit{dest}.
Optionally, compilation \textit{flags} can be defined via |\def| commands.
This command line makes the \TeX{} engine believe
it is compiling the file \textit{target}
whose content is specified as the latter parameter.
The provided code then forwards the processing to
\textit{main} or \textit{dest} as described in \secref{sec:forward}.

%%%%%%%%%%%%%%%%%%%%%%%%%%%%%%%%%%%%%%%%%%%%%%%%%%%%%%%%%%%%%%%%%%%%%%%%%%%%%%%%
\subsection{Include by Input}
\label{sec:input}

Including child documents by |\include| has some restrictions by design.
Most notably, the content of a child document always occupies
its own set of pages; pages cannot be shared between child documents.
Usually, this behaviour makes perfect sense
because each child document contain an essential part of the document.
However, in some situations it may be desirable to compose
a document from a collection of parts
without having mandatory page breaks between then.
For this case, the package
provides a mechanism to include parts
by |\input| which can also be processed individually.
However, by construction this mechanism
requires manual handling of the content to be output.

%%%%%%%%%%%%%%%%%%%%%%%%%%%%%%%%%%%%%%%%
\DescribeMacro{\ifchilddocmanual}
The main file should be prepared as usual, see \secref{sec:include}.
However, the document body must make a distinction
between processing of an individual part and of the main document, e.g.:
%
\begin{center}
\begin{tabular}{l}
|\ifchilddocmanual|\\
|\input{\childdocname}|\\
|\||else|\\
\textit{document body with }|\input{|\textit{part}|}|\\
|\||fi|
\end{tabular}
\end{center}
%
The conditional |\ifchilddocmanual| is true whenever
a part to be included by |\input| is being compiled,
and the name of the part is stored in |\childdocname|.

%%%%%%%%%%%%%%%%%%%%%%%%%%%%%%%%%%%%%%%%
\DescribeMacro{\childdocby}
Each part to be included by |\input| should start with:
%
\begin{center}
\begin{tabular}{l}
|\input{childdoc.def}|\\
|\childdocby{|\textit{main}|}|\\
\end{tabular}
\end{center}
%
The directive |\childdocby| is similar to |\childdocof|
described in \secref{sec:include},
but the subsequent selection of content must be done manually.
To that end, both |\ifchilddoc| and |\ifchilddocmanual|
will be true upon processing of a part,
and the name of the part is stored in |\childdocname|.
Note that |\jobname| will be set to the filename of the current part
so that each part receives an individual |.aux| file
that does not interfere with the |.aux| file(s) of the main document.
This behaviour can be altered by the alternative form
|\childdocby[*]{|\textit{main}|}| (with a non-empty optional argument)
which uses the |.aux| file of the main document
by setting |\jobname| to \textit{main}.

%%%%%%%%%%%%%%%%%%%%%%%%%%%%%%%%%%%%%%%%%%%%%%%%%%%%%%%%%%%%%%%%%%%%%%%%%%%%%%%%
\subsection{Driver Development}
\label{sec:driver}

The \textsf{childdoc} mechanism can also be use for the development
of definition files such as \LaTeX{} styles or classes.
This case differs from the above setup with multiple parts
included by |\include| in that no |\includeonly| should be invoked.
This can be achieved by starting the include file
(before |\ProvidesPackage|) with:
%
\begin{center}
\begin{tabular}{l}
|\input{childdoc.def}|\\
|\childdocforward{|\textit{main}|}|\\
\end{tabular}
\end{center}
%
or alternatively with:
%
\begin{center}
\begin{tabular}{l}
|\input{childdoc.def}|\\
|\childdocby{|\textit{main}|}|\\
\end{tabular}
\end{center}
%
Both forms have slightly different effects as described above.
The main file is prepared as usual, see \secref{sec:include}.

%%%%%%%%%%%%%%%%%%%%%%%%%%%%%%%%%%%%%%%%%%%%%%%%%%%%%%%%%%%%%%%%%%%%%%%%%%%%%%%%
\subsection{Legacy Detection}
\label{sec:detection}

The directive |\childdocmain| in the main file can detect
whether the complete document or merely a child is to be compiled
even without using the directive |\childdocof|.
This method is deprecated because it is less robust
and there is no compelling reason to use it;
it is merely provided for backward compatibility
and it may be removed in future versions.

If the detection mechanism is to be used,
it is mandatory to correctly specify
the filename of the main file as the argument of |\childdocmain|:
%
\begin{center}
\begin{tabular}{l}
|\input{childdoc.def}|\\
|\childdocmain{|\textit{main}|}|\\
\end{tabular}
\end{center}
%
If |\jobname| does not match the argument \textit{main} of |\childdocmain|,
it is assumed that |\jobname| points to the child file to be compiled.
When using |\childdocmain| with the main file specified as argument,
it suffices to start a child file
with just |\input{|\textit{main}|}|
without loading of the package and using |\childdocof|.
If instead all processing is done
with the appropriate \textsf{childdoc} directives,
the argument of \textit{main} of |\childdocmain| can be empty.

An alternative version of the command line processing described
in \secref{sec:commandline} using the detection mechanism reads:
%
\begin{center}
|... -jobname "|\textit{target}|" "|[\textit{flags}]%
[|\def\jobname{|\textit{dest}|}|]|\input{|\textit{main}|}"|
\end{center}

%%%%%%%%%%%%%%%%%%%%%%%%%%%%%%%%%%%%%%%%%%%%%%%%%%%%%%%%%%%%%%%%%%%%%%%%%%%%%%%%
\subsection{Manual Code}
\label{sec:manual}

In case one cannot be certain whether the definitions file |childdoc.def|
is installed on the target \TeX{} distribution
and one prefers not to ship it,
it is conceivable to paste a few relevant commands into the sources.

To that end, drop all statements |\input{childdoc.def}|
and perform the replacements as outlined below.
Instead of |\childdocmain{|\textit{main}|}| add the following code
to the top of the main file:
%
\begin{center}
\begin{tabular}{l}
|\||ifdefined\childdocname\endinput\||fi\newif\ifchilddoc|\\
|\edef\childdocname{\scantokens\expandafter{\jobname\noexpand}}|\\
|\def\childdocmain{|\textit{main}|}\||ifx\childdocmain\childdocname\||else|\\
|\childdoctrue\includeonly{\childdocname}\let\jobname\childdocmain\||fi|\\
\end{tabular}
\end{center}
%
Instead of |\childdocof{|\textit{main}|}| just include the main file
at the top of each child file:
%
\begin{center}
|\input{|\textit{main}|}|
\end{center}
%
A simple redirection |\childdocforward{|\textit{dest}|}| is achieved by:
%
\begin{center}
|\def\jobname{|\textit{dest}|}\input{\jobname}|
\end{center}
%
The redirection with prefix
|\childdocforwardprefix[|\textit{prefix}|]{|\textit{dest}|}|
is accomplished by:
%
\begin{center}
\begin{tabular}{l}
|{\edef\jobname{\scantokens\expandafter{\jobname\noexpand}}|\\
|\def\redirectjob |\textit{prefix}|#1~~~{\gdef\jobname{|\textit{dest}|#1}}|\\
|\expandafter\redirectjob\jobname~~~}\input{\jobname}|
\end{tabular}
\end{center}

In an alternative approach,
child documents can be compiled by a specific command line
without additional code or specific definitions:
%
\begin{center}
|... -jobname "|\textit{target}|" "|[\textit{flags}]%
|\includeonly{|\textit{dest}|}\input{|\textit{main}|}"|
\end{center}
%

%%%%%%%%%%%%%%%%%%%%%%%%%%%%%%%%%%%%%%%%%%%%%%%%%%%%%%%%%%%%%%%%%%%%%%%%%%%%%%%%
%%%%%%%%%%%%%%%%%%%%%%%%%%%%%%%%%%%%%%%%%%%%%%%%%%%%%%%%%%%%%%%%%%%%%%%%%%%%%%%%
\section{Information}

%%%%%%%%%%%%%%%%%%%%%%%%%%%%%%%%%%%%%%%%%%%%%%%%%%%%%%%%%%%%%%%%%%%%%%%%%%%%%%%%
\subsection{Copyright}

Copyright \copyright{} 2017--2018 Niklas Beisert

This work may be distributed and/or modified under the
conditions of the \LaTeX{} Project Public License, either version 1.3
of this license or (at your option) any later version.
The latest version of this license is in
  \url{http://www.latex-project.org/lppl.txt}
and version 1.3 or later is part of all distributions of \LaTeX{}
version 2005/12/01 or later.

This work has the LPPL maintenance status `maintained'.

The Current Maintainer of this work is Niklas Beisert.

This work consists of the files |README.txt|, |childdoc.ins| and |childdoc.dtx|
as well as the derived files |childdoc.def|, |cdocsamp.tex|
with |cdocsch1.tex|, |cdocsch2.tex|, |cdocspt3.tex|, |cdocspt4.tex|,
|cdocsdrf.tex|, |cdocsfn1.tex|, |cdocsfn2.tex|
as well as |childdoc.pdf|.

%%%%%%%%%%%%%%%%%%%%%%%%%%%%%%%%%%%%%%%%%%%%%%%%%%%%%%%%%%%%%%%%%%%%%%%%%%%%%%%%
\subsection{Files and Installation}

The package consists of the files:
%
\begin{center}
\begin{tabular}{ll}
    |README.txt|   & readme file \\
    |childdoc.ins| & installation file \\
    |childdoc.dtx| & source file \\
    |childdoc.def| & definition file \\
    |cdocsamp.tex| & sample main file \\
    |cdocsch1.tex| & sample include file \\
    |cdocsch2.tex| & sample include file \\
    |cdocspt3.tex| & sample part file \\
    |cdocspt4.tex| & sample part file \\
    |cdocsdrf.tex| & sample redirection file \\
    |cdocsfn1.tex| & sample redirection file \\
    |cdocsfn2.tex| & sample redirection file \\
    |childdoc.pdf| & manual
\end{tabular}
\end{center}
%
The distribution consists of the files
|README.txt|, |childdoc.ins| and |childdoc.dtx|.
%
\begin{itemize}
\item
Run (pdf)\LaTeX{} on |childdoc.dtx|
to compile the manual |childdoc.pdf| (this file).
\item
Run \LaTeX{} on |childdoc.ins| to create the definitions file |childdoc.def|
and the sample |cdocsamp.tex| with include files
|cdocsch1.tex|, |cdocsch2.tex|, |cdocspt3.tex|, |cdocspt4.tex|,
|cdocsdrf.tex|, |cdocsfn1.tex|, |cdocsfn2.tex|.
Then copy the file |childdoc.def| to an appropriate directory of your \LaTeX{}
distribution, e.g.\ \textit{texmf-root}|/tex/latex/childdoc|.
\end{itemize}

%%%%%%%%%%%%%%%%%%%%%%%%%%%%%%%%%%%%%%%%%%%%%%%%%%%%%%%%%%%%%%%%%%%%%%%%%%%%%%%%
\subsection{Related CTAN Packages}

There are several other packages which offer a similar functionality:
%
\begin{itemize}
\item
The packages
\href{http://ctan.org/pkg/docmute}{\textsf{docmute}},
\href{http://ctan.org/pkg/includex}{\textsf{includex}} and
\href{http://ctan.org/pkg/standalone}{\textsf{standalone}}
provide commands to include only the document body of
a child file thus allowing both files to be compiled individually.
\item
The packages \href{http://ctan.org/pkg/subdocs}{\textsf{subdocs}}
and \href{http://ctan.org/pkg/subfiles}{\textsf{subfiles}}
provide structures in which the main and child documents can be
encapsulated and allowing them to be compiled individually.
The inclusion mechanism is different from the conventional |\include|.
\item
The package \href{http://ctan.org/pkg/combine}{\textsf{combine}}
is an elaborate solution to combine several documents into one.
\end{itemize}
%
See also the CTAN topic \href{http://ctan.org/topic/subdocs}{\textsf{subdocs}}
for further related packages.
The present package differs from the above solutions in that
a document structure constructed with the conventional |\include| mechanism
just needs two extra commands at the top of every file
such that all constituent files can be compiled individually.

%%%%%%%%%%%%%%%%%%%%%%%%%%%%%%%%%%%%%%%%%%%%%%%%%%%%%%%%%%%%%%%%%%%%%%%%%%%%%%%%
%\subsection{Feature Suggestions}
%
%The following is a list of features which may be useful for future
%versions of this package:
%%
%\begin{itemize}
%\item
%\ldots
%\end{itemize}

%%%%%%%%%%%%%%%%%%%%%%%%%%%%%%%%%%%%%%%%%%%%%%%%%%%%%%%%%%%%%%%%%%%%%%%%%%%%%%%%
\subsection{Revision History}

%%%%%%%%%%%%%%%%%%%%%%%%%%%%%%%%%%%%%%%%
\paragraph{v2.0:} 2018/12/30

\begin{itemize}
\item
immediate forward processing
\item
added |\childdocby| mechanism
\item
manual restructured
\end{itemize}

%%%%%%%%%%%%%%%%%%%%%%%%%%%%%%%%%%%%%%%%
\paragraph{v1.6:} 2018/01/17

\begin{itemize}
\item
application for development of include files
\item
corrections to manual
\end{itemize}

%%%%%%%%%%%%%%%%%%%%%%%%%%%%%%%%%%%%%%%%
\paragraph{v1.5:} 2017/05/21

\begin{itemize}
\item
more complete structuring introduced
\item
|\childdocof| introduced
\item
|\childdoc| renamed to |\childdocmain|
\item
|\childredirect| renamed to |\childdocforward| and |\childdocforwardprefix|
and functionality expanded
\end{itemize}

%%%%%%%%%%%%%%%%%%%%%%%%%%%%%%%%%%%%%%%%
\paragraph{v1.0:} 2017/04/27

\begin{itemize}
\item
manual and install package
\item
first version published on CTAN
\end{itemize}

%%%%%%%%%%%%%%%%%%%%%%%%%%%%%%%%%%%%%%%%
\paragraph{v0.6:} 2017/04/26

\begin{itemize}
\item
redirection mechanism added
\end{itemize}

%%%%%%%%%%%%%%%%%%%%%%%%%%%%%%%%%%%%%%%%
\paragraph{v0.5:} 2017/04/26

\begin{itemize}
\item
functionality in definition file
\end{itemize}


%%%%%%%%%%%%%%%%%%%%%%%%%%%%%%%%%%%%%%%%%%%%%%%%%%%%%%%%%%%%%%%%%%%%%%%%%%%%%%%%
%%%%%%%%%%%%%%%%%%%%%%%%%%%%%%%%%%%%%%%%%%%%%%%%%%%%%%%%%%%%%%%%%%%%%%%%%%%%%%%%
%%%%%%%%%%%%%%%%%%%%%%%%%%%%%%%%%%%%%%%%%%%%%%%%%%%%%%%%%%%%%%%%%%%%%%%%%%%%%%%%
\appendix

\settowidth\MacroIndent{\rmfamily\scriptsize 000\ }

 \DocInput{childdoc.dtx}

\end{document}
%</driver>
% \fi
%
% %%%%%%%%%%%%%%%%%%%%%%%%%%%%%%%%%%%%%%%%%%%%%%%%%%%%%%%%%%%%%%%%%%%%%%%%%%%%%%
% %%%%%%%%%%%%%%%%%%%%%%%%%%%%%%%%%%%%%%%%%%%%%%%%%%%%%%%%%%%%%%%%%%%%%%%%%%%%%%
% \section{Sample}
%\iffalse
%<*samplemain>
%\fi
%
% The following presents a sample document
% with two chapters, two parts, a title page,
% a compile flag as well as three forwarding files to set the flag.
% It consists of eight |.tex| files:
% \begin{center}
% \begin{tabular}{ll}
% |cdocsamp.tex|&main file\\
% |cdocsch1.tex|&include file for chapter 1\\
% |cdocsch2.tex|&include file for chapter 2\\
% |cdocspt3.tex|&include file for part 3\\
% |cdocspt4.tex|&include file for part 4\\
% |cdocsdrf.tex|&forwarding file for main file in draft mode\\
% |cdocsfi1.tex|&forwarding file for final version of chapter 1\\
% |cdocsfi2.tex|&forwarding file for final version of chapter 2\\
% \end{tabular}
% \end{center}
% Each of the eight files can be compiled directly by the \LaTeX{} compiler.
%
% %%%%%%%%%%%%%%%%%%%%%%%%%%%%%%%%%%%%%%
% \paragraph{Main File.}
%
% The main file is called |cdocsamp.tex|.
%
% Load the \textsf{childdoc} definitions and
% declare the filename for the main document:
%    \begin{macrocode}
\input{childdoc.def}
\childdocmain{}
%    \end{macrocode}

% Optional override for |\version| flag:
%    \begin{macrocode}
%%\ifchilddoc\else\providecommand{\version}{draft}\fi
%    \end{macrocode}

% Define the default values for the |\version| flag
% (|final| for the main file and |draft| for childs):
%    \begin{macrocode}
\ifchilddoc
\providecommand{\version}{draft}
\else
\providecommand{\version}{final}
\fi
%    \end{macrocode}

% Load the standard document class:
%    \begin{macrocode}
\documentclass[12pt]{article}
%    \end{macrocode}

% Start the document body:
%    \begin{macrocode}
\begin{document}
%    \end{macrocode}

% Declare a title page.
% Print title, part of document being processed and version flag:
%    \begin{macrocode}
\addtocounter{page}{-1}
\begin{center}
{\LARGE\bfseries{}childdoc example\par}
\vspace{1cm}
\ifchilddoc
\ifchilddocmanual part\else chapter\fi:
`\childdocname' of `\childdocjob'\par
\else
main document: `\childdocjob'\par
\fi
version: \version\par
\end{center}
\newpage
%    \end{macrocode}

% Manually include selected file,
% otherwise process as usual:
%    \begin{macrocode}
\ifchilddocmanual
\section*{part `\childdocname'}
\input{\childdocname}
\else
%    \end{macrocode}

% Include the two chapters:
%    \begin{macrocode}
\include{cdocsch1}
\include{cdocsch2}
%    \end{macrocode}

% Include the two parts unless only chapters should be displayed:
%    \begin{macrocode}
\ifchilddoc\else
\section{part three}
\input{cdocspt3}
\section{part four}
\input{cdocspt4}
\fi
%    \end{macrocode}

% Process as usual until here:
%    \begin{macrocode}
\fi
%    \end{macrocode}

% End of document body:
%    \begin{macrocode}
\end{document}
%    \end{macrocode}
%\iffalse
%</samplemain>
%\fi
%
% %%%%%%%%%%%%%%%%%%%%%%%%%%%%%%%%%%%%%%
% \paragraph{Chapter Include Files.}
%
% The include files are called |cdocsch1.tex| and |cdocsch2.tex|.
%
%\iffalse
%<*samplechap1|samplechap2>
%\fi

% Optional override for |\version| flag:
%    \begin{macrocode}
%%\providecommand{\version}{final}
%    \end{macrocode}

% Include the main document:
%    \begin{macrocode}
\input{childdoc.def}
\childdocof{cdocsamp}
%    \end{macrocode}

%\iffalse
%</samplechap1|samplechap2>
%\fi
%
%\iffalse
%<*samplechap1>
%\fi
% Some text for chapter 1:
%    \begin{macrocode}
\section{one}
some text in chapter one
%    \end{macrocode}

%\iffalse
%</samplechap1>
%\fi
% Some text for chapter 2:
%\iffalse
%<*samplechap2>
%\fi
%    \begin{macrocode}
\section{two}
more text in chapter two
%    \end{macrocode}

%\iffalse
%</samplechap2>
%\fi
%
% %%%%%%%%%%%%%%%%%%%%%%%%%%%%%%%%%%%%%%
% \paragraph{Part Include Files.}
%
% The include files are called |cdocspt3.tex| and |cdocspt4.tex|.
%
%\iffalse
%<*samplepart3|samplepart4>
%\fi

% Optional override for |\version| flag:
%    \begin{macrocode}
%%\providecommand{\version}{final}
%    \end{macrocode}

% Include the main document:
%    \begin{macrocode}
\input{childdoc.def}
\childdocby{cdocsamp}
%    \end{macrocode}

%\iffalse
%</samplepart3|samplepart4>
%\fi
%
%\iffalse
%<*samplepart3>
%\fi
% Some text for part 3:
%    \begin{macrocode}
some text in part three
%    \end{macrocode}

%\iffalse
%</samplepart3>
%\fi
% Some text for part 4:
%\iffalse
%<*samplepart4>
%\fi
%    \begin{macrocode}
more text in part four
%    \end{macrocode}

%\iffalse
%</samplepart4>
%\fi
%
% %%%%%%%%%%%%%%%%%%%%%%%%%%%%%%%%%%%%%%
% \paragraph{Forwarding for a Complete Draft.}
%
% The following forwarding file |cdocsdrf.tex|
% compiles the main document in draft mode:
%\iffalse
%<*sampledraft>
%\fi
%    \begin{macrocode}
\def\version{draft}
\input{childdoc.def}
\childdocforward{cdocsamp}
%    \end{macrocode}

%\iffalse
%</sampledraft>
%\fi
%
% %%%%%%%%%%%%%%%%%%%%%%%%%%%%%%%%%%%%%%
% \paragraph{Forwarding for Final Version of the Chapters.}
%
% The following forwarding files |cdocsfn1.tex| and |cdocsfn2.tex|
% (with identical content)
% compile the final versions of the child documents
% |cdocsch1.tex| and |cdocsch2.tex|, respectively:
%\iffalse
%<*samplefinal>
%\fi
%    \begin{macrocode}
\def\version{final}
\input{childdoc.def}
\childdocforwardprefix[cdocsamp]{cdocsfn}{cdocsch}
%    \end{macrocode}

%\iffalse
%</samplefinal>
%\fi
%
% %%%%%%%%%%%%%%%%%%%%%%%%%%%%%%%%%%%%%%
% \paragraph{Command Line Processing.}
%
% The following three command lines generate the output files
% |cdocscld|, |cdocscl1| and |cdocscl2|
% which should be identical to
% |cdocsdrf|, |cdocsch1| and |cdocsfn2|, respectively:
% \begin{center}
% \begin{tabular}{l}
% |latex -jobname cdocscld \|\\
% |  "\def\version{draft}\input{childdoc.def}\childdocforward{cdocsamp}"|\\
% |latex -jobname cdocscl1 \|\\
% |  "\input{childdoc.def}\childdocforward[cdocsamp]{cdocsch1}"|\\
% |latex -jobname cdocscl2 \|\\
% |  "\def\version{final}\input{childdoc.def}\childdocforward{cdocsch2}"|
% \end{tabular}
% \end{center}
% Note that the trailing backslash on each first line
% merely continues the input to the second line
% (for convenient cut ant paste).
% Furthermore, the command |latex| can be replaced by any
% of its alternative versions such as |pdflatex|.
%
% %%%%%%%%%%%%%%%%%%%%%%%%%%%%%%%%%%%%%%%%%%%%%%%%%%%%%%%%%%%%%%%%%%%%%%%%%%%%%%
% %%%%%%%%%%%%%%%%%%%%%%%%%%%%%%%%%%%%%%%%%%%%%%%%%%%%%%%%%%%%%%%%%%%%%%%%%%%%%%
% \section{Implementation}
%\iffalse
%<*package>
%\fi
%
% This section describes the definitions file |childdoc.def|.

% The definitions cannot be loaded using |\usepackage| or |\RequirePackage|
% which has a mechanism to prevent loading a style file more than once.
% When loading the definitions by means of |\input|
% multiple instances have to be prevented manually:
%\iffalse
%This code needs to be before the `\ProvidesFile' directive
%which is defined at the beginning of this file.
%Therefore it is also placed there and commented out here.
%</package>
%<*discard>
%\fi
%    \begin{macrocode}
\ifdefined\childdocmain\endinput\fi
%    \end{macrocode}
%\iffalse
%</discard>
%<*package>
%\fi
%
% \macro{\ifchilddoc}
% \macro{\ifchilddocmanual}
% The conditional |\ifchilddoc| tells whether a
% child (true) or main (false) document is being compiled.
% The conditional |\ifchilddocmanual| tells whether
% the |\includeonly| mechanism is used (false) or
% the selection of child files must be performed manually (true).
% The definitions initialise to false:
%    \begin{macrocode}
\newif\ifchilddoc
\newif\ifchilddocmanual
%    \end{macrocode}

% \macro{\childdocname}
% \macro{\childdocjob}
% The macro |\childdocname| stores the name of the main document
% to be compiled. The macro |\childdocjob| stores the name of
% the document on which the \LaTeX{} compiler was originally invoked.
% The content of |\jobname| cannot be compared
% to filenames specified in the source due to different catcodes.
% The following code rescans |\jobname|, stores the result
% in |\childdocname| and saves a copy in |\childdocjob|:
%    \begin{macrocode}
\edef\childdocname{\scantokens\expandafter{\jobname\noexpand}}
\let\childdocjob\childdocname
%    \end{macrocode}

% \macro{\childdocdisable}
% The macro |\childdocdisable| prevents the main file
% from being processed more than once.
% At this stage, the main document command |\childdocmain|
% is assumed to be called once again where it should do nothing.
% Any subsequent call to it should prevent
% a secondary processing of the main document
% It overwrites the forwarding commands
% |\childdocof| and |\childdocforward|
% with empty macros to prevent further inclusions of the main document:
%    \begin{macrocode}
\newcommand{\childdocdisable}
{
  \renewcommand{\childdocmain}[1]{\renewcommand{\childdocmain}[1]{\endinput}}
  \renewcommand{\childdocof}[1]{}
  \renewcommand{\childdocby}[2][]{}
  \renewcommand{\childdocforward}[2][]{}
  \renewcommand{\childdocdisable}{}
}
%    \end{macrocode}

% \macro{\childdocmain}
% The macro |\childdocmain| is to be called at the top of the main file
% with nothing or the main filename (without extension) as argument.
% First, it breaks loops.
% If the argument is not empty and does not match |\childdocname|
% (which is set by the first inclusion of |childdoc.def|),
% |\ifchilddoc| is set to true, |\includeonly| is applied to the child file
% and |\jobname| is set to the main file
% (for proper handling of |.aux| files):
%    \begin{macrocode}
\newcommand{\childdocmain}[1]
{
  \childdocdisable\childdocmain{}
  \if?#1?\else
    \begingroup
      \def\childdoctmp{#1}
      \ifx\childdoctmp\childdocname
        \def\childdoctmp{}
      \else
        \def\childdoctmp
        {
          \childdoctrue
          \includeonly{\childdocname}
          \def\childdocjob{#1}
          \def\jobname{#1}
        }
      \fi
      \expandafter
    \endgroup
    \childdoctmp
  \fi
}
%    \end{macrocode}

% \macro{\childdocof}
% The command |\childdocof| redirects
% compilation to the main file |#1|.
%    \begin{macrocode}
\newcommand{\childdocof}[1]
{
  \childdocdisable
  \childdoctrue
  \includeonly{\childdocname}
  \def\jobname{#1}
  \def\childdocjob{#1}
  \input{#1}
}
%    \end{macrocode}

% \macro{\childdocby}
% The command |\childdocby| ....
%    \begin{macrocode}
\newcommand{\childdocby}[2][]
{
  \childdocdisable
  \childdoctrue
  \childdocmanualtrue
  \if?#1?\else
    \def\jobname{#2}
  \fi
  \def\childdocjob{#2}
  \input{#2}
  \endinput
}
%    \end{macrocode}

% \macro{\childdocforward}
% The command |\childdocforward| redirects
% compilation to the main file or
% (if the optional argument is given) a child file.
% Parameters are set as if the main file
% or a child file starting with |\childdocof| was compiled.
% Then compilation is handed over to the main file:
%    \begin{macrocode}
\newcommand{\childdocforward}[2][]
{
  \begingroup
    \if?#1?
      \def\childdoctmp
      {
        \def\childdocname{#2}
        \def\childdocjob{#2}
        \def\jobname{#2}
        \input{#2}
        \endinput
      }
    \else
      \def\childdoctmp
      {
        \childdocdisable
        \def\childdocname{#2}
        \childdoctrue
        \includeonly{#2}
        \def\childdocjob{#1}
        \def\jobname{#1}
        \input{#1}
        \endinput
      }
    \fi
    \expandafter
  \endgroup
  \childdoctmp
}
%    \end{macrocode}

% \macro{\childdocforwardprefix}
% The command |\childdocforwardprefix| redirects
% compilation to the main or a child file by means of a pattern.
% The prefix |#1| in the current filename is replaced by |#2|
% and the suffix of the current filename is kept
% (it is assumed that the filename does not contain the substring `|~~~|'
% which is used as a delimiter).
% Compilation is handed over to the new file by |\childdocforward|:
%    \begin{macrocode}
\newcommand{\childdocforwardprefix}[3][]
{
  \begingroup
    \def\childdocextract #2##1~~~{\def\childdoctmp{\childdocforward[#1]{#3##1}}}
    \expandafter\childdocextract\childdocname~~~
    \expandafter
  \endgroup
  \childdoctmp
}
%    \end{macrocode}

% \macro{\childdoc}
% The deprecated macro |\childdoc| is a legacy version of |\childdocmain|:
%    \begin{macrocode}
\newcommand{\childdoc}{\childdocmain}
%    \end{macrocode}

% \macro{\childdocredirect}
% The deprecated macro |\childdocredirect| is a legacy version
% of |\childdocforward| and |\childdocforwardprefix|:
%    \begin{macrocode}
\newcommand{\childdocredirect}[2][]
{
  \begingroup
    \if?#1?
      \def\childdoctmp{\childdocforward{#2}}
    \else
      \def\childdoctmp{\childdocforwardprefix{#1}{#2}}
    \fi
    \expandafter
  \endgroup
  \childdoctmp
}
%    \end{macrocode}

%\iffalse
%</package>
%\fi
%
\endinput
|\\
|\childdocforwardprefix[|\textit{main}|]{|\textit{prefix}|}{|\textit{dest}|}|
\end{tabular}
\end{center}
%
the destination file is determined by a pattern
depending on the current file:
To make this work, the current file must be called
`{\textit{prefix}\hspace{0.2em}\textit{suffix}}'
with \textit{prefix} matching precisely the argument.
Processing is then passed on to the file
`{\textit{dest}\hspace{0.2em}\textit{suffix}}'.
Surely, the same effect is achieved by
directly specifying the
argument `{\textit{dest}\hspace{0.2em}\textit{suffix}}'
in the first form.
However, that requires to set up a different file
for each child. With the alternative form of the command
all these files can have exactly the same content
which simplifies setting them up and maintaining them.

For example, the following file |draft.tex|
with a compilation flag |\version| as described in \secref{sec:flags}
compiles the main document as a draft:
%
\begin{center}
\begin{tabular}{l}
|\def\version{draft}|\\
|% \iffalse
%
% childdoc.dtx Copyright (C) 2017-2018 Niklas Beisert
%
% This work may be distributed and/or modified under the
% conditions of the LaTeX Project Public License, either version 1.3
% of this license or (at your option) any later version.
% The latest version of this license is in
%   http://www.latex-project.org/lppl.txt
% and version 1.3 or later is part of all distributions of LaTeX
% version 2005/12/01 or later.
%
% This work has the LPPL maintenance status `maintained'.
%
% The Current Maintainer of this work is Niklas Beisert.
%
% This work consists of the files childdoc.dtx and childdoc.ins
% and the derived files childdoc.def and cdocsamp.tex with
% cdocsch1.tex, cdocsch2.tex, cdocsdrf.tex, cdocsfn1.tex, cdocsfn2.tex.
%
%<package>\ifdefined\childdocmain\endinput\fi
%<package>\ProvidesFile{childdoc.def}[2018/12/30 v2.0 child document driver]
%<samplemain>\ProvidesFile{cdocsamp.tex}[2018/12/30 v2.0 sample for childdoc]
%<*driver>
%\ProvidesFile{childdoc.drv}[2018/12/30 v2.0 childdoc reference manual file]
\PassOptionsToClass{10pt,a4paper}{article}
\documentclass{ltxdoc}

\usepackage[margin=35mm]{geometry}
\usepackage{hyperref}
\usepackage{hyperxmp}
\usepackage[usenames]{color}

\hypersetup{colorlinks=true}
\hypersetup{pdfstartview=FitH}
\hypersetup{pdfpagemode=UseNone}
\hypersetup{pdfsource={}}
\hypersetup{pdflang={en-UK}}
\hypersetup{pdfcopyright={Copyright 2017-2018 Niklas Beisert.
  This work may be distributed and/or modified under the
  conditions of the LaTeX Project Public License, either version 1.3
  of this license or (at your option) any later version.}}
\hypersetup{pdflicenseurl={http://www.latex-project.org/lppl.txt}}
\hypersetup{pdfcontactaddress={ETH Zurich, ITP, HIT K,
  Wolfgang-Pauli-Strasse 27}}
\hypersetup{pdfcontactpostcode={8093}}
\hypersetup{pdfcontactcity={Zurich}}
\hypersetup{pdfcontactcountry={Switzerland}}
\hypersetup{pdfcontactemail={nbeisert@itp.phys.ethz.ch}}
\hypersetup{pdfcontacturl={http://people.phys.ethz.ch/\xmptilde nbeisert/}}

\newcommand{\secref}[1]{\hyperref[#1]{section \ref*{#1}}}

\parskip1ex
\parindent0pt
\let\olditemize\itemize
\def\itemize{\olditemize\parskip0pt}

\begin{document}

\title{The \textsf{childdoc} Package}
\hypersetup{pdftitle={The childdoc Package}}
\author{Niklas Beisert\\[2ex]
  Institut f\"ur Theoretische Physik\\
  Eidgen\"ossische Technische Hochschule Z\"urich\\
  Wolfgang-Pauli-Strasse 27, 8093 Z\"urich, Switzerland\\[1ex]
  \href{mailto:nbeisert@itp.phys.ethz.ch}
  {\texttt{nbeisert@itp.phys.ethz.ch}}}
\hypersetup{pdfauthor={Niklas Beisert}}
\hypersetup{pdfsubject={Manual for the LaTeX2e Package childdoc}}
\date{30 December 2018, \textsf{v2.0}}
\maketitle

\begin{abstract}\noindent
\textsf{childdoc} is a \LaTeXe{} package
that enables the direct compilation
of document sections included by |\include|
to individual files.
\end{abstract}

\begingroup
\parskip0ex
\tableofcontents
\endgroup

%%%%%%%%%%%%%%%%%%%%%%%%%%%%%%%%%%%%%%%%%%%%%%%%%%%%%%%%%%%%%%%%%%%%%%%%%%%%%%%%
%%%%%%%%%%%%%%%%%%%%%%%%%%%%%%%%%%%%%%%%%%%%%%%%%%%%%%%%%%%%%%%%%%%%%%%%%%%%%%%%
\section{Introduction}

\LaTeX{} provides a mechanism to structure a large document (such as a book)
into a main file and several child files (containing the chapters)
using the |\include| command.
This mechanism is beneficial for documents
which span hundreds of pages in order to
make the source file(s) more manageable.
Moreover, compilation can be restricted to
selected child files by means of the |\includeonly| command.
The latter feature can be used to reduce the compilation time while editing
(this was significantly more useful in the earlier days of \LaTeX{})
or to generate a smaller document which is easier to navigate.
Another application of |\includeonly| is to generate
documents consisting of selected parts of the complete document.

However, there are a few drawbacks of the plain |\include| mechanism:
\begin{itemize}
\item
The child files cannot be compiled on their own,
they can only be compiled via the main file.
A naive editing environment
(such as a text editor with an option
to have the current file processed by \LaTeX)
may require one to switch to the main file before compiling;
attempting to compile the child file produces errors.
\item
The main file must be modified (each time)
to adjust the |\includeonly| command
to the present needs. This easily leaves the main file in a messy state.
\item
The generated document will always carry the filename
of the main document. This is inconvenient if
several child files are to be compiled and
to be kept for distribution.
\end{itemize}

The present package provides a simple interface
to make child files individually compilable by \LaTeX{}.
Compiling a child file then has the same effect as compiling
the main file with an |\includeonly| command
to select the appropriate child.
Moreover the generated document will carry the name of the child
rather than the main file.
This resolves all three above issues.

This feature is meant to make the editing of books,
thesis documents and lecture notes somewhat more convenient.
However, the package can also be used efficiently for
composing a series of documents (such as exercise sheets)
which are typically distributed individually.
It then assists the author in generating the individual documents
(potentially in different versions)
as well as a document containing the collected series.
Another application is in developing style files
or other kinds of included material
where compilation of the style file could redirect
to a sample or test file.

%%%%%%%%%%%%%%%%%%%%%%%%%%%%%%%%%%%%%%%%%%%%%%%%%%%%%%%%%%%%%%%%%%%%%%%%%%%%%%%%
%%%%%%%%%%%%%%%%%%%%%%%%%%%%%%%%%%%%%%%%%%%%%%%%%%%%%%%%%%%%%%%%%%%%%%%%%%%%%%%%
\section{Usage}

First of all, the package \textsf{childdoc} is \emph{not} a standard
\LaTeXe{} |.sty| style file! Therefore it needs to be invoked in
a non-standard way.

%%%%%%%%%%%%%%%%%%%%%%%%%%%%%%%%%%%%%%%%%%%%%%%%%%%%%%%%%%%%%%%%%%%%%%%%%%%%%%%%
\subsection{Included Files}
\label{sec:include}

%%%%%%%%%%%%%%%%%%%%%%%%%%%%%%%%%%%%%%%%
\DescribeMacro{\childdocmain}
To use the package, add the commands
\begin{center}
\begin{tabular}{l}
|\input{childdoc.def}|\\
|\childdocmain{}|\\
\end{tabular}
\end{center}
at the very top of the main \LaTeX{} file,
in particular \emph{before} the |\documentclass| statement!
The argument of |\childdocmain| should be left empty
(but it must be present).

%%%%%%%%%%%%%%%%%%%%%%%%%%%%%%%%%%%%%%%%
\DescribeMacro{\childdocof}
Furthermore, add the commands
\begin{center}
\begin{tabular}{l}
|\input{childdoc.def}|\\
|\childdocof{|\textit{main}|}|\\
\end{tabular}
\end{center}
at the top of every child file \textit{child}
which is included by |\include{|\textit{child}|}|
from within the main file
(or at least for those files to be compiled individually).
The argument \textit{main} must be the filename of the main file.

There are a couple of
considerations in setting up the main and child documents:

%%%%%%%%%%%%%%%%%%%%%%%%%%%%%%%%%%%%%%%%
\paragraph{Restrictions.}

Please note the following restrictions:
\begin{itemize}
\item
|\childdocmain| must be called with one argument \textit{main}
to ensure compatibility with earlier version of the package.
It must either be empty (|\childdocmain{}|)
or precisely match the filename of the main file in which it is specified.
See \secref{sec:detection} for further information.
\item
The filename \textit{main} must be specified without the |.tex| extension.
\item
The filename \textit{main} is case sensitive
(even in case-insensitive file systems)
due to internal string comparison.
\item
The argument \textit{main} should be fully expanded, it cannot be a macro.
\item
Subdirectories and special characters should be avoided in filenames.
\item
The command |\childdocmain{|\textit{main}|}| must be followed by a whitespace.
It should not be followed immediately by another command
or by a comment mark `|%|'.
This is because the \TeX{} parser reads the token immediately following
the argument of |\childdocmain| and puts it
at the beginning of every child section;
however, a white\-space is ignored.
\end{itemize}

%%%%%%%%%%%%%%%%%%%%%%%%%%%%%%%%%%%%%%%%
\paragraph{Content of Main File.}

It is advisable to place all content in the child files included by |\include|.
Any output contained in the main file will appear in all child documents
unless suppressed manually;
it cannot be suppressed automatically by the |\includeonly| directive
and thus should normally be avoided.
A method to include some content in the main file
by means of conditional processing is described in \secref{sec:conditional}.

%%%%%%%%%%%%%%%%%%%%%%%%%%%%%%%%%%%%%%%%
\paragraph{Page Numbering.}

When only a part of the document is compiled,
the appropriate numbering of pages
(as well as other status parameters)
is determined from the |.aux| files.
The latter contain information from previous passes.
However this information needs to propagate through
all intermediate child documents.
Therefore the page numbering in child documents may well
be inconsistent until the complete document is compiled at least once.

A useful (if unconventional) way to always ensure a consistent
page numbering is to restart the numbering in each child document
and denote the pages by `\textit{child}|.|\textit{page}'
where \textit{child} represents the chapter/section number of the child file.
This can be achieved by the command
|\numberwithin{page}{|\textit{child}|}|
of the \textsf{amsmath} package
where \textit{child} can be |chapter| or |section|
depending on the chosen structuring.
Alternatively, one can modify the macro |\thepage| appropriately
and reset the counter |page| at the start of each child file.

%%%%%%%%%%%%%%%%%%%%%%%%%%%%%%%%%%%%%%%%%%%%%%%%%%%%%%%%%%%%%%%%%%%%%%%%%%%%%%%%
\subsection{Conditional Processing}
\label{sec:conditional}

The package provides a mechanism to compile different versions
of a document. To customise the versions further some conditional processing
can come in handy to distinguish which version is being compiled.
The package provides two macros to describe the compilation context:

%%%%%%%%%%%%%%%%%%%%%%%%%%%%%%%%%%%%%%%%
\DescribeMacro{\ifchilddoc}
The conditional |\ifchilddoc| distinguishes between the compilation of
child documents and the main document:
%
\begin{center}
|\ifchilddoc |\textit{child-code}| |[|\||else |\textit{main-code}]| \||fi|
\end{center}

%%%%%%%%%%%%%%%%%%%%%%%%%%%%%%%%%%%%%%%%
\DescribeMacro{\childdocname}
\DescribeMacro{\childdocjob}
The macro |\childdocname| contains the filename (without extension)
of the main or child file being processed.
Note that |\childdocjob| will always contain the name of the main file.

%%%%%%%%%%%%%%%%%%%%%%%%%%%%%%%%%%%%%%%%
\paragraph{Title Page.}

Conditional processing can be used to include a title or banner page
in the main document when proper precautions are taken.
Importantly, the code in the main file should ensure that the page counter
(as well as other status parameters which are stored in the |.aux| files)
takes the same value after the conditional processing.
Otherwise the page numbers may take divergent values
depending on which part is compiled.

For example, a title page could be declared by:
%
\begin{center}
\begin{tabular}{l}
|\ifchilddoc\||else|\\
|\addtocounter{page}{-1}|\\
\textit{code for title page}\\
|\newpage|\\
|\||fi|
\end{tabular}
\end{center}
%
A banner page for the child documents can be generated by:
%
\begin{center}
\begin{tabular}{l}
|\ifchilddoc|\\
|\addtocounter{page}{-1}|\\
\textit{code for banner page}\\
|\newpage|\\
|\||fi|
\end{tabular}
\end{center}
%
Here one could write a message such as:
\begin{center}
|This is the part \childdocname{} of \childdocjob{}.|
\end{center}

%%%%%%%%%%%%%%%%%%%%%%%%%%%%%%%%%%%%%%%%%%%%%%%%%%%%%%%%%%%%%%%%%%%%%%%%%%%%%%%%
\subsection{Flags}
\label{sec:flags}

The package makes it easy to generate different versions
of the main or child documents.
To this end compilation flags can be defined
and assigned different default values.
They will be particularly useful in conjunction
with the forwarding mechanism described in \secref{sec:forward}.

For example, it may be useful to have a flag |\version|
which can be set to |draft| or |final|.
The document source will contain some conditional code
depending on the value of |\version|.
Suppose further, the flag should default to |final| for the main file
and to |draft| for child files
which is a natural assignment for editing the document.
This is achieved by placing the following code
in the preamble of the main document
(below the |\childdocmain| directive):
%
\begin{center}
\begin{tabular}{l}
|\ifchilddoc|\\
|\providecommand{\version}{draft}|\\
|\||else|\\
|\providecommand{\version}{final}|\\
|\||fi|
\end{tabular}
\end{center}
%
The definition by |\providecommand| makes sure
that previous definitions are not overwritten.
Further statements |\providecommand{\version}{...}|
can thus be added before the above code to override it.

For the main file, one might add a line
(between |\childdocmain| and the above block)
%
\begin{center}
|%\ifchilddoc\||else\providecommand{\version}{draft}\||fi|
\end{center}
%
which can be uncommented to produce a draft version.
Likewise one can add a line to the very top of a child file
(above the |\childdocof{|\textit{main}|}| directive)
%
\begin{center}
|%\providecommand{\version}{final}|
\end{center}
%
which can be uncommented to produce the final version of this child document.

%%%%%%%%%%%%%%%%%%%%%%%%%%%%%%%%%%%%%%%%%%%%%%%%%%%%%%%%%%%%%%%%%%%%%%%%%%%%%%%%
\subsection{Forwarding}
\label{sec:forward}

Different versions of the main or child documents
using compilation flags as described in \secref{sec:flags}
can be (permanently) stored in different files
for convenient compilation, viewing and distribution.
To this end, the package defines a command
to pass on compilation to a different file:

%%%%%%%%%%%%%%%%%%%%%%%%%%%%%%%%%%%%%%%%
\DescribeMacro{\childdocforward}
The command |\childdocforward| redirects processing to
another source file:
%
\begin{center}
\begin{tabular}{l}
|\input{childdoc.def}|\\
|\childdocforward[|\textit{main}|]{|\textit{dest}|}|\\
\end{tabular}
\end{center}
%
The argument \textit{dest} is the destination file
(without extension).
It should be the main file or one of the child files.
Note that further \textsf{childdoc} directives
such as |\childdocof| and |\childdocforward|
in the indicated file will be processed in this form.
The optional argument \textit{main}
passes on directly to the main file \textit{main}
while pretending to compile the child \textit{dest}.
This form behaves as if \textit{dest}
issues |\childdocof{|\textit{main}|}| right away,
and no further \textsf{childdoc} directives will be processed.

%%%%%%%%%%%%%%%%%%%%%%%%%%%%%%%%%%%%%%%%
\DescribeMacro{\...prefix}
In the alternative form |\childdocforwardprefix|,
%
\begin{center}
\begin{tabular}{l}
|\input{childdoc.def}|\\
|\childdocforwardprefix[|\textit{main}|]{|\textit{prefix}|}{|\textit{dest}|}|
\end{tabular}
\end{center}
%
the destination file is determined by a pattern
depending on the current file:
To make this work, the current file must be called
`{\textit{prefix}\hspace{0.2em}\textit{suffix}}'
with \textit{prefix} matching precisely the argument.
Processing is then passed on to the file
`{\textit{dest}\hspace{0.2em}\textit{suffix}}'.
Surely, the same effect is achieved by
directly specifying the
argument `{\textit{dest}\hspace{0.2em}\textit{suffix}}'
in the first form.
However, that requires to set up a different file
for each child. With the alternative form of the command
all these files can have exactly the same content
which simplifies setting them up and maintaining them.

For example, the following file |draft.tex|
with a compilation flag |\version| as described in \secref{sec:flags}
compiles the main document as a draft:
%
\begin{center}
\begin{tabular}{l}
|\def\version{draft}|\\
|\input{childdoc.def}|\\
|\childdocforward{|\textit{main}|}|
\end{tabular}
\end{center}
%
Likewise, the following files |final|\textit{nn}|.tex|
compile the final version of the child document
|child|\textit{nn}|.tex|:
%
\begin{center}
\begin{tabular}{l}
|\def\version{final}|\\
|\input{childdoc.def}|\\
|\childdocforwardprefix{final}{child}|
\end{tabular}
\end{center}
%

Note that when several versions of a main file and/or of each child file
are to be generated, it may be convenient to set up a |Makefile| or
shell script to automatise the process.

%%%%%%%%%%%%%%%%%%%%%%%%%%%%%%%%%%%%%%%%%%%%%%%%%%%%%%%%%%%%%%%%%%%%%%%%%%%%%%%%
\subsection{Command Line Processing}
\label{sec:commandline}

The effect of redirection files can also be achieved by invoking
the \LaTeX{} compiler with a more elaborate command line.
Most conveniently this should be done as part
of a shell script or a |Makefile|.

When using \textsf{childdoc} in the main file, the following
command lines effectively perform a redirection
(note that depending on the shell being used,
backslashes may have to be doubled: `|\|' $\to$ `|\\|'):
%
\begin{center}
|... -jobname "|\textit{target}|" |\\|"|[\textit{flags}]%
|\input{childdoc.def}\childdocforward[|\textit{main}|]{|\textit{dest}|}"|
\end{center}
%
Here \textit{target} is the name of the output file,
\textit{main} is the name of the main file
and \textit{dest} is the name of the main or child file to be processed
(all filenames without extensions).
The optional argument \textit{main} can be omitted
if \textit{main} matches \textit{dest}.
Optionally, compilation \textit{flags} can be defined via |\def| commands.
This command line makes the \TeX{} engine believe
it is compiling the file \textit{target}
whose content is specified as the latter parameter.
The provided code then forwards the processing to
\textit{main} or \textit{dest} as described in \secref{sec:forward}.

%%%%%%%%%%%%%%%%%%%%%%%%%%%%%%%%%%%%%%%%%%%%%%%%%%%%%%%%%%%%%%%%%%%%%%%%%%%%%%%%
\subsection{Include by Input}
\label{sec:input}

Including child documents by |\include| has some restrictions by design.
Most notably, the content of a child document always occupies
its own set of pages; pages cannot be shared between child documents.
Usually, this behaviour makes perfect sense
because each child document contain an essential part of the document.
However, in some situations it may be desirable to compose
a document from a collection of parts
without having mandatory page breaks between then.
For this case, the package
provides a mechanism to include parts
by |\input| which can also be processed individually.
However, by construction this mechanism
requires manual handling of the content to be output.

%%%%%%%%%%%%%%%%%%%%%%%%%%%%%%%%%%%%%%%%
\DescribeMacro{\ifchilddocmanual}
The main file should be prepared as usual, see \secref{sec:include}.
However, the document body must make a distinction
between processing of an individual part and of the main document, e.g.:
%
\begin{center}
\begin{tabular}{l}
|\ifchilddocmanual|\\
|\input{\childdocname}|\\
|\||else|\\
\textit{document body with }|\input{|\textit{part}|}|\\
|\||fi|
\end{tabular}
\end{center}
%
The conditional |\ifchilddocmanual| is true whenever
a part to be included by |\input| is being compiled,
and the name of the part is stored in |\childdocname|.

%%%%%%%%%%%%%%%%%%%%%%%%%%%%%%%%%%%%%%%%
\DescribeMacro{\childdocby}
Each part to be included by |\input| should start with:
%
\begin{center}
\begin{tabular}{l}
|\input{childdoc.def}|\\
|\childdocby{|\textit{main}|}|\\
\end{tabular}
\end{center}
%
The directive |\childdocby| is similar to |\childdocof|
described in \secref{sec:include},
but the subsequent selection of content must be done manually.
To that end, both |\ifchilddoc| and |\ifchilddocmanual|
will be true upon processing of a part,
and the name of the part is stored in |\childdocname|.
Note that |\jobname| will be set to the filename of the current part
so that each part receives an individual |.aux| file
that does not interfere with the |.aux| file(s) of the main document.
This behaviour can be altered by the alternative form
|\childdocby[*]{|\textit{main}|}| (with a non-empty optional argument)
which uses the |.aux| file of the main document
by setting |\jobname| to \textit{main}.

%%%%%%%%%%%%%%%%%%%%%%%%%%%%%%%%%%%%%%%%%%%%%%%%%%%%%%%%%%%%%%%%%%%%%%%%%%%%%%%%
\subsection{Driver Development}
\label{sec:driver}

The \textsf{childdoc} mechanism can also be use for the development
of definition files such as \LaTeX{} styles or classes.
This case differs from the above setup with multiple parts
included by |\include| in that no |\includeonly| should be invoked.
This can be achieved by starting the include file
(before |\ProvidesPackage|) with:
%
\begin{center}
\begin{tabular}{l}
|\input{childdoc.def}|\\
|\childdocforward{|\textit{main}|}|\\
\end{tabular}
\end{center}
%
or alternatively with:
%
\begin{center}
\begin{tabular}{l}
|\input{childdoc.def}|\\
|\childdocby{|\textit{main}|}|\\
\end{tabular}
\end{center}
%
Both forms have slightly different effects as described above.
The main file is prepared as usual, see \secref{sec:include}.

%%%%%%%%%%%%%%%%%%%%%%%%%%%%%%%%%%%%%%%%%%%%%%%%%%%%%%%%%%%%%%%%%%%%%%%%%%%%%%%%
\subsection{Legacy Detection}
\label{sec:detection}

The directive |\childdocmain| in the main file can detect
whether the complete document or merely a child is to be compiled
even without using the directive |\childdocof|.
This method is deprecated because it is less robust
and there is no compelling reason to use it;
it is merely provided for backward compatibility
and it may be removed in future versions.

If the detection mechanism is to be used,
it is mandatory to correctly specify
the filename of the main file as the argument of |\childdocmain|:
%
\begin{center}
\begin{tabular}{l}
|\input{childdoc.def}|\\
|\childdocmain{|\textit{main}|}|\\
\end{tabular}
\end{center}
%
If |\jobname| does not match the argument \textit{main} of |\childdocmain|,
it is assumed that |\jobname| points to the child file to be compiled.
When using |\childdocmain| with the main file specified as argument,
it suffices to start a child file
with just |\input{|\textit{main}|}|
without loading of the package and using |\childdocof|.
If instead all processing is done
with the appropriate \textsf{childdoc} directives,
the argument of \textit{main} of |\childdocmain| can be empty.

An alternative version of the command line processing described
in \secref{sec:commandline} using the detection mechanism reads:
%
\begin{center}
|... -jobname "|\textit{target}|" "|[\textit{flags}]%
[|\def\jobname{|\textit{dest}|}|]|\input{|\textit{main}|}"|
\end{center}

%%%%%%%%%%%%%%%%%%%%%%%%%%%%%%%%%%%%%%%%%%%%%%%%%%%%%%%%%%%%%%%%%%%%%%%%%%%%%%%%
\subsection{Manual Code}
\label{sec:manual}

In case one cannot be certain whether the definitions file |childdoc.def|
is installed on the target \TeX{} distribution
and one prefers not to ship it,
it is conceivable to paste a few relevant commands into the sources.

To that end, drop all statements |\input{childdoc.def}|
and perform the replacements as outlined below.
Instead of |\childdocmain{|\textit{main}|}| add the following code
to the top of the main file:
%
\begin{center}
\begin{tabular}{l}
|\||ifdefined\childdocname\endinput\||fi\newif\ifchilddoc|\\
|\edef\childdocname{\scantokens\expandafter{\jobname\noexpand}}|\\
|\def\childdocmain{|\textit{main}|}\||ifx\childdocmain\childdocname\||else|\\
|\childdoctrue\includeonly{\childdocname}\let\jobname\childdocmain\||fi|\\
\end{tabular}
\end{center}
%
Instead of |\childdocof{|\textit{main}|}| just include the main file
at the top of each child file:
%
\begin{center}
|\input{|\textit{main}|}|
\end{center}
%
A simple redirection |\childdocforward{|\textit{dest}|}| is achieved by:
%
\begin{center}
|\def\jobname{|\textit{dest}|}\input{\jobname}|
\end{center}
%
The redirection with prefix
|\childdocforwardprefix[|\textit{prefix}|]{|\textit{dest}|}|
is accomplished by:
%
\begin{center}
\begin{tabular}{l}
|{\edef\jobname{\scantokens\expandafter{\jobname\noexpand}}|\\
|\def\redirectjob |\textit{prefix}|#1~~~{\gdef\jobname{|\textit{dest}|#1}}|\\
|\expandafter\redirectjob\jobname~~~}\input{\jobname}|
\end{tabular}
\end{center}

In an alternative approach,
child documents can be compiled by a specific command line
without additional code or specific definitions:
%
\begin{center}
|... -jobname "|\textit{target}|" "|[\textit{flags}]%
|\includeonly{|\textit{dest}|}\input{|\textit{main}|}"|
\end{center}
%

%%%%%%%%%%%%%%%%%%%%%%%%%%%%%%%%%%%%%%%%%%%%%%%%%%%%%%%%%%%%%%%%%%%%%%%%%%%%%%%%
%%%%%%%%%%%%%%%%%%%%%%%%%%%%%%%%%%%%%%%%%%%%%%%%%%%%%%%%%%%%%%%%%%%%%%%%%%%%%%%%
\section{Information}

%%%%%%%%%%%%%%%%%%%%%%%%%%%%%%%%%%%%%%%%%%%%%%%%%%%%%%%%%%%%%%%%%%%%%%%%%%%%%%%%
\subsection{Copyright}

Copyright \copyright{} 2017--2018 Niklas Beisert

This work may be distributed and/or modified under the
conditions of the \LaTeX{} Project Public License, either version 1.3
of this license or (at your option) any later version.
The latest version of this license is in
  \url{http://www.latex-project.org/lppl.txt}
and version 1.3 or later is part of all distributions of \LaTeX{}
version 2005/12/01 or later.

This work has the LPPL maintenance status `maintained'.

The Current Maintainer of this work is Niklas Beisert.

This work consists of the files |README.txt|, |childdoc.ins| and |childdoc.dtx|
as well as the derived files |childdoc.def|, |cdocsamp.tex|
with |cdocsch1.tex|, |cdocsch2.tex|, |cdocspt3.tex|, |cdocspt4.tex|,
|cdocsdrf.tex|, |cdocsfn1.tex|, |cdocsfn2.tex|
as well as |childdoc.pdf|.

%%%%%%%%%%%%%%%%%%%%%%%%%%%%%%%%%%%%%%%%%%%%%%%%%%%%%%%%%%%%%%%%%%%%%%%%%%%%%%%%
\subsection{Files and Installation}

The package consists of the files:
%
\begin{center}
\begin{tabular}{ll}
    |README.txt|   & readme file \\
    |childdoc.ins| & installation file \\
    |childdoc.dtx| & source file \\
    |childdoc.def| & definition file \\
    |cdocsamp.tex| & sample main file \\
    |cdocsch1.tex| & sample include file \\
    |cdocsch2.tex| & sample include file \\
    |cdocspt3.tex| & sample part file \\
    |cdocspt4.tex| & sample part file \\
    |cdocsdrf.tex| & sample redirection file \\
    |cdocsfn1.tex| & sample redirection file \\
    |cdocsfn2.tex| & sample redirection file \\
    |childdoc.pdf| & manual
\end{tabular}
\end{center}
%
The distribution consists of the files
|README.txt|, |childdoc.ins| and |childdoc.dtx|.
%
\begin{itemize}
\item
Run (pdf)\LaTeX{} on |childdoc.dtx|
to compile the manual |childdoc.pdf| (this file).
\item
Run \LaTeX{} on |childdoc.ins| to create the definitions file |childdoc.def|
and the sample |cdocsamp.tex| with include files
|cdocsch1.tex|, |cdocsch2.tex|, |cdocspt3.tex|, |cdocspt4.tex|,
|cdocsdrf.tex|, |cdocsfn1.tex|, |cdocsfn2.tex|.
Then copy the file |childdoc.def| to an appropriate directory of your \LaTeX{}
distribution, e.g.\ \textit{texmf-root}|/tex/latex/childdoc|.
\end{itemize}

%%%%%%%%%%%%%%%%%%%%%%%%%%%%%%%%%%%%%%%%%%%%%%%%%%%%%%%%%%%%%%%%%%%%%%%%%%%%%%%%
\subsection{Related CTAN Packages}

There are several other packages which offer a similar functionality:
%
\begin{itemize}
\item
The packages
\href{http://ctan.org/pkg/docmute}{\textsf{docmute}},
\href{http://ctan.org/pkg/includex}{\textsf{includex}} and
\href{http://ctan.org/pkg/standalone}{\textsf{standalone}}
provide commands to include only the document body of
a child file thus allowing both files to be compiled individually.
\item
The packages \href{http://ctan.org/pkg/subdocs}{\textsf{subdocs}}
and \href{http://ctan.org/pkg/subfiles}{\textsf{subfiles}}
provide structures in which the main and child documents can be
encapsulated and allowing them to be compiled individually.
The inclusion mechanism is different from the conventional |\include|.
\item
The package \href{http://ctan.org/pkg/combine}{\textsf{combine}}
is an elaborate solution to combine several documents into one.
\end{itemize}
%
See also the CTAN topic \href{http://ctan.org/topic/subdocs}{\textsf{subdocs}}
for further related packages.
The present package differs from the above solutions in that
a document structure constructed with the conventional |\include| mechanism
just needs two extra commands at the top of every file
such that all constituent files can be compiled individually.

%%%%%%%%%%%%%%%%%%%%%%%%%%%%%%%%%%%%%%%%%%%%%%%%%%%%%%%%%%%%%%%%%%%%%%%%%%%%%%%%
%\subsection{Feature Suggestions}
%
%The following is a list of features which may be useful for future
%versions of this package:
%%
%\begin{itemize}
%\item
%\ldots
%\end{itemize}

%%%%%%%%%%%%%%%%%%%%%%%%%%%%%%%%%%%%%%%%%%%%%%%%%%%%%%%%%%%%%%%%%%%%%%%%%%%%%%%%
\subsection{Revision History}

%%%%%%%%%%%%%%%%%%%%%%%%%%%%%%%%%%%%%%%%
\paragraph{v2.0:} 2018/12/30

\begin{itemize}
\item
immediate forward processing
\item
added |\childdocby| mechanism
\item
manual restructured
\end{itemize}

%%%%%%%%%%%%%%%%%%%%%%%%%%%%%%%%%%%%%%%%
\paragraph{v1.6:} 2018/01/17

\begin{itemize}
\item
application for development of include files
\item
corrections to manual
\end{itemize}

%%%%%%%%%%%%%%%%%%%%%%%%%%%%%%%%%%%%%%%%
\paragraph{v1.5:} 2017/05/21

\begin{itemize}
\item
more complete structuring introduced
\item
|\childdocof| introduced
\item
|\childdoc| renamed to |\childdocmain|
\item
|\childredirect| renamed to |\childdocforward| and |\childdocforwardprefix|
and functionality expanded
\end{itemize}

%%%%%%%%%%%%%%%%%%%%%%%%%%%%%%%%%%%%%%%%
\paragraph{v1.0:} 2017/04/27

\begin{itemize}
\item
manual and install package
\item
first version published on CTAN
\end{itemize}

%%%%%%%%%%%%%%%%%%%%%%%%%%%%%%%%%%%%%%%%
\paragraph{v0.6:} 2017/04/26

\begin{itemize}
\item
redirection mechanism added
\end{itemize}

%%%%%%%%%%%%%%%%%%%%%%%%%%%%%%%%%%%%%%%%
\paragraph{v0.5:} 2017/04/26

\begin{itemize}
\item
functionality in definition file
\end{itemize}


%%%%%%%%%%%%%%%%%%%%%%%%%%%%%%%%%%%%%%%%%%%%%%%%%%%%%%%%%%%%%%%%%%%%%%%%%%%%%%%%
%%%%%%%%%%%%%%%%%%%%%%%%%%%%%%%%%%%%%%%%%%%%%%%%%%%%%%%%%%%%%%%%%%%%%%%%%%%%%%%%
%%%%%%%%%%%%%%%%%%%%%%%%%%%%%%%%%%%%%%%%%%%%%%%%%%%%%%%%%%%%%%%%%%%%%%%%%%%%%%%%
\appendix

\settowidth\MacroIndent{\rmfamily\scriptsize 000\ }

 \DocInput{childdoc.dtx}

\end{document}
%</driver>
% \fi
%
% %%%%%%%%%%%%%%%%%%%%%%%%%%%%%%%%%%%%%%%%%%%%%%%%%%%%%%%%%%%%%%%%%%%%%%%%%%%%%%
% %%%%%%%%%%%%%%%%%%%%%%%%%%%%%%%%%%%%%%%%%%%%%%%%%%%%%%%%%%%%%%%%%%%%%%%%%%%%%%
% \section{Sample}
%\iffalse
%<*samplemain>
%\fi
%
% The following presents a sample document
% with two chapters, two parts, a title page,
% a compile flag as well as three forwarding files to set the flag.
% It consists of eight |.tex| files:
% \begin{center}
% \begin{tabular}{ll}
% |cdocsamp.tex|&main file\\
% |cdocsch1.tex|&include file for chapter 1\\
% |cdocsch2.tex|&include file for chapter 2\\
% |cdocspt3.tex|&include file for part 3\\
% |cdocspt4.tex|&include file for part 4\\
% |cdocsdrf.tex|&forwarding file for main file in draft mode\\
% |cdocsfi1.tex|&forwarding file for final version of chapter 1\\
% |cdocsfi2.tex|&forwarding file for final version of chapter 2\\
% \end{tabular}
% \end{center}
% Each of the eight files can be compiled directly by the \LaTeX{} compiler.
%
% %%%%%%%%%%%%%%%%%%%%%%%%%%%%%%%%%%%%%%
% \paragraph{Main File.}
%
% The main file is called |cdocsamp.tex|.
%
% Load the \textsf{childdoc} definitions and
% declare the filename for the main document:
%    \begin{macrocode}
\input{childdoc.def}
\childdocmain{}
%    \end{macrocode}

% Optional override for |\version| flag:
%    \begin{macrocode}
%%\ifchilddoc\else\providecommand{\version}{draft}\fi
%    \end{macrocode}

% Define the default values for the |\version| flag
% (|final| for the main file and |draft| for childs):
%    \begin{macrocode}
\ifchilddoc
\providecommand{\version}{draft}
\else
\providecommand{\version}{final}
\fi
%    \end{macrocode}

% Load the standard document class:
%    \begin{macrocode}
\documentclass[12pt]{article}
%    \end{macrocode}

% Start the document body:
%    \begin{macrocode}
\begin{document}
%    \end{macrocode}

% Declare a title page.
% Print title, part of document being processed and version flag:
%    \begin{macrocode}
\addtocounter{page}{-1}
\begin{center}
{\LARGE\bfseries{}childdoc example\par}
\vspace{1cm}
\ifchilddoc
\ifchilddocmanual part\else chapter\fi:
`\childdocname' of `\childdocjob'\par
\else
main document: `\childdocjob'\par
\fi
version: \version\par
\end{center}
\newpage
%    \end{macrocode}

% Manually include selected file,
% otherwise process as usual:
%    \begin{macrocode}
\ifchilddocmanual
\section*{part `\childdocname'}
\input{\childdocname}
\else
%    \end{macrocode}

% Include the two chapters:
%    \begin{macrocode}
\include{cdocsch1}
\include{cdocsch2}
%    \end{macrocode}

% Include the two parts unless only chapters should be displayed:
%    \begin{macrocode}
\ifchilddoc\else
\section{part three}
\input{cdocspt3}
\section{part four}
\input{cdocspt4}
\fi
%    \end{macrocode}

% Process as usual until here:
%    \begin{macrocode}
\fi
%    \end{macrocode}

% End of document body:
%    \begin{macrocode}
\end{document}
%    \end{macrocode}
%\iffalse
%</samplemain>
%\fi
%
% %%%%%%%%%%%%%%%%%%%%%%%%%%%%%%%%%%%%%%
% \paragraph{Chapter Include Files.}
%
% The include files are called |cdocsch1.tex| and |cdocsch2.tex|.
%
%\iffalse
%<*samplechap1|samplechap2>
%\fi

% Optional override for |\version| flag:
%    \begin{macrocode}
%%\providecommand{\version}{final}
%    \end{macrocode}

% Include the main document:
%    \begin{macrocode}
\input{childdoc.def}
\childdocof{cdocsamp}
%    \end{macrocode}

%\iffalse
%</samplechap1|samplechap2>
%\fi
%
%\iffalse
%<*samplechap1>
%\fi
% Some text for chapter 1:
%    \begin{macrocode}
\section{one}
some text in chapter one
%    \end{macrocode}

%\iffalse
%</samplechap1>
%\fi
% Some text for chapter 2:
%\iffalse
%<*samplechap2>
%\fi
%    \begin{macrocode}
\section{two}
more text in chapter two
%    \end{macrocode}

%\iffalse
%</samplechap2>
%\fi
%
% %%%%%%%%%%%%%%%%%%%%%%%%%%%%%%%%%%%%%%
% \paragraph{Part Include Files.}
%
% The include files are called |cdocspt3.tex| and |cdocspt4.tex|.
%
%\iffalse
%<*samplepart3|samplepart4>
%\fi

% Optional override for |\version| flag:
%    \begin{macrocode}
%%\providecommand{\version}{final}
%    \end{macrocode}

% Include the main document:
%    \begin{macrocode}
\input{childdoc.def}
\childdocby{cdocsamp}
%    \end{macrocode}

%\iffalse
%</samplepart3|samplepart4>
%\fi
%
%\iffalse
%<*samplepart3>
%\fi
% Some text for part 3:
%    \begin{macrocode}
some text in part three
%    \end{macrocode}

%\iffalse
%</samplepart3>
%\fi
% Some text for part 4:
%\iffalse
%<*samplepart4>
%\fi
%    \begin{macrocode}
more text in part four
%    \end{macrocode}

%\iffalse
%</samplepart4>
%\fi
%
% %%%%%%%%%%%%%%%%%%%%%%%%%%%%%%%%%%%%%%
% \paragraph{Forwarding for a Complete Draft.}
%
% The following forwarding file |cdocsdrf.tex|
% compiles the main document in draft mode:
%\iffalse
%<*sampledraft>
%\fi
%    \begin{macrocode}
\def\version{draft}
\input{childdoc.def}
\childdocforward{cdocsamp}
%    \end{macrocode}

%\iffalse
%</sampledraft>
%\fi
%
% %%%%%%%%%%%%%%%%%%%%%%%%%%%%%%%%%%%%%%
% \paragraph{Forwarding for Final Version of the Chapters.}
%
% The following forwarding files |cdocsfn1.tex| and |cdocsfn2.tex|
% (with identical content)
% compile the final versions of the child documents
% |cdocsch1.tex| and |cdocsch2.tex|, respectively:
%\iffalse
%<*samplefinal>
%\fi
%    \begin{macrocode}
\def\version{final}
\input{childdoc.def}
\childdocforwardprefix[cdocsamp]{cdocsfn}{cdocsch}
%    \end{macrocode}

%\iffalse
%</samplefinal>
%\fi
%
% %%%%%%%%%%%%%%%%%%%%%%%%%%%%%%%%%%%%%%
% \paragraph{Command Line Processing.}
%
% The following three command lines generate the output files
% |cdocscld|, |cdocscl1| and |cdocscl2|
% which should be identical to
% |cdocsdrf|, |cdocsch1| and |cdocsfn2|, respectively:
% \begin{center}
% \begin{tabular}{l}
% |latex -jobname cdocscld \|\\
% |  "\def\version{draft}\input{childdoc.def}\childdocforward{cdocsamp}"|\\
% |latex -jobname cdocscl1 \|\\
% |  "\input{childdoc.def}\childdocforward[cdocsamp]{cdocsch1}"|\\
% |latex -jobname cdocscl2 \|\\
% |  "\def\version{final}\input{childdoc.def}\childdocforward{cdocsch2}"|
% \end{tabular}
% \end{center}
% Note that the trailing backslash on each first line
% merely continues the input to the second line
% (for convenient cut ant paste).
% Furthermore, the command |latex| can be replaced by any
% of its alternative versions such as |pdflatex|.
%
% %%%%%%%%%%%%%%%%%%%%%%%%%%%%%%%%%%%%%%%%%%%%%%%%%%%%%%%%%%%%%%%%%%%%%%%%%%%%%%
% %%%%%%%%%%%%%%%%%%%%%%%%%%%%%%%%%%%%%%%%%%%%%%%%%%%%%%%%%%%%%%%%%%%%%%%%%%%%%%
% \section{Implementation}
%\iffalse
%<*package>
%\fi
%
% This section describes the definitions file |childdoc.def|.

% The definitions cannot be loaded using |\usepackage| or |\RequirePackage|
% which has a mechanism to prevent loading a style file more than once.
% When loading the definitions by means of |\input|
% multiple instances have to be prevented manually:
%\iffalse
%This code needs to be before the `\ProvidesFile' directive
%which is defined at the beginning of this file.
%Therefore it is also placed there and commented out here.
%</package>
%<*discard>
%\fi
%    \begin{macrocode}
\ifdefined\childdocmain\endinput\fi
%    \end{macrocode}
%\iffalse
%</discard>
%<*package>
%\fi
%
% \macro{\ifchilddoc}
% \macro{\ifchilddocmanual}
% The conditional |\ifchilddoc| tells whether a
% child (true) or main (false) document is being compiled.
% The conditional |\ifchilddocmanual| tells whether
% the |\includeonly| mechanism is used (false) or
% the selection of child files must be performed manually (true).
% The definitions initialise to false:
%    \begin{macrocode}
\newif\ifchilddoc
\newif\ifchilddocmanual
%    \end{macrocode}

% \macro{\childdocname}
% \macro{\childdocjob}
% The macro |\childdocname| stores the name of the main document
% to be compiled. The macro |\childdocjob| stores the name of
% the document on which the \LaTeX{} compiler was originally invoked.
% The content of |\jobname| cannot be compared
% to filenames specified in the source due to different catcodes.
% The following code rescans |\jobname|, stores the result
% in |\childdocname| and saves a copy in |\childdocjob|:
%    \begin{macrocode}
\edef\childdocname{\scantokens\expandafter{\jobname\noexpand}}
\let\childdocjob\childdocname
%    \end{macrocode}

% \macro{\childdocdisable}
% The macro |\childdocdisable| prevents the main file
% from being processed more than once.
% At this stage, the main document command |\childdocmain|
% is assumed to be called once again where it should do nothing.
% Any subsequent call to it should prevent
% a secondary processing of the main document
% It overwrites the forwarding commands
% |\childdocof| and |\childdocforward|
% with empty macros to prevent further inclusions of the main document:
%    \begin{macrocode}
\newcommand{\childdocdisable}
{
  \renewcommand{\childdocmain}[1]{\renewcommand{\childdocmain}[1]{\endinput}}
  \renewcommand{\childdocof}[1]{}
  \renewcommand{\childdocby}[2][]{}
  \renewcommand{\childdocforward}[2][]{}
  \renewcommand{\childdocdisable}{}
}
%    \end{macrocode}

% \macro{\childdocmain}
% The macro |\childdocmain| is to be called at the top of the main file
% with nothing or the main filename (without extension) as argument.
% First, it breaks loops.
% If the argument is not empty and does not match |\childdocname|
% (which is set by the first inclusion of |childdoc.def|),
% |\ifchilddoc| is set to true, |\includeonly| is applied to the child file
% and |\jobname| is set to the main file
% (for proper handling of |.aux| files):
%    \begin{macrocode}
\newcommand{\childdocmain}[1]
{
  \childdocdisable\childdocmain{}
  \if?#1?\else
    \begingroup
      \def\childdoctmp{#1}
      \ifx\childdoctmp\childdocname
        \def\childdoctmp{}
      \else
        \def\childdoctmp
        {
          \childdoctrue
          \includeonly{\childdocname}
          \def\childdocjob{#1}
          \def\jobname{#1}
        }
      \fi
      \expandafter
    \endgroup
    \childdoctmp
  \fi
}
%    \end{macrocode}

% \macro{\childdocof}
% The command |\childdocof| redirects
% compilation to the main file |#1|.
%    \begin{macrocode}
\newcommand{\childdocof}[1]
{
  \childdocdisable
  \childdoctrue
  \includeonly{\childdocname}
  \def\jobname{#1}
  \def\childdocjob{#1}
  \input{#1}
}
%    \end{macrocode}

% \macro{\childdocby}
% The command |\childdocby| ....
%    \begin{macrocode}
\newcommand{\childdocby}[2][]
{
  \childdocdisable
  \childdoctrue
  \childdocmanualtrue
  \if?#1?\else
    \def\jobname{#2}
  \fi
  \def\childdocjob{#2}
  \input{#2}
  \endinput
}
%    \end{macrocode}

% \macro{\childdocforward}
% The command |\childdocforward| redirects
% compilation to the main file or
% (if the optional argument is given) a child file.
% Parameters are set as if the main file
% or a child file starting with |\childdocof| was compiled.
% Then compilation is handed over to the main file:
%    \begin{macrocode}
\newcommand{\childdocforward}[2][]
{
  \begingroup
    \if?#1?
      \def\childdoctmp
      {
        \def\childdocname{#2}
        \def\childdocjob{#2}
        \def\jobname{#2}
        \input{#2}
        \endinput
      }
    \else
      \def\childdoctmp
      {
        \childdocdisable
        \def\childdocname{#2}
        \childdoctrue
        \includeonly{#2}
        \def\childdocjob{#1}
        \def\jobname{#1}
        \input{#1}
        \endinput
      }
    \fi
    \expandafter
  \endgroup
  \childdoctmp
}
%    \end{macrocode}

% \macro{\childdocforwardprefix}
% The command |\childdocforwardprefix| redirects
% compilation to the main or a child file by means of a pattern.
% The prefix |#1| in the current filename is replaced by |#2|
% and the suffix of the current filename is kept
% (it is assumed that the filename does not contain the substring `|~~~|'
% which is used as a delimiter).
% Compilation is handed over to the new file by |\childdocforward|:
%    \begin{macrocode}
\newcommand{\childdocforwardprefix}[3][]
{
  \begingroup
    \def\childdocextract #2##1~~~{\def\childdoctmp{\childdocforward[#1]{#3##1}}}
    \expandafter\childdocextract\childdocname~~~
    \expandafter
  \endgroup
  \childdoctmp
}
%    \end{macrocode}

% \macro{\childdoc}
% The deprecated macro |\childdoc| is a legacy version of |\childdocmain|:
%    \begin{macrocode}
\newcommand{\childdoc}{\childdocmain}
%    \end{macrocode}

% \macro{\childdocredirect}
% The deprecated macro |\childdocredirect| is a legacy version
% of |\childdocforward| and |\childdocforwardprefix|:
%    \begin{macrocode}
\newcommand{\childdocredirect}[2][]
{
  \begingroup
    \if?#1?
      \def\childdoctmp{\childdocforward{#2}}
    \else
      \def\childdoctmp{\childdocforwardprefix{#1}{#2}}
    \fi
    \expandafter
  \endgroup
  \childdoctmp
}
%    \end{macrocode}

%\iffalse
%</package>
%\fi
%
\endinput
|\\
|\childdocforward{|\textit{main}|}|
\end{tabular}
\end{center}
%
Likewise, the following files |final|\textit{nn}|.tex|
compile the final version of the child document
|child|\textit{nn}|.tex|:
%
\begin{center}
\begin{tabular}{l}
|\def\version{final}|\\
|% \iffalse
%
% childdoc.dtx Copyright (C) 2017-2018 Niklas Beisert
%
% This work may be distributed and/or modified under the
% conditions of the LaTeX Project Public License, either version 1.3
% of this license or (at your option) any later version.
% The latest version of this license is in
%   http://www.latex-project.org/lppl.txt
% and version 1.3 or later is part of all distributions of LaTeX
% version 2005/12/01 or later.
%
% This work has the LPPL maintenance status `maintained'.
%
% The Current Maintainer of this work is Niklas Beisert.
%
% This work consists of the files childdoc.dtx and childdoc.ins
% and the derived files childdoc.def and cdocsamp.tex with
% cdocsch1.tex, cdocsch2.tex, cdocsdrf.tex, cdocsfn1.tex, cdocsfn2.tex.
%
%<package>\ifdefined\childdocmain\endinput\fi
%<package>\ProvidesFile{childdoc.def}[2018/12/30 v2.0 child document driver]
%<samplemain>\ProvidesFile{cdocsamp.tex}[2018/12/30 v2.0 sample for childdoc]
%<*driver>
%\ProvidesFile{childdoc.drv}[2018/12/30 v2.0 childdoc reference manual file]
\PassOptionsToClass{10pt,a4paper}{article}
\documentclass{ltxdoc}

\usepackage[margin=35mm]{geometry}
\usepackage{hyperref}
\usepackage{hyperxmp}
\usepackage[usenames]{color}

\hypersetup{colorlinks=true}
\hypersetup{pdfstartview=FitH}
\hypersetup{pdfpagemode=UseNone}
\hypersetup{pdfsource={}}
\hypersetup{pdflang={en-UK}}
\hypersetup{pdfcopyright={Copyright 2017-2018 Niklas Beisert.
  This work may be distributed and/or modified under the
  conditions of the LaTeX Project Public License, either version 1.3
  of this license or (at your option) any later version.}}
\hypersetup{pdflicenseurl={http://www.latex-project.org/lppl.txt}}
\hypersetup{pdfcontactaddress={ETH Zurich, ITP, HIT K,
  Wolfgang-Pauli-Strasse 27}}
\hypersetup{pdfcontactpostcode={8093}}
\hypersetup{pdfcontactcity={Zurich}}
\hypersetup{pdfcontactcountry={Switzerland}}
\hypersetup{pdfcontactemail={nbeisert@itp.phys.ethz.ch}}
\hypersetup{pdfcontacturl={http://people.phys.ethz.ch/\xmptilde nbeisert/}}

\newcommand{\secref}[1]{\hyperref[#1]{section \ref*{#1}}}

\parskip1ex
\parindent0pt
\let\olditemize\itemize
\def\itemize{\olditemize\parskip0pt}

\begin{document}

\title{The \textsf{childdoc} Package}
\hypersetup{pdftitle={The childdoc Package}}
\author{Niklas Beisert\\[2ex]
  Institut f\"ur Theoretische Physik\\
  Eidgen\"ossische Technische Hochschule Z\"urich\\
  Wolfgang-Pauli-Strasse 27, 8093 Z\"urich, Switzerland\\[1ex]
  \href{mailto:nbeisert@itp.phys.ethz.ch}
  {\texttt{nbeisert@itp.phys.ethz.ch}}}
\hypersetup{pdfauthor={Niklas Beisert}}
\hypersetup{pdfsubject={Manual for the LaTeX2e Package childdoc}}
\date{30 December 2018, \textsf{v2.0}}
\maketitle

\begin{abstract}\noindent
\textsf{childdoc} is a \LaTeXe{} package
that enables the direct compilation
of document sections included by |\include|
to individual files.
\end{abstract}

\begingroup
\parskip0ex
\tableofcontents
\endgroup

%%%%%%%%%%%%%%%%%%%%%%%%%%%%%%%%%%%%%%%%%%%%%%%%%%%%%%%%%%%%%%%%%%%%%%%%%%%%%%%%
%%%%%%%%%%%%%%%%%%%%%%%%%%%%%%%%%%%%%%%%%%%%%%%%%%%%%%%%%%%%%%%%%%%%%%%%%%%%%%%%
\section{Introduction}

\LaTeX{} provides a mechanism to structure a large document (such as a book)
into a main file and several child files (containing the chapters)
using the |\include| command.
This mechanism is beneficial for documents
which span hundreds of pages in order to
make the source file(s) more manageable.
Moreover, compilation can be restricted to
selected child files by means of the |\includeonly| command.
The latter feature can be used to reduce the compilation time while editing
(this was significantly more useful in the earlier days of \LaTeX{})
or to generate a smaller document which is easier to navigate.
Another application of |\includeonly| is to generate
documents consisting of selected parts of the complete document.

However, there are a few drawbacks of the plain |\include| mechanism:
\begin{itemize}
\item
The child files cannot be compiled on their own,
they can only be compiled via the main file.
A naive editing environment
(such as a text editor with an option
to have the current file processed by \LaTeX)
may require one to switch to the main file before compiling;
attempting to compile the child file produces errors.
\item
The main file must be modified (each time)
to adjust the |\includeonly| command
to the present needs. This easily leaves the main file in a messy state.
\item
The generated document will always carry the filename
of the main document. This is inconvenient if
several child files are to be compiled and
to be kept for distribution.
\end{itemize}

The present package provides a simple interface
to make child files individually compilable by \LaTeX{}.
Compiling a child file then has the same effect as compiling
the main file with an |\includeonly| command
to select the appropriate child.
Moreover the generated document will carry the name of the child
rather than the main file.
This resolves all three above issues.

This feature is meant to make the editing of books,
thesis documents and lecture notes somewhat more convenient.
However, the package can also be used efficiently for
composing a series of documents (such as exercise sheets)
which are typically distributed individually.
It then assists the author in generating the individual documents
(potentially in different versions)
as well as a document containing the collected series.
Another application is in developing style files
or other kinds of included material
where compilation of the style file could redirect
to a sample or test file.

%%%%%%%%%%%%%%%%%%%%%%%%%%%%%%%%%%%%%%%%%%%%%%%%%%%%%%%%%%%%%%%%%%%%%%%%%%%%%%%%
%%%%%%%%%%%%%%%%%%%%%%%%%%%%%%%%%%%%%%%%%%%%%%%%%%%%%%%%%%%%%%%%%%%%%%%%%%%%%%%%
\section{Usage}

First of all, the package \textsf{childdoc} is \emph{not} a standard
\LaTeXe{} |.sty| style file! Therefore it needs to be invoked in
a non-standard way.

%%%%%%%%%%%%%%%%%%%%%%%%%%%%%%%%%%%%%%%%%%%%%%%%%%%%%%%%%%%%%%%%%%%%%%%%%%%%%%%%
\subsection{Included Files}
\label{sec:include}

%%%%%%%%%%%%%%%%%%%%%%%%%%%%%%%%%%%%%%%%
\DescribeMacro{\childdocmain}
To use the package, add the commands
\begin{center}
\begin{tabular}{l}
|\input{childdoc.def}|\\
|\childdocmain{}|\\
\end{tabular}
\end{center}
at the very top of the main \LaTeX{} file,
in particular \emph{before} the |\documentclass| statement!
The argument of |\childdocmain| should be left empty
(but it must be present).

%%%%%%%%%%%%%%%%%%%%%%%%%%%%%%%%%%%%%%%%
\DescribeMacro{\childdocof}
Furthermore, add the commands
\begin{center}
\begin{tabular}{l}
|\input{childdoc.def}|\\
|\childdocof{|\textit{main}|}|\\
\end{tabular}
\end{center}
at the top of every child file \textit{child}
which is included by |\include{|\textit{child}|}|
from within the main file
(or at least for those files to be compiled individually).
The argument \textit{main} must be the filename of the main file.

There are a couple of
considerations in setting up the main and child documents:

%%%%%%%%%%%%%%%%%%%%%%%%%%%%%%%%%%%%%%%%
\paragraph{Restrictions.}

Please note the following restrictions:
\begin{itemize}
\item
|\childdocmain| must be called with one argument \textit{main}
to ensure compatibility with earlier version of the package.
It must either be empty (|\childdocmain{}|)
or precisely match the filename of the main file in which it is specified.
See \secref{sec:detection} for further information.
\item
The filename \textit{main} must be specified without the |.tex| extension.
\item
The filename \textit{main} is case sensitive
(even in case-insensitive file systems)
due to internal string comparison.
\item
The argument \textit{main} should be fully expanded, it cannot be a macro.
\item
Subdirectories and special characters should be avoided in filenames.
\item
The command |\childdocmain{|\textit{main}|}| must be followed by a whitespace.
It should not be followed immediately by another command
or by a comment mark `|%|'.
This is because the \TeX{} parser reads the token immediately following
the argument of |\childdocmain| and puts it
at the beginning of every child section;
however, a white\-space is ignored.
\end{itemize}

%%%%%%%%%%%%%%%%%%%%%%%%%%%%%%%%%%%%%%%%
\paragraph{Content of Main File.}

It is advisable to place all content in the child files included by |\include|.
Any output contained in the main file will appear in all child documents
unless suppressed manually;
it cannot be suppressed automatically by the |\includeonly| directive
and thus should normally be avoided.
A method to include some content in the main file
by means of conditional processing is described in \secref{sec:conditional}.

%%%%%%%%%%%%%%%%%%%%%%%%%%%%%%%%%%%%%%%%
\paragraph{Page Numbering.}

When only a part of the document is compiled,
the appropriate numbering of pages
(as well as other status parameters)
is determined from the |.aux| files.
The latter contain information from previous passes.
However this information needs to propagate through
all intermediate child documents.
Therefore the page numbering in child documents may well
be inconsistent until the complete document is compiled at least once.

A useful (if unconventional) way to always ensure a consistent
page numbering is to restart the numbering in each child document
and denote the pages by `\textit{child}|.|\textit{page}'
where \textit{child} represents the chapter/section number of the child file.
This can be achieved by the command
|\numberwithin{page}{|\textit{child}|}|
of the \textsf{amsmath} package
where \textit{child} can be |chapter| or |section|
depending on the chosen structuring.
Alternatively, one can modify the macro |\thepage| appropriately
and reset the counter |page| at the start of each child file.

%%%%%%%%%%%%%%%%%%%%%%%%%%%%%%%%%%%%%%%%%%%%%%%%%%%%%%%%%%%%%%%%%%%%%%%%%%%%%%%%
\subsection{Conditional Processing}
\label{sec:conditional}

The package provides a mechanism to compile different versions
of a document. To customise the versions further some conditional processing
can come in handy to distinguish which version is being compiled.
The package provides two macros to describe the compilation context:

%%%%%%%%%%%%%%%%%%%%%%%%%%%%%%%%%%%%%%%%
\DescribeMacro{\ifchilddoc}
The conditional |\ifchilddoc| distinguishes between the compilation of
child documents and the main document:
%
\begin{center}
|\ifchilddoc |\textit{child-code}| |[|\||else |\textit{main-code}]| \||fi|
\end{center}

%%%%%%%%%%%%%%%%%%%%%%%%%%%%%%%%%%%%%%%%
\DescribeMacro{\childdocname}
\DescribeMacro{\childdocjob}
The macro |\childdocname| contains the filename (without extension)
of the main or child file being processed.
Note that |\childdocjob| will always contain the name of the main file.

%%%%%%%%%%%%%%%%%%%%%%%%%%%%%%%%%%%%%%%%
\paragraph{Title Page.}

Conditional processing can be used to include a title or banner page
in the main document when proper precautions are taken.
Importantly, the code in the main file should ensure that the page counter
(as well as other status parameters which are stored in the |.aux| files)
takes the same value after the conditional processing.
Otherwise the page numbers may take divergent values
depending on which part is compiled.

For example, a title page could be declared by:
%
\begin{center}
\begin{tabular}{l}
|\ifchilddoc\||else|\\
|\addtocounter{page}{-1}|\\
\textit{code for title page}\\
|\newpage|\\
|\||fi|
\end{tabular}
\end{center}
%
A banner page for the child documents can be generated by:
%
\begin{center}
\begin{tabular}{l}
|\ifchilddoc|\\
|\addtocounter{page}{-1}|\\
\textit{code for banner page}\\
|\newpage|\\
|\||fi|
\end{tabular}
\end{center}
%
Here one could write a message such as:
\begin{center}
|This is the part \childdocname{} of \childdocjob{}.|
\end{center}

%%%%%%%%%%%%%%%%%%%%%%%%%%%%%%%%%%%%%%%%%%%%%%%%%%%%%%%%%%%%%%%%%%%%%%%%%%%%%%%%
\subsection{Flags}
\label{sec:flags}

The package makes it easy to generate different versions
of the main or child documents.
To this end compilation flags can be defined
and assigned different default values.
They will be particularly useful in conjunction
with the forwarding mechanism described in \secref{sec:forward}.

For example, it may be useful to have a flag |\version|
which can be set to |draft| or |final|.
The document source will contain some conditional code
depending on the value of |\version|.
Suppose further, the flag should default to |final| for the main file
and to |draft| for child files
which is a natural assignment for editing the document.
This is achieved by placing the following code
in the preamble of the main document
(below the |\childdocmain| directive):
%
\begin{center}
\begin{tabular}{l}
|\ifchilddoc|\\
|\providecommand{\version}{draft}|\\
|\||else|\\
|\providecommand{\version}{final}|\\
|\||fi|
\end{tabular}
\end{center}
%
The definition by |\providecommand| makes sure
that previous definitions are not overwritten.
Further statements |\providecommand{\version}{...}|
can thus be added before the above code to override it.

For the main file, one might add a line
(between |\childdocmain| and the above block)
%
\begin{center}
|%\ifchilddoc\||else\providecommand{\version}{draft}\||fi|
\end{center}
%
which can be uncommented to produce a draft version.
Likewise one can add a line to the very top of a child file
(above the |\childdocof{|\textit{main}|}| directive)
%
\begin{center}
|%\providecommand{\version}{final}|
\end{center}
%
which can be uncommented to produce the final version of this child document.

%%%%%%%%%%%%%%%%%%%%%%%%%%%%%%%%%%%%%%%%%%%%%%%%%%%%%%%%%%%%%%%%%%%%%%%%%%%%%%%%
\subsection{Forwarding}
\label{sec:forward}

Different versions of the main or child documents
using compilation flags as described in \secref{sec:flags}
can be (permanently) stored in different files
for convenient compilation, viewing and distribution.
To this end, the package defines a command
to pass on compilation to a different file:

%%%%%%%%%%%%%%%%%%%%%%%%%%%%%%%%%%%%%%%%
\DescribeMacro{\childdocforward}
The command |\childdocforward| redirects processing to
another source file:
%
\begin{center}
\begin{tabular}{l}
|\input{childdoc.def}|\\
|\childdocforward[|\textit{main}|]{|\textit{dest}|}|\\
\end{tabular}
\end{center}
%
The argument \textit{dest} is the destination file
(without extension).
It should be the main file or one of the child files.
Note that further \textsf{childdoc} directives
such as |\childdocof| and |\childdocforward|
in the indicated file will be processed in this form.
The optional argument \textit{main}
passes on directly to the main file \textit{main}
while pretending to compile the child \textit{dest}.
This form behaves as if \textit{dest}
issues |\childdocof{|\textit{main}|}| right away,
and no further \textsf{childdoc} directives will be processed.

%%%%%%%%%%%%%%%%%%%%%%%%%%%%%%%%%%%%%%%%
\DescribeMacro{\...prefix}
In the alternative form |\childdocforwardprefix|,
%
\begin{center}
\begin{tabular}{l}
|\input{childdoc.def}|\\
|\childdocforwardprefix[|\textit{main}|]{|\textit{prefix}|}{|\textit{dest}|}|
\end{tabular}
\end{center}
%
the destination file is determined by a pattern
depending on the current file:
To make this work, the current file must be called
`{\textit{prefix}\hspace{0.2em}\textit{suffix}}'
with \textit{prefix} matching precisely the argument.
Processing is then passed on to the file
`{\textit{dest}\hspace{0.2em}\textit{suffix}}'.
Surely, the same effect is achieved by
directly specifying the
argument `{\textit{dest}\hspace{0.2em}\textit{suffix}}'
in the first form.
However, that requires to set up a different file
for each child. With the alternative form of the command
all these files can have exactly the same content
which simplifies setting them up and maintaining them.

For example, the following file |draft.tex|
with a compilation flag |\version| as described in \secref{sec:flags}
compiles the main document as a draft:
%
\begin{center}
\begin{tabular}{l}
|\def\version{draft}|\\
|\input{childdoc.def}|\\
|\childdocforward{|\textit{main}|}|
\end{tabular}
\end{center}
%
Likewise, the following files |final|\textit{nn}|.tex|
compile the final version of the child document
|child|\textit{nn}|.tex|:
%
\begin{center}
\begin{tabular}{l}
|\def\version{final}|\\
|\input{childdoc.def}|\\
|\childdocforwardprefix{final}{child}|
\end{tabular}
\end{center}
%

Note that when several versions of a main file and/or of each child file
are to be generated, it may be convenient to set up a |Makefile| or
shell script to automatise the process.

%%%%%%%%%%%%%%%%%%%%%%%%%%%%%%%%%%%%%%%%%%%%%%%%%%%%%%%%%%%%%%%%%%%%%%%%%%%%%%%%
\subsection{Command Line Processing}
\label{sec:commandline}

The effect of redirection files can also be achieved by invoking
the \LaTeX{} compiler with a more elaborate command line.
Most conveniently this should be done as part
of a shell script or a |Makefile|.

When using \textsf{childdoc} in the main file, the following
command lines effectively perform a redirection
(note that depending on the shell being used,
backslashes may have to be doubled: `|\|' $\to$ `|\\|'):
%
\begin{center}
|... -jobname "|\textit{target}|" |\\|"|[\textit{flags}]%
|\input{childdoc.def}\childdocforward[|\textit{main}|]{|\textit{dest}|}"|
\end{center}
%
Here \textit{target} is the name of the output file,
\textit{main} is the name of the main file
and \textit{dest} is the name of the main or child file to be processed
(all filenames without extensions).
The optional argument \textit{main} can be omitted
if \textit{main} matches \textit{dest}.
Optionally, compilation \textit{flags} can be defined via |\def| commands.
This command line makes the \TeX{} engine believe
it is compiling the file \textit{target}
whose content is specified as the latter parameter.
The provided code then forwards the processing to
\textit{main} or \textit{dest} as described in \secref{sec:forward}.

%%%%%%%%%%%%%%%%%%%%%%%%%%%%%%%%%%%%%%%%%%%%%%%%%%%%%%%%%%%%%%%%%%%%%%%%%%%%%%%%
\subsection{Include by Input}
\label{sec:input}

Including child documents by |\include| has some restrictions by design.
Most notably, the content of a child document always occupies
its own set of pages; pages cannot be shared between child documents.
Usually, this behaviour makes perfect sense
because each child document contain an essential part of the document.
However, in some situations it may be desirable to compose
a document from a collection of parts
without having mandatory page breaks between then.
For this case, the package
provides a mechanism to include parts
by |\input| which can also be processed individually.
However, by construction this mechanism
requires manual handling of the content to be output.

%%%%%%%%%%%%%%%%%%%%%%%%%%%%%%%%%%%%%%%%
\DescribeMacro{\ifchilddocmanual}
The main file should be prepared as usual, see \secref{sec:include}.
However, the document body must make a distinction
between processing of an individual part and of the main document, e.g.:
%
\begin{center}
\begin{tabular}{l}
|\ifchilddocmanual|\\
|\input{\childdocname}|\\
|\||else|\\
\textit{document body with }|\input{|\textit{part}|}|\\
|\||fi|
\end{tabular}
\end{center}
%
The conditional |\ifchilddocmanual| is true whenever
a part to be included by |\input| is being compiled,
and the name of the part is stored in |\childdocname|.

%%%%%%%%%%%%%%%%%%%%%%%%%%%%%%%%%%%%%%%%
\DescribeMacro{\childdocby}
Each part to be included by |\input| should start with:
%
\begin{center}
\begin{tabular}{l}
|\input{childdoc.def}|\\
|\childdocby{|\textit{main}|}|\\
\end{tabular}
\end{center}
%
The directive |\childdocby| is similar to |\childdocof|
described in \secref{sec:include},
but the subsequent selection of content must be done manually.
To that end, both |\ifchilddoc| and |\ifchilddocmanual|
will be true upon processing of a part,
and the name of the part is stored in |\childdocname|.
Note that |\jobname| will be set to the filename of the current part
so that each part receives an individual |.aux| file
that does not interfere with the |.aux| file(s) of the main document.
This behaviour can be altered by the alternative form
|\childdocby[*]{|\textit{main}|}| (with a non-empty optional argument)
which uses the |.aux| file of the main document
by setting |\jobname| to \textit{main}.

%%%%%%%%%%%%%%%%%%%%%%%%%%%%%%%%%%%%%%%%%%%%%%%%%%%%%%%%%%%%%%%%%%%%%%%%%%%%%%%%
\subsection{Driver Development}
\label{sec:driver}

The \textsf{childdoc} mechanism can also be use for the development
of definition files such as \LaTeX{} styles or classes.
This case differs from the above setup with multiple parts
included by |\include| in that no |\includeonly| should be invoked.
This can be achieved by starting the include file
(before |\ProvidesPackage|) with:
%
\begin{center}
\begin{tabular}{l}
|\input{childdoc.def}|\\
|\childdocforward{|\textit{main}|}|\\
\end{tabular}
\end{center}
%
or alternatively with:
%
\begin{center}
\begin{tabular}{l}
|\input{childdoc.def}|\\
|\childdocby{|\textit{main}|}|\\
\end{tabular}
\end{center}
%
Both forms have slightly different effects as described above.
The main file is prepared as usual, see \secref{sec:include}.

%%%%%%%%%%%%%%%%%%%%%%%%%%%%%%%%%%%%%%%%%%%%%%%%%%%%%%%%%%%%%%%%%%%%%%%%%%%%%%%%
\subsection{Legacy Detection}
\label{sec:detection}

The directive |\childdocmain| in the main file can detect
whether the complete document or merely a child is to be compiled
even without using the directive |\childdocof|.
This method is deprecated because it is less robust
and there is no compelling reason to use it;
it is merely provided for backward compatibility
and it may be removed in future versions.

If the detection mechanism is to be used,
it is mandatory to correctly specify
the filename of the main file as the argument of |\childdocmain|:
%
\begin{center}
\begin{tabular}{l}
|\input{childdoc.def}|\\
|\childdocmain{|\textit{main}|}|\\
\end{tabular}
\end{center}
%
If |\jobname| does not match the argument \textit{main} of |\childdocmain|,
it is assumed that |\jobname| points to the child file to be compiled.
When using |\childdocmain| with the main file specified as argument,
it suffices to start a child file
with just |\input{|\textit{main}|}|
without loading of the package and using |\childdocof|.
If instead all processing is done
with the appropriate \textsf{childdoc} directives,
the argument of \textit{main} of |\childdocmain| can be empty.

An alternative version of the command line processing described
in \secref{sec:commandline} using the detection mechanism reads:
%
\begin{center}
|... -jobname "|\textit{target}|" "|[\textit{flags}]%
[|\def\jobname{|\textit{dest}|}|]|\input{|\textit{main}|}"|
\end{center}

%%%%%%%%%%%%%%%%%%%%%%%%%%%%%%%%%%%%%%%%%%%%%%%%%%%%%%%%%%%%%%%%%%%%%%%%%%%%%%%%
\subsection{Manual Code}
\label{sec:manual}

In case one cannot be certain whether the definitions file |childdoc.def|
is installed on the target \TeX{} distribution
and one prefers not to ship it,
it is conceivable to paste a few relevant commands into the sources.

To that end, drop all statements |\input{childdoc.def}|
and perform the replacements as outlined below.
Instead of |\childdocmain{|\textit{main}|}| add the following code
to the top of the main file:
%
\begin{center}
\begin{tabular}{l}
|\||ifdefined\childdocname\endinput\||fi\newif\ifchilddoc|\\
|\edef\childdocname{\scantokens\expandafter{\jobname\noexpand}}|\\
|\def\childdocmain{|\textit{main}|}\||ifx\childdocmain\childdocname\||else|\\
|\childdoctrue\includeonly{\childdocname}\let\jobname\childdocmain\||fi|\\
\end{tabular}
\end{center}
%
Instead of |\childdocof{|\textit{main}|}| just include the main file
at the top of each child file:
%
\begin{center}
|\input{|\textit{main}|}|
\end{center}
%
A simple redirection |\childdocforward{|\textit{dest}|}| is achieved by:
%
\begin{center}
|\def\jobname{|\textit{dest}|}\input{\jobname}|
\end{center}
%
The redirection with prefix
|\childdocforwardprefix[|\textit{prefix}|]{|\textit{dest}|}|
is accomplished by:
%
\begin{center}
\begin{tabular}{l}
|{\edef\jobname{\scantokens\expandafter{\jobname\noexpand}}|\\
|\def\redirectjob |\textit{prefix}|#1~~~{\gdef\jobname{|\textit{dest}|#1}}|\\
|\expandafter\redirectjob\jobname~~~}\input{\jobname}|
\end{tabular}
\end{center}

In an alternative approach,
child documents can be compiled by a specific command line
without additional code or specific definitions:
%
\begin{center}
|... -jobname "|\textit{target}|" "|[\textit{flags}]%
|\includeonly{|\textit{dest}|}\input{|\textit{main}|}"|
\end{center}
%

%%%%%%%%%%%%%%%%%%%%%%%%%%%%%%%%%%%%%%%%%%%%%%%%%%%%%%%%%%%%%%%%%%%%%%%%%%%%%%%%
%%%%%%%%%%%%%%%%%%%%%%%%%%%%%%%%%%%%%%%%%%%%%%%%%%%%%%%%%%%%%%%%%%%%%%%%%%%%%%%%
\section{Information}

%%%%%%%%%%%%%%%%%%%%%%%%%%%%%%%%%%%%%%%%%%%%%%%%%%%%%%%%%%%%%%%%%%%%%%%%%%%%%%%%
\subsection{Copyright}

Copyright \copyright{} 2017--2018 Niklas Beisert

This work may be distributed and/or modified under the
conditions of the \LaTeX{} Project Public License, either version 1.3
of this license or (at your option) any later version.
The latest version of this license is in
  \url{http://www.latex-project.org/lppl.txt}
and version 1.3 or later is part of all distributions of \LaTeX{}
version 2005/12/01 or later.

This work has the LPPL maintenance status `maintained'.

The Current Maintainer of this work is Niklas Beisert.

This work consists of the files |README.txt|, |childdoc.ins| and |childdoc.dtx|
as well as the derived files |childdoc.def|, |cdocsamp.tex|
with |cdocsch1.tex|, |cdocsch2.tex|, |cdocspt3.tex|, |cdocspt4.tex|,
|cdocsdrf.tex|, |cdocsfn1.tex|, |cdocsfn2.tex|
as well as |childdoc.pdf|.

%%%%%%%%%%%%%%%%%%%%%%%%%%%%%%%%%%%%%%%%%%%%%%%%%%%%%%%%%%%%%%%%%%%%%%%%%%%%%%%%
\subsection{Files and Installation}

The package consists of the files:
%
\begin{center}
\begin{tabular}{ll}
    |README.txt|   & readme file \\
    |childdoc.ins| & installation file \\
    |childdoc.dtx| & source file \\
    |childdoc.def| & definition file \\
    |cdocsamp.tex| & sample main file \\
    |cdocsch1.tex| & sample include file \\
    |cdocsch2.tex| & sample include file \\
    |cdocspt3.tex| & sample part file \\
    |cdocspt4.tex| & sample part file \\
    |cdocsdrf.tex| & sample redirection file \\
    |cdocsfn1.tex| & sample redirection file \\
    |cdocsfn2.tex| & sample redirection file \\
    |childdoc.pdf| & manual
\end{tabular}
\end{center}
%
The distribution consists of the files
|README.txt|, |childdoc.ins| and |childdoc.dtx|.
%
\begin{itemize}
\item
Run (pdf)\LaTeX{} on |childdoc.dtx|
to compile the manual |childdoc.pdf| (this file).
\item
Run \LaTeX{} on |childdoc.ins| to create the definitions file |childdoc.def|
and the sample |cdocsamp.tex| with include files
|cdocsch1.tex|, |cdocsch2.tex|, |cdocspt3.tex|, |cdocspt4.tex|,
|cdocsdrf.tex|, |cdocsfn1.tex|, |cdocsfn2.tex|.
Then copy the file |childdoc.def| to an appropriate directory of your \LaTeX{}
distribution, e.g.\ \textit{texmf-root}|/tex/latex/childdoc|.
\end{itemize}

%%%%%%%%%%%%%%%%%%%%%%%%%%%%%%%%%%%%%%%%%%%%%%%%%%%%%%%%%%%%%%%%%%%%%%%%%%%%%%%%
\subsection{Related CTAN Packages}

There are several other packages which offer a similar functionality:
%
\begin{itemize}
\item
The packages
\href{http://ctan.org/pkg/docmute}{\textsf{docmute}},
\href{http://ctan.org/pkg/includex}{\textsf{includex}} and
\href{http://ctan.org/pkg/standalone}{\textsf{standalone}}
provide commands to include only the document body of
a child file thus allowing both files to be compiled individually.
\item
The packages \href{http://ctan.org/pkg/subdocs}{\textsf{subdocs}}
and \href{http://ctan.org/pkg/subfiles}{\textsf{subfiles}}
provide structures in which the main and child documents can be
encapsulated and allowing them to be compiled individually.
The inclusion mechanism is different from the conventional |\include|.
\item
The package \href{http://ctan.org/pkg/combine}{\textsf{combine}}
is an elaborate solution to combine several documents into one.
\end{itemize}
%
See also the CTAN topic \href{http://ctan.org/topic/subdocs}{\textsf{subdocs}}
for further related packages.
The present package differs from the above solutions in that
a document structure constructed with the conventional |\include| mechanism
just needs two extra commands at the top of every file
such that all constituent files can be compiled individually.

%%%%%%%%%%%%%%%%%%%%%%%%%%%%%%%%%%%%%%%%%%%%%%%%%%%%%%%%%%%%%%%%%%%%%%%%%%%%%%%%
%\subsection{Feature Suggestions}
%
%The following is a list of features which may be useful for future
%versions of this package:
%%
%\begin{itemize}
%\item
%\ldots
%\end{itemize}

%%%%%%%%%%%%%%%%%%%%%%%%%%%%%%%%%%%%%%%%%%%%%%%%%%%%%%%%%%%%%%%%%%%%%%%%%%%%%%%%
\subsection{Revision History}

%%%%%%%%%%%%%%%%%%%%%%%%%%%%%%%%%%%%%%%%
\paragraph{v2.0:} 2018/12/30

\begin{itemize}
\item
immediate forward processing
\item
added |\childdocby| mechanism
\item
manual restructured
\end{itemize}

%%%%%%%%%%%%%%%%%%%%%%%%%%%%%%%%%%%%%%%%
\paragraph{v1.6:} 2018/01/17

\begin{itemize}
\item
application for development of include files
\item
corrections to manual
\end{itemize}

%%%%%%%%%%%%%%%%%%%%%%%%%%%%%%%%%%%%%%%%
\paragraph{v1.5:} 2017/05/21

\begin{itemize}
\item
more complete structuring introduced
\item
|\childdocof| introduced
\item
|\childdoc| renamed to |\childdocmain|
\item
|\childredirect| renamed to |\childdocforward| and |\childdocforwardprefix|
and functionality expanded
\end{itemize}

%%%%%%%%%%%%%%%%%%%%%%%%%%%%%%%%%%%%%%%%
\paragraph{v1.0:} 2017/04/27

\begin{itemize}
\item
manual and install package
\item
first version published on CTAN
\end{itemize}

%%%%%%%%%%%%%%%%%%%%%%%%%%%%%%%%%%%%%%%%
\paragraph{v0.6:} 2017/04/26

\begin{itemize}
\item
redirection mechanism added
\end{itemize}

%%%%%%%%%%%%%%%%%%%%%%%%%%%%%%%%%%%%%%%%
\paragraph{v0.5:} 2017/04/26

\begin{itemize}
\item
functionality in definition file
\end{itemize}


%%%%%%%%%%%%%%%%%%%%%%%%%%%%%%%%%%%%%%%%%%%%%%%%%%%%%%%%%%%%%%%%%%%%%%%%%%%%%%%%
%%%%%%%%%%%%%%%%%%%%%%%%%%%%%%%%%%%%%%%%%%%%%%%%%%%%%%%%%%%%%%%%%%%%%%%%%%%%%%%%
%%%%%%%%%%%%%%%%%%%%%%%%%%%%%%%%%%%%%%%%%%%%%%%%%%%%%%%%%%%%%%%%%%%%%%%%%%%%%%%%
\appendix

\settowidth\MacroIndent{\rmfamily\scriptsize 000\ }

 \DocInput{childdoc.dtx}

\end{document}
%</driver>
% \fi
%
% %%%%%%%%%%%%%%%%%%%%%%%%%%%%%%%%%%%%%%%%%%%%%%%%%%%%%%%%%%%%%%%%%%%%%%%%%%%%%%
% %%%%%%%%%%%%%%%%%%%%%%%%%%%%%%%%%%%%%%%%%%%%%%%%%%%%%%%%%%%%%%%%%%%%%%%%%%%%%%
% \section{Sample}
%\iffalse
%<*samplemain>
%\fi
%
% The following presents a sample document
% with two chapters, two parts, a title page,
% a compile flag as well as three forwarding files to set the flag.
% It consists of eight |.tex| files:
% \begin{center}
% \begin{tabular}{ll}
% |cdocsamp.tex|&main file\\
% |cdocsch1.tex|&include file for chapter 1\\
% |cdocsch2.tex|&include file for chapter 2\\
% |cdocspt3.tex|&include file for part 3\\
% |cdocspt4.tex|&include file for part 4\\
% |cdocsdrf.tex|&forwarding file for main file in draft mode\\
% |cdocsfi1.tex|&forwarding file for final version of chapter 1\\
% |cdocsfi2.tex|&forwarding file for final version of chapter 2\\
% \end{tabular}
% \end{center}
% Each of the eight files can be compiled directly by the \LaTeX{} compiler.
%
% %%%%%%%%%%%%%%%%%%%%%%%%%%%%%%%%%%%%%%
% \paragraph{Main File.}
%
% The main file is called |cdocsamp.tex|.
%
% Load the \textsf{childdoc} definitions and
% declare the filename for the main document:
%    \begin{macrocode}
\input{childdoc.def}
\childdocmain{}
%    \end{macrocode}

% Optional override for |\version| flag:
%    \begin{macrocode}
%%\ifchilddoc\else\providecommand{\version}{draft}\fi
%    \end{macrocode}

% Define the default values for the |\version| flag
% (|final| for the main file and |draft| for childs):
%    \begin{macrocode}
\ifchilddoc
\providecommand{\version}{draft}
\else
\providecommand{\version}{final}
\fi
%    \end{macrocode}

% Load the standard document class:
%    \begin{macrocode}
\documentclass[12pt]{article}
%    \end{macrocode}

% Start the document body:
%    \begin{macrocode}
\begin{document}
%    \end{macrocode}

% Declare a title page.
% Print title, part of document being processed and version flag:
%    \begin{macrocode}
\addtocounter{page}{-1}
\begin{center}
{\LARGE\bfseries{}childdoc example\par}
\vspace{1cm}
\ifchilddoc
\ifchilddocmanual part\else chapter\fi:
`\childdocname' of `\childdocjob'\par
\else
main document: `\childdocjob'\par
\fi
version: \version\par
\end{center}
\newpage
%    \end{macrocode}

% Manually include selected file,
% otherwise process as usual:
%    \begin{macrocode}
\ifchilddocmanual
\section*{part `\childdocname'}
\input{\childdocname}
\else
%    \end{macrocode}

% Include the two chapters:
%    \begin{macrocode}
\include{cdocsch1}
\include{cdocsch2}
%    \end{macrocode}

% Include the two parts unless only chapters should be displayed:
%    \begin{macrocode}
\ifchilddoc\else
\section{part three}
\input{cdocspt3}
\section{part four}
\input{cdocspt4}
\fi
%    \end{macrocode}

% Process as usual until here:
%    \begin{macrocode}
\fi
%    \end{macrocode}

% End of document body:
%    \begin{macrocode}
\end{document}
%    \end{macrocode}
%\iffalse
%</samplemain>
%\fi
%
% %%%%%%%%%%%%%%%%%%%%%%%%%%%%%%%%%%%%%%
% \paragraph{Chapter Include Files.}
%
% The include files are called |cdocsch1.tex| and |cdocsch2.tex|.
%
%\iffalse
%<*samplechap1|samplechap2>
%\fi

% Optional override for |\version| flag:
%    \begin{macrocode}
%%\providecommand{\version}{final}
%    \end{macrocode}

% Include the main document:
%    \begin{macrocode}
\input{childdoc.def}
\childdocof{cdocsamp}
%    \end{macrocode}

%\iffalse
%</samplechap1|samplechap2>
%\fi
%
%\iffalse
%<*samplechap1>
%\fi
% Some text for chapter 1:
%    \begin{macrocode}
\section{one}
some text in chapter one
%    \end{macrocode}

%\iffalse
%</samplechap1>
%\fi
% Some text for chapter 2:
%\iffalse
%<*samplechap2>
%\fi
%    \begin{macrocode}
\section{two}
more text in chapter two
%    \end{macrocode}

%\iffalse
%</samplechap2>
%\fi
%
% %%%%%%%%%%%%%%%%%%%%%%%%%%%%%%%%%%%%%%
% \paragraph{Part Include Files.}
%
% The include files are called |cdocspt3.tex| and |cdocspt4.tex|.
%
%\iffalse
%<*samplepart3|samplepart4>
%\fi

% Optional override for |\version| flag:
%    \begin{macrocode}
%%\providecommand{\version}{final}
%    \end{macrocode}

% Include the main document:
%    \begin{macrocode}
\input{childdoc.def}
\childdocby{cdocsamp}
%    \end{macrocode}

%\iffalse
%</samplepart3|samplepart4>
%\fi
%
%\iffalse
%<*samplepart3>
%\fi
% Some text for part 3:
%    \begin{macrocode}
some text in part three
%    \end{macrocode}

%\iffalse
%</samplepart3>
%\fi
% Some text for part 4:
%\iffalse
%<*samplepart4>
%\fi
%    \begin{macrocode}
more text in part four
%    \end{macrocode}

%\iffalse
%</samplepart4>
%\fi
%
% %%%%%%%%%%%%%%%%%%%%%%%%%%%%%%%%%%%%%%
% \paragraph{Forwarding for a Complete Draft.}
%
% The following forwarding file |cdocsdrf.tex|
% compiles the main document in draft mode:
%\iffalse
%<*sampledraft>
%\fi
%    \begin{macrocode}
\def\version{draft}
\input{childdoc.def}
\childdocforward{cdocsamp}
%    \end{macrocode}

%\iffalse
%</sampledraft>
%\fi
%
% %%%%%%%%%%%%%%%%%%%%%%%%%%%%%%%%%%%%%%
% \paragraph{Forwarding for Final Version of the Chapters.}
%
% The following forwarding files |cdocsfn1.tex| and |cdocsfn2.tex|
% (with identical content)
% compile the final versions of the child documents
% |cdocsch1.tex| and |cdocsch2.tex|, respectively:
%\iffalse
%<*samplefinal>
%\fi
%    \begin{macrocode}
\def\version{final}
\input{childdoc.def}
\childdocforwardprefix[cdocsamp]{cdocsfn}{cdocsch}
%    \end{macrocode}

%\iffalse
%</samplefinal>
%\fi
%
% %%%%%%%%%%%%%%%%%%%%%%%%%%%%%%%%%%%%%%
% \paragraph{Command Line Processing.}
%
% The following three command lines generate the output files
% |cdocscld|, |cdocscl1| and |cdocscl2|
% which should be identical to
% |cdocsdrf|, |cdocsch1| and |cdocsfn2|, respectively:
% \begin{center}
% \begin{tabular}{l}
% |latex -jobname cdocscld \|\\
% |  "\def\version{draft}\input{childdoc.def}\childdocforward{cdocsamp}"|\\
% |latex -jobname cdocscl1 \|\\
% |  "\input{childdoc.def}\childdocforward[cdocsamp]{cdocsch1}"|\\
% |latex -jobname cdocscl2 \|\\
% |  "\def\version{final}\input{childdoc.def}\childdocforward{cdocsch2}"|
% \end{tabular}
% \end{center}
% Note that the trailing backslash on each first line
% merely continues the input to the second line
% (for convenient cut ant paste).
% Furthermore, the command |latex| can be replaced by any
% of its alternative versions such as |pdflatex|.
%
% %%%%%%%%%%%%%%%%%%%%%%%%%%%%%%%%%%%%%%%%%%%%%%%%%%%%%%%%%%%%%%%%%%%%%%%%%%%%%%
% %%%%%%%%%%%%%%%%%%%%%%%%%%%%%%%%%%%%%%%%%%%%%%%%%%%%%%%%%%%%%%%%%%%%%%%%%%%%%%
% \section{Implementation}
%\iffalse
%<*package>
%\fi
%
% This section describes the definitions file |childdoc.def|.

% The definitions cannot be loaded using |\usepackage| or |\RequirePackage|
% which has a mechanism to prevent loading a style file more than once.
% When loading the definitions by means of |\input|
% multiple instances have to be prevented manually:
%\iffalse
%This code needs to be before the `\ProvidesFile' directive
%which is defined at the beginning of this file.
%Therefore it is also placed there and commented out here.
%</package>
%<*discard>
%\fi
%    \begin{macrocode}
\ifdefined\childdocmain\endinput\fi
%    \end{macrocode}
%\iffalse
%</discard>
%<*package>
%\fi
%
% \macro{\ifchilddoc}
% \macro{\ifchilddocmanual}
% The conditional |\ifchilddoc| tells whether a
% child (true) or main (false) document is being compiled.
% The conditional |\ifchilddocmanual| tells whether
% the |\includeonly| mechanism is used (false) or
% the selection of child files must be performed manually (true).
% The definitions initialise to false:
%    \begin{macrocode}
\newif\ifchilddoc
\newif\ifchilddocmanual
%    \end{macrocode}

% \macro{\childdocname}
% \macro{\childdocjob}
% The macro |\childdocname| stores the name of the main document
% to be compiled. The macro |\childdocjob| stores the name of
% the document on which the \LaTeX{} compiler was originally invoked.
% The content of |\jobname| cannot be compared
% to filenames specified in the source due to different catcodes.
% The following code rescans |\jobname|, stores the result
% in |\childdocname| and saves a copy in |\childdocjob|:
%    \begin{macrocode}
\edef\childdocname{\scantokens\expandafter{\jobname\noexpand}}
\let\childdocjob\childdocname
%    \end{macrocode}

% \macro{\childdocdisable}
% The macro |\childdocdisable| prevents the main file
% from being processed more than once.
% At this stage, the main document command |\childdocmain|
% is assumed to be called once again where it should do nothing.
% Any subsequent call to it should prevent
% a secondary processing of the main document
% It overwrites the forwarding commands
% |\childdocof| and |\childdocforward|
% with empty macros to prevent further inclusions of the main document:
%    \begin{macrocode}
\newcommand{\childdocdisable}
{
  \renewcommand{\childdocmain}[1]{\renewcommand{\childdocmain}[1]{\endinput}}
  \renewcommand{\childdocof}[1]{}
  \renewcommand{\childdocby}[2][]{}
  \renewcommand{\childdocforward}[2][]{}
  \renewcommand{\childdocdisable}{}
}
%    \end{macrocode}

% \macro{\childdocmain}
% The macro |\childdocmain| is to be called at the top of the main file
% with nothing or the main filename (without extension) as argument.
% First, it breaks loops.
% If the argument is not empty and does not match |\childdocname|
% (which is set by the first inclusion of |childdoc.def|),
% |\ifchilddoc| is set to true, |\includeonly| is applied to the child file
% and |\jobname| is set to the main file
% (for proper handling of |.aux| files):
%    \begin{macrocode}
\newcommand{\childdocmain}[1]
{
  \childdocdisable\childdocmain{}
  \if?#1?\else
    \begingroup
      \def\childdoctmp{#1}
      \ifx\childdoctmp\childdocname
        \def\childdoctmp{}
      \else
        \def\childdoctmp
        {
          \childdoctrue
          \includeonly{\childdocname}
          \def\childdocjob{#1}
          \def\jobname{#1}
        }
      \fi
      \expandafter
    \endgroup
    \childdoctmp
  \fi
}
%    \end{macrocode}

% \macro{\childdocof}
% The command |\childdocof| redirects
% compilation to the main file |#1|.
%    \begin{macrocode}
\newcommand{\childdocof}[1]
{
  \childdocdisable
  \childdoctrue
  \includeonly{\childdocname}
  \def\jobname{#1}
  \def\childdocjob{#1}
  \input{#1}
}
%    \end{macrocode}

% \macro{\childdocby}
% The command |\childdocby| ....
%    \begin{macrocode}
\newcommand{\childdocby}[2][]
{
  \childdocdisable
  \childdoctrue
  \childdocmanualtrue
  \if?#1?\else
    \def\jobname{#2}
  \fi
  \def\childdocjob{#2}
  \input{#2}
  \endinput
}
%    \end{macrocode}

% \macro{\childdocforward}
% The command |\childdocforward| redirects
% compilation to the main file or
% (if the optional argument is given) a child file.
% Parameters are set as if the main file
% or a child file starting with |\childdocof| was compiled.
% Then compilation is handed over to the main file:
%    \begin{macrocode}
\newcommand{\childdocforward}[2][]
{
  \begingroup
    \if?#1?
      \def\childdoctmp
      {
        \def\childdocname{#2}
        \def\childdocjob{#2}
        \def\jobname{#2}
        \input{#2}
        \endinput
      }
    \else
      \def\childdoctmp
      {
        \childdocdisable
        \def\childdocname{#2}
        \childdoctrue
        \includeonly{#2}
        \def\childdocjob{#1}
        \def\jobname{#1}
        \input{#1}
        \endinput
      }
    \fi
    \expandafter
  \endgroup
  \childdoctmp
}
%    \end{macrocode}

% \macro{\childdocforwardprefix}
% The command |\childdocforwardprefix| redirects
% compilation to the main or a child file by means of a pattern.
% The prefix |#1| in the current filename is replaced by |#2|
% and the suffix of the current filename is kept
% (it is assumed that the filename does not contain the substring `|~~~|'
% which is used as a delimiter).
% Compilation is handed over to the new file by |\childdocforward|:
%    \begin{macrocode}
\newcommand{\childdocforwardprefix}[3][]
{
  \begingroup
    \def\childdocextract #2##1~~~{\def\childdoctmp{\childdocforward[#1]{#3##1}}}
    \expandafter\childdocextract\childdocname~~~
    \expandafter
  \endgroup
  \childdoctmp
}
%    \end{macrocode}

% \macro{\childdoc}
% The deprecated macro |\childdoc| is a legacy version of |\childdocmain|:
%    \begin{macrocode}
\newcommand{\childdoc}{\childdocmain}
%    \end{macrocode}

% \macro{\childdocredirect}
% The deprecated macro |\childdocredirect| is a legacy version
% of |\childdocforward| and |\childdocforwardprefix|:
%    \begin{macrocode}
\newcommand{\childdocredirect}[2][]
{
  \begingroup
    \if?#1?
      \def\childdoctmp{\childdocforward{#2}}
    \else
      \def\childdoctmp{\childdocforwardprefix{#1}{#2}}
    \fi
    \expandafter
  \endgroup
  \childdoctmp
}
%    \end{macrocode}

%\iffalse
%</package>
%\fi
%
\endinput
|\\
|\childdocforwardprefix{final}{child}|
\end{tabular}
\end{center}
%

Note that when several versions of a main file and/or of each child file
are to be generated, it may be convenient to set up a |Makefile| or
shell script to automatise the process.

%%%%%%%%%%%%%%%%%%%%%%%%%%%%%%%%%%%%%%%%%%%%%%%%%%%%%%%%%%%%%%%%%%%%%%%%%%%%%%%%
\subsection{Command Line Processing}
\label{sec:commandline}

The effect of redirection files can also be achieved by invoking
the \LaTeX{} compiler with a more elaborate command line.
Most conveniently this should be done as part
of a shell script or a |Makefile|.

When using \textsf{childdoc} in the main file, the following
command lines effectively perform a redirection
(note that depending on the shell being used,
backslashes may have to be doubled: `|\|' $\to$ `|\\|'):
%
\begin{center}
|... -jobname "|\textit{target}|" |\\|"|[\textit{flags}]%
|% \iffalse
%
% childdoc.dtx Copyright (C) 2017-2018 Niklas Beisert
%
% This work may be distributed and/or modified under the
% conditions of the LaTeX Project Public License, either version 1.3
% of this license or (at your option) any later version.
% The latest version of this license is in
%   http://www.latex-project.org/lppl.txt
% and version 1.3 or later is part of all distributions of LaTeX
% version 2005/12/01 or later.
%
% This work has the LPPL maintenance status `maintained'.
%
% The Current Maintainer of this work is Niklas Beisert.
%
% This work consists of the files childdoc.dtx and childdoc.ins
% and the derived files childdoc.def and cdocsamp.tex with
% cdocsch1.tex, cdocsch2.tex, cdocsdrf.tex, cdocsfn1.tex, cdocsfn2.tex.
%
%<package>\ifdefined\childdocmain\endinput\fi
%<package>\ProvidesFile{childdoc.def}[2018/12/30 v2.0 child document driver]
%<samplemain>\ProvidesFile{cdocsamp.tex}[2018/12/30 v2.0 sample for childdoc]
%<*driver>
%\ProvidesFile{childdoc.drv}[2018/12/30 v2.0 childdoc reference manual file]
\PassOptionsToClass{10pt,a4paper}{article}
\documentclass{ltxdoc}

\usepackage[margin=35mm]{geometry}
\usepackage{hyperref}
\usepackage{hyperxmp}
\usepackage[usenames]{color}

\hypersetup{colorlinks=true}
\hypersetup{pdfstartview=FitH}
\hypersetup{pdfpagemode=UseNone}
\hypersetup{pdfsource={}}
\hypersetup{pdflang={en-UK}}
\hypersetup{pdfcopyright={Copyright 2017-2018 Niklas Beisert.
  This work may be distributed and/or modified under the
  conditions of the LaTeX Project Public License, either version 1.3
  of this license or (at your option) any later version.}}
\hypersetup{pdflicenseurl={http://www.latex-project.org/lppl.txt}}
\hypersetup{pdfcontactaddress={ETH Zurich, ITP, HIT K,
  Wolfgang-Pauli-Strasse 27}}
\hypersetup{pdfcontactpostcode={8093}}
\hypersetup{pdfcontactcity={Zurich}}
\hypersetup{pdfcontactcountry={Switzerland}}
\hypersetup{pdfcontactemail={nbeisert@itp.phys.ethz.ch}}
\hypersetup{pdfcontacturl={http://people.phys.ethz.ch/\xmptilde nbeisert/}}

\newcommand{\secref}[1]{\hyperref[#1]{section \ref*{#1}}}

\parskip1ex
\parindent0pt
\let\olditemize\itemize
\def\itemize{\olditemize\parskip0pt}

\begin{document}

\title{The \textsf{childdoc} Package}
\hypersetup{pdftitle={The childdoc Package}}
\author{Niklas Beisert\\[2ex]
  Institut f\"ur Theoretische Physik\\
  Eidgen\"ossische Technische Hochschule Z\"urich\\
  Wolfgang-Pauli-Strasse 27, 8093 Z\"urich, Switzerland\\[1ex]
  \href{mailto:nbeisert@itp.phys.ethz.ch}
  {\texttt{nbeisert@itp.phys.ethz.ch}}}
\hypersetup{pdfauthor={Niklas Beisert}}
\hypersetup{pdfsubject={Manual for the LaTeX2e Package childdoc}}
\date{30 December 2018, \textsf{v2.0}}
\maketitle

\begin{abstract}\noindent
\textsf{childdoc} is a \LaTeXe{} package
that enables the direct compilation
of document sections included by |\include|
to individual files.
\end{abstract}

\begingroup
\parskip0ex
\tableofcontents
\endgroup

%%%%%%%%%%%%%%%%%%%%%%%%%%%%%%%%%%%%%%%%%%%%%%%%%%%%%%%%%%%%%%%%%%%%%%%%%%%%%%%%
%%%%%%%%%%%%%%%%%%%%%%%%%%%%%%%%%%%%%%%%%%%%%%%%%%%%%%%%%%%%%%%%%%%%%%%%%%%%%%%%
\section{Introduction}

\LaTeX{} provides a mechanism to structure a large document (such as a book)
into a main file and several child files (containing the chapters)
using the |\include| command.
This mechanism is beneficial for documents
which span hundreds of pages in order to
make the source file(s) more manageable.
Moreover, compilation can be restricted to
selected child files by means of the |\includeonly| command.
The latter feature can be used to reduce the compilation time while editing
(this was significantly more useful in the earlier days of \LaTeX{})
or to generate a smaller document which is easier to navigate.
Another application of |\includeonly| is to generate
documents consisting of selected parts of the complete document.

However, there are a few drawbacks of the plain |\include| mechanism:
\begin{itemize}
\item
The child files cannot be compiled on their own,
they can only be compiled via the main file.
A naive editing environment
(such as a text editor with an option
to have the current file processed by \LaTeX)
may require one to switch to the main file before compiling;
attempting to compile the child file produces errors.
\item
The main file must be modified (each time)
to adjust the |\includeonly| command
to the present needs. This easily leaves the main file in a messy state.
\item
The generated document will always carry the filename
of the main document. This is inconvenient if
several child files are to be compiled and
to be kept for distribution.
\end{itemize}

The present package provides a simple interface
to make child files individually compilable by \LaTeX{}.
Compiling a child file then has the same effect as compiling
the main file with an |\includeonly| command
to select the appropriate child.
Moreover the generated document will carry the name of the child
rather than the main file.
This resolves all three above issues.

This feature is meant to make the editing of books,
thesis documents and lecture notes somewhat more convenient.
However, the package can also be used efficiently for
composing a series of documents (such as exercise sheets)
which are typically distributed individually.
It then assists the author in generating the individual documents
(potentially in different versions)
as well as a document containing the collected series.
Another application is in developing style files
or other kinds of included material
where compilation of the style file could redirect
to a sample or test file.

%%%%%%%%%%%%%%%%%%%%%%%%%%%%%%%%%%%%%%%%%%%%%%%%%%%%%%%%%%%%%%%%%%%%%%%%%%%%%%%%
%%%%%%%%%%%%%%%%%%%%%%%%%%%%%%%%%%%%%%%%%%%%%%%%%%%%%%%%%%%%%%%%%%%%%%%%%%%%%%%%
\section{Usage}

First of all, the package \textsf{childdoc} is \emph{not} a standard
\LaTeXe{} |.sty| style file! Therefore it needs to be invoked in
a non-standard way.

%%%%%%%%%%%%%%%%%%%%%%%%%%%%%%%%%%%%%%%%%%%%%%%%%%%%%%%%%%%%%%%%%%%%%%%%%%%%%%%%
\subsection{Included Files}
\label{sec:include}

%%%%%%%%%%%%%%%%%%%%%%%%%%%%%%%%%%%%%%%%
\DescribeMacro{\childdocmain}
To use the package, add the commands
\begin{center}
\begin{tabular}{l}
|\input{childdoc.def}|\\
|\childdocmain{}|\\
\end{tabular}
\end{center}
at the very top of the main \LaTeX{} file,
in particular \emph{before} the |\documentclass| statement!
The argument of |\childdocmain| should be left empty
(but it must be present).

%%%%%%%%%%%%%%%%%%%%%%%%%%%%%%%%%%%%%%%%
\DescribeMacro{\childdocof}
Furthermore, add the commands
\begin{center}
\begin{tabular}{l}
|\input{childdoc.def}|\\
|\childdocof{|\textit{main}|}|\\
\end{tabular}
\end{center}
at the top of every child file \textit{child}
which is included by |\include{|\textit{child}|}|
from within the main file
(or at least for those files to be compiled individually).
The argument \textit{main} must be the filename of the main file.

There are a couple of
considerations in setting up the main and child documents:

%%%%%%%%%%%%%%%%%%%%%%%%%%%%%%%%%%%%%%%%
\paragraph{Restrictions.}

Please note the following restrictions:
\begin{itemize}
\item
|\childdocmain| must be called with one argument \textit{main}
to ensure compatibility with earlier version of the package.
It must either be empty (|\childdocmain{}|)
or precisely match the filename of the main file in which it is specified.
See \secref{sec:detection} for further information.
\item
The filename \textit{main} must be specified without the |.tex| extension.
\item
The filename \textit{main} is case sensitive
(even in case-insensitive file systems)
due to internal string comparison.
\item
The argument \textit{main} should be fully expanded, it cannot be a macro.
\item
Subdirectories and special characters should be avoided in filenames.
\item
The command |\childdocmain{|\textit{main}|}| must be followed by a whitespace.
It should not be followed immediately by another command
or by a comment mark `|%|'.
This is because the \TeX{} parser reads the token immediately following
the argument of |\childdocmain| and puts it
at the beginning of every child section;
however, a white\-space is ignored.
\end{itemize}

%%%%%%%%%%%%%%%%%%%%%%%%%%%%%%%%%%%%%%%%
\paragraph{Content of Main File.}

It is advisable to place all content in the child files included by |\include|.
Any output contained in the main file will appear in all child documents
unless suppressed manually;
it cannot be suppressed automatically by the |\includeonly| directive
and thus should normally be avoided.
A method to include some content in the main file
by means of conditional processing is described in \secref{sec:conditional}.

%%%%%%%%%%%%%%%%%%%%%%%%%%%%%%%%%%%%%%%%
\paragraph{Page Numbering.}

When only a part of the document is compiled,
the appropriate numbering of pages
(as well as other status parameters)
is determined from the |.aux| files.
The latter contain information from previous passes.
However this information needs to propagate through
all intermediate child documents.
Therefore the page numbering in child documents may well
be inconsistent until the complete document is compiled at least once.

A useful (if unconventional) way to always ensure a consistent
page numbering is to restart the numbering in each child document
and denote the pages by `\textit{child}|.|\textit{page}'
where \textit{child} represents the chapter/section number of the child file.
This can be achieved by the command
|\numberwithin{page}{|\textit{child}|}|
of the \textsf{amsmath} package
where \textit{child} can be |chapter| or |section|
depending on the chosen structuring.
Alternatively, one can modify the macro |\thepage| appropriately
and reset the counter |page| at the start of each child file.

%%%%%%%%%%%%%%%%%%%%%%%%%%%%%%%%%%%%%%%%%%%%%%%%%%%%%%%%%%%%%%%%%%%%%%%%%%%%%%%%
\subsection{Conditional Processing}
\label{sec:conditional}

The package provides a mechanism to compile different versions
of a document. To customise the versions further some conditional processing
can come in handy to distinguish which version is being compiled.
The package provides two macros to describe the compilation context:

%%%%%%%%%%%%%%%%%%%%%%%%%%%%%%%%%%%%%%%%
\DescribeMacro{\ifchilddoc}
The conditional |\ifchilddoc| distinguishes between the compilation of
child documents and the main document:
%
\begin{center}
|\ifchilddoc |\textit{child-code}| |[|\||else |\textit{main-code}]| \||fi|
\end{center}

%%%%%%%%%%%%%%%%%%%%%%%%%%%%%%%%%%%%%%%%
\DescribeMacro{\childdocname}
\DescribeMacro{\childdocjob}
The macro |\childdocname| contains the filename (without extension)
of the main or child file being processed.
Note that |\childdocjob| will always contain the name of the main file.

%%%%%%%%%%%%%%%%%%%%%%%%%%%%%%%%%%%%%%%%
\paragraph{Title Page.}

Conditional processing can be used to include a title or banner page
in the main document when proper precautions are taken.
Importantly, the code in the main file should ensure that the page counter
(as well as other status parameters which are stored in the |.aux| files)
takes the same value after the conditional processing.
Otherwise the page numbers may take divergent values
depending on which part is compiled.

For example, a title page could be declared by:
%
\begin{center}
\begin{tabular}{l}
|\ifchilddoc\||else|\\
|\addtocounter{page}{-1}|\\
\textit{code for title page}\\
|\newpage|\\
|\||fi|
\end{tabular}
\end{center}
%
A banner page for the child documents can be generated by:
%
\begin{center}
\begin{tabular}{l}
|\ifchilddoc|\\
|\addtocounter{page}{-1}|\\
\textit{code for banner page}\\
|\newpage|\\
|\||fi|
\end{tabular}
\end{center}
%
Here one could write a message such as:
\begin{center}
|This is the part \childdocname{} of \childdocjob{}.|
\end{center}

%%%%%%%%%%%%%%%%%%%%%%%%%%%%%%%%%%%%%%%%%%%%%%%%%%%%%%%%%%%%%%%%%%%%%%%%%%%%%%%%
\subsection{Flags}
\label{sec:flags}

The package makes it easy to generate different versions
of the main or child documents.
To this end compilation flags can be defined
and assigned different default values.
They will be particularly useful in conjunction
with the forwarding mechanism described in \secref{sec:forward}.

For example, it may be useful to have a flag |\version|
which can be set to |draft| or |final|.
The document source will contain some conditional code
depending on the value of |\version|.
Suppose further, the flag should default to |final| for the main file
and to |draft| for child files
which is a natural assignment for editing the document.
This is achieved by placing the following code
in the preamble of the main document
(below the |\childdocmain| directive):
%
\begin{center}
\begin{tabular}{l}
|\ifchilddoc|\\
|\providecommand{\version}{draft}|\\
|\||else|\\
|\providecommand{\version}{final}|\\
|\||fi|
\end{tabular}
\end{center}
%
The definition by |\providecommand| makes sure
that previous definitions are not overwritten.
Further statements |\providecommand{\version}{...}|
can thus be added before the above code to override it.

For the main file, one might add a line
(between |\childdocmain| and the above block)
%
\begin{center}
|%\ifchilddoc\||else\providecommand{\version}{draft}\||fi|
\end{center}
%
which can be uncommented to produce a draft version.
Likewise one can add a line to the very top of a child file
(above the |\childdocof{|\textit{main}|}| directive)
%
\begin{center}
|%\providecommand{\version}{final}|
\end{center}
%
which can be uncommented to produce the final version of this child document.

%%%%%%%%%%%%%%%%%%%%%%%%%%%%%%%%%%%%%%%%%%%%%%%%%%%%%%%%%%%%%%%%%%%%%%%%%%%%%%%%
\subsection{Forwarding}
\label{sec:forward}

Different versions of the main or child documents
using compilation flags as described in \secref{sec:flags}
can be (permanently) stored in different files
for convenient compilation, viewing and distribution.
To this end, the package defines a command
to pass on compilation to a different file:

%%%%%%%%%%%%%%%%%%%%%%%%%%%%%%%%%%%%%%%%
\DescribeMacro{\childdocforward}
The command |\childdocforward| redirects processing to
another source file:
%
\begin{center}
\begin{tabular}{l}
|\input{childdoc.def}|\\
|\childdocforward[|\textit{main}|]{|\textit{dest}|}|\\
\end{tabular}
\end{center}
%
The argument \textit{dest} is the destination file
(without extension).
It should be the main file or one of the child files.
Note that further \textsf{childdoc} directives
such as |\childdocof| and |\childdocforward|
in the indicated file will be processed in this form.
The optional argument \textit{main}
passes on directly to the main file \textit{main}
while pretending to compile the child \textit{dest}.
This form behaves as if \textit{dest}
issues |\childdocof{|\textit{main}|}| right away,
and no further \textsf{childdoc} directives will be processed.

%%%%%%%%%%%%%%%%%%%%%%%%%%%%%%%%%%%%%%%%
\DescribeMacro{\...prefix}
In the alternative form |\childdocforwardprefix|,
%
\begin{center}
\begin{tabular}{l}
|\input{childdoc.def}|\\
|\childdocforwardprefix[|\textit{main}|]{|\textit{prefix}|}{|\textit{dest}|}|
\end{tabular}
\end{center}
%
the destination file is determined by a pattern
depending on the current file:
To make this work, the current file must be called
`{\textit{prefix}\hspace{0.2em}\textit{suffix}}'
with \textit{prefix} matching precisely the argument.
Processing is then passed on to the file
`{\textit{dest}\hspace{0.2em}\textit{suffix}}'.
Surely, the same effect is achieved by
directly specifying the
argument `{\textit{dest}\hspace{0.2em}\textit{suffix}}'
in the first form.
However, that requires to set up a different file
for each child. With the alternative form of the command
all these files can have exactly the same content
which simplifies setting them up and maintaining them.

For example, the following file |draft.tex|
with a compilation flag |\version| as described in \secref{sec:flags}
compiles the main document as a draft:
%
\begin{center}
\begin{tabular}{l}
|\def\version{draft}|\\
|\input{childdoc.def}|\\
|\childdocforward{|\textit{main}|}|
\end{tabular}
\end{center}
%
Likewise, the following files |final|\textit{nn}|.tex|
compile the final version of the child document
|child|\textit{nn}|.tex|:
%
\begin{center}
\begin{tabular}{l}
|\def\version{final}|\\
|\input{childdoc.def}|\\
|\childdocforwardprefix{final}{child}|
\end{tabular}
\end{center}
%

Note that when several versions of a main file and/or of each child file
are to be generated, it may be convenient to set up a |Makefile| or
shell script to automatise the process.

%%%%%%%%%%%%%%%%%%%%%%%%%%%%%%%%%%%%%%%%%%%%%%%%%%%%%%%%%%%%%%%%%%%%%%%%%%%%%%%%
\subsection{Command Line Processing}
\label{sec:commandline}

The effect of redirection files can also be achieved by invoking
the \LaTeX{} compiler with a more elaborate command line.
Most conveniently this should be done as part
of a shell script or a |Makefile|.

When using \textsf{childdoc} in the main file, the following
command lines effectively perform a redirection
(note that depending on the shell being used,
backslashes may have to be doubled: `|\|' $\to$ `|\\|'):
%
\begin{center}
|... -jobname "|\textit{target}|" |\\|"|[\textit{flags}]%
|\input{childdoc.def}\childdocforward[|\textit{main}|]{|\textit{dest}|}"|
\end{center}
%
Here \textit{target} is the name of the output file,
\textit{main} is the name of the main file
and \textit{dest} is the name of the main or child file to be processed
(all filenames without extensions).
The optional argument \textit{main} can be omitted
if \textit{main} matches \textit{dest}.
Optionally, compilation \textit{flags} can be defined via |\def| commands.
This command line makes the \TeX{} engine believe
it is compiling the file \textit{target}
whose content is specified as the latter parameter.
The provided code then forwards the processing to
\textit{main} or \textit{dest} as described in \secref{sec:forward}.

%%%%%%%%%%%%%%%%%%%%%%%%%%%%%%%%%%%%%%%%%%%%%%%%%%%%%%%%%%%%%%%%%%%%%%%%%%%%%%%%
\subsection{Include by Input}
\label{sec:input}

Including child documents by |\include| has some restrictions by design.
Most notably, the content of a child document always occupies
its own set of pages; pages cannot be shared between child documents.
Usually, this behaviour makes perfect sense
because each child document contain an essential part of the document.
However, in some situations it may be desirable to compose
a document from a collection of parts
without having mandatory page breaks between then.
For this case, the package
provides a mechanism to include parts
by |\input| which can also be processed individually.
However, by construction this mechanism
requires manual handling of the content to be output.

%%%%%%%%%%%%%%%%%%%%%%%%%%%%%%%%%%%%%%%%
\DescribeMacro{\ifchilddocmanual}
The main file should be prepared as usual, see \secref{sec:include}.
However, the document body must make a distinction
between processing of an individual part and of the main document, e.g.:
%
\begin{center}
\begin{tabular}{l}
|\ifchilddocmanual|\\
|\input{\childdocname}|\\
|\||else|\\
\textit{document body with }|\input{|\textit{part}|}|\\
|\||fi|
\end{tabular}
\end{center}
%
The conditional |\ifchilddocmanual| is true whenever
a part to be included by |\input| is being compiled,
and the name of the part is stored in |\childdocname|.

%%%%%%%%%%%%%%%%%%%%%%%%%%%%%%%%%%%%%%%%
\DescribeMacro{\childdocby}
Each part to be included by |\input| should start with:
%
\begin{center}
\begin{tabular}{l}
|\input{childdoc.def}|\\
|\childdocby{|\textit{main}|}|\\
\end{tabular}
\end{center}
%
The directive |\childdocby| is similar to |\childdocof|
described in \secref{sec:include},
but the subsequent selection of content must be done manually.
To that end, both |\ifchilddoc| and |\ifchilddocmanual|
will be true upon processing of a part,
and the name of the part is stored in |\childdocname|.
Note that |\jobname| will be set to the filename of the current part
so that each part receives an individual |.aux| file
that does not interfere with the |.aux| file(s) of the main document.
This behaviour can be altered by the alternative form
|\childdocby[*]{|\textit{main}|}| (with a non-empty optional argument)
which uses the |.aux| file of the main document
by setting |\jobname| to \textit{main}.

%%%%%%%%%%%%%%%%%%%%%%%%%%%%%%%%%%%%%%%%%%%%%%%%%%%%%%%%%%%%%%%%%%%%%%%%%%%%%%%%
\subsection{Driver Development}
\label{sec:driver}

The \textsf{childdoc} mechanism can also be use for the development
of definition files such as \LaTeX{} styles or classes.
This case differs from the above setup with multiple parts
included by |\include| in that no |\includeonly| should be invoked.
This can be achieved by starting the include file
(before |\ProvidesPackage|) with:
%
\begin{center}
\begin{tabular}{l}
|\input{childdoc.def}|\\
|\childdocforward{|\textit{main}|}|\\
\end{tabular}
\end{center}
%
or alternatively with:
%
\begin{center}
\begin{tabular}{l}
|\input{childdoc.def}|\\
|\childdocby{|\textit{main}|}|\\
\end{tabular}
\end{center}
%
Both forms have slightly different effects as described above.
The main file is prepared as usual, see \secref{sec:include}.

%%%%%%%%%%%%%%%%%%%%%%%%%%%%%%%%%%%%%%%%%%%%%%%%%%%%%%%%%%%%%%%%%%%%%%%%%%%%%%%%
\subsection{Legacy Detection}
\label{sec:detection}

The directive |\childdocmain| in the main file can detect
whether the complete document or merely a child is to be compiled
even without using the directive |\childdocof|.
This method is deprecated because it is less robust
and there is no compelling reason to use it;
it is merely provided for backward compatibility
and it may be removed in future versions.

If the detection mechanism is to be used,
it is mandatory to correctly specify
the filename of the main file as the argument of |\childdocmain|:
%
\begin{center}
\begin{tabular}{l}
|\input{childdoc.def}|\\
|\childdocmain{|\textit{main}|}|\\
\end{tabular}
\end{center}
%
If |\jobname| does not match the argument \textit{main} of |\childdocmain|,
it is assumed that |\jobname| points to the child file to be compiled.
When using |\childdocmain| with the main file specified as argument,
it suffices to start a child file
with just |\input{|\textit{main}|}|
without loading of the package and using |\childdocof|.
If instead all processing is done
with the appropriate \textsf{childdoc} directives,
the argument of \textit{main} of |\childdocmain| can be empty.

An alternative version of the command line processing described
in \secref{sec:commandline} using the detection mechanism reads:
%
\begin{center}
|... -jobname "|\textit{target}|" "|[\textit{flags}]%
[|\def\jobname{|\textit{dest}|}|]|\input{|\textit{main}|}"|
\end{center}

%%%%%%%%%%%%%%%%%%%%%%%%%%%%%%%%%%%%%%%%%%%%%%%%%%%%%%%%%%%%%%%%%%%%%%%%%%%%%%%%
\subsection{Manual Code}
\label{sec:manual}

In case one cannot be certain whether the definitions file |childdoc.def|
is installed on the target \TeX{} distribution
and one prefers not to ship it,
it is conceivable to paste a few relevant commands into the sources.

To that end, drop all statements |\input{childdoc.def}|
and perform the replacements as outlined below.
Instead of |\childdocmain{|\textit{main}|}| add the following code
to the top of the main file:
%
\begin{center}
\begin{tabular}{l}
|\||ifdefined\childdocname\endinput\||fi\newif\ifchilddoc|\\
|\edef\childdocname{\scantokens\expandafter{\jobname\noexpand}}|\\
|\def\childdocmain{|\textit{main}|}\||ifx\childdocmain\childdocname\||else|\\
|\childdoctrue\includeonly{\childdocname}\let\jobname\childdocmain\||fi|\\
\end{tabular}
\end{center}
%
Instead of |\childdocof{|\textit{main}|}| just include the main file
at the top of each child file:
%
\begin{center}
|\input{|\textit{main}|}|
\end{center}
%
A simple redirection |\childdocforward{|\textit{dest}|}| is achieved by:
%
\begin{center}
|\def\jobname{|\textit{dest}|}\input{\jobname}|
\end{center}
%
The redirection with prefix
|\childdocforwardprefix[|\textit{prefix}|]{|\textit{dest}|}|
is accomplished by:
%
\begin{center}
\begin{tabular}{l}
|{\edef\jobname{\scantokens\expandafter{\jobname\noexpand}}|\\
|\def\redirectjob |\textit{prefix}|#1~~~{\gdef\jobname{|\textit{dest}|#1}}|\\
|\expandafter\redirectjob\jobname~~~}\input{\jobname}|
\end{tabular}
\end{center}

In an alternative approach,
child documents can be compiled by a specific command line
without additional code or specific definitions:
%
\begin{center}
|... -jobname "|\textit{target}|" "|[\textit{flags}]%
|\includeonly{|\textit{dest}|}\input{|\textit{main}|}"|
\end{center}
%

%%%%%%%%%%%%%%%%%%%%%%%%%%%%%%%%%%%%%%%%%%%%%%%%%%%%%%%%%%%%%%%%%%%%%%%%%%%%%%%%
%%%%%%%%%%%%%%%%%%%%%%%%%%%%%%%%%%%%%%%%%%%%%%%%%%%%%%%%%%%%%%%%%%%%%%%%%%%%%%%%
\section{Information}

%%%%%%%%%%%%%%%%%%%%%%%%%%%%%%%%%%%%%%%%%%%%%%%%%%%%%%%%%%%%%%%%%%%%%%%%%%%%%%%%
\subsection{Copyright}

Copyright \copyright{} 2017--2018 Niklas Beisert

This work may be distributed and/or modified under the
conditions of the \LaTeX{} Project Public License, either version 1.3
of this license or (at your option) any later version.
The latest version of this license is in
  \url{http://www.latex-project.org/lppl.txt}
and version 1.3 or later is part of all distributions of \LaTeX{}
version 2005/12/01 or later.

This work has the LPPL maintenance status `maintained'.

The Current Maintainer of this work is Niklas Beisert.

This work consists of the files |README.txt|, |childdoc.ins| and |childdoc.dtx|
as well as the derived files |childdoc.def|, |cdocsamp.tex|
with |cdocsch1.tex|, |cdocsch2.tex|, |cdocspt3.tex|, |cdocspt4.tex|,
|cdocsdrf.tex|, |cdocsfn1.tex|, |cdocsfn2.tex|
as well as |childdoc.pdf|.

%%%%%%%%%%%%%%%%%%%%%%%%%%%%%%%%%%%%%%%%%%%%%%%%%%%%%%%%%%%%%%%%%%%%%%%%%%%%%%%%
\subsection{Files and Installation}

The package consists of the files:
%
\begin{center}
\begin{tabular}{ll}
    |README.txt|   & readme file \\
    |childdoc.ins| & installation file \\
    |childdoc.dtx| & source file \\
    |childdoc.def| & definition file \\
    |cdocsamp.tex| & sample main file \\
    |cdocsch1.tex| & sample include file \\
    |cdocsch2.tex| & sample include file \\
    |cdocspt3.tex| & sample part file \\
    |cdocspt4.tex| & sample part file \\
    |cdocsdrf.tex| & sample redirection file \\
    |cdocsfn1.tex| & sample redirection file \\
    |cdocsfn2.tex| & sample redirection file \\
    |childdoc.pdf| & manual
\end{tabular}
\end{center}
%
The distribution consists of the files
|README.txt|, |childdoc.ins| and |childdoc.dtx|.
%
\begin{itemize}
\item
Run (pdf)\LaTeX{} on |childdoc.dtx|
to compile the manual |childdoc.pdf| (this file).
\item
Run \LaTeX{} on |childdoc.ins| to create the definitions file |childdoc.def|
and the sample |cdocsamp.tex| with include files
|cdocsch1.tex|, |cdocsch2.tex|, |cdocspt3.tex|, |cdocspt4.tex|,
|cdocsdrf.tex|, |cdocsfn1.tex|, |cdocsfn2.tex|.
Then copy the file |childdoc.def| to an appropriate directory of your \LaTeX{}
distribution, e.g.\ \textit{texmf-root}|/tex/latex/childdoc|.
\end{itemize}

%%%%%%%%%%%%%%%%%%%%%%%%%%%%%%%%%%%%%%%%%%%%%%%%%%%%%%%%%%%%%%%%%%%%%%%%%%%%%%%%
\subsection{Related CTAN Packages}

There are several other packages which offer a similar functionality:
%
\begin{itemize}
\item
The packages
\href{http://ctan.org/pkg/docmute}{\textsf{docmute}},
\href{http://ctan.org/pkg/includex}{\textsf{includex}} and
\href{http://ctan.org/pkg/standalone}{\textsf{standalone}}
provide commands to include only the document body of
a child file thus allowing both files to be compiled individually.
\item
The packages \href{http://ctan.org/pkg/subdocs}{\textsf{subdocs}}
and \href{http://ctan.org/pkg/subfiles}{\textsf{subfiles}}
provide structures in which the main and child documents can be
encapsulated and allowing them to be compiled individually.
The inclusion mechanism is different from the conventional |\include|.
\item
The package \href{http://ctan.org/pkg/combine}{\textsf{combine}}
is an elaborate solution to combine several documents into one.
\end{itemize}
%
See also the CTAN topic \href{http://ctan.org/topic/subdocs}{\textsf{subdocs}}
for further related packages.
The present package differs from the above solutions in that
a document structure constructed with the conventional |\include| mechanism
just needs two extra commands at the top of every file
such that all constituent files can be compiled individually.

%%%%%%%%%%%%%%%%%%%%%%%%%%%%%%%%%%%%%%%%%%%%%%%%%%%%%%%%%%%%%%%%%%%%%%%%%%%%%%%%
%\subsection{Feature Suggestions}
%
%The following is a list of features which may be useful for future
%versions of this package:
%%
%\begin{itemize}
%\item
%\ldots
%\end{itemize}

%%%%%%%%%%%%%%%%%%%%%%%%%%%%%%%%%%%%%%%%%%%%%%%%%%%%%%%%%%%%%%%%%%%%%%%%%%%%%%%%
\subsection{Revision History}

%%%%%%%%%%%%%%%%%%%%%%%%%%%%%%%%%%%%%%%%
\paragraph{v2.0:} 2018/12/30

\begin{itemize}
\item
immediate forward processing
\item
added |\childdocby| mechanism
\item
manual restructured
\end{itemize}

%%%%%%%%%%%%%%%%%%%%%%%%%%%%%%%%%%%%%%%%
\paragraph{v1.6:} 2018/01/17

\begin{itemize}
\item
application for development of include files
\item
corrections to manual
\end{itemize}

%%%%%%%%%%%%%%%%%%%%%%%%%%%%%%%%%%%%%%%%
\paragraph{v1.5:} 2017/05/21

\begin{itemize}
\item
more complete structuring introduced
\item
|\childdocof| introduced
\item
|\childdoc| renamed to |\childdocmain|
\item
|\childredirect| renamed to |\childdocforward| and |\childdocforwardprefix|
and functionality expanded
\end{itemize}

%%%%%%%%%%%%%%%%%%%%%%%%%%%%%%%%%%%%%%%%
\paragraph{v1.0:} 2017/04/27

\begin{itemize}
\item
manual and install package
\item
first version published on CTAN
\end{itemize}

%%%%%%%%%%%%%%%%%%%%%%%%%%%%%%%%%%%%%%%%
\paragraph{v0.6:} 2017/04/26

\begin{itemize}
\item
redirection mechanism added
\end{itemize}

%%%%%%%%%%%%%%%%%%%%%%%%%%%%%%%%%%%%%%%%
\paragraph{v0.5:} 2017/04/26

\begin{itemize}
\item
functionality in definition file
\end{itemize}


%%%%%%%%%%%%%%%%%%%%%%%%%%%%%%%%%%%%%%%%%%%%%%%%%%%%%%%%%%%%%%%%%%%%%%%%%%%%%%%%
%%%%%%%%%%%%%%%%%%%%%%%%%%%%%%%%%%%%%%%%%%%%%%%%%%%%%%%%%%%%%%%%%%%%%%%%%%%%%%%%
%%%%%%%%%%%%%%%%%%%%%%%%%%%%%%%%%%%%%%%%%%%%%%%%%%%%%%%%%%%%%%%%%%%%%%%%%%%%%%%%
\appendix

\settowidth\MacroIndent{\rmfamily\scriptsize 000\ }

 \DocInput{childdoc.dtx}

\end{document}
%</driver>
% \fi
%
% %%%%%%%%%%%%%%%%%%%%%%%%%%%%%%%%%%%%%%%%%%%%%%%%%%%%%%%%%%%%%%%%%%%%%%%%%%%%%%
% %%%%%%%%%%%%%%%%%%%%%%%%%%%%%%%%%%%%%%%%%%%%%%%%%%%%%%%%%%%%%%%%%%%%%%%%%%%%%%
% \section{Sample}
%\iffalse
%<*samplemain>
%\fi
%
% The following presents a sample document
% with two chapters, two parts, a title page,
% a compile flag as well as three forwarding files to set the flag.
% It consists of eight |.tex| files:
% \begin{center}
% \begin{tabular}{ll}
% |cdocsamp.tex|&main file\\
% |cdocsch1.tex|&include file for chapter 1\\
% |cdocsch2.tex|&include file for chapter 2\\
% |cdocspt3.tex|&include file for part 3\\
% |cdocspt4.tex|&include file for part 4\\
% |cdocsdrf.tex|&forwarding file for main file in draft mode\\
% |cdocsfi1.tex|&forwarding file for final version of chapter 1\\
% |cdocsfi2.tex|&forwarding file for final version of chapter 2\\
% \end{tabular}
% \end{center}
% Each of the eight files can be compiled directly by the \LaTeX{} compiler.
%
% %%%%%%%%%%%%%%%%%%%%%%%%%%%%%%%%%%%%%%
% \paragraph{Main File.}
%
% The main file is called |cdocsamp.tex|.
%
% Load the \textsf{childdoc} definitions and
% declare the filename for the main document:
%    \begin{macrocode}
\input{childdoc.def}
\childdocmain{}
%    \end{macrocode}

% Optional override for |\version| flag:
%    \begin{macrocode}
%%\ifchilddoc\else\providecommand{\version}{draft}\fi
%    \end{macrocode}

% Define the default values for the |\version| flag
% (|final| for the main file and |draft| for childs):
%    \begin{macrocode}
\ifchilddoc
\providecommand{\version}{draft}
\else
\providecommand{\version}{final}
\fi
%    \end{macrocode}

% Load the standard document class:
%    \begin{macrocode}
\documentclass[12pt]{article}
%    \end{macrocode}

% Start the document body:
%    \begin{macrocode}
\begin{document}
%    \end{macrocode}

% Declare a title page.
% Print title, part of document being processed and version flag:
%    \begin{macrocode}
\addtocounter{page}{-1}
\begin{center}
{\LARGE\bfseries{}childdoc example\par}
\vspace{1cm}
\ifchilddoc
\ifchilddocmanual part\else chapter\fi:
`\childdocname' of `\childdocjob'\par
\else
main document: `\childdocjob'\par
\fi
version: \version\par
\end{center}
\newpage
%    \end{macrocode}

% Manually include selected file,
% otherwise process as usual:
%    \begin{macrocode}
\ifchilddocmanual
\section*{part `\childdocname'}
\input{\childdocname}
\else
%    \end{macrocode}

% Include the two chapters:
%    \begin{macrocode}
\include{cdocsch1}
\include{cdocsch2}
%    \end{macrocode}

% Include the two parts unless only chapters should be displayed:
%    \begin{macrocode}
\ifchilddoc\else
\section{part three}
\input{cdocspt3}
\section{part four}
\input{cdocspt4}
\fi
%    \end{macrocode}

% Process as usual until here:
%    \begin{macrocode}
\fi
%    \end{macrocode}

% End of document body:
%    \begin{macrocode}
\end{document}
%    \end{macrocode}
%\iffalse
%</samplemain>
%\fi
%
% %%%%%%%%%%%%%%%%%%%%%%%%%%%%%%%%%%%%%%
% \paragraph{Chapter Include Files.}
%
% The include files are called |cdocsch1.tex| and |cdocsch2.tex|.
%
%\iffalse
%<*samplechap1|samplechap2>
%\fi

% Optional override for |\version| flag:
%    \begin{macrocode}
%%\providecommand{\version}{final}
%    \end{macrocode}

% Include the main document:
%    \begin{macrocode}
\input{childdoc.def}
\childdocof{cdocsamp}
%    \end{macrocode}

%\iffalse
%</samplechap1|samplechap2>
%\fi
%
%\iffalse
%<*samplechap1>
%\fi
% Some text for chapter 1:
%    \begin{macrocode}
\section{one}
some text in chapter one
%    \end{macrocode}

%\iffalse
%</samplechap1>
%\fi
% Some text for chapter 2:
%\iffalse
%<*samplechap2>
%\fi
%    \begin{macrocode}
\section{two}
more text in chapter two
%    \end{macrocode}

%\iffalse
%</samplechap2>
%\fi
%
% %%%%%%%%%%%%%%%%%%%%%%%%%%%%%%%%%%%%%%
% \paragraph{Part Include Files.}
%
% The include files are called |cdocspt3.tex| and |cdocspt4.tex|.
%
%\iffalse
%<*samplepart3|samplepart4>
%\fi

% Optional override for |\version| flag:
%    \begin{macrocode}
%%\providecommand{\version}{final}
%    \end{macrocode}

% Include the main document:
%    \begin{macrocode}
\input{childdoc.def}
\childdocby{cdocsamp}
%    \end{macrocode}

%\iffalse
%</samplepart3|samplepart4>
%\fi
%
%\iffalse
%<*samplepart3>
%\fi
% Some text for part 3:
%    \begin{macrocode}
some text in part three
%    \end{macrocode}

%\iffalse
%</samplepart3>
%\fi
% Some text for part 4:
%\iffalse
%<*samplepart4>
%\fi
%    \begin{macrocode}
more text in part four
%    \end{macrocode}

%\iffalse
%</samplepart4>
%\fi
%
% %%%%%%%%%%%%%%%%%%%%%%%%%%%%%%%%%%%%%%
% \paragraph{Forwarding for a Complete Draft.}
%
% The following forwarding file |cdocsdrf.tex|
% compiles the main document in draft mode:
%\iffalse
%<*sampledraft>
%\fi
%    \begin{macrocode}
\def\version{draft}
\input{childdoc.def}
\childdocforward{cdocsamp}
%    \end{macrocode}

%\iffalse
%</sampledraft>
%\fi
%
% %%%%%%%%%%%%%%%%%%%%%%%%%%%%%%%%%%%%%%
% \paragraph{Forwarding for Final Version of the Chapters.}
%
% The following forwarding files |cdocsfn1.tex| and |cdocsfn2.tex|
% (with identical content)
% compile the final versions of the child documents
% |cdocsch1.tex| and |cdocsch2.tex|, respectively:
%\iffalse
%<*samplefinal>
%\fi
%    \begin{macrocode}
\def\version{final}
\input{childdoc.def}
\childdocforwardprefix[cdocsamp]{cdocsfn}{cdocsch}
%    \end{macrocode}

%\iffalse
%</samplefinal>
%\fi
%
% %%%%%%%%%%%%%%%%%%%%%%%%%%%%%%%%%%%%%%
% \paragraph{Command Line Processing.}
%
% The following three command lines generate the output files
% |cdocscld|, |cdocscl1| and |cdocscl2|
% which should be identical to
% |cdocsdrf|, |cdocsch1| and |cdocsfn2|, respectively:
% \begin{center}
% \begin{tabular}{l}
% |latex -jobname cdocscld \|\\
% |  "\def\version{draft}\input{childdoc.def}\childdocforward{cdocsamp}"|\\
% |latex -jobname cdocscl1 \|\\
% |  "\input{childdoc.def}\childdocforward[cdocsamp]{cdocsch1}"|\\
% |latex -jobname cdocscl2 \|\\
% |  "\def\version{final}\input{childdoc.def}\childdocforward{cdocsch2}"|
% \end{tabular}
% \end{center}
% Note that the trailing backslash on each first line
% merely continues the input to the second line
% (for convenient cut ant paste).
% Furthermore, the command |latex| can be replaced by any
% of its alternative versions such as |pdflatex|.
%
% %%%%%%%%%%%%%%%%%%%%%%%%%%%%%%%%%%%%%%%%%%%%%%%%%%%%%%%%%%%%%%%%%%%%%%%%%%%%%%
% %%%%%%%%%%%%%%%%%%%%%%%%%%%%%%%%%%%%%%%%%%%%%%%%%%%%%%%%%%%%%%%%%%%%%%%%%%%%%%
% \section{Implementation}
%\iffalse
%<*package>
%\fi
%
% This section describes the definitions file |childdoc.def|.

% The definitions cannot be loaded using |\usepackage| or |\RequirePackage|
% which has a mechanism to prevent loading a style file more than once.
% When loading the definitions by means of |\input|
% multiple instances have to be prevented manually:
%\iffalse
%This code needs to be before the `\ProvidesFile' directive
%which is defined at the beginning of this file.
%Therefore it is also placed there and commented out here.
%</package>
%<*discard>
%\fi
%    \begin{macrocode}
\ifdefined\childdocmain\endinput\fi
%    \end{macrocode}
%\iffalse
%</discard>
%<*package>
%\fi
%
% \macro{\ifchilddoc}
% \macro{\ifchilddocmanual}
% The conditional |\ifchilddoc| tells whether a
% child (true) or main (false) document is being compiled.
% The conditional |\ifchilddocmanual| tells whether
% the |\includeonly| mechanism is used (false) or
% the selection of child files must be performed manually (true).
% The definitions initialise to false:
%    \begin{macrocode}
\newif\ifchilddoc
\newif\ifchilddocmanual
%    \end{macrocode}

% \macro{\childdocname}
% \macro{\childdocjob}
% The macro |\childdocname| stores the name of the main document
% to be compiled. The macro |\childdocjob| stores the name of
% the document on which the \LaTeX{} compiler was originally invoked.
% The content of |\jobname| cannot be compared
% to filenames specified in the source due to different catcodes.
% The following code rescans |\jobname|, stores the result
% in |\childdocname| and saves a copy in |\childdocjob|:
%    \begin{macrocode}
\edef\childdocname{\scantokens\expandafter{\jobname\noexpand}}
\let\childdocjob\childdocname
%    \end{macrocode}

% \macro{\childdocdisable}
% The macro |\childdocdisable| prevents the main file
% from being processed more than once.
% At this stage, the main document command |\childdocmain|
% is assumed to be called once again where it should do nothing.
% Any subsequent call to it should prevent
% a secondary processing of the main document
% It overwrites the forwarding commands
% |\childdocof| and |\childdocforward|
% with empty macros to prevent further inclusions of the main document:
%    \begin{macrocode}
\newcommand{\childdocdisable}
{
  \renewcommand{\childdocmain}[1]{\renewcommand{\childdocmain}[1]{\endinput}}
  \renewcommand{\childdocof}[1]{}
  \renewcommand{\childdocby}[2][]{}
  \renewcommand{\childdocforward}[2][]{}
  \renewcommand{\childdocdisable}{}
}
%    \end{macrocode}

% \macro{\childdocmain}
% The macro |\childdocmain| is to be called at the top of the main file
% with nothing or the main filename (without extension) as argument.
% First, it breaks loops.
% If the argument is not empty and does not match |\childdocname|
% (which is set by the first inclusion of |childdoc.def|),
% |\ifchilddoc| is set to true, |\includeonly| is applied to the child file
% and |\jobname| is set to the main file
% (for proper handling of |.aux| files):
%    \begin{macrocode}
\newcommand{\childdocmain}[1]
{
  \childdocdisable\childdocmain{}
  \if?#1?\else
    \begingroup
      \def\childdoctmp{#1}
      \ifx\childdoctmp\childdocname
        \def\childdoctmp{}
      \else
        \def\childdoctmp
        {
          \childdoctrue
          \includeonly{\childdocname}
          \def\childdocjob{#1}
          \def\jobname{#1}
        }
      \fi
      \expandafter
    \endgroup
    \childdoctmp
  \fi
}
%    \end{macrocode}

% \macro{\childdocof}
% The command |\childdocof| redirects
% compilation to the main file |#1|.
%    \begin{macrocode}
\newcommand{\childdocof}[1]
{
  \childdocdisable
  \childdoctrue
  \includeonly{\childdocname}
  \def\jobname{#1}
  \def\childdocjob{#1}
  \input{#1}
}
%    \end{macrocode}

% \macro{\childdocby}
% The command |\childdocby| ....
%    \begin{macrocode}
\newcommand{\childdocby}[2][]
{
  \childdocdisable
  \childdoctrue
  \childdocmanualtrue
  \if?#1?\else
    \def\jobname{#2}
  \fi
  \def\childdocjob{#2}
  \input{#2}
  \endinput
}
%    \end{macrocode}

% \macro{\childdocforward}
% The command |\childdocforward| redirects
% compilation to the main file or
% (if the optional argument is given) a child file.
% Parameters are set as if the main file
% or a child file starting with |\childdocof| was compiled.
% Then compilation is handed over to the main file:
%    \begin{macrocode}
\newcommand{\childdocforward}[2][]
{
  \begingroup
    \if?#1?
      \def\childdoctmp
      {
        \def\childdocname{#2}
        \def\childdocjob{#2}
        \def\jobname{#2}
        \input{#2}
        \endinput
      }
    \else
      \def\childdoctmp
      {
        \childdocdisable
        \def\childdocname{#2}
        \childdoctrue
        \includeonly{#2}
        \def\childdocjob{#1}
        \def\jobname{#1}
        \input{#1}
        \endinput
      }
    \fi
    \expandafter
  \endgroup
  \childdoctmp
}
%    \end{macrocode}

% \macro{\childdocforwardprefix}
% The command |\childdocforwardprefix| redirects
% compilation to the main or a child file by means of a pattern.
% The prefix |#1| in the current filename is replaced by |#2|
% and the suffix of the current filename is kept
% (it is assumed that the filename does not contain the substring `|~~~|'
% which is used as a delimiter).
% Compilation is handed over to the new file by |\childdocforward|:
%    \begin{macrocode}
\newcommand{\childdocforwardprefix}[3][]
{
  \begingroup
    \def\childdocextract #2##1~~~{\def\childdoctmp{\childdocforward[#1]{#3##1}}}
    \expandafter\childdocextract\childdocname~~~
    \expandafter
  \endgroup
  \childdoctmp
}
%    \end{macrocode}

% \macro{\childdoc}
% The deprecated macro |\childdoc| is a legacy version of |\childdocmain|:
%    \begin{macrocode}
\newcommand{\childdoc}{\childdocmain}
%    \end{macrocode}

% \macro{\childdocredirect}
% The deprecated macro |\childdocredirect| is a legacy version
% of |\childdocforward| and |\childdocforwardprefix|:
%    \begin{macrocode}
\newcommand{\childdocredirect}[2][]
{
  \begingroup
    \if?#1?
      \def\childdoctmp{\childdocforward{#2}}
    \else
      \def\childdoctmp{\childdocforwardprefix{#1}{#2}}
    \fi
    \expandafter
  \endgroup
  \childdoctmp
}
%    \end{macrocode}

%\iffalse
%</package>
%\fi
%
\endinput
\childdocforward[|\textit{main}|]{|\textit{dest}|}"|
\end{center}
%
Here \textit{target} is the name of the output file,
\textit{main} is the name of the main file
and \textit{dest} is the name of the main or child file to be processed
(all filenames without extensions).
The optional argument \textit{main} can be omitted
if \textit{main} matches \textit{dest}.
Optionally, compilation \textit{flags} can be defined via |\def| commands.
This command line makes the \TeX{} engine believe
it is compiling the file \textit{target}
whose content is specified as the latter parameter.
The provided code then forwards the processing to
\textit{main} or \textit{dest} as described in \secref{sec:forward}.

%%%%%%%%%%%%%%%%%%%%%%%%%%%%%%%%%%%%%%%%%%%%%%%%%%%%%%%%%%%%%%%%%%%%%%%%%%%%%%%%
\subsection{Include by Input}
\label{sec:input}

Including child documents by |\include| has some restrictions by design.
Most notably, the content of a child document always occupies
its own set of pages; pages cannot be shared between child documents.
Usually, this behaviour makes perfect sense
because each child document contain an essential part of the document.
However, in some situations it may be desirable to compose
a document from a collection of parts
without having mandatory page breaks between then.
For this case, the package
provides a mechanism to include parts
by |\input| which can also be processed individually.
However, by construction this mechanism
requires manual handling of the content to be output.

%%%%%%%%%%%%%%%%%%%%%%%%%%%%%%%%%%%%%%%%
\DescribeMacro{\ifchilddocmanual}
The main file should be prepared as usual, see \secref{sec:include}.
However, the document body must make a distinction
between processing of an individual part and of the main document, e.g.:
%
\begin{center}
\begin{tabular}{l}
|\ifchilddocmanual|\\
|\input{\childdocname}|\\
|\||else|\\
\textit{document body with }|\input{|\textit{part}|}|\\
|\||fi|
\end{tabular}
\end{center}
%
The conditional |\ifchilddocmanual| is true whenever
a part to be included by |\input| is being compiled,
and the name of the part is stored in |\childdocname|.

%%%%%%%%%%%%%%%%%%%%%%%%%%%%%%%%%%%%%%%%
\DescribeMacro{\childdocby}
Each part to be included by |\input| should start with:
%
\begin{center}
\begin{tabular}{l}
|% \iffalse
%
% childdoc.dtx Copyright (C) 2017-2018 Niklas Beisert
%
% This work may be distributed and/or modified under the
% conditions of the LaTeX Project Public License, either version 1.3
% of this license or (at your option) any later version.
% The latest version of this license is in
%   http://www.latex-project.org/lppl.txt
% and version 1.3 or later is part of all distributions of LaTeX
% version 2005/12/01 or later.
%
% This work has the LPPL maintenance status `maintained'.
%
% The Current Maintainer of this work is Niklas Beisert.
%
% This work consists of the files childdoc.dtx and childdoc.ins
% and the derived files childdoc.def and cdocsamp.tex with
% cdocsch1.tex, cdocsch2.tex, cdocsdrf.tex, cdocsfn1.tex, cdocsfn2.tex.
%
%<package>\ifdefined\childdocmain\endinput\fi
%<package>\ProvidesFile{childdoc.def}[2018/12/30 v2.0 child document driver]
%<samplemain>\ProvidesFile{cdocsamp.tex}[2018/12/30 v2.0 sample for childdoc]
%<*driver>
%\ProvidesFile{childdoc.drv}[2018/12/30 v2.0 childdoc reference manual file]
\PassOptionsToClass{10pt,a4paper}{article}
\documentclass{ltxdoc}

\usepackage[margin=35mm]{geometry}
\usepackage{hyperref}
\usepackage{hyperxmp}
\usepackage[usenames]{color}

\hypersetup{colorlinks=true}
\hypersetup{pdfstartview=FitH}
\hypersetup{pdfpagemode=UseNone}
\hypersetup{pdfsource={}}
\hypersetup{pdflang={en-UK}}
\hypersetup{pdfcopyright={Copyright 2017-2018 Niklas Beisert.
  This work may be distributed and/or modified under the
  conditions of the LaTeX Project Public License, either version 1.3
  of this license or (at your option) any later version.}}
\hypersetup{pdflicenseurl={http://www.latex-project.org/lppl.txt}}
\hypersetup{pdfcontactaddress={ETH Zurich, ITP, HIT K,
  Wolfgang-Pauli-Strasse 27}}
\hypersetup{pdfcontactpostcode={8093}}
\hypersetup{pdfcontactcity={Zurich}}
\hypersetup{pdfcontactcountry={Switzerland}}
\hypersetup{pdfcontactemail={nbeisert@itp.phys.ethz.ch}}
\hypersetup{pdfcontacturl={http://people.phys.ethz.ch/\xmptilde nbeisert/}}

\newcommand{\secref}[1]{\hyperref[#1]{section \ref*{#1}}}

\parskip1ex
\parindent0pt
\let\olditemize\itemize
\def\itemize{\olditemize\parskip0pt}

\begin{document}

\title{The \textsf{childdoc} Package}
\hypersetup{pdftitle={The childdoc Package}}
\author{Niklas Beisert\\[2ex]
  Institut f\"ur Theoretische Physik\\
  Eidgen\"ossische Technische Hochschule Z\"urich\\
  Wolfgang-Pauli-Strasse 27, 8093 Z\"urich, Switzerland\\[1ex]
  \href{mailto:nbeisert@itp.phys.ethz.ch}
  {\texttt{nbeisert@itp.phys.ethz.ch}}}
\hypersetup{pdfauthor={Niklas Beisert}}
\hypersetup{pdfsubject={Manual for the LaTeX2e Package childdoc}}
\date{30 December 2018, \textsf{v2.0}}
\maketitle

\begin{abstract}\noindent
\textsf{childdoc} is a \LaTeXe{} package
that enables the direct compilation
of document sections included by |\include|
to individual files.
\end{abstract}

\begingroup
\parskip0ex
\tableofcontents
\endgroup

%%%%%%%%%%%%%%%%%%%%%%%%%%%%%%%%%%%%%%%%%%%%%%%%%%%%%%%%%%%%%%%%%%%%%%%%%%%%%%%%
%%%%%%%%%%%%%%%%%%%%%%%%%%%%%%%%%%%%%%%%%%%%%%%%%%%%%%%%%%%%%%%%%%%%%%%%%%%%%%%%
\section{Introduction}

\LaTeX{} provides a mechanism to structure a large document (such as a book)
into a main file and several child files (containing the chapters)
using the |\include| command.
This mechanism is beneficial for documents
which span hundreds of pages in order to
make the source file(s) more manageable.
Moreover, compilation can be restricted to
selected child files by means of the |\includeonly| command.
The latter feature can be used to reduce the compilation time while editing
(this was significantly more useful in the earlier days of \LaTeX{})
or to generate a smaller document which is easier to navigate.
Another application of |\includeonly| is to generate
documents consisting of selected parts of the complete document.

However, there are a few drawbacks of the plain |\include| mechanism:
\begin{itemize}
\item
The child files cannot be compiled on their own,
they can only be compiled via the main file.
A naive editing environment
(such as a text editor with an option
to have the current file processed by \LaTeX)
may require one to switch to the main file before compiling;
attempting to compile the child file produces errors.
\item
The main file must be modified (each time)
to adjust the |\includeonly| command
to the present needs. This easily leaves the main file in a messy state.
\item
The generated document will always carry the filename
of the main document. This is inconvenient if
several child files are to be compiled and
to be kept for distribution.
\end{itemize}

The present package provides a simple interface
to make child files individually compilable by \LaTeX{}.
Compiling a child file then has the same effect as compiling
the main file with an |\includeonly| command
to select the appropriate child.
Moreover the generated document will carry the name of the child
rather than the main file.
This resolves all three above issues.

This feature is meant to make the editing of books,
thesis documents and lecture notes somewhat more convenient.
However, the package can also be used efficiently for
composing a series of documents (such as exercise sheets)
which are typically distributed individually.
It then assists the author in generating the individual documents
(potentially in different versions)
as well as a document containing the collected series.
Another application is in developing style files
or other kinds of included material
where compilation of the style file could redirect
to a sample or test file.

%%%%%%%%%%%%%%%%%%%%%%%%%%%%%%%%%%%%%%%%%%%%%%%%%%%%%%%%%%%%%%%%%%%%%%%%%%%%%%%%
%%%%%%%%%%%%%%%%%%%%%%%%%%%%%%%%%%%%%%%%%%%%%%%%%%%%%%%%%%%%%%%%%%%%%%%%%%%%%%%%
\section{Usage}

First of all, the package \textsf{childdoc} is \emph{not} a standard
\LaTeXe{} |.sty| style file! Therefore it needs to be invoked in
a non-standard way.

%%%%%%%%%%%%%%%%%%%%%%%%%%%%%%%%%%%%%%%%%%%%%%%%%%%%%%%%%%%%%%%%%%%%%%%%%%%%%%%%
\subsection{Included Files}
\label{sec:include}

%%%%%%%%%%%%%%%%%%%%%%%%%%%%%%%%%%%%%%%%
\DescribeMacro{\childdocmain}
To use the package, add the commands
\begin{center}
\begin{tabular}{l}
|\input{childdoc.def}|\\
|\childdocmain{}|\\
\end{tabular}
\end{center}
at the very top of the main \LaTeX{} file,
in particular \emph{before} the |\documentclass| statement!
The argument of |\childdocmain| should be left empty
(but it must be present).

%%%%%%%%%%%%%%%%%%%%%%%%%%%%%%%%%%%%%%%%
\DescribeMacro{\childdocof}
Furthermore, add the commands
\begin{center}
\begin{tabular}{l}
|\input{childdoc.def}|\\
|\childdocof{|\textit{main}|}|\\
\end{tabular}
\end{center}
at the top of every child file \textit{child}
which is included by |\include{|\textit{child}|}|
from within the main file
(or at least for those files to be compiled individually).
The argument \textit{main} must be the filename of the main file.

There are a couple of
considerations in setting up the main and child documents:

%%%%%%%%%%%%%%%%%%%%%%%%%%%%%%%%%%%%%%%%
\paragraph{Restrictions.}

Please note the following restrictions:
\begin{itemize}
\item
|\childdocmain| must be called with one argument \textit{main}
to ensure compatibility with earlier version of the package.
It must either be empty (|\childdocmain{}|)
or precisely match the filename of the main file in which it is specified.
See \secref{sec:detection} for further information.
\item
The filename \textit{main} must be specified without the |.tex| extension.
\item
The filename \textit{main} is case sensitive
(even in case-insensitive file systems)
due to internal string comparison.
\item
The argument \textit{main} should be fully expanded, it cannot be a macro.
\item
Subdirectories and special characters should be avoided in filenames.
\item
The command |\childdocmain{|\textit{main}|}| must be followed by a whitespace.
It should not be followed immediately by another command
or by a comment mark `|%|'.
This is because the \TeX{} parser reads the token immediately following
the argument of |\childdocmain| and puts it
at the beginning of every child section;
however, a white\-space is ignored.
\end{itemize}

%%%%%%%%%%%%%%%%%%%%%%%%%%%%%%%%%%%%%%%%
\paragraph{Content of Main File.}

It is advisable to place all content in the child files included by |\include|.
Any output contained in the main file will appear in all child documents
unless suppressed manually;
it cannot be suppressed automatically by the |\includeonly| directive
and thus should normally be avoided.
A method to include some content in the main file
by means of conditional processing is described in \secref{sec:conditional}.

%%%%%%%%%%%%%%%%%%%%%%%%%%%%%%%%%%%%%%%%
\paragraph{Page Numbering.}

When only a part of the document is compiled,
the appropriate numbering of pages
(as well as other status parameters)
is determined from the |.aux| files.
The latter contain information from previous passes.
However this information needs to propagate through
all intermediate child documents.
Therefore the page numbering in child documents may well
be inconsistent until the complete document is compiled at least once.

A useful (if unconventional) way to always ensure a consistent
page numbering is to restart the numbering in each child document
and denote the pages by `\textit{child}|.|\textit{page}'
where \textit{child} represents the chapter/section number of the child file.
This can be achieved by the command
|\numberwithin{page}{|\textit{child}|}|
of the \textsf{amsmath} package
where \textit{child} can be |chapter| or |section|
depending on the chosen structuring.
Alternatively, one can modify the macro |\thepage| appropriately
and reset the counter |page| at the start of each child file.

%%%%%%%%%%%%%%%%%%%%%%%%%%%%%%%%%%%%%%%%%%%%%%%%%%%%%%%%%%%%%%%%%%%%%%%%%%%%%%%%
\subsection{Conditional Processing}
\label{sec:conditional}

The package provides a mechanism to compile different versions
of a document. To customise the versions further some conditional processing
can come in handy to distinguish which version is being compiled.
The package provides two macros to describe the compilation context:

%%%%%%%%%%%%%%%%%%%%%%%%%%%%%%%%%%%%%%%%
\DescribeMacro{\ifchilddoc}
The conditional |\ifchilddoc| distinguishes between the compilation of
child documents and the main document:
%
\begin{center}
|\ifchilddoc |\textit{child-code}| |[|\||else |\textit{main-code}]| \||fi|
\end{center}

%%%%%%%%%%%%%%%%%%%%%%%%%%%%%%%%%%%%%%%%
\DescribeMacro{\childdocname}
\DescribeMacro{\childdocjob}
The macro |\childdocname| contains the filename (without extension)
of the main or child file being processed.
Note that |\childdocjob| will always contain the name of the main file.

%%%%%%%%%%%%%%%%%%%%%%%%%%%%%%%%%%%%%%%%
\paragraph{Title Page.}

Conditional processing can be used to include a title or banner page
in the main document when proper precautions are taken.
Importantly, the code in the main file should ensure that the page counter
(as well as other status parameters which are stored in the |.aux| files)
takes the same value after the conditional processing.
Otherwise the page numbers may take divergent values
depending on which part is compiled.

For example, a title page could be declared by:
%
\begin{center}
\begin{tabular}{l}
|\ifchilddoc\||else|\\
|\addtocounter{page}{-1}|\\
\textit{code for title page}\\
|\newpage|\\
|\||fi|
\end{tabular}
\end{center}
%
A banner page for the child documents can be generated by:
%
\begin{center}
\begin{tabular}{l}
|\ifchilddoc|\\
|\addtocounter{page}{-1}|\\
\textit{code for banner page}\\
|\newpage|\\
|\||fi|
\end{tabular}
\end{center}
%
Here one could write a message such as:
\begin{center}
|This is the part \childdocname{} of \childdocjob{}.|
\end{center}

%%%%%%%%%%%%%%%%%%%%%%%%%%%%%%%%%%%%%%%%%%%%%%%%%%%%%%%%%%%%%%%%%%%%%%%%%%%%%%%%
\subsection{Flags}
\label{sec:flags}

The package makes it easy to generate different versions
of the main or child documents.
To this end compilation flags can be defined
and assigned different default values.
They will be particularly useful in conjunction
with the forwarding mechanism described in \secref{sec:forward}.

For example, it may be useful to have a flag |\version|
which can be set to |draft| or |final|.
The document source will contain some conditional code
depending on the value of |\version|.
Suppose further, the flag should default to |final| for the main file
and to |draft| for child files
which is a natural assignment for editing the document.
This is achieved by placing the following code
in the preamble of the main document
(below the |\childdocmain| directive):
%
\begin{center}
\begin{tabular}{l}
|\ifchilddoc|\\
|\providecommand{\version}{draft}|\\
|\||else|\\
|\providecommand{\version}{final}|\\
|\||fi|
\end{tabular}
\end{center}
%
The definition by |\providecommand| makes sure
that previous definitions are not overwritten.
Further statements |\providecommand{\version}{...}|
can thus be added before the above code to override it.

For the main file, one might add a line
(between |\childdocmain| and the above block)
%
\begin{center}
|%\ifchilddoc\||else\providecommand{\version}{draft}\||fi|
\end{center}
%
which can be uncommented to produce a draft version.
Likewise one can add a line to the very top of a child file
(above the |\childdocof{|\textit{main}|}| directive)
%
\begin{center}
|%\providecommand{\version}{final}|
\end{center}
%
which can be uncommented to produce the final version of this child document.

%%%%%%%%%%%%%%%%%%%%%%%%%%%%%%%%%%%%%%%%%%%%%%%%%%%%%%%%%%%%%%%%%%%%%%%%%%%%%%%%
\subsection{Forwarding}
\label{sec:forward}

Different versions of the main or child documents
using compilation flags as described in \secref{sec:flags}
can be (permanently) stored in different files
for convenient compilation, viewing and distribution.
To this end, the package defines a command
to pass on compilation to a different file:

%%%%%%%%%%%%%%%%%%%%%%%%%%%%%%%%%%%%%%%%
\DescribeMacro{\childdocforward}
The command |\childdocforward| redirects processing to
another source file:
%
\begin{center}
\begin{tabular}{l}
|\input{childdoc.def}|\\
|\childdocforward[|\textit{main}|]{|\textit{dest}|}|\\
\end{tabular}
\end{center}
%
The argument \textit{dest} is the destination file
(without extension).
It should be the main file or one of the child files.
Note that further \textsf{childdoc} directives
such as |\childdocof| and |\childdocforward|
in the indicated file will be processed in this form.
The optional argument \textit{main}
passes on directly to the main file \textit{main}
while pretending to compile the child \textit{dest}.
This form behaves as if \textit{dest}
issues |\childdocof{|\textit{main}|}| right away,
and no further \textsf{childdoc} directives will be processed.

%%%%%%%%%%%%%%%%%%%%%%%%%%%%%%%%%%%%%%%%
\DescribeMacro{\...prefix}
In the alternative form |\childdocforwardprefix|,
%
\begin{center}
\begin{tabular}{l}
|\input{childdoc.def}|\\
|\childdocforwardprefix[|\textit{main}|]{|\textit{prefix}|}{|\textit{dest}|}|
\end{tabular}
\end{center}
%
the destination file is determined by a pattern
depending on the current file:
To make this work, the current file must be called
`{\textit{prefix}\hspace{0.2em}\textit{suffix}}'
with \textit{prefix} matching precisely the argument.
Processing is then passed on to the file
`{\textit{dest}\hspace{0.2em}\textit{suffix}}'.
Surely, the same effect is achieved by
directly specifying the
argument `{\textit{dest}\hspace{0.2em}\textit{suffix}}'
in the first form.
However, that requires to set up a different file
for each child. With the alternative form of the command
all these files can have exactly the same content
which simplifies setting them up and maintaining them.

For example, the following file |draft.tex|
with a compilation flag |\version| as described in \secref{sec:flags}
compiles the main document as a draft:
%
\begin{center}
\begin{tabular}{l}
|\def\version{draft}|\\
|\input{childdoc.def}|\\
|\childdocforward{|\textit{main}|}|
\end{tabular}
\end{center}
%
Likewise, the following files |final|\textit{nn}|.tex|
compile the final version of the child document
|child|\textit{nn}|.tex|:
%
\begin{center}
\begin{tabular}{l}
|\def\version{final}|\\
|\input{childdoc.def}|\\
|\childdocforwardprefix{final}{child}|
\end{tabular}
\end{center}
%

Note that when several versions of a main file and/or of each child file
are to be generated, it may be convenient to set up a |Makefile| or
shell script to automatise the process.

%%%%%%%%%%%%%%%%%%%%%%%%%%%%%%%%%%%%%%%%%%%%%%%%%%%%%%%%%%%%%%%%%%%%%%%%%%%%%%%%
\subsection{Command Line Processing}
\label{sec:commandline}

The effect of redirection files can also be achieved by invoking
the \LaTeX{} compiler with a more elaborate command line.
Most conveniently this should be done as part
of a shell script or a |Makefile|.

When using \textsf{childdoc} in the main file, the following
command lines effectively perform a redirection
(note that depending on the shell being used,
backslashes may have to be doubled: `|\|' $\to$ `|\\|'):
%
\begin{center}
|... -jobname "|\textit{target}|" |\\|"|[\textit{flags}]%
|\input{childdoc.def}\childdocforward[|\textit{main}|]{|\textit{dest}|}"|
\end{center}
%
Here \textit{target} is the name of the output file,
\textit{main} is the name of the main file
and \textit{dest} is the name of the main or child file to be processed
(all filenames without extensions).
The optional argument \textit{main} can be omitted
if \textit{main} matches \textit{dest}.
Optionally, compilation \textit{flags} can be defined via |\def| commands.
This command line makes the \TeX{} engine believe
it is compiling the file \textit{target}
whose content is specified as the latter parameter.
The provided code then forwards the processing to
\textit{main} or \textit{dest} as described in \secref{sec:forward}.

%%%%%%%%%%%%%%%%%%%%%%%%%%%%%%%%%%%%%%%%%%%%%%%%%%%%%%%%%%%%%%%%%%%%%%%%%%%%%%%%
\subsection{Include by Input}
\label{sec:input}

Including child documents by |\include| has some restrictions by design.
Most notably, the content of a child document always occupies
its own set of pages; pages cannot be shared between child documents.
Usually, this behaviour makes perfect sense
because each child document contain an essential part of the document.
However, in some situations it may be desirable to compose
a document from a collection of parts
without having mandatory page breaks between then.
For this case, the package
provides a mechanism to include parts
by |\input| which can also be processed individually.
However, by construction this mechanism
requires manual handling of the content to be output.

%%%%%%%%%%%%%%%%%%%%%%%%%%%%%%%%%%%%%%%%
\DescribeMacro{\ifchilddocmanual}
The main file should be prepared as usual, see \secref{sec:include}.
However, the document body must make a distinction
between processing of an individual part and of the main document, e.g.:
%
\begin{center}
\begin{tabular}{l}
|\ifchilddocmanual|\\
|\input{\childdocname}|\\
|\||else|\\
\textit{document body with }|\input{|\textit{part}|}|\\
|\||fi|
\end{tabular}
\end{center}
%
The conditional |\ifchilddocmanual| is true whenever
a part to be included by |\input| is being compiled,
and the name of the part is stored in |\childdocname|.

%%%%%%%%%%%%%%%%%%%%%%%%%%%%%%%%%%%%%%%%
\DescribeMacro{\childdocby}
Each part to be included by |\input| should start with:
%
\begin{center}
\begin{tabular}{l}
|\input{childdoc.def}|\\
|\childdocby{|\textit{main}|}|\\
\end{tabular}
\end{center}
%
The directive |\childdocby| is similar to |\childdocof|
described in \secref{sec:include},
but the subsequent selection of content must be done manually.
To that end, both |\ifchilddoc| and |\ifchilddocmanual|
will be true upon processing of a part,
and the name of the part is stored in |\childdocname|.
Note that |\jobname| will be set to the filename of the current part
so that each part receives an individual |.aux| file
that does not interfere with the |.aux| file(s) of the main document.
This behaviour can be altered by the alternative form
|\childdocby[*]{|\textit{main}|}| (with a non-empty optional argument)
which uses the |.aux| file of the main document
by setting |\jobname| to \textit{main}.

%%%%%%%%%%%%%%%%%%%%%%%%%%%%%%%%%%%%%%%%%%%%%%%%%%%%%%%%%%%%%%%%%%%%%%%%%%%%%%%%
\subsection{Driver Development}
\label{sec:driver}

The \textsf{childdoc} mechanism can also be use for the development
of definition files such as \LaTeX{} styles or classes.
This case differs from the above setup with multiple parts
included by |\include| in that no |\includeonly| should be invoked.
This can be achieved by starting the include file
(before |\ProvidesPackage|) with:
%
\begin{center}
\begin{tabular}{l}
|\input{childdoc.def}|\\
|\childdocforward{|\textit{main}|}|\\
\end{tabular}
\end{center}
%
or alternatively with:
%
\begin{center}
\begin{tabular}{l}
|\input{childdoc.def}|\\
|\childdocby{|\textit{main}|}|\\
\end{tabular}
\end{center}
%
Both forms have slightly different effects as described above.
The main file is prepared as usual, see \secref{sec:include}.

%%%%%%%%%%%%%%%%%%%%%%%%%%%%%%%%%%%%%%%%%%%%%%%%%%%%%%%%%%%%%%%%%%%%%%%%%%%%%%%%
\subsection{Legacy Detection}
\label{sec:detection}

The directive |\childdocmain| in the main file can detect
whether the complete document or merely a child is to be compiled
even without using the directive |\childdocof|.
This method is deprecated because it is less robust
and there is no compelling reason to use it;
it is merely provided for backward compatibility
and it may be removed in future versions.

If the detection mechanism is to be used,
it is mandatory to correctly specify
the filename of the main file as the argument of |\childdocmain|:
%
\begin{center}
\begin{tabular}{l}
|\input{childdoc.def}|\\
|\childdocmain{|\textit{main}|}|\\
\end{tabular}
\end{center}
%
If |\jobname| does not match the argument \textit{main} of |\childdocmain|,
it is assumed that |\jobname| points to the child file to be compiled.
When using |\childdocmain| with the main file specified as argument,
it suffices to start a child file
with just |\input{|\textit{main}|}|
without loading of the package and using |\childdocof|.
If instead all processing is done
with the appropriate \textsf{childdoc} directives,
the argument of \textit{main} of |\childdocmain| can be empty.

An alternative version of the command line processing described
in \secref{sec:commandline} using the detection mechanism reads:
%
\begin{center}
|... -jobname "|\textit{target}|" "|[\textit{flags}]%
[|\def\jobname{|\textit{dest}|}|]|\input{|\textit{main}|}"|
\end{center}

%%%%%%%%%%%%%%%%%%%%%%%%%%%%%%%%%%%%%%%%%%%%%%%%%%%%%%%%%%%%%%%%%%%%%%%%%%%%%%%%
\subsection{Manual Code}
\label{sec:manual}

In case one cannot be certain whether the definitions file |childdoc.def|
is installed on the target \TeX{} distribution
and one prefers not to ship it,
it is conceivable to paste a few relevant commands into the sources.

To that end, drop all statements |\input{childdoc.def}|
and perform the replacements as outlined below.
Instead of |\childdocmain{|\textit{main}|}| add the following code
to the top of the main file:
%
\begin{center}
\begin{tabular}{l}
|\||ifdefined\childdocname\endinput\||fi\newif\ifchilddoc|\\
|\edef\childdocname{\scantokens\expandafter{\jobname\noexpand}}|\\
|\def\childdocmain{|\textit{main}|}\||ifx\childdocmain\childdocname\||else|\\
|\childdoctrue\includeonly{\childdocname}\let\jobname\childdocmain\||fi|\\
\end{tabular}
\end{center}
%
Instead of |\childdocof{|\textit{main}|}| just include the main file
at the top of each child file:
%
\begin{center}
|\input{|\textit{main}|}|
\end{center}
%
A simple redirection |\childdocforward{|\textit{dest}|}| is achieved by:
%
\begin{center}
|\def\jobname{|\textit{dest}|}\input{\jobname}|
\end{center}
%
The redirection with prefix
|\childdocforwardprefix[|\textit{prefix}|]{|\textit{dest}|}|
is accomplished by:
%
\begin{center}
\begin{tabular}{l}
|{\edef\jobname{\scantokens\expandafter{\jobname\noexpand}}|\\
|\def\redirectjob |\textit{prefix}|#1~~~{\gdef\jobname{|\textit{dest}|#1}}|\\
|\expandafter\redirectjob\jobname~~~}\input{\jobname}|
\end{tabular}
\end{center}

In an alternative approach,
child documents can be compiled by a specific command line
without additional code or specific definitions:
%
\begin{center}
|... -jobname "|\textit{target}|" "|[\textit{flags}]%
|\includeonly{|\textit{dest}|}\input{|\textit{main}|}"|
\end{center}
%

%%%%%%%%%%%%%%%%%%%%%%%%%%%%%%%%%%%%%%%%%%%%%%%%%%%%%%%%%%%%%%%%%%%%%%%%%%%%%%%%
%%%%%%%%%%%%%%%%%%%%%%%%%%%%%%%%%%%%%%%%%%%%%%%%%%%%%%%%%%%%%%%%%%%%%%%%%%%%%%%%
\section{Information}

%%%%%%%%%%%%%%%%%%%%%%%%%%%%%%%%%%%%%%%%%%%%%%%%%%%%%%%%%%%%%%%%%%%%%%%%%%%%%%%%
\subsection{Copyright}

Copyright \copyright{} 2017--2018 Niklas Beisert

This work may be distributed and/or modified under the
conditions of the \LaTeX{} Project Public License, either version 1.3
of this license or (at your option) any later version.
The latest version of this license is in
  \url{http://www.latex-project.org/lppl.txt}
and version 1.3 or later is part of all distributions of \LaTeX{}
version 2005/12/01 or later.

This work has the LPPL maintenance status `maintained'.

The Current Maintainer of this work is Niklas Beisert.

This work consists of the files |README.txt|, |childdoc.ins| and |childdoc.dtx|
as well as the derived files |childdoc.def|, |cdocsamp.tex|
with |cdocsch1.tex|, |cdocsch2.tex|, |cdocspt3.tex|, |cdocspt4.tex|,
|cdocsdrf.tex|, |cdocsfn1.tex|, |cdocsfn2.tex|
as well as |childdoc.pdf|.

%%%%%%%%%%%%%%%%%%%%%%%%%%%%%%%%%%%%%%%%%%%%%%%%%%%%%%%%%%%%%%%%%%%%%%%%%%%%%%%%
\subsection{Files and Installation}

The package consists of the files:
%
\begin{center}
\begin{tabular}{ll}
    |README.txt|   & readme file \\
    |childdoc.ins| & installation file \\
    |childdoc.dtx| & source file \\
    |childdoc.def| & definition file \\
    |cdocsamp.tex| & sample main file \\
    |cdocsch1.tex| & sample include file \\
    |cdocsch2.tex| & sample include file \\
    |cdocspt3.tex| & sample part file \\
    |cdocspt4.tex| & sample part file \\
    |cdocsdrf.tex| & sample redirection file \\
    |cdocsfn1.tex| & sample redirection file \\
    |cdocsfn2.tex| & sample redirection file \\
    |childdoc.pdf| & manual
\end{tabular}
\end{center}
%
The distribution consists of the files
|README.txt|, |childdoc.ins| and |childdoc.dtx|.
%
\begin{itemize}
\item
Run (pdf)\LaTeX{} on |childdoc.dtx|
to compile the manual |childdoc.pdf| (this file).
\item
Run \LaTeX{} on |childdoc.ins| to create the definitions file |childdoc.def|
and the sample |cdocsamp.tex| with include files
|cdocsch1.tex|, |cdocsch2.tex|, |cdocspt3.tex|, |cdocspt4.tex|,
|cdocsdrf.tex|, |cdocsfn1.tex|, |cdocsfn2.tex|.
Then copy the file |childdoc.def| to an appropriate directory of your \LaTeX{}
distribution, e.g.\ \textit{texmf-root}|/tex/latex/childdoc|.
\end{itemize}

%%%%%%%%%%%%%%%%%%%%%%%%%%%%%%%%%%%%%%%%%%%%%%%%%%%%%%%%%%%%%%%%%%%%%%%%%%%%%%%%
\subsection{Related CTAN Packages}

There are several other packages which offer a similar functionality:
%
\begin{itemize}
\item
The packages
\href{http://ctan.org/pkg/docmute}{\textsf{docmute}},
\href{http://ctan.org/pkg/includex}{\textsf{includex}} and
\href{http://ctan.org/pkg/standalone}{\textsf{standalone}}
provide commands to include only the document body of
a child file thus allowing both files to be compiled individually.
\item
The packages \href{http://ctan.org/pkg/subdocs}{\textsf{subdocs}}
and \href{http://ctan.org/pkg/subfiles}{\textsf{subfiles}}
provide structures in which the main and child documents can be
encapsulated and allowing them to be compiled individually.
The inclusion mechanism is different from the conventional |\include|.
\item
The package \href{http://ctan.org/pkg/combine}{\textsf{combine}}
is an elaborate solution to combine several documents into one.
\end{itemize}
%
See also the CTAN topic \href{http://ctan.org/topic/subdocs}{\textsf{subdocs}}
for further related packages.
The present package differs from the above solutions in that
a document structure constructed with the conventional |\include| mechanism
just needs two extra commands at the top of every file
such that all constituent files can be compiled individually.

%%%%%%%%%%%%%%%%%%%%%%%%%%%%%%%%%%%%%%%%%%%%%%%%%%%%%%%%%%%%%%%%%%%%%%%%%%%%%%%%
%\subsection{Feature Suggestions}
%
%The following is a list of features which may be useful for future
%versions of this package:
%%
%\begin{itemize}
%\item
%\ldots
%\end{itemize}

%%%%%%%%%%%%%%%%%%%%%%%%%%%%%%%%%%%%%%%%%%%%%%%%%%%%%%%%%%%%%%%%%%%%%%%%%%%%%%%%
\subsection{Revision History}

%%%%%%%%%%%%%%%%%%%%%%%%%%%%%%%%%%%%%%%%
\paragraph{v2.0:} 2018/12/30

\begin{itemize}
\item
immediate forward processing
\item
added |\childdocby| mechanism
\item
manual restructured
\end{itemize}

%%%%%%%%%%%%%%%%%%%%%%%%%%%%%%%%%%%%%%%%
\paragraph{v1.6:} 2018/01/17

\begin{itemize}
\item
application for development of include files
\item
corrections to manual
\end{itemize}

%%%%%%%%%%%%%%%%%%%%%%%%%%%%%%%%%%%%%%%%
\paragraph{v1.5:} 2017/05/21

\begin{itemize}
\item
more complete structuring introduced
\item
|\childdocof| introduced
\item
|\childdoc| renamed to |\childdocmain|
\item
|\childredirect| renamed to |\childdocforward| and |\childdocforwardprefix|
and functionality expanded
\end{itemize}

%%%%%%%%%%%%%%%%%%%%%%%%%%%%%%%%%%%%%%%%
\paragraph{v1.0:} 2017/04/27

\begin{itemize}
\item
manual and install package
\item
first version published on CTAN
\end{itemize}

%%%%%%%%%%%%%%%%%%%%%%%%%%%%%%%%%%%%%%%%
\paragraph{v0.6:} 2017/04/26

\begin{itemize}
\item
redirection mechanism added
\end{itemize}

%%%%%%%%%%%%%%%%%%%%%%%%%%%%%%%%%%%%%%%%
\paragraph{v0.5:} 2017/04/26

\begin{itemize}
\item
functionality in definition file
\end{itemize}


%%%%%%%%%%%%%%%%%%%%%%%%%%%%%%%%%%%%%%%%%%%%%%%%%%%%%%%%%%%%%%%%%%%%%%%%%%%%%%%%
%%%%%%%%%%%%%%%%%%%%%%%%%%%%%%%%%%%%%%%%%%%%%%%%%%%%%%%%%%%%%%%%%%%%%%%%%%%%%%%%
%%%%%%%%%%%%%%%%%%%%%%%%%%%%%%%%%%%%%%%%%%%%%%%%%%%%%%%%%%%%%%%%%%%%%%%%%%%%%%%%
\appendix

\settowidth\MacroIndent{\rmfamily\scriptsize 000\ }

 \DocInput{childdoc.dtx}

\end{document}
%</driver>
% \fi
%
% %%%%%%%%%%%%%%%%%%%%%%%%%%%%%%%%%%%%%%%%%%%%%%%%%%%%%%%%%%%%%%%%%%%%%%%%%%%%%%
% %%%%%%%%%%%%%%%%%%%%%%%%%%%%%%%%%%%%%%%%%%%%%%%%%%%%%%%%%%%%%%%%%%%%%%%%%%%%%%
% \section{Sample}
%\iffalse
%<*samplemain>
%\fi
%
% The following presents a sample document
% with two chapters, two parts, a title page,
% a compile flag as well as three forwarding files to set the flag.
% It consists of eight |.tex| files:
% \begin{center}
% \begin{tabular}{ll}
% |cdocsamp.tex|&main file\\
% |cdocsch1.tex|&include file for chapter 1\\
% |cdocsch2.tex|&include file for chapter 2\\
% |cdocspt3.tex|&include file for part 3\\
% |cdocspt4.tex|&include file for part 4\\
% |cdocsdrf.tex|&forwarding file for main file in draft mode\\
% |cdocsfi1.tex|&forwarding file for final version of chapter 1\\
% |cdocsfi2.tex|&forwarding file for final version of chapter 2\\
% \end{tabular}
% \end{center}
% Each of the eight files can be compiled directly by the \LaTeX{} compiler.
%
% %%%%%%%%%%%%%%%%%%%%%%%%%%%%%%%%%%%%%%
% \paragraph{Main File.}
%
% The main file is called |cdocsamp.tex|.
%
% Load the \textsf{childdoc} definitions and
% declare the filename for the main document:
%    \begin{macrocode}
\input{childdoc.def}
\childdocmain{}
%    \end{macrocode}

% Optional override for |\version| flag:
%    \begin{macrocode}
%%\ifchilddoc\else\providecommand{\version}{draft}\fi
%    \end{macrocode}

% Define the default values for the |\version| flag
% (|final| for the main file and |draft| for childs):
%    \begin{macrocode}
\ifchilddoc
\providecommand{\version}{draft}
\else
\providecommand{\version}{final}
\fi
%    \end{macrocode}

% Load the standard document class:
%    \begin{macrocode}
\documentclass[12pt]{article}
%    \end{macrocode}

% Start the document body:
%    \begin{macrocode}
\begin{document}
%    \end{macrocode}

% Declare a title page.
% Print title, part of document being processed and version flag:
%    \begin{macrocode}
\addtocounter{page}{-1}
\begin{center}
{\LARGE\bfseries{}childdoc example\par}
\vspace{1cm}
\ifchilddoc
\ifchilddocmanual part\else chapter\fi:
`\childdocname' of `\childdocjob'\par
\else
main document: `\childdocjob'\par
\fi
version: \version\par
\end{center}
\newpage
%    \end{macrocode}

% Manually include selected file,
% otherwise process as usual:
%    \begin{macrocode}
\ifchilddocmanual
\section*{part `\childdocname'}
\input{\childdocname}
\else
%    \end{macrocode}

% Include the two chapters:
%    \begin{macrocode}
\include{cdocsch1}
\include{cdocsch2}
%    \end{macrocode}

% Include the two parts unless only chapters should be displayed:
%    \begin{macrocode}
\ifchilddoc\else
\section{part three}
\input{cdocspt3}
\section{part four}
\input{cdocspt4}
\fi
%    \end{macrocode}

% Process as usual until here:
%    \begin{macrocode}
\fi
%    \end{macrocode}

% End of document body:
%    \begin{macrocode}
\end{document}
%    \end{macrocode}
%\iffalse
%</samplemain>
%\fi
%
% %%%%%%%%%%%%%%%%%%%%%%%%%%%%%%%%%%%%%%
% \paragraph{Chapter Include Files.}
%
% The include files are called |cdocsch1.tex| and |cdocsch2.tex|.
%
%\iffalse
%<*samplechap1|samplechap2>
%\fi

% Optional override for |\version| flag:
%    \begin{macrocode}
%%\providecommand{\version}{final}
%    \end{macrocode}

% Include the main document:
%    \begin{macrocode}
\input{childdoc.def}
\childdocof{cdocsamp}
%    \end{macrocode}

%\iffalse
%</samplechap1|samplechap2>
%\fi
%
%\iffalse
%<*samplechap1>
%\fi
% Some text for chapter 1:
%    \begin{macrocode}
\section{one}
some text in chapter one
%    \end{macrocode}

%\iffalse
%</samplechap1>
%\fi
% Some text for chapter 2:
%\iffalse
%<*samplechap2>
%\fi
%    \begin{macrocode}
\section{two}
more text in chapter two
%    \end{macrocode}

%\iffalse
%</samplechap2>
%\fi
%
% %%%%%%%%%%%%%%%%%%%%%%%%%%%%%%%%%%%%%%
% \paragraph{Part Include Files.}
%
% The include files are called |cdocspt3.tex| and |cdocspt4.tex|.
%
%\iffalse
%<*samplepart3|samplepart4>
%\fi

% Optional override for |\version| flag:
%    \begin{macrocode}
%%\providecommand{\version}{final}
%    \end{macrocode}

% Include the main document:
%    \begin{macrocode}
\input{childdoc.def}
\childdocby{cdocsamp}
%    \end{macrocode}

%\iffalse
%</samplepart3|samplepart4>
%\fi
%
%\iffalse
%<*samplepart3>
%\fi
% Some text for part 3:
%    \begin{macrocode}
some text in part three
%    \end{macrocode}

%\iffalse
%</samplepart3>
%\fi
% Some text for part 4:
%\iffalse
%<*samplepart4>
%\fi
%    \begin{macrocode}
more text in part four
%    \end{macrocode}

%\iffalse
%</samplepart4>
%\fi
%
% %%%%%%%%%%%%%%%%%%%%%%%%%%%%%%%%%%%%%%
% \paragraph{Forwarding for a Complete Draft.}
%
% The following forwarding file |cdocsdrf.tex|
% compiles the main document in draft mode:
%\iffalse
%<*sampledraft>
%\fi
%    \begin{macrocode}
\def\version{draft}
\input{childdoc.def}
\childdocforward{cdocsamp}
%    \end{macrocode}

%\iffalse
%</sampledraft>
%\fi
%
% %%%%%%%%%%%%%%%%%%%%%%%%%%%%%%%%%%%%%%
% \paragraph{Forwarding for Final Version of the Chapters.}
%
% The following forwarding files |cdocsfn1.tex| and |cdocsfn2.tex|
% (with identical content)
% compile the final versions of the child documents
% |cdocsch1.tex| and |cdocsch2.tex|, respectively:
%\iffalse
%<*samplefinal>
%\fi
%    \begin{macrocode}
\def\version{final}
\input{childdoc.def}
\childdocforwardprefix[cdocsamp]{cdocsfn}{cdocsch}
%    \end{macrocode}

%\iffalse
%</samplefinal>
%\fi
%
% %%%%%%%%%%%%%%%%%%%%%%%%%%%%%%%%%%%%%%
% \paragraph{Command Line Processing.}
%
% The following three command lines generate the output files
% |cdocscld|, |cdocscl1| and |cdocscl2|
% which should be identical to
% |cdocsdrf|, |cdocsch1| and |cdocsfn2|, respectively:
% \begin{center}
% \begin{tabular}{l}
% |latex -jobname cdocscld \|\\
% |  "\def\version{draft}\input{childdoc.def}\childdocforward{cdocsamp}"|\\
% |latex -jobname cdocscl1 \|\\
% |  "\input{childdoc.def}\childdocforward[cdocsamp]{cdocsch1}"|\\
% |latex -jobname cdocscl2 \|\\
% |  "\def\version{final}\input{childdoc.def}\childdocforward{cdocsch2}"|
% \end{tabular}
% \end{center}
% Note that the trailing backslash on each first line
% merely continues the input to the second line
% (for convenient cut ant paste).
% Furthermore, the command |latex| can be replaced by any
% of its alternative versions such as |pdflatex|.
%
% %%%%%%%%%%%%%%%%%%%%%%%%%%%%%%%%%%%%%%%%%%%%%%%%%%%%%%%%%%%%%%%%%%%%%%%%%%%%%%
% %%%%%%%%%%%%%%%%%%%%%%%%%%%%%%%%%%%%%%%%%%%%%%%%%%%%%%%%%%%%%%%%%%%%%%%%%%%%%%
% \section{Implementation}
%\iffalse
%<*package>
%\fi
%
% This section describes the definitions file |childdoc.def|.

% The definitions cannot be loaded using |\usepackage| or |\RequirePackage|
% which has a mechanism to prevent loading a style file more than once.
% When loading the definitions by means of |\input|
% multiple instances have to be prevented manually:
%\iffalse
%This code needs to be before the `\ProvidesFile' directive
%which is defined at the beginning of this file.
%Therefore it is also placed there and commented out here.
%</package>
%<*discard>
%\fi
%    \begin{macrocode}
\ifdefined\childdocmain\endinput\fi
%    \end{macrocode}
%\iffalse
%</discard>
%<*package>
%\fi
%
% \macro{\ifchilddoc}
% \macro{\ifchilddocmanual}
% The conditional |\ifchilddoc| tells whether a
% child (true) or main (false) document is being compiled.
% The conditional |\ifchilddocmanual| tells whether
% the |\includeonly| mechanism is used (false) or
% the selection of child files must be performed manually (true).
% The definitions initialise to false:
%    \begin{macrocode}
\newif\ifchilddoc
\newif\ifchilddocmanual
%    \end{macrocode}

% \macro{\childdocname}
% \macro{\childdocjob}
% The macro |\childdocname| stores the name of the main document
% to be compiled. The macro |\childdocjob| stores the name of
% the document on which the \LaTeX{} compiler was originally invoked.
% The content of |\jobname| cannot be compared
% to filenames specified in the source due to different catcodes.
% The following code rescans |\jobname|, stores the result
% in |\childdocname| and saves a copy in |\childdocjob|:
%    \begin{macrocode}
\edef\childdocname{\scantokens\expandafter{\jobname\noexpand}}
\let\childdocjob\childdocname
%    \end{macrocode}

% \macro{\childdocdisable}
% The macro |\childdocdisable| prevents the main file
% from being processed more than once.
% At this stage, the main document command |\childdocmain|
% is assumed to be called once again where it should do nothing.
% Any subsequent call to it should prevent
% a secondary processing of the main document
% It overwrites the forwarding commands
% |\childdocof| and |\childdocforward|
% with empty macros to prevent further inclusions of the main document:
%    \begin{macrocode}
\newcommand{\childdocdisable}
{
  \renewcommand{\childdocmain}[1]{\renewcommand{\childdocmain}[1]{\endinput}}
  \renewcommand{\childdocof}[1]{}
  \renewcommand{\childdocby}[2][]{}
  \renewcommand{\childdocforward}[2][]{}
  \renewcommand{\childdocdisable}{}
}
%    \end{macrocode}

% \macro{\childdocmain}
% The macro |\childdocmain| is to be called at the top of the main file
% with nothing or the main filename (without extension) as argument.
% First, it breaks loops.
% If the argument is not empty and does not match |\childdocname|
% (which is set by the first inclusion of |childdoc.def|),
% |\ifchilddoc| is set to true, |\includeonly| is applied to the child file
% and |\jobname| is set to the main file
% (for proper handling of |.aux| files):
%    \begin{macrocode}
\newcommand{\childdocmain}[1]
{
  \childdocdisable\childdocmain{}
  \if?#1?\else
    \begingroup
      \def\childdoctmp{#1}
      \ifx\childdoctmp\childdocname
        \def\childdoctmp{}
      \else
        \def\childdoctmp
        {
          \childdoctrue
          \includeonly{\childdocname}
          \def\childdocjob{#1}
          \def\jobname{#1}
        }
      \fi
      \expandafter
    \endgroup
    \childdoctmp
  \fi
}
%    \end{macrocode}

% \macro{\childdocof}
% The command |\childdocof| redirects
% compilation to the main file |#1|.
%    \begin{macrocode}
\newcommand{\childdocof}[1]
{
  \childdocdisable
  \childdoctrue
  \includeonly{\childdocname}
  \def\jobname{#1}
  \def\childdocjob{#1}
  \input{#1}
}
%    \end{macrocode}

% \macro{\childdocby}
% The command |\childdocby| ....
%    \begin{macrocode}
\newcommand{\childdocby}[2][]
{
  \childdocdisable
  \childdoctrue
  \childdocmanualtrue
  \if?#1?\else
    \def\jobname{#2}
  \fi
  \def\childdocjob{#2}
  \input{#2}
  \endinput
}
%    \end{macrocode}

% \macro{\childdocforward}
% The command |\childdocforward| redirects
% compilation to the main file or
% (if the optional argument is given) a child file.
% Parameters are set as if the main file
% or a child file starting with |\childdocof| was compiled.
% Then compilation is handed over to the main file:
%    \begin{macrocode}
\newcommand{\childdocforward}[2][]
{
  \begingroup
    \if?#1?
      \def\childdoctmp
      {
        \def\childdocname{#2}
        \def\childdocjob{#2}
        \def\jobname{#2}
        \input{#2}
        \endinput
      }
    \else
      \def\childdoctmp
      {
        \childdocdisable
        \def\childdocname{#2}
        \childdoctrue
        \includeonly{#2}
        \def\childdocjob{#1}
        \def\jobname{#1}
        \input{#1}
        \endinput
      }
    \fi
    \expandafter
  \endgroup
  \childdoctmp
}
%    \end{macrocode}

% \macro{\childdocforwardprefix}
% The command |\childdocforwardprefix| redirects
% compilation to the main or a child file by means of a pattern.
% The prefix |#1| in the current filename is replaced by |#2|
% and the suffix of the current filename is kept
% (it is assumed that the filename does not contain the substring `|~~~|'
% which is used as a delimiter).
% Compilation is handed over to the new file by |\childdocforward|:
%    \begin{macrocode}
\newcommand{\childdocforwardprefix}[3][]
{
  \begingroup
    \def\childdocextract #2##1~~~{\def\childdoctmp{\childdocforward[#1]{#3##1}}}
    \expandafter\childdocextract\childdocname~~~
    \expandafter
  \endgroup
  \childdoctmp
}
%    \end{macrocode}

% \macro{\childdoc}
% The deprecated macro |\childdoc| is a legacy version of |\childdocmain|:
%    \begin{macrocode}
\newcommand{\childdoc}{\childdocmain}
%    \end{macrocode}

% \macro{\childdocredirect}
% The deprecated macro |\childdocredirect| is a legacy version
% of |\childdocforward| and |\childdocforwardprefix|:
%    \begin{macrocode}
\newcommand{\childdocredirect}[2][]
{
  \begingroup
    \if?#1?
      \def\childdoctmp{\childdocforward{#2}}
    \else
      \def\childdoctmp{\childdocforwardprefix{#1}{#2}}
    \fi
    \expandafter
  \endgroup
  \childdoctmp
}
%    \end{macrocode}

%\iffalse
%</package>
%\fi
%
\endinput
|\\
|\childdocby{|\textit{main}|}|\\
\end{tabular}
\end{center}
%
The directive |\childdocby| is similar to |\childdocof|
described in \secref{sec:include},
but the subsequent selection of content must be done manually.
To that end, both |\ifchilddoc| and |\ifchilddocmanual|
will be true upon processing of a part,
and the name of the part is stored in |\childdocname|.
Note that |\jobname| will be set to the filename of the current part
so that each part receives an individual |.aux| file
that does not interfere with the |.aux| file(s) of the main document.
This behaviour can be altered by the alternative form
|\childdocby[*]{|\textit{main}|}| (with a non-empty optional argument)
which uses the |.aux| file of the main document
by setting |\jobname| to \textit{main}.

%%%%%%%%%%%%%%%%%%%%%%%%%%%%%%%%%%%%%%%%%%%%%%%%%%%%%%%%%%%%%%%%%%%%%%%%%%%%%%%%
\subsection{Driver Development}
\label{sec:driver}

The \textsf{childdoc} mechanism can also be use for the development
of definition files such as \LaTeX{} styles or classes.
This case differs from the above setup with multiple parts
included by |\include| in that no |\includeonly| should be invoked.
This can be achieved by starting the include file
(before |\ProvidesPackage|) with:
%
\begin{center}
\begin{tabular}{l}
|% \iffalse
%
% childdoc.dtx Copyright (C) 2017-2018 Niklas Beisert
%
% This work may be distributed and/or modified under the
% conditions of the LaTeX Project Public License, either version 1.3
% of this license or (at your option) any later version.
% The latest version of this license is in
%   http://www.latex-project.org/lppl.txt
% and version 1.3 or later is part of all distributions of LaTeX
% version 2005/12/01 or later.
%
% This work has the LPPL maintenance status `maintained'.
%
% The Current Maintainer of this work is Niklas Beisert.
%
% This work consists of the files childdoc.dtx and childdoc.ins
% and the derived files childdoc.def and cdocsamp.tex with
% cdocsch1.tex, cdocsch2.tex, cdocsdrf.tex, cdocsfn1.tex, cdocsfn2.tex.
%
%<package>\ifdefined\childdocmain\endinput\fi
%<package>\ProvidesFile{childdoc.def}[2018/12/30 v2.0 child document driver]
%<samplemain>\ProvidesFile{cdocsamp.tex}[2018/12/30 v2.0 sample for childdoc]
%<*driver>
%\ProvidesFile{childdoc.drv}[2018/12/30 v2.0 childdoc reference manual file]
\PassOptionsToClass{10pt,a4paper}{article}
\documentclass{ltxdoc}

\usepackage[margin=35mm]{geometry}
\usepackage{hyperref}
\usepackage{hyperxmp}
\usepackage[usenames]{color}

\hypersetup{colorlinks=true}
\hypersetup{pdfstartview=FitH}
\hypersetup{pdfpagemode=UseNone}
\hypersetup{pdfsource={}}
\hypersetup{pdflang={en-UK}}
\hypersetup{pdfcopyright={Copyright 2017-2018 Niklas Beisert.
  This work may be distributed and/or modified under the
  conditions of the LaTeX Project Public License, either version 1.3
  of this license or (at your option) any later version.}}
\hypersetup{pdflicenseurl={http://www.latex-project.org/lppl.txt}}
\hypersetup{pdfcontactaddress={ETH Zurich, ITP, HIT K,
  Wolfgang-Pauli-Strasse 27}}
\hypersetup{pdfcontactpostcode={8093}}
\hypersetup{pdfcontactcity={Zurich}}
\hypersetup{pdfcontactcountry={Switzerland}}
\hypersetup{pdfcontactemail={nbeisert@itp.phys.ethz.ch}}
\hypersetup{pdfcontacturl={http://people.phys.ethz.ch/\xmptilde nbeisert/}}

\newcommand{\secref}[1]{\hyperref[#1]{section \ref*{#1}}}

\parskip1ex
\parindent0pt
\let\olditemize\itemize
\def\itemize{\olditemize\parskip0pt}

\begin{document}

\title{The \textsf{childdoc} Package}
\hypersetup{pdftitle={The childdoc Package}}
\author{Niklas Beisert\\[2ex]
  Institut f\"ur Theoretische Physik\\
  Eidgen\"ossische Technische Hochschule Z\"urich\\
  Wolfgang-Pauli-Strasse 27, 8093 Z\"urich, Switzerland\\[1ex]
  \href{mailto:nbeisert@itp.phys.ethz.ch}
  {\texttt{nbeisert@itp.phys.ethz.ch}}}
\hypersetup{pdfauthor={Niklas Beisert}}
\hypersetup{pdfsubject={Manual for the LaTeX2e Package childdoc}}
\date{30 December 2018, \textsf{v2.0}}
\maketitle

\begin{abstract}\noindent
\textsf{childdoc} is a \LaTeXe{} package
that enables the direct compilation
of document sections included by |\include|
to individual files.
\end{abstract}

\begingroup
\parskip0ex
\tableofcontents
\endgroup

%%%%%%%%%%%%%%%%%%%%%%%%%%%%%%%%%%%%%%%%%%%%%%%%%%%%%%%%%%%%%%%%%%%%%%%%%%%%%%%%
%%%%%%%%%%%%%%%%%%%%%%%%%%%%%%%%%%%%%%%%%%%%%%%%%%%%%%%%%%%%%%%%%%%%%%%%%%%%%%%%
\section{Introduction}

\LaTeX{} provides a mechanism to structure a large document (such as a book)
into a main file and several child files (containing the chapters)
using the |\include| command.
This mechanism is beneficial for documents
which span hundreds of pages in order to
make the source file(s) more manageable.
Moreover, compilation can be restricted to
selected child files by means of the |\includeonly| command.
The latter feature can be used to reduce the compilation time while editing
(this was significantly more useful in the earlier days of \LaTeX{})
or to generate a smaller document which is easier to navigate.
Another application of |\includeonly| is to generate
documents consisting of selected parts of the complete document.

However, there are a few drawbacks of the plain |\include| mechanism:
\begin{itemize}
\item
The child files cannot be compiled on their own,
they can only be compiled via the main file.
A naive editing environment
(such as a text editor with an option
to have the current file processed by \LaTeX)
may require one to switch to the main file before compiling;
attempting to compile the child file produces errors.
\item
The main file must be modified (each time)
to adjust the |\includeonly| command
to the present needs. This easily leaves the main file in a messy state.
\item
The generated document will always carry the filename
of the main document. This is inconvenient if
several child files are to be compiled and
to be kept for distribution.
\end{itemize}

The present package provides a simple interface
to make child files individually compilable by \LaTeX{}.
Compiling a child file then has the same effect as compiling
the main file with an |\includeonly| command
to select the appropriate child.
Moreover the generated document will carry the name of the child
rather than the main file.
This resolves all three above issues.

This feature is meant to make the editing of books,
thesis documents and lecture notes somewhat more convenient.
However, the package can also be used efficiently for
composing a series of documents (such as exercise sheets)
which are typically distributed individually.
It then assists the author in generating the individual documents
(potentially in different versions)
as well as a document containing the collected series.
Another application is in developing style files
or other kinds of included material
where compilation of the style file could redirect
to a sample or test file.

%%%%%%%%%%%%%%%%%%%%%%%%%%%%%%%%%%%%%%%%%%%%%%%%%%%%%%%%%%%%%%%%%%%%%%%%%%%%%%%%
%%%%%%%%%%%%%%%%%%%%%%%%%%%%%%%%%%%%%%%%%%%%%%%%%%%%%%%%%%%%%%%%%%%%%%%%%%%%%%%%
\section{Usage}

First of all, the package \textsf{childdoc} is \emph{not} a standard
\LaTeXe{} |.sty| style file! Therefore it needs to be invoked in
a non-standard way.

%%%%%%%%%%%%%%%%%%%%%%%%%%%%%%%%%%%%%%%%%%%%%%%%%%%%%%%%%%%%%%%%%%%%%%%%%%%%%%%%
\subsection{Included Files}
\label{sec:include}

%%%%%%%%%%%%%%%%%%%%%%%%%%%%%%%%%%%%%%%%
\DescribeMacro{\childdocmain}
To use the package, add the commands
\begin{center}
\begin{tabular}{l}
|\input{childdoc.def}|\\
|\childdocmain{}|\\
\end{tabular}
\end{center}
at the very top of the main \LaTeX{} file,
in particular \emph{before} the |\documentclass| statement!
The argument of |\childdocmain| should be left empty
(but it must be present).

%%%%%%%%%%%%%%%%%%%%%%%%%%%%%%%%%%%%%%%%
\DescribeMacro{\childdocof}
Furthermore, add the commands
\begin{center}
\begin{tabular}{l}
|\input{childdoc.def}|\\
|\childdocof{|\textit{main}|}|\\
\end{tabular}
\end{center}
at the top of every child file \textit{child}
which is included by |\include{|\textit{child}|}|
from within the main file
(or at least for those files to be compiled individually).
The argument \textit{main} must be the filename of the main file.

There are a couple of
considerations in setting up the main and child documents:

%%%%%%%%%%%%%%%%%%%%%%%%%%%%%%%%%%%%%%%%
\paragraph{Restrictions.}

Please note the following restrictions:
\begin{itemize}
\item
|\childdocmain| must be called with one argument \textit{main}
to ensure compatibility with earlier version of the package.
It must either be empty (|\childdocmain{}|)
or precisely match the filename of the main file in which it is specified.
See \secref{sec:detection} for further information.
\item
The filename \textit{main} must be specified without the |.tex| extension.
\item
The filename \textit{main} is case sensitive
(even in case-insensitive file systems)
due to internal string comparison.
\item
The argument \textit{main} should be fully expanded, it cannot be a macro.
\item
Subdirectories and special characters should be avoided in filenames.
\item
The command |\childdocmain{|\textit{main}|}| must be followed by a whitespace.
It should not be followed immediately by another command
or by a comment mark `|%|'.
This is because the \TeX{} parser reads the token immediately following
the argument of |\childdocmain| and puts it
at the beginning of every child section;
however, a white\-space is ignored.
\end{itemize}

%%%%%%%%%%%%%%%%%%%%%%%%%%%%%%%%%%%%%%%%
\paragraph{Content of Main File.}

It is advisable to place all content in the child files included by |\include|.
Any output contained in the main file will appear in all child documents
unless suppressed manually;
it cannot be suppressed automatically by the |\includeonly| directive
and thus should normally be avoided.
A method to include some content in the main file
by means of conditional processing is described in \secref{sec:conditional}.

%%%%%%%%%%%%%%%%%%%%%%%%%%%%%%%%%%%%%%%%
\paragraph{Page Numbering.}

When only a part of the document is compiled,
the appropriate numbering of pages
(as well as other status parameters)
is determined from the |.aux| files.
The latter contain information from previous passes.
However this information needs to propagate through
all intermediate child documents.
Therefore the page numbering in child documents may well
be inconsistent until the complete document is compiled at least once.

A useful (if unconventional) way to always ensure a consistent
page numbering is to restart the numbering in each child document
and denote the pages by `\textit{child}|.|\textit{page}'
where \textit{child} represents the chapter/section number of the child file.
This can be achieved by the command
|\numberwithin{page}{|\textit{child}|}|
of the \textsf{amsmath} package
where \textit{child} can be |chapter| or |section|
depending on the chosen structuring.
Alternatively, one can modify the macro |\thepage| appropriately
and reset the counter |page| at the start of each child file.

%%%%%%%%%%%%%%%%%%%%%%%%%%%%%%%%%%%%%%%%%%%%%%%%%%%%%%%%%%%%%%%%%%%%%%%%%%%%%%%%
\subsection{Conditional Processing}
\label{sec:conditional}

The package provides a mechanism to compile different versions
of a document. To customise the versions further some conditional processing
can come in handy to distinguish which version is being compiled.
The package provides two macros to describe the compilation context:

%%%%%%%%%%%%%%%%%%%%%%%%%%%%%%%%%%%%%%%%
\DescribeMacro{\ifchilddoc}
The conditional |\ifchilddoc| distinguishes between the compilation of
child documents and the main document:
%
\begin{center}
|\ifchilddoc |\textit{child-code}| |[|\||else |\textit{main-code}]| \||fi|
\end{center}

%%%%%%%%%%%%%%%%%%%%%%%%%%%%%%%%%%%%%%%%
\DescribeMacro{\childdocname}
\DescribeMacro{\childdocjob}
The macro |\childdocname| contains the filename (without extension)
of the main or child file being processed.
Note that |\childdocjob| will always contain the name of the main file.

%%%%%%%%%%%%%%%%%%%%%%%%%%%%%%%%%%%%%%%%
\paragraph{Title Page.}

Conditional processing can be used to include a title or banner page
in the main document when proper precautions are taken.
Importantly, the code in the main file should ensure that the page counter
(as well as other status parameters which are stored in the |.aux| files)
takes the same value after the conditional processing.
Otherwise the page numbers may take divergent values
depending on which part is compiled.

For example, a title page could be declared by:
%
\begin{center}
\begin{tabular}{l}
|\ifchilddoc\||else|\\
|\addtocounter{page}{-1}|\\
\textit{code for title page}\\
|\newpage|\\
|\||fi|
\end{tabular}
\end{center}
%
A banner page for the child documents can be generated by:
%
\begin{center}
\begin{tabular}{l}
|\ifchilddoc|\\
|\addtocounter{page}{-1}|\\
\textit{code for banner page}\\
|\newpage|\\
|\||fi|
\end{tabular}
\end{center}
%
Here one could write a message such as:
\begin{center}
|This is the part \childdocname{} of \childdocjob{}.|
\end{center}

%%%%%%%%%%%%%%%%%%%%%%%%%%%%%%%%%%%%%%%%%%%%%%%%%%%%%%%%%%%%%%%%%%%%%%%%%%%%%%%%
\subsection{Flags}
\label{sec:flags}

The package makes it easy to generate different versions
of the main or child documents.
To this end compilation flags can be defined
and assigned different default values.
They will be particularly useful in conjunction
with the forwarding mechanism described in \secref{sec:forward}.

For example, it may be useful to have a flag |\version|
which can be set to |draft| or |final|.
The document source will contain some conditional code
depending on the value of |\version|.
Suppose further, the flag should default to |final| for the main file
and to |draft| for child files
which is a natural assignment for editing the document.
This is achieved by placing the following code
in the preamble of the main document
(below the |\childdocmain| directive):
%
\begin{center}
\begin{tabular}{l}
|\ifchilddoc|\\
|\providecommand{\version}{draft}|\\
|\||else|\\
|\providecommand{\version}{final}|\\
|\||fi|
\end{tabular}
\end{center}
%
The definition by |\providecommand| makes sure
that previous definitions are not overwritten.
Further statements |\providecommand{\version}{...}|
can thus be added before the above code to override it.

For the main file, one might add a line
(between |\childdocmain| and the above block)
%
\begin{center}
|%\ifchilddoc\||else\providecommand{\version}{draft}\||fi|
\end{center}
%
which can be uncommented to produce a draft version.
Likewise one can add a line to the very top of a child file
(above the |\childdocof{|\textit{main}|}| directive)
%
\begin{center}
|%\providecommand{\version}{final}|
\end{center}
%
which can be uncommented to produce the final version of this child document.

%%%%%%%%%%%%%%%%%%%%%%%%%%%%%%%%%%%%%%%%%%%%%%%%%%%%%%%%%%%%%%%%%%%%%%%%%%%%%%%%
\subsection{Forwarding}
\label{sec:forward}

Different versions of the main or child documents
using compilation flags as described in \secref{sec:flags}
can be (permanently) stored in different files
for convenient compilation, viewing and distribution.
To this end, the package defines a command
to pass on compilation to a different file:

%%%%%%%%%%%%%%%%%%%%%%%%%%%%%%%%%%%%%%%%
\DescribeMacro{\childdocforward}
The command |\childdocforward| redirects processing to
another source file:
%
\begin{center}
\begin{tabular}{l}
|\input{childdoc.def}|\\
|\childdocforward[|\textit{main}|]{|\textit{dest}|}|\\
\end{tabular}
\end{center}
%
The argument \textit{dest} is the destination file
(without extension).
It should be the main file or one of the child files.
Note that further \textsf{childdoc} directives
such as |\childdocof| and |\childdocforward|
in the indicated file will be processed in this form.
The optional argument \textit{main}
passes on directly to the main file \textit{main}
while pretending to compile the child \textit{dest}.
This form behaves as if \textit{dest}
issues |\childdocof{|\textit{main}|}| right away,
and no further \textsf{childdoc} directives will be processed.

%%%%%%%%%%%%%%%%%%%%%%%%%%%%%%%%%%%%%%%%
\DescribeMacro{\...prefix}
In the alternative form |\childdocforwardprefix|,
%
\begin{center}
\begin{tabular}{l}
|\input{childdoc.def}|\\
|\childdocforwardprefix[|\textit{main}|]{|\textit{prefix}|}{|\textit{dest}|}|
\end{tabular}
\end{center}
%
the destination file is determined by a pattern
depending on the current file:
To make this work, the current file must be called
`{\textit{prefix}\hspace{0.2em}\textit{suffix}}'
with \textit{prefix} matching precisely the argument.
Processing is then passed on to the file
`{\textit{dest}\hspace{0.2em}\textit{suffix}}'.
Surely, the same effect is achieved by
directly specifying the
argument `{\textit{dest}\hspace{0.2em}\textit{suffix}}'
in the first form.
However, that requires to set up a different file
for each child. With the alternative form of the command
all these files can have exactly the same content
which simplifies setting them up and maintaining them.

For example, the following file |draft.tex|
with a compilation flag |\version| as described in \secref{sec:flags}
compiles the main document as a draft:
%
\begin{center}
\begin{tabular}{l}
|\def\version{draft}|\\
|\input{childdoc.def}|\\
|\childdocforward{|\textit{main}|}|
\end{tabular}
\end{center}
%
Likewise, the following files |final|\textit{nn}|.tex|
compile the final version of the child document
|child|\textit{nn}|.tex|:
%
\begin{center}
\begin{tabular}{l}
|\def\version{final}|\\
|\input{childdoc.def}|\\
|\childdocforwardprefix{final}{child}|
\end{tabular}
\end{center}
%

Note that when several versions of a main file and/or of each child file
are to be generated, it may be convenient to set up a |Makefile| or
shell script to automatise the process.

%%%%%%%%%%%%%%%%%%%%%%%%%%%%%%%%%%%%%%%%%%%%%%%%%%%%%%%%%%%%%%%%%%%%%%%%%%%%%%%%
\subsection{Command Line Processing}
\label{sec:commandline}

The effect of redirection files can also be achieved by invoking
the \LaTeX{} compiler with a more elaborate command line.
Most conveniently this should be done as part
of a shell script or a |Makefile|.

When using \textsf{childdoc} in the main file, the following
command lines effectively perform a redirection
(note that depending on the shell being used,
backslashes may have to be doubled: `|\|' $\to$ `|\\|'):
%
\begin{center}
|... -jobname "|\textit{target}|" |\\|"|[\textit{flags}]%
|\input{childdoc.def}\childdocforward[|\textit{main}|]{|\textit{dest}|}"|
\end{center}
%
Here \textit{target} is the name of the output file,
\textit{main} is the name of the main file
and \textit{dest} is the name of the main or child file to be processed
(all filenames without extensions).
The optional argument \textit{main} can be omitted
if \textit{main} matches \textit{dest}.
Optionally, compilation \textit{flags} can be defined via |\def| commands.
This command line makes the \TeX{} engine believe
it is compiling the file \textit{target}
whose content is specified as the latter parameter.
The provided code then forwards the processing to
\textit{main} or \textit{dest} as described in \secref{sec:forward}.

%%%%%%%%%%%%%%%%%%%%%%%%%%%%%%%%%%%%%%%%%%%%%%%%%%%%%%%%%%%%%%%%%%%%%%%%%%%%%%%%
\subsection{Include by Input}
\label{sec:input}

Including child documents by |\include| has some restrictions by design.
Most notably, the content of a child document always occupies
its own set of pages; pages cannot be shared between child documents.
Usually, this behaviour makes perfect sense
because each child document contain an essential part of the document.
However, in some situations it may be desirable to compose
a document from a collection of parts
without having mandatory page breaks between then.
For this case, the package
provides a mechanism to include parts
by |\input| which can also be processed individually.
However, by construction this mechanism
requires manual handling of the content to be output.

%%%%%%%%%%%%%%%%%%%%%%%%%%%%%%%%%%%%%%%%
\DescribeMacro{\ifchilddocmanual}
The main file should be prepared as usual, see \secref{sec:include}.
However, the document body must make a distinction
between processing of an individual part and of the main document, e.g.:
%
\begin{center}
\begin{tabular}{l}
|\ifchilddocmanual|\\
|\input{\childdocname}|\\
|\||else|\\
\textit{document body with }|\input{|\textit{part}|}|\\
|\||fi|
\end{tabular}
\end{center}
%
The conditional |\ifchilddocmanual| is true whenever
a part to be included by |\input| is being compiled,
and the name of the part is stored in |\childdocname|.

%%%%%%%%%%%%%%%%%%%%%%%%%%%%%%%%%%%%%%%%
\DescribeMacro{\childdocby}
Each part to be included by |\input| should start with:
%
\begin{center}
\begin{tabular}{l}
|\input{childdoc.def}|\\
|\childdocby{|\textit{main}|}|\\
\end{tabular}
\end{center}
%
The directive |\childdocby| is similar to |\childdocof|
described in \secref{sec:include},
but the subsequent selection of content must be done manually.
To that end, both |\ifchilddoc| and |\ifchilddocmanual|
will be true upon processing of a part,
and the name of the part is stored in |\childdocname|.
Note that |\jobname| will be set to the filename of the current part
so that each part receives an individual |.aux| file
that does not interfere with the |.aux| file(s) of the main document.
This behaviour can be altered by the alternative form
|\childdocby[*]{|\textit{main}|}| (with a non-empty optional argument)
which uses the |.aux| file of the main document
by setting |\jobname| to \textit{main}.

%%%%%%%%%%%%%%%%%%%%%%%%%%%%%%%%%%%%%%%%%%%%%%%%%%%%%%%%%%%%%%%%%%%%%%%%%%%%%%%%
\subsection{Driver Development}
\label{sec:driver}

The \textsf{childdoc} mechanism can also be use for the development
of definition files such as \LaTeX{} styles or classes.
This case differs from the above setup with multiple parts
included by |\include| in that no |\includeonly| should be invoked.
This can be achieved by starting the include file
(before |\ProvidesPackage|) with:
%
\begin{center}
\begin{tabular}{l}
|\input{childdoc.def}|\\
|\childdocforward{|\textit{main}|}|\\
\end{tabular}
\end{center}
%
or alternatively with:
%
\begin{center}
\begin{tabular}{l}
|\input{childdoc.def}|\\
|\childdocby{|\textit{main}|}|\\
\end{tabular}
\end{center}
%
Both forms have slightly different effects as described above.
The main file is prepared as usual, see \secref{sec:include}.

%%%%%%%%%%%%%%%%%%%%%%%%%%%%%%%%%%%%%%%%%%%%%%%%%%%%%%%%%%%%%%%%%%%%%%%%%%%%%%%%
\subsection{Legacy Detection}
\label{sec:detection}

The directive |\childdocmain| in the main file can detect
whether the complete document or merely a child is to be compiled
even without using the directive |\childdocof|.
This method is deprecated because it is less robust
and there is no compelling reason to use it;
it is merely provided for backward compatibility
and it may be removed in future versions.

If the detection mechanism is to be used,
it is mandatory to correctly specify
the filename of the main file as the argument of |\childdocmain|:
%
\begin{center}
\begin{tabular}{l}
|\input{childdoc.def}|\\
|\childdocmain{|\textit{main}|}|\\
\end{tabular}
\end{center}
%
If |\jobname| does not match the argument \textit{main} of |\childdocmain|,
it is assumed that |\jobname| points to the child file to be compiled.
When using |\childdocmain| with the main file specified as argument,
it suffices to start a child file
with just |\input{|\textit{main}|}|
without loading of the package and using |\childdocof|.
If instead all processing is done
with the appropriate \textsf{childdoc} directives,
the argument of \textit{main} of |\childdocmain| can be empty.

An alternative version of the command line processing described
in \secref{sec:commandline} using the detection mechanism reads:
%
\begin{center}
|... -jobname "|\textit{target}|" "|[\textit{flags}]%
[|\def\jobname{|\textit{dest}|}|]|\input{|\textit{main}|}"|
\end{center}

%%%%%%%%%%%%%%%%%%%%%%%%%%%%%%%%%%%%%%%%%%%%%%%%%%%%%%%%%%%%%%%%%%%%%%%%%%%%%%%%
\subsection{Manual Code}
\label{sec:manual}

In case one cannot be certain whether the definitions file |childdoc.def|
is installed on the target \TeX{} distribution
and one prefers not to ship it,
it is conceivable to paste a few relevant commands into the sources.

To that end, drop all statements |\input{childdoc.def}|
and perform the replacements as outlined below.
Instead of |\childdocmain{|\textit{main}|}| add the following code
to the top of the main file:
%
\begin{center}
\begin{tabular}{l}
|\||ifdefined\childdocname\endinput\||fi\newif\ifchilddoc|\\
|\edef\childdocname{\scantokens\expandafter{\jobname\noexpand}}|\\
|\def\childdocmain{|\textit{main}|}\||ifx\childdocmain\childdocname\||else|\\
|\childdoctrue\includeonly{\childdocname}\let\jobname\childdocmain\||fi|\\
\end{tabular}
\end{center}
%
Instead of |\childdocof{|\textit{main}|}| just include the main file
at the top of each child file:
%
\begin{center}
|\input{|\textit{main}|}|
\end{center}
%
A simple redirection |\childdocforward{|\textit{dest}|}| is achieved by:
%
\begin{center}
|\def\jobname{|\textit{dest}|}\input{\jobname}|
\end{center}
%
The redirection with prefix
|\childdocforwardprefix[|\textit{prefix}|]{|\textit{dest}|}|
is accomplished by:
%
\begin{center}
\begin{tabular}{l}
|{\edef\jobname{\scantokens\expandafter{\jobname\noexpand}}|\\
|\def\redirectjob |\textit{prefix}|#1~~~{\gdef\jobname{|\textit{dest}|#1}}|\\
|\expandafter\redirectjob\jobname~~~}\input{\jobname}|
\end{tabular}
\end{center}

In an alternative approach,
child documents can be compiled by a specific command line
without additional code or specific definitions:
%
\begin{center}
|... -jobname "|\textit{target}|" "|[\textit{flags}]%
|\includeonly{|\textit{dest}|}\input{|\textit{main}|}"|
\end{center}
%

%%%%%%%%%%%%%%%%%%%%%%%%%%%%%%%%%%%%%%%%%%%%%%%%%%%%%%%%%%%%%%%%%%%%%%%%%%%%%%%%
%%%%%%%%%%%%%%%%%%%%%%%%%%%%%%%%%%%%%%%%%%%%%%%%%%%%%%%%%%%%%%%%%%%%%%%%%%%%%%%%
\section{Information}

%%%%%%%%%%%%%%%%%%%%%%%%%%%%%%%%%%%%%%%%%%%%%%%%%%%%%%%%%%%%%%%%%%%%%%%%%%%%%%%%
\subsection{Copyright}

Copyright \copyright{} 2017--2018 Niklas Beisert

This work may be distributed and/or modified under the
conditions of the \LaTeX{} Project Public License, either version 1.3
of this license or (at your option) any later version.
The latest version of this license is in
  \url{http://www.latex-project.org/lppl.txt}
and version 1.3 or later is part of all distributions of \LaTeX{}
version 2005/12/01 or later.

This work has the LPPL maintenance status `maintained'.

The Current Maintainer of this work is Niklas Beisert.

This work consists of the files |README.txt|, |childdoc.ins| and |childdoc.dtx|
as well as the derived files |childdoc.def|, |cdocsamp.tex|
with |cdocsch1.tex|, |cdocsch2.tex|, |cdocspt3.tex|, |cdocspt4.tex|,
|cdocsdrf.tex|, |cdocsfn1.tex|, |cdocsfn2.tex|
as well as |childdoc.pdf|.

%%%%%%%%%%%%%%%%%%%%%%%%%%%%%%%%%%%%%%%%%%%%%%%%%%%%%%%%%%%%%%%%%%%%%%%%%%%%%%%%
\subsection{Files and Installation}

The package consists of the files:
%
\begin{center}
\begin{tabular}{ll}
    |README.txt|   & readme file \\
    |childdoc.ins| & installation file \\
    |childdoc.dtx| & source file \\
    |childdoc.def| & definition file \\
    |cdocsamp.tex| & sample main file \\
    |cdocsch1.tex| & sample include file \\
    |cdocsch2.tex| & sample include file \\
    |cdocspt3.tex| & sample part file \\
    |cdocspt4.tex| & sample part file \\
    |cdocsdrf.tex| & sample redirection file \\
    |cdocsfn1.tex| & sample redirection file \\
    |cdocsfn2.tex| & sample redirection file \\
    |childdoc.pdf| & manual
\end{tabular}
\end{center}
%
The distribution consists of the files
|README.txt|, |childdoc.ins| and |childdoc.dtx|.
%
\begin{itemize}
\item
Run (pdf)\LaTeX{} on |childdoc.dtx|
to compile the manual |childdoc.pdf| (this file).
\item
Run \LaTeX{} on |childdoc.ins| to create the definitions file |childdoc.def|
and the sample |cdocsamp.tex| with include files
|cdocsch1.tex|, |cdocsch2.tex|, |cdocspt3.tex|, |cdocspt4.tex|,
|cdocsdrf.tex|, |cdocsfn1.tex|, |cdocsfn2.tex|.
Then copy the file |childdoc.def| to an appropriate directory of your \LaTeX{}
distribution, e.g.\ \textit{texmf-root}|/tex/latex/childdoc|.
\end{itemize}

%%%%%%%%%%%%%%%%%%%%%%%%%%%%%%%%%%%%%%%%%%%%%%%%%%%%%%%%%%%%%%%%%%%%%%%%%%%%%%%%
\subsection{Related CTAN Packages}

There are several other packages which offer a similar functionality:
%
\begin{itemize}
\item
The packages
\href{http://ctan.org/pkg/docmute}{\textsf{docmute}},
\href{http://ctan.org/pkg/includex}{\textsf{includex}} and
\href{http://ctan.org/pkg/standalone}{\textsf{standalone}}
provide commands to include only the document body of
a child file thus allowing both files to be compiled individually.
\item
The packages \href{http://ctan.org/pkg/subdocs}{\textsf{subdocs}}
and \href{http://ctan.org/pkg/subfiles}{\textsf{subfiles}}
provide structures in which the main and child documents can be
encapsulated and allowing them to be compiled individually.
The inclusion mechanism is different from the conventional |\include|.
\item
The package \href{http://ctan.org/pkg/combine}{\textsf{combine}}
is an elaborate solution to combine several documents into one.
\end{itemize}
%
See also the CTAN topic \href{http://ctan.org/topic/subdocs}{\textsf{subdocs}}
for further related packages.
The present package differs from the above solutions in that
a document structure constructed with the conventional |\include| mechanism
just needs two extra commands at the top of every file
such that all constituent files can be compiled individually.

%%%%%%%%%%%%%%%%%%%%%%%%%%%%%%%%%%%%%%%%%%%%%%%%%%%%%%%%%%%%%%%%%%%%%%%%%%%%%%%%
%\subsection{Feature Suggestions}
%
%The following is a list of features which may be useful for future
%versions of this package:
%%
%\begin{itemize}
%\item
%\ldots
%\end{itemize}

%%%%%%%%%%%%%%%%%%%%%%%%%%%%%%%%%%%%%%%%%%%%%%%%%%%%%%%%%%%%%%%%%%%%%%%%%%%%%%%%
\subsection{Revision History}

%%%%%%%%%%%%%%%%%%%%%%%%%%%%%%%%%%%%%%%%
\paragraph{v2.0:} 2018/12/30

\begin{itemize}
\item
immediate forward processing
\item
added |\childdocby| mechanism
\item
manual restructured
\end{itemize}

%%%%%%%%%%%%%%%%%%%%%%%%%%%%%%%%%%%%%%%%
\paragraph{v1.6:} 2018/01/17

\begin{itemize}
\item
application for development of include files
\item
corrections to manual
\end{itemize}

%%%%%%%%%%%%%%%%%%%%%%%%%%%%%%%%%%%%%%%%
\paragraph{v1.5:} 2017/05/21

\begin{itemize}
\item
more complete structuring introduced
\item
|\childdocof| introduced
\item
|\childdoc| renamed to |\childdocmain|
\item
|\childredirect| renamed to |\childdocforward| and |\childdocforwardprefix|
and functionality expanded
\end{itemize}

%%%%%%%%%%%%%%%%%%%%%%%%%%%%%%%%%%%%%%%%
\paragraph{v1.0:} 2017/04/27

\begin{itemize}
\item
manual and install package
\item
first version published on CTAN
\end{itemize}

%%%%%%%%%%%%%%%%%%%%%%%%%%%%%%%%%%%%%%%%
\paragraph{v0.6:} 2017/04/26

\begin{itemize}
\item
redirection mechanism added
\end{itemize}

%%%%%%%%%%%%%%%%%%%%%%%%%%%%%%%%%%%%%%%%
\paragraph{v0.5:} 2017/04/26

\begin{itemize}
\item
functionality in definition file
\end{itemize}


%%%%%%%%%%%%%%%%%%%%%%%%%%%%%%%%%%%%%%%%%%%%%%%%%%%%%%%%%%%%%%%%%%%%%%%%%%%%%%%%
%%%%%%%%%%%%%%%%%%%%%%%%%%%%%%%%%%%%%%%%%%%%%%%%%%%%%%%%%%%%%%%%%%%%%%%%%%%%%%%%
%%%%%%%%%%%%%%%%%%%%%%%%%%%%%%%%%%%%%%%%%%%%%%%%%%%%%%%%%%%%%%%%%%%%%%%%%%%%%%%%
\appendix

\settowidth\MacroIndent{\rmfamily\scriptsize 000\ }

 \DocInput{childdoc.dtx}

\end{document}
%</driver>
% \fi
%
% %%%%%%%%%%%%%%%%%%%%%%%%%%%%%%%%%%%%%%%%%%%%%%%%%%%%%%%%%%%%%%%%%%%%%%%%%%%%%%
% %%%%%%%%%%%%%%%%%%%%%%%%%%%%%%%%%%%%%%%%%%%%%%%%%%%%%%%%%%%%%%%%%%%%%%%%%%%%%%
% \section{Sample}
%\iffalse
%<*samplemain>
%\fi
%
% The following presents a sample document
% with two chapters, two parts, a title page,
% a compile flag as well as three forwarding files to set the flag.
% It consists of eight |.tex| files:
% \begin{center}
% \begin{tabular}{ll}
% |cdocsamp.tex|&main file\\
% |cdocsch1.tex|&include file for chapter 1\\
% |cdocsch2.tex|&include file for chapter 2\\
% |cdocspt3.tex|&include file for part 3\\
% |cdocspt4.tex|&include file for part 4\\
% |cdocsdrf.tex|&forwarding file for main file in draft mode\\
% |cdocsfi1.tex|&forwarding file for final version of chapter 1\\
% |cdocsfi2.tex|&forwarding file for final version of chapter 2\\
% \end{tabular}
% \end{center}
% Each of the eight files can be compiled directly by the \LaTeX{} compiler.
%
% %%%%%%%%%%%%%%%%%%%%%%%%%%%%%%%%%%%%%%
% \paragraph{Main File.}
%
% The main file is called |cdocsamp.tex|.
%
% Load the \textsf{childdoc} definitions and
% declare the filename for the main document:
%    \begin{macrocode}
\input{childdoc.def}
\childdocmain{}
%    \end{macrocode}

% Optional override for |\version| flag:
%    \begin{macrocode}
%%\ifchilddoc\else\providecommand{\version}{draft}\fi
%    \end{macrocode}

% Define the default values for the |\version| flag
% (|final| for the main file and |draft| for childs):
%    \begin{macrocode}
\ifchilddoc
\providecommand{\version}{draft}
\else
\providecommand{\version}{final}
\fi
%    \end{macrocode}

% Load the standard document class:
%    \begin{macrocode}
\documentclass[12pt]{article}
%    \end{macrocode}

% Start the document body:
%    \begin{macrocode}
\begin{document}
%    \end{macrocode}

% Declare a title page.
% Print title, part of document being processed and version flag:
%    \begin{macrocode}
\addtocounter{page}{-1}
\begin{center}
{\LARGE\bfseries{}childdoc example\par}
\vspace{1cm}
\ifchilddoc
\ifchilddocmanual part\else chapter\fi:
`\childdocname' of `\childdocjob'\par
\else
main document: `\childdocjob'\par
\fi
version: \version\par
\end{center}
\newpage
%    \end{macrocode}

% Manually include selected file,
% otherwise process as usual:
%    \begin{macrocode}
\ifchilddocmanual
\section*{part `\childdocname'}
\input{\childdocname}
\else
%    \end{macrocode}

% Include the two chapters:
%    \begin{macrocode}
\include{cdocsch1}
\include{cdocsch2}
%    \end{macrocode}

% Include the two parts unless only chapters should be displayed:
%    \begin{macrocode}
\ifchilddoc\else
\section{part three}
\input{cdocspt3}
\section{part four}
\input{cdocspt4}
\fi
%    \end{macrocode}

% Process as usual until here:
%    \begin{macrocode}
\fi
%    \end{macrocode}

% End of document body:
%    \begin{macrocode}
\end{document}
%    \end{macrocode}
%\iffalse
%</samplemain>
%\fi
%
% %%%%%%%%%%%%%%%%%%%%%%%%%%%%%%%%%%%%%%
% \paragraph{Chapter Include Files.}
%
% The include files are called |cdocsch1.tex| and |cdocsch2.tex|.
%
%\iffalse
%<*samplechap1|samplechap2>
%\fi

% Optional override for |\version| flag:
%    \begin{macrocode}
%%\providecommand{\version}{final}
%    \end{macrocode}

% Include the main document:
%    \begin{macrocode}
\input{childdoc.def}
\childdocof{cdocsamp}
%    \end{macrocode}

%\iffalse
%</samplechap1|samplechap2>
%\fi
%
%\iffalse
%<*samplechap1>
%\fi
% Some text for chapter 1:
%    \begin{macrocode}
\section{one}
some text in chapter one
%    \end{macrocode}

%\iffalse
%</samplechap1>
%\fi
% Some text for chapter 2:
%\iffalse
%<*samplechap2>
%\fi
%    \begin{macrocode}
\section{two}
more text in chapter two
%    \end{macrocode}

%\iffalse
%</samplechap2>
%\fi
%
% %%%%%%%%%%%%%%%%%%%%%%%%%%%%%%%%%%%%%%
% \paragraph{Part Include Files.}
%
% The include files are called |cdocspt3.tex| and |cdocspt4.tex|.
%
%\iffalse
%<*samplepart3|samplepart4>
%\fi

% Optional override for |\version| flag:
%    \begin{macrocode}
%%\providecommand{\version}{final}
%    \end{macrocode}

% Include the main document:
%    \begin{macrocode}
\input{childdoc.def}
\childdocby{cdocsamp}
%    \end{macrocode}

%\iffalse
%</samplepart3|samplepart4>
%\fi
%
%\iffalse
%<*samplepart3>
%\fi
% Some text for part 3:
%    \begin{macrocode}
some text in part three
%    \end{macrocode}

%\iffalse
%</samplepart3>
%\fi
% Some text for part 4:
%\iffalse
%<*samplepart4>
%\fi
%    \begin{macrocode}
more text in part four
%    \end{macrocode}

%\iffalse
%</samplepart4>
%\fi
%
% %%%%%%%%%%%%%%%%%%%%%%%%%%%%%%%%%%%%%%
% \paragraph{Forwarding for a Complete Draft.}
%
% The following forwarding file |cdocsdrf.tex|
% compiles the main document in draft mode:
%\iffalse
%<*sampledraft>
%\fi
%    \begin{macrocode}
\def\version{draft}
\input{childdoc.def}
\childdocforward{cdocsamp}
%    \end{macrocode}

%\iffalse
%</sampledraft>
%\fi
%
% %%%%%%%%%%%%%%%%%%%%%%%%%%%%%%%%%%%%%%
% \paragraph{Forwarding for Final Version of the Chapters.}
%
% The following forwarding files |cdocsfn1.tex| and |cdocsfn2.tex|
% (with identical content)
% compile the final versions of the child documents
% |cdocsch1.tex| and |cdocsch2.tex|, respectively:
%\iffalse
%<*samplefinal>
%\fi
%    \begin{macrocode}
\def\version{final}
\input{childdoc.def}
\childdocforwardprefix[cdocsamp]{cdocsfn}{cdocsch}
%    \end{macrocode}

%\iffalse
%</samplefinal>
%\fi
%
% %%%%%%%%%%%%%%%%%%%%%%%%%%%%%%%%%%%%%%
% \paragraph{Command Line Processing.}
%
% The following three command lines generate the output files
% |cdocscld|, |cdocscl1| and |cdocscl2|
% which should be identical to
% |cdocsdrf|, |cdocsch1| and |cdocsfn2|, respectively:
% \begin{center}
% \begin{tabular}{l}
% |latex -jobname cdocscld \|\\
% |  "\def\version{draft}\input{childdoc.def}\childdocforward{cdocsamp}"|\\
% |latex -jobname cdocscl1 \|\\
% |  "\input{childdoc.def}\childdocforward[cdocsamp]{cdocsch1}"|\\
% |latex -jobname cdocscl2 \|\\
% |  "\def\version{final}\input{childdoc.def}\childdocforward{cdocsch2}"|
% \end{tabular}
% \end{center}
% Note that the trailing backslash on each first line
% merely continues the input to the second line
% (for convenient cut ant paste).
% Furthermore, the command |latex| can be replaced by any
% of its alternative versions such as |pdflatex|.
%
% %%%%%%%%%%%%%%%%%%%%%%%%%%%%%%%%%%%%%%%%%%%%%%%%%%%%%%%%%%%%%%%%%%%%%%%%%%%%%%
% %%%%%%%%%%%%%%%%%%%%%%%%%%%%%%%%%%%%%%%%%%%%%%%%%%%%%%%%%%%%%%%%%%%%%%%%%%%%%%
% \section{Implementation}
%\iffalse
%<*package>
%\fi
%
% This section describes the definitions file |childdoc.def|.

% The definitions cannot be loaded using |\usepackage| or |\RequirePackage|
% which has a mechanism to prevent loading a style file more than once.
% When loading the definitions by means of |\input|
% multiple instances have to be prevented manually:
%\iffalse
%This code needs to be before the `\ProvidesFile' directive
%which is defined at the beginning of this file.
%Therefore it is also placed there and commented out here.
%</package>
%<*discard>
%\fi
%    \begin{macrocode}
\ifdefined\childdocmain\endinput\fi
%    \end{macrocode}
%\iffalse
%</discard>
%<*package>
%\fi
%
% \macro{\ifchilddoc}
% \macro{\ifchilddocmanual}
% The conditional |\ifchilddoc| tells whether a
% child (true) or main (false) document is being compiled.
% The conditional |\ifchilddocmanual| tells whether
% the |\includeonly| mechanism is used (false) or
% the selection of child files must be performed manually (true).
% The definitions initialise to false:
%    \begin{macrocode}
\newif\ifchilddoc
\newif\ifchilddocmanual
%    \end{macrocode}

% \macro{\childdocname}
% \macro{\childdocjob}
% The macro |\childdocname| stores the name of the main document
% to be compiled. The macro |\childdocjob| stores the name of
% the document on which the \LaTeX{} compiler was originally invoked.
% The content of |\jobname| cannot be compared
% to filenames specified in the source due to different catcodes.
% The following code rescans |\jobname|, stores the result
% in |\childdocname| and saves a copy in |\childdocjob|:
%    \begin{macrocode}
\edef\childdocname{\scantokens\expandafter{\jobname\noexpand}}
\let\childdocjob\childdocname
%    \end{macrocode}

% \macro{\childdocdisable}
% The macro |\childdocdisable| prevents the main file
% from being processed more than once.
% At this stage, the main document command |\childdocmain|
% is assumed to be called once again where it should do nothing.
% Any subsequent call to it should prevent
% a secondary processing of the main document
% It overwrites the forwarding commands
% |\childdocof| and |\childdocforward|
% with empty macros to prevent further inclusions of the main document:
%    \begin{macrocode}
\newcommand{\childdocdisable}
{
  \renewcommand{\childdocmain}[1]{\renewcommand{\childdocmain}[1]{\endinput}}
  \renewcommand{\childdocof}[1]{}
  \renewcommand{\childdocby}[2][]{}
  \renewcommand{\childdocforward}[2][]{}
  \renewcommand{\childdocdisable}{}
}
%    \end{macrocode}

% \macro{\childdocmain}
% The macro |\childdocmain| is to be called at the top of the main file
% with nothing or the main filename (without extension) as argument.
% First, it breaks loops.
% If the argument is not empty and does not match |\childdocname|
% (which is set by the first inclusion of |childdoc.def|),
% |\ifchilddoc| is set to true, |\includeonly| is applied to the child file
% and |\jobname| is set to the main file
% (for proper handling of |.aux| files):
%    \begin{macrocode}
\newcommand{\childdocmain}[1]
{
  \childdocdisable\childdocmain{}
  \if?#1?\else
    \begingroup
      \def\childdoctmp{#1}
      \ifx\childdoctmp\childdocname
        \def\childdoctmp{}
      \else
        \def\childdoctmp
        {
          \childdoctrue
          \includeonly{\childdocname}
          \def\childdocjob{#1}
          \def\jobname{#1}
        }
      \fi
      \expandafter
    \endgroup
    \childdoctmp
  \fi
}
%    \end{macrocode}

% \macro{\childdocof}
% The command |\childdocof| redirects
% compilation to the main file |#1|.
%    \begin{macrocode}
\newcommand{\childdocof}[1]
{
  \childdocdisable
  \childdoctrue
  \includeonly{\childdocname}
  \def\jobname{#1}
  \def\childdocjob{#1}
  \input{#1}
}
%    \end{macrocode}

% \macro{\childdocby}
% The command |\childdocby| ....
%    \begin{macrocode}
\newcommand{\childdocby}[2][]
{
  \childdocdisable
  \childdoctrue
  \childdocmanualtrue
  \if?#1?\else
    \def\jobname{#2}
  \fi
  \def\childdocjob{#2}
  \input{#2}
  \endinput
}
%    \end{macrocode}

% \macro{\childdocforward}
% The command |\childdocforward| redirects
% compilation to the main file or
% (if the optional argument is given) a child file.
% Parameters are set as if the main file
% or a child file starting with |\childdocof| was compiled.
% Then compilation is handed over to the main file:
%    \begin{macrocode}
\newcommand{\childdocforward}[2][]
{
  \begingroup
    \if?#1?
      \def\childdoctmp
      {
        \def\childdocname{#2}
        \def\childdocjob{#2}
        \def\jobname{#2}
        \input{#2}
        \endinput
      }
    \else
      \def\childdoctmp
      {
        \childdocdisable
        \def\childdocname{#2}
        \childdoctrue
        \includeonly{#2}
        \def\childdocjob{#1}
        \def\jobname{#1}
        \input{#1}
        \endinput
      }
    \fi
    \expandafter
  \endgroup
  \childdoctmp
}
%    \end{macrocode}

% \macro{\childdocforwardprefix}
% The command |\childdocforwardprefix| redirects
% compilation to the main or a child file by means of a pattern.
% The prefix |#1| in the current filename is replaced by |#2|
% and the suffix of the current filename is kept
% (it is assumed that the filename does not contain the substring `|~~~|'
% which is used as a delimiter).
% Compilation is handed over to the new file by |\childdocforward|:
%    \begin{macrocode}
\newcommand{\childdocforwardprefix}[3][]
{
  \begingroup
    \def\childdocextract #2##1~~~{\def\childdoctmp{\childdocforward[#1]{#3##1}}}
    \expandafter\childdocextract\childdocname~~~
    \expandafter
  \endgroup
  \childdoctmp
}
%    \end{macrocode}

% \macro{\childdoc}
% The deprecated macro |\childdoc| is a legacy version of |\childdocmain|:
%    \begin{macrocode}
\newcommand{\childdoc}{\childdocmain}
%    \end{macrocode}

% \macro{\childdocredirect}
% The deprecated macro |\childdocredirect| is a legacy version
% of |\childdocforward| and |\childdocforwardprefix|:
%    \begin{macrocode}
\newcommand{\childdocredirect}[2][]
{
  \begingroup
    \if?#1?
      \def\childdoctmp{\childdocforward{#2}}
    \else
      \def\childdoctmp{\childdocforwardprefix{#1}{#2}}
    \fi
    \expandafter
  \endgroup
  \childdoctmp
}
%    \end{macrocode}

%\iffalse
%</package>
%\fi
%
\endinput
|\\
|\childdocforward{|\textit{main}|}|\\
\end{tabular}
\end{center}
%
or alternatively with:
%
\begin{center}
\begin{tabular}{l}
|% \iffalse
%
% childdoc.dtx Copyright (C) 2017-2018 Niklas Beisert
%
% This work may be distributed and/or modified under the
% conditions of the LaTeX Project Public License, either version 1.3
% of this license or (at your option) any later version.
% The latest version of this license is in
%   http://www.latex-project.org/lppl.txt
% and version 1.3 or later is part of all distributions of LaTeX
% version 2005/12/01 or later.
%
% This work has the LPPL maintenance status `maintained'.
%
% The Current Maintainer of this work is Niklas Beisert.
%
% This work consists of the files childdoc.dtx and childdoc.ins
% and the derived files childdoc.def and cdocsamp.tex with
% cdocsch1.tex, cdocsch2.tex, cdocsdrf.tex, cdocsfn1.tex, cdocsfn2.tex.
%
%<package>\ifdefined\childdocmain\endinput\fi
%<package>\ProvidesFile{childdoc.def}[2018/12/30 v2.0 child document driver]
%<samplemain>\ProvidesFile{cdocsamp.tex}[2018/12/30 v2.0 sample for childdoc]
%<*driver>
%\ProvidesFile{childdoc.drv}[2018/12/30 v2.0 childdoc reference manual file]
\PassOptionsToClass{10pt,a4paper}{article}
\documentclass{ltxdoc}

\usepackage[margin=35mm]{geometry}
\usepackage{hyperref}
\usepackage{hyperxmp}
\usepackage[usenames]{color}

\hypersetup{colorlinks=true}
\hypersetup{pdfstartview=FitH}
\hypersetup{pdfpagemode=UseNone}
\hypersetup{pdfsource={}}
\hypersetup{pdflang={en-UK}}
\hypersetup{pdfcopyright={Copyright 2017-2018 Niklas Beisert.
  This work may be distributed and/or modified under the
  conditions of the LaTeX Project Public License, either version 1.3
  of this license or (at your option) any later version.}}
\hypersetup{pdflicenseurl={http://www.latex-project.org/lppl.txt}}
\hypersetup{pdfcontactaddress={ETH Zurich, ITP, HIT K,
  Wolfgang-Pauli-Strasse 27}}
\hypersetup{pdfcontactpostcode={8093}}
\hypersetup{pdfcontactcity={Zurich}}
\hypersetup{pdfcontactcountry={Switzerland}}
\hypersetup{pdfcontactemail={nbeisert@itp.phys.ethz.ch}}
\hypersetup{pdfcontacturl={http://people.phys.ethz.ch/\xmptilde nbeisert/}}

\newcommand{\secref}[1]{\hyperref[#1]{section \ref*{#1}}}

\parskip1ex
\parindent0pt
\let\olditemize\itemize
\def\itemize{\olditemize\parskip0pt}

\begin{document}

\title{The \textsf{childdoc} Package}
\hypersetup{pdftitle={The childdoc Package}}
\author{Niklas Beisert\\[2ex]
  Institut f\"ur Theoretische Physik\\
  Eidgen\"ossische Technische Hochschule Z\"urich\\
  Wolfgang-Pauli-Strasse 27, 8093 Z\"urich, Switzerland\\[1ex]
  \href{mailto:nbeisert@itp.phys.ethz.ch}
  {\texttt{nbeisert@itp.phys.ethz.ch}}}
\hypersetup{pdfauthor={Niklas Beisert}}
\hypersetup{pdfsubject={Manual for the LaTeX2e Package childdoc}}
\date{30 December 2018, \textsf{v2.0}}
\maketitle

\begin{abstract}\noindent
\textsf{childdoc} is a \LaTeXe{} package
that enables the direct compilation
of document sections included by |\include|
to individual files.
\end{abstract}

\begingroup
\parskip0ex
\tableofcontents
\endgroup

%%%%%%%%%%%%%%%%%%%%%%%%%%%%%%%%%%%%%%%%%%%%%%%%%%%%%%%%%%%%%%%%%%%%%%%%%%%%%%%%
%%%%%%%%%%%%%%%%%%%%%%%%%%%%%%%%%%%%%%%%%%%%%%%%%%%%%%%%%%%%%%%%%%%%%%%%%%%%%%%%
\section{Introduction}

\LaTeX{} provides a mechanism to structure a large document (such as a book)
into a main file and several child files (containing the chapters)
using the |\include| command.
This mechanism is beneficial for documents
which span hundreds of pages in order to
make the source file(s) more manageable.
Moreover, compilation can be restricted to
selected child files by means of the |\includeonly| command.
The latter feature can be used to reduce the compilation time while editing
(this was significantly more useful in the earlier days of \LaTeX{})
or to generate a smaller document which is easier to navigate.
Another application of |\includeonly| is to generate
documents consisting of selected parts of the complete document.

However, there are a few drawbacks of the plain |\include| mechanism:
\begin{itemize}
\item
The child files cannot be compiled on their own,
they can only be compiled via the main file.
A naive editing environment
(such as a text editor with an option
to have the current file processed by \LaTeX)
may require one to switch to the main file before compiling;
attempting to compile the child file produces errors.
\item
The main file must be modified (each time)
to adjust the |\includeonly| command
to the present needs. This easily leaves the main file in a messy state.
\item
The generated document will always carry the filename
of the main document. This is inconvenient if
several child files are to be compiled and
to be kept for distribution.
\end{itemize}

The present package provides a simple interface
to make child files individually compilable by \LaTeX{}.
Compiling a child file then has the same effect as compiling
the main file with an |\includeonly| command
to select the appropriate child.
Moreover the generated document will carry the name of the child
rather than the main file.
This resolves all three above issues.

This feature is meant to make the editing of books,
thesis documents and lecture notes somewhat more convenient.
However, the package can also be used efficiently for
composing a series of documents (such as exercise sheets)
which are typically distributed individually.
It then assists the author in generating the individual documents
(potentially in different versions)
as well as a document containing the collected series.
Another application is in developing style files
or other kinds of included material
where compilation of the style file could redirect
to a sample or test file.

%%%%%%%%%%%%%%%%%%%%%%%%%%%%%%%%%%%%%%%%%%%%%%%%%%%%%%%%%%%%%%%%%%%%%%%%%%%%%%%%
%%%%%%%%%%%%%%%%%%%%%%%%%%%%%%%%%%%%%%%%%%%%%%%%%%%%%%%%%%%%%%%%%%%%%%%%%%%%%%%%
\section{Usage}

First of all, the package \textsf{childdoc} is \emph{not} a standard
\LaTeXe{} |.sty| style file! Therefore it needs to be invoked in
a non-standard way.

%%%%%%%%%%%%%%%%%%%%%%%%%%%%%%%%%%%%%%%%%%%%%%%%%%%%%%%%%%%%%%%%%%%%%%%%%%%%%%%%
\subsection{Included Files}
\label{sec:include}

%%%%%%%%%%%%%%%%%%%%%%%%%%%%%%%%%%%%%%%%
\DescribeMacro{\childdocmain}
To use the package, add the commands
\begin{center}
\begin{tabular}{l}
|\input{childdoc.def}|\\
|\childdocmain{}|\\
\end{tabular}
\end{center}
at the very top of the main \LaTeX{} file,
in particular \emph{before} the |\documentclass| statement!
The argument of |\childdocmain| should be left empty
(but it must be present).

%%%%%%%%%%%%%%%%%%%%%%%%%%%%%%%%%%%%%%%%
\DescribeMacro{\childdocof}
Furthermore, add the commands
\begin{center}
\begin{tabular}{l}
|\input{childdoc.def}|\\
|\childdocof{|\textit{main}|}|\\
\end{tabular}
\end{center}
at the top of every child file \textit{child}
which is included by |\include{|\textit{child}|}|
from within the main file
(or at least for those files to be compiled individually).
The argument \textit{main} must be the filename of the main file.

There are a couple of
considerations in setting up the main and child documents:

%%%%%%%%%%%%%%%%%%%%%%%%%%%%%%%%%%%%%%%%
\paragraph{Restrictions.}

Please note the following restrictions:
\begin{itemize}
\item
|\childdocmain| must be called with one argument \textit{main}
to ensure compatibility with earlier version of the package.
It must either be empty (|\childdocmain{}|)
or precisely match the filename of the main file in which it is specified.
See \secref{sec:detection} for further information.
\item
The filename \textit{main} must be specified without the |.tex| extension.
\item
The filename \textit{main} is case sensitive
(even in case-insensitive file systems)
due to internal string comparison.
\item
The argument \textit{main} should be fully expanded, it cannot be a macro.
\item
Subdirectories and special characters should be avoided in filenames.
\item
The command |\childdocmain{|\textit{main}|}| must be followed by a whitespace.
It should not be followed immediately by another command
or by a comment mark `|%|'.
This is because the \TeX{} parser reads the token immediately following
the argument of |\childdocmain| and puts it
at the beginning of every child section;
however, a white\-space is ignored.
\end{itemize}

%%%%%%%%%%%%%%%%%%%%%%%%%%%%%%%%%%%%%%%%
\paragraph{Content of Main File.}

It is advisable to place all content in the child files included by |\include|.
Any output contained in the main file will appear in all child documents
unless suppressed manually;
it cannot be suppressed automatically by the |\includeonly| directive
and thus should normally be avoided.
A method to include some content in the main file
by means of conditional processing is described in \secref{sec:conditional}.

%%%%%%%%%%%%%%%%%%%%%%%%%%%%%%%%%%%%%%%%
\paragraph{Page Numbering.}

When only a part of the document is compiled,
the appropriate numbering of pages
(as well as other status parameters)
is determined from the |.aux| files.
The latter contain information from previous passes.
However this information needs to propagate through
all intermediate child documents.
Therefore the page numbering in child documents may well
be inconsistent until the complete document is compiled at least once.

A useful (if unconventional) way to always ensure a consistent
page numbering is to restart the numbering in each child document
and denote the pages by `\textit{child}|.|\textit{page}'
where \textit{child} represents the chapter/section number of the child file.
This can be achieved by the command
|\numberwithin{page}{|\textit{child}|}|
of the \textsf{amsmath} package
where \textit{child} can be |chapter| or |section|
depending on the chosen structuring.
Alternatively, one can modify the macro |\thepage| appropriately
and reset the counter |page| at the start of each child file.

%%%%%%%%%%%%%%%%%%%%%%%%%%%%%%%%%%%%%%%%%%%%%%%%%%%%%%%%%%%%%%%%%%%%%%%%%%%%%%%%
\subsection{Conditional Processing}
\label{sec:conditional}

The package provides a mechanism to compile different versions
of a document. To customise the versions further some conditional processing
can come in handy to distinguish which version is being compiled.
The package provides two macros to describe the compilation context:

%%%%%%%%%%%%%%%%%%%%%%%%%%%%%%%%%%%%%%%%
\DescribeMacro{\ifchilddoc}
The conditional |\ifchilddoc| distinguishes between the compilation of
child documents and the main document:
%
\begin{center}
|\ifchilddoc |\textit{child-code}| |[|\||else |\textit{main-code}]| \||fi|
\end{center}

%%%%%%%%%%%%%%%%%%%%%%%%%%%%%%%%%%%%%%%%
\DescribeMacro{\childdocname}
\DescribeMacro{\childdocjob}
The macro |\childdocname| contains the filename (without extension)
of the main or child file being processed.
Note that |\childdocjob| will always contain the name of the main file.

%%%%%%%%%%%%%%%%%%%%%%%%%%%%%%%%%%%%%%%%
\paragraph{Title Page.}

Conditional processing can be used to include a title or banner page
in the main document when proper precautions are taken.
Importantly, the code in the main file should ensure that the page counter
(as well as other status parameters which are stored in the |.aux| files)
takes the same value after the conditional processing.
Otherwise the page numbers may take divergent values
depending on which part is compiled.

For example, a title page could be declared by:
%
\begin{center}
\begin{tabular}{l}
|\ifchilddoc\||else|\\
|\addtocounter{page}{-1}|\\
\textit{code for title page}\\
|\newpage|\\
|\||fi|
\end{tabular}
\end{center}
%
A banner page for the child documents can be generated by:
%
\begin{center}
\begin{tabular}{l}
|\ifchilddoc|\\
|\addtocounter{page}{-1}|\\
\textit{code for banner page}\\
|\newpage|\\
|\||fi|
\end{tabular}
\end{center}
%
Here one could write a message such as:
\begin{center}
|This is the part \childdocname{} of \childdocjob{}.|
\end{center}

%%%%%%%%%%%%%%%%%%%%%%%%%%%%%%%%%%%%%%%%%%%%%%%%%%%%%%%%%%%%%%%%%%%%%%%%%%%%%%%%
\subsection{Flags}
\label{sec:flags}

The package makes it easy to generate different versions
of the main or child documents.
To this end compilation flags can be defined
and assigned different default values.
They will be particularly useful in conjunction
with the forwarding mechanism described in \secref{sec:forward}.

For example, it may be useful to have a flag |\version|
which can be set to |draft| or |final|.
The document source will contain some conditional code
depending on the value of |\version|.
Suppose further, the flag should default to |final| for the main file
and to |draft| for child files
which is a natural assignment for editing the document.
This is achieved by placing the following code
in the preamble of the main document
(below the |\childdocmain| directive):
%
\begin{center}
\begin{tabular}{l}
|\ifchilddoc|\\
|\providecommand{\version}{draft}|\\
|\||else|\\
|\providecommand{\version}{final}|\\
|\||fi|
\end{tabular}
\end{center}
%
The definition by |\providecommand| makes sure
that previous definitions are not overwritten.
Further statements |\providecommand{\version}{...}|
can thus be added before the above code to override it.

For the main file, one might add a line
(between |\childdocmain| and the above block)
%
\begin{center}
|%\ifchilddoc\||else\providecommand{\version}{draft}\||fi|
\end{center}
%
which can be uncommented to produce a draft version.
Likewise one can add a line to the very top of a child file
(above the |\childdocof{|\textit{main}|}| directive)
%
\begin{center}
|%\providecommand{\version}{final}|
\end{center}
%
which can be uncommented to produce the final version of this child document.

%%%%%%%%%%%%%%%%%%%%%%%%%%%%%%%%%%%%%%%%%%%%%%%%%%%%%%%%%%%%%%%%%%%%%%%%%%%%%%%%
\subsection{Forwarding}
\label{sec:forward}

Different versions of the main or child documents
using compilation flags as described in \secref{sec:flags}
can be (permanently) stored in different files
for convenient compilation, viewing and distribution.
To this end, the package defines a command
to pass on compilation to a different file:

%%%%%%%%%%%%%%%%%%%%%%%%%%%%%%%%%%%%%%%%
\DescribeMacro{\childdocforward}
The command |\childdocforward| redirects processing to
another source file:
%
\begin{center}
\begin{tabular}{l}
|\input{childdoc.def}|\\
|\childdocforward[|\textit{main}|]{|\textit{dest}|}|\\
\end{tabular}
\end{center}
%
The argument \textit{dest} is the destination file
(without extension).
It should be the main file or one of the child files.
Note that further \textsf{childdoc} directives
such as |\childdocof| and |\childdocforward|
in the indicated file will be processed in this form.
The optional argument \textit{main}
passes on directly to the main file \textit{main}
while pretending to compile the child \textit{dest}.
This form behaves as if \textit{dest}
issues |\childdocof{|\textit{main}|}| right away,
and no further \textsf{childdoc} directives will be processed.

%%%%%%%%%%%%%%%%%%%%%%%%%%%%%%%%%%%%%%%%
\DescribeMacro{\...prefix}
In the alternative form |\childdocforwardprefix|,
%
\begin{center}
\begin{tabular}{l}
|\input{childdoc.def}|\\
|\childdocforwardprefix[|\textit{main}|]{|\textit{prefix}|}{|\textit{dest}|}|
\end{tabular}
\end{center}
%
the destination file is determined by a pattern
depending on the current file:
To make this work, the current file must be called
`{\textit{prefix}\hspace{0.2em}\textit{suffix}}'
with \textit{prefix} matching precisely the argument.
Processing is then passed on to the file
`{\textit{dest}\hspace{0.2em}\textit{suffix}}'.
Surely, the same effect is achieved by
directly specifying the
argument `{\textit{dest}\hspace{0.2em}\textit{suffix}}'
in the first form.
However, that requires to set up a different file
for each child. With the alternative form of the command
all these files can have exactly the same content
which simplifies setting them up and maintaining them.

For example, the following file |draft.tex|
with a compilation flag |\version| as described in \secref{sec:flags}
compiles the main document as a draft:
%
\begin{center}
\begin{tabular}{l}
|\def\version{draft}|\\
|\input{childdoc.def}|\\
|\childdocforward{|\textit{main}|}|
\end{tabular}
\end{center}
%
Likewise, the following files |final|\textit{nn}|.tex|
compile the final version of the child document
|child|\textit{nn}|.tex|:
%
\begin{center}
\begin{tabular}{l}
|\def\version{final}|\\
|\input{childdoc.def}|\\
|\childdocforwardprefix{final}{child}|
\end{tabular}
\end{center}
%

Note that when several versions of a main file and/or of each child file
are to be generated, it may be convenient to set up a |Makefile| or
shell script to automatise the process.

%%%%%%%%%%%%%%%%%%%%%%%%%%%%%%%%%%%%%%%%%%%%%%%%%%%%%%%%%%%%%%%%%%%%%%%%%%%%%%%%
\subsection{Command Line Processing}
\label{sec:commandline}

The effect of redirection files can also be achieved by invoking
the \LaTeX{} compiler with a more elaborate command line.
Most conveniently this should be done as part
of a shell script or a |Makefile|.

When using \textsf{childdoc} in the main file, the following
command lines effectively perform a redirection
(note that depending on the shell being used,
backslashes may have to be doubled: `|\|' $\to$ `|\\|'):
%
\begin{center}
|... -jobname "|\textit{target}|" |\\|"|[\textit{flags}]%
|\input{childdoc.def}\childdocforward[|\textit{main}|]{|\textit{dest}|}"|
\end{center}
%
Here \textit{target} is the name of the output file,
\textit{main} is the name of the main file
and \textit{dest} is the name of the main or child file to be processed
(all filenames without extensions).
The optional argument \textit{main} can be omitted
if \textit{main} matches \textit{dest}.
Optionally, compilation \textit{flags} can be defined via |\def| commands.
This command line makes the \TeX{} engine believe
it is compiling the file \textit{target}
whose content is specified as the latter parameter.
The provided code then forwards the processing to
\textit{main} or \textit{dest} as described in \secref{sec:forward}.

%%%%%%%%%%%%%%%%%%%%%%%%%%%%%%%%%%%%%%%%%%%%%%%%%%%%%%%%%%%%%%%%%%%%%%%%%%%%%%%%
\subsection{Include by Input}
\label{sec:input}

Including child documents by |\include| has some restrictions by design.
Most notably, the content of a child document always occupies
its own set of pages; pages cannot be shared between child documents.
Usually, this behaviour makes perfect sense
because each child document contain an essential part of the document.
However, in some situations it may be desirable to compose
a document from a collection of parts
without having mandatory page breaks between then.
For this case, the package
provides a mechanism to include parts
by |\input| which can also be processed individually.
However, by construction this mechanism
requires manual handling of the content to be output.

%%%%%%%%%%%%%%%%%%%%%%%%%%%%%%%%%%%%%%%%
\DescribeMacro{\ifchilddocmanual}
The main file should be prepared as usual, see \secref{sec:include}.
However, the document body must make a distinction
between processing of an individual part and of the main document, e.g.:
%
\begin{center}
\begin{tabular}{l}
|\ifchilddocmanual|\\
|\input{\childdocname}|\\
|\||else|\\
\textit{document body with }|\input{|\textit{part}|}|\\
|\||fi|
\end{tabular}
\end{center}
%
The conditional |\ifchilddocmanual| is true whenever
a part to be included by |\input| is being compiled,
and the name of the part is stored in |\childdocname|.

%%%%%%%%%%%%%%%%%%%%%%%%%%%%%%%%%%%%%%%%
\DescribeMacro{\childdocby}
Each part to be included by |\input| should start with:
%
\begin{center}
\begin{tabular}{l}
|\input{childdoc.def}|\\
|\childdocby{|\textit{main}|}|\\
\end{tabular}
\end{center}
%
The directive |\childdocby| is similar to |\childdocof|
described in \secref{sec:include},
but the subsequent selection of content must be done manually.
To that end, both |\ifchilddoc| and |\ifchilddocmanual|
will be true upon processing of a part,
and the name of the part is stored in |\childdocname|.
Note that |\jobname| will be set to the filename of the current part
so that each part receives an individual |.aux| file
that does not interfere with the |.aux| file(s) of the main document.
This behaviour can be altered by the alternative form
|\childdocby[*]{|\textit{main}|}| (with a non-empty optional argument)
which uses the |.aux| file of the main document
by setting |\jobname| to \textit{main}.

%%%%%%%%%%%%%%%%%%%%%%%%%%%%%%%%%%%%%%%%%%%%%%%%%%%%%%%%%%%%%%%%%%%%%%%%%%%%%%%%
\subsection{Driver Development}
\label{sec:driver}

The \textsf{childdoc} mechanism can also be use for the development
of definition files such as \LaTeX{} styles or classes.
This case differs from the above setup with multiple parts
included by |\include| in that no |\includeonly| should be invoked.
This can be achieved by starting the include file
(before |\ProvidesPackage|) with:
%
\begin{center}
\begin{tabular}{l}
|\input{childdoc.def}|\\
|\childdocforward{|\textit{main}|}|\\
\end{tabular}
\end{center}
%
or alternatively with:
%
\begin{center}
\begin{tabular}{l}
|\input{childdoc.def}|\\
|\childdocby{|\textit{main}|}|\\
\end{tabular}
\end{center}
%
Both forms have slightly different effects as described above.
The main file is prepared as usual, see \secref{sec:include}.

%%%%%%%%%%%%%%%%%%%%%%%%%%%%%%%%%%%%%%%%%%%%%%%%%%%%%%%%%%%%%%%%%%%%%%%%%%%%%%%%
\subsection{Legacy Detection}
\label{sec:detection}

The directive |\childdocmain| in the main file can detect
whether the complete document or merely a child is to be compiled
even without using the directive |\childdocof|.
This method is deprecated because it is less robust
and there is no compelling reason to use it;
it is merely provided for backward compatibility
and it may be removed in future versions.

If the detection mechanism is to be used,
it is mandatory to correctly specify
the filename of the main file as the argument of |\childdocmain|:
%
\begin{center}
\begin{tabular}{l}
|\input{childdoc.def}|\\
|\childdocmain{|\textit{main}|}|\\
\end{tabular}
\end{center}
%
If |\jobname| does not match the argument \textit{main} of |\childdocmain|,
it is assumed that |\jobname| points to the child file to be compiled.
When using |\childdocmain| with the main file specified as argument,
it suffices to start a child file
with just |\input{|\textit{main}|}|
without loading of the package and using |\childdocof|.
If instead all processing is done
with the appropriate \textsf{childdoc} directives,
the argument of \textit{main} of |\childdocmain| can be empty.

An alternative version of the command line processing described
in \secref{sec:commandline} using the detection mechanism reads:
%
\begin{center}
|... -jobname "|\textit{target}|" "|[\textit{flags}]%
[|\def\jobname{|\textit{dest}|}|]|\input{|\textit{main}|}"|
\end{center}

%%%%%%%%%%%%%%%%%%%%%%%%%%%%%%%%%%%%%%%%%%%%%%%%%%%%%%%%%%%%%%%%%%%%%%%%%%%%%%%%
\subsection{Manual Code}
\label{sec:manual}

In case one cannot be certain whether the definitions file |childdoc.def|
is installed on the target \TeX{} distribution
and one prefers not to ship it,
it is conceivable to paste a few relevant commands into the sources.

To that end, drop all statements |\input{childdoc.def}|
and perform the replacements as outlined below.
Instead of |\childdocmain{|\textit{main}|}| add the following code
to the top of the main file:
%
\begin{center}
\begin{tabular}{l}
|\||ifdefined\childdocname\endinput\||fi\newif\ifchilddoc|\\
|\edef\childdocname{\scantokens\expandafter{\jobname\noexpand}}|\\
|\def\childdocmain{|\textit{main}|}\||ifx\childdocmain\childdocname\||else|\\
|\childdoctrue\includeonly{\childdocname}\let\jobname\childdocmain\||fi|\\
\end{tabular}
\end{center}
%
Instead of |\childdocof{|\textit{main}|}| just include the main file
at the top of each child file:
%
\begin{center}
|\input{|\textit{main}|}|
\end{center}
%
A simple redirection |\childdocforward{|\textit{dest}|}| is achieved by:
%
\begin{center}
|\def\jobname{|\textit{dest}|}\input{\jobname}|
\end{center}
%
The redirection with prefix
|\childdocforwardprefix[|\textit{prefix}|]{|\textit{dest}|}|
is accomplished by:
%
\begin{center}
\begin{tabular}{l}
|{\edef\jobname{\scantokens\expandafter{\jobname\noexpand}}|\\
|\def\redirectjob |\textit{prefix}|#1~~~{\gdef\jobname{|\textit{dest}|#1}}|\\
|\expandafter\redirectjob\jobname~~~}\input{\jobname}|
\end{tabular}
\end{center}

In an alternative approach,
child documents can be compiled by a specific command line
without additional code or specific definitions:
%
\begin{center}
|... -jobname "|\textit{target}|" "|[\textit{flags}]%
|\includeonly{|\textit{dest}|}\input{|\textit{main}|}"|
\end{center}
%

%%%%%%%%%%%%%%%%%%%%%%%%%%%%%%%%%%%%%%%%%%%%%%%%%%%%%%%%%%%%%%%%%%%%%%%%%%%%%%%%
%%%%%%%%%%%%%%%%%%%%%%%%%%%%%%%%%%%%%%%%%%%%%%%%%%%%%%%%%%%%%%%%%%%%%%%%%%%%%%%%
\section{Information}

%%%%%%%%%%%%%%%%%%%%%%%%%%%%%%%%%%%%%%%%%%%%%%%%%%%%%%%%%%%%%%%%%%%%%%%%%%%%%%%%
\subsection{Copyright}

Copyright \copyright{} 2017--2018 Niklas Beisert

This work may be distributed and/or modified under the
conditions of the \LaTeX{} Project Public License, either version 1.3
of this license or (at your option) any later version.
The latest version of this license is in
  \url{http://www.latex-project.org/lppl.txt}
and version 1.3 or later is part of all distributions of \LaTeX{}
version 2005/12/01 or later.

This work has the LPPL maintenance status `maintained'.

The Current Maintainer of this work is Niklas Beisert.

This work consists of the files |README.txt|, |childdoc.ins| and |childdoc.dtx|
as well as the derived files |childdoc.def|, |cdocsamp.tex|
with |cdocsch1.tex|, |cdocsch2.tex|, |cdocspt3.tex|, |cdocspt4.tex|,
|cdocsdrf.tex|, |cdocsfn1.tex|, |cdocsfn2.tex|
as well as |childdoc.pdf|.

%%%%%%%%%%%%%%%%%%%%%%%%%%%%%%%%%%%%%%%%%%%%%%%%%%%%%%%%%%%%%%%%%%%%%%%%%%%%%%%%
\subsection{Files and Installation}

The package consists of the files:
%
\begin{center}
\begin{tabular}{ll}
    |README.txt|   & readme file \\
    |childdoc.ins| & installation file \\
    |childdoc.dtx| & source file \\
    |childdoc.def| & definition file \\
    |cdocsamp.tex| & sample main file \\
    |cdocsch1.tex| & sample include file \\
    |cdocsch2.tex| & sample include file \\
    |cdocspt3.tex| & sample part file \\
    |cdocspt4.tex| & sample part file \\
    |cdocsdrf.tex| & sample redirection file \\
    |cdocsfn1.tex| & sample redirection file \\
    |cdocsfn2.tex| & sample redirection file \\
    |childdoc.pdf| & manual
\end{tabular}
\end{center}
%
The distribution consists of the files
|README.txt|, |childdoc.ins| and |childdoc.dtx|.
%
\begin{itemize}
\item
Run (pdf)\LaTeX{} on |childdoc.dtx|
to compile the manual |childdoc.pdf| (this file).
\item
Run \LaTeX{} on |childdoc.ins| to create the definitions file |childdoc.def|
and the sample |cdocsamp.tex| with include files
|cdocsch1.tex|, |cdocsch2.tex|, |cdocspt3.tex|, |cdocspt4.tex|,
|cdocsdrf.tex|, |cdocsfn1.tex|, |cdocsfn2.tex|.
Then copy the file |childdoc.def| to an appropriate directory of your \LaTeX{}
distribution, e.g.\ \textit{texmf-root}|/tex/latex/childdoc|.
\end{itemize}

%%%%%%%%%%%%%%%%%%%%%%%%%%%%%%%%%%%%%%%%%%%%%%%%%%%%%%%%%%%%%%%%%%%%%%%%%%%%%%%%
\subsection{Related CTAN Packages}

There are several other packages which offer a similar functionality:
%
\begin{itemize}
\item
The packages
\href{http://ctan.org/pkg/docmute}{\textsf{docmute}},
\href{http://ctan.org/pkg/includex}{\textsf{includex}} and
\href{http://ctan.org/pkg/standalone}{\textsf{standalone}}
provide commands to include only the document body of
a child file thus allowing both files to be compiled individually.
\item
The packages \href{http://ctan.org/pkg/subdocs}{\textsf{subdocs}}
and \href{http://ctan.org/pkg/subfiles}{\textsf{subfiles}}
provide structures in which the main and child documents can be
encapsulated and allowing them to be compiled individually.
The inclusion mechanism is different from the conventional |\include|.
\item
The package \href{http://ctan.org/pkg/combine}{\textsf{combine}}
is an elaborate solution to combine several documents into one.
\end{itemize}
%
See also the CTAN topic \href{http://ctan.org/topic/subdocs}{\textsf{subdocs}}
for further related packages.
The present package differs from the above solutions in that
a document structure constructed with the conventional |\include| mechanism
just needs two extra commands at the top of every file
such that all constituent files can be compiled individually.

%%%%%%%%%%%%%%%%%%%%%%%%%%%%%%%%%%%%%%%%%%%%%%%%%%%%%%%%%%%%%%%%%%%%%%%%%%%%%%%%
%\subsection{Feature Suggestions}
%
%The following is a list of features which may be useful for future
%versions of this package:
%%
%\begin{itemize}
%\item
%\ldots
%\end{itemize}

%%%%%%%%%%%%%%%%%%%%%%%%%%%%%%%%%%%%%%%%%%%%%%%%%%%%%%%%%%%%%%%%%%%%%%%%%%%%%%%%
\subsection{Revision History}

%%%%%%%%%%%%%%%%%%%%%%%%%%%%%%%%%%%%%%%%
\paragraph{v2.0:} 2018/12/30

\begin{itemize}
\item
immediate forward processing
\item
added |\childdocby| mechanism
\item
manual restructured
\end{itemize}

%%%%%%%%%%%%%%%%%%%%%%%%%%%%%%%%%%%%%%%%
\paragraph{v1.6:} 2018/01/17

\begin{itemize}
\item
application for development of include files
\item
corrections to manual
\end{itemize}

%%%%%%%%%%%%%%%%%%%%%%%%%%%%%%%%%%%%%%%%
\paragraph{v1.5:} 2017/05/21

\begin{itemize}
\item
more complete structuring introduced
\item
|\childdocof| introduced
\item
|\childdoc| renamed to |\childdocmain|
\item
|\childredirect| renamed to |\childdocforward| and |\childdocforwardprefix|
and functionality expanded
\end{itemize}

%%%%%%%%%%%%%%%%%%%%%%%%%%%%%%%%%%%%%%%%
\paragraph{v1.0:} 2017/04/27

\begin{itemize}
\item
manual and install package
\item
first version published on CTAN
\end{itemize}

%%%%%%%%%%%%%%%%%%%%%%%%%%%%%%%%%%%%%%%%
\paragraph{v0.6:} 2017/04/26

\begin{itemize}
\item
redirection mechanism added
\end{itemize}

%%%%%%%%%%%%%%%%%%%%%%%%%%%%%%%%%%%%%%%%
\paragraph{v0.5:} 2017/04/26

\begin{itemize}
\item
functionality in definition file
\end{itemize}


%%%%%%%%%%%%%%%%%%%%%%%%%%%%%%%%%%%%%%%%%%%%%%%%%%%%%%%%%%%%%%%%%%%%%%%%%%%%%%%%
%%%%%%%%%%%%%%%%%%%%%%%%%%%%%%%%%%%%%%%%%%%%%%%%%%%%%%%%%%%%%%%%%%%%%%%%%%%%%%%%
%%%%%%%%%%%%%%%%%%%%%%%%%%%%%%%%%%%%%%%%%%%%%%%%%%%%%%%%%%%%%%%%%%%%%%%%%%%%%%%%
\appendix

\settowidth\MacroIndent{\rmfamily\scriptsize 000\ }

 \DocInput{childdoc.dtx}

\end{document}
%</driver>
% \fi
%
% %%%%%%%%%%%%%%%%%%%%%%%%%%%%%%%%%%%%%%%%%%%%%%%%%%%%%%%%%%%%%%%%%%%%%%%%%%%%%%
% %%%%%%%%%%%%%%%%%%%%%%%%%%%%%%%%%%%%%%%%%%%%%%%%%%%%%%%%%%%%%%%%%%%%%%%%%%%%%%
% \section{Sample}
%\iffalse
%<*samplemain>
%\fi
%
% The following presents a sample document
% with two chapters, two parts, a title page,
% a compile flag as well as three forwarding files to set the flag.
% It consists of eight |.tex| files:
% \begin{center}
% \begin{tabular}{ll}
% |cdocsamp.tex|&main file\\
% |cdocsch1.tex|&include file for chapter 1\\
% |cdocsch2.tex|&include file for chapter 2\\
% |cdocspt3.tex|&include file for part 3\\
% |cdocspt4.tex|&include file for part 4\\
% |cdocsdrf.tex|&forwarding file for main file in draft mode\\
% |cdocsfi1.tex|&forwarding file for final version of chapter 1\\
% |cdocsfi2.tex|&forwarding file for final version of chapter 2\\
% \end{tabular}
% \end{center}
% Each of the eight files can be compiled directly by the \LaTeX{} compiler.
%
% %%%%%%%%%%%%%%%%%%%%%%%%%%%%%%%%%%%%%%
% \paragraph{Main File.}
%
% The main file is called |cdocsamp.tex|.
%
% Load the \textsf{childdoc} definitions and
% declare the filename for the main document:
%    \begin{macrocode}
\input{childdoc.def}
\childdocmain{}
%    \end{macrocode}

% Optional override for |\version| flag:
%    \begin{macrocode}
%%\ifchilddoc\else\providecommand{\version}{draft}\fi
%    \end{macrocode}

% Define the default values for the |\version| flag
% (|final| for the main file and |draft| for childs):
%    \begin{macrocode}
\ifchilddoc
\providecommand{\version}{draft}
\else
\providecommand{\version}{final}
\fi
%    \end{macrocode}

% Load the standard document class:
%    \begin{macrocode}
\documentclass[12pt]{article}
%    \end{macrocode}

% Start the document body:
%    \begin{macrocode}
\begin{document}
%    \end{macrocode}

% Declare a title page.
% Print title, part of document being processed and version flag:
%    \begin{macrocode}
\addtocounter{page}{-1}
\begin{center}
{\LARGE\bfseries{}childdoc example\par}
\vspace{1cm}
\ifchilddoc
\ifchilddocmanual part\else chapter\fi:
`\childdocname' of `\childdocjob'\par
\else
main document: `\childdocjob'\par
\fi
version: \version\par
\end{center}
\newpage
%    \end{macrocode}

% Manually include selected file,
% otherwise process as usual:
%    \begin{macrocode}
\ifchilddocmanual
\section*{part `\childdocname'}
\input{\childdocname}
\else
%    \end{macrocode}

% Include the two chapters:
%    \begin{macrocode}
\include{cdocsch1}
\include{cdocsch2}
%    \end{macrocode}

% Include the two parts unless only chapters should be displayed:
%    \begin{macrocode}
\ifchilddoc\else
\section{part three}
\input{cdocspt3}
\section{part four}
\input{cdocspt4}
\fi
%    \end{macrocode}

% Process as usual until here:
%    \begin{macrocode}
\fi
%    \end{macrocode}

% End of document body:
%    \begin{macrocode}
\end{document}
%    \end{macrocode}
%\iffalse
%</samplemain>
%\fi
%
% %%%%%%%%%%%%%%%%%%%%%%%%%%%%%%%%%%%%%%
% \paragraph{Chapter Include Files.}
%
% The include files are called |cdocsch1.tex| and |cdocsch2.tex|.
%
%\iffalse
%<*samplechap1|samplechap2>
%\fi

% Optional override for |\version| flag:
%    \begin{macrocode}
%%\providecommand{\version}{final}
%    \end{macrocode}

% Include the main document:
%    \begin{macrocode}
\input{childdoc.def}
\childdocof{cdocsamp}
%    \end{macrocode}

%\iffalse
%</samplechap1|samplechap2>
%\fi
%
%\iffalse
%<*samplechap1>
%\fi
% Some text for chapter 1:
%    \begin{macrocode}
\section{one}
some text in chapter one
%    \end{macrocode}

%\iffalse
%</samplechap1>
%\fi
% Some text for chapter 2:
%\iffalse
%<*samplechap2>
%\fi
%    \begin{macrocode}
\section{two}
more text in chapter two
%    \end{macrocode}

%\iffalse
%</samplechap2>
%\fi
%
% %%%%%%%%%%%%%%%%%%%%%%%%%%%%%%%%%%%%%%
% \paragraph{Part Include Files.}
%
% The include files are called |cdocspt3.tex| and |cdocspt4.tex|.
%
%\iffalse
%<*samplepart3|samplepart4>
%\fi

% Optional override for |\version| flag:
%    \begin{macrocode}
%%\providecommand{\version}{final}
%    \end{macrocode}

% Include the main document:
%    \begin{macrocode}
\input{childdoc.def}
\childdocby{cdocsamp}
%    \end{macrocode}

%\iffalse
%</samplepart3|samplepart4>
%\fi
%
%\iffalse
%<*samplepart3>
%\fi
% Some text for part 3:
%    \begin{macrocode}
some text in part three
%    \end{macrocode}

%\iffalse
%</samplepart3>
%\fi
% Some text for part 4:
%\iffalse
%<*samplepart4>
%\fi
%    \begin{macrocode}
more text in part four
%    \end{macrocode}

%\iffalse
%</samplepart4>
%\fi
%
% %%%%%%%%%%%%%%%%%%%%%%%%%%%%%%%%%%%%%%
% \paragraph{Forwarding for a Complete Draft.}
%
% The following forwarding file |cdocsdrf.tex|
% compiles the main document in draft mode:
%\iffalse
%<*sampledraft>
%\fi
%    \begin{macrocode}
\def\version{draft}
\input{childdoc.def}
\childdocforward{cdocsamp}
%    \end{macrocode}

%\iffalse
%</sampledraft>
%\fi
%
% %%%%%%%%%%%%%%%%%%%%%%%%%%%%%%%%%%%%%%
% \paragraph{Forwarding for Final Version of the Chapters.}
%
% The following forwarding files |cdocsfn1.tex| and |cdocsfn2.tex|
% (with identical content)
% compile the final versions of the child documents
% |cdocsch1.tex| and |cdocsch2.tex|, respectively:
%\iffalse
%<*samplefinal>
%\fi
%    \begin{macrocode}
\def\version{final}
\input{childdoc.def}
\childdocforwardprefix[cdocsamp]{cdocsfn}{cdocsch}
%    \end{macrocode}

%\iffalse
%</samplefinal>
%\fi
%
% %%%%%%%%%%%%%%%%%%%%%%%%%%%%%%%%%%%%%%
% \paragraph{Command Line Processing.}
%
% The following three command lines generate the output files
% |cdocscld|, |cdocscl1| and |cdocscl2|
% which should be identical to
% |cdocsdrf|, |cdocsch1| and |cdocsfn2|, respectively:
% \begin{center}
% \begin{tabular}{l}
% |latex -jobname cdocscld \|\\
% |  "\def\version{draft}\input{childdoc.def}\childdocforward{cdocsamp}"|\\
% |latex -jobname cdocscl1 \|\\
% |  "\input{childdoc.def}\childdocforward[cdocsamp]{cdocsch1}"|\\
% |latex -jobname cdocscl2 \|\\
% |  "\def\version{final}\input{childdoc.def}\childdocforward{cdocsch2}"|
% \end{tabular}
% \end{center}
% Note that the trailing backslash on each first line
% merely continues the input to the second line
% (for convenient cut ant paste).
% Furthermore, the command |latex| can be replaced by any
% of its alternative versions such as |pdflatex|.
%
% %%%%%%%%%%%%%%%%%%%%%%%%%%%%%%%%%%%%%%%%%%%%%%%%%%%%%%%%%%%%%%%%%%%%%%%%%%%%%%
% %%%%%%%%%%%%%%%%%%%%%%%%%%%%%%%%%%%%%%%%%%%%%%%%%%%%%%%%%%%%%%%%%%%%%%%%%%%%%%
% \section{Implementation}
%\iffalse
%<*package>
%\fi
%
% This section describes the definitions file |childdoc.def|.

% The definitions cannot be loaded using |\usepackage| or |\RequirePackage|
% which has a mechanism to prevent loading a style file more than once.
% When loading the definitions by means of |\input|
% multiple instances have to be prevented manually:
%\iffalse
%This code needs to be before the `\ProvidesFile' directive
%which is defined at the beginning of this file.
%Therefore it is also placed there and commented out here.
%</package>
%<*discard>
%\fi
%    \begin{macrocode}
\ifdefined\childdocmain\endinput\fi
%    \end{macrocode}
%\iffalse
%</discard>
%<*package>
%\fi
%
% \macro{\ifchilddoc}
% \macro{\ifchilddocmanual}
% The conditional |\ifchilddoc| tells whether a
% child (true) or main (false) document is being compiled.
% The conditional |\ifchilddocmanual| tells whether
% the |\includeonly| mechanism is used (false) or
% the selection of child files must be performed manually (true).
% The definitions initialise to false:
%    \begin{macrocode}
\newif\ifchilddoc
\newif\ifchilddocmanual
%    \end{macrocode}

% \macro{\childdocname}
% \macro{\childdocjob}
% The macro |\childdocname| stores the name of the main document
% to be compiled. The macro |\childdocjob| stores the name of
% the document on which the \LaTeX{} compiler was originally invoked.
% The content of |\jobname| cannot be compared
% to filenames specified in the source due to different catcodes.
% The following code rescans |\jobname|, stores the result
% in |\childdocname| and saves a copy in |\childdocjob|:
%    \begin{macrocode}
\edef\childdocname{\scantokens\expandafter{\jobname\noexpand}}
\let\childdocjob\childdocname
%    \end{macrocode}

% \macro{\childdocdisable}
% The macro |\childdocdisable| prevents the main file
% from being processed more than once.
% At this stage, the main document command |\childdocmain|
% is assumed to be called once again where it should do nothing.
% Any subsequent call to it should prevent
% a secondary processing of the main document
% It overwrites the forwarding commands
% |\childdocof| and |\childdocforward|
% with empty macros to prevent further inclusions of the main document:
%    \begin{macrocode}
\newcommand{\childdocdisable}
{
  \renewcommand{\childdocmain}[1]{\renewcommand{\childdocmain}[1]{\endinput}}
  \renewcommand{\childdocof}[1]{}
  \renewcommand{\childdocby}[2][]{}
  \renewcommand{\childdocforward}[2][]{}
  \renewcommand{\childdocdisable}{}
}
%    \end{macrocode}

% \macro{\childdocmain}
% The macro |\childdocmain| is to be called at the top of the main file
% with nothing or the main filename (without extension) as argument.
% First, it breaks loops.
% If the argument is not empty and does not match |\childdocname|
% (which is set by the first inclusion of |childdoc.def|),
% |\ifchilddoc| is set to true, |\includeonly| is applied to the child file
% and |\jobname| is set to the main file
% (for proper handling of |.aux| files):
%    \begin{macrocode}
\newcommand{\childdocmain}[1]
{
  \childdocdisable\childdocmain{}
  \if?#1?\else
    \begingroup
      \def\childdoctmp{#1}
      \ifx\childdoctmp\childdocname
        \def\childdoctmp{}
      \else
        \def\childdoctmp
        {
          \childdoctrue
          \includeonly{\childdocname}
          \def\childdocjob{#1}
          \def\jobname{#1}
        }
      \fi
      \expandafter
    \endgroup
    \childdoctmp
  \fi
}
%    \end{macrocode}

% \macro{\childdocof}
% The command |\childdocof| redirects
% compilation to the main file |#1|.
%    \begin{macrocode}
\newcommand{\childdocof}[1]
{
  \childdocdisable
  \childdoctrue
  \includeonly{\childdocname}
  \def\jobname{#1}
  \def\childdocjob{#1}
  \input{#1}
}
%    \end{macrocode}

% \macro{\childdocby}
% The command |\childdocby| ....
%    \begin{macrocode}
\newcommand{\childdocby}[2][]
{
  \childdocdisable
  \childdoctrue
  \childdocmanualtrue
  \if?#1?\else
    \def\jobname{#2}
  \fi
  \def\childdocjob{#2}
  \input{#2}
  \endinput
}
%    \end{macrocode}

% \macro{\childdocforward}
% The command |\childdocforward| redirects
% compilation to the main file or
% (if the optional argument is given) a child file.
% Parameters are set as if the main file
% or a child file starting with |\childdocof| was compiled.
% Then compilation is handed over to the main file:
%    \begin{macrocode}
\newcommand{\childdocforward}[2][]
{
  \begingroup
    \if?#1?
      \def\childdoctmp
      {
        \def\childdocname{#2}
        \def\childdocjob{#2}
        \def\jobname{#2}
        \input{#2}
        \endinput
      }
    \else
      \def\childdoctmp
      {
        \childdocdisable
        \def\childdocname{#2}
        \childdoctrue
        \includeonly{#2}
        \def\childdocjob{#1}
        \def\jobname{#1}
        \input{#1}
        \endinput
      }
    \fi
    \expandafter
  \endgroup
  \childdoctmp
}
%    \end{macrocode}

% \macro{\childdocforwardprefix}
% The command |\childdocforwardprefix| redirects
% compilation to the main or a child file by means of a pattern.
% The prefix |#1| in the current filename is replaced by |#2|
% and the suffix of the current filename is kept
% (it is assumed that the filename does not contain the substring `|~~~|'
% which is used as a delimiter).
% Compilation is handed over to the new file by |\childdocforward|:
%    \begin{macrocode}
\newcommand{\childdocforwardprefix}[3][]
{
  \begingroup
    \def\childdocextract #2##1~~~{\def\childdoctmp{\childdocforward[#1]{#3##1}}}
    \expandafter\childdocextract\childdocname~~~
    \expandafter
  \endgroup
  \childdoctmp
}
%    \end{macrocode}

% \macro{\childdoc}
% The deprecated macro |\childdoc| is a legacy version of |\childdocmain|:
%    \begin{macrocode}
\newcommand{\childdoc}{\childdocmain}
%    \end{macrocode}

% \macro{\childdocredirect}
% The deprecated macro |\childdocredirect| is a legacy version
% of |\childdocforward| and |\childdocforwardprefix|:
%    \begin{macrocode}
\newcommand{\childdocredirect}[2][]
{
  \begingroup
    \if?#1?
      \def\childdoctmp{\childdocforward{#2}}
    \else
      \def\childdoctmp{\childdocforwardprefix{#1}{#2}}
    \fi
    \expandafter
  \endgroup
  \childdoctmp
}
%    \end{macrocode}

%\iffalse
%</package>
%\fi
%
\endinput
|\\
|\childdocby{|\textit{main}|}|\\
\end{tabular}
\end{center}
%
Both forms have slightly different effects as described above.
The main file is prepared as usual, see \secref{sec:include}.

%%%%%%%%%%%%%%%%%%%%%%%%%%%%%%%%%%%%%%%%%%%%%%%%%%%%%%%%%%%%%%%%%%%%%%%%%%%%%%%%
\subsection{Legacy Detection}
\label{sec:detection}

The directive |\childdocmain| in the main file can detect
whether the complete document or merely a child is to be compiled
even without using the directive |\childdocof|.
This method is deprecated because it is less robust
and there is no compelling reason to use it;
it is merely provided for backward compatibility
and it may be removed in future versions.

If the detection mechanism is to be used,
it is mandatory to correctly specify
the filename of the main file as the argument of |\childdocmain|:
%
\begin{center}
\begin{tabular}{l}
|% \iffalse
%
% childdoc.dtx Copyright (C) 2017-2018 Niklas Beisert
%
% This work may be distributed and/or modified under the
% conditions of the LaTeX Project Public License, either version 1.3
% of this license or (at your option) any later version.
% The latest version of this license is in
%   http://www.latex-project.org/lppl.txt
% and version 1.3 or later is part of all distributions of LaTeX
% version 2005/12/01 or later.
%
% This work has the LPPL maintenance status `maintained'.
%
% The Current Maintainer of this work is Niklas Beisert.
%
% This work consists of the files childdoc.dtx and childdoc.ins
% and the derived files childdoc.def and cdocsamp.tex with
% cdocsch1.tex, cdocsch2.tex, cdocsdrf.tex, cdocsfn1.tex, cdocsfn2.tex.
%
%<package>\ifdefined\childdocmain\endinput\fi
%<package>\ProvidesFile{childdoc.def}[2018/12/30 v2.0 child document driver]
%<samplemain>\ProvidesFile{cdocsamp.tex}[2018/12/30 v2.0 sample for childdoc]
%<*driver>
%\ProvidesFile{childdoc.drv}[2018/12/30 v2.0 childdoc reference manual file]
\PassOptionsToClass{10pt,a4paper}{article}
\documentclass{ltxdoc}

\usepackage[margin=35mm]{geometry}
\usepackage{hyperref}
\usepackage{hyperxmp}
\usepackage[usenames]{color}

\hypersetup{colorlinks=true}
\hypersetup{pdfstartview=FitH}
\hypersetup{pdfpagemode=UseNone}
\hypersetup{pdfsource={}}
\hypersetup{pdflang={en-UK}}
\hypersetup{pdfcopyright={Copyright 2017-2018 Niklas Beisert.
  This work may be distributed and/or modified under the
  conditions of the LaTeX Project Public License, either version 1.3
  of this license or (at your option) any later version.}}
\hypersetup{pdflicenseurl={http://www.latex-project.org/lppl.txt}}
\hypersetup{pdfcontactaddress={ETH Zurich, ITP, HIT K,
  Wolfgang-Pauli-Strasse 27}}
\hypersetup{pdfcontactpostcode={8093}}
\hypersetup{pdfcontactcity={Zurich}}
\hypersetup{pdfcontactcountry={Switzerland}}
\hypersetup{pdfcontactemail={nbeisert@itp.phys.ethz.ch}}
\hypersetup{pdfcontacturl={http://people.phys.ethz.ch/\xmptilde nbeisert/}}

\newcommand{\secref}[1]{\hyperref[#1]{section \ref*{#1}}}

\parskip1ex
\parindent0pt
\let\olditemize\itemize
\def\itemize{\olditemize\parskip0pt}

\begin{document}

\title{The \textsf{childdoc} Package}
\hypersetup{pdftitle={The childdoc Package}}
\author{Niklas Beisert\\[2ex]
  Institut f\"ur Theoretische Physik\\
  Eidgen\"ossische Technische Hochschule Z\"urich\\
  Wolfgang-Pauli-Strasse 27, 8093 Z\"urich, Switzerland\\[1ex]
  \href{mailto:nbeisert@itp.phys.ethz.ch}
  {\texttt{nbeisert@itp.phys.ethz.ch}}}
\hypersetup{pdfauthor={Niklas Beisert}}
\hypersetup{pdfsubject={Manual for the LaTeX2e Package childdoc}}
\date{30 December 2018, \textsf{v2.0}}
\maketitle

\begin{abstract}\noindent
\textsf{childdoc} is a \LaTeXe{} package
that enables the direct compilation
of document sections included by |\include|
to individual files.
\end{abstract}

\begingroup
\parskip0ex
\tableofcontents
\endgroup

%%%%%%%%%%%%%%%%%%%%%%%%%%%%%%%%%%%%%%%%%%%%%%%%%%%%%%%%%%%%%%%%%%%%%%%%%%%%%%%%
%%%%%%%%%%%%%%%%%%%%%%%%%%%%%%%%%%%%%%%%%%%%%%%%%%%%%%%%%%%%%%%%%%%%%%%%%%%%%%%%
\section{Introduction}

\LaTeX{} provides a mechanism to structure a large document (such as a book)
into a main file and several child files (containing the chapters)
using the |\include| command.
This mechanism is beneficial for documents
which span hundreds of pages in order to
make the source file(s) more manageable.
Moreover, compilation can be restricted to
selected child files by means of the |\includeonly| command.
The latter feature can be used to reduce the compilation time while editing
(this was significantly more useful in the earlier days of \LaTeX{})
or to generate a smaller document which is easier to navigate.
Another application of |\includeonly| is to generate
documents consisting of selected parts of the complete document.

However, there are a few drawbacks of the plain |\include| mechanism:
\begin{itemize}
\item
The child files cannot be compiled on their own,
they can only be compiled via the main file.
A naive editing environment
(such as a text editor with an option
to have the current file processed by \LaTeX)
may require one to switch to the main file before compiling;
attempting to compile the child file produces errors.
\item
The main file must be modified (each time)
to adjust the |\includeonly| command
to the present needs. This easily leaves the main file in a messy state.
\item
The generated document will always carry the filename
of the main document. This is inconvenient if
several child files are to be compiled and
to be kept for distribution.
\end{itemize}

The present package provides a simple interface
to make child files individually compilable by \LaTeX{}.
Compiling a child file then has the same effect as compiling
the main file with an |\includeonly| command
to select the appropriate child.
Moreover the generated document will carry the name of the child
rather than the main file.
This resolves all three above issues.

This feature is meant to make the editing of books,
thesis documents and lecture notes somewhat more convenient.
However, the package can also be used efficiently for
composing a series of documents (such as exercise sheets)
which are typically distributed individually.
It then assists the author in generating the individual documents
(potentially in different versions)
as well as a document containing the collected series.
Another application is in developing style files
or other kinds of included material
where compilation of the style file could redirect
to a sample or test file.

%%%%%%%%%%%%%%%%%%%%%%%%%%%%%%%%%%%%%%%%%%%%%%%%%%%%%%%%%%%%%%%%%%%%%%%%%%%%%%%%
%%%%%%%%%%%%%%%%%%%%%%%%%%%%%%%%%%%%%%%%%%%%%%%%%%%%%%%%%%%%%%%%%%%%%%%%%%%%%%%%
\section{Usage}

First of all, the package \textsf{childdoc} is \emph{not} a standard
\LaTeXe{} |.sty| style file! Therefore it needs to be invoked in
a non-standard way.

%%%%%%%%%%%%%%%%%%%%%%%%%%%%%%%%%%%%%%%%%%%%%%%%%%%%%%%%%%%%%%%%%%%%%%%%%%%%%%%%
\subsection{Included Files}
\label{sec:include}

%%%%%%%%%%%%%%%%%%%%%%%%%%%%%%%%%%%%%%%%
\DescribeMacro{\childdocmain}
To use the package, add the commands
\begin{center}
\begin{tabular}{l}
|\input{childdoc.def}|\\
|\childdocmain{}|\\
\end{tabular}
\end{center}
at the very top of the main \LaTeX{} file,
in particular \emph{before} the |\documentclass| statement!
The argument of |\childdocmain| should be left empty
(but it must be present).

%%%%%%%%%%%%%%%%%%%%%%%%%%%%%%%%%%%%%%%%
\DescribeMacro{\childdocof}
Furthermore, add the commands
\begin{center}
\begin{tabular}{l}
|\input{childdoc.def}|\\
|\childdocof{|\textit{main}|}|\\
\end{tabular}
\end{center}
at the top of every child file \textit{child}
which is included by |\include{|\textit{child}|}|
from within the main file
(or at least for those files to be compiled individually).
The argument \textit{main} must be the filename of the main file.

There are a couple of
considerations in setting up the main and child documents:

%%%%%%%%%%%%%%%%%%%%%%%%%%%%%%%%%%%%%%%%
\paragraph{Restrictions.}

Please note the following restrictions:
\begin{itemize}
\item
|\childdocmain| must be called with one argument \textit{main}
to ensure compatibility with earlier version of the package.
It must either be empty (|\childdocmain{}|)
or precisely match the filename of the main file in which it is specified.
See \secref{sec:detection} for further information.
\item
The filename \textit{main} must be specified without the |.tex| extension.
\item
The filename \textit{main} is case sensitive
(even in case-insensitive file systems)
due to internal string comparison.
\item
The argument \textit{main} should be fully expanded, it cannot be a macro.
\item
Subdirectories and special characters should be avoided in filenames.
\item
The command |\childdocmain{|\textit{main}|}| must be followed by a whitespace.
It should not be followed immediately by another command
or by a comment mark `|%|'.
This is because the \TeX{} parser reads the token immediately following
the argument of |\childdocmain| and puts it
at the beginning of every child section;
however, a white\-space is ignored.
\end{itemize}

%%%%%%%%%%%%%%%%%%%%%%%%%%%%%%%%%%%%%%%%
\paragraph{Content of Main File.}

It is advisable to place all content in the child files included by |\include|.
Any output contained in the main file will appear in all child documents
unless suppressed manually;
it cannot be suppressed automatically by the |\includeonly| directive
and thus should normally be avoided.
A method to include some content in the main file
by means of conditional processing is described in \secref{sec:conditional}.

%%%%%%%%%%%%%%%%%%%%%%%%%%%%%%%%%%%%%%%%
\paragraph{Page Numbering.}

When only a part of the document is compiled,
the appropriate numbering of pages
(as well as other status parameters)
is determined from the |.aux| files.
The latter contain information from previous passes.
However this information needs to propagate through
all intermediate child documents.
Therefore the page numbering in child documents may well
be inconsistent until the complete document is compiled at least once.

A useful (if unconventional) way to always ensure a consistent
page numbering is to restart the numbering in each child document
and denote the pages by `\textit{child}|.|\textit{page}'
where \textit{child} represents the chapter/section number of the child file.
This can be achieved by the command
|\numberwithin{page}{|\textit{child}|}|
of the \textsf{amsmath} package
where \textit{child} can be |chapter| or |section|
depending on the chosen structuring.
Alternatively, one can modify the macro |\thepage| appropriately
and reset the counter |page| at the start of each child file.

%%%%%%%%%%%%%%%%%%%%%%%%%%%%%%%%%%%%%%%%%%%%%%%%%%%%%%%%%%%%%%%%%%%%%%%%%%%%%%%%
\subsection{Conditional Processing}
\label{sec:conditional}

The package provides a mechanism to compile different versions
of a document. To customise the versions further some conditional processing
can come in handy to distinguish which version is being compiled.
The package provides two macros to describe the compilation context:

%%%%%%%%%%%%%%%%%%%%%%%%%%%%%%%%%%%%%%%%
\DescribeMacro{\ifchilddoc}
The conditional |\ifchilddoc| distinguishes between the compilation of
child documents and the main document:
%
\begin{center}
|\ifchilddoc |\textit{child-code}| |[|\||else |\textit{main-code}]| \||fi|
\end{center}

%%%%%%%%%%%%%%%%%%%%%%%%%%%%%%%%%%%%%%%%
\DescribeMacro{\childdocname}
\DescribeMacro{\childdocjob}
The macro |\childdocname| contains the filename (without extension)
of the main or child file being processed.
Note that |\childdocjob| will always contain the name of the main file.

%%%%%%%%%%%%%%%%%%%%%%%%%%%%%%%%%%%%%%%%
\paragraph{Title Page.}

Conditional processing can be used to include a title or banner page
in the main document when proper precautions are taken.
Importantly, the code in the main file should ensure that the page counter
(as well as other status parameters which are stored in the |.aux| files)
takes the same value after the conditional processing.
Otherwise the page numbers may take divergent values
depending on which part is compiled.

For example, a title page could be declared by:
%
\begin{center}
\begin{tabular}{l}
|\ifchilddoc\||else|\\
|\addtocounter{page}{-1}|\\
\textit{code for title page}\\
|\newpage|\\
|\||fi|
\end{tabular}
\end{center}
%
A banner page for the child documents can be generated by:
%
\begin{center}
\begin{tabular}{l}
|\ifchilddoc|\\
|\addtocounter{page}{-1}|\\
\textit{code for banner page}\\
|\newpage|\\
|\||fi|
\end{tabular}
\end{center}
%
Here one could write a message such as:
\begin{center}
|This is the part \childdocname{} of \childdocjob{}.|
\end{center}

%%%%%%%%%%%%%%%%%%%%%%%%%%%%%%%%%%%%%%%%%%%%%%%%%%%%%%%%%%%%%%%%%%%%%%%%%%%%%%%%
\subsection{Flags}
\label{sec:flags}

The package makes it easy to generate different versions
of the main or child documents.
To this end compilation flags can be defined
and assigned different default values.
They will be particularly useful in conjunction
with the forwarding mechanism described in \secref{sec:forward}.

For example, it may be useful to have a flag |\version|
which can be set to |draft| or |final|.
The document source will contain some conditional code
depending on the value of |\version|.
Suppose further, the flag should default to |final| for the main file
and to |draft| for child files
which is a natural assignment for editing the document.
This is achieved by placing the following code
in the preamble of the main document
(below the |\childdocmain| directive):
%
\begin{center}
\begin{tabular}{l}
|\ifchilddoc|\\
|\providecommand{\version}{draft}|\\
|\||else|\\
|\providecommand{\version}{final}|\\
|\||fi|
\end{tabular}
\end{center}
%
The definition by |\providecommand| makes sure
that previous definitions are not overwritten.
Further statements |\providecommand{\version}{...}|
can thus be added before the above code to override it.

For the main file, one might add a line
(between |\childdocmain| and the above block)
%
\begin{center}
|%\ifchilddoc\||else\providecommand{\version}{draft}\||fi|
\end{center}
%
which can be uncommented to produce a draft version.
Likewise one can add a line to the very top of a child file
(above the |\childdocof{|\textit{main}|}| directive)
%
\begin{center}
|%\providecommand{\version}{final}|
\end{center}
%
which can be uncommented to produce the final version of this child document.

%%%%%%%%%%%%%%%%%%%%%%%%%%%%%%%%%%%%%%%%%%%%%%%%%%%%%%%%%%%%%%%%%%%%%%%%%%%%%%%%
\subsection{Forwarding}
\label{sec:forward}

Different versions of the main or child documents
using compilation flags as described in \secref{sec:flags}
can be (permanently) stored in different files
for convenient compilation, viewing and distribution.
To this end, the package defines a command
to pass on compilation to a different file:

%%%%%%%%%%%%%%%%%%%%%%%%%%%%%%%%%%%%%%%%
\DescribeMacro{\childdocforward}
The command |\childdocforward| redirects processing to
another source file:
%
\begin{center}
\begin{tabular}{l}
|\input{childdoc.def}|\\
|\childdocforward[|\textit{main}|]{|\textit{dest}|}|\\
\end{tabular}
\end{center}
%
The argument \textit{dest} is the destination file
(without extension).
It should be the main file or one of the child files.
Note that further \textsf{childdoc} directives
such as |\childdocof| and |\childdocforward|
in the indicated file will be processed in this form.
The optional argument \textit{main}
passes on directly to the main file \textit{main}
while pretending to compile the child \textit{dest}.
This form behaves as if \textit{dest}
issues |\childdocof{|\textit{main}|}| right away,
and no further \textsf{childdoc} directives will be processed.

%%%%%%%%%%%%%%%%%%%%%%%%%%%%%%%%%%%%%%%%
\DescribeMacro{\...prefix}
In the alternative form |\childdocforwardprefix|,
%
\begin{center}
\begin{tabular}{l}
|\input{childdoc.def}|\\
|\childdocforwardprefix[|\textit{main}|]{|\textit{prefix}|}{|\textit{dest}|}|
\end{tabular}
\end{center}
%
the destination file is determined by a pattern
depending on the current file:
To make this work, the current file must be called
`{\textit{prefix}\hspace{0.2em}\textit{suffix}}'
with \textit{prefix} matching precisely the argument.
Processing is then passed on to the file
`{\textit{dest}\hspace{0.2em}\textit{suffix}}'.
Surely, the same effect is achieved by
directly specifying the
argument `{\textit{dest}\hspace{0.2em}\textit{suffix}}'
in the first form.
However, that requires to set up a different file
for each child. With the alternative form of the command
all these files can have exactly the same content
which simplifies setting them up and maintaining them.

For example, the following file |draft.tex|
with a compilation flag |\version| as described in \secref{sec:flags}
compiles the main document as a draft:
%
\begin{center}
\begin{tabular}{l}
|\def\version{draft}|\\
|\input{childdoc.def}|\\
|\childdocforward{|\textit{main}|}|
\end{tabular}
\end{center}
%
Likewise, the following files |final|\textit{nn}|.tex|
compile the final version of the child document
|child|\textit{nn}|.tex|:
%
\begin{center}
\begin{tabular}{l}
|\def\version{final}|\\
|\input{childdoc.def}|\\
|\childdocforwardprefix{final}{child}|
\end{tabular}
\end{center}
%

Note that when several versions of a main file and/or of each child file
are to be generated, it may be convenient to set up a |Makefile| or
shell script to automatise the process.

%%%%%%%%%%%%%%%%%%%%%%%%%%%%%%%%%%%%%%%%%%%%%%%%%%%%%%%%%%%%%%%%%%%%%%%%%%%%%%%%
\subsection{Command Line Processing}
\label{sec:commandline}

The effect of redirection files can also be achieved by invoking
the \LaTeX{} compiler with a more elaborate command line.
Most conveniently this should be done as part
of a shell script or a |Makefile|.

When using \textsf{childdoc} in the main file, the following
command lines effectively perform a redirection
(note that depending on the shell being used,
backslashes may have to be doubled: `|\|' $\to$ `|\\|'):
%
\begin{center}
|... -jobname "|\textit{target}|" |\\|"|[\textit{flags}]%
|\input{childdoc.def}\childdocforward[|\textit{main}|]{|\textit{dest}|}"|
\end{center}
%
Here \textit{target} is the name of the output file,
\textit{main} is the name of the main file
and \textit{dest} is the name of the main or child file to be processed
(all filenames without extensions).
The optional argument \textit{main} can be omitted
if \textit{main} matches \textit{dest}.
Optionally, compilation \textit{flags} can be defined via |\def| commands.
This command line makes the \TeX{} engine believe
it is compiling the file \textit{target}
whose content is specified as the latter parameter.
The provided code then forwards the processing to
\textit{main} or \textit{dest} as described in \secref{sec:forward}.

%%%%%%%%%%%%%%%%%%%%%%%%%%%%%%%%%%%%%%%%%%%%%%%%%%%%%%%%%%%%%%%%%%%%%%%%%%%%%%%%
\subsection{Include by Input}
\label{sec:input}

Including child documents by |\include| has some restrictions by design.
Most notably, the content of a child document always occupies
its own set of pages; pages cannot be shared between child documents.
Usually, this behaviour makes perfect sense
because each child document contain an essential part of the document.
However, in some situations it may be desirable to compose
a document from a collection of parts
without having mandatory page breaks between then.
For this case, the package
provides a mechanism to include parts
by |\input| which can also be processed individually.
However, by construction this mechanism
requires manual handling of the content to be output.

%%%%%%%%%%%%%%%%%%%%%%%%%%%%%%%%%%%%%%%%
\DescribeMacro{\ifchilddocmanual}
The main file should be prepared as usual, see \secref{sec:include}.
However, the document body must make a distinction
between processing of an individual part and of the main document, e.g.:
%
\begin{center}
\begin{tabular}{l}
|\ifchilddocmanual|\\
|\input{\childdocname}|\\
|\||else|\\
\textit{document body with }|\input{|\textit{part}|}|\\
|\||fi|
\end{tabular}
\end{center}
%
The conditional |\ifchilddocmanual| is true whenever
a part to be included by |\input| is being compiled,
and the name of the part is stored in |\childdocname|.

%%%%%%%%%%%%%%%%%%%%%%%%%%%%%%%%%%%%%%%%
\DescribeMacro{\childdocby}
Each part to be included by |\input| should start with:
%
\begin{center}
\begin{tabular}{l}
|\input{childdoc.def}|\\
|\childdocby{|\textit{main}|}|\\
\end{tabular}
\end{center}
%
The directive |\childdocby| is similar to |\childdocof|
described in \secref{sec:include},
but the subsequent selection of content must be done manually.
To that end, both |\ifchilddoc| and |\ifchilddocmanual|
will be true upon processing of a part,
and the name of the part is stored in |\childdocname|.
Note that |\jobname| will be set to the filename of the current part
so that each part receives an individual |.aux| file
that does not interfere with the |.aux| file(s) of the main document.
This behaviour can be altered by the alternative form
|\childdocby[*]{|\textit{main}|}| (with a non-empty optional argument)
which uses the |.aux| file of the main document
by setting |\jobname| to \textit{main}.

%%%%%%%%%%%%%%%%%%%%%%%%%%%%%%%%%%%%%%%%%%%%%%%%%%%%%%%%%%%%%%%%%%%%%%%%%%%%%%%%
\subsection{Driver Development}
\label{sec:driver}

The \textsf{childdoc} mechanism can also be use for the development
of definition files such as \LaTeX{} styles or classes.
This case differs from the above setup with multiple parts
included by |\include| in that no |\includeonly| should be invoked.
This can be achieved by starting the include file
(before |\ProvidesPackage|) with:
%
\begin{center}
\begin{tabular}{l}
|\input{childdoc.def}|\\
|\childdocforward{|\textit{main}|}|\\
\end{tabular}
\end{center}
%
or alternatively with:
%
\begin{center}
\begin{tabular}{l}
|\input{childdoc.def}|\\
|\childdocby{|\textit{main}|}|\\
\end{tabular}
\end{center}
%
Both forms have slightly different effects as described above.
The main file is prepared as usual, see \secref{sec:include}.

%%%%%%%%%%%%%%%%%%%%%%%%%%%%%%%%%%%%%%%%%%%%%%%%%%%%%%%%%%%%%%%%%%%%%%%%%%%%%%%%
\subsection{Legacy Detection}
\label{sec:detection}

The directive |\childdocmain| in the main file can detect
whether the complete document or merely a child is to be compiled
even without using the directive |\childdocof|.
This method is deprecated because it is less robust
and there is no compelling reason to use it;
it is merely provided for backward compatibility
and it may be removed in future versions.

If the detection mechanism is to be used,
it is mandatory to correctly specify
the filename of the main file as the argument of |\childdocmain|:
%
\begin{center}
\begin{tabular}{l}
|\input{childdoc.def}|\\
|\childdocmain{|\textit{main}|}|\\
\end{tabular}
\end{center}
%
If |\jobname| does not match the argument \textit{main} of |\childdocmain|,
it is assumed that |\jobname| points to the child file to be compiled.
When using |\childdocmain| with the main file specified as argument,
it suffices to start a child file
with just |\input{|\textit{main}|}|
without loading of the package and using |\childdocof|.
If instead all processing is done
with the appropriate \textsf{childdoc} directives,
the argument of \textit{main} of |\childdocmain| can be empty.

An alternative version of the command line processing described
in \secref{sec:commandline} using the detection mechanism reads:
%
\begin{center}
|... -jobname "|\textit{target}|" "|[\textit{flags}]%
[|\def\jobname{|\textit{dest}|}|]|\input{|\textit{main}|}"|
\end{center}

%%%%%%%%%%%%%%%%%%%%%%%%%%%%%%%%%%%%%%%%%%%%%%%%%%%%%%%%%%%%%%%%%%%%%%%%%%%%%%%%
\subsection{Manual Code}
\label{sec:manual}

In case one cannot be certain whether the definitions file |childdoc.def|
is installed on the target \TeX{} distribution
and one prefers not to ship it,
it is conceivable to paste a few relevant commands into the sources.

To that end, drop all statements |\input{childdoc.def}|
and perform the replacements as outlined below.
Instead of |\childdocmain{|\textit{main}|}| add the following code
to the top of the main file:
%
\begin{center}
\begin{tabular}{l}
|\||ifdefined\childdocname\endinput\||fi\newif\ifchilddoc|\\
|\edef\childdocname{\scantokens\expandafter{\jobname\noexpand}}|\\
|\def\childdocmain{|\textit{main}|}\||ifx\childdocmain\childdocname\||else|\\
|\childdoctrue\includeonly{\childdocname}\let\jobname\childdocmain\||fi|\\
\end{tabular}
\end{center}
%
Instead of |\childdocof{|\textit{main}|}| just include the main file
at the top of each child file:
%
\begin{center}
|\input{|\textit{main}|}|
\end{center}
%
A simple redirection |\childdocforward{|\textit{dest}|}| is achieved by:
%
\begin{center}
|\def\jobname{|\textit{dest}|}\input{\jobname}|
\end{center}
%
The redirection with prefix
|\childdocforwardprefix[|\textit{prefix}|]{|\textit{dest}|}|
is accomplished by:
%
\begin{center}
\begin{tabular}{l}
|{\edef\jobname{\scantokens\expandafter{\jobname\noexpand}}|\\
|\def\redirectjob |\textit{prefix}|#1~~~{\gdef\jobname{|\textit{dest}|#1}}|\\
|\expandafter\redirectjob\jobname~~~}\input{\jobname}|
\end{tabular}
\end{center}

In an alternative approach,
child documents can be compiled by a specific command line
without additional code or specific definitions:
%
\begin{center}
|... -jobname "|\textit{target}|" "|[\textit{flags}]%
|\includeonly{|\textit{dest}|}\input{|\textit{main}|}"|
\end{center}
%

%%%%%%%%%%%%%%%%%%%%%%%%%%%%%%%%%%%%%%%%%%%%%%%%%%%%%%%%%%%%%%%%%%%%%%%%%%%%%%%%
%%%%%%%%%%%%%%%%%%%%%%%%%%%%%%%%%%%%%%%%%%%%%%%%%%%%%%%%%%%%%%%%%%%%%%%%%%%%%%%%
\section{Information}

%%%%%%%%%%%%%%%%%%%%%%%%%%%%%%%%%%%%%%%%%%%%%%%%%%%%%%%%%%%%%%%%%%%%%%%%%%%%%%%%
\subsection{Copyright}

Copyright \copyright{} 2017--2018 Niklas Beisert

This work may be distributed and/or modified under the
conditions of the \LaTeX{} Project Public License, either version 1.3
of this license or (at your option) any later version.
The latest version of this license is in
  \url{http://www.latex-project.org/lppl.txt}
and version 1.3 or later is part of all distributions of \LaTeX{}
version 2005/12/01 or later.

This work has the LPPL maintenance status `maintained'.

The Current Maintainer of this work is Niklas Beisert.

This work consists of the files |README.txt|, |childdoc.ins| and |childdoc.dtx|
as well as the derived files |childdoc.def|, |cdocsamp.tex|
with |cdocsch1.tex|, |cdocsch2.tex|, |cdocspt3.tex|, |cdocspt4.tex|,
|cdocsdrf.tex|, |cdocsfn1.tex|, |cdocsfn2.tex|
as well as |childdoc.pdf|.

%%%%%%%%%%%%%%%%%%%%%%%%%%%%%%%%%%%%%%%%%%%%%%%%%%%%%%%%%%%%%%%%%%%%%%%%%%%%%%%%
\subsection{Files and Installation}

The package consists of the files:
%
\begin{center}
\begin{tabular}{ll}
    |README.txt|   & readme file \\
    |childdoc.ins| & installation file \\
    |childdoc.dtx| & source file \\
    |childdoc.def| & definition file \\
    |cdocsamp.tex| & sample main file \\
    |cdocsch1.tex| & sample include file \\
    |cdocsch2.tex| & sample include file \\
    |cdocspt3.tex| & sample part file \\
    |cdocspt4.tex| & sample part file \\
    |cdocsdrf.tex| & sample redirection file \\
    |cdocsfn1.tex| & sample redirection file \\
    |cdocsfn2.tex| & sample redirection file \\
    |childdoc.pdf| & manual
\end{tabular}
\end{center}
%
The distribution consists of the files
|README.txt|, |childdoc.ins| and |childdoc.dtx|.
%
\begin{itemize}
\item
Run (pdf)\LaTeX{} on |childdoc.dtx|
to compile the manual |childdoc.pdf| (this file).
\item
Run \LaTeX{} on |childdoc.ins| to create the definitions file |childdoc.def|
and the sample |cdocsamp.tex| with include files
|cdocsch1.tex|, |cdocsch2.tex|, |cdocspt3.tex|, |cdocspt4.tex|,
|cdocsdrf.tex|, |cdocsfn1.tex|, |cdocsfn2.tex|.
Then copy the file |childdoc.def| to an appropriate directory of your \LaTeX{}
distribution, e.g.\ \textit{texmf-root}|/tex/latex/childdoc|.
\end{itemize}

%%%%%%%%%%%%%%%%%%%%%%%%%%%%%%%%%%%%%%%%%%%%%%%%%%%%%%%%%%%%%%%%%%%%%%%%%%%%%%%%
\subsection{Related CTAN Packages}

There are several other packages which offer a similar functionality:
%
\begin{itemize}
\item
The packages
\href{http://ctan.org/pkg/docmute}{\textsf{docmute}},
\href{http://ctan.org/pkg/includex}{\textsf{includex}} and
\href{http://ctan.org/pkg/standalone}{\textsf{standalone}}
provide commands to include only the document body of
a child file thus allowing both files to be compiled individually.
\item
The packages \href{http://ctan.org/pkg/subdocs}{\textsf{subdocs}}
and \href{http://ctan.org/pkg/subfiles}{\textsf{subfiles}}
provide structures in which the main and child documents can be
encapsulated and allowing them to be compiled individually.
The inclusion mechanism is different from the conventional |\include|.
\item
The package \href{http://ctan.org/pkg/combine}{\textsf{combine}}
is an elaborate solution to combine several documents into one.
\end{itemize}
%
See also the CTAN topic \href{http://ctan.org/topic/subdocs}{\textsf{subdocs}}
for further related packages.
The present package differs from the above solutions in that
a document structure constructed with the conventional |\include| mechanism
just needs two extra commands at the top of every file
such that all constituent files can be compiled individually.

%%%%%%%%%%%%%%%%%%%%%%%%%%%%%%%%%%%%%%%%%%%%%%%%%%%%%%%%%%%%%%%%%%%%%%%%%%%%%%%%
%\subsection{Feature Suggestions}
%
%The following is a list of features which may be useful for future
%versions of this package:
%%
%\begin{itemize}
%\item
%\ldots
%\end{itemize}

%%%%%%%%%%%%%%%%%%%%%%%%%%%%%%%%%%%%%%%%%%%%%%%%%%%%%%%%%%%%%%%%%%%%%%%%%%%%%%%%
\subsection{Revision History}

%%%%%%%%%%%%%%%%%%%%%%%%%%%%%%%%%%%%%%%%
\paragraph{v2.0:} 2018/12/30

\begin{itemize}
\item
immediate forward processing
\item
added |\childdocby| mechanism
\item
manual restructured
\end{itemize}

%%%%%%%%%%%%%%%%%%%%%%%%%%%%%%%%%%%%%%%%
\paragraph{v1.6:} 2018/01/17

\begin{itemize}
\item
application for development of include files
\item
corrections to manual
\end{itemize}

%%%%%%%%%%%%%%%%%%%%%%%%%%%%%%%%%%%%%%%%
\paragraph{v1.5:} 2017/05/21

\begin{itemize}
\item
more complete structuring introduced
\item
|\childdocof| introduced
\item
|\childdoc| renamed to |\childdocmain|
\item
|\childredirect| renamed to |\childdocforward| and |\childdocforwardprefix|
and functionality expanded
\end{itemize}

%%%%%%%%%%%%%%%%%%%%%%%%%%%%%%%%%%%%%%%%
\paragraph{v1.0:} 2017/04/27

\begin{itemize}
\item
manual and install package
\item
first version published on CTAN
\end{itemize}

%%%%%%%%%%%%%%%%%%%%%%%%%%%%%%%%%%%%%%%%
\paragraph{v0.6:} 2017/04/26

\begin{itemize}
\item
redirection mechanism added
\end{itemize}

%%%%%%%%%%%%%%%%%%%%%%%%%%%%%%%%%%%%%%%%
\paragraph{v0.5:} 2017/04/26

\begin{itemize}
\item
functionality in definition file
\end{itemize}


%%%%%%%%%%%%%%%%%%%%%%%%%%%%%%%%%%%%%%%%%%%%%%%%%%%%%%%%%%%%%%%%%%%%%%%%%%%%%%%%
%%%%%%%%%%%%%%%%%%%%%%%%%%%%%%%%%%%%%%%%%%%%%%%%%%%%%%%%%%%%%%%%%%%%%%%%%%%%%%%%
%%%%%%%%%%%%%%%%%%%%%%%%%%%%%%%%%%%%%%%%%%%%%%%%%%%%%%%%%%%%%%%%%%%%%%%%%%%%%%%%
\appendix

\settowidth\MacroIndent{\rmfamily\scriptsize 000\ }

 \DocInput{childdoc.dtx}

\end{document}
%</driver>
% \fi
%
% %%%%%%%%%%%%%%%%%%%%%%%%%%%%%%%%%%%%%%%%%%%%%%%%%%%%%%%%%%%%%%%%%%%%%%%%%%%%%%
% %%%%%%%%%%%%%%%%%%%%%%%%%%%%%%%%%%%%%%%%%%%%%%%%%%%%%%%%%%%%%%%%%%%%%%%%%%%%%%
% \section{Sample}
%\iffalse
%<*samplemain>
%\fi
%
% The following presents a sample document
% with two chapters, two parts, a title page,
% a compile flag as well as three forwarding files to set the flag.
% It consists of eight |.tex| files:
% \begin{center}
% \begin{tabular}{ll}
% |cdocsamp.tex|&main file\\
% |cdocsch1.tex|&include file for chapter 1\\
% |cdocsch2.tex|&include file for chapter 2\\
% |cdocspt3.tex|&include file for part 3\\
% |cdocspt4.tex|&include file for part 4\\
% |cdocsdrf.tex|&forwarding file for main file in draft mode\\
% |cdocsfi1.tex|&forwarding file for final version of chapter 1\\
% |cdocsfi2.tex|&forwarding file for final version of chapter 2\\
% \end{tabular}
% \end{center}
% Each of the eight files can be compiled directly by the \LaTeX{} compiler.
%
% %%%%%%%%%%%%%%%%%%%%%%%%%%%%%%%%%%%%%%
% \paragraph{Main File.}
%
% The main file is called |cdocsamp.tex|.
%
% Load the \textsf{childdoc} definitions and
% declare the filename for the main document:
%    \begin{macrocode}
\input{childdoc.def}
\childdocmain{}
%    \end{macrocode}

% Optional override for |\version| flag:
%    \begin{macrocode}
%%\ifchilddoc\else\providecommand{\version}{draft}\fi
%    \end{macrocode}

% Define the default values for the |\version| flag
% (|final| for the main file and |draft| for childs):
%    \begin{macrocode}
\ifchilddoc
\providecommand{\version}{draft}
\else
\providecommand{\version}{final}
\fi
%    \end{macrocode}

% Load the standard document class:
%    \begin{macrocode}
\documentclass[12pt]{article}
%    \end{macrocode}

% Start the document body:
%    \begin{macrocode}
\begin{document}
%    \end{macrocode}

% Declare a title page.
% Print title, part of document being processed and version flag:
%    \begin{macrocode}
\addtocounter{page}{-1}
\begin{center}
{\LARGE\bfseries{}childdoc example\par}
\vspace{1cm}
\ifchilddoc
\ifchilddocmanual part\else chapter\fi:
`\childdocname' of `\childdocjob'\par
\else
main document: `\childdocjob'\par
\fi
version: \version\par
\end{center}
\newpage
%    \end{macrocode}

% Manually include selected file,
% otherwise process as usual:
%    \begin{macrocode}
\ifchilddocmanual
\section*{part `\childdocname'}
\input{\childdocname}
\else
%    \end{macrocode}

% Include the two chapters:
%    \begin{macrocode}
\include{cdocsch1}
\include{cdocsch2}
%    \end{macrocode}

% Include the two parts unless only chapters should be displayed:
%    \begin{macrocode}
\ifchilddoc\else
\section{part three}
\input{cdocspt3}
\section{part four}
\input{cdocspt4}
\fi
%    \end{macrocode}

% Process as usual until here:
%    \begin{macrocode}
\fi
%    \end{macrocode}

% End of document body:
%    \begin{macrocode}
\end{document}
%    \end{macrocode}
%\iffalse
%</samplemain>
%\fi
%
% %%%%%%%%%%%%%%%%%%%%%%%%%%%%%%%%%%%%%%
% \paragraph{Chapter Include Files.}
%
% The include files are called |cdocsch1.tex| and |cdocsch2.tex|.
%
%\iffalse
%<*samplechap1|samplechap2>
%\fi

% Optional override for |\version| flag:
%    \begin{macrocode}
%%\providecommand{\version}{final}
%    \end{macrocode}

% Include the main document:
%    \begin{macrocode}
\input{childdoc.def}
\childdocof{cdocsamp}
%    \end{macrocode}

%\iffalse
%</samplechap1|samplechap2>
%\fi
%
%\iffalse
%<*samplechap1>
%\fi
% Some text for chapter 1:
%    \begin{macrocode}
\section{one}
some text in chapter one
%    \end{macrocode}

%\iffalse
%</samplechap1>
%\fi
% Some text for chapter 2:
%\iffalse
%<*samplechap2>
%\fi
%    \begin{macrocode}
\section{two}
more text in chapter two
%    \end{macrocode}

%\iffalse
%</samplechap2>
%\fi
%
% %%%%%%%%%%%%%%%%%%%%%%%%%%%%%%%%%%%%%%
% \paragraph{Part Include Files.}
%
% The include files are called |cdocspt3.tex| and |cdocspt4.tex|.
%
%\iffalse
%<*samplepart3|samplepart4>
%\fi

% Optional override for |\version| flag:
%    \begin{macrocode}
%%\providecommand{\version}{final}
%    \end{macrocode}

% Include the main document:
%    \begin{macrocode}
\input{childdoc.def}
\childdocby{cdocsamp}
%    \end{macrocode}

%\iffalse
%</samplepart3|samplepart4>
%\fi
%
%\iffalse
%<*samplepart3>
%\fi
% Some text for part 3:
%    \begin{macrocode}
some text in part three
%    \end{macrocode}

%\iffalse
%</samplepart3>
%\fi
% Some text for part 4:
%\iffalse
%<*samplepart4>
%\fi
%    \begin{macrocode}
more text in part four
%    \end{macrocode}

%\iffalse
%</samplepart4>
%\fi
%
% %%%%%%%%%%%%%%%%%%%%%%%%%%%%%%%%%%%%%%
% \paragraph{Forwarding for a Complete Draft.}
%
% The following forwarding file |cdocsdrf.tex|
% compiles the main document in draft mode:
%\iffalse
%<*sampledraft>
%\fi
%    \begin{macrocode}
\def\version{draft}
\input{childdoc.def}
\childdocforward{cdocsamp}
%    \end{macrocode}

%\iffalse
%</sampledraft>
%\fi
%
% %%%%%%%%%%%%%%%%%%%%%%%%%%%%%%%%%%%%%%
% \paragraph{Forwarding for Final Version of the Chapters.}
%
% The following forwarding files |cdocsfn1.tex| and |cdocsfn2.tex|
% (with identical content)
% compile the final versions of the child documents
% |cdocsch1.tex| and |cdocsch2.tex|, respectively:
%\iffalse
%<*samplefinal>
%\fi
%    \begin{macrocode}
\def\version{final}
\input{childdoc.def}
\childdocforwardprefix[cdocsamp]{cdocsfn}{cdocsch}
%    \end{macrocode}

%\iffalse
%</samplefinal>
%\fi
%
% %%%%%%%%%%%%%%%%%%%%%%%%%%%%%%%%%%%%%%
% \paragraph{Command Line Processing.}
%
% The following three command lines generate the output files
% |cdocscld|, |cdocscl1| and |cdocscl2|
% which should be identical to
% |cdocsdrf|, |cdocsch1| and |cdocsfn2|, respectively:
% \begin{center}
% \begin{tabular}{l}
% |latex -jobname cdocscld \|\\
% |  "\def\version{draft}\input{childdoc.def}\childdocforward{cdocsamp}"|\\
% |latex -jobname cdocscl1 \|\\
% |  "\input{childdoc.def}\childdocforward[cdocsamp]{cdocsch1}"|\\
% |latex -jobname cdocscl2 \|\\
% |  "\def\version{final}\input{childdoc.def}\childdocforward{cdocsch2}"|
% \end{tabular}
% \end{center}
% Note that the trailing backslash on each first line
% merely continues the input to the second line
% (for convenient cut ant paste).
% Furthermore, the command |latex| can be replaced by any
% of its alternative versions such as |pdflatex|.
%
% %%%%%%%%%%%%%%%%%%%%%%%%%%%%%%%%%%%%%%%%%%%%%%%%%%%%%%%%%%%%%%%%%%%%%%%%%%%%%%
% %%%%%%%%%%%%%%%%%%%%%%%%%%%%%%%%%%%%%%%%%%%%%%%%%%%%%%%%%%%%%%%%%%%%%%%%%%%%%%
% \section{Implementation}
%\iffalse
%<*package>
%\fi
%
% This section describes the definitions file |childdoc.def|.

% The definitions cannot be loaded using |\usepackage| or |\RequirePackage|
% which has a mechanism to prevent loading a style file more than once.
% When loading the definitions by means of |\input|
% multiple instances have to be prevented manually:
%\iffalse
%This code needs to be before the `\ProvidesFile' directive
%which is defined at the beginning of this file.
%Therefore it is also placed there and commented out here.
%</package>
%<*discard>
%\fi
%    \begin{macrocode}
\ifdefined\childdocmain\endinput\fi
%    \end{macrocode}
%\iffalse
%</discard>
%<*package>
%\fi
%
% \macro{\ifchilddoc}
% \macro{\ifchilddocmanual}
% The conditional |\ifchilddoc| tells whether a
% child (true) or main (false) document is being compiled.
% The conditional |\ifchilddocmanual| tells whether
% the |\includeonly| mechanism is used (false) or
% the selection of child files must be performed manually (true).
% The definitions initialise to false:
%    \begin{macrocode}
\newif\ifchilddoc
\newif\ifchilddocmanual
%    \end{macrocode}

% \macro{\childdocname}
% \macro{\childdocjob}
% The macro |\childdocname| stores the name of the main document
% to be compiled. The macro |\childdocjob| stores the name of
% the document on which the \LaTeX{} compiler was originally invoked.
% The content of |\jobname| cannot be compared
% to filenames specified in the source due to different catcodes.
% The following code rescans |\jobname|, stores the result
% in |\childdocname| and saves a copy in |\childdocjob|:
%    \begin{macrocode}
\edef\childdocname{\scantokens\expandafter{\jobname\noexpand}}
\let\childdocjob\childdocname
%    \end{macrocode}

% \macro{\childdocdisable}
% The macro |\childdocdisable| prevents the main file
% from being processed more than once.
% At this stage, the main document command |\childdocmain|
% is assumed to be called once again where it should do nothing.
% Any subsequent call to it should prevent
% a secondary processing of the main document
% It overwrites the forwarding commands
% |\childdocof| and |\childdocforward|
% with empty macros to prevent further inclusions of the main document:
%    \begin{macrocode}
\newcommand{\childdocdisable}
{
  \renewcommand{\childdocmain}[1]{\renewcommand{\childdocmain}[1]{\endinput}}
  \renewcommand{\childdocof}[1]{}
  \renewcommand{\childdocby}[2][]{}
  \renewcommand{\childdocforward}[2][]{}
  \renewcommand{\childdocdisable}{}
}
%    \end{macrocode}

% \macro{\childdocmain}
% The macro |\childdocmain| is to be called at the top of the main file
% with nothing or the main filename (without extension) as argument.
% First, it breaks loops.
% If the argument is not empty and does not match |\childdocname|
% (which is set by the first inclusion of |childdoc.def|),
% |\ifchilddoc| is set to true, |\includeonly| is applied to the child file
% and |\jobname| is set to the main file
% (for proper handling of |.aux| files):
%    \begin{macrocode}
\newcommand{\childdocmain}[1]
{
  \childdocdisable\childdocmain{}
  \if?#1?\else
    \begingroup
      \def\childdoctmp{#1}
      \ifx\childdoctmp\childdocname
        \def\childdoctmp{}
      \else
        \def\childdoctmp
        {
          \childdoctrue
          \includeonly{\childdocname}
          \def\childdocjob{#1}
          \def\jobname{#1}
        }
      \fi
      \expandafter
    \endgroup
    \childdoctmp
  \fi
}
%    \end{macrocode}

% \macro{\childdocof}
% The command |\childdocof| redirects
% compilation to the main file |#1|.
%    \begin{macrocode}
\newcommand{\childdocof}[1]
{
  \childdocdisable
  \childdoctrue
  \includeonly{\childdocname}
  \def\jobname{#1}
  \def\childdocjob{#1}
  \input{#1}
}
%    \end{macrocode}

% \macro{\childdocby}
% The command |\childdocby| ....
%    \begin{macrocode}
\newcommand{\childdocby}[2][]
{
  \childdocdisable
  \childdoctrue
  \childdocmanualtrue
  \if?#1?\else
    \def\jobname{#2}
  \fi
  \def\childdocjob{#2}
  \input{#2}
  \endinput
}
%    \end{macrocode}

% \macro{\childdocforward}
% The command |\childdocforward| redirects
% compilation to the main file or
% (if the optional argument is given) a child file.
% Parameters are set as if the main file
% or a child file starting with |\childdocof| was compiled.
% Then compilation is handed over to the main file:
%    \begin{macrocode}
\newcommand{\childdocforward}[2][]
{
  \begingroup
    \if?#1?
      \def\childdoctmp
      {
        \def\childdocname{#2}
        \def\childdocjob{#2}
        \def\jobname{#2}
        \input{#2}
        \endinput
      }
    \else
      \def\childdoctmp
      {
        \childdocdisable
        \def\childdocname{#2}
        \childdoctrue
        \includeonly{#2}
        \def\childdocjob{#1}
        \def\jobname{#1}
        \input{#1}
        \endinput
      }
    \fi
    \expandafter
  \endgroup
  \childdoctmp
}
%    \end{macrocode}

% \macro{\childdocforwardprefix}
% The command |\childdocforwardprefix| redirects
% compilation to the main or a child file by means of a pattern.
% The prefix |#1| in the current filename is replaced by |#2|
% and the suffix of the current filename is kept
% (it is assumed that the filename does not contain the substring `|~~~|'
% which is used as a delimiter).
% Compilation is handed over to the new file by |\childdocforward|:
%    \begin{macrocode}
\newcommand{\childdocforwardprefix}[3][]
{
  \begingroup
    \def\childdocextract #2##1~~~{\def\childdoctmp{\childdocforward[#1]{#3##1}}}
    \expandafter\childdocextract\childdocname~~~
    \expandafter
  \endgroup
  \childdoctmp
}
%    \end{macrocode}

% \macro{\childdoc}
% The deprecated macro |\childdoc| is a legacy version of |\childdocmain|:
%    \begin{macrocode}
\newcommand{\childdoc}{\childdocmain}
%    \end{macrocode}

% \macro{\childdocredirect}
% The deprecated macro |\childdocredirect| is a legacy version
% of |\childdocforward| and |\childdocforwardprefix|:
%    \begin{macrocode}
\newcommand{\childdocredirect}[2][]
{
  \begingroup
    \if?#1?
      \def\childdoctmp{\childdocforward{#2}}
    \else
      \def\childdoctmp{\childdocforwardprefix{#1}{#2}}
    \fi
    \expandafter
  \endgroup
  \childdoctmp
}
%    \end{macrocode}

%\iffalse
%</package>
%\fi
%
\endinput
|\\
|\childdocmain{|\textit{main}|}|\\
\end{tabular}
\end{center}
%
If |\jobname| does not match the argument \textit{main} of |\childdocmain|,
it is assumed that |\jobname| points to the child file to be compiled.
When using |\childdocmain| with the main file specified as argument,
it suffices to start a child file
with just |\input{|\textit{main}|}|
without loading of the package and using |\childdocof|.
If instead all processing is done
with the appropriate \textsf{childdoc} directives,
the argument of \textit{main} of |\childdocmain| can be empty.

An alternative version of the command line processing described
in \secref{sec:commandline} using the detection mechanism reads:
%
\begin{center}
|... -jobname "|\textit{target}|" "|[\textit{flags}]%
[|\def\jobname{|\textit{dest}|}|]|\input{|\textit{main}|}"|
\end{center}

%%%%%%%%%%%%%%%%%%%%%%%%%%%%%%%%%%%%%%%%%%%%%%%%%%%%%%%%%%%%%%%%%%%%%%%%%%%%%%%%
\subsection{Manual Code}
\label{sec:manual}

In case one cannot be certain whether the definitions file |childdoc.def|
is installed on the target \TeX{} distribution
and one prefers not to ship it,
it is conceivable to paste a few relevant commands into the sources.

To that end, drop all statements |% \iffalse
%
% childdoc.dtx Copyright (C) 2017-2018 Niklas Beisert
%
% This work may be distributed and/or modified under the
% conditions of the LaTeX Project Public License, either version 1.3
% of this license or (at your option) any later version.
% The latest version of this license is in
%   http://www.latex-project.org/lppl.txt
% and version 1.3 or later is part of all distributions of LaTeX
% version 2005/12/01 or later.
%
% This work has the LPPL maintenance status `maintained'.
%
% The Current Maintainer of this work is Niklas Beisert.
%
% This work consists of the files childdoc.dtx and childdoc.ins
% and the derived files childdoc.def and cdocsamp.tex with
% cdocsch1.tex, cdocsch2.tex, cdocsdrf.tex, cdocsfn1.tex, cdocsfn2.tex.
%
%<package>\ifdefined\childdocmain\endinput\fi
%<package>\ProvidesFile{childdoc.def}[2018/12/30 v2.0 child document driver]
%<samplemain>\ProvidesFile{cdocsamp.tex}[2018/12/30 v2.0 sample for childdoc]
%<*driver>
%\ProvidesFile{childdoc.drv}[2018/12/30 v2.0 childdoc reference manual file]
\PassOptionsToClass{10pt,a4paper}{article}
\documentclass{ltxdoc}

\usepackage[margin=35mm]{geometry}
\usepackage{hyperref}
\usepackage{hyperxmp}
\usepackage[usenames]{color}

\hypersetup{colorlinks=true}
\hypersetup{pdfstartview=FitH}
\hypersetup{pdfpagemode=UseNone}
\hypersetup{pdfsource={}}
\hypersetup{pdflang={en-UK}}
\hypersetup{pdfcopyright={Copyright 2017-2018 Niklas Beisert.
  This work may be distributed and/or modified under the
  conditions of the LaTeX Project Public License, either version 1.3
  of this license or (at your option) any later version.}}
\hypersetup{pdflicenseurl={http://www.latex-project.org/lppl.txt}}
\hypersetup{pdfcontactaddress={ETH Zurich, ITP, HIT K,
  Wolfgang-Pauli-Strasse 27}}
\hypersetup{pdfcontactpostcode={8093}}
\hypersetup{pdfcontactcity={Zurich}}
\hypersetup{pdfcontactcountry={Switzerland}}
\hypersetup{pdfcontactemail={nbeisert@itp.phys.ethz.ch}}
\hypersetup{pdfcontacturl={http://people.phys.ethz.ch/\xmptilde nbeisert/}}

\newcommand{\secref}[1]{\hyperref[#1]{section \ref*{#1}}}

\parskip1ex
\parindent0pt
\let\olditemize\itemize
\def\itemize{\olditemize\parskip0pt}

\begin{document}

\title{The \textsf{childdoc} Package}
\hypersetup{pdftitle={The childdoc Package}}
\author{Niklas Beisert\\[2ex]
  Institut f\"ur Theoretische Physik\\
  Eidgen\"ossische Technische Hochschule Z\"urich\\
  Wolfgang-Pauli-Strasse 27, 8093 Z\"urich, Switzerland\\[1ex]
  \href{mailto:nbeisert@itp.phys.ethz.ch}
  {\texttt{nbeisert@itp.phys.ethz.ch}}}
\hypersetup{pdfauthor={Niklas Beisert}}
\hypersetup{pdfsubject={Manual for the LaTeX2e Package childdoc}}
\date{30 December 2018, \textsf{v2.0}}
\maketitle

\begin{abstract}\noindent
\textsf{childdoc} is a \LaTeXe{} package
that enables the direct compilation
of document sections included by |\include|
to individual files.
\end{abstract}

\begingroup
\parskip0ex
\tableofcontents
\endgroup

%%%%%%%%%%%%%%%%%%%%%%%%%%%%%%%%%%%%%%%%%%%%%%%%%%%%%%%%%%%%%%%%%%%%%%%%%%%%%%%%
%%%%%%%%%%%%%%%%%%%%%%%%%%%%%%%%%%%%%%%%%%%%%%%%%%%%%%%%%%%%%%%%%%%%%%%%%%%%%%%%
\section{Introduction}

\LaTeX{} provides a mechanism to structure a large document (such as a book)
into a main file and several child files (containing the chapters)
using the |\include| command.
This mechanism is beneficial for documents
which span hundreds of pages in order to
make the source file(s) more manageable.
Moreover, compilation can be restricted to
selected child files by means of the |\includeonly| command.
The latter feature can be used to reduce the compilation time while editing
(this was significantly more useful in the earlier days of \LaTeX{})
or to generate a smaller document which is easier to navigate.
Another application of |\includeonly| is to generate
documents consisting of selected parts of the complete document.

However, there are a few drawbacks of the plain |\include| mechanism:
\begin{itemize}
\item
The child files cannot be compiled on their own,
they can only be compiled via the main file.
A naive editing environment
(such as a text editor with an option
to have the current file processed by \LaTeX)
may require one to switch to the main file before compiling;
attempting to compile the child file produces errors.
\item
The main file must be modified (each time)
to adjust the |\includeonly| command
to the present needs. This easily leaves the main file in a messy state.
\item
The generated document will always carry the filename
of the main document. This is inconvenient if
several child files are to be compiled and
to be kept for distribution.
\end{itemize}

The present package provides a simple interface
to make child files individually compilable by \LaTeX{}.
Compiling a child file then has the same effect as compiling
the main file with an |\includeonly| command
to select the appropriate child.
Moreover the generated document will carry the name of the child
rather than the main file.
This resolves all three above issues.

This feature is meant to make the editing of books,
thesis documents and lecture notes somewhat more convenient.
However, the package can also be used efficiently for
composing a series of documents (such as exercise sheets)
which are typically distributed individually.
It then assists the author in generating the individual documents
(potentially in different versions)
as well as a document containing the collected series.
Another application is in developing style files
or other kinds of included material
where compilation of the style file could redirect
to a sample or test file.

%%%%%%%%%%%%%%%%%%%%%%%%%%%%%%%%%%%%%%%%%%%%%%%%%%%%%%%%%%%%%%%%%%%%%%%%%%%%%%%%
%%%%%%%%%%%%%%%%%%%%%%%%%%%%%%%%%%%%%%%%%%%%%%%%%%%%%%%%%%%%%%%%%%%%%%%%%%%%%%%%
\section{Usage}

First of all, the package \textsf{childdoc} is \emph{not} a standard
\LaTeXe{} |.sty| style file! Therefore it needs to be invoked in
a non-standard way.

%%%%%%%%%%%%%%%%%%%%%%%%%%%%%%%%%%%%%%%%%%%%%%%%%%%%%%%%%%%%%%%%%%%%%%%%%%%%%%%%
\subsection{Included Files}
\label{sec:include}

%%%%%%%%%%%%%%%%%%%%%%%%%%%%%%%%%%%%%%%%
\DescribeMacro{\childdocmain}
To use the package, add the commands
\begin{center}
\begin{tabular}{l}
|\input{childdoc.def}|\\
|\childdocmain{}|\\
\end{tabular}
\end{center}
at the very top of the main \LaTeX{} file,
in particular \emph{before} the |\documentclass| statement!
The argument of |\childdocmain| should be left empty
(but it must be present).

%%%%%%%%%%%%%%%%%%%%%%%%%%%%%%%%%%%%%%%%
\DescribeMacro{\childdocof}
Furthermore, add the commands
\begin{center}
\begin{tabular}{l}
|\input{childdoc.def}|\\
|\childdocof{|\textit{main}|}|\\
\end{tabular}
\end{center}
at the top of every child file \textit{child}
which is included by |\include{|\textit{child}|}|
from within the main file
(or at least for those files to be compiled individually).
The argument \textit{main} must be the filename of the main file.

There are a couple of
considerations in setting up the main and child documents:

%%%%%%%%%%%%%%%%%%%%%%%%%%%%%%%%%%%%%%%%
\paragraph{Restrictions.}

Please note the following restrictions:
\begin{itemize}
\item
|\childdocmain| must be called with one argument \textit{main}
to ensure compatibility with earlier version of the package.
It must either be empty (|\childdocmain{}|)
or precisely match the filename of the main file in which it is specified.
See \secref{sec:detection} for further information.
\item
The filename \textit{main} must be specified without the |.tex| extension.
\item
The filename \textit{main} is case sensitive
(even in case-insensitive file systems)
due to internal string comparison.
\item
The argument \textit{main} should be fully expanded, it cannot be a macro.
\item
Subdirectories and special characters should be avoided in filenames.
\item
The command |\childdocmain{|\textit{main}|}| must be followed by a whitespace.
It should not be followed immediately by another command
or by a comment mark `|%|'.
This is because the \TeX{} parser reads the token immediately following
the argument of |\childdocmain| and puts it
at the beginning of every child section;
however, a white\-space is ignored.
\end{itemize}

%%%%%%%%%%%%%%%%%%%%%%%%%%%%%%%%%%%%%%%%
\paragraph{Content of Main File.}

It is advisable to place all content in the child files included by |\include|.
Any output contained in the main file will appear in all child documents
unless suppressed manually;
it cannot be suppressed automatically by the |\includeonly| directive
and thus should normally be avoided.
A method to include some content in the main file
by means of conditional processing is described in \secref{sec:conditional}.

%%%%%%%%%%%%%%%%%%%%%%%%%%%%%%%%%%%%%%%%
\paragraph{Page Numbering.}

When only a part of the document is compiled,
the appropriate numbering of pages
(as well as other status parameters)
is determined from the |.aux| files.
The latter contain information from previous passes.
However this information needs to propagate through
all intermediate child documents.
Therefore the page numbering in child documents may well
be inconsistent until the complete document is compiled at least once.

A useful (if unconventional) way to always ensure a consistent
page numbering is to restart the numbering in each child document
and denote the pages by `\textit{child}|.|\textit{page}'
where \textit{child} represents the chapter/section number of the child file.
This can be achieved by the command
|\numberwithin{page}{|\textit{child}|}|
of the \textsf{amsmath} package
where \textit{child} can be |chapter| or |section|
depending on the chosen structuring.
Alternatively, one can modify the macro |\thepage| appropriately
and reset the counter |page| at the start of each child file.

%%%%%%%%%%%%%%%%%%%%%%%%%%%%%%%%%%%%%%%%%%%%%%%%%%%%%%%%%%%%%%%%%%%%%%%%%%%%%%%%
\subsection{Conditional Processing}
\label{sec:conditional}

The package provides a mechanism to compile different versions
of a document. To customise the versions further some conditional processing
can come in handy to distinguish which version is being compiled.
The package provides two macros to describe the compilation context:

%%%%%%%%%%%%%%%%%%%%%%%%%%%%%%%%%%%%%%%%
\DescribeMacro{\ifchilddoc}
The conditional |\ifchilddoc| distinguishes between the compilation of
child documents and the main document:
%
\begin{center}
|\ifchilddoc |\textit{child-code}| |[|\||else |\textit{main-code}]| \||fi|
\end{center}

%%%%%%%%%%%%%%%%%%%%%%%%%%%%%%%%%%%%%%%%
\DescribeMacro{\childdocname}
\DescribeMacro{\childdocjob}
The macro |\childdocname| contains the filename (without extension)
of the main or child file being processed.
Note that |\childdocjob| will always contain the name of the main file.

%%%%%%%%%%%%%%%%%%%%%%%%%%%%%%%%%%%%%%%%
\paragraph{Title Page.}

Conditional processing can be used to include a title or banner page
in the main document when proper precautions are taken.
Importantly, the code in the main file should ensure that the page counter
(as well as other status parameters which are stored in the |.aux| files)
takes the same value after the conditional processing.
Otherwise the page numbers may take divergent values
depending on which part is compiled.

For example, a title page could be declared by:
%
\begin{center}
\begin{tabular}{l}
|\ifchilddoc\||else|\\
|\addtocounter{page}{-1}|\\
\textit{code for title page}\\
|\newpage|\\
|\||fi|
\end{tabular}
\end{center}
%
A banner page for the child documents can be generated by:
%
\begin{center}
\begin{tabular}{l}
|\ifchilddoc|\\
|\addtocounter{page}{-1}|\\
\textit{code for banner page}\\
|\newpage|\\
|\||fi|
\end{tabular}
\end{center}
%
Here one could write a message such as:
\begin{center}
|This is the part \childdocname{} of \childdocjob{}.|
\end{center}

%%%%%%%%%%%%%%%%%%%%%%%%%%%%%%%%%%%%%%%%%%%%%%%%%%%%%%%%%%%%%%%%%%%%%%%%%%%%%%%%
\subsection{Flags}
\label{sec:flags}

The package makes it easy to generate different versions
of the main or child documents.
To this end compilation flags can be defined
and assigned different default values.
They will be particularly useful in conjunction
with the forwarding mechanism described in \secref{sec:forward}.

For example, it may be useful to have a flag |\version|
which can be set to |draft| or |final|.
The document source will contain some conditional code
depending on the value of |\version|.
Suppose further, the flag should default to |final| for the main file
and to |draft| for child files
which is a natural assignment for editing the document.
This is achieved by placing the following code
in the preamble of the main document
(below the |\childdocmain| directive):
%
\begin{center}
\begin{tabular}{l}
|\ifchilddoc|\\
|\providecommand{\version}{draft}|\\
|\||else|\\
|\providecommand{\version}{final}|\\
|\||fi|
\end{tabular}
\end{center}
%
The definition by |\providecommand| makes sure
that previous definitions are not overwritten.
Further statements |\providecommand{\version}{...}|
can thus be added before the above code to override it.

For the main file, one might add a line
(between |\childdocmain| and the above block)
%
\begin{center}
|%\ifchilddoc\||else\providecommand{\version}{draft}\||fi|
\end{center}
%
which can be uncommented to produce a draft version.
Likewise one can add a line to the very top of a child file
(above the |\childdocof{|\textit{main}|}| directive)
%
\begin{center}
|%\providecommand{\version}{final}|
\end{center}
%
which can be uncommented to produce the final version of this child document.

%%%%%%%%%%%%%%%%%%%%%%%%%%%%%%%%%%%%%%%%%%%%%%%%%%%%%%%%%%%%%%%%%%%%%%%%%%%%%%%%
\subsection{Forwarding}
\label{sec:forward}

Different versions of the main or child documents
using compilation flags as described in \secref{sec:flags}
can be (permanently) stored in different files
for convenient compilation, viewing and distribution.
To this end, the package defines a command
to pass on compilation to a different file:

%%%%%%%%%%%%%%%%%%%%%%%%%%%%%%%%%%%%%%%%
\DescribeMacro{\childdocforward}
The command |\childdocforward| redirects processing to
another source file:
%
\begin{center}
\begin{tabular}{l}
|\input{childdoc.def}|\\
|\childdocforward[|\textit{main}|]{|\textit{dest}|}|\\
\end{tabular}
\end{center}
%
The argument \textit{dest} is the destination file
(without extension).
It should be the main file or one of the child files.
Note that further \textsf{childdoc} directives
such as |\childdocof| and |\childdocforward|
in the indicated file will be processed in this form.
The optional argument \textit{main}
passes on directly to the main file \textit{main}
while pretending to compile the child \textit{dest}.
This form behaves as if \textit{dest}
issues |\childdocof{|\textit{main}|}| right away,
and no further \textsf{childdoc} directives will be processed.

%%%%%%%%%%%%%%%%%%%%%%%%%%%%%%%%%%%%%%%%
\DescribeMacro{\...prefix}
In the alternative form |\childdocforwardprefix|,
%
\begin{center}
\begin{tabular}{l}
|\input{childdoc.def}|\\
|\childdocforwardprefix[|\textit{main}|]{|\textit{prefix}|}{|\textit{dest}|}|
\end{tabular}
\end{center}
%
the destination file is determined by a pattern
depending on the current file:
To make this work, the current file must be called
`{\textit{prefix}\hspace{0.2em}\textit{suffix}}'
with \textit{prefix} matching precisely the argument.
Processing is then passed on to the file
`{\textit{dest}\hspace{0.2em}\textit{suffix}}'.
Surely, the same effect is achieved by
directly specifying the
argument `{\textit{dest}\hspace{0.2em}\textit{suffix}}'
in the first form.
However, that requires to set up a different file
for each child. With the alternative form of the command
all these files can have exactly the same content
which simplifies setting them up and maintaining them.

For example, the following file |draft.tex|
with a compilation flag |\version| as described in \secref{sec:flags}
compiles the main document as a draft:
%
\begin{center}
\begin{tabular}{l}
|\def\version{draft}|\\
|\input{childdoc.def}|\\
|\childdocforward{|\textit{main}|}|
\end{tabular}
\end{center}
%
Likewise, the following files |final|\textit{nn}|.tex|
compile the final version of the child document
|child|\textit{nn}|.tex|:
%
\begin{center}
\begin{tabular}{l}
|\def\version{final}|\\
|\input{childdoc.def}|\\
|\childdocforwardprefix{final}{child}|
\end{tabular}
\end{center}
%

Note that when several versions of a main file and/or of each child file
are to be generated, it may be convenient to set up a |Makefile| or
shell script to automatise the process.

%%%%%%%%%%%%%%%%%%%%%%%%%%%%%%%%%%%%%%%%%%%%%%%%%%%%%%%%%%%%%%%%%%%%%%%%%%%%%%%%
\subsection{Command Line Processing}
\label{sec:commandline}

The effect of redirection files can also be achieved by invoking
the \LaTeX{} compiler with a more elaborate command line.
Most conveniently this should be done as part
of a shell script or a |Makefile|.

When using \textsf{childdoc} in the main file, the following
command lines effectively perform a redirection
(note that depending on the shell being used,
backslashes may have to be doubled: `|\|' $\to$ `|\\|'):
%
\begin{center}
|... -jobname "|\textit{target}|" |\\|"|[\textit{flags}]%
|\input{childdoc.def}\childdocforward[|\textit{main}|]{|\textit{dest}|}"|
\end{center}
%
Here \textit{target} is the name of the output file,
\textit{main} is the name of the main file
and \textit{dest} is the name of the main or child file to be processed
(all filenames without extensions).
The optional argument \textit{main} can be omitted
if \textit{main} matches \textit{dest}.
Optionally, compilation \textit{flags} can be defined via |\def| commands.
This command line makes the \TeX{} engine believe
it is compiling the file \textit{target}
whose content is specified as the latter parameter.
The provided code then forwards the processing to
\textit{main} or \textit{dest} as described in \secref{sec:forward}.

%%%%%%%%%%%%%%%%%%%%%%%%%%%%%%%%%%%%%%%%%%%%%%%%%%%%%%%%%%%%%%%%%%%%%%%%%%%%%%%%
\subsection{Include by Input}
\label{sec:input}

Including child documents by |\include| has some restrictions by design.
Most notably, the content of a child document always occupies
its own set of pages; pages cannot be shared between child documents.
Usually, this behaviour makes perfect sense
because each child document contain an essential part of the document.
However, in some situations it may be desirable to compose
a document from a collection of parts
without having mandatory page breaks between then.
For this case, the package
provides a mechanism to include parts
by |\input| which can also be processed individually.
However, by construction this mechanism
requires manual handling of the content to be output.

%%%%%%%%%%%%%%%%%%%%%%%%%%%%%%%%%%%%%%%%
\DescribeMacro{\ifchilddocmanual}
The main file should be prepared as usual, see \secref{sec:include}.
However, the document body must make a distinction
between processing of an individual part and of the main document, e.g.:
%
\begin{center}
\begin{tabular}{l}
|\ifchilddocmanual|\\
|\input{\childdocname}|\\
|\||else|\\
\textit{document body with }|\input{|\textit{part}|}|\\
|\||fi|
\end{tabular}
\end{center}
%
The conditional |\ifchilddocmanual| is true whenever
a part to be included by |\input| is being compiled,
and the name of the part is stored in |\childdocname|.

%%%%%%%%%%%%%%%%%%%%%%%%%%%%%%%%%%%%%%%%
\DescribeMacro{\childdocby}
Each part to be included by |\input| should start with:
%
\begin{center}
\begin{tabular}{l}
|\input{childdoc.def}|\\
|\childdocby{|\textit{main}|}|\\
\end{tabular}
\end{center}
%
The directive |\childdocby| is similar to |\childdocof|
described in \secref{sec:include},
but the subsequent selection of content must be done manually.
To that end, both |\ifchilddoc| and |\ifchilddocmanual|
will be true upon processing of a part,
and the name of the part is stored in |\childdocname|.
Note that |\jobname| will be set to the filename of the current part
so that each part receives an individual |.aux| file
that does not interfere with the |.aux| file(s) of the main document.
This behaviour can be altered by the alternative form
|\childdocby[*]{|\textit{main}|}| (with a non-empty optional argument)
which uses the |.aux| file of the main document
by setting |\jobname| to \textit{main}.

%%%%%%%%%%%%%%%%%%%%%%%%%%%%%%%%%%%%%%%%%%%%%%%%%%%%%%%%%%%%%%%%%%%%%%%%%%%%%%%%
\subsection{Driver Development}
\label{sec:driver}

The \textsf{childdoc} mechanism can also be use for the development
of definition files such as \LaTeX{} styles or classes.
This case differs from the above setup with multiple parts
included by |\include| in that no |\includeonly| should be invoked.
This can be achieved by starting the include file
(before |\ProvidesPackage|) with:
%
\begin{center}
\begin{tabular}{l}
|\input{childdoc.def}|\\
|\childdocforward{|\textit{main}|}|\\
\end{tabular}
\end{center}
%
or alternatively with:
%
\begin{center}
\begin{tabular}{l}
|\input{childdoc.def}|\\
|\childdocby{|\textit{main}|}|\\
\end{tabular}
\end{center}
%
Both forms have slightly different effects as described above.
The main file is prepared as usual, see \secref{sec:include}.

%%%%%%%%%%%%%%%%%%%%%%%%%%%%%%%%%%%%%%%%%%%%%%%%%%%%%%%%%%%%%%%%%%%%%%%%%%%%%%%%
\subsection{Legacy Detection}
\label{sec:detection}

The directive |\childdocmain| in the main file can detect
whether the complete document or merely a child is to be compiled
even without using the directive |\childdocof|.
This method is deprecated because it is less robust
and there is no compelling reason to use it;
it is merely provided for backward compatibility
and it may be removed in future versions.

If the detection mechanism is to be used,
it is mandatory to correctly specify
the filename of the main file as the argument of |\childdocmain|:
%
\begin{center}
\begin{tabular}{l}
|\input{childdoc.def}|\\
|\childdocmain{|\textit{main}|}|\\
\end{tabular}
\end{center}
%
If |\jobname| does not match the argument \textit{main} of |\childdocmain|,
it is assumed that |\jobname| points to the child file to be compiled.
When using |\childdocmain| with the main file specified as argument,
it suffices to start a child file
with just |\input{|\textit{main}|}|
without loading of the package and using |\childdocof|.
If instead all processing is done
with the appropriate \textsf{childdoc} directives,
the argument of \textit{main} of |\childdocmain| can be empty.

An alternative version of the command line processing described
in \secref{sec:commandline} using the detection mechanism reads:
%
\begin{center}
|... -jobname "|\textit{target}|" "|[\textit{flags}]%
[|\def\jobname{|\textit{dest}|}|]|\input{|\textit{main}|}"|
\end{center}

%%%%%%%%%%%%%%%%%%%%%%%%%%%%%%%%%%%%%%%%%%%%%%%%%%%%%%%%%%%%%%%%%%%%%%%%%%%%%%%%
\subsection{Manual Code}
\label{sec:manual}

In case one cannot be certain whether the definitions file |childdoc.def|
is installed on the target \TeX{} distribution
and one prefers not to ship it,
it is conceivable to paste a few relevant commands into the sources.

To that end, drop all statements |\input{childdoc.def}|
and perform the replacements as outlined below.
Instead of |\childdocmain{|\textit{main}|}| add the following code
to the top of the main file:
%
\begin{center}
\begin{tabular}{l}
|\||ifdefined\childdocname\endinput\||fi\newif\ifchilddoc|\\
|\edef\childdocname{\scantokens\expandafter{\jobname\noexpand}}|\\
|\def\childdocmain{|\textit{main}|}\||ifx\childdocmain\childdocname\||else|\\
|\childdoctrue\includeonly{\childdocname}\let\jobname\childdocmain\||fi|\\
\end{tabular}
\end{center}
%
Instead of |\childdocof{|\textit{main}|}| just include the main file
at the top of each child file:
%
\begin{center}
|\input{|\textit{main}|}|
\end{center}
%
A simple redirection |\childdocforward{|\textit{dest}|}| is achieved by:
%
\begin{center}
|\def\jobname{|\textit{dest}|}\input{\jobname}|
\end{center}
%
The redirection with prefix
|\childdocforwardprefix[|\textit{prefix}|]{|\textit{dest}|}|
is accomplished by:
%
\begin{center}
\begin{tabular}{l}
|{\edef\jobname{\scantokens\expandafter{\jobname\noexpand}}|\\
|\def\redirectjob |\textit{prefix}|#1~~~{\gdef\jobname{|\textit{dest}|#1}}|\\
|\expandafter\redirectjob\jobname~~~}\input{\jobname}|
\end{tabular}
\end{center}

In an alternative approach,
child documents can be compiled by a specific command line
without additional code or specific definitions:
%
\begin{center}
|... -jobname "|\textit{target}|" "|[\textit{flags}]%
|\includeonly{|\textit{dest}|}\input{|\textit{main}|}"|
\end{center}
%

%%%%%%%%%%%%%%%%%%%%%%%%%%%%%%%%%%%%%%%%%%%%%%%%%%%%%%%%%%%%%%%%%%%%%%%%%%%%%%%%
%%%%%%%%%%%%%%%%%%%%%%%%%%%%%%%%%%%%%%%%%%%%%%%%%%%%%%%%%%%%%%%%%%%%%%%%%%%%%%%%
\section{Information}

%%%%%%%%%%%%%%%%%%%%%%%%%%%%%%%%%%%%%%%%%%%%%%%%%%%%%%%%%%%%%%%%%%%%%%%%%%%%%%%%
\subsection{Copyright}

Copyright \copyright{} 2017--2018 Niklas Beisert

This work may be distributed and/or modified under the
conditions of the \LaTeX{} Project Public License, either version 1.3
of this license or (at your option) any later version.
The latest version of this license is in
  \url{http://www.latex-project.org/lppl.txt}
and version 1.3 or later is part of all distributions of \LaTeX{}
version 2005/12/01 or later.

This work has the LPPL maintenance status `maintained'.

The Current Maintainer of this work is Niklas Beisert.

This work consists of the files |README.txt|, |childdoc.ins| and |childdoc.dtx|
as well as the derived files |childdoc.def|, |cdocsamp.tex|
with |cdocsch1.tex|, |cdocsch2.tex|, |cdocspt3.tex|, |cdocspt4.tex|,
|cdocsdrf.tex|, |cdocsfn1.tex|, |cdocsfn2.tex|
as well as |childdoc.pdf|.

%%%%%%%%%%%%%%%%%%%%%%%%%%%%%%%%%%%%%%%%%%%%%%%%%%%%%%%%%%%%%%%%%%%%%%%%%%%%%%%%
\subsection{Files and Installation}

The package consists of the files:
%
\begin{center}
\begin{tabular}{ll}
    |README.txt|   & readme file \\
    |childdoc.ins| & installation file \\
    |childdoc.dtx| & source file \\
    |childdoc.def| & definition file \\
    |cdocsamp.tex| & sample main file \\
    |cdocsch1.tex| & sample include file \\
    |cdocsch2.tex| & sample include file \\
    |cdocspt3.tex| & sample part file \\
    |cdocspt4.tex| & sample part file \\
    |cdocsdrf.tex| & sample redirection file \\
    |cdocsfn1.tex| & sample redirection file \\
    |cdocsfn2.tex| & sample redirection file \\
    |childdoc.pdf| & manual
\end{tabular}
\end{center}
%
The distribution consists of the files
|README.txt|, |childdoc.ins| and |childdoc.dtx|.
%
\begin{itemize}
\item
Run (pdf)\LaTeX{} on |childdoc.dtx|
to compile the manual |childdoc.pdf| (this file).
\item
Run \LaTeX{} on |childdoc.ins| to create the definitions file |childdoc.def|
and the sample |cdocsamp.tex| with include files
|cdocsch1.tex|, |cdocsch2.tex|, |cdocspt3.tex|, |cdocspt4.tex|,
|cdocsdrf.tex|, |cdocsfn1.tex|, |cdocsfn2.tex|.
Then copy the file |childdoc.def| to an appropriate directory of your \LaTeX{}
distribution, e.g.\ \textit{texmf-root}|/tex/latex/childdoc|.
\end{itemize}

%%%%%%%%%%%%%%%%%%%%%%%%%%%%%%%%%%%%%%%%%%%%%%%%%%%%%%%%%%%%%%%%%%%%%%%%%%%%%%%%
\subsection{Related CTAN Packages}

There are several other packages which offer a similar functionality:
%
\begin{itemize}
\item
The packages
\href{http://ctan.org/pkg/docmute}{\textsf{docmute}},
\href{http://ctan.org/pkg/includex}{\textsf{includex}} and
\href{http://ctan.org/pkg/standalone}{\textsf{standalone}}
provide commands to include only the document body of
a child file thus allowing both files to be compiled individually.
\item
The packages \href{http://ctan.org/pkg/subdocs}{\textsf{subdocs}}
and \href{http://ctan.org/pkg/subfiles}{\textsf{subfiles}}
provide structures in which the main and child documents can be
encapsulated and allowing them to be compiled individually.
The inclusion mechanism is different from the conventional |\include|.
\item
The package \href{http://ctan.org/pkg/combine}{\textsf{combine}}
is an elaborate solution to combine several documents into one.
\end{itemize}
%
See also the CTAN topic \href{http://ctan.org/topic/subdocs}{\textsf{subdocs}}
for further related packages.
The present package differs from the above solutions in that
a document structure constructed with the conventional |\include| mechanism
just needs two extra commands at the top of every file
such that all constituent files can be compiled individually.

%%%%%%%%%%%%%%%%%%%%%%%%%%%%%%%%%%%%%%%%%%%%%%%%%%%%%%%%%%%%%%%%%%%%%%%%%%%%%%%%
%\subsection{Feature Suggestions}
%
%The following is a list of features which may be useful for future
%versions of this package:
%%
%\begin{itemize}
%\item
%\ldots
%\end{itemize}

%%%%%%%%%%%%%%%%%%%%%%%%%%%%%%%%%%%%%%%%%%%%%%%%%%%%%%%%%%%%%%%%%%%%%%%%%%%%%%%%
\subsection{Revision History}

%%%%%%%%%%%%%%%%%%%%%%%%%%%%%%%%%%%%%%%%
\paragraph{v2.0:} 2018/12/30

\begin{itemize}
\item
immediate forward processing
\item
added |\childdocby| mechanism
\item
manual restructured
\end{itemize}

%%%%%%%%%%%%%%%%%%%%%%%%%%%%%%%%%%%%%%%%
\paragraph{v1.6:} 2018/01/17

\begin{itemize}
\item
application for development of include files
\item
corrections to manual
\end{itemize}

%%%%%%%%%%%%%%%%%%%%%%%%%%%%%%%%%%%%%%%%
\paragraph{v1.5:} 2017/05/21

\begin{itemize}
\item
more complete structuring introduced
\item
|\childdocof| introduced
\item
|\childdoc| renamed to |\childdocmain|
\item
|\childredirect| renamed to |\childdocforward| and |\childdocforwardprefix|
and functionality expanded
\end{itemize}

%%%%%%%%%%%%%%%%%%%%%%%%%%%%%%%%%%%%%%%%
\paragraph{v1.0:} 2017/04/27

\begin{itemize}
\item
manual and install package
\item
first version published on CTAN
\end{itemize}

%%%%%%%%%%%%%%%%%%%%%%%%%%%%%%%%%%%%%%%%
\paragraph{v0.6:} 2017/04/26

\begin{itemize}
\item
redirection mechanism added
\end{itemize}

%%%%%%%%%%%%%%%%%%%%%%%%%%%%%%%%%%%%%%%%
\paragraph{v0.5:} 2017/04/26

\begin{itemize}
\item
functionality in definition file
\end{itemize}


%%%%%%%%%%%%%%%%%%%%%%%%%%%%%%%%%%%%%%%%%%%%%%%%%%%%%%%%%%%%%%%%%%%%%%%%%%%%%%%%
%%%%%%%%%%%%%%%%%%%%%%%%%%%%%%%%%%%%%%%%%%%%%%%%%%%%%%%%%%%%%%%%%%%%%%%%%%%%%%%%
%%%%%%%%%%%%%%%%%%%%%%%%%%%%%%%%%%%%%%%%%%%%%%%%%%%%%%%%%%%%%%%%%%%%%%%%%%%%%%%%
\appendix

\settowidth\MacroIndent{\rmfamily\scriptsize 000\ }

 \DocInput{childdoc.dtx}

\end{document}
%</driver>
% \fi
%
% %%%%%%%%%%%%%%%%%%%%%%%%%%%%%%%%%%%%%%%%%%%%%%%%%%%%%%%%%%%%%%%%%%%%%%%%%%%%%%
% %%%%%%%%%%%%%%%%%%%%%%%%%%%%%%%%%%%%%%%%%%%%%%%%%%%%%%%%%%%%%%%%%%%%%%%%%%%%%%
% \section{Sample}
%\iffalse
%<*samplemain>
%\fi
%
% The following presents a sample document
% with two chapters, two parts, a title page,
% a compile flag as well as three forwarding files to set the flag.
% It consists of eight |.tex| files:
% \begin{center}
% \begin{tabular}{ll}
% |cdocsamp.tex|&main file\\
% |cdocsch1.tex|&include file for chapter 1\\
% |cdocsch2.tex|&include file for chapter 2\\
% |cdocspt3.tex|&include file for part 3\\
% |cdocspt4.tex|&include file for part 4\\
% |cdocsdrf.tex|&forwarding file for main file in draft mode\\
% |cdocsfi1.tex|&forwarding file for final version of chapter 1\\
% |cdocsfi2.tex|&forwarding file for final version of chapter 2\\
% \end{tabular}
% \end{center}
% Each of the eight files can be compiled directly by the \LaTeX{} compiler.
%
% %%%%%%%%%%%%%%%%%%%%%%%%%%%%%%%%%%%%%%
% \paragraph{Main File.}
%
% The main file is called |cdocsamp.tex|.
%
% Load the \textsf{childdoc} definitions and
% declare the filename for the main document:
%    \begin{macrocode}
\input{childdoc.def}
\childdocmain{}
%    \end{macrocode}

% Optional override for |\version| flag:
%    \begin{macrocode}
%%\ifchilddoc\else\providecommand{\version}{draft}\fi
%    \end{macrocode}

% Define the default values for the |\version| flag
% (|final| for the main file and |draft| for childs):
%    \begin{macrocode}
\ifchilddoc
\providecommand{\version}{draft}
\else
\providecommand{\version}{final}
\fi
%    \end{macrocode}

% Load the standard document class:
%    \begin{macrocode}
\documentclass[12pt]{article}
%    \end{macrocode}

% Start the document body:
%    \begin{macrocode}
\begin{document}
%    \end{macrocode}

% Declare a title page.
% Print title, part of document being processed and version flag:
%    \begin{macrocode}
\addtocounter{page}{-1}
\begin{center}
{\LARGE\bfseries{}childdoc example\par}
\vspace{1cm}
\ifchilddoc
\ifchilddocmanual part\else chapter\fi:
`\childdocname' of `\childdocjob'\par
\else
main document: `\childdocjob'\par
\fi
version: \version\par
\end{center}
\newpage
%    \end{macrocode}

% Manually include selected file,
% otherwise process as usual:
%    \begin{macrocode}
\ifchilddocmanual
\section*{part `\childdocname'}
\input{\childdocname}
\else
%    \end{macrocode}

% Include the two chapters:
%    \begin{macrocode}
\include{cdocsch1}
\include{cdocsch2}
%    \end{macrocode}

% Include the two parts unless only chapters should be displayed:
%    \begin{macrocode}
\ifchilddoc\else
\section{part three}
\input{cdocspt3}
\section{part four}
\input{cdocspt4}
\fi
%    \end{macrocode}

% Process as usual until here:
%    \begin{macrocode}
\fi
%    \end{macrocode}

% End of document body:
%    \begin{macrocode}
\end{document}
%    \end{macrocode}
%\iffalse
%</samplemain>
%\fi
%
% %%%%%%%%%%%%%%%%%%%%%%%%%%%%%%%%%%%%%%
% \paragraph{Chapter Include Files.}
%
% The include files are called |cdocsch1.tex| and |cdocsch2.tex|.
%
%\iffalse
%<*samplechap1|samplechap2>
%\fi

% Optional override for |\version| flag:
%    \begin{macrocode}
%%\providecommand{\version}{final}
%    \end{macrocode}

% Include the main document:
%    \begin{macrocode}
\input{childdoc.def}
\childdocof{cdocsamp}
%    \end{macrocode}

%\iffalse
%</samplechap1|samplechap2>
%\fi
%
%\iffalse
%<*samplechap1>
%\fi
% Some text for chapter 1:
%    \begin{macrocode}
\section{one}
some text in chapter one
%    \end{macrocode}

%\iffalse
%</samplechap1>
%\fi
% Some text for chapter 2:
%\iffalse
%<*samplechap2>
%\fi
%    \begin{macrocode}
\section{two}
more text in chapter two
%    \end{macrocode}

%\iffalse
%</samplechap2>
%\fi
%
% %%%%%%%%%%%%%%%%%%%%%%%%%%%%%%%%%%%%%%
% \paragraph{Part Include Files.}
%
% The include files are called |cdocspt3.tex| and |cdocspt4.tex|.
%
%\iffalse
%<*samplepart3|samplepart4>
%\fi

% Optional override for |\version| flag:
%    \begin{macrocode}
%%\providecommand{\version}{final}
%    \end{macrocode}

% Include the main document:
%    \begin{macrocode}
\input{childdoc.def}
\childdocby{cdocsamp}
%    \end{macrocode}

%\iffalse
%</samplepart3|samplepart4>
%\fi
%
%\iffalse
%<*samplepart3>
%\fi
% Some text for part 3:
%    \begin{macrocode}
some text in part three
%    \end{macrocode}

%\iffalse
%</samplepart3>
%\fi
% Some text for part 4:
%\iffalse
%<*samplepart4>
%\fi
%    \begin{macrocode}
more text in part four
%    \end{macrocode}

%\iffalse
%</samplepart4>
%\fi
%
% %%%%%%%%%%%%%%%%%%%%%%%%%%%%%%%%%%%%%%
% \paragraph{Forwarding for a Complete Draft.}
%
% The following forwarding file |cdocsdrf.tex|
% compiles the main document in draft mode:
%\iffalse
%<*sampledraft>
%\fi
%    \begin{macrocode}
\def\version{draft}
\input{childdoc.def}
\childdocforward{cdocsamp}
%    \end{macrocode}

%\iffalse
%</sampledraft>
%\fi
%
% %%%%%%%%%%%%%%%%%%%%%%%%%%%%%%%%%%%%%%
% \paragraph{Forwarding for Final Version of the Chapters.}
%
% The following forwarding files |cdocsfn1.tex| and |cdocsfn2.tex|
% (with identical content)
% compile the final versions of the child documents
% |cdocsch1.tex| and |cdocsch2.tex|, respectively:
%\iffalse
%<*samplefinal>
%\fi
%    \begin{macrocode}
\def\version{final}
\input{childdoc.def}
\childdocforwardprefix[cdocsamp]{cdocsfn}{cdocsch}
%    \end{macrocode}

%\iffalse
%</samplefinal>
%\fi
%
% %%%%%%%%%%%%%%%%%%%%%%%%%%%%%%%%%%%%%%
% \paragraph{Command Line Processing.}
%
% The following three command lines generate the output files
% |cdocscld|, |cdocscl1| and |cdocscl2|
% which should be identical to
% |cdocsdrf|, |cdocsch1| and |cdocsfn2|, respectively:
% \begin{center}
% \begin{tabular}{l}
% |latex -jobname cdocscld \|\\
% |  "\def\version{draft}\input{childdoc.def}\childdocforward{cdocsamp}"|\\
% |latex -jobname cdocscl1 \|\\
% |  "\input{childdoc.def}\childdocforward[cdocsamp]{cdocsch1}"|\\
% |latex -jobname cdocscl2 \|\\
% |  "\def\version{final}\input{childdoc.def}\childdocforward{cdocsch2}"|
% \end{tabular}
% \end{center}
% Note that the trailing backslash on each first line
% merely continues the input to the second line
% (for convenient cut ant paste).
% Furthermore, the command |latex| can be replaced by any
% of its alternative versions such as |pdflatex|.
%
% %%%%%%%%%%%%%%%%%%%%%%%%%%%%%%%%%%%%%%%%%%%%%%%%%%%%%%%%%%%%%%%%%%%%%%%%%%%%%%
% %%%%%%%%%%%%%%%%%%%%%%%%%%%%%%%%%%%%%%%%%%%%%%%%%%%%%%%%%%%%%%%%%%%%%%%%%%%%%%
% \section{Implementation}
%\iffalse
%<*package>
%\fi
%
% This section describes the definitions file |childdoc.def|.

% The definitions cannot be loaded using |\usepackage| or |\RequirePackage|
% which has a mechanism to prevent loading a style file more than once.
% When loading the definitions by means of |\input|
% multiple instances have to be prevented manually:
%\iffalse
%This code needs to be before the `\ProvidesFile' directive
%which is defined at the beginning of this file.
%Therefore it is also placed there and commented out here.
%</package>
%<*discard>
%\fi
%    \begin{macrocode}
\ifdefined\childdocmain\endinput\fi
%    \end{macrocode}
%\iffalse
%</discard>
%<*package>
%\fi
%
% \macro{\ifchilddoc}
% \macro{\ifchilddocmanual}
% The conditional |\ifchilddoc| tells whether a
% child (true) or main (false) document is being compiled.
% The conditional |\ifchilddocmanual| tells whether
% the |\includeonly| mechanism is used (false) or
% the selection of child files must be performed manually (true).
% The definitions initialise to false:
%    \begin{macrocode}
\newif\ifchilddoc
\newif\ifchilddocmanual
%    \end{macrocode}

% \macro{\childdocname}
% \macro{\childdocjob}
% The macro |\childdocname| stores the name of the main document
% to be compiled. The macro |\childdocjob| stores the name of
% the document on which the \LaTeX{} compiler was originally invoked.
% The content of |\jobname| cannot be compared
% to filenames specified in the source due to different catcodes.
% The following code rescans |\jobname|, stores the result
% in |\childdocname| and saves a copy in |\childdocjob|:
%    \begin{macrocode}
\edef\childdocname{\scantokens\expandafter{\jobname\noexpand}}
\let\childdocjob\childdocname
%    \end{macrocode}

% \macro{\childdocdisable}
% The macro |\childdocdisable| prevents the main file
% from being processed more than once.
% At this stage, the main document command |\childdocmain|
% is assumed to be called once again where it should do nothing.
% Any subsequent call to it should prevent
% a secondary processing of the main document
% It overwrites the forwarding commands
% |\childdocof| and |\childdocforward|
% with empty macros to prevent further inclusions of the main document:
%    \begin{macrocode}
\newcommand{\childdocdisable}
{
  \renewcommand{\childdocmain}[1]{\renewcommand{\childdocmain}[1]{\endinput}}
  \renewcommand{\childdocof}[1]{}
  \renewcommand{\childdocby}[2][]{}
  \renewcommand{\childdocforward}[2][]{}
  \renewcommand{\childdocdisable}{}
}
%    \end{macrocode}

% \macro{\childdocmain}
% The macro |\childdocmain| is to be called at the top of the main file
% with nothing or the main filename (without extension) as argument.
% First, it breaks loops.
% If the argument is not empty and does not match |\childdocname|
% (which is set by the first inclusion of |childdoc.def|),
% |\ifchilddoc| is set to true, |\includeonly| is applied to the child file
% and |\jobname| is set to the main file
% (for proper handling of |.aux| files):
%    \begin{macrocode}
\newcommand{\childdocmain}[1]
{
  \childdocdisable\childdocmain{}
  \if?#1?\else
    \begingroup
      \def\childdoctmp{#1}
      \ifx\childdoctmp\childdocname
        \def\childdoctmp{}
      \else
        \def\childdoctmp
        {
          \childdoctrue
          \includeonly{\childdocname}
          \def\childdocjob{#1}
          \def\jobname{#1}
        }
      \fi
      \expandafter
    \endgroup
    \childdoctmp
  \fi
}
%    \end{macrocode}

% \macro{\childdocof}
% The command |\childdocof| redirects
% compilation to the main file |#1|.
%    \begin{macrocode}
\newcommand{\childdocof}[1]
{
  \childdocdisable
  \childdoctrue
  \includeonly{\childdocname}
  \def\jobname{#1}
  \def\childdocjob{#1}
  \input{#1}
}
%    \end{macrocode}

% \macro{\childdocby}
% The command |\childdocby| ....
%    \begin{macrocode}
\newcommand{\childdocby}[2][]
{
  \childdocdisable
  \childdoctrue
  \childdocmanualtrue
  \if?#1?\else
    \def\jobname{#2}
  \fi
  \def\childdocjob{#2}
  \input{#2}
  \endinput
}
%    \end{macrocode}

% \macro{\childdocforward}
% The command |\childdocforward| redirects
% compilation to the main file or
% (if the optional argument is given) a child file.
% Parameters are set as if the main file
% or a child file starting with |\childdocof| was compiled.
% Then compilation is handed over to the main file:
%    \begin{macrocode}
\newcommand{\childdocforward}[2][]
{
  \begingroup
    \if?#1?
      \def\childdoctmp
      {
        \def\childdocname{#2}
        \def\childdocjob{#2}
        \def\jobname{#2}
        \input{#2}
        \endinput
      }
    \else
      \def\childdoctmp
      {
        \childdocdisable
        \def\childdocname{#2}
        \childdoctrue
        \includeonly{#2}
        \def\childdocjob{#1}
        \def\jobname{#1}
        \input{#1}
        \endinput
      }
    \fi
    \expandafter
  \endgroup
  \childdoctmp
}
%    \end{macrocode}

% \macro{\childdocforwardprefix}
% The command |\childdocforwardprefix| redirects
% compilation to the main or a child file by means of a pattern.
% The prefix |#1| in the current filename is replaced by |#2|
% and the suffix of the current filename is kept
% (it is assumed that the filename does not contain the substring `|~~~|'
% which is used as a delimiter).
% Compilation is handed over to the new file by |\childdocforward|:
%    \begin{macrocode}
\newcommand{\childdocforwardprefix}[3][]
{
  \begingroup
    \def\childdocextract #2##1~~~{\def\childdoctmp{\childdocforward[#1]{#3##1}}}
    \expandafter\childdocextract\childdocname~~~
    \expandafter
  \endgroup
  \childdoctmp
}
%    \end{macrocode}

% \macro{\childdoc}
% The deprecated macro |\childdoc| is a legacy version of |\childdocmain|:
%    \begin{macrocode}
\newcommand{\childdoc}{\childdocmain}
%    \end{macrocode}

% \macro{\childdocredirect}
% The deprecated macro |\childdocredirect| is a legacy version
% of |\childdocforward| and |\childdocforwardprefix|:
%    \begin{macrocode}
\newcommand{\childdocredirect}[2][]
{
  \begingroup
    \if?#1?
      \def\childdoctmp{\childdocforward{#2}}
    \else
      \def\childdoctmp{\childdocforwardprefix{#1}{#2}}
    \fi
    \expandafter
  \endgroup
  \childdoctmp
}
%    \end{macrocode}

%\iffalse
%</package>
%\fi
%
\endinput
|
and perform the replacements as outlined below.
Instead of |\childdocmain{|\textit{main}|}| add the following code
to the top of the main file:
%
\begin{center}
\begin{tabular}{l}
|\||ifdefined\childdocname\endinput\||fi\newif\ifchilddoc|\\
|\edef\childdocname{\scantokens\expandafter{\jobname\noexpand}}|\\
|\def\childdocmain{|\textit{main}|}\||ifx\childdocmain\childdocname\||else|\\
|\childdoctrue\includeonly{\childdocname}\let\jobname\childdocmain\||fi|\\
\end{tabular}
\end{center}
%
Instead of |\childdocof{|\textit{main}|}| just include the main file
at the top of each child file:
%
\begin{center}
|\input{|\textit{main}|}|
\end{center}
%
A simple redirection |\childdocforward{|\textit{dest}|}| is achieved by:
%
\begin{center}
|\def\jobname{|\textit{dest}|}\input{\jobname}|
\end{center}
%
The redirection with prefix
|\childdocforwardprefix[|\textit{prefix}|]{|\textit{dest}|}|
is accomplished by:
%
\begin{center}
\begin{tabular}{l}
|{\edef\jobname{\scantokens\expandafter{\jobname\noexpand}}|\\
|\def\redirectjob |\textit{prefix}|#1~~~{\gdef\jobname{|\textit{dest}|#1}}|\\
|\expandafter\redirectjob\jobname~~~}\input{\jobname}|
\end{tabular}
\end{center}

In an alternative approach,
child documents can be compiled by a specific command line
without additional code or specific definitions:
%
\begin{center}
|... -jobname "|\textit{target}|" "|[\textit{flags}]%
|\includeonly{|\textit{dest}|}\input{|\textit{main}|}"|
\end{center}
%

%%%%%%%%%%%%%%%%%%%%%%%%%%%%%%%%%%%%%%%%%%%%%%%%%%%%%%%%%%%%%%%%%%%%%%%%%%%%%%%%
%%%%%%%%%%%%%%%%%%%%%%%%%%%%%%%%%%%%%%%%%%%%%%%%%%%%%%%%%%%%%%%%%%%%%%%%%%%%%%%%
\section{Information}

%%%%%%%%%%%%%%%%%%%%%%%%%%%%%%%%%%%%%%%%%%%%%%%%%%%%%%%%%%%%%%%%%%%%%%%%%%%%%%%%
\subsection{Copyright}

Copyright \copyright{} 2017--2018 Niklas Beisert

This work may be distributed and/or modified under the
conditions of the \LaTeX{} Project Public License, either version 1.3
of this license or (at your option) any later version.
The latest version of this license is in
  \url{http://www.latex-project.org/lppl.txt}
and version 1.3 or later is part of all distributions of \LaTeX{}
version 2005/12/01 or later.

This work has the LPPL maintenance status `maintained'.

The Current Maintainer of this work is Niklas Beisert.

This work consists of the files |README.txt|, |childdoc.ins| and |childdoc.dtx|
as well as the derived files |childdoc.def|, |cdocsamp.tex|
with |cdocsch1.tex|, |cdocsch2.tex|, |cdocspt3.tex|, |cdocspt4.tex|,
|cdocsdrf.tex|, |cdocsfn1.tex|, |cdocsfn2.tex|
as well as |childdoc.pdf|.

%%%%%%%%%%%%%%%%%%%%%%%%%%%%%%%%%%%%%%%%%%%%%%%%%%%%%%%%%%%%%%%%%%%%%%%%%%%%%%%%
\subsection{Files and Installation}

The package consists of the files:
%
\begin{center}
\begin{tabular}{ll}
    |README.txt|   & readme file \\
    |childdoc.ins| & installation file \\
    |childdoc.dtx| & source file \\
    |childdoc.def| & definition file \\
    |cdocsamp.tex| & sample main file \\
    |cdocsch1.tex| & sample include file \\
    |cdocsch2.tex| & sample include file \\
    |cdocspt3.tex| & sample part file \\
    |cdocspt4.tex| & sample part file \\
    |cdocsdrf.tex| & sample redirection file \\
    |cdocsfn1.tex| & sample redirection file \\
    |cdocsfn2.tex| & sample redirection file \\
    |childdoc.pdf| & manual
\end{tabular}
\end{center}
%
The distribution consists of the files
|README.txt|, |childdoc.ins| and |childdoc.dtx|.
%
\begin{itemize}
\item
Run (pdf)\LaTeX{} on |childdoc.dtx|
to compile the manual |childdoc.pdf| (this file).
\item
Run \LaTeX{} on |childdoc.ins| to create the definitions file |childdoc.def|
and the sample |cdocsamp.tex| with include files
|cdocsch1.tex|, |cdocsch2.tex|, |cdocspt3.tex|, |cdocspt4.tex|,
|cdocsdrf.tex|, |cdocsfn1.tex|, |cdocsfn2.tex|.
Then copy the file |childdoc.def| to an appropriate directory of your \LaTeX{}
distribution, e.g.\ \textit{texmf-root}|/tex/latex/childdoc|.
\end{itemize}

%%%%%%%%%%%%%%%%%%%%%%%%%%%%%%%%%%%%%%%%%%%%%%%%%%%%%%%%%%%%%%%%%%%%%%%%%%%%%%%%
\subsection{Related CTAN Packages}

There are several other packages which offer a similar functionality:
%
\begin{itemize}
\item
The packages
\href{http://ctan.org/pkg/docmute}{\textsf{docmute}},
\href{http://ctan.org/pkg/includex}{\textsf{includex}} and
\href{http://ctan.org/pkg/standalone}{\textsf{standalone}}
provide commands to include only the document body of
a child file thus allowing both files to be compiled individually.
\item
The packages \href{http://ctan.org/pkg/subdocs}{\textsf{subdocs}}
and \href{http://ctan.org/pkg/subfiles}{\textsf{subfiles}}
provide structures in which the main and child documents can be
encapsulated and allowing them to be compiled individually.
The inclusion mechanism is different from the conventional |\include|.
\item
The package \href{http://ctan.org/pkg/combine}{\textsf{combine}}
is an elaborate solution to combine several documents into one.
\end{itemize}
%
See also the CTAN topic \href{http://ctan.org/topic/subdocs}{\textsf{subdocs}}
for further related packages.
The present package differs from the above solutions in that
a document structure constructed with the conventional |\include| mechanism
just needs two extra commands at the top of every file
such that all constituent files can be compiled individually.

%%%%%%%%%%%%%%%%%%%%%%%%%%%%%%%%%%%%%%%%%%%%%%%%%%%%%%%%%%%%%%%%%%%%%%%%%%%%%%%%
%\subsection{Feature Suggestions}
%
%The following is a list of features which may be useful for future
%versions of this package:
%%
%\begin{itemize}
%\item
%\ldots
%\end{itemize}

%%%%%%%%%%%%%%%%%%%%%%%%%%%%%%%%%%%%%%%%%%%%%%%%%%%%%%%%%%%%%%%%%%%%%%%%%%%%%%%%
\subsection{Revision History}

%%%%%%%%%%%%%%%%%%%%%%%%%%%%%%%%%%%%%%%%
\paragraph{v2.0:} 2018/12/30

\begin{itemize}
\item
immediate forward processing
\item
added |\childdocby| mechanism
\item
manual restructured
\end{itemize}

%%%%%%%%%%%%%%%%%%%%%%%%%%%%%%%%%%%%%%%%
\paragraph{v1.6:} 2018/01/17

\begin{itemize}
\item
application for development of include files
\item
corrections to manual
\end{itemize}

%%%%%%%%%%%%%%%%%%%%%%%%%%%%%%%%%%%%%%%%
\paragraph{v1.5:} 2017/05/21

\begin{itemize}
\item
more complete structuring introduced
\item
|\childdocof| introduced
\item
|\childdoc| renamed to |\childdocmain|
\item
|\childredirect| renamed to |\childdocforward| and |\childdocforwardprefix|
and functionality expanded
\end{itemize}

%%%%%%%%%%%%%%%%%%%%%%%%%%%%%%%%%%%%%%%%
\paragraph{v1.0:} 2017/04/27

\begin{itemize}
\item
manual and install package
\item
first version published on CTAN
\end{itemize}

%%%%%%%%%%%%%%%%%%%%%%%%%%%%%%%%%%%%%%%%
\paragraph{v0.6:} 2017/04/26

\begin{itemize}
\item
redirection mechanism added
\end{itemize}

%%%%%%%%%%%%%%%%%%%%%%%%%%%%%%%%%%%%%%%%
\paragraph{v0.5:} 2017/04/26

\begin{itemize}
\item
functionality in definition file
\end{itemize}


%%%%%%%%%%%%%%%%%%%%%%%%%%%%%%%%%%%%%%%%%%%%%%%%%%%%%%%%%%%%%%%%%%%%%%%%%%%%%%%%
%%%%%%%%%%%%%%%%%%%%%%%%%%%%%%%%%%%%%%%%%%%%%%%%%%%%%%%%%%%%%%%%%%%%%%%%%%%%%%%%
%%%%%%%%%%%%%%%%%%%%%%%%%%%%%%%%%%%%%%%%%%%%%%%%%%%%%%%%%%%%%%%%%%%%%%%%%%%%%%%%
\appendix

\settowidth\MacroIndent{\rmfamily\scriptsize 000\ }

 \DocInput{childdoc.dtx}

\end{document}
%</driver>
% \fi
%
% %%%%%%%%%%%%%%%%%%%%%%%%%%%%%%%%%%%%%%%%%%%%%%%%%%%%%%%%%%%%%%%%%%%%%%%%%%%%%%
% %%%%%%%%%%%%%%%%%%%%%%%%%%%%%%%%%%%%%%%%%%%%%%%%%%%%%%%%%%%%%%%%%%%%%%%%%%%%%%
% \section{Sample}
%\iffalse
%<*samplemain>
%\fi
%
% The following presents a sample document
% with two chapters, two parts, a title page,
% a compile flag as well as three forwarding files to set the flag.
% It consists of eight |.tex| files:
% \begin{center}
% \begin{tabular}{ll}
% |cdocsamp.tex|&main file\\
% |cdocsch1.tex|&include file for chapter 1\\
% |cdocsch2.tex|&include file for chapter 2\\
% |cdocspt3.tex|&include file for part 3\\
% |cdocspt4.tex|&include file for part 4\\
% |cdocsdrf.tex|&forwarding file for main file in draft mode\\
% |cdocsfi1.tex|&forwarding file for final version of chapter 1\\
% |cdocsfi2.tex|&forwarding file for final version of chapter 2\\
% \end{tabular}
% \end{center}
% Each of the eight files can be compiled directly by the \LaTeX{} compiler.
%
% %%%%%%%%%%%%%%%%%%%%%%%%%%%%%%%%%%%%%%
% \paragraph{Main File.}
%
% The main file is called |cdocsamp.tex|.
%
% Load the \textsf{childdoc} definitions and
% declare the filename for the main document:
%    \begin{macrocode}
% \iffalse
%
% childdoc.dtx Copyright (C) 2017-2018 Niklas Beisert
%
% This work may be distributed and/or modified under the
% conditions of the LaTeX Project Public License, either version 1.3
% of this license or (at your option) any later version.
% The latest version of this license is in
%   http://www.latex-project.org/lppl.txt
% and version 1.3 or later is part of all distributions of LaTeX
% version 2005/12/01 or later.
%
% This work has the LPPL maintenance status `maintained'.
%
% The Current Maintainer of this work is Niklas Beisert.
%
% This work consists of the files childdoc.dtx and childdoc.ins
% and the derived files childdoc.def and cdocsamp.tex with
% cdocsch1.tex, cdocsch2.tex, cdocsdrf.tex, cdocsfn1.tex, cdocsfn2.tex.
%
%<package>\ifdefined\childdocmain\endinput\fi
%<package>\ProvidesFile{childdoc.def}[2018/12/30 v2.0 child document driver]
%<samplemain>\ProvidesFile{cdocsamp.tex}[2018/12/30 v2.0 sample for childdoc]
%<*driver>
%\ProvidesFile{childdoc.drv}[2018/12/30 v2.0 childdoc reference manual file]
\PassOptionsToClass{10pt,a4paper}{article}
\documentclass{ltxdoc}

\usepackage[margin=35mm]{geometry}
\usepackage{hyperref}
\usepackage{hyperxmp}
\usepackage[usenames]{color}

\hypersetup{colorlinks=true}
\hypersetup{pdfstartview=FitH}
\hypersetup{pdfpagemode=UseNone}
\hypersetup{pdfsource={}}
\hypersetup{pdflang={en-UK}}
\hypersetup{pdfcopyright={Copyright 2017-2018 Niklas Beisert.
  This work may be distributed and/or modified under the
  conditions of the LaTeX Project Public License, either version 1.3
  of this license or (at your option) any later version.}}
\hypersetup{pdflicenseurl={http://www.latex-project.org/lppl.txt}}
\hypersetup{pdfcontactaddress={ETH Zurich, ITP, HIT K,
  Wolfgang-Pauli-Strasse 27}}
\hypersetup{pdfcontactpostcode={8093}}
\hypersetup{pdfcontactcity={Zurich}}
\hypersetup{pdfcontactcountry={Switzerland}}
\hypersetup{pdfcontactemail={nbeisert@itp.phys.ethz.ch}}
\hypersetup{pdfcontacturl={http://people.phys.ethz.ch/\xmptilde nbeisert/}}

\newcommand{\secref}[1]{\hyperref[#1]{section \ref*{#1}}}

\parskip1ex
\parindent0pt
\let\olditemize\itemize
\def\itemize{\olditemize\parskip0pt}

\begin{document}

\title{The \textsf{childdoc} Package}
\hypersetup{pdftitle={The childdoc Package}}
\author{Niklas Beisert\\[2ex]
  Institut f\"ur Theoretische Physik\\
  Eidgen\"ossische Technische Hochschule Z\"urich\\
  Wolfgang-Pauli-Strasse 27, 8093 Z\"urich, Switzerland\\[1ex]
  \href{mailto:nbeisert@itp.phys.ethz.ch}
  {\texttt{nbeisert@itp.phys.ethz.ch}}}
\hypersetup{pdfauthor={Niklas Beisert}}
\hypersetup{pdfsubject={Manual for the LaTeX2e Package childdoc}}
\date{30 December 2018, \textsf{v2.0}}
\maketitle

\begin{abstract}\noindent
\textsf{childdoc} is a \LaTeXe{} package
that enables the direct compilation
of document sections included by |\include|
to individual files.
\end{abstract}

\begingroup
\parskip0ex
\tableofcontents
\endgroup

%%%%%%%%%%%%%%%%%%%%%%%%%%%%%%%%%%%%%%%%%%%%%%%%%%%%%%%%%%%%%%%%%%%%%%%%%%%%%%%%
%%%%%%%%%%%%%%%%%%%%%%%%%%%%%%%%%%%%%%%%%%%%%%%%%%%%%%%%%%%%%%%%%%%%%%%%%%%%%%%%
\section{Introduction}

\LaTeX{} provides a mechanism to structure a large document (such as a book)
into a main file and several child files (containing the chapters)
using the |\include| command.
This mechanism is beneficial for documents
which span hundreds of pages in order to
make the source file(s) more manageable.
Moreover, compilation can be restricted to
selected child files by means of the |\includeonly| command.
The latter feature can be used to reduce the compilation time while editing
(this was significantly more useful in the earlier days of \LaTeX{})
or to generate a smaller document which is easier to navigate.
Another application of |\includeonly| is to generate
documents consisting of selected parts of the complete document.

However, there are a few drawbacks of the plain |\include| mechanism:
\begin{itemize}
\item
The child files cannot be compiled on their own,
they can only be compiled via the main file.
A naive editing environment
(such as a text editor with an option
to have the current file processed by \LaTeX)
may require one to switch to the main file before compiling;
attempting to compile the child file produces errors.
\item
The main file must be modified (each time)
to adjust the |\includeonly| command
to the present needs. This easily leaves the main file in a messy state.
\item
The generated document will always carry the filename
of the main document. This is inconvenient if
several child files are to be compiled and
to be kept for distribution.
\end{itemize}

The present package provides a simple interface
to make child files individually compilable by \LaTeX{}.
Compiling a child file then has the same effect as compiling
the main file with an |\includeonly| command
to select the appropriate child.
Moreover the generated document will carry the name of the child
rather than the main file.
This resolves all three above issues.

This feature is meant to make the editing of books,
thesis documents and lecture notes somewhat more convenient.
However, the package can also be used efficiently for
composing a series of documents (such as exercise sheets)
which are typically distributed individually.
It then assists the author in generating the individual documents
(potentially in different versions)
as well as a document containing the collected series.
Another application is in developing style files
or other kinds of included material
where compilation of the style file could redirect
to a sample or test file.

%%%%%%%%%%%%%%%%%%%%%%%%%%%%%%%%%%%%%%%%%%%%%%%%%%%%%%%%%%%%%%%%%%%%%%%%%%%%%%%%
%%%%%%%%%%%%%%%%%%%%%%%%%%%%%%%%%%%%%%%%%%%%%%%%%%%%%%%%%%%%%%%%%%%%%%%%%%%%%%%%
\section{Usage}

First of all, the package \textsf{childdoc} is \emph{not} a standard
\LaTeXe{} |.sty| style file! Therefore it needs to be invoked in
a non-standard way.

%%%%%%%%%%%%%%%%%%%%%%%%%%%%%%%%%%%%%%%%%%%%%%%%%%%%%%%%%%%%%%%%%%%%%%%%%%%%%%%%
\subsection{Included Files}
\label{sec:include}

%%%%%%%%%%%%%%%%%%%%%%%%%%%%%%%%%%%%%%%%
\DescribeMacro{\childdocmain}
To use the package, add the commands
\begin{center}
\begin{tabular}{l}
|\input{childdoc.def}|\\
|\childdocmain{}|\\
\end{tabular}
\end{center}
at the very top of the main \LaTeX{} file,
in particular \emph{before} the |\documentclass| statement!
The argument of |\childdocmain| should be left empty
(but it must be present).

%%%%%%%%%%%%%%%%%%%%%%%%%%%%%%%%%%%%%%%%
\DescribeMacro{\childdocof}
Furthermore, add the commands
\begin{center}
\begin{tabular}{l}
|\input{childdoc.def}|\\
|\childdocof{|\textit{main}|}|\\
\end{tabular}
\end{center}
at the top of every child file \textit{child}
which is included by |\include{|\textit{child}|}|
from within the main file
(or at least for those files to be compiled individually).
The argument \textit{main} must be the filename of the main file.

There are a couple of
considerations in setting up the main and child documents:

%%%%%%%%%%%%%%%%%%%%%%%%%%%%%%%%%%%%%%%%
\paragraph{Restrictions.}

Please note the following restrictions:
\begin{itemize}
\item
|\childdocmain| must be called with one argument \textit{main}
to ensure compatibility with earlier version of the package.
It must either be empty (|\childdocmain{}|)
or precisely match the filename of the main file in which it is specified.
See \secref{sec:detection} for further information.
\item
The filename \textit{main} must be specified without the |.tex| extension.
\item
The filename \textit{main} is case sensitive
(even in case-insensitive file systems)
due to internal string comparison.
\item
The argument \textit{main} should be fully expanded, it cannot be a macro.
\item
Subdirectories and special characters should be avoided in filenames.
\item
The command |\childdocmain{|\textit{main}|}| must be followed by a whitespace.
It should not be followed immediately by another command
or by a comment mark `|%|'.
This is because the \TeX{} parser reads the token immediately following
the argument of |\childdocmain| and puts it
at the beginning of every child section;
however, a white\-space is ignored.
\end{itemize}

%%%%%%%%%%%%%%%%%%%%%%%%%%%%%%%%%%%%%%%%
\paragraph{Content of Main File.}

It is advisable to place all content in the child files included by |\include|.
Any output contained in the main file will appear in all child documents
unless suppressed manually;
it cannot be suppressed automatically by the |\includeonly| directive
and thus should normally be avoided.
A method to include some content in the main file
by means of conditional processing is described in \secref{sec:conditional}.

%%%%%%%%%%%%%%%%%%%%%%%%%%%%%%%%%%%%%%%%
\paragraph{Page Numbering.}

When only a part of the document is compiled,
the appropriate numbering of pages
(as well as other status parameters)
is determined from the |.aux| files.
The latter contain information from previous passes.
However this information needs to propagate through
all intermediate child documents.
Therefore the page numbering in child documents may well
be inconsistent until the complete document is compiled at least once.

A useful (if unconventional) way to always ensure a consistent
page numbering is to restart the numbering in each child document
and denote the pages by `\textit{child}|.|\textit{page}'
where \textit{child} represents the chapter/section number of the child file.
This can be achieved by the command
|\numberwithin{page}{|\textit{child}|}|
of the \textsf{amsmath} package
where \textit{child} can be |chapter| or |section|
depending on the chosen structuring.
Alternatively, one can modify the macro |\thepage| appropriately
and reset the counter |page| at the start of each child file.

%%%%%%%%%%%%%%%%%%%%%%%%%%%%%%%%%%%%%%%%%%%%%%%%%%%%%%%%%%%%%%%%%%%%%%%%%%%%%%%%
\subsection{Conditional Processing}
\label{sec:conditional}

The package provides a mechanism to compile different versions
of a document. To customise the versions further some conditional processing
can come in handy to distinguish which version is being compiled.
The package provides two macros to describe the compilation context:

%%%%%%%%%%%%%%%%%%%%%%%%%%%%%%%%%%%%%%%%
\DescribeMacro{\ifchilddoc}
The conditional |\ifchilddoc| distinguishes between the compilation of
child documents and the main document:
%
\begin{center}
|\ifchilddoc |\textit{child-code}| |[|\||else |\textit{main-code}]| \||fi|
\end{center}

%%%%%%%%%%%%%%%%%%%%%%%%%%%%%%%%%%%%%%%%
\DescribeMacro{\childdocname}
\DescribeMacro{\childdocjob}
The macro |\childdocname| contains the filename (without extension)
of the main or child file being processed.
Note that |\childdocjob| will always contain the name of the main file.

%%%%%%%%%%%%%%%%%%%%%%%%%%%%%%%%%%%%%%%%
\paragraph{Title Page.}

Conditional processing can be used to include a title or banner page
in the main document when proper precautions are taken.
Importantly, the code in the main file should ensure that the page counter
(as well as other status parameters which are stored in the |.aux| files)
takes the same value after the conditional processing.
Otherwise the page numbers may take divergent values
depending on which part is compiled.

For example, a title page could be declared by:
%
\begin{center}
\begin{tabular}{l}
|\ifchilddoc\||else|\\
|\addtocounter{page}{-1}|\\
\textit{code for title page}\\
|\newpage|\\
|\||fi|
\end{tabular}
\end{center}
%
A banner page for the child documents can be generated by:
%
\begin{center}
\begin{tabular}{l}
|\ifchilddoc|\\
|\addtocounter{page}{-1}|\\
\textit{code for banner page}\\
|\newpage|\\
|\||fi|
\end{tabular}
\end{center}
%
Here one could write a message such as:
\begin{center}
|This is the part \childdocname{} of \childdocjob{}.|
\end{center}

%%%%%%%%%%%%%%%%%%%%%%%%%%%%%%%%%%%%%%%%%%%%%%%%%%%%%%%%%%%%%%%%%%%%%%%%%%%%%%%%
\subsection{Flags}
\label{sec:flags}

The package makes it easy to generate different versions
of the main or child documents.
To this end compilation flags can be defined
and assigned different default values.
They will be particularly useful in conjunction
with the forwarding mechanism described in \secref{sec:forward}.

For example, it may be useful to have a flag |\version|
which can be set to |draft| or |final|.
The document source will contain some conditional code
depending on the value of |\version|.
Suppose further, the flag should default to |final| for the main file
and to |draft| for child files
which is a natural assignment for editing the document.
This is achieved by placing the following code
in the preamble of the main document
(below the |\childdocmain| directive):
%
\begin{center}
\begin{tabular}{l}
|\ifchilddoc|\\
|\providecommand{\version}{draft}|\\
|\||else|\\
|\providecommand{\version}{final}|\\
|\||fi|
\end{tabular}
\end{center}
%
The definition by |\providecommand| makes sure
that previous definitions are not overwritten.
Further statements |\providecommand{\version}{...}|
can thus be added before the above code to override it.

For the main file, one might add a line
(between |\childdocmain| and the above block)
%
\begin{center}
|%\ifchilddoc\||else\providecommand{\version}{draft}\||fi|
\end{center}
%
which can be uncommented to produce a draft version.
Likewise one can add a line to the very top of a child file
(above the |\childdocof{|\textit{main}|}| directive)
%
\begin{center}
|%\providecommand{\version}{final}|
\end{center}
%
which can be uncommented to produce the final version of this child document.

%%%%%%%%%%%%%%%%%%%%%%%%%%%%%%%%%%%%%%%%%%%%%%%%%%%%%%%%%%%%%%%%%%%%%%%%%%%%%%%%
\subsection{Forwarding}
\label{sec:forward}

Different versions of the main or child documents
using compilation flags as described in \secref{sec:flags}
can be (permanently) stored in different files
for convenient compilation, viewing and distribution.
To this end, the package defines a command
to pass on compilation to a different file:

%%%%%%%%%%%%%%%%%%%%%%%%%%%%%%%%%%%%%%%%
\DescribeMacro{\childdocforward}
The command |\childdocforward| redirects processing to
another source file:
%
\begin{center}
\begin{tabular}{l}
|\input{childdoc.def}|\\
|\childdocforward[|\textit{main}|]{|\textit{dest}|}|\\
\end{tabular}
\end{center}
%
The argument \textit{dest} is the destination file
(without extension).
It should be the main file or one of the child files.
Note that further \textsf{childdoc} directives
such as |\childdocof| and |\childdocforward|
in the indicated file will be processed in this form.
The optional argument \textit{main}
passes on directly to the main file \textit{main}
while pretending to compile the child \textit{dest}.
This form behaves as if \textit{dest}
issues |\childdocof{|\textit{main}|}| right away,
and no further \textsf{childdoc} directives will be processed.

%%%%%%%%%%%%%%%%%%%%%%%%%%%%%%%%%%%%%%%%
\DescribeMacro{\...prefix}
In the alternative form |\childdocforwardprefix|,
%
\begin{center}
\begin{tabular}{l}
|\input{childdoc.def}|\\
|\childdocforwardprefix[|\textit{main}|]{|\textit{prefix}|}{|\textit{dest}|}|
\end{tabular}
\end{center}
%
the destination file is determined by a pattern
depending on the current file:
To make this work, the current file must be called
`{\textit{prefix}\hspace{0.2em}\textit{suffix}}'
with \textit{prefix} matching precisely the argument.
Processing is then passed on to the file
`{\textit{dest}\hspace{0.2em}\textit{suffix}}'.
Surely, the same effect is achieved by
directly specifying the
argument `{\textit{dest}\hspace{0.2em}\textit{suffix}}'
in the first form.
However, that requires to set up a different file
for each child. With the alternative form of the command
all these files can have exactly the same content
which simplifies setting them up and maintaining them.

For example, the following file |draft.tex|
with a compilation flag |\version| as described in \secref{sec:flags}
compiles the main document as a draft:
%
\begin{center}
\begin{tabular}{l}
|\def\version{draft}|\\
|\input{childdoc.def}|\\
|\childdocforward{|\textit{main}|}|
\end{tabular}
\end{center}
%
Likewise, the following files |final|\textit{nn}|.tex|
compile the final version of the child document
|child|\textit{nn}|.tex|:
%
\begin{center}
\begin{tabular}{l}
|\def\version{final}|\\
|\input{childdoc.def}|\\
|\childdocforwardprefix{final}{child}|
\end{tabular}
\end{center}
%

Note that when several versions of a main file and/or of each child file
are to be generated, it may be convenient to set up a |Makefile| or
shell script to automatise the process.

%%%%%%%%%%%%%%%%%%%%%%%%%%%%%%%%%%%%%%%%%%%%%%%%%%%%%%%%%%%%%%%%%%%%%%%%%%%%%%%%
\subsection{Command Line Processing}
\label{sec:commandline}

The effect of redirection files can also be achieved by invoking
the \LaTeX{} compiler with a more elaborate command line.
Most conveniently this should be done as part
of a shell script or a |Makefile|.

When using \textsf{childdoc} in the main file, the following
command lines effectively perform a redirection
(note that depending on the shell being used,
backslashes may have to be doubled: `|\|' $\to$ `|\\|'):
%
\begin{center}
|... -jobname "|\textit{target}|" |\\|"|[\textit{flags}]%
|\input{childdoc.def}\childdocforward[|\textit{main}|]{|\textit{dest}|}"|
\end{center}
%
Here \textit{target} is the name of the output file,
\textit{main} is the name of the main file
and \textit{dest} is the name of the main or child file to be processed
(all filenames without extensions).
The optional argument \textit{main} can be omitted
if \textit{main} matches \textit{dest}.
Optionally, compilation \textit{flags} can be defined via |\def| commands.
This command line makes the \TeX{} engine believe
it is compiling the file \textit{target}
whose content is specified as the latter parameter.
The provided code then forwards the processing to
\textit{main} or \textit{dest} as described in \secref{sec:forward}.

%%%%%%%%%%%%%%%%%%%%%%%%%%%%%%%%%%%%%%%%%%%%%%%%%%%%%%%%%%%%%%%%%%%%%%%%%%%%%%%%
\subsection{Include by Input}
\label{sec:input}

Including child documents by |\include| has some restrictions by design.
Most notably, the content of a child document always occupies
its own set of pages; pages cannot be shared between child documents.
Usually, this behaviour makes perfect sense
because each child document contain an essential part of the document.
However, in some situations it may be desirable to compose
a document from a collection of parts
without having mandatory page breaks between then.
For this case, the package
provides a mechanism to include parts
by |\input| which can also be processed individually.
However, by construction this mechanism
requires manual handling of the content to be output.

%%%%%%%%%%%%%%%%%%%%%%%%%%%%%%%%%%%%%%%%
\DescribeMacro{\ifchilddocmanual}
The main file should be prepared as usual, see \secref{sec:include}.
However, the document body must make a distinction
between processing of an individual part and of the main document, e.g.:
%
\begin{center}
\begin{tabular}{l}
|\ifchilddocmanual|\\
|\input{\childdocname}|\\
|\||else|\\
\textit{document body with }|\input{|\textit{part}|}|\\
|\||fi|
\end{tabular}
\end{center}
%
The conditional |\ifchilddocmanual| is true whenever
a part to be included by |\input| is being compiled,
and the name of the part is stored in |\childdocname|.

%%%%%%%%%%%%%%%%%%%%%%%%%%%%%%%%%%%%%%%%
\DescribeMacro{\childdocby}
Each part to be included by |\input| should start with:
%
\begin{center}
\begin{tabular}{l}
|\input{childdoc.def}|\\
|\childdocby{|\textit{main}|}|\\
\end{tabular}
\end{center}
%
The directive |\childdocby| is similar to |\childdocof|
described in \secref{sec:include},
but the subsequent selection of content must be done manually.
To that end, both |\ifchilddoc| and |\ifchilddocmanual|
will be true upon processing of a part,
and the name of the part is stored in |\childdocname|.
Note that |\jobname| will be set to the filename of the current part
so that each part receives an individual |.aux| file
that does not interfere with the |.aux| file(s) of the main document.
This behaviour can be altered by the alternative form
|\childdocby[*]{|\textit{main}|}| (with a non-empty optional argument)
which uses the |.aux| file of the main document
by setting |\jobname| to \textit{main}.

%%%%%%%%%%%%%%%%%%%%%%%%%%%%%%%%%%%%%%%%%%%%%%%%%%%%%%%%%%%%%%%%%%%%%%%%%%%%%%%%
\subsection{Driver Development}
\label{sec:driver}

The \textsf{childdoc} mechanism can also be use for the development
of definition files such as \LaTeX{} styles or classes.
This case differs from the above setup with multiple parts
included by |\include| in that no |\includeonly| should be invoked.
This can be achieved by starting the include file
(before |\ProvidesPackage|) with:
%
\begin{center}
\begin{tabular}{l}
|\input{childdoc.def}|\\
|\childdocforward{|\textit{main}|}|\\
\end{tabular}
\end{center}
%
or alternatively with:
%
\begin{center}
\begin{tabular}{l}
|\input{childdoc.def}|\\
|\childdocby{|\textit{main}|}|\\
\end{tabular}
\end{center}
%
Both forms have slightly different effects as described above.
The main file is prepared as usual, see \secref{sec:include}.

%%%%%%%%%%%%%%%%%%%%%%%%%%%%%%%%%%%%%%%%%%%%%%%%%%%%%%%%%%%%%%%%%%%%%%%%%%%%%%%%
\subsection{Legacy Detection}
\label{sec:detection}

The directive |\childdocmain| in the main file can detect
whether the complete document or merely a child is to be compiled
even without using the directive |\childdocof|.
This method is deprecated because it is less robust
and there is no compelling reason to use it;
it is merely provided for backward compatibility
and it may be removed in future versions.

If the detection mechanism is to be used,
it is mandatory to correctly specify
the filename of the main file as the argument of |\childdocmain|:
%
\begin{center}
\begin{tabular}{l}
|\input{childdoc.def}|\\
|\childdocmain{|\textit{main}|}|\\
\end{tabular}
\end{center}
%
If |\jobname| does not match the argument \textit{main} of |\childdocmain|,
it is assumed that |\jobname| points to the child file to be compiled.
When using |\childdocmain| with the main file specified as argument,
it suffices to start a child file
with just |\input{|\textit{main}|}|
without loading of the package and using |\childdocof|.
If instead all processing is done
with the appropriate \textsf{childdoc} directives,
the argument of \textit{main} of |\childdocmain| can be empty.

An alternative version of the command line processing described
in \secref{sec:commandline} using the detection mechanism reads:
%
\begin{center}
|... -jobname "|\textit{target}|" "|[\textit{flags}]%
[|\def\jobname{|\textit{dest}|}|]|\input{|\textit{main}|}"|
\end{center}

%%%%%%%%%%%%%%%%%%%%%%%%%%%%%%%%%%%%%%%%%%%%%%%%%%%%%%%%%%%%%%%%%%%%%%%%%%%%%%%%
\subsection{Manual Code}
\label{sec:manual}

In case one cannot be certain whether the definitions file |childdoc.def|
is installed on the target \TeX{} distribution
and one prefers not to ship it,
it is conceivable to paste a few relevant commands into the sources.

To that end, drop all statements |\input{childdoc.def}|
and perform the replacements as outlined below.
Instead of |\childdocmain{|\textit{main}|}| add the following code
to the top of the main file:
%
\begin{center}
\begin{tabular}{l}
|\||ifdefined\childdocname\endinput\||fi\newif\ifchilddoc|\\
|\edef\childdocname{\scantokens\expandafter{\jobname\noexpand}}|\\
|\def\childdocmain{|\textit{main}|}\||ifx\childdocmain\childdocname\||else|\\
|\childdoctrue\includeonly{\childdocname}\let\jobname\childdocmain\||fi|\\
\end{tabular}
\end{center}
%
Instead of |\childdocof{|\textit{main}|}| just include the main file
at the top of each child file:
%
\begin{center}
|\input{|\textit{main}|}|
\end{center}
%
A simple redirection |\childdocforward{|\textit{dest}|}| is achieved by:
%
\begin{center}
|\def\jobname{|\textit{dest}|}\input{\jobname}|
\end{center}
%
The redirection with prefix
|\childdocforwardprefix[|\textit{prefix}|]{|\textit{dest}|}|
is accomplished by:
%
\begin{center}
\begin{tabular}{l}
|{\edef\jobname{\scantokens\expandafter{\jobname\noexpand}}|\\
|\def\redirectjob |\textit{prefix}|#1~~~{\gdef\jobname{|\textit{dest}|#1}}|\\
|\expandafter\redirectjob\jobname~~~}\input{\jobname}|
\end{tabular}
\end{center}

In an alternative approach,
child documents can be compiled by a specific command line
without additional code or specific definitions:
%
\begin{center}
|... -jobname "|\textit{target}|" "|[\textit{flags}]%
|\includeonly{|\textit{dest}|}\input{|\textit{main}|}"|
\end{center}
%

%%%%%%%%%%%%%%%%%%%%%%%%%%%%%%%%%%%%%%%%%%%%%%%%%%%%%%%%%%%%%%%%%%%%%%%%%%%%%%%%
%%%%%%%%%%%%%%%%%%%%%%%%%%%%%%%%%%%%%%%%%%%%%%%%%%%%%%%%%%%%%%%%%%%%%%%%%%%%%%%%
\section{Information}

%%%%%%%%%%%%%%%%%%%%%%%%%%%%%%%%%%%%%%%%%%%%%%%%%%%%%%%%%%%%%%%%%%%%%%%%%%%%%%%%
\subsection{Copyright}

Copyright \copyright{} 2017--2018 Niklas Beisert

This work may be distributed and/or modified under the
conditions of the \LaTeX{} Project Public License, either version 1.3
of this license or (at your option) any later version.
The latest version of this license is in
  \url{http://www.latex-project.org/lppl.txt}
and version 1.3 or later is part of all distributions of \LaTeX{}
version 2005/12/01 or later.

This work has the LPPL maintenance status `maintained'.

The Current Maintainer of this work is Niklas Beisert.

This work consists of the files |README.txt|, |childdoc.ins| and |childdoc.dtx|
as well as the derived files |childdoc.def|, |cdocsamp.tex|
with |cdocsch1.tex|, |cdocsch2.tex|, |cdocspt3.tex|, |cdocspt4.tex|,
|cdocsdrf.tex|, |cdocsfn1.tex|, |cdocsfn2.tex|
as well as |childdoc.pdf|.

%%%%%%%%%%%%%%%%%%%%%%%%%%%%%%%%%%%%%%%%%%%%%%%%%%%%%%%%%%%%%%%%%%%%%%%%%%%%%%%%
\subsection{Files and Installation}

The package consists of the files:
%
\begin{center}
\begin{tabular}{ll}
    |README.txt|   & readme file \\
    |childdoc.ins| & installation file \\
    |childdoc.dtx| & source file \\
    |childdoc.def| & definition file \\
    |cdocsamp.tex| & sample main file \\
    |cdocsch1.tex| & sample include file \\
    |cdocsch2.tex| & sample include file \\
    |cdocspt3.tex| & sample part file \\
    |cdocspt4.tex| & sample part file \\
    |cdocsdrf.tex| & sample redirection file \\
    |cdocsfn1.tex| & sample redirection file \\
    |cdocsfn2.tex| & sample redirection file \\
    |childdoc.pdf| & manual
\end{tabular}
\end{center}
%
The distribution consists of the files
|README.txt|, |childdoc.ins| and |childdoc.dtx|.
%
\begin{itemize}
\item
Run (pdf)\LaTeX{} on |childdoc.dtx|
to compile the manual |childdoc.pdf| (this file).
\item
Run \LaTeX{} on |childdoc.ins| to create the definitions file |childdoc.def|
and the sample |cdocsamp.tex| with include files
|cdocsch1.tex|, |cdocsch2.tex|, |cdocspt3.tex|, |cdocspt4.tex|,
|cdocsdrf.tex|, |cdocsfn1.tex|, |cdocsfn2.tex|.
Then copy the file |childdoc.def| to an appropriate directory of your \LaTeX{}
distribution, e.g.\ \textit{texmf-root}|/tex/latex/childdoc|.
\end{itemize}

%%%%%%%%%%%%%%%%%%%%%%%%%%%%%%%%%%%%%%%%%%%%%%%%%%%%%%%%%%%%%%%%%%%%%%%%%%%%%%%%
\subsection{Related CTAN Packages}

There are several other packages which offer a similar functionality:
%
\begin{itemize}
\item
The packages
\href{http://ctan.org/pkg/docmute}{\textsf{docmute}},
\href{http://ctan.org/pkg/includex}{\textsf{includex}} and
\href{http://ctan.org/pkg/standalone}{\textsf{standalone}}
provide commands to include only the document body of
a child file thus allowing both files to be compiled individually.
\item
The packages \href{http://ctan.org/pkg/subdocs}{\textsf{subdocs}}
and \href{http://ctan.org/pkg/subfiles}{\textsf{subfiles}}
provide structures in which the main and child documents can be
encapsulated and allowing them to be compiled individually.
The inclusion mechanism is different from the conventional |\include|.
\item
The package \href{http://ctan.org/pkg/combine}{\textsf{combine}}
is an elaborate solution to combine several documents into one.
\end{itemize}
%
See also the CTAN topic \href{http://ctan.org/topic/subdocs}{\textsf{subdocs}}
for further related packages.
The present package differs from the above solutions in that
a document structure constructed with the conventional |\include| mechanism
just needs two extra commands at the top of every file
such that all constituent files can be compiled individually.

%%%%%%%%%%%%%%%%%%%%%%%%%%%%%%%%%%%%%%%%%%%%%%%%%%%%%%%%%%%%%%%%%%%%%%%%%%%%%%%%
%\subsection{Feature Suggestions}
%
%The following is a list of features which may be useful for future
%versions of this package:
%%
%\begin{itemize}
%\item
%\ldots
%\end{itemize}

%%%%%%%%%%%%%%%%%%%%%%%%%%%%%%%%%%%%%%%%%%%%%%%%%%%%%%%%%%%%%%%%%%%%%%%%%%%%%%%%
\subsection{Revision History}

%%%%%%%%%%%%%%%%%%%%%%%%%%%%%%%%%%%%%%%%
\paragraph{v2.0:} 2018/12/30

\begin{itemize}
\item
immediate forward processing
\item
added |\childdocby| mechanism
\item
manual restructured
\end{itemize}

%%%%%%%%%%%%%%%%%%%%%%%%%%%%%%%%%%%%%%%%
\paragraph{v1.6:} 2018/01/17

\begin{itemize}
\item
application for development of include files
\item
corrections to manual
\end{itemize}

%%%%%%%%%%%%%%%%%%%%%%%%%%%%%%%%%%%%%%%%
\paragraph{v1.5:} 2017/05/21

\begin{itemize}
\item
more complete structuring introduced
\item
|\childdocof| introduced
\item
|\childdoc| renamed to |\childdocmain|
\item
|\childredirect| renamed to |\childdocforward| and |\childdocforwardprefix|
and functionality expanded
\end{itemize}

%%%%%%%%%%%%%%%%%%%%%%%%%%%%%%%%%%%%%%%%
\paragraph{v1.0:} 2017/04/27

\begin{itemize}
\item
manual and install package
\item
first version published on CTAN
\end{itemize}

%%%%%%%%%%%%%%%%%%%%%%%%%%%%%%%%%%%%%%%%
\paragraph{v0.6:} 2017/04/26

\begin{itemize}
\item
redirection mechanism added
\end{itemize}

%%%%%%%%%%%%%%%%%%%%%%%%%%%%%%%%%%%%%%%%
\paragraph{v0.5:} 2017/04/26

\begin{itemize}
\item
functionality in definition file
\end{itemize}


%%%%%%%%%%%%%%%%%%%%%%%%%%%%%%%%%%%%%%%%%%%%%%%%%%%%%%%%%%%%%%%%%%%%%%%%%%%%%%%%
%%%%%%%%%%%%%%%%%%%%%%%%%%%%%%%%%%%%%%%%%%%%%%%%%%%%%%%%%%%%%%%%%%%%%%%%%%%%%%%%
%%%%%%%%%%%%%%%%%%%%%%%%%%%%%%%%%%%%%%%%%%%%%%%%%%%%%%%%%%%%%%%%%%%%%%%%%%%%%%%%
\appendix

\settowidth\MacroIndent{\rmfamily\scriptsize 000\ }

 \DocInput{childdoc.dtx}

\end{document}
%</driver>
% \fi
%
% %%%%%%%%%%%%%%%%%%%%%%%%%%%%%%%%%%%%%%%%%%%%%%%%%%%%%%%%%%%%%%%%%%%%%%%%%%%%%%
% %%%%%%%%%%%%%%%%%%%%%%%%%%%%%%%%%%%%%%%%%%%%%%%%%%%%%%%%%%%%%%%%%%%%%%%%%%%%%%
% \section{Sample}
%\iffalse
%<*samplemain>
%\fi
%
% The following presents a sample document
% with two chapters, two parts, a title page,
% a compile flag as well as three forwarding files to set the flag.
% It consists of eight |.tex| files:
% \begin{center}
% \begin{tabular}{ll}
% |cdocsamp.tex|&main file\\
% |cdocsch1.tex|&include file for chapter 1\\
% |cdocsch2.tex|&include file for chapter 2\\
% |cdocspt3.tex|&include file for part 3\\
% |cdocspt4.tex|&include file for part 4\\
% |cdocsdrf.tex|&forwarding file for main file in draft mode\\
% |cdocsfi1.tex|&forwarding file for final version of chapter 1\\
% |cdocsfi2.tex|&forwarding file for final version of chapter 2\\
% \end{tabular}
% \end{center}
% Each of the eight files can be compiled directly by the \LaTeX{} compiler.
%
% %%%%%%%%%%%%%%%%%%%%%%%%%%%%%%%%%%%%%%
% \paragraph{Main File.}
%
% The main file is called |cdocsamp.tex|.
%
% Load the \textsf{childdoc} definitions and
% declare the filename for the main document:
%    \begin{macrocode}
\input{childdoc.def}
\childdocmain{}
%    \end{macrocode}

% Optional override for |\version| flag:
%    \begin{macrocode}
%%\ifchilddoc\else\providecommand{\version}{draft}\fi
%    \end{macrocode}

% Define the default values for the |\version| flag
% (|final| for the main file and |draft| for childs):
%    \begin{macrocode}
\ifchilddoc
\providecommand{\version}{draft}
\else
\providecommand{\version}{final}
\fi
%    \end{macrocode}

% Load the standard document class:
%    \begin{macrocode}
\documentclass[12pt]{article}
%    \end{macrocode}

% Start the document body:
%    \begin{macrocode}
\begin{document}
%    \end{macrocode}

% Declare a title page.
% Print title, part of document being processed and version flag:
%    \begin{macrocode}
\addtocounter{page}{-1}
\begin{center}
{\LARGE\bfseries{}childdoc example\par}
\vspace{1cm}
\ifchilddoc
\ifchilddocmanual part\else chapter\fi:
`\childdocname' of `\childdocjob'\par
\else
main document: `\childdocjob'\par
\fi
version: \version\par
\end{center}
\newpage
%    \end{macrocode}

% Manually include selected file,
% otherwise process as usual:
%    \begin{macrocode}
\ifchilddocmanual
\section*{part `\childdocname'}
\input{\childdocname}
\else
%    \end{macrocode}

% Include the two chapters:
%    \begin{macrocode}
\include{cdocsch1}
\include{cdocsch2}
%    \end{macrocode}

% Include the two parts unless only chapters should be displayed:
%    \begin{macrocode}
\ifchilddoc\else
\section{part three}
\input{cdocspt3}
\section{part four}
\input{cdocspt4}
\fi
%    \end{macrocode}

% Process as usual until here:
%    \begin{macrocode}
\fi
%    \end{macrocode}

% End of document body:
%    \begin{macrocode}
\end{document}
%    \end{macrocode}
%\iffalse
%</samplemain>
%\fi
%
% %%%%%%%%%%%%%%%%%%%%%%%%%%%%%%%%%%%%%%
% \paragraph{Chapter Include Files.}
%
% The include files are called |cdocsch1.tex| and |cdocsch2.tex|.
%
%\iffalse
%<*samplechap1|samplechap2>
%\fi

% Optional override for |\version| flag:
%    \begin{macrocode}
%%\providecommand{\version}{final}
%    \end{macrocode}

% Include the main document:
%    \begin{macrocode}
\input{childdoc.def}
\childdocof{cdocsamp}
%    \end{macrocode}

%\iffalse
%</samplechap1|samplechap2>
%\fi
%
%\iffalse
%<*samplechap1>
%\fi
% Some text for chapter 1:
%    \begin{macrocode}
\section{one}
some text in chapter one
%    \end{macrocode}

%\iffalse
%</samplechap1>
%\fi
% Some text for chapter 2:
%\iffalse
%<*samplechap2>
%\fi
%    \begin{macrocode}
\section{two}
more text in chapter two
%    \end{macrocode}

%\iffalse
%</samplechap2>
%\fi
%
% %%%%%%%%%%%%%%%%%%%%%%%%%%%%%%%%%%%%%%
% \paragraph{Part Include Files.}
%
% The include files are called |cdocspt3.tex| and |cdocspt4.tex|.
%
%\iffalse
%<*samplepart3|samplepart4>
%\fi

% Optional override for |\version| flag:
%    \begin{macrocode}
%%\providecommand{\version}{final}
%    \end{macrocode}

% Include the main document:
%    \begin{macrocode}
\input{childdoc.def}
\childdocby{cdocsamp}
%    \end{macrocode}

%\iffalse
%</samplepart3|samplepart4>
%\fi
%
%\iffalse
%<*samplepart3>
%\fi
% Some text for part 3:
%    \begin{macrocode}
some text in part three
%    \end{macrocode}

%\iffalse
%</samplepart3>
%\fi
% Some text for part 4:
%\iffalse
%<*samplepart4>
%\fi
%    \begin{macrocode}
more text in part four
%    \end{macrocode}

%\iffalse
%</samplepart4>
%\fi
%
% %%%%%%%%%%%%%%%%%%%%%%%%%%%%%%%%%%%%%%
% \paragraph{Forwarding for a Complete Draft.}
%
% The following forwarding file |cdocsdrf.tex|
% compiles the main document in draft mode:
%\iffalse
%<*sampledraft>
%\fi
%    \begin{macrocode}
\def\version{draft}
\input{childdoc.def}
\childdocforward{cdocsamp}
%    \end{macrocode}

%\iffalse
%</sampledraft>
%\fi
%
% %%%%%%%%%%%%%%%%%%%%%%%%%%%%%%%%%%%%%%
% \paragraph{Forwarding for Final Version of the Chapters.}
%
% The following forwarding files |cdocsfn1.tex| and |cdocsfn2.tex|
% (with identical content)
% compile the final versions of the child documents
% |cdocsch1.tex| and |cdocsch2.tex|, respectively:
%\iffalse
%<*samplefinal>
%\fi
%    \begin{macrocode}
\def\version{final}
\input{childdoc.def}
\childdocforwardprefix[cdocsamp]{cdocsfn}{cdocsch}
%    \end{macrocode}

%\iffalse
%</samplefinal>
%\fi
%
% %%%%%%%%%%%%%%%%%%%%%%%%%%%%%%%%%%%%%%
% \paragraph{Command Line Processing.}
%
% The following three command lines generate the output files
% |cdocscld|, |cdocscl1| and |cdocscl2|
% which should be identical to
% |cdocsdrf|, |cdocsch1| and |cdocsfn2|, respectively:
% \begin{center}
% \begin{tabular}{l}
% |latex -jobname cdocscld \|\\
% |  "\def\version{draft}\input{childdoc.def}\childdocforward{cdocsamp}"|\\
% |latex -jobname cdocscl1 \|\\
% |  "\input{childdoc.def}\childdocforward[cdocsamp]{cdocsch1}"|\\
% |latex -jobname cdocscl2 \|\\
% |  "\def\version{final}\input{childdoc.def}\childdocforward{cdocsch2}"|
% \end{tabular}
% \end{center}
% Note that the trailing backslash on each first line
% merely continues the input to the second line
% (for convenient cut ant paste).
% Furthermore, the command |latex| can be replaced by any
% of its alternative versions such as |pdflatex|.
%
% %%%%%%%%%%%%%%%%%%%%%%%%%%%%%%%%%%%%%%%%%%%%%%%%%%%%%%%%%%%%%%%%%%%%%%%%%%%%%%
% %%%%%%%%%%%%%%%%%%%%%%%%%%%%%%%%%%%%%%%%%%%%%%%%%%%%%%%%%%%%%%%%%%%%%%%%%%%%%%
% \section{Implementation}
%\iffalse
%<*package>
%\fi
%
% This section describes the definitions file |childdoc.def|.

% The definitions cannot be loaded using |\usepackage| or |\RequirePackage|
% which has a mechanism to prevent loading a style file more than once.
% When loading the definitions by means of |\input|
% multiple instances have to be prevented manually:
%\iffalse
%This code needs to be before the `\ProvidesFile' directive
%which is defined at the beginning of this file.
%Therefore it is also placed there and commented out here.
%</package>
%<*discard>
%\fi
%    \begin{macrocode}
\ifdefined\childdocmain\endinput\fi
%    \end{macrocode}
%\iffalse
%</discard>
%<*package>
%\fi
%
% \macro{\ifchilddoc}
% \macro{\ifchilddocmanual}
% The conditional |\ifchilddoc| tells whether a
% child (true) or main (false) document is being compiled.
% The conditional |\ifchilddocmanual| tells whether
% the |\includeonly| mechanism is used (false) or
% the selection of child files must be performed manually (true).
% The definitions initialise to false:
%    \begin{macrocode}
\newif\ifchilddoc
\newif\ifchilddocmanual
%    \end{macrocode}

% \macro{\childdocname}
% \macro{\childdocjob}
% The macro |\childdocname| stores the name of the main document
% to be compiled. The macro |\childdocjob| stores the name of
% the document on which the \LaTeX{} compiler was originally invoked.
% The content of |\jobname| cannot be compared
% to filenames specified in the source due to different catcodes.
% The following code rescans |\jobname|, stores the result
% in |\childdocname| and saves a copy in |\childdocjob|:
%    \begin{macrocode}
\edef\childdocname{\scantokens\expandafter{\jobname\noexpand}}
\let\childdocjob\childdocname
%    \end{macrocode}

% \macro{\childdocdisable}
% The macro |\childdocdisable| prevents the main file
% from being processed more than once.
% At this stage, the main document command |\childdocmain|
% is assumed to be called once again where it should do nothing.
% Any subsequent call to it should prevent
% a secondary processing of the main document
% It overwrites the forwarding commands
% |\childdocof| and |\childdocforward|
% with empty macros to prevent further inclusions of the main document:
%    \begin{macrocode}
\newcommand{\childdocdisable}
{
  \renewcommand{\childdocmain}[1]{\renewcommand{\childdocmain}[1]{\endinput}}
  \renewcommand{\childdocof}[1]{}
  \renewcommand{\childdocby}[2][]{}
  \renewcommand{\childdocforward}[2][]{}
  \renewcommand{\childdocdisable}{}
}
%    \end{macrocode}

% \macro{\childdocmain}
% The macro |\childdocmain| is to be called at the top of the main file
% with nothing or the main filename (without extension) as argument.
% First, it breaks loops.
% If the argument is not empty and does not match |\childdocname|
% (which is set by the first inclusion of |childdoc.def|),
% |\ifchilddoc| is set to true, |\includeonly| is applied to the child file
% and |\jobname| is set to the main file
% (for proper handling of |.aux| files):
%    \begin{macrocode}
\newcommand{\childdocmain}[1]
{
  \childdocdisable\childdocmain{}
  \if?#1?\else
    \begingroup
      \def\childdoctmp{#1}
      \ifx\childdoctmp\childdocname
        \def\childdoctmp{}
      \else
        \def\childdoctmp
        {
          \childdoctrue
          \includeonly{\childdocname}
          \def\childdocjob{#1}
          \def\jobname{#1}
        }
      \fi
      \expandafter
    \endgroup
    \childdoctmp
  \fi
}
%    \end{macrocode}

% \macro{\childdocof}
% The command |\childdocof| redirects
% compilation to the main file |#1|.
%    \begin{macrocode}
\newcommand{\childdocof}[1]
{
  \childdocdisable
  \childdoctrue
  \includeonly{\childdocname}
  \def\jobname{#1}
  \def\childdocjob{#1}
  \input{#1}
}
%    \end{macrocode}

% \macro{\childdocby}
% The command |\childdocby| ....
%    \begin{macrocode}
\newcommand{\childdocby}[2][]
{
  \childdocdisable
  \childdoctrue
  \childdocmanualtrue
  \if?#1?\else
    \def\jobname{#2}
  \fi
  \def\childdocjob{#2}
  \input{#2}
  \endinput
}
%    \end{macrocode}

% \macro{\childdocforward}
% The command |\childdocforward| redirects
% compilation to the main file or
% (if the optional argument is given) a child file.
% Parameters are set as if the main file
% or a child file starting with |\childdocof| was compiled.
% Then compilation is handed over to the main file:
%    \begin{macrocode}
\newcommand{\childdocforward}[2][]
{
  \begingroup
    \if?#1?
      \def\childdoctmp
      {
        \def\childdocname{#2}
        \def\childdocjob{#2}
        \def\jobname{#2}
        \input{#2}
        \endinput
      }
    \else
      \def\childdoctmp
      {
        \childdocdisable
        \def\childdocname{#2}
        \childdoctrue
        \includeonly{#2}
        \def\childdocjob{#1}
        \def\jobname{#1}
        \input{#1}
        \endinput
      }
    \fi
    \expandafter
  \endgroup
  \childdoctmp
}
%    \end{macrocode}

% \macro{\childdocforwardprefix}
% The command |\childdocforwardprefix| redirects
% compilation to the main or a child file by means of a pattern.
% The prefix |#1| in the current filename is replaced by |#2|
% and the suffix of the current filename is kept
% (it is assumed that the filename does not contain the substring `|~~~|'
% which is used as a delimiter).
% Compilation is handed over to the new file by |\childdocforward|:
%    \begin{macrocode}
\newcommand{\childdocforwardprefix}[3][]
{
  \begingroup
    \def\childdocextract #2##1~~~{\def\childdoctmp{\childdocforward[#1]{#3##1}}}
    \expandafter\childdocextract\childdocname~~~
    \expandafter
  \endgroup
  \childdoctmp
}
%    \end{macrocode}

% \macro{\childdoc}
% The deprecated macro |\childdoc| is a legacy version of |\childdocmain|:
%    \begin{macrocode}
\newcommand{\childdoc}{\childdocmain}
%    \end{macrocode}

% \macro{\childdocredirect}
% The deprecated macro |\childdocredirect| is a legacy version
% of |\childdocforward| and |\childdocforwardprefix|:
%    \begin{macrocode}
\newcommand{\childdocredirect}[2][]
{
  \begingroup
    \if?#1?
      \def\childdoctmp{\childdocforward{#2}}
    \else
      \def\childdoctmp{\childdocforwardprefix{#1}{#2}}
    \fi
    \expandafter
  \endgroup
  \childdoctmp
}
%    \end{macrocode}

%\iffalse
%</package>
%\fi
%
\endinput

\childdocmain{}
%    \end{macrocode}

% Optional override for |\version| flag:
%    \begin{macrocode}
%%\ifchilddoc\else\providecommand{\version}{draft}\fi
%    \end{macrocode}

% Define the default values for the |\version| flag
% (|final| for the main file and |draft| for childs):
%    \begin{macrocode}
\ifchilddoc
\providecommand{\version}{draft}
\else
\providecommand{\version}{final}
\fi
%    \end{macrocode}

% Load the standard document class:
%    \begin{macrocode}
\documentclass[12pt]{article}
%    \end{macrocode}

% Start the document body:
%    \begin{macrocode}
\begin{document}
%    \end{macrocode}

% Declare a title page.
% Print title, part of document being processed and version flag:
%    \begin{macrocode}
\addtocounter{page}{-1}
\begin{center}
{\LARGE\bfseries{}childdoc example\par}
\vspace{1cm}
\ifchilddoc
\ifchilddocmanual part\else chapter\fi:
`\childdocname' of `\childdocjob'\par
\else
main document: `\childdocjob'\par
\fi
version: \version\par
\end{center}
\newpage
%    \end{macrocode}

% Manually include selected file,
% otherwise process as usual:
%    \begin{macrocode}
\ifchilddocmanual
\section*{part `\childdocname'}
\input{\childdocname}
\else
%    \end{macrocode}

% Include the two chapters:
%    \begin{macrocode}
\include{cdocsch1}
\include{cdocsch2}
%    \end{macrocode}

% Include the two parts unless only chapters should be displayed:
%    \begin{macrocode}
\ifchilddoc\else
\section{part three}
\input{cdocspt3}
\section{part four}
\input{cdocspt4}
\fi
%    \end{macrocode}

% Process as usual until here:
%    \begin{macrocode}
\fi
%    \end{macrocode}

% End of document body:
%    \begin{macrocode}
\end{document}
%    \end{macrocode}
%\iffalse
%</samplemain>
%\fi
%
% %%%%%%%%%%%%%%%%%%%%%%%%%%%%%%%%%%%%%%
% \paragraph{Chapter Include Files.}
%
% The include files are called |cdocsch1.tex| and |cdocsch2.tex|.
%
%\iffalse
%<*samplechap1|samplechap2>
%\fi

% Optional override for |\version| flag:
%    \begin{macrocode}
%%\providecommand{\version}{final}
%    \end{macrocode}

% Include the main document:
%    \begin{macrocode}
% \iffalse
%
% childdoc.dtx Copyright (C) 2017-2018 Niklas Beisert
%
% This work may be distributed and/or modified under the
% conditions of the LaTeX Project Public License, either version 1.3
% of this license or (at your option) any later version.
% The latest version of this license is in
%   http://www.latex-project.org/lppl.txt
% and version 1.3 or later is part of all distributions of LaTeX
% version 2005/12/01 or later.
%
% This work has the LPPL maintenance status `maintained'.
%
% The Current Maintainer of this work is Niklas Beisert.
%
% This work consists of the files childdoc.dtx and childdoc.ins
% and the derived files childdoc.def and cdocsamp.tex with
% cdocsch1.tex, cdocsch2.tex, cdocsdrf.tex, cdocsfn1.tex, cdocsfn2.tex.
%
%<package>\ifdefined\childdocmain\endinput\fi
%<package>\ProvidesFile{childdoc.def}[2018/12/30 v2.0 child document driver]
%<samplemain>\ProvidesFile{cdocsamp.tex}[2018/12/30 v2.0 sample for childdoc]
%<*driver>
%\ProvidesFile{childdoc.drv}[2018/12/30 v2.0 childdoc reference manual file]
\PassOptionsToClass{10pt,a4paper}{article}
\documentclass{ltxdoc}

\usepackage[margin=35mm]{geometry}
\usepackage{hyperref}
\usepackage{hyperxmp}
\usepackage[usenames]{color}

\hypersetup{colorlinks=true}
\hypersetup{pdfstartview=FitH}
\hypersetup{pdfpagemode=UseNone}
\hypersetup{pdfsource={}}
\hypersetup{pdflang={en-UK}}
\hypersetup{pdfcopyright={Copyright 2017-2018 Niklas Beisert.
  This work may be distributed and/or modified under the
  conditions of the LaTeX Project Public License, either version 1.3
  of this license or (at your option) any later version.}}
\hypersetup{pdflicenseurl={http://www.latex-project.org/lppl.txt}}
\hypersetup{pdfcontactaddress={ETH Zurich, ITP, HIT K,
  Wolfgang-Pauli-Strasse 27}}
\hypersetup{pdfcontactpostcode={8093}}
\hypersetup{pdfcontactcity={Zurich}}
\hypersetup{pdfcontactcountry={Switzerland}}
\hypersetup{pdfcontactemail={nbeisert@itp.phys.ethz.ch}}
\hypersetup{pdfcontacturl={http://people.phys.ethz.ch/\xmptilde nbeisert/}}

\newcommand{\secref}[1]{\hyperref[#1]{section \ref*{#1}}}

\parskip1ex
\parindent0pt
\let\olditemize\itemize
\def\itemize{\olditemize\parskip0pt}

\begin{document}

\title{The \textsf{childdoc} Package}
\hypersetup{pdftitle={The childdoc Package}}
\author{Niklas Beisert\\[2ex]
  Institut f\"ur Theoretische Physik\\
  Eidgen\"ossische Technische Hochschule Z\"urich\\
  Wolfgang-Pauli-Strasse 27, 8093 Z\"urich, Switzerland\\[1ex]
  \href{mailto:nbeisert@itp.phys.ethz.ch}
  {\texttt{nbeisert@itp.phys.ethz.ch}}}
\hypersetup{pdfauthor={Niklas Beisert}}
\hypersetup{pdfsubject={Manual for the LaTeX2e Package childdoc}}
\date{30 December 2018, \textsf{v2.0}}
\maketitle

\begin{abstract}\noindent
\textsf{childdoc} is a \LaTeXe{} package
that enables the direct compilation
of document sections included by |\include|
to individual files.
\end{abstract}

\begingroup
\parskip0ex
\tableofcontents
\endgroup

%%%%%%%%%%%%%%%%%%%%%%%%%%%%%%%%%%%%%%%%%%%%%%%%%%%%%%%%%%%%%%%%%%%%%%%%%%%%%%%%
%%%%%%%%%%%%%%%%%%%%%%%%%%%%%%%%%%%%%%%%%%%%%%%%%%%%%%%%%%%%%%%%%%%%%%%%%%%%%%%%
\section{Introduction}

\LaTeX{} provides a mechanism to structure a large document (such as a book)
into a main file and several child files (containing the chapters)
using the |\include| command.
This mechanism is beneficial for documents
which span hundreds of pages in order to
make the source file(s) more manageable.
Moreover, compilation can be restricted to
selected child files by means of the |\includeonly| command.
The latter feature can be used to reduce the compilation time while editing
(this was significantly more useful in the earlier days of \LaTeX{})
or to generate a smaller document which is easier to navigate.
Another application of |\includeonly| is to generate
documents consisting of selected parts of the complete document.

However, there are a few drawbacks of the plain |\include| mechanism:
\begin{itemize}
\item
The child files cannot be compiled on their own,
they can only be compiled via the main file.
A naive editing environment
(such as a text editor with an option
to have the current file processed by \LaTeX)
may require one to switch to the main file before compiling;
attempting to compile the child file produces errors.
\item
The main file must be modified (each time)
to adjust the |\includeonly| command
to the present needs. This easily leaves the main file in a messy state.
\item
The generated document will always carry the filename
of the main document. This is inconvenient if
several child files are to be compiled and
to be kept for distribution.
\end{itemize}

The present package provides a simple interface
to make child files individually compilable by \LaTeX{}.
Compiling a child file then has the same effect as compiling
the main file with an |\includeonly| command
to select the appropriate child.
Moreover the generated document will carry the name of the child
rather than the main file.
This resolves all three above issues.

This feature is meant to make the editing of books,
thesis documents and lecture notes somewhat more convenient.
However, the package can also be used efficiently for
composing a series of documents (such as exercise sheets)
which are typically distributed individually.
It then assists the author in generating the individual documents
(potentially in different versions)
as well as a document containing the collected series.
Another application is in developing style files
or other kinds of included material
where compilation of the style file could redirect
to a sample or test file.

%%%%%%%%%%%%%%%%%%%%%%%%%%%%%%%%%%%%%%%%%%%%%%%%%%%%%%%%%%%%%%%%%%%%%%%%%%%%%%%%
%%%%%%%%%%%%%%%%%%%%%%%%%%%%%%%%%%%%%%%%%%%%%%%%%%%%%%%%%%%%%%%%%%%%%%%%%%%%%%%%
\section{Usage}

First of all, the package \textsf{childdoc} is \emph{not} a standard
\LaTeXe{} |.sty| style file! Therefore it needs to be invoked in
a non-standard way.

%%%%%%%%%%%%%%%%%%%%%%%%%%%%%%%%%%%%%%%%%%%%%%%%%%%%%%%%%%%%%%%%%%%%%%%%%%%%%%%%
\subsection{Included Files}
\label{sec:include}

%%%%%%%%%%%%%%%%%%%%%%%%%%%%%%%%%%%%%%%%
\DescribeMacro{\childdocmain}
To use the package, add the commands
\begin{center}
\begin{tabular}{l}
|\input{childdoc.def}|\\
|\childdocmain{}|\\
\end{tabular}
\end{center}
at the very top of the main \LaTeX{} file,
in particular \emph{before} the |\documentclass| statement!
The argument of |\childdocmain| should be left empty
(but it must be present).

%%%%%%%%%%%%%%%%%%%%%%%%%%%%%%%%%%%%%%%%
\DescribeMacro{\childdocof}
Furthermore, add the commands
\begin{center}
\begin{tabular}{l}
|\input{childdoc.def}|\\
|\childdocof{|\textit{main}|}|\\
\end{tabular}
\end{center}
at the top of every child file \textit{child}
which is included by |\include{|\textit{child}|}|
from within the main file
(or at least for those files to be compiled individually).
The argument \textit{main} must be the filename of the main file.

There are a couple of
considerations in setting up the main and child documents:

%%%%%%%%%%%%%%%%%%%%%%%%%%%%%%%%%%%%%%%%
\paragraph{Restrictions.}

Please note the following restrictions:
\begin{itemize}
\item
|\childdocmain| must be called with one argument \textit{main}
to ensure compatibility with earlier version of the package.
It must either be empty (|\childdocmain{}|)
or precisely match the filename of the main file in which it is specified.
See \secref{sec:detection} for further information.
\item
The filename \textit{main} must be specified without the |.tex| extension.
\item
The filename \textit{main} is case sensitive
(even in case-insensitive file systems)
due to internal string comparison.
\item
The argument \textit{main} should be fully expanded, it cannot be a macro.
\item
Subdirectories and special characters should be avoided in filenames.
\item
The command |\childdocmain{|\textit{main}|}| must be followed by a whitespace.
It should not be followed immediately by another command
or by a comment mark `|%|'.
This is because the \TeX{} parser reads the token immediately following
the argument of |\childdocmain| and puts it
at the beginning of every child section;
however, a white\-space is ignored.
\end{itemize}

%%%%%%%%%%%%%%%%%%%%%%%%%%%%%%%%%%%%%%%%
\paragraph{Content of Main File.}

It is advisable to place all content in the child files included by |\include|.
Any output contained in the main file will appear in all child documents
unless suppressed manually;
it cannot be suppressed automatically by the |\includeonly| directive
and thus should normally be avoided.
A method to include some content in the main file
by means of conditional processing is described in \secref{sec:conditional}.

%%%%%%%%%%%%%%%%%%%%%%%%%%%%%%%%%%%%%%%%
\paragraph{Page Numbering.}

When only a part of the document is compiled,
the appropriate numbering of pages
(as well as other status parameters)
is determined from the |.aux| files.
The latter contain information from previous passes.
However this information needs to propagate through
all intermediate child documents.
Therefore the page numbering in child documents may well
be inconsistent until the complete document is compiled at least once.

A useful (if unconventional) way to always ensure a consistent
page numbering is to restart the numbering in each child document
and denote the pages by `\textit{child}|.|\textit{page}'
where \textit{child} represents the chapter/section number of the child file.
This can be achieved by the command
|\numberwithin{page}{|\textit{child}|}|
of the \textsf{amsmath} package
where \textit{child} can be |chapter| or |section|
depending on the chosen structuring.
Alternatively, one can modify the macro |\thepage| appropriately
and reset the counter |page| at the start of each child file.

%%%%%%%%%%%%%%%%%%%%%%%%%%%%%%%%%%%%%%%%%%%%%%%%%%%%%%%%%%%%%%%%%%%%%%%%%%%%%%%%
\subsection{Conditional Processing}
\label{sec:conditional}

The package provides a mechanism to compile different versions
of a document. To customise the versions further some conditional processing
can come in handy to distinguish which version is being compiled.
The package provides two macros to describe the compilation context:

%%%%%%%%%%%%%%%%%%%%%%%%%%%%%%%%%%%%%%%%
\DescribeMacro{\ifchilddoc}
The conditional |\ifchilddoc| distinguishes between the compilation of
child documents and the main document:
%
\begin{center}
|\ifchilddoc |\textit{child-code}| |[|\||else |\textit{main-code}]| \||fi|
\end{center}

%%%%%%%%%%%%%%%%%%%%%%%%%%%%%%%%%%%%%%%%
\DescribeMacro{\childdocname}
\DescribeMacro{\childdocjob}
The macro |\childdocname| contains the filename (without extension)
of the main or child file being processed.
Note that |\childdocjob| will always contain the name of the main file.

%%%%%%%%%%%%%%%%%%%%%%%%%%%%%%%%%%%%%%%%
\paragraph{Title Page.}

Conditional processing can be used to include a title or banner page
in the main document when proper precautions are taken.
Importantly, the code in the main file should ensure that the page counter
(as well as other status parameters which are stored in the |.aux| files)
takes the same value after the conditional processing.
Otherwise the page numbers may take divergent values
depending on which part is compiled.

For example, a title page could be declared by:
%
\begin{center}
\begin{tabular}{l}
|\ifchilddoc\||else|\\
|\addtocounter{page}{-1}|\\
\textit{code for title page}\\
|\newpage|\\
|\||fi|
\end{tabular}
\end{center}
%
A banner page for the child documents can be generated by:
%
\begin{center}
\begin{tabular}{l}
|\ifchilddoc|\\
|\addtocounter{page}{-1}|\\
\textit{code for banner page}\\
|\newpage|\\
|\||fi|
\end{tabular}
\end{center}
%
Here one could write a message such as:
\begin{center}
|This is the part \childdocname{} of \childdocjob{}.|
\end{center}

%%%%%%%%%%%%%%%%%%%%%%%%%%%%%%%%%%%%%%%%%%%%%%%%%%%%%%%%%%%%%%%%%%%%%%%%%%%%%%%%
\subsection{Flags}
\label{sec:flags}

The package makes it easy to generate different versions
of the main or child documents.
To this end compilation flags can be defined
and assigned different default values.
They will be particularly useful in conjunction
with the forwarding mechanism described in \secref{sec:forward}.

For example, it may be useful to have a flag |\version|
which can be set to |draft| or |final|.
The document source will contain some conditional code
depending on the value of |\version|.
Suppose further, the flag should default to |final| for the main file
and to |draft| for child files
which is a natural assignment for editing the document.
This is achieved by placing the following code
in the preamble of the main document
(below the |\childdocmain| directive):
%
\begin{center}
\begin{tabular}{l}
|\ifchilddoc|\\
|\providecommand{\version}{draft}|\\
|\||else|\\
|\providecommand{\version}{final}|\\
|\||fi|
\end{tabular}
\end{center}
%
The definition by |\providecommand| makes sure
that previous definitions are not overwritten.
Further statements |\providecommand{\version}{...}|
can thus be added before the above code to override it.

For the main file, one might add a line
(between |\childdocmain| and the above block)
%
\begin{center}
|%\ifchilddoc\||else\providecommand{\version}{draft}\||fi|
\end{center}
%
which can be uncommented to produce a draft version.
Likewise one can add a line to the very top of a child file
(above the |\childdocof{|\textit{main}|}| directive)
%
\begin{center}
|%\providecommand{\version}{final}|
\end{center}
%
which can be uncommented to produce the final version of this child document.

%%%%%%%%%%%%%%%%%%%%%%%%%%%%%%%%%%%%%%%%%%%%%%%%%%%%%%%%%%%%%%%%%%%%%%%%%%%%%%%%
\subsection{Forwarding}
\label{sec:forward}

Different versions of the main or child documents
using compilation flags as described in \secref{sec:flags}
can be (permanently) stored in different files
for convenient compilation, viewing and distribution.
To this end, the package defines a command
to pass on compilation to a different file:

%%%%%%%%%%%%%%%%%%%%%%%%%%%%%%%%%%%%%%%%
\DescribeMacro{\childdocforward}
The command |\childdocforward| redirects processing to
another source file:
%
\begin{center}
\begin{tabular}{l}
|\input{childdoc.def}|\\
|\childdocforward[|\textit{main}|]{|\textit{dest}|}|\\
\end{tabular}
\end{center}
%
The argument \textit{dest} is the destination file
(without extension).
It should be the main file or one of the child files.
Note that further \textsf{childdoc} directives
such as |\childdocof| and |\childdocforward|
in the indicated file will be processed in this form.
The optional argument \textit{main}
passes on directly to the main file \textit{main}
while pretending to compile the child \textit{dest}.
This form behaves as if \textit{dest}
issues |\childdocof{|\textit{main}|}| right away,
and no further \textsf{childdoc} directives will be processed.

%%%%%%%%%%%%%%%%%%%%%%%%%%%%%%%%%%%%%%%%
\DescribeMacro{\...prefix}
In the alternative form |\childdocforwardprefix|,
%
\begin{center}
\begin{tabular}{l}
|\input{childdoc.def}|\\
|\childdocforwardprefix[|\textit{main}|]{|\textit{prefix}|}{|\textit{dest}|}|
\end{tabular}
\end{center}
%
the destination file is determined by a pattern
depending on the current file:
To make this work, the current file must be called
`{\textit{prefix}\hspace{0.2em}\textit{suffix}}'
with \textit{prefix} matching precisely the argument.
Processing is then passed on to the file
`{\textit{dest}\hspace{0.2em}\textit{suffix}}'.
Surely, the same effect is achieved by
directly specifying the
argument `{\textit{dest}\hspace{0.2em}\textit{suffix}}'
in the first form.
However, that requires to set up a different file
for each child. With the alternative form of the command
all these files can have exactly the same content
which simplifies setting them up and maintaining them.

For example, the following file |draft.tex|
with a compilation flag |\version| as described in \secref{sec:flags}
compiles the main document as a draft:
%
\begin{center}
\begin{tabular}{l}
|\def\version{draft}|\\
|\input{childdoc.def}|\\
|\childdocforward{|\textit{main}|}|
\end{tabular}
\end{center}
%
Likewise, the following files |final|\textit{nn}|.tex|
compile the final version of the child document
|child|\textit{nn}|.tex|:
%
\begin{center}
\begin{tabular}{l}
|\def\version{final}|\\
|\input{childdoc.def}|\\
|\childdocforwardprefix{final}{child}|
\end{tabular}
\end{center}
%

Note that when several versions of a main file and/or of each child file
are to be generated, it may be convenient to set up a |Makefile| or
shell script to automatise the process.

%%%%%%%%%%%%%%%%%%%%%%%%%%%%%%%%%%%%%%%%%%%%%%%%%%%%%%%%%%%%%%%%%%%%%%%%%%%%%%%%
\subsection{Command Line Processing}
\label{sec:commandline}

The effect of redirection files can also be achieved by invoking
the \LaTeX{} compiler with a more elaborate command line.
Most conveniently this should be done as part
of a shell script or a |Makefile|.

When using \textsf{childdoc} in the main file, the following
command lines effectively perform a redirection
(note that depending on the shell being used,
backslashes may have to be doubled: `|\|' $\to$ `|\\|'):
%
\begin{center}
|... -jobname "|\textit{target}|" |\\|"|[\textit{flags}]%
|\input{childdoc.def}\childdocforward[|\textit{main}|]{|\textit{dest}|}"|
\end{center}
%
Here \textit{target} is the name of the output file,
\textit{main} is the name of the main file
and \textit{dest} is the name of the main or child file to be processed
(all filenames without extensions).
The optional argument \textit{main} can be omitted
if \textit{main} matches \textit{dest}.
Optionally, compilation \textit{flags} can be defined via |\def| commands.
This command line makes the \TeX{} engine believe
it is compiling the file \textit{target}
whose content is specified as the latter parameter.
The provided code then forwards the processing to
\textit{main} or \textit{dest} as described in \secref{sec:forward}.

%%%%%%%%%%%%%%%%%%%%%%%%%%%%%%%%%%%%%%%%%%%%%%%%%%%%%%%%%%%%%%%%%%%%%%%%%%%%%%%%
\subsection{Include by Input}
\label{sec:input}

Including child documents by |\include| has some restrictions by design.
Most notably, the content of a child document always occupies
its own set of pages; pages cannot be shared between child documents.
Usually, this behaviour makes perfect sense
because each child document contain an essential part of the document.
However, in some situations it may be desirable to compose
a document from a collection of parts
without having mandatory page breaks between then.
For this case, the package
provides a mechanism to include parts
by |\input| which can also be processed individually.
However, by construction this mechanism
requires manual handling of the content to be output.

%%%%%%%%%%%%%%%%%%%%%%%%%%%%%%%%%%%%%%%%
\DescribeMacro{\ifchilddocmanual}
The main file should be prepared as usual, see \secref{sec:include}.
However, the document body must make a distinction
between processing of an individual part and of the main document, e.g.:
%
\begin{center}
\begin{tabular}{l}
|\ifchilddocmanual|\\
|\input{\childdocname}|\\
|\||else|\\
\textit{document body with }|\input{|\textit{part}|}|\\
|\||fi|
\end{tabular}
\end{center}
%
The conditional |\ifchilddocmanual| is true whenever
a part to be included by |\input| is being compiled,
and the name of the part is stored in |\childdocname|.

%%%%%%%%%%%%%%%%%%%%%%%%%%%%%%%%%%%%%%%%
\DescribeMacro{\childdocby}
Each part to be included by |\input| should start with:
%
\begin{center}
\begin{tabular}{l}
|\input{childdoc.def}|\\
|\childdocby{|\textit{main}|}|\\
\end{tabular}
\end{center}
%
The directive |\childdocby| is similar to |\childdocof|
described in \secref{sec:include},
but the subsequent selection of content must be done manually.
To that end, both |\ifchilddoc| and |\ifchilddocmanual|
will be true upon processing of a part,
and the name of the part is stored in |\childdocname|.
Note that |\jobname| will be set to the filename of the current part
so that each part receives an individual |.aux| file
that does not interfere with the |.aux| file(s) of the main document.
This behaviour can be altered by the alternative form
|\childdocby[*]{|\textit{main}|}| (with a non-empty optional argument)
which uses the |.aux| file of the main document
by setting |\jobname| to \textit{main}.

%%%%%%%%%%%%%%%%%%%%%%%%%%%%%%%%%%%%%%%%%%%%%%%%%%%%%%%%%%%%%%%%%%%%%%%%%%%%%%%%
\subsection{Driver Development}
\label{sec:driver}

The \textsf{childdoc} mechanism can also be use for the development
of definition files such as \LaTeX{} styles or classes.
This case differs from the above setup with multiple parts
included by |\include| in that no |\includeonly| should be invoked.
This can be achieved by starting the include file
(before |\ProvidesPackage|) with:
%
\begin{center}
\begin{tabular}{l}
|\input{childdoc.def}|\\
|\childdocforward{|\textit{main}|}|\\
\end{tabular}
\end{center}
%
or alternatively with:
%
\begin{center}
\begin{tabular}{l}
|\input{childdoc.def}|\\
|\childdocby{|\textit{main}|}|\\
\end{tabular}
\end{center}
%
Both forms have slightly different effects as described above.
The main file is prepared as usual, see \secref{sec:include}.

%%%%%%%%%%%%%%%%%%%%%%%%%%%%%%%%%%%%%%%%%%%%%%%%%%%%%%%%%%%%%%%%%%%%%%%%%%%%%%%%
\subsection{Legacy Detection}
\label{sec:detection}

The directive |\childdocmain| in the main file can detect
whether the complete document or merely a child is to be compiled
even without using the directive |\childdocof|.
This method is deprecated because it is less robust
and there is no compelling reason to use it;
it is merely provided for backward compatibility
and it may be removed in future versions.

If the detection mechanism is to be used,
it is mandatory to correctly specify
the filename of the main file as the argument of |\childdocmain|:
%
\begin{center}
\begin{tabular}{l}
|\input{childdoc.def}|\\
|\childdocmain{|\textit{main}|}|\\
\end{tabular}
\end{center}
%
If |\jobname| does not match the argument \textit{main} of |\childdocmain|,
it is assumed that |\jobname| points to the child file to be compiled.
When using |\childdocmain| with the main file specified as argument,
it suffices to start a child file
with just |\input{|\textit{main}|}|
without loading of the package and using |\childdocof|.
If instead all processing is done
with the appropriate \textsf{childdoc} directives,
the argument of \textit{main} of |\childdocmain| can be empty.

An alternative version of the command line processing described
in \secref{sec:commandline} using the detection mechanism reads:
%
\begin{center}
|... -jobname "|\textit{target}|" "|[\textit{flags}]%
[|\def\jobname{|\textit{dest}|}|]|\input{|\textit{main}|}"|
\end{center}

%%%%%%%%%%%%%%%%%%%%%%%%%%%%%%%%%%%%%%%%%%%%%%%%%%%%%%%%%%%%%%%%%%%%%%%%%%%%%%%%
\subsection{Manual Code}
\label{sec:manual}

In case one cannot be certain whether the definitions file |childdoc.def|
is installed on the target \TeX{} distribution
and one prefers not to ship it,
it is conceivable to paste a few relevant commands into the sources.

To that end, drop all statements |\input{childdoc.def}|
and perform the replacements as outlined below.
Instead of |\childdocmain{|\textit{main}|}| add the following code
to the top of the main file:
%
\begin{center}
\begin{tabular}{l}
|\||ifdefined\childdocname\endinput\||fi\newif\ifchilddoc|\\
|\edef\childdocname{\scantokens\expandafter{\jobname\noexpand}}|\\
|\def\childdocmain{|\textit{main}|}\||ifx\childdocmain\childdocname\||else|\\
|\childdoctrue\includeonly{\childdocname}\let\jobname\childdocmain\||fi|\\
\end{tabular}
\end{center}
%
Instead of |\childdocof{|\textit{main}|}| just include the main file
at the top of each child file:
%
\begin{center}
|\input{|\textit{main}|}|
\end{center}
%
A simple redirection |\childdocforward{|\textit{dest}|}| is achieved by:
%
\begin{center}
|\def\jobname{|\textit{dest}|}\input{\jobname}|
\end{center}
%
The redirection with prefix
|\childdocforwardprefix[|\textit{prefix}|]{|\textit{dest}|}|
is accomplished by:
%
\begin{center}
\begin{tabular}{l}
|{\edef\jobname{\scantokens\expandafter{\jobname\noexpand}}|\\
|\def\redirectjob |\textit{prefix}|#1~~~{\gdef\jobname{|\textit{dest}|#1}}|\\
|\expandafter\redirectjob\jobname~~~}\input{\jobname}|
\end{tabular}
\end{center}

In an alternative approach,
child documents can be compiled by a specific command line
without additional code or specific definitions:
%
\begin{center}
|... -jobname "|\textit{target}|" "|[\textit{flags}]%
|\includeonly{|\textit{dest}|}\input{|\textit{main}|}"|
\end{center}
%

%%%%%%%%%%%%%%%%%%%%%%%%%%%%%%%%%%%%%%%%%%%%%%%%%%%%%%%%%%%%%%%%%%%%%%%%%%%%%%%%
%%%%%%%%%%%%%%%%%%%%%%%%%%%%%%%%%%%%%%%%%%%%%%%%%%%%%%%%%%%%%%%%%%%%%%%%%%%%%%%%
\section{Information}

%%%%%%%%%%%%%%%%%%%%%%%%%%%%%%%%%%%%%%%%%%%%%%%%%%%%%%%%%%%%%%%%%%%%%%%%%%%%%%%%
\subsection{Copyright}

Copyright \copyright{} 2017--2018 Niklas Beisert

This work may be distributed and/or modified under the
conditions of the \LaTeX{} Project Public License, either version 1.3
of this license or (at your option) any later version.
The latest version of this license is in
  \url{http://www.latex-project.org/lppl.txt}
and version 1.3 or later is part of all distributions of \LaTeX{}
version 2005/12/01 or later.

This work has the LPPL maintenance status `maintained'.

The Current Maintainer of this work is Niklas Beisert.

This work consists of the files |README.txt|, |childdoc.ins| and |childdoc.dtx|
as well as the derived files |childdoc.def|, |cdocsamp.tex|
with |cdocsch1.tex|, |cdocsch2.tex|, |cdocspt3.tex|, |cdocspt4.tex|,
|cdocsdrf.tex|, |cdocsfn1.tex|, |cdocsfn2.tex|
as well as |childdoc.pdf|.

%%%%%%%%%%%%%%%%%%%%%%%%%%%%%%%%%%%%%%%%%%%%%%%%%%%%%%%%%%%%%%%%%%%%%%%%%%%%%%%%
\subsection{Files and Installation}

The package consists of the files:
%
\begin{center}
\begin{tabular}{ll}
    |README.txt|   & readme file \\
    |childdoc.ins| & installation file \\
    |childdoc.dtx| & source file \\
    |childdoc.def| & definition file \\
    |cdocsamp.tex| & sample main file \\
    |cdocsch1.tex| & sample include file \\
    |cdocsch2.tex| & sample include file \\
    |cdocspt3.tex| & sample part file \\
    |cdocspt4.tex| & sample part file \\
    |cdocsdrf.tex| & sample redirection file \\
    |cdocsfn1.tex| & sample redirection file \\
    |cdocsfn2.tex| & sample redirection file \\
    |childdoc.pdf| & manual
\end{tabular}
\end{center}
%
The distribution consists of the files
|README.txt|, |childdoc.ins| and |childdoc.dtx|.
%
\begin{itemize}
\item
Run (pdf)\LaTeX{} on |childdoc.dtx|
to compile the manual |childdoc.pdf| (this file).
\item
Run \LaTeX{} on |childdoc.ins| to create the definitions file |childdoc.def|
and the sample |cdocsamp.tex| with include files
|cdocsch1.tex|, |cdocsch2.tex|, |cdocspt3.tex|, |cdocspt4.tex|,
|cdocsdrf.tex|, |cdocsfn1.tex|, |cdocsfn2.tex|.
Then copy the file |childdoc.def| to an appropriate directory of your \LaTeX{}
distribution, e.g.\ \textit{texmf-root}|/tex/latex/childdoc|.
\end{itemize}

%%%%%%%%%%%%%%%%%%%%%%%%%%%%%%%%%%%%%%%%%%%%%%%%%%%%%%%%%%%%%%%%%%%%%%%%%%%%%%%%
\subsection{Related CTAN Packages}

There are several other packages which offer a similar functionality:
%
\begin{itemize}
\item
The packages
\href{http://ctan.org/pkg/docmute}{\textsf{docmute}},
\href{http://ctan.org/pkg/includex}{\textsf{includex}} and
\href{http://ctan.org/pkg/standalone}{\textsf{standalone}}
provide commands to include only the document body of
a child file thus allowing both files to be compiled individually.
\item
The packages \href{http://ctan.org/pkg/subdocs}{\textsf{subdocs}}
and \href{http://ctan.org/pkg/subfiles}{\textsf{subfiles}}
provide structures in which the main and child documents can be
encapsulated and allowing them to be compiled individually.
The inclusion mechanism is different from the conventional |\include|.
\item
The package \href{http://ctan.org/pkg/combine}{\textsf{combine}}
is an elaborate solution to combine several documents into one.
\end{itemize}
%
See also the CTAN topic \href{http://ctan.org/topic/subdocs}{\textsf{subdocs}}
for further related packages.
The present package differs from the above solutions in that
a document structure constructed with the conventional |\include| mechanism
just needs two extra commands at the top of every file
such that all constituent files can be compiled individually.

%%%%%%%%%%%%%%%%%%%%%%%%%%%%%%%%%%%%%%%%%%%%%%%%%%%%%%%%%%%%%%%%%%%%%%%%%%%%%%%%
%\subsection{Feature Suggestions}
%
%The following is a list of features which may be useful for future
%versions of this package:
%%
%\begin{itemize}
%\item
%\ldots
%\end{itemize}

%%%%%%%%%%%%%%%%%%%%%%%%%%%%%%%%%%%%%%%%%%%%%%%%%%%%%%%%%%%%%%%%%%%%%%%%%%%%%%%%
\subsection{Revision History}

%%%%%%%%%%%%%%%%%%%%%%%%%%%%%%%%%%%%%%%%
\paragraph{v2.0:} 2018/12/30

\begin{itemize}
\item
immediate forward processing
\item
added |\childdocby| mechanism
\item
manual restructured
\end{itemize}

%%%%%%%%%%%%%%%%%%%%%%%%%%%%%%%%%%%%%%%%
\paragraph{v1.6:} 2018/01/17

\begin{itemize}
\item
application for development of include files
\item
corrections to manual
\end{itemize}

%%%%%%%%%%%%%%%%%%%%%%%%%%%%%%%%%%%%%%%%
\paragraph{v1.5:} 2017/05/21

\begin{itemize}
\item
more complete structuring introduced
\item
|\childdocof| introduced
\item
|\childdoc| renamed to |\childdocmain|
\item
|\childredirect| renamed to |\childdocforward| and |\childdocforwardprefix|
and functionality expanded
\end{itemize}

%%%%%%%%%%%%%%%%%%%%%%%%%%%%%%%%%%%%%%%%
\paragraph{v1.0:} 2017/04/27

\begin{itemize}
\item
manual and install package
\item
first version published on CTAN
\end{itemize}

%%%%%%%%%%%%%%%%%%%%%%%%%%%%%%%%%%%%%%%%
\paragraph{v0.6:} 2017/04/26

\begin{itemize}
\item
redirection mechanism added
\end{itemize}

%%%%%%%%%%%%%%%%%%%%%%%%%%%%%%%%%%%%%%%%
\paragraph{v0.5:} 2017/04/26

\begin{itemize}
\item
functionality in definition file
\end{itemize}


%%%%%%%%%%%%%%%%%%%%%%%%%%%%%%%%%%%%%%%%%%%%%%%%%%%%%%%%%%%%%%%%%%%%%%%%%%%%%%%%
%%%%%%%%%%%%%%%%%%%%%%%%%%%%%%%%%%%%%%%%%%%%%%%%%%%%%%%%%%%%%%%%%%%%%%%%%%%%%%%%
%%%%%%%%%%%%%%%%%%%%%%%%%%%%%%%%%%%%%%%%%%%%%%%%%%%%%%%%%%%%%%%%%%%%%%%%%%%%%%%%
\appendix

\settowidth\MacroIndent{\rmfamily\scriptsize 000\ }

 \DocInput{childdoc.dtx}

\end{document}
%</driver>
% \fi
%
% %%%%%%%%%%%%%%%%%%%%%%%%%%%%%%%%%%%%%%%%%%%%%%%%%%%%%%%%%%%%%%%%%%%%%%%%%%%%%%
% %%%%%%%%%%%%%%%%%%%%%%%%%%%%%%%%%%%%%%%%%%%%%%%%%%%%%%%%%%%%%%%%%%%%%%%%%%%%%%
% \section{Sample}
%\iffalse
%<*samplemain>
%\fi
%
% The following presents a sample document
% with two chapters, two parts, a title page,
% a compile flag as well as three forwarding files to set the flag.
% It consists of eight |.tex| files:
% \begin{center}
% \begin{tabular}{ll}
% |cdocsamp.tex|&main file\\
% |cdocsch1.tex|&include file for chapter 1\\
% |cdocsch2.tex|&include file for chapter 2\\
% |cdocspt3.tex|&include file for part 3\\
% |cdocspt4.tex|&include file for part 4\\
% |cdocsdrf.tex|&forwarding file for main file in draft mode\\
% |cdocsfi1.tex|&forwarding file for final version of chapter 1\\
% |cdocsfi2.tex|&forwarding file for final version of chapter 2\\
% \end{tabular}
% \end{center}
% Each of the eight files can be compiled directly by the \LaTeX{} compiler.
%
% %%%%%%%%%%%%%%%%%%%%%%%%%%%%%%%%%%%%%%
% \paragraph{Main File.}
%
% The main file is called |cdocsamp.tex|.
%
% Load the \textsf{childdoc} definitions and
% declare the filename for the main document:
%    \begin{macrocode}
\input{childdoc.def}
\childdocmain{}
%    \end{macrocode}

% Optional override for |\version| flag:
%    \begin{macrocode}
%%\ifchilddoc\else\providecommand{\version}{draft}\fi
%    \end{macrocode}

% Define the default values for the |\version| flag
% (|final| for the main file and |draft| for childs):
%    \begin{macrocode}
\ifchilddoc
\providecommand{\version}{draft}
\else
\providecommand{\version}{final}
\fi
%    \end{macrocode}

% Load the standard document class:
%    \begin{macrocode}
\documentclass[12pt]{article}
%    \end{macrocode}

% Start the document body:
%    \begin{macrocode}
\begin{document}
%    \end{macrocode}

% Declare a title page.
% Print title, part of document being processed and version flag:
%    \begin{macrocode}
\addtocounter{page}{-1}
\begin{center}
{\LARGE\bfseries{}childdoc example\par}
\vspace{1cm}
\ifchilddoc
\ifchilddocmanual part\else chapter\fi:
`\childdocname' of `\childdocjob'\par
\else
main document: `\childdocjob'\par
\fi
version: \version\par
\end{center}
\newpage
%    \end{macrocode}

% Manually include selected file,
% otherwise process as usual:
%    \begin{macrocode}
\ifchilddocmanual
\section*{part `\childdocname'}
\input{\childdocname}
\else
%    \end{macrocode}

% Include the two chapters:
%    \begin{macrocode}
\include{cdocsch1}
\include{cdocsch2}
%    \end{macrocode}

% Include the two parts unless only chapters should be displayed:
%    \begin{macrocode}
\ifchilddoc\else
\section{part three}
\input{cdocspt3}
\section{part four}
\input{cdocspt4}
\fi
%    \end{macrocode}

% Process as usual until here:
%    \begin{macrocode}
\fi
%    \end{macrocode}

% End of document body:
%    \begin{macrocode}
\end{document}
%    \end{macrocode}
%\iffalse
%</samplemain>
%\fi
%
% %%%%%%%%%%%%%%%%%%%%%%%%%%%%%%%%%%%%%%
% \paragraph{Chapter Include Files.}
%
% The include files are called |cdocsch1.tex| and |cdocsch2.tex|.
%
%\iffalse
%<*samplechap1|samplechap2>
%\fi

% Optional override for |\version| flag:
%    \begin{macrocode}
%%\providecommand{\version}{final}
%    \end{macrocode}

% Include the main document:
%    \begin{macrocode}
\input{childdoc.def}
\childdocof{cdocsamp}
%    \end{macrocode}

%\iffalse
%</samplechap1|samplechap2>
%\fi
%
%\iffalse
%<*samplechap1>
%\fi
% Some text for chapter 1:
%    \begin{macrocode}
\section{one}
some text in chapter one
%    \end{macrocode}

%\iffalse
%</samplechap1>
%\fi
% Some text for chapter 2:
%\iffalse
%<*samplechap2>
%\fi
%    \begin{macrocode}
\section{two}
more text in chapter two
%    \end{macrocode}

%\iffalse
%</samplechap2>
%\fi
%
% %%%%%%%%%%%%%%%%%%%%%%%%%%%%%%%%%%%%%%
% \paragraph{Part Include Files.}
%
% The include files are called |cdocspt3.tex| and |cdocspt4.tex|.
%
%\iffalse
%<*samplepart3|samplepart4>
%\fi

% Optional override for |\version| flag:
%    \begin{macrocode}
%%\providecommand{\version}{final}
%    \end{macrocode}

% Include the main document:
%    \begin{macrocode}
\input{childdoc.def}
\childdocby{cdocsamp}
%    \end{macrocode}

%\iffalse
%</samplepart3|samplepart4>
%\fi
%
%\iffalse
%<*samplepart3>
%\fi
% Some text for part 3:
%    \begin{macrocode}
some text in part three
%    \end{macrocode}

%\iffalse
%</samplepart3>
%\fi
% Some text for part 4:
%\iffalse
%<*samplepart4>
%\fi
%    \begin{macrocode}
more text in part four
%    \end{macrocode}

%\iffalse
%</samplepart4>
%\fi
%
% %%%%%%%%%%%%%%%%%%%%%%%%%%%%%%%%%%%%%%
% \paragraph{Forwarding for a Complete Draft.}
%
% The following forwarding file |cdocsdrf.tex|
% compiles the main document in draft mode:
%\iffalse
%<*sampledraft>
%\fi
%    \begin{macrocode}
\def\version{draft}
\input{childdoc.def}
\childdocforward{cdocsamp}
%    \end{macrocode}

%\iffalse
%</sampledraft>
%\fi
%
% %%%%%%%%%%%%%%%%%%%%%%%%%%%%%%%%%%%%%%
% \paragraph{Forwarding for Final Version of the Chapters.}
%
% The following forwarding files |cdocsfn1.tex| and |cdocsfn2.tex|
% (with identical content)
% compile the final versions of the child documents
% |cdocsch1.tex| and |cdocsch2.tex|, respectively:
%\iffalse
%<*samplefinal>
%\fi
%    \begin{macrocode}
\def\version{final}
\input{childdoc.def}
\childdocforwardprefix[cdocsamp]{cdocsfn}{cdocsch}
%    \end{macrocode}

%\iffalse
%</samplefinal>
%\fi
%
% %%%%%%%%%%%%%%%%%%%%%%%%%%%%%%%%%%%%%%
% \paragraph{Command Line Processing.}
%
% The following three command lines generate the output files
% |cdocscld|, |cdocscl1| and |cdocscl2|
% which should be identical to
% |cdocsdrf|, |cdocsch1| and |cdocsfn2|, respectively:
% \begin{center}
% \begin{tabular}{l}
% |latex -jobname cdocscld \|\\
% |  "\def\version{draft}\input{childdoc.def}\childdocforward{cdocsamp}"|\\
% |latex -jobname cdocscl1 \|\\
% |  "\input{childdoc.def}\childdocforward[cdocsamp]{cdocsch1}"|\\
% |latex -jobname cdocscl2 \|\\
% |  "\def\version{final}\input{childdoc.def}\childdocforward{cdocsch2}"|
% \end{tabular}
% \end{center}
% Note that the trailing backslash on each first line
% merely continues the input to the second line
% (for convenient cut ant paste).
% Furthermore, the command |latex| can be replaced by any
% of its alternative versions such as |pdflatex|.
%
% %%%%%%%%%%%%%%%%%%%%%%%%%%%%%%%%%%%%%%%%%%%%%%%%%%%%%%%%%%%%%%%%%%%%%%%%%%%%%%
% %%%%%%%%%%%%%%%%%%%%%%%%%%%%%%%%%%%%%%%%%%%%%%%%%%%%%%%%%%%%%%%%%%%%%%%%%%%%%%
% \section{Implementation}
%\iffalse
%<*package>
%\fi
%
% This section describes the definitions file |childdoc.def|.

% The definitions cannot be loaded using |\usepackage| or |\RequirePackage|
% which has a mechanism to prevent loading a style file more than once.
% When loading the definitions by means of |\input|
% multiple instances have to be prevented manually:
%\iffalse
%This code needs to be before the `\ProvidesFile' directive
%which is defined at the beginning of this file.
%Therefore it is also placed there and commented out here.
%</package>
%<*discard>
%\fi
%    \begin{macrocode}
\ifdefined\childdocmain\endinput\fi
%    \end{macrocode}
%\iffalse
%</discard>
%<*package>
%\fi
%
% \macro{\ifchilddoc}
% \macro{\ifchilddocmanual}
% The conditional |\ifchilddoc| tells whether a
% child (true) or main (false) document is being compiled.
% The conditional |\ifchilddocmanual| tells whether
% the |\includeonly| mechanism is used (false) or
% the selection of child files must be performed manually (true).
% The definitions initialise to false:
%    \begin{macrocode}
\newif\ifchilddoc
\newif\ifchilddocmanual
%    \end{macrocode}

% \macro{\childdocname}
% \macro{\childdocjob}
% The macro |\childdocname| stores the name of the main document
% to be compiled. The macro |\childdocjob| stores the name of
% the document on which the \LaTeX{} compiler was originally invoked.
% The content of |\jobname| cannot be compared
% to filenames specified in the source due to different catcodes.
% The following code rescans |\jobname|, stores the result
% in |\childdocname| and saves a copy in |\childdocjob|:
%    \begin{macrocode}
\edef\childdocname{\scantokens\expandafter{\jobname\noexpand}}
\let\childdocjob\childdocname
%    \end{macrocode}

% \macro{\childdocdisable}
% The macro |\childdocdisable| prevents the main file
% from being processed more than once.
% At this stage, the main document command |\childdocmain|
% is assumed to be called once again where it should do nothing.
% Any subsequent call to it should prevent
% a secondary processing of the main document
% It overwrites the forwarding commands
% |\childdocof| and |\childdocforward|
% with empty macros to prevent further inclusions of the main document:
%    \begin{macrocode}
\newcommand{\childdocdisable}
{
  \renewcommand{\childdocmain}[1]{\renewcommand{\childdocmain}[1]{\endinput}}
  \renewcommand{\childdocof}[1]{}
  \renewcommand{\childdocby}[2][]{}
  \renewcommand{\childdocforward}[2][]{}
  \renewcommand{\childdocdisable}{}
}
%    \end{macrocode}

% \macro{\childdocmain}
% The macro |\childdocmain| is to be called at the top of the main file
% with nothing or the main filename (without extension) as argument.
% First, it breaks loops.
% If the argument is not empty and does not match |\childdocname|
% (which is set by the first inclusion of |childdoc.def|),
% |\ifchilddoc| is set to true, |\includeonly| is applied to the child file
% and |\jobname| is set to the main file
% (for proper handling of |.aux| files):
%    \begin{macrocode}
\newcommand{\childdocmain}[1]
{
  \childdocdisable\childdocmain{}
  \if?#1?\else
    \begingroup
      \def\childdoctmp{#1}
      \ifx\childdoctmp\childdocname
        \def\childdoctmp{}
      \else
        \def\childdoctmp
        {
          \childdoctrue
          \includeonly{\childdocname}
          \def\childdocjob{#1}
          \def\jobname{#1}
        }
      \fi
      \expandafter
    \endgroup
    \childdoctmp
  \fi
}
%    \end{macrocode}

% \macro{\childdocof}
% The command |\childdocof| redirects
% compilation to the main file |#1|.
%    \begin{macrocode}
\newcommand{\childdocof}[1]
{
  \childdocdisable
  \childdoctrue
  \includeonly{\childdocname}
  \def\jobname{#1}
  \def\childdocjob{#1}
  \input{#1}
}
%    \end{macrocode}

% \macro{\childdocby}
% The command |\childdocby| ....
%    \begin{macrocode}
\newcommand{\childdocby}[2][]
{
  \childdocdisable
  \childdoctrue
  \childdocmanualtrue
  \if?#1?\else
    \def\jobname{#2}
  \fi
  \def\childdocjob{#2}
  \input{#2}
  \endinput
}
%    \end{macrocode}

% \macro{\childdocforward}
% The command |\childdocforward| redirects
% compilation to the main file or
% (if the optional argument is given) a child file.
% Parameters are set as if the main file
% or a child file starting with |\childdocof| was compiled.
% Then compilation is handed over to the main file:
%    \begin{macrocode}
\newcommand{\childdocforward}[2][]
{
  \begingroup
    \if?#1?
      \def\childdoctmp
      {
        \def\childdocname{#2}
        \def\childdocjob{#2}
        \def\jobname{#2}
        \input{#2}
        \endinput
      }
    \else
      \def\childdoctmp
      {
        \childdocdisable
        \def\childdocname{#2}
        \childdoctrue
        \includeonly{#2}
        \def\childdocjob{#1}
        \def\jobname{#1}
        \input{#1}
        \endinput
      }
    \fi
    \expandafter
  \endgroup
  \childdoctmp
}
%    \end{macrocode}

% \macro{\childdocforwardprefix}
% The command |\childdocforwardprefix| redirects
% compilation to the main or a child file by means of a pattern.
% The prefix |#1| in the current filename is replaced by |#2|
% and the suffix of the current filename is kept
% (it is assumed that the filename does not contain the substring `|~~~|'
% which is used as a delimiter).
% Compilation is handed over to the new file by |\childdocforward|:
%    \begin{macrocode}
\newcommand{\childdocforwardprefix}[3][]
{
  \begingroup
    \def\childdocextract #2##1~~~{\def\childdoctmp{\childdocforward[#1]{#3##1}}}
    \expandafter\childdocextract\childdocname~~~
    \expandafter
  \endgroup
  \childdoctmp
}
%    \end{macrocode}

% \macro{\childdoc}
% The deprecated macro |\childdoc| is a legacy version of |\childdocmain|:
%    \begin{macrocode}
\newcommand{\childdoc}{\childdocmain}
%    \end{macrocode}

% \macro{\childdocredirect}
% The deprecated macro |\childdocredirect| is a legacy version
% of |\childdocforward| and |\childdocforwardprefix|:
%    \begin{macrocode}
\newcommand{\childdocredirect}[2][]
{
  \begingroup
    \if?#1?
      \def\childdoctmp{\childdocforward{#2}}
    \else
      \def\childdoctmp{\childdocforwardprefix{#1}{#2}}
    \fi
    \expandafter
  \endgroup
  \childdoctmp
}
%    \end{macrocode}

%\iffalse
%</package>
%\fi
%
\endinput

\childdocof{cdocsamp}
%    \end{macrocode}

%\iffalse
%</samplechap1|samplechap2>
%\fi
%
%\iffalse
%<*samplechap1>
%\fi
% Some text for chapter 1:
%    \begin{macrocode}
\section{one}
some text in chapter one
%    \end{macrocode}

%\iffalse
%</samplechap1>
%\fi
% Some text for chapter 2:
%\iffalse
%<*samplechap2>
%\fi
%    \begin{macrocode}
\section{two}
more text in chapter two
%    \end{macrocode}

%\iffalse
%</samplechap2>
%\fi
%
% %%%%%%%%%%%%%%%%%%%%%%%%%%%%%%%%%%%%%%
% \paragraph{Part Include Files.}
%
% The include files are called |cdocspt3.tex| and |cdocspt4.tex|.
%
%\iffalse
%<*samplepart3|samplepart4>
%\fi

% Optional override for |\version| flag:
%    \begin{macrocode}
%%\providecommand{\version}{final}
%    \end{macrocode}

% Include the main document:
%    \begin{macrocode}
% \iffalse
%
% childdoc.dtx Copyright (C) 2017-2018 Niklas Beisert
%
% This work may be distributed and/or modified under the
% conditions of the LaTeX Project Public License, either version 1.3
% of this license or (at your option) any later version.
% The latest version of this license is in
%   http://www.latex-project.org/lppl.txt
% and version 1.3 or later is part of all distributions of LaTeX
% version 2005/12/01 or later.
%
% This work has the LPPL maintenance status `maintained'.
%
% The Current Maintainer of this work is Niklas Beisert.
%
% This work consists of the files childdoc.dtx and childdoc.ins
% and the derived files childdoc.def and cdocsamp.tex with
% cdocsch1.tex, cdocsch2.tex, cdocsdrf.tex, cdocsfn1.tex, cdocsfn2.tex.
%
%<package>\ifdefined\childdocmain\endinput\fi
%<package>\ProvidesFile{childdoc.def}[2018/12/30 v2.0 child document driver]
%<samplemain>\ProvidesFile{cdocsamp.tex}[2018/12/30 v2.0 sample for childdoc]
%<*driver>
%\ProvidesFile{childdoc.drv}[2018/12/30 v2.0 childdoc reference manual file]
\PassOptionsToClass{10pt,a4paper}{article}
\documentclass{ltxdoc}

\usepackage[margin=35mm]{geometry}
\usepackage{hyperref}
\usepackage{hyperxmp}
\usepackage[usenames]{color}

\hypersetup{colorlinks=true}
\hypersetup{pdfstartview=FitH}
\hypersetup{pdfpagemode=UseNone}
\hypersetup{pdfsource={}}
\hypersetup{pdflang={en-UK}}
\hypersetup{pdfcopyright={Copyright 2017-2018 Niklas Beisert.
  This work may be distributed and/or modified under the
  conditions of the LaTeX Project Public License, either version 1.3
  of this license or (at your option) any later version.}}
\hypersetup{pdflicenseurl={http://www.latex-project.org/lppl.txt}}
\hypersetup{pdfcontactaddress={ETH Zurich, ITP, HIT K,
  Wolfgang-Pauli-Strasse 27}}
\hypersetup{pdfcontactpostcode={8093}}
\hypersetup{pdfcontactcity={Zurich}}
\hypersetup{pdfcontactcountry={Switzerland}}
\hypersetup{pdfcontactemail={nbeisert@itp.phys.ethz.ch}}
\hypersetup{pdfcontacturl={http://people.phys.ethz.ch/\xmptilde nbeisert/}}

\newcommand{\secref}[1]{\hyperref[#1]{section \ref*{#1}}}

\parskip1ex
\parindent0pt
\let\olditemize\itemize
\def\itemize{\olditemize\parskip0pt}

\begin{document}

\title{The \textsf{childdoc} Package}
\hypersetup{pdftitle={The childdoc Package}}
\author{Niklas Beisert\\[2ex]
  Institut f\"ur Theoretische Physik\\
  Eidgen\"ossische Technische Hochschule Z\"urich\\
  Wolfgang-Pauli-Strasse 27, 8093 Z\"urich, Switzerland\\[1ex]
  \href{mailto:nbeisert@itp.phys.ethz.ch}
  {\texttt{nbeisert@itp.phys.ethz.ch}}}
\hypersetup{pdfauthor={Niklas Beisert}}
\hypersetup{pdfsubject={Manual for the LaTeX2e Package childdoc}}
\date{30 December 2018, \textsf{v2.0}}
\maketitle

\begin{abstract}\noindent
\textsf{childdoc} is a \LaTeXe{} package
that enables the direct compilation
of document sections included by |\include|
to individual files.
\end{abstract}

\begingroup
\parskip0ex
\tableofcontents
\endgroup

%%%%%%%%%%%%%%%%%%%%%%%%%%%%%%%%%%%%%%%%%%%%%%%%%%%%%%%%%%%%%%%%%%%%%%%%%%%%%%%%
%%%%%%%%%%%%%%%%%%%%%%%%%%%%%%%%%%%%%%%%%%%%%%%%%%%%%%%%%%%%%%%%%%%%%%%%%%%%%%%%
\section{Introduction}

\LaTeX{} provides a mechanism to structure a large document (such as a book)
into a main file and several child files (containing the chapters)
using the |\include| command.
This mechanism is beneficial for documents
which span hundreds of pages in order to
make the source file(s) more manageable.
Moreover, compilation can be restricted to
selected child files by means of the |\includeonly| command.
The latter feature can be used to reduce the compilation time while editing
(this was significantly more useful in the earlier days of \LaTeX{})
or to generate a smaller document which is easier to navigate.
Another application of |\includeonly| is to generate
documents consisting of selected parts of the complete document.

However, there are a few drawbacks of the plain |\include| mechanism:
\begin{itemize}
\item
The child files cannot be compiled on their own,
they can only be compiled via the main file.
A naive editing environment
(such as a text editor with an option
to have the current file processed by \LaTeX)
may require one to switch to the main file before compiling;
attempting to compile the child file produces errors.
\item
The main file must be modified (each time)
to adjust the |\includeonly| command
to the present needs. This easily leaves the main file in a messy state.
\item
The generated document will always carry the filename
of the main document. This is inconvenient if
several child files are to be compiled and
to be kept for distribution.
\end{itemize}

The present package provides a simple interface
to make child files individually compilable by \LaTeX{}.
Compiling a child file then has the same effect as compiling
the main file with an |\includeonly| command
to select the appropriate child.
Moreover the generated document will carry the name of the child
rather than the main file.
This resolves all three above issues.

This feature is meant to make the editing of books,
thesis documents and lecture notes somewhat more convenient.
However, the package can also be used efficiently for
composing a series of documents (such as exercise sheets)
which are typically distributed individually.
It then assists the author in generating the individual documents
(potentially in different versions)
as well as a document containing the collected series.
Another application is in developing style files
or other kinds of included material
where compilation of the style file could redirect
to a sample or test file.

%%%%%%%%%%%%%%%%%%%%%%%%%%%%%%%%%%%%%%%%%%%%%%%%%%%%%%%%%%%%%%%%%%%%%%%%%%%%%%%%
%%%%%%%%%%%%%%%%%%%%%%%%%%%%%%%%%%%%%%%%%%%%%%%%%%%%%%%%%%%%%%%%%%%%%%%%%%%%%%%%
\section{Usage}

First of all, the package \textsf{childdoc} is \emph{not} a standard
\LaTeXe{} |.sty| style file! Therefore it needs to be invoked in
a non-standard way.

%%%%%%%%%%%%%%%%%%%%%%%%%%%%%%%%%%%%%%%%%%%%%%%%%%%%%%%%%%%%%%%%%%%%%%%%%%%%%%%%
\subsection{Included Files}
\label{sec:include}

%%%%%%%%%%%%%%%%%%%%%%%%%%%%%%%%%%%%%%%%
\DescribeMacro{\childdocmain}
To use the package, add the commands
\begin{center}
\begin{tabular}{l}
|\input{childdoc.def}|\\
|\childdocmain{}|\\
\end{tabular}
\end{center}
at the very top of the main \LaTeX{} file,
in particular \emph{before} the |\documentclass| statement!
The argument of |\childdocmain| should be left empty
(but it must be present).

%%%%%%%%%%%%%%%%%%%%%%%%%%%%%%%%%%%%%%%%
\DescribeMacro{\childdocof}
Furthermore, add the commands
\begin{center}
\begin{tabular}{l}
|\input{childdoc.def}|\\
|\childdocof{|\textit{main}|}|\\
\end{tabular}
\end{center}
at the top of every child file \textit{child}
which is included by |\include{|\textit{child}|}|
from within the main file
(or at least for those files to be compiled individually).
The argument \textit{main} must be the filename of the main file.

There are a couple of
considerations in setting up the main and child documents:

%%%%%%%%%%%%%%%%%%%%%%%%%%%%%%%%%%%%%%%%
\paragraph{Restrictions.}

Please note the following restrictions:
\begin{itemize}
\item
|\childdocmain| must be called with one argument \textit{main}
to ensure compatibility with earlier version of the package.
It must either be empty (|\childdocmain{}|)
or precisely match the filename of the main file in which it is specified.
See \secref{sec:detection} for further information.
\item
The filename \textit{main} must be specified without the |.tex| extension.
\item
The filename \textit{main} is case sensitive
(even in case-insensitive file systems)
due to internal string comparison.
\item
The argument \textit{main} should be fully expanded, it cannot be a macro.
\item
Subdirectories and special characters should be avoided in filenames.
\item
The command |\childdocmain{|\textit{main}|}| must be followed by a whitespace.
It should not be followed immediately by another command
or by a comment mark `|%|'.
This is because the \TeX{} parser reads the token immediately following
the argument of |\childdocmain| and puts it
at the beginning of every child section;
however, a white\-space is ignored.
\end{itemize}

%%%%%%%%%%%%%%%%%%%%%%%%%%%%%%%%%%%%%%%%
\paragraph{Content of Main File.}

It is advisable to place all content in the child files included by |\include|.
Any output contained in the main file will appear in all child documents
unless suppressed manually;
it cannot be suppressed automatically by the |\includeonly| directive
and thus should normally be avoided.
A method to include some content in the main file
by means of conditional processing is described in \secref{sec:conditional}.

%%%%%%%%%%%%%%%%%%%%%%%%%%%%%%%%%%%%%%%%
\paragraph{Page Numbering.}

When only a part of the document is compiled,
the appropriate numbering of pages
(as well as other status parameters)
is determined from the |.aux| files.
The latter contain information from previous passes.
However this information needs to propagate through
all intermediate child documents.
Therefore the page numbering in child documents may well
be inconsistent until the complete document is compiled at least once.

A useful (if unconventional) way to always ensure a consistent
page numbering is to restart the numbering in each child document
and denote the pages by `\textit{child}|.|\textit{page}'
where \textit{child} represents the chapter/section number of the child file.
This can be achieved by the command
|\numberwithin{page}{|\textit{child}|}|
of the \textsf{amsmath} package
where \textit{child} can be |chapter| or |section|
depending on the chosen structuring.
Alternatively, one can modify the macro |\thepage| appropriately
and reset the counter |page| at the start of each child file.

%%%%%%%%%%%%%%%%%%%%%%%%%%%%%%%%%%%%%%%%%%%%%%%%%%%%%%%%%%%%%%%%%%%%%%%%%%%%%%%%
\subsection{Conditional Processing}
\label{sec:conditional}

The package provides a mechanism to compile different versions
of a document. To customise the versions further some conditional processing
can come in handy to distinguish which version is being compiled.
The package provides two macros to describe the compilation context:

%%%%%%%%%%%%%%%%%%%%%%%%%%%%%%%%%%%%%%%%
\DescribeMacro{\ifchilddoc}
The conditional |\ifchilddoc| distinguishes between the compilation of
child documents and the main document:
%
\begin{center}
|\ifchilddoc |\textit{child-code}| |[|\||else |\textit{main-code}]| \||fi|
\end{center}

%%%%%%%%%%%%%%%%%%%%%%%%%%%%%%%%%%%%%%%%
\DescribeMacro{\childdocname}
\DescribeMacro{\childdocjob}
The macro |\childdocname| contains the filename (without extension)
of the main or child file being processed.
Note that |\childdocjob| will always contain the name of the main file.

%%%%%%%%%%%%%%%%%%%%%%%%%%%%%%%%%%%%%%%%
\paragraph{Title Page.}

Conditional processing can be used to include a title or banner page
in the main document when proper precautions are taken.
Importantly, the code in the main file should ensure that the page counter
(as well as other status parameters which are stored in the |.aux| files)
takes the same value after the conditional processing.
Otherwise the page numbers may take divergent values
depending on which part is compiled.

For example, a title page could be declared by:
%
\begin{center}
\begin{tabular}{l}
|\ifchilddoc\||else|\\
|\addtocounter{page}{-1}|\\
\textit{code for title page}\\
|\newpage|\\
|\||fi|
\end{tabular}
\end{center}
%
A banner page for the child documents can be generated by:
%
\begin{center}
\begin{tabular}{l}
|\ifchilddoc|\\
|\addtocounter{page}{-1}|\\
\textit{code for banner page}\\
|\newpage|\\
|\||fi|
\end{tabular}
\end{center}
%
Here one could write a message such as:
\begin{center}
|This is the part \childdocname{} of \childdocjob{}.|
\end{center}

%%%%%%%%%%%%%%%%%%%%%%%%%%%%%%%%%%%%%%%%%%%%%%%%%%%%%%%%%%%%%%%%%%%%%%%%%%%%%%%%
\subsection{Flags}
\label{sec:flags}

The package makes it easy to generate different versions
of the main or child documents.
To this end compilation flags can be defined
and assigned different default values.
They will be particularly useful in conjunction
with the forwarding mechanism described in \secref{sec:forward}.

For example, it may be useful to have a flag |\version|
which can be set to |draft| or |final|.
The document source will contain some conditional code
depending on the value of |\version|.
Suppose further, the flag should default to |final| for the main file
and to |draft| for child files
which is a natural assignment for editing the document.
This is achieved by placing the following code
in the preamble of the main document
(below the |\childdocmain| directive):
%
\begin{center}
\begin{tabular}{l}
|\ifchilddoc|\\
|\providecommand{\version}{draft}|\\
|\||else|\\
|\providecommand{\version}{final}|\\
|\||fi|
\end{tabular}
\end{center}
%
The definition by |\providecommand| makes sure
that previous definitions are not overwritten.
Further statements |\providecommand{\version}{...}|
can thus be added before the above code to override it.

For the main file, one might add a line
(between |\childdocmain| and the above block)
%
\begin{center}
|%\ifchilddoc\||else\providecommand{\version}{draft}\||fi|
\end{center}
%
which can be uncommented to produce a draft version.
Likewise one can add a line to the very top of a child file
(above the |\childdocof{|\textit{main}|}| directive)
%
\begin{center}
|%\providecommand{\version}{final}|
\end{center}
%
which can be uncommented to produce the final version of this child document.

%%%%%%%%%%%%%%%%%%%%%%%%%%%%%%%%%%%%%%%%%%%%%%%%%%%%%%%%%%%%%%%%%%%%%%%%%%%%%%%%
\subsection{Forwarding}
\label{sec:forward}

Different versions of the main or child documents
using compilation flags as described in \secref{sec:flags}
can be (permanently) stored in different files
for convenient compilation, viewing and distribution.
To this end, the package defines a command
to pass on compilation to a different file:

%%%%%%%%%%%%%%%%%%%%%%%%%%%%%%%%%%%%%%%%
\DescribeMacro{\childdocforward}
The command |\childdocforward| redirects processing to
another source file:
%
\begin{center}
\begin{tabular}{l}
|\input{childdoc.def}|\\
|\childdocforward[|\textit{main}|]{|\textit{dest}|}|\\
\end{tabular}
\end{center}
%
The argument \textit{dest} is the destination file
(without extension).
It should be the main file or one of the child files.
Note that further \textsf{childdoc} directives
such as |\childdocof| and |\childdocforward|
in the indicated file will be processed in this form.
The optional argument \textit{main}
passes on directly to the main file \textit{main}
while pretending to compile the child \textit{dest}.
This form behaves as if \textit{dest}
issues |\childdocof{|\textit{main}|}| right away,
and no further \textsf{childdoc} directives will be processed.

%%%%%%%%%%%%%%%%%%%%%%%%%%%%%%%%%%%%%%%%
\DescribeMacro{\...prefix}
In the alternative form |\childdocforwardprefix|,
%
\begin{center}
\begin{tabular}{l}
|\input{childdoc.def}|\\
|\childdocforwardprefix[|\textit{main}|]{|\textit{prefix}|}{|\textit{dest}|}|
\end{tabular}
\end{center}
%
the destination file is determined by a pattern
depending on the current file:
To make this work, the current file must be called
`{\textit{prefix}\hspace{0.2em}\textit{suffix}}'
with \textit{prefix} matching precisely the argument.
Processing is then passed on to the file
`{\textit{dest}\hspace{0.2em}\textit{suffix}}'.
Surely, the same effect is achieved by
directly specifying the
argument `{\textit{dest}\hspace{0.2em}\textit{suffix}}'
in the first form.
However, that requires to set up a different file
for each child. With the alternative form of the command
all these files can have exactly the same content
which simplifies setting them up and maintaining them.

For example, the following file |draft.tex|
with a compilation flag |\version| as described in \secref{sec:flags}
compiles the main document as a draft:
%
\begin{center}
\begin{tabular}{l}
|\def\version{draft}|\\
|\input{childdoc.def}|\\
|\childdocforward{|\textit{main}|}|
\end{tabular}
\end{center}
%
Likewise, the following files |final|\textit{nn}|.tex|
compile the final version of the child document
|child|\textit{nn}|.tex|:
%
\begin{center}
\begin{tabular}{l}
|\def\version{final}|\\
|\input{childdoc.def}|\\
|\childdocforwardprefix{final}{child}|
\end{tabular}
\end{center}
%

Note that when several versions of a main file and/or of each child file
are to be generated, it may be convenient to set up a |Makefile| or
shell script to automatise the process.

%%%%%%%%%%%%%%%%%%%%%%%%%%%%%%%%%%%%%%%%%%%%%%%%%%%%%%%%%%%%%%%%%%%%%%%%%%%%%%%%
\subsection{Command Line Processing}
\label{sec:commandline}

The effect of redirection files can also be achieved by invoking
the \LaTeX{} compiler with a more elaborate command line.
Most conveniently this should be done as part
of a shell script or a |Makefile|.

When using \textsf{childdoc} in the main file, the following
command lines effectively perform a redirection
(note that depending on the shell being used,
backslashes may have to be doubled: `|\|' $\to$ `|\\|'):
%
\begin{center}
|... -jobname "|\textit{target}|" |\\|"|[\textit{flags}]%
|\input{childdoc.def}\childdocforward[|\textit{main}|]{|\textit{dest}|}"|
\end{center}
%
Here \textit{target} is the name of the output file,
\textit{main} is the name of the main file
and \textit{dest} is the name of the main or child file to be processed
(all filenames without extensions).
The optional argument \textit{main} can be omitted
if \textit{main} matches \textit{dest}.
Optionally, compilation \textit{flags} can be defined via |\def| commands.
This command line makes the \TeX{} engine believe
it is compiling the file \textit{target}
whose content is specified as the latter parameter.
The provided code then forwards the processing to
\textit{main} or \textit{dest} as described in \secref{sec:forward}.

%%%%%%%%%%%%%%%%%%%%%%%%%%%%%%%%%%%%%%%%%%%%%%%%%%%%%%%%%%%%%%%%%%%%%%%%%%%%%%%%
\subsection{Include by Input}
\label{sec:input}

Including child documents by |\include| has some restrictions by design.
Most notably, the content of a child document always occupies
its own set of pages; pages cannot be shared between child documents.
Usually, this behaviour makes perfect sense
because each child document contain an essential part of the document.
However, in some situations it may be desirable to compose
a document from a collection of parts
without having mandatory page breaks between then.
For this case, the package
provides a mechanism to include parts
by |\input| which can also be processed individually.
However, by construction this mechanism
requires manual handling of the content to be output.

%%%%%%%%%%%%%%%%%%%%%%%%%%%%%%%%%%%%%%%%
\DescribeMacro{\ifchilddocmanual}
The main file should be prepared as usual, see \secref{sec:include}.
However, the document body must make a distinction
between processing of an individual part and of the main document, e.g.:
%
\begin{center}
\begin{tabular}{l}
|\ifchilddocmanual|\\
|\input{\childdocname}|\\
|\||else|\\
\textit{document body with }|\input{|\textit{part}|}|\\
|\||fi|
\end{tabular}
\end{center}
%
The conditional |\ifchilddocmanual| is true whenever
a part to be included by |\input| is being compiled,
and the name of the part is stored in |\childdocname|.

%%%%%%%%%%%%%%%%%%%%%%%%%%%%%%%%%%%%%%%%
\DescribeMacro{\childdocby}
Each part to be included by |\input| should start with:
%
\begin{center}
\begin{tabular}{l}
|\input{childdoc.def}|\\
|\childdocby{|\textit{main}|}|\\
\end{tabular}
\end{center}
%
The directive |\childdocby| is similar to |\childdocof|
described in \secref{sec:include},
but the subsequent selection of content must be done manually.
To that end, both |\ifchilddoc| and |\ifchilddocmanual|
will be true upon processing of a part,
and the name of the part is stored in |\childdocname|.
Note that |\jobname| will be set to the filename of the current part
so that each part receives an individual |.aux| file
that does not interfere with the |.aux| file(s) of the main document.
This behaviour can be altered by the alternative form
|\childdocby[*]{|\textit{main}|}| (with a non-empty optional argument)
which uses the |.aux| file of the main document
by setting |\jobname| to \textit{main}.

%%%%%%%%%%%%%%%%%%%%%%%%%%%%%%%%%%%%%%%%%%%%%%%%%%%%%%%%%%%%%%%%%%%%%%%%%%%%%%%%
\subsection{Driver Development}
\label{sec:driver}

The \textsf{childdoc} mechanism can also be use for the development
of definition files such as \LaTeX{} styles or classes.
This case differs from the above setup with multiple parts
included by |\include| in that no |\includeonly| should be invoked.
This can be achieved by starting the include file
(before |\ProvidesPackage|) with:
%
\begin{center}
\begin{tabular}{l}
|\input{childdoc.def}|\\
|\childdocforward{|\textit{main}|}|\\
\end{tabular}
\end{center}
%
or alternatively with:
%
\begin{center}
\begin{tabular}{l}
|\input{childdoc.def}|\\
|\childdocby{|\textit{main}|}|\\
\end{tabular}
\end{center}
%
Both forms have slightly different effects as described above.
The main file is prepared as usual, see \secref{sec:include}.

%%%%%%%%%%%%%%%%%%%%%%%%%%%%%%%%%%%%%%%%%%%%%%%%%%%%%%%%%%%%%%%%%%%%%%%%%%%%%%%%
\subsection{Legacy Detection}
\label{sec:detection}

The directive |\childdocmain| in the main file can detect
whether the complete document or merely a child is to be compiled
even without using the directive |\childdocof|.
This method is deprecated because it is less robust
and there is no compelling reason to use it;
it is merely provided for backward compatibility
and it may be removed in future versions.

If the detection mechanism is to be used,
it is mandatory to correctly specify
the filename of the main file as the argument of |\childdocmain|:
%
\begin{center}
\begin{tabular}{l}
|\input{childdoc.def}|\\
|\childdocmain{|\textit{main}|}|\\
\end{tabular}
\end{center}
%
If |\jobname| does not match the argument \textit{main} of |\childdocmain|,
it is assumed that |\jobname| points to the child file to be compiled.
When using |\childdocmain| with the main file specified as argument,
it suffices to start a child file
with just |\input{|\textit{main}|}|
without loading of the package and using |\childdocof|.
If instead all processing is done
with the appropriate \textsf{childdoc} directives,
the argument of \textit{main} of |\childdocmain| can be empty.

An alternative version of the command line processing described
in \secref{sec:commandline} using the detection mechanism reads:
%
\begin{center}
|... -jobname "|\textit{target}|" "|[\textit{flags}]%
[|\def\jobname{|\textit{dest}|}|]|\input{|\textit{main}|}"|
\end{center}

%%%%%%%%%%%%%%%%%%%%%%%%%%%%%%%%%%%%%%%%%%%%%%%%%%%%%%%%%%%%%%%%%%%%%%%%%%%%%%%%
\subsection{Manual Code}
\label{sec:manual}

In case one cannot be certain whether the definitions file |childdoc.def|
is installed on the target \TeX{} distribution
and one prefers not to ship it,
it is conceivable to paste a few relevant commands into the sources.

To that end, drop all statements |\input{childdoc.def}|
and perform the replacements as outlined below.
Instead of |\childdocmain{|\textit{main}|}| add the following code
to the top of the main file:
%
\begin{center}
\begin{tabular}{l}
|\||ifdefined\childdocname\endinput\||fi\newif\ifchilddoc|\\
|\edef\childdocname{\scantokens\expandafter{\jobname\noexpand}}|\\
|\def\childdocmain{|\textit{main}|}\||ifx\childdocmain\childdocname\||else|\\
|\childdoctrue\includeonly{\childdocname}\let\jobname\childdocmain\||fi|\\
\end{tabular}
\end{center}
%
Instead of |\childdocof{|\textit{main}|}| just include the main file
at the top of each child file:
%
\begin{center}
|\input{|\textit{main}|}|
\end{center}
%
A simple redirection |\childdocforward{|\textit{dest}|}| is achieved by:
%
\begin{center}
|\def\jobname{|\textit{dest}|}\input{\jobname}|
\end{center}
%
The redirection with prefix
|\childdocforwardprefix[|\textit{prefix}|]{|\textit{dest}|}|
is accomplished by:
%
\begin{center}
\begin{tabular}{l}
|{\edef\jobname{\scantokens\expandafter{\jobname\noexpand}}|\\
|\def\redirectjob |\textit{prefix}|#1~~~{\gdef\jobname{|\textit{dest}|#1}}|\\
|\expandafter\redirectjob\jobname~~~}\input{\jobname}|
\end{tabular}
\end{center}

In an alternative approach,
child documents can be compiled by a specific command line
without additional code or specific definitions:
%
\begin{center}
|... -jobname "|\textit{target}|" "|[\textit{flags}]%
|\includeonly{|\textit{dest}|}\input{|\textit{main}|}"|
\end{center}
%

%%%%%%%%%%%%%%%%%%%%%%%%%%%%%%%%%%%%%%%%%%%%%%%%%%%%%%%%%%%%%%%%%%%%%%%%%%%%%%%%
%%%%%%%%%%%%%%%%%%%%%%%%%%%%%%%%%%%%%%%%%%%%%%%%%%%%%%%%%%%%%%%%%%%%%%%%%%%%%%%%
\section{Information}

%%%%%%%%%%%%%%%%%%%%%%%%%%%%%%%%%%%%%%%%%%%%%%%%%%%%%%%%%%%%%%%%%%%%%%%%%%%%%%%%
\subsection{Copyright}

Copyright \copyright{} 2017--2018 Niklas Beisert

This work may be distributed and/or modified under the
conditions of the \LaTeX{} Project Public License, either version 1.3
of this license or (at your option) any later version.
The latest version of this license is in
  \url{http://www.latex-project.org/lppl.txt}
and version 1.3 or later is part of all distributions of \LaTeX{}
version 2005/12/01 or later.

This work has the LPPL maintenance status `maintained'.

The Current Maintainer of this work is Niklas Beisert.

This work consists of the files |README.txt|, |childdoc.ins| and |childdoc.dtx|
as well as the derived files |childdoc.def|, |cdocsamp.tex|
with |cdocsch1.tex|, |cdocsch2.tex|, |cdocspt3.tex|, |cdocspt4.tex|,
|cdocsdrf.tex|, |cdocsfn1.tex|, |cdocsfn2.tex|
as well as |childdoc.pdf|.

%%%%%%%%%%%%%%%%%%%%%%%%%%%%%%%%%%%%%%%%%%%%%%%%%%%%%%%%%%%%%%%%%%%%%%%%%%%%%%%%
\subsection{Files and Installation}

The package consists of the files:
%
\begin{center}
\begin{tabular}{ll}
    |README.txt|   & readme file \\
    |childdoc.ins| & installation file \\
    |childdoc.dtx| & source file \\
    |childdoc.def| & definition file \\
    |cdocsamp.tex| & sample main file \\
    |cdocsch1.tex| & sample include file \\
    |cdocsch2.tex| & sample include file \\
    |cdocspt3.tex| & sample part file \\
    |cdocspt4.tex| & sample part file \\
    |cdocsdrf.tex| & sample redirection file \\
    |cdocsfn1.tex| & sample redirection file \\
    |cdocsfn2.tex| & sample redirection file \\
    |childdoc.pdf| & manual
\end{tabular}
\end{center}
%
The distribution consists of the files
|README.txt|, |childdoc.ins| and |childdoc.dtx|.
%
\begin{itemize}
\item
Run (pdf)\LaTeX{} on |childdoc.dtx|
to compile the manual |childdoc.pdf| (this file).
\item
Run \LaTeX{} on |childdoc.ins| to create the definitions file |childdoc.def|
and the sample |cdocsamp.tex| with include files
|cdocsch1.tex|, |cdocsch2.tex|, |cdocspt3.tex|, |cdocspt4.tex|,
|cdocsdrf.tex|, |cdocsfn1.tex|, |cdocsfn2.tex|.
Then copy the file |childdoc.def| to an appropriate directory of your \LaTeX{}
distribution, e.g.\ \textit{texmf-root}|/tex/latex/childdoc|.
\end{itemize}

%%%%%%%%%%%%%%%%%%%%%%%%%%%%%%%%%%%%%%%%%%%%%%%%%%%%%%%%%%%%%%%%%%%%%%%%%%%%%%%%
\subsection{Related CTAN Packages}

There are several other packages which offer a similar functionality:
%
\begin{itemize}
\item
The packages
\href{http://ctan.org/pkg/docmute}{\textsf{docmute}},
\href{http://ctan.org/pkg/includex}{\textsf{includex}} and
\href{http://ctan.org/pkg/standalone}{\textsf{standalone}}
provide commands to include only the document body of
a child file thus allowing both files to be compiled individually.
\item
The packages \href{http://ctan.org/pkg/subdocs}{\textsf{subdocs}}
and \href{http://ctan.org/pkg/subfiles}{\textsf{subfiles}}
provide structures in which the main and child documents can be
encapsulated and allowing them to be compiled individually.
The inclusion mechanism is different from the conventional |\include|.
\item
The package \href{http://ctan.org/pkg/combine}{\textsf{combine}}
is an elaborate solution to combine several documents into one.
\end{itemize}
%
See also the CTAN topic \href{http://ctan.org/topic/subdocs}{\textsf{subdocs}}
for further related packages.
The present package differs from the above solutions in that
a document structure constructed with the conventional |\include| mechanism
just needs two extra commands at the top of every file
such that all constituent files can be compiled individually.

%%%%%%%%%%%%%%%%%%%%%%%%%%%%%%%%%%%%%%%%%%%%%%%%%%%%%%%%%%%%%%%%%%%%%%%%%%%%%%%%
%\subsection{Feature Suggestions}
%
%The following is a list of features which may be useful for future
%versions of this package:
%%
%\begin{itemize}
%\item
%\ldots
%\end{itemize}

%%%%%%%%%%%%%%%%%%%%%%%%%%%%%%%%%%%%%%%%%%%%%%%%%%%%%%%%%%%%%%%%%%%%%%%%%%%%%%%%
\subsection{Revision History}

%%%%%%%%%%%%%%%%%%%%%%%%%%%%%%%%%%%%%%%%
\paragraph{v2.0:} 2018/12/30

\begin{itemize}
\item
immediate forward processing
\item
added |\childdocby| mechanism
\item
manual restructured
\end{itemize}

%%%%%%%%%%%%%%%%%%%%%%%%%%%%%%%%%%%%%%%%
\paragraph{v1.6:} 2018/01/17

\begin{itemize}
\item
application for development of include files
\item
corrections to manual
\end{itemize}

%%%%%%%%%%%%%%%%%%%%%%%%%%%%%%%%%%%%%%%%
\paragraph{v1.5:} 2017/05/21

\begin{itemize}
\item
more complete structuring introduced
\item
|\childdocof| introduced
\item
|\childdoc| renamed to |\childdocmain|
\item
|\childredirect| renamed to |\childdocforward| and |\childdocforwardprefix|
and functionality expanded
\end{itemize}

%%%%%%%%%%%%%%%%%%%%%%%%%%%%%%%%%%%%%%%%
\paragraph{v1.0:} 2017/04/27

\begin{itemize}
\item
manual and install package
\item
first version published on CTAN
\end{itemize}

%%%%%%%%%%%%%%%%%%%%%%%%%%%%%%%%%%%%%%%%
\paragraph{v0.6:} 2017/04/26

\begin{itemize}
\item
redirection mechanism added
\end{itemize}

%%%%%%%%%%%%%%%%%%%%%%%%%%%%%%%%%%%%%%%%
\paragraph{v0.5:} 2017/04/26

\begin{itemize}
\item
functionality in definition file
\end{itemize}


%%%%%%%%%%%%%%%%%%%%%%%%%%%%%%%%%%%%%%%%%%%%%%%%%%%%%%%%%%%%%%%%%%%%%%%%%%%%%%%%
%%%%%%%%%%%%%%%%%%%%%%%%%%%%%%%%%%%%%%%%%%%%%%%%%%%%%%%%%%%%%%%%%%%%%%%%%%%%%%%%
%%%%%%%%%%%%%%%%%%%%%%%%%%%%%%%%%%%%%%%%%%%%%%%%%%%%%%%%%%%%%%%%%%%%%%%%%%%%%%%%
\appendix

\settowidth\MacroIndent{\rmfamily\scriptsize 000\ }

 \DocInput{childdoc.dtx}

\end{document}
%</driver>
% \fi
%
% %%%%%%%%%%%%%%%%%%%%%%%%%%%%%%%%%%%%%%%%%%%%%%%%%%%%%%%%%%%%%%%%%%%%%%%%%%%%%%
% %%%%%%%%%%%%%%%%%%%%%%%%%%%%%%%%%%%%%%%%%%%%%%%%%%%%%%%%%%%%%%%%%%%%%%%%%%%%%%
% \section{Sample}
%\iffalse
%<*samplemain>
%\fi
%
% The following presents a sample document
% with two chapters, two parts, a title page,
% a compile flag as well as three forwarding files to set the flag.
% It consists of eight |.tex| files:
% \begin{center}
% \begin{tabular}{ll}
% |cdocsamp.tex|&main file\\
% |cdocsch1.tex|&include file for chapter 1\\
% |cdocsch2.tex|&include file for chapter 2\\
% |cdocspt3.tex|&include file for part 3\\
% |cdocspt4.tex|&include file for part 4\\
% |cdocsdrf.tex|&forwarding file for main file in draft mode\\
% |cdocsfi1.tex|&forwarding file for final version of chapter 1\\
% |cdocsfi2.tex|&forwarding file for final version of chapter 2\\
% \end{tabular}
% \end{center}
% Each of the eight files can be compiled directly by the \LaTeX{} compiler.
%
% %%%%%%%%%%%%%%%%%%%%%%%%%%%%%%%%%%%%%%
% \paragraph{Main File.}
%
% The main file is called |cdocsamp.tex|.
%
% Load the \textsf{childdoc} definitions and
% declare the filename for the main document:
%    \begin{macrocode}
\input{childdoc.def}
\childdocmain{}
%    \end{macrocode}

% Optional override for |\version| flag:
%    \begin{macrocode}
%%\ifchilddoc\else\providecommand{\version}{draft}\fi
%    \end{macrocode}

% Define the default values for the |\version| flag
% (|final| for the main file and |draft| for childs):
%    \begin{macrocode}
\ifchilddoc
\providecommand{\version}{draft}
\else
\providecommand{\version}{final}
\fi
%    \end{macrocode}

% Load the standard document class:
%    \begin{macrocode}
\documentclass[12pt]{article}
%    \end{macrocode}

% Start the document body:
%    \begin{macrocode}
\begin{document}
%    \end{macrocode}

% Declare a title page.
% Print title, part of document being processed and version flag:
%    \begin{macrocode}
\addtocounter{page}{-1}
\begin{center}
{\LARGE\bfseries{}childdoc example\par}
\vspace{1cm}
\ifchilddoc
\ifchilddocmanual part\else chapter\fi:
`\childdocname' of `\childdocjob'\par
\else
main document: `\childdocjob'\par
\fi
version: \version\par
\end{center}
\newpage
%    \end{macrocode}

% Manually include selected file,
% otherwise process as usual:
%    \begin{macrocode}
\ifchilddocmanual
\section*{part `\childdocname'}
\input{\childdocname}
\else
%    \end{macrocode}

% Include the two chapters:
%    \begin{macrocode}
\include{cdocsch1}
\include{cdocsch2}
%    \end{macrocode}

% Include the two parts unless only chapters should be displayed:
%    \begin{macrocode}
\ifchilddoc\else
\section{part three}
\input{cdocspt3}
\section{part four}
\input{cdocspt4}
\fi
%    \end{macrocode}

% Process as usual until here:
%    \begin{macrocode}
\fi
%    \end{macrocode}

% End of document body:
%    \begin{macrocode}
\end{document}
%    \end{macrocode}
%\iffalse
%</samplemain>
%\fi
%
% %%%%%%%%%%%%%%%%%%%%%%%%%%%%%%%%%%%%%%
% \paragraph{Chapter Include Files.}
%
% The include files are called |cdocsch1.tex| and |cdocsch2.tex|.
%
%\iffalse
%<*samplechap1|samplechap2>
%\fi

% Optional override for |\version| flag:
%    \begin{macrocode}
%%\providecommand{\version}{final}
%    \end{macrocode}

% Include the main document:
%    \begin{macrocode}
\input{childdoc.def}
\childdocof{cdocsamp}
%    \end{macrocode}

%\iffalse
%</samplechap1|samplechap2>
%\fi
%
%\iffalse
%<*samplechap1>
%\fi
% Some text for chapter 1:
%    \begin{macrocode}
\section{one}
some text in chapter one
%    \end{macrocode}

%\iffalse
%</samplechap1>
%\fi
% Some text for chapter 2:
%\iffalse
%<*samplechap2>
%\fi
%    \begin{macrocode}
\section{two}
more text in chapter two
%    \end{macrocode}

%\iffalse
%</samplechap2>
%\fi
%
% %%%%%%%%%%%%%%%%%%%%%%%%%%%%%%%%%%%%%%
% \paragraph{Part Include Files.}
%
% The include files are called |cdocspt3.tex| and |cdocspt4.tex|.
%
%\iffalse
%<*samplepart3|samplepart4>
%\fi

% Optional override for |\version| flag:
%    \begin{macrocode}
%%\providecommand{\version}{final}
%    \end{macrocode}

% Include the main document:
%    \begin{macrocode}
\input{childdoc.def}
\childdocby{cdocsamp}
%    \end{macrocode}

%\iffalse
%</samplepart3|samplepart4>
%\fi
%
%\iffalse
%<*samplepart3>
%\fi
% Some text for part 3:
%    \begin{macrocode}
some text in part three
%    \end{macrocode}

%\iffalse
%</samplepart3>
%\fi
% Some text for part 4:
%\iffalse
%<*samplepart4>
%\fi
%    \begin{macrocode}
more text in part four
%    \end{macrocode}

%\iffalse
%</samplepart4>
%\fi
%
% %%%%%%%%%%%%%%%%%%%%%%%%%%%%%%%%%%%%%%
% \paragraph{Forwarding for a Complete Draft.}
%
% The following forwarding file |cdocsdrf.tex|
% compiles the main document in draft mode:
%\iffalse
%<*sampledraft>
%\fi
%    \begin{macrocode}
\def\version{draft}
\input{childdoc.def}
\childdocforward{cdocsamp}
%    \end{macrocode}

%\iffalse
%</sampledraft>
%\fi
%
% %%%%%%%%%%%%%%%%%%%%%%%%%%%%%%%%%%%%%%
% \paragraph{Forwarding for Final Version of the Chapters.}
%
% The following forwarding files |cdocsfn1.tex| and |cdocsfn2.tex|
% (with identical content)
% compile the final versions of the child documents
% |cdocsch1.tex| and |cdocsch2.tex|, respectively:
%\iffalse
%<*samplefinal>
%\fi
%    \begin{macrocode}
\def\version{final}
\input{childdoc.def}
\childdocforwardprefix[cdocsamp]{cdocsfn}{cdocsch}
%    \end{macrocode}

%\iffalse
%</samplefinal>
%\fi
%
% %%%%%%%%%%%%%%%%%%%%%%%%%%%%%%%%%%%%%%
% \paragraph{Command Line Processing.}
%
% The following three command lines generate the output files
% |cdocscld|, |cdocscl1| and |cdocscl2|
% which should be identical to
% |cdocsdrf|, |cdocsch1| and |cdocsfn2|, respectively:
% \begin{center}
% \begin{tabular}{l}
% |latex -jobname cdocscld \|\\
% |  "\def\version{draft}\input{childdoc.def}\childdocforward{cdocsamp}"|\\
% |latex -jobname cdocscl1 \|\\
% |  "\input{childdoc.def}\childdocforward[cdocsamp]{cdocsch1}"|\\
% |latex -jobname cdocscl2 \|\\
% |  "\def\version{final}\input{childdoc.def}\childdocforward{cdocsch2}"|
% \end{tabular}
% \end{center}
% Note that the trailing backslash on each first line
% merely continues the input to the second line
% (for convenient cut ant paste).
% Furthermore, the command |latex| can be replaced by any
% of its alternative versions such as |pdflatex|.
%
% %%%%%%%%%%%%%%%%%%%%%%%%%%%%%%%%%%%%%%%%%%%%%%%%%%%%%%%%%%%%%%%%%%%%%%%%%%%%%%
% %%%%%%%%%%%%%%%%%%%%%%%%%%%%%%%%%%%%%%%%%%%%%%%%%%%%%%%%%%%%%%%%%%%%%%%%%%%%%%
% \section{Implementation}
%\iffalse
%<*package>
%\fi
%
% This section describes the definitions file |childdoc.def|.

% The definitions cannot be loaded using |\usepackage| or |\RequirePackage|
% which has a mechanism to prevent loading a style file more than once.
% When loading the definitions by means of |\input|
% multiple instances have to be prevented manually:
%\iffalse
%This code needs to be before the `\ProvidesFile' directive
%which is defined at the beginning of this file.
%Therefore it is also placed there and commented out here.
%</package>
%<*discard>
%\fi
%    \begin{macrocode}
\ifdefined\childdocmain\endinput\fi
%    \end{macrocode}
%\iffalse
%</discard>
%<*package>
%\fi
%
% \macro{\ifchilddoc}
% \macro{\ifchilddocmanual}
% The conditional |\ifchilddoc| tells whether a
% child (true) or main (false) document is being compiled.
% The conditional |\ifchilddocmanual| tells whether
% the |\includeonly| mechanism is used (false) or
% the selection of child files must be performed manually (true).
% The definitions initialise to false:
%    \begin{macrocode}
\newif\ifchilddoc
\newif\ifchilddocmanual
%    \end{macrocode}

% \macro{\childdocname}
% \macro{\childdocjob}
% The macro |\childdocname| stores the name of the main document
% to be compiled. The macro |\childdocjob| stores the name of
% the document on which the \LaTeX{} compiler was originally invoked.
% The content of |\jobname| cannot be compared
% to filenames specified in the source due to different catcodes.
% The following code rescans |\jobname|, stores the result
% in |\childdocname| and saves a copy in |\childdocjob|:
%    \begin{macrocode}
\edef\childdocname{\scantokens\expandafter{\jobname\noexpand}}
\let\childdocjob\childdocname
%    \end{macrocode}

% \macro{\childdocdisable}
% The macro |\childdocdisable| prevents the main file
% from being processed more than once.
% At this stage, the main document command |\childdocmain|
% is assumed to be called once again where it should do nothing.
% Any subsequent call to it should prevent
% a secondary processing of the main document
% It overwrites the forwarding commands
% |\childdocof| and |\childdocforward|
% with empty macros to prevent further inclusions of the main document:
%    \begin{macrocode}
\newcommand{\childdocdisable}
{
  \renewcommand{\childdocmain}[1]{\renewcommand{\childdocmain}[1]{\endinput}}
  \renewcommand{\childdocof}[1]{}
  \renewcommand{\childdocby}[2][]{}
  \renewcommand{\childdocforward}[2][]{}
  \renewcommand{\childdocdisable}{}
}
%    \end{macrocode}

% \macro{\childdocmain}
% The macro |\childdocmain| is to be called at the top of the main file
% with nothing or the main filename (without extension) as argument.
% First, it breaks loops.
% If the argument is not empty and does not match |\childdocname|
% (which is set by the first inclusion of |childdoc.def|),
% |\ifchilddoc| is set to true, |\includeonly| is applied to the child file
% and |\jobname| is set to the main file
% (for proper handling of |.aux| files):
%    \begin{macrocode}
\newcommand{\childdocmain}[1]
{
  \childdocdisable\childdocmain{}
  \if?#1?\else
    \begingroup
      \def\childdoctmp{#1}
      \ifx\childdoctmp\childdocname
        \def\childdoctmp{}
      \else
        \def\childdoctmp
        {
          \childdoctrue
          \includeonly{\childdocname}
          \def\childdocjob{#1}
          \def\jobname{#1}
        }
      \fi
      \expandafter
    \endgroup
    \childdoctmp
  \fi
}
%    \end{macrocode}

% \macro{\childdocof}
% The command |\childdocof| redirects
% compilation to the main file |#1|.
%    \begin{macrocode}
\newcommand{\childdocof}[1]
{
  \childdocdisable
  \childdoctrue
  \includeonly{\childdocname}
  \def\jobname{#1}
  \def\childdocjob{#1}
  \input{#1}
}
%    \end{macrocode}

% \macro{\childdocby}
% The command |\childdocby| ....
%    \begin{macrocode}
\newcommand{\childdocby}[2][]
{
  \childdocdisable
  \childdoctrue
  \childdocmanualtrue
  \if?#1?\else
    \def\jobname{#2}
  \fi
  \def\childdocjob{#2}
  \input{#2}
  \endinput
}
%    \end{macrocode}

% \macro{\childdocforward}
% The command |\childdocforward| redirects
% compilation to the main file or
% (if the optional argument is given) a child file.
% Parameters are set as if the main file
% or a child file starting with |\childdocof| was compiled.
% Then compilation is handed over to the main file:
%    \begin{macrocode}
\newcommand{\childdocforward}[2][]
{
  \begingroup
    \if?#1?
      \def\childdoctmp
      {
        \def\childdocname{#2}
        \def\childdocjob{#2}
        \def\jobname{#2}
        \input{#2}
        \endinput
      }
    \else
      \def\childdoctmp
      {
        \childdocdisable
        \def\childdocname{#2}
        \childdoctrue
        \includeonly{#2}
        \def\childdocjob{#1}
        \def\jobname{#1}
        \input{#1}
        \endinput
      }
    \fi
    \expandafter
  \endgroup
  \childdoctmp
}
%    \end{macrocode}

% \macro{\childdocforwardprefix}
% The command |\childdocforwardprefix| redirects
% compilation to the main or a child file by means of a pattern.
% The prefix |#1| in the current filename is replaced by |#2|
% and the suffix of the current filename is kept
% (it is assumed that the filename does not contain the substring `|~~~|'
% which is used as a delimiter).
% Compilation is handed over to the new file by |\childdocforward|:
%    \begin{macrocode}
\newcommand{\childdocforwardprefix}[3][]
{
  \begingroup
    \def\childdocextract #2##1~~~{\def\childdoctmp{\childdocforward[#1]{#3##1}}}
    \expandafter\childdocextract\childdocname~~~
    \expandafter
  \endgroup
  \childdoctmp
}
%    \end{macrocode}

% \macro{\childdoc}
% The deprecated macro |\childdoc| is a legacy version of |\childdocmain|:
%    \begin{macrocode}
\newcommand{\childdoc}{\childdocmain}
%    \end{macrocode}

% \macro{\childdocredirect}
% The deprecated macro |\childdocredirect| is a legacy version
% of |\childdocforward| and |\childdocforwardprefix|:
%    \begin{macrocode}
\newcommand{\childdocredirect}[2][]
{
  \begingroup
    \if?#1?
      \def\childdoctmp{\childdocforward{#2}}
    \else
      \def\childdoctmp{\childdocforwardprefix{#1}{#2}}
    \fi
    \expandafter
  \endgroup
  \childdoctmp
}
%    \end{macrocode}

%\iffalse
%</package>
%\fi
%
\endinput

\childdocby{cdocsamp}
%    \end{macrocode}

%\iffalse
%</samplepart3|samplepart4>
%\fi
%
%\iffalse
%<*samplepart3>
%\fi
% Some text for part 3:
%    \begin{macrocode}
some text in part three
%    \end{macrocode}

%\iffalse
%</samplepart3>
%\fi
% Some text for part 4:
%\iffalse
%<*samplepart4>
%\fi
%    \begin{macrocode}
more text in part four
%    \end{macrocode}

%\iffalse
%</samplepart4>
%\fi
%
% %%%%%%%%%%%%%%%%%%%%%%%%%%%%%%%%%%%%%%
% \paragraph{Forwarding for a Complete Draft.}
%
% The following forwarding file |cdocsdrf.tex|
% compiles the main document in draft mode:
%\iffalse
%<*sampledraft>
%\fi
%    \begin{macrocode}
\def\version{draft}
% \iffalse
%
% childdoc.dtx Copyright (C) 2017-2018 Niklas Beisert
%
% This work may be distributed and/or modified under the
% conditions of the LaTeX Project Public License, either version 1.3
% of this license or (at your option) any later version.
% The latest version of this license is in
%   http://www.latex-project.org/lppl.txt
% and version 1.3 or later is part of all distributions of LaTeX
% version 2005/12/01 or later.
%
% This work has the LPPL maintenance status `maintained'.
%
% The Current Maintainer of this work is Niklas Beisert.
%
% This work consists of the files childdoc.dtx and childdoc.ins
% and the derived files childdoc.def and cdocsamp.tex with
% cdocsch1.tex, cdocsch2.tex, cdocsdrf.tex, cdocsfn1.tex, cdocsfn2.tex.
%
%<package>\ifdefined\childdocmain\endinput\fi
%<package>\ProvidesFile{childdoc.def}[2018/12/30 v2.0 child document driver]
%<samplemain>\ProvidesFile{cdocsamp.tex}[2018/12/30 v2.0 sample for childdoc]
%<*driver>
%\ProvidesFile{childdoc.drv}[2018/12/30 v2.0 childdoc reference manual file]
\PassOptionsToClass{10pt,a4paper}{article}
\documentclass{ltxdoc}

\usepackage[margin=35mm]{geometry}
\usepackage{hyperref}
\usepackage{hyperxmp}
\usepackage[usenames]{color}

\hypersetup{colorlinks=true}
\hypersetup{pdfstartview=FitH}
\hypersetup{pdfpagemode=UseNone}
\hypersetup{pdfsource={}}
\hypersetup{pdflang={en-UK}}
\hypersetup{pdfcopyright={Copyright 2017-2018 Niklas Beisert.
  This work may be distributed and/or modified under the
  conditions of the LaTeX Project Public License, either version 1.3
  of this license or (at your option) any later version.}}
\hypersetup{pdflicenseurl={http://www.latex-project.org/lppl.txt}}
\hypersetup{pdfcontactaddress={ETH Zurich, ITP, HIT K,
  Wolfgang-Pauli-Strasse 27}}
\hypersetup{pdfcontactpostcode={8093}}
\hypersetup{pdfcontactcity={Zurich}}
\hypersetup{pdfcontactcountry={Switzerland}}
\hypersetup{pdfcontactemail={nbeisert@itp.phys.ethz.ch}}
\hypersetup{pdfcontacturl={http://people.phys.ethz.ch/\xmptilde nbeisert/}}

\newcommand{\secref}[1]{\hyperref[#1]{section \ref*{#1}}}

\parskip1ex
\parindent0pt
\let\olditemize\itemize
\def\itemize{\olditemize\parskip0pt}

\begin{document}

\title{The \textsf{childdoc} Package}
\hypersetup{pdftitle={The childdoc Package}}
\author{Niklas Beisert\\[2ex]
  Institut f\"ur Theoretische Physik\\
  Eidgen\"ossische Technische Hochschule Z\"urich\\
  Wolfgang-Pauli-Strasse 27, 8093 Z\"urich, Switzerland\\[1ex]
  \href{mailto:nbeisert@itp.phys.ethz.ch}
  {\texttt{nbeisert@itp.phys.ethz.ch}}}
\hypersetup{pdfauthor={Niklas Beisert}}
\hypersetup{pdfsubject={Manual for the LaTeX2e Package childdoc}}
\date{30 December 2018, \textsf{v2.0}}
\maketitle

\begin{abstract}\noindent
\textsf{childdoc} is a \LaTeXe{} package
that enables the direct compilation
of document sections included by |\include|
to individual files.
\end{abstract}

\begingroup
\parskip0ex
\tableofcontents
\endgroup

%%%%%%%%%%%%%%%%%%%%%%%%%%%%%%%%%%%%%%%%%%%%%%%%%%%%%%%%%%%%%%%%%%%%%%%%%%%%%%%%
%%%%%%%%%%%%%%%%%%%%%%%%%%%%%%%%%%%%%%%%%%%%%%%%%%%%%%%%%%%%%%%%%%%%%%%%%%%%%%%%
\section{Introduction}

\LaTeX{} provides a mechanism to structure a large document (such as a book)
into a main file and several child files (containing the chapters)
using the |\include| command.
This mechanism is beneficial for documents
which span hundreds of pages in order to
make the source file(s) more manageable.
Moreover, compilation can be restricted to
selected child files by means of the |\includeonly| command.
The latter feature can be used to reduce the compilation time while editing
(this was significantly more useful in the earlier days of \LaTeX{})
or to generate a smaller document which is easier to navigate.
Another application of |\includeonly| is to generate
documents consisting of selected parts of the complete document.

However, there are a few drawbacks of the plain |\include| mechanism:
\begin{itemize}
\item
The child files cannot be compiled on their own,
they can only be compiled via the main file.
A naive editing environment
(such as a text editor with an option
to have the current file processed by \LaTeX)
may require one to switch to the main file before compiling;
attempting to compile the child file produces errors.
\item
The main file must be modified (each time)
to adjust the |\includeonly| command
to the present needs. This easily leaves the main file in a messy state.
\item
The generated document will always carry the filename
of the main document. This is inconvenient if
several child files are to be compiled and
to be kept for distribution.
\end{itemize}

The present package provides a simple interface
to make child files individually compilable by \LaTeX{}.
Compiling a child file then has the same effect as compiling
the main file with an |\includeonly| command
to select the appropriate child.
Moreover the generated document will carry the name of the child
rather than the main file.
This resolves all three above issues.

This feature is meant to make the editing of books,
thesis documents and lecture notes somewhat more convenient.
However, the package can also be used efficiently for
composing a series of documents (such as exercise sheets)
which are typically distributed individually.
It then assists the author in generating the individual documents
(potentially in different versions)
as well as a document containing the collected series.
Another application is in developing style files
or other kinds of included material
where compilation of the style file could redirect
to a sample or test file.

%%%%%%%%%%%%%%%%%%%%%%%%%%%%%%%%%%%%%%%%%%%%%%%%%%%%%%%%%%%%%%%%%%%%%%%%%%%%%%%%
%%%%%%%%%%%%%%%%%%%%%%%%%%%%%%%%%%%%%%%%%%%%%%%%%%%%%%%%%%%%%%%%%%%%%%%%%%%%%%%%
\section{Usage}

First of all, the package \textsf{childdoc} is \emph{not} a standard
\LaTeXe{} |.sty| style file! Therefore it needs to be invoked in
a non-standard way.

%%%%%%%%%%%%%%%%%%%%%%%%%%%%%%%%%%%%%%%%%%%%%%%%%%%%%%%%%%%%%%%%%%%%%%%%%%%%%%%%
\subsection{Included Files}
\label{sec:include}

%%%%%%%%%%%%%%%%%%%%%%%%%%%%%%%%%%%%%%%%
\DescribeMacro{\childdocmain}
To use the package, add the commands
\begin{center}
\begin{tabular}{l}
|\input{childdoc.def}|\\
|\childdocmain{}|\\
\end{tabular}
\end{center}
at the very top of the main \LaTeX{} file,
in particular \emph{before} the |\documentclass| statement!
The argument of |\childdocmain| should be left empty
(but it must be present).

%%%%%%%%%%%%%%%%%%%%%%%%%%%%%%%%%%%%%%%%
\DescribeMacro{\childdocof}
Furthermore, add the commands
\begin{center}
\begin{tabular}{l}
|\input{childdoc.def}|\\
|\childdocof{|\textit{main}|}|\\
\end{tabular}
\end{center}
at the top of every child file \textit{child}
which is included by |\include{|\textit{child}|}|
from within the main file
(or at least for those files to be compiled individually).
The argument \textit{main} must be the filename of the main file.

There are a couple of
considerations in setting up the main and child documents:

%%%%%%%%%%%%%%%%%%%%%%%%%%%%%%%%%%%%%%%%
\paragraph{Restrictions.}

Please note the following restrictions:
\begin{itemize}
\item
|\childdocmain| must be called with one argument \textit{main}
to ensure compatibility with earlier version of the package.
It must either be empty (|\childdocmain{}|)
or precisely match the filename of the main file in which it is specified.
See \secref{sec:detection} for further information.
\item
The filename \textit{main} must be specified without the |.tex| extension.
\item
The filename \textit{main} is case sensitive
(even in case-insensitive file systems)
due to internal string comparison.
\item
The argument \textit{main} should be fully expanded, it cannot be a macro.
\item
Subdirectories and special characters should be avoided in filenames.
\item
The command |\childdocmain{|\textit{main}|}| must be followed by a whitespace.
It should not be followed immediately by another command
or by a comment mark `|%|'.
This is because the \TeX{} parser reads the token immediately following
the argument of |\childdocmain| and puts it
at the beginning of every child section;
however, a white\-space is ignored.
\end{itemize}

%%%%%%%%%%%%%%%%%%%%%%%%%%%%%%%%%%%%%%%%
\paragraph{Content of Main File.}

It is advisable to place all content in the child files included by |\include|.
Any output contained in the main file will appear in all child documents
unless suppressed manually;
it cannot be suppressed automatically by the |\includeonly| directive
and thus should normally be avoided.
A method to include some content in the main file
by means of conditional processing is described in \secref{sec:conditional}.

%%%%%%%%%%%%%%%%%%%%%%%%%%%%%%%%%%%%%%%%
\paragraph{Page Numbering.}

When only a part of the document is compiled,
the appropriate numbering of pages
(as well as other status parameters)
is determined from the |.aux| files.
The latter contain information from previous passes.
However this information needs to propagate through
all intermediate child documents.
Therefore the page numbering in child documents may well
be inconsistent until the complete document is compiled at least once.

A useful (if unconventional) way to always ensure a consistent
page numbering is to restart the numbering in each child document
and denote the pages by `\textit{child}|.|\textit{page}'
where \textit{child} represents the chapter/section number of the child file.
This can be achieved by the command
|\numberwithin{page}{|\textit{child}|}|
of the \textsf{amsmath} package
where \textit{child} can be |chapter| or |section|
depending on the chosen structuring.
Alternatively, one can modify the macro |\thepage| appropriately
and reset the counter |page| at the start of each child file.

%%%%%%%%%%%%%%%%%%%%%%%%%%%%%%%%%%%%%%%%%%%%%%%%%%%%%%%%%%%%%%%%%%%%%%%%%%%%%%%%
\subsection{Conditional Processing}
\label{sec:conditional}

The package provides a mechanism to compile different versions
of a document. To customise the versions further some conditional processing
can come in handy to distinguish which version is being compiled.
The package provides two macros to describe the compilation context:

%%%%%%%%%%%%%%%%%%%%%%%%%%%%%%%%%%%%%%%%
\DescribeMacro{\ifchilddoc}
The conditional |\ifchilddoc| distinguishes between the compilation of
child documents and the main document:
%
\begin{center}
|\ifchilddoc |\textit{child-code}| |[|\||else |\textit{main-code}]| \||fi|
\end{center}

%%%%%%%%%%%%%%%%%%%%%%%%%%%%%%%%%%%%%%%%
\DescribeMacro{\childdocname}
\DescribeMacro{\childdocjob}
The macro |\childdocname| contains the filename (without extension)
of the main or child file being processed.
Note that |\childdocjob| will always contain the name of the main file.

%%%%%%%%%%%%%%%%%%%%%%%%%%%%%%%%%%%%%%%%
\paragraph{Title Page.}

Conditional processing can be used to include a title or banner page
in the main document when proper precautions are taken.
Importantly, the code in the main file should ensure that the page counter
(as well as other status parameters which are stored in the |.aux| files)
takes the same value after the conditional processing.
Otherwise the page numbers may take divergent values
depending on which part is compiled.

For example, a title page could be declared by:
%
\begin{center}
\begin{tabular}{l}
|\ifchilddoc\||else|\\
|\addtocounter{page}{-1}|\\
\textit{code for title page}\\
|\newpage|\\
|\||fi|
\end{tabular}
\end{center}
%
A banner page for the child documents can be generated by:
%
\begin{center}
\begin{tabular}{l}
|\ifchilddoc|\\
|\addtocounter{page}{-1}|\\
\textit{code for banner page}\\
|\newpage|\\
|\||fi|
\end{tabular}
\end{center}
%
Here one could write a message such as:
\begin{center}
|This is the part \childdocname{} of \childdocjob{}.|
\end{center}

%%%%%%%%%%%%%%%%%%%%%%%%%%%%%%%%%%%%%%%%%%%%%%%%%%%%%%%%%%%%%%%%%%%%%%%%%%%%%%%%
\subsection{Flags}
\label{sec:flags}

The package makes it easy to generate different versions
of the main or child documents.
To this end compilation flags can be defined
and assigned different default values.
They will be particularly useful in conjunction
with the forwarding mechanism described in \secref{sec:forward}.

For example, it may be useful to have a flag |\version|
which can be set to |draft| or |final|.
The document source will contain some conditional code
depending on the value of |\version|.
Suppose further, the flag should default to |final| for the main file
and to |draft| for child files
which is a natural assignment for editing the document.
This is achieved by placing the following code
in the preamble of the main document
(below the |\childdocmain| directive):
%
\begin{center}
\begin{tabular}{l}
|\ifchilddoc|\\
|\providecommand{\version}{draft}|\\
|\||else|\\
|\providecommand{\version}{final}|\\
|\||fi|
\end{tabular}
\end{center}
%
The definition by |\providecommand| makes sure
that previous definitions are not overwritten.
Further statements |\providecommand{\version}{...}|
can thus be added before the above code to override it.

For the main file, one might add a line
(between |\childdocmain| and the above block)
%
\begin{center}
|%\ifchilddoc\||else\providecommand{\version}{draft}\||fi|
\end{center}
%
which can be uncommented to produce a draft version.
Likewise one can add a line to the very top of a child file
(above the |\childdocof{|\textit{main}|}| directive)
%
\begin{center}
|%\providecommand{\version}{final}|
\end{center}
%
which can be uncommented to produce the final version of this child document.

%%%%%%%%%%%%%%%%%%%%%%%%%%%%%%%%%%%%%%%%%%%%%%%%%%%%%%%%%%%%%%%%%%%%%%%%%%%%%%%%
\subsection{Forwarding}
\label{sec:forward}

Different versions of the main or child documents
using compilation flags as described in \secref{sec:flags}
can be (permanently) stored in different files
for convenient compilation, viewing and distribution.
To this end, the package defines a command
to pass on compilation to a different file:

%%%%%%%%%%%%%%%%%%%%%%%%%%%%%%%%%%%%%%%%
\DescribeMacro{\childdocforward}
The command |\childdocforward| redirects processing to
another source file:
%
\begin{center}
\begin{tabular}{l}
|\input{childdoc.def}|\\
|\childdocforward[|\textit{main}|]{|\textit{dest}|}|\\
\end{tabular}
\end{center}
%
The argument \textit{dest} is the destination file
(without extension).
It should be the main file or one of the child files.
Note that further \textsf{childdoc} directives
such as |\childdocof| and |\childdocforward|
in the indicated file will be processed in this form.
The optional argument \textit{main}
passes on directly to the main file \textit{main}
while pretending to compile the child \textit{dest}.
This form behaves as if \textit{dest}
issues |\childdocof{|\textit{main}|}| right away,
and no further \textsf{childdoc} directives will be processed.

%%%%%%%%%%%%%%%%%%%%%%%%%%%%%%%%%%%%%%%%
\DescribeMacro{\...prefix}
In the alternative form |\childdocforwardprefix|,
%
\begin{center}
\begin{tabular}{l}
|\input{childdoc.def}|\\
|\childdocforwardprefix[|\textit{main}|]{|\textit{prefix}|}{|\textit{dest}|}|
\end{tabular}
\end{center}
%
the destination file is determined by a pattern
depending on the current file:
To make this work, the current file must be called
`{\textit{prefix}\hspace{0.2em}\textit{suffix}}'
with \textit{prefix} matching precisely the argument.
Processing is then passed on to the file
`{\textit{dest}\hspace{0.2em}\textit{suffix}}'.
Surely, the same effect is achieved by
directly specifying the
argument `{\textit{dest}\hspace{0.2em}\textit{suffix}}'
in the first form.
However, that requires to set up a different file
for each child. With the alternative form of the command
all these files can have exactly the same content
which simplifies setting them up and maintaining them.

For example, the following file |draft.tex|
with a compilation flag |\version| as described in \secref{sec:flags}
compiles the main document as a draft:
%
\begin{center}
\begin{tabular}{l}
|\def\version{draft}|\\
|\input{childdoc.def}|\\
|\childdocforward{|\textit{main}|}|
\end{tabular}
\end{center}
%
Likewise, the following files |final|\textit{nn}|.tex|
compile the final version of the child document
|child|\textit{nn}|.tex|:
%
\begin{center}
\begin{tabular}{l}
|\def\version{final}|\\
|\input{childdoc.def}|\\
|\childdocforwardprefix{final}{child}|
\end{tabular}
\end{center}
%

Note that when several versions of a main file and/or of each child file
are to be generated, it may be convenient to set up a |Makefile| or
shell script to automatise the process.

%%%%%%%%%%%%%%%%%%%%%%%%%%%%%%%%%%%%%%%%%%%%%%%%%%%%%%%%%%%%%%%%%%%%%%%%%%%%%%%%
\subsection{Command Line Processing}
\label{sec:commandline}

The effect of redirection files can also be achieved by invoking
the \LaTeX{} compiler with a more elaborate command line.
Most conveniently this should be done as part
of a shell script or a |Makefile|.

When using \textsf{childdoc} in the main file, the following
command lines effectively perform a redirection
(note that depending on the shell being used,
backslashes may have to be doubled: `|\|' $\to$ `|\\|'):
%
\begin{center}
|... -jobname "|\textit{target}|" |\\|"|[\textit{flags}]%
|\input{childdoc.def}\childdocforward[|\textit{main}|]{|\textit{dest}|}"|
\end{center}
%
Here \textit{target} is the name of the output file,
\textit{main} is the name of the main file
and \textit{dest} is the name of the main or child file to be processed
(all filenames without extensions).
The optional argument \textit{main} can be omitted
if \textit{main} matches \textit{dest}.
Optionally, compilation \textit{flags} can be defined via |\def| commands.
This command line makes the \TeX{} engine believe
it is compiling the file \textit{target}
whose content is specified as the latter parameter.
The provided code then forwards the processing to
\textit{main} or \textit{dest} as described in \secref{sec:forward}.

%%%%%%%%%%%%%%%%%%%%%%%%%%%%%%%%%%%%%%%%%%%%%%%%%%%%%%%%%%%%%%%%%%%%%%%%%%%%%%%%
\subsection{Include by Input}
\label{sec:input}

Including child documents by |\include| has some restrictions by design.
Most notably, the content of a child document always occupies
its own set of pages; pages cannot be shared between child documents.
Usually, this behaviour makes perfect sense
because each child document contain an essential part of the document.
However, in some situations it may be desirable to compose
a document from a collection of parts
without having mandatory page breaks between then.
For this case, the package
provides a mechanism to include parts
by |\input| which can also be processed individually.
However, by construction this mechanism
requires manual handling of the content to be output.

%%%%%%%%%%%%%%%%%%%%%%%%%%%%%%%%%%%%%%%%
\DescribeMacro{\ifchilddocmanual}
The main file should be prepared as usual, see \secref{sec:include}.
However, the document body must make a distinction
between processing of an individual part and of the main document, e.g.:
%
\begin{center}
\begin{tabular}{l}
|\ifchilddocmanual|\\
|\input{\childdocname}|\\
|\||else|\\
\textit{document body with }|\input{|\textit{part}|}|\\
|\||fi|
\end{tabular}
\end{center}
%
The conditional |\ifchilddocmanual| is true whenever
a part to be included by |\input| is being compiled,
and the name of the part is stored in |\childdocname|.

%%%%%%%%%%%%%%%%%%%%%%%%%%%%%%%%%%%%%%%%
\DescribeMacro{\childdocby}
Each part to be included by |\input| should start with:
%
\begin{center}
\begin{tabular}{l}
|\input{childdoc.def}|\\
|\childdocby{|\textit{main}|}|\\
\end{tabular}
\end{center}
%
The directive |\childdocby| is similar to |\childdocof|
described in \secref{sec:include},
but the subsequent selection of content must be done manually.
To that end, both |\ifchilddoc| and |\ifchilddocmanual|
will be true upon processing of a part,
and the name of the part is stored in |\childdocname|.
Note that |\jobname| will be set to the filename of the current part
so that each part receives an individual |.aux| file
that does not interfere with the |.aux| file(s) of the main document.
This behaviour can be altered by the alternative form
|\childdocby[*]{|\textit{main}|}| (with a non-empty optional argument)
which uses the |.aux| file of the main document
by setting |\jobname| to \textit{main}.

%%%%%%%%%%%%%%%%%%%%%%%%%%%%%%%%%%%%%%%%%%%%%%%%%%%%%%%%%%%%%%%%%%%%%%%%%%%%%%%%
\subsection{Driver Development}
\label{sec:driver}

The \textsf{childdoc} mechanism can also be use for the development
of definition files such as \LaTeX{} styles or classes.
This case differs from the above setup with multiple parts
included by |\include| in that no |\includeonly| should be invoked.
This can be achieved by starting the include file
(before |\ProvidesPackage|) with:
%
\begin{center}
\begin{tabular}{l}
|\input{childdoc.def}|\\
|\childdocforward{|\textit{main}|}|\\
\end{tabular}
\end{center}
%
or alternatively with:
%
\begin{center}
\begin{tabular}{l}
|\input{childdoc.def}|\\
|\childdocby{|\textit{main}|}|\\
\end{tabular}
\end{center}
%
Both forms have slightly different effects as described above.
The main file is prepared as usual, see \secref{sec:include}.

%%%%%%%%%%%%%%%%%%%%%%%%%%%%%%%%%%%%%%%%%%%%%%%%%%%%%%%%%%%%%%%%%%%%%%%%%%%%%%%%
\subsection{Legacy Detection}
\label{sec:detection}

The directive |\childdocmain| in the main file can detect
whether the complete document or merely a child is to be compiled
even without using the directive |\childdocof|.
This method is deprecated because it is less robust
and there is no compelling reason to use it;
it is merely provided for backward compatibility
and it may be removed in future versions.

If the detection mechanism is to be used,
it is mandatory to correctly specify
the filename of the main file as the argument of |\childdocmain|:
%
\begin{center}
\begin{tabular}{l}
|\input{childdoc.def}|\\
|\childdocmain{|\textit{main}|}|\\
\end{tabular}
\end{center}
%
If |\jobname| does not match the argument \textit{main} of |\childdocmain|,
it is assumed that |\jobname| points to the child file to be compiled.
When using |\childdocmain| with the main file specified as argument,
it suffices to start a child file
with just |\input{|\textit{main}|}|
without loading of the package and using |\childdocof|.
If instead all processing is done
with the appropriate \textsf{childdoc} directives,
the argument of \textit{main} of |\childdocmain| can be empty.

An alternative version of the command line processing described
in \secref{sec:commandline} using the detection mechanism reads:
%
\begin{center}
|... -jobname "|\textit{target}|" "|[\textit{flags}]%
[|\def\jobname{|\textit{dest}|}|]|\input{|\textit{main}|}"|
\end{center}

%%%%%%%%%%%%%%%%%%%%%%%%%%%%%%%%%%%%%%%%%%%%%%%%%%%%%%%%%%%%%%%%%%%%%%%%%%%%%%%%
\subsection{Manual Code}
\label{sec:manual}

In case one cannot be certain whether the definitions file |childdoc.def|
is installed on the target \TeX{} distribution
and one prefers not to ship it,
it is conceivable to paste a few relevant commands into the sources.

To that end, drop all statements |\input{childdoc.def}|
and perform the replacements as outlined below.
Instead of |\childdocmain{|\textit{main}|}| add the following code
to the top of the main file:
%
\begin{center}
\begin{tabular}{l}
|\||ifdefined\childdocname\endinput\||fi\newif\ifchilddoc|\\
|\edef\childdocname{\scantokens\expandafter{\jobname\noexpand}}|\\
|\def\childdocmain{|\textit{main}|}\||ifx\childdocmain\childdocname\||else|\\
|\childdoctrue\includeonly{\childdocname}\let\jobname\childdocmain\||fi|\\
\end{tabular}
\end{center}
%
Instead of |\childdocof{|\textit{main}|}| just include the main file
at the top of each child file:
%
\begin{center}
|\input{|\textit{main}|}|
\end{center}
%
A simple redirection |\childdocforward{|\textit{dest}|}| is achieved by:
%
\begin{center}
|\def\jobname{|\textit{dest}|}\input{\jobname}|
\end{center}
%
The redirection with prefix
|\childdocforwardprefix[|\textit{prefix}|]{|\textit{dest}|}|
is accomplished by:
%
\begin{center}
\begin{tabular}{l}
|{\edef\jobname{\scantokens\expandafter{\jobname\noexpand}}|\\
|\def\redirectjob |\textit{prefix}|#1~~~{\gdef\jobname{|\textit{dest}|#1}}|\\
|\expandafter\redirectjob\jobname~~~}\input{\jobname}|
\end{tabular}
\end{center}

In an alternative approach,
child documents can be compiled by a specific command line
without additional code or specific definitions:
%
\begin{center}
|... -jobname "|\textit{target}|" "|[\textit{flags}]%
|\includeonly{|\textit{dest}|}\input{|\textit{main}|}"|
\end{center}
%

%%%%%%%%%%%%%%%%%%%%%%%%%%%%%%%%%%%%%%%%%%%%%%%%%%%%%%%%%%%%%%%%%%%%%%%%%%%%%%%%
%%%%%%%%%%%%%%%%%%%%%%%%%%%%%%%%%%%%%%%%%%%%%%%%%%%%%%%%%%%%%%%%%%%%%%%%%%%%%%%%
\section{Information}

%%%%%%%%%%%%%%%%%%%%%%%%%%%%%%%%%%%%%%%%%%%%%%%%%%%%%%%%%%%%%%%%%%%%%%%%%%%%%%%%
\subsection{Copyright}

Copyright \copyright{} 2017--2018 Niklas Beisert

This work may be distributed and/or modified under the
conditions of the \LaTeX{} Project Public License, either version 1.3
of this license or (at your option) any later version.
The latest version of this license is in
  \url{http://www.latex-project.org/lppl.txt}
and version 1.3 or later is part of all distributions of \LaTeX{}
version 2005/12/01 or later.

This work has the LPPL maintenance status `maintained'.

The Current Maintainer of this work is Niklas Beisert.

This work consists of the files |README.txt|, |childdoc.ins| and |childdoc.dtx|
as well as the derived files |childdoc.def|, |cdocsamp.tex|
with |cdocsch1.tex|, |cdocsch2.tex|, |cdocspt3.tex|, |cdocspt4.tex|,
|cdocsdrf.tex|, |cdocsfn1.tex|, |cdocsfn2.tex|
as well as |childdoc.pdf|.

%%%%%%%%%%%%%%%%%%%%%%%%%%%%%%%%%%%%%%%%%%%%%%%%%%%%%%%%%%%%%%%%%%%%%%%%%%%%%%%%
\subsection{Files and Installation}

The package consists of the files:
%
\begin{center}
\begin{tabular}{ll}
    |README.txt|   & readme file \\
    |childdoc.ins| & installation file \\
    |childdoc.dtx| & source file \\
    |childdoc.def| & definition file \\
    |cdocsamp.tex| & sample main file \\
    |cdocsch1.tex| & sample include file \\
    |cdocsch2.tex| & sample include file \\
    |cdocspt3.tex| & sample part file \\
    |cdocspt4.tex| & sample part file \\
    |cdocsdrf.tex| & sample redirection file \\
    |cdocsfn1.tex| & sample redirection file \\
    |cdocsfn2.tex| & sample redirection file \\
    |childdoc.pdf| & manual
\end{tabular}
\end{center}
%
The distribution consists of the files
|README.txt|, |childdoc.ins| and |childdoc.dtx|.
%
\begin{itemize}
\item
Run (pdf)\LaTeX{} on |childdoc.dtx|
to compile the manual |childdoc.pdf| (this file).
\item
Run \LaTeX{} on |childdoc.ins| to create the definitions file |childdoc.def|
and the sample |cdocsamp.tex| with include files
|cdocsch1.tex|, |cdocsch2.tex|, |cdocspt3.tex|, |cdocspt4.tex|,
|cdocsdrf.tex|, |cdocsfn1.tex|, |cdocsfn2.tex|.
Then copy the file |childdoc.def| to an appropriate directory of your \LaTeX{}
distribution, e.g.\ \textit{texmf-root}|/tex/latex/childdoc|.
\end{itemize}

%%%%%%%%%%%%%%%%%%%%%%%%%%%%%%%%%%%%%%%%%%%%%%%%%%%%%%%%%%%%%%%%%%%%%%%%%%%%%%%%
\subsection{Related CTAN Packages}

There are several other packages which offer a similar functionality:
%
\begin{itemize}
\item
The packages
\href{http://ctan.org/pkg/docmute}{\textsf{docmute}},
\href{http://ctan.org/pkg/includex}{\textsf{includex}} and
\href{http://ctan.org/pkg/standalone}{\textsf{standalone}}
provide commands to include only the document body of
a child file thus allowing both files to be compiled individually.
\item
The packages \href{http://ctan.org/pkg/subdocs}{\textsf{subdocs}}
and \href{http://ctan.org/pkg/subfiles}{\textsf{subfiles}}
provide structures in which the main and child documents can be
encapsulated and allowing them to be compiled individually.
The inclusion mechanism is different from the conventional |\include|.
\item
The package \href{http://ctan.org/pkg/combine}{\textsf{combine}}
is an elaborate solution to combine several documents into one.
\end{itemize}
%
See also the CTAN topic \href{http://ctan.org/topic/subdocs}{\textsf{subdocs}}
for further related packages.
The present package differs from the above solutions in that
a document structure constructed with the conventional |\include| mechanism
just needs two extra commands at the top of every file
such that all constituent files can be compiled individually.

%%%%%%%%%%%%%%%%%%%%%%%%%%%%%%%%%%%%%%%%%%%%%%%%%%%%%%%%%%%%%%%%%%%%%%%%%%%%%%%%
%\subsection{Feature Suggestions}
%
%The following is a list of features which may be useful for future
%versions of this package:
%%
%\begin{itemize}
%\item
%\ldots
%\end{itemize}

%%%%%%%%%%%%%%%%%%%%%%%%%%%%%%%%%%%%%%%%%%%%%%%%%%%%%%%%%%%%%%%%%%%%%%%%%%%%%%%%
\subsection{Revision History}

%%%%%%%%%%%%%%%%%%%%%%%%%%%%%%%%%%%%%%%%
\paragraph{v2.0:} 2018/12/30

\begin{itemize}
\item
immediate forward processing
\item
added |\childdocby| mechanism
\item
manual restructured
\end{itemize}

%%%%%%%%%%%%%%%%%%%%%%%%%%%%%%%%%%%%%%%%
\paragraph{v1.6:} 2018/01/17

\begin{itemize}
\item
application for development of include files
\item
corrections to manual
\end{itemize}

%%%%%%%%%%%%%%%%%%%%%%%%%%%%%%%%%%%%%%%%
\paragraph{v1.5:} 2017/05/21

\begin{itemize}
\item
more complete structuring introduced
\item
|\childdocof| introduced
\item
|\childdoc| renamed to |\childdocmain|
\item
|\childredirect| renamed to |\childdocforward| and |\childdocforwardprefix|
and functionality expanded
\end{itemize}

%%%%%%%%%%%%%%%%%%%%%%%%%%%%%%%%%%%%%%%%
\paragraph{v1.0:} 2017/04/27

\begin{itemize}
\item
manual and install package
\item
first version published on CTAN
\end{itemize}

%%%%%%%%%%%%%%%%%%%%%%%%%%%%%%%%%%%%%%%%
\paragraph{v0.6:} 2017/04/26

\begin{itemize}
\item
redirection mechanism added
\end{itemize}

%%%%%%%%%%%%%%%%%%%%%%%%%%%%%%%%%%%%%%%%
\paragraph{v0.5:} 2017/04/26

\begin{itemize}
\item
functionality in definition file
\end{itemize}


%%%%%%%%%%%%%%%%%%%%%%%%%%%%%%%%%%%%%%%%%%%%%%%%%%%%%%%%%%%%%%%%%%%%%%%%%%%%%%%%
%%%%%%%%%%%%%%%%%%%%%%%%%%%%%%%%%%%%%%%%%%%%%%%%%%%%%%%%%%%%%%%%%%%%%%%%%%%%%%%%
%%%%%%%%%%%%%%%%%%%%%%%%%%%%%%%%%%%%%%%%%%%%%%%%%%%%%%%%%%%%%%%%%%%%%%%%%%%%%%%%
\appendix

\settowidth\MacroIndent{\rmfamily\scriptsize 000\ }

 \DocInput{childdoc.dtx}

\end{document}
%</driver>
% \fi
%
% %%%%%%%%%%%%%%%%%%%%%%%%%%%%%%%%%%%%%%%%%%%%%%%%%%%%%%%%%%%%%%%%%%%%%%%%%%%%%%
% %%%%%%%%%%%%%%%%%%%%%%%%%%%%%%%%%%%%%%%%%%%%%%%%%%%%%%%%%%%%%%%%%%%%%%%%%%%%%%
% \section{Sample}
%\iffalse
%<*samplemain>
%\fi
%
% The following presents a sample document
% with two chapters, two parts, a title page,
% a compile flag as well as three forwarding files to set the flag.
% It consists of eight |.tex| files:
% \begin{center}
% \begin{tabular}{ll}
% |cdocsamp.tex|&main file\\
% |cdocsch1.tex|&include file for chapter 1\\
% |cdocsch2.tex|&include file for chapter 2\\
% |cdocspt3.tex|&include file for part 3\\
% |cdocspt4.tex|&include file for part 4\\
% |cdocsdrf.tex|&forwarding file for main file in draft mode\\
% |cdocsfi1.tex|&forwarding file for final version of chapter 1\\
% |cdocsfi2.tex|&forwarding file for final version of chapter 2\\
% \end{tabular}
% \end{center}
% Each of the eight files can be compiled directly by the \LaTeX{} compiler.
%
% %%%%%%%%%%%%%%%%%%%%%%%%%%%%%%%%%%%%%%
% \paragraph{Main File.}
%
% The main file is called |cdocsamp.tex|.
%
% Load the \textsf{childdoc} definitions and
% declare the filename for the main document:
%    \begin{macrocode}
\input{childdoc.def}
\childdocmain{}
%    \end{macrocode}

% Optional override for |\version| flag:
%    \begin{macrocode}
%%\ifchilddoc\else\providecommand{\version}{draft}\fi
%    \end{macrocode}

% Define the default values for the |\version| flag
% (|final| for the main file and |draft| for childs):
%    \begin{macrocode}
\ifchilddoc
\providecommand{\version}{draft}
\else
\providecommand{\version}{final}
\fi
%    \end{macrocode}

% Load the standard document class:
%    \begin{macrocode}
\documentclass[12pt]{article}
%    \end{macrocode}

% Start the document body:
%    \begin{macrocode}
\begin{document}
%    \end{macrocode}

% Declare a title page.
% Print title, part of document being processed and version flag:
%    \begin{macrocode}
\addtocounter{page}{-1}
\begin{center}
{\LARGE\bfseries{}childdoc example\par}
\vspace{1cm}
\ifchilddoc
\ifchilddocmanual part\else chapter\fi:
`\childdocname' of `\childdocjob'\par
\else
main document: `\childdocjob'\par
\fi
version: \version\par
\end{center}
\newpage
%    \end{macrocode}

% Manually include selected file,
% otherwise process as usual:
%    \begin{macrocode}
\ifchilddocmanual
\section*{part `\childdocname'}
\input{\childdocname}
\else
%    \end{macrocode}

% Include the two chapters:
%    \begin{macrocode}
\include{cdocsch1}
\include{cdocsch2}
%    \end{macrocode}

% Include the two parts unless only chapters should be displayed:
%    \begin{macrocode}
\ifchilddoc\else
\section{part three}
\input{cdocspt3}
\section{part four}
\input{cdocspt4}
\fi
%    \end{macrocode}

% Process as usual until here:
%    \begin{macrocode}
\fi
%    \end{macrocode}

% End of document body:
%    \begin{macrocode}
\end{document}
%    \end{macrocode}
%\iffalse
%</samplemain>
%\fi
%
% %%%%%%%%%%%%%%%%%%%%%%%%%%%%%%%%%%%%%%
% \paragraph{Chapter Include Files.}
%
% The include files are called |cdocsch1.tex| and |cdocsch2.tex|.
%
%\iffalse
%<*samplechap1|samplechap2>
%\fi

% Optional override for |\version| flag:
%    \begin{macrocode}
%%\providecommand{\version}{final}
%    \end{macrocode}

% Include the main document:
%    \begin{macrocode}
\input{childdoc.def}
\childdocof{cdocsamp}
%    \end{macrocode}

%\iffalse
%</samplechap1|samplechap2>
%\fi
%
%\iffalse
%<*samplechap1>
%\fi
% Some text for chapter 1:
%    \begin{macrocode}
\section{one}
some text in chapter one
%    \end{macrocode}

%\iffalse
%</samplechap1>
%\fi
% Some text for chapter 2:
%\iffalse
%<*samplechap2>
%\fi
%    \begin{macrocode}
\section{two}
more text in chapter two
%    \end{macrocode}

%\iffalse
%</samplechap2>
%\fi
%
% %%%%%%%%%%%%%%%%%%%%%%%%%%%%%%%%%%%%%%
% \paragraph{Part Include Files.}
%
% The include files are called |cdocspt3.tex| and |cdocspt4.tex|.
%
%\iffalse
%<*samplepart3|samplepart4>
%\fi

% Optional override for |\version| flag:
%    \begin{macrocode}
%%\providecommand{\version}{final}
%    \end{macrocode}

% Include the main document:
%    \begin{macrocode}
\input{childdoc.def}
\childdocby{cdocsamp}
%    \end{macrocode}

%\iffalse
%</samplepart3|samplepart4>
%\fi
%
%\iffalse
%<*samplepart3>
%\fi
% Some text for part 3:
%    \begin{macrocode}
some text in part three
%    \end{macrocode}

%\iffalse
%</samplepart3>
%\fi
% Some text for part 4:
%\iffalse
%<*samplepart4>
%\fi
%    \begin{macrocode}
more text in part four
%    \end{macrocode}

%\iffalse
%</samplepart4>
%\fi
%
% %%%%%%%%%%%%%%%%%%%%%%%%%%%%%%%%%%%%%%
% \paragraph{Forwarding for a Complete Draft.}
%
% The following forwarding file |cdocsdrf.tex|
% compiles the main document in draft mode:
%\iffalse
%<*sampledraft>
%\fi
%    \begin{macrocode}
\def\version{draft}
\input{childdoc.def}
\childdocforward{cdocsamp}
%    \end{macrocode}

%\iffalse
%</sampledraft>
%\fi
%
% %%%%%%%%%%%%%%%%%%%%%%%%%%%%%%%%%%%%%%
% \paragraph{Forwarding for Final Version of the Chapters.}
%
% The following forwarding files |cdocsfn1.tex| and |cdocsfn2.tex|
% (with identical content)
% compile the final versions of the child documents
% |cdocsch1.tex| and |cdocsch2.tex|, respectively:
%\iffalse
%<*samplefinal>
%\fi
%    \begin{macrocode}
\def\version{final}
\input{childdoc.def}
\childdocforwardprefix[cdocsamp]{cdocsfn}{cdocsch}
%    \end{macrocode}

%\iffalse
%</samplefinal>
%\fi
%
% %%%%%%%%%%%%%%%%%%%%%%%%%%%%%%%%%%%%%%
% \paragraph{Command Line Processing.}
%
% The following three command lines generate the output files
% |cdocscld|, |cdocscl1| and |cdocscl2|
% which should be identical to
% |cdocsdrf|, |cdocsch1| and |cdocsfn2|, respectively:
% \begin{center}
% \begin{tabular}{l}
% |latex -jobname cdocscld \|\\
% |  "\def\version{draft}\input{childdoc.def}\childdocforward{cdocsamp}"|\\
% |latex -jobname cdocscl1 \|\\
% |  "\input{childdoc.def}\childdocforward[cdocsamp]{cdocsch1}"|\\
% |latex -jobname cdocscl2 \|\\
% |  "\def\version{final}\input{childdoc.def}\childdocforward{cdocsch2}"|
% \end{tabular}
% \end{center}
% Note that the trailing backslash on each first line
% merely continues the input to the second line
% (for convenient cut ant paste).
% Furthermore, the command |latex| can be replaced by any
% of its alternative versions such as |pdflatex|.
%
% %%%%%%%%%%%%%%%%%%%%%%%%%%%%%%%%%%%%%%%%%%%%%%%%%%%%%%%%%%%%%%%%%%%%%%%%%%%%%%
% %%%%%%%%%%%%%%%%%%%%%%%%%%%%%%%%%%%%%%%%%%%%%%%%%%%%%%%%%%%%%%%%%%%%%%%%%%%%%%
% \section{Implementation}
%\iffalse
%<*package>
%\fi
%
% This section describes the definitions file |childdoc.def|.

% The definitions cannot be loaded using |\usepackage| or |\RequirePackage|
% which has a mechanism to prevent loading a style file more than once.
% When loading the definitions by means of |\input|
% multiple instances have to be prevented manually:
%\iffalse
%This code needs to be before the `\ProvidesFile' directive
%which is defined at the beginning of this file.
%Therefore it is also placed there and commented out here.
%</package>
%<*discard>
%\fi
%    \begin{macrocode}
\ifdefined\childdocmain\endinput\fi
%    \end{macrocode}
%\iffalse
%</discard>
%<*package>
%\fi
%
% \macro{\ifchilddoc}
% \macro{\ifchilddocmanual}
% The conditional |\ifchilddoc| tells whether a
% child (true) or main (false) document is being compiled.
% The conditional |\ifchilddocmanual| tells whether
% the |\includeonly| mechanism is used (false) or
% the selection of child files must be performed manually (true).
% The definitions initialise to false:
%    \begin{macrocode}
\newif\ifchilddoc
\newif\ifchilddocmanual
%    \end{macrocode}

% \macro{\childdocname}
% \macro{\childdocjob}
% The macro |\childdocname| stores the name of the main document
% to be compiled. The macro |\childdocjob| stores the name of
% the document on which the \LaTeX{} compiler was originally invoked.
% The content of |\jobname| cannot be compared
% to filenames specified in the source due to different catcodes.
% The following code rescans |\jobname|, stores the result
% in |\childdocname| and saves a copy in |\childdocjob|:
%    \begin{macrocode}
\edef\childdocname{\scantokens\expandafter{\jobname\noexpand}}
\let\childdocjob\childdocname
%    \end{macrocode}

% \macro{\childdocdisable}
% The macro |\childdocdisable| prevents the main file
% from being processed more than once.
% At this stage, the main document command |\childdocmain|
% is assumed to be called once again where it should do nothing.
% Any subsequent call to it should prevent
% a secondary processing of the main document
% It overwrites the forwarding commands
% |\childdocof| and |\childdocforward|
% with empty macros to prevent further inclusions of the main document:
%    \begin{macrocode}
\newcommand{\childdocdisable}
{
  \renewcommand{\childdocmain}[1]{\renewcommand{\childdocmain}[1]{\endinput}}
  \renewcommand{\childdocof}[1]{}
  \renewcommand{\childdocby}[2][]{}
  \renewcommand{\childdocforward}[2][]{}
  \renewcommand{\childdocdisable}{}
}
%    \end{macrocode}

% \macro{\childdocmain}
% The macro |\childdocmain| is to be called at the top of the main file
% with nothing or the main filename (without extension) as argument.
% First, it breaks loops.
% If the argument is not empty and does not match |\childdocname|
% (which is set by the first inclusion of |childdoc.def|),
% |\ifchilddoc| is set to true, |\includeonly| is applied to the child file
% and |\jobname| is set to the main file
% (for proper handling of |.aux| files):
%    \begin{macrocode}
\newcommand{\childdocmain}[1]
{
  \childdocdisable\childdocmain{}
  \if?#1?\else
    \begingroup
      \def\childdoctmp{#1}
      \ifx\childdoctmp\childdocname
        \def\childdoctmp{}
      \else
        \def\childdoctmp
        {
          \childdoctrue
          \includeonly{\childdocname}
          \def\childdocjob{#1}
          \def\jobname{#1}
        }
      \fi
      \expandafter
    \endgroup
    \childdoctmp
  \fi
}
%    \end{macrocode}

% \macro{\childdocof}
% The command |\childdocof| redirects
% compilation to the main file |#1|.
%    \begin{macrocode}
\newcommand{\childdocof}[1]
{
  \childdocdisable
  \childdoctrue
  \includeonly{\childdocname}
  \def\jobname{#1}
  \def\childdocjob{#1}
  \input{#1}
}
%    \end{macrocode}

% \macro{\childdocby}
% The command |\childdocby| ....
%    \begin{macrocode}
\newcommand{\childdocby}[2][]
{
  \childdocdisable
  \childdoctrue
  \childdocmanualtrue
  \if?#1?\else
    \def\jobname{#2}
  \fi
  \def\childdocjob{#2}
  \input{#2}
  \endinput
}
%    \end{macrocode}

% \macro{\childdocforward}
% The command |\childdocforward| redirects
% compilation to the main file or
% (if the optional argument is given) a child file.
% Parameters are set as if the main file
% or a child file starting with |\childdocof| was compiled.
% Then compilation is handed over to the main file:
%    \begin{macrocode}
\newcommand{\childdocforward}[2][]
{
  \begingroup
    \if?#1?
      \def\childdoctmp
      {
        \def\childdocname{#2}
        \def\childdocjob{#2}
        \def\jobname{#2}
        \input{#2}
        \endinput
      }
    \else
      \def\childdoctmp
      {
        \childdocdisable
        \def\childdocname{#2}
        \childdoctrue
        \includeonly{#2}
        \def\childdocjob{#1}
        \def\jobname{#1}
        \input{#1}
        \endinput
      }
    \fi
    \expandafter
  \endgroup
  \childdoctmp
}
%    \end{macrocode}

% \macro{\childdocforwardprefix}
% The command |\childdocforwardprefix| redirects
% compilation to the main or a child file by means of a pattern.
% The prefix |#1| in the current filename is replaced by |#2|
% and the suffix of the current filename is kept
% (it is assumed that the filename does not contain the substring `|~~~|'
% which is used as a delimiter).
% Compilation is handed over to the new file by |\childdocforward|:
%    \begin{macrocode}
\newcommand{\childdocforwardprefix}[3][]
{
  \begingroup
    \def\childdocextract #2##1~~~{\def\childdoctmp{\childdocforward[#1]{#3##1}}}
    \expandafter\childdocextract\childdocname~~~
    \expandafter
  \endgroup
  \childdoctmp
}
%    \end{macrocode}

% \macro{\childdoc}
% The deprecated macro |\childdoc| is a legacy version of |\childdocmain|:
%    \begin{macrocode}
\newcommand{\childdoc}{\childdocmain}
%    \end{macrocode}

% \macro{\childdocredirect}
% The deprecated macro |\childdocredirect| is a legacy version
% of |\childdocforward| and |\childdocforwardprefix|:
%    \begin{macrocode}
\newcommand{\childdocredirect}[2][]
{
  \begingroup
    \if?#1?
      \def\childdoctmp{\childdocforward{#2}}
    \else
      \def\childdoctmp{\childdocforwardprefix{#1}{#2}}
    \fi
    \expandafter
  \endgroup
  \childdoctmp
}
%    \end{macrocode}

%\iffalse
%</package>
%\fi
%
\endinput

\childdocforward{cdocsamp}
%    \end{macrocode}

%\iffalse
%</sampledraft>
%\fi
%
% %%%%%%%%%%%%%%%%%%%%%%%%%%%%%%%%%%%%%%
% \paragraph{Forwarding for Final Version of the Chapters.}
%
% The following forwarding files |cdocsfn1.tex| and |cdocsfn2.tex|
% (with identical content)
% compile the final versions of the child documents
% |cdocsch1.tex| and |cdocsch2.tex|, respectively:
%\iffalse
%<*samplefinal>
%\fi
%    \begin{macrocode}
\def\version{final}
% \iffalse
%
% childdoc.dtx Copyright (C) 2017-2018 Niklas Beisert
%
% This work may be distributed and/or modified under the
% conditions of the LaTeX Project Public License, either version 1.3
% of this license or (at your option) any later version.
% The latest version of this license is in
%   http://www.latex-project.org/lppl.txt
% and version 1.3 or later is part of all distributions of LaTeX
% version 2005/12/01 or later.
%
% This work has the LPPL maintenance status `maintained'.
%
% The Current Maintainer of this work is Niklas Beisert.
%
% This work consists of the files childdoc.dtx and childdoc.ins
% and the derived files childdoc.def and cdocsamp.tex with
% cdocsch1.tex, cdocsch2.tex, cdocsdrf.tex, cdocsfn1.tex, cdocsfn2.tex.
%
%<package>\ifdefined\childdocmain\endinput\fi
%<package>\ProvidesFile{childdoc.def}[2018/12/30 v2.0 child document driver]
%<samplemain>\ProvidesFile{cdocsamp.tex}[2018/12/30 v2.0 sample for childdoc]
%<*driver>
%\ProvidesFile{childdoc.drv}[2018/12/30 v2.0 childdoc reference manual file]
\PassOptionsToClass{10pt,a4paper}{article}
\documentclass{ltxdoc}

\usepackage[margin=35mm]{geometry}
\usepackage{hyperref}
\usepackage{hyperxmp}
\usepackage[usenames]{color}

\hypersetup{colorlinks=true}
\hypersetup{pdfstartview=FitH}
\hypersetup{pdfpagemode=UseNone}
\hypersetup{pdfsource={}}
\hypersetup{pdflang={en-UK}}
\hypersetup{pdfcopyright={Copyright 2017-2018 Niklas Beisert.
  This work may be distributed and/or modified under the
  conditions of the LaTeX Project Public License, either version 1.3
  of this license or (at your option) any later version.}}
\hypersetup{pdflicenseurl={http://www.latex-project.org/lppl.txt}}
\hypersetup{pdfcontactaddress={ETH Zurich, ITP, HIT K,
  Wolfgang-Pauli-Strasse 27}}
\hypersetup{pdfcontactpostcode={8093}}
\hypersetup{pdfcontactcity={Zurich}}
\hypersetup{pdfcontactcountry={Switzerland}}
\hypersetup{pdfcontactemail={nbeisert@itp.phys.ethz.ch}}
\hypersetup{pdfcontacturl={http://people.phys.ethz.ch/\xmptilde nbeisert/}}

\newcommand{\secref}[1]{\hyperref[#1]{section \ref*{#1}}}

\parskip1ex
\parindent0pt
\let\olditemize\itemize
\def\itemize{\olditemize\parskip0pt}

\begin{document}

\title{The \textsf{childdoc} Package}
\hypersetup{pdftitle={The childdoc Package}}
\author{Niklas Beisert\\[2ex]
  Institut f\"ur Theoretische Physik\\
  Eidgen\"ossische Technische Hochschule Z\"urich\\
  Wolfgang-Pauli-Strasse 27, 8093 Z\"urich, Switzerland\\[1ex]
  \href{mailto:nbeisert@itp.phys.ethz.ch}
  {\texttt{nbeisert@itp.phys.ethz.ch}}}
\hypersetup{pdfauthor={Niklas Beisert}}
\hypersetup{pdfsubject={Manual for the LaTeX2e Package childdoc}}
\date{30 December 2018, \textsf{v2.0}}
\maketitle

\begin{abstract}\noindent
\textsf{childdoc} is a \LaTeXe{} package
that enables the direct compilation
of document sections included by |\include|
to individual files.
\end{abstract}

\begingroup
\parskip0ex
\tableofcontents
\endgroup

%%%%%%%%%%%%%%%%%%%%%%%%%%%%%%%%%%%%%%%%%%%%%%%%%%%%%%%%%%%%%%%%%%%%%%%%%%%%%%%%
%%%%%%%%%%%%%%%%%%%%%%%%%%%%%%%%%%%%%%%%%%%%%%%%%%%%%%%%%%%%%%%%%%%%%%%%%%%%%%%%
\section{Introduction}

\LaTeX{} provides a mechanism to structure a large document (such as a book)
into a main file and several child files (containing the chapters)
using the |\include| command.
This mechanism is beneficial for documents
which span hundreds of pages in order to
make the source file(s) more manageable.
Moreover, compilation can be restricted to
selected child files by means of the |\includeonly| command.
The latter feature can be used to reduce the compilation time while editing
(this was significantly more useful in the earlier days of \LaTeX{})
or to generate a smaller document which is easier to navigate.
Another application of |\includeonly| is to generate
documents consisting of selected parts of the complete document.

However, there are a few drawbacks of the plain |\include| mechanism:
\begin{itemize}
\item
The child files cannot be compiled on their own,
they can only be compiled via the main file.
A naive editing environment
(such as a text editor with an option
to have the current file processed by \LaTeX)
may require one to switch to the main file before compiling;
attempting to compile the child file produces errors.
\item
The main file must be modified (each time)
to adjust the |\includeonly| command
to the present needs. This easily leaves the main file in a messy state.
\item
The generated document will always carry the filename
of the main document. This is inconvenient if
several child files are to be compiled and
to be kept for distribution.
\end{itemize}

The present package provides a simple interface
to make child files individually compilable by \LaTeX{}.
Compiling a child file then has the same effect as compiling
the main file with an |\includeonly| command
to select the appropriate child.
Moreover the generated document will carry the name of the child
rather than the main file.
This resolves all three above issues.

This feature is meant to make the editing of books,
thesis documents and lecture notes somewhat more convenient.
However, the package can also be used efficiently for
composing a series of documents (such as exercise sheets)
which are typically distributed individually.
It then assists the author in generating the individual documents
(potentially in different versions)
as well as a document containing the collected series.
Another application is in developing style files
or other kinds of included material
where compilation of the style file could redirect
to a sample or test file.

%%%%%%%%%%%%%%%%%%%%%%%%%%%%%%%%%%%%%%%%%%%%%%%%%%%%%%%%%%%%%%%%%%%%%%%%%%%%%%%%
%%%%%%%%%%%%%%%%%%%%%%%%%%%%%%%%%%%%%%%%%%%%%%%%%%%%%%%%%%%%%%%%%%%%%%%%%%%%%%%%
\section{Usage}

First of all, the package \textsf{childdoc} is \emph{not} a standard
\LaTeXe{} |.sty| style file! Therefore it needs to be invoked in
a non-standard way.

%%%%%%%%%%%%%%%%%%%%%%%%%%%%%%%%%%%%%%%%%%%%%%%%%%%%%%%%%%%%%%%%%%%%%%%%%%%%%%%%
\subsection{Included Files}
\label{sec:include}

%%%%%%%%%%%%%%%%%%%%%%%%%%%%%%%%%%%%%%%%
\DescribeMacro{\childdocmain}
To use the package, add the commands
\begin{center}
\begin{tabular}{l}
|\input{childdoc.def}|\\
|\childdocmain{}|\\
\end{tabular}
\end{center}
at the very top of the main \LaTeX{} file,
in particular \emph{before} the |\documentclass| statement!
The argument of |\childdocmain| should be left empty
(but it must be present).

%%%%%%%%%%%%%%%%%%%%%%%%%%%%%%%%%%%%%%%%
\DescribeMacro{\childdocof}
Furthermore, add the commands
\begin{center}
\begin{tabular}{l}
|\input{childdoc.def}|\\
|\childdocof{|\textit{main}|}|\\
\end{tabular}
\end{center}
at the top of every child file \textit{child}
which is included by |\include{|\textit{child}|}|
from within the main file
(or at least for those files to be compiled individually).
The argument \textit{main} must be the filename of the main file.

There are a couple of
considerations in setting up the main and child documents:

%%%%%%%%%%%%%%%%%%%%%%%%%%%%%%%%%%%%%%%%
\paragraph{Restrictions.}

Please note the following restrictions:
\begin{itemize}
\item
|\childdocmain| must be called with one argument \textit{main}
to ensure compatibility with earlier version of the package.
It must either be empty (|\childdocmain{}|)
or precisely match the filename of the main file in which it is specified.
See \secref{sec:detection} for further information.
\item
The filename \textit{main} must be specified without the |.tex| extension.
\item
The filename \textit{main} is case sensitive
(even in case-insensitive file systems)
due to internal string comparison.
\item
The argument \textit{main} should be fully expanded, it cannot be a macro.
\item
Subdirectories and special characters should be avoided in filenames.
\item
The command |\childdocmain{|\textit{main}|}| must be followed by a whitespace.
It should not be followed immediately by another command
or by a comment mark `|%|'.
This is because the \TeX{} parser reads the token immediately following
the argument of |\childdocmain| and puts it
at the beginning of every child section;
however, a white\-space is ignored.
\end{itemize}

%%%%%%%%%%%%%%%%%%%%%%%%%%%%%%%%%%%%%%%%
\paragraph{Content of Main File.}

It is advisable to place all content in the child files included by |\include|.
Any output contained in the main file will appear in all child documents
unless suppressed manually;
it cannot be suppressed automatically by the |\includeonly| directive
and thus should normally be avoided.
A method to include some content in the main file
by means of conditional processing is described in \secref{sec:conditional}.

%%%%%%%%%%%%%%%%%%%%%%%%%%%%%%%%%%%%%%%%
\paragraph{Page Numbering.}

When only a part of the document is compiled,
the appropriate numbering of pages
(as well as other status parameters)
is determined from the |.aux| files.
The latter contain information from previous passes.
However this information needs to propagate through
all intermediate child documents.
Therefore the page numbering in child documents may well
be inconsistent until the complete document is compiled at least once.

A useful (if unconventional) way to always ensure a consistent
page numbering is to restart the numbering in each child document
and denote the pages by `\textit{child}|.|\textit{page}'
where \textit{child} represents the chapter/section number of the child file.
This can be achieved by the command
|\numberwithin{page}{|\textit{child}|}|
of the \textsf{amsmath} package
where \textit{child} can be |chapter| or |section|
depending on the chosen structuring.
Alternatively, one can modify the macro |\thepage| appropriately
and reset the counter |page| at the start of each child file.

%%%%%%%%%%%%%%%%%%%%%%%%%%%%%%%%%%%%%%%%%%%%%%%%%%%%%%%%%%%%%%%%%%%%%%%%%%%%%%%%
\subsection{Conditional Processing}
\label{sec:conditional}

The package provides a mechanism to compile different versions
of a document. To customise the versions further some conditional processing
can come in handy to distinguish which version is being compiled.
The package provides two macros to describe the compilation context:

%%%%%%%%%%%%%%%%%%%%%%%%%%%%%%%%%%%%%%%%
\DescribeMacro{\ifchilddoc}
The conditional |\ifchilddoc| distinguishes between the compilation of
child documents and the main document:
%
\begin{center}
|\ifchilddoc |\textit{child-code}| |[|\||else |\textit{main-code}]| \||fi|
\end{center}

%%%%%%%%%%%%%%%%%%%%%%%%%%%%%%%%%%%%%%%%
\DescribeMacro{\childdocname}
\DescribeMacro{\childdocjob}
The macro |\childdocname| contains the filename (without extension)
of the main or child file being processed.
Note that |\childdocjob| will always contain the name of the main file.

%%%%%%%%%%%%%%%%%%%%%%%%%%%%%%%%%%%%%%%%
\paragraph{Title Page.}

Conditional processing can be used to include a title or banner page
in the main document when proper precautions are taken.
Importantly, the code in the main file should ensure that the page counter
(as well as other status parameters which are stored in the |.aux| files)
takes the same value after the conditional processing.
Otherwise the page numbers may take divergent values
depending on which part is compiled.

For example, a title page could be declared by:
%
\begin{center}
\begin{tabular}{l}
|\ifchilddoc\||else|\\
|\addtocounter{page}{-1}|\\
\textit{code for title page}\\
|\newpage|\\
|\||fi|
\end{tabular}
\end{center}
%
A banner page for the child documents can be generated by:
%
\begin{center}
\begin{tabular}{l}
|\ifchilddoc|\\
|\addtocounter{page}{-1}|\\
\textit{code for banner page}\\
|\newpage|\\
|\||fi|
\end{tabular}
\end{center}
%
Here one could write a message such as:
\begin{center}
|This is the part \childdocname{} of \childdocjob{}.|
\end{center}

%%%%%%%%%%%%%%%%%%%%%%%%%%%%%%%%%%%%%%%%%%%%%%%%%%%%%%%%%%%%%%%%%%%%%%%%%%%%%%%%
\subsection{Flags}
\label{sec:flags}

The package makes it easy to generate different versions
of the main or child documents.
To this end compilation flags can be defined
and assigned different default values.
They will be particularly useful in conjunction
with the forwarding mechanism described in \secref{sec:forward}.

For example, it may be useful to have a flag |\version|
which can be set to |draft| or |final|.
The document source will contain some conditional code
depending on the value of |\version|.
Suppose further, the flag should default to |final| for the main file
and to |draft| for child files
which is a natural assignment for editing the document.
This is achieved by placing the following code
in the preamble of the main document
(below the |\childdocmain| directive):
%
\begin{center}
\begin{tabular}{l}
|\ifchilddoc|\\
|\providecommand{\version}{draft}|\\
|\||else|\\
|\providecommand{\version}{final}|\\
|\||fi|
\end{tabular}
\end{center}
%
The definition by |\providecommand| makes sure
that previous definitions are not overwritten.
Further statements |\providecommand{\version}{...}|
can thus be added before the above code to override it.

For the main file, one might add a line
(between |\childdocmain| and the above block)
%
\begin{center}
|%\ifchilddoc\||else\providecommand{\version}{draft}\||fi|
\end{center}
%
which can be uncommented to produce a draft version.
Likewise one can add a line to the very top of a child file
(above the |\childdocof{|\textit{main}|}| directive)
%
\begin{center}
|%\providecommand{\version}{final}|
\end{center}
%
which can be uncommented to produce the final version of this child document.

%%%%%%%%%%%%%%%%%%%%%%%%%%%%%%%%%%%%%%%%%%%%%%%%%%%%%%%%%%%%%%%%%%%%%%%%%%%%%%%%
\subsection{Forwarding}
\label{sec:forward}

Different versions of the main or child documents
using compilation flags as described in \secref{sec:flags}
can be (permanently) stored in different files
for convenient compilation, viewing and distribution.
To this end, the package defines a command
to pass on compilation to a different file:

%%%%%%%%%%%%%%%%%%%%%%%%%%%%%%%%%%%%%%%%
\DescribeMacro{\childdocforward}
The command |\childdocforward| redirects processing to
another source file:
%
\begin{center}
\begin{tabular}{l}
|\input{childdoc.def}|\\
|\childdocforward[|\textit{main}|]{|\textit{dest}|}|\\
\end{tabular}
\end{center}
%
The argument \textit{dest} is the destination file
(without extension).
It should be the main file or one of the child files.
Note that further \textsf{childdoc} directives
such as |\childdocof| and |\childdocforward|
in the indicated file will be processed in this form.
The optional argument \textit{main}
passes on directly to the main file \textit{main}
while pretending to compile the child \textit{dest}.
This form behaves as if \textit{dest}
issues |\childdocof{|\textit{main}|}| right away,
and no further \textsf{childdoc} directives will be processed.

%%%%%%%%%%%%%%%%%%%%%%%%%%%%%%%%%%%%%%%%
\DescribeMacro{\...prefix}
In the alternative form |\childdocforwardprefix|,
%
\begin{center}
\begin{tabular}{l}
|\input{childdoc.def}|\\
|\childdocforwardprefix[|\textit{main}|]{|\textit{prefix}|}{|\textit{dest}|}|
\end{tabular}
\end{center}
%
the destination file is determined by a pattern
depending on the current file:
To make this work, the current file must be called
`{\textit{prefix}\hspace{0.2em}\textit{suffix}}'
with \textit{prefix} matching precisely the argument.
Processing is then passed on to the file
`{\textit{dest}\hspace{0.2em}\textit{suffix}}'.
Surely, the same effect is achieved by
directly specifying the
argument `{\textit{dest}\hspace{0.2em}\textit{suffix}}'
in the first form.
However, that requires to set up a different file
for each child. With the alternative form of the command
all these files can have exactly the same content
which simplifies setting them up and maintaining them.

For example, the following file |draft.tex|
with a compilation flag |\version| as described in \secref{sec:flags}
compiles the main document as a draft:
%
\begin{center}
\begin{tabular}{l}
|\def\version{draft}|\\
|\input{childdoc.def}|\\
|\childdocforward{|\textit{main}|}|
\end{tabular}
\end{center}
%
Likewise, the following files |final|\textit{nn}|.tex|
compile the final version of the child document
|child|\textit{nn}|.tex|:
%
\begin{center}
\begin{tabular}{l}
|\def\version{final}|\\
|\input{childdoc.def}|\\
|\childdocforwardprefix{final}{child}|
\end{tabular}
\end{center}
%

Note that when several versions of a main file and/or of each child file
are to be generated, it may be convenient to set up a |Makefile| or
shell script to automatise the process.

%%%%%%%%%%%%%%%%%%%%%%%%%%%%%%%%%%%%%%%%%%%%%%%%%%%%%%%%%%%%%%%%%%%%%%%%%%%%%%%%
\subsection{Command Line Processing}
\label{sec:commandline}

The effect of redirection files can also be achieved by invoking
the \LaTeX{} compiler with a more elaborate command line.
Most conveniently this should be done as part
of a shell script or a |Makefile|.

When using \textsf{childdoc} in the main file, the following
command lines effectively perform a redirection
(note that depending on the shell being used,
backslashes may have to be doubled: `|\|' $\to$ `|\\|'):
%
\begin{center}
|... -jobname "|\textit{target}|" |\\|"|[\textit{flags}]%
|\input{childdoc.def}\childdocforward[|\textit{main}|]{|\textit{dest}|}"|
\end{center}
%
Here \textit{target} is the name of the output file,
\textit{main} is the name of the main file
and \textit{dest} is the name of the main or child file to be processed
(all filenames without extensions).
The optional argument \textit{main} can be omitted
if \textit{main} matches \textit{dest}.
Optionally, compilation \textit{flags} can be defined via |\def| commands.
This command line makes the \TeX{} engine believe
it is compiling the file \textit{target}
whose content is specified as the latter parameter.
The provided code then forwards the processing to
\textit{main} or \textit{dest} as described in \secref{sec:forward}.

%%%%%%%%%%%%%%%%%%%%%%%%%%%%%%%%%%%%%%%%%%%%%%%%%%%%%%%%%%%%%%%%%%%%%%%%%%%%%%%%
\subsection{Include by Input}
\label{sec:input}

Including child documents by |\include| has some restrictions by design.
Most notably, the content of a child document always occupies
its own set of pages; pages cannot be shared between child documents.
Usually, this behaviour makes perfect sense
because each child document contain an essential part of the document.
However, in some situations it may be desirable to compose
a document from a collection of parts
without having mandatory page breaks between then.
For this case, the package
provides a mechanism to include parts
by |\input| which can also be processed individually.
However, by construction this mechanism
requires manual handling of the content to be output.

%%%%%%%%%%%%%%%%%%%%%%%%%%%%%%%%%%%%%%%%
\DescribeMacro{\ifchilddocmanual}
The main file should be prepared as usual, see \secref{sec:include}.
However, the document body must make a distinction
between processing of an individual part and of the main document, e.g.:
%
\begin{center}
\begin{tabular}{l}
|\ifchilddocmanual|\\
|\input{\childdocname}|\\
|\||else|\\
\textit{document body with }|\input{|\textit{part}|}|\\
|\||fi|
\end{tabular}
\end{center}
%
The conditional |\ifchilddocmanual| is true whenever
a part to be included by |\input| is being compiled,
and the name of the part is stored in |\childdocname|.

%%%%%%%%%%%%%%%%%%%%%%%%%%%%%%%%%%%%%%%%
\DescribeMacro{\childdocby}
Each part to be included by |\input| should start with:
%
\begin{center}
\begin{tabular}{l}
|\input{childdoc.def}|\\
|\childdocby{|\textit{main}|}|\\
\end{tabular}
\end{center}
%
The directive |\childdocby| is similar to |\childdocof|
described in \secref{sec:include},
but the subsequent selection of content must be done manually.
To that end, both |\ifchilddoc| and |\ifchilddocmanual|
will be true upon processing of a part,
and the name of the part is stored in |\childdocname|.
Note that |\jobname| will be set to the filename of the current part
so that each part receives an individual |.aux| file
that does not interfere with the |.aux| file(s) of the main document.
This behaviour can be altered by the alternative form
|\childdocby[*]{|\textit{main}|}| (with a non-empty optional argument)
which uses the |.aux| file of the main document
by setting |\jobname| to \textit{main}.

%%%%%%%%%%%%%%%%%%%%%%%%%%%%%%%%%%%%%%%%%%%%%%%%%%%%%%%%%%%%%%%%%%%%%%%%%%%%%%%%
\subsection{Driver Development}
\label{sec:driver}

The \textsf{childdoc} mechanism can also be use for the development
of definition files such as \LaTeX{} styles or classes.
This case differs from the above setup with multiple parts
included by |\include| in that no |\includeonly| should be invoked.
This can be achieved by starting the include file
(before |\ProvidesPackage|) with:
%
\begin{center}
\begin{tabular}{l}
|\input{childdoc.def}|\\
|\childdocforward{|\textit{main}|}|\\
\end{tabular}
\end{center}
%
or alternatively with:
%
\begin{center}
\begin{tabular}{l}
|\input{childdoc.def}|\\
|\childdocby{|\textit{main}|}|\\
\end{tabular}
\end{center}
%
Both forms have slightly different effects as described above.
The main file is prepared as usual, see \secref{sec:include}.

%%%%%%%%%%%%%%%%%%%%%%%%%%%%%%%%%%%%%%%%%%%%%%%%%%%%%%%%%%%%%%%%%%%%%%%%%%%%%%%%
\subsection{Legacy Detection}
\label{sec:detection}

The directive |\childdocmain| in the main file can detect
whether the complete document or merely a child is to be compiled
even without using the directive |\childdocof|.
This method is deprecated because it is less robust
and there is no compelling reason to use it;
it is merely provided for backward compatibility
and it may be removed in future versions.

If the detection mechanism is to be used,
it is mandatory to correctly specify
the filename of the main file as the argument of |\childdocmain|:
%
\begin{center}
\begin{tabular}{l}
|\input{childdoc.def}|\\
|\childdocmain{|\textit{main}|}|\\
\end{tabular}
\end{center}
%
If |\jobname| does not match the argument \textit{main} of |\childdocmain|,
it is assumed that |\jobname| points to the child file to be compiled.
When using |\childdocmain| with the main file specified as argument,
it suffices to start a child file
with just |\input{|\textit{main}|}|
without loading of the package and using |\childdocof|.
If instead all processing is done
with the appropriate \textsf{childdoc} directives,
the argument of \textit{main} of |\childdocmain| can be empty.

An alternative version of the command line processing described
in \secref{sec:commandline} using the detection mechanism reads:
%
\begin{center}
|... -jobname "|\textit{target}|" "|[\textit{flags}]%
[|\def\jobname{|\textit{dest}|}|]|\input{|\textit{main}|}"|
\end{center}

%%%%%%%%%%%%%%%%%%%%%%%%%%%%%%%%%%%%%%%%%%%%%%%%%%%%%%%%%%%%%%%%%%%%%%%%%%%%%%%%
\subsection{Manual Code}
\label{sec:manual}

In case one cannot be certain whether the definitions file |childdoc.def|
is installed on the target \TeX{} distribution
and one prefers not to ship it,
it is conceivable to paste a few relevant commands into the sources.

To that end, drop all statements |\input{childdoc.def}|
and perform the replacements as outlined below.
Instead of |\childdocmain{|\textit{main}|}| add the following code
to the top of the main file:
%
\begin{center}
\begin{tabular}{l}
|\||ifdefined\childdocname\endinput\||fi\newif\ifchilddoc|\\
|\edef\childdocname{\scantokens\expandafter{\jobname\noexpand}}|\\
|\def\childdocmain{|\textit{main}|}\||ifx\childdocmain\childdocname\||else|\\
|\childdoctrue\includeonly{\childdocname}\let\jobname\childdocmain\||fi|\\
\end{tabular}
\end{center}
%
Instead of |\childdocof{|\textit{main}|}| just include the main file
at the top of each child file:
%
\begin{center}
|\input{|\textit{main}|}|
\end{center}
%
A simple redirection |\childdocforward{|\textit{dest}|}| is achieved by:
%
\begin{center}
|\def\jobname{|\textit{dest}|}\input{\jobname}|
\end{center}
%
The redirection with prefix
|\childdocforwardprefix[|\textit{prefix}|]{|\textit{dest}|}|
is accomplished by:
%
\begin{center}
\begin{tabular}{l}
|{\edef\jobname{\scantokens\expandafter{\jobname\noexpand}}|\\
|\def\redirectjob |\textit{prefix}|#1~~~{\gdef\jobname{|\textit{dest}|#1}}|\\
|\expandafter\redirectjob\jobname~~~}\input{\jobname}|
\end{tabular}
\end{center}

In an alternative approach,
child documents can be compiled by a specific command line
without additional code or specific definitions:
%
\begin{center}
|... -jobname "|\textit{target}|" "|[\textit{flags}]%
|\includeonly{|\textit{dest}|}\input{|\textit{main}|}"|
\end{center}
%

%%%%%%%%%%%%%%%%%%%%%%%%%%%%%%%%%%%%%%%%%%%%%%%%%%%%%%%%%%%%%%%%%%%%%%%%%%%%%%%%
%%%%%%%%%%%%%%%%%%%%%%%%%%%%%%%%%%%%%%%%%%%%%%%%%%%%%%%%%%%%%%%%%%%%%%%%%%%%%%%%
\section{Information}

%%%%%%%%%%%%%%%%%%%%%%%%%%%%%%%%%%%%%%%%%%%%%%%%%%%%%%%%%%%%%%%%%%%%%%%%%%%%%%%%
\subsection{Copyright}

Copyright \copyright{} 2017--2018 Niklas Beisert

This work may be distributed and/or modified under the
conditions of the \LaTeX{} Project Public License, either version 1.3
of this license or (at your option) any later version.
The latest version of this license is in
  \url{http://www.latex-project.org/lppl.txt}
and version 1.3 or later is part of all distributions of \LaTeX{}
version 2005/12/01 or later.

This work has the LPPL maintenance status `maintained'.

The Current Maintainer of this work is Niklas Beisert.

This work consists of the files |README.txt|, |childdoc.ins| and |childdoc.dtx|
as well as the derived files |childdoc.def|, |cdocsamp.tex|
with |cdocsch1.tex|, |cdocsch2.tex|, |cdocspt3.tex|, |cdocspt4.tex|,
|cdocsdrf.tex|, |cdocsfn1.tex|, |cdocsfn2.tex|
as well as |childdoc.pdf|.

%%%%%%%%%%%%%%%%%%%%%%%%%%%%%%%%%%%%%%%%%%%%%%%%%%%%%%%%%%%%%%%%%%%%%%%%%%%%%%%%
\subsection{Files and Installation}

The package consists of the files:
%
\begin{center}
\begin{tabular}{ll}
    |README.txt|   & readme file \\
    |childdoc.ins| & installation file \\
    |childdoc.dtx| & source file \\
    |childdoc.def| & definition file \\
    |cdocsamp.tex| & sample main file \\
    |cdocsch1.tex| & sample include file \\
    |cdocsch2.tex| & sample include file \\
    |cdocspt3.tex| & sample part file \\
    |cdocspt4.tex| & sample part file \\
    |cdocsdrf.tex| & sample redirection file \\
    |cdocsfn1.tex| & sample redirection file \\
    |cdocsfn2.tex| & sample redirection file \\
    |childdoc.pdf| & manual
\end{tabular}
\end{center}
%
The distribution consists of the files
|README.txt|, |childdoc.ins| and |childdoc.dtx|.
%
\begin{itemize}
\item
Run (pdf)\LaTeX{} on |childdoc.dtx|
to compile the manual |childdoc.pdf| (this file).
\item
Run \LaTeX{} on |childdoc.ins| to create the definitions file |childdoc.def|
and the sample |cdocsamp.tex| with include files
|cdocsch1.tex|, |cdocsch2.tex|, |cdocspt3.tex|, |cdocspt4.tex|,
|cdocsdrf.tex|, |cdocsfn1.tex|, |cdocsfn2.tex|.
Then copy the file |childdoc.def| to an appropriate directory of your \LaTeX{}
distribution, e.g.\ \textit{texmf-root}|/tex/latex/childdoc|.
\end{itemize}

%%%%%%%%%%%%%%%%%%%%%%%%%%%%%%%%%%%%%%%%%%%%%%%%%%%%%%%%%%%%%%%%%%%%%%%%%%%%%%%%
\subsection{Related CTAN Packages}

There are several other packages which offer a similar functionality:
%
\begin{itemize}
\item
The packages
\href{http://ctan.org/pkg/docmute}{\textsf{docmute}},
\href{http://ctan.org/pkg/includex}{\textsf{includex}} and
\href{http://ctan.org/pkg/standalone}{\textsf{standalone}}
provide commands to include only the document body of
a child file thus allowing both files to be compiled individually.
\item
The packages \href{http://ctan.org/pkg/subdocs}{\textsf{subdocs}}
and \href{http://ctan.org/pkg/subfiles}{\textsf{subfiles}}
provide structures in which the main and child documents can be
encapsulated and allowing them to be compiled individually.
The inclusion mechanism is different from the conventional |\include|.
\item
The package \href{http://ctan.org/pkg/combine}{\textsf{combine}}
is an elaborate solution to combine several documents into one.
\end{itemize}
%
See also the CTAN topic \href{http://ctan.org/topic/subdocs}{\textsf{subdocs}}
for further related packages.
The present package differs from the above solutions in that
a document structure constructed with the conventional |\include| mechanism
just needs two extra commands at the top of every file
such that all constituent files can be compiled individually.

%%%%%%%%%%%%%%%%%%%%%%%%%%%%%%%%%%%%%%%%%%%%%%%%%%%%%%%%%%%%%%%%%%%%%%%%%%%%%%%%
%\subsection{Feature Suggestions}
%
%The following is a list of features which may be useful for future
%versions of this package:
%%
%\begin{itemize}
%\item
%\ldots
%\end{itemize}

%%%%%%%%%%%%%%%%%%%%%%%%%%%%%%%%%%%%%%%%%%%%%%%%%%%%%%%%%%%%%%%%%%%%%%%%%%%%%%%%
\subsection{Revision History}

%%%%%%%%%%%%%%%%%%%%%%%%%%%%%%%%%%%%%%%%
\paragraph{v2.0:} 2018/12/30

\begin{itemize}
\item
immediate forward processing
\item
added |\childdocby| mechanism
\item
manual restructured
\end{itemize}

%%%%%%%%%%%%%%%%%%%%%%%%%%%%%%%%%%%%%%%%
\paragraph{v1.6:} 2018/01/17

\begin{itemize}
\item
application for development of include files
\item
corrections to manual
\end{itemize}

%%%%%%%%%%%%%%%%%%%%%%%%%%%%%%%%%%%%%%%%
\paragraph{v1.5:} 2017/05/21

\begin{itemize}
\item
more complete structuring introduced
\item
|\childdocof| introduced
\item
|\childdoc| renamed to |\childdocmain|
\item
|\childredirect| renamed to |\childdocforward| and |\childdocforwardprefix|
and functionality expanded
\end{itemize}

%%%%%%%%%%%%%%%%%%%%%%%%%%%%%%%%%%%%%%%%
\paragraph{v1.0:} 2017/04/27

\begin{itemize}
\item
manual and install package
\item
first version published on CTAN
\end{itemize}

%%%%%%%%%%%%%%%%%%%%%%%%%%%%%%%%%%%%%%%%
\paragraph{v0.6:} 2017/04/26

\begin{itemize}
\item
redirection mechanism added
\end{itemize}

%%%%%%%%%%%%%%%%%%%%%%%%%%%%%%%%%%%%%%%%
\paragraph{v0.5:} 2017/04/26

\begin{itemize}
\item
functionality in definition file
\end{itemize}


%%%%%%%%%%%%%%%%%%%%%%%%%%%%%%%%%%%%%%%%%%%%%%%%%%%%%%%%%%%%%%%%%%%%%%%%%%%%%%%%
%%%%%%%%%%%%%%%%%%%%%%%%%%%%%%%%%%%%%%%%%%%%%%%%%%%%%%%%%%%%%%%%%%%%%%%%%%%%%%%%
%%%%%%%%%%%%%%%%%%%%%%%%%%%%%%%%%%%%%%%%%%%%%%%%%%%%%%%%%%%%%%%%%%%%%%%%%%%%%%%%
\appendix

\settowidth\MacroIndent{\rmfamily\scriptsize 000\ }

 \DocInput{childdoc.dtx}

\end{document}
%</driver>
% \fi
%
% %%%%%%%%%%%%%%%%%%%%%%%%%%%%%%%%%%%%%%%%%%%%%%%%%%%%%%%%%%%%%%%%%%%%%%%%%%%%%%
% %%%%%%%%%%%%%%%%%%%%%%%%%%%%%%%%%%%%%%%%%%%%%%%%%%%%%%%%%%%%%%%%%%%%%%%%%%%%%%
% \section{Sample}
%\iffalse
%<*samplemain>
%\fi
%
% The following presents a sample document
% with two chapters, two parts, a title page,
% a compile flag as well as three forwarding files to set the flag.
% It consists of eight |.tex| files:
% \begin{center}
% \begin{tabular}{ll}
% |cdocsamp.tex|&main file\\
% |cdocsch1.tex|&include file for chapter 1\\
% |cdocsch2.tex|&include file for chapter 2\\
% |cdocspt3.tex|&include file for part 3\\
% |cdocspt4.tex|&include file for part 4\\
% |cdocsdrf.tex|&forwarding file for main file in draft mode\\
% |cdocsfi1.tex|&forwarding file for final version of chapter 1\\
% |cdocsfi2.tex|&forwarding file for final version of chapter 2\\
% \end{tabular}
% \end{center}
% Each of the eight files can be compiled directly by the \LaTeX{} compiler.
%
% %%%%%%%%%%%%%%%%%%%%%%%%%%%%%%%%%%%%%%
% \paragraph{Main File.}
%
% The main file is called |cdocsamp.tex|.
%
% Load the \textsf{childdoc} definitions and
% declare the filename for the main document:
%    \begin{macrocode}
\input{childdoc.def}
\childdocmain{}
%    \end{macrocode}

% Optional override for |\version| flag:
%    \begin{macrocode}
%%\ifchilddoc\else\providecommand{\version}{draft}\fi
%    \end{macrocode}

% Define the default values for the |\version| flag
% (|final| for the main file and |draft| for childs):
%    \begin{macrocode}
\ifchilddoc
\providecommand{\version}{draft}
\else
\providecommand{\version}{final}
\fi
%    \end{macrocode}

% Load the standard document class:
%    \begin{macrocode}
\documentclass[12pt]{article}
%    \end{macrocode}

% Start the document body:
%    \begin{macrocode}
\begin{document}
%    \end{macrocode}

% Declare a title page.
% Print title, part of document being processed and version flag:
%    \begin{macrocode}
\addtocounter{page}{-1}
\begin{center}
{\LARGE\bfseries{}childdoc example\par}
\vspace{1cm}
\ifchilddoc
\ifchilddocmanual part\else chapter\fi:
`\childdocname' of `\childdocjob'\par
\else
main document: `\childdocjob'\par
\fi
version: \version\par
\end{center}
\newpage
%    \end{macrocode}

% Manually include selected file,
% otherwise process as usual:
%    \begin{macrocode}
\ifchilddocmanual
\section*{part `\childdocname'}
\input{\childdocname}
\else
%    \end{macrocode}

% Include the two chapters:
%    \begin{macrocode}
\include{cdocsch1}
\include{cdocsch2}
%    \end{macrocode}

% Include the two parts unless only chapters should be displayed:
%    \begin{macrocode}
\ifchilddoc\else
\section{part three}
\input{cdocspt3}
\section{part four}
\input{cdocspt4}
\fi
%    \end{macrocode}

% Process as usual until here:
%    \begin{macrocode}
\fi
%    \end{macrocode}

% End of document body:
%    \begin{macrocode}
\end{document}
%    \end{macrocode}
%\iffalse
%</samplemain>
%\fi
%
% %%%%%%%%%%%%%%%%%%%%%%%%%%%%%%%%%%%%%%
% \paragraph{Chapter Include Files.}
%
% The include files are called |cdocsch1.tex| and |cdocsch2.tex|.
%
%\iffalse
%<*samplechap1|samplechap2>
%\fi

% Optional override for |\version| flag:
%    \begin{macrocode}
%%\providecommand{\version}{final}
%    \end{macrocode}

% Include the main document:
%    \begin{macrocode}
\input{childdoc.def}
\childdocof{cdocsamp}
%    \end{macrocode}

%\iffalse
%</samplechap1|samplechap2>
%\fi
%
%\iffalse
%<*samplechap1>
%\fi
% Some text for chapter 1:
%    \begin{macrocode}
\section{one}
some text in chapter one
%    \end{macrocode}

%\iffalse
%</samplechap1>
%\fi
% Some text for chapter 2:
%\iffalse
%<*samplechap2>
%\fi
%    \begin{macrocode}
\section{two}
more text in chapter two
%    \end{macrocode}

%\iffalse
%</samplechap2>
%\fi
%
% %%%%%%%%%%%%%%%%%%%%%%%%%%%%%%%%%%%%%%
% \paragraph{Part Include Files.}
%
% The include files are called |cdocspt3.tex| and |cdocspt4.tex|.
%
%\iffalse
%<*samplepart3|samplepart4>
%\fi

% Optional override for |\version| flag:
%    \begin{macrocode}
%%\providecommand{\version}{final}
%    \end{macrocode}

% Include the main document:
%    \begin{macrocode}
\input{childdoc.def}
\childdocby{cdocsamp}
%    \end{macrocode}

%\iffalse
%</samplepart3|samplepart4>
%\fi
%
%\iffalse
%<*samplepart3>
%\fi
% Some text for part 3:
%    \begin{macrocode}
some text in part three
%    \end{macrocode}

%\iffalse
%</samplepart3>
%\fi
% Some text for part 4:
%\iffalse
%<*samplepart4>
%\fi
%    \begin{macrocode}
more text in part four
%    \end{macrocode}

%\iffalse
%</samplepart4>
%\fi
%
% %%%%%%%%%%%%%%%%%%%%%%%%%%%%%%%%%%%%%%
% \paragraph{Forwarding for a Complete Draft.}
%
% The following forwarding file |cdocsdrf.tex|
% compiles the main document in draft mode:
%\iffalse
%<*sampledraft>
%\fi
%    \begin{macrocode}
\def\version{draft}
\input{childdoc.def}
\childdocforward{cdocsamp}
%    \end{macrocode}

%\iffalse
%</sampledraft>
%\fi
%
% %%%%%%%%%%%%%%%%%%%%%%%%%%%%%%%%%%%%%%
% \paragraph{Forwarding for Final Version of the Chapters.}
%
% The following forwarding files |cdocsfn1.tex| and |cdocsfn2.tex|
% (with identical content)
% compile the final versions of the child documents
% |cdocsch1.tex| and |cdocsch2.tex|, respectively:
%\iffalse
%<*samplefinal>
%\fi
%    \begin{macrocode}
\def\version{final}
\input{childdoc.def}
\childdocforwardprefix[cdocsamp]{cdocsfn}{cdocsch}
%    \end{macrocode}

%\iffalse
%</samplefinal>
%\fi
%
% %%%%%%%%%%%%%%%%%%%%%%%%%%%%%%%%%%%%%%
% \paragraph{Command Line Processing.}
%
% The following three command lines generate the output files
% |cdocscld|, |cdocscl1| and |cdocscl2|
% which should be identical to
% |cdocsdrf|, |cdocsch1| and |cdocsfn2|, respectively:
% \begin{center}
% \begin{tabular}{l}
% |latex -jobname cdocscld \|\\
% |  "\def\version{draft}\input{childdoc.def}\childdocforward{cdocsamp}"|\\
% |latex -jobname cdocscl1 \|\\
% |  "\input{childdoc.def}\childdocforward[cdocsamp]{cdocsch1}"|\\
% |latex -jobname cdocscl2 \|\\
% |  "\def\version{final}\input{childdoc.def}\childdocforward{cdocsch2}"|
% \end{tabular}
% \end{center}
% Note that the trailing backslash on each first line
% merely continues the input to the second line
% (for convenient cut ant paste).
% Furthermore, the command |latex| can be replaced by any
% of its alternative versions such as |pdflatex|.
%
% %%%%%%%%%%%%%%%%%%%%%%%%%%%%%%%%%%%%%%%%%%%%%%%%%%%%%%%%%%%%%%%%%%%%%%%%%%%%%%
% %%%%%%%%%%%%%%%%%%%%%%%%%%%%%%%%%%%%%%%%%%%%%%%%%%%%%%%%%%%%%%%%%%%%%%%%%%%%%%
% \section{Implementation}
%\iffalse
%<*package>
%\fi
%
% This section describes the definitions file |childdoc.def|.

% The definitions cannot be loaded using |\usepackage| or |\RequirePackage|
% which has a mechanism to prevent loading a style file more than once.
% When loading the definitions by means of |\input|
% multiple instances have to be prevented manually:
%\iffalse
%This code needs to be before the `\ProvidesFile' directive
%which is defined at the beginning of this file.
%Therefore it is also placed there and commented out here.
%</package>
%<*discard>
%\fi
%    \begin{macrocode}
\ifdefined\childdocmain\endinput\fi
%    \end{macrocode}
%\iffalse
%</discard>
%<*package>
%\fi
%
% \macro{\ifchilddoc}
% \macro{\ifchilddocmanual}
% The conditional |\ifchilddoc| tells whether a
% child (true) or main (false) document is being compiled.
% The conditional |\ifchilddocmanual| tells whether
% the |\includeonly| mechanism is used (false) or
% the selection of child files must be performed manually (true).
% The definitions initialise to false:
%    \begin{macrocode}
\newif\ifchilddoc
\newif\ifchilddocmanual
%    \end{macrocode}

% \macro{\childdocname}
% \macro{\childdocjob}
% The macro |\childdocname| stores the name of the main document
% to be compiled. The macro |\childdocjob| stores the name of
% the document on which the \LaTeX{} compiler was originally invoked.
% The content of |\jobname| cannot be compared
% to filenames specified in the source due to different catcodes.
% The following code rescans |\jobname|, stores the result
% in |\childdocname| and saves a copy in |\childdocjob|:
%    \begin{macrocode}
\edef\childdocname{\scantokens\expandafter{\jobname\noexpand}}
\let\childdocjob\childdocname
%    \end{macrocode}

% \macro{\childdocdisable}
% The macro |\childdocdisable| prevents the main file
% from being processed more than once.
% At this stage, the main document command |\childdocmain|
% is assumed to be called once again where it should do nothing.
% Any subsequent call to it should prevent
% a secondary processing of the main document
% It overwrites the forwarding commands
% |\childdocof| and |\childdocforward|
% with empty macros to prevent further inclusions of the main document:
%    \begin{macrocode}
\newcommand{\childdocdisable}
{
  \renewcommand{\childdocmain}[1]{\renewcommand{\childdocmain}[1]{\endinput}}
  \renewcommand{\childdocof}[1]{}
  \renewcommand{\childdocby}[2][]{}
  \renewcommand{\childdocforward}[2][]{}
  \renewcommand{\childdocdisable}{}
}
%    \end{macrocode}

% \macro{\childdocmain}
% The macro |\childdocmain| is to be called at the top of the main file
% with nothing or the main filename (without extension) as argument.
% First, it breaks loops.
% If the argument is not empty and does not match |\childdocname|
% (which is set by the first inclusion of |childdoc.def|),
% |\ifchilddoc| is set to true, |\includeonly| is applied to the child file
% and |\jobname| is set to the main file
% (for proper handling of |.aux| files):
%    \begin{macrocode}
\newcommand{\childdocmain}[1]
{
  \childdocdisable\childdocmain{}
  \if?#1?\else
    \begingroup
      \def\childdoctmp{#1}
      \ifx\childdoctmp\childdocname
        \def\childdoctmp{}
      \else
        \def\childdoctmp
        {
          \childdoctrue
          \includeonly{\childdocname}
          \def\childdocjob{#1}
          \def\jobname{#1}
        }
      \fi
      \expandafter
    \endgroup
    \childdoctmp
  \fi
}
%    \end{macrocode}

% \macro{\childdocof}
% The command |\childdocof| redirects
% compilation to the main file |#1|.
%    \begin{macrocode}
\newcommand{\childdocof}[1]
{
  \childdocdisable
  \childdoctrue
  \includeonly{\childdocname}
  \def\jobname{#1}
  \def\childdocjob{#1}
  \input{#1}
}
%    \end{macrocode}

% \macro{\childdocby}
% The command |\childdocby| ....
%    \begin{macrocode}
\newcommand{\childdocby}[2][]
{
  \childdocdisable
  \childdoctrue
  \childdocmanualtrue
  \if?#1?\else
    \def\jobname{#2}
  \fi
  \def\childdocjob{#2}
  \input{#2}
  \endinput
}
%    \end{macrocode}

% \macro{\childdocforward}
% The command |\childdocforward| redirects
% compilation to the main file or
% (if the optional argument is given) a child file.
% Parameters are set as if the main file
% or a child file starting with |\childdocof| was compiled.
% Then compilation is handed over to the main file:
%    \begin{macrocode}
\newcommand{\childdocforward}[2][]
{
  \begingroup
    \if?#1?
      \def\childdoctmp
      {
        \def\childdocname{#2}
        \def\childdocjob{#2}
        \def\jobname{#2}
        \input{#2}
        \endinput
      }
    \else
      \def\childdoctmp
      {
        \childdocdisable
        \def\childdocname{#2}
        \childdoctrue
        \includeonly{#2}
        \def\childdocjob{#1}
        \def\jobname{#1}
        \input{#1}
        \endinput
      }
    \fi
    \expandafter
  \endgroup
  \childdoctmp
}
%    \end{macrocode}

% \macro{\childdocforwardprefix}
% The command |\childdocforwardprefix| redirects
% compilation to the main or a child file by means of a pattern.
% The prefix |#1| in the current filename is replaced by |#2|
% and the suffix of the current filename is kept
% (it is assumed that the filename does not contain the substring `|~~~|'
% which is used as a delimiter).
% Compilation is handed over to the new file by |\childdocforward|:
%    \begin{macrocode}
\newcommand{\childdocforwardprefix}[3][]
{
  \begingroup
    \def\childdocextract #2##1~~~{\def\childdoctmp{\childdocforward[#1]{#3##1}}}
    \expandafter\childdocextract\childdocname~~~
    \expandafter
  \endgroup
  \childdoctmp
}
%    \end{macrocode}

% \macro{\childdoc}
% The deprecated macro |\childdoc| is a legacy version of |\childdocmain|:
%    \begin{macrocode}
\newcommand{\childdoc}{\childdocmain}
%    \end{macrocode}

% \macro{\childdocredirect}
% The deprecated macro |\childdocredirect| is a legacy version
% of |\childdocforward| and |\childdocforwardprefix|:
%    \begin{macrocode}
\newcommand{\childdocredirect}[2][]
{
  \begingroup
    \if?#1?
      \def\childdoctmp{\childdocforward{#2}}
    \else
      \def\childdoctmp{\childdocforwardprefix{#1}{#2}}
    \fi
    \expandafter
  \endgroup
  \childdoctmp
}
%    \end{macrocode}

%\iffalse
%</package>
%\fi
%
\endinput

\childdocforwardprefix[cdocsamp]{cdocsfn}{cdocsch}
%    \end{macrocode}

%\iffalse
%</samplefinal>
%\fi
%
% %%%%%%%%%%%%%%%%%%%%%%%%%%%%%%%%%%%%%%
% \paragraph{Command Line Processing.}
%
% The following three command lines generate the output files
% |cdocscld|, |cdocscl1| and |cdocscl2|
% which should be identical to
% |cdocsdrf|, |cdocsch1| and |cdocsfn2|, respectively:
% \begin{center}
% \begin{tabular}{l}
% |latex -jobname cdocscld \|\\
% |  "\def\version{draft}% \iffalse
%
% childdoc.dtx Copyright (C) 2017-2018 Niklas Beisert
%
% This work may be distributed and/or modified under the
% conditions of the LaTeX Project Public License, either version 1.3
% of this license or (at your option) any later version.
% The latest version of this license is in
%   http://www.latex-project.org/lppl.txt
% and version 1.3 or later is part of all distributions of LaTeX
% version 2005/12/01 or later.
%
% This work has the LPPL maintenance status `maintained'.
%
% The Current Maintainer of this work is Niklas Beisert.
%
% This work consists of the files childdoc.dtx and childdoc.ins
% and the derived files childdoc.def and cdocsamp.tex with
% cdocsch1.tex, cdocsch2.tex, cdocsdrf.tex, cdocsfn1.tex, cdocsfn2.tex.
%
%<package>\ifdefined\childdocmain\endinput\fi
%<package>\ProvidesFile{childdoc.def}[2018/12/30 v2.0 child document driver]
%<samplemain>\ProvidesFile{cdocsamp.tex}[2018/12/30 v2.0 sample for childdoc]
%<*driver>
%\ProvidesFile{childdoc.drv}[2018/12/30 v2.0 childdoc reference manual file]
\PassOptionsToClass{10pt,a4paper}{article}
\documentclass{ltxdoc}

\usepackage[margin=35mm]{geometry}
\usepackage{hyperref}
\usepackage{hyperxmp}
\usepackage[usenames]{color}

\hypersetup{colorlinks=true}
\hypersetup{pdfstartview=FitH}
\hypersetup{pdfpagemode=UseNone}
\hypersetup{pdfsource={}}
\hypersetup{pdflang={en-UK}}
\hypersetup{pdfcopyright={Copyright 2017-2018 Niklas Beisert.
  This work may be distributed and/or modified under the
  conditions of the LaTeX Project Public License, either version 1.3
  of this license or (at your option) any later version.}}
\hypersetup{pdflicenseurl={http://www.latex-project.org/lppl.txt}}
\hypersetup{pdfcontactaddress={ETH Zurich, ITP, HIT K,
  Wolfgang-Pauli-Strasse 27}}
\hypersetup{pdfcontactpostcode={8093}}
\hypersetup{pdfcontactcity={Zurich}}
\hypersetup{pdfcontactcountry={Switzerland}}
\hypersetup{pdfcontactemail={nbeisert@itp.phys.ethz.ch}}
\hypersetup{pdfcontacturl={http://people.phys.ethz.ch/\xmptilde nbeisert/}}

\newcommand{\secref}[1]{\hyperref[#1]{section \ref*{#1}}}

\parskip1ex
\parindent0pt
\let\olditemize\itemize
\def\itemize{\olditemize\parskip0pt}

\begin{document}

\title{The \textsf{childdoc} Package}
\hypersetup{pdftitle={The childdoc Package}}
\author{Niklas Beisert\\[2ex]
  Institut f\"ur Theoretische Physik\\
  Eidgen\"ossische Technische Hochschule Z\"urich\\
  Wolfgang-Pauli-Strasse 27, 8093 Z\"urich, Switzerland\\[1ex]
  \href{mailto:nbeisert@itp.phys.ethz.ch}
  {\texttt{nbeisert@itp.phys.ethz.ch}}}
\hypersetup{pdfauthor={Niklas Beisert}}
\hypersetup{pdfsubject={Manual for the LaTeX2e Package childdoc}}
\date{30 December 2018, \textsf{v2.0}}
\maketitle

\begin{abstract}\noindent
\textsf{childdoc} is a \LaTeXe{} package
that enables the direct compilation
of document sections included by |\include|
to individual files.
\end{abstract}

\begingroup
\parskip0ex
\tableofcontents
\endgroup

%%%%%%%%%%%%%%%%%%%%%%%%%%%%%%%%%%%%%%%%%%%%%%%%%%%%%%%%%%%%%%%%%%%%%%%%%%%%%%%%
%%%%%%%%%%%%%%%%%%%%%%%%%%%%%%%%%%%%%%%%%%%%%%%%%%%%%%%%%%%%%%%%%%%%%%%%%%%%%%%%
\section{Introduction}

\LaTeX{} provides a mechanism to structure a large document (such as a book)
into a main file and several child files (containing the chapters)
using the |\include| command.
This mechanism is beneficial for documents
which span hundreds of pages in order to
make the source file(s) more manageable.
Moreover, compilation can be restricted to
selected child files by means of the |\includeonly| command.
The latter feature can be used to reduce the compilation time while editing
(this was significantly more useful in the earlier days of \LaTeX{})
or to generate a smaller document which is easier to navigate.
Another application of |\includeonly| is to generate
documents consisting of selected parts of the complete document.

However, there are a few drawbacks of the plain |\include| mechanism:
\begin{itemize}
\item
The child files cannot be compiled on their own,
they can only be compiled via the main file.
A naive editing environment
(such as a text editor with an option
to have the current file processed by \LaTeX)
may require one to switch to the main file before compiling;
attempting to compile the child file produces errors.
\item
The main file must be modified (each time)
to adjust the |\includeonly| command
to the present needs. This easily leaves the main file in a messy state.
\item
The generated document will always carry the filename
of the main document. This is inconvenient if
several child files are to be compiled and
to be kept for distribution.
\end{itemize}

The present package provides a simple interface
to make child files individually compilable by \LaTeX{}.
Compiling a child file then has the same effect as compiling
the main file with an |\includeonly| command
to select the appropriate child.
Moreover the generated document will carry the name of the child
rather than the main file.
This resolves all three above issues.

This feature is meant to make the editing of books,
thesis documents and lecture notes somewhat more convenient.
However, the package can also be used efficiently for
composing a series of documents (such as exercise sheets)
which are typically distributed individually.
It then assists the author in generating the individual documents
(potentially in different versions)
as well as a document containing the collected series.
Another application is in developing style files
or other kinds of included material
where compilation of the style file could redirect
to a sample or test file.

%%%%%%%%%%%%%%%%%%%%%%%%%%%%%%%%%%%%%%%%%%%%%%%%%%%%%%%%%%%%%%%%%%%%%%%%%%%%%%%%
%%%%%%%%%%%%%%%%%%%%%%%%%%%%%%%%%%%%%%%%%%%%%%%%%%%%%%%%%%%%%%%%%%%%%%%%%%%%%%%%
\section{Usage}

First of all, the package \textsf{childdoc} is \emph{not} a standard
\LaTeXe{} |.sty| style file! Therefore it needs to be invoked in
a non-standard way.

%%%%%%%%%%%%%%%%%%%%%%%%%%%%%%%%%%%%%%%%%%%%%%%%%%%%%%%%%%%%%%%%%%%%%%%%%%%%%%%%
\subsection{Included Files}
\label{sec:include}

%%%%%%%%%%%%%%%%%%%%%%%%%%%%%%%%%%%%%%%%
\DescribeMacro{\childdocmain}
To use the package, add the commands
\begin{center}
\begin{tabular}{l}
|\input{childdoc.def}|\\
|\childdocmain{}|\\
\end{tabular}
\end{center}
at the very top of the main \LaTeX{} file,
in particular \emph{before} the |\documentclass| statement!
The argument of |\childdocmain| should be left empty
(but it must be present).

%%%%%%%%%%%%%%%%%%%%%%%%%%%%%%%%%%%%%%%%
\DescribeMacro{\childdocof}
Furthermore, add the commands
\begin{center}
\begin{tabular}{l}
|\input{childdoc.def}|\\
|\childdocof{|\textit{main}|}|\\
\end{tabular}
\end{center}
at the top of every child file \textit{child}
which is included by |\include{|\textit{child}|}|
from within the main file
(or at least for those files to be compiled individually).
The argument \textit{main} must be the filename of the main file.

There are a couple of
considerations in setting up the main and child documents:

%%%%%%%%%%%%%%%%%%%%%%%%%%%%%%%%%%%%%%%%
\paragraph{Restrictions.}

Please note the following restrictions:
\begin{itemize}
\item
|\childdocmain| must be called with one argument \textit{main}
to ensure compatibility with earlier version of the package.
It must either be empty (|\childdocmain{}|)
or precisely match the filename of the main file in which it is specified.
See \secref{sec:detection} for further information.
\item
The filename \textit{main} must be specified without the |.tex| extension.
\item
The filename \textit{main} is case sensitive
(even in case-insensitive file systems)
due to internal string comparison.
\item
The argument \textit{main} should be fully expanded, it cannot be a macro.
\item
Subdirectories and special characters should be avoided in filenames.
\item
The command |\childdocmain{|\textit{main}|}| must be followed by a whitespace.
It should not be followed immediately by another command
or by a comment mark `|%|'.
This is because the \TeX{} parser reads the token immediately following
the argument of |\childdocmain| and puts it
at the beginning of every child section;
however, a white\-space is ignored.
\end{itemize}

%%%%%%%%%%%%%%%%%%%%%%%%%%%%%%%%%%%%%%%%
\paragraph{Content of Main File.}

It is advisable to place all content in the child files included by |\include|.
Any output contained in the main file will appear in all child documents
unless suppressed manually;
it cannot be suppressed automatically by the |\includeonly| directive
and thus should normally be avoided.
A method to include some content in the main file
by means of conditional processing is described in \secref{sec:conditional}.

%%%%%%%%%%%%%%%%%%%%%%%%%%%%%%%%%%%%%%%%
\paragraph{Page Numbering.}

When only a part of the document is compiled,
the appropriate numbering of pages
(as well as other status parameters)
is determined from the |.aux| files.
The latter contain information from previous passes.
However this information needs to propagate through
all intermediate child documents.
Therefore the page numbering in child documents may well
be inconsistent until the complete document is compiled at least once.

A useful (if unconventional) way to always ensure a consistent
page numbering is to restart the numbering in each child document
and denote the pages by `\textit{child}|.|\textit{page}'
where \textit{child} represents the chapter/section number of the child file.
This can be achieved by the command
|\numberwithin{page}{|\textit{child}|}|
of the \textsf{amsmath} package
where \textit{child} can be |chapter| or |section|
depending on the chosen structuring.
Alternatively, one can modify the macro |\thepage| appropriately
and reset the counter |page| at the start of each child file.

%%%%%%%%%%%%%%%%%%%%%%%%%%%%%%%%%%%%%%%%%%%%%%%%%%%%%%%%%%%%%%%%%%%%%%%%%%%%%%%%
\subsection{Conditional Processing}
\label{sec:conditional}

The package provides a mechanism to compile different versions
of a document. To customise the versions further some conditional processing
can come in handy to distinguish which version is being compiled.
The package provides two macros to describe the compilation context:

%%%%%%%%%%%%%%%%%%%%%%%%%%%%%%%%%%%%%%%%
\DescribeMacro{\ifchilddoc}
The conditional |\ifchilddoc| distinguishes between the compilation of
child documents and the main document:
%
\begin{center}
|\ifchilddoc |\textit{child-code}| |[|\||else |\textit{main-code}]| \||fi|
\end{center}

%%%%%%%%%%%%%%%%%%%%%%%%%%%%%%%%%%%%%%%%
\DescribeMacro{\childdocname}
\DescribeMacro{\childdocjob}
The macro |\childdocname| contains the filename (without extension)
of the main or child file being processed.
Note that |\childdocjob| will always contain the name of the main file.

%%%%%%%%%%%%%%%%%%%%%%%%%%%%%%%%%%%%%%%%
\paragraph{Title Page.}

Conditional processing can be used to include a title or banner page
in the main document when proper precautions are taken.
Importantly, the code in the main file should ensure that the page counter
(as well as other status parameters which are stored in the |.aux| files)
takes the same value after the conditional processing.
Otherwise the page numbers may take divergent values
depending on which part is compiled.

For example, a title page could be declared by:
%
\begin{center}
\begin{tabular}{l}
|\ifchilddoc\||else|\\
|\addtocounter{page}{-1}|\\
\textit{code for title page}\\
|\newpage|\\
|\||fi|
\end{tabular}
\end{center}
%
A banner page for the child documents can be generated by:
%
\begin{center}
\begin{tabular}{l}
|\ifchilddoc|\\
|\addtocounter{page}{-1}|\\
\textit{code for banner page}\\
|\newpage|\\
|\||fi|
\end{tabular}
\end{center}
%
Here one could write a message such as:
\begin{center}
|This is the part \childdocname{} of \childdocjob{}.|
\end{center}

%%%%%%%%%%%%%%%%%%%%%%%%%%%%%%%%%%%%%%%%%%%%%%%%%%%%%%%%%%%%%%%%%%%%%%%%%%%%%%%%
\subsection{Flags}
\label{sec:flags}

The package makes it easy to generate different versions
of the main or child documents.
To this end compilation flags can be defined
and assigned different default values.
They will be particularly useful in conjunction
with the forwarding mechanism described in \secref{sec:forward}.

For example, it may be useful to have a flag |\version|
which can be set to |draft| or |final|.
The document source will contain some conditional code
depending on the value of |\version|.
Suppose further, the flag should default to |final| for the main file
and to |draft| for child files
which is a natural assignment for editing the document.
This is achieved by placing the following code
in the preamble of the main document
(below the |\childdocmain| directive):
%
\begin{center}
\begin{tabular}{l}
|\ifchilddoc|\\
|\providecommand{\version}{draft}|\\
|\||else|\\
|\providecommand{\version}{final}|\\
|\||fi|
\end{tabular}
\end{center}
%
The definition by |\providecommand| makes sure
that previous definitions are not overwritten.
Further statements |\providecommand{\version}{...}|
can thus be added before the above code to override it.

For the main file, one might add a line
(between |\childdocmain| and the above block)
%
\begin{center}
|%\ifchilddoc\||else\providecommand{\version}{draft}\||fi|
\end{center}
%
which can be uncommented to produce a draft version.
Likewise one can add a line to the very top of a child file
(above the |\childdocof{|\textit{main}|}| directive)
%
\begin{center}
|%\providecommand{\version}{final}|
\end{center}
%
which can be uncommented to produce the final version of this child document.

%%%%%%%%%%%%%%%%%%%%%%%%%%%%%%%%%%%%%%%%%%%%%%%%%%%%%%%%%%%%%%%%%%%%%%%%%%%%%%%%
\subsection{Forwarding}
\label{sec:forward}

Different versions of the main or child documents
using compilation flags as described in \secref{sec:flags}
can be (permanently) stored in different files
for convenient compilation, viewing and distribution.
To this end, the package defines a command
to pass on compilation to a different file:

%%%%%%%%%%%%%%%%%%%%%%%%%%%%%%%%%%%%%%%%
\DescribeMacro{\childdocforward}
The command |\childdocforward| redirects processing to
another source file:
%
\begin{center}
\begin{tabular}{l}
|\input{childdoc.def}|\\
|\childdocforward[|\textit{main}|]{|\textit{dest}|}|\\
\end{tabular}
\end{center}
%
The argument \textit{dest} is the destination file
(without extension).
It should be the main file or one of the child files.
Note that further \textsf{childdoc} directives
such as |\childdocof| and |\childdocforward|
in the indicated file will be processed in this form.
The optional argument \textit{main}
passes on directly to the main file \textit{main}
while pretending to compile the child \textit{dest}.
This form behaves as if \textit{dest}
issues |\childdocof{|\textit{main}|}| right away,
and no further \textsf{childdoc} directives will be processed.

%%%%%%%%%%%%%%%%%%%%%%%%%%%%%%%%%%%%%%%%
\DescribeMacro{\...prefix}
In the alternative form |\childdocforwardprefix|,
%
\begin{center}
\begin{tabular}{l}
|\input{childdoc.def}|\\
|\childdocforwardprefix[|\textit{main}|]{|\textit{prefix}|}{|\textit{dest}|}|
\end{tabular}
\end{center}
%
the destination file is determined by a pattern
depending on the current file:
To make this work, the current file must be called
`{\textit{prefix}\hspace{0.2em}\textit{suffix}}'
with \textit{prefix} matching precisely the argument.
Processing is then passed on to the file
`{\textit{dest}\hspace{0.2em}\textit{suffix}}'.
Surely, the same effect is achieved by
directly specifying the
argument `{\textit{dest}\hspace{0.2em}\textit{suffix}}'
in the first form.
However, that requires to set up a different file
for each child. With the alternative form of the command
all these files can have exactly the same content
which simplifies setting them up and maintaining them.

For example, the following file |draft.tex|
with a compilation flag |\version| as described in \secref{sec:flags}
compiles the main document as a draft:
%
\begin{center}
\begin{tabular}{l}
|\def\version{draft}|\\
|\input{childdoc.def}|\\
|\childdocforward{|\textit{main}|}|
\end{tabular}
\end{center}
%
Likewise, the following files |final|\textit{nn}|.tex|
compile the final version of the child document
|child|\textit{nn}|.tex|:
%
\begin{center}
\begin{tabular}{l}
|\def\version{final}|\\
|\input{childdoc.def}|\\
|\childdocforwardprefix{final}{child}|
\end{tabular}
\end{center}
%

Note that when several versions of a main file and/or of each child file
are to be generated, it may be convenient to set up a |Makefile| or
shell script to automatise the process.

%%%%%%%%%%%%%%%%%%%%%%%%%%%%%%%%%%%%%%%%%%%%%%%%%%%%%%%%%%%%%%%%%%%%%%%%%%%%%%%%
\subsection{Command Line Processing}
\label{sec:commandline}

The effect of redirection files can also be achieved by invoking
the \LaTeX{} compiler with a more elaborate command line.
Most conveniently this should be done as part
of a shell script or a |Makefile|.

When using \textsf{childdoc} in the main file, the following
command lines effectively perform a redirection
(note that depending on the shell being used,
backslashes may have to be doubled: `|\|' $\to$ `|\\|'):
%
\begin{center}
|... -jobname "|\textit{target}|" |\\|"|[\textit{flags}]%
|\input{childdoc.def}\childdocforward[|\textit{main}|]{|\textit{dest}|}"|
\end{center}
%
Here \textit{target} is the name of the output file,
\textit{main} is the name of the main file
and \textit{dest} is the name of the main or child file to be processed
(all filenames without extensions).
The optional argument \textit{main} can be omitted
if \textit{main} matches \textit{dest}.
Optionally, compilation \textit{flags} can be defined via |\def| commands.
This command line makes the \TeX{} engine believe
it is compiling the file \textit{target}
whose content is specified as the latter parameter.
The provided code then forwards the processing to
\textit{main} or \textit{dest} as described in \secref{sec:forward}.

%%%%%%%%%%%%%%%%%%%%%%%%%%%%%%%%%%%%%%%%%%%%%%%%%%%%%%%%%%%%%%%%%%%%%%%%%%%%%%%%
\subsection{Include by Input}
\label{sec:input}

Including child documents by |\include| has some restrictions by design.
Most notably, the content of a child document always occupies
its own set of pages; pages cannot be shared between child documents.
Usually, this behaviour makes perfect sense
because each child document contain an essential part of the document.
However, in some situations it may be desirable to compose
a document from a collection of parts
without having mandatory page breaks between then.
For this case, the package
provides a mechanism to include parts
by |\input| which can also be processed individually.
However, by construction this mechanism
requires manual handling of the content to be output.

%%%%%%%%%%%%%%%%%%%%%%%%%%%%%%%%%%%%%%%%
\DescribeMacro{\ifchilddocmanual}
The main file should be prepared as usual, see \secref{sec:include}.
However, the document body must make a distinction
between processing of an individual part and of the main document, e.g.:
%
\begin{center}
\begin{tabular}{l}
|\ifchilddocmanual|\\
|\input{\childdocname}|\\
|\||else|\\
\textit{document body with }|\input{|\textit{part}|}|\\
|\||fi|
\end{tabular}
\end{center}
%
The conditional |\ifchilddocmanual| is true whenever
a part to be included by |\input| is being compiled,
and the name of the part is stored in |\childdocname|.

%%%%%%%%%%%%%%%%%%%%%%%%%%%%%%%%%%%%%%%%
\DescribeMacro{\childdocby}
Each part to be included by |\input| should start with:
%
\begin{center}
\begin{tabular}{l}
|\input{childdoc.def}|\\
|\childdocby{|\textit{main}|}|\\
\end{tabular}
\end{center}
%
The directive |\childdocby| is similar to |\childdocof|
described in \secref{sec:include},
but the subsequent selection of content must be done manually.
To that end, both |\ifchilddoc| and |\ifchilddocmanual|
will be true upon processing of a part,
and the name of the part is stored in |\childdocname|.
Note that |\jobname| will be set to the filename of the current part
so that each part receives an individual |.aux| file
that does not interfere with the |.aux| file(s) of the main document.
This behaviour can be altered by the alternative form
|\childdocby[*]{|\textit{main}|}| (with a non-empty optional argument)
which uses the |.aux| file of the main document
by setting |\jobname| to \textit{main}.

%%%%%%%%%%%%%%%%%%%%%%%%%%%%%%%%%%%%%%%%%%%%%%%%%%%%%%%%%%%%%%%%%%%%%%%%%%%%%%%%
\subsection{Driver Development}
\label{sec:driver}

The \textsf{childdoc} mechanism can also be use for the development
of definition files such as \LaTeX{} styles or classes.
This case differs from the above setup with multiple parts
included by |\include| in that no |\includeonly| should be invoked.
This can be achieved by starting the include file
(before |\ProvidesPackage|) with:
%
\begin{center}
\begin{tabular}{l}
|\input{childdoc.def}|\\
|\childdocforward{|\textit{main}|}|\\
\end{tabular}
\end{center}
%
or alternatively with:
%
\begin{center}
\begin{tabular}{l}
|\input{childdoc.def}|\\
|\childdocby{|\textit{main}|}|\\
\end{tabular}
\end{center}
%
Both forms have slightly different effects as described above.
The main file is prepared as usual, see \secref{sec:include}.

%%%%%%%%%%%%%%%%%%%%%%%%%%%%%%%%%%%%%%%%%%%%%%%%%%%%%%%%%%%%%%%%%%%%%%%%%%%%%%%%
\subsection{Legacy Detection}
\label{sec:detection}

The directive |\childdocmain| in the main file can detect
whether the complete document or merely a child is to be compiled
even without using the directive |\childdocof|.
This method is deprecated because it is less robust
and there is no compelling reason to use it;
it is merely provided for backward compatibility
and it may be removed in future versions.

If the detection mechanism is to be used,
it is mandatory to correctly specify
the filename of the main file as the argument of |\childdocmain|:
%
\begin{center}
\begin{tabular}{l}
|\input{childdoc.def}|\\
|\childdocmain{|\textit{main}|}|\\
\end{tabular}
\end{center}
%
If |\jobname| does not match the argument \textit{main} of |\childdocmain|,
it is assumed that |\jobname| points to the child file to be compiled.
When using |\childdocmain| with the main file specified as argument,
it suffices to start a child file
with just |\input{|\textit{main}|}|
without loading of the package and using |\childdocof|.
If instead all processing is done
with the appropriate \textsf{childdoc} directives,
the argument of \textit{main} of |\childdocmain| can be empty.

An alternative version of the command line processing described
in \secref{sec:commandline} using the detection mechanism reads:
%
\begin{center}
|... -jobname "|\textit{target}|" "|[\textit{flags}]%
[|\def\jobname{|\textit{dest}|}|]|\input{|\textit{main}|}"|
\end{center}

%%%%%%%%%%%%%%%%%%%%%%%%%%%%%%%%%%%%%%%%%%%%%%%%%%%%%%%%%%%%%%%%%%%%%%%%%%%%%%%%
\subsection{Manual Code}
\label{sec:manual}

In case one cannot be certain whether the definitions file |childdoc.def|
is installed on the target \TeX{} distribution
and one prefers not to ship it,
it is conceivable to paste a few relevant commands into the sources.

To that end, drop all statements |\input{childdoc.def}|
and perform the replacements as outlined below.
Instead of |\childdocmain{|\textit{main}|}| add the following code
to the top of the main file:
%
\begin{center}
\begin{tabular}{l}
|\||ifdefined\childdocname\endinput\||fi\newif\ifchilddoc|\\
|\edef\childdocname{\scantokens\expandafter{\jobname\noexpand}}|\\
|\def\childdocmain{|\textit{main}|}\||ifx\childdocmain\childdocname\||else|\\
|\childdoctrue\includeonly{\childdocname}\let\jobname\childdocmain\||fi|\\
\end{tabular}
\end{center}
%
Instead of |\childdocof{|\textit{main}|}| just include the main file
at the top of each child file:
%
\begin{center}
|\input{|\textit{main}|}|
\end{center}
%
A simple redirection |\childdocforward{|\textit{dest}|}| is achieved by:
%
\begin{center}
|\def\jobname{|\textit{dest}|}\input{\jobname}|
\end{center}
%
The redirection with prefix
|\childdocforwardprefix[|\textit{prefix}|]{|\textit{dest}|}|
is accomplished by:
%
\begin{center}
\begin{tabular}{l}
|{\edef\jobname{\scantokens\expandafter{\jobname\noexpand}}|\\
|\def\redirectjob |\textit{prefix}|#1~~~{\gdef\jobname{|\textit{dest}|#1}}|\\
|\expandafter\redirectjob\jobname~~~}\input{\jobname}|
\end{tabular}
\end{center}

In an alternative approach,
child documents can be compiled by a specific command line
without additional code or specific definitions:
%
\begin{center}
|... -jobname "|\textit{target}|" "|[\textit{flags}]%
|\includeonly{|\textit{dest}|}\input{|\textit{main}|}"|
\end{center}
%

%%%%%%%%%%%%%%%%%%%%%%%%%%%%%%%%%%%%%%%%%%%%%%%%%%%%%%%%%%%%%%%%%%%%%%%%%%%%%%%%
%%%%%%%%%%%%%%%%%%%%%%%%%%%%%%%%%%%%%%%%%%%%%%%%%%%%%%%%%%%%%%%%%%%%%%%%%%%%%%%%
\section{Information}

%%%%%%%%%%%%%%%%%%%%%%%%%%%%%%%%%%%%%%%%%%%%%%%%%%%%%%%%%%%%%%%%%%%%%%%%%%%%%%%%
\subsection{Copyright}

Copyright \copyright{} 2017--2018 Niklas Beisert

This work may be distributed and/or modified under the
conditions of the \LaTeX{} Project Public License, either version 1.3
of this license or (at your option) any later version.
The latest version of this license is in
  \url{http://www.latex-project.org/lppl.txt}
and version 1.3 or later is part of all distributions of \LaTeX{}
version 2005/12/01 or later.

This work has the LPPL maintenance status `maintained'.

The Current Maintainer of this work is Niklas Beisert.

This work consists of the files |README.txt|, |childdoc.ins| and |childdoc.dtx|
as well as the derived files |childdoc.def|, |cdocsamp.tex|
with |cdocsch1.tex|, |cdocsch2.tex|, |cdocspt3.tex|, |cdocspt4.tex|,
|cdocsdrf.tex|, |cdocsfn1.tex|, |cdocsfn2.tex|
as well as |childdoc.pdf|.

%%%%%%%%%%%%%%%%%%%%%%%%%%%%%%%%%%%%%%%%%%%%%%%%%%%%%%%%%%%%%%%%%%%%%%%%%%%%%%%%
\subsection{Files and Installation}

The package consists of the files:
%
\begin{center}
\begin{tabular}{ll}
    |README.txt|   & readme file \\
    |childdoc.ins| & installation file \\
    |childdoc.dtx| & source file \\
    |childdoc.def| & definition file \\
    |cdocsamp.tex| & sample main file \\
    |cdocsch1.tex| & sample include file \\
    |cdocsch2.tex| & sample include file \\
    |cdocspt3.tex| & sample part file \\
    |cdocspt4.tex| & sample part file \\
    |cdocsdrf.tex| & sample redirection file \\
    |cdocsfn1.tex| & sample redirection file \\
    |cdocsfn2.tex| & sample redirection file \\
    |childdoc.pdf| & manual
\end{tabular}
\end{center}
%
The distribution consists of the files
|README.txt|, |childdoc.ins| and |childdoc.dtx|.
%
\begin{itemize}
\item
Run (pdf)\LaTeX{} on |childdoc.dtx|
to compile the manual |childdoc.pdf| (this file).
\item
Run \LaTeX{} on |childdoc.ins| to create the definitions file |childdoc.def|
and the sample |cdocsamp.tex| with include files
|cdocsch1.tex|, |cdocsch2.tex|, |cdocspt3.tex|, |cdocspt4.tex|,
|cdocsdrf.tex|, |cdocsfn1.tex|, |cdocsfn2.tex|.
Then copy the file |childdoc.def| to an appropriate directory of your \LaTeX{}
distribution, e.g.\ \textit{texmf-root}|/tex/latex/childdoc|.
\end{itemize}

%%%%%%%%%%%%%%%%%%%%%%%%%%%%%%%%%%%%%%%%%%%%%%%%%%%%%%%%%%%%%%%%%%%%%%%%%%%%%%%%
\subsection{Related CTAN Packages}

There are several other packages which offer a similar functionality:
%
\begin{itemize}
\item
The packages
\href{http://ctan.org/pkg/docmute}{\textsf{docmute}},
\href{http://ctan.org/pkg/includex}{\textsf{includex}} and
\href{http://ctan.org/pkg/standalone}{\textsf{standalone}}
provide commands to include only the document body of
a child file thus allowing both files to be compiled individually.
\item
The packages \href{http://ctan.org/pkg/subdocs}{\textsf{subdocs}}
and \href{http://ctan.org/pkg/subfiles}{\textsf{subfiles}}
provide structures in which the main and child documents can be
encapsulated and allowing them to be compiled individually.
The inclusion mechanism is different from the conventional |\include|.
\item
The package \href{http://ctan.org/pkg/combine}{\textsf{combine}}
is an elaborate solution to combine several documents into one.
\end{itemize}
%
See also the CTAN topic \href{http://ctan.org/topic/subdocs}{\textsf{subdocs}}
for further related packages.
The present package differs from the above solutions in that
a document structure constructed with the conventional |\include| mechanism
just needs two extra commands at the top of every file
such that all constituent files can be compiled individually.

%%%%%%%%%%%%%%%%%%%%%%%%%%%%%%%%%%%%%%%%%%%%%%%%%%%%%%%%%%%%%%%%%%%%%%%%%%%%%%%%
%\subsection{Feature Suggestions}
%
%The following is a list of features which may be useful for future
%versions of this package:
%%
%\begin{itemize}
%\item
%\ldots
%\end{itemize}

%%%%%%%%%%%%%%%%%%%%%%%%%%%%%%%%%%%%%%%%%%%%%%%%%%%%%%%%%%%%%%%%%%%%%%%%%%%%%%%%
\subsection{Revision History}

%%%%%%%%%%%%%%%%%%%%%%%%%%%%%%%%%%%%%%%%
\paragraph{v2.0:} 2018/12/30

\begin{itemize}
\item
immediate forward processing
\item
added |\childdocby| mechanism
\item
manual restructured
\end{itemize}

%%%%%%%%%%%%%%%%%%%%%%%%%%%%%%%%%%%%%%%%
\paragraph{v1.6:} 2018/01/17

\begin{itemize}
\item
application for development of include files
\item
corrections to manual
\end{itemize}

%%%%%%%%%%%%%%%%%%%%%%%%%%%%%%%%%%%%%%%%
\paragraph{v1.5:} 2017/05/21

\begin{itemize}
\item
more complete structuring introduced
\item
|\childdocof| introduced
\item
|\childdoc| renamed to |\childdocmain|
\item
|\childredirect| renamed to |\childdocforward| and |\childdocforwardprefix|
and functionality expanded
\end{itemize}

%%%%%%%%%%%%%%%%%%%%%%%%%%%%%%%%%%%%%%%%
\paragraph{v1.0:} 2017/04/27

\begin{itemize}
\item
manual and install package
\item
first version published on CTAN
\end{itemize}

%%%%%%%%%%%%%%%%%%%%%%%%%%%%%%%%%%%%%%%%
\paragraph{v0.6:} 2017/04/26

\begin{itemize}
\item
redirection mechanism added
\end{itemize}

%%%%%%%%%%%%%%%%%%%%%%%%%%%%%%%%%%%%%%%%
\paragraph{v0.5:} 2017/04/26

\begin{itemize}
\item
functionality in definition file
\end{itemize}


%%%%%%%%%%%%%%%%%%%%%%%%%%%%%%%%%%%%%%%%%%%%%%%%%%%%%%%%%%%%%%%%%%%%%%%%%%%%%%%%
%%%%%%%%%%%%%%%%%%%%%%%%%%%%%%%%%%%%%%%%%%%%%%%%%%%%%%%%%%%%%%%%%%%%%%%%%%%%%%%%
%%%%%%%%%%%%%%%%%%%%%%%%%%%%%%%%%%%%%%%%%%%%%%%%%%%%%%%%%%%%%%%%%%%%%%%%%%%%%%%%
\appendix

\settowidth\MacroIndent{\rmfamily\scriptsize 000\ }

 \DocInput{childdoc.dtx}

\end{document}
%</driver>
% \fi
%
% %%%%%%%%%%%%%%%%%%%%%%%%%%%%%%%%%%%%%%%%%%%%%%%%%%%%%%%%%%%%%%%%%%%%%%%%%%%%%%
% %%%%%%%%%%%%%%%%%%%%%%%%%%%%%%%%%%%%%%%%%%%%%%%%%%%%%%%%%%%%%%%%%%%%%%%%%%%%%%
% \section{Sample}
%\iffalse
%<*samplemain>
%\fi
%
% The following presents a sample document
% with two chapters, two parts, a title page,
% a compile flag as well as three forwarding files to set the flag.
% It consists of eight |.tex| files:
% \begin{center}
% \begin{tabular}{ll}
% |cdocsamp.tex|&main file\\
% |cdocsch1.tex|&include file for chapter 1\\
% |cdocsch2.tex|&include file for chapter 2\\
% |cdocspt3.tex|&include file for part 3\\
% |cdocspt4.tex|&include file for part 4\\
% |cdocsdrf.tex|&forwarding file for main file in draft mode\\
% |cdocsfi1.tex|&forwarding file for final version of chapter 1\\
% |cdocsfi2.tex|&forwarding file for final version of chapter 2\\
% \end{tabular}
% \end{center}
% Each of the eight files can be compiled directly by the \LaTeX{} compiler.
%
% %%%%%%%%%%%%%%%%%%%%%%%%%%%%%%%%%%%%%%
% \paragraph{Main File.}
%
% The main file is called |cdocsamp.tex|.
%
% Load the \textsf{childdoc} definitions and
% declare the filename for the main document:
%    \begin{macrocode}
\input{childdoc.def}
\childdocmain{}
%    \end{macrocode}

% Optional override for |\version| flag:
%    \begin{macrocode}
%%\ifchilddoc\else\providecommand{\version}{draft}\fi
%    \end{macrocode}

% Define the default values for the |\version| flag
% (|final| for the main file and |draft| for childs):
%    \begin{macrocode}
\ifchilddoc
\providecommand{\version}{draft}
\else
\providecommand{\version}{final}
\fi
%    \end{macrocode}

% Load the standard document class:
%    \begin{macrocode}
\documentclass[12pt]{article}
%    \end{macrocode}

% Start the document body:
%    \begin{macrocode}
\begin{document}
%    \end{macrocode}

% Declare a title page.
% Print title, part of document being processed and version flag:
%    \begin{macrocode}
\addtocounter{page}{-1}
\begin{center}
{\LARGE\bfseries{}childdoc example\par}
\vspace{1cm}
\ifchilddoc
\ifchilddocmanual part\else chapter\fi:
`\childdocname' of `\childdocjob'\par
\else
main document: `\childdocjob'\par
\fi
version: \version\par
\end{center}
\newpage
%    \end{macrocode}

% Manually include selected file,
% otherwise process as usual:
%    \begin{macrocode}
\ifchilddocmanual
\section*{part `\childdocname'}
\input{\childdocname}
\else
%    \end{macrocode}

% Include the two chapters:
%    \begin{macrocode}
\include{cdocsch1}
\include{cdocsch2}
%    \end{macrocode}

% Include the two parts unless only chapters should be displayed:
%    \begin{macrocode}
\ifchilddoc\else
\section{part three}
\input{cdocspt3}
\section{part four}
\input{cdocspt4}
\fi
%    \end{macrocode}

% Process as usual until here:
%    \begin{macrocode}
\fi
%    \end{macrocode}

% End of document body:
%    \begin{macrocode}
\end{document}
%    \end{macrocode}
%\iffalse
%</samplemain>
%\fi
%
% %%%%%%%%%%%%%%%%%%%%%%%%%%%%%%%%%%%%%%
% \paragraph{Chapter Include Files.}
%
% The include files are called |cdocsch1.tex| and |cdocsch2.tex|.
%
%\iffalse
%<*samplechap1|samplechap2>
%\fi

% Optional override for |\version| flag:
%    \begin{macrocode}
%%\providecommand{\version}{final}
%    \end{macrocode}

% Include the main document:
%    \begin{macrocode}
\input{childdoc.def}
\childdocof{cdocsamp}
%    \end{macrocode}

%\iffalse
%</samplechap1|samplechap2>
%\fi
%
%\iffalse
%<*samplechap1>
%\fi
% Some text for chapter 1:
%    \begin{macrocode}
\section{one}
some text in chapter one
%    \end{macrocode}

%\iffalse
%</samplechap1>
%\fi
% Some text for chapter 2:
%\iffalse
%<*samplechap2>
%\fi
%    \begin{macrocode}
\section{two}
more text in chapter two
%    \end{macrocode}

%\iffalse
%</samplechap2>
%\fi
%
% %%%%%%%%%%%%%%%%%%%%%%%%%%%%%%%%%%%%%%
% \paragraph{Part Include Files.}
%
% The include files are called |cdocspt3.tex| and |cdocspt4.tex|.
%
%\iffalse
%<*samplepart3|samplepart4>
%\fi

% Optional override for |\version| flag:
%    \begin{macrocode}
%%\providecommand{\version}{final}
%    \end{macrocode}

% Include the main document:
%    \begin{macrocode}
\input{childdoc.def}
\childdocby{cdocsamp}
%    \end{macrocode}

%\iffalse
%</samplepart3|samplepart4>
%\fi
%
%\iffalse
%<*samplepart3>
%\fi
% Some text for part 3:
%    \begin{macrocode}
some text in part three
%    \end{macrocode}

%\iffalse
%</samplepart3>
%\fi
% Some text for part 4:
%\iffalse
%<*samplepart4>
%\fi
%    \begin{macrocode}
more text in part four
%    \end{macrocode}

%\iffalse
%</samplepart4>
%\fi
%
% %%%%%%%%%%%%%%%%%%%%%%%%%%%%%%%%%%%%%%
% \paragraph{Forwarding for a Complete Draft.}
%
% The following forwarding file |cdocsdrf.tex|
% compiles the main document in draft mode:
%\iffalse
%<*sampledraft>
%\fi
%    \begin{macrocode}
\def\version{draft}
\input{childdoc.def}
\childdocforward{cdocsamp}
%    \end{macrocode}

%\iffalse
%</sampledraft>
%\fi
%
% %%%%%%%%%%%%%%%%%%%%%%%%%%%%%%%%%%%%%%
% \paragraph{Forwarding for Final Version of the Chapters.}
%
% The following forwarding files |cdocsfn1.tex| and |cdocsfn2.tex|
% (with identical content)
% compile the final versions of the child documents
% |cdocsch1.tex| and |cdocsch2.tex|, respectively:
%\iffalse
%<*samplefinal>
%\fi
%    \begin{macrocode}
\def\version{final}
\input{childdoc.def}
\childdocforwardprefix[cdocsamp]{cdocsfn}{cdocsch}
%    \end{macrocode}

%\iffalse
%</samplefinal>
%\fi
%
% %%%%%%%%%%%%%%%%%%%%%%%%%%%%%%%%%%%%%%
% \paragraph{Command Line Processing.}
%
% The following three command lines generate the output files
% |cdocscld|, |cdocscl1| and |cdocscl2|
% which should be identical to
% |cdocsdrf|, |cdocsch1| and |cdocsfn2|, respectively:
% \begin{center}
% \begin{tabular}{l}
% |latex -jobname cdocscld \|\\
% |  "\def\version{draft}\input{childdoc.def}\childdocforward{cdocsamp}"|\\
% |latex -jobname cdocscl1 \|\\
% |  "\input{childdoc.def}\childdocforward[cdocsamp]{cdocsch1}"|\\
% |latex -jobname cdocscl2 \|\\
% |  "\def\version{final}\input{childdoc.def}\childdocforward{cdocsch2}"|
% \end{tabular}
% \end{center}
% Note that the trailing backslash on each first line
% merely continues the input to the second line
% (for convenient cut ant paste).
% Furthermore, the command |latex| can be replaced by any
% of its alternative versions such as |pdflatex|.
%
% %%%%%%%%%%%%%%%%%%%%%%%%%%%%%%%%%%%%%%%%%%%%%%%%%%%%%%%%%%%%%%%%%%%%%%%%%%%%%%
% %%%%%%%%%%%%%%%%%%%%%%%%%%%%%%%%%%%%%%%%%%%%%%%%%%%%%%%%%%%%%%%%%%%%%%%%%%%%%%
% \section{Implementation}
%\iffalse
%<*package>
%\fi
%
% This section describes the definitions file |childdoc.def|.

% The definitions cannot be loaded using |\usepackage| or |\RequirePackage|
% which has a mechanism to prevent loading a style file more than once.
% When loading the definitions by means of |\input|
% multiple instances have to be prevented manually:
%\iffalse
%This code needs to be before the `\ProvidesFile' directive
%which is defined at the beginning of this file.
%Therefore it is also placed there and commented out here.
%</package>
%<*discard>
%\fi
%    \begin{macrocode}
\ifdefined\childdocmain\endinput\fi
%    \end{macrocode}
%\iffalse
%</discard>
%<*package>
%\fi
%
% \macro{\ifchilddoc}
% \macro{\ifchilddocmanual}
% The conditional |\ifchilddoc| tells whether a
% child (true) or main (false) document is being compiled.
% The conditional |\ifchilddocmanual| tells whether
% the |\includeonly| mechanism is used (false) or
% the selection of child files must be performed manually (true).
% The definitions initialise to false:
%    \begin{macrocode}
\newif\ifchilddoc
\newif\ifchilddocmanual
%    \end{macrocode}

% \macro{\childdocname}
% \macro{\childdocjob}
% The macro |\childdocname| stores the name of the main document
% to be compiled. The macro |\childdocjob| stores the name of
% the document on which the \LaTeX{} compiler was originally invoked.
% The content of |\jobname| cannot be compared
% to filenames specified in the source due to different catcodes.
% The following code rescans |\jobname|, stores the result
% in |\childdocname| and saves a copy in |\childdocjob|:
%    \begin{macrocode}
\edef\childdocname{\scantokens\expandafter{\jobname\noexpand}}
\let\childdocjob\childdocname
%    \end{macrocode}

% \macro{\childdocdisable}
% The macro |\childdocdisable| prevents the main file
% from being processed more than once.
% At this stage, the main document command |\childdocmain|
% is assumed to be called once again where it should do nothing.
% Any subsequent call to it should prevent
% a secondary processing of the main document
% It overwrites the forwarding commands
% |\childdocof| and |\childdocforward|
% with empty macros to prevent further inclusions of the main document:
%    \begin{macrocode}
\newcommand{\childdocdisable}
{
  \renewcommand{\childdocmain}[1]{\renewcommand{\childdocmain}[1]{\endinput}}
  \renewcommand{\childdocof}[1]{}
  \renewcommand{\childdocby}[2][]{}
  \renewcommand{\childdocforward}[2][]{}
  \renewcommand{\childdocdisable}{}
}
%    \end{macrocode}

% \macro{\childdocmain}
% The macro |\childdocmain| is to be called at the top of the main file
% with nothing or the main filename (without extension) as argument.
% First, it breaks loops.
% If the argument is not empty and does not match |\childdocname|
% (which is set by the first inclusion of |childdoc.def|),
% |\ifchilddoc| is set to true, |\includeonly| is applied to the child file
% and |\jobname| is set to the main file
% (for proper handling of |.aux| files):
%    \begin{macrocode}
\newcommand{\childdocmain}[1]
{
  \childdocdisable\childdocmain{}
  \if?#1?\else
    \begingroup
      \def\childdoctmp{#1}
      \ifx\childdoctmp\childdocname
        \def\childdoctmp{}
      \else
        \def\childdoctmp
        {
          \childdoctrue
          \includeonly{\childdocname}
          \def\childdocjob{#1}
          \def\jobname{#1}
        }
      \fi
      \expandafter
    \endgroup
    \childdoctmp
  \fi
}
%    \end{macrocode}

% \macro{\childdocof}
% The command |\childdocof| redirects
% compilation to the main file |#1|.
%    \begin{macrocode}
\newcommand{\childdocof}[1]
{
  \childdocdisable
  \childdoctrue
  \includeonly{\childdocname}
  \def\jobname{#1}
  \def\childdocjob{#1}
  \input{#1}
}
%    \end{macrocode}

% \macro{\childdocby}
% The command |\childdocby| ....
%    \begin{macrocode}
\newcommand{\childdocby}[2][]
{
  \childdocdisable
  \childdoctrue
  \childdocmanualtrue
  \if?#1?\else
    \def\jobname{#2}
  \fi
  \def\childdocjob{#2}
  \input{#2}
  \endinput
}
%    \end{macrocode}

% \macro{\childdocforward}
% The command |\childdocforward| redirects
% compilation to the main file or
% (if the optional argument is given) a child file.
% Parameters are set as if the main file
% or a child file starting with |\childdocof| was compiled.
% Then compilation is handed over to the main file:
%    \begin{macrocode}
\newcommand{\childdocforward}[2][]
{
  \begingroup
    \if?#1?
      \def\childdoctmp
      {
        \def\childdocname{#2}
        \def\childdocjob{#2}
        \def\jobname{#2}
        \input{#2}
        \endinput
      }
    \else
      \def\childdoctmp
      {
        \childdocdisable
        \def\childdocname{#2}
        \childdoctrue
        \includeonly{#2}
        \def\childdocjob{#1}
        \def\jobname{#1}
        \input{#1}
        \endinput
      }
    \fi
    \expandafter
  \endgroup
  \childdoctmp
}
%    \end{macrocode}

% \macro{\childdocforwardprefix}
% The command |\childdocforwardprefix| redirects
% compilation to the main or a child file by means of a pattern.
% The prefix |#1| in the current filename is replaced by |#2|
% and the suffix of the current filename is kept
% (it is assumed that the filename does not contain the substring `|~~~|'
% which is used as a delimiter).
% Compilation is handed over to the new file by |\childdocforward|:
%    \begin{macrocode}
\newcommand{\childdocforwardprefix}[3][]
{
  \begingroup
    \def\childdocextract #2##1~~~{\def\childdoctmp{\childdocforward[#1]{#3##1}}}
    \expandafter\childdocextract\childdocname~~~
    \expandafter
  \endgroup
  \childdoctmp
}
%    \end{macrocode}

% \macro{\childdoc}
% The deprecated macro |\childdoc| is a legacy version of |\childdocmain|:
%    \begin{macrocode}
\newcommand{\childdoc}{\childdocmain}
%    \end{macrocode}

% \macro{\childdocredirect}
% The deprecated macro |\childdocredirect| is a legacy version
% of |\childdocforward| and |\childdocforwardprefix|:
%    \begin{macrocode}
\newcommand{\childdocredirect}[2][]
{
  \begingroup
    \if?#1?
      \def\childdoctmp{\childdocforward{#2}}
    \else
      \def\childdoctmp{\childdocforwardprefix{#1}{#2}}
    \fi
    \expandafter
  \endgroup
  \childdoctmp
}
%    \end{macrocode}

%\iffalse
%</package>
%\fi
%
\endinput
\childdocforward{cdocsamp}"|\\
% |latex -jobname cdocscl1 \|\\
% |  "% \iffalse
%
% childdoc.dtx Copyright (C) 2017-2018 Niklas Beisert
%
% This work may be distributed and/or modified under the
% conditions of the LaTeX Project Public License, either version 1.3
% of this license or (at your option) any later version.
% The latest version of this license is in
%   http://www.latex-project.org/lppl.txt
% and version 1.3 or later is part of all distributions of LaTeX
% version 2005/12/01 or later.
%
% This work has the LPPL maintenance status `maintained'.
%
% The Current Maintainer of this work is Niklas Beisert.
%
% This work consists of the files childdoc.dtx and childdoc.ins
% and the derived files childdoc.def and cdocsamp.tex with
% cdocsch1.tex, cdocsch2.tex, cdocsdrf.tex, cdocsfn1.tex, cdocsfn2.tex.
%
%<package>\ifdefined\childdocmain\endinput\fi
%<package>\ProvidesFile{childdoc.def}[2018/12/30 v2.0 child document driver]
%<samplemain>\ProvidesFile{cdocsamp.tex}[2018/12/30 v2.0 sample for childdoc]
%<*driver>
%\ProvidesFile{childdoc.drv}[2018/12/30 v2.0 childdoc reference manual file]
\PassOptionsToClass{10pt,a4paper}{article}
\documentclass{ltxdoc}

\usepackage[margin=35mm]{geometry}
\usepackage{hyperref}
\usepackage{hyperxmp}
\usepackage[usenames]{color}

\hypersetup{colorlinks=true}
\hypersetup{pdfstartview=FitH}
\hypersetup{pdfpagemode=UseNone}
\hypersetup{pdfsource={}}
\hypersetup{pdflang={en-UK}}
\hypersetup{pdfcopyright={Copyright 2017-2018 Niklas Beisert.
  This work may be distributed and/or modified under the
  conditions of the LaTeX Project Public License, either version 1.3
  of this license or (at your option) any later version.}}
\hypersetup{pdflicenseurl={http://www.latex-project.org/lppl.txt}}
\hypersetup{pdfcontactaddress={ETH Zurich, ITP, HIT K,
  Wolfgang-Pauli-Strasse 27}}
\hypersetup{pdfcontactpostcode={8093}}
\hypersetup{pdfcontactcity={Zurich}}
\hypersetup{pdfcontactcountry={Switzerland}}
\hypersetup{pdfcontactemail={nbeisert@itp.phys.ethz.ch}}
\hypersetup{pdfcontacturl={http://people.phys.ethz.ch/\xmptilde nbeisert/}}

\newcommand{\secref}[1]{\hyperref[#1]{section \ref*{#1}}}

\parskip1ex
\parindent0pt
\let\olditemize\itemize
\def\itemize{\olditemize\parskip0pt}

\begin{document}

\title{The \textsf{childdoc} Package}
\hypersetup{pdftitle={The childdoc Package}}
\author{Niklas Beisert\\[2ex]
  Institut f\"ur Theoretische Physik\\
  Eidgen\"ossische Technische Hochschule Z\"urich\\
  Wolfgang-Pauli-Strasse 27, 8093 Z\"urich, Switzerland\\[1ex]
  \href{mailto:nbeisert@itp.phys.ethz.ch}
  {\texttt{nbeisert@itp.phys.ethz.ch}}}
\hypersetup{pdfauthor={Niklas Beisert}}
\hypersetup{pdfsubject={Manual for the LaTeX2e Package childdoc}}
\date{30 December 2018, \textsf{v2.0}}
\maketitle

\begin{abstract}\noindent
\textsf{childdoc} is a \LaTeXe{} package
that enables the direct compilation
of document sections included by |\include|
to individual files.
\end{abstract}

\begingroup
\parskip0ex
\tableofcontents
\endgroup

%%%%%%%%%%%%%%%%%%%%%%%%%%%%%%%%%%%%%%%%%%%%%%%%%%%%%%%%%%%%%%%%%%%%%%%%%%%%%%%%
%%%%%%%%%%%%%%%%%%%%%%%%%%%%%%%%%%%%%%%%%%%%%%%%%%%%%%%%%%%%%%%%%%%%%%%%%%%%%%%%
\section{Introduction}

\LaTeX{} provides a mechanism to structure a large document (such as a book)
into a main file and several child files (containing the chapters)
using the |\include| command.
This mechanism is beneficial for documents
which span hundreds of pages in order to
make the source file(s) more manageable.
Moreover, compilation can be restricted to
selected child files by means of the |\includeonly| command.
The latter feature can be used to reduce the compilation time while editing
(this was significantly more useful in the earlier days of \LaTeX{})
or to generate a smaller document which is easier to navigate.
Another application of |\includeonly| is to generate
documents consisting of selected parts of the complete document.

However, there are a few drawbacks of the plain |\include| mechanism:
\begin{itemize}
\item
The child files cannot be compiled on their own,
they can only be compiled via the main file.
A naive editing environment
(such as a text editor with an option
to have the current file processed by \LaTeX)
may require one to switch to the main file before compiling;
attempting to compile the child file produces errors.
\item
The main file must be modified (each time)
to adjust the |\includeonly| command
to the present needs. This easily leaves the main file in a messy state.
\item
The generated document will always carry the filename
of the main document. This is inconvenient if
several child files are to be compiled and
to be kept for distribution.
\end{itemize}

The present package provides a simple interface
to make child files individually compilable by \LaTeX{}.
Compiling a child file then has the same effect as compiling
the main file with an |\includeonly| command
to select the appropriate child.
Moreover the generated document will carry the name of the child
rather than the main file.
This resolves all three above issues.

This feature is meant to make the editing of books,
thesis documents and lecture notes somewhat more convenient.
However, the package can also be used efficiently for
composing a series of documents (such as exercise sheets)
which are typically distributed individually.
It then assists the author in generating the individual documents
(potentially in different versions)
as well as a document containing the collected series.
Another application is in developing style files
or other kinds of included material
where compilation of the style file could redirect
to a sample or test file.

%%%%%%%%%%%%%%%%%%%%%%%%%%%%%%%%%%%%%%%%%%%%%%%%%%%%%%%%%%%%%%%%%%%%%%%%%%%%%%%%
%%%%%%%%%%%%%%%%%%%%%%%%%%%%%%%%%%%%%%%%%%%%%%%%%%%%%%%%%%%%%%%%%%%%%%%%%%%%%%%%
\section{Usage}

First of all, the package \textsf{childdoc} is \emph{not} a standard
\LaTeXe{} |.sty| style file! Therefore it needs to be invoked in
a non-standard way.

%%%%%%%%%%%%%%%%%%%%%%%%%%%%%%%%%%%%%%%%%%%%%%%%%%%%%%%%%%%%%%%%%%%%%%%%%%%%%%%%
\subsection{Included Files}
\label{sec:include}

%%%%%%%%%%%%%%%%%%%%%%%%%%%%%%%%%%%%%%%%
\DescribeMacro{\childdocmain}
To use the package, add the commands
\begin{center}
\begin{tabular}{l}
|\input{childdoc.def}|\\
|\childdocmain{}|\\
\end{tabular}
\end{center}
at the very top of the main \LaTeX{} file,
in particular \emph{before} the |\documentclass| statement!
The argument of |\childdocmain| should be left empty
(but it must be present).

%%%%%%%%%%%%%%%%%%%%%%%%%%%%%%%%%%%%%%%%
\DescribeMacro{\childdocof}
Furthermore, add the commands
\begin{center}
\begin{tabular}{l}
|\input{childdoc.def}|\\
|\childdocof{|\textit{main}|}|\\
\end{tabular}
\end{center}
at the top of every child file \textit{child}
which is included by |\include{|\textit{child}|}|
from within the main file
(or at least for those files to be compiled individually).
The argument \textit{main} must be the filename of the main file.

There are a couple of
considerations in setting up the main and child documents:

%%%%%%%%%%%%%%%%%%%%%%%%%%%%%%%%%%%%%%%%
\paragraph{Restrictions.}

Please note the following restrictions:
\begin{itemize}
\item
|\childdocmain| must be called with one argument \textit{main}
to ensure compatibility with earlier version of the package.
It must either be empty (|\childdocmain{}|)
or precisely match the filename of the main file in which it is specified.
See \secref{sec:detection} for further information.
\item
The filename \textit{main} must be specified without the |.tex| extension.
\item
The filename \textit{main} is case sensitive
(even in case-insensitive file systems)
due to internal string comparison.
\item
The argument \textit{main} should be fully expanded, it cannot be a macro.
\item
Subdirectories and special characters should be avoided in filenames.
\item
The command |\childdocmain{|\textit{main}|}| must be followed by a whitespace.
It should not be followed immediately by another command
or by a comment mark `|%|'.
This is because the \TeX{} parser reads the token immediately following
the argument of |\childdocmain| and puts it
at the beginning of every child section;
however, a white\-space is ignored.
\end{itemize}

%%%%%%%%%%%%%%%%%%%%%%%%%%%%%%%%%%%%%%%%
\paragraph{Content of Main File.}

It is advisable to place all content in the child files included by |\include|.
Any output contained in the main file will appear in all child documents
unless suppressed manually;
it cannot be suppressed automatically by the |\includeonly| directive
and thus should normally be avoided.
A method to include some content in the main file
by means of conditional processing is described in \secref{sec:conditional}.

%%%%%%%%%%%%%%%%%%%%%%%%%%%%%%%%%%%%%%%%
\paragraph{Page Numbering.}

When only a part of the document is compiled,
the appropriate numbering of pages
(as well as other status parameters)
is determined from the |.aux| files.
The latter contain information from previous passes.
However this information needs to propagate through
all intermediate child documents.
Therefore the page numbering in child documents may well
be inconsistent until the complete document is compiled at least once.

A useful (if unconventional) way to always ensure a consistent
page numbering is to restart the numbering in each child document
and denote the pages by `\textit{child}|.|\textit{page}'
where \textit{child} represents the chapter/section number of the child file.
This can be achieved by the command
|\numberwithin{page}{|\textit{child}|}|
of the \textsf{amsmath} package
where \textit{child} can be |chapter| or |section|
depending on the chosen structuring.
Alternatively, one can modify the macro |\thepage| appropriately
and reset the counter |page| at the start of each child file.

%%%%%%%%%%%%%%%%%%%%%%%%%%%%%%%%%%%%%%%%%%%%%%%%%%%%%%%%%%%%%%%%%%%%%%%%%%%%%%%%
\subsection{Conditional Processing}
\label{sec:conditional}

The package provides a mechanism to compile different versions
of a document. To customise the versions further some conditional processing
can come in handy to distinguish which version is being compiled.
The package provides two macros to describe the compilation context:

%%%%%%%%%%%%%%%%%%%%%%%%%%%%%%%%%%%%%%%%
\DescribeMacro{\ifchilddoc}
The conditional |\ifchilddoc| distinguishes between the compilation of
child documents and the main document:
%
\begin{center}
|\ifchilddoc |\textit{child-code}| |[|\||else |\textit{main-code}]| \||fi|
\end{center}

%%%%%%%%%%%%%%%%%%%%%%%%%%%%%%%%%%%%%%%%
\DescribeMacro{\childdocname}
\DescribeMacro{\childdocjob}
The macro |\childdocname| contains the filename (without extension)
of the main or child file being processed.
Note that |\childdocjob| will always contain the name of the main file.

%%%%%%%%%%%%%%%%%%%%%%%%%%%%%%%%%%%%%%%%
\paragraph{Title Page.}

Conditional processing can be used to include a title or banner page
in the main document when proper precautions are taken.
Importantly, the code in the main file should ensure that the page counter
(as well as other status parameters which are stored in the |.aux| files)
takes the same value after the conditional processing.
Otherwise the page numbers may take divergent values
depending on which part is compiled.

For example, a title page could be declared by:
%
\begin{center}
\begin{tabular}{l}
|\ifchilddoc\||else|\\
|\addtocounter{page}{-1}|\\
\textit{code for title page}\\
|\newpage|\\
|\||fi|
\end{tabular}
\end{center}
%
A banner page for the child documents can be generated by:
%
\begin{center}
\begin{tabular}{l}
|\ifchilddoc|\\
|\addtocounter{page}{-1}|\\
\textit{code for banner page}\\
|\newpage|\\
|\||fi|
\end{tabular}
\end{center}
%
Here one could write a message such as:
\begin{center}
|This is the part \childdocname{} of \childdocjob{}.|
\end{center}

%%%%%%%%%%%%%%%%%%%%%%%%%%%%%%%%%%%%%%%%%%%%%%%%%%%%%%%%%%%%%%%%%%%%%%%%%%%%%%%%
\subsection{Flags}
\label{sec:flags}

The package makes it easy to generate different versions
of the main or child documents.
To this end compilation flags can be defined
and assigned different default values.
They will be particularly useful in conjunction
with the forwarding mechanism described in \secref{sec:forward}.

For example, it may be useful to have a flag |\version|
which can be set to |draft| or |final|.
The document source will contain some conditional code
depending on the value of |\version|.
Suppose further, the flag should default to |final| for the main file
and to |draft| for child files
which is a natural assignment for editing the document.
This is achieved by placing the following code
in the preamble of the main document
(below the |\childdocmain| directive):
%
\begin{center}
\begin{tabular}{l}
|\ifchilddoc|\\
|\providecommand{\version}{draft}|\\
|\||else|\\
|\providecommand{\version}{final}|\\
|\||fi|
\end{tabular}
\end{center}
%
The definition by |\providecommand| makes sure
that previous definitions are not overwritten.
Further statements |\providecommand{\version}{...}|
can thus be added before the above code to override it.

For the main file, one might add a line
(between |\childdocmain| and the above block)
%
\begin{center}
|%\ifchilddoc\||else\providecommand{\version}{draft}\||fi|
\end{center}
%
which can be uncommented to produce a draft version.
Likewise one can add a line to the very top of a child file
(above the |\childdocof{|\textit{main}|}| directive)
%
\begin{center}
|%\providecommand{\version}{final}|
\end{center}
%
which can be uncommented to produce the final version of this child document.

%%%%%%%%%%%%%%%%%%%%%%%%%%%%%%%%%%%%%%%%%%%%%%%%%%%%%%%%%%%%%%%%%%%%%%%%%%%%%%%%
\subsection{Forwarding}
\label{sec:forward}

Different versions of the main or child documents
using compilation flags as described in \secref{sec:flags}
can be (permanently) stored in different files
for convenient compilation, viewing and distribution.
To this end, the package defines a command
to pass on compilation to a different file:

%%%%%%%%%%%%%%%%%%%%%%%%%%%%%%%%%%%%%%%%
\DescribeMacro{\childdocforward}
The command |\childdocforward| redirects processing to
another source file:
%
\begin{center}
\begin{tabular}{l}
|\input{childdoc.def}|\\
|\childdocforward[|\textit{main}|]{|\textit{dest}|}|\\
\end{tabular}
\end{center}
%
The argument \textit{dest} is the destination file
(without extension).
It should be the main file or one of the child files.
Note that further \textsf{childdoc} directives
such as |\childdocof| and |\childdocforward|
in the indicated file will be processed in this form.
The optional argument \textit{main}
passes on directly to the main file \textit{main}
while pretending to compile the child \textit{dest}.
This form behaves as if \textit{dest}
issues |\childdocof{|\textit{main}|}| right away,
and no further \textsf{childdoc} directives will be processed.

%%%%%%%%%%%%%%%%%%%%%%%%%%%%%%%%%%%%%%%%
\DescribeMacro{\...prefix}
In the alternative form |\childdocforwardprefix|,
%
\begin{center}
\begin{tabular}{l}
|\input{childdoc.def}|\\
|\childdocforwardprefix[|\textit{main}|]{|\textit{prefix}|}{|\textit{dest}|}|
\end{tabular}
\end{center}
%
the destination file is determined by a pattern
depending on the current file:
To make this work, the current file must be called
`{\textit{prefix}\hspace{0.2em}\textit{suffix}}'
with \textit{prefix} matching precisely the argument.
Processing is then passed on to the file
`{\textit{dest}\hspace{0.2em}\textit{suffix}}'.
Surely, the same effect is achieved by
directly specifying the
argument `{\textit{dest}\hspace{0.2em}\textit{suffix}}'
in the first form.
However, that requires to set up a different file
for each child. With the alternative form of the command
all these files can have exactly the same content
which simplifies setting them up and maintaining them.

For example, the following file |draft.tex|
with a compilation flag |\version| as described in \secref{sec:flags}
compiles the main document as a draft:
%
\begin{center}
\begin{tabular}{l}
|\def\version{draft}|\\
|\input{childdoc.def}|\\
|\childdocforward{|\textit{main}|}|
\end{tabular}
\end{center}
%
Likewise, the following files |final|\textit{nn}|.tex|
compile the final version of the child document
|child|\textit{nn}|.tex|:
%
\begin{center}
\begin{tabular}{l}
|\def\version{final}|\\
|\input{childdoc.def}|\\
|\childdocforwardprefix{final}{child}|
\end{tabular}
\end{center}
%

Note that when several versions of a main file and/or of each child file
are to be generated, it may be convenient to set up a |Makefile| or
shell script to automatise the process.

%%%%%%%%%%%%%%%%%%%%%%%%%%%%%%%%%%%%%%%%%%%%%%%%%%%%%%%%%%%%%%%%%%%%%%%%%%%%%%%%
\subsection{Command Line Processing}
\label{sec:commandline}

The effect of redirection files can also be achieved by invoking
the \LaTeX{} compiler with a more elaborate command line.
Most conveniently this should be done as part
of a shell script or a |Makefile|.

When using \textsf{childdoc} in the main file, the following
command lines effectively perform a redirection
(note that depending on the shell being used,
backslashes may have to be doubled: `|\|' $\to$ `|\\|'):
%
\begin{center}
|... -jobname "|\textit{target}|" |\\|"|[\textit{flags}]%
|\input{childdoc.def}\childdocforward[|\textit{main}|]{|\textit{dest}|}"|
\end{center}
%
Here \textit{target} is the name of the output file,
\textit{main} is the name of the main file
and \textit{dest} is the name of the main or child file to be processed
(all filenames without extensions).
The optional argument \textit{main} can be omitted
if \textit{main} matches \textit{dest}.
Optionally, compilation \textit{flags} can be defined via |\def| commands.
This command line makes the \TeX{} engine believe
it is compiling the file \textit{target}
whose content is specified as the latter parameter.
The provided code then forwards the processing to
\textit{main} or \textit{dest} as described in \secref{sec:forward}.

%%%%%%%%%%%%%%%%%%%%%%%%%%%%%%%%%%%%%%%%%%%%%%%%%%%%%%%%%%%%%%%%%%%%%%%%%%%%%%%%
\subsection{Include by Input}
\label{sec:input}

Including child documents by |\include| has some restrictions by design.
Most notably, the content of a child document always occupies
its own set of pages; pages cannot be shared between child documents.
Usually, this behaviour makes perfect sense
because each child document contain an essential part of the document.
However, in some situations it may be desirable to compose
a document from a collection of parts
without having mandatory page breaks between then.
For this case, the package
provides a mechanism to include parts
by |\input| which can also be processed individually.
However, by construction this mechanism
requires manual handling of the content to be output.

%%%%%%%%%%%%%%%%%%%%%%%%%%%%%%%%%%%%%%%%
\DescribeMacro{\ifchilddocmanual}
The main file should be prepared as usual, see \secref{sec:include}.
However, the document body must make a distinction
between processing of an individual part and of the main document, e.g.:
%
\begin{center}
\begin{tabular}{l}
|\ifchilddocmanual|\\
|\input{\childdocname}|\\
|\||else|\\
\textit{document body with }|\input{|\textit{part}|}|\\
|\||fi|
\end{tabular}
\end{center}
%
The conditional |\ifchilddocmanual| is true whenever
a part to be included by |\input| is being compiled,
and the name of the part is stored in |\childdocname|.

%%%%%%%%%%%%%%%%%%%%%%%%%%%%%%%%%%%%%%%%
\DescribeMacro{\childdocby}
Each part to be included by |\input| should start with:
%
\begin{center}
\begin{tabular}{l}
|\input{childdoc.def}|\\
|\childdocby{|\textit{main}|}|\\
\end{tabular}
\end{center}
%
The directive |\childdocby| is similar to |\childdocof|
described in \secref{sec:include},
but the subsequent selection of content must be done manually.
To that end, both |\ifchilddoc| and |\ifchilddocmanual|
will be true upon processing of a part,
and the name of the part is stored in |\childdocname|.
Note that |\jobname| will be set to the filename of the current part
so that each part receives an individual |.aux| file
that does not interfere with the |.aux| file(s) of the main document.
This behaviour can be altered by the alternative form
|\childdocby[*]{|\textit{main}|}| (with a non-empty optional argument)
which uses the |.aux| file of the main document
by setting |\jobname| to \textit{main}.

%%%%%%%%%%%%%%%%%%%%%%%%%%%%%%%%%%%%%%%%%%%%%%%%%%%%%%%%%%%%%%%%%%%%%%%%%%%%%%%%
\subsection{Driver Development}
\label{sec:driver}

The \textsf{childdoc} mechanism can also be use for the development
of definition files such as \LaTeX{} styles or classes.
This case differs from the above setup with multiple parts
included by |\include| in that no |\includeonly| should be invoked.
This can be achieved by starting the include file
(before |\ProvidesPackage|) with:
%
\begin{center}
\begin{tabular}{l}
|\input{childdoc.def}|\\
|\childdocforward{|\textit{main}|}|\\
\end{tabular}
\end{center}
%
or alternatively with:
%
\begin{center}
\begin{tabular}{l}
|\input{childdoc.def}|\\
|\childdocby{|\textit{main}|}|\\
\end{tabular}
\end{center}
%
Both forms have slightly different effects as described above.
The main file is prepared as usual, see \secref{sec:include}.

%%%%%%%%%%%%%%%%%%%%%%%%%%%%%%%%%%%%%%%%%%%%%%%%%%%%%%%%%%%%%%%%%%%%%%%%%%%%%%%%
\subsection{Legacy Detection}
\label{sec:detection}

The directive |\childdocmain| in the main file can detect
whether the complete document or merely a child is to be compiled
even without using the directive |\childdocof|.
This method is deprecated because it is less robust
and there is no compelling reason to use it;
it is merely provided for backward compatibility
and it may be removed in future versions.

If the detection mechanism is to be used,
it is mandatory to correctly specify
the filename of the main file as the argument of |\childdocmain|:
%
\begin{center}
\begin{tabular}{l}
|\input{childdoc.def}|\\
|\childdocmain{|\textit{main}|}|\\
\end{tabular}
\end{center}
%
If |\jobname| does not match the argument \textit{main} of |\childdocmain|,
it is assumed that |\jobname| points to the child file to be compiled.
When using |\childdocmain| with the main file specified as argument,
it suffices to start a child file
with just |\input{|\textit{main}|}|
without loading of the package and using |\childdocof|.
If instead all processing is done
with the appropriate \textsf{childdoc} directives,
the argument of \textit{main} of |\childdocmain| can be empty.

An alternative version of the command line processing described
in \secref{sec:commandline} using the detection mechanism reads:
%
\begin{center}
|... -jobname "|\textit{target}|" "|[\textit{flags}]%
[|\def\jobname{|\textit{dest}|}|]|\input{|\textit{main}|}"|
\end{center}

%%%%%%%%%%%%%%%%%%%%%%%%%%%%%%%%%%%%%%%%%%%%%%%%%%%%%%%%%%%%%%%%%%%%%%%%%%%%%%%%
\subsection{Manual Code}
\label{sec:manual}

In case one cannot be certain whether the definitions file |childdoc.def|
is installed on the target \TeX{} distribution
and one prefers not to ship it,
it is conceivable to paste a few relevant commands into the sources.

To that end, drop all statements |\input{childdoc.def}|
and perform the replacements as outlined below.
Instead of |\childdocmain{|\textit{main}|}| add the following code
to the top of the main file:
%
\begin{center}
\begin{tabular}{l}
|\||ifdefined\childdocname\endinput\||fi\newif\ifchilddoc|\\
|\edef\childdocname{\scantokens\expandafter{\jobname\noexpand}}|\\
|\def\childdocmain{|\textit{main}|}\||ifx\childdocmain\childdocname\||else|\\
|\childdoctrue\includeonly{\childdocname}\let\jobname\childdocmain\||fi|\\
\end{tabular}
\end{center}
%
Instead of |\childdocof{|\textit{main}|}| just include the main file
at the top of each child file:
%
\begin{center}
|\input{|\textit{main}|}|
\end{center}
%
A simple redirection |\childdocforward{|\textit{dest}|}| is achieved by:
%
\begin{center}
|\def\jobname{|\textit{dest}|}\input{\jobname}|
\end{center}
%
The redirection with prefix
|\childdocforwardprefix[|\textit{prefix}|]{|\textit{dest}|}|
is accomplished by:
%
\begin{center}
\begin{tabular}{l}
|{\edef\jobname{\scantokens\expandafter{\jobname\noexpand}}|\\
|\def\redirectjob |\textit{prefix}|#1~~~{\gdef\jobname{|\textit{dest}|#1}}|\\
|\expandafter\redirectjob\jobname~~~}\input{\jobname}|
\end{tabular}
\end{center}

In an alternative approach,
child documents can be compiled by a specific command line
without additional code or specific definitions:
%
\begin{center}
|... -jobname "|\textit{target}|" "|[\textit{flags}]%
|\includeonly{|\textit{dest}|}\input{|\textit{main}|}"|
\end{center}
%

%%%%%%%%%%%%%%%%%%%%%%%%%%%%%%%%%%%%%%%%%%%%%%%%%%%%%%%%%%%%%%%%%%%%%%%%%%%%%%%%
%%%%%%%%%%%%%%%%%%%%%%%%%%%%%%%%%%%%%%%%%%%%%%%%%%%%%%%%%%%%%%%%%%%%%%%%%%%%%%%%
\section{Information}

%%%%%%%%%%%%%%%%%%%%%%%%%%%%%%%%%%%%%%%%%%%%%%%%%%%%%%%%%%%%%%%%%%%%%%%%%%%%%%%%
\subsection{Copyright}

Copyright \copyright{} 2017--2018 Niklas Beisert

This work may be distributed and/or modified under the
conditions of the \LaTeX{} Project Public License, either version 1.3
of this license or (at your option) any later version.
The latest version of this license is in
  \url{http://www.latex-project.org/lppl.txt}
and version 1.3 or later is part of all distributions of \LaTeX{}
version 2005/12/01 or later.

This work has the LPPL maintenance status `maintained'.

The Current Maintainer of this work is Niklas Beisert.

This work consists of the files |README.txt|, |childdoc.ins| and |childdoc.dtx|
as well as the derived files |childdoc.def|, |cdocsamp.tex|
with |cdocsch1.tex|, |cdocsch2.tex|, |cdocspt3.tex|, |cdocspt4.tex|,
|cdocsdrf.tex|, |cdocsfn1.tex|, |cdocsfn2.tex|
as well as |childdoc.pdf|.

%%%%%%%%%%%%%%%%%%%%%%%%%%%%%%%%%%%%%%%%%%%%%%%%%%%%%%%%%%%%%%%%%%%%%%%%%%%%%%%%
\subsection{Files and Installation}

The package consists of the files:
%
\begin{center}
\begin{tabular}{ll}
    |README.txt|   & readme file \\
    |childdoc.ins| & installation file \\
    |childdoc.dtx| & source file \\
    |childdoc.def| & definition file \\
    |cdocsamp.tex| & sample main file \\
    |cdocsch1.tex| & sample include file \\
    |cdocsch2.tex| & sample include file \\
    |cdocspt3.tex| & sample part file \\
    |cdocspt4.tex| & sample part file \\
    |cdocsdrf.tex| & sample redirection file \\
    |cdocsfn1.tex| & sample redirection file \\
    |cdocsfn2.tex| & sample redirection file \\
    |childdoc.pdf| & manual
\end{tabular}
\end{center}
%
The distribution consists of the files
|README.txt|, |childdoc.ins| and |childdoc.dtx|.
%
\begin{itemize}
\item
Run (pdf)\LaTeX{} on |childdoc.dtx|
to compile the manual |childdoc.pdf| (this file).
\item
Run \LaTeX{} on |childdoc.ins| to create the definitions file |childdoc.def|
and the sample |cdocsamp.tex| with include files
|cdocsch1.tex|, |cdocsch2.tex|, |cdocspt3.tex|, |cdocspt4.tex|,
|cdocsdrf.tex|, |cdocsfn1.tex|, |cdocsfn2.tex|.
Then copy the file |childdoc.def| to an appropriate directory of your \LaTeX{}
distribution, e.g.\ \textit{texmf-root}|/tex/latex/childdoc|.
\end{itemize}

%%%%%%%%%%%%%%%%%%%%%%%%%%%%%%%%%%%%%%%%%%%%%%%%%%%%%%%%%%%%%%%%%%%%%%%%%%%%%%%%
\subsection{Related CTAN Packages}

There are several other packages which offer a similar functionality:
%
\begin{itemize}
\item
The packages
\href{http://ctan.org/pkg/docmute}{\textsf{docmute}},
\href{http://ctan.org/pkg/includex}{\textsf{includex}} and
\href{http://ctan.org/pkg/standalone}{\textsf{standalone}}
provide commands to include only the document body of
a child file thus allowing both files to be compiled individually.
\item
The packages \href{http://ctan.org/pkg/subdocs}{\textsf{subdocs}}
and \href{http://ctan.org/pkg/subfiles}{\textsf{subfiles}}
provide structures in which the main and child documents can be
encapsulated and allowing them to be compiled individually.
The inclusion mechanism is different from the conventional |\include|.
\item
The package \href{http://ctan.org/pkg/combine}{\textsf{combine}}
is an elaborate solution to combine several documents into one.
\end{itemize}
%
See also the CTAN topic \href{http://ctan.org/topic/subdocs}{\textsf{subdocs}}
for further related packages.
The present package differs from the above solutions in that
a document structure constructed with the conventional |\include| mechanism
just needs two extra commands at the top of every file
such that all constituent files can be compiled individually.

%%%%%%%%%%%%%%%%%%%%%%%%%%%%%%%%%%%%%%%%%%%%%%%%%%%%%%%%%%%%%%%%%%%%%%%%%%%%%%%%
%\subsection{Feature Suggestions}
%
%The following is a list of features which may be useful for future
%versions of this package:
%%
%\begin{itemize}
%\item
%\ldots
%\end{itemize}

%%%%%%%%%%%%%%%%%%%%%%%%%%%%%%%%%%%%%%%%%%%%%%%%%%%%%%%%%%%%%%%%%%%%%%%%%%%%%%%%
\subsection{Revision History}

%%%%%%%%%%%%%%%%%%%%%%%%%%%%%%%%%%%%%%%%
\paragraph{v2.0:} 2018/12/30

\begin{itemize}
\item
immediate forward processing
\item
added |\childdocby| mechanism
\item
manual restructured
\end{itemize}

%%%%%%%%%%%%%%%%%%%%%%%%%%%%%%%%%%%%%%%%
\paragraph{v1.6:} 2018/01/17

\begin{itemize}
\item
application for development of include files
\item
corrections to manual
\end{itemize}

%%%%%%%%%%%%%%%%%%%%%%%%%%%%%%%%%%%%%%%%
\paragraph{v1.5:} 2017/05/21

\begin{itemize}
\item
more complete structuring introduced
\item
|\childdocof| introduced
\item
|\childdoc| renamed to |\childdocmain|
\item
|\childredirect| renamed to |\childdocforward| and |\childdocforwardprefix|
and functionality expanded
\end{itemize}

%%%%%%%%%%%%%%%%%%%%%%%%%%%%%%%%%%%%%%%%
\paragraph{v1.0:} 2017/04/27

\begin{itemize}
\item
manual and install package
\item
first version published on CTAN
\end{itemize}

%%%%%%%%%%%%%%%%%%%%%%%%%%%%%%%%%%%%%%%%
\paragraph{v0.6:} 2017/04/26

\begin{itemize}
\item
redirection mechanism added
\end{itemize}

%%%%%%%%%%%%%%%%%%%%%%%%%%%%%%%%%%%%%%%%
\paragraph{v0.5:} 2017/04/26

\begin{itemize}
\item
functionality in definition file
\end{itemize}


%%%%%%%%%%%%%%%%%%%%%%%%%%%%%%%%%%%%%%%%%%%%%%%%%%%%%%%%%%%%%%%%%%%%%%%%%%%%%%%%
%%%%%%%%%%%%%%%%%%%%%%%%%%%%%%%%%%%%%%%%%%%%%%%%%%%%%%%%%%%%%%%%%%%%%%%%%%%%%%%%
%%%%%%%%%%%%%%%%%%%%%%%%%%%%%%%%%%%%%%%%%%%%%%%%%%%%%%%%%%%%%%%%%%%%%%%%%%%%%%%%
\appendix

\settowidth\MacroIndent{\rmfamily\scriptsize 000\ }

 \DocInput{childdoc.dtx}

\end{document}
%</driver>
% \fi
%
% %%%%%%%%%%%%%%%%%%%%%%%%%%%%%%%%%%%%%%%%%%%%%%%%%%%%%%%%%%%%%%%%%%%%%%%%%%%%%%
% %%%%%%%%%%%%%%%%%%%%%%%%%%%%%%%%%%%%%%%%%%%%%%%%%%%%%%%%%%%%%%%%%%%%%%%%%%%%%%
% \section{Sample}
%\iffalse
%<*samplemain>
%\fi
%
% The following presents a sample document
% with two chapters, two parts, a title page,
% a compile flag as well as three forwarding files to set the flag.
% It consists of eight |.tex| files:
% \begin{center}
% \begin{tabular}{ll}
% |cdocsamp.tex|&main file\\
% |cdocsch1.tex|&include file for chapter 1\\
% |cdocsch2.tex|&include file for chapter 2\\
% |cdocspt3.tex|&include file for part 3\\
% |cdocspt4.tex|&include file for part 4\\
% |cdocsdrf.tex|&forwarding file for main file in draft mode\\
% |cdocsfi1.tex|&forwarding file for final version of chapter 1\\
% |cdocsfi2.tex|&forwarding file for final version of chapter 2\\
% \end{tabular}
% \end{center}
% Each of the eight files can be compiled directly by the \LaTeX{} compiler.
%
% %%%%%%%%%%%%%%%%%%%%%%%%%%%%%%%%%%%%%%
% \paragraph{Main File.}
%
% The main file is called |cdocsamp.tex|.
%
% Load the \textsf{childdoc} definitions and
% declare the filename for the main document:
%    \begin{macrocode}
\input{childdoc.def}
\childdocmain{}
%    \end{macrocode}

% Optional override for |\version| flag:
%    \begin{macrocode}
%%\ifchilddoc\else\providecommand{\version}{draft}\fi
%    \end{macrocode}

% Define the default values for the |\version| flag
% (|final| for the main file and |draft| for childs):
%    \begin{macrocode}
\ifchilddoc
\providecommand{\version}{draft}
\else
\providecommand{\version}{final}
\fi
%    \end{macrocode}

% Load the standard document class:
%    \begin{macrocode}
\documentclass[12pt]{article}
%    \end{macrocode}

% Start the document body:
%    \begin{macrocode}
\begin{document}
%    \end{macrocode}

% Declare a title page.
% Print title, part of document being processed and version flag:
%    \begin{macrocode}
\addtocounter{page}{-1}
\begin{center}
{\LARGE\bfseries{}childdoc example\par}
\vspace{1cm}
\ifchilddoc
\ifchilddocmanual part\else chapter\fi:
`\childdocname' of `\childdocjob'\par
\else
main document: `\childdocjob'\par
\fi
version: \version\par
\end{center}
\newpage
%    \end{macrocode}

% Manually include selected file,
% otherwise process as usual:
%    \begin{macrocode}
\ifchilddocmanual
\section*{part `\childdocname'}
\input{\childdocname}
\else
%    \end{macrocode}

% Include the two chapters:
%    \begin{macrocode}
\include{cdocsch1}
\include{cdocsch2}
%    \end{macrocode}

% Include the two parts unless only chapters should be displayed:
%    \begin{macrocode}
\ifchilddoc\else
\section{part three}
\input{cdocspt3}
\section{part four}
\input{cdocspt4}
\fi
%    \end{macrocode}

% Process as usual until here:
%    \begin{macrocode}
\fi
%    \end{macrocode}

% End of document body:
%    \begin{macrocode}
\end{document}
%    \end{macrocode}
%\iffalse
%</samplemain>
%\fi
%
% %%%%%%%%%%%%%%%%%%%%%%%%%%%%%%%%%%%%%%
% \paragraph{Chapter Include Files.}
%
% The include files are called |cdocsch1.tex| and |cdocsch2.tex|.
%
%\iffalse
%<*samplechap1|samplechap2>
%\fi

% Optional override for |\version| flag:
%    \begin{macrocode}
%%\providecommand{\version}{final}
%    \end{macrocode}

% Include the main document:
%    \begin{macrocode}
\input{childdoc.def}
\childdocof{cdocsamp}
%    \end{macrocode}

%\iffalse
%</samplechap1|samplechap2>
%\fi
%
%\iffalse
%<*samplechap1>
%\fi
% Some text for chapter 1:
%    \begin{macrocode}
\section{one}
some text in chapter one
%    \end{macrocode}

%\iffalse
%</samplechap1>
%\fi
% Some text for chapter 2:
%\iffalse
%<*samplechap2>
%\fi
%    \begin{macrocode}
\section{two}
more text in chapter two
%    \end{macrocode}

%\iffalse
%</samplechap2>
%\fi
%
% %%%%%%%%%%%%%%%%%%%%%%%%%%%%%%%%%%%%%%
% \paragraph{Part Include Files.}
%
% The include files are called |cdocspt3.tex| and |cdocspt4.tex|.
%
%\iffalse
%<*samplepart3|samplepart4>
%\fi

% Optional override for |\version| flag:
%    \begin{macrocode}
%%\providecommand{\version}{final}
%    \end{macrocode}

% Include the main document:
%    \begin{macrocode}
\input{childdoc.def}
\childdocby{cdocsamp}
%    \end{macrocode}

%\iffalse
%</samplepart3|samplepart4>
%\fi
%
%\iffalse
%<*samplepart3>
%\fi
% Some text for part 3:
%    \begin{macrocode}
some text in part three
%    \end{macrocode}

%\iffalse
%</samplepart3>
%\fi
% Some text for part 4:
%\iffalse
%<*samplepart4>
%\fi
%    \begin{macrocode}
more text in part four
%    \end{macrocode}

%\iffalse
%</samplepart4>
%\fi
%
% %%%%%%%%%%%%%%%%%%%%%%%%%%%%%%%%%%%%%%
% \paragraph{Forwarding for a Complete Draft.}
%
% The following forwarding file |cdocsdrf.tex|
% compiles the main document in draft mode:
%\iffalse
%<*sampledraft>
%\fi
%    \begin{macrocode}
\def\version{draft}
\input{childdoc.def}
\childdocforward{cdocsamp}
%    \end{macrocode}

%\iffalse
%</sampledraft>
%\fi
%
% %%%%%%%%%%%%%%%%%%%%%%%%%%%%%%%%%%%%%%
% \paragraph{Forwarding for Final Version of the Chapters.}
%
% The following forwarding files |cdocsfn1.tex| and |cdocsfn2.tex|
% (with identical content)
% compile the final versions of the child documents
% |cdocsch1.tex| and |cdocsch2.tex|, respectively:
%\iffalse
%<*samplefinal>
%\fi
%    \begin{macrocode}
\def\version{final}
\input{childdoc.def}
\childdocforwardprefix[cdocsamp]{cdocsfn}{cdocsch}
%    \end{macrocode}

%\iffalse
%</samplefinal>
%\fi
%
% %%%%%%%%%%%%%%%%%%%%%%%%%%%%%%%%%%%%%%
% \paragraph{Command Line Processing.}
%
% The following three command lines generate the output files
% |cdocscld|, |cdocscl1| and |cdocscl2|
% which should be identical to
% |cdocsdrf|, |cdocsch1| and |cdocsfn2|, respectively:
% \begin{center}
% \begin{tabular}{l}
% |latex -jobname cdocscld \|\\
% |  "\def\version{draft}\input{childdoc.def}\childdocforward{cdocsamp}"|\\
% |latex -jobname cdocscl1 \|\\
% |  "\input{childdoc.def}\childdocforward[cdocsamp]{cdocsch1}"|\\
% |latex -jobname cdocscl2 \|\\
% |  "\def\version{final}\input{childdoc.def}\childdocforward{cdocsch2}"|
% \end{tabular}
% \end{center}
% Note that the trailing backslash on each first line
% merely continues the input to the second line
% (for convenient cut ant paste).
% Furthermore, the command |latex| can be replaced by any
% of its alternative versions such as |pdflatex|.
%
% %%%%%%%%%%%%%%%%%%%%%%%%%%%%%%%%%%%%%%%%%%%%%%%%%%%%%%%%%%%%%%%%%%%%%%%%%%%%%%
% %%%%%%%%%%%%%%%%%%%%%%%%%%%%%%%%%%%%%%%%%%%%%%%%%%%%%%%%%%%%%%%%%%%%%%%%%%%%%%
% \section{Implementation}
%\iffalse
%<*package>
%\fi
%
% This section describes the definitions file |childdoc.def|.

% The definitions cannot be loaded using |\usepackage| or |\RequirePackage|
% which has a mechanism to prevent loading a style file more than once.
% When loading the definitions by means of |\input|
% multiple instances have to be prevented manually:
%\iffalse
%This code needs to be before the `\ProvidesFile' directive
%which is defined at the beginning of this file.
%Therefore it is also placed there and commented out here.
%</package>
%<*discard>
%\fi
%    \begin{macrocode}
\ifdefined\childdocmain\endinput\fi
%    \end{macrocode}
%\iffalse
%</discard>
%<*package>
%\fi
%
% \macro{\ifchilddoc}
% \macro{\ifchilddocmanual}
% The conditional |\ifchilddoc| tells whether a
% child (true) or main (false) document is being compiled.
% The conditional |\ifchilddocmanual| tells whether
% the |\includeonly| mechanism is used (false) or
% the selection of child files must be performed manually (true).
% The definitions initialise to false:
%    \begin{macrocode}
\newif\ifchilddoc
\newif\ifchilddocmanual
%    \end{macrocode}

% \macro{\childdocname}
% \macro{\childdocjob}
% The macro |\childdocname| stores the name of the main document
% to be compiled. The macro |\childdocjob| stores the name of
% the document on which the \LaTeX{} compiler was originally invoked.
% The content of |\jobname| cannot be compared
% to filenames specified in the source due to different catcodes.
% The following code rescans |\jobname|, stores the result
% in |\childdocname| and saves a copy in |\childdocjob|:
%    \begin{macrocode}
\edef\childdocname{\scantokens\expandafter{\jobname\noexpand}}
\let\childdocjob\childdocname
%    \end{macrocode}

% \macro{\childdocdisable}
% The macro |\childdocdisable| prevents the main file
% from being processed more than once.
% At this stage, the main document command |\childdocmain|
% is assumed to be called once again where it should do nothing.
% Any subsequent call to it should prevent
% a secondary processing of the main document
% It overwrites the forwarding commands
% |\childdocof| and |\childdocforward|
% with empty macros to prevent further inclusions of the main document:
%    \begin{macrocode}
\newcommand{\childdocdisable}
{
  \renewcommand{\childdocmain}[1]{\renewcommand{\childdocmain}[1]{\endinput}}
  \renewcommand{\childdocof}[1]{}
  \renewcommand{\childdocby}[2][]{}
  \renewcommand{\childdocforward}[2][]{}
  \renewcommand{\childdocdisable}{}
}
%    \end{macrocode}

% \macro{\childdocmain}
% The macro |\childdocmain| is to be called at the top of the main file
% with nothing or the main filename (without extension) as argument.
% First, it breaks loops.
% If the argument is not empty and does not match |\childdocname|
% (which is set by the first inclusion of |childdoc.def|),
% |\ifchilddoc| is set to true, |\includeonly| is applied to the child file
% and |\jobname| is set to the main file
% (for proper handling of |.aux| files):
%    \begin{macrocode}
\newcommand{\childdocmain}[1]
{
  \childdocdisable\childdocmain{}
  \if?#1?\else
    \begingroup
      \def\childdoctmp{#1}
      \ifx\childdoctmp\childdocname
        \def\childdoctmp{}
      \else
        \def\childdoctmp
        {
          \childdoctrue
          \includeonly{\childdocname}
          \def\childdocjob{#1}
          \def\jobname{#1}
        }
      \fi
      \expandafter
    \endgroup
    \childdoctmp
  \fi
}
%    \end{macrocode}

% \macro{\childdocof}
% The command |\childdocof| redirects
% compilation to the main file |#1|.
%    \begin{macrocode}
\newcommand{\childdocof}[1]
{
  \childdocdisable
  \childdoctrue
  \includeonly{\childdocname}
  \def\jobname{#1}
  \def\childdocjob{#1}
  \input{#1}
}
%    \end{macrocode}

% \macro{\childdocby}
% The command |\childdocby| ....
%    \begin{macrocode}
\newcommand{\childdocby}[2][]
{
  \childdocdisable
  \childdoctrue
  \childdocmanualtrue
  \if?#1?\else
    \def\jobname{#2}
  \fi
  \def\childdocjob{#2}
  \input{#2}
  \endinput
}
%    \end{macrocode}

% \macro{\childdocforward}
% The command |\childdocforward| redirects
% compilation to the main file or
% (if the optional argument is given) a child file.
% Parameters are set as if the main file
% or a child file starting with |\childdocof| was compiled.
% Then compilation is handed over to the main file:
%    \begin{macrocode}
\newcommand{\childdocforward}[2][]
{
  \begingroup
    \if?#1?
      \def\childdoctmp
      {
        \def\childdocname{#2}
        \def\childdocjob{#2}
        \def\jobname{#2}
        \input{#2}
        \endinput
      }
    \else
      \def\childdoctmp
      {
        \childdocdisable
        \def\childdocname{#2}
        \childdoctrue
        \includeonly{#2}
        \def\childdocjob{#1}
        \def\jobname{#1}
        \input{#1}
        \endinput
      }
    \fi
    \expandafter
  \endgroup
  \childdoctmp
}
%    \end{macrocode}

% \macro{\childdocforwardprefix}
% The command |\childdocforwardprefix| redirects
% compilation to the main or a child file by means of a pattern.
% The prefix |#1| in the current filename is replaced by |#2|
% and the suffix of the current filename is kept
% (it is assumed that the filename does not contain the substring `|~~~|'
% which is used as a delimiter).
% Compilation is handed over to the new file by |\childdocforward|:
%    \begin{macrocode}
\newcommand{\childdocforwardprefix}[3][]
{
  \begingroup
    \def\childdocextract #2##1~~~{\def\childdoctmp{\childdocforward[#1]{#3##1}}}
    \expandafter\childdocextract\childdocname~~~
    \expandafter
  \endgroup
  \childdoctmp
}
%    \end{macrocode}

% \macro{\childdoc}
% The deprecated macro |\childdoc| is a legacy version of |\childdocmain|:
%    \begin{macrocode}
\newcommand{\childdoc}{\childdocmain}
%    \end{macrocode}

% \macro{\childdocredirect}
% The deprecated macro |\childdocredirect| is a legacy version
% of |\childdocforward| and |\childdocforwardprefix|:
%    \begin{macrocode}
\newcommand{\childdocredirect}[2][]
{
  \begingroup
    \if?#1?
      \def\childdoctmp{\childdocforward{#2}}
    \else
      \def\childdoctmp{\childdocforwardprefix{#1}{#2}}
    \fi
    \expandafter
  \endgroup
  \childdoctmp
}
%    \end{macrocode}

%\iffalse
%</package>
%\fi
%
\endinput
\childdocforward[cdocsamp]{cdocsch1}"|\\
% |latex -jobname cdocscl2 \|\\
% |  "\def\version{final}% \iffalse
%
% childdoc.dtx Copyright (C) 2017-2018 Niklas Beisert
%
% This work may be distributed and/or modified under the
% conditions of the LaTeX Project Public License, either version 1.3
% of this license or (at your option) any later version.
% The latest version of this license is in
%   http://www.latex-project.org/lppl.txt
% and version 1.3 or later is part of all distributions of LaTeX
% version 2005/12/01 or later.
%
% This work has the LPPL maintenance status `maintained'.
%
% The Current Maintainer of this work is Niklas Beisert.
%
% This work consists of the files childdoc.dtx and childdoc.ins
% and the derived files childdoc.def and cdocsamp.tex with
% cdocsch1.tex, cdocsch2.tex, cdocsdrf.tex, cdocsfn1.tex, cdocsfn2.tex.
%
%<package>\ifdefined\childdocmain\endinput\fi
%<package>\ProvidesFile{childdoc.def}[2018/12/30 v2.0 child document driver]
%<samplemain>\ProvidesFile{cdocsamp.tex}[2018/12/30 v2.0 sample for childdoc]
%<*driver>
%\ProvidesFile{childdoc.drv}[2018/12/30 v2.0 childdoc reference manual file]
\PassOptionsToClass{10pt,a4paper}{article}
\documentclass{ltxdoc}

\usepackage[margin=35mm]{geometry}
\usepackage{hyperref}
\usepackage{hyperxmp}
\usepackage[usenames]{color}

\hypersetup{colorlinks=true}
\hypersetup{pdfstartview=FitH}
\hypersetup{pdfpagemode=UseNone}
\hypersetup{pdfsource={}}
\hypersetup{pdflang={en-UK}}
\hypersetup{pdfcopyright={Copyright 2017-2018 Niklas Beisert.
  This work may be distributed and/or modified under the
  conditions of the LaTeX Project Public License, either version 1.3
  of this license or (at your option) any later version.}}
\hypersetup{pdflicenseurl={http://www.latex-project.org/lppl.txt}}
\hypersetup{pdfcontactaddress={ETH Zurich, ITP, HIT K,
  Wolfgang-Pauli-Strasse 27}}
\hypersetup{pdfcontactpostcode={8093}}
\hypersetup{pdfcontactcity={Zurich}}
\hypersetup{pdfcontactcountry={Switzerland}}
\hypersetup{pdfcontactemail={nbeisert@itp.phys.ethz.ch}}
\hypersetup{pdfcontacturl={http://people.phys.ethz.ch/\xmptilde nbeisert/}}

\newcommand{\secref}[1]{\hyperref[#1]{section \ref*{#1}}}

\parskip1ex
\parindent0pt
\let\olditemize\itemize
\def\itemize{\olditemize\parskip0pt}

\begin{document}

\title{The \textsf{childdoc} Package}
\hypersetup{pdftitle={The childdoc Package}}
\author{Niklas Beisert\\[2ex]
  Institut f\"ur Theoretische Physik\\
  Eidgen\"ossische Technische Hochschule Z\"urich\\
  Wolfgang-Pauli-Strasse 27, 8093 Z\"urich, Switzerland\\[1ex]
  \href{mailto:nbeisert@itp.phys.ethz.ch}
  {\texttt{nbeisert@itp.phys.ethz.ch}}}
\hypersetup{pdfauthor={Niklas Beisert}}
\hypersetup{pdfsubject={Manual for the LaTeX2e Package childdoc}}
\date{30 December 2018, \textsf{v2.0}}
\maketitle

\begin{abstract}\noindent
\textsf{childdoc} is a \LaTeXe{} package
that enables the direct compilation
of document sections included by |\include|
to individual files.
\end{abstract}

\begingroup
\parskip0ex
\tableofcontents
\endgroup

%%%%%%%%%%%%%%%%%%%%%%%%%%%%%%%%%%%%%%%%%%%%%%%%%%%%%%%%%%%%%%%%%%%%%%%%%%%%%%%%
%%%%%%%%%%%%%%%%%%%%%%%%%%%%%%%%%%%%%%%%%%%%%%%%%%%%%%%%%%%%%%%%%%%%%%%%%%%%%%%%
\section{Introduction}

\LaTeX{} provides a mechanism to structure a large document (such as a book)
into a main file and several child files (containing the chapters)
using the |\include| command.
This mechanism is beneficial for documents
which span hundreds of pages in order to
make the source file(s) more manageable.
Moreover, compilation can be restricted to
selected child files by means of the |\includeonly| command.
The latter feature can be used to reduce the compilation time while editing
(this was significantly more useful in the earlier days of \LaTeX{})
or to generate a smaller document which is easier to navigate.
Another application of |\includeonly| is to generate
documents consisting of selected parts of the complete document.

However, there are a few drawbacks of the plain |\include| mechanism:
\begin{itemize}
\item
The child files cannot be compiled on their own,
they can only be compiled via the main file.
A naive editing environment
(such as a text editor with an option
to have the current file processed by \LaTeX)
may require one to switch to the main file before compiling;
attempting to compile the child file produces errors.
\item
The main file must be modified (each time)
to adjust the |\includeonly| command
to the present needs. This easily leaves the main file in a messy state.
\item
The generated document will always carry the filename
of the main document. This is inconvenient if
several child files are to be compiled and
to be kept for distribution.
\end{itemize}

The present package provides a simple interface
to make child files individually compilable by \LaTeX{}.
Compiling a child file then has the same effect as compiling
the main file with an |\includeonly| command
to select the appropriate child.
Moreover the generated document will carry the name of the child
rather than the main file.
This resolves all three above issues.

This feature is meant to make the editing of books,
thesis documents and lecture notes somewhat more convenient.
However, the package can also be used efficiently for
composing a series of documents (such as exercise sheets)
which are typically distributed individually.
It then assists the author in generating the individual documents
(potentially in different versions)
as well as a document containing the collected series.
Another application is in developing style files
or other kinds of included material
where compilation of the style file could redirect
to a sample or test file.

%%%%%%%%%%%%%%%%%%%%%%%%%%%%%%%%%%%%%%%%%%%%%%%%%%%%%%%%%%%%%%%%%%%%%%%%%%%%%%%%
%%%%%%%%%%%%%%%%%%%%%%%%%%%%%%%%%%%%%%%%%%%%%%%%%%%%%%%%%%%%%%%%%%%%%%%%%%%%%%%%
\section{Usage}

First of all, the package \textsf{childdoc} is \emph{not} a standard
\LaTeXe{} |.sty| style file! Therefore it needs to be invoked in
a non-standard way.

%%%%%%%%%%%%%%%%%%%%%%%%%%%%%%%%%%%%%%%%%%%%%%%%%%%%%%%%%%%%%%%%%%%%%%%%%%%%%%%%
\subsection{Included Files}
\label{sec:include}

%%%%%%%%%%%%%%%%%%%%%%%%%%%%%%%%%%%%%%%%
\DescribeMacro{\childdocmain}
To use the package, add the commands
\begin{center}
\begin{tabular}{l}
|\input{childdoc.def}|\\
|\childdocmain{}|\\
\end{tabular}
\end{center}
at the very top of the main \LaTeX{} file,
in particular \emph{before} the |\documentclass| statement!
The argument of |\childdocmain| should be left empty
(but it must be present).

%%%%%%%%%%%%%%%%%%%%%%%%%%%%%%%%%%%%%%%%
\DescribeMacro{\childdocof}
Furthermore, add the commands
\begin{center}
\begin{tabular}{l}
|\input{childdoc.def}|\\
|\childdocof{|\textit{main}|}|\\
\end{tabular}
\end{center}
at the top of every child file \textit{child}
which is included by |\include{|\textit{child}|}|
from within the main file
(or at least for those files to be compiled individually).
The argument \textit{main} must be the filename of the main file.

There are a couple of
considerations in setting up the main and child documents:

%%%%%%%%%%%%%%%%%%%%%%%%%%%%%%%%%%%%%%%%
\paragraph{Restrictions.}

Please note the following restrictions:
\begin{itemize}
\item
|\childdocmain| must be called with one argument \textit{main}
to ensure compatibility with earlier version of the package.
It must either be empty (|\childdocmain{}|)
or precisely match the filename of the main file in which it is specified.
See \secref{sec:detection} for further information.
\item
The filename \textit{main} must be specified without the |.tex| extension.
\item
The filename \textit{main} is case sensitive
(even in case-insensitive file systems)
due to internal string comparison.
\item
The argument \textit{main} should be fully expanded, it cannot be a macro.
\item
Subdirectories and special characters should be avoided in filenames.
\item
The command |\childdocmain{|\textit{main}|}| must be followed by a whitespace.
It should not be followed immediately by another command
or by a comment mark `|%|'.
This is because the \TeX{} parser reads the token immediately following
the argument of |\childdocmain| and puts it
at the beginning of every child section;
however, a white\-space is ignored.
\end{itemize}

%%%%%%%%%%%%%%%%%%%%%%%%%%%%%%%%%%%%%%%%
\paragraph{Content of Main File.}

It is advisable to place all content in the child files included by |\include|.
Any output contained in the main file will appear in all child documents
unless suppressed manually;
it cannot be suppressed automatically by the |\includeonly| directive
and thus should normally be avoided.
A method to include some content in the main file
by means of conditional processing is described in \secref{sec:conditional}.

%%%%%%%%%%%%%%%%%%%%%%%%%%%%%%%%%%%%%%%%
\paragraph{Page Numbering.}

When only a part of the document is compiled,
the appropriate numbering of pages
(as well as other status parameters)
is determined from the |.aux| files.
The latter contain information from previous passes.
However this information needs to propagate through
all intermediate child documents.
Therefore the page numbering in child documents may well
be inconsistent until the complete document is compiled at least once.

A useful (if unconventional) way to always ensure a consistent
page numbering is to restart the numbering in each child document
and denote the pages by `\textit{child}|.|\textit{page}'
where \textit{child} represents the chapter/section number of the child file.
This can be achieved by the command
|\numberwithin{page}{|\textit{child}|}|
of the \textsf{amsmath} package
where \textit{child} can be |chapter| or |section|
depending on the chosen structuring.
Alternatively, one can modify the macro |\thepage| appropriately
and reset the counter |page| at the start of each child file.

%%%%%%%%%%%%%%%%%%%%%%%%%%%%%%%%%%%%%%%%%%%%%%%%%%%%%%%%%%%%%%%%%%%%%%%%%%%%%%%%
\subsection{Conditional Processing}
\label{sec:conditional}

The package provides a mechanism to compile different versions
of a document. To customise the versions further some conditional processing
can come in handy to distinguish which version is being compiled.
The package provides two macros to describe the compilation context:

%%%%%%%%%%%%%%%%%%%%%%%%%%%%%%%%%%%%%%%%
\DescribeMacro{\ifchilddoc}
The conditional |\ifchilddoc| distinguishes between the compilation of
child documents and the main document:
%
\begin{center}
|\ifchilddoc |\textit{child-code}| |[|\||else |\textit{main-code}]| \||fi|
\end{center}

%%%%%%%%%%%%%%%%%%%%%%%%%%%%%%%%%%%%%%%%
\DescribeMacro{\childdocname}
\DescribeMacro{\childdocjob}
The macro |\childdocname| contains the filename (without extension)
of the main or child file being processed.
Note that |\childdocjob| will always contain the name of the main file.

%%%%%%%%%%%%%%%%%%%%%%%%%%%%%%%%%%%%%%%%
\paragraph{Title Page.}

Conditional processing can be used to include a title or banner page
in the main document when proper precautions are taken.
Importantly, the code in the main file should ensure that the page counter
(as well as other status parameters which are stored in the |.aux| files)
takes the same value after the conditional processing.
Otherwise the page numbers may take divergent values
depending on which part is compiled.

For example, a title page could be declared by:
%
\begin{center}
\begin{tabular}{l}
|\ifchilddoc\||else|\\
|\addtocounter{page}{-1}|\\
\textit{code for title page}\\
|\newpage|\\
|\||fi|
\end{tabular}
\end{center}
%
A banner page for the child documents can be generated by:
%
\begin{center}
\begin{tabular}{l}
|\ifchilddoc|\\
|\addtocounter{page}{-1}|\\
\textit{code for banner page}\\
|\newpage|\\
|\||fi|
\end{tabular}
\end{center}
%
Here one could write a message such as:
\begin{center}
|This is the part \childdocname{} of \childdocjob{}.|
\end{center}

%%%%%%%%%%%%%%%%%%%%%%%%%%%%%%%%%%%%%%%%%%%%%%%%%%%%%%%%%%%%%%%%%%%%%%%%%%%%%%%%
\subsection{Flags}
\label{sec:flags}

The package makes it easy to generate different versions
of the main or child documents.
To this end compilation flags can be defined
and assigned different default values.
They will be particularly useful in conjunction
with the forwarding mechanism described in \secref{sec:forward}.

For example, it may be useful to have a flag |\version|
which can be set to |draft| or |final|.
The document source will contain some conditional code
depending on the value of |\version|.
Suppose further, the flag should default to |final| for the main file
and to |draft| for child files
which is a natural assignment for editing the document.
This is achieved by placing the following code
in the preamble of the main document
(below the |\childdocmain| directive):
%
\begin{center}
\begin{tabular}{l}
|\ifchilddoc|\\
|\providecommand{\version}{draft}|\\
|\||else|\\
|\providecommand{\version}{final}|\\
|\||fi|
\end{tabular}
\end{center}
%
The definition by |\providecommand| makes sure
that previous definitions are not overwritten.
Further statements |\providecommand{\version}{...}|
can thus be added before the above code to override it.

For the main file, one might add a line
(between |\childdocmain| and the above block)
%
\begin{center}
|%\ifchilddoc\||else\providecommand{\version}{draft}\||fi|
\end{center}
%
which can be uncommented to produce a draft version.
Likewise one can add a line to the very top of a child file
(above the |\childdocof{|\textit{main}|}| directive)
%
\begin{center}
|%\providecommand{\version}{final}|
\end{center}
%
which can be uncommented to produce the final version of this child document.

%%%%%%%%%%%%%%%%%%%%%%%%%%%%%%%%%%%%%%%%%%%%%%%%%%%%%%%%%%%%%%%%%%%%%%%%%%%%%%%%
\subsection{Forwarding}
\label{sec:forward}

Different versions of the main or child documents
using compilation flags as described in \secref{sec:flags}
can be (permanently) stored in different files
for convenient compilation, viewing and distribution.
To this end, the package defines a command
to pass on compilation to a different file:

%%%%%%%%%%%%%%%%%%%%%%%%%%%%%%%%%%%%%%%%
\DescribeMacro{\childdocforward}
The command |\childdocforward| redirects processing to
another source file:
%
\begin{center}
\begin{tabular}{l}
|\input{childdoc.def}|\\
|\childdocforward[|\textit{main}|]{|\textit{dest}|}|\\
\end{tabular}
\end{center}
%
The argument \textit{dest} is the destination file
(without extension).
It should be the main file or one of the child files.
Note that further \textsf{childdoc} directives
such as |\childdocof| and |\childdocforward|
in the indicated file will be processed in this form.
The optional argument \textit{main}
passes on directly to the main file \textit{main}
while pretending to compile the child \textit{dest}.
This form behaves as if \textit{dest}
issues |\childdocof{|\textit{main}|}| right away,
and no further \textsf{childdoc} directives will be processed.

%%%%%%%%%%%%%%%%%%%%%%%%%%%%%%%%%%%%%%%%
\DescribeMacro{\...prefix}
In the alternative form |\childdocforwardprefix|,
%
\begin{center}
\begin{tabular}{l}
|\input{childdoc.def}|\\
|\childdocforwardprefix[|\textit{main}|]{|\textit{prefix}|}{|\textit{dest}|}|
\end{tabular}
\end{center}
%
the destination file is determined by a pattern
depending on the current file:
To make this work, the current file must be called
`{\textit{prefix}\hspace{0.2em}\textit{suffix}}'
with \textit{prefix} matching precisely the argument.
Processing is then passed on to the file
`{\textit{dest}\hspace{0.2em}\textit{suffix}}'.
Surely, the same effect is achieved by
directly specifying the
argument `{\textit{dest}\hspace{0.2em}\textit{suffix}}'
in the first form.
However, that requires to set up a different file
for each child. With the alternative form of the command
all these files can have exactly the same content
which simplifies setting them up and maintaining them.

For example, the following file |draft.tex|
with a compilation flag |\version| as described in \secref{sec:flags}
compiles the main document as a draft:
%
\begin{center}
\begin{tabular}{l}
|\def\version{draft}|\\
|\input{childdoc.def}|\\
|\childdocforward{|\textit{main}|}|
\end{tabular}
\end{center}
%
Likewise, the following files |final|\textit{nn}|.tex|
compile the final version of the child document
|child|\textit{nn}|.tex|:
%
\begin{center}
\begin{tabular}{l}
|\def\version{final}|\\
|\input{childdoc.def}|\\
|\childdocforwardprefix{final}{child}|
\end{tabular}
\end{center}
%

Note that when several versions of a main file and/or of each child file
are to be generated, it may be convenient to set up a |Makefile| or
shell script to automatise the process.

%%%%%%%%%%%%%%%%%%%%%%%%%%%%%%%%%%%%%%%%%%%%%%%%%%%%%%%%%%%%%%%%%%%%%%%%%%%%%%%%
\subsection{Command Line Processing}
\label{sec:commandline}

The effect of redirection files can also be achieved by invoking
the \LaTeX{} compiler with a more elaborate command line.
Most conveniently this should be done as part
of a shell script or a |Makefile|.

When using \textsf{childdoc} in the main file, the following
command lines effectively perform a redirection
(note that depending on the shell being used,
backslashes may have to be doubled: `|\|' $\to$ `|\\|'):
%
\begin{center}
|... -jobname "|\textit{target}|" |\\|"|[\textit{flags}]%
|\input{childdoc.def}\childdocforward[|\textit{main}|]{|\textit{dest}|}"|
\end{center}
%
Here \textit{target} is the name of the output file,
\textit{main} is the name of the main file
and \textit{dest} is the name of the main or child file to be processed
(all filenames without extensions).
The optional argument \textit{main} can be omitted
if \textit{main} matches \textit{dest}.
Optionally, compilation \textit{flags} can be defined via |\def| commands.
This command line makes the \TeX{} engine believe
it is compiling the file \textit{target}
whose content is specified as the latter parameter.
The provided code then forwards the processing to
\textit{main} or \textit{dest} as described in \secref{sec:forward}.

%%%%%%%%%%%%%%%%%%%%%%%%%%%%%%%%%%%%%%%%%%%%%%%%%%%%%%%%%%%%%%%%%%%%%%%%%%%%%%%%
\subsection{Include by Input}
\label{sec:input}

Including child documents by |\include| has some restrictions by design.
Most notably, the content of a child document always occupies
its own set of pages; pages cannot be shared between child documents.
Usually, this behaviour makes perfect sense
because each child document contain an essential part of the document.
However, in some situations it may be desirable to compose
a document from a collection of parts
without having mandatory page breaks between then.
For this case, the package
provides a mechanism to include parts
by |\input| which can also be processed individually.
However, by construction this mechanism
requires manual handling of the content to be output.

%%%%%%%%%%%%%%%%%%%%%%%%%%%%%%%%%%%%%%%%
\DescribeMacro{\ifchilddocmanual}
The main file should be prepared as usual, see \secref{sec:include}.
However, the document body must make a distinction
between processing of an individual part and of the main document, e.g.:
%
\begin{center}
\begin{tabular}{l}
|\ifchilddocmanual|\\
|\input{\childdocname}|\\
|\||else|\\
\textit{document body with }|\input{|\textit{part}|}|\\
|\||fi|
\end{tabular}
\end{center}
%
The conditional |\ifchilddocmanual| is true whenever
a part to be included by |\input| is being compiled,
and the name of the part is stored in |\childdocname|.

%%%%%%%%%%%%%%%%%%%%%%%%%%%%%%%%%%%%%%%%
\DescribeMacro{\childdocby}
Each part to be included by |\input| should start with:
%
\begin{center}
\begin{tabular}{l}
|\input{childdoc.def}|\\
|\childdocby{|\textit{main}|}|\\
\end{tabular}
\end{center}
%
The directive |\childdocby| is similar to |\childdocof|
described in \secref{sec:include},
but the subsequent selection of content must be done manually.
To that end, both |\ifchilddoc| and |\ifchilddocmanual|
will be true upon processing of a part,
and the name of the part is stored in |\childdocname|.
Note that |\jobname| will be set to the filename of the current part
so that each part receives an individual |.aux| file
that does not interfere with the |.aux| file(s) of the main document.
This behaviour can be altered by the alternative form
|\childdocby[*]{|\textit{main}|}| (with a non-empty optional argument)
which uses the |.aux| file of the main document
by setting |\jobname| to \textit{main}.

%%%%%%%%%%%%%%%%%%%%%%%%%%%%%%%%%%%%%%%%%%%%%%%%%%%%%%%%%%%%%%%%%%%%%%%%%%%%%%%%
\subsection{Driver Development}
\label{sec:driver}

The \textsf{childdoc} mechanism can also be use for the development
of definition files such as \LaTeX{} styles or classes.
This case differs from the above setup with multiple parts
included by |\include| in that no |\includeonly| should be invoked.
This can be achieved by starting the include file
(before |\ProvidesPackage|) with:
%
\begin{center}
\begin{tabular}{l}
|\input{childdoc.def}|\\
|\childdocforward{|\textit{main}|}|\\
\end{tabular}
\end{center}
%
or alternatively with:
%
\begin{center}
\begin{tabular}{l}
|\input{childdoc.def}|\\
|\childdocby{|\textit{main}|}|\\
\end{tabular}
\end{center}
%
Both forms have slightly different effects as described above.
The main file is prepared as usual, see \secref{sec:include}.

%%%%%%%%%%%%%%%%%%%%%%%%%%%%%%%%%%%%%%%%%%%%%%%%%%%%%%%%%%%%%%%%%%%%%%%%%%%%%%%%
\subsection{Legacy Detection}
\label{sec:detection}

The directive |\childdocmain| in the main file can detect
whether the complete document or merely a child is to be compiled
even without using the directive |\childdocof|.
This method is deprecated because it is less robust
and there is no compelling reason to use it;
it is merely provided for backward compatibility
and it may be removed in future versions.

If the detection mechanism is to be used,
it is mandatory to correctly specify
the filename of the main file as the argument of |\childdocmain|:
%
\begin{center}
\begin{tabular}{l}
|\input{childdoc.def}|\\
|\childdocmain{|\textit{main}|}|\\
\end{tabular}
\end{center}
%
If |\jobname| does not match the argument \textit{main} of |\childdocmain|,
it is assumed that |\jobname| points to the child file to be compiled.
When using |\childdocmain| with the main file specified as argument,
it suffices to start a child file
with just |\input{|\textit{main}|}|
without loading of the package and using |\childdocof|.
If instead all processing is done
with the appropriate \textsf{childdoc} directives,
the argument of \textit{main} of |\childdocmain| can be empty.

An alternative version of the command line processing described
in \secref{sec:commandline} using the detection mechanism reads:
%
\begin{center}
|... -jobname "|\textit{target}|" "|[\textit{flags}]%
[|\def\jobname{|\textit{dest}|}|]|\input{|\textit{main}|}"|
\end{center}

%%%%%%%%%%%%%%%%%%%%%%%%%%%%%%%%%%%%%%%%%%%%%%%%%%%%%%%%%%%%%%%%%%%%%%%%%%%%%%%%
\subsection{Manual Code}
\label{sec:manual}

In case one cannot be certain whether the definitions file |childdoc.def|
is installed on the target \TeX{} distribution
and one prefers not to ship it,
it is conceivable to paste a few relevant commands into the sources.

To that end, drop all statements |\input{childdoc.def}|
and perform the replacements as outlined below.
Instead of |\childdocmain{|\textit{main}|}| add the following code
to the top of the main file:
%
\begin{center}
\begin{tabular}{l}
|\||ifdefined\childdocname\endinput\||fi\newif\ifchilddoc|\\
|\edef\childdocname{\scantokens\expandafter{\jobname\noexpand}}|\\
|\def\childdocmain{|\textit{main}|}\||ifx\childdocmain\childdocname\||else|\\
|\childdoctrue\includeonly{\childdocname}\let\jobname\childdocmain\||fi|\\
\end{tabular}
\end{center}
%
Instead of |\childdocof{|\textit{main}|}| just include the main file
at the top of each child file:
%
\begin{center}
|\input{|\textit{main}|}|
\end{center}
%
A simple redirection |\childdocforward{|\textit{dest}|}| is achieved by:
%
\begin{center}
|\def\jobname{|\textit{dest}|}\input{\jobname}|
\end{center}
%
The redirection with prefix
|\childdocforwardprefix[|\textit{prefix}|]{|\textit{dest}|}|
is accomplished by:
%
\begin{center}
\begin{tabular}{l}
|{\edef\jobname{\scantokens\expandafter{\jobname\noexpand}}|\\
|\def\redirectjob |\textit{prefix}|#1~~~{\gdef\jobname{|\textit{dest}|#1}}|\\
|\expandafter\redirectjob\jobname~~~}\input{\jobname}|
\end{tabular}
\end{center}

In an alternative approach,
child documents can be compiled by a specific command line
without additional code or specific definitions:
%
\begin{center}
|... -jobname "|\textit{target}|" "|[\textit{flags}]%
|\includeonly{|\textit{dest}|}\input{|\textit{main}|}"|
\end{center}
%

%%%%%%%%%%%%%%%%%%%%%%%%%%%%%%%%%%%%%%%%%%%%%%%%%%%%%%%%%%%%%%%%%%%%%%%%%%%%%%%%
%%%%%%%%%%%%%%%%%%%%%%%%%%%%%%%%%%%%%%%%%%%%%%%%%%%%%%%%%%%%%%%%%%%%%%%%%%%%%%%%
\section{Information}

%%%%%%%%%%%%%%%%%%%%%%%%%%%%%%%%%%%%%%%%%%%%%%%%%%%%%%%%%%%%%%%%%%%%%%%%%%%%%%%%
\subsection{Copyright}

Copyright \copyright{} 2017--2018 Niklas Beisert

This work may be distributed and/or modified under the
conditions of the \LaTeX{} Project Public License, either version 1.3
of this license or (at your option) any later version.
The latest version of this license is in
  \url{http://www.latex-project.org/lppl.txt}
and version 1.3 or later is part of all distributions of \LaTeX{}
version 2005/12/01 or later.

This work has the LPPL maintenance status `maintained'.

The Current Maintainer of this work is Niklas Beisert.

This work consists of the files |README.txt|, |childdoc.ins| and |childdoc.dtx|
as well as the derived files |childdoc.def|, |cdocsamp.tex|
with |cdocsch1.tex|, |cdocsch2.tex|, |cdocspt3.tex|, |cdocspt4.tex|,
|cdocsdrf.tex|, |cdocsfn1.tex|, |cdocsfn2.tex|
as well as |childdoc.pdf|.

%%%%%%%%%%%%%%%%%%%%%%%%%%%%%%%%%%%%%%%%%%%%%%%%%%%%%%%%%%%%%%%%%%%%%%%%%%%%%%%%
\subsection{Files and Installation}

The package consists of the files:
%
\begin{center}
\begin{tabular}{ll}
    |README.txt|   & readme file \\
    |childdoc.ins| & installation file \\
    |childdoc.dtx| & source file \\
    |childdoc.def| & definition file \\
    |cdocsamp.tex| & sample main file \\
    |cdocsch1.tex| & sample include file \\
    |cdocsch2.tex| & sample include file \\
    |cdocspt3.tex| & sample part file \\
    |cdocspt4.tex| & sample part file \\
    |cdocsdrf.tex| & sample redirection file \\
    |cdocsfn1.tex| & sample redirection file \\
    |cdocsfn2.tex| & sample redirection file \\
    |childdoc.pdf| & manual
\end{tabular}
\end{center}
%
The distribution consists of the files
|README.txt|, |childdoc.ins| and |childdoc.dtx|.
%
\begin{itemize}
\item
Run (pdf)\LaTeX{} on |childdoc.dtx|
to compile the manual |childdoc.pdf| (this file).
\item
Run \LaTeX{} on |childdoc.ins| to create the definitions file |childdoc.def|
and the sample |cdocsamp.tex| with include files
|cdocsch1.tex|, |cdocsch2.tex|, |cdocspt3.tex|, |cdocspt4.tex|,
|cdocsdrf.tex|, |cdocsfn1.tex|, |cdocsfn2.tex|.
Then copy the file |childdoc.def| to an appropriate directory of your \LaTeX{}
distribution, e.g.\ \textit{texmf-root}|/tex/latex/childdoc|.
\end{itemize}

%%%%%%%%%%%%%%%%%%%%%%%%%%%%%%%%%%%%%%%%%%%%%%%%%%%%%%%%%%%%%%%%%%%%%%%%%%%%%%%%
\subsection{Related CTAN Packages}

There are several other packages which offer a similar functionality:
%
\begin{itemize}
\item
The packages
\href{http://ctan.org/pkg/docmute}{\textsf{docmute}},
\href{http://ctan.org/pkg/includex}{\textsf{includex}} and
\href{http://ctan.org/pkg/standalone}{\textsf{standalone}}
provide commands to include only the document body of
a child file thus allowing both files to be compiled individually.
\item
The packages \href{http://ctan.org/pkg/subdocs}{\textsf{subdocs}}
and \href{http://ctan.org/pkg/subfiles}{\textsf{subfiles}}
provide structures in which the main and child documents can be
encapsulated and allowing them to be compiled individually.
The inclusion mechanism is different from the conventional |\include|.
\item
The package \href{http://ctan.org/pkg/combine}{\textsf{combine}}
is an elaborate solution to combine several documents into one.
\end{itemize}
%
See also the CTAN topic \href{http://ctan.org/topic/subdocs}{\textsf{subdocs}}
for further related packages.
The present package differs from the above solutions in that
a document structure constructed with the conventional |\include| mechanism
just needs two extra commands at the top of every file
such that all constituent files can be compiled individually.

%%%%%%%%%%%%%%%%%%%%%%%%%%%%%%%%%%%%%%%%%%%%%%%%%%%%%%%%%%%%%%%%%%%%%%%%%%%%%%%%
%\subsection{Feature Suggestions}
%
%The following is a list of features which may be useful for future
%versions of this package:
%%
%\begin{itemize}
%\item
%\ldots
%\end{itemize}

%%%%%%%%%%%%%%%%%%%%%%%%%%%%%%%%%%%%%%%%%%%%%%%%%%%%%%%%%%%%%%%%%%%%%%%%%%%%%%%%
\subsection{Revision History}

%%%%%%%%%%%%%%%%%%%%%%%%%%%%%%%%%%%%%%%%
\paragraph{v2.0:} 2018/12/30

\begin{itemize}
\item
immediate forward processing
\item
added |\childdocby| mechanism
\item
manual restructured
\end{itemize}

%%%%%%%%%%%%%%%%%%%%%%%%%%%%%%%%%%%%%%%%
\paragraph{v1.6:} 2018/01/17

\begin{itemize}
\item
application for development of include files
\item
corrections to manual
\end{itemize}

%%%%%%%%%%%%%%%%%%%%%%%%%%%%%%%%%%%%%%%%
\paragraph{v1.5:} 2017/05/21

\begin{itemize}
\item
more complete structuring introduced
\item
|\childdocof| introduced
\item
|\childdoc| renamed to |\childdocmain|
\item
|\childredirect| renamed to |\childdocforward| and |\childdocforwardprefix|
and functionality expanded
\end{itemize}

%%%%%%%%%%%%%%%%%%%%%%%%%%%%%%%%%%%%%%%%
\paragraph{v1.0:} 2017/04/27

\begin{itemize}
\item
manual and install package
\item
first version published on CTAN
\end{itemize}

%%%%%%%%%%%%%%%%%%%%%%%%%%%%%%%%%%%%%%%%
\paragraph{v0.6:} 2017/04/26

\begin{itemize}
\item
redirection mechanism added
\end{itemize}

%%%%%%%%%%%%%%%%%%%%%%%%%%%%%%%%%%%%%%%%
\paragraph{v0.5:} 2017/04/26

\begin{itemize}
\item
functionality in definition file
\end{itemize}


%%%%%%%%%%%%%%%%%%%%%%%%%%%%%%%%%%%%%%%%%%%%%%%%%%%%%%%%%%%%%%%%%%%%%%%%%%%%%%%%
%%%%%%%%%%%%%%%%%%%%%%%%%%%%%%%%%%%%%%%%%%%%%%%%%%%%%%%%%%%%%%%%%%%%%%%%%%%%%%%%
%%%%%%%%%%%%%%%%%%%%%%%%%%%%%%%%%%%%%%%%%%%%%%%%%%%%%%%%%%%%%%%%%%%%%%%%%%%%%%%%
\appendix

\settowidth\MacroIndent{\rmfamily\scriptsize 000\ }

 \DocInput{childdoc.dtx}

\end{document}
%</driver>
% \fi
%
% %%%%%%%%%%%%%%%%%%%%%%%%%%%%%%%%%%%%%%%%%%%%%%%%%%%%%%%%%%%%%%%%%%%%%%%%%%%%%%
% %%%%%%%%%%%%%%%%%%%%%%%%%%%%%%%%%%%%%%%%%%%%%%%%%%%%%%%%%%%%%%%%%%%%%%%%%%%%%%
% \section{Sample}
%\iffalse
%<*samplemain>
%\fi
%
% The following presents a sample document
% with two chapters, two parts, a title page,
% a compile flag as well as three forwarding files to set the flag.
% It consists of eight |.tex| files:
% \begin{center}
% \begin{tabular}{ll}
% |cdocsamp.tex|&main file\\
% |cdocsch1.tex|&include file for chapter 1\\
% |cdocsch2.tex|&include file for chapter 2\\
% |cdocspt3.tex|&include file for part 3\\
% |cdocspt4.tex|&include file for part 4\\
% |cdocsdrf.tex|&forwarding file for main file in draft mode\\
% |cdocsfi1.tex|&forwarding file for final version of chapter 1\\
% |cdocsfi2.tex|&forwarding file for final version of chapter 2\\
% \end{tabular}
% \end{center}
% Each of the eight files can be compiled directly by the \LaTeX{} compiler.
%
% %%%%%%%%%%%%%%%%%%%%%%%%%%%%%%%%%%%%%%
% \paragraph{Main File.}
%
% The main file is called |cdocsamp.tex|.
%
% Load the \textsf{childdoc} definitions and
% declare the filename for the main document:
%    \begin{macrocode}
\input{childdoc.def}
\childdocmain{}
%    \end{macrocode}

% Optional override for |\version| flag:
%    \begin{macrocode}
%%\ifchilddoc\else\providecommand{\version}{draft}\fi
%    \end{macrocode}

% Define the default values for the |\version| flag
% (|final| for the main file and |draft| for childs):
%    \begin{macrocode}
\ifchilddoc
\providecommand{\version}{draft}
\else
\providecommand{\version}{final}
\fi
%    \end{macrocode}

% Load the standard document class:
%    \begin{macrocode}
\documentclass[12pt]{article}
%    \end{macrocode}

% Start the document body:
%    \begin{macrocode}
\begin{document}
%    \end{macrocode}

% Declare a title page.
% Print title, part of document being processed and version flag:
%    \begin{macrocode}
\addtocounter{page}{-1}
\begin{center}
{\LARGE\bfseries{}childdoc example\par}
\vspace{1cm}
\ifchilddoc
\ifchilddocmanual part\else chapter\fi:
`\childdocname' of `\childdocjob'\par
\else
main document: `\childdocjob'\par
\fi
version: \version\par
\end{center}
\newpage
%    \end{macrocode}

% Manually include selected file,
% otherwise process as usual:
%    \begin{macrocode}
\ifchilddocmanual
\section*{part `\childdocname'}
\input{\childdocname}
\else
%    \end{macrocode}

% Include the two chapters:
%    \begin{macrocode}
\include{cdocsch1}
\include{cdocsch2}
%    \end{macrocode}

% Include the two parts unless only chapters should be displayed:
%    \begin{macrocode}
\ifchilddoc\else
\section{part three}
\input{cdocspt3}
\section{part four}
\input{cdocspt4}
\fi
%    \end{macrocode}

% Process as usual until here:
%    \begin{macrocode}
\fi
%    \end{macrocode}

% End of document body:
%    \begin{macrocode}
\end{document}
%    \end{macrocode}
%\iffalse
%</samplemain>
%\fi
%
% %%%%%%%%%%%%%%%%%%%%%%%%%%%%%%%%%%%%%%
% \paragraph{Chapter Include Files.}
%
% The include files are called |cdocsch1.tex| and |cdocsch2.tex|.
%
%\iffalse
%<*samplechap1|samplechap2>
%\fi

% Optional override for |\version| flag:
%    \begin{macrocode}
%%\providecommand{\version}{final}
%    \end{macrocode}

% Include the main document:
%    \begin{macrocode}
\input{childdoc.def}
\childdocof{cdocsamp}
%    \end{macrocode}

%\iffalse
%</samplechap1|samplechap2>
%\fi
%
%\iffalse
%<*samplechap1>
%\fi
% Some text for chapter 1:
%    \begin{macrocode}
\section{one}
some text in chapter one
%    \end{macrocode}

%\iffalse
%</samplechap1>
%\fi
% Some text for chapter 2:
%\iffalse
%<*samplechap2>
%\fi
%    \begin{macrocode}
\section{two}
more text in chapter two
%    \end{macrocode}

%\iffalse
%</samplechap2>
%\fi
%
% %%%%%%%%%%%%%%%%%%%%%%%%%%%%%%%%%%%%%%
% \paragraph{Part Include Files.}
%
% The include files are called |cdocspt3.tex| and |cdocspt4.tex|.
%
%\iffalse
%<*samplepart3|samplepart4>
%\fi

% Optional override for |\version| flag:
%    \begin{macrocode}
%%\providecommand{\version}{final}
%    \end{macrocode}

% Include the main document:
%    \begin{macrocode}
\input{childdoc.def}
\childdocby{cdocsamp}
%    \end{macrocode}

%\iffalse
%</samplepart3|samplepart4>
%\fi
%
%\iffalse
%<*samplepart3>
%\fi
% Some text for part 3:
%    \begin{macrocode}
some text in part three
%    \end{macrocode}

%\iffalse
%</samplepart3>
%\fi
% Some text for part 4:
%\iffalse
%<*samplepart4>
%\fi
%    \begin{macrocode}
more text in part four
%    \end{macrocode}

%\iffalse
%</samplepart4>
%\fi
%
% %%%%%%%%%%%%%%%%%%%%%%%%%%%%%%%%%%%%%%
% \paragraph{Forwarding for a Complete Draft.}
%
% The following forwarding file |cdocsdrf.tex|
% compiles the main document in draft mode:
%\iffalse
%<*sampledraft>
%\fi
%    \begin{macrocode}
\def\version{draft}
\input{childdoc.def}
\childdocforward{cdocsamp}
%    \end{macrocode}

%\iffalse
%</sampledraft>
%\fi
%
% %%%%%%%%%%%%%%%%%%%%%%%%%%%%%%%%%%%%%%
% \paragraph{Forwarding for Final Version of the Chapters.}
%
% The following forwarding files |cdocsfn1.tex| and |cdocsfn2.tex|
% (with identical content)
% compile the final versions of the child documents
% |cdocsch1.tex| and |cdocsch2.tex|, respectively:
%\iffalse
%<*samplefinal>
%\fi
%    \begin{macrocode}
\def\version{final}
\input{childdoc.def}
\childdocforwardprefix[cdocsamp]{cdocsfn}{cdocsch}
%    \end{macrocode}

%\iffalse
%</samplefinal>
%\fi
%
% %%%%%%%%%%%%%%%%%%%%%%%%%%%%%%%%%%%%%%
% \paragraph{Command Line Processing.}
%
% The following three command lines generate the output files
% |cdocscld|, |cdocscl1| and |cdocscl2|
% which should be identical to
% |cdocsdrf|, |cdocsch1| and |cdocsfn2|, respectively:
% \begin{center}
% \begin{tabular}{l}
% |latex -jobname cdocscld \|\\
% |  "\def\version{draft}\input{childdoc.def}\childdocforward{cdocsamp}"|\\
% |latex -jobname cdocscl1 \|\\
% |  "\input{childdoc.def}\childdocforward[cdocsamp]{cdocsch1}"|\\
% |latex -jobname cdocscl2 \|\\
% |  "\def\version{final}\input{childdoc.def}\childdocforward{cdocsch2}"|
% \end{tabular}
% \end{center}
% Note that the trailing backslash on each first line
% merely continues the input to the second line
% (for convenient cut ant paste).
% Furthermore, the command |latex| can be replaced by any
% of its alternative versions such as |pdflatex|.
%
% %%%%%%%%%%%%%%%%%%%%%%%%%%%%%%%%%%%%%%%%%%%%%%%%%%%%%%%%%%%%%%%%%%%%%%%%%%%%%%
% %%%%%%%%%%%%%%%%%%%%%%%%%%%%%%%%%%%%%%%%%%%%%%%%%%%%%%%%%%%%%%%%%%%%%%%%%%%%%%
% \section{Implementation}
%\iffalse
%<*package>
%\fi
%
% This section describes the definitions file |childdoc.def|.

% The definitions cannot be loaded using |\usepackage| or |\RequirePackage|
% which has a mechanism to prevent loading a style file more than once.
% When loading the definitions by means of |\input|
% multiple instances have to be prevented manually:
%\iffalse
%This code needs to be before the `\ProvidesFile' directive
%which is defined at the beginning of this file.
%Therefore it is also placed there and commented out here.
%</package>
%<*discard>
%\fi
%    \begin{macrocode}
\ifdefined\childdocmain\endinput\fi
%    \end{macrocode}
%\iffalse
%</discard>
%<*package>
%\fi
%
% \macro{\ifchilddoc}
% \macro{\ifchilddocmanual}
% The conditional |\ifchilddoc| tells whether a
% child (true) or main (false) document is being compiled.
% The conditional |\ifchilddocmanual| tells whether
% the |\includeonly| mechanism is used (false) or
% the selection of child files must be performed manually (true).
% The definitions initialise to false:
%    \begin{macrocode}
\newif\ifchilddoc
\newif\ifchilddocmanual
%    \end{macrocode}

% \macro{\childdocname}
% \macro{\childdocjob}
% The macro |\childdocname| stores the name of the main document
% to be compiled. The macro |\childdocjob| stores the name of
% the document on which the \LaTeX{} compiler was originally invoked.
% The content of |\jobname| cannot be compared
% to filenames specified in the source due to different catcodes.
% The following code rescans |\jobname|, stores the result
% in |\childdocname| and saves a copy in |\childdocjob|:
%    \begin{macrocode}
\edef\childdocname{\scantokens\expandafter{\jobname\noexpand}}
\let\childdocjob\childdocname
%    \end{macrocode}

% \macro{\childdocdisable}
% The macro |\childdocdisable| prevents the main file
% from being processed more than once.
% At this stage, the main document command |\childdocmain|
% is assumed to be called once again where it should do nothing.
% Any subsequent call to it should prevent
% a secondary processing of the main document
% It overwrites the forwarding commands
% |\childdocof| and |\childdocforward|
% with empty macros to prevent further inclusions of the main document:
%    \begin{macrocode}
\newcommand{\childdocdisable}
{
  \renewcommand{\childdocmain}[1]{\renewcommand{\childdocmain}[1]{\endinput}}
  \renewcommand{\childdocof}[1]{}
  \renewcommand{\childdocby}[2][]{}
  \renewcommand{\childdocforward}[2][]{}
  \renewcommand{\childdocdisable}{}
}
%    \end{macrocode}

% \macro{\childdocmain}
% The macro |\childdocmain| is to be called at the top of the main file
% with nothing or the main filename (without extension) as argument.
% First, it breaks loops.
% If the argument is not empty and does not match |\childdocname|
% (which is set by the first inclusion of |childdoc.def|),
% |\ifchilddoc| is set to true, |\includeonly| is applied to the child file
% and |\jobname| is set to the main file
% (for proper handling of |.aux| files):
%    \begin{macrocode}
\newcommand{\childdocmain}[1]
{
  \childdocdisable\childdocmain{}
  \if?#1?\else
    \begingroup
      \def\childdoctmp{#1}
      \ifx\childdoctmp\childdocname
        \def\childdoctmp{}
      \else
        \def\childdoctmp
        {
          \childdoctrue
          \includeonly{\childdocname}
          \def\childdocjob{#1}
          \def\jobname{#1}
        }
      \fi
      \expandafter
    \endgroup
    \childdoctmp
  \fi
}
%    \end{macrocode}

% \macro{\childdocof}
% The command |\childdocof| redirects
% compilation to the main file |#1|.
%    \begin{macrocode}
\newcommand{\childdocof}[1]
{
  \childdocdisable
  \childdoctrue
  \includeonly{\childdocname}
  \def\jobname{#1}
  \def\childdocjob{#1}
  \input{#1}
}
%    \end{macrocode}

% \macro{\childdocby}
% The command |\childdocby| ....
%    \begin{macrocode}
\newcommand{\childdocby}[2][]
{
  \childdocdisable
  \childdoctrue
  \childdocmanualtrue
  \if?#1?\else
    \def\jobname{#2}
  \fi
  \def\childdocjob{#2}
  \input{#2}
  \endinput
}
%    \end{macrocode}

% \macro{\childdocforward}
% The command |\childdocforward| redirects
% compilation to the main file or
% (if the optional argument is given) a child file.
% Parameters are set as if the main file
% or a child file starting with |\childdocof| was compiled.
% Then compilation is handed over to the main file:
%    \begin{macrocode}
\newcommand{\childdocforward}[2][]
{
  \begingroup
    \if?#1?
      \def\childdoctmp
      {
        \def\childdocname{#2}
        \def\childdocjob{#2}
        \def\jobname{#2}
        \input{#2}
        \endinput
      }
    \else
      \def\childdoctmp
      {
        \childdocdisable
        \def\childdocname{#2}
        \childdoctrue
        \includeonly{#2}
        \def\childdocjob{#1}
        \def\jobname{#1}
        \input{#1}
        \endinput
      }
    \fi
    \expandafter
  \endgroup
  \childdoctmp
}
%    \end{macrocode}

% \macro{\childdocforwardprefix}
% The command |\childdocforwardprefix| redirects
% compilation to the main or a child file by means of a pattern.
% The prefix |#1| in the current filename is replaced by |#2|
% and the suffix of the current filename is kept
% (it is assumed that the filename does not contain the substring `|~~~|'
% which is used as a delimiter).
% Compilation is handed over to the new file by |\childdocforward|:
%    \begin{macrocode}
\newcommand{\childdocforwardprefix}[3][]
{
  \begingroup
    \def\childdocextract #2##1~~~{\def\childdoctmp{\childdocforward[#1]{#3##1}}}
    \expandafter\childdocextract\childdocname~~~
    \expandafter
  \endgroup
  \childdoctmp
}
%    \end{macrocode}

% \macro{\childdoc}
% The deprecated macro |\childdoc| is a legacy version of |\childdocmain|:
%    \begin{macrocode}
\newcommand{\childdoc}{\childdocmain}
%    \end{macrocode}

% \macro{\childdocredirect}
% The deprecated macro |\childdocredirect| is a legacy version
% of |\childdocforward| and |\childdocforwardprefix|:
%    \begin{macrocode}
\newcommand{\childdocredirect}[2][]
{
  \begingroup
    \if?#1?
      \def\childdoctmp{\childdocforward{#2}}
    \else
      \def\childdoctmp{\childdocforwardprefix{#1}{#2}}
    \fi
    \expandafter
  \endgroup
  \childdoctmp
}
%    \end{macrocode}

%\iffalse
%</package>
%\fi
%
\endinput
\childdocforward{cdocsch2}"|
% \end{tabular}
% \end{center}
% Note that the trailing backslash on each first line
% merely continues the input to the second line
% (for convenient cut ant paste).
% Furthermore, the command |latex| can be replaced by any
% of its alternative versions such as |pdflatex|.
%
% %%%%%%%%%%%%%%%%%%%%%%%%%%%%%%%%%%%%%%%%%%%%%%%%%%%%%%%%%%%%%%%%%%%%%%%%%%%%%%
% %%%%%%%%%%%%%%%%%%%%%%%%%%%%%%%%%%%%%%%%%%%%%%%%%%%%%%%%%%%%%%%%%%%%%%%%%%%%%%
% \section{Implementation}
%\iffalse
%<*package>
%\fi
%
% This section describes the definitions file |childdoc.def|.

% The definitions cannot be loaded using |\usepackage| or |\RequirePackage|
% which has a mechanism to prevent loading a style file more than once.
% When loading the definitions by means of |\input|
% multiple instances have to be prevented manually:
%\iffalse
%This code needs to be before the `\ProvidesFile' directive
%which is defined at the beginning of this file.
%Therefore it is also placed there and commented out here.
%</package>
%<*discard>
%\fi
%    \begin{macrocode}
\ifdefined\childdocmain\endinput\fi
%    \end{macrocode}
%\iffalse
%</discard>
%<*package>
%\fi
%
% \macro{\ifchilddoc}
% \macro{\ifchilddocmanual}
% The conditional |\ifchilddoc| tells whether a
% child (true) or main (false) document is being compiled.
% The conditional |\ifchilddocmanual| tells whether
% the |\includeonly| mechanism is used (false) or
% the selection of child files must be performed manually (true).
% The definitions initialise to false:
%    \begin{macrocode}
\newif\ifchilddoc
\newif\ifchilddocmanual
%    \end{macrocode}

% \macro{\childdocname}
% \macro{\childdocjob}
% The macro |\childdocname| stores the name of the main document
% to be compiled. The macro |\childdocjob| stores the name of
% the document on which the \LaTeX{} compiler was originally invoked.
% The content of |\jobname| cannot be compared
% to filenames specified in the source due to different catcodes.
% The following code rescans |\jobname|, stores the result
% in |\childdocname| and saves a copy in |\childdocjob|:
%    \begin{macrocode}
\edef\childdocname{\scantokens\expandafter{\jobname\noexpand}}
\let\childdocjob\childdocname
%    \end{macrocode}

% \macro{\childdocdisable}
% The macro |\childdocdisable| prevents the main file
% from being processed more than once.
% At this stage, the main document command |\childdocmain|
% is assumed to be called once again where it should do nothing.
% Any subsequent call to it should prevent
% a secondary processing of the main document
% It overwrites the forwarding commands
% |\childdocof| and |\childdocforward|
% with empty macros to prevent further inclusions of the main document:
%    \begin{macrocode}
\newcommand{\childdocdisable}
{
  \renewcommand{\childdocmain}[1]{\renewcommand{\childdocmain}[1]{\endinput}}
  \renewcommand{\childdocof}[1]{}
  \renewcommand{\childdocby}[2][]{}
  \renewcommand{\childdocforward}[2][]{}
  \renewcommand{\childdocdisable}{}
}
%    \end{macrocode}

% \macro{\childdocmain}
% The macro |\childdocmain| is to be called at the top of the main file
% with nothing or the main filename (without extension) as argument.
% First, it breaks loops.
% If the argument is not empty and does not match |\childdocname|
% (which is set by the first inclusion of |childdoc.def|),
% |\ifchilddoc| is set to true, |\includeonly| is applied to the child file
% and |\jobname| is set to the main file
% (for proper handling of |.aux| files):
%    \begin{macrocode}
\newcommand{\childdocmain}[1]
{
  \childdocdisable\childdocmain{}
  \if?#1?\else
    \begingroup
      \def\childdoctmp{#1}
      \ifx\childdoctmp\childdocname
        \def\childdoctmp{}
      \else
        \def\childdoctmp
        {
          \childdoctrue
          \includeonly{\childdocname}
          \def\childdocjob{#1}
          \def\jobname{#1}
        }
      \fi
      \expandafter
    \endgroup
    \childdoctmp
  \fi
}
%    \end{macrocode}

% \macro{\childdocof}
% The command |\childdocof| redirects
% compilation to the main file |#1|.
%    \begin{macrocode}
\newcommand{\childdocof}[1]
{
  \childdocdisable
  \childdoctrue
  \includeonly{\childdocname}
  \def\jobname{#1}
  \def\childdocjob{#1}
  \input{#1}
}
%    \end{macrocode}

% \macro{\childdocby}
% The command |\childdocby| ....
%    \begin{macrocode}
\newcommand{\childdocby}[2][]
{
  \childdocdisable
  \childdoctrue
  \childdocmanualtrue
  \if?#1?\else
    \def\jobname{#2}
  \fi
  \def\childdocjob{#2}
  \input{#2}
  \endinput
}
%    \end{macrocode}

% \macro{\childdocforward}
% The command |\childdocforward| redirects
% compilation to the main file or
% (if the optional argument is given) a child file.
% Parameters are set as if the main file
% or a child file starting with |\childdocof| was compiled.
% Then compilation is handed over to the main file:
%    \begin{macrocode}
\newcommand{\childdocforward}[2][]
{
  \begingroup
    \if?#1?
      \def\childdoctmp
      {
        \def\childdocname{#2}
        \def\childdocjob{#2}
        \def\jobname{#2}
        \input{#2}
        \endinput
      }
    \else
      \def\childdoctmp
      {
        \childdocdisable
        \def\childdocname{#2}
        \childdoctrue
        \includeonly{#2}
        \def\childdocjob{#1}
        \def\jobname{#1}
        \input{#1}
        \endinput
      }
    \fi
    \expandafter
  \endgroup
  \childdoctmp
}
%    \end{macrocode}

% \macro{\childdocforwardprefix}
% The command |\childdocforwardprefix| redirects
% compilation to the main or a child file by means of a pattern.
% The prefix |#1| in the current filename is replaced by |#2|
% and the suffix of the current filename is kept
% (it is assumed that the filename does not contain the substring `|~~~|'
% which is used as a delimiter).
% Compilation is handed over to the new file by |\childdocforward|:
%    \begin{macrocode}
\newcommand{\childdocforwardprefix}[3][]
{
  \begingroup
    \def\childdocextract #2##1~~~{\def\childdoctmp{\childdocforward[#1]{#3##1}}}
    \expandafter\childdocextract\childdocname~~~
    \expandafter
  \endgroup
  \childdoctmp
}
%    \end{macrocode}

% \macro{\childdoc}
% The deprecated macro |\childdoc| is a legacy version of |\childdocmain|:
%    \begin{macrocode}
\newcommand{\childdoc}{\childdocmain}
%    \end{macrocode}

% \macro{\childdocredirect}
% The deprecated macro |\childdocredirect| is a legacy version
% of |\childdocforward| and |\childdocforwardprefix|:
%    \begin{macrocode}
\newcommand{\childdocredirect}[2][]
{
  \begingroup
    \if?#1?
      \def\childdoctmp{\childdocforward{#2}}
    \else
      \def\childdoctmp{\childdocforwardprefix{#1}{#2}}
    \fi
    \expandafter
  \endgroup
  \childdoctmp
}
%    \end{macrocode}

%\iffalse
%</package>
%\fi
%
\endinput
|\\
|\childdocby{|\textit{main}|}|\\
\end{tabular}
\end{center}
%
Both forms have slightly different effects as described above.
The main file is prepared as usual, see \secref{sec:include}.

%%%%%%%%%%%%%%%%%%%%%%%%%%%%%%%%%%%%%%%%%%%%%%%%%%%%%%%%%%%%%%%%%%%%%%%%%%%%%%%%
\subsection{Legacy Detection}
\label{sec:detection}

The directive |\childdocmain| in the main file can detect
whether the complete document or merely a child is to be compiled
even without using the directive |\childdocof|.
This method is deprecated because it is less robust
and there is no compelling reason to use it;
it is merely provided for backward compatibility
and it may be removed in future versions.

If the detection mechanism is to be used,
it is mandatory to correctly specify
the filename of the main file as the argument of |\childdocmain|:
%
\begin{center}
\begin{tabular}{l}
|% \iffalse
%
% childdoc.dtx Copyright (C) 2017-2018 Niklas Beisert
%
% This work may be distributed and/or modified under the
% conditions of the LaTeX Project Public License, either version 1.3
% of this license or (at your option) any later version.
% The latest version of this license is in
%   http://www.latex-project.org/lppl.txt
% and version 1.3 or later is part of all distributions of LaTeX
% version 2005/12/01 or later.
%
% This work has the LPPL maintenance status `maintained'.
%
% The Current Maintainer of this work is Niklas Beisert.
%
% This work consists of the files childdoc.dtx and childdoc.ins
% and the derived files childdoc.def and cdocsamp.tex with
% cdocsch1.tex, cdocsch2.tex, cdocsdrf.tex, cdocsfn1.tex, cdocsfn2.tex.
%
%<package>\ifdefined\childdocmain\endinput\fi
%<package>\ProvidesFile{childdoc.def}[2018/12/30 v2.0 child document driver]
%<samplemain>\ProvidesFile{cdocsamp.tex}[2018/12/30 v2.0 sample for childdoc]
%<*driver>
%\ProvidesFile{childdoc.drv}[2018/12/30 v2.0 childdoc reference manual file]
\PassOptionsToClass{10pt,a4paper}{article}
\documentclass{ltxdoc}

\usepackage[margin=35mm]{geometry}
\usepackage{hyperref}
\usepackage{hyperxmp}
\usepackage[usenames]{color}

\hypersetup{colorlinks=true}
\hypersetup{pdfstartview=FitH}
\hypersetup{pdfpagemode=UseNone}
\hypersetup{pdfsource={}}
\hypersetup{pdflang={en-UK}}
\hypersetup{pdfcopyright={Copyright 2017-2018 Niklas Beisert.
  This work may be distributed and/or modified under the
  conditions of the LaTeX Project Public License, either version 1.3
  of this license or (at your option) any later version.}}
\hypersetup{pdflicenseurl={http://www.latex-project.org/lppl.txt}}
\hypersetup{pdfcontactaddress={ETH Zurich, ITP, HIT K,
  Wolfgang-Pauli-Strasse 27}}
\hypersetup{pdfcontactpostcode={8093}}
\hypersetup{pdfcontactcity={Zurich}}
\hypersetup{pdfcontactcountry={Switzerland}}
\hypersetup{pdfcontactemail={nbeisert@itp.phys.ethz.ch}}
\hypersetup{pdfcontacturl={http://people.phys.ethz.ch/\xmptilde nbeisert/}}

\newcommand{\secref}[1]{\hyperref[#1]{section \ref*{#1}}}

\parskip1ex
\parindent0pt
\let\olditemize\itemize
\def\itemize{\olditemize\parskip0pt}

\begin{document}

\title{The \textsf{childdoc} Package}
\hypersetup{pdftitle={The childdoc Package}}
\author{Niklas Beisert\\[2ex]
  Institut f\"ur Theoretische Physik\\
  Eidgen\"ossische Technische Hochschule Z\"urich\\
  Wolfgang-Pauli-Strasse 27, 8093 Z\"urich, Switzerland\\[1ex]
  \href{mailto:nbeisert@itp.phys.ethz.ch}
  {\texttt{nbeisert@itp.phys.ethz.ch}}}
\hypersetup{pdfauthor={Niklas Beisert}}
\hypersetup{pdfsubject={Manual for the LaTeX2e Package childdoc}}
\date{30 December 2018, \textsf{v2.0}}
\maketitle

\begin{abstract}\noindent
\textsf{childdoc} is a \LaTeXe{} package
that enables the direct compilation
of document sections included by |\include|
to individual files.
\end{abstract}

\begingroup
\parskip0ex
\tableofcontents
\endgroup

%%%%%%%%%%%%%%%%%%%%%%%%%%%%%%%%%%%%%%%%%%%%%%%%%%%%%%%%%%%%%%%%%%%%%%%%%%%%%%%%
%%%%%%%%%%%%%%%%%%%%%%%%%%%%%%%%%%%%%%%%%%%%%%%%%%%%%%%%%%%%%%%%%%%%%%%%%%%%%%%%
\section{Introduction}

\LaTeX{} provides a mechanism to structure a large document (such as a book)
into a main file and several child files (containing the chapters)
using the |\include| command.
This mechanism is beneficial for documents
which span hundreds of pages in order to
make the source file(s) more manageable.
Moreover, compilation can be restricted to
selected child files by means of the |\includeonly| command.
The latter feature can be used to reduce the compilation time while editing
(this was significantly more useful in the earlier days of \LaTeX{})
or to generate a smaller document which is easier to navigate.
Another application of |\includeonly| is to generate
documents consisting of selected parts of the complete document.

However, there are a few drawbacks of the plain |\include| mechanism:
\begin{itemize}
\item
The child files cannot be compiled on their own,
they can only be compiled via the main file.
A naive editing environment
(such as a text editor with an option
to have the current file processed by \LaTeX)
may require one to switch to the main file before compiling;
attempting to compile the child file produces errors.
\item
The main file must be modified (each time)
to adjust the |\includeonly| command
to the present needs. This easily leaves the main file in a messy state.
\item
The generated document will always carry the filename
of the main document. This is inconvenient if
several child files are to be compiled and
to be kept for distribution.
\end{itemize}

The present package provides a simple interface
to make child files individually compilable by \LaTeX{}.
Compiling a child file then has the same effect as compiling
the main file with an |\includeonly| command
to select the appropriate child.
Moreover the generated document will carry the name of the child
rather than the main file.
This resolves all three above issues.

This feature is meant to make the editing of books,
thesis documents and lecture notes somewhat more convenient.
However, the package can also be used efficiently for
composing a series of documents (such as exercise sheets)
which are typically distributed individually.
It then assists the author in generating the individual documents
(potentially in different versions)
as well as a document containing the collected series.
Another application is in developing style files
or other kinds of included material
where compilation of the style file could redirect
to a sample or test file.

%%%%%%%%%%%%%%%%%%%%%%%%%%%%%%%%%%%%%%%%%%%%%%%%%%%%%%%%%%%%%%%%%%%%%%%%%%%%%%%%
%%%%%%%%%%%%%%%%%%%%%%%%%%%%%%%%%%%%%%%%%%%%%%%%%%%%%%%%%%%%%%%%%%%%%%%%%%%%%%%%
\section{Usage}

First of all, the package \textsf{childdoc} is \emph{not} a standard
\LaTeXe{} |.sty| style file! Therefore it needs to be invoked in
a non-standard way.

%%%%%%%%%%%%%%%%%%%%%%%%%%%%%%%%%%%%%%%%%%%%%%%%%%%%%%%%%%%%%%%%%%%%%%%%%%%%%%%%
\subsection{Included Files}
\label{sec:include}

%%%%%%%%%%%%%%%%%%%%%%%%%%%%%%%%%%%%%%%%
\DescribeMacro{\childdocmain}
To use the package, add the commands
\begin{center}
\begin{tabular}{l}
|% \iffalse
%
% childdoc.dtx Copyright (C) 2017-2018 Niklas Beisert
%
% This work may be distributed and/or modified under the
% conditions of the LaTeX Project Public License, either version 1.3
% of this license or (at your option) any later version.
% The latest version of this license is in
%   http://www.latex-project.org/lppl.txt
% and version 1.3 or later is part of all distributions of LaTeX
% version 2005/12/01 or later.
%
% This work has the LPPL maintenance status `maintained'.
%
% The Current Maintainer of this work is Niklas Beisert.
%
% This work consists of the files childdoc.dtx and childdoc.ins
% and the derived files childdoc.def and cdocsamp.tex with
% cdocsch1.tex, cdocsch2.tex, cdocsdrf.tex, cdocsfn1.tex, cdocsfn2.tex.
%
%<package>\ifdefined\childdocmain\endinput\fi
%<package>\ProvidesFile{childdoc.def}[2018/12/30 v2.0 child document driver]
%<samplemain>\ProvidesFile{cdocsamp.tex}[2018/12/30 v2.0 sample for childdoc]
%<*driver>
%\ProvidesFile{childdoc.drv}[2018/12/30 v2.0 childdoc reference manual file]
\PassOptionsToClass{10pt,a4paper}{article}
\documentclass{ltxdoc}

\usepackage[margin=35mm]{geometry}
\usepackage{hyperref}
\usepackage{hyperxmp}
\usepackage[usenames]{color}

\hypersetup{colorlinks=true}
\hypersetup{pdfstartview=FitH}
\hypersetup{pdfpagemode=UseNone}
\hypersetup{pdfsource={}}
\hypersetup{pdflang={en-UK}}
\hypersetup{pdfcopyright={Copyright 2017-2018 Niklas Beisert.
  This work may be distributed and/or modified under the
  conditions of the LaTeX Project Public License, either version 1.3
  of this license or (at your option) any later version.}}
\hypersetup{pdflicenseurl={http://www.latex-project.org/lppl.txt}}
\hypersetup{pdfcontactaddress={ETH Zurich, ITP, HIT K,
  Wolfgang-Pauli-Strasse 27}}
\hypersetup{pdfcontactpostcode={8093}}
\hypersetup{pdfcontactcity={Zurich}}
\hypersetup{pdfcontactcountry={Switzerland}}
\hypersetup{pdfcontactemail={nbeisert@itp.phys.ethz.ch}}
\hypersetup{pdfcontacturl={http://people.phys.ethz.ch/\xmptilde nbeisert/}}

\newcommand{\secref}[1]{\hyperref[#1]{section \ref*{#1}}}

\parskip1ex
\parindent0pt
\let\olditemize\itemize
\def\itemize{\olditemize\parskip0pt}

\begin{document}

\title{The \textsf{childdoc} Package}
\hypersetup{pdftitle={The childdoc Package}}
\author{Niklas Beisert\\[2ex]
  Institut f\"ur Theoretische Physik\\
  Eidgen\"ossische Technische Hochschule Z\"urich\\
  Wolfgang-Pauli-Strasse 27, 8093 Z\"urich, Switzerland\\[1ex]
  \href{mailto:nbeisert@itp.phys.ethz.ch}
  {\texttt{nbeisert@itp.phys.ethz.ch}}}
\hypersetup{pdfauthor={Niklas Beisert}}
\hypersetup{pdfsubject={Manual for the LaTeX2e Package childdoc}}
\date{30 December 2018, \textsf{v2.0}}
\maketitle

\begin{abstract}\noindent
\textsf{childdoc} is a \LaTeXe{} package
that enables the direct compilation
of document sections included by |\include|
to individual files.
\end{abstract}

\begingroup
\parskip0ex
\tableofcontents
\endgroup

%%%%%%%%%%%%%%%%%%%%%%%%%%%%%%%%%%%%%%%%%%%%%%%%%%%%%%%%%%%%%%%%%%%%%%%%%%%%%%%%
%%%%%%%%%%%%%%%%%%%%%%%%%%%%%%%%%%%%%%%%%%%%%%%%%%%%%%%%%%%%%%%%%%%%%%%%%%%%%%%%
\section{Introduction}

\LaTeX{} provides a mechanism to structure a large document (such as a book)
into a main file and several child files (containing the chapters)
using the |\include| command.
This mechanism is beneficial for documents
which span hundreds of pages in order to
make the source file(s) more manageable.
Moreover, compilation can be restricted to
selected child files by means of the |\includeonly| command.
The latter feature can be used to reduce the compilation time while editing
(this was significantly more useful in the earlier days of \LaTeX{})
or to generate a smaller document which is easier to navigate.
Another application of |\includeonly| is to generate
documents consisting of selected parts of the complete document.

However, there are a few drawbacks of the plain |\include| mechanism:
\begin{itemize}
\item
The child files cannot be compiled on their own,
they can only be compiled via the main file.
A naive editing environment
(such as a text editor with an option
to have the current file processed by \LaTeX)
may require one to switch to the main file before compiling;
attempting to compile the child file produces errors.
\item
The main file must be modified (each time)
to adjust the |\includeonly| command
to the present needs. This easily leaves the main file in a messy state.
\item
The generated document will always carry the filename
of the main document. This is inconvenient if
several child files are to be compiled and
to be kept for distribution.
\end{itemize}

The present package provides a simple interface
to make child files individually compilable by \LaTeX{}.
Compiling a child file then has the same effect as compiling
the main file with an |\includeonly| command
to select the appropriate child.
Moreover the generated document will carry the name of the child
rather than the main file.
This resolves all three above issues.

This feature is meant to make the editing of books,
thesis documents and lecture notes somewhat more convenient.
However, the package can also be used efficiently for
composing a series of documents (such as exercise sheets)
which are typically distributed individually.
It then assists the author in generating the individual documents
(potentially in different versions)
as well as a document containing the collected series.
Another application is in developing style files
or other kinds of included material
where compilation of the style file could redirect
to a sample or test file.

%%%%%%%%%%%%%%%%%%%%%%%%%%%%%%%%%%%%%%%%%%%%%%%%%%%%%%%%%%%%%%%%%%%%%%%%%%%%%%%%
%%%%%%%%%%%%%%%%%%%%%%%%%%%%%%%%%%%%%%%%%%%%%%%%%%%%%%%%%%%%%%%%%%%%%%%%%%%%%%%%
\section{Usage}

First of all, the package \textsf{childdoc} is \emph{not} a standard
\LaTeXe{} |.sty| style file! Therefore it needs to be invoked in
a non-standard way.

%%%%%%%%%%%%%%%%%%%%%%%%%%%%%%%%%%%%%%%%%%%%%%%%%%%%%%%%%%%%%%%%%%%%%%%%%%%%%%%%
\subsection{Included Files}
\label{sec:include}

%%%%%%%%%%%%%%%%%%%%%%%%%%%%%%%%%%%%%%%%
\DescribeMacro{\childdocmain}
To use the package, add the commands
\begin{center}
\begin{tabular}{l}
|\input{childdoc.def}|\\
|\childdocmain{}|\\
\end{tabular}
\end{center}
at the very top of the main \LaTeX{} file,
in particular \emph{before} the |\documentclass| statement!
The argument of |\childdocmain| should be left empty
(but it must be present).

%%%%%%%%%%%%%%%%%%%%%%%%%%%%%%%%%%%%%%%%
\DescribeMacro{\childdocof}
Furthermore, add the commands
\begin{center}
\begin{tabular}{l}
|\input{childdoc.def}|\\
|\childdocof{|\textit{main}|}|\\
\end{tabular}
\end{center}
at the top of every child file \textit{child}
which is included by |\include{|\textit{child}|}|
from within the main file
(or at least for those files to be compiled individually).
The argument \textit{main} must be the filename of the main file.

There are a couple of
considerations in setting up the main and child documents:

%%%%%%%%%%%%%%%%%%%%%%%%%%%%%%%%%%%%%%%%
\paragraph{Restrictions.}

Please note the following restrictions:
\begin{itemize}
\item
|\childdocmain| must be called with one argument \textit{main}
to ensure compatibility with earlier version of the package.
It must either be empty (|\childdocmain{}|)
or precisely match the filename of the main file in which it is specified.
See \secref{sec:detection} for further information.
\item
The filename \textit{main} must be specified without the |.tex| extension.
\item
The filename \textit{main} is case sensitive
(even in case-insensitive file systems)
due to internal string comparison.
\item
The argument \textit{main} should be fully expanded, it cannot be a macro.
\item
Subdirectories and special characters should be avoided in filenames.
\item
The command |\childdocmain{|\textit{main}|}| must be followed by a whitespace.
It should not be followed immediately by another command
or by a comment mark `|%|'.
This is because the \TeX{} parser reads the token immediately following
the argument of |\childdocmain| and puts it
at the beginning of every child section;
however, a white\-space is ignored.
\end{itemize}

%%%%%%%%%%%%%%%%%%%%%%%%%%%%%%%%%%%%%%%%
\paragraph{Content of Main File.}

It is advisable to place all content in the child files included by |\include|.
Any output contained in the main file will appear in all child documents
unless suppressed manually;
it cannot be suppressed automatically by the |\includeonly| directive
and thus should normally be avoided.
A method to include some content in the main file
by means of conditional processing is described in \secref{sec:conditional}.

%%%%%%%%%%%%%%%%%%%%%%%%%%%%%%%%%%%%%%%%
\paragraph{Page Numbering.}

When only a part of the document is compiled,
the appropriate numbering of pages
(as well as other status parameters)
is determined from the |.aux| files.
The latter contain information from previous passes.
However this information needs to propagate through
all intermediate child documents.
Therefore the page numbering in child documents may well
be inconsistent until the complete document is compiled at least once.

A useful (if unconventional) way to always ensure a consistent
page numbering is to restart the numbering in each child document
and denote the pages by `\textit{child}|.|\textit{page}'
where \textit{child} represents the chapter/section number of the child file.
This can be achieved by the command
|\numberwithin{page}{|\textit{child}|}|
of the \textsf{amsmath} package
where \textit{child} can be |chapter| or |section|
depending on the chosen structuring.
Alternatively, one can modify the macro |\thepage| appropriately
and reset the counter |page| at the start of each child file.

%%%%%%%%%%%%%%%%%%%%%%%%%%%%%%%%%%%%%%%%%%%%%%%%%%%%%%%%%%%%%%%%%%%%%%%%%%%%%%%%
\subsection{Conditional Processing}
\label{sec:conditional}

The package provides a mechanism to compile different versions
of a document. To customise the versions further some conditional processing
can come in handy to distinguish which version is being compiled.
The package provides two macros to describe the compilation context:

%%%%%%%%%%%%%%%%%%%%%%%%%%%%%%%%%%%%%%%%
\DescribeMacro{\ifchilddoc}
The conditional |\ifchilddoc| distinguishes between the compilation of
child documents and the main document:
%
\begin{center}
|\ifchilddoc |\textit{child-code}| |[|\||else |\textit{main-code}]| \||fi|
\end{center}

%%%%%%%%%%%%%%%%%%%%%%%%%%%%%%%%%%%%%%%%
\DescribeMacro{\childdocname}
\DescribeMacro{\childdocjob}
The macro |\childdocname| contains the filename (without extension)
of the main or child file being processed.
Note that |\childdocjob| will always contain the name of the main file.

%%%%%%%%%%%%%%%%%%%%%%%%%%%%%%%%%%%%%%%%
\paragraph{Title Page.}

Conditional processing can be used to include a title or banner page
in the main document when proper precautions are taken.
Importantly, the code in the main file should ensure that the page counter
(as well as other status parameters which are stored in the |.aux| files)
takes the same value after the conditional processing.
Otherwise the page numbers may take divergent values
depending on which part is compiled.

For example, a title page could be declared by:
%
\begin{center}
\begin{tabular}{l}
|\ifchilddoc\||else|\\
|\addtocounter{page}{-1}|\\
\textit{code for title page}\\
|\newpage|\\
|\||fi|
\end{tabular}
\end{center}
%
A banner page for the child documents can be generated by:
%
\begin{center}
\begin{tabular}{l}
|\ifchilddoc|\\
|\addtocounter{page}{-1}|\\
\textit{code for banner page}\\
|\newpage|\\
|\||fi|
\end{tabular}
\end{center}
%
Here one could write a message such as:
\begin{center}
|This is the part \childdocname{} of \childdocjob{}.|
\end{center}

%%%%%%%%%%%%%%%%%%%%%%%%%%%%%%%%%%%%%%%%%%%%%%%%%%%%%%%%%%%%%%%%%%%%%%%%%%%%%%%%
\subsection{Flags}
\label{sec:flags}

The package makes it easy to generate different versions
of the main or child documents.
To this end compilation flags can be defined
and assigned different default values.
They will be particularly useful in conjunction
with the forwarding mechanism described in \secref{sec:forward}.

For example, it may be useful to have a flag |\version|
which can be set to |draft| or |final|.
The document source will contain some conditional code
depending on the value of |\version|.
Suppose further, the flag should default to |final| for the main file
and to |draft| for child files
which is a natural assignment for editing the document.
This is achieved by placing the following code
in the preamble of the main document
(below the |\childdocmain| directive):
%
\begin{center}
\begin{tabular}{l}
|\ifchilddoc|\\
|\providecommand{\version}{draft}|\\
|\||else|\\
|\providecommand{\version}{final}|\\
|\||fi|
\end{tabular}
\end{center}
%
The definition by |\providecommand| makes sure
that previous definitions are not overwritten.
Further statements |\providecommand{\version}{...}|
can thus be added before the above code to override it.

For the main file, one might add a line
(between |\childdocmain| and the above block)
%
\begin{center}
|%\ifchilddoc\||else\providecommand{\version}{draft}\||fi|
\end{center}
%
which can be uncommented to produce a draft version.
Likewise one can add a line to the very top of a child file
(above the |\childdocof{|\textit{main}|}| directive)
%
\begin{center}
|%\providecommand{\version}{final}|
\end{center}
%
which can be uncommented to produce the final version of this child document.

%%%%%%%%%%%%%%%%%%%%%%%%%%%%%%%%%%%%%%%%%%%%%%%%%%%%%%%%%%%%%%%%%%%%%%%%%%%%%%%%
\subsection{Forwarding}
\label{sec:forward}

Different versions of the main or child documents
using compilation flags as described in \secref{sec:flags}
can be (permanently) stored in different files
for convenient compilation, viewing and distribution.
To this end, the package defines a command
to pass on compilation to a different file:

%%%%%%%%%%%%%%%%%%%%%%%%%%%%%%%%%%%%%%%%
\DescribeMacro{\childdocforward}
The command |\childdocforward| redirects processing to
another source file:
%
\begin{center}
\begin{tabular}{l}
|\input{childdoc.def}|\\
|\childdocforward[|\textit{main}|]{|\textit{dest}|}|\\
\end{tabular}
\end{center}
%
The argument \textit{dest} is the destination file
(without extension).
It should be the main file or one of the child files.
Note that further \textsf{childdoc} directives
such as |\childdocof| and |\childdocforward|
in the indicated file will be processed in this form.
The optional argument \textit{main}
passes on directly to the main file \textit{main}
while pretending to compile the child \textit{dest}.
This form behaves as if \textit{dest}
issues |\childdocof{|\textit{main}|}| right away,
and no further \textsf{childdoc} directives will be processed.

%%%%%%%%%%%%%%%%%%%%%%%%%%%%%%%%%%%%%%%%
\DescribeMacro{\...prefix}
In the alternative form |\childdocforwardprefix|,
%
\begin{center}
\begin{tabular}{l}
|\input{childdoc.def}|\\
|\childdocforwardprefix[|\textit{main}|]{|\textit{prefix}|}{|\textit{dest}|}|
\end{tabular}
\end{center}
%
the destination file is determined by a pattern
depending on the current file:
To make this work, the current file must be called
`{\textit{prefix}\hspace{0.2em}\textit{suffix}}'
with \textit{prefix} matching precisely the argument.
Processing is then passed on to the file
`{\textit{dest}\hspace{0.2em}\textit{suffix}}'.
Surely, the same effect is achieved by
directly specifying the
argument `{\textit{dest}\hspace{0.2em}\textit{suffix}}'
in the first form.
However, that requires to set up a different file
for each child. With the alternative form of the command
all these files can have exactly the same content
which simplifies setting them up and maintaining them.

For example, the following file |draft.tex|
with a compilation flag |\version| as described in \secref{sec:flags}
compiles the main document as a draft:
%
\begin{center}
\begin{tabular}{l}
|\def\version{draft}|\\
|\input{childdoc.def}|\\
|\childdocforward{|\textit{main}|}|
\end{tabular}
\end{center}
%
Likewise, the following files |final|\textit{nn}|.tex|
compile the final version of the child document
|child|\textit{nn}|.tex|:
%
\begin{center}
\begin{tabular}{l}
|\def\version{final}|\\
|\input{childdoc.def}|\\
|\childdocforwardprefix{final}{child}|
\end{tabular}
\end{center}
%

Note that when several versions of a main file and/or of each child file
are to be generated, it may be convenient to set up a |Makefile| or
shell script to automatise the process.

%%%%%%%%%%%%%%%%%%%%%%%%%%%%%%%%%%%%%%%%%%%%%%%%%%%%%%%%%%%%%%%%%%%%%%%%%%%%%%%%
\subsection{Command Line Processing}
\label{sec:commandline}

The effect of redirection files can also be achieved by invoking
the \LaTeX{} compiler with a more elaborate command line.
Most conveniently this should be done as part
of a shell script or a |Makefile|.

When using \textsf{childdoc} in the main file, the following
command lines effectively perform a redirection
(note that depending on the shell being used,
backslashes may have to be doubled: `|\|' $\to$ `|\\|'):
%
\begin{center}
|... -jobname "|\textit{target}|" |\\|"|[\textit{flags}]%
|\input{childdoc.def}\childdocforward[|\textit{main}|]{|\textit{dest}|}"|
\end{center}
%
Here \textit{target} is the name of the output file,
\textit{main} is the name of the main file
and \textit{dest} is the name of the main or child file to be processed
(all filenames without extensions).
The optional argument \textit{main} can be omitted
if \textit{main} matches \textit{dest}.
Optionally, compilation \textit{flags} can be defined via |\def| commands.
This command line makes the \TeX{} engine believe
it is compiling the file \textit{target}
whose content is specified as the latter parameter.
The provided code then forwards the processing to
\textit{main} or \textit{dest} as described in \secref{sec:forward}.

%%%%%%%%%%%%%%%%%%%%%%%%%%%%%%%%%%%%%%%%%%%%%%%%%%%%%%%%%%%%%%%%%%%%%%%%%%%%%%%%
\subsection{Include by Input}
\label{sec:input}

Including child documents by |\include| has some restrictions by design.
Most notably, the content of a child document always occupies
its own set of pages; pages cannot be shared between child documents.
Usually, this behaviour makes perfect sense
because each child document contain an essential part of the document.
However, in some situations it may be desirable to compose
a document from a collection of parts
without having mandatory page breaks between then.
For this case, the package
provides a mechanism to include parts
by |\input| which can also be processed individually.
However, by construction this mechanism
requires manual handling of the content to be output.

%%%%%%%%%%%%%%%%%%%%%%%%%%%%%%%%%%%%%%%%
\DescribeMacro{\ifchilddocmanual}
The main file should be prepared as usual, see \secref{sec:include}.
However, the document body must make a distinction
between processing of an individual part and of the main document, e.g.:
%
\begin{center}
\begin{tabular}{l}
|\ifchilddocmanual|\\
|\input{\childdocname}|\\
|\||else|\\
\textit{document body with }|\input{|\textit{part}|}|\\
|\||fi|
\end{tabular}
\end{center}
%
The conditional |\ifchilddocmanual| is true whenever
a part to be included by |\input| is being compiled,
and the name of the part is stored in |\childdocname|.

%%%%%%%%%%%%%%%%%%%%%%%%%%%%%%%%%%%%%%%%
\DescribeMacro{\childdocby}
Each part to be included by |\input| should start with:
%
\begin{center}
\begin{tabular}{l}
|\input{childdoc.def}|\\
|\childdocby{|\textit{main}|}|\\
\end{tabular}
\end{center}
%
The directive |\childdocby| is similar to |\childdocof|
described in \secref{sec:include},
but the subsequent selection of content must be done manually.
To that end, both |\ifchilddoc| and |\ifchilddocmanual|
will be true upon processing of a part,
and the name of the part is stored in |\childdocname|.
Note that |\jobname| will be set to the filename of the current part
so that each part receives an individual |.aux| file
that does not interfere with the |.aux| file(s) of the main document.
This behaviour can be altered by the alternative form
|\childdocby[*]{|\textit{main}|}| (with a non-empty optional argument)
which uses the |.aux| file of the main document
by setting |\jobname| to \textit{main}.

%%%%%%%%%%%%%%%%%%%%%%%%%%%%%%%%%%%%%%%%%%%%%%%%%%%%%%%%%%%%%%%%%%%%%%%%%%%%%%%%
\subsection{Driver Development}
\label{sec:driver}

The \textsf{childdoc} mechanism can also be use for the development
of definition files such as \LaTeX{} styles or classes.
This case differs from the above setup with multiple parts
included by |\include| in that no |\includeonly| should be invoked.
This can be achieved by starting the include file
(before |\ProvidesPackage|) with:
%
\begin{center}
\begin{tabular}{l}
|\input{childdoc.def}|\\
|\childdocforward{|\textit{main}|}|\\
\end{tabular}
\end{center}
%
or alternatively with:
%
\begin{center}
\begin{tabular}{l}
|\input{childdoc.def}|\\
|\childdocby{|\textit{main}|}|\\
\end{tabular}
\end{center}
%
Both forms have slightly different effects as described above.
The main file is prepared as usual, see \secref{sec:include}.

%%%%%%%%%%%%%%%%%%%%%%%%%%%%%%%%%%%%%%%%%%%%%%%%%%%%%%%%%%%%%%%%%%%%%%%%%%%%%%%%
\subsection{Legacy Detection}
\label{sec:detection}

The directive |\childdocmain| in the main file can detect
whether the complete document or merely a child is to be compiled
even without using the directive |\childdocof|.
This method is deprecated because it is less robust
and there is no compelling reason to use it;
it is merely provided for backward compatibility
and it may be removed in future versions.

If the detection mechanism is to be used,
it is mandatory to correctly specify
the filename of the main file as the argument of |\childdocmain|:
%
\begin{center}
\begin{tabular}{l}
|\input{childdoc.def}|\\
|\childdocmain{|\textit{main}|}|\\
\end{tabular}
\end{center}
%
If |\jobname| does not match the argument \textit{main} of |\childdocmain|,
it is assumed that |\jobname| points to the child file to be compiled.
When using |\childdocmain| with the main file specified as argument,
it suffices to start a child file
with just |\input{|\textit{main}|}|
without loading of the package and using |\childdocof|.
If instead all processing is done
with the appropriate \textsf{childdoc} directives,
the argument of \textit{main} of |\childdocmain| can be empty.

An alternative version of the command line processing described
in \secref{sec:commandline} using the detection mechanism reads:
%
\begin{center}
|... -jobname "|\textit{target}|" "|[\textit{flags}]%
[|\def\jobname{|\textit{dest}|}|]|\input{|\textit{main}|}"|
\end{center}

%%%%%%%%%%%%%%%%%%%%%%%%%%%%%%%%%%%%%%%%%%%%%%%%%%%%%%%%%%%%%%%%%%%%%%%%%%%%%%%%
\subsection{Manual Code}
\label{sec:manual}

In case one cannot be certain whether the definitions file |childdoc.def|
is installed on the target \TeX{} distribution
and one prefers not to ship it,
it is conceivable to paste a few relevant commands into the sources.

To that end, drop all statements |\input{childdoc.def}|
and perform the replacements as outlined below.
Instead of |\childdocmain{|\textit{main}|}| add the following code
to the top of the main file:
%
\begin{center}
\begin{tabular}{l}
|\||ifdefined\childdocname\endinput\||fi\newif\ifchilddoc|\\
|\edef\childdocname{\scantokens\expandafter{\jobname\noexpand}}|\\
|\def\childdocmain{|\textit{main}|}\||ifx\childdocmain\childdocname\||else|\\
|\childdoctrue\includeonly{\childdocname}\let\jobname\childdocmain\||fi|\\
\end{tabular}
\end{center}
%
Instead of |\childdocof{|\textit{main}|}| just include the main file
at the top of each child file:
%
\begin{center}
|\input{|\textit{main}|}|
\end{center}
%
A simple redirection |\childdocforward{|\textit{dest}|}| is achieved by:
%
\begin{center}
|\def\jobname{|\textit{dest}|}\input{\jobname}|
\end{center}
%
The redirection with prefix
|\childdocforwardprefix[|\textit{prefix}|]{|\textit{dest}|}|
is accomplished by:
%
\begin{center}
\begin{tabular}{l}
|{\edef\jobname{\scantokens\expandafter{\jobname\noexpand}}|\\
|\def\redirectjob |\textit{prefix}|#1~~~{\gdef\jobname{|\textit{dest}|#1}}|\\
|\expandafter\redirectjob\jobname~~~}\input{\jobname}|
\end{tabular}
\end{center}

In an alternative approach,
child documents can be compiled by a specific command line
without additional code or specific definitions:
%
\begin{center}
|... -jobname "|\textit{target}|" "|[\textit{flags}]%
|\includeonly{|\textit{dest}|}\input{|\textit{main}|}"|
\end{center}
%

%%%%%%%%%%%%%%%%%%%%%%%%%%%%%%%%%%%%%%%%%%%%%%%%%%%%%%%%%%%%%%%%%%%%%%%%%%%%%%%%
%%%%%%%%%%%%%%%%%%%%%%%%%%%%%%%%%%%%%%%%%%%%%%%%%%%%%%%%%%%%%%%%%%%%%%%%%%%%%%%%
\section{Information}

%%%%%%%%%%%%%%%%%%%%%%%%%%%%%%%%%%%%%%%%%%%%%%%%%%%%%%%%%%%%%%%%%%%%%%%%%%%%%%%%
\subsection{Copyright}

Copyright \copyright{} 2017--2018 Niklas Beisert

This work may be distributed and/or modified under the
conditions of the \LaTeX{} Project Public License, either version 1.3
of this license or (at your option) any later version.
The latest version of this license is in
  \url{http://www.latex-project.org/lppl.txt}
and version 1.3 or later is part of all distributions of \LaTeX{}
version 2005/12/01 or later.

This work has the LPPL maintenance status `maintained'.

The Current Maintainer of this work is Niklas Beisert.

This work consists of the files |README.txt|, |childdoc.ins| and |childdoc.dtx|
as well as the derived files |childdoc.def|, |cdocsamp.tex|
with |cdocsch1.tex|, |cdocsch2.tex|, |cdocspt3.tex|, |cdocspt4.tex|,
|cdocsdrf.tex|, |cdocsfn1.tex|, |cdocsfn2.tex|
as well as |childdoc.pdf|.

%%%%%%%%%%%%%%%%%%%%%%%%%%%%%%%%%%%%%%%%%%%%%%%%%%%%%%%%%%%%%%%%%%%%%%%%%%%%%%%%
\subsection{Files and Installation}

The package consists of the files:
%
\begin{center}
\begin{tabular}{ll}
    |README.txt|   & readme file \\
    |childdoc.ins| & installation file \\
    |childdoc.dtx| & source file \\
    |childdoc.def| & definition file \\
    |cdocsamp.tex| & sample main file \\
    |cdocsch1.tex| & sample include file \\
    |cdocsch2.tex| & sample include file \\
    |cdocspt3.tex| & sample part file \\
    |cdocspt4.tex| & sample part file \\
    |cdocsdrf.tex| & sample redirection file \\
    |cdocsfn1.tex| & sample redirection file \\
    |cdocsfn2.tex| & sample redirection file \\
    |childdoc.pdf| & manual
\end{tabular}
\end{center}
%
The distribution consists of the files
|README.txt|, |childdoc.ins| and |childdoc.dtx|.
%
\begin{itemize}
\item
Run (pdf)\LaTeX{} on |childdoc.dtx|
to compile the manual |childdoc.pdf| (this file).
\item
Run \LaTeX{} on |childdoc.ins| to create the definitions file |childdoc.def|
and the sample |cdocsamp.tex| with include files
|cdocsch1.tex|, |cdocsch2.tex|, |cdocspt3.tex|, |cdocspt4.tex|,
|cdocsdrf.tex|, |cdocsfn1.tex|, |cdocsfn2.tex|.
Then copy the file |childdoc.def| to an appropriate directory of your \LaTeX{}
distribution, e.g.\ \textit{texmf-root}|/tex/latex/childdoc|.
\end{itemize}

%%%%%%%%%%%%%%%%%%%%%%%%%%%%%%%%%%%%%%%%%%%%%%%%%%%%%%%%%%%%%%%%%%%%%%%%%%%%%%%%
\subsection{Related CTAN Packages}

There are several other packages which offer a similar functionality:
%
\begin{itemize}
\item
The packages
\href{http://ctan.org/pkg/docmute}{\textsf{docmute}},
\href{http://ctan.org/pkg/includex}{\textsf{includex}} and
\href{http://ctan.org/pkg/standalone}{\textsf{standalone}}
provide commands to include only the document body of
a child file thus allowing both files to be compiled individually.
\item
The packages \href{http://ctan.org/pkg/subdocs}{\textsf{subdocs}}
and \href{http://ctan.org/pkg/subfiles}{\textsf{subfiles}}
provide structures in which the main and child documents can be
encapsulated and allowing them to be compiled individually.
The inclusion mechanism is different from the conventional |\include|.
\item
The package \href{http://ctan.org/pkg/combine}{\textsf{combine}}
is an elaborate solution to combine several documents into one.
\end{itemize}
%
See also the CTAN topic \href{http://ctan.org/topic/subdocs}{\textsf{subdocs}}
for further related packages.
The present package differs from the above solutions in that
a document structure constructed with the conventional |\include| mechanism
just needs two extra commands at the top of every file
such that all constituent files can be compiled individually.

%%%%%%%%%%%%%%%%%%%%%%%%%%%%%%%%%%%%%%%%%%%%%%%%%%%%%%%%%%%%%%%%%%%%%%%%%%%%%%%%
%\subsection{Feature Suggestions}
%
%The following is a list of features which may be useful for future
%versions of this package:
%%
%\begin{itemize}
%\item
%\ldots
%\end{itemize}

%%%%%%%%%%%%%%%%%%%%%%%%%%%%%%%%%%%%%%%%%%%%%%%%%%%%%%%%%%%%%%%%%%%%%%%%%%%%%%%%
\subsection{Revision History}

%%%%%%%%%%%%%%%%%%%%%%%%%%%%%%%%%%%%%%%%
\paragraph{v2.0:} 2018/12/30

\begin{itemize}
\item
immediate forward processing
\item
added |\childdocby| mechanism
\item
manual restructured
\end{itemize}

%%%%%%%%%%%%%%%%%%%%%%%%%%%%%%%%%%%%%%%%
\paragraph{v1.6:} 2018/01/17

\begin{itemize}
\item
application for development of include files
\item
corrections to manual
\end{itemize}

%%%%%%%%%%%%%%%%%%%%%%%%%%%%%%%%%%%%%%%%
\paragraph{v1.5:} 2017/05/21

\begin{itemize}
\item
more complete structuring introduced
\item
|\childdocof| introduced
\item
|\childdoc| renamed to |\childdocmain|
\item
|\childredirect| renamed to |\childdocforward| and |\childdocforwardprefix|
and functionality expanded
\end{itemize}

%%%%%%%%%%%%%%%%%%%%%%%%%%%%%%%%%%%%%%%%
\paragraph{v1.0:} 2017/04/27

\begin{itemize}
\item
manual and install package
\item
first version published on CTAN
\end{itemize}

%%%%%%%%%%%%%%%%%%%%%%%%%%%%%%%%%%%%%%%%
\paragraph{v0.6:} 2017/04/26

\begin{itemize}
\item
redirection mechanism added
\end{itemize}

%%%%%%%%%%%%%%%%%%%%%%%%%%%%%%%%%%%%%%%%
\paragraph{v0.5:} 2017/04/26

\begin{itemize}
\item
functionality in definition file
\end{itemize}


%%%%%%%%%%%%%%%%%%%%%%%%%%%%%%%%%%%%%%%%%%%%%%%%%%%%%%%%%%%%%%%%%%%%%%%%%%%%%%%%
%%%%%%%%%%%%%%%%%%%%%%%%%%%%%%%%%%%%%%%%%%%%%%%%%%%%%%%%%%%%%%%%%%%%%%%%%%%%%%%%
%%%%%%%%%%%%%%%%%%%%%%%%%%%%%%%%%%%%%%%%%%%%%%%%%%%%%%%%%%%%%%%%%%%%%%%%%%%%%%%%
\appendix

\settowidth\MacroIndent{\rmfamily\scriptsize 000\ }

 \DocInput{childdoc.dtx}

\end{document}
%</driver>
% \fi
%
% %%%%%%%%%%%%%%%%%%%%%%%%%%%%%%%%%%%%%%%%%%%%%%%%%%%%%%%%%%%%%%%%%%%%%%%%%%%%%%
% %%%%%%%%%%%%%%%%%%%%%%%%%%%%%%%%%%%%%%%%%%%%%%%%%%%%%%%%%%%%%%%%%%%%%%%%%%%%%%
% \section{Sample}
%\iffalse
%<*samplemain>
%\fi
%
% The following presents a sample document
% with two chapters, two parts, a title page,
% a compile flag as well as three forwarding files to set the flag.
% It consists of eight |.tex| files:
% \begin{center}
% \begin{tabular}{ll}
% |cdocsamp.tex|&main file\\
% |cdocsch1.tex|&include file for chapter 1\\
% |cdocsch2.tex|&include file for chapter 2\\
% |cdocspt3.tex|&include file for part 3\\
% |cdocspt4.tex|&include file for part 4\\
% |cdocsdrf.tex|&forwarding file for main file in draft mode\\
% |cdocsfi1.tex|&forwarding file for final version of chapter 1\\
% |cdocsfi2.tex|&forwarding file for final version of chapter 2\\
% \end{tabular}
% \end{center}
% Each of the eight files can be compiled directly by the \LaTeX{} compiler.
%
% %%%%%%%%%%%%%%%%%%%%%%%%%%%%%%%%%%%%%%
% \paragraph{Main File.}
%
% The main file is called |cdocsamp.tex|.
%
% Load the \textsf{childdoc} definitions and
% declare the filename for the main document:
%    \begin{macrocode}
\input{childdoc.def}
\childdocmain{}
%    \end{macrocode}

% Optional override for |\version| flag:
%    \begin{macrocode}
%%\ifchilddoc\else\providecommand{\version}{draft}\fi
%    \end{macrocode}

% Define the default values for the |\version| flag
% (|final| for the main file and |draft| for childs):
%    \begin{macrocode}
\ifchilddoc
\providecommand{\version}{draft}
\else
\providecommand{\version}{final}
\fi
%    \end{macrocode}

% Load the standard document class:
%    \begin{macrocode}
\documentclass[12pt]{article}
%    \end{macrocode}

% Start the document body:
%    \begin{macrocode}
\begin{document}
%    \end{macrocode}

% Declare a title page.
% Print title, part of document being processed and version flag:
%    \begin{macrocode}
\addtocounter{page}{-1}
\begin{center}
{\LARGE\bfseries{}childdoc example\par}
\vspace{1cm}
\ifchilddoc
\ifchilddocmanual part\else chapter\fi:
`\childdocname' of `\childdocjob'\par
\else
main document: `\childdocjob'\par
\fi
version: \version\par
\end{center}
\newpage
%    \end{macrocode}

% Manually include selected file,
% otherwise process as usual:
%    \begin{macrocode}
\ifchilddocmanual
\section*{part `\childdocname'}
\input{\childdocname}
\else
%    \end{macrocode}

% Include the two chapters:
%    \begin{macrocode}
\include{cdocsch1}
\include{cdocsch2}
%    \end{macrocode}

% Include the two parts unless only chapters should be displayed:
%    \begin{macrocode}
\ifchilddoc\else
\section{part three}
\input{cdocspt3}
\section{part four}
\input{cdocspt4}
\fi
%    \end{macrocode}

% Process as usual until here:
%    \begin{macrocode}
\fi
%    \end{macrocode}

% End of document body:
%    \begin{macrocode}
\end{document}
%    \end{macrocode}
%\iffalse
%</samplemain>
%\fi
%
% %%%%%%%%%%%%%%%%%%%%%%%%%%%%%%%%%%%%%%
% \paragraph{Chapter Include Files.}
%
% The include files are called |cdocsch1.tex| and |cdocsch2.tex|.
%
%\iffalse
%<*samplechap1|samplechap2>
%\fi

% Optional override for |\version| flag:
%    \begin{macrocode}
%%\providecommand{\version}{final}
%    \end{macrocode}

% Include the main document:
%    \begin{macrocode}
\input{childdoc.def}
\childdocof{cdocsamp}
%    \end{macrocode}

%\iffalse
%</samplechap1|samplechap2>
%\fi
%
%\iffalse
%<*samplechap1>
%\fi
% Some text for chapter 1:
%    \begin{macrocode}
\section{one}
some text in chapter one
%    \end{macrocode}

%\iffalse
%</samplechap1>
%\fi
% Some text for chapter 2:
%\iffalse
%<*samplechap2>
%\fi
%    \begin{macrocode}
\section{two}
more text in chapter two
%    \end{macrocode}

%\iffalse
%</samplechap2>
%\fi
%
% %%%%%%%%%%%%%%%%%%%%%%%%%%%%%%%%%%%%%%
% \paragraph{Part Include Files.}
%
% The include files are called |cdocspt3.tex| and |cdocspt4.tex|.
%
%\iffalse
%<*samplepart3|samplepart4>
%\fi

% Optional override for |\version| flag:
%    \begin{macrocode}
%%\providecommand{\version}{final}
%    \end{macrocode}

% Include the main document:
%    \begin{macrocode}
\input{childdoc.def}
\childdocby{cdocsamp}
%    \end{macrocode}

%\iffalse
%</samplepart3|samplepart4>
%\fi
%
%\iffalse
%<*samplepart3>
%\fi
% Some text for part 3:
%    \begin{macrocode}
some text in part three
%    \end{macrocode}

%\iffalse
%</samplepart3>
%\fi
% Some text for part 4:
%\iffalse
%<*samplepart4>
%\fi
%    \begin{macrocode}
more text in part four
%    \end{macrocode}

%\iffalse
%</samplepart4>
%\fi
%
% %%%%%%%%%%%%%%%%%%%%%%%%%%%%%%%%%%%%%%
% \paragraph{Forwarding for a Complete Draft.}
%
% The following forwarding file |cdocsdrf.tex|
% compiles the main document in draft mode:
%\iffalse
%<*sampledraft>
%\fi
%    \begin{macrocode}
\def\version{draft}
\input{childdoc.def}
\childdocforward{cdocsamp}
%    \end{macrocode}

%\iffalse
%</sampledraft>
%\fi
%
% %%%%%%%%%%%%%%%%%%%%%%%%%%%%%%%%%%%%%%
% \paragraph{Forwarding for Final Version of the Chapters.}
%
% The following forwarding files |cdocsfn1.tex| and |cdocsfn2.tex|
% (with identical content)
% compile the final versions of the child documents
% |cdocsch1.tex| and |cdocsch2.tex|, respectively:
%\iffalse
%<*samplefinal>
%\fi
%    \begin{macrocode}
\def\version{final}
\input{childdoc.def}
\childdocforwardprefix[cdocsamp]{cdocsfn}{cdocsch}
%    \end{macrocode}

%\iffalse
%</samplefinal>
%\fi
%
% %%%%%%%%%%%%%%%%%%%%%%%%%%%%%%%%%%%%%%
% \paragraph{Command Line Processing.}
%
% The following three command lines generate the output files
% |cdocscld|, |cdocscl1| and |cdocscl2|
% which should be identical to
% |cdocsdrf|, |cdocsch1| and |cdocsfn2|, respectively:
% \begin{center}
% \begin{tabular}{l}
% |latex -jobname cdocscld \|\\
% |  "\def\version{draft}\input{childdoc.def}\childdocforward{cdocsamp}"|\\
% |latex -jobname cdocscl1 \|\\
% |  "\input{childdoc.def}\childdocforward[cdocsamp]{cdocsch1}"|\\
% |latex -jobname cdocscl2 \|\\
% |  "\def\version{final}\input{childdoc.def}\childdocforward{cdocsch2}"|
% \end{tabular}
% \end{center}
% Note that the trailing backslash on each first line
% merely continues the input to the second line
% (for convenient cut ant paste).
% Furthermore, the command |latex| can be replaced by any
% of its alternative versions such as |pdflatex|.
%
% %%%%%%%%%%%%%%%%%%%%%%%%%%%%%%%%%%%%%%%%%%%%%%%%%%%%%%%%%%%%%%%%%%%%%%%%%%%%%%
% %%%%%%%%%%%%%%%%%%%%%%%%%%%%%%%%%%%%%%%%%%%%%%%%%%%%%%%%%%%%%%%%%%%%%%%%%%%%%%
% \section{Implementation}
%\iffalse
%<*package>
%\fi
%
% This section describes the definitions file |childdoc.def|.

% The definitions cannot be loaded using |\usepackage| or |\RequirePackage|
% which has a mechanism to prevent loading a style file more than once.
% When loading the definitions by means of |\input|
% multiple instances have to be prevented manually:
%\iffalse
%This code needs to be before the `\ProvidesFile' directive
%which is defined at the beginning of this file.
%Therefore it is also placed there and commented out here.
%</package>
%<*discard>
%\fi
%    \begin{macrocode}
\ifdefined\childdocmain\endinput\fi
%    \end{macrocode}
%\iffalse
%</discard>
%<*package>
%\fi
%
% \macro{\ifchilddoc}
% \macro{\ifchilddocmanual}
% The conditional |\ifchilddoc| tells whether a
% child (true) or main (false) document is being compiled.
% The conditional |\ifchilddocmanual| tells whether
% the |\includeonly| mechanism is used (false) or
% the selection of child files must be performed manually (true).
% The definitions initialise to false:
%    \begin{macrocode}
\newif\ifchilddoc
\newif\ifchilddocmanual
%    \end{macrocode}

% \macro{\childdocname}
% \macro{\childdocjob}
% The macro |\childdocname| stores the name of the main document
% to be compiled. The macro |\childdocjob| stores the name of
% the document on which the \LaTeX{} compiler was originally invoked.
% The content of |\jobname| cannot be compared
% to filenames specified in the source due to different catcodes.
% The following code rescans |\jobname|, stores the result
% in |\childdocname| and saves a copy in |\childdocjob|:
%    \begin{macrocode}
\edef\childdocname{\scantokens\expandafter{\jobname\noexpand}}
\let\childdocjob\childdocname
%    \end{macrocode}

% \macro{\childdocdisable}
% The macro |\childdocdisable| prevents the main file
% from being processed more than once.
% At this stage, the main document command |\childdocmain|
% is assumed to be called once again where it should do nothing.
% Any subsequent call to it should prevent
% a secondary processing of the main document
% It overwrites the forwarding commands
% |\childdocof| and |\childdocforward|
% with empty macros to prevent further inclusions of the main document:
%    \begin{macrocode}
\newcommand{\childdocdisable}
{
  \renewcommand{\childdocmain}[1]{\renewcommand{\childdocmain}[1]{\endinput}}
  \renewcommand{\childdocof}[1]{}
  \renewcommand{\childdocby}[2][]{}
  \renewcommand{\childdocforward}[2][]{}
  \renewcommand{\childdocdisable}{}
}
%    \end{macrocode}

% \macro{\childdocmain}
% The macro |\childdocmain| is to be called at the top of the main file
% with nothing or the main filename (without extension) as argument.
% First, it breaks loops.
% If the argument is not empty and does not match |\childdocname|
% (which is set by the first inclusion of |childdoc.def|),
% |\ifchilddoc| is set to true, |\includeonly| is applied to the child file
% and |\jobname| is set to the main file
% (for proper handling of |.aux| files):
%    \begin{macrocode}
\newcommand{\childdocmain}[1]
{
  \childdocdisable\childdocmain{}
  \if?#1?\else
    \begingroup
      \def\childdoctmp{#1}
      \ifx\childdoctmp\childdocname
        \def\childdoctmp{}
      \else
        \def\childdoctmp
        {
          \childdoctrue
          \includeonly{\childdocname}
          \def\childdocjob{#1}
          \def\jobname{#1}
        }
      \fi
      \expandafter
    \endgroup
    \childdoctmp
  \fi
}
%    \end{macrocode}

% \macro{\childdocof}
% The command |\childdocof| redirects
% compilation to the main file |#1|.
%    \begin{macrocode}
\newcommand{\childdocof}[1]
{
  \childdocdisable
  \childdoctrue
  \includeonly{\childdocname}
  \def\jobname{#1}
  \def\childdocjob{#1}
  \input{#1}
}
%    \end{macrocode}

% \macro{\childdocby}
% The command |\childdocby| ....
%    \begin{macrocode}
\newcommand{\childdocby}[2][]
{
  \childdocdisable
  \childdoctrue
  \childdocmanualtrue
  \if?#1?\else
    \def\jobname{#2}
  \fi
  \def\childdocjob{#2}
  \input{#2}
  \endinput
}
%    \end{macrocode}

% \macro{\childdocforward}
% The command |\childdocforward| redirects
% compilation to the main file or
% (if the optional argument is given) a child file.
% Parameters are set as if the main file
% or a child file starting with |\childdocof| was compiled.
% Then compilation is handed over to the main file:
%    \begin{macrocode}
\newcommand{\childdocforward}[2][]
{
  \begingroup
    \if?#1?
      \def\childdoctmp
      {
        \def\childdocname{#2}
        \def\childdocjob{#2}
        \def\jobname{#2}
        \input{#2}
        \endinput
      }
    \else
      \def\childdoctmp
      {
        \childdocdisable
        \def\childdocname{#2}
        \childdoctrue
        \includeonly{#2}
        \def\childdocjob{#1}
        \def\jobname{#1}
        \input{#1}
        \endinput
      }
    \fi
    \expandafter
  \endgroup
  \childdoctmp
}
%    \end{macrocode}

% \macro{\childdocforwardprefix}
% The command |\childdocforwardprefix| redirects
% compilation to the main or a child file by means of a pattern.
% The prefix |#1| in the current filename is replaced by |#2|
% and the suffix of the current filename is kept
% (it is assumed that the filename does not contain the substring `|~~~|'
% which is used as a delimiter).
% Compilation is handed over to the new file by |\childdocforward|:
%    \begin{macrocode}
\newcommand{\childdocforwardprefix}[3][]
{
  \begingroup
    \def\childdocextract #2##1~~~{\def\childdoctmp{\childdocforward[#1]{#3##1}}}
    \expandafter\childdocextract\childdocname~~~
    \expandafter
  \endgroup
  \childdoctmp
}
%    \end{macrocode}

% \macro{\childdoc}
% The deprecated macro |\childdoc| is a legacy version of |\childdocmain|:
%    \begin{macrocode}
\newcommand{\childdoc}{\childdocmain}
%    \end{macrocode}

% \macro{\childdocredirect}
% The deprecated macro |\childdocredirect| is a legacy version
% of |\childdocforward| and |\childdocforwardprefix|:
%    \begin{macrocode}
\newcommand{\childdocredirect}[2][]
{
  \begingroup
    \if?#1?
      \def\childdoctmp{\childdocforward{#2}}
    \else
      \def\childdoctmp{\childdocforwardprefix{#1}{#2}}
    \fi
    \expandafter
  \endgroup
  \childdoctmp
}
%    \end{macrocode}

%\iffalse
%</package>
%\fi
%
\endinput
|\\
|\childdocmain{}|\\
\end{tabular}
\end{center}
at the very top of the main \LaTeX{} file,
in particular \emph{before} the |\documentclass| statement!
The argument of |\childdocmain| should be left empty
(but it must be present).

%%%%%%%%%%%%%%%%%%%%%%%%%%%%%%%%%%%%%%%%
\DescribeMacro{\childdocof}
Furthermore, add the commands
\begin{center}
\begin{tabular}{l}
|% \iffalse
%
% childdoc.dtx Copyright (C) 2017-2018 Niklas Beisert
%
% This work may be distributed and/or modified under the
% conditions of the LaTeX Project Public License, either version 1.3
% of this license or (at your option) any later version.
% The latest version of this license is in
%   http://www.latex-project.org/lppl.txt
% and version 1.3 or later is part of all distributions of LaTeX
% version 2005/12/01 or later.
%
% This work has the LPPL maintenance status `maintained'.
%
% The Current Maintainer of this work is Niklas Beisert.
%
% This work consists of the files childdoc.dtx and childdoc.ins
% and the derived files childdoc.def and cdocsamp.tex with
% cdocsch1.tex, cdocsch2.tex, cdocsdrf.tex, cdocsfn1.tex, cdocsfn2.tex.
%
%<package>\ifdefined\childdocmain\endinput\fi
%<package>\ProvidesFile{childdoc.def}[2018/12/30 v2.0 child document driver]
%<samplemain>\ProvidesFile{cdocsamp.tex}[2018/12/30 v2.0 sample for childdoc]
%<*driver>
%\ProvidesFile{childdoc.drv}[2018/12/30 v2.0 childdoc reference manual file]
\PassOptionsToClass{10pt,a4paper}{article}
\documentclass{ltxdoc}

\usepackage[margin=35mm]{geometry}
\usepackage{hyperref}
\usepackage{hyperxmp}
\usepackage[usenames]{color}

\hypersetup{colorlinks=true}
\hypersetup{pdfstartview=FitH}
\hypersetup{pdfpagemode=UseNone}
\hypersetup{pdfsource={}}
\hypersetup{pdflang={en-UK}}
\hypersetup{pdfcopyright={Copyright 2017-2018 Niklas Beisert.
  This work may be distributed and/or modified under the
  conditions of the LaTeX Project Public License, either version 1.3
  of this license or (at your option) any later version.}}
\hypersetup{pdflicenseurl={http://www.latex-project.org/lppl.txt}}
\hypersetup{pdfcontactaddress={ETH Zurich, ITP, HIT K,
  Wolfgang-Pauli-Strasse 27}}
\hypersetup{pdfcontactpostcode={8093}}
\hypersetup{pdfcontactcity={Zurich}}
\hypersetup{pdfcontactcountry={Switzerland}}
\hypersetup{pdfcontactemail={nbeisert@itp.phys.ethz.ch}}
\hypersetup{pdfcontacturl={http://people.phys.ethz.ch/\xmptilde nbeisert/}}

\newcommand{\secref}[1]{\hyperref[#1]{section \ref*{#1}}}

\parskip1ex
\parindent0pt
\let\olditemize\itemize
\def\itemize{\olditemize\parskip0pt}

\begin{document}

\title{The \textsf{childdoc} Package}
\hypersetup{pdftitle={The childdoc Package}}
\author{Niklas Beisert\\[2ex]
  Institut f\"ur Theoretische Physik\\
  Eidgen\"ossische Technische Hochschule Z\"urich\\
  Wolfgang-Pauli-Strasse 27, 8093 Z\"urich, Switzerland\\[1ex]
  \href{mailto:nbeisert@itp.phys.ethz.ch}
  {\texttt{nbeisert@itp.phys.ethz.ch}}}
\hypersetup{pdfauthor={Niklas Beisert}}
\hypersetup{pdfsubject={Manual for the LaTeX2e Package childdoc}}
\date{30 December 2018, \textsf{v2.0}}
\maketitle

\begin{abstract}\noindent
\textsf{childdoc} is a \LaTeXe{} package
that enables the direct compilation
of document sections included by |\include|
to individual files.
\end{abstract}

\begingroup
\parskip0ex
\tableofcontents
\endgroup

%%%%%%%%%%%%%%%%%%%%%%%%%%%%%%%%%%%%%%%%%%%%%%%%%%%%%%%%%%%%%%%%%%%%%%%%%%%%%%%%
%%%%%%%%%%%%%%%%%%%%%%%%%%%%%%%%%%%%%%%%%%%%%%%%%%%%%%%%%%%%%%%%%%%%%%%%%%%%%%%%
\section{Introduction}

\LaTeX{} provides a mechanism to structure a large document (such as a book)
into a main file and several child files (containing the chapters)
using the |\include| command.
This mechanism is beneficial for documents
which span hundreds of pages in order to
make the source file(s) more manageable.
Moreover, compilation can be restricted to
selected child files by means of the |\includeonly| command.
The latter feature can be used to reduce the compilation time while editing
(this was significantly more useful in the earlier days of \LaTeX{})
or to generate a smaller document which is easier to navigate.
Another application of |\includeonly| is to generate
documents consisting of selected parts of the complete document.

However, there are a few drawbacks of the plain |\include| mechanism:
\begin{itemize}
\item
The child files cannot be compiled on their own,
they can only be compiled via the main file.
A naive editing environment
(such as a text editor with an option
to have the current file processed by \LaTeX)
may require one to switch to the main file before compiling;
attempting to compile the child file produces errors.
\item
The main file must be modified (each time)
to adjust the |\includeonly| command
to the present needs. This easily leaves the main file in a messy state.
\item
The generated document will always carry the filename
of the main document. This is inconvenient if
several child files are to be compiled and
to be kept for distribution.
\end{itemize}

The present package provides a simple interface
to make child files individually compilable by \LaTeX{}.
Compiling a child file then has the same effect as compiling
the main file with an |\includeonly| command
to select the appropriate child.
Moreover the generated document will carry the name of the child
rather than the main file.
This resolves all three above issues.

This feature is meant to make the editing of books,
thesis documents and lecture notes somewhat more convenient.
However, the package can also be used efficiently for
composing a series of documents (such as exercise sheets)
which are typically distributed individually.
It then assists the author in generating the individual documents
(potentially in different versions)
as well as a document containing the collected series.
Another application is in developing style files
or other kinds of included material
where compilation of the style file could redirect
to a sample or test file.

%%%%%%%%%%%%%%%%%%%%%%%%%%%%%%%%%%%%%%%%%%%%%%%%%%%%%%%%%%%%%%%%%%%%%%%%%%%%%%%%
%%%%%%%%%%%%%%%%%%%%%%%%%%%%%%%%%%%%%%%%%%%%%%%%%%%%%%%%%%%%%%%%%%%%%%%%%%%%%%%%
\section{Usage}

First of all, the package \textsf{childdoc} is \emph{not} a standard
\LaTeXe{} |.sty| style file! Therefore it needs to be invoked in
a non-standard way.

%%%%%%%%%%%%%%%%%%%%%%%%%%%%%%%%%%%%%%%%%%%%%%%%%%%%%%%%%%%%%%%%%%%%%%%%%%%%%%%%
\subsection{Included Files}
\label{sec:include}

%%%%%%%%%%%%%%%%%%%%%%%%%%%%%%%%%%%%%%%%
\DescribeMacro{\childdocmain}
To use the package, add the commands
\begin{center}
\begin{tabular}{l}
|\input{childdoc.def}|\\
|\childdocmain{}|\\
\end{tabular}
\end{center}
at the very top of the main \LaTeX{} file,
in particular \emph{before} the |\documentclass| statement!
The argument of |\childdocmain| should be left empty
(but it must be present).

%%%%%%%%%%%%%%%%%%%%%%%%%%%%%%%%%%%%%%%%
\DescribeMacro{\childdocof}
Furthermore, add the commands
\begin{center}
\begin{tabular}{l}
|\input{childdoc.def}|\\
|\childdocof{|\textit{main}|}|\\
\end{tabular}
\end{center}
at the top of every child file \textit{child}
which is included by |\include{|\textit{child}|}|
from within the main file
(or at least for those files to be compiled individually).
The argument \textit{main} must be the filename of the main file.

There are a couple of
considerations in setting up the main and child documents:

%%%%%%%%%%%%%%%%%%%%%%%%%%%%%%%%%%%%%%%%
\paragraph{Restrictions.}

Please note the following restrictions:
\begin{itemize}
\item
|\childdocmain| must be called with one argument \textit{main}
to ensure compatibility with earlier version of the package.
It must either be empty (|\childdocmain{}|)
or precisely match the filename of the main file in which it is specified.
See \secref{sec:detection} for further information.
\item
The filename \textit{main} must be specified without the |.tex| extension.
\item
The filename \textit{main} is case sensitive
(even in case-insensitive file systems)
due to internal string comparison.
\item
The argument \textit{main} should be fully expanded, it cannot be a macro.
\item
Subdirectories and special characters should be avoided in filenames.
\item
The command |\childdocmain{|\textit{main}|}| must be followed by a whitespace.
It should not be followed immediately by another command
or by a comment mark `|%|'.
This is because the \TeX{} parser reads the token immediately following
the argument of |\childdocmain| and puts it
at the beginning of every child section;
however, a white\-space is ignored.
\end{itemize}

%%%%%%%%%%%%%%%%%%%%%%%%%%%%%%%%%%%%%%%%
\paragraph{Content of Main File.}

It is advisable to place all content in the child files included by |\include|.
Any output contained in the main file will appear in all child documents
unless suppressed manually;
it cannot be suppressed automatically by the |\includeonly| directive
and thus should normally be avoided.
A method to include some content in the main file
by means of conditional processing is described in \secref{sec:conditional}.

%%%%%%%%%%%%%%%%%%%%%%%%%%%%%%%%%%%%%%%%
\paragraph{Page Numbering.}

When only a part of the document is compiled,
the appropriate numbering of pages
(as well as other status parameters)
is determined from the |.aux| files.
The latter contain information from previous passes.
However this information needs to propagate through
all intermediate child documents.
Therefore the page numbering in child documents may well
be inconsistent until the complete document is compiled at least once.

A useful (if unconventional) way to always ensure a consistent
page numbering is to restart the numbering in each child document
and denote the pages by `\textit{child}|.|\textit{page}'
where \textit{child} represents the chapter/section number of the child file.
This can be achieved by the command
|\numberwithin{page}{|\textit{child}|}|
of the \textsf{amsmath} package
where \textit{child} can be |chapter| or |section|
depending on the chosen structuring.
Alternatively, one can modify the macro |\thepage| appropriately
and reset the counter |page| at the start of each child file.

%%%%%%%%%%%%%%%%%%%%%%%%%%%%%%%%%%%%%%%%%%%%%%%%%%%%%%%%%%%%%%%%%%%%%%%%%%%%%%%%
\subsection{Conditional Processing}
\label{sec:conditional}

The package provides a mechanism to compile different versions
of a document. To customise the versions further some conditional processing
can come in handy to distinguish which version is being compiled.
The package provides two macros to describe the compilation context:

%%%%%%%%%%%%%%%%%%%%%%%%%%%%%%%%%%%%%%%%
\DescribeMacro{\ifchilddoc}
The conditional |\ifchilddoc| distinguishes between the compilation of
child documents and the main document:
%
\begin{center}
|\ifchilddoc |\textit{child-code}| |[|\||else |\textit{main-code}]| \||fi|
\end{center}

%%%%%%%%%%%%%%%%%%%%%%%%%%%%%%%%%%%%%%%%
\DescribeMacro{\childdocname}
\DescribeMacro{\childdocjob}
The macro |\childdocname| contains the filename (without extension)
of the main or child file being processed.
Note that |\childdocjob| will always contain the name of the main file.

%%%%%%%%%%%%%%%%%%%%%%%%%%%%%%%%%%%%%%%%
\paragraph{Title Page.}

Conditional processing can be used to include a title or banner page
in the main document when proper precautions are taken.
Importantly, the code in the main file should ensure that the page counter
(as well as other status parameters which are stored in the |.aux| files)
takes the same value after the conditional processing.
Otherwise the page numbers may take divergent values
depending on which part is compiled.

For example, a title page could be declared by:
%
\begin{center}
\begin{tabular}{l}
|\ifchilddoc\||else|\\
|\addtocounter{page}{-1}|\\
\textit{code for title page}\\
|\newpage|\\
|\||fi|
\end{tabular}
\end{center}
%
A banner page for the child documents can be generated by:
%
\begin{center}
\begin{tabular}{l}
|\ifchilddoc|\\
|\addtocounter{page}{-1}|\\
\textit{code for banner page}\\
|\newpage|\\
|\||fi|
\end{tabular}
\end{center}
%
Here one could write a message such as:
\begin{center}
|This is the part \childdocname{} of \childdocjob{}.|
\end{center}

%%%%%%%%%%%%%%%%%%%%%%%%%%%%%%%%%%%%%%%%%%%%%%%%%%%%%%%%%%%%%%%%%%%%%%%%%%%%%%%%
\subsection{Flags}
\label{sec:flags}

The package makes it easy to generate different versions
of the main or child documents.
To this end compilation flags can be defined
and assigned different default values.
They will be particularly useful in conjunction
with the forwarding mechanism described in \secref{sec:forward}.

For example, it may be useful to have a flag |\version|
which can be set to |draft| or |final|.
The document source will contain some conditional code
depending on the value of |\version|.
Suppose further, the flag should default to |final| for the main file
and to |draft| for child files
which is a natural assignment for editing the document.
This is achieved by placing the following code
in the preamble of the main document
(below the |\childdocmain| directive):
%
\begin{center}
\begin{tabular}{l}
|\ifchilddoc|\\
|\providecommand{\version}{draft}|\\
|\||else|\\
|\providecommand{\version}{final}|\\
|\||fi|
\end{tabular}
\end{center}
%
The definition by |\providecommand| makes sure
that previous definitions are not overwritten.
Further statements |\providecommand{\version}{...}|
can thus be added before the above code to override it.

For the main file, one might add a line
(between |\childdocmain| and the above block)
%
\begin{center}
|%\ifchilddoc\||else\providecommand{\version}{draft}\||fi|
\end{center}
%
which can be uncommented to produce a draft version.
Likewise one can add a line to the very top of a child file
(above the |\childdocof{|\textit{main}|}| directive)
%
\begin{center}
|%\providecommand{\version}{final}|
\end{center}
%
which can be uncommented to produce the final version of this child document.

%%%%%%%%%%%%%%%%%%%%%%%%%%%%%%%%%%%%%%%%%%%%%%%%%%%%%%%%%%%%%%%%%%%%%%%%%%%%%%%%
\subsection{Forwarding}
\label{sec:forward}

Different versions of the main or child documents
using compilation flags as described in \secref{sec:flags}
can be (permanently) stored in different files
for convenient compilation, viewing and distribution.
To this end, the package defines a command
to pass on compilation to a different file:

%%%%%%%%%%%%%%%%%%%%%%%%%%%%%%%%%%%%%%%%
\DescribeMacro{\childdocforward}
The command |\childdocforward| redirects processing to
another source file:
%
\begin{center}
\begin{tabular}{l}
|\input{childdoc.def}|\\
|\childdocforward[|\textit{main}|]{|\textit{dest}|}|\\
\end{tabular}
\end{center}
%
The argument \textit{dest} is the destination file
(without extension).
It should be the main file or one of the child files.
Note that further \textsf{childdoc} directives
such as |\childdocof| and |\childdocforward|
in the indicated file will be processed in this form.
The optional argument \textit{main}
passes on directly to the main file \textit{main}
while pretending to compile the child \textit{dest}.
This form behaves as if \textit{dest}
issues |\childdocof{|\textit{main}|}| right away,
and no further \textsf{childdoc} directives will be processed.

%%%%%%%%%%%%%%%%%%%%%%%%%%%%%%%%%%%%%%%%
\DescribeMacro{\...prefix}
In the alternative form |\childdocforwardprefix|,
%
\begin{center}
\begin{tabular}{l}
|\input{childdoc.def}|\\
|\childdocforwardprefix[|\textit{main}|]{|\textit{prefix}|}{|\textit{dest}|}|
\end{tabular}
\end{center}
%
the destination file is determined by a pattern
depending on the current file:
To make this work, the current file must be called
`{\textit{prefix}\hspace{0.2em}\textit{suffix}}'
with \textit{prefix} matching precisely the argument.
Processing is then passed on to the file
`{\textit{dest}\hspace{0.2em}\textit{suffix}}'.
Surely, the same effect is achieved by
directly specifying the
argument `{\textit{dest}\hspace{0.2em}\textit{suffix}}'
in the first form.
However, that requires to set up a different file
for each child. With the alternative form of the command
all these files can have exactly the same content
which simplifies setting them up and maintaining them.

For example, the following file |draft.tex|
with a compilation flag |\version| as described in \secref{sec:flags}
compiles the main document as a draft:
%
\begin{center}
\begin{tabular}{l}
|\def\version{draft}|\\
|\input{childdoc.def}|\\
|\childdocforward{|\textit{main}|}|
\end{tabular}
\end{center}
%
Likewise, the following files |final|\textit{nn}|.tex|
compile the final version of the child document
|child|\textit{nn}|.tex|:
%
\begin{center}
\begin{tabular}{l}
|\def\version{final}|\\
|\input{childdoc.def}|\\
|\childdocforwardprefix{final}{child}|
\end{tabular}
\end{center}
%

Note that when several versions of a main file and/or of each child file
are to be generated, it may be convenient to set up a |Makefile| or
shell script to automatise the process.

%%%%%%%%%%%%%%%%%%%%%%%%%%%%%%%%%%%%%%%%%%%%%%%%%%%%%%%%%%%%%%%%%%%%%%%%%%%%%%%%
\subsection{Command Line Processing}
\label{sec:commandline}

The effect of redirection files can also be achieved by invoking
the \LaTeX{} compiler with a more elaborate command line.
Most conveniently this should be done as part
of a shell script or a |Makefile|.

When using \textsf{childdoc} in the main file, the following
command lines effectively perform a redirection
(note that depending on the shell being used,
backslashes may have to be doubled: `|\|' $\to$ `|\\|'):
%
\begin{center}
|... -jobname "|\textit{target}|" |\\|"|[\textit{flags}]%
|\input{childdoc.def}\childdocforward[|\textit{main}|]{|\textit{dest}|}"|
\end{center}
%
Here \textit{target} is the name of the output file,
\textit{main} is the name of the main file
and \textit{dest} is the name of the main or child file to be processed
(all filenames without extensions).
The optional argument \textit{main} can be omitted
if \textit{main} matches \textit{dest}.
Optionally, compilation \textit{flags} can be defined via |\def| commands.
This command line makes the \TeX{} engine believe
it is compiling the file \textit{target}
whose content is specified as the latter parameter.
The provided code then forwards the processing to
\textit{main} or \textit{dest} as described in \secref{sec:forward}.

%%%%%%%%%%%%%%%%%%%%%%%%%%%%%%%%%%%%%%%%%%%%%%%%%%%%%%%%%%%%%%%%%%%%%%%%%%%%%%%%
\subsection{Include by Input}
\label{sec:input}

Including child documents by |\include| has some restrictions by design.
Most notably, the content of a child document always occupies
its own set of pages; pages cannot be shared between child documents.
Usually, this behaviour makes perfect sense
because each child document contain an essential part of the document.
However, in some situations it may be desirable to compose
a document from a collection of parts
without having mandatory page breaks between then.
For this case, the package
provides a mechanism to include parts
by |\input| which can also be processed individually.
However, by construction this mechanism
requires manual handling of the content to be output.

%%%%%%%%%%%%%%%%%%%%%%%%%%%%%%%%%%%%%%%%
\DescribeMacro{\ifchilddocmanual}
The main file should be prepared as usual, see \secref{sec:include}.
However, the document body must make a distinction
between processing of an individual part and of the main document, e.g.:
%
\begin{center}
\begin{tabular}{l}
|\ifchilddocmanual|\\
|\input{\childdocname}|\\
|\||else|\\
\textit{document body with }|\input{|\textit{part}|}|\\
|\||fi|
\end{tabular}
\end{center}
%
The conditional |\ifchilddocmanual| is true whenever
a part to be included by |\input| is being compiled,
and the name of the part is stored in |\childdocname|.

%%%%%%%%%%%%%%%%%%%%%%%%%%%%%%%%%%%%%%%%
\DescribeMacro{\childdocby}
Each part to be included by |\input| should start with:
%
\begin{center}
\begin{tabular}{l}
|\input{childdoc.def}|\\
|\childdocby{|\textit{main}|}|\\
\end{tabular}
\end{center}
%
The directive |\childdocby| is similar to |\childdocof|
described in \secref{sec:include},
but the subsequent selection of content must be done manually.
To that end, both |\ifchilddoc| and |\ifchilddocmanual|
will be true upon processing of a part,
and the name of the part is stored in |\childdocname|.
Note that |\jobname| will be set to the filename of the current part
so that each part receives an individual |.aux| file
that does not interfere with the |.aux| file(s) of the main document.
This behaviour can be altered by the alternative form
|\childdocby[*]{|\textit{main}|}| (with a non-empty optional argument)
which uses the |.aux| file of the main document
by setting |\jobname| to \textit{main}.

%%%%%%%%%%%%%%%%%%%%%%%%%%%%%%%%%%%%%%%%%%%%%%%%%%%%%%%%%%%%%%%%%%%%%%%%%%%%%%%%
\subsection{Driver Development}
\label{sec:driver}

The \textsf{childdoc} mechanism can also be use for the development
of definition files such as \LaTeX{} styles or classes.
This case differs from the above setup with multiple parts
included by |\include| in that no |\includeonly| should be invoked.
This can be achieved by starting the include file
(before |\ProvidesPackage|) with:
%
\begin{center}
\begin{tabular}{l}
|\input{childdoc.def}|\\
|\childdocforward{|\textit{main}|}|\\
\end{tabular}
\end{center}
%
or alternatively with:
%
\begin{center}
\begin{tabular}{l}
|\input{childdoc.def}|\\
|\childdocby{|\textit{main}|}|\\
\end{tabular}
\end{center}
%
Both forms have slightly different effects as described above.
The main file is prepared as usual, see \secref{sec:include}.

%%%%%%%%%%%%%%%%%%%%%%%%%%%%%%%%%%%%%%%%%%%%%%%%%%%%%%%%%%%%%%%%%%%%%%%%%%%%%%%%
\subsection{Legacy Detection}
\label{sec:detection}

The directive |\childdocmain| in the main file can detect
whether the complete document or merely a child is to be compiled
even without using the directive |\childdocof|.
This method is deprecated because it is less robust
and there is no compelling reason to use it;
it is merely provided for backward compatibility
and it may be removed in future versions.

If the detection mechanism is to be used,
it is mandatory to correctly specify
the filename of the main file as the argument of |\childdocmain|:
%
\begin{center}
\begin{tabular}{l}
|\input{childdoc.def}|\\
|\childdocmain{|\textit{main}|}|\\
\end{tabular}
\end{center}
%
If |\jobname| does not match the argument \textit{main} of |\childdocmain|,
it is assumed that |\jobname| points to the child file to be compiled.
When using |\childdocmain| with the main file specified as argument,
it suffices to start a child file
with just |\input{|\textit{main}|}|
without loading of the package and using |\childdocof|.
If instead all processing is done
with the appropriate \textsf{childdoc} directives,
the argument of \textit{main} of |\childdocmain| can be empty.

An alternative version of the command line processing described
in \secref{sec:commandline} using the detection mechanism reads:
%
\begin{center}
|... -jobname "|\textit{target}|" "|[\textit{flags}]%
[|\def\jobname{|\textit{dest}|}|]|\input{|\textit{main}|}"|
\end{center}

%%%%%%%%%%%%%%%%%%%%%%%%%%%%%%%%%%%%%%%%%%%%%%%%%%%%%%%%%%%%%%%%%%%%%%%%%%%%%%%%
\subsection{Manual Code}
\label{sec:manual}

In case one cannot be certain whether the definitions file |childdoc.def|
is installed on the target \TeX{} distribution
and one prefers not to ship it,
it is conceivable to paste a few relevant commands into the sources.

To that end, drop all statements |\input{childdoc.def}|
and perform the replacements as outlined below.
Instead of |\childdocmain{|\textit{main}|}| add the following code
to the top of the main file:
%
\begin{center}
\begin{tabular}{l}
|\||ifdefined\childdocname\endinput\||fi\newif\ifchilddoc|\\
|\edef\childdocname{\scantokens\expandafter{\jobname\noexpand}}|\\
|\def\childdocmain{|\textit{main}|}\||ifx\childdocmain\childdocname\||else|\\
|\childdoctrue\includeonly{\childdocname}\let\jobname\childdocmain\||fi|\\
\end{tabular}
\end{center}
%
Instead of |\childdocof{|\textit{main}|}| just include the main file
at the top of each child file:
%
\begin{center}
|\input{|\textit{main}|}|
\end{center}
%
A simple redirection |\childdocforward{|\textit{dest}|}| is achieved by:
%
\begin{center}
|\def\jobname{|\textit{dest}|}\input{\jobname}|
\end{center}
%
The redirection with prefix
|\childdocforwardprefix[|\textit{prefix}|]{|\textit{dest}|}|
is accomplished by:
%
\begin{center}
\begin{tabular}{l}
|{\edef\jobname{\scantokens\expandafter{\jobname\noexpand}}|\\
|\def\redirectjob |\textit{prefix}|#1~~~{\gdef\jobname{|\textit{dest}|#1}}|\\
|\expandafter\redirectjob\jobname~~~}\input{\jobname}|
\end{tabular}
\end{center}

In an alternative approach,
child documents can be compiled by a specific command line
without additional code or specific definitions:
%
\begin{center}
|... -jobname "|\textit{target}|" "|[\textit{flags}]%
|\includeonly{|\textit{dest}|}\input{|\textit{main}|}"|
\end{center}
%

%%%%%%%%%%%%%%%%%%%%%%%%%%%%%%%%%%%%%%%%%%%%%%%%%%%%%%%%%%%%%%%%%%%%%%%%%%%%%%%%
%%%%%%%%%%%%%%%%%%%%%%%%%%%%%%%%%%%%%%%%%%%%%%%%%%%%%%%%%%%%%%%%%%%%%%%%%%%%%%%%
\section{Information}

%%%%%%%%%%%%%%%%%%%%%%%%%%%%%%%%%%%%%%%%%%%%%%%%%%%%%%%%%%%%%%%%%%%%%%%%%%%%%%%%
\subsection{Copyright}

Copyright \copyright{} 2017--2018 Niklas Beisert

This work may be distributed and/or modified under the
conditions of the \LaTeX{} Project Public License, either version 1.3
of this license or (at your option) any later version.
The latest version of this license is in
  \url{http://www.latex-project.org/lppl.txt}
and version 1.3 or later is part of all distributions of \LaTeX{}
version 2005/12/01 or later.

This work has the LPPL maintenance status `maintained'.

The Current Maintainer of this work is Niklas Beisert.

This work consists of the files |README.txt|, |childdoc.ins| and |childdoc.dtx|
as well as the derived files |childdoc.def|, |cdocsamp.tex|
with |cdocsch1.tex|, |cdocsch2.tex|, |cdocspt3.tex|, |cdocspt4.tex|,
|cdocsdrf.tex|, |cdocsfn1.tex|, |cdocsfn2.tex|
as well as |childdoc.pdf|.

%%%%%%%%%%%%%%%%%%%%%%%%%%%%%%%%%%%%%%%%%%%%%%%%%%%%%%%%%%%%%%%%%%%%%%%%%%%%%%%%
\subsection{Files and Installation}

The package consists of the files:
%
\begin{center}
\begin{tabular}{ll}
    |README.txt|   & readme file \\
    |childdoc.ins| & installation file \\
    |childdoc.dtx| & source file \\
    |childdoc.def| & definition file \\
    |cdocsamp.tex| & sample main file \\
    |cdocsch1.tex| & sample include file \\
    |cdocsch2.tex| & sample include file \\
    |cdocspt3.tex| & sample part file \\
    |cdocspt4.tex| & sample part file \\
    |cdocsdrf.tex| & sample redirection file \\
    |cdocsfn1.tex| & sample redirection file \\
    |cdocsfn2.tex| & sample redirection file \\
    |childdoc.pdf| & manual
\end{tabular}
\end{center}
%
The distribution consists of the files
|README.txt|, |childdoc.ins| and |childdoc.dtx|.
%
\begin{itemize}
\item
Run (pdf)\LaTeX{} on |childdoc.dtx|
to compile the manual |childdoc.pdf| (this file).
\item
Run \LaTeX{} on |childdoc.ins| to create the definitions file |childdoc.def|
and the sample |cdocsamp.tex| with include files
|cdocsch1.tex|, |cdocsch2.tex|, |cdocspt3.tex|, |cdocspt4.tex|,
|cdocsdrf.tex|, |cdocsfn1.tex|, |cdocsfn2.tex|.
Then copy the file |childdoc.def| to an appropriate directory of your \LaTeX{}
distribution, e.g.\ \textit{texmf-root}|/tex/latex/childdoc|.
\end{itemize}

%%%%%%%%%%%%%%%%%%%%%%%%%%%%%%%%%%%%%%%%%%%%%%%%%%%%%%%%%%%%%%%%%%%%%%%%%%%%%%%%
\subsection{Related CTAN Packages}

There are several other packages which offer a similar functionality:
%
\begin{itemize}
\item
The packages
\href{http://ctan.org/pkg/docmute}{\textsf{docmute}},
\href{http://ctan.org/pkg/includex}{\textsf{includex}} and
\href{http://ctan.org/pkg/standalone}{\textsf{standalone}}
provide commands to include only the document body of
a child file thus allowing both files to be compiled individually.
\item
The packages \href{http://ctan.org/pkg/subdocs}{\textsf{subdocs}}
and \href{http://ctan.org/pkg/subfiles}{\textsf{subfiles}}
provide structures in which the main and child documents can be
encapsulated and allowing them to be compiled individually.
The inclusion mechanism is different from the conventional |\include|.
\item
The package \href{http://ctan.org/pkg/combine}{\textsf{combine}}
is an elaborate solution to combine several documents into one.
\end{itemize}
%
See also the CTAN topic \href{http://ctan.org/topic/subdocs}{\textsf{subdocs}}
for further related packages.
The present package differs from the above solutions in that
a document structure constructed with the conventional |\include| mechanism
just needs two extra commands at the top of every file
such that all constituent files can be compiled individually.

%%%%%%%%%%%%%%%%%%%%%%%%%%%%%%%%%%%%%%%%%%%%%%%%%%%%%%%%%%%%%%%%%%%%%%%%%%%%%%%%
%\subsection{Feature Suggestions}
%
%The following is a list of features which may be useful for future
%versions of this package:
%%
%\begin{itemize}
%\item
%\ldots
%\end{itemize}

%%%%%%%%%%%%%%%%%%%%%%%%%%%%%%%%%%%%%%%%%%%%%%%%%%%%%%%%%%%%%%%%%%%%%%%%%%%%%%%%
\subsection{Revision History}

%%%%%%%%%%%%%%%%%%%%%%%%%%%%%%%%%%%%%%%%
\paragraph{v2.0:} 2018/12/30

\begin{itemize}
\item
immediate forward processing
\item
added |\childdocby| mechanism
\item
manual restructured
\end{itemize}

%%%%%%%%%%%%%%%%%%%%%%%%%%%%%%%%%%%%%%%%
\paragraph{v1.6:} 2018/01/17

\begin{itemize}
\item
application for development of include files
\item
corrections to manual
\end{itemize}

%%%%%%%%%%%%%%%%%%%%%%%%%%%%%%%%%%%%%%%%
\paragraph{v1.5:} 2017/05/21

\begin{itemize}
\item
more complete structuring introduced
\item
|\childdocof| introduced
\item
|\childdoc| renamed to |\childdocmain|
\item
|\childredirect| renamed to |\childdocforward| and |\childdocforwardprefix|
and functionality expanded
\end{itemize}

%%%%%%%%%%%%%%%%%%%%%%%%%%%%%%%%%%%%%%%%
\paragraph{v1.0:} 2017/04/27

\begin{itemize}
\item
manual and install package
\item
first version published on CTAN
\end{itemize}

%%%%%%%%%%%%%%%%%%%%%%%%%%%%%%%%%%%%%%%%
\paragraph{v0.6:} 2017/04/26

\begin{itemize}
\item
redirection mechanism added
\end{itemize}

%%%%%%%%%%%%%%%%%%%%%%%%%%%%%%%%%%%%%%%%
\paragraph{v0.5:} 2017/04/26

\begin{itemize}
\item
functionality in definition file
\end{itemize}


%%%%%%%%%%%%%%%%%%%%%%%%%%%%%%%%%%%%%%%%%%%%%%%%%%%%%%%%%%%%%%%%%%%%%%%%%%%%%%%%
%%%%%%%%%%%%%%%%%%%%%%%%%%%%%%%%%%%%%%%%%%%%%%%%%%%%%%%%%%%%%%%%%%%%%%%%%%%%%%%%
%%%%%%%%%%%%%%%%%%%%%%%%%%%%%%%%%%%%%%%%%%%%%%%%%%%%%%%%%%%%%%%%%%%%%%%%%%%%%%%%
\appendix

\settowidth\MacroIndent{\rmfamily\scriptsize 000\ }

 \DocInput{childdoc.dtx}

\end{document}
%</driver>
% \fi
%
% %%%%%%%%%%%%%%%%%%%%%%%%%%%%%%%%%%%%%%%%%%%%%%%%%%%%%%%%%%%%%%%%%%%%%%%%%%%%%%
% %%%%%%%%%%%%%%%%%%%%%%%%%%%%%%%%%%%%%%%%%%%%%%%%%%%%%%%%%%%%%%%%%%%%%%%%%%%%%%
% \section{Sample}
%\iffalse
%<*samplemain>
%\fi
%
% The following presents a sample document
% with two chapters, two parts, a title page,
% a compile flag as well as three forwarding files to set the flag.
% It consists of eight |.tex| files:
% \begin{center}
% \begin{tabular}{ll}
% |cdocsamp.tex|&main file\\
% |cdocsch1.tex|&include file for chapter 1\\
% |cdocsch2.tex|&include file for chapter 2\\
% |cdocspt3.tex|&include file for part 3\\
% |cdocspt4.tex|&include file for part 4\\
% |cdocsdrf.tex|&forwarding file for main file in draft mode\\
% |cdocsfi1.tex|&forwarding file for final version of chapter 1\\
% |cdocsfi2.tex|&forwarding file for final version of chapter 2\\
% \end{tabular}
% \end{center}
% Each of the eight files can be compiled directly by the \LaTeX{} compiler.
%
% %%%%%%%%%%%%%%%%%%%%%%%%%%%%%%%%%%%%%%
% \paragraph{Main File.}
%
% The main file is called |cdocsamp.tex|.
%
% Load the \textsf{childdoc} definitions and
% declare the filename for the main document:
%    \begin{macrocode}
\input{childdoc.def}
\childdocmain{}
%    \end{macrocode}

% Optional override for |\version| flag:
%    \begin{macrocode}
%%\ifchilddoc\else\providecommand{\version}{draft}\fi
%    \end{macrocode}

% Define the default values for the |\version| flag
% (|final| for the main file and |draft| for childs):
%    \begin{macrocode}
\ifchilddoc
\providecommand{\version}{draft}
\else
\providecommand{\version}{final}
\fi
%    \end{macrocode}

% Load the standard document class:
%    \begin{macrocode}
\documentclass[12pt]{article}
%    \end{macrocode}

% Start the document body:
%    \begin{macrocode}
\begin{document}
%    \end{macrocode}

% Declare a title page.
% Print title, part of document being processed and version flag:
%    \begin{macrocode}
\addtocounter{page}{-1}
\begin{center}
{\LARGE\bfseries{}childdoc example\par}
\vspace{1cm}
\ifchilddoc
\ifchilddocmanual part\else chapter\fi:
`\childdocname' of `\childdocjob'\par
\else
main document: `\childdocjob'\par
\fi
version: \version\par
\end{center}
\newpage
%    \end{macrocode}

% Manually include selected file,
% otherwise process as usual:
%    \begin{macrocode}
\ifchilddocmanual
\section*{part `\childdocname'}
\input{\childdocname}
\else
%    \end{macrocode}

% Include the two chapters:
%    \begin{macrocode}
\include{cdocsch1}
\include{cdocsch2}
%    \end{macrocode}

% Include the two parts unless only chapters should be displayed:
%    \begin{macrocode}
\ifchilddoc\else
\section{part three}
\input{cdocspt3}
\section{part four}
\input{cdocspt4}
\fi
%    \end{macrocode}

% Process as usual until here:
%    \begin{macrocode}
\fi
%    \end{macrocode}

% End of document body:
%    \begin{macrocode}
\end{document}
%    \end{macrocode}
%\iffalse
%</samplemain>
%\fi
%
% %%%%%%%%%%%%%%%%%%%%%%%%%%%%%%%%%%%%%%
% \paragraph{Chapter Include Files.}
%
% The include files are called |cdocsch1.tex| and |cdocsch2.tex|.
%
%\iffalse
%<*samplechap1|samplechap2>
%\fi

% Optional override for |\version| flag:
%    \begin{macrocode}
%%\providecommand{\version}{final}
%    \end{macrocode}

% Include the main document:
%    \begin{macrocode}
\input{childdoc.def}
\childdocof{cdocsamp}
%    \end{macrocode}

%\iffalse
%</samplechap1|samplechap2>
%\fi
%
%\iffalse
%<*samplechap1>
%\fi
% Some text for chapter 1:
%    \begin{macrocode}
\section{one}
some text in chapter one
%    \end{macrocode}

%\iffalse
%</samplechap1>
%\fi
% Some text for chapter 2:
%\iffalse
%<*samplechap2>
%\fi
%    \begin{macrocode}
\section{two}
more text in chapter two
%    \end{macrocode}

%\iffalse
%</samplechap2>
%\fi
%
% %%%%%%%%%%%%%%%%%%%%%%%%%%%%%%%%%%%%%%
% \paragraph{Part Include Files.}
%
% The include files are called |cdocspt3.tex| and |cdocspt4.tex|.
%
%\iffalse
%<*samplepart3|samplepart4>
%\fi

% Optional override for |\version| flag:
%    \begin{macrocode}
%%\providecommand{\version}{final}
%    \end{macrocode}

% Include the main document:
%    \begin{macrocode}
\input{childdoc.def}
\childdocby{cdocsamp}
%    \end{macrocode}

%\iffalse
%</samplepart3|samplepart4>
%\fi
%
%\iffalse
%<*samplepart3>
%\fi
% Some text for part 3:
%    \begin{macrocode}
some text in part three
%    \end{macrocode}

%\iffalse
%</samplepart3>
%\fi
% Some text for part 4:
%\iffalse
%<*samplepart4>
%\fi
%    \begin{macrocode}
more text in part four
%    \end{macrocode}

%\iffalse
%</samplepart4>
%\fi
%
% %%%%%%%%%%%%%%%%%%%%%%%%%%%%%%%%%%%%%%
% \paragraph{Forwarding for a Complete Draft.}
%
% The following forwarding file |cdocsdrf.tex|
% compiles the main document in draft mode:
%\iffalse
%<*sampledraft>
%\fi
%    \begin{macrocode}
\def\version{draft}
\input{childdoc.def}
\childdocforward{cdocsamp}
%    \end{macrocode}

%\iffalse
%</sampledraft>
%\fi
%
% %%%%%%%%%%%%%%%%%%%%%%%%%%%%%%%%%%%%%%
% \paragraph{Forwarding for Final Version of the Chapters.}
%
% The following forwarding files |cdocsfn1.tex| and |cdocsfn2.tex|
% (with identical content)
% compile the final versions of the child documents
% |cdocsch1.tex| and |cdocsch2.tex|, respectively:
%\iffalse
%<*samplefinal>
%\fi
%    \begin{macrocode}
\def\version{final}
\input{childdoc.def}
\childdocforwardprefix[cdocsamp]{cdocsfn}{cdocsch}
%    \end{macrocode}

%\iffalse
%</samplefinal>
%\fi
%
% %%%%%%%%%%%%%%%%%%%%%%%%%%%%%%%%%%%%%%
% \paragraph{Command Line Processing.}
%
% The following three command lines generate the output files
% |cdocscld|, |cdocscl1| and |cdocscl2|
% which should be identical to
% |cdocsdrf|, |cdocsch1| and |cdocsfn2|, respectively:
% \begin{center}
% \begin{tabular}{l}
% |latex -jobname cdocscld \|\\
% |  "\def\version{draft}\input{childdoc.def}\childdocforward{cdocsamp}"|\\
% |latex -jobname cdocscl1 \|\\
% |  "\input{childdoc.def}\childdocforward[cdocsamp]{cdocsch1}"|\\
% |latex -jobname cdocscl2 \|\\
% |  "\def\version{final}\input{childdoc.def}\childdocforward{cdocsch2}"|
% \end{tabular}
% \end{center}
% Note that the trailing backslash on each first line
% merely continues the input to the second line
% (for convenient cut ant paste).
% Furthermore, the command |latex| can be replaced by any
% of its alternative versions such as |pdflatex|.
%
% %%%%%%%%%%%%%%%%%%%%%%%%%%%%%%%%%%%%%%%%%%%%%%%%%%%%%%%%%%%%%%%%%%%%%%%%%%%%%%
% %%%%%%%%%%%%%%%%%%%%%%%%%%%%%%%%%%%%%%%%%%%%%%%%%%%%%%%%%%%%%%%%%%%%%%%%%%%%%%
% \section{Implementation}
%\iffalse
%<*package>
%\fi
%
% This section describes the definitions file |childdoc.def|.

% The definitions cannot be loaded using |\usepackage| or |\RequirePackage|
% which has a mechanism to prevent loading a style file more than once.
% When loading the definitions by means of |\input|
% multiple instances have to be prevented manually:
%\iffalse
%This code needs to be before the `\ProvidesFile' directive
%which is defined at the beginning of this file.
%Therefore it is also placed there and commented out here.
%</package>
%<*discard>
%\fi
%    \begin{macrocode}
\ifdefined\childdocmain\endinput\fi
%    \end{macrocode}
%\iffalse
%</discard>
%<*package>
%\fi
%
% \macro{\ifchilddoc}
% \macro{\ifchilddocmanual}
% The conditional |\ifchilddoc| tells whether a
% child (true) or main (false) document is being compiled.
% The conditional |\ifchilddocmanual| tells whether
% the |\includeonly| mechanism is used (false) or
% the selection of child files must be performed manually (true).
% The definitions initialise to false:
%    \begin{macrocode}
\newif\ifchilddoc
\newif\ifchilddocmanual
%    \end{macrocode}

% \macro{\childdocname}
% \macro{\childdocjob}
% The macro |\childdocname| stores the name of the main document
% to be compiled. The macro |\childdocjob| stores the name of
% the document on which the \LaTeX{} compiler was originally invoked.
% The content of |\jobname| cannot be compared
% to filenames specified in the source due to different catcodes.
% The following code rescans |\jobname|, stores the result
% in |\childdocname| and saves a copy in |\childdocjob|:
%    \begin{macrocode}
\edef\childdocname{\scantokens\expandafter{\jobname\noexpand}}
\let\childdocjob\childdocname
%    \end{macrocode}

% \macro{\childdocdisable}
% The macro |\childdocdisable| prevents the main file
% from being processed more than once.
% At this stage, the main document command |\childdocmain|
% is assumed to be called once again where it should do nothing.
% Any subsequent call to it should prevent
% a secondary processing of the main document
% It overwrites the forwarding commands
% |\childdocof| and |\childdocforward|
% with empty macros to prevent further inclusions of the main document:
%    \begin{macrocode}
\newcommand{\childdocdisable}
{
  \renewcommand{\childdocmain}[1]{\renewcommand{\childdocmain}[1]{\endinput}}
  \renewcommand{\childdocof}[1]{}
  \renewcommand{\childdocby}[2][]{}
  \renewcommand{\childdocforward}[2][]{}
  \renewcommand{\childdocdisable}{}
}
%    \end{macrocode}

% \macro{\childdocmain}
% The macro |\childdocmain| is to be called at the top of the main file
% with nothing or the main filename (without extension) as argument.
% First, it breaks loops.
% If the argument is not empty and does not match |\childdocname|
% (which is set by the first inclusion of |childdoc.def|),
% |\ifchilddoc| is set to true, |\includeonly| is applied to the child file
% and |\jobname| is set to the main file
% (for proper handling of |.aux| files):
%    \begin{macrocode}
\newcommand{\childdocmain}[1]
{
  \childdocdisable\childdocmain{}
  \if?#1?\else
    \begingroup
      \def\childdoctmp{#1}
      \ifx\childdoctmp\childdocname
        \def\childdoctmp{}
      \else
        \def\childdoctmp
        {
          \childdoctrue
          \includeonly{\childdocname}
          \def\childdocjob{#1}
          \def\jobname{#1}
        }
      \fi
      \expandafter
    \endgroup
    \childdoctmp
  \fi
}
%    \end{macrocode}

% \macro{\childdocof}
% The command |\childdocof| redirects
% compilation to the main file |#1|.
%    \begin{macrocode}
\newcommand{\childdocof}[1]
{
  \childdocdisable
  \childdoctrue
  \includeonly{\childdocname}
  \def\jobname{#1}
  \def\childdocjob{#1}
  \input{#1}
}
%    \end{macrocode}

% \macro{\childdocby}
% The command |\childdocby| ....
%    \begin{macrocode}
\newcommand{\childdocby}[2][]
{
  \childdocdisable
  \childdoctrue
  \childdocmanualtrue
  \if?#1?\else
    \def\jobname{#2}
  \fi
  \def\childdocjob{#2}
  \input{#2}
  \endinput
}
%    \end{macrocode}

% \macro{\childdocforward}
% The command |\childdocforward| redirects
% compilation to the main file or
% (if the optional argument is given) a child file.
% Parameters are set as if the main file
% or a child file starting with |\childdocof| was compiled.
% Then compilation is handed over to the main file:
%    \begin{macrocode}
\newcommand{\childdocforward}[2][]
{
  \begingroup
    \if?#1?
      \def\childdoctmp
      {
        \def\childdocname{#2}
        \def\childdocjob{#2}
        \def\jobname{#2}
        \input{#2}
        \endinput
      }
    \else
      \def\childdoctmp
      {
        \childdocdisable
        \def\childdocname{#2}
        \childdoctrue
        \includeonly{#2}
        \def\childdocjob{#1}
        \def\jobname{#1}
        \input{#1}
        \endinput
      }
    \fi
    \expandafter
  \endgroup
  \childdoctmp
}
%    \end{macrocode}

% \macro{\childdocforwardprefix}
% The command |\childdocforwardprefix| redirects
% compilation to the main or a child file by means of a pattern.
% The prefix |#1| in the current filename is replaced by |#2|
% and the suffix of the current filename is kept
% (it is assumed that the filename does not contain the substring `|~~~|'
% which is used as a delimiter).
% Compilation is handed over to the new file by |\childdocforward|:
%    \begin{macrocode}
\newcommand{\childdocforwardprefix}[3][]
{
  \begingroup
    \def\childdocextract #2##1~~~{\def\childdoctmp{\childdocforward[#1]{#3##1}}}
    \expandafter\childdocextract\childdocname~~~
    \expandafter
  \endgroup
  \childdoctmp
}
%    \end{macrocode}

% \macro{\childdoc}
% The deprecated macro |\childdoc| is a legacy version of |\childdocmain|:
%    \begin{macrocode}
\newcommand{\childdoc}{\childdocmain}
%    \end{macrocode}

% \macro{\childdocredirect}
% The deprecated macro |\childdocredirect| is a legacy version
% of |\childdocforward| and |\childdocforwardprefix|:
%    \begin{macrocode}
\newcommand{\childdocredirect}[2][]
{
  \begingroup
    \if?#1?
      \def\childdoctmp{\childdocforward{#2}}
    \else
      \def\childdoctmp{\childdocforwardprefix{#1}{#2}}
    \fi
    \expandafter
  \endgroup
  \childdoctmp
}
%    \end{macrocode}

%\iffalse
%</package>
%\fi
%
\endinput
|\\
|\childdocof{|\textit{main}|}|\\
\end{tabular}
\end{center}
at the top of every child file \textit{child}
which is included by |\include{|\textit{child}|}|
from within the main file
(or at least for those files to be compiled individually).
The argument \textit{main} must be the filename of the main file.

There are a couple of
considerations in setting up the main and child documents:

%%%%%%%%%%%%%%%%%%%%%%%%%%%%%%%%%%%%%%%%
\paragraph{Restrictions.}

Please note the following restrictions:
\begin{itemize}
\item
|\childdocmain| must be called with one argument \textit{main}
to ensure compatibility with earlier version of the package.
It must either be empty (|\childdocmain{}|)
or precisely match the filename of the main file in which it is specified.
See \secref{sec:detection} for further information.
\item
The filename \textit{main} must be specified without the |.tex| extension.
\item
The filename \textit{main} is case sensitive
(even in case-insensitive file systems)
due to internal string comparison.
\item
The argument \textit{main} should be fully expanded, it cannot be a macro.
\item
Subdirectories and special characters should be avoided in filenames.
\item
The command |\childdocmain{|\textit{main}|}| must be followed by a whitespace.
It should not be followed immediately by another command
or by a comment mark `|%|'.
This is because the \TeX{} parser reads the token immediately following
the argument of |\childdocmain| and puts it
at the beginning of every child section;
however, a white\-space is ignored.
\end{itemize}

%%%%%%%%%%%%%%%%%%%%%%%%%%%%%%%%%%%%%%%%
\paragraph{Content of Main File.}

It is advisable to place all content in the child files included by |\include|.
Any output contained in the main file will appear in all child documents
unless suppressed manually;
it cannot be suppressed automatically by the |\includeonly| directive
and thus should normally be avoided.
A method to include some content in the main file
by means of conditional processing is described in \secref{sec:conditional}.

%%%%%%%%%%%%%%%%%%%%%%%%%%%%%%%%%%%%%%%%
\paragraph{Page Numbering.}

When only a part of the document is compiled,
the appropriate numbering of pages
(as well as other status parameters)
is determined from the |.aux| files.
The latter contain information from previous passes.
However this information needs to propagate through
all intermediate child documents.
Therefore the page numbering in child documents may well
be inconsistent until the complete document is compiled at least once.

A useful (if unconventional) way to always ensure a consistent
page numbering is to restart the numbering in each child document
and denote the pages by `\textit{child}|.|\textit{page}'
where \textit{child} represents the chapter/section number of the child file.
This can be achieved by the command
|\numberwithin{page}{|\textit{child}|}|
of the \textsf{amsmath} package
where \textit{child} can be |chapter| or |section|
depending on the chosen structuring.
Alternatively, one can modify the macro |\thepage| appropriately
and reset the counter |page| at the start of each child file.

%%%%%%%%%%%%%%%%%%%%%%%%%%%%%%%%%%%%%%%%%%%%%%%%%%%%%%%%%%%%%%%%%%%%%%%%%%%%%%%%
\subsection{Conditional Processing}
\label{sec:conditional}

The package provides a mechanism to compile different versions
of a document. To customise the versions further some conditional processing
can come in handy to distinguish which version is being compiled.
The package provides two macros to describe the compilation context:

%%%%%%%%%%%%%%%%%%%%%%%%%%%%%%%%%%%%%%%%
\DescribeMacro{\ifchilddoc}
The conditional |\ifchilddoc| distinguishes between the compilation of
child documents and the main document:
%
\begin{center}
|\ifchilddoc |\textit{child-code}| |[|\||else |\textit{main-code}]| \||fi|
\end{center}

%%%%%%%%%%%%%%%%%%%%%%%%%%%%%%%%%%%%%%%%
\DescribeMacro{\childdocname}
\DescribeMacro{\childdocjob}
The macro |\childdocname| contains the filename (without extension)
of the main or child file being processed.
Note that |\childdocjob| will always contain the name of the main file.

%%%%%%%%%%%%%%%%%%%%%%%%%%%%%%%%%%%%%%%%
\paragraph{Title Page.}

Conditional processing can be used to include a title or banner page
in the main document when proper precautions are taken.
Importantly, the code in the main file should ensure that the page counter
(as well as other status parameters which are stored in the |.aux| files)
takes the same value after the conditional processing.
Otherwise the page numbers may take divergent values
depending on which part is compiled.

For example, a title page could be declared by:
%
\begin{center}
\begin{tabular}{l}
|\ifchilddoc\||else|\\
|\addtocounter{page}{-1}|\\
\textit{code for title page}\\
|\newpage|\\
|\||fi|
\end{tabular}
\end{center}
%
A banner page for the child documents can be generated by:
%
\begin{center}
\begin{tabular}{l}
|\ifchilddoc|\\
|\addtocounter{page}{-1}|\\
\textit{code for banner page}\\
|\newpage|\\
|\||fi|
\end{tabular}
\end{center}
%
Here one could write a message such as:
\begin{center}
|This is the part \childdocname{} of \childdocjob{}.|
\end{center}

%%%%%%%%%%%%%%%%%%%%%%%%%%%%%%%%%%%%%%%%%%%%%%%%%%%%%%%%%%%%%%%%%%%%%%%%%%%%%%%%
\subsection{Flags}
\label{sec:flags}

The package makes it easy to generate different versions
of the main or child documents.
To this end compilation flags can be defined
and assigned different default values.
They will be particularly useful in conjunction
with the forwarding mechanism described in \secref{sec:forward}.

For example, it may be useful to have a flag |\version|
which can be set to |draft| or |final|.
The document source will contain some conditional code
depending on the value of |\version|.
Suppose further, the flag should default to |final| for the main file
and to |draft| for child files
which is a natural assignment for editing the document.
This is achieved by placing the following code
in the preamble of the main document
(below the |\childdocmain| directive):
%
\begin{center}
\begin{tabular}{l}
|\ifchilddoc|\\
|\providecommand{\version}{draft}|\\
|\||else|\\
|\providecommand{\version}{final}|\\
|\||fi|
\end{tabular}
\end{center}
%
The definition by |\providecommand| makes sure
that previous definitions are not overwritten.
Further statements |\providecommand{\version}{...}|
can thus be added before the above code to override it.

For the main file, one might add a line
(between |\childdocmain| and the above block)
%
\begin{center}
|%\ifchilddoc\||else\providecommand{\version}{draft}\||fi|
\end{center}
%
which can be uncommented to produce a draft version.
Likewise one can add a line to the very top of a child file
(above the |\childdocof{|\textit{main}|}| directive)
%
\begin{center}
|%\providecommand{\version}{final}|
\end{center}
%
which can be uncommented to produce the final version of this child document.

%%%%%%%%%%%%%%%%%%%%%%%%%%%%%%%%%%%%%%%%%%%%%%%%%%%%%%%%%%%%%%%%%%%%%%%%%%%%%%%%
\subsection{Forwarding}
\label{sec:forward}

Different versions of the main or child documents
using compilation flags as described in \secref{sec:flags}
can be (permanently) stored in different files
for convenient compilation, viewing and distribution.
To this end, the package defines a command
to pass on compilation to a different file:

%%%%%%%%%%%%%%%%%%%%%%%%%%%%%%%%%%%%%%%%
\DescribeMacro{\childdocforward}
The command |\childdocforward| redirects processing to
another source file:
%
\begin{center}
\begin{tabular}{l}
|% \iffalse
%
% childdoc.dtx Copyright (C) 2017-2018 Niklas Beisert
%
% This work may be distributed and/or modified under the
% conditions of the LaTeX Project Public License, either version 1.3
% of this license or (at your option) any later version.
% The latest version of this license is in
%   http://www.latex-project.org/lppl.txt
% and version 1.3 or later is part of all distributions of LaTeX
% version 2005/12/01 or later.
%
% This work has the LPPL maintenance status `maintained'.
%
% The Current Maintainer of this work is Niklas Beisert.
%
% This work consists of the files childdoc.dtx and childdoc.ins
% and the derived files childdoc.def and cdocsamp.tex with
% cdocsch1.tex, cdocsch2.tex, cdocsdrf.tex, cdocsfn1.tex, cdocsfn2.tex.
%
%<package>\ifdefined\childdocmain\endinput\fi
%<package>\ProvidesFile{childdoc.def}[2018/12/30 v2.0 child document driver]
%<samplemain>\ProvidesFile{cdocsamp.tex}[2018/12/30 v2.0 sample for childdoc]
%<*driver>
%\ProvidesFile{childdoc.drv}[2018/12/30 v2.0 childdoc reference manual file]
\PassOptionsToClass{10pt,a4paper}{article}
\documentclass{ltxdoc}

\usepackage[margin=35mm]{geometry}
\usepackage{hyperref}
\usepackage{hyperxmp}
\usepackage[usenames]{color}

\hypersetup{colorlinks=true}
\hypersetup{pdfstartview=FitH}
\hypersetup{pdfpagemode=UseNone}
\hypersetup{pdfsource={}}
\hypersetup{pdflang={en-UK}}
\hypersetup{pdfcopyright={Copyright 2017-2018 Niklas Beisert.
  This work may be distributed and/or modified under the
  conditions of the LaTeX Project Public License, either version 1.3
  of this license or (at your option) any later version.}}
\hypersetup{pdflicenseurl={http://www.latex-project.org/lppl.txt}}
\hypersetup{pdfcontactaddress={ETH Zurich, ITP, HIT K,
  Wolfgang-Pauli-Strasse 27}}
\hypersetup{pdfcontactpostcode={8093}}
\hypersetup{pdfcontactcity={Zurich}}
\hypersetup{pdfcontactcountry={Switzerland}}
\hypersetup{pdfcontactemail={nbeisert@itp.phys.ethz.ch}}
\hypersetup{pdfcontacturl={http://people.phys.ethz.ch/\xmptilde nbeisert/}}

\newcommand{\secref}[1]{\hyperref[#1]{section \ref*{#1}}}

\parskip1ex
\parindent0pt
\let\olditemize\itemize
\def\itemize{\olditemize\parskip0pt}

\begin{document}

\title{The \textsf{childdoc} Package}
\hypersetup{pdftitle={The childdoc Package}}
\author{Niklas Beisert\\[2ex]
  Institut f\"ur Theoretische Physik\\
  Eidgen\"ossische Technische Hochschule Z\"urich\\
  Wolfgang-Pauli-Strasse 27, 8093 Z\"urich, Switzerland\\[1ex]
  \href{mailto:nbeisert@itp.phys.ethz.ch}
  {\texttt{nbeisert@itp.phys.ethz.ch}}}
\hypersetup{pdfauthor={Niklas Beisert}}
\hypersetup{pdfsubject={Manual for the LaTeX2e Package childdoc}}
\date{30 December 2018, \textsf{v2.0}}
\maketitle

\begin{abstract}\noindent
\textsf{childdoc} is a \LaTeXe{} package
that enables the direct compilation
of document sections included by |\include|
to individual files.
\end{abstract}

\begingroup
\parskip0ex
\tableofcontents
\endgroup

%%%%%%%%%%%%%%%%%%%%%%%%%%%%%%%%%%%%%%%%%%%%%%%%%%%%%%%%%%%%%%%%%%%%%%%%%%%%%%%%
%%%%%%%%%%%%%%%%%%%%%%%%%%%%%%%%%%%%%%%%%%%%%%%%%%%%%%%%%%%%%%%%%%%%%%%%%%%%%%%%
\section{Introduction}

\LaTeX{} provides a mechanism to structure a large document (such as a book)
into a main file and several child files (containing the chapters)
using the |\include| command.
This mechanism is beneficial for documents
which span hundreds of pages in order to
make the source file(s) more manageable.
Moreover, compilation can be restricted to
selected child files by means of the |\includeonly| command.
The latter feature can be used to reduce the compilation time while editing
(this was significantly more useful in the earlier days of \LaTeX{})
or to generate a smaller document which is easier to navigate.
Another application of |\includeonly| is to generate
documents consisting of selected parts of the complete document.

However, there are a few drawbacks of the plain |\include| mechanism:
\begin{itemize}
\item
The child files cannot be compiled on their own,
they can only be compiled via the main file.
A naive editing environment
(such as a text editor with an option
to have the current file processed by \LaTeX)
may require one to switch to the main file before compiling;
attempting to compile the child file produces errors.
\item
The main file must be modified (each time)
to adjust the |\includeonly| command
to the present needs. This easily leaves the main file in a messy state.
\item
The generated document will always carry the filename
of the main document. This is inconvenient if
several child files are to be compiled and
to be kept for distribution.
\end{itemize}

The present package provides a simple interface
to make child files individually compilable by \LaTeX{}.
Compiling a child file then has the same effect as compiling
the main file with an |\includeonly| command
to select the appropriate child.
Moreover the generated document will carry the name of the child
rather than the main file.
This resolves all three above issues.

This feature is meant to make the editing of books,
thesis documents and lecture notes somewhat more convenient.
However, the package can also be used efficiently for
composing a series of documents (such as exercise sheets)
which are typically distributed individually.
It then assists the author in generating the individual documents
(potentially in different versions)
as well as a document containing the collected series.
Another application is in developing style files
or other kinds of included material
where compilation of the style file could redirect
to a sample or test file.

%%%%%%%%%%%%%%%%%%%%%%%%%%%%%%%%%%%%%%%%%%%%%%%%%%%%%%%%%%%%%%%%%%%%%%%%%%%%%%%%
%%%%%%%%%%%%%%%%%%%%%%%%%%%%%%%%%%%%%%%%%%%%%%%%%%%%%%%%%%%%%%%%%%%%%%%%%%%%%%%%
\section{Usage}

First of all, the package \textsf{childdoc} is \emph{not} a standard
\LaTeXe{} |.sty| style file! Therefore it needs to be invoked in
a non-standard way.

%%%%%%%%%%%%%%%%%%%%%%%%%%%%%%%%%%%%%%%%%%%%%%%%%%%%%%%%%%%%%%%%%%%%%%%%%%%%%%%%
\subsection{Included Files}
\label{sec:include}

%%%%%%%%%%%%%%%%%%%%%%%%%%%%%%%%%%%%%%%%
\DescribeMacro{\childdocmain}
To use the package, add the commands
\begin{center}
\begin{tabular}{l}
|\input{childdoc.def}|\\
|\childdocmain{}|\\
\end{tabular}
\end{center}
at the very top of the main \LaTeX{} file,
in particular \emph{before} the |\documentclass| statement!
The argument of |\childdocmain| should be left empty
(but it must be present).

%%%%%%%%%%%%%%%%%%%%%%%%%%%%%%%%%%%%%%%%
\DescribeMacro{\childdocof}
Furthermore, add the commands
\begin{center}
\begin{tabular}{l}
|\input{childdoc.def}|\\
|\childdocof{|\textit{main}|}|\\
\end{tabular}
\end{center}
at the top of every child file \textit{child}
which is included by |\include{|\textit{child}|}|
from within the main file
(or at least for those files to be compiled individually).
The argument \textit{main} must be the filename of the main file.

There are a couple of
considerations in setting up the main and child documents:

%%%%%%%%%%%%%%%%%%%%%%%%%%%%%%%%%%%%%%%%
\paragraph{Restrictions.}

Please note the following restrictions:
\begin{itemize}
\item
|\childdocmain| must be called with one argument \textit{main}
to ensure compatibility with earlier version of the package.
It must either be empty (|\childdocmain{}|)
or precisely match the filename of the main file in which it is specified.
See \secref{sec:detection} for further information.
\item
The filename \textit{main} must be specified without the |.tex| extension.
\item
The filename \textit{main} is case sensitive
(even in case-insensitive file systems)
due to internal string comparison.
\item
The argument \textit{main} should be fully expanded, it cannot be a macro.
\item
Subdirectories and special characters should be avoided in filenames.
\item
The command |\childdocmain{|\textit{main}|}| must be followed by a whitespace.
It should not be followed immediately by another command
or by a comment mark `|%|'.
This is because the \TeX{} parser reads the token immediately following
the argument of |\childdocmain| and puts it
at the beginning of every child section;
however, a white\-space is ignored.
\end{itemize}

%%%%%%%%%%%%%%%%%%%%%%%%%%%%%%%%%%%%%%%%
\paragraph{Content of Main File.}

It is advisable to place all content in the child files included by |\include|.
Any output contained in the main file will appear in all child documents
unless suppressed manually;
it cannot be suppressed automatically by the |\includeonly| directive
and thus should normally be avoided.
A method to include some content in the main file
by means of conditional processing is described in \secref{sec:conditional}.

%%%%%%%%%%%%%%%%%%%%%%%%%%%%%%%%%%%%%%%%
\paragraph{Page Numbering.}

When only a part of the document is compiled,
the appropriate numbering of pages
(as well as other status parameters)
is determined from the |.aux| files.
The latter contain information from previous passes.
However this information needs to propagate through
all intermediate child documents.
Therefore the page numbering in child documents may well
be inconsistent until the complete document is compiled at least once.

A useful (if unconventional) way to always ensure a consistent
page numbering is to restart the numbering in each child document
and denote the pages by `\textit{child}|.|\textit{page}'
where \textit{child} represents the chapter/section number of the child file.
This can be achieved by the command
|\numberwithin{page}{|\textit{child}|}|
of the \textsf{amsmath} package
where \textit{child} can be |chapter| or |section|
depending on the chosen structuring.
Alternatively, one can modify the macro |\thepage| appropriately
and reset the counter |page| at the start of each child file.

%%%%%%%%%%%%%%%%%%%%%%%%%%%%%%%%%%%%%%%%%%%%%%%%%%%%%%%%%%%%%%%%%%%%%%%%%%%%%%%%
\subsection{Conditional Processing}
\label{sec:conditional}

The package provides a mechanism to compile different versions
of a document. To customise the versions further some conditional processing
can come in handy to distinguish which version is being compiled.
The package provides two macros to describe the compilation context:

%%%%%%%%%%%%%%%%%%%%%%%%%%%%%%%%%%%%%%%%
\DescribeMacro{\ifchilddoc}
The conditional |\ifchilddoc| distinguishes between the compilation of
child documents and the main document:
%
\begin{center}
|\ifchilddoc |\textit{child-code}| |[|\||else |\textit{main-code}]| \||fi|
\end{center}

%%%%%%%%%%%%%%%%%%%%%%%%%%%%%%%%%%%%%%%%
\DescribeMacro{\childdocname}
\DescribeMacro{\childdocjob}
The macro |\childdocname| contains the filename (without extension)
of the main or child file being processed.
Note that |\childdocjob| will always contain the name of the main file.

%%%%%%%%%%%%%%%%%%%%%%%%%%%%%%%%%%%%%%%%
\paragraph{Title Page.}

Conditional processing can be used to include a title or banner page
in the main document when proper precautions are taken.
Importantly, the code in the main file should ensure that the page counter
(as well as other status parameters which are stored in the |.aux| files)
takes the same value after the conditional processing.
Otherwise the page numbers may take divergent values
depending on which part is compiled.

For example, a title page could be declared by:
%
\begin{center}
\begin{tabular}{l}
|\ifchilddoc\||else|\\
|\addtocounter{page}{-1}|\\
\textit{code for title page}\\
|\newpage|\\
|\||fi|
\end{tabular}
\end{center}
%
A banner page for the child documents can be generated by:
%
\begin{center}
\begin{tabular}{l}
|\ifchilddoc|\\
|\addtocounter{page}{-1}|\\
\textit{code for banner page}\\
|\newpage|\\
|\||fi|
\end{tabular}
\end{center}
%
Here one could write a message such as:
\begin{center}
|This is the part \childdocname{} of \childdocjob{}.|
\end{center}

%%%%%%%%%%%%%%%%%%%%%%%%%%%%%%%%%%%%%%%%%%%%%%%%%%%%%%%%%%%%%%%%%%%%%%%%%%%%%%%%
\subsection{Flags}
\label{sec:flags}

The package makes it easy to generate different versions
of the main or child documents.
To this end compilation flags can be defined
and assigned different default values.
They will be particularly useful in conjunction
with the forwarding mechanism described in \secref{sec:forward}.

For example, it may be useful to have a flag |\version|
which can be set to |draft| or |final|.
The document source will contain some conditional code
depending on the value of |\version|.
Suppose further, the flag should default to |final| for the main file
and to |draft| for child files
which is a natural assignment for editing the document.
This is achieved by placing the following code
in the preamble of the main document
(below the |\childdocmain| directive):
%
\begin{center}
\begin{tabular}{l}
|\ifchilddoc|\\
|\providecommand{\version}{draft}|\\
|\||else|\\
|\providecommand{\version}{final}|\\
|\||fi|
\end{tabular}
\end{center}
%
The definition by |\providecommand| makes sure
that previous definitions are not overwritten.
Further statements |\providecommand{\version}{...}|
can thus be added before the above code to override it.

For the main file, one might add a line
(between |\childdocmain| and the above block)
%
\begin{center}
|%\ifchilddoc\||else\providecommand{\version}{draft}\||fi|
\end{center}
%
which can be uncommented to produce a draft version.
Likewise one can add a line to the very top of a child file
(above the |\childdocof{|\textit{main}|}| directive)
%
\begin{center}
|%\providecommand{\version}{final}|
\end{center}
%
which can be uncommented to produce the final version of this child document.

%%%%%%%%%%%%%%%%%%%%%%%%%%%%%%%%%%%%%%%%%%%%%%%%%%%%%%%%%%%%%%%%%%%%%%%%%%%%%%%%
\subsection{Forwarding}
\label{sec:forward}

Different versions of the main or child documents
using compilation flags as described in \secref{sec:flags}
can be (permanently) stored in different files
for convenient compilation, viewing and distribution.
To this end, the package defines a command
to pass on compilation to a different file:

%%%%%%%%%%%%%%%%%%%%%%%%%%%%%%%%%%%%%%%%
\DescribeMacro{\childdocforward}
The command |\childdocforward| redirects processing to
another source file:
%
\begin{center}
\begin{tabular}{l}
|\input{childdoc.def}|\\
|\childdocforward[|\textit{main}|]{|\textit{dest}|}|\\
\end{tabular}
\end{center}
%
The argument \textit{dest} is the destination file
(without extension).
It should be the main file or one of the child files.
Note that further \textsf{childdoc} directives
such as |\childdocof| and |\childdocforward|
in the indicated file will be processed in this form.
The optional argument \textit{main}
passes on directly to the main file \textit{main}
while pretending to compile the child \textit{dest}.
This form behaves as if \textit{dest}
issues |\childdocof{|\textit{main}|}| right away,
and no further \textsf{childdoc} directives will be processed.

%%%%%%%%%%%%%%%%%%%%%%%%%%%%%%%%%%%%%%%%
\DescribeMacro{\...prefix}
In the alternative form |\childdocforwardprefix|,
%
\begin{center}
\begin{tabular}{l}
|\input{childdoc.def}|\\
|\childdocforwardprefix[|\textit{main}|]{|\textit{prefix}|}{|\textit{dest}|}|
\end{tabular}
\end{center}
%
the destination file is determined by a pattern
depending on the current file:
To make this work, the current file must be called
`{\textit{prefix}\hspace{0.2em}\textit{suffix}}'
with \textit{prefix} matching precisely the argument.
Processing is then passed on to the file
`{\textit{dest}\hspace{0.2em}\textit{suffix}}'.
Surely, the same effect is achieved by
directly specifying the
argument `{\textit{dest}\hspace{0.2em}\textit{suffix}}'
in the first form.
However, that requires to set up a different file
for each child. With the alternative form of the command
all these files can have exactly the same content
which simplifies setting them up and maintaining them.

For example, the following file |draft.tex|
with a compilation flag |\version| as described in \secref{sec:flags}
compiles the main document as a draft:
%
\begin{center}
\begin{tabular}{l}
|\def\version{draft}|\\
|\input{childdoc.def}|\\
|\childdocforward{|\textit{main}|}|
\end{tabular}
\end{center}
%
Likewise, the following files |final|\textit{nn}|.tex|
compile the final version of the child document
|child|\textit{nn}|.tex|:
%
\begin{center}
\begin{tabular}{l}
|\def\version{final}|\\
|\input{childdoc.def}|\\
|\childdocforwardprefix{final}{child}|
\end{tabular}
\end{center}
%

Note that when several versions of a main file and/or of each child file
are to be generated, it may be convenient to set up a |Makefile| or
shell script to automatise the process.

%%%%%%%%%%%%%%%%%%%%%%%%%%%%%%%%%%%%%%%%%%%%%%%%%%%%%%%%%%%%%%%%%%%%%%%%%%%%%%%%
\subsection{Command Line Processing}
\label{sec:commandline}

The effect of redirection files can also be achieved by invoking
the \LaTeX{} compiler with a more elaborate command line.
Most conveniently this should be done as part
of a shell script or a |Makefile|.

When using \textsf{childdoc} in the main file, the following
command lines effectively perform a redirection
(note that depending on the shell being used,
backslashes may have to be doubled: `|\|' $\to$ `|\\|'):
%
\begin{center}
|... -jobname "|\textit{target}|" |\\|"|[\textit{flags}]%
|\input{childdoc.def}\childdocforward[|\textit{main}|]{|\textit{dest}|}"|
\end{center}
%
Here \textit{target} is the name of the output file,
\textit{main} is the name of the main file
and \textit{dest} is the name of the main or child file to be processed
(all filenames without extensions).
The optional argument \textit{main} can be omitted
if \textit{main} matches \textit{dest}.
Optionally, compilation \textit{flags} can be defined via |\def| commands.
This command line makes the \TeX{} engine believe
it is compiling the file \textit{target}
whose content is specified as the latter parameter.
The provided code then forwards the processing to
\textit{main} or \textit{dest} as described in \secref{sec:forward}.

%%%%%%%%%%%%%%%%%%%%%%%%%%%%%%%%%%%%%%%%%%%%%%%%%%%%%%%%%%%%%%%%%%%%%%%%%%%%%%%%
\subsection{Include by Input}
\label{sec:input}

Including child documents by |\include| has some restrictions by design.
Most notably, the content of a child document always occupies
its own set of pages; pages cannot be shared between child documents.
Usually, this behaviour makes perfect sense
because each child document contain an essential part of the document.
However, in some situations it may be desirable to compose
a document from a collection of parts
without having mandatory page breaks between then.
For this case, the package
provides a mechanism to include parts
by |\input| which can also be processed individually.
However, by construction this mechanism
requires manual handling of the content to be output.

%%%%%%%%%%%%%%%%%%%%%%%%%%%%%%%%%%%%%%%%
\DescribeMacro{\ifchilddocmanual}
The main file should be prepared as usual, see \secref{sec:include}.
However, the document body must make a distinction
between processing of an individual part and of the main document, e.g.:
%
\begin{center}
\begin{tabular}{l}
|\ifchilddocmanual|\\
|\input{\childdocname}|\\
|\||else|\\
\textit{document body with }|\input{|\textit{part}|}|\\
|\||fi|
\end{tabular}
\end{center}
%
The conditional |\ifchilddocmanual| is true whenever
a part to be included by |\input| is being compiled,
and the name of the part is stored in |\childdocname|.

%%%%%%%%%%%%%%%%%%%%%%%%%%%%%%%%%%%%%%%%
\DescribeMacro{\childdocby}
Each part to be included by |\input| should start with:
%
\begin{center}
\begin{tabular}{l}
|\input{childdoc.def}|\\
|\childdocby{|\textit{main}|}|\\
\end{tabular}
\end{center}
%
The directive |\childdocby| is similar to |\childdocof|
described in \secref{sec:include},
but the subsequent selection of content must be done manually.
To that end, both |\ifchilddoc| and |\ifchilddocmanual|
will be true upon processing of a part,
and the name of the part is stored in |\childdocname|.
Note that |\jobname| will be set to the filename of the current part
so that each part receives an individual |.aux| file
that does not interfere with the |.aux| file(s) of the main document.
This behaviour can be altered by the alternative form
|\childdocby[*]{|\textit{main}|}| (with a non-empty optional argument)
which uses the |.aux| file of the main document
by setting |\jobname| to \textit{main}.

%%%%%%%%%%%%%%%%%%%%%%%%%%%%%%%%%%%%%%%%%%%%%%%%%%%%%%%%%%%%%%%%%%%%%%%%%%%%%%%%
\subsection{Driver Development}
\label{sec:driver}

The \textsf{childdoc} mechanism can also be use for the development
of definition files such as \LaTeX{} styles or classes.
This case differs from the above setup with multiple parts
included by |\include| in that no |\includeonly| should be invoked.
This can be achieved by starting the include file
(before |\ProvidesPackage|) with:
%
\begin{center}
\begin{tabular}{l}
|\input{childdoc.def}|\\
|\childdocforward{|\textit{main}|}|\\
\end{tabular}
\end{center}
%
or alternatively with:
%
\begin{center}
\begin{tabular}{l}
|\input{childdoc.def}|\\
|\childdocby{|\textit{main}|}|\\
\end{tabular}
\end{center}
%
Both forms have slightly different effects as described above.
The main file is prepared as usual, see \secref{sec:include}.

%%%%%%%%%%%%%%%%%%%%%%%%%%%%%%%%%%%%%%%%%%%%%%%%%%%%%%%%%%%%%%%%%%%%%%%%%%%%%%%%
\subsection{Legacy Detection}
\label{sec:detection}

The directive |\childdocmain| in the main file can detect
whether the complete document or merely a child is to be compiled
even without using the directive |\childdocof|.
This method is deprecated because it is less robust
and there is no compelling reason to use it;
it is merely provided for backward compatibility
and it may be removed in future versions.

If the detection mechanism is to be used,
it is mandatory to correctly specify
the filename of the main file as the argument of |\childdocmain|:
%
\begin{center}
\begin{tabular}{l}
|\input{childdoc.def}|\\
|\childdocmain{|\textit{main}|}|\\
\end{tabular}
\end{center}
%
If |\jobname| does not match the argument \textit{main} of |\childdocmain|,
it is assumed that |\jobname| points to the child file to be compiled.
When using |\childdocmain| with the main file specified as argument,
it suffices to start a child file
with just |\input{|\textit{main}|}|
without loading of the package and using |\childdocof|.
If instead all processing is done
with the appropriate \textsf{childdoc} directives,
the argument of \textit{main} of |\childdocmain| can be empty.

An alternative version of the command line processing described
in \secref{sec:commandline} using the detection mechanism reads:
%
\begin{center}
|... -jobname "|\textit{target}|" "|[\textit{flags}]%
[|\def\jobname{|\textit{dest}|}|]|\input{|\textit{main}|}"|
\end{center}

%%%%%%%%%%%%%%%%%%%%%%%%%%%%%%%%%%%%%%%%%%%%%%%%%%%%%%%%%%%%%%%%%%%%%%%%%%%%%%%%
\subsection{Manual Code}
\label{sec:manual}

In case one cannot be certain whether the definitions file |childdoc.def|
is installed on the target \TeX{} distribution
and one prefers not to ship it,
it is conceivable to paste a few relevant commands into the sources.

To that end, drop all statements |\input{childdoc.def}|
and perform the replacements as outlined below.
Instead of |\childdocmain{|\textit{main}|}| add the following code
to the top of the main file:
%
\begin{center}
\begin{tabular}{l}
|\||ifdefined\childdocname\endinput\||fi\newif\ifchilddoc|\\
|\edef\childdocname{\scantokens\expandafter{\jobname\noexpand}}|\\
|\def\childdocmain{|\textit{main}|}\||ifx\childdocmain\childdocname\||else|\\
|\childdoctrue\includeonly{\childdocname}\let\jobname\childdocmain\||fi|\\
\end{tabular}
\end{center}
%
Instead of |\childdocof{|\textit{main}|}| just include the main file
at the top of each child file:
%
\begin{center}
|\input{|\textit{main}|}|
\end{center}
%
A simple redirection |\childdocforward{|\textit{dest}|}| is achieved by:
%
\begin{center}
|\def\jobname{|\textit{dest}|}\input{\jobname}|
\end{center}
%
The redirection with prefix
|\childdocforwardprefix[|\textit{prefix}|]{|\textit{dest}|}|
is accomplished by:
%
\begin{center}
\begin{tabular}{l}
|{\edef\jobname{\scantokens\expandafter{\jobname\noexpand}}|\\
|\def\redirectjob |\textit{prefix}|#1~~~{\gdef\jobname{|\textit{dest}|#1}}|\\
|\expandafter\redirectjob\jobname~~~}\input{\jobname}|
\end{tabular}
\end{center}

In an alternative approach,
child documents can be compiled by a specific command line
without additional code or specific definitions:
%
\begin{center}
|... -jobname "|\textit{target}|" "|[\textit{flags}]%
|\includeonly{|\textit{dest}|}\input{|\textit{main}|}"|
\end{center}
%

%%%%%%%%%%%%%%%%%%%%%%%%%%%%%%%%%%%%%%%%%%%%%%%%%%%%%%%%%%%%%%%%%%%%%%%%%%%%%%%%
%%%%%%%%%%%%%%%%%%%%%%%%%%%%%%%%%%%%%%%%%%%%%%%%%%%%%%%%%%%%%%%%%%%%%%%%%%%%%%%%
\section{Information}

%%%%%%%%%%%%%%%%%%%%%%%%%%%%%%%%%%%%%%%%%%%%%%%%%%%%%%%%%%%%%%%%%%%%%%%%%%%%%%%%
\subsection{Copyright}

Copyright \copyright{} 2017--2018 Niklas Beisert

This work may be distributed and/or modified under the
conditions of the \LaTeX{} Project Public License, either version 1.3
of this license or (at your option) any later version.
The latest version of this license is in
  \url{http://www.latex-project.org/lppl.txt}
and version 1.3 or later is part of all distributions of \LaTeX{}
version 2005/12/01 or later.

This work has the LPPL maintenance status `maintained'.

The Current Maintainer of this work is Niklas Beisert.

This work consists of the files |README.txt|, |childdoc.ins| and |childdoc.dtx|
as well as the derived files |childdoc.def|, |cdocsamp.tex|
with |cdocsch1.tex|, |cdocsch2.tex|, |cdocspt3.tex|, |cdocspt4.tex|,
|cdocsdrf.tex|, |cdocsfn1.tex|, |cdocsfn2.tex|
as well as |childdoc.pdf|.

%%%%%%%%%%%%%%%%%%%%%%%%%%%%%%%%%%%%%%%%%%%%%%%%%%%%%%%%%%%%%%%%%%%%%%%%%%%%%%%%
\subsection{Files and Installation}

The package consists of the files:
%
\begin{center}
\begin{tabular}{ll}
    |README.txt|   & readme file \\
    |childdoc.ins| & installation file \\
    |childdoc.dtx| & source file \\
    |childdoc.def| & definition file \\
    |cdocsamp.tex| & sample main file \\
    |cdocsch1.tex| & sample include file \\
    |cdocsch2.tex| & sample include file \\
    |cdocspt3.tex| & sample part file \\
    |cdocspt4.tex| & sample part file \\
    |cdocsdrf.tex| & sample redirection file \\
    |cdocsfn1.tex| & sample redirection file \\
    |cdocsfn2.tex| & sample redirection file \\
    |childdoc.pdf| & manual
\end{tabular}
\end{center}
%
The distribution consists of the files
|README.txt|, |childdoc.ins| and |childdoc.dtx|.
%
\begin{itemize}
\item
Run (pdf)\LaTeX{} on |childdoc.dtx|
to compile the manual |childdoc.pdf| (this file).
\item
Run \LaTeX{} on |childdoc.ins| to create the definitions file |childdoc.def|
and the sample |cdocsamp.tex| with include files
|cdocsch1.tex|, |cdocsch2.tex|, |cdocspt3.tex|, |cdocspt4.tex|,
|cdocsdrf.tex|, |cdocsfn1.tex|, |cdocsfn2.tex|.
Then copy the file |childdoc.def| to an appropriate directory of your \LaTeX{}
distribution, e.g.\ \textit{texmf-root}|/tex/latex/childdoc|.
\end{itemize}

%%%%%%%%%%%%%%%%%%%%%%%%%%%%%%%%%%%%%%%%%%%%%%%%%%%%%%%%%%%%%%%%%%%%%%%%%%%%%%%%
\subsection{Related CTAN Packages}

There are several other packages which offer a similar functionality:
%
\begin{itemize}
\item
The packages
\href{http://ctan.org/pkg/docmute}{\textsf{docmute}},
\href{http://ctan.org/pkg/includex}{\textsf{includex}} and
\href{http://ctan.org/pkg/standalone}{\textsf{standalone}}
provide commands to include only the document body of
a child file thus allowing both files to be compiled individually.
\item
The packages \href{http://ctan.org/pkg/subdocs}{\textsf{subdocs}}
and \href{http://ctan.org/pkg/subfiles}{\textsf{subfiles}}
provide structures in which the main and child documents can be
encapsulated and allowing them to be compiled individually.
The inclusion mechanism is different from the conventional |\include|.
\item
The package \href{http://ctan.org/pkg/combine}{\textsf{combine}}
is an elaborate solution to combine several documents into one.
\end{itemize}
%
See also the CTAN topic \href{http://ctan.org/topic/subdocs}{\textsf{subdocs}}
for further related packages.
The present package differs from the above solutions in that
a document structure constructed with the conventional |\include| mechanism
just needs two extra commands at the top of every file
such that all constituent files can be compiled individually.

%%%%%%%%%%%%%%%%%%%%%%%%%%%%%%%%%%%%%%%%%%%%%%%%%%%%%%%%%%%%%%%%%%%%%%%%%%%%%%%%
%\subsection{Feature Suggestions}
%
%The following is a list of features which may be useful for future
%versions of this package:
%%
%\begin{itemize}
%\item
%\ldots
%\end{itemize}

%%%%%%%%%%%%%%%%%%%%%%%%%%%%%%%%%%%%%%%%%%%%%%%%%%%%%%%%%%%%%%%%%%%%%%%%%%%%%%%%
\subsection{Revision History}

%%%%%%%%%%%%%%%%%%%%%%%%%%%%%%%%%%%%%%%%
\paragraph{v2.0:} 2018/12/30

\begin{itemize}
\item
immediate forward processing
\item
added |\childdocby| mechanism
\item
manual restructured
\end{itemize}

%%%%%%%%%%%%%%%%%%%%%%%%%%%%%%%%%%%%%%%%
\paragraph{v1.6:} 2018/01/17

\begin{itemize}
\item
application for development of include files
\item
corrections to manual
\end{itemize}

%%%%%%%%%%%%%%%%%%%%%%%%%%%%%%%%%%%%%%%%
\paragraph{v1.5:} 2017/05/21

\begin{itemize}
\item
more complete structuring introduced
\item
|\childdocof| introduced
\item
|\childdoc| renamed to |\childdocmain|
\item
|\childredirect| renamed to |\childdocforward| and |\childdocforwardprefix|
and functionality expanded
\end{itemize}

%%%%%%%%%%%%%%%%%%%%%%%%%%%%%%%%%%%%%%%%
\paragraph{v1.0:} 2017/04/27

\begin{itemize}
\item
manual and install package
\item
first version published on CTAN
\end{itemize}

%%%%%%%%%%%%%%%%%%%%%%%%%%%%%%%%%%%%%%%%
\paragraph{v0.6:} 2017/04/26

\begin{itemize}
\item
redirection mechanism added
\end{itemize}

%%%%%%%%%%%%%%%%%%%%%%%%%%%%%%%%%%%%%%%%
\paragraph{v0.5:} 2017/04/26

\begin{itemize}
\item
functionality in definition file
\end{itemize}


%%%%%%%%%%%%%%%%%%%%%%%%%%%%%%%%%%%%%%%%%%%%%%%%%%%%%%%%%%%%%%%%%%%%%%%%%%%%%%%%
%%%%%%%%%%%%%%%%%%%%%%%%%%%%%%%%%%%%%%%%%%%%%%%%%%%%%%%%%%%%%%%%%%%%%%%%%%%%%%%%
%%%%%%%%%%%%%%%%%%%%%%%%%%%%%%%%%%%%%%%%%%%%%%%%%%%%%%%%%%%%%%%%%%%%%%%%%%%%%%%%
\appendix

\settowidth\MacroIndent{\rmfamily\scriptsize 000\ }

 \DocInput{childdoc.dtx}

\end{document}
%</driver>
% \fi
%
% %%%%%%%%%%%%%%%%%%%%%%%%%%%%%%%%%%%%%%%%%%%%%%%%%%%%%%%%%%%%%%%%%%%%%%%%%%%%%%
% %%%%%%%%%%%%%%%%%%%%%%%%%%%%%%%%%%%%%%%%%%%%%%%%%%%%%%%%%%%%%%%%%%%%%%%%%%%%%%
% \section{Sample}
%\iffalse
%<*samplemain>
%\fi
%
% The following presents a sample document
% with two chapters, two parts, a title page,
% a compile flag as well as three forwarding files to set the flag.
% It consists of eight |.tex| files:
% \begin{center}
% \begin{tabular}{ll}
% |cdocsamp.tex|&main file\\
% |cdocsch1.tex|&include file for chapter 1\\
% |cdocsch2.tex|&include file for chapter 2\\
% |cdocspt3.tex|&include file for part 3\\
% |cdocspt4.tex|&include file for part 4\\
% |cdocsdrf.tex|&forwarding file for main file in draft mode\\
% |cdocsfi1.tex|&forwarding file for final version of chapter 1\\
% |cdocsfi2.tex|&forwarding file for final version of chapter 2\\
% \end{tabular}
% \end{center}
% Each of the eight files can be compiled directly by the \LaTeX{} compiler.
%
% %%%%%%%%%%%%%%%%%%%%%%%%%%%%%%%%%%%%%%
% \paragraph{Main File.}
%
% The main file is called |cdocsamp.tex|.
%
% Load the \textsf{childdoc} definitions and
% declare the filename for the main document:
%    \begin{macrocode}
\input{childdoc.def}
\childdocmain{}
%    \end{macrocode}

% Optional override for |\version| flag:
%    \begin{macrocode}
%%\ifchilddoc\else\providecommand{\version}{draft}\fi
%    \end{macrocode}

% Define the default values for the |\version| flag
% (|final| for the main file and |draft| for childs):
%    \begin{macrocode}
\ifchilddoc
\providecommand{\version}{draft}
\else
\providecommand{\version}{final}
\fi
%    \end{macrocode}

% Load the standard document class:
%    \begin{macrocode}
\documentclass[12pt]{article}
%    \end{macrocode}

% Start the document body:
%    \begin{macrocode}
\begin{document}
%    \end{macrocode}

% Declare a title page.
% Print title, part of document being processed and version flag:
%    \begin{macrocode}
\addtocounter{page}{-1}
\begin{center}
{\LARGE\bfseries{}childdoc example\par}
\vspace{1cm}
\ifchilddoc
\ifchilddocmanual part\else chapter\fi:
`\childdocname' of `\childdocjob'\par
\else
main document: `\childdocjob'\par
\fi
version: \version\par
\end{center}
\newpage
%    \end{macrocode}

% Manually include selected file,
% otherwise process as usual:
%    \begin{macrocode}
\ifchilddocmanual
\section*{part `\childdocname'}
\input{\childdocname}
\else
%    \end{macrocode}

% Include the two chapters:
%    \begin{macrocode}
\include{cdocsch1}
\include{cdocsch2}
%    \end{macrocode}

% Include the two parts unless only chapters should be displayed:
%    \begin{macrocode}
\ifchilddoc\else
\section{part three}
\input{cdocspt3}
\section{part four}
\input{cdocspt4}
\fi
%    \end{macrocode}

% Process as usual until here:
%    \begin{macrocode}
\fi
%    \end{macrocode}

% End of document body:
%    \begin{macrocode}
\end{document}
%    \end{macrocode}
%\iffalse
%</samplemain>
%\fi
%
% %%%%%%%%%%%%%%%%%%%%%%%%%%%%%%%%%%%%%%
% \paragraph{Chapter Include Files.}
%
% The include files are called |cdocsch1.tex| and |cdocsch2.tex|.
%
%\iffalse
%<*samplechap1|samplechap2>
%\fi

% Optional override for |\version| flag:
%    \begin{macrocode}
%%\providecommand{\version}{final}
%    \end{macrocode}

% Include the main document:
%    \begin{macrocode}
\input{childdoc.def}
\childdocof{cdocsamp}
%    \end{macrocode}

%\iffalse
%</samplechap1|samplechap2>
%\fi
%
%\iffalse
%<*samplechap1>
%\fi
% Some text for chapter 1:
%    \begin{macrocode}
\section{one}
some text in chapter one
%    \end{macrocode}

%\iffalse
%</samplechap1>
%\fi
% Some text for chapter 2:
%\iffalse
%<*samplechap2>
%\fi
%    \begin{macrocode}
\section{two}
more text in chapter two
%    \end{macrocode}

%\iffalse
%</samplechap2>
%\fi
%
% %%%%%%%%%%%%%%%%%%%%%%%%%%%%%%%%%%%%%%
% \paragraph{Part Include Files.}
%
% The include files are called |cdocspt3.tex| and |cdocspt4.tex|.
%
%\iffalse
%<*samplepart3|samplepart4>
%\fi

% Optional override for |\version| flag:
%    \begin{macrocode}
%%\providecommand{\version}{final}
%    \end{macrocode}

% Include the main document:
%    \begin{macrocode}
\input{childdoc.def}
\childdocby{cdocsamp}
%    \end{macrocode}

%\iffalse
%</samplepart3|samplepart4>
%\fi
%
%\iffalse
%<*samplepart3>
%\fi
% Some text for part 3:
%    \begin{macrocode}
some text in part three
%    \end{macrocode}

%\iffalse
%</samplepart3>
%\fi
% Some text for part 4:
%\iffalse
%<*samplepart4>
%\fi
%    \begin{macrocode}
more text in part four
%    \end{macrocode}

%\iffalse
%</samplepart4>
%\fi
%
% %%%%%%%%%%%%%%%%%%%%%%%%%%%%%%%%%%%%%%
% \paragraph{Forwarding for a Complete Draft.}
%
% The following forwarding file |cdocsdrf.tex|
% compiles the main document in draft mode:
%\iffalse
%<*sampledraft>
%\fi
%    \begin{macrocode}
\def\version{draft}
\input{childdoc.def}
\childdocforward{cdocsamp}
%    \end{macrocode}

%\iffalse
%</sampledraft>
%\fi
%
% %%%%%%%%%%%%%%%%%%%%%%%%%%%%%%%%%%%%%%
% \paragraph{Forwarding for Final Version of the Chapters.}
%
% The following forwarding files |cdocsfn1.tex| and |cdocsfn2.tex|
% (with identical content)
% compile the final versions of the child documents
% |cdocsch1.tex| and |cdocsch2.tex|, respectively:
%\iffalse
%<*samplefinal>
%\fi
%    \begin{macrocode}
\def\version{final}
\input{childdoc.def}
\childdocforwardprefix[cdocsamp]{cdocsfn}{cdocsch}
%    \end{macrocode}

%\iffalse
%</samplefinal>
%\fi
%
% %%%%%%%%%%%%%%%%%%%%%%%%%%%%%%%%%%%%%%
% \paragraph{Command Line Processing.}
%
% The following three command lines generate the output files
% |cdocscld|, |cdocscl1| and |cdocscl2|
% which should be identical to
% |cdocsdrf|, |cdocsch1| and |cdocsfn2|, respectively:
% \begin{center}
% \begin{tabular}{l}
% |latex -jobname cdocscld \|\\
% |  "\def\version{draft}\input{childdoc.def}\childdocforward{cdocsamp}"|\\
% |latex -jobname cdocscl1 \|\\
% |  "\input{childdoc.def}\childdocforward[cdocsamp]{cdocsch1}"|\\
% |latex -jobname cdocscl2 \|\\
% |  "\def\version{final}\input{childdoc.def}\childdocforward{cdocsch2}"|
% \end{tabular}
% \end{center}
% Note that the trailing backslash on each first line
% merely continues the input to the second line
% (for convenient cut ant paste).
% Furthermore, the command |latex| can be replaced by any
% of its alternative versions such as |pdflatex|.
%
% %%%%%%%%%%%%%%%%%%%%%%%%%%%%%%%%%%%%%%%%%%%%%%%%%%%%%%%%%%%%%%%%%%%%%%%%%%%%%%
% %%%%%%%%%%%%%%%%%%%%%%%%%%%%%%%%%%%%%%%%%%%%%%%%%%%%%%%%%%%%%%%%%%%%%%%%%%%%%%
% \section{Implementation}
%\iffalse
%<*package>
%\fi
%
% This section describes the definitions file |childdoc.def|.

% The definitions cannot be loaded using |\usepackage| or |\RequirePackage|
% which has a mechanism to prevent loading a style file more than once.
% When loading the definitions by means of |\input|
% multiple instances have to be prevented manually:
%\iffalse
%This code needs to be before the `\ProvidesFile' directive
%which is defined at the beginning of this file.
%Therefore it is also placed there and commented out here.
%</package>
%<*discard>
%\fi
%    \begin{macrocode}
\ifdefined\childdocmain\endinput\fi
%    \end{macrocode}
%\iffalse
%</discard>
%<*package>
%\fi
%
% \macro{\ifchilddoc}
% \macro{\ifchilddocmanual}
% The conditional |\ifchilddoc| tells whether a
% child (true) or main (false) document is being compiled.
% The conditional |\ifchilddocmanual| tells whether
% the |\includeonly| mechanism is used (false) or
% the selection of child files must be performed manually (true).
% The definitions initialise to false:
%    \begin{macrocode}
\newif\ifchilddoc
\newif\ifchilddocmanual
%    \end{macrocode}

% \macro{\childdocname}
% \macro{\childdocjob}
% The macro |\childdocname| stores the name of the main document
% to be compiled. The macro |\childdocjob| stores the name of
% the document on which the \LaTeX{} compiler was originally invoked.
% The content of |\jobname| cannot be compared
% to filenames specified in the source due to different catcodes.
% The following code rescans |\jobname|, stores the result
% in |\childdocname| and saves a copy in |\childdocjob|:
%    \begin{macrocode}
\edef\childdocname{\scantokens\expandafter{\jobname\noexpand}}
\let\childdocjob\childdocname
%    \end{macrocode}

% \macro{\childdocdisable}
% The macro |\childdocdisable| prevents the main file
% from being processed more than once.
% At this stage, the main document command |\childdocmain|
% is assumed to be called once again where it should do nothing.
% Any subsequent call to it should prevent
% a secondary processing of the main document
% It overwrites the forwarding commands
% |\childdocof| and |\childdocforward|
% with empty macros to prevent further inclusions of the main document:
%    \begin{macrocode}
\newcommand{\childdocdisable}
{
  \renewcommand{\childdocmain}[1]{\renewcommand{\childdocmain}[1]{\endinput}}
  \renewcommand{\childdocof}[1]{}
  \renewcommand{\childdocby}[2][]{}
  \renewcommand{\childdocforward}[2][]{}
  \renewcommand{\childdocdisable}{}
}
%    \end{macrocode}

% \macro{\childdocmain}
% The macro |\childdocmain| is to be called at the top of the main file
% with nothing or the main filename (without extension) as argument.
% First, it breaks loops.
% If the argument is not empty and does not match |\childdocname|
% (which is set by the first inclusion of |childdoc.def|),
% |\ifchilddoc| is set to true, |\includeonly| is applied to the child file
% and |\jobname| is set to the main file
% (for proper handling of |.aux| files):
%    \begin{macrocode}
\newcommand{\childdocmain}[1]
{
  \childdocdisable\childdocmain{}
  \if?#1?\else
    \begingroup
      \def\childdoctmp{#1}
      \ifx\childdoctmp\childdocname
        \def\childdoctmp{}
      \else
        \def\childdoctmp
        {
          \childdoctrue
          \includeonly{\childdocname}
          \def\childdocjob{#1}
          \def\jobname{#1}
        }
      \fi
      \expandafter
    \endgroup
    \childdoctmp
  \fi
}
%    \end{macrocode}

% \macro{\childdocof}
% The command |\childdocof| redirects
% compilation to the main file |#1|.
%    \begin{macrocode}
\newcommand{\childdocof}[1]
{
  \childdocdisable
  \childdoctrue
  \includeonly{\childdocname}
  \def\jobname{#1}
  \def\childdocjob{#1}
  \input{#1}
}
%    \end{macrocode}

% \macro{\childdocby}
% The command |\childdocby| ....
%    \begin{macrocode}
\newcommand{\childdocby}[2][]
{
  \childdocdisable
  \childdoctrue
  \childdocmanualtrue
  \if?#1?\else
    \def\jobname{#2}
  \fi
  \def\childdocjob{#2}
  \input{#2}
  \endinput
}
%    \end{macrocode}

% \macro{\childdocforward}
% The command |\childdocforward| redirects
% compilation to the main file or
% (if the optional argument is given) a child file.
% Parameters are set as if the main file
% or a child file starting with |\childdocof| was compiled.
% Then compilation is handed over to the main file:
%    \begin{macrocode}
\newcommand{\childdocforward}[2][]
{
  \begingroup
    \if?#1?
      \def\childdoctmp
      {
        \def\childdocname{#2}
        \def\childdocjob{#2}
        \def\jobname{#2}
        \input{#2}
        \endinput
      }
    \else
      \def\childdoctmp
      {
        \childdocdisable
        \def\childdocname{#2}
        \childdoctrue
        \includeonly{#2}
        \def\childdocjob{#1}
        \def\jobname{#1}
        \input{#1}
        \endinput
      }
    \fi
    \expandafter
  \endgroup
  \childdoctmp
}
%    \end{macrocode}

% \macro{\childdocforwardprefix}
% The command |\childdocforwardprefix| redirects
% compilation to the main or a child file by means of a pattern.
% The prefix |#1| in the current filename is replaced by |#2|
% and the suffix of the current filename is kept
% (it is assumed that the filename does not contain the substring `|~~~|'
% which is used as a delimiter).
% Compilation is handed over to the new file by |\childdocforward|:
%    \begin{macrocode}
\newcommand{\childdocforwardprefix}[3][]
{
  \begingroup
    \def\childdocextract #2##1~~~{\def\childdoctmp{\childdocforward[#1]{#3##1}}}
    \expandafter\childdocextract\childdocname~~~
    \expandafter
  \endgroup
  \childdoctmp
}
%    \end{macrocode}

% \macro{\childdoc}
% The deprecated macro |\childdoc| is a legacy version of |\childdocmain|:
%    \begin{macrocode}
\newcommand{\childdoc}{\childdocmain}
%    \end{macrocode}

% \macro{\childdocredirect}
% The deprecated macro |\childdocredirect| is a legacy version
% of |\childdocforward| and |\childdocforwardprefix|:
%    \begin{macrocode}
\newcommand{\childdocredirect}[2][]
{
  \begingroup
    \if?#1?
      \def\childdoctmp{\childdocforward{#2}}
    \else
      \def\childdoctmp{\childdocforwardprefix{#1}{#2}}
    \fi
    \expandafter
  \endgroup
  \childdoctmp
}
%    \end{macrocode}

%\iffalse
%</package>
%\fi
%
\endinput
|\\
|\childdocforward[|\textit{main}|]{|\textit{dest}|}|\\
\end{tabular}
\end{center}
%
The argument \textit{dest} is the destination file
(without extension).
It should be the main file or one of the child files.
Note that further \textsf{childdoc} directives
such as |\childdocof| and |\childdocforward|
in the indicated file will be processed in this form.
The optional argument \textit{main}
passes on directly to the main file \textit{main}
while pretending to compile the child \textit{dest}.
This form behaves as if \textit{dest}
issues |\childdocof{|\textit{main}|}| right away,
and no further \textsf{childdoc} directives will be processed.

%%%%%%%%%%%%%%%%%%%%%%%%%%%%%%%%%%%%%%%%
\DescribeMacro{\...prefix}
In the alternative form |\childdocforwardprefix|,
%
\begin{center}
\begin{tabular}{l}
|% \iffalse
%
% childdoc.dtx Copyright (C) 2017-2018 Niklas Beisert
%
% This work may be distributed and/or modified under the
% conditions of the LaTeX Project Public License, either version 1.3
% of this license or (at your option) any later version.
% The latest version of this license is in
%   http://www.latex-project.org/lppl.txt
% and version 1.3 or later is part of all distributions of LaTeX
% version 2005/12/01 or later.
%
% This work has the LPPL maintenance status `maintained'.
%
% The Current Maintainer of this work is Niklas Beisert.
%
% This work consists of the files childdoc.dtx and childdoc.ins
% and the derived files childdoc.def and cdocsamp.tex with
% cdocsch1.tex, cdocsch2.tex, cdocsdrf.tex, cdocsfn1.tex, cdocsfn2.tex.
%
%<package>\ifdefined\childdocmain\endinput\fi
%<package>\ProvidesFile{childdoc.def}[2018/12/30 v2.0 child document driver]
%<samplemain>\ProvidesFile{cdocsamp.tex}[2018/12/30 v2.0 sample for childdoc]
%<*driver>
%\ProvidesFile{childdoc.drv}[2018/12/30 v2.0 childdoc reference manual file]
\PassOptionsToClass{10pt,a4paper}{article}
\documentclass{ltxdoc}

\usepackage[margin=35mm]{geometry}
\usepackage{hyperref}
\usepackage{hyperxmp}
\usepackage[usenames]{color}

\hypersetup{colorlinks=true}
\hypersetup{pdfstartview=FitH}
\hypersetup{pdfpagemode=UseNone}
\hypersetup{pdfsource={}}
\hypersetup{pdflang={en-UK}}
\hypersetup{pdfcopyright={Copyright 2017-2018 Niklas Beisert.
  This work may be distributed and/or modified under the
  conditions of the LaTeX Project Public License, either version 1.3
  of this license or (at your option) any later version.}}
\hypersetup{pdflicenseurl={http://www.latex-project.org/lppl.txt}}
\hypersetup{pdfcontactaddress={ETH Zurich, ITP, HIT K,
  Wolfgang-Pauli-Strasse 27}}
\hypersetup{pdfcontactpostcode={8093}}
\hypersetup{pdfcontactcity={Zurich}}
\hypersetup{pdfcontactcountry={Switzerland}}
\hypersetup{pdfcontactemail={nbeisert@itp.phys.ethz.ch}}
\hypersetup{pdfcontacturl={http://people.phys.ethz.ch/\xmptilde nbeisert/}}

\newcommand{\secref}[1]{\hyperref[#1]{section \ref*{#1}}}

\parskip1ex
\parindent0pt
\let\olditemize\itemize
\def\itemize{\olditemize\parskip0pt}

\begin{document}

\title{The \textsf{childdoc} Package}
\hypersetup{pdftitle={The childdoc Package}}
\author{Niklas Beisert\\[2ex]
  Institut f\"ur Theoretische Physik\\
  Eidgen\"ossische Technische Hochschule Z\"urich\\
  Wolfgang-Pauli-Strasse 27, 8093 Z\"urich, Switzerland\\[1ex]
  \href{mailto:nbeisert@itp.phys.ethz.ch}
  {\texttt{nbeisert@itp.phys.ethz.ch}}}
\hypersetup{pdfauthor={Niklas Beisert}}
\hypersetup{pdfsubject={Manual for the LaTeX2e Package childdoc}}
\date{30 December 2018, \textsf{v2.0}}
\maketitle

\begin{abstract}\noindent
\textsf{childdoc} is a \LaTeXe{} package
that enables the direct compilation
of document sections included by |\include|
to individual files.
\end{abstract}

\begingroup
\parskip0ex
\tableofcontents
\endgroup

%%%%%%%%%%%%%%%%%%%%%%%%%%%%%%%%%%%%%%%%%%%%%%%%%%%%%%%%%%%%%%%%%%%%%%%%%%%%%%%%
%%%%%%%%%%%%%%%%%%%%%%%%%%%%%%%%%%%%%%%%%%%%%%%%%%%%%%%%%%%%%%%%%%%%%%%%%%%%%%%%
\section{Introduction}

\LaTeX{} provides a mechanism to structure a large document (such as a book)
into a main file and several child files (containing the chapters)
using the |\include| command.
This mechanism is beneficial for documents
which span hundreds of pages in order to
make the source file(s) more manageable.
Moreover, compilation can be restricted to
selected child files by means of the |\includeonly| command.
The latter feature can be used to reduce the compilation time while editing
(this was significantly more useful in the earlier days of \LaTeX{})
or to generate a smaller document which is easier to navigate.
Another application of |\includeonly| is to generate
documents consisting of selected parts of the complete document.

However, there are a few drawbacks of the plain |\include| mechanism:
\begin{itemize}
\item
The child files cannot be compiled on their own,
they can only be compiled via the main file.
A naive editing environment
(such as a text editor with an option
to have the current file processed by \LaTeX)
may require one to switch to the main file before compiling;
attempting to compile the child file produces errors.
\item
The main file must be modified (each time)
to adjust the |\includeonly| command
to the present needs. This easily leaves the main file in a messy state.
\item
The generated document will always carry the filename
of the main document. This is inconvenient if
several child files are to be compiled and
to be kept for distribution.
\end{itemize}

The present package provides a simple interface
to make child files individually compilable by \LaTeX{}.
Compiling a child file then has the same effect as compiling
the main file with an |\includeonly| command
to select the appropriate child.
Moreover the generated document will carry the name of the child
rather than the main file.
This resolves all three above issues.

This feature is meant to make the editing of books,
thesis documents and lecture notes somewhat more convenient.
However, the package can also be used efficiently for
composing a series of documents (such as exercise sheets)
which are typically distributed individually.
It then assists the author in generating the individual documents
(potentially in different versions)
as well as a document containing the collected series.
Another application is in developing style files
or other kinds of included material
where compilation of the style file could redirect
to a sample or test file.

%%%%%%%%%%%%%%%%%%%%%%%%%%%%%%%%%%%%%%%%%%%%%%%%%%%%%%%%%%%%%%%%%%%%%%%%%%%%%%%%
%%%%%%%%%%%%%%%%%%%%%%%%%%%%%%%%%%%%%%%%%%%%%%%%%%%%%%%%%%%%%%%%%%%%%%%%%%%%%%%%
\section{Usage}

First of all, the package \textsf{childdoc} is \emph{not} a standard
\LaTeXe{} |.sty| style file! Therefore it needs to be invoked in
a non-standard way.

%%%%%%%%%%%%%%%%%%%%%%%%%%%%%%%%%%%%%%%%%%%%%%%%%%%%%%%%%%%%%%%%%%%%%%%%%%%%%%%%
\subsection{Included Files}
\label{sec:include}

%%%%%%%%%%%%%%%%%%%%%%%%%%%%%%%%%%%%%%%%
\DescribeMacro{\childdocmain}
To use the package, add the commands
\begin{center}
\begin{tabular}{l}
|\input{childdoc.def}|\\
|\childdocmain{}|\\
\end{tabular}
\end{center}
at the very top of the main \LaTeX{} file,
in particular \emph{before} the |\documentclass| statement!
The argument of |\childdocmain| should be left empty
(but it must be present).

%%%%%%%%%%%%%%%%%%%%%%%%%%%%%%%%%%%%%%%%
\DescribeMacro{\childdocof}
Furthermore, add the commands
\begin{center}
\begin{tabular}{l}
|\input{childdoc.def}|\\
|\childdocof{|\textit{main}|}|\\
\end{tabular}
\end{center}
at the top of every child file \textit{child}
which is included by |\include{|\textit{child}|}|
from within the main file
(or at least for those files to be compiled individually).
The argument \textit{main} must be the filename of the main file.

There are a couple of
considerations in setting up the main and child documents:

%%%%%%%%%%%%%%%%%%%%%%%%%%%%%%%%%%%%%%%%
\paragraph{Restrictions.}

Please note the following restrictions:
\begin{itemize}
\item
|\childdocmain| must be called with one argument \textit{main}
to ensure compatibility with earlier version of the package.
It must either be empty (|\childdocmain{}|)
or precisely match the filename of the main file in which it is specified.
See \secref{sec:detection} for further information.
\item
The filename \textit{main} must be specified without the |.tex| extension.
\item
The filename \textit{main} is case sensitive
(even in case-insensitive file systems)
due to internal string comparison.
\item
The argument \textit{main} should be fully expanded, it cannot be a macro.
\item
Subdirectories and special characters should be avoided in filenames.
\item
The command |\childdocmain{|\textit{main}|}| must be followed by a whitespace.
It should not be followed immediately by another command
or by a comment mark `|%|'.
This is because the \TeX{} parser reads the token immediately following
the argument of |\childdocmain| and puts it
at the beginning of every child section;
however, a white\-space is ignored.
\end{itemize}

%%%%%%%%%%%%%%%%%%%%%%%%%%%%%%%%%%%%%%%%
\paragraph{Content of Main File.}

It is advisable to place all content in the child files included by |\include|.
Any output contained in the main file will appear in all child documents
unless suppressed manually;
it cannot be suppressed automatically by the |\includeonly| directive
and thus should normally be avoided.
A method to include some content in the main file
by means of conditional processing is described in \secref{sec:conditional}.

%%%%%%%%%%%%%%%%%%%%%%%%%%%%%%%%%%%%%%%%
\paragraph{Page Numbering.}

When only a part of the document is compiled,
the appropriate numbering of pages
(as well as other status parameters)
is determined from the |.aux| files.
The latter contain information from previous passes.
However this information needs to propagate through
all intermediate child documents.
Therefore the page numbering in child documents may well
be inconsistent until the complete document is compiled at least once.

A useful (if unconventional) way to always ensure a consistent
page numbering is to restart the numbering in each child document
and denote the pages by `\textit{child}|.|\textit{page}'
where \textit{child} represents the chapter/section number of the child file.
This can be achieved by the command
|\numberwithin{page}{|\textit{child}|}|
of the \textsf{amsmath} package
where \textit{child} can be |chapter| or |section|
depending on the chosen structuring.
Alternatively, one can modify the macro |\thepage| appropriately
and reset the counter |page| at the start of each child file.

%%%%%%%%%%%%%%%%%%%%%%%%%%%%%%%%%%%%%%%%%%%%%%%%%%%%%%%%%%%%%%%%%%%%%%%%%%%%%%%%
\subsection{Conditional Processing}
\label{sec:conditional}

The package provides a mechanism to compile different versions
of a document. To customise the versions further some conditional processing
can come in handy to distinguish which version is being compiled.
The package provides two macros to describe the compilation context:

%%%%%%%%%%%%%%%%%%%%%%%%%%%%%%%%%%%%%%%%
\DescribeMacro{\ifchilddoc}
The conditional |\ifchilddoc| distinguishes between the compilation of
child documents and the main document:
%
\begin{center}
|\ifchilddoc |\textit{child-code}| |[|\||else |\textit{main-code}]| \||fi|
\end{center}

%%%%%%%%%%%%%%%%%%%%%%%%%%%%%%%%%%%%%%%%
\DescribeMacro{\childdocname}
\DescribeMacro{\childdocjob}
The macro |\childdocname| contains the filename (without extension)
of the main or child file being processed.
Note that |\childdocjob| will always contain the name of the main file.

%%%%%%%%%%%%%%%%%%%%%%%%%%%%%%%%%%%%%%%%
\paragraph{Title Page.}

Conditional processing can be used to include a title or banner page
in the main document when proper precautions are taken.
Importantly, the code in the main file should ensure that the page counter
(as well as other status parameters which are stored in the |.aux| files)
takes the same value after the conditional processing.
Otherwise the page numbers may take divergent values
depending on which part is compiled.

For example, a title page could be declared by:
%
\begin{center}
\begin{tabular}{l}
|\ifchilddoc\||else|\\
|\addtocounter{page}{-1}|\\
\textit{code for title page}\\
|\newpage|\\
|\||fi|
\end{tabular}
\end{center}
%
A banner page for the child documents can be generated by:
%
\begin{center}
\begin{tabular}{l}
|\ifchilddoc|\\
|\addtocounter{page}{-1}|\\
\textit{code for banner page}\\
|\newpage|\\
|\||fi|
\end{tabular}
\end{center}
%
Here one could write a message such as:
\begin{center}
|This is the part \childdocname{} of \childdocjob{}.|
\end{center}

%%%%%%%%%%%%%%%%%%%%%%%%%%%%%%%%%%%%%%%%%%%%%%%%%%%%%%%%%%%%%%%%%%%%%%%%%%%%%%%%
\subsection{Flags}
\label{sec:flags}

The package makes it easy to generate different versions
of the main or child documents.
To this end compilation flags can be defined
and assigned different default values.
They will be particularly useful in conjunction
with the forwarding mechanism described in \secref{sec:forward}.

For example, it may be useful to have a flag |\version|
which can be set to |draft| or |final|.
The document source will contain some conditional code
depending on the value of |\version|.
Suppose further, the flag should default to |final| for the main file
and to |draft| for child files
which is a natural assignment for editing the document.
This is achieved by placing the following code
in the preamble of the main document
(below the |\childdocmain| directive):
%
\begin{center}
\begin{tabular}{l}
|\ifchilddoc|\\
|\providecommand{\version}{draft}|\\
|\||else|\\
|\providecommand{\version}{final}|\\
|\||fi|
\end{tabular}
\end{center}
%
The definition by |\providecommand| makes sure
that previous definitions are not overwritten.
Further statements |\providecommand{\version}{...}|
can thus be added before the above code to override it.

For the main file, one might add a line
(between |\childdocmain| and the above block)
%
\begin{center}
|%\ifchilddoc\||else\providecommand{\version}{draft}\||fi|
\end{center}
%
which can be uncommented to produce a draft version.
Likewise one can add a line to the very top of a child file
(above the |\childdocof{|\textit{main}|}| directive)
%
\begin{center}
|%\providecommand{\version}{final}|
\end{center}
%
which can be uncommented to produce the final version of this child document.

%%%%%%%%%%%%%%%%%%%%%%%%%%%%%%%%%%%%%%%%%%%%%%%%%%%%%%%%%%%%%%%%%%%%%%%%%%%%%%%%
\subsection{Forwarding}
\label{sec:forward}

Different versions of the main or child documents
using compilation flags as described in \secref{sec:flags}
can be (permanently) stored in different files
for convenient compilation, viewing and distribution.
To this end, the package defines a command
to pass on compilation to a different file:

%%%%%%%%%%%%%%%%%%%%%%%%%%%%%%%%%%%%%%%%
\DescribeMacro{\childdocforward}
The command |\childdocforward| redirects processing to
another source file:
%
\begin{center}
\begin{tabular}{l}
|\input{childdoc.def}|\\
|\childdocforward[|\textit{main}|]{|\textit{dest}|}|\\
\end{tabular}
\end{center}
%
The argument \textit{dest} is the destination file
(without extension).
It should be the main file or one of the child files.
Note that further \textsf{childdoc} directives
such as |\childdocof| and |\childdocforward|
in the indicated file will be processed in this form.
The optional argument \textit{main}
passes on directly to the main file \textit{main}
while pretending to compile the child \textit{dest}.
This form behaves as if \textit{dest}
issues |\childdocof{|\textit{main}|}| right away,
and no further \textsf{childdoc} directives will be processed.

%%%%%%%%%%%%%%%%%%%%%%%%%%%%%%%%%%%%%%%%
\DescribeMacro{\...prefix}
In the alternative form |\childdocforwardprefix|,
%
\begin{center}
\begin{tabular}{l}
|\input{childdoc.def}|\\
|\childdocforwardprefix[|\textit{main}|]{|\textit{prefix}|}{|\textit{dest}|}|
\end{tabular}
\end{center}
%
the destination file is determined by a pattern
depending on the current file:
To make this work, the current file must be called
`{\textit{prefix}\hspace{0.2em}\textit{suffix}}'
with \textit{prefix} matching precisely the argument.
Processing is then passed on to the file
`{\textit{dest}\hspace{0.2em}\textit{suffix}}'.
Surely, the same effect is achieved by
directly specifying the
argument `{\textit{dest}\hspace{0.2em}\textit{suffix}}'
in the first form.
However, that requires to set up a different file
for each child. With the alternative form of the command
all these files can have exactly the same content
which simplifies setting them up and maintaining them.

For example, the following file |draft.tex|
with a compilation flag |\version| as described in \secref{sec:flags}
compiles the main document as a draft:
%
\begin{center}
\begin{tabular}{l}
|\def\version{draft}|\\
|\input{childdoc.def}|\\
|\childdocforward{|\textit{main}|}|
\end{tabular}
\end{center}
%
Likewise, the following files |final|\textit{nn}|.tex|
compile the final version of the child document
|child|\textit{nn}|.tex|:
%
\begin{center}
\begin{tabular}{l}
|\def\version{final}|\\
|\input{childdoc.def}|\\
|\childdocforwardprefix{final}{child}|
\end{tabular}
\end{center}
%

Note that when several versions of a main file and/or of each child file
are to be generated, it may be convenient to set up a |Makefile| or
shell script to automatise the process.

%%%%%%%%%%%%%%%%%%%%%%%%%%%%%%%%%%%%%%%%%%%%%%%%%%%%%%%%%%%%%%%%%%%%%%%%%%%%%%%%
\subsection{Command Line Processing}
\label{sec:commandline}

The effect of redirection files can also be achieved by invoking
the \LaTeX{} compiler with a more elaborate command line.
Most conveniently this should be done as part
of a shell script or a |Makefile|.

When using \textsf{childdoc} in the main file, the following
command lines effectively perform a redirection
(note that depending on the shell being used,
backslashes may have to be doubled: `|\|' $\to$ `|\\|'):
%
\begin{center}
|... -jobname "|\textit{target}|" |\\|"|[\textit{flags}]%
|\input{childdoc.def}\childdocforward[|\textit{main}|]{|\textit{dest}|}"|
\end{center}
%
Here \textit{target} is the name of the output file,
\textit{main} is the name of the main file
and \textit{dest} is the name of the main or child file to be processed
(all filenames without extensions).
The optional argument \textit{main} can be omitted
if \textit{main} matches \textit{dest}.
Optionally, compilation \textit{flags} can be defined via |\def| commands.
This command line makes the \TeX{} engine believe
it is compiling the file \textit{target}
whose content is specified as the latter parameter.
The provided code then forwards the processing to
\textit{main} or \textit{dest} as described in \secref{sec:forward}.

%%%%%%%%%%%%%%%%%%%%%%%%%%%%%%%%%%%%%%%%%%%%%%%%%%%%%%%%%%%%%%%%%%%%%%%%%%%%%%%%
\subsection{Include by Input}
\label{sec:input}

Including child documents by |\include| has some restrictions by design.
Most notably, the content of a child document always occupies
its own set of pages; pages cannot be shared between child documents.
Usually, this behaviour makes perfect sense
because each child document contain an essential part of the document.
However, in some situations it may be desirable to compose
a document from a collection of parts
without having mandatory page breaks between then.
For this case, the package
provides a mechanism to include parts
by |\input| which can also be processed individually.
However, by construction this mechanism
requires manual handling of the content to be output.

%%%%%%%%%%%%%%%%%%%%%%%%%%%%%%%%%%%%%%%%
\DescribeMacro{\ifchilddocmanual}
The main file should be prepared as usual, see \secref{sec:include}.
However, the document body must make a distinction
between processing of an individual part and of the main document, e.g.:
%
\begin{center}
\begin{tabular}{l}
|\ifchilddocmanual|\\
|\input{\childdocname}|\\
|\||else|\\
\textit{document body with }|\input{|\textit{part}|}|\\
|\||fi|
\end{tabular}
\end{center}
%
The conditional |\ifchilddocmanual| is true whenever
a part to be included by |\input| is being compiled,
and the name of the part is stored in |\childdocname|.

%%%%%%%%%%%%%%%%%%%%%%%%%%%%%%%%%%%%%%%%
\DescribeMacro{\childdocby}
Each part to be included by |\input| should start with:
%
\begin{center}
\begin{tabular}{l}
|\input{childdoc.def}|\\
|\childdocby{|\textit{main}|}|\\
\end{tabular}
\end{center}
%
The directive |\childdocby| is similar to |\childdocof|
described in \secref{sec:include},
but the subsequent selection of content must be done manually.
To that end, both |\ifchilddoc| and |\ifchilddocmanual|
will be true upon processing of a part,
and the name of the part is stored in |\childdocname|.
Note that |\jobname| will be set to the filename of the current part
so that each part receives an individual |.aux| file
that does not interfere with the |.aux| file(s) of the main document.
This behaviour can be altered by the alternative form
|\childdocby[*]{|\textit{main}|}| (with a non-empty optional argument)
which uses the |.aux| file of the main document
by setting |\jobname| to \textit{main}.

%%%%%%%%%%%%%%%%%%%%%%%%%%%%%%%%%%%%%%%%%%%%%%%%%%%%%%%%%%%%%%%%%%%%%%%%%%%%%%%%
\subsection{Driver Development}
\label{sec:driver}

The \textsf{childdoc} mechanism can also be use for the development
of definition files such as \LaTeX{} styles or classes.
This case differs from the above setup with multiple parts
included by |\include| in that no |\includeonly| should be invoked.
This can be achieved by starting the include file
(before |\ProvidesPackage|) with:
%
\begin{center}
\begin{tabular}{l}
|\input{childdoc.def}|\\
|\childdocforward{|\textit{main}|}|\\
\end{tabular}
\end{center}
%
or alternatively with:
%
\begin{center}
\begin{tabular}{l}
|\input{childdoc.def}|\\
|\childdocby{|\textit{main}|}|\\
\end{tabular}
\end{center}
%
Both forms have slightly different effects as described above.
The main file is prepared as usual, see \secref{sec:include}.

%%%%%%%%%%%%%%%%%%%%%%%%%%%%%%%%%%%%%%%%%%%%%%%%%%%%%%%%%%%%%%%%%%%%%%%%%%%%%%%%
\subsection{Legacy Detection}
\label{sec:detection}

The directive |\childdocmain| in the main file can detect
whether the complete document or merely a child is to be compiled
even without using the directive |\childdocof|.
This method is deprecated because it is less robust
and there is no compelling reason to use it;
it is merely provided for backward compatibility
and it may be removed in future versions.

If the detection mechanism is to be used,
it is mandatory to correctly specify
the filename of the main file as the argument of |\childdocmain|:
%
\begin{center}
\begin{tabular}{l}
|\input{childdoc.def}|\\
|\childdocmain{|\textit{main}|}|\\
\end{tabular}
\end{center}
%
If |\jobname| does not match the argument \textit{main} of |\childdocmain|,
it is assumed that |\jobname| points to the child file to be compiled.
When using |\childdocmain| with the main file specified as argument,
it suffices to start a child file
with just |\input{|\textit{main}|}|
without loading of the package and using |\childdocof|.
If instead all processing is done
with the appropriate \textsf{childdoc} directives,
the argument of \textit{main} of |\childdocmain| can be empty.

An alternative version of the command line processing described
in \secref{sec:commandline} using the detection mechanism reads:
%
\begin{center}
|... -jobname "|\textit{target}|" "|[\textit{flags}]%
[|\def\jobname{|\textit{dest}|}|]|\input{|\textit{main}|}"|
\end{center}

%%%%%%%%%%%%%%%%%%%%%%%%%%%%%%%%%%%%%%%%%%%%%%%%%%%%%%%%%%%%%%%%%%%%%%%%%%%%%%%%
\subsection{Manual Code}
\label{sec:manual}

In case one cannot be certain whether the definitions file |childdoc.def|
is installed on the target \TeX{} distribution
and one prefers not to ship it,
it is conceivable to paste a few relevant commands into the sources.

To that end, drop all statements |\input{childdoc.def}|
and perform the replacements as outlined below.
Instead of |\childdocmain{|\textit{main}|}| add the following code
to the top of the main file:
%
\begin{center}
\begin{tabular}{l}
|\||ifdefined\childdocname\endinput\||fi\newif\ifchilddoc|\\
|\edef\childdocname{\scantokens\expandafter{\jobname\noexpand}}|\\
|\def\childdocmain{|\textit{main}|}\||ifx\childdocmain\childdocname\||else|\\
|\childdoctrue\includeonly{\childdocname}\let\jobname\childdocmain\||fi|\\
\end{tabular}
\end{center}
%
Instead of |\childdocof{|\textit{main}|}| just include the main file
at the top of each child file:
%
\begin{center}
|\input{|\textit{main}|}|
\end{center}
%
A simple redirection |\childdocforward{|\textit{dest}|}| is achieved by:
%
\begin{center}
|\def\jobname{|\textit{dest}|}\input{\jobname}|
\end{center}
%
The redirection with prefix
|\childdocforwardprefix[|\textit{prefix}|]{|\textit{dest}|}|
is accomplished by:
%
\begin{center}
\begin{tabular}{l}
|{\edef\jobname{\scantokens\expandafter{\jobname\noexpand}}|\\
|\def\redirectjob |\textit{prefix}|#1~~~{\gdef\jobname{|\textit{dest}|#1}}|\\
|\expandafter\redirectjob\jobname~~~}\input{\jobname}|
\end{tabular}
\end{center}

In an alternative approach,
child documents can be compiled by a specific command line
without additional code or specific definitions:
%
\begin{center}
|... -jobname "|\textit{target}|" "|[\textit{flags}]%
|\includeonly{|\textit{dest}|}\input{|\textit{main}|}"|
\end{center}
%

%%%%%%%%%%%%%%%%%%%%%%%%%%%%%%%%%%%%%%%%%%%%%%%%%%%%%%%%%%%%%%%%%%%%%%%%%%%%%%%%
%%%%%%%%%%%%%%%%%%%%%%%%%%%%%%%%%%%%%%%%%%%%%%%%%%%%%%%%%%%%%%%%%%%%%%%%%%%%%%%%
\section{Information}

%%%%%%%%%%%%%%%%%%%%%%%%%%%%%%%%%%%%%%%%%%%%%%%%%%%%%%%%%%%%%%%%%%%%%%%%%%%%%%%%
\subsection{Copyright}

Copyright \copyright{} 2017--2018 Niklas Beisert

This work may be distributed and/or modified under the
conditions of the \LaTeX{} Project Public License, either version 1.3
of this license or (at your option) any later version.
The latest version of this license is in
  \url{http://www.latex-project.org/lppl.txt}
and version 1.3 or later is part of all distributions of \LaTeX{}
version 2005/12/01 or later.

This work has the LPPL maintenance status `maintained'.

The Current Maintainer of this work is Niklas Beisert.

This work consists of the files |README.txt|, |childdoc.ins| and |childdoc.dtx|
as well as the derived files |childdoc.def|, |cdocsamp.tex|
with |cdocsch1.tex|, |cdocsch2.tex|, |cdocspt3.tex|, |cdocspt4.tex|,
|cdocsdrf.tex|, |cdocsfn1.tex|, |cdocsfn2.tex|
as well as |childdoc.pdf|.

%%%%%%%%%%%%%%%%%%%%%%%%%%%%%%%%%%%%%%%%%%%%%%%%%%%%%%%%%%%%%%%%%%%%%%%%%%%%%%%%
\subsection{Files and Installation}

The package consists of the files:
%
\begin{center}
\begin{tabular}{ll}
    |README.txt|   & readme file \\
    |childdoc.ins| & installation file \\
    |childdoc.dtx| & source file \\
    |childdoc.def| & definition file \\
    |cdocsamp.tex| & sample main file \\
    |cdocsch1.tex| & sample include file \\
    |cdocsch2.tex| & sample include file \\
    |cdocspt3.tex| & sample part file \\
    |cdocspt4.tex| & sample part file \\
    |cdocsdrf.tex| & sample redirection file \\
    |cdocsfn1.tex| & sample redirection file \\
    |cdocsfn2.tex| & sample redirection file \\
    |childdoc.pdf| & manual
\end{tabular}
\end{center}
%
The distribution consists of the files
|README.txt|, |childdoc.ins| and |childdoc.dtx|.
%
\begin{itemize}
\item
Run (pdf)\LaTeX{} on |childdoc.dtx|
to compile the manual |childdoc.pdf| (this file).
\item
Run \LaTeX{} on |childdoc.ins| to create the definitions file |childdoc.def|
and the sample |cdocsamp.tex| with include files
|cdocsch1.tex|, |cdocsch2.tex|, |cdocspt3.tex|, |cdocspt4.tex|,
|cdocsdrf.tex|, |cdocsfn1.tex|, |cdocsfn2.tex|.
Then copy the file |childdoc.def| to an appropriate directory of your \LaTeX{}
distribution, e.g.\ \textit{texmf-root}|/tex/latex/childdoc|.
\end{itemize}

%%%%%%%%%%%%%%%%%%%%%%%%%%%%%%%%%%%%%%%%%%%%%%%%%%%%%%%%%%%%%%%%%%%%%%%%%%%%%%%%
\subsection{Related CTAN Packages}

There are several other packages which offer a similar functionality:
%
\begin{itemize}
\item
The packages
\href{http://ctan.org/pkg/docmute}{\textsf{docmute}},
\href{http://ctan.org/pkg/includex}{\textsf{includex}} and
\href{http://ctan.org/pkg/standalone}{\textsf{standalone}}
provide commands to include only the document body of
a child file thus allowing both files to be compiled individually.
\item
The packages \href{http://ctan.org/pkg/subdocs}{\textsf{subdocs}}
and \href{http://ctan.org/pkg/subfiles}{\textsf{subfiles}}
provide structures in which the main and child documents can be
encapsulated and allowing them to be compiled individually.
The inclusion mechanism is different from the conventional |\include|.
\item
The package \href{http://ctan.org/pkg/combine}{\textsf{combine}}
is an elaborate solution to combine several documents into one.
\end{itemize}
%
See also the CTAN topic \href{http://ctan.org/topic/subdocs}{\textsf{subdocs}}
for further related packages.
The present package differs from the above solutions in that
a document structure constructed with the conventional |\include| mechanism
just needs two extra commands at the top of every file
such that all constituent files can be compiled individually.

%%%%%%%%%%%%%%%%%%%%%%%%%%%%%%%%%%%%%%%%%%%%%%%%%%%%%%%%%%%%%%%%%%%%%%%%%%%%%%%%
%\subsection{Feature Suggestions}
%
%The following is a list of features which may be useful for future
%versions of this package:
%%
%\begin{itemize}
%\item
%\ldots
%\end{itemize}

%%%%%%%%%%%%%%%%%%%%%%%%%%%%%%%%%%%%%%%%%%%%%%%%%%%%%%%%%%%%%%%%%%%%%%%%%%%%%%%%
\subsection{Revision History}

%%%%%%%%%%%%%%%%%%%%%%%%%%%%%%%%%%%%%%%%
\paragraph{v2.0:} 2018/12/30

\begin{itemize}
\item
immediate forward processing
\item
added |\childdocby| mechanism
\item
manual restructured
\end{itemize}

%%%%%%%%%%%%%%%%%%%%%%%%%%%%%%%%%%%%%%%%
\paragraph{v1.6:} 2018/01/17

\begin{itemize}
\item
application for development of include files
\item
corrections to manual
\end{itemize}

%%%%%%%%%%%%%%%%%%%%%%%%%%%%%%%%%%%%%%%%
\paragraph{v1.5:} 2017/05/21

\begin{itemize}
\item
more complete structuring introduced
\item
|\childdocof| introduced
\item
|\childdoc| renamed to |\childdocmain|
\item
|\childredirect| renamed to |\childdocforward| and |\childdocforwardprefix|
and functionality expanded
\end{itemize}

%%%%%%%%%%%%%%%%%%%%%%%%%%%%%%%%%%%%%%%%
\paragraph{v1.0:} 2017/04/27

\begin{itemize}
\item
manual and install package
\item
first version published on CTAN
\end{itemize}

%%%%%%%%%%%%%%%%%%%%%%%%%%%%%%%%%%%%%%%%
\paragraph{v0.6:} 2017/04/26

\begin{itemize}
\item
redirection mechanism added
\end{itemize}

%%%%%%%%%%%%%%%%%%%%%%%%%%%%%%%%%%%%%%%%
\paragraph{v0.5:} 2017/04/26

\begin{itemize}
\item
functionality in definition file
\end{itemize}


%%%%%%%%%%%%%%%%%%%%%%%%%%%%%%%%%%%%%%%%%%%%%%%%%%%%%%%%%%%%%%%%%%%%%%%%%%%%%%%%
%%%%%%%%%%%%%%%%%%%%%%%%%%%%%%%%%%%%%%%%%%%%%%%%%%%%%%%%%%%%%%%%%%%%%%%%%%%%%%%%
%%%%%%%%%%%%%%%%%%%%%%%%%%%%%%%%%%%%%%%%%%%%%%%%%%%%%%%%%%%%%%%%%%%%%%%%%%%%%%%%
\appendix

\settowidth\MacroIndent{\rmfamily\scriptsize 000\ }

 \DocInput{childdoc.dtx}

\end{document}
%</driver>
% \fi
%
% %%%%%%%%%%%%%%%%%%%%%%%%%%%%%%%%%%%%%%%%%%%%%%%%%%%%%%%%%%%%%%%%%%%%%%%%%%%%%%
% %%%%%%%%%%%%%%%%%%%%%%%%%%%%%%%%%%%%%%%%%%%%%%%%%%%%%%%%%%%%%%%%%%%%%%%%%%%%%%
% \section{Sample}
%\iffalse
%<*samplemain>
%\fi
%
% The following presents a sample document
% with two chapters, two parts, a title page,
% a compile flag as well as three forwarding files to set the flag.
% It consists of eight |.tex| files:
% \begin{center}
% \begin{tabular}{ll}
% |cdocsamp.tex|&main file\\
% |cdocsch1.tex|&include file for chapter 1\\
% |cdocsch2.tex|&include file for chapter 2\\
% |cdocspt3.tex|&include file for part 3\\
% |cdocspt4.tex|&include file for part 4\\
% |cdocsdrf.tex|&forwarding file for main file in draft mode\\
% |cdocsfi1.tex|&forwarding file for final version of chapter 1\\
% |cdocsfi2.tex|&forwarding file for final version of chapter 2\\
% \end{tabular}
% \end{center}
% Each of the eight files can be compiled directly by the \LaTeX{} compiler.
%
% %%%%%%%%%%%%%%%%%%%%%%%%%%%%%%%%%%%%%%
% \paragraph{Main File.}
%
% The main file is called |cdocsamp.tex|.
%
% Load the \textsf{childdoc} definitions and
% declare the filename for the main document:
%    \begin{macrocode}
\input{childdoc.def}
\childdocmain{}
%    \end{macrocode}

% Optional override for |\version| flag:
%    \begin{macrocode}
%%\ifchilddoc\else\providecommand{\version}{draft}\fi
%    \end{macrocode}

% Define the default values for the |\version| flag
% (|final| for the main file and |draft| for childs):
%    \begin{macrocode}
\ifchilddoc
\providecommand{\version}{draft}
\else
\providecommand{\version}{final}
\fi
%    \end{macrocode}

% Load the standard document class:
%    \begin{macrocode}
\documentclass[12pt]{article}
%    \end{macrocode}

% Start the document body:
%    \begin{macrocode}
\begin{document}
%    \end{macrocode}

% Declare a title page.
% Print title, part of document being processed and version flag:
%    \begin{macrocode}
\addtocounter{page}{-1}
\begin{center}
{\LARGE\bfseries{}childdoc example\par}
\vspace{1cm}
\ifchilddoc
\ifchilddocmanual part\else chapter\fi:
`\childdocname' of `\childdocjob'\par
\else
main document: `\childdocjob'\par
\fi
version: \version\par
\end{center}
\newpage
%    \end{macrocode}

% Manually include selected file,
% otherwise process as usual:
%    \begin{macrocode}
\ifchilddocmanual
\section*{part `\childdocname'}
\input{\childdocname}
\else
%    \end{macrocode}

% Include the two chapters:
%    \begin{macrocode}
\include{cdocsch1}
\include{cdocsch2}
%    \end{macrocode}

% Include the two parts unless only chapters should be displayed:
%    \begin{macrocode}
\ifchilddoc\else
\section{part three}
\input{cdocspt3}
\section{part four}
\input{cdocspt4}
\fi
%    \end{macrocode}

% Process as usual until here:
%    \begin{macrocode}
\fi
%    \end{macrocode}

% End of document body:
%    \begin{macrocode}
\end{document}
%    \end{macrocode}
%\iffalse
%</samplemain>
%\fi
%
% %%%%%%%%%%%%%%%%%%%%%%%%%%%%%%%%%%%%%%
% \paragraph{Chapter Include Files.}
%
% The include files are called |cdocsch1.tex| and |cdocsch2.tex|.
%
%\iffalse
%<*samplechap1|samplechap2>
%\fi

% Optional override for |\version| flag:
%    \begin{macrocode}
%%\providecommand{\version}{final}
%    \end{macrocode}

% Include the main document:
%    \begin{macrocode}
\input{childdoc.def}
\childdocof{cdocsamp}
%    \end{macrocode}

%\iffalse
%</samplechap1|samplechap2>
%\fi
%
%\iffalse
%<*samplechap1>
%\fi
% Some text for chapter 1:
%    \begin{macrocode}
\section{one}
some text in chapter one
%    \end{macrocode}

%\iffalse
%</samplechap1>
%\fi
% Some text for chapter 2:
%\iffalse
%<*samplechap2>
%\fi
%    \begin{macrocode}
\section{two}
more text in chapter two
%    \end{macrocode}

%\iffalse
%</samplechap2>
%\fi
%
% %%%%%%%%%%%%%%%%%%%%%%%%%%%%%%%%%%%%%%
% \paragraph{Part Include Files.}
%
% The include files are called |cdocspt3.tex| and |cdocspt4.tex|.
%
%\iffalse
%<*samplepart3|samplepart4>
%\fi

% Optional override for |\version| flag:
%    \begin{macrocode}
%%\providecommand{\version}{final}
%    \end{macrocode}

% Include the main document:
%    \begin{macrocode}
\input{childdoc.def}
\childdocby{cdocsamp}
%    \end{macrocode}

%\iffalse
%</samplepart3|samplepart4>
%\fi
%
%\iffalse
%<*samplepart3>
%\fi
% Some text for part 3:
%    \begin{macrocode}
some text in part three
%    \end{macrocode}

%\iffalse
%</samplepart3>
%\fi
% Some text for part 4:
%\iffalse
%<*samplepart4>
%\fi
%    \begin{macrocode}
more text in part four
%    \end{macrocode}

%\iffalse
%</samplepart4>
%\fi
%
% %%%%%%%%%%%%%%%%%%%%%%%%%%%%%%%%%%%%%%
% \paragraph{Forwarding for a Complete Draft.}
%
% The following forwarding file |cdocsdrf.tex|
% compiles the main document in draft mode:
%\iffalse
%<*sampledraft>
%\fi
%    \begin{macrocode}
\def\version{draft}
\input{childdoc.def}
\childdocforward{cdocsamp}
%    \end{macrocode}

%\iffalse
%</sampledraft>
%\fi
%
% %%%%%%%%%%%%%%%%%%%%%%%%%%%%%%%%%%%%%%
% \paragraph{Forwarding for Final Version of the Chapters.}
%
% The following forwarding files |cdocsfn1.tex| and |cdocsfn2.tex|
% (with identical content)
% compile the final versions of the child documents
% |cdocsch1.tex| and |cdocsch2.tex|, respectively:
%\iffalse
%<*samplefinal>
%\fi
%    \begin{macrocode}
\def\version{final}
\input{childdoc.def}
\childdocforwardprefix[cdocsamp]{cdocsfn}{cdocsch}
%    \end{macrocode}

%\iffalse
%</samplefinal>
%\fi
%
% %%%%%%%%%%%%%%%%%%%%%%%%%%%%%%%%%%%%%%
% \paragraph{Command Line Processing.}
%
% The following three command lines generate the output files
% |cdocscld|, |cdocscl1| and |cdocscl2|
% which should be identical to
% |cdocsdrf|, |cdocsch1| and |cdocsfn2|, respectively:
% \begin{center}
% \begin{tabular}{l}
% |latex -jobname cdocscld \|\\
% |  "\def\version{draft}\input{childdoc.def}\childdocforward{cdocsamp}"|\\
% |latex -jobname cdocscl1 \|\\
% |  "\input{childdoc.def}\childdocforward[cdocsamp]{cdocsch1}"|\\
% |latex -jobname cdocscl2 \|\\
% |  "\def\version{final}\input{childdoc.def}\childdocforward{cdocsch2}"|
% \end{tabular}
% \end{center}
% Note that the trailing backslash on each first line
% merely continues the input to the second line
% (for convenient cut ant paste).
% Furthermore, the command |latex| can be replaced by any
% of its alternative versions such as |pdflatex|.
%
% %%%%%%%%%%%%%%%%%%%%%%%%%%%%%%%%%%%%%%%%%%%%%%%%%%%%%%%%%%%%%%%%%%%%%%%%%%%%%%
% %%%%%%%%%%%%%%%%%%%%%%%%%%%%%%%%%%%%%%%%%%%%%%%%%%%%%%%%%%%%%%%%%%%%%%%%%%%%%%
% \section{Implementation}
%\iffalse
%<*package>
%\fi
%
% This section describes the definitions file |childdoc.def|.

% The definitions cannot be loaded using |\usepackage| or |\RequirePackage|
% which has a mechanism to prevent loading a style file more than once.
% When loading the definitions by means of |\input|
% multiple instances have to be prevented manually:
%\iffalse
%This code needs to be before the `\ProvidesFile' directive
%which is defined at the beginning of this file.
%Therefore it is also placed there and commented out here.
%</package>
%<*discard>
%\fi
%    \begin{macrocode}
\ifdefined\childdocmain\endinput\fi
%    \end{macrocode}
%\iffalse
%</discard>
%<*package>
%\fi
%
% \macro{\ifchilddoc}
% \macro{\ifchilddocmanual}
% The conditional |\ifchilddoc| tells whether a
% child (true) or main (false) document is being compiled.
% The conditional |\ifchilddocmanual| tells whether
% the |\includeonly| mechanism is used (false) or
% the selection of child files must be performed manually (true).
% The definitions initialise to false:
%    \begin{macrocode}
\newif\ifchilddoc
\newif\ifchilddocmanual
%    \end{macrocode}

% \macro{\childdocname}
% \macro{\childdocjob}
% The macro |\childdocname| stores the name of the main document
% to be compiled. The macro |\childdocjob| stores the name of
% the document on which the \LaTeX{} compiler was originally invoked.
% The content of |\jobname| cannot be compared
% to filenames specified in the source due to different catcodes.
% The following code rescans |\jobname|, stores the result
% in |\childdocname| and saves a copy in |\childdocjob|:
%    \begin{macrocode}
\edef\childdocname{\scantokens\expandafter{\jobname\noexpand}}
\let\childdocjob\childdocname
%    \end{macrocode}

% \macro{\childdocdisable}
% The macro |\childdocdisable| prevents the main file
% from being processed more than once.
% At this stage, the main document command |\childdocmain|
% is assumed to be called once again where it should do nothing.
% Any subsequent call to it should prevent
% a secondary processing of the main document
% It overwrites the forwarding commands
% |\childdocof| and |\childdocforward|
% with empty macros to prevent further inclusions of the main document:
%    \begin{macrocode}
\newcommand{\childdocdisable}
{
  \renewcommand{\childdocmain}[1]{\renewcommand{\childdocmain}[1]{\endinput}}
  \renewcommand{\childdocof}[1]{}
  \renewcommand{\childdocby}[2][]{}
  \renewcommand{\childdocforward}[2][]{}
  \renewcommand{\childdocdisable}{}
}
%    \end{macrocode}

% \macro{\childdocmain}
% The macro |\childdocmain| is to be called at the top of the main file
% with nothing or the main filename (without extension) as argument.
% First, it breaks loops.
% If the argument is not empty and does not match |\childdocname|
% (which is set by the first inclusion of |childdoc.def|),
% |\ifchilddoc| is set to true, |\includeonly| is applied to the child file
% and |\jobname| is set to the main file
% (for proper handling of |.aux| files):
%    \begin{macrocode}
\newcommand{\childdocmain}[1]
{
  \childdocdisable\childdocmain{}
  \if?#1?\else
    \begingroup
      \def\childdoctmp{#1}
      \ifx\childdoctmp\childdocname
        \def\childdoctmp{}
      \else
        \def\childdoctmp
        {
          \childdoctrue
          \includeonly{\childdocname}
          \def\childdocjob{#1}
          \def\jobname{#1}
        }
      \fi
      \expandafter
    \endgroup
    \childdoctmp
  \fi
}
%    \end{macrocode}

% \macro{\childdocof}
% The command |\childdocof| redirects
% compilation to the main file |#1|.
%    \begin{macrocode}
\newcommand{\childdocof}[1]
{
  \childdocdisable
  \childdoctrue
  \includeonly{\childdocname}
  \def\jobname{#1}
  \def\childdocjob{#1}
  \input{#1}
}
%    \end{macrocode}

% \macro{\childdocby}
% The command |\childdocby| ....
%    \begin{macrocode}
\newcommand{\childdocby}[2][]
{
  \childdocdisable
  \childdoctrue
  \childdocmanualtrue
  \if?#1?\else
    \def\jobname{#2}
  \fi
  \def\childdocjob{#2}
  \input{#2}
  \endinput
}
%    \end{macrocode}

% \macro{\childdocforward}
% The command |\childdocforward| redirects
% compilation to the main file or
% (if the optional argument is given) a child file.
% Parameters are set as if the main file
% or a child file starting with |\childdocof| was compiled.
% Then compilation is handed over to the main file:
%    \begin{macrocode}
\newcommand{\childdocforward}[2][]
{
  \begingroup
    \if?#1?
      \def\childdoctmp
      {
        \def\childdocname{#2}
        \def\childdocjob{#2}
        \def\jobname{#2}
        \input{#2}
        \endinput
      }
    \else
      \def\childdoctmp
      {
        \childdocdisable
        \def\childdocname{#2}
        \childdoctrue
        \includeonly{#2}
        \def\childdocjob{#1}
        \def\jobname{#1}
        \input{#1}
        \endinput
      }
    \fi
    \expandafter
  \endgroup
  \childdoctmp
}
%    \end{macrocode}

% \macro{\childdocforwardprefix}
% The command |\childdocforwardprefix| redirects
% compilation to the main or a child file by means of a pattern.
% The prefix |#1| in the current filename is replaced by |#2|
% and the suffix of the current filename is kept
% (it is assumed that the filename does not contain the substring `|~~~|'
% which is used as a delimiter).
% Compilation is handed over to the new file by |\childdocforward|:
%    \begin{macrocode}
\newcommand{\childdocforwardprefix}[3][]
{
  \begingroup
    \def\childdocextract #2##1~~~{\def\childdoctmp{\childdocforward[#1]{#3##1}}}
    \expandafter\childdocextract\childdocname~~~
    \expandafter
  \endgroup
  \childdoctmp
}
%    \end{macrocode}

% \macro{\childdoc}
% The deprecated macro |\childdoc| is a legacy version of |\childdocmain|:
%    \begin{macrocode}
\newcommand{\childdoc}{\childdocmain}
%    \end{macrocode}

% \macro{\childdocredirect}
% The deprecated macro |\childdocredirect| is a legacy version
% of |\childdocforward| and |\childdocforwardprefix|:
%    \begin{macrocode}
\newcommand{\childdocredirect}[2][]
{
  \begingroup
    \if?#1?
      \def\childdoctmp{\childdocforward{#2}}
    \else
      \def\childdoctmp{\childdocforwardprefix{#1}{#2}}
    \fi
    \expandafter
  \endgroup
  \childdoctmp
}
%    \end{macrocode}

%\iffalse
%</package>
%\fi
%
\endinput
|\\
|\childdocforwardprefix[|\textit{main}|]{|\textit{prefix}|}{|\textit{dest}|}|
\end{tabular}
\end{center}
%
the destination file is determined by a pattern
depending on the current file:
To make this work, the current file must be called
`{\textit{prefix}\hspace{0.2em}\textit{suffix}}'
with \textit{prefix} matching precisely the argument.
Processing is then passed on to the file
`{\textit{dest}\hspace{0.2em}\textit{suffix}}'.
Surely, the same effect is achieved by
directly specifying the
argument `{\textit{dest}\hspace{0.2em}\textit{suffix}}'
in the first form.
However, that requires to set up a different file
for each child. With the alternative form of the command
all these files can have exactly the same content
which simplifies setting them up and maintaining them.

For example, the following file |draft.tex|
with a compilation flag |\version| as described in \secref{sec:flags}
compiles the main document as a draft:
%
\begin{center}
\begin{tabular}{l}
|\def\version{draft}|\\
|% \iffalse
%
% childdoc.dtx Copyright (C) 2017-2018 Niklas Beisert
%
% This work may be distributed and/or modified under the
% conditions of the LaTeX Project Public License, either version 1.3
% of this license or (at your option) any later version.
% The latest version of this license is in
%   http://www.latex-project.org/lppl.txt
% and version 1.3 or later is part of all distributions of LaTeX
% version 2005/12/01 or later.
%
% This work has the LPPL maintenance status `maintained'.
%
% The Current Maintainer of this work is Niklas Beisert.
%
% This work consists of the files childdoc.dtx and childdoc.ins
% and the derived files childdoc.def and cdocsamp.tex with
% cdocsch1.tex, cdocsch2.tex, cdocsdrf.tex, cdocsfn1.tex, cdocsfn2.tex.
%
%<package>\ifdefined\childdocmain\endinput\fi
%<package>\ProvidesFile{childdoc.def}[2018/12/30 v2.0 child document driver]
%<samplemain>\ProvidesFile{cdocsamp.tex}[2018/12/30 v2.0 sample for childdoc]
%<*driver>
%\ProvidesFile{childdoc.drv}[2018/12/30 v2.0 childdoc reference manual file]
\PassOptionsToClass{10pt,a4paper}{article}
\documentclass{ltxdoc}

\usepackage[margin=35mm]{geometry}
\usepackage{hyperref}
\usepackage{hyperxmp}
\usepackage[usenames]{color}

\hypersetup{colorlinks=true}
\hypersetup{pdfstartview=FitH}
\hypersetup{pdfpagemode=UseNone}
\hypersetup{pdfsource={}}
\hypersetup{pdflang={en-UK}}
\hypersetup{pdfcopyright={Copyright 2017-2018 Niklas Beisert.
  This work may be distributed and/or modified under the
  conditions of the LaTeX Project Public License, either version 1.3
  of this license or (at your option) any later version.}}
\hypersetup{pdflicenseurl={http://www.latex-project.org/lppl.txt}}
\hypersetup{pdfcontactaddress={ETH Zurich, ITP, HIT K,
  Wolfgang-Pauli-Strasse 27}}
\hypersetup{pdfcontactpostcode={8093}}
\hypersetup{pdfcontactcity={Zurich}}
\hypersetup{pdfcontactcountry={Switzerland}}
\hypersetup{pdfcontactemail={nbeisert@itp.phys.ethz.ch}}
\hypersetup{pdfcontacturl={http://people.phys.ethz.ch/\xmptilde nbeisert/}}

\newcommand{\secref}[1]{\hyperref[#1]{section \ref*{#1}}}

\parskip1ex
\parindent0pt
\let\olditemize\itemize
\def\itemize{\olditemize\parskip0pt}

\begin{document}

\title{The \textsf{childdoc} Package}
\hypersetup{pdftitle={The childdoc Package}}
\author{Niklas Beisert\\[2ex]
  Institut f\"ur Theoretische Physik\\
  Eidgen\"ossische Technische Hochschule Z\"urich\\
  Wolfgang-Pauli-Strasse 27, 8093 Z\"urich, Switzerland\\[1ex]
  \href{mailto:nbeisert@itp.phys.ethz.ch}
  {\texttt{nbeisert@itp.phys.ethz.ch}}}
\hypersetup{pdfauthor={Niklas Beisert}}
\hypersetup{pdfsubject={Manual for the LaTeX2e Package childdoc}}
\date{30 December 2018, \textsf{v2.0}}
\maketitle

\begin{abstract}\noindent
\textsf{childdoc} is a \LaTeXe{} package
that enables the direct compilation
of document sections included by |\include|
to individual files.
\end{abstract}

\begingroup
\parskip0ex
\tableofcontents
\endgroup

%%%%%%%%%%%%%%%%%%%%%%%%%%%%%%%%%%%%%%%%%%%%%%%%%%%%%%%%%%%%%%%%%%%%%%%%%%%%%%%%
%%%%%%%%%%%%%%%%%%%%%%%%%%%%%%%%%%%%%%%%%%%%%%%%%%%%%%%%%%%%%%%%%%%%%%%%%%%%%%%%
\section{Introduction}

\LaTeX{} provides a mechanism to structure a large document (such as a book)
into a main file and several child files (containing the chapters)
using the |\include| command.
This mechanism is beneficial for documents
which span hundreds of pages in order to
make the source file(s) more manageable.
Moreover, compilation can be restricted to
selected child files by means of the |\includeonly| command.
The latter feature can be used to reduce the compilation time while editing
(this was significantly more useful in the earlier days of \LaTeX{})
or to generate a smaller document which is easier to navigate.
Another application of |\includeonly| is to generate
documents consisting of selected parts of the complete document.

However, there are a few drawbacks of the plain |\include| mechanism:
\begin{itemize}
\item
The child files cannot be compiled on their own,
they can only be compiled via the main file.
A naive editing environment
(such as a text editor with an option
to have the current file processed by \LaTeX)
may require one to switch to the main file before compiling;
attempting to compile the child file produces errors.
\item
The main file must be modified (each time)
to adjust the |\includeonly| command
to the present needs. This easily leaves the main file in a messy state.
\item
The generated document will always carry the filename
of the main document. This is inconvenient if
several child files are to be compiled and
to be kept for distribution.
\end{itemize}

The present package provides a simple interface
to make child files individually compilable by \LaTeX{}.
Compiling a child file then has the same effect as compiling
the main file with an |\includeonly| command
to select the appropriate child.
Moreover the generated document will carry the name of the child
rather than the main file.
This resolves all three above issues.

This feature is meant to make the editing of books,
thesis documents and lecture notes somewhat more convenient.
However, the package can also be used efficiently for
composing a series of documents (such as exercise sheets)
which are typically distributed individually.
It then assists the author in generating the individual documents
(potentially in different versions)
as well as a document containing the collected series.
Another application is in developing style files
or other kinds of included material
where compilation of the style file could redirect
to a sample or test file.

%%%%%%%%%%%%%%%%%%%%%%%%%%%%%%%%%%%%%%%%%%%%%%%%%%%%%%%%%%%%%%%%%%%%%%%%%%%%%%%%
%%%%%%%%%%%%%%%%%%%%%%%%%%%%%%%%%%%%%%%%%%%%%%%%%%%%%%%%%%%%%%%%%%%%%%%%%%%%%%%%
\section{Usage}

First of all, the package \textsf{childdoc} is \emph{not} a standard
\LaTeXe{} |.sty| style file! Therefore it needs to be invoked in
a non-standard way.

%%%%%%%%%%%%%%%%%%%%%%%%%%%%%%%%%%%%%%%%%%%%%%%%%%%%%%%%%%%%%%%%%%%%%%%%%%%%%%%%
\subsection{Included Files}
\label{sec:include}

%%%%%%%%%%%%%%%%%%%%%%%%%%%%%%%%%%%%%%%%
\DescribeMacro{\childdocmain}
To use the package, add the commands
\begin{center}
\begin{tabular}{l}
|\input{childdoc.def}|\\
|\childdocmain{}|\\
\end{tabular}
\end{center}
at the very top of the main \LaTeX{} file,
in particular \emph{before} the |\documentclass| statement!
The argument of |\childdocmain| should be left empty
(but it must be present).

%%%%%%%%%%%%%%%%%%%%%%%%%%%%%%%%%%%%%%%%
\DescribeMacro{\childdocof}
Furthermore, add the commands
\begin{center}
\begin{tabular}{l}
|\input{childdoc.def}|\\
|\childdocof{|\textit{main}|}|\\
\end{tabular}
\end{center}
at the top of every child file \textit{child}
which is included by |\include{|\textit{child}|}|
from within the main file
(or at least for those files to be compiled individually).
The argument \textit{main} must be the filename of the main file.

There are a couple of
considerations in setting up the main and child documents:

%%%%%%%%%%%%%%%%%%%%%%%%%%%%%%%%%%%%%%%%
\paragraph{Restrictions.}

Please note the following restrictions:
\begin{itemize}
\item
|\childdocmain| must be called with one argument \textit{main}
to ensure compatibility with earlier version of the package.
It must either be empty (|\childdocmain{}|)
or precisely match the filename of the main file in which it is specified.
See \secref{sec:detection} for further information.
\item
The filename \textit{main} must be specified without the |.tex| extension.
\item
The filename \textit{main} is case sensitive
(even in case-insensitive file systems)
due to internal string comparison.
\item
The argument \textit{main} should be fully expanded, it cannot be a macro.
\item
Subdirectories and special characters should be avoided in filenames.
\item
The command |\childdocmain{|\textit{main}|}| must be followed by a whitespace.
It should not be followed immediately by another command
or by a comment mark `|%|'.
This is because the \TeX{} parser reads the token immediately following
the argument of |\childdocmain| and puts it
at the beginning of every child section;
however, a white\-space is ignored.
\end{itemize}

%%%%%%%%%%%%%%%%%%%%%%%%%%%%%%%%%%%%%%%%
\paragraph{Content of Main File.}

It is advisable to place all content in the child files included by |\include|.
Any output contained in the main file will appear in all child documents
unless suppressed manually;
it cannot be suppressed automatically by the |\includeonly| directive
and thus should normally be avoided.
A method to include some content in the main file
by means of conditional processing is described in \secref{sec:conditional}.

%%%%%%%%%%%%%%%%%%%%%%%%%%%%%%%%%%%%%%%%
\paragraph{Page Numbering.}

When only a part of the document is compiled,
the appropriate numbering of pages
(as well as other status parameters)
is determined from the |.aux| files.
The latter contain information from previous passes.
However this information needs to propagate through
all intermediate child documents.
Therefore the page numbering in child documents may well
be inconsistent until the complete document is compiled at least once.

A useful (if unconventional) way to always ensure a consistent
page numbering is to restart the numbering in each child document
and denote the pages by `\textit{child}|.|\textit{page}'
where \textit{child} represents the chapter/section number of the child file.
This can be achieved by the command
|\numberwithin{page}{|\textit{child}|}|
of the \textsf{amsmath} package
where \textit{child} can be |chapter| or |section|
depending on the chosen structuring.
Alternatively, one can modify the macro |\thepage| appropriately
and reset the counter |page| at the start of each child file.

%%%%%%%%%%%%%%%%%%%%%%%%%%%%%%%%%%%%%%%%%%%%%%%%%%%%%%%%%%%%%%%%%%%%%%%%%%%%%%%%
\subsection{Conditional Processing}
\label{sec:conditional}

The package provides a mechanism to compile different versions
of a document. To customise the versions further some conditional processing
can come in handy to distinguish which version is being compiled.
The package provides two macros to describe the compilation context:

%%%%%%%%%%%%%%%%%%%%%%%%%%%%%%%%%%%%%%%%
\DescribeMacro{\ifchilddoc}
The conditional |\ifchilddoc| distinguishes between the compilation of
child documents and the main document:
%
\begin{center}
|\ifchilddoc |\textit{child-code}| |[|\||else |\textit{main-code}]| \||fi|
\end{center}

%%%%%%%%%%%%%%%%%%%%%%%%%%%%%%%%%%%%%%%%
\DescribeMacro{\childdocname}
\DescribeMacro{\childdocjob}
The macro |\childdocname| contains the filename (without extension)
of the main or child file being processed.
Note that |\childdocjob| will always contain the name of the main file.

%%%%%%%%%%%%%%%%%%%%%%%%%%%%%%%%%%%%%%%%
\paragraph{Title Page.}

Conditional processing can be used to include a title or banner page
in the main document when proper precautions are taken.
Importantly, the code in the main file should ensure that the page counter
(as well as other status parameters which are stored in the |.aux| files)
takes the same value after the conditional processing.
Otherwise the page numbers may take divergent values
depending on which part is compiled.

For example, a title page could be declared by:
%
\begin{center}
\begin{tabular}{l}
|\ifchilddoc\||else|\\
|\addtocounter{page}{-1}|\\
\textit{code for title page}\\
|\newpage|\\
|\||fi|
\end{tabular}
\end{center}
%
A banner page for the child documents can be generated by:
%
\begin{center}
\begin{tabular}{l}
|\ifchilddoc|\\
|\addtocounter{page}{-1}|\\
\textit{code for banner page}\\
|\newpage|\\
|\||fi|
\end{tabular}
\end{center}
%
Here one could write a message such as:
\begin{center}
|This is the part \childdocname{} of \childdocjob{}.|
\end{center}

%%%%%%%%%%%%%%%%%%%%%%%%%%%%%%%%%%%%%%%%%%%%%%%%%%%%%%%%%%%%%%%%%%%%%%%%%%%%%%%%
\subsection{Flags}
\label{sec:flags}

The package makes it easy to generate different versions
of the main or child documents.
To this end compilation flags can be defined
and assigned different default values.
They will be particularly useful in conjunction
with the forwarding mechanism described in \secref{sec:forward}.

For example, it may be useful to have a flag |\version|
which can be set to |draft| or |final|.
The document source will contain some conditional code
depending on the value of |\version|.
Suppose further, the flag should default to |final| for the main file
and to |draft| for child files
which is a natural assignment for editing the document.
This is achieved by placing the following code
in the preamble of the main document
(below the |\childdocmain| directive):
%
\begin{center}
\begin{tabular}{l}
|\ifchilddoc|\\
|\providecommand{\version}{draft}|\\
|\||else|\\
|\providecommand{\version}{final}|\\
|\||fi|
\end{tabular}
\end{center}
%
The definition by |\providecommand| makes sure
that previous definitions are not overwritten.
Further statements |\providecommand{\version}{...}|
can thus be added before the above code to override it.

For the main file, one might add a line
(between |\childdocmain| and the above block)
%
\begin{center}
|%\ifchilddoc\||else\providecommand{\version}{draft}\||fi|
\end{center}
%
which can be uncommented to produce a draft version.
Likewise one can add a line to the very top of a child file
(above the |\childdocof{|\textit{main}|}| directive)
%
\begin{center}
|%\providecommand{\version}{final}|
\end{center}
%
which can be uncommented to produce the final version of this child document.

%%%%%%%%%%%%%%%%%%%%%%%%%%%%%%%%%%%%%%%%%%%%%%%%%%%%%%%%%%%%%%%%%%%%%%%%%%%%%%%%
\subsection{Forwarding}
\label{sec:forward}

Different versions of the main or child documents
using compilation flags as described in \secref{sec:flags}
can be (permanently) stored in different files
for convenient compilation, viewing and distribution.
To this end, the package defines a command
to pass on compilation to a different file:

%%%%%%%%%%%%%%%%%%%%%%%%%%%%%%%%%%%%%%%%
\DescribeMacro{\childdocforward}
The command |\childdocforward| redirects processing to
another source file:
%
\begin{center}
\begin{tabular}{l}
|\input{childdoc.def}|\\
|\childdocforward[|\textit{main}|]{|\textit{dest}|}|\\
\end{tabular}
\end{center}
%
The argument \textit{dest} is the destination file
(without extension).
It should be the main file or one of the child files.
Note that further \textsf{childdoc} directives
such as |\childdocof| and |\childdocforward|
in the indicated file will be processed in this form.
The optional argument \textit{main}
passes on directly to the main file \textit{main}
while pretending to compile the child \textit{dest}.
This form behaves as if \textit{dest}
issues |\childdocof{|\textit{main}|}| right away,
and no further \textsf{childdoc} directives will be processed.

%%%%%%%%%%%%%%%%%%%%%%%%%%%%%%%%%%%%%%%%
\DescribeMacro{\...prefix}
In the alternative form |\childdocforwardprefix|,
%
\begin{center}
\begin{tabular}{l}
|\input{childdoc.def}|\\
|\childdocforwardprefix[|\textit{main}|]{|\textit{prefix}|}{|\textit{dest}|}|
\end{tabular}
\end{center}
%
the destination file is determined by a pattern
depending on the current file:
To make this work, the current file must be called
`{\textit{prefix}\hspace{0.2em}\textit{suffix}}'
with \textit{prefix} matching precisely the argument.
Processing is then passed on to the file
`{\textit{dest}\hspace{0.2em}\textit{suffix}}'.
Surely, the same effect is achieved by
directly specifying the
argument `{\textit{dest}\hspace{0.2em}\textit{suffix}}'
in the first form.
However, that requires to set up a different file
for each child. With the alternative form of the command
all these files can have exactly the same content
which simplifies setting them up and maintaining them.

For example, the following file |draft.tex|
with a compilation flag |\version| as described in \secref{sec:flags}
compiles the main document as a draft:
%
\begin{center}
\begin{tabular}{l}
|\def\version{draft}|\\
|\input{childdoc.def}|\\
|\childdocforward{|\textit{main}|}|
\end{tabular}
\end{center}
%
Likewise, the following files |final|\textit{nn}|.tex|
compile the final version of the child document
|child|\textit{nn}|.tex|:
%
\begin{center}
\begin{tabular}{l}
|\def\version{final}|\\
|\input{childdoc.def}|\\
|\childdocforwardprefix{final}{child}|
\end{tabular}
\end{center}
%

Note that when several versions of a main file and/or of each child file
are to be generated, it may be convenient to set up a |Makefile| or
shell script to automatise the process.

%%%%%%%%%%%%%%%%%%%%%%%%%%%%%%%%%%%%%%%%%%%%%%%%%%%%%%%%%%%%%%%%%%%%%%%%%%%%%%%%
\subsection{Command Line Processing}
\label{sec:commandline}

The effect of redirection files can also be achieved by invoking
the \LaTeX{} compiler with a more elaborate command line.
Most conveniently this should be done as part
of a shell script or a |Makefile|.

When using \textsf{childdoc} in the main file, the following
command lines effectively perform a redirection
(note that depending on the shell being used,
backslashes may have to be doubled: `|\|' $\to$ `|\\|'):
%
\begin{center}
|... -jobname "|\textit{target}|" |\\|"|[\textit{flags}]%
|\input{childdoc.def}\childdocforward[|\textit{main}|]{|\textit{dest}|}"|
\end{center}
%
Here \textit{target} is the name of the output file,
\textit{main} is the name of the main file
and \textit{dest} is the name of the main or child file to be processed
(all filenames without extensions).
The optional argument \textit{main} can be omitted
if \textit{main} matches \textit{dest}.
Optionally, compilation \textit{flags} can be defined via |\def| commands.
This command line makes the \TeX{} engine believe
it is compiling the file \textit{target}
whose content is specified as the latter parameter.
The provided code then forwards the processing to
\textit{main} or \textit{dest} as described in \secref{sec:forward}.

%%%%%%%%%%%%%%%%%%%%%%%%%%%%%%%%%%%%%%%%%%%%%%%%%%%%%%%%%%%%%%%%%%%%%%%%%%%%%%%%
\subsection{Include by Input}
\label{sec:input}

Including child documents by |\include| has some restrictions by design.
Most notably, the content of a child document always occupies
its own set of pages; pages cannot be shared between child documents.
Usually, this behaviour makes perfect sense
because each child document contain an essential part of the document.
However, in some situations it may be desirable to compose
a document from a collection of parts
without having mandatory page breaks between then.
For this case, the package
provides a mechanism to include parts
by |\input| which can also be processed individually.
However, by construction this mechanism
requires manual handling of the content to be output.

%%%%%%%%%%%%%%%%%%%%%%%%%%%%%%%%%%%%%%%%
\DescribeMacro{\ifchilddocmanual}
The main file should be prepared as usual, see \secref{sec:include}.
However, the document body must make a distinction
between processing of an individual part and of the main document, e.g.:
%
\begin{center}
\begin{tabular}{l}
|\ifchilddocmanual|\\
|\input{\childdocname}|\\
|\||else|\\
\textit{document body with }|\input{|\textit{part}|}|\\
|\||fi|
\end{tabular}
\end{center}
%
The conditional |\ifchilddocmanual| is true whenever
a part to be included by |\input| is being compiled,
and the name of the part is stored in |\childdocname|.

%%%%%%%%%%%%%%%%%%%%%%%%%%%%%%%%%%%%%%%%
\DescribeMacro{\childdocby}
Each part to be included by |\input| should start with:
%
\begin{center}
\begin{tabular}{l}
|\input{childdoc.def}|\\
|\childdocby{|\textit{main}|}|\\
\end{tabular}
\end{center}
%
The directive |\childdocby| is similar to |\childdocof|
described in \secref{sec:include},
but the subsequent selection of content must be done manually.
To that end, both |\ifchilddoc| and |\ifchilddocmanual|
will be true upon processing of a part,
and the name of the part is stored in |\childdocname|.
Note that |\jobname| will be set to the filename of the current part
so that each part receives an individual |.aux| file
that does not interfere with the |.aux| file(s) of the main document.
This behaviour can be altered by the alternative form
|\childdocby[*]{|\textit{main}|}| (with a non-empty optional argument)
which uses the |.aux| file of the main document
by setting |\jobname| to \textit{main}.

%%%%%%%%%%%%%%%%%%%%%%%%%%%%%%%%%%%%%%%%%%%%%%%%%%%%%%%%%%%%%%%%%%%%%%%%%%%%%%%%
\subsection{Driver Development}
\label{sec:driver}

The \textsf{childdoc} mechanism can also be use for the development
of definition files such as \LaTeX{} styles or classes.
This case differs from the above setup with multiple parts
included by |\include| in that no |\includeonly| should be invoked.
This can be achieved by starting the include file
(before |\ProvidesPackage|) with:
%
\begin{center}
\begin{tabular}{l}
|\input{childdoc.def}|\\
|\childdocforward{|\textit{main}|}|\\
\end{tabular}
\end{center}
%
or alternatively with:
%
\begin{center}
\begin{tabular}{l}
|\input{childdoc.def}|\\
|\childdocby{|\textit{main}|}|\\
\end{tabular}
\end{center}
%
Both forms have slightly different effects as described above.
The main file is prepared as usual, see \secref{sec:include}.

%%%%%%%%%%%%%%%%%%%%%%%%%%%%%%%%%%%%%%%%%%%%%%%%%%%%%%%%%%%%%%%%%%%%%%%%%%%%%%%%
\subsection{Legacy Detection}
\label{sec:detection}

The directive |\childdocmain| in the main file can detect
whether the complete document or merely a child is to be compiled
even without using the directive |\childdocof|.
This method is deprecated because it is less robust
and there is no compelling reason to use it;
it is merely provided for backward compatibility
and it may be removed in future versions.

If the detection mechanism is to be used,
it is mandatory to correctly specify
the filename of the main file as the argument of |\childdocmain|:
%
\begin{center}
\begin{tabular}{l}
|\input{childdoc.def}|\\
|\childdocmain{|\textit{main}|}|\\
\end{tabular}
\end{center}
%
If |\jobname| does not match the argument \textit{main} of |\childdocmain|,
it is assumed that |\jobname| points to the child file to be compiled.
When using |\childdocmain| with the main file specified as argument,
it suffices to start a child file
with just |\input{|\textit{main}|}|
without loading of the package and using |\childdocof|.
If instead all processing is done
with the appropriate \textsf{childdoc} directives,
the argument of \textit{main} of |\childdocmain| can be empty.

An alternative version of the command line processing described
in \secref{sec:commandline} using the detection mechanism reads:
%
\begin{center}
|... -jobname "|\textit{target}|" "|[\textit{flags}]%
[|\def\jobname{|\textit{dest}|}|]|\input{|\textit{main}|}"|
\end{center}

%%%%%%%%%%%%%%%%%%%%%%%%%%%%%%%%%%%%%%%%%%%%%%%%%%%%%%%%%%%%%%%%%%%%%%%%%%%%%%%%
\subsection{Manual Code}
\label{sec:manual}

In case one cannot be certain whether the definitions file |childdoc.def|
is installed on the target \TeX{} distribution
and one prefers not to ship it,
it is conceivable to paste a few relevant commands into the sources.

To that end, drop all statements |\input{childdoc.def}|
and perform the replacements as outlined below.
Instead of |\childdocmain{|\textit{main}|}| add the following code
to the top of the main file:
%
\begin{center}
\begin{tabular}{l}
|\||ifdefined\childdocname\endinput\||fi\newif\ifchilddoc|\\
|\edef\childdocname{\scantokens\expandafter{\jobname\noexpand}}|\\
|\def\childdocmain{|\textit{main}|}\||ifx\childdocmain\childdocname\||else|\\
|\childdoctrue\includeonly{\childdocname}\let\jobname\childdocmain\||fi|\\
\end{tabular}
\end{center}
%
Instead of |\childdocof{|\textit{main}|}| just include the main file
at the top of each child file:
%
\begin{center}
|\input{|\textit{main}|}|
\end{center}
%
A simple redirection |\childdocforward{|\textit{dest}|}| is achieved by:
%
\begin{center}
|\def\jobname{|\textit{dest}|}\input{\jobname}|
\end{center}
%
The redirection with prefix
|\childdocforwardprefix[|\textit{prefix}|]{|\textit{dest}|}|
is accomplished by:
%
\begin{center}
\begin{tabular}{l}
|{\edef\jobname{\scantokens\expandafter{\jobname\noexpand}}|\\
|\def\redirectjob |\textit{prefix}|#1~~~{\gdef\jobname{|\textit{dest}|#1}}|\\
|\expandafter\redirectjob\jobname~~~}\input{\jobname}|
\end{tabular}
\end{center}

In an alternative approach,
child documents can be compiled by a specific command line
without additional code or specific definitions:
%
\begin{center}
|... -jobname "|\textit{target}|" "|[\textit{flags}]%
|\includeonly{|\textit{dest}|}\input{|\textit{main}|}"|
\end{center}
%

%%%%%%%%%%%%%%%%%%%%%%%%%%%%%%%%%%%%%%%%%%%%%%%%%%%%%%%%%%%%%%%%%%%%%%%%%%%%%%%%
%%%%%%%%%%%%%%%%%%%%%%%%%%%%%%%%%%%%%%%%%%%%%%%%%%%%%%%%%%%%%%%%%%%%%%%%%%%%%%%%
\section{Information}

%%%%%%%%%%%%%%%%%%%%%%%%%%%%%%%%%%%%%%%%%%%%%%%%%%%%%%%%%%%%%%%%%%%%%%%%%%%%%%%%
\subsection{Copyright}

Copyright \copyright{} 2017--2018 Niklas Beisert

This work may be distributed and/or modified under the
conditions of the \LaTeX{} Project Public License, either version 1.3
of this license or (at your option) any later version.
The latest version of this license is in
  \url{http://www.latex-project.org/lppl.txt}
and version 1.3 or later is part of all distributions of \LaTeX{}
version 2005/12/01 or later.

This work has the LPPL maintenance status `maintained'.

The Current Maintainer of this work is Niklas Beisert.

This work consists of the files |README.txt|, |childdoc.ins| and |childdoc.dtx|
as well as the derived files |childdoc.def|, |cdocsamp.tex|
with |cdocsch1.tex|, |cdocsch2.tex|, |cdocspt3.tex|, |cdocspt4.tex|,
|cdocsdrf.tex|, |cdocsfn1.tex|, |cdocsfn2.tex|
as well as |childdoc.pdf|.

%%%%%%%%%%%%%%%%%%%%%%%%%%%%%%%%%%%%%%%%%%%%%%%%%%%%%%%%%%%%%%%%%%%%%%%%%%%%%%%%
\subsection{Files and Installation}

The package consists of the files:
%
\begin{center}
\begin{tabular}{ll}
    |README.txt|   & readme file \\
    |childdoc.ins| & installation file \\
    |childdoc.dtx| & source file \\
    |childdoc.def| & definition file \\
    |cdocsamp.tex| & sample main file \\
    |cdocsch1.tex| & sample include file \\
    |cdocsch2.tex| & sample include file \\
    |cdocspt3.tex| & sample part file \\
    |cdocspt4.tex| & sample part file \\
    |cdocsdrf.tex| & sample redirection file \\
    |cdocsfn1.tex| & sample redirection file \\
    |cdocsfn2.tex| & sample redirection file \\
    |childdoc.pdf| & manual
\end{tabular}
\end{center}
%
The distribution consists of the files
|README.txt|, |childdoc.ins| and |childdoc.dtx|.
%
\begin{itemize}
\item
Run (pdf)\LaTeX{} on |childdoc.dtx|
to compile the manual |childdoc.pdf| (this file).
\item
Run \LaTeX{} on |childdoc.ins| to create the definitions file |childdoc.def|
and the sample |cdocsamp.tex| with include files
|cdocsch1.tex|, |cdocsch2.tex|, |cdocspt3.tex|, |cdocspt4.tex|,
|cdocsdrf.tex|, |cdocsfn1.tex|, |cdocsfn2.tex|.
Then copy the file |childdoc.def| to an appropriate directory of your \LaTeX{}
distribution, e.g.\ \textit{texmf-root}|/tex/latex/childdoc|.
\end{itemize}

%%%%%%%%%%%%%%%%%%%%%%%%%%%%%%%%%%%%%%%%%%%%%%%%%%%%%%%%%%%%%%%%%%%%%%%%%%%%%%%%
\subsection{Related CTAN Packages}

There are several other packages which offer a similar functionality:
%
\begin{itemize}
\item
The packages
\href{http://ctan.org/pkg/docmute}{\textsf{docmute}},
\href{http://ctan.org/pkg/includex}{\textsf{includex}} and
\href{http://ctan.org/pkg/standalone}{\textsf{standalone}}
provide commands to include only the document body of
a child file thus allowing both files to be compiled individually.
\item
The packages \href{http://ctan.org/pkg/subdocs}{\textsf{subdocs}}
and \href{http://ctan.org/pkg/subfiles}{\textsf{subfiles}}
provide structures in which the main and child documents can be
encapsulated and allowing them to be compiled individually.
The inclusion mechanism is different from the conventional |\include|.
\item
The package \href{http://ctan.org/pkg/combine}{\textsf{combine}}
is an elaborate solution to combine several documents into one.
\end{itemize}
%
See also the CTAN topic \href{http://ctan.org/topic/subdocs}{\textsf{subdocs}}
for further related packages.
The present package differs from the above solutions in that
a document structure constructed with the conventional |\include| mechanism
just needs two extra commands at the top of every file
such that all constituent files can be compiled individually.

%%%%%%%%%%%%%%%%%%%%%%%%%%%%%%%%%%%%%%%%%%%%%%%%%%%%%%%%%%%%%%%%%%%%%%%%%%%%%%%%
%\subsection{Feature Suggestions}
%
%The following is a list of features which may be useful for future
%versions of this package:
%%
%\begin{itemize}
%\item
%\ldots
%\end{itemize}

%%%%%%%%%%%%%%%%%%%%%%%%%%%%%%%%%%%%%%%%%%%%%%%%%%%%%%%%%%%%%%%%%%%%%%%%%%%%%%%%
\subsection{Revision History}

%%%%%%%%%%%%%%%%%%%%%%%%%%%%%%%%%%%%%%%%
\paragraph{v2.0:} 2018/12/30

\begin{itemize}
\item
immediate forward processing
\item
added |\childdocby| mechanism
\item
manual restructured
\end{itemize}

%%%%%%%%%%%%%%%%%%%%%%%%%%%%%%%%%%%%%%%%
\paragraph{v1.6:} 2018/01/17

\begin{itemize}
\item
application for development of include files
\item
corrections to manual
\end{itemize}

%%%%%%%%%%%%%%%%%%%%%%%%%%%%%%%%%%%%%%%%
\paragraph{v1.5:} 2017/05/21

\begin{itemize}
\item
more complete structuring introduced
\item
|\childdocof| introduced
\item
|\childdoc| renamed to |\childdocmain|
\item
|\childredirect| renamed to |\childdocforward| and |\childdocforwardprefix|
and functionality expanded
\end{itemize}

%%%%%%%%%%%%%%%%%%%%%%%%%%%%%%%%%%%%%%%%
\paragraph{v1.0:} 2017/04/27

\begin{itemize}
\item
manual and install package
\item
first version published on CTAN
\end{itemize}

%%%%%%%%%%%%%%%%%%%%%%%%%%%%%%%%%%%%%%%%
\paragraph{v0.6:} 2017/04/26

\begin{itemize}
\item
redirection mechanism added
\end{itemize}

%%%%%%%%%%%%%%%%%%%%%%%%%%%%%%%%%%%%%%%%
\paragraph{v0.5:} 2017/04/26

\begin{itemize}
\item
functionality in definition file
\end{itemize}


%%%%%%%%%%%%%%%%%%%%%%%%%%%%%%%%%%%%%%%%%%%%%%%%%%%%%%%%%%%%%%%%%%%%%%%%%%%%%%%%
%%%%%%%%%%%%%%%%%%%%%%%%%%%%%%%%%%%%%%%%%%%%%%%%%%%%%%%%%%%%%%%%%%%%%%%%%%%%%%%%
%%%%%%%%%%%%%%%%%%%%%%%%%%%%%%%%%%%%%%%%%%%%%%%%%%%%%%%%%%%%%%%%%%%%%%%%%%%%%%%%
\appendix

\settowidth\MacroIndent{\rmfamily\scriptsize 000\ }

 \DocInput{childdoc.dtx}

\end{document}
%</driver>
% \fi
%
% %%%%%%%%%%%%%%%%%%%%%%%%%%%%%%%%%%%%%%%%%%%%%%%%%%%%%%%%%%%%%%%%%%%%%%%%%%%%%%
% %%%%%%%%%%%%%%%%%%%%%%%%%%%%%%%%%%%%%%%%%%%%%%%%%%%%%%%%%%%%%%%%%%%%%%%%%%%%%%
% \section{Sample}
%\iffalse
%<*samplemain>
%\fi
%
% The following presents a sample document
% with two chapters, two parts, a title page,
% a compile flag as well as three forwarding files to set the flag.
% It consists of eight |.tex| files:
% \begin{center}
% \begin{tabular}{ll}
% |cdocsamp.tex|&main file\\
% |cdocsch1.tex|&include file for chapter 1\\
% |cdocsch2.tex|&include file for chapter 2\\
% |cdocspt3.tex|&include file for part 3\\
% |cdocspt4.tex|&include file for part 4\\
% |cdocsdrf.tex|&forwarding file for main file in draft mode\\
% |cdocsfi1.tex|&forwarding file for final version of chapter 1\\
% |cdocsfi2.tex|&forwarding file for final version of chapter 2\\
% \end{tabular}
% \end{center}
% Each of the eight files can be compiled directly by the \LaTeX{} compiler.
%
% %%%%%%%%%%%%%%%%%%%%%%%%%%%%%%%%%%%%%%
% \paragraph{Main File.}
%
% The main file is called |cdocsamp.tex|.
%
% Load the \textsf{childdoc} definitions and
% declare the filename for the main document:
%    \begin{macrocode}
\input{childdoc.def}
\childdocmain{}
%    \end{macrocode}

% Optional override for |\version| flag:
%    \begin{macrocode}
%%\ifchilddoc\else\providecommand{\version}{draft}\fi
%    \end{macrocode}

% Define the default values for the |\version| flag
% (|final| for the main file and |draft| for childs):
%    \begin{macrocode}
\ifchilddoc
\providecommand{\version}{draft}
\else
\providecommand{\version}{final}
\fi
%    \end{macrocode}

% Load the standard document class:
%    \begin{macrocode}
\documentclass[12pt]{article}
%    \end{macrocode}

% Start the document body:
%    \begin{macrocode}
\begin{document}
%    \end{macrocode}

% Declare a title page.
% Print title, part of document being processed and version flag:
%    \begin{macrocode}
\addtocounter{page}{-1}
\begin{center}
{\LARGE\bfseries{}childdoc example\par}
\vspace{1cm}
\ifchilddoc
\ifchilddocmanual part\else chapter\fi:
`\childdocname' of `\childdocjob'\par
\else
main document: `\childdocjob'\par
\fi
version: \version\par
\end{center}
\newpage
%    \end{macrocode}

% Manually include selected file,
% otherwise process as usual:
%    \begin{macrocode}
\ifchilddocmanual
\section*{part `\childdocname'}
\input{\childdocname}
\else
%    \end{macrocode}

% Include the two chapters:
%    \begin{macrocode}
\include{cdocsch1}
\include{cdocsch2}
%    \end{macrocode}

% Include the two parts unless only chapters should be displayed:
%    \begin{macrocode}
\ifchilddoc\else
\section{part three}
\input{cdocspt3}
\section{part four}
\input{cdocspt4}
\fi
%    \end{macrocode}

% Process as usual until here:
%    \begin{macrocode}
\fi
%    \end{macrocode}

% End of document body:
%    \begin{macrocode}
\end{document}
%    \end{macrocode}
%\iffalse
%</samplemain>
%\fi
%
% %%%%%%%%%%%%%%%%%%%%%%%%%%%%%%%%%%%%%%
% \paragraph{Chapter Include Files.}
%
% The include files are called |cdocsch1.tex| and |cdocsch2.tex|.
%
%\iffalse
%<*samplechap1|samplechap2>
%\fi

% Optional override for |\version| flag:
%    \begin{macrocode}
%%\providecommand{\version}{final}
%    \end{macrocode}

% Include the main document:
%    \begin{macrocode}
\input{childdoc.def}
\childdocof{cdocsamp}
%    \end{macrocode}

%\iffalse
%</samplechap1|samplechap2>
%\fi
%
%\iffalse
%<*samplechap1>
%\fi
% Some text for chapter 1:
%    \begin{macrocode}
\section{one}
some text in chapter one
%    \end{macrocode}

%\iffalse
%</samplechap1>
%\fi
% Some text for chapter 2:
%\iffalse
%<*samplechap2>
%\fi
%    \begin{macrocode}
\section{two}
more text in chapter two
%    \end{macrocode}

%\iffalse
%</samplechap2>
%\fi
%
% %%%%%%%%%%%%%%%%%%%%%%%%%%%%%%%%%%%%%%
% \paragraph{Part Include Files.}
%
% The include files are called |cdocspt3.tex| and |cdocspt4.tex|.
%
%\iffalse
%<*samplepart3|samplepart4>
%\fi

% Optional override for |\version| flag:
%    \begin{macrocode}
%%\providecommand{\version}{final}
%    \end{macrocode}

% Include the main document:
%    \begin{macrocode}
\input{childdoc.def}
\childdocby{cdocsamp}
%    \end{macrocode}

%\iffalse
%</samplepart3|samplepart4>
%\fi
%
%\iffalse
%<*samplepart3>
%\fi
% Some text for part 3:
%    \begin{macrocode}
some text in part three
%    \end{macrocode}

%\iffalse
%</samplepart3>
%\fi
% Some text for part 4:
%\iffalse
%<*samplepart4>
%\fi
%    \begin{macrocode}
more text in part four
%    \end{macrocode}

%\iffalse
%</samplepart4>
%\fi
%
% %%%%%%%%%%%%%%%%%%%%%%%%%%%%%%%%%%%%%%
% \paragraph{Forwarding for a Complete Draft.}
%
% The following forwarding file |cdocsdrf.tex|
% compiles the main document in draft mode:
%\iffalse
%<*sampledraft>
%\fi
%    \begin{macrocode}
\def\version{draft}
\input{childdoc.def}
\childdocforward{cdocsamp}
%    \end{macrocode}

%\iffalse
%</sampledraft>
%\fi
%
% %%%%%%%%%%%%%%%%%%%%%%%%%%%%%%%%%%%%%%
% \paragraph{Forwarding for Final Version of the Chapters.}
%
% The following forwarding files |cdocsfn1.tex| and |cdocsfn2.tex|
% (with identical content)
% compile the final versions of the child documents
% |cdocsch1.tex| and |cdocsch2.tex|, respectively:
%\iffalse
%<*samplefinal>
%\fi
%    \begin{macrocode}
\def\version{final}
\input{childdoc.def}
\childdocforwardprefix[cdocsamp]{cdocsfn}{cdocsch}
%    \end{macrocode}

%\iffalse
%</samplefinal>
%\fi
%
% %%%%%%%%%%%%%%%%%%%%%%%%%%%%%%%%%%%%%%
% \paragraph{Command Line Processing.}
%
% The following three command lines generate the output files
% |cdocscld|, |cdocscl1| and |cdocscl2|
% which should be identical to
% |cdocsdrf|, |cdocsch1| and |cdocsfn2|, respectively:
% \begin{center}
% \begin{tabular}{l}
% |latex -jobname cdocscld \|\\
% |  "\def\version{draft}\input{childdoc.def}\childdocforward{cdocsamp}"|\\
% |latex -jobname cdocscl1 \|\\
% |  "\input{childdoc.def}\childdocforward[cdocsamp]{cdocsch1}"|\\
% |latex -jobname cdocscl2 \|\\
% |  "\def\version{final}\input{childdoc.def}\childdocforward{cdocsch2}"|
% \end{tabular}
% \end{center}
% Note that the trailing backslash on each first line
% merely continues the input to the second line
% (for convenient cut ant paste).
% Furthermore, the command |latex| can be replaced by any
% of its alternative versions such as |pdflatex|.
%
% %%%%%%%%%%%%%%%%%%%%%%%%%%%%%%%%%%%%%%%%%%%%%%%%%%%%%%%%%%%%%%%%%%%%%%%%%%%%%%
% %%%%%%%%%%%%%%%%%%%%%%%%%%%%%%%%%%%%%%%%%%%%%%%%%%%%%%%%%%%%%%%%%%%%%%%%%%%%%%
% \section{Implementation}
%\iffalse
%<*package>
%\fi
%
% This section describes the definitions file |childdoc.def|.

% The definitions cannot be loaded using |\usepackage| or |\RequirePackage|
% which has a mechanism to prevent loading a style file more than once.
% When loading the definitions by means of |\input|
% multiple instances have to be prevented manually:
%\iffalse
%This code needs to be before the `\ProvidesFile' directive
%which is defined at the beginning of this file.
%Therefore it is also placed there and commented out here.
%</package>
%<*discard>
%\fi
%    \begin{macrocode}
\ifdefined\childdocmain\endinput\fi
%    \end{macrocode}
%\iffalse
%</discard>
%<*package>
%\fi
%
% \macro{\ifchilddoc}
% \macro{\ifchilddocmanual}
% The conditional |\ifchilddoc| tells whether a
% child (true) or main (false) document is being compiled.
% The conditional |\ifchilddocmanual| tells whether
% the |\includeonly| mechanism is used (false) or
% the selection of child files must be performed manually (true).
% The definitions initialise to false:
%    \begin{macrocode}
\newif\ifchilddoc
\newif\ifchilddocmanual
%    \end{macrocode}

% \macro{\childdocname}
% \macro{\childdocjob}
% The macro |\childdocname| stores the name of the main document
% to be compiled. The macro |\childdocjob| stores the name of
% the document on which the \LaTeX{} compiler was originally invoked.
% The content of |\jobname| cannot be compared
% to filenames specified in the source due to different catcodes.
% The following code rescans |\jobname|, stores the result
% in |\childdocname| and saves a copy in |\childdocjob|:
%    \begin{macrocode}
\edef\childdocname{\scantokens\expandafter{\jobname\noexpand}}
\let\childdocjob\childdocname
%    \end{macrocode}

% \macro{\childdocdisable}
% The macro |\childdocdisable| prevents the main file
% from being processed more than once.
% At this stage, the main document command |\childdocmain|
% is assumed to be called once again where it should do nothing.
% Any subsequent call to it should prevent
% a secondary processing of the main document
% It overwrites the forwarding commands
% |\childdocof| and |\childdocforward|
% with empty macros to prevent further inclusions of the main document:
%    \begin{macrocode}
\newcommand{\childdocdisable}
{
  \renewcommand{\childdocmain}[1]{\renewcommand{\childdocmain}[1]{\endinput}}
  \renewcommand{\childdocof}[1]{}
  \renewcommand{\childdocby}[2][]{}
  \renewcommand{\childdocforward}[2][]{}
  \renewcommand{\childdocdisable}{}
}
%    \end{macrocode}

% \macro{\childdocmain}
% The macro |\childdocmain| is to be called at the top of the main file
% with nothing or the main filename (without extension) as argument.
% First, it breaks loops.
% If the argument is not empty and does not match |\childdocname|
% (which is set by the first inclusion of |childdoc.def|),
% |\ifchilddoc| is set to true, |\includeonly| is applied to the child file
% and |\jobname| is set to the main file
% (for proper handling of |.aux| files):
%    \begin{macrocode}
\newcommand{\childdocmain}[1]
{
  \childdocdisable\childdocmain{}
  \if?#1?\else
    \begingroup
      \def\childdoctmp{#1}
      \ifx\childdoctmp\childdocname
        \def\childdoctmp{}
      \else
        \def\childdoctmp
        {
          \childdoctrue
          \includeonly{\childdocname}
          \def\childdocjob{#1}
          \def\jobname{#1}
        }
      \fi
      \expandafter
    \endgroup
    \childdoctmp
  \fi
}
%    \end{macrocode}

% \macro{\childdocof}
% The command |\childdocof| redirects
% compilation to the main file |#1|.
%    \begin{macrocode}
\newcommand{\childdocof}[1]
{
  \childdocdisable
  \childdoctrue
  \includeonly{\childdocname}
  \def\jobname{#1}
  \def\childdocjob{#1}
  \input{#1}
}
%    \end{macrocode}

% \macro{\childdocby}
% The command |\childdocby| ....
%    \begin{macrocode}
\newcommand{\childdocby}[2][]
{
  \childdocdisable
  \childdoctrue
  \childdocmanualtrue
  \if?#1?\else
    \def\jobname{#2}
  \fi
  \def\childdocjob{#2}
  \input{#2}
  \endinput
}
%    \end{macrocode}

% \macro{\childdocforward}
% The command |\childdocforward| redirects
% compilation to the main file or
% (if the optional argument is given) a child file.
% Parameters are set as if the main file
% or a child file starting with |\childdocof| was compiled.
% Then compilation is handed over to the main file:
%    \begin{macrocode}
\newcommand{\childdocforward}[2][]
{
  \begingroup
    \if?#1?
      \def\childdoctmp
      {
        \def\childdocname{#2}
        \def\childdocjob{#2}
        \def\jobname{#2}
        \input{#2}
        \endinput
      }
    \else
      \def\childdoctmp
      {
        \childdocdisable
        \def\childdocname{#2}
        \childdoctrue
        \includeonly{#2}
        \def\childdocjob{#1}
        \def\jobname{#1}
        \input{#1}
        \endinput
      }
    \fi
    \expandafter
  \endgroup
  \childdoctmp
}
%    \end{macrocode}

% \macro{\childdocforwardprefix}
% The command |\childdocforwardprefix| redirects
% compilation to the main or a child file by means of a pattern.
% The prefix |#1| in the current filename is replaced by |#2|
% and the suffix of the current filename is kept
% (it is assumed that the filename does not contain the substring `|~~~|'
% which is used as a delimiter).
% Compilation is handed over to the new file by |\childdocforward|:
%    \begin{macrocode}
\newcommand{\childdocforwardprefix}[3][]
{
  \begingroup
    \def\childdocextract #2##1~~~{\def\childdoctmp{\childdocforward[#1]{#3##1}}}
    \expandafter\childdocextract\childdocname~~~
    \expandafter
  \endgroup
  \childdoctmp
}
%    \end{macrocode}

% \macro{\childdoc}
% The deprecated macro |\childdoc| is a legacy version of |\childdocmain|:
%    \begin{macrocode}
\newcommand{\childdoc}{\childdocmain}
%    \end{macrocode}

% \macro{\childdocredirect}
% The deprecated macro |\childdocredirect| is a legacy version
% of |\childdocforward| and |\childdocforwardprefix|:
%    \begin{macrocode}
\newcommand{\childdocredirect}[2][]
{
  \begingroup
    \if?#1?
      \def\childdoctmp{\childdocforward{#2}}
    \else
      \def\childdoctmp{\childdocforwardprefix{#1}{#2}}
    \fi
    \expandafter
  \endgroup
  \childdoctmp
}
%    \end{macrocode}

%\iffalse
%</package>
%\fi
%
\endinput
|\\
|\childdocforward{|\textit{main}|}|
\end{tabular}
\end{center}
%
Likewise, the following files |final|\textit{nn}|.tex|
compile the final version of the child document
|child|\textit{nn}|.tex|:
%
\begin{center}
\begin{tabular}{l}
|\def\version{final}|\\
|% \iffalse
%
% childdoc.dtx Copyright (C) 2017-2018 Niklas Beisert
%
% This work may be distributed and/or modified under the
% conditions of the LaTeX Project Public License, either version 1.3
% of this license or (at your option) any later version.
% The latest version of this license is in
%   http://www.latex-project.org/lppl.txt
% and version 1.3 or later is part of all distributions of LaTeX
% version 2005/12/01 or later.
%
% This work has the LPPL maintenance status `maintained'.
%
% The Current Maintainer of this work is Niklas Beisert.
%
% This work consists of the files childdoc.dtx and childdoc.ins
% and the derived files childdoc.def and cdocsamp.tex with
% cdocsch1.tex, cdocsch2.tex, cdocsdrf.tex, cdocsfn1.tex, cdocsfn2.tex.
%
%<package>\ifdefined\childdocmain\endinput\fi
%<package>\ProvidesFile{childdoc.def}[2018/12/30 v2.0 child document driver]
%<samplemain>\ProvidesFile{cdocsamp.tex}[2018/12/30 v2.0 sample for childdoc]
%<*driver>
%\ProvidesFile{childdoc.drv}[2018/12/30 v2.0 childdoc reference manual file]
\PassOptionsToClass{10pt,a4paper}{article}
\documentclass{ltxdoc}

\usepackage[margin=35mm]{geometry}
\usepackage{hyperref}
\usepackage{hyperxmp}
\usepackage[usenames]{color}

\hypersetup{colorlinks=true}
\hypersetup{pdfstartview=FitH}
\hypersetup{pdfpagemode=UseNone}
\hypersetup{pdfsource={}}
\hypersetup{pdflang={en-UK}}
\hypersetup{pdfcopyright={Copyright 2017-2018 Niklas Beisert.
  This work may be distributed and/or modified under the
  conditions of the LaTeX Project Public License, either version 1.3
  of this license or (at your option) any later version.}}
\hypersetup{pdflicenseurl={http://www.latex-project.org/lppl.txt}}
\hypersetup{pdfcontactaddress={ETH Zurich, ITP, HIT K,
  Wolfgang-Pauli-Strasse 27}}
\hypersetup{pdfcontactpostcode={8093}}
\hypersetup{pdfcontactcity={Zurich}}
\hypersetup{pdfcontactcountry={Switzerland}}
\hypersetup{pdfcontactemail={nbeisert@itp.phys.ethz.ch}}
\hypersetup{pdfcontacturl={http://people.phys.ethz.ch/\xmptilde nbeisert/}}

\newcommand{\secref}[1]{\hyperref[#1]{section \ref*{#1}}}

\parskip1ex
\parindent0pt
\let\olditemize\itemize
\def\itemize{\olditemize\parskip0pt}

\begin{document}

\title{The \textsf{childdoc} Package}
\hypersetup{pdftitle={The childdoc Package}}
\author{Niklas Beisert\\[2ex]
  Institut f\"ur Theoretische Physik\\
  Eidgen\"ossische Technische Hochschule Z\"urich\\
  Wolfgang-Pauli-Strasse 27, 8093 Z\"urich, Switzerland\\[1ex]
  \href{mailto:nbeisert@itp.phys.ethz.ch}
  {\texttt{nbeisert@itp.phys.ethz.ch}}}
\hypersetup{pdfauthor={Niklas Beisert}}
\hypersetup{pdfsubject={Manual for the LaTeX2e Package childdoc}}
\date{30 December 2018, \textsf{v2.0}}
\maketitle

\begin{abstract}\noindent
\textsf{childdoc} is a \LaTeXe{} package
that enables the direct compilation
of document sections included by |\include|
to individual files.
\end{abstract}

\begingroup
\parskip0ex
\tableofcontents
\endgroup

%%%%%%%%%%%%%%%%%%%%%%%%%%%%%%%%%%%%%%%%%%%%%%%%%%%%%%%%%%%%%%%%%%%%%%%%%%%%%%%%
%%%%%%%%%%%%%%%%%%%%%%%%%%%%%%%%%%%%%%%%%%%%%%%%%%%%%%%%%%%%%%%%%%%%%%%%%%%%%%%%
\section{Introduction}

\LaTeX{} provides a mechanism to structure a large document (such as a book)
into a main file and several child files (containing the chapters)
using the |\include| command.
This mechanism is beneficial for documents
which span hundreds of pages in order to
make the source file(s) more manageable.
Moreover, compilation can be restricted to
selected child files by means of the |\includeonly| command.
The latter feature can be used to reduce the compilation time while editing
(this was significantly more useful in the earlier days of \LaTeX{})
or to generate a smaller document which is easier to navigate.
Another application of |\includeonly| is to generate
documents consisting of selected parts of the complete document.

However, there are a few drawbacks of the plain |\include| mechanism:
\begin{itemize}
\item
The child files cannot be compiled on their own,
they can only be compiled via the main file.
A naive editing environment
(such as a text editor with an option
to have the current file processed by \LaTeX)
may require one to switch to the main file before compiling;
attempting to compile the child file produces errors.
\item
The main file must be modified (each time)
to adjust the |\includeonly| command
to the present needs. This easily leaves the main file in a messy state.
\item
The generated document will always carry the filename
of the main document. This is inconvenient if
several child files are to be compiled and
to be kept for distribution.
\end{itemize}

The present package provides a simple interface
to make child files individually compilable by \LaTeX{}.
Compiling a child file then has the same effect as compiling
the main file with an |\includeonly| command
to select the appropriate child.
Moreover the generated document will carry the name of the child
rather than the main file.
This resolves all three above issues.

This feature is meant to make the editing of books,
thesis documents and lecture notes somewhat more convenient.
However, the package can also be used efficiently for
composing a series of documents (such as exercise sheets)
which are typically distributed individually.
It then assists the author in generating the individual documents
(potentially in different versions)
as well as a document containing the collected series.
Another application is in developing style files
or other kinds of included material
where compilation of the style file could redirect
to a sample or test file.

%%%%%%%%%%%%%%%%%%%%%%%%%%%%%%%%%%%%%%%%%%%%%%%%%%%%%%%%%%%%%%%%%%%%%%%%%%%%%%%%
%%%%%%%%%%%%%%%%%%%%%%%%%%%%%%%%%%%%%%%%%%%%%%%%%%%%%%%%%%%%%%%%%%%%%%%%%%%%%%%%
\section{Usage}

First of all, the package \textsf{childdoc} is \emph{not} a standard
\LaTeXe{} |.sty| style file! Therefore it needs to be invoked in
a non-standard way.

%%%%%%%%%%%%%%%%%%%%%%%%%%%%%%%%%%%%%%%%%%%%%%%%%%%%%%%%%%%%%%%%%%%%%%%%%%%%%%%%
\subsection{Included Files}
\label{sec:include}

%%%%%%%%%%%%%%%%%%%%%%%%%%%%%%%%%%%%%%%%
\DescribeMacro{\childdocmain}
To use the package, add the commands
\begin{center}
\begin{tabular}{l}
|\input{childdoc.def}|\\
|\childdocmain{}|\\
\end{tabular}
\end{center}
at the very top of the main \LaTeX{} file,
in particular \emph{before} the |\documentclass| statement!
The argument of |\childdocmain| should be left empty
(but it must be present).

%%%%%%%%%%%%%%%%%%%%%%%%%%%%%%%%%%%%%%%%
\DescribeMacro{\childdocof}
Furthermore, add the commands
\begin{center}
\begin{tabular}{l}
|\input{childdoc.def}|\\
|\childdocof{|\textit{main}|}|\\
\end{tabular}
\end{center}
at the top of every child file \textit{child}
which is included by |\include{|\textit{child}|}|
from within the main file
(or at least for those files to be compiled individually).
The argument \textit{main} must be the filename of the main file.

There are a couple of
considerations in setting up the main and child documents:

%%%%%%%%%%%%%%%%%%%%%%%%%%%%%%%%%%%%%%%%
\paragraph{Restrictions.}

Please note the following restrictions:
\begin{itemize}
\item
|\childdocmain| must be called with one argument \textit{main}
to ensure compatibility with earlier version of the package.
It must either be empty (|\childdocmain{}|)
or precisely match the filename of the main file in which it is specified.
See \secref{sec:detection} for further information.
\item
The filename \textit{main} must be specified without the |.tex| extension.
\item
The filename \textit{main} is case sensitive
(even in case-insensitive file systems)
due to internal string comparison.
\item
The argument \textit{main} should be fully expanded, it cannot be a macro.
\item
Subdirectories and special characters should be avoided in filenames.
\item
The command |\childdocmain{|\textit{main}|}| must be followed by a whitespace.
It should not be followed immediately by another command
or by a comment mark `|%|'.
This is because the \TeX{} parser reads the token immediately following
the argument of |\childdocmain| and puts it
at the beginning of every child section;
however, a white\-space is ignored.
\end{itemize}

%%%%%%%%%%%%%%%%%%%%%%%%%%%%%%%%%%%%%%%%
\paragraph{Content of Main File.}

It is advisable to place all content in the child files included by |\include|.
Any output contained in the main file will appear in all child documents
unless suppressed manually;
it cannot be suppressed automatically by the |\includeonly| directive
and thus should normally be avoided.
A method to include some content in the main file
by means of conditional processing is described in \secref{sec:conditional}.

%%%%%%%%%%%%%%%%%%%%%%%%%%%%%%%%%%%%%%%%
\paragraph{Page Numbering.}

When only a part of the document is compiled,
the appropriate numbering of pages
(as well as other status parameters)
is determined from the |.aux| files.
The latter contain information from previous passes.
However this information needs to propagate through
all intermediate child documents.
Therefore the page numbering in child documents may well
be inconsistent until the complete document is compiled at least once.

A useful (if unconventional) way to always ensure a consistent
page numbering is to restart the numbering in each child document
and denote the pages by `\textit{child}|.|\textit{page}'
where \textit{child} represents the chapter/section number of the child file.
This can be achieved by the command
|\numberwithin{page}{|\textit{child}|}|
of the \textsf{amsmath} package
where \textit{child} can be |chapter| or |section|
depending on the chosen structuring.
Alternatively, one can modify the macro |\thepage| appropriately
and reset the counter |page| at the start of each child file.

%%%%%%%%%%%%%%%%%%%%%%%%%%%%%%%%%%%%%%%%%%%%%%%%%%%%%%%%%%%%%%%%%%%%%%%%%%%%%%%%
\subsection{Conditional Processing}
\label{sec:conditional}

The package provides a mechanism to compile different versions
of a document. To customise the versions further some conditional processing
can come in handy to distinguish which version is being compiled.
The package provides two macros to describe the compilation context:

%%%%%%%%%%%%%%%%%%%%%%%%%%%%%%%%%%%%%%%%
\DescribeMacro{\ifchilddoc}
The conditional |\ifchilddoc| distinguishes between the compilation of
child documents and the main document:
%
\begin{center}
|\ifchilddoc |\textit{child-code}| |[|\||else |\textit{main-code}]| \||fi|
\end{center}

%%%%%%%%%%%%%%%%%%%%%%%%%%%%%%%%%%%%%%%%
\DescribeMacro{\childdocname}
\DescribeMacro{\childdocjob}
The macro |\childdocname| contains the filename (without extension)
of the main or child file being processed.
Note that |\childdocjob| will always contain the name of the main file.

%%%%%%%%%%%%%%%%%%%%%%%%%%%%%%%%%%%%%%%%
\paragraph{Title Page.}

Conditional processing can be used to include a title or banner page
in the main document when proper precautions are taken.
Importantly, the code in the main file should ensure that the page counter
(as well as other status parameters which are stored in the |.aux| files)
takes the same value after the conditional processing.
Otherwise the page numbers may take divergent values
depending on which part is compiled.

For example, a title page could be declared by:
%
\begin{center}
\begin{tabular}{l}
|\ifchilddoc\||else|\\
|\addtocounter{page}{-1}|\\
\textit{code for title page}\\
|\newpage|\\
|\||fi|
\end{tabular}
\end{center}
%
A banner page for the child documents can be generated by:
%
\begin{center}
\begin{tabular}{l}
|\ifchilddoc|\\
|\addtocounter{page}{-1}|\\
\textit{code for banner page}\\
|\newpage|\\
|\||fi|
\end{tabular}
\end{center}
%
Here one could write a message such as:
\begin{center}
|This is the part \childdocname{} of \childdocjob{}.|
\end{center}

%%%%%%%%%%%%%%%%%%%%%%%%%%%%%%%%%%%%%%%%%%%%%%%%%%%%%%%%%%%%%%%%%%%%%%%%%%%%%%%%
\subsection{Flags}
\label{sec:flags}

The package makes it easy to generate different versions
of the main or child documents.
To this end compilation flags can be defined
and assigned different default values.
They will be particularly useful in conjunction
with the forwarding mechanism described in \secref{sec:forward}.

For example, it may be useful to have a flag |\version|
which can be set to |draft| or |final|.
The document source will contain some conditional code
depending on the value of |\version|.
Suppose further, the flag should default to |final| for the main file
and to |draft| for child files
which is a natural assignment for editing the document.
This is achieved by placing the following code
in the preamble of the main document
(below the |\childdocmain| directive):
%
\begin{center}
\begin{tabular}{l}
|\ifchilddoc|\\
|\providecommand{\version}{draft}|\\
|\||else|\\
|\providecommand{\version}{final}|\\
|\||fi|
\end{tabular}
\end{center}
%
The definition by |\providecommand| makes sure
that previous definitions are not overwritten.
Further statements |\providecommand{\version}{...}|
can thus be added before the above code to override it.

For the main file, one might add a line
(between |\childdocmain| and the above block)
%
\begin{center}
|%\ifchilddoc\||else\providecommand{\version}{draft}\||fi|
\end{center}
%
which can be uncommented to produce a draft version.
Likewise one can add a line to the very top of a child file
(above the |\childdocof{|\textit{main}|}| directive)
%
\begin{center}
|%\providecommand{\version}{final}|
\end{center}
%
which can be uncommented to produce the final version of this child document.

%%%%%%%%%%%%%%%%%%%%%%%%%%%%%%%%%%%%%%%%%%%%%%%%%%%%%%%%%%%%%%%%%%%%%%%%%%%%%%%%
\subsection{Forwarding}
\label{sec:forward}

Different versions of the main or child documents
using compilation flags as described in \secref{sec:flags}
can be (permanently) stored in different files
for convenient compilation, viewing and distribution.
To this end, the package defines a command
to pass on compilation to a different file:

%%%%%%%%%%%%%%%%%%%%%%%%%%%%%%%%%%%%%%%%
\DescribeMacro{\childdocforward}
The command |\childdocforward| redirects processing to
another source file:
%
\begin{center}
\begin{tabular}{l}
|\input{childdoc.def}|\\
|\childdocforward[|\textit{main}|]{|\textit{dest}|}|\\
\end{tabular}
\end{center}
%
The argument \textit{dest} is the destination file
(without extension).
It should be the main file or one of the child files.
Note that further \textsf{childdoc} directives
such as |\childdocof| and |\childdocforward|
in the indicated file will be processed in this form.
The optional argument \textit{main}
passes on directly to the main file \textit{main}
while pretending to compile the child \textit{dest}.
This form behaves as if \textit{dest}
issues |\childdocof{|\textit{main}|}| right away,
and no further \textsf{childdoc} directives will be processed.

%%%%%%%%%%%%%%%%%%%%%%%%%%%%%%%%%%%%%%%%
\DescribeMacro{\...prefix}
In the alternative form |\childdocforwardprefix|,
%
\begin{center}
\begin{tabular}{l}
|\input{childdoc.def}|\\
|\childdocforwardprefix[|\textit{main}|]{|\textit{prefix}|}{|\textit{dest}|}|
\end{tabular}
\end{center}
%
the destination file is determined by a pattern
depending on the current file:
To make this work, the current file must be called
`{\textit{prefix}\hspace{0.2em}\textit{suffix}}'
with \textit{prefix} matching precisely the argument.
Processing is then passed on to the file
`{\textit{dest}\hspace{0.2em}\textit{suffix}}'.
Surely, the same effect is achieved by
directly specifying the
argument `{\textit{dest}\hspace{0.2em}\textit{suffix}}'
in the first form.
However, that requires to set up a different file
for each child. With the alternative form of the command
all these files can have exactly the same content
which simplifies setting them up and maintaining them.

For example, the following file |draft.tex|
with a compilation flag |\version| as described in \secref{sec:flags}
compiles the main document as a draft:
%
\begin{center}
\begin{tabular}{l}
|\def\version{draft}|\\
|\input{childdoc.def}|\\
|\childdocforward{|\textit{main}|}|
\end{tabular}
\end{center}
%
Likewise, the following files |final|\textit{nn}|.tex|
compile the final version of the child document
|child|\textit{nn}|.tex|:
%
\begin{center}
\begin{tabular}{l}
|\def\version{final}|\\
|\input{childdoc.def}|\\
|\childdocforwardprefix{final}{child}|
\end{tabular}
\end{center}
%

Note that when several versions of a main file and/or of each child file
are to be generated, it may be convenient to set up a |Makefile| or
shell script to automatise the process.

%%%%%%%%%%%%%%%%%%%%%%%%%%%%%%%%%%%%%%%%%%%%%%%%%%%%%%%%%%%%%%%%%%%%%%%%%%%%%%%%
\subsection{Command Line Processing}
\label{sec:commandline}

The effect of redirection files can also be achieved by invoking
the \LaTeX{} compiler with a more elaborate command line.
Most conveniently this should be done as part
of a shell script or a |Makefile|.

When using \textsf{childdoc} in the main file, the following
command lines effectively perform a redirection
(note that depending on the shell being used,
backslashes may have to be doubled: `|\|' $\to$ `|\\|'):
%
\begin{center}
|... -jobname "|\textit{target}|" |\\|"|[\textit{flags}]%
|\input{childdoc.def}\childdocforward[|\textit{main}|]{|\textit{dest}|}"|
\end{center}
%
Here \textit{target} is the name of the output file,
\textit{main} is the name of the main file
and \textit{dest} is the name of the main or child file to be processed
(all filenames without extensions).
The optional argument \textit{main} can be omitted
if \textit{main} matches \textit{dest}.
Optionally, compilation \textit{flags} can be defined via |\def| commands.
This command line makes the \TeX{} engine believe
it is compiling the file \textit{target}
whose content is specified as the latter parameter.
The provided code then forwards the processing to
\textit{main} or \textit{dest} as described in \secref{sec:forward}.

%%%%%%%%%%%%%%%%%%%%%%%%%%%%%%%%%%%%%%%%%%%%%%%%%%%%%%%%%%%%%%%%%%%%%%%%%%%%%%%%
\subsection{Include by Input}
\label{sec:input}

Including child documents by |\include| has some restrictions by design.
Most notably, the content of a child document always occupies
its own set of pages; pages cannot be shared between child documents.
Usually, this behaviour makes perfect sense
because each child document contain an essential part of the document.
However, in some situations it may be desirable to compose
a document from a collection of parts
without having mandatory page breaks between then.
For this case, the package
provides a mechanism to include parts
by |\input| which can also be processed individually.
However, by construction this mechanism
requires manual handling of the content to be output.

%%%%%%%%%%%%%%%%%%%%%%%%%%%%%%%%%%%%%%%%
\DescribeMacro{\ifchilddocmanual}
The main file should be prepared as usual, see \secref{sec:include}.
However, the document body must make a distinction
between processing of an individual part and of the main document, e.g.:
%
\begin{center}
\begin{tabular}{l}
|\ifchilddocmanual|\\
|\input{\childdocname}|\\
|\||else|\\
\textit{document body with }|\input{|\textit{part}|}|\\
|\||fi|
\end{tabular}
\end{center}
%
The conditional |\ifchilddocmanual| is true whenever
a part to be included by |\input| is being compiled,
and the name of the part is stored in |\childdocname|.

%%%%%%%%%%%%%%%%%%%%%%%%%%%%%%%%%%%%%%%%
\DescribeMacro{\childdocby}
Each part to be included by |\input| should start with:
%
\begin{center}
\begin{tabular}{l}
|\input{childdoc.def}|\\
|\childdocby{|\textit{main}|}|\\
\end{tabular}
\end{center}
%
The directive |\childdocby| is similar to |\childdocof|
described in \secref{sec:include},
but the subsequent selection of content must be done manually.
To that end, both |\ifchilddoc| and |\ifchilddocmanual|
will be true upon processing of a part,
and the name of the part is stored in |\childdocname|.
Note that |\jobname| will be set to the filename of the current part
so that each part receives an individual |.aux| file
that does not interfere with the |.aux| file(s) of the main document.
This behaviour can be altered by the alternative form
|\childdocby[*]{|\textit{main}|}| (with a non-empty optional argument)
which uses the |.aux| file of the main document
by setting |\jobname| to \textit{main}.

%%%%%%%%%%%%%%%%%%%%%%%%%%%%%%%%%%%%%%%%%%%%%%%%%%%%%%%%%%%%%%%%%%%%%%%%%%%%%%%%
\subsection{Driver Development}
\label{sec:driver}

The \textsf{childdoc} mechanism can also be use for the development
of definition files such as \LaTeX{} styles or classes.
This case differs from the above setup with multiple parts
included by |\include| in that no |\includeonly| should be invoked.
This can be achieved by starting the include file
(before |\ProvidesPackage|) with:
%
\begin{center}
\begin{tabular}{l}
|\input{childdoc.def}|\\
|\childdocforward{|\textit{main}|}|\\
\end{tabular}
\end{center}
%
or alternatively with:
%
\begin{center}
\begin{tabular}{l}
|\input{childdoc.def}|\\
|\childdocby{|\textit{main}|}|\\
\end{tabular}
\end{center}
%
Both forms have slightly different effects as described above.
The main file is prepared as usual, see \secref{sec:include}.

%%%%%%%%%%%%%%%%%%%%%%%%%%%%%%%%%%%%%%%%%%%%%%%%%%%%%%%%%%%%%%%%%%%%%%%%%%%%%%%%
\subsection{Legacy Detection}
\label{sec:detection}

The directive |\childdocmain| in the main file can detect
whether the complete document or merely a child is to be compiled
even without using the directive |\childdocof|.
This method is deprecated because it is less robust
and there is no compelling reason to use it;
it is merely provided for backward compatibility
and it may be removed in future versions.

If the detection mechanism is to be used,
it is mandatory to correctly specify
the filename of the main file as the argument of |\childdocmain|:
%
\begin{center}
\begin{tabular}{l}
|\input{childdoc.def}|\\
|\childdocmain{|\textit{main}|}|\\
\end{tabular}
\end{center}
%
If |\jobname| does not match the argument \textit{main} of |\childdocmain|,
it is assumed that |\jobname| points to the child file to be compiled.
When using |\childdocmain| with the main file specified as argument,
it suffices to start a child file
with just |\input{|\textit{main}|}|
without loading of the package and using |\childdocof|.
If instead all processing is done
with the appropriate \textsf{childdoc} directives,
the argument of \textit{main} of |\childdocmain| can be empty.

An alternative version of the command line processing described
in \secref{sec:commandline} using the detection mechanism reads:
%
\begin{center}
|... -jobname "|\textit{target}|" "|[\textit{flags}]%
[|\def\jobname{|\textit{dest}|}|]|\input{|\textit{main}|}"|
\end{center}

%%%%%%%%%%%%%%%%%%%%%%%%%%%%%%%%%%%%%%%%%%%%%%%%%%%%%%%%%%%%%%%%%%%%%%%%%%%%%%%%
\subsection{Manual Code}
\label{sec:manual}

In case one cannot be certain whether the definitions file |childdoc.def|
is installed on the target \TeX{} distribution
and one prefers not to ship it,
it is conceivable to paste a few relevant commands into the sources.

To that end, drop all statements |\input{childdoc.def}|
and perform the replacements as outlined below.
Instead of |\childdocmain{|\textit{main}|}| add the following code
to the top of the main file:
%
\begin{center}
\begin{tabular}{l}
|\||ifdefined\childdocname\endinput\||fi\newif\ifchilddoc|\\
|\edef\childdocname{\scantokens\expandafter{\jobname\noexpand}}|\\
|\def\childdocmain{|\textit{main}|}\||ifx\childdocmain\childdocname\||else|\\
|\childdoctrue\includeonly{\childdocname}\let\jobname\childdocmain\||fi|\\
\end{tabular}
\end{center}
%
Instead of |\childdocof{|\textit{main}|}| just include the main file
at the top of each child file:
%
\begin{center}
|\input{|\textit{main}|}|
\end{center}
%
A simple redirection |\childdocforward{|\textit{dest}|}| is achieved by:
%
\begin{center}
|\def\jobname{|\textit{dest}|}\input{\jobname}|
\end{center}
%
The redirection with prefix
|\childdocforwardprefix[|\textit{prefix}|]{|\textit{dest}|}|
is accomplished by:
%
\begin{center}
\begin{tabular}{l}
|{\edef\jobname{\scantokens\expandafter{\jobname\noexpand}}|\\
|\def\redirectjob |\textit{prefix}|#1~~~{\gdef\jobname{|\textit{dest}|#1}}|\\
|\expandafter\redirectjob\jobname~~~}\input{\jobname}|
\end{tabular}
\end{center}

In an alternative approach,
child documents can be compiled by a specific command line
without additional code or specific definitions:
%
\begin{center}
|... -jobname "|\textit{target}|" "|[\textit{flags}]%
|\includeonly{|\textit{dest}|}\input{|\textit{main}|}"|
\end{center}
%

%%%%%%%%%%%%%%%%%%%%%%%%%%%%%%%%%%%%%%%%%%%%%%%%%%%%%%%%%%%%%%%%%%%%%%%%%%%%%%%%
%%%%%%%%%%%%%%%%%%%%%%%%%%%%%%%%%%%%%%%%%%%%%%%%%%%%%%%%%%%%%%%%%%%%%%%%%%%%%%%%
\section{Information}

%%%%%%%%%%%%%%%%%%%%%%%%%%%%%%%%%%%%%%%%%%%%%%%%%%%%%%%%%%%%%%%%%%%%%%%%%%%%%%%%
\subsection{Copyright}

Copyright \copyright{} 2017--2018 Niklas Beisert

This work may be distributed and/or modified under the
conditions of the \LaTeX{} Project Public License, either version 1.3
of this license or (at your option) any later version.
The latest version of this license is in
  \url{http://www.latex-project.org/lppl.txt}
and version 1.3 or later is part of all distributions of \LaTeX{}
version 2005/12/01 or later.

This work has the LPPL maintenance status `maintained'.

The Current Maintainer of this work is Niklas Beisert.

This work consists of the files |README.txt|, |childdoc.ins| and |childdoc.dtx|
as well as the derived files |childdoc.def|, |cdocsamp.tex|
with |cdocsch1.tex|, |cdocsch2.tex|, |cdocspt3.tex|, |cdocspt4.tex|,
|cdocsdrf.tex|, |cdocsfn1.tex|, |cdocsfn2.tex|
as well as |childdoc.pdf|.

%%%%%%%%%%%%%%%%%%%%%%%%%%%%%%%%%%%%%%%%%%%%%%%%%%%%%%%%%%%%%%%%%%%%%%%%%%%%%%%%
\subsection{Files and Installation}

The package consists of the files:
%
\begin{center}
\begin{tabular}{ll}
    |README.txt|   & readme file \\
    |childdoc.ins| & installation file \\
    |childdoc.dtx| & source file \\
    |childdoc.def| & definition file \\
    |cdocsamp.tex| & sample main file \\
    |cdocsch1.tex| & sample include file \\
    |cdocsch2.tex| & sample include file \\
    |cdocspt3.tex| & sample part file \\
    |cdocspt4.tex| & sample part file \\
    |cdocsdrf.tex| & sample redirection file \\
    |cdocsfn1.tex| & sample redirection file \\
    |cdocsfn2.tex| & sample redirection file \\
    |childdoc.pdf| & manual
\end{tabular}
\end{center}
%
The distribution consists of the files
|README.txt|, |childdoc.ins| and |childdoc.dtx|.
%
\begin{itemize}
\item
Run (pdf)\LaTeX{} on |childdoc.dtx|
to compile the manual |childdoc.pdf| (this file).
\item
Run \LaTeX{} on |childdoc.ins| to create the definitions file |childdoc.def|
and the sample |cdocsamp.tex| with include files
|cdocsch1.tex|, |cdocsch2.tex|, |cdocspt3.tex|, |cdocspt4.tex|,
|cdocsdrf.tex|, |cdocsfn1.tex|, |cdocsfn2.tex|.
Then copy the file |childdoc.def| to an appropriate directory of your \LaTeX{}
distribution, e.g.\ \textit{texmf-root}|/tex/latex/childdoc|.
\end{itemize}

%%%%%%%%%%%%%%%%%%%%%%%%%%%%%%%%%%%%%%%%%%%%%%%%%%%%%%%%%%%%%%%%%%%%%%%%%%%%%%%%
\subsection{Related CTAN Packages}

There are several other packages which offer a similar functionality:
%
\begin{itemize}
\item
The packages
\href{http://ctan.org/pkg/docmute}{\textsf{docmute}},
\href{http://ctan.org/pkg/includex}{\textsf{includex}} and
\href{http://ctan.org/pkg/standalone}{\textsf{standalone}}
provide commands to include only the document body of
a child file thus allowing both files to be compiled individually.
\item
The packages \href{http://ctan.org/pkg/subdocs}{\textsf{subdocs}}
and \href{http://ctan.org/pkg/subfiles}{\textsf{subfiles}}
provide structures in which the main and child documents can be
encapsulated and allowing them to be compiled individually.
The inclusion mechanism is different from the conventional |\include|.
\item
The package \href{http://ctan.org/pkg/combine}{\textsf{combine}}
is an elaborate solution to combine several documents into one.
\end{itemize}
%
See also the CTAN topic \href{http://ctan.org/topic/subdocs}{\textsf{subdocs}}
for further related packages.
The present package differs from the above solutions in that
a document structure constructed with the conventional |\include| mechanism
just needs two extra commands at the top of every file
such that all constituent files can be compiled individually.

%%%%%%%%%%%%%%%%%%%%%%%%%%%%%%%%%%%%%%%%%%%%%%%%%%%%%%%%%%%%%%%%%%%%%%%%%%%%%%%%
%\subsection{Feature Suggestions}
%
%The following is a list of features which may be useful for future
%versions of this package:
%%
%\begin{itemize}
%\item
%\ldots
%\end{itemize}

%%%%%%%%%%%%%%%%%%%%%%%%%%%%%%%%%%%%%%%%%%%%%%%%%%%%%%%%%%%%%%%%%%%%%%%%%%%%%%%%
\subsection{Revision History}

%%%%%%%%%%%%%%%%%%%%%%%%%%%%%%%%%%%%%%%%
\paragraph{v2.0:} 2018/12/30

\begin{itemize}
\item
immediate forward processing
\item
added |\childdocby| mechanism
\item
manual restructured
\end{itemize}

%%%%%%%%%%%%%%%%%%%%%%%%%%%%%%%%%%%%%%%%
\paragraph{v1.6:} 2018/01/17

\begin{itemize}
\item
application for development of include files
\item
corrections to manual
\end{itemize}

%%%%%%%%%%%%%%%%%%%%%%%%%%%%%%%%%%%%%%%%
\paragraph{v1.5:} 2017/05/21

\begin{itemize}
\item
more complete structuring introduced
\item
|\childdocof| introduced
\item
|\childdoc| renamed to |\childdocmain|
\item
|\childredirect| renamed to |\childdocforward| and |\childdocforwardprefix|
and functionality expanded
\end{itemize}

%%%%%%%%%%%%%%%%%%%%%%%%%%%%%%%%%%%%%%%%
\paragraph{v1.0:} 2017/04/27

\begin{itemize}
\item
manual and install package
\item
first version published on CTAN
\end{itemize}

%%%%%%%%%%%%%%%%%%%%%%%%%%%%%%%%%%%%%%%%
\paragraph{v0.6:} 2017/04/26

\begin{itemize}
\item
redirection mechanism added
\end{itemize}

%%%%%%%%%%%%%%%%%%%%%%%%%%%%%%%%%%%%%%%%
\paragraph{v0.5:} 2017/04/26

\begin{itemize}
\item
functionality in definition file
\end{itemize}


%%%%%%%%%%%%%%%%%%%%%%%%%%%%%%%%%%%%%%%%%%%%%%%%%%%%%%%%%%%%%%%%%%%%%%%%%%%%%%%%
%%%%%%%%%%%%%%%%%%%%%%%%%%%%%%%%%%%%%%%%%%%%%%%%%%%%%%%%%%%%%%%%%%%%%%%%%%%%%%%%
%%%%%%%%%%%%%%%%%%%%%%%%%%%%%%%%%%%%%%%%%%%%%%%%%%%%%%%%%%%%%%%%%%%%%%%%%%%%%%%%
\appendix

\settowidth\MacroIndent{\rmfamily\scriptsize 000\ }

 \DocInput{childdoc.dtx}

\end{document}
%</driver>
% \fi
%
% %%%%%%%%%%%%%%%%%%%%%%%%%%%%%%%%%%%%%%%%%%%%%%%%%%%%%%%%%%%%%%%%%%%%%%%%%%%%%%
% %%%%%%%%%%%%%%%%%%%%%%%%%%%%%%%%%%%%%%%%%%%%%%%%%%%%%%%%%%%%%%%%%%%%%%%%%%%%%%
% \section{Sample}
%\iffalse
%<*samplemain>
%\fi
%
% The following presents a sample document
% with two chapters, two parts, a title page,
% a compile flag as well as three forwarding files to set the flag.
% It consists of eight |.tex| files:
% \begin{center}
% \begin{tabular}{ll}
% |cdocsamp.tex|&main file\\
% |cdocsch1.tex|&include file for chapter 1\\
% |cdocsch2.tex|&include file for chapter 2\\
% |cdocspt3.tex|&include file for part 3\\
% |cdocspt4.tex|&include file for part 4\\
% |cdocsdrf.tex|&forwarding file for main file in draft mode\\
% |cdocsfi1.tex|&forwarding file for final version of chapter 1\\
% |cdocsfi2.tex|&forwarding file for final version of chapter 2\\
% \end{tabular}
% \end{center}
% Each of the eight files can be compiled directly by the \LaTeX{} compiler.
%
% %%%%%%%%%%%%%%%%%%%%%%%%%%%%%%%%%%%%%%
% \paragraph{Main File.}
%
% The main file is called |cdocsamp.tex|.
%
% Load the \textsf{childdoc} definitions and
% declare the filename for the main document:
%    \begin{macrocode}
\input{childdoc.def}
\childdocmain{}
%    \end{macrocode}

% Optional override for |\version| flag:
%    \begin{macrocode}
%%\ifchilddoc\else\providecommand{\version}{draft}\fi
%    \end{macrocode}

% Define the default values for the |\version| flag
% (|final| for the main file and |draft| for childs):
%    \begin{macrocode}
\ifchilddoc
\providecommand{\version}{draft}
\else
\providecommand{\version}{final}
\fi
%    \end{macrocode}

% Load the standard document class:
%    \begin{macrocode}
\documentclass[12pt]{article}
%    \end{macrocode}

% Start the document body:
%    \begin{macrocode}
\begin{document}
%    \end{macrocode}

% Declare a title page.
% Print title, part of document being processed and version flag:
%    \begin{macrocode}
\addtocounter{page}{-1}
\begin{center}
{\LARGE\bfseries{}childdoc example\par}
\vspace{1cm}
\ifchilddoc
\ifchilddocmanual part\else chapter\fi:
`\childdocname' of `\childdocjob'\par
\else
main document: `\childdocjob'\par
\fi
version: \version\par
\end{center}
\newpage
%    \end{macrocode}

% Manually include selected file,
% otherwise process as usual:
%    \begin{macrocode}
\ifchilddocmanual
\section*{part `\childdocname'}
\input{\childdocname}
\else
%    \end{macrocode}

% Include the two chapters:
%    \begin{macrocode}
\include{cdocsch1}
\include{cdocsch2}
%    \end{macrocode}

% Include the two parts unless only chapters should be displayed:
%    \begin{macrocode}
\ifchilddoc\else
\section{part three}
\input{cdocspt3}
\section{part four}
\input{cdocspt4}
\fi
%    \end{macrocode}

% Process as usual until here:
%    \begin{macrocode}
\fi
%    \end{macrocode}

% End of document body:
%    \begin{macrocode}
\end{document}
%    \end{macrocode}
%\iffalse
%</samplemain>
%\fi
%
% %%%%%%%%%%%%%%%%%%%%%%%%%%%%%%%%%%%%%%
% \paragraph{Chapter Include Files.}
%
% The include files are called |cdocsch1.tex| and |cdocsch2.tex|.
%
%\iffalse
%<*samplechap1|samplechap2>
%\fi

% Optional override for |\version| flag:
%    \begin{macrocode}
%%\providecommand{\version}{final}
%    \end{macrocode}

% Include the main document:
%    \begin{macrocode}
\input{childdoc.def}
\childdocof{cdocsamp}
%    \end{macrocode}

%\iffalse
%</samplechap1|samplechap2>
%\fi
%
%\iffalse
%<*samplechap1>
%\fi
% Some text for chapter 1:
%    \begin{macrocode}
\section{one}
some text in chapter one
%    \end{macrocode}

%\iffalse
%</samplechap1>
%\fi
% Some text for chapter 2:
%\iffalse
%<*samplechap2>
%\fi
%    \begin{macrocode}
\section{two}
more text in chapter two
%    \end{macrocode}

%\iffalse
%</samplechap2>
%\fi
%
% %%%%%%%%%%%%%%%%%%%%%%%%%%%%%%%%%%%%%%
% \paragraph{Part Include Files.}
%
% The include files are called |cdocspt3.tex| and |cdocspt4.tex|.
%
%\iffalse
%<*samplepart3|samplepart4>
%\fi

% Optional override for |\version| flag:
%    \begin{macrocode}
%%\providecommand{\version}{final}
%    \end{macrocode}

% Include the main document:
%    \begin{macrocode}
\input{childdoc.def}
\childdocby{cdocsamp}
%    \end{macrocode}

%\iffalse
%</samplepart3|samplepart4>
%\fi
%
%\iffalse
%<*samplepart3>
%\fi
% Some text for part 3:
%    \begin{macrocode}
some text in part three
%    \end{macrocode}

%\iffalse
%</samplepart3>
%\fi
% Some text for part 4:
%\iffalse
%<*samplepart4>
%\fi
%    \begin{macrocode}
more text in part four
%    \end{macrocode}

%\iffalse
%</samplepart4>
%\fi
%
% %%%%%%%%%%%%%%%%%%%%%%%%%%%%%%%%%%%%%%
% \paragraph{Forwarding for a Complete Draft.}
%
% The following forwarding file |cdocsdrf.tex|
% compiles the main document in draft mode:
%\iffalse
%<*sampledraft>
%\fi
%    \begin{macrocode}
\def\version{draft}
\input{childdoc.def}
\childdocforward{cdocsamp}
%    \end{macrocode}

%\iffalse
%</sampledraft>
%\fi
%
% %%%%%%%%%%%%%%%%%%%%%%%%%%%%%%%%%%%%%%
% \paragraph{Forwarding for Final Version of the Chapters.}
%
% The following forwarding files |cdocsfn1.tex| and |cdocsfn2.tex|
% (with identical content)
% compile the final versions of the child documents
% |cdocsch1.tex| and |cdocsch2.tex|, respectively:
%\iffalse
%<*samplefinal>
%\fi
%    \begin{macrocode}
\def\version{final}
\input{childdoc.def}
\childdocforwardprefix[cdocsamp]{cdocsfn}{cdocsch}
%    \end{macrocode}

%\iffalse
%</samplefinal>
%\fi
%
% %%%%%%%%%%%%%%%%%%%%%%%%%%%%%%%%%%%%%%
% \paragraph{Command Line Processing.}
%
% The following three command lines generate the output files
% |cdocscld|, |cdocscl1| and |cdocscl2|
% which should be identical to
% |cdocsdrf|, |cdocsch1| and |cdocsfn2|, respectively:
% \begin{center}
% \begin{tabular}{l}
% |latex -jobname cdocscld \|\\
% |  "\def\version{draft}\input{childdoc.def}\childdocforward{cdocsamp}"|\\
% |latex -jobname cdocscl1 \|\\
% |  "\input{childdoc.def}\childdocforward[cdocsamp]{cdocsch1}"|\\
% |latex -jobname cdocscl2 \|\\
% |  "\def\version{final}\input{childdoc.def}\childdocforward{cdocsch2}"|
% \end{tabular}
% \end{center}
% Note that the trailing backslash on each first line
% merely continues the input to the second line
% (for convenient cut ant paste).
% Furthermore, the command |latex| can be replaced by any
% of its alternative versions such as |pdflatex|.
%
% %%%%%%%%%%%%%%%%%%%%%%%%%%%%%%%%%%%%%%%%%%%%%%%%%%%%%%%%%%%%%%%%%%%%%%%%%%%%%%
% %%%%%%%%%%%%%%%%%%%%%%%%%%%%%%%%%%%%%%%%%%%%%%%%%%%%%%%%%%%%%%%%%%%%%%%%%%%%%%
% \section{Implementation}
%\iffalse
%<*package>
%\fi
%
% This section describes the definitions file |childdoc.def|.

% The definitions cannot be loaded using |\usepackage| or |\RequirePackage|
% which has a mechanism to prevent loading a style file more than once.
% When loading the definitions by means of |\input|
% multiple instances have to be prevented manually:
%\iffalse
%This code needs to be before the `\ProvidesFile' directive
%which is defined at the beginning of this file.
%Therefore it is also placed there and commented out here.
%</package>
%<*discard>
%\fi
%    \begin{macrocode}
\ifdefined\childdocmain\endinput\fi
%    \end{macrocode}
%\iffalse
%</discard>
%<*package>
%\fi
%
% \macro{\ifchilddoc}
% \macro{\ifchilddocmanual}
% The conditional |\ifchilddoc| tells whether a
% child (true) or main (false) document is being compiled.
% The conditional |\ifchilddocmanual| tells whether
% the |\includeonly| mechanism is used (false) or
% the selection of child files must be performed manually (true).
% The definitions initialise to false:
%    \begin{macrocode}
\newif\ifchilddoc
\newif\ifchilddocmanual
%    \end{macrocode}

% \macro{\childdocname}
% \macro{\childdocjob}
% The macro |\childdocname| stores the name of the main document
% to be compiled. The macro |\childdocjob| stores the name of
% the document on which the \LaTeX{} compiler was originally invoked.
% The content of |\jobname| cannot be compared
% to filenames specified in the source due to different catcodes.
% The following code rescans |\jobname|, stores the result
% in |\childdocname| and saves a copy in |\childdocjob|:
%    \begin{macrocode}
\edef\childdocname{\scantokens\expandafter{\jobname\noexpand}}
\let\childdocjob\childdocname
%    \end{macrocode}

% \macro{\childdocdisable}
% The macro |\childdocdisable| prevents the main file
% from being processed more than once.
% At this stage, the main document command |\childdocmain|
% is assumed to be called once again where it should do nothing.
% Any subsequent call to it should prevent
% a secondary processing of the main document
% It overwrites the forwarding commands
% |\childdocof| and |\childdocforward|
% with empty macros to prevent further inclusions of the main document:
%    \begin{macrocode}
\newcommand{\childdocdisable}
{
  \renewcommand{\childdocmain}[1]{\renewcommand{\childdocmain}[1]{\endinput}}
  \renewcommand{\childdocof}[1]{}
  \renewcommand{\childdocby}[2][]{}
  \renewcommand{\childdocforward}[2][]{}
  \renewcommand{\childdocdisable}{}
}
%    \end{macrocode}

% \macro{\childdocmain}
% The macro |\childdocmain| is to be called at the top of the main file
% with nothing or the main filename (without extension) as argument.
% First, it breaks loops.
% If the argument is not empty and does not match |\childdocname|
% (which is set by the first inclusion of |childdoc.def|),
% |\ifchilddoc| is set to true, |\includeonly| is applied to the child file
% and |\jobname| is set to the main file
% (for proper handling of |.aux| files):
%    \begin{macrocode}
\newcommand{\childdocmain}[1]
{
  \childdocdisable\childdocmain{}
  \if?#1?\else
    \begingroup
      \def\childdoctmp{#1}
      \ifx\childdoctmp\childdocname
        \def\childdoctmp{}
      \else
        \def\childdoctmp
        {
          \childdoctrue
          \includeonly{\childdocname}
          \def\childdocjob{#1}
          \def\jobname{#1}
        }
      \fi
      \expandafter
    \endgroup
    \childdoctmp
  \fi
}
%    \end{macrocode}

% \macro{\childdocof}
% The command |\childdocof| redirects
% compilation to the main file |#1|.
%    \begin{macrocode}
\newcommand{\childdocof}[1]
{
  \childdocdisable
  \childdoctrue
  \includeonly{\childdocname}
  \def\jobname{#1}
  \def\childdocjob{#1}
  \input{#1}
}
%    \end{macrocode}

% \macro{\childdocby}
% The command |\childdocby| ....
%    \begin{macrocode}
\newcommand{\childdocby}[2][]
{
  \childdocdisable
  \childdoctrue
  \childdocmanualtrue
  \if?#1?\else
    \def\jobname{#2}
  \fi
  \def\childdocjob{#2}
  \input{#2}
  \endinput
}
%    \end{macrocode}

% \macro{\childdocforward}
% The command |\childdocforward| redirects
% compilation to the main file or
% (if the optional argument is given) a child file.
% Parameters are set as if the main file
% or a child file starting with |\childdocof| was compiled.
% Then compilation is handed over to the main file:
%    \begin{macrocode}
\newcommand{\childdocforward}[2][]
{
  \begingroup
    \if?#1?
      \def\childdoctmp
      {
        \def\childdocname{#2}
        \def\childdocjob{#2}
        \def\jobname{#2}
        \input{#2}
        \endinput
      }
    \else
      \def\childdoctmp
      {
        \childdocdisable
        \def\childdocname{#2}
        \childdoctrue
        \includeonly{#2}
        \def\childdocjob{#1}
        \def\jobname{#1}
        \input{#1}
        \endinput
      }
    \fi
    \expandafter
  \endgroup
  \childdoctmp
}
%    \end{macrocode}

% \macro{\childdocforwardprefix}
% The command |\childdocforwardprefix| redirects
% compilation to the main or a child file by means of a pattern.
% The prefix |#1| in the current filename is replaced by |#2|
% and the suffix of the current filename is kept
% (it is assumed that the filename does not contain the substring `|~~~|'
% which is used as a delimiter).
% Compilation is handed over to the new file by |\childdocforward|:
%    \begin{macrocode}
\newcommand{\childdocforwardprefix}[3][]
{
  \begingroup
    \def\childdocextract #2##1~~~{\def\childdoctmp{\childdocforward[#1]{#3##1}}}
    \expandafter\childdocextract\childdocname~~~
    \expandafter
  \endgroup
  \childdoctmp
}
%    \end{macrocode}

% \macro{\childdoc}
% The deprecated macro |\childdoc| is a legacy version of |\childdocmain|:
%    \begin{macrocode}
\newcommand{\childdoc}{\childdocmain}
%    \end{macrocode}

% \macro{\childdocredirect}
% The deprecated macro |\childdocredirect| is a legacy version
% of |\childdocforward| and |\childdocforwardprefix|:
%    \begin{macrocode}
\newcommand{\childdocredirect}[2][]
{
  \begingroup
    \if?#1?
      \def\childdoctmp{\childdocforward{#2}}
    \else
      \def\childdoctmp{\childdocforwardprefix{#1}{#2}}
    \fi
    \expandafter
  \endgroup
  \childdoctmp
}
%    \end{macrocode}

%\iffalse
%</package>
%\fi
%
\endinput
|\\
|\childdocforwardprefix{final}{child}|
\end{tabular}
\end{center}
%

Note that when several versions of a main file and/or of each child file
are to be generated, it may be convenient to set up a |Makefile| or
shell script to automatise the process.

%%%%%%%%%%%%%%%%%%%%%%%%%%%%%%%%%%%%%%%%%%%%%%%%%%%%%%%%%%%%%%%%%%%%%%%%%%%%%%%%
\subsection{Command Line Processing}
\label{sec:commandline}

The effect of redirection files can also be achieved by invoking
the \LaTeX{} compiler with a more elaborate command line.
Most conveniently this should be done as part
of a shell script or a |Makefile|.

When using \textsf{childdoc} in the main file, the following
command lines effectively perform a redirection
(note that depending on the shell being used,
backslashes may have to be doubled: `|\|' $\to$ `|\\|'):
%
\begin{center}
|... -jobname "|\textit{target}|" |\\|"|[\textit{flags}]%
|% \iffalse
%
% childdoc.dtx Copyright (C) 2017-2018 Niklas Beisert
%
% This work may be distributed and/or modified under the
% conditions of the LaTeX Project Public License, either version 1.3
% of this license or (at your option) any later version.
% The latest version of this license is in
%   http://www.latex-project.org/lppl.txt
% and version 1.3 or later is part of all distributions of LaTeX
% version 2005/12/01 or later.
%
% This work has the LPPL maintenance status `maintained'.
%
% The Current Maintainer of this work is Niklas Beisert.
%
% This work consists of the files childdoc.dtx and childdoc.ins
% and the derived files childdoc.def and cdocsamp.tex with
% cdocsch1.tex, cdocsch2.tex, cdocsdrf.tex, cdocsfn1.tex, cdocsfn2.tex.
%
%<package>\ifdefined\childdocmain\endinput\fi
%<package>\ProvidesFile{childdoc.def}[2018/12/30 v2.0 child document driver]
%<samplemain>\ProvidesFile{cdocsamp.tex}[2018/12/30 v2.0 sample for childdoc]
%<*driver>
%\ProvidesFile{childdoc.drv}[2018/12/30 v2.0 childdoc reference manual file]
\PassOptionsToClass{10pt,a4paper}{article}
\documentclass{ltxdoc}

\usepackage[margin=35mm]{geometry}
\usepackage{hyperref}
\usepackage{hyperxmp}
\usepackage[usenames]{color}

\hypersetup{colorlinks=true}
\hypersetup{pdfstartview=FitH}
\hypersetup{pdfpagemode=UseNone}
\hypersetup{pdfsource={}}
\hypersetup{pdflang={en-UK}}
\hypersetup{pdfcopyright={Copyright 2017-2018 Niklas Beisert.
  This work may be distributed and/or modified under the
  conditions of the LaTeX Project Public License, either version 1.3
  of this license or (at your option) any later version.}}
\hypersetup{pdflicenseurl={http://www.latex-project.org/lppl.txt}}
\hypersetup{pdfcontactaddress={ETH Zurich, ITP, HIT K,
  Wolfgang-Pauli-Strasse 27}}
\hypersetup{pdfcontactpostcode={8093}}
\hypersetup{pdfcontactcity={Zurich}}
\hypersetup{pdfcontactcountry={Switzerland}}
\hypersetup{pdfcontactemail={nbeisert@itp.phys.ethz.ch}}
\hypersetup{pdfcontacturl={http://people.phys.ethz.ch/\xmptilde nbeisert/}}

\newcommand{\secref}[1]{\hyperref[#1]{section \ref*{#1}}}

\parskip1ex
\parindent0pt
\let\olditemize\itemize
\def\itemize{\olditemize\parskip0pt}

\begin{document}

\title{The \textsf{childdoc} Package}
\hypersetup{pdftitle={The childdoc Package}}
\author{Niklas Beisert\\[2ex]
  Institut f\"ur Theoretische Physik\\
  Eidgen\"ossische Technische Hochschule Z\"urich\\
  Wolfgang-Pauli-Strasse 27, 8093 Z\"urich, Switzerland\\[1ex]
  \href{mailto:nbeisert@itp.phys.ethz.ch}
  {\texttt{nbeisert@itp.phys.ethz.ch}}}
\hypersetup{pdfauthor={Niklas Beisert}}
\hypersetup{pdfsubject={Manual for the LaTeX2e Package childdoc}}
\date{30 December 2018, \textsf{v2.0}}
\maketitle

\begin{abstract}\noindent
\textsf{childdoc} is a \LaTeXe{} package
that enables the direct compilation
of document sections included by |\include|
to individual files.
\end{abstract}

\begingroup
\parskip0ex
\tableofcontents
\endgroup

%%%%%%%%%%%%%%%%%%%%%%%%%%%%%%%%%%%%%%%%%%%%%%%%%%%%%%%%%%%%%%%%%%%%%%%%%%%%%%%%
%%%%%%%%%%%%%%%%%%%%%%%%%%%%%%%%%%%%%%%%%%%%%%%%%%%%%%%%%%%%%%%%%%%%%%%%%%%%%%%%
\section{Introduction}

\LaTeX{} provides a mechanism to structure a large document (such as a book)
into a main file and several child files (containing the chapters)
using the |\include| command.
This mechanism is beneficial for documents
which span hundreds of pages in order to
make the source file(s) more manageable.
Moreover, compilation can be restricted to
selected child files by means of the |\includeonly| command.
The latter feature can be used to reduce the compilation time while editing
(this was significantly more useful in the earlier days of \LaTeX{})
or to generate a smaller document which is easier to navigate.
Another application of |\includeonly| is to generate
documents consisting of selected parts of the complete document.

However, there are a few drawbacks of the plain |\include| mechanism:
\begin{itemize}
\item
The child files cannot be compiled on their own,
they can only be compiled via the main file.
A naive editing environment
(such as a text editor with an option
to have the current file processed by \LaTeX)
may require one to switch to the main file before compiling;
attempting to compile the child file produces errors.
\item
The main file must be modified (each time)
to adjust the |\includeonly| command
to the present needs. This easily leaves the main file in a messy state.
\item
The generated document will always carry the filename
of the main document. This is inconvenient if
several child files are to be compiled and
to be kept for distribution.
\end{itemize}

The present package provides a simple interface
to make child files individually compilable by \LaTeX{}.
Compiling a child file then has the same effect as compiling
the main file with an |\includeonly| command
to select the appropriate child.
Moreover the generated document will carry the name of the child
rather than the main file.
This resolves all three above issues.

This feature is meant to make the editing of books,
thesis documents and lecture notes somewhat more convenient.
However, the package can also be used efficiently for
composing a series of documents (such as exercise sheets)
which are typically distributed individually.
It then assists the author in generating the individual documents
(potentially in different versions)
as well as a document containing the collected series.
Another application is in developing style files
or other kinds of included material
where compilation of the style file could redirect
to a sample or test file.

%%%%%%%%%%%%%%%%%%%%%%%%%%%%%%%%%%%%%%%%%%%%%%%%%%%%%%%%%%%%%%%%%%%%%%%%%%%%%%%%
%%%%%%%%%%%%%%%%%%%%%%%%%%%%%%%%%%%%%%%%%%%%%%%%%%%%%%%%%%%%%%%%%%%%%%%%%%%%%%%%
\section{Usage}

First of all, the package \textsf{childdoc} is \emph{not} a standard
\LaTeXe{} |.sty| style file! Therefore it needs to be invoked in
a non-standard way.

%%%%%%%%%%%%%%%%%%%%%%%%%%%%%%%%%%%%%%%%%%%%%%%%%%%%%%%%%%%%%%%%%%%%%%%%%%%%%%%%
\subsection{Included Files}
\label{sec:include}

%%%%%%%%%%%%%%%%%%%%%%%%%%%%%%%%%%%%%%%%
\DescribeMacro{\childdocmain}
To use the package, add the commands
\begin{center}
\begin{tabular}{l}
|\input{childdoc.def}|\\
|\childdocmain{}|\\
\end{tabular}
\end{center}
at the very top of the main \LaTeX{} file,
in particular \emph{before} the |\documentclass| statement!
The argument of |\childdocmain| should be left empty
(but it must be present).

%%%%%%%%%%%%%%%%%%%%%%%%%%%%%%%%%%%%%%%%
\DescribeMacro{\childdocof}
Furthermore, add the commands
\begin{center}
\begin{tabular}{l}
|\input{childdoc.def}|\\
|\childdocof{|\textit{main}|}|\\
\end{tabular}
\end{center}
at the top of every child file \textit{child}
which is included by |\include{|\textit{child}|}|
from within the main file
(or at least for those files to be compiled individually).
The argument \textit{main} must be the filename of the main file.

There are a couple of
considerations in setting up the main and child documents:

%%%%%%%%%%%%%%%%%%%%%%%%%%%%%%%%%%%%%%%%
\paragraph{Restrictions.}

Please note the following restrictions:
\begin{itemize}
\item
|\childdocmain| must be called with one argument \textit{main}
to ensure compatibility with earlier version of the package.
It must either be empty (|\childdocmain{}|)
or precisely match the filename of the main file in which it is specified.
See \secref{sec:detection} for further information.
\item
The filename \textit{main} must be specified without the |.tex| extension.
\item
The filename \textit{main} is case sensitive
(even in case-insensitive file systems)
due to internal string comparison.
\item
The argument \textit{main} should be fully expanded, it cannot be a macro.
\item
Subdirectories and special characters should be avoided in filenames.
\item
The command |\childdocmain{|\textit{main}|}| must be followed by a whitespace.
It should not be followed immediately by another command
or by a comment mark `|%|'.
This is because the \TeX{} parser reads the token immediately following
the argument of |\childdocmain| and puts it
at the beginning of every child section;
however, a white\-space is ignored.
\end{itemize}

%%%%%%%%%%%%%%%%%%%%%%%%%%%%%%%%%%%%%%%%
\paragraph{Content of Main File.}

It is advisable to place all content in the child files included by |\include|.
Any output contained in the main file will appear in all child documents
unless suppressed manually;
it cannot be suppressed automatically by the |\includeonly| directive
and thus should normally be avoided.
A method to include some content in the main file
by means of conditional processing is described in \secref{sec:conditional}.

%%%%%%%%%%%%%%%%%%%%%%%%%%%%%%%%%%%%%%%%
\paragraph{Page Numbering.}

When only a part of the document is compiled,
the appropriate numbering of pages
(as well as other status parameters)
is determined from the |.aux| files.
The latter contain information from previous passes.
However this information needs to propagate through
all intermediate child documents.
Therefore the page numbering in child documents may well
be inconsistent until the complete document is compiled at least once.

A useful (if unconventional) way to always ensure a consistent
page numbering is to restart the numbering in each child document
and denote the pages by `\textit{child}|.|\textit{page}'
where \textit{child} represents the chapter/section number of the child file.
This can be achieved by the command
|\numberwithin{page}{|\textit{child}|}|
of the \textsf{amsmath} package
where \textit{child} can be |chapter| or |section|
depending on the chosen structuring.
Alternatively, one can modify the macro |\thepage| appropriately
and reset the counter |page| at the start of each child file.

%%%%%%%%%%%%%%%%%%%%%%%%%%%%%%%%%%%%%%%%%%%%%%%%%%%%%%%%%%%%%%%%%%%%%%%%%%%%%%%%
\subsection{Conditional Processing}
\label{sec:conditional}

The package provides a mechanism to compile different versions
of a document. To customise the versions further some conditional processing
can come in handy to distinguish which version is being compiled.
The package provides two macros to describe the compilation context:

%%%%%%%%%%%%%%%%%%%%%%%%%%%%%%%%%%%%%%%%
\DescribeMacro{\ifchilddoc}
The conditional |\ifchilddoc| distinguishes between the compilation of
child documents and the main document:
%
\begin{center}
|\ifchilddoc |\textit{child-code}| |[|\||else |\textit{main-code}]| \||fi|
\end{center}

%%%%%%%%%%%%%%%%%%%%%%%%%%%%%%%%%%%%%%%%
\DescribeMacro{\childdocname}
\DescribeMacro{\childdocjob}
The macro |\childdocname| contains the filename (without extension)
of the main or child file being processed.
Note that |\childdocjob| will always contain the name of the main file.

%%%%%%%%%%%%%%%%%%%%%%%%%%%%%%%%%%%%%%%%
\paragraph{Title Page.}

Conditional processing can be used to include a title or banner page
in the main document when proper precautions are taken.
Importantly, the code in the main file should ensure that the page counter
(as well as other status parameters which are stored in the |.aux| files)
takes the same value after the conditional processing.
Otherwise the page numbers may take divergent values
depending on which part is compiled.

For example, a title page could be declared by:
%
\begin{center}
\begin{tabular}{l}
|\ifchilddoc\||else|\\
|\addtocounter{page}{-1}|\\
\textit{code for title page}\\
|\newpage|\\
|\||fi|
\end{tabular}
\end{center}
%
A banner page for the child documents can be generated by:
%
\begin{center}
\begin{tabular}{l}
|\ifchilddoc|\\
|\addtocounter{page}{-1}|\\
\textit{code for banner page}\\
|\newpage|\\
|\||fi|
\end{tabular}
\end{center}
%
Here one could write a message such as:
\begin{center}
|This is the part \childdocname{} of \childdocjob{}.|
\end{center}

%%%%%%%%%%%%%%%%%%%%%%%%%%%%%%%%%%%%%%%%%%%%%%%%%%%%%%%%%%%%%%%%%%%%%%%%%%%%%%%%
\subsection{Flags}
\label{sec:flags}

The package makes it easy to generate different versions
of the main or child documents.
To this end compilation flags can be defined
and assigned different default values.
They will be particularly useful in conjunction
with the forwarding mechanism described in \secref{sec:forward}.

For example, it may be useful to have a flag |\version|
which can be set to |draft| or |final|.
The document source will contain some conditional code
depending on the value of |\version|.
Suppose further, the flag should default to |final| for the main file
and to |draft| for child files
which is a natural assignment for editing the document.
This is achieved by placing the following code
in the preamble of the main document
(below the |\childdocmain| directive):
%
\begin{center}
\begin{tabular}{l}
|\ifchilddoc|\\
|\providecommand{\version}{draft}|\\
|\||else|\\
|\providecommand{\version}{final}|\\
|\||fi|
\end{tabular}
\end{center}
%
The definition by |\providecommand| makes sure
that previous definitions are not overwritten.
Further statements |\providecommand{\version}{...}|
can thus be added before the above code to override it.

For the main file, one might add a line
(between |\childdocmain| and the above block)
%
\begin{center}
|%\ifchilddoc\||else\providecommand{\version}{draft}\||fi|
\end{center}
%
which can be uncommented to produce a draft version.
Likewise one can add a line to the very top of a child file
(above the |\childdocof{|\textit{main}|}| directive)
%
\begin{center}
|%\providecommand{\version}{final}|
\end{center}
%
which can be uncommented to produce the final version of this child document.

%%%%%%%%%%%%%%%%%%%%%%%%%%%%%%%%%%%%%%%%%%%%%%%%%%%%%%%%%%%%%%%%%%%%%%%%%%%%%%%%
\subsection{Forwarding}
\label{sec:forward}

Different versions of the main or child documents
using compilation flags as described in \secref{sec:flags}
can be (permanently) stored in different files
for convenient compilation, viewing and distribution.
To this end, the package defines a command
to pass on compilation to a different file:

%%%%%%%%%%%%%%%%%%%%%%%%%%%%%%%%%%%%%%%%
\DescribeMacro{\childdocforward}
The command |\childdocforward| redirects processing to
another source file:
%
\begin{center}
\begin{tabular}{l}
|\input{childdoc.def}|\\
|\childdocforward[|\textit{main}|]{|\textit{dest}|}|\\
\end{tabular}
\end{center}
%
The argument \textit{dest} is the destination file
(without extension).
It should be the main file or one of the child files.
Note that further \textsf{childdoc} directives
such as |\childdocof| and |\childdocforward|
in the indicated file will be processed in this form.
The optional argument \textit{main}
passes on directly to the main file \textit{main}
while pretending to compile the child \textit{dest}.
This form behaves as if \textit{dest}
issues |\childdocof{|\textit{main}|}| right away,
and no further \textsf{childdoc} directives will be processed.

%%%%%%%%%%%%%%%%%%%%%%%%%%%%%%%%%%%%%%%%
\DescribeMacro{\...prefix}
In the alternative form |\childdocforwardprefix|,
%
\begin{center}
\begin{tabular}{l}
|\input{childdoc.def}|\\
|\childdocforwardprefix[|\textit{main}|]{|\textit{prefix}|}{|\textit{dest}|}|
\end{tabular}
\end{center}
%
the destination file is determined by a pattern
depending on the current file:
To make this work, the current file must be called
`{\textit{prefix}\hspace{0.2em}\textit{suffix}}'
with \textit{prefix} matching precisely the argument.
Processing is then passed on to the file
`{\textit{dest}\hspace{0.2em}\textit{suffix}}'.
Surely, the same effect is achieved by
directly specifying the
argument `{\textit{dest}\hspace{0.2em}\textit{suffix}}'
in the first form.
However, that requires to set up a different file
for each child. With the alternative form of the command
all these files can have exactly the same content
which simplifies setting them up and maintaining them.

For example, the following file |draft.tex|
with a compilation flag |\version| as described in \secref{sec:flags}
compiles the main document as a draft:
%
\begin{center}
\begin{tabular}{l}
|\def\version{draft}|\\
|\input{childdoc.def}|\\
|\childdocforward{|\textit{main}|}|
\end{tabular}
\end{center}
%
Likewise, the following files |final|\textit{nn}|.tex|
compile the final version of the child document
|child|\textit{nn}|.tex|:
%
\begin{center}
\begin{tabular}{l}
|\def\version{final}|\\
|\input{childdoc.def}|\\
|\childdocforwardprefix{final}{child}|
\end{tabular}
\end{center}
%

Note that when several versions of a main file and/or of each child file
are to be generated, it may be convenient to set up a |Makefile| or
shell script to automatise the process.

%%%%%%%%%%%%%%%%%%%%%%%%%%%%%%%%%%%%%%%%%%%%%%%%%%%%%%%%%%%%%%%%%%%%%%%%%%%%%%%%
\subsection{Command Line Processing}
\label{sec:commandline}

The effect of redirection files can also be achieved by invoking
the \LaTeX{} compiler with a more elaborate command line.
Most conveniently this should be done as part
of a shell script or a |Makefile|.

When using \textsf{childdoc} in the main file, the following
command lines effectively perform a redirection
(note that depending on the shell being used,
backslashes may have to be doubled: `|\|' $\to$ `|\\|'):
%
\begin{center}
|... -jobname "|\textit{target}|" |\\|"|[\textit{flags}]%
|\input{childdoc.def}\childdocforward[|\textit{main}|]{|\textit{dest}|}"|
\end{center}
%
Here \textit{target} is the name of the output file,
\textit{main} is the name of the main file
and \textit{dest} is the name of the main or child file to be processed
(all filenames without extensions).
The optional argument \textit{main} can be omitted
if \textit{main} matches \textit{dest}.
Optionally, compilation \textit{flags} can be defined via |\def| commands.
This command line makes the \TeX{} engine believe
it is compiling the file \textit{target}
whose content is specified as the latter parameter.
The provided code then forwards the processing to
\textit{main} or \textit{dest} as described in \secref{sec:forward}.

%%%%%%%%%%%%%%%%%%%%%%%%%%%%%%%%%%%%%%%%%%%%%%%%%%%%%%%%%%%%%%%%%%%%%%%%%%%%%%%%
\subsection{Include by Input}
\label{sec:input}

Including child documents by |\include| has some restrictions by design.
Most notably, the content of a child document always occupies
its own set of pages; pages cannot be shared between child documents.
Usually, this behaviour makes perfect sense
because each child document contain an essential part of the document.
However, in some situations it may be desirable to compose
a document from a collection of parts
without having mandatory page breaks between then.
For this case, the package
provides a mechanism to include parts
by |\input| which can also be processed individually.
However, by construction this mechanism
requires manual handling of the content to be output.

%%%%%%%%%%%%%%%%%%%%%%%%%%%%%%%%%%%%%%%%
\DescribeMacro{\ifchilddocmanual}
The main file should be prepared as usual, see \secref{sec:include}.
However, the document body must make a distinction
between processing of an individual part and of the main document, e.g.:
%
\begin{center}
\begin{tabular}{l}
|\ifchilddocmanual|\\
|\input{\childdocname}|\\
|\||else|\\
\textit{document body with }|\input{|\textit{part}|}|\\
|\||fi|
\end{tabular}
\end{center}
%
The conditional |\ifchilddocmanual| is true whenever
a part to be included by |\input| is being compiled,
and the name of the part is stored in |\childdocname|.

%%%%%%%%%%%%%%%%%%%%%%%%%%%%%%%%%%%%%%%%
\DescribeMacro{\childdocby}
Each part to be included by |\input| should start with:
%
\begin{center}
\begin{tabular}{l}
|\input{childdoc.def}|\\
|\childdocby{|\textit{main}|}|\\
\end{tabular}
\end{center}
%
The directive |\childdocby| is similar to |\childdocof|
described in \secref{sec:include},
but the subsequent selection of content must be done manually.
To that end, both |\ifchilddoc| and |\ifchilddocmanual|
will be true upon processing of a part,
and the name of the part is stored in |\childdocname|.
Note that |\jobname| will be set to the filename of the current part
so that each part receives an individual |.aux| file
that does not interfere with the |.aux| file(s) of the main document.
This behaviour can be altered by the alternative form
|\childdocby[*]{|\textit{main}|}| (with a non-empty optional argument)
which uses the |.aux| file of the main document
by setting |\jobname| to \textit{main}.

%%%%%%%%%%%%%%%%%%%%%%%%%%%%%%%%%%%%%%%%%%%%%%%%%%%%%%%%%%%%%%%%%%%%%%%%%%%%%%%%
\subsection{Driver Development}
\label{sec:driver}

The \textsf{childdoc} mechanism can also be use for the development
of definition files such as \LaTeX{} styles or classes.
This case differs from the above setup with multiple parts
included by |\include| in that no |\includeonly| should be invoked.
This can be achieved by starting the include file
(before |\ProvidesPackage|) with:
%
\begin{center}
\begin{tabular}{l}
|\input{childdoc.def}|\\
|\childdocforward{|\textit{main}|}|\\
\end{tabular}
\end{center}
%
or alternatively with:
%
\begin{center}
\begin{tabular}{l}
|\input{childdoc.def}|\\
|\childdocby{|\textit{main}|}|\\
\end{tabular}
\end{center}
%
Both forms have slightly different effects as described above.
The main file is prepared as usual, see \secref{sec:include}.

%%%%%%%%%%%%%%%%%%%%%%%%%%%%%%%%%%%%%%%%%%%%%%%%%%%%%%%%%%%%%%%%%%%%%%%%%%%%%%%%
\subsection{Legacy Detection}
\label{sec:detection}

The directive |\childdocmain| in the main file can detect
whether the complete document or merely a child is to be compiled
even without using the directive |\childdocof|.
This method is deprecated because it is less robust
and there is no compelling reason to use it;
it is merely provided for backward compatibility
and it may be removed in future versions.

If the detection mechanism is to be used,
it is mandatory to correctly specify
the filename of the main file as the argument of |\childdocmain|:
%
\begin{center}
\begin{tabular}{l}
|\input{childdoc.def}|\\
|\childdocmain{|\textit{main}|}|\\
\end{tabular}
\end{center}
%
If |\jobname| does not match the argument \textit{main} of |\childdocmain|,
it is assumed that |\jobname| points to the child file to be compiled.
When using |\childdocmain| with the main file specified as argument,
it suffices to start a child file
with just |\input{|\textit{main}|}|
without loading of the package and using |\childdocof|.
If instead all processing is done
with the appropriate \textsf{childdoc} directives,
the argument of \textit{main} of |\childdocmain| can be empty.

An alternative version of the command line processing described
in \secref{sec:commandline} using the detection mechanism reads:
%
\begin{center}
|... -jobname "|\textit{target}|" "|[\textit{flags}]%
[|\def\jobname{|\textit{dest}|}|]|\input{|\textit{main}|}"|
\end{center}

%%%%%%%%%%%%%%%%%%%%%%%%%%%%%%%%%%%%%%%%%%%%%%%%%%%%%%%%%%%%%%%%%%%%%%%%%%%%%%%%
\subsection{Manual Code}
\label{sec:manual}

In case one cannot be certain whether the definitions file |childdoc.def|
is installed on the target \TeX{} distribution
and one prefers not to ship it,
it is conceivable to paste a few relevant commands into the sources.

To that end, drop all statements |\input{childdoc.def}|
and perform the replacements as outlined below.
Instead of |\childdocmain{|\textit{main}|}| add the following code
to the top of the main file:
%
\begin{center}
\begin{tabular}{l}
|\||ifdefined\childdocname\endinput\||fi\newif\ifchilddoc|\\
|\edef\childdocname{\scantokens\expandafter{\jobname\noexpand}}|\\
|\def\childdocmain{|\textit{main}|}\||ifx\childdocmain\childdocname\||else|\\
|\childdoctrue\includeonly{\childdocname}\let\jobname\childdocmain\||fi|\\
\end{tabular}
\end{center}
%
Instead of |\childdocof{|\textit{main}|}| just include the main file
at the top of each child file:
%
\begin{center}
|\input{|\textit{main}|}|
\end{center}
%
A simple redirection |\childdocforward{|\textit{dest}|}| is achieved by:
%
\begin{center}
|\def\jobname{|\textit{dest}|}\input{\jobname}|
\end{center}
%
The redirection with prefix
|\childdocforwardprefix[|\textit{prefix}|]{|\textit{dest}|}|
is accomplished by:
%
\begin{center}
\begin{tabular}{l}
|{\edef\jobname{\scantokens\expandafter{\jobname\noexpand}}|\\
|\def\redirectjob |\textit{prefix}|#1~~~{\gdef\jobname{|\textit{dest}|#1}}|\\
|\expandafter\redirectjob\jobname~~~}\input{\jobname}|
\end{tabular}
\end{center}

In an alternative approach,
child documents can be compiled by a specific command line
without additional code or specific definitions:
%
\begin{center}
|... -jobname "|\textit{target}|" "|[\textit{flags}]%
|\includeonly{|\textit{dest}|}\input{|\textit{main}|}"|
\end{center}
%

%%%%%%%%%%%%%%%%%%%%%%%%%%%%%%%%%%%%%%%%%%%%%%%%%%%%%%%%%%%%%%%%%%%%%%%%%%%%%%%%
%%%%%%%%%%%%%%%%%%%%%%%%%%%%%%%%%%%%%%%%%%%%%%%%%%%%%%%%%%%%%%%%%%%%%%%%%%%%%%%%
\section{Information}

%%%%%%%%%%%%%%%%%%%%%%%%%%%%%%%%%%%%%%%%%%%%%%%%%%%%%%%%%%%%%%%%%%%%%%%%%%%%%%%%
\subsection{Copyright}

Copyright \copyright{} 2017--2018 Niklas Beisert

This work may be distributed and/or modified under the
conditions of the \LaTeX{} Project Public License, either version 1.3
of this license or (at your option) any later version.
The latest version of this license is in
  \url{http://www.latex-project.org/lppl.txt}
and version 1.3 or later is part of all distributions of \LaTeX{}
version 2005/12/01 or later.

This work has the LPPL maintenance status `maintained'.

The Current Maintainer of this work is Niklas Beisert.

This work consists of the files |README.txt|, |childdoc.ins| and |childdoc.dtx|
as well as the derived files |childdoc.def|, |cdocsamp.tex|
with |cdocsch1.tex|, |cdocsch2.tex|, |cdocspt3.tex|, |cdocspt4.tex|,
|cdocsdrf.tex|, |cdocsfn1.tex|, |cdocsfn2.tex|
as well as |childdoc.pdf|.

%%%%%%%%%%%%%%%%%%%%%%%%%%%%%%%%%%%%%%%%%%%%%%%%%%%%%%%%%%%%%%%%%%%%%%%%%%%%%%%%
\subsection{Files and Installation}

The package consists of the files:
%
\begin{center}
\begin{tabular}{ll}
    |README.txt|   & readme file \\
    |childdoc.ins| & installation file \\
    |childdoc.dtx| & source file \\
    |childdoc.def| & definition file \\
    |cdocsamp.tex| & sample main file \\
    |cdocsch1.tex| & sample include file \\
    |cdocsch2.tex| & sample include file \\
    |cdocspt3.tex| & sample part file \\
    |cdocspt4.tex| & sample part file \\
    |cdocsdrf.tex| & sample redirection file \\
    |cdocsfn1.tex| & sample redirection file \\
    |cdocsfn2.tex| & sample redirection file \\
    |childdoc.pdf| & manual
\end{tabular}
\end{center}
%
The distribution consists of the files
|README.txt|, |childdoc.ins| and |childdoc.dtx|.
%
\begin{itemize}
\item
Run (pdf)\LaTeX{} on |childdoc.dtx|
to compile the manual |childdoc.pdf| (this file).
\item
Run \LaTeX{} on |childdoc.ins| to create the definitions file |childdoc.def|
and the sample |cdocsamp.tex| with include files
|cdocsch1.tex|, |cdocsch2.tex|, |cdocspt3.tex|, |cdocspt4.tex|,
|cdocsdrf.tex|, |cdocsfn1.tex|, |cdocsfn2.tex|.
Then copy the file |childdoc.def| to an appropriate directory of your \LaTeX{}
distribution, e.g.\ \textit{texmf-root}|/tex/latex/childdoc|.
\end{itemize}

%%%%%%%%%%%%%%%%%%%%%%%%%%%%%%%%%%%%%%%%%%%%%%%%%%%%%%%%%%%%%%%%%%%%%%%%%%%%%%%%
\subsection{Related CTAN Packages}

There are several other packages which offer a similar functionality:
%
\begin{itemize}
\item
The packages
\href{http://ctan.org/pkg/docmute}{\textsf{docmute}},
\href{http://ctan.org/pkg/includex}{\textsf{includex}} and
\href{http://ctan.org/pkg/standalone}{\textsf{standalone}}
provide commands to include only the document body of
a child file thus allowing both files to be compiled individually.
\item
The packages \href{http://ctan.org/pkg/subdocs}{\textsf{subdocs}}
and \href{http://ctan.org/pkg/subfiles}{\textsf{subfiles}}
provide structures in which the main and child documents can be
encapsulated and allowing them to be compiled individually.
The inclusion mechanism is different from the conventional |\include|.
\item
The package \href{http://ctan.org/pkg/combine}{\textsf{combine}}
is an elaborate solution to combine several documents into one.
\end{itemize}
%
See also the CTAN topic \href{http://ctan.org/topic/subdocs}{\textsf{subdocs}}
for further related packages.
The present package differs from the above solutions in that
a document structure constructed with the conventional |\include| mechanism
just needs two extra commands at the top of every file
such that all constituent files can be compiled individually.

%%%%%%%%%%%%%%%%%%%%%%%%%%%%%%%%%%%%%%%%%%%%%%%%%%%%%%%%%%%%%%%%%%%%%%%%%%%%%%%%
%\subsection{Feature Suggestions}
%
%The following is a list of features which may be useful for future
%versions of this package:
%%
%\begin{itemize}
%\item
%\ldots
%\end{itemize}

%%%%%%%%%%%%%%%%%%%%%%%%%%%%%%%%%%%%%%%%%%%%%%%%%%%%%%%%%%%%%%%%%%%%%%%%%%%%%%%%
\subsection{Revision History}

%%%%%%%%%%%%%%%%%%%%%%%%%%%%%%%%%%%%%%%%
\paragraph{v2.0:} 2018/12/30

\begin{itemize}
\item
immediate forward processing
\item
added |\childdocby| mechanism
\item
manual restructured
\end{itemize}

%%%%%%%%%%%%%%%%%%%%%%%%%%%%%%%%%%%%%%%%
\paragraph{v1.6:} 2018/01/17

\begin{itemize}
\item
application for development of include files
\item
corrections to manual
\end{itemize}

%%%%%%%%%%%%%%%%%%%%%%%%%%%%%%%%%%%%%%%%
\paragraph{v1.5:} 2017/05/21

\begin{itemize}
\item
more complete structuring introduced
\item
|\childdocof| introduced
\item
|\childdoc| renamed to |\childdocmain|
\item
|\childredirect| renamed to |\childdocforward| and |\childdocforwardprefix|
and functionality expanded
\end{itemize}

%%%%%%%%%%%%%%%%%%%%%%%%%%%%%%%%%%%%%%%%
\paragraph{v1.0:} 2017/04/27

\begin{itemize}
\item
manual and install package
\item
first version published on CTAN
\end{itemize}

%%%%%%%%%%%%%%%%%%%%%%%%%%%%%%%%%%%%%%%%
\paragraph{v0.6:} 2017/04/26

\begin{itemize}
\item
redirection mechanism added
\end{itemize}

%%%%%%%%%%%%%%%%%%%%%%%%%%%%%%%%%%%%%%%%
\paragraph{v0.5:} 2017/04/26

\begin{itemize}
\item
functionality in definition file
\end{itemize}


%%%%%%%%%%%%%%%%%%%%%%%%%%%%%%%%%%%%%%%%%%%%%%%%%%%%%%%%%%%%%%%%%%%%%%%%%%%%%%%%
%%%%%%%%%%%%%%%%%%%%%%%%%%%%%%%%%%%%%%%%%%%%%%%%%%%%%%%%%%%%%%%%%%%%%%%%%%%%%%%%
%%%%%%%%%%%%%%%%%%%%%%%%%%%%%%%%%%%%%%%%%%%%%%%%%%%%%%%%%%%%%%%%%%%%%%%%%%%%%%%%
\appendix

\settowidth\MacroIndent{\rmfamily\scriptsize 000\ }

 \DocInput{childdoc.dtx}

\end{document}
%</driver>
% \fi
%
% %%%%%%%%%%%%%%%%%%%%%%%%%%%%%%%%%%%%%%%%%%%%%%%%%%%%%%%%%%%%%%%%%%%%%%%%%%%%%%
% %%%%%%%%%%%%%%%%%%%%%%%%%%%%%%%%%%%%%%%%%%%%%%%%%%%%%%%%%%%%%%%%%%%%%%%%%%%%%%
% \section{Sample}
%\iffalse
%<*samplemain>
%\fi
%
% The following presents a sample document
% with two chapters, two parts, a title page,
% a compile flag as well as three forwarding files to set the flag.
% It consists of eight |.tex| files:
% \begin{center}
% \begin{tabular}{ll}
% |cdocsamp.tex|&main file\\
% |cdocsch1.tex|&include file for chapter 1\\
% |cdocsch2.tex|&include file for chapter 2\\
% |cdocspt3.tex|&include file for part 3\\
% |cdocspt4.tex|&include file for part 4\\
% |cdocsdrf.tex|&forwarding file for main file in draft mode\\
% |cdocsfi1.tex|&forwarding file for final version of chapter 1\\
% |cdocsfi2.tex|&forwarding file for final version of chapter 2\\
% \end{tabular}
% \end{center}
% Each of the eight files can be compiled directly by the \LaTeX{} compiler.
%
% %%%%%%%%%%%%%%%%%%%%%%%%%%%%%%%%%%%%%%
% \paragraph{Main File.}
%
% The main file is called |cdocsamp.tex|.
%
% Load the \textsf{childdoc} definitions and
% declare the filename for the main document:
%    \begin{macrocode}
\input{childdoc.def}
\childdocmain{}
%    \end{macrocode}

% Optional override for |\version| flag:
%    \begin{macrocode}
%%\ifchilddoc\else\providecommand{\version}{draft}\fi
%    \end{macrocode}

% Define the default values for the |\version| flag
% (|final| for the main file and |draft| for childs):
%    \begin{macrocode}
\ifchilddoc
\providecommand{\version}{draft}
\else
\providecommand{\version}{final}
\fi
%    \end{macrocode}

% Load the standard document class:
%    \begin{macrocode}
\documentclass[12pt]{article}
%    \end{macrocode}

% Start the document body:
%    \begin{macrocode}
\begin{document}
%    \end{macrocode}

% Declare a title page.
% Print title, part of document being processed and version flag:
%    \begin{macrocode}
\addtocounter{page}{-1}
\begin{center}
{\LARGE\bfseries{}childdoc example\par}
\vspace{1cm}
\ifchilddoc
\ifchilddocmanual part\else chapter\fi:
`\childdocname' of `\childdocjob'\par
\else
main document: `\childdocjob'\par
\fi
version: \version\par
\end{center}
\newpage
%    \end{macrocode}

% Manually include selected file,
% otherwise process as usual:
%    \begin{macrocode}
\ifchilddocmanual
\section*{part `\childdocname'}
\input{\childdocname}
\else
%    \end{macrocode}

% Include the two chapters:
%    \begin{macrocode}
\include{cdocsch1}
\include{cdocsch2}
%    \end{macrocode}

% Include the two parts unless only chapters should be displayed:
%    \begin{macrocode}
\ifchilddoc\else
\section{part three}
\input{cdocspt3}
\section{part four}
\input{cdocspt4}
\fi
%    \end{macrocode}

% Process as usual until here:
%    \begin{macrocode}
\fi
%    \end{macrocode}

% End of document body:
%    \begin{macrocode}
\end{document}
%    \end{macrocode}
%\iffalse
%</samplemain>
%\fi
%
% %%%%%%%%%%%%%%%%%%%%%%%%%%%%%%%%%%%%%%
% \paragraph{Chapter Include Files.}
%
% The include files are called |cdocsch1.tex| and |cdocsch2.tex|.
%
%\iffalse
%<*samplechap1|samplechap2>
%\fi

% Optional override for |\version| flag:
%    \begin{macrocode}
%%\providecommand{\version}{final}
%    \end{macrocode}

% Include the main document:
%    \begin{macrocode}
\input{childdoc.def}
\childdocof{cdocsamp}
%    \end{macrocode}

%\iffalse
%</samplechap1|samplechap2>
%\fi
%
%\iffalse
%<*samplechap1>
%\fi
% Some text for chapter 1:
%    \begin{macrocode}
\section{one}
some text in chapter one
%    \end{macrocode}

%\iffalse
%</samplechap1>
%\fi
% Some text for chapter 2:
%\iffalse
%<*samplechap2>
%\fi
%    \begin{macrocode}
\section{two}
more text in chapter two
%    \end{macrocode}

%\iffalse
%</samplechap2>
%\fi
%
% %%%%%%%%%%%%%%%%%%%%%%%%%%%%%%%%%%%%%%
% \paragraph{Part Include Files.}
%
% The include files are called |cdocspt3.tex| and |cdocspt4.tex|.
%
%\iffalse
%<*samplepart3|samplepart4>
%\fi

% Optional override for |\version| flag:
%    \begin{macrocode}
%%\providecommand{\version}{final}
%    \end{macrocode}

% Include the main document:
%    \begin{macrocode}
\input{childdoc.def}
\childdocby{cdocsamp}
%    \end{macrocode}

%\iffalse
%</samplepart3|samplepart4>
%\fi
%
%\iffalse
%<*samplepart3>
%\fi
% Some text for part 3:
%    \begin{macrocode}
some text in part three
%    \end{macrocode}

%\iffalse
%</samplepart3>
%\fi
% Some text for part 4:
%\iffalse
%<*samplepart4>
%\fi
%    \begin{macrocode}
more text in part four
%    \end{macrocode}

%\iffalse
%</samplepart4>
%\fi
%
% %%%%%%%%%%%%%%%%%%%%%%%%%%%%%%%%%%%%%%
% \paragraph{Forwarding for a Complete Draft.}
%
% The following forwarding file |cdocsdrf.tex|
% compiles the main document in draft mode:
%\iffalse
%<*sampledraft>
%\fi
%    \begin{macrocode}
\def\version{draft}
\input{childdoc.def}
\childdocforward{cdocsamp}
%    \end{macrocode}

%\iffalse
%</sampledraft>
%\fi
%
% %%%%%%%%%%%%%%%%%%%%%%%%%%%%%%%%%%%%%%
% \paragraph{Forwarding for Final Version of the Chapters.}
%
% The following forwarding files |cdocsfn1.tex| and |cdocsfn2.tex|
% (with identical content)
% compile the final versions of the child documents
% |cdocsch1.tex| and |cdocsch2.tex|, respectively:
%\iffalse
%<*samplefinal>
%\fi
%    \begin{macrocode}
\def\version{final}
\input{childdoc.def}
\childdocforwardprefix[cdocsamp]{cdocsfn}{cdocsch}
%    \end{macrocode}

%\iffalse
%</samplefinal>
%\fi
%
% %%%%%%%%%%%%%%%%%%%%%%%%%%%%%%%%%%%%%%
% \paragraph{Command Line Processing.}
%
% The following three command lines generate the output files
% |cdocscld|, |cdocscl1| and |cdocscl2|
% which should be identical to
% |cdocsdrf|, |cdocsch1| and |cdocsfn2|, respectively:
% \begin{center}
% \begin{tabular}{l}
% |latex -jobname cdocscld \|\\
% |  "\def\version{draft}\input{childdoc.def}\childdocforward{cdocsamp}"|\\
% |latex -jobname cdocscl1 \|\\
% |  "\input{childdoc.def}\childdocforward[cdocsamp]{cdocsch1}"|\\
% |latex -jobname cdocscl2 \|\\
% |  "\def\version{final}\input{childdoc.def}\childdocforward{cdocsch2}"|
% \end{tabular}
% \end{center}
% Note that the trailing backslash on each first line
% merely continues the input to the second line
% (for convenient cut ant paste).
% Furthermore, the command |latex| can be replaced by any
% of its alternative versions such as |pdflatex|.
%
% %%%%%%%%%%%%%%%%%%%%%%%%%%%%%%%%%%%%%%%%%%%%%%%%%%%%%%%%%%%%%%%%%%%%%%%%%%%%%%
% %%%%%%%%%%%%%%%%%%%%%%%%%%%%%%%%%%%%%%%%%%%%%%%%%%%%%%%%%%%%%%%%%%%%%%%%%%%%%%
% \section{Implementation}
%\iffalse
%<*package>
%\fi
%
% This section describes the definitions file |childdoc.def|.

% The definitions cannot be loaded using |\usepackage| or |\RequirePackage|
% which has a mechanism to prevent loading a style file more than once.
% When loading the definitions by means of |\input|
% multiple instances have to be prevented manually:
%\iffalse
%This code needs to be before the `\ProvidesFile' directive
%which is defined at the beginning of this file.
%Therefore it is also placed there and commented out here.
%</package>
%<*discard>
%\fi
%    \begin{macrocode}
\ifdefined\childdocmain\endinput\fi
%    \end{macrocode}
%\iffalse
%</discard>
%<*package>
%\fi
%
% \macro{\ifchilddoc}
% \macro{\ifchilddocmanual}
% The conditional |\ifchilddoc| tells whether a
% child (true) or main (false) document is being compiled.
% The conditional |\ifchilddocmanual| tells whether
% the |\includeonly| mechanism is used (false) or
% the selection of child files must be performed manually (true).
% The definitions initialise to false:
%    \begin{macrocode}
\newif\ifchilddoc
\newif\ifchilddocmanual
%    \end{macrocode}

% \macro{\childdocname}
% \macro{\childdocjob}
% The macro |\childdocname| stores the name of the main document
% to be compiled. The macro |\childdocjob| stores the name of
% the document on which the \LaTeX{} compiler was originally invoked.
% The content of |\jobname| cannot be compared
% to filenames specified in the source due to different catcodes.
% The following code rescans |\jobname|, stores the result
% in |\childdocname| and saves a copy in |\childdocjob|:
%    \begin{macrocode}
\edef\childdocname{\scantokens\expandafter{\jobname\noexpand}}
\let\childdocjob\childdocname
%    \end{macrocode}

% \macro{\childdocdisable}
% The macro |\childdocdisable| prevents the main file
% from being processed more than once.
% At this stage, the main document command |\childdocmain|
% is assumed to be called once again where it should do nothing.
% Any subsequent call to it should prevent
% a secondary processing of the main document
% It overwrites the forwarding commands
% |\childdocof| and |\childdocforward|
% with empty macros to prevent further inclusions of the main document:
%    \begin{macrocode}
\newcommand{\childdocdisable}
{
  \renewcommand{\childdocmain}[1]{\renewcommand{\childdocmain}[1]{\endinput}}
  \renewcommand{\childdocof}[1]{}
  \renewcommand{\childdocby}[2][]{}
  \renewcommand{\childdocforward}[2][]{}
  \renewcommand{\childdocdisable}{}
}
%    \end{macrocode}

% \macro{\childdocmain}
% The macro |\childdocmain| is to be called at the top of the main file
% with nothing or the main filename (without extension) as argument.
% First, it breaks loops.
% If the argument is not empty and does not match |\childdocname|
% (which is set by the first inclusion of |childdoc.def|),
% |\ifchilddoc| is set to true, |\includeonly| is applied to the child file
% and |\jobname| is set to the main file
% (for proper handling of |.aux| files):
%    \begin{macrocode}
\newcommand{\childdocmain}[1]
{
  \childdocdisable\childdocmain{}
  \if?#1?\else
    \begingroup
      \def\childdoctmp{#1}
      \ifx\childdoctmp\childdocname
        \def\childdoctmp{}
      \else
        \def\childdoctmp
        {
          \childdoctrue
          \includeonly{\childdocname}
          \def\childdocjob{#1}
          \def\jobname{#1}
        }
      \fi
      \expandafter
    \endgroup
    \childdoctmp
  \fi
}
%    \end{macrocode}

% \macro{\childdocof}
% The command |\childdocof| redirects
% compilation to the main file |#1|.
%    \begin{macrocode}
\newcommand{\childdocof}[1]
{
  \childdocdisable
  \childdoctrue
  \includeonly{\childdocname}
  \def\jobname{#1}
  \def\childdocjob{#1}
  \input{#1}
}
%    \end{macrocode}

% \macro{\childdocby}
% The command |\childdocby| ....
%    \begin{macrocode}
\newcommand{\childdocby}[2][]
{
  \childdocdisable
  \childdoctrue
  \childdocmanualtrue
  \if?#1?\else
    \def\jobname{#2}
  \fi
  \def\childdocjob{#2}
  \input{#2}
  \endinput
}
%    \end{macrocode}

% \macro{\childdocforward}
% The command |\childdocforward| redirects
% compilation to the main file or
% (if the optional argument is given) a child file.
% Parameters are set as if the main file
% or a child file starting with |\childdocof| was compiled.
% Then compilation is handed over to the main file:
%    \begin{macrocode}
\newcommand{\childdocforward}[2][]
{
  \begingroup
    \if?#1?
      \def\childdoctmp
      {
        \def\childdocname{#2}
        \def\childdocjob{#2}
        \def\jobname{#2}
        \input{#2}
        \endinput
      }
    \else
      \def\childdoctmp
      {
        \childdocdisable
        \def\childdocname{#2}
        \childdoctrue
        \includeonly{#2}
        \def\childdocjob{#1}
        \def\jobname{#1}
        \input{#1}
        \endinput
      }
    \fi
    \expandafter
  \endgroup
  \childdoctmp
}
%    \end{macrocode}

% \macro{\childdocforwardprefix}
% The command |\childdocforwardprefix| redirects
% compilation to the main or a child file by means of a pattern.
% The prefix |#1| in the current filename is replaced by |#2|
% and the suffix of the current filename is kept
% (it is assumed that the filename does not contain the substring `|~~~|'
% which is used as a delimiter).
% Compilation is handed over to the new file by |\childdocforward|:
%    \begin{macrocode}
\newcommand{\childdocforwardprefix}[3][]
{
  \begingroup
    \def\childdocextract #2##1~~~{\def\childdoctmp{\childdocforward[#1]{#3##1}}}
    \expandafter\childdocextract\childdocname~~~
    \expandafter
  \endgroup
  \childdoctmp
}
%    \end{macrocode}

% \macro{\childdoc}
% The deprecated macro |\childdoc| is a legacy version of |\childdocmain|:
%    \begin{macrocode}
\newcommand{\childdoc}{\childdocmain}
%    \end{macrocode}

% \macro{\childdocredirect}
% The deprecated macro |\childdocredirect| is a legacy version
% of |\childdocforward| and |\childdocforwardprefix|:
%    \begin{macrocode}
\newcommand{\childdocredirect}[2][]
{
  \begingroup
    \if?#1?
      \def\childdoctmp{\childdocforward{#2}}
    \else
      \def\childdoctmp{\childdocforwardprefix{#1}{#2}}
    \fi
    \expandafter
  \endgroup
  \childdoctmp
}
%    \end{macrocode}

%\iffalse
%</package>
%\fi
%
\endinput
\childdocforward[|\textit{main}|]{|\textit{dest}|}"|
\end{center}
%
Here \textit{target} is the name of the output file,
\textit{main} is the name of the main file
and \textit{dest} is the name of the main or child file to be processed
(all filenames without extensions).
The optional argument \textit{main} can be omitted
if \textit{main} matches \textit{dest}.
Optionally, compilation \textit{flags} can be defined via |\def| commands.
This command line makes the \TeX{} engine believe
it is compiling the file \textit{target}
whose content is specified as the latter parameter.
The provided code then forwards the processing to
\textit{main} or \textit{dest} as described in \secref{sec:forward}.

%%%%%%%%%%%%%%%%%%%%%%%%%%%%%%%%%%%%%%%%%%%%%%%%%%%%%%%%%%%%%%%%%%%%%%%%%%%%%%%%
\subsection{Include by Input}
\label{sec:input}

Including child documents by |\include| has some restrictions by design.
Most notably, the content of a child document always occupies
its own set of pages; pages cannot be shared between child documents.
Usually, this behaviour makes perfect sense
because each child document contain an essential part of the document.
However, in some situations it may be desirable to compose
a document from a collection of parts
without having mandatory page breaks between then.
For this case, the package
provides a mechanism to include parts
by |\input| which can also be processed individually.
However, by construction this mechanism
requires manual handling of the content to be output.

%%%%%%%%%%%%%%%%%%%%%%%%%%%%%%%%%%%%%%%%
\DescribeMacro{\ifchilddocmanual}
The main file should be prepared as usual, see \secref{sec:include}.
However, the document body must make a distinction
between processing of an individual part and of the main document, e.g.:
%
\begin{center}
\begin{tabular}{l}
|\ifchilddocmanual|\\
|\input{\childdocname}|\\
|\||else|\\
\textit{document body with }|\input{|\textit{part}|}|\\
|\||fi|
\end{tabular}
\end{center}
%
The conditional |\ifchilddocmanual| is true whenever
a part to be included by |\input| is being compiled,
and the name of the part is stored in |\childdocname|.

%%%%%%%%%%%%%%%%%%%%%%%%%%%%%%%%%%%%%%%%
\DescribeMacro{\childdocby}
Each part to be included by |\input| should start with:
%
\begin{center}
\begin{tabular}{l}
|% \iffalse
%
% childdoc.dtx Copyright (C) 2017-2018 Niklas Beisert
%
% This work may be distributed and/or modified under the
% conditions of the LaTeX Project Public License, either version 1.3
% of this license or (at your option) any later version.
% The latest version of this license is in
%   http://www.latex-project.org/lppl.txt
% and version 1.3 or later is part of all distributions of LaTeX
% version 2005/12/01 or later.
%
% This work has the LPPL maintenance status `maintained'.
%
% The Current Maintainer of this work is Niklas Beisert.
%
% This work consists of the files childdoc.dtx and childdoc.ins
% and the derived files childdoc.def and cdocsamp.tex with
% cdocsch1.tex, cdocsch2.tex, cdocsdrf.tex, cdocsfn1.tex, cdocsfn2.tex.
%
%<package>\ifdefined\childdocmain\endinput\fi
%<package>\ProvidesFile{childdoc.def}[2018/12/30 v2.0 child document driver]
%<samplemain>\ProvidesFile{cdocsamp.tex}[2018/12/30 v2.0 sample for childdoc]
%<*driver>
%\ProvidesFile{childdoc.drv}[2018/12/30 v2.0 childdoc reference manual file]
\PassOptionsToClass{10pt,a4paper}{article}
\documentclass{ltxdoc}

\usepackage[margin=35mm]{geometry}
\usepackage{hyperref}
\usepackage{hyperxmp}
\usepackage[usenames]{color}

\hypersetup{colorlinks=true}
\hypersetup{pdfstartview=FitH}
\hypersetup{pdfpagemode=UseNone}
\hypersetup{pdfsource={}}
\hypersetup{pdflang={en-UK}}
\hypersetup{pdfcopyright={Copyright 2017-2018 Niklas Beisert.
  This work may be distributed and/or modified under the
  conditions of the LaTeX Project Public License, either version 1.3
  of this license or (at your option) any later version.}}
\hypersetup{pdflicenseurl={http://www.latex-project.org/lppl.txt}}
\hypersetup{pdfcontactaddress={ETH Zurich, ITP, HIT K,
  Wolfgang-Pauli-Strasse 27}}
\hypersetup{pdfcontactpostcode={8093}}
\hypersetup{pdfcontactcity={Zurich}}
\hypersetup{pdfcontactcountry={Switzerland}}
\hypersetup{pdfcontactemail={nbeisert@itp.phys.ethz.ch}}
\hypersetup{pdfcontacturl={http://people.phys.ethz.ch/\xmptilde nbeisert/}}

\newcommand{\secref}[1]{\hyperref[#1]{section \ref*{#1}}}

\parskip1ex
\parindent0pt
\let\olditemize\itemize
\def\itemize{\olditemize\parskip0pt}

\begin{document}

\title{The \textsf{childdoc} Package}
\hypersetup{pdftitle={The childdoc Package}}
\author{Niklas Beisert\\[2ex]
  Institut f\"ur Theoretische Physik\\
  Eidgen\"ossische Technische Hochschule Z\"urich\\
  Wolfgang-Pauli-Strasse 27, 8093 Z\"urich, Switzerland\\[1ex]
  \href{mailto:nbeisert@itp.phys.ethz.ch}
  {\texttt{nbeisert@itp.phys.ethz.ch}}}
\hypersetup{pdfauthor={Niklas Beisert}}
\hypersetup{pdfsubject={Manual for the LaTeX2e Package childdoc}}
\date{30 December 2018, \textsf{v2.0}}
\maketitle

\begin{abstract}\noindent
\textsf{childdoc} is a \LaTeXe{} package
that enables the direct compilation
of document sections included by |\include|
to individual files.
\end{abstract}

\begingroup
\parskip0ex
\tableofcontents
\endgroup

%%%%%%%%%%%%%%%%%%%%%%%%%%%%%%%%%%%%%%%%%%%%%%%%%%%%%%%%%%%%%%%%%%%%%%%%%%%%%%%%
%%%%%%%%%%%%%%%%%%%%%%%%%%%%%%%%%%%%%%%%%%%%%%%%%%%%%%%%%%%%%%%%%%%%%%%%%%%%%%%%
\section{Introduction}

\LaTeX{} provides a mechanism to structure a large document (such as a book)
into a main file and several child files (containing the chapters)
using the |\include| command.
This mechanism is beneficial for documents
which span hundreds of pages in order to
make the source file(s) more manageable.
Moreover, compilation can be restricted to
selected child files by means of the |\includeonly| command.
The latter feature can be used to reduce the compilation time while editing
(this was significantly more useful in the earlier days of \LaTeX{})
or to generate a smaller document which is easier to navigate.
Another application of |\includeonly| is to generate
documents consisting of selected parts of the complete document.

However, there are a few drawbacks of the plain |\include| mechanism:
\begin{itemize}
\item
The child files cannot be compiled on their own,
they can only be compiled via the main file.
A naive editing environment
(such as a text editor with an option
to have the current file processed by \LaTeX)
may require one to switch to the main file before compiling;
attempting to compile the child file produces errors.
\item
The main file must be modified (each time)
to adjust the |\includeonly| command
to the present needs. This easily leaves the main file in a messy state.
\item
The generated document will always carry the filename
of the main document. This is inconvenient if
several child files are to be compiled and
to be kept for distribution.
\end{itemize}

The present package provides a simple interface
to make child files individually compilable by \LaTeX{}.
Compiling a child file then has the same effect as compiling
the main file with an |\includeonly| command
to select the appropriate child.
Moreover the generated document will carry the name of the child
rather than the main file.
This resolves all three above issues.

This feature is meant to make the editing of books,
thesis documents and lecture notes somewhat more convenient.
However, the package can also be used efficiently for
composing a series of documents (such as exercise sheets)
which are typically distributed individually.
It then assists the author in generating the individual documents
(potentially in different versions)
as well as a document containing the collected series.
Another application is in developing style files
or other kinds of included material
where compilation of the style file could redirect
to a sample or test file.

%%%%%%%%%%%%%%%%%%%%%%%%%%%%%%%%%%%%%%%%%%%%%%%%%%%%%%%%%%%%%%%%%%%%%%%%%%%%%%%%
%%%%%%%%%%%%%%%%%%%%%%%%%%%%%%%%%%%%%%%%%%%%%%%%%%%%%%%%%%%%%%%%%%%%%%%%%%%%%%%%
\section{Usage}

First of all, the package \textsf{childdoc} is \emph{not} a standard
\LaTeXe{} |.sty| style file! Therefore it needs to be invoked in
a non-standard way.

%%%%%%%%%%%%%%%%%%%%%%%%%%%%%%%%%%%%%%%%%%%%%%%%%%%%%%%%%%%%%%%%%%%%%%%%%%%%%%%%
\subsection{Included Files}
\label{sec:include}

%%%%%%%%%%%%%%%%%%%%%%%%%%%%%%%%%%%%%%%%
\DescribeMacro{\childdocmain}
To use the package, add the commands
\begin{center}
\begin{tabular}{l}
|\input{childdoc.def}|\\
|\childdocmain{}|\\
\end{tabular}
\end{center}
at the very top of the main \LaTeX{} file,
in particular \emph{before} the |\documentclass| statement!
The argument of |\childdocmain| should be left empty
(but it must be present).

%%%%%%%%%%%%%%%%%%%%%%%%%%%%%%%%%%%%%%%%
\DescribeMacro{\childdocof}
Furthermore, add the commands
\begin{center}
\begin{tabular}{l}
|\input{childdoc.def}|\\
|\childdocof{|\textit{main}|}|\\
\end{tabular}
\end{center}
at the top of every child file \textit{child}
which is included by |\include{|\textit{child}|}|
from within the main file
(or at least for those files to be compiled individually).
The argument \textit{main} must be the filename of the main file.

There are a couple of
considerations in setting up the main and child documents:

%%%%%%%%%%%%%%%%%%%%%%%%%%%%%%%%%%%%%%%%
\paragraph{Restrictions.}

Please note the following restrictions:
\begin{itemize}
\item
|\childdocmain| must be called with one argument \textit{main}
to ensure compatibility with earlier version of the package.
It must either be empty (|\childdocmain{}|)
or precisely match the filename of the main file in which it is specified.
See \secref{sec:detection} for further information.
\item
The filename \textit{main} must be specified without the |.tex| extension.
\item
The filename \textit{main} is case sensitive
(even in case-insensitive file systems)
due to internal string comparison.
\item
The argument \textit{main} should be fully expanded, it cannot be a macro.
\item
Subdirectories and special characters should be avoided in filenames.
\item
The command |\childdocmain{|\textit{main}|}| must be followed by a whitespace.
It should not be followed immediately by another command
or by a comment mark `|%|'.
This is because the \TeX{} parser reads the token immediately following
the argument of |\childdocmain| and puts it
at the beginning of every child section;
however, a white\-space is ignored.
\end{itemize}

%%%%%%%%%%%%%%%%%%%%%%%%%%%%%%%%%%%%%%%%
\paragraph{Content of Main File.}

It is advisable to place all content in the child files included by |\include|.
Any output contained in the main file will appear in all child documents
unless suppressed manually;
it cannot be suppressed automatically by the |\includeonly| directive
and thus should normally be avoided.
A method to include some content in the main file
by means of conditional processing is described in \secref{sec:conditional}.

%%%%%%%%%%%%%%%%%%%%%%%%%%%%%%%%%%%%%%%%
\paragraph{Page Numbering.}

When only a part of the document is compiled,
the appropriate numbering of pages
(as well as other status parameters)
is determined from the |.aux| files.
The latter contain information from previous passes.
However this information needs to propagate through
all intermediate child documents.
Therefore the page numbering in child documents may well
be inconsistent until the complete document is compiled at least once.

A useful (if unconventional) way to always ensure a consistent
page numbering is to restart the numbering in each child document
and denote the pages by `\textit{child}|.|\textit{page}'
where \textit{child} represents the chapter/section number of the child file.
This can be achieved by the command
|\numberwithin{page}{|\textit{child}|}|
of the \textsf{amsmath} package
where \textit{child} can be |chapter| or |section|
depending on the chosen structuring.
Alternatively, one can modify the macro |\thepage| appropriately
and reset the counter |page| at the start of each child file.

%%%%%%%%%%%%%%%%%%%%%%%%%%%%%%%%%%%%%%%%%%%%%%%%%%%%%%%%%%%%%%%%%%%%%%%%%%%%%%%%
\subsection{Conditional Processing}
\label{sec:conditional}

The package provides a mechanism to compile different versions
of a document. To customise the versions further some conditional processing
can come in handy to distinguish which version is being compiled.
The package provides two macros to describe the compilation context:

%%%%%%%%%%%%%%%%%%%%%%%%%%%%%%%%%%%%%%%%
\DescribeMacro{\ifchilddoc}
The conditional |\ifchilddoc| distinguishes between the compilation of
child documents and the main document:
%
\begin{center}
|\ifchilddoc |\textit{child-code}| |[|\||else |\textit{main-code}]| \||fi|
\end{center}

%%%%%%%%%%%%%%%%%%%%%%%%%%%%%%%%%%%%%%%%
\DescribeMacro{\childdocname}
\DescribeMacro{\childdocjob}
The macro |\childdocname| contains the filename (without extension)
of the main or child file being processed.
Note that |\childdocjob| will always contain the name of the main file.

%%%%%%%%%%%%%%%%%%%%%%%%%%%%%%%%%%%%%%%%
\paragraph{Title Page.}

Conditional processing can be used to include a title or banner page
in the main document when proper precautions are taken.
Importantly, the code in the main file should ensure that the page counter
(as well as other status parameters which are stored in the |.aux| files)
takes the same value after the conditional processing.
Otherwise the page numbers may take divergent values
depending on which part is compiled.

For example, a title page could be declared by:
%
\begin{center}
\begin{tabular}{l}
|\ifchilddoc\||else|\\
|\addtocounter{page}{-1}|\\
\textit{code for title page}\\
|\newpage|\\
|\||fi|
\end{tabular}
\end{center}
%
A banner page for the child documents can be generated by:
%
\begin{center}
\begin{tabular}{l}
|\ifchilddoc|\\
|\addtocounter{page}{-1}|\\
\textit{code for banner page}\\
|\newpage|\\
|\||fi|
\end{tabular}
\end{center}
%
Here one could write a message such as:
\begin{center}
|This is the part \childdocname{} of \childdocjob{}.|
\end{center}

%%%%%%%%%%%%%%%%%%%%%%%%%%%%%%%%%%%%%%%%%%%%%%%%%%%%%%%%%%%%%%%%%%%%%%%%%%%%%%%%
\subsection{Flags}
\label{sec:flags}

The package makes it easy to generate different versions
of the main or child documents.
To this end compilation flags can be defined
and assigned different default values.
They will be particularly useful in conjunction
with the forwarding mechanism described in \secref{sec:forward}.

For example, it may be useful to have a flag |\version|
which can be set to |draft| or |final|.
The document source will contain some conditional code
depending on the value of |\version|.
Suppose further, the flag should default to |final| for the main file
and to |draft| for child files
which is a natural assignment for editing the document.
This is achieved by placing the following code
in the preamble of the main document
(below the |\childdocmain| directive):
%
\begin{center}
\begin{tabular}{l}
|\ifchilddoc|\\
|\providecommand{\version}{draft}|\\
|\||else|\\
|\providecommand{\version}{final}|\\
|\||fi|
\end{tabular}
\end{center}
%
The definition by |\providecommand| makes sure
that previous definitions are not overwritten.
Further statements |\providecommand{\version}{...}|
can thus be added before the above code to override it.

For the main file, one might add a line
(between |\childdocmain| and the above block)
%
\begin{center}
|%\ifchilddoc\||else\providecommand{\version}{draft}\||fi|
\end{center}
%
which can be uncommented to produce a draft version.
Likewise one can add a line to the very top of a child file
(above the |\childdocof{|\textit{main}|}| directive)
%
\begin{center}
|%\providecommand{\version}{final}|
\end{center}
%
which can be uncommented to produce the final version of this child document.

%%%%%%%%%%%%%%%%%%%%%%%%%%%%%%%%%%%%%%%%%%%%%%%%%%%%%%%%%%%%%%%%%%%%%%%%%%%%%%%%
\subsection{Forwarding}
\label{sec:forward}

Different versions of the main or child documents
using compilation flags as described in \secref{sec:flags}
can be (permanently) stored in different files
for convenient compilation, viewing and distribution.
To this end, the package defines a command
to pass on compilation to a different file:

%%%%%%%%%%%%%%%%%%%%%%%%%%%%%%%%%%%%%%%%
\DescribeMacro{\childdocforward}
The command |\childdocforward| redirects processing to
another source file:
%
\begin{center}
\begin{tabular}{l}
|\input{childdoc.def}|\\
|\childdocforward[|\textit{main}|]{|\textit{dest}|}|\\
\end{tabular}
\end{center}
%
The argument \textit{dest} is the destination file
(without extension).
It should be the main file or one of the child files.
Note that further \textsf{childdoc} directives
such as |\childdocof| and |\childdocforward|
in the indicated file will be processed in this form.
The optional argument \textit{main}
passes on directly to the main file \textit{main}
while pretending to compile the child \textit{dest}.
This form behaves as if \textit{dest}
issues |\childdocof{|\textit{main}|}| right away,
and no further \textsf{childdoc} directives will be processed.

%%%%%%%%%%%%%%%%%%%%%%%%%%%%%%%%%%%%%%%%
\DescribeMacro{\...prefix}
In the alternative form |\childdocforwardprefix|,
%
\begin{center}
\begin{tabular}{l}
|\input{childdoc.def}|\\
|\childdocforwardprefix[|\textit{main}|]{|\textit{prefix}|}{|\textit{dest}|}|
\end{tabular}
\end{center}
%
the destination file is determined by a pattern
depending on the current file:
To make this work, the current file must be called
`{\textit{prefix}\hspace{0.2em}\textit{suffix}}'
with \textit{prefix} matching precisely the argument.
Processing is then passed on to the file
`{\textit{dest}\hspace{0.2em}\textit{suffix}}'.
Surely, the same effect is achieved by
directly specifying the
argument `{\textit{dest}\hspace{0.2em}\textit{suffix}}'
in the first form.
However, that requires to set up a different file
for each child. With the alternative form of the command
all these files can have exactly the same content
which simplifies setting them up and maintaining them.

For example, the following file |draft.tex|
with a compilation flag |\version| as described in \secref{sec:flags}
compiles the main document as a draft:
%
\begin{center}
\begin{tabular}{l}
|\def\version{draft}|\\
|\input{childdoc.def}|\\
|\childdocforward{|\textit{main}|}|
\end{tabular}
\end{center}
%
Likewise, the following files |final|\textit{nn}|.tex|
compile the final version of the child document
|child|\textit{nn}|.tex|:
%
\begin{center}
\begin{tabular}{l}
|\def\version{final}|\\
|\input{childdoc.def}|\\
|\childdocforwardprefix{final}{child}|
\end{tabular}
\end{center}
%

Note that when several versions of a main file and/or of each child file
are to be generated, it may be convenient to set up a |Makefile| or
shell script to automatise the process.

%%%%%%%%%%%%%%%%%%%%%%%%%%%%%%%%%%%%%%%%%%%%%%%%%%%%%%%%%%%%%%%%%%%%%%%%%%%%%%%%
\subsection{Command Line Processing}
\label{sec:commandline}

The effect of redirection files can also be achieved by invoking
the \LaTeX{} compiler with a more elaborate command line.
Most conveniently this should be done as part
of a shell script or a |Makefile|.

When using \textsf{childdoc} in the main file, the following
command lines effectively perform a redirection
(note that depending on the shell being used,
backslashes may have to be doubled: `|\|' $\to$ `|\\|'):
%
\begin{center}
|... -jobname "|\textit{target}|" |\\|"|[\textit{flags}]%
|\input{childdoc.def}\childdocforward[|\textit{main}|]{|\textit{dest}|}"|
\end{center}
%
Here \textit{target} is the name of the output file,
\textit{main} is the name of the main file
and \textit{dest} is the name of the main or child file to be processed
(all filenames without extensions).
The optional argument \textit{main} can be omitted
if \textit{main} matches \textit{dest}.
Optionally, compilation \textit{flags} can be defined via |\def| commands.
This command line makes the \TeX{} engine believe
it is compiling the file \textit{target}
whose content is specified as the latter parameter.
The provided code then forwards the processing to
\textit{main} or \textit{dest} as described in \secref{sec:forward}.

%%%%%%%%%%%%%%%%%%%%%%%%%%%%%%%%%%%%%%%%%%%%%%%%%%%%%%%%%%%%%%%%%%%%%%%%%%%%%%%%
\subsection{Include by Input}
\label{sec:input}

Including child documents by |\include| has some restrictions by design.
Most notably, the content of a child document always occupies
its own set of pages; pages cannot be shared between child documents.
Usually, this behaviour makes perfect sense
because each child document contain an essential part of the document.
However, in some situations it may be desirable to compose
a document from a collection of parts
without having mandatory page breaks between then.
For this case, the package
provides a mechanism to include parts
by |\input| which can also be processed individually.
However, by construction this mechanism
requires manual handling of the content to be output.

%%%%%%%%%%%%%%%%%%%%%%%%%%%%%%%%%%%%%%%%
\DescribeMacro{\ifchilddocmanual}
The main file should be prepared as usual, see \secref{sec:include}.
However, the document body must make a distinction
between processing of an individual part and of the main document, e.g.:
%
\begin{center}
\begin{tabular}{l}
|\ifchilddocmanual|\\
|\input{\childdocname}|\\
|\||else|\\
\textit{document body with }|\input{|\textit{part}|}|\\
|\||fi|
\end{tabular}
\end{center}
%
The conditional |\ifchilddocmanual| is true whenever
a part to be included by |\input| is being compiled,
and the name of the part is stored in |\childdocname|.

%%%%%%%%%%%%%%%%%%%%%%%%%%%%%%%%%%%%%%%%
\DescribeMacro{\childdocby}
Each part to be included by |\input| should start with:
%
\begin{center}
\begin{tabular}{l}
|\input{childdoc.def}|\\
|\childdocby{|\textit{main}|}|\\
\end{tabular}
\end{center}
%
The directive |\childdocby| is similar to |\childdocof|
described in \secref{sec:include},
but the subsequent selection of content must be done manually.
To that end, both |\ifchilddoc| and |\ifchilddocmanual|
will be true upon processing of a part,
and the name of the part is stored in |\childdocname|.
Note that |\jobname| will be set to the filename of the current part
so that each part receives an individual |.aux| file
that does not interfere with the |.aux| file(s) of the main document.
This behaviour can be altered by the alternative form
|\childdocby[*]{|\textit{main}|}| (with a non-empty optional argument)
which uses the |.aux| file of the main document
by setting |\jobname| to \textit{main}.

%%%%%%%%%%%%%%%%%%%%%%%%%%%%%%%%%%%%%%%%%%%%%%%%%%%%%%%%%%%%%%%%%%%%%%%%%%%%%%%%
\subsection{Driver Development}
\label{sec:driver}

The \textsf{childdoc} mechanism can also be use for the development
of definition files such as \LaTeX{} styles or classes.
This case differs from the above setup with multiple parts
included by |\include| in that no |\includeonly| should be invoked.
This can be achieved by starting the include file
(before |\ProvidesPackage|) with:
%
\begin{center}
\begin{tabular}{l}
|\input{childdoc.def}|\\
|\childdocforward{|\textit{main}|}|\\
\end{tabular}
\end{center}
%
or alternatively with:
%
\begin{center}
\begin{tabular}{l}
|\input{childdoc.def}|\\
|\childdocby{|\textit{main}|}|\\
\end{tabular}
\end{center}
%
Both forms have slightly different effects as described above.
The main file is prepared as usual, see \secref{sec:include}.

%%%%%%%%%%%%%%%%%%%%%%%%%%%%%%%%%%%%%%%%%%%%%%%%%%%%%%%%%%%%%%%%%%%%%%%%%%%%%%%%
\subsection{Legacy Detection}
\label{sec:detection}

The directive |\childdocmain| in the main file can detect
whether the complete document or merely a child is to be compiled
even without using the directive |\childdocof|.
This method is deprecated because it is less robust
and there is no compelling reason to use it;
it is merely provided for backward compatibility
and it may be removed in future versions.

If the detection mechanism is to be used,
it is mandatory to correctly specify
the filename of the main file as the argument of |\childdocmain|:
%
\begin{center}
\begin{tabular}{l}
|\input{childdoc.def}|\\
|\childdocmain{|\textit{main}|}|\\
\end{tabular}
\end{center}
%
If |\jobname| does not match the argument \textit{main} of |\childdocmain|,
it is assumed that |\jobname| points to the child file to be compiled.
When using |\childdocmain| with the main file specified as argument,
it suffices to start a child file
with just |\input{|\textit{main}|}|
without loading of the package and using |\childdocof|.
If instead all processing is done
with the appropriate \textsf{childdoc} directives,
the argument of \textit{main} of |\childdocmain| can be empty.

An alternative version of the command line processing described
in \secref{sec:commandline} using the detection mechanism reads:
%
\begin{center}
|... -jobname "|\textit{target}|" "|[\textit{flags}]%
[|\def\jobname{|\textit{dest}|}|]|\input{|\textit{main}|}"|
\end{center}

%%%%%%%%%%%%%%%%%%%%%%%%%%%%%%%%%%%%%%%%%%%%%%%%%%%%%%%%%%%%%%%%%%%%%%%%%%%%%%%%
\subsection{Manual Code}
\label{sec:manual}

In case one cannot be certain whether the definitions file |childdoc.def|
is installed on the target \TeX{} distribution
and one prefers not to ship it,
it is conceivable to paste a few relevant commands into the sources.

To that end, drop all statements |\input{childdoc.def}|
and perform the replacements as outlined below.
Instead of |\childdocmain{|\textit{main}|}| add the following code
to the top of the main file:
%
\begin{center}
\begin{tabular}{l}
|\||ifdefined\childdocname\endinput\||fi\newif\ifchilddoc|\\
|\edef\childdocname{\scantokens\expandafter{\jobname\noexpand}}|\\
|\def\childdocmain{|\textit{main}|}\||ifx\childdocmain\childdocname\||else|\\
|\childdoctrue\includeonly{\childdocname}\let\jobname\childdocmain\||fi|\\
\end{tabular}
\end{center}
%
Instead of |\childdocof{|\textit{main}|}| just include the main file
at the top of each child file:
%
\begin{center}
|\input{|\textit{main}|}|
\end{center}
%
A simple redirection |\childdocforward{|\textit{dest}|}| is achieved by:
%
\begin{center}
|\def\jobname{|\textit{dest}|}\input{\jobname}|
\end{center}
%
The redirection with prefix
|\childdocforwardprefix[|\textit{prefix}|]{|\textit{dest}|}|
is accomplished by:
%
\begin{center}
\begin{tabular}{l}
|{\edef\jobname{\scantokens\expandafter{\jobname\noexpand}}|\\
|\def\redirectjob |\textit{prefix}|#1~~~{\gdef\jobname{|\textit{dest}|#1}}|\\
|\expandafter\redirectjob\jobname~~~}\input{\jobname}|
\end{tabular}
\end{center}

In an alternative approach,
child documents can be compiled by a specific command line
without additional code or specific definitions:
%
\begin{center}
|... -jobname "|\textit{target}|" "|[\textit{flags}]%
|\includeonly{|\textit{dest}|}\input{|\textit{main}|}"|
\end{center}
%

%%%%%%%%%%%%%%%%%%%%%%%%%%%%%%%%%%%%%%%%%%%%%%%%%%%%%%%%%%%%%%%%%%%%%%%%%%%%%%%%
%%%%%%%%%%%%%%%%%%%%%%%%%%%%%%%%%%%%%%%%%%%%%%%%%%%%%%%%%%%%%%%%%%%%%%%%%%%%%%%%
\section{Information}

%%%%%%%%%%%%%%%%%%%%%%%%%%%%%%%%%%%%%%%%%%%%%%%%%%%%%%%%%%%%%%%%%%%%%%%%%%%%%%%%
\subsection{Copyright}

Copyright \copyright{} 2017--2018 Niklas Beisert

This work may be distributed and/or modified under the
conditions of the \LaTeX{} Project Public License, either version 1.3
of this license or (at your option) any later version.
The latest version of this license is in
  \url{http://www.latex-project.org/lppl.txt}
and version 1.3 or later is part of all distributions of \LaTeX{}
version 2005/12/01 or later.

This work has the LPPL maintenance status `maintained'.

The Current Maintainer of this work is Niklas Beisert.

This work consists of the files |README.txt|, |childdoc.ins| and |childdoc.dtx|
as well as the derived files |childdoc.def|, |cdocsamp.tex|
with |cdocsch1.tex|, |cdocsch2.tex|, |cdocspt3.tex|, |cdocspt4.tex|,
|cdocsdrf.tex|, |cdocsfn1.tex|, |cdocsfn2.tex|
as well as |childdoc.pdf|.

%%%%%%%%%%%%%%%%%%%%%%%%%%%%%%%%%%%%%%%%%%%%%%%%%%%%%%%%%%%%%%%%%%%%%%%%%%%%%%%%
\subsection{Files and Installation}

The package consists of the files:
%
\begin{center}
\begin{tabular}{ll}
    |README.txt|   & readme file \\
    |childdoc.ins| & installation file \\
    |childdoc.dtx| & source file \\
    |childdoc.def| & definition file \\
    |cdocsamp.tex| & sample main file \\
    |cdocsch1.tex| & sample include file \\
    |cdocsch2.tex| & sample include file \\
    |cdocspt3.tex| & sample part file \\
    |cdocspt4.tex| & sample part file \\
    |cdocsdrf.tex| & sample redirection file \\
    |cdocsfn1.tex| & sample redirection file \\
    |cdocsfn2.tex| & sample redirection file \\
    |childdoc.pdf| & manual
\end{tabular}
\end{center}
%
The distribution consists of the files
|README.txt|, |childdoc.ins| and |childdoc.dtx|.
%
\begin{itemize}
\item
Run (pdf)\LaTeX{} on |childdoc.dtx|
to compile the manual |childdoc.pdf| (this file).
\item
Run \LaTeX{} on |childdoc.ins| to create the definitions file |childdoc.def|
and the sample |cdocsamp.tex| with include files
|cdocsch1.tex|, |cdocsch2.tex|, |cdocspt3.tex|, |cdocspt4.tex|,
|cdocsdrf.tex|, |cdocsfn1.tex|, |cdocsfn2.tex|.
Then copy the file |childdoc.def| to an appropriate directory of your \LaTeX{}
distribution, e.g.\ \textit{texmf-root}|/tex/latex/childdoc|.
\end{itemize}

%%%%%%%%%%%%%%%%%%%%%%%%%%%%%%%%%%%%%%%%%%%%%%%%%%%%%%%%%%%%%%%%%%%%%%%%%%%%%%%%
\subsection{Related CTAN Packages}

There are several other packages which offer a similar functionality:
%
\begin{itemize}
\item
The packages
\href{http://ctan.org/pkg/docmute}{\textsf{docmute}},
\href{http://ctan.org/pkg/includex}{\textsf{includex}} and
\href{http://ctan.org/pkg/standalone}{\textsf{standalone}}
provide commands to include only the document body of
a child file thus allowing both files to be compiled individually.
\item
The packages \href{http://ctan.org/pkg/subdocs}{\textsf{subdocs}}
and \href{http://ctan.org/pkg/subfiles}{\textsf{subfiles}}
provide structures in which the main and child documents can be
encapsulated and allowing them to be compiled individually.
The inclusion mechanism is different from the conventional |\include|.
\item
The package \href{http://ctan.org/pkg/combine}{\textsf{combine}}
is an elaborate solution to combine several documents into one.
\end{itemize}
%
See also the CTAN topic \href{http://ctan.org/topic/subdocs}{\textsf{subdocs}}
for further related packages.
The present package differs from the above solutions in that
a document structure constructed with the conventional |\include| mechanism
just needs two extra commands at the top of every file
such that all constituent files can be compiled individually.

%%%%%%%%%%%%%%%%%%%%%%%%%%%%%%%%%%%%%%%%%%%%%%%%%%%%%%%%%%%%%%%%%%%%%%%%%%%%%%%%
%\subsection{Feature Suggestions}
%
%The following is a list of features which may be useful for future
%versions of this package:
%%
%\begin{itemize}
%\item
%\ldots
%\end{itemize}

%%%%%%%%%%%%%%%%%%%%%%%%%%%%%%%%%%%%%%%%%%%%%%%%%%%%%%%%%%%%%%%%%%%%%%%%%%%%%%%%
\subsection{Revision History}

%%%%%%%%%%%%%%%%%%%%%%%%%%%%%%%%%%%%%%%%
\paragraph{v2.0:} 2018/12/30

\begin{itemize}
\item
immediate forward processing
\item
added |\childdocby| mechanism
\item
manual restructured
\end{itemize}

%%%%%%%%%%%%%%%%%%%%%%%%%%%%%%%%%%%%%%%%
\paragraph{v1.6:} 2018/01/17

\begin{itemize}
\item
application for development of include files
\item
corrections to manual
\end{itemize}

%%%%%%%%%%%%%%%%%%%%%%%%%%%%%%%%%%%%%%%%
\paragraph{v1.5:} 2017/05/21

\begin{itemize}
\item
more complete structuring introduced
\item
|\childdocof| introduced
\item
|\childdoc| renamed to |\childdocmain|
\item
|\childredirect| renamed to |\childdocforward| and |\childdocforwardprefix|
and functionality expanded
\end{itemize}

%%%%%%%%%%%%%%%%%%%%%%%%%%%%%%%%%%%%%%%%
\paragraph{v1.0:} 2017/04/27

\begin{itemize}
\item
manual and install package
\item
first version published on CTAN
\end{itemize}

%%%%%%%%%%%%%%%%%%%%%%%%%%%%%%%%%%%%%%%%
\paragraph{v0.6:} 2017/04/26

\begin{itemize}
\item
redirection mechanism added
\end{itemize}

%%%%%%%%%%%%%%%%%%%%%%%%%%%%%%%%%%%%%%%%
\paragraph{v0.5:} 2017/04/26

\begin{itemize}
\item
functionality in definition file
\end{itemize}


%%%%%%%%%%%%%%%%%%%%%%%%%%%%%%%%%%%%%%%%%%%%%%%%%%%%%%%%%%%%%%%%%%%%%%%%%%%%%%%%
%%%%%%%%%%%%%%%%%%%%%%%%%%%%%%%%%%%%%%%%%%%%%%%%%%%%%%%%%%%%%%%%%%%%%%%%%%%%%%%%
%%%%%%%%%%%%%%%%%%%%%%%%%%%%%%%%%%%%%%%%%%%%%%%%%%%%%%%%%%%%%%%%%%%%%%%%%%%%%%%%
\appendix

\settowidth\MacroIndent{\rmfamily\scriptsize 000\ }

 \DocInput{childdoc.dtx}

\end{document}
%</driver>
% \fi
%
% %%%%%%%%%%%%%%%%%%%%%%%%%%%%%%%%%%%%%%%%%%%%%%%%%%%%%%%%%%%%%%%%%%%%%%%%%%%%%%
% %%%%%%%%%%%%%%%%%%%%%%%%%%%%%%%%%%%%%%%%%%%%%%%%%%%%%%%%%%%%%%%%%%%%%%%%%%%%%%
% \section{Sample}
%\iffalse
%<*samplemain>
%\fi
%
% The following presents a sample document
% with two chapters, two parts, a title page,
% a compile flag as well as three forwarding files to set the flag.
% It consists of eight |.tex| files:
% \begin{center}
% \begin{tabular}{ll}
% |cdocsamp.tex|&main file\\
% |cdocsch1.tex|&include file for chapter 1\\
% |cdocsch2.tex|&include file for chapter 2\\
% |cdocspt3.tex|&include file for part 3\\
% |cdocspt4.tex|&include file for part 4\\
% |cdocsdrf.tex|&forwarding file for main file in draft mode\\
% |cdocsfi1.tex|&forwarding file for final version of chapter 1\\
% |cdocsfi2.tex|&forwarding file for final version of chapter 2\\
% \end{tabular}
% \end{center}
% Each of the eight files can be compiled directly by the \LaTeX{} compiler.
%
% %%%%%%%%%%%%%%%%%%%%%%%%%%%%%%%%%%%%%%
% \paragraph{Main File.}
%
% The main file is called |cdocsamp.tex|.
%
% Load the \textsf{childdoc} definitions and
% declare the filename for the main document:
%    \begin{macrocode}
\input{childdoc.def}
\childdocmain{}
%    \end{macrocode}

% Optional override for |\version| flag:
%    \begin{macrocode}
%%\ifchilddoc\else\providecommand{\version}{draft}\fi
%    \end{macrocode}

% Define the default values for the |\version| flag
% (|final| for the main file and |draft| for childs):
%    \begin{macrocode}
\ifchilddoc
\providecommand{\version}{draft}
\else
\providecommand{\version}{final}
\fi
%    \end{macrocode}

% Load the standard document class:
%    \begin{macrocode}
\documentclass[12pt]{article}
%    \end{macrocode}

% Start the document body:
%    \begin{macrocode}
\begin{document}
%    \end{macrocode}

% Declare a title page.
% Print title, part of document being processed and version flag:
%    \begin{macrocode}
\addtocounter{page}{-1}
\begin{center}
{\LARGE\bfseries{}childdoc example\par}
\vspace{1cm}
\ifchilddoc
\ifchilddocmanual part\else chapter\fi:
`\childdocname' of `\childdocjob'\par
\else
main document: `\childdocjob'\par
\fi
version: \version\par
\end{center}
\newpage
%    \end{macrocode}

% Manually include selected file,
% otherwise process as usual:
%    \begin{macrocode}
\ifchilddocmanual
\section*{part `\childdocname'}
\input{\childdocname}
\else
%    \end{macrocode}

% Include the two chapters:
%    \begin{macrocode}
\include{cdocsch1}
\include{cdocsch2}
%    \end{macrocode}

% Include the two parts unless only chapters should be displayed:
%    \begin{macrocode}
\ifchilddoc\else
\section{part three}
\input{cdocspt3}
\section{part four}
\input{cdocspt4}
\fi
%    \end{macrocode}

% Process as usual until here:
%    \begin{macrocode}
\fi
%    \end{macrocode}

% End of document body:
%    \begin{macrocode}
\end{document}
%    \end{macrocode}
%\iffalse
%</samplemain>
%\fi
%
% %%%%%%%%%%%%%%%%%%%%%%%%%%%%%%%%%%%%%%
% \paragraph{Chapter Include Files.}
%
% The include files are called |cdocsch1.tex| and |cdocsch2.tex|.
%
%\iffalse
%<*samplechap1|samplechap2>
%\fi

% Optional override for |\version| flag:
%    \begin{macrocode}
%%\providecommand{\version}{final}
%    \end{macrocode}

% Include the main document:
%    \begin{macrocode}
\input{childdoc.def}
\childdocof{cdocsamp}
%    \end{macrocode}

%\iffalse
%</samplechap1|samplechap2>
%\fi
%
%\iffalse
%<*samplechap1>
%\fi
% Some text for chapter 1:
%    \begin{macrocode}
\section{one}
some text in chapter one
%    \end{macrocode}

%\iffalse
%</samplechap1>
%\fi
% Some text for chapter 2:
%\iffalse
%<*samplechap2>
%\fi
%    \begin{macrocode}
\section{two}
more text in chapter two
%    \end{macrocode}

%\iffalse
%</samplechap2>
%\fi
%
% %%%%%%%%%%%%%%%%%%%%%%%%%%%%%%%%%%%%%%
% \paragraph{Part Include Files.}
%
% The include files are called |cdocspt3.tex| and |cdocspt4.tex|.
%
%\iffalse
%<*samplepart3|samplepart4>
%\fi

% Optional override for |\version| flag:
%    \begin{macrocode}
%%\providecommand{\version}{final}
%    \end{macrocode}

% Include the main document:
%    \begin{macrocode}
\input{childdoc.def}
\childdocby{cdocsamp}
%    \end{macrocode}

%\iffalse
%</samplepart3|samplepart4>
%\fi
%
%\iffalse
%<*samplepart3>
%\fi
% Some text for part 3:
%    \begin{macrocode}
some text in part three
%    \end{macrocode}

%\iffalse
%</samplepart3>
%\fi
% Some text for part 4:
%\iffalse
%<*samplepart4>
%\fi
%    \begin{macrocode}
more text in part four
%    \end{macrocode}

%\iffalse
%</samplepart4>
%\fi
%
% %%%%%%%%%%%%%%%%%%%%%%%%%%%%%%%%%%%%%%
% \paragraph{Forwarding for a Complete Draft.}
%
% The following forwarding file |cdocsdrf.tex|
% compiles the main document in draft mode:
%\iffalse
%<*sampledraft>
%\fi
%    \begin{macrocode}
\def\version{draft}
\input{childdoc.def}
\childdocforward{cdocsamp}
%    \end{macrocode}

%\iffalse
%</sampledraft>
%\fi
%
% %%%%%%%%%%%%%%%%%%%%%%%%%%%%%%%%%%%%%%
% \paragraph{Forwarding for Final Version of the Chapters.}
%
% The following forwarding files |cdocsfn1.tex| and |cdocsfn2.tex|
% (with identical content)
% compile the final versions of the child documents
% |cdocsch1.tex| and |cdocsch2.tex|, respectively:
%\iffalse
%<*samplefinal>
%\fi
%    \begin{macrocode}
\def\version{final}
\input{childdoc.def}
\childdocforwardprefix[cdocsamp]{cdocsfn}{cdocsch}
%    \end{macrocode}

%\iffalse
%</samplefinal>
%\fi
%
% %%%%%%%%%%%%%%%%%%%%%%%%%%%%%%%%%%%%%%
% \paragraph{Command Line Processing.}
%
% The following three command lines generate the output files
% |cdocscld|, |cdocscl1| and |cdocscl2|
% which should be identical to
% |cdocsdrf|, |cdocsch1| and |cdocsfn2|, respectively:
% \begin{center}
% \begin{tabular}{l}
% |latex -jobname cdocscld \|\\
% |  "\def\version{draft}\input{childdoc.def}\childdocforward{cdocsamp}"|\\
% |latex -jobname cdocscl1 \|\\
% |  "\input{childdoc.def}\childdocforward[cdocsamp]{cdocsch1}"|\\
% |latex -jobname cdocscl2 \|\\
% |  "\def\version{final}\input{childdoc.def}\childdocforward{cdocsch2}"|
% \end{tabular}
% \end{center}
% Note that the trailing backslash on each first line
% merely continues the input to the second line
% (for convenient cut ant paste).
% Furthermore, the command |latex| can be replaced by any
% of its alternative versions such as |pdflatex|.
%
% %%%%%%%%%%%%%%%%%%%%%%%%%%%%%%%%%%%%%%%%%%%%%%%%%%%%%%%%%%%%%%%%%%%%%%%%%%%%%%
% %%%%%%%%%%%%%%%%%%%%%%%%%%%%%%%%%%%%%%%%%%%%%%%%%%%%%%%%%%%%%%%%%%%%%%%%%%%%%%
% \section{Implementation}
%\iffalse
%<*package>
%\fi
%
% This section describes the definitions file |childdoc.def|.

% The definitions cannot be loaded using |\usepackage| or |\RequirePackage|
% which has a mechanism to prevent loading a style file more than once.
% When loading the definitions by means of |\input|
% multiple instances have to be prevented manually:
%\iffalse
%This code needs to be before the `\ProvidesFile' directive
%which is defined at the beginning of this file.
%Therefore it is also placed there and commented out here.
%</package>
%<*discard>
%\fi
%    \begin{macrocode}
\ifdefined\childdocmain\endinput\fi
%    \end{macrocode}
%\iffalse
%</discard>
%<*package>
%\fi
%
% \macro{\ifchilddoc}
% \macro{\ifchilddocmanual}
% The conditional |\ifchilddoc| tells whether a
% child (true) or main (false) document is being compiled.
% The conditional |\ifchilddocmanual| tells whether
% the |\includeonly| mechanism is used (false) or
% the selection of child files must be performed manually (true).
% The definitions initialise to false:
%    \begin{macrocode}
\newif\ifchilddoc
\newif\ifchilddocmanual
%    \end{macrocode}

% \macro{\childdocname}
% \macro{\childdocjob}
% The macro |\childdocname| stores the name of the main document
% to be compiled. The macro |\childdocjob| stores the name of
% the document on which the \LaTeX{} compiler was originally invoked.
% The content of |\jobname| cannot be compared
% to filenames specified in the source due to different catcodes.
% The following code rescans |\jobname|, stores the result
% in |\childdocname| and saves a copy in |\childdocjob|:
%    \begin{macrocode}
\edef\childdocname{\scantokens\expandafter{\jobname\noexpand}}
\let\childdocjob\childdocname
%    \end{macrocode}

% \macro{\childdocdisable}
% The macro |\childdocdisable| prevents the main file
% from being processed more than once.
% At this stage, the main document command |\childdocmain|
% is assumed to be called once again where it should do nothing.
% Any subsequent call to it should prevent
% a secondary processing of the main document
% It overwrites the forwarding commands
% |\childdocof| and |\childdocforward|
% with empty macros to prevent further inclusions of the main document:
%    \begin{macrocode}
\newcommand{\childdocdisable}
{
  \renewcommand{\childdocmain}[1]{\renewcommand{\childdocmain}[1]{\endinput}}
  \renewcommand{\childdocof}[1]{}
  \renewcommand{\childdocby}[2][]{}
  \renewcommand{\childdocforward}[2][]{}
  \renewcommand{\childdocdisable}{}
}
%    \end{macrocode}

% \macro{\childdocmain}
% The macro |\childdocmain| is to be called at the top of the main file
% with nothing or the main filename (without extension) as argument.
% First, it breaks loops.
% If the argument is not empty and does not match |\childdocname|
% (which is set by the first inclusion of |childdoc.def|),
% |\ifchilddoc| is set to true, |\includeonly| is applied to the child file
% and |\jobname| is set to the main file
% (for proper handling of |.aux| files):
%    \begin{macrocode}
\newcommand{\childdocmain}[1]
{
  \childdocdisable\childdocmain{}
  \if?#1?\else
    \begingroup
      \def\childdoctmp{#1}
      \ifx\childdoctmp\childdocname
        \def\childdoctmp{}
      \else
        \def\childdoctmp
        {
          \childdoctrue
          \includeonly{\childdocname}
          \def\childdocjob{#1}
          \def\jobname{#1}
        }
      \fi
      \expandafter
    \endgroup
    \childdoctmp
  \fi
}
%    \end{macrocode}

% \macro{\childdocof}
% The command |\childdocof| redirects
% compilation to the main file |#1|.
%    \begin{macrocode}
\newcommand{\childdocof}[1]
{
  \childdocdisable
  \childdoctrue
  \includeonly{\childdocname}
  \def\jobname{#1}
  \def\childdocjob{#1}
  \input{#1}
}
%    \end{macrocode}

% \macro{\childdocby}
% The command |\childdocby| ....
%    \begin{macrocode}
\newcommand{\childdocby}[2][]
{
  \childdocdisable
  \childdoctrue
  \childdocmanualtrue
  \if?#1?\else
    \def\jobname{#2}
  \fi
  \def\childdocjob{#2}
  \input{#2}
  \endinput
}
%    \end{macrocode}

% \macro{\childdocforward}
% The command |\childdocforward| redirects
% compilation to the main file or
% (if the optional argument is given) a child file.
% Parameters are set as if the main file
% or a child file starting with |\childdocof| was compiled.
% Then compilation is handed over to the main file:
%    \begin{macrocode}
\newcommand{\childdocforward}[2][]
{
  \begingroup
    \if?#1?
      \def\childdoctmp
      {
        \def\childdocname{#2}
        \def\childdocjob{#2}
        \def\jobname{#2}
        \input{#2}
        \endinput
      }
    \else
      \def\childdoctmp
      {
        \childdocdisable
        \def\childdocname{#2}
        \childdoctrue
        \includeonly{#2}
        \def\childdocjob{#1}
        \def\jobname{#1}
        \input{#1}
        \endinput
      }
    \fi
    \expandafter
  \endgroup
  \childdoctmp
}
%    \end{macrocode}

% \macro{\childdocforwardprefix}
% The command |\childdocforwardprefix| redirects
% compilation to the main or a child file by means of a pattern.
% The prefix |#1| in the current filename is replaced by |#2|
% and the suffix of the current filename is kept
% (it is assumed that the filename does not contain the substring `|~~~|'
% which is used as a delimiter).
% Compilation is handed over to the new file by |\childdocforward|:
%    \begin{macrocode}
\newcommand{\childdocforwardprefix}[3][]
{
  \begingroup
    \def\childdocextract #2##1~~~{\def\childdoctmp{\childdocforward[#1]{#3##1}}}
    \expandafter\childdocextract\childdocname~~~
    \expandafter
  \endgroup
  \childdoctmp
}
%    \end{macrocode}

% \macro{\childdoc}
% The deprecated macro |\childdoc| is a legacy version of |\childdocmain|:
%    \begin{macrocode}
\newcommand{\childdoc}{\childdocmain}
%    \end{macrocode}

% \macro{\childdocredirect}
% The deprecated macro |\childdocredirect| is a legacy version
% of |\childdocforward| and |\childdocforwardprefix|:
%    \begin{macrocode}
\newcommand{\childdocredirect}[2][]
{
  \begingroup
    \if?#1?
      \def\childdoctmp{\childdocforward{#2}}
    \else
      \def\childdoctmp{\childdocforwardprefix{#1}{#2}}
    \fi
    \expandafter
  \endgroup
  \childdoctmp
}
%    \end{macrocode}

%\iffalse
%</package>
%\fi
%
\endinput
|\\
|\childdocby{|\textit{main}|}|\\
\end{tabular}
\end{center}
%
The directive |\childdocby| is similar to |\childdocof|
described in \secref{sec:include},
but the subsequent selection of content must be done manually.
To that end, both |\ifchilddoc| and |\ifchilddocmanual|
will be true upon processing of a part,
and the name of the part is stored in |\childdocname|.
Note that |\jobname| will be set to the filename of the current part
so that each part receives an individual |.aux| file
that does not interfere with the |.aux| file(s) of the main document.
This behaviour can be altered by the alternative form
|\childdocby[*]{|\textit{main}|}| (with a non-empty optional argument)
which uses the |.aux| file of the main document
by setting |\jobname| to \textit{main}.

%%%%%%%%%%%%%%%%%%%%%%%%%%%%%%%%%%%%%%%%%%%%%%%%%%%%%%%%%%%%%%%%%%%%%%%%%%%%%%%%
\subsection{Driver Development}
\label{sec:driver}

The \textsf{childdoc} mechanism can also be use for the development
of definition files such as \LaTeX{} styles or classes.
This case differs from the above setup with multiple parts
included by |\include| in that no |\includeonly| should be invoked.
This can be achieved by starting the include file
(before |\ProvidesPackage|) with:
%
\begin{center}
\begin{tabular}{l}
|% \iffalse
%
% childdoc.dtx Copyright (C) 2017-2018 Niklas Beisert
%
% This work may be distributed and/or modified under the
% conditions of the LaTeX Project Public License, either version 1.3
% of this license or (at your option) any later version.
% The latest version of this license is in
%   http://www.latex-project.org/lppl.txt
% and version 1.3 or later is part of all distributions of LaTeX
% version 2005/12/01 or later.
%
% This work has the LPPL maintenance status `maintained'.
%
% The Current Maintainer of this work is Niklas Beisert.
%
% This work consists of the files childdoc.dtx and childdoc.ins
% and the derived files childdoc.def and cdocsamp.tex with
% cdocsch1.tex, cdocsch2.tex, cdocsdrf.tex, cdocsfn1.tex, cdocsfn2.tex.
%
%<package>\ifdefined\childdocmain\endinput\fi
%<package>\ProvidesFile{childdoc.def}[2018/12/30 v2.0 child document driver]
%<samplemain>\ProvidesFile{cdocsamp.tex}[2018/12/30 v2.0 sample for childdoc]
%<*driver>
%\ProvidesFile{childdoc.drv}[2018/12/30 v2.0 childdoc reference manual file]
\PassOptionsToClass{10pt,a4paper}{article}
\documentclass{ltxdoc}

\usepackage[margin=35mm]{geometry}
\usepackage{hyperref}
\usepackage{hyperxmp}
\usepackage[usenames]{color}

\hypersetup{colorlinks=true}
\hypersetup{pdfstartview=FitH}
\hypersetup{pdfpagemode=UseNone}
\hypersetup{pdfsource={}}
\hypersetup{pdflang={en-UK}}
\hypersetup{pdfcopyright={Copyright 2017-2018 Niklas Beisert.
  This work may be distributed and/or modified under the
  conditions of the LaTeX Project Public License, either version 1.3
  of this license or (at your option) any later version.}}
\hypersetup{pdflicenseurl={http://www.latex-project.org/lppl.txt}}
\hypersetup{pdfcontactaddress={ETH Zurich, ITP, HIT K,
  Wolfgang-Pauli-Strasse 27}}
\hypersetup{pdfcontactpostcode={8093}}
\hypersetup{pdfcontactcity={Zurich}}
\hypersetup{pdfcontactcountry={Switzerland}}
\hypersetup{pdfcontactemail={nbeisert@itp.phys.ethz.ch}}
\hypersetup{pdfcontacturl={http://people.phys.ethz.ch/\xmptilde nbeisert/}}

\newcommand{\secref}[1]{\hyperref[#1]{section \ref*{#1}}}

\parskip1ex
\parindent0pt
\let\olditemize\itemize
\def\itemize{\olditemize\parskip0pt}

\begin{document}

\title{The \textsf{childdoc} Package}
\hypersetup{pdftitle={The childdoc Package}}
\author{Niklas Beisert\\[2ex]
  Institut f\"ur Theoretische Physik\\
  Eidgen\"ossische Technische Hochschule Z\"urich\\
  Wolfgang-Pauli-Strasse 27, 8093 Z\"urich, Switzerland\\[1ex]
  \href{mailto:nbeisert@itp.phys.ethz.ch}
  {\texttt{nbeisert@itp.phys.ethz.ch}}}
\hypersetup{pdfauthor={Niklas Beisert}}
\hypersetup{pdfsubject={Manual for the LaTeX2e Package childdoc}}
\date{30 December 2018, \textsf{v2.0}}
\maketitle

\begin{abstract}\noindent
\textsf{childdoc} is a \LaTeXe{} package
that enables the direct compilation
of document sections included by |\include|
to individual files.
\end{abstract}

\begingroup
\parskip0ex
\tableofcontents
\endgroup

%%%%%%%%%%%%%%%%%%%%%%%%%%%%%%%%%%%%%%%%%%%%%%%%%%%%%%%%%%%%%%%%%%%%%%%%%%%%%%%%
%%%%%%%%%%%%%%%%%%%%%%%%%%%%%%%%%%%%%%%%%%%%%%%%%%%%%%%%%%%%%%%%%%%%%%%%%%%%%%%%
\section{Introduction}

\LaTeX{} provides a mechanism to structure a large document (such as a book)
into a main file and several child files (containing the chapters)
using the |\include| command.
This mechanism is beneficial for documents
which span hundreds of pages in order to
make the source file(s) more manageable.
Moreover, compilation can be restricted to
selected child files by means of the |\includeonly| command.
The latter feature can be used to reduce the compilation time while editing
(this was significantly more useful in the earlier days of \LaTeX{})
or to generate a smaller document which is easier to navigate.
Another application of |\includeonly| is to generate
documents consisting of selected parts of the complete document.

However, there are a few drawbacks of the plain |\include| mechanism:
\begin{itemize}
\item
The child files cannot be compiled on their own,
they can only be compiled via the main file.
A naive editing environment
(such as a text editor with an option
to have the current file processed by \LaTeX)
may require one to switch to the main file before compiling;
attempting to compile the child file produces errors.
\item
The main file must be modified (each time)
to adjust the |\includeonly| command
to the present needs. This easily leaves the main file in a messy state.
\item
The generated document will always carry the filename
of the main document. This is inconvenient if
several child files are to be compiled and
to be kept for distribution.
\end{itemize}

The present package provides a simple interface
to make child files individually compilable by \LaTeX{}.
Compiling a child file then has the same effect as compiling
the main file with an |\includeonly| command
to select the appropriate child.
Moreover the generated document will carry the name of the child
rather than the main file.
This resolves all three above issues.

This feature is meant to make the editing of books,
thesis documents and lecture notes somewhat more convenient.
However, the package can also be used efficiently for
composing a series of documents (such as exercise sheets)
which are typically distributed individually.
It then assists the author in generating the individual documents
(potentially in different versions)
as well as a document containing the collected series.
Another application is in developing style files
or other kinds of included material
where compilation of the style file could redirect
to a sample or test file.

%%%%%%%%%%%%%%%%%%%%%%%%%%%%%%%%%%%%%%%%%%%%%%%%%%%%%%%%%%%%%%%%%%%%%%%%%%%%%%%%
%%%%%%%%%%%%%%%%%%%%%%%%%%%%%%%%%%%%%%%%%%%%%%%%%%%%%%%%%%%%%%%%%%%%%%%%%%%%%%%%
\section{Usage}

First of all, the package \textsf{childdoc} is \emph{not} a standard
\LaTeXe{} |.sty| style file! Therefore it needs to be invoked in
a non-standard way.

%%%%%%%%%%%%%%%%%%%%%%%%%%%%%%%%%%%%%%%%%%%%%%%%%%%%%%%%%%%%%%%%%%%%%%%%%%%%%%%%
\subsection{Included Files}
\label{sec:include}

%%%%%%%%%%%%%%%%%%%%%%%%%%%%%%%%%%%%%%%%
\DescribeMacro{\childdocmain}
To use the package, add the commands
\begin{center}
\begin{tabular}{l}
|\input{childdoc.def}|\\
|\childdocmain{}|\\
\end{tabular}
\end{center}
at the very top of the main \LaTeX{} file,
in particular \emph{before} the |\documentclass| statement!
The argument of |\childdocmain| should be left empty
(but it must be present).

%%%%%%%%%%%%%%%%%%%%%%%%%%%%%%%%%%%%%%%%
\DescribeMacro{\childdocof}
Furthermore, add the commands
\begin{center}
\begin{tabular}{l}
|\input{childdoc.def}|\\
|\childdocof{|\textit{main}|}|\\
\end{tabular}
\end{center}
at the top of every child file \textit{child}
which is included by |\include{|\textit{child}|}|
from within the main file
(or at least for those files to be compiled individually).
The argument \textit{main} must be the filename of the main file.

There are a couple of
considerations in setting up the main and child documents:

%%%%%%%%%%%%%%%%%%%%%%%%%%%%%%%%%%%%%%%%
\paragraph{Restrictions.}

Please note the following restrictions:
\begin{itemize}
\item
|\childdocmain| must be called with one argument \textit{main}
to ensure compatibility with earlier version of the package.
It must either be empty (|\childdocmain{}|)
or precisely match the filename of the main file in which it is specified.
See \secref{sec:detection} for further information.
\item
The filename \textit{main} must be specified without the |.tex| extension.
\item
The filename \textit{main} is case sensitive
(even in case-insensitive file systems)
due to internal string comparison.
\item
The argument \textit{main} should be fully expanded, it cannot be a macro.
\item
Subdirectories and special characters should be avoided in filenames.
\item
The command |\childdocmain{|\textit{main}|}| must be followed by a whitespace.
It should not be followed immediately by another command
or by a comment mark `|%|'.
This is because the \TeX{} parser reads the token immediately following
the argument of |\childdocmain| and puts it
at the beginning of every child section;
however, a white\-space is ignored.
\end{itemize}

%%%%%%%%%%%%%%%%%%%%%%%%%%%%%%%%%%%%%%%%
\paragraph{Content of Main File.}

It is advisable to place all content in the child files included by |\include|.
Any output contained in the main file will appear in all child documents
unless suppressed manually;
it cannot be suppressed automatically by the |\includeonly| directive
and thus should normally be avoided.
A method to include some content in the main file
by means of conditional processing is described in \secref{sec:conditional}.

%%%%%%%%%%%%%%%%%%%%%%%%%%%%%%%%%%%%%%%%
\paragraph{Page Numbering.}

When only a part of the document is compiled,
the appropriate numbering of pages
(as well as other status parameters)
is determined from the |.aux| files.
The latter contain information from previous passes.
However this information needs to propagate through
all intermediate child documents.
Therefore the page numbering in child documents may well
be inconsistent until the complete document is compiled at least once.

A useful (if unconventional) way to always ensure a consistent
page numbering is to restart the numbering in each child document
and denote the pages by `\textit{child}|.|\textit{page}'
where \textit{child} represents the chapter/section number of the child file.
This can be achieved by the command
|\numberwithin{page}{|\textit{child}|}|
of the \textsf{amsmath} package
where \textit{child} can be |chapter| or |section|
depending on the chosen structuring.
Alternatively, one can modify the macro |\thepage| appropriately
and reset the counter |page| at the start of each child file.

%%%%%%%%%%%%%%%%%%%%%%%%%%%%%%%%%%%%%%%%%%%%%%%%%%%%%%%%%%%%%%%%%%%%%%%%%%%%%%%%
\subsection{Conditional Processing}
\label{sec:conditional}

The package provides a mechanism to compile different versions
of a document. To customise the versions further some conditional processing
can come in handy to distinguish which version is being compiled.
The package provides two macros to describe the compilation context:

%%%%%%%%%%%%%%%%%%%%%%%%%%%%%%%%%%%%%%%%
\DescribeMacro{\ifchilddoc}
The conditional |\ifchilddoc| distinguishes between the compilation of
child documents and the main document:
%
\begin{center}
|\ifchilddoc |\textit{child-code}| |[|\||else |\textit{main-code}]| \||fi|
\end{center}

%%%%%%%%%%%%%%%%%%%%%%%%%%%%%%%%%%%%%%%%
\DescribeMacro{\childdocname}
\DescribeMacro{\childdocjob}
The macro |\childdocname| contains the filename (without extension)
of the main or child file being processed.
Note that |\childdocjob| will always contain the name of the main file.

%%%%%%%%%%%%%%%%%%%%%%%%%%%%%%%%%%%%%%%%
\paragraph{Title Page.}

Conditional processing can be used to include a title or banner page
in the main document when proper precautions are taken.
Importantly, the code in the main file should ensure that the page counter
(as well as other status parameters which are stored in the |.aux| files)
takes the same value after the conditional processing.
Otherwise the page numbers may take divergent values
depending on which part is compiled.

For example, a title page could be declared by:
%
\begin{center}
\begin{tabular}{l}
|\ifchilddoc\||else|\\
|\addtocounter{page}{-1}|\\
\textit{code for title page}\\
|\newpage|\\
|\||fi|
\end{tabular}
\end{center}
%
A banner page for the child documents can be generated by:
%
\begin{center}
\begin{tabular}{l}
|\ifchilddoc|\\
|\addtocounter{page}{-1}|\\
\textit{code for banner page}\\
|\newpage|\\
|\||fi|
\end{tabular}
\end{center}
%
Here one could write a message such as:
\begin{center}
|This is the part \childdocname{} of \childdocjob{}.|
\end{center}

%%%%%%%%%%%%%%%%%%%%%%%%%%%%%%%%%%%%%%%%%%%%%%%%%%%%%%%%%%%%%%%%%%%%%%%%%%%%%%%%
\subsection{Flags}
\label{sec:flags}

The package makes it easy to generate different versions
of the main or child documents.
To this end compilation flags can be defined
and assigned different default values.
They will be particularly useful in conjunction
with the forwarding mechanism described in \secref{sec:forward}.

For example, it may be useful to have a flag |\version|
which can be set to |draft| or |final|.
The document source will contain some conditional code
depending on the value of |\version|.
Suppose further, the flag should default to |final| for the main file
and to |draft| for child files
which is a natural assignment for editing the document.
This is achieved by placing the following code
in the preamble of the main document
(below the |\childdocmain| directive):
%
\begin{center}
\begin{tabular}{l}
|\ifchilddoc|\\
|\providecommand{\version}{draft}|\\
|\||else|\\
|\providecommand{\version}{final}|\\
|\||fi|
\end{tabular}
\end{center}
%
The definition by |\providecommand| makes sure
that previous definitions are not overwritten.
Further statements |\providecommand{\version}{...}|
can thus be added before the above code to override it.

For the main file, one might add a line
(between |\childdocmain| and the above block)
%
\begin{center}
|%\ifchilddoc\||else\providecommand{\version}{draft}\||fi|
\end{center}
%
which can be uncommented to produce a draft version.
Likewise one can add a line to the very top of a child file
(above the |\childdocof{|\textit{main}|}| directive)
%
\begin{center}
|%\providecommand{\version}{final}|
\end{center}
%
which can be uncommented to produce the final version of this child document.

%%%%%%%%%%%%%%%%%%%%%%%%%%%%%%%%%%%%%%%%%%%%%%%%%%%%%%%%%%%%%%%%%%%%%%%%%%%%%%%%
\subsection{Forwarding}
\label{sec:forward}

Different versions of the main or child documents
using compilation flags as described in \secref{sec:flags}
can be (permanently) stored in different files
for convenient compilation, viewing and distribution.
To this end, the package defines a command
to pass on compilation to a different file:

%%%%%%%%%%%%%%%%%%%%%%%%%%%%%%%%%%%%%%%%
\DescribeMacro{\childdocforward}
The command |\childdocforward| redirects processing to
another source file:
%
\begin{center}
\begin{tabular}{l}
|\input{childdoc.def}|\\
|\childdocforward[|\textit{main}|]{|\textit{dest}|}|\\
\end{tabular}
\end{center}
%
The argument \textit{dest} is the destination file
(without extension).
It should be the main file or one of the child files.
Note that further \textsf{childdoc} directives
such as |\childdocof| and |\childdocforward|
in the indicated file will be processed in this form.
The optional argument \textit{main}
passes on directly to the main file \textit{main}
while pretending to compile the child \textit{dest}.
This form behaves as if \textit{dest}
issues |\childdocof{|\textit{main}|}| right away,
and no further \textsf{childdoc} directives will be processed.

%%%%%%%%%%%%%%%%%%%%%%%%%%%%%%%%%%%%%%%%
\DescribeMacro{\...prefix}
In the alternative form |\childdocforwardprefix|,
%
\begin{center}
\begin{tabular}{l}
|\input{childdoc.def}|\\
|\childdocforwardprefix[|\textit{main}|]{|\textit{prefix}|}{|\textit{dest}|}|
\end{tabular}
\end{center}
%
the destination file is determined by a pattern
depending on the current file:
To make this work, the current file must be called
`{\textit{prefix}\hspace{0.2em}\textit{suffix}}'
with \textit{prefix} matching precisely the argument.
Processing is then passed on to the file
`{\textit{dest}\hspace{0.2em}\textit{suffix}}'.
Surely, the same effect is achieved by
directly specifying the
argument `{\textit{dest}\hspace{0.2em}\textit{suffix}}'
in the first form.
However, that requires to set up a different file
for each child. With the alternative form of the command
all these files can have exactly the same content
which simplifies setting them up and maintaining them.

For example, the following file |draft.tex|
with a compilation flag |\version| as described in \secref{sec:flags}
compiles the main document as a draft:
%
\begin{center}
\begin{tabular}{l}
|\def\version{draft}|\\
|\input{childdoc.def}|\\
|\childdocforward{|\textit{main}|}|
\end{tabular}
\end{center}
%
Likewise, the following files |final|\textit{nn}|.tex|
compile the final version of the child document
|child|\textit{nn}|.tex|:
%
\begin{center}
\begin{tabular}{l}
|\def\version{final}|\\
|\input{childdoc.def}|\\
|\childdocforwardprefix{final}{child}|
\end{tabular}
\end{center}
%

Note that when several versions of a main file and/or of each child file
are to be generated, it may be convenient to set up a |Makefile| or
shell script to automatise the process.

%%%%%%%%%%%%%%%%%%%%%%%%%%%%%%%%%%%%%%%%%%%%%%%%%%%%%%%%%%%%%%%%%%%%%%%%%%%%%%%%
\subsection{Command Line Processing}
\label{sec:commandline}

The effect of redirection files can also be achieved by invoking
the \LaTeX{} compiler with a more elaborate command line.
Most conveniently this should be done as part
of a shell script or a |Makefile|.

When using \textsf{childdoc} in the main file, the following
command lines effectively perform a redirection
(note that depending on the shell being used,
backslashes may have to be doubled: `|\|' $\to$ `|\\|'):
%
\begin{center}
|... -jobname "|\textit{target}|" |\\|"|[\textit{flags}]%
|\input{childdoc.def}\childdocforward[|\textit{main}|]{|\textit{dest}|}"|
\end{center}
%
Here \textit{target} is the name of the output file,
\textit{main} is the name of the main file
and \textit{dest} is the name of the main or child file to be processed
(all filenames without extensions).
The optional argument \textit{main} can be omitted
if \textit{main} matches \textit{dest}.
Optionally, compilation \textit{flags} can be defined via |\def| commands.
This command line makes the \TeX{} engine believe
it is compiling the file \textit{target}
whose content is specified as the latter parameter.
The provided code then forwards the processing to
\textit{main} or \textit{dest} as described in \secref{sec:forward}.

%%%%%%%%%%%%%%%%%%%%%%%%%%%%%%%%%%%%%%%%%%%%%%%%%%%%%%%%%%%%%%%%%%%%%%%%%%%%%%%%
\subsection{Include by Input}
\label{sec:input}

Including child documents by |\include| has some restrictions by design.
Most notably, the content of a child document always occupies
its own set of pages; pages cannot be shared between child documents.
Usually, this behaviour makes perfect sense
because each child document contain an essential part of the document.
However, in some situations it may be desirable to compose
a document from a collection of parts
without having mandatory page breaks between then.
For this case, the package
provides a mechanism to include parts
by |\input| which can also be processed individually.
However, by construction this mechanism
requires manual handling of the content to be output.

%%%%%%%%%%%%%%%%%%%%%%%%%%%%%%%%%%%%%%%%
\DescribeMacro{\ifchilddocmanual}
The main file should be prepared as usual, see \secref{sec:include}.
However, the document body must make a distinction
between processing of an individual part and of the main document, e.g.:
%
\begin{center}
\begin{tabular}{l}
|\ifchilddocmanual|\\
|\input{\childdocname}|\\
|\||else|\\
\textit{document body with }|\input{|\textit{part}|}|\\
|\||fi|
\end{tabular}
\end{center}
%
The conditional |\ifchilddocmanual| is true whenever
a part to be included by |\input| is being compiled,
and the name of the part is stored in |\childdocname|.

%%%%%%%%%%%%%%%%%%%%%%%%%%%%%%%%%%%%%%%%
\DescribeMacro{\childdocby}
Each part to be included by |\input| should start with:
%
\begin{center}
\begin{tabular}{l}
|\input{childdoc.def}|\\
|\childdocby{|\textit{main}|}|\\
\end{tabular}
\end{center}
%
The directive |\childdocby| is similar to |\childdocof|
described in \secref{sec:include},
but the subsequent selection of content must be done manually.
To that end, both |\ifchilddoc| and |\ifchilddocmanual|
will be true upon processing of a part,
and the name of the part is stored in |\childdocname|.
Note that |\jobname| will be set to the filename of the current part
so that each part receives an individual |.aux| file
that does not interfere with the |.aux| file(s) of the main document.
This behaviour can be altered by the alternative form
|\childdocby[*]{|\textit{main}|}| (with a non-empty optional argument)
which uses the |.aux| file of the main document
by setting |\jobname| to \textit{main}.

%%%%%%%%%%%%%%%%%%%%%%%%%%%%%%%%%%%%%%%%%%%%%%%%%%%%%%%%%%%%%%%%%%%%%%%%%%%%%%%%
\subsection{Driver Development}
\label{sec:driver}

The \textsf{childdoc} mechanism can also be use for the development
of definition files such as \LaTeX{} styles or classes.
This case differs from the above setup with multiple parts
included by |\include| in that no |\includeonly| should be invoked.
This can be achieved by starting the include file
(before |\ProvidesPackage|) with:
%
\begin{center}
\begin{tabular}{l}
|\input{childdoc.def}|\\
|\childdocforward{|\textit{main}|}|\\
\end{tabular}
\end{center}
%
or alternatively with:
%
\begin{center}
\begin{tabular}{l}
|\input{childdoc.def}|\\
|\childdocby{|\textit{main}|}|\\
\end{tabular}
\end{center}
%
Both forms have slightly different effects as described above.
The main file is prepared as usual, see \secref{sec:include}.

%%%%%%%%%%%%%%%%%%%%%%%%%%%%%%%%%%%%%%%%%%%%%%%%%%%%%%%%%%%%%%%%%%%%%%%%%%%%%%%%
\subsection{Legacy Detection}
\label{sec:detection}

The directive |\childdocmain| in the main file can detect
whether the complete document or merely a child is to be compiled
even without using the directive |\childdocof|.
This method is deprecated because it is less robust
and there is no compelling reason to use it;
it is merely provided for backward compatibility
and it may be removed in future versions.

If the detection mechanism is to be used,
it is mandatory to correctly specify
the filename of the main file as the argument of |\childdocmain|:
%
\begin{center}
\begin{tabular}{l}
|\input{childdoc.def}|\\
|\childdocmain{|\textit{main}|}|\\
\end{tabular}
\end{center}
%
If |\jobname| does not match the argument \textit{main} of |\childdocmain|,
it is assumed that |\jobname| points to the child file to be compiled.
When using |\childdocmain| with the main file specified as argument,
it suffices to start a child file
with just |\input{|\textit{main}|}|
without loading of the package and using |\childdocof|.
If instead all processing is done
with the appropriate \textsf{childdoc} directives,
the argument of \textit{main} of |\childdocmain| can be empty.

An alternative version of the command line processing described
in \secref{sec:commandline} using the detection mechanism reads:
%
\begin{center}
|... -jobname "|\textit{target}|" "|[\textit{flags}]%
[|\def\jobname{|\textit{dest}|}|]|\input{|\textit{main}|}"|
\end{center}

%%%%%%%%%%%%%%%%%%%%%%%%%%%%%%%%%%%%%%%%%%%%%%%%%%%%%%%%%%%%%%%%%%%%%%%%%%%%%%%%
\subsection{Manual Code}
\label{sec:manual}

In case one cannot be certain whether the definitions file |childdoc.def|
is installed on the target \TeX{} distribution
and one prefers not to ship it,
it is conceivable to paste a few relevant commands into the sources.

To that end, drop all statements |\input{childdoc.def}|
and perform the replacements as outlined below.
Instead of |\childdocmain{|\textit{main}|}| add the following code
to the top of the main file:
%
\begin{center}
\begin{tabular}{l}
|\||ifdefined\childdocname\endinput\||fi\newif\ifchilddoc|\\
|\edef\childdocname{\scantokens\expandafter{\jobname\noexpand}}|\\
|\def\childdocmain{|\textit{main}|}\||ifx\childdocmain\childdocname\||else|\\
|\childdoctrue\includeonly{\childdocname}\let\jobname\childdocmain\||fi|\\
\end{tabular}
\end{center}
%
Instead of |\childdocof{|\textit{main}|}| just include the main file
at the top of each child file:
%
\begin{center}
|\input{|\textit{main}|}|
\end{center}
%
A simple redirection |\childdocforward{|\textit{dest}|}| is achieved by:
%
\begin{center}
|\def\jobname{|\textit{dest}|}\input{\jobname}|
\end{center}
%
The redirection with prefix
|\childdocforwardprefix[|\textit{prefix}|]{|\textit{dest}|}|
is accomplished by:
%
\begin{center}
\begin{tabular}{l}
|{\edef\jobname{\scantokens\expandafter{\jobname\noexpand}}|\\
|\def\redirectjob |\textit{prefix}|#1~~~{\gdef\jobname{|\textit{dest}|#1}}|\\
|\expandafter\redirectjob\jobname~~~}\input{\jobname}|
\end{tabular}
\end{center}

In an alternative approach,
child documents can be compiled by a specific command line
without additional code or specific definitions:
%
\begin{center}
|... -jobname "|\textit{target}|" "|[\textit{flags}]%
|\includeonly{|\textit{dest}|}\input{|\textit{main}|}"|
\end{center}
%

%%%%%%%%%%%%%%%%%%%%%%%%%%%%%%%%%%%%%%%%%%%%%%%%%%%%%%%%%%%%%%%%%%%%%%%%%%%%%%%%
%%%%%%%%%%%%%%%%%%%%%%%%%%%%%%%%%%%%%%%%%%%%%%%%%%%%%%%%%%%%%%%%%%%%%%%%%%%%%%%%
\section{Information}

%%%%%%%%%%%%%%%%%%%%%%%%%%%%%%%%%%%%%%%%%%%%%%%%%%%%%%%%%%%%%%%%%%%%%%%%%%%%%%%%
\subsection{Copyright}

Copyright \copyright{} 2017--2018 Niklas Beisert

This work may be distributed and/or modified under the
conditions of the \LaTeX{} Project Public License, either version 1.3
of this license or (at your option) any later version.
The latest version of this license is in
  \url{http://www.latex-project.org/lppl.txt}
and version 1.3 or later is part of all distributions of \LaTeX{}
version 2005/12/01 or later.

This work has the LPPL maintenance status `maintained'.

The Current Maintainer of this work is Niklas Beisert.

This work consists of the files |README.txt|, |childdoc.ins| and |childdoc.dtx|
as well as the derived files |childdoc.def|, |cdocsamp.tex|
with |cdocsch1.tex|, |cdocsch2.tex|, |cdocspt3.tex|, |cdocspt4.tex|,
|cdocsdrf.tex|, |cdocsfn1.tex|, |cdocsfn2.tex|
as well as |childdoc.pdf|.

%%%%%%%%%%%%%%%%%%%%%%%%%%%%%%%%%%%%%%%%%%%%%%%%%%%%%%%%%%%%%%%%%%%%%%%%%%%%%%%%
\subsection{Files and Installation}

The package consists of the files:
%
\begin{center}
\begin{tabular}{ll}
    |README.txt|   & readme file \\
    |childdoc.ins| & installation file \\
    |childdoc.dtx| & source file \\
    |childdoc.def| & definition file \\
    |cdocsamp.tex| & sample main file \\
    |cdocsch1.tex| & sample include file \\
    |cdocsch2.tex| & sample include file \\
    |cdocspt3.tex| & sample part file \\
    |cdocspt4.tex| & sample part file \\
    |cdocsdrf.tex| & sample redirection file \\
    |cdocsfn1.tex| & sample redirection file \\
    |cdocsfn2.tex| & sample redirection file \\
    |childdoc.pdf| & manual
\end{tabular}
\end{center}
%
The distribution consists of the files
|README.txt|, |childdoc.ins| and |childdoc.dtx|.
%
\begin{itemize}
\item
Run (pdf)\LaTeX{} on |childdoc.dtx|
to compile the manual |childdoc.pdf| (this file).
\item
Run \LaTeX{} on |childdoc.ins| to create the definitions file |childdoc.def|
and the sample |cdocsamp.tex| with include files
|cdocsch1.tex|, |cdocsch2.tex|, |cdocspt3.tex|, |cdocspt4.tex|,
|cdocsdrf.tex|, |cdocsfn1.tex|, |cdocsfn2.tex|.
Then copy the file |childdoc.def| to an appropriate directory of your \LaTeX{}
distribution, e.g.\ \textit{texmf-root}|/tex/latex/childdoc|.
\end{itemize}

%%%%%%%%%%%%%%%%%%%%%%%%%%%%%%%%%%%%%%%%%%%%%%%%%%%%%%%%%%%%%%%%%%%%%%%%%%%%%%%%
\subsection{Related CTAN Packages}

There are several other packages which offer a similar functionality:
%
\begin{itemize}
\item
The packages
\href{http://ctan.org/pkg/docmute}{\textsf{docmute}},
\href{http://ctan.org/pkg/includex}{\textsf{includex}} and
\href{http://ctan.org/pkg/standalone}{\textsf{standalone}}
provide commands to include only the document body of
a child file thus allowing both files to be compiled individually.
\item
The packages \href{http://ctan.org/pkg/subdocs}{\textsf{subdocs}}
and \href{http://ctan.org/pkg/subfiles}{\textsf{subfiles}}
provide structures in which the main and child documents can be
encapsulated and allowing them to be compiled individually.
The inclusion mechanism is different from the conventional |\include|.
\item
The package \href{http://ctan.org/pkg/combine}{\textsf{combine}}
is an elaborate solution to combine several documents into one.
\end{itemize}
%
See also the CTAN topic \href{http://ctan.org/topic/subdocs}{\textsf{subdocs}}
for further related packages.
The present package differs from the above solutions in that
a document structure constructed with the conventional |\include| mechanism
just needs two extra commands at the top of every file
such that all constituent files can be compiled individually.

%%%%%%%%%%%%%%%%%%%%%%%%%%%%%%%%%%%%%%%%%%%%%%%%%%%%%%%%%%%%%%%%%%%%%%%%%%%%%%%%
%\subsection{Feature Suggestions}
%
%The following is a list of features which may be useful for future
%versions of this package:
%%
%\begin{itemize}
%\item
%\ldots
%\end{itemize}

%%%%%%%%%%%%%%%%%%%%%%%%%%%%%%%%%%%%%%%%%%%%%%%%%%%%%%%%%%%%%%%%%%%%%%%%%%%%%%%%
\subsection{Revision History}

%%%%%%%%%%%%%%%%%%%%%%%%%%%%%%%%%%%%%%%%
\paragraph{v2.0:} 2018/12/30

\begin{itemize}
\item
immediate forward processing
\item
added |\childdocby| mechanism
\item
manual restructured
\end{itemize}

%%%%%%%%%%%%%%%%%%%%%%%%%%%%%%%%%%%%%%%%
\paragraph{v1.6:} 2018/01/17

\begin{itemize}
\item
application for development of include files
\item
corrections to manual
\end{itemize}

%%%%%%%%%%%%%%%%%%%%%%%%%%%%%%%%%%%%%%%%
\paragraph{v1.5:} 2017/05/21

\begin{itemize}
\item
more complete structuring introduced
\item
|\childdocof| introduced
\item
|\childdoc| renamed to |\childdocmain|
\item
|\childredirect| renamed to |\childdocforward| and |\childdocforwardprefix|
and functionality expanded
\end{itemize}

%%%%%%%%%%%%%%%%%%%%%%%%%%%%%%%%%%%%%%%%
\paragraph{v1.0:} 2017/04/27

\begin{itemize}
\item
manual and install package
\item
first version published on CTAN
\end{itemize}

%%%%%%%%%%%%%%%%%%%%%%%%%%%%%%%%%%%%%%%%
\paragraph{v0.6:} 2017/04/26

\begin{itemize}
\item
redirection mechanism added
\end{itemize}

%%%%%%%%%%%%%%%%%%%%%%%%%%%%%%%%%%%%%%%%
\paragraph{v0.5:} 2017/04/26

\begin{itemize}
\item
functionality in definition file
\end{itemize}


%%%%%%%%%%%%%%%%%%%%%%%%%%%%%%%%%%%%%%%%%%%%%%%%%%%%%%%%%%%%%%%%%%%%%%%%%%%%%%%%
%%%%%%%%%%%%%%%%%%%%%%%%%%%%%%%%%%%%%%%%%%%%%%%%%%%%%%%%%%%%%%%%%%%%%%%%%%%%%%%%
%%%%%%%%%%%%%%%%%%%%%%%%%%%%%%%%%%%%%%%%%%%%%%%%%%%%%%%%%%%%%%%%%%%%%%%%%%%%%%%%
\appendix

\settowidth\MacroIndent{\rmfamily\scriptsize 000\ }

 \DocInput{childdoc.dtx}

\end{document}
%</driver>
% \fi
%
% %%%%%%%%%%%%%%%%%%%%%%%%%%%%%%%%%%%%%%%%%%%%%%%%%%%%%%%%%%%%%%%%%%%%%%%%%%%%%%
% %%%%%%%%%%%%%%%%%%%%%%%%%%%%%%%%%%%%%%%%%%%%%%%%%%%%%%%%%%%%%%%%%%%%%%%%%%%%%%
% \section{Sample}
%\iffalse
%<*samplemain>
%\fi
%
% The following presents a sample document
% with two chapters, two parts, a title page,
% a compile flag as well as three forwarding files to set the flag.
% It consists of eight |.tex| files:
% \begin{center}
% \begin{tabular}{ll}
% |cdocsamp.tex|&main file\\
% |cdocsch1.tex|&include file for chapter 1\\
% |cdocsch2.tex|&include file for chapter 2\\
% |cdocspt3.tex|&include file for part 3\\
% |cdocspt4.tex|&include file for part 4\\
% |cdocsdrf.tex|&forwarding file for main file in draft mode\\
% |cdocsfi1.tex|&forwarding file for final version of chapter 1\\
% |cdocsfi2.tex|&forwarding file for final version of chapter 2\\
% \end{tabular}
% \end{center}
% Each of the eight files can be compiled directly by the \LaTeX{} compiler.
%
% %%%%%%%%%%%%%%%%%%%%%%%%%%%%%%%%%%%%%%
% \paragraph{Main File.}
%
% The main file is called |cdocsamp.tex|.
%
% Load the \textsf{childdoc} definitions and
% declare the filename for the main document:
%    \begin{macrocode}
\input{childdoc.def}
\childdocmain{}
%    \end{macrocode}

% Optional override for |\version| flag:
%    \begin{macrocode}
%%\ifchilddoc\else\providecommand{\version}{draft}\fi
%    \end{macrocode}

% Define the default values for the |\version| flag
% (|final| for the main file and |draft| for childs):
%    \begin{macrocode}
\ifchilddoc
\providecommand{\version}{draft}
\else
\providecommand{\version}{final}
\fi
%    \end{macrocode}

% Load the standard document class:
%    \begin{macrocode}
\documentclass[12pt]{article}
%    \end{macrocode}

% Start the document body:
%    \begin{macrocode}
\begin{document}
%    \end{macrocode}

% Declare a title page.
% Print title, part of document being processed and version flag:
%    \begin{macrocode}
\addtocounter{page}{-1}
\begin{center}
{\LARGE\bfseries{}childdoc example\par}
\vspace{1cm}
\ifchilddoc
\ifchilddocmanual part\else chapter\fi:
`\childdocname' of `\childdocjob'\par
\else
main document: `\childdocjob'\par
\fi
version: \version\par
\end{center}
\newpage
%    \end{macrocode}

% Manually include selected file,
% otherwise process as usual:
%    \begin{macrocode}
\ifchilddocmanual
\section*{part `\childdocname'}
\input{\childdocname}
\else
%    \end{macrocode}

% Include the two chapters:
%    \begin{macrocode}
\include{cdocsch1}
\include{cdocsch2}
%    \end{macrocode}

% Include the two parts unless only chapters should be displayed:
%    \begin{macrocode}
\ifchilddoc\else
\section{part three}
\input{cdocspt3}
\section{part four}
\input{cdocspt4}
\fi
%    \end{macrocode}

% Process as usual until here:
%    \begin{macrocode}
\fi
%    \end{macrocode}

% End of document body:
%    \begin{macrocode}
\end{document}
%    \end{macrocode}
%\iffalse
%</samplemain>
%\fi
%
% %%%%%%%%%%%%%%%%%%%%%%%%%%%%%%%%%%%%%%
% \paragraph{Chapter Include Files.}
%
% The include files are called |cdocsch1.tex| and |cdocsch2.tex|.
%
%\iffalse
%<*samplechap1|samplechap2>
%\fi

% Optional override for |\version| flag:
%    \begin{macrocode}
%%\providecommand{\version}{final}
%    \end{macrocode}

% Include the main document:
%    \begin{macrocode}
\input{childdoc.def}
\childdocof{cdocsamp}
%    \end{macrocode}

%\iffalse
%</samplechap1|samplechap2>
%\fi
%
%\iffalse
%<*samplechap1>
%\fi
% Some text for chapter 1:
%    \begin{macrocode}
\section{one}
some text in chapter one
%    \end{macrocode}

%\iffalse
%</samplechap1>
%\fi
% Some text for chapter 2:
%\iffalse
%<*samplechap2>
%\fi
%    \begin{macrocode}
\section{two}
more text in chapter two
%    \end{macrocode}

%\iffalse
%</samplechap2>
%\fi
%
% %%%%%%%%%%%%%%%%%%%%%%%%%%%%%%%%%%%%%%
% \paragraph{Part Include Files.}
%
% The include files are called |cdocspt3.tex| and |cdocspt4.tex|.
%
%\iffalse
%<*samplepart3|samplepart4>
%\fi

% Optional override for |\version| flag:
%    \begin{macrocode}
%%\providecommand{\version}{final}
%    \end{macrocode}

% Include the main document:
%    \begin{macrocode}
\input{childdoc.def}
\childdocby{cdocsamp}
%    \end{macrocode}

%\iffalse
%</samplepart3|samplepart4>
%\fi
%
%\iffalse
%<*samplepart3>
%\fi
% Some text for part 3:
%    \begin{macrocode}
some text in part three
%    \end{macrocode}

%\iffalse
%</samplepart3>
%\fi
% Some text for part 4:
%\iffalse
%<*samplepart4>
%\fi
%    \begin{macrocode}
more text in part four
%    \end{macrocode}

%\iffalse
%</samplepart4>
%\fi
%
% %%%%%%%%%%%%%%%%%%%%%%%%%%%%%%%%%%%%%%
% \paragraph{Forwarding for a Complete Draft.}
%
% The following forwarding file |cdocsdrf.tex|
% compiles the main document in draft mode:
%\iffalse
%<*sampledraft>
%\fi
%    \begin{macrocode}
\def\version{draft}
\input{childdoc.def}
\childdocforward{cdocsamp}
%    \end{macrocode}

%\iffalse
%</sampledraft>
%\fi
%
% %%%%%%%%%%%%%%%%%%%%%%%%%%%%%%%%%%%%%%
% \paragraph{Forwarding for Final Version of the Chapters.}
%
% The following forwarding files |cdocsfn1.tex| and |cdocsfn2.tex|
% (with identical content)
% compile the final versions of the child documents
% |cdocsch1.tex| and |cdocsch2.tex|, respectively:
%\iffalse
%<*samplefinal>
%\fi
%    \begin{macrocode}
\def\version{final}
\input{childdoc.def}
\childdocforwardprefix[cdocsamp]{cdocsfn}{cdocsch}
%    \end{macrocode}

%\iffalse
%</samplefinal>
%\fi
%
% %%%%%%%%%%%%%%%%%%%%%%%%%%%%%%%%%%%%%%
% \paragraph{Command Line Processing.}
%
% The following three command lines generate the output files
% |cdocscld|, |cdocscl1| and |cdocscl2|
% which should be identical to
% |cdocsdrf|, |cdocsch1| and |cdocsfn2|, respectively:
% \begin{center}
% \begin{tabular}{l}
% |latex -jobname cdocscld \|\\
% |  "\def\version{draft}\input{childdoc.def}\childdocforward{cdocsamp}"|\\
% |latex -jobname cdocscl1 \|\\
% |  "\input{childdoc.def}\childdocforward[cdocsamp]{cdocsch1}"|\\
% |latex -jobname cdocscl2 \|\\
% |  "\def\version{final}\input{childdoc.def}\childdocforward{cdocsch2}"|
% \end{tabular}
% \end{center}
% Note that the trailing backslash on each first line
% merely continues the input to the second line
% (for convenient cut ant paste).
% Furthermore, the command |latex| can be replaced by any
% of its alternative versions such as |pdflatex|.
%
% %%%%%%%%%%%%%%%%%%%%%%%%%%%%%%%%%%%%%%%%%%%%%%%%%%%%%%%%%%%%%%%%%%%%%%%%%%%%%%
% %%%%%%%%%%%%%%%%%%%%%%%%%%%%%%%%%%%%%%%%%%%%%%%%%%%%%%%%%%%%%%%%%%%%%%%%%%%%%%
% \section{Implementation}
%\iffalse
%<*package>
%\fi
%
% This section describes the definitions file |childdoc.def|.

% The definitions cannot be loaded using |\usepackage| or |\RequirePackage|
% which has a mechanism to prevent loading a style file more than once.
% When loading the definitions by means of |\input|
% multiple instances have to be prevented manually:
%\iffalse
%This code needs to be before the `\ProvidesFile' directive
%which is defined at the beginning of this file.
%Therefore it is also placed there and commented out here.
%</package>
%<*discard>
%\fi
%    \begin{macrocode}
\ifdefined\childdocmain\endinput\fi
%    \end{macrocode}
%\iffalse
%</discard>
%<*package>
%\fi
%
% \macro{\ifchilddoc}
% \macro{\ifchilddocmanual}
% The conditional |\ifchilddoc| tells whether a
% child (true) or main (false) document is being compiled.
% The conditional |\ifchilddocmanual| tells whether
% the |\includeonly| mechanism is used (false) or
% the selection of child files must be performed manually (true).
% The definitions initialise to false:
%    \begin{macrocode}
\newif\ifchilddoc
\newif\ifchilddocmanual
%    \end{macrocode}

% \macro{\childdocname}
% \macro{\childdocjob}
% The macro |\childdocname| stores the name of the main document
% to be compiled. The macro |\childdocjob| stores the name of
% the document on which the \LaTeX{} compiler was originally invoked.
% The content of |\jobname| cannot be compared
% to filenames specified in the source due to different catcodes.
% The following code rescans |\jobname|, stores the result
% in |\childdocname| and saves a copy in |\childdocjob|:
%    \begin{macrocode}
\edef\childdocname{\scantokens\expandafter{\jobname\noexpand}}
\let\childdocjob\childdocname
%    \end{macrocode}

% \macro{\childdocdisable}
% The macro |\childdocdisable| prevents the main file
% from being processed more than once.
% At this stage, the main document command |\childdocmain|
% is assumed to be called once again where it should do nothing.
% Any subsequent call to it should prevent
% a secondary processing of the main document
% It overwrites the forwarding commands
% |\childdocof| and |\childdocforward|
% with empty macros to prevent further inclusions of the main document:
%    \begin{macrocode}
\newcommand{\childdocdisable}
{
  \renewcommand{\childdocmain}[1]{\renewcommand{\childdocmain}[1]{\endinput}}
  \renewcommand{\childdocof}[1]{}
  \renewcommand{\childdocby}[2][]{}
  \renewcommand{\childdocforward}[2][]{}
  \renewcommand{\childdocdisable}{}
}
%    \end{macrocode}

% \macro{\childdocmain}
% The macro |\childdocmain| is to be called at the top of the main file
% with nothing or the main filename (without extension) as argument.
% First, it breaks loops.
% If the argument is not empty and does not match |\childdocname|
% (which is set by the first inclusion of |childdoc.def|),
% |\ifchilddoc| is set to true, |\includeonly| is applied to the child file
% and |\jobname| is set to the main file
% (for proper handling of |.aux| files):
%    \begin{macrocode}
\newcommand{\childdocmain}[1]
{
  \childdocdisable\childdocmain{}
  \if?#1?\else
    \begingroup
      \def\childdoctmp{#1}
      \ifx\childdoctmp\childdocname
        \def\childdoctmp{}
      \else
        \def\childdoctmp
        {
          \childdoctrue
          \includeonly{\childdocname}
          \def\childdocjob{#1}
          \def\jobname{#1}
        }
      \fi
      \expandafter
    \endgroup
    \childdoctmp
  \fi
}
%    \end{macrocode}

% \macro{\childdocof}
% The command |\childdocof| redirects
% compilation to the main file |#1|.
%    \begin{macrocode}
\newcommand{\childdocof}[1]
{
  \childdocdisable
  \childdoctrue
  \includeonly{\childdocname}
  \def\jobname{#1}
  \def\childdocjob{#1}
  \input{#1}
}
%    \end{macrocode}

% \macro{\childdocby}
% The command |\childdocby| ....
%    \begin{macrocode}
\newcommand{\childdocby}[2][]
{
  \childdocdisable
  \childdoctrue
  \childdocmanualtrue
  \if?#1?\else
    \def\jobname{#2}
  \fi
  \def\childdocjob{#2}
  \input{#2}
  \endinput
}
%    \end{macrocode}

% \macro{\childdocforward}
% The command |\childdocforward| redirects
% compilation to the main file or
% (if the optional argument is given) a child file.
% Parameters are set as if the main file
% or a child file starting with |\childdocof| was compiled.
% Then compilation is handed over to the main file:
%    \begin{macrocode}
\newcommand{\childdocforward}[2][]
{
  \begingroup
    \if?#1?
      \def\childdoctmp
      {
        \def\childdocname{#2}
        \def\childdocjob{#2}
        \def\jobname{#2}
        \input{#2}
        \endinput
      }
    \else
      \def\childdoctmp
      {
        \childdocdisable
        \def\childdocname{#2}
        \childdoctrue
        \includeonly{#2}
        \def\childdocjob{#1}
        \def\jobname{#1}
        \input{#1}
        \endinput
      }
    \fi
    \expandafter
  \endgroup
  \childdoctmp
}
%    \end{macrocode}

% \macro{\childdocforwardprefix}
% The command |\childdocforwardprefix| redirects
% compilation to the main or a child file by means of a pattern.
% The prefix |#1| in the current filename is replaced by |#2|
% and the suffix of the current filename is kept
% (it is assumed that the filename does not contain the substring `|~~~|'
% which is used as a delimiter).
% Compilation is handed over to the new file by |\childdocforward|:
%    \begin{macrocode}
\newcommand{\childdocforwardprefix}[3][]
{
  \begingroup
    \def\childdocextract #2##1~~~{\def\childdoctmp{\childdocforward[#1]{#3##1}}}
    \expandafter\childdocextract\childdocname~~~
    \expandafter
  \endgroup
  \childdoctmp
}
%    \end{macrocode}

% \macro{\childdoc}
% The deprecated macro |\childdoc| is a legacy version of |\childdocmain|:
%    \begin{macrocode}
\newcommand{\childdoc}{\childdocmain}
%    \end{macrocode}

% \macro{\childdocredirect}
% The deprecated macro |\childdocredirect| is a legacy version
% of |\childdocforward| and |\childdocforwardprefix|:
%    \begin{macrocode}
\newcommand{\childdocredirect}[2][]
{
  \begingroup
    \if?#1?
      \def\childdoctmp{\childdocforward{#2}}
    \else
      \def\childdoctmp{\childdocforwardprefix{#1}{#2}}
    \fi
    \expandafter
  \endgroup
  \childdoctmp
}
%    \end{macrocode}

%\iffalse
%</package>
%\fi
%
\endinput
|\\
|\childdocforward{|\textit{main}|}|\\
\end{tabular}
\end{center}
%
or alternatively with:
%
\begin{center}
\begin{tabular}{l}
|% \iffalse
%
% childdoc.dtx Copyright (C) 2017-2018 Niklas Beisert
%
% This work may be distributed and/or modified under the
% conditions of the LaTeX Project Public License, either version 1.3
% of this license or (at your option) any later version.
% The latest version of this license is in
%   http://www.latex-project.org/lppl.txt
% and version 1.3 or later is part of all distributions of LaTeX
% version 2005/12/01 or later.
%
% This work has the LPPL maintenance status `maintained'.
%
% The Current Maintainer of this work is Niklas Beisert.
%
% This work consists of the files childdoc.dtx and childdoc.ins
% and the derived files childdoc.def and cdocsamp.tex with
% cdocsch1.tex, cdocsch2.tex, cdocsdrf.tex, cdocsfn1.tex, cdocsfn2.tex.
%
%<package>\ifdefined\childdocmain\endinput\fi
%<package>\ProvidesFile{childdoc.def}[2018/12/30 v2.0 child document driver]
%<samplemain>\ProvidesFile{cdocsamp.tex}[2018/12/30 v2.0 sample for childdoc]
%<*driver>
%\ProvidesFile{childdoc.drv}[2018/12/30 v2.0 childdoc reference manual file]
\PassOptionsToClass{10pt,a4paper}{article}
\documentclass{ltxdoc}

\usepackage[margin=35mm]{geometry}
\usepackage{hyperref}
\usepackage{hyperxmp}
\usepackage[usenames]{color}

\hypersetup{colorlinks=true}
\hypersetup{pdfstartview=FitH}
\hypersetup{pdfpagemode=UseNone}
\hypersetup{pdfsource={}}
\hypersetup{pdflang={en-UK}}
\hypersetup{pdfcopyright={Copyright 2017-2018 Niklas Beisert.
  This work may be distributed and/or modified under the
  conditions of the LaTeX Project Public License, either version 1.3
  of this license or (at your option) any later version.}}
\hypersetup{pdflicenseurl={http://www.latex-project.org/lppl.txt}}
\hypersetup{pdfcontactaddress={ETH Zurich, ITP, HIT K,
  Wolfgang-Pauli-Strasse 27}}
\hypersetup{pdfcontactpostcode={8093}}
\hypersetup{pdfcontactcity={Zurich}}
\hypersetup{pdfcontactcountry={Switzerland}}
\hypersetup{pdfcontactemail={nbeisert@itp.phys.ethz.ch}}
\hypersetup{pdfcontacturl={http://people.phys.ethz.ch/\xmptilde nbeisert/}}

\newcommand{\secref}[1]{\hyperref[#1]{section \ref*{#1}}}

\parskip1ex
\parindent0pt
\let\olditemize\itemize
\def\itemize{\olditemize\parskip0pt}

\begin{document}

\title{The \textsf{childdoc} Package}
\hypersetup{pdftitle={The childdoc Package}}
\author{Niklas Beisert\\[2ex]
  Institut f\"ur Theoretische Physik\\
  Eidgen\"ossische Technische Hochschule Z\"urich\\
  Wolfgang-Pauli-Strasse 27, 8093 Z\"urich, Switzerland\\[1ex]
  \href{mailto:nbeisert@itp.phys.ethz.ch}
  {\texttt{nbeisert@itp.phys.ethz.ch}}}
\hypersetup{pdfauthor={Niklas Beisert}}
\hypersetup{pdfsubject={Manual for the LaTeX2e Package childdoc}}
\date{30 December 2018, \textsf{v2.0}}
\maketitle

\begin{abstract}\noindent
\textsf{childdoc} is a \LaTeXe{} package
that enables the direct compilation
of document sections included by |\include|
to individual files.
\end{abstract}

\begingroup
\parskip0ex
\tableofcontents
\endgroup

%%%%%%%%%%%%%%%%%%%%%%%%%%%%%%%%%%%%%%%%%%%%%%%%%%%%%%%%%%%%%%%%%%%%%%%%%%%%%%%%
%%%%%%%%%%%%%%%%%%%%%%%%%%%%%%%%%%%%%%%%%%%%%%%%%%%%%%%%%%%%%%%%%%%%%%%%%%%%%%%%
\section{Introduction}

\LaTeX{} provides a mechanism to structure a large document (such as a book)
into a main file and several child files (containing the chapters)
using the |\include| command.
This mechanism is beneficial for documents
which span hundreds of pages in order to
make the source file(s) more manageable.
Moreover, compilation can be restricted to
selected child files by means of the |\includeonly| command.
The latter feature can be used to reduce the compilation time while editing
(this was significantly more useful in the earlier days of \LaTeX{})
or to generate a smaller document which is easier to navigate.
Another application of |\includeonly| is to generate
documents consisting of selected parts of the complete document.

However, there are a few drawbacks of the plain |\include| mechanism:
\begin{itemize}
\item
The child files cannot be compiled on their own,
they can only be compiled via the main file.
A naive editing environment
(such as a text editor with an option
to have the current file processed by \LaTeX)
may require one to switch to the main file before compiling;
attempting to compile the child file produces errors.
\item
The main file must be modified (each time)
to adjust the |\includeonly| command
to the present needs. This easily leaves the main file in a messy state.
\item
The generated document will always carry the filename
of the main document. This is inconvenient if
several child files are to be compiled and
to be kept for distribution.
\end{itemize}

The present package provides a simple interface
to make child files individually compilable by \LaTeX{}.
Compiling a child file then has the same effect as compiling
the main file with an |\includeonly| command
to select the appropriate child.
Moreover the generated document will carry the name of the child
rather than the main file.
This resolves all three above issues.

This feature is meant to make the editing of books,
thesis documents and lecture notes somewhat more convenient.
However, the package can also be used efficiently for
composing a series of documents (such as exercise sheets)
which are typically distributed individually.
It then assists the author in generating the individual documents
(potentially in different versions)
as well as a document containing the collected series.
Another application is in developing style files
or other kinds of included material
where compilation of the style file could redirect
to a sample or test file.

%%%%%%%%%%%%%%%%%%%%%%%%%%%%%%%%%%%%%%%%%%%%%%%%%%%%%%%%%%%%%%%%%%%%%%%%%%%%%%%%
%%%%%%%%%%%%%%%%%%%%%%%%%%%%%%%%%%%%%%%%%%%%%%%%%%%%%%%%%%%%%%%%%%%%%%%%%%%%%%%%
\section{Usage}

First of all, the package \textsf{childdoc} is \emph{not} a standard
\LaTeXe{} |.sty| style file! Therefore it needs to be invoked in
a non-standard way.

%%%%%%%%%%%%%%%%%%%%%%%%%%%%%%%%%%%%%%%%%%%%%%%%%%%%%%%%%%%%%%%%%%%%%%%%%%%%%%%%
\subsection{Included Files}
\label{sec:include}

%%%%%%%%%%%%%%%%%%%%%%%%%%%%%%%%%%%%%%%%
\DescribeMacro{\childdocmain}
To use the package, add the commands
\begin{center}
\begin{tabular}{l}
|\input{childdoc.def}|\\
|\childdocmain{}|\\
\end{tabular}
\end{center}
at the very top of the main \LaTeX{} file,
in particular \emph{before} the |\documentclass| statement!
The argument of |\childdocmain| should be left empty
(but it must be present).

%%%%%%%%%%%%%%%%%%%%%%%%%%%%%%%%%%%%%%%%
\DescribeMacro{\childdocof}
Furthermore, add the commands
\begin{center}
\begin{tabular}{l}
|\input{childdoc.def}|\\
|\childdocof{|\textit{main}|}|\\
\end{tabular}
\end{center}
at the top of every child file \textit{child}
which is included by |\include{|\textit{child}|}|
from within the main file
(or at least for those files to be compiled individually).
The argument \textit{main} must be the filename of the main file.

There are a couple of
considerations in setting up the main and child documents:

%%%%%%%%%%%%%%%%%%%%%%%%%%%%%%%%%%%%%%%%
\paragraph{Restrictions.}

Please note the following restrictions:
\begin{itemize}
\item
|\childdocmain| must be called with one argument \textit{main}
to ensure compatibility with earlier version of the package.
It must either be empty (|\childdocmain{}|)
or precisely match the filename of the main file in which it is specified.
See \secref{sec:detection} for further information.
\item
The filename \textit{main} must be specified without the |.tex| extension.
\item
The filename \textit{main} is case sensitive
(even in case-insensitive file systems)
due to internal string comparison.
\item
The argument \textit{main} should be fully expanded, it cannot be a macro.
\item
Subdirectories and special characters should be avoided in filenames.
\item
The command |\childdocmain{|\textit{main}|}| must be followed by a whitespace.
It should not be followed immediately by another command
or by a comment mark `|%|'.
This is because the \TeX{} parser reads the token immediately following
the argument of |\childdocmain| and puts it
at the beginning of every child section;
however, a white\-space is ignored.
\end{itemize}

%%%%%%%%%%%%%%%%%%%%%%%%%%%%%%%%%%%%%%%%
\paragraph{Content of Main File.}

It is advisable to place all content in the child files included by |\include|.
Any output contained in the main file will appear in all child documents
unless suppressed manually;
it cannot be suppressed automatically by the |\includeonly| directive
and thus should normally be avoided.
A method to include some content in the main file
by means of conditional processing is described in \secref{sec:conditional}.

%%%%%%%%%%%%%%%%%%%%%%%%%%%%%%%%%%%%%%%%
\paragraph{Page Numbering.}

When only a part of the document is compiled,
the appropriate numbering of pages
(as well as other status parameters)
is determined from the |.aux| files.
The latter contain information from previous passes.
However this information needs to propagate through
all intermediate child documents.
Therefore the page numbering in child documents may well
be inconsistent until the complete document is compiled at least once.

A useful (if unconventional) way to always ensure a consistent
page numbering is to restart the numbering in each child document
and denote the pages by `\textit{child}|.|\textit{page}'
where \textit{child} represents the chapter/section number of the child file.
This can be achieved by the command
|\numberwithin{page}{|\textit{child}|}|
of the \textsf{amsmath} package
where \textit{child} can be |chapter| or |section|
depending on the chosen structuring.
Alternatively, one can modify the macro |\thepage| appropriately
and reset the counter |page| at the start of each child file.

%%%%%%%%%%%%%%%%%%%%%%%%%%%%%%%%%%%%%%%%%%%%%%%%%%%%%%%%%%%%%%%%%%%%%%%%%%%%%%%%
\subsection{Conditional Processing}
\label{sec:conditional}

The package provides a mechanism to compile different versions
of a document. To customise the versions further some conditional processing
can come in handy to distinguish which version is being compiled.
The package provides two macros to describe the compilation context:

%%%%%%%%%%%%%%%%%%%%%%%%%%%%%%%%%%%%%%%%
\DescribeMacro{\ifchilddoc}
The conditional |\ifchilddoc| distinguishes between the compilation of
child documents and the main document:
%
\begin{center}
|\ifchilddoc |\textit{child-code}| |[|\||else |\textit{main-code}]| \||fi|
\end{center}

%%%%%%%%%%%%%%%%%%%%%%%%%%%%%%%%%%%%%%%%
\DescribeMacro{\childdocname}
\DescribeMacro{\childdocjob}
The macro |\childdocname| contains the filename (without extension)
of the main or child file being processed.
Note that |\childdocjob| will always contain the name of the main file.

%%%%%%%%%%%%%%%%%%%%%%%%%%%%%%%%%%%%%%%%
\paragraph{Title Page.}

Conditional processing can be used to include a title or banner page
in the main document when proper precautions are taken.
Importantly, the code in the main file should ensure that the page counter
(as well as other status parameters which are stored in the |.aux| files)
takes the same value after the conditional processing.
Otherwise the page numbers may take divergent values
depending on which part is compiled.

For example, a title page could be declared by:
%
\begin{center}
\begin{tabular}{l}
|\ifchilddoc\||else|\\
|\addtocounter{page}{-1}|\\
\textit{code for title page}\\
|\newpage|\\
|\||fi|
\end{tabular}
\end{center}
%
A banner page for the child documents can be generated by:
%
\begin{center}
\begin{tabular}{l}
|\ifchilddoc|\\
|\addtocounter{page}{-1}|\\
\textit{code for banner page}\\
|\newpage|\\
|\||fi|
\end{tabular}
\end{center}
%
Here one could write a message such as:
\begin{center}
|This is the part \childdocname{} of \childdocjob{}.|
\end{center}

%%%%%%%%%%%%%%%%%%%%%%%%%%%%%%%%%%%%%%%%%%%%%%%%%%%%%%%%%%%%%%%%%%%%%%%%%%%%%%%%
\subsection{Flags}
\label{sec:flags}

The package makes it easy to generate different versions
of the main or child documents.
To this end compilation flags can be defined
and assigned different default values.
They will be particularly useful in conjunction
with the forwarding mechanism described in \secref{sec:forward}.

For example, it may be useful to have a flag |\version|
which can be set to |draft| or |final|.
The document source will contain some conditional code
depending on the value of |\version|.
Suppose further, the flag should default to |final| for the main file
and to |draft| for child files
which is a natural assignment for editing the document.
This is achieved by placing the following code
in the preamble of the main document
(below the |\childdocmain| directive):
%
\begin{center}
\begin{tabular}{l}
|\ifchilddoc|\\
|\providecommand{\version}{draft}|\\
|\||else|\\
|\providecommand{\version}{final}|\\
|\||fi|
\end{tabular}
\end{center}
%
The definition by |\providecommand| makes sure
that previous definitions are not overwritten.
Further statements |\providecommand{\version}{...}|
can thus be added before the above code to override it.

For the main file, one might add a line
(between |\childdocmain| and the above block)
%
\begin{center}
|%\ifchilddoc\||else\providecommand{\version}{draft}\||fi|
\end{center}
%
which can be uncommented to produce a draft version.
Likewise one can add a line to the very top of a child file
(above the |\childdocof{|\textit{main}|}| directive)
%
\begin{center}
|%\providecommand{\version}{final}|
\end{center}
%
which can be uncommented to produce the final version of this child document.

%%%%%%%%%%%%%%%%%%%%%%%%%%%%%%%%%%%%%%%%%%%%%%%%%%%%%%%%%%%%%%%%%%%%%%%%%%%%%%%%
\subsection{Forwarding}
\label{sec:forward}

Different versions of the main or child documents
using compilation flags as described in \secref{sec:flags}
can be (permanently) stored in different files
for convenient compilation, viewing and distribution.
To this end, the package defines a command
to pass on compilation to a different file:

%%%%%%%%%%%%%%%%%%%%%%%%%%%%%%%%%%%%%%%%
\DescribeMacro{\childdocforward}
The command |\childdocforward| redirects processing to
another source file:
%
\begin{center}
\begin{tabular}{l}
|\input{childdoc.def}|\\
|\childdocforward[|\textit{main}|]{|\textit{dest}|}|\\
\end{tabular}
\end{center}
%
The argument \textit{dest} is the destination file
(without extension).
It should be the main file or one of the child files.
Note that further \textsf{childdoc} directives
such as |\childdocof| and |\childdocforward|
in the indicated file will be processed in this form.
The optional argument \textit{main}
passes on directly to the main file \textit{main}
while pretending to compile the child \textit{dest}.
This form behaves as if \textit{dest}
issues |\childdocof{|\textit{main}|}| right away,
and no further \textsf{childdoc} directives will be processed.

%%%%%%%%%%%%%%%%%%%%%%%%%%%%%%%%%%%%%%%%
\DescribeMacro{\...prefix}
In the alternative form |\childdocforwardprefix|,
%
\begin{center}
\begin{tabular}{l}
|\input{childdoc.def}|\\
|\childdocforwardprefix[|\textit{main}|]{|\textit{prefix}|}{|\textit{dest}|}|
\end{tabular}
\end{center}
%
the destination file is determined by a pattern
depending on the current file:
To make this work, the current file must be called
`{\textit{prefix}\hspace{0.2em}\textit{suffix}}'
with \textit{prefix} matching precisely the argument.
Processing is then passed on to the file
`{\textit{dest}\hspace{0.2em}\textit{suffix}}'.
Surely, the same effect is achieved by
directly specifying the
argument `{\textit{dest}\hspace{0.2em}\textit{suffix}}'
in the first form.
However, that requires to set up a different file
for each child. With the alternative form of the command
all these files can have exactly the same content
which simplifies setting them up and maintaining them.

For example, the following file |draft.tex|
with a compilation flag |\version| as described in \secref{sec:flags}
compiles the main document as a draft:
%
\begin{center}
\begin{tabular}{l}
|\def\version{draft}|\\
|\input{childdoc.def}|\\
|\childdocforward{|\textit{main}|}|
\end{tabular}
\end{center}
%
Likewise, the following files |final|\textit{nn}|.tex|
compile the final version of the child document
|child|\textit{nn}|.tex|:
%
\begin{center}
\begin{tabular}{l}
|\def\version{final}|\\
|\input{childdoc.def}|\\
|\childdocforwardprefix{final}{child}|
\end{tabular}
\end{center}
%

Note that when several versions of a main file and/or of each child file
are to be generated, it may be convenient to set up a |Makefile| or
shell script to automatise the process.

%%%%%%%%%%%%%%%%%%%%%%%%%%%%%%%%%%%%%%%%%%%%%%%%%%%%%%%%%%%%%%%%%%%%%%%%%%%%%%%%
\subsection{Command Line Processing}
\label{sec:commandline}

The effect of redirection files can also be achieved by invoking
the \LaTeX{} compiler with a more elaborate command line.
Most conveniently this should be done as part
of a shell script or a |Makefile|.

When using \textsf{childdoc} in the main file, the following
command lines effectively perform a redirection
(note that depending on the shell being used,
backslashes may have to be doubled: `|\|' $\to$ `|\\|'):
%
\begin{center}
|... -jobname "|\textit{target}|" |\\|"|[\textit{flags}]%
|\input{childdoc.def}\childdocforward[|\textit{main}|]{|\textit{dest}|}"|
\end{center}
%
Here \textit{target} is the name of the output file,
\textit{main} is the name of the main file
and \textit{dest} is the name of the main or child file to be processed
(all filenames without extensions).
The optional argument \textit{main} can be omitted
if \textit{main} matches \textit{dest}.
Optionally, compilation \textit{flags} can be defined via |\def| commands.
This command line makes the \TeX{} engine believe
it is compiling the file \textit{target}
whose content is specified as the latter parameter.
The provided code then forwards the processing to
\textit{main} or \textit{dest} as described in \secref{sec:forward}.

%%%%%%%%%%%%%%%%%%%%%%%%%%%%%%%%%%%%%%%%%%%%%%%%%%%%%%%%%%%%%%%%%%%%%%%%%%%%%%%%
\subsection{Include by Input}
\label{sec:input}

Including child documents by |\include| has some restrictions by design.
Most notably, the content of a child document always occupies
its own set of pages; pages cannot be shared between child documents.
Usually, this behaviour makes perfect sense
because each child document contain an essential part of the document.
However, in some situations it may be desirable to compose
a document from a collection of parts
without having mandatory page breaks between then.
For this case, the package
provides a mechanism to include parts
by |\input| which can also be processed individually.
However, by construction this mechanism
requires manual handling of the content to be output.

%%%%%%%%%%%%%%%%%%%%%%%%%%%%%%%%%%%%%%%%
\DescribeMacro{\ifchilddocmanual}
The main file should be prepared as usual, see \secref{sec:include}.
However, the document body must make a distinction
between processing of an individual part and of the main document, e.g.:
%
\begin{center}
\begin{tabular}{l}
|\ifchilddocmanual|\\
|\input{\childdocname}|\\
|\||else|\\
\textit{document body with }|\input{|\textit{part}|}|\\
|\||fi|
\end{tabular}
\end{center}
%
The conditional |\ifchilddocmanual| is true whenever
a part to be included by |\input| is being compiled,
and the name of the part is stored in |\childdocname|.

%%%%%%%%%%%%%%%%%%%%%%%%%%%%%%%%%%%%%%%%
\DescribeMacro{\childdocby}
Each part to be included by |\input| should start with:
%
\begin{center}
\begin{tabular}{l}
|\input{childdoc.def}|\\
|\childdocby{|\textit{main}|}|\\
\end{tabular}
\end{center}
%
The directive |\childdocby| is similar to |\childdocof|
described in \secref{sec:include},
but the subsequent selection of content must be done manually.
To that end, both |\ifchilddoc| and |\ifchilddocmanual|
will be true upon processing of a part,
and the name of the part is stored in |\childdocname|.
Note that |\jobname| will be set to the filename of the current part
so that each part receives an individual |.aux| file
that does not interfere with the |.aux| file(s) of the main document.
This behaviour can be altered by the alternative form
|\childdocby[*]{|\textit{main}|}| (with a non-empty optional argument)
which uses the |.aux| file of the main document
by setting |\jobname| to \textit{main}.

%%%%%%%%%%%%%%%%%%%%%%%%%%%%%%%%%%%%%%%%%%%%%%%%%%%%%%%%%%%%%%%%%%%%%%%%%%%%%%%%
\subsection{Driver Development}
\label{sec:driver}

The \textsf{childdoc} mechanism can also be use for the development
of definition files such as \LaTeX{} styles or classes.
This case differs from the above setup with multiple parts
included by |\include| in that no |\includeonly| should be invoked.
This can be achieved by starting the include file
(before |\ProvidesPackage|) with:
%
\begin{center}
\begin{tabular}{l}
|\input{childdoc.def}|\\
|\childdocforward{|\textit{main}|}|\\
\end{tabular}
\end{center}
%
or alternatively with:
%
\begin{center}
\begin{tabular}{l}
|\input{childdoc.def}|\\
|\childdocby{|\textit{main}|}|\\
\end{tabular}
\end{center}
%
Both forms have slightly different effects as described above.
The main file is prepared as usual, see \secref{sec:include}.

%%%%%%%%%%%%%%%%%%%%%%%%%%%%%%%%%%%%%%%%%%%%%%%%%%%%%%%%%%%%%%%%%%%%%%%%%%%%%%%%
\subsection{Legacy Detection}
\label{sec:detection}

The directive |\childdocmain| in the main file can detect
whether the complete document or merely a child is to be compiled
even without using the directive |\childdocof|.
This method is deprecated because it is less robust
and there is no compelling reason to use it;
it is merely provided for backward compatibility
and it may be removed in future versions.

If the detection mechanism is to be used,
it is mandatory to correctly specify
the filename of the main file as the argument of |\childdocmain|:
%
\begin{center}
\begin{tabular}{l}
|\input{childdoc.def}|\\
|\childdocmain{|\textit{main}|}|\\
\end{tabular}
\end{center}
%
If |\jobname| does not match the argument \textit{main} of |\childdocmain|,
it is assumed that |\jobname| points to the child file to be compiled.
When using |\childdocmain| with the main file specified as argument,
it suffices to start a child file
with just |\input{|\textit{main}|}|
without loading of the package and using |\childdocof|.
If instead all processing is done
with the appropriate \textsf{childdoc} directives,
the argument of \textit{main} of |\childdocmain| can be empty.

An alternative version of the command line processing described
in \secref{sec:commandline} using the detection mechanism reads:
%
\begin{center}
|... -jobname "|\textit{target}|" "|[\textit{flags}]%
[|\def\jobname{|\textit{dest}|}|]|\input{|\textit{main}|}"|
\end{center}

%%%%%%%%%%%%%%%%%%%%%%%%%%%%%%%%%%%%%%%%%%%%%%%%%%%%%%%%%%%%%%%%%%%%%%%%%%%%%%%%
\subsection{Manual Code}
\label{sec:manual}

In case one cannot be certain whether the definitions file |childdoc.def|
is installed on the target \TeX{} distribution
and one prefers not to ship it,
it is conceivable to paste a few relevant commands into the sources.

To that end, drop all statements |\input{childdoc.def}|
and perform the replacements as outlined below.
Instead of |\childdocmain{|\textit{main}|}| add the following code
to the top of the main file:
%
\begin{center}
\begin{tabular}{l}
|\||ifdefined\childdocname\endinput\||fi\newif\ifchilddoc|\\
|\edef\childdocname{\scantokens\expandafter{\jobname\noexpand}}|\\
|\def\childdocmain{|\textit{main}|}\||ifx\childdocmain\childdocname\||else|\\
|\childdoctrue\includeonly{\childdocname}\let\jobname\childdocmain\||fi|\\
\end{tabular}
\end{center}
%
Instead of |\childdocof{|\textit{main}|}| just include the main file
at the top of each child file:
%
\begin{center}
|\input{|\textit{main}|}|
\end{center}
%
A simple redirection |\childdocforward{|\textit{dest}|}| is achieved by:
%
\begin{center}
|\def\jobname{|\textit{dest}|}\input{\jobname}|
\end{center}
%
The redirection with prefix
|\childdocforwardprefix[|\textit{prefix}|]{|\textit{dest}|}|
is accomplished by:
%
\begin{center}
\begin{tabular}{l}
|{\edef\jobname{\scantokens\expandafter{\jobname\noexpand}}|\\
|\def\redirectjob |\textit{prefix}|#1~~~{\gdef\jobname{|\textit{dest}|#1}}|\\
|\expandafter\redirectjob\jobname~~~}\input{\jobname}|
\end{tabular}
\end{center}

In an alternative approach,
child documents can be compiled by a specific command line
without additional code or specific definitions:
%
\begin{center}
|... -jobname "|\textit{target}|" "|[\textit{flags}]%
|\includeonly{|\textit{dest}|}\input{|\textit{main}|}"|
\end{center}
%

%%%%%%%%%%%%%%%%%%%%%%%%%%%%%%%%%%%%%%%%%%%%%%%%%%%%%%%%%%%%%%%%%%%%%%%%%%%%%%%%
%%%%%%%%%%%%%%%%%%%%%%%%%%%%%%%%%%%%%%%%%%%%%%%%%%%%%%%%%%%%%%%%%%%%%%%%%%%%%%%%
\section{Information}

%%%%%%%%%%%%%%%%%%%%%%%%%%%%%%%%%%%%%%%%%%%%%%%%%%%%%%%%%%%%%%%%%%%%%%%%%%%%%%%%
\subsection{Copyright}

Copyright \copyright{} 2017--2018 Niklas Beisert

This work may be distributed and/or modified under the
conditions of the \LaTeX{} Project Public License, either version 1.3
of this license or (at your option) any later version.
The latest version of this license is in
  \url{http://www.latex-project.org/lppl.txt}
and version 1.3 or later is part of all distributions of \LaTeX{}
version 2005/12/01 or later.

This work has the LPPL maintenance status `maintained'.

The Current Maintainer of this work is Niklas Beisert.

This work consists of the files |README.txt|, |childdoc.ins| and |childdoc.dtx|
as well as the derived files |childdoc.def|, |cdocsamp.tex|
with |cdocsch1.tex|, |cdocsch2.tex|, |cdocspt3.tex|, |cdocspt4.tex|,
|cdocsdrf.tex|, |cdocsfn1.tex|, |cdocsfn2.tex|
as well as |childdoc.pdf|.

%%%%%%%%%%%%%%%%%%%%%%%%%%%%%%%%%%%%%%%%%%%%%%%%%%%%%%%%%%%%%%%%%%%%%%%%%%%%%%%%
\subsection{Files and Installation}

The package consists of the files:
%
\begin{center}
\begin{tabular}{ll}
    |README.txt|   & readme file \\
    |childdoc.ins| & installation file \\
    |childdoc.dtx| & source file \\
    |childdoc.def| & definition file \\
    |cdocsamp.tex| & sample main file \\
    |cdocsch1.tex| & sample include file \\
    |cdocsch2.tex| & sample include file \\
    |cdocspt3.tex| & sample part file \\
    |cdocspt4.tex| & sample part file \\
    |cdocsdrf.tex| & sample redirection file \\
    |cdocsfn1.tex| & sample redirection file \\
    |cdocsfn2.tex| & sample redirection file \\
    |childdoc.pdf| & manual
\end{tabular}
\end{center}
%
The distribution consists of the files
|README.txt|, |childdoc.ins| and |childdoc.dtx|.
%
\begin{itemize}
\item
Run (pdf)\LaTeX{} on |childdoc.dtx|
to compile the manual |childdoc.pdf| (this file).
\item
Run \LaTeX{} on |childdoc.ins| to create the definitions file |childdoc.def|
and the sample |cdocsamp.tex| with include files
|cdocsch1.tex|, |cdocsch2.tex|, |cdocspt3.tex|, |cdocspt4.tex|,
|cdocsdrf.tex|, |cdocsfn1.tex|, |cdocsfn2.tex|.
Then copy the file |childdoc.def| to an appropriate directory of your \LaTeX{}
distribution, e.g.\ \textit{texmf-root}|/tex/latex/childdoc|.
\end{itemize}

%%%%%%%%%%%%%%%%%%%%%%%%%%%%%%%%%%%%%%%%%%%%%%%%%%%%%%%%%%%%%%%%%%%%%%%%%%%%%%%%
\subsection{Related CTAN Packages}

There are several other packages which offer a similar functionality:
%
\begin{itemize}
\item
The packages
\href{http://ctan.org/pkg/docmute}{\textsf{docmute}},
\href{http://ctan.org/pkg/includex}{\textsf{includex}} and
\href{http://ctan.org/pkg/standalone}{\textsf{standalone}}
provide commands to include only the document body of
a child file thus allowing both files to be compiled individually.
\item
The packages \href{http://ctan.org/pkg/subdocs}{\textsf{subdocs}}
and \href{http://ctan.org/pkg/subfiles}{\textsf{subfiles}}
provide structures in which the main and child documents can be
encapsulated and allowing them to be compiled individually.
The inclusion mechanism is different from the conventional |\include|.
\item
The package \href{http://ctan.org/pkg/combine}{\textsf{combine}}
is an elaborate solution to combine several documents into one.
\end{itemize}
%
See also the CTAN topic \href{http://ctan.org/topic/subdocs}{\textsf{subdocs}}
for further related packages.
The present package differs from the above solutions in that
a document structure constructed with the conventional |\include| mechanism
just needs two extra commands at the top of every file
such that all constituent files can be compiled individually.

%%%%%%%%%%%%%%%%%%%%%%%%%%%%%%%%%%%%%%%%%%%%%%%%%%%%%%%%%%%%%%%%%%%%%%%%%%%%%%%%
%\subsection{Feature Suggestions}
%
%The following is a list of features which may be useful for future
%versions of this package:
%%
%\begin{itemize}
%\item
%\ldots
%\end{itemize}

%%%%%%%%%%%%%%%%%%%%%%%%%%%%%%%%%%%%%%%%%%%%%%%%%%%%%%%%%%%%%%%%%%%%%%%%%%%%%%%%
\subsection{Revision History}

%%%%%%%%%%%%%%%%%%%%%%%%%%%%%%%%%%%%%%%%
\paragraph{v2.0:} 2018/12/30

\begin{itemize}
\item
immediate forward processing
\item
added |\childdocby| mechanism
\item
manual restructured
\end{itemize}

%%%%%%%%%%%%%%%%%%%%%%%%%%%%%%%%%%%%%%%%
\paragraph{v1.6:} 2018/01/17

\begin{itemize}
\item
application for development of include files
\item
corrections to manual
\end{itemize}

%%%%%%%%%%%%%%%%%%%%%%%%%%%%%%%%%%%%%%%%
\paragraph{v1.5:} 2017/05/21

\begin{itemize}
\item
more complete structuring introduced
\item
|\childdocof| introduced
\item
|\childdoc| renamed to |\childdocmain|
\item
|\childredirect| renamed to |\childdocforward| and |\childdocforwardprefix|
and functionality expanded
\end{itemize}

%%%%%%%%%%%%%%%%%%%%%%%%%%%%%%%%%%%%%%%%
\paragraph{v1.0:} 2017/04/27

\begin{itemize}
\item
manual and install package
\item
first version published on CTAN
\end{itemize}

%%%%%%%%%%%%%%%%%%%%%%%%%%%%%%%%%%%%%%%%
\paragraph{v0.6:} 2017/04/26

\begin{itemize}
\item
redirection mechanism added
\end{itemize}

%%%%%%%%%%%%%%%%%%%%%%%%%%%%%%%%%%%%%%%%
\paragraph{v0.5:} 2017/04/26

\begin{itemize}
\item
functionality in definition file
\end{itemize}


%%%%%%%%%%%%%%%%%%%%%%%%%%%%%%%%%%%%%%%%%%%%%%%%%%%%%%%%%%%%%%%%%%%%%%%%%%%%%%%%
%%%%%%%%%%%%%%%%%%%%%%%%%%%%%%%%%%%%%%%%%%%%%%%%%%%%%%%%%%%%%%%%%%%%%%%%%%%%%%%%
%%%%%%%%%%%%%%%%%%%%%%%%%%%%%%%%%%%%%%%%%%%%%%%%%%%%%%%%%%%%%%%%%%%%%%%%%%%%%%%%
\appendix

\settowidth\MacroIndent{\rmfamily\scriptsize 000\ }

 \DocInput{childdoc.dtx}

\end{document}
%</driver>
% \fi
%
% %%%%%%%%%%%%%%%%%%%%%%%%%%%%%%%%%%%%%%%%%%%%%%%%%%%%%%%%%%%%%%%%%%%%%%%%%%%%%%
% %%%%%%%%%%%%%%%%%%%%%%%%%%%%%%%%%%%%%%%%%%%%%%%%%%%%%%%%%%%%%%%%%%%%%%%%%%%%%%
% \section{Sample}
%\iffalse
%<*samplemain>
%\fi
%
% The following presents a sample document
% with two chapters, two parts, a title page,
% a compile flag as well as three forwarding files to set the flag.
% It consists of eight |.tex| files:
% \begin{center}
% \begin{tabular}{ll}
% |cdocsamp.tex|&main file\\
% |cdocsch1.tex|&include file for chapter 1\\
% |cdocsch2.tex|&include file for chapter 2\\
% |cdocspt3.tex|&include file for part 3\\
% |cdocspt4.tex|&include file for part 4\\
% |cdocsdrf.tex|&forwarding file for main file in draft mode\\
% |cdocsfi1.tex|&forwarding file for final version of chapter 1\\
% |cdocsfi2.tex|&forwarding file for final version of chapter 2\\
% \end{tabular}
% \end{center}
% Each of the eight files can be compiled directly by the \LaTeX{} compiler.
%
% %%%%%%%%%%%%%%%%%%%%%%%%%%%%%%%%%%%%%%
% \paragraph{Main File.}
%
% The main file is called |cdocsamp.tex|.
%
% Load the \textsf{childdoc} definitions and
% declare the filename for the main document:
%    \begin{macrocode}
\input{childdoc.def}
\childdocmain{}
%    \end{macrocode}

% Optional override for |\version| flag:
%    \begin{macrocode}
%%\ifchilddoc\else\providecommand{\version}{draft}\fi
%    \end{macrocode}

% Define the default values for the |\version| flag
% (|final| for the main file and |draft| for childs):
%    \begin{macrocode}
\ifchilddoc
\providecommand{\version}{draft}
\else
\providecommand{\version}{final}
\fi
%    \end{macrocode}

% Load the standard document class:
%    \begin{macrocode}
\documentclass[12pt]{article}
%    \end{macrocode}

% Start the document body:
%    \begin{macrocode}
\begin{document}
%    \end{macrocode}

% Declare a title page.
% Print title, part of document being processed and version flag:
%    \begin{macrocode}
\addtocounter{page}{-1}
\begin{center}
{\LARGE\bfseries{}childdoc example\par}
\vspace{1cm}
\ifchilddoc
\ifchilddocmanual part\else chapter\fi:
`\childdocname' of `\childdocjob'\par
\else
main document: `\childdocjob'\par
\fi
version: \version\par
\end{center}
\newpage
%    \end{macrocode}

% Manually include selected file,
% otherwise process as usual:
%    \begin{macrocode}
\ifchilddocmanual
\section*{part `\childdocname'}
\input{\childdocname}
\else
%    \end{macrocode}

% Include the two chapters:
%    \begin{macrocode}
\include{cdocsch1}
\include{cdocsch2}
%    \end{macrocode}

% Include the two parts unless only chapters should be displayed:
%    \begin{macrocode}
\ifchilddoc\else
\section{part three}
\input{cdocspt3}
\section{part four}
\input{cdocspt4}
\fi
%    \end{macrocode}

% Process as usual until here:
%    \begin{macrocode}
\fi
%    \end{macrocode}

% End of document body:
%    \begin{macrocode}
\end{document}
%    \end{macrocode}
%\iffalse
%</samplemain>
%\fi
%
% %%%%%%%%%%%%%%%%%%%%%%%%%%%%%%%%%%%%%%
% \paragraph{Chapter Include Files.}
%
% The include files are called |cdocsch1.tex| and |cdocsch2.tex|.
%
%\iffalse
%<*samplechap1|samplechap2>
%\fi

% Optional override for |\version| flag:
%    \begin{macrocode}
%%\providecommand{\version}{final}
%    \end{macrocode}

% Include the main document:
%    \begin{macrocode}
\input{childdoc.def}
\childdocof{cdocsamp}
%    \end{macrocode}

%\iffalse
%</samplechap1|samplechap2>
%\fi
%
%\iffalse
%<*samplechap1>
%\fi
% Some text for chapter 1:
%    \begin{macrocode}
\section{one}
some text in chapter one
%    \end{macrocode}

%\iffalse
%</samplechap1>
%\fi
% Some text for chapter 2:
%\iffalse
%<*samplechap2>
%\fi
%    \begin{macrocode}
\section{two}
more text in chapter two
%    \end{macrocode}

%\iffalse
%</samplechap2>
%\fi
%
% %%%%%%%%%%%%%%%%%%%%%%%%%%%%%%%%%%%%%%
% \paragraph{Part Include Files.}
%
% The include files are called |cdocspt3.tex| and |cdocspt4.tex|.
%
%\iffalse
%<*samplepart3|samplepart4>
%\fi

% Optional override for |\version| flag:
%    \begin{macrocode}
%%\providecommand{\version}{final}
%    \end{macrocode}

% Include the main document:
%    \begin{macrocode}
\input{childdoc.def}
\childdocby{cdocsamp}
%    \end{macrocode}

%\iffalse
%</samplepart3|samplepart4>
%\fi
%
%\iffalse
%<*samplepart3>
%\fi
% Some text for part 3:
%    \begin{macrocode}
some text in part three
%    \end{macrocode}

%\iffalse
%</samplepart3>
%\fi
% Some text for part 4:
%\iffalse
%<*samplepart4>
%\fi
%    \begin{macrocode}
more text in part four
%    \end{macrocode}

%\iffalse
%</samplepart4>
%\fi
%
% %%%%%%%%%%%%%%%%%%%%%%%%%%%%%%%%%%%%%%
% \paragraph{Forwarding for a Complete Draft.}
%
% The following forwarding file |cdocsdrf.tex|
% compiles the main document in draft mode:
%\iffalse
%<*sampledraft>
%\fi
%    \begin{macrocode}
\def\version{draft}
\input{childdoc.def}
\childdocforward{cdocsamp}
%    \end{macrocode}

%\iffalse
%</sampledraft>
%\fi
%
% %%%%%%%%%%%%%%%%%%%%%%%%%%%%%%%%%%%%%%
% \paragraph{Forwarding for Final Version of the Chapters.}
%
% The following forwarding files |cdocsfn1.tex| and |cdocsfn2.tex|
% (with identical content)
% compile the final versions of the child documents
% |cdocsch1.tex| and |cdocsch2.tex|, respectively:
%\iffalse
%<*samplefinal>
%\fi
%    \begin{macrocode}
\def\version{final}
\input{childdoc.def}
\childdocforwardprefix[cdocsamp]{cdocsfn}{cdocsch}
%    \end{macrocode}

%\iffalse
%</samplefinal>
%\fi
%
% %%%%%%%%%%%%%%%%%%%%%%%%%%%%%%%%%%%%%%
% \paragraph{Command Line Processing.}
%
% The following three command lines generate the output files
% |cdocscld|, |cdocscl1| and |cdocscl2|
% which should be identical to
% |cdocsdrf|, |cdocsch1| and |cdocsfn2|, respectively:
% \begin{center}
% \begin{tabular}{l}
% |latex -jobname cdocscld \|\\
% |  "\def\version{draft}\input{childdoc.def}\childdocforward{cdocsamp}"|\\
% |latex -jobname cdocscl1 \|\\
% |  "\input{childdoc.def}\childdocforward[cdocsamp]{cdocsch1}"|\\
% |latex -jobname cdocscl2 \|\\
% |  "\def\version{final}\input{childdoc.def}\childdocforward{cdocsch2}"|
% \end{tabular}
% \end{center}
% Note that the trailing backslash on each first line
% merely continues the input to the second line
% (for convenient cut ant paste).
% Furthermore, the command |latex| can be replaced by any
% of its alternative versions such as |pdflatex|.
%
% %%%%%%%%%%%%%%%%%%%%%%%%%%%%%%%%%%%%%%%%%%%%%%%%%%%%%%%%%%%%%%%%%%%%%%%%%%%%%%
% %%%%%%%%%%%%%%%%%%%%%%%%%%%%%%%%%%%%%%%%%%%%%%%%%%%%%%%%%%%%%%%%%%%%%%%%%%%%%%
% \section{Implementation}
%\iffalse
%<*package>
%\fi
%
% This section describes the definitions file |childdoc.def|.

% The definitions cannot be loaded using |\usepackage| or |\RequirePackage|
% which has a mechanism to prevent loading a style file more than once.
% When loading the definitions by means of |\input|
% multiple instances have to be prevented manually:
%\iffalse
%This code needs to be before the `\ProvidesFile' directive
%which is defined at the beginning of this file.
%Therefore it is also placed there and commented out here.
%</package>
%<*discard>
%\fi
%    \begin{macrocode}
\ifdefined\childdocmain\endinput\fi
%    \end{macrocode}
%\iffalse
%</discard>
%<*package>
%\fi
%
% \macro{\ifchilddoc}
% \macro{\ifchilddocmanual}
% The conditional |\ifchilddoc| tells whether a
% child (true) or main (false) document is being compiled.
% The conditional |\ifchilddocmanual| tells whether
% the |\includeonly| mechanism is used (false) or
% the selection of child files must be performed manually (true).
% The definitions initialise to false:
%    \begin{macrocode}
\newif\ifchilddoc
\newif\ifchilddocmanual
%    \end{macrocode}

% \macro{\childdocname}
% \macro{\childdocjob}
% The macro |\childdocname| stores the name of the main document
% to be compiled. The macro |\childdocjob| stores the name of
% the document on which the \LaTeX{} compiler was originally invoked.
% The content of |\jobname| cannot be compared
% to filenames specified in the source due to different catcodes.
% The following code rescans |\jobname|, stores the result
% in |\childdocname| and saves a copy in |\childdocjob|:
%    \begin{macrocode}
\edef\childdocname{\scantokens\expandafter{\jobname\noexpand}}
\let\childdocjob\childdocname
%    \end{macrocode}

% \macro{\childdocdisable}
% The macro |\childdocdisable| prevents the main file
% from being processed more than once.
% At this stage, the main document command |\childdocmain|
% is assumed to be called once again where it should do nothing.
% Any subsequent call to it should prevent
% a secondary processing of the main document
% It overwrites the forwarding commands
% |\childdocof| and |\childdocforward|
% with empty macros to prevent further inclusions of the main document:
%    \begin{macrocode}
\newcommand{\childdocdisable}
{
  \renewcommand{\childdocmain}[1]{\renewcommand{\childdocmain}[1]{\endinput}}
  \renewcommand{\childdocof}[1]{}
  \renewcommand{\childdocby}[2][]{}
  \renewcommand{\childdocforward}[2][]{}
  \renewcommand{\childdocdisable}{}
}
%    \end{macrocode}

% \macro{\childdocmain}
% The macro |\childdocmain| is to be called at the top of the main file
% with nothing or the main filename (without extension) as argument.
% First, it breaks loops.
% If the argument is not empty and does not match |\childdocname|
% (which is set by the first inclusion of |childdoc.def|),
% |\ifchilddoc| is set to true, |\includeonly| is applied to the child file
% and |\jobname| is set to the main file
% (for proper handling of |.aux| files):
%    \begin{macrocode}
\newcommand{\childdocmain}[1]
{
  \childdocdisable\childdocmain{}
  \if?#1?\else
    \begingroup
      \def\childdoctmp{#1}
      \ifx\childdoctmp\childdocname
        \def\childdoctmp{}
      \else
        \def\childdoctmp
        {
          \childdoctrue
          \includeonly{\childdocname}
          \def\childdocjob{#1}
          \def\jobname{#1}
        }
      \fi
      \expandafter
    \endgroup
    \childdoctmp
  \fi
}
%    \end{macrocode}

% \macro{\childdocof}
% The command |\childdocof| redirects
% compilation to the main file |#1|.
%    \begin{macrocode}
\newcommand{\childdocof}[1]
{
  \childdocdisable
  \childdoctrue
  \includeonly{\childdocname}
  \def\jobname{#1}
  \def\childdocjob{#1}
  \input{#1}
}
%    \end{macrocode}

% \macro{\childdocby}
% The command |\childdocby| ....
%    \begin{macrocode}
\newcommand{\childdocby}[2][]
{
  \childdocdisable
  \childdoctrue
  \childdocmanualtrue
  \if?#1?\else
    \def\jobname{#2}
  \fi
  \def\childdocjob{#2}
  \input{#2}
  \endinput
}
%    \end{macrocode}

% \macro{\childdocforward}
% The command |\childdocforward| redirects
% compilation to the main file or
% (if the optional argument is given) a child file.
% Parameters are set as if the main file
% or a child file starting with |\childdocof| was compiled.
% Then compilation is handed over to the main file:
%    \begin{macrocode}
\newcommand{\childdocforward}[2][]
{
  \begingroup
    \if?#1?
      \def\childdoctmp
      {
        \def\childdocname{#2}
        \def\childdocjob{#2}
        \def\jobname{#2}
        \input{#2}
        \endinput
      }
    \else
      \def\childdoctmp
      {
        \childdocdisable
        \def\childdocname{#2}
        \childdoctrue
        \includeonly{#2}
        \def\childdocjob{#1}
        \def\jobname{#1}
        \input{#1}
        \endinput
      }
    \fi
    \expandafter
  \endgroup
  \childdoctmp
}
%    \end{macrocode}

% \macro{\childdocforwardprefix}
% The command |\childdocforwardprefix| redirects
% compilation to the main or a child file by means of a pattern.
% The prefix |#1| in the current filename is replaced by |#2|
% and the suffix of the current filename is kept
% (it is assumed that the filename does not contain the substring `|~~~|'
% which is used as a delimiter).
% Compilation is handed over to the new file by |\childdocforward|:
%    \begin{macrocode}
\newcommand{\childdocforwardprefix}[3][]
{
  \begingroup
    \def\childdocextract #2##1~~~{\def\childdoctmp{\childdocforward[#1]{#3##1}}}
    \expandafter\childdocextract\childdocname~~~
    \expandafter
  \endgroup
  \childdoctmp
}
%    \end{macrocode}

% \macro{\childdoc}
% The deprecated macro |\childdoc| is a legacy version of |\childdocmain|:
%    \begin{macrocode}
\newcommand{\childdoc}{\childdocmain}
%    \end{macrocode}

% \macro{\childdocredirect}
% The deprecated macro |\childdocredirect| is a legacy version
% of |\childdocforward| and |\childdocforwardprefix|:
%    \begin{macrocode}
\newcommand{\childdocredirect}[2][]
{
  \begingroup
    \if?#1?
      \def\childdoctmp{\childdocforward{#2}}
    \else
      \def\childdoctmp{\childdocforwardprefix{#1}{#2}}
    \fi
    \expandafter
  \endgroup
  \childdoctmp
}
%    \end{macrocode}

%\iffalse
%</package>
%\fi
%
\endinput
|\\
|\childdocby{|\textit{main}|}|\\
\end{tabular}
\end{center}
%
Both forms have slightly different effects as described above.
The main file is prepared as usual, see \secref{sec:include}.

%%%%%%%%%%%%%%%%%%%%%%%%%%%%%%%%%%%%%%%%%%%%%%%%%%%%%%%%%%%%%%%%%%%%%%%%%%%%%%%%
\subsection{Legacy Detection}
\label{sec:detection}

The directive |\childdocmain| in the main file can detect
whether the complete document or merely a child is to be compiled
even without using the directive |\childdocof|.
This method is deprecated because it is less robust
and there is no compelling reason to use it;
it is merely provided for backward compatibility
and it may be removed in future versions.

If the detection mechanism is to be used,
it is mandatory to correctly specify
the filename of the main file as the argument of |\childdocmain|:
%
\begin{center}
\begin{tabular}{l}
|% \iffalse
%
% childdoc.dtx Copyright (C) 2017-2018 Niklas Beisert
%
% This work may be distributed and/or modified under the
% conditions of the LaTeX Project Public License, either version 1.3
% of this license or (at your option) any later version.
% The latest version of this license is in
%   http://www.latex-project.org/lppl.txt
% and version 1.3 or later is part of all distributions of LaTeX
% version 2005/12/01 or later.
%
% This work has the LPPL maintenance status `maintained'.
%
% The Current Maintainer of this work is Niklas Beisert.
%
% This work consists of the files childdoc.dtx and childdoc.ins
% and the derived files childdoc.def and cdocsamp.tex with
% cdocsch1.tex, cdocsch2.tex, cdocsdrf.tex, cdocsfn1.tex, cdocsfn2.tex.
%
%<package>\ifdefined\childdocmain\endinput\fi
%<package>\ProvidesFile{childdoc.def}[2018/12/30 v2.0 child document driver]
%<samplemain>\ProvidesFile{cdocsamp.tex}[2018/12/30 v2.0 sample for childdoc]
%<*driver>
%\ProvidesFile{childdoc.drv}[2018/12/30 v2.0 childdoc reference manual file]
\PassOptionsToClass{10pt,a4paper}{article}
\documentclass{ltxdoc}

\usepackage[margin=35mm]{geometry}
\usepackage{hyperref}
\usepackage{hyperxmp}
\usepackage[usenames]{color}

\hypersetup{colorlinks=true}
\hypersetup{pdfstartview=FitH}
\hypersetup{pdfpagemode=UseNone}
\hypersetup{pdfsource={}}
\hypersetup{pdflang={en-UK}}
\hypersetup{pdfcopyright={Copyright 2017-2018 Niklas Beisert.
  This work may be distributed and/or modified under the
  conditions of the LaTeX Project Public License, either version 1.3
  of this license or (at your option) any later version.}}
\hypersetup{pdflicenseurl={http://www.latex-project.org/lppl.txt}}
\hypersetup{pdfcontactaddress={ETH Zurich, ITP, HIT K,
  Wolfgang-Pauli-Strasse 27}}
\hypersetup{pdfcontactpostcode={8093}}
\hypersetup{pdfcontactcity={Zurich}}
\hypersetup{pdfcontactcountry={Switzerland}}
\hypersetup{pdfcontactemail={nbeisert@itp.phys.ethz.ch}}
\hypersetup{pdfcontacturl={http://people.phys.ethz.ch/\xmptilde nbeisert/}}

\newcommand{\secref}[1]{\hyperref[#1]{section \ref*{#1}}}

\parskip1ex
\parindent0pt
\let\olditemize\itemize
\def\itemize{\olditemize\parskip0pt}

\begin{document}

\title{The \textsf{childdoc} Package}
\hypersetup{pdftitle={The childdoc Package}}
\author{Niklas Beisert\\[2ex]
  Institut f\"ur Theoretische Physik\\
  Eidgen\"ossische Technische Hochschule Z\"urich\\
  Wolfgang-Pauli-Strasse 27, 8093 Z\"urich, Switzerland\\[1ex]
  \href{mailto:nbeisert@itp.phys.ethz.ch}
  {\texttt{nbeisert@itp.phys.ethz.ch}}}
\hypersetup{pdfauthor={Niklas Beisert}}
\hypersetup{pdfsubject={Manual for the LaTeX2e Package childdoc}}
\date{30 December 2018, \textsf{v2.0}}
\maketitle

\begin{abstract}\noindent
\textsf{childdoc} is a \LaTeXe{} package
that enables the direct compilation
of document sections included by |\include|
to individual files.
\end{abstract}

\begingroup
\parskip0ex
\tableofcontents
\endgroup

%%%%%%%%%%%%%%%%%%%%%%%%%%%%%%%%%%%%%%%%%%%%%%%%%%%%%%%%%%%%%%%%%%%%%%%%%%%%%%%%
%%%%%%%%%%%%%%%%%%%%%%%%%%%%%%%%%%%%%%%%%%%%%%%%%%%%%%%%%%%%%%%%%%%%%%%%%%%%%%%%
\section{Introduction}

\LaTeX{} provides a mechanism to structure a large document (such as a book)
into a main file and several child files (containing the chapters)
using the |\include| command.
This mechanism is beneficial for documents
which span hundreds of pages in order to
make the source file(s) more manageable.
Moreover, compilation can be restricted to
selected child files by means of the |\includeonly| command.
The latter feature can be used to reduce the compilation time while editing
(this was significantly more useful in the earlier days of \LaTeX{})
or to generate a smaller document which is easier to navigate.
Another application of |\includeonly| is to generate
documents consisting of selected parts of the complete document.

However, there are a few drawbacks of the plain |\include| mechanism:
\begin{itemize}
\item
The child files cannot be compiled on their own,
they can only be compiled via the main file.
A naive editing environment
(such as a text editor with an option
to have the current file processed by \LaTeX)
may require one to switch to the main file before compiling;
attempting to compile the child file produces errors.
\item
The main file must be modified (each time)
to adjust the |\includeonly| command
to the present needs. This easily leaves the main file in a messy state.
\item
The generated document will always carry the filename
of the main document. This is inconvenient if
several child files are to be compiled and
to be kept for distribution.
\end{itemize}

The present package provides a simple interface
to make child files individually compilable by \LaTeX{}.
Compiling a child file then has the same effect as compiling
the main file with an |\includeonly| command
to select the appropriate child.
Moreover the generated document will carry the name of the child
rather than the main file.
This resolves all three above issues.

This feature is meant to make the editing of books,
thesis documents and lecture notes somewhat more convenient.
However, the package can also be used efficiently for
composing a series of documents (such as exercise sheets)
which are typically distributed individually.
It then assists the author in generating the individual documents
(potentially in different versions)
as well as a document containing the collected series.
Another application is in developing style files
or other kinds of included material
where compilation of the style file could redirect
to a sample or test file.

%%%%%%%%%%%%%%%%%%%%%%%%%%%%%%%%%%%%%%%%%%%%%%%%%%%%%%%%%%%%%%%%%%%%%%%%%%%%%%%%
%%%%%%%%%%%%%%%%%%%%%%%%%%%%%%%%%%%%%%%%%%%%%%%%%%%%%%%%%%%%%%%%%%%%%%%%%%%%%%%%
\section{Usage}

First of all, the package \textsf{childdoc} is \emph{not} a standard
\LaTeXe{} |.sty| style file! Therefore it needs to be invoked in
a non-standard way.

%%%%%%%%%%%%%%%%%%%%%%%%%%%%%%%%%%%%%%%%%%%%%%%%%%%%%%%%%%%%%%%%%%%%%%%%%%%%%%%%
\subsection{Included Files}
\label{sec:include}

%%%%%%%%%%%%%%%%%%%%%%%%%%%%%%%%%%%%%%%%
\DescribeMacro{\childdocmain}
To use the package, add the commands
\begin{center}
\begin{tabular}{l}
|\input{childdoc.def}|\\
|\childdocmain{}|\\
\end{tabular}
\end{center}
at the very top of the main \LaTeX{} file,
in particular \emph{before} the |\documentclass| statement!
The argument of |\childdocmain| should be left empty
(but it must be present).

%%%%%%%%%%%%%%%%%%%%%%%%%%%%%%%%%%%%%%%%
\DescribeMacro{\childdocof}
Furthermore, add the commands
\begin{center}
\begin{tabular}{l}
|\input{childdoc.def}|\\
|\childdocof{|\textit{main}|}|\\
\end{tabular}
\end{center}
at the top of every child file \textit{child}
which is included by |\include{|\textit{child}|}|
from within the main file
(or at least for those files to be compiled individually).
The argument \textit{main} must be the filename of the main file.

There are a couple of
considerations in setting up the main and child documents:

%%%%%%%%%%%%%%%%%%%%%%%%%%%%%%%%%%%%%%%%
\paragraph{Restrictions.}

Please note the following restrictions:
\begin{itemize}
\item
|\childdocmain| must be called with one argument \textit{main}
to ensure compatibility with earlier version of the package.
It must either be empty (|\childdocmain{}|)
or precisely match the filename of the main file in which it is specified.
See \secref{sec:detection} for further information.
\item
The filename \textit{main} must be specified without the |.tex| extension.
\item
The filename \textit{main} is case sensitive
(even in case-insensitive file systems)
due to internal string comparison.
\item
The argument \textit{main} should be fully expanded, it cannot be a macro.
\item
Subdirectories and special characters should be avoided in filenames.
\item
The command |\childdocmain{|\textit{main}|}| must be followed by a whitespace.
It should not be followed immediately by another command
or by a comment mark `|%|'.
This is because the \TeX{} parser reads the token immediately following
the argument of |\childdocmain| and puts it
at the beginning of every child section;
however, a white\-space is ignored.
\end{itemize}

%%%%%%%%%%%%%%%%%%%%%%%%%%%%%%%%%%%%%%%%
\paragraph{Content of Main File.}

It is advisable to place all content in the child files included by |\include|.
Any output contained in the main file will appear in all child documents
unless suppressed manually;
it cannot be suppressed automatically by the |\includeonly| directive
and thus should normally be avoided.
A method to include some content in the main file
by means of conditional processing is described in \secref{sec:conditional}.

%%%%%%%%%%%%%%%%%%%%%%%%%%%%%%%%%%%%%%%%
\paragraph{Page Numbering.}

When only a part of the document is compiled,
the appropriate numbering of pages
(as well as other status parameters)
is determined from the |.aux| files.
The latter contain information from previous passes.
However this information needs to propagate through
all intermediate child documents.
Therefore the page numbering in child documents may well
be inconsistent until the complete document is compiled at least once.

A useful (if unconventional) way to always ensure a consistent
page numbering is to restart the numbering in each child document
and denote the pages by `\textit{child}|.|\textit{page}'
where \textit{child} represents the chapter/section number of the child file.
This can be achieved by the command
|\numberwithin{page}{|\textit{child}|}|
of the \textsf{amsmath} package
where \textit{child} can be |chapter| or |section|
depending on the chosen structuring.
Alternatively, one can modify the macro |\thepage| appropriately
and reset the counter |page| at the start of each child file.

%%%%%%%%%%%%%%%%%%%%%%%%%%%%%%%%%%%%%%%%%%%%%%%%%%%%%%%%%%%%%%%%%%%%%%%%%%%%%%%%
\subsection{Conditional Processing}
\label{sec:conditional}

The package provides a mechanism to compile different versions
of a document. To customise the versions further some conditional processing
can come in handy to distinguish which version is being compiled.
The package provides two macros to describe the compilation context:

%%%%%%%%%%%%%%%%%%%%%%%%%%%%%%%%%%%%%%%%
\DescribeMacro{\ifchilddoc}
The conditional |\ifchilddoc| distinguishes between the compilation of
child documents and the main document:
%
\begin{center}
|\ifchilddoc |\textit{child-code}| |[|\||else |\textit{main-code}]| \||fi|
\end{center}

%%%%%%%%%%%%%%%%%%%%%%%%%%%%%%%%%%%%%%%%
\DescribeMacro{\childdocname}
\DescribeMacro{\childdocjob}
The macro |\childdocname| contains the filename (without extension)
of the main or child file being processed.
Note that |\childdocjob| will always contain the name of the main file.

%%%%%%%%%%%%%%%%%%%%%%%%%%%%%%%%%%%%%%%%
\paragraph{Title Page.}

Conditional processing can be used to include a title or banner page
in the main document when proper precautions are taken.
Importantly, the code in the main file should ensure that the page counter
(as well as other status parameters which are stored in the |.aux| files)
takes the same value after the conditional processing.
Otherwise the page numbers may take divergent values
depending on which part is compiled.

For example, a title page could be declared by:
%
\begin{center}
\begin{tabular}{l}
|\ifchilddoc\||else|\\
|\addtocounter{page}{-1}|\\
\textit{code for title page}\\
|\newpage|\\
|\||fi|
\end{tabular}
\end{center}
%
A banner page for the child documents can be generated by:
%
\begin{center}
\begin{tabular}{l}
|\ifchilddoc|\\
|\addtocounter{page}{-1}|\\
\textit{code for banner page}\\
|\newpage|\\
|\||fi|
\end{tabular}
\end{center}
%
Here one could write a message such as:
\begin{center}
|This is the part \childdocname{} of \childdocjob{}.|
\end{center}

%%%%%%%%%%%%%%%%%%%%%%%%%%%%%%%%%%%%%%%%%%%%%%%%%%%%%%%%%%%%%%%%%%%%%%%%%%%%%%%%
\subsection{Flags}
\label{sec:flags}

The package makes it easy to generate different versions
of the main or child documents.
To this end compilation flags can be defined
and assigned different default values.
They will be particularly useful in conjunction
with the forwarding mechanism described in \secref{sec:forward}.

For example, it may be useful to have a flag |\version|
which can be set to |draft| or |final|.
The document source will contain some conditional code
depending on the value of |\version|.
Suppose further, the flag should default to |final| for the main file
and to |draft| for child files
which is a natural assignment for editing the document.
This is achieved by placing the following code
in the preamble of the main document
(below the |\childdocmain| directive):
%
\begin{center}
\begin{tabular}{l}
|\ifchilddoc|\\
|\providecommand{\version}{draft}|\\
|\||else|\\
|\providecommand{\version}{final}|\\
|\||fi|
\end{tabular}
\end{center}
%
The definition by |\providecommand| makes sure
that previous definitions are not overwritten.
Further statements |\providecommand{\version}{...}|
can thus be added before the above code to override it.

For the main file, one might add a line
(between |\childdocmain| and the above block)
%
\begin{center}
|%\ifchilddoc\||else\providecommand{\version}{draft}\||fi|
\end{center}
%
which can be uncommented to produce a draft version.
Likewise one can add a line to the very top of a child file
(above the |\childdocof{|\textit{main}|}| directive)
%
\begin{center}
|%\providecommand{\version}{final}|
\end{center}
%
which can be uncommented to produce the final version of this child document.

%%%%%%%%%%%%%%%%%%%%%%%%%%%%%%%%%%%%%%%%%%%%%%%%%%%%%%%%%%%%%%%%%%%%%%%%%%%%%%%%
\subsection{Forwarding}
\label{sec:forward}

Different versions of the main or child documents
using compilation flags as described in \secref{sec:flags}
can be (permanently) stored in different files
for convenient compilation, viewing and distribution.
To this end, the package defines a command
to pass on compilation to a different file:

%%%%%%%%%%%%%%%%%%%%%%%%%%%%%%%%%%%%%%%%
\DescribeMacro{\childdocforward}
The command |\childdocforward| redirects processing to
another source file:
%
\begin{center}
\begin{tabular}{l}
|\input{childdoc.def}|\\
|\childdocforward[|\textit{main}|]{|\textit{dest}|}|\\
\end{tabular}
\end{center}
%
The argument \textit{dest} is the destination file
(without extension).
It should be the main file or one of the child files.
Note that further \textsf{childdoc} directives
such as |\childdocof| and |\childdocforward|
in the indicated file will be processed in this form.
The optional argument \textit{main}
passes on directly to the main file \textit{main}
while pretending to compile the child \textit{dest}.
This form behaves as if \textit{dest}
issues |\childdocof{|\textit{main}|}| right away,
and no further \textsf{childdoc} directives will be processed.

%%%%%%%%%%%%%%%%%%%%%%%%%%%%%%%%%%%%%%%%
\DescribeMacro{\...prefix}
In the alternative form |\childdocforwardprefix|,
%
\begin{center}
\begin{tabular}{l}
|\input{childdoc.def}|\\
|\childdocforwardprefix[|\textit{main}|]{|\textit{prefix}|}{|\textit{dest}|}|
\end{tabular}
\end{center}
%
the destination file is determined by a pattern
depending on the current file:
To make this work, the current file must be called
`{\textit{prefix}\hspace{0.2em}\textit{suffix}}'
with \textit{prefix} matching precisely the argument.
Processing is then passed on to the file
`{\textit{dest}\hspace{0.2em}\textit{suffix}}'.
Surely, the same effect is achieved by
directly specifying the
argument `{\textit{dest}\hspace{0.2em}\textit{suffix}}'
in the first form.
However, that requires to set up a different file
for each child. With the alternative form of the command
all these files can have exactly the same content
which simplifies setting them up and maintaining them.

For example, the following file |draft.tex|
with a compilation flag |\version| as described in \secref{sec:flags}
compiles the main document as a draft:
%
\begin{center}
\begin{tabular}{l}
|\def\version{draft}|\\
|\input{childdoc.def}|\\
|\childdocforward{|\textit{main}|}|
\end{tabular}
\end{center}
%
Likewise, the following files |final|\textit{nn}|.tex|
compile the final version of the child document
|child|\textit{nn}|.tex|:
%
\begin{center}
\begin{tabular}{l}
|\def\version{final}|\\
|\input{childdoc.def}|\\
|\childdocforwardprefix{final}{child}|
\end{tabular}
\end{center}
%

Note that when several versions of a main file and/or of each child file
are to be generated, it may be convenient to set up a |Makefile| or
shell script to automatise the process.

%%%%%%%%%%%%%%%%%%%%%%%%%%%%%%%%%%%%%%%%%%%%%%%%%%%%%%%%%%%%%%%%%%%%%%%%%%%%%%%%
\subsection{Command Line Processing}
\label{sec:commandline}

The effect of redirection files can also be achieved by invoking
the \LaTeX{} compiler with a more elaborate command line.
Most conveniently this should be done as part
of a shell script or a |Makefile|.

When using \textsf{childdoc} in the main file, the following
command lines effectively perform a redirection
(note that depending on the shell being used,
backslashes may have to be doubled: `|\|' $\to$ `|\\|'):
%
\begin{center}
|... -jobname "|\textit{target}|" |\\|"|[\textit{flags}]%
|\input{childdoc.def}\childdocforward[|\textit{main}|]{|\textit{dest}|}"|
\end{center}
%
Here \textit{target} is the name of the output file,
\textit{main} is the name of the main file
and \textit{dest} is the name of the main or child file to be processed
(all filenames without extensions).
The optional argument \textit{main} can be omitted
if \textit{main} matches \textit{dest}.
Optionally, compilation \textit{flags} can be defined via |\def| commands.
This command line makes the \TeX{} engine believe
it is compiling the file \textit{target}
whose content is specified as the latter parameter.
The provided code then forwards the processing to
\textit{main} or \textit{dest} as described in \secref{sec:forward}.

%%%%%%%%%%%%%%%%%%%%%%%%%%%%%%%%%%%%%%%%%%%%%%%%%%%%%%%%%%%%%%%%%%%%%%%%%%%%%%%%
\subsection{Include by Input}
\label{sec:input}

Including child documents by |\include| has some restrictions by design.
Most notably, the content of a child document always occupies
its own set of pages; pages cannot be shared between child documents.
Usually, this behaviour makes perfect sense
because each child document contain an essential part of the document.
However, in some situations it may be desirable to compose
a document from a collection of parts
without having mandatory page breaks between then.
For this case, the package
provides a mechanism to include parts
by |\input| which can also be processed individually.
However, by construction this mechanism
requires manual handling of the content to be output.

%%%%%%%%%%%%%%%%%%%%%%%%%%%%%%%%%%%%%%%%
\DescribeMacro{\ifchilddocmanual}
The main file should be prepared as usual, see \secref{sec:include}.
However, the document body must make a distinction
between processing of an individual part and of the main document, e.g.:
%
\begin{center}
\begin{tabular}{l}
|\ifchilddocmanual|\\
|\input{\childdocname}|\\
|\||else|\\
\textit{document body with }|\input{|\textit{part}|}|\\
|\||fi|
\end{tabular}
\end{center}
%
The conditional |\ifchilddocmanual| is true whenever
a part to be included by |\input| is being compiled,
and the name of the part is stored in |\childdocname|.

%%%%%%%%%%%%%%%%%%%%%%%%%%%%%%%%%%%%%%%%
\DescribeMacro{\childdocby}
Each part to be included by |\input| should start with:
%
\begin{center}
\begin{tabular}{l}
|\input{childdoc.def}|\\
|\childdocby{|\textit{main}|}|\\
\end{tabular}
\end{center}
%
The directive |\childdocby| is similar to |\childdocof|
described in \secref{sec:include},
but the subsequent selection of content must be done manually.
To that end, both |\ifchilddoc| and |\ifchilddocmanual|
will be true upon processing of a part,
and the name of the part is stored in |\childdocname|.
Note that |\jobname| will be set to the filename of the current part
so that each part receives an individual |.aux| file
that does not interfere with the |.aux| file(s) of the main document.
This behaviour can be altered by the alternative form
|\childdocby[*]{|\textit{main}|}| (with a non-empty optional argument)
which uses the |.aux| file of the main document
by setting |\jobname| to \textit{main}.

%%%%%%%%%%%%%%%%%%%%%%%%%%%%%%%%%%%%%%%%%%%%%%%%%%%%%%%%%%%%%%%%%%%%%%%%%%%%%%%%
\subsection{Driver Development}
\label{sec:driver}

The \textsf{childdoc} mechanism can also be use for the development
of definition files such as \LaTeX{} styles or classes.
This case differs from the above setup with multiple parts
included by |\include| in that no |\includeonly| should be invoked.
This can be achieved by starting the include file
(before |\ProvidesPackage|) with:
%
\begin{center}
\begin{tabular}{l}
|\input{childdoc.def}|\\
|\childdocforward{|\textit{main}|}|\\
\end{tabular}
\end{center}
%
or alternatively with:
%
\begin{center}
\begin{tabular}{l}
|\input{childdoc.def}|\\
|\childdocby{|\textit{main}|}|\\
\end{tabular}
\end{center}
%
Both forms have slightly different effects as described above.
The main file is prepared as usual, see \secref{sec:include}.

%%%%%%%%%%%%%%%%%%%%%%%%%%%%%%%%%%%%%%%%%%%%%%%%%%%%%%%%%%%%%%%%%%%%%%%%%%%%%%%%
\subsection{Legacy Detection}
\label{sec:detection}

The directive |\childdocmain| in the main file can detect
whether the complete document or merely a child is to be compiled
even without using the directive |\childdocof|.
This method is deprecated because it is less robust
and there is no compelling reason to use it;
it is merely provided for backward compatibility
and it may be removed in future versions.

If the detection mechanism is to be used,
it is mandatory to correctly specify
the filename of the main file as the argument of |\childdocmain|:
%
\begin{center}
\begin{tabular}{l}
|\input{childdoc.def}|\\
|\childdocmain{|\textit{main}|}|\\
\end{tabular}
\end{center}
%
If |\jobname| does not match the argument \textit{main} of |\childdocmain|,
it is assumed that |\jobname| points to the child file to be compiled.
When using |\childdocmain| with the main file specified as argument,
it suffices to start a child file
with just |\input{|\textit{main}|}|
without loading of the package and using |\childdocof|.
If instead all processing is done
with the appropriate \textsf{childdoc} directives,
the argument of \textit{main} of |\childdocmain| can be empty.

An alternative version of the command line processing described
in \secref{sec:commandline} using the detection mechanism reads:
%
\begin{center}
|... -jobname "|\textit{target}|" "|[\textit{flags}]%
[|\def\jobname{|\textit{dest}|}|]|\input{|\textit{main}|}"|
\end{center}

%%%%%%%%%%%%%%%%%%%%%%%%%%%%%%%%%%%%%%%%%%%%%%%%%%%%%%%%%%%%%%%%%%%%%%%%%%%%%%%%
\subsection{Manual Code}
\label{sec:manual}

In case one cannot be certain whether the definitions file |childdoc.def|
is installed on the target \TeX{} distribution
and one prefers not to ship it,
it is conceivable to paste a few relevant commands into the sources.

To that end, drop all statements |\input{childdoc.def}|
and perform the replacements as outlined below.
Instead of |\childdocmain{|\textit{main}|}| add the following code
to the top of the main file:
%
\begin{center}
\begin{tabular}{l}
|\||ifdefined\childdocname\endinput\||fi\newif\ifchilddoc|\\
|\edef\childdocname{\scantokens\expandafter{\jobname\noexpand}}|\\
|\def\childdocmain{|\textit{main}|}\||ifx\childdocmain\childdocname\||else|\\
|\childdoctrue\includeonly{\childdocname}\let\jobname\childdocmain\||fi|\\
\end{tabular}
\end{center}
%
Instead of |\childdocof{|\textit{main}|}| just include the main file
at the top of each child file:
%
\begin{center}
|\input{|\textit{main}|}|
\end{center}
%
A simple redirection |\childdocforward{|\textit{dest}|}| is achieved by:
%
\begin{center}
|\def\jobname{|\textit{dest}|}\input{\jobname}|
\end{center}
%
The redirection with prefix
|\childdocforwardprefix[|\textit{prefix}|]{|\textit{dest}|}|
is accomplished by:
%
\begin{center}
\begin{tabular}{l}
|{\edef\jobname{\scantokens\expandafter{\jobname\noexpand}}|\\
|\def\redirectjob |\textit{prefix}|#1~~~{\gdef\jobname{|\textit{dest}|#1}}|\\
|\expandafter\redirectjob\jobname~~~}\input{\jobname}|
\end{tabular}
\end{center}

In an alternative approach,
child documents can be compiled by a specific command line
without additional code or specific definitions:
%
\begin{center}
|... -jobname "|\textit{target}|" "|[\textit{flags}]%
|\includeonly{|\textit{dest}|}\input{|\textit{main}|}"|
\end{center}
%

%%%%%%%%%%%%%%%%%%%%%%%%%%%%%%%%%%%%%%%%%%%%%%%%%%%%%%%%%%%%%%%%%%%%%%%%%%%%%%%%
%%%%%%%%%%%%%%%%%%%%%%%%%%%%%%%%%%%%%%%%%%%%%%%%%%%%%%%%%%%%%%%%%%%%%%%%%%%%%%%%
\section{Information}

%%%%%%%%%%%%%%%%%%%%%%%%%%%%%%%%%%%%%%%%%%%%%%%%%%%%%%%%%%%%%%%%%%%%%%%%%%%%%%%%
\subsection{Copyright}

Copyright \copyright{} 2017--2018 Niklas Beisert

This work may be distributed and/or modified under the
conditions of the \LaTeX{} Project Public License, either version 1.3
of this license or (at your option) any later version.
The latest version of this license is in
  \url{http://www.latex-project.org/lppl.txt}
and version 1.3 or later is part of all distributions of \LaTeX{}
version 2005/12/01 or later.

This work has the LPPL maintenance status `maintained'.

The Current Maintainer of this work is Niklas Beisert.

This work consists of the files |README.txt|, |childdoc.ins| and |childdoc.dtx|
as well as the derived files |childdoc.def|, |cdocsamp.tex|
with |cdocsch1.tex|, |cdocsch2.tex|, |cdocspt3.tex|, |cdocspt4.tex|,
|cdocsdrf.tex|, |cdocsfn1.tex|, |cdocsfn2.tex|
as well as |childdoc.pdf|.

%%%%%%%%%%%%%%%%%%%%%%%%%%%%%%%%%%%%%%%%%%%%%%%%%%%%%%%%%%%%%%%%%%%%%%%%%%%%%%%%
\subsection{Files and Installation}

The package consists of the files:
%
\begin{center}
\begin{tabular}{ll}
    |README.txt|   & readme file \\
    |childdoc.ins| & installation file \\
    |childdoc.dtx| & source file \\
    |childdoc.def| & definition file \\
    |cdocsamp.tex| & sample main file \\
    |cdocsch1.tex| & sample include file \\
    |cdocsch2.tex| & sample include file \\
    |cdocspt3.tex| & sample part file \\
    |cdocspt4.tex| & sample part file \\
    |cdocsdrf.tex| & sample redirection file \\
    |cdocsfn1.tex| & sample redirection file \\
    |cdocsfn2.tex| & sample redirection file \\
    |childdoc.pdf| & manual
\end{tabular}
\end{center}
%
The distribution consists of the files
|README.txt|, |childdoc.ins| and |childdoc.dtx|.
%
\begin{itemize}
\item
Run (pdf)\LaTeX{} on |childdoc.dtx|
to compile the manual |childdoc.pdf| (this file).
\item
Run \LaTeX{} on |childdoc.ins| to create the definitions file |childdoc.def|
and the sample |cdocsamp.tex| with include files
|cdocsch1.tex|, |cdocsch2.tex|, |cdocspt3.tex|, |cdocspt4.tex|,
|cdocsdrf.tex|, |cdocsfn1.tex|, |cdocsfn2.tex|.
Then copy the file |childdoc.def| to an appropriate directory of your \LaTeX{}
distribution, e.g.\ \textit{texmf-root}|/tex/latex/childdoc|.
\end{itemize}

%%%%%%%%%%%%%%%%%%%%%%%%%%%%%%%%%%%%%%%%%%%%%%%%%%%%%%%%%%%%%%%%%%%%%%%%%%%%%%%%
\subsection{Related CTAN Packages}

There are several other packages which offer a similar functionality:
%
\begin{itemize}
\item
The packages
\href{http://ctan.org/pkg/docmute}{\textsf{docmute}},
\href{http://ctan.org/pkg/includex}{\textsf{includex}} and
\href{http://ctan.org/pkg/standalone}{\textsf{standalone}}
provide commands to include only the document body of
a child file thus allowing both files to be compiled individually.
\item
The packages \href{http://ctan.org/pkg/subdocs}{\textsf{subdocs}}
and \href{http://ctan.org/pkg/subfiles}{\textsf{subfiles}}
provide structures in which the main and child documents can be
encapsulated and allowing them to be compiled individually.
The inclusion mechanism is different from the conventional |\include|.
\item
The package \href{http://ctan.org/pkg/combine}{\textsf{combine}}
is an elaborate solution to combine several documents into one.
\end{itemize}
%
See also the CTAN topic \href{http://ctan.org/topic/subdocs}{\textsf{subdocs}}
for further related packages.
The present package differs from the above solutions in that
a document structure constructed with the conventional |\include| mechanism
just needs two extra commands at the top of every file
such that all constituent files can be compiled individually.

%%%%%%%%%%%%%%%%%%%%%%%%%%%%%%%%%%%%%%%%%%%%%%%%%%%%%%%%%%%%%%%%%%%%%%%%%%%%%%%%
%\subsection{Feature Suggestions}
%
%The following is a list of features which may be useful for future
%versions of this package:
%%
%\begin{itemize}
%\item
%\ldots
%\end{itemize}

%%%%%%%%%%%%%%%%%%%%%%%%%%%%%%%%%%%%%%%%%%%%%%%%%%%%%%%%%%%%%%%%%%%%%%%%%%%%%%%%
\subsection{Revision History}

%%%%%%%%%%%%%%%%%%%%%%%%%%%%%%%%%%%%%%%%
\paragraph{v2.0:} 2018/12/30

\begin{itemize}
\item
immediate forward processing
\item
added |\childdocby| mechanism
\item
manual restructured
\end{itemize}

%%%%%%%%%%%%%%%%%%%%%%%%%%%%%%%%%%%%%%%%
\paragraph{v1.6:} 2018/01/17

\begin{itemize}
\item
application for development of include files
\item
corrections to manual
\end{itemize}

%%%%%%%%%%%%%%%%%%%%%%%%%%%%%%%%%%%%%%%%
\paragraph{v1.5:} 2017/05/21

\begin{itemize}
\item
more complete structuring introduced
\item
|\childdocof| introduced
\item
|\childdoc| renamed to |\childdocmain|
\item
|\childredirect| renamed to |\childdocforward| and |\childdocforwardprefix|
and functionality expanded
\end{itemize}

%%%%%%%%%%%%%%%%%%%%%%%%%%%%%%%%%%%%%%%%
\paragraph{v1.0:} 2017/04/27

\begin{itemize}
\item
manual and install package
\item
first version published on CTAN
\end{itemize}

%%%%%%%%%%%%%%%%%%%%%%%%%%%%%%%%%%%%%%%%
\paragraph{v0.6:} 2017/04/26

\begin{itemize}
\item
redirection mechanism added
\end{itemize}

%%%%%%%%%%%%%%%%%%%%%%%%%%%%%%%%%%%%%%%%
\paragraph{v0.5:} 2017/04/26

\begin{itemize}
\item
functionality in definition file
\end{itemize}


%%%%%%%%%%%%%%%%%%%%%%%%%%%%%%%%%%%%%%%%%%%%%%%%%%%%%%%%%%%%%%%%%%%%%%%%%%%%%%%%
%%%%%%%%%%%%%%%%%%%%%%%%%%%%%%%%%%%%%%%%%%%%%%%%%%%%%%%%%%%%%%%%%%%%%%%%%%%%%%%%
%%%%%%%%%%%%%%%%%%%%%%%%%%%%%%%%%%%%%%%%%%%%%%%%%%%%%%%%%%%%%%%%%%%%%%%%%%%%%%%%
\appendix

\settowidth\MacroIndent{\rmfamily\scriptsize 000\ }

 \DocInput{childdoc.dtx}

\end{document}
%</driver>
% \fi
%
% %%%%%%%%%%%%%%%%%%%%%%%%%%%%%%%%%%%%%%%%%%%%%%%%%%%%%%%%%%%%%%%%%%%%%%%%%%%%%%
% %%%%%%%%%%%%%%%%%%%%%%%%%%%%%%%%%%%%%%%%%%%%%%%%%%%%%%%%%%%%%%%%%%%%%%%%%%%%%%
% \section{Sample}
%\iffalse
%<*samplemain>
%\fi
%
% The following presents a sample document
% with two chapters, two parts, a title page,
% a compile flag as well as three forwarding files to set the flag.
% It consists of eight |.tex| files:
% \begin{center}
% \begin{tabular}{ll}
% |cdocsamp.tex|&main file\\
% |cdocsch1.tex|&include file for chapter 1\\
% |cdocsch2.tex|&include file for chapter 2\\
% |cdocspt3.tex|&include file for part 3\\
% |cdocspt4.tex|&include file for part 4\\
% |cdocsdrf.tex|&forwarding file for main file in draft mode\\
% |cdocsfi1.tex|&forwarding file for final version of chapter 1\\
% |cdocsfi2.tex|&forwarding file for final version of chapter 2\\
% \end{tabular}
% \end{center}
% Each of the eight files can be compiled directly by the \LaTeX{} compiler.
%
% %%%%%%%%%%%%%%%%%%%%%%%%%%%%%%%%%%%%%%
% \paragraph{Main File.}
%
% The main file is called |cdocsamp.tex|.
%
% Load the \textsf{childdoc} definitions and
% declare the filename for the main document:
%    \begin{macrocode}
\input{childdoc.def}
\childdocmain{}
%    \end{macrocode}

% Optional override for |\version| flag:
%    \begin{macrocode}
%%\ifchilddoc\else\providecommand{\version}{draft}\fi
%    \end{macrocode}

% Define the default values for the |\version| flag
% (|final| for the main file and |draft| for childs):
%    \begin{macrocode}
\ifchilddoc
\providecommand{\version}{draft}
\else
\providecommand{\version}{final}
\fi
%    \end{macrocode}

% Load the standard document class:
%    \begin{macrocode}
\documentclass[12pt]{article}
%    \end{macrocode}

% Start the document body:
%    \begin{macrocode}
\begin{document}
%    \end{macrocode}

% Declare a title page.
% Print title, part of document being processed and version flag:
%    \begin{macrocode}
\addtocounter{page}{-1}
\begin{center}
{\LARGE\bfseries{}childdoc example\par}
\vspace{1cm}
\ifchilddoc
\ifchilddocmanual part\else chapter\fi:
`\childdocname' of `\childdocjob'\par
\else
main document: `\childdocjob'\par
\fi
version: \version\par
\end{center}
\newpage
%    \end{macrocode}

% Manually include selected file,
% otherwise process as usual:
%    \begin{macrocode}
\ifchilddocmanual
\section*{part `\childdocname'}
\input{\childdocname}
\else
%    \end{macrocode}

% Include the two chapters:
%    \begin{macrocode}
\include{cdocsch1}
\include{cdocsch2}
%    \end{macrocode}

% Include the two parts unless only chapters should be displayed:
%    \begin{macrocode}
\ifchilddoc\else
\section{part three}
\input{cdocspt3}
\section{part four}
\input{cdocspt4}
\fi
%    \end{macrocode}

% Process as usual until here:
%    \begin{macrocode}
\fi
%    \end{macrocode}

% End of document body:
%    \begin{macrocode}
\end{document}
%    \end{macrocode}
%\iffalse
%</samplemain>
%\fi
%
% %%%%%%%%%%%%%%%%%%%%%%%%%%%%%%%%%%%%%%
% \paragraph{Chapter Include Files.}
%
% The include files are called |cdocsch1.tex| and |cdocsch2.tex|.
%
%\iffalse
%<*samplechap1|samplechap2>
%\fi

% Optional override for |\version| flag:
%    \begin{macrocode}
%%\providecommand{\version}{final}
%    \end{macrocode}

% Include the main document:
%    \begin{macrocode}
\input{childdoc.def}
\childdocof{cdocsamp}
%    \end{macrocode}

%\iffalse
%</samplechap1|samplechap2>
%\fi
%
%\iffalse
%<*samplechap1>
%\fi
% Some text for chapter 1:
%    \begin{macrocode}
\section{one}
some text in chapter one
%    \end{macrocode}

%\iffalse
%</samplechap1>
%\fi
% Some text for chapter 2:
%\iffalse
%<*samplechap2>
%\fi
%    \begin{macrocode}
\section{two}
more text in chapter two
%    \end{macrocode}

%\iffalse
%</samplechap2>
%\fi
%
% %%%%%%%%%%%%%%%%%%%%%%%%%%%%%%%%%%%%%%
% \paragraph{Part Include Files.}
%
% The include files are called |cdocspt3.tex| and |cdocspt4.tex|.
%
%\iffalse
%<*samplepart3|samplepart4>
%\fi

% Optional override for |\version| flag:
%    \begin{macrocode}
%%\providecommand{\version}{final}
%    \end{macrocode}

% Include the main document:
%    \begin{macrocode}
\input{childdoc.def}
\childdocby{cdocsamp}
%    \end{macrocode}

%\iffalse
%</samplepart3|samplepart4>
%\fi
%
%\iffalse
%<*samplepart3>
%\fi
% Some text for part 3:
%    \begin{macrocode}
some text in part three
%    \end{macrocode}

%\iffalse
%</samplepart3>
%\fi
% Some text for part 4:
%\iffalse
%<*samplepart4>
%\fi
%    \begin{macrocode}
more text in part four
%    \end{macrocode}

%\iffalse
%</samplepart4>
%\fi
%
% %%%%%%%%%%%%%%%%%%%%%%%%%%%%%%%%%%%%%%
% \paragraph{Forwarding for a Complete Draft.}
%
% The following forwarding file |cdocsdrf.tex|
% compiles the main document in draft mode:
%\iffalse
%<*sampledraft>
%\fi
%    \begin{macrocode}
\def\version{draft}
\input{childdoc.def}
\childdocforward{cdocsamp}
%    \end{macrocode}

%\iffalse
%</sampledraft>
%\fi
%
% %%%%%%%%%%%%%%%%%%%%%%%%%%%%%%%%%%%%%%
% \paragraph{Forwarding for Final Version of the Chapters.}
%
% The following forwarding files |cdocsfn1.tex| and |cdocsfn2.tex|
% (with identical content)
% compile the final versions of the child documents
% |cdocsch1.tex| and |cdocsch2.tex|, respectively:
%\iffalse
%<*samplefinal>
%\fi
%    \begin{macrocode}
\def\version{final}
\input{childdoc.def}
\childdocforwardprefix[cdocsamp]{cdocsfn}{cdocsch}
%    \end{macrocode}

%\iffalse
%</samplefinal>
%\fi
%
% %%%%%%%%%%%%%%%%%%%%%%%%%%%%%%%%%%%%%%
% \paragraph{Command Line Processing.}
%
% The following three command lines generate the output files
% |cdocscld|, |cdocscl1| and |cdocscl2|
% which should be identical to
% |cdocsdrf|, |cdocsch1| and |cdocsfn2|, respectively:
% \begin{center}
% \begin{tabular}{l}
% |latex -jobname cdocscld \|\\
% |  "\def\version{draft}\input{childdoc.def}\childdocforward{cdocsamp}"|\\
% |latex -jobname cdocscl1 \|\\
% |  "\input{childdoc.def}\childdocforward[cdocsamp]{cdocsch1}"|\\
% |latex -jobname cdocscl2 \|\\
% |  "\def\version{final}\input{childdoc.def}\childdocforward{cdocsch2}"|
% \end{tabular}
% \end{center}
% Note that the trailing backslash on each first line
% merely continues the input to the second line
% (for convenient cut ant paste).
% Furthermore, the command |latex| can be replaced by any
% of its alternative versions such as |pdflatex|.
%
% %%%%%%%%%%%%%%%%%%%%%%%%%%%%%%%%%%%%%%%%%%%%%%%%%%%%%%%%%%%%%%%%%%%%%%%%%%%%%%
% %%%%%%%%%%%%%%%%%%%%%%%%%%%%%%%%%%%%%%%%%%%%%%%%%%%%%%%%%%%%%%%%%%%%%%%%%%%%%%
% \section{Implementation}
%\iffalse
%<*package>
%\fi
%
% This section describes the definitions file |childdoc.def|.

% The definitions cannot be loaded using |\usepackage| or |\RequirePackage|
% which has a mechanism to prevent loading a style file more than once.
% When loading the definitions by means of |\input|
% multiple instances have to be prevented manually:
%\iffalse
%This code needs to be before the `\ProvidesFile' directive
%which is defined at the beginning of this file.
%Therefore it is also placed there and commented out here.
%</package>
%<*discard>
%\fi
%    \begin{macrocode}
\ifdefined\childdocmain\endinput\fi
%    \end{macrocode}
%\iffalse
%</discard>
%<*package>
%\fi
%
% \macro{\ifchilddoc}
% \macro{\ifchilddocmanual}
% The conditional |\ifchilddoc| tells whether a
% child (true) or main (false) document is being compiled.
% The conditional |\ifchilddocmanual| tells whether
% the |\includeonly| mechanism is used (false) or
% the selection of child files must be performed manually (true).
% The definitions initialise to false:
%    \begin{macrocode}
\newif\ifchilddoc
\newif\ifchilddocmanual
%    \end{macrocode}

% \macro{\childdocname}
% \macro{\childdocjob}
% The macro |\childdocname| stores the name of the main document
% to be compiled. The macro |\childdocjob| stores the name of
% the document on which the \LaTeX{} compiler was originally invoked.
% The content of |\jobname| cannot be compared
% to filenames specified in the source due to different catcodes.
% The following code rescans |\jobname|, stores the result
% in |\childdocname| and saves a copy in |\childdocjob|:
%    \begin{macrocode}
\edef\childdocname{\scantokens\expandafter{\jobname\noexpand}}
\let\childdocjob\childdocname
%    \end{macrocode}

% \macro{\childdocdisable}
% The macro |\childdocdisable| prevents the main file
% from being processed more than once.
% At this stage, the main document command |\childdocmain|
% is assumed to be called once again where it should do nothing.
% Any subsequent call to it should prevent
% a secondary processing of the main document
% It overwrites the forwarding commands
% |\childdocof| and |\childdocforward|
% with empty macros to prevent further inclusions of the main document:
%    \begin{macrocode}
\newcommand{\childdocdisable}
{
  \renewcommand{\childdocmain}[1]{\renewcommand{\childdocmain}[1]{\endinput}}
  \renewcommand{\childdocof}[1]{}
  \renewcommand{\childdocby}[2][]{}
  \renewcommand{\childdocforward}[2][]{}
  \renewcommand{\childdocdisable}{}
}
%    \end{macrocode}

% \macro{\childdocmain}
% The macro |\childdocmain| is to be called at the top of the main file
% with nothing or the main filename (without extension) as argument.
% First, it breaks loops.
% If the argument is not empty and does not match |\childdocname|
% (which is set by the first inclusion of |childdoc.def|),
% |\ifchilddoc| is set to true, |\includeonly| is applied to the child file
% and |\jobname| is set to the main file
% (for proper handling of |.aux| files):
%    \begin{macrocode}
\newcommand{\childdocmain}[1]
{
  \childdocdisable\childdocmain{}
  \if?#1?\else
    \begingroup
      \def\childdoctmp{#1}
      \ifx\childdoctmp\childdocname
        \def\childdoctmp{}
      \else
        \def\childdoctmp
        {
          \childdoctrue
          \includeonly{\childdocname}
          \def\childdocjob{#1}
          \def\jobname{#1}
        }
      \fi
      \expandafter
    \endgroup
    \childdoctmp
  \fi
}
%    \end{macrocode}

% \macro{\childdocof}
% The command |\childdocof| redirects
% compilation to the main file |#1|.
%    \begin{macrocode}
\newcommand{\childdocof}[1]
{
  \childdocdisable
  \childdoctrue
  \includeonly{\childdocname}
  \def\jobname{#1}
  \def\childdocjob{#1}
  \input{#1}
}
%    \end{macrocode}

% \macro{\childdocby}
% The command |\childdocby| ....
%    \begin{macrocode}
\newcommand{\childdocby}[2][]
{
  \childdocdisable
  \childdoctrue
  \childdocmanualtrue
  \if?#1?\else
    \def\jobname{#2}
  \fi
  \def\childdocjob{#2}
  \input{#2}
  \endinput
}
%    \end{macrocode}

% \macro{\childdocforward}
% The command |\childdocforward| redirects
% compilation to the main file or
% (if the optional argument is given) a child file.
% Parameters are set as if the main file
% or a child file starting with |\childdocof| was compiled.
% Then compilation is handed over to the main file:
%    \begin{macrocode}
\newcommand{\childdocforward}[2][]
{
  \begingroup
    \if?#1?
      \def\childdoctmp
      {
        \def\childdocname{#2}
        \def\childdocjob{#2}
        \def\jobname{#2}
        \input{#2}
        \endinput
      }
    \else
      \def\childdoctmp
      {
        \childdocdisable
        \def\childdocname{#2}
        \childdoctrue
        \includeonly{#2}
        \def\childdocjob{#1}
        \def\jobname{#1}
        \input{#1}
        \endinput
      }
    \fi
    \expandafter
  \endgroup
  \childdoctmp
}
%    \end{macrocode}

% \macro{\childdocforwardprefix}
% The command |\childdocforwardprefix| redirects
% compilation to the main or a child file by means of a pattern.
% The prefix |#1| in the current filename is replaced by |#2|
% and the suffix of the current filename is kept
% (it is assumed that the filename does not contain the substring `|~~~|'
% which is used as a delimiter).
% Compilation is handed over to the new file by |\childdocforward|:
%    \begin{macrocode}
\newcommand{\childdocforwardprefix}[3][]
{
  \begingroup
    \def\childdocextract #2##1~~~{\def\childdoctmp{\childdocforward[#1]{#3##1}}}
    \expandafter\childdocextract\childdocname~~~
    \expandafter
  \endgroup
  \childdoctmp
}
%    \end{macrocode}

% \macro{\childdoc}
% The deprecated macro |\childdoc| is a legacy version of |\childdocmain|:
%    \begin{macrocode}
\newcommand{\childdoc}{\childdocmain}
%    \end{macrocode}

% \macro{\childdocredirect}
% The deprecated macro |\childdocredirect| is a legacy version
% of |\childdocforward| and |\childdocforwardprefix|:
%    \begin{macrocode}
\newcommand{\childdocredirect}[2][]
{
  \begingroup
    \if?#1?
      \def\childdoctmp{\childdocforward{#2}}
    \else
      \def\childdoctmp{\childdocforwardprefix{#1}{#2}}
    \fi
    \expandafter
  \endgroup
  \childdoctmp
}
%    \end{macrocode}

%\iffalse
%</package>
%\fi
%
\endinput
|\\
|\childdocmain{|\textit{main}|}|\\
\end{tabular}
\end{center}
%
If |\jobname| does not match the argument \textit{main} of |\childdocmain|,
it is assumed that |\jobname| points to the child file to be compiled.
When using |\childdocmain| with the main file specified as argument,
it suffices to start a child file
with just |\input{|\textit{main}|}|
without loading of the package and using |\childdocof|.
If instead all processing is done
with the appropriate \textsf{childdoc} directives,
the argument of \textit{main} of |\childdocmain| can be empty.

An alternative version of the command line processing described
in \secref{sec:commandline} using the detection mechanism reads:
%
\begin{center}
|... -jobname "|\textit{target}|" "|[\textit{flags}]%
[|\def\jobname{|\textit{dest}|}|]|\input{|\textit{main}|}"|
\end{center}

%%%%%%%%%%%%%%%%%%%%%%%%%%%%%%%%%%%%%%%%%%%%%%%%%%%%%%%%%%%%%%%%%%%%%%%%%%%%%%%%
\subsection{Manual Code}
\label{sec:manual}

In case one cannot be certain whether the definitions file |childdoc.def|
is installed on the target \TeX{} distribution
and one prefers not to ship it,
it is conceivable to paste a few relevant commands into the sources.

To that end, drop all statements |% \iffalse
%
% childdoc.dtx Copyright (C) 2017-2018 Niklas Beisert
%
% This work may be distributed and/or modified under the
% conditions of the LaTeX Project Public License, either version 1.3
% of this license or (at your option) any later version.
% The latest version of this license is in
%   http://www.latex-project.org/lppl.txt
% and version 1.3 or later is part of all distributions of LaTeX
% version 2005/12/01 or later.
%
% This work has the LPPL maintenance status `maintained'.
%
% The Current Maintainer of this work is Niklas Beisert.
%
% This work consists of the files childdoc.dtx and childdoc.ins
% and the derived files childdoc.def and cdocsamp.tex with
% cdocsch1.tex, cdocsch2.tex, cdocsdrf.tex, cdocsfn1.tex, cdocsfn2.tex.
%
%<package>\ifdefined\childdocmain\endinput\fi
%<package>\ProvidesFile{childdoc.def}[2018/12/30 v2.0 child document driver]
%<samplemain>\ProvidesFile{cdocsamp.tex}[2018/12/30 v2.0 sample for childdoc]
%<*driver>
%\ProvidesFile{childdoc.drv}[2018/12/30 v2.0 childdoc reference manual file]
\PassOptionsToClass{10pt,a4paper}{article}
\documentclass{ltxdoc}

\usepackage[margin=35mm]{geometry}
\usepackage{hyperref}
\usepackage{hyperxmp}
\usepackage[usenames]{color}

\hypersetup{colorlinks=true}
\hypersetup{pdfstartview=FitH}
\hypersetup{pdfpagemode=UseNone}
\hypersetup{pdfsource={}}
\hypersetup{pdflang={en-UK}}
\hypersetup{pdfcopyright={Copyright 2017-2018 Niklas Beisert.
  This work may be distributed and/or modified under the
  conditions of the LaTeX Project Public License, either version 1.3
  of this license or (at your option) any later version.}}
\hypersetup{pdflicenseurl={http://www.latex-project.org/lppl.txt}}
\hypersetup{pdfcontactaddress={ETH Zurich, ITP, HIT K,
  Wolfgang-Pauli-Strasse 27}}
\hypersetup{pdfcontactpostcode={8093}}
\hypersetup{pdfcontactcity={Zurich}}
\hypersetup{pdfcontactcountry={Switzerland}}
\hypersetup{pdfcontactemail={nbeisert@itp.phys.ethz.ch}}
\hypersetup{pdfcontacturl={http://people.phys.ethz.ch/\xmptilde nbeisert/}}

\newcommand{\secref}[1]{\hyperref[#1]{section \ref*{#1}}}

\parskip1ex
\parindent0pt
\let\olditemize\itemize
\def\itemize{\olditemize\parskip0pt}

\begin{document}

\title{The \textsf{childdoc} Package}
\hypersetup{pdftitle={The childdoc Package}}
\author{Niklas Beisert\\[2ex]
  Institut f\"ur Theoretische Physik\\
  Eidgen\"ossische Technische Hochschule Z\"urich\\
  Wolfgang-Pauli-Strasse 27, 8093 Z\"urich, Switzerland\\[1ex]
  \href{mailto:nbeisert@itp.phys.ethz.ch}
  {\texttt{nbeisert@itp.phys.ethz.ch}}}
\hypersetup{pdfauthor={Niklas Beisert}}
\hypersetup{pdfsubject={Manual for the LaTeX2e Package childdoc}}
\date{30 December 2018, \textsf{v2.0}}
\maketitle

\begin{abstract}\noindent
\textsf{childdoc} is a \LaTeXe{} package
that enables the direct compilation
of document sections included by |\include|
to individual files.
\end{abstract}

\begingroup
\parskip0ex
\tableofcontents
\endgroup

%%%%%%%%%%%%%%%%%%%%%%%%%%%%%%%%%%%%%%%%%%%%%%%%%%%%%%%%%%%%%%%%%%%%%%%%%%%%%%%%
%%%%%%%%%%%%%%%%%%%%%%%%%%%%%%%%%%%%%%%%%%%%%%%%%%%%%%%%%%%%%%%%%%%%%%%%%%%%%%%%
\section{Introduction}

\LaTeX{} provides a mechanism to structure a large document (such as a book)
into a main file and several child files (containing the chapters)
using the |\include| command.
This mechanism is beneficial for documents
which span hundreds of pages in order to
make the source file(s) more manageable.
Moreover, compilation can be restricted to
selected child files by means of the |\includeonly| command.
The latter feature can be used to reduce the compilation time while editing
(this was significantly more useful in the earlier days of \LaTeX{})
or to generate a smaller document which is easier to navigate.
Another application of |\includeonly| is to generate
documents consisting of selected parts of the complete document.

However, there are a few drawbacks of the plain |\include| mechanism:
\begin{itemize}
\item
The child files cannot be compiled on their own,
they can only be compiled via the main file.
A naive editing environment
(such as a text editor with an option
to have the current file processed by \LaTeX)
may require one to switch to the main file before compiling;
attempting to compile the child file produces errors.
\item
The main file must be modified (each time)
to adjust the |\includeonly| command
to the present needs. This easily leaves the main file in a messy state.
\item
The generated document will always carry the filename
of the main document. This is inconvenient if
several child files are to be compiled and
to be kept for distribution.
\end{itemize}

The present package provides a simple interface
to make child files individually compilable by \LaTeX{}.
Compiling a child file then has the same effect as compiling
the main file with an |\includeonly| command
to select the appropriate child.
Moreover the generated document will carry the name of the child
rather than the main file.
This resolves all three above issues.

This feature is meant to make the editing of books,
thesis documents and lecture notes somewhat more convenient.
However, the package can also be used efficiently for
composing a series of documents (such as exercise sheets)
which are typically distributed individually.
It then assists the author in generating the individual documents
(potentially in different versions)
as well as a document containing the collected series.
Another application is in developing style files
or other kinds of included material
where compilation of the style file could redirect
to a sample or test file.

%%%%%%%%%%%%%%%%%%%%%%%%%%%%%%%%%%%%%%%%%%%%%%%%%%%%%%%%%%%%%%%%%%%%%%%%%%%%%%%%
%%%%%%%%%%%%%%%%%%%%%%%%%%%%%%%%%%%%%%%%%%%%%%%%%%%%%%%%%%%%%%%%%%%%%%%%%%%%%%%%
\section{Usage}

First of all, the package \textsf{childdoc} is \emph{not} a standard
\LaTeXe{} |.sty| style file! Therefore it needs to be invoked in
a non-standard way.

%%%%%%%%%%%%%%%%%%%%%%%%%%%%%%%%%%%%%%%%%%%%%%%%%%%%%%%%%%%%%%%%%%%%%%%%%%%%%%%%
\subsection{Included Files}
\label{sec:include}

%%%%%%%%%%%%%%%%%%%%%%%%%%%%%%%%%%%%%%%%
\DescribeMacro{\childdocmain}
To use the package, add the commands
\begin{center}
\begin{tabular}{l}
|\input{childdoc.def}|\\
|\childdocmain{}|\\
\end{tabular}
\end{center}
at the very top of the main \LaTeX{} file,
in particular \emph{before} the |\documentclass| statement!
The argument of |\childdocmain| should be left empty
(but it must be present).

%%%%%%%%%%%%%%%%%%%%%%%%%%%%%%%%%%%%%%%%
\DescribeMacro{\childdocof}
Furthermore, add the commands
\begin{center}
\begin{tabular}{l}
|\input{childdoc.def}|\\
|\childdocof{|\textit{main}|}|\\
\end{tabular}
\end{center}
at the top of every child file \textit{child}
which is included by |\include{|\textit{child}|}|
from within the main file
(or at least for those files to be compiled individually).
The argument \textit{main} must be the filename of the main file.

There are a couple of
considerations in setting up the main and child documents:

%%%%%%%%%%%%%%%%%%%%%%%%%%%%%%%%%%%%%%%%
\paragraph{Restrictions.}

Please note the following restrictions:
\begin{itemize}
\item
|\childdocmain| must be called with one argument \textit{main}
to ensure compatibility with earlier version of the package.
It must either be empty (|\childdocmain{}|)
or precisely match the filename of the main file in which it is specified.
See \secref{sec:detection} for further information.
\item
The filename \textit{main} must be specified without the |.tex| extension.
\item
The filename \textit{main} is case sensitive
(even in case-insensitive file systems)
due to internal string comparison.
\item
The argument \textit{main} should be fully expanded, it cannot be a macro.
\item
Subdirectories and special characters should be avoided in filenames.
\item
The command |\childdocmain{|\textit{main}|}| must be followed by a whitespace.
It should not be followed immediately by another command
or by a comment mark `|%|'.
This is because the \TeX{} parser reads the token immediately following
the argument of |\childdocmain| and puts it
at the beginning of every child section;
however, a white\-space is ignored.
\end{itemize}

%%%%%%%%%%%%%%%%%%%%%%%%%%%%%%%%%%%%%%%%
\paragraph{Content of Main File.}

It is advisable to place all content in the child files included by |\include|.
Any output contained in the main file will appear in all child documents
unless suppressed manually;
it cannot be suppressed automatically by the |\includeonly| directive
and thus should normally be avoided.
A method to include some content in the main file
by means of conditional processing is described in \secref{sec:conditional}.

%%%%%%%%%%%%%%%%%%%%%%%%%%%%%%%%%%%%%%%%
\paragraph{Page Numbering.}

When only a part of the document is compiled,
the appropriate numbering of pages
(as well as other status parameters)
is determined from the |.aux| files.
The latter contain information from previous passes.
However this information needs to propagate through
all intermediate child documents.
Therefore the page numbering in child documents may well
be inconsistent until the complete document is compiled at least once.

A useful (if unconventional) way to always ensure a consistent
page numbering is to restart the numbering in each child document
and denote the pages by `\textit{child}|.|\textit{page}'
where \textit{child} represents the chapter/section number of the child file.
This can be achieved by the command
|\numberwithin{page}{|\textit{child}|}|
of the \textsf{amsmath} package
where \textit{child} can be |chapter| or |section|
depending on the chosen structuring.
Alternatively, one can modify the macro |\thepage| appropriately
and reset the counter |page| at the start of each child file.

%%%%%%%%%%%%%%%%%%%%%%%%%%%%%%%%%%%%%%%%%%%%%%%%%%%%%%%%%%%%%%%%%%%%%%%%%%%%%%%%
\subsection{Conditional Processing}
\label{sec:conditional}

The package provides a mechanism to compile different versions
of a document. To customise the versions further some conditional processing
can come in handy to distinguish which version is being compiled.
The package provides two macros to describe the compilation context:

%%%%%%%%%%%%%%%%%%%%%%%%%%%%%%%%%%%%%%%%
\DescribeMacro{\ifchilddoc}
The conditional |\ifchilddoc| distinguishes between the compilation of
child documents and the main document:
%
\begin{center}
|\ifchilddoc |\textit{child-code}| |[|\||else |\textit{main-code}]| \||fi|
\end{center}

%%%%%%%%%%%%%%%%%%%%%%%%%%%%%%%%%%%%%%%%
\DescribeMacro{\childdocname}
\DescribeMacro{\childdocjob}
The macro |\childdocname| contains the filename (without extension)
of the main or child file being processed.
Note that |\childdocjob| will always contain the name of the main file.

%%%%%%%%%%%%%%%%%%%%%%%%%%%%%%%%%%%%%%%%
\paragraph{Title Page.}

Conditional processing can be used to include a title or banner page
in the main document when proper precautions are taken.
Importantly, the code in the main file should ensure that the page counter
(as well as other status parameters which are stored in the |.aux| files)
takes the same value after the conditional processing.
Otherwise the page numbers may take divergent values
depending on which part is compiled.

For example, a title page could be declared by:
%
\begin{center}
\begin{tabular}{l}
|\ifchilddoc\||else|\\
|\addtocounter{page}{-1}|\\
\textit{code for title page}\\
|\newpage|\\
|\||fi|
\end{tabular}
\end{center}
%
A banner page for the child documents can be generated by:
%
\begin{center}
\begin{tabular}{l}
|\ifchilddoc|\\
|\addtocounter{page}{-1}|\\
\textit{code for banner page}\\
|\newpage|\\
|\||fi|
\end{tabular}
\end{center}
%
Here one could write a message such as:
\begin{center}
|This is the part \childdocname{} of \childdocjob{}.|
\end{center}

%%%%%%%%%%%%%%%%%%%%%%%%%%%%%%%%%%%%%%%%%%%%%%%%%%%%%%%%%%%%%%%%%%%%%%%%%%%%%%%%
\subsection{Flags}
\label{sec:flags}

The package makes it easy to generate different versions
of the main or child documents.
To this end compilation flags can be defined
and assigned different default values.
They will be particularly useful in conjunction
with the forwarding mechanism described in \secref{sec:forward}.

For example, it may be useful to have a flag |\version|
which can be set to |draft| or |final|.
The document source will contain some conditional code
depending on the value of |\version|.
Suppose further, the flag should default to |final| for the main file
and to |draft| for child files
which is a natural assignment for editing the document.
This is achieved by placing the following code
in the preamble of the main document
(below the |\childdocmain| directive):
%
\begin{center}
\begin{tabular}{l}
|\ifchilddoc|\\
|\providecommand{\version}{draft}|\\
|\||else|\\
|\providecommand{\version}{final}|\\
|\||fi|
\end{tabular}
\end{center}
%
The definition by |\providecommand| makes sure
that previous definitions are not overwritten.
Further statements |\providecommand{\version}{...}|
can thus be added before the above code to override it.

For the main file, one might add a line
(between |\childdocmain| and the above block)
%
\begin{center}
|%\ifchilddoc\||else\providecommand{\version}{draft}\||fi|
\end{center}
%
which can be uncommented to produce a draft version.
Likewise one can add a line to the very top of a child file
(above the |\childdocof{|\textit{main}|}| directive)
%
\begin{center}
|%\providecommand{\version}{final}|
\end{center}
%
which can be uncommented to produce the final version of this child document.

%%%%%%%%%%%%%%%%%%%%%%%%%%%%%%%%%%%%%%%%%%%%%%%%%%%%%%%%%%%%%%%%%%%%%%%%%%%%%%%%
\subsection{Forwarding}
\label{sec:forward}

Different versions of the main or child documents
using compilation flags as described in \secref{sec:flags}
can be (permanently) stored in different files
for convenient compilation, viewing and distribution.
To this end, the package defines a command
to pass on compilation to a different file:

%%%%%%%%%%%%%%%%%%%%%%%%%%%%%%%%%%%%%%%%
\DescribeMacro{\childdocforward}
The command |\childdocforward| redirects processing to
another source file:
%
\begin{center}
\begin{tabular}{l}
|\input{childdoc.def}|\\
|\childdocforward[|\textit{main}|]{|\textit{dest}|}|\\
\end{tabular}
\end{center}
%
The argument \textit{dest} is the destination file
(without extension).
It should be the main file or one of the child files.
Note that further \textsf{childdoc} directives
such as |\childdocof| and |\childdocforward|
in the indicated file will be processed in this form.
The optional argument \textit{main}
passes on directly to the main file \textit{main}
while pretending to compile the child \textit{dest}.
This form behaves as if \textit{dest}
issues |\childdocof{|\textit{main}|}| right away,
and no further \textsf{childdoc} directives will be processed.

%%%%%%%%%%%%%%%%%%%%%%%%%%%%%%%%%%%%%%%%
\DescribeMacro{\...prefix}
In the alternative form |\childdocforwardprefix|,
%
\begin{center}
\begin{tabular}{l}
|\input{childdoc.def}|\\
|\childdocforwardprefix[|\textit{main}|]{|\textit{prefix}|}{|\textit{dest}|}|
\end{tabular}
\end{center}
%
the destination file is determined by a pattern
depending on the current file:
To make this work, the current file must be called
`{\textit{prefix}\hspace{0.2em}\textit{suffix}}'
with \textit{prefix} matching precisely the argument.
Processing is then passed on to the file
`{\textit{dest}\hspace{0.2em}\textit{suffix}}'.
Surely, the same effect is achieved by
directly specifying the
argument `{\textit{dest}\hspace{0.2em}\textit{suffix}}'
in the first form.
However, that requires to set up a different file
for each child. With the alternative form of the command
all these files can have exactly the same content
which simplifies setting them up and maintaining them.

For example, the following file |draft.tex|
with a compilation flag |\version| as described in \secref{sec:flags}
compiles the main document as a draft:
%
\begin{center}
\begin{tabular}{l}
|\def\version{draft}|\\
|\input{childdoc.def}|\\
|\childdocforward{|\textit{main}|}|
\end{tabular}
\end{center}
%
Likewise, the following files |final|\textit{nn}|.tex|
compile the final version of the child document
|child|\textit{nn}|.tex|:
%
\begin{center}
\begin{tabular}{l}
|\def\version{final}|\\
|\input{childdoc.def}|\\
|\childdocforwardprefix{final}{child}|
\end{tabular}
\end{center}
%

Note that when several versions of a main file and/or of each child file
are to be generated, it may be convenient to set up a |Makefile| or
shell script to automatise the process.

%%%%%%%%%%%%%%%%%%%%%%%%%%%%%%%%%%%%%%%%%%%%%%%%%%%%%%%%%%%%%%%%%%%%%%%%%%%%%%%%
\subsection{Command Line Processing}
\label{sec:commandline}

The effect of redirection files can also be achieved by invoking
the \LaTeX{} compiler with a more elaborate command line.
Most conveniently this should be done as part
of a shell script or a |Makefile|.

When using \textsf{childdoc} in the main file, the following
command lines effectively perform a redirection
(note that depending on the shell being used,
backslashes may have to be doubled: `|\|' $\to$ `|\\|'):
%
\begin{center}
|... -jobname "|\textit{target}|" |\\|"|[\textit{flags}]%
|\input{childdoc.def}\childdocforward[|\textit{main}|]{|\textit{dest}|}"|
\end{center}
%
Here \textit{target} is the name of the output file,
\textit{main} is the name of the main file
and \textit{dest} is the name of the main or child file to be processed
(all filenames without extensions).
The optional argument \textit{main} can be omitted
if \textit{main} matches \textit{dest}.
Optionally, compilation \textit{flags} can be defined via |\def| commands.
This command line makes the \TeX{} engine believe
it is compiling the file \textit{target}
whose content is specified as the latter parameter.
The provided code then forwards the processing to
\textit{main} or \textit{dest} as described in \secref{sec:forward}.

%%%%%%%%%%%%%%%%%%%%%%%%%%%%%%%%%%%%%%%%%%%%%%%%%%%%%%%%%%%%%%%%%%%%%%%%%%%%%%%%
\subsection{Include by Input}
\label{sec:input}

Including child documents by |\include| has some restrictions by design.
Most notably, the content of a child document always occupies
its own set of pages; pages cannot be shared between child documents.
Usually, this behaviour makes perfect sense
because each child document contain an essential part of the document.
However, in some situations it may be desirable to compose
a document from a collection of parts
without having mandatory page breaks between then.
For this case, the package
provides a mechanism to include parts
by |\input| which can also be processed individually.
However, by construction this mechanism
requires manual handling of the content to be output.

%%%%%%%%%%%%%%%%%%%%%%%%%%%%%%%%%%%%%%%%
\DescribeMacro{\ifchilddocmanual}
The main file should be prepared as usual, see \secref{sec:include}.
However, the document body must make a distinction
between processing of an individual part and of the main document, e.g.:
%
\begin{center}
\begin{tabular}{l}
|\ifchilddocmanual|\\
|\input{\childdocname}|\\
|\||else|\\
\textit{document body with }|\input{|\textit{part}|}|\\
|\||fi|
\end{tabular}
\end{center}
%
The conditional |\ifchilddocmanual| is true whenever
a part to be included by |\input| is being compiled,
and the name of the part is stored in |\childdocname|.

%%%%%%%%%%%%%%%%%%%%%%%%%%%%%%%%%%%%%%%%
\DescribeMacro{\childdocby}
Each part to be included by |\input| should start with:
%
\begin{center}
\begin{tabular}{l}
|\input{childdoc.def}|\\
|\childdocby{|\textit{main}|}|\\
\end{tabular}
\end{center}
%
The directive |\childdocby| is similar to |\childdocof|
described in \secref{sec:include},
but the subsequent selection of content must be done manually.
To that end, both |\ifchilddoc| and |\ifchilddocmanual|
will be true upon processing of a part,
and the name of the part is stored in |\childdocname|.
Note that |\jobname| will be set to the filename of the current part
so that each part receives an individual |.aux| file
that does not interfere with the |.aux| file(s) of the main document.
This behaviour can be altered by the alternative form
|\childdocby[*]{|\textit{main}|}| (with a non-empty optional argument)
which uses the |.aux| file of the main document
by setting |\jobname| to \textit{main}.

%%%%%%%%%%%%%%%%%%%%%%%%%%%%%%%%%%%%%%%%%%%%%%%%%%%%%%%%%%%%%%%%%%%%%%%%%%%%%%%%
\subsection{Driver Development}
\label{sec:driver}

The \textsf{childdoc} mechanism can also be use for the development
of definition files such as \LaTeX{} styles or classes.
This case differs from the above setup with multiple parts
included by |\include| in that no |\includeonly| should be invoked.
This can be achieved by starting the include file
(before |\ProvidesPackage|) with:
%
\begin{center}
\begin{tabular}{l}
|\input{childdoc.def}|\\
|\childdocforward{|\textit{main}|}|\\
\end{tabular}
\end{center}
%
or alternatively with:
%
\begin{center}
\begin{tabular}{l}
|\input{childdoc.def}|\\
|\childdocby{|\textit{main}|}|\\
\end{tabular}
\end{center}
%
Both forms have slightly different effects as described above.
The main file is prepared as usual, see \secref{sec:include}.

%%%%%%%%%%%%%%%%%%%%%%%%%%%%%%%%%%%%%%%%%%%%%%%%%%%%%%%%%%%%%%%%%%%%%%%%%%%%%%%%
\subsection{Legacy Detection}
\label{sec:detection}

The directive |\childdocmain| in the main file can detect
whether the complete document or merely a child is to be compiled
even without using the directive |\childdocof|.
This method is deprecated because it is less robust
and there is no compelling reason to use it;
it is merely provided for backward compatibility
and it may be removed in future versions.

If the detection mechanism is to be used,
it is mandatory to correctly specify
the filename of the main file as the argument of |\childdocmain|:
%
\begin{center}
\begin{tabular}{l}
|\input{childdoc.def}|\\
|\childdocmain{|\textit{main}|}|\\
\end{tabular}
\end{center}
%
If |\jobname| does not match the argument \textit{main} of |\childdocmain|,
it is assumed that |\jobname| points to the child file to be compiled.
When using |\childdocmain| with the main file specified as argument,
it suffices to start a child file
with just |\input{|\textit{main}|}|
without loading of the package and using |\childdocof|.
If instead all processing is done
with the appropriate \textsf{childdoc} directives,
the argument of \textit{main} of |\childdocmain| can be empty.

An alternative version of the command line processing described
in \secref{sec:commandline} using the detection mechanism reads:
%
\begin{center}
|... -jobname "|\textit{target}|" "|[\textit{flags}]%
[|\def\jobname{|\textit{dest}|}|]|\input{|\textit{main}|}"|
\end{center}

%%%%%%%%%%%%%%%%%%%%%%%%%%%%%%%%%%%%%%%%%%%%%%%%%%%%%%%%%%%%%%%%%%%%%%%%%%%%%%%%
\subsection{Manual Code}
\label{sec:manual}

In case one cannot be certain whether the definitions file |childdoc.def|
is installed on the target \TeX{} distribution
and one prefers not to ship it,
it is conceivable to paste a few relevant commands into the sources.

To that end, drop all statements |\input{childdoc.def}|
and perform the replacements as outlined below.
Instead of |\childdocmain{|\textit{main}|}| add the following code
to the top of the main file:
%
\begin{center}
\begin{tabular}{l}
|\||ifdefined\childdocname\endinput\||fi\newif\ifchilddoc|\\
|\edef\childdocname{\scantokens\expandafter{\jobname\noexpand}}|\\
|\def\childdocmain{|\textit{main}|}\||ifx\childdocmain\childdocname\||else|\\
|\childdoctrue\includeonly{\childdocname}\let\jobname\childdocmain\||fi|\\
\end{tabular}
\end{center}
%
Instead of |\childdocof{|\textit{main}|}| just include the main file
at the top of each child file:
%
\begin{center}
|\input{|\textit{main}|}|
\end{center}
%
A simple redirection |\childdocforward{|\textit{dest}|}| is achieved by:
%
\begin{center}
|\def\jobname{|\textit{dest}|}\input{\jobname}|
\end{center}
%
The redirection with prefix
|\childdocforwardprefix[|\textit{prefix}|]{|\textit{dest}|}|
is accomplished by:
%
\begin{center}
\begin{tabular}{l}
|{\edef\jobname{\scantokens\expandafter{\jobname\noexpand}}|\\
|\def\redirectjob |\textit{prefix}|#1~~~{\gdef\jobname{|\textit{dest}|#1}}|\\
|\expandafter\redirectjob\jobname~~~}\input{\jobname}|
\end{tabular}
\end{center}

In an alternative approach,
child documents can be compiled by a specific command line
without additional code or specific definitions:
%
\begin{center}
|... -jobname "|\textit{target}|" "|[\textit{flags}]%
|\includeonly{|\textit{dest}|}\input{|\textit{main}|}"|
\end{center}
%

%%%%%%%%%%%%%%%%%%%%%%%%%%%%%%%%%%%%%%%%%%%%%%%%%%%%%%%%%%%%%%%%%%%%%%%%%%%%%%%%
%%%%%%%%%%%%%%%%%%%%%%%%%%%%%%%%%%%%%%%%%%%%%%%%%%%%%%%%%%%%%%%%%%%%%%%%%%%%%%%%
\section{Information}

%%%%%%%%%%%%%%%%%%%%%%%%%%%%%%%%%%%%%%%%%%%%%%%%%%%%%%%%%%%%%%%%%%%%%%%%%%%%%%%%
\subsection{Copyright}

Copyright \copyright{} 2017--2018 Niklas Beisert

This work may be distributed and/or modified under the
conditions of the \LaTeX{} Project Public License, either version 1.3
of this license or (at your option) any later version.
The latest version of this license is in
  \url{http://www.latex-project.org/lppl.txt}
and version 1.3 or later is part of all distributions of \LaTeX{}
version 2005/12/01 or later.

This work has the LPPL maintenance status `maintained'.

The Current Maintainer of this work is Niklas Beisert.

This work consists of the files |README.txt|, |childdoc.ins| and |childdoc.dtx|
as well as the derived files |childdoc.def|, |cdocsamp.tex|
with |cdocsch1.tex|, |cdocsch2.tex|, |cdocspt3.tex|, |cdocspt4.tex|,
|cdocsdrf.tex|, |cdocsfn1.tex|, |cdocsfn2.tex|
as well as |childdoc.pdf|.

%%%%%%%%%%%%%%%%%%%%%%%%%%%%%%%%%%%%%%%%%%%%%%%%%%%%%%%%%%%%%%%%%%%%%%%%%%%%%%%%
\subsection{Files and Installation}

The package consists of the files:
%
\begin{center}
\begin{tabular}{ll}
    |README.txt|   & readme file \\
    |childdoc.ins| & installation file \\
    |childdoc.dtx| & source file \\
    |childdoc.def| & definition file \\
    |cdocsamp.tex| & sample main file \\
    |cdocsch1.tex| & sample include file \\
    |cdocsch2.tex| & sample include file \\
    |cdocspt3.tex| & sample part file \\
    |cdocspt4.tex| & sample part file \\
    |cdocsdrf.tex| & sample redirection file \\
    |cdocsfn1.tex| & sample redirection file \\
    |cdocsfn2.tex| & sample redirection file \\
    |childdoc.pdf| & manual
\end{tabular}
\end{center}
%
The distribution consists of the files
|README.txt|, |childdoc.ins| and |childdoc.dtx|.
%
\begin{itemize}
\item
Run (pdf)\LaTeX{} on |childdoc.dtx|
to compile the manual |childdoc.pdf| (this file).
\item
Run \LaTeX{} on |childdoc.ins| to create the definitions file |childdoc.def|
and the sample |cdocsamp.tex| with include files
|cdocsch1.tex|, |cdocsch2.tex|, |cdocspt3.tex|, |cdocspt4.tex|,
|cdocsdrf.tex|, |cdocsfn1.tex|, |cdocsfn2.tex|.
Then copy the file |childdoc.def| to an appropriate directory of your \LaTeX{}
distribution, e.g.\ \textit{texmf-root}|/tex/latex/childdoc|.
\end{itemize}

%%%%%%%%%%%%%%%%%%%%%%%%%%%%%%%%%%%%%%%%%%%%%%%%%%%%%%%%%%%%%%%%%%%%%%%%%%%%%%%%
\subsection{Related CTAN Packages}

There are several other packages which offer a similar functionality:
%
\begin{itemize}
\item
The packages
\href{http://ctan.org/pkg/docmute}{\textsf{docmute}},
\href{http://ctan.org/pkg/includex}{\textsf{includex}} and
\href{http://ctan.org/pkg/standalone}{\textsf{standalone}}
provide commands to include only the document body of
a child file thus allowing both files to be compiled individually.
\item
The packages \href{http://ctan.org/pkg/subdocs}{\textsf{subdocs}}
and \href{http://ctan.org/pkg/subfiles}{\textsf{subfiles}}
provide structures in which the main and child documents can be
encapsulated and allowing them to be compiled individually.
The inclusion mechanism is different from the conventional |\include|.
\item
The package \href{http://ctan.org/pkg/combine}{\textsf{combine}}
is an elaborate solution to combine several documents into one.
\end{itemize}
%
See also the CTAN topic \href{http://ctan.org/topic/subdocs}{\textsf{subdocs}}
for further related packages.
The present package differs from the above solutions in that
a document structure constructed with the conventional |\include| mechanism
just needs two extra commands at the top of every file
such that all constituent files can be compiled individually.

%%%%%%%%%%%%%%%%%%%%%%%%%%%%%%%%%%%%%%%%%%%%%%%%%%%%%%%%%%%%%%%%%%%%%%%%%%%%%%%%
%\subsection{Feature Suggestions}
%
%The following is a list of features which may be useful for future
%versions of this package:
%%
%\begin{itemize}
%\item
%\ldots
%\end{itemize}

%%%%%%%%%%%%%%%%%%%%%%%%%%%%%%%%%%%%%%%%%%%%%%%%%%%%%%%%%%%%%%%%%%%%%%%%%%%%%%%%
\subsection{Revision History}

%%%%%%%%%%%%%%%%%%%%%%%%%%%%%%%%%%%%%%%%
\paragraph{v2.0:} 2018/12/30

\begin{itemize}
\item
immediate forward processing
\item
added |\childdocby| mechanism
\item
manual restructured
\end{itemize}

%%%%%%%%%%%%%%%%%%%%%%%%%%%%%%%%%%%%%%%%
\paragraph{v1.6:} 2018/01/17

\begin{itemize}
\item
application for development of include files
\item
corrections to manual
\end{itemize}

%%%%%%%%%%%%%%%%%%%%%%%%%%%%%%%%%%%%%%%%
\paragraph{v1.5:} 2017/05/21

\begin{itemize}
\item
more complete structuring introduced
\item
|\childdocof| introduced
\item
|\childdoc| renamed to |\childdocmain|
\item
|\childredirect| renamed to |\childdocforward| and |\childdocforwardprefix|
and functionality expanded
\end{itemize}

%%%%%%%%%%%%%%%%%%%%%%%%%%%%%%%%%%%%%%%%
\paragraph{v1.0:} 2017/04/27

\begin{itemize}
\item
manual and install package
\item
first version published on CTAN
\end{itemize}

%%%%%%%%%%%%%%%%%%%%%%%%%%%%%%%%%%%%%%%%
\paragraph{v0.6:} 2017/04/26

\begin{itemize}
\item
redirection mechanism added
\end{itemize}

%%%%%%%%%%%%%%%%%%%%%%%%%%%%%%%%%%%%%%%%
\paragraph{v0.5:} 2017/04/26

\begin{itemize}
\item
functionality in definition file
\end{itemize}


%%%%%%%%%%%%%%%%%%%%%%%%%%%%%%%%%%%%%%%%%%%%%%%%%%%%%%%%%%%%%%%%%%%%%%%%%%%%%%%%
%%%%%%%%%%%%%%%%%%%%%%%%%%%%%%%%%%%%%%%%%%%%%%%%%%%%%%%%%%%%%%%%%%%%%%%%%%%%%%%%
%%%%%%%%%%%%%%%%%%%%%%%%%%%%%%%%%%%%%%%%%%%%%%%%%%%%%%%%%%%%%%%%%%%%%%%%%%%%%%%%
\appendix

\settowidth\MacroIndent{\rmfamily\scriptsize 000\ }

 \DocInput{childdoc.dtx}

\end{document}
%</driver>
% \fi
%
% %%%%%%%%%%%%%%%%%%%%%%%%%%%%%%%%%%%%%%%%%%%%%%%%%%%%%%%%%%%%%%%%%%%%%%%%%%%%%%
% %%%%%%%%%%%%%%%%%%%%%%%%%%%%%%%%%%%%%%%%%%%%%%%%%%%%%%%%%%%%%%%%%%%%%%%%%%%%%%
% \section{Sample}
%\iffalse
%<*samplemain>
%\fi
%
% The following presents a sample document
% with two chapters, two parts, a title page,
% a compile flag as well as three forwarding files to set the flag.
% It consists of eight |.tex| files:
% \begin{center}
% \begin{tabular}{ll}
% |cdocsamp.tex|&main file\\
% |cdocsch1.tex|&include file for chapter 1\\
% |cdocsch2.tex|&include file for chapter 2\\
% |cdocspt3.tex|&include file for part 3\\
% |cdocspt4.tex|&include file for part 4\\
% |cdocsdrf.tex|&forwarding file for main file in draft mode\\
% |cdocsfi1.tex|&forwarding file for final version of chapter 1\\
% |cdocsfi2.tex|&forwarding file for final version of chapter 2\\
% \end{tabular}
% \end{center}
% Each of the eight files can be compiled directly by the \LaTeX{} compiler.
%
% %%%%%%%%%%%%%%%%%%%%%%%%%%%%%%%%%%%%%%
% \paragraph{Main File.}
%
% The main file is called |cdocsamp.tex|.
%
% Load the \textsf{childdoc} definitions and
% declare the filename for the main document:
%    \begin{macrocode}
\input{childdoc.def}
\childdocmain{}
%    \end{macrocode}

% Optional override for |\version| flag:
%    \begin{macrocode}
%%\ifchilddoc\else\providecommand{\version}{draft}\fi
%    \end{macrocode}

% Define the default values for the |\version| flag
% (|final| for the main file and |draft| for childs):
%    \begin{macrocode}
\ifchilddoc
\providecommand{\version}{draft}
\else
\providecommand{\version}{final}
\fi
%    \end{macrocode}

% Load the standard document class:
%    \begin{macrocode}
\documentclass[12pt]{article}
%    \end{macrocode}

% Start the document body:
%    \begin{macrocode}
\begin{document}
%    \end{macrocode}

% Declare a title page.
% Print title, part of document being processed and version flag:
%    \begin{macrocode}
\addtocounter{page}{-1}
\begin{center}
{\LARGE\bfseries{}childdoc example\par}
\vspace{1cm}
\ifchilddoc
\ifchilddocmanual part\else chapter\fi:
`\childdocname' of `\childdocjob'\par
\else
main document: `\childdocjob'\par
\fi
version: \version\par
\end{center}
\newpage
%    \end{macrocode}

% Manually include selected file,
% otherwise process as usual:
%    \begin{macrocode}
\ifchilddocmanual
\section*{part `\childdocname'}
\input{\childdocname}
\else
%    \end{macrocode}

% Include the two chapters:
%    \begin{macrocode}
\include{cdocsch1}
\include{cdocsch2}
%    \end{macrocode}

% Include the two parts unless only chapters should be displayed:
%    \begin{macrocode}
\ifchilddoc\else
\section{part three}
\input{cdocspt3}
\section{part four}
\input{cdocspt4}
\fi
%    \end{macrocode}

% Process as usual until here:
%    \begin{macrocode}
\fi
%    \end{macrocode}

% End of document body:
%    \begin{macrocode}
\end{document}
%    \end{macrocode}
%\iffalse
%</samplemain>
%\fi
%
% %%%%%%%%%%%%%%%%%%%%%%%%%%%%%%%%%%%%%%
% \paragraph{Chapter Include Files.}
%
% The include files are called |cdocsch1.tex| and |cdocsch2.tex|.
%
%\iffalse
%<*samplechap1|samplechap2>
%\fi

% Optional override for |\version| flag:
%    \begin{macrocode}
%%\providecommand{\version}{final}
%    \end{macrocode}

% Include the main document:
%    \begin{macrocode}
\input{childdoc.def}
\childdocof{cdocsamp}
%    \end{macrocode}

%\iffalse
%</samplechap1|samplechap2>
%\fi
%
%\iffalse
%<*samplechap1>
%\fi
% Some text for chapter 1:
%    \begin{macrocode}
\section{one}
some text in chapter one
%    \end{macrocode}

%\iffalse
%</samplechap1>
%\fi
% Some text for chapter 2:
%\iffalse
%<*samplechap2>
%\fi
%    \begin{macrocode}
\section{two}
more text in chapter two
%    \end{macrocode}

%\iffalse
%</samplechap2>
%\fi
%
% %%%%%%%%%%%%%%%%%%%%%%%%%%%%%%%%%%%%%%
% \paragraph{Part Include Files.}
%
% The include files are called |cdocspt3.tex| and |cdocspt4.tex|.
%
%\iffalse
%<*samplepart3|samplepart4>
%\fi

% Optional override for |\version| flag:
%    \begin{macrocode}
%%\providecommand{\version}{final}
%    \end{macrocode}

% Include the main document:
%    \begin{macrocode}
\input{childdoc.def}
\childdocby{cdocsamp}
%    \end{macrocode}

%\iffalse
%</samplepart3|samplepart4>
%\fi
%
%\iffalse
%<*samplepart3>
%\fi
% Some text for part 3:
%    \begin{macrocode}
some text in part three
%    \end{macrocode}

%\iffalse
%</samplepart3>
%\fi
% Some text for part 4:
%\iffalse
%<*samplepart4>
%\fi
%    \begin{macrocode}
more text in part four
%    \end{macrocode}

%\iffalse
%</samplepart4>
%\fi
%
% %%%%%%%%%%%%%%%%%%%%%%%%%%%%%%%%%%%%%%
% \paragraph{Forwarding for a Complete Draft.}
%
% The following forwarding file |cdocsdrf.tex|
% compiles the main document in draft mode:
%\iffalse
%<*sampledraft>
%\fi
%    \begin{macrocode}
\def\version{draft}
\input{childdoc.def}
\childdocforward{cdocsamp}
%    \end{macrocode}

%\iffalse
%</sampledraft>
%\fi
%
% %%%%%%%%%%%%%%%%%%%%%%%%%%%%%%%%%%%%%%
% \paragraph{Forwarding for Final Version of the Chapters.}
%
% The following forwarding files |cdocsfn1.tex| and |cdocsfn2.tex|
% (with identical content)
% compile the final versions of the child documents
% |cdocsch1.tex| and |cdocsch2.tex|, respectively:
%\iffalse
%<*samplefinal>
%\fi
%    \begin{macrocode}
\def\version{final}
\input{childdoc.def}
\childdocforwardprefix[cdocsamp]{cdocsfn}{cdocsch}
%    \end{macrocode}

%\iffalse
%</samplefinal>
%\fi
%
% %%%%%%%%%%%%%%%%%%%%%%%%%%%%%%%%%%%%%%
% \paragraph{Command Line Processing.}
%
% The following three command lines generate the output files
% |cdocscld|, |cdocscl1| and |cdocscl2|
% which should be identical to
% |cdocsdrf|, |cdocsch1| and |cdocsfn2|, respectively:
% \begin{center}
% \begin{tabular}{l}
% |latex -jobname cdocscld \|\\
% |  "\def\version{draft}\input{childdoc.def}\childdocforward{cdocsamp}"|\\
% |latex -jobname cdocscl1 \|\\
% |  "\input{childdoc.def}\childdocforward[cdocsamp]{cdocsch1}"|\\
% |latex -jobname cdocscl2 \|\\
% |  "\def\version{final}\input{childdoc.def}\childdocforward{cdocsch2}"|
% \end{tabular}
% \end{center}
% Note that the trailing backslash on each first line
% merely continues the input to the second line
% (for convenient cut ant paste).
% Furthermore, the command |latex| can be replaced by any
% of its alternative versions such as |pdflatex|.
%
% %%%%%%%%%%%%%%%%%%%%%%%%%%%%%%%%%%%%%%%%%%%%%%%%%%%%%%%%%%%%%%%%%%%%%%%%%%%%%%
% %%%%%%%%%%%%%%%%%%%%%%%%%%%%%%%%%%%%%%%%%%%%%%%%%%%%%%%%%%%%%%%%%%%%%%%%%%%%%%
% \section{Implementation}
%\iffalse
%<*package>
%\fi
%
% This section describes the definitions file |childdoc.def|.

% The definitions cannot be loaded using |\usepackage| or |\RequirePackage|
% which has a mechanism to prevent loading a style file more than once.
% When loading the definitions by means of |\input|
% multiple instances have to be prevented manually:
%\iffalse
%This code needs to be before the `\ProvidesFile' directive
%which is defined at the beginning of this file.
%Therefore it is also placed there and commented out here.
%</package>
%<*discard>
%\fi
%    \begin{macrocode}
\ifdefined\childdocmain\endinput\fi
%    \end{macrocode}
%\iffalse
%</discard>
%<*package>
%\fi
%
% \macro{\ifchilddoc}
% \macro{\ifchilddocmanual}
% The conditional |\ifchilddoc| tells whether a
% child (true) or main (false) document is being compiled.
% The conditional |\ifchilddocmanual| tells whether
% the |\includeonly| mechanism is used (false) or
% the selection of child files must be performed manually (true).
% The definitions initialise to false:
%    \begin{macrocode}
\newif\ifchilddoc
\newif\ifchilddocmanual
%    \end{macrocode}

% \macro{\childdocname}
% \macro{\childdocjob}
% The macro |\childdocname| stores the name of the main document
% to be compiled. The macro |\childdocjob| stores the name of
% the document on which the \LaTeX{} compiler was originally invoked.
% The content of |\jobname| cannot be compared
% to filenames specified in the source due to different catcodes.
% The following code rescans |\jobname|, stores the result
% in |\childdocname| and saves a copy in |\childdocjob|:
%    \begin{macrocode}
\edef\childdocname{\scantokens\expandafter{\jobname\noexpand}}
\let\childdocjob\childdocname
%    \end{macrocode}

% \macro{\childdocdisable}
% The macro |\childdocdisable| prevents the main file
% from being processed more than once.
% At this stage, the main document command |\childdocmain|
% is assumed to be called once again where it should do nothing.
% Any subsequent call to it should prevent
% a secondary processing of the main document
% It overwrites the forwarding commands
% |\childdocof| and |\childdocforward|
% with empty macros to prevent further inclusions of the main document:
%    \begin{macrocode}
\newcommand{\childdocdisable}
{
  \renewcommand{\childdocmain}[1]{\renewcommand{\childdocmain}[1]{\endinput}}
  \renewcommand{\childdocof}[1]{}
  \renewcommand{\childdocby}[2][]{}
  \renewcommand{\childdocforward}[2][]{}
  \renewcommand{\childdocdisable}{}
}
%    \end{macrocode}

% \macro{\childdocmain}
% The macro |\childdocmain| is to be called at the top of the main file
% with nothing or the main filename (without extension) as argument.
% First, it breaks loops.
% If the argument is not empty and does not match |\childdocname|
% (which is set by the first inclusion of |childdoc.def|),
% |\ifchilddoc| is set to true, |\includeonly| is applied to the child file
% and |\jobname| is set to the main file
% (for proper handling of |.aux| files):
%    \begin{macrocode}
\newcommand{\childdocmain}[1]
{
  \childdocdisable\childdocmain{}
  \if?#1?\else
    \begingroup
      \def\childdoctmp{#1}
      \ifx\childdoctmp\childdocname
        \def\childdoctmp{}
      \else
        \def\childdoctmp
        {
          \childdoctrue
          \includeonly{\childdocname}
          \def\childdocjob{#1}
          \def\jobname{#1}
        }
      \fi
      \expandafter
    \endgroup
    \childdoctmp
  \fi
}
%    \end{macrocode}

% \macro{\childdocof}
% The command |\childdocof| redirects
% compilation to the main file |#1|.
%    \begin{macrocode}
\newcommand{\childdocof}[1]
{
  \childdocdisable
  \childdoctrue
  \includeonly{\childdocname}
  \def\jobname{#1}
  \def\childdocjob{#1}
  \input{#1}
}
%    \end{macrocode}

% \macro{\childdocby}
% The command |\childdocby| ....
%    \begin{macrocode}
\newcommand{\childdocby}[2][]
{
  \childdocdisable
  \childdoctrue
  \childdocmanualtrue
  \if?#1?\else
    \def\jobname{#2}
  \fi
  \def\childdocjob{#2}
  \input{#2}
  \endinput
}
%    \end{macrocode}

% \macro{\childdocforward}
% The command |\childdocforward| redirects
% compilation to the main file or
% (if the optional argument is given) a child file.
% Parameters are set as if the main file
% or a child file starting with |\childdocof| was compiled.
% Then compilation is handed over to the main file:
%    \begin{macrocode}
\newcommand{\childdocforward}[2][]
{
  \begingroup
    \if?#1?
      \def\childdoctmp
      {
        \def\childdocname{#2}
        \def\childdocjob{#2}
        \def\jobname{#2}
        \input{#2}
        \endinput
      }
    \else
      \def\childdoctmp
      {
        \childdocdisable
        \def\childdocname{#2}
        \childdoctrue
        \includeonly{#2}
        \def\childdocjob{#1}
        \def\jobname{#1}
        \input{#1}
        \endinput
      }
    \fi
    \expandafter
  \endgroup
  \childdoctmp
}
%    \end{macrocode}

% \macro{\childdocforwardprefix}
% The command |\childdocforwardprefix| redirects
% compilation to the main or a child file by means of a pattern.
% The prefix |#1| in the current filename is replaced by |#2|
% and the suffix of the current filename is kept
% (it is assumed that the filename does not contain the substring `|~~~|'
% which is used as a delimiter).
% Compilation is handed over to the new file by |\childdocforward|:
%    \begin{macrocode}
\newcommand{\childdocforwardprefix}[3][]
{
  \begingroup
    \def\childdocextract #2##1~~~{\def\childdoctmp{\childdocforward[#1]{#3##1}}}
    \expandafter\childdocextract\childdocname~~~
    \expandafter
  \endgroup
  \childdoctmp
}
%    \end{macrocode}

% \macro{\childdoc}
% The deprecated macro |\childdoc| is a legacy version of |\childdocmain|:
%    \begin{macrocode}
\newcommand{\childdoc}{\childdocmain}
%    \end{macrocode}

% \macro{\childdocredirect}
% The deprecated macro |\childdocredirect| is a legacy version
% of |\childdocforward| and |\childdocforwardprefix|:
%    \begin{macrocode}
\newcommand{\childdocredirect}[2][]
{
  \begingroup
    \if?#1?
      \def\childdoctmp{\childdocforward{#2}}
    \else
      \def\childdoctmp{\childdocforwardprefix{#1}{#2}}
    \fi
    \expandafter
  \endgroup
  \childdoctmp
}
%    \end{macrocode}

%\iffalse
%</package>
%\fi
%
\endinput
|
and perform the replacements as outlined below.
Instead of |\childdocmain{|\textit{main}|}| add the following code
to the top of the main file:
%
\begin{center}
\begin{tabular}{l}
|\||ifdefined\childdocname\endinput\||fi\newif\ifchilddoc|\\
|\edef\childdocname{\scantokens\expandafter{\jobname\noexpand}}|\\
|\def\childdocmain{|\textit{main}|}\||ifx\childdocmain\childdocname\||else|\\
|\childdoctrue\includeonly{\childdocname}\let\jobname\childdocmain\||fi|\\
\end{tabular}
\end{center}
%
Instead of |\childdocof{|\textit{main}|}| just include the main file
at the top of each child file:
%
\begin{center}
|\input{|\textit{main}|}|
\end{center}
%
A simple redirection |\childdocforward{|\textit{dest}|}| is achieved by:
%
\begin{center}
|\def\jobname{|\textit{dest}|}\input{\jobname}|
\end{center}
%
The redirection with prefix
|\childdocforwardprefix[|\textit{prefix}|]{|\textit{dest}|}|
is accomplished by:
%
\begin{center}
\begin{tabular}{l}
|{\edef\jobname{\scantokens\expandafter{\jobname\noexpand}}|\\
|\def\redirectjob |\textit{prefix}|#1~~~{\gdef\jobname{|\textit{dest}|#1}}|\\
|\expandafter\redirectjob\jobname~~~}\input{\jobname}|
\end{tabular}
\end{center}

In an alternative approach,
child documents can be compiled by a specific command line
without additional code or specific definitions:
%
\begin{center}
|... -jobname "|\textit{target}|" "|[\textit{flags}]%
|\includeonly{|\textit{dest}|}\input{|\textit{main}|}"|
\end{center}
%

%%%%%%%%%%%%%%%%%%%%%%%%%%%%%%%%%%%%%%%%%%%%%%%%%%%%%%%%%%%%%%%%%%%%%%%%%%%%%%%%
%%%%%%%%%%%%%%%%%%%%%%%%%%%%%%%%%%%%%%%%%%%%%%%%%%%%%%%%%%%%%%%%%%%%%%%%%%%%%%%%
\section{Information}

%%%%%%%%%%%%%%%%%%%%%%%%%%%%%%%%%%%%%%%%%%%%%%%%%%%%%%%%%%%%%%%%%%%%%%%%%%%%%%%%
\subsection{Copyright}

Copyright \copyright{} 2017--2018 Niklas Beisert

This work may be distributed and/or modified under the
conditions of the \LaTeX{} Project Public License, either version 1.3
of this license or (at your option) any later version.
The latest version of this license is in
  \url{http://www.latex-project.org/lppl.txt}
and version 1.3 or later is part of all distributions of \LaTeX{}
version 2005/12/01 or later.

This work has the LPPL maintenance status `maintained'.

The Current Maintainer of this work is Niklas Beisert.

This work consists of the files |README.txt|, |childdoc.ins| and |childdoc.dtx|
as well as the derived files |childdoc.def|, |cdocsamp.tex|
with |cdocsch1.tex|, |cdocsch2.tex|, |cdocspt3.tex|, |cdocspt4.tex|,
|cdocsdrf.tex|, |cdocsfn1.tex|, |cdocsfn2.tex|
as well as |childdoc.pdf|.

%%%%%%%%%%%%%%%%%%%%%%%%%%%%%%%%%%%%%%%%%%%%%%%%%%%%%%%%%%%%%%%%%%%%%%%%%%%%%%%%
\subsection{Files and Installation}

The package consists of the files:
%
\begin{center}
\begin{tabular}{ll}
    |README.txt|   & readme file \\
    |childdoc.ins| & installation file \\
    |childdoc.dtx| & source file \\
    |childdoc.def| & definition file \\
    |cdocsamp.tex| & sample main file \\
    |cdocsch1.tex| & sample include file \\
    |cdocsch2.tex| & sample include file \\
    |cdocspt3.tex| & sample part file \\
    |cdocspt4.tex| & sample part file \\
    |cdocsdrf.tex| & sample redirection file \\
    |cdocsfn1.tex| & sample redirection file \\
    |cdocsfn2.tex| & sample redirection file \\
    |childdoc.pdf| & manual
\end{tabular}
\end{center}
%
The distribution consists of the files
|README.txt|, |childdoc.ins| and |childdoc.dtx|.
%
\begin{itemize}
\item
Run (pdf)\LaTeX{} on |childdoc.dtx|
to compile the manual |childdoc.pdf| (this file).
\item
Run \LaTeX{} on |childdoc.ins| to create the definitions file |childdoc.def|
and the sample |cdocsamp.tex| with include files
|cdocsch1.tex|, |cdocsch2.tex|, |cdocspt3.tex|, |cdocspt4.tex|,
|cdocsdrf.tex|, |cdocsfn1.tex|, |cdocsfn2.tex|.
Then copy the file |childdoc.def| to an appropriate directory of your \LaTeX{}
distribution, e.g.\ \textit{texmf-root}|/tex/latex/childdoc|.
\end{itemize}

%%%%%%%%%%%%%%%%%%%%%%%%%%%%%%%%%%%%%%%%%%%%%%%%%%%%%%%%%%%%%%%%%%%%%%%%%%%%%%%%
\subsection{Related CTAN Packages}

There are several other packages which offer a similar functionality:
%
\begin{itemize}
\item
The packages
\href{http://ctan.org/pkg/docmute}{\textsf{docmute}},
\href{http://ctan.org/pkg/includex}{\textsf{includex}} and
\href{http://ctan.org/pkg/standalone}{\textsf{standalone}}
provide commands to include only the document body of
a child file thus allowing both files to be compiled individually.
\item
The packages \href{http://ctan.org/pkg/subdocs}{\textsf{subdocs}}
and \href{http://ctan.org/pkg/subfiles}{\textsf{subfiles}}
provide structures in which the main and child documents can be
encapsulated and allowing them to be compiled individually.
The inclusion mechanism is different from the conventional |\include|.
\item
The package \href{http://ctan.org/pkg/combine}{\textsf{combine}}
is an elaborate solution to combine several documents into one.
\end{itemize}
%
See also the CTAN topic \href{http://ctan.org/topic/subdocs}{\textsf{subdocs}}
for further related packages.
The present package differs from the above solutions in that
a document structure constructed with the conventional |\include| mechanism
just needs two extra commands at the top of every file
such that all constituent files can be compiled individually.

%%%%%%%%%%%%%%%%%%%%%%%%%%%%%%%%%%%%%%%%%%%%%%%%%%%%%%%%%%%%%%%%%%%%%%%%%%%%%%%%
%\subsection{Feature Suggestions}
%
%The following is a list of features which may be useful for future
%versions of this package:
%%
%\begin{itemize}
%\item
%\ldots
%\end{itemize}

%%%%%%%%%%%%%%%%%%%%%%%%%%%%%%%%%%%%%%%%%%%%%%%%%%%%%%%%%%%%%%%%%%%%%%%%%%%%%%%%
\subsection{Revision History}

%%%%%%%%%%%%%%%%%%%%%%%%%%%%%%%%%%%%%%%%
\paragraph{v2.0:} 2018/12/30

\begin{itemize}
\item
immediate forward processing
\item
added |\childdocby| mechanism
\item
manual restructured
\end{itemize}

%%%%%%%%%%%%%%%%%%%%%%%%%%%%%%%%%%%%%%%%
\paragraph{v1.6:} 2018/01/17

\begin{itemize}
\item
application for development of include files
\item
corrections to manual
\end{itemize}

%%%%%%%%%%%%%%%%%%%%%%%%%%%%%%%%%%%%%%%%
\paragraph{v1.5:} 2017/05/21

\begin{itemize}
\item
more complete structuring introduced
\item
|\childdocof| introduced
\item
|\childdoc| renamed to |\childdocmain|
\item
|\childredirect| renamed to |\childdocforward| and |\childdocforwardprefix|
and functionality expanded
\end{itemize}

%%%%%%%%%%%%%%%%%%%%%%%%%%%%%%%%%%%%%%%%
\paragraph{v1.0:} 2017/04/27

\begin{itemize}
\item
manual and install package
\item
first version published on CTAN
\end{itemize}

%%%%%%%%%%%%%%%%%%%%%%%%%%%%%%%%%%%%%%%%
\paragraph{v0.6:} 2017/04/26

\begin{itemize}
\item
redirection mechanism added
\end{itemize}

%%%%%%%%%%%%%%%%%%%%%%%%%%%%%%%%%%%%%%%%
\paragraph{v0.5:} 2017/04/26

\begin{itemize}
\item
functionality in definition file
\end{itemize}


%%%%%%%%%%%%%%%%%%%%%%%%%%%%%%%%%%%%%%%%%%%%%%%%%%%%%%%%%%%%%%%%%%%%%%%%%%%%%%%%
%%%%%%%%%%%%%%%%%%%%%%%%%%%%%%%%%%%%%%%%%%%%%%%%%%%%%%%%%%%%%%%%%%%%%%%%%%%%%%%%
%%%%%%%%%%%%%%%%%%%%%%%%%%%%%%%%%%%%%%%%%%%%%%%%%%%%%%%%%%%%%%%%%%%%%%%%%%%%%%%%
\appendix

\settowidth\MacroIndent{\rmfamily\scriptsize 000\ }

 \DocInput{childdoc.dtx}

\end{document}
%</driver>
% \fi
%
% %%%%%%%%%%%%%%%%%%%%%%%%%%%%%%%%%%%%%%%%%%%%%%%%%%%%%%%%%%%%%%%%%%%%%%%%%%%%%%
% %%%%%%%%%%%%%%%%%%%%%%%%%%%%%%%%%%%%%%%%%%%%%%%%%%%%%%%%%%%%%%%%%%%%%%%%%%%%%%
% \section{Sample}
%\iffalse
%<*samplemain>
%\fi
%
% The following presents a sample document
% with two chapters, two parts, a title page,
% a compile flag as well as three forwarding files to set the flag.
% It consists of eight |.tex| files:
% \begin{center}
% \begin{tabular}{ll}
% |cdocsamp.tex|&main file\\
% |cdocsch1.tex|&include file for chapter 1\\
% |cdocsch2.tex|&include file for chapter 2\\
% |cdocspt3.tex|&include file for part 3\\
% |cdocspt4.tex|&include file for part 4\\
% |cdocsdrf.tex|&forwarding file for main file in draft mode\\
% |cdocsfi1.tex|&forwarding file for final version of chapter 1\\
% |cdocsfi2.tex|&forwarding file for final version of chapter 2\\
% \end{tabular}
% \end{center}
% Each of the eight files can be compiled directly by the \LaTeX{} compiler.
%
% %%%%%%%%%%%%%%%%%%%%%%%%%%%%%%%%%%%%%%
% \paragraph{Main File.}
%
% The main file is called |cdocsamp.tex|.
%
% Load the \textsf{childdoc} definitions and
% declare the filename for the main document:
%    \begin{macrocode}
% \iffalse
%
% childdoc.dtx Copyright (C) 2017-2018 Niklas Beisert
%
% This work may be distributed and/or modified under the
% conditions of the LaTeX Project Public License, either version 1.3
% of this license or (at your option) any later version.
% The latest version of this license is in
%   http://www.latex-project.org/lppl.txt
% and version 1.3 or later is part of all distributions of LaTeX
% version 2005/12/01 or later.
%
% This work has the LPPL maintenance status `maintained'.
%
% The Current Maintainer of this work is Niklas Beisert.
%
% This work consists of the files childdoc.dtx and childdoc.ins
% and the derived files childdoc.def and cdocsamp.tex with
% cdocsch1.tex, cdocsch2.tex, cdocsdrf.tex, cdocsfn1.tex, cdocsfn2.tex.
%
%<package>\ifdefined\childdocmain\endinput\fi
%<package>\ProvidesFile{childdoc.def}[2018/12/30 v2.0 child document driver]
%<samplemain>\ProvidesFile{cdocsamp.tex}[2018/12/30 v2.0 sample for childdoc]
%<*driver>
%\ProvidesFile{childdoc.drv}[2018/12/30 v2.0 childdoc reference manual file]
\PassOptionsToClass{10pt,a4paper}{article}
\documentclass{ltxdoc}

\usepackage[margin=35mm]{geometry}
\usepackage{hyperref}
\usepackage{hyperxmp}
\usepackage[usenames]{color}

\hypersetup{colorlinks=true}
\hypersetup{pdfstartview=FitH}
\hypersetup{pdfpagemode=UseNone}
\hypersetup{pdfsource={}}
\hypersetup{pdflang={en-UK}}
\hypersetup{pdfcopyright={Copyright 2017-2018 Niklas Beisert.
  This work may be distributed and/or modified under the
  conditions of the LaTeX Project Public License, either version 1.3
  of this license or (at your option) any later version.}}
\hypersetup{pdflicenseurl={http://www.latex-project.org/lppl.txt}}
\hypersetup{pdfcontactaddress={ETH Zurich, ITP, HIT K,
  Wolfgang-Pauli-Strasse 27}}
\hypersetup{pdfcontactpostcode={8093}}
\hypersetup{pdfcontactcity={Zurich}}
\hypersetup{pdfcontactcountry={Switzerland}}
\hypersetup{pdfcontactemail={nbeisert@itp.phys.ethz.ch}}
\hypersetup{pdfcontacturl={http://people.phys.ethz.ch/\xmptilde nbeisert/}}

\newcommand{\secref}[1]{\hyperref[#1]{section \ref*{#1}}}

\parskip1ex
\parindent0pt
\let\olditemize\itemize
\def\itemize{\olditemize\parskip0pt}

\begin{document}

\title{The \textsf{childdoc} Package}
\hypersetup{pdftitle={The childdoc Package}}
\author{Niklas Beisert\\[2ex]
  Institut f\"ur Theoretische Physik\\
  Eidgen\"ossische Technische Hochschule Z\"urich\\
  Wolfgang-Pauli-Strasse 27, 8093 Z\"urich, Switzerland\\[1ex]
  \href{mailto:nbeisert@itp.phys.ethz.ch}
  {\texttt{nbeisert@itp.phys.ethz.ch}}}
\hypersetup{pdfauthor={Niklas Beisert}}
\hypersetup{pdfsubject={Manual for the LaTeX2e Package childdoc}}
\date{30 December 2018, \textsf{v2.0}}
\maketitle

\begin{abstract}\noindent
\textsf{childdoc} is a \LaTeXe{} package
that enables the direct compilation
of document sections included by |\include|
to individual files.
\end{abstract}

\begingroup
\parskip0ex
\tableofcontents
\endgroup

%%%%%%%%%%%%%%%%%%%%%%%%%%%%%%%%%%%%%%%%%%%%%%%%%%%%%%%%%%%%%%%%%%%%%%%%%%%%%%%%
%%%%%%%%%%%%%%%%%%%%%%%%%%%%%%%%%%%%%%%%%%%%%%%%%%%%%%%%%%%%%%%%%%%%%%%%%%%%%%%%
\section{Introduction}

\LaTeX{} provides a mechanism to structure a large document (such as a book)
into a main file and several child files (containing the chapters)
using the |\include| command.
This mechanism is beneficial for documents
which span hundreds of pages in order to
make the source file(s) more manageable.
Moreover, compilation can be restricted to
selected child files by means of the |\includeonly| command.
The latter feature can be used to reduce the compilation time while editing
(this was significantly more useful in the earlier days of \LaTeX{})
or to generate a smaller document which is easier to navigate.
Another application of |\includeonly| is to generate
documents consisting of selected parts of the complete document.

However, there are a few drawbacks of the plain |\include| mechanism:
\begin{itemize}
\item
The child files cannot be compiled on their own,
they can only be compiled via the main file.
A naive editing environment
(such as a text editor with an option
to have the current file processed by \LaTeX)
may require one to switch to the main file before compiling;
attempting to compile the child file produces errors.
\item
The main file must be modified (each time)
to adjust the |\includeonly| command
to the present needs. This easily leaves the main file in a messy state.
\item
The generated document will always carry the filename
of the main document. This is inconvenient if
several child files are to be compiled and
to be kept for distribution.
\end{itemize}

The present package provides a simple interface
to make child files individually compilable by \LaTeX{}.
Compiling a child file then has the same effect as compiling
the main file with an |\includeonly| command
to select the appropriate child.
Moreover the generated document will carry the name of the child
rather than the main file.
This resolves all three above issues.

This feature is meant to make the editing of books,
thesis documents and lecture notes somewhat more convenient.
However, the package can also be used efficiently for
composing a series of documents (such as exercise sheets)
which are typically distributed individually.
It then assists the author in generating the individual documents
(potentially in different versions)
as well as a document containing the collected series.
Another application is in developing style files
or other kinds of included material
where compilation of the style file could redirect
to a sample or test file.

%%%%%%%%%%%%%%%%%%%%%%%%%%%%%%%%%%%%%%%%%%%%%%%%%%%%%%%%%%%%%%%%%%%%%%%%%%%%%%%%
%%%%%%%%%%%%%%%%%%%%%%%%%%%%%%%%%%%%%%%%%%%%%%%%%%%%%%%%%%%%%%%%%%%%%%%%%%%%%%%%
\section{Usage}

First of all, the package \textsf{childdoc} is \emph{not} a standard
\LaTeXe{} |.sty| style file! Therefore it needs to be invoked in
a non-standard way.

%%%%%%%%%%%%%%%%%%%%%%%%%%%%%%%%%%%%%%%%%%%%%%%%%%%%%%%%%%%%%%%%%%%%%%%%%%%%%%%%
\subsection{Included Files}
\label{sec:include}

%%%%%%%%%%%%%%%%%%%%%%%%%%%%%%%%%%%%%%%%
\DescribeMacro{\childdocmain}
To use the package, add the commands
\begin{center}
\begin{tabular}{l}
|\input{childdoc.def}|\\
|\childdocmain{}|\\
\end{tabular}
\end{center}
at the very top of the main \LaTeX{} file,
in particular \emph{before} the |\documentclass| statement!
The argument of |\childdocmain| should be left empty
(but it must be present).

%%%%%%%%%%%%%%%%%%%%%%%%%%%%%%%%%%%%%%%%
\DescribeMacro{\childdocof}
Furthermore, add the commands
\begin{center}
\begin{tabular}{l}
|\input{childdoc.def}|\\
|\childdocof{|\textit{main}|}|\\
\end{tabular}
\end{center}
at the top of every child file \textit{child}
which is included by |\include{|\textit{child}|}|
from within the main file
(or at least for those files to be compiled individually).
The argument \textit{main} must be the filename of the main file.

There are a couple of
considerations in setting up the main and child documents:

%%%%%%%%%%%%%%%%%%%%%%%%%%%%%%%%%%%%%%%%
\paragraph{Restrictions.}

Please note the following restrictions:
\begin{itemize}
\item
|\childdocmain| must be called with one argument \textit{main}
to ensure compatibility with earlier version of the package.
It must either be empty (|\childdocmain{}|)
or precisely match the filename of the main file in which it is specified.
See \secref{sec:detection} for further information.
\item
The filename \textit{main} must be specified without the |.tex| extension.
\item
The filename \textit{main} is case sensitive
(even in case-insensitive file systems)
due to internal string comparison.
\item
The argument \textit{main} should be fully expanded, it cannot be a macro.
\item
Subdirectories and special characters should be avoided in filenames.
\item
The command |\childdocmain{|\textit{main}|}| must be followed by a whitespace.
It should not be followed immediately by another command
or by a comment mark `|%|'.
This is because the \TeX{} parser reads the token immediately following
the argument of |\childdocmain| and puts it
at the beginning of every child section;
however, a white\-space is ignored.
\end{itemize}

%%%%%%%%%%%%%%%%%%%%%%%%%%%%%%%%%%%%%%%%
\paragraph{Content of Main File.}

It is advisable to place all content in the child files included by |\include|.
Any output contained in the main file will appear in all child documents
unless suppressed manually;
it cannot be suppressed automatically by the |\includeonly| directive
and thus should normally be avoided.
A method to include some content in the main file
by means of conditional processing is described in \secref{sec:conditional}.

%%%%%%%%%%%%%%%%%%%%%%%%%%%%%%%%%%%%%%%%
\paragraph{Page Numbering.}

When only a part of the document is compiled,
the appropriate numbering of pages
(as well as other status parameters)
is determined from the |.aux| files.
The latter contain information from previous passes.
However this information needs to propagate through
all intermediate child documents.
Therefore the page numbering in child documents may well
be inconsistent until the complete document is compiled at least once.

A useful (if unconventional) way to always ensure a consistent
page numbering is to restart the numbering in each child document
and denote the pages by `\textit{child}|.|\textit{page}'
where \textit{child} represents the chapter/section number of the child file.
This can be achieved by the command
|\numberwithin{page}{|\textit{child}|}|
of the \textsf{amsmath} package
where \textit{child} can be |chapter| or |section|
depending on the chosen structuring.
Alternatively, one can modify the macro |\thepage| appropriately
and reset the counter |page| at the start of each child file.

%%%%%%%%%%%%%%%%%%%%%%%%%%%%%%%%%%%%%%%%%%%%%%%%%%%%%%%%%%%%%%%%%%%%%%%%%%%%%%%%
\subsection{Conditional Processing}
\label{sec:conditional}

The package provides a mechanism to compile different versions
of a document. To customise the versions further some conditional processing
can come in handy to distinguish which version is being compiled.
The package provides two macros to describe the compilation context:

%%%%%%%%%%%%%%%%%%%%%%%%%%%%%%%%%%%%%%%%
\DescribeMacro{\ifchilddoc}
The conditional |\ifchilddoc| distinguishes between the compilation of
child documents and the main document:
%
\begin{center}
|\ifchilddoc |\textit{child-code}| |[|\||else |\textit{main-code}]| \||fi|
\end{center}

%%%%%%%%%%%%%%%%%%%%%%%%%%%%%%%%%%%%%%%%
\DescribeMacro{\childdocname}
\DescribeMacro{\childdocjob}
The macro |\childdocname| contains the filename (without extension)
of the main or child file being processed.
Note that |\childdocjob| will always contain the name of the main file.

%%%%%%%%%%%%%%%%%%%%%%%%%%%%%%%%%%%%%%%%
\paragraph{Title Page.}

Conditional processing can be used to include a title or banner page
in the main document when proper precautions are taken.
Importantly, the code in the main file should ensure that the page counter
(as well as other status parameters which are stored in the |.aux| files)
takes the same value after the conditional processing.
Otherwise the page numbers may take divergent values
depending on which part is compiled.

For example, a title page could be declared by:
%
\begin{center}
\begin{tabular}{l}
|\ifchilddoc\||else|\\
|\addtocounter{page}{-1}|\\
\textit{code for title page}\\
|\newpage|\\
|\||fi|
\end{tabular}
\end{center}
%
A banner page for the child documents can be generated by:
%
\begin{center}
\begin{tabular}{l}
|\ifchilddoc|\\
|\addtocounter{page}{-1}|\\
\textit{code for banner page}\\
|\newpage|\\
|\||fi|
\end{tabular}
\end{center}
%
Here one could write a message such as:
\begin{center}
|This is the part \childdocname{} of \childdocjob{}.|
\end{center}

%%%%%%%%%%%%%%%%%%%%%%%%%%%%%%%%%%%%%%%%%%%%%%%%%%%%%%%%%%%%%%%%%%%%%%%%%%%%%%%%
\subsection{Flags}
\label{sec:flags}

The package makes it easy to generate different versions
of the main or child documents.
To this end compilation flags can be defined
and assigned different default values.
They will be particularly useful in conjunction
with the forwarding mechanism described in \secref{sec:forward}.

For example, it may be useful to have a flag |\version|
which can be set to |draft| or |final|.
The document source will contain some conditional code
depending on the value of |\version|.
Suppose further, the flag should default to |final| for the main file
and to |draft| for child files
which is a natural assignment for editing the document.
This is achieved by placing the following code
in the preamble of the main document
(below the |\childdocmain| directive):
%
\begin{center}
\begin{tabular}{l}
|\ifchilddoc|\\
|\providecommand{\version}{draft}|\\
|\||else|\\
|\providecommand{\version}{final}|\\
|\||fi|
\end{tabular}
\end{center}
%
The definition by |\providecommand| makes sure
that previous definitions are not overwritten.
Further statements |\providecommand{\version}{...}|
can thus be added before the above code to override it.

For the main file, one might add a line
(between |\childdocmain| and the above block)
%
\begin{center}
|%\ifchilddoc\||else\providecommand{\version}{draft}\||fi|
\end{center}
%
which can be uncommented to produce a draft version.
Likewise one can add a line to the very top of a child file
(above the |\childdocof{|\textit{main}|}| directive)
%
\begin{center}
|%\providecommand{\version}{final}|
\end{center}
%
which can be uncommented to produce the final version of this child document.

%%%%%%%%%%%%%%%%%%%%%%%%%%%%%%%%%%%%%%%%%%%%%%%%%%%%%%%%%%%%%%%%%%%%%%%%%%%%%%%%
\subsection{Forwarding}
\label{sec:forward}

Different versions of the main or child documents
using compilation flags as described in \secref{sec:flags}
can be (permanently) stored in different files
for convenient compilation, viewing and distribution.
To this end, the package defines a command
to pass on compilation to a different file:

%%%%%%%%%%%%%%%%%%%%%%%%%%%%%%%%%%%%%%%%
\DescribeMacro{\childdocforward}
The command |\childdocforward| redirects processing to
another source file:
%
\begin{center}
\begin{tabular}{l}
|\input{childdoc.def}|\\
|\childdocforward[|\textit{main}|]{|\textit{dest}|}|\\
\end{tabular}
\end{center}
%
The argument \textit{dest} is the destination file
(without extension).
It should be the main file or one of the child files.
Note that further \textsf{childdoc} directives
such as |\childdocof| and |\childdocforward|
in the indicated file will be processed in this form.
The optional argument \textit{main}
passes on directly to the main file \textit{main}
while pretending to compile the child \textit{dest}.
This form behaves as if \textit{dest}
issues |\childdocof{|\textit{main}|}| right away,
and no further \textsf{childdoc} directives will be processed.

%%%%%%%%%%%%%%%%%%%%%%%%%%%%%%%%%%%%%%%%
\DescribeMacro{\...prefix}
In the alternative form |\childdocforwardprefix|,
%
\begin{center}
\begin{tabular}{l}
|\input{childdoc.def}|\\
|\childdocforwardprefix[|\textit{main}|]{|\textit{prefix}|}{|\textit{dest}|}|
\end{tabular}
\end{center}
%
the destination file is determined by a pattern
depending on the current file:
To make this work, the current file must be called
`{\textit{prefix}\hspace{0.2em}\textit{suffix}}'
with \textit{prefix} matching precisely the argument.
Processing is then passed on to the file
`{\textit{dest}\hspace{0.2em}\textit{suffix}}'.
Surely, the same effect is achieved by
directly specifying the
argument `{\textit{dest}\hspace{0.2em}\textit{suffix}}'
in the first form.
However, that requires to set up a different file
for each child. With the alternative form of the command
all these files can have exactly the same content
which simplifies setting them up and maintaining them.

For example, the following file |draft.tex|
with a compilation flag |\version| as described in \secref{sec:flags}
compiles the main document as a draft:
%
\begin{center}
\begin{tabular}{l}
|\def\version{draft}|\\
|\input{childdoc.def}|\\
|\childdocforward{|\textit{main}|}|
\end{tabular}
\end{center}
%
Likewise, the following files |final|\textit{nn}|.tex|
compile the final version of the child document
|child|\textit{nn}|.tex|:
%
\begin{center}
\begin{tabular}{l}
|\def\version{final}|\\
|\input{childdoc.def}|\\
|\childdocforwardprefix{final}{child}|
\end{tabular}
\end{center}
%

Note that when several versions of a main file and/or of each child file
are to be generated, it may be convenient to set up a |Makefile| or
shell script to automatise the process.

%%%%%%%%%%%%%%%%%%%%%%%%%%%%%%%%%%%%%%%%%%%%%%%%%%%%%%%%%%%%%%%%%%%%%%%%%%%%%%%%
\subsection{Command Line Processing}
\label{sec:commandline}

The effect of redirection files can also be achieved by invoking
the \LaTeX{} compiler with a more elaborate command line.
Most conveniently this should be done as part
of a shell script or a |Makefile|.

When using \textsf{childdoc} in the main file, the following
command lines effectively perform a redirection
(note that depending on the shell being used,
backslashes may have to be doubled: `|\|' $\to$ `|\\|'):
%
\begin{center}
|... -jobname "|\textit{target}|" |\\|"|[\textit{flags}]%
|\input{childdoc.def}\childdocforward[|\textit{main}|]{|\textit{dest}|}"|
\end{center}
%
Here \textit{target} is the name of the output file,
\textit{main} is the name of the main file
and \textit{dest} is the name of the main or child file to be processed
(all filenames without extensions).
The optional argument \textit{main} can be omitted
if \textit{main} matches \textit{dest}.
Optionally, compilation \textit{flags} can be defined via |\def| commands.
This command line makes the \TeX{} engine believe
it is compiling the file \textit{target}
whose content is specified as the latter parameter.
The provided code then forwards the processing to
\textit{main} or \textit{dest} as described in \secref{sec:forward}.

%%%%%%%%%%%%%%%%%%%%%%%%%%%%%%%%%%%%%%%%%%%%%%%%%%%%%%%%%%%%%%%%%%%%%%%%%%%%%%%%
\subsection{Include by Input}
\label{sec:input}

Including child documents by |\include| has some restrictions by design.
Most notably, the content of a child document always occupies
its own set of pages; pages cannot be shared between child documents.
Usually, this behaviour makes perfect sense
because each child document contain an essential part of the document.
However, in some situations it may be desirable to compose
a document from a collection of parts
without having mandatory page breaks between then.
For this case, the package
provides a mechanism to include parts
by |\input| which can also be processed individually.
However, by construction this mechanism
requires manual handling of the content to be output.

%%%%%%%%%%%%%%%%%%%%%%%%%%%%%%%%%%%%%%%%
\DescribeMacro{\ifchilddocmanual}
The main file should be prepared as usual, see \secref{sec:include}.
However, the document body must make a distinction
between processing of an individual part and of the main document, e.g.:
%
\begin{center}
\begin{tabular}{l}
|\ifchilddocmanual|\\
|\input{\childdocname}|\\
|\||else|\\
\textit{document body with }|\input{|\textit{part}|}|\\
|\||fi|
\end{tabular}
\end{center}
%
The conditional |\ifchilddocmanual| is true whenever
a part to be included by |\input| is being compiled,
and the name of the part is stored in |\childdocname|.

%%%%%%%%%%%%%%%%%%%%%%%%%%%%%%%%%%%%%%%%
\DescribeMacro{\childdocby}
Each part to be included by |\input| should start with:
%
\begin{center}
\begin{tabular}{l}
|\input{childdoc.def}|\\
|\childdocby{|\textit{main}|}|\\
\end{tabular}
\end{center}
%
The directive |\childdocby| is similar to |\childdocof|
described in \secref{sec:include},
but the subsequent selection of content must be done manually.
To that end, both |\ifchilddoc| and |\ifchilddocmanual|
will be true upon processing of a part,
and the name of the part is stored in |\childdocname|.
Note that |\jobname| will be set to the filename of the current part
so that each part receives an individual |.aux| file
that does not interfere with the |.aux| file(s) of the main document.
This behaviour can be altered by the alternative form
|\childdocby[*]{|\textit{main}|}| (with a non-empty optional argument)
which uses the |.aux| file of the main document
by setting |\jobname| to \textit{main}.

%%%%%%%%%%%%%%%%%%%%%%%%%%%%%%%%%%%%%%%%%%%%%%%%%%%%%%%%%%%%%%%%%%%%%%%%%%%%%%%%
\subsection{Driver Development}
\label{sec:driver}

The \textsf{childdoc} mechanism can also be use for the development
of definition files such as \LaTeX{} styles or classes.
This case differs from the above setup with multiple parts
included by |\include| in that no |\includeonly| should be invoked.
This can be achieved by starting the include file
(before |\ProvidesPackage|) with:
%
\begin{center}
\begin{tabular}{l}
|\input{childdoc.def}|\\
|\childdocforward{|\textit{main}|}|\\
\end{tabular}
\end{center}
%
or alternatively with:
%
\begin{center}
\begin{tabular}{l}
|\input{childdoc.def}|\\
|\childdocby{|\textit{main}|}|\\
\end{tabular}
\end{center}
%
Both forms have slightly different effects as described above.
The main file is prepared as usual, see \secref{sec:include}.

%%%%%%%%%%%%%%%%%%%%%%%%%%%%%%%%%%%%%%%%%%%%%%%%%%%%%%%%%%%%%%%%%%%%%%%%%%%%%%%%
\subsection{Legacy Detection}
\label{sec:detection}

The directive |\childdocmain| in the main file can detect
whether the complete document or merely a child is to be compiled
even without using the directive |\childdocof|.
This method is deprecated because it is less robust
and there is no compelling reason to use it;
it is merely provided for backward compatibility
and it may be removed in future versions.

If the detection mechanism is to be used,
it is mandatory to correctly specify
the filename of the main file as the argument of |\childdocmain|:
%
\begin{center}
\begin{tabular}{l}
|\input{childdoc.def}|\\
|\childdocmain{|\textit{main}|}|\\
\end{tabular}
\end{center}
%
If |\jobname| does not match the argument \textit{main} of |\childdocmain|,
it is assumed that |\jobname| points to the child file to be compiled.
When using |\childdocmain| with the main file specified as argument,
it suffices to start a child file
with just |\input{|\textit{main}|}|
without loading of the package and using |\childdocof|.
If instead all processing is done
with the appropriate \textsf{childdoc} directives,
the argument of \textit{main} of |\childdocmain| can be empty.

An alternative version of the command line processing described
in \secref{sec:commandline} using the detection mechanism reads:
%
\begin{center}
|... -jobname "|\textit{target}|" "|[\textit{flags}]%
[|\def\jobname{|\textit{dest}|}|]|\input{|\textit{main}|}"|
\end{center}

%%%%%%%%%%%%%%%%%%%%%%%%%%%%%%%%%%%%%%%%%%%%%%%%%%%%%%%%%%%%%%%%%%%%%%%%%%%%%%%%
\subsection{Manual Code}
\label{sec:manual}

In case one cannot be certain whether the definitions file |childdoc.def|
is installed on the target \TeX{} distribution
and one prefers not to ship it,
it is conceivable to paste a few relevant commands into the sources.

To that end, drop all statements |\input{childdoc.def}|
and perform the replacements as outlined below.
Instead of |\childdocmain{|\textit{main}|}| add the following code
to the top of the main file:
%
\begin{center}
\begin{tabular}{l}
|\||ifdefined\childdocname\endinput\||fi\newif\ifchilddoc|\\
|\edef\childdocname{\scantokens\expandafter{\jobname\noexpand}}|\\
|\def\childdocmain{|\textit{main}|}\||ifx\childdocmain\childdocname\||else|\\
|\childdoctrue\includeonly{\childdocname}\let\jobname\childdocmain\||fi|\\
\end{tabular}
\end{center}
%
Instead of |\childdocof{|\textit{main}|}| just include the main file
at the top of each child file:
%
\begin{center}
|\input{|\textit{main}|}|
\end{center}
%
A simple redirection |\childdocforward{|\textit{dest}|}| is achieved by:
%
\begin{center}
|\def\jobname{|\textit{dest}|}\input{\jobname}|
\end{center}
%
The redirection with prefix
|\childdocforwardprefix[|\textit{prefix}|]{|\textit{dest}|}|
is accomplished by:
%
\begin{center}
\begin{tabular}{l}
|{\edef\jobname{\scantokens\expandafter{\jobname\noexpand}}|\\
|\def\redirectjob |\textit{prefix}|#1~~~{\gdef\jobname{|\textit{dest}|#1}}|\\
|\expandafter\redirectjob\jobname~~~}\input{\jobname}|
\end{tabular}
\end{center}

In an alternative approach,
child documents can be compiled by a specific command line
without additional code or specific definitions:
%
\begin{center}
|... -jobname "|\textit{target}|" "|[\textit{flags}]%
|\includeonly{|\textit{dest}|}\input{|\textit{main}|}"|
\end{center}
%

%%%%%%%%%%%%%%%%%%%%%%%%%%%%%%%%%%%%%%%%%%%%%%%%%%%%%%%%%%%%%%%%%%%%%%%%%%%%%%%%
%%%%%%%%%%%%%%%%%%%%%%%%%%%%%%%%%%%%%%%%%%%%%%%%%%%%%%%%%%%%%%%%%%%%%%%%%%%%%%%%
\section{Information}

%%%%%%%%%%%%%%%%%%%%%%%%%%%%%%%%%%%%%%%%%%%%%%%%%%%%%%%%%%%%%%%%%%%%%%%%%%%%%%%%
\subsection{Copyright}

Copyright \copyright{} 2017--2018 Niklas Beisert

This work may be distributed and/or modified under the
conditions of the \LaTeX{} Project Public License, either version 1.3
of this license or (at your option) any later version.
The latest version of this license is in
  \url{http://www.latex-project.org/lppl.txt}
and version 1.3 or later is part of all distributions of \LaTeX{}
version 2005/12/01 or later.

This work has the LPPL maintenance status `maintained'.

The Current Maintainer of this work is Niklas Beisert.

This work consists of the files |README.txt|, |childdoc.ins| and |childdoc.dtx|
as well as the derived files |childdoc.def|, |cdocsamp.tex|
with |cdocsch1.tex|, |cdocsch2.tex|, |cdocspt3.tex|, |cdocspt4.tex|,
|cdocsdrf.tex|, |cdocsfn1.tex|, |cdocsfn2.tex|
as well as |childdoc.pdf|.

%%%%%%%%%%%%%%%%%%%%%%%%%%%%%%%%%%%%%%%%%%%%%%%%%%%%%%%%%%%%%%%%%%%%%%%%%%%%%%%%
\subsection{Files and Installation}

The package consists of the files:
%
\begin{center}
\begin{tabular}{ll}
    |README.txt|   & readme file \\
    |childdoc.ins| & installation file \\
    |childdoc.dtx| & source file \\
    |childdoc.def| & definition file \\
    |cdocsamp.tex| & sample main file \\
    |cdocsch1.tex| & sample include file \\
    |cdocsch2.tex| & sample include file \\
    |cdocspt3.tex| & sample part file \\
    |cdocspt4.tex| & sample part file \\
    |cdocsdrf.tex| & sample redirection file \\
    |cdocsfn1.tex| & sample redirection file \\
    |cdocsfn2.tex| & sample redirection file \\
    |childdoc.pdf| & manual
\end{tabular}
\end{center}
%
The distribution consists of the files
|README.txt|, |childdoc.ins| and |childdoc.dtx|.
%
\begin{itemize}
\item
Run (pdf)\LaTeX{} on |childdoc.dtx|
to compile the manual |childdoc.pdf| (this file).
\item
Run \LaTeX{} on |childdoc.ins| to create the definitions file |childdoc.def|
and the sample |cdocsamp.tex| with include files
|cdocsch1.tex|, |cdocsch2.tex|, |cdocspt3.tex|, |cdocspt4.tex|,
|cdocsdrf.tex|, |cdocsfn1.tex|, |cdocsfn2.tex|.
Then copy the file |childdoc.def| to an appropriate directory of your \LaTeX{}
distribution, e.g.\ \textit{texmf-root}|/tex/latex/childdoc|.
\end{itemize}

%%%%%%%%%%%%%%%%%%%%%%%%%%%%%%%%%%%%%%%%%%%%%%%%%%%%%%%%%%%%%%%%%%%%%%%%%%%%%%%%
\subsection{Related CTAN Packages}

There are several other packages which offer a similar functionality:
%
\begin{itemize}
\item
The packages
\href{http://ctan.org/pkg/docmute}{\textsf{docmute}},
\href{http://ctan.org/pkg/includex}{\textsf{includex}} and
\href{http://ctan.org/pkg/standalone}{\textsf{standalone}}
provide commands to include only the document body of
a child file thus allowing both files to be compiled individually.
\item
The packages \href{http://ctan.org/pkg/subdocs}{\textsf{subdocs}}
and \href{http://ctan.org/pkg/subfiles}{\textsf{subfiles}}
provide structures in which the main and child documents can be
encapsulated and allowing them to be compiled individually.
The inclusion mechanism is different from the conventional |\include|.
\item
The package \href{http://ctan.org/pkg/combine}{\textsf{combine}}
is an elaborate solution to combine several documents into one.
\end{itemize}
%
See also the CTAN topic \href{http://ctan.org/topic/subdocs}{\textsf{subdocs}}
for further related packages.
The present package differs from the above solutions in that
a document structure constructed with the conventional |\include| mechanism
just needs two extra commands at the top of every file
such that all constituent files can be compiled individually.

%%%%%%%%%%%%%%%%%%%%%%%%%%%%%%%%%%%%%%%%%%%%%%%%%%%%%%%%%%%%%%%%%%%%%%%%%%%%%%%%
%\subsection{Feature Suggestions}
%
%The following is a list of features which may be useful for future
%versions of this package:
%%
%\begin{itemize}
%\item
%\ldots
%\end{itemize}

%%%%%%%%%%%%%%%%%%%%%%%%%%%%%%%%%%%%%%%%%%%%%%%%%%%%%%%%%%%%%%%%%%%%%%%%%%%%%%%%
\subsection{Revision History}

%%%%%%%%%%%%%%%%%%%%%%%%%%%%%%%%%%%%%%%%
\paragraph{v2.0:} 2018/12/30

\begin{itemize}
\item
immediate forward processing
\item
added |\childdocby| mechanism
\item
manual restructured
\end{itemize}

%%%%%%%%%%%%%%%%%%%%%%%%%%%%%%%%%%%%%%%%
\paragraph{v1.6:} 2018/01/17

\begin{itemize}
\item
application for development of include files
\item
corrections to manual
\end{itemize}

%%%%%%%%%%%%%%%%%%%%%%%%%%%%%%%%%%%%%%%%
\paragraph{v1.5:} 2017/05/21

\begin{itemize}
\item
more complete structuring introduced
\item
|\childdocof| introduced
\item
|\childdoc| renamed to |\childdocmain|
\item
|\childredirect| renamed to |\childdocforward| and |\childdocforwardprefix|
and functionality expanded
\end{itemize}

%%%%%%%%%%%%%%%%%%%%%%%%%%%%%%%%%%%%%%%%
\paragraph{v1.0:} 2017/04/27

\begin{itemize}
\item
manual and install package
\item
first version published on CTAN
\end{itemize}

%%%%%%%%%%%%%%%%%%%%%%%%%%%%%%%%%%%%%%%%
\paragraph{v0.6:} 2017/04/26

\begin{itemize}
\item
redirection mechanism added
\end{itemize}

%%%%%%%%%%%%%%%%%%%%%%%%%%%%%%%%%%%%%%%%
\paragraph{v0.5:} 2017/04/26

\begin{itemize}
\item
functionality in definition file
\end{itemize}


%%%%%%%%%%%%%%%%%%%%%%%%%%%%%%%%%%%%%%%%%%%%%%%%%%%%%%%%%%%%%%%%%%%%%%%%%%%%%%%%
%%%%%%%%%%%%%%%%%%%%%%%%%%%%%%%%%%%%%%%%%%%%%%%%%%%%%%%%%%%%%%%%%%%%%%%%%%%%%%%%
%%%%%%%%%%%%%%%%%%%%%%%%%%%%%%%%%%%%%%%%%%%%%%%%%%%%%%%%%%%%%%%%%%%%%%%%%%%%%%%%
\appendix

\settowidth\MacroIndent{\rmfamily\scriptsize 000\ }

 \DocInput{childdoc.dtx}

\end{document}
%</driver>
% \fi
%
% %%%%%%%%%%%%%%%%%%%%%%%%%%%%%%%%%%%%%%%%%%%%%%%%%%%%%%%%%%%%%%%%%%%%%%%%%%%%%%
% %%%%%%%%%%%%%%%%%%%%%%%%%%%%%%%%%%%%%%%%%%%%%%%%%%%%%%%%%%%%%%%%%%%%%%%%%%%%%%
% \section{Sample}
%\iffalse
%<*samplemain>
%\fi
%
% The following presents a sample document
% with two chapters, two parts, a title page,
% a compile flag as well as three forwarding files to set the flag.
% It consists of eight |.tex| files:
% \begin{center}
% \begin{tabular}{ll}
% |cdocsamp.tex|&main file\\
% |cdocsch1.tex|&include file for chapter 1\\
% |cdocsch2.tex|&include file for chapter 2\\
% |cdocspt3.tex|&include file for part 3\\
% |cdocspt4.tex|&include file for part 4\\
% |cdocsdrf.tex|&forwarding file for main file in draft mode\\
% |cdocsfi1.tex|&forwarding file for final version of chapter 1\\
% |cdocsfi2.tex|&forwarding file for final version of chapter 2\\
% \end{tabular}
% \end{center}
% Each of the eight files can be compiled directly by the \LaTeX{} compiler.
%
% %%%%%%%%%%%%%%%%%%%%%%%%%%%%%%%%%%%%%%
% \paragraph{Main File.}
%
% The main file is called |cdocsamp.tex|.
%
% Load the \textsf{childdoc} definitions and
% declare the filename for the main document:
%    \begin{macrocode}
\input{childdoc.def}
\childdocmain{}
%    \end{macrocode}

% Optional override for |\version| flag:
%    \begin{macrocode}
%%\ifchilddoc\else\providecommand{\version}{draft}\fi
%    \end{macrocode}

% Define the default values for the |\version| flag
% (|final| for the main file and |draft| for childs):
%    \begin{macrocode}
\ifchilddoc
\providecommand{\version}{draft}
\else
\providecommand{\version}{final}
\fi
%    \end{macrocode}

% Load the standard document class:
%    \begin{macrocode}
\documentclass[12pt]{article}
%    \end{macrocode}

% Start the document body:
%    \begin{macrocode}
\begin{document}
%    \end{macrocode}

% Declare a title page.
% Print title, part of document being processed and version flag:
%    \begin{macrocode}
\addtocounter{page}{-1}
\begin{center}
{\LARGE\bfseries{}childdoc example\par}
\vspace{1cm}
\ifchilddoc
\ifchilddocmanual part\else chapter\fi:
`\childdocname' of `\childdocjob'\par
\else
main document: `\childdocjob'\par
\fi
version: \version\par
\end{center}
\newpage
%    \end{macrocode}

% Manually include selected file,
% otherwise process as usual:
%    \begin{macrocode}
\ifchilddocmanual
\section*{part `\childdocname'}
\input{\childdocname}
\else
%    \end{macrocode}

% Include the two chapters:
%    \begin{macrocode}
\include{cdocsch1}
\include{cdocsch2}
%    \end{macrocode}

% Include the two parts unless only chapters should be displayed:
%    \begin{macrocode}
\ifchilddoc\else
\section{part three}
\input{cdocspt3}
\section{part four}
\input{cdocspt4}
\fi
%    \end{macrocode}

% Process as usual until here:
%    \begin{macrocode}
\fi
%    \end{macrocode}

% End of document body:
%    \begin{macrocode}
\end{document}
%    \end{macrocode}
%\iffalse
%</samplemain>
%\fi
%
% %%%%%%%%%%%%%%%%%%%%%%%%%%%%%%%%%%%%%%
% \paragraph{Chapter Include Files.}
%
% The include files are called |cdocsch1.tex| and |cdocsch2.tex|.
%
%\iffalse
%<*samplechap1|samplechap2>
%\fi

% Optional override for |\version| flag:
%    \begin{macrocode}
%%\providecommand{\version}{final}
%    \end{macrocode}

% Include the main document:
%    \begin{macrocode}
\input{childdoc.def}
\childdocof{cdocsamp}
%    \end{macrocode}

%\iffalse
%</samplechap1|samplechap2>
%\fi
%
%\iffalse
%<*samplechap1>
%\fi
% Some text for chapter 1:
%    \begin{macrocode}
\section{one}
some text in chapter one
%    \end{macrocode}

%\iffalse
%</samplechap1>
%\fi
% Some text for chapter 2:
%\iffalse
%<*samplechap2>
%\fi
%    \begin{macrocode}
\section{two}
more text in chapter two
%    \end{macrocode}

%\iffalse
%</samplechap2>
%\fi
%
% %%%%%%%%%%%%%%%%%%%%%%%%%%%%%%%%%%%%%%
% \paragraph{Part Include Files.}
%
% The include files are called |cdocspt3.tex| and |cdocspt4.tex|.
%
%\iffalse
%<*samplepart3|samplepart4>
%\fi

% Optional override for |\version| flag:
%    \begin{macrocode}
%%\providecommand{\version}{final}
%    \end{macrocode}

% Include the main document:
%    \begin{macrocode}
\input{childdoc.def}
\childdocby{cdocsamp}
%    \end{macrocode}

%\iffalse
%</samplepart3|samplepart4>
%\fi
%
%\iffalse
%<*samplepart3>
%\fi
% Some text for part 3:
%    \begin{macrocode}
some text in part three
%    \end{macrocode}

%\iffalse
%</samplepart3>
%\fi
% Some text for part 4:
%\iffalse
%<*samplepart4>
%\fi
%    \begin{macrocode}
more text in part four
%    \end{macrocode}

%\iffalse
%</samplepart4>
%\fi
%
% %%%%%%%%%%%%%%%%%%%%%%%%%%%%%%%%%%%%%%
% \paragraph{Forwarding for a Complete Draft.}
%
% The following forwarding file |cdocsdrf.tex|
% compiles the main document in draft mode:
%\iffalse
%<*sampledraft>
%\fi
%    \begin{macrocode}
\def\version{draft}
\input{childdoc.def}
\childdocforward{cdocsamp}
%    \end{macrocode}

%\iffalse
%</sampledraft>
%\fi
%
% %%%%%%%%%%%%%%%%%%%%%%%%%%%%%%%%%%%%%%
% \paragraph{Forwarding for Final Version of the Chapters.}
%
% The following forwarding files |cdocsfn1.tex| and |cdocsfn2.tex|
% (with identical content)
% compile the final versions of the child documents
% |cdocsch1.tex| and |cdocsch2.tex|, respectively:
%\iffalse
%<*samplefinal>
%\fi
%    \begin{macrocode}
\def\version{final}
\input{childdoc.def}
\childdocforwardprefix[cdocsamp]{cdocsfn}{cdocsch}
%    \end{macrocode}

%\iffalse
%</samplefinal>
%\fi
%
% %%%%%%%%%%%%%%%%%%%%%%%%%%%%%%%%%%%%%%
% \paragraph{Command Line Processing.}
%
% The following three command lines generate the output files
% |cdocscld|, |cdocscl1| and |cdocscl2|
% which should be identical to
% |cdocsdrf|, |cdocsch1| and |cdocsfn2|, respectively:
% \begin{center}
% \begin{tabular}{l}
% |latex -jobname cdocscld \|\\
% |  "\def\version{draft}\input{childdoc.def}\childdocforward{cdocsamp}"|\\
% |latex -jobname cdocscl1 \|\\
% |  "\input{childdoc.def}\childdocforward[cdocsamp]{cdocsch1}"|\\
% |latex -jobname cdocscl2 \|\\
% |  "\def\version{final}\input{childdoc.def}\childdocforward{cdocsch2}"|
% \end{tabular}
% \end{center}
% Note that the trailing backslash on each first line
% merely continues the input to the second line
% (for convenient cut ant paste).
% Furthermore, the command |latex| can be replaced by any
% of its alternative versions such as |pdflatex|.
%
% %%%%%%%%%%%%%%%%%%%%%%%%%%%%%%%%%%%%%%%%%%%%%%%%%%%%%%%%%%%%%%%%%%%%%%%%%%%%%%
% %%%%%%%%%%%%%%%%%%%%%%%%%%%%%%%%%%%%%%%%%%%%%%%%%%%%%%%%%%%%%%%%%%%%%%%%%%%%%%
% \section{Implementation}
%\iffalse
%<*package>
%\fi
%
% This section describes the definitions file |childdoc.def|.

% The definitions cannot be loaded using |\usepackage| or |\RequirePackage|
% which has a mechanism to prevent loading a style file more than once.
% When loading the definitions by means of |\input|
% multiple instances have to be prevented manually:
%\iffalse
%This code needs to be before the `\ProvidesFile' directive
%which is defined at the beginning of this file.
%Therefore it is also placed there and commented out here.
%</package>
%<*discard>
%\fi
%    \begin{macrocode}
\ifdefined\childdocmain\endinput\fi
%    \end{macrocode}
%\iffalse
%</discard>
%<*package>
%\fi
%
% \macro{\ifchilddoc}
% \macro{\ifchilddocmanual}
% The conditional |\ifchilddoc| tells whether a
% child (true) or main (false) document is being compiled.
% The conditional |\ifchilddocmanual| tells whether
% the |\includeonly| mechanism is used (false) or
% the selection of child files must be performed manually (true).
% The definitions initialise to false:
%    \begin{macrocode}
\newif\ifchilddoc
\newif\ifchilddocmanual
%    \end{macrocode}

% \macro{\childdocname}
% \macro{\childdocjob}
% The macro |\childdocname| stores the name of the main document
% to be compiled. The macro |\childdocjob| stores the name of
% the document on which the \LaTeX{} compiler was originally invoked.
% The content of |\jobname| cannot be compared
% to filenames specified in the source due to different catcodes.
% The following code rescans |\jobname|, stores the result
% in |\childdocname| and saves a copy in |\childdocjob|:
%    \begin{macrocode}
\edef\childdocname{\scantokens\expandafter{\jobname\noexpand}}
\let\childdocjob\childdocname
%    \end{macrocode}

% \macro{\childdocdisable}
% The macro |\childdocdisable| prevents the main file
% from being processed more than once.
% At this stage, the main document command |\childdocmain|
% is assumed to be called once again where it should do nothing.
% Any subsequent call to it should prevent
% a secondary processing of the main document
% It overwrites the forwarding commands
% |\childdocof| and |\childdocforward|
% with empty macros to prevent further inclusions of the main document:
%    \begin{macrocode}
\newcommand{\childdocdisable}
{
  \renewcommand{\childdocmain}[1]{\renewcommand{\childdocmain}[1]{\endinput}}
  \renewcommand{\childdocof}[1]{}
  \renewcommand{\childdocby}[2][]{}
  \renewcommand{\childdocforward}[2][]{}
  \renewcommand{\childdocdisable}{}
}
%    \end{macrocode}

% \macro{\childdocmain}
% The macro |\childdocmain| is to be called at the top of the main file
% with nothing or the main filename (without extension) as argument.
% First, it breaks loops.
% If the argument is not empty and does not match |\childdocname|
% (which is set by the first inclusion of |childdoc.def|),
% |\ifchilddoc| is set to true, |\includeonly| is applied to the child file
% and |\jobname| is set to the main file
% (for proper handling of |.aux| files):
%    \begin{macrocode}
\newcommand{\childdocmain}[1]
{
  \childdocdisable\childdocmain{}
  \if?#1?\else
    \begingroup
      \def\childdoctmp{#1}
      \ifx\childdoctmp\childdocname
        \def\childdoctmp{}
      \else
        \def\childdoctmp
        {
          \childdoctrue
          \includeonly{\childdocname}
          \def\childdocjob{#1}
          \def\jobname{#1}
        }
      \fi
      \expandafter
    \endgroup
    \childdoctmp
  \fi
}
%    \end{macrocode}

% \macro{\childdocof}
% The command |\childdocof| redirects
% compilation to the main file |#1|.
%    \begin{macrocode}
\newcommand{\childdocof}[1]
{
  \childdocdisable
  \childdoctrue
  \includeonly{\childdocname}
  \def\jobname{#1}
  \def\childdocjob{#1}
  \input{#1}
}
%    \end{macrocode}

% \macro{\childdocby}
% The command |\childdocby| ....
%    \begin{macrocode}
\newcommand{\childdocby}[2][]
{
  \childdocdisable
  \childdoctrue
  \childdocmanualtrue
  \if?#1?\else
    \def\jobname{#2}
  \fi
  \def\childdocjob{#2}
  \input{#2}
  \endinput
}
%    \end{macrocode}

% \macro{\childdocforward}
% The command |\childdocforward| redirects
% compilation to the main file or
% (if the optional argument is given) a child file.
% Parameters are set as if the main file
% or a child file starting with |\childdocof| was compiled.
% Then compilation is handed over to the main file:
%    \begin{macrocode}
\newcommand{\childdocforward}[2][]
{
  \begingroup
    \if?#1?
      \def\childdoctmp
      {
        \def\childdocname{#2}
        \def\childdocjob{#2}
        \def\jobname{#2}
        \input{#2}
        \endinput
      }
    \else
      \def\childdoctmp
      {
        \childdocdisable
        \def\childdocname{#2}
        \childdoctrue
        \includeonly{#2}
        \def\childdocjob{#1}
        \def\jobname{#1}
        \input{#1}
        \endinput
      }
    \fi
    \expandafter
  \endgroup
  \childdoctmp
}
%    \end{macrocode}

% \macro{\childdocforwardprefix}
% The command |\childdocforwardprefix| redirects
% compilation to the main or a child file by means of a pattern.
% The prefix |#1| in the current filename is replaced by |#2|
% and the suffix of the current filename is kept
% (it is assumed that the filename does not contain the substring `|~~~|'
% which is used as a delimiter).
% Compilation is handed over to the new file by |\childdocforward|:
%    \begin{macrocode}
\newcommand{\childdocforwardprefix}[3][]
{
  \begingroup
    \def\childdocextract #2##1~~~{\def\childdoctmp{\childdocforward[#1]{#3##1}}}
    \expandafter\childdocextract\childdocname~~~
    \expandafter
  \endgroup
  \childdoctmp
}
%    \end{macrocode}

% \macro{\childdoc}
% The deprecated macro |\childdoc| is a legacy version of |\childdocmain|:
%    \begin{macrocode}
\newcommand{\childdoc}{\childdocmain}
%    \end{macrocode}

% \macro{\childdocredirect}
% The deprecated macro |\childdocredirect| is a legacy version
% of |\childdocforward| and |\childdocforwardprefix|:
%    \begin{macrocode}
\newcommand{\childdocredirect}[2][]
{
  \begingroup
    \if?#1?
      \def\childdoctmp{\childdocforward{#2}}
    \else
      \def\childdoctmp{\childdocforwardprefix{#1}{#2}}
    \fi
    \expandafter
  \endgroup
  \childdoctmp
}
%    \end{macrocode}

%\iffalse
%</package>
%\fi
%
\endinput

\childdocmain{}
%    \end{macrocode}

% Optional override for |\version| flag:
%    \begin{macrocode}
%%\ifchilddoc\else\providecommand{\version}{draft}\fi
%    \end{macrocode}

% Define the default values for the |\version| flag
% (|final| for the main file and |draft| for childs):
%    \begin{macrocode}
\ifchilddoc
\providecommand{\version}{draft}
\else
\providecommand{\version}{final}
\fi
%    \end{macrocode}

% Load the standard document class:
%    \begin{macrocode}
\documentclass[12pt]{article}
%    \end{macrocode}

% Start the document body:
%    \begin{macrocode}
\begin{document}
%    \end{macrocode}

% Declare a title page.
% Print title, part of document being processed and version flag:
%    \begin{macrocode}
\addtocounter{page}{-1}
\begin{center}
{\LARGE\bfseries{}childdoc example\par}
\vspace{1cm}
\ifchilddoc
\ifchilddocmanual part\else chapter\fi:
`\childdocname' of `\childdocjob'\par
\else
main document: `\childdocjob'\par
\fi
version: \version\par
\end{center}
\newpage
%    \end{macrocode}

% Manually include selected file,
% otherwise process as usual:
%    \begin{macrocode}
\ifchilddocmanual
\section*{part `\childdocname'}
\input{\childdocname}
\else
%    \end{macrocode}

% Include the two chapters:
%    \begin{macrocode}
\include{cdocsch1}
\include{cdocsch2}
%    \end{macrocode}

% Include the two parts unless only chapters should be displayed:
%    \begin{macrocode}
\ifchilddoc\else
\section{part three}
\input{cdocspt3}
\section{part four}
\input{cdocspt4}
\fi
%    \end{macrocode}

% Process as usual until here:
%    \begin{macrocode}
\fi
%    \end{macrocode}

% End of document body:
%    \begin{macrocode}
\end{document}
%    \end{macrocode}
%\iffalse
%</samplemain>
%\fi
%
% %%%%%%%%%%%%%%%%%%%%%%%%%%%%%%%%%%%%%%
% \paragraph{Chapter Include Files.}
%
% The include files are called |cdocsch1.tex| and |cdocsch2.tex|.
%
%\iffalse
%<*samplechap1|samplechap2>
%\fi

% Optional override for |\version| flag:
%    \begin{macrocode}
%%\providecommand{\version}{final}
%    \end{macrocode}

% Include the main document:
%    \begin{macrocode}
% \iffalse
%
% childdoc.dtx Copyright (C) 2017-2018 Niklas Beisert
%
% This work may be distributed and/or modified under the
% conditions of the LaTeX Project Public License, either version 1.3
% of this license or (at your option) any later version.
% The latest version of this license is in
%   http://www.latex-project.org/lppl.txt
% and version 1.3 or later is part of all distributions of LaTeX
% version 2005/12/01 or later.
%
% This work has the LPPL maintenance status `maintained'.
%
% The Current Maintainer of this work is Niklas Beisert.
%
% This work consists of the files childdoc.dtx and childdoc.ins
% and the derived files childdoc.def and cdocsamp.tex with
% cdocsch1.tex, cdocsch2.tex, cdocsdrf.tex, cdocsfn1.tex, cdocsfn2.tex.
%
%<package>\ifdefined\childdocmain\endinput\fi
%<package>\ProvidesFile{childdoc.def}[2018/12/30 v2.0 child document driver]
%<samplemain>\ProvidesFile{cdocsamp.tex}[2018/12/30 v2.0 sample for childdoc]
%<*driver>
%\ProvidesFile{childdoc.drv}[2018/12/30 v2.0 childdoc reference manual file]
\PassOptionsToClass{10pt,a4paper}{article}
\documentclass{ltxdoc}

\usepackage[margin=35mm]{geometry}
\usepackage{hyperref}
\usepackage{hyperxmp}
\usepackage[usenames]{color}

\hypersetup{colorlinks=true}
\hypersetup{pdfstartview=FitH}
\hypersetup{pdfpagemode=UseNone}
\hypersetup{pdfsource={}}
\hypersetup{pdflang={en-UK}}
\hypersetup{pdfcopyright={Copyright 2017-2018 Niklas Beisert.
  This work may be distributed and/or modified under the
  conditions of the LaTeX Project Public License, either version 1.3
  of this license or (at your option) any later version.}}
\hypersetup{pdflicenseurl={http://www.latex-project.org/lppl.txt}}
\hypersetup{pdfcontactaddress={ETH Zurich, ITP, HIT K,
  Wolfgang-Pauli-Strasse 27}}
\hypersetup{pdfcontactpostcode={8093}}
\hypersetup{pdfcontactcity={Zurich}}
\hypersetup{pdfcontactcountry={Switzerland}}
\hypersetup{pdfcontactemail={nbeisert@itp.phys.ethz.ch}}
\hypersetup{pdfcontacturl={http://people.phys.ethz.ch/\xmptilde nbeisert/}}

\newcommand{\secref}[1]{\hyperref[#1]{section \ref*{#1}}}

\parskip1ex
\parindent0pt
\let\olditemize\itemize
\def\itemize{\olditemize\parskip0pt}

\begin{document}

\title{The \textsf{childdoc} Package}
\hypersetup{pdftitle={The childdoc Package}}
\author{Niklas Beisert\\[2ex]
  Institut f\"ur Theoretische Physik\\
  Eidgen\"ossische Technische Hochschule Z\"urich\\
  Wolfgang-Pauli-Strasse 27, 8093 Z\"urich, Switzerland\\[1ex]
  \href{mailto:nbeisert@itp.phys.ethz.ch}
  {\texttt{nbeisert@itp.phys.ethz.ch}}}
\hypersetup{pdfauthor={Niklas Beisert}}
\hypersetup{pdfsubject={Manual for the LaTeX2e Package childdoc}}
\date{30 December 2018, \textsf{v2.0}}
\maketitle

\begin{abstract}\noindent
\textsf{childdoc} is a \LaTeXe{} package
that enables the direct compilation
of document sections included by |\include|
to individual files.
\end{abstract}

\begingroup
\parskip0ex
\tableofcontents
\endgroup

%%%%%%%%%%%%%%%%%%%%%%%%%%%%%%%%%%%%%%%%%%%%%%%%%%%%%%%%%%%%%%%%%%%%%%%%%%%%%%%%
%%%%%%%%%%%%%%%%%%%%%%%%%%%%%%%%%%%%%%%%%%%%%%%%%%%%%%%%%%%%%%%%%%%%%%%%%%%%%%%%
\section{Introduction}

\LaTeX{} provides a mechanism to structure a large document (such as a book)
into a main file and several child files (containing the chapters)
using the |\include| command.
This mechanism is beneficial for documents
which span hundreds of pages in order to
make the source file(s) more manageable.
Moreover, compilation can be restricted to
selected child files by means of the |\includeonly| command.
The latter feature can be used to reduce the compilation time while editing
(this was significantly more useful in the earlier days of \LaTeX{})
or to generate a smaller document which is easier to navigate.
Another application of |\includeonly| is to generate
documents consisting of selected parts of the complete document.

However, there are a few drawbacks of the plain |\include| mechanism:
\begin{itemize}
\item
The child files cannot be compiled on their own,
they can only be compiled via the main file.
A naive editing environment
(such as a text editor with an option
to have the current file processed by \LaTeX)
may require one to switch to the main file before compiling;
attempting to compile the child file produces errors.
\item
The main file must be modified (each time)
to adjust the |\includeonly| command
to the present needs. This easily leaves the main file in a messy state.
\item
The generated document will always carry the filename
of the main document. This is inconvenient if
several child files are to be compiled and
to be kept for distribution.
\end{itemize}

The present package provides a simple interface
to make child files individually compilable by \LaTeX{}.
Compiling a child file then has the same effect as compiling
the main file with an |\includeonly| command
to select the appropriate child.
Moreover the generated document will carry the name of the child
rather than the main file.
This resolves all three above issues.

This feature is meant to make the editing of books,
thesis documents and lecture notes somewhat more convenient.
However, the package can also be used efficiently for
composing a series of documents (such as exercise sheets)
which are typically distributed individually.
It then assists the author in generating the individual documents
(potentially in different versions)
as well as a document containing the collected series.
Another application is in developing style files
or other kinds of included material
where compilation of the style file could redirect
to a sample or test file.

%%%%%%%%%%%%%%%%%%%%%%%%%%%%%%%%%%%%%%%%%%%%%%%%%%%%%%%%%%%%%%%%%%%%%%%%%%%%%%%%
%%%%%%%%%%%%%%%%%%%%%%%%%%%%%%%%%%%%%%%%%%%%%%%%%%%%%%%%%%%%%%%%%%%%%%%%%%%%%%%%
\section{Usage}

First of all, the package \textsf{childdoc} is \emph{not} a standard
\LaTeXe{} |.sty| style file! Therefore it needs to be invoked in
a non-standard way.

%%%%%%%%%%%%%%%%%%%%%%%%%%%%%%%%%%%%%%%%%%%%%%%%%%%%%%%%%%%%%%%%%%%%%%%%%%%%%%%%
\subsection{Included Files}
\label{sec:include}

%%%%%%%%%%%%%%%%%%%%%%%%%%%%%%%%%%%%%%%%
\DescribeMacro{\childdocmain}
To use the package, add the commands
\begin{center}
\begin{tabular}{l}
|\input{childdoc.def}|\\
|\childdocmain{}|\\
\end{tabular}
\end{center}
at the very top of the main \LaTeX{} file,
in particular \emph{before} the |\documentclass| statement!
The argument of |\childdocmain| should be left empty
(but it must be present).

%%%%%%%%%%%%%%%%%%%%%%%%%%%%%%%%%%%%%%%%
\DescribeMacro{\childdocof}
Furthermore, add the commands
\begin{center}
\begin{tabular}{l}
|\input{childdoc.def}|\\
|\childdocof{|\textit{main}|}|\\
\end{tabular}
\end{center}
at the top of every child file \textit{child}
which is included by |\include{|\textit{child}|}|
from within the main file
(or at least for those files to be compiled individually).
The argument \textit{main} must be the filename of the main file.

There are a couple of
considerations in setting up the main and child documents:

%%%%%%%%%%%%%%%%%%%%%%%%%%%%%%%%%%%%%%%%
\paragraph{Restrictions.}

Please note the following restrictions:
\begin{itemize}
\item
|\childdocmain| must be called with one argument \textit{main}
to ensure compatibility with earlier version of the package.
It must either be empty (|\childdocmain{}|)
or precisely match the filename of the main file in which it is specified.
See \secref{sec:detection} for further information.
\item
The filename \textit{main} must be specified without the |.tex| extension.
\item
The filename \textit{main} is case sensitive
(even in case-insensitive file systems)
due to internal string comparison.
\item
The argument \textit{main} should be fully expanded, it cannot be a macro.
\item
Subdirectories and special characters should be avoided in filenames.
\item
The command |\childdocmain{|\textit{main}|}| must be followed by a whitespace.
It should not be followed immediately by another command
or by a comment mark `|%|'.
This is because the \TeX{} parser reads the token immediately following
the argument of |\childdocmain| and puts it
at the beginning of every child section;
however, a white\-space is ignored.
\end{itemize}

%%%%%%%%%%%%%%%%%%%%%%%%%%%%%%%%%%%%%%%%
\paragraph{Content of Main File.}

It is advisable to place all content in the child files included by |\include|.
Any output contained in the main file will appear in all child documents
unless suppressed manually;
it cannot be suppressed automatically by the |\includeonly| directive
and thus should normally be avoided.
A method to include some content in the main file
by means of conditional processing is described in \secref{sec:conditional}.

%%%%%%%%%%%%%%%%%%%%%%%%%%%%%%%%%%%%%%%%
\paragraph{Page Numbering.}

When only a part of the document is compiled,
the appropriate numbering of pages
(as well as other status parameters)
is determined from the |.aux| files.
The latter contain information from previous passes.
However this information needs to propagate through
all intermediate child documents.
Therefore the page numbering in child documents may well
be inconsistent until the complete document is compiled at least once.

A useful (if unconventional) way to always ensure a consistent
page numbering is to restart the numbering in each child document
and denote the pages by `\textit{child}|.|\textit{page}'
where \textit{child} represents the chapter/section number of the child file.
This can be achieved by the command
|\numberwithin{page}{|\textit{child}|}|
of the \textsf{amsmath} package
where \textit{child} can be |chapter| or |section|
depending on the chosen structuring.
Alternatively, one can modify the macro |\thepage| appropriately
and reset the counter |page| at the start of each child file.

%%%%%%%%%%%%%%%%%%%%%%%%%%%%%%%%%%%%%%%%%%%%%%%%%%%%%%%%%%%%%%%%%%%%%%%%%%%%%%%%
\subsection{Conditional Processing}
\label{sec:conditional}

The package provides a mechanism to compile different versions
of a document. To customise the versions further some conditional processing
can come in handy to distinguish which version is being compiled.
The package provides two macros to describe the compilation context:

%%%%%%%%%%%%%%%%%%%%%%%%%%%%%%%%%%%%%%%%
\DescribeMacro{\ifchilddoc}
The conditional |\ifchilddoc| distinguishes between the compilation of
child documents and the main document:
%
\begin{center}
|\ifchilddoc |\textit{child-code}| |[|\||else |\textit{main-code}]| \||fi|
\end{center}

%%%%%%%%%%%%%%%%%%%%%%%%%%%%%%%%%%%%%%%%
\DescribeMacro{\childdocname}
\DescribeMacro{\childdocjob}
The macro |\childdocname| contains the filename (without extension)
of the main or child file being processed.
Note that |\childdocjob| will always contain the name of the main file.

%%%%%%%%%%%%%%%%%%%%%%%%%%%%%%%%%%%%%%%%
\paragraph{Title Page.}

Conditional processing can be used to include a title or banner page
in the main document when proper precautions are taken.
Importantly, the code in the main file should ensure that the page counter
(as well as other status parameters which are stored in the |.aux| files)
takes the same value after the conditional processing.
Otherwise the page numbers may take divergent values
depending on which part is compiled.

For example, a title page could be declared by:
%
\begin{center}
\begin{tabular}{l}
|\ifchilddoc\||else|\\
|\addtocounter{page}{-1}|\\
\textit{code for title page}\\
|\newpage|\\
|\||fi|
\end{tabular}
\end{center}
%
A banner page for the child documents can be generated by:
%
\begin{center}
\begin{tabular}{l}
|\ifchilddoc|\\
|\addtocounter{page}{-1}|\\
\textit{code for banner page}\\
|\newpage|\\
|\||fi|
\end{tabular}
\end{center}
%
Here one could write a message such as:
\begin{center}
|This is the part \childdocname{} of \childdocjob{}.|
\end{center}

%%%%%%%%%%%%%%%%%%%%%%%%%%%%%%%%%%%%%%%%%%%%%%%%%%%%%%%%%%%%%%%%%%%%%%%%%%%%%%%%
\subsection{Flags}
\label{sec:flags}

The package makes it easy to generate different versions
of the main or child documents.
To this end compilation flags can be defined
and assigned different default values.
They will be particularly useful in conjunction
with the forwarding mechanism described in \secref{sec:forward}.

For example, it may be useful to have a flag |\version|
which can be set to |draft| or |final|.
The document source will contain some conditional code
depending on the value of |\version|.
Suppose further, the flag should default to |final| for the main file
and to |draft| for child files
which is a natural assignment for editing the document.
This is achieved by placing the following code
in the preamble of the main document
(below the |\childdocmain| directive):
%
\begin{center}
\begin{tabular}{l}
|\ifchilddoc|\\
|\providecommand{\version}{draft}|\\
|\||else|\\
|\providecommand{\version}{final}|\\
|\||fi|
\end{tabular}
\end{center}
%
The definition by |\providecommand| makes sure
that previous definitions are not overwritten.
Further statements |\providecommand{\version}{...}|
can thus be added before the above code to override it.

For the main file, one might add a line
(between |\childdocmain| and the above block)
%
\begin{center}
|%\ifchilddoc\||else\providecommand{\version}{draft}\||fi|
\end{center}
%
which can be uncommented to produce a draft version.
Likewise one can add a line to the very top of a child file
(above the |\childdocof{|\textit{main}|}| directive)
%
\begin{center}
|%\providecommand{\version}{final}|
\end{center}
%
which can be uncommented to produce the final version of this child document.

%%%%%%%%%%%%%%%%%%%%%%%%%%%%%%%%%%%%%%%%%%%%%%%%%%%%%%%%%%%%%%%%%%%%%%%%%%%%%%%%
\subsection{Forwarding}
\label{sec:forward}

Different versions of the main or child documents
using compilation flags as described in \secref{sec:flags}
can be (permanently) stored in different files
for convenient compilation, viewing and distribution.
To this end, the package defines a command
to pass on compilation to a different file:

%%%%%%%%%%%%%%%%%%%%%%%%%%%%%%%%%%%%%%%%
\DescribeMacro{\childdocforward}
The command |\childdocforward| redirects processing to
another source file:
%
\begin{center}
\begin{tabular}{l}
|\input{childdoc.def}|\\
|\childdocforward[|\textit{main}|]{|\textit{dest}|}|\\
\end{tabular}
\end{center}
%
The argument \textit{dest} is the destination file
(without extension).
It should be the main file or one of the child files.
Note that further \textsf{childdoc} directives
such as |\childdocof| and |\childdocforward|
in the indicated file will be processed in this form.
The optional argument \textit{main}
passes on directly to the main file \textit{main}
while pretending to compile the child \textit{dest}.
This form behaves as if \textit{dest}
issues |\childdocof{|\textit{main}|}| right away,
and no further \textsf{childdoc} directives will be processed.

%%%%%%%%%%%%%%%%%%%%%%%%%%%%%%%%%%%%%%%%
\DescribeMacro{\...prefix}
In the alternative form |\childdocforwardprefix|,
%
\begin{center}
\begin{tabular}{l}
|\input{childdoc.def}|\\
|\childdocforwardprefix[|\textit{main}|]{|\textit{prefix}|}{|\textit{dest}|}|
\end{tabular}
\end{center}
%
the destination file is determined by a pattern
depending on the current file:
To make this work, the current file must be called
`{\textit{prefix}\hspace{0.2em}\textit{suffix}}'
with \textit{prefix} matching precisely the argument.
Processing is then passed on to the file
`{\textit{dest}\hspace{0.2em}\textit{suffix}}'.
Surely, the same effect is achieved by
directly specifying the
argument `{\textit{dest}\hspace{0.2em}\textit{suffix}}'
in the first form.
However, that requires to set up a different file
for each child. With the alternative form of the command
all these files can have exactly the same content
which simplifies setting them up and maintaining them.

For example, the following file |draft.tex|
with a compilation flag |\version| as described in \secref{sec:flags}
compiles the main document as a draft:
%
\begin{center}
\begin{tabular}{l}
|\def\version{draft}|\\
|\input{childdoc.def}|\\
|\childdocforward{|\textit{main}|}|
\end{tabular}
\end{center}
%
Likewise, the following files |final|\textit{nn}|.tex|
compile the final version of the child document
|child|\textit{nn}|.tex|:
%
\begin{center}
\begin{tabular}{l}
|\def\version{final}|\\
|\input{childdoc.def}|\\
|\childdocforwardprefix{final}{child}|
\end{tabular}
\end{center}
%

Note that when several versions of a main file and/or of each child file
are to be generated, it may be convenient to set up a |Makefile| or
shell script to automatise the process.

%%%%%%%%%%%%%%%%%%%%%%%%%%%%%%%%%%%%%%%%%%%%%%%%%%%%%%%%%%%%%%%%%%%%%%%%%%%%%%%%
\subsection{Command Line Processing}
\label{sec:commandline}

The effect of redirection files can also be achieved by invoking
the \LaTeX{} compiler with a more elaborate command line.
Most conveniently this should be done as part
of a shell script or a |Makefile|.

When using \textsf{childdoc} in the main file, the following
command lines effectively perform a redirection
(note that depending on the shell being used,
backslashes may have to be doubled: `|\|' $\to$ `|\\|'):
%
\begin{center}
|... -jobname "|\textit{target}|" |\\|"|[\textit{flags}]%
|\input{childdoc.def}\childdocforward[|\textit{main}|]{|\textit{dest}|}"|
\end{center}
%
Here \textit{target} is the name of the output file,
\textit{main} is the name of the main file
and \textit{dest} is the name of the main or child file to be processed
(all filenames without extensions).
The optional argument \textit{main} can be omitted
if \textit{main} matches \textit{dest}.
Optionally, compilation \textit{flags} can be defined via |\def| commands.
This command line makes the \TeX{} engine believe
it is compiling the file \textit{target}
whose content is specified as the latter parameter.
The provided code then forwards the processing to
\textit{main} or \textit{dest} as described in \secref{sec:forward}.

%%%%%%%%%%%%%%%%%%%%%%%%%%%%%%%%%%%%%%%%%%%%%%%%%%%%%%%%%%%%%%%%%%%%%%%%%%%%%%%%
\subsection{Include by Input}
\label{sec:input}

Including child documents by |\include| has some restrictions by design.
Most notably, the content of a child document always occupies
its own set of pages; pages cannot be shared between child documents.
Usually, this behaviour makes perfect sense
because each child document contain an essential part of the document.
However, in some situations it may be desirable to compose
a document from a collection of parts
without having mandatory page breaks between then.
For this case, the package
provides a mechanism to include parts
by |\input| which can also be processed individually.
However, by construction this mechanism
requires manual handling of the content to be output.

%%%%%%%%%%%%%%%%%%%%%%%%%%%%%%%%%%%%%%%%
\DescribeMacro{\ifchilddocmanual}
The main file should be prepared as usual, see \secref{sec:include}.
However, the document body must make a distinction
between processing of an individual part and of the main document, e.g.:
%
\begin{center}
\begin{tabular}{l}
|\ifchilddocmanual|\\
|\input{\childdocname}|\\
|\||else|\\
\textit{document body with }|\input{|\textit{part}|}|\\
|\||fi|
\end{tabular}
\end{center}
%
The conditional |\ifchilddocmanual| is true whenever
a part to be included by |\input| is being compiled,
and the name of the part is stored in |\childdocname|.

%%%%%%%%%%%%%%%%%%%%%%%%%%%%%%%%%%%%%%%%
\DescribeMacro{\childdocby}
Each part to be included by |\input| should start with:
%
\begin{center}
\begin{tabular}{l}
|\input{childdoc.def}|\\
|\childdocby{|\textit{main}|}|\\
\end{tabular}
\end{center}
%
The directive |\childdocby| is similar to |\childdocof|
described in \secref{sec:include},
but the subsequent selection of content must be done manually.
To that end, both |\ifchilddoc| and |\ifchilddocmanual|
will be true upon processing of a part,
and the name of the part is stored in |\childdocname|.
Note that |\jobname| will be set to the filename of the current part
so that each part receives an individual |.aux| file
that does not interfere with the |.aux| file(s) of the main document.
This behaviour can be altered by the alternative form
|\childdocby[*]{|\textit{main}|}| (with a non-empty optional argument)
which uses the |.aux| file of the main document
by setting |\jobname| to \textit{main}.

%%%%%%%%%%%%%%%%%%%%%%%%%%%%%%%%%%%%%%%%%%%%%%%%%%%%%%%%%%%%%%%%%%%%%%%%%%%%%%%%
\subsection{Driver Development}
\label{sec:driver}

The \textsf{childdoc} mechanism can also be use for the development
of definition files such as \LaTeX{} styles or classes.
This case differs from the above setup with multiple parts
included by |\include| in that no |\includeonly| should be invoked.
This can be achieved by starting the include file
(before |\ProvidesPackage|) with:
%
\begin{center}
\begin{tabular}{l}
|\input{childdoc.def}|\\
|\childdocforward{|\textit{main}|}|\\
\end{tabular}
\end{center}
%
or alternatively with:
%
\begin{center}
\begin{tabular}{l}
|\input{childdoc.def}|\\
|\childdocby{|\textit{main}|}|\\
\end{tabular}
\end{center}
%
Both forms have slightly different effects as described above.
The main file is prepared as usual, see \secref{sec:include}.

%%%%%%%%%%%%%%%%%%%%%%%%%%%%%%%%%%%%%%%%%%%%%%%%%%%%%%%%%%%%%%%%%%%%%%%%%%%%%%%%
\subsection{Legacy Detection}
\label{sec:detection}

The directive |\childdocmain| in the main file can detect
whether the complete document or merely a child is to be compiled
even without using the directive |\childdocof|.
This method is deprecated because it is less robust
and there is no compelling reason to use it;
it is merely provided for backward compatibility
and it may be removed in future versions.

If the detection mechanism is to be used,
it is mandatory to correctly specify
the filename of the main file as the argument of |\childdocmain|:
%
\begin{center}
\begin{tabular}{l}
|\input{childdoc.def}|\\
|\childdocmain{|\textit{main}|}|\\
\end{tabular}
\end{center}
%
If |\jobname| does not match the argument \textit{main} of |\childdocmain|,
it is assumed that |\jobname| points to the child file to be compiled.
When using |\childdocmain| with the main file specified as argument,
it suffices to start a child file
with just |\input{|\textit{main}|}|
without loading of the package and using |\childdocof|.
If instead all processing is done
with the appropriate \textsf{childdoc} directives,
the argument of \textit{main} of |\childdocmain| can be empty.

An alternative version of the command line processing described
in \secref{sec:commandline} using the detection mechanism reads:
%
\begin{center}
|... -jobname "|\textit{target}|" "|[\textit{flags}]%
[|\def\jobname{|\textit{dest}|}|]|\input{|\textit{main}|}"|
\end{center}

%%%%%%%%%%%%%%%%%%%%%%%%%%%%%%%%%%%%%%%%%%%%%%%%%%%%%%%%%%%%%%%%%%%%%%%%%%%%%%%%
\subsection{Manual Code}
\label{sec:manual}

In case one cannot be certain whether the definitions file |childdoc.def|
is installed on the target \TeX{} distribution
and one prefers not to ship it,
it is conceivable to paste a few relevant commands into the sources.

To that end, drop all statements |\input{childdoc.def}|
and perform the replacements as outlined below.
Instead of |\childdocmain{|\textit{main}|}| add the following code
to the top of the main file:
%
\begin{center}
\begin{tabular}{l}
|\||ifdefined\childdocname\endinput\||fi\newif\ifchilddoc|\\
|\edef\childdocname{\scantokens\expandafter{\jobname\noexpand}}|\\
|\def\childdocmain{|\textit{main}|}\||ifx\childdocmain\childdocname\||else|\\
|\childdoctrue\includeonly{\childdocname}\let\jobname\childdocmain\||fi|\\
\end{tabular}
\end{center}
%
Instead of |\childdocof{|\textit{main}|}| just include the main file
at the top of each child file:
%
\begin{center}
|\input{|\textit{main}|}|
\end{center}
%
A simple redirection |\childdocforward{|\textit{dest}|}| is achieved by:
%
\begin{center}
|\def\jobname{|\textit{dest}|}\input{\jobname}|
\end{center}
%
The redirection with prefix
|\childdocforwardprefix[|\textit{prefix}|]{|\textit{dest}|}|
is accomplished by:
%
\begin{center}
\begin{tabular}{l}
|{\edef\jobname{\scantokens\expandafter{\jobname\noexpand}}|\\
|\def\redirectjob |\textit{prefix}|#1~~~{\gdef\jobname{|\textit{dest}|#1}}|\\
|\expandafter\redirectjob\jobname~~~}\input{\jobname}|
\end{tabular}
\end{center}

In an alternative approach,
child documents can be compiled by a specific command line
without additional code or specific definitions:
%
\begin{center}
|... -jobname "|\textit{target}|" "|[\textit{flags}]%
|\includeonly{|\textit{dest}|}\input{|\textit{main}|}"|
\end{center}
%

%%%%%%%%%%%%%%%%%%%%%%%%%%%%%%%%%%%%%%%%%%%%%%%%%%%%%%%%%%%%%%%%%%%%%%%%%%%%%%%%
%%%%%%%%%%%%%%%%%%%%%%%%%%%%%%%%%%%%%%%%%%%%%%%%%%%%%%%%%%%%%%%%%%%%%%%%%%%%%%%%
\section{Information}

%%%%%%%%%%%%%%%%%%%%%%%%%%%%%%%%%%%%%%%%%%%%%%%%%%%%%%%%%%%%%%%%%%%%%%%%%%%%%%%%
\subsection{Copyright}

Copyright \copyright{} 2017--2018 Niklas Beisert

This work may be distributed and/or modified under the
conditions of the \LaTeX{} Project Public License, either version 1.3
of this license or (at your option) any later version.
The latest version of this license is in
  \url{http://www.latex-project.org/lppl.txt}
and version 1.3 or later is part of all distributions of \LaTeX{}
version 2005/12/01 or later.

This work has the LPPL maintenance status `maintained'.

The Current Maintainer of this work is Niklas Beisert.

This work consists of the files |README.txt|, |childdoc.ins| and |childdoc.dtx|
as well as the derived files |childdoc.def|, |cdocsamp.tex|
with |cdocsch1.tex|, |cdocsch2.tex|, |cdocspt3.tex|, |cdocspt4.tex|,
|cdocsdrf.tex|, |cdocsfn1.tex|, |cdocsfn2.tex|
as well as |childdoc.pdf|.

%%%%%%%%%%%%%%%%%%%%%%%%%%%%%%%%%%%%%%%%%%%%%%%%%%%%%%%%%%%%%%%%%%%%%%%%%%%%%%%%
\subsection{Files and Installation}

The package consists of the files:
%
\begin{center}
\begin{tabular}{ll}
    |README.txt|   & readme file \\
    |childdoc.ins| & installation file \\
    |childdoc.dtx| & source file \\
    |childdoc.def| & definition file \\
    |cdocsamp.tex| & sample main file \\
    |cdocsch1.tex| & sample include file \\
    |cdocsch2.tex| & sample include file \\
    |cdocspt3.tex| & sample part file \\
    |cdocspt4.tex| & sample part file \\
    |cdocsdrf.tex| & sample redirection file \\
    |cdocsfn1.tex| & sample redirection file \\
    |cdocsfn2.tex| & sample redirection file \\
    |childdoc.pdf| & manual
\end{tabular}
\end{center}
%
The distribution consists of the files
|README.txt|, |childdoc.ins| and |childdoc.dtx|.
%
\begin{itemize}
\item
Run (pdf)\LaTeX{} on |childdoc.dtx|
to compile the manual |childdoc.pdf| (this file).
\item
Run \LaTeX{} on |childdoc.ins| to create the definitions file |childdoc.def|
and the sample |cdocsamp.tex| with include files
|cdocsch1.tex|, |cdocsch2.tex|, |cdocspt3.tex|, |cdocspt4.tex|,
|cdocsdrf.tex|, |cdocsfn1.tex|, |cdocsfn2.tex|.
Then copy the file |childdoc.def| to an appropriate directory of your \LaTeX{}
distribution, e.g.\ \textit{texmf-root}|/tex/latex/childdoc|.
\end{itemize}

%%%%%%%%%%%%%%%%%%%%%%%%%%%%%%%%%%%%%%%%%%%%%%%%%%%%%%%%%%%%%%%%%%%%%%%%%%%%%%%%
\subsection{Related CTAN Packages}

There are several other packages which offer a similar functionality:
%
\begin{itemize}
\item
The packages
\href{http://ctan.org/pkg/docmute}{\textsf{docmute}},
\href{http://ctan.org/pkg/includex}{\textsf{includex}} and
\href{http://ctan.org/pkg/standalone}{\textsf{standalone}}
provide commands to include only the document body of
a child file thus allowing both files to be compiled individually.
\item
The packages \href{http://ctan.org/pkg/subdocs}{\textsf{subdocs}}
and \href{http://ctan.org/pkg/subfiles}{\textsf{subfiles}}
provide structures in which the main and child documents can be
encapsulated and allowing them to be compiled individually.
The inclusion mechanism is different from the conventional |\include|.
\item
The package \href{http://ctan.org/pkg/combine}{\textsf{combine}}
is an elaborate solution to combine several documents into one.
\end{itemize}
%
See also the CTAN topic \href{http://ctan.org/topic/subdocs}{\textsf{subdocs}}
for further related packages.
The present package differs from the above solutions in that
a document structure constructed with the conventional |\include| mechanism
just needs two extra commands at the top of every file
such that all constituent files can be compiled individually.

%%%%%%%%%%%%%%%%%%%%%%%%%%%%%%%%%%%%%%%%%%%%%%%%%%%%%%%%%%%%%%%%%%%%%%%%%%%%%%%%
%\subsection{Feature Suggestions}
%
%The following is a list of features which may be useful for future
%versions of this package:
%%
%\begin{itemize}
%\item
%\ldots
%\end{itemize}

%%%%%%%%%%%%%%%%%%%%%%%%%%%%%%%%%%%%%%%%%%%%%%%%%%%%%%%%%%%%%%%%%%%%%%%%%%%%%%%%
\subsection{Revision History}

%%%%%%%%%%%%%%%%%%%%%%%%%%%%%%%%%%%%%%%%
\paragraph{v2.0:} 2018/12/30

\begin{itemize}
\item
immediate forward processing
\item
added |\childdocby| mechanism
\item
manual restructured
\end{itemize}

%%%%%%%%%%%%%%%%%%%%%%%%%%%%%%%%%%%%%%%%
\paragraph{v1.6:} 2018/01/17

\begin{itemize}
\item
application for development of include files
\item
corrections to manual
\end{itemize}

%%%%%%%%%%%%%%%%%%%%%%%%%%%%%%%%%%%%%%%%
\paragraph{v1.5:} 2017/05/21

\begin{itemize}
\item
more complete structuring introduced
\item
|\childdocof| introduced
\item
|\childdoc| renamed to |\childdocmain|
\item
|\childredirect| renamed to |\childdocforward| and |\childdocforwardprefix|
and functionality expanded
\end{itemize}

%%%%%%%%%%%%%%%%%%%%%%%%%%%%%%%%%%%%%%%%
\paragraph{v1.0:} 2017/04/27

\begin{itemize}
\item
manual and install package
\item
first version published on CTAN
\end{itemize}

%%%%%%%%%%%%%%%%%%%%%%%%%%%%%%%%%%%%%%%%
\paragraph{v0.6:} 2017/04/26

\begin{itemize}
\item
redirection mechanism added
\end{itemize}

%%%%%%%%%%%%%%%%%%%%%%%%%%%%%%%%%%%%%%%%
\paragraph{v0.5:} 2017/04/26

\begin{itemize}
\item
functionality in definition file
\end{itemize}


%%%%%%%%%%%%%%%%%%%%%%%%%%%%%%%%%%%%%%%%%%%%%%%%%%%%%%%%%%%%%%%%%%%%%%%%%%%%%%%%
%%%%%%%%%%%%%%%%%%%%%%%%%%%%%%%%%%%%%%%%%%%%%%%%%%%%%%%%%%%%%%%%%%%%%%%%%%%%%%%%
%%%%%%%%%%%%%%%%%%%%%%%%%%%%%%%%%%%%%%%%%%%%%%%%%%%%%%%%%%%%%%%%%%%%%%%%%%%%%%%%
\appendix

\settowidth\MacroIndent{\rmfamily\scriptsize 000\ }

 \DocInput{childdoc.dtx}

\end{document}
%</driver>
% \fi
%
% %%%%%%%%%%%%%%%%%%%%%%%%%%%%%%%%%%%%%%%%%%%%%%%%%%%%%%%%%%%%%%%%%%%%%%%%%%%%%%
% %%%%%%%%%%%%%%%%%%%%%%%%%%%%%%%%%%%%%%%%%%%%%%%%%%%%%%%%%%%%%%%%%%%%%%%%%%%%%%
% \section{Sample}
%\iffalse
%<*samplemain>
%\fi
%
% The following presents a sample document
% with two chapters, two parts, a title page,
% a compile flag as well as three forwarding files to set the flag.
% It consists of eight |.tex| files:
% \begin{center}
% \begin{tabular}{ll}
% |cdocsamp.tex|&main file\\
% |cdocsch1.tex|&include file for chapter 1\\
% |cdocsch2.tex|&include file for chapter 2\\
% |cdocspt3.tex|&include file for part 3\\
% |cdocspt4.tex|&include file for part 4\\
% |cdocsdrf.tex|&forwarding file for main file in draft mode\\
% |cdocsfi1.tex|&forwarding file for final version of chapter 1\\
% |cdocsfi2.tex|&forwarding file for final version of chapter 2\\
% \end{tabular}
% \end{center}
% Each of the eight files can be compiled directly by the \LaTeX{} compiler.
%
% %%%%%%%%%%%%%%%%%%%%%%%%%%%%%%%%%%%%%%
% \paragraph{Main File.}
%
% The main file is called |cdocsamp.tex|.
%
% Load the \textsf{childdoc} definitions and
% declare the filename for the main document:
%    \begin{macrocode}
\input{childdoc.def}
\childdocmain{}
%    \end{macrocode}

% Optional override for |\version| flag:
%    \begin{macrocode}
%%\ifchilddoc\else\providecommand{\version}{draft}\fi
%    \end{macrocode}

% Define the default values for the |\version| flag
% (|final| for the main file and |draft| for childs):
%    \begin{macrocode}
\ifchilddoc
\providecommand{\version}{draft}
\else
\providecommand{\version}{final}
\fi
%    \end{macrocode}

% Load the standard document class:
%    \begin{macrocode}
\documentclass[12pt]{article}
%    \end{macrocode}

% Start the document body:
%    \begin{macrocode}
\begin{document}
%    \end{macrocode}

% Declare a title page.
% Print title, part of document being processed and version flag:
%    \begin{macrocode}
\addtocounter{page}{-1}
\begin{center}
{\LARGE\bfseries{}childdoc example\par}
\vspace{1cm}
\ifchilddoc
\ifchilddocmanual part\else chapter\fi:
`\childdocname' of `\childdocjob'\par
\else
main document: `\childdocjob'\par
\fi
version: \version\par
\end{center}
\newpage
%    \end{macrocode}

% Manually include selected file,
% otherwise process as usual:
%    \begin{macrocode}
\ifchilddocmanual
\section*{part `\childdocname'}
\input{\childdocname}
\else
%    \end{macrocode}

% Include the two chapters:
%    \begin{macrocode}
\include{cdocsch1}
\include{cdocsch2}
%    \end{macrocode}

% Include the two parts unless only chapters should be displayed:
%    \begin{macrocode}
\ifchilddoc\else
\section{part three}
\input{cdocspt3}
\section{part four}
\input{cdocspt4}
\fi
%    \end{macrocode}

% Process as usual until here:
%    \begin{macrocode}
\fi
%    \end{macrocode}

% End of document body:
%    \begin{macrocode}
\end{document}
%    \end{macrocode}
%\iffalse
%</samplemain>
%\fi
%
% %%%%%%%%%%%%%%%%%%%%%%%%%%%%%%%%%%%%%%
% \paragraph{Chapter Include Files.}
%
% The include files are called |cdocsch1.tex| and |cdocsch2.tex|.
%
%\iffalse
%<*samplechap1|samplechap2>
%\fi

% Optional override for |\version| flag:
%    \begin{macrocode}
%%\providecommand{\version}{final}
%    \end{macrocode}

% Include the main document:
%    \begin{macrocode}
\input{childdoc.def}
\childdocof{cdocsamp}
%    \end{macrocode}

%\iffalse
%</samplechap1|samplechap2>
%\fi
%
%\iffalse
%<*samplechap1>
%\fi
% Some text for chapter 1:
%    \begin{macrocode}
\section{one}
some text in chapter one
%    \end{macrocode}

%\iffalse
%</samplechap1>
%\fi
% Some text for chapter 2:
%\iffalse
%<*samplechap2>
%\fi
%    \begin{macrocode}
\section{two}
more text in chapter two
%    \end{macrocode}

%\iffalse
%</samplechap2>
%\fi
%
% %%%%%%%%%%%%%%%%%%%%%%%%%%%%%%%%%%%%%%
% \paragraph{Part Include Files.}
%
% The include files are called |cdocspt3.tex| and |cdocspt4.tex|.
%
%\iffalse
%<*samplepart3|samplepart4>
%\fi

% Optional override for |\version| flag:
%    \begin{macrocode}
%%\providecommand{\version}{final}
%    \end{macrocode}

% Include the main document:
%    \begin{macrocode}
\input{childdoc.def}
\childdocby{cdocsamp}
%    \end{macrocode}

%\iffalse
%</samplepart3|samplepart4>
%\fi
%
%\iffalse
%<*samplepart3>
%\fi
% Some text for part 3:
%    \begin{macrocode}
some text in part three
%    \end{macrocode}

%\iffalse
%</samplepart3>
%\fi
% Some text for part 4:
%\iffalse
%<*samplepart4>
%\fi
%    \begin{macrocode}
more text in part four
%    \end{macrocode}

%\iffalse
%</samplepart4>
%\fi
%
% %%%%%%%%%%%%%%%%%%%%%%%%%%%%%%%%%%%%%%
% \paragraph{Forwarding for a Complete Draft.}
%
% The following forwarding file |cdocsdrf.tex|
% compiles the main document in draft mode:
%\iffalse
%<*sampledraft>
%\fi
%    \begin{macrocode}
\def\version{draft}
\input{childdoc.def}
\childdocforward{cdocsamp}
%    \end{macrocode}

%\iffalse
%</sampledraft>
%\fi
%
% %%%%%%%%%%%%%%%%%%%%%%%%%%%%%%%%%%%%%%
% \paragraph{Forwarding for Final Version of the Chapters.}
%
% The following forwarding files |cdocsfn1.tex| and |cdocsfn2.tex|
% (with identical content)
% compile the final versions of the child documents
% |cdocsch1.tex| and |cdocsch2.tex|, respectively:
%\iffalse
%<*samplefinal>
%\fi
%    \begin{macrocode}
\def\version{final}
\input{childdoc.def}
\childdocforwardprefix[cdocsamp]{cdocsfn}{cdocsch}
%    \end{macrocode}

%\iffalse
%</samplefinal>
%\fi
%
% %%%%%%%%%%%%%%%%%%%%%%%%%%%%%%%%%%%%%%
% \paragraph{Command Line Processing.}
%
% The following three command lines generate the output files
% |cdocscld|, |cdocscl1| and |cdocscl2|
% which should be identical to
% |cdocsdrf|, |cdocsch1| and |cdocsfn2|, respectively:
% \begin{center}
% \begin{tabular}{l}
% |latex -jobname cdocscld \|\\
% |  "\def\version{draft}\input{childdoc.def}\childdocforward{cdocsamp}"|\\
% |latex -jobname cdocscl1 \|\\
% |  "\input{childdoc.def}\childdocforward[cdocsamp]{cdocsch1}"|\\
% |latex -jobname cdocscl2 \|\\
% |  "\def\version{final}\input{childdoc.def}\childdocforward{cdocsch2}"|
% \end{tabular}
% \end{center}
% Note that the trailing backslash on each first line
% merely continues the input to the second line
% (for convenient cut ant paste).
% Furthermore, the command |latex| can be replaced by any
% of its alternative versions such as |pdflatex|.
%
% %%%%%%%%%%%%%%%%%%%%%%%%%%%%%%%%%%%%%%%%%%%%%%%%%%%%%%%%%%%%%%%%%%%%%%%%%%%%%%
% %%%%%%%%%%%%%%%%%%%%%%%%%%%%%%%%%%%%%%%%%%%%%%%%%%%%%%%%%%%%%%%%%%%%%%%%%%%%%%
% \section{Implementation}
%\iffalse
%<*package>
%\fi
%
% This section describes the definitions file |childdoc.def|.

% The definitions cannot be loaded using |\usepackage| or |\RequirePackage|
% which has a mechanism to prevent loading a style file more than once.
% When loading the definitions by means of |\input|
% multiple instances have to be prevented manually:
%\iffalse
%This code needs to be before the `\ProvidesFile' directive
%which is defined at the beginning of this file.
%Therefore it is also placed there and commented out here.
%</package>
%<*discard>
%\fi
%    \begin{macrocode}
\ifdefined\childdocmain\endinput\fi
%    \end{macrocode}
%\iffalse
%</discard>
%<*package>
%\fi
%
% \macro{\ifchilddoc}
% \macro{\ifchilddocmanual}
% The conditional |\ifchilddoc| tells whether a
% child (true) or main (false) document is being compiled.
% The conditional |\ifchilddocmanual| tells whether
% the |\includeonly| mechanism is used (false) or
% the selection of child files must be performed manually (true).
% The definitions initialise to false:
%    \begin{macrocode}
\newif\ifchilddoc
\newif\ifchilddocmanual
%    \end{macrocode}

% \macro{\childdocname}
% \macro{\childdocjob}
% The macro |\childdocname| stores the name of the main document
% to be compiled. The macro |\childdocjob| stores the name of
% the document on which the \LaTeX{} compiler was originally invoked.
% The content of |\jobname| cannot be compared
% to filenames specified in the source due to different catcodes.
% The following code rescans |\jobname|, stores the result
% in |\childdocname| and saves a copy in |\childdocjob|:
%    \begin{macrocode}
\edef\childdocname{\scantokens\expandafter{\jobname\noexpand}}
\let\childdocjob\childdocname
%    \end{macrocode}

% \macro{\childdocdisable}
% The macro |\childdocdisable| prevents the main file
% from being processed more than once.
% At this stage, the main document command |\childdocmain|
% is assumed to be called once again where it should do nothing.
% Any subsequent call to it should prevent
% a secondary processing of the main document
% It overwrites the forwarding commands
% |\childdocof| and |\childdocforward|
% with empty macros to prevent further inclusions of the main document:
%    \begin{macrocode}
\newcommand{\childdocdisable}
{
  \renewcommand{\childdocmain}[1]{\renewcommand{\childdocmain}[1]{\endinput}}
  \renewcommand{\childdocof}[1]{}
  \renewcommand{\childdocby}[2][]{}
  \renewcommand{\childdocforward}[2][]{}
  \renewcommand{\childdocdisable}{}
}
%    \end{macrocode}

% \macro{\childdocmain}
% The macro |\childdocmain| is to be called at the top of the main file
% with nothing or the main filename (without extension) as argument.
% First, it breaks loops.
% If the argument is not empty and does not match |\childdocname|
% (which is set by the first inclusion of |childdoc.def|),
% |\ifchilddoc| is set to true, |\includeonly| is applied to the child file
% and |\jobname| is set to the main file
% (for proper handling of |.aux| files):
%    \begin{macrocode}
\newcommand{\childdocmain}[1]
{
  \childdocdisable\childdocmain{}
  \if?#1?\else
    \begingroup
      \def\childdoctmp{#1}
      \ifx\childdoctmp\childdocname
        \def\childdoctmp{}
      \else
        \def\childdoctmp
        {
          \childdoctrue
          \includeonly{\childdocname}
          \def\childdocjob{#1}
          \def\jobname{#1}
        }
      \fi
      \expandafter
    \endgroup
    \childdoctmp
  \fi
}
%    \end{macrocode}

% \macro{\childdocof}
% The command |\childdocof| redirects
% compilation to the main file |#1|.
%    \begin{macrocode}
\newcommand{\childdocof}[1]
{
  \childdocdisable
  \childdoctrue
  \includeonly{\childdocname}
  \def\jobname{#1}
  \def\childdocjob{#1}
  \input{#1}
}
%    \end{macrocode}

% \macro{\childdocby}
% The command |\childdocby| ....
%    \begin{macrocode}
\newcommand{\childdocby}[2][]
{
  \childdocdisable
  \childdoctrue
  \childdocmanualtrue
  \if?#1?\else
    \def\jobname{#2}
  \fi
  \def\childdocjob{#2}
  \input{#2}
  \endinput
}
%    \end{macrocode}

% \macro{\childdocforward}
% The command |\childdocforward| redirects
% compilation to the main file or
% (if the optional argument is given) a child file.
% Parameters are set as if the main file
% or a child file starting with |\childdocof| was compiled.
% Then compilation is handed over to the main file:
%    \begin{macrocode}
\newcommand{\childdocforward}[2][]
{
  \begingroup
    \if?#1?
      \def\childdoctmp
      {
        \def\childdocname{#2}
        \def\childdocjob{#2}
        \def\jobname{#2}
        \input{#2}
        \endinput
      }
    \else
      \def\childdoctmp
      {
        \childdocdisable
        \def\childdocname{#2}
        \childdoctrue
        \includeonly{#2}
        \def\childdocjob{#1}
        \def\jobname{#1}
        \input{#1}
        \endinput
      }
    \fi
    \expandafter
  \endgroup
  \childdoctmp
}
%    \end{macrocode}

% \macro{\childdocforwardprefix}
% The command |\childdocforwardprefix| redirects
% compilation to the main or a child file by means of a pattern.
% The prefix |#1| in the current filename is replaced by |#2|
% and the suffix of the current filename is kept
% (it is assumed that the filename does not contain the substring `|~~~|'
% which is used as a delimiter).
% Compilation is handed over to the new file by |\childdocforward|:
%    \begin{macrocode}
\newcommand{\childdocforwardprefix}[3][]
{
  \begingroup
    \def\childdocextract #2##1~~~{\def\childdoctmp{\childdocforward[#1]{#3##1}}}
    \expandafter\childdocextract\childdocname~~~
    \expandafter
  \endgroup
  \childdoctmp
}
%    \end{macrocode}

% \macro{\childdoc}
% The deprecated macro |\childdoc| is a legacy version of |\childdocmain|:
%    \begin{macrocode}
\newcommand{\childdoc}{\childdocmain}
%    \end{macrocode}

% \macro{\childdocredirect}
% The deprecated macro |\childdocredirect| is a legacy version
% of |\childdocforward| and |\childdocforwardprefix|:
%    \begin{macrocode}
\newcommand{\childdocredirect}[2][]
{
  \begingroup
    \if?#1?
      \def\childdoctmp{\childdocforward{#2}}
    \else
      \def\childdoctmp{\childdocforwardprefix{#1}{#2}}
    \fi
    \expandafter
  \endgroup
  \childdoctmp
}
%    \end{macrocode}

%\iffalse
%</package>
%\fi
%
\endinput

\childdocof{cdocsamp}
%    \end{macrocode}

%\iffalse
%</samplechap1|samplechap2>
%\fi
%
%\iffalse
%<*samplechap1>
%\fi
% Some text for chapter 1:
%    \begin{macrocode}
\section{one}
some text in chapter one
%    \end{macrocode}

%\iffalse
%</samplechap1>
%\fi
% Some text for chapter 2:
%\iffalse
%<*samplechap2>
%\fi
%    \begin{macrocode}
\section{two}
more text in chapter two
%    \end{macrocode}

%\iffalse
%</samplechap2>
%\fi
%
% %%%%%%%%%%%%%%%%%%%%%%%%%%%%%%%%%%%%%%
% \paragraph{Part Include Files.}
%
% The include files are called |cdocspt3.tex| and |cdocspt4.tex|.
%
%\iffalse
%<*samplepart3|samplepart4>
%\fi

% Optional override for |\version| flag:
%    \begin{macrocode}
%%\providecommand{\version}{final}
%    \end{macrocode}

% Include the main document:
%    \begin{macrocode}
% \iffalse
%
% childdoc.dtx Copyright (C) 2017-2018 Niklas Beisert
%
% This work may be distributed and/or modified under the
% conditions of the LaTeX Project Public License, either version 1.3
% of this license or (at your option) any later version.
% The latest version of this license is in
%   http://www.latex-project.org/lppl.txt
% and version 1.3 or later is part of all distributions of LaTeX
% version 2005/12/01 or later.
%
% This work has the LPPL maintenance status `maintained'.
%
% The Current Maintainer of this work is Niklas Beisert.
%
% This work consists of the files childdoc.dtx and childdoc.ins
% and the derived files childdoc.def and cdocsamp.tex with
% cdocsch1.tex, cdocsch2.tex, cdocsdrf.tex, cdocsfn1.tex, cdocsfn2.tex.
%
%<package>\ifdefined\childdocmain\endinput\fi
%<package>\ProvidesFile{childdoc.def}[2018/12/30 v2.0 child document driver]
%<samplemain>\ProvidesFile{cdocsamp.tex}[2018/12/30 v2.0 sample for childdoc]
%<*driver>
%\ProvidesFile{childdoc.drv}[2018/12/30 v2.0 childdoc reference manual file]
\PassOptionsToClass{10pt,a4paper}{article}
\documentclass{ltxdoc}

\usepackage[margin=35mm]{geometry}
\usepackage{hyperref}
\usepackage{hyperxmp}
\usepackage[usenames]{color}

\hypersetup{colorlinks=true}
\hypersetup{pdfstartview=FitH}
\hypersetup{pdfpagemode=UseNone}
\hypersetup{pdfsource={}}
\hypersetup{pdflang={en-UK}}
\hypersetup{pdfcopyright={Copyright 2017-2018 Niklas Beisert.
  This work may be distributed and/or modified under the
  conditions of the LaTeX Project Public License, either version 1.3
  of this license or (at your option) any later version.}}
\hypersetup{pdflicenseurl={http://www.latex-project.org/lppl.txt}}
\hypersetup{pdfcontactaddress={ETH Zurich, ITP, HIT K,
  Wolfgang-Pauli-Strasse 27}}
\hypersetup{pdfcontactpostcode={8093}}
\hypersetup{pdfcontactcity={Zurich}}
\hypersetup{pdfcontactcountry={Switzerland}}
\hypersetup{pdfcontactemail={nbeisert@itp.phys.ethz.ch}}
\hypersetup{pdfcontacturl={http://people.phys.ethz.ch/\xmptilde nbeisert/}}

\newcommand{\secref}[1]{\hyperref[#1]{section \ref*{#1}}}

\parskip1ex
\parindent0pt
\let\olditemize\itemize
\def\itemize{\olditemize\parskip0pt}

\begin{document}

\title{The \textsf{childdoc} Package}
\hypersetup{pdftitle={The childdoc Package}}
\author{Niklas Beisert\\[2ex]
  Institut f\"ur Theoretische Physik\\
  Eidgen\"ossische Technische Hochschule Z\"urich\\
  Wolfgang-Pauli-Strasse 27, 8093 Z\"urich, Switzerland\\[1ex]
  \href{mailto:nbeisert@itp.phys.ethz.ch}
  {\texttt{nbeisert@itp.phys.ethz.ch}}}
\hypersetup{pdfauthor={Niklas Beisert}}
\hypersetup{pdfsubject={Manual for the LaTeX2e Package childdoc}}
\date{30 December 2018, \textsf{v2.0}}
\maketitle

\begin{abstract}\noindent
\textsf{childdoc} is a \LaTeXe{} package
that enables the direct compilation
of document sections included by |\include|
to individual files.
\end{abstract}

\begingroup
\parskip0ex
\tableofcontents
\endgroup

%%%%%%%%%%%%%%%%%%%%%%%%%%%%%%%%%%%%%%%%%%%%%%%%%%%%%%%%%%%%%%%%%%%%%%%%%%%%%%%%
%%%%%%%%%%%%%%%%%%%%%%%%%%%%%%%%%%%%%%%%%%%%%%%%%%%%%%%%%%%%%%%%%%%%%%%%%%%%%%%%
\section{Introduction}

\LaTeX{} provides a mechanism to structure a large document (such as a book)
into a main file and several child files (containing the chapters)
using the |\include| command.
This mechanism is beneficial for documents
which span hundreds of pages in order to
make the source file(s) more manageable.
Moreover, compilation can be restricted to
selected child files by means of the |\includeonly| command.
The latter feature can be used to reduce the compilation time while editing
(this was significantly more useful in the earlier days of \LaTeX{})
or to generate a smaller document which is easier to navigate.
Another application of |\includeonly| is to generate
documents consisting of selected parts of the complete document.

However, there are a few drawbacks of the plain |\include| mechanism:
\begin{itemize}
\item
The child files cannot be compiled on their own,
they can only be compiled via the main file.
A naive editing environment
(such as a text editor with an option
to have the current file processed by \LaTeX)
may require one to switch to the main file before compiling;
attempting to compile the child file produces errors.
\item
The main file must be modified (each time)
to adjust the |\includeonly| command
to the present needs. This easily leaves the main file in a messy state.
\item
The generated document will always carry the filename
of the main document. This is inconvenient if
several child files are to be compiled and
to be kept for distribution.
\end{itemize}

The present package provides a simple interface
to make child files individually compilable by \LaTeX{}.
Compiling a child file then has the same effect as compiling
the main file with an |\includeonly| command
to select the appropriate child.
Moreover the generated document will carry the name of the child
rather than the main file.
This resolves all three above issues.

This feature is meant to make the editing of books,
thesis documents and lecture notes somewhat more convenient.
However, the package can also be used efficiently for
composing a series of documents (such as exercise sheets)
which are typically distributed individually.
It then assists the author in generating the individual documents
(potentially in different versions)
as well as a document containing the collected series.
Another application is in developing style files
or other kinds of included material
where compilation of the style file could redirect
to a sample or test file.

%%%%%%%%%%%%%%%%%%%%%%%%%%%%%%%%%%%%%%%%%%%%%%%%%%%%%%%%%%%%%%%%%%%%%%%%%%%%%%%%
%%%%%%%%%%%%%%%%%%%%%%%%%%%%%%%%%%%%%%%%%%%%%%%%%%%%%%%%%%%%%%%%%%%%%%%%%%%%%%%%
\section{Usage}

First of all, the package \textsf{childdoc} is \emph{not} a standard
\LaTeXe{} |.sty| style file! Therefore it needs to be invoked in
a non-standard way.

%%%%%%%%%%%%%%%%%%%%%%%%%%%%%%%%%%%%%%%%%%%%%%%%%%%%%%%%%%%%%%%%%%%%%%%%%%%%%%%%
\subsection{Included Files}
\label{sec:include}

%%%%%%%%%%%%%%%%%%%%%%%%%%%%%%%%%%%%%%%%
\DescribeMacro{\childdocmain}
To use the package, add the commands
\begin{center}
\begin{tabular}{l}
|\input{childdoc.def}|\\
|\childdocmain{}|\\
\end{tabular}
\end{center}
at the very top of the main \LaTeX{} file,
in particular \emph{before} the |\documentclass| statement!
The argument of |\childdocmain| should be left empty
(but it must be present).

%%%%%%%%%%%%%%%%%%%%%%%%%%%%%%%%%%%%%%%%
\DescribeMacro{\childdocof}
Furthermore, add the commands
\begin{center}
\begin{tabular}{l}
|\input{childdoc.def}|\\
|\childdocof{|\textit{main}|}|\\
\end{tabular}
\end{center}
at the top of every child file \textit{child}
which is included by |\include{|\textit{child}|}|
from within the main file
(or at least for those files to be compiled individually).
The argument \textit{main} must be the filename of the main file.

There are a couple of
considerations in setting up the main and child documents:

%%%%%%%%%%%%%%%%%%%%%%%%%%%%%%%%%%%%%%%%
\paragraph{Restrictions.}

Please note the following restrictions:
\begin{itemize}
\item
|\childdocmain| must be called with one argument \textit{main}
to ensure compatibility with earlier version of the package.
It must either be empty (|\childdocmain{}|)
or precisely match the filename of the main file in which it is specified.
See \secref{sec:detection} for further information.
\item
The filename \textit{main} must be specified without the |.tex| extension.
\item
The filename \textit{main} is case sensitive
(even in case-insensitive file systems)
due to internal string comparison.
\item
The argument \textit{main} should be fully expanded, it cannot be a macro.
\item
Subdirectories and special characters should be avoided in filenames.
\item
The command |\childdocmain{|\textit{main}|}| must be followed by a whitespace.
It should not be followed immediately by another command
or by a comment mark `|%|'.
This is because the \TeX{} parser reads the token immediately following
the argument of |\childdocmain| and puts it
at the beginning of every child section;
however, a white\-space is ignored.
\end{itemize}

%%%%%%%%%%%%%%%%%%%%%%%%%%%%%%%%%%%%%%%%
\paragraph{Content of Main File.}

It is advisable to place all content in the child files included by |\include|.
Any output contained in the main file will appear in all child documents
unless suppressed manually;
it cannot be suppressed automatically by the |\includeonly| directive
and thus should normally be avoided.
A method to include some content in the main file
by means of conditional processing is described in \secref{sec:conditional}.

%%%%%%%%%%%%%%%%%%%%%%%%%%%%%%%%%%%%%%%%
\paragraph{Page Numbering.}

When only a part of the document is compiled,
the appropriate numbering of pages
(as well as other status parameters)
is determined from the |.aux| files.
The latter contain information from previous passes.
However this information needs to propagate through
all intermediate child documents.
Therefore the page numbering in child documents may well
be inconsistent until the complete document is compiled at least once.

A useful (if unconventional) way to always ensure a consistent
page numbering is to restart the numbering in each child document
and denote the pages by `\textit{child}|.|\textit{page}'
where \textit{child} represents the chapter/section number of the child file.
This can be achieved by the command
|\numberwithin{page}{|\textit{child}|}|
of the \textsf{amsmath} package
where \textit{child} can be |chapter| or |section|
depending on the chosen structuring.
Alternatively, one can modify the macro |\thepage| appropriately
and reset the counter |page| at the start of each child file.

%%%%%%%%%%%%%%%%%%%%%%%%%%%%%%%%%%%%%%%%%%%%%%%%%%%%%%%%%%%%%%%%%%%%%%%%%%%%%%%%
\subsection{Conditional Processing}
\label{sec:conditional}

The package provides a mechanism to compile different versions
of a document. To customise the versions further some conditional processing
can come in handy to distinguish which version is being compiled.
The package provides two macros to describe the compilation context:

%%%%%%%%%%%%%%%%%%%%%%%%%%%%%%%%%%%%%%%%
\DescribeMacro{\ifchilddoc}
The conditional |\ifchilddoc| distinguishes between the compilation of
child documents and the main document:
%
\begin{center}
|\ifchilddoc |\textit{child-code}| |[|\||else |\textit{main-code}]| \||fi|
\end{center}

%%%%%%%%%%%%%%%%%%%%%%%%%%%%%%%%%%%%%%%%
\DescribeMacro{\childdocname}
\DescribeMacro{\childdocjob}
The macro |\childdocname| contains the filename (without extension)
of the main or child file being processed.
Note that |\childdocjob| will always contain the name of the main file.

%%%%%%%%%%%%%%%%%%%%%%%%%%%%%%%%%%%%%%%%
\paragraph{Title Page.}

Conditional processing can be used to include a title or banner page
in the main document when proper precautions are taken.
Importantly, the code in the main file should ensure that the page counter
(as well as other status parameters which are stored in the |.aux| files)
takes the same value after the conditional processing.
Otherwise the page numbers may take divergent values
depending on which part is compiled.

For example, a title page could be declared by:
%
\begin{center}
\begin{tabular}{l}
|\ifchilddoc\||else|\\
|\addtocounter{page}{-1}|\\
\textit{code for title page}\\
|\newpage|\\
|\||fi|
\end{tabular}
\end{center}
%
A banner page for the child documents can be generated by:
%
\begin{center}
\begin{tabular}{l}
|\ifchilddoc|\\
|\addtocounter{page}{-1}|\\
\textit{code for banner page}\\
|\newpage|\\
|\||fi|
\end{tabular}
\end{center}
%
Here one could write a message such as:
\begin{center}
|This is the part \childdocname{} of \childdocjob{}.|
\end{center}

%%%%%%%%%%%%%%%%%%%%%%%%%%%%%%%%%%%%%%%%%%%%%%%%%%%%%%%%%%%%%%%%%%%%%%%%%%%%%%%%
\subsection{Flags}
\label{sec:flags}

The package makes it easy to generate different versions
of the main or child documents.
To this end compilation flags can be defined
and assigned different default values.
They will be particularly useful in conjunction
with the forwarding mechanism described in \secref{sec:forward}.

For example, it may be useful to have a flag |\version|
which can be set to |draft| or |final|.
The document source will contain some conditional code
depending on the value of |\version|.
Suppose further, the flag should default to |final| for the main file
and to |draft| for child files
which is a natural assignment for editing the document.
This is achieved by placing the following code
in the preamble of the main document
(below the |\childdocmain| directive):
%
\begin{center}
\begin{tabular}{l}
|\ifchilddoc|\\
|\providecommand{\version}{draft}|\\
|\||else|\\
|\providecommand{\version}{final}|\\
|\||fi|
\end{tabular}
\end{center}
%
The definition by |\providecommand| makes sure
that previous definitions are not overwritten.
Further statements |\providecommand{\version}{...}|
can thus be added before the above code to override it.

For the main file, one might add a line
(between |\childdocmain| and the above block)
%
\begin{center}
|%\ifchilddoc\||else\providecommand{\version}{draft}\||fi|
\end{center}
%
which can be uncommented to produce a draft version.
Likewise one can add a line to the very top of a child file
(above the |\childdocof{|\textit{main}|}| directive)
%
\begin{center}
|%\providecommand{\version}{final}|
\end{center}
%
which can be uncommented to produce the final version of this child document.

%%%%%%%%%%%%%%%%%%%%%%%%%%%%%%%%%%%%%%%%%%%%%%%%%%%%%%%%%%%%%%%%%%%%%%%%%%%%%%%%
\subsection{Forwarding}
\label{sec:forward}

Different versions of the main or child documents
using compilation flags as described in \secref{sec:flags}
can be (permanently) stored in different files
for convenient compilation, viewing and distribution.
To this end, the package defines a command
to pass on compilation to a different file:

%%%%%%%%%%%%%%%%%%%%%%%%%%%%%%%%%%%%%%%%
\DescribeMacro{\childdocforward}
The command |\childdocforward| redirects processing to
another source file:
%
\begin{center}
\begin{tabular}{l}
|\input{childdoc.def}|\\
|\childdocforward[|\textit{main}|]{|\textit{dest}|}|\\
\end{tabular}
\end{center}
%
The argument \textit{dest} is the destination file
(without extension).
It should be the main file or one of the child files.
Note that further \textsf{childdoc} directives
such as |\childdocof| and |\childdocforward|
in the indicated file will be processed in this form.
The optional argument \textit{main}
passes on directly to the main file \textit{main}
while pretending to compile the child \textit{dest}.
This form behaves as if \textit{dest}
issues |\childdocof{|\textit{main}|}| right away,
and no further \textsf{childdoc} directives will be processed.

%%%%%%%%%%%%%%%%%%%%%%%%%%%%%%%%%%%%%%%%
\DescribeMacro{\...prefix}
In the alternative form |\childdocforwardprefix|,
%
\begin{center}
\begin{tabular}{l}
|\input{childdoc.def}|\\
|\childdocforwardprefix[|\textit{main}|]{|\textit{prefix}|}{|\textit{dest}|}|
\end{tabular}
\end{center}
%
the destination file is determined by a pattern
depending on the current file:
To make this work, the current file must be called
`{\textit{prefix}\hspace{0.2em}\textit{suffix}}'
with \textit{prefix} matching precisely the argument.
Processing is then passed on to the file
`{\textit{dest}\hspace{0.2em}\textit{suffix}}'.
Surely, the same effect is achieved by
directly specifying the
argument `{\textit{dest}\hspace{0.2em}\textit{suffix}}'
in the first form.
However, that requires to set up a different file
for each child. With the alternative form of the command
all these files can have exactly the same content
which simplifies setting them up and maintaining them.

For example, the following file |draft.tex|
with a compilation flag |\version| as described in \secref{sec:flags}
compiles the main document as a draft:
%
\begin{center}
\begin{tabular}{l}
|\def\version{draft}|\\
|\input{childdoc.def}|\\
|\childdocforward{|\textit{main}|}|
\end{tabular}
\end{center}
%
Likewise, the following files |final|\textit{nn}|.tex|
compile the final version of the child document
|child|\textit{nn}|.tex|:
%
\begin{center}
\begin{tabular}{l}
|\def\version{final}|\\
|\input{childdoc.def}|\\
|\childdocforwardprefix{final}{child}|
\end{tabular}
\end{center}
%

Note that when several versions of a main file and/or of each child file
are to be generated, it may be convenient to set up a |Makefile| or
shell script to automatise the process.

%%%%%%%%%%%%%%%%%%%%%%%%%%%%%%%%%%%%%%%%%%%%%%%%%%%%%%%%%%%%%%%%%%%%%%%%%%%%%%%%
\subsection{Command Line Processing}
\label{sec:commandline}

The effect of redirection files can also be achieved by invoking
the \LaTeX{} compiler with a more elaborate command line.
Most conveniently this should be done as part
of a shell script or a |Makefile|.

When using \textsf{childdoc} in the main file, the following
command lines effectively perform a redirection
(note that depending on the shell being used,
backslashes may have to be doubled: `|\|' $\to$ `|\\|'):
%
\begin{center}
|... -jobname "|\textit{target}|" |\\|"|[\textit{flags}]%
|\input{childdoc.def}\childdocforward[|\textit{main}|]{|\textit{dest}|}"|
\end{center}
%
Here \textit{target} is the name of the output file,
\textit{main} is the name of the main file
and \textit{dest} is the name of the main or child file to be processed
(all filenames without extensions).
The optional argument \textit{main} can be omitted
if \textit{main} matches \textit{dest}.
Optionally, compilation \textit{flags} can be defined via |\def| commands.
This command line makes the \TeX{} engine believe
it is compiling the file \textit{target}
whose content is specified as the latter parameter.
The provided code then forwards the processing to
\textit{main} or \textit{dest} as described in \secref{sec:forward}.

%%%%%%%%%%%%%%%%%%%%%%%%%%%%%%%%%%%%%%%%%%%%%%%%%%%%%%%%%%%%%%%%%%%%%%%%%%%%%%%%
\subsection{Include by Input}
\label{sec:input}

Including child documents by |\include| has some restrictions by design.
Most notably, the content of a child document always occupies
its own set of pages; pages cannot be shared between child documents.
Usually, this behaviour makes perfect sense
because each child document contain an essential part of the document.
However, in some situations it may be desirable to compose
a document from a collection of parts
without having mandatory page breaks between then.
For this case, the package
provides a mechanism to include parts
by |\input| which can also be processed individually.
However, by construction this mechanism
requires manual handling of the content to be output.

%%%%%%%%%%%%%%%%%%%%%%%%%%%%%%%%%%%%%%%%
\DescribeMacro{\ifchilddocmanual}
The main file should be prepared as usual, see \secref{sec:include}.
However, the document body must make a distinction
between processing of an individual part and of the main document, e.g.:
%
\begin{center}
\begin{tabular}{l}
|\ifchilddocmanual|\\
|\input{\childdocname}|\\
|\||else|\\
\textit{document body with }|\input{|\textit{part}|}|\\
|\||fi|
\end{tabular}
\end{center}
%
The conditional |\ifchilddocmanual| is true whenever
a part to be included by |\input| is being compiled,
and the name of the part is stored in |\childdocname|.

%%%%%%%%%%%%%%%%%%%%%%%%%%%%%%%%%%%%%%%%
\DescribeMacro{\childdocby}
Each part to be included by |\input| should start with:
%
\begin{center}
\begin{tabular}{l}
|\input{childdoc.def}|\\
|\childdocby{|\textit{main}|}|\\
\end{tabular}
\end{center}
%
The directive |\childdocby| is similar to |\childdocof|
described in \secref{sec:include},
but the subsequent selection of content must be done manually.
To that end, both |\ifchilddoc| and |\ifchilddocmanual|
will be true upon processing of a part,
and the name of the part is stored in |\childdocname|.
Note that |\jobname| will be set to the filename of the current part
so that each part receives an individual |.aux| file
that does not interfere with the |.aux| file(s) of the main document.
This behaviour can be altered by the alternative form
|\childdocby[*]{|\textit{main}|}| (with a non-empty optional argument)
which uses the |.aux| file of the main document
by setting |\jobname| to \textit{main}.

%%%%%%%%%%%%%%%%%%%%%%%%%%%%%%%%%%%%%%%%%%%%%%%%%%%%%%%%%%%%%%%%%%%%%%%%%%%%%%%%
\subsection{Driver Development}
\label{sec:driver}

The \textsf{childdoc} mechanism can also be use for the development
of definition files such as \LaTeX{} styles or classes.
This case differs from the above setup with multiple parts
included by |\include| in that no |\includeonly| should be invoked.
This can be achieved by starting the include file
(before |\ProvidesPackage|) with:
%
\begin{center}
\begin{tabular}{l}
|\input{childdoc.def}|\\
|\childdocforward{|\textit{main}|}|\\
\end{tabular}
\end{center}
%
or alternatively with:
%
\begin{center}
\begin{tabular}{l}
|\input{childdoc.def}|\\
|\childdocby{|\textit{main}|}|\\
\end{tabular}
\end{center}
%
Both forms have slightly different effects as described above.
The main file is prepared as usual, see \secref{sec:include}.

%%%%%%%%%%%%%%%%%%%%%%%%%%%%%%%%%%%%%%%%%%%%%%%%%%%%%%%%%%%%%%%%%%%%%%%%%%%%%%%%
\subsection{Legacy Detection}
\label{sec:detection}

The directive |\childdocmain| in the main file can detect
whether the complete document or merely a child is to be compiled
even without using the directive |\childdocof|.
This method is deprecated because it is less robust
and there is no compelling reason to use it;
it is merely provided for backward compatibility
and it may be removed in future versions.

If the detection mechanism is to be used,
it is mandatory to correctly specify
the filename of the main file as the argument of |\childdocmain|:
%
\begin{center}
\begin{tabular}{l}
|\input{childdoc.def}|\\
|\childdocmain{|\textit{main}|}|\\
\end{tabular}
\end{center}
%
If |\jobname| does not match the argument \textit{main} of |\childdocmain|,
it is assumed that |\jobname| points to the child file to be compiled.
When using |\childdocmain| with the main file specified as argument,
it suffices to start a child file
with just |\input{|\textit{main}|}|
without loading of the package and using |\childdocof|.
If instead all processing is done
with the appropriate \textsf{childdoc} directives,
the argument of \textit{main} of |\childdocmain| can be empty.

An alternative version of the command line processing described
in \secref{sec:commandline} using the detection mechanism reads:
%
\begin{center}
|... -jobname "|\textit{target}|" "|[\textit{flags}]%
[|\def\jobname{|\textit{dest}|}|]|\input{|\textit{main}|}"|
\end{center}

%%%%%%%%%%%%%%%%%%%%%%%%%%%%%%%%%%%%%%%%%%%%%%%%%%%%%%%%%%%%%%%%%%%%%%%%%%%%%%%%
\subsection{Manual Code}
\label{sec:manual}

In case one cannot be certain whether the definitions file |childdoc.def|
is installed on the target \TeX{} distribution
and one prefers not to ship it,
it is conceivable to paste a few relevant commands into the sources.

To that end, drop all statements |\input{childdoc.def}|
and perform the replacements as outlined below.
Instead of |\childdocmain{|\textit{main}|}| add the following code
to the top of the main file:
%
\begin{center}
\begin{tabular}{l}
|\||ifdefined\childdocname\endinput\||fi\newif\ifchilddoc|\\
|\edef\childdocname{\scantokens\expandafter{\jobname\noexpand}}|\\
|\def\childdocmain{|\textit{main}|}\||ifx\childdocmain\childdocname\||else|\\
|\childdoctrue\includeonly{\childdocname}\let\jobname\childdocmain\||fi|\\
\end{tabular}
\end{center}
%
Instead of |\childdocof{|\textit{main}|}| just include the main file
at the top of each child file:
%
\begin{center}
|\input{|\textit{main}|}|
\end{center}
%
A simple redirection |\childdocforward{|\textit{dest}|}| is achieved by:
%
\begin{center}
|\def\jobname{|\textit{dest}|}\input{\jobname}|
\end{center}
%
The redirection with prefix
|\childdocforwardprefix[|\textit{prefix}|]{|\textit{dest}|}|
is accomplished by:
%
\begin{center}
\begin{tabular}{l}
|{\edef\jobname{\scantokens\expandafter{\jobname\noexpand}}|\\
|\def\redirectjob |\textit{prefix}|#1~~~{\gdef\jobname{|\textit{dest}|#1}}|\\
|\expandafter\redirectjob\jobname~~~}\input{\jobname}|
\end{tabular}
\end{center}

In an alternative approach,
child documents can be compiled by a specific command line
without additional code or specific definitions:
%
\begin{center}
|... -jobname "|\textit{target}|" "|[\textit{flags}]%
|\includeonly{|\textit{dest}|}\input{|\textit{main}|}"|
\end{center}
%

%%%%%%%%%%%%%%%%%%%%%%%%%%%%%%%%%%%%%%%%%%%%%%%%%%%%%%%%%%%%%%%%%%%%%%%%%%%%%%%%
%%%%%%%%%%%%%%%%%%%%%%%%%%%%%%%%%%%%%%%%%%%%%%%%%%%%%%%%%%%%%%%%%%%%%%%%%%%%%%%%
\section{Information}

%%%%%%%%%%%%%%%%%%%%%%%%%%%%%%%%%%%%%%%%%%%%%%%%%%%%%%%%%%%%%%%%%%%%%%%%%%%%%%%%
\subsection{Copyright}

Copyright \copyright{} 2017--2018 Niklas Beisert

This work may be distributed and/or modified under the
conditions of the \LaTeX{} Project Public License, either version 1.3
of this license or (at your option) any later version.
The latest version of this license is in
  \url{http://www.latex-project.org/lppl.txt}
and version 1.3 or later is part of all distributions of \LaTeX{}
version 2005/12/01 or later.

This work has the LPPL maintenance status `maintained'.

The Current Maintainer of this work is Niklas Beisert.

This work consists of the files |README.txt|, |childdoc.ins| and |childdoc.dtx|
as well as the derived files |childdoc.def|, |cdocsamp.tex|
with |cdocsch1.tex|, |cdocsch2.tex|, |cdocspt3.tex|, |cdocspt4.tex|,
|cdocsdrf.tex|, |cdocsfn1.tex|, |cdocsfn2.tex|
as well as |childdoc.pdf|.

%%%%%%%%%%%%%%%%%%%%%%%%%%%%%%%%%%%%%%%%%%%%%%%%%%%%%%%%%%%%%%%%%%%%%%%%%%%%%%%%
\subsection{Files and Installation}

The package consists of the files:
%
\begin{center}
\begin{tabular}{ll}
    |README.txt|   & readme file \\
    |childdoc.ins| & installation file \\
    |childdoc.dtx| & source file \\
    |childdoc.def| & definition file \\
    |cdocsamp.tex| & sample main file \\
    |cdocsch1.tex| & sample include file \\
    |cdocsch2.tex| & sample include file \\
    |cdocspt3.tex| & sample part file \\
    |cdocspt4.tex| & sample part file \\
    |cdocsdrf.tex| & sample redirection file \\
    |cdocsfn1.tex| & sample redirection file \\
    |cdocsfn2.tex| & sample redirection file \\
    |childdoc.pdf| & manual
\end{tabular}
\end{center}
%
The distribution consists of the files
|README.txt|, |childdoc.ins| and |childdoc.dtx|.
%
\begin{itemize}
\item
Run (pdf)\LaTeX{} on |childdoc.dtx|
to compile the manual |childdoc.pdf| (this file).
\item
Run \LaTeX{} on |childdoc.ins| to create the definitions file |childdoc.def|
and the sample |cdocsamp.tex| with include files
|cdocsch1.tex|, |cdocsch2.tex|, |cdocspt3.tex|, |cdocspt4.tex|,
|cdocsdrf.tex|, |cdocsfn1.tex|, |cdocsfn2.tex|.
Then copy the file |childdoc.def| to an appropriate directory of your \LaTeX{}
distribution, e.g.\ \textit{texmf-root}|/tex/latex/childdoc|.
\end{itemize}

%%%%%%%%%%%%%%%%%%%%%%%%%%%%%%%%%%%%%%%%%%%%%%%%%%%%%%%%%%%%%%%%%%%%%%%%%%%%%%%%
\subsection{Related CTAN Packages}

There are several other packages which offer a similar functionality:
%
\begin{itemize}
\item
The packages
\href{http://ctan.org/pkg/docmute}{\textsf{docmute}},
\href{http://ctan.org/pkg/includex}{\textsf{includex}} and
\href{http://ctan.org/pkg/standalone}{\textsf{standalone}}
provide commands to include only the document body of
a child file thus allowing both files to be compiled individually.
\item
The packages \href{http://ctan.org/pkg/subdocs}{\textsf{subdocs}}
and \href{http://ctan.org/pkg/subfiles}{\textsf{subfiles}}
provide structures in which the main and child documents can be
encapsulated and allowing them to be compiled individually.
The inclusion mechanism is different from the conventional |\include|.
\item
The package \href{http://ctan.org/pkg/combine}{\textsf{combine}}
is an elaborate solution to combine several documents into one.
\end{itemize}
%
See also the CTAN topic \href{http://ctan.org/topic/subdocs}{\textsf{subdocs}}
for further related packages.
The present package differs from the above solutions in that
a document structure constructed with the conventional |\include| mechanism
just needs two extra commands at the top of every file
such that all constituent files can be compiled individually.

%%%%%%%%%%%%%%%%%%%%%%%%%%%%%%%%%%%%%%%%%%%%%%%%%%%%%%%%%%%%%%%%%%%%%%%%%%%%%%%%
%\subsection{Feature Suggestions}
%
%The following is a list of features which may be useful for future
%versions of this package:
%%
%\begin{itemize}
%\item
%\ldots
%\end{itemize}

%%%%%%%%%%%%%%%%%%%%%%%%%%%%%%%%%%%%%%%%%%%%%%%%%%%%%%%%%%%%%%%%%%%%%%%%%%%%%%%%
\subsection{Revision History}

%%%%%%%%%%%%%%%%%%%%%%%%%%%%%%%%%%%%%%%%
\paragraph{v2.0:} 2018/12/30

\begin{itemize}
\item
immediate forward processing
\item
added |\childdocby| mechanism
\item
manual restructured
\end{itemize}

%%%%%%%%%%%%%%%%%%%%%%%%%%%%%%%%%%%%%%%%
\paragraph{v1.6:} 2018/01/17

\begin{itemize}
\item
application for development of include files
\item
corrections to manual
\end{itemize}

%%%%%%%%%%%%%%%%%%%%%%%%%%%%%%%%%%%%%%%%
\paragraph{v1.5:} 2017/05/21

\begin{itemize}
\item
more complete structuring introduced
\item
|\childdocof| introduced
\item
|\childdoc| renamed to |\childdocmain|
\item
|\childredirect| renamed to |\childdocforward| and |\childdocforwardprefix|
and functionality expanded
\end{itemize}

%%%%%%%%%%%%%%%%%%%%%%%%%%%%%%%%%%%%%%%%
\paragraph{v1.0:} 2017/04/27

\begin{itemize}
\item
manual and install package
\item
first version published on CTAN
\end{itemize}

%%%%%%%%%%%%%%%%%%%%%%%%%%%%%%%%%%%%%%%%
\paragraph{v0.6:} 2017/04/26

\begin{itemize}
\item
redirection mechanism added
\end{itemize}

%%%%%%%%%%%%%%%%%%%%%%%%%%%%%%%%%%%%%%%%
\paragraph{v0.5:} 2017/04/26

\begin{itemize}
\item
functionality in definition file
\end{itemize}


%%%%%%%%%%%%%%%%%%%%%%%%%%%%%%%%%%%%%%%%%%%%%%%%%%%%%%%%%%%%%%%%%%%%%%%%%%%%%%%%
%%%%%%%%%%%%%%%%%%%%%%%%%%%%%%%%%%%%%%%%%%%%%%%%%%%%%%%%%%%%%%%%%%%%%%%%%%%%%%%%
%%%%%%%%%%%%%%%%%%%%%%%%%%%%%%%%%%%%%%%%%%%%%%%%%%%%%%%%%%%%%%%%%%%%%%%%%%%%%%%%
\appendix

\settowidth\MacroIndent{\rmfamily\scriptsize 000\ }

 \DocInput{childdoc.dtx}

\end{document}
%</driver>
% \fi
%
% %%%%%%%%%%%%%%%%%%%%%%%%%%%%%%%%%%%%%%%%%%%%%%%%%%%%%%%%%%%%%%%%%%%%%%%%%%%%%%
% %%%%%%%%%%%%%%%%%%%%%%%%%%%%%%%%%%%%%%%%%%%%%%%%%%%%%%%%%%%%%%%%%%%%%%%%%%%%%%
% \section{Sample}
%\iffalse
%<*samplemain>
%\fi
%
% The following presents a sample document
% with two chapters, two parts, a title page,
% a compile flag as well as three forwarding files to set the flag.
% It consists of eight |.tex| files:
% \begin{center}
% \begin{tabular}{ll}
% |cdocsamp.tex|&main file\\
% |cdocsch1.tex|&include file for chapter 1\\
% |cdocsch2.tex|&include file for chapter 2\\
% |cdocspt3.tex|&include file for part 3\\
% |cdocspt4.tex|&include file for part 4\\
% |cdocsdrf.tex|&forwarding file for main file in draft mode\\
% |cdocsfi1.tex|&forwarding file for final version of chapter 1\\
% |cdocsfi2.tex|&forwarding file for final version of chapter 2\\
% \end{tabular}
% \end{center}
% Each of the eight files can be compiled directly by the \LaTeX{} compiler.
%
% %%%%%%%%%%%%%%%%%%%%%%%%%%%%%%%%%%%%%%
% \paragraph{Main File.}
%
% The main file is called |cdocsamp.tex|.
%
% Load the \textsf{childdoc} definitions and
% declare the filename for the main document:
%    \begin{macrocode}
\input{childdoc.def}
\childdocmain{}
%    \end{macrocode}

% Optional override for |\version| flag:
%    \begin{macrocode}
%%\ifchilddoc\else\providecommand{\version}{draft}\fi
%    \end{macrocode}

% Define the default values for the |\version| flag
% (|final| for the main file and |draft| for childs):
%    \begin{macrocode}
\ifchilddoc
\providecommand{\version}{draft}
\else
\providecommand{\version}{final}
\fi
%    \end{macrocode}

% Load the standard document class:
%    \begin{macrocode}
\documentclass[12pt]{article}
%    \end{macrocode}

% Start the document body:
%    \begin{macrocode}
\begin{document}
%    \end{macrocode}

% Declare a title page.
% Print title, part of document being processed and version flag:
%    \begin{macrocode}
\addtocounter{page}{-1}
\begin{center}
{\LARGE\bfseries{}childdoc example\par}
\vspace{1cm}
\ifchilddoc
\ifchilddocmanual part\else chapter\fi:
`\childdocname' of `\childdocjob'\par
\else
main document: `\childdocjob'\par
\fi
version: \version\par
\end{center}
\newpage
%    \end{macrocode}

% Manually include selected file,
% otherwise process as usual:
%    \begin{macrocode}
\ifchilddocmanual
\section*{part `\childdocname'}
\input{\childdocname}
\else
%    \end{macrocode}

% Include the two chapters:
%    \begin{macrocode}
\include{cdocsch1}
\include{cdocsch2}
%    \end{macrocode}

% Include the two parts unless only chapters should be displayed:
%    \begin{macrocode}
\ifchilddoc\else
\section{part three}
\input{cdocspt3}
\section{part four}
\input{cdocspt4}
\fi
%    \end{macrocode}

% Process as usual until here:
%    \begin{macrocode}
\fi
%    \end{macrocode}

% End of document body:
%    \begin{macrocode}
\end{document}
%    \end{macrocode}
%\iffalse
%</samplemain>
%\fi
%
% %%%%%%%%%%%%%%%%%%%%%%%%%%%%%%%%%%%%%%
% \paragraph{Chapter Include Files.}
%
% The include files are called |cdocsch1.tex| and |cdocsch2.tex|.
%
%\iffalse
%<*samplechap1|samplechap2>
%\fi

% Optional override for |\version| flag:
%    \begin{macrocode}
%%\providecommand{\version}{final}
%    \end{macrocode}

% Include the main document:
%    \begin{macrocode}
\input{childdoc.def}
\childdocof{cdocsamp}
%    \end{macrocode}

%\iffalse
%</samplechap1|samplechap2>
%\fi
%
%\iffalse
%<*samplechap1>
%\fi
% Some text for chapter 1:
%    \begin{macrocode}
\section{one}
some text in chapter one
%    \end{macrocode}

%\iffalse
%</samplechap1>
%\fi
% Some text for chapter 2:
%\iffalse
%<*samplechap2>
%\fi
%    \begin{macrocode}
\section{two}
more text in chapter two
%    \end{macrocode}

%\iffalse
%</samplechap2>
%\fi
%
% %%%%%%%%%%%%%%%%%%%%%%%%%%%%%%%%%%%%%%
% \paragraph{Part Include Files.}
%
% The include files are called |cdocspt3.tex| and |cdocspt4.tex|.
%
%\iffalse
%<*samplepart3|samplepart4>
%\fi

% Optional override for |\version| flag:
%    \begin{macrocode}
%%\providecommand{\version}{final}
%    \end{macrocode}

% Include the main document:
%    \begin{macrocode}
\input{childdoc.def}
\childdocby{cdocsamp}
%    \end{macrocode}

%\iffalse
%</samplepart3|samplepart4>
%\fi
%
%\iffalse
%<*samplepart3>
%\fi
% Some text for part 3:
%    \begin{macrocode}
some text in part three
%    \end{macrocode}

%\iffalse
%</samplepart3>
%\fi
% Some text for part 4:
%\iffalse
%<*samplepart4>
%\fi
%    \begin{macrocode}
more text in part four
%    \end{macrocode}

%\iffalse
%</samplepart4>
%\fi
%
% %%%%%%%%%%%%%%%%%%%%%%%%%%%%%%%%%%%%%%
% \paragraph{Forwarding for a Complete Draft.}
%
% The following forwarding file |cdocsdrf.tex|
% compiles the main document in draft mode:
%\iffalse
%<*sampledraft>
%\fi
%    \begin{macrocode}
\def\version{draft}
\input{childdoc.def}
\childdocforward{cdocsamp}
%    \end{macrocode}

%\iffalse
%</sampledraft>
%\fi
%
% %%%%%%%%%%%%%%%%%%%%%%%%%%%%%%%%%%%%%%
% \paragraph{Forwarding for Final Version of the Chapters.}
%
% The following forwarding files |cdocsfn1.tex| and |cdocsfn2.tex|
% (with identical content)
% compile the final versions of the child documents
% |cdocsch1.tex| and |cdocsch2.tex|, respectively:
%\iffalse
%<*samplefinal>
%\fi
%    \begin{macrocode}
\def\version{final}
\input{childdoc.def}
\childdocforwardprefix[cdocsamp]{cdocsfn}{cdocsch}
%    \end{macrocode}

%\iffalse
%</samplefinal>
%\fi
%
% %%%%%%%%%%%%%%%%%%%%%%%%%%%%%%%%%%%%%%
% \paragraph{Command Line Processing.}
%
% The following three command lines generate the output files
% |cdocscld|, |cdocscl1| and |cdocscl2|
% which should be identical to
% |cdocsdrf|, |cdocsch1| and |cdocsfn2|, respectively:
% \begin{center}
% \begin{tabular}{l}
% |latex -jobname cdocscld \|\\
% |  "\def\version{draft}\input{childdoc.def}\childdocforward{cdocsamp}"|\\
% |latex -jobname cdocscl1 \|\\
% |  "\input{childdoc.def}\childdocforward[cdocsamp]{cdocsch1}"|\\
% |latex -jobname cdocscl2 \|\\
% |  "\def\version{final}\input{childdoc.def}\childdocforward{cdocsch2}"|
% \end{tabular}
% \end{center}
% Note that the trailing backslash on each first line
% merely continues the input to the second line
% (for convenient cut ant paste).
% Furthermore, the command |latex| can be replaced by any
% of its alternative versions such as |pdflatex|.
%
% %%%%%%%%%%%%%%%%%%%%%%%%%%%%%%%%%%%%%%%%%%%%%%%%%%%%%%%%%%%%%%%%%%%%%%%%%%%%%%
% %%%%%%%%%%%%%%%%%%%%%%%%%%%%%%%%%%%%%%%%%%%%%%%%%%%%%%%%%%%%%%%%%%%%%%%%%%%%%%
% \section{Implementation}
%\iffalse
%<*package>
%\fi
%
% This section describes the definitions file |childdoc.def|.

% The definitions cannot be loaded using |\usepackage| or |\RequirePackage|
% which has a mechanism to prevent loading a style file more than once.
% When loading the definitions by means of |\input|
% multiple instances have to be prevented manually:
%\iffalse
%This code needs to be before the `\ProvidesFile' directive
%which is defined at the beginning of this file.
%Therefore it is also placed there and commented out here.
%</package>
%<*discard>
%\fi
%    \begin{macrocode}
\ifdefined\childdocmain\endinput\fi
%    \end{macrocode}
%\iffalse
%</discard>
%<*package>
%\fi
%
% \macro{\ifchilddoc}
% \macro{\ifchilddocmanual}
% The conditional |\ifchilddoc| tells whether a
% child (true) or main (false) document is being compiled.
% The conditional |\ifchilddocmanual| tells whether
% the |\includeonly| mechanism is used (false) or
% the selection of child files must be performed manually (true).
% The definitions initialise to false:
%    \begin{macrocode}
\newif\ifchilddoc
\newif\ifchilddocmanual
%    \end{macrocode}

% \macro{\childdocname}
% \macro{\childdocjob}
% The macro |\childdocname| stores the name of the main document
% to be compiled. The macro |\childdocjob| stores the name of
% the document on which the \LaTeX{} compiler was originally invoked.
% The content of |\jobname| cannot be compared
% to filenames specified in the source due to different catcodes.
% The following code rescans |\jobname|, stores the result
% in |\childdocname| and saves a copy in |\childdocjob|:
%    \begin{macrocode}
\edef\childdocname{\scantokens\expandafter{\jobname\noexpand}}
\let\childdocjob\childdocname
%    \end{macrocode}

% \macro{\childdocdisable}
% The macro |\childdocdisable| prevents the main file
% from being processed more than once.
% At this stage, the main document command |\childdocmain|
% is assumed to be called once again where it should do nothing.
% Any subsequent call to it should prevent
% a secondary processing of the main document
% It overwrites the forwarding commands
% |\childdocof| and |\childdocforward|
% with empty macros to prevent further inclusions of the main document:
%    \begin{macrocode}
\newcommand{\childdocdisable}
{
  \renewcommand{\childdocmain}[1]{\renewcommand{\childdocmain}[1]{\endinput}}
  \renewcommand{\childdocof}[1]{}
  \renewcommand{\childdocby}[2][]{}
  \renewcommand{\childdocforward}[2][]{}
  \renewcommand{\childdocdisable}{}
}
%    \end{macrocode}

% \macro{\childdocmain}
% The macro |\childdocmain| is to be called at the top of the main file
% with nothing or the main filename (without extension) as argument.
% First, it breaks loops.
% If the argument is not empty and does not match |\childdocname|
% (which is set by the first inclusion of |childdoc.def|),
% |\ifchilddoc| is set to true, |\includeonly| is applied to the child file
% and |\jobname| is set to the main file
% (for proper handling of |.aux| files):
%    \begin{macrocode}
\newcommand{\childdocmain}[1]
{
  \childdocdisable\childdocmain{}
  \if?#1?\else
    \begingroup
      \def\childdoctmp{#1}
      \ifx\childdoctmp\childdocname
        \def\childdoctmp{}
      \else
        \def\childdoctmp
        {
          \childdoctrue
          \includeonly{\childdocname}
          \def\childdocjob{#1}
          \def\jobname{#1}
        }
      \fi
      \expandafter
    \endgroup
    \childdoctmp
  \fi
}
%    \end{macrocode}

% \macro{\childdocof}
% The command |\childdocof| redirects
% compilation to the main file |#1|.
%    \begin{macrocode}
\newcommand{\childdocof}[1]
{
  \childdocdisable
  \childdoctrue
  \includeonly{\childdocname}
  \def\jobname{#1}
  \def\childdocjob{#1}
  \input{#1}
}
%    \end{macrocode}

% \macro{\childdocby}
% The command |\childdocby| ....
%    \begin{macrocode}
\newcommand{\childdocby}[2][]
{
  \childdocdisable
  \childdoctrue
  \childdocmanualtrue
  \if?#1?\else
    \def\jobname{#2}
  \fi
  \def\childdocjob{#2}
  \input{#2}
  \endinput
}
%    \end{macrocode}

% \macro{\childdocforward}
% The command |\childdocforward| redirects
% compilation to the main file or
% (if the optional argument is given) a child file.
% Parameters are set as if the main file
% or a child file starting with |\childdocof| was compiled.
% Then compilation is handed over to the main file:
%    \begin{macrocode}
\newcommand{\childdocforward}[2][]
{
  \begingroup
    \if?#1?
      \def\childdoctmp
      {
        \def\childdocname{#2}
        \def\childdocjob{#2}
        \def\jobname{#2}
        \input{#2}
        \endinput
      }
    \else
      \def\childdoctmp
      {
        \childdocdisable
        \def\childdocname{#2}
        \childdoctrue
        \includeonly{#2}
        \def\childdocjob{#1}
        \def\jobname{#1}
        \input{#1}
        \endinput
      }
    \fi
    \expandafter
  \endgroup
  \childdoctmp
}
%    \end{macrocode}

% \macro{\childdocforwardprefix}
% The command |\childdocforwardprefix| redirects
% compilation to the main or a child file by means of a pattern.
% The prefix |#1| in the current filename is replaced by |#2|
% and the suffix of the current filename is kept
% (it is assumed that the filename does not contain the substring `|~~~|'
% which is used as a delimiter).
% Compilation is handed over to the new file by |\childdocforward|:
%    \begin{macrocode}
\newcommand{\childdocforwardprefix}[3][]
{
  \begingroup
    \def\childdocextract #2##1~~~{\def\childdoctmp{\childdocforward[#1]{#3##1}}}
    \expandafter\childdocextract\childdocname~~~
    \expandafter
  \endgroup
  \childdoctmp
}
%    \end{macrocode}

% \macro{\childdoc}
% The deprecated macro |\childdoc| is a legacy version of |\childdocmain|:
%    \begin{macrocode}
\newcommand{\childdoc}{\childdocmain}
%    \end{macrocode}

% \macro{\childdocredirect}
% The deprecated macro |\childdocredirect| is a legacy version
% of |\childdocforward| and |\childdocforwardprefix|:
%    \begin{macrocode}
\newcommand{\childdocredirect}[2][]
{
  \begingroup
    \if?#1?
      \def\childdoctmp{\childdocforward{#2}}
    \else
      \def\childdoctmp{\childdocforwardprefix{#1}{#2}}
    \fi
    \expandafter
  \endgroup
  \childdoctmp
}
%    \end{macrocode}

%\iffalse
%</package>
%\fi
%
\endinput

\childdocby{cdocsamp}
%    \end{macrocode}

%\iffalse
%</samplepart3|samplepart4>
%\fi
%
%\iffalse
%<*samplepart3>
%\fi
% Some text for part 3:
%    \begin{macrocode}
some text in part three
%    \end{macrocode}

%\iffalse
%</samplepart3>
%\fi
% Some text for part 4:
%\iffalse
%<*samplepart4>
%\fi
%    \begin{macrocode}
more text in part four
%    \end{macrocode}

%\iffalse
%</samplepart4>
%\fi
%
% %%%%%%%%%%%%%%%%%%%%%%%%%%%%%%%%%%%%%%
% \paragraph{Forwarding for a Complete Draft.}
%
% The following forwarding file |cdocsdrf.tex|
% compiles the main document in draft mode:
%\iffalse
%<*sampledraft>
%\fi
%    \begin{macrocode}
\def\version{draft}
% \iffalse
%
% childdoc.dtx Copyright (C) 2017-2018 Niklas Beisert
%
% This work may be distributed and/or modified under the
% conditions of the LaTeX Project Public License, either version 1.3
% of this license or (at your option) any later version.
% The latest version of this license is in
%   http://www.latex-project.org/lppl.txt
% and version 1.3 or later is part of all distributions of LaTeX
% version 2005/12/01 or later.
%
% This work has the LPPL maintenance status `maintained'.
%
% The Current Maintainer of this work is Niklas Beisert.
%
% This work consists of the files childdoc.dtx and childdoc.ins
% and the derived files childdoc.def and cdocsamp.tex with
% cdocsch1.tex, cdocsch2.tex, cdocsdrf.tex, cdocsfn1.tex, cdocsfn2.tex.
%
%<package>\ifdefined\childdocmain\endinput\fi
%<package>\ProvidesFile{childdoc.def}[2018/12/30 v2.0 child document driver]
%<samplemain>\ProvidesFile{cdocsamp.tex}[2018/12/30 v2.0 sample for childdoc]
%<*driver>
%\ProvidesFile{childdoc.drv}[2018/12/30 v2.0 childdoc reference manual file]
\PassOptionsToClass{10pt,a4paper}{article}
\documentclass{ltxdoc}

\usepackage[margin=35mm]{geometry}
\usepackage{hyperref}
\usepackage{hyperxmp}
\usepackage[usenames]{color}

\hypersetup{colorlinks=true}
\hypersetup{pdfstartview=FitH}
\hypersetup{pdfpagemode=UseNone}
\hypersetup{pdfsource={}}
\hypersetup{pdflang={en-UK}}
\hypersetup{pdfcopyright={Copyright 2017-2018 Niklas Beisert.
  This work may be distributed and/or modified under the
  conditions of the LaTeX Project Public License, either version 1.3
  of this license or (at your option) any later version.}}
\hypersetup{pdflicenseurl={http://www.latex-project.org/lppl.txt}}
\hypersetup{pdfcontactaddress={ETH Zurich, ITP, HIT K,
  Wolfgang-Pauli-Strasse 27}}
\hypersetup{pdfcontactpostcode={8093}}
\hypersetup{pdfcontactcity={Zurich}}
\hypersetup{pdfcontactcountry={Switzerland}}
\hypersetup{pdfcontactemail={nbeisert@itp.phys.ethz.ch}}
\hypersetup{pdfcontacturl={http://people.phys.ethz.ch/\xmptilde nbeisert/}}

\newcommand{\secref}[1]{\hyperref[#1]{section \ref*{#1}}}

\parskip1ex
\parindent0pt
\let\olditemize\itemize
\def\itemize{\olditemize\parskip0pt}

\begin{document}

\title{The \textsf{childdoc} Package}
\hypersetup{pdftitle={The childdoc Package}}
\author{Niklas Beisert\\[2ex]
  Institut f\"ur Theoretische Physik\\
  Eidgen\"ossische Technische Hochschule Z\"urich\\
  Wolfgang-Pauli-Strasse 27, 8093 Z\"urich, Switzerland\\[1ex]
  \href{mailto:nbeisert@itp.phys.ethz.ch}
  {\texttt{nbeisert@itp.phys.ethz.ch}}}
\hypersetup{pdfauthor={Niklas Beisert}}
\hypersetup{pdfsubject={Manual for the LaTeX2e Package childdoc}}
\date{30 December 2018, \textsf{v2.0}}
\maketitle

\begin{abstract}\noindent
\textsf{childdoc} is a \LaTeXe{} package
that enables the direct compilation
of document sections included by |\include|
to individual files.
\end{abstract}

\begingroup
\parskip0ex
\tableofcontents
\endgroup

%%%%%%%%%%%%%%%%%%%%%%%%%%%%%%%%%%%%%%%%%%%%%%%%%%%%%%%%%%%%%%%%%%%%%%%%%%%%%%%%
%%%%%%%%%%%%%%%%%%%%%%%%%%%%%%%%%%%%%%%%%%%%%%%%%%%%%%%%%%%%%%%%%%%%%%%%%%%%%%%%
\section{Introduction}

\LaTeX{} provides a mechanism to structure a large document (such as a book)
into a main file and several child files (containing the chapters)
using the |\include| command.
This mechanism is beneficial for documents
which span hundreds of pages in order to
make the source file(s) more manageable.
Moreover, compilation can be restricted to
selected child files by means of the |\includeonly| command.
The latter feature can be used to reduce the compilation time while editing
(this was significantly more useful in the earlier days of \LaTeX{})
or to generate a smaller document which is easier to navigate.
Another application of |\includeonly| is to generate
documents consisting of selected parts of the complete document.

However, there are a few drawbacks of the plain |\include| mechanism:
\begin{itemize}
\item
The child files cannot be compiled on their own,
they can only be compiled via the main file.
A naive editing environment
(such as a text editor with an option
to have the current file processed by \LaTeX)
may require one to switch to the main file before compiling;
attempting to compile the child file produces errors.
\item
The main file must be modified (each time)
to adjust the |\includeonly| command
to the present needs. This easily leaves the main file in a messy state.
\item
The generated document will always carry the filename
of the main document. This is inconvenient if
several child files are to be compiled and
to be kept for distribution.
\end{itemize}

The present package provides a simple interface
to make child files individually compilable by \LaTeX{}.
Compiling a child file then has the same effect as compiling
the main file with an |\includeonly| command
to select the appropriate child.
Moreover the generated document will carry the name of the child
rather than the main file.
This resolves all three above issues.

This feature is meant to make the editing of books,
thesis documents and lecture notes somewhat more convenient.
However, the package can also be used efficiently for
composing a series of documents (such as exercise sheets)
which are typically distributed individually.
It then assists the author in generating the individual documents
(potentially in different versions)
as well as a document containing the collected series.
Another application is in developing style files
or other kinds of included material
where compilation of the style file could redirect
to a sample or test file.

%%%%%%%%%%%%%%%%%%%%%%%%%%%%%%%%%%%%%%%%%%%%%%%%%%%%%%%%%%%%%%%%%%%%%%%%%%%%%%%%
%%%%%%%%%%%%%%%%%%%%%%%%%%%%%%%%%%%%%%%%%%%%%%%%%%%%%%%%%%%%%%%%%%%%%%%%%%%%%%%%
\section{Usage}

First of all, the package \textsf{childdoc} is \emph{not} a standard
\LaTeXe{} |.sty| style file! Therefore it needs to be invoked in
a non-standard way.

%%%%%%%%%%%%%%%%%%%%%%%%%%%%%%%%%%%%%%%%%%%%%%%%%%%%%%%%%%%%%%%%%%%%%%%%%%%%%%%%
\subsection{Included Files}
\label{sec:include}

%%%%%%%%%%%%%%%%%%%%%%%%%%%%%%%%%%%%%%%%
\DescribeMacro{\childdocmain}
To use the package, add the commands
\begin{center}
\begin{tabular}{l}
|\input{childdoc.def}|\\
|\childdocmain{}|\\
\end{tabular}
\end{center}
at the very top of the main \LaTeX{} file,
in particular \emph{before} the |\documentclass| statement!
The argument of |\childdocmain| should be left empty
(but it must be present).

%%%%%%%%%%%%%%%%%%%%%%%%%%%%%%%%%%%%%%%%
\DescribeMacro{\childdocof}
Furthermore, add the commands
\begin{center}
\begin{tabular}{l}
|\input{childdoc.def}|\\
|\childdocof{|\textit{main}|}|\\
\end{tabular}
\end{center}
at the top of every child file \textit{child}
which is included by |\include{|\textit{child}|}|
from within the main file
(or at least for those files to be compiled individually).
The argument \textit{main} must be the filename of the main file.

There are a couple of
considerations in setting up the main and child documents:

%%%%%%%%%%%%%%%%%%%%%%%%%%%%%%%%%%%%%%%%
\paragraph{Restrictions.}

Please note the following restrictions:
\begin{itemize}
\item
|\childdocmain| must be called with one argument \textit{main}
to ensure compatibility with earlier version of the package.
It must either be empty (|\childdocmain{}|)
or precisely match the filename of the main file in which it is specified.
See \secref{sec:detection} for further information.
\item
The filename \textit{main} must be specified without the |.tex| extension.
\item
The filename \textit{main} is case sensitive
(even in case-insensitive file systems)
due to internal string comparison.
\item
The argument \textit{main} should be fully expanded, it cannot be a macro.
\item
Subdirectories and special characters should be avoided in filenames.
\item
The command |\childdocmain{|\textit{main}|}| must be followed by a whitespace.
It should not be followed immediately by another command
or by a comment mark `|%|'.
This is because the \TeX{} parser reads the token immediately following
the argument of |\childdocmain| and puts it
at the beginning of every child section;
however, a white\-space is ignored.
\end{itemize}

%%%%%%%%%%%%%%%%%%%%%%%%%%%%%%%%%%%%%%%%
\paragraph{Content of Main File.}

It is advisable to place all content in the child files included by |\include|.
Any output contained in the main file will appear in all child documents
unless suppressed manually;
it cannot be suppressed automatically by the |\includeonly| directive
and thus should normally be avoided.
A method to include some content in the main file
by means of conditional processing is described in \secref{sec:conditional}.

%%%%%%%%%%%%%%%%%%%%%%%%%%%%%%%%%%%%%%%%
\paragraph{Page Numbering.}

When only a part of the document is compiled,
the appropriate numbering of pages
(as well as other status parameters)
is determined from the |.aux| files.
The latter contain information from previous passes.
However this information needs to propagate through
all intermediate child documents.
Therefore the page numbering in child documents may well
be inconsistent until the complete document is compiled at least once.

A useful (if unconventional) way to always ensure a consistent
page numbering is to restart the numbering in each child document
and denote the pages by `\textit{child}|.|\textit{page}'
where \textit{child} represents the chapter/section number of the child file.
This can be achieved by the command
|\numberwithin{page}{|\textit{child}|}|
of the \textsf{amsmath} package
where \textit{child} can be |chapter| or |section|
depending on the chosen structuring.
Alternatively, one can modify the macro |\thepage| appropriately
and reset the counter |page| at the start of each child file.

%%%%%%%%%%%%%%%%%%%%%%%%%%%%%%%%%%%%%%%%%%%%%%%%%%%%%%%%%%%%%%%%%%%%%%%%%%%%%%%%
\subsection{Conditional Processing}
\label{sec:conditional}

The package provides a mechanism to compile different versions
of a document. To customise the versions further some conditional processing
can come in handy to distinguish which version is being compiled.
The package provides two macros to describe the compilation context:

%%%%%%%%%%%%%%%%%%%%%%%%%%%%%%%%%%%%%%%%
\DescribeMacro{\ifchilddoc}
The conditional |\ifchilddoc| distinguishes between the compilation of
child documents and the main document:
%
\begin{center}
|\ifchilddoc |\textit{child-code}| |[|\||else |\textit{main-code}]| \||fi|
\end{center}

%%%%%%%%%%%%%%%%%%%%%%%%%%%%%%%%%%%%%%%%
\DescribeMacro{\childdocname}
\DescribeMacro{\childdocjob}
The macro |\childdocname| contains the filename (without extension)
of the main or child file being processed.
Note that |\childdocjob| will always contain the name of the main file.

%%%%%%%%%%%%%%%%%%%%%%%%%%%%%%%%%%%%%%%%
\paragraph{Title Page.}

Conditional processing can be used to include a title or banner page
in the main document when proper precautions are taken.
Importantly, the code in the main file should ensure that the page counter
(as well as other status parameters which are stored in the |.aux| files)
takes the same value after the conditional processing.
Otherwise the page numbers may take divergent values
depending on which part is compiled.

For example, a title page could be declared by:
%
\begin{center}
\begin{tabular}{l}
|\ifchilddoc\||else|\\
|\addtocounter{page}{-1}|\\
\textit{code for title page}\\
|\newpage|\\
|\||fi|
\end{tabular}
\end{center}
%
A banner page for the child documents can be generated by:
%
\begin{center}
\begin{tabular}{l}
|\ifchilddoc|\\
|\addtocounter{page}{-1}|\\
\textit{code for banner page}\\
|\newpage|\\
|\||fi|
\end{tabular}
\end{center}
%
Here one could write a message such as:
\begin{center}
|This is the part \childdocname{} of \childdocjob{}.|
\end{center}

%%%%%%%%%%%%%%%%%%%%%%%%%%%%%%%%%%%%%%%%%%%%%%%%%%%%%%%%%%%%%%%%%%%%%%%%%%%%%%%%
\subsection{Flags}
\label{sec:flags}

The package makes it easy to generate different versions
of the main or child documents.
To this end compilation flags can be defined
and assigned different default values.
They will be particularly useful in conjunction
with the forwarding mechanism described in \secref{sec:forward}.

For example, it may be useful to have a flag |\version|
which can be set to |draft| or |final|.
The document source will contain some conditional code
depending on the value of |\version|.
Suppose further, the flag should default to |final| for the main file
and to |draft| for child files
which is a natural assignment for editing the document.
This is achieved by placing the following code
in the preamble of the main document
(below the |\childdocmain| directive):
%
\begin{center}
\begin{tabular}{l}
|\ifchilddoc|\\
|\providecommand{\version}{draft}|\\
|\||else|\\
|\providecommand{\version}{final}|\\
|\||fi|
\end{tabular}
\end{center}
%
The definition by |\providecommand| makes sure
that previous definitions are not overwritten.
Further statements |\providecommand{\version}{...}|
can thus be added before the above code to override it.

For the main file, one might add a line
(between |\childdocmain| and the above block)
%
\begin{center}
|%\ifchilddoc\||else\providecommand{\version}{draft}\||fi|
\end{center}
%
which can be uncommented to produce a draft version.
Likewise one can add a line to the very top of a child file
(above the |\childdocof{|\textit{main}|}| directive)
%
\begin{center}
|%\providecommand{\version}{final}|
\end{center}
%
which can be uncommented to produce the final version of this child document.

%%%%%%%%%%%%%%%%%%%%%%%%%%%%%%%%%%%%%%%%%%%%%%%%%%%%%%%%%%%%%%%%%%%%%%%%%%%%%%%%
\subsection{Forwarding}
\label{sec:forward}

Different versions of the main or child documents
using compilation flags as described in \secref{sec:flags}
can be (permanently) stored in different files
for convenient compilation, viewing and distribution.
To this end, the package defines a command
to pass on compilation to a different file:

%%%%%%%%%%%%%%%%%%%%%%%%%%%%%%%%%%%%%%%%
\DescribeMacro{\childdocforward}
The command |\childdocforward| redirects processing to
another source file:
%
\begin{center}
\begin{tabular}{l}
|\input{childdoc.def}|\\
|\childdocforward[|\textit{main}|]{|\textit{dest}|}|\\
\end{tabular}
\end{center}
%
The argument \textit{dest} is the destination file
(without extension).
It should be the main file or one of the child files.
Note that further \textsf{childdoc} directives
such as |\childdocof| and |\childdocforward|
in the indicated file will be processed in this form.
The optional argument \textit{main}
passes on directly to the main file \textit{main}
while pretending to compile the child \textit{dest}.
This form behaves as if \textit{dest}
issues |\childdocof{|\textit{main}|}| right away,
and no further \textsf{childdoc} directives will be processed.

%%%%%%%%%%%%%%%%%%%%%%%%%%%%%%%%%%%%%%%%
\DescribeMacro{\...prefix}
In the alternative form |\childdocforwardprefix|,
%
\begin{center}
\begin{tabular}{l}
|\input{childdoc.def}|\\
|\childdocforwardprefix[|\textit{main}|]{|\textit{prefix}|}{|\textit{dest}|}|
\end{tabular}
\end{center}
%
the destination file is determined by a pattern
depending on the current file:
To make this work, the current file must be called
`{\textit{prefix}\hspace{0.2em}\textit{suffix}}'
with \textit{prefix} matching precisely the argument.
Processing is then passed on to the file
`{\textit{dest}\hspace{0.2em}\textit{suffix}}'.
Surely, the same effect is achieved by
directly specifying the
argument `{\textit{dest}\hspace{0.2em}\textit{suffix}}'
in the first form.
However, that requires to set up a different file
for each child. With the alternative form of the command
all these files can have exactly the same content
which simplifies setting them up and maintaining them.

For example, the following file |draft.tex|
with a compilation flag |\version| as described in \secref{sec:flags}
compiles the main document as a draft:
%
\begin{center}
\begin{tabular}{l}
|\def\version{draft}|\\
|\input{childdoc.def}|\\
|\childdocforward{|\textit{main}|}|
\end{tabular}
\end{center}
%
Likewise, the following files |final|\textit{nn}|.tex|
compile the final version of the child document
|child|\textit{nn}|.tex|:
%
\begin{center}
\begin{tabular}{l}
|\def\version{final}|\\
|\input{childdoc.def}|\\
|\childdocforwardprefix{final}{child}|
\end{tabular}
\end{center}
%

Note that when several versions of a main file and/or of each child file
are to be generated, it may be convenient to set up a |Makefile| or
shell script to automatise the process.

%%%%%%%%%%%%%%%%%%%%%%%%%%%%%%%%%%%%%%%%%%%%%%%%%%%%%%%%%%%%%%%%%%%%%%%%%%%%%%%%
\subsection{Command Line Processing}
\label{sec:commandline}

The effect of redirection files can also be achieved by invoking
the \LaTeX{} compiler with a more elaborate command line.
Most conveniently this should be done as part
of a shell script or a |Makefile|.

When using \textsf{childdoc} in the main file, the following
command lines effectively perform a redirection
(note that depending on the shell being used,
backslashes may have to be doubled: `|\|' $\to$ `|\\|'):
%
\begin{center}
|... -jobname "|\textit{target}|" |\\|"|[\textit{flags}]%
|\input{childdoc.def}\childdocforward[|\textit{main}|]{|\textit{dest}|}"|
\end{center}
%
Here \textit{target} is the name of the output file,
\textit{main} is the name of the main file
and \textit{dest} is the name of the main or child file to be processed
(all filenames without extensions).
The optional argument \textit{main} can be omitted
if \textit{main} matches \textit{dest}.
Optionally, compilation \textit{flags} can be defined via |\def| commands.
This command line makes the \TeX{} engine believe
it is compiling the file \textit{target}
whose content is specified as the latter parameter.
The provided code then forwards the processing to
\textit{main} or \textit{dest} as described in \secref{sec:forward}.

%%%%%%%%%%%%%%%%%%%%%%%%%%%%%%%%%%%%%%%%%%%%%%%%%%%%%%%%%%%%%%%%%%%%%%%%%%%%%%%%
\subsection{Include by Input}
\label{sec:input}

Including child documents by |\include| has some restrictions by design.
Most notably, the content of a child document always occupies
its own set of pages; pages cannot be shared between child documents.
Usually, this behaviour makes perfect sense
because each child document contain an essential part of the document.
However, in some situations it may be desirable to compose
a document from a collection of parts
without having mandatory page breaks between then.
For this case, the package
provides a mechanism to include parts
by |\input| which can also be processed individually.
However, by construction this mechanism
requires manual handling of the content to be output.

%%%%%%%%%%%%%%%%%%%%%%%%%%%%%%%%%%%%%%%%
\DescribeMacro{\ifchilddocmanual}
The main file should be prepared as usual, see \secref{sec:include}.
However, the document body must make a distinction
between processing of an individual part and of the main document, e.g.:
%
\begin{center}
\begin{tabular}{l}
|\ifchilddocmanual|\\
|\input{\childdocname}|\\
|\||else|\\
\textit{document body with }|\input{|\textit{part}|}|\\
|\||fi|
\end{tabular}
\end{center}
%
The conditional |\ifchilddocmanual| is true whenever
a part to be included by |\input| is being compiled,
and the name of the part is stored in |\childdocname|.

%%%%%%%%%%%%%%%%%%%%%%%%%%%%%%%%%%%%%%%%
\DescribeMacro{\childdocby}
Each part to be included by |\input| should start with:
%
\begin{center}
\begin{tabular}{l}
|\input{childdoc.def}|\\
|\childdocby{|\textit{main}|}|\\
\end{tabular}
\end{center}
%
The directive |\childdocby| is similar to |\childdocof|
described in \secref{sec:include},
but the subsequent selection of content must be done manually.
To that end, both |\ifchilddoc| and |\ifchilddocmanual|
will be true upon processing of a part,
and the name of the part is stored in |\childdocname|.
Note that |\jobname| will be set to the filename of the current part
so that each part receives an individual |.aux| file
that does not interfere with the |.aux| file(s) of the main document.
This behaviour can be altered by the alternative form
|\childdocby[*]{|\textit{main}|}| (with a non-empty optional argument)
which uses the |.aux| file of the main document
by setting |\jobname| to \textit{main}.

%%%%%%%%%%%%%%%%%%%%%%%%%%%%%%%%%%%%%%%%%%%%%%%%%%%%%%%%%%%%%%%%%%%%%%%%%%%%%%%%
\subsection{Driver Development}
\label{sec:driver}

The \textsf{childdoc} mechanism can also be use for the development
of definition files such as \LaTeX{} styles or classes.
This case differs from the above setup with multiple parts
included by |\include| in that no |\includeonly| should be invoked.
This can be achieved by starting the include file
(before |\ProvidesPackage|) with:
%
\begin{center}
\begin{tabular}{l}
|\input{childdoc.def}|\\
|\childdocforward{|\textit{main}|}|\\
\end{tabular}
\end{center}
%
or alternatively with:
%
\begin{center}
\begin{tabular}{l}
|\input{childdoc.def}|\\
|\childdocby{|\textit{main}|}|\\
\end{tabular}
\end{center}
%
Both forms have slightly different effects as described above.
The main file is prepared as usual, see \secref{sec:include}.

%%%%%%%%%%%%%%%%%%%%%%%%%%%%%%%%%%%%%%%%%%%%%%%%%%%%%%%%%%%%%%%%%%%%%%%%%%%%%%%%
\subsection{Legacy Detection}
\label{sec:detection}

The directive |\childdocmain| in the main file can detect
whether the complete document or merely a child is to be compiled
even without using the directive |\childdocof|.
This method is deprecated because it is less robust
and there is no compelling reason to use it;
it is merely provided for backward compatibility
and it may be removed in future versions.

If the detection mechanism is to be used,
it is mandatory to correctly specify
the filename of the main file as the argument of |\childdocmain|:
%
\begin{center}
\begin{tabular}{l}
|\input{childdoc.def}|\\
|\childdocmain{|\textit{main}|}|\\
\end{tabular}
\end{center}
%
If |\jobname| does not match the argument \textit{main} of |\childdocmain|,
it is assumed that |\jobname| points to the child file to be compiled.
When using |\childdocmain| with the main file specified as argument,
it suffices to start a child file
with just |\input{|\textit{main}|}|
without loading of the package and using |\childdocof|.
If instead all processing is done
with the appropriate \textsf{childdoc} directives,
the argument of \textit{main} of |\childdocmain| can be empty.

An alternative version of the command line processing described
in \secref{sec:commandline} using the detection mechanism reads:
%
\begin{center}
|... -jobname "|\textit{target}|" "|[\textit{flags}]%
[|\def\jobname{|\textit{dest}|}|]|\input{|\textit{main}|}"|
\end{center}

%%%%%%%%%%%%%%%%%%%%%%%%%%%%%%%%%%%%%%%%%%%%%%%%%%%%%%%%%%%%%%%%%%%%%%%%%%%%%%%%
\subsection{Manual Code}
\label{sec:manual}

In case one cannot be certain whether the definitions file |childdoc.def|
is installed on the target \TeX{} distribution
and one prefers not to ship it,
it is conceivable to paste a few relevant commands into the sources.

To that end, drop all statements |\input{childdoc.def}|
and perform the replacements as outlined below.
Instead of |\childdocmain{|\textit{main}|}| add the following code
to the top of the main file:
%
\begin{center}
\begin{tabular}{l}
|\||ifdefined\childdocname\endinput\||fi\newif\ifchilddoc|\\
|\edef\childdocname{\scantokens\expandafter{\jobname\noexpand}}|\\
|\def\childdocmain{|\textit{main}|}\||ifx\childdocmain\childdocname\||else|\\
|\childdoctrue\includeonly{\childdocname}\let\jobname\childdocmain\||fi|\\
\end{tabular}
\end{center}
%
Instead of |\childdocof{|\textit{main}|}| just include the main file
at the top of each child file:
%
\begin{center}
|\input{|\textit{main}|}|
\end{center}
%
A simple redirection |\childdocforward{|\textit{dest}|}| is achieved by:
%
\begin{center}
|\def\jobname{|\textit{dest}|}\input{\jobname}|
\end{center}
%
The redirection with prefix
|\childdocforwardprefix[|\textit{prefix}|]{|\textit{dest}|}|
is accomplished by:
%
\begin{center}
\begin{tabular}{l}
|{\edef\jobname{\scantokens\expandafter{\jobname\noexpand}}|\\
|\def\redirectjob |\textit{prefix}|#1~~~{\gdef\jobname{|\textit{dest}|#1}}|\\
|\expandafter\redirectjob\jobname~~~}\input{\jobname}|
\end{tabular}
\end{center}

In an alternative approach,
child documents can be compiled by a specific command line
without additional code or specific definitions:
%
\begin{center}
|... -jobname "|\textit{target}|" "|[\textit{flags}]%
|\includeonly{|\textit{dest}|}\input{|\textit{main}|}"|
\end{center}
%

%%%%%%%%%%%%%%%%%%%%%%%%%%%%%%%%%%%%%%%%%%%%%%%%%%%%%%%%%%%%%%%%%%%%%%%%%%%%%%%%
%%%%%%%%%%%%%%%%%%%%%%%%%%%%%%%%%%%%%%%%%%%%%%%%%%%%%%%%%%%%%%%%%%%%%%%%%%%%%%%%
\section{Information}

%%%%%%%%%%%%%%%%%%%%%%%%%%%%%%%%%%%%%%%%%%%%%%%%%%%%%%%%%%%%%%%%%%%%%%%%%%%%%%%%
\subsection{Copyright}

Copyright \copyright{} 2017--2018 Niklas Beisert

This work may be distributed and/or modified under the
conditions of the \LaTeX{} Project Public License, either version 1.3
of this license or (at your option) any later version.
The latest version of this license is in
  \url{http://www.latex-project.org/lppl.txt}
and version 1.3 or later is part of all distributions of \LaTeX{}
version 2005/12/01 or later.

This work has the LPPL maintenance status `maintained'.

The Current Maintainer of this work is Niklas Beisert.

This work consists of the files |README.txt|, |childdoc.ins| and |childdoc.dtx|
as well as the derived files |childdoc.def|, |cdocsamp.tex|
with |cdocsch1.tex|, |cdocsch2.tex|, |cdocspt3.tex|, |cdocspt4.tex|,
|cdocsdrf.tex|, |cdocsfn1.tex|, |cdocsfn2.tex|
as well as |childdoc.pdf|.

%%%%%%%%%%%%%%%%%%%%%%%%%%%%%%%%%%%%%%%%%%%%%%%%%%%%%%%%%%%%%%%%%%%%%%%%%%%%%%%%
\subsection{Files and Installation}

The package consists of the files:
%
\begin{center}
\begin{tabular}{ll}
    |README.txt|   & readme file \\
    |childdoc.ins| & installation file \\
    |childdoc.dtx| & source file \\
    |childdoc.def| & definition file \\
    |cdocsamp.tex| & sample main file \\
    |cdocsch1.tex| & sample include file \\
    |cdocsch2.tex| & sample include file \\
    |cdocspt3.tex| & sample part file \\
    |cdocspt4.tex| & sample part file \\
    |cdocsdrf.tex| & sample redirection file \\
    |cdocsfn1.tex| & sample redirection file \\
    |cdocsfn2.tex| & sample redirection file \\
    |childdoc.pdf| & manual
\end{tabular}
\end{center}
%
The distribution consists of the files
|README.txt|, |childdoc.ins| and |childdoc.dtx|.
%
\begin{itemize}
\item
Run (pdf)\LaTeX{} on |childdoc.dtx|
to compile the manual |childdoc.pdf| (this file).
\item
Run \LaTeX{} on |childdoc.ins| to create the definitions file |childdoc.def|
and the sample |cdocsamp.tex| with include files
|cdocsch1.tex|, |cdocsch2.tex|, |cdocspt3.tex|, |cdocspt4.tex|,
|cdocsdrf.tex|, |cdocsfn1.tex|, |cdocsfn2.tex|.
Then copy the file |childdoc.def| to an appropriate directory of your \LaTeX{}
distribution, e.g.\ \textit{texmf-root}|/tex/latex/childdoc|.
\end{itemize}

%%%%%%%%%%%%%%%%%%%%%%%%%%%%%%%%%%%%%%%%%%%%%%%%%%%%%%%%%%%%%%%%%%%%%%%%%%%%%%%%
\subsection{Related CTAN Packages}

There are several other packages which offer a similar functionality:
%
\begin{itemize}
\item
The packages
\href{http://ctan.org/pkg/docmute}{\textsf{docmute}},
\href{http://ctan.org/pkg/includex}{\textsf{includex}} and
\href{http://ctan.org/pkg/standalone}{\textsf{standalone}}
provide commands to include only the document body of
a child file thus allowing both files to be compiled individually.
\item
The packages \href{http://ctan.org/pkg/subdocs}{\textsf{subdocs}}
and \href{http://ctan.org/pkg/subfiles}{\textsf{subfiles}}
provide structures in which the main and child documents can be
encapsulated and allowing them to be compiled individually.
The inclusion mechanism is different from the conventional |\include|.
\item
The package \href{http://ctan.org/pkg/combine}{\textsf{combine}}
is an elaborate solution to combine several documents into one.
\end{itemize}
%
See also the CTAN topic \href{http://ctan.org/topic/subdocs}{\textsf{subdocs}}
for further related packages.
The present package differs from the above solutions in that
a document structure constructed with the conventional |\include| mechanism
just needs two extra commands at the top of every file
such that all constituent files can be compiled individually.

%%%%%%%%%%%%%%%%%%%%%%%%%%%%%%%%%%%%%%%%%%%%%%%%%%%%%%%%%%%%%%%%%%%%%%%%%%%%%%%%
%\subsection{Feature Suggestions}
%
%The following is a list of features which may be useful for future
%versions of this package:
%%
%\begin{itemize}
%\item
%\ldots
%\end{itemize}

%%%%%%%%%%%%%%%%%%%%%%%%%%%%%%%%%%%%%%%%%%%%%%%%%%%%%%%%%%%%%%%%%%%%%%%%%%%%%%%%
\subsection{Revision History}

%%%%%%%%%%%%%%%%%%%%%%%%%%%%%%%%%%%%%%%%
\paragraph{v2.0:} 2018/12/30

\begin{itemize}
\item
immediate forward processing
\item
added |\childdocby| mechanism
\item
manual restructured
\end{itemize}

%%%%%%%%%%%%%%%%%%%%%%%%%%%%%%%%%%%%%%%%
\paragraph{v1.6:} 2018/01/17

\begin{itemize}
\item
application for development of include files
\item
corrections to manual
\end{itemize}

%%%%%%%%%%%%%%%%%%%%%%%%%%%%%%%%%%%%%%%%
\paragraph{v1.5:} 2017/05/21

\begin{itemize}
\item
more complete structuring introduced
\item
|\childdocof| introduced
\item
|\childdoc| renamed to |\childdocmain|
\item
|\childredirect| renamed to |\childdocforward| and |\childdocforwardprefix|
and functionality expanded
\end{itemize}

%%%%%%%%%%%%%%%%%%%%%%%%%%%%%%%%%%%%%%%%
\paragraph{v1.0:} 2017/04/27

\begin{itemize}
\item
manual and install package
\item
first version published on CTAN
\end{itemize}

%%%%%%%%%%%%%%%%%%%%%%%%%%%%%%%%%%%%%%%%
\paragraph{v0.6:} 2017/04/26

\begin{itemize}
\item
redirection mechanism added
\end{itemize}

%%%%%%%%%%%%%%%%%%%%%%%%%%%%%%%%%%%%%%%%
\paragraph{v0.5:} 2017/04/26

\begin{itemize}
\item
functionality in definition file
\end{itemize}


%%%%%%%%%%%%%%%%%%%%%%%%%%%%%%%%%%%%%%%%%%%%%%%%%%%%%%%%%%%%%%%%%%%%%%%%%%%%%%%%
%%%%%%%%%%%%%%%%%%%%%%%%%%%%%%%%%%%%%%%%%%%%%%%%%%%%%%%%%%%%%%%%%%%%%%%%%%%%%%%%
%%%%%%%%%%%%%%%%%%%%%%%%%%%%%%%%%%%%%%%%%%%%%%%%%%%%%%%%%%%%%%%%%%%%%%%%%%%%%%%%
\appendix

\settowidth\MacroIndent{\rmfamily\scriptsize 000\ }

 \DocInput{childdoc.dtx}

\end{document}
%</driver>
% \fi
%
% %%%%%%%%%%%%%%%%%%%%%%%%%%%%%%%%%%%%%%%%%%%%%%%%%%%%%%%%%%%%%%%%%%%%%%%%%%%%%%
% %%%%%%%%%%%%%%%%%%%%%%%%%%%%%%%%%%%%%%%%%%%%%%%%%%%%%%%%%%%%%%%%%%%%%%%%%%%%%%
% \section{Sample}
%\iffalse
%<*samplemain>
%\fi
%
% The following presents a sample document
% with two chapters, two parts, a title page,
% a compile flag as well as three forwarding files to set the flag.
% It consists of eight |.tex| files:
% \begin{center}
% \begin{tabular}{ll}
% |cdocsamp.tex|&main file\\
% |cdocsch1.tex|&include file for chapter 1\\
% |cdocsch2.tex|&include file for chapter 2\\
% |cdocspt3.tex|&include file for part 3\\
% |cdocspt4.tex|&include file for part 4\\
% |cdocsdrf.tex|&forwarding file for main file in draft mode\\
% |cdocsfi1.tex|&forwarding file for final version of chapter 1\\
% |cdocsfi2.tex|&forwarding file for final version of chapter 2\\
% \end{tabular}
% \end{center}
% Each of the eight files can be compiled directly by the \LaTeX{} compiler.
%
% %%%%%%%%%%%%%%%%%%%%%%%%%%%%%%%%%%%%%%
% \paragraph{Main File.}
%
% The main file is called |cdocsamp.tex|.
%
% Load the \textsf{childdoc} definitions and
% declare the filename for the main document:
%    \begin{macrocode}
\input{childdoc.def}
\childdocmain{}
%    \end{macrocode}

% Optional override for |\version| flag:
%    \begin{macrocode}
%%\ifchilddoc\else\providecommand{\version}{draft}\fi
%    \end{macrocode}

% Define the default values for the |\version| flag
% (|final| for the main file and |draft| for childs):
%    \begin{macrocode}
\ifchilddoc
\providecommand{\version}{draft}
\else
\providecommand{\version}{final}
\fi
%    \end{macrocode}

% Load the standard document class:
%    \begin{macrocode}
\documentclass[12pt]{article}
%    \end{macrocode}

% Start the document body:
%    \begin{macrocode}
\begin{document}
%    \end{macrocode}

% Declare a title page.
% Print title, part of document being processed and version flag:
%    \begin{macrocode}
\addtocounter{page}{-1}
\begin{center}
{\LARGE\bfseries{}childdoc example\par}
\vspace{1cm}
\ifchilddoc
\ifchilddocmanual part\else chapter\fi:
`\childdocname' of `\childdocjob'\par
\else
main document: `\childdocjob'\par
\fi
version: \version\par
\end{center}
\newpage
%    \end{macrocode}

% Manually include selected file,
% otherwise process as usual:
%    \begin{macrocode}
\ifchilddocmanual
\section*{part `\childdocname'}
\input{\childdocname}
\else
%    \end{macrocode}

% Include the two chapters:
%    \begin{macrocode}
\include{cdocsch1}
\include{cdocsch2}
%    \end{macrocode}

% Include the two parts unless only chapters should be displayed:
%    \begin{macrocode}
\ifchilddoc\else
\section{part three}
\input{cdocspt3}
\section{part four}
\input{cdocspt4}
\fi
%    \end{macrocode}

% Process as usual until here:
%    \begin{macrocode}
\fi
%    \end{macrocode}

% End of document body:
%    \begin{macrocode}
\end{document}
%    \end{macrocode}
%\iffalse
%</samplemain>
%\fi
%
% %%%%%%%%%%%%%%%%%%%%%%%%%%%%%%%%%%%%%%
% \paragraph{Chapter Include Files.}
%
% The include files are called |cdocsch1.tex| and |cdocsch2.tex|.
%
%\iffalse
%<*samplechap1|samplechap2>
%\fi

% Optional override for |\version| flag:
%    \begin{macrocode}
%%\providecommand{\version}{final}
%    \end{macrocode}

% Include the main document:
%    \begin{macrocode}
\input{childdoc.def}
\childdocof{cdocsamp}
%    \end{macrocode}

%\iffalse
%</samplechap1|samplechap2>
%\fi
%
%\iffalse
%<*samplechap1>
%\fi
% Some text for chapter 1:
%    \begin{macrocode}
\section{one}
some text in chapter one
%    \end{macrocode}

%\iffalse
%</samplechap1>
%\fi
% Some text for chapter 2:
%\iffalse
%<*samplechap2>
%\fi
%    \begin{macrocode}
\section{two}
more text in chapter two
%    \end{macrocode}

%\iffalse
%</samplechap2>
%\fi
%
% %%%%%%%%%%%%%%%%%%%%%%%%%%%%%%%%%%%%%%
% \paragraph{Part Include Files.}
%
% The include files are called |cdocspt3.tex| and |cdocspt4.tex|.
%
%\iffalse
%<*samplepart3|samplepart4>
%\fi

% Optional override for |\version| flag:
%    \begin{macrocode}
%%\providecommand{\version}{final}
%    \end{macrocode}

% Include the main document:
%    \begin{macrocode}
\input{childdoc.def}
\childdocby{cdocsamp}
%    \end{macrocode}

%\iffalse
%</samplepart3|samplepart4>
%\fi
%
%\iffalse
%<*samplepart3>
%\fi
% Some text for part 3:
%    \begin{macrocode}
some text in part three
%    \end{macrocode}

%\iffalse
%</samplepart3>
%\fi
% Some text for part 4:
%\iffalse
%<*samplepart4>
%\fi
%    \begin{macrocode}
more text in part four
%    \end{macrocode}

%\iffalse
%</samplepart4>
%\fi
%
% %%%%%%%%%%%%%%%%%%%%%%%%%%%%%%%%%%%%%%
% \paragraph{Forwarding for a Complete Draft.}
%
% The following forwarding file |cdocsdrf.tex|
% compiles the main document in draft mode:
%\iffalse
%<*sampledraft>
%\fi
%    \begin{macrocode}
\def\version{draft}
\input{childdoc.def}
\childdocforward{cdocsamp}
%    \end{macrocode}

%\iffalse
%</sampledraft>
%\fi
%
% %%%%%%%%%%%%%%%%%%%%%%%%%%%%%%%%%%%%%%
% \paragraph{Forwarding for Final Version of the Chapters.}
%
% The following forwarding files |cdocsfn1.tex| and |cdocsfn2.tex|
% (with identical content)
% compile the final versions of the child documents
% |cdocsch1.tex| and |cdocsch2.tex|, respectively:
%\iffalse
%<*samplefinal>
%\fi
%    \begin{macrocode}
\def\version{final}
\input{childdoc.def}
\childdocforwardprefix[cdocsamp]{cdocsfn}{cdocsch}
%    \end{macrocode}

%\iffalse
%</samplefinal>
%\fi
%
% %%%%%%%%%%%%%%%%%%%%%%%%%%%%%%%%%%%%%%
% \paragraph{Command Line Processing.}
%
% The following three command lines generate the output files
% |cdocscld|, |cdocscl1| and |cdocscl2|
% which should be identical to
% |cdocsdrf|, |cdocsch1| and |cdocsfn2|, respectively:
% \begin{center}
% \begin{tabular}{l}
% |latex -jobname cdocscld \|\\
% |  "\def\version{draft}\input{childdoc.def}\childdocforward{cdocsamp}"|\\
% |latex -jobname cdocscl1 \|\\
% |  "\input{childdoc.def}\childdocforward[cdocsamp]{cdocsch1}"|\\
% |latex -jobname cdocscl2 \|\\
% |  "\def\version{final}\input{childdoc.def}\childdocforward{cdocsch2}"|
% \end{tabular}
% \end{center}
% Note that the trailing backslash on each first line
% merely continues the input to the second line
% (for convenient cut ant paste).
% Furthermore, the command |latex| can be replaced by any
% of its alternative versions such as |pdflatex|.
%
% %%%%%%%%%%%%%%%%%%%%%%%%%%%%%%%%%%%%%%%%%%%%%%%%%%%%%%%%%%%%%%%%%%%%%%%%%%%%%%
% %%%%%%%%%%%%%%%%%%%%%%%%%%%%%%%%%%%%%%%%%%%%%%%%%%%%%%%%%%%%%%%%%%%%%%%%%%%%%%
% \section{Implementation}
%\iffalse
%<*package>
%\fi
%
% This section describes the definitions file |childdoc.def|.

% The definitions cannot be loaded using |\usepackage| or |\RequirePackage|
% which has a mechanism to prevent loading a style file more than once.
% When loading the definitions by means of |\input|
% multiple instances have to be prevented manually:
%\iffalse
%This code needs to be before the `\ProvidesFile' directive
%which is defined at the beginning of this file.
%Therefore it is also placed there and commented out here.
%</package>
%<*discard>
%\fi
%    \begin{macrocode}
\ifdefined\childdocmain\endinput\fi
%    \end{macrocode}
%\iffalse
%</discard>
%<*package>
%\fi
%
% \macro{\ifchilddoc}
% \macro{\ifchilddocmanual}
% The conditional |\ifchilddoc| tells whether a
% child (true) or main (false) document is being compiled.
% The conditional |\ifchilddocmanual| tells whether
% the |\includeonly| mechanism is used (false) or
% the selection of child files must be performed manually (true).
% The definitions initialise to false:
%    \begin{macrocode}
\newif\ifchilddoc
\newif\ifchilddocmanual
%    \end{macrocode}

% \macro{\childdocname}
% \macro{\childdocjob}
% The macro |\childdocname| stores the name of the main document
% to be compiled. The macro |\childdocjob| stores the name of
% the document on which the \LaTeX{} compiler was originally invoked.
% The content of |\jobname| cannot be compared
% to filenames specified in the source due to different catcodes.
% The following code rescans |\jobname|, stores the result
% in |\childdocname| and saves a copy in |\childdocjob|:
%    \begin{macrocode}
\edef\childdocname{\scantokens\expandafter{\jobname\noexpand}}
\let\childdocjob\childdocname
%    \end{macrocode}

% \macro{\childdocdisable}
% The macro |\childdocdisable| prevents the main file
% from being processed more than once.
% At this stage, the main document command |\childdocmain|
% is assumed to be called once again where it should do nothing.
% Any subsequent call to it should prevent
% a secondary processing of the main document
% It overwrites the forwarding commands
% |\childdocof| and |\childdocforward|
% with empty macros to prevent further inclusions of the main document:
%    \begin{macrocode}
\newcommand{\childdocdisable}
{
  \renewcommand{\childdocmain}[1]{\renewcommand{\childdocmain}[1]{\endinput}}
  \renewcommand{\childdocof}[1]{}
  \renewcommand{\childdocby}[2][]{}
  \renewcommand{\childdocforward}[2][]{}
  \renewcommand{\childdocdisable}{}
}
%    \end{macrocode}

% \macro{\childdocmain}
% The macro |\childdocmain| is to be called at the top of the main file
% with nothing or the main filename (without extension) as argument.
% First, it breaks loops.
% If the argument is not empty and does not match |\childdocname|
% (which is set by the first inclusion of |childdoc.def|),
% |\ifchilddoc| is set to true, |\includeonly| is applied to the child file
% and |\jobname| is set to the main file
% (for proper handling of |.aux| files):
%    \begin{macrocode}
\newcommand{\childdocmain}[1]
{
  \childdocdisable\childdocmain{}
  \if?#1?\else
    \begingroup
      \def\childdoctmp{#1}
      \ifx\childdoctmp\childdocname
        \def\childdoctmp{}
      \else
        \def\childdoctmp
        {
          \childdoctrue
          \includeonly{\childdocname}
          \def\childdocjob{#1}
          \def\jobname{#1}
        }
      \fi
      \expandafter
    \endgroup
    \childdoctmp
  \fi
}
%    \end{macrocode}

% \macro{\childdocof}
% The command |\childdocof| redirects
% compilation to the main file |#1|.
%    \begin{macrocode}
\newcommand{\childdocof}[1]
{
  \childdocdisable
  \childdoctrue
  \includeonly{\childdocname}
  \def\jobname{#1}
  \def\childdocjob{#1}
  \input{#1}
}
%    \end{macrocode}

% \macro{\childdocby}
% The command |\childdocby| ....
%    \begin{macrocode}
\newcommand{\childdocby}[2][]
{
  \childdocdisable
  \childdoctrue
  \childdocmanualtrue
  \if?#1?\else
    \def\jobname{#2}
  \fi
  \def\childdocjob{#2}
  \input{#2}
  \endinput
}
%    \end{macrocode}

% \macro{\childdocforward}
% The command |\childdocforward| redirects
% compilation to the main file or
% (if the optional argument is given) a child file.
% Parameters are set as if the main file
% or a child file starting with |\childdocof| was compiled.
% Then compilation is handed over to the main file:
%    \begin{macrocode}
\newcommand{\childdocforward}[2][]
{
  \begingroup
    \if?#1?
      \def\childdoctmp
      {
        \def\childdocname{#2}
        \def\childdocjob{#2}
        \def\jobname{#2}
        \input{#2}
        \endinput
      }
    \else
      \def\childdoctmp
      {
        \childdocdisable
        \def\childdocname{#2}
        \childdoctrue
        \includeonly{#2}
        \def\childdocjob{#1}
        \def\jobname{#1}
        \input{#1}
        \endinput
      }
    \fi
    \expandafter
  \endgroup
  \childdoctmp
}
%    \end{macrocode}

% \macro{\childdocforwardprefix}
% The command |\childdocforwardprefix| redirects
% compilation to the main or a child file by means of a pattern.
% The prefix |#1| in the current filename is replaced by |#2|
% and the suffix of the current filename is kept
% (it is assumed that the filename does not contain the substring `|~~~|'
% which is used as a delimiter).
% Compilation is handed over to the new file by |\childdocforward|:
%    \begin{macrocode}
\newcommand{\childdocforwardprefix}[3][]
{
  \begingroup
    \def\childdocextract #2##1~~~{\def\childdoctmp{\childdocforward[#1]{#3##1}}}
    \expandafter\childdocextract\childdocname~~~
    \expandafter
  \endgroup
  \childdoctmp
}
%    \end{macrocode}

% \macro{\childdoc}
% The deprecated macro |\childdoc| is a legacy version of |\childdocmain|:
%    \begin{macrocode}
\newcommand{\childdoc}{\childdocmain}
%    \end{macrocode}

% \macro{\childdocredirect}
% The deprecated macro |\childdocredirect| is a legacy version
% of |\childdocforward| and |\childdocforwardprefix|:
%    \begin{macrocode}
\newcommand{\childdocredirect}[2][]
{
  \begingroup
    \if?#1?
      \def\childdoctmp{\childdocforward{#2}}
    \else
      \def\childdoctmp{\childdocforwardprefix{#1}{#2}}
    \fi
    \expandafter
  \endgroup
  \childdoctmp
}
%    \end{macrocode}

%\iffalse
%</package>
%\fi
%
\endinput

\childdocforward{cdocsamp}
%    \end{macrocode}

%\iffalse
%</sampledraft>
%\fi
%
% %%%%%%%%%%%%%%%%%%%%%%%%%%%%%%%%%%%%%%
% \paragraph{Forwarding for Final Version of the Chapters.}
%
% The following forwarding files |cdocsfn1.tex| and |cdocsfn2.tex|
% (with identical content)
% compile the final versions of the child documents
% |cdocsch1.tex| and |cdocsch2.tex|, respectively:
%\iffalse
%<*samplefinal>
%\fi
%    \begin{macrocode}
\def\version{final}
% \iffalse
%
% childdoc.dtx Copyright (C) 2017-2018 Niklas Beisert
%
% This work may be distributed and/or modified under the
% conditions of the LaTeX Project Public License, either version 1.3
% of this license or (at your option) any later version.
% The latest version of this license is in
%   http://www.latex-project.org/lppl.txt
% and version 1.3 or later is part of all distributions of LaTeX
% version 2005/12/01 or later.
%
% This work has the LPPL maintenance status `maintained'.
%
% The Current Maintainer of this work is Niklas Beisert.
%
% This work consists of the files childdoc.dtx and childdoc.ins
% and the derived files childdoc.def and cdocsamp.tex with
% cdocsch1.tex, cdocsch2.tex, cdocsdrf.tex, cdocsfn1.tex, cdocsfn2.tex.
%
%<package>\ifdefined\childdocmain\endinput\fi
%<package>\ProvidesFile{childdoc.def}[2018/12/30 v2.0 child document driver]
%<samplemain>\ProvidesFile{cdocsamp.tex}[2018/12/30 v2.0 sample for childdoc]
%<*driver>
%\ProvidesFile{childdoc.drv}[2018/12/30 v2.0 childdoc reference manual file]
\PassOptionsToClass{10pt,a4paper}{article}
\documentclass{ltxdoc}

\usepackage[margin=35mm]{geometry}
\usepackage{hyperref}
\usepackage{hyperxmp}
\usepackage[usenames]{color}

\hypersetup{colorlinks=true}
\hypersetup{pdfstartview=FitH}
\hypersetup{pdfpagemode=UseNone}
\hypersetup{pdfsource={}}
\hypersetup{pdflang={en-UK}}
\hypersetup{pdfcopyright={Copyright 2017-2018 Niklas Beisert.
  This work may be distributed and/or modified under the
  conditions of the LaTeX Project Public License, either version 1.3
  of this license or (at your option) any later version.}}
\hypersetup{pdflicenseurl={http://www.latex-project.org/lppl.txt}}
\hypersetup{pdfcontactaddress={ETH Zurich, ITP, HIT K,
  Wolfgang-Pauli-Strasse 27}}
\hypersetup{pdfcontactpostcode={8093}}
\hypersetup{pdfcontactcity={Zurich}}
\hypersetup{pdfcontactcountry={Switzerland}}
\hypersetup{pdfcontactemail={nbeisert@itp.phys.ethz.ch}}
\hypersetup{pdfcontacturl={http://people.phys.ethz.ch/\xmptilde nbeisert/}}

\newcommand{\secref}[1]{\hyperref[#1]{section \ref*{#1}}}

\parskip1ex
\parindent0pt
\let\olditemize\itemize
\def\itemize{\olditemize\parskip0pt}

\begin{document}

\title{The \textsf{childdoc} Package}
\hypersetup{pdftitle={The childdoc Package}}
\author{Niklas Beisert\\[2ex]
  Institut f\"ur Theoretische Physik\\
  Eidgen\"ossische Technische Hochschule Z\"urich\\
  Wolfgang-Pauli-Strasse 27, 8093 Z\"urich, Switzerland\\[1ex]
  \href{mailto:nbeisert@itp.phys.ethz.ch}
  {\texttt{nbeisert@itp.phys.ethz.ch}}}
\hypersetup{pdfauthor={Niklas Beisert}}
\hypersetup{pdfsubject={Manual for the LaTeX2e Package childdoc}}
\date{30 December 2018, \textsf{v2.0}}
\maketitle

\begin{abstract}\noindent
\textsf{childdoc} is a \LaTeXe{} package
that enables the direct compilation
of document sections included by |\include|
to individual files.
\end{abstract}

\begingroup
\parskip0ex
\tableofcontents
\endgroup

%%%%%%%%%%%%%%%%%%%%%%%%%%%%%%%%%%%%%%%%%%%%%%%%%%%%%%%%%%%%%%%%%%%%%%%%%%%%%%%%
%%%%%%%%%%%%%%%%%%%%%%%%%%%%%%%%%%%%%%%%%%%%%%%%%%%%%%%%%%%%%%%%%%%%%%%%%%%%%%%%
\section{Introduction}

\LaTeX{} provides a mechanism to structure a large document (such as a book)
into a main file and several child files (containing the chapters)
using the |\include| command.
This mechanism is beneficial for documents
which span hundreds of pages in order to
make the source file(s) more manageable.
Moreover, compilation can be restricted to
selected child files by means of the |\includeonly| command.
The latter feature can be used to reduce the compilation time while editing
(this was significantly more useful in the earlier days of \LaTeX{})
or to generate a smaller document which is easier to navigate.
Another application of |\includeonly| is to generate
documents consisting of selected parts of the complete document.

However, there are a few drawbacks of the plain |\include| mechanism:
\begin{itemize}
\item
The child files cannot be compiled on their own,
they can only be compiled via the main file.
A naive editing environment
(such as a text editor with an option
to have the current file processed by \LaTeX)
may require one to switch to the main file before compiling;
attempting to compile the child file produces errors.
\item
The main file must be modified (each time)
to adjust the |\includeonly| command
to the present needs. This easily leaves the main file in a messy state.
\item
The generated document will always carry the filename
of the main document. This is inconvenient if
several child files are to be compiled and
to be kept for distribution.
\end{itemize}

The present package provides a simple interface
to make child files individually compilable by \LaTeX{}.
Compiling a child file then has the same effect as compiling
the main file with an |\includeonly| command
to select the appropriate child.
Moreover the generated document will carry the name of the child
rather than the main file.
This resolves all three above issues.

This feature is meant to make the editing of books,
thesis documents and lecture notes somewhat more convenient.
However, the package can also be used efficiently for
composing a series of documents (such as exercise sheets)
which are typically distributed individually.
It then assists the author in generating the individual documents
(potentially in different versions)
as well as a document containing the collected series.
Another application is in developing style files
or other kinds of included material
where compilation of the style file could redirect
to a sample or test file.

%%%%%%%%%%%%%%%%%%%%%%%%%%%%%%%%%%%%%%%%%%%%%%%%%%%%%%%%%%%%%%%%%%%%%%%%%%%%%%%%
%%%%%%%%%%%%%%%%%%%%%%%%%%%%%%%%%%%%%%%%%%%%%%%%%%%%%%%%%%%%%%%%%%%%%%%%%%%%%%%%
\section{Usage}

First of all, the package \textsf{childdoc} is \emph{not} a standard
\LaTeXe{} |.sty| style file! Therefore it needs to be invoked in
a non-standard way.

%%%%%%%%%%%%%%%%%%%%%%%%%%%%%%%%%%%%%%%%%%%%%%%%%%%%%%%%%%%%%%%%%%%%%%%%%%%%%%%%
\subsection{Included Files}
\label{sec:include}

%%%%%%%%%%%%%%%%%%%%%%%%%%%%%%%%%%%%%%%%
\DescribeMacro{\childdocmain}
To use the package, add the commands
\begin{center}
\begin{tabular}{l}
|\input{childdoc.def}|\\
|\childdocmain{}|\\
\end{tabular}
\end{center}
at the very top of the main \LaTeX{} file,
in particular \emph{before} the |\documentclass| statement!
The argument of |\childdocmain| should be left empty
(but it must be present).

%%%%%%%%%%%%%%%%%%%%%%%%%%%%%%%%%%%%%%%%
\DescribeMacro{\childdocof}
Furthermore, add the commands
\begin{center}
\begin{tabular}{l}
|\input{childdoc.def}|\\
|\childdocof{|\textit{main}|}|\\
\end{tabular}
\end{center}
at the top of every child file \textit{child}
which is included by |\include{|\textit{child}|}|
from within the main file
(or at least for those files to be compiled individually).
The argument \textit{main} must be the filename of the main file.

There are a couple of
considerations in setting up the main and child documents:

%%%%%%%%%%%%%%%%%%%%%%%%%%%%%%%%%%%%%%%%
\paragraph{Restrictions.}

Please note the following restrictions:
\begin{itemize}
\item
|\childdocmain| must be called with one argument \textit{main}
to ensure compatibility with earlier version of the package.
It must either be empty (|\childdocmain{}|)
or precisely match the filename of the main file in which it is specified.
See \secref{sec:detection} for further information.
\item
The filename \textit{main} must be specified without the |.tex| extension.
\item
The filename \textit{main} is case sensitive
(even in case-insensitive file systems)
due to internal string comparison.
\item
The argument \textit{main} should be fully expanded, it cannot be a macro.
\item
Subdirectories and special characters should be avoided in filenames.
\item
The command |\childdocmain{|\textit{main}|}| must be followed by a whitespace.
It should not be followed immediately by another command
or by a comment mark `|%|'.
This is because the \TeX{} parser reads the token immediately following
the argument of |\childdocmain| and puts it
at the beginning of every child section;
however, a white\-space is ignored.
\end{itemize}

%%%%%%%%%%%%%%%%%%%%%%%%%%%%%%%%%%%%%%%%
\paragraph{Content of Main File.}

It is advisable to place all content in the child files included by |\include|.
Any output contained in the main file will appear in all child documents
unless suppressed manually;
it cannot be suppressed automatically by the |\includeonly| directive
and thus should normally be avoided.
A method to include some content in the main file
by means of conditional processing is described in \secref{sec:conditional}.

%%%%%%%%%%%%%%%%%%%%%%%%%%%%%%%%%%%%%%%%
\paragraph{Page Numbering.}

When only a part of the document is compiled,
the appropriate numbering of pages
(as well as other status parameters)
is determined from the |.aux| files.
The latter contain information from previous passes.
However this information needs to propagate through
all intermediate child documents.
Therefore the page numbering in child documents may well
be inconsistent until the complete document is compiled at least once.

A useful (if unconventional) way to always ensure a consistent
page numbering is to restart the numbering in each child document
and denote the pages by `\textit{child}|.|\textit{page}'
where \textit{child} represents the chapter/section number of the child file.
This can be achieved by the command
|\numberwithin{page}{|\textit{child}|}|
of the \textsf{amsmath} package
where \textit{child} can be |chapter| or |section|
depending on the chosen structuring.
Alternatively, one can modify the macro |\thepage| appropriately
and reset the counter |page| at the start of each child file.

%%%%%%%%%%%%%%%%%%%%%%%%%%%%%%%%%%%%%%%%%%%%%%%%%%%%%%%%%%%%%%%%%%%%%%%%%%%%%%%%
\subsection{Conditional Processing}
\label{sec:conditional}

The package provides a mechanism to compile different versions
of a document. To customise the versions further some conditional processing
can come in handy to distinguish which version is being compiled.
The package provides two macros to describe the compilation context:

%%%%%%%%%%%%%%%%%%%%%%%%%%%%%%%%%%%%%%%%
\DescribeMacro{\ifchilddoc}
The conditional |\ifchilddoc| distinguishes between the compilation of
child documents and the main document:
%
\begin{center}
|\ifchilddoc |\textit{child-code}| |[|\||else |\textit{main-code}]| \||fi|
\end{center}

%%%%%%%%%%%%%%%%%%%%%%%%%%%%%%%%%%%%%%%%
\DescribeMacro{\childdocname}
\DescribeMacro{\childdocjob}
The macro |\childdocname| contains the filename (without extension)
of the main or child file being processed.
Note that |\childdocjob| will always contain the name of the main file.

%%%%%%%%%%%%%%%%%%%%%%%%%%%%%%%%%%%%%%%%
\paragraph{Title Page.}

Conditional processing can be used to include a title or banner page
in the main document when proper precautions are taken.
Importantly, the code in the main file should ensure that the page counter
(as well as other status parameters which are stored in the |.aux| files)
takes the same value after the conditional processing.
Otherwise the page numbers may take divergent values
depending on which part is compiled.

For example, a title page could be declared by:
%
\begin{center}
\begin{tabular}{l}
|\ifchilddoc\||else|\\
|\addtocounter{page}{-1}|\\
\textit{code for title page}\\
|\newpage|\\
|\||fi|
\end{tabular}
\end{center}
%
A banner page for the child documents can be generated by:
%
\begin{center}
\begin{tabular}{l}
|\ifchilddoc|\\
|\addtocounter{page}{-1}|\\
\textit{code for banner page}\\
|\newpage|\\
|\||fi|
\end{tabular}
\end{center}
%
Here one could write a message such as:
\begin{center}
|This is the part \childdocname{} of \childdocjob{}.|
\end{center}

%%%%%%%%%%%%%%%%%%%%%%%%%%%%%%%%%%%%%%%%%%%%%%%%%%%%%%%%%%%%%%%%%%%%%%%%%%%%%%%%
\subsection{Flags}
\label{sec:flags}

The package makes it easy to generate different versions
of the main or child documents.
To this end compilation flags can be defined
and assigned different default values.
They will be particularly useful in conjunction
with the forwarding mechanism described in \secref{sec:forward}.

For example, it may be useful to have a flag |\version|
which can be set to |draft| or |final|.
The document source will contain some conditional code
depending on the value of |\version|.
Suppose further, the flag should default to |final| for the main file
and to |draft| for child files
which is a natural assignment for editing the document.
This is achieved by placing the following code
in the preamble of the main document
(below the |\childdocmain| directive):
%
\begin{center}
\begin{tabular}{l}
|\ifchilddoc|\\
|\providecommand{\version}{draft}|\\
|\||else|\\
|\providecommand{\version}{final}|\\
|\||fi|
\end{tabular}
\end{center}
%
The definition by |\providecommand| makes sure
that previous definitions are not overwritten.
Further statements |\providecommand{\version}{...}|
can thus be added before the above code to override it.

For the main file, one might add a line
(between |\childdocmain| and the above block)
%
\begin{center}
|%\ifchilddoc\||else\providecommand{\version}{draft}\||fi|
\end{center}
%
which can be uncommented to produce a draft version.
Likewise one can add a line to the very top of a child file
(above the |\childdocof{|\textit{main}|}| directive)
%
\begin{center}
|%\providecommand{\version}{final}|
\end{center}
%
which can be uncommented to produce the final version of this child document.

%%%%%%%%%%%%%%%%%%%%%%%%%%%%%%%%%%%%%%%%%%%%%%%%%%%%%%%%%%%%%%%%%%%%%%%%%%%%%%%%
\subsection{Forwarding}
\label{sec:forward}

Different versions of the main or child documents
using compilation flags as described in \secref{sec:flags}
can be (permanently) stored in different files
for convenient compilation, viewing and distribution.
To this end, the package defines a command
to pass on compilation to a different file:

%%%%%%%%%%%%%%%%%%%%%%%%%%%%%%%%%%%%%%%%
\DescribeMacro{\childdocforward}
The command |\childdocforward| redirects processing to
another source file:
%
\begin{center}
\begin{tabular}{l}
|\input{childdoc.def}|\\
|\childdocforward[|\textit{main}|]{|\textit{dest}|}|\\
\end{tabular}
\end{center}
%
The argument \textit{dest} is the destination file
(without extension).
It should be the main file or one of the child files.
Note that further \textsf{childdoc} directives
such as |\childdocof| and |\childdocforward|
in the indicated file will be processed in this form.
The optional argument \textit{main}
passes on directly to the main file \textit{main}
while pretending to compile the child \textit{dest}.
This form behaves as if \textit{dest}
issues |\childdocof{|\textit{main}|}| right away,
and no further \textsf{childdoc} directives will be processed.

%%%%%%%%%%%%%%%%%%%%%%%%%%%%%%%%%%%%%%%%
\DescribeMacro{\...prefix}
In the alternative form |\childdocforwardprefix|,
%
\begin{center}
\begin{tabular}{l}
|\input{childdoc.def}|\\
|\childdocforwardprefix[|\textit{main}|]{|\textit{prefix}|}{|\textit{dest}|}|
\end{tabular}
\end{center}
%
the destination file is determined by a pattern
depending on the current file:
To make this work, the current file must be called
`{\textit{prefix}\hspace{0.2em}\textit{suffix}}'
with \textit{prefix} matching precisely the argument.
Processing is then passed on to the file
`{\textit{dest}\hspace{0.2em}\textit{suffix}}'.
Surely, the same effect is achieved by
directly specifying the
argument `{\textit{dest}\hspace{0.2em}\textit{suffix}}'
in the first form.
However, that requires to set up a different file
for each child. With the alternative form of the command
all these files can have exactly the same content
which simplifies setting them up and maintaining them.

For example, the following file |draft.tex|
with a compilation flag |\version| as described in \secref{sec:flags}
compiles the main document as a draft:
%
\begin{center}
\begin{tabular}{l}
|\def\version{draft}|\\
|\input{childdoc.def}|\\
|\childdocforward{|\textit{main}|}|
\end{tabular}
\end{center}
%
Likewise, the following files |final|\textit{nn}|.tex|
compile the final version of the child document
|child|\textit{nn}|.tex|:
%
\begin{center}
\begin{tabular}{l}
|\def\version{final}|\\
|\input{childdoc.def}|\\
|\childdocforwardprefix{final}{child}|
\end{tabular}
\end{center}
%

Note that when several versions of a main file and/or of each child file
are to be generated, it may be convenient to set up a |Makefile| or
shell script to automatise the process.

%%%%%%%%%%%%%%%%%%%%%%%%%%%%%%%%%%%%%%%%%%%%%%%%%%%%%%%%%%%%%%%%%%%%%%%%%%%%%%%%
\subsection{Command Line Processing}
\label{sec:commandline}

The effect of redirection files can also be achieved by invoking
the \LaTeX{} compiler with a more elaborate command line.
Most conveniently this should be done as part
of a shell script or a |Makefile|.

When using \textsf{childdoc} in the main file, the following
command lines effectively perform a redirection
(note that depending on the shell being used,
backslashes may have to be doubled: `|\|' $\to$ `|\\|'):
%
\begin{center}
|... -jobname "|\textit{target}|" |\\|"|[\textit{flags}]%
|\input{childdoc.def}\childdocforward[|\textit{main}|]{|\textit{dest}|}"|
\end{center}
%
Here \textit{target} is the name of the output file,
\textit{main} is the name of the main file
and \textit{dest} is the name of the main or child file to be processed
(all filenames without extensions).
The optional argument \textit{main} can be omitted
if \textit{main} matches \textit{dest}.
Optionally, compilation \textit{flags} can be defined via |\def| commands.
This command line makes the \TeX{} engine believe
it is compiling the file \textit{target}
whose content is specified as the latter parameter.
The provided code then forwards the processing to
\textit{main} or \textit{dest} as described in \secref{sec:forward}.

%%%%%%%%%%%%%%%%%%%%%%%%%%%%%%%%%%%%%%%%%%%%%%%%%%%%%%%%%%%%%%%%%%%%%%%%%%%%%%%%
\subsection{Include by Input}
\label{sec:input}

Including child documents by |\include| has some restrictions by design.
Most notably, the content of a child document always occupies
its own set of pages; pages cannot be shared between child documents.
Usually, this behaviour makes perfect sense
because each child document contain an essential part of the document.
However, in some situations it may be desirable to compose
a document from a collection of parts
without having mandatory page breaks between then.
For this case, the package
provides a mechanism to include parts
by |\input| which can also be processed individually.
However, by construction this mechanism
requires manual handling of the content to be output.

%%%%%%%%%%%%%%%%%%%%%%%%%%%%%%%%%%%%%%%%
\DescribeMacro{\ifchilddocmanual}
The main file should be prepared as usual, see \secref{sec:include}.
However, the document body must make a distinction
between processing of an individual part and of the main document, e.g.:
%
\begin{center}
\begin{tabular}{l}
|\ifchilddocmanual|\\
|\input{\childdocname}|\\
|\||else|\\
\textit{document body with }|\input{|\textit{part}|}|\\
|\||fi|
\end{tabular}
\end{center}
%
The conditional |\ifchilddocmanual| is true whenever
a part to be included by |\input| is being compiled,
and the name of the part is stored in |\childdocname|.

%%%%%%%%%%%%%%%%%%%%%%%%%%%%%%%%%%%%%%%%
\DescribeMacro{\childdocby}
Each part to be included by |\input| should start with:
%
\begin{center}
\begin{tabular}{l}
|\input{childdoc.def}|\\
|\childdocby{|\textit{main}|}|\\
\end{tabular}
\end{center}
%
The directive |\childdocby| is similar to |\childdocof|
described in \secref{sec:include},
but the subsequent selection of content must be done manually.
To that end, both |\ifchilddoc| and |\ifchilddocmanual|
will be true upon processing of a part,
and the name of the part is stored in |\childdocname|.
Note that |\jobname| will be set to the filename of the current part
so that each part receives an individual |.aux| file
that does not interfere with the |.aux| file(s) of the main document.
This behaviour can be altered by the alternative form
|\childdocby[*]{|\textit{main}|}| (with a non-empty optional argument)
which uses the |.aux| file of the main document
by setting |\jobname| to \textit{main}.

%%%%%%%%%%%%%%%%%%%%%%%%%%%%%%%%%%%%%%%%%%%%%%%%%%%%%%%%%%%%%%%%%%%%%%%%%%%%%%%%
\subsection{Driver Development}
\label{sec:driver}

The \textsf{childdoc} mechanism can also be use for the development
of definition files such as \LaTeX{} styles or classes.
This case differs from the above setup with multiple parts
included by |\include| in that no |\includeonly| should be invoked.
This can be achieved by starting the include file
(before |\ProvidesPackage|) with:
%
\begin{center}
\begin{tabular}{l}
|\input{childdoc.def}|\\
|\childdocforward{|\textit{main}|}|\\
\end{tabular}
\end{center}
%
or alternatively with:
%
\begin{center}
\begin{tabular}{l}
|\input{childdoc.def}|\\
|\childdocby{|\textit{main}|}|\\
\end{tabular}
\end{center}
%
Both forms have slightly different effects as described above.
The main file is prepared as usual, see \secref{sec:include}.

%%%%%%%%%%%%%%%%%%%%%%%%%%%%%%%%%%%%%%%%%%%%%%%%%%%%%%%%%%%%%%%%%%%%%%%%%%%%%%%%
\subsection{Legacy Detection}
\label{sec:detection}

The directive |\childdocmain| in the main file can detect
whether the complete document or merely a child is to be compiled
even without using the directive |\childdocof|.
This method is deprecated because it is less robust
and there is no compelling reason to use it;
it is merely provided for backward compatibility
and it may be removed in future versions.

If the detection mechanism is to be used,
it is mandatory to correctly specify
the filename of the main file as the argument of |\childdocmain|:
%
\begin{center}
\begin{tabular}{l}
|\input{childdoc.def}|\\
|\childdocmain{|\textit{main}|}|\\
\end{tabular}
\end{center}
%
If |\jobname| does not match the argument \textit{main} of |\childdocmain|,
it is assumed that |\jobname| points to the child file to be compiled.
When using |\childdocmain| with the main file specified as argument,
it suffices to start a child file
with just |\input{|\textit{main}|}|
without loading of the package and using |\childdocof|.
If instead all processing is done
with the appropriate \textsf{childdoc} directives,
the argument of \textit{main} of |\childdocmain| can be empty.

An alternative version of the command line processing described
in \secref{sec:commandline} using the detection mechanism reads:
%
\begin{center}
|... -jobname "|\textit{target}|" "|[\textit{flags}]%
[|\def\jobname{|\textit{dest}|}|]|\input{|\textit{main}|}"|
\end{center}

%%%%%%%%%%%%%%%%%%%%%%%%%%%%%%%%%%%%%%%%%%%%%%%%%%%%%%%%%%%%%%%%%%%%%%%%%%%%%%%%
\subsection{Manual Code}
\label{sec:manual}

In case one cannot be certain whether the definitions file |childdoc.def|
is installed on the target \TeX{} distribution
and one prefers not to ship it,
it is conceivable to paste a few relevant commands into the sources.

To that end, drop all statements |\input{childdoc.def}|
and perform the replacements as outlined below.
Instead of |\childdocmain{|\textit{main}|}| add the following code
to the top of the main file:
%
\begin{center}
\begin{tabular}{l}
|\||ifdefined\childdocname\endinput\||fi\newif\ifchilddoc|\\
|\edef\childdocname{\scantokens\expandafter{\jobname\noexpand}}|\\
|\def\childdocmain{|\textit{main}|}\||ifx\childdocmain\childdocname\||else|\\
|\childdoctrue\includeonly{\childdocname}\let\jobname\childdocmain\||fi|\\
\end{tabular}
\end{center}
%
Instead of |\childdocof{|\textit{main}|}| just include the main file
at the top of each child file:
%
\begin{center}
|\input{|\textit{main}|}|
\end{center}
%
A simple redirection |\childdocforward{|\textit{dest}|}| is achieved by:
%
\begin{center}
|\def\jobname{|\textit{dest}|}\input{\jobname}|
\end{center}
%
The redirection with prefix
|\childdocforwardprefix[|\textit{prefix}|]{|\textit{dest}|}|
is accomplished by:
%
\begin{center}
\begin{tabular}{l}
|{\edef\jobname{\scantokens\expandafter{\jobname\noexpand}}|\\
|\def\redirectjob |\textit{prefix}|#1~~~{\gdef\jobname{|\textit{dest}|#1}}|\\
|\expandafter\redirectjob\jobname~~~}\input{\jobname}|
\end{tabular}
\end{center}

In an alternative approach,
child documents can be compiled by a specific command line
without additional code or specific definitions:
%
\begin{center}
|... -jobname "|\textit{target}|" "|[\textit{flags}]%
|\includeonly{|\textit{dest}|}\input{|\textit{main}|}"|
\end{center}
%

%%%%%%%%%%%%%%%%%%%%%%%%%%%%%%%%%%%%%%%%%%%%%%%%%%%%%%%%%%%%%%%%%%%%%%%%%%%%%%%%
%%%%%%%%%%%%%%%%%%%%%%%%%%%%%%%%%%%%%%%%%%%%%%%%%%%%%%%%%%%%%%%%%%%%%%%%%%%%%%%%
\section{Information}

%%%%%%%%%%%%%%%%%%%%%%%%%%%%%%%%%%%%%%%%%%%%%%%%%%%%%%%%%%%%%%%%%%%%%%%%%%%%%%%%
\subsection{Copyright}

Copyright \copyright{} 2017--2018 Niklas Beisert

This work may be distributed and/or modified under the
conditions of the \LaTeX{} Project Public License, either version 1.3
of this license or (at your option) any later version.
The latest version of this license is in
  \url{http://www.latex-project.org/lppl.txt}
and version 1.3 or later is part of all distributions of \LaTeX{}
version 2005/12/01 or later.

This work has the LPPL maintenance status `maintained'.

The Current Maintainer of this work is Niklas Beisert.

This work consists of the files |README.txt|, |childdoc.ins| and |childdoc.dtx|
as well as the derived files |childdoc.def|, |cdocsamp.tex|
with |cdocsch1.tex|, |cdocsch2.tex|, |cdocspt3.tex|, |cdocspt4.tex|,
|cdocsdrf.tex|, |cdocsfn1.tex|, |cdocsfn2.tex|
as well as |childdoc.pdf|.

%%%%%%%%%%%%%%%%%%%%%%%%%%%%%%%%%%%%%%%%%%%%%%%%%%%%%%%%%%%%%%%%%%%%%%%%%%%%%%%%
\subsection{Files and Installation}

The package consists of the files:
%
\begin{center}
\begin{tabular}{ll}
    |README.txt|   & readme file \\
    |childdoc.ins| & installation file \\
    |childdoc.dtx| & source file \\
    |childdoc.def| & definition file \\
    |cdocsamp.tex| & sample main file \\
    |cdocsch1.tex| & sample include file \\
    |cdocsch2.tex| & sample include file \\
    |cdocspt3.tex| & sample part file \\
    |cdocspt4.tex| & sample part file \\
    |cdocsdrf.tex| & sample redirection file \\
    |cdocsfn1.tex| & sample redirection file \\
    |cdocsfn2.tex| & sample redirection file \\
    |childdoc.pdf| & manual
\end{tabular}
\end{center}
%
The distribution consists of the files
|README.txt|, |childdoc.ins| and |childdoc.dtx|.
%
\begin{itemize}
\item
Run (pdf)\LaTeX{} on |childdoc.dtx|
to compile the manual |childdoc.pdf| (this file).
\item
Run \LaTeX{} on |childdoc.ins| to create the definitions file |childdoc.def|
and the sample |cdocsamp.tex| with include files
|cdocsch1.tex|, |cdocsch2.tex|, |cdocspt3.tex|, |cdocspt4.tex|,
|cdocsdrf.tex|, |cdocsfn1.tex|, |cdocsfn2.tex|.
Then copy the file |childdoc.def| to an appropriate directory of your \LaTeX{}
distribution, e.g.\ \textit{texmf-root}|/tex/latex/childdoc|.
\end{itemize}

%%%%%%%%%%%%%%%%%%%%%%%%%%%%%%%%%%%%%%%%%%%%%%%%%%%%%%%%%%%%%%%%%%%%%%%%%%%%%%%%
\subsection{Related CTAN Packages}

There are several other packages which offer a similar functionality:
%
\begin{itemize}
\item
The packages
\href{http://ctan.org/pkg/docmute}{\textsf{docmute}},
\href{http://ctan.org/pkg/includex}{\textsf{includex}} and
\href{http://ctan.org/pkg/standalone}{\textsf{standalone}}
provide commands to include only the document body of
a child file thus allowing both files to be compiled individually.
\item
The packages \href{http://ctan.org/pkg/subdocs}{\textsf{subdocs}}
and \href{http://ctan.org/pkg/subfiles}{\textsf{subfiles}}
provide structures in which the main and child documents can be
encapsulated and allowing them to be compiled individually.
The inclusion mechanism is different from the conventional |\include|.
\item
The package \href{http://ctan.org/pkg/combine}{\textsf{combine}}
is an elaborate solution to combine several documents into one.
\end{itemize}
%
See also the CTAN topic \href{http://ctan.org/topic/subdocs}{\textsf{subdocs}}
for further related packages.
The present package differs from the above solutions in that
a document structure constructed with the conventional |\include| mechanism
just needs two extra commands at the top of every file
such that all constituent files can be compiled individually.

%%%%%%%%%%%%%%%%%%%%%%%%%%%%%%%%%%%%%%%%%%%%%%%%%%%%%%%%%%%%%%%%%%%%%%%%%%%%%%%%
%\subsection{Feature Suggestions}
%
%The following is a list of features which may be useful for future
%versions of this package:
%%
%\begin{itemize}
%\item
%\ldots
%\end{itemize}

%%%%%%%%%%%%%%%%%%%%%%%%%%%%%%%%%%%%%%%%%%%%%%%%%%%%%%%%%%%%%%%%%%%%%%%%%%%%%%%%
\subsection{Revision History}

%%%%%%%%%%%%%%%%%%%%%%%%%%%%%%%%%%%%%%%%
\paragraph{v2.0:} 2018/12/30

\begin{itemize}
\item
immediate forward processing
\item
added |\childdocby| mechanism
\item
manual restructured
\end{itemize}

%%%%%%%%%%%%%%%%%%%%%%%%%%%%%%%%%%%%%%%%
\paragraph{v1.6:} 2018/01/17

\begin{itemize}
\item
application for development of include files
\item
corrections to manual
\end{itemize}

%%%%%%%%%%%%%%%%%%%%%%%%%%%%%%%%%%%%%%%%
\paragraph{v1.5:} 2017/05/21

\begin{itemize}
\item
more complete structuring introduced
\item
|\childdocof| introduced
\item
|\childdoc| renamed to |\childdocmain|
\item
|\childredirect| renamed to |\childdocforward| and |\childdocforwardprefix|
and functionality expanded
\end{itemize}

%%%%%%%%%%%%%%%%%%%%%%%%%%%%%%%%%%%%%%%%
\paragraph{v1.0:} 2017/04/27

\begin{itemize}
\item
manual and install package
\item
first version published on CTAN
\end{itemize}

%%%%%%%%%%%%%%%%%%%%%%%%%%%%%%%%%%%%%%%%
\paragraph{v0.6:} 2017/04/26

\begin{itemize}
\item
redirection mechanism added
\end{itemize}

%%%%%%%%%%%%%%%%%%%%%%%%%%%%%%%%%%%%%%%%
\paragraph{v0.5:} 2017/04/26

\begin{itemize}
\item
functionality in definition file
\end{itemize}


%%%%%%%%%%%%%%%%%%%%%%%%%%%%%%%%%%%%%%%%%%%%%%%%%%%%%%%%%%%%%%%%%%%%%%%%%%%%%%%%
%%%%%%%%%%%%%%%%%%%%%%%%%%%%%%%%%%%%%%%%%%%%%%%%%%%%%%%%%%%%%%%%%%%%%%%%%%%%%%%%
%%%%%%%%%%%%%%%%%%%%%%%%%%%%%%%%%%%%%%%%%%%%%%%%%%%%%%%%%%%%%%%%%%%%%%%%%%%%%%%%
\appendix

\settowidth\MacroIndent{\rmfamily\scriptsize 000\ }

 \DocInput{childdoc.dtx}

\end{document}
%</driver>
% \fi
%
% %%%%%%%%%%%%%%%%%%%%%%%%%%%%%%%%%%%%%%%%%%%%%%%%%%%%%%%%%%%%%%%%%%%%%%%%%%%%%%
% %%%%%%%%%%%%%%%%%%%%%%%%%%%%%%%%%%%%%%%%%%%%%%%%%%%%%%%%%%%%%%%%%%%%%%%%%%%%%%
% \section{Sample}
%\iffalse
%<*samplemain>
%\fi
%
% The following presents a sample document
% with two chapters, two parts, a title page,
% a compile flag as well as three forwarding files to set the flag.
% It consists of eight |.tex| files:
% \begin{center}
% \begin{tabular}{ll}
% |cdocsamp.tex|&main file\\
% |cdocsch1.tex|&include file for chapter 1\\
% |cdocsch2.tex|&include file for chapter 2\\
% |cdocspt3.tex|&include file for part 3\\
% |cdocspt4.tex|&include file for part 4\\
% |cdocsdrf.tex|&forwarding file for main file in draft mode\\
% |cdocsfi1.tex|&forwarding file for final version of chapter 1\\
% |cdocsfi2.tex|&forwarding file for final version of chapter 2\\
% \end{tabular}
% \end{center}
% Each of the eight files can be compiled directly by the \LaTeX{} compiler.
%
% %%%%%%%%%%%%%%%%%%%%%%%%%%%%%%%%%%%%%%
% \paragraph{Main File.}
%
% The main file is called |cdocsamp.tex|.
%
% Load the \textsf{childdoc} definitions and
% declare the filename for the main document:
%    \begin{macrocode}
\input{childdoc.def}
\childdocmain{}
%    \end{macrocode}

% Optional override for |\version| flag:
%    \begin{macrocode}
%%\ifchilddoc\else\providecommand{\version}{draft}\fi
%    \end{macrocode}

% Define the default values for the |\version| flag
% (|final| for the main file and |draft| for childs):
%    \begin{macrocode}
\ifchilddoc
\providecommand{\version}{draft}
\else
\providecommand{\version}{final}
\fi
%    \end{macrocode}

% Load the standard document class:
%    \begin{macrocode}
\documentclass[12pt]{article}
%    \end{macrocode}

% Start the document body:
%    \begin{macrocode}
\begin{document}
%    \end{macrocode}

% Declare a title page.
% Print title, part of document being processed and version flag:
%    \begin{macrocode}
\addtocounter{page}{-1}
\begin{center}
{\LARGE\bfseries{}childdoc example\par}
\vspace{1cm}
\ifchilddoc
\ifchilddocmanual part\else chapter\fi:
`\childdocname' of `\childdocjob'\par
\else
main document: `\childdocjob'\par
\fi
version: \version\par
\end{center}
\newpage
%    \end{macrocode}

% Manually include selected file,
% otherwise process as usual:
%    \begin{macrocode}
\ifchilddocmanual
\section*{part `\childdocname'}
\input{\childdocname}
\else
%    \end{macrocode}

% Include the two chapters:
%    \begin{macrocode}
\include{cdocsch1}
\include{cdocsch2}
%    \end{macrocode}

% Include the two parts unless only chapters should be displayed:
%    \begin{macrocode}
\ifchilddoc\else
\section{part three}
\input{cdocspt3}
\section{part four}
\input{cdocspt4}
\fi
%    \end{macrocode}

% Process as usual until here:
%    \begin{macrocode}
\fi
%    \end{macrocode}

% End of document body:
%    \begin{macrocode}
\end{document}
%    \end{macrocode}
%\iffalse
%</samplemain>
%\fi
%
% %%%%%%%%%%%%%%%%%%%%%%%%%%%%%%%%%%%%%%
% \paragraph{Chapter Include Files.}
%
% The include files are called |cdocsch1.tex| and |cdocsch2.tex|.
%
%\iffalse
%<*samplechap1|samplechap2>
%\fi

% Optional override for |\version| flag:
%    \begin{macrocode}
%%\providecommand{\version}{final}
%    \end{macrocode}

% Include the main document:
%    \begin{macrocode}
\input{childdoc.def}
\childdocof{cdocsamp}
%    \end{macrocode}

%\iffalse
%</samplechap1|samplechap2>
%\fi
%
%\iffalse
%<*samplechap1>
%\fi
% Some text for chapter 1:
%    \begin{macrocode}
\section{one}
some text in chapter one
%    \end{macrocode}

%\iffalse
%</samplechap1>
%\fi
% Some text for chapter 2:
%\iffalse
%<*samplechap2>
%\fi
%    \begin{macrocode}
\section{two}
more text in chapter two
%    \end{macrocode}

%\iffalse
%</samplechap2>
%\fi
%
% %%%%%%%%%%%%%%%%%%%%%%%%%%%%%%%%%%%%%%
% \paragraph{Part Include Files.}
%
% The include files are called |cdocspt3.tex| and |cdocspt4.tex|.
%
%\iffalse
%<*samplepart3|samplepart4>
%\fi

% Optional override for |\version| flag:
%    \begin{macrocode}
%%\providecommand{\version}{final}
%    \end{macrocode}

% Include the main document:
%    \begin{macrocode}
\input{childdoc.def}
\childdocby{cdocsamp}
%    \end{macrocode}

%\iffalse
%</samplepart3|samplepart4>
%\fi
%
%\iffalse
%<*samplepart3>
%\fi
% Some text for part 3:
%    \begin{macrocode}
some text in part three
%    \end{macrocode}

%\iffalse
%</samplepart3>
%\fi
% Some text for part 4:
%\iffalse
%<*samplepart4>
%\fi
%    \begin{macrocode}
more text in part four
%    \end{macrocode}

%\iffalse
%</samplepart4>
%\fi
%
% %%%%%%%%%%%%%%%%%%%%%%%%%%%%%%%%%%%%%%
% \paragraph{Forwarding for a Complete Draft.}
%
% The following forwarding file |cdocsdrf.tex|
% compiles the main document in draft mode:
%\iffalse
%<*sampledraft>
%\fi
%    \begin{macrocode}
\def\version{draft}
\input{childdoc.def}
\childdocforward{cdocsamp}
%    \end{macrocode}

%\iffalse
%</sampledraft>
%\fi
%
% %%%%%%%%%%%%%%%%%%%%%%%%%%%%%%%%%%%%%%
% \paragraph{Forwarding for Final Version of the Chapters.}
%
% The following forwarding files |cdocsfn1.tex| and |cdocsfn2.tex|
% (with identical content)
% compile the final versions of the child documents
% |cdocsch1.tex| and |cdocsch2.tex|, respectively:
%\iffalse
%<*samplefinal>
%\fi
%    \begin{macrocode}
\def\version{final}
\input{childdoc.def}
\childdocforwardprefix[cdocsamp]{cdocsfn}{cdocsch}
%    \end{macrocode}

%\iffalse
%</samplefinal>
%\fi
%
% %%%%%%%%%%%%%%%%%%%%%%%%%%%%%%%%%%%%%%
% \paragraph{Command Line Processing.}
%
% The following three command lines generate the output files
% |cdocscld|, |cdocscl1| and |cdocscl2|
% which should be identical to
% |cdocsdrf|, |cdocsch1| and |cdocsfn2|, respectively:
% \begin{center}
% \begin{tabular}{l}
% |latex -jobname cdocscld \|\\
% |  "\def\version{draft}\input{childdoc.def}\childdocforward{cdocsamp}"|\\
% |latex -jobname cdocscl1 \|\\
% |  "\input{childdoc.def}\childdocforward[cdocsamp]{cdocsch1}"|\\
% |latex -jobname cdocscl2 \|\\
% |  "\def\version{final}\input{childdoc.def}\childdocforward{cdocsch2}"|
% \end{tabular}
% \end{center}
% Note that the trailing backslash on each first line
% merely continues the input to the second line
% (for convenient cut ant paste).
% Furthermore, the command |latex| can be replaced by any
% of its alternative versions such as |pdflatex|.
%
% %%%%%%%%%%%%%%%%%%%%%%%%%%%%%%%%%%%%%%%%%%%%%%%%%%%%%%%%%%%%%%%%%%%%%%%%%%%%%%
% %%%%%%%%%%%%%%%%%%%%%%%%%%%%%%%%%%%%%%%%%%%%%%%%%%%%%%%%%%%%%%%%%%%%%%%%%%%%%%
% \section{Implementation}
%\iffalse
%<*package>
%\fi
%
% This section describes the definitions file |childdoc.def|.

% The definitions cannot be loaded using |\usepackage| or |\RequirePackage|
% which has a mechanism to prevent loading a style file more than once.
% When loading the definitions by means of |\input|
% multiple instances have to be prevented manually:
%\iffalse
%This code needs to be before the `\ProvidesFile' directive
%which is defined at the beginning of this file.
%Therefore it is also placed there and commented out here.
%</package>
%<*discard>
%\fi
%    \begin{macrocode}
\ifdefined\childdocmain\endinput\fi
%    \end{macrocode}
%\iffalse
%</discard>
%<*package>
%\fi
%
% \macro{\ifchilddoc}
% \macro{\ifchilddocmanual}
% The conditional |\ifchilddoc| tells whether a
% child (true) or main (false) document is being compiled.
% The conditional |\ifchilddocmanual| tells whether
% the |\includeonly| mechanism is used (false) or
% the selection of child files must be performed manually (true).
% The definitions initialise to false:
%    \begin{macrocode}
\newif\ifchilddoc
\newif\ifchilddocmanual
%    \end{macrocode}

% \macro{\childdocname}
% \macro{\childdocjob}
% The macro |\childdocname| stores the name of the main document
% to be compiled. The macro |\childdocjob| stores the name of
% the document on which the \LaTeX{} compiler was originally invoked.
% The content of |\jobname| cannot be compared
% to filenames specified in the source due to different catcodes.
% The following code rescans |\jobname|, stores the result
% in |\childdocname| and saves a copy in |\childdocjob|:
%    \begin{macrocode}
\edef\childdocname{\scantokens\expandafter{\jobname\noexpand}}
\let\childdocjob\childdocname
%    \end{macrocode}

% \macro{\childdocdisable}
% The macro |\childdocdisable| prevents the main file
% from being processed more than once.
% At this stage, the main document command |\childdocmain|
% is assumed to be called once again where it should do nothing.
% Any subsequent call to it should prevent
% a secondary processing of the main document
% It overwrites the forwarding commands
% |\childdocof| and |\childdocforward|
% with empty macros to prevent further inclusions of the main document:
%    \begin{macrocode}
\newcommand{\childdocdisable}
{
  \renewcommand{\childdocmain}[1]{\renewcommand{\childdocmain}[1]{\endinput}}
  \renewcommand{\childdocof}[1]{}
  \renewcommand{\childdocby}[2][]{}
  \renewcommand{\childdocforward}[2][]{}
  \renewcommand{\childdocdisable}{}
}
%    \end{macrocode}

% \macro{\childdocmain}
% The macro |\childdocmain| is to be called at the top of the main file
% with nothing or the main filename (without extension) as argument.
% First, it breaks loops.
% If the argument is not empty and does not match |\childdocname|
% (which is set by the first inclusion of |childdoc.def|),
% |\ifchilddoc| is set to true, |\includeonly| is applied to the child file
% and |\jobname| is set to the main file
% (for proper handling of |.aux| files):
%    \begin{macrocode}
\newcommand{\childdocmain}[1]
{
  \childdocdisable\childdocmain{}
  \if?#1?\else
    \begingroup
      \def\childdoctmp{#1}
      \ifx\childdoctmp\childdocname
        \def\childdoctmp{}
      \else
        \def\childdoctmp
        {
          \childdoctrue
          \includeonly{\childdocname}
          \def\childdocjob{#1}
          \def\jobname{#1}
        }
      \fi
      \expandafter
    \endgroup
    \childdoctmp
  \fi
}
%    \end{macrocode}

% \macro{\childdocof}
% The command |\childdocof| redirects
% compilation to the main file |#1|.
%    \begin{macrocode}
\newcommand{\childdocof}[1]
{
  \childdocdisable
  \childdoctrue
  \includeonly{\childdocname}
  \def\jobname{#1}
  \def\childdocjob{#1}
  \input{#1}
}
%    \end{macrocode}

% \macro{\childdocby}
% The command |\childdocby| ....
%    \begin{macrocode}
\newcommand{\childdocby}[2][]
{
  \childdocdisable
  \childdoctrue
  \childdocmanualtrue
  \if?#1?\else
    \def\jobname{#2}
  \fi
  \def\childdocjob{#2}
  \input{#2}
  \endinput
}
%    \end{macrocode}

% \macro{\childdocforward}
% The command |\childdocforward| redirects
% compilation to the main file or
% (if the optional argument is given) a child file.
% Parameters are set as if the main file
% or a child file starting with |\childdocof| was compiled.
% Then compilation is handed over to the main file:
%    \begin{macrocode}
\newcommand{\childdocforward}[2][]
{
  \begingroup
    \if?#1?
      \def\childdoctmp
      {
        \def\childdocname{#2}
        \def\childdocjob{#2}
        \def\jobname{#2}
        \input{#2}
        \endinput
      }
    \else
      \def\childdoctmp
      {
        \childdocdisable
        \def\childdocname{#2}
        \childdoctrue
        \includeonly{#2}
        \def\childdocjob{#1}
        \def\jobname{#1}
        \input{#1}
        \endinput
      }
    \fi
    \expandafter
  \endgroup
  \childdoctmp
}
%    \end{macrocode}

% \macro{\childdocforwardprefix}
% The command |\childdocforwardprefix| redirects
% compilation to the main or a child file by means of a pattern.
% The prefix |#1| in the current filename is replaced by |#2|
% and the suffix of the current filename is kept
% (it is assumed that the filename does not contain the substring `|~~~|'
% which is used as a delimiter).
% Compilation is handed over to the new file by |\childdocforward|:
%    \begin{macrocode}
\newcommand{\childdocforwardprefix}[3][]
{
  \begingroup
    \def\childdocextract #2##1~~~{\def\childdoctmp{\childdocforward[#1]{#3##1}}}
    \expandafter\childdocextract\childdocname~~~
    \expandafter
  \endgroup
  \childdoctmp
}
%    \end{macrocode}

% \macro{\childdoc}
% The deprecated macro |\childdoc| is a legacy version of |\childdocmain|:
%    \begin{macrocode}
\newcommand{\childdoc}{\childdocmain}
%    \end{macrocode}

% \macro{\childdocredirect}
% The deprecated macro |\childdocredirect| is a legacy version
% of |\childdocforward| and |\childdocforwardprefix|:
%    \begin{macrocode}
\newcommand{\childdocredirect}[2][]
{
  \begingroup
    \if?#1?
      \def\childdoctmp{\childdocforward{#2}}
    \else
      \def\childdoctmp{\childdocforwardprefix{#1}{#2}}
    \fi
    \expandafter
  \endgroup
  \childdoctmp
}
%    \end{macrocode}

%\iffalse
%</package>
%\fi
%
\endinput

\childdocforwardprefix[cdocsamp]{cdocsfn}{cdocsch}
%    \end{macrocode}

%\iffalse
%</samplefinal>
%\fi
%
% %%%%%%%%%%%%%%%%%%%%%%%%%%%%%%%%%%%%%%
% \paragraph{Command Line Processing.}
%
% The following three command lines generate the output files
% |cdocscld|, |cdocscl1| and |cdocscl2|
% which should be identical to
% |cdocsdrf|, |cdocsch1| and |cdocsfn2|, respectively:
% \begin{center}
% \begin{tabular}{l}
% |latex -jobname cdocscld \|\\
% |  "\def\version{draft}% \iffalse
%
% childdoc.dtx Copyright (C) 2017-2018 Niklas Beisert
%
% This work may be distributed and/or modified under the
% conditions of the LaTeX Project Public License, either version 1.3
% of this license or (at your option) any later version.
% The latest version of this license is in
%   http://www.latex-project.org/lppl.txt
% and version 1.3 or later is part of all distributions of LaTeX
% version 2005/12/01 or later.
%
% This work has the LPPL maintenance status `maintained'.
%
% The Current Maintainer of this work is Niklas Beisert.
%
% This work consists of the files childdoc.dtx and childdoc.ins
% and the derived files childdoc.def and cdocsamp.tex with
% cdocsch1.tex, cdocsch2.tex, cdocsdrf.tex, cdocsfn1.tex, cdocsfn2.tex.
%
%<package>\ifdefined\childdocmain\endinput\fi
%<package>\ProvidesFile{childdoc.def}[2018/12/30 v2.0 child document driver]
%<samplemain>\ProvidesFile{cdocsamp.tex}[2018/12/30 v2.0 sample for childdoc]
%<*driver>
%\ProvidesFile{childdoc.drv}[2018/12/30 v2.0 childdoc reference manual file]
\PassOptionsToClass{10pt,a4paper}{article}
\documentclass{ltxdoc}

\usepackage[margin=35mm]{geometry}
\usepackage{hyperref}
\usepackage{hyperxmp}
\usepackage[usenames]{color}

\hypersetup{colorlinks=true}
\hypersetup{pdfstartview=FitH}
\hypersetup{pdfpagemode=UseNone}
\hypersetup{pdfsource={}}
\hypersetup{pdflang={en-UK}}
\hypersetup{pdfcopyright={Copyright 2017-2018 Niklas Beisert.
  This work may be distributed and/or modified under the
  conditions of the LaTeX Project Public License, either version 1.3
  of this license or (at your option) any later version.}}
\hypersetup{pdflicenseurl={http://www.latex-project.org/lppl.txt}}
\hypersetup{pdfcontactaddress={ETH Zurich, ITP, HIT K,
  Wolfgang-Pauli-Strasse 27}}
\hypersetup{pdfcontactpostcode={8093}}
\hypersetup{pdfcontactcity={Zurich}}
\hypersetup{pdfcontactcountry={Switzerland}}
\hypersetup{pdfcontactemail={nbeisert@itp.phys.ethz.ch}}
\hypersetup{pdfcontacturl={http://people.phys.ethz.ch/\xmptilde nbeisert/}}

\newcommand{\secref}[1]{\hyperref[#1]{section \ref*{#1}}}

\parskip1ex
\parindent0pt
\let\olditemize\itemize
\def\itemize{\olditemize\parskip0pt}

\begin{document}

\title{The \textsf{childdoc} Package}
\hypersetup{pdftitle={The childdoc Package}}
\author{Niklas Beisert\\[2ex]
  Institut f\"ur Theoretische Physik\\
  Eidgen\"ossische Technische Hochschule Z\"urich\\
  Wolfgang-Pauli-Strasse 27, 8093 Z\"urich, Switzerland\\[1ex]
  \href{mailto:nbeisert@itp.phys.ethz.ch}
  {\texttt{nbeisert@itp.phys.ethz.ch}}}
\hypersetup{pdfauthor={Niklas Beisert}}
\hypersetup{pdfsubject={Manual for the LaTeX2e Package childdoc}}
\date{30 December 2018, \textsf{v2.0}}
\maketitle

\begin{abstract}\noindent
\textsf{childdoc} is a \LaTeXe{} package
that enables the direct compilation
of document sections included by |\include|
to individual files.
\end{abstract}

\begingroup
\parskip0ex
\tableofcontents
\endgroup

%%%%%%%%%%%%%%%%%%%%%%%%%%%%%%%%%%%%%%%%%%%%%%%%%%%%%%%%%%%%%%%%%%%%%%%%%%%%%%%%
%%%%%%%%%%%%%%%%%%%%%%%%%%%%%%%%%%%%%%%%%%%%%%%%%%%%%%%%%%%%%%%%%%%%%%%%%%%%%%%%
\section{Introduction}

\LaTeX{} provides a mechanism to structure a large document (such as a book)
into a main file and several child files (containing the chapters)
using the |\include| command.
This mechanism is beneficial for documents
which span hundreds of pages in order to
make the source file(s) more manageable.
Moreover, compilation can be restricted to
selected child files by means of the |\includeonly| command.
The latter feature can be used to reduce the compilation time while editing
(this was significantly more useful in the earlier days of \LaTeX{})
or to generate a smaller document which is easier to navigate.
Another application of |\includeonly| is to generate
documents consisting of selected parts of the complete document.

However, there are a few drawbacks of the plain |\include| mechanism:
\begin{itemize}
\item
The child files cannot be compiled on their own,
they can only be compiled via the main file.
A naive editing environment
(such as a text editor with an option
to have the current file processed by \LaTeX)
may require one to switch to the main file before compiling;
attempting to compile the child file produces errors.
\item
The main file must be modified (each time)
to adjust the |\includeonly| command
to the present needs. This easily leaves the main file in a messy state.
\item
The generated document will always carry the filename
of the main document. This is inconvenient if
several child files are to be compiled and
to be kept for distribution.
\end{itemize}

The present package provides a simple interface
to make child files individually compilable by \LaTeX{}.
Compiling a child file then has the same effect as compiling
the main file with an |\includeonly| command
to select the appropriate child.
Moreover the generated document will carry the name of the child
rather than the main file.
This resolves all three above issues.

This feature is meant to make the editing of books,
thesis documents and lecture notes somewhat more convenient.
However, the package can also be used efficiently for
composing a series of documents (such as exercise sheets)
which are typically distributed individually.
It then assists the author in generating the individual documents
(potentially in different versions)
as well as a document containing the collected series.
Another application is in developing style files
or other kinds of included material
where compilation of the style file could redirect
to a sample or test file.

%%%%%%%%%%%%%%%%%%%%%%%%%%%%%%%%%%%%%%%%%%%%%%%%%%%%%%%%%%%%%%%%%%%%%%%%%%%%%%%%
%%%%%%%%%%%%%%%%%%%%%%%%%%%%%%%%%%%%%%%%%%%%%%%%%%%%%%%%%%%%%%%%%%%%%%%%%%%%%%%%
\section{Usage}

First of all, the package \textsf{childdoc} is \emph{not} a standard
\LaTeXe{} |.sty| style file! Therefore it needs to be invoked in
a non-standard way.

%%%%%%%%%%%%%%%%%%%%%%%%%%%%%%%%%%%%%%%%%%%%%%%%%%%%%%%%%%%%%%%%%%%%%%%%%%%%%%%%
\subsection{Included Files}
\label{sec:include}

%%%%%%%%%%%%%%%%%%%%%%%%%%%%%%%%%%%%%%%%
\DescribeMacro{\childdocmain}
To use the package, add the commands
\begin{center}
\begin{tabular}{l}
|\input{childdoc.def}|\\
|\childdocmain{}|\\
\end{tabular}
\end{center}
at the very top of the main \LaTeX{} file,
in particular \emph{before} the |\documentclass| statement!
The argument of |\childdocmain| should be left empty
(but it must be present).

%%%%%%%%%%%%%%%%%%%%%%%%%%%%%%%%%%%%%%%%
\DescribeMacro{\childdocof}
Furthermore, add the commands
\begin{center}
\begin{tabular}{l}
|\input{childdoc.def}|\\
|\childdocof{|\textit{main}|}|\\
\end{tabular}
\end{center}
at the top of every child file \textit{child}
which is included by |\include{|\textit{child}|}|
from within the main file
(or at least for those files to be compiled individually).
The argument \textit{main} must be the filename of the main file.

There are a couple of
considerations in setting up the main and child documents:

%%%%%%%%%%%%%%%%%%%%%%%%%%%%%%%%%%%%%%%%
\paragraph{Restrictions.}

Please note the following restrictions:
\begin{itemize}
\item
|\childdocmain| must be called with one argument \textit{main}
to ensure compatibility with earlier version of the package.
It must either be empty (|\childdocmain{}|)
or precisely match the filename of the main file in which it is specified.
See \secref{sec:detection} for further information.
\item
The filename \textit{main} must be specified without the |.tex| extension.
\item
The filename \textit{main} is case sensitive
(even in case-insensitive file systems)
due to internal string comparison.
\item
The argument \textit{main} should be fully expanded, it cannot be a macro.
\item
Subdirectories and special characters should be avoided in filenames.
\item
The command |\childdocmain{|\textit{main}|}| must be followed by a whitespace.
It should not be followed immediately by another command
or by a comment mark `|%|'.
This is because the \TeX{} parser reads the token immediately following
the argument of |\childdocmain| and puts it
at the beginning of every child section;
however, a white\-space is ignored.
\end{itemize}

%%%%%%%%%%%%%%%%%%%%%%%%%%%%%%%%%%%%%%%%
\paragraph{Content of Main File.}

It is advisable to place all content in the child files included by |\include|.
Any output contained in the main file will appear in all child documents
unless suppressed manually;
it cannot be suppressed automatically by the |\includeonly| directive
and thus should normally be avoided.
A method to include some content in the main file
by means of conditional processing is described in \secref{sec:conditional}.

%%%%%%%%%%%%%%%%%%%%%%%%%%%%%%%%%%%%%%%%
\paragraph{Page Numbering.}

When only a part of the document is compiled,
the appropriate numbering of pages
(as well as other status parameters)
is determined from the |.aux| files.
The latter contain information from previous passes.
However this information needs to propagate through
all intermediate child documents.
Therefore the page numbering in child documents may well
be inconsistent until the complete document is compiled at least once.

A useful (if unconventional) way to always ensure a consistent
page numbering is to restart the numbering in each child document
and denote the pages by `\textit{child}|.|\textit{page}'
where \textit{child} represents the chapter/section number of the child file.
This can be achieved by the command
|\numberwithin{page}{|\textit{child}|}|
of the \textsf{amsmath} package
where \textit{child} can be |chapter| or |section|
depending on the chosen structuring.
Alternatively, one can modify the macro |\thepage| appropriately
and reset the counter |page| at the start of each child file.

%%%%%%%%%%%%%%%%%%%%%%%%%%%%%%%%%%%%%%%%%%%%%%%%%%%%%%%%%%%%%%%%%%%%%%%%%%%%%%%%
\subsection{Conditional Processing}
\label{sec:conditional}

The package provides a mechanism to compile different versions
of a document. To customise the versions further some conditional processing
can come in handy to distinguish which version is being compiled.
The package provides two macros to describe the compilation context:

%%%%%%%%%%%%%%%%%%%%%%%%%%%%%%%%%%%%%%%%
\DescribeMacro{\ifchilddoc}
The conditional |\ifchilddoc| distinguishes between the compilation of
child documents and the main document:
%
\begin{center}
|\ifchilddoc |\textit{child-code}| |[|\||else |\textit{main-code}]| \||fi|
\end{center}

%%%%%%%%%%%%%%%%%%%%%%%%%%%%%%%%%%%%%%%%
\DescribeMacro{\childdocname}
\DescribeMacro{\childdocjob}
The macro |\childdocname| contains the filename (without extension)
of the main or child file being processed.
Note that |\childdocjob| will always contain the name of the main file.

%%%%%%%%%%%%%%%%%%%%%%%%%%%%%%%%%%%%%%%%
\paragraph{Title Page.}

Conditional processing can be used to include a title or banner page
in the main document when proper precautions are taken.
Importantly, the code in the main file should ensure that the page counter
(as well as other status parameters which are stored in the |.aux| files)
takes the same value after the conditional processing.
Otherwise the page numbers may take divergent values
depending on which part is compiled.

For example, a title page could be declared by:
%
\begin{center}
\begin{tabular}{l}
|\ifchilddoc\||else|\\
|\addtocounter{page}{-1}|\\
\textit{code for title page}\\
|\newpage|\\
|\||fi|
\end{tabular}
\end{center}
%
A banner page for the child documents can be generated by:
%
\begin{center}
\begin{tabular}{l}
|\ifchilddoc|\\
|\addtocounter{page}{-1}|\\
\textit{code for banner page}\\
|\newpage|\\
|\||fi|
\end{tabular}
\end{center}
%
Here one could write a message such as:
\begin{center}
|This is the part \childdocname{} of \childdocjob{}.|
\end{center}

%%%%%%%%%%%%%%%%%%%%%%%%%%%%%%%%%%%%%%%%%%%%%%%%%%%%%%%%%%%%%%%%%%%%%%%%%%%%%%%%
\subsection{Flags}
\label{sec:flags}

The package makes it easy to generate different versions
of the main or child documents.
To this end compilation flags can be defined
and assigned different default values.
They will be particularly useful in conjunction
with the forwarding mechanism described in \secref{sec:forward}.

For example, it may be useful to have a flag |\version|
which can be set to |draft| or |final|.
The document source will contain some conditional code
depending on the value of |\version|.
Suppose further, the flag should default to |final| for the main file
and to |draft| for child files
which is a natural assignment for editing the document.
This is achieved by placing the following code
in the preamble of the main document
(below the |\childdocmain| directive):
%
\begin{center}
\begin{tabular}{l}
|\ifchilddoc|\\
|\providecommand{\version}{draft}|\\
|\||else|\\
|\providecommand{\version}{final}|\\
|\||fi|
\end{tabular}
\end{center}
%
The definition by |\providecommand| makes sure
that previous definitions are not overwritten.
Further statements |\providecommand{\version}{...}|
can thus be added before the above code to override it.

For the main file, one might add a line
(between |\childdocmain| and the above block)
%
\begin{center}
|%\ifchilddoc\||else\providecommand{\version}{draft}\||fi|
\end{center}
%
which can be uncommented to produce a draft version.
Likewise one can add a line to the very top of a child file
(above the |\childdocof{|\textit{main}|}| directive)
%
\begin{center}
|%\providecommand{\version}{final}|
\end{center}
%
which can be uncommented to produce the final version of this child document.

%%%%%%%%%%%%%%%%%%%%%%%%%%%%%%%%%%%%%%%%%%%%%%%%%%%%%%%%%%%%%%%%%%%%%%%%%%%%%%%%
\subsection{Forwarding}
\label{sec:forward}

Different versions of the main or child documents
using compilation flags as described in \secref{sec:flags}
can be (permanently) stored in different files
for convenient compilation, viewing and distribution.
To this end, the package defines a command
to pass on compilation to a different file:

%%%%%%%%%%%%%%%%%%%%%%%%%%%%%%%%%%%%%%%%
\DescribeMacro{\childdocforward}
The command |\childdocforward| redirects processing to
another source file:
%
\begin{center}
\begin{tabular}{l}
|\input{childdoc.def}|\\
|\childdocforward[|\textit{main}|]{|\textit{dest}|}|\\
\end{tabular}
\end{center}
%
The argument \textit{dest} is the destination file
(without extension).
It should be the main file or one of the child files.
Note that further \textsf{childdoc} directives
such as |\childdocof| and |\childdocforward|
in the indicated file will be processed in this form.
The optional argument \textit{main}
passes on directly to the main file \textit{main}
while pretending to compile the child \textit{dest}.
This form behaves as if \textit{dest}
issues |\childdocof{|\textit{main}|}| right away,
and no further \textsf{childdoc} directives will be processed.

%%%%%%%%%%%%%%%%%%%%%%%%%%%%%%%%%%%%%%%%
\DescribeMacro{\...prefix}
In the alternative form |\childdocforwardprefix|,
%
\begin{center}
\begin{tabular}{l}
|\input{childdoc.def}|\\
|\childdocforwardprefix[|\textit{main}|]{|\textit{prefix}|}{|\textit{dest}|}|
\end{tabular}
\end{center}
%
the destination file is determined by a pattern
depending on the current file:
To make this work, the current file must be called
`{\textit{prefix}\hspace{0.2em}\textit{suffix}}'
with \textit{prefix} matching precisely the argument.
Processing is then passed on to the file
`{\textit{dest}\hspace{0.2em}\textit{suffix}}'.
Surely, the same effect is achieved by
directly specifying the
argument `{\textit{dest}\hspace{0.2em}\textit{suffix}}'
in the first form.
However, that requires to set up a different file
for each child. With the alternative form of the command
all these files can have exactly the same content
which simplifies setting them up and maintaining them.

For example, the following file |draft.tex|
with a compilation flag |\version| as described in \secref{sec:flags}
compiles the main document as a draft:
%
\begin{center}
\begin{tabular}{l}
|\def\version{draft}|\\
|\input{childdoc.def}|\\
|\childdocforward{|\textit{main}|}|
\end{tabular}
\end{center}
%
Likewise, the following files |final|\textit{nn}|.tex|
compile the final version of the child document
|child|\textit{nn}|.tex|:
%
\begin{center}
\begin{tabular}{l}
|\def\version{final}|\\
|\input{childdoc.def}|\\
|\childdocforwardprefix{final}{child}|
\end{tabular}
\end{center}
%

Note that when several versions of a main file and/or of each child file
are to be generated, it may be convenient to set up a |Makefile| or
shell script to automatise the process.

%%%%%%%%%%%%%%%%%%%%%%%%%%%%%%%%%%%%%%%%%%%%%%%%%%%%%%%%%%%%%%%%%%%%%%%%%%%%%%%%
\subsection{Command Line Processing}
\label{sec:commandline}

The effect of redirection files can also be achieved by invoking
the \LaTeX{} compiler with a more elaborate command line.
Most conveniently this should be done as part
of a shell script or a |Makefile|.

When using \textsf{childdoc} in the main file, the following
command lines effectively perform a redirection
(note that depending on the shell being used,
backslashes may have to be doubled: `|\|' $\to$ `|\\|'):
%
\begin{center}
|... -jobname "|\textit{target}|" |\\|"|[\textit{flags}]%
|\input{childdoc.def}\childdocforward[|\textit{main}|]{|\textit{dest}|}"|
\end{center}
%
Here \textit{target} is the name of the output file,
\textit{main} is the name of the main file
and \textit{dest} is the name of the main or child file to be processed
(all filenames without extensions).
The optional argument \textit{main} can be omitted
if \textit{main} matches \textit{dest}.
Optionally, compilation \textit{flags} can be defined via |\def| commands.
This command line makes the \TeX{} engine believe
it is compiling the file \textit{target}
whose content is specified as the latter parameter.
The provided code then forwards the processing to
\textit{main} or \textit{dest} as described in \secref{sec:forward}.

%%%%%%%%%%%%%%%%%%%%%%%%%%%%%%%%%%%%%%%%%%%%%%%%%%%%%%%%%%%%%%%%%%%%%%%%%%%%%%%%
\subsection{Include by Input}
\label{sec:input}

Including child documents by |\include| has some restrictions by design.
Most notably, the content of a child document always occupies
its own set of pages; pages cannot be shared between child documents.
Usually, this behaviour makes perfect sense
because each child document contain an essential part of the document.
However, in some situations it may be desirable to compose
a document from a collection of parts
without having mandatory page breaks between then.
For this case, the package
provides a mechanism to include parts
by |\input| which can also be processed individually.
However, by construction this mechanism
requires manual handling of the content to be output.

%%%%%%%%%%%%%%%%%%%%%%%%%%%%%%%%%%%%%%%%
\DescribeMacro{\ifchilddocmanual}
The main file should be prepared as usual, see \secref{sec:include}.
However, the document body must make a distinction
between processing of an individual part and of the main document, e.g.:
%
\begin{center}
\begin{tabular}{l}
|\ifchilddocmanual|\\
|\input{\childdocname}|\\
|\||else|\\
\textit{document body with }|\input{|\textit{part}|}|\\
|\||fi|
\end{tabular}
\end{center}
%
The conditional |\ifchilddocmanual| is true whenever
a part to be included by |\input| is being compiled,
and the name of the part is stored in |\childdocname|.

%%%%%%%%%%%%%%%%%%%%%%%%%%%%%%%%%%%%%%%%
\DescribeMacro{\childdocby}
Each part to be included by |\input| should start with:
%
\begin{center}
\begin{tabular}{l}
|\input{childdoc.def}|\\
|\childdocby{|\textit{main}|}|\\
\end{tabular}
\end{center}
%
The directive |\childdocby| is similar to |\childdocof|
described in \secref{sec:include},
but the subsequent selection of content must be done manually.
To that end, both |\ifchilddoc| and |\ifchilddocmanual|
will be true upon processing of a part,
and the name of the part is stored in |\childdocname|.
Note that |\jobname| will be set to the filename of the current part
so that each part receives an individual |.aux| file
that does not interfere with the |.aux| file(s) of the main document.
This behaviour can be altered by the alternative form
|\childdocby[*]{|\textit{main}|}| (with a non-empty optional argument)
which uses the |.aux| file of the main document
by setting |\jobname| to \textit{main}.

%%%%%%%%%%%%%%%%%%%%%%%%%%%%%%%%%%%%%%%%%%%%%%%%%%%%%%%%%%%%%%%%%%%%%%%%%%%%%%%%
\subsection{Driver Development}
\label{sec:driver}

The \textsf{childdoc} mechanism can also be use for the development
of definition files such as \LaTeX{} styles or classes.
This case differs from the above setup with multiple parts
included by |\include| in that no |\includeonly| should be invoked.
This can be achieved by starting the include file
(before |\ProvidesPackage|) with:
%
\begin{center}
\begin{tabular}{l}
|\input{childdoc.def}|\\
|\childdocforward{|\textit{main}|}|\\
\end{tabular}
\end{center}
%
or alternatively with:
%
\begin{center}
\begin{tabular}{l}
|\input{childdoc.def}|\\
|\childdocby{|\textit{main}|}|\\
\end{tabular}
\end{center}
%
Both forms have slightly different effects as described above.
The main file is prepared as usual, see \secref{sec:include}.

%%%%%%%%%%%%%%%%%%%%%%%%%%%%%%%%%%%%%%%%%%%%%%%%%%%%%%%%%%%%%%%%%%%%%%%%%%%%%%%%
\subsection{Legacy Detection}
\label{sec:detection}

The directive |\childdocmain| in the main file can detect
whether the complete document or merely a child is to be compiled
even without using the directive |\childdocof|.
This method is deprecated because it is less robust
and there is no compelling reason to use it;
it is merely provided for backward compatibility
and it may be removed in future versions.

If the detection mechanism is to be used,
it is mandatory to correctly specify
the filename of the main file as the argument of |\childdocmain|:
%
\begin{center}
\begin{tabular}{l}
|\input{childdoc.def}|\\
|\childdocmain{|\textit{main}|}|\\
\end{tabular}
\end{center}
%
If |\jobname| does not match the argument \textit{main} of |\childdocmain|,
it is assumed that |\jobname| points to the child file to be compiled.
When using |\childdocmain| with the main file specified as argument,
it suffices to start a child file
with just |\input{|\textit{main}|}|
without loading of the package and using |\childdocof|.
If instead all processing is done
with the appropriate \textsf{childdoc} directives,
the argument of \textit{main} of |\childdocmain| can be empty.

An alternative version of the command line processing described
in \secref{sec:commandline} using the detection mechanism reads:
%
\begin{center}
|... -jobname "|\textit{target}|" "|[\textit{flags}]%
[|\def\jobname{|\textit{dest}|}|]|\input{|\textit{main}|}"|
\end{center}

%%%%%%%%%%%%%%%%%%%%%%%%%%%%%%%%%%%%%%%%%%%%%%%%%%%%%%%%%%%%%%%%%%%%%%%%%%%%%%%%
\subsection{Manual Code}
\label{sec:manual}

In case one cannot be certain whether the definitions file |childdoc.def|
is installed on the target \TeX{} distribution
and one prefers not to ship it,
it is conceivable to paste a few relevant commands into the sources.

To that end, drop all statements |\input{childdoc.def}|
and perform the replacements as outlined below.
Instead of |\childdocmain{|\textit{main}|}| add the following code
to the top of the main file:
%
\begin{center}
\begin{tabular}{l}
|\||ifdefined\childdocname\endinput\||fi\newif\ifchilddoc|\\
|\edef\childdocname{\scantokens\expandafter{\jobname\noexpand}}|\\
|\def\childdocmain{|\textit{main}|}\||ifx\childdocmain\childdocname\||else|\\
|\childdoctrue\includeonly{\childdocname}\let\jobname\childdocmain\||fi|\\
\end{tabular}
\end{center}
%
Instead of |\childdocof{|\textit{main}|}| just include the main file
at the top of each child file:
%
\begin{center}
|\input{|\textit{main}|}|
\end{center}
%
A simple redirection |\childdocforward{|\textit{dest}|}| is achieved by:
%
\begin{center}
|\def\jobname{|\textit{dest}|}\input{\jobname}|
\end{center}
%
The redirection with prefix
|\childdocforwardprefix[|\textit{prefix}|]{|\textit{dest}|}|
is accomplished by:
%
\begin{center}
\begin{tabular}{l}
|{\edef\jobname{\scantokens\expandafter{\jobname\noexpand}}|\\
|\def\redirectjob |\textit{prefix}|#1~~~{\gdef\jobname{|\textit{dest}|#1}}|\\
|\expandafter\redirectjob\jobname~~~}\input{\jobname}|
\end{tabular}
\end{center}

In an alternative approach,
child documents can be compiled by a specific command line
without additional code or specific definitions:
%
\begin{center}
|... -jobname "|\textit{target}|" "|[\textit{flags}]%
|\includeonly{|\textit{dest}|}\input{|\textit{main}|}"|
\end{center}
%

%%%%%%%%%%%%%%%%%%%%%%%%%%%%%%%%%%%%%%%%%%%%%%%%%%%%%%%%%%%%%%%%%%%%%%%%%%%%%%%%
%%%%%%%%%%%%%%%%%%%%%%%%%%%%%%%%%%%%%%%%%%%%%%%%%%%%%%%%%%%%%%%%%%%%%%%%%%%%%%%%
\section{Information}

%%%%%%%%%%%%%%%%%%%%%%%%%%%%%%%%%%%%%%%%%%%%%%%%%%%%%%%%%%%%%%%%%%%%%%%%%%%%%%%%
\subsection{Copyright}

Copyright \copyright{} 2017--2018 Niklas Beisert

This work may be distributed and/or modified under the
conditions of the \LaTeX{} Project Public License, either version 1.3
of this license or (at your option) any later version.
The latest version of this license is in
  \url{http://www.latex-project.org/lppl.txt}
and version 1.3 or later is part of all distributions of \LaTeX{}
version 2005/12/01 or later.

This work has the LPPL maintenance status `maintained'.

The Current Maintainer of this work is Niklas Beisert.

This work consists of the files |README.txt|, |childdoc.ins| and |childdoc.dtx|
as well as the derived files |childdoc.def|, |cdocsamp.tex|
with |cdocsch1.tex|, |cdocsch2.tex|, |cdocspt3.tex|, |cdocspt4.tex|,
|cdocsdrf.tex|, |cdocsfn1.tex|, |cdocsfn2.tex|
as well as |childdoc.pdf|.

%%%%%%%%%%%%%%%%%%%%%%%%%%%%%%%%%%%%%%%%%%%%%%%%%%%%%%%%%%%%%%%%%%%%%%%%%%%%%%%%
\subsection{Files and Installation}

The package consists of the files:
%
\begin{center}
\begin{tabular}{ll}
    |README.txt|   & readme file \\
    |childdoc.ins| & installation file \\
    |childdoc.dtx| & source file \\
    |childdoc.def| & definition file \\
    |cdocsamp.tex| & sample main file \\
    |cdocsch1.tex| & sample include file \\
    |cdocsch2.tex| & sample include file \\
    |cdocspt3.tex| & sample part file \\
    |cdocspt4.tex| & sample part file \\
    |cdocsdrf.tex| & sample redirection file \\
    |cdocsfn1.tex| & sample redirection file \\
    |cdocsfn2.tex| & sample redirection file \\
    |childdoc.pdf| & manual
\end{tabular}
\end{center}
%
The distribution consists of the files
|README.txt|, |childdoc.ins| and |childdoc.dtx|.
%
\begin{itemize}
\item
Run (pdf)\LaTeX{} on |childdoc.dtx|
to compile the manual |childdoc.pdf| (this file).
\item
Run \LaTeX{} on |childdoc.ins| to create the definitions file |childdoc.def|
and the sample |cdocsamp.tex| with include files
|cdocsch1.tex|, |cdocsch2.tex|, |cdocspt3.tex|, |cdocspt4.tex|,
|cdocsdrf.tex|, |cdocsfn1.tex|, |cdocsfn2.tex|.
Then copy the file |childdoc.def| to an appropriate directory of your \LaTeX{}
distribution, e.g.\ \textit{texmf-root}|/tex/latex/childdoc|.
\end{itemize}

%%%%%%%%%%%%%%%%%%%%%%%%%%%%%%%%%%%%%%%%%%%%%%%%%%%%%%%%%%%%%%%%%%%%%%%%%%%%%%%%
\subsection{Related CTAN Packages}

There are several other packages which offer a similar functionality:
%
\begin{itemize}
\item
The packages
\href{http://ctan.org/pkg/docmute}{\textsf{docmute}},
\href{http://ctan.org/pkg/includex}{\textsf{includex}} and
\href{http://ctan.org/pkg/standalone}{\textsf{standalone}}
provide commands to include only the document body of
a child file thus allowing both files to be compiled individually.
\item
The packages \href{http://ctan.org/pkg/subdocs}{\textsf{subdocs}}
and \href{http://ctan.org/pkg/subfiles}{\textsf{subfiles}}
provide structures in which the main and child documents can be
encapsulated and allowing them to be compiled individually.
The inclusion mechanism is different from the conventional |\include|.
\item
The package \href{http://ctan.org/pkg/combine}{\textsf{combine}}
is an elaborate solution to combine several documents into one.
\end{itemize}
%
See also the CTAN topic \href{http://ctan.org/topic/subdocs}{\textsf{subdocs}}
for further related packages.
The present package differs from the above solutions in that
a document structure constructed with the conventional |\include| mechanism
just needs two extra commands at the top of every file
such that all constituent files can be compiled individually.

%%%%%%%%%%%%%%%%%%%%%%%%%%%%%%%%%%%%%%%%%%%%%%%%%%%%%%%%%%%%%%%%%%%%%%%%%%%%%%%%
%\subsection{Feature Suggestions}
%
%The following is a list of features which may be useful for future
%versions of this package:
%%
%\begin{itemize}
%\item
%\ldots
%\end{itemize}

%%%%%%%%%%%%%%%%%%%%%%%%%%%%%%%%%%%%%%%%%%%%%%%%%%%%%%%%%%%%%%%%%%%%%%%%%%%%%%%%
\subsection{Revision History}

%%%%%%%%%%%%%%%%%%%%%%%%%%%%%%%%%%%%%%%%
\paragraph{v2.0:} 2018/12/30

\begin{itemize}
\item
immediate forward processing
\item
added |\childdocby| mechanism
\item
manual restructured
\end{itemize}

%%%%%%%%%%%%%%%%%%%%%%%%%%%%%%%%%%%%%%%%
\paragraph{v1.6:} 2018/01/17

\begin{itemize}
\item
application for development of include files
\item
corrections to manual
\end{itemize}

%%%%%%%%%%%%%%%%%%%%%%%%%%%%%%%%%%%%%%%%
\paragraph{v1.5:} 2017/05/21

\begin{itemize}
\item
more complete structuring introduced
\item
|\childdocof| introduced
\item
|\childdoc| renamed to |\childdocmain|
\item
|\childredirect| renamed to |\childdocforward| and |\childdocforwardprefix|
and functionality expanded
\end{itemize}

%%%%%%%%%%%%%%%%%%%%%%%%%%%%%%%%%%%%%%%%
\paragraph{v1.0:} 2017/04/27

\begin{itemize}
\item
manual and install package
\item
first version published on CTAN
\end{itemize}

%%%%%%%%%%%%%%%%%%%%%%%%%%%%%%%%%%%%%%%%
\paragraph{v0.6:} 2017/04/26

\begin{itemize}
\item
redirection mechanism added
\end{itemize}

%%%%%%%%%%%%%%%%%%%%%%%%%%%%%%%%%%%%%%%%
\paragraph{v0.5:} 2017/04/26

\begin{itemize}
\item
functionality in definition file
\end{itemize}


%%%%%%%%%%%%%%%%%%%%%%%%%%%%%%%%%%%%%%%%%%%%%%%%%%%%%%%%%%%%%%%%%%%%%%%%%%%%%%%%
%%%%%%%%%%%%%%%%%%%%%%%%%%%%%%%%%%%%%%%%%%%%%%%%%%%%%%%%%%%%%%%%%%%%%%%%%%%%%%%%
%%%%%%%%%%%%%%%%%%%%%%%%%%%%%%%%%%%%%%%%%%%%%%%%%%%%%%%%%%%%%%%%%%%%%%%%%%%%%%%%
\appendix

\settowidth\MacroIndent{\rmfamily\scriptsize 000\ }

 \DocInput{childdoc.dtx}

\end{document}
%</driver>
% \fi
%
% %%%%%%%%%%%%%%%%%%%%%%%%%%%%%%%%%%%%%%%%%%%%%%%%%%%%%%%%%%%%%%%%%%%%%%%%%%%%%%
% %%%%%%%%%%%%%%%%%%%%%%%%%%%%%%%%%%%%%%%%%%%%%%%%%%%%%%%%%%%%%%%%%%%%%%%%%%%%%%
% \section{Sample}
%\iffalse
%<*samplemain>
%\fi
%
% The following presents a sample document
% with two chapters, two parts, a title page,
% a compile flag as well as three forwarding files to set the flag.
% It consists of eight |.tex| files:
% \begin{center}
% \begin{tabular}{ll}
% |cdocsamp.tex|&main file\\
% |cdocsch1.tex|&include file for chapter 1\\
% |cdocsch2.tex|&include file for chapter 2\\
% |cdocspt3.tex|&include file for part 3\\
% |cdocspt4.tex|&include file for part 4\\
% |cdocsdrf.tex|&forwarding file for main file in draft mode\\
% |cdocsfi1.tex|&forwarding file for final version of chapter 1\\
% |cdocsfi2.tex|&forwarding file for final version of chapter 2\\
% \end{tabular}
% \end{center}
% Each of the eight files can be compiled directly by the \LaTeX{} compiler.
%
% %%%%%%%%%%%%%%%%%%%%%%%%%%%%%%%%%%%%%%
% \paragraph{Main File.}
%
% The main file is called |cdocsamp.tex|.
%
% Load the \textsf{childdoc} definitions and
% declare the filename for the main document:
%    \begin{macrocode}
\input{childdoc.def}
\childdocmain{}
%    \end{macrocode}

% Optional override for |\version| flag:
%    \begin{macrocode}
%%\ifchilddoc\else\providecommand{\version}{draft}\fi
%    \end{macrocode}

% Define the default values for the |\version| flag
% (|final| for the main file and |draft| for childs):
%    \begin{macrocode}
\ifchilddoc
\providecommand{\version}{draft}
\else
\providecommand{\version}{final}
\fi
%    \end{macrocode}

% Load the standard document class:
%    \begin{macrocode}
\documentclass[12pt]{article}
%    \end{macrocode}

% Start the document body:
%    \begin{macrocode}
\begin{document}
%    \end{macrocode}

% Declare a title page.
% Print title, part of document being processed and version flag:
%    \begin{macrocode}
\addtocounter{page}{-1}
\begin{center}
{\LARGE\bfseries{}childdoc example\par}
\vspace{1cm}
\ifchilddoc
\ifchilddocmanual part\else chapter\fi:
`\childdocname' of `\childdocjob'\par
\else
main document: `\childdocjob'\par
\fi
version: \version\par
\end{center}
\newpage
%    \end{macrocode}

% Manually include selected file,
% otherwise process as usual:
%    \begin{macrocode}
\ifchilddocmanual
\section*{part `\childdocname'}
\input{\childdocname}
\else
%    \end{macrocode}

% Include the two chapters:
%    \begin{macrocode}
\include{cdocsch1}
\include{cdocsch2}
%    \end{macrocode}

% Include the two parts unless only chapters should be displayed:
%    \begin{macrocode}
\ifchilddoc\else
\section{part three}
\input{cdocspt3}
\section{part four}
\input{cdocspt4}
\fi
%    \end{macrocode}

% Process as usual until here:
%    \begin{macrocode}
\fi
%    \end{macrocode}

% End of document body:
%    \begin{macrocode}
\end{document}
%    \end{macrocode}
%\iffalse
%</samplemain>
%\fi
%
% %%%%%%%%%%%%%%%%%%%%%%%%%%%%%%%%%%%%%%
% \paragraph{Chapter Include Files.}
%
% The include files are called |cdocsch1.tex| and |cdocsch2.tex|.
%
%\iffalse
%<*samplechap1|samplechap2>
%\fi

% Optional override for |\version| flag:
%    \begin{macrocode}
%%\providecommand{\version}{final}
%    \end{macrocode}

% Include the main document:
%    \begin{macrocode}
\input{childdoc.def}
\childdocof{cdocsamp}
%    \end{macrocode}

%\iffalse
%</samplechap1|samplechap2>
%\fi
%
%\iffalse
%<*samplechap1>
%\fi
% Some text for chapter 1:
%    \begin{macrocode}
\section{one}
some text in chapter one
%    \end{macrocode}

%\iffalse
%</samplechap1>
%\fi
% Some text for chapter 2:
%\iffalse
%<*samplechap2>
%\fi
%    \begin{macrocode}
\section{two}
more text in chapter two
%    \end{macrocode}

%\iffalse
%</samplechap2>
%\fi
%
% %%%%%%%%%%%%%%%%%%%%%%%%%%%%%%%%%%%%%%
% \paragraph{Part Include Files.}
%
% The include files are called |cdocspt3.tex| and |cdocspt4.tex|.
%
%\iffalse
%<*samplepart3|samplepart4>
%\fi

% Optional override for |\version| flag:
%    \begin{macrocode}
%%\providecommand{\version}{final}
%    \end{macrocode}

% Include the main document:
%    \begin{macrocode}
\input{childdoc.def}
\childdocby{cdocsamp}
%    \end{macrocode}

%\iffalse
%</samplepart3|samplepart4>
%\fi
%
%\iffalse
%<*samplepart3>
%\fi
% Some text for part 3:
%    \begin{macrocode}
some text in part three
%    \end{macrocode}

%\iffalse
%</samplepart3>
%\fi
% Some text for part 4:
%\iffalse
%<*samplepart4>
%\fi
%    \begin{macrocode}
more text in part four
%    \end{macrocode}

%\iffalse
%</samplepart4>
%\fi
%
% %%%%%%%%%%%%%%%%%%%%%%%%%%%%%%%%%%%%%%
% \paragraph{Forwarding for a Complete Draft.}
%
% The following forwarding file |cdocsdrf.tex|
% compiles the main document in draft mode:
%\iffalse
%<*sampledraft>
%\fi
%    \begin{macrocode}
\def\version{draft}
\input{childdoc.def}
\childdocforward{cdocsamp}
%    \end{macrocode}

%\iffalse
%</sampledraft>
%\fi
%
% %%%%%%%%%%%%%%%%%%%%%%%%%%%%%%%%%%%%%%
% \paragraph{Forwarding for Final Version of the Chapters.}
%
% The following forwarding files |cdocsfn1.tex| and |cdocsfn2.tex|
% (with identical content)
% compile the final versions of the child documents
% |cdocsch1.tex| and |cdocsch2.tex|, respectively:
%\iffalse
%<*samplefinal>
%\fi
%    \begin{macrocode}
\def\version{final}
\input{childdoc.def}
\childdocforwardprefix[cdocsamp]{cdocsfn}{cdocsch}
%    \end{macrocode}

%\iffalse
%</samplefinal>
%\fi
%
% %%%%%%%%%%%%%%%%%%%%%%%%%%%%%%%%%%%%%%
% \paragraph{Command Line Processing.}
%
% The following three command lines generate the output files
% |cdocscld|, |cdocscl1| and |cdocscl2|
% which should be identical to
% |cdocsdrf|, |cdocsch1| and |cdocsfn2|, respectively:
% \begin{center}
% \begin{tabular}{l}
% |latex -jobname cdocscld \|\\
% |  "\def\version{draft}\input{childdoc.def}\childdocforward{cdocsamp}"|\\
% |latex -jobname cdocscl1 \|\\
% |  "\input{childdoc.def}\childdocforward[cdocsamp]{cdocsch1}"|\\
% |latex -jobname cdocscl2 \|\\
% |  "\def\version{final}\input{childdoc.def}\childdocforward{cdocsch2}"|
% \end{tabular}
% \end{center}
% Note that the trailing backslash on each first line
% merely continues the input to the second line
% (for convenient cut ant paste).
% Furthermore, the command |latex| can be replaced by any
% of its alternative versions such as |pdflatex|.
%
% %%%%%%%%%%%%%%%%%%%%%%%%%%%%%%%%%%%%%%%%%%%%%%%%%%%%%%%%%%%%%%%%%%%%%%%%%%%%%%
% %%%%%%%%%%%%%%%%%%%%%%%%%%%%%%%%%%%%%%%%%%%%%%%%%%%%%%%%%%%%%%%%%%%%%%%%%%%%%%
% \section{Implementation}
%\iffalse
%<*package>
%\fi
%
% This section describes the definitions file |childdoc.def|.

% The definitions cannot be loaded using |\usepackage| or |\RequirePackage|
% which has a mechanism to prevent loading a style file more than once.
% When loading the definitions by means of |\input|
% multiple instances have to be prevented manually:
%\iffalse
%This code needs to be before the `\ProvidesFile' directive
%which is defined at the beginning of this file.
%Therefore it is also placed there and commented out here.
%</package>
%<*discard>
%\fi
%    \begin{macrocode}
\ifdefined\childdocmain\endinput\fi
%    \end{macrocode}
%\iffalse
%</discard>
%<*package>
%\fi
%
% \macro{\ifchilddoc}
% \macro{\ifchilddocmanual}
% The conditional |\ifchilddoc| tells whether a
% child (true) or main (false) document is being compiled.
% The conditional |\ifchilddocmanual| tells whether
% the |\includeonly| mechanism is used (false) or
% the selection of child files must be performed manually (true).
% The definitions initialise to false:
%    \begin{macrocode}
\newif\ifchilddoc
\newif\ifchilddocmanual
%    \end{macrocode}

% \macro{\childdocname}
% \macro{\childdocjob}
% The macro |\childdocname| stores the name of the main document
% to be compiled. The macro |\childdocjob| stores the name of
% the document on which the \LaTeX{} compiler was originally invoked.
% The content of |\jobname| cannot be compared
% to filenames specified in the source due to different catcodes.
% The following code rescans |\jobname|, stores the result
% in |\childdocname| and saves a copy in |\childdocjob|:
%    \begin{macrocode}
\edef\childdocname{\scantokens\expandafter{\jobname\noexpand}}
\let\childdocjob\childdocname
%    \end{macrocode}

% \macro{\childdocdisable}
% The macro |\childdocdisable| prevents the main file
% from being processed more than once.
% At this stage, the main document command |\childdocmain|
% is assumed to be called once again where it should do nothing.
% Any subsequent call to it should prevent
% a secondary processing of the main document
% It overwrites the forwarding commands
% |\childdocof| and |\childdocforward|
% with empty macros to prevent further inclusions of the main document:
%    \begin{macrocode}
\newcommand{\childdocdisable}
{
  \renewcommand{\childdocmain}[1]{\renewcommand{\childdocmain}[1]{\endinput}}
  \renewcommand{\childdocof}[1]{}
  \renewcommand{\childdocby}[2][]{}
  \renewcommand{\childdocforward}[2][]{}
  \renewcommand{\childdocdisable}{}
}
%    \end{macrocode}

% \macro{\childdocmain}
% The macro |\childdocmain| is to be called at the top of the main file
% with nothing or the main filename (without extension) as argument.
% First, it breaks loops.
% If the argument is not empty and does not match |\childdocname|
% (which is set by the first inclusion of |childdoc.def|),
% |\ifchilddoc| is set to true, |\includeonly| is applied to the child file
% and |\jobname| is set to the main file
% (for proper handling of |.aux| files):
%    \begin{macrocode}
\newcommand{\childdocmain}[1]
{
  \childdocdisable\childdocmain{}
  \if?#1?\else
    \begingroup
      \def\childdoctmp{#1}
      \ifx\childdoctmp\childdocname
        \def\childdoctmp{}
      \else
        \def\childdoctmp
        {
          \childdoctrue
          \includeonly{\childdocname}
          \def\childdocjob{#1}
          \def\jobname{#1}
        }
      \fi
      \expandafter
    \endgroup
    \childdoctmp
  \fi
}
%    \end{macrocode}

% \macro{\childdocof}
% The command |\childdocof| redirects
% compilation to the main file |#1|.
%    \begin{macrocode}
\newcommand{\childdocof}[1]
{
  \childdocdisable
  \childdoctrue
  \includeonly{\childdocname}
  \def\jobname{#1}
  \def\childdocjob{#1}
  \input{#1}
}
%    \end{macrocode}

% \macro{\childdocby}
% The command |\childdocby| ....
%    \begin{macrocode}
\newcommand{\childdocby}[2][]
{
  \childdocdisable
  \childdoctrue
  \childdocmanualtrue
  \if?#1?\else
    \def\jobname{#2}
  \fi
  \def\childdocjob{#2}
  \input{#2}
  \endinput
}
%    \end{macrocode}

% \macro{\childdocforward}
% The command |\childdocforward| redirects
% compilation to the main file or
% (if the optional argument is given) a child file.
% Parameters are set as if the main file
% or a child file starting with |\childdocof| was compiled.
% Then compilation is handed over to the main file:
%    \begin{macrocode}
\newcommand{\childdocforward}[2][]
{
  \begingroup
    \if?#1?
      \def\childdoctmp
      {
        \def\childdocname{#2}
        \def\childdocjob{#2}
        \def\jobname{#2}
        \input{#2}
        \endinput
      }
    \else
      \def\childdoctmp
      {
        \childdocdisable
        \def\childdocname{#2}
        \childdoctrue
        \includeonly{#2}
        \def\childdocjob{#1}
        \def\jobname{#1}
        \input{#1}
        \endinput
      }
    \fi
    \expandafter
  \endgroup
  \childdoctmp
}
%    \end{macrocode}

% \macro{\childdocforwardprefix}
% The command |\childdocforwardprefix| redirects
% compilation to the main or a child file by means of a pattern.
% The prefix |#1| in the current filename is replaced by |#2|
% and the suffix of the current filename is kept
% (it is assumed that the filename does not contain the substring `|~~~|'
% which is used as a delimiter).
% Compilation is handed over to the new file by |\childdocforward|:
%    \begin{macrocode}
\newcommand{\childdocforwardprefix}[3][]
{
  \begingroup
    \def\childdocextract #2##1~~~{\def\childdoctmp{\childdocforward[#1]{#3##1}}}
    \expandafter\childdocextract\childdocname~~~
    \expandafter
  \endgroup
  \childdoctmp
}
%    \end{macrocode}

% \macro{\childdoc}
% The deprecated macro |\childdoc| is a legacy version of |\childdocmain|:
%    \begin{macrocode}
\newcommand{\childdoc}{\childdocmain}
%    \end{macrocode}

% \macro{\childdocredirect}
% The deprecated macro |\childdocredirect| is a legacy version
% of |\childdocforward| and |\childdocforwardprefix|:
%    \begin{macrocode}
\newcommand{\childdocredirect}[2][]
{
  \begingroup
    \if?#1?
      \def\childdoctmp{\childdocforward{#2}}
    \else
      \def\childdoctmp{\childdocforwardprefix{#1}{#2}}
    \fi
    \expandafter
  \endgroup
  \childdoctmp
}
%    \end{macrocode}

%\iffalse
%</package>
%\fi
%
\endinput
\childdocforward{cdocsamp}"|\\
% |latex -jobname cdocscl1 \|\\
% |  "% \iffalse
%
% childdoc.dtx Copyright (C) 2017-2018 Niklas Beisert
%
% This work may be distributed and/or modified under the
% conditions of the LaTeX Project Public License, either version 1.3
% of this license or (at your option) any later version.
% The latest version of this license is in
%   http://www.latex-project.org/lppl.txt
% and version 1.3 or later is part of all distributions of LaTeX
% version 2005/12/01 or later.
%
% This work has the LPPL maintenance status `maintained'.
%
% The Current Maintainer of this work is Niklas Beisert.
%
% This work consists of the files childdoc.dtx and childdoc.ins
% and the derived files childdoc.def and cdocsamp.tex with
% cdocsch1.tex, cdocsch2.tex, cdocsdrf.tex, cdocsfn1.tex, cdocsfn2.tex.
%
%<package>\ifdefined\childdocmain\endinput\fi
%<package>\ProvidesFile{childdoc.def}[2018/12/30 v2.0 child document driver]
%<samplemain>\ProvidesFile{cdocsamp.tex}[2018/12/30 v2.0 sample for childdoc]
%<*driver>
%\ProvidesFile{childdoc.drv}[2018/12/30 v2.0 childdoc reference manual file]
\PassOptionsToClass{10pt,a4paper}{article}
\documentclass{ltxdoc}

\usepackage[margin=35mm]{geometry}
\usepackage{hyperref}
\usepackage{hyperxmp}
\usepackage[usenames]{color}

\hypersetup{colorlinks=true}
\hypersetup{pdfstartview=FitH}
\hypersetup{pdfpagemode=UseNone}
\hypersetup{pdfsource={}}
\hypersetup{pdflang={en-UK}}
\hypersetup{pdfcopyright={Copyright 2017-2018 Niklas Beisert.
  This work may be distributed and/or modified under the
  conditions of the LaTeX Project Public License, either version 1.3
  of this license or (at your option) any later version.}}
\hypersetup{pdflicenseurl={http://www.latex-project.org/lppl.txt}}
\hypersetup{pdfcontactaddress={ETH Zurich, ITP, HIT K,
  Wolfgang-Pauli-Strasse 27}}
\hypersetup{pdfcontactpostcode={8093}}
\hypersetup{pdfcontactcity={Zurich}}
\hypersetup{pdfcontactcountry={Switzerland}}
\hypersetup{pdfcontactemail={nbeisert@itp.phys.ethz.ch}}
\hypersetup{pdfcontacturl={http://people.phys.ethz.ch/\xmptilde nbeisert/}}

\newcommand{\secref}[1]{\hyperref[#1]{section \ref*{#1}}}

\parskip1ex
\parindent0pt
\let\olditemize\itemize
\def\itemize{\olditemize\parskip0pt}

\begin{document}

\title{The \textsf{childdoc} Package}
\hypersetup{pdftitle={The childdoc Package}}
\author{Niklas Beisert\\[2ex]
  Institut f\"ur Theoretische Physik\\
  Eidgen\"ossische Technische Hochschule Z\"urich\\
  Wolfgang-Pauli-Strasse 27, 8093 Z\"urich, Switzerland\\[1ex]
  \href{mailto:nbeisert@itp.phys.ethz.ch}
  {\texttt{nbeisert@itp.phys.ethz.ch}}}
\hypersetup{pdfauthor={Niklas Beisert}}
\hypersetup{pdfsubject={Manual for the LaTeX2e Package childdoc}}
\date{30 December 2018, \textsf{v2.0}}
\maketitle

\begin{abstract}\noindent
\textsf{childdoc} is a \LaTeXe{} package
that enables the direct compilation
of document sections included by |\include|
to individual files.
\end{abstract}

\begingroup
\parskip0ex
\tableofcontents
\endgroup

%%%%%%%%%%%%%%%%%%%%%%%%%%%%%%%%%%%%%%%%%%%%%%%%%%%%%%%%%%%%%%%%%%%%%%%%%%%%%%%%
%%%%%%%%%%%%%%%%%%%%%%%%%%%%%%%%%%%%%%%%%%%%%%%%%%%%%%%%%%%%%%%%%%%%%%%%%%%%%%%%
\section{Introduction}

\LaTeX{} provides a mechanism to structure a large document (such as a book)
into a main file and several child files (containing the chapters)
using the |\include| command.
This mechanism is beneficial for documents
which span hundreds of pages in order to
make the source file(s) more manageable.
Moreover, compilation can be restricted to
selected child files by means of the |\includeonly| command.
The latter feature can be used to reduce the compilation time while editing
(this was significantly more useful in the earlier days of \LaTeX{})
or to generate a smaller document which is easier to navigate.
Another application of |\includeonly| is to generate
documents consisting of selected parts of the complete document.

However, there are a few drawbacks of the plain |\include| mechanism:
\begin{itemize}
\item
The child files cannot be compiled on their own,
they can only be compiled via the main file.
A naive editing environment
(such as a text editor with an option
to have the current file processed by \LaTeX)
may require one to switch to the main file before compiling;
attempting to compile the child file produces errors.
\item
The main file must be modified (each time)
to adjust the |\includeonly| command
to the present needs. This easily leaves the main file in a messy state.
\item
The generated document will always carry the filename
of the main document. This is inconvenient if
several child files are to be compiled and
to be kept for distribution.
\end{itemize}

The present package provides a simple interface
to make child files individually compilable by \LaTeX{}.
Compiling a child file then has the same effect as compiling
the main file with an |\includeonly| command
to select the appropriate child.
Moreover the generated document will carry the name of the child
rather than the main file.
This resolves all three above issues.

This feature is meant to make the editing of books,
thesis documents and lecture notes somewhat more convenient.
However, the package can also be used efficiently for
composing a series of documents (such as exercise sheets)
which are typically distributed individually.
It then assists the author in generating the individual documents
(potentially in different versions)
as well as a document containing the collected series.
Another application is in developing style files
or other kinds of included material
where compilation of the style file could redirect
to a sample or test file.

%%%%%%%%%%%%%%%%%%%%%%%%%%%%%%%%%%%%%%%%%%%%%%%%%%%%%%%%%%%%%%%%%%%%%%%%%%%%%%%%
%%%%%%%%%%%%%%%%%%%%%%%%%%%%%%%%%%%%%%%%%%%%%%%%%%%%%%%%%%%%%%%%%%%%%%%%%%%%%%%%
\section{Usage}

First of all, the package \textsf{childdoc} is \emph{not} a standard
\LaTeXe{} |.sty| style file! Therefore it needs to be invoked in
a non-standard way.

%%%%%%%%%%%%%%%%%%%%%%%%%%%%%%%%%%%%%%%%%%%%%%%%%%%%%%%%%%%%%%%%%%%%%%%%%%%%%%%%
\subsection{Included Files}
\label{sec:include}

%%%%%%%%%%%%%%%%%%%%%%%%%%%%%%%%%%%%%%%%
\DescribeMacro{\childdocmain}
To use the package, add the commands
\begin{center}
\begin{tabular}{l}
|\input{childdoc.def}|\\
|\childdocmain{}|\\
\end{tabular}
\end{center}
at the very top of the main \LaTeX{} file,
in particular \emph{before} the |\documentclass| statement!
The argument of |\childdocmain| should be left empty
(but it must be present).

%%%%%%%%%%%%%%%%%%%%%%%%%%%%%%%%%%%%%%%%
\DescribeMacro{\childdocof}
Furthermore, add the commands
\begin{center}
\begin{tabular}{l}
|\input{childdoc.def}|\\
|\childdocof{|\textit{main}|}|\\
\end{tabular}
\end{center}
at the top of every child file \textit{child}
which is included by |\include{|\textit{child}|}|
from within the main file
(or at least for those files to be compiled individually).
The argument \textit{main} must be the filename of the main file.

There are a couple of
considerations in setting up the main and child documents:

%%%%%%%%%%%%%%%%%%%%%%%%%%%%%%%%%%%%%%%%
\paragraph{Restrictions.}

Please note the following restrictions:
\begin{itemize}
\item
|\childdocmain| must be called with one argument \textit{main}
to ensure compatibility with earlier version of the package.
It must either be empty (|\childdocmain{}|)
or precisely match the filename of the main file in which it is specified.
See \secref{sec:detection} for further information.
\item
The filename \textit{main} must be specified without the |.tex| extension.
\item
The filename \textit{main} is case sensitive
(even in case-insensitive file systems)
due to internal string comparison.
\item
The argument \textit{main} should be fully expanded, it cannot be a macro.
\item
Subdirectories and special characters should be avoided in filenames.
\item
The command |\childdocmain{|\textit{main}|}| must be followed by a whitespace.
It should not be followed immediately by another command
or by a comment mark `|%|'.
This is because the \TeX{} parser reads the token immediately following
the argument of |\childdocmain| and puts it
at the beginning of every child section;
however, a white\-space is ignored.
\end{itemize}

%%%%%%%%%%%%%%%%%%%%%%%%%%%%%%%%%%%%%%%%
\paragraph{Content of Main File.}

It is advisable to place all content in the child files included by |\include|.
Any output contained in the main file will appear in all child documents
unless suppressed manually;
it cannot be suppressed automatically by the |\includeonly| directive
and thus should normally be avoided.
A method to include some content in the main file
by means of conditional processing is described in \secref{sec:conditional}.

%%%%%%%%%%%%%%%%%%%%%%%%%%%%%%%%%%%%%%%%
\paragraph{Page Numbering.}

When only a part of the document is compiled,
the appropriate numbering of pages
(as well as other status parameters)
is determined from the |.aux| files.
The latter contain information from previous passes.
However this information needs to propagate through
all intermediate child documents.
Therefore the page numbering in child documents may well
be inconsistent until the complete document is compiled at least once.

A useful (if unconventional) way to always ensure a consistent
page numbering is to restart the numbering in each child document
and denote the pages by `\textit{child}|.|\textit{page}'
where \textit{child} represents the chapter/section number of the child file.
This can be achieved by the command
|\numberwithin{page}{|\textit{child}|}|
of the \textsf{amsmath} package
where \textit{child} can be |chapter| or |section|
depending on the chosen structuring.
Alternatively, one can modify the macro |\thepage| appropriately
and reset the counter |page| at the start of each child file.

%%%%%%%%%%%%%%%%%%%%%%%%%%%%%%%%%%%%%%%%%%%%%%%%%%%%%%%%%%%%%%%%%%%%%%%%%%%%%%%%
\subsection{Conditional Processing}
\label{sec:conditional}

The package provides a mechanism to compile different versions
of a document. To customise the versions further some conditional processing
can come in handy to distinguish which version is being compiled.
The package provides two macros to describe the compilation context:

%%%%%%%%%%%%%%%%%%%%%%%%%%%%%%%%%%%%%%%%
\DescribeMacro{\ifchilddoc}
The conditional |\ifchilddoc| distinguishes between the compilation of
child documents and the main document:
%
\begin{center}
|\ifchilddoc |\textit{child-code}| |[|\||else |\textit{main-code}]| \||fi|
\end{center}

%%%%%%%%%%%%%%%%%%%%%%%%%%%%%%%%%%%%%%%%
\DescribeMacro{\childdocname}
\DescribeMacro{\childdocjob}
The macro |\childdocname| contains the filename (without extension)
of the main or child file being processed.
Note that |\childdocjob| will always contain the name of the main file.

%%%%%%%%%%%%%%%%%%%%%%%%%%%%%%%%%%%%%%%%
\paragraph{Title Page.}

Conditional processing can be used to include a title or banner page
in the main document when proper precautions are taken.
Importantly, the code in the main file should ensure that the page counter
(as well as other status parameters which are stored in the |.aux| files)
takes the same value after the conditional processing.
Otherwise the page numbers may take divergent values
depending on which part is compiled.

For example, a title page could be declared by:
%
\begin{center}
\begin{tabular}{l}
|\ifchilddoc\||else|\\
|\addtocounter{page}{-1}|\\
\textit{code for title page}\\
|\newpage|\\
|\||fi|
\end{tabular}
\end{center}
%
A banner page for the child documents can be generated by:
%
\begin{center}
\begin{tabular}{l}
|\ifchilddoc|\\
|\addtocounter{page}{-1}|\\
\textit{code for banner page}\\
|\newpage|\\
|\||fi|
\end{tabular}
\end{center}
%
Here one could write a message such as:
\begin{center}
|This is the part \childdocname{} of \childdocjob{}.|
\end{center}

%%%%%%%%%%%%%%%%%%%%%%%%%%%%%%%%%%%%%%%%%%%%%%%%%%%%%%%%%%%%%%%%%%%%%%%%%%%%%%%%
\subsection{Flags}
\label{sec:flags}

The package makes it easy to generate different versions
of the main or child documents.
To this end compilation flags can be defined
and assigned different default values.
They will be particularly useful in conjunction
with the forwarding mechanism described in \secref{sec:forward}.

For example, it may be useful to have a flag |\version|
which can be set to |draft| or |final|.
The document source will contain some conditional code
depending on the value of |\version|.
Suppose further, the flag should default to |final| for the main file
and to |draft| for child files
which is a natural assignment for editing the document.
This is achieved by placing the following code
in the preamble of the main document
(below the |\childdocmain| directive):
%
\begin{center}
\begin{tabular}{l}
|\ifchilddoc|\\
|\providecommand{\version}{draft}|\\
|\||else|\\
|\providecommand{\version}{final}|\\
|\||fi|
\end{tabular}
\end{center}
%
The definition by |\providecommand| makes sure
that previous definitions are not overwritten.
Further statements |\providecommand{\version}{...}|
can thus be added before the above code to override it.

For the main file, one might add a line
(between |\childdocmain| and the above block)
%
\begin{center}
|%\ifchilddoc\||else\providecommand{\version}{draft}\||fi|
\end{center}
%
which can be uncommented to produce a draft version.
Likewise one can add a line to the very top of a child file
(above the |\childdocof{|\textit{main}|}| directive)
%
\begin{center}
|%\providecommand{\version}{final}|
\end{center}
%
which can be uncommented to produce the final version of this child document.

%%%%%%%%%%%%%%%%%%%%%%%%%%%%%%%%%%%%%%%%%%%%%%%%%%%%%%%%%%%%%%%%%%%%%%%%%%%%%%%%
\subsection{Forwarding}
\label{sec:forward}

Different versions of the main or child documents
using compilation flags as described in \secref{sec:flags}
can be (permanently) stored in different files
for convenient compilation, viewing and distribution.
To this end, the package defines a command
to pass on compilation to a different file:

%%%%%%%%%%%%%%%%%%%%%%%%%%%%%%%%%%%%%%%%
\DescribeMacro{\childdocforward}
The command |\childdocforward| redirects processing to
another source file:
%
\begin{center}
\begin{tabular}{l}
|\input{childdoc.def}|\\
|\childdocforward[|\textit{main}|]{|\textit{dest}|}|\\
\end{tabular}
\end{center}
%
The argument \textit{dest} is the destination file
(without extension).
It should be the main file or one of the child files.
Note that further \textsf{childdoc} directives
such as |\childdocof| and |\childdocforward|
in the indicated file will be processed in this form.
The optional argument \textit{main}
passes on directly to the main file \textit{main}
while pretending to compile the child \textit{dest}.
This form behaves as if \textit{dest}
issues |\childdocof{|\textit{main}|}| right away,
and no further \textsf{childdoc} directives will be processed.

%%%%%%%%%%%%%%%%%%%%%%%%%%%%%%%%%%%%%%%%
\DescribeMacro{\...prefix}
In the alternative form |\childdocforwardprefix|,
%
\begin{center}
\begin{tabular}{l}
|\input{childdoc.def}|\\
|\childdocforwardprefix[|\textit{main}|]{|\textit{prefix}|}{|\textit{dest}|}|
\end{tabular}
\end{center}
%
the destination file is determined by a pattern
depending on the current file:
To make this work, the current file must be called
`{\textit{prefix}\hspace{0.2em}\textit{suffix}}'
with \textit{prefix} matching precisely the argument.
Processing is then passed on to the file
`{\textit{dest}\hspace{0.2em}\textit{suffix}}'.
Surely, the same effect is achieved by
directly specifying the
argument `{\textit{dest}\hspace{0.2em}\textit{suffix}}'
in the first form.
However, that requires to set up a different file
for each child. With the alternative form of the command
all these files can have exactly the same content
which simplifies setting them up and maintaining them.

For example, the following file |draft.tex|
with a compilation flag |\version| as described in \secref{sec:flags}
compiles the main document as a draft:
%
\begin{center}
\begin{tabular}{l}
|\def\version{draft}|\\
|\input{childdoc.def}|\\
|\childdocforward{|\textit{main}|}|
\end{tabular}
\end{center}
%
Likewise, the following files |final|\textit{nn}|.tex|
compile the final version of the child document
|child|\textit{nn}|.tex|:
%
\begin{center}
\begin{tabular}{l}
|\def\version{final}|\\
|\input{childdoc.def}|\\
|\childdocforwardprefix{final}{child}|
\end{tabular}
\end{center}
%

Note that when several versions of a main file and/or of each child file
are to be generated, it may be convenient to set up a |Makefile| or
shell script to automatise the process.

%%%%%%%%%%%%%%%%%%%%%%%%%%%%%%%%%%%%%%%%%%%%%%%%%%%%%%%%%%%%%%%%%%%%%%%%%%%%%%%%
\subsection{Command Line Processing}
\label{sec:commandline}

The effect of redirection files can also be achieved by invoking
the \LaTeX{} compiler with a more elaborate command line.
Most conveniently this should be done as part
of a shell script or a |Makefile|.

When using \textsf{childdoc} in the main file, the following
command lines effectively perform a redirection
(note that depending on the shell being used,
backslashes may have to be doubled: `|\|' $\to$ `|\\|'):
%
\begin{center}
|... -jobname "|\textit{target}|" |\\|"|[\textit{flags}]%
|\input{childdoc.def}\childdocforward[|\textit{main}|]{|\textit{dest}|}"|
\end{center}
%
Here \textit{target} is the name of the output file,
\textit{main} is the name of the main file
and \textit{dest} is the name of the main or child file to be processed
(all filenames without extensions).
The optional argument \textit{main} can be omitted
if \textit{main} matches \textit{dest}.
Optionally, compilation \textit{flags} can be defined via |\def| commands.
This command line makes the \TeX{} engine believe
it is compiling the file \textit{target}
whose content is specified as the latter parameter.
The provided code then forwards the processing to
\textit{main} or \textit{dest} as described in \secref{sec:forward}.

%%%%%%%%%%%%%%%%%%%%%%%%%%%%%%%%%%%%%%%%%%%%%%%%%%%%%%%%%%%%%%%%%%%%%%%%%%%%%%%%
\subsection{Include by Input}
\label{sec:input}

Including child documents by |\include| has some restrictions by design.
Most notably, the content of a child document always occupies
its own set of pages; pages cannot be shared between child documents.
Usually, this behaviour makes perfect sense
because each child document contain an essential part of the document.
However, in some situations it may be desirable to compose
a document from a collection of parts
without having mandatory page breaks between then.
For this case, the package
provides a mechanism to include parts
by |\input| which can also be processed individually.
However, by construction this mechanism
requires manual handling of the content to be output.

%%%%%%%%%%%%%%%%%%%%%%%%%%%%%%%%%%%%%%%%
\DescribeMacro{\ifchilddocmanual}
The main file should be prepared as usual, see \secref{sec:include}.
However, the document body must make a distinction
between processing of an individual part and of the main document, e.g.:
%
\begin{center}
\begin{tabular}{l}
|\ifchilddocmanual|\\
|\input{\childdocname}|\\
|\||else|\\
\textit{document body with }|\input{|\textit{part}|}|\\
|\||fi|
\end{tabular}
\end{center}
%
The conditional |\ifchilddocmanual| is true whenever
a part to be included by |\input| is being compiled,
and the name of the part is stored in |\childdocname|.

%%%%%%%%%%%%%%%%%%%%%%%%%%%%%%%%%%%%%%%%
\DescribeMacro{\childdocby}
Each part to be included by |\input| should start with:
%
\begin{center}
\begin{tabular}{l}
|\input{childdoc.def}|\\
|\childdocby{|\textit{main}|}|\\
\end{tabular}
\end{center}
%
The directive |\childdocby| is similar to |\childdocof|
described in \secref{sec:include},
but the subsequent selection of content must be done manually.
To that end, both |\ifchilddoc| and |\ifchilddocmanual|
will be true upon processing of a part,
and the name of the part is stored in |\childdocname|.
Note that |\jobname| will be set to the filename of the current part
so that each part receives an individual |.aux| file
that does not interfere with the |.aux| file(s) of the main document.
This behaviour can be altered by the alternative form
|\childdocby[*]{|\textit{main}|}| (with a non-empty optional argument)
which uses the |.aux| file of the main document
by setting |\jobname| to \textit{main}.

%%%%%%%%%%%%%%%%%%%%%%%%%%%%%%%%%%%%%%%%%%%%%%%%%%%%%%%%%%%%%%%%%%%%%%%%%%%%%%%%
\subsection{Driver Development}
\label{sec:driver}

The \textsf{childdoc} mechanism can also be use for the development
of definition files such as \LaTeX{} styles or classes.
This case differs from the above setup with multiple parts
included by |\include| in that no |\includeonly| should be invoked.
This can be achieved by starting the include file
(before |\ProvidesPackage|) with:
%
\begin{center}
\begin{tabular}{l}
|\input{childdoc.def}|\\
|\childdocforward{|\textit{main}|}|\\
\end{tabular}
\end{center}
%
or alternatively with:
%
\begin{center}
\begin{tabular}{l}
|\input{childdoc.def}|\\
|\childdocby{|\textit{main}|}|\\
\end{tabular}
\end{center}
%
Both forms have slightly different effects as described above.
The main file is prepared as usual, see \secref{sec:include}.

%%%%%%%%%%%%%%%%%%%%%%%%%%%%%%%%%%%%%%%%%%%%%%%%%%%%%%%%%%%%%%%%%%%%%%%%%%%%%%%%
\subsection{Legacy Detection}
\label{sec:detection}

The directive |\childdocmain| in the main file can detect
whether the complete document or merely a child is to be compiled
even without using the directive |\childdocof|.
This method is deprecated because it is less robust
and there is no compelling reason to use it;
it is merely provided for backward compatibility
and it may be removed in future versions.

If the detection mechanism is to be used,
it is mandatory to correctly specify
the filename of the main file as the argument of |\childdocmain|:
%
\begin{center}
\begin{tabular}{l}
|\input{childdoc.def}|\\
|\childdocmain{|\textit{main}|}|\\
\end{tabular}
\end{center}
%
If |\jobname| does not match the argument \textit{main} of |\childdocmain|,
it is assumed that |\jobname| points to the child file to be compiled.
When using |\childdocmain| with the main file specified as argument,
it suffices to start a child file
with just |\input{|\textit{main}|}|
without loading of the package and using |\childdocof|.
If instead all processing is done
with the appropriate \textsf{childdoc} directives,
the argument of \textit{main} of |\childdocmain| can be empty.

An alternative version of the command line processing described
in \secref{sec:commandline} using the detection mechanism reads:
%
\begin{center}
|... -jobname "|\textit{target}|" "|[\textit{flags}]%
[|\def\jobname{|\textit{dest}|}|]|\input{|\textit{main}|}"|
\end{center}

%%%%%%%%%%%%%%%%%%%%%%%%%%%%%%%%%%%%%%%%%%%%%%%%%%%%%%%%%%%%%%%%%%%%%%%%%%%%%%%%
\subsection{Manual Code}
\label{sec:manual}

In case one cannot be certain whether the definitions file |childdoc.def|
is installed on the target \TeX{} distribution
and one prefers not to ship it,
it is conceivable to paste a few relevant commands into the sources.

To that end, drop all statements |\input{childdoc.def}|
and perform the replacements as outlined below.
Instead of |\childdocmain{|\textit{main}|}| add the following code
to the top of the main file:
%
\begin{center}
\begin{tabular}{l}
|\||ifdefined\childdocname\endinput\||fi\newif\ifchilddoc|\\
|\edef\childdocname{\scantokens\expandafter{\jobname\noexpand}}|\\
|\def\childdocmain{|\textit{main}|}\||ifx\childdocmain\childdocname\||else|\\
|\childdoctrue\includeonly{\childdocname}\let\jobname\childdocmain\||fi|\\
\end{tabular}
\end{center}
%
Instead of |\childdocof{|\textit{main}|}| just include the main file
at the top of each child file:
%
\begin{center}
|\input{|\textit{main}|}|
\end{center}
%
A simple redirection |\childdocforward{|\textit{dest}|}| is achieved by:
%
\begin{center}
|\def\jobname{|\textit{dest}|}\input{\jobname}|
\end{center}
%
The redirection with prefix
|\childdocforwardprefix[|\textit{prefix}|]{|\textit{dest}|}|
is accomplished by:
%
\begin{center}
\begin{tabular}{l}
|{\edef\jobname{\scantokens\expandafter{\jobname\noexpand}}|\\
|\def\redirectjob |\textit{prefix}|#1~~~{\gdef\jobname{|\textit{dest}|#1}}|\\
|\expandafter\redirectjob\jobname~~~}\input{\jobname}|
\end{tabular}
\end{center}

In an alternative approach,
child documents can be compiled by a specific command line
without additional code or specific definitions:
%
\begin{center}
|... -jobname "|\textit{target}|" "|[\textit{flags}]%
|\includeonly{|\textit{dest}|}\input{|\textit{main}|}"|
\end{center}
%

%%%%%%%%%%%%%%%%%%%%%%%%%%%%%%%%%%%%%%%%%%%%%%%%%%%%%%%%%%%%%%%%%%%%%%%%%%%%%%%%
%%%%%%%%%%%%%%%%%%%%%%%%%%%%%%%%%%%%%%%%%%%%%%%%%%%%%%%%%%%%%%%%%%%%%%%%%%%%%%%%
\section{Information}

%%%%%%%%%%%%%%%%%%%%%%%%%%%%%%%%%%%%%%%%%%%%%%%%%%%%%%%%%%%%%%%%%%%%%%%%%%%%%%%%
\subsection{Copyright}

Copyright \copyright{} 2017--2018 Niklas Beisert

This work may be distributed and/or modified under the
conditions of the \LaTeX{} Project Public License, either version 1.3
of this license or (at your option) any later version.
The latest version of this license is in
  \url{http://www.latex-project.org/lppl.txt}
and version 1.3 or later is part of all distributions of \LaTeX{}
version 2005/12/01 or later.

This work has the LPPL maintenance status `maintained'.

The Current Maintainer of this work is Niklas Beisert.

This work consists of the files |README.txt|, |childdoc.ins| and |childdoc.dtx|
as well as the derived files |childdoc.def|, |cdocsamp.tex|
with |cdocsch1.tex|, |cdocsch2.tex|, |cdocspt3.tex|, |cdocspt4.tex|,
|cdocsdrf.tex|, |cdocsfn1.tex|, |cdocsfn2.tex|
as well as |childdoc.pdf|.

%%%%%%%%%%%%%%%%%%%%%%%%%%%%%%%%%%%%%%%%%%%%%%%%%%%%%%%%%%%%%%%%%%%%%%%%%%%%%%%%
\subsection{Files and Installation}

The package consists of the files:
%
\begin{center}
\begin{tabular}{ll}
    |README.txt|   & readme file \\
    |childdoc.ins| & installation file \\
    |childdoc.dtx| & source file \\
    |childdoc.def| & definition file \\
    |cdocsamp.tex| & sample main file \\
    |cdocsch1.tex| & sample include file \\
    |cdocsch2.tex| & sample include file \\
    |cdocspt3.tex| & sample part file \\
    |cdocspt4.tex| & sample part file \\
    |cdocsdrf.tex| & sample redirection file \\
    |cdocsfn1.tex| & sample redirection file \\
    |cdocsfn2.tex| & sample redirection file \\
    |childdoc.pdf| & manual
\end{tabular}
\end{center}
%
The distribution consists of the files
|README.txt|, |childdoc.ins| and |childdoc.dtx|.
%
\begin{itemize}
\item
Run (pdf)\LaTeX{} on |childdoc.dtx|
to compile the manual |childdoc.pdf| (this file).
\item
Run \LaTeX{} on |childdoc.ins| to create the definitions file |childdoc.def|
and the sample |cdocsamp.tex| with include files
|cdocsch1.tex|, |cdocsch2.tex|, |cdocspt3.tex|, |cdocspt4.tex|,
|cdocsdrf.tex|, |cdocsfn1.tex|, |cdocsfn2.tex|.
Then copy the file |childdoc.def| to an appropriate directory of your \LaTeX{}
distribution, e.g.\ \textit{texmf-root}|/tex/latex/childdoc|.
\end{itemize}

%%%%%%%%%%%%%%%%%%%%%%%%%%%%%%%%%%%%%%%%%%%%%%%%%%%%%%%%%%%%%%%%%%%%%%%%%%%%%%%%
\subsection{Related CTAN Packages}

There are several other packages which offer a similar functionality:
%
\begin{itemize}
\item
The packages
\href{http://ctan.org/pkg/docmute}{\textsf{docmute}},
\href{http://ctan.org/pkg/includex}{\textsf{includex}} and
\href{http://ctan.org/pkg/standalone}{\textsf{standalone}}
provide commands to include only the document body of
a child file thus allowing both files to be compiled individually.
\item
The packages \href{http://ctan.org/pkg/subdocs}{\textsf{subdocs}}
and \href{http://ctan.org/pkg/subfiles}{\textsf{subfiles}}
provide structures in which the main and child documents can be
encapsulated and allowing them to be compiled individually.
The inclusion mechanism is different from the conventional |\include|.
\item
The package \href{http://ctan.org/pkg/combine}{\textsf{combine}}
is an elaborate solution to combine several documents into one.
\end{itemize}
%
See also the CTAN topic \href{http://ctan.org/topic/subdocs}{\textsf{subdocs}}
for further related packages.
The present package differs from the above solutions in that
a document structure constructed with the conventional |\include| mechanism
just needs two extra commands at the top of every file
such that all constituent files can be compiled individually.

%%%%%%%%%%%%%%%%%%%%%%%%%%%%%%%%%%%%%%%%%%%%%%%%%%%%%%%%%%%%%%%%%%%%%%%%%%%%%%%%
%\subsection{Feature Suggestions}
%
%The following is a list of features which may be useful for future
%versions of this package:
%%
%\begin{itemize}
%\item
%\ldots
%\end{itemize}

%%%%%%%%%%%%%%%%%%%%%%%%%%%%%%%%%%%%%%%%%%%%%%%%%%%%%%%%%%%%%%%%%%%%%%%%%%%%%%%%
\subsection{Revision History}

%%%%%%%%%%%%%%%%%%%%%%%%%%%%%%%%%%%%%%%%
\paragraph{v2.0:} 2018/12/30

\begin{itemize}
\item
immediate forward processing
\item
added |\childdocby| mechanism
\item
manual restructured
\end{itemize}

%%%%%%%%%%%%%%%%%%%%%%%%%%%%%%%%%%%%%%%%
\paragraph{v1.6:} 2018/01/17

\begin{itemize}
\item
application for development of include files
\item
corrections to manual
\end{itemize}

%%%%%%%%%%%%%%%%%%%%%%%%%%%%%%%%%%%%%%%%
\paragraph{v1.5:} 2017/05/21

\begin{itemize}
\item
more complete structuring introduced
\item
|\childdocof| introduced
\item
|\childdoc| renamed to |\childdocmain|
\item
|\childredirect| renamed to |\childdocforward| and |\childdocforwardprefix|
and functionality expanded
\end{itemize}

%%%%%%%%%%%%%%%%%%%%%%%%%%%%%%%%%%%%%%%%
\paragraph{v1.0:} 2017/04/27

\begin{itemize}
\item
manual and install package
\item
first version published on CTAN
\end{itemize}

%%%%%%%%%%%%%%%%%%%%%%%%%%%%%%%%%%%%%%%%
\paragraph{v0.6:} 2017/04/26

\begin{itemize}
\item
redirection mechanism added
\end{itemize}

%%%%%%%%%%%%%%%%%%%%%%%%%%%%%%%%%%%%%%%%
\paragraph{v0.5:} 2017/04/26

\begin{itemize}
\item
functionality in definition file
\end{itemize}


%%%%%%%%%%%%%%%%%%%%%%%%%%%%%%%%%%%%%%%%%%%%%%%%%%%%%%%%%%%%%%%%%%%%%%%%%%%%%%%%
%%%%%%%%%%%%%%%%%%%%%%%%%%%%%%%%%%%%%%%%%%%%%%%%%%%%%%%%%%%%%%%%%%%%%%%%%%%%%%%%
%%%%%%%%%%%%%%%%%%%%%%%%%%%%%%%%%%%%%%%%%%%%%%%%%%%%%%%%%%%%%%%%%%%%%%%%%%%%%%%%
\appendix

\settowidth\MacroIndent{\rmfamily\scriptsize 000\ }

 \DocInput{childdoc.dtx}

\end{document}
%</driver>
% \fi
%
% %%%%%%%%%%%%%%%%%%%%%%%%%%%%%%%%%%%%%%%%%%%%%%%%%%%%%%%%%%%%%%%%%%%%%%%%%%%%%%
% %%%%%%%%%%%%%%%%%%%%%%%%%%%%%%%%%%%%%%%%%%%%%%%%%%%%%%%%%%%%%%%%%%%%%%%%%%%%%%
% \section{Sample}
%\iffalse
%<*samplemain>
%\fi
%
% The following presents a sample document
% with two chapters, two parts, a title page,
% a compile flag as well as three forwarding files to set the flag.
% It consists of eight |.tex| files:
% \begin{center}
% \begin{tabular}{ll}
% |cdocsamp.tex|&main file\\
% |cdocsch1.tex|&include file for chapter 1\\
% |cdocsch2.tex|&include file for chapter 2\\
% |cdocspt3.tex|&include file for part 3\\
% |cdocspt4.tex|&include file for part 4\\
% |cdocsdrf.tex|&forwarding file for main file in draft mode\\
% |cdocsfi1.tex|&forwarding file for final version of chapter 1\\
% |cdocsfi2.tex|&forwarding file for final version of chapter 2\\
% \end{tabular}
% \end{center}
% Each of the eight files can be compiled directly by the \LaTeX{} compiler.
%
% %%%%%%%%%%%%%%%%%%%%%%%%%%%%%%%%%%%%%%
% \paragraph{Main File.}
%
% The main file is called |cdocsamp.tex|.
%
% Load the \textsf{childdoc} definitions and
% declare the filename for the main document:
%    \begin{macrocode}
\input{childdoc.def}
\childdocmain{}
%    \end{macrocode}

% Optional override for |\version| flag:
%    \begin{macrocode}
%%\ifchilddoc\else\providecommand{\version}{draft}\fi
%    \end{macrocode}

% Define the default values for the |\version| flag
% (|final| for the main file and |draft| for childs):
%    \begin{macrocode}
\ifchilddoc
\providecommand{\version}{draft}
\else
\providecommand{\version}{final}
\fi
%    \end{macrocode}

% Load the standard document class:
%    \begin{macrocode}
\documentclass[12pt]{article}
%    \end{macrocode}

% Start the document body:
%    \begin{macrocode}
\begin{document}
%    \end{macrocode}

% Declare a title page.
% Print title, part of document being processed and version flag:
%    \begin{macrocode}
\addtocounter{page}{-1}
\begin{center}
{\LARGE\bfseries{}childdoc example\par}
\vspace{1cm}
\ifchilddoc
\ifchilddocmanual part\else chapter\fi:
`\childdocname' of `\childdocjob'\par
\else
main document: `\childdocjob'\par
\fi
version: \version\par
\end{center}
\newpage
%    \end{macrocode}

% Manually include selected file,
% otherwise process as usual:
%    \begin{macrocode}
\ifchilddocmanual
\section*{part `\childdocname'}
\input{\childdocname}
\else
%    \end{macrocode}

% Include the two chapters:
%    \begin{macrocode}
\include{cdocsch1}
\include{cdocsch2}
%    \end{macrocode}

% Include the two parts unless only chapters should be displayed:
%    \begin{macrocode}
\ifchilddoc\else
\section{part three}
\input{cdocspt3}
\section{part four}
\input{cdocspt4}
\fi
%    \end{macrocode}

% Process as usual until here:
%    \begin{macrocode}
\fi
%    \end{macrocode}

% End of document body:
%    \begin{macrocode}
\end{document}
%    \end{macrocode}
%\iffalse
%</samplemain>
%\fi
%
% %%%%%%%%%%%%%%%%%%%%%%%%%%%%%%%%%%%%%%
% \paragraph{Chapter Include Files.}
%
% The include files are called |cdocsch1.tex| and |cdocsch2.tex|.
%
%\iffalse
%<*samplechap1|samplechap2>
%\fi

% Optional override for |\version| flag:
%    \begin{macrocode}
%%\providecommand{\version}{final}
%    \end{macrocode}

% Include the main document:
%    \begin{macrocode}
\input{childdoc.def}
\childdocof{cdocsamp}
%    \end{macrocode}

%\iffalse
%</samplechap1|samplechap2>
%\fi
%
%\iffalse
%<*samplechap1>
%\fi
% Some text for chapter 1:
%    \begin{macrocode}
\section{one}
some text in chapter one
%    \end{macrocode}

%\iffalse
%</samplechap1>
%\fi
% Some text for chapter 2:
%\iffalse
%<*samplechap2>
%\fi
%    \begin{macrocode}
\section{two}
more text in chapter two
%    \end{macrocode}

%\iffalse
%</samplechap2>
%\fi
%
% %%%%%%%%%%%%%%%%%%%%%%%%%%%%%%%%%%%%%%
% \paragraph{Part Include Files.}
%
% The include files are called |cdocspt3.tex| and |cdocspt4.tex|.
%
%\iffalse
%<*samplepart3|samplepart4>
%\fi

% Optional override for |\version| flag:
%    \begin{macrocode}
%%\providecommand{\version}{final}
%    \end{macrocode}

% Include the main document:
%    \begin{macrocode}
\input{childdoc.def}
\childdocby{cdocsamp}
%    \end{macrocode}

%\iffalse
%</samplepart3|samplepart4>
%\fi
%
%\iffalse
%<*samplepart3>
%\fi
% Some text for part 3:
%    \begin{macrocode}
some text in part three
%    \end{macrocode}

%\iffalse
%</samplepart3>
%\fi
% Some text for part 4:
%\iffalse
%<*samplepart4>
%\fi
%    \begin{macrocode}
more text in part four
%    \end{macrocode}

%\iffalse
%</samplepart4>
%\fi
%
% %%%%%%%%%%%%%%%%%%%%%%%%%%%%%%%%%%%%%%
% \paragraph{Forwarding for a Complete Draft.}
%
% The following forwarding file |cdocsdrf.tex|
% compiles the main document in draft mode:
%\iffalse
%<*sampledraft>
%\fi
%    \begin{macrocode}
\def\version{draft}
\input{childdoc.def}
\childdocforward{cdocsamp}
%    \end{macrocode}

%\iffalse
%</sampledraft>
%\fi
%
% %%%%%%%%%%%%%%%%%%%%%%%%%%%%%%%%%%%%%%
% \paragraph{Forwarding for Final Version of the Chapters.}
%
% The following forwarding files |cdocsfn1.tex| and |cdocsfn2.tex|
% (with identical content)
% compile the final versions of the child documents
% |cdocsch1.tex| and |cdocsch2.tex|, respectively:
%\iffalse
%<*samplefinal>
%\fi
%    \begin{macrocode}
\def\version{final}
\input{childdoc.def}
\childdocforwardprefix[cdocsamp]{cdocsfn}{cdocsch}
%    \end{macrocode}

%\iffalse
%</samplefinal>
%\fi
%
% %%%%%%%%%%%%%%%%%%%%%%%%%%%%%%%%%%%%%%
% \paragraph{Command Line Processing.}
%
% The following three command lines generate the output files
% |cdocscld|, |cdocscl1| and |cdocscl2|
% which should be identical to
% |cdocsdrf|, |cdocsch1| and |cdocsfn2|, respectively:
% \begin{center}
% \begin{tabular}{l}
% |latex -jobname cdocscld \|\\
% |  "\def\version{draft}\input{childdoc.def}\childdocforward{cdocsamp}"|\\
% |latex -jobname cdocscl1 \|\\
% |  "\input{childdoc.def}\childdocforward[cdocsamp]{cdocsch1}"|\\
% |latex -jobname cdocscl2 \|\\
% |  "\def\version{final}\input{childdoc.def}\childdocforward{cdocsch2}"|
% \end{tabular}
% \end{center}
% Note that the trailing backslash on each first line
% merely continues the input to the second line
% (for convenient cut ant paste).
% Furthermore, the command |latex| can be replaced by any
% of its alternative versions such as |pdflatex|.
%
% %%%%%%%%%%%%%%%%%%%%%%%%%%%%%%%%%%%%%%%%%%%%%%%%%%%%%%%%%%%%%%%%%%%%%%%%%%%%%%
% %%%%%%%%%%%%%%%%%%%%%%%%%%%%%%%%%%%%%%%%%%%%%%%%%%%%%%%%%%%%%%%%%%%%%%%%%%%%%%
% \section{Implementation}
%\iffalse
%<*package>
%\fi
%
% This section describes the definitions file |childdoc.def|.

% The definitions cannot be loaded using |\usepackage| or |\RequirePackage|
% which has a mechanism to prevent loading a style file more than once.
% When loading the definitions by means of |\input|
% multiple instances have to be prevented manually:
%\iffalse
%This code needs to be before the `\ProvidesFile' directive
%which is defined at the beginning of this file.
%Therefore it is also placed there and commented out here.
%</package>
%<*discard>
%\fi
%    \begin{macrocode}
\ifdefined\childdocmain\endinput\fi
%    \end{macrocode}
%\iffalse
%</discard>
%<*package>
%\fi
%
% \macro{\ifchilddoc}
% \macro{\ifchilddocmanual}
% The conditional |\ifchilddoc| tells whether a
% child (true) or main (false) document is being compiled.
% The conditional |\ifchilddocmanual| tells whether
% the |\includeonly| mechanism is used (false) or
% the selection of child files must be performed manually (true).
% The definitions initialise to false:
%    \begin{macrocode}
\newif\ifchilddoc
\newif\ifchilddocmanual
%    \end{macrocode}

% \macro{\childdocname}
% \macro{\childdocjob}
% The macro |\childdocname| stores the name of the main document
% to be compiled. The macro |\childdocjob| stores the name of
% the document on which the \LaTeX{} compiler was originally invoked.
% The content of |\jobname| cannot be compared
% to filenames specified in the source due to different catcodes.
% The following code rescans |\jobname|, stores the result
% in |\childdocname| and saves a copy in |\childdocjob|:
%    \begin{macrocode}
\edef\childdocname{\scantokens\expandafter{\jobname\noexpand}}
\let\childdocjob\childdocname
%    \end{macrocode}

% \macro{\childdocdisable}
% The macro |\childdocdisable| prevents the main file
% from being processed more than once.
% At this stage, the main document command |\childdocmain|
% is assumed to be called once again where it should do nothing.
% Any subsequent call to it should prevent
% a secondary processing of the main document
% It overwrites the forwarding commands
% |\childdocof| and |\childdocforward|
% with empty macros to prevent further inclusions of the main document:
%    \begin{macrocode}
\newcommand{\childdocdisable}
{
  \renewcommand{\childdocmain}[1]{\renewcommand{\childdocmain}[1]{\endinput}}
  \renewcommand{\childdocof}[1]{}
  \renewcommand{\childdocby}[2][]{}
  \renewcommand{\childdocforward}[2][]{}
  \renewcommand{\childdocdisable}{}
}
%    \end{macrocode}

% \macro{\childdocmain}
% The macro |\childdocmain| is to be called at the top of the main file
% with nothing or the main filename (without extension) as argument.
% First, it breaks loops.
% If the argument is not empty and does not match |\childdocname|
% (which is set by the first inclusion of |childdoc.def|),
% |\ifchilddoc| is set to true, |\includeonly| is applied to the child file
% and |\jobname| is set to the main file
% (for proper handling of |.aux| files):
%    \begin{macrocode}
\newcommand{\childdocmain}[1]
{
  \childdocdisable\childdocmain{}
  \if?#1?\else
    \begingroup
      \def\childdoctmp{#1}
      \ifx\childdoctmp\childdocname
        \def\childdoctmp{}
      \else
        \def\childdoctmp
        {
          \childdoctrue
          \includeonly{\childdocname}
          \def\childdocjob{#1}
          \def\jobname{#1}
        }
      \fi
      \expandafter
    \endgroup
    \childdoctmp
  \fi
}
%    \end{macrocode}

% \macro{\childdocof}
% The command |\childdocof| redirects
% compilation to the main file |#1|.
%    \begin{macrocode}
\newcommand{\childdocof}[1]
{
  \childdocdisable
  \childdoctrue
  \includeonly{\childdocname}
  \def\jobname{#1}
  \def\childdocjob{#1}
  \input{#1}
}
%    \end{macrocode}

% \macro{\childdocby}
% The command |\childdocby| ....
%    \begin{macrocode}
\newcommand{\childdocby}[2][]
{
  \childdocdisable
  \childdoctrue
  \childdocmanualtrue
  \if?#1?\else
    \def\jobname{#2}
  \fi
  \def\childdocjob{#2}
  \input{#2}
  \endinput
}
%    \end{macrocode}

% \macro{\childdocforward}
% The command |\childdocforward| redirects
% compilation to the main file or
% (if the optional argument is given) a child file.
% Parameters are set as if the main file
% or a child file starting with |\childdocof| was compiled.
% Then compilation is handed over to the main file:
%    \begin{macrocode}
\newcommand{\childdocforward}[2][]
{
  \begingroup
    \if?#1?
      \def\childdoctmp
      {
        \def\childdocname{#2}
        \def\childdocjob{#2}
        \def\jobname{#2}
        \input{#2}
        \endinput
      }
    \else
      \def\childdoctmp
      {
        \childdocdisable
        \def\childdocname{#2}
        \childdoctrue
        \includeonly{#2}
        \def\childdocjob{#1}
        \def\jobname{#1}
        \input{#1}
        \endinput
      }
    \fi
    \expandafter
  \endgroup
  \childdoctmp
}
%    \end{macrocode}

% \macro{\childdocforwardprefix}
% The command |\childdocforwardprefix| redirects
% compilation to the main or a child file by means of a pattern.
% The prefix |#1| in the current filename is replaced by |#2|
% and the suffix of the current filename is kept
% (it is assumed that the filename does not contain the substring `|~~~|'
% which is used as a delimiter).
% Compilation is handed over to the new file by |\childdocforward|:
%    \begin{macrocode}
\newcommand{\childdocforwardprefix}[3][]
{
  \begingroup
    \def\childdocextract #2##1~~~{\def\childdoctmp{\childdocforward[#1]{#3##1}}}
    \expandafter\childdocextract\childdocname~~~
    \expandafter
  \endgroup
  \childdoctmp
}
%    \end{macrocode}

% \macro{\childdoc}
% The deprecated macro |\childdoc| is a legacy version of |\childdocmain|:
%    \begin{macrocode}
\newcommand{\childdoc}{\childdocmain}
%    \end{macrocode}

% \macro{\childdocredirect}
% The deprecated macro |\childdocredirect| is a legacy version
% of |\childdocforward| and |\childdocforwardprefix|:
%    \begin{macrocode}
\newcommand{\childdocredirect}[2][]
{
  \begingroup
    \if?#1?
      \def\childdoctmp{\childdocforward{#2}}
    \else
      \def\childdoctmp{\childdocforwardprefix{#1}{#2}}
    \fi
    \expandafter
  \endgroup
  \childdoctmp
}
%    \end{macrocode}

%\iffalse
%</package>
%\fi
%
\endinput
\childdocforward[cdocsamp]{cdocsch1}"|\\
% |latex -jobname cdocscl2 \|\\
% |  "\def\version{final}% \iffalse
%
% childdoc.dtx Copyright (C) 2017-2018 Niklas Beisert
%
% This work may be distributed and/or modified under the
% conditions of the LaTeX Project Public License, either version 1.3
% of this license or (at your option) any later version.
% The latest version of this license is in
%   http://www.latex-project.org/lppl.txt
% and version 1.3 or later is part of all distributions of LaTeX
% version 2005/12/01 or later.
%
% This work has the LPPL maintenance status `maintained'.
%
% The Current Maintainer of this work is Niklas Beisert.
%
% This work consists of the files childdoc.dtx and childdoc.ins
% and the derived files childdoc.def and cdocsamp.tex with
% cdocsch1.tex, cdocsch2.tex, cdocsdrf.tex, cdocsfn1.tex, cdocsfn2.tex.
%
%<package>\ifdefined\childdocmain\endinput\fi
%<package>\ProvidesFile{childdoc.def}[2018/12/30 v2.0 child document driver]
%<samplemain>\ProvidesFile{cdocsamp.tex}[2018/12/30 v2.0 sample for childdoc]
%<*driver>
%\ProvidesFile{childdoc.drv}[2018/12/30 v2.0 childdoc reference manual file]
\PassOptionsToClass{10pt,a4paper}{article}
\documentclass{ltxdoc}

\usepackage[margin=35mm]{geometry}
\usepackage{hyperref}
\usepackage{hyperxmp}
\usepackage[usenames]{color}

\hypersetup{colorlinks=true}
\hypersetup{pdfstartview=FitH}
\hypersetup{pdfpagemode=UseNone}
\hypersetup{pdfsource={}}
\hypersetup{pdflang={en-UK}}
\hypersetup{pdfcopyright={Copyright 2017-2018 Niklas Beisert.
  This work may be distributed and/or modified under the
  conditions of the LaTeX Project Public License, either version 1.3
  of this license or (at your option) any later version.}}
\hypersetup{pdflicenseurl={http://www.latex-project.org/lppl.txt}}
\hypersetup{pdfcontactaddress={ETH Zurich, ITP, HIT K,
  Wolfgang-Pauli-Strasse 27}}
\hypersetup{pdfcontactpostcode={8093}}
\hypersetup{pdfcontactcity={Zurich}}
\hypersetup{pdfcontactcountry={Switzerland}}
\hypersetup{pdfcontactemail={nbeisert@itp.phys.ethz.ch}}
\hypersetup{pdfcontacturl={http://people.phys.ethz.ch/\xmptilde nbeisert/}}

\newcommand{\secref}[1]{\hyperref[#1]{section \ref*{#1}}}

\parskip1ex
\parindent0pt
\let\olditemize\itemize
\def\itemize{\olditemize\parskip0pt}

\begin{document}

\title{The \textsf{childdoc} Package}
\hypersetup{pdftitle={The childdoc Package}}
\author{Niklas Beisert\\[2ex]
  Institut f\"ur Theoretische Physik\\
  Eidgen\"ossische Technische Hochschule Z\"urich\\
  Wolfgang-Pauli-Strasse 27, 8093 Z\"urich, Switzerland\\[1ex]
  \href{mailto:nbeisert@itp.phys.ethz.ch}
  {\texttt{nbeisert@itp.phys.ethz.ch}}}
\hypersetup{pdfauthor={Niklas Beisert}}
\hypersetup{pdfsubject={Manual for the LaTeX2e Package childdoc}}
\date{30 December 2018, \textsf{v2.0}}
\maketitle

\begin{abstract}\noindent
\textsf{childdoc} is a \LaTeXe{} package
that enables the direct compilation
of document sections included by |\include|
to individual files.
\end{abstract}

\begingroup
\parskip0ex
\tableofcontents
\endgroup

%%%%%%%%%%%%%%%%%%%%%%%%%%%%%%%%%%%%%%%%%%%%%%%%%%%%%%%%%%%%%%%%%%%%%%%%%%%%%%%%
%%%%%%%%%%%%%%%%%%%%%%%%%%%%%%%%%%%%%%%%%%%%%%%%%%%%%%%%%%%%%%%%%%%%%%%%%%%%%%%%
\section{Introduction}

\LaTeX{} provides a mechanism to structure a large document (such as a book)
into a main file and several child files (containing the chapters)
using the |\include| command.
This mechanism is beneficial for documents
which span hundreds of pages in order to
make the source file(s) more manageable.
Moreover, compilation can be restricted to
selected child files by means of the |\includeonly| command.
The latter feature can be used to reduce the compilation time while editing
(this was significantly more useful in the earlier days of \LaTeX{})
or to generate a smaller document which is easier to navigate.
Another application of |\includeonly| is to generate
documents consisting of selected parts of the complete document.

However, there are a few drawbacks of the plain |\include| mechanism:
\begin{itemize}
\item
The child files cannot be compiled on their own,
they can only be compiled via the main file.
A naive editing environment
(such as a text editor with an option
to have the current file processed by \LaTeX)
may require one to switch to the main file before compiling;
attempting to compile the child file produces errors.
\item
The main file must be modified (each time)
to adjust the |\includeonly| command
to the present needs. This easily leaves the main file in a messy state.
\item
The generated document will always carry the filename
of the main document. This is inconvenient if
several child files are to be compiled and
to be kept for distribution.
\end{itemize}

The present package provides a simple interface
to make child files individually compilable by \LaTeX{}.
Compiling a child file then has the same effect as compiling
the main file with an |\includeonly| command
to select the appropriate child.
Moreover the generated document will carry the name of the child
rather than the main file.
This resolves all three above issues.

This feature is meant to make the editing of books,
thesis documents and lecture notes somewhat more convenient.
However, the package can also be used efficiently for
composing a series of documents (such as exercise sheets)
which are typically distributed individually.
It then assists the author in generating the individual documents
(potentially in different versions)
as well as a document containing the collected series.
Another application is in developing style files
or other kinds of included material
where compilation of the style file could redirect
to a sample or test file.

%%%%%%%%%%%%%%%%%%%%%%%%%%%%%%%%%%%%%%%%%%%%%%%%%%%%%%%%%%%%%%%%%%%%%%%%%%%%%%%%
%%%%%%%%%%%%%%%%%%%%%%%%%%%%%%%%%%%%%%%%%%%%%%%%%%%%%%%%%%%%%%%%%%%%%%%%%%%%%%%%
\section{Usage}

First of all, the package \textsf{childdoc} is \emph{not} a standard
\LaTeXe{} |.sty| style file! Therefore it needs to be invoked in
a non-standard way.

%%%%%%%%%%%%%%%%%%%%%%%%%%%%%%%%%%%%%%%%%%%%%%%%%%%%%%%%%%%%%%%%%%%%%%%%%%%%%%%%
\subsection{Included Files}
\label{sec:include}

%%%%%%%%%%%%%%%%%%%%%%%%%%%%%%%%%%%%%%%%
\DescribeMacro{\childdocmain}
To use the package, add the commands
\begin{center}
\begin{tabular}{l}
|\input{childdoc.def}|\\
|\childdocmain{}|\\
\end{tabular}
\end{center}
at the very top of the main \LaTeX{} file,
in particular \emph{before} the |\documentclass| statement!
The argument of |\childdocmain| should be left empty
(but it must be present).

%%%%%%%%%%%%%%%%%%%%%%%%%%%%%%%%%%%%%%%%
\DescribeMacro{\childdocof}
Furthermore, add the commands
\begin{center}
\begin{tabular}{l}
|\input{childdoc.def}|\\
|\childdocof{|\textit{main}|}|\\
\end{tabular}
\end{center}
at the top of every child file \textit{child}
which is included by |\include{|\textit{child}|}|
from within the main file
(or at least for those files to be compiled individually).
The argument \textit{main} must be the filename of the main file.

There are a couple of
considerations in setting up the main and child documents:

%%%%%%%%%%%%%%%%%%%%%%%%%%%%%%%%%%%%%%%%
\paragraph{Restrictions.}

Please note the following restrictions:
\begin{itemize}
\item
|\childdocmain| must be called with one argument \textit{main}
to ensure compatibility with earlier version of the package.
It must either be empty (|\childdocmain{}|)
or precisely match the filename of the main file in which it is specified.
See \secref{sec:detection} for further information.
\item
The filename \textit{main} must be specified without the |.tex| extension.
\item
The filename \textit{main} is case sensitive
(even in case-insensitive file systems)
due to internal string comparison.
\item
The argument \textit{main} should be fully expanded, it cannot be a macro.
\item
Subdirectories and special characters should be avoided in filenames.
\item
The command |\childdocmain{|\textit{main}|}| must be followed by a whitespace.
It should not be followed immediately by another command
or by a comment mark `|%|'.
This is because the \TeX{} parser reads the token immediately following
the argument of |\childdocmain| and puts it
at the beginning of every child section;
however, a white\-space is ignored.
\end{itemize}

%%%%%%%%%%%%%%%%%%%%%%%%%%%%%%%%%%%%%%%%
\paragraph{Content of Main File.}

It is advisable to place all content in the child files included by |\include|.
Any output contained in the main file will appear in all child documents
unless suppressed manually;
it cannot be suppressed automatically by the |\includeonly| directive
and thus should normally be avoided.
A method to include some content in the main file
by means of conditional processing is described in \secref{sec:conditional}.

%%%%%%%%%%%%%%%%%%%%%%%%%%%%%%%%%%%%%%%%
\paragraph{Page Numbering.}

When only a part of the document is compiled,
the appropriate numbering of pages
(as well as other status parameters)
is determined from the |.aux| files.
The latter contain information from previous passes.
However this information needs to propagate through
all intermediate child documents.
Therefore the page numbering in child documents may well
be inconsistent until the complete document is compiled at least once.

A useful (if unconventional) way to always ensure a consistent
page numbering is to restart the numbering in each child document
and denote the pages by `\textit{child}|.|\textit{page}'
where \textit{child} represents the chapter/section number of the child file.
This can be achieved by the command
|\numberwithin{page}{|\textit{child}|}|
of the \textsf{amsmath} package
where \textit{child} can be |chapter| or |section|
depending on the chosen structuring.
Alternatively, one can modify the macro |\thepage| appropriately
and reset the counter |page| at the start of each child file.

%%%%%%%%%%%%%%%%%%%%%%%%%%%%%%%%%%%%%%%%%%%%%%%%%%%%%%%%%%%%%%%%%%%%%%%%%%%%%%%%
\subsection{Conditional Processing}
\label{sec:conditional}

The package provides a mechanism to compile different versions
of a document. To customise the versions further some conditional processing
can come in handy to distinguish which version is being compiled.
The package provides two macros to describe the compilation context:

%%%%%%%%%%%%%%%%%%%%%%%%%%%%%%%%%%%%%%%%
\DescribeMacro{\ifchilddoc}
The conditional |\ifchilddoc| distinguishes between the compilation of
child documents and the main document:
%
\begin{center}
|\ifchilddoc |\textit{child-code}| |[|\||else |\textit{main-code}]| \||fi|
\end{center}

%%%%%%%%%%%%%%%%%%%%%%%%%%%%%%%%%%%%%%%%
\DescribeMacro{\childdocname}
\DescribeMacro{\childdocjob}
The macro |\childdocname| contains the filename (without extension)
of the main or child file being processed.
Note that |\childdocjob| will always contain the name of the main file.

%%%%%%%%%%%%%%%%%%%%%%%%%%%%%%%%%%%%%%%%
\paragraph{Title Page.}

Conditional processing can be used to include a title or banner page
in the main document when proper precautions are taken.
Importantly, the code in the main file should ensure that the page counter
(as well as other status parameters which are stored in the |.aux| files)
takes the same value after the conditional processing.
Otherwise the page numbers may take divergent values
depending on which part is compiled.

For example, a title page could be declared by:
%
\begin{center}
\begin{tabular}{l}
|\ifchilddoc\||else|\\
|\addtocounter{page}{-1}|\\
\textit{code for title page}\\
|\newpage|\\
|\||fi|
\end{tabular}
\end{center}
%
A banner page for the child documents can be generated by:
%
\begin{center}
\begin{tabular}{l}
|\ifchilddoc|\\
|\addtocounter{page}{-1}|\\
\textit{code for banner page}\\
|\newpage|\\
|\||fi|
\end{tabular}
\end{center}
%
Here one could write a message such as:
\begin{center}
|This is the part \childdocname{} of \childdocjob{}.|
\end{center}

%%%%%%%%%%%%%%%%%%%%%%%%%%%%%%%%%%%%%%%%%%%%%%%%%%%%%%%%%%%%%%%%%%%%%%%%%%%%%%%%
\subsection{Flags}
\label{sec:flags}

The package makes it easy to generate different versions
of the main or child documents.
To this end compilation flags can be defined
and assigned different default values.
They will be particularly useful in conjunction
with the forwarding mechanism described in \secref{sec:forward}.

For example, it may be useful to have a flag |\version|
which can be set to |draft| or |final|.
The document source will contain some conditional code
depending on the value of |\version|.
Suppose further, the flag should default to |final| for the main file
and to |draft| for child files
which is a natural assignment for editing the document.
This is achieved by placing the following code
in the preamble of the main document
(below the |\childdocmain| directive):
%
\begin{center}
\begin{tabular}{l}
|\ifchilddoc|\\
|\providecommand{\version}{draft}|\\
|\||else|\\
|\providecommand{\version}{final}|\\
|\||fi|
\end{tabular}
\end{center}
%
The definition by |\providecommand| makes sure
that previous definitions are not overwritten.
Further statements |\providecommand{\version}{...}|
can thus be added before the above code to override it.

For the main file, one might add a line
(between |\childdocmain| and the above block)
%
\begin{center}
|%\ifchilddoc\||else\providecommand{\version}{draft}\||fi|
\end{center}
%
which can be uncommented to produce a draft version.
Likewise one can add a line to the very top of a child file
(above the |\childdocof{|\textit{main}|}| directive)
%
\begin{center}
|%\providecommand{\version}{final}|
\end{center}
%
which can be uncommented to produce the final version of this child document.

%%%%%%%%%%%%%%%%%%%%%%%%%%%%%%%%%%%%%%%%%%%%%%%%%%%%%%%%%%%%%%%%%%%%%%%%%%%%%%%%
\subsection{Forwarding}
\label{sec:forward}

Different versions of the main or child documents
using compilation flags as described in \secref{sec:flags}
can be (permanently) stored in different files
for convenient compilation, viewing and distribution.
To this end, the package defines a command
to pass on compilation to a different file:

%%%%%%%%%%%%%%%%%%%%%%%%%%%%%%%%%%%%%%%%
\DescribeMacro{\childdocforward}
The command |\childdocforward| redirects processing to
another source file:
%
\begin{center}
\begin{tabular}{l}
|\input{childdoc.def}|\\
|\childdocforward[|\textit{main}|]{|\textit{dest}|}|\\
\end{tabular}
\end{center}
%
The argument \textit{dest} is the destination file
(without extension).
It should be the main file or one of the child files.
Note that further \textsf{childdoc} directives
such as |\childdocof| and |\childdocforward|
in the indicated file will be processed in this form.
The optional argument \textit{main}
passes on directly to the main file \textit{main}
while pretending to compile the child \textit{dest}.
This form behaves as if \textit{dest}
issues |\childdocof{|\textit{main}|}| right away,
and no further \textsf{childdoc} directives will be processed.

%%%%%%%%%%%%%%%%%%%%%%%%%%%%%%%%%%%%%%%%
\DescribeMacro{\...prefix}
In the alternative form |\childdocforwardprefix|,
%
\begin{center}
\begin{tabular}{l}
|\input{childdoc.def}|\\
|\childdocforwardprefix[|\textit{main}|]{|\textit{prefix}|}{|\textit{dest}|}|
\end{tabular}
\end{center}
%
the destination file is determined by a pattern
depending on the current file:
To make this work, the current file must be called
`{\textit{prefix}\hspace{0.2em}\textit{suffix}}'
with \textit{prefix} matching precisely the argument.
Processing is then passed on to the file
`{\textit{dest}\hspace{0.2em}\textit{suffix}}'.
Surely, the same effect is achieved by
directly specifying the
argument `{\textit{dest}\hspace{0.2em}\textit{suffix}}'
in the first form.
However, that requires to set up a different file
for each child. With the alternative form of the command
all these files can have exactly the same content
which simplifies setting them up and maintaining them.

For example, the following file |draft.tex|
with a compilation flag |\version| as described in \secref{sec:flags}
compiles the main document as a draft:
%
\begin{center}
\begin{tabular}{l}
|\def\version{draft}|\\
|\input{childdoc.def}|\\
|\childdocforward{|\textit{main}|}|
\end{tabular}
\end{center}
%
Likewise, the following files |final|\textit{nn}|.tex|
compile the final version of the child document
|child|\textit{nn}|.tex|:
%
\begin{center}
\begin{tabular}{l}
|\def\version{final}|\\
|\input{childdoc.def}|\\
|\childdocforwardprefix{final}{child}|
\end{tabular}
\end{center}
%

Note that when several versions of a main file and/or of each child file
are to be generated, it may be convenient to set up a |Makefile| or
shell script to automatise the process.

%%%%%%%%%%%%%%%%%%%%%%%%%%%%%%%%%%%%%%%%%%%%%%%%%%%%%%%%%%%%%%%%%%%%%%%%%%%%%%%%
\subsection{Command Line Processing}
\label{sec:commandline}

The effect of redirection files can also be achieved by invoking
the \LaTeX{} compiler with a more elaborate command line.
Most conveniently this should be done as part
of a shell script or a |Makefile|.

When using \textsf{childdoc} in the main file, the following
command lines effectively perform a redirection
(note that depending on the shell being used,
backslashes may have to be doubled: `|\|' $\to$ `|\\|'):
%
\begin{center}
|... -jobname "|\textit{target}|" |\\|"|[\textit{flags}]%
|\input{childdoc.def}\childdocforward[|\textit{main}|]{|\textit{dest}|}"|
\end{center}
%
Here \textit{target} is the name of the output file,
\textit{main} is the name of the main file
and \textit{dest} is the name of the main or child file to be processed
(all filenames without extensions).
The optional argument \textit{main} can be omitted
if \textit{main} matches \textit{dest}.
Optionally, compilation \textit{flags} can be defined via |\def| commands.
This command line makes the \TeX{} engine believe
it is compiling the file \textit{target}
whose content is specified as the latter parameter.
The provided code then forwards the processing to
\textit{main} or \textit{dest} as described in \secref{sec:forward}.

%%%%%%%%%%%%%%%%%%%%%%%%%%%%%%%%%%%%%%%%%%%%%%%%%%%%%%%%%%%%%%%%%%%%%%%%%%%%%%%%
\subsection{Include by Input}
\label{sec:input}

Including child documents by |\include| has some restrictions by design.
Most notably, the content of a child document always occupies
its own set of pages; pages cannot be shared between child documents.
Usually, this behaviour makes perfect sense
because each child document contain an essential part of the document.
However, in some situations it may be desirable to compose
a document from a collection of parts
without having mandatory page breaks between then.
For this case, the package
provides a mechanism to include parts
by |\input| which can also be processed individually.
However, by construction this mechanism
requires manual handling of the content to be output.

%%%%%%%%%%%%%%%%%%%%%%%%%%%%%%%%%%%%%%%%
\DescribeMacro{\ifchilddocmanual}
The main file should be prepared as usual, see \secref{sec:include}.
However, the document body must make a distinction
between processing of an individual part and of the main document, e.g.:
%
\begin{center}
\begin{tabular}{l}
|\ifchilddocmanual|\\
|\input{\childdocname}|\\
|\||else|\\
\textit{document body with }|\input{|\textit{part}|}|\\
|\||fi|
\end{tabular}
\end{center}
%
The conditional |\ifchilddocmanual| is true whenever
a part to be included by |\input| is being compiled,
and the name of the part is stored in |\childdocname|.

%%%%%%%%%%%%%%%%%%%%%%%%%%%%%%%%%%%%%%%%
\DescribeMacro{\childdocby}
Each part to be included by |\input| should start with:
%
\begin{center}
\begin{tabular}{l}
|\input{childdoc.def}|\\
|\childdocby{|\textit{main}|}|\\
\end{tabular}
\end{center}
%
The directive |\childdocby| is similar to |\childdocof|
described in \secref{sec:include},
but the subsequent selection of content must be done manually.
To that end, both |\ifchilddoc| and |\ifchilddocmanual|
will be true upon processing of a part,
and the name of the part is stored in |\childdocname|.
Note that |\jobname| will be set to the filename of the current part
so that each part receives an individual |.aux| file
that does not interfere with the |.aux| file(s) of the main document.
This behaviour can be altered by the alternative form
|\childdocby[*]{|\textit{main}|}| (with a non-empty optional argument)
which uses the |.aux| file of the main document
by setting |\jobname| to \textit{main}.

%%%%%%%%%%%%%%%%%%%%%%%%%%%%%%%%%%%%%%%%%%%%%%%%%%%%%%%%%%%%%%%%%%%%%%%%%%%%%%%%
\subsection{Driver Development}
\label{sec:driver}

The \textsf{childdoc} mechanism can also be use for the development
of definition files such as \LaTeX{} styles or classes.
This case differs from the above setup with multiple parts
included by |\include| in that no |\includeonly| should be invoked.
This can be achieved by starting the include file
(before |\ProvidesPackage|) with:
%
\begin{center}
\begin{tabular}{l}
|\input{childdoc.def}|\\
|\childdocforward{|\textit{main}|}|\\
\end{tabular}
\end{center}
%
or alternatively with:
%
\begin{center}
\begin{tabular}{l}
|\input{childdoc.def}|\\
|\childdocby{|\textit{main}|}|\\
\end{tabular}
\end{center}
%
Both forms have slightly different effects as described above.
The main file is prepared as usual, see \secref{sec:include}.

%%%%%%%%%%%%%%%%%%%%%%%%%%%%%%%%%%%%%%%%%%%%%%%%%%%%%%%%%%%%%%%%%%%%%%%%%%%%%%%%
\subsection{Legacy Detection}
\label{sec:detection}

The directive |\childdocmain| in the main file can detect
whether the complete document or merely a child is to be compiled
even without using the directive |\childdocof|.
This method is deprecated because it is less robust
and there is no compelling reason to use it;
it is merely provided for backward compatibility
and it may be removed in future versions.

If the detection mechanism is to be used,
it is mandatory to correctly specify
the filename of the main file as the argument of |\childdocmain|:
%
\begin{center}
\begin{tabular}{l}
|\input{childdoc.def}|\\
|\childdocmain{|\textit{main}|}|\\
\end{tabular}
\end{center}
%
If |\jobname| does not match the argument \textit{main} of |\childdocmain|,
it is assumed that |\jobname| points to the child file to be compiled.
When using |\childdocmain| with the main file specified as argument,
it suffices to start a child file
with just |\input{|\textit{main}|}|
without loading of the package and using |\childdocof|.
If instead all processing is done
with the appropriate \textsf{childdoc} directives,
the argument of \textit{main} of |\childdocmain| can be empty.

An alternative version of the command line processing described
in \secref{sec:commandline} using the detection mechanism reads:
%
\begin{center}
|... -jobname "|\textit{target}|" "|[\textit{flags}]%
[|\def\jobname{|\textit{dest}|}|]|\input{|\textit{main}|}"|
\end{center}

%%%%%%%%%%%%%%%%%%%%%%%%%%%%%%%%%%%%%%%%%%%%%%%%%%%%%%%%%%%%%%%%%%%%%%%%%%%%%%%%
\subsection{Manual Code}
\label{sec:manual}

In case one cannot be certain whether the definitions file |childdoc.def|
is installed on the target \TeX{} distribution
and one prefers not to ship it,
it is conceivable to paste a few relevant commands into the sources.

To that end, drop all statements |\input{childdoc.def}|
and perform the replacements as outlined below.
Instead of |\childdocmain{|\textit{main}|}| add the following code
to the top of the main file:
%
\begin{center}
\begin{tabular}{l}
|\||ifdefined\childdocname\endinput\||fi\newif\ifchilddoc|\\
|\edef\childdocname{\scantokens\expandafter{\jobname\noexpand}}|\\
|\def\childdocmain{|\textit{main}|}\||ifx\childdocmain\childdocname\||else|\\
|\childdoctrue\includeonly{\childdocname}\let\jobname\childdocmain\||fi|\\
\end{tabular}
\end{center}
%
Instead of |\childdocof{|\textit{main}|}| just include the main file
at the top of each child file:
%
\begin{center}
|\input{|\textit{main}|}|
\end{center}
%
A simple redirection |\childdocforward{|\textit{dest}|}| is achieved by:
%
\begin{center}
|\def\jobname{|\textit{dest}|}\input{\jobname}|
\end{center}
%
The redirection with prefix
|\childdocforwardprefix[|\textit{prefix}|]{|\textit{dest}|}|
is accomplished by:
%
\begin{center}
\begin{tabular}{l}
|{\edef\jobname{\scantokens\expandafter{\jobname\noexpand}}|\\
|\def\redirectjob |\textit{prefix}|#1~~~{\gdef\jobname{|\textit{dest}|#1}}|\\
|\expandafter\redirectjob\jobname~~~}\input{\jobname}|
\end{tabular}
\end{center}

In an alternative approach,
child documents can be compiled by a specific command line
without additional code or specific definitions:
%
\begin{center}
|... -jobname "|\textit{target}|" "|[\textit{flags}]%
|\includeonly{|\textit{dest}|}\input{|\textit{main}|}"|
\end{center}
%

%%%%%%%%%%%%%%%%%%%%%%%%%%%%%%%%%%%%%%%%%%%%%%%%%%%%%%%%%%%%%%%%%%%%%%%%%%%%%%%%
%%%%%%%%%%%%%%%%%%%%%%%%%%%%%%%%%%%%%%%%%%%%%%%%%%%%%%%%%%%%%%%%%%%%%%%%%%%%%%%%
\section{Information}

%%%%%%%%%%%%%%%%%%%%%%%%%%%%%%%%%%%%%%%%%%%%%%%%%%%%%%%%%%%%%%%%%%%%%%%%%%%%%%%%
\subsection{Copyright}

Copyright \copyright{} 2017--2018 Niklas Beisert

This work may be distributed and/or modified under the
conditions of the \LaTeX{} Project Public License, either version 1.3
of this license or (at your option) any later version.
The latest version of this license is in
  \url{http://www.latex-project.org/lppl.txt}
and version 1.3 or later is part of all distributions of \LaTeX{}
version 2005/12/01 or later.

This work has the LPPL maintenance status `maintained'.

The Current Maintainer of this work is Niklas Beisert.

This work consists of the files |README.txt|, |childdoc.ins| and |childdoc.dtx|
as well as the derived files |childdoc.def|, |cdocsamp.tex|
with |cdocsch1.tex|, |cdocsch2.tex|, |cdocspt3.tex|, |cdocspt4.tex|,
|cdocsdrf.tex|, |cdocsfn1.tex|, |cdocsfn2.tex|
as well as |childdoc.pdf|.

%%%%%%%%%%%%%%%%%%%%%%%%%%%%%%%%%%%%%%%%%%%%%%%%%%%%%%%%%%%%%%%%%%%%%%%%%%%%%%%%
\subsection{Files and Installation}

The package consists of the files:
%
\begin{center}
\begin{tabular}{ll}
    |README.txt|   & readme file \\
    |childdoc.ins| & installation file \\
    |childdoc.dtx| & source file \\
    |childdoc.def| & definition file \\
    |cdocsamp.tex| & sample main file \\
    |cdocsch1.tex| & sample include file \\
    |cdocsch2.tex| & sample include file \\
    |cdocspt3.tex| & sample part file \\
    |cdocspt4.tex| & sample part file \\
    |cdocsdrf.tex| & sample redirection file \\
    |cdocsfn1.tex| & sample redirection file \\
    |cdocsfn2.tex| & sample redirection file \\
    |childdoc.pdf| & manual
\end{tabular}
\end{center}
%
The distribution consists of the files
|README.txt|, |childdoc.ins| and |childdoc.dtx|.
%
\begin{itemize}
\item
Run (pdf)\LaTeX{} on |childdoc.dtx|
to compile the manual |childdoc.pdf| (this file).
\item
Run \LaTeX{} on |childdoc.ins| to create the definitions file |childdoc.def|
and the sample |cdocsamp.tex| with include files
|cdocsch1.tex|, |cdocsch2.tex|, |cdocspt3.tex|, |cdocspt4.tex|,
|cdocsdrf.tex|, |cdocsfn1.tex|, |cdocsfn2.tex|.
Then copy the file |childdoc.def| to an appropriate directory of your \LaTeX{}
distribution, e.g.\ \textit{texmf-root}|/tex/latex/childdoc|.
\end{itemize}

%%%%%%%%%%%%%%%%%%%%%%%%%%%%%%%%%%%%%%%%%%%%%%%%%%%%%%%%%%%%%%%%%%%%%%%%%%%%%%%%
\subsection{Related CTAN Packages}

There are several other packages which offer a similar functionality:
%
\begin{itemize}
\item
The packages
\href{http://ctan.org/pkg/docmute}{\textsf{docmute}},
\href{http://ctan.org/pkg/includex}{\textsf{includex}} and
\href{http://ctan.org/pkg/standalone}{\textsf{standalone}}
provide commands to include only the document body of
a child file thus allowing both files to be compiled individually.
\item
The packages \href{http://ctan.org/pkg/subdocs}{\textsf{subdocs}}
and \href{http://ctan.org/pkg/subfiles}{\textsf{subfiles}}
provide structures in which the main and child documents can be
encapsulated and allowing them to be compiled individually.
The inclusion mechanism is different from the conventional |\include|.
\item
The package \href{http://ctan.org/pkg/combine}{\textsf{combine}}
is an elaborate solution to combine several documents into one.
\end{itemize}
%
See also the CTAN topic \href{http://ctan.org/topic/subdocs}{\textsf{subdocs}}
for further related packages.
The present package differs from the above solutions in that
a document structure constructed with the conventional |\include| mechanism
just needs two extra commands at the top of every file
such that all constituent files can be compiled individually.

%%%%%%%%%%%%%%%%%%%%%%%%%%%%%%%%%%%%%%%%%%%%%%%%%%%%%%%%%%%%%%%%%%%%%%%%%%%%%%%%
%\subsection{Feature Suggestions}
%
%The following is a list of features which may be useful for future
%versions of this package:
%%
%\begin{itemize}
%\item
%\ldots
%\end{itemize}

%%%%%%%%%%%%%%%%%%%%%%%%%%%%%%%%%%%%%%%%%%%%%%%%%%%%%%%%%%%%%%%%%%%%%%%%%%%%%%%%
\subsection{Revision History}

%%%%%%%%%%%%%%%%%%%%%%%%%%%%%%%%%%%%%%%%
\paragraph{v2.0:} 2018/12/30

\begin{itemize}
\item
immediate forward processing
\item
added |\childdocby| mechanism
\item
manual restructured
\end{itemize}

%%%%%%%%%%%%%%%%%%%%%%%%%%%%%%%%%%%%%%%%
\paragraph{v1.6:} 2018/01/17

\begin{itemize}
\item
application for development of include files
\item
corrections to manual
\end{itemize}

%%%%%%%%%%%%%%%%%%%%%%%%%%%%%%%%%%%%%%%%
\paragraph{v1.5:} 2017/05/21

\begin{itemize}
\item
more complete structuring introduced
\item
|\childdocof| introduced
\item
|\childdoc| renamed to |\childdocmain|
\item
|\childredirect| renamed to |\childdocforward| and |\childdocforwardprefix|
and functionality expanded
\end{itemize}

%%%%%%%%%%%%%%%%%%%%%%%%%%%%%%%%%%%%%%%%
\paragraph{v1.0:} 2017/04/27

\begin{itemize}
\item
manual and install package
\item
first version published on CTAN
\end{itemize}

%%%%%%%%%%%%%%%%%%%%%%%%%%%%%%%%%%%%%%%%
\paragraph{v0.6:} 2017/04/26

\begin{itemize}
\item
redirection mechanism added
\end{itemize}

%%%%%%%%%%%%%%%%%%%%%%%%%%%%%%%%%%%%%%%%
\paragraph{v0.5:} 2017/04/26

\begin{itemize}
\item
functionality in definition file
\end{itemize}


%%%%%%%%%%%%%%%%%%%%%%%%%%%%%%%%%%%%%%%%%%%%%%%%%%%%%%%%%%%%%%%%%%%%%%%%%%%%%%%%
%%%%%%%%%%%%%%%%%%%%%%%%%%%%%%%%%%%%%%%%%%%%%%%%%%%%%%%%%%%%%%%%%%%%%%%%%%%%%%%%
%%%%%%%%%%%%%%%%%%%%%%%%%%%%%%%%%%%%%%%%%%%%%%%%%%%%%%%%%%%%%%%%%%%%%%%%%%%%%%%%
\appendix

\settowidth\MacroIndent{\rmfamily\scriptsize 000\ }

 \DocInput{childdoc.dtx}

\end{document}
%</driver>
% \fi
%
% %%%%%%%%%%%%%%%%%%%%%%%%%%%%%%%%%%%%%%%%%%%%%%%%%%%%%%%%%%%%%%%%%%%%%%%%%%%%%%
% %%%%%%%%%%%%%%%%%%%%%%%%%%%%%%%%%%%%%%%%%%%%%%%%%%%%%%%%%%%%%%%%%%%%%%%%%%%%%%
% \section{Sample}
%\iffalse
%<*samplemain>
%\fi
%
% The following presents a sample document
% with two chapters, two parts, a title page,
% a compile flag as well as three forwarding files to set the flag.
% It consists of eight |.tex| files:
% \begin{center}
% \begin{tabular}{ll}
% |cdocsamp.tex|&main file\\
% |cdocsch1.tex|&include file for chapter 1\\
% |cdocsch2.tex|&include file for chapter 2\\
% |cdocspt3.tex|&include file for part 3\\
% |cdocspt4.tex|&include file for part 4\\
% |cdocsdrf.tex|&forwarding file for main file in draft mode\\
% |cdocsfi1.tex|&forwarding file for final version of chapter 1\\
% |cdocsfi2.tex|&forwarding file for final version of chapter 2\\
% \end{tabular}
% \end{center}
% Each of the eight files can be compiled directly by the \LaTeX{} compiler.
%
% %%%%%%%%%%%%%%%%%%%%%%%%%%%%%%%%%%%%%%
% \paragraph{Main File.}
%
% The main file is called |cdocsamp.tex|.
%
% Load the \textsf{childdoc} definitions and
% declare the filename for the main document:
%    \begin{macrocode}
\input{childdoc.def}
\childdocmain{}
%    \end{macrocode}

% Optional override for |\version| flag:
%    \begin{macrocode}
%%\ifchilddoc\else\providecommand{\version}{draft}\fi
%    \end{macrocode}

% Define the default values for the |\version| flag
% (|final| for the main file and |draft| for childs):
%    \begin{macrocode}
\ifchilddoc
\providecommand{\version}{draft}
\else
\providecommand{\version}{final}
\fi
%    \end{macrocode}

% Load the standard document class:
%    \begin{macrocode}
\documentclass[12pt]{article}
%    \end{macrocode}

% Start the document body:
%    \begin{macrocode}
\begin{document}
%    \end{macrocode}

% Declare a title page.
% Print title, part of document being processed and version flag:
%    \begin{macrocode}
\addtocounter{page}{-1}
\begin{center}
{\LARGE\bfseries{}childdoc example\par}
\vspace{1cm}
\ifchilddoc
\ifchilddocmanual part\else chapter\fi:
`\childdocname' of `\childdocjob'\par
\else
main document: `\childdocjob'\par
\fi
version: \version\par
\end{center}
\newpage
%    \end{macrocode}

% Manually include selected file,
% otherwise process as usual:
%    \begin{macrocode}
\ifchilddocmanual
\section*{part `\childdocname'}
\input{\childdocname}
\else
%    \end{macrocode}

% Include the two chapters:
%    \begin{macrocode}
\include{cdocsch1}
\include{cdocsch2}
%    \end{macrocode}

% Include the two parts unless only chapters should be displayed:
%    \begin{macrocode}
\ifchilddoc\else
\section{part three}
\input{cdocspt3}
\section{part four}
\input{cdocspt4}
\fi
%    \end{macrocode}

% Process as usual until here:
%    \begin{macrocode}
\fi
%    \end{macrocode}

% End of document body:
%    \begin{macrocode}
\end{document}
%    \end{macrocode}
%\iffalse
%</samplemain>
%\fi
%
% %%%%%%%%%%%%%%%%%%%%%%%%%%%%%%%%%%%%%%
% \paragraph{Chapter Include Files.}
%
% The include files are called |cdocsch1.tex| and |cdocsch2.tex|.
%
%\iffalse
%<*samplechap1|samplechap2>
%\fi

% Optional override for |\version| flag:
%    \begin{macrocode}
%%\providecommand{\version}{final}
%    \end{macrocode}

% Include the main document:
%    \begin{macrocode}
\input{childdoc.def}
\childdocof{cdocsamp}
%    \end{macrocode}

%\iffalse
%</samplechap1|samplechap2>
%\fi
%
%\iffalse
%<*samplechap1>
%\fi
% Some text for chapter 1:
%    \begin{macrocode}
\section{one}
some text in chapter one
%    \end{macrocode}

%\iffalse
%</samplechap1>
%\fi
% Some text for chapter 2:
%\iffalse
%<*samplechap2>
%\fi
%    \begin{macrocode}
\section{two}
more text in chapter two
%    \end{macrocode}

%\iffalse
%</samplechap2>
%\fi
%
% %%%%%%%%%%%%%%%%%%%%%%%%%%%%%%%%%%%%%%
% \paragraph{Part Include Files.}
%
% The include files are called |cdocspt3.tex| and |cdocspt4.tex|.
%
%\iffalse
%<*samplepart3|samplepart4>
%\fi

% Optional override for |\version| flag:
%    \begin{macrocode}
%%\providecommand{\version}{final}
%    \end{macrocode}

% Include the main document:
%    \begin{macrocode}
\input{childdoc.def}
\childdocby{cdocsamp}
%    \end{macrocode}

%\iffalse
%</samplepart3|samplepart4>
%\fi
%
%\iffalse
%<*samplepart3>
%\fi
% Some text for part 3:
%    \begin{macrocode}
some text in part three
%    \end{macrocode}

%\iffalse
%</samplepart3>
%\fi
% Some text for part 4:
%\iffalse
%<*samplepart4>
%\fi
%    \begin{macrocode}
more text in part four
%    \end{macrocode}

%\iffalse
%</samplepart4>
%\fi
%
% %%%%%%%%%%%%%%%%%%%%%%%%%%%%%%%%%%%%%%
% \paragraph{Forwarding for a Complete Draft.}
%
% The following forwarding file |cdocsdrf.tex|
% compiles the main document in draft mode:
%\iffalse
%<*sampledraft>
%\fi
%    \begin{macrocode}
\def\version{draft}
\input{childdoc.def}
\childdocforward{cdocsamp}
%    \end{macrocode}

%\iffalse
%</sampledraft>
%\fi
%
% %%%%%%%%%%%%%%%%%%%%%%%%%%%%%%%%%%%%%%
% \paragraph{Forwarding for Final Version of the Chapters.}
%
% The following forwarding files |cdocsfn1.tex| and |cdocsfn2.tex|
% (with identical content)
% compile the final versions of the child documents
% |cdocsch1.tex| and |cdocsch2.tex|, respectively:
%\iffalse
%<*samplefinal>
%\fi
%    \begin{macrocode}
\def\version{final}
\input{childdoc.def}
\childdocforwardprefix[cdocsamp]{cdocsfn}{cdocsch}
%    \end{macrocode}

%\iffalse
%</samplefinal>
%\fi
%
% %%%%%%%%%%%%%%%%%%%%%%%%%%%%%%%%%%%%%%
% \paragraph{Command Line Processing.}
%
% The following three command lines generate the output files
% |cdocscld|, |cdocscl1| and |cdocscl2|
% which should be identical to
% |cdocsdrf|, |cdocsch1| and |cdocsfn2|, respectively:
% \begin{center}
% \begin{tabular}{l}
% |latex -jobname cdocscld \|\\
% |  "\def\version{draft}\input{childdoc.def}\childdocforward{cdocsamp}"|\\
% |latex -jobname cdocscl1 \|\\
% |  "\input{childdoc.def}\childdocforward[cdocsamp]{cdocsch1}"|\\
% |latex -jobname cdocscl2 \|\\
% |  "\def\version{final}\input{childdoc.def}\childdocforward{cdocsch2}"|
% \end{tabular}
% \end{center}
% Note that the trailing backslash on each first line
% merely continues the input to the second line
% (for convenient cut ant paste).
% Furthermore, the command |latex| can be replaced by any
% of its alternative versions such as |pdflatex|.
%
% %%%%%%%%%%%%%%%%%%%%%%%%%%%%%%%%%%%%%%%%%%%%%%%%%%%%%%%%%%%%%%%%%%%%%%%%%%%%%%
% %%%%%%%%%%%%%%%%%%%%%%%%%%%%%%%%%%%%%%%%%%%%%%%%%%%%%%%%%%%%%%%%%%%%%%%%%%%%%%
% \section{Implementation}
%\iffalse
%<*package>
%\fi
%
% This section describes the definitions file |childdoc.def|.

% The definitions cannot be loaded using |\usepackage| or |\RequirePackage|
% which has a mechanism to prevent loading a style file more than once.
% When loading the definitions by means of |\input|
% multiple instances have to be prevented manually:
%\iffalse
%This code needs to be before the `\ProvidesFile' directive
%which is defined at the beginning of this file.
%Therefore it is also placed there and commented out here.
%</package>
%<*discard>
%\fi
%    \begin{macrocode}
\ifdefined\childdocmain\endinput\fi
%    \end{macrocode}
%\iffalse
%</discard>
%<*package>
%\fi
%
% \macro{\ifchilddoc}
% \macro{\ifchilddocmanual}
% The conditional |\ifchilddoc| tells whether a
% child (true) or main (false) document is being compiled.
% The conditional |\ifchilddocmanual| tells whether
% the |\includeonly| mechanism is used (false) or
% the selection of child files must be performed manually (true).
% The definitions initialise to false:
%    \begin{macrocode}
\newif\ifchilddoc
\newif\ifchilddocmanual
%    \end{macrocode}

% \macro{\childdocname}
% \macro{\childdocjob}
% The macro |\childdocname| stores the name of the main document
% to be compiled. The macro |\childdocjob| stores the name of
% the document on which the \LaTeX{} compiler was originally invoked.
% The content of |\jobname| cannot be compared
% to filenames specified in the source due to different catcodes.
% The following code rescans |\jobname|, stores the result
% in |\childdocname| and saves a copy in |\childdocjob|:
%    \begin{macrocode}
\edef\childdocname{\scantokens\expandafter{\jobname\noexpand}}
\let\childdocjob\childdocname
%    \end{macrocode}

% \macro{\childdocdisable}
% The macro |\childdocdisable| prevents the main file
% from being processed more than once.
% At this stage, the main document command |\childdocmain|
% is assumed to be called once again where it should do nothing.
% Any subsequent call to it should prevent
% a secondary processing of the main document
% It overwrites the forwarding commands
% |\childdocof| and |\childdocforward|
% with empty macros to prevent further inclusions of the main document:
%    \begin{macrocode}
\newcommand{\childdocdisable}
{
  \renewcommand{\childdocmain}[1]{\renewcommand{\childdocmain}[1]{\endinput}}
  \renewcommand{\childdocof}[1]{}
  \renewcommand{\childdocby}[2][]{}
  \renewcommand{\childdocforward}[2][]{}
  \renewcommand{\childdocdisable}{}
}
%    \end{macrocode}

% \macro{\childdocmain}
% The macro |\childdocmain| is to be called at the top of the main file
% with nothing or the main filename (without extension) as argument.
% First, it breaks loops.
% If the argument is not empty and does not match |\childdocname|
% (which is set by the first inclusion of |childdoc.def|),
% |\ifchilddoc| is set to true, |\includeonly| is applied to the child file
% and |\jobname| is set to the main file
% (for proper handling of |.aux| files):
%    \begin{macrocode}
\newcommand{\childdocmain}[1]
{
  \childdocdisable\childdocmain{}
  \if?#1?\else
    \begingroup
      \def\childdoctmp{#1}
      \ifx\childdoctmp\childdocname
        \def\childdoctmp{}
      \else
        \def\childdoctmp
        {
          \childdoctrue
          \includeonly{\childdocname}
          \def\childdocjob{#1}
          \def\jobname{#1}
        }
      \fi
      \expandafter
    \endgroup
    \childdoctmp
  \fi
}
%    \end{macrocode}

% \macro{\childdocof}
% The command |\childdocof| redirects
% compilation to the main file |#1|.
%    \begin{macrocode}
\newcommand{\childdocof}[1]
{
  \childdocdisable
  \childdoctrue
  \includeonly{\childdocname}
  \def\jobname{#1}
  \def\childdocjob{#1}
  \input{#1}
}
%    \end{macrocode}

% \macro{\childdocby}
% The command |\childdocby| ....
%    \begin{macrocode}
\newcommand{\childdocby}[2][]
{
  \childdocdisable
  \childdoctrue
  \childdocmanualtrue
  \if?#1?\else
    \def\jobname{#2}
  \fi
  \def\childdocjob{#2}
  \input{#2}
  \endinput
}
%    \end{macrocode}

% \macro{\childdocforward}
% The command |\childdocforward| redirects
% compilation to the main file or
% (if the optional argument is given) a child file.
% Parameters are set as if the main file
% or a child file starting with |\childdocof| was compiled.
% Then compilation is handed over to the main file:
%    \begin{macrocode}
\newcommand{\childdocforward}[2][]
{
  \begingroup
    \if?#1?
      \def\childdoctmp
      {
        \def\childdocname{#2}
        \def\childdocjob{#2}
        \def\jobname{#2}
        \input{#2}
        \endinput
      }
    \else
      \def\childdoctmp
      {
        \childdocdisable
        \def\childdocname{#2}
        \childdoctrue
        \includeonly{#2}
        \def\childdocjob{#1}
        \def\jobname{#1}
        \input{#1}
        \endinput
      }
    \fi
    \expandafter
  \endgroup
  \childdoctmp
}
%    \end{macrocode}

% \macro{\childdocforwardprefix}
% The command |\childdocforwardprefix| redirects
% compilation to the main or a child file by means of a pattern.
% The prefix |#1| in the current filename is replaced by |#2|
% and the suffix of the current filename is kept
% (it is assumed that the filename does not contain the substring `|~~~|'
% which is used as a delimiter).
% Compilation is handed over to the new file by |\childdocforward|:
%    \begin{macrocode}
\newcommand{\childdocforwardprefix}[3][]
{
  \begingroup
    \def\childdocextract #2##1~~~{\def\childdoctmp{\childdocforward[#1]{#3##1}}}
    \expandafter\childdocextract\childdocname~~~
    \expandafter
  \endgroup
  \childdoctmp
}
%    \end{macrocode}

% \macro{\childdoc}
% The deprecated macro |\childdoc| is a legacy version of |\childdocmain|:
%    \begin{macrocode}
\newcommand{\childdoc}{\childdocmain}
%    \end{macrocode}

% \macro{\childdocredirect}
% The deprecated macro |\childdocredirect| is a legacy version
% of |\childdocforward| and |\childdocforwardprefix|:
%    \begin{macrocode}
\newcommand{\childdocredirect}[2][]
{
  \begingroup
    \if?#1?
      \def\childdoctmp{\childdocforward{#2}}
    \else
      \def\childdoctmp{\childdocforwardprefix{#1}{#2}}
    \fi
    \expandafter
  \endgroup
  \childdoctmp
}
%    \end{macrocode}

%\iffalse
%</package>
%\fi
%
\endinput
\childdocforward{cdocsch2}"|
% \end{tabular}
% \end{center}
% Note that the trailing backslash on each first line
% merely continues the input to the second line
% (for convenient cut ant paste).
% Furthermore, the command |latex| can be replaced by any
% of its alternative versions such as |pdflatex|.
%
% %%%%%%%%%%%%%%%%%%%%%%%%%%%%%%%%%%%%%%%%%%%%%%%%%%%%%%%%%%%%%%%%%%%%%%%%%%%%%%
% %%%%%%%%%%%%%%%%%%%%%%%%%%%%%%%%%%%%%%%%%%%%%%%%%%%%%%%%%%%%%%%%%%%%%%%%%%%%%%
% \section{Implementation}
%\iffalse
%<*package>
%\fi
%
% This section describes the definitions file |childdoc.def|.

% The definitions cannot be loaded using |\usepackage| or |\RequirePackage|
% which has a mechanism to prevent loading a style file more than once.
% When loading the definitions by means of |\input|
% multiple instances have to be prevented manually:
%\iffalse
%This code needs to be before the `\ProvidesFile' directive
%which is defined at the beginning of this file.
%Therefore it is also placed there and commented out here.
%</package>
%<*discard>
%\fi
%    \begin{macrocode}
\ifdefined\childdocmain\endinput\fi
%    \end{macrocode}
%\iffalse
%</discard>
%<*package>
%\fi
%
% \macro{\ifchilddoc}
% \macro{\ifchilddocmanual}
% The conditional |\ifchilddoc| tells whether a
% child (true) or main (false) document is being compiled.
% The conditional |\ifchilddocmanual| tells whether
% the |\includeonly| mechanism is used (false) or
% the selection of child files must be performed manually (true).
% The definitions initialise to false:
%    \begin{macrocode}
\newif\ifchilddoc
\newif\ifchilddocmanual
%    \end{macrocode}

% \macro{\childdocname}
% \macro{\childdocjob}
% The macro |\childdocname| stores the name of the main document
% to be compiled. The macro |\childdocjob| stores the name of
% the document on which the \LaTeX{} compiler was originally invoked.
% The content of |\jobname| cannot be compared
% to filenames specified in the source due to different catcodes.
% The following code rescans |\jobname|, stores the result
% in |\childdocname| and saves a copy in |\childdocjob|:
%    \begin{macrocode}
\edef\childdocname{\scantokens\expandafter{\jobname\noexpand}}
\let\childdocjob\childdocname
%    \end{macrocode}

% \macro{\childdocdisable}
% The macro |\childdocdisable| prevents the main file
% from being processed more than once.
% At this stage, the main document command |\childdocmain|
% is assumed to be called once again where it should do nothing.
% Any subsequent call to it should prevent
% a secondary processing of the main document
% It overwrites the forwarding commands
% |\childdocof| and |\childdocforward|
% with empty macros to prevent further inclusions of the main document:
%    \begin{macrocode}
\newcommand{\childdocdisable}
{
  \renewcommand{\childdocmain}[1]{\renewcommand{\childdocmain}[1]{\endinput}}
  \renewcommand{\childdocof}[1]{}
  \renewcommand{\childdocby}[2][]{}
  \renewcommand{\childdocforward}[2][]{}
  \renewcommand{\childdocdisable}{}
}
%    \end{macrocode}

% \macro{\childdocmain}
% The macro |\childdocmain| is to be called at the top of the main file
% with nothing or the main filename (without extension) as argument.
% First, it breaks loops.
% If the argument is not empty and does not match |\childdocname|
% (which is set by the first inclusion of |childdoc.def|),
% |\ifchilddoc| is set to true, |\includeonly| is applied to the child file
% and |\jobname| is set to the main file
% (for proper handling of |.aux| files):
%    \begin{macrocode}
\newcommand{\childdocmain}[1]
{
  \childdocdisable\childdocmain{}
  \if?#1?\else
    \begingroup
      \def\childdoctmp{#1}
      \ifx\childdoctmp\childdocname
        \def\childdoctmp{}
      \else
        \def\childdoctmp
        {
          \childdoctrue
          \includeonly{\childdocname}
          \def\childdocjob{#1}
          \def\jobname{#1}
        }
      \fi
      \expandafter
    \endgroup
    \childdoctmp
  \fi
}
%    \end{macrocode}

% \macro{\childdocof}
% The command |\childdocof| redirects
% compilation to the main file |#1|.
%    \begin{macrocode}
\newcommand{\childdocof}[1]
{
  \childdocdisable
  \childdoctrue
  \includeonly{\childdocname}
  \def\jobname{#1}
  \def\childdocjob{#1}
  \input{#1}
}
%    \end{macrocode}

% \macro{\childdocby}
% The command |\childdocby| ....
%    \begin{macrocode}
\newcommand{\childdocby}[2][]
{
  \childdocdisable
  \childdoctrue
  \childdocmanualtrue
  \if?#1?\else
    \def\jobname{#2}
  \fi
  \def\childdocjob{#2}
  \input{#2}
  \endinput
}
%    \end{macrocode}

% \macro{\childdocforward}
% The command |\childdocforward| redirects
% compilation to the main file or
% (if the optional argument is given) a child file.
% Parameters are set as if the main file
% or a child file starting with |\childdocof| was compiled.
% Then compilation is handed over to the main file:
%    \begin{macrocode}
\newcommand{\childdocforward}[2][]
{
  \begingroup
    \if?#1?
      \def\childdoctmp
      {
        \def\childdocname{#2}
        \def\childdocjob{#2}
        \def\jobname{#2}
        \input{#2}
        \endinput
      }
    \else
      \def\childdoctmp
      {
        \childdocdisable
        \def\childdocname{#2}
        \childdoctrue
        \includeonly{#2}
        \def\childdocjob{#1}
        \def\jobname{#1}
        \input{#1}
        \endinput
      }
    \fi
    \expandafter
  \endgroup
  \childdoctmp
}
%    \end{macrocode}

% \macro{\childdocforwardprefix}
% The command |\childdocforwardprefix| redirects
% compilation to the main or a child file by means of a pattern.
% The prefix |#1| in the current filename is replaced by |#2|
% and the suffix of the current filename is kept
% (it is assumed that the filename does not contain the substring `|~~~|'
% which is used as a delimiter).
% Compilation is handed over to the new file by |\childdocforward|:
%    \begin{macrocode}
\newcommand{\childdocforwardprefix}[3][]
{
  \begingroup
    \def\childdocextract #2##1~~~{\def\childdoctmp{\childdocforward[#1]{#3##1}}}
    \expandafter\childdocextract\childdocname~~~
    \expandafter
  \endgroup
  \childdoctmp
}
%    \end{macrocode}

% \macro{\childdoc}
% The deprecated macro |\childdoc| is a legacy version of |\childdocmain|:
%    \begin{macrocode}
\newcommand{\childdoc}{\childdocmain}
%    \end{macrocode}

% \macro{\childdocredirect}
% The deprecated macro |\childdocredirect| is a legacy version
% of |\childdocforward| and |\childdocforwardprefix|:
%    \begin{macrocode}
\newcommand{\childdocredirect}[2][]
{
  \begingroup
    \if?#1?
      \def\childdoctmp{\childdocforward{#2}}
    \else
      \def\childdoctmp{\childdocforwardprefix{#1}{#2}}
    \fi
    \expandafter
  \endgroup
  \childdoctmp
}
%    \end{macrocode}

%\iffalse
%</package>
%\fi
%
\endinput
|\\
|\childdocmain{|\textit{main}|}|\\
\end{tabular}
\end{center}
%
If |\jobname| does not match the argument \textit{main} of |\childdocmain|,
it is assumed that |\jobname| points to the child file to be compiled.
When using |\childdocmain| with the main file specified as argument,
it suffices to start a child file
with just |\input{|\textit{main}|}|
without loading of the package and using |\childdocof|.
If instead all processing is done
with the appropriate \textsf{childdoc} directives,
the argument of \textit{main} of |\childdocmain| can be empty.

An alternative version of the command line processing described
in \secref{sec:commandline} using the detection mechanism reads:
%
\begin{center}
|... -jobname "|\textit{target}|" "|[\textit{flags}]%
[|\def\jobname{|\textit{dest}|}|]|\input{|\textit{main}|}"|
\end{center}

%%%%%%%%%%%%%%%%%%%%%%%%%%%%%%%%%%%%%%%%%%%%%%%%%%%%%%%%%%%%%%%%%%%%%%%%%%%%%%%%
\subsection{Manual Code}
\label{sec:manual}

In case one cannot be certain whether the definitions file |childdoc.def|
is installed on the target \TeX{} distribution
and one prefers not to ship it,
it is conceivable to paste a few relevant commands into the sources.

To that end, drop all statements |% \iffalse
%
% childdoc.dtx Copyright (C) 2017-2018 Niklas Beisert
%
% This work may be distributed and/or modified under the
% conditions of the LaTeX Project Public License, either version 1.3
% of this license or (at your option) any later version.
% The latest version of this license is in
%   http://www.latex-project.org/lppl.txt
% and version 1.3 or later is part of all distributions of LaTeX
% version 2005/12/01 or later.
%
% This work has the LPPL maintenance status `maintained'.
%
% The Current Maintainer of this work is Niklas Beisert.
%
% This work consists of the files childdoc.dtx and childdoc.ins
% and the derived files childdoc.def and cdocsamp.tex with
% cdocsch1.tex, cdocsch2.tex, cdocsdrf.tex, cdocsfn1.tex, cdocsfn2.tex.
%
%<package>\ifdefined\childdocmain\endinput\fi
%<package>\ProvidesFile{childdoc.def}[2018/12/30 v2.0 child document driver]
%<samplemain>\ProvidesFile{cdocsamp.tex}[2018/12/30 v2.0 sample for childdoc]
%<*driver>
%\ProvidesFile{childdoc.drv}[2018/12/30 v2.0 childdoc reference manual file]
\PassOptionsToClass{10pt,a4paper}{article}
\documentclass{ltxdoc}

\usepackage[margin=35mm]{geometry}
\usepackage{hyperref}
\usepackage{hyperxmp}
\usepackage[usenames]{color}

\hypersetup{colorlinks=true}
\hypersetup{pdfstartview=FitH}
\hypersetup{pdfpagemode=UseNone}
\hypersetup{pdfsource={}}
\hypersetup{pdflang={en-UK}}
\hypersetup{pdfcopyright={Copyright 2017-2018 Niklas Beisert.
  This work may be distributed and/or modified under the
  conditions of the LaTeX Project Public License, either version 1.3
  of this license or (at your option) any later version.}}
\hypersetup{pdflicenseurl={http://www.latex-project.org/lppl.txt}}
\hypersetup{pdfcontactaddress={ETH Zurich, ITP, HIT K,
  Wolfgang-Pauli-Strasse 27}}
\hypersetup{pdfcontactpostcode={8093}}
\hypersetup{pdfcontactcity={Zurich}}
\hypersetup{pdfcontactcountry={Switzerland}}
\hypersetup{pdfcontactemail={nbeisert@itp.phys.ethz.ch}}
\hypersetup{pdfcontacturl={http://people.phys.ethz.ch/\xmptilde nbeisert/}}

\newcommand{\secref}[1]{\hyperref[#1]{section \ref*{#1}}}

\parskip1ex
\parindent0pt
\let\olditemize\itemize
\def\itemize{\olditemize\parskip0pt}

\begin{document}

\title{The \textsf{childdoc} Package}
\hypersetup{pdftitle={The childdoc Package}}
\author{Niklas Beisert\\[2ex]
  Institut f\"ur Theoretische Physik\\
  Eidgen\"ossische Technische Hochschule Z\"urich\\
  Wolfgang-Pauli-Strasse 27, 8093 Z\"urich, Switzerland\\[1ex]
  \href{mailto:nbeisert@itp.phys.ethz.ch}
  {\texttt{nbeisert@itp.phys.ethz.ch}}}
\hypersetup{pdfauthor={Niklas Beisert}}
\hypersetup{pdfsubject={Manual for the LaTeX2e Package childdoc}}
\date{30 December 2018, \textsf{v2.0}}
\maketitle

\begin{abstract}\noindent
\textsf{childdoc} is a \LaTeXe{} package
that enables the direct compilation
of document sections included by |\include|
to individual files.
\end{abstract}

\begingroup
\parskip0ex
\tableofcontents
\endgroup

%%%%%%%%%%%%%%%%%%%%%%%%%%%%%%%%%%%%%%%%%%%%%%%%%%%%%%%%%%%%%%%%%%%%%%%%%%%%%%%%
%%%%%%%%%%%%%%%%%%%%%%%%%%%%%%%%%%%%%%%%%%%%%%%%%%%%%%%%%%%%%%%%%%%%%%%%%%%%%%%%
\section{Introduction}

\LaTeX{} provides a mechanism to structure a large document (such as a book)
into a main file and several child files (containing the chapters)
using the |\include| command.
This mechanism is beneficial for documents
which span hundreds of pages in order to
make the source file(s) more manageable.
Moreover, compilation can be restricted to
selected child files by means of the |\includeonly| command.
The latter feature can be used to reduce the compilation time while editing
(this was significantly more useful in the earlier days of \LaTeX{})
or to generate a smaller document which is easier to navigate.
Another application of |\includeonly| is to generate
documents consisting of selected parts of the complete document.

However, there are a few drawbacks of the plain |\include| mechanism:
\begin{itemize}
\item
The child files cannot be compiled on their own,
they can only be compiled via the main file.
A naive editing environment
(such as a text editor with an option
to have the current file processed by \LaTeX)
may require one to switch to the main file before compiling;
attempting to compile the child file produces errors.
\item
The main file must be modified (each time)
to adjust the |\includeonly| command
to the present needs. This easily leaves the main file in a messy state.
\item
The generated document will always carry the filename
of the main document. This is inconvenient if
several child files are to be compiled and
to be kept for distribution.
\end{itemize}

The present package provides a simple interface
to make child files individually compilable by \LaTeX{}.
Compiling a child file then has the same effect as compiling
the main file with an |\includeonly| command
to select the appropriate child.
Moreover the generated document will carry the name of the child
rather than the main file.
This resolves all three above issues.

This feature is meant to make the editing of books,
thesis documents and lecture notes somewhat more convenient.
However, the package can also be used efficiently for
composing a series of documents (such as exercise sheets)
which are typically distributed individually.
It then assists the author in generating the individual documents
(potentially in different versions)
as well as a document containing the collected series.
Another application is in developing style files
or other kinds of included material
where compilation of the style file could redirect
to a sample or test file.

%%%%%%%%%%%%%%%%%%%%%%%%%%%%%%%%%%%%%%%%%%%%%%%%%%%%%%%%%%%%%%%%%%%%%%%%%%%%%%%%
%%%%%%%%%%%%%%%%%%%%%%%%%%%%%%%%%%%%%%%%%%%%%%%%%%%%%%%%%%%%%%%%%%%%%%%%%%%%%%%%
\section{Usage}

First of all, the package \textsf{childdoc} is \emph{not} a standard
\LaTeXe{} |.sty| style file! Therefore it needs to be invoked in
a non-standard way.

%%%%%%%%%%%%%%%%%%%%%%%%%%%%%%%%%%%%%%%%%%%%%%%%%%%%%%%%%%%%%%%%%%%%%%%%%%%%%%%%
\subsection{Included Files}
\label{sec:include}

%%%%%%%%%%%%%%%%%%%%%%%%%%%%%%%%%%%%%%%%
\DescribeMacro{\childdocmain}
To use the package, add the commands
\begin{center}
\begin{tabular}{l}
|% \iffalse
%
% childdoc.dtx Copyright (C) 2017-2018 Niklas Beisert
%
% This work may be distributed and/or modified under the
% conditions of the LaTeX Project Public License, either version 1.3
% of this license or (at your option) any later version.
% The latest version of this license is in
%   http://www.latex-project.org/lppl.txt
% and version 1.3 or later is part of all distributions of LaTeX
% version 2005/12/01 or later.
%
% This work has the LPPL maintenance status `maintained'.
%
% The Current Maintainer of this work is Niklas Beisert.
%
% This work consists of the files childdoc.dtx and childdoc.ins
% and the derived files childdoc.def and cdocsamp.tex with
% cdocsch1.tex, cdocsch2.tex, cdocsdrf.tex, cdocsfn1.tex, cdocsfn2.tex.
%
%<package>\ifdefined\childdocmain\endinput\fi
%<package>\ProvidesFile{childdoc.def}[2018/12/30 v2.0 child document driver]
%<samplemain>\ProvidesFile{cdocsamp.tex}[2018/12/30 v2.0 sample for childdoc]
%<*driver>
%\ProvidesFile{childdoc.drv}[2018/12/30 v2.0 childdoc reference manual file]
\PassOptionsToClass{10pt,a4paper}{article}
\documentclass{ltxdoc}

\usepackage[margin=35mm]{geometry}
\usepackage{hyperref}
\usepackage{hyperxmp}
\usepackage[usenames]{color}

\hypersetup{colorlinks=true}
\hypersetup{pdfstartview=FitH}
\hypersetup{pdfpagemode=UseNone}
\hypersetup{pdfsource={}}
\hypersetup{pdflang={en-UK}}
\hypersetup{pdfcopyright={Copyright 2017-2018 Niklas Beisert.
  This work may be distributed and/or modified under the
  conditions of the LaTeX Project Public License, either version 1.3
  of this license or (at your option) any later version.}}
\hypersetup{pdflicenseurl={http://www.latex-project.org/lppl.txt}}
\hypersetup{pdfcontactaddress={ETH Zurich, ITP, HIT K,
  Wolfgang-Pauli-Strasse 27}}
\hypersetup{pdfcontactpostcode={8093}}
\hypersetup{pdfcontactcity={Zurich}}
\hypersetup{pdfcontactcountry={Switzerland}}
\hypersetup{pdfcontactemail={nbeisert@itp.phys.ethz.ch}}
\hypersetup{pdfcontacturl={http://people.phys.ethz.ch/\xmptilde nbeisert/}}

\newcommand{\secref}[1]{\hyperref[#1]{section \ref*{#1}}}

\parskip1ex
\parindent0pt
\let\olditemize\itemize
\def\itemize{\olditemize\parskip0pt}

\begin{document}

\title{The \textsf{childdoc} Package}
\hypersetup{pdftitle={The childdoc Package}}
\author{Niklas Beisert\\[2ex]
  Institut f\"ur Theoretische Physik\\
  Eidgen\"ossische Technische Hochschule Z\"urich\\
  Wolfgang-Pauli-Strasse 27, 8093 Z\"urich, Switzerland\\[1ex]
  \href{mailto:nbeisert@itp.phys.ethz.ch}
  {\texttt{nbeisert@itp.phys.ethz.ch}}}
\hypersetup{pdfauthor={Niklas Beisert}}
\hypersetup{pdfsubject={Manual for the LaTeX2e Package childdoc}}
\date{30 December 2018, \textsf{v2.0}}
\maketitle

\begin{abstract}\noindent
\textsf{childdoc} is a \LaTeXe{} package
that enables the direct compilation
of document sections included by |\include|
to individual files.
\end{abstract}

\begingroup
\parskip0ex
\tableofcontents
\endgroup

%%%%%%%%%%%%%%%%%%%%%%%%%%%%%%%%%%%%%%%%%%%%%%%%%%%%%%%%%%%%%%%%%%%%%%%%%%%%%%%%
%%%%%%%%%%%%%%%%%%%%%%%%%%%%%%%%%%%%%%%%%%%%%%%%%%%%%%%%%%%%%%%%%%%%%%%%%%%%%%%%
\section{Introduction}

\LaTeX{} provides a mechanism to structure a large document (such as a book)
into a main file and several child files (containing the chapters)
using the |\include| command.
This mechanism is beneficial for documents
which span hundreds of pages in order to
make the source file(s) more manageable.
Moreover, compilation can be restricted to
selected child files by means of the |\includeonly| command.
The latter feature can be used to reduce the compilation time while editing
(this was significantly more useful in the earlier days of \LaTeX{})
or to generate a smaller document which is easier to navigate.
Another application of |\includeonly| is to generate
documents consisting of selected parts of the complete document.

However, there are a few drawbacks of the plain |\include| mechanism:
\begin{itemize}
\item
The child files cannot be compiled on their own,
they can only be compiled via the main file.
A naive editing environment
(such as a text editor with an option
to have the current file processed by \LaTeX)
may require one to switch to the main file before compiling;
attempting to compile the child file produces errors.
\item
The main file must be modified (each time)
to adjust the |\includeonly| command
to the present needs. This easily leaves the main file in a messy state.
\item
The generated document will always carry the filename
of the main document. This is inconvenient if
several child files are to be compiled and
to be kept for distribution.
\end{itemize}

The present package provides a simple interface
to make child files individually compilable by \LaTeX{}.
Compiling a child file then has the same effect as compiling
the main file with an |\includeonly| command
to select the appropriate child.
Moreover the generated document will carry the name of the child
rather than the main file.
This resolves all three above issues.

This feature is meant to make the editing of books,
thesis documents and lecture notes somewhat more convenient.
However, the package can also be used efficiently for
composing a series of documents (such as exercise sheets)
which are typically distributed individually.
It then assists the author in generating the individual documents
(potentially in different versions)
as well as a document containing the collected series.
Another application is in developing style files
or other kinds of included material
where compilation of the style file could redirect
to a sample or test file.

%%%%%%%%%%%%%%%%%%%%%%%%%%%%%%%%%%%%%%%%%%%%%%%%%%%%%%%%%%%%%%%%%%%%%%%%%%%%%%%%
%%%%%%%%%%%%%%%%%%%%%%%%%%%%%%%%%%%%%%%%%%%%%%%%%%%%%%%%%%%%%%%%%%%%%%%%%%%%%%%%
\section{Usage}

First of all, the package \textsf{childdoc} is \emph{not} a standard
\LaTeXe{} |.sty| style file! Therefore it needs to be invoked in
a non-standard way.

%%%%%%%%%%%%%%%%%%%%%%%%%%%%%%%%%%%%%%%%%%%%%%%%%%%%%%%%%%%%%%%%%%%%%%%%%%%%%%%%
\subsection{Included Files}
\label{sec:include}

%%%%%%%%%%%%%%%%%%%%%%%%%%%%%%%%%%%%%%%%
\DescribeMacro{\childdocmain}
To use the package, add the commands
\begin{center}
\begin{tabular}{l}
|\input{childdoc.def}|\\
|\childdocmain{}|\\
\end{tabular}
\end{center}
at the very top of the main \LaTeX{} file,
in particular \emph{before} the |\documentclass| statement!
The argument of |\childdocmain| should be left empty
(but it must be present).

%%%%%%%%%%%%%%%%%%%%%%%%%%%%%%%%%%%%%%%%
\DescribeMacro{\childdocof}
Furthermore, add the commands
\begin{center}
\begin{tabular}{l}
|\input{childdoc.def}|\\
|\childdocof{|\textit{main}|}|\\
\end{tabular}
\end{center}
at the top of every child file \textit{child}
which is included by |\include{|\textit{child}|}|
from within the main file
(or at least for those files to be compiled individually).
The argument \textit{main} must be the filename of the main file.

There are a couple of
considerations in setting up the main and child documents:

%%%%%%%%%%%%%%%%%%%%%%%%%%%%%%%%%%%%%%%%
\paragraph{Restrictions.}

Please note the following restrictions:
\begin{itemize}
\item
|\childdocmain| must be called with one argument \textit{main}
to ensure compatibility with earlier version of the package.
It must either be empty (|\childdocmain{}|)
or precisely match the filename of the main file in which it is specified.
See \secref{sec:detection} for further information.
\item
The filename \textit{main} must be specified without the |.tex| extension.
\item
The filename \textit{main} is case sensitive
(even in case-insensitive file systems)
due to internal string comparison.
\item
The argument \textit{main} should be fully expanded, it cannot be a macro.
\item
Subdirectories and special characters should be avoided in filenames.
\item
The command |\childdocmain{|\textit{main}|}| must be followed by a whitespace.
It should not be followed immediately by another command
or by a comment mark `|%|'.
This is because the \TeX{} parser reads the token immediately following
the argument of |\childdocmain| and puts it
at the beginning of every child section;
however, a white\-space is ignored.
\end{itemize}

%%%%%%%%%%%%%%%%%%%%%%%%%%%%%%%%%%%%%%%%
\paragraph{Content of Main File.}

It is advisable to place all content in the child files included by |\include|.
Any output contained in the main file will appear in all child documents
unless suppressed manually;
it cannot be suppressed automatically by the |\includeonly| directive
and thus should normally be avoided.
A method to include some content in the main file
by means of conditional processing is described in \secref{sec:conditional}.

%%%%%%%%%%%%%%%%%%%%%%%%%%%%%%%%%%%%%%%%
\paragraph{Page Numbering.}

When only a part of the document is compiled,
the appropriate numbering of pages
(as well as other status parameters)
is determined from the |.aux| files.
The latter contain information from previous passes.
However this information needs to propagate through
all intermediate child documents.
Therefore the page numbering in child documents may well
be inconsistent until the complete document is compiled at least once.

A useful (if unconventional) way to always ensure a consistent
page numbering is to restart the numbering in each child document
and denote the pages by `\textit{child}|.|\textit{page}'
where \textit{child} represents the chapter/section number of the child file.
This can be achieved by the command
|\numberwithin{page}{|\textit{child}|}|
of the \textsf{amsmath} package
where \textit{child} can be |chapter| or |section|
depending on the chosen structuring.
Alternatively, one can modify the macro |\thepage| appropriately
and reset the counter |page| at the start of each child file.

%%%%%%%%%%%%%%%%%%%%%%%%%%%%%%%%%%%%%%%%%%%%%%%%%%%%%%%%%%%%%%%%%%%%%%%%%%%%%%%%
\subsection{Conditional Processing}
\label{sec:conditional}

The package provides a mechanism to compile different versions
of a document. To customise the versions further some conditional processing
can come in handy to distinguish which version is being compiled.
The package provides two macros to describe the compilation context:

%%%%%%%%%%%%%%%%%%%%%%%%%%%%%%%%%%%%%%%%
\DescribeMacro{\ifchilddoc}
The conditional |\ifchilddoc| distinguishes between the compilation of
child documents and the main document:
%
\begin{center}
|\ifchilddoc |\textit{child-code}| |[|\||else |\textit{main-code}]| \||fi|
\end{center}

%%%%%%%%%%%%%%%%%%%%%%%%%%%%%%%%%%%%%%%%
\DescribeMacro{\childdocname}
\DescribeMacro{\childdocjob}
The macro |\childdocname| contains the filename (without extension)
of the main or child file being processed.
Note that |\childdocjob| will always contain the name of the main file.

%%%%%%%%%%%%%%%%%%%%%%%%%%%%%%%%%%%%%%%%
\paragraph{Title Page.}

Conditional processing can be used to include a title or banner page
in the main document when proper precautions are taken.
Importantly, the code in the main file should ensure that the page counter
(as well as other status parameters which are stored in the |.aux| files)
takes the same value after the conditional processing.
Otherwise the page numbers may take divergent values
depending on which part is compiled.

For example, a title page could be declared by:
%
\begin{center}
\begin{tabular}{l}
|\ifchilddoc\||else|\\
|\addtocounter{page}{-1}|\\
\textit{code for title page}\\
|\newpage|\\
|\||fi|
\end{tabular}
\end{center}
%
A banner page for the child documents can be generated by:
%
\begin{center}
\begin{tabular}{l}
|\ifchilddoc|\\
|\addtocounter{page}{-1}|\\
\textit{code for banner page}\\
|\newpage|\\
|\||fi|
\end{tabular}
\end{center}
%
Here one could write a message such as:
\begin{center}
|This is the part \childdocname{} of \childdocjob{}.|
\end{center}

%%%%%%%%%%%%%%%%%%%%%%%%%%%%%%%%%%%%%%%%%%%%%%%%%%%%%%%%%%%%%%%%%%%%%%%%%%%%%%%%
\subsection{Flags}
\label{sec:flags}

The package makes it easy to generate different versions
of the main or child documents.
To this end compilation flags can be defined
and assigned different default values.
They will be particularly useful in conjunction
with the forwarding mechanism described in \secref{sec:forward}.

For example, it may be useful to have a flag |\version|
which can be set to |draft| or |final|.
The document source will contain some conditional code
depending on the value of |\version|.
Suppose further, the flag should default to |final| for the main file
and to |draft| for child files
which is a natural assignment for editing the document.
This is achieved by placing the following code
in the preamble of the main document
(below the |\childdocmain| directive):
%
\begin{center}
\begin{tabular}{l}
|\ifchilddoc|\\
|\providecommand{\version}{draft}|\\
|\||else|\\
|\providecommand{\version}{final}|\\
|\||fi|
\end{tabular}
\end{center}
%
The definition by |\providecommand| makes sure
that previous definitions are not overwritten.
Further statements |\providecommand{\version}{...}|
can thus be added before the above code to override it.

For the main file, one might add a line
(between |\childdocmain| and the above block)
%
\begin{center}
|%\ifchilddoc\||else\providecommand{\version}{draft}\||fi|
\end{center}
%
which can be uncommented to produce a draft version.
Likewise one can add a line to the very top of a child file
(above the |\childdocof{|\textit{main}|}| directive)
%
\begin{center}
|%\providecommand{\version}{final}|
\end{center}
%
which can be uncommented to produce the final version of this child document.

%%%%%%%%%%%%%%%%%%%%%%%%%%%%%%%%%%%%%%%%%%%%%%%%%%%%%%%%%%%%%%%%%%%%%%%%%%%%%%%%
\subsection{Forwarding}
\label{sec:forward}

Different versions of the main or child documents
using compilation flags as described in \secref{sec:flags}
can be (permanently) stored in different files
for convenient compilation, viewing and distribution.
To this end, the package defines a command
to pass on compilation to a different file:

%%%%%%%%%%%%%%%%%%%%%%%%%%%%%%%%%%%%%%%%
\DescribeMacro{\childdocforward}
The command |\childdocforward| redirects processing to
another source file:
%
\begin{center}
\begin{tabular}{l}
|\input{childdoc.def}|\\
|\childdocforward[|\textit{main}|]{|\textit{dest}|}|\\
\end{tabular}
\end{center}
%
The argument \textit{dest} is the destination file
(without extension).
It should be the main file or one of the child files.
Note that further \textsf{childdoc} directives
such as |\childdocof| and |\childdocforward|
in the indicated file will be processed in this form.
The optional argument \textit{main}
passes on directly to the main file \textit{main}
while pretending to compile the child \textit{dest}.
This form behaves as if \textit{dest}
issues |\childdocof{|\textit{main}|}| right away,
and no further \textsf{childdoc} directives will be processed.

%%%%%%%%%%%%%%%%%%%%%%%%%%%%%%%%%%%%%%%%
\DescribeMacro{\...prefix}
In the alternative form |\childdocforwardprefix|,
%
\begin{center}
\begin{tabular}{l}
|\input{childdoc.def}|\\
|\childdocforwardprefix[|\textit{main}|]{|\textit{prefix}|}{|\textit{dest}|}|
\end{tabular}
\end{center}
%
the destination file is determined by a pattern
depending on the current file:
To make this work, the current file must be called
`{\textit{prefix}\hspace{0.2em}\textit{suffix}}'
with \textit{prefix} matching precisely the argument.
Processing is then passed on to the file
`{\textit{dest}\hspace{0.2em}\textit{suffix}}'.
Surely, the same effect is achieved by
directly specifying the
argument `{\textit{dest}\hspace{0.2em}\textit{suffix}}'
in the first form.
However, that requires to set up a different file
for each child. With the alternative form of the command
all these files can have exactly the same content
which simplifies setting them up and maintaining them.

For example, the following file |draft.tex|
with a compilation flag |\version| as described in \secref{sec:flags}
compiles the main document as a draft:
%
\begin{center}
\begin{tabular}{l}
|\def\version{draft}|\\
|\input{childdoc.def}|\\
|\childdocforward{|\textit{main}|}|
\end{tabular}
\end{center}
%
Likewise, the following files |final|\textit{nn}|.tex|
compile the final version of the child document
|child|\textit{nn}|.tex|:
%
\begin{center}
\begin{tabular}{l}
|\def\version{final}|\\
|\input{childdoc.def}|\\
|\childdocforwardprefix{final}{child}|
\end{tabular}
\end{center}
%

Note that when several versions of a main file and/or of each child file
are to be generated, it may be convenient to set up a |Makefile| or
shell script to automatise the process.

%%%%%%%%%%%%%%%%%%%%%%%%%%%%%%%%%%%%%%%%%%%%%%%%%%%%%%%%%%%%%%%%%%%%%%%%%%%%%%%%
\subsection{Command Line Processing}
\label{sec:commandline}

The effect of redirection files can also be achieved by invoking
the \LaTeX{} compiler with a more elaborate command line.
Most conveniently this should be done as part
of a shell script or a |Makefile|.

When using \textsf{childdoc} in the main file, the following
command lines effectively perform a redirection
(note that depending on the shell being used,
backslashes may have to be doubled: `|\|' $\to$ `|\\|'):
%
\begin{center}
|... -jobname "|\textit{target}|" |\\|"|[\textit{flags}]%
|\input{childdoc.def}\childdocforward[|\textit{main}|]{|\textit{dest}|}"|
\end{center}
%
Here \textit{target} is the name of the output file,
\textit{main} is the name of the main file
and \textit{dest} is the name of the main or child file to be processed
(all filenames without extensions).
The optional argument \textit{main} can be omitted
if \textit{main} matches \textit{dest}.
Optionally, compilation \textit{flags} can be defined via |\def| commands.
This command line makes the \TeX{} engine believe
it is compiling the file \textit{target}
whose content is specified as the latter parameter.
The provided code then forwards the processing to
\textit{main} or \textit{dest} as described in \secref{sec:forward}.

%%%%%%%%%%%%%%%%%%%%%%%%%%%%%%%%%%%%%%%%%%%%%%%%%%%%%%%%%%%%%%%%%%%%%%%%%%%%%%%%
\subsection{Include by Input}
\label{sec:input}

Including child documents by |\include| has some restrictions by design.
Most notably, the content of a child document always occupies
its own set of pages; pages cannot be shared between child documents.
Usually, this behaviour makes perfect sense
because each child document contain an essential part of the document.
However, in some situations it may be desirable to compose
a document from a collection of parts
without having mandatory page breaks between then.
For this case, the package
provides a mechanism to include parts
by |\input| which can also be processed individually.
However, by construction this mechanism
requires manual handling of the content to be output.

%%%%%%%%%%%%%%%%%%%%%%%%%%%%%%%%%%%%%%%%
\DescribeMacro{\ifchilddocmanual}
The main file should be prepared as usual, see \secref{sec:include}.
However, the document body must make a distinction
between processing of an individual part and of the main document, e.g.:
%
\begin{center}
\begin{tabular}{l}
|\ifchilddocmanual|\\
|\input{\childdocname}|\\
|\||else|\\
\textit{document body with }|\input{|\textit{part}|}|\\
|\||fi|
\end{tabular}
\end{center}
%
The conditional |\ifchilddocmanual| is true whenever
a part to be included by |\input| is being compiled,
and the name of the part is stored in |\childdocname|.

%%%%%%%%%%%%%%%%%%%%%%%%%%%%%%%%%%%%%%%%
\DescribeMacro{\childdocby}
Each part to be included by |\input| should start with:
%
\begin{center}
\begin{tabular}{l}
|\input{childdoc.def}|\\
|\childdocby{|\textit{main}|}|\\
\end{tabular}
\end{center}
%
The directive |\childdocby| is similar to |\childdocof|
described in \secref{sec:include},
but the subsequent selection of content must be done manually.
To that end, both |\ifchilddoc| and |\ifchilddocmanual|
will be true upon processing of a part,
and the name of the part is stored in |\childdocname|.
Note that |\jobname| will be set to the filename of the current part
so that each part receives an individual |.aux| file
that does not interfere with the |.aux| file(s) of the main document.
This behaviour can be altered by the alternative form
|\childdocby[*]{|\textit{main}|}| (with a non-empty optional argument)
which uses the |.aux| file of the main document
by setting |\jobname| to \textit{main}.

%%%%%%%%%%%%%%%%%%%%%%%%%%%%%%%%%%%%%%%%%%%%%%%%%%%%%%%%%%%%%%%%%%%%%%%%%%%%%%%%
\subsection{Driver Development}
\label{sec:driver}

The \textsf{childdoc} mechanism can also be use for the development
of definition files such as \LaTeX{} styles or classes.
This case differs from the above setup with multiple parts
included by |\include| in that no |\includeonly| should be invoked.
This can be achieved by starting the include file
(before |\ProvidesPackage|) with:
%
\begin{center}
\begin{tabular}{l}
|\input{childdoc.def}|\\
|\childdocforward{|\textit{main}|}|\\
\end{tabular}
\end{center}
%
or alternatively with:
%
\begin{center}
\begin{tabular}{l}
|\input{childdoc.def}|\\
|\childdocby{|\textit{main}|}|\\
\end{tabular}
\end{center}
%
Both forms have slightly different effects as described above.
The main file is prepared as usual, see \secref{sec:include}.

%%%%%%%%%%%%%%%%%%%%%%%%%%%%%%%%%%%%%%%%%%%%%%%%%%%%%%%%%%%%%%%%%%%%%%%%%%%%%%%%
\subsection{Legacy Detection}
\label{sec:detection}

The directive |\childdocmain| in the main file can detect
whether the complete document or merely a child is to be compiled
even without using the directive |\childdocof|.
This method is deprecated because it is less robust
and there is no compelling reason to use it;
it is merely provided for backward compatibility
and it may be removed in future versions.

If the detection mechanism is to be used,
it is mandatory to correctly specify
the filename of the main file as the argument of |\childdocmain|:
%
\begin{center}
\begin{tabular}{l}
|\input{childdoc.def}|\\
|\childdocmain{|\textit{main}|}|\\
\end{tabular}
\end{center}
%
If |\jobname| does not match the argument \textit{main} of |\childdocmain|,
it is assumed that |\jobname| points to the child file to be compiled.
When using |\childdocmain| with the main file specified as argument,
it suffices to start a child file
with just |\input{|\textit{main}|}|
without loading of the package and using |\childdocof|.
If instead all processing is done
with the appropriate \textsf{childdoc} directives,
the argument of \textit{main} of |\childdocmain| can be empty.

An alternative version of the command line processing described
in \secref{sec:commandline} using the detection mechanism reads:
%
\begin{center}
|... -jobname "|\textit{target}|" "|[\textit{flags}]%
[|\def\jobname{|\textit{dest}|}|]|\input{|\textit{main}|}"|
\end{center}

%%%%%%%%%%%%%%%%%%%%%%%%%%%%%%%%%%%%%%%%%%%%%%%%%%%%%%%%%%%%%%%%%%%%%%%%%%%%%%%%
\subsection{Manual Code}
\label{sec:manual}

In case one cannot be certain whether the definitions file |childdoc.def|
is installed on the target \TeX{} distribution
and one prefers not to ship it,
it is conceivable to paste a few relevant commands into the sources.

To that end, drop all statements |\input{childdoc.def}|
and perform the replacements as outlined below.
Instead of |\childdocmain{|\textit{main}|}| add the following code
to the top of the main file:
%
\begin{center}
\begin{tabular}{l}
|\||ifdefined\childdocname\endinput\||fi\newif\ifchilddoc|\\
|\edef\childdocname{\scantokens\expandafter{\jobname\noexpand}}|\\
|\def\childdocmain{|\textit{main}|}\||ifx\childdocmain\childdocname\||else|\\
|\childdoctrue\includeonly{\childdocname}\let\jobname\childdocmain\||fi|\\
\end{tabular}
\end{center}
%
Instead of |\childdocof{|\textit{main}|}| just include the main file
at the top of each child file:
%
\begin{center}
|\input{|\textit{main}|}|
\end{center}
%
A simple redirection |\childdocforward{|\textit{dest}|}| is achieved by:
%
\begin{center}
|\def\jobname{|\textit{dest}|}\input{\jobname}|
\end{center}
%
The redirection with prefix
|\childdocforwardprefix[|\textit{prefix}|]{|\textit{dest}|}|
is accomplished by:
%
\begin{center}
\begin{tabular}{l}
|{\edef\jobname{\scantokens\expandafter{\jobname\noexpand}}|\\
|\def\redirectjob |\textit{prefix}|#1~~~{\gdef\jobname{|\textit{dest}|#1}}|\\
|\expandafter\redirectjob\jobname~~~}\input{\jobname}|
\end{tabular}
\end{center}

In an alternative approach,
child documents can be compiled by a specific command line
without additional code or specific definitions:
%
\begin{center}
|... -jobname "|\textit{target}|" "|[\textit{flags}]%
|\includeonly{|\textit{dest}|}\input{|\textit{main}|}"|
\end{center}
%

%%%%%%%%%%%%%%%%%%%%%%%%%%%%%%%%%%%%%%%%%%%%%%%%%%%%%%%%%%%%%%%%%%%%%%%%%%%%%%%%
%%%%%%%%%%%%%%%%%%%%%%%%%%%%%%%%%%%%%%%%%%%%%%%%%%%%%%%%%%%%%%%%%%%%%%%%%%%%%%%%
\section{Information}

%%%%%%%%%%%%%%%%%%%%%%%%%%%%%%%%%%%%%%%%%%%%%%%%%%%%%%%%%%%%%%%%%%%%%%%%%%%%%%%%
\subsection{Copyright}

Copyright \copyright{} 2017--2018 Niklas Beisert

This work may be distributed and/or modified under the
conditions of the \LaTeX{} Project Public License, either version 1.3
of this license or (at your option) any later version.
The latest version of this license is in
  \url{http://www.latex-project.org/lppl.txt}
and version 1.3 or later is part of all distributions of \LaTeX{}
version 2005/12/01 or later.

This work has the LPPL maintenance status `maintained'.

The Current Maintainer of this work is Niklas Beisert.

This work consists of the files |README.txt|, |childdoc.ins| and |childdoc.dtx|
as well as the derived files |childdoc.def|, |cdocsamp.tex|
with |cdocsch1.tex|, |cdocsch2.tex|, |cdocspt3.tex|, |cdocspt4.tex|,
|cdocsdrf.tex|, |cdocsfn1.tex|, |cdocsfn2.tex|
as well as |childdoc.pdf|.

%%%%%%%%%%%%%%%%%%%%%%%%%%%%%%%%%%%%%%%%%%%%%%%%%%%%%%%%%%%%%%%%%%%%%%%%%%%%%%%%
\subsection{Files and Installation}

The package consists of the files:
%
\begin{center}
\begin{tabular}{ll}
    |README.txt|   & readme file \\
    |childdoc.ins| & installation file \\
    |childdoc.dtx| & source file \\
    |childdoc.def| & definition file \\
    |cdocsamp.tex| & sample main file \\
    |cdocsch1.tex| & sample include file \\
    |cdocsch2.tex| & sample include file \\
    |cdocspt3.tex| & sample part file \\
    |cdocspt4.tex| & sample part file \\
    |cdocsdrf.tex| & sample redirection file \\
    |cdocsfn1.tex| & sample redirection file \\
    |cdocsfn2.tex| & sample redirection file \\
    |childdoc.pdf| & manual
\end{tabular}
\end{center}
%
The distribution consists of the files
|README.txt|, |childdoc.ins| and |childdoc.dtx|.
%
\begin{itemize}
\item
Run (pdf)\LaTeX{} on |childdoc.dtx|
to compile the manual |childdoc.pdf| (this file).
\item
Run \LaTeX{} on |childdoc.ins| to create the definitions file |childdoc.def|
and the sample |cdocsamp.tex| with include files
|cdocsch1.tex|, |cdocsch2.tex|, |cdocspt3.tex|, |cdocspt4.tex|,
|cdocsdrf.tex|, |cdocsfn1.tex|, |cdocsfn2.tex|.
Then copy the file |childdoc.def| to an appropriate directory of your \LaTeX{}
distribution, e.g.\ \textit{texmf-root}|/tex/latex/childdoc|.
\end{itemize}

%%%%%%%%%%%%%%%%%%%%%%%%%%%%%%%%%%%%%%%%%%%%%%%%%%%%%%%%%%%%%%%%%%%%%%%%%%%%%%%%
\subsection{Related CTAN Packages}

There are several other packages which offer a similar functionality:
%
\begin{itemize}
\item
The packages
\href{http://ctan.org/pkg/docmute}{\textsf{docmute}},
\href{http://ctan.org/pkg/includex}{\textsf{includex}} and
\href{http://ctan.org/pkg/standalone}{\textsf{standalone}}
provide commands to include only the document body of
a child file thus allowing both files to be compiled individually.
\item
The packages \href{http://ctan.org/pkg/subdocs}{\textsf{subdocs}}
and \href{http://ctan.org/pkg/subfiles}{\textsf{subfiles}}
provide structures in which the main and child documents can be
encapsulated and allowing them to be compiled individually.
The inclusion mechanism is different from the conventional |\include|.
\item
The package \href{http://ctan.org/pkg/combine}{\textsf{combine}}
is an elaborate solution to combine several documents into one.
\end{itemize}
%
See also the CTAN topic \href{http://ctan.org/topic/subdocs}{\textsf{subdocs}}
for further related packages.
The present package differs from the above solutions in that
a document structure constructed with the conventional |\include| mechanism
just needs two extra commands at the top of every file
such that all constituent files can be compiled individually.

%%%%%%%%%%%%%%%%%%%%%%%%%%%%%%%%%%%%%%%%%%%%%%%%%%%%%%%%%%%%%%%%%%%%%%%%%%%%%%%%
%\subsection{Feature Suggestions}
%
%The following is a list of features which may be useful for future
%versions of this package:
%%
%\begin{itemize}
%\item
%\ldots
%\end{itemize}

%%%%%%%%%%%%%%%%%%%%%%%%%%%%%%%%%%%%%%%%%%%%%%%%%%%%%%%%%%%%%%%%%%%%%%%%%%%%%%%%
\subsection{Revision History}

%%%%%%%%%%%%%%%%%%%%%%%%%%%%%%%%%%%%%%%%
\paragraph{v2.0:} 2018/12/30

\begin{itemize}
\item
immediate forward processing
\item
added |\childdocby| mechanism
\item
manual restructured
\end{itemize}

%%%%%%%%%%%%%%%%%%%%%%%%%%%%%%%%%%%%%%%%
\paragraph{v1.6:} 2018/01/17

\begin{itemize}
\item
application for development of include files
\item
corrections to manual
\end{itemize}

%%%%%%%%%%%%%%%%%%%%%%%%%%%%%%%%%%%%%%%%
\paragraph{v1.5:} 2017/05/21

\begin{itemize}
\item
more complete structuring introduced
\item
|\childdocof| introduced
\item
|\childdoc| renamed to |\childdocmain|
\item
|\childredirect| renamed to |\childdocforward| and |\childdocforwardprefix|
and functionality expanded
\end{itemize}

%%%%%%%%%%%%%%%%%%%%%%%%%%%%%%%%%%%%%%%%
\paragraph{v1.0:} 2017/04/27

\begin{itemize}
\item
manual and install package
\item
first version published on CTAN
\end{itemize}

%%%%%%%%%%%%%%%%%%%%%%%%%%%%%%%%%%%%%%%%
\paragraph{v0.6:} 2017/04/26

\begin{itemize}
\item
redirection mechanism added
\end{itemize}

%%%%%%%%%%%%%%%%%%%%%%%%%%%%%%%%%%%%%%%%
\paragraph{v0.5:} 2017/04/26

\begin{itemize}
\item
functionality in definition file
\end{itemize}


%%%%%%%%%%%%%%%%%%%%%%%%%%%%%%%%%%%%%%%%%%%%%%%%%%%%%%%%%%%%%%%%%%%%%%%%%%%%%%%%
%%%%%%%%%%%%%%%%%%%%%%%%%%%%%%%%%%%%%%%%%%%%%%%%%%%%%%%%%%%%%%%%%%%%%%%%%%%%%%%%
%%%%%%%%%%%%%%%%%%%%%%%%%%%%%%%%%%%%%%%%%%%%%%%%%%%%%%%%%%%%%%%%%%%%%%%%%%%%%%%%
\appendix

\settowidth\MacroIndent{\rmfamily\scriptsize 000\ }

 \DocInput{childdoc.dtx}

\end{document}
%</driver>
% \fi
%
% %%%%%%%%%%%%%%%%%%%%%%%%%%%%%%%%%%%%%%%%%%%%%%%%%%%%%%%%%%%%%%%%%%%%%%%%%%%%%%
% %%%%%%%%%%%%%%%%%%%%%%%%%%%%%%%%%%%%%%%%%%%%%%%%%%%%%%%%%%%%%%%%%%%%%%%%%%%%%%
% \section{Sample}
%\iffalse
%<*samplemain>
%\fi
%
% The following presents a sample document
% with two chapters, two parts, a title page,
% a compile flag as well as three forwarding files to set the flag.
% It consists of eight |.tex| files:
% \begin{center}
% \begin{tabular}{ll}
% |cdocsamp.tex|&main file\\
% |cdocsch1.tex|&include file for chapter 1\\
% |cdocsch2.tex|&include file for chapter 2\\
% |cdocspt3.tex|&include file for part 3\\
% |cdocspt4.tex|&include file for part 4\\
% |cdocsdrf.tex|&forwarding file for main file in draft mode\\
% |cdocsfi1.tex|&forwarding file for final version of chapter 1\\
% |cdocsfi2.tex|&forwarding file for final version of chapter 2\\
% \end{tabular}
% \end{center}
% Each of the eight files can be compiled directly by the \LaTeX{} compiler.
%
% %%%%%%%%%%%%%%%%%%%%%%%%%%%%%%%%%%%%%%
% \paragraph{Main File.}
%
% The main file is called |cdocsamp.tex|.
%
% Load the \textsf{childdoc} definitions and
% declare the filename for the main document:
%    \begin{macrocode}
\input{childdoc.def}
\childdocmain{}
%    \end{macrocode}

% Optional override for |\version| flag:
%    \begin{macrocode}
%%\ifchilddoc\else\providecommand{\version}{draft}\fi
%    \end{macrocode}

% Define the default values for the |\version| flag
% (|final| for the main file and |draft| for childs):
%    \begin{macrocode}
\ifchilddoc
\providecommand{\version}{draft}
\else
\providecommand{\version}{final}
\fi
%    \end{macrocode}

% Load the standard document class:
%    \begin{macrocode}
\documentclass[12pt]{article}
%    \end{macrocode}

% Start the document body:
%    \begin{macrocode}
\begin{document}
%    \end{macrocode}

% Declare a title page.
% Print title, part of document being processed and version flag:
%    \begin{macrocode}
\addtocounter{page}{-1}
\begin{center}
{\LARGE\bfseries{}childdoc example\par}
\vspace{1cm}
\ifchilddoc
\ifchilddocmanual part\else chapter\fi:
`\childdocname' of `\childdocjob'\par
\else
main document: `\childdocjob'\par
\fi
version: \version\par
\end{center}
\newpage
%    \end{macrocode}

% Manually include selected file,
% otherwise process as usual:
%    \begin{macrocode}
\ifchilddocmanual
\section*{part `\childdocname'}
\input{\childdocname}
\else
%    \end{macrocode}

% Include the two chapters:
%    \begin{macrocode}
\include{cdocsch1}
\include{cdocsch2}
%    \end{macrocode}

% Include the two parts unless only chapters should be displayed:
%    \begin{macrocode}
\ifchilddoc\else
\section{part three}
\input{cdocspt3}
\section{part four}
\input{cdocspt4}
\fi
%    \end{macrocode}

% Process as usual until here:
%    \begin{macrocode}
\fi
%    \end{macrocode}

% End of document body:
%    \begin{macrocode}
\end{document}
%    \end{macrocode}
%\iffalse
%</samplemain>
%\fi
%
% %%%%%%%%%%%%%%%%%%%%%%%%%%%%%%%%%%%%%%
% \paragraph{Chapter Include Files.}
%
% The include files are called |cdocsch1.tex| and |cdocsch2.tex|.
%
%\iffalse
%<*samplechap1|samplechap2>
%\fi

% Optional override for |\version| flag:
%    \begin{macrocode}
%%\providecommand{\version}{final}
%    \end{macrocode}

% Include the main document:
%    \begin{macrocode}
\input{childdoc.def}
\childdocof{cdocsamp}
%    \end{macrocode}

%\iffalse
%</samplechap1|samplechap2>
%\fi
%
%\iffalse
%<*samplechap1>
%\fi
% Some text for chapter 1:
%    \begin{macrocode}
\section{one}
some text in chapter one
%    \end{macrocode}

%\iffalse
%</samplechap1>
%\fi
% Some text for chapter 2:
%\iffalse
%<*samplechap2>
%\fi
%    \begin{macrocode}
\section{two}
more text in chapter two
%    \end{macrocode}

%\iffalse
%</samplechap2>
%\fi
%
% %%%%%%%%%%%%%%%%%%%%%%%%%%%%%%%%%%%%%%
% \paragraph{Part Include Files.}
%
% The include files are called |cdocspt3.tex| and |cdocspt4.tex|.
%
%\iffalse
%<*samplepart3|samplepart4>
%\fi

% Optional override for |\version| flag:
%    \begin{macrocode}
%%\providecommand{\version}{final}
%    \end{macrocode}

% Include the main document:
%    \begin{macrocode}
\input{childdoc.def}
\childdocby{cdocsamp}
%    \end{macrocode}

%\iffalse
%</samplepart3|samplepart4>
%\fi
%
%\iffalse
%<*samplepart3>
%\fi
% Some text for part 3:
%    \begin{macrocode}
some text in part three
%    \end{macrocode}

%\iffalse
%</samplepart3>
%\fi
% Some text for part 4:
%\iffalse
%<*samplepart4>
%\fi
%    \begin{macrocode}
more text in part four
%    \end{macrocode}

%\iffalse
%</samplepart4>
%\fi
%
% %%%%%%%%%%%%%%%%%%%%%%%%%%%%%%%%%%%%%%
% \paragraph{Forwarding for a Complete Draft.}
%
% The following forwarding file |cdocsdrf.tex|
% compiles the main document in draft mode:
%\iffalse
%<*sampledraft>
%\fi
%    \begin{macrocode}
\def\version{draft}
\input{childdoc.def}
\childdocforward{cdocsamp}
%    \end{macrocode}

%\iffalse
%</sampledraft>
%\fi
%
% %%%%%%%%%%%%%%%%%%%%%%%%%%%%%%%%%%%%%%
% \paragraph{Forwarding for Final Version of the Chapters.}
%
% The following forwarding files |cdocsfn1.tex| and |cdocsfn2.tex|
% (with identical content)
% compile the final versions of the child documents
% |cdocsch1.tex| and |cdocsch2.tex|, respectively:
%\iffalse
%<*samplefinal>
%\fi
%    \begin{macrocode}
\def\version{final}
\input{childdoc.def}
\childdocforwardprefix[cdocsamp]{cdocsfn}{cdocsch}
%    \end{macrocode}

%\iffalse
%</samplefinal>
%\fi
%
% %%%%%%%%%%%%%%%%%%%%%%%%%%%%%%%%%%%%%%
% \paragraph{Command Line Processing.}
%
% The following three command lines generate the output files
% |cdocscld|, |cdocscl1| and |cdocscl2|
% which should be identical to
% |cdocsdrf|, |cdocsch1| and |cdocsfn2|, respectively:
% \begin{center}
% \begin{tabular}{l}
% |latex -jobname cdocscld \|\\
% |  "\def\version{draft}\input{childdoc.def}\childdocforward{cdocsamp}"|\\
% |latex -jobname cdocscl1 \|\\
% |  "\input{childdoc.def}\childdocforward[cdocsamp]{cdocsch1}"|\\
% |latex -jobname cdocscl2 \|\\
% |  "\def\version{final}\input{childdoc.def}\childdocforward{cdocsch2}"|
% \end{tabular}
% \end{center}
% Note that the trailing backslash on each first line
% merely continues the input to the second line
% (for convenient cut ant paste).
% Furthermore, the command |latex| can be replaced by any
% of its alternative versions such as |pdflatex|.
%
% %%%%%%%%%%%%%%%%%%%%%%%%%%%%%%%%%%%%%%%%%%%%%%%%%%%%%%%%%%%%%%%%%%%%%%%%%%%%%%
% %%%%%%%%%%%%%%%%%%%%%%%%%%%%%%%%%%%%%%%%%%%%%%%%%%%%%%%%%%%%%%%%%%%%%%%%%%%%%%
% \section{Implementation}
%\iffalse
%<*package>
%\fi
%
% This section describes the definitions file |childdoc.def|.

% The definitions cannot be loaded using |\usepackage| or |\RequirePackage|
% which has a mechanism to prevent loading a style file more than once.
% When loading the definitions by means of |\input|
% multiple instances have to be prevented manually:
%\iffalse
%This code needs to be before the `\ProvidesFile' directive
%which is defined at the beginning of this file.
%Therefore it is also placed there and commented out here.
%</package>
%<*discard>
%\fi
%    \begin{macrocode}
\ifdefined\childdocmain\endinput\fi
%    \end{macrocode}
%\iffalse
%</discard>
%<*package>
%\fi
%
% \macro{\ifchilddoc}
% \macro{\ifchilddocmanual}
% The conditional |\ifchilddoc| tells whether a
% child (true) or main (false) document is being compiled.
% The conditional |\ifchilddocmanual| tells whether
% the |\includeonly| mechanism is used (false) or
% the selection of child files must be performed manually (true).
% The definitions initialise to false:
%    \begin{macrocode}
\newif\ifchilddoc
\newif\ifchilddocmanual
%    \end{macrocode}

% \macro{\childdocname}
% \macro{\childdocjob}
% The macro |\childdocname| stores the name of the main document
% to be compiled. The macro |\childdocjob| stores the name of
% the document on which the \LaTeX{} compiler was originally invoked.
% The content of |\jobname| cannot be compared
% to filenames specified in the source due to different catcodes.
% The following code rescans |\jobname|, stores the result
% in |\childdocname| and saves a copy in |\childdocjob|:
%    \begin{macrocode}
\edef\childdocname{\scantokens\expandafter{\jobname\noexpand}}
\let\childdocjob\childdocname
%    \end{macrocode}

% \macro{\childdocdisable}
% The macro |\childdocdisable| prevents the main file
% from being processed more than once.
% At this stage, the main document command |\childdocmain|
% is assumed to be called once again where it should do nothing.
% Any subsequent call to it should prevent
% a secondary processing of the main document
% It overwrites the forwarding commands
% |\childdocof| and |\childdocforward|
% with empty macros to prevent further inclusions of the main document:
%    \begin{macrocode}
\newcommand{\childdocdisable}
{
  \renewcommand{\childdocmain}[1]{\renewcommand{\childdocmain}[1]{\endinput}}
  \renewcommand{\childdocof}[1]{}
  \renewcommand{\childdocby}[2][]{}
  \renewcommand{\childdocforward}[2][]{}
  \renewcommand{\childdocdisable}{}
}
%    \end{macrocode}

% \macro{\childdocmain}
% The macro |\childdocmain| is to be called at the top of the main file
% with nothing or the main filename (without extension) as argument.
% First, it breaks loops.
% If the argument is not empty and does not match |\childdocname|
% (which is set by the first inclusion of |childdoc.def|),
% |\ifchilddoc| is set to true, |\includeonly| is applied to the child file
% and |\jobname| is set to the main file
% (for proper handling of |.aux| files):
%    \begin{macrocode}
\newcommand{\childdocmain}[1]
{
  \childdocdisable\childdocmain{}
  \if?#1?\else
    \begingroup
      \def\childdoctmp{#1}
      \ifx\childdoctmp\childdocname
        \def\childdoctmp{}
      \else
        \def\childdoctmp
        {
          \childdoctrue
          \includeonly{\childdocname}
          \def\childdocjob{#1}
          \def\jobname{#1}
        }
      \fi
      \expandafter
    \endgroup
    \childdoctmp
  \fi
}
%    \end{macrocode}

% \macro{\childdocof}
% The command |\childdocof| redirects
% compilation to the main file |#1|.
%    \begin{macrocode}
\newcommand{\childdocof}[1]
{
  \childdocdisable
  \childdoctrue
  \includeonly{\childdocname}
  \def\jobname{#1}
  \def\childdocjob{#1}
  \input{#1}
}
%    \end{macrocode}

% \macro{\childdocby}
% The command |\childdocby| ....
%    \begin{macrocode}
\newcommand{\childdocby}[2][]
{
  \childdocdisable
  \childdoctrue
  \childdocmanualtrue
  \if?#1?\else
    \def\jobname{#2}
  \fi
  \def\childdocjob{#2}
  \input{#2}
  \endinput
}
%    \end{macrocode}

% \macro{\childdocforward}
% The command |\childdocforward| redirects
% compilation to the main file or
% (if the optional argument is given) a child file.
% Parameters are set as if the main file
% or a child file starting with |\childdocof| was compiled.
% Then compilation is handed over to the main file:
%    \begin{macrocode}
\newcommand{\childdocforward}[2][]
{
  \begingroup
    \if?#1?
      \def\childdoctmp
      {
        \def\childdocname{#2}
        \def\childdocjob{#2}
        \def\jobname{#2}
        \input{#2}
        \endinput
      }
    \else
      \def\childdoctmp
      {
        \childdocdisable
        \def\childdocname{#2}
        \childdoctrue
        \includeonly{#2}
        \def\childdocjob{#1}
        \def\jobname{#1}
        \input{#1}
        \endinput
      }
    \fi
    \expandafter
  \endgroup
  \childdoctmp
}
%    \end{macrocode}

% \macro{\childdocforwardprefix}
% The command |\childdocforwardprefix| redirects
% compilation to the main or a child file by means of a pattern.
% The prefix |#1| in the current filename is replaced by |#2|
% and the suffix of the current filename is kept
% (it is assumed that the filename does not contain the substring `|~~~|'
% which is used as a delimiter).
% Compilation is handed over to the new file by |\childdocforward|:
%    \begin{macrocode}
\newcommand{\childdocforwardprefix}[3][]
{
  \begingroup
    \def\childdocextract #2##1~~~{\def\childdoctmp{\childdocforward[#1]{#3##1}}}
    \expandafter\childdocextract\childdocname~~~
    \expandafter
  \endgroup
  \childdoctmp
}
%    \end{macrocode}

% \macro{\childdoc}
% The deprecated macro |\childdoc| is a legacy version of |\childdocmain|:
%    \begin{macrocode}
\newcommand{\childdoc}{\childdocmain}
%    \end{macrocode}

% \macro{\childdocredirect}
% The deprecated macro |\childdocredirect| is a legacy version
% of |\childdocforward| and |\childdocforwardprefix|:
%    \begin{macrocode}
\newcommand{\childdocredirect}[2][]
{
  \begingroup
    \if?#1?
      \def\childdoctmp{\childdocforward{#2}}
    \else
      \def\childdoctmp{\childdocforwardprefix{#1}{#2}}
    \fi
    \expandafter
  \endgroup
  \childdoctmp
}
%    \end{macrocode}

%\iffalse
%</package>
%\fi
%
\endinput
|\\
|\childdocmain{}|\\
\end{tabular}
\end{center}
at the very top of the main \LaTeX{} file,
in particular \emph{before} the |\documentclass| statement!
The argument of |\childdocmain| should be left empty
(but it must be present).

%%%%%%%%%%%%%%%%%%%%%%%%%%%%%%%%%%%%%%%%
\DescribeMacro{\childdocof}
Furthermore, add the commands
\begin{center}
\begin{tabular}{l}
|% \iffalse
%
% childdoc.dtx Copyright (C) 2017-2018 Niklas Beisert
%
% This work may be distributed and/or modified under the
% conditions of the LaTeX Project Public License, either version 1.3
% of this license or (at your option) any later version.
% The latest version of this license is in
%   http://www.latex-project.org/lppl.txt
% and version 1.3 or later is part of all distributions of LaTeX
% version 2005/12/01 or later.
%
% This work has the LPPL maintenance status `maintained'.
%
% The Current Maintainer of this work is Niklas Beisert.
%
% This work consists of the files childdoc.dtx and childdoc.ins
% and the derived files childdoc.def and cdocsamp.tex with
% cdocsch1.tex, cdocsch2.tex, cdocsdrf.tex, cdocsfn1.tex, cdocsfn2.tex.
%
%<package>\ifdefined\childdocmain\endinput\fi
%<package>\ProvidesFile{childdoc.def}[2018/12/30 v2.0 child document driver]
%<samplemain>\ProvidesFile{cdocsamp.tex}[2018/12/30 v2.0 sample for childdoc]
%<*driver>
%\ProvidesFile{childdoc.drv}[2018/12/30 v2.0 childdoc reference manual file]
\PassOptionsToClass{10pt,a4paper}{article}
\documentclass{ltxdoc}

\usepackage[margin=35mm]{geometry}
\usepackage{hyperref}
\usepackage{hyperxmp}
\usepackage[usenames]{color}

\hypersetup{colorlinks=true}
\hypersetup{pdfstartview=FitH}
\hypersetup{pdfpagemode=UseNone}
\hypersetup{pdfsource={}}
\hypersetup{pdflang={en-UK}}
\hypersetup{pdfcopyright={Copyright 2017-2018 Niklas Beisert.
  This work may be distributed and/or modified under the
  conditions of the LaTeX Project Public License, either version 1.3
  of this license or (at your option) any later version.}}
\hypersetup{pdflicenseurl={http://www.latex-project.org/lppl.txt}}
\hypersetup{pdfcontactaddress={ETH Zurich, ITP, HIT K,
  Wolfgang-Pauli-Strasse 27}}
\hypersetup{pdfcontactpostcode={8093}}
\hypersetup{pdfcontactcity={Zurich}}
\hypersetup{pdfcontactcountry={Switzerland}}
\hypersetup{pdfcontactemail={nbeisert@itp.phys.ethz.ch}}
\hypersetup{pdfcontacturl={http://people.phys.ethz.ch/\xmptilde nbeisert/}}

\newcommand{\secref}[1]{\hyperref[#1]{section \ref*{#1}}}

\parskip1ex
\parindent0pt
\let\olditemize\itemize
\def\itemize{\olditemize\parskip0pt}

\begin{document}

\title{The \textsf{childdoc} Package}
\hypersetup{pdftitle={The childdoc Package}}
\author{Niklas Beisert\\[2ex]
  Institut f\"ur Theoretische Physik\\
  Eidgen\"ossische Technische Hochschule Z\"urich\\
  Wolfgang-Pauli-Strasse 27, 8093 Z\"urich, Switzerland\\[1ex]
  \href{mailto:nbeisert@itp.phys.ethz.ch}
  {\texttt{nbeisert@itp.phys.ethz.ch}}}
\hypersetup{pdfauthor={Niklas Beisert}}
\hypersetup{pdfsubject={Manual for the LaTeX2e Package childdoc}}
\date{30 December 2018, \textsf{v2.0}}
\maketitle

\begin{abstract}\noindent
\textsf{childdoc} is a \LaTeXe{} package
that enables the direct compilation
of document sections included by |\include|
to individual files.
\end{abstract}

\begingroup
\parskip0ex
\tableofcontents
\endgroup

%%%%%%%%%%%%%%%%%%%%%%%%%%%%%%%%%%%%%%%%%%%%%%%%%%%%%%%%%%%%%%%%%%%%%%%%%%%%%%%%
%%%%%%%%%%%%%%%%%%%%%%%%%%%%%%%%%%%%%%%%%%%%%%%%%%%%%%%%%%%%%%%%%%%%%%%%%%%%%%%%
\section{Introduction}

\LaTeX{} provides a mechanism to structure a large document (such as a book)
into a main file and several child files (containing the chapters)
using the |\include| command.
This mechanism is beneficial for documents
which span hundreds of pages in order to
make the source file(s) more manageable.
Moreover, compilation can be restricted to
selected child files by means of the |\includeonly| command.
The latter feature can be used to reduce the compilation time while editing
(this was significantly more useful in the earlier days of \LaTeX{})
or to generate a smaller document which is easier to navigate.
Another application of |\includeonly| is to generate
documents consisting of selected parts of the complete document.

However, there are a few drawbacks of the plain |\include| mechanism:
\begin{itemize}
\item
The child files cannot be compiled on their own,
they can only be compiled via the main file.
A naive editing environment
(such as a text editor with an option
to have the current file processed by \LaTeX)
may require one to switch to the main file before compiling;
attempting to compile the child file produces errors.
\item
The main file must be modified (each time)
to adjust the |\includeonly| command
to the present needs. This easily leaves the main file in a messy state.
\item
The generated document will always carry the filename
of the main document. This is inconvenient if
several child files are to be compiled and
to be kept for distribution.
\end{itemize}

The present package provides a simple interface
to make child files individually compilable by \LaTeX{}.
Compiling a child file then has the same effect as compiling
the main file with an |\includeonly| command
to select the appropriate child.
Moreover the generated document will carry the name of the child
rather than the main file.
This resolves all three above issues.

This feature is meant to make the editing of books,
thesis documents and lecture notes somewhat more convenient.
However, the package can also be used efficiently for
composing a series of documents (such as exercise sheets)
which are typically distributed individually.
It then assists the author in generating the individual documents
(potentially in different versions)
as well as a document containing the collected series.
Another application is in developing style files
or other kinds of included material
where compilation of the style file could redirect
to a sample or test file.

%%%%%%%%%%%%%%%%%%%%%%%%%%%%%%%%%%%%%%%%%%%%%%%%%%%%%%%%%%%%%%%%%%%%%%%%%%%%%%%%
%%%%%%%%%%%%%%%%%%%%%%%%%%%%%%%%%%%%%%%%%%%%%%%%%%%%%%%%%%%%%%%%%%%%%%%%%%%%%%%%
\section{Usage}

First of all, the package \textsf{childdoc} is \emph{not} a standard
\LaTeXe{} |.sty| style file! Therefore it needs to be invoked in
a non-standard way.

%%%%%%%%%%%%%%%%%%%%%%%%%%%%%%%%%%%%%%%%%%%%%%%%%%%%%%%%%%%%%%%%%%%%%%%%%%%%%%%%
\subsection{Included Files}
\label{sec:include}

%%%%%%%%%%%%%%%%%%%%%%%%%%%%%%%%%%%%%%%%
\DescribeMacro{\childdocmain}
To use the package, add the commands
\begin{center}
\begin{tabular}{l}
|\input{childdoc.def}|\\
|\childdocmain{}|\\
\end{tabular}
\end{center}
at the very top of the main \LaTeX{} file,
in particular \emph{before} the |\documentclass| statement!
The argument of |\childdocmain| should be left empty
(but it must be present).

%%%%%%%%%%%%%%%%%%%%%%%%%%%%%%%%%%%%%%%%
\DescribeMacro{\childdocof}
Furthermore, add the commands
\begin{center}
\begin{tabular}{l}
|\input{childdoc.def}|\\
|\childdocof{|\textit{main}|}|\\
\end{tabular}
\end{center}
at the top of every child file \textit{child}
which is included by |\include{|\textit{child}|}|
from within the main file
(or at least for those files to be compiled individually).
The argument \textit{main} must be the filename of the main file.

There are a couple of
considerations in setting up the main and child documents:

%%%%%%%%%%%%%%%%%%%%%%%%%%%%%%%%%%%%%%%%
\paragraph{Restrictions.}

Please note the following restrictions:
\begin{itemize}
\item
|\childdocmain| must be called with one argument \textit{main}
to ensure compatibility with earlier version of the package.
It must either be empty (|\childdocmain{}|)
or precisely match the filename of the main file in which it is specified.
See \secref{sec:detection} for further information.
\item
The filename \textit{main} must be specified without the |.tex| extension.
\item
The filename \textit{main} is case sensitive
(even in case-insensitive file systems)
due to internal string comparison.
\item
The argument \textit{main} should be fully expanded, it cannot be a macro.
\item
Subdirectories and special characters should be avoided in filenames.
\item
The command |\childdocmain{|\textit{main}|}| must be followed by a whitespace.
It should not be followed immediately by another command
or by a comment mark `|%|'.
This is because the \TeX{} parser reads the token immediately following
the argument of |\childdocmain| and puts it
at the beginning of every child section;
however, a white\-space is ignored.
\end{itemize}

%%%%%%%%%%%%%%%%%%%%%%%%%%%%%%%%%%%%%%%%
\paragraph{Content of Main File.}

It is advisable to place all content in the child files included by |\include|.
Any output contained in the main file will appear in all child documents
unless suppressed manually;
it cannot be suppressed automatically by the |\includeonly| directive
and thus should normally be avoided.
A method to include some content in the main file
by means of conditional processing is described in \secref{sec:conditional}.

%%%%%%%%%%%%%%%%%%%%%%%%%%%%%%%%%%%%%%%%
\paragraph{Page Numbering.}

When only a part of the document is compiled,
the appropriate numbering of pages
(as well as other status parameters)
is determined from the |.aux| files.
The latter contain information from previous passes.
However this information needs to propagate through
all intermediate child documents.
Therefore the page numbering in child documents may well
be inconsistent until the complete document is compiled at least once.

A useful (if unconventional) way to always ensure a consistent
page numbering is to restart the numbering in each child document
and denote the pages by `\textit{child}|.|\textit{page}'
where \textit{child} represents the chapter/section number of the child file.
This can be achieved by the command
|\numberwithin{page}{|\textit{child}|}|
of the \textsf{amsmath} package
where \textit{child} can be |chapter| or |section|
depending on the chosen structuring.
Alternatively, one can modify the macro |\thepage| appropriately
and reset the counter |page| at the start of each child file.

%%%%%%%%%%%%%%%%%%%%%%%%%%%%%%%%%%%%%%%%%%%%%%%%%%%%%%%%%%%%%%%%%%%%%%%%%%%%%%%%
\subsection{Conditional Processing}
\label{sec:conditional}

The package provides a mechanism to compile different versions
of a document. To customise the versions further some conditional processing
can come in handy to distinguish which version is being compiled.
The package provides two macros to describe the compilation context:

%%%%%%%%%%%%%%%%%%%%%%%%%%%%%%%%%%%%%%%%
\DescribeMacro{\ifchilddoc}
The conditional |\ifchilddoc| distinguishes between the compilation of
child documents and the main document:
%
\begin{center}
|\ifchilddoc |\textit{child-code}| |[|\||else |\textit{main-code}]| \||fi|
\end{center}

%%%%%%%%%%%%%%%%%%%%%%%%%%%%%%%%%%%%%%%%
\DescribeMacro{\childdocname}
\DescribeMacro{\childdocjob}
The macro |\childdocname| contains the filename (without extension)
of the main or child file being processed.
Note that |\childdocjob| will always contain the name of the main file.

%%%%%%%%%%%%%%%%%%%%%%%%%%%%%%%%%%%%%%%%
\paragraph{Title Page.}

Conditional processing can be used to include a title or banner page
in the main document when proper precautions are taken.
Importantly, the code in the main file should ensure that the page counter
(as well as other status parameters which are stored in the |.aux| files)
takes the same value after the conditional processing.
Otherwise the page numbers may take divergent values
depending on which part is compiled.

For example, a title page could be declared by:
%
\begin{center}
\begin{tabular}{l}
|\ifchilddoc\||else|\\
|\addtocounter{page}{-1}|\\
\textit{code for title page}\\
|\newpage|\\
|\||fi|
\end{tabular}
\end{center}
%
A banner page for the child documents can be generated by:
%
\begin{center}
\begin{tabular}{l}
|\ifchilddoc|\\
|\addtocounter{page}{-1}|\\
\textit{code for banner page}\\
|\newpage|\\
|\||fi|
\end{tabular}
\end{center}
%
Here one could write a message such as:
\begin{center}
|This is the part \childdocname{} of \childdocjob{}.|
\end{center}

%%%%%%%%%%%%%%%%%%%%%%%%%%%%%%%%%%%%%%%%%%%%%%%%%%%%%%%%%%%%%%%%%%%%%%%%%%%%%%%%
\subsection{Flags}
\label{sec:flags}

The package makes it easy to generate different versions
of the main or child documents.
To this end compilation flags can be defined
and assigned different default values.
They will be particularly useful in conjunction
with the forwarding mechanism described in \secref{sec:forward}.

For example, it may be useful to have a flag |\version|
which can be set to |draft| or |final|.
The document source will contain some conditional code
depending on the value of |\version|.
Suppose further, the flag should default to |final| for the main file
and to |draft| for child files
which is a natural assignment for editing the document.
This is achieved by placing the following code
in the preamble of the main document
(below the |\childdocmain| directive):
%
\begin{center}
\begin{tabular}{l}
|\ifchilddoc|\\
|\providecommand{\version}{draft}|\\
|\||else|\\
|\providecommand{\version}{final}|\\
|\||fi|
\end{tabular}
\end{center}
%
The definition by |\providecommand| makes sure
that previous definitions are not overwritten.
Further statements |\providecommand{\version}{...}|
can thus be added before the above code to override it.

For the main file, one might add a line
(between |\childdocmain| and the above block)
%
\begin{center}
|%\ifchilddoc\||else\providecommand{\version}{draft}\||fi|
\end{center}
%
which can be uncommented to produce a draft version.
Likewise one can add a line to the very top of a child file
(above the |\childdocof{|\textit{main}|}| directive)
%
\begin{center}
|%\providecommand{\version}{final}|
\end{center}
%
which can be uncommented to produce the final version of this child document.

%%%%%%%%%%%%%%%%%%%%%%%%%%%%%%%%%%%%%%%%%%%%%%%%%%%%%%%%%%%%%%%%%%%%%%%%%%%%%%%%
\subsection{Forwarding}
\label{sec:forward}

Different versions of the main or child documents
using compilation flags as described in \secref{sec:flags}
can be (permanently) stored in different files
for convenient compilation, viewing and distribution.
To this end, the package defines a command
to pass on compilation to a different file:

%%%%%%%%%%%%%%%%%%%%%%%%%%%%%%%%%%%%%%%%
\DescribeMacro{\childdocforward}
The command |\childdocforward| redirects processing to
another source file:
%
\begin{center}
\begin{tabular}{l}
|\input{childdoc.def}|\\
|\childdocforward[|\textit{main}|]{|\textit{dest}|}|\\
\end{tabular}
\end{center}
%
The argument \textit{dest} is the destination file
(without extension).
It should be the main file or one of the child files.
Note that further \textsf{childdoc} directives
such as |\childdocof| and |\childdocforward|
in the indicated file will be processed in this form.
The optional argument \textit{main}
passes on directly to the main file \textit{main}
while pretending to compile the child \textit{dest}.
This form behaves as if \textit{dest}
issues |\childdocof{|\textit{main}|}| right away,
and no further \textsf{childdoc} directives will be processed.

%%%%%%%%%%%%%%%%%%%%%%%%%%%%%%%%%%%%%%%%
\DescribeMacro{\...prefix}
In the alternative form |\childdocforwardprefix|,
%
\begin{center}
\begin{tabular}{l}
|\input{childdoc.def}|\\
|\childdocforwardprefix[|\textit{main}|]{|\textit{prefix}|}{|\textit{dest}|}|
\end{tabular}
\end{center}
%
the destination file is determined by a pattern
depending on the current file:
To make this work, the current file must be called
`{\textit{prefix}\hspace{0.2em}\textit{suffix}}'
with \textit{prefix} matching precisely the argument.
Processing is then passed on to the file
`{\textit{dest}\hspace{0.2em}\textit{suffix}}'.
Surely, the same effect is achieved by
directly specifying the
argument `{\textit{dest}\hspace{0.2em}\textit{suffix}}'
in the first form.
However, that requires to set up a different file
for each child. With the alternative form of the command
all these files can have exactly the same content
which simplifies setting them up and maintaining them.

For example, the following file |draft.tex|
with a compilation flag |\version| as described in \secref{sec:flags}
compiles the main document as a draft:
%
\begin{center}
\begin{tabular}{l}
|\def\version{draft}|\\
|\input{childdoc.def}|\\
|\childdocforward{|\textit{main}|}|
\end{tabular}
\end{center}
%
Likewise, the following files |final|\textit{nn}|.tex|
compile the final version of the child document
|child|\textit{nn}|.tex|:
%
\begin{center}
\begin{tabular}{l}
|\def\version{final}|\\
|\input{childdoc.def}|\\
|\childdocforwardprefix{final}{child}|
\end{tabular}
\end{center}
%

Note that when several versions of a main file and/or of each child file
are to be generated, it may be convenient to set up a |Makefile| or
shell script to automatise the process.

%%%%%%%%%%%%%%%%%%%%%%%%%%%%%%%%%%%%%%%%%%%%%%%%%%%%%%%%%%%%%%%%%%%%%%%%%%%%%%%%
\subsection{Command Line Processing}
\label{sec:commandline}

The effect of redirection files can also be achieved by invoking
the \LaTeX{} compiler with a more elaborate command line.
Most conveniently this should be done as part
of a shell script or a |Makefile|.

When using \textsf{childdoc} in the main file, the following
command lines effectively perform a redirection
(note that depending on the shell being used,
backslashes may have to be doubled: `|\|' $\to$ `|\\|'):
%
\begin{center}
|... -jobname "|\textit{target}|" |\\|"|[\textit{flags}]%
|\input{childdoc.def}\childdocforward[|\textit{main}|]{|\textit{dest}|}"|
\end{center}
%
Here \textit{target} is the name of the output file,
\textit{main} is the name of the main file
and \textit{dest} is the name of the main or child file to be processed
(all filenames without extensions).
The optional argument \textit{main} can be omitted
if \textit{main} matches \textit{dest}.
Optionally, compilation \textit{flags} can be defined via |\def| commands.
This command line makes the \TeX{} engine believe
it is compiling the file \textit{target}
whose content is specified as the latter parameter.
The provided code then forwards the processing to
\textit{main} or \textit{dest} as described in \secref{sec:forward}.

%%%%%%%%%%%%%%%%%%%%%%%%%%%%%%%%%%%%%%%%%%%%%%%%%%%%%%%%%%%%%%%%%%%%%%%%%%%%%%%%
\subsection{Include by Input}
\label{sec:input}

Including child documents by |\include| has some restrictions by design.
Most notably, the content of a child document always occupies
its own set of pages; pages cannot be shared between child documents.
Usually, this behaviour makes perfect sense
because each child document contain an essential part of the document.
However, in some situations it may be desirable to compose
a document from a collection of parts
without having mandatory page breaks between then.
For this case, the package
provides a mechanism to include parts
by |\input| which can also be processed individually.
However, by construction this mechanism
requires manual handling of the content to be output.

%%%%%%%%%%%%%%%%%%%%%%%%%%%%%%%%%%%%%%%%
\DescribeMacro{\ifchilddocmanual}
The main file should be prepared as usual, see \secref{sec:include}.
However, the document body must make a distinction
between processing of an individual part and of the main document, e.g.:
%
\begin{center}
\begin{tabular}{l}
|\ifchilddocmanual|\\
|\input{\childdocname}|\\
|\||else|\\
\textit{document body with }|\input{|\textit{part}|}|\\
|\||fi|
\end{tabular}
\end{center}
%
The conditional |\ifchilddocmanual| is true whenever
a part to be included by |\input| is being compiled,
and the name of the part is stored in |\childdocname|.

%%%%%%%%%%%%%%%%%%%%%%%%%%%%%%%%%%%%%%%%
\DescribeMacro{\childdocby}
Each part to be included by |\input| should start with:
%
\begin{center}
\begin{tabular}{l}
|\input{childdoc.def}|\\
|\childdocby{|\textit{main}|}|\\
\end{tabular}
\end{center}
%
The directive |\childdocby| is similar to |\childdocof|
described in \secref{sec:include},
but the subsequent selection of content must be done manually.
To that end, both |\ifchilddoc| and |\ifchilddocmanual|
will be true upon processing of a part,
and the name of the part is stored in |\childdocname|.
Note that |\jobname| will be set to the filename of the current part
so that each part receives an individual |.aux| file
that does not interfere with the |.aux| file(s) of the main document.
This behaviour can be altered by the alternative form
|\childdocby[*]{|\textit{main}|}| (with a non-empty optional argument)
which uses the |.aux| file of the main document
by setting |\jobname| to \textit{main}.

%%%%%%%%%%%%%%%%%%%%%%%%%%%%%%%%%%%%%%%%%%%%%%%%%%%%%%%%%%%%%%%%%%%%%%%%%%%%%%%%
\subsection{Driver Development}
\label{sec:driver}

The \textsf{childdoc} mechanism can also be use for the development
of definition files such as \LaTeX{} styles or classes.
This case differs from the above setup with multiple parts
included by |\include| in that no |\includeonly| should be invoked.
This can be achieved by starting the include file
(before |\ProvidesPackage|) with:
%
\begin{center}
\begin{tabular}{l}
|\input{childdoc.def}|\\
|\childdocforward{|\textit{main}|}|\\
\end{tabular}
\end{center}
%
or alternatively with:
%
\begin{center}
\begin{tabular}{l}
|\input{childdoc.def}|\\
|\childdocby{|\textit{main}|}|\\
\end{tabular}
\end{center}
%
Both forms have slightly different effects as described above.
The main file is prepared as usual, see \secref{sec:include}.

%%%%%%%%%%%%%%%%%%%%%%%%%%%%%%%%%%%%%%%%%%%%%%%%%%%%%%%%%%%%%%%%%%%%%%%%%%%%%%%%
\subsection{Legacy Detection}
\label{sec:detection}

The directive |\childdocmain| in the main file can detect
whether the complete document or merely a child is to be compiled
even without using the directive |\childdocof|.
This method is deprecated because it is less robust
and there is no compelling reason to use it;
it is merely provided for backward compatibility
and it may be removed in future versions.

If the detection mechanism is to be used,
it is mandatory to correctly specify
the filename of the main file as the argument of |\childdocmain|:
%
\begin{center}
\begin{tabular}{l}
|\input{childdoc.def}|\\
|\childdocmain{|\textit{main}|}|\\
\end{tabular}
\end{center}
%
If |\jobname| does not match the argument \textit{main} of |\childdocmain|,
it is assumed that |\jobname| points to the child file to be compiled.
When using |\childdocmain| with the main file specified as argument,
it suffices to start a child file
with just |\input{|\textit{main}|}|
without loading of the package and using |\childdocof|.
If instead all processing is done
with the appropriate \textsf{childdoc} directives,
the argument of \textit{main} of |\childdocmain| can be empty.

An alternative version of the command line processing described
in \secref{sec:commandline} using the detection mechanism reads:
%
\begin{center}
|... -jobname "|\textit{target}|" "|[\textit{flags}]%
[|\def\jobname{|\textit{dest}|}|]|\input{|\textit{main}|}"|
\end{center}

%%%%%%%%%%%%%%%%%%%%%%%%%%%%%%%%%%%%%%%%%%%%%%%%%%%%%%%%%%%%%%%%%%%%%%%%%%%%%%%%
\subsection{Manual Code}
\label{sec:manual}

In case one cannot be certain whether the definitions file |childdoc.def|
is installed on the target \TeX{} distribution
and one prefers not to ship it,
it is conceivable to paste a few relevant commands into the sources.

To that end, drop all statements |\input{childdoc.def}|
and perform the replacements as outlined below.
Instead of |\childdocmain{|\textit{main}|}| add the following code
to the top of the main file:
%
\begin{center}
\begin{tabular}{l}
|\||ifdefined\childdocname\endinput\||fi\newif\ifchilddoc|\\
|\edef\childdocname{\scantokens\expandafter{\jobname\noexpand}}|\\
|\def\childdocmain{|\textit{main}|}\||ifx\childdocmain\childdocname\||else|\\
|\childdoctrue\includeonly{\childdocname}\let\jobname\childdocmain\||fi|\\
\end{tabular}
\end{center}
%
Instead of |\childdocof{|\textit{main}|}| just include the main file
at the top of each child file:
%
\begin{center}
|\input{|\textit{main}|}|
\end{center}
%
A simple redirection |\childdocforward{|\textit{dest}|}| is achieved by:
%
\begin{center}
|\def\jobname{|\textit{dest}|}\input{\jobname}|
\end{center}
%
The redirection with prefix
|\childdocforwardprefix[|\textit{prefix}|]{|\textit{dest}|}|
is accomplished by:
%
\begin{center}
\begin{tabular}{l}
|{\edef\jobname{\scantokens\expandafter{\jobname\noexpand}}|\\
|\def\redirectjob |\textit{prefix}|#1~~~{\gdef\jobname{|\textit{dest}|#1}}|\\
|\expandafter\redirectjob\jobname~~~}\input{\jobname}|
\end{tabular}
\end{center}

In an alternative approach,
child documents can be compiled by a specific command line
without additional code or specific definitions:
%
\begin{center}
|... -jobname "|\textit{target}|" "|[\textit{flags}]%
|\includeonly{|\textit{dest}|}\input{|\textit{main}|}"|
\end{center}
%

%%%%%%%%%%%%%%%%%%%%%%%%%%%%%%%%%%%%%%%%%%%%%%%%%%%%%%%%%%%%%%%%%%%%%%%%%%%%%%%%
%%%%%%%%%%%%%%%%%%%%%%%%%%%%%%%%%%%%%%%%%%%%%%%%%%%%%%%%%%%%%%%%%%%%%%%%%%%%%%%%
\section{Information}

%%%%%%%%%%%%%%%%%%%%%%%%%%%%%%%%%%%%%%%%%%%%%%%%%%%%%%%%%%%%%%%%%%%%%%%%%%%%%%%%
\subsection{Copyright}

Copyright \copyright{} 2017--2018 Niklas Beisert

This work may be distributed and/or modified under the
conditions of the \LaTeX{} Project Public License, either version 1.3
of this license or (at your option) any later version.
The latest version of this license is in
  \url{http://www.latex-project.org/lppl.txt}
and version 1.3 or later is part of all distributions of \LaTeX{}
version 2005/12/01 or later.

This work has the LPPL maintenance status `maintained'.

The Current Maintainer of this work is Niklas Beisert.

This work consists of the files |README.txt|, |childdoc.ins| and |childdoc.dtx|
as well as the derived files |childdoc.def|, |cdocsamp.tex|
with |cdocsch1.tex|, |cdocsch2.tex|, |cdocspt3.tex|, |cdocspt4.tex|,
|cdocsdrf.tex|, |cdocsfn1.tex|, |cdocsfn2.tex|
as well as |childdoc.pdf|.

%%%%%%%%%%%%%%%%%%%%%%%%%%%%%%%%%%%%%%%%%%%%%%%%%%%%%%%%%%%%%%%%%%%%%%%%%%%%%%%%
\subsection{Files and Installation}

The package consists of the files:
%
\begin{center}
\begin{tabular}{ll}
    |README.txt|   & readme file \\
    |childdoc.ins| & installation file \\
    |childdoc.dtx| & source file \\
    |childdoc.def| & definition file \\
    |cdocsamp.tex| & sample main file \\
    |cdocsch1.tex| & sample include file \\
    |cdocsch2.tex| & sample include file \\
    |cdocspt3.tex| & sample part file \\
    |cdocspt4.tex| & sample part file \\
    |cdocsdrf.tex| & sample redirection file \\
    |cdocsfn1.tex| & sample redirection file \\
    |cdocsfn2.tex| & sample redirection file \\
    |childdoc.pdf| & manual
\end{tabular}
\end{center}
%
The distribution consists of the files
|README.txt|, |childdoc.ins| and |childdoc.dtx|.
%
\begin{itemize}
\item
Run (pdf)\LaTeX{} on |childdoc.dtx|
to compile the manual |childdoc.pdf| (this file).
\item
Run \LaTeX{} on |childdoc.ins| to create the definitions file |childdoc.def|
and the sample |cdocsamp.tex| with include files
|cdocsch1.tex|, |cdocsch2.tex|, |cdocspt3.tex|, |cdocspt4.tex|,
|cdocsdrf.tex|, |cdocsfn1.tex|, |cdocsfn2.tex|.
Then copy the file |childdoc.def| to an appropriate directory of your \LaTeX{}
distribution, e.g.\ \textit{texmf-root}|/tex/latex/childdoc|.
\end{itemize}

%%%%%%%%%%%%%%%%%%%%%%%%%%%%%%%%%%%%%%%%%%%%%%%%%%%%%%%%%%%%%%%%%%%%%%%%%%%%%%%%
\subsection{Related CTAN Packages}

There are several other packages which offer a similar functionality:
%
\begin{itemize}
\item
The packages
\href{http://ctan.org/pkg/docmute}{\textsf{docmute}},
\href{http://ctan.org/pkg/includex}{\textsf{includex}} and
\href{http://ctan.org/pkg/standalone}{\textsf{standalone}}
provide commands to include only the document body of
a child file thus allowing both files to be compiled individually.
\item
The packages \href{http://ctan.org/pkg/subdocs}{\textsf{subdocs}}
and \href{http://ctan.org/pkg/subfiles}{\textsf{subfiles}}
provide structures in which the main and child documents can be
encapsulated and allowing them to be compiled individually.
The inclusion mechanism is different from the conventional |\include|.
\item
The package \href{http://ctan.org/pkg/combine}{\textsf{combine}}
is an elaborate solution to combine several documents into one.
\end{itemize}
%
See also the CTAN topic \href{http://ctan.org/topic/subdocs}{\textsf{subdocs}}
for further related packages.
The present package differs from the above solutions in that
a document structure constructed with the conventional |\include| mechanism
just needs two extra commands at the top of every file
such that all constituent files can be compiled individually.

%%%%%%%%%%%%%%%%%%%%%%%%%%%%%%%%%%%%%%%%%%%%%%%%%%%%%%%%%%%%%%%%%%%%%%%%%%%%%%%%
%\subsection{Feature Suggestions}
%
%The following is a list of features which may be useful for future
%versions of this package:
%%
%\begin{itemize}
%\item
%\ldots
%\end{itemize}

%%%%%%%%%%%%%%%%%%%%%%%%%%%%%%%%%%%%%%%%%%%%%%%%%%%%%%%%%%%%%%%%%%%%%%%%%%%%%%%%
\subsection{Revision History}

%%%%%%%%%%%%%%%%%%%%%%%%%%%%%%%%%%%%%%%%
\paragraph{v2.0:} 2018/12/30

\begin{itemize}
\item
immediate forward processing
\item
added |\childdocby| mechanism
\item
manual restructured
\end{itemize}

%%%%%%%%%%%%%%%%%%%%%%%%%%%%%%%%%%%%%%%%
\paragraph{v1.6:} 2018/01/17

\begin{itemize}
\item
application for development of include files
\item
corrections to manual
\end{itemize}

%%%%%%%%%%%%%%%%%%%%%%%%%%%%%%%%%%%%%%%%
\paragraph{v1.5:} 2017/05/21

\begin{itemize}
\item
more complete structuring introduced
\item
|\childdocof| introduced
\item
|\childdoc| renamed to |\childdocmain|
\item
|\childredirect| renamed to |\childdocforward| and |\childdocforwardprefix|
and functionality expanded
\end{itemize}

%%%%%%%%%%%%%%%%%%%%%%%%%%%%%%%%%%%%%%%%
\paragraph{v1.0:} 2017/04/27

\begin{itemize}
\item
manual and install package
\item
first version published on CTAN
\end{itemize}

%%%%%%%%%%%%%%%%%%%%%%%%%%%%%%%%%%%%%%%%
\paragraph{v0.6:} 2017/04/26

\begin{itemize}
\item
redirection mechanism added
\end{itemize}

%%%%%%%%%%%%%%%%%%%%%%%%%%%%%%%%%%%%%%%%
\paragraph{v0.5:} 2017/04/26

\begin{itemize}
\item
functionality in definition file
\end{itemize}


%%%%%%%%%%%%%%%%%%%%%%%%%%%%%%%%%%%%%%%%%%%%%%%%%%%%%%%%%%%%%%%%%%%%%%%%%%%%%%%%
%%%%%%%%%%%%%%%%%%%%%%%%%%%%%%%%%%%%%%%%%%%%%%%%%%%%%%%%%%%%%%%%%%%%%%%%%%%%%%%%
%%%%%%%%%%%%%%%%%%%%%%%%%%%%%%%%%%%%%%%%%%%%%%%%%%%%%%%%%%%%%%%%%%%%%%%%%%%%%%%%
\appendix

\settowidth\MacroIndent{\rmfamily\scriptsize 000\ }

 \DocInput{childdoc.dtx}

\end{document}
%</driver>
% \fi
%
% %%%%%%%%%%%%%%%%%%%%%%%%%%%%%%%%%%%%%%%%%%%%%%%%%%%%%%%%%%%%%%%%%%%%%%%%%%%%%%
% %%%%%%%%%%%%%%%%%%%%%%%%%%%%%%%%%%%%%%%%%%%%%%%%%%%%%%%%%%%%%%%%%%%%%%%%%%%%%%
% \section{Sample}
%\iffalse
%<*samplemain>
%\fi
%
% The following presents a sample document
% with two chapters, two parts, a title page,
% a compile flag as well as three forwarding files to set the flag.
% It consists of eight |.tex| files:
% \begin{center}
% \begin{tabular}{ll}
% |cdocsamp.tex|&main file\\
% |cdocsch1.tex|&include file for chapter 1\\
% |cdocsch2.tex|&include file for chapter 2\\
% |cdocspt3.tex|&include file for part 3\\
% |cdocspt4.tex|&include file for part 4\\
% |cdocsdrf.tex|&forwarding file for main file in draft mode\\
% |cdocsfi1.tex|&forwarding file for final version of chapter 1\\
% |cdocsfi2.tex|&forwarding file for final version of chapter 2\\
% \end{tabular}
% \end{center}
% Each of the eight files can be compiled directly by the \LaTeX{} compiler.
%
% %%%%%%%%%%%%%%%%%%%%%%%%%%%%%%%%%%%%%%
% \paragraph{Main File.}
%
% The main file is called |cdocsamp.tex|.
%
% Load the \textsf{childdoc} definitions and
% declare the filename for the main document:
%    \begin{macrocode}
\input{childdoc.def}
\childdocmain{}
%    \end{macrocode}

% Optional override for |\version| flag:
%    \begin{macrocode}
%%\ifchilddoc\else\providecommand{\version}{draft}\fi
%    \end{macrocode}

% Define the default values for the |\version| flag
% (|final| for the main file and |draft| for childs):
%    \begin{macrocode}
\ifchilddoc
\providecommand{\version}{draft}
\else
\providecommand{\version}{final}
\fi
%    \end{macrocode}

% Load the standard document class:
%    \begin{macrocode}
\documentclass[12pt]{article}
%    \end{macrocode}

% Start the document body:
%    \begin{macrocode}
\begin{document}
%    \end{macrocode}

% Declare a title page.
% Print title, part of document being processed and version flag:
%    \begin{macrocode}
\addtocounter{page}{-1}
\begin{center}
{\LARGE\bfseries{}childdoc example\par}
\vspace{1cm}
\ifchilddoc
\ifchilddocmanual part\else chapter\fi:
`\childdocname' of `\childdocjob'\par
\else
main document: `\childdocjob'\par
\fi
version: \version\par
\end{center}
\newpage
%    \end{macrocode}

% Manually include selected file,
% otherwise process as usual:
%    \begin{macrocode}
\ifchilddocmanual
\section*{part `\childdocname'}
\input{\childdocname}
\else
%    \end{macrocode}

% Include the two chapters:
%    \begin{macrocode}
\include{cdocsch1}
\include{cdocsch2}
%    \end{macrocode}

% Include the two parts unless only chapters should be displayed:
%    \begin{macrocode}
\ifchilddoc\else
\section{part three}
\input{cdocspt3}
\section{part four}
\input{cdocspt4}
\fi
%    \end{macrocode}

% Process as usual until here:
%    \begin{macrocode}
\fi
%    \end{macrocode}

% End of document body:
%    \begin{macrocode}
\end{document}
%    \end{macrocode}
%\iffalse
%</samplemain>
%\fi
%
% %%%%%%%%%%%%%%%%%%%%%%%%%%%%%%%%%%%%%%
% \paragraph{Chapter Include Files.}
%
% The include files are called |cdocsch1.tex| and |cdocsch2.tex|.
%
%\iffalse
%<*samplechap1|samplechap2>
%\fi

% Optional override for |\version| flag:
%    \begin{macrocode}
%%\providecommand{\version}{final}
%    \end{macrocode}

% Include the main document:
%    \begin{macrocode}
\input{childdoc.def}
\childdocof{cdocsamp}
%    \end{macrocode}

%\iffalse
%</samplechap1|samplechap2>
%\fi
%
%\iffalse
%<*samplechap1>
%\fi
% Some text for chapter 1:
%    \begin{macrocode}
\section{one}
some text in chapter one
%    \end{macrocode}

%\iffalse
%</samplechap1>
%\fi
% Some text for chapter 2:
%\iffalse
%<*samplechap2>
%\fi
%    \begin{macrocode}
\section{two}
more text in chapter two
%    \end{macrocode}

%\iffalse
%</samplechap2>
%\fi
%
% %%%%%%%%%%%%%%%%%%%%%%%%%%%%%%%%%%%%%%
% \paragraph{Part Include Files.}
%
% The include files are called |cdocspt3.tex| and |cdocspt4.tex|.
%
%\iffalse
%<*samplepart3|samplepart4>
%\fi

% Optional override for |\version| flag:
%    \begin{macrocode}
%%\providecommand{\version}{final}
%    \end{macrocode}

% Include the main document:
%    \begin{macrocode}
\input{childdoc.def}
\childdocby{cdocsamp}
%    \end{macrocode}

%\iffalse
%</samplepart3|samplepart4>
%\fi
%
%\iffalse
%<*samplepart3>
%\fi
% Some text for part 3:
%    \begin{macrocode}
some text in part three
%    \end{macrocode}

%\iffalse
%</samplepart3>
%\fi
% Some text for part 4:
%\iffalse
%<*samplepart4>
%\fi
%    \begin{macrocode}
more text in part four
%    \end{macrocode}

%\iffalse
%</samplepart4>
%\fi
%
% %%%%%%%%%%%%%%%%%%%%%%%%%%%%%%%%%%%%%%
% \paragraph{Forwarding for a Complete Draft.}
%
% The following forwarding file |cdocsdrf.tex|
% compiles the main document in draft mode:
%\iffalse
%<*sampledraft>
%\fi
%    \begin{macrocode}
\def\version{draft}
\input{childdoc.def}
\childdocforward{cdocsamp}
%    \end{macrocode}

%\iffalse
%</sampledraft>
%\fi
%
% %%%%%%%%%%%%%%%%%%%%%%%%%%%%%%%%%%%%%%
% \paragraph{Forwarding for Final Version of the Chapters.}
%
% The following forwarding files |cdocsfn1.tex| and |cdocsfn2.tex|
% (with identical content)
% compile the final versions of the child documents
% |cdocsch1.tex| and |cdocsch2.tex|, respectively:
%\iffalse
%<*samplefinal>
%\fi
%    \begin{macrocode}
\def\version{final}
\input{childdoc.def}
\childdocforwardprefix[cdocsamp]{cdocsfn}{cdocsch}
%    \end{macrocode}

%\iffalse
%</samplefinal>
%\fi
%
% %%%%%%%%%%%%%%%%%%%%%%%%%%%%%%%%%%%%%%
% \paragraph{Command Line Processing.}
%
% The following three command lines generate the output files
% |cdocscld|, |cdocscl1| and |cdocscl2|
% which should be identical to
% |cdocsdrf|, |cdocsch1| and |cdocsfn2|, respectively:
% \begin{center}
% \begin{tabular}{l}
% |latex -jobname cdocscld \|\\
% |  "\def\version{draft}\input{childdoc.def}\childdocforward{cdocsamp}"|\\
% |latex -jobname cdocscl1 \|\\
% |  "\input{childdoc.def}\childdocforward[cdocsamp]{cdocsch1}"|\\
% |latex -jobname cdocscl2 \|\\
% |  "\def\version{final}\input{childdoc.def}\childdocforward{cdocsch2}"|
% \end{tabular}
% \end{center}
% Note that the trailing backslash on each first line
% merely continues the input to the second line
% (for convenient cut ant paste).
% Furthermore, the command |latex| can be replaced by any
% of its alternative versions such as |pdflatex|.
%
% %%%%%%%%%%%%%%%%%%%%%%%%%%%%%%%%%%%%%%%%%%%%%%%%%%%%%%%%%%%%%%%%%%%%%%%%%%%%%%
% %%%%%%%%%%%%%%%%%%%%%%%%%%%%%%%%%%%%%%%%%%%%%%%%%%%%%%%%%%%%%%%%%%%%%%%%%%%%%%
% \section{Implementation}
%\iffalse
%<*package>
%\fi
%
% This section describes the definitions file |childdoc.def|.

% The definitions cannot be loaded using |\usepackage| or |\RequirePackage|
% which has a mechanism to prevent loading a style file more than once.
% When loading the definitions by means of |\input|
% multiple instances have to be prevented manually:
%\iffalse
%This code needs to be before the `\ProvidesFile' directive
%which is defined at the beginning of this file.
%Therefore it is also placed there and commented out here.
%</package>
%<*discard>
%\fi
%    \begin{macrocode}
\ifdefined\childdocmain\endinput\fi
%    \end{macrocode}
%\iffalse
%</discard>
%<*package>
%\fi
%
% \macro{\ifchilddoc}
% \macro{\ifchilddocmanual}
% The conditional |\ifchilddoc| tells whether a
% child (true) or main (false) document is being compiled.
% The conditional |\ifchilddocmanual| tells whether
% the |\includeonly| mechanism is used (false) or
% the selection of child files must be performed manually (true).
% The definitions initialise to false:
%    \begin{macrocode}
\newif\ifchilddoc
\newif\ifchilddocmanual
%    \end{macrocode}

% \macro{\childdocname}
% \macro{\childdocjob}
% The macro |\childdocname| stores the name of the main document
% to be compiled. The macro |\childdocjob| stores the name of
% the document on which the \LaTeX{} compiler was originally invoked.
% The content of |\jobname| cannot be compared
% to filenames specified in the source due to different catcodes.
% The following code rescans |\jobname|, stores the result
% in |\childdocname| and saves a copy in |\childdocjob|:
%    \begin{macrocode}
\edef\childdocname{\scantokens\expandafter{\jobname\noexpand}}
\let\childdocjob\childdocname
%    \end{macrocode}

% \macro{\childdocdisable}
% The macro |\childdocdisable| prevents the main file
% from being processed more than once.
% At this stage, the main document command |\childdocmain|
% is assumed to be called once again where it should do nothing.
% Any subsequent call to it should prevent
% a secondary processing of the main document
% It overwrites the forwarding commands
% |\childdocof| and |\childdocforward|
% with empty macros to prevent further inclusions of the main document:
%    \begin{macrocode}
\newcommand{\childdocdisable}
{
  \renewcommand{\childdocmain}[1]{\renewcommand{\childdocmain}[1]{\endinput}}
  \renewcommand{\childdocof}[1]{}
  \renewcommand{\childdocby}[2][]{}
  \renewcommand{\childdocforward}[2][]{}
  \renewcommand{\childdocdisable}{}
}
%    \end{macrocode}

% \macro{\childdocmain}
% The macro |\childdocmain| is to be called at the top of the main file
% with nothing or the main filename (without extension) as argument.
% First, it breaks loops.
% If the argument is not empty and does not match |\childdocname|
% (which is set by the first inclusion of |childdoc.def|),
% |\ifchilddoc| is set to true, |\includeonly| is applied to the child file
% and |\jobname| is set to the main file
% (for proper handling of |.aux| files):
%    \begin{macrocode}
\newcommand{\childdocmain}[1]
{
  \childdocdisable\childdocmain{}
  \if?#1?\else
    \begingroup
      \def\childdoctmp{#1}
      \ifx\childdoctmp\childdocname
        \def\childdoctmp{}
      \else
        \def\childdoctmp
        {
          \childdoctrue
          \includeonly{\childdocname}
          \def\childdocjob{#1}
          \def\jobname{#1}
        }
      \fi
      \expandafter
    \endgroup
    \childdoctmp
  \fi
}
%    \end{macrocode}

% \macro{\childdocof}
% The command |\childdocof| redirects
% compilation to the main file |#1|.
%    \begin{macrocode}
\newcommand{\childdocof}[1]
{
  \childdocdisable
  \childdoctrue
  \includeonly{\childdocname}
  \def\jobname{#1}
  \def\childdocjob{#1}
  \input{#1}
}
%    \end{macrocode}

% \macro{\childdocby}
% The command |\childdocby| ....
%    \begin{macrocode}
\newcommand{\childdocby}[2][]
{
  \childdocdisable
  \childdoctrue
  \childdocmanualtrue
  \if?#1?\else
    \def\jobname{#2}
  \fi
  \def\childdocjob{#2}
  \input{#2}
  \endinput
}
%    \end{macrocode}

% \macro{\childdocforward}
% The command |\childdocforward| redirects
% compilation to the main file or
% (if the optional argument is given) a child file.
% Parameters are set as if the main file
% or a child file starting with |\childdocof| was compiled.
% Then compilation is handed over to the main file:
%    \begin{macrocode}
\newcommand{\childdocforward}[2][]
{
  \begingroup
    \if?#1?
      \def\childdoctmp
      {
        \def\childdocname{#2}
        \def\childdocjob{#2}
        \def\jobname{#2}
        \input{#2}
        \endinput
      }
    \else
      \def\childdoctmp
      {
        \childdocdisable
        \def\childdocname{#2}
        \childdoctrue
        \includeonly{#2}
        \def\childdocjob{#1}
        \def\jobname{#1}
        \input{#1}
        \endinput
      }
    \fi
    \expandafter
  \endgroup
  \childdoctmp
}
%    \end{macrocode}

% \macro{\childdocforwardprefix}
% The command |\childdocforwardprefix| redirects
% compilation to the main or a child file by means of a pattern.
% The prefix |#1| in the current filename is replaced by |#2|
% and the suffix of the current filename is kept
% (it is assumed that the filename does not contain the substring `|~~~|'
% which is used as a delimiter).
% Compilation is handed over to the new file by |\childdocforward|:
%    \begin{macrocode}
\newcommand{\childdocforwardprefix}[3][]
{
  \begingroup
    \def\childdocextract #2##1~~~{\def\childdoctmp{\childdocforward[#1]{#3##1}}}
    \expandafter\childdocextract\childdocname~~~
    \expandafter
  \endgroup
  \childdoctmp
}
%    \end{macrocode}

% \macro{\childdoc}
% The deprecated macro |\childdoc| is a legacy version of |\childdocmain|:
%    \begin{macrocode}
\newcommand{\childdoc}{\childdocmain}
%    \end{macrocode}

% \macro{\childdocredirect}
% The deprecated macro |\childdocredirect| is a legacy version
% of |\childdocforward| and |\childdocforwardprefix|:
%    \begin{macrocode}
\newcommand{\childdocredirect}[2][]
{
  \begingroup
    \if?#1?
      \def\childdoctmp{\childdocforward{#2}}
    \else
      \def\childdoctmp{\childdocforwardprefix{#1}{#2}}
    \fi
    \expandafter
  \endgroup
  \childdoctmp
}
%    \end{macrocode}

%\iffalse
%</package>
%\fi
%
\endinput
|\\
|\childdocof{|\textit{main}|}|\\
\end{tabular}
\end{center}
at the top of every child file \textit{child}
which is included by |\include{|\textit{child}|}|
from within the main file
(or at least for those files to be compiled individually).
The argument \textit{main} must be the filename of the main file.

There are a couple of
considerations in setting up the main and child documents:

%%%%%%%%%%%%%%%%%%%%%%%%%%%%%%%%%%%%%%%%
\paragraph{Restrictions.}

Please note the following restrictions:
\begin{itemize}
\item
|\childdocmain| must be called with one argument \textit{main}
to ensure compatibility with earlier version of the package.
It must either be empty (|\childdocmain{}|)
or precisely match the filename of the main file in which it is specified.
See \secref{sec:detection} for further information.
\item
The filename \textit{main} must be specified without the |.tex| extension.
\item
The filename \textit{main} is case sensitive
(even in case-insensitive file systems)
due to internal string comparison.
\item
The argument \textit{main} should be fully expanded, it cannot be a macro.
\item
Subdirectories and special characters should be avoided in filenames.
\item
The command |\childdocmain{|\textit{main}|}| must be followed by a whitespace.
It should not be followed immediately by another command
or by a comment mark `|%|'.
This is because the \TeX{} parser reads the token immediately following
the argument of |\childdocmain| and puts it
at the beginning of every child section;
however, a white\-space is ignored.
\end{itemize}

%%%%%%%%%%%%%%%%%%%%%%%%%%%%%%%%%%%%%%%%
\paragraph{Content of Main File.}

It is advisable to place all content in the child files included by |\include|.
Any output contained in the main file will appear in all child documents
unless suppressed manually;
it cannot be suppressed automatically by the |\includeonly| directive
and thus should normally be avoided.
A method to include some content in the main file
by means of conditional processing is described in \secref{sec:conditional}.

%%%%%%%%%%%%%%%%%%%%%%%%%%%%%%%%%%%%%%%%
\paragraph{Page Numbering.}

When only a part of the document is compiled,
the appropriate numbering of pages
(as well as other status parameters)
is determined from the |.aux| files.
The latter contain information from previous passes.
However this information needs to propagate through
all intermediate child documents.
Therefore the page numbering in child documents may well
be inconsistent until the complete document is compiled at least once.

A useful (if unconventional) way to always ensure a consistent
page numbering is to restart the numbering in each child document
and denote the pages by `\textit{child}|.|\textit{page}'
where \textit{child} represents the chapter/section number of the child file.
This can be achieved by the command
|\numberwithin{page}{|\textit{child}|}|
of the \textsf{amsmath} package
where \textit{child} can be |chapter| or |section|
depending on the chosen structuring.
Alternatively, one can modify the macro |\thepage| appropriately
and reset the counter |page| at the start of each child file.

%%%%%%%%%%%%%%%%%%%%%%%%%%%%%%%%%%%%%%%%%%%%%%%%%%%%%%%%%%%%%%%%%%%%%%%%%%%%%%%%
\subsection{Conditional Processing}
\label{sec:conditional}

The package provides a mechanism to compile different versions
of a document. To customise the versions further some conditional processing
can come in handy to distinguish which version is being compiled.
The package provides two macros to describe the compilation context:

%%%%%%%%%%%%%%%%%%%%%%%%%%%%%%%%%%%%%%%%
\DescribeMacro{\ifchilddoc}
The conditional |\ifchilddoc| distinguishes between the compilation of
child documents and the main document:
%
\begin{center}
|\ifchilddoc |\textit{child-code}| |[|\||else |\textit{main-code}]| \||fi|
\end{center}

%%%%%%%%%%%%%%%%%%%%%%%%%%%%%%%%%%%%%%%%
\DescribeMacro{\childdocname}
\DescribeMacro{\childdocjob}
The macro |\childdocname| contains the filename (without extension)
of the main or child file being processed.
Note that |\childdocjob| will always contain the name of the main file.

%%%%%%%%%%%%%%%%%%%%%%%%%%%%%%%%%%%%%%%%
\paragraph{Title Page.}

Conditional processing can be used to include a title or banner page
in the main document when proper precautions are taken.
Importantly, the code in the main file should ensure that the page counter
(as well as other status parameters which are stored in the |.aux| files)
takes the same value after the conditional processing.
Otherwise the page numbers may take divergent values
depending on which part is compiled.

For example, a title page could be declared by:
%
\begin{center}
\begin{tabular}{l}
|\ifchilddoc\||else|\\
|\addtocounter{page}{-1}|\\
\textit{code for title page}\\
|\newpage|\\
|\||fi|
\end{tabular}
\end{center}
%
A banner page for the child documents can be generated by:
%
\begin{center}
\begin{tabular}{l}
|\ifchilddoc|\\
|\addtocounter{page}{-1}|\\
\textit{code for banner page}\\
|\newpage|\\
|\||fi|
\end{tabular}
\end{center}
%
Here one could write a message such as:
\begin{center}
|This is the part \childdocname{} of \childdocjob{}.|
\end{center}

%%%%%%%%%%%%%%%%%%%%%%%%%%%%%%%%%%%%%%%%%%%%%%%%%%%%%%%%%%%%%%%%%%%%%%%%%%%%%%%%
\subsection{Flags}
\label{sec:flags}

The package makes it easy to generate different versions
of the main or child documents.
To this end compilation flags can be defined
and assigned different default values.
They will be particularly useful in conjunction
with the forwarding mechanism described in \secref{sec:forward}.

For example, it may be useful to have a flag |\version|
which can be set to |draft| or |final|.
The document source will contain some conditional code
depending on the value of |\version|.
Suppose further, the flag should default to |final| for the main file
and to |draft| for child files
which is a natural assignment for editing the document.
This is achieved by placing the following code
in the preamble of the main document
(below the |\childdocmain| directive):
%
\begin{center}
\begin{tabular}{l}
|\ifchilddoc|\\
|\providecommand{\version}{draft}|\\
|\||else|\\
|\providecommand{\version}{final}|\\
|\||fi|
\end{tabular}
\end{center}
%
The definition by |\providecommand| makes sure
that previous definitions are not overwritten.
Further statements |\providecommand{\version}{...}|
can thus be added before the above code to override it.

For the main file, one might add a line
(between |\childdocmain| and the above block)
%
\begin{center}
|%\ifchilddoc\||else\providecommand{\version}{draft}\||fi|
\end{center}
%
which can be uncommented to produce a draft version.
Likewise one can add a line to the very top of a child file
(above the |\childdocof{|\textit{main}|}| directive)
%
\begin{center}
|%\providecommand{\version}{final}|
\end{center}
%
which can be uncommented to produce the final version of this child document.

%%%%%%%%%%%%%%%%%%%%%%%%%%%%%%%%%%%%%%%%%%%%%%%%%%%%%%%%%%%%%%%%%%%%%%%%%%%%%%%%
\subsection{Forwarding}
\label{sec:forward}

Different versions of the main or child documents
using compilation flags as described in \secref{sec:flags}
can be (permanently) stored in different files
for convenient compilation, viewing and distribution.
To this end, the package defines a command
to pass on compilation to a different file:

%%%%%%%%%%%%%%%%%%%%%%%%%%%%%%%%%%%%%%%%
\DescribeMacro{\childdocforward}
The command |\childdocforward| redirects processing to
another source file:
%
\begin{center}
\begin{tabular}{l}
|% \iffalse
%
% childdoc.dtx Copyright (C) 2017-2018 Niklas Beisert
%
% This work may be distributed and/or modified under the
% conditions of the LaTeX Project Public License, either version 1.3
% of this license or (at your option) any later version.
% The latest version of this license is in
%   http://www.latex-project.org/lppl.txt
% and version 1.3 or later is part of all distributions of LaTeX
% version 2005/12/01 or later.
%
% This work has the LPPL maintenance status `maintained'.
%
% The Current Maintainer of this work is Niklas Beisert.
%
% This work consists of the files childdoc.dtx and childdoc.ins
% and the derived files childdoc.def and cdocsamp.tex with
% cdocsch1.tex, cdocsch2.tex, cdocsdrf.tex, cdocsfn1.tex, cdocsfn2.tex.
%
%<package>\ifdefined\childdocmain\endinput\fi
%<package>\ProvidesFile{childdoc.def}[2018/12/30 v2.0 child document driver]
%<samplemain>\ProvidesFile{cdocsamp.tex}[2018/12/30 v2.0 sample for childdoc]
%<*driver>
%\ProvidesFile{childdoc.drv}[2018/12/30 v2.0 childdoc reference manual file]
\PassOptionsToClass{10pt,a4paper}{article}
\documentclass{ltxdoc}

\usepackage[margin=35mm]{geometry}
\usepackage{hyperref}
\usepackage{hyperxmp}
\usepackage[usenames]{color}

\hypersetup{colorlinks=true}
\hypersetup{pdfstartview=FitH}
\hypersetup{pdfpagemode=UseNone}
\hypersetup{pdfsource={}}
\hypersetup{pdflang={en-UK}}
\hypersetup{pdfcopyright={Copyright 2017-2018 Niklas Beisert.
  This work may be distributed and/or modified under the
  conditions of the LaTeX Project Public License, either version 1.3
  of this license or (at your option) any later version.}}
\hypersetup{pdflicenseurl={http://www.latex-project.org/lppl.txt}}
\hypersetup{pdfcontactaddress={ETH Zurich, ITP, HIT K,
  Wolfgang-Pauli-Strasse 27}}
\hypersetup{pdfcontactpostcode={8093}}
\hypersetup{pdfcontactcity={Zurich}}
\hypersetup{pdfcontactcountry={Switzerland}}
\hypersetup{pdfcontactemail={nbeisert@itp.phys.ethz.ch}}
\hypersetup{pdfcontacturl={http://people.phys.ethz.ch/\xmptilde nbeisert/}}

\newcommand{\secref}[1]{\hyperref[#1]{section \ref*{#1}}}

\parskip1ex
\parindent0pt
\let\olditemize\itemize
\def\itemize{\olditemize\parskip0pt}

\begin{document}

\title{The \textsf{childdoc} Package}
\hypersetup{pdftitle={The childdoc Package}}
\author{Niklas Beisert\\[2ex]
  Institut f\"ur Theoretische Physik\\
  Eidgen\"ossische Technische Hochschule Z\"urich\\
  Wolfgang-Pauli-Strasse 27, 8093 Z\"urich, Switzerland\\[1ex]
  \href{mailto:nbeisert@itp.phys.ethz.ch}
  {\texttt{nbeisert@itp.phys.ethz.ch}}}
\hypersetup{pdfauthor={Niklas Beisert}}
\hypersetup{pdfsubject={Manual for the LaTeX2e Package childdoc}}
\date{30 December 2018, \textsf{v2.0}}
\maketitle

\begin{abstract}\noindent
\textsf{childdoc} is a \LaTeXe{} package
that enables the direct compilation
of document sections included by |\include|
to individual files.
\end{abstract}

\begingroup
\parskip0ex
\tableofcontents
\endgroup

%%%%%%%%%%%%%%%%%%%%%%%%%%%%%%%%%%%%%%%%%%%%%%%%%%%%%%%%%%%%%%%%%%%%%%%%%%%%%%%%
%%%%%%%%%%%%%%%%%%%%%%%%%%%%%%%%%%%%%%%%%%%%%%%%%%%%%%%%%%%%%%%%%%%%%%%%%%%%%%%%
\section{Introduction}

\LaTeX{} provides a mechanism to structure a large document (such as a book)
into a main file and several child files (containing the chapters)
using the |\include| command.
This mechanism is beneficial for documents
which span hundreds of pages in order to
make the source file(s) more manageable.
Moreover, compilation can be restricted to
selected child files by means of the |\includeonly| command.
The latter feature can be used to reduce the compilation time while editing
(this was significantly more useful in the earlier days of \LaTeX{})
or to generate a smaller document which is easier to navigate.
Another application of |\includeonly| is to generate
documents consisting of selected parts of the complete document.

However, there are a few drawbacks of the plain |\include| mechanism:
\begin{itemize}
\item
The child files cannot be compiled on their own,
they can only be compiled via the main file.
A naive editing environment
(such as a text editor with an option
to have the current file processed by \LaTeX)
may require one to switch to the main file before compiling;
attempting to compile the child file produces errors.
\item
The main file must be modified (each time)
to adjust the |\includeonly| command
to the present needs. This easily leaves the main file in a messy state.
\item
The generated document will always carry the filename
of the main document. This is inconvenient if
several child files are to be compiled and
to be kept for distribution.
\end{itemize}

The present package provides a simple interface
to make child files individually compilable by \LaTeX{}.
Compiling a child file then has the same effect as compiling
the main file with an |\includeonly| command
to select the appropriate child.
Moreover the generated document will carry the name of the child
rather than the main file.
This resolves all three above issues.

This feature is meant to make the editing of books,
thesis documents and lecture notes somewhat more convenient.
However, the package can also be used efficiently for
composing a series of documents (such as exercise sheets)
which are typically distributed individually.
It then assists the author in generating the individual documents
(potentially in different versions)
as well as a document containing the collected series.
Another application is in developing style files
or other kinds of included material
where compilation of the style file could redirect
to a sample or test file.

%%%%%%%%%%%%%%%%%%%%%%%%%%%%%%%%%%%%%%%%%%%%%%%%%%%%%%%%%%%%%%%%%%%%%%%%%%%%%%%%
%%%%%%%%%%%%%%%%%%%%%%%%%%%%%%%%%%%%%%%%%%%%%%%%%%%%%%%%%%%%%%%%%%%%%%%%%%%%%%%%
\section{Usage}

First of all, the package \textsf{childdoc} is \emph{not} a standard
\LaTeXe{} |.sty| style file! Therefore it needs to be invoked in
a non-standard way.

%%%%%%%%%%%%%%%%%%%%%%%%%%%%%%%%%%%%%%%%%%%%%%%%%%%%%%%%%%%%%%%%%%%%%%%%%%%%%%%%
\subsection{Included Files}
\label{sec:include}

%%%%%%%%%%%%%%%%%%%%%%%%%%%%%%%%%%%%%%%%
\DescribeMacro{\childdocmain}
To use the package, add the commands
\begin{center}
\begin{tabular}{l}
|\input{childdoc.def}|\\
|\childdocmain{}|\\
\end{tabular}
\end{center}
at the very top of the main \LaTeX{} file,
in particular \emph{before} the |\documentclass| statement!
The argument of |\childdocmain| should be left empty
(but it must be present).

%%%%%%%%%%%%%%%%%%%%%%%%%%%%%%%%%%%%%%%%
\DescribeMacro{\childdocof}
Furthermore, add the commands
\begin{center}
\begin{tabular}{l}
|\input{childdoc.def}|\\
|\childdocof{|\textit{main}|}|\\
\end{tabular}
\end{center}
at the top of every child file \textit{child}
which is included by |\include{|\textit{child}|}|
from within the main file
(or at least for those files to be compiled individually).
The argument \textit{main} must be the filename of the main file.

There are a couple of
considerations in setting up the main and child documents:

%%%%%%%%%%%%%%%%%%%%%%%%%%%%%%%%%%%%%%%%
\paragraph{Restrictions.}

Please note the following restrictions:
\begin{itemize}
\item
|\childdocmain| must be called with one argument \textit{main}
to ensure compatibility with earlier version of the package.
It must either be empty (|\childdocmain{}|)
or precisely match the filename of the main file in which it is specified.
See \secref{sec:detection} for further information.
\item
The filename \textit{main} must be specified without the |.tex| extension.
\item
The filename \textit{main} is case sensitive
(even in case-insensitive file systems)
due to internal string comparison.
\item
The argument \textit{main} should be fully expanded, it cannot be a macro.
\item
Subdirectories and special characters should be avoided in filenames.
\item
The command |\childdocmain{|\textit{main}|}| must be followed by a whitespace.
It should not be followed immediately by another command
or by a comment mark `|%|'.
This is because the \TeX{} parser reads the token immediately following
the argument of |\childdocmain| and puts it
at the beginning of every child section;
however, a white\-space is ignored.
\end{itemize}

%%%%%%%%%%%%%%%%%%%%%%%%%%%%%%%%%%%%%%%%
\paragraph{Content of Main File.}

It is advisable to place all content in the child files included by |\include|.
Any output contained in the main file will appear in all child documents
unless suppressed manually;
it cannot be suppressed automatically by the |\includeonly| directive
and thus should normally be avoided.
A method to include some content in the main file
by means of conditional processing is described in \secref{sec:conditional}.

%%%%%%%%%%%%%%%%%%%%%%%%%%%%%%%%%%%%%%%%
\paragraph{Page Numbering.}

When only a part of the document is compiled,
the appropriate numbering of pages
(as well as other status parameters)
is determined from the |.aux| files.
The latter contain information from previous passes.
However this information needs to propagate through
all intermediate child documents.
Therefore the page numbering in child documents may well
be inconsistent until the complete document is compiled at least once.

A useful (if unconventional) way to always ensure a consistent
page numbering is to restart the numbering in each child document
and denote the pages by `\textit{child}|.|\textit{page}'
where \textit{child} represents the chapter/section number of the child file.
This can be achieved by the command
|\numberwithin{page}{|\textit{child}|}|
of the \textsf{amsmath} package
where \textit{child} can be |chapter| or |section|
depending on the chosen structuring.
Alternatively, one can modify the macro |\thepage| appropriately
and reset the counter |page| at the start of each child file.

%%%%%%%%%%%%%%%%%%%%%%%%%%%%%%%%%%%%%%%%%%%%%%%%%%%%%%%%%%%%%%%%%%%%%%%%%%%%%%%%
\subsection{Conditional Processing}
\label{sec:conditional}

The package provides a mechanism to compile different versions
of a document. To customise the versions further some conditional processing
can come in handy to distinguish which version is being compiled.
The package provides two macros to describe the compilation context:

%%%%%%%%%%%%%%%%%%%%%%%%%%%%%%%%%%%%%%%%
\DescribeMacro{\ifchilddoc}
The conditional |\ifchilddoc| distinguishes between the compilation of
child documents and the main document:
%
\begin{center}
|\ifchilddoc |\textit{child-code}| |[|\||else |\textit{main-code}]| \||fi|
\end{center}

%%%%%%%%%%%%%%%%%%%%%%%%%%%%%%%%%%%%%%%%
\DescribeMacro{\childdocname}
\DescribeMacro{\childdocjob}
The macro |\childdocname| contains the filename (without extension)
of the main or child file being processed.
Note that |\childdocjob| will always contain the name of the main file.

%%%%%%%%%%%%%%%%%%%%%%%%%%%%%%%%%%%%%%%%
\paragraph{Title Page.}

Conditional processing can be used to include a title or banner page
in the main document when proper precautions are taken.
Importantly, the code in the main file should ensure that the page counter
(as well as other status parameters which are stored in the |.aux| files)
takes the same value after the conditional processing.
Otherwise the page numbers may take divergent values
depending on which part is compiled.

For example, a title page could be declared by:
%
\begin{center}
\begin{tabular}{l}
|\ifchilddoc\||else|\\
|\addtocounter{page}{-1}|\\
\textit{code for title page}\\
|\newpage|\\
|\||fi|
\end{tabular}
\end{center}
%
A banner page for the child documents can be generated by:
%
\begin{center}
\begin{tabular}{l}
|\ifchilddoc|\\
|\addtocounter{page}{-1}|\\
\textit{code for banner page}\\
|\newpage|\\
|\||fi|
\end{tabular}
\end{center}
%
Here one could write a message such as:
\begin{center}
|This is the part \childdocname{} of \childdocjob{}.|
\end{center}

%%%%%%%%%%%%%%%%%%%%%%%%%%%%%%%%%%%%%%%%%%%%%%%%%%%%%%%%%%%%%%%%%%%%%%%%%%%%%%%%
\subsection{Flags}
\label{sec:flags}

The package makes it easy to generate different versions
of the main or child documents.
To this end compilation flags can be defined
and assigned different default values.
They will be particularly useful in conjunction
with the forwarding mechanism described in \secref{sec:forward}.

For example, it may be useful to have a flag |\version|
which can be set to |draft| or |final|.
The document source will contain some conditional code
depending on the value of |\version|.
Suppose further, the flag should default to |final| for the main file
and to |draft| for child files
which is a natural assignment for editing the document.
This is achieved by placing the following code
in the preamble of the main document
(below the |\childdocmain| directive):
%
\begin{center}
\begin{tabular}{l}
|\ifchilddoc|\\
|\providecommand{\version}{draft}|\\
|\||else|\\
|\providecommand{\version}{final}|\\
|\||fi|
\end{tabular}
\end{center}
%
The definition by |\providecommand| makes sure
that previous definitions are not overwritten.
Further statements |\providecommand{\version}{...}|
can thus be added before the above code to override it.

For the main file, one might add a line
(between |\childdocmain| and the above block)
%
\begin{center}
|%\ifchilddoc\||else\providecommand{\version}{draft}\||fi|
\end{center}
%
which can be uncommented to produce a draft version.
Likewise one can add a line to the very top of a child file
(above the |\childdocof{|\textit{main}|}| directive)
%
\begin{center}
|%\providecommand{\version}{final}|
\end{center}
%
which can be uncommented to produce the final version of this child document.

%%%%%%%%%%%%%%%%%%%%%%%%%%%%%%%%%%%%%%%%%%%%%%%%%%%%%%%%%%%%%%%%%%%%%%%%%%%%%%%%
\subsection{Forwarding}
\label{sec:forward}

Different versions of the main or child documents
using compilation flags as described in \secref{sec:flags}
can be (permanently) stored in different files
for convenient compilation, viewing and distribution.
To this end, the package defines a command
to pass on compilation to a different file:

%%%%%%%%%%%%%%%%%%%%%%%%%%%%%%%%%%%%%%%%
\DescribeMacro{\childdocforward}
The command |\childdocforward| redirects processing to
another source file:
%
\begin{center}
\begin{tabular}{l}
|\input{childdoc.def}|\\
|\childdocforward[|\textit{main}|]{|\textit{dest}|}|\\
\end{tabular}
\end{center}
%
The argument \textit{dest} is the destination file
(without extension).
It should be the main file or one of the child files.
Note that further \textsf{childdoc} directives
such as |\childdocof| and |\childdocforward|
in the indicated file will be processed in this form.
The optional argument \textit{main}
passes on directly to the main file \textit{main}
while pretending to compile the child \textit{dest}.
This form behaves as if \textit{dest}
issues |\childdocof{|\textit{main}|}| right away,
and no further \textsf{childdoc} directives will be processed.

%%%%%%%%%%%%%%%%%%%%%%%%%%%%%%%%%%%%%%%%
\DescribeMacro{\...prefix}
In the alternative form |\childdocforwardprefix|,
%
\begin{center}
\begin{tabular}{l}
|\input{childdoc.def}|\\
|\childdocforwardprefix[|\textit{main}|]{|\textit{prefix}|}{|\textit{dest}|}|
\end{tabular}
\end{center}
%
the destination file is determined by a pattern
depending on the current file:
To make this work, the current file must be called
`{\textit{prefix}\hspace{0.2em}\textit{suffix}}'
with \textit{prefix} matching precisely the argument.
Processing is then passed on to the file
`{\textit{dest}\hspace{0.2em}\textit{suffix}}'.
Surely, the same effect is achieved by
directly specifying the
argument `{\textit{dest}\hspace{0.2em}\textit{suffix}}'
in the first form.
However, that requires to set up a different file
for each child. With the alternative form of the command
all these files can have exactly the same content
which simplifies setting them up and maintaining them.

For example, the following file |draft.tex|
with a compilation flag |\version| as described in \secref{sec:flags}
compiles the main document as a draft:
%
\begin{center}
\begin{tabular}{l}
|\def\version{draft}|\\
|\input{childdoc.def}|\\
|\childdocforward{|\textit{main}|}|
\end{tabular}
\end{center}
%
Likewise, the following files |final|\textit{nn}|.tex|
compile the final version of the child document
|child|\textit{nn}|.tex|:
%
\begin{center}
\begin{tabular}{l}
|\def\version{final}|\\
|\input{childdoc.def}|\\
|\childdocforwardprefix{final}{child}|
\end{tabular}
\end{center}
%

Note that when several versions of a main file and/or of each child file
are to be generated, it may be convenient to set up a |Makefile| or
shell script to automatise the process.

%%%%%%%%%%%%%%%%%%%%%%%%%%%%%%%%%%%%%%%%%%%%%%%%%%%%%%%%%%%%%%%%%%%%%%%%%%%%%%%%
\subsection{Command Line Processing}
\label{sec:commandline}

The effect of redirection files can also be achieved by invoking
the \LaTeX{} compiler with a more elaborate command line.
Most conveniently this should be done as part
of a shell script or a |Makefile|.

When using \textsf{childdoc} in the main file, the following
command lines effectively perform a redirection
(note that depending on the shell being used,
backslashes may have to be doubled: `|\|' $\to$ `|\\|'):
%
\begin{center}
|... -jobname "|\textit{target}|" |\\|"|[\textit{flags}]%
|\input{childdoc.def}\childdocforward[|\textit{main}|]{|\textit{dest}|}"|
\end{center}
%
Here \textit{target} is the name of the output file,
\textit{main} is the name of the main file
and \textit{dest} is the name of the main or child file to be processed
(all filenames without extensions).
The optional argument \textit{main} can be omitted
if \textit{main} matches \textit{dest}.
Optionally, compilation \textit{flags} can be defined via |\def| commands.
This command line makes the \TeX{} engine believe
it is compiling the file \textit{target}
whose content is specified as the latter parameter.
The provided code then forwards the processing to
\textit{main} or \textit{dest} as described in \secref{sec:forward}.

%%%%%%%%%%%%%%%%%%%%%%%%%%%%%%%%%%%%%%%%%%%%%%%%%%%%%%%%%%%%%%%%%%%%%%%%%%%%%%%%
\subsection{Include by Input}
\label{sec:input}

Including child documents by |\include| has some restrictions by design.
Most notably, the content of a child document always occupies
its own set of pages; pages cannot be shared between child documents.
Usually, this behaviour makes perfect sense
because each child document contain an essential part of the document.
However, in some situations it may be desirable to compose
a document from a collection of parts
without having mandatory page breaks between then.
For this case, the package
provides a mechanism to include parts
by |\input| which can also be processed individually.
However, by construction this mechanism
requires manual handling of the content to be output.

%%%%%%%%%%%%%%%%%%%%%%%%%%%%%%%%%%%%%%%%
\DescribeMacro{\ifchilddocmanual}
The main file should be prepared as usual, see \secref{sec:include}.
However, the document body must make a distinction
between processing of an individual part and of the main document, e.g.:
%
\begin{center}
\begin{tabular}{l}
|\ifchilddocmanual|\\
|\input{\childdocname}|\\
|\||else|\\
\textit{document body with }|\input{|\textit{part}|}|\\
|\||fi|
\end{tabular}
\end{center}
%
The conditional |\ifchilddocmanual| is true whenever
a part to be included by |\input| is being compiled,
and the name of the part is stored in |\childdocname|.

%%%%%%%%%%%%%%%%%%%%%%%%%%%%%%%%%%%%%%%%
\DescribeMacro{\childdocby}
Each part to be included by |\input| should start with:
%
\begin{center}
\begin{tabular}{l}
|\input{childdoc.def}|\\
|\childdocby{|\textit{main}|}|\\
\end{tabular}
\end{center}
%
The directive |\childdocby| is similar to |\childdocof|
described in \secref{sec:include},
but the subsequent selection of content must be done manually.
To that end, both |\ifchilddoc| and |\ifchilddocmanual|
will be true upon processing of a part,
and the name of the part is stored in |\childdocname|.
Note that |\jobname| will be set to the filename of the current part
so that each part receives an individual |.aux| file
that does not interfere with the |.aux| file(s) of the main document.
This behaviour can be altered by the alternative form
|\childdocby[*]{|\textit{main}|}| (with a non-empty optional argument)
which uses the |.aux| file of the main document
by setting |\jobname| to \textit{main}.

%%%%%%%%%%%%%%%%%%%%%%%%%%%%%%%%%%%%%%%%%%%%%%%%%%%%%%%%%%%%%%%%%%%%%%%%%%%%%%%%
\subsection{Driver Development}
\label{sec:driver}

The \textsf{childdoc} mechanism can also be use for the development
of definition files such as \LaTeX{} styles or classes.
This case differs from the above setup with multiple parts
included by |\include| in that no |\includeonly| should be invoked.
This can be achieved by starting the include file
(before |\ProvidesPackage|) with:
%
\begin{center}
\begin{tabular}{l}
|\input{childdoc.def}|\\
|\childdocforward{|\textit{main}|}|\\
\end{tabular}
\end{center}
%
or alternatively with:
%
\begin{center}
\begin{tabular}{l}
|\input{childdoc.def}|\\
|\childdocby{|\textit{main}|}|\\
\end{tabular}
\end{center}
%
Both forms have slightly different effects as described above.
The main file is prepared as usual, see \secref{sec:include}.

%%%%%%%%%%%%%%%%%%%%%%%%%%%%%%%%%%%%%%%%%%%%%%%%%%%%%%%%%%%%%%%%%%%%%%%%%%%%%%%%
\subsection{Legacy Detection}
\label{sec:detection}

The directive |\childdocmain| in the main file can detect
whether the complete document or merely a child is to be compiled
even without using the directive |\childdocof|.
This method is deprecated because it is less robust
and there is no compelling reason to use it;
it is merely provided for backward compatibility
and it may be removed in future versions.

If the detection mechanism is to be used,
it is mandatory to correctly specify
the filename of the main file as the argument of |\childdocmain|:
%
\begin{center}
\begin{tabular}{l}
|\input{childdoc.def}|\\
|\childdocmain{|\textit{main}|}|\\
\end{tabular}
\end{center}
%
If |\jobname| does not match the argument \textit{main} of |\childdocmain|,
it is assumed that |\jobname| points to the child file to be compiled.
When using |\childdocmain| with the main file specified as argument,
it suffices to start a child file
with just |\input{|\textit{main}|}|
without loading of the package and using |\childdocof|.
If instead all processing is done
with the appropriate \textsf{childdoc} directives,
the argument of \textit{main} of |\childdocmain| can be empty.

An alternative version of the command line processing described
in \secref{sec:commandline} using the detection mechanism reads:
%
\begin{center}
|... -jobname "|\textit{target}|" "|[\textit{flags}]%
[|\def\jobname{|\textit{dest}|}|]|\input{|\textit{main}|}"|
\end{center}

%%%%%%%%%%%%%%%%%%%%%%%%%%%%%%%%%%%%%%%%%%%%%%%%%%%%%%%%%%%%%%%%%%%%%%%%%%%%%%%%
\subsection{Manual Code}
\label{sec:manual}

In case one cannot be certain whether the definitions file |childdoc.def|
is installed on the target \TeX{} distribution
and one prefers not to ship it,
it is conceivable to paste a few relevant commands into the sources.

To that end, drop all statements |\input{childdoc.def}|
and perform the replacements as outlined below.
Instead of |\childdocmain{|\textit{main}|}| add the following code
to the top of the main file:
%
\begin{center}
\begin{tabular}{l}
|\||ifdefined\childdocname\endinput\||fi\newif\ifchilddoc|\\
|\edef\childdocname{\scantokens\expandafter{\jobname\noexpand}}|\\
|\def\childdocmain{|\textit{main}|}\||ifx\childdocmain\childdocname\||else|\\
|\childdoctrue\includeonly{\childdocname}\let\jobname\childdocmain\||fi|\\
\end{tabular}
\end{center}
%
Instead of |\childdocof{|\textit{main}|}| just include the main file
at the top of each child file:
%
\begin{center}
|\input{|\textit{main}|}|
\end{center}
%
A simple redirection |\childdocforward{|\textit{dest}|}| is achieved by:
%
\begin{center}
|\def\jobname{|\textit{dest}|}\input{\jobname}|
\end{center}
%
The redirection with prefix
|\childdocforwardprefix[|\textit{prefix}|]{|\textit{dest}|}|
is accomplished by:
%
\begin{center}
\begin{tabular}{l}
|{\edef\jobname{\scantokens\expandafter{\jobname\noexpand}}|\\
|\def\redirectjob |\textit{prefix}|#1~~~{\gdef\jobname{|\textit{dest}|#1}}|\\
|\expandafter\redirectjob\jobname~~~}\input{\jobname}|
\end{tabular}
\end{center}

In an alternative approach,
child documents can be compiled by a specific command line
without additional code or specific definitions:
%
\begin{center}
|... -jobname "|\textit{target}|" "|[\textit{flags}]%
|\includeonly{|\textit{dest}|}\input{|\textit{main}|}"|
\end{center}
%

%%%%%%%%%%%%%%%%%%%%%%%%%%%%%%%%%%%%%%%%%%%%%%%%%%%%%%%%%%%%%%%%%%%%%%%%%%%%%%%%
%%%%%%%%%%%%%%%%%%%%%%%%%%%%%%%%%%%%%%%%%%%%%%%%%%%%%%%%%%%%%%%%%%%%%%%%%%%%%%%%
\section{Information}

%%%%%%%%%%%%%%%%%%%%%%%%%%%%%%%%%%%%%%%%%%%%%%%%%%%%%%%%%%%%%%%%%%%%%%%%%%%%%%%%
\subsection{Copyright}

Copyright \copyright{} 2017--2018 Niklas Beisert

This work may be distributed and/or modified under the
conditions of the \LaTeX{} Project Public License, either version 1.3
of this license or (at your option) any later version.
The latest version of this license is in
  \url{http://www.latex-project.org/lppl.txt}
and version 1.3 or later is part of all distributions of \LaTeX{}
version 2005/12/01 or later.

This work has the LPPL maintenance status `maintained'.

The Current Maintainer of this work is Niklas Beisert.

This work consists of the files |README.txt|, |childdoc.ins| and |childdoc.dtx|
as well as the derived files |childdoc.def|, |cdocsamp.tex|
with |cdocsch1.tex|, |cdocsch2.tex|, |cdocspt3.tex|, |cdocspt4.tex|,
|cdocsdrf.tex|, |cdocsfn1.tex|, |cdocsfn2.tex|
as well as |childdoc.pdf|.

%%%%%%%%%%%%%%%%%%%%%%%%%%%%%%%%%%%%%%%%%%%%%%%%%%%%%%%%%%%%%%%%%%%%%%%%%%%%%%%%
\subsection{Files and Installation}

The package consists of the files:
%
\begin{center}
\begin{tabular}{ll}
    |README.txt|   & readme file \\
    |childdoc.ins| & installation file \\
    |childdoc.dtx| & source file \\
    |childdoc.def| & definition file \\
    |cdocsamp.tex| & sample main file \\
    |cdocsch1.tex| & sample include file \\
    |cdocsch2.tex| & sample include file \\
    |cdocspt3.tex| & sample part file \\
    |cdocspt4.tex| & sample part file \\
    |cdocsdrf.tex| & sample redirection file \\
    |cdocsfn1.tex| & sample redirection file \\
    |cdocsfn2.tex| & sample redirection file \\
    |childdoc.pdf| & manual
\end{tabular}
\end{center}
%
The distribution consists of the files
|README.txt|, |childdoc.ins| and |childdoc.dtx|.
%
\begin{itemize}
\item
Run (pdf)\LaTeX{} on |childdoc.dtx|
to compile the manual |childdoc.pdf| (this file).
\item
Run \LaTeX{} on |childdoc.ins| to create the definitions file |childdoc.def|
and the sample |cdocsamp.tex| with include files
|cdocsch1.tex|, |cdocsch2.tex|, |cdocspt3.tex|, |cdocspt4.tex|,
|cdocsdrf.tex|, |cdocsfn1.tex|, |cdocsfn2.tex|.
Then copy the file |childdoc.def| to an appropriate directory of your \LaTeX{}
distribution, e.g.\ \textit{texmf-root}|/tex/latex/childdoc|.
\end{itemize}

%%%%%%%%%%%%%%%%%%%%%%%%%%%%%%%%%%%%%%%%%%%%%%%%%%%%%%%%%%%%%%%%%%%%%%%%%%%%%%%%
\subsection{Related CTAN Packages}

There are several other packages which offer a similar functionality:
%
\begin{itemize}
\item
The packages
\href{http://ctan.org/pkg/docmute}{\textsf{docmute}},
\href{http://ctan.org/pkg/includex}{\textsf{includex}} and
\href{http://ctan.org/pkg/standalone}{\textsf{standalone}}
provide commands to include only the document body of
a child file thus allowing both files to be compiled individually.
\item
The packages \href{http://ctan.org/pkg/subdocs}{\textsf{subdocs}}
and \href{http://ctan.org/pkg/subfiles}{\textsf{subfiles}}
provide structures in which the main and child documents can be
encapsulated and allowing them to be compiled individually.
The inclusion mechanism is different from the conventional |\include|.
\item
The package \href{http://ctan.org/pkg/combine}{\textsf{combine}}
is an elaborate solution to combine several documents into one.
\end{itemize}
%
See also the CTAN topic \href{http://ctan.org/topic/subdocs}{\textsf{subdocs}}
for further related packages.
The present package differs from the above solutions in that
a document structure constructed with the conventional |\include| mechanism
just needs two extra commands at the top of every file
such that all constituent files can be compiled individually.

%%%%%%%%%%%%%%%%%%%%%%%%%%%%%%%%%%%%%%%%%%%%%%%%%%%%%%%%%%%%%%%%%%%%%%%%%%%%%%%%
%\subsection{Feature Suggestions}
%
%The following is a list of features which may be useful for future
%versions of this package:
%%
%\begin{itemize}
%\item
%\ldots
%\end{itemize}

%%%%%%%%%%%%%%%%%%%%%%%%%%%%%%%%%%%%%%%%%%%%%%%%%%%%%%%%%%%%%%%%%%%%%%%%%%%%%%%%
\subsection{Revision History}

%%%%%%%%%%%%%%%%%%%%%%%%%%%%%%%%%%%%%%%%
\paragraph{v2.0:} 2018/12/30

\begin{itemize}
\item
immediate forward processing
\item
added |\childdocby| mechanism
\item
manual restructured
\end{itemize}

%%%%%%%%%%%%%%%%%%%%%%%%%%%%%%%%%%%%%%%%
\paragraph{v1.6:} 2018/01/17

\begin{itemize}
\item
application for development of include files
\item
corrections to manual
\end{itemize}

%%%%%%%%%%%%%%%%%%%%%%%%%%%%%%%%%%%%%%%%
\paragraph{v1.5:} 2017/05/21

\begin{itemize}
\item
more complete structuring introduced
\item
|\childdocof| introduced
\item
|\childdoc| renamed to |\childdocmain|
\item
|\childredirect| renamed to |\childdocforward| and |\childdocforwardprefix|
and functionality expanded
\end{itemize}

%%%%%%%%%%%%%%%%%%%%%%%%%%%%%%%%%%%%%%%%
\paragraph{v1.0:} 2017/04/27

\begin{itemize}
\item
manual and install package
\item
first version published on CTAN
\end{itemize}

%%%%%%%%%%%%%%%%%%%%%%%%%%%%%%%%%%%%%%%%
\paragraph{v0.6:} 2017/04/26

\begin{itemize}
\item
redirection mechanism added
\end{itemize}

%%%%%%%%%%%%%%%%%%%%%%%%%%%%%%%%%%%%%%%%
\paragraph{v0.5:} 2017/04/26

\begin{itemize}
\item
functionality in definition file
\end{itemize}


%%%%%%%%%%%%%%%%%%%%%%%%%%%%%%%%%%%%%%%%%%%%%%%%%%%%%%%%%%%%%%%%%%%%%%%%%%%%%%%%
%%%%%%%%%%%%%%%%%%%%%%%%%%%%%%%%%%%%%%%%%%%%%%%%%%%%%%%%%%%%%%%%%%%%%%%%%%%%%%%%
%%%%%%%%%%%%%%%%%%%%%%%%%%%%%%%%%%%%%%%%%%%%%%%%%%%%%%%%%%%%%%%%%%%%%%%%%%%%%%%%
\appendix

\settowidth\MacroIndent{\rmfamily\scriptsize 000\ }

 \DocInput{childdoc.dtx}

\end{document}
%</driver>
% \fi
%
% %%%%%%%%%%%%%%%%%%%%%%%%%%%%%%%%%%%%%%%%%%%%%%%%%%%%%%%%%%%%%%%%%%%%%%%%%%%%%%
% %%%%%%%%%%%%%%%%%%%%%%%%%%%%%%%%%%%%%%%%%%%%%%%%%%%%%%%%%%%%%%%%%%%%%%%%%%%%%%
% \section{Sample}
%\iffalse
%<*samplemain>
%\fi
%
% The following presents a sample document
% with two chapters, two parts, a title page,
% a compile flag as well as three forwarding files to set the flag.
% It consists of eight |.tex| files:
% \begin{center}
% \begin{tabular}{ll}
% |cdocsamp.tex|&main file\\
% |cdocsch1.tex|&include file for chapter 1\\
% |cdocsch2.tex|&include file for chapter 2\\
% |cdocspt3.tex|&include file for part 3\\
% |cdocspt4.tex|&include file for part 4\\
% |cdocsdrf.tex|&forwarding file for main file in draft mode\\
% |cdocsfi1.tex|&forwarding file for final version of chapter 1\\
% |cdocsfi2.tex|&forwarding file for final version of chapter 2\\
% \end{tabular}
% \end{center}
% Each of the eight files can be compiled directly by the \LaTeX{} compiler.
%
% %%%%%%%%%%%%%%%%%%%%%%%%%%%%%%%%%%%%%%
% \paragraph{Main File.}
%
% The main file is called |cdocsamp.tex|.
%
% Load the \textsf{childdoc} definitions and
% declare the filename for the main document:
%    \begin{macrocode}
\input{childdoc.def}
\childdocmain{}
%    \end{macrocode}

% Optional override for |\version| flag:
%    \begin{macrocode}
%%\ifchilddoc\else\providecommand{\version}{draft}\fi
%    \end{macrocode}

% Define the default values for the |\version| flag
% (|final| for the main file and |draft| for childs):
%    \begin{macrocode}
\ifchilddoc
\providecommand{\version}{draft}
\else
\providecommand{\version}{final}
\fi
%    \end{macrocode}

% Load the standard document class:
%    \begin{macrocode}
\documentclass[12pt]{article}
%    \end{macrocode}

% Start the document body:
%    \begin{macrocode}
\begin{document}
%    \end{macrocode}

% Declare a title page.
% Print title, part of document being processed and version flag:
%    \begin{macrocode}
\addtocounter{page}{-1}
\begin{center}
{\LARGE\bfseries{}childdoc example\par}
\vspace{1cm}
\ifchilddoc
\ifchilddocmanual part\else chapter\fi:
`\childdocname' of `\childdocjob'\par
\else
main document: `\childdocjob'\par
\fi
version: \version\par
\end{center}
\newpage
%    \end{macrocode}

% Manually include selected file,
% otherwise process as usual:
%    \begin{macrocode}
\ifchilddocmanual
\section*{part `\childdocname'}
\input{\childdocname}
\else
%    \end{macrocode}

% Include the two chapters:
%    \begin{macrocode}
\include{cdocsch1}
\include{cdocsch2}
%    \end{macrocode}

% Include the two parts unless only chapters should be displayed:
%    \begin{macrocode}
\ifchilddoc\else
\section{part three}
\input{cdocspt3}
\section{part four}
\input{cdocspt4}
\fi
%    \end{macrocode}

% Process as usual until here:
%    \begin{macrocode}
\fi
%    \end{macrocode}

% End of document body:
%    \begin{macrocode}
\end{document}
%    \end{macrocode}
%\iffalse
%</samplemain>
%\fi
%
% %%%%%%%%%%%%%%%%%%%%%%%%%%%%%%%%%%%%%%
% \paragraph{Chapter Include Files.}
%
% The include files are called |cdocsch1.tex| and |cdocsch2.tex|.
%
%\iffalse
%<*samplechap1|samplechap2>
%\fi

% Optional override for |\version| flag:
%    \begin{macrocode}
%%\providecommand{\version}{final}
%    \end{macrocode}

% Include the main document:
%    \begin{macrocode}
\input{childdoc.def}
\childdocof{cdocsamp}
%    \end{macrocode}

%\iffalse
%</samplechap1|samplechap2>
%\fi
%
%\iffalse
%<*samplechap1>
%\fi
% Some text for chapter 1:
%    \begin{macrocode}
\section{one}
some text in chapter one
%    \end{macrocode}

%\iffalse
%</samplechap1>
%\fi
% Some text for chapter 2:
%\iffalse
%<*samplechap2>
%\fi
%    \begin{macrocode}
\section{two}
more text in chapter two
%    \end{macrocode}

%\iffalse
%</samplechap2>
%\fi
%
% %%%%%%%%%%%%%%%%%%%%%%%%%%%%%%%%%%%%%%
% \paragraph{Part Include Files.}
%
% The include files are called |cdocspt3.tex| and |cdocspt4.tex|.
%
%\iffalse
%<*samplepart3|samplepart4>
%\fi

% Optional override for |\version| flag:
%    \begin{macrocode}
%%\providecommand{\version}{final}
%    \end{macrocode}

% Include the main document:
%    \begin{macrocode}
\input{childdoc.def}
\childdocby{cdocsamp}
%    \end{macrocode}

%\iffalse
%</samplepart3|samplepart4>
%\fi
%
%\iffalse
%<*samplepart3>
%\fi
% Some text for part 3:
%    \begin{macrocode}
some text in part three
%    \end{macrocode}

%\iffalse
%</samplepart3>
%\fi
% Some text for part 4:
%\iffalse
%<*samplepart4>
%\fi
%    \begin{macrocode}
more text in part four
%    \end{macrocode}

%\iffalse
%</samplepart4>
%\fi
%
% %%%%%%%%%%%%%%%%%%%%%%%%%%%%%%%%%%%%%%
% \paragraph{Forwarding for a Complete Draft.}
%
% The following forwarding file |cdocsdrf.tex|
% compiles the main document in draft mode:
%\iffalse
%<*sampledraft>
%\fi
%    \begin{macrocode}
\def\version{draft}
\input{childdoc.def}
\childdocforward{cdocsamp}
%    \end{macrocode}

%\iffalse
%</sampledraft>
%\fi
%
% %%%%%%%%%%%%%%%%%%%%%%%%%%%%%%%%%%%%%%
% \paragraph{Forwarding for Final Version of the Chapters.}
%
% The following forwarding files |cdocsfn1.tex| and |cdocsfn2.tex|
% (with identical content)
% compile the final versions of the child documents
% |cdocsch1.tex| and |cdocsch2.tex|, respectively:
%\iffalse
%<*samplefinal>
%\fi
%    \begin{macrocode}
\def\version{final}
\input{childdoc.def}
\childdocforwardprefix[cdocsamp]{cdocsfn}{cdocsch}
%    \end{macrocode}

%\iffalse
%</samplefinal>
%\fi
%
% %%%%%%%%%%%%%%%%%%%%%%%%%%%%%%%%%%%%%%
% \paragraph{Command Line Processing.}
%
% The following three command lines generate the output files
% |cdocscld|, |cdocscl1| and |cdocscl2|
% which should be identical to
% |cdocsdrf|, |cdocsch1| and |cdocsfn2|, respectively:
% \begin{center}
% \begin{tabular}{l}
% |latex -jobname cdocscld \|\\
% |  "\def\version{draft}\input{childdoc.def}\childdocforward{cdocsamp}"|\\
% |latex -jobname cdocscl1 \|\\
% |  "\input{childdoc.def}\childdocforward[cdocsamp]{cdocsch1}"|\\
% |latex -jobname cdocscl2 \|\\
% |  "\def\version{final}\input{childdoc.def}\childdocforward{cdocsch2}"|
% \end{tabular}
% \end{center}
% Note that the trailing backslash on each first line
% merely continues the input to the second line
% (for convenient cut ant paste).
% Furthermore, the command |latex| can be replaced by any
% of its alternative versions such as |pdflatex|.
%
% %%%%%%%%%%%%%%%%%%%%%%%%%%%%%%%%%%%%%%%%%%%%%%%%%%%%%%%%%%%%%%%%%%%%%%%%%%%%%%
% %%%%%%%%%%%%%%%%%%%%%%%%%%%%%%%%%%%%%%%%%%%%%%%%%%%%%%%%%%%%%%%%%%%%%%%%%%%%%%
% \section{Implementation}
%\iffalse
%<*package>
%\fi
%
% This section describes the definitions file |childdoc.def|.

% The definitions cannot be loaded using |\usepackage| or |\RequirePackage|
% which has a mechanism to prevent loading a style file more than once.
% When loading the definitions by means of |\input|
% multiple instances have to be prevented manually:
%\iffalse
%This code needs to be before the `\ProvidesFile' directive
%which is defined at the beginning of this file.
%Therefore it is also placed there and commented out here.
%</package>
%<*discard>
%\fi
%    \begin{macrocode}
\ifdefined\childdocmain\endinput\fi
%    \end{macrocode}
%\iffalse
%</discard>
%<*package>
%\fi
%
% \macro{\ifchilddoc}
% \macro{\ifchilddocmanual}
% The conditional |\ifchilddoc| tells whether a
% child (true) or main (false) document is being compiled.
% The conditional |\ifchilddocmanual| tells whether
% the |\includeonly| mechanism is used (false) or
% the selection of child files must be performed manually (true).
% The definitions initialise to false:
%    \begin{macrocode}
\newif\ifchilddoc
\newif\ifchilddocmanual
%    \end{macrocode}

% \macro{\childdocname}
% \macro{\childdocjob}
% The macro |\childdocname| stores the name of the main document
% to be compiled. The macro |\childdocjob| stores the name of
% the document on which the \LaTeX{} compiler was originally invoked.
% The content of |\jobname| cannot be compared
% to filenames specified in the source due to different catcodes.
% The following code rescans |\jobname|, stores the result
% in |\childdocname| and saves a copy in |\childdocjob|:
%    \begin{macrocode}
\edef\childdocname{\scantokens\expandafter{\jobname\noexpand}}
\let\childdocjob\childdocname
%    \end{macrocode}

% \macro{\childdocdisable}
% The macro |\childdocdisable| prevents the main file
% from being processed more than once.
% At this stage, the main document command |\childdocmain|
% is assumed to be called once again where it should do nothing.
% Any subsequent call to it should prevent
% a secondary processing of the main document
% It overwrites the forwarding commands
% |\childdocof| and |\childdocforward|
% with empty macros to prevent further inclusions of the main document:
%    \begin{macrocode}
\newcommand{\childdocdisable}
{
  \renewcommand{\childdocmain}[1]{\renewcommand{\childdocmain}[1]{\endinput}}
  \renewcommand{\childdocof}[1]{}
  \renewcommand{\childdocby}[2][]{}
  \renewcommand{\childdocforward}[2][]{}
  \renewcommand{\childdocdisable}{}
}
%    \end{macrocode}

% \macro{\childdocmain}
% The macro |\childdocmain| is to be called at the top of the main file
% with nothing or the main filename (without extension) as argument.
% First, it breaks loops.
% If the argument is not empty and does not match |\childdocname|
% (which is set by the first inclusion of |childdoc.def|),
% |\ifchilddoc| is set to true, |\includeonly| is applied to the child file
% and |\jobname| is set to the main file
% (for proper handling of |.aux| files):
%    \begin{macrocode}
\newcommand{\childdocmain}[1]
{
  \childdocdisable\childdocmain{}
  \if?#1?\else
    \begingroup
      \def\childdoctmp{#1}
      \ifx\childdoctmp\childdocname
        \def\childdoctmp{}
      \else
        \def\childdoctmp
        {
          \childdoctrue
          \includeonly{\childdocname}
          \def\childdocjob{#1}
          \def\jobname{#1}
        }
      \fi
      \expandafter
    \endgroup
    \childdoctmp
  \fi
}
%    \end{macrocode}

% \macro{\childdocof}
% The command |\childdocof| redirects
% compilation to the main file |#1|.
%    \begin{macrocode}
\newcommand{\childdocof}[1]
{
  \childdocdisable
  \childdoctrue
  \includeonly{\childdocname}
  \def\jobname{#1}
  \def\childdocjob{#1}
  \input{#1}
}
%    \end{macrocode}

% \macro{\childdocby}
% The command |\childdocby| ....
%    \begin{macrocode}
\newcommand{\childdocby}[2][]
{
  \childdocdisable
  \childdoctrue
  \childdocmanualtrue
  \if?#1?\else
    \def\jobname{#2}
  \fi
  \def\childdocjob{#2}
  \input{#2}
  \endinput
}
%    \end{macrocode}

% \macro{\childdocforward}
% The command |\childdocforward| redirects
% compilation to the main file or
% (if the optional argument is given) a child file.
% Parameters are set as if the main file
% or a child file starting with |\childdocof| was compiled.
% Then compilation is handed over to the main file:
%    \begin{macrocode}
\newcommand{\childdocforward}[2][]
{
  \begingroup
    \if?#1?
      \def\childdoctmp
      {
        \def\childdocname{#2}
        \def\childdocjob{#2}
        \def\jobname{#2}
        \input{#2}
        \endinput
      }
    \else
      \def\childdoctmp
      {
        \childdocdisable
        \def\childdocname{#2}
        \childdoctrue
        \includeonly{#2}
        \def\childdocjob{#1}
        \def\jobname{#1}
        \input{#1}
        \endinput
      }
    \fi
    \expandafter
  \endgroup
  \childdoctmp
}
%    \end{macrocode}

% \macro{\childdocforwardprefix}
% The command |\childdocforwardprefix| redirects
% compilation to the main or a child file by means of a pattern.
% The prefix |#1| in the current filename is replaced by |#2|
% and the suffix of the current filename is kept
% (it is assumed that the filename does not contain the substring `|~~~|'
% which is used as a delimiter).
% Compilation is handed over to the new file by |\childdocforward|:
%    \begin{macrocode}
\newcommand{\childdocforwardprefix}[3][]
{
  \begingroup
    \def\childdocextract #2##1~~~{\def\childdoctmp{\childdocforward[#1]{#3##1}}}
    \expandafter\childdocextract\childdocname~~~
    \expandafter
  \endgroup
  \childdoctmp
}
%    \end{macrocode}

% \macro{\childdoc}
% The deprecated macro |\childdoc| is a legacy version of |\childdocmain|:
%    \begin{macrocode}
\newcommand{\childdoc}{\childdocmain}
%    \end{macrocode}

% \macro{\childdocredirect}
% The deprecated macro |\childdocredirect| is a legacy version
% of |\childdocforward| and |\childdocforwardprefix|:
%    \begin{macrocode}
\newcommand{\childdocredirect}[2][]
{
  \begingroup
    \if?#1?
      \def\childdoctmp{\childdocforward{#2}}
    \else
      \def\childdoctmp{\childdocforwardprefix{#1}{#2}}
    \fi
    \expandafter
  \endgroup
  \childdoctmp
}
%    \end{macrocode}

%\iffalse
%</package>
%\fi
%
\endinput
|\\
|\childdocforward[|\textit{main}|]{|\textit{dest}|}|\\
\end{tabular}
\end{center}
%
The argument \textit{dest} is the destination file
(without extension).
It should be the main file or one of the child files.
Note that further \textsf{childdoc} directives
such as |\childdocof| and |\childdocforward|
in the indicated file will be processed in this form.
The optional argument \textit{main}
passes on directly to the main file \textit{main}
while pretending to compile the child \textit{dest}.
This form behaves as if \textit{dest}
issues |\childdocof{|\textit{main}|}| right away,
and no further \textsf{childdoc} directives will be processed.

%%%%%%%%%%%%%%%%%%%%%%%%%%%%%%%%%%%%%%%%
\DescribeMacro{\...prefix}
In the alternative form |\childdocforwardprefix|,
%
\begin{center}
\begin{tabular}{l}
|% \iffalse
%
% childdoc.dtx Copyright (C) 2017-2018 Niklas Beisert
%
% This work may be distributed and/or modified under the
% conditions of the LaTeX Project Public License, either version 1.3
% of this license or (at your option) any later version.
% The latest version of this license is in
%   http://www.latex-project.org/lppl.txt
% and version 1.3 or later is part of all distributions of LaTeX
% version 2005/12/01 or later.
%
% This work has the LPPL maintenance status `maintained'.
%
% The Current Maintainer of this work is Niklas Beisert.
%
% This work consists of the files childdoc.dtx and childdoc.ins
% and the derived files childdoc.def and cdocsamp.tex with
% cdocsch1.tex, cdocsch2.tex, cdocsdrf.tex, cdocsfn1.tex, cdocsfn2.tex.
%
%<package>\ifdefined\childdocmain\endinput\fi
%<package>\ProvidesFile{childdoc.def}[2018/12/30 v2.0 child document driver]
%<samplemain>\ProvidesFile{cdocsamp.tex}[2018/12/30 v2.0 sample for childdoc]
%<*driver>
%\ProvidesFile{childdoc.drv}[2018/12/30 v2.0 childdoc reference manual file]
\PassOptionsToClass{10pt,a4paper}{article}
\documentclass{ltxdoc}

\usepackage[margin=35mm]{geometry}
\usepackage{hyperref}
\usepackage{hyperxmp}
\usepackage[usenames]{color}

\hypersetup{colorlinks=true}
\hypersetup{pdfstartview=FitH}
\hypersetup{pdfpagemode=UseNone}
\hypersetup{pdfsource={}}
\hypersetup{pdflang={en-UK}}
\hypersetup{pdfcopyright={Copyright 2017-2018 Niklas Beisert.
  This work may be distributed and/or modified under the
  conditions of the LaTeX Project Public License, either version 1.3
  of this license or (at your option) any later version.}}
\hypersetup{pdflicenseurl={http://www.latex-project.org/lppl.txt}}
\hypersetup{pdfcontactaddress={ETH Zurich, ITP, HIT K,
  Wolfgang-Pauli-Strasse 27}}
\hypersetup{pdfcontactpostcode={8093}}
\hypersetup{pdfcontactcity={Zurich}}
\hypersetup{pdfcontactcountry={Switzerland}}
\hypersetup{pdfcontactemail={nbeisert@itp.phys.ethz.ch}}
\hypersetup{pdfcontacturl={http://people.phys.ethz.ch/\xmptilde nbeisert/}}

\newcommand{\secref}[1]{\hyperref[#1]{section \ref*{#1}}}

\parskip1ex
\parindent0pt
\let\olditemize\itemize
\def\itemize{\olditemize\parskip0pt}

\begin{document}

\title{The \textsf{childdoc} Package}
\hypersetup{pdftitle={The childdoc Package}}
\author{Niklas Beisert\\[2ex]
  Institut f\"ur Theoretische Physik\\
  Eidgen\"ossische Technische Hochschule Z\"urich\\
  Wolfgang-Pauli-Strasse 27, 8093 Z\"urich, Switzerland\\[1ex]
  \href{mailto:nbeisert@itp.phys.ethz.ch}
  {\texttt{nbeisert@itp.phys.ethz.ch}}}
\hypersetup{pdfauthor={Niklas Beisert}}
\hypersetup{pdfsubject={Manual for the LaTeX2e Package childdoc}}
\date{30 December 2018, \textsf{v2.0}}
\maketitle

\begin{abstract}\noindent
\textsf{childdoc} is a \LaTeXe{} package
that enables the direct compilation
of document sections included by |\include|
to individual files.
\end{abstract}

\begingroup
\parskip0ex
\tableofcontents
\endgroup

%%%%%%%%%%%%%%%%%%%%%%%%%%%%%%%%%%%%%%%%%%%%%%%%%%%%%%%%%%%%%%%%%%%%%%%%%%%%%%%%
%%%%%%%%%%%%%%%%%%%%%%%%%%%%%%%%%%%%%%%%%%%%%%%%%%%%%%%%%%%%%%%%%%%%%%%%%%%%%%%%
\section{Introduction}

\LaTeX{} provides a mechanism to structure a large document (such as a book)
into a main file and several child files (containing the chapters)
using the |\include| command.
This mechanism is beneficial for documents
which span hundreds of pages in order to
make the source file(s) more manageable.
Moreover, compilation can be restricted to
selected child files by means of the |\includeonly| command.
The latter feature can be used to reduce the compilation time while editing
(this was significantly more useful in the earlier days of \LaTeX{})
or to generate a smaller document which is easier to navigate.
Another application of |\includeonly| is to generate
documents consisting of selected parts of the complete document.

However, there are a few drawbacks of the plain |\include| mechanism:
\begin{itemize}
\item
The child files cannot be compiled on their own,
they can only be compiled via the main file.
A naive editing environment
(such as a text editor with an option
to have the current file processed by \LaTeX)
may require one to switch to the main file before compiling;
attempting to compile the child file produces errors.
\item
The main file must be modified (each time)
to adjust the |\includeonly| command
to the present needs. This easily leaves the main file in a messy state.
\item
The generated document will always carry the filename
of the main document. This is inconvenient if
several child files are to be compiled and
to be kept for distribution.
\end{itemize}

The present package provides a simple interface
to make child files individually compilable by \LaTeX{}.
Compiling a child file then has the same effect as compiling
the main file with an |\includeonly| command
to select the appropriate child.
Moreover the generated document will carry the name of the child
rather than the main file.
This resolves all three above issues.

This feature is meant to make the editing of books,
thesis documents and lecture notes somewhat more convenient.
However, the package can also be used efficiently for
composing a series of documents (such as exercise sheets)
which are typically distributed individually.
It then assists the author in generating the individual documents
(potentially in different versions)
as well as a document containing the collected series.
Another application is in developing style files
or other kinds of included material
where compilation of the style file could redirect
to a sample or test file.

%%%%%%%%%%%%%%%%%%%%%%%%%%%%%%%%%%%%%%%%%%%%%%%%%%%%%%%%%%%%%%%%%%%%%%%%%%%%%%%%
%%%%%%%%%%%%%%%%%%%%%%%%%%%%%%%%%%%%%%%%%%%%%%%%%%%%%%%%%%%%%%%%%%%%%%%%%%%%%%%%
\section{Usage}

First of all, the package \textsf{childdoc} is \emph{not} a standard
\LaTeXe{} |.sty| style file! Therefore it needs to be invoked in
a non-standard way.

%%%%%%%%%%%%%%%%%%%%%%%%%%%%%%%%%%%%%%%%%%%%%%%%%%%%%%%%%%%%%%%%%%%%%%%%%%%%%%%%
\subsection{Included Files}
\label{sec:include}

%%%%%%%%%%%%%%%%%%%%%%%%%%%%%%%%%%%%%%%%
\DescribeMacro{\childdocmain}
To use the package, add the commands
\begin{center}
\begin{tabular}{l}
|\input{childdoc.def}|\\
|\childdocmain{}|\\
\end{tabular}
\end{center}
at the very top of the main \LaTeX{} file,
in particular \emph{before} the |\documentclass| statement!
The argument of |\childdocmain| should be left empty
(but it must be present).

%%%%%%%%%%%%%%%%%%%%%%%%%%%%%%%%%%%%%%%%
\DescribeMacro{\childdocof}
Furthermore, add the commands
\begin{center}
\begin{tabular}{l}
|\input{childdoc.def}|\\
|\childdocof{|\textit{main}|}|\\
\end{tabular}
\end{center}
at the top of every child file \textit{child}
which is included by |\include{|\textit{child}|}|
from within the main file
(or at least for those files to be compiled individually).
The argument \textit{main} must be the filename of the main file.

There are a couple of
considerations in setting up the main and child documents:

%%%%%%%%%%%%%%%%%%%%%%%%%%%%%%%%%%%%%%%%
\paragraph{Restrictions.}

Please note the following restrictions:
\begin{itemize}
\item
|\childdocmain| must be called with one argument \textit{main}
to ensure compatibility with earlier version of the package.
It must either be empty (|\childdocmain{}|)
or precisely match the filename of the main file in which it is specified.
See \secref{sec:detection} for further information.
\item
The filename \textit{main} must be specified without the |.tex| extension.
\item
The filename \textit{main} is case sensitive
(even in case-insensitive file systems)
due to internal string comparison.
\item
The argument \textit{main} should be fully expanded, it cannot be a macro.
\item
Subdirectories and special characters should be avoided in filenames.
\item
The command |\childdocmain{|\textit{main}|}| must be followed by a whitespace.
It should not be followed immediately by another command
or by a comment mark `|%|'.
This is because the \TeX{} parser reads the token immediately following
the argument of |\childdocmain| and puts it
at the beginning of every child section;
however, a white\-space is ignored.
\end{itemize}

%%%%%%%%%%%%%%%%%%%%%%%%%%%%%%%%%%%%%%%%
\paragraph{Content of Main File.}

It is advisable to place all content in the child files included by |\include|.
Any output contained in the main file will appear in all child documents
unless suppressed manually;
it cannot be suppressed automatically by the |\includeonly| directive
and thus should normally be avoided.
A method to include some content in the main file
by means of conditional processing is described in \secref{sec:conditional}.

%%%%%%%%%%%%%%%%%%%%%%%%%%%%%%%%%%%%%%%%
\paragraph{Page Numbering.}

When only a part of the document is compiled,
the appropriate numbering of pages
(as well as other status parameters)
is determined from the |.aux| files.
The latter contain information from previous passes.
However this information needs to propagate through
all intermediate child documents.
Therefore the page numbering in child documents may well
be inconsistent until the complete document is compiled at least once.

A useful (if unconventional) way to always ensure a consistent
page numbering is to restart the numbering in each child document
and denote the pages by `\textit{child}|.|\textit{page}'
where \textit{child} represents the chapter/section number of the child file.
This can be achieved by the command
|\numberwithin{page}{|\textit{child}|}|
of the \textsf{amsmath} package
where \textit{child} can be |chapter| or |section|
depending on the chosen structuring.
Alternatively, one can modify the macro |\thepage| appropriately
and reset the counter |page| at the start of each child file.

%%%%%%%%%%%%%%%%%%%%%%%%%%%%%%%%%%%%%%%%%%%%%%%%%%%%%%%%%%%%%%%%%%%%%%%%%%%%%%%%
\subsection{Conditional Processing}
\label{sec:conditional}

The package provides a mechanism to compile different versions
of a document. To customise the versions further some conditional processing
can come in handy to distinguish which version is being compiled.
The package provides two macros to describe the compilation context:

%%%%%%%%%%%%%%%%%%%%%%%%%%%%%%%%%%%%%%%%
\DescribeMacro{\ifchilddoc}
The conditional |\ifchilddoc| distinguishes between the compilation of
child documents and the main document:
%
\begin{center}
|\ifchilddoc |\textit{child-code}| |[|\||else |\textit{main-code}]| \||fi|
\end{center}

%%%%%%%%%%%%%%%%%%%%%%%%%%%%%%%%%%%%%%%%
\DescribeMacro{\childdocname}
\DescribeMacro{\childdocjob}
The macro |\childdocname| contains the filename (without extension)
of the main or child file being processed.
Note that |\childdocjob| will always contain the name of the main file.

%%%%%%%%%%%%%%%%%%%%%%%%%%%%%%%%%%%%%%%%
\paragraph{Title Page.}

Conditional processing can be used to include a title or banner page
in the main document when proper precautions are taken.
Importantly, the code in the main file should ensure that the page counter
(as well as other status parameters which are stored in the |.aux| files)
takes the same value after the conditional processing.
Otherwise the page numbers may take divergent values
depending on which part is compiled.

For example, a title page could be declared by:
%
\begin{center}
\begin{tabular}{l}
|\ifchilddoc\||else|\\
|\addtocounter{page}{-1}|\\
\textit{code for title page}\\
|\newpage|\\
|\||fi|
\end{tabular}
\end{center}
%
A banner page for the child documents can be generated by:
%
\begin{center}
\begin{tabular}{l}
|\ifchilddoc|\\
|\addtocounter{page}{-1}|\\
\textit{code for banner page}\\
|\newpage|\\
|\||fi|
\end{tabular}
\end{center}
%
Here one could write a message such as:
\begin{center}
|This is the part \childdocname{} of \childdocjob{}.|
\end{center}

%%%%%%%%%%%%%%%%%%%%%%%%%%%%%%%%%%%%%%%%%%%%%%%%%%%%%%%%%%%%%%%%%%%%%%%%%%%%%%%%
\subsection{Flags}
\label{sec:flags}

The package makes it easy to generate different versions
of the main or child documents.
To this end compilation flags can be defined
and assigned different default values.
They will be particularly useful in conjunction
with the forwarding mechanism described in \secref{sec:forward}.

For example, it may be useful to have a flag |\version|
which can be set to |draft| or |final|.
The document source will contain some conditional code
depending on the value of |\version|.
Suppose further, the flag should default to |final| for the main file
and to |draft| for child files
which is a natural assignment for editing the document.
This is achieved by placing the following code
in the preamble of the main document
(below the |\childdocmain| directive):
%
\begin{center}
\begin{tabular}{l}
|\ifchilddoc|\\
|\providecommand{\version}{draft}|\\
|\||else|\\
|\providecommand{\version}{final}|\\
|\||fi|
\end{tabular}
\end{center}
%
The definition by |\providecommand| makes sure
that previous definitions are not overwritten.
Further statements |\providecommand{\version}{...}|
can thus be added before the above code to override it.

For the main file, one might add a line
(between |\childdocmain| and the above block)
%
\begin{center}
|%\ifchilddoc\||else\providecommand{\version}{draft}\||fi|
\end{center}
%
which can be uncommented to produce a draft version.
Likewise one can add a line to the very top of a child file
(above the |\childdocof{|\textit{main}|}| directive)
%
\begin{center}
|%\providecommand{\version}{final}|
\end{center}
%
which can be uncommented to produce the final version of this child document.

%%%%%%%%%%%%%%%%%%%%%%%%%%%%%%%%%%%%%%%%%%%%%%%%%%%%%%%%%%%%%%%%%%%%%%%%%%%%%%%%
\subsection{Forwarding}
\label{sec:forward}

Different versions of the main or child documents
using compilation flags as described in \secref{sec:flags}
can be (permanently) stored in different files
for convenient compilation, viewing and distribution.
To this end, the package defines a command
to pass on compilation to a different file:

%%%%%%%%%%%%%%%%%%%%%%%%%%%%%%%%%%%%%%%%
\DescribeMacro{\childdocforward}
The command |\childdocforward| redirects processing to
another source file:
%
\begin{center}
\begin{tabular}{l}
|\input{childdoc.def}|\\
|\childdocforward[|\textit{main}|]{|\textit{dest}|}|\\
\end{tabular}
\end{center}
%
The argument \textit{dest} is the destination file
(without extension).
It should be the main file or one of the child files.
Note that further \textsf{childdoc} directives
such as |\childdocof| and |\childdocforward|
in the indicated file will be processed in this form.
The optional argument \textit{main}
passes on directly to the main file \textit{main}
while pretending to compile the child \textit{dest}.
This form behaves as if \textit{dest}
issues |\childdocof{|\textit{main}|}| right away,
and no further \textsf{childdoc} directives will be processed.

%%%%%%%%%%%%%%%%%%%%%%%%%%%%%%%%%%%%%%%%
\DescribeMacro{\...prefix}
In the alternative form |\childdocforwardprefix|,
%
\begin{center}
\begin{tabular}{l}
|\input{childdoc.def}|\\
|\childdocforwardprefix[|\textit{main}|]{|\textit{prefix}|}{|\textit{dest}|}|
\end{tabular}
\end{center}
%
the destination file is determined by a pattern
depending on the current file:
To make this work, the current file must be called
`{\textit{prefix}\hspace{0.2em}\textit{suffix}}'
with \textit{prefix} matching precisely the argument.
Processing is then passed on to the file
`{\textit{dest}\hspace{0.2em}\textit{suffix}}'.
Surely, the same effect is achieved by
directly specifying the
argument `{\textit{dest}\hspace{0.2em}\textit{suffix}}'
in the first form.
However, that requires to set up a different file
for each child. With the alternative form of the command
all these files can have exactly the same content
which simplifies setting them up and maintaining them.

For example, the following file |draft.tex|
with a compilation flag |\version| as described in \secref{sec:flags}
compiles the main document as a draft:
%
\begin{center}
\begin{tabular}{l}
|\def\version{draft}|\\
|\input{childdoc.def}|\\
|\childdocforward{|\textit{main}|}|
\end{tabular}
\end{center}
%
Likewise, the following files |final|\textit{nn}|.tex|
compile the final version of the child document
|child|\textit{nn}|.tex|:
%
\begin{center}
\begin{tabular}{l}
|\def\version{final}|\\
|\input{childdoc.def}|\\
|\childdocforwardprefix{final}{child}|
\end{tabular}
\end{center}
%

Note that when several versions of a main file and/or of each child file
are to be generated, it may be convenient to set up a |Makefile| or
shell script to automatise the process.

%%%%%%%%%%%%%%%%%%%%%%%%%%%%%%%%%%%%%%%%%%%%%%%%%%%%%%%%%%%%%%%%%%%%%%%%%%%%%%%%
\subsection{Command Line Processing}
\label{sec:commandline}

The effect of redirection files can also be achieved by invoking
the \LaTeX{} compiler with a more elaborate command line.
Most conveniently this should be done as part
of a shell script or a |Makefile|.

When using \textsf{childdoc} in the main file, the following
command lines effectively perform a redirection
(note that depending on the shell being used,
backslashes may have to be doubled: `|\|' $\to$ `|\\|'):
%
\begin{center}
|... -jobname "|\textit{target}|" |\\|"|[\textit{flags}]%
|\input{childdoc.def}\childdocforward[|\textit{main}|]{|\textit{dest}|}"|
\end{center}
%
Here \textit{target} is the name of the output file,
\textit{main} is the name of the main file
and \textit{dest} is the name of the main or child file to be processed
(all filenames without extensions).
The optional argument \textit{main} can be omitted
if \textit{main} matches \textit{dest}.
Optionally, compilation \textit{flags} can be defined via |\def| commands.
This command line makes the \TeX{} engine believe
it is compiling the file \textit{target}
whose content is specified as the latter parameter.
The provided code then forwards the processing to
\textit{main} or \textit{dest} as described in \secref{sec:forward}.

%%%%%%%%%%%%%%%%%%%%%%%%%%%%%%%%%%%%%%%%%%%%%%%%%%%%%%%%%%%%%%%%%%%%%%%%%%%%%%%%
\subsection{Include by Input}
\label{sec:input}

Including child documents by |\include| has some restrictions by design.
Most notably, the content of a child document always occupies
its own set of pages; pages cannot be shared between child documents.
Usually, this behaviour makes perfect sense
because each child document contain an essential part of the document.
However, in some situations it may be desirable to compose
a document from a collection of parts
without having mandatory page breaks between then.
For this case, the package
provides a mechanism to include parts
by |\input| which can also be processed individually.
However, by construction this mechanism
requires manual handling of the content to be output.

%%%%%%%%%%%%%%%%%%%%%%%%%%%%%%%%%%%%%%%%
\DescribeMacro{\ifchilddocmanual}
The main file should be prepared as usual, see \secref{sec:include}.
However, the document body must make a distinction
between processing of an individual part and of the main document, e.g.:
%
\begin{center}
\begin{tabular}{l}
|\ifchilddocmanual|\\
|\input{\childdocname}|\\
|\||else|\\
\textit{document body with }|\input{|\textit{part}|}|\\
|\||fi|
\end{tabular}
\end{center}
%
The conditional |\ifchilddocmanual| is true whenever
a part to be included by |\input| is being compiled,
and the name of the part is stored in |\childdocname|.

%%%%%%%%%%%%%%%%%%%%%%%%%%%%%%%%%%%%%%%%
\DescribeMacro{\childdocby}
Each part to be included by |\input| should start with:
%
\begin{center}
\begin{tabular}{l}
|\input{childdoc.def}|\\
|\childdocby{|\textit{main}|}|\\
\end{tabular}
\end{center}
%
The directive |\childdocby| is similar to |\childdocof|
described in \secref{sec:include},
but the subsequent selection of content must be done manually.
To that end, both |\ifchilddoc| and |\ifchilddocmanual|
will be true upon processing of a part,
and the name of the part is stored in |\childdocname|.
Note that |\jobname| will be set to the filename of the current part
so that each part receives an individual |.aux| file
that does not interfere with the |.aux| file(s) of the main document.
This behaviour can be altered by the alternative form
|\childdocby[*]{|\textit{main}|}| (with a non-empty optional argument)
which uses the |.aux| file of the main document
by setting |\jobname| to \textit{main}.

%%%%%%%%%%%%%%%%%%%%%%%%%%%%%%%%%%%%%%%%%%%%%%%%%%%%%%%%%%%%%%%%%%%%%%%%%%%%%%%%
\subsection{Driver Development}
\label{sec:driver}

The \textsf{childdoc} mechanism can also be use for the development
of definition files such as \LaTeX{} styles or classes.
This case differs from the above setup with multiple parts
included by |\include| in that no |\includeonly| should be invoked.
This can be achieved by starting the include file
(before |\ProvidesPackage|) with:
%
\begin{center}
\begin{tabular}{l}
|\input{childdoc.def}|\\
|\childdocforward{|\textit{main}|}|\\
\end{tabular}
\end{center}
%
or alternatively with:
%
\begin{center}
\begin{tabular}{l}
|\input{childdoc.def}|\\
|\childdocby{|\textit{main}|}|\\
\end{tabular}
\end{center}
%
Both forms have slightly different effects as described above.
The main file is prepared as usual, see \secref{sec:include}.

%%%%%%%%%%%%%%%%%%%%%%%%%%%%%%%%%%%%%%%%%%%%%%%%%%%%%%%%%%%%%%%%%%%%%%%%%%%%%%%%
\subsection{Legacy Detection}
\label{sec:detection}

The directive |\childdocmain| in the main file can detect
whether the complete document or merely a child is to be compiled
even without using the directive |\childdocof|.
This method is deprecated because it is less robust
and there is no compelling reason to use it;
it is merely provided for backward compatibility
and it may be removed in future versions.

If the detection mechanism is to be used,
it is mandatory to correctly specify
the filename of the main file as the argument of |\childdocmain|:
%
\begin{center}
\begin{tabular}{l}
|\input{childdoc.def}|\\
|\childdocmain{|\textit{main}|}|\\
\end{tabular}
\end{center}
%
If |\jobname| does not match the argument \textit{main} of |\childdocmain|,
it is assumed that |\jobname| points to the child file to be compiled.
When using |\childdocmain| with the main file specified as argument,
it suffices to start a child file
with just |\input{|\textit{main}|}|
without loading of the package and using |\childdocof|.
If instead all processing is done
with the appropriate \textsf{childdoc} directives,
the argument of \textit{main} of |\childdocmain| can be empty.

An alternative version of the command line processing described
in \secref{sec:commandline} using the detection mechanism reads:
%
\begin{center}
|... -jobname "|\textit{target}|" "|[\textit{flags}]%
[|\def\jobname{|\textit{dest}|}|]|\input{|\textit{main}|}"|
\end{center}

%%%%%%%%%%%%%%%%%%%%%%%%%%%%%%%%%%%%%%%%%%%%%%%%%%%%%%%%%%%%%%%%%%%%%%%%%%%%%%%%
\subsection{Manual Code}
\label{sec:manual}

In case one cannot be certain whether the definitions file |childdoc.def|
is installed on the target \TeX{} distribution
and one prefers not to ship it,
it is conceivable to paste a few relevant commands into the sources.

To that end, drop all statements |\input{childdoc.def}|
and perform the replacements as outlined below.
Instead of |\childdocmain{|\textit{main}|}| add the following code
to the top of the main file:
%
\begin{center}
\begin{tabular}{l}
|\||ifdefined\childdocname\endinput\||fi\newif\ifchilddoc|\\
|\edef\childdocname{\scantokens\expandafter{\jobname\noexpand}}|\\
|\def\childdocmain{|\textit{main}|}\||ifx\childdocmain\childdocname\||else|\\
|\childdoctrue\includeonly{\childdocname}\let\jobname\childdocmain\||fi|\\
\end{tabular}
\end{center}
%
Instead of |\childdocof{|\textit{main}|}| just include the main file
at the top of each child file:
%
\begin{center}
|\input{|\textit{main}|}|
\end{center}
%
A simple redirection |\childdocforward{|\textit{dest}|}| is achieved by:
%
\begin{center}
|\def\jobname{|\textit{dest}|}\input{\jobname}|
\end{center}
%
The redirection with prefix
|\childdocforwardprefix[|\textit{prefix}|]{|\textit{dest}|}|
is accomplished by:
%
\begin{center}
\begin{tabular}{l}
|{\edef\jobname{\scantokens\expandafter{\jobname\noexpand}}|\\
|\def\redirectjob |\textit{prefix}|#1~~~{\gdef\jobname{|\textit{dest}|#1}}|\\
|\expandafter\redirectjob\jobname~~~}\input{\jobname}|
\end{tabular}
\end{center}

In an alternative approach,
child documents can be compiled by a specific command line
without additional code or specific definitions:
%
\begin{center}
|... -jobname "|\textit{target}|" "|[\textit{flags}]%
|\includeonly{|\textit{dest}|}\input{|\textit{main}|}"|
\end{center}
%

%%%%%%%%%%%%%%%%%%%%%%%%%%%%%%%%%%%%%%%%%%%%%%%%%%%%%%%%%%%%%%%%%%%%%%%%%%%%%%%%
%%%%%%%%%%%%%%%%%%%%%%%%%%%%%%%%%%%%%%%%%%%%%%%%%%%%%%%%%%%%%%%%%%%%%%%%%%%%%%%%
\section{Information}

%%%%%%%%%%%%%%%%%%%%%%%%%%%%%%%%%%%%%%%%%%%%%%%%%%%%%%%%%%%%%%%%%%%%%%%%%%%%%%%%
\subsection{Copyright}

Copyright \copyright{} 2017--2018 Niklas Beisert

This work may be distributed and/or modified under the
conditions of the \LaTeX{} Project Public License, either version 1.3
of this license or (at your option) any later version.
The latest version of this license is in
  \url{http://www.latex-project.org/lppl.txt}
and version 1.3 or later is part of all distributions of \LaTeX{}
version 2005/12/01 or later.

This work has the LPPL maintenance status `maintained'.

The Current Maintainer of this work is Niklas Beisert.

This work consists of the files |README.txt|, |childdoc.ins| and |childdoc.dtx|
as well as the derived files |childdoc.def|, |cdocsamp.tex|
with |cdocsch1.tex|, |cdocsch2.tex|, |cdocspt3.tex|, |cdocspt4.tex|,
|cdocsdrf.tex|, |cdocsfn1.tex|, |cdocsfn2.tex|
as well as |childdoc.pdf|.

%%%%%%%%%%%%%%%%%%%%%%%%%%%%%%%%%%%%%%%%%%%%%%%%%%%%%%%%%%%%%%%%%%%%%%%%%%%%%%%%
\subsection{Files and Installation}

The package consists of the files:
%
\begin{center}
\begin{tabular}{ll}
    |README.txt|   & readme file \\
    |childdoc.ins| & installation file \\
    |childdoc.dtx| & source file \\
    |childdoc.def| & definition file \\
    |cdocsamp.tex| & sample main file \\
    |cdocsch1.tex| & sample include file \\
    |cdocsch2.tex| & sample include file \\
    |cdocspt3.tex| & sample part file \\
    |cdocspt4.tex| & sample part file \\
    |cdocsdrf.tex| & sample redirection file \\
    |cdocsfn1.tex| & sample redirection file \\
    |cdocsfn2.tex| & sample redirection file \\
    |childdoc.pdf| & manual
\end{tabular}
\end{center}
%
The distribution consists of the files
|README.txt|, |childdoc.ins| and |childdoc.dtx|.
%
\begin{itemize}
\item
Run (pdf)\LaTeX{} on |childdoc.dtx|
to compile the manual |childdoc.pdf| (this file).
\item
Run \LaTeX{} on |childdoc.ins| to create the definitions file |childdoc.def|
and the sample |cdocsamp.tex| with include files
|cdocsch1.tex|, |cdocsch2.tex|, |cdocspt3.tex|, |cdocspt4.tex|,
|cdocsdrf.tex|, |cdocsfn1.tex|, |cdocsfn2.tex|.
Then copy the file |childdoc.def| to an appropriate directory of your \LaTeX{}
distribution, e.g.\ \textit{texmf-root}|/tex/latex/childdoc|.
\end{itemize}

%%%%%%%%%%%%%%%%%%%%%%%%%%%%%%%%%%%%%%%%%%%%%%%%%%%%%%%%%%%%%%%%%%%%%%%%%%%%%%%%
\subsection{Related CTAN Packages}

There are several other packages which offer a similar functionality:
%
\begin{itemize}
\item
The packages
\href{http://ctan.org/pkg/docmute}{\textsf{docmute}},
\href{http://ctan.org/pkg/includex}{\textsf{includex}} and
\href{http://ctan.org/pkg/standalone}{\textsf{standalone}}
provide commands to include only the document body of
a child file thus allowing both files to be compiled individually.
\item
The packages \href{http://ctan.org/pkg/subdocs}{\textsf{subdocs}}
and \href{http://ctan.org/pkg/subfiles}{\textsf{subfiles}}
provide structures in which the main and child documents can be
encapsulated and allowing them to be compiled individually.
The inclusion mechanism is different from the conventional |\include|.
\item
The package \href{http://ctan.org/pkg/combine}{\textsf{combine}}
is an elaborate solution to combine several documents into one.
\end{itemize}
%
See also the CTAN topic \href{http://ctan.org/topic/subdocs}{\textsf{subdocs}}
for further related packages.
The present package differs from the above solutions in that
a document structure constructed with the conventional |\include| mechanism
just needs two extra commands at the top of every file
such that all constituent files can be compiled individually.

%%%%%%%%%%%%%%%%%%%%%%%%%%%%%%%%%%%%%%%%%%%%%%%%%%%%%%%%%%%%%%%%%%%%%%%%%%%%%%%%
%\subsection{Feature Suggestions}
%
%The following is a list of features which may be useful for future
%versions of this package:
%%
%\begin{itemize}
%\item
%\ldots
%\end{itemize}

%%%%%%%%%%%%%%%%%%%%%%%%%%%%%%%%%%%%%%%%%%%%%%%%%%%%%%%%%%%%%%%%%%%%%%%%%%%%%%%%
\subsection{Revision History}

%%%%%%%%%%%%%%%%%%%%%%%%%%%%%%%%%%%%%%%%
\paragraph{v2.0:} 2018/12/30

\begin{itemize}
\item
immediate forward processing
\item
added |\childdocby| mechanism
\item
manual restructured
\end{itemize}

%%%%%%%%%%%%%%%%%%%%%%%%%%%%%%%%%%%%%%%%
\paragraph{v1.6:} 2018/01/17

\begin{itemize}
\item
application for development of include files
\item
corrections to manual
\end{itemize}

%%%%%%%%%%%%%%%%%%%%%%%%%%%%%%%%%%%%%%%%
\paragraph{v1.5:} 2017/05/21

\begin{itemize}
\item
more complete structuring introduced
\item
|\childdocof| introduced
\item
|\childdoc| renamed to |\childdocmain|
\item
|\childredirect| renamed to |\childdocforward| and |\childdocforwardprefix|
and functionality expanded
\end{itemize}

%%%%%%%%%%%%%%%%%%%%%%%%%%%%%%%%%%%%%%%%
\paragraph{v1.0:} 2017/04/27

\begin{itemize}
\item
manual and install package
\item
first version published on CTAN
\end{itemize}

%%%%%%%%%%%%%%%%%%%%%%%%%%%%%%%%%%%%%%%%
\paragraph{v0.6:} 2017/04/26

\begin{itemize}
\item
redirection mechanism added
\end{itemize}

%%%%%%%%%%%%%%%%%%%%%%%%%%%%%%%%%%%%%%%%
\paragraph{v0.5:} 2017/04/26

\begin{itemize}
\item
functionality in definition file
\end{itemize}


%%%%%%%%%%%%%%%%%%%%%%%%%%%%%%%%%%%%%%%%%%%%%%%%%%%%%%%%%%%%%%%%%%%%%%%%%%%%%%%%
%%%%%%%%%%%%%%%%%%%%%%%%%%%%%%%%%%%%%%%%%%%%%%%%%%%%%%%%%%%%%%%%%%%%%%%%%%%%%%%%
%%%%%%%%%%%%%%%%%%%%%%%%%%%%%%%%%%%%%%%%%%%%%%%%%%%%%%%%%%%%%%%%%%%%%%%%%%%%%%%%
\appendix

\settowidth\MacroIndent{\rmfamily\scriptsize 000\ }

 \DocInput{childdoc.dtx}

\end{document}
%</driver>
% \fi
%
% %%%%%%%%%%%%%%%%%%%%%%%%%%%%%%%%%%%%%%%%%%%%%%%%%%%%%%%%%%%%%%%%%%%%%%%%%%%%%%
% %%%%%%%%%%%%%%%%%%%%%%%%%%%%%%%%%%%%%%%%%%%%%%%%%%%%%%%%%%%%%%%%%%%%%%%%%%%%%%
% \section{Sample}
%\iffalse
%<*samplemain>
%\fi
%
% The following presents a sample document
% with two chapters, two parts, a title page,
% a compile flag as well as three forwarding files to set the flag.
% It consists of eight |.tex| files:
% \begin{center}
% \begin{tabular}{ll}
% |cdocsamp.tex|&main file\\
% |cdocsch1.tex|&include file for chapter 1\\
% |cdocsch2.tex|&include file for chapter 2\\
% |cdocspt3.tex|&include file for part 3\\
% |cdocspt4.tex|&include file for part 4\\
% |cdocsdrf.tex|&forwarding file for main file in draft mode\\
% |cdocsfi1.tex|&forwarding file for final version of chapter 1\\
% |cdocsfi2.tex|&forwarding file for final version of chapter 2\\
% \end{tabular}
% \end{center}
% Each of the eight files can be compiled directly by the \LaTeX{} compiler.
%
% %%%%%%%%%%%%%%%%%%%%%%%%%%%%%%%%%%%%%%
% \paragraph{Main File.}
%
% The main file is called |cdocsamp.tex|.
%
% Load the \textsf{childdoc} definitions and
% declare the filename for the main document:
%    \begin{macrocode}
\input{childdoc.def}
\childdocmain{}
%    \end{macrocode}

% Optional override for |\version| flag:
%    \begin{macrocode}
%%\ifchilddoc\else\providecommand{\version}{draft}\fi
%    \end{macrocode}

% Define the default values for the |\version| flag
% (|final| for the main file and |draft| for childs):
%    \begin{macrocode}
\ifchilddoc
\providecommand{\version}{draft}
\else
\providecommand{\version}{final}
\fi
%    \end{macrocode}

% Load the standard document class:
%    \begin{macrocode}
\documentclass[12pt]{article}
%    \end{macrocode}

% Start the document body:
%    \begin{macrocode}
\begin{document}
%    \end{macrocode}

% Declare a title page.
% Print title, part of document being processed and version flag:
%    \begin{macrocode}
\addtocounter{page}{-1}
\begin{center}
{\LARGE\bfseries{}childdoc example\par}
\vspace{1cm}
\ifchilddoc
\ifchilddocmanual part\else chapter\fi:
`\childdocname' of `\childdocjob'\par
\else
main document: `\childdocjob'\par
\fi
version: \version\par
\end{center}
\newpage
%    \end{macrocode}

% Manually include selected file,
% otherwise process as usual:
%    \begin{macrocode}
\ifchilddocmanual
\section*{part `\childdocname'}
\input{\childdocname}
\else
%    \end{macrocode}

% Include the two chapters:
%    \begin{macrocode}
\include{cdocsch1}
\include{cdocsch2}
%    \end{macrocode}

% Include the two parts unless only chapters should be displayed:
%    \begin{macrocode}
\ifchilddoc\else
\section{part three}
\input{cdocspt3}
\section{part four}
\input{cdocspt4}
\fi
%    \end{macrocode}

% Process as usual until here:
%    \begin{macrocode}
\fi
%    \end{macrocode}

% End of document body:
%    \begin{macrocode}
\end{document}
%    \end{macrocode}
%\iffalse
%</samplemain>
%\fi
%
% %%%%%%%%%%%%%%%%%%%%%%%%%%%%%%%%%%%%%%
% \paragraph{Chapter Include Files.}
%
% The include files are called |cdocsch1.tex| and |cdocsch2.tex|.
%
%\iffalse
%<*samplechap1|samplechap2>
%\fi

% Optional override for |\version| flag:
%    \begin{macrocode}
%%\providecommand{\version}{final}
%    \end{macrocode}

% Include the main document:
%    \begin{macrocode}
\input{childdoc.def}
\childdocof{cdocsamp}
%    \end{macrocode}

%\iffalse
%</samplechap1|samplechap2>
%\fi
%
%\iffalse
%<*samplechap1>
%\fi
% Some text for chapter 1:
%    \begin{macrocode}
\section{one}
some text in chapter one
%    \end{macrocode}

%\iffalse
%</samplechap1>
%\fi
% Some text for chapter 2:
%\iffalse
%<*samplechap2>
%\fi
%    \begin{macrocode}
\section{two}
more text in chapter two
%    \end{macrocode}

%\iffalse
%</samplechap2>
%\fi
%
% %%%%%%%%%%%%%%%%%%%%%%%%%%%%%%%%%%%%%%
% \paragraph{Part Include Files.}
%
% The include files are called |cdocspt3.tex| and |cdocspt4.tex|.
%
%\iffalse
%<*samplepart3|samplepart4>
%\fi

% Optional override for |\version| flag:
%    \begin{macrocode}
%%\providecommand{\version}{final}
%    \end{macrocode}

% Include the main document:
%    \begin{macrocode}
\input{childdoc.def}
\childdocby{cdocsamp}
%    \end{macrocode}

%\iffalse
%</samplepart3|samplepart4>
%\fi
%
%\iffalse
%<*samplepart3>
%\fi
% Some text for part 3:
%    \begin{macrocode}
some text in part three
%    \end{macrocode}

%\iffalse
%</samplepart3>
%\fi
% Some text for part 4:
%\iffalse
%<*samplepart4>
%\fi
%    \begin{macrocode}
more text in part four
%    \end{macrocode}

%\iffalse
%</samplepart4>
%\fi
%
% %%%%%%%%%%%%%%%%%%%%%%%%%%%%%%%%%%%%%%
% \paragraph{Forwarding for a Complete Draft.}
%
% The following forwarding file |cdocsdrf.tex|
% compiles the main document in draft mode:
%\iffalse
%<*sampledraft>
%\fi
%    \begin{macrocode}
\def\version{draft}
\input{childdoc.def}
\childdocforward{cdocsamp}
%    \end{macrocode}

%\iffalse
%</sampledraft>
%\fi
%
% %%%%%%%%%%%%%%%%%%%%%%%%%%%%%%%%%%%%%%
% \paragraph{Forwarding for Final Version of the Chapters.}
%
% The following forwarding files |cdocsfn1.tex| and |cdocsfn2.tex|
% (with identical content)
% compile the final versions of the child documents
% |cdocsch1.tex| and |cdocsch2.tex|, respectively:
%\iffalse
%<*samplefinal>
%\fi
%    \begin{macrocode}
\def\version{final}
\input{childdoc.def}
\childdocforwardprefix[cdocsamp]{cdocsfn}{cdocsch}
%    \end{macrocode}

%\iffalse
%</samplefinal>
%\fi
%
% %%%%%%%%%%%%%%%%%%%%%%%%%%%%%%%%%%%%%%
% \paragraph{Command Line Processing.}
%
% The following three command lines generate the output files
% |cdocscld|, |cdocscl1| and |cdocscl2|
% which should be identical to
% |cdocsdrf|, |cdocsch1| and |cdocsfn2|, respectively:
% \begin{center}
% \begin{tabular}{l}
% |latex -jobname cdocscld \|\\
% |  "\def\version{draft}\input{childdoc.def}\childdocforward{cdocsamp}"|\\
% |latex -jobname cdocscl1 \|\\
% |  "\input{childdoc.def}\childdocforward[cdocsamp]{cdocsch1}"|\\
% |latex -jobname cdocscl2 \|\\
% |  "\def\version{final}\input{childdoc.def}\childdocforward{cdocsch2}"|
% \end{tabular}
% \end{center}
% Note that the trailing backslash on each first line
% merely continues the input to the second line
% (for convenient cut ant paste).
% Furthermore, the command |latex| can be replaced by any
% of its alternative versions such as |pdflatex|.
%
% %%%%%%%%%%%%%%%%%%%%%%%%%%%%%%%%%%%%%%%%%%%%%%%%%%%%%%%%%%%%%%%%%%%%%%%%%%%%%%
% %%%%%%%%%%%%%%%%%%%%%%%%%%%%%%%%%%%%%%%%%%%%%%%%%%%%%%%%%%%%%%%%%%%%%%%%%%%%%%
% \section{Implementation}
%\iffalse
%<*package>
%\fi
%
% This section describes the definitions file |childdoc.def|.

% The definitions cannot be loaded using |\usepackage| or |\RequirePackage|
% which has a mechanism to prevent loading a style file more than once.
% When loading the definitions by means of |\input|
% multiple instances have to be prevented manually:
%\iffalse
%This code needs to be before the `\ProvidesFile' directive
%which is defined at the beginning of this file.
%Therefore it is also placed there and commented out here.
%</package>
%<*discard>
%\fi
%    \begin{macrocode}
\ifdefined\childdocmain\endinput\fi
%    \end{macrocode}
%\iffalse
%</discard>
%<*package>
%\fi
%
% \macro{\ifchilddoc}
% \macro{\ifchilddocmanual}
% The conditional |\ifchilddoc| tells whether a
% child (true) or main (false) document is being compiled.
% The conditional |\ifchilddocmanual| tells whether
% the |\includeonly| mechanism is used (false) or
% the selection of child files must be performed manually (true).
% The definitions initialise to false:
%    \begin{macrocode}
\newif\ifchilddoc
\newif\ifchilddocmanual
%    \end{macrocode}

% \macro{\childdocname}
% \macro{\childdocjob}
% The macro |\childdocname| stores the name of the main document
% to be compiled. The macro |\childdocjob| stores the name of
% the document on which the \LaTeX{} compiler was originally invoked.
% The content of |\jobname| cannot be compared
% to filenames specified in the source due to different catcodes.
% The following code rescans |\jobname|, stores the result
% in |\childdocname| and saves a copy in |\childdocjob|:
%    \begin{macrocode}
\edef\childdocname{\scantokens\expandafter{\jobname\noexpand}}
\let\childdocjob\childdocname
%    \end{macrocode}

% \macro{\childdocdisable}
% The macro |\childdocdisable| prevents the main file
% from being processed more than once.
% At this stage, the main document command |\childdocmain|
% is assumed to be called once again where it should do nothing.
% Any subsequent call to it should prevent
% a secondary processing of the main document
% It overwrites the forwarding commands
% |\childdocof| and |\childdocforward|
% with empty macros to prevent further inclusions of the main document:
%    \begin{macrocode}
\newcommand{\childdocdisable}
{
  \renewcommand{\childdocmain}[1]{\renewcommand{\childdocmain}[1]{\endinput}}
  \renewcommand{\childdocof}[1]{}
  \renewcommand{\childdocby}[2][]{}
  \renewcommand{\childdocforward}[2][]{}
  \renewcommand{\childdocdisable}{}
}
%    \end{macrocode}

% \macro{\childdocmain}
% The macro |\childdocmain| is to be called at the top of the main file
% with nothing or the main filename (without extension) as argument.
% First, it breaks loops.
% If the argument is not empty and does not match |\childdocname|
% (which is set by the first inclusion of |childdoc.def|),
% |\ifchilddoc| is set to true, |\includeonly| is applied to the child file
% and |\jobname| is set to the main file
% (for proper handling of |.aux| files):
%    \begin{macrocode}
\newcommand{\childdocmain}[1]
{
  \childdocdisable\childdocmain{}
  \if?#1?\else
    \begingroup
      \def\childdoctmp{#1}
      \ifx\childdoctmp\childdocname
        \def\childdoctmp{}
      \else
        \def\childdoctmp
        {
          \childdoctrue
          \includeonly{\childdocname}
          \def\childdocjob{#1}
          \def\jobname{#1}
        }
      \fi
      \expandafter
    \endgroup
    \childdoctmp
  \fi
}
%    \end{macrocode}

% \macro{\childdocof}
% The command |\childdocof| redirects
% compilation to the main file |#1|.
%    \begin{macrocode}
\newcommand{\childdocof}[1]
{
  \childdocdisable
  \childdoctrue
  \includeonly{\childdocname}
  \def\jobname{#1}
  \def\childdocjob{#1}
  \input{#1}
}
%    \end{macrocode}

% \macro{\childdocby}
% The command |\childdocby| ....
%    \begin{macrocode}
\newcommand{\childdocby}[2][]
{
  \childdocdisable
  \childdoctrue
  \childdocmanualtrue
  \if?#1?\else
    \def\jobname{#2}
  \fi
  \def\childdocjob{#2}
  \input{#2}
  \endinput
}
%    \end{macrocode}

% \macro{\childdocforward}
% The command |\childdocforward| redirects
% compilation to the main file or
% (if the optional argument is given) a child file.
% Parameters are set as if the main file
% or a child file starting with |\childdocof| was compiled.
% Then compilation is handed over to the main file:
%    \begin{macrocode}
\newcommand{\childdocforward}[2][]
{
  \begingroup
    \if?#1?
      \def\childdoctmp
      {
        \def\childdocname{#2}
        \def\childdocjob{#2}
        \def\jobname{#2}
        \input{#2}
        \endinput
      }
    \else
      \def\childdoctmp
      {
        \childdocdisable
        \def\childdocname{#2}
        \childdoctrue
        \includeonly{#2}
        \def\childdocjob{#1}
        \def\jobname{#1}
        \input{#1}
        \endinput
      }
    \fi
    \expandafter
  \endgroup
  \childdoctmp
}
%    \end{macrocode}

% \macro{\childdocforwardprefix}
% The command |\childdocforwardprefix| redirects
% compilation to the main or a child file by means of a pattern.
% The prefix |#1| in the current filename is replaced by |#2|
% and the suffix of the current filename is kept
% (it is assumed that the filename does not contain the substring `|~~~|'
% which is used as a delimiter).
% Compilation is handed over to the new file by |\childdocforward|:
%    \begin{macrocode}
\newcommand{\childdocforwardprefix}[3][]
{
  \begingroup
    \def\childdocextract #2##1~~~{\def\childdoctmp{\childdocforward[#1]{#3##1}}}
    \expandafter\childdocextract\childdocname~~~
    \expandafter
  \endgroup
  \childdoctmp
}
%    \end{macrocode}

% \macro{\childdoc}
% The deprecated macro |\childdoc| is a legacy version of |\childdocmain|:
%    \begin{macrocode}
\newcommand{\childdoc}{\childdocmain}
%    \end{macrocode}

% \macro{\childdocredirect}
% The deprecated macro |\childdocredirect| is a legacy version
% of |\childdocforward| and |\childdocforwardprefix|:
%    \begin{macrocode}
\newcommand{\childdocredirect}[2][]
{
  \begingroup
    \if?#1?
      \def\childdoctmp{\childdocforward{#2}}
    \else
      \def\childdoctmp{\childdocforwardprefix{#1}{#2}}
    \fi
    \expandafter
  \endgroup
  \childdoctmp
}
%    \end{macrocode}

%\iffalse
%</package>
%\fi
%
\endinput
|\\
|\childdocforwardprefix[|\textit{main}|]{|\textit{prefix}|}{|\textit{dest}|}|
\end{tabular}
\end{center}
%
the destination file is determined by a pattern
depending on the current file:
To make this work, the current file must be called
`{\textit{prefix}\hspace{0.2em}\textit{suffix}}'
with \textit{prefix} matching precisely the argument.
Processing is then passed on to the file
`{\textit{dest}\hspace{0.2em}\textit{suffix}}'.
Surely, the same effect is achieved by
directly specifying the
argument `{\textit{dest}\hspace{0.2em}\textit{suffix}}'
in the first form.
However, that requires to set up a different file
for each child. With the alternative form of the command
all these files can have exactly the same content
which simplifies setting them up and maintaining them.

For example, the following file |draft.tex|
with a compilation flag |\version| as described in \secref{sec:flags}
compiles the main document as a draft:
%
\begin{center}
\begin{tabular}{l}
|\def\version{draft}|\\
|% \iffalse
%
% childdoc.dtx Copyright (C) 2017-2018 Niklas Beisert
%
% This work may be distributed and/or modified under the
% conditions of the LaTeX Project Public License, either version 1.3
% of this license or (at your option) any later version.
% The latest version of this license is in
%   http://www.latex-project.org/lppl.txt
% and version 1.3 or later is part of all distributions of LaTeX
% version 2005/12/01 or later.
%
% This work has the LPPL maintenance status `maintained'.
%
% The Current Maintainer of this work is Niklas Beisert.
%
% This work consists of the files childdoc.dtx and childdoc.ins
% and the derived files childdoc.def and cdocsamp.tex with
% cdocsch1.tex, cdocsch2.tex, cdocsdrf.tex, cdocsfn1.tex, cdocsfn2.tex.
%
%<package>\ifdefined\childdocmain\endinput\fi
%<package>\ProvidesFile{childdoc.def}[2018/12/30 v2.0 child document driver]
%<samplemain>\ProvidesFile{cdocsamp.tex}[2018/12/30 v2.0 sample for childdoc]
%<*driver>
%\ProvidesFile{childdoc.drv}[2018/12/30 v2.0 childdoc reference manual file]
\PassOptionsToClass{10pt,a4paper}{article}
\documentclass{ltxdoc}

\usepackage[margin=35mm]{geometry}
\usepackage{hyperref}
\usepackage{hyperxmp}
\usepackage[usenames]{color}

\hypersetup{colorlinks=true}
\hypersetup{pdfstartview=FitH}
\hypersetup{pdfpagemode=UseNone}
\hypersetup{pdfsource={}}
\hypersetup{pdflang={en-UK}}
\hypersetup{pdfcopyright={Copyright 2017-2018 Niklas Beisert.
  This work may be distributed and/or modified under the
  conditions of the LaTeX Project Public License, either version 1.3
  of this license or (at your option) any later version.}}
\hypersetup{pdflicenseurl={http://www.latex-project.org/lppl.txt}}
\hypersetup{pdfcontactaddress={ETH Zurich, ITP, HIT K,
  Wolfgang-Pauli-Strasse 27}}
\hypersetup{pdfcontactpostcode={8093}}
\hypersetup{pdfcontactcity={Zurich}}
\hypersetup{pdfcontactcountry={Switzerland}}
\hypersetup{pdfcontactemail={nbeisert@itp.phys.ethz.ch}}
\hypersetup{pdfcontacturl={http://people.phys.ethz.ch/\xmptilde nbeisert/}}

\newcommand{\secref}[1]{\hyperref[#1]{section \ref*{#1}}}

\parskip1ex
\parindent0pt
\let\olditemize\itemize
\def\itemize{\olditemize\parskip0pt}

\begin{document}

\title{The \textsf{childdoc} Package}
\hypersetup{pdftitle={The childdoc Package}}
\author{Niklas Beisert\\[2ex]
  Institut f\"ur Theoretische Physik\\
  Eidgen\"ossische Technische Hochschule Z\"urich\\
  Wolfgang-Pauli-Strasse 27, 8093 Z\"urich, Switzerland\\[1ex]
  \href{mailto:nbeisert@itp.phys.ethz.ch}
  {\texttt{nbeisert@itp.phys.ethz.ch}}}
\hypersetup{pdfauthor={Niklas Beisert}}
\hypersetup{pdfsubject={Manual for the LaTeX2e Package childdoc}}
\date{30 December 2018, \textsf{v2.0}}
\maketitle

\begin{abstract}\noindent
\textsf{childdoc} is a \LaTeXe{} package
that enables the direct compilation
of document sections included by |\include|
to individual files.
\end{abstract}

\begingroup
\parskip0ex
\tableofcontents
\endgroup

%%%%%%%%%%%%%%%%%%%%%%%%%%%%%%%%%%%%%%%%%%%%%%%%%%%%%%%%%%%%%%%%%%%%%%%%%%%%%%%%
%%%%%%%%%%%%%%%%%%%%%%%%%%%%%%%%%%%%%%%%%%%%%%%%%%%%%%%%%%%%%%%%%%%%%%%%%%%%%%%%
\section{Introduction}

\LaTeX{} provides a mechanism to structure a large document (such as a book)
into a main file and several child files (containing the chapters)
using the |\include| command.
This mechanism is beneficial for documents
which span hundreds of pages in order to
make the source file(s) more manageable.
Moreover, compilation can be restricted to
selected child files by means of the |\includeonly| command.
The latter feature can be used to reduce the compilation time while editing
(this was significantly more useful in the earlier days of \LaTeX{})
or to generate a smaller document which is easier to navigate.
Another application of |\includeonly| is to generate
documents consisting of selected parts of the complete document.

However, there are a few drawbacks of the plain |\include| mechanism:
\begin{itemize}
\item
The child files cannot be compiled on their own,
they can only be compiled via the main file.
A naive editing environment
(such as a text editor with an option
to have the current file processed by \LaTeX)
may require one to switch to the main file before compiling;
attempting to compile the child file produces errors.
\item
The main file must be modified (each time)
to adjust the |\includeonly| command
to the present needs. This easily leaves the main file in a messy state.
\item
The generated document will always carry the filename
of the main document. This is inconvenient if
several child files are to be compiled and
to be kept for distribution.
\end{itemize}

The present package provides a simple interface
to make child files individually compilable by \LaTeX{}.
Compiling a child file then has the same effect as compiling
the main file with an |\includeonly| command
to select the appropriate child.
Moreover the generated document will carry the name of the child
rather than the main file.
This resolves all three above issues.

This feature is meant to make the editing of books,
thesis documents and lecture notes somewhat more convenient.
However, the package can also be used efficiently for
composing a series of documents (such as exercise sheets)
which are typically distributed individually.
It then assists the author in generating the individual documents
(potentially in different versions)
as well as a document containing the collected series.
Another application is in developing style files
or other kinds of included material
where compilation of the style file could redirect
to a sample or test file.

%%%%%%%%%%%%%%%%%%%%%%%%%%%%%%%%%%%%%%%%%%%%%%%%%%%%%%%%%%%%%%%%%%%%%%%%%%%%%%%%
%%%%%%%%%%%%%%%%%%%%%%%%%%%%%%%%%%%%%%%%%%%%%%%%%%%%%%%%%%%%%%%%%%%%%%%%%%%%%%%%
\section{Usage}

First of all, the package \textsf{childdoc} is \emph{not} a standard
\LaTeXe{} |.sty| style file! Therefore it needs to be invoked in
a non-standard way.

%%%%%%%%%%%%%%%%%%%%%%%%%%%%%%%%%%%%%%%%%%%%%%%%%%%%%%%%%%%%%%%%%%%%%%%%%%%%%%%%
\subsection{Included Files}
\label{sec:include}

%%%%%%%%%%%%%%%%%%%%%%%%%%%%%%%%%%%%%%%%
\DescribeMacro{\childdocmain}
To use the package, add the commands
\begin{center}
\begin{tabular}{l}
|\input{childdoc.def}|\\
|\childdocmain{}|\\
\end{tabular}
\end{center}
at the very top of the main \LaTeX{} file,
in particular \emph{before} the |\documentclass| statement!
The argument of |\childdocmain| should be left empty
(but it must be present).

%%%%%%%%%%%%%%%%%%%%%%%%%%%%%%%%%%%%%%%%
\DescribeMacro{\childdocof}
Furthermore, add the commands
\begin{center}
\begin{tabular}{l}
|\input{childdoc.def}|\\
|\childdocof{|\textit{main}|}|\\
\end{tabular}
\end{center}
at the top of every child file \textit{child}
which is included by |\include{|\textit{child}|}|
from within the main file
(or at least for those files to be compiled individually).
The argument \textit{main} must be the filename of the main file.

There are a couple of
considerations in setting up the main and child documents:

%%%%%%%%%%%%%%%%%%%%%%%%%%%%%%%%%%%%%%%%
\paragraph{Restrictions.}

Please note the following restrictions:
\begin{itemize}
\item
|\childdocmain| must be called with one argument \textit{main}
to ensure compatibility with earlier version of the package.
It must either be empty (|\childdocmain{}|)
or precisely match the filename of the main file in which it is specified.
See \secref{sec:detection} for further information.
\item
The filename \textit{main} must be specified without the |.tex| extension.
\item
The filename \textit{main} is case sensitive
(even in case-insensitive file systems)
due to internal string comparison.
\item
The argument \textit{main} should be fully expanded, it cannot be a macro.
\item
Subdirectories and special characters should be avoided in filenames.
\item
The command |\childdocmain{|\textit{main}|}| must be followed by a whitespace.
It should not be followed immediately by another command
or by a comment mark `|%|'.
This is because the \TeX{} parser reads the token immediately following
the argument of |\childdocmain| and puts it
at the beginning of every child section;
however, a white\-space is ignored.
\end{itemize}

%%%%%%%%%%%%%%%%%%%%%%%%%%%%%%%%%%%%%%%%
\paragraph{Content of Main File.}

It is advisable to place all content in the child files included by |\include|.
Any output contained in the main file will appear in all child documents
unless suppressed manually;
it cannot be suppressed automatically by the |\includeonly| directive
and thus should normally be avoided.
A method to include some content in the main file
by means of conditional processing is described in \secref{sec:conditional}.

%%%%%%%%%%%%%%%%%%%%%%%%%%%%%%%%%%%%%%%%
\paragraph{Page Numbering.}

When only a part of the document is compiled,
the appropriate numbering of pages
(as well as other status parameters)
is determined from the |.aux| files.
The latter contain information from previous passes.
However this information needs to propagate through
all intermediate child documents.
Therefore the page numbering in child documents may well
be inconsistent until the complete document is compiled at least once.

A useful (if unconventional) way to always ensure a consistent
page numbering is to restart the numbering in each child document
and denote the pages by `\textit{child}|.|\textit{page}'
where \textit{child} represents the chapter/section number of the child file.
This can be achieved by the command
|\numberwithin{page}{|\textit{child}|}|
of the \textsf{amsmath} package
where \textit{child} can be |chapter| or |section|
depending on the chosen structuring.
Alternatively, one can modify the macro |\thepage| appropriately
and reset the counter |page| at the start of each child file.

%%%%%%%%%%%%%%%%%%%%%%%%%%%%%%%%%%%%%%%%%%%%%%%%%%%%%%%%%%%%%%%%%%%%%%%%%%%%%%%%
\subsection{Conditional Processing}
\label{sec:conditional}

The package provides a mechanism to compile different versions
of a document. To customise the versions further some conditional processing
can come in handy to distinguish which version is being compiled.
The package provides two macros to describe the compilation context:

%%%%%%%%%%%%%%%%%%%%%%%%%%%%%%%%%%%%%%%%
\DescribeMacro{\ifchilddoc}
The conditional |\ifchilddoc| distinguishes between the compilation of
child documents and the main document:
%
\begin{center}
|\ifchilddoc |\textit{child-code}| |[|\||else |\textit{main-code}]| \||fi|
\end{center}

%%%%%%%%%%%%%%%%%%%%%%%%%%%%%%%%%%%%%%%%
\DescribeMacro{\childdocname}
\DescribeMacro{\childdocjob}
The macro |\childdocname| contains the filename (without extension)
of the main or child file being processed.
Note that |\childdocjob| will always contain the name of the main file.

%%%%%%%%%%%%%%%%%%%%%%%%%%%%%%%%%%%%%%%%
\paragraph{Title Page.}

Conditional processing can be used to include a title or banner page
in the main document when proper precautions are taken.
Importantly, the code in the main file should ensure that the page counter
(as well as other status parameters which are stored in the |.aux| files)
takes the same value after the conditional processing.
Otherwise the page numbers may take divergent values
depending on which part is compiled.

For example, a title page could be declared by:
%
\begin{center}
\begin{tabular}{l}
|\ifchilddoc\||else|\\
|\addtocounter{page}{-1}|\\
\textit{code for title page}\\
|\newpage|\\
|\||fi|
\end{tabular}
\end{center}
%
A banner page for the child documents can be generated by:
%
\begin{center}
\begin{tabular}{l}
|\ifchilddoc|\\
|\addtocounter{page}{-1}|\\
\textit{code for banner page}\\
|\newpage|\\
|\||fi|
\end{tabular}
\end{center}
%
Here one could write a message such as:
\begin{center}
|This is the part \childdocname{} of \childdocjob{}.|
\end{center}

%%%%%%%%%%%%%%%%%%%%%%%%%%%%%%%%%%%%%%%%%%%%%%%%%%%%%%%%%%%%%%%%%%%%%%%%%%%%%%%%
\subsection{Flags}
\label{sec:flags}

The package makes it easy to generate different versions
of the main or child documents.
To this end compilation flags can be defined
and assigned different default values.
They will be particularly useful in conjunction
with the forwarding mechanism described in \secref{sec:forward}.

For example, it may be useful to have a flag |\version|
which can be set to |draft| or |final|.
The document source will contain some conditional code
depending on the value of |\version|.
Suppose further, the flag should default to |final| for the main file
and to |draft| for child files
which is a natural assignment for editing the document.
This is achieved by placing the following code
in the preamble of the main document
(below the |\childdocmain| directive):
%
\begin{center}
\begin{tabular}{l}
|\ifchilddoc|\\
|\providecommand{\version}{draft}|\\
|\||else|\\
|\providecommand{\version}{final}|\\
|\||fi|
\end{tabular}
\end{center}
%
The definition by |\providecommand| makes sure
that previous definitions are not overwritten.
Further statements |\providecommand{\version}{...}|
can thus be added before the above code to override it.

For the main file, one might add a line
(between |\childdocmain| and the above block)
%
\begin{center}
|%\ifchilddoc\||else\providecommand{\version}{draft}\||fi|
\end{center}
%
which can be uncommented to produce a draft version.
Likewise one can add a line to the very top of a child file
(above the |\childdocof{|\textit{main}|}| directive)
%
\begin{center}
|%\providecommand{\version}{final}|
\end{center}
%
which can be uncommented to produce the final version of this child document.

%%%%%%%%%%%%%%%%%%%%%%%%%%%%%%%%%%%%%%%%%%%%%%%%%%%%%%%%%%%%%%%%%%%%%%%%%%%%%%%%
\subsection{Forwarding}
\label{sec:forward}

Different versions of the main or child documents
using compilation flags as described in \secref{sec:flags}
can be (permanently) stored in different files
for convenient compilation, viewing and distribution.
To this end, the package defines a command
to pass on compilation to a different file:

%%%%%%%%%%%%%%%%%%%%%%%%%%%%%%%%%%%%%%%%
\DescribeMacro{\childdocforward}
The command |\childdocforward| redirects processing to
another source file:
%
\begin{center}
\begin{tabular}{l}
|\input{childdoc.def}|\\
|\childdocforward[|\textit{main}|]{|\textit{dest}|}|\\
\end{tabular}
\end{center}
%
The argument \textit{dest} is the destination file
(without extension).
It should be the main file or one of the child files.
Note that further \textsf{childdoc} directives
such as |\childdocof| and |\childdocforward|
in the indicated file will be processed in this form.
The optional argument \textit{main}
passes on directly to the main file \textit{main}
while pretending to compile the child \textit{dest}.
This form behaves as if \textit{dest}
issues |\childdocof{|\textit{main}|}| right away,
and no further \textsf{childdoc} directives will be processed.

%%%%%%%%%%%%%%%%%%%%%%%%%%%%%%%%%%%%%%%%
\DescribeMacro{\...prefix}
In the alternative form |\childdocforwardprefix|,
%
\begin{center}
\begin{tabular}{l}
|\input{childdoc.def}|\\
|\childdocforwardprefix[|\textit{main}|]{|\textit{prefix}|}{|\textit{dest}|}|
\end{tabular}
\end{center}
%
the destination file is determined by a pattern
depending on the current file:
To make this work, the current file must be called
`{\textit{prefix}\hspace{0.2em}\textit{suffix}}'
with \textit{prefix} matching precisely the argument.
Processing is then passed on to the file
`{\textit{dest}\hspace{0.2em}\textit{suffix}}'.
Surely, the same effect is achieved by
directly specifying the
argument `{\textit{dest}\hspace{0.2em}\textit{suffix}}'
in the first form.
However, that requires to set up a different file
for each child. With the alternative form of the command
all these files can have exactly the same content
which simplifies setting them up and maintaining them.

For example, the following file |draft.tex|
with a compilation flag |\version| as described in \secref{sec:flags}
compiles the main document as a draft:
%
\begin{center}
\begin{tabular}{l}
|\def\version{draft}|\\
|\input{childdoc.def}|\\
|\childdocforward{|\textit{main}|}|
\end{tabular}
\end{center}
%
Likewise, the following files |final|\textit{nn}|.tex|
compile the final version of the child document
|child|\textit{nn}|.tex|:
%
\begin{center}
\begin{tabular}{l}
|\def\version{final}|\\
|\input{childdoc.def}|\\
|\childdocforwardprefix{final}{child}|
\end{tabular}
\end{center}
%

Note that when several versions of a main file and/or of each child file
are to be generated, it may be convenient to set up a |Makefile| or
shell script to automatise the process.

%%%%%%%%%%%%%%%%%%%%%%%%%%%%%%%%%%%%%%%%%%%%%%%%%%%%%%%%%%%%%%%%%%%%%%%%%%%%%%%%
\subsection{Command Line Processing}
\label{sec:commandline}

The effect of redirection files can also be achieved by invoking
the \LaTeX{} compiler with a more elaborate command line.
Most conveniently this should be done as part
of a shell script or a |Makefile|.

When using \textsf{childdoc} in the main file, the following
command lines effectively perform a redirection
(note that depending on the shell being used,
backslashes may have to be doubled: `|\|' $\to$ `|\\|'):
%
\begin{center}
|... -jobname "|\textit{target}|" |\\|"|[\textit{flags}]%
|\input{childdoc.def}\childdocforward[|\textit{main}|]{|\textit{dest}|}"|
\end{center}
%
Here \textit{target} is the name of the output file,
\textit{main} is the name of the main file
and \textit{dest} is the name of the main or child file to be processed
(all filenames without extensions).
The optional argument \textit{main} can be omitted
if \textit{main} matches \textit{dest}.
Optionally, compilation \textit{flags} can be defined via |\def| commands.
This command line makes the \TeX{} engine believe
it is compiling the file \textit{target}
whose content is specified as the latter parameter.
The provided code then forwards the processing to
\textit{main} or \textit{dest} as described in \secref{sec:forward}.

%%%%%%%%%%%%%%%%%%%%%%%%%%%%%%%%%%%%%%%%%%%%%%%%%%%%%%%%%%%%%%%%%%%%%%%%%%%%%%%%
\subsection{Include by Input}
\label{sec:input}

Including child documents by |\include| has some restrictions by design.
Most notably, the content of a child document always occupies
its own set of pages; pages cannot be shared between child documents.
Usually, this behaviour makes perfect sense
because each child document contain an essential part of the document.
However, in some situations it may be desirable to compose
a document from a collection of parts
without having mandatory page breaks between then.
For this case, the package
provides a mechanism to include parts
by |\input| which can also be processed individually.
However, by construction this mechanism
requires manual handling of the content to be output.

%%%%%%%%%%%%%%%%%%%%%%%%%%%%%%%%%%%%%%%%
\DescribeMacro{\ifchilddocmanual}
The main file should be prepared as usual, see \secref{sec:include}.
However, the document body must make a distinction
between processing of an individual part and of the main document, e.g.:
%
\begin{center}
\begin{tabular}{l}
|\ifchilddocmanual|\\
|\input{\childdocname}|\\
|\||else|\\
\textit{document body with }|\input{|\textit{part}|}|\\
|\||fi|
\end{tabular}
\end{center}
%
The conditional |\ifchilddocmanual| is true whenever
a part to be included by |\input| is being compiled,
and the name of the part is stored in |\childdocname|.

%%%%%%%%%%%%%%%%%%%%%%%%%%%%%%%%%%%%%%%%
\DescribeMacro{\childdocby}
Each part to be included by |\input| should start with:
%
\begin{center}
\begin{tabular}{l}
|\input{childdoc.def}|\\
|\childdocby{|\textit{main}|}|\\
\end{tabular}
\end{center}
%
The directive |\childdocby| is similar to |\childdocof|
described in \secref{sec:include},
but the subsequent selection of content must be done manually.
To that end, both |\ifchilddoc| and |\ifchilddocmanual|
will be true upon processing of a part,
and the name of the part is stored in |\childdocname|.
Note that |\jobname| will be set to the filename of the current part
so that each part receives an individual |.aux| file
that does not interfere with the |.aux| file(s) of the main document.
This behaviour can be altered by the alternative form
|\childdocby[*]{|\textit{main}|}| (with a non-empty optional argument)
which uses the |.aux| file of the main document
by setting |\jobname| to \textit{main}.

%%%%%%%%%%%%%%%%%%%%%%%%%%%%%%%%%%%%%%%%%%%%%%%%%%%%%%%%%%%%%%%%%%%%%%%%%%%%%%%%
\subsection{Driver Development}
\label{sec:driver}

The \textsf{childdoc} mechanism can also be use for the development
of definition files such as \LaTeX{} styles or classes.
This case differs from the above setup with multiple parts
included by |\include| in that no |\includeonly| should be invoked.
This can be achieved by starting the include file
(before |\ProvidesPackage|) with:
%
\begin{center}
\begin{tabular}{l}
|\input{childdoc.def}|\\
|\childdocforward{|\textit{main}|}|\\
\end{tabular}
\end{center}
%
or alternatively with:
%
\begin{center}
\begin{tabular}{l}
|\input{childdoc.def}|\\
|\childdocby{|\textit{main}|}|\\
\end{tabular}
\end{center}
%
Both forms have slightly different effects as described above.
The main file is prepared as usual, see \secref{sec:include}.

%%%%%%%%%%%%%%%%%%%%%%%%%%%%%%%%%%%%%%%%%%%%%%%%%%%%%%%%%%%%%%%%%%%%%%%%%%%%%%%%
\subsection{Legacy Detection}
\label{sec:detection}

The directive |\childdocmain| in the main file can detect
whether the complete document or merely a child is to be compiled
even without using the directive |\childdocof|.
This method is deprecated because it is less robust
and there is no compelling reason to use it;
it is merely provided for backward compatibility
and it may be removed in future versions.

If the detection mechanism is to be used,
it is mandatory to correctly specify
the filename of the main file as the argument of |\childdocmain|:
%
\begin{center}
\begin{tabular}{l}
|\input{childdoc.def}|\\
|\childdocmain{|\textit{main}|}|\\
\end{tabular}
\end{center}
%
If |\jobname| does not match the argument \textit{main} of |\childdocmain|,
it is assumed that |\jobname| points to the child file to be compiled.
When using |\childdocmain| with the main file specified as argument,
it suffices to start a child file
with just |\input{|\textit{main}|}|
without loading of the package and using |\childdocof|.
If instead all processing is done
with the appropriate \textsf{childdoc} directives,
the argument of \textit{main} of |\childdocmain| can be empty.

An alternative version of the command line processing described
in \secref{sec:commandline} using the detection mechanism reads:
%
\begin{center}
|... -jobname "|\textit{target}|" "|[\textit{flags}]%
[|\def\jobname{|\textit{dest}|}|]|\input{|\textit{main}|}"|
\end{center}

%%%%%%%%%%%%%%%%%%%%%%%%%%%%%%%%%%%%%%%%%%%%%%%%%%%%%%%%%%%%%%%%%%%%%%%%%%%%%%%%
\subsection{Manual Code}
\label{sec:manual}

In case one cannot be certain whether the definitions file |childdoc.def|
is installed on the target \TeX{} distribution
and one prefers not to ship it,
it is conceivable to paste a few relevant commands into the sources.

To that end, drop all statements |\input{childdoc.def}|
and perform the replacements as outlined below.
Instead of |\childdocmain{|\textit{main}|}| add the following code
to the top of the main file:
%
\begin{center}
\begin{tabular}{l}
|\||ifdefined\childdocname\endinput\||fi\newif\ifchilddoc|\\
|\edef\childdocname{\scantokens\expandafter{\jobname\noexpand}}|\\
|\def\childdocmain{|\textit{main}|}\||ifx\childdocmain\childdocname\||else|\\
|\childdoctrue\includeonly{\childdocname}\let\jobname\childdocmain\||fi|\\
\end{tabular}
\end{center}
%
Instead of |\childdocof{|\textit{main}|}| just include the main file
at the top of each child file:
%
\begin{center}
|\input{|\textit{main}|}|
\end{center}
%
A simple redirection |\childdocforward{|\textit{dest}|}| is achieved by:
%
\begin{center}
|\def\jobname{|\textit{dest}|}\input{\jobname}|
\end{center}
%
The redirection with prefix
|\childdocforwardprefix[|\textit{prefix}|]{|\textit{dest}|}|
is accomplished by:
%
\begin{center}
\begin{tabular}{l}
|{\edef\jobname{\scantokens\expandafter{\jobname\noexpand}}|\\
|\def\redirectjob |\textit{prefix}|#1~~~{\gdef\jobname{|\textit{dest}|#1}}|\\
|\expandafter\redirectjob\jobname~~~}\input{\jobname}|
\end{tabular}
\end{center}

In an alternative approach,
child documents can be compiled by a specific command line
without additional code or specific definitions:
%
\begin{center}
|... -jobname "|\textit{target}|" "|[\textit{flags}]%
|\includeonly{|\textit{dest}|}\input{|\textit{main}|}"|
\end{center}
%

%%%%%%%%%%%%%%%%%%%%%%%%%%%%%%%%%%%%%%%%%%%%%%%%%%%%%%%%%%%%%%%%%%%%%%%%%%%%%%%%
%%%%%%%%%%%%%%%%%%%%%%%%%%%%%%%%%%%%%%%%%%%%%%%%%%%%%%%%%%%%%%%%%%%%%%%%%%%%%%%%
\section{Information}

%%%%%%%%%%%%%%%%%%%%%%%%%%%%%%%%%%%%%%%%%%%%%%%%%%%%%%%%%%%%%%%%%%%%%%%%%%%%%%%%
\subsection{Copyright}

Copyright \copyright{} 2017--2018 Niklas Beisert

This work may be distributed and/or modified under the
conditions of the \LaTeX{} Project Public License, either version 1.3
of this license or (at your option) any later version.
The latest version of this license is in
  \url{http://www.latex-project.org/lppl.txt}
and version 1.3 or later is part of all distributions of \LaTeX{}
version 2005/12/01 or later.

This work has the LPPL maintenance status `maintained'.

The Current Maintainer of this work is Niklas Beisert.

This work consists of the files |README.txt|, |childdoc.ins| and |childdoc.dtx|
as well as the derived files |childdoc.def|, |cdocsamp.tex|
with |cdocsch1.tex|, |cdocsch2.tex|, |cdocspt3.tex|, |cdocspt4.tex|,
|cdocsdrf.tex|, |cdocsfn1.tex|, |cdocsfn2.tex|
as well as |childdoc.pdf|.

%%%%%%%%%%%%%%%%%%%%%%%%%%%%%%%%%%%%%%%%%%%%%%%%%%%%%%%%%%%%%%%%%%%%%%%%%%%%%%%%
\subsection{Files and Installation}

The package consists of the files:
%
\begin{center}
\begin{tabular}{ll}
    |README.txt|   & readme file \\
    |childdoc.ins| & installation file \\
    |childdoc.dtx| & source file \\
    |childdoc.def| & definition file \\
    |cdocsamp.tex| & sample main file \\
    |cdocsch1.tex| & sample include file \\
    |cdocsch2.tex| & sample include file \\
    |cdocspt3.tex| & sample part file \\
    |cdocspt4.tex| & sample part file \\
    |cdocsdrf.tex| & sample redirection file \\
    |cdocsfn1.tex| & sample redirection file \\
    |cdocsfn2.tex| & sample redirection file \\
    |childdoc.pdf| & manual
\end{tabular}
\end{center}
%
The distribution consists of the files
|README.txt|, |childdoc.ins| and |childdoc.dtx|.
%
\begin{itemize}
\item
Run (pdf)\LaTeX{} on |childdoc.dtx|
to compile the manual |childdoc.pdf| (this file).
\item
Run \LaTeX{} on |childdoc.ins| to create the definitions file |childdoc.def|
and the sample |cdocsamp.tex| with include files
|cdocsch1.tex|, |cdocsch2.tex|, |cdocspt3.tex|, |cdocspt4.tex|,
|cdocsdrf.tex|, |cdocsfn1.tex|, |cdocsfn2.tex|.
Then copy the file |childdoc.def| to an appropriate directory of your \LaTeX{}
distribution, e.g.\ \textit{texmf-root}|/tex/latex/childdoc|.
\end{itemize}

%%%%%%%%%%%%%%%%%%%%%%%%%%%%%%%%%%%%%%%%%%%%%%%%%%%%%%%%%%%%%%%%%%%%%%%%%%%%%%%%
\subsection{Related CTAN Packages}

There are several other packages which offer a similar functionality:
%
\begin{itemize}
\item
The packages
\href{http://ctan.org/pkg/docmute}{\textsf{docmute}},
\href{http://ctan.org/pkg/includex}{\textsf{includex}} and
\href{http://ctan.org/pkg/standalone}{\textsf{standalone}}
provide commands to include only the document body of
a child file thus allowing both files to be compiled individually.
\item
The packages \href{http://ctan.org/pkg/subdocs}{\textsf{subdocs}}
and \href{http://ctan.org/pkg/subfiles}{\textsf{subfiles}}
provide structures in which the main and child documents can be
encapsulated and allowing them to be compiled individually.
The inclusion mechanism is different from the conventional |\include|.
\item
The package \href{http://ctan.org/pkg/combine}{\textsf{combine}}
is an elaborate solution to combine several documents into one.
\end{itemize}
%
See also the CTAN topic \href{http://ctan.org/topic/subdocs}{\textsf{subdocs}}
for further related packages.
The present package differs from the above solutions in that
a document structure constructed with the conventional |\include| mechanism
just needs two extra commands at the top of every file
such that all constituent files can be compiled individually.

%%%%%%%%%%%%%%%%%%%%%%%%%%%%%%%%%%%%%%%%%%%%%%%%%%%%%%%%%%%%%%%%%%%%%%%%%%%%%%%%
%\subsection{Feature Suggestions}
%
%The following is a list of features which may be useful for future
%versions of this package:
%%
%\begin{itemize}
%\item
%\ldots
%\end{itemize}

%%%%%%%%%%%%%%%%%%%%%%%%%%%%%%%%%%%%%%%%%%%%%%%%%%%%%%%%%%%%%%%%%%%%%%%%%%%%%%%%
\subsection{Revision History}

%%%%%%%%%%%%%%%%%%%%%%%%%%%%%%%%%%%%%%%%
\paragraph{v2.0:} 2018/12/30

\begin{itemize}
\item
immediate forward processing
\item
added |\childdocby| mechanism
\item
manual restructured
\end{itemize}

%%%%%%%%%%%%%%%%%%%%%%%%%%%%%%%%%%%%%%%%
\paragraph{v1.6:} 2018/01/17

\begin{itemize}
\item
application for development of include files
\item
corrections to manual
\end{itemize}

%%%%%%%%%%%%%%%%%%%%%%%%%%%%%%%%%%%%%%%%
\paragraph{v1.5:} 2017/05/21

\begin{itemize}
\item
more complete structuring introduced
\item
|\childdocof| introduced
\item
|\childdoc| renamed to |\childdocmain|
\item
|\childredirect| renamed to |\childdocforward| and |\childdocforwardprefix|
and functionality expanded
\end{itemize}

%%%%%%%%%%%%%%%%%%%%%%%%%%%%%%%%%%%%%%%%
\paragraph{v1.0:} 2017/04/27

\begin{itemize}
\item
manual and install package
\item
first version published on CTAN
\end{itemize}

%%%%%%%%%%%%%%%%%%%%%%%%%%%%%%%%%%%%%%%%
\paragraph{v0.6:} 2017/04/26

\begin{itemize}
\item
redirection mechanism added
\end{itemize}

%%%%%%%%%%%%%%%%%%%%%%%%%%%%%%%%%%%%%%%%
\paragraph{v0.5:} 2017/04/26

\begin{itemize}
\item
functionality in definition file
\end{itemize}


%%%%%%%%%%%%%%%%%%%%%%%%%%%%%%%%%%%%%%%%%%%%%%%%%%%%%%%%%%%%%%%%%%%%%%%%%%%%%%%%
%%%%%%%%%%%%%%%%%%%%%%%%%%%%%%%%%%%%%%%%%%%%%%%%%%%%%%%%%%%%%%%%%%%%%%%%%%%%%%%%
%%%%%%%%%%%%%%%%%%%%%%%%%%%%%%%%%%%%%%%%%%%%%%%%%%%%%%%%%%%%%%%%%%%%%%%%%%%%%%%%
\appendix

\settowidth\MacroIndent{\rmfamily\scriptsize 000\ }

 \DocInput{childdoc.dtx}

\end{document}
%</driver>
% \fi
%
% %%%%%%%%%%%%%%%%%%%%%%%%%%%%%%%%%%%%%%%%%%%%%%%%%%%%%%%%%%%%%%%%%%%%%%%%%%%%%%
% %%%%%%%%%%%%%%%%%%%%%%%%%%%%%%%%%%%%%%%%%%%%%%%%%%%%%%%%%%%%%%%%%%%%%%%%%%%%%%
% \section{Sample}
%\iffalse
%<*samplemain>
%\fi
%
% The following presents a sample document
% with two chapters, two parts, a title page,
% a compile flag as well as three forwarding files to set the flag.
% It consists of eight |.tex| files:
% \begin{center}
% \begin{tabular}{ll}
% |cdocsamp.tex|&main file\\
% |cdocsch1.tex|&include file for chapter 1\\
% |cdocsch2.tex|&include file for chapter 2\\
% |cdocspt3.tex|&include file for part 3\\
% |cdocspt4.tex|&include file for part 4\\
% |cdocsdrf.tex|&forwarding file for main file in draft mode\\
% |cdocsfi1.tex|&forwarding file for final version of chapter 1\\
% |cdocsfi2.tex|&forwarding file for final version of chapter 2\\
% \end{tabular}
% \end{center}
% Each of the eight files can be compiled directly by the \LaTeX{} compiler.
%
% %%%%%%%%%%%%%%%%%%%%%%%%%%%%%%%%%%%%%%
% \paragraph{Main File.}
%
% The main file is called |cdocsamp.tex|.
%
% Load the \textsf{childdoc} definitions and
% declare the filename for the main document:
%    \begin{macrocode}
\input{childdoc.def}
\childdocmain{}
%    \end{macrocode}

% Optional override for |\version| flag:
%    \begin{macrocode}
%%\ifchilddoc\else\providecommand{\version}{draft}\fi
%    \end{macrocode}

% Define the default values for the |\version| flag
% (|final| for the main file and |draft| for childs):
%    \begin{macrocode}
\ifchilddoc
\providecommand{\version}{draft}
\else
\providecommand{\version}{final}
\fi
%    \end{macrocode}

% Load the standard document class:
%    \begin{macrocode}
\documentclass[12pt]{article}
%    \end{macrocode}

% Start the document body:
%    \begin{macrocode}
\begin{document}
%    \end{macrocode}

% Declare a title page.
% Print title, part of document being processed and version flag:
%    \begin{macrocode}
\addtocounter{page}{-1}
\begin{center}
{\LARGE\bfseries{}childdoc example\par}
\vspace{1cm}
\ifchilddoc
\ifchilddocmanual part\else chapter\fi:
`\childdocname' of `\childdocjob'\par
\else
main document: `\childdocjob'\par
\fi
version: \version\par
\end{center}
\newpage
%    \end{macrocode}

% Manually include selected file,
% otherwise process as usual:
%    \begin{macrocode}
\ifchilddocmanual
\section*{part `\childdocname'}
\input{\childdocname}
\else
%    \end{macrocode}

% Include the two chapters:
%    \begin{macrocode}
\include{cdocsch1}
\include{cdocsch2}
%    \end{macrocode}

% Include the two parts unless only chapters should be displayed:
%    \begin{macrocode}
\ifchilddoc\else
\section{part three}
\input{cdocspt3}
\section{part four}
\input{cdocspt4}
\fi
%    \end{macrocode}

% Process as usual until here:
%    \begin{macrocode}
\fi
%    \end{macrocode}

% End of document body:
%    \begin{macrocode}
\end{document}
%    \end{macrocode}
%\iffalse
%</samplemain>
%\fi
%
% %%%%%%%%%%%%%%%%%%%%%%%%%%%%%%%%%%%%%%
% \paragraph{Chapter Include Files.}
%
% The include files are called |cdocsch1.tex| and |cdocsch2.tex|.
%
%\iffalse
%<*samplechap1|samplechap2>
%\fi

% Optional override for |\version| flag:
%    \begin{macrocode}
%%\providecommand{\version}{final}
%    \end{macrocode}

% Include the main document:
%    \begin{macrocode}
\input{childdoc.def}
\childdocof{cdocsamp}
%    \end{macrocode}

%\iffalse
%</samplechap1|samplechap2>
%\fi
%
%\iffalse
%<*samplechap1>
%\fi
% Some text for chapter 1:
%    \begin{macrocode}
\section{one}
some text in chapter one
%    \end{macrocode}

%\iffalse
%</samplechap1>
%\fi
% Some text for chapter 2:
%\iffalse
%<*samplechap2>
%\fi
%    \begin{macrocode}
\section{two}
more text in chapter two
%    \end{macrocode}

%\iffalse
%</samplechap2>
%\fi
%
% %%%%%%%%%%%%%%%%%%%%%%%%%%%%%%%%%%%%%%
% \paragraph{Part Include Files.}
%
% The include files are called |cdocspt3.tex| and |cdocspt4.tex|.
%
%\iffalse
%<*samplepart3|samplepart4>
%\fi

% Optional override for |\version| flag:
%    \begin{macrocode}
%%\providecommand{\version}{final}
%    \end{macrocode}

% Include the main document:
%    \begin{macrocode}
\input{childdoc.def}
\childdocby{cdocsamp}
%    \end{macrocode}

%\iffalse
%</samplepart3|samplepart4>
%\fi
%
%\iffalse
%<*samplepart3>
%\fi
% Some text for part 3:
%    \begin{macrocode}
some text in part three
%    \end{macrocode}

%\iffalse
%</samplepart3>
%\fi
% Some text for part 4:
%\iffalse
%<*samplepart4>
%\fi
%    \begin{macrocode}
more text in part four
%    \end{macrocode}

%\iffalse
%</samplepart4>
%\fi
%
% %%%%%%%%%%%%%%%%%%%%%%%%%%%%%%%%%%%%%%
% \paragraph{Forwarding for a Complete Draft.}
%
% The following forwarding file |cdocsdrf.tex|
% compiles the main document in draft mode:
%\iffalse
%<*sampledraft>
%\fi
%    \begin{macrocode}
\def\version{draft}
\input{childdoc.def}
\childdocforward{cdocsamp}
%    \end{macrocode}

%\iffalse
%</sampledraft>
%\fi
%
% %%%%%%%%%%%%%%%%%%%%%%%%%%%%%%%%%%%%%%
% \paragraph{Forwarding for Final Version of the Chapters.}
%
% The following forwarding files |cdocsfn1.tex| and |cdocsfn2.tex|
% (with identical content)
% compile the final versions of the child documents
% |cdocsch1.tex| and |cdocsch2.tex|, respectively:
%\iffalse
%<*samplefinal>
%\fi
%    \begin{macrocode}
\def\version{final}
\input{childdoc.def}
\childdocforwardprefix[cdocsamp]{cdocsfn}{cdocsch}
%    \end{macrocode}

%\iffalse
%</samplefinal>
%\fi
%
% %%%%%%%%%%%%%%%%%%%%%%%%%%%%%%%%%%%%%%
% \paragraph{Command Line Processing.}
%
% The following three command lines generate the output files
% |cdocscld|, |cdocscl1| and |cdocscl2|
% which should be identical to
% |cdocsdrf|, |cdocsch1| and |cdocsfn2|, respectively:
% \begin{center}
% \begin{tabular}{l}
% |latex -jobname cdocscld \|\\
% |  "\def\version{draft}\input{childdoc.def}\childdocforward{cdocsamp}"|\\
% |latex -jobname cdocscl1 \|\\
% |  "\input{childdoc.def}\childdocforward[cdocsamp]{cdocsch1}"|\\
% |latex -jobname cdocscl2 \|\\
% |  "\def\version{final}\input{childdoc.def}\childdocforward{cdocsch2}"|
% \end{tabular}
% \end{center}
% Note that the trailing backslash on each first line
% merely continues the input to the second line
% (for convenient cut ant paste).
% Furthermore, the command |latex| can be replaced by any
% of its alternative versions such as |pdflatex|.
%
% %%%%%%%%%%%%%%%%%%%%%%%%%%%%%%%%%%%%%%%%%%%%%%%%%%%%%%%%%%%%%%%%%%%%%%%%%%%%%%
% %%%%%%%%%%%%%%%%%%%%%%%%%%%%%%%%%%%%%%%%%%%%%%%%%%%%%%%%%%%%%%%%%%%%%%%%%%%%%%
% \section{Implementation}
%\iffalse
%<*package>
%\fi
%
% This section describes the definitions file |childdoc.def|.

% The definitions cannot be loaded using |\usepackage| or |\RequirePackage|
% which has a mechanism to prevent loading a style file more than once.
% When loading the definitions by means of |\input|
% multiple instances have to be prevented manually:
%\iffalse
%This code needs to be before the `\ProvidesFile' directive
%which is defined at the beginning of this file.
%Therefore it is also placed there and commented out here.
%</package>
%<*discard>
%\fi
%    \begin{macrocode}
\ifdefined\childdocmain\endinput\fi
%    \end{macrocode}
%\iffalse
%</discard>
%<*package>
%\fi
%
% \macro{\ifchilddoc}
% \macro{\ifchilddocmanual}
% The conditional |\ifchilddoc| tells whether a
% child (true) or main (false) document is being compiled.
% The conditional |\ifchilddocmanual| tells whether
% the |\includeonly| mechanism is used (false) or
% the selection of child files must be performed manually (true).
% The definitions initialise to false:
%    \begin{macrocode}
\newif\ifchilddoc
\newif\ifchilddocmanual
%    \end{macrocode}

% \macro{\childdocname}
% \macro{\childdocjob}
% The macro |\childdocname| stores the name of the main document
% to be compiled. The macro |\childdocjob| stores the name of
% the document on which the \LaTeX{} compiler was originally invoked.
% The content of |\jobname| cannot be compared
% to filenames specified in the source due to different catcodes.
% The following code rescans |\jobname|, stores the result
% in |\childdocname| and saves a copy in |\childdocjob|:
%    \begin{macrocode}
\edef\childdocname{\scantokens\expandafter{\jobname\noexpand}}
\let\childdocjob\childdocname
%    \end{macrocode}

% \macro{\childdocdisable}
% The macro |\childdocdisable| prevents the main file
% from being processed more than once.
% At this stage, the main document command |\childdocmain|
% is assumed to be called once again where it should do nothing.
% Any subsequent call to it should prevent
% a secondary processing of the main document
% It overwrites the forwarding commands
% |\childdocof| and |\childdocforward|
% with empty macros to prevent further inclusions of the main document:
%    \begin{macrocode}
\newcommand{\childdocdisable}
{
  \renewcommand{\childdocmain}[1]{\renewcommand{\childdocmain}[1]{\endinput}}
  \renewcommand{\childdocof}[1]{}
  \renewcommand{\childdocby}[2][]{}
  \renewcommand{\childdocforward}[2][]{}
  \renewcommand{\childdocdisable}{}
}
%    \end{macrocode}

% \macro{\childdocmain}
% The macro |\childdocmain| is to be called at the top of the main file
% with nothing or the main filename (without extension) as argument.
% First, it breaks loops.
% If the argument is not empty and does not match |\childdocname|
% (which is set by the first inclusion of |childdoc.def|),
% |\ifchilddoc| is set to true, |\includeonly| is applied to the child file
% and |\jobname| is set to the main file
% (for proper handling of |.aux| files):
%    \begin{macrocode}
\newcommand{\childdocmain}[1]
{
  \childdocdisable\childdocmain{}
  \if?#1?\else
    \begingroup
      \def\childdoctmp{#1}
      \ifx\childdoctmp\childdocname
        \def\childdoctmp{}
      \else
        \def\childdoctmp
        {
          \childdoctrue
          \includeonly{\childdocname}
          \def\childdocjob{#1}
          \def\jobname{#1}
        }
      \fi
      \expandafter
    \endgroup
    \childdoctmp
  \fi
}
%    \end{macrocode}

% \macro{\childdocof}
% The command |\childdocof| redirects
% compilation to the main file |#1|.
%    \begin{macrocode}
\newcommand{\childdocof}[1]
{
  \childdocdisable
  \childdoctrue
  \includeonly{\childdocname}
  \def\jobname{#1}
  \def\childdocjob{#1}
  \input{#1}
}
%    \end{macrocode}

% \macro{\childdocby}
% The command |\childdocby| ....
%    \begin{macrocode}
\newcommand{\childdocby}[2][]
{
  \childdocdisable
  \childdoctrue
  \childdocmanualtrue
  \if?#1?\else
    \def\jobname{#2}
  \fi
  \def\childdocjob{#2}
  \input{#2}
  \endinput
}
%    \end{macrocode}

% \macro{\childdocforward}
% The command |\childdocforward| redirects
% compilation to the main file or
% (if the optional argument is given) a child file.
% Parameters are set as if the main file
% or a child file starting with |\childdocof| was compiled.
% Then compilation is handed over to the main file:
%    \begin{macrocode}
\newcommand{\childdocforward}[2][]
{
  \begingroup
    \if?#1?
      \def\childdoctmp
      {
        \def\childdocname{#2}
        \def\childdocjob{#2}
        \def\jobname{#2}
        \input{#2}
        \endinput
      }
    \else
      \def\childdoctmp
      {
        \childdocdisable
        \def\childdocname{#2}
        \childdoctrue
        \includeonly{#2}
        \def\childdocjob{#1}
        \def\jobname{#1}
        \input{#1}
        \endinput
      }
    \fi
    \expandafter
  \endgroup
  \childdoctmp
}
%    \end{macrocode}

% \macro{\childdocforwardprefix}
% The command |\childdocforwardprefix| redirects
% compilation to the main or a child file by means of a pattern.
% The prefix |#1| in the current filename is replaced by |#2|
% and the suffix of the current filename is kept
% (it is assumed that the filename does not contain the substring `|~~~|'
% which is used as a delimiter).
% Compilation is handed over to the new file by |\childdocforward|:
%    \begin{macrocode}
\newcommand{\childdocforwardprefix}[3][]
{
  \begingroup
    \def\childdocextract #2##1~~~{\def\childdoctmp{\childdocforward[#1]{#3##1}}}
    \expandafter\childdocextract\childdocname~~~
    \expandafter
  \endgroup
  \childdoctmp
}
%    \end{macrocode}

% \macro{\childdoc}
% The deprecated macro |\childdoc| is a legacy version of |\childdocmain|:
%    \begin{macrocode}
\newcommand{\childdoc}{\childdocmain}
%    \end{macrocode}

% \macro{\childdocredirect}
% The deprecated macro |\childdocredirect| is a legacy version
% of |\childdocforward| and |\childdocforwardprefix|:
%    \begin{macrocode}
\newcommand{\childdocredirect}[2][]
{
  \begingroup
    \if?#1?
      \def\childdoctmp{\childdocforward{#2}}
    \else
      \def\childdoctmp{\childdocforwardprefix{#1}{#2}}
    \fi
    \expandafter
  \endgroup
  \childdoctmp
}
%    \end{macrocode}

%\iffalse
%</package>
%\fi
%
\endinput
|\\
|\childdocforward{|\textit{main}|}|
\end{tabular}
\end{center}
%
Likewise, the following files |final|\textit{nn}|.tex|
compile the final version of the child document
|child|\textit{nn}|.tex|:
%
\begin{center}
\begin{tabular}{l}
|\def\version{final}|\\
|% \iffalse
%
% childdoc.dtx Copyright (C) 2017-2018 Niklas Beisert
%
% This work may be distributed and/or modified under the
% conditions of the LaTeX Project Public License, either version 1.3
% of this license or (at your option) any later version.
% The latest version of this license is in
%   http://www.latex-project.org/lppl.txt
% and version 1.3 or later is part of all distributions of LaTeX
% version 2005/12/01 or later.
%
% This work has the LPPL maintenance status `maintained'.
%
% The Current Maintainer of this work is Niklas Beisert.
%
% This work consists of the files childdoc.dtx and childdoc.ins
% and the derived files childdoc.def and cdocsamp.tex with
% cdocsch1.tex, cdocsch2.tex, cdocsdrf.tex, cdocsfn1.tex, cdocsfn2.tex.
%
%<package>\ifdefined\childdocmain\endinput\fi
%<package>\ProvidesFile{childdoc.def}[2018/12/30 v2.0 child document driver]
%<samplemain>\ProvidesFile{cdocsamp.tex}[2018/12/30 v2.0 sample for childdoc]
%<*driver>
%\ProvidesFile{childdoc.drv}[2018/12/30 v2.0 childdoc reference manual file]
\PassOptionsToClass{10pt,a4paper}{article}
\documentclass{ltxdoc}

\usepackage[margin=35mm]{geometry}
\usepackage{hyperref}
\usepackage{hyperxmp}
\usepackage[usenames]{color}

\hypersetup{colorlinks=true}
\hypersetup{pdfstartview=FitH}
\hypersetup{pdfpagemode=UseNone}
\hypersetup{pdfsource={}}
\hypersetup{pdflang={en-UK}}
\hypersetup{pdfcopyright={Copyright 2017-2018 Niklas Beisert.
  This work may be distributed and/or modified under the
  conditions of the LaTeX Project Public License, either version 1.3
  of this license or (at your option) any later version.}}
\hypersetup{pdflicenseurl={http://www.latex-project.org/lppl.txt}}
\hypersetup{pdfcontactaddress={ETH Zurich, ITP, HIT K,
  Wolfgang-Pauli-Strasse 27}}
\hypersetup{pdfcontactpostcode={8093}}
\hypersetup{pdfcontactcity={Zurich}}
\hypersetup{pdfcontactcountry={Switzerland}}
\hypersetup{pdfcontactemail={nbeisert@itp.phys.ethz.ch}}
\hypersetup{pdfcontacturl={http://people.phys.ethz.ch/\xmptilde nbeisert/}}

\newcommand{\secref}[1]{\hyperref[#1]{section \ref*{#1}}}

\parskip1ex
\parindent0pt
\let\olditemize\itemize
\def\itemize{\olditemize\parskip0pt}

\begin{document}

\title{The \textsf{childdoc} Package}
\hypersetup{pdftitle={The childdoc Package}}
\author{Niklas Beisert\\[2ex]
  Institut f\"ur Theoretische Physik\\
  Eidgen\"ossische Technische Hochschule Z\"urich\\
  Wolfgang-Pauli-Strasse 27, 8093 Z\"urich, Switzerland\\[1ex]
  \href{mailto:nbeisert@itp.phys.ethz.ch}
  {\texttt{nbeisert@itp.phys.ethz.ch}}}
\hypersetup{pdfauthor={Niklas Beisert}}
\hypersetup{pdfsubject={Manual for the LaTeX2e Package childdoc}}
\date{30 December 2018, \textsf{v2.0}}
\maketitle

\begin{abstract}\noindent
\textsf{childdoc} is a \LaTeXe{} package
that enables the direct compilation
of document sections included by |\include|
to individual files.
\end{abstract}

\begingroup
\parskip0ex
\tableofcontents
\endgroup

%%%%%%%%%%%%%%%%%%%%%%%%%%%%%%%%%%%%%%%%%%%%%%%%%%%%%%%%%%%%%%%%%%%%%%%%%%%%%%%%
%%%%%%%%%%%%%%%%%%%%%%%%%%%%%%%%%%%%%%%%%%%%%%%%%%%%%%%%%%%%%%%%%%%%%%%%%%%%%%%%
\section{Introduction}

\LaTeX{} provides a mechanism to structure a large document (such as a book)
into a main file and several child files (containing the chapters)
using the |\include| command.
This mechanism is beneficial for documents
which span hundreds of pages in order to
make the source file(s) more manageable.
Moreover, compilation can be restricted to
selected child files by means of the |\includeonly| command.
The latter feature can be used to reduce the compilation time while editing
(this was significantly more useful in the earlier days of \LaTeX{})
or to generate a smaller document which is easier to navigate.
Another application of |\includeonly| is to generate
documents consisting of selected parts of the complete document.

However, there are a few drawbacks of the plain |\include| mechanism:
\begin{itemize}
\item
The child files cannot be compiled on their own,
they can only be compiled via the main file.
A naive editing environment
(such as a text editor with an option
to have the current file processed by \LaTeX)
may require one to switch to the main file before compiling;
attempting to compile the child file produces errors.
\item
The main file must be modified (each time)
to adjust the |\includeonly| command
to the present needs. This easily leaves the main file in a messy state.
\item
The generated document will always carry the filename
of the main document. This is inconvenient if
several child files are to be compiled and
to be kept for distribution.
\end{itemize}

The present package provides a simple interface
to make child files individually compilable by \LaTeX{}.
Compiling a child file then has the same effect as compiling
the main file with an |\includeonly| command
to select the appropriate child.
Moreover the generated document will carry the name of the child
rather than the main file.
This resolves all three above issues.

This feature is meant to make the editing of books,
thesis documents and lecture notes somewhat more convenient.
However, the package can also be used efficiently for
composing a series of documents (such as exercise sheets)
which are typically distributed individually.
It then assists the author in generating the individual documents
(potentially in different versions)
as well as a document containing the collected series.
Another application is in developing style files
or other kinds of included material
where compilation of the style file could redirect
to a sample or test file.

%%%%%%%%%%%%%%%%%%%%%%%%%%%%%%%%%%%%%%%%%%%%%%%%%%%%%%%%%%%%%%%%%%%%%%%%%%%%%%%%
%%%%%%%%%%%%%%%%%%%%%%%%%%%%%%%%%%%%%%%%%%%%%%%%%%%%%%%%%%%%%%%%%%%%%%%%%%%%%%%%
\section{Usage}

First of all, the package \textsf{childdoc} is \emph{not} a standard
\LaTeXe{} |.sty| style file! Therefore it needs to be invoked in
a non-standard way.

%%%%%%%%%%%%%%%%%%%%%%%%%%%%%%%%%%%%%%%%%%%%%%%%%%%%%%%%%%%%%%%%%%%%%%%%%%%%%%%%
\subsection{Included Files}
\label{sec:include}

%%%%%%%%%%%%%%%%%%%%%%%%%%%%%%%%%%%%%%%%
\DescribeMacro{\childdocmain}
To use the package, add the commands
\begin{center}
\begin{tabular}{l}
|\input{childdoc.def}|\\
|\childdocmain{}|\\
\end{tabular}
\end{center}
at the very top of the main \LaTeX{} file,
in particular \emph{before} the |\documentclass| statement!
The argument of |\childdocmain| should be left empty
(but it must be present).

%%%%%%%%%%%%%%%%%%%%%%%%%%%%%%%%%%%%%%%%
\DescribeMacro{\childdocof}
Furthermore, add the commands
\begin{center}
\begin{tabular}{l}
|\input{childdoc.def}|\\
|\childdocof{|\textit{main}|}|\\
\end{tabular}
\end{center}
at the top of every child file \textit{child}
which is included by |\include{|\textit{child}|}|
from within the main file
(or at least for those files to be compiled individually).
The argument \textit{main} must be the filename of the main file.

There are a couple of
considerations in setting up the main and child documents:

%%%%%%%%%%%%%%%%%%%%%%%%%%%%%%%%%%%%%%%%
\paragraph{Restrictions.}

Please note the following restrictions:
\begin{itemize}
\item
|\childdocmain| must be called with one argument \textit{main}
to ensure compatibility with earlier version of the package.
It must either be empty (|\childdocmain{}|)
or precisely match the filename of the main file in which it is specified.
See \secref{sec:detection} for further information.
\item
The filename \textit{main} must be specified without the |.tex| extension.
\item
The filename \textit{main} is case sensitive
(even in case-insensitive file systems)
due to internal string comparison.
\item
The argument \textit{main} should be fully expanded, it cannot be a macro.
\item
Subdirectories and special characters should be avoided in filenames.
\item
The command |\childdocmain{|\textit{main}|}| must be followed by a whitespace.
It should not be followed immediately by another command
or by a comment mark `|%|'.
This is because the \TeX{} parser reads the token immediately following
the argument of |\childdocmain| and puts it
at the beginning of every child section;
however, a white\-space is ignored.
\end{itemize}

%%%%%%%%%%%%%%%%%%%%%%%%%%%%%%%%%%%%%%%%
\paragraph{Content of Main File.}

It is advisable to place all content in the child files included by |\include|.
Any output contained in the main file will appear in all child documents
unless suppressed manually;
it cannot be suppressed automatically by the |\includeonly| directive
and thus should normally be avoided.
A method to include some content in the main file
by means of conditional processing is described in \secref{sec:conditional}.

%%%%%%%%%%%%%%%%%%%%%%%%%%%%%%%%%%%%%%%%
\paragraph{Page Numbering.}

When only a part of the document is compiled,
the appropriate numbering of pages
(as well as other status parameters)
is determined from the |.aux| files.
The latter contain information from previous passes.
However this information needs to propagate through
all intermediate child documents.
Therefore the page numbering in child documents may well
be inconsistent until the complete document is compiled at least once.

A useful (if unconventional) way to always ensure a consistent
page numbering is to restart the numbering in each child document
and denote the pages by `\textit{child}|.|\textit{page}'
where \textit{child} represents the chapter/section number of the child file.
This can be achieved by the command
|\numberwithin{page}{|\textit{child}|}|
of the \textsf{amsmath} package
where \textit{child} can be |chapter| or |section|
depending on the chosen structuring.
Alternatively, one can modify the macro |\thepage| appropriately
and reset the counter |page| at the start of each child file.

%%%%%%%%%%%%%%%%%%%%%%%%%%%%%%%%%%%%%%%%%%%%%%%%%%%%%%%%%%%%%%%%%%%%%%%%%%%%%%%%
\subsection{Conditional Processing}
\label{sec:conditional}

The package provides a mechanism to compile different versions
of a document. To customise the versions further some conditional processing
can come in handy to distinguish which version is being compiled.
The package provides two macros to describe the compilation context:

%%%%%%%%%%%%%%%%%%%%%%%%%%%%%%%%%%%%%%%%
\DescribeMacro{\ifchilddoc}
The conditional |\ifchilddoc| distinguishes between the compilation of
child documents and the main document:
%
\begin{center}
|\ifchilddoc |\textit{child-code}| |[|\||else |\textit{main-code}]| \||fi|
\end{center}

%%%%%%%%%%%%%%%%%%%%%%%%%%%%%%%%%%%%%%%%
\DescribeMacro{\childdocname}
\DescribeMacro{\childdocjob}
The macro |\childdocname| contains the filename (without extension)
of the main or child file being processed.
Note that |\childdocjob| will always contain the name of the main file.

%%%%%%%%%%%%%%%%%%%%%%%%%%%%%%%%%%%%%%%%
\paragraph{Title Page.}

Conditional processing can be used to include a title or banner page
in the main document when proper precautions are taken.
Importantly, the code in the main file should ensure that the page counter
(as well as other status parameters which are stored in the |.aux| files)
takes the same value after the conditional processing.
Otherwise the page numbers may take divergent values
depending on which part is compiled.

For example, a title page could be declared by:
%
\begin{center}
\begin{tabular}{l}
|\ifchilddoc\||else|\\
|\addtocounter{page}{-1}|\\
\textit{code for title page}\\
|\newpage|\\
|\||fi|
\end{tabular}
\end{center}
%
A banner page for the child documents can be generated by:
%
\begin{center}
\begin{tabular}{l}
|\ifchilddoc|\\
|\addtocounter{page}{-1}|\\
\textit{code for banner page}\\
|\newpage|\\
|\||fi|
\end{tabular}
\end{center}
%
Here one could write a message such as:
\begin{center}
|This is the part \childdocname{} of \childdocjob{}.|
\end{center}

%%%%%%%%%%%%%%%%%%%%%%%%%%%%%%%%%%%%%%%%%%%%%%%%%%%%%%%%%%%%%%%%%%%%%%%%%%%%%%%%
\subsection{Flags}
\label{sec:flags}

The package makes it easy to generate different versions
of the main or child documents.
To this end compilation flags can be defined
and assigned different default values.
They will be particularly useful in conjunction
with the forwarding mechanism described in \secref{sec:forward}.

For example, it may be useful to have a flag |\version|
which can be set to |draft| or |final|.
The document source will contain some conditional code
depending on the value of |\version|.
Suppose further, the flag should default to |final| for the main file
and to |draft| for child files
which is a natural assignment for editing the document.
This is achieved by placing the following code
in the preamble of the main document
(below the |\childdocmain| directive):
%
\begin{center}
\begin{tabular}{l}
|\ifchilddoc|\\
|\providecommand{\version}{draft}|\\
|\||else|\\
|\providecommand{\version}{final}|\\
|\||fi|
\end{tabular}
\end{center}
%
The definition by |\providecommand| makes sure
that previous definitions are not overwritten.
Further statements |\providecommand{\version}{...}|
can thus be added before the above code to override it.

For the main file, one might add a line
(between |\childdocmain| and the above block)
%
\begin{center}
|%\ifchilddoc\||else\providecommand{\version}{draft}\||fi|
\end{center}
%
which can be uncommented to produce a draft version.
Likewise one can add a line to the very top of a child file
(above the |\childdocof{|\textit{main}|}| directive)
%
\begin{center}
|%\providecommand{\version}{final}|
\end{center}
%
which can be uncommented to produce the final version of this child document.

%%%%%%%%%%%%%%%%%%%%%%%%%%%%%%%%%%%%%%%%%%%%%%%%%%%%%%%%%%%%%%%%%%%%%%%%%%%%%%%%
\subsection{Forwarding}
\label{sec:forward}

Different versions of the main or child documents
using compilation flags as described in \secref{sec:flags}
can be (permanently) stored in different files
for convenient compilation, viewing and distribution.
To this end, the package defines a command
to pass on compilation to a different file:

%%%%%%%%%%%%%%%%%%%%%%%%%%%%%%%%%%%%%%%%
\DescribeMacro{\childdocforward}
The command |\childdocforward| redirects processing to
another source file:
%
\begin{center}
\begin{tabular}{l}
|\input{childdoc.def}|\\
|\childdocforward[|\textit{main}|]{|\textit{dest}|}|\\
\end{tabular}
\end{center}
%
The argument \textit{dest} is the destination file
(without extension).
It should be the main file or one of the child files.
Note that further \textsf{childdoc} directives
such as |\childdocof| and |\childdocforward|
in the indicated file will be processed in this form.
The optional argument \textit{main}
passes on directly to the main file \textit{main}
while pretending to compile the child \textit{dest}.
This form behaves as if \textit{dest}
issues |\childdocof{|\textit{main}|}| right away,
and no further \textsf{childdoc} directives will be processed.

%%%%%%%%%%%%%%%%%%%%%%%%%%%%%%%%%%%%%%%%
\DescribeMacro{\...prefix}
In the alternative form |\childdocforwardprefix|,
%
\begin{center}
\begin{tabular}{l}
|\input{childdoc.def}|\\
|\childdocforwardprefix[|\textit{main}|]{|\textit{prefix}|}{|\textit{dest}|}|
\end{tabular}
\end{center}
%
the destination file is determined by a pattern
depending on the current file:
To make this work, the current file must be called
`{\textit{prefix}\hspace{0.2em}\textit{suffix}}'
with \textit{prefix} matching precisely the argument.
Processing is then passed on to the file
`{\textit{dest}\hspace{0.2em}\textit{suffix}}'.
Surely, the same effect is achieved by
directly specifying the
argument `{\textit{dest}\hspace{0.2em}\textit{suffix}}'
in the first form.
However, that requires to set up a different file
for each child. With the alternative form of the command
all these files can have exactly the same content
which simplifies setting them up and maintaining them.

For example, the following file |draft.tex|
with a compilation flag |\version| as described in \secref{sec:flags}
compiles the main document as a draft:
%
\begin{center}
\begin{tabular}{l}
|\def\version{draft}|\\
|\input{childdoc.def}|\\
|\childdocforward{|\textit{main}|}|
\end{tabular}
\end{center}
%
Likewise, the following files |final|\textit{nn}|.tex|
compile the final version of the child document
|child|\textit{nn}|.tex|:
%
\begin{center}
\begin{tabular}{l}
|\def\version{final}|\\
|\input{childdoc.def}|\\
|\childdocforwardprefix{final}{child}|
\end{tabular}
\end{center}
%

Note that when several versions of a main file and/or of each child file
are to be generated, it may be convenient to set up a |Makefile| or
shell script to automatise the process.

%%%%%%%%%%%%%%%%%%%%%%%%%%%%%%%%%%%%%%%%%%%%%%%%%%%%%%%%%%%%%%%%%%%%%%%%%%%%%%%%
\subsection{Command Line Processing}
\label{sec:commandline}

The effect of redirection files can also be achieved by invoking
the \LaTeX{} compiler with a more elaborate command line.
Most conveniently this should be done as part
of a shell script or a |Makefile|.

When using \textsf{childdoc} in the main file, the following
command lines effectively perform a redirection
(note that depending on the shell being used,
backslashes may have to be doubled: `|\|' $\to$ `|\\|'):
%
\begin{center}
|... -jobname "|\textit{target}|" |\\|"|[\textit{flags}]%
|\input{childdoc.def}\childdocforward[|\textit{main}|]{|\textit{dest}|}"|
\end{center}
%
Here \textit{target} is the name of the output file,
\textit{main} is the name of the main file
and \textit{dest} is the name of the main or child file to be processed
(all filenames without extensions).
The optional argument \textit{main} can be omitted
if \textit{main} matches \textit{dest}.
Optionally, compilation \textit{flags} can be defined via |\def| commands.
This command line makes the \TeX{} engine believe
it is compiling the file \textit{target}
whose content is specified as the latter parameter.
The provided code then forwards the processing to
\textit{main} or \textit{dest} as described in \secref{sec:forward}.

%%%%%%%%%%%%%%%%%%%%%%%%%%%%%%%%%%%%%%%%%%%%%%%%%%%%%%%%%%%%%%%%%%%%%%%%%%%%%%%%
\subsection{Include by Input}
\label{sec:input}

Including child documents by |\include| has some restrictions by design.
Most notably, the content of a child document always occupies
its own set of pages; pages cannot be shared between child documents.
Usually, this behaviour makes perfect sense
because each child document contain an essential part of the document.
However, in some situations it may be desirable to compose
a document from a collection of parts
without having mandatory page breaks between then.
For this case, the package
provides a mechanism to include parts
by |\input| which can also be processed individually.
However, by construction this mechanism
requires manual handling of the content to be output.

%%%%%%%%%%%%%%%%%%%%%%%%%%%%%%%%%%%%%%%%
\DescribeMacro{\ifchilddocmanual}
The main file should be prepared as usual, see \secref{sec:include}.
However, the document body must make a distinction
between processing of an individual part and of the main document, e.g.:
%
\begin{center}
\begin{tabular}{l}
|\ifchilddocmanual|\\
|\input{\childdocname}|\\
|\||else|\\
\textit{document body with }|\input{|\textit{part}|}|\\
|\||fi|
\end{tabular}
\end{center}
%
The conditional |\ifchilddocmanual| is true whenever
a part to be included by |\input| is being compiled,
and the name of the part is stored in |\childdocname|.

%%%%%%%%%%%%%%%%%%%%%%%%%%%%%%%%%%%%%%%%
\DescribeMacro{\childdocby}
Each part to be included by |\input| should start with:
%
\begin{center}
\begin{tabular}{l}
|\input{childdoc.def}|\\
|\childdocby{|\textit{main}|}|\\
\end{tabular}
\end{center}
%
The directive |\childdocby| is similar to |\childdocof|
described in \secref{sec:include},
but the subsequent selection of content must be done manually.
To that end, both |\ifchilddoc| and |\ifchilddocmanual|
will be true upon processing of a part,
and the name of the part is stored in |\childdocname|.
Note that |\jobname| will be set to the filename of the current part
so that each part receives an individual |.aux| file
that does not interfere with the |.aux| file(s) of the main document.
This behaviour can be altered by the alternative form
|\childdocby[*]{|\textit{main}|}| (with a non-empty optional argument)
which uses the |.aux| file of the main document
by setting |\jobname| to \textit{main}.

%%%%%%%%%%%%%%%%%%%%%%%%%%%%%%%%%%%%%%%%%%%%%%%%%%%%%%%%%%%%%%%%%%%%%%%%%%%%%%%%
\subsection{Driver Development}
\label{sec:driver}

The \textsf{childdoc} mechanism can also be use for the development
of definition files such as \LaTeX{} styles or classes.
This case differs from the above setup with multiple parts
included by |\include| in that no |\includeonly| should be invoked.
This can be achieved by starting the include file
(before |\ProvidesPackage|) with:
%
\begin{center}
\begin{tabular}{l}
|\input{childdoc.def}|\\
|\childdocforward{|\textit{main}|}|\\
\end{tabular}
\end{center}
%
or alternatively with:
%
\begin{center}
\begin{tabular}{l}
|\input{childdoc.def}|\\
|\childdocby{|\textit{main}|}|\\
\end{tabular}
\end{center}
%
Both forms have slightly different effects as described above.
The main file is prepared as usual, see \secref{sec:include}.

%%%%%%%%%%%%%%%%%%%%%%%%%%%%%%%%%%%%%%%%%%%%%%%%%%%%%%%%%%%%%%%%%%%%%%%%%%%%%%%%
\subsection{Legacy Detection}
\label{sec:detection}

The directive |\childdocmain| in the main file can detect
whether the complete document or merely a child is to be compiled
even without using the directive |\childdocof|.
This method is deprecated because it is less robust
and there is no compelling reason to use it;
it is merely provided for backward compatibility
and it may be removed in future versions.

If the detection mechanism is to be used,
it is mandatory to correctly specify
the filename of the main file as the argument of |\childdocmain|:
%
\begin{center}
\begin{tabular}{l}
|\input{childdoc.def}|\\
|\childdocmain{|\textit{main}|}|\\
\end{tabular}
\end{center}
%
If |\jobname| does not match the argument \textit{main} of |\childdocmain|,
it is assumed that |\jobname| points to the child file to be compiled.
When using |\childdocmain| with the main file specified as argument,
it suffices to start a child file
with just |\input{|\textit{main}|}|
without loading of the package and using |\childdocof|.
If instead all processing is done
with the appropriate \textsf{childdoc} directives,
the argument of \textit{main} of |\childdocmain| can be empty.

An alternative version of the command line processing described
in \secref{sec:commandline} using the detection mechanism reads:
%
\begin{center}
|... -jobname "|\textit{target}|" "|[\textit{flags}]%
[|\def\jobname{|\textit{dest}|}|]|\input{|\textit{main}|}"|
\end{center}

%%%%%%%%%%%%%%%%%%%%%%%%%%%%%%%%%%%%%%%%%%%%%%%%%%%%%%%%%%%%%%%%%%%%%%%%%%%%%%%%
\subsection{Manual Code}
\label{sec:manual}

In case one cannot be certain whether the definitions file |childdoc.def|
is installed on the target \TeX{} distribution
and one prefers not to ship it,
it is conceivable to paste a few relevant commands into the sources.

To that end, drop all statements |\input{childdoc.def}|
and perform the replacements as outlined below.
Instead of |\childdocmain{|\textit{main}|}| add the following code
to the top of the main file:
%
\begin{center}
\begin{tabular}{l}
|\||ifdefined\childdocname\endinput\||fi\newif\ifchilddoc|\\
|\edef\childdocname{\scantokens\expandafter{\jobname\noexpand}}|\\
|\def\childdocmain{|\textit{main}|}\||ifx\childdocmain\childdocname\||else|\\
|\childdoctrue\includeonly{\childdocname}\let\jobname\childdocmain\||fi|\\
\end{tabular}
\end{center}
%
Instead of |\childdocof{|\textit{main}|}| just include the main file
at the top of each child file:
%
\begin{center}
|\input{|\textit{main}|}|
\end{center}
%
A simple redirection |\childdocforward{|\textit{dest}|}| is achieved by:
%
\begin{center}
|\def\jobname{|\textit{dest}|}\input{\jobname}|
\end{center}
%
The redirection with prefix
|\childdocforwardprefix[|\textit{prefix}|]{|\textit{dest}|}|
is accomplished by:
%
\begin{center}
\begin{tabular}{l}
|{\edef\jobname{\scantokens\expandafter{\jobname\noexpand}}|\\
|\def\redirectjob |\textit{prefix}|#1~~~{\gdef\jobname{|\textit{dest}|#1}}|\\
|\expandafter\redirectjob\jobname~~~}\input{\jobname}|
\end{tabular}
\end{center}

In an alternative approach,
child documents can be compiled by a specific command line
without additional code or specific definitions:
%
\begin{center}
|... -jobname "|\textit{target}|" "|[\textit{flags}]%
|\includeonly{|\textit{dest}|}\input{|\textit{main}|}"|
\end{center}
%

%%%%%%%%%%%%%%%%%%%%%%%%%%%%%%%%%%%%%%%%%%%%%%%%%%%%%%%%%%%%%%%%%%%%%%%%%%%%%%%%
%%%%%%%%%%%%%%%%%%%%%%%%%%%%%%%%%%%%%%%%%%%%%%%%%%%%%%%%%%%%%%%%%%%%%%%%%%%%%%%%
\section{Information}

%%%%%%%%%%%%%%%%%%%%%%%%%%%%%%%%%%%%%%%%%%%%%%%%%%%%%%%%%%%%%%%%%%%%%%%%%%%%%%%%
\subsection{Copyright}

Copyright \copyright{} 2017--2018 Niklas Beisert

This work may be distributed and/or modified under the
conditions of the \LaTeX{} Project Public License, either version 1.3
of this license or (at your option) any later version.
The latest version of this license is in
  \url{http://www.latex-project.org/lppl.txt}
and version 1.3 or later is part of all distributions of \LaTeX{}
version 2005/12/01 or later.

This work has the LPPL maintenance status `maintained'.

The Current Maintainer of this work is Niklas Beisert.

This work consists of the files |README.txt|, |childdoc.ins| and |childdoc.dtx|
as well as the derived files |childdoc.def|, |cdocsamp.tex|
with |cdocsch1.tex|, |cdocsch2.tex|, |cdocspt3.tex|, |cdocspt4.tex|,
|cdocsdrf.tex|, |cdocsfn1.tex|, |cdocsfn2.tex|
as well as |childdoc.pdf|.

%%%%%%%%%%%%%%%%%%%%%%%%%%%%%%%%%%%%%%%%%%%%%%%%%%%%%%%%%%%%%%%%%%%%%%%%%%%%%%%%
\subsection{Files and Installation}

The package consists of the files:
%
\begin{center}
\begin{tabular}{ll}
    |README.txt|   & readme file \\
    |childdoc.ins| & installation file \\
    |childdoc.dtx| & source file \\
    |childdoc.def| & definition file \\
    |cdocsamp.tex| & sample main file \\
    |cdocsch1.tex| & sample include file \\
    |cdocsch2.tex| & sample include file \\
    |cdocspt3.tex| & sample part file \\
    |cdocspt4.tex| & sample part file \\
    |cdocsdrf.tex| & sample redirection file \\
    |cdocsfn1.tex| & sample redirection file \\
    |cdocsfn2.tex| & sample redirection file \\
    |childdoc.pdf| & manual
\end{tabular}
\end{center}
%
The distribution consists of the files
|README.txt|, |childdoc.ins| and |childdoc.dtx|.
%
\begin{itemize}
\item
Run (pdf)\LaTeX{} on |childdoc.dtx|
to compile the manual |childdoc.pdf| (this file).
\item
Run \LaTeX{} on |childdoc.ins| to create the definitions file |childdoc.def|
and the sample |cdocsamp.tex| with include files
|cdocsch1.tex|, |cdocsch2.tex|, |cdocspt3.tex|, |cdocspt4.tex|,
|cdocsdrf.tex|, |cdocsfn1.tex|, |cdocsfn2.tex|.
Then copy the file |childdoc.def| to an appropriate directory of your \LaTeX{}
distribution, e.g.\ \textit{texmf-root}|/tex/latex/childdoc|.
\end{itemize}

%%%%%%%%%%%%%%%%%%%%%%%%%%%%%%%%%%%%%%%%%%%%%%%%%%%%%%%%%%%%%%%%%%%%%%%%%%%%%%%%
\subsection{Related CTAN Packages}

There are several other packages which offer a similar functionality:
%
\begin{itemize}
\item
The packages
\href{http://ctan.org/pkg/docmute}{\textsf{docmute}},
\href{http://ctan.org/pkg/includex}{\textsf{includex}} and
\href{http://ctan.org/pkg/standalone}{\textsf{standalone}}
provide commands to include only the document body of
a child file thus allowing both files to be compiled individually.
\item
The packages \href{http://ctan.org/pkg/subdocs}{\textsf{subdocs}}
and \href{http://ctan.org/pkg/subfiles}{\textsf{subfiles}}
provide structures in which the main and child documents can be
encapsulated and allowing them to be compiled individually.
The inclusion mechanism is different from the conventional |\include|.
\item
The package \href{http://ctan.org/pkg/combine}{\textsf{combine}}
is an elaborate solution to combine several documents into one.
\end{itemize}
%
See also the CTAN topic \href{http://ctan.org/topic/subdocs}{\textsf{subdocs}}
for further related packages.
The present package differs from the above solutions in that
a document structure constructed with the conventional |\include| mechanism
just needs two extra commands at the top of every file
such that all constituent files can be compiled individually.

%%%%%%%%%%%%%%%%%%%%%%%%%%%%%%%%%%%%%%%%%%%%%%%%%%%%%%%%%%%%%%%%%%%%%%%%%%%%%%%%
%\subsection{Feature Suggestions}
%
%The following is a list of features which may be useful for future
%versions of this package:
%%
%\begin{itemize}
%\item
%\ldots
%\end{itemize}

%%%%%%%%%%%%%%%%%%%%%%%%%%%%%%%%%%%%%%%%%%%%%%%%%%%%%%%%%%%%%%%%%%%%%%%%%%%%%%%%
\subsection{Revision History}

%%%%%%%%%%%%%%%%%%%%%%%%%%%%%%%%%%%%%%%%
\paragraph{v2.0:} 2018/12/30

\begin{itemize}
\item
immediate forward processing
\item
added |\childdocby| mechanism
\item
manual restructured
\end{itemize}

%%%%%%%%%%%%%%%%%%%%%%%%%%%%%%%%%%%%%%%%
\paragraph{v1.6:} 2018/01/17

\begin{itemize}
\item
application for development of include files
\item
corrections to manual
\end{itemize}

%%%%%%%%%%%%%%%%%%%%%%%%%%%%%%%%%%%%%%%%
\paragraph{v1.5:} 2017/05/21

\begin{itemize}
\item
more complete structuring introduced
\item
|\childdocof| introduced
\item
|\childdoc| renamed to |\childdocmain|
\item
|\childredirect| renamed to |\childdocforward| and |\childdocforwardprefix|
and functionality expanded
\end{itemize}

%%%%%%%%%%%%%%%%%%%%%%%%%%%%%%%%%%%%%%%%
\paragraph{v1.0:} 2017/04/27

\begin{itemize}
\item
manual and install package
\item
first version published on CTAN
\end{itemize}

%%%%%%%%%%%%%%%%%%%%%%%%%%%%%%%%%%%%%%%%
\paragraph{v0.6:} 2017/04/26

\begin{itemize}
\item
redirection mechanism added
\end{itemize}

%%%%%%%%%%%%%%%%%%%%%%%%%%%%%%%%%%%%%%%%
\paragraph{v0.5:} 2017/04/26

\begin{itemize}
\item
functionality in definition file
\end{itemize}


%%%%%%%%%%%%%%%%%%%%%%%%%%%%%%%%%%%%%%%%%%%%%%%%%%%%%%%%%%%%%%%%%%%%%%%%%%%%%%%%
%%%%%%%%%%%%%%%%%%%%%%%%%%%%%%%%%%%%%%%%%%%%%%%%%%%%%%%%%%%%%%%%%%%%%%%%%%%%%%%%
%%%%%%%%%%%%%%%%%%%%%%%%%%%%%%%%%%%%%%%%%%%%%%%%%%%%%%%%%%%%%%%%%%%%%%%%%%%%%%%%
\appendix

\settowidth\MacroIndent{\rmfamily\scriptsize 000\ }

 \DocInput{childdoc.dtx}

\end{document}
%</driver>
% \fi
%
% %%%%%%%%%%%%%%%%%%%%%%%%%%%%%%%%%%%%%%%%%%%%%%%%%%%%%%%%%%%%%%%%%%%%%%%%%%%%%%
% %%%%%%%%%%%%%%%%%%%%%%%%%%%%%%%%%%%%%%%%%%%%%%%%%%%%%%%%%%%%%%%%%%%%%%%%%%%%%%
% \section{Sample}
%\iffalse
%<*samplemain>
%\fi
%
% The following presents a sample document
% with two chapters, two parts, a title page,
% a compile flag as well as three forwarding files to set the flag.
% It consists of eight |.tex| files:
% \begin{center}
% \begin{tabular}{ll}
% |cdocsamp.tex|&main file\\
% |cdocsch1.tex|&include file for chapter 1\\
% |cdocsch2.tex|&include file for chapter 2\\
% |cdocspt3.tex|&include file for part 3\\
% |cdocspt4.tex|&include file for part 4\\
% |cdocsdrf.tex|&forwarding file for main file in draft mode\\
% |cdocsfi1.tex|&forwarding file for final version of chapter 1\\
% |cdocsfi2.tex|&forwarding file for final version of chapter 2\\
% \end{tabular}
% \end{center}
% Each of the eight files can be compiled directly by the \LaTeX{} compiler.
%
% %%%%%%%%%%%%%%%%%%%%%%%%%%%%%%%%%%%%%%
% \paragraph{Main File.}
%
% The main file is called |cdocsamp.tex|.
%
% Load the \textsf{childdoc} definitions and
% declare the filename for the main document:
%    \begin{macrocode}
\input{childdoc.def}
\childdocmain{}
%    \end{macrocode}

% Optional override for |\version| flag:
%    \begin{macrocode}
%%\ifchilddoc\else\providecommand{\version}{draft}\fi
%    \end{macrocode}

% Define the default values for the |\version| flag
% (|final| for the main file and |draft| for childs):
%    \begin{macrocode}
\ifchilddoc
\providecommand{\version}{draft}
\else
\providecommand{\version}{final}
\fi
%    \end{macrocode}

% Load the standard document class:
%    \begin{macrocode}
\documentclass[12pt]{article}
%    \end{macrocode}

% Start the document body:
%    \begin{macrocode}
\begin{document}
%    \end{macrocode}

% Declare a title page.
% Print title, part of document being processed and version flag:
%    \begin{macrocode}
\addtocounter{page}{-1}
\begin{center}
{\LARGE\bfseries{}childdoc example\par}
\vspace{1cm}
\ifchilddoc
\ifchilddocmanual part\else chapter\fi:
`\childdocname' of `\childdocjob'\par
\else
main document: `\childdocjob'\par
\fi
version: \version\par
\end{center}
\newpage
%    \end{macrocode}

% Manually include selected file,
% otherwise process as usual:
%    \begin{macrocode}
\ifchilddocmanual
\section*{part `\childdocname'}
\input{\childdocname}
\else
%    \end{macrocode}

% Include the two chapters:
%    \begin{macrocode}
\include{cdocsch1}
\include{cdocsch2}
%    \end{macrocode}

% Include the two parts unless only chapters should be displayed:
%    \begin{macrocode}
\ifchilddoc\else
\section{part three}
\input{cdocspt3}
\section{part four}
\input{cdocspt4}
\fi
%    \end{macrocode}

% Process as usual until here:
%    \begin{macrocode}
\fi
%    \end{macrocode}

% End of document body:
%    \begin{macrocode}
\end{document}
%    \end{macrocode}
%\iffalse
%</samplemain>
%\fi
%
% %%%%%%%%%%%%%%%%%%%%%%%%%%%%%%%%%%%%%%
% \paragraph{Chapter Include Files.}
%
% The include files are called |cdocsch1.tex| and |cdocsch2.tex|.
%
%\iffalse
%<*samplechap1|samplechap2>
%\fi

% Optional override for |\version| flag:
%    \begin{macrocode}
%%\providecommand{\version}{final}
%    \end{macrocode}

% Include the main document:
%    \begin{macrocode}
\input{childdoc.def}
\childdocof{cdocsamp}
%    \end{macrocode}

%\iffalse
%</samplechap1|samplechap2>
%\fi
%
%\iffalse
%<*samplechap1>
%\fi
% Some text for chapter 1:
%    \begin{macrocode}
\section{one}
some text in chapter one
%    \end{macrocode}

%\iffalse
%</samplechap1>
%\fi
% Some text for chapter 2:
%\iffalse
%<*samplechap2>
%\fi
%    \begin{macrocode}
\section{two}
more text in chapter two
%    \end{macrocode}

%\iffalse
%</samplechap2>
%\fi
%
% %%%%%%%%%%%%%%%%%%%%%%%%%%%%%%%%%%%%%%
% \paragraph{Part Include Files.}
%
% The include files are called |cdocspt3.tex| and |cdocspt4.tex|.
%
%\iffalse
%<*samplepart3|samplepart4>
%\fi

% Optional override for |\version| flag:
%    \begin{macrocode}
%%\providecommand{\version}{final}
%    \end{macrocode}

% Include the main document:
%    \begin{macrocode}
\input{childdoc.def}
\childdocby{cdocsamp}
%    \end{macrocode}

%\iffalse
%</samplepart3|samplepart4>
%\fi
%
%\iffalse
%<*samplepart3>
%\fi
% Some text for part 3:
%    \begin{macrocode}
some text in part three
%    \end{macrocode}

%\iffalse
%</samplepart3>
%\fi
% Some text for part 4:
%\iffalse
%<*samplepart4>
%\fi
%    \begin{macrocode}
more text in part four
%    \end{macrocode}

%\iffalse
%</samplepart4>
%\fi
%
% %%%%%%%%%%%%%%%%%%%%%%%%%%%%%%%%%%%%%%
% \paragraph{Forwarding for a Complete Draft.}
%
% The following forwarding file |cdocsdrf.tex|
% compiles the main document in draft mode:
%\iffalse
%<*sampledraft>
%\fi
%    \begin{macrocode}
\def\version{draft}
\input{childdoc.def}
\childdocforward{cdocsamp}
%    \end{macrocode}

%\iffalse
%</sampledraft>
%\fi
%
% %%%%%%%%%%%%%%%%%%%%%%%%%%%%%%%%%%%%%%
% \paragraph{Forwarding for Final Version of the Chapters.}
%
% The following forwarding files |cdocsfn1.tex| and |cdocsfn2.tex|
% (with identical content)
% compile the final versions of the child documents
% |cdocsch1.tex| and |cdocsch2.tex|, respectively:
%\iffalse
%<*samplefinal>
%\fi
%    \begin{macrocode}
\def\version{final}
\input{childdoc.def}
\childdocforwardprefix[cdocsamp]{cdocsfn}{cdocsch}
%    \end{macrocode}

%\iffalse
%</samplefinal>
%\fi
%
% %%%%%%%%%%%%%%%%%%%%%%%%%%%%%%%%%%%%%%
% \paragraph{Command Line Processing.}
%
% The following three command lines generate the output files
% |cdocscld|, |cdocscl1| and |cdocscl2|
% which should be identical to
% |cdocsdrf|, |cdocsch1| and |cdocsfn2|, respectively:
% \begin{center}
% \begin{tabular}{l}
% |latex -jobname cdocscld \|\\
% |  "\def\version{draft}\input{childdoc.def}\childdocforward{cdocsamp}"|\\
% |latex -jobname cdocscl1 \|\\
% |  "\input{childdoc.def}\childdocforward[cdocsamp]{cdocsch1}"|\\
% |latex -jobname cdocscl2 \|\\
% |  "\def\version{final}\input{childdoc.def}\childdocforward{cdocsch2}"|
% \end{tabular}
% \end{center}
% Note that the trailing backslash on each first line
% merely continues the input to the second line
% (for convenient cut ant paste).
% Furthermore, the command |latex| can be replaced by any
% of its alternative versions such as |pdflatex|.
%
% %%%%%%%%%%%%%%%%%%%%%%%%%%%%%%%%%%%%%%%%%%%%%%%%%%%%%%%%%%%%%%%%%%%%%%%%%%%%%%
% %%%%%%%%%%%%%%%%%%%%%%%%%%%%%%%%%%%%%%%%%%%%%%%%%%%%%%%%%%%%%%%%%%%%%%%%%%%%%%
% \section{Implementation}
%\iffalse
%<*package>
%\fi
%
% This section describes the definitions file |childdoc.def|.

% The definitions cannot be loaded using |\usepackage| or |\RequirePackage|
% which has a mechanism to prevent loading a style file more than once.
% When loading the definitions by means of |\input|
% multiple instances have to be prevented manually:
%\iffalse
%This code needs to be before the `\ProvidesFile' directive
%which is defined at the beginning of this file.
%Therefore it is also placed there and commented out here.
%</package>
%<*discard>
%\fi
%    \begin{macrocode}
\ifdefined\childdocmain\endinput\fi
%    \end{macrocode}
%\iffalse
%</discard>
%<*package>
%\fi
%
% \macro{\ifchilddoc}
% \macro{\ifchilddocmanual}
% The conditional |\ifchilddoc| tells whether a
% child (true) or main (false) document is being compiled.
% The conditional |\ifchilddocmanual| tells whether
% the |\includeonly| mechanism is used (false) or
% the selection of child files must be performed manually (true).
% The definitions initialise to false:
%    \begin{macrocode}
\newif\ifchilddoc
\newif\ifchilddocmanual
%    \end{macrocode}

% \macro{\childdocname}
% \macro{\childdocjob}
% The macro |\childdocname| stores the name of the main document
% to be compiled. The macro |\childdocjob| stores the name of
% the document on which the \LaTeX{} compiler was originally invoked.
% The content of |\jobname| cannot be compared
% to filenames specified in the source due to different catcodes.
% The following code rescans |\jobname|, stores the result
% in |\childdocname| and saves a copy in |\childdocjob|:
%    \begin{macrocode}
\edef\childdocname{\scantokens\expandafter{\jobname\noexpand}}
\let\childdocjob\childdocname
%    \end{macrocode}

% \macro{\childdocdisable}
% The macro |\childdocdisable| prevents the main file
% from being processed more than once.
% At this stage, the main document command |\childdocmain|
% is assumed to be called once again where it should do nothing.
% Any subsequent call to it should prevent
% a secondary processing of the main document
% It overwrites the forwarding commands
% |\childdocof| and |\childdocforward|
% with empty macros to prevent further inclusions of the main document:
%    \begin{macrocode}
\newcommand{\childdocdisable}
{
  \renewcommand{\childdocmain}[1]{\renewcommand{\childdocmain}[1]{\endinput}}
  \renewcommand{\childdocof}[1]{}
  \renewcommand{\childdocby}[2][]{}
  \renewcommand{\childdocforward}[2][]{}
  \renewcommand{\childdocdisable}{}
}
%    \end{macrocode}

% \macro{\childdocmain}
% The macro |\childdocmain| is to be called at the top of the main file
% with nothing or the main filename (without extension) as argument.
% First, it breaks loops.
% If the argument is not empty and does not match |\childdocname|
% (which is set by the first inclusion of |childdoc.def|),
% |\ifchilddoc| is set to true, |\includeonly| is applied to the child file
% and |\jobname| is set to the main file
% (for proper handling of |.aux| files):
%    \begin{macrocode}
\newcommand{\childdocmain}[1]
{
  \childdocdisable\childdocmain{}
  \if?#1?\else
    \begingroup
      \def\childdoctmp{#1}
      \ifx\childdoctmp\childdocname
        \def\childdoctmp{}
      \else
        \def\childdoctmp
        {
          \childdoctrue
          \includeonly{\childdocname}
          \def\childdocjob{#1}
          \def\jobname{#1}
        }
      \fi
      \expandafter
    \endgroup
    \childdoctmp
  \fi
}
%    \end{macrocode}

% \macro{\childdocof}
% The command |\childdocof| redirects
% compilation to the main file |#1|.
%    \begin{macrocode}
\newcommand{\childdocof}[1]
{
  \childdocdisable
  \childdoctrue
  \includeonly{\childdocname}
  \def\jobname{#1}
  \def\childdocjob{#1}
  \input{#1}
}
%    \end{macrocode}

% \macro{\childdocby}
% The command |\childdocby| ....
%    \begin{macrocode}
\newcommand{\childdocby}[2][]
{
  \childdocdisable
  \childdoctrue
  \childdocmanualtrue
  \if?#1?\else
    \def\jobname{#2}
  \fi
  \def\childdocjob{#2}
  \input{#2}
  \endinput
}
%    \end{macrocode}

% \macro{\childdocforward}
% The command |\childdocforward| redirects
% compilation to the main file or
% (if the optional argument is given) a child file.
% Parameters are set as if the main file
% or a child file starting with |\childdocof| was compiled.
% Then compilation is handed over to the main file:
%    \begin{macrocode}
\newcommand{\childdocforward}[2][]
{
  \begingroup
    \if?#1?
      \def\childdoctmp
      {
        \def\childdocname{#2}
        \def\childdocjob{#2}
        \def\jobname{#2}
        \input{#2}
        \endinput
      }
    \else
      \def\childdoctmp
      {
        \childdocdisable
        \def\childdocname{#2}
        \childdoctrue
        \includeonly{#2}
        \def\childdocjob{#1}
        \def\jobname{#1}
        \input{#1}
        \endinput
      }
    \fi
    \expandafter
  \endgroup
  \childdoctmp
}
%    \end{macrocode}

% \macro{\childdocforwardprefix}
% The command |\childdocforwardprefix| redirects
% compilation to the main or a child file by means of a pattern.
% The prefix |#1| in the current filename is replaced by |#2|
% and the suffix of the current filename is kept
% (it is assumed that the filename does not contain the substring `|~~~|'
% which is used as a delimiter).
% Compilation is handed over to the new file by |\childdocforward|:
%    \begin{macrocode}
\newcommand{\childdocforwardprefix}[3][]
{
  \begingroup
    \def\childdocextract #2##1~~~{\def\childdoctmp{\childdocforward[#1]{#3##1}}}
    \expandafter\childdocextract\childdocname~~~
    \expandafter
  \endgroup
  \childdoctmp
}
%    \end{macrocode}

% \macro{\childdoc}
% The deprecated macro |\childdoc| is a legacy version of |\childdocmain|:
%    \begin{macrocode}
\newcommand{\childdoc}{\childdocmain}
%    \end{macrocode}

% \macro{\childdocredirect}
% The deprecated macro |\childdocredirect| is a legacy version
% of |\childdocforward| and |\childdocforwardprefix|:
%    \begin{macrocode}
\newcommand{\childdocredirect}[2][]
{
  \begingroup
    \if?#1?
      \def\childdoctmp{\childdocforward{#2}}
    \else
      \def\childdoctmp{\childdocforwardprefix{#1}{#2}}
    \fi
    \expandafter
  \endgroup
  \childdoctmp
}
%    \end{macrocode}

%\iffalse
%</package>
%\fi
%
\endinput
|\\
|\childdocforwardprefix{final}{child}|
\end{tabular}
\end{center}
%

Note that when several versions of a main file and/or of each child file
are to be generated, it may be convenient to set up a |Makefile| or
shell script to automatise the process.

%%%%%%%%%%%%%%%%%%%%%%%%%%%%%%%%%%%%%%%%%%%%%%%%%%%%%%%%%%%%%%%%%%%%%%%%%%%%%%%%
\subsection{Command Line Processing}
\label{sec:commandline}

The effect of redirection files can also be achieved by invoking
the \LaTeX{} compiler with a more elaborate command line.
Most conveniently this should be done as part
of a shell script or a |Makefile|.

When using \textsf{childdoc} in the main file, the following
command lines effectively perform a redirection
(note that depending on the shell being used,
backslashes may have to be doubled: `|\|' $\to$ `|\\|'):
%
\begin{center}
|... -jobname "|\textit{target}|" |\\|"|[\textit{flags}]%
|% \iffalse
%
% childdoc.dtx Copyright (C) 2017-2018 Niklas Beisert
%
% This work may be distributed and/or modified under the
% conditions of the LaTeX Project Public License, either version 1.3
% of this license or (at your option) any later version.
% The latest version of this license is in
%   http://www.latex-project.org/lppl.txt
% and version 1.3 or later is part of all distributions of LaTeX
% version 2005/12/01 or later.
%
% This work has the LPPL maintenance status `maintained'.
%
% The Current Maintainer of this work is Niklas Beisert.
%
% This work consists of the files childdoc.dtx and childdoc.ins
% and the derived files childdoc.def and cdocsamp.tex with
% cdocsch1.tex, cdocsch2.tex, cdocsdrf.tex, cdocsfn1.tex, cdocsfn2.tex.
%
%<package>\ifdefined\childdocmain\endinput\fi
%<package>\ProvidesFile{childdoc.def}[2018/12/30 v2.0 child document driver]
%<samplemain>\ProvidesFile{cdocsamp.tex}[2018/12/30 v2.0 sample for childdoc]
%<*driver>
%\ProvidesFile{childdoc.drv}[2018/12/30 v2.0 childdoc reference manual file]
\PassOptionsToClass{10pt,a4paper}{article}
\documentclass{ltxdoc}

\usepackage[margin=35mm]{geometry}
\usepackage{hyperref}
\usepackage{hyperxmp}
\usepackage[usenames]{color}

\hypersetup{colorlinks=true}
\hypersetup{pdfstartview=FitH}
\hypersetup{pdfpagemode=UseNone}
\hypersetup{pdfsource={}}
\hypersetup{pdflang={en-UK}}
\hypersetup{pdfcopyright={Copyright 2017-2018 Niklas Beisert.
  This work may be distributed and/or modified under the
  conditions of the LaTeX Project Public License, either version 1.3
  of this license or (at your option) any later version.}}
\hypersetup{pdflicenseurl={http://www.latex-project.org/lppl.txt}}
\hypersetup{pdfcontactaddress={ETH Zurich, ITP, HIT K,
  Wolfgang-Pauli-Strasse 27}}
\hypersetup{pdfcontactpostcode={8093}}
\hypersetup{pdfcontactcity={Zurich}}
\hypersetup{pdfcontactcountry={Switzerland}}
\hypersetup{pdfcontactemail={nbeisert@itp.phys.ethz.ch}}
\hypersetup{pdfcontacturl={http://people.phys.ethz.ch/\xmptilde nbeisert/}}

\newcommand{\secref}[1]{\hyperref[#1]{section \ref*{#1}}}

\parskip1ex
\parindent0pt
\let\olditemize\itemize
\def\itemize{\olditemize\parskip0pt}

\begin{document}

\title{The \textsf{childdoc} Package}
\hypersetup{pdftitle={The childdoc Package}}
\author{Niklas Beisert\\[2ex]
  Institut f\"ur Theoretische Physik\\
  Eidgen\"ossische Technische Hochschule Z\"urich\\
  Wolfgang-Pauli-Strasse 27, 8093 Z\"urich, Switzerland\\[1ex]
  \href{mailto:nbeisert@itp.phys.ethz.ch}
  {\texttt{nbeisert@itp.phys.ethz.ch}}}
\hypersetup{pdfauthor={Niklas Beisert}}
\hypersetup{pdfsubject={Manual for the LaTeX2e Package childdoc}}
\date{30 December 2018, \textsf{v2.0}}
\maketitle

\begin{abstract}\noindent
\textsf{childdoc} is a \LaTeXe{} package
that enables the direct compilation
of document sections included by |\include|
to individual files.
\end{abstract}

\begingroup
\parskip0ex
\tableofcontents
\endgroup

%%%%%%%%%%%%%%%%%%%%%%%%%%%%%%%%%%%%%%%%%%%%%%%%%%%%%%%%%%%%%%%%%%%%%%%%%%%%%%%%
%%%%%%%%%%%%%%%%%%%%%%%%%%%%%%%%%%%%%%%%%%%%%%%%%%%%%%%%%%%%%%%%%%%%%%%%%%%%%%%%
\section{Introduction}

\LaTeX{} provides a mechanism to structure a large document (such as a book)
into a main file and several child files (containing the chapters)
using the |\include| command.
This mechanism is beneficial for documents
which span hundreds of pages in order to
make the source file(s) more manageable.
Moreover, compilation can be restricted to
selected child files by means of the |\includeonly| command.
The latter feature can be used to reduce the compilation time while editing
(this was significantly more useful in the earlier days of \LaTeX{})
or to generate a smaller document which is easier to navigate.
Another application of |\includeonly| is to generate
documents consisting of selected parts of the complete document.

However, there are a few drawbacks of the plain |\include| mechanism:
\begin{itemize}
\item
The child files cannot be compiled on their own,
they can only be compiled via the main file.
A naive editing environment
(such as a text editor with an option
to have the current file processed by \LaTeX)
may require one to switch to the main file before compiling;
attempting to compile the child file produces errors.
\item
The main file must be modified (each time)
to adjust the |\includeonly| command
to the present needs. This easily leaves the main file in a messy state.
\item
The generated document will always carry the filename
of the main document. This is inconvenient if
several child files are to be compiled and
to be kept for distribution.
\end{itemize}

The present package provides a simple interface
to make child files individually compilable by \LaTeX{}.
Compiling a child file then has the same effect as compiling
the main file with an |\includeonly| command
to select the appropriate child.
Moreover the generated document will carry the name of the child
rather than the main file.
This resolves all three above issues.

This feature is meant to make the editing of books,
thesis documents and lecture notes somewhat more convenient.
However, the package can also be used efficiently for
composing a series of documents (such as exercise sheets)
which are typically distributed individually.
It then assists the author in generating the individual documents
(potentially in different versions)
as well as a document containing the collected series.
Another application is in developing style files
or other kinds of included material
where compilation of the style file could redirect
to a sample or test file.

%%%%%%%%%%%%%%%%%%%%%%%%%%%%%%%%%%%%%%%%%%%%%%%%%%%%%%%%%%%%%%%%%%%%%%%%%%%%%%%%
%%%%%%%%%%%%%%%%%%%%%%%%%%%%%%%%%%%%%%%%%%%%%%%%%%%%%%%%%%%%%%%%%%%%%%%%%%%%%%%%
\section{Usage}

First of all, the package \textsf{childdoc} is \emph{not} a standard
\LaTeXe{} |.sty| style file! Therefore it needs to be invoked in
a non-standard way.

%%%%%%%%%%%%%%%%%%%%%%%%%%%%%%%%%%%%%%%%%%%%%%%%%%%%%%%%%%%%%%%%%%%%%%%%%%%%%%%%
\subsection{Included Files}
\label{sec:include}

%%%%%%%%%%%%%%%%%%%%%%%%%%%%%%%%%%%%%%%%
\DescribeMacro{\childdocmain}
To use the package, add the commands
\begin{center}
\begin{tabular}{l}
|\input{childdoc.def}|\\
|\childdocmain{}|\\
\end{tabular}
\end{center}
at the very top of the main \LaTeX{} file,
in particular \emph{before} the |\documentclass| statement!
The argument of |\childdocmain| should be left empty
(but it must be present).

%%%%%%%%%%%%%%%%%%%%%%%%%%%%%%%%%%%%%%%%
\DescribeMacro{\childdocof}
Furthermore, add the commands
\begin{center}
\begin{tabular}{l}
|\input{childdoc.def}|\\
|\childdocof{|\textit{main}|}|\\
\end{tabular}
\end{center}
at the top of every child file \textit{child}
which is included by |\include{|\textit{child}|}|
from within the main file
(or at least for those files to be compiled individually).
The argument \textit{main} must be the filename of the main file.

There are a couple of
considerations in setting up the main and child documents:

%%%%%%%%%%%%%%%%%%%%%%%%%%%%%%%%%%%%%%%%
\paragraph{Restrictions.}

Please note the following restrictions:
\begin{itemize}
\item
|\childdocmain| must be called with one argument \textit{main}
to ensure compatibility with earlier version of the package.
It must either be empty (|\childdocmain{}|)
or precisely match the filename of the main file in which it is specified.
See \secref{sec:detection} for further information.
\item
The filename \textit{main} must be specified without the |.tex| extension.
\item
The filename \textit{main} is case sensitive
(even in case-insensitive file systems)
due to internal string comparison.
\item
The argument \textit{main} should be fully expanded, it cannot be a macro.
\item
Subdirectories and special characters should be avoided in filenames.
\item
The command |\childdocmain{|\textit{main}|}| must be followed by a whitespace.
It should not be followed immediately by another command
or by a comment mark `|%|'.
This is because the \TeX{} parser reads the token immediately following
the argument of |\childdocmain| and puts it
at the beginning of every child section;
however, a white\-space is ignored.
\end{itemize}

%%%%%%%%%%%%%%%%%%%%%%%%%%%%%%%%%%%%%%%%
\paragraph{Content of Main File.}

It is advisable to place all content in the child files included by |\include|.
Any output contained in the main file will appear in all child documents
unless suppressed manually;
it cannot be suppressed automatically by the |\includeonly| directive
and thus should normally be avoided.
A method to include some content in the main file
by means of conditional processing is described in \secref{sec:conditional}.

%%%%%%%%%%%%%%%%%%%%%%%%%%%%%%%%%%%%%%%%
\paragraph{Page Numbering.}

When only a part of the document is compiled,
the appropriate numbering of pages
(as well as other status parameters)
is determined from the |.aux| files.
The latter contain information from previous passes.
However this information needs to propagate through
all intermediate child documents.
Therefore the page numbering in child documents may well
be inconsistent until the complete document is compiled at least once.

A useful (if unconventional) way to always ensure a consistent
page numbering is to restart the numbering in each child document
and denote the pages by `\textit{child}|.|\textit{page}'
where \textit{child} represents the chapter/section number of the child file.
This can be achieved by the command
|\numberwithin{page}{|\textit{child}|}|
of the \textsf{amsmath} package
where \textit{child} can be |chapter| or |section|
depending on the chosen structuring.
Alternatively, one can modify the macro |\thepage| appropriately
and reset the counter |page| at the start of each child file.

%%%%%%%%%%%%%%%%%%%%%%%%%%%%%%%%%%%%%%%%%%%%%%%%%%%%%%%%%%%%%%%%%%%%%%%%%%%%%%%%
\subsection{Conditional Processing}
\label{sec:conditional}

The package provides a mechanism to compile different versions
of a document. To customise the versions further some conditional processing
can come in handy to distinguish which version is being compiled.
The package provides two macros to describe the compilation context:

%%%%%%%%%%%%%%%%%%%%%%%%%%%%%%%%%%%%%%%%
\DescribeMacro{\ifchilddoc}
The conditional |\ifchilddoc| distinguishes between the compilation of
child documents and the main document:
%
\begin{center}
|\ifchilddoc |\textit{child-code}| |[|\||else |\textit{main-code}]| \||fi|
\end{center}

%%%%%%%%%%%%%%%%%%%%%%%%%%%%%%%%%%%%%%%%
\DescribeMacro{\childdocname}
\DescribeMacro{\childdocjob}
The macro |\childdocname| contains the filename (without extension)
of the main or child file being processed.
Note that |\childdocjob| will always contain the name of the main file.

%%%%%%%%%%%%%%%%%%%%%%%%%%%%%%%%%%%%%%%%
\paragraph{Title Page.}

Conditional processing can be used to include a title or banner page
in the main document when proper precautions are taken.
Importantly, the code in the main file should ensure that the page counter
(as well as other status parameters which are stored in the |.aux| files)
takes the same value after the conditional processing.
Otherwise the page numbers may take divergent values
depending on which part is compiled.

For example, a title page could be declared by:
%
\begin{center}
\begin{tabular}{l}
|\ifchilddoc\||else|\\
|\addtocounter{page}{-1}|\\
\textit{code for title page}\\
|\newpage|\\
|\||fi|
\end{tabular}
\end{center}
%
A banner page for the child documents can be generated by:
%
\begin{center}
\begin{tabular}{l}
|\ifchilddoc|\\
|\addtocounter{page}{-1}|\\
\textit{code for banner page}\\
|\newpage|\\
|\||fi|
\end{tabular}
\end{center}
%
Here one could write a message such as:
\begin{center}
|This is the part \childdocname{} of \childdocjob{}.|
\end{center}

%%%%%%%%%%%%%%%%%%%%%%%%%%%%%%%%%%%%%%%%%%%%%%%%%%%%%%%%%%%%%%%%%%%%%%%%%%%%%%%%
\subsection{Flags}
\label{sec:flags}

The package makes it easy to generate different versions
of the main or child documents.
To this end compilation flags can be defined
and assigned different default values.
They will be particularly useful in conjunction
with the forwarding mechanism described in \secref{sec:forward}.

For example, it may be useful to have a flag |\version|
which can be set to |draft| or |final|.
The document source will contain some conditional code
depending on the value of |\version|.
Suppose further, the flag should default to |final| for the main file
and to |draft| for child files
which is a natural assignment for editing the document.
This is achieved by placing the following code
in the preamble of the main document
(below the |\childdocmain| directive):
%
\begin{center}
\begin{tabular}{l}
|\ifchilddoc|\\
|\providecommand{\version}{draft}|\\
|\||else|\\
|\providecommand{\version}{final}|\\
|\||fi|
\end{tabular}
\end{center}
%
The definition by |\providecommand| makes sure
that previous definitions are not overwritten.
Further statements |\providecommand{\version}{...}|
can thus be added before the above code to override it.

For the main file, one might add a line
(between |\childdocmain| and the above block)
%
\begin{center}
|%\ifchilddoc\||else\providecommand{\version}{draft}\||fi|
\end{center}
%
which can be uncommented to produce a draft version.
Likewise one can add a line to the very top of a child file
(above the |\childdocof{|\textit{main}|}| directive)
%
\begin{center}
|%\providecommand{\version}{final}|
\end{center}
%
which can be uncommented to produce the final version of this child document.

%%%%%%%%%%%%%%%%%%%%%%%%%%%%%%%%%%%%%%%%%%%%%%%%%%%%%%%%%%%%%%%%%%%%%%%%%%%%%%%%
\subsection{Forwarding}
\label{sec:forward}

Different versions of the main or child documents
using compilation flags as described in \secref{sec:flags}
can be (permanently) stored in different files
for convenient compilation, viewing and distribution.
To this end, the package defines a command
to pass on compilation to a different file:

%%%%%%%%%%%%%%%%%%%%%%%%%%%%%%%%%%%%%%%%
\DescribeMacro{\childdocforward}
The command |\childdocforward| redirects processing to
another source file:
%
\begin{center}
\begin{tabular}{l}
|\input{childdoc.def}|\\
|\childdocforward[|\textit{main}|]{|\textit{dest}|}|\\
\end{tabular}
\end{center}
%
The argument \textit{dest} is the destination file
(without extension).
It should be the main file or one of the child files.
Note that further \textsf{childdoc} directives
such as |\childdocof| and |\childdocforward|
in the indicated file will be processed in this form.
The optional argument \textit{main}
passes on directly to the main file \textit{main}
while pretending to compile the child \textit{dest}.
This form behaves as if \textit{dest}
issues |\childdocof{|\textit{main}|}| right away,
and no further \textsf{childdoc} directives will be processed.

%%%%%%%%%%%%%%%%%%%%%%%%%%%%%%%%%%%%%%%%
\DescribeMacro{\...prefix}
In the alternative form |\childdocforwardprefix|,
%
\begin{center}
\begin{tabular}{l}
|\input{childdoc.def}|\\
|\childdocforwardprefix[|\textit{main}|]{|\textit{prefix}|}{|\textit{dest}|}|
\end{tabular}
\end{center}
%
the destination file is determined by a pattern
depending on the current file:
To make this work, the current file must be called
`{\textit{prefix}\hspace{0.2em}\textit{suffix}}'
with \textit{prefix} matching precisely the argument.
Processing is then passed on to the file
`{\textit{dest}\hspace{0.2em}\textit{suffix}}'.
Surely, the same effect is achieved by
directly specifying the
argument `{\textit{dest}\hspace{0.2em}\textit{suffix}}'
in the first form.
However, that requires to set up a different file
for each child. With the alternative form of the command
all these files can have exactly the same content
which simplifies setting them up and maintaining them.

For example, the following file |draft.tex|
with a compilation flag |\version| as described in \secref{sec:flags}
compiles the main document as a draft:
%
\begin{center}
\begin{tabular}{l}
|\def\version{draft}|\\
|\input{childdoc.def}|\\
|\childdocforward{|\textit{main}|}|
\end{tabular}
\end{center}
%
Likewise, the following files |final|\textit{nn}|.tex|
compile the final version of the child document
|child|\textit{nn}|.tex|:
%
\begin{center}
\begin{tabular}{l}
|\def\version{final}|\\
|\input{childdoc.def}|\\
|\childdocforwardprefix{final}{child}|
\end{tabular}
\end{center}
%

Note that when several versions of a main file and/or of each child file
are to be generated, it may be convenient to set up a |Makefile| or
shell script to automatise the process.

%%%%%%%%%%%%%%%%%%%%%%%%%%%%%%%%%%%%%%%%%%%%%%%%%%%%%%%%%%%%%%%%%%%%%%%%%%%%%%%%
\subsection{Command Line Processing}
\label{sec:commandline}

The effect of redirection files can also be achieved by invoking
the \LaTeX{} compiler with a more elaborate command line.
Most conveniently this should be done as part
of a shell script or a |Makefile|.

When using \textsf{childdoc} in the main file, the following
command lines effectively perform a redirection
(note that depending on the shell being used,
backslashes may have to be doubled: `|\|' $\to$ `|\\|'):
%
\begin{center}
|... -jobname "|\textit{target}|" |\\|"|[\textit{flags}]%
|\input{childdoc.def}\childdocforward[|\textit{main}|]{|\textit{dest}|}"|
\end{center}
%
Here \textit{target} is the name of the output file,
\textit{main} is the name of the main file
and \textit{dest} is the name of the main or child file to be processed
(all filenames without extensions).
The optional argument \textit{main} can be omitted
if \textit{main} matches \textit{dest}.
Optionally, compilation \textit{flags} can be defined via |\def| commands.
This command line makes the \TeX{} engine believe
it is compiling the file \textit{target}
whose content is specified as the latter parameter.
The provided code then forwards the processing to
\textit{main} or \textit{dest} as described in \secref{sec:forward}.

%%%%%%%%%%%%%%%%%%%%%%%%%%%%%%%%%%%%%%%%%%%%%%%%%%%%%%%%%%%%%%%%%%%%%%%%%%%%%%%%
\subsection{Include by Input}
\label{sec:input}

Including child documents by |\include| has some restrictions by design.
Most notably, the content of a child document always occupies
its own set of pages; pages cannot be shared between child documents.
Usually, this behaviour makes perfect sense
because each child document contain an essential part of the document.
However, in some situations it may be desirable to compose
a document from a collection of parts
without having mandatory page breaks between then.
For this case, the package
provides a mechanism to include parts
by |\input| which can also be processed individually.
However, by construction this mechanism
requires manual handling of the content to be output.

%%%%%%%%%%%%%%%%%%%%%%%%%%%%%%%%%%%%%%%%
\DescribeMacro{\ifchilddocmanual}
The main file should be prepared as usual, see \secref{sec:include}.
However, the document body must make a distinction
between processing of an individual part and of the main document, e.g.:
%
\begin{center}
\begin{tabular}{l}
|\ifchilddocmanual|\\
|\input{\childdocname}|\\
|\||else|\\
\textit{document body with }|\input{|\textit{part}|}|\\
|\||fi|
\end{tabular}
\end{center}
%
The conditional |\ifchilddocmanual| is true whenever
a part to be included by |\input| is being compiled,
and the name of the part is stored in |\childdocname|.

%%%%%%%%%%%%%%%%%%%%%%%%%%%%%%%%%%%%%%%%
\DescribeMacro{\childdocby}
Each part to be included by |\input| should start with:
%
\begin{center}
\begin{tabular}{l}
|\input{childdoc.def}|\\
|\childdocby{|\textit{main}|}|\\
\end{tabular}
\end{center}
%
The directive |\childdocby| is similar to |\childdocof|
described in \secref{sec:include},
but the subsequent selection of content must be done manually.
To that end, both |\ifchilddoc| and |\ifchilddocmanual|
will be true upon processing of a part,
and the name of the part is stored in |\childdocname|.
Note that |\jobname| will be set to the filename of the current part
so that each part receives an individual |.aux| file
that does not interfere with the |.aux| file(s) of the main document.
This behaviour can be altered by the alternative form
|\childdocby[*]{|\textit{main}|}| (with a non-empty optional argument)
which uses the |.aux| file of the main document
by setting |\jobname| to \textit{main}.

%%%%%%%%%%%%%%%%%%%%%%%%%%%%%%%%%%%%%%%%%%%%%%%%%%%%%%%%%%%%%%%%%%%%%%%%%%%%%%%%
\subsection{Driver Development}
\label{sec:driver}

The \textsf{childdoc} mechanism can also be use for the development
of definition files such as \LaTeX{} styles or classes.
This case differs from the above setup with multiple parts
included by |\include| in that no |\includeonly| should be invoked.
This can be achieved by starting the include file
(before |\ProvidesPackage|) with:
%
\begin{center}
\begin{tabular}{l}
|\input{childdoc.def}|\\
|\childdocforward{|\textit{main}|}|\\
\end{tabular}
\end{center}
%
or alternatively with:
%
\begin{center}
\begin{tabular}{l}
|\input{childdoc.def}|\\
|\childdocby{|\textit{main}|}|\\
\end{tabular}
\end{center}
%
Both forms have slightly different effects as described above.
The main file is prepared as usual, see \secref{sec:include}.

%%%%%%%%%%%%%%%%%%%%%%%%%%%%%%%%%%%%%%%%%%%%%%%%%%%%%%%%%%%%%%%%%%%%%%%%%%%%%%%%
\subsection{Legacy Detection}
\label{sec:detection}

The directive |\childdocmain| in the main file can detect
whether the complete document or merely a child is to be compiled
even without using the directive |\childdocof|.
This method is deprecated because it is less robust
and there is no compelling reason to use it;
it is merely provided for backward compatibility
and it may be removed in future versions.

If the detection mechanism is to be used,
it is mandatory to correctly specify
the filename of the main file as the argument of |\childdocmain|:
%
\begin{center}
\begin{tabular}{l}
|\input{childdoc.def}|\\
|\childdocmain{|\textit{main}|}|\\
\end{tabular}
\end{center}
%
If |\jobname| does not match the argument \textit{main} of |\childdocmain|,
it is assumed that |\jobname| points to the child file to be compiled.
When using |\childdocmain| with the main file specified as argument,
it suffices to start a child file
with just |\input{|\textit{main}|}|
without loading of the package and using |\childdocof|.
If instead all processing is done
with the appropriate \textsf{childdoc} directives,
the argument of \textit{main} of |\childdocmain| can be empty.

An alternative version of the command line processing described
in \secref{sec:commandline} using the detection mechanism reads:
%
\begin{center}
|... -jobname "|\textit{target}|" "|[\textit{flags}]%
[|\def\jobname{|\textit{dest}|}|]|\input{|\textit{main}|}"|
\end{center}

%%%%%%%%%%%%%%%%%%%%%%%%%%%%%%%%%%%%%%%%%%%%%%%%%%%%%%%%%%%%%%%%%%%%%%%%%%%%%%%%
\subsection{Manual Code}
\label{sec:manual}

In case one cannot be certain whether the definitions file |childdoc.def|
is installed on the target \TeX{} distribution
and one prefers not to ship it,
it is conceivable to paste a few relevant commands into the sources.

To that end, drop all statements |\input{childdoc.def}|
and perform the replacements as outlined below.
Instead of |\childdocmain{|\textit{main}|}| add the following code
to the top of the main file:
%
\begin{center}
\begin{tabular}{l}
|\||ifdefined\childdocname\endinput\||fi\newif\ifchilddoc|\\
|\edef\childdocname{\scantokens\expandafter{\jobname\noexpand}}|\\
|\def\childdocmain{|\textit{main}|}\||ifx\childdocmain\childdocname\||else|\\
|\childdoctrue\includeonly{\childdocname}\let\jobname\childdocmain\||fi|\\
\end{tabular}
\end{center}
%
Instead of |\childdocof{|\textit{main}|}| just include the main file
at the top of each child file:
%
\begin{center}
|\input{|\textit{main}|}|
\end{center}
%
A simple redirection |\childdocforward{|\textit{dest}|}| is achieved by:
%
\begin{center}
|\def\jobname{|\textit{dest}|}\input{\jobname}|
\end{center}
%
The redirection with prefix
|\childdocforwardprefix[|\textit{prefix}|]{|\textit{dest}|}|
is accomplished by:
%
\begin{center}
\begin{tabular}{l}
|{\edef\jobname{\scantokens\expandafter{\jobname\noexpand}}|\\
|\def\redirectjob |\textit{prefix}|#1~~~{\gdef\jobname{|\textit{dest}|#1}}|\\
|\expandafter\redirectjob\jobname~~~}\input{\jobname}|
\end{tabular}
\end{center}

In an alternative approach,
child documents can be compiled by a specific command line
without additional code or specific definitions:
%
\begin{center}
|... -jobname "|\textit{target}|" "|[\textit{flags}]%
|\includeonly{|\textit{dest}|}\input{|\textit{main}|}"|
\end{center}
%

%%%%%%%%%%%%%%%%%%%%%%%%%%%%%%%%%%%%%%%%%%%%%%%%%%%%%%%%%%%%%%%%%%%%%%%%%%%%%%%%
%%%%%%%%%%%%%%%%%%%%%%%%%%%%%%%%%%%%%%%%%%%%%%%%%%%%%%%%%%%%%%%%%%%%%%%%%%%%%%%%
\section{Information}

%%%%%%%%%%%%%%%%%%%%%%%%%%%%%%%%%%%%%%%%%%%%%%%%%%%%%%%%%%%%%%%%%%%%%%%%%%%%%%%%
\subsection{Copyright}

Copyright \copyright{} 2017--2018 Niklas Beisert

This work may be distributed and/or modified under the
conditions of the \LaTeX{} Project Public License, either version 1.3
of this license or (at your option) any later version.
The latest version of this license is in
  \url{http://www.latex-project.org/lppl.txt}
and version 1.3 or later is part of all distributions of \LaTeX{}
version 2005/12/01 or later.

This work has the LPPL maintenance status `maintained'.

The Current Maintainer of this work is Niklas Beisert.

This work consists of the files |README.txt|, |childdoc.ins| and |childdoc.dtx|
as well as the derived files |childdoc.def|, |cdocsamp.tex|
with |cdocsch1.tex|, |cdocsch2.tex|, |cdocspt3.tex|, |cdocspt4.tex|,
|cdocsdrf.tex|, |cdocsfn1.tex|, |cdocsfn2.tex|
as well as |childdoc.pdf|.

%%%%%%%%%%%%%%%%%%%%%%%%%%%%%%%%%%%%%%%%%%%%%%%%%%%%%%%%%%%%%%%%%%%%%%%%%%%%%%%%
\subsection{Files and Installation}

The package consists of the files:
%
\begin{center}
\begin{tabular}{ll}
    |README.txt|   & readme file \\
    |childdoc.ins| & installation file \\
    |childdoc.dtx| & source file \\
    |childdoc.def| & definition file \\
    |cdocsamp.tex| & sample main file \\
    |cdocsch1.tex| & sample include file \\
    |cdocsch2.tex| & sample include file \\
    |cdocspt3.tex| & sample part file \\
    |cdocspt4.tex| & sample part file \\
    |cdocsdrf.tex| & sample redirection file \\
    |cdocsfn1.tex| & sample redirection file \\
    |cdocsfn2.tex| & sample redirection file \\
    |childdoc.pdf| & manual
\end{tabular}
\end{center}
%
The distribution consists of the files
|README.txt|, |childdoc.ins| and |childdoc.dtx|.
%
\begin{itemize}
\item
Run (pdf)\LaTeX{} on |childdoc.dtx|
to compile the manual |childdoc.pdf| (this file).
\item
Run \LaTeX{} on |childdoc.ins| to create the definitions file |childdoc.def|
and the sample |cdocsamp.tex| with include files
|cdocsch1.tex|, |cdocsch2.tex|, |cdocspt3.tex|, |cdocspt4.tex|,
|cdocsdrf.tex|, |cdocsfn1.tex|, |cdocsfn2.tex|.
Then copy the file |childdoc.def| to an appropriate directory of your \LaTeX{}
distribution, e.g.\ \textit{texmf-root}|/tex/latex/childdoc|.
\end{itemize}

%%%%%%%%%%%%%%%%%%%%%%%%%%%%%%%%%%%%%%%%%%%%%%%%%%%%%%%%%%%%%%%%%%%%%%%%%%%%%%%%
\subsection{Related CTAN Packages}

There are several other packages which offer a similar functionality:
%
\begin{itemize}
\item
The packages
\href{http://ctan.org/pkg/docmute}{\textsf{docmute}},
\href{http://ctan.org/pkg/includex}{\textsf{includex}} and
\href{http://ctan.org/pkg/standalone}{\textsf{standalone}}
provide commands to include only the document body of
a child file thus allowing both files to be compiled individually.
\item
The packages \href{http://ctan.org/pkg/subdocs}{\textsf{subdocs}}
and \href{http://ctan.org/pkg/subfiles}{\textsf{subfiles}}
provide structures in which the main and child documents can be
encapsulated and allowing them to be compiled individually.
The inclusion mechanism is different from the conventional |\include|.
\item
The package \href{http://ctan.org/pkg/combine}{\textsf{combine}}
is an elaborate solution to combine several documents into one.
\end{itemize}
%
See also the CTAN topic \href{http://ctan.org/topic/subdocs}{\textsf{subdocs}}
for further related packages.
The present package differs from the above solutions in that
a document structure constructed with the conventional |\include| mechanism
just needs two extra commands at the top of every file
such that all constituent files can be compiled individually.

%%%%%%%%%%%%%%%%%%%%%%%%%%%%%%%%%%%%%%%%%%%%%%%%%%%%%%%%%%%%%%%%%%%%%%%%%%%%%%%%
%\subsection{Feature Suggestions}
%
%The following is a list of features which may be useful for future
%versions of this package:
%%
%\begin{itemize}
%\item
%\ldots
%\end{itemize}

%%%%%%%%%%%%%%%%%%%%%%%%%%%%%%%%%%%%%%%%%%%%%%%%%%%%%%%%%%%%%%%%%%%%%%%%%%%%%%%%
\subsection{Revision History}

%%%%%%%%%%%%%%%%%%%%%%%%%%%%%%%%%%%%%%%%
\paragraph{v2.0:} 2018/12/30

\begin{itemize}
\item
immediate forward processing
\item
added |\childdocby| mechanism
\item
manual restructured
\end{itemize}

%%%%%%%%%%%%%%%%%%%%%%%%%%%%%%%%%%%%%%%%
\paragraph{v1.6:} 2018/01/17

\begin{itemize}
\item
application for development of include files
\item
corrections to manual
\end{itemize}

%%%%%%%%%%%%%%%%%%%%%%%%%%%%%%%%%%%%%%%%
\paragraph{v1.5:} 2017/05/21

\begin{itemize}
\item
more complete structuring introduced
\item
|\childdocof| introduced
\item
|\childdoc| renamed to |\childdocmain|
\item
|\childredirect| renamed to |\childdocforward| and |\childdocforwardprefix|
and functionality expanded
\end{itemize}

%%%%%%%%%%%%%%%%%%%%%%%%%%%%%%%%%%%%%%%%
\paragraph{v1.0:} 2017/04/27

\begin{itemize}
\item
manual and install package
\item
first version published on CTAN
\end{itemize}

%%%%%%%%%%%%%%%%%%%%%%%%%%%%%%%%%%%%%%%%
\paragraph{v0.6:} 2017/04/26

\begin{itemize}
\item
redirection mechanism added
\end{itemize}

%%%%%%%%%%%%%%%%%%%%%%%%%%%%%%%%%%%%%%%%
\paragraph{v0.5:} 2017/04/26

\begin{itemize}
\item
functionality in definition file
\end{itemize}


%%%%%%%%%%%%%%%%%%%%%%%%%%%%%%%%%%%%%%%%%%%%%%%%%%%%%%%%%%%%%%%%%%%%%%%%%%%%%%%%
%%%%%%%%%%%%%%%%%%%%%%%%%%%%%%%%%%%%%%%%%%%%%%%%%%%%%%%%%%%%%%%%%%%%%%%%%%%%%%%%
%%%%%%%%%%%%%%%%%%%%%%%%%%%%%%%%%%%%%%%%%%%%%%%%%%%%%%%%%%%%%%%%%%%%%%%%%%%%%%%%
\appendix

\settowidth\MacroIndent{\rmfamily\scriptsize 000\ }

 \DocInput{childdoc.dtx}

\end{document}
%</driver>
% \fi
%
% %%%%%%%%%%%%%%%%%%%%%%%%%%%%%%%%%%%%%%%%%%%%%%%%%%%%%%%%%%%%%%%%%%%%%%%%%%%%%%
% %%%%%%%%%%%%%%%%%%%%%%%%%%%%%%%%%%%%%%%%%%%%%%%%%%%%%%%%%%%%%%%%%%%%%%%%%%%%%%
% \section{Sample}
%\iffalse
%<*samplemain>
%\fi
%
% The following presents a sample document
% with two chapters, two parts, a title page,
% a compile flag as well as three forwarding files to set the flag.
% It consists of eight |.tex| files:
% \begin{center}
% \begin{tabular}{ll}
% |cdocsamp.tex|&main file\\
% |cdocsch1.tex|&include file for chapter 1\\
% |cdocsch2.tex|&include file for chapter 2\\
% |cdocspt3.tex|&include file for part 3\\
% |cdocspt4.tex|&include file for part 4\\
% |cdocsdrf.tex|&forwarding file for main file in draft mode\\
% |cdocsfi1.tex|&forwarding file for final version of chapter 1\\
% |cdocsfi2.tex|&forwarding file for final version of chapter 2\\
% \end{tabular}
% \end{center}
% Each of the eight files can be compiled directly by the \LaTeX{} compiler.
%
% %%%%%%%%%%%%%%%%%%%%%%%%%%%%%%%%%%%%%%
% \paragraph{Main File.}
%
% The main file is called |cdocsamp.tex|.
%
% Load the \textsf{childdoc} definitions and
% declare the filename for the main document:
%    \begin{macrocode}
\input{childdoc.def}
\childdocmain{}
%    \end{macrocode}

% Optional override for |\version| flag:
%    \begin{macrocode}
%%\ifchilddoc\else\providecommand{\version}{draft}\fi
%    \end{macrocode}

% Define the default values for the |\version| flag
% (|final| for the main file and |draft| for childs):
%    \begin{macrocode}
\ifchilddoc
\providecommand{\version}{draft}
\else
\providecommand{\version}{final}
\fi
%    \end{macrocode}

% Load the standard document class:
%    \begin{macrocode}
\documentclass[12pt]{article}
%    \end{macrocode}

% Start the document body:
%    \begin{macrocode}
\begin{document}
%    \end{macrocode}

% Declare a title page.
% Print title, part of document being processed and version flag:
%    \begin{macrocode}
\addtocounter{page}{-1}
\begin{center}
{\LARGE\bfseries{}childdoc example\par}
\vspace{1cm}
\ifchilddoc
\ifchilddocmanual part\else chapter\fi:
`\childdocname' of `\childdocjob'\par
\else
main document: `\childdocjob'\par
\fi
version: \version\par
\end{center}
\newpage
%    \end{macrocode}

% Manually include selected file,
% otherwise process as usual:
%    \begin{macrocode}
\ifchilddocmanual
\section*{part `\childdocname'}
\input{\childdocname}
\else
%    \end{macrocode}

% Include the two chapters:
%    \begin{macrocode}
\include{cdocsch1}
\include{cdocsch2}
%    \end{macrocode}

% Include the two parts unless only chapters should be displayed:
%    \begin{macrocode}
\ifchilddoc\else
\section{part three}
\input{cdocspt3}
\section{part four}
\input{cdocspt4}
\fi
%    \end{macrocode}

% Process as usual until here:
%    \begin{macrocode}
\fi
%    \end{macrocode}

% End of document body:
%    \begin{macrocode}
\end{document}
%    \end{macrocode}
%\iffalse
%</samplemain>
%\fi
%
% %%%%%%%%%%%%%%%%%%%%%%%%%%%%%%%%%%%%%%
% \paragraph{Chapter Include Files.}
%
% The include files are called |cdocsch1.tex| and |cdocsch2.tex|.
%
%\iffalse
%<*samplechap1|samplechap2>
%\fi

% Optional override for |\version| flag:
%    \begin{macrocode}
%%\providecommand{\version}{final}
%    \end{macrocode}

% Include the main document:
%    \begin{macrocode}
\input{childdoc.def}
\childdocof{cdocsamp}
%    \end{macrocode}

%\iffalse
%</samplechap1|samplechap2>
%\fi
%
%\iffalse
%<*samplechap1>
%\fi
% Some text for chapter 1:
%    \begin{macrocode}
\section{one}
some text in chapter one
%    \end{macrocode}

%\iffalse
%</samplechap1>
%\fi
% Some text for chapter 2:
%\iffalse
%<*samplechap2>
%\fi
%    \begin{macrocode}
\section{two}
more text in chapter two
%    \end{macrocode}

%\iffalse
%</samplechap2>
%\fi
%
% %%%%%%%%%%%%%%%%%%%%%%%%%%%%%%%%%%%%%%
% \paragraph{Part Include Files.}
%
% The include files are called |cdocspt3.tex| and |cdocspt4.tex|.
%
%\iffalse
%<*samplepart3|samplepart4>
%\fi

% Optional override for |\version| flag:
%    \begin{macrocode}
%%\providecommand{\version}{final}
%    \end{macrocode}

% Include the main document:
%    \begin{macrocode}
\input{childdoc.def}
\childdocby{cdocsamp}
%    \end{macrocode}

%\iffalse
%</samplepart3|samplepart4>
%\fi
%
%\iffalse
%<*samplepart3>
%\fi
% Some text for part 3:
%    \begin{macrocode}
some text in part three
%    \end{macrocode}

%\iffalse
%</samplepart3>
%\fi
% Some text for part 4:
%\iffalse
%<*samplepart4>
%\fi
%    \begin{macrocode}
more text in part four
%    \end{macrocode}

%\iffalse
%</samplepart4>
%\fi
%
% %%%%%%%%%%%%%%%%%%%%%%%%%%%%%%%%%%%%%%
% \paragraph{Forwarding for a Complete Draft.}
%
% The following forwarding file |cdocsdrf.tex|
% compiles the main document in draft mode:
%\iffalse
%<*sampledraft>
%\fi
%    \begin{macrocode}
\def\version{draft}
\input{childdoc.def}
\childdocforward{cdocsamp}
%    \end{macrocode}

%\iffalse
%</sampledraft>
%\fi
%
% %%%%%%%%%%%%%%%%%%%%%%%%%%%%%%%%%%%%%%
% \paragraph{Forwarding for Final Version of the Chapters.}
%
% The following forwarding files |cdocsfn1.tex| and |cdocsfn2.tex|
% (with identical content)
% compile the final versions of the child documents
% |cdocsch1.tex| and |cdocsch2.tex|, respectively:
%\iffalse
%<*samplefinal>
%\fi
%    \begin{macrocode}
\def\version{final}
\input{childdoc.def}
\childdocforwardprefix[cdocsamp]{cdocsfn}{cdocsch}
%    \end{macrocode}

%\iffalse
%</samplefinal>
%\fi
%
% %%%%%%%%%%%%%%%%%%%%%%%%%%%%%%%%%%%%%%
% \paragraph{Command Line Processing.}
%
% The following three command lines generate the output files
% |cdocscld|, |cdocscl1| and |cdocscl2|
% which should be identical to
% |cdocsdrf|, |cdocsch1| and |cdocsfn2|, respectively:
% \begin{center}
% \begin{tabular}{l}
% |latex -jobname cdocscld \|\\
% |  "\def\version{draft}\input{childdoc.def}\childdocforward{cdocsamp}"|\\
% |latex -jobname cdocscl1 \|\\
% |  "\input{childdoc.def}\childdocforward[cdocsamp]{cdocsch1}"|\\
% |latex -jobname cdocscl2 \|\\
% |  "\def\version{final}\input{childdoc.def}\childdocforward{cdocsch2}"|
% \end{tabular}
% \end{center}
% Note that the trailing backslash on each first line
% merely continues the input to the second line
% (for convenient cut ant paste).
% Furthermore, the command |latex| can be replaced by any
% of its alternative versions such as |pdflatex|.
%
% %%%%%%%%%%%%%%%%%%%%%%%%%%%%%%%%%%%%%%%%%%%%%%%%%%%%%%%%%%%%%%%%%%%%%%%%%%%%%%
% %%%%%%%%%%%%%%%%%%%%%%%%%%%%%%%%%%%%%%%%%%%%%%%%%%%%%%%%%%%%%%%%%%%%%%%%%%%%%%
% \section{Implementation}
%\iffalse
%<*package>
%\fi
%
% This section describes the definitions file |childdoc.def|.

% The definitions cannot be loaded using |\usepackage| or |\RequirePackage|
% which has a mechanism to prevent loading a style file more than once.
% When loading the definitions by means of |\input|
% multiple instances have to be prevented manually:
%\iffalse
%This code needs to be before the `\ProvidesFile' directive
%which is defined at the beginning of this file.
%Therefore it is also placed there and commented out here.
%</package>
%<*discard>
%\fi
%    \begin{macrocode}
\ifdefined\childdocmain\endinput\fi
%    \end{macrocode}
%\iffalse
%</discard>
%<*package>
%\fi
%
% \macro{\ifchilddoc}
% \macro{\ifchilddocmanual}
% The conditional |\ifchilddoc| tells whether a
% child (true) or main (false) document is being compiled.
% The conditional |\ifchilddocmanual| tells whether
% the |\includeonly| mechanism is used (false) or
% the selection of child files must be performed manually (true).
% The definitions initialise to false:
%    \begin{macrocode}
\newif\ifchilddoc
\newif\ifchilddocmanual
%    \end{macrocode}

% \macro{\childdocname}
% \macro{\childdocjob}
% The macro |\childdocname| stores the name of the main document
% to be compiled. The macro |\childdocjob| stores the name of
% the document on which the \LaTeX{} compiler was originally invoked.
% The content of |\jobname| cannot be compared
% to filenames specified in the source due to different catcodes.
% The following code rescans |\jobname|, stores the result
% in |\childdocname| and saves a copy in |\childdocjob|:
%    \begin{macrocode}
\edef\childdocname{\scantokens\expandafter{\jobname\noexpand}}
\let\childdocjob\childdocname
%    \end{macrocode}

% \macro{\childdocdisable}
% The macro |\childdocdisable| prevents the main file
% from being processed more than once.
% At this stage, the main document command |\childdocmain|
% is assumed to be called once again where it should do nothing.
% Any subsequent call to it should prevent
% a secondary processing of the main document
% It overwrites the forwarding commands
% |\childdocof| and |\childdocforward|
% with empty macros to prevent further inclusions of the main document:
%    \begin{macrocode}
\newcommand{\childdocdisable}
{
  \renewcommand{\childdocmain}[1]{\renewcommand{\childdocmain}[1]{\endinput}}
  \renewcommand{\childdocof}[1]{}
  \renewcommand{\childdocby}[2][]{}
  \renewcommand{\childdocforward}[2][]{}
  \renewcommand{\childdocdisable}{}
}
%    \end{macrocode}

% \macro{\childdocmain}
% The macro |\childdocmain| is to be called at the top of the main file
% with nothing or the main filename (without extension) as argument.
% First, it breaks loops.
% If the argument is not empty and does not match |\childdocname|
% (which is set by the first inclusion of |childdoc.def|),
% |\ifchilddoc| is set to true, |\includeonly| is applied to the child file
% and |\jobname| is set to the main file
% (for proper handling of |.aux| files):
%    \begin{macrocode}
\newcommand{\childdocmain}[1]
{
  \childdocdisable\childdocmain{}
  \if?#1?\else
    \begingroup
      \def\childdoctmp{#1}
      \ifx\childdoctmp\childdocname
        \def\childdoctmp{}
      \else
        \def\childdoctmp
        {
          \childdoctrue
          \includeonly{\childdocname}
          \def\childdocjob{#1}
          \def\jobname{#1}
        }
      \fi
      \expandafter
    \endgroup
    \childdoctmp
  \fi
}
%    \end{macrocode}

% \macro{\childdocof}
% The command |\childdocof| redirects
% compilation to the main file |#1|.
%    \begin{macrocode}
\newcommand{\childdocof}[1]
{
  \childdocdisable
  \childdoctrue
  \includeonly{\childdocname}
  \def\jobname{#1}
  \def\childdocjob{#1}
  \input{#1}
}
%    \end{macrocode}

% \macro{\childdocby}
% The command |\childdocby| ....
%    \begin{macrocode}
\newcommand{\childdocby}[2][]
{
  \childdocdisable
  \childdoctrue
  \childdocmanualtrue
  \if?#1?\else
    \def\jobname{#2}
  \fi
  \def\childdocjob{#2}
  \input{#2}
  \endinput
}
%    \end{macrocode}

% \macro{\childdocforward}
% The command |\childdocforward| redirects
% compilation to the main file or
% (if the optional argument is given) a child file.
% Parameters are set as if the main file
% or a child file starting with |\childdocof| was compiled.
% Then compilation is handed over to the main file:
%    \begin{macrocode}
\newcommand{\childdocforward}[2][]
{
  \begingroup
    \if?#1?
      \def\childdoctmp
      {
        \def\childdocname{#2}
        \def\childdocjob{#2}
        \def\jobname{#2}
        \input{#2}
        \endinput
      }
    \else
      \def\childdoctmp
      {
        \childdocdisable
        \def\childdocname{#2}
        \childdoctrue
        \includeonly{#2}
        \def\childdocjob{#1}
        \def\jobname{#1}
        \input{#1}
        \endinput
      }
    \fi
    \expandafter
  \endgroup
  \childdoctmp
}
%    \end{macrocode}

% \macro{\childdocforwardprefix}
% The command |\childdocforwardprefix| redirects
% compilation to the main or a child file by means of a pattern.
% The prefix |#1| in the current filename is replaced by |#2|
% and the suffix of the current filename is kept
% (it is assumed that the filename does not contain the substring `|~~~|'
% which is used as a delimiter).
% Compilation is handed over to the new file by |\childdocforward|:
%    \begin{macrocode}
\newcommand{\childdocforwardprefix}[3][]
{
  \begingroup
    \def\childdocextract #2##1~~~{\def\childdoctmp{\childdocforward[#1]{#3##1}}}
    \expandafter\childdocextract\childdocname~~~
    \expandafter
  \endgroup
  \childdoctmp
}
%    \end{macrocode}

% \macro{\childdoc}
% The deprecated macro |\childdoc| is a legacy version of |\childdocmain|:
%    \begin{macrocode}
\newcommand{\childdoc}{\childdocmain}
%    \end{macrocode}

% \macro{\childdocredirect}
% The deprecated macro |\childdocredirect| is a legacy version
% of |\childdocforward| and |\childdocforwardprefix|:
%    \begin{macrocode}
\newcommand{\childdocredirect}[2][]
{
  \begingroup
    \if?#1?
      \def\childdoctmp{\childdocforward{#2}}
    \else
      \def\childdoctmp{\childdocforwardprefix{#1}{#2}}
    \fi
    \expandafter
  \endgroup
  \childdoctmp
}
%    \end{macrocode}

%\iffalse
%</package>
%\fi
%
\endinput
\childdocforward[|\textit{main}|]{|\textit{dest}|}"|
\end{center}
%
Here \textit{target} is the name of the output file,
\textit{main} is the name of the main file
and \textit{dest} is the name of the main or child file to be processed
(all filenames without extensions).
The optional argument \textit{main} can be omitted
if \textit{main} matches \textit{dest}.
Optionally, compilation \textit{flags} can be defined via |\def| commands.
This command line makes the \TeX{} engine believe
it is compiling the file \textit{target}
whose content is specified as the latter parameter.
The provided code then forwards the processing to
\textit{main} or \textit{dest} as described in \secref{sec:forward}.

%%%%%%%%%%%%%%%%%%%%%%%%%%%%%%%%%%%%%%%%%%%%%%%%%%%%%%%%%%%%%%%%%%%%%%%%%%%%%%%%
\subsection{Include by Input}
\label{sec:input}

Including child documents by |\include| has some restrictions by design.
Most notably, the content of a child document always occupies
its own set of pages; pages cannot be shared between child documents.
Usually, this behaviour makes perfect sense
because each child document contain an essential part of the document.
However, in some situations it may be desirable to compose
a document from a collection of parts
without having mandatory page breaks between then.
For this case, the package
provides a mechanism to include parts
by |\input| which can also be processed individually.
However, by construction this mechanism
requires manual handling of the content to be output.

%%%%%%%%%%%%%%%%%%%%%%%%%%%%%%%%%%%%%%%%
\DescribeMacro{\ifchilddocmanual}
The main file should be prepared as usual, see \secref{sec:include}.
However, the document body must make a distinction
between processing of an individual part and of the main document, e.g.:
%
\begin{center}
\begin{tabular}{l}
|\ifchilddocmanual|\\
|\input{\childdocname}|\\
|\||else|\\
\textit{document body with }|\input{|\textit{part}|}|\\
|\||fi|
\end{tabular}
\end{center}
%
The conditional |\ifchilddocmanual| is true whenever
a part to be included by |\input| is being compiled,
and the name of the part is stored in |\childdocname|.

%%%%%%%%%%%%%%%%%%%%%%%%%%%%%%%%%%%%%%%%
\DescribeMacro{\childdocby}
Each part to be included by |\input| should start with:
%
\begin{center}
\begin{tabular}{l}
|% \iffalse
%
% childdoc.dtx Copyright (C) 2017-2018 Niklas Beisert
%
% This work may be distributed and/or modified under the
% conditions of the LaTeX Project Public License, either version 1.3
% of this license or (at your option) any later version.
% The latest version of this license is in
%   http://www.latex-project.org/lppl.txt
% and version 1.3 or later is part of all distributions of LaTeX
% version 2005/12/01 or later.
%
% This work has the LPPL maintenance status `maintained'.
%
% The Current Maintainer of this work is Niklas Beisert.
%
% This work consists of the files childdoc.dtx and childdoc.ins
% and the derived files childdoc.def and cdocsamp.tex with
% cdocsch1.tex, cdocsch2.tex, cdocsdrf.tex, cdocsfn1.tex, cdocsfn2.tex.
%
%<package>\ifdefined\childdocmain\endinput\fi
%<package>\ProvidesFile{childdoc.def}[2018/12/30 v2.0 child document driver]
%<samplemain>\ProvidesFile{cdocsamp.tex}[2018/12/30 v2.0 sample for childdoc]
%<*driver>
%\ProvidesFile{childdoc.drv}[2018/12/30 v2.0 childdoc reference manual file]
\PassOptionsToClass{10pt,a4paper}{article}
\documentclass{ltxdoc}

\usepackage[margin=35mm]{geometry}
\usepackage{hyperref}
\usepackage{hyperxmp}
\usepackage[usenames]{color}

\hypersetup{colorlinks=true}
\hypersetup{pdfstartview=FitH}
\hypersetup{pdfpagemode=UseNone}
\hypersetup{pdfsource={}}
\hypersetup{pdflang={en-UK}}
\hypersetup{pdfcopyright={Copyright 2017-2018 Niklas Beisert.
  This work may be distributed and/or modified under the
  conditions of the LaTeX Project Public License, either version 1.3
  of this license or (at your option) any later version.}}
\hypersetup{pdflicenseurl={http://www.latex-project.org/lppl.txt}}
\hypersetup{pdfcontactaddress={ETH Zurich, ITP, HIT K,
  Wolfgang-Pauli-Strasse 27}}
\hypersetup{pdfcontactpostcode={8093}}
\hypersetup{pdfcontactcity={Zurich}}
\hypersetup{pdfcontactcountry={Switzerland}}
\hypersetup{pdfcontactemail={nbeisert@itp.phys.ethz.ch}}
\hypersetup{pdfcontacturl={http://people.phys.ethz.ch/\xmptilde nbeisert/}}

\newcommand{\secref}[1]{\hyperref[#1]{section \ref*{#1}}}

\parskip1ex
\parindent0pt
\let\olditemize\itemize
\def\itemize{\olditemize\parskip0pt}

\begin{document}

\title{The \textsf{childdoc} Package}
\hypersetup{pdftitle={The childdoc Package}}
\author{Niklas Beisert\\[2ex]
  Institut f\"ur Theoretische Physik\\
  Eidgen\"ossische Technische Hochschule Z\"urich\\
  Wolfgang-Pauli-Strasse 27, 8093 Z\"urich, Switzerland\\[1ex]
  \href{mailto:nbeisert@itp.phys.ethz.ch}
  {\texttt{nbeisert@itp.phys.ethz.ch}}}
\hypersetup{pdfauthor={Niklas Beisert}}
\hypersetup{pdfsubject={Manual for the LaTeX2e Package childdoc}}
\date{30 December 2018, \textsf{v2.0}}
\maketitle

\begin{abstract}\noindent
\textsf{childdoc} is a \LaTeXe{} package
that enables the direct compilation
of document sections included by |\include|
to individual files.
\end{abstract}

\begingroup
\parskip0ex
\tableofcontents
\endgroup

%%%%%%%%%%%%%%%%%%%%%%%%%%%%%%%%%%%%%%%%%%%%%%%%%%%%%%%%%%%%%%%%%%%%%%%%%%%%%%%%
%%%%%%%%%%%%%%%%%%%%%%%%%%%%%%%%%%%%%%%%%%%%%%%%%%%%%%%%%%%%%%%%%%%%%%%%%%%%%%%%
\section{Introduction}

\LaTeX{} provides a mechanism to structure a large document (such as a book)
into a main file and several child files (containing the chapters)
using the |\include| command.
This mechanism is beneficial for documents
which span hundreds of pages in order to
make the source file(s) more manageable.
Moreover, compilation can be restricted to
selected child files by means of the |\includeonly| command.
The latter feature can be used to reduce the compilation time while editing
(this was significantly more useful in the earlier days of \LaTeX{})
or to generate a smaller document which is easier to navigate.
Another application of |\includeonly| is to generate
documents consisting of selected parts of the complete document.

However, there are a few drawbacks of the plain |\include| mechanism:
\begin{itemize}
\item
The child files cannot be compiled on their own,
they can only be compiled via the main file.
A naive editing environment
(such as a text editor with an option
to have the current file processed by \LaTeX)
may require one to switch to the main file before compiling;
attempting to compile the child file produces errors.
\item
The main file must be modified (each time)
to adjust the |\includeonly| command
to the present needs. This easily leaves the main file in a messy state.
\item
The generated document will always carry the filename
of the main document. This is inconvenient if
several child files are to be compiled and
to be kept for distribution.
\end{itemize}

The present package provides a simple interface
to make child files individually compilable by \LaTeX{}.
Compiling a child file then has the same effect as compiling
the main file with an |\includeonly| command
to select the appropriate child.
Moreover the generated document will carry the name of the child
rather than the main file.
This resolves all three above issues.

This feature is meant to make the editing of books,
thesis documents and lecture notes somewhat more convenient.
However, the package can also be used efficiently for
composing a series of documents (such as exercise sheets)
which are typically distributed individually.
It then assists the author in generating the individual documents
(potentially in different versions)
as well as a document containing the collected series.
Another application is in developing style files
or other kinds of included material
where compilation of the style file could redirect
to a sample or test file.

%%%%%%%%%%%%%%%%%%%%%%%%%%%%%%%%%%%%%%%%%%%%%%%%%%%%%%%%%%%%%%%%%%%%%%%%%%%%%%%%
%%%%%%%%%%%%%%%%%%%%%%%%%%%%%%%%%%%%%%%%%%%%%%%%%%%%%%%%%%%%%%%%%%%%%%%%%%%%%%%%
\section{Usage}

First of all, the package \textsf{childdoc} is \emph{not} a standard
\LaTeXe{} |.sty| style file! Therefore it needs to be invoked in
a non-standard way.

%%%%%%%%%%%%%%%%%%%%%%%%%%%%%%%%%%%%%%%%%%%%%%%%%%%%%%%%%%%%%%%%%%%%%%%%%%%%%%%%
\subsection{Included Files}
\label{sec:include}

%%%%%%%%%%%%%%%%%%%%%%%%%%%%%%%%%%%%%%%%
\DescribeMacro{\childdocmain}
To use the package, add the commands
\begin{center}
\begin{tabular}{l}
|\input{childdoc.def}|\\
|\childdocmain{}|\\
\end{tabular}
\end{center}
at the very top of the main \LaTeX{} file,
in particular \emph{before} the |\documentclass| statement!
The argument of |\childdocmain| should be left empty
(but it must be present).

%%%%%%%%%%%%%%%%%%%%%%%%%%%%%%%%%%%%%%%%
\DescribeMacro{\childdocof}
Furthermore, add the commands
\begin{center}
\begin{tabular}{l}
|\input{childdoc.def}|\\
|\childdocof{|\textit{main}|}|\\
\end{tabular}
\end{center}
at the top of every child file \textit{child}
which is included by |\include{|\textit{child}|}|
from within the main file
(or at least for those files to be compiled individually).
The argument \textit{main} must be the filename of the main file.

There are a couple of
considerations in setting up the main and child documents:

%%%%%%%%%%%%%%%%%%%%%%%%%%%%%%%%%%%%%%%%
\paragraph{Restrictions.}

Please note the following restrictions:
\begin{itemize}
\item
|\childdocmain| must be called with one argument \textit{main}
to ensure compatibility with earlier version of the package.
It must either be empty (|\childdocmain{}|)
or precisely match the filename of the main file in which it is specified.
See \secref{sec:detection} for further information.
\item
The filename \textit{main} must be specified without the |.tex| extension.
\item
The filename \textit{main} is case sensitive
(even in case-insensitive file systems)
due to internal string comparison.
\item
The argument \textit{main} should be fully expanded, it cannot be a macro.
\item
Subdirectories and special characters should be avoided in filenames.
\item
The command |\childdocmain{|\textit{main}|}| must be followed by a whitespace.
It should not be followed immediately by another command
or by a comment mark `|%|'.
This is because the \TeX{} parser reads the token immediately following
the argument of |\childdocmain| and puts it
at the beginning of every child section;
however, a white\-space is ignored.
\end{itemize}

%%%%%%%%%%%%%%%%%%%%%%%%%%%%%%%%%%%%%%%%
\paragraph{Content of Main File.}

It is advisable to place all content in the child files included by |\include|.
Any output contained in the main file will appear in all child documents
unless suppressed manually;
it cannot be suppressed automatically by the |\includeonly| directive
and thus should normally be avoided.
A method to include some content in the main file
by means of conditional processing is described in \secref{sec:conditional}.

%%%%%%%%%%%%%%%%%%%%%%%%%%%%%%%%%%%%%%%%
\paragraph{Page Numbering.}

When only a part of the document is compiled,
the appropriate numbering of pages
(as well as other status parameters)
is determined from the |.aux| files.
The latter contain information from previous passes.
However this information needs to propagate through
all intermediate child documents.
Therefore the page numbering in child documents may well
be inconsistent until the complete document is compiled at least once.

A useful (if unconventional) way to always ensure a consistent
page numbering is to restart the numbering in each child document
and denote the pages by `\textit{child}|.|\textit{page}'
where \textit{child} represents the chapter/section number of the child file.
This can be achieved by the command
|\numberwithin{page}{|\textit{child}|}|
of the \textsf{amsmath} package
where \textit{child} can be |chapter| or |section|
depending on the chosen structuring.
Alternatively, one can modify the macro |\thepage| appropriately
and reset the counter |page| at the start of each child file.

%%%%%%%%%%%%%%%%%%%%%%%%%%%%%%%%%%%%%%%%%%%%%%%%%%%%%%%%%%%%%%%%%%%%%%%%%%%%%%%%
\subsection{Conditional Processing}
\label{sec:conditional}

The package provides a mechanism to compile different versions
of a document. To customise the versions further some conditional processing
can come in handy to distinguish which version is being compiled.
The package provides two macros to describe the compilation context:

%%%%%%%%%%%%%%%%%%%%%%%%%%%%%%%%%%%%%%%%
\DescribeMacro{\ifchilddoc}
The conditional |\ifchilddoc| distinguishes between the compilation of
child documents and the main document:
%
\begin{center}
|\ifchilddoc |\textit{child-code}| |[|\||else |\textit{main-code}]| \||fi|
\end{center}

%%%%%%%%%%%%%%%%%%%%%%%%%%%%%%%%%%%%%%%%
\DescribeMacro{\childdocname}
\DescribeMacro{\childdocjob}
The macro |\childdocname| contains the filename (without extension)
of the main or child file being processed.
Note that |\childdocjob| will always contain the name of the main file.

%%%%%%%%%%%%%%%%%%%%%%%%%%%%%%%%%%%%%%%%
\paragraph{Title Page.}

Conditional processing can be used to include a title or banner page
in the main document when proper precautions are taken.
Importantly, the code in the main file should ensure that the page counter
(as well as other status parameters which are stored in the |.aux| files)
takes the same value after the conditional processing.
Otherwise the page numbers may take divergent values
depending on which part is compiled.

For example, a title page could be declared by:
%
\begin{center}
\begin{tabular}{l}
|\ifchilddoc\||else|\\
|\addtocounter{page}{-1}|\\
\textit{code for title page}\\
|\newpage|\\
|\||fi|
\end{tabular}
\end{center}
%
A banner page for the child documents can be generated by:
%
\begin{center}
\begin{tabular}{l}
|\ifchilddoc|\\
|\addtocounter{page}{-1}|\\
\textit{code for banner page}\\
|\newpage|\\
|\||fi|
\end{tabular}
\end{center}
%
Here one could write a message such as:
\begin{center}
|This is the part \childdocname{} of \childdocjob{}.|
\end{center}

%%%%%%%%%%%%%%%%%%%%%%%%%%%%%%%%%%%%%%%%%%%%%%%%%%%%%%%%%%%%%%%%%%%%%%%%%%%%%%%%
\subsection{Flags}
\label{sec:flags}

The package makes it easy to generate different versions
of the main or child documents.
To this end compilation flags can be defined
and assigned different default values.
They will be particularly useful in conjunction
with the forwarding mechanism described in \secref{sec:forward}.

For example, it may be useful to have a flag |\version|
which can be set to |draft| or |final|.
The document source will contain some conditional code
depending on the value of |\version|.
Suppose further, the flag should default to |final| for the main file
and to |draft| for child files
which is a natural assignment for editing the document.
This is achieved by placing the following code
in the preamble of the main document
(below the |\childdocmain| directive):
%
\begin{center}
\begin{tabular}{l}
|\ifchilddoc|\\
|\providecommand{\version}{draft}|\\
|\||else|\\
|\providecommand{\version}{final}|\\
|\||fi|
\end{tabular}
\end{center}
%
The definition by |\providecommand| makes sure
that previous definitions are not overwritten.
Further statements |\providecommand{\version}{...}|
can thus be added before the above code to override it.

For the main file, one might add a line
(between |\childdocmain| and the above block)
%
\begin{center}
|%\ifchilddoc\||else\providecommand{\version}{draft}\||fi|
\end{center}
%
which can be uncommented to produce a draft version.
Likewise one can add a line to the very top of a child file
(above the |\childdocof{|\textit{main}|}| directive)
%
\begin{center}
|%\providecommand{\version}{final}|
\end{center}
%
which can be uncommented to produce the final version of this child document.

%%%%%%%%%%%%%%%%%%%%%%%%%%%%%%%%%%%%%%%%%%%%%%%%%%%%%%%%%%%%%%%%%%%%%%%%%%%%%%%%
\subsection{Forwarding}
\label{sec:forward}

Different versions of the main or child documents
using compilation flags as described in \secref{sec:flags}
can be (permanently) stored in different files
for convenient compilation, viewing and distribution.
To this end, the package defines a command
to pass on compilation to a different file:

%%%%%%%%%%%%%%%%%%%%%%%%%%%%%%%%%%%%%%%%
\DescribeMacro{\childdocforward}
The command |\childdocforward| redirects processing to
another source file:
%
\begin{center}
\begin{tabular}{l}
|\input{childdoc.def}|\\
|\childdocforward[|\textit{main}|]{|\textit{dest}|}|\\
\end{tabular}
\end{center}
%
The argument \textit{dest} is the destination file
(without extension).
It should be the main file or one of the child files.
Note that further \textsf{childdoc} directives
such as |\childdocof| and |\childdocforward|
in the indicated file will be processed in this form.
The optional argument \textit{main}
passes on directly to the main file \textit{main}
while pretending to compile the child \textit{dest}.
This form behaves as if \textit{dest}
issues |\childdocof{|\textit{main}|}| right away,
and no further \textsf{childdoc} directives will be processed.

%%%%%%%%%%%%%%%%%%%%%%%%%%%%%%%%%%%%%%%%
\DescribeMacro{\...prefix}
In the alternative form |\childdocforwardprefix|,
%
\begin{center}
\begin{tabular}{l}
|\input{childdoc.def}|\\
|\childdocforwardprefix[|\textit{main}|]{|\textit{prefix}|}{|\textit{dest}|}|
\end{tabular}
\end{center}
%
the destination file is determined by a pattern
depending on the current file:
To make this work, the current file must be called
`{\textit{prefix}\hspace{0.2em}\textit{suffix}}'
with \textit{prefix} matching precisely the argument.
Processing is then passed on to the file
`{\textit{dest}\hspace{0.2em}\textit{suffix}}'.
Surely, the same effect is achieved by
directly specifying the
argument `{\textit{dest}\hspace{0.2em}\textit{suffix}}'
in the first form.
However, that requires to set up a different file
for each child. With the alternative form of the command
all these files can have exactly the same content
which simplifies setting them up and maintaining them.

For example, the following file |draft.tex|
with a compilation flag |\version| as described in \secref{sec:flags}
compiles the main document as a draft:
%
\begin{center}
\begin{tabular}{l}
|\def\version{draft}|\\
|\input{childdoc.def}|\\
|\childdocforward{|\textit{main}|}|
\end{tabular}
\end{center}
%
Likewise, the following files |final|\textit{nn}|.tex|
compile the final version of the child document
|child|\textit{nn}|.tex|:
%
\begin{center}
\begin{tabular}{l}
|\def\version{final}|\\
|\input{childdoc.def}|\\
|\childdocforwardprefix{final}{child}|
\end{tabular}
\end{center}
%

Note that when several versions of a main file and/or of each child file
are to be generated, it may be convenient to set up a |Makefile| or
shell script to automatise the process.

%%%%%%%%%%%%%%%%%%%%%%%%%%%%%%%%%%%%%%%%%%%%%%%%%%%%%%%%%%%%%%%%%%%%%%%%%%%%%%%%
\subsection{Command Line Processing}
\label{sec:commandline}

The effect of redirection files can also be achieved by invoking
the \LaTeX{} compiler with a more elaborate command line.
Most conveniently this should be done as part
of a shell script or a |Makefile|.

When using \textsf{childdoc} in the main file, the following
command lines effectively perform a redirection
(note that depending on the shell being used,
backslashes may have to be doubled: `|\|' $\to$ `|\\|'):
%
\begin{center}
|... -jobname "|\textit{target}|" |\\|"|[\textit{flags}]%
|\input{childdoc.def}\childdocforward[|\textit{main}|]{|\textit{dest}|}"|
\end{center}
%
Here \textit{target} is the name of the output file,
\textit{main} is the name of the main file
and \textit{dest} is the name of the main or child file to be processed
(all filenames without extensions).
The optional argument \textit{main} can be omitted
if \textit{main} matches \textit{dest}.
Optionally, compilation \textit{flags} can be defined via |\def| commands.
This command line makes the \TeX{} engine believe
it is compiling the file \textit{target}
whose content is specified as the latter parameter.
The provided code then forwards the processing to
\textit{main} or \textit{dest} as described in \secref{sec:forward}.

%%%%%%%%%%%%%%%%%%%%%%%%%%%%%%%%%%%%%%%%%%%%%%%%%%%%%%%%%%%%%%%%%%%%%%%%%%%%%%%%
\subsection{Include by Input}
\label{sec:input}

Including child documents by |\include| has some restrictions by design.
Most notably, the content of a child document always occupies
its own set of pages; pages cannot be shared between child documents.
Usually, this behaviour makes perfect sense
because each child document contain an essential part of the document.
However, in some situations it may be desirable to compose
a document from a collection of parts
without having mandatory page breaks between then.
For this case, the package
provides a mechanism to include parts
by |\input| which can also be processed individually.
However, by construction this mechanism
requires manual handling of the content to be output.

%%%%%%%%%%%%%%%%%%%%%%%%%%%%%%%%%%%%%%%%
\DescribeMacro{\ifchilddocmanual}
The main file should be prepared as usual, see \secref{sec:include}.
However, the document body must make a distinction
between processing of an individual part and of the main document, e.g.:
%
\begin{center}
\begin{tabular}{l}
|\ifchilddocmanual|\\
|\input{\childdocname}|\\
|\||else|\\
\textit{document body with }|\input{|\textit{part}|}|\\
|\||fi|
\end{tabular}
\end{center}
%
The conditional |\ifchilddocmanual| is true whenever
a part to be included by |\input| is being compiled,
and the name of the part is stored in |\childdocname|.

%%%%%%%%%%%%%%%%%%%%%%%%%%%%%%%%%%%%%%%%
\DescribeMacro{\childdocby}
Each part to be included by |\input| should start with:
%
\begin{center}
\begin{tabular}{l}
|\input{childdoc.def}|\\
|\childdocby{|\textit{main}|}|\\
\end{tabular}
\end{center}
%
The directive |\childdocby| is similar to |\childdocof|
described in \secref{sec:include},
but the subsequent selection of content must be done manually.
To that end, both |\ifchilddoc| and |\ifchilddocmanual|
will be true upon processing of a part,
and the name of the part is stored in |\childdocname|.
Note that |\jobname| will be set to the filename of the current part
so that each part receives an individual |.aux| file
that does not interfere with the |.aux| file(s) of the main document.
This behaviour can be altered by the alternative form
|\childdocby[*]{|\textit{main}|}| (with a non-empty optional argument)
which uses the |.aux| file of the main document
by setting |\jobname| to \textit{main}.

%%%%%%%%%%%%%%%%%%%%%%%%%%%%%%%%%%%%%%%%%%%%%%%%%%%%%%%%%%%%%%%%%%%%%%%%%%%%%%%%
\subsection{Driver Development}
\label{sec:driver}

The \textsf{childdoc} mechanism can also be use for the development
of definition files such as \LaTeX{} styles or classes.
This case differs from the above setup with multiple parts
included by |\include| in that no |\includeonly| should be invoked.
This can be achieved by starting the include file
(before |\ProvidesPackage|) with:
%
\begin{center}
\begin{tabular}{l}
|\input{childdoc.def}|\\
|\childdocforward{|\textit{main}|}|\\
\end{tabular}
\end{center}
%
or alternatively with:
%
\begin{center}
\begin{tabular}{l}
|\input{childdoc.def}|\\
|\childdocby{|\textit{main}|}|\\
\end{tabular}
\end{center}
%
Both forms have slightly different effects as described above.
The main file is prepared as usual, see \secref{sec:include}.

%%%%%%%%%%%%%%%%%%%%%%%%%%%%%%%%%%%%%%%%%%%%%%%%%%%%%%%%%%%%%%%%%%%%%%%%%%%%%%%%
\subsection{Legacy Detection}
\label{sec:detection}

The directive |\childdocmain| in the main file can detect
whether the complete document or merely a child is to be compiled
even without using the directive |\childdocof|.
This method is deprecated because it is less robust
and there is no compelling reason to use it;
it is merely provided for backward compatibility
and it may be removed in future versions.

If the detection mechanism is to be used,
it is mandatory to correctly specify
the filename of the main file as the argument of |\childdocmain|:
%
\begin{center}
\begin{tabular}{l}
|\input{childdoc.def}|\\
|\childdocmain{|\textit{main}|}|\\
\end{tabular}
\end{center}
%
If |\jobname| does not match the argument \textit{main} of |\childdocmain|,
it is assumed that |\jobname| points to the child file to be compiled.
When using |\childdocmain| with the main file specified as argument,
it suffices to start a child file
with just |\input{|\textit{main}|}|
without loading of the package and using |\childdocof|.
If instead all processing is done
with the appropriate \textsf{childdoc} directives,
the argument of \textit{main} of |\childdocmain| can be empty.

An alternative version of the command line processing described
in \secref{sec:commandline} using the detection mechanism reads:
%
\begin{center}
|... -jobname "|\textit{target}|" "|[\textit{flags}]%
[|\def\jobname{|\textit{dest}|}|]|\input{|\textit{main}|}"|
\end{center}

%%%%%%%%%%%%%%%%%%%%%%%%%%%%%%%%%%%%%%%%%%%%%%%%%%%%%%%%%%%%%%%%%%%%%%%%%%%%%%%%
\subsection{Manual Code}
\label{sec:manual}

In case one cannot be certain whether the definitions file |childdoc.def|
is installed on the target \TeX{} distribution
and one prefers not to ship it,
it is conceivable to paste a few relevant commands into the sources.

To that end, drop all statements |\input{childdoc.def}|
and perform the replacements as outlined below.
Instead of |\childdocmain{|\textit{main}|}| add the following code
to the top of the main file:
%
\begin{center}
\begin{tabular}{l}
|\||ifdefined\childdocname\endinput\||fi\newif\ifchilddoc|\\
|\edef\childdocname{\scantokens\expandafter{\jobname\noexpand}}|\\
|\def\childdocmain{|\textit{main}|}\||ifx\childdocmain\childdocname\||else|\\
|\childdoctrue\includeonly{\childdocname}\let\jobname\childdocmain\||fi|\\
\end{tabular}
\end{center}
%
Instead of |\childdocof{|\textit{main}|}| just include the main file
at the top of each child file:
%
\begin{center}
|\input{|\textit{main}|}|
\end{center}
%
A simple redirection |\childdocforward{|\textit{dest}|}| is achieved by:
%
\begin{center}
|\def\jobname{|\textit{dest}|}\input{\jobname}|
\end{center}
%
The redirection with prefix
|\childdocforwardprefix[|\textit{prefix}|]{|\textit{dest}|}|
is accomplished by:
%
\begin{center}
\begin{tabular}{l}
|{\edef\jobname{\scantokens\expandafter{\jobname\noexpand}}|\\
|\def\redirectjob |\textit{prefix}|#1~~~{\gdef\jobname{|\textit{dest}|#1}}|\\
|\expandafter\redirectjob\jobname~~~}\input{\jobname}|
\end{tabular}
\end{center}

In an alternative approach,
child documents can be compiled by a specific command line
without additional code or specific definitions:
%
\begin{center}
|... -jobname "|\textit{target}|" "|[\textit{flags}]%
|\includeonly{|\textit{dest}|}\input{|\textit{main}|}"|
\end{center}
%

%%%%%%%%%%%%%%%%%%%%%%%%%%%%%%%%%%%%%%%%%%%%%%%%%%%%%%%%%%%%%%%%%%%%%%%%%%%%%%%%
%%%%%%%%%%%%%%%%%%%%%%%%%%%%%%%%%%%%%%%%%%%%%%%%%%%%%%%%%%%%%%%%%%%%%%%%%%%%%%%%
\section{Information}

%%%%%%%%%%%%%%%%%%%%%%%%%%%%%%%%%%%%%%%%%%%%%%%%%%%%%%%%%%%%%%%%%%%%%%%%%%%%%%%%
\subsection{Copyright}

Copyright \copyright{} 2017--2018 Niklas Beisert

This work may be distributed and/or modified under the
conditions of the \LaTeX{} Project Public License, either version 1.3
of this license or (at your option) any later version.
The latest version of this license is in
  \url{http://www.latex-project.org/lppl.txt}
and version 1.3 or later is part of all distributions of \LaTeX{}
version 2005/12/01 or later.

This work has the LPPL maintenance status `maintained'.

The Current Maintainer of this work is Niklas Beisert.

This work consists of the files |README.txt|, |childdoc.ins| and |childdoc.dtx|
as well as the derived files |childdoc.def|, |cdocsamp.tex|
with |cdocsch1.tex|, |cdocsch2.tex|, |cdocspt3.tex|, |cdocspt4.tex|,
|cdocsdrf.tex|, |cdocsfn1.tex|, |cdocsfn2.tex|
as well as |childdoc.pdf|.

%%%%%%%%%%%%%%%%%%%%%%%%%%%%%%%%%%%%%%%%%%%%%%%%%%%%%%%%%%%%%%%%%%%%%%%%%%%%%%%%
\subsection{Files and Installation}

The package consists of the files:
%
\begin{center}
\begin{tabular}{ll}
    |README.txt|   & readme file \\
    |childdoc.ins| & installation file \\
    |childdoc.dtx| & source file \\
    |childdoc.def| & definition file \\
    |cdocsamp.tex| & sample main file \\
    |cdocsch1.tex| & sample include file \\
    |cdocsch2.tex| & sample include file \\
    |cdocspt3.tex| & sample part file \\
    |cdocspt4.tex| & sample part file \\
    |cdocsdrf.tex| & sample redirection file \\
    |cdocsfn1.tex| & sample redirection file \\
    |cdocsfn2.tex| & sample redirection file \\
    |childdoc.pdf| & manual
\end{tabular}
\end{center}
%
The distribution consists of the files
|README.txt|, |childdoc.ins| and |childdoc.dtx|.
%
\begin{itemize}
\item
Run (pdf)\LaTeX{} on |childdoc.dtx|
to compile the manual |childdoc.pdf| (this file).
\item
Run \LaTeX{} on |childdoc.ins| to create the definitions file |childdoc.def|
and the sample |cdocsamp.tex| with include files
|cdocsch1.tex|, |cdocsch2.tex|, |cdocspt3.tex|, |cdocspt4.tex|,
|cdocsdrf.tex|, |cdocsfn1.tex|, |cdocsfn2.tex|.
Then copy the file |childdoc.def| to an appropriate directory of your \LaTeX{}
distribution, e.g.\ \textit{texmf-root}|/tex/latex/childdoc|.
\end{itemize}

%%%%%%%%%%%%%%%%%%%%%%%%%%%%%%%%%%%%%%%%%%%%%%%%%%%%%%%%%%%%%%%%%%%%%%%%%%%%%%%%
\subsection{Related CTAN Packages}

There are several other packages which offer a similar functionality:
%
\begin{itemize}
\item
The packages
\href{http://ctan.org/pkg/docmute}{\textsf{docmute}},
\href{http://ctan.org/pkg/includex}{\textsf{includex}} and
\href{http://ctan.org/pkg/standalone}{\textsf{standalone}}
provide commands to include only the document body of
a child file thus allowing both files to be compiled individually.
\item
The packages \href{http://ctan.org/pkg/subdocs}{\textsf{subdocs}}
and \href{http://ctan.org/pkg/subfiles}{\textsf{subfiles}}
provide structures in which the main and child documents can be
encapsulated and allowing them to be compiled individually.
The inclusion mechanism is different from the conventional |\include|.
\item
The package \href{http://ctan.org/pkg/combine}{\textsf{combine}}
is an elaborate solution to combine several documents into one.
\end{itemize}
%
See also the CTAN topic \href{http://ctan.org/topic/subdocs}{\textsf{subdocs}}
for further related packages.
The present package differs from the above solutions in that
a document structure constructed with the conventional |\include| mechanism
just needs two extra commands at the top of every file
such that all constituent files can be compiled individually.

%%%%%%%%%%%%%%%%%%%%%%%%%%%%%%%%%%%%%%%%%%%%%%%%%%%%%%%%%%%%%%%%%%%%%%%%%%%%%%%%
%\subsection{Feature Suggestions}
%
%The following is a list of features which may be useful for future
%versions of this package:
%%
%\begin{itemize}
%\item
%\ldots
%\end{itemize}

%%%%%%%%%%%%%%%%%%%%%%%%%%%%%%%%%%%%%%%%%%%%%%%%%%%%%%%%%%%%%%%%%%%%%%%%%%%%%%%%
\subsection{Revision History}

%%%%%%%%%%%%%%%%%%%%%%%%%%%%%%%%%%%%%%%%
\paragraph{v2.0:} 2018/12/30

\begin{itemize}
\item
immediate forward processing
\item
added |\childdocby| mechanism
\item
manual restructured
\end{itemize}

%%%%%%%%%%%%%%%%%%%%%%%%%%%%%%%%%%%%%%%%
\paragraph{v1.6:} 2018/01/17

\begin{itemize}
\item
application for development of include files
\item
corrections to manual
\end{itemize}

%%%%%%%%%%%%%%%%%%%%%%%%%%%%%%%%%%%%%%%%
\paragraph{v1.5:} 2017/05/21

\begin{itemize}
\item
more complete structuring introduced
\item
|\childdocof| introduced
\item
|\childdoc| renamed to |\childdocmain|
\item
|\childredirect| renamed to |\childdocforward| and |\childdocforwardprefix|
and functionality expanded
\end{itemize}

%%%%%%%%%%%%%%%%%%%%%%%%%%%%%%%%%%%%%%%%
\paragraph{v1.0:} 2017/04/27

\begin{itemize}
\item
manual and install package
\item
first version published on CTAN
\end{itemize}

%%%%%%%%%%%%%%%%%%%%%%%%%%%%%%%%%%%%%%%%
\paragraph{v0.6:} 2017/04/26

\begin{itemize}
\item
redirection mechanism added
\end{itemize}

%%%%%%%%%%%%%%%%%%%%%%%%%%%%%%%%%%%%%%%%
\paragraph{v0.5:} 2017/04/26

\begin{itemize}
\item
functionality in definition file
\end{itemize}


%%%%%%%%%%%%%%%%%%%%%%%%%%%%%%%%%%%%%%%%%%%%%%%%%%%%%%%%%%%%%%%%%%%%%%%%%%%%%%%%
%%%%%%%%%%%%%%%%%%%%%%%%%%%%%%%%%%%%%%%%%%%%%%%%%%%%%%%%%%%%%%%%%%%%%%%%%%%%%%%%
%%%%%%%%%%%%%%%%%%%%%%%%%%%%%%%%%%%%%%%%%%%%%%%%%%%%%%%%%%%%%%%%%%%%%%%%%%%%%%%%
\appendix

\settowidth\MacroIndent{\rmfamily\scriptsize 000\ }

 \DocInput{childdoc.dtx}

\end{document}
%</driver>
% \fi
%
% %%%%%%%%%%%%%%%%%%%%%%%%%%%%%%%%%%%%%%%%%%%%%%%%%%%%%%%%%%%%%%%%%%%%%%%%%%%%%%
% %%%%%%%%%%%%%%%%%%%%%%%%%%%%%%%%%%%%%%%%%%%%%%%%%%%%%%%%%%%%%%%%%%%%%%%%%%%%%%
% \section{Sample}
%\iffalse
%<*samplemain>
%\fi
%
% The following presents a sample document
% with two chapters, two parts, a title page,
% a compile flag as well as three forwarding files to set the flag.
% It consists of eight |.tex| files:
% \begin{center}
% \begin{tabular}{ll}
% |cdocsamp.tex|&main file\\
% |cdocsch1.tex|&include file for chapter 1\\
% |cdocsch2.tex|&include file for chapter 2\\
% |cdocspt3.tex|&include file for part 3\\
% |cdocspt4.tex|&include file for part 4\\
% |cdocsdrf.tex|&forwarding file for main file in draft mode\\
% |cdocsfi1.tex|&forwarding file for final version of chapter 1\\
% |cdocsfi2.tex|&forwarding file for final version of chapter 2\\
% \end{tabular}
% \end{center}
% Each of the eight files can be compiled directly by the \LaTeX{} compiler.
%
% %%%%%%%%%%%%%%%%%%%%%%%%%%%%%%%%%%%%%%
% \paragraph{Main File.}
%
% The main file is called |cdocsamp.tex|.
%
% Load the \textsf{childdoc} definitions and
% declare the filename for the main document:
%    \begin{macrocode}
\input{childdoc.def}
\childdocmain{}
%    \end{macrocode}

% Optional override for |\version| flag:
%    \begin{macrocode}
%%\ifchilddoc\else\providecommand{\version}{draft}\fi
%    \end{macrocode}

% Define the default values for the |\version| flag
% (|final| for the main file and |draft| for childs):
%    \begin{macrocode}
\ifchilddoc
\providecommand{\version}{draft}
\else
\providecommand{\version}{final}
\fi
%    \end{macrocode}

% Load the standard document class:
%    \begin{macrocode}
\documentclass[12pt]{article}
%    \end{macrocode}

% Start the document body:
%    \begin{macrocode}
\begin{document}
%    \end{macrocode}

% Declare a title page.
% Print title, part of document being processed and version flag:
%    \begin{macrocode}
\addtocounter{page}{-1}
\begin{center}
{\LARGE\bfseries{}childdoc example\par}
\vspace{1cm}
\ifchilddoc
\ifchilddocmanual part\else chapter\fi:
`\childdocname' of `\childdocjob'\par
\else
main document: `\childdocjob'\par
\fi
version: \version\par
\end{center}
\newpage
%    \end{macrocode}

% Manually include selected file,
% otherwise process as usual:
%    \begin{macrocode}
\ifchilddocmanual
\section*{part `\childdocname'}
\input{\childdocname}
\else
%    \end{macrocode}

% Include the two chapters:
%    \begin{macrocode}
\include{cdocsch1}
\include{cdocsch2}
%    \end{macrocode}

% Include the two parts unless only chapters should be displayed:
%    \begin{macrocode}
\ifchilddoc\else
\section{part three}
\input{cdocspt3}
\section{part four}
\input{cdocspt4}
\fi
%    \end{macrocode}

% Process as usual until here:
%    \begin{macrocode}
\fi
%    \end{macrocode}

% End of document body:
%    \begin{macrocode}
\end{document}
%    \end{macrocode}
%\iffalse
%</samplemain>
%\fi
%
% %%%%%%%%%%%%%%%%%%%%%%%%%%%%%%%%%%%%%%
% \paragraph{Chapter Include Files.}
%
% The include files are called |cdocsch1.tex| and |cdocsch2.tex|.
%
%\iffalse
%<*samplechap1|samplechap2>
%\fi

% Optional override for |\version| flag:
%    \begin{macrocode}
%%\providecommand{\version}{final}
%    \end{macrocode}

% Include the main document:
%    \begin{macrocode}
\input{childdoc.def}
\childdocof{cdocsamp}
%    \end{macrocode}

%\iffalse
%</samplechap1|samplechap2>
%\fi
%
%\iffalse
%<*samplechap1>
%\fi
% Some text for chapter 1:
%    \begin{macrocode}
\section{one}
some text in chapter one
%    \end{macrocode}

%\iffalse
%</samplechap1>
%\fi
% Some text for chapter 2:
%\iffalse
%<*samplechap2>
%\fi
%    \begin{macrocode}
\section{two}
more text in chapter two
%    \end{macrocode}

%\iffalse
%</samplechap2>
%\fi
%
% %%%%%%%%%%%%%%%%%%%%%%%%%%%%%%%%%%%%%%
% \paragraph{Part Include Files.}
%
% The include files are called |cdocspt3.tex| and |cdocspt4.tex|.
%
%\iffalse
%<*samplepart3|samplepart4>
%\fi

% Optional override for |\version| flag:
%    \begin{macrocode}
%%\providecommand{\version}{final}
%    \end{macrocode}

% Include the main document:
%    \begin{macrocode}
\input{childdoc.def}
\childdocby{cdocsamp}
%    \end{macrocode}

%\iffalse
%</samplepart3|samplepart4>
%\fi
%
%\iffalse
%<*samplepart3>
%\fi
% Some text for part 3:
%    \begin{macrocode}
some text in part three
%    \end{macrocode}

%\iffalse
%</samplepart3>
%\fi
% Some text for part 4:
%\iffalse
%<*samplepart4>
%\fi
%    \begin{macrocode}
more text in part four
%    \end{macrocode}

%\iffalse
%</samplepart4>
%\fi
%
% %%%%%%%%%%%%%%%%%%%%%%%%%%%%%%%%%%%%%%
% \paragraph{Forwarding for a Complete Draft.}
%
% The following forwarding file |cdocsdrf.tex|
% compiles the main document in draft mode:
%\iffalse
%<*sampledraft>
%\fi
%    \begin{macrocode}
\def\version{draft}
\input{childdoc.def}
\childdocforward{cdocsamp}
%    \end{macrocode}

%\iffalse
%</sampledraft>
%\fi
%
% %%%%%%%%%%%%%%%%%%%%%%%%%%%%%%%%%%%%%%
% \paragraph{Forwarding for Final Version of the Chapters.}
%
% The following forwarding files |cdocsfn1.tex| and |cdocsfn2.tex|
% (with identical content)
% compile the final versions of the child documents
% |cdocsch1.tex| and |cdocsch2.tex|, respectively:
%\iffalse
%<*samplefinal>
%\fi
%    \begin{macrocode}
\def\version{final}
\input{childdoc.def}
\childdocforwardprefix[cdocsamp]{cdocsfn}{cdocsch}
%    \end{macrocode}

%\iffalse
%</samplefinal>
%\fi
%
% %%%%%%%%%%%%%%%%%%%%%%%%%%%%%%%%%%%%%%
% \paragraph{Command Line Processing.}
%
% The following three command lines generate the output files
% |cdocscld|, |cdocscl1| and |cdocscl2|
% which should be identical to
% |cdocsdrf|, |cdocsch1| and |cdocsfn2|, respectively:
% \begin{center}
% \begin{tabular}{l}
% |latex -jobname cdocscld \|\\
% |  "\def\version{draft}\input{childdoc.def}\childdocforward{cdocsamp}"|\\
% |latex -jobname cdocscl1 \|\\
% |  "\input{childdoc.def}\childdocforward[cdocsamp]{cdocsch1}"|\\
% |latex -jobname cdocscl2 \|\\
% |  "\def\version{final}\input{childdoc.def}\childdocforward{cdocsch2}"|
% \end{tabular}
% \end{center}
% Note that the trailing backslash on each first line
% merely continues the input to the second line
% (for convenient cut ant paste).
% Furthermore, the command |latex| can be replaced by any
% of its alternative versions such as |pdflatex|.
%
% %%%%%%%%%%%%%%%%%%%%%%%%%%%%%%%%%%%%%%%%%%%%%%%%%%%%%%%%%%%%%%%%%%%%%%%%%%%%%%
% %%%%%%%%%%%%%%%%%%%%%%%%%%%%%%%%%%%%%%%%%%%%%%%%%%%%%%%%%%%%%%%%%%%%%%%%%%%%%%
% \section{Implementation}
%\iffalse
%<*package>
%\fi
%
% This section describes the definitions file |childdoc.def|.

% The definitions cannot be loaded using |\usepackage| or |\RequirePackage|
% which has a mechanism to prevent loading a style file more than once.
% When loading the definitions by means of |\input|
% multiple instances have to be prevented manually:
%\iffalse
%This code needs to be before the `\ProvidesFile' directive
%which is defined at the beginning of this file.
%Therefore it is also placed there and commented out here.
%</package>
%<*discard>
%\fi
%    \begin{macrocode}
\ifdefined\childdocmain\endinput\fi
%    \end{macrocode}
%\iffalse
%</discard>
%<*package>
%\fi
%
% \macro{\ifchilddoc}
% \macro{\ifchilddocmanual}
% The conditional |\ifchilddoc| tells whether a
% child (true) or main (false) document is being compiled.
% The conditional |\ifchilddocmanual| tells whether
% the |\includeonly| mechanism is used (false) or
% the selection of child files must be performed manually (true).
% The definitions initialise to false:
%    \begin{macrocode}
\newif\ifchilddoc
\newif\ifchilddocmanual
%    \end{macrocode}

% \macro{\childdocname}
% \macro{\childdocjob}
% The macro |\childdocname| stores the name of the main document
% to be compiled. The macro |\childdocjob| stores the name of
% the document on which the \LaTeX{} compiler was originally invoked.
% The content of |\jobname| cannot be compared
% to filenames specified in the source due to different catcodes.
% The following code rescans |\jobname|, stores the result
% in |\childdocname| and saves a copy in |\childdocjob|:
%    \begin{macrocode}
\edef\childdocname{\scantokens\expandafter{\jobname\noexpand}}
\let\childdocjob\childdocname
%    \end{macrocode}

% \macro{\childdocdisable}
% The macro |\childdocdisable| prevents the main file
% from being processed more than once.
% At this stage, the main document command |\childdocmain|
% is assumed to be called once again where it should do nothing.
% Any subsequent call to it should prevent
% a secondary processing of the main document
% It overwrites the forwarding commands
% |\childdocof| and |\childdocforward|
% with empty macros to prevent further inclusions of the main document:
%    \begin{macrocode}
\newcommand{\childdocdisable}
{
  \renewcommand{\childdocmain}[1]{\renewcommand{\childdocmain}[1]{\endinput}}
  \renewcommand{\childdocof}[1]{}
  \renewcommand{\childdocby}[2][]{}
  \renewcommand{\childdocforward}[2][]{}
  \renewcommand{\childdocdisable}{}
}
%    \end{macrocode}

% \macro{\childdocmain}
% The macro |\childdocmain| is to be called at the top of the main file
% with nothing or the main filename (without extension) as argument.
% First, it breaks loops.
% If the argument is not empty and does not match |\childdocname|
% (which is set by the first inclusion of |childdoc.def|),
% |\ifchilddoc| is set to true, |\includeonly| is applied to the child file
% and |\jobname| is set to the main file
% (for proper handling of |.aux| files):
%    \begin{macrocode}
\newcommand{\childdocmain}[1]
{
  \childdocdisable\childdocmain{}
  \if?#1?\else
    \begingroup
      \def\childdoctmp{#1}
      \ifx\childdoctmp\childdocname
        \def\childdoctmp{}
      \else
        \def\childdoctmp
        {
          \childdoctrue
          \includeonly{\childdocname}
          \def\childdocjob{#1}
          \def\jobname{#1}
        }
      \fi
      \expandafter
    \endgroup
    \childdoctmp
  \fi
}
%    \end{macrocode}

% \macro{\childdocof}
% The command |\childdocof| redirects
% compilation to the main file |#1|.
%    \begin{macrocode}
\newcommand{\childdocof}[1]
{
  \childdocdisable
  \childdoctrue
  \includeonly{\childdocname}
  \def\jobname{#1}
  \def\childdocjob{#1}
  \input{#1}
}
%    \end{macrocode}

% \macro{\childdocby}
% The command |\childdocby| ....
%    \begin{macrocode}
\newcommand{\childdocby}[2][]
{
  \childdocdisable
  \childdoctrue
  \childdocmanualtrue
  \if?#1?\else
    \def\jobname{#2}
  \fi
  \def\childdocjob{#2}
  \input{#2}
  \endinput
}
%    \end{macrocode}

% \macro{\childdocforward}
% The command |\childdocforward| redirects
% compilation to the main file or
% (if the optional argument is given) a child file.
% Parameters are set as if the main file
% or a child file starting with |\childdocof| was compiled.
% Then compilation is handed over to the main file:
%    \begin{macrocode}
\newcommand{\childdocforward}[2][]
{
  \begingroup
    \if?#1?
      \def\childdoctmp
      {
        \def\childdocname{#2}
        \def\childdocjob{#2}
        \def\jobname{#2}
        \input{#2}
        \endinput
      }
    \else
      \def\childdoctmp
      {
        \childdocdisable
        \def\childdocname{#2}
        \childdoctrue
        \includeonly{#2}
        \def\childdocjob{#1}
        \def\jobname{#1}
        \input{#1}
        \endinput
      }
    \fi
    \expandafter
  \endgroup
  \childdoctmp
}
%    \end{macrocode}

% \macro{\childdocforwardprefix}
% The command |\childdocforwardprefix| redirects
% compilation to the main or a child file by means of a pattern.
% The prefix |#1| in the current filename is replaced by |#2|
% and the suffix of the current filename is kept
% (it is assumed that the filename does not contain the substring `|~~~|'
% which is used as a delimiter).
% Compilation is handed over to the new file by |\childdocforward|:
%    \begin{macrocode}
\newcommand{\childdocforwardprefix}[3][]
{
  \begingroup
    \def\childdocextract #2##1~~~{\def\childdoctmp{\childdocforward[#1]{#3##1}}}
    \expandafter\childdocextract\childdocname~~~
    \expandafter
  \endgroup
  \childdoctmp
}
%    \end{macrocode}

% \macro{\childdoc}
% The deprecated macro |\childdoc| is a legacy version of |\childdocmain|:
%    \begin{macrocode}
\newcommand{\childdoc}{\childdocmain}
%    \end{macrocode}

% \macro{\childdocredirect}
% The deprecated macro |\childdocredirect| is a legacy version
% of |\childdocforward| and |\childdocforwardprefix|:
%    \begin{macrocode}
\newcommand{\childdocredirect}[2][]
{
  \begingroup
    \if?#1?
      \def\childdoctmp{\childdocforward{#2}}
    \else
      \def\childdoctmp{\childdocforwardprefix{#1}{#2}}
    \fi
    \expandafter
  \endgroup
  \childdoctmp
}
%    \end{macrocode}

%\iffalse
%</package>
%\fi
%
\endinput
|\\
|\childdocby{|\textit{main}|}|\\
\end{tabular}
\end{center}
%
The directive |\childdocby| is similar to |\childdocof|
described in \secref{sec:include},
but the subsequent selection of content must be done manually.
To that end, both |\ifchilddoc| and |\ifchilddocmanual|
will be true upon processing of a part,
and the name of the part is stored in |\childdocname|.
Note that |\jobname| will be set to the filename of the current part
so that each part receives an individual |.aux| file
that does not interfere with the |.aux| file(s) of the main document.
This behaviour can be altered by the alternative form
|\childdocby[*]{|\textit{main}|}| (with a non-empty optional argument)
which uses the |.aux| file of the main document
by setting |\jobname| to \textit{main}.

%%%%%%%%%%%%%%%%%%%%%%%%%%%%%%%%%%%%%%%%%%%%%%%%%%%%%%%%%%%%%%%%%%%%%%%%%%%%%%%%
\subsection{Driver Development}
\label{sec:driver}

The \textsf{childdoc} mechanism can also be use for the development
of definition files such as \LaTeX{} styles or classes.
This case differs from the above setup with multiple parts
included by |\include| in that no |\includeonly| should be invoked.
This can be achieved by starting the include file
(before |\ProvidesPackage|) with:
%
\begin{center}
\begin{tabular}{l}
|% \iffalse
%
% childdoc.dtx Copyright (C) 2017-2018 Niklas Beisert
%
% This work may be distributed and/or modified under the
% conditions of the LaTeX Project Public License, either version 1.3
% of this license or (at your option) any later version.
% The latest version of this license is in
%   http://www.latex-project.org/lppl.txt
% and version 1.3 or later is part of all distributions of LaTeX
% version 2005/12/01 or later.
%
% This work has the LPPL maintenance status `maintained'.
%
% The Current Maintainer of this work is Niklas Beisert.
%
% This work consists of the files childdoc.dtx and childdoc.ins
% and the derived files childdoc.def and cdocsamp.tex with
% cdocsch1.tex, cdocsch2.tex, cdocsdrf.tex, cdocsfn1.tex, cdocsfn2.tex.
%
%<package>\ifdefined\childdocmain\endinput\fi
%<package>\ProvidesFile{childdoc.def}[2018/12/30 v2.0 child document driver]
%<samplemain>\ProvidesFile{cdocsamp.tex}[2018/12/30 v2.0 sample for childdoc]
%<*driver>
%\ProvidesFile{childdoc.drv}[2018/12/30 v2.0 childdoc reference manual file]
\PassOptionsToClass{10pt,a4paper}{article}
\documentclass{ltxdoc}

\usepackage[margin=35mm]{geometry}
\usepackage{hyperref}
\usepackage{hyperxmp}
\usepackage[usenames]{color}

\hypersetup{colorlinks=true}
\hypersetup{pdfstartview=FitH}
\hypersetup{pdfpagemode=UseNone}
\hypersetup{pdfsource={}}
\hypersetup{pdflang={en-UK}}
\hypersetup{pdfcopyright={Copyright 2017-2018 Niklas Beisert.
  This work may be distributed and/or modified under the
  conditions of the LaTeX Project Public License, either version 1.3
  of this license or (at your option) any later version.}}
\hypersetup{pdflicenseurl={http://www.latex-project.org/lppl.txt}}
\hypersetup{pdfcontactaddress={ETH Zurich, ITP, HIT K,
  Wolfgang-Pauli-Strasse 27}}
\hypersetup{pdfcontactpostcode={8093}}
\hypersetup{pdfcontactcity={Zurich}}
\hypersetup{pdfcontactcountry={Switzerland}}
\hypersetup{pdfcontactemail={nbeisert@itp.phys.ethz.ch}}
\hypersetup{pdfcontacturl={http://people.phys.ethz.ch/\xmptilde nbeisert/}}

\newcommand{\secref}[1]{\hyperref[#1]{section \ref*{#1}}}

\parskip1ex
\parindent0pt
\let\olditemize\itemize
\def\itemize{\olditemize\parskip0pt}

\begin{document}

\title{The \textsf{childdoc} Package}
\hypersetup{pdftitle={The childdoc Package}}
\author{Niklas Beisert\\[2ex]
  Institut f\"ur Theoretische Physik\\
  Eidgen\"ossische Technische Hochschule Z\"urich\\
  Wolfgang-Pauli-Strasse 27, 8093 Z\"urich, Switzerland\\[1ex]
  \href{mailto:nbeisert@itp.phys.ethz.ch}
  {\texttt{nbeisert@itp.phys.ethz.ch}}}
\hypersetup{pdfauthor={Niklas Beisert}}
\hypersetup{pdfsubject={Manual for the LaTeX2e Package childdoc}}
\date{30 December 2018, \textsf{v2.0}}
\maketitle

\begin{abstract}\noindent
\textsf{childdoc} is a \LaTeXe{} package
that enables the direct compilation
of document sections included by |\include|
to individual files.
\end{abstract}

\begingroup
\parskip0ex
\tableofcontents
\endgroup

%%%%%%%%%%%%%%%%%%%%%%%%%%%%%%%%%%%%%%%%%%%%%%%%%%%%%%%%%%%%%%%%%%%%%%%%%%%%%%%%
%%%%%%%%%%%%%%%%%%%%%%%%%%%%%%%%%%%%%%%%%%%%%%%%%%%%%%%%%%%%%%%%%%%%%%%%%%%%%%%%
\section{Introduction}

\LaTeX{} provides a mechanism to structure a large document (such as a book)
into a main file and several child files (containing the chapters)
using the |\include| command.
This mechanism is beneficial for documents
which span hundreds of pages in order to
make the source file(s) more manageable.
Moreover, compilation can be restricted to
selected child files by means of the |\includeonly| command.
The latter feature can be used to reduce the compilation time while editing
(this was significantly more useful in the earlier days of \LaTeX{})
or to generate a smaller document which is easier to navigate.
Another application of |\includeonly| is to generate
documents consisting of selected parts of the complete document.

However, there are a few drawbacks of the plain |\include| mechanism:
\begin{itemize}
\item
The child files cannot be compiled on their own,
they can only be compiled via the main file.
A naive editing environment
(such as a text editor with an option
to have the current file processed by \LaTeX)
may require one to switch to the main file before compiling;
attempting to compile the child file produces errors.
\item
The main file must be modified (each time)
to adjust the |\includeonly| command
to the present needs. This easily leaves the main file in a messy state.
\item
The generated document will always carry the filename
of the main document. This is inconvenient if
several child files are to be compiled and
to be kept for distribution.
\end{itemize}

The present package provides a simple interface
to make child files individually compilable by \LaTeX{}.
Compiling a child file then has the same effect as compiling
the main file with an |\includeonly| command
to select the appropriate child.
Moreover the generated document will carry the name of the child
rather than the main file.
This resolves all three above issues.

This feature is meant to make the editing of books,
thesis documents and lecture notes somewhat more convenient.
However, the package can also be used efficiently for
composing a series of documents (such as exercise sheets)
which are typically distributed individually.
It then assists the author in generating the individual documents
(potentially in different versions)
as well as a document containing the collected series.
Another application is in developing style files
or other kinds of included material
where compilation of the style file could redirect
to a sample or test file.

%%%%%%%%%%%%%%%%%%%%%%%%%%%%%%%%%%%%%%%%%%%%%%%%%%%%%%%%%%%%%%%%%%%%%%%%%%%%%%%%
%%%%%%%%%%%%%%%%%%%%%%%%%%%%%%%%%%%%%%%%%%%%%%%%%%%%%%%%%%%%%%%%%%%%%%%%%%%%%%%%
\section{Usage}

First of all, the package \textsf{childdoc} is \emph{not} a standard
\LaTeXe{} |.sty| style file! Therefore it needs to be invoked in
a non-standard way.

%%%%%%%%%%%%%%%%%%%%%%%%%%%%%%%%%%%%%%%%%%%%%%%%%%%%%%%%%%%%%%%%%%%%%%%%%%%%%%%%
\subsection{Included Files}
\label{sec:include}

%%%%%%%%%%%%%%%%%%%%%%%%%%%%%%%%%%%%%%%%
\DescribeMacro{\childdocmain}
To use the package, add the commands
\begin{center}
\begin{tabular}{l}
|\input{childdoc.def}|\\
|\childdocmain{}|\\
\end{tabular}
\end{center}
at the very top of the main \LaTeX{} file,
in particular \emph{before} the |\documentclass| statement!
The argument of |\childdocmain| should be left empty
(but it must be present).

%%%%%%%%%%%%%%%%%%%%%%%%%%%%%%%%%%%%%%%%
\DescribeMacro{\childdocof}
Furthermore, add the commands
\begin{center}
\begin{tabular}{l}
|\input{childdoc.def}|\\
|\childdocof{|\textit{main}|}|\\
\end{tabular}
\end{center}
at the top of every child file \textit{child}
which is included by |\include{|\textit{child}|}|
from within the main file
(or at least for those files to be compiled individually).
The argument \textit{main} must be the filename of the main file.

There are a couple of
considerations in setting up the main and child documents:

%%%%%%%%%%%%%%%%%%%%%%%%%%%%%%%%%%%%%%%%
\paragraph{Restrictions.}

Please note the following restrictions:
\begin{itemize}
\item
|\childdocmain| must be called with one argument \textit{main}
to ensure compatibility with earlier version of the package.
It must either be empty (|\childdocmain{}|)
or precisely match the filename of the main file in which it is specified.
See \secref{sec:detection} for further information.
\item
The filename \textit{main} must be specified without the |.tex| extension.
\item
The filename \textit{main} is case sensitive
(even in case-insensitive file systems)
due to internal string comparison.
\item
The argument \textit{main} should be fully expanded, it cannot be a macro.
\item
Subdirectories and special characters should be avoided in filenames.
\item
The command |\childdocmain{|\textit{main}|}| must be followed by a whitespace.
It should not be followed immediately by another command
or by a comment mark `|%|'.
This is because the \TeX{} parser reads the token immediately following
the argument of |\childdocmain| and puts it
at the beginning of every child section;
however, a white\-space is ignored.
\end{itemize}

%%%%%%%%%%%%%%%%%%%%%%%%%%%%%%%%%%%%%%%%
\paragraph{Content of Main File.}

It is advisable to place all content in the child files included by |\include|.
Any output contained in the main file will appear in all child documents
unless suppressed manually;
it cannot be suppressed automatically by the |\includeonly| directive
and thus should normally be avoided.
A method to include some content in the main file
by means of conditional processing is described in \secref{sec:conditional}.

%%%%%%%%%%%%%%%%%%%%%%%%%%%%%%%%%%%%%%%%
\paragraph{Page Numbering.}

When only a part of the document is compiled,
the appropriate numbering of pages
(as well as other status parameters)
is determined from the |.aux| files.
The latter contain information from previous passes.
However this information needs to propagate through
all intermediate child documents.
Therefore the page numbering in child documents may well
be inconsistent until the complete document is compiled at least once.

A useful (if unconventional) way to always ensure a consistent
page numbering is to restart the numbering in each child document
and denote the pages by `\textit{child}|.|\textit{page}'
where \textit{child} represents the chapter/section number of the child file.
This can be achieved by the command
|\numberwithin{page}{|\textit{child}|}|
of the \textsf{amsmath} package
where \textit{child} can be |chapter| or |section|
depending on the chosen structuring.
Alternatively, one can modify the macro |\thepage| appropriately
and reset the counter |page| at the start of each child file.

%%%%%%%%%%%%%%%%%%%%%%%%%%%%%%%%%%%%%%%%%%%%%%%%%%%%%%%%%%%%%%%%%%%%%%%%%%%%%%%%
\subsection{Conditional Processing}
\label{sec:conditional}

The package provides a mechanism to compile different versions
of a document. To customise the versions further some conditional processing
can come in handy to distinguish which version is being compiled.
The package provides two macros to describe the compilation context:

%%%%%%%%%%%%%%%%%%%%%%%%%%%%%%%%%%%%%%%%
\DescribeMacro{\ifchilddoc}
The conditional |\ifchilddoc| distinguishes between the compilation of
child documents and the main document:
%
\begin{center}
|\ifchilddoc |\textit{child-code}| |[|\||else |\textit{main-code}]| \||fi|
\end{center}

%%%%%%%%%%%%%%%%%%%%%%%%%%%%%%%%%%%%%%%%
\DescribeMacro{\childdocname}
\DescribeMacro{\childdocjob}
The macro |\childdocname| contains the filename (without extension)
of the main or child file being processed.
Note that |\childdocjob| will always contain the name of the main file.

%%%%%%%%%%%%%%%%%%%%%%%%%%%%%%%%%%%%%%%%
\paragraph{Title Page.}

Conditional processing can be used to include a title or banner page
in the main document when proper precautions are taken.
Importantly, the code in the main file should ensure that the page counter
(as well as other status parameters which are stored in the |.aux| files)
takes the same value after the conditional processing.
Otherwise the page numbers may take divergent values
depending on which part is compiled.

For example, a title page could be declared by:
%
\begin{center}
\begin{tabular}{l}
|\ifchilddoc\||else|\\
|\addtocounter{page}{-1}|\\
\textit{code for title page}\\
|\newpage|\\
|\||fi|
\end{tabular}
\end{center}
%
A banner page for the child documents can be generated by:
%
\begin{center}
\begin{tabular}{l}
|\ifchilddoc|\\
|\addtocounter{page}{-1}|\\
\textit{code for banner page}\\
|\newpage|\\
|\||fi|
\end{tabular}
\end{center}
%
Here one could write a message such as:
\begin{center}
|This is the part \childdocname{} of \childdocjob{}.|
\end{center}

%%%%%%%%%%%%%%%%%%%%%%%%%%%%%%%%%%%%%%%%%%%%%%%%%%%%%%%%%%%%%%%%%%%%%%%%%%%%%%%%
\subsection{Flags}
\label{sec:flags}

The package makes it easy to generate different versions
of the main or child documents.
To this end compilation flags can be defined
and assigned different default values.
They will be particularly useful in conjunction
with the forwarding mechanism described in \secref{sec:forward}.

For example, it may be useful to have a flag |\version|
which can be set to |draft| or |final|.
The document source will contain some conditional code
depending on the value of |\version|.
Suppose further, the flag should default to |final| for the main file
and to |draft| for child files
which is a natural assignment for editing the document.
This is achieved by placing the following code
in the preamble of the main document
(below the |\childdocmain| directive):
%
\begin{center}
\begin{tabular}{l}
|\ifchilddoc|\\
|\providecommand{\version}{draft}|\\
|\||else|\\
|\providecommand{\version}{final}|\\
|\||fi|
\end{tabular}
\end{center}
%
The definition by |\providecommand| makes sure
that previous definitions are not overwritten.
Further statements |\providecommand{\version}{...}|
can thus be added before the above code to override it.

For the main file, one might add a line
(between |\childdocmain| and the above block)
%
\begin{center}
|%\ifchilddoc\||else\providecommand{\version}{draft}\||fi|
\end{center}
%
which can be uncommented to produce a draft version.
Likewise one can add a line to the very top of a child file
(above the |\childdocof{|\textit{main}|}| directive)
%
\begin{center}
|%\providecommand{\version}{final}|
\end{center}
%
which can be uncommented to produce the final version of this child document.

%%%%%%%%%%%%%%%%%%%%%%%%%%%%%%%%%%%%%%%%%%%%%%%%%%%%%%%%%%%%%%%%%%%%%%%%%%%%%%%%
\subsection{Forwarding}
\label{sec:forward}

Different versions of the main or child documents
using compilation flags as described in \secref{sec:flags}
can be (permanently) stored in different files
for convenient compilation, viewing and distribution.
To this end, the package defines a command
to pass on compilation to a different file:

%%%%%%%%%%%%%%%%%%%%%%%%%%%%%%%%%%%%%%%%
\DescribeMacro{\childdocforward}
The command |\childdocforward| redirects processing to
another source file:
%
\begin{center}
\begin{tabular}{l}
|\input{childdoc.def}|\\
|\childdocforward[|\textit{main}|]{|\textit{dest}|}|\\
\end{tabular}
\end{center}
%
The argument \textit{dest} is the destination file
(without extension).
It should be the main file or one of the child files.
Note that further \textsf{childdoc} directives
such as |\childdocof| and |\childdocforward|
in the indicated file will be processed in this form.
The optional argument \textit{main}
passes on directly to the main file \textit{main}
while pretending to compile the child \textit{dest}.
This form behaves as if \textit{dest}
issues |\childdocof{|\textit{main}|}| right away,
and no further \textsf{childdoc} directives will be processed.

%%%%%%%%%%%%%%%%%%%%%%%%%%%%%%%%%%%%%%%%
\DescribeMacro{\...prefix}
In the alternative form |\childdocforwardprefix|,
%
\begin{center}
\begin{tabular}{l}
|\input{childdoc.def}|\\
|\childdocforwardprefix[|\textit{main}|]{|\textit{prefix}|}{|\textit{dest}|}|
\end{tabular}
\end{center}
%
the destination file is determined by a pattern
depending on the current file:
To make this work, the current file must be called
`{\textit{prefix}\hspace{0.2em}\textit{suffix}}'
with \textit{prefix} matching precisely the argument.
Processing is then passed on to the file
`{\textit{dest}\hspace{0.2em}\textit{suffix}}'.
Surely, the same effect is achieved by
directly specifying the
argument `{\textit{dest}\hspace{0.2em}\textit{suffix}}'
in the first form.
However, that requires to set up a different file
for each child. With the alternative form of the command
all these files can have exactly the same content
which simplifies setting them up and maintaining them.

For example, the following file |draft.tex|
with a compilation flag |\version| as described in \secref{sec:flags}
compiles the main document as a draft:
%
\begin{center}
\begin{tabular}{l}
|\def\version{draft}|\\
|\input{childdoc.def}|\\
|\childdocforward{|\textit{main}|}|
\end{tabular}
\end{center}
%
Likewise, the following files |final|\textit{nn}|.tex|
compile the final version of the child document
|child|\textit{nn}|.tex|:
%
\begin{center}
\begin{tabular}{l}
|\def\version{final}|\\
|\input{childdoc.def}|\\
|\childdocforwardprefix{final}{child}|
\end{tabular}
\end{center}
%

Note that when several versions of a main file and/or of each child file
are to be generated, it may be convenient to set up a |Makefile| or
shell script to automatise the process.

%%%%%%%%%%%%%%%%%%%%%%%%%%%%%%%%%%%%%%%%%%%%%%%%%%%%%%%%%%%%%%%%%%%%%%%%%%%%%%%%
\subsection{Command Line Processing}
\label{sec:commandline}

The effect of redirection files can also be achieved by invoking
the \LaTeX{} compiler with a more elaborate command line.
Most conveniently this should be done as part
of a shell script or a |Makefile|.

When using \textsf{childdoc} in the main file, the following
command lines effectively perform a redirection
(note that depending on the shell being used,
backslashes may have to be doubled: `|\|' $\to$ `|\\|'):
%
\begin{center}
|... -jobname "|\textit{target}|" |\\|"|[\textit{flags}]%
|\input{childdoc.def}\childdocforward[|\textit{main}|]{|\textit{dest}|}"|
\end{center}
%
Here \textit{target} is the name of the output file,
\textit{main} is the name of the main file
and \textit{dest} is the name of the main or child file to be processed
(all filenames without extensions).
The optional argument \textit{main} can be omitted
if \textit{main} matches \textit{dest}.
Optionally, compilation \textit{flags} can be defined via |\def| commands.
This command line makes the \TeX{} engine believe
it is compiling the file \textit{target}
whose content is specified as the latter parameter.
The provided code then forwards the processing to
\textit{main} or \textit{dest} as described in \secref{sec:forward}.

%%%%%%%%%%%%%%%%%%%%%%%%%%%%%%%%%%%%%%%%%%%%%%%%%%%%%%%%%%%%%%%%%%%%%%%%%%%%%%%%
\subsection{Include by Input}
\label{sec:input}

Including child documents by |\include| has some restrictions by design.
Most notably, the content of a child document always occupies
its own set of pages; pages cannot be shared between child documents.
Usually, this behaviour makes perfect sense
because each child document contain an essential part of the document.
However, in some situations it may be desirable to compose
a document from a collection of parts
without having mandatory page breaks between then.
For this case, the package
provides a mechanism to include parts
by |\input| which can also be processed individually.
However, by construction this mechanism
requires manual handling of the content to be output.

%%%%%%%%%%%%%%%%%%%%%%%%%%%%%%%%%%%%%%%%
\DescribeMacro{\ifchilddocmanual}
The main file should be prepared as usual, see \secref{sec:include}.
However, the document body must make a distinction
between processing of an individual part and of the main document, e.g.:
%
\begin{center}
\begin{tabular}{l}
|\ifchilddocmanual|\\
|\input{\childdocname}|\\
|\||else|\\
\textit{document body with }|\input{|\textit{part}|}|\\
|\||fi|
\end{tabular}
\end{center}
%
The conditional |\ifchilddocmanual| is true whenever
a part to be included by |\input| is being compiled,
and the name of the part is stored in |\childdocname|.

%%%%%%%%%%%%%%%%%%%%%%%%%%%%%%%%%%%%%%%%
\DescribeMacro{\childdocby}
Each part to be included by |\input| should start with:
%
\begin{center}
\begin{tabular}{l}
|\input{childdoc.def}|\\
|\childdocby{|\textit{main}|}|\\
\end{tabular}
\end{center}
%
The directive |\childdocby| is similar to |\childdocof|
described in \secref{sec:include},
but the subsequent selection of content must be done manually.
To that end, both |\ifchilddoc| and |\ifchilddocmanual|
will be true upon processing of a part,
and the name of the part is stored in |\childdocname|.
Note that |\jobname| will be set to the filename of the current part
so that each part receives an individual |.aux| file
that does not interfere with the |.aux| file(s) of the main document.
This behaviour can be altered by the alternative form
|\childdocby[*]{|\textit{main}|}| (with a non-empty optional argument)
which uses the |.aux| file of the main document
by setting |\jobname| to \textit{main}.

%%%%%%%%%%%%%%%%%%%%%%%%%%%%%%%%%%%%%%%%%%%%%%%%%%%%%%%%%%%%%%%%%%%%%%%%%%%%%%%%
\subsection{Driver Development}
\label{sec:driver}

The \textsf{childdoc} mechanism can also be use for the development
of definition files such as \LaTeX{} styles or classes.
This case differs from the above setup with multiple parts
included by |\include| in that no |\includeonly| should be invoked.
This can be achieved by starting the include file
(before |\ProvidesPackage|) with:
%
\begin{center}
\begin{tabular}{l}
|\input{childdoc.def}|\\
|\childdocforward{|\textit{main}|}|\\
\end{tabular}
\end{center}
%
or alternatively with:
%
\begin{center}
\begin{tabular}{l}
|\input{childdoc.def}|\\
|\childdocby{|\textit{main}|}|\\
\end{tabular}
\end{center}
%
Both forms have slightly different effects as described above.
The main file is prepared as usual, see \secref{sec:include}.

%%%%%%%%%%%%%%%%%%%%%%%%%%%%%%%%%%%%%%%%%%%%%%%%%%%%%%%%%%%%%%%%%%%%%%%%%%%%%%%%
\subsection{Legacy Detection}
\label{sec:detection}

The directive |\childdocmain| in the main file can detect
whether the complete document or merely a child is to be compiled
even without using the directive |\childdocof|.
This method is deprecated because it is less robust
and there is no compelling reason to use it;
it is merely provided for backward compatibility
and it may be removed in future versions.

If the detection mechanism is to be used,
it is mandatory to correctly specify
the filename of the main file as the argument of |\childdocmain|:
%
\begin{center}
\begin{tabular}{l}
|\input{childdoc.def}|\\
|\childdocmain{|\textit{main}|}|\\
\end{tabular}
\end{center}
%
If |\jobname| does not match the argument \textit{main} of |\childdocmain|,
it is assumed that |\jobname| points to the child file to be compiled.
When using |\childdocmain| with the main file specified as argument,
it suffices to start a child file
with just |\input{|\textit{main}|}|
without loading of the package and using |\childdocof|.
If instead all processing is done
with the appropriate \textsf{childdoc} directives,
the argument of \textit{main} of |\childdocmain| can be empty.

An alternative version of the command line processing described
in \secref{sec:commandline} using the detection mechanism reads:
%
\begin{center}
|... -jobname "|\textit{target}|" "|[\textit{flags}]%
[|\def\jobname{|\textit{dest}|}|]|\input{|\textit{main}|}"|
\end{center}

%%%%%%%%%%%%%%%%%%%%%%%%%%%%%%%%%%%%%%%%%%%%%%%%%%%%%%%%%%%%%%%%%%%%%%%%%%%%%%%%
\subsection{Manual Code}
\label{sec:manual}

In case one cannot be certain whether the definitions file |childdoc.def|
is installed on the target \TeX{} distribution
and one prefers not to ship it,
it is conceivable to paste a few relevant commands into the sources.

To that end, drop all statements |\input{childdoc.def}|
and perform the replacements as outlined below.
Instead of |\childdocmain{|\textit{main}|}| add the following code
to the top of the main file:
%
\begin{center}
\begin{tabular}{l}
|\||ifdefined\childdocname\endinput\||fi\newif\ifchilddoc|\\
|\edef\childdocname{\scantokens\expandafter{\jobname\noexpand}}|\\
|\def\childdocmain{|\textit{main}|}\||ifx\childdocmain\childdocname\||else|\\
|\childdoctrue\includeonly{\childdocname}\let\jobname\childdocmain\||fi|\\
\end{tabular}
\end{center}
%
Instead of |\childdocof{|\textit{main}|}| just include the main file
at the top of each child file:
%
\begin{center}
|\input{|\textit{main}|}|
\end{center}
%
A simple redirection |\childdocforward{|\textit{dest}|}| is achieved by:
%
\begin{center}
|\def\jobname{|\textit{dest}|}\input{\jobname}|
\end{center}
%
The redirection with prefix
|\childdocforwardprefix[|\textit{prefix}|]{|\textit{dest}|}|
is accomplished by:
%
\begin{center}
\begin{tabular}{l}
|{\edef\jobname{\scantokens\expandafter{\jobname\noexpand}}|\\
|\def\redirectjob |\textit{prefix}|#1~~~{\gdef\jobname{|\textit{dest}|#1}}|\\
|\expandafter\redirectjob\jobname~~~}\input{\jobname}|
\end{tabular}
\end{center}

In an alternative approach,
child documents can be compiled by a specific command line
without additional code or specific definitions:
%
\begin{center}
|... -jobname "|\textit{target}|" "|[\textit{flags}]%
|\includeonly{|\textit{dest}|}\input{|\textit{main}|}"|
\end{center}
%

%%%%%%%%%%%%%%%%%%%%%%%%%%%%%%%%%%%%%%%%%%%%%%%%%%%%%%%%%%%%%%%%%%%%%%%%%%%%%%%%
%%%%%%%%%%%%%%%%%%%%%%%%%%%%%%%%%%%%%%%%%%%%%%%%%%%%%%%%%%%%%%%%%%%%%%%%%%%%%%%%
\section{Information}

%%%%%%%%%%%%%%%%%%%%%%%%%%%%%%%%%%%%%%%%%%%%%%%%%%%%%%%%%%%%%%%%%%%%%%%%%%%%%%%%
\subsection{Copyright}

Copyright \copyright{} 2017--2018 Niklas Beisert

This work may be distributed and/or modified under the
conditions of the \LaTeX{} Project Public License, either version 1.3
of this license or (at your option) any later version.
The latest version of this license is in
  \url{http://www.latex-project.org/lppl.txt}
and version 1.3 or later is part of all distributions of \LaTeX{}
version 2005/12/01 or later.

This work has the LPPL maintenance status `maintained'.

The Current Maintainer of this work is Niklas Beisert.

This work consists of the files |README.txt|, |childdoc.ins| and |childdoc.dtx|
as well as the derived files |childdoc.def|, |cdocsamp.tex|
with |cdocsch1.tex|, |cdocsch2.tex|, |cdocspt3.tex|, |cdocspt4.tex|,
|cdocsdrf.tex|, |cdocsfn1.tex|, |cdocsfn2.tex|
as well as |childdoc.pdf|.

%%%%%%%%%%%%%%%%%%%%%%%%%%%%%%%%%%%%%%%%%%%%%%%%%%%%%%%%%%%%%%%%%%%%%%%%%%%%%%%%
\subsection{Files and Installation}

The package consists of the files:
%
\begin{center}
\begin{tabular}{ll}
    |README.txt|   & readme file \\
    |childdoc.ins| & installation file \\
    |childdoc.dtx| & source file \\
    |childdoc.def| & definition file \\
    |cdocsamp.tex| & sample main file \\
    |cdocsch1.tex| & sample include file \\
    |cdocsch2.tex| & sample include file \\
    |cdocspt3.tex| & sample part file \\
    |cdocspt4.tex| & sample part file \\
    |cdocsdrf.tex| & sample redirection file \\
    |cdocsfn1.tex| & sample redirection file \\
    |cdocsfn2.tex| & sample redirection file \\
    |childdoc.pdf| & manual
\end{tabular}
\end{center}
%
The distribution consists of the files
|README.txt|, |childdoc.ins| and |childdoc.dtx|.
%
\begin{itemize}
\item
Run (pdf)\LaTeX{} on |childdoc.dtx|
to compile the manual |childdoc.pdf| (this file).
\item
Run \LaTeX{} on |childdoc.ins| to create the definitions file |childdoc.def|
and the sample |cdocsamp.tex| with include files
|cdocsch1.tex|, |cdocsch2.tex|, |cdocspt3.tex|, |cdocspt4.tex|,
|cdocsdrf.tex|, |cdocsfn1.tex|, |cdocsfn2.tex|.
Then copy the file |childdoc.def| to an appropriate directory of your \LaTeX{}
distribution, e.g.\ \textit{texmf-root}|/tex/latex/childdoc|.
\end{itemize}

%%%%%%%%%%%%%%%%%%%%%%%%%%%%%%%%%%%%%%%%%%%%%%%%%%%%%%%%%%%%%%%%%%%%%%%%%%%%%%%%
\subsection{Related CTAN Packages}

There are several other packages which offer a similar functionality:
%
\begin{itemize}
\item
The packages
\href{http://ctan.org/pkg/docmute}{\textsf{docmute}},
\href{http://ctan.org/pkg/includex}{\textsf{includex}} and
\href{http://ctan.org/pkg/standalone}{\textsf{standalone}}
provide commands to include only the document body of
a child file thus allowing both files to be compiled individually.
\item
The packages \href{http://ctan.org/pkg/subdocs}{\textsf{subdocs}}
and \href{http://ctan.org/pkg/subfiles}{\textsf{subfiles}}
provide structures in which the main and child documents can be
encapsulated and allowing them to be compiled individually.
The inclusion mechanism is different from the conventional |\include|.
\item
The package \href{http://ctan.org/pkg/combine}{\textsf{combine}}
is an elaborate solution to combine several documents into one.
\end{itemize}
%
See also the CTAN topic \href{http://ctan.org/topic/subdocs}{\textsf{subdocs}}
for further related packages.
The present package differs from the above solutions in that
a document structure constructed with the conventional |\include| mechanism
just needs two extra commands at the top of every file
such that all constituent files can be compiled individually.

%%%%%%%%%%%%%%%%%%%%%%%%%%%%%%%%%%%%%%%%%%%%%%%%%%%%%%%%%%%%%%%%%%%%%%%%%%%%%%%%
%\subsection{Feature Suggestions}
%
%The following is a list of features which may be useful for future
%versions of this package:
%%
%\begin{itemize}
%\item
%\ldots
%\end{itemize}

%%%%%%%%%%%%%%%%%%%%%%%%%%%%%%%%%%%%%%%%%%%%%%%%%%%%%%%%%%%%%%%%%%%%%%%%%%%%%%%%
\subsection{Revision History}

%%%%%%%%%%%%%%%%%%%%%%%%%%%%%%%%%%%%%%%%
\paragraph{v2.0:} 2018/12/30

\begin{itemize}
\item
immediate forward processing
\item
added |\childdocby| mechanism
\item
manual restructured
\end{itemize}

%%%%%%%%%%%%%%%%%%%%%%%%%%%%%%%%%%%%%%%%
\paragraph{v1.6:} 2018/01/17

\begin{itemize}
\item
application for development of include files
\item
corrections to manual
\end{itemize}

%%%%%%%%%%%%%%%%%%%%%%%%%%%%%%%%%%%%%%%%
\paragraph{v1.5:} 2017/05/21

\begin{itemize}
\item
more complete structuring introduced
\item
|\childdocof| introduced
\item
|\childdoc| renamed to |\childdocmain|
\item
|\childredirect| renamed to |\childdocforward| and |\childdocforwardprefix|
and functionality expanded
\end{itemize}

%%%%%%%%%%%%%%%%%%%%%%%%%%%%%%%%%%%%%%%%
\paragraph{v1.0:} 2017/04/27

\begin{itemize}
\item
manual and install package
\item
first version published on CTAN
\end{itemize}

%%%%%%%%%%%%%%%%%%%%%%%%%%%%%%%%%%%%%%%%
\paragraph{v0.6:} 2017/04/26

\begin{itemize}
\item
redirection mechanism added
\end{itemize}

%%%%%%%%%%%%%%%%%%%%%%%%%%%%%%%%%%%%%%%%
\paragraph{v0.5:} 2017/04/26

\begin{itemize}
\item
functionality in definition file
\end{itemize}


%%%%%%%%%%%%%%%%%%%%%%%%%%%%%%%%%%%%%%%%%%%%%%%%%%%%%%%%%%%%%%%%%%%%%%%%%%%%%%%%
%%%%%%%%%%%%%%%%%%%%%%%%%%%%%%%%%%%%%%%%%%%%%%%%%%%%%%%%%%%%%%%%%%%%%%%%%%%%%%%%
%%%%%%%%%%%%%%%%%%%%%%%%%%%%%%%%%%%%%%%%%%%%%%%%%%%%%%%%%%%%%%%%%%%%%%%%%%%%%%%%
\appendix

\settowidth\MacroIndent{\rmfamily\scriptsize 000\ }

 \DocInput{childdoc.dtx}

\end{document}
%</driver>
% \fi
%
% %%%%%%%%%%%%%%%%%%%%%%%%%%%%%%%%%%%%%%%%%%%%%%%%%%%%%%%%%%%%%%%%%%%%%%%%%%%%%%
% %%%%%%%%%%%%%%%%%%%%%%%%%%%%%%%%%%%%%%%%%%%%%%%%%%%%%%%%%%%%%%%%%%%%%%%%%%%%%%
% \section{Sample}
%\iffalse
%<*samplemain>
%\fi
%
% The following presents a sample document
% with two chapters, two parts, a title page,
% a compile flag as well as three forwarding files to set the flag.
% It consists of eight |.tex| files:
% \begin{center}
% \begin{tabular}{ll}
% |cdocsamp.tex|&main file\\
% |cdocsch1.tex|&include file for chapter 1\\
% |cdocsch2.tex|&include file for chapter 2\\
% |cdocspt3.tex|&include file for part 3\\
% |cdocspt4.tex|&include file for part 4\\
% |cdocsdrf.tex|&forwarding file for main file in draft mode\\
% |cdocsfi1.tex|&forwarding file for final version of chapter 1\\
% |cdocsfi2.tex|&forwarding file for final version of chapter 2\\
% \end{tabular}
% \end{center}
% Each of the eight files can be compiled directly by the \LaTeX{} compiler.
%
% %%%%%%%%%%%%%%%%%%%%%%%%%%%%%%%%%%%%%%
% \paragraph{Main File.}
%
% The main file is called |cdocsamp.tex|.
%
% Load the \textsf{childdoc} definitions and
% declare the filename for the main document:
%    \begin{macrocode}
\input{childdoc.def}
\childdocmain{}
%    \end{macrocode}

% Optional override for |\version| flag:
%    \begin{macrocode}
%%\ifchilddoc\else\providecommand{\version}{draft}\fi
%    \end{macrocode}

% Define the default values for the |\version| flag
% (|final| for the main file and |draft| for childs):
%    \begin{macrocode}
\ifchilddoc
\providecommand{\version}{draft}
\else
\providecommand{\version}{final}
\fi
%    \end{macrocode}

% Load the standard document class:
%    \begin{macrocode}
\documentclass[12pt]{article}
%    \end{macrocode}

% Start the document body:
%    \begin{macrocode}
\begin{document}
%    \end{macrocode}

% Declare a title page.
% Print title, part of document being processed and version flag:
%    \begin{macrocode}
\addtocounter{page}{-1}
\begin{center}
{\LARGE\bfseries{}childdoc example\par}
\vspace{1cm}
\ifchilddoc
\ifchilddocmanual part\else chapter\fi:
`\childdocname' of `\childdocjob'\par
\else
main document: `\childdocjob'\par
\fi
version: \version\par
\end{center}
\newpage
%    \end{macrocode}

% Manually include selected file,
% otherwise process as usual:
%    \begin{macrocode}
\ifchilddocmanual
\section*{part `\childdocname'}
\input{\childdocname}
\else
%    \end{macrocode}

% Include the two chapters:
%    \begin{macrocode}
\include{cdocsch1}
\include{cdocsch2}
%    \end{macrocode}

% Include the two parts unless only chapters should be displayed:
%    \begin{macrocode}
\ifchilddoc\else
\section{part three}
\input{cdocspt3}
\section{part four}
\input{cdocspt4}
\fi
%    \end{macrocode}

% Process as usual until here:
%    \begin{macrocode}
\fi
%    \end{macrocode}

% End of document body:
%    \begin{macrocode}
\end{document}
%    \end{macrocode}
%\iffalse
%</samplemain>
%\fi
%
% %%%%%%%%%%%%%%%%%%%%%%%%%%%%%%%%%%%%%%
% \paragraph{Chapter Include Files.}
%
% The include files are called |cdocsch1.tex| and |cdocsch2.tex|.
%
%\iffalse
%<*samplechap1|samplechap2>
%\fi

% Optional override for |\version| flag:
%    \begin{macrocode}
%%\providecommand{\version}{final}
%    \end{macrocode}

% Include the main document:
%    \begin{macrocode}
\input{childdoc.def}
\childdocof{cdocsamp}
%    \end{macrocode}

%\iffalse
%</samplechap1|samplechap2>
%\fi
%
%\iffalse
%<*samplechap1>
%\fi
% Some text for chapter 1:
%    \begin{macrocode}
\section{one}
some text in chapter one
%    \end{macrocode}

%\iffalse
%</samplechap1>
%\fi
% Some text for chapter 2:
%\iffalse
%<*samplechap2>
%\fi
%    \begin{macrocode}
\section{two}
more text in chapter two
%    \end{macrocode}

%\iffalse
%</samplechap2>
%\fi
%
% %%%%%%%%%%%%%%%%%%%%%%%%%%%%%%%%%%%%%%
% \paragraph{Part Include Files.}
%
% The include files are called |cdocspt3.tex| and |cdocspt4.tex|.
%
%\iffalse
%<*samplepart3|samplepart4>
%\fi

% Optional override for |\version| flag:
%    \begin{macrocode}
%%\providecommand{\version}{final}
%    \end{macrocode}

% Include the main document:
%    \begin{macrocode}
\input{childdoc.def}
\childdocby{cdocsamp}
%    \end{macrocode}

%\iffalse
%</samplepart3|samplepart4>
%\fi
%
%\iffalse
%<*samplepart3>
%\fi
% Some text for part 3:
%    \begin{macrocode}
some text in part three
%    \end{macrocode}

%\iffalse
%</samplepart3>
%\fi
% Some text for part 4:
%\iffalse
%<*samplepart4>
%\fi
%    \begin{macrocode}
more text in part four
%    \end{macrocode}

%\iffalse
%</samplepart4>
%\fi
%
% %%%%%%%%%%%%%%%%%%%%%%%%%%%%%%%%%%%%%%
% \paragraph{Forwarding for a Complete Draft.}
%
% The following forwarding file |cdocsdrf.tex|
% compiles the main document in draft mode:
%\iffalse
%<*sampledraft>
%\fi
%    \begin{macrocode}
\def\version{draft}
\input{childdoc.def}
\childdocforward{cdocsamp}
%    \end{macrocode}

%\iffalse
%</sampledraft>
%\fi
%
% %%%%%%%%%%%%%%%%%%%%%%%%%%%%%%%%%%%%%%
% \paragraph{Forwarding for Final Version of the Chapters.}
%
% The following forwarding files |cdocsfn1.tex| and |cdocsfn2.tex|
% (with identical content)
% compile the final versions of the child documents
% |cdocsch1.tex| and |cdocsch2.tex|, respectively:
%\iffalse
%<*samplefinal>
%\fi
%    \begin{macrocode}
\def\version{final}
\input{childdoc.def}
\childdocforwardprefix[cdocsamp]{cdocsfn}{cdocsch}
%    \end{macrocode}

%\iffalse
%</samplefinal>
%\fi
%
% %%%%%%%%%%%%%%%%%%%%%%%%%%%%%%%%%%%%%%
% \paragraph{Command Line Processing.}
%
% The following three command lines generate the output files
% |cdocscld|, |cdocscl1| and |cdocscl2|
% which should be identical to
% |cdocsdrf|, |cdocsch1| and |cdocsfn2|, respectively:
% \begin{center}
% \begin{tabular}{l}
% |latex -jobname cdocscld \|\\
% |  "\def\version{draft}\input{childdoc.def}\childdocforward{cdocsamp}"|\\
% |latex -jobname cdocscl1 \|\\
% |  "\input{childdoc.def}\childdocforward[cdocsamp]{cdocsch1}"|\\
% |latex -jobname cdocscl2 \|\\
% |  "\def\version{final}\input{childdoc.def}\childdocforward{cdocsch2}"|
% \end{tabular}
% \end{center}
% Note that the trailing backslash on each first line
% merely continues the input to the second line
% (for convenient cut ant paste).
% Furthermore, the command |latex| can be replaced by any
% of its alternative versions such as |pdflatex|.
%
% %%%%%%%%%%%%%%%%%%%%%%%%%%%%%%%%%%%%%%%%%%%%%%%%%%%%%%%%%%%%%%%%%%%%%%%%%%%%%%
% %%%%%%%%%%%%%%%%%%%%%%%%%%%%%%%%%%%%%%%%%%%%%%%%%%%%%%%%%%%%%%%%%%%%%%%%%%%%%%
% \section{Implementation}
%\iffalse
%<*package>
%\fi
%
% This section describes the definitions file |childdoc.def|.

% The definitions cannot be loaded using |\usepackage| or |\RequirePackage|
% which has a mechanism to prevent loading a style file more than once.
% When loading the definitions by means of |\input|
% multiple instances have to be prevented manually:
%\iffalse
%This code needs to be before the `\ProvidesFile' directive
%which is defined at the beginning of this file.
%Therefore it is also placed there and commented out here.
%</package>
%<*discard>
%\fi
%    \begin{macrocode}
\ifdefined\childdocmain\endinput\fi
%    \end{macrocode}
%\iffalse
%</discard>
%<*package>
%\fi
%
% \macro{\ifchilddoc}
% \macro{\ifchilddocmanual}
% The conditional |\ifchilddoc| tells whether a
% child (true) or main (false) document is being compiled.
% The conditional |\ifchilddocmanual| tells whether
% the |\includeonly| mechanism is used (false) or
% the selection of child files must be performed manually (true).
% The definitions initialise to false:
%    \begin{macrocode}
\newif\ifchilddoc
\newif\ifchilddocmanual
%    \end{macrocode}

% \macro{\childdocname}
% \macro{\childdocjob}
% The macro |\childdocname| stores the name of the main document
% to be compiled. The macro |\childdocjob| stores the name of
% the document on which the \LaTeX{} compiler was originally invoked.
% The content of |\jobname| cannot be compared
% to filenames specified in the source due to different catcodes.
% The following code rescans |\jobname|, stores the result
% in |\childdocname| and saves a copy in |\childdocjob|:
%    \begin{macrocode}
\edef\childdocname{\scantokens\expandafter{\jobname\noexpand}}
\let\childdocjob\childdocname
%    \end{macrocode}

% \macro{\childdocdisable}
% The macro |\childdocdisable| prevents the main file
% from being processed more than once.
% At this stage, the main document command |\childdocmain|
% is assumed to be called once again where it should do nothing.
% Any subsequent call to it should prevent
% a secondary processing of the main document
% It overwrites the forwarding commands
% |\childdocof| and |\childdocforward|
% with empty macros to prevent further inclusions of the main document:
%    \begin{macrocode}
\newcommand{\childdocdisable}
{
  \renewcommand{\childdocmain}[1]{\renewcommand{\childdocmain}[1]{\endinput}}
  \renewcommand{\childdocof}[1]{}
  \renewcommand{\childdocby}[2][]{}
  \renewcommand{\childdocforward}[2][]{}
  \renewcommand{\childdocdisable}{}
}
%    \end{macrocode}

% \macro{\childdocmain}
% The macro |\childdocmain| is to be called at the top of the main file
% with nothing or the main filename (without extension) as argument.
% First, it breaks loops.
% If the argument is not empty and does not match |\childdocname|
% (which is set by the first inclusion of |childdoc.def|),
% |\ifchilddoc| is set to true, |\includeonly| is applied to the child file
% and |\jobname| is set to the main file
% (for proper handling of |.aux| files):
%    \begin{macrocode}
\newcommand{\childdocmain}[1]
{
  \childdocdisable\childdocmain{}
  \if?#1?\else
    \begingroup
      \def\childdoctmp{#1}
      \ifx\childdoctmp\childdocname
        \def\childdoctmp{}
      \else
        \def\childdoctmp
        {
          \childdoctrue
          \includeonly{\childdocname}
          \def\childdocjob{#1}
          \def\jobname{#1}
        }
      \fi
      \expandafter
    \endgroup
    \childdoctmp
  \fi
}
%    \end{macrocode}

% \macro{\childdocof}
% The command |\childdocof| redirects
% compilation to the main file |#1|.
%    \begin{macrocode}
\newcommand{\childdocof}[1]
{
  \childdocdisable
  \childdoctrue
  \includeonly{\childdocname}
  \def\jobname{#1}
  \def\childdocjob{#1}
  \input{#1}
}
%    \end{macrocode}

% \macro{\childdocby}
% The command |\childdocby| ....
%    \begin{macrocode}
\newcommand{\childdocby}[2][]
{
  \childdocdisable
  \childdoctrue
  \childdocmanualtrue
  \if?#1?\else
    \def\jobname{#2}
  \fi
  \def\childdocjob{#2}
  \input{#2}
  \endinput
}
%    \end{macrocode}

% \macro{\childdocforward}
% The command |\childdocforward| redirects
% compilation to the main file or
% (if the optional argument is given) a child file.
% Parameters are set as if the main file
% or a child file starting with |\childdocof| was compiled.
% Then compilation is handed over to the main file:
%    \begin{macrocode}
\newcommand{\childdocforward}[2][]
{
  \begingroup
    \if?#1?
      \def\childdoctmp
      {
        \def\childdocname{#2}
        \def\childdocjob{#2}
        \def\jobname{#2}
        \input{#2}
        \endinput
      }
    \else
      \def\childdoctmp
      {
        \childdocdisable
        \def\childdocname{#2}
        \childdoctrue
        \includeonly{#2}
        \def\childdocjob{#1}
        \def\jobname{#1}
        \input{#1}
        \endinput
      }
    \fi
    \expandafter
  \endgroup
  \childdoctmp
}
%    \end{macrocode}

% \macro{\childdocforwardprefix}
% The command |\childdocforwardprefix| redirects
% compilation to the main or a child file by means of a pattern.
% The prefix |#1| in the current filename is replaced by |#2|
% and the suffix of the current filename is kept
% (it is assumed that the filename does not contain the substring `|~~~|'
% which is used as a delimiter).
% Compilation is handed over to the new file by |\childdocforward|:
%    \begin{macrocode}
\newcommand{\childdocforwardprefix}[3][]
{
  \begingroup
    \def\childdocextract #2##1~~~{\def\childdoctmp{\childdocforward[#1]{#3##1}}}
    \expandafter\childdocextract\childdocname~~~
    \expandafter
  \endgroup
  \childdoctmp
}
%    \end{macrocode}

% \macro{\childdoc}
% The deprecated macro |\childdoc| is a legacy version of |\childdocmain|:
%    \begin{macrocode}
\newcommand{\childdoc}{\childdocmain}
%    \end{macrocode}

% \macro{\childdocredirect}
% The deprecated macro |\childdocredirect| is a legacy version
% of |\childdocforward| and |\childdocforwardprefix|:
%    \begin{macrocode}
\newcommand{\childdocredirect}[2][]
{
  \begingroup
    \if?#1?
      \def\childdoctmp{\childdocforward{#2}}
    \else
      \def\childdoctmp{\childdocforwardprefix{#1}{#2}}
    \fi
    \expandafter
  \endgroup
  \childdoctmp
}
%    \end{macrocode}

%\iffalse
%</package>
%\fi
%
\endinput
|\\
|\childdocforward{|\textit{main}|}|\\
\end{tabular}
\end{center}
%
or alternatively with:
%
\begin{center}
\begin{tabular}{l}
|% \iffalse
%
% childdoc.dtx Copyright (C) 2017-2018 Niklas Beisert
%
% This work may be distributed and/or modified under the
% conditions of the LaTeX Project Public License, either version 1.3
% of this license or (at your option) any later version.
% The latest version of this license is in
%   http://www.latex-project.org/lppl.txt
% and version 1.3 or later is part of all distributions of LaTeX
% version 2005/12/01 or later.
%
% This work has the LPPL maintenance status `maintained'.
%
% The Current Maintainer of this work is Niklas Beisert.
%
% This work consists of the files childdoc.dtx and childdoc.ins
% and the derived files childdoc.def and cdocsamp.tex with
% cdocsch1.tex, cdocsch2.tex, cdocsdrf.tex, cdocsfn1.tex, cdocsfn2.tex.
%
%<package>\ifdefined\childdocmain\endinput\fi
%<package>\ProvidesFile{childdoc.def}[2018/12/30 v2.0 child document driver]
%<samplemain>\ProvidesFile{cdocsamp.tex}[2018/12/30 v2.0 sample for childdoc]
%<*driver>
%\ProvidesFile{childdoc.drv}[2018/12/30 v2.0 childdoc reference manual file]
\PassOptionsToClass{10pt,a4paper}{article}
\documentclass{ltxdoc}

\usepackage[margin=35mm]{geometry}
\usepackage{hyperref}
\usepackage{hyperxmp}
\usepackage[usenames]{color}

\hypersetup{colorlinks=true}
\hypersetup{pdfstartview=FitH}
\hypersetup{pdfpagemode=UseNone}
\hypersetup{pdfsource={}}
\hypersetup{pdflang={en-UK}}
\hypersetup{pdfcopyright={Copyright 2017-2018 Niklas Beisert.
  This work may be distributed and/or modified under the
  conditions of the LaTeX Project Public License, either version 1.3
  of this license or (at your option) any later version.}}
\hypersetup{pdflicenseurl={http://www.latex-project.org/lppl.txt}}
\hypersetup{pdfcontactaddress={ETH Zurich, ITP, HIT K,
  Wolfgang-Pauli-Strasse 27}}
\hypersetup{pdfcontactpostcode={8093}}
\hypersetup{pdfcontactcity={Zurich}}
\hypersetup{pdfcontactcountry={Switzerland}}
\hypersetup{pdfcontactemail={nbeisert@itp.phys.ethz.ch}}
\hypersetup{pdfcontacturl={http://people.phys.ethz.ch/\xmptilde nbeisert/}}

\newcommand{\secref}[1]{\hyperref[#1]{section \ref*{#1}}}

\parskip1ex
\parindent0pt
\let\olditemize\itemize
\def\itemize{\olditemize\parskip0pt}

\begin{document}

\title{The \textsf{childdoc} Package}
\hypersetup{pdftitle={The childdoc Package}}
\author{Niklas Beisert\\[2ex]
  Institut f\"ur Theoretische Physik\\
  Eidgen\"ossische Technische Hochschule Z\"urich\\
  Wolfgang-Pauli-Strasse 27, 8093 Z\"urich, Switzerland\\[1ex]
  \href{mailto:nbeisert@itp.phys.ethz.ch}
  {\texttt{nbeisert@itp.phys.ethz.ch}}}
\hypersetup{pdfauthor={Niklas Beisert}}
\hypersetup{pdfsubject={Manual for the LaTeX2e Package childdoc}}
\date{30 December 2018, \textsf{v2.0}}
\maketitle

\begin{abstract}\noindent
\textsf{childdoc} is a \LaTeXe{} package
that enables the direct compilation
of document sections included by |\include|
to individual files.
\end{abstract}

\begingroup
\parskip0ex
\tableofcontents
\endgroup

%%%%%%%%%%%%%%%%%%%%%%%%%%%%%%%%%%%%%%%%%%%%%%%%%%%%%%%%%%%%%%%%%%%%%%%%%%%%%%%%
%%%%%%%%%%%%%%%%%%%%%%%%%%%%%%%%%%%%%%%%%%%%%%%%%%%%%%%%%%%%%%%%%%%%%%%%%%%%%%%%
\section{Introduction}

\LaTeX{} provides a mechanism to structure a large document (such as a book)
into a main file and several child files (containing the chapters)
using the |\include| command.
This mechanism is beneficial for documents
which span hundreds of pages in order to
make the source file(s) more manageable.
Moreover, compilation can be restricted to
selected child files by means of the |\includeonly| command.
The latter feature can be used to reduce the compilation time while editing
(this was significantly more useful in the earlier days of \LaTeX{})
or to generate a smaller document which is easier to navigate.
Another application of |\includeonly| is to generate
documents consisting of selected parts of the complete document.

However, there are a few drawbacks of the plain |\include| mechanism:
\begin{itemize}
\item
The child files cannot be compiled on their own,
they can only be compiled via the main file.
A naive editing environment
(such as a text editor with an option
to have the current file processed by \LaTeX)
may require one to switch to the main file before compiling;
attempting to compile the child file produces errors.
\item
The main file must be modified (each time)
to adjust the |\includeonly| command
to the present needs. This easily leaves the main file in a messy state.
\item
The generated document will always carry the filename
of the main document. This is inconvenient if
several child files are to be compiled and
to be kept for distribution.
\end{itemize}

The present package provides a simple interface
to make child files individually compilable by \LaTeX{}.
Compiling a child file then has the same effect as compiling
the main file with an |\includeonly| command
to select the appropriate child.
Moreover the generated document will carry the name of the child
rather than the main file.
This resolves all three above issues.

This feature is meant to make the editing of books,
thesis documents and lecture notes somewhat more convenient.
However, the package can also be used efficiently for
composing a series of documents (such as exercise sheets)
which are typically distributed individually.
It then assists the author in generating the individual documents
(potentially in different versions)
as well as a document containing the collected series.
Another application is in developing style files
or other kinds of included material
where compilation of the style file could redirect
to a sample or test file.

%%%%%%%%%%%%%%%%%%%%%%%%%%%%%%%%%%%%%%%%%%%%%%%%%%%%%%%%%%%%%%%%%%%%%%%%%%%%%%%%
%%%%%%%%%%%%%%%%%%%%%%%%%%%%%%%%%%%%%%%%%%%%%%%%%%%%%%%%%%%%%%%%%%%%%%%%%%%%%%%%
\section{Usage}

First of all, the package \textsf{childdoc} is \emph{not} a standard
\LaTeXe{} |.sty| style file! Therefore it needs to be invoked in
a non-standard way.

%%%%%%%%%%%%%%%%%%%%%%%%%%%%%%%%%%%%%%%%%%%%%%%%%%%%%%%%%%%%%%%%%%%%%%%%%%%%%%%%
\subsection{Included Files}
\label{sec:include}

%%%%%%%%%%%%%%%%%%%%%%%%%%%%%%%%%%%%%%%%
\DescribeMacro{\childdocmain}
To use the package, add the commands
\begin{center}
\begin{tabular}{l}
|\input{childdoc.def}|\\
|\childdocmain{}|\\
\end{tabular}
\end{center}
at the very top of the main \LaTeX{} file,
in particular \emph{before} the |\documentclass| statement!
The argument of |\childdocmain| should be left empty
(but it must be present).

%%%%%%%%%%%%%%%%%%%%%%%%%%%%%%%%%%%%%%%%
\DescribeMacro{\childdocof}
Furthermore, add the commands
\begin{center}
\begin{tabular}{l}
|\input{childdoc.def}|\\
|\childdocof{|\textit{main}|}|\\
\end{tabular}
\end{center}
at the top of every child file \textit{child}
which is included by |\include{|\textit{child}|}|
from within the main file
(or at least for those files to be compiled individually).
The argument \textit{main} must be the filename of the main file.

There are a couple of
considerations in setting up the main and child documents:

%%%%%%%%%%%%%%%%%%%%%%%%%%%%%%%%%%%%%%%%
\paragraph{Restrictions.}

Please note the following restrictions:
\begin{itemize}
\item
|\childdocmain| must be called with one argument \textit{main}
to ensure compatibility with earlier version of the package.
It must either be empty (|\childdocmain{}|)
or precisely match the filename of the main file in which it is specified.
See \secref{sec:detection} for further information.
\item
The filename \textit{main} must be specified without the |.tex| extension.
\item
The filename \textit{main} is case sensitive
(even in case-insensitive file systems)
due to internal string comparison.
\item
The argument \textit{main} should be fully expanded, it cannot be a macro.
\item
Subdirectories and special characters should be avoided in filenames.
\item
The command |\childdocmain{|\textit{main}|}| must be followed by a whitespace.
It should not be followed immediately by another command
or by a comment mark `|%|'.
This is because the \TeX{} parser reads the token immediately following
the argument of |\childdocmain| and puts it
at the beginning of every child section;
however, a white\-space is ignored.
\end{itemize}

%%%%%%%%%%%%%%%%%%%%%%%%%%%%%%%%%%%%%%%%
\paragraph{Content of Main File.}

It is advisable to place all content in the child files included by |\include|.
Any output contained in the main file will appear in all child documents
unless suppressed manually;
it cannot be suppressed automatically by the |\includeonly| directive
and thus should normally be avoided.
A method to include some content in the main file
by means of conditional processing is described in \secref{sec:conditional}.

%%%%%%%%%%%%%%%%%%%%%%%%%%%%%%%%%%%%%%%%
\paragraph{Page Numbering.}

When only a part of the document is compiled,
the appropriate numbering of pages
(as well as other status parameters)
is determined from the |.aux| files.
The latter contain information from previous passes.
However this information needs to propagate through
all intermediate child documents.
Therefore the page numbering in child documents may well
be inconsistent until the complete document is compiled at least once.

A useful (if unconventional) way to always ensure a consistent
page numbering is to restart the numbering in each child document
and denote the pages by `\textit{child}|.|\textit{page}'
where \textit{child} represents the chapter/section number of the child file.
This can be achieved by the command
|\numberwithin{page}{|\textit{child}|}|
of the \textsf{amsmath} package
where \textit{child} can be |chapter| or |section|
depending on the chosen structuring.
Alternatively, one can modify the macro |\thepage| appropriately
and reset the counter |page| at the start of each child file.

%%%%%%%%%%%%%%%%%%%%%%%%%%%%%%%%%%%%%%%%%%%%%%%%%%%%%%%%%%%%%%%%%%%%%%%%%%%%%%%%
\subsection{Conditional Processing}
\label{sec:conditional}

The package provides a mechanism to compile different versions
of a document. To customise the versions further some conditional processing
can come in handy to distinguish which version is being compiled.
The package provides two macros to describe the compilation context:

%%%%%%%%%%%%%%%%%%%%%%%%%%%%%%%%%%%%%%%%
\DescribeMacro{\ifchilddoc}
The conditional |\ifchilddoc| distinguishes between the compilation of
child documents and the main document:
%
\begin{center}
|\ifchilddoc |\textit{child-code}| |[|\||else |\textit{main-code}]| \||fi|
\end{center}

%%%%%%%%%%%%%%%%%%%%%%%%%%%%%%%%%%%%%%%%
\DescribeMacro{\childdocname}
\DescribeMacro{\childdocjob}
The macro |\childdocname| contains the filename (without extension)
of the main or child file being processed.
Note that |\childdocjob| will always contain the name of the main file.

%%%%%%%%%%%%%%%%%%%%%%%%%%%%%%%%%%%%%%%%
\paragraph{Title Page.}

Conditional processing can be used to include a title or banner page
in the main document when proper precautions are taken.
Importantly, the code in the main file should ensure that the page counter
(as well as other status parameters which are stored in the |.aux| files)
takes the same value after the conditional processing.
Otherwise the page numbers may take divergent values
depending on which part is compiled.

For example, a title page could be declared by:
%
\begin{center}
\begin{tabular}{l}
|\ifchilddoc\||else|\\
|\addtocounter{page}{-1}|\\
\textit{code for title page}\\
|\newpage|\\
|\||fi|
\end{tabular}
\end{center}
%
A banner page for the child documents can be generated by:
%
\begin{center}
\begin{tabular}{l}
|\ifchilddoc|\\
|\addtocounter{page}{-1}|\\
\textit{code for banner page}\\
|\newpage|\\
|\||fi|
\end{tabular}
\end{center}
%
Here one could write a message such as:
\begin{center}
|This is the part \childdocname{} of \childdocjob{}.|
\end{center}

%%%%%%%%%%%%%%%%%%%%%%%%%%%%%%%%%%%%%%%%%%%%%%%%%%%%%%%%%%%%%%%%%%%%%%%%%%%%%%%%
\subsection{Flags}
\label{sec:flags}

The package makes it easy to generate different versions
of the main or child documents.
To this end compilation flags can be defined
and assigned different default values.
They will be particularly useful in conjunction
with the forwarding mechanism described in \secref{sec:forward}.

For example, it may be useful to have a flag |\version|
which can be set to |draft| or |final|.
The document source will contain some conditional code
depending on the value of |\version|.
Suppose further, the flag should default to |final| for the main file
and to |draft| for child files
which is a natural assignment for editing the document.
This is achieved by placing the following code
in the preamble of the main document
(below the |\childdocmain| directive):
%
\begin{center}
\begin{tabular}{l}
|\ifchilddoc|\\
|\providecommand{\version}{draft}|\\
|\||else|\\
|\providecommand{\version}{final}|\\
|\||fi|
\end{tabular}
\end{center}
%
The definition by |\providecommand| makes sure
that previous definitions are not overwritten.
Further statements |\providecommand{\version}{...}|
can thus be added before the above code to override it.

For the main file, one might add a line
(between |\childdocmain| and the above block)
%
\begin{center}
|%\ifchilddoc\||else\providecommand{\version}{draft}\||fi|
\end{center}
%
which can be uncommented to produce a draft version.
Likewise one can add a line to the very top of a child file
(above the |\childdocof{|\textit{main}|}| directive)
%
\begin{center}
|%\providecommand{\version}{final}|
\end{center}
%
which can be uncommented to produce the final version of this child document.

%%%%%%%%%%%%%%%%%%%%%%%%%%%%%%%%%%%%%%%%%%%%%%%%%%%%%%%%%%%%%%%%%%%%%%%%%%%%%%%%
\subsection{Forwarding}
\label{sec:forward}

Different versions of the main or child documents
using compilation flags as described in \secref{sec:flags}
can be (permanently) stored in different files
for convenient compilation, viewing and distribution.
To this end, the package defines a command
to pass on compilation to a different file:

%%%%%%%%%%%%%%%%%%%%%%%%%%%%%%%%%%%%%%%%
\DescribeMacro{\childdocforward}
The command |\childdocforward| redirects processing to
another source file:
%
\begin{center}
\begin{tabular}{l}
|\input{childdoc.def}|\\
|\childdocforward[|\textit{main}|]{|\textit{dest}|}|\\
\end{tabular}
\end{center}
%
The argument \textit{dest} is the destination file
(without extension).
It should be the main file or one of the child files.
Note that further \textsf{childdoc} directives
such as |\childdocof| and |\childdocforward|
in the indicated file will be processed in this form.
The optional argument \textit{main}
passes on directly to the main file \textit{main}
while pretending to compile the child \textit{dest}.
This form behaves as if \textit{dest}
issues |\childdocof{|\textit{main}|}| right away,
and no further \textsf{childdoc} directives will be processed.

%%%%%%%%%%%%%%%%%%%%%%%%%%%%%%%%%%%%%%%%
\DescribeMacro{\...prefix}
In the alternative form |\childdocforwardprefix|,
%
\begin{center}
\begin{tabular}{l}
|\input{childdoc.def}|\\
|\childdocforwardprefix[|\textit{main}|]{|\textit{prefix}|}{|\textit{dest}|}|
\end{tabular}
\end{center}
%
the destination file is determined by a pattern
depending on the current file:
To make this work, the current file must be called
`{\textit{prefix}\hspace{0.2em}\textit{suffix}}'
with \textit{prefix} matching precisely the argument.
Processing is then passed on to the file
`{\textit{dest}\hspace{0.2em}\textit{suffix}}'.
Surely, the same effect is achieved by
directly specifying the
argument `{\textit{dest}\hspace{0.2em}\textit{suffix}}'
in the first form.
However, that requires to set up a different file
for each child. With the alternative form of the command
all these files can have exactly the same content
which simplifies setting them up and maintaining them.

For example, the following file |draft.tex|
with a compilation flag |\version| as described in \secref{sec:flags}
compiles the main document as a draft:
%
\begin{center}
\begin{tabular}{l}
|\def\version{draft}|\\
|\input{childdoc.def}|\\
|\childdocforward{|\textit{main}|}|
\end{tabular}
\end{center}
%
Likewise, the following files |final|\textit{nn}|.tex|
compile the final version of the child document
|child|\textit{nn}|.tex|:
%
\begin{center}
\begin{tabular}{l}
|\def\version{final}|\\
|\input{childdoc.def}|\\
|\childdocforwardprefix{final}{child}|
\end{tabular}
\end{center}
%

Note that when several versions of a main file and/or of each child file
are to be generated, it may be convenient to set up a |Makefile| or
shell script to automatise the process.

%%%%%%%%%%%%%%%%%%%%%%%%%%%%%%%%%%%%%%%%%%%%%%%%%%%%%%%%%%%%%%%%%%%%%%%%%%%%%%%%
\subsection{Command Line Processing}
\label{sec:commandline}

The effect of redirection files can also be achieved by invoking
the \LaTeX{} compiler with a more elaborate command line.
Most conveniently this should be done as part
of a shell script or a |Makefile|.

When using \textsf{childdoc} in the main file, the following
command lines effectively perform a redirection
(note that depending on the shell being used,
backslashes may have to be doubled: `|\|' $\to$ `|\\|'):
%
\begin{center}
|... -jobname "|\textit{target}|" |\\|"|[\textit{flags}]%
|\input{childdoc.def}\childdocforward[|\textit{main}|]{|\textit{dest}|}"|
\end{center}
%
Here \textit{target} is the name of the output file,
\textit{main} is the name of the main file
and \textit{dest} is the name of the main or child file to be processed
(all filenames without extensions).
The optional argument \textit{main} can be omitted
if \textit{main} matches \textit{dest}.
Optionally, compilation \textit{flags} can be defined via |\def| commands.
This command line makes the \TeX{} engine believe
it is compiling the file \textit{target}
whose content is specified as the latter parameter.
The provided code then forwards the processing to
\textit{main} or \textit{dest} as described in \secref{sec:forward}.

%%%%%%%%%%%%%%%%%%%%%%%%%%%%%%%%%%%%%%%%%%%%%%%%%%%%%%%%%%%%%%%%%%%%%%%%%%%%%%%%
\subsection{Include by Input}
\label{sec:input}

Including child documents by |\include| has some restrictions by design.
Most notably, the content of a child document always occupies
its own set of pages; pages cannot be shared between child documents.
Usually, this behaviour makes perfect sense
because each child document contain an essential part of the document.
However, in some situations it may be desirable to compose
a document from a collection of parts
without having mandatory page breaks between then.
For this case, the package
provides a mechanism to include parts
by |\input| which can also be processed individually.
However, by construction this mechanism
requires manual handling of the content to be output.

%%%%%%%%%%%%%%%%%%%%%%%%%%%%%%%%%%%%%%%%
\DescribeMacro{\ifchilddocmanual}
The main file should be prepared as usual, see \secref{sec:include}.
However, the document body must make a distinction
between processing of an individual part and of the main document, e.g.:
%
\begin{center}
\begin{tabular}{l}
|\ifchilddocmanual|\\
|\input{\childdocname}|\\
|\||else|\\
\textit{document body with }|\input{|\textit{part}|}|\\
|\||fi|
\end{tabular}
\end{center}
%
The conditional |\ifchilddocmanual| is true whenever
a part to be included by |\input| is being compiled,
and the name of the part is stored in |\childdocname|.

%%%%%%%%%%%%%%%%%%%%%%%%%%%%%%%%%%%%%%%%
\DescribeMacro{\childdocby}
Each part to be included by |\input| should start with:
%
\begin{center}
\begin{tabular}{l}
|\input{childdoc.def}|\\
|\childdocby{|\textit{main}|}|\\
\end{tabular}
\end{center}
%
The directive |\childdocby| is similar to |\childdocof|
described in \secref{sec:include},
but the subsequent selection of content must be done manually.
To that end, both |\ifchilddoc| and |\ifchilddocmanual|
will be true upon processing of a part,
and the name of the part is stored in |\childdocname|.
Note that |\jobname| will be set to the filename of the current part
so that each part receives an individual |.aux| file
that does not interfere with the |.aux| file(s) of the main document.
This behaviour can be altered by the alternative form
|\childdocby[*]{|\textit{main}|}| (with a non-empty optional argument)
which uses the |.aux| file of the main document
by setting |\jobname| to \textit{main}.

%%%%%%%%%%%%%%%%%%%%%%%%%%%%%%%%%%%%%%%%%%%%%%%%%%%%%%%%%%%%%%%%%%%%%%%%%%%%%%%%
\subsection{Driver Development}
\label{sec:driver}

The \textsf{childdoc} mechanism can also be use for the development
of definition files such as \LaTeX{} styles or classes.
This case differs from the above setup with multiple parts
included by |\include| in that no |\includeonly| should be invoked.
This can be achieved by starting the include file
(before |\ProvidesPackage|) with:
%
\begin{center}
\begin{tabular}{l}
|\input{childdoc.def}|\\
|\childdocforward{|\textit{main}|}|\\
\end{tabular}
\end{center}
%
or alternatively with:
%
\begin{center}
\begin{tabular}{l}
|\input{childdoc.def}|\\
|\childdocby{|\textit{main}|}|\\
\end{tabular}
\end{center}
%
Both forms have slightly different effects as described above.
The main file is prepared as usual, see \secref{sec:include}.

%%%%%%%%%%%%%%%%%%%%%%%%%%%%%%%%%%%%%%%%%%%%%%%%%%%%%%%%%%%%%%%%%%%%%%%%%%%%%%%%
\subsection{Legacy Detection}
\label{sec:detection}

The directive |\childdocmain| in the main file can detect
whether the complete document or merely a child is to be compiled
even without using the directive |\childdocof|.
This method is deprecated because it is less robust
and there is no compelling reason to use it;
it is merely provided for backward compatibility
and it may be removed in future versions.

If the detection mechanism is to be used,
it is mandatory to correctly specify
the filename of the main file as the argument of |\childdocmain|:
%
\begin{center}
\begin{tabular}{l}
|\input{childdoc.def}|\\
|\childdocmain{|\textit{main}|}|\\
\end{tabular}
\end{center}
%
If |\jobname| does not match the argument \textit{main} of |\childdocmain|,
it is assumed that |\jobname| points to the child file to be compiled.
When using |\childdocmain| with the main file specified as argument,
it suffices to start a child file
with just |\input{|\textit{main}|}|
without loading of the package and using |\childdocof|.
If instead all processing is done
with the appropriate \textsf{childdoc} directives,
the argument of \textit{main} of |\childdocmain| can be empty.

An alternative version of the command line processing described
in \secref{sec:commandline} using the detection mechanism reads:
%
\begin{center}
|... -jobname "|\textit{target}|" "|[\textit{flags}]%
[|\def\jobname{|\textit{dest}|}|]|\input{|\textit{main}|}"|
\end{center}

%%%%%%%%%%%%%%%%%%%%%%%%%%%%%%%%%%%%%%%%%%%%%%%%%%%%%%%%%%%%%%%%%%%%%%%%%%%%%%%%
\subsection{Manual Code}
\label{sec:manual}

In case one cannot be certain whether the definitions file |childdoc.def|
is installed on the target \TeX{} distribution
and one prefers not to ship it,
it is conceivable to paste a few relevant commands into the sources.

To that end, drop all statements |\input{childdoc.def}|
and perform the replacements as outlined below.
Instead of |\childdocmain{|\textit{main}|}| add the following code
to the top of the main file:
%
\begin{center}
\begin{tabular}{l}
|\||ifdefined\childdocname\endinput\||fi\newif\ifchilddoc|\\
|\edef\childdocname{\scantokens\expandafter{\jobname\noexpand}}|\\
|\def\childdocmain{|\textit{main}|}\||ifx\childdocmain\childdocname\||else|\\
|\childdoctrue\includeonly{\childdocname}\let\jobname\childdocmain\||fi|\\
\end{tabular}
\end{center}
%
Instead of |\childdocof{|\textit{main}|}| just include the main file
at the top of each child file:
%
\begin{center}
|\input{|\textit{main}|}|
\end{center}
%
A simple redirection |\childdocforward{|\textit{dest}|}| is achieved by:
%
\begin{center}
|\def\jobname{|\textit{dest}|}\input{\jobname}|
\end{center}
%
The redirection with prefix
|\childdocforwardprefix[|\textit{prefix}|]{|\textit{dest}|}|
is accomplished by:
%
\begin{center}
\begin{tabular}{l}
|{\edef\jobname{\scantokens\expandafter{\jobname\noexpand}}|\\
|\def\redirectjob |\textit{prefix}|#1~~~{\gdef\jobname{|\textit{dest}|#1}}|\\
|\expandafter\redirectjob\jobname~~~}\input{\jobname}|
\end{tabular}
\end{center}

In an alternative approach,
child documents can be compiled by a specific command line
without additional code or specific definitions:
%
\begin{center}
|... -jobname "|\textit{target}|" "|[\textit{flags}]%
|\includeonly{|\textit{dest}|}\input{|\textit{main}|}"|
\end{center}
%

%%%%%%%%%%%%%%%%%%%%%%%%%%%%%%%%%%%%%%%%%%%%%%%%%%%%%%%%%%%%%%%%%%%%%%%%%%%%%%%%
%%%%%%%%%%%%%%%%%%%%%%%%%%%%%%%%%%%%%%%%%%%%%%%%%%%%%%%%%%%%%%%%%%%%%%%%%%%%%%%%
\section{Information}

%%%%%%%%%%%%%%%%%%%%%%%%%%%%%%%%%%%%%%%%%%%%%%%%%%%%%%%%%%%%%%%%%%%%%%%%%%%%%%%%
\subsection{Copyright}

Copyright \copyright{} 2017--2018 Niklas Beisert

This work may be distributed and/or modified under the
conditions of the \LaTeX{} Project Public License, either version 1.3
of this license or (at your option) any later version.
The latest version of this license is in
  \url{http://www.latex-project.org/lppl.txt}
and version 1.3 or later is part of all distributions of \LaTeX{}
version 2005/12/01 or later.

This work has the LPPL maintenance status `maintained'.

The Current Maintainer of this work is Niklas Beisert.

This work consists of the files |README.txt|, |childdoc.ins| and |childdoc.dtx|
as well as the derived files |childdoc.def|, |cdocsamp.tex|
with |cdocsch1.tex|, |cdocsch2.tex|, |cdocspt3.tex|, |cdocspt4.tex|,
|cdocsdrf.tex|, |cdocsfn1.tex|, |cdocsfn2.tex|
as well as |childdoc.pdf|.

%%%%%%%%%%%%%%%%%%%%%%%%%%%%%%%%%%%%%%%%%%%%%%%%%%%%%%%%%%%%%%%%%%%%%%%%%%%%%%%%
\subsection{Files and Installation}

The package consists of the files:
%
\begin{center}
\begin{tabular}{ll}
    |README.txt|   & readme file \\
    |childdoc.ins| & installation file \\
    |childdoc.dtx| & source file \\
    |childdoc.def| & definition file \\
    |cdocsamp.tex| & sample main file \\
    |cdocsch1.tex| & sample include file \\
    |cdocsch2.tex| & sample include file \\
    |cdocspt3.tex| & sample part file \\
    |cdocspt4.tex| & sample part file \\
    |cdocsdrf.tex| & sample redirection file \\
    |cdocsfn1.tex| & sample redirection file \\
    |cdocsfn2.tex| & sample redirection file \\
    |childdoc.pdf| & manual
\end{tabular}
\end{center}
%
The distribution consists of the files
|README.txt|, |childdoc.ins| and |childdoc.dtx|.
%
\begin{itemize}
\item
Run (pdf)\LaTeX{} on |childdoc.dtx|
to compile the manual |childdoc.pdf| (this file).
\item
Run \LaTeX{} on |childdoc.ins| to create the definitions file |childdoc.def|
and the sample |cdocsamp.tex| with include files
|cdocsch1.tex|, |cdocsch2.tex|, |cdocspt3.tex|, |cdocspt4.tex|,
|cdocsdrf.tex|, |cdocsfn1.tex|, |cdocsfn2.tex|.
Then copy the file |childdoc.def| to an appropriate directory of your \LaTeX{}
distribution, e.g.\ \textit{texmf-root}|/tex/latex/childdoc|.
\end{itemize}

%%%%%%%%%%%%%%%%%%%%%%%%%%%%%%%%%%%%%%%%%%%%%%%%%%%%%%%%%%%%%%%%%%%%%%%%%%%%%%%%
\subsection{Related CTAN Packages}

There are several other packages which offer a similar functionality:
%
\begin{itemize}
\item
The packages
\href{http://ctan.org/pkg/docmute}{\textsf{docmute}},
\href{http://ctan.org/pkg/includex}{\textsf{includex}} and
\href{http://ctan.org/pkg/standalone}{\textsf{standalone}}
provide commands to include only the document body of
a child file thus allowing both files to be compiled individually.
\item
The packages \href{http://ctan.org/pkg/subdocs}{\textsf{subdocs}}
and \href{http://ctan.org/pkg/subfiles}{\textsf{subfiles}}
provide structures in which the main and child documents can be
encapsulated and allowing them to be compiled individually.
The inclusion mechanism is different from the conventional |\include|.
\item
The package \href{http://ctan.org/pkg/combine}{\textsf{combine}}
is an elaborate solution to combine several documents into one.
\end{itemize}
%
See also the CTAN topic \href{http://ctan.org/topic/subdocs}{\textsf{subdocs}}
for further related packages.
The present package differs from the above solutions in that
a document structure constructed with the conventional |\include| mechanism
just needs two extra commands at the top of every file
such that all constituent files can be compiled individually.

%%%%%%%%%%%%%%%%%%%%%%%%%%%%%%%%%%%%%%%%%%%%%%%%%%%%%%%%%%%%%%%%%%%%%%%%%%%%%%%%
%\subsection{Feature Suggestions}
%
%The following is a list of features which may be useful for future
%versions of this package:
%%
%\begin{itemize}
%\item
%\ldots
%\end{itemize}

%%%%%%%%%%%%%%%%%%%%%%%%%%%%%%%%%%%%%%%%%%%%%%%%%%%%%%%%%%%%%%%%%%%%%%%%%%%%%%%%
\subsection{Revision History}

%%%%%%%%%%%%%%%%%%%%%%%%%%%%%%%%%%%%%%%%
\paragraph{v2.0:} 2018/12/30

\begin{itemize}
\item
immediate forward processing
\item
added |\childdocby| mechanism
\item
manual restructured
\end{itemize}

%%%%%%%%%%%%%%%%%%%%%%%%%%%%%%%%%%%%%%%%
\paragraph{v1.6:} 2018/01/17

\begin{itemize}
\item
application for development of include files
\item
corrections to manual
\end{itemize}

%%%%%%%%%%%%%%%%%%%%%%%%%%%%%%%%%%%%%%%%
\paragraph{v1.5:} 2017/05/21

\begin{itemize}
\item
more complete structuring introduced
\item
|\childdocof| introduced
\item
|\childdoc| renamed to |\childdocmain|
\item
|\childredirect| renamed to |\childdocforward| and |\childdocforwardprefix|
and functionality expanded
\end{itemize}

%%%%%%%%%%%%%%%%%%%%%%%%%%%%%%%%%%%%%%%%
\paragraph{v1.0:} 2017/04/27

\begin{itemize}
\item
manual and install package
\item
first version published on CTAN
\end{itemize}

%%%%%%%%%%%%%%%%%%%%%%%%%%%%%%%%%%%%%%%%
\paragraph{v0.6:} 2017/04/26

\begin{itemize}
\item
redirection mechanism added
\end{itemize}

%%%%%%%%%%%%%%%%%%%%%%%%%%%%%%%%%%%%%%%%
\paragraph{v0.5:} 2017/04/26

\begin{itemize}
\item
functionality in definition file
\end{itemize}


%%%%%%%%%%%%%%%%%%%%%%%%%%%%%%%%%%%%%%%%%%%%%%%%%%%%%%%%%%%%%%%%%%%%%%%%%%%%%%%%
%%%%%%%%%%%%%%%%%%%%%%%%%%%%%%%%%%%%%%%%%%%%%%%%%%%%%%%%%%%%%%%%%%%%%%%%%%%%%%%%
%%%%%%%%%%%%%%%%%%%%%%%%%%%%%%%%%%%%%%%%%%%%%%%%%%%%%%%%%%%%%%%%%%%%%%%%%%%%%%%%
\appendix

\settowidth\MacroIndent{\rmfamily\scriptsize 000\ }

 \DocInput{childdoc.dtx}

\end{document}
%</driver>
% \fi
%
% %%%%%%%%%%%%%%%%%%%%%%%%%%%%%%%%%%%%%%%%%%%%%%%%%%%%%%%%%%%%%%%%%%%%%%%%%%%%%%
% %%%%%%%%%%%%%%%%%%%%%%%%%%%%%%%%%%%%%%%%%%%%%%%%%%%%%%%%%%%%%%%%%%%%%%%%%%%%%%
% \section{Sample}
%\iffalse
%<*samplemain>
%\fi
%
% The following presents a sample document
% with two chapters, two parts, a title page,
% a compile flag as well as three forwarding files to set the flag.
% It consists of eight |.tex| files:
% \begin{center}
% \begin{tabular}{ll}
% |cdocsamp.tex|&main file\\
% |cdocsch1.tex|&include file for chapter 1\\
% |cdocsch2.tex|&include file for chapter 2\\
% |cdocspt3.tex|&include file for part 3\\
% |cdocspt4.tex|&include file for part 4\\
% |cdocsdrf.tex|&forwarding file for main file in draft mode\\
% |cdocsfi1.tex|&forwarding file for final version of chapter 1\\
% |cdocsfi2.tex|&forwarding file for final version of chapter 2\\
% \end{tabular}
% \end{center}
% Each of the eight files can be compiled directly by the \LaTeX{} compiler.
%
% %%%%%%%%%%%%%%%%%%%%%%%%%%%%%%%%%%%%%%
% \paragraph{Main File.}
%
% The main file is called |cdocsamp.tex|.
%
% Load the \textsf{childdoc} definitions and
% declare the filename for the main document:
%    \begin{macrocode}
\input{childdoc.def}
\childdocmain{}
%    \end{macrocode}

% Optional override for |\version| flag:
%    \begin{macrocode}
%%\ifchilddoc\else\providecommand{\version}{draft}\fi
%    \end{macrocode}

% Define the default values for the |\version| flag
% (|final| for the main file and |draft| for childs):
%    \begin{macrocode}
\ifchilddoc
\providecommand{\version}{draft}
\else
\providecommand{\version}{final}
\fi
%    \end{macrocode}

% Load the standard document class:
%    \begin{macrocode}
\documentclass[12pt]{article}
%    \end{macrocode}

% Start the document body:
%    \begin{macrocode}
\begin{document}
%    \end{macrocode}

% Declare a title page.
% Print title, part of document being processed and version flag:
%    \begin{macrocode}
\addtocounter{page}{-1}
\begin{center}
{\LARGE\bfseries{}childdoc example\par}
\vspace{1cm}
\ifchilddoc
\ifchilddocmanual part\else chapter\fi:
`\childdocname' of `\childdocjob'\par
\else
main document: `\childdocjob'\par
\fi
version: \version\par
\end{center}
\newpage
%    \end{macrocode}

% Manually include selected file,
% otherwise process as usual:
%    \begin{macrocode}
\ifchilddocmanual
\section*{part `\childdocname'}
\input{\childdocname}
\else
%    \end{macrocode}

% Include the two chapters:
%    \begin{macrocode}
\include{cdocsch1}
\include{cdocsch2}
%    \end{macrocode}

% Include the two parts unless only chapters should be displayed:
%    \begin{macrocode}
\ifchilddoc\else
\section{part three}
\input{cdocspt3}
\section{part four}
\input{cdocspt4}
\fi
%    \end{macrocode}

% Process as usual until here:
%    \begin{macrocode}
\fi
%    \end{macrocode}

% End of document body:
%    \begin{macrocode}
\end{document}
%    \end{macrocode}
%\iffalse
%</samplemain>
%\fi
%
% %%%%%%%%%%%%%%%%%%%%%%%%%%%%%%%%%%%%%%
% \paragraph{Chapter Include Files.}
%
% The include files are called |cdocsch1.tex| and |cdocsch2.tex|.
%
%\iffalse
%<*samplechap1|samplechap2>
%\fi

% Optional override for |\version| flag:
%    \begin{macrocode}
%%\providecommand{\version}{final}
%    \end{macrocode}

% Include the main document:
%    \begin{macrocode}
\input{childdoc.def}
\childdocof{cdocsamp}
%    \end{macrocode}

%\iffalse
%</samplechap1|samplechap2>
%\fi
%
%\iffalse
%<*samplechap1>
%\fi
% Some text for chapter 1:
%    \begin{macrocode}
\section{one}
some text in chapter one
%    \end{macrocode}

%\iffalse
%</samplechap1>
%\fi
% Some text for chapter 2:
%\iffalse
%<*samplechap2>
%\fi
%    \begin{macrocode}
\section{two}
more text in chapter two
%    \end{macrocode}

%\iffalse
%</samplechap2>
%\fi
%
% %%%%%%%%%%%%%%%%%%%%%%%%%%%%%%%%%%%%%%
% \paragraph{Part Include Files.}
%
% The include files are called |cdocspt3.tex| and |cdocspt4.tex|.
%
%\iffalse
%<*samplepart3|samplepart4>
%\fi

% Optional override for |\version| flag:
%    \begin{macrocode}
%%\providecommand{\version}{final}
%    \end{macrocode}

% Include the main document:
%    \begin{macrocode}
\input{childdoc.def}
\childdocby{cdocsamp}
%    \end{macrocode}

%\iffalse
%</samplepart3|samplepart4>
%\fi
%
%\iffalse
%<*samplepart3>
%\fi
% Some text for part 3:
%    \begin{macrocode}
some text in part three
%    \end{macrocode}

%\iffalse
%</samplepart3>
%\fi
% Some text for part 4:
%\iffalse
%<*samplepart4>
%\fi
%    \begin{macrocode}
more text in part four
%    \end{macrocode}

%\iffalse
%</samplepart4>
%\fi
%
% %%%%%%%%%%%%%%%%%%%%%%%%%%%%%%%%%%%%%%
% \paragraph{Forwarding for a Complete Draft.}
%
% The following forwarding file |cdocsdrf.tex|
% compiles the main document in draft mode:
%\iffalse
%<*sampledraft>
%\fi
%    \begin{macrocode}
\def\version{draft}
\input{childdoc.def}
\childdocforward{cdocsamp}
%    \end{macrocode}

%\iffalse
%</sampledraft>
%\fi
%
% %%%%%%%%%%%%%%%%%%%%%%%%%%%%%%%%%%%%%%
% \paragraph{Forwarding for Final Version of the Chapters.}
%
% The following forwarding files |cdocsfn1.tex| and |cdocsfn2.tex|
% (with identical content)
% compile the final versions of the child documents
% |cdocsch1.tex| and |cdocsch2.tex|, respectively:
%\iffalse
%<*samplefinal>
%\fi
%    \begin{macrocode}
\def\version{final}
\input{childdoc.def}
\childdocforwardprefix[cdocsamp]{cdocsfn}{cdocsch}
%    \end{macrocode}

%\iffalse
%</samplefinal>
%\fi
%
% %%%%%%%%%%%%%%%%%%%%%%%%%%%%%%%%%%%%%%
% \paragraph{Command Line Processing.}
%
% The following three command lines generate the output files
% |cdocscld|, |cdocscl1| and |cdocscl2|
% which should be identical to
% |cdocsdrf|, |cdocsch1| and |cdocsfn2|, respectively:
% \begin{center}
% \begin{tabular}{l}
% |latex -jobname cdocscld \|\\
% |  "\def\version{draft}\input{childdoc.def}\childdocforward{cdocsamp}"|\\
% |latex -jobname cdocscl1 \|\\
% |  "\input{childdoc.def}\childdocforward[cdocsamp]{cdocsch1}"|\\
% |latex -jobname cdocscl2 \|\\
% |  "\def\version{final}\input{childdoc.def}\childdocforward{cdocsch2}"|
% \end{tabular}
% \end{center}
% Note that the trailing backslash on each first line
% merely continues the input to the second line
% (for convenient cut ant paste).
% Furthermore, the command |latex| can be replaced by any
% of its alternative versions such as |pdflatex|.
%
% %%%%%%%%%%%%%%%%%%%%%%%%%%%%%%%%%%%%%%%%%%%%%%%%%%%%%%%%%%%%%%%%%%%%%%%%%%%%%%
% %%%%%%%%%%%%%%%%%%%%%%%%%%%%%%%%%%%%%%%%%%%%%%%%%%%%%%%%%%%%%%%%%%%%%%%%%%%%%%
% \section{Implementation}
%\iffalse
%<*package>
%\fi
%
% This section describes the definitions file |childdoc.def|.

% The definitions cannot be loaded using |\usepackage| or |\RequirePackage|
% which has a mechanism to prevent loading a style file more than once.
% When loading the definitions by means of |\input|
% multiple instances have to be prevented manually:
%\iffalse
%This code needs to be before the `\ProvidesFile' directive
%which is defined at the beginning of this file.
%Therefore it is also placed there and commented out here.
%</package>
%<*discard>
%\fi
%    \begin{macrocode}
\ifdefined\childdocmain\endinput\fi
%    \end{macrocode}
%\iffalse
%</discard>
%<*package>
%\fi
%
% \macro{\ifchilddoc}
% \macro{\ifchilddocmanual}
% The conditional |\ifchilddoc| tells whether a
% child (true) or main (false) document is being compiled.
% The conditional |\ifchilddocmanual| tells whether
% the |\includeonly| mechanism is used (false) or
% the selection of child files must be performed manually (true).
% The definitions initialise to false:
%    \begin{macrocode}
\newif\ifchilddoc
\newif\ifchilddocmanual
%    \end{macrocode}

% \macro{\childdocname}
% \macro{\childdocjob}
% The macro |\childdocname| stores the name of the main document
% to be compiled. The macro |\childdocjob| stores the name of
% the document on which the \LaTeX{} compiler was originally invoked.
% The content of |\jobname| cannot be compared
% to filenames specified in the source due to different catcodes.
% The following code rescans |\jobname|, stores the result
% in |\childdocname| and saves a copy in |\childdocjob|:
%    \begin{macrocode}
\edef\childdocname{\scantokens\expandafter{\jobname\noexpand}}
\let\childdocjob\childdocname
%    \end{macrocode}

% \macro{\childdocdisable}
% The macro |\childdocdisable| prevents the main file
% from being processed more than once.
% At this stage, the main document command |\childdocmain|
% is assumed to be called once again where it should do nothing.
% Any subsequent call to it should prevent
% a secondary processing of the main document
% It overwrites the forwarding commands
% |\childdocof| and |\childdocforward|
% with empty macros to prevent further inclusions of the main document:
%    \begin{macrocode}
\newcommand{\childdocdisable}
{
  \renewcommand{\childdocmain}[1]{\renewcommand{\childdocmain}[1]{\endinput}}
  \renewcommand{\childdocof}[1]{}
  \renewcommand{\childdocby}[2][]{}
  \renewcommand{\childdocforward}[2][]{}
  \renewcommand{\childdocdisable}{}
}
%    \end{macrocode}

% \macro{\childdocmain}
% The macro |\childdocmain| is to be called at the top of the main file
% with nothing or the main filename (without extension) as argument.
% First, it breaks loops.
% If the argument is not empty and does not match |\childdocname|
% (which is set by the first inclusion of |childdoc.def|),
% |\ifchilddoc| is set to true, |\includeonly| is applied to the child file
% and |\jobname| is set to the main file
% (for proper handling of |.aux| files):
%    \begin{macrocode}
\newcommand{\childdocmain}[1]
{
  \childdocdisable\childdocmain{}
  \if?#1?\else
    \begingroup
      \def\childdoctmp{#1}
      \ifx\childdoctmp\childdocname
        \def\childdoctmp{}
      \else
        \def\childdoctmp
        {
          \childdoctrue
          \includeonly{\childdocname}
          \def\childdocjob{#1}
          \def\jobname{#1}
        }
      \fi
      \expandafter
    \endgroup
    \childdoctmp
  \fi
}
%    \end{macrocode}

% \macro{\childdocof}
% The command |\childdocof| redirects
% compilation to the main file |#1|.
%    \begin{macrocode}
\newcommand{\childdocof}[1]
{
  \childdocdisable
  \childdoctrue
  \includeonly{\childdocname}
  \def\jobname{#1}
  \def\childdocjob{#1}
  \input{#1}
}
%    \end{macrocode}

% \macro{\childdocby}
% The command |\childdocby| ....
%    \begin{macrocode}
\newcommand{\childdocby}[2][]
{
  \childdocdisable
  \childdoctrue
  \childdocmanualtrue
  \if?#1?\else
    \def\jobname{#2}
  \fi
  \def\childdocjob{#2}
  \input{#2}
  \endinput
}
%    \end{macrocode}

% \macro{\childdocforward}
% The command |\childdocforward| redirects
% compilation to the main file or
% (if the optional argument is given) a child file.
% Parameters are set as if the main file
% or a child file starting with |\childdocof| was compiled.
% Then compilation is handed over to the main file:
%    \begin{macrocode}
\newcommand{\childdocforward}[2][]
{
  \begingroup
    \if?#1?
      \def\childdoctmp
      {
        \def\childdocname{#2}
        \def\childdocjob{#2}
        \def\jobname{#2}
        \input{#2}
        \endinput
      }
    \else
      \def\childdoctmp
      {
        \childdocdisable
        \def\childdocname{#2}
        \childdoctrue
        \includeonly{#2}
        \def\childdocjob{#1}
        \def\jobname{#1}
        \input{#1}
        \endinput
      }
    \fi
    \expandafter
  \endgroup
  \childdoctmp
}
%    \end{macrocode}

% \macro{\childdocforwardprefix}
% The command |\childdocforwardprefix| redirects
% compilation to the main or a child file by means of a pattern.
% The prefix |#1| in the current filename is replaced by |#2|
% and the suffix of the current filename is kept
% (it is assumed that the filename does not contain the substring `|~~~|'
% which is used as a delimiter).
% Compilation is handed over to the new file by |\childdocforward|:
%    \begin{macrocode}
\newcommand{\childdocforwardprefix}[3][]
{
  \begingroup
    \def\childdocextract #2##1~~~{\def\childdoctmp{\childdocforward[#1]{#3##1}}}
    \expandafter\childdocextract\childdocname~~~
    \expandafter
  \endgroup
  \childdoctmp
}
%    \end{macrocode}

% \macro{\childdoc}
% The deprecated macro |\childdoc| is a legacy version of |\childdocmain|:
%    \begin{macrocode}
\newcommand{\childdoc}{\childdocmain}
%    \end{macrocode}

% \macro{\childdocredirect}
% The deprecated macro |\childdocredirect| is a legacy version
% of |\childdocforward| and |\childdocforwardprefix|:
%    \begin{macrocode}
\newcommand{\childdocredirect}[2][]
{
  \begingroup
    \if?#1?
      \def\childdoctmp{\childdocforward{#2}}
    \else
      \def\childdoctmp{\childdocforwardprefix{#1}{#2}}
    \fi
    \expandafter
  \endgroup
  \childdoctmp
}
%    \end{macrocode}

%\iffalse
%</package>
%\fi
%
\endinput
|\\
|\childdocby{|\textit{main}|}|\\
\end{tabular}
\end{center}
%
Both forms have slightly different effects as described above.
The main file is prepared as usual, see \secref{sec:include}.

%%%%%%%%%%%%%%%%%%%%%%%%%%%%%%%%%%%%%%%%%%%%%%%%%%%%%%%%%%%%%%%%%%%%%%%%%%%%%%%%
\subsection{Legacy Detection}
\label{sec:detection}

The directive |\childdocmain| in the main file can detect
whether the complete document or merely a child is to be compiled
even without using the directive |\childdocof|.
This method is deprecated because it is less robust
and there is no compelling reason to use it;
it is merely provided for backward compatibility
and it may be removed in future versions.

If the detection mechanism is to be used,
it is mandatory to correctly specify
the filename of the main file as the argument of |\childdocmain|:
%
\begin{center}
\begin{tabular}{l}
|% \iffalse
%
% childdoc.dtx Copyright (C) 2017-2018 Niklas Beisert
%
% This work may be distributed and/or modified under the
% conditions of the LaTeX Project Public License, either version 1.3
% of this license or (at your option) any later version.
% The latest version of this license is in
%   http://www.latex-project.org/lppl.txt
% and version 1.3 or later is part of all distributions of LaTeX
% version 2005/12/01 or later.
%
% This work has the LPPL maintenance status `maintained'.
%
% The Current Maintainer of this work is Niklas Beisert.
%
% This work consists of the files childdoc.dtx and childdoc.ins
% and the derived files childdoc.def and cdocsamp.tex with
% cdocsch1.tex, cdocsch2.tex, cdocsdrf.tex, cdocsfn1.tex, cdocsfn2.tex.
%
%<package>\ifdefined\childdocmain\endinput\fi
%<package>\ProvidesFile{childdoc.def}[2018/12/30 v2.0 child document driver]
%<samplemain>\ProvidesFile{cdocsamp.tex}[2018/12/30 v2.0 sample for childdoc]
%<*driver>
%\ProvidesFile{childdoc.drv}[2018/12/30 v2.0 childdoc reference manual file]
\PassOptionsToClass{10pt,a4paper}{article}
\documentclass{ltxdoc}

\usepackage[margin=35mm]{geometry}
\usepackage{hyperref}
\usepackage{hyperxmp}
\usepackage[usenames]{color}

\hypersetup{colorlinks=true}
\hypersetup{pdfstartview=FitH}
\hypersetup{pdfpagemode=UseNone}
\hypersetup{pdfsource={}}
\hypersetup{pdflang={en-UK}}
\hypersetup{pdfcopyright={Copyright 2017-2018 Niklas Beisert.
  This work may be distributed and/or modified under the
  conditions of the LaTeX Project Public License, either version 1.3
  of this license or (at your option) any later version.}}
\hypersetup{pdflicenseurl={http://www.latex-project.org/lppl.txt}}
\hypersetup{pdfcontactaddress={ETH Zurich, ITP, HIT K,
  Wolfgang-Pauli-Strasse 27}}
\hypersetup{pdfcontactpostcode={8093}}
\hypersetup{pdfcontactcity={Zurich}}
\hypersetup{pdfcontactcountry={Switzerland}}
\hypersetup{pdfcontactemail={nbeisert@itp.phys.ethz.ch}}
\hypersetup{pdfcontacturl={http://people.phys.ethz.ch/\xmptilde nbeisert/}}

\newcommand{\secref}[1]{\hyperref[#1]{section \ref*{#1}}}

\parskip1ex
\parindent0pt
\let\olditemize\itemize
\def\itemize{\olditemize\parskip0pt}

\begin{document}

\title{The \textsf{childdoc} Package}
\hypersetup{pdftitle={The childdoc Package}}
\author{Niklas Beisert\\[2ex]
  Institut f\"ur Theoretische Physik\\
  Eidgen\"ossische Technische Hochschule Z\"urich\\
  Wolfgang-Pauli-Strasse 27, 8093 Z\"urich, Switzerland\\[1ex]
  \href{mailto:nbeisert@itp.phys.ethz.ch}
  {\texttt{nbeisert@itp.phys.ethz.ch}}}
\hypersetup{pdfauthor={Niklas Beisert}}
\hypersetup{pdfsubject={Manual for the LaTeX2e Package childdoc}}
\date{30 December 2018, \textsf{v2.0}}
\maketitle

\begin{abstract}\noindent
\textsf{childdoc} is a \LaTeXe{} package
that enables the direct compilation
of document sections included by |\include|
to individual files.
\end{abstract}

\begingroup
\parskip0ex
\tableofcontents
\endgroup

%%%%%%%%%%%%%%%%%%%%%%%%%%%%%%%%%%%%%%%%%%%%%%%%%%%%%%%%%%%%%%%%%%%%%%%%%%%%%%%%
%%%%%%%%%%%%%%%%%%%%%%%%%%%%%%%%%%%%%%%%%%%%%%%%%%%%%%%%%%%%%%%%%%%%%%%%%%%%%%%%
\section{Introduction}

\LaTeX{} provides a mechanism to structure a large document (such as a book)
into a main file and several child files (containing the chapters)
using the |\include| command.
This mechanism is beneficial for documents
which span hundreds of pages in order to
make the source file(s) more manageable.
Moreover, compilation can be restricted to
selected child files by means of the |\includeonly| command.
The latter feature can be used to reduce the compilation time while editing
(this was significantly more useful in the earlier days of \LaTeX{})
or to generate a smaller document which is easier to navigate.
Another application of |\includeonly| is to generate
documents consisting of selected parts of the complete document.

However, there are a few drawbacks of the plain |\include| mechanism:
\begin{itemize}
\item
The child files cannot be compiled on their own,
they can only be compiled via the main file.
A naive editing environment
(such as a text editor with an option
to have the current file processed by \LaTeX)
may require one to switch to the main file before compiling;
attempting to compile the child file produces errors.
\item
The main file must be modified (each time)
to adjust the |\includeonly| command
to the present needs. This easily leaves the main file in a messy state.
\item
The generated document will always carry the filename
of the main document. This is inconvenient if
several child files are to be compiled and
to be kept for distribution.
\end{itemize}

The present package provides a simple interface
to make child files individually compilable by \LaTeX{}.
Compiling a child file then has the same effect as compiling
the main file with an |\includeonly| command
to select the appropriate child.
Moreover the generated document will carry the name of the child
rather than the main file.
This resolves all three above issues.

This feature is meant to make the editing of books,
thesis documents and lecture notes somewhat more convenient.
However, the package can also be used efficiently for
composing a series of documents (such as exercise sheets)
which are typically distributed individually.
It then assists the author in generating the individual documents
(potentially in different versions)
as well as a document containing the collected series.
Another application is in developing style files
or other kinds of included material
where compilation of the style file could redirect
to a sample or test file.

%%%%%%%%%%%%%%%%%%%%%%%%%%%%%%%%%%%%%%%%%%%%%%%%%%%%%%%%%%%%%%%%%%%%%%%%%%%%%%%%
%%%%%%%%%%%%%%%%%%%%%%%%%%%%%%%%%%%%%%%%%%%%%%%%%%%%%%%%%%%%%%%%%%%%%%%%%%%%%%%%
\section{Usage}

First of all, the package \textsf{childdoc} is \emph{not} a standard
\LaTeXe{} |.sty| style file! Therefore it needs to be invoked in
a non-standard way.

%%%%%%%%%%%%%%%%%%%%%%%%%%%%%%%%%%%%%%%%%%%%%%%%%%%%%%%%%%%%%%%%%%%%%%%%%%%%%%%%
\subsection{Included Files}
\label{sec:include}

%%%%%%%%%%%%%%%%%%%%%%%%%%%%%%%%%%%%%%%%
\DescribeMacro{\childdocmain}
To use the package, add the commands
\begin{center}
\begin{tabular}{l}
|\input{childdoc.def}|\\
|\childdocmain{}|\\
\end{tabular}
\end{center}
at the very top of the main \LaTeX{} file,
in particular \emph{before} the |\documentclass| statement!
The argument of |\childdocmain| should be left empty
(but it must be present).

%%%%%%%%%%%%%%%%%%%%%%%%%%%%%%%%%%%%%%%%
\DescribeMacro{\childdocof}
Furthermore, add the commands
\begin{center}
\begin{tabular}{l}
|\input{childdoc.def}|\\
|\childdocof{|\textit{main}|}|\\
\end{tabular}
\end{center}
at the top of every child file \textit{child}
which is included by |\include{|\textit{child}|}|
from within the main file
(or at least for those files to be compiled individually).
The argument \textit{main} must be the filename of the main file.

There are a couple of
considerations in setting up the main and child documents:

%%%%%%%%%%%%%%%%%%%%%%%%%%%%%%%%%%%%%%%%
\paragraph{Restrictions.}

Please note the following restrictions:
\begin{itemize}
\item
|\childdocmain| must be called with one argument \textit{main}
to ensure compatibility with earlier version of the package.
It must either be empty (|\childdocmain{}|)
or precisely match the filename of the main file in which it is specified.
See \secref{sec:detection} for further information.
\item
The filename \textit{main} must be specified without the |.tex| extension.
\item
The filename \textit{main} is case sensitive
(even in case-insensitive file systems)
due to internal string comparison.
\item
The argument \textit{main} should be fully expanded, it cannot be a macro.
\item
Subdirectories and special characters should be avoided in filenames.
\item
The command |\childdocmain{|\textit{main}|}| must be followed by a whitespace.
It should not be followed immediately by another command
or by a comment mark `|%|'.
This is because the \TeX{} parser reads the token immediately following
the argument of |\childdocmain| and puts it
at the beginning of every child section;
however, a white\-space is ignored.
\end{itemize}

%%%%%%%%%%%%%%%%%%%%%%%%%%%%%%%%%%%%%%%%
\paragraph{Content of Main File.}

It is advisable to place all content in the child files included by |\include|.
Any output contained in the main file will appear in all child documents
unless suppressed manually;
it cannot be suppressed automatically by the |\includeonly| directive
and thus should normally be avoided.
A method to include some content in the main file
by means of conditional processing is described in \secref{sec:conditional}.

%%%%%%%%%%%%%%%%%%%%%%%%%%%%%%%%%%%%%%%%
\paragraph{Page Numbering.}

When only a part of the document is compiled,
the appropriate numbering of pages
(as well as other status parameters)
is determined from the |.aux| files.
The latter contain information from previous passes.
However this information needs to propagate through
all intermediate child documents.
Therefore the page numbering in child documents may well
be inconsistent until the complete document is compiled at least once.

A useful (if unconventional) way to always ensure a consistent
page numbering is to restart the numbering in each child document
and denote the pages by `\textit{child}|.|\textit{page}'
where \textit{child} represents the chapter/section number of the child file.
This can be achieved by the command
|\numberwithin{page}{|\textit{child}|}|
of the \textsf{amsmath} package
where \textit{child} can be |chapter| or |section|
depending on the chosen structuring.
Alternatively, one can modify the macro |\thepage| appropriately
and reset the counter |page| at the start of each child file.

%%%%%%%%%%%%%%%%%%%%%%%%%%%%%%%%%%%%%%%%%%%%%%%%%%%%%%%%%%%%%%%%%%%%%%%%%%%%%%%%
\subsection{Conditional Processing}
\label{sec:conditional}

The package provides a mechanism to compile different versions
of a document. To customise the versions further some conditional processing
can come in handy to distinguish which version is being compiled.
The package provides two macros to describe the compilation context:

%%%%%%%%%%%%%%%%%%%%%%%%%%%%%%%%%%%%%%%%
\DescribeMacro{\ifchilddoc}
The conditional |\ifchilddoc| distinguishes between the compilation of
child documents and the main document:
%
\begin{center}
|\ifchilddoc |\textit{child-code}| |[|\||else |\textit{main-code}]| \||fi|
\end{center}

%%%%%%%%%%%%%%%%%%%%%%%%%%%%%%%%%%%%%%%%
\DescribeMacro{\childdocname}
\DescribeMacro{\childdocjob}
The macro |\childdocname| contains the filename (without extension)
of the main or child file being processed.
Note that |\childdocjob| will always contain the name of the main file.

%%%%%%%%%%%%%%%%%%%%%%%%%%%%%%%%%%%%%%%%
\paragraph{Title Page.}

Conditional processing can be used to include a title or banner page
in the main document when proper precautions are taken.
Importantly, the code in the main file should ensure that the page counter
(as well as other status parameters which are stored in the |.aux| files)
takes the same value after the conditional processing.
Otherwise the page numbers may take divergent values
depending on which part is compiled.

For example, a title page could be declared by:
%
\begin{center}
\begin{tabular}{l}
|\ifchilddoc\||else|\\
|\addtocounter{page}{-1}|\\
\textit{code for title page}\\
|\newpage|\\
|\||fi|
\end{tabular}
\end{center}
%
A banner page for the child documents can be generated by:
%
\begin{center}
\begin{tabular}{l}
|\ifchilddoc|\\
|\addtocounter{page}{-1}|\\
\textit{code for banner page}\\
|\newpage|\\
|\||fi|
\end{tabular}
\end{center}
%
Here one could write a message such as:
\begin{center}
|This is the part \childdocname{} of \childdocjob{}.|
\end{center}

%%%%%%%%%%%%%%%%%%%%%%%%%%%%%%%%%%%%%%%%%%%%%%%%%%%%%%%%%%%%%%%%%%%%%%%%%%%%%%%%
\subsection{Flags}
\label{sec:flags}

The package makes it easy to generate different versions
of the main or child documents.
To this end compilation flags can be defined
and assigned different default values.
They will be particularly useful in conjunction
with the forwarding mechanism described in \secref{sec:forward}.

For example, it may be useful to have a flag |\version|
which can be set to |draft| or |final|.
The document source will contain some conditional code
depending on the value of |\version|.
Suppose further, the flag should default to |final| for the main file
and to |draft| for child files
which is a natural assignment for editing the document.
This is achieved by placing the following code
in the preamble of the main document
(below the |\childdocmain| directive):
%
\begin{center}
\begin{tabular}{l}
|\ifchilddoc|\\
|\providecommand{\version}{draft}|\\
|\||else|\\
|\providecommand{\version}{final}|\\
|\||fi|
\end{tabular}
\end{center}
%
The definition by |\providecommand| makes sure
that previous definitions are not overwritten.
Further statements |\providecommand{\version}{...}|
can thus be added before the above code to override it.

For the main file, one might add a line
(between |\childdocmain| and the above block)
%
\begin{center}
|%\ifchilddoc\||else\providecommand{\version}{draft}\||fi|
\end{center}
%
which can be uncommented to produce a draft version.
Likewise one can add a line to the very top of a child file
(above the |\childdocof{|\textit{main}|}| directive)
%
\begin{center}
|%\providecommand{\version}{final}|
\end{center}
%
which can be uncommented to produce the final version of this child document.

%%%%%%%%%%%%%%%%%%%%%%%%%%%%%%%%%%%%%%%%%%%%%%%%%%%%%%%%%%%%%%%%%%%%%%%%%%%%%%%%
\subsection{Forwarding}
\label{sec:forward}

Different versions of the main or child documents
using compilation flags as described in \secref{sec:flags}
can be (permanently) stored in different files
for convenient compilation, viewing and distribution.
To this end, the package defines a command
to pass on compilation to a different file:

%%%%%%%%%%%%%%%%%%%%%%%%%%%%%%%%%%%%%%%%
\DescribeMacro{\childdocforward}
The command |\childdocforward| redirects processing to
another source file:
%
\begin{center}
\begin{tabular}{l}
|\input{childdoc.def}|\\
|\childdocforward[|\textit{main}|]{|\textit{dest}|}|\\
\end{tabular}
\end{center}
%
The argument \textit{dest} is the destination file
(without extension).
It should be the main file or one of the child files.
Note that further \textsf{childdoc} directives
such as |\childdocof| and |\childdocforward|
in the indicated file will be processed in this form.
The optional argument \textit{main}
passes on directly to the main file \textit{main}
while pretending to compile the child \textit{dest}.
This form behaves as if \textit{dest}
issues |\childdocof{|\textit{main}|}| right away,
and no further \textsf{childdoc} directives will be processed.

%%%%%%%%%%%%%%%%%%%%%%%%%%%%%%%%%%%%%%%%
\DescribeMacro{\...prefix}
In the alternative form |\childdocforwardprefix|,
%
\begin{center}
\begin{tabular}{l}
|\input{childdoc.def}|\\
|\childdocforwardprefix[|\textit{main}|]{|\textit{prefix}|}{|\textit{dest}|}|
\end{tabular}
\end{center}
%
the destination file is determined by a pattern
depending on the current file:
To make this work, the current file must be called
`{\textit{prefix}\hspace{0.2em}\textit{suffix}}'
with \textit{prefix} matching precisely the argument.
Processing is then passed on to the file
`{\textit{dest}\hspace{0.2em}\textit{suffix}}'.
Surely, the same effect is achieved by
directly specifying the
argument `{\textit{dest}\hspace{0.2em}\textit{suffix}}'
in the first form.
However, that requires to set up a different file
for each child. With the alternative form of the command
all these files can have exactly the same content
which simplifies setting them up and maintaining them.

For example, the following file |draft.tex|
with a compilation flag |\version| as described in \secref{sec:flags}
compiles the main document as a draft:
%
\begin{center}
\begin{tabular}{l}
|\def\version{draft}|\\
|\input{childdoc.def}|\\
|\childdocforward{|\textit{main}|}|
\end{tabular}
\end{center}
%
Likewise, the following files |final|\textit{nn}|.tex|
compile the final version of the child document
|child|\textit{nn}|.tex|:
%
\begin{center}
\begin{tabular}{l}
|\def\version{final}|\\
|\input{childdoc.def}|\\
|\childdocforwardprefix{final}{child}|
\end{tabular}
\end{center}
%

Note that when several versions of a main file and/or of each child file
are to be generated, it may be convenient to set up a |Makefile| or
shell script to automatise the process.

%%%%%%%%%%%%%%%%%%%%%%%%%%%%%%%%%%%%%%%%%%%%%%%%%%%%%%%%%%%%%%%%%%%%%%%%%%%%%%%%
\subsection{Command Line Processing}
\label{sec:commandline}

The effect of redirection files can also be achieved by invoking
the \LaTeX{} compiler with a more elaborate command line.
Most conveniently this should be done as part
of a shell script or a |Makefile|.

When using \textsf{childdoc} in the main file, the following
command lines effectively perform a redirection
(note that depending on the shell being used,
backslashes may have to be doubled: `|\|' $\to$ `|\\|'):
%
\begin{center}
|... -jobname "|\textit{target}|" |\\|"|[\textit{flags}]%
|\input{childdoc.def}\childdocforward[|\textit{main}|]{|\textit{dest}|}"|
\end{center}
%
Here \textit{target} is the name of the output file,
\textit{main} is the name of the main file
and \textit{dest} is the name of the main or child file to be processed
(all filenames without extensions).
The optional argument \textit{main} can be omitted
if \textit{main} matches \textit{dest}.
Optionally, compilation \textit{flags} can be defined via |\def| commands.
This command line makes the \TeX{} engine believe
it is compiling the file \textit{target}
whose content is specified as the latter parameter.
The provided code then forwards the processing to
\textit{main} or \textit{dest} as described in \secref{sec:forward}.

%%%%%%%%%%%%%%%%%%%%%%%%%%%%%%%%%%%%%%%%%%%%%%%%%%%%%%%%%%%%%%%%%%%%%%%%%%%%%%%%
\subsection{Include by Input}
\label{sec:input}

Including child documents by |\include| has some restrictions by design.
Most notably, the content of a child document always occupies
its own set of pages; pages cannot be shared between child documents.
Usually, this behaviour makes perfect sense
because each child document contain an essential part of the document.
However, in some situations it may be desirable to compose
a document from a collection of parts
without having mandatory page breaks between then.
For this case, the package
provides a mechanism to include parts
by |\input| which can also be processed individually.
However, by construction this mechanism
requires manual handling of the content to be output.

%%%%%%%%%%%%%%%%%%%%%%%%%%%%%%%%%%%%%%%%
\DescribeMacro{\ifchilddocmanual}
The main file should be prepared as usual, see \secref{sec:include}.
However, the document body must make a distinction
between processing of an individual part and of the main document, e.g.:
%
\begin{center}
\begin{tabular}{l}
|\ifchilddocmanual|\\
|\input{\childdocname}|\\
|\||else|\\
\textit{document body with }|\input{|\textit{part}|}|\\
|\||fi|
\end{tabular}
\end{center}
%
The conditional |\ifchilddocmanual| is true whenever
a part to be included by |\input| is being compiled,
and the name of the part is stored in |\childdocname|.

%%%%%%%%%%%%%%%%%%%%%%%%%%%%%%%%%%%%%%%%
\DescribeMacro{\childdocby}
Each part to be included by |\input| should start with:
%
\begin{center}
\begin{tabular}{l}
|\input{childdoc.def}|\\
|\childdocby{|\textit{main}|}|\\
\end{tabular}
\end{center}
%
The directive |\childdocby| is similar to |\childdocof|
described in \secref{sec:include},
but the subsequent selection of content must be done manually.
To that end, both |\ifchilddoc| and |\ifchilddocmanual|
will be true upon processing of a part,
and the name of the part is stored in |\childdocname|.
Note that |\jobname| will be set to the filename of the current part
so that each part receives an individual |.aux| file
that does not interfere with the |.aux| file(s) of the main document.
This behaviour can be altered by the alternative form
|\childdocby[*]{|\textit{main}|}| (with a non-empty optional argument)
which uses the |.aux| file of the main document
by setting |\jobname| to \textit{main}.

%%%%%%%%%%%%%%%%%%%%%%%%%%%%%%%%%%%%%%%%%%%%%%%%%%%%%%%%%%%%%%%%%%%%%%%%%%%%%%%%
\subsection{Driver Development}
\label{sec:driver}

The \textsf{childdoc} mechanism can also be use for the development
of definition files such as \LaTeX{} styles or classes.
This case differs from the above setup with multiple parts
included by |\include| in that no |\includeonly| should be invoked.
This can be achieved by starting the include file
(before |\ProvidesPackage|) with:
%
\begin{center}
\begin{tabular}{l}
|\input{childdoc.def}|\\
|\childdocforward{|\textit{main}|}|\\
\end{tabular}
\end{center}
%
or alternatively with:
%
\begin{center}
\begin{tabular}{l}
|\input{childdoc.def}|\\
|\childdocby{|\textit{main}|}|\\
\end{tabular}
\end{center}
%
Both forms have slightly different effects as described above.
The main file is prepared as usual, see \secref{sec:include}.

%%%%%%%%%%%%%%%%%%%%%%%%%%%%%%%%%%%%%%%%%%%%%%%%%%%%%%%%%%%%%%%%%%%%%%%%%%%%%%%%
\subsection{Legacy Detection}
\label{sec:detection}

The directive |\childdocmain| in the main file can detect
whether the complete document or merely a child is to be compiled
even without using the directive |\childdocof|.
This method is deprecated because it is less robust
and there is no compelling reason to use it;
it is merely provided for backward compatibility
and it may be removed in future versions.

If the detection mechanism is to be used,
it is mandatory to correctly specify
the filename of the main file as the argument of |\childdocmain|:
%
\begin{center}
\begin{tabular}{l}
|\input{childdoc.def}|\\
|\childdocmain{|\textit{main}|}|\\
\end{tabular}
\end{center}
%
If |\jobname| does not match the argument \textit{main} of |\childdocmain|,
it is assumed that |\jobname| points to the child file to be compiled.
When using |\childdocmain| with the main file specified as argument,
it suffices to start a child file
with just |\input{|\textit{main}|}|
without loading of the package and using |\childdocof|.
If instead all processing is done
with the appropriate \textsf{childdoc} directives,
the argument of \textit{main} of |\childdocmain| can be empty.

An alternative version of the command line processing described
in \secref{sec:commandline} using the detection mechanism reads:
%
\begin{center}
|... -jobname "|\textit{target}|" "|[\textit{flags}]%
[|\def\jobname{|\textit{dest}|}|]|\input{|\textit{main}|}"|
\end{center}

%%%%%%%%%%%%%%%%%%%%%%%%%%%%%%%%%%%%%%%%%%%%%%%%%%%%%%%%%%%%%%%%%%%%%%%%%%%%%%%%
\subsection{Manual Code}
\label{sec:manual}

In case one cannot be certain whether the definitions file |childdoc.def|
is installed on the target \TeX{} distribution
and one prefers not to ship it,
it is conceivable to paste a few relevant commands into the sources.

To that end, drop all statements |\input{childdoc.def}|
and perform the replacements as outlined below.
Instead of |\childdocmain{|\textit{main}|}| add the following code
to the top of the main file:
%
\begin{center}
\begin{tabular}{l}
|\||ifdefined\childdocname\endinput\||fi\newif\ifchilddoc|\\
|\edef\childdocname{\scantokens\expandafter{\jobname\noexpand}}|\\
|\def\childdocmain{|\textit{main}|}\||ifx\childdocmain\childdocname\||else|\\
|\childdoctrue\includeonly{\childdocname}\let\jobname\childdocmain\||fi|\\
\end{tabular}
\end{center}
%
Instead of |\childdocof{|\textit{main}|}| just include the main file
at the top of each child file:
%
\begin{center}
|\input{|\textit{main}|}|
\end{center}
%
A simple redirection |\childdocforward{|\textit{dest}|}| is achieved by:
%
\begin{center}
|\def\jobname{|\textit{dest}|}\input{\jobname}|
\end{center}
%
The redirection with prefix
|\childdocforwardprefix[|\textit{prefix}|]{|\textit{dest}|}|
is accomplished by:
%
\begin{center}
\begin{tabular}{l}
|{\edef\jobname{\scantokens\expandafter{\jobname\noexpand}}|\\
|\def\redirectjob |\textit{prefix}|#1~~~{\gdef\jobname{|\textit{dest}|#1}}|\\
|\expandafter\redirectjob\jobname~~~}\input{\jobname}|
\end{tabular}
\end{center}

In an alternative approach,
child documents can be compiled by a specific command line
without additional code or specific definitions:
%
\begin{center}
|... -jobname "|\textit{target}|" "|[\textit{flags}]%
|\includeonly{|\textit{dest}|}\input{|\textit{main}|}"|
\end{center}
%

%%%%%%%%%%%%%%%%%%%%%%%%%%%%%%%%%%%%%%%%%%%%%%%%%%%%%%%%%%%%%%%%%%%%%%%%%%%%%%%%
%%%%%%%%%%%%%%%%%%%%%%%%%%%%%%%%%%%%%%%%%%%%%%%%%%%%%%%%%%%%%%%%%%%%%%%%%%%%%%%%
\section{Information}

%%%%%%%%%%%%%%%%%%%%%%%%%%%%%%%%%%%%%%%%%%%%%%%%%%%%%%%%%%%%%%%%%%%%%%%%%%%%%%%%
\subsection{Copyright}

Copyright \copyright{} 2017--2018 Niklas Beisert

This work may be distributed and/or modified under the
conditions of the \LaTeX{} Project Public License, either version 1.3
of this license or (at your option) any later version.
The latest version of this license is in
  \url{http://www.latex-project.org/lppl.txt}
and version 1.3 or later is part of all distributions of \LaTeX{}
version 2005/12/01 or later.

This work has the LPPL maintenance status `maintained'.

The Current Maintainer of this work is Niklas Beisert.

This work consists of the files |README.txt|, |childdoc.ins| and |childdoc.dtx|
as well as the derived files |childdoc.def|, |cdocsamp.tex|
with |cdocsch1.tex|, |cdocsch2.tex|, |cdocspt3.tex|, |cdocspt4.tex|,
|cdocsdrf.tex|, |cdocsfn1.tex|, |cdocsfn2.tex|
as well as |childdoc.pdf|.

%%%%%%%%%%%%%%%%%%%%%%%%%%%%%%%%%%%%%%%%%%%%%%%%%%%%%%%%%%%%%%%%%%%%%%%%%%%%%%%%
\subsection{Files and Installation}

The package consists of the files:
%
\begin{center}
\begin{tabular}{ll}
    |README.txt|   & readme file \\
    |childdoc.ins| & installation file \\
    |childdoc.dtx| & source file \\
    |childdoc.def| & definition file \\
    |cdocsamp.tex| & sample main file \\
    |cdocsch1.tex| & sample include file \\
    |cdocsch2.tex| & sample include file \\
    |cdocspt3.tex| & sample part file \\
    |cdocspt4.tex| & sample part file \\
    |cdocsdrf.tex| & sample redirection file \\
    |cdocsfn1.tex| & sample redirection file \\
    |cdocsfn2.tex| & sample redirection file \\
    |childdoc.pdf| & manual
\end{tabular}
\end{center}
%
The distribution consists of the files
|README.txt|, |childdoc.ins| and |childdoc.dtx|.
%
\begin{itemize}
\item
Run (pdf)\LaTeX{} on |childdoc.dtx|
to compile the manual |childdoc.pdf| (this file).
\item
Run \LaTeX{} on |childdoc.ins| to create the definitions file |childdoc.def|
and the sample |cdocsamp.tex| with include files
|cdocsch1.tex|, |cdocsch2.tex|, |cdocspt3.tex|, |cdocspt4.tex|,
|cdocsdrf.tex|, |cdocsfn1.tex|, |cdocsfn2.tex|.
Then copy the file |childdoc.def| to an appropriate directory of your \LaTeX{}
distribution, e.g.\ \textit{texmf-root}|/tex/latex/childdoc|.
\end{itemize}

%%%%%%%%%%%%%%%%%%%%%%%%%%%%%%%%%%%%%%%%%%%%%%%%%%%%%%%%%%%%%%%%%%%%%%%%%%%%%%%%
\subsection{Related CTAN Packages}

There are several other packages which offer a similar functionality:
%
\begin{itemize}
\item
The packages
\href{http://ctan.org/pkg/docmute}{\textsf{docmute}},
\href{http://ctan.org/pkg/includex}{\textsf{includex}} and
\href{http://ctan.org/pkg/standalone}{\textsf{standalone}}
provide commands to include only the document body of
a child file thus allowing both files to be compiled individually.
\item
The packages \href{http://ctan.org/pkg/subdocs}{\textsf{subdocs}}
and \href{http://ctan.org/pkg/subfiles}{\textsf{subfiles}}
provide structures in which the main and child documents can be
encapsulated and allowing them to be compiled individually.
The inclusion mechanism is different from the conventional |\include|.
\item
The package \href{http://ctan.org/pkg/combine}{\textsf{combine}}
is an elaborate solution to combine several documents into one.
\end{itemize}
%
See also the CTAN topic \href{http://ctan.org/topic/subdocs}{\textsf{subdocs}}
for further related packages.
The present package differs from the above solutions in that
a document structure constructed with the conventional |\include| mechanism
just needs two extra commands at the top of every file
such that all constituent files can be compiled individually.

%%%%%%%%%%%%%%%%%%%%%%%%%%%%%%%%%%%%%%%%%%%%%%%%%%%%%%%%%%%%%%%%%%%%%%%%%%%%%%%%
%\subsection{Feature Suggestions}
%
%The following is a list of features which may be useful for future
%versions of this package:
%%
%\begin{itemize}
%\item
%\ldots
%\end{itemize}

%%%%%%%%%%%%%%%%%%%%%%%%%%%%%%%%%%%%%%%%%%%%%%%%%%%%%%%%%%%%%%%%%%%%%%%%%%%%%%%%
\subsection{Revision History}

%%%%%%%%%%%%%%%%%%%%%%%%%%%%%%%%%%%%%%%%
\paragraph{v2.0:} 2018/12/30

\begin{itemize}
\item
immediate forward processing
\item
added |\childdocby| mechanism
\item
manual restructured
\end{itemize}

%%%%%%%%%%%%%%%%%%%%%%%%%%%%%%%%%%%%%%%%
\paragraph{v1.6:} 2018/01/17

\begin{itemize}
\item
application for development of include files
\item
corrections to manual
\end{itemize}

%%%%%%%%%%%%%%%%%%%%%%%%%%%%%%%%%%%%%%%%
\paragraph{v1.5:} 2017/05/21

\begin{itemize}
\item
more complete structuring introduced
\item
|\childdocof| introduced
\item
|\childdoc| renamed to |\childdocmain|
\item
|\childredirect| renamed to |\childdocforward| and |\childdocforwardprefix|
and functionality expanded
\end{itemize}

%%%%%%%%%%%%%%%%%%%%%%%%%%%%%%%%%%%%%%%%
\paragraph{v1.0:} 2017/04/27

\begin{itemize}
\item
manual and install package
\item
first version published on CTAN
\end{itemize}

%%%%%%%%%%%%%%%%%%%%%%%%%%%%%%%%%%%%%%%%
\paragraph{v0.6:} 2017/04/26

\begin{itemize}
\item
redirection mechanism added
\end{itemize}

%%%%%%%%%%%%%%%%%%%%%%%%%%%%%%%%%%%%%%%%
\paragraph{v0.5:} 2017/04/26

\begin{itemize}
\item
functionality in definition file
\end{itemize}


%%%%%%%%%%%%%%%%%%%%%%%%%%%%%%%%%%%%%%%%%%%%%%%%%%%%%%%%%%%%%%%%%%%%%%%%%%%%%%%%
%%%%%%%%%%%%%%%%%%%%%%%%%%%%%%%%%%%%%%%%%%%%%%%%%%%%%%%%%%%%%%%%%%%%%%%%%%%%%%%%
%%%%%%%%%%%%%%%%%%%%%%%%%%%%%%%%%%%%%%%%%%%%%%%%%%%%%%%%%%%%%%%%%%%%%%%%%%%%%%%%
\appendix

\settowidth\MacroIndent{\rmfamily\scriptsize 000\ }

 \DocInput{childdoc.dtx}

\end{document}
%</driver>
% \fi
%
% %%%%%%%%%%%%%%%%%%%%%%%%%%%%%%%%%%%%%%%%%%%%%%%%%%%%%%%%%%%%%%%%%%%%%%%%%%%%%%
% %%%%%%%%%%%%%%%%%%%%%%%%%%%%%%%%%%%%%%%%%%%%%%%%%%%%%%%%%%%%%%%%%%%%%%%%%%%%%%
% \section{Sample}
%\iffalse
%<*samplemain>
%\fi
%
% The following presents a sample document
% with two chapters, two parts, a title page,
% a compile flag as well as three forwarding files to set the flag.
% It consists of eight |.tex| files:
% \begin{center}
% \begin{tabular}{ll}
% |cdocsamp.tex|&main file\\
% |cdocsch1.tex|&include file for chapter 1\\
% |cdocsch2.tex|&include file for chapter 2\\
% |cdocspt3.tex|&include file for part 3\\
% |cdocspt4.tex|&include file for part 4\\
% |cdocsdrf.tex|&forwarding file for main file in draft mode\\
% |cdocsfi1.tex|&forwarding file for final version of chapter 1\\
% |cdocsfi2.tex|&forwarding file for final version of chapter 2\\
% \end{tabular}
% \end{center}
% Each of the eight files can be compiled directly by the \LaTeX{} compiler.
%
% %%%%%%%%%%%%%%%%%%%%%%%%%%%%%%%%%%%%%%
% \paragraph{Main File.}
%
% The main file is called |cdocsamp.tex|.
%
% Load the \textsf{childdoc} definitions and
% declare the filename for the main document:
%    \begin{macrocode}
\input{childdoc.def}
\childdocmain{}
%    \end{macrocode}

% Optional override for |\version| flag:
%    \begin{macrocode}
%%\ifchilddoc\else\providecommand{\version}{draft}\fi
%    \end{macrocode}

% Define the default values for the |\version| flag
% (|final| for the main file and |draft| for childs):
%    \begin{macrocode}
\ifchilddoc
\providecommand{\version}{draft}
\else
\providecommand{\version}{final}
\fi
%    \end{macrocode}

% Load the standard document class:
%    \begin{macrocode}
\documentclass[12pt]{article}
%    \end{macrocode}

% Start the document body:
%    \begin{macrocode}
\begin{document}
%    \end{macrocode}

% Declare a title page.
% Print title, part of document being processed and version flag:
%    \begin{macrocode}
\addtocounter{page}{-1}
\begin{center}
{\LARGE\bfseries{}childdoc example\par}
\vspace{1cm}
\ifchilddoc
\ifchilddocmanual part\else chapter\fi:
`\childdocname' of `\childdocjob'\par
\else
main document: `\childdocjob'\par
\fi
version: \version\par
\end{center}
\newpage
%    \end{macrocode}

% Manually include selected file,
% otherwise process as usual:
%    \begin{macrocode}
\ifchilddocmanual
\section*{part `\childdocname'}
\input{\childdocname}
\else
%    \end{macrocode}

% Include the two chapters:
%    \begin{macrocode}
\include{cdocsch1}
\include{cdocsch2}
%    \end{macrocode}

% Include the two parts unless only chapters should be displayed:
%    \begin{macrocode}
\ifchilddoc\else
\section{part three}
\input{cdocspt3}
\section{part four}
\input{cdocspt4}
\fi
%    \end{macrocode}

% Process as usual until here:
%    \begin{macrocode}
\fi
%    \end{macrocode}

% End of document body:
%    \begin{macrocode}
\end{document}
%    \end{macrocode}
%\iffalse
%</samplemain>
%\fi
%
% %%%%%%%%%%%%%%%%%%%%%%%%%%%%%%%%%%%%%%
% \paragraph{Chapter Include Files.}
%
% The include files are called |cdocsch1.tex| and |cdocsch2.tex|.
%
%\iffalse
%<*samplechap1|samplechap2>
%\fi

% Optional override for |\version| flag:
%    \begin{macrocode}
%%\providecommand{\version}{final}
%    \end{macrocode}

% Include the main document:
%    \begin{macrocode}
\input{childdoc.def}
\childdocof{cdocsamp}
%    \end{macrocode}

%\iffalse
%</samplechap1|samplechap2>
%\fi
%
%\iffalse
%<*samplechap1>
%\fi
% Some text for chapter 1:
%    \begin{macrocode}
\section{one}
some text in chapter one
%    \end{macrocode}

%\iffalse
%</samplechap1>
%\fi
% Some text for chapter 2:
%\iffalse
%<*samplechap2>
%\fi
%    \begin{macrocode}
\section{two}
more text in chapter two
%    \end{macrocode}

%\iffalse
%</samplechap2>
%\fi
%
% %%%%%%%%%%%%%%%%%%%%%%%%%%%%%%%%%%%%%%
% \paragraph{Part Include Files.}
%
% The include files are called |cdocspt3.tex| and |cdocspt4.tex|.
%
%\iffalse
%<*samplepart3|samplepart4>
%\fi

% Optional override for |\version| flag:
%    \begin{macrocode}
%%\providecommand{\version}{final}
%    \end{macrocode}

% Include the main document:
%    \begin{macrocode}
\input{childdoc.def}
\childdocby{cdocsamp}
%    \end{macrocode}

%\iffalse
%</samplepart3|samplepart4>
%\fi
%
%\iffalse
%<*samplepart3>
%\fi
% Some text for part 3:
%    \begin{macrocode}
some text in part three
%    \end{macrocode}

%\iffalse
%</samplepart3>
%\fi
% Some text for part 4:
%\iffalse
%<*samplepart4>
%\fi
%    \begin{macrocode}
more text in part four
%    \end{macrocode}

%\iffalse
%</samplepart4>
%\fi
%
% %%%%%%%%%%%%%%%%%%%%%%%%%%%%%%%%%%%%%%
% \paragraph{Forwarding for a Complete Draft.}
%
% The following forwarding file |cdocsdrf.tex|
% compiles the main document in draft mode:
%\iffalse
%<*sampledraft>
%\fi
%    \begin{macrocode}
\def\version{draft}
\input{childdoc.def}
\childdocforward{cdocsamp}
%    \end{macrocode}

%\iffalse
%</sampledraft>
%\fi
%
% %%%%%%%%%%%%%%%%%%%%%%%%%%%%%%%%%%%%%%
% \paragraph{Forwarding for Final Version of the Chapters.}
%
% The following forwarding files |cdocsfn1.tex| and |cdocsfn2.tex|
% (with identical content)
% compile the final versions of the child documents
% |cdocsch1.tex| and |cdocsch2.tex|, respectively:
%\iffalse
%<*samplefinal>
%\fi
%    \begin{macrocode}
\def\version{final}
\input{childdoc.def}
\childdocforwardprefix[cdocsamp]{cdocsfn}{cdocsch}
%    \end{macrocode}

%\iffalse
%</samplefinal>
%\fi
%
% %%%%%%%%%%%%%%%%%%%%%%%%%%%%%%%%%%%%%%
% \paragraph{Command Line Processing.}
%
% The following three command lines generate the output files
% |cdocscld|, |cdocscl1| and |cdocscl2|
% which should be identical to
% |cdocsdrf|, |cdocsch1| and |cdocsfn2|, respectively:
% \begin{center}
% \begin{tabular}{l}
% |latex -jobname cdocscld \|\\
% |  "\def\version{draft}\input{childdoc.def}\childdocforward{cdocsamp}"|\\
% |latex -jobname cdocscl1 \|\\
% |  "\input{childdoc.def}\childdocforward[cdocsamp]{cdocsch1}"|\\
% |latex -jobname cdocscl2 \|\\
% |  "\def\version{final}\input{childdoc.def}\childdocforward{cdocsch2}"|
% \end{tabular}
% \end{center}
% Note that the trailing backslash on each first line
% merely continues the input to the second line
% (for convenient cut ant paste).
% Furthermore, the command |latex| can be replaced by any
% of its alternative versions such as |pdflatex|.
%
% %%%%%%%%%%%%%%%%%%%%%%%%%%%%%%%%%%%%%%%%%%%%%%%%%%%%%%%%%%%%%%%%%%%%%%%%%%%%%%
% %%%%%%%%%%%%%%%%%%%%%%%%%%%%%%%%%%%%%%%%%%%%%%%%%%%%%%%%%%%%%%%%%%%%%%%%%%%%%%
% \section{Implementation}
%\iffalse
%<*package>
%\fi
%
% This section describes the definitions file |childdoc.def|.

% The definitions cannot be loaded using |\usepackage| or |\RequirePackage|
% which has a mechanism to prevent loading a style file more than once.
% When loading the definitions by means of |\input|
% multiple instances have to be prevented manually:
%\iffalse
%This code needs to be before the `\ProvidesFile' directive
%which is defined at the beginning of this file.
%Therefore it is also placed there and commented out here.
%</package>
%<*discard>
%\fi
%    \begin{macrocode}
\ifdefined\childdocmain\endinput\fi
%    \end{macrocode}
%\iffalse
%</discard>
%<*package>
%\fi
%
% \macro{\ifchilddoc}
% \macro{\ifchilddocmanual}
% The conditional |\ifchilddoc| tells whether a
% child (true) or main (false) document is being compiled.
% The conditional |\ifchilddocmanual| tells whether
% the |\includeonly| mechanism is used (false) or
% the selection of child files must be performed manually (true).
% The definitions initialise to false:
%    \begin{macrocode}
\newif\ifchilddoc
\newif\ifchilddocmanual
%    \end{macrocode}

% \macro{\childdocname}
% \macro{\childdocjob}
% The macro |\childdocname| stores the name of the main document
% to be compiled. The macro |\childdocjob| stores the name of
% the document on which the \LaTeX{} compiler was originally invoked.
% The content of |\jobname| cannot be compared
% to filenames specified in the source due to different catcodes.
% The following code rescans |\jobname|, stores the result
% in |\childdocname| and saves a copy in |\childdocjob|:
%    \begin{macrocode}
\edef\childdocname{\scantokens\expandafter{\jobname\noexpand}}
\let\childdocjob\childdocname
%    \end{macrocode}

% \macro{\childdocdisable}
% The macro |\childdocdisable| prevents the main file
% from being processed more than once.
% At this stage, the main document command |\childdocmain|
% is assumed to be called once again where it should do nothing.
% Any subsequent call to it should prevent
% a secondary processing of the main document
% It overwrites the forwarding commands
% |\childdocof| and |\childdocforward|
% with empty macros to prevent further inclusions of the main document:
%    \begin{macrocode}
\newcommand{\childdocdisable}
{
  \renewcommand{\childdocmain}[1]{\renewcommand{\childdocmain}[1]{\endinput}}
  \renewcommand{\childdocof}[1]{}
  \renewcommand{\childdocby}[2][]{}
  \renewcommand{\childdocforward}[2][]{}
  \renewcommand{\childdocdisable}{}
}
%    \end{macrocode}

% \macro{\childdocmain}
% The macro |\childdocmain| is to be called at the top of the main file
% with nothing or the main filename (without extension) as argument.
% First, it breaks loops.
% If the argument is not empty and does not match |\childdocname|
% (which is set by the first inclusion of |childdoc.def|),
% |\ifchilddoc| is set to true, |\includeonly| is applied to the child file
% and |\jobname| is set to the main file
% (for proper handling of |.aux| files):
%    \begin{macrocode}
\newcommand{\childdocmain}[1]
{
  \childdocdisable\childdocmain{}
  \if?#1?\else
    \begingroup
      \def\childdoctmp{#1}
      \ifx\childdoctmp\childdocname
        \def\childdoctmp{}
      \else
        \def\childdoctmp
        {
          \childdoctrue
          \includeonly{\childdocname}
          \def\childdocjob{#1}
          \def\jobname{#1}
        }
      \fi
      \expandafter
    \endgroup
    \childdoctmp
  \fi
}
%    \end{macrocode}

% \macro{\childdocof}
% The command |\childdocof| redirects
% compilation to the main file |#1|.
%    \begin{macrocode}
\newcommand{\childdocof}[1]
{
  \childdocdisable
  \childdoctrue
  \includeonly{\childdocname}
  \def\jobname{#1}
  \def\childdocjob{#1}
  \input{#1}
}
%    \end{macrocode}

% \macro{\childdocby}
% The command |\childdocby| ....
%    \begin{macrocode}
\newcommand{\childdocby}[2][]
{
  \childdocdisable
  \childdoctrue
  \childdocmanualtrue
  \if?#1?\else
    \def\jobname{#2}
  \fi
  \def\childdocjob{#2}
  \input{#2}
  \endinput
}
%    \end{macrocode}

% \macro{\childdocforward}
% The command |\childdocforward| redirects
% compilation to the main file or
% (if the optional argument is given) a child file.
% Parameters are set as if the main file
% or a child file starting with |\childdocof| was compiled.
% Then compilation is handed over to the main file:
%    \begin{macrocode}
\newcommand{\childdocforward}[2][]
{
  \begingroup
    \if?#1?
      \def\childdoctmp
      {
        \def\childdocname{#2}
        \def\childdocjob{#2}
        \def\jobname{#2}
        \input{#2}
        \endinput
      }
    \else
      \def\childdoctmp
      {
        \childdocdisable
        \def\childdocname{#2}
        \childdoctrue
        \includeonly{#2}
        \def\childdocjob{#1}
        \def\jobname{#1}
        \input{#1}
        \endinput
      }
    \fi
    \expandafter
  \endgroup
  \childdoctmp
}
%    \end{macrocode}

% \macro{\childdocforwardprefix}
% The command |\childdocforwardprefix| redirects
% compilation to the main or a child file by means of a pattern.
% The prefix |#1| in the current filename is replaced by |#2|
% and the suffix of the current filename is kept
% (it is assumed that the filename does not contain the substring `|~~~|'
% which is used as a delimiter).
% Compilation is handed over to the new file by |\childdocforward|:
%    \begin{macrocode}
\newcommand{\childdocforwardprefix}[3][]
{
  \begingroup
    \def\childdocextract #2##1~~~{\def\childdoctmp{\childdocforward[#1]{#3##1}}}
    \expandafter\childdocextract\childdocname~~~
    \expandafter
  \endgroup
  \childdoctmp
}
%    \end{macrocode}

% \macro{\childdoc}
% The deprecated macro |\childdoc| is a legacy version of |\childdocmain|:
%    \begin{macrocode}
\newcommand{\childdoc}{\childdocmain}
%    \end{macrocode}

% \macro{\childdocredirect}
% The deprecated macro |\childdocredirect| is a legacy version
% of |\childdocforward| and |\childdocforwardprefix|:
%    \begin{macrocode}
\newcommand{\childdocredirect}[2][]
{
  \begingroup
    \if?#1?
      \def\childdoctmp{\childdocforward{#2}}
    \else
      \def\childdoctmp{\childdocforwardprefix{#1}{#2}}
    \fi
    \expandafter
  \endgroup
  \childdoctmp
}
%    \end{macrocode}

%\iffalse
%</package>
%\fi
%
\endinput
|\\
|\childdocmain{|\textit{main}|}|\\
\end{tabular}
\end{center}
%
If |\jobname| does not match the argument \textit{main} of |\childdocmain|,
it is assumed that |\jobname| points to the child file to be compiled.
When using |\childdocmain| with the main file specified as argument,
it suffices to start a child file
with just |\input{|\textit{main}|}|
without loading of the package and using |\childdocof|.
If instead all processing is done
with the appropriate \textsf{childdoc} directives,
the argument of \textit{main} of |\childdocmain| can be empty.

An alternative version of the command line processing described
in \secref{sec:commandline} using the detection mechanism reads:
%
\begin{center}
|... -jobname "|\textit{target}|" "|[\textit{flags}]%
[|\def\jobname{|\textit{dest}|}|]|\input{|\textit{main}|}"|
\end{center}

%%%%%%%%%%%%%%%%%%%%%%%%%%%%%%%%%%%%%%%%%%%%%%%%%%%%%%%%%%%%%%%%%%%%%%%%%%%%%%%%
\subsection{Manual Code}
\label{sec:manual}

In case one cannot be certain whether the definitions file |childdoc.def|
is installed on the target \TeX{} distribution
and one prefers not to ship it,
it is conceivable to paste a few relevant commands into the sources.

To that end, drop all statements |% \iffalse
%
% childdoc.dtx Copyright (C) 2017-2018 Niklas Beisert
%
% This work may be distributed and/or modified under the
% conditions of the LaTeX Project Public License, either version 1.3
% of this license or (at your option) any later version.
% The latest version of this license is in
%   http://www.latex-project.org/lppl.txt
% and version 1.3 or later is part of all distributions of LaTeX
% version 2005/12/01 or later.
%
% This work has the LPPL maintenance status `maintained'.
%
% The Current Maintainer of this work is Niklas Beisert.
%
% This work consists of the files childdoc.dtx and childdoc.ins
% and the derived files childdoc.def and cdocsamp.tex with
% cdocsch1.tex, cdocsch2.tex, cdocsdrf.tex, cdocsfn1.tex, cdocsfn2.tex.
%
%<package>\ifdefined\childdocmain\endinput\fi
%<package>\ProvidesFile{childdoc.def}[2018/12/30 v2.0 child document driver]
%<samplemain>\ProvidesFile{cdocsamp.tex}[2018/12/30 v2.0 sample for childdoc]
%<*driver>
%\ProvidesFile{childdoc.drv}[2018/12/30 v2.0 childdoc reference manual file]
\PassOptionsToClass{10pt,a4paper}{article}
\documentclass{ltxdoc}

\usepackage[margin=35mm]{geometry}
\usepackage{hyperref}
\usepackage{hyperxmp}
\usepackage[usenames]{color}

\hypersetup{colorlinks=true}
\hypersetup{pdfstartview=FitH}
\hypersetup{pdfpagemode=UseNone}
\hypersetup{pdfsource={}}
\hypersetup{pdflang={en-UK}}
\hypersetup{pdfcopyright={Copyright 2017-2018 Niklas Beisert.
  This work may be distributed and/or modified under the
  conditions of the LaTeX Project Public License, either version 1.3
  of this license or (at your option) any later version.}}
\hypersetup{pdflicenseurl={http://www.latex-project.org/lppl.txt}}
\hypersetup{pdfcontactaddress={ETH Zurich, ITP, HIT K,
  Wolfgang-Pauli-Strasse 27}}
\hypersetup{pdfcontactpostcode={8093}}
\hypersetup{pdfcontactcity={Zurich}}
\hypersetup{pdfcontactcountry={Switzerland}}
\hypersetup{pdfcontactemail={nbeisert@itp.phys.ethz.ch}}
\hypersetup{pdfcontacturl={http://people.phys.ethz.ch/\xmptilde nbeisert/}}

\newcommand{\secref}[1]{\hyperref[#1]{section \ref*{#1}}}

\parskip1ex
\parindent0pt
\let\olditemize\itemize
\def\itemize{\olditemize\parskip0pt}

\begin{document}

\title{The \textsf{childdoc} Package}
\hypersetup{pdftitle={The childdoc Package}}
\author{Niklas Beisert\\[2ex]
  Institut f\"ur Theoretische Physik\\
  Eidgen\"ossische Technische Hochschule Z\"urich\\
  Wolfgang-Pauli-Strasse 27, 8093 Z\"urich, Switzerland\\[1ex]
  \href{mailto:nbeisert@itp.phys.ethz.ch}
  {\texttt{nbeisert@itp.phys.ethz.ch}}}
\hypersetup{pdfauthor={Niklas Beisert}}
\hypersetup{pdfsubject={Manual for the LaTeX2e Package childdoc}}
\date{30 December 2018, \textsf{v2.0}}
\maketitle

\begin{abstract}\noindent
\textsf{childdoc} is a \LaTeXe{} package
that enables the direct compilation
of document sections included by |\include|
to individual files.
\end{abstract}

\begingroup
\parskip0ex
\tableofcontents
\endgroup

%%%%%%%%%%%%%%%%%%%%%%%%%%%%%%%%%%%%%%%%%%%%%%%%%%%%%%%%%%%%%%%%%%%%%%%%%%%%%%%%
%%%%%%%%%%%%%%%%%%%%%%%%%%%%%%%%%%%%%%%%%%%%%%%%%%%%%%%%%%%%%%%%%%%%%%%%%%%%%%%%
\section{Introduction}

\LaTeX{} provides a mechanism to structure a large document (such as a book)
into a main file and several child files (containing the chapters)
using the |\include| command.
This mechanism is beneficial for documents
which span hundreds of pages in order to
make the source file(s) more manageable.
Moreover, compilation can be restricted to
selected child files by means of the |\includeonly| command.
The latter feature can be used to reduce the compilation time while editing
(this was significantly more useful in the earlier days of \LaTeX{})
or to generate a smaller document which is easier to navigate.
Another application of |\includeonly| is to generate
documents consisting of selected parts of the complete document.

However, there are a few drawbacks of the plain |\include| mechanism:
\begin{itemize}
\item
The child files cannot be compiled on their own,
they can only be compiled via the main file.
A naive editing environment
(such as a text editor with an option
to have the current file processed by \LaTeX)
may require one to switch to the main file before compiling;
attempting to compile the child file produces errors.
\item
The main file must be modified (each time)
to adjust the |\includeonly| command
to the present needs. This easily leaves the main file in a messy state.
\item
The generated document will always carry the filename
of the main document. This is inconvenient if
several child files are to be compiled and
to be kept for distribution.
\end{itemize}

The present package provides a simple interface
to make child files individually compilable by \LaTeX{}.
Compiling a child file then has the same effect as compiling
the main file with an |\includeonly| command
to select the appropriate child.
Moreover the generated document will carry the name of the child
rather than the main file.
This resolves all three above issues.

This feature is meant to make the editing of books,
thesis documents and lecture notes somewhat more convenient.
However, the package can also be used efficiently for
composing a series of documents (such as exercise sheets)
which are typically distributed individually.
It then assists the author in generating the individual documents
(potentially in different versions)
as well as a document containing the collected series.
Another application is in developing style files
or other kinds of included material
where compilation of the style file could redirect
to a sample or test file.

%%%%%%%%%%%%%%%%%%%%%%%%%%%%%%%%%%%%%%%%%%%%%%%%%%%%%%%%%%%%%%%%%%%%%%%%%%%%%%%%
%%%%%%%%%%%%%%%%%%%%%%%%%%%%%%%%%%%%%%%%%%%%%%%%%%%%%%%%%%%%%%%%%%%%%%%%%%%%%%%%
\section{Usage}

First of all, the package \textsf{childdoc} is \emph{not} a standard
\LaTeXe{} |.sty| style file! Therefore it needs to be invoked in
a non-standard way.

%%%%%%%%%%%%%%%%%%%%%%%%%%%%%%%%%%%%%%%%%%%%%%%%%%%%%%%%%%%%%%%%%%%%%%%%%%%%%%%%
\subsection{Included Files}
\label{sec:include}

%%%%%%%%%%%%%%%%%%%%%%%%%%%%%%%%%%%%%%%%
\DescribeMacro{\childdocmain}
To use the package, add the commands
\begin{center}
\begin{tabular}{l}
|\input{childdoc.def}|\\
|\childdocmain{}|\\
\end{tabular}
\end{center}
at the very top of the main \LaTeX{} file,
in particular \emph{before} the |\documentclass| statement!
The argument of |\childdocmain| should be left empty
(but it must be present).

%%%%%%%%%%%%%%%%%%%%%%%%%%%%%%%%%%%%%%%%
\DescribeMacro{\childdocof}
Furthermore, add the commands
\begin{center}
\begin{tabular}{l}
|\input{childdoc.def}|\\
|\childdocof{|\textit{main}|}|\\
\end{tabular}
\end{center}
at the top of every child file \textit{child}
which is included by |\include{|\textit{child}|}|
from within the main file
(or at least for those files to be compiled individually).
The argument \textit{main} must be the filename of the main file.

There are a couple of
considerations in setting up the main and child documents:

%%%%%%%%%%%%%%%%%%%%%%%%%%%%%%%%%%%%%%%%
\paragraph{Restrictions.}

Please note the following restrictions:
\begin{itemize}
\item
|\childdocmain| must be called with one argument \textit{main}
to ensure compatibility with earlier version of the package.
It must either be empty (|\childdocmain{}|)
or precisely match the filename of the main file in which it is specified.
See \secref{sec:detection} for further information.
\item
The filename \textit{main} must be specified without the |.tex| extension.
\item
The filename \textit{main} is case sensitive
(even in case-insensitive file systems)
due to internal string comparison.
\item
The argument \textit{main} should be fully expanded, it cannot be a macro.
\item
Subdirectories and special characters should be avoided in filenames.
\item
The command |\childdocmain{|\textit{main}|}| must be followed by a whitespace.
It should not be followed immediately by another command
or by a comment mark `|%|'.
This is because the \TeX{} parser reads the token immediately following
the argument of |\childdocmain| and puts it
at the beginning of every child section;
however, a white\-space is ignored.
\end{itemize}

%%%%%%%%%%%%%%%%%%%%%%%%%%%%%%%%%%%%%%%%
\paragraph{Content of Main File.}

It is advisable to place all content in the child files included by |\include|.
Any output contained in the main file will appear in all child documents
unless suppressed manually;
it cannot be suppressed automatically by the |\includeonly| directive
and thus should normally be avoided.
A method to include some content in the main file
by means of conditional processing is described in \secref{sec:conditional}.

%%%%%%%%%%%%%%%%%%%%%%%%%%%%%%%%%%%%%%%%
\paragraph{Page Numbering.}

When only a part of the document is compiled,
the appropriate numbering of pages
(as well as other status parameters)
is determined from the |.aux| files.
The latter contain information from previous passes.
However this information needs to propagate through
all intermediate child documents.
Therefore the page numbering in child documents may well
be inconsistent until the complete document is compiled at least once.

A useful (if unconventional) way to always ensure a consistent
page numbering is to restart the numbering in each child document
and denote the pages by `\textit{child}|.|\textit{page}'
where \textit{child} represents the chapter/section number of the child file.
This can be achieved by the command
|\numberwithin{page}{|\textit{child}|}|
of the \textsf{amsmath} package
where \textit{child} can be |chapter| or |section|
depending on the chosen structuring.
Alternatively, one can modify the macro |\thepage| appropriately
and reset the counter |page| at the start of each child file.

%%%%%%%%%%%%%%%%%%%%%%%%%%%%%%%%%%%%%%%%%%%%%%%%%%%%%%%%%%%%%%%%%%%%%%%%%%%%%%%%
\subsection{Conditional Processing}
\label{sec:conditional}

The package provides a mechanism to compile different versions
of a document. To customise the versions further some conditional processing
can come in handy to distinguish which version is being compiled.
The package provides two macros to describe the compilation context:

%%%%%%%%%%%%%%%%%%%%%%%%%%%%%%%%%%%%%%%%
\DescribeMacro{\ifchilddoc}
The conditional |\ifchilddoc| distinguishes between the compilation of
child documents and the main document:
%
\begin{center}
|\ifchilddoc |\textit{child-code}| |[|\||else |\textit{main-code}]| \||fi|
\end{center}

%%%%%%%%%%%%%%%%%%%%%%%%%%%%%%%%%%%%%%%%
\DescribeMacro{\childdocname}
\DescribeMacro{\childdocjob}
The macro |\childdocname| contains the filename (without extension)
of the main or child file being processed.
Note that |\childdocjob| will always contain the name of the main file.

%%%%%%%%%%%%%%%%%%%%%%%%%%%%%%%%%%%%%%%%
\paragraph{Title Page.}

Conditional processing can be used to include a title or banner page
in the main document when proper precautions are taken.
Importantly, the code in the main file should ensure that the page counter
(as well as other status parameters which are stored in the |.aux| files)
takes the same value after the conditional processing.
Otherwise the page numbers may take divergent values
depending on which part is compiled.

For example, a title page could be declared by:
%
\begin{center}
\begin{tabular}{l}
|\ifchilddoc\||else|\\
|\addtocounter{page}{-1}|\\
\textit{code for title page}\\
|\newpage|\\
|\||fi|
\end{tabular}
\end{center}
%
A banner page for the child documents can be generated by:
%
\begin{center}
\begin{tabular}{l}
|\ifchilddoc|\\
|\addtocounter{page}{-1}|\\
\textit{code for banner page}\\
|\newpage|\\
|\||fi|
\end{tabular}
\end{center}
%
Here one could write a message such as:
\begin{center}
|This is the part \childdocname{} of \childdocjob{}.|
\end{center}

%%%%%%%%%%%%%%%%%%%%%%%%%%%%%%%%%%%%%%%%%%%%%%%%%%%%%%%%%%%%%%%%%%%%%%%%%%%%%%%%
\subsection{Flags}
\label{sec:flags}

The package makes it easy to generate different versions
of the main or child documents.
To this end compilation flags can be defined
and assigned different default values.
They will be particularly useful in conjunction
with the forwarding mechanism described in \secref{sec:forward}.

For example, it may be useful to have a flag |\version|
which can be set to |draft| or |final|.
The document source will contain some conditional code
depending on the value of |\version|.
Suppose further, the flag should default to |final| for the main file
and to |draft| for child files
which is a natural assignment for editing the document.
This is achieved by placing the following code
in the preamble of the main document
(below the |\childdocmain| directive):
%
\begin{center}
\begin{tabular}{l}
|\ifchilddoc|\\
|\providecommand{\version}{draft}|\\
|\||else|\\
|\providecommand{\version}{final}|\\
|\||fi|
\end{tabular}
\end{center}
%
The definition by |\providecommand| makes sure
that previous definitions are not overwritten.
Further statements |\providecommand{\version}{...}|
can thus be added before the above code to override it.

For the main file, one might add a line
(between |\childdocmain| and the above block)
%
\begin{center}
|%\ifchilddoc\||else\providecommand{\version}{draft}\||fi|
\end{center}
%
which can be uncommented to produce a draft version.
Likewise one can add a line to the very top of a child file
(above the |\childdocof{|\textit{main}|}| directive)
%
\begin{center}
|%\providecommand{\version}{final}|
\end{center}
%
which can be uncommented to produce the final version of this child document.

%%%%%%%%%%%%%%%%%%%%%%%%%%%%%%%%%%%%%%%%%%%%%%%%%%%%%%%%%%%%%%%%%%%%%%%%%%%%%%%%
\subsection{Forwarding}
\label{sec:forward}

Different versions of the main or child documents
using compilation flags as described in \secref{sec:flags}
can be (permanently) stored in different files
for convenient compilation, viewing and distribution.
To this end, the package defines a command
to pass on compilation to a different file:

%%%%%%%%%%%%%%%%%%%%%%%%%%%%%%%%%%%%%%%%
\DescribeMacro{\childdocforward}
The command |\childdocforward| redirects processing to
another source file:
%
\begin{center}
\begin{tabular}{l}
|\input{childdoc.def}|\\
|\childdocforward[|\textit{main}|]{|\textit{dest}|}|\\
\end{tabular}
\end{center}
%
The argument \textit{dest} is the destination file
(without extension).
It should be the main file or one of the child files.
Note that further \textsf{childdoc} directives
such as |\childdocof| and |\childdocforward|
in the indicated file will be processed in this form.
The optional argument \textit{main}
passes on directly to the main file \textit{main}
while pretending to compile the child \textit{dest}.
This form behaves as if \textit{dest}
issues |\childdocof{|\textit{main}|}| right away,
and no further \textsf{childdoc} directives will be processed.

%%%%%%%%%%%%%%%%%%%%%%%%%%%%%%%%%%%%%%%%
\DescribeMacro{\...prefix}
In the alternative form |\childdocforwardprefix|,
%
\begin{center}
\begin{tabular}{l}
|\input{childdoc.def}|\\
|\childdocforwardprefix[|\textit{main}|]{|\textit{prefix}|}{|\textit{dest}|}|
\end{tabular}
\end{center}
%
the destination file is determined by a pattern
depending on the current file:
To make this work, the current file must be called
`{\textit{prefix}\hspace{0.2em}\textit{suffix}}'
with \textit{prefix} matching precisely the argument.
Processing is then passed on to the file
`{\textit{dest}\hspace{0.2em}\textit{suffix}}'.
Surely, the same effect is achieved by
directly specifying the
argument `{\textit{dest}\hspace{0.2em}\textit{suffix}}'
in the first form.
However, that requires to set up a different file
for each child. With the alternative form of the command
all these files can have exactly the same content
which simplifies setting them up and maintaining them.

For example, the following file |draft.tex|
with a compilation flag |\version| as described in \secref{sec:flags}
compiles the main document as a draft:
%
\begin{center}
\begin{tabular}{l}
|\def\version{draft}|\\
|\input{childdoc.def}|\\
|\childdocforward{|\textit{main}|}|
\end{tabular}
\end{center}
%
Likewise, the following files |final|\textit{nn}|.tex|
compile the final version of the child document
|child|\textit{nn}|.tex|:
%
\begin{center}
\begin{tabular}{l}
|\def\version{final}|\\
|\input{childdoc.def}|\\
|\childdocforwardprefix{final}{child}|
\end{tabular}
\end{center}
%

Note that when several versions of a main file and/or of each child file
are to be generated, it may be convenient to set up a |Makefile| or
shell script to automatise the process.

%%%%%%%%%%%%%%%%%%%%%%%%%%%%%%%%%%%%%%%%%%%%%%%%%%%%%%%%%%%%%%%%%%%%%%%%%%%%%%%%
\subsection{Command Line Processing}
\label{sec:commandline}

The effect of redirection files can also be achieved by invoking
the \LaTeX{} compiler with a more elaborate command line.
Most conveniently this should be done as part
of a shell script or a |Makefile|.

When using \textsf{childdoc} in the main file, the following
command lines effectively perform a redirection
(note that depending on the shell being used,
backslashes may have to be doubled: `|\|' $\to$ `|\\|'):
%
\begin{center}
|... -jobname "|\textit{target}|" |\\|"|[\textit{flags}]%
|\input{childdoc.def}\childdocforward[|\textit{main}|]{|\textit{dest}|}"|
\end{center}
%
Here \textit{target} is the name of the output file,
\textit{main} is the name of the main file
and \textit{dest} is the name of the main or child file to be processed
(all filenames without extensions).
The optional argument \textit{main} can be omitted
if \textit{main} matches \textit{dest}.
Optionally, compilation \textit{flags} can be defined via |\def| commands.
This command line makes the \TeX{} engine believe
it is compiling the file \textit{target}
whose content is specified as the latter parameter.
The provided code then forwards the processing to
\textit{main} or \textit{dest} as described in \secref{sec:forward}.

%%%%%%%%%%%%%%%%%%%%%%%%%%%%%%%%%%%%%%%%%%%%%%%%%%%%%%%%%%%%%%%%%%%%%%%%%%%%%%%%
\subsection{Include by Input}
\label{sec:input}

Including child documents by |\include| has some restrictions by design.
Most notably, the content of a child document always occupies
its own set of pages; pages cannot be shared between child documents.
Usually, this behaviour makes perfect sense
because each child document contain an essential part of the document.
However, in some situations it may be desirable to compose
a document from a collection of parts
without having mandatory page breaks between then.
For this case, the package
provides a mechanism to include parts
by |\input| which can also be processed individually.
However, by construction this mechanism
requires manual handling of the content to be output.

%%%%%%%%%%%%%%%%%%%%%%%%%%%%%%%%%%%%%%%%
\DescribeMacro{\ifchilddocmanual}
The main file should be prepared as usual, see \secref{sec:include}.
However, the document body must make a distinction
between processing of an individual part and of the main document, e.g.:
%
\begin{center}
\begin{tabular}{l}
|\ifchilddocmanual|\\
|\input{\childdocname}|\\
|\||else|\\
\textit{document body with }|\input{|\textit{part}|}|\\
|\||fi|
\end{tabular}
\end{center}
%
The conditional |\ifchilddocmanual| is true whenever
a part to be included by |\input| is being compiled,
and the name of the part is stored in |\childdocname|.

%%%%%%%%%%%%%%%%%%%%%%%%%%%%%%%%%%%%%%%%
\DescribeMacro{\childdocby}
Each part to be included by |\input| should start with:
%
\begin{center}
\begin{tabular}{l}
|\input{childdoc.def}|\\
|\childdocby{|\textit{main}|}|\\
\end{tabular}
\end{center}
%
The directive |\childdocby| is similar to |\childdocof|
described in \secref{sec:include},
but the subsequent selection of content must be done manually.
To that end, both |\ifchilddoc| and |\ifchilddocmanual|
will be true upon processing of a part,
and the name of the part is stored in |\childdocname|.
Note that |\jobname| will be set to the filename of the current part
so that each part receives an individual |.aux| file
that does not interfere with the |.aux| file(s) of the main document.
This behaviour can be altered by the alternative form
|\childdocby[*]{|\textit{main}|}| (with a non-empty optional argument)
which uses the |.aux| file of the main document
by setting |\jobname| to \textit{main}.

%%%%%%%%%%%%%%%%%%%%%%%%%%%%%%%%%%%%%%%%%%%%%%%%%%%%%%%%%%%%%%%%%%%%%%%%%%%%%%%%
\subsection{Driver Development}
\label{sec:driver}

The \textsf{childdoc} mechanism can also be use for the development
of definition files such as \LaTeX{} styles or classes.
This case differs from the above setup with multiple parts
included by |\include| in that no |\includeonly| should be invoked.
This can be achieved by starting the include file
(before |\ProvidesPackage|) with:
%
\begin{center}
\begin{tabular}{l}
|\input{childdoc.def}|\\
|\childdocforward{|\textit{main}|}|\\
\end{tabular}
\end{center}
%
or alternatively with:
%
\begin{center}
\begin{tabular}{l}
|\input{childdoc.def}|\\
|\childdocby{|\textit{main}|}|\\
\end{tabular}
\end{center}
%
Both forms have slightly different effects as described above.
The main file is prepared as usual, see \secref{sec:include}.

%%%%%%%%%%%%%%%%%%%%%%%%%%%%%%%%%%%%%%%%%%%%%%%%%%%%%%%%%%%%%%%%%%%%%%%%%%%%%%%%
\subsection{Legacy Detection}
\label{sec:detection}

The directive |\childdocmain| in the main file can detect
whether the complete document or merely a child is to be compiled
even without using the directive |\childdocof|.
This method is deprecated because it is less robust
and there is no compelling reason to use it;
it is merely provided for backward compatibility
and it may be removed in future versions.

If the detection mechanism is to be used,
it is mandatory to correctly specify
the filename of the main file as the argument of |\childdocmain|:
%
\begin{center}
\begin{tabular}{l}
|\input{childdoc.def}|\\
|\childdocmain{|\textit{main}|}|\\
\end{tabular}
\end{center}
%
If |\jobname| does not match the argument \textit{main} of |\childdocmain|,
it is assumed that |\jobname| points to the child file to be compiled.
When using |\childdocmain| with the main file specified as argument,
it suffices to start a child file
with just |\input{|\textit{main}|}|
without loading of the package and using |\childdocof|.
If instead all processing is done
with the appropriate \textsf{childdoc} directives,
the argument of \textit{main} of |\childdocmain| can be empty.

An alternative version of the command line processing described
in \secref{sec:commandline} using the detection mechanism reads:
%
\begin{center}
|... -jobname "|\textit{target}|" "|[\textit{flags}]%
[|\def\jobname{|\textit{dest}|}|]|\input{|\textit{main}|}"|
\end{center}

%%%%%%%%%%%%%%%%%%%%%%%%%%%%%%%%%%%%%%%%%%%%%%%%%%%%%%%%%%%%%%%%%%%%%%%%%%%%%%%%
\subsection{Manual Code}
\label{sec:manual}

In case one cannot be certain whether the definitions file |childdoc.def|
is installed on the target \TeX{} distribution
and one prefers not to ship it,
it is conceivable to paste a few relevant commands into the sources.

To that end, drop all statements |\input{childdoc.def}|
and perform the replacements as outlined below.
Instead of |\childdocmain{|\textit{main}|}| add the following code
to the top of the main file:
%
\begin{center}
\begin{tabular}{l}
|\||ifdefined\childdocname\endinput\||fi\newif\ifchilddoc|\\
|\edef\childdocname{\scantokens\expandafter{\jobname\noexpand}}|\\
|\def\childdocmain{|\textit{main}|}\||ifx\childdocmain\childdocname\||else|\\
|\childdoctrue\includeonly{\childdocname}\let\jobname\childdocmain\||fi|\\
\end{tabular}
\end{center}
%
Instead of |\childdocof{|\textit{main}|}| just include the main file
at the top of each child file:
%
\begin{center}
|\input{|\textit{main}|}|
\end{center}
%
A simple redirection |\childdocforward{|\textit{dest}|}| is achieved by:
%
\begin{center}
|\def\jobname{|\textit{dest}|}\input{\jobname}|
\end{center}
%
The redirection with prefix
|\childdocforwardprefix[|\textit{prefix}|]{|\textit{dest}|}|
is accomplished by:
%
\begin{center}
\begin{tabular}{l}
|{\edef\jobname{\scantokens\expandafter{\jobname\noexpand}}|\\
|\def\redirectjob |\textit{prefix}|#1~~~{\gdef\jobname{|\textit{dest}|#1}}|\\
|\expandafter\redirectjob\jobname~~~}\input{\jobname}|
\end{tabular}
\end{center}

In an alternative approach,
child documents can be compiled by a specific command line
without additional code or specific definitions:
%
\begin{center}
|... -jobname "|\textit{target}|" "|[\textit{flags}]%
|\includeonly{|\textit{dest}|}\input{|\textit{main}|}"|
\end{center}
%

%%%%%%%%%%%%%%%%%%%%%%%%%%%%%%%%%%%%%%%%%%%%%%%%%%%%%%%%%%%%%%%%%%%%%%%%%%%%%%%%
%%%%%%%%%%%%%%%%%%%%%%%%%%%%%%%%%%%%%%%%%%%%%%%%%%%%%%%%%%%%%%%%%%%%%%%%%%%%%%%%
\section{Information}

%%%%%%%%%%%%%%%%%%%%%%%%%%%%%%%%%%%%%%%%%%%%%%%%%%%%%%%%%%%%%%%%%%%%%%%%%%%%%%%%
\subsection{Copyright}

Copyright \copyright{} 2017--2018 Niklas Beisert

This work may be distributed and/or modified under the
conditions of the \LaTeX{} Project Public License, either version 1.3
of this license or (at your option) any later version.
The latest version of this license is in
  \url{http://www.latex-project.org/lppl.txt}
and version 1.3 or later is part of all distributions of \LaTeX{}
version 2005/12/01 or later.

This work has the LPPL maintenance status `maintained'.

The Current Maintainer of this work is Niklas Beisert.

This work consists of the files |README.txt|, |childdoc.ins| and |childdoc.dtx|
as well as the derived files |childdoc.def|, |cdocsamp.tex|
with |cdocsch1.tex|, |cdocsch2.tex|, |cdocspt3.tex|, |cdocspt4.tex|,
|cdocsdrf.tex|, |cdocsfn1.tex|, |cdocsfn2.tex|
as well as |childdoc.pdf|.

%%%%%%%%%%%%%%%%%%%%%%%%%%%%%%%%%%%%%%%%%%%%%%%%%%%%%%%%%%%%%%%%%%%%%%%%%%%%%%%%
\subsection{Files and Installation}

The package consists of the files:
%
\begin{center}
\begin{tabular}{ll}
    |README.txt|   & readme file \\
    |childdoc.ins| & installation file \\
    |childdoc.dtx| & source file \\
    |childdoc.def| & definition file \\
    |cdocsamp.tex| & sample main file \\
    |cdocsch1.tex| & sample include file \\
    |cdocsch2.tex| & sample include file \\
    |cdocspt3.tex| & sample part file \\
    |cdocspt4.tex| & sample part file \\
    |cdocsdrf.tex| & sample redirection file \\
    |cdocsfn1.tex| & sample redirection file \\
    |cdocsfn2.tex| & sample redirection file \\
    |childdoc.pdf| & manual
\end{tabular}
\end{center}
%
The distribution consists of the files
|README.txt|, |childdoc.ins| and |childdoc.dtx|.
%
\begin{itemize}
\item
Run (pdf)\LaTeX{} on |childdoc.dtx|
to compile the manual |childdoc.pdf| (this file).
\item
Run \LaTeX{} on |childdoc.ins| to create the definitions file |childdoc.def|
and the sample |cdocsamp.tex| with include files
|cdocsch1.tex|, |cdocsch2.tex|, |cdocspt3.tex|, |cdocspt4.tex|,
|cdocsdrf.tex|, |cdocsfn1.tex|, |cdocsfn2.tex|.
Then copy the file |childdoc.def| to an appropriate directory of your \LaTeX{}
distribution, e.g.\ \textit{texmf-root}|/tex/latex/childdoc|.
\end{itemize}

%%%%%%%%%%%%%%%%%%%%%%%%%%%%%%%%%%%%%%%%%%%%%%%%%%%%%%%%%%%%%%%%%%%%%%%%%%%%%%%%
\subsection{Related CTAN Packages}

There are several other packages which offer a similar functionality:
%
\begin{itemize}
\item
The packages
\href{http://ctan.org/pkg/docmute}{\textsf{docmute}},
\href{http://ctan.org/pkg/includex}{\textsf{includex}} and
\href{http://ctan.org/pkg/standalone}{\textsf{standalone}}
provide commands to include only the document body of
a child file thus allowing both files to be compiled individually.
\item
The packages \href{http://ctan.org/pkg/subdocs}{\textsf{subdocs}}
and \href{http://ctan.org/pkg/subfiles}{\textsf{subfiles}}
provide structures in which the main and child documents can be
encapsulated and allowing them to be compiled individually.
The inclusion mechanism is different from the conventional |\include|.
\item
The package \href{http://ctan.org/pkg/combine}{\textsf{combine}}
is an elaborate solution to combine several documents into one.
\end{itemize}
%
See also the CTAN topic \href{http://ctan.org/topic/subdocs}{\textsf{subdocs}}
for further related packages.
The present package differs from the above solutions in that
a document structure constructed with the conventional |\include| mechanism
just needs two extra commands at the top of every file
such that all constituent files can be compiled individually.

%%%%%%%%%%%%%%%%%%%%%%%%%%%%%%%%%%%%%%%%%%%%%%%%%%%%%%%%%%%%%%%%%%%%%%%%%%%%%%%%
%\subsection{Feature Suggestions}
%
%The following is a list of features which may be useful for future
%versions of this package:
%%
%\begin{itemize}
%\item
%\ldots
%\end{itemize}

%%%%%%%%%%%%%%%%%%%%%%%%%%%%%%%%%%%%%%%%%%%%%%%%%%%%%%%%%%%%%%%%%%%%%%%%%%%%%%%%
\subsection{Revision History}

%%%%%%%%%%%%%%%%%%%%%%%%%%%%%%%%%%%%%%%%
\paragraph{v2.0:} 2018/12/30

\begin{itemize}
\item
immediate forward processing
\item
added |\childdocby| mechanism
\item
manual restructured
\end{itemize}

%%%%%%%%%%%%%%%%%%%%%%%%%%%%%%%%%%%%%%%%
\paragraph{v1.6:} 2018/01/17

\begin{itemize}
\item
application for development of include files
\item
corrections to manual
\end{itemize}

%%%%%%%%%%%%%%%%%%%%%%%%%%%%%%%%%%%%%%%%
\paragraph{v1.5:} 2017/05/21

\begin{itemize}
\item
more complete structuring introduced
\item
|\childdocof| introduced
\item
|\childdoc| renamed to |\childdocmain|
\item
|\childredirect| renamed to |\childdocforward| and |\childdocforwardprefix|
and functionality expanded
\end{itemize}

%%%%%%%%%%%%%%%%%%%%%%%%%%%%%%%%%%%%%%%%
\paragraph{v1.0:} 2017/04/27

\begin{itemize}
\item
manual and install package
\item
first version published on CTAN
\end{itemize}

%%%%%%%%%%%%%%%%%%%%%%%%%%%%%%%%%%%%%%%%
\paragraph{v0.6:} 2017/04/26

\begin{itemize}
\item
redirection mechanism added
\end{itemize}

%%%%%%%%%%%%%%%%%%%%%%%%%%%%%%%%%%%%%%%%
\paragraph{v0.5:} 2017/04/26

\begin{itemize}
\item
functionality in definition file
\end{itemize}


%%%%%%%%%%%%%%%%%%%%%%%%%%%%%%%%%%%%%%%%%%%%%%%%%%%%%%%%%%%%%%%%%%%%%%%%%%%%%%%%
%%%%%%%%%%%%%%%%%%%%%%%%%%%%%%%%%%%%%%%%%%%%%%%%%%%%%%%%%%%%%%%%%%%%%%%%%%%%%%%%
%%%%%%%%%%%%%%%%%%%%%%%%%%%%%%%%%%%%%%%%%%%%%%%%%%%%%%%%%%%%%%%%%%%%%%%%%%%%%%%%
\appendix

\settowidth\MacroIndent{\rmfamily\scriptsize 000\ }

 \DocInput{childdoc.dtx}

\end{document}
%</driver>
% \fi
%
% %%%%%%%%%%%%%%%%%%%%%%%%%%%%%%%%%%%%%%%%%%%%%%%%%%%%%%%%%%%%%%%%%%%%%%%%%%%%%%
% %%%%%%%%%%%%%%%%%%%%%%%%%%%%%%%%%%%%%%%%%%%%%%%%%%%%%%%%%%%%%%%%%%%%%%%%%%%%%%
% \section{Sample}
%\iffalse
%<*samplemain>
%\fi
%
% The following presents a sample document
% with two chapters, two parts, a title page,
% a compile flag as well as three forwarding files to set the flag.
% It consists of eight |.tex| files:
% \begin{center}
% \begin{tabular}{ll}
% |cdocsamp.tex|&main file\\
% |cdocsch1.tex|&include file for chapter 1\\
% |cdocsch2.tex|&include file for chapter 2\\
% |cdocspt3.tex|&include file for part 3\\
% |cdocspt4.tex|&include file for part 4\\
% |cdocsdrf.tex|&forwarding file for main file in draft mode\\
% |cdocsfi1.tex|&forwarding file for final version of chapter 1\\
% |cdocsfi2.tex|&forwarding file for final version of chapter 2\\
% \end{tabular}
% \end{center}
% Each of the eight files can be compiled directly by the \LaTeX{} compiler.
%
% %%%%%%%%%%%%%%%%%%%%%%%%%%%%%%%%%%%%%%
% \paragraph{Main File.}
%
% The main file is called |cdocsamp.tex|.
%
% Load the \textsf{childdoc} definitions and
% declare the filename for the main document:
%    \begin{macrocode}
\input{childdoc.def}
\childdocmain{}
%    \end{macrocode}

% Optional override for |\version| flag:
%    \begin{macrocode}
%%\ifchilddoc\else\providecommand{\version}{draft}\fi
%    \end{macrocode}

% Define the default values for the |\version| flag
% (|final| for the main file and |draft| for childs):
%    \begin{macrocode}
\ifchilddoc
\providecommand{\version}{draft}
\else
\providecommand{\version}{final}
\fi
%    \end{macrocode}

% Load the standard document class:
%    \begin{macrocode}
\documentclass[12pt]{article}
%    \end{macrocode}

% Start the document body:
%    \begin{macrocode}
\begin{document}
%    \end{macrocode}

% Declare a title page.
% Print title, part of document being processed and version flag:
%    \begin{macrocode}
\addtocounter{page}{-1}
\begin{center}
{\LARGE\bfseries{}childdoc example\par}
\vspace{1cm}
\ifchilddoc
\ifchilddocmanual part\else chapter\fi:
`\childdocname' of `\childdocjob'\par
\else
main document: `\childdocjob'\par
\fi
version: \version\par
\end{center}
\newpage
%    \end{macrocode}

% Manually include selected file,
% otherwise process as usual:
%    \begin{macrocode}
\ifchilddocmanual
\section*{part `\childdocname'}
\input{\childdocname}
\else
%    \end{macrocode}

% Include the two chapters:
%    \begin{macrocode}
\include{cdocsch1}
\include{cdocsch2}
%    \end{macrocode}

% Include the two parts unless only chapters should be displayed:
%    \begin{macrocode}
\ifchilddoc\else
\section{part three}
\input{cdocspt3}
\section{part four}
\input{cdocspt4}
\fi
%    \end{macrocode}

% Process as usual until here:
%    \begin{macrocode}
\fi
%    \end{macrocode}

% End of document body:
%    \begin{macrocode}
\end{document}
%    \end{macrocode}
%\iffalse
%</samplemain>
%\fi
%
% %%%%%%%%%%%%%%%%%%%%%%%%%%%%%%%%%%%%%%
% \paragraph{Chapter Include Files.}
%
% The include files are called |cdocsch1.tex| and |cdocsch2.tex|.
%
%\iffalse
%<*samplechap1|samplechap2>
%\fi

% Optional override for |\version| flag:
%    \begin{macrocode}
%%\providecommand{\version}{final}
%    \end{macrocode}

% Include the main document:
%    \begin{macrocode}
\input{childdoc.def}
\childdocof{cdocsamp}
%    \end{macrocode}

%\iffalse
%</samplechap1|samplechap2>
%\fi
%
%\iffalse
%<*samplechap1>
%\fi
% Some text for chapter 1:
%    \begin{macrocode}
\section{one}
some text in chapter one
%    \end{macrocode}

%\iffalse
%</samplechap1>
%\fi
% Some text for chapter 2:
%\iffalse
%<*samplechap2>
%\fi
%    \begin{macrocode}
\section{two}
more text in chapter two
%    \end{macrocode}

%\iffalse
%</samplechap2>
%\fi
%
% %%%%%%%%%%%%%%%%%%%%%%%%%%%%%%%%%%%%%%
% \paragraph{Part Include Files.}
%
% The include files are called |cdocspt3.tex| and |cdocspt4.tex|.
%
%\iffalse
%<*samplepart3|samplepart4>
%\fi

% Optional override for |\version| flag:
%    \begin{macrocode}
%%\providecommand{\version}{final}
%    \end{macrocode}

% Include the main document:
%    \begin{macrocode}
\input{childdoc.def}
\childdocby{cdocsamp}
%    \end{macrocode}

%\iffalse
%</samplepart3|samplepart4>
%\fi
%
%\iffalse
%<*samplepart3>
%\fi
% Some text for part 3:
%    \begin{macrocode}
some text in part three
%    \end{macrocode}

%\iffalse
%</samplepart3>
%\fi
% Some text for part 4:
%\iffalse
%<*samplepart4>
%\fi
%    \begin{macrocode}
more text in part four
%    \end{macrocode}

%\iffalse
%</samplepart4>
%\fi
%
% %%%%%%%%%%%%%%%%%%%%%%%%%%%%%%%%%%%%%%
% \paragraph{Forwarding for a Complete Draft.}
%
% The following forwarding file |cdocsdrf.tex|
% compiles the main document in draft mode:
%\iffalse
%<*sampledraft>
%\fi
%    \begin{macrocode}
\def\version{draft}
\input{childdoc.def}
\childdocforward{cdocsamp}
%    \end{macrocode}

%\iffalse
%</sampledraft>
%\fi
%
% %%%%%%%%%%%%%%%%%%%%%%%%%%%%%%%%%%%%%%
% \paragraph{Forwarding for Final Version of the Chapters.}
%
% The following forwarding files |cdocsfn1.tex| and |cdocsfn2.tex|
% (with identical content)
% compile the final versions of the child documents
% |cdocsch1.tex| and |cdocsch2.tex|, respectively:
%\iffalse
%<*samplefinal>
%\fi
%    \begin{macrocode}
\def\version{final}
\input{childdoc.def}
\childdocforwardprefix[cdocsamp]{cdocsfn}{cdocsch}
%    \end{macrocode}

%\iffalse
%</samplefinal>
%\fi
%
% %%%%%%%%%%%%%%%%%%%%%%%%%%%%%%%%%%%%%%
% \paragraph{Command Line Processing.}
%
% The following three command lines generate the output files
% |cdocscld|, |cdocscl1| and |cdocscl2|
% which should be identical to
% |cdocsdrf|, |cdocsch1| and |cdocsfn2|, respectively:
% \begin{center}
% \begin{tabular}{l}
% |latex -jobname cdocscld \|\\
% |  "\def\version{draft}\input{childdoc.def}\childdocforward{cdocsamp}"|\\
% |latex -jobname cdocscl1 \|\\
% |  "\input{childdoc.def}\childdocforward[cdocsamp]{cdocsch1}"|\\
% |latex -jobname cdocscl2 \|\\
% |  "\def\version{final}\input{childdoc.def}\childdocforward{cdocsch2}"|
% \end{tabular}
% \end{center}
% Note that the trailing backslash on each first line
% merely continues the input to the second line
% (for convenient cut ant paste).
% Furthermore, the command |latex| can be replaced by any
% of its alternative versions such as |pdflatex|.
%
% %%%%%%%%%%%%%%%%%%%%%%%%%%%%%%%%%%%%%%%%%%%%%%%%%%%%%%%%%%%%%%%%%%%%%%%%%%%%%%
% %%%%%%%%%%%%%%%%%%%%%%%%%%%%%%%%%%%%%%%%%%%%%%%%%%%%%%%%%%%%%%%%%%%%%%%%%%%%%%
% \section{Implementation}
%\iffalse
%<*package>
%\fi
%
% This section describes the definitions file |childdoc.def|.

% The definitions cannot be loaded using |\usepackage| or |\RequirePackage|
% which has a mechanism to prevent loading a style file more than once.
% When loading the definitions by means of |\input|
% multiple instances have to be prevented manually:
%\iffalse
%This code needs to be before the `\ProvidesFile' directive
%which is defined at the beginning of this file.
%Therefore it is also placed there and commented out here.
%</package>
%<*discard>
%\fi
%    \begin{macrocode}
\ifdefined\childdocmain\endinput\fi
%    \end{macrocode}
%\iffalse
%</discard>
%<*package>
%\fi
%
% \macro{\ifchilddoc}
% \macro{\ifchilddocmanual}
% The conditional |\ifchilddoc| tells whether a
% child (true) or main (false) document is being compiled.
% The conditional |\ifchilddocmanual| tells whether
% the |\includeonly| mechanism is used (false) or
% the selection of child files must be performed manually (true).
% The definitions initialise to false:
%    \begin{macrocode}
\newif\ifchilddoc
\newif\ifchilddocmanual
%    \end{macrocode}

% \macro{\childdocname}
% \macro{\childdocjob}
% The macro |\childdocname| stores the name of the main document
% to be compiled. The macro |\childdocjob| stores the name of
% the document on which the \LaTeX{} compiler was originally invoked.
% The content of |\jobname| cannot be compared
% to filenames specified in the source due to different catcodes.
% The following code rescans |\jobname|, stores the result
% in |\childdocname| and saves a copy in |\childdocjob|:
%    \begin{macrocode}
\edef\childdocname{\scantokens\expandafter{\jobname\noexpand}}
\let\childdocjob\childdocname
%    \end{macrocode}

% \macro{\childdocdisable}
% The macro |\childdocdisable| prevents the main file
% from being processed more than once.
% At this stage, the main document command |\childdocmain|
% is assumed to be called once again where it should do nothing.
% Any subsequent call to it should prevent
% a secondary processing of the main document
% It overwrites the forwarding commands
% |\childdocof| and |\childdocforward|
% with empty macros to prevent further inclusions of the main document:
%    \begin{macrocode}
\newcommand{\childdocdisable}
{
  \renewcommand{\childdocmain}[1]{\renewcommand{\childdocmain}[1]{\endinput}}
  \renewcommand{\childdocof}[1]{}
  \renewcommand{\childdocby}[2][]{}
  \renewcommand{\childdocforward}[2][]{}
  \renewcommand{\childdocdisable}{}
}
%    \end{macrocode}

% \macro{\childdocmain}
% The macro |\childdocmain| is to be called at the top of the main file
% with nothing or the main filename (without extension) as argument.
% First, it breaks loops.
% If the argument is not empty and does not match |\childdocname|
% (which is set by the first inclusion of |childdoc.def|),
% |\ifchilddoc| is set to true, |\includeonly| is applied to the child file
% and |\jobname| is set to the main file
% (for proper handling of |.aux| files):
%    \begin{macrocode}
\newcommand{\childdocmain}[1]
{
  \childdocdisable\childdocmain{}
  \if?#1?\else
    \begingroup
      \def\childdoctmp{#1}
      \ifx\childdoctmp\childdocname
        \def\childdoctmp{}
      \else
        \def\childdoctmp
        {
          \childdoctrue
          \includeonly{\childdocname}
          \def\childdocjob{#1}
          \def\jobname{#1}
        }
      \fi
      \expandafter
    \endgroup
    \childdoctmp
  \fi
}
%    \end{macrocode}

% \macro{\childdocof}
% The command |\childdocof| redirects
% compilation to the main file |#1|.
%    \begin{macrocode}
\newcommand{\childdocof}[1]
{
  \childdocdisable
  \childdoctrue
  \includeonly{\childdocname}
  \def\jobname{#1}
  \def\childdocjob{#1}
  \input{#1}
}
%    \end{macrocode}

% \macro{\childdocby}
% The command |\childdocby| ....
%    \begin{macrocode}
\newcommand{\childdocby}[2][]
{
  \childdocdisable
  \childdoctrue
  \childdocmanualtrue
  \if?#1?\else
    \def\jobname{#2}
  \fi
  \def\childdocjob{#2}
  \input{#2}
  \endinput
}
%    \end{macrocode}

% \macro{\childdocforward}
% The command |\childdocforward| redirects
% compilation to the main file or
% (if the optional argument is given) a child file.
% Parameters are set as if the main file
% or a child file starting with |\childdocof| was compiled.
% Then compilation is handed over to the main file:
%    \begin{macrocode}
\newcommand{\childdocforward}[2][]
{
  \begingroup
    \if?#1?
      \def\childdoctmp
      {
        \def\childdocname{#2}
        \def\childdocjob{#2}
        \def\jobname{#2}
        \input{#2}
        \endinput
      }
    \else
      \def\childdoctmp
      {
        \childdocdisable
        \def\childdocname{#2}
        \childdoctrue
        \includeonly{#2}
        \def\childdocjob{#1}
        \def\jobname{#1}
        \input{#1}
        \endinput
      }
    \fi
    \expandafter
  \endgroup
  \childdoctmp
}
%    \end{macrocode}

% \macro{\childdocforwardprefix}
% The command |\childdocforwardprefix| redirects
% compilation to the main or a child file by means of a pattern.
% The prefix |#1| in the current filename is replaced by |#2|
% and the suffix of the current filename is kept
% (it is assumed that the filename does not contain the substring `|~~~|'
% which is used as a delimiter).
% Compilation is handed over to the new file by |\childdocforward|:
%    \begin{macrocode}
\newcommand{\childdocforwardprefix}[3][]
{
  \begingroup
    \def\childdocextract #2##1~~~{\def\childdoctmp{\childdocforward[#1]{#3##1}}}
    \expandafter\childdocextract\childdocname~~~
    \expandafter
  \endgroup
  \childdoctmp
}
%    \end{macrocode}

% \macro{\childdoc}
% The deprecated macro |\childdoc| is a legacy version of |\childdocmain|:
%    \begin{macrocode}
\newcommand{\childdoc}{\childdocmain}
%    \end{macrocode}

% \macro{\childdocredirect}
% The deprecated macro |\childdocredirect| is a legacy version
% of |\childdocforward| and |\childdocforwardprefix|:
%    \begin{macrocode}
\newcommand{\childdocredirect}[2][]
{
  \begingroup
    \if?#1?
      \def\childdoctmp{\childdocforward{#2}}
    \else
      \def\childdoctmp{\childdocforwardprefix{#1}{#2}}
    \fi
    \expandafter
  \endgroup
  \childdoctmp
}
%    \end{macrocode}

%\iffalse
%</package>
%\fi
%
\endinput
|
and perform the replacements as outlined below.
Instead of |\childdocmain{|\textit{main}|}| add the following code
to the top of the main file:
%
\begin{center}
\begin{tabular}{l}
|\||ifdefined\childdocname\endinput\||fi\newif\ifchilddoc|\\
|\edef\childdocname{\scantokens\expandafter{\jobname\noexpand}}|\\
|\def\childdocmain{|\textit{main}|}\||ifx\childdocmain\childdocname\||else|\\
|\childdoctrue\includeonly{\childdocname}\let\jobname\childdocmain\||fi|\\
\end{tabular}
\end{center}
%
Instead of |\childdocof{|\textit{main}|}| just include the main file
at the top of each child file:
%
\begin{center}
|\input{|\textit{main}|}|
\end{center}
%
A simple redirection |\childdocforward{|\textit{dest}|}| is achieved by:
%
\begin{center}
|\def\jobname{|\textit{dest}|}\input{\jobname}|
\end{center}
%
The redirection with prefix
|\childdocforwardprefix[|\textit{prefix}|]{|\textit{dest}|}|
is accomplished by:
%
\begin{center}
\begin{tabular}{l}
|{\edef\jobname{\scantokens\expandafter{\jobname\noexpand}}|\\
|\def\redirectjob |\textit{prefix}|#1~~~{\gdef\jobname{|\textit{dest}|#1}}|\\
|\expandafter\redirectjob\jobname~~~}\input{\jobname}|
\end{tabular}
\end{center}

In an alternative approach,
child documents can be compiled by a specific command line
without additional code or specific definitions:
%
\begin{center}
|... -jobname "|\textit{target}|" "|[\textit{flags}]%
|\includeonly{|\textit{dest}|}\input{|\textit{main}|}"|
\end{center}
%

%%%%%%%%%%%%%%%%%%%%%%%%%%%%%%%%%%%%%%%%%%%%%%%%%%%%%%%%%%%%%%%%%%%%%%%%%%%%%%%%
%%%%%%%%%%%%%%%%%%%%%%%%%%%%%%%%%%%%%%%%%%%%%%%%%%%%%%%%%%%%%%%%%%%%%%%%%%%%%%%%
\section{Information}

%%%%%%%%%%%%%%%%%%%%%%%%%%%%%%%%%%%%%%%%%%%%%%%%%%%%%%%%%%%%%%%%%%%%%%%%%%%%%%%%
\subsection{Copyright}

Copyright \copyright{} 2017--2018 Niklas Beisert

This work may be distributed and/or modified under the
conditions of the \LaTeX{} Project Public License, either version 1.3
of this license or (at your option) any later version.
The latest version of this license is in
  \url{http://www.latex-project.org/lppl.txt}
and version 1.3 or later is part of all distributions of \LaTeX{}
version 2005/12/01 or later.

This work has the LPPL maintenance status `maintained'.

The Current Maintainer of this work is Niklas Beisert.

This work consists of the files |README.txt|, |childdoc.ins| and |childdoc.dtx|
as well as the derived files |childdoc.def|, |cdocsamp.tex|
with |cdocsch1.tex|, |cdocsch2.tex|, |cdocspt3.tex|, |cdocspt4.tex|,
|cdocsdrf.tex|, |cdocsfn1.tex|, |cdocsfn2.tex|
as well as |childdoc.pdf|.

%%%%%%%%%%%%%%%%%%%%%%%%%%%%%%%%%%%%%%%%%%%%%%%%%%%%%%%%%%%%%%%%%%%%%%%%%%%%%%%%
\subsection{Files and Installation}

The package consists of the files:
%
\begin{center}
\begin{tabular}{ll}
    |README.txt|   & readme file \\
    |childdoc.ins| & installation file \\
    |childdoc.dtx| & source file \\
    |childdoc.def| & definition file \\
    |cdocsamp.tex| & sample main file \\
    |cdocsch1.tex| & sample include file \\
    |cdocsch2.tex| & sample include file \\
    |cdocspt3.tex| & sample part file \\
    |cdocspt4.tex| & sample part file \\
    |cdocsdrf.tex| & sample redirection file \\
    |cdocsfn1.tex| & sample redirection file \\
    |cdocsfn2.tex| & sample redirection file \\
    |childdoc.pdf| & manual
\end{tabular}
\end{center}
%
The distribution consists of the files
|README.txt|, |childdoc.ins| and |childdoc.dtx|.
%
\begin{itemize}
\item
Run (pdf)\LaTeX{} on |childdoc.dtx|
to compile the manual |childdoc.pdf| (this file).
\item
Run \LaTeX{} on |childdoc.ins| to create the definitions file |childdoc.def|
and the sample |cdocsamp.tex| with include files
|cdocsch1.tex|, |cdocsch2.tex|, |cdocspt3.tex|, |cdocspt4.tex|,
|cdocsdrf.tex|, |cdocsfn1.tex|, |cdocsfn2.tex|.
Then copy the file |childdoc.def| to an appropriate directory of your \LaTeX{}
distribution, e.g.\ \textit{texmf-root}|/tex/latex/childdoc|.
\end{itemize}

%%%%%%%%%%%%%%%%%%%%%%%%%%%%%%%%%%%%%%%%%%%%%%%%%%%%%%%%%%%%%%%%%%%%%%%%%%%%%%%%
\subsection{Related CTAN Packages}

There are several other packages which offer a similar functionality:
%
\begin{itemize}
\item
The packages
\href{http://ctan.org/pkg/docmute}{\textsf{docmute}},
\href{http://ctan.org/pkg/includex}{\textsf{includex}} and
\href{http://ctan.org/pkg/standalone}{\textsf{standalone}}
provide commands to include only the document body of
a child file thus allowing both files to be compiled individually.
\item
The packages \href{http://ctan.org/pkg/subdocs}{\textsf{subdocs}}
and \href{http://ctan.org/pkg/subfiles}{\textsf{subfiles}}
provide structures in which the main and child documents can be
encapsulated and allowing them to be compiled individually.
The inclusion mechanism is different from the conventional |\include|.
\item
The package \href{http://ctan.org/pkg/combine}{\textsf{combine}}
is an elaborate solution to combine several documents into one.
\end{itemize}
%
See also the CTAN topic \href{http://ctan.org/topic/subdocs}{\textsf{subdocs}}
for further related packages.
The present package differs from the above solutions in that
a document structure constructed with the conventional |\include| mechanism
just needs two extra commands at the top of every file
such that all constituent files can be compiled individually.

%%%%%%%%%%%%%%%%%%%%%%%%%%%%%%%%%%%%%%%%%%%%%%%%%%%%%%%%%%%%%%%%%%%%%%%%%%%%%%%%
%\subsection{Feature Suggestions}
%
%The following is a list of features which may be useful for future
%versions of this package:
%%
%\begin{itemize}
%\item
%\ldots
%\end{itemize}

%%%%%%%%%%%%%%%%%%%%%%%%%%%%%%%%%%%%%%%%%%%%%%%%%%%%%%%%%%%%%%%%%%%%%%%%%%%%%%%%
\subsection{Revision History}

%%%%%%%%%%%%%%%%%%%%%%%%%%%%%%%%%%%%%%%%
\paragraph{v2.0:} 2018/12/30

\begin{itemize}
\item
immediate forward processing
\item
added |\childdocby| mechanism
\item
manual restructured
\end{itemize}

%%%%%%%%%%%%%%%%%%%%%%%%%%%%%%%%%%%%%%%%
\paragraph{v1.6:} 2018/01/17

\begin{itemize}
\item
application for development of include files
\item
corrections to manual
\end{itemize}

%%%%%%%%%%%%%%%%%%%%%%%%%%%%%%%%%%%%%%%%
\paragraph{v1.5:} 2017/05/21

\begin{itemize}
\item
more complete structuring introduced
\item
|\childdocof| introduced
\item
|\childdoc| renamed to |\childdocmain|
\item
|\childredirect| renamed to |\childdocforward| and |\childdocforwardprefix|
and functionality expanded
\end{itemize}

%%%%%%%%%%%%%%%%%%%%%%%%%%%%%%%%%%%%%%%%
\paragraph{v1.0:} 2017/04/27

\begin{itemize}
\item
manual and install package
\item
first version published on CTAN
\end{itemize}

%%%%%%%%%%%%%%%%%%%%%%%%%%%%%%%%%%%%%%%%
\paragraph{v0.6:} 2017/04/26

\begin{itemize}
\item
redirection mechanism added
\end{itemize}

%%%%%%%%%%%%%%%%%%%%%%%%%%%%%%%%%%%%%%%%
\paragraph{v0.5:} 2017/04/26

\begin{itemize}
\item
functionality in definition file
\end{itemize}


%%%%%%%%%%%%%%%%%%%%%%%%%%%%%%%%%%%%%%%%%%%%%%%%%%%%%%%%%%%%%%%%%%%%%%%%%%%%%%%%
%%%%%%%%%%%%%%%%%%%%%%%%%%%%%%%%%%%%%%%%%%%%%%%%%%%%%%%%%%%%%%%%%%%%%%%%%%%%%%%%
%%%%%%%%%%%%%%%%%%%%%%%%%%%%%%%%%%%%%%%%%%%%%%%%%%%%%%%%%%%%%%%%%%%%%%%%%%%%%%%%
\appendix

\settowidth\MacroIndent{\rmfamily\scriptsize 000\ }

 \DocInput{childdoc.dtx}

\end{document}
%</driver>
% \fi
%
% %%%%%%%%%%%%%%%%%%%%%%%%%%%%%%%%%%%%%%%%%%%%%%%%%%%%%%%%%%%%%%%%%%%%%%%%%%%%%%
% %%%%%%%%%%%%%%%%%%%%%%%%%%%%%%%%%%%%%%%%%%%%%%%%%%%%%%%%%%%%%%%%%%%%%%%%%%%%%%
% \section{Sample}
%\iffalse
%<*samplemain>
%\fi
%
% The following presents a sample document
% with two chapters, two parts, a title page,
% a compile flag as well as three forwarding files to set the flag.
% It consists of eight |.tex| files:
% \begin{center}
% \begin{tabular}{ll}
% |cdocsamp.tex|&main file\\
% |cdocsch1.tex|&include file for chapter 1\\
% |cdocsch2.tex|&include file for chapter 2\\
% |cdocspt3.tex|&include file for part 3\\
% |cdocspt4.tex|&include file for part 4\\
% |cdocsdrf.tex|&forwarding file for main file in draft mode\\
% |cdocsfi1.tex|&forwarding file for final version of chapter 1\\
% |cdocsfi2.tex|&forwarding file for final version of chapter 2\\
% \end{tabular}
% \end{center}
% Each of the eight files can be compiled directly by the \LaTeX{} compiler.
%
% %%%%%%%%%%%%%%%%%%%%%%%%%%%%%%%%%%%%%%
% \paragraph{Main File.}
%
% The main file is called |cdocsamp.tex|.
%
% Load the \textsf{childdoc} definitions and
% declare the filename for the main document:
%    \begin{macrocode}
% \iffalse
%
% childdoc.dtx Copyright (C) 2017-2018 Niklas Beisert
%
% This work may be distributed and/or modified under the
% conditions of the LaTeX Project Public License, either version 1.3
% of this license or (at your option) any later version.
% The latest version of this license is in
%   http://www.latex-project.org/lppl.txt
% and version 1.3 or later is part of all distributions of LaTeX
% version 2005/12/01 or later.
%
% This work has the LPPL maintenance status `maintained'.
%
% The Current Maintainer of this work is Niklas Beisert.
%
% This work consists of the files childdoc.dtx and childdoc.ins
% and the derived files childdoc.def and cdocsamp.tex with
% cdocsch1.tex, cdocsch2.tex, cdocsdrf.tex, cdocsfn1.tex, cdocsfn2.tex.
%
%<package>\ifdefined\childdocmain\endinput\fi
%<package>\ProvidesFile{childdoc.def}[2018/12/30 v2.0 child document driver]
%<samplemain>\ProvidesFile{cdocsamp.tex}[2018/12/30 v2.0 sample for childdoc]
%<*driver>
%\ProvidesFile{childdoc.drv}[2018/12/30 v2.0 childdoc reference manual file]
\PassOptionsToClass{10pt,a4paper}{article}
\documentclass{ltxdoc}

\usepackage[margin=35mm]{geometry}
\usepackage{hyperref}
\usepackage{hyperxmp}
\usepackage[usenames]{color}

\hypersetup{colorlinks=true}
\hypersetup{pdfstartview=FitH}
\hypersetup{pdfpagemode=UseNone}
\hypersetup{pdfsource={}}
\hypersetup{pdflang={en-UK}}
\hypersetup{pdfcopyright={Copyright 2017-2018 Niklas Beisert.
  This work may be distributed and/or modified under the
  conditions of the LaTeX Project Public License, either version 1.3
  of this license or (at your option) any later version.}}
\hypersetup{pdflicenseurl={http://www.latex-project.org/lppl.txt}}
\hypersetup{pdfcontactaddress={ETH Zurich, ITP, HIT K,
  Wolfgang-Pauli-Strasse 27}}
\hypersetup{pdfcontactpostcode={8093}}
\hypersetup{pdfcontactcity={Zurich}}
\hypersetup{pdfcontactcountry={Switzerland}}
\hypersetup{pdfcontactemail={nbeisert@itp.phys.ethz.ch}}
\hypersetup{pdfcontacturl={http://people.phys.ethz.ch/\xmptilde nbeisert/}}

\newcommand{\secref}[1]{\hyperref[#1]{section \ref*{#1}}}

\parskip1ex
\parindent0pt
\let\olditemize\itemize
\def\itemize{\olditemize\parskip0pt}

\begin{document}

\title{The \textsf{childdoc} Package}
\hypersetup{pdftitle={The childdoc Package}}
\author{Niklas Beisert\\[2ex]
  Institut f\"ur Theoretische Physik\\
  Eidgen\"ossische Technische Hochschule Z\"urich\\
  Wolfgang-Pauli-Strasse 27, 8093 Z\"urich, Switzerland\\[1ex]
  \href{mailto:nbeisert@itp.phys.ethz.ch}
  {\texttt{nbeisert@itp.phys.ethz.ch}}}
\hypersetup{pdfauthor={Niklas Beisert}}
\hypersetup{pdfsubject={Manual for the LaTeX2e Package childdoc}}
\date{30 December 2018, \textsf{v2.0}}
\maketitle

\begin{abstract}\noindent
\textsf{childdoc} is a \LaTeXe{} package
that enables the direct compilation
of document sections included by |\include|
to individual files.
\end{abstract}

\begingroup
\parskip0ex
\tableofcontents
\endgroup

%%%%%%%%%%%%%%%%%%%%%%%%%%%%%%%%%%%%%%%%%%%%%%%%%%%%%%%%%%%%%%%%%%%%%%%%%%%%%%%%
%%%%%%%%%%%%%%%%%%%%%%%%%%%%%%%%%%%%%%%%%%%%%%%%%%%%%%%%%%%%%%%%%%%%%%%%%%%%%%%%
\section{Introduction}

\LaTeX{} provides a mechanism to structure a large document (such as a book)
into a main file and several child files (containing the chapters)
using the |\include| command.
This mechanism is beneficial for documents
which span hundreds of pages in order to
make the source file(s) more manageable.
Moreover, compilation can be restricted to
selected child files by means of the |\includeonly| command.
The latter feature can be used to reduce the compilation time while editing
(this was significantly more useful in the earlier days of \LaTeX{})
or to generate a smaller document which is easier to navigate.
Another application of |\includeonly| is to generate
documents consisting of selected parts of the complete document.

However, there are a few drawbacks of the plain |\include| mechanism:
\begin{itemize}
\item
The child files cannot be compiled on their own,
they can only be compiled via the main file.
A naive editing environment
(such as a text editor with an option
to have the current file processed by \LaTeX)
may require one to switch to the main file before compiling;
attempting to compile the child file produces errors.
\item
The main file must be modified (each time)
to adjust the |\includeonly| command
to the present needs. This easily leaves the main file in a messy state.
\item
The generated document will always carry the filename
of the main document. This is inconvenient if
several child files are to be compiled and
to be kept for distribution.
\end{itemize}

The present package provides a simple interface
to make child files individually compilable by \LaTeX{}.
Compiling a child file then has the same effect as compiling
the main file with an |\includeonly| command
to select the appropriate child.
Moreover the generated document will carry the name of the child
rather than the main file.
This resolves all three above issues.

This feature is meant to make the editing of books,
thesis documents and lecture notes somewhat more convenient.
However, the package can also be used efficiently for
composing a series of documents (such as exercise sheets)
which are typically distributed individually.
It then assists the author in generating the individual documents
(potentially in different versions)
as well as a document containing the collected series.
Another application is in developing style files
or other kinds of included material
where compilation of the style file could redirect
to a sample or test file.

%%%%%%%%%%%%%%%%%%%%%%%%%%%%%%%%%%%%%%%%%%%%%%%%%%%%%%%%%%%%%%%%%%%%%%%%%%%%%%%%
%%%%%%%%%%%%%%%%%%%%%%%%%%%%%%%%%%%%%%%%%%%%%%%%%%%%%%%%%%%%%%%%%%%%%%%%%%%%%%%%
\section{Usage}

First of all, the package \textsf{childdoc} is \emph{not} a standard
\LaTeXe{} |.sty| style file! Therefore it needs to be invoked in
a non-standard way.

%%%%%%%%%%%%%%%%%%%%%%%%%%%%%%%%%%%%%%%%%%%%%%%%%%%%%%%%%%%%%%%%%%%%%%%%%%%%%%%%
\subsection{Included Files}
\label{sec:include}

%%%%%%%%%%%%%%%%%%%%%%%%%%%%%%%%%%%%%%%%
\DescribeMacro{\childdocmain}
To use the package, add the commands
\begin{center}
\begin{tabular}{l}
|\input{childdoc.def}|\\
|\childdocmain{}|\\
\end{tabular}
\end{center}
at the very top of the main \LaTeX{} file,
in particular \emph{before} the |\documentclass| statement!
The argument of |\childdocmain| should be left empty
(but it must be present).

%%%%%%%%%%%%%%%%%%%%%%%%%%%%%%%%%%%%%%%%
\DescribeMacro{\childdocof}
Furthermore, add the commands
\begin{center}
\begin{tabular}{l}
|\input{childdoc.def}|\\
|\childdocof{|\textit{main}|}|\\
\end{tabular}
\end{center}
at the top of every child file \textit{child}
which is included by |\include{|\textit{child}|}|
from within the main file
(or at least for those files to be compiled individually).
The argument \textit{main} must be the filename of the main file.

There are a couple of
considerations in setting up the main and child documents:

%%%%%%%%%%%%%%%%%%%%%%%%%%%%%%%%%%%%%%%%
\paragraph{Restrictions.}

Please note the following restrictions:
\begin{itemize}
\item
|\childdocmain| must be called with one argument \textit{main}
to ensure compatibility with earlier version of the package.
It must either be empty (|\childdocmain{}|)
or precisely match the filename of the main file in which it is specified.
See \secref{sec:detection} for further information.
\item
The filename \textit{main} must be specified without the |.tex| extension.
\item
The filename \textit{main} is case sensitive
(even in case-insensitive file systems)
due to internal string comparison.
\item
The argument \textit{main} should be fully expanded, it cannot be a macro.
\item
Subdirectories and special characters should be avoided in filenames.
\item
The command |\childdocmain{|\textit{main}|}| must be followed by a whitespace.
It should not be followed immediately by another command
or by a comment mark `|%|'.
This is because the \TeX{} parser reads the token immediately following
the argument of |\childdocmain| and puts it
at the beginning of every child section;
however, a white\-space is ignored.
\end{itemize}

%%%%%%%%%%%%%%%%%%%%%%%%%%%%%%%%%%%%%%%%
\paragraph{Content of Main File.}

It is advisable to place all content in the child files included by |\include|.
Any output contained in the main file will appear in all child documents
unless suppressed manually;
it cannot be suppressed automatically by the |\includeonly| directive
and thus should normally be avoided.
A method to include some content in the main file
by means of conditional processing is described in \secref{sec:conditional}.

%%%%%%%%%%%%%%%%%%%%%%%%%%%%%%%%%%%%%%%%
\paragraph{Page Numbering.}

When only a part of the document is compiled,
the appropriate numbering of pages
(as well as other status parameters)
is determined from the |.aux| files.
The latter contain information from previous passes.
However this information needs to propagate through
all intermediate child documents.
Therefore the page numbering in child documents may well
be inconsistent until the complete document is compiled at least once.

A useful (if unconventional) way to always ensure a consistent
page numbering is to restart the numbering in each child document
and denote the pages by `\textit{child}|.|\textit{page}'
where \textit{child} represents the chapter/section number of the child file.
This can be achieved by the command
|\numberwithin{page}{|\textit{child}|}|
of the \textsf{amsmath} package
where \textit{child} can be |chapter| or |section|
depending on the chosen structuring.
Alternatively, one can modify the macro |\thepage| appropriately
and reset the counter |page| at the start of each child file.

%%%%%%%%%%%%%%%%%%%%%%%%%%%%%%%%%%%%%%%%%%%%%%%%%%%%%%%%%%%%%%%%%%%%%%%%%%%%%%%%
\subsection{Conditional Processing}
\label{sec:conditional}

The package provides a mechanism to compile different versions
of a document. To customise the versions further some conditional processing
can come in handy to distinguish which version is being compiled.
The package provides two macros to describe the compilation context:

%%%%%%%%%%%%%%%%%%%%%%%%%%%%%%%%%%%%%%%%
\DescribeMacro{\ifchilddoc}
The conditional |\ifchilddoc| distinguishes between the compilation of
child documents and the main document:
%
\begin{center}
|\ifchilddoc |\textit{child-code}| |[|\||else |\textit{main-code}]| \||fi|
\end{center}

%%%%%%%%%%%%%%%%%%%%%%%%%%%%%%%%%%%%%%%%
\DescribeMacro{\childdocname}
\DescribeMacro{\childdocjob}
The macro |\childdocname| contains the filename (without extension)
of the main or child file being processed.
Note that |\childdocjob| will always contain the name of the main file.

%%%%%%%%%%%%%%%%%%%%%%%%%%%%%%%%%%%%%%%%
\paragraph{Title Page.}

Conditional processing can be used to include a title or banner page
in the main document when proper precautions are taken.
Importantly, the code in the main file should ensure that the page counter
(as well as other status parameters which are stored in the |.aux| files)
takes the same value after the conditional processing.
Otherwise the page numbers may take divergent values
depending on which part is compiled.

For example, a title page could be declared by:
%
\begin{center}
\begin{tabular}{l}
|\ifchilddoc\||else|\\
|\addtocounter{page}{-1}|\\
\textit{code for title page}\\
|\newpage|\\
|\||fi|
\end{tabular}
\end{center}
%
A banner page for the child documents can be generated by:
%
\begin{center}
\begin{tabular}{l}
|\ifchilddoc|\\
|\addtocounter{page}{-1}|\\
\textit{code for banner page}\\
|\newpage|\\
|\||fi|
\end{tabular}
\end{center}
%
Here one could write a message such as:
\begin{center}
|This is the part \childdocname{} of \childdocjob{}.|
\end{center}

%%%%%%%%%%%%%%%%%%%%%%%%%%%%%%%%%%%%%%%%%%%%%%%%%%%%%%%%%%%%%%%%%%%%%%%%%%%%%%%%
\subsection{Flags}
\label{sec:flags}

The package makes it easy to generate different versions
of the main or child documents.
To this end compilation flags can be defined
and assigned different default values.
They will be particularly useful in conjunction
with the forwarding mechanism described in \secref{sec:forward}.

For example, it may be useful to have a flag |\version|
which can be set to |draft| or |final|.
The document source will contain some conditional code
depending on the value of |\version|.
Suppose further, the flag should default to |final| for the main file
and to |draft| for child files
which is a natural assignment for editing the document.
This is achieved by placing the following code
in the preamble of the main document
(below the |\childdocmain| directive):
%
\begin{center}
\begin{tabular}{l}
|\ifchilddoc|\\
|\providecommand{\version}{draft}|\\
|\||else|\\
|\providecommand{\version}{final}|\\
|\||fi|
\end{tabular}
\end{center}
%
The definition by |\providecommand| makes sure
that previous definitions are not overwritten.
Further statements |\providecommand{\version}{...}|
can thus be added before the above code to override it.

For the main file, one might add a line
(between |\childdocmain| and the above block)
%
\begin{center}
|%\ifchilddoc\||else\providecommand{\version}{draft}\||fi|
\end{center}
%
which can be uncommented to produce a draft version.
Likewise one can add a line to the very top of a child file
(above the |\childdocof{|\textit{main}|}| directive)
%
\begin{center}
|%\providecommand{\version}{final}|
\end{center}
%
which can be uncommented to produce the final version of this child document.

%%%%%%%%%%%%%%%%%%%%%%%%%%%%%%%%%%%%%%%%%%%%%%%%%%%%%%%%%%%%%%%%%%%%%%%%%%%%%%%%
\subsection{Forwarding}
\label{sec:forward}

Different versions of the main or child documents
using compilation flags as described in \secref{sec:flags}
can be (permanently) stored in different files
for convenient compilation, viewing and distribution.
To this end, the package defines a command
to pass on compilation to a different file:

%%%%%%%%%%%%%%%%%%%%%%%%%%%%%%%%%%%%%%%%
\DescribeMacro{\childdocforward}
The command |\childdocforward| redirects processing to
another source file:
%
\begin{center}
\begin{tabular}{l}
|\input{childdoc.def}|\\
|\childdocforward[|\textit{main}|]{|\textit{dest}|}|\\
\end{tabular}
\end{center}
%
The argument \textit{dest} is the destination file
(without extension).
It should be the main file or one of the child files.
Note that further \textsf{childdoc} directives
such as |\childdocof| and |\childdocforward|
in the indicated file will be processed in this form.
The optional argument \textit{main}
passes on directly to the main file \textit{main}
while pretending to compile the child \textit{dest}.
This form behaves as if \textit{dest}
issues |\childdocof{|\textit{main}|}| right away,
and no further \textsf{childdoc} directives will be processed.

%%%%%%%%%%%%%%%%%%%%%%%%%%%%%%%%%%%%%%%%
\DescribeMacro{\...prefix}
In the alternative form |\childdocforwardprefix|,
%
\begin{center}
\begin{tabular}{l}
|\input{childdoc.def}|\\
|\childdocforwardprefix[|\textit{main}|]{|\textit{prefix}|}{|\textit{dest}|}|
\end{tabular}
\end{center}
%
the destination file is determined by a pattern
depending on the current file:
To make this work, the current file must be called
`{\textit{prefix}\hspace{0.2em}\textit{suffix}}'
with \textit{prefix} matching precisely the argument.
Processing is then passed on to the file
`{\textit{dest}\hspace{0.2em}\textit{suffix}}'.
Surely, the same effect is achieved by
directly specifying the
argument `{\textit{dest}\hspace{0.2em}\textit{suffix}}'
in the first form.
However, that requires to set up a different file
for each child. With the alternative form of the command
all these files can have exactly the same content
which simplifies setting them up and maintaining them.

For example, the following file |draft.tex|
with a compilation flag |\version| as described in \secref{sec:flags}
compiles the main document as a draft:
%
\begin{center}
\begin{tabular}{l}
|\def\version{draft}|\\
|\input{childdoc.def}|\\
|\childdocforward{|\textit{main}|}|
\end{tabular}
\end{center}
%
Likewise, the following files |final|\textit{nn}|.tex|
compile the final version of the child document
|child|\textit{nn}|.tex|:
%
\begin{center}
\begin{tabular}{l}
|\def\version{final}|\\
|\input{childdoc.def}|\\
|\childdocforwardprefix{final}{child}|
\end{tabular}
\end{center}
%

Note that when several versions of a main file and/or of each child file
are to be generated, it may be convenient to set up a |Makefile| or
shell script to automatise the process.

%%%%%%%%%%%%%%%%%%%%%%%%%%%%%%%%%%%%%%%%%%%%%%%%%%%%%%%%%%%%%%%%%%%%%%%%%%%%%%%%
\subsection{Command Line Processing}
\label{sec:commandline}

The effect of redirection files can also be achieved by invoking
the \LaTeX{} compiler with a more elaborate command line.
Most conveniently this should be done as part
of a shell script or a |Makefile|.

When using \textsf{childdoc} in the main file, the following
command lines effectively perform a redirection
(note that depending on the shell being used,
backslashes may have to be doubled: `|\|' $\to$ `|\\|'):
%
\begin{center}
|... -jobname "|\textit{target}|" |\\|"|[\textit{flags}]%
|\input{childdoc.def}\childdocforward[|\textit{main}|]{|\textit{dest}|}"|
\end{center}
%
Here \textit{target} is the name of the output file,
\textit{main} is the name of the main file
and \textit{dest} is the name of the main or child file to be processed
(all filenames without extensions).
The optional argument \textit{main} can be omitted
if \textit{main} matches \textit{dest}.
Optionally, compilation \textit{flags} can be defined via |\def| commands.
This command line makes the \TeX{} engine believe
it is compiling the file \textit{target}
whose content is specified as the latter parameter.
The provided code then forwards the processing to
\textit{main} or \textit{dest} as described in \secref{sec:forward}.

%%%%%%%%%%%%%%%%%%%%%%%%%%%%%%%%%%%%%%%%%%%%%%%%%%%%%%%%%%%%%%%%%%%%%%%%%%%%%%%%
\subsection{Include by Input}
\label{sec:input}

Including child documents by |\include| has some restrictions by design.
Most notably, the content of a child document always occupies
its own set of pages; pages cannot be shared between child documents.
Usually, this behaviour makes perfect sense
because each child document contain an essential part of the document.
However, in some situations it may be desirable to compose
a document from a collection of parts
without having mandatory page breaks between then.
For this case, the package
provides a mechanism to include parts
by |\input| which can also be processed individually.
However, by construction this mechanism
requires manual handling of the content to be output.

%%%%%%%%%%%%%%%%%%%%%%%%%%%%%%%%%%%%%%%%
\DescribeMacro{\ifchilddocmanual}
The main file should be prepared as usual, see \secref{sec:include}.
However, the document body must make a distinction
between processing of an individual part and of the main document, e.g.:
%
\begin{center}
\begin{tabular}{l}
|\ifchilddocmanual|\\
|\input{\childdocname}|\\
|\||else|\\
\textit{document body with }|\input{|\textit{part}|}|\\
|\||fi|
\end{tabular}
\end{center}
%
The conditional |\ifchilddocmanual| is true whenever
a part to be included by |\input| is being compiled,
and the name of the part is stored in |\childdocname|.

%%%%%%%%%%%%%%%%%%%%%%%%%%%%%%%%%%%%%%%%
\DescribeMacro{\childdocby}
Each part to be included by |\input| should start with:
%
\begin{center}
\begin{tabular}{l}
|\input{childdoc.def}|\\
|\childdocby{|\textit{main}|}|\\
\end{tabular}
\end{center}
%
The directive |\childdocby| is similar to |\childdocof|
described in \secref{sec:include},
but the subsequent selection of content must be done manually.
To that end, both |\ifchilddoc| and |\ifchilddocmanual|
will be true upon processing of a part,
and the name of the part is stored in |\childdocname|.
Note that |\jobname| will be set to the filename of the current part
so that each part receives an individual |.aux| file
that does not interfere with the |.aux| file(s) of the main document.
This behaviour can be altered by the alternative form
|\childdocby[*]{|\textit{main}|}| (with a non-empty optional argument)
which uses the |.aux| file of the main document
by setting |\jobname| to \textit{main}.

%%%%%%%%%%%%%%%%%%%%%%%%%%%%%%%%%%%%%%%%%%%%%%%%%%%%%%%%%%%%%%%%%%%%%%%%%%%%%%%%
\subsection{Driver Development}
\label{sec:driver}

The \textsf{childdoc} mechanism can also be use for the development
of definition files such as \LaTeX{} styles or classes.
This case differs from the above setup with multiple parts
included by |\include| in that no |\includeonly| should be invoked.
This can be achieved by starting the include file
(before |\ProvidesPackage|) with:
%
\begin{center}
\begin{tabular}{l}
|\input{childdoc.def}|\\
|\childdocforward{|\textit{main}|}|\\
\end{tabular}
\end{center}
%
or alternatively with:
%
\begin{center}
\begin{tabular}{l}
|\input{childdoc.def}|\\
|\childdocby{|\textit{main}|}|\\
\end{tabular}
\end{center}
%
Both forms have slightly different effects as described above.
The main file is prepared as usual, see \secref{sec:include}.

%%%%%%%%%%%%%%%%%%%%%%%%%%%%%%%%%%%%%%%%%%%%%%%%%%%%%%%%%%%%%%%%%%%%%%%%%%%%%%%%
\subsection{Legacy Detection}
\label{sec:detection}

The directive |\childdocmain| in the main file can detect
whether the complete document or merely a child is to be compiled
even without using the directive |\childdocof|.
This method is deprecated because it is less robust
and there is no compelling reason to use it;
it is merely provided for backward compatibility
and it may be removed in future versions.

If the detection mechanism is to be used,
it is mandatory to correctly specify
the filename of the main file as the argument of |\childdocmain|:
%
\begin{center}
\begin{tabular}{l}
|\input{childdoc.def}|\\
|\childdocmain{|\textit{main}|}|\\
\end{tabular}
\end{center}
%
If |\jobname| does not match the argument \textit{main} of |\childdocmain|,
it is assumed that |\jobname| points to the child file to be compiled.
When using |\childdocmain| with the main file specified as argument,
it suffices to start a child file
with just |\input{|\textit{main}|}|
without loading of the package and using |\childdocof|.
If instead all processing is done
with the appropriate \textsf{childdoc} directives,
the argument of \textit{main} of |\childdocmain| can be empty.

An alternative version of the command line processing described
in \secref{sec:commandline} using the detection mechanism reads:
%
\begin{center}
|... -jobname "|\textit{target}|" "|[\textit{flags}]%
[|\def\jobname{|\textit{dest}|}|]|\input{|\textit{main}|}"|
\end{center}

%%%%%%%%%%%%%%%%%%%%%%%%%%%%%%%%%%%%%%%%%%%%%%%%%%%%%%%%%%%%%%%%%%%%%%%%%%%%%%%%
\subsection{Manual Code}
\label{sec:manual}

In case one cannot be certain whether the definitions file |childdoc.def|
is installed on the target \TeX{} distribution
and one prefers not to ship it,
it is conceivable to paste a few relevant commands into the sources.

To that end, drop all statements |\input{childdoc.def}|
and perform the replacements as outlined below.
Instead of |\childdocmain{|\textit{main}|}| add the following code
to the top of the main file:
%
\begin{center}
\begin{tabular}{l}
|\||ifdefined\childdocname\endinput\||fi\newif\ifchilddoc|\\
|\edef\childdocname{\scantokens\expandafter{\jobname\noexpand}}|\\
|\def\childdocmain{|\textit{main}|}\||ifx\childdocmain\childdocname\||else|\\
|\childdoctrue\includeonly{\childdocname}\let\jobname\childdocmain\||fi|\\
\end{tabular}
\end{center}
%
Instead of |\childdocof{|\textit{main}|}| just include the main file
at the top of each child file:
%
\begin{center}
|\input{|\textit{main}|}|
\end{center}
%
A simple redirection |\childdocforward{|\textit{dest}|}| is achieved by:
%
\begin{center}
|\def\jobname{|\textit{dest}|}\input{\jobname}|
\end{center}
%
The redirection with prefix
|\childdocforwardprefix[|\textit{prefix}|]{|\textit{dest}|}|
is accomplished by:
%
\begin{center}
\begin{tabular}{l}
|{\edef\jobname{\scantokens\expandafter{\jobname\noexpand}}|\\
|\def\redirectjob |\textit{prefix}|#1~~~{\gdef\jobname{|\textit{dest}|#1}}|\\
|\expandafter\redirectjob\jobname~~~}\input{\jobname}|
\end{tabular}
\end{center}

In an alternative approach,
child documents can be compiled by a specific command line
without additional code or specific definitions:
%
\begin{center}
|... -jobname "|\textit{target}|" "|[\textit{flags}]%
|\includeonly{|\textit{dest}|}\input{|\textit{main}|}"|
\end{center}
%

%%%%%%%%%%%%%%%%%%%%%%%%%%%%%%%%%%%%%%%%%%%%%%%%%%%%%%%%%%%%%%%%%%%%%%%%%%%%%%%%
%%%%%%%%%%%%%%%%%%%%%%%%%%%%%%%%%%%%%%%%%%%%%%%%%%%%%%%%%%%%%%%%%%%%%%%%%%%%%%%%
\section{Information}

%%%%%%%%%%%%%%%%%%%%%%%%%%%%%%%%%%%%%%%%%%%%%%%%%%%%%%%%%%%%%%%%%%%%%%%%%%%%%%%%
\subsection{Copyright}

Copyright \copyright{} 2017--2018 Niklas Beisert

This work may be distributed and/or modified under the
conditions of the \LaTeX{} Project Public License, either version 1.3
of this license or (at your option) any later version.
The latest version of this license is in
  \url{http://www.latex-project.org/lppl.txt}
and version 1.3 or later is part of all distributions of \LaTeX{}
version 2005/12/01 or later.

This work has the LPPL maintenance status `maintained'.

The Current Maintainer of this work is Niklas Beisert.

This work consists of the files |README.txt|, |childdoc.ins| and |childdoc.dtx|
as well as the derived files |childdoc.def|, |cdocsamp.tex|
with |cdocsch1.tex|, |cdocsch2.tex|, |cdocspt3.tex|, |cdocspt4.tex|,
|cdocsdrf.tex|, |cdocsfn1.tex|, |cdocsfn2.tex|
as well as |childdoc.pdf|.

%%%%%%%%%%%%%%%%%%%%%%%%%%%%%%%%%%%%%%%%%%%%%%%%%%%%%%%%%%%%%%%%%%%%%%%%%%%%%%%%
\subsection{Files and Installation}

The package consists of the files:
%
\begin{center}
\begin{tabular}{ll}
    |README.txt|   & readme file \\
    |childdoc.ins| & installation file \\
    |childdoc.dtx| & source file \\
    |childdoc.def| & definition file \\
    |cdocsamp.tex| & sample main file \\
    |cdocsch1.tex| & sample include file \\
    |cdocsch2.tex| & sample include file \\
    |cdocspt3.tex| & sample part file \\
    |cdocspt4.tex| & sample part file \\
    |cdocsdrf.tex| & sample redirection file \\
    |cdocsfn1.tex| & sample redirection file \\
    |cdocsfn2.tex| & sample redirection file \\
    |childdoc.pdf| & manual
\end{tabular}
\end{center}
%
The distribution consists of the files
|README.txt|, |childdoc.ins| and |childdoc.dtx|.
%
\begin{itemize}
\item
Run (pdf)\LaTeX{} on |childdoc.dtx|
to compile the manual |childdoc.pdf| (this file).
\item
Run \LaTeX{} on |childdoc.ins| to create the definitions file |childdoc.def|
and the sample |cdocsamp.tex| with include files
|cdocsch1.tex|, |cdocsch2.tex|, |cdocspt3.tex|, |cdocspt4.tex|,
|cdocsdrf.tex|, |cdocsfn1.tex|, |cdocsfn2.tex|.
Then copy the file |childdoc.def| to an appropriate directory of your \LaTeX{}
distribution, e.g.\ \textit{texmf-root}|/tex/latex/childdoc|.
\end{itemize}

%%%%%%%%%%%%%%%%%%%%%%%%%%%%%%%%%%%%%%%%%%%%%%%%%%%%%%%%%%%%%%%%%%%%%%%%%%%%%%%%
\subsection{Related CTAN Packages}

There are several other packages which offer a similar functionality:
%
\begin{itemize}
\item
The packages
\href{http://ctan.org/pkg/docmute}{\textsf{docmute}},
\href{http://ctan.org/pkg/includex}{\textsf{includex}} and
\href{http://ctan.org/pkg/standalone}{\textsf{standalone}}
provide commands to include only the document body of
a child file thus allowing both files to be compiled individually.
\item
The packages \href{http://ctan.org/pkg/subdocs}{\textsf{subdocs}}
and \href{http://ctan.org/pkg/subfiles}{\textsf{subfiles}}
provide structures in which the main and child documents can be
encapsulated and allowing them to be compiled individually.
The inclusion mechanism is different from the conventional |\include|.
\item
The package \href{http://ctan.org/pkg/combine}{\textsf{combine}}
is an elaborate solution to combine several documents into one.
\end{itemize}
%
See also the CTAN topic \href{http://ctan.org/topic/subdocs}{\textsf{subdocs}}
for further related packages.
The present package differs from the above solutions in that
a document structure constructed with the conventional |\include| mechanism
just needs two extra commands at the top of every file
such that all constituent files can be compiled individually.

%%%%%%%%%%%%%%%%%%%%%%%%%%%%%%%%%%%%%%%%%%%%%%%%%%%%%%%%%%%%%%%%%%%%%%%%%%%%%%%%
%\subsection{Feature Suggestions}
%
%The following is a list of features which may be useful for future
%versions of this package:
%%
%\begin{itemize}
%\item
%\ldots
%\end{itemize}

%%%%%%%%%%%%%%%%%%%%%%%%%%%%%%%%%%%%%%%%%%%%%%%%%%%%%%%%%%%%%%%%%%%%%%%%%%%%%%%%
\subsection{Revision History}

%%%%%%%%%%%%%%%%%%%%%%%%%%%%%%%%%%%%%%%%
\paragraph{v2.0:} 2018/12/30

\begin{itemize}
\item
immediate forward processing
\item
added |\childdocby| mechanism
\item
manual restructured
\end{itemize}

%%%%%%%%%%%%%%%%%%%%%%%%%%%%%%%%%%%%%%%%
\paragraph{v1.6:} 2018/01/17

\begin{itemize}
\item
application for development of include files
\item
corrections to manual
\end{itemize}

%%%%%%%%%%%%%%%%%%%%%%%%%%%%%%%%%%%%%%%%
\paragraph{v1.5:} 2017/05/21

\begin{itemize}
\item
more complete structuring introduced
\item
|\childdocof| introduced
\item
|\childdoc| renamed to |\childdocmain|
\item
|\childredirect| renamed to |\childdocforward| and |\childdocforwardprefix|
and functionality expanded
\end{itemize}

%%%%%%%%%%%%%%%%%%%%%%%%%%%%%%%%%%%%%%%%
\paragraph{v1.0:} 2017/04/27

\begin{itemize}
\item
manual and install package
\item
first version published on CTAN
\end{itemize}

%%%%%%%%%%%%%%%%%%%%%%%%%%%%%%%%%%%%%%%%
\paragraph{v0.6:} 2017/04/26

\begin{itemize}
\item
redirection mechanism added
\end{itemize}

%%%%%%%%%%%%%%%%%%%%%%%%%%%%%%%%%%%%%%%%
\paragraph{v0.5:} 2017/04/26

\begin{itemize}
\item
functionality in definition file
\end{itemize}


%%%%%%%%%%%%%%%%%%%%%%%%%%%%%%%%%%%%%%%%%%%%%%%%%%%%%%%%%%%%%%%%%%%%%%%%%%%%%%%%
%%%%%%%%%%%%%%%%%%%%%%%%%%%%%%%%%%%%%%%%%%%%%%%%%%%%%%%%%%%%%%%%%%%%%%%%%%%%%%%%
%%%%%%%%%%%%%%%%%%%%%%%%%%%%%%%%%%%%%%%%%%%%%%%%%%%%%%%%%%%%%%%%%%%%%%%%%%%%%%%%
\appendix

\settowidth\MacroIndent{\rmfamily\scriptsize 000\ }

 \DocInput{childdoc.dtx}

\end{document}
%</driver>
% \fi
%
% %%%%%%%%%%%%%%%%%%%%%%%%%%%%%%%%%%%%%%%%%%%%%%%%%%%%%%%%%%%%%%%%%%%%%%%%%%%%%%
% %%%%%%%%%%%%%%%%%%%%%%%%%%%%%%%%%%%%%%%%%%%%%%%%%%%%%%%%%%%%%%%%%%%%%%%%%%%%%%
% \section{Sample}
%\iffalse
%<*samplemain>
%\fi
%
% The following presents a sample document
% with two chapters, two parts, a title page,
% a compile flag as well as three forwarding files to set the flag.
% It consists of eight |.tex| files:
% \begin{center}
% \begin{tabular}{ll}
% |cdocsamp.tex|&main file\\
% |cdocsch1.tex|&include file for chapter 1\\
% |cdocsch2.tex|&include file for chapter 2\\
% |cdocspt3.tex|&include file for part 3\\
% |cdocspt4.tex|&include file for part 4\\
% |cdocsdrf.tex|&forwarding file for main file in draft mode\\
% |cdocsfi1.tex|&forwarding file for final version of chapter 1\\
% |cdocsfi2.tex|&forwarding file for final version of chapter 2\\
% \end{tabular}
% \end{center}
% Each of the eight files can be compiled directly by the \LaTeX{} compiler.
%
% %%%%%%%%%%%%%%%%%%%%%%%%%%%%%%%%%%%%%%
% \paragraph{Main File.}
%
% The main file is called |cdocsamp.tex|.
%
% Load the \textsf{childdoc} definitions and
% declare the filename for the main document:
%    \begin{macrocode}
\input{childdoc.def}
\childdocmain{}
%    \end{macrocode}

% Optional override for |\version| flag:
%    \begin{macrocode}
%%\ifchilddoc\else\providecommand{\version}{draft}\fi
%    \end{macrocode}

% Define the default values for the |\version| flag
% (|final| for the main file and |draft| for childs):
%    \begin{macrocode}
\ifchilddoc
\providecommand{\version}{draft}
\else
\providecommand{\version}{final}
\fi
%    \end{macrocode}

% Load the standard document class:
%    \begin{macrocode}
\documentclass[12pt]{article}
%    \end{macrocode}

% Start the document body:
%    \begin{macrocode}
\begin{document}
%    \end{macrocode}

% Declare a title page.
% Print title, part of document being processed and version flag:
%    \begin{macrocode}
\addtocounter{page}{-1}
\begin{center}
{\LARGE\bfseries{}childdoc example\par}
\vspace{1cm}
\ifchilddoc
\ifchilddocmanual part\else chapter\fi:
`\childdocname' of `\childdocjob'\par
\else
main document: `\childdocjob'\par
\fi
version: \version\par
\end{center}
\newpage
%    \end{macrocode}

% Manually include selected file,
% otherwise process as usual:
%    \begin{macrocode}
\ifchilddocmanual
\section*{part `\childdocname'}
\input{\childdocname}
\else
%    \end{macrocode}

% Include the two chapters:
%    \begin{macrocode}
\include{cdocsch1}
\include{cdocsch2}
%    \end{macrocode}

% Include the two parts unless only chapters should be displayed:
%    \begin{macrocode}
\ifchilddoc\else
\section{part three}
\input{cdocspt3}
\section{part four}
\input{cdocspt4}
\fi
%    \end{macrocode}

% Process as usual until here:
%    \begin{macrocode}
\fi
%    \end{macrocode}

% End of document body:
%    \begin{macrocode}
\end{document}
%    \end{macrocode}
%\iffalse
%</samplemain>
%\fi
%
% %%%%%%%%%%%%%%%%%%%%%%%%%%%%%%%%%%%%%%
% \paragraph{Chapter Include Files.}
%
% The include files are called |cdocsch1.tex| and |cdocsch2.tex|.
%
%\iffalse
%<*samplechap1|samplechap2>
%\fi

% Optional override for |\version| flag:
%    \begin{macrocode}
%%\providecommand{\version}{final}
%    \end{macrocode}

% Include the main document:
%    \begin{macrocode}
\input{childdoc.def}
\childdocof{cdocsamp}
%    \end{macrocode}

%\iffalse
%</samplechap1|samplechap2>
%\fi
%
%\iffalse
%<*samplechap1>
%\fi
% Some text for chapter 1:
%    \begin{macrocode}
\section{one}
some text in chapter one
%    \end{macrocode}

%\iffalse
%</samplechap1>
%\fi
% Some text for chapter 2:
%\iffalse
%<*samplechap2>
%\fi
%    \begin{macrocode}
\section{two}
more text in chapter two
%    \end{macrocode}

%\iffalse
%</samplechap2>
%\fi
%
% %%%%%%%%%%%%%%%%%%%%%%%%%%%%%%%%%%%%%%
% \paragraph{Part Include Files.}
%
% The include files are called |cdocspt3.tex| and |cdocspt4.tex|.
%
%\iffalse
%<*samplepart3|samplepart4>
%\fi

% Optional override for |\version| flag:
%    \begin{macrocode}
%%\providecommand{\version}{final}
%    \end{macrocode}

% Include the main document:
%    \begin{macrocode}
\input{childdoc.def}
\childdocby{cdocsamp}
%    \end{macrocode}

%\iffalse
%</samplepart3|samplepart4>
%\fi
%
%\iffalse
%<*samplepart3>
%\fi
% Some text for part 3:
%    \begin{macrocode}
some text in part three
%    \end{macrocode}

%\iffalse
%</samplepart3>
%\fi
% Some text for part 4:
%\iffalse
%<*samplepart4>
%\fi
%    \begin{macrocode}
more text in part four
%    \end{macrocode}

%\iffalse
%</samplepart4>
%\fi
%
% %%%%%%%%%%%%%%%%%%%%%%%%%%%%%%%%%%%%%%
% \paragraph{Forwarding for a Complete Draft.}
%
% The following forwarding file |cdocsdrf.tex|
% compiles the main document in draft mode:
%\iffalse
%<*sampledraft>
%\fi
%    \begin{macrocode}
\def\version{draft}
\input{childdoc.def}
\childdocforward{cdocsamp}
%    \end{macrocode}

%\iffalse
%</sampledraft>
%\fi
%
% %%%%%%%%%%%%%%%%%%%%%%%%%%%%%%%%%%%%%%
% \paragraph{Forwarding for Final Version of the Chapters.}
%
% The following forwarding files |cdocsfn1.tex| and |cdocsfn2.tex|
% (with identical content)
% compile the final versions of the child documents
% |cdocsch1.tex| and |cdocsch2.tex|, respectively:
%\iffalse
%<*samplefinal>
%\fi
%    \begin{macrocode}
\def\version{final}
\input{childdoc.def}
\childdocforwardprefix[cdocsamp]{cdocsfn}{cdocsch}
%    \end{macrocode}

%\iffalse
%</samplefinal>
%\fi
%
% %%%%%%%%%%%%%%%%%%%%%%%%%%%%%%%%%%%%%%
% \paragraph{Command Line Processing.}
%
% The following three command lines generate the output files
% |cdocscld|, |cdocscl1| and |cdocscl2|
% which should be identical to
% |cdocsdrf|, |cdocsch1| and |cdocsfn2|, respectively:
% \begin{center}
% \begin{tabular}{l}
% |latex -jobname cdocscld \|\\
% |  "\def\version{draft}\input{childdoc.def}\childdocforward{cdocsamp}"|\\
% |latex -jobname cdocscl1 \|\\
% |  "\input{childdoc.def}\childdocforward[cdocsamp]{cdocsch1}"|\\
% |latex -jobname cdocscl2 \|\\
% |  "\def\version{final}\input{childdoc.def}\childdocforward{cdocsch2}"|
% \end{tabular}
% \end{center}
% Note that the trailing backslash on each first line
% merely continues the input to the second line
% (for convenient cut ant paste).
% Furthermore, the command |latex| can be replaced by any
% of its alternative versions such as |pdflatex|.
%
% %%%%%%%%%%%%%%%%%%%%%%%%%%%%%%%%%%%%%%%%%%%%%%%%%%%%%%%%%%%%%%%%%%%%%%%%%%%%%%
% %%%%%%%%%%%%%%%%%%%%%%%%%%%%%%%%%%%%%%%%%%%%%%%%%%%%%%%%%%%%%%%%%%%%%%%%%%%%%%
% \section{Implementation}
%\iffalse
%<*package>
%\fi
%
% This section describes the definitions file |childdoc.def|.

% The definitions cannot be loaded using |\usepackage| or |\RequirePackage|
% which has a mechanism to prevent loading a style file more than once.
% When loading the definitions by means of |\input|
% multiple instances have to be prevented manually:
%\iffalse
%This code needs to be before the `\ProvidesFile' directive
%which is defined at the beginning of this file.
%Therefore it is also placed there and commented out here.
%</package>
%<*discard>
%\fi
%    \begin{macrocode}
\ifdefined\childdocmain\endinput\fi
%    \end{macrocode}
%\iffalse
%</discard>
%<*package>
%\fi
%
% \macro{\ifchilddoc}
% \macro{\ifchilddocmanual}
% The conditional |\ifchilddoc| tells whether a
% child (true) or main (false) document is being compiled.
% The conditional |\ifchilddocmanual| tells whether
% the |\includeonly| mechanism is used (false) or
% the selection of child files must be performed manually (true).
% The definitions initialise to false:
%    \begin{macrocode}
\newif\ifchilddoc
\newif\ifchilddocmanual
%    \end{macrocode}

% \macro{\childdocname}
% \macro{\childdocjob}
% The macro |\childdocname| stores the name of the main document
% to be compiled. The macro |\childdocjob| stores the name of
% the document on which the \LaTeX{} compiler was originally invoked.
% The content of |\jobname| cannot be compared
% to filenames specified in the source due to different catcodes.
% The following code rescans |\jobname|, stores the result
% in |\childdocname| and saves a copy in |\childdocjob|:
%    \begin{macrocode}
\edef\childdocname{\scantokens\expandafter{\jobname\noexpand}}
\let\childdocjob\childdocname
%    \end{macrocode}

% \macro{\childdocdisable}
% The macro |\childdocdisable| prevents the main file
% from being processed more than once.
% At this stage, the main document command |\childdocmain|
% is assumed to be called once again where it should do nothing.
% Any subsequent call to it should prevent
% a secondary processing of the main document
% It overwrites the forwarding commands
% |\childdocof| and |\childdocforward|
% with empty macros to prevent further inclusions of the main document:
%    \begin{macrocode}
\newcommand{\childdocdisable}
{
  \renewcommand{\childdocmain}[1]{\renewcommand{\childdocmain}[1]{\endinput}}
  \renewcommand{\childdocof}[1]{}
  \renewcommand{\childdocby}[2][]{}
  \renewcommand{\childdocforward}[2][]{}
  \renewcommand{\childdocdisable}{}
}
%    \end{macrocode}

% \macro{\childdocmain}
% The macro |\childdocmain| is to be called at the top of the main file
% with nothing or the main filename (without extension) as argument.
% First, it breaks loops.
% If the argument is not empty and does not match |\childdocname|
% (which is set by the first inclusion of |childdoc.def|),
% |\ifchilddoc| is set to true, |\includeonly| is applied to the child file
% and |\jobname| is set to the main file
% (for proper handling of |.aux| files):
%    \begin{macrocode}
\newcommand{\childdocmain}[1]
{
  \childdocdisable\childdocmain{}
  \if?#1?\else
    \begingroup
      \def\childdoctmp{#1}
      \ifx\childdoctmp\childdocname
        \def\childdoctmp{}
      \else
        \def\childdoctmp
        {
          \childdoctrue
          \includeonly{\childdocname}
          \def\childdocjob{#1}
          \def\jobname{#1}
        }
      \fi
      \expandafter
    \endgroup
    \childdoctmp
  \fi
}
%    \end{macrocode}

% \macro{\childdocof}
% The command |\childdocof| redirects
% compilation to the main file |#1|.
%    \begin{macrocode}
\newcommand{\childdocof}[1]
{
  \childdocdisable
  \childdoctrue
  \includeonly{\childdocname}
  \def\jobname{#1}
  \def\childdocjob{#1}
  \input{#1}
}
%    \end{macrocode}

% \macro{\childdocby}
% The command |\childdocby| ....
%    \begin{macrocode}
\newcommand{\childdocby}[2][]
{
  \childdocdisable
  \childdoctrue
  \childdocmanualtrue
  \if?#1?\else
    \def\jobname{#2}
  \fi
  \def\childdocjob{#2}
  \input{#2}
  \endinput
}
%    \end{macrocode}

% \macro{\childdocforward}
% The command |\childdocforward| redirects
% compilation to the main file or
% (if the optional argument is given) a child file.
% Parameters are set as if the main file
% or a child file starting with |\childdocof| was compiled.
% Then compilation is handed over to the main file:
%    \begin{macrocode}
\newcommand{\childdocforward}[2][]
{
  \begingroup
    \if?#1?
      \def\childdoctmp
      {
        \def\childdocname{#2}
        \def\childdocjob{#2}
        \def\jobname{#2}
        \input{#2}
        \endinput
      }
    \else
      \def\childdoctmp
      {
        \childdocdisable
        \def\childdocname{#2}
        \childdoctrue
        \includeonly{#2}
        \def\childdocjob{#1}
        \def\jobname{#1}
        \input{#1}
        \endinput
      }
    \fi
    \expandafter
  \endgroup
  \childdoctmp
}
%    \end{macrocode}

% \macro{\childdocforwardprefix}
% The command |\childdocforwardprefix| redirects
% compilation to the main or a child file by means of a pattern.
% The prefix |#1| in the current filename is replaced by |#2|
% and the suffix of the current filename is kept
% (it is assumed that the filename does not contain the substring `|~~~|'
% which is used as a delimiter).
% Compilation is handed over to the new file by |\childdocforward|:
%    \begin{macrocode}
\newcommand{\childdocforwardprefix}[3][]
{
  \begingroup
    \def\childdocextract #2##1~~~{\def\childdoctmp{\childdocforward[#1]{#3##1}}}
    \expandafter\childdocextract\childdocname~~~
    \expandafter
  \endgroup
  \childdoctmp
}
%    \end{macrocode}

% \macro{\childdoc}
% The deprecated macro |\childdoc| is a legacy version of |\childdocmain|:
%    \begin{macrocode}
\newcommand{\childdoc}{\childdocmain}
%    \end{macrocode}

% \macro{\childdocredirect}
% The deprecated macro |\childdocredirect| is a legacy version
% of |\childdocforward| and |\childdocforwardprefix|:
%    \begin{macrocode}
\newcommand{\childdocredirect}[2][]
{
  \begingroup
    \if?#1?
      \def\childdoctmp{\childdocforward{#2}}
    \else
      \def\childdoctmp{\childdocforwardprefix{#1}{#2}}
    \fi
    \expandafter
  \endgroup
  \childdoctmp
}
%    \end{macrocode}

%\iffalse
%</package>
%\fi
%
\endinput

\childdocmain{}
%    \end{macrocode}

% Optional override for |\version| flag:
%    \begin{macrocode}
%%\ifchilddoc\else\providecommand{\version}{draft}\fi
%    \end{macrocode}

% Define the default values for the |\version| flag
% (|final| for the main file and |draft| for childs):
%    \begin{macrocode}
\ifchilddoc
\providecommand{\version}{draft}
\else
\providecommand{\version}{final}
\fi
%    \end{macrocode}

% Load the standard document class:
%    \begin{macrocode}
\documentclass[12pt]{article}
%    \end{macrocode}

% Start the document body:
%    \begin{macrocode}
\begin{document}
%    \end{macrocode}

% Declare a title page.
% Print title, part of document being processed and version flag:
%    \begin{macrocode}
\addtocounter{page}{-1}
\begin{center}
{\LARGE\bfseries{}childdoc example\par}
\vspace{1cm}
\ifchilddoc
\ifchilddocmanual part\else chapter\fi:
`\childdocname' of `\childdocjob'\par
\else
main document: `\childdocjob'\par
\fi
version: \version\par
\end{center}
\newpage
%    \end{macrocode}

% Manually include selected file,
% otherwise process as usual:
%    \begin{macrocode}
\ifchilddocmanual
\section*{part `\childdocname'}
\input{\childdocname}
\else
%    \end{macrocode}

% Include the two chapters:
%    \begin{macrocode}
\include{cdocsch1}
\include{cdocsch2}
%    \end{macrocode}

% Include the two parts unless only chapters should be displayed:
%    \begin{macrocode}
\ifchilddoc\else
\section{part three}
\input{cdocspt3}
\section{part four}
\input{cdocspt4}
\fi
%    \end{macrocode}

% Process as usual until here:
%    \begin{macrocode}
\fi
%    \end{macrocode}

% End of document body:
%    \begin{macrocode}
\end{document}
%    \end{macrocode}
%\iffalse
%</samplemain>
%\fi
%
% %%%%%%%%%%%%%%%%%%%%%%%%%%%%%%%%%%%%%%
% \paragraph{Chapter Include Files.}
%
% The include files are called |cdocsch1.tex| and |cdocsch2.tex|.
%
%\iffalse
%<*samplechap1|samplechap2>
%\fi

% Optional override for |\version| flag:
%    \begin{macrocode}
%%\providecommand{\version}{final}
%    \end{macrocode}

% Include the main document:
%    \begin{macrocode}
% \iffalse
%
% childdoc.dtx Copyright (C) 2017-2018 Niklas Beisert
%
% This work may be distributed and/or modified under the
% conditions of the LaTeX Project Public License, either version 1.3
% of this license or (at your option) any later version.
% The latest version of this license is in
%   http://www.latex-project.org/lppl.txt
% and version 1.3 or later is part of all distributions of LaTeX
% version 2005/12/01 or later.
%
% This work has the LPPL maintenance status `maintained'.
%
% The Current Maintainer of this work is Niklas Beisert.
%
% This work consists of the files childdoc.dtx and childdoc.ins
% and the derived files childdoc.def and cdocsamp.tex with
% cdocsch1.tex, cdocsch2.tex, cdocsdrf.tex, cdocsfn1.tex, cdocsfn2.tex.
%
%<package>\ifdefined\childdocmain\endinput\fi
%<package>\ProvidesFile{childdoc.def}[2018/12/30 v2.0 child document driver]
%<samplemain>\ProvidesFile{cdocsamp.tex}[2018/12/30 v2.0 sample for childdoc]
%<*driver>
%\ProvidesFile{childdoc.drv}[2018/12/30 v2.0 childdoc reference manual file]
\PassOptionsToClass{10pt,a4paper}{article}
\documentclass{ltxdoc}

\usepackage[margin=35mm]{geometry}
\usepackage{hyperref}
\usepackage{hyperxmp}
\usepackage[usenames]{color}

\hypersetup{colorlinks=true}
\hypersetup{pdfstartview=FitH}
\hypersetup{pdfpagemode=UseNone}
\hypersetup{pdfsource={}}
\hypersetup{pdflang={en-UK}}
\hypersetup{pdfcopyright={Copyright 2017-2018 Niklas Beisert.
  This work may be distributed and/or modified under the
  conditions of the LaTeX Project Public License, either version 1.3
  of this license or (at your option) any later version.}}
\hypersetup{pdflicenseurl={http://www.latex-project.org/lppl.txt}}
\hypersetup{pdfcontactaddress={ETH Zurich, ITP, HIT K,
  Wolfgang-Pauli-Strasse 27}}
\hypersetup{pdfcontactpostcode={8093}}
\hypersetup{pdfcontactcity={Zurich}}
\hypersetup{pdfcontactcountry={Switzerland}}
\hypersetup{pdfcontactemail={nbeisert@itp.phys.ethz.ch}}
\hypersetup{pdfcontacturl={http://people.phys.ethz.ch/\xmptilde nbeisert/}}

\newcommand{\secref}[1]{\hyperref[#1]{section \ref*{#1}}}

\parskip1ex
\parindent0pt
\let\olditemize\itemize
\def\itemize{\olditemize\parskip0pt}

\begin{document}

\title{The \textsf{childdoc} Package}
\hypersetup{pdftitle={The childdoc Package}}
\author{Niklas Beisert\\[2ex]
  Institut f\"ur Theoretische Physik\\
  Eidgen\"ossische Technische Hochschule Z\"urich\\
  Wolfgang-Pauli-Strasse 27, 8093 Z\"urich, Switzerland\\[1ex]
  \href{mailto:nbeisert@itp.phys.ethz.ch}
  {\texttt{nbeisert@itp.phys.ethz.ch}}}
\hypersetup{pdfauthor={Niklas Beisert}}
\hypersetup{pdfsubject={Manual for the LaTeX2e Package childdoc}}
\date{30 December 2018, \textsf{v2.0}}
\maketitle

\begin{abstract}\noindent
\textsf{childdoc} is a \LaTeXe{} package
that enables the direct compilation
of document sections included by |\include|
to individual files.
\end{abstract}

\begingroup
\parskip0ex
\tableofcontents
\endgroup

%%%%%%%%%%%%%%%%%%%%%%%%%%%%%%%%%%%%%%%%%%%%%%%%%%%%%%%%%%%%%%%%%%%%%%%%%%%%%%%%
%%%%%%%%%%%%%%%%%%%%%%%%%%%%%%%%%%%%%%%%%%%%%%%%%%%%%%%%%%%%%%%%%%%%%%%%%%%%%%%%
\section{Introduction}

\LaTeX{} provides a mechanism to structure a large document (such as a book)
into a main file and several child files (containing the chapters)
using the |\include| command.
This mechanism is beneficial for documents
which span hundreds of pages in order to
make the source file(s) more manageable.
Moreover, compilation can be restricted to
selected child files by means of the |\includeonly| command.
The latter feature can be used to reduce the compilation time while editing
(this was significantly more useful in the earlier days of \LaTeX{})
or to generate a smaller document which is easier to navigate.
Another application of |\includeonly| is to generate
documents consisting of selected parts of the complete document.

However, there are a few drawbacks of the plain |\include| mechanism:
\begin{itemize}
\item
The child files cannot be compiled on their own,
they can only be compiled via the main file.
A naive editing environment
(such as a text editor with an option
to have the current file processed by \LaTeX)
may require one to switch to the main file before compiling;
attempting to compile the child file produces errors.
\item
The main file must be modified (each time)
to adjust the |\includeonly| command
to the present needs. This easily leaves the main file in a messy state.
\item
The generated document will always carry the filename
of the main document. This is inconvenient if
several child files are to be compiled and
to be kept for distribution.
\end{itemize}

The present package provides a simple interface
to make child files individually compilable by \LaTeX{}.
Compiling a child file then has the same effect as compiling
the main file with an |\includeonly| command
to select the appropriate child.
Moreover the generated document will carry the name of the child
rather than the main file.
This resolves all three above issues.

This feature is meant to make the editing of books,
thesis documents and lecture notes somewhat more convenient.
However, the package can also be used efficiently for
composing a series of documents (such as exercise sheets)
which are typically distributed individually.
It then assists the author in generating the individual documents
(potentially in different versions)
as well as a document containing the collected series.
Another application is in developing style files
or other kinds of included material
where compilation of the style file could redirect
to a sample or test file.

%%%%%%%%%%%%%%%%%%%%%%%%%%%%%%%%%%%%%%%%%%%%%%%%%%%%%%%%%%%%%%%%%%%%%%%%%%%%%%%%
%%%%%%%%%%%%%%%%%%%%%%%%%%%%%%%%%%%%%%%%%%%%%%%%%%%%%%%%%%%%%%%%%%%%%%%%%%%%%%%%
\section{Usage}

First of all, the package \textsf{childdoc} is \emph{not} a standard
\LaTeXe{} |.sty| style file! Therefore it needs to be invoked in
a non-standard way.

%%%%%%%%%%%%%%%%%%%%%%%%%%%%%%%%%%%%%%%%%%%%%%%%%%%%%%%%%%%%%%%%%%%%%%%%%%%%%%%%
\subsection{Included Files}
\label{sec:include}

%%%%%%%%%%%%%%%%%%%%%%%%%%%%%%%%%%%%%%%%
\DescribeMacro{\childdocmain}
To use the package, add the commands
\begin{center}
\begin{tabular}{l}
|\input{childdoc.def}|\\
|\childdocmain{}|\\
\end{tabular}
\end{center}
at the very top of the main \LaTeX{} file,
in particular \emph{before} the |\documentclass| statement!
The argument of |\childdocmain| should be left empty
(but it must be present).

%%%%%%%%%%%%%%%%%%%%%%%%%%%%%%%%%%%%%%%%
\DescribeMacro{\childdocof}
Furthermore, add the commands
\begin{center}
\begin{tabular}{l}
|\input{childdoc.def}|\\
|\childdocof{|\textit{main}|}|\\
\end{tabular}
\end{center}
at the top of every child file \textit{child}
which is included by |\include{|\textit{child}|}|
from within the main file
(or at least for those files to be compiled individually).
The argument \textit{main} must be the filename of the main file.

There are a couple of
considerations in setting up the main and child documents:

%%%%%%%%%%%%%%%%%%%%%%%%%%%%%%%%%%%%%%%%
\paragraph{Restrictions.}

Please note the following restrictions:
\begin{itemize}
\item
|\childdocmain| must be called with one argument \textit{main}
to ensure compatibility with earlier version of the package.
It must either be empty (|\childdocmain{}|)
or precisely match the filename of the main file in which it is specified.
See \secref{sec:detection} for further information.
\item
The filename \textit{main} must be specified without the |.tex| extension.
\item
The filename \textit{main} is case sensitive
(even in case-insensitive file systems)
due to internal string comparison.
\item
The argument \textit{main} should be fully expanded, it cannot be a macro.
\item
Subdirectories and special characters should be avoided in filenames.
\item
The command |\childdocmain{|\textit{main}|}| must be followed by a whitespace.
It should not be followed immediately by another command
or by a comment mark `|%|'.
This is because the \TeX{} parser reads the token immediately following
the argument of |\childdocmain| and puts it
at the beginning of every child section;
however, a white\-space is ignored.
\end{itemize}

%%%%%%%%%%%%%%%%%%%%%%%%%%%%%%%%%%%%%%%%
\paragraph{Content of Main File.}

It is advisable to place all content in the child files included by |\include|.
Any output contained in the main file will appear in all child documents
unless suppressed manually;
it cannot be suppressed automatically by the |\includeonly| directive
and thus should normally be avoided.
A method to include some content in the main file
by means of conditional processing is described in \secref{sec:conditional}.

%%%%%%%%%%%%%%%%%%%%%%%%%%%%%%%%%%%%%%%%
\paragraph{Page Numbering.}

When only a part of the document is compiled,
the appropriate numbering of pages
(as well as other status parameters)
is determined from the |.aux| files.
The latter contain information from previous passes.
However this information needs to propagate through
all intermediate child documents.
Therefore the page numbering in child documents may well
be inconsistent until the complete document is compiled at least once.

A useful (if unconventional) way to always ensure a consistent
page numbering is to restart the numbering in each child document
and denote the pages by `\textit{child}|.|\textit{page}'
where \textit{child} represents the chapter/section number of the child file.
This can be achieved by the command
|\numberwithin{page}{|\textit{child}|}|
of the \textsf{amsmath} package
where \textit{child} can be |chapter| or |section|
depending on the chosen structuring.
Alternatively, one can modify the macro |\thepage| appropriately
and reset the counter |page| at the start of each child file.

%%%%%%%%%%%%%%%%%%%%%%%%%%%%%%%%%%%%%%%%%%%%%%%%%%%%%%%%%%%%%%%%%%%%%%%%%%%%%%%%
\subsection{Conditional Processing}
\label{sec:conditional}

The package provides a mechanism to compile different versions
of a document. To customise the versions further some conditional processing
can come in handy to distinguish which version is being compiled.
The package provides two macros to describe the compilation context:

%%%%%%%%%%%%%%%%%%%%%%%%%%%%%%%%%%%%%%%%
\DescribeMacro{\ifchilddoc}
The conditional |\ifchilddoc| distinguishes between the compilation of
child documents and the main document:
%
\begin{center}
|\ifchilddoc |\textit{child-code}| |[|\||else |\textit{main-code}]| \||fi|
\end{center}

%%%%%%%%%%%%%%%%%%%%%%%%%%%%%%%%%%%%%%%%
\DescribeMacro{\childdocname}
\DescribeMacro{\childdocjob}
The macro |\childdocname| contains the filename (without extension)
of the main or child file being processed.
Note that |\childdocjob| will always contain the name of the main file.

%%%%%%%%%%%%%%%%%%%%%%%%%%%%%%%%%%%%%%%%
\paragraph{Title Page.}

Conditional processing can be used to include a title or banner page
in the main document when proper precautions are taken.
Importantly, the code in the main file should ensure that the page counter
(as well as other status parameters which are stored in the |.aux| files)
takes the same value after the conditional processing.
Otherwise the page numbers may take divergent values
depending on which part is compiled.

For example, a title page could be declared by:
%
\begin{center}
\begin{tabular}{l}
|\ifchilddoc\||else|\\
|\addtocounter{page}{-1}|\\
\textit{code for title page}\\
|\newpage|\\
|\||fi|
\end{tabular}
\end{center}
%
A banner page for the child documents can be generated by:
%
\begin{center}
\begin{tabular}{l}
|\ifchilddoc|\\
|\addtocounter{page}{-1}|\\
\textit{code for banner page}\\
|\newpage|\\
|\||fi|
\end{tabular}
\end{center}
%
Here one could write a message such as:
\begin{center}
|This is the part \childdocname{} of \childdocjob{}.|
\end{center}

%%%%%%%%%%%%%%%%%%%%%%%%%%%%%%%%%%%%%%%%%%%%%%%%%%%%%%%%%%%%%%%%%%%%%%%%%%%%%%%%
\subsection{Flags}
\label{sec:flags}

The package makes it easy to generate different versions
of the main or child documents.
To this end compilation flags can be defined
and assigned different default values.
They will be particularly useful in conjunction
with the forwarding mechanism described in \secref{sec:forward}.

For example, it may be useful to have a flag |\version|
which can be set to |draft| or |final|.
The document source will contain some conditional code
depending on the value of |\version|.
Suppose further, the flag should default to |final| for the main file
and to |draft| for child files
which is a natural assignment for editing the document.
This is achieved by placing the following code
in the preamble of the main document
(below the |\childdocmain| directive):
%
\begin{center}
\begin{tabular}{l}
|\ifchilddoc|\\
|\providecommand{\version}{draft}|\\
|\||else|\\
|\providecommand{\version}{final}|\\
|\||fi|
\end{tabular}
\end{center}
%
The definition by |\providecommand| makes sure
that previous definitions are not overwritten.
Further statements |\providecommand{\version}{...}|
can thus be added before the above code to override it.

For the main file, one might add a line
(between |\childdocmain| and the above block)
%
\begin{center}
|%\ifchilddoc\||else\providecommand{\version}{draft}\||fi|
\end{center}
%
which can be uncommented to produce a draft version.
Likewise one can add a line to the very top of a child file
(above the |\childdocof{|\textit{main}|}| directive)
%
\begin{center}
|%\providecommand{\version}{final}|
\end{center}
%
which can be uncommented to produce the final version of this child document.

%%%%%%%%%%%%%%%%%%%%%%%%%%%%%%%%%%%%%%%%%%%%%%%%%%%%%%%%%%%%%%%%%%%%%%%%%%%%%%%%
\subsection{Forwarding}
\label{sec:forward}

Different versions of the main or child documents
using compilation flags as described in \secref{sec:flags}
can be (permanently) stored in different files
for convenient compilation, viewing and distribution.
To this end, the package defines a command
to pass on compilation to a different file:

%%%%%%%%%%%%%%%%%%%%%%%%%%%%%%%%%%%%%%%%
\DescribeMacro{\childdocforward}
The command |\childdocforward| redirects processing to
another source file:
%
\begin{center}
\begin{tabular}{l}
|\input{childdoc.def}|\\
|\childdocforward[|\textit{main}|]{|\textit{dest}|}|\\
\end{tabular}
\end{center}
%
The argument \textit{dest} is the destination file
(without extension).
It should be the main file or one of the child files.
Note that further \textsf{childdoc} directives
such as |\childdocof| and |\childdocforward|
in the indicated file will be processed in this form.
The optional argument \textit{main}
passes on directly to the main file \textit{main}
while pretending to compile the child \textit{dest}.
This form behaves as if \textit{dest}
issues |\childdocof{|\textit{main}|}| right away,
and no further \textsf{childdoc} directives will be processed.

%%%%%%%%%%%%%%%%%%%%%%%%%%%%%%%%%%%%%%%%
\DescribeMacro{\...prefix}
In the alternative form |\childdocforwardprefix|,
%
\begin{center}
\begin{tabular}{l}
|\input{childdoc.def}|\\
|\childdocforwardprefix[|\textit{main}|]{|\textit{prefix}|}{|\textit{dest}|}|
\end{tabular}
\end{center}
%
the destination file is determined by a pattern
depending on the current file:
To make this work, the current file must be called
`{\textit{prefix}\hspace{0.2em}\textit{suffix}}'
with \textit{prefix} matching precisely the argument.
Processing is then passed on to the file
`{\textit{dest}\hspace{0.2em}\textit{suffix}}'.
Surely, the same effect is achieved by
directly specifying the
argument `{\textit{dest}\hspace{0.2em}\textit{suffix}}'
in the first form.
However, that requires to set up a different file
for each child. With the alternative form of the command
all these files can have exactly the same content
which simplifies setting them up and maintaining them.

For example, the following file |draft.tex|
with a compilation flag |\version| as described in \secref{sec:flags}
compiles the main document as a draft:
%
\begin{center}
\begin{tabular}{l}
|\def\version{draft}|\\
|\input{childdoc.def}|\\
|\childdocforward{|\textit{main}|}|
\end{tabular}
\end{center}
%
Likewise, the following files |final|\textit{nn}|.tex|
compile the final version of the child document
|child|\textit{nn}|.tex|:
%
\begin{center}
\begin{tabular}{l}
|\def\version{final}|\\
|\input{childdoc.def}|\\
|\childdocforwardprefix{final}{child}|
\end{tabular}
\end{center}
%

Note that when several versions of a main file and/or of each child file
are to be generated, it may be convenient to set up a |Makefile| or
shell script to automatise the process.

%%%%%%%%%%%%%%%%%%%%%%%%%%%%%%%%%%%%%%%%%%%%%%%%%%%%%%%%%%%%%%%%%%%%%%%%%%%%%%%%
\subsection{Command Line Processing}
\label{sec:commandline}

The effect of redirection files can also be achieved by invoking
the \LaTeX{} compiler with a more elaborate command line.
Most conveniently this should be done as part
of a shell script or a |Makefile|.

When using \textsf{childdoc} in the main file, the following
command lines effectively perform a redirection
(note that depending on the shell being used,
backslashes may have to be doubled: `|\|' $\to$ `|\\|'):
%
\begin{center}
|... -jobname "|\textit{target}|" |\\|"|[\textit{flags}]%
|\input{childdoc.def}\childdocforward[|\textit{main}|]{|\textit{dest}|}"|
\end{center}
%
Here \textit{target} is the name of the output file,
\textit{main} is the name of the main file
and \textit{dest} is the name of the main or child file to be processed
(all filenames without extensions).
The optional argument \textit{main} can be omitted
if \textit{main} matches \textit{dest}.
Optionally, compilation \textit{flags} can be defined via |\def| commands.
This command line makes the \TeX{} engine believe
it is compiling the file \textit{target}
whose content is specified as the latter parameter.
The provided code then forwards the processing to
\textit{main} or \textit{dest} as described in \secref{sec:forward}.

%%%%%%%%%%%%%%%%%%%%%%%%%%%%%%%%%%%%%%%%%%%%%%%%%%%%%%%%%%%%%%%%%%%%%%%%%%%%%%%%
\subsection{Include by Input}
\label{sec:input}

Including child documents by |\include| has some restrictions by design.
Most notably, the content of a child document always occupies
its own set of pages; pages cannot be shared between child documents.
Usually, this behaviour makes perfect sense
because each child document contain an essential part of the document.
However, in some situations it may be desirable to compose
a document from a collection of parts
without having mandatory page breaks between then.
For this case, the package
provides a mechanism to include parts
by |\input| which can also be processed individually.
However, by construction this mechanism
requires manual handling of the content to be output.

%%%%%%%%%%%%%%%%%%%%%%%%%%%%%%%%%%%%%%%%
\DescribeMacro{\ifchilddocmanual}
The main file should be prepared as usual, see \secref{sec:include}.
However, the document body must make a distinction
between processing of an individual part and of the main document, e.g.:
%
\begin{center}
\begin{tabular}{l}
|\ifchilddocmanual|\\
|\input{\childdocname}|\\
|\||else|\\
\textit{document body with }|\input{|\textit{part}|}|\\
|\||fi|
\end{tabular}
\end{center}
%
The conditional |\ifchilddocmanual| is true whenever
a part to be included by |\input| is being compiled,
and the name of the part is stored in |\childdocname|.

%%%%%%%%%%%%%%%%%%%%%%%%%%%%%%%%%%%%%%%%
\DescribeMacro{\childdocby}
Each part to be included by |\input| should start with:
%
\begin{center}
\begin{tabular}{l}
|\input{childdoc.def}|\\
|\childdocby{|\textit{main}|}|\\
\end{tabular}
\end{center}
%
The directive |\childdocby| is similar to |\childdocof|
described in \secref{sec:include},
but the subsequent selection of content must be done manually.
To that end, both |\ifchilddoc| and |\ifchilddocmanual|
will be true upon processing of a part,
and the name of the part is stored in |\childdocname|.
Note that |\jobname| will be set to the filename of the current part
so that each part receives an individual |.aux| file
that does not interfere with the |.aux| file(s) of the main document.
This behaviour can be altered by the alternative form
|\childdocby[*]{|\textit{main}|}| (with a non-empty optional argument)
which uses the |.aux| file of the main document
by setting |\jobname| to \textit{main}.

%%%%%%%%%%%%%%%%%%%%%%%%%%%%%%%%%%%%%%%%%%%%%%%%%%%%%%%%%%%%%%%%%%%%%%%%%%%%%%%%
\subsection{Driver Development}
\label{sec:driver}

The \textsf{childdoc} mechanism can also be use for the development
of definition files such as \LaTeX{} styles or classes.
This case differs from the above setup with multiple parts
included by |\include| in that no |\includeonly| should be invoked.
This can be achieved by starting the include file
(before |\ProvidesPackage|) with:
%
\begin{center}
\begin{tabular}{l}
|\input{childdoc.def}|\\
|\childdocforward{|\textit{main}|}|\\
\end{tabular}
\end{center}
%
or alternatively with:
%
\begin{center}
\begin{tabular}{l}
|\input{childdoc.def}|\\
|\childdocby{|\textit{main}|}|\\
\end{tabular}
\end{center}
%
Both forms have slightly different effects as described above.
The main file is prepared as usual, see \secref{sec:include}.

%%%%%%%%%%%%%%%%%%%%%%%%%%%%%%%%%%%%%%%%%%%%%%%%%%%%%%%%%%%%%%%%%%%%%%%%%%%%%%%%
\subsection{Legacy Detection}
\label{sec:detection}

The directive |\childdocmain| in the main file can detect
whether the complete document or merely a child is to be compiled
even without using the directive |\childdocof|.
This method is deprecated because it is less robust
and there is no compelling reason to use it;
it is merely provided for backward compatibility
and it may be removed in future versions.

If the detection mechanism is to be used,
it is mandatory to correctly specify
the filename of the main file as the argument of |\childdocmain|:
%
\begin{center}
\begin{tabular}{l}
|\input{childdoc.def}|\\
|\childdocmain{|\textit{main}|}|\\
\end{tabular}
\end{center}
%
If |\jobname| does not match the argument \textit{main} of |\childdocmain|,
it is assumed that |\jobname| points to the child file to be compiled.
When using |\childdocmain| with the main file specified as argument,
it suffices to start a child file
with just |\input{|\textit{main}|}|
without loading of the package and using |\childdocof|.
If instead all processing is done
with the appropriate \textsf{childdoc} directives,
the argument of \textit{main} of |\childdocmain| can be empty.

An alternative version of the command line processing described
in \secref{sec:commandline} using the detection mechanism reads:
%
\begin{center}
|... -jobname "|\textit{target}|" "|[\textit{flags}]%
[|\def\jobname{|\textit{dest}|}|]|\input{|\textit{main}|}"|
\end{center}

%%%%%%%%%%%%%%%%%%%%%%%%%%%%%%%%%%%%%%%%%%%%%%%%%%%%%%%%%%%%%%%%%%%%%%%%%%%%%%%%
\subsection{Manual Code}
\label{sec:manual}

In case one cannot be certain whether the definitions file |childdoc.def|
is installed on the target \TeX{} distribution
and one prefers not to ship it,
it is conceivable to paste a few relevant commands into the sources.

To that end, drop all statements |\input{childdoc.def}|
and perform the replacements as outlined below.
Instead of |\childdocmain{|\textit{main}|}| add the following code
to the top of the main file:
%
\begin{center}
\begin{tabular}{l}
|\||ifdefined\childdocname\endinput\||fi\newif\ifchilddoc|\\
|\edef\childdocname{\scantokens\expandafter{\jobname\noexpand}}|\\
|\def\childdocmain{|\textit{main}|}\||ifx\childdocmain\childdocname\||else|\\
|\childdoctrue\includeonly{\childdocname}\let\jobname\childdocmain\||fi|\\
\end{tabular}
\end{center}
%
Instead of |\childdocof{|\textit{main}|}| just include the main file
at the top of each child file:
%
\begin{center}
|\input{|\textit{main}|}|
\end{center}
%
A simple redirection |\childdocforward{|\textit{dest}|}| is achieved by:
%
\begin{center}
|\def\jobname{|\textit{dest}|}\input{\jobname}|
\end{center}
%
The redirection with prefix
|\childdocforwardprefix[|\textit{prefix}|]{|\textit{dest}|}|
is accomplished by:
%
\begin{center}
\begin{tabular}{l}
|{\edef\jobname{\scantokens\expandafter{\jobname\noexpand}}|\\
|\def\redirectjob |\textit{prefix}|#1~~~{\gdef\jobname{|\textit{dest}|#1}}|\\
|\expandafter\redirectjob\jobname~~~}\input{\jobname}|
\end{tabular}
\end{center}

In an alternative approach,
child documents can be compiled by a specific command line
without additional code or specific definitions:
%
\begin{center}
|... -jobname "|\textit{target}|" "|[\textit{flags}]%
|\includeonly{|\textit{dest}|}\input{|\textit{main}|}"|
\end{center}
%

%%%%%%%%%%%%%%%%%%%%%%%%%%%%%%%%%%%%%%%%%%%%%%%%%%%%%%%%%%%%%%%%%%%%%%%%%%%%%%%%
%%%%%%%%%%%%%%%%%%%%%%%%%%%%%%%%%%%%%%%%%%%%%%%%%%%%%%%%%%%%%%%%%%%%%%%%%%%%%%%%
\section{Information}

%%%%%%%%%%%%%%%%%%%%%%%%%%%%%%%%%%%%%%%%%%%%%%%%%%%%%%%%%%%%%%%%%%%%%%%%%%%%%%%%
\subsection{Copyright}

Copyright \copyright{} 2017--2018 Niklas Beisert

This work may be distributed and/or modified under the
conditions of the \LaTeX{} Project Public License, either version 1.3
of this license or (at your option) any later version.
The latest version of this license is in
  \url{http://www.latex-project.org/lppl.txt}
and version 1.3 or later is part of all distributions of \LaTeX{}
version 2005/12/01 or later.

This work has the LPPL maintenance status `maintained'.

The Current Maintainer of this work is Niklas Beisert.

This work consists of the files |README.txt|, |childdoc.ins| and |childdoc.dtx|
as well as the derived files |childdoc.def|, |cdocsamp.tex|
with |cdocsch1.tex|, |cdocsch2.tex|, |cdocspt3.tex|, |cdocspt4.tex|,
|cdocsdrf.tex|, |cdocsfn1.tex|, |cdocsfn2.tex|
as well as |childdoc.pdf|.

%%%%%%%%%%%%%%%%%%%%%%%%%%%%%%%%%%%%%%%%%%%%%%%%%%%%%%%%%%%%%%%%%%%%%%%%%%%%%%%%
\subsection{Files and Installation}

The package consists of the files:
%
\begin{center}
\begin{tabular}{ll}
    |README.txt|   & readme file \\
    |childdoc.ins| & installation file \\
    |childdoc.dtx| & source file \\
    |childdoc.def| & definition file \\
    |cdocsamp.tex| & sample main file \\
    |cdocsch1.tex| & sample include file \\
    |cdocsch2.tex| & sample include file \\
    |cdocspt3.tex| & sample part file \\
    |cdocspt4.tex| & sample part file \\
    |cdocsdrf.tex| & sample redirection file \\
    |cdocsfn1.tex| & sample redirection file \\
    |cdocsfn2.tex| & sample redirection file \\
    |childdoc.pdf| & manual
\end{tabular}
\end{center}
%
The distribution consists of the files
|README.txt|, |childdoc.ins| and |childdoc.dtx|.
%
\begin{itemize}
\item
Run (pdf)\LaTeX{} on |childdoc.dtx|
to compile the manual |childdoc.pdf| (this file).
\item
Run \LaTeX{} on |childdoc.ins| to create the definitions file |childdoc.def|
and the sample |cdocsamp.tex| with include files
|cdocsch1.tex|, |cdocsch2.tex|, |cdocspt3.tex|, |cdocspt4.tex|,
|cdocsdrf.tex|, |cdocsfn1.tex|, |cdocsfn2.tex|.
Then copy the file |childdoc.def| to an appropriate directory of your \LaTeX{}
distribution, e.g.\ \textit{texmf-root}|/tex/latex/childdoc|.
\end{itemize}

%%%%%%%%%%%%%%%%%%%%%%%%%%%%%%%%%%%%%%%%%%%%%%%%%%%%%%%%%%%%%%%%%%%%%%%%%%%%%%%%
\subsection{Related CTAN Packages}

There are several other packages which offer a similar functionality:
%
\begin{itemize}
\item
The packages
\href{http://ctan.org/pkg/docmute}{\textsf{docmute}},
\href{http://ctan.org/pkg/includex}{\textsf{includex}} and
\href{http://ctan.org/pkg/standalone}{\textsf{standalone}}
provide commands to include only the document body of
a child file thus allowing both files to be compiled individually.
\item
The packages \href{http://ctan.org/pkg/subdocs}{\textsf{subdocs}}
and \href{http://ctan.org/pkg/subfiles}{\textsf{subfiles}}
provide structures in which the main and child documents can be
encapsulated and allowing them to be compiled individually.
The inclusion mechanism is different from the conventional |\include|.
\item
The package \href{http://ctan.org/pkg/combine}{\textsf{combine}}
is an elaborate solution to combine several documents into one.
\end{itemize}
%
See also the CTAN topic \href{http://ctan.org/topic/subdocs}{\textsf{subdocs}}
for further related packages.
The present package differs from the above solutions in that
a document structure constructed with the conventional |\include| mechanism
just needs two extra commands at the top of every file
such that all constituent files can be compiled individually.

%%%%%%%%%%%%%%%%%%%%%%%%%%%%%%%%%%%%%%%%%%%%%%%%%%%%%%%%%%%%%%%%%%%%%%%%%%%%%%%%
%\subsection{Feature Suggestions}
%
%The following is a list of features which may be useful for future
%versions of this package:
%%
%\begin{itemize}
%\item
%\ldots
%\end{itemize}

%%%%%%%%%%%%%%%%%%%%%%%%%%%%%%%%%%%%%%%%%%%%%%%%%%%%%%%%%%%%%%%%%%%%%%%%%%%%%%%%
\subsection{Revision History}

%%%%%%%%%%%%%%%%%%%%%%%%%%%%%%%%%%%%%%%%
\paragraph{v2.0:} 2018/12/30

\begin{itemize}
\item
immediate forward processing
\item
added |\childdocby| mechanism
\item
manual restructured
\end{itemize}

%%%%%%%%%%%%%%%%%%%%%%%%%%%%%%%%%%%%%%%%
\paragraph{v1.6:} 2018/01/17

\begin{itemize}
\item
application for development of include files
\item
corrections to manual
\end{itemize}

%%%%%%%%%%%%%%%%%%%%%%%%%%%%%%%%%%%%%%%%
\paragraph{v1.5:} 2017/05/21

\begin{itemize}
\item
more complete structuring introduced
\item
|\childdocof| introduced
\item
|\childdoc| renamed to |\childdocmain|
\item
|\childredirect| renamed to |\childdocforward| and |\childdocforwardprefix|
and functionality expanded
\end{itemize}

%%%%%%%%%%%%%%%%%%%%%%%%%%%%%%%%%%%%%%%%
\paragraph{v1.0:} 2017/04/27

\begin{itemize}
\item
manual and install package
\item
first version published on CTAN
\end{itemize}

%%%%%%%%%%%%%%%%%%%%%%%%%%%%%%%%%%%%%%%%
\paragraph{v0.6:} 2017/04/26

\begin{itemize}
\item
redirection mechanism added
\end{itemize}

%%%%%%%%%%%%%%%%%%%%%%%%%%%%%%%%%%%%%%%%
\paragraph{v0.5:} 2017/04/26

\begin{itemize}
\item
functionality in definition file
\end{itemize}


%%%%%%%%%%%%%%%%%%%%%%%%%%%%%%%%%%%%%%%%%%%%%%%%%%%%%%%%%%%%%%%%%%%%%%%%%%%%%%%%
%%%%%%%%%%%%%%%%%%%%%%%%%%%%%%%%%%%%%%%%%%%%%%%%%%%%%%%%%%%%%%%%%%%%%%%%%%%%%%%%
%%%%%%%%%%%%%%%%%%%%%%%%%%%%%%%%%%%%%%%%%%%%%%%%%%%%%%%%%%%%%%%%%%%%%%%%%%%%%%%%
\appendix

\settowidth\MacroIndent{\rmfamily\scriptsize 000\ }

 \DocInput{childdoc.dtx}

\end{document}
%</driver>
% \fi
%
% %%%%%%%%%%%%%%%%%%%%%%%%%%%%%%%%%%%%%%%%%%%%%%%%%%%%%%%%%%%%%%%%%%%%%%%%%%%%%%
% %%%%%%%%%%%%%%%%%%%%%%%%%%%%%%%%%%%%%%%%%%%%%%%%%%%%%%%%%%%%%%%%%%%%%%%%%%%%%%
% \section{Sample}
%\iffalse
%<*samplemain>
%\fi
%
% The following presents a sample document
% with two chapters, two parts, a title page,
% a compile flag as well as three forwarding files to set the flag.
% It consists of eight |.tex| files:
% \begin{center}
% \begin{tabular}{ll}
% |cdocsamp.tex|&main file\\
% |cdocsch1.tex|&include file for chapter 1\\
% |cdocsch2.tex|&include file for chapter 2\\
% |cdocspt3.tex|&include file for part 3\\
% |cdocspt4.tex|&include file for part 4\\
% |cdocsdrf.tex|&forwarding file for main file in draft mode\\
% |cdocsfi1.tex|&forwarding file for final version of chapter 1\\
% |cdocsfi2.tex|&forwarding file for final version of chapter 2\\
% \end{tabular}
% \end{center}
% Each of the eight files can be compiled directly by the \LaTeX{} compiler.
%
% %%%%%%%%%%%%%%%%%%%%%%%%%%%%%%%%%%%%%%
% \paragraph{Main File.}
%
% The main file is called |cdocsamp.tex|.
%
% Load the \textsf{childdoc} definitions and
% declare the filename for the main document:
%    \begin{macrocode}
\input{childdoc.def}
\childdocmain{}
%    \end{macrocode}

% Optional override for |\version| flag:
%    \begin{macrocode}
%%\ifchilddoc\else\providecommand{\version}{draft}\fi
%    \end{macrocode}

% Define the default values for the |\version| flag
% (|final| for the main file and |draft| for childs):
%    \begin{macrocode}
\ifchilddoc
\providecommand{\version}{draft}
\else
\providecommand{\version}{final}
\fi
%    \end{macrocode}

% Load the standard document class:
%    \begin{macrocode}
\documentclass[12pt]{article}
%    \end{macrocode}

% Start the document body:
%    \begin{macrocode}
\begin{document}
%    \end{macrocode}

% Declare a title page.
% Print title, part of document being processed and version flag:
%    \begin{macrocode}
\addtocounter{page}{-1}
\begin{center}
{\LARGE\bfseries{}childdoc example\par}
\vspace{1cm}
\ifchilddoc
\ifchilddocmanual part\else chapter\fi:
`\childdocname' of `\childdocjob'\par
\else
main document: `\childdocjob'\par
\fi
version: \version\par
\end{center}
\newpage
%    \end{macrocode}

% Manually include selected file,
% otherwise process as usual:
%    \begin{macrocode}
\ifchilddocmanual
\section*{part `\childdocname'}
\input{\childdocname}
\else
%    \end{macrocode}

% Include the two chapters:
%    \begin{macrocode}
\include{cdocsch1}
\include{cdocsch2}
%    \end{macrocode}

% Include the two parts unless only chapters should be displayed:
%    \begin{macrocode}
\ifchilddoc\else
\section{part three}
\input{cdocspt3}
\section{part four}
\input{cdocspt4}
\fi
%    \end{macrocode}

% Process as usual until here:
%    \begin{macrocode}
\fi
%    \end{macrocode}

% End of document body:
%    \begin{macrocode}
\end{document}
%    \end{macrocode}
%\iffalse
%</samplemain>
%\fi
%
% %%%%%%%%%%%%%%%%%%%%%%%%%%%%%%%%%%%%%%
% \paragraph{Chapter Include Files.}
%
% The include files are called |cdocsch1.tex| and |cdocsch2.tex|.
%
%\iffalse
%<*samplechap1|samplechap2>
%\fi

% Optional override for |\version| flag:
%    \begin{macrocode}
%%\providecommand{\version}{final}
%    \end{macrocode}

% Include the main document:
%    \begin{macrocode}
\input{childdoc.def}
\childdocof{cdocsamp}
%    \end{macrocode}

%\iffalse
%</samplechap1|samplechap2>
%\fi
%
%\iffalse
%<*samplechap1>
%\fi
% Some text for chapter 1:
%    \begin{macrocode}
\section{one}
some text in chapter one
%    \end{macrocode}

%\iffalse
%</samplechap1>
%\fi
% Some text for chapter 2:
%\iffalse
%<*samplechap2>
%\fi
%    \begin{macrocode}
\section{two}
more text in chapter two
%    \end{macrocode}

%\iffalse
%</samplechap2>
%\fi
%
% %%%%%%%%%%%%%%%%%%%%%%%%%%%%%%%%%%%%%%
% \paragraph{Part Include Files.}
%
% The include files are called |cdocspt3.tex| and |cdocspt4.tex|.
%
%\iffalse
%<*samplepart3|samplepart4>
%\fi

% Optional override for |\version| flag:
%    \begin{macrocode}
%%\providecommand{\version}{final}
%    \end{macrocode}

% Include the main document:
%    \begin{macrocode}
\input{childdoc.def}
\childdocby{cdocsamp}
%    \end{macrocode}

%\iffalse
%</samplepart3|samplepart4>
%\fi
%
%\iffalse
%<*samplepart3>
%\fi
% Some text for part 3:
%    \begin{macrocode}
some text in part three
%    \end{macrocode}

%\iffalse
%</samplepart3>
%\fi
% Some text for part 4:
%\iffalse
%<*samplepart4>
%\fi
%    \begin{macrocode}
more text in part four
%    \end{macrocode}

%\iffalse
%</samplepart4>
%\fi
%
% %%%%%%%%%%%%%%%%%%%%%%%%%%%%%%%%%%%%%%
% \paragraph{Forwarding for a Complete Draft.}
%
% The following forwarding file |cdocsdrf.tex|
% compiles the main document in draft mode:
%\iffalse
%<*sampledraft>
%\fi
%    \begin{macrocode}
\def\version{draft}
\input{childdoc.def}
\childdocforward{cdocsamp}
%    \end{macrocode}

%\iffalse
%</sampledraft>
%\fi
%
% %%%%%%%%%%%%%%%%%%%%%%%%%%%%%%%%%%%%%%
% \paragraph{Forwarding for Final Version of the Chapters.}
%
% The following forwarding files |cdocsfn1.tex| and |cdocsfn2.tex|
% (with identical content)
% compile the final versions of the child documents
% |cdocsch1.tex| and |cdocsch2.tex|, respectively:
%\iffalse
%<*samplefinal>
%\fi
%    \begin{macrocode}
\def\version{final}
\input{childdoc.def}
\childdocforwardprefix[cdocsamp]{cdocsfn}{cdocsch}
%    \end{macrocode}

%\iffalse
%</samplefinal>
%\fi
%
% %%%%%%%%%%%%%%%%%%%%%%%%%%%%%%%%%%%%%%
% \paragraph{Command Line Processing.}
%
% The following three command lines generate the output files
% |cdocscld|, |cdocscl1| and |cdocscl2|
% which should be identical to
% |cdocsdrf|, |cdocsch1| and |cdocsfn2|, respectively:
% \begin{center}
% \begin{tabular}{l}
% |latex -jobname cdocscld \|\\
% |  "\def\version{draft}\input{childdoc.def}\childdocforward{cdocsamp}"|\\
% |latex -jobname cdocscl1 \|\\
% |  "\input{childdoc.def}\childdocforward[cdocsamp]{cdocsch1}"|\\
% |latex -jobname cdocscl2 \|\\
% |  "\def\version{final}\input{childdoc.def}\childdocforward{cdocsch2}"|
% \end{tabular}
% \end{center}
% Note that the trailing backslash on each first line
% merely continues the input to the second line
% (for convenient cut ant paste).
% Furthermore, the command |latex| can be replaced by any
% of its alternative versions such as |pdflatex|.
%
% %%%%%%%%%%%%%%%%%%%%%%%%%%%%%%%%%%%%%%%%%%%%%%%%%%%%%%%%%%%%%%%%%%%%%%%%%%%%%%
% %%%%%%%%%%%%%%%%%%%%%%%%%%%%%%%%%%%%%%%%%%%%%%%%%%%%%%%%%%%%%%%%%%%%%%%%%%%%%%
% \section{Implementation}
%\iffalse
%<*package>
%\fi
%
% This section describes the definitions file |childdoc.def|.

% The definitions cannot be loaded using |\usepackage| or |\RequirePackage|
% which has a mechanism to prevent loading a style file more than once.
% When loading the definitions by means of |\input|
% multiple instances have to be prevented manually:
%\iffalse
%This code needs to be before the `\ProvidesFile' directive
%which is defined at the beginning of this file.
%Therefore it is also placed there and commented out here.
%</package>
%<*discard>
%\fi
%    \begin{macrocode}
\ifdefined\childdocmain\endinput\fi
%    \end{macrocode}
%\iffalse
%</discard>
%<*package>
%\fi
%
% \macro{\ifchilddoc}
% \macro{\ifchilddocmanual}
% The conditional |\ifchilddoc| tells whether a
% child (true) or main (false) document is being compiled.
% The conditional |\ifchilddocmanual| tells whether
% the |\includeonly| mechanism is used (false) or
% the selection of child files must be performed manually (true).
% The definitions initialise to false:
%    \begin{macrocode}
\newif\ifchilddoc
\newif\ifchilddocmanual
%    \end{macrocode}

% \macro{\childdocname}
% \macro{\childdocjob}
% The macro |\childdocname| stores the name of the main document
% to be compiled. The macro |\childdocjob| stores the name of
% the document on which the \LaTeX{} compiler was originally invoked.
% The content of |\jobname| cannot be compared
% to filenames specified in the source due to different catcodes.
% The following code rescans |\jobname|, stores the result
% in |\childdocname| and saves a copy in |\childdocjob|:
%    \begin{macrocode}
\edef\childdocname{\scantokens\expandafter{\jobname\noexpand}}
\let\childdocjob\childdocname
%    \end{macrocode}

% \macro{\childdocdisable}
% The macro |\childdocdisable| prevents the main file
% from being processed more than once.
% At this stage, the main document command |\childdocmain|
% is assumed to be called once again where it should do nothing.
% Any subsequent call to it should prevent
% a secondary processing of the main document
% It overwrites the forwarding commands
% |\childdocof| and |\childdocforward|
% with empty macros to prevent further inclusions of the main document:
%    \begin{macrocode}
\newcommand{\childdocdisable}
{
  \renewcommand{\childdocmain}[1]{\renewcommand{\childdocmain}[1]{\endinput}}
  \renewcommand{\childdocof}[1]{}
  \renewcommand{\childdocby}[2][]{}
  \renewcommand{\childdocforward}[2][]{}
  \renewcommand{\childdocdisable}{}
}
%    \end{macrocode}

% \macro{\childdocmain}
% The macro |\childdocmain| is to be called at the top of the main file
% with nothing or the main filename (without extension) as argument.
% First, it breaks loops.
% If the argument is not empty and does not match |\childdocname|
% (which is set by the first inclusion of |childdoc.def|),
% |\ifchilddoc| is set to true, |\includeonly| is applied to the child file
% and |\jobname| is set to the main file
% (for proper handling of |.aux| files):
%    \begin{macrocode}
\newcommand{\childdocmain}[1]
{
  \childdocdisable\childdocmain{}
  \if?#1?\else
    \begingroup
      \def\childdoctmp{#1}
      \ifx\childdoctmp\childdocname
        \def\childdoctmp{}
      \else
        \def\childdoctmp
        {
          \childdoctrue
          \includeonly{\childdocname}
          \def\childdocjob{#1}
          \def\jobname{#1}
        }
      \fi
      \expandafter
    \endgroup
    \childdoctmp
  \fi
}
%    \end{macrocode}

% \macro{\childdocof}
% The command |\childdocof| redirects
% compilation to the main file |#1|.
%    \begin{macrocode}
\newcommand{\childdocof}[1]
{
  \childdocdisable
  \childdoctrue
  \includeonly{\childdocname}
  \def\jobname{#1}
  \def\childdocjob{#1}
  \input{#1}
}
%    \end{macrocode}

% \macro{\childdocby}
% The command |\childdocby| ....
%    \begin{macrocode}
\newcommand{\childdocby}[2][]
{
  \childdocdisable
  \childdoctrue
  \childdocmanualtrue
  \if?#1?\else
    \def\jobname{#2}
  \fi
  \def\childdocjob{#2}
  \input{#2}
  \endinput
}
%    \end{macrocode}

% \macro{\childdocforward}
% The command |\childdocforward| redirects
% compilation to the main file or
% (if the optional argument is given) a child file.
% Parameters are set as if the main file
% or a child file starting with |\childdocof| was compiled.
% Then compilation is handed over to the main file:
%    \begin{macrocode}
\newcommand{\childdocforward}[2][]
{
  \begingroup
    \if?#1?
      \def\childdoctmp
      {
        \def\childdocname{#2}
        \def\childdocjob{#2}
        \def\jobname{#2}
        \input{#2}
        \endinput
      }
    \else
      \def\childdoctmp
      {
        \childdocdisable
        \def\childdocname{#2}
        \childdoctrue
        \includeonly{#2}
        \def\childdocjob{#1}
        \def\jobname{#1}
        \input{#1}
        \endinput
      }
    \fi
    \expandafter
  \endgroup
  \childdoctmp
}
%    \end{macrocode}

% \macro{\childdocforwardprefix}
% The command |\childdocforwardprefix| redirects
% compilation to the main or a child file by means of a pattern.
% The prefix |#1| in the current filename is replaced by |#2|
% and the suffix of the current filename is kept
% (it is assumed that the filename does not contain the substring `|~~~|'
% which is used as a delimiter).
% Compilation is handed over to the new file by |\childdocforward|:
%    \begin{macrocode}
\newcommand{\childdocforwardprefix}[3][]
{
  \begingroup
    \def\childdocextract #2##1~~~{\def\childdoctmp{\childdocforward[#1]{#3##1}}}
    \expandafter\childdocextract\childdocname~~~
    \expandafter
  \endgroup
  \childdoctmp
}
%    \end{macrocode}

% \macro{\childdoc}
% The deprecated macro |\childdoc| is a legacy version of |\childdocmain|:
%    \begin{macrocode}
\newcommand{\childdoc}{\childdocmain}
%    \end{macrocode}

% \macro{\childdocredirect}
% The deprecated macro |\childdocredirect| is a legacy version
% of |\childdocforward| and |\childdocforwardprefix|:
%    \begin{macrocode}
\newcommand{\childdocredirect}[2][]
{
  \begingroup
    \if?#1?
      \def\childdoctmp{\childdocforward{#2}}
    \else
      \def\childdoctmp{\childdocforwardprefix{#1}{#2}}
    \fi
    \expandafter
  \endgroup
  \childdoctmp
}
%    \end{macrocode}

%\iffalse
%</package>
%\fi
%
\endinput

\childdocof{cdocsamp}
%    \end{macrocode}

%\iffalse
%</samplechap1|samplechap2>
%\fi
%
%\iffalse
%<*samplechap1>
%\fi
% Some text for chapter 1:
%    \begin{macrocode}
\section{one}
some text in chapter one
%    \end{macrocode}

%\iffalse
%</samplechap1>
%\fi
% Some text for chapter 2:
%\iffalse
%<*samplechap2>
%\fi
%    \begin{macrocode}
\section{two}
more text in chapter two
%    \end{macrocode}

%\iffalse
%</samplechap2>
%\fi
%
% %%%%%%%%%%%%%%%%%%%%%%%%%%%%%%%%%%%%%%
% \paragraph{Part Include Files.}
%
% The include files are called |cdocspt3.tex| and |cdocspt4.tex|.
%
%\iffalse
%<*samplepart3|samplepart4>
%\fi

% Optional override for |\version| flag:
%    \begin{macrocode}
%%\providecommand{\version}{final}
%    \end{macrocode}

% Include the main document:
%    \begin{macrocode}
% \iffalse
%
% childdoc.dtx Copyright (C) 2017-2018 Niklas Beisert
%
% This work may be distributed and/or modified under the
% conditions of the LaTeX Project Public License, either version 1.3
% of this license or (at your option) any later version.
% The latest version of this license is in
%   http://www.latex-project.org/lppl.txt
% and version 1.3 or later is part of all distributions of LaTeX
% version 2005/12/01 or later.
%
% This work has the LPPL maintenance status `maintained'.
%
% The Current Maintainer of this work is Niklas Beisert.
%
% This work consists of the files childdoc.dtx and childdoc.ins
% and the derived files childdoc.def and cdocsamp.tex with
% cdocsch1.tex, cdocsch2.tex, cdocsdrf.tex, cdocsfn1.tex, cdocsfn2.tex.
%
%<package>\ifdefined\childdocmain\endinput\fi
%<package>\ProvidesFile{childdoc.def}[2018/12/30 v2.0 child document driver]
%<samplemain>\ProvidesFile{cdocsamp.tex}[2018/12/30 v2.0 sample for childdoc]
%<*driver>
%\ProvidesFile{childdoc.drv}[2018/12/30 v2.0 childdoc reference manual file]
\PassOptionsToClass{10pt,a4paper}{article}
\documentclass{ltxdoc}

\usepackage[margin=35mm]{geometry}
\usepackage{hyperref}
\usepackage{hyperxmp}
\usepackage[usenames]{color}

\hypersetup{colorlinks=true}
\hypersetup{pdfstartview=FitH}
\hypersetup{pdfpagemode=UseNone}
\hypersetup{pdfsource={}}
\hypersetup{pdflang={en-UK}}
\hypersetup{pdfcopyright={Copyright 2017-2018 Niklas Beisert.
  This work may be distributed and/or modified under the
  conditions of the LaTeX Project Public License, either version 1.3
  of this license or (at your option) any later version.}}
\hypersetup{pdflicenseurl={http://www.latex-project.org/lppl.txt}}
\hypersetup{pdfcontactaddress={ETH Zurich, ITP, HIT K,
  Wolfgang-Pauli-Strasse 27}}
\hypersetup{pdfcontactpostcode={8093}}
\hypersetup{pdfcontactcity={Zurich}}
\hypersetup{pdfcontactcountry={Switzerland}}
\hypersetup{pdfcontactemail={nbeisert@itp.phys.ethz.ch}}
\hypersetup{pdfcontacturl={http://people.phys.ethz.ch/\xmptilde nbeisert/}}

\newcommand{\secref}[1]{\hyperref[#1]{section \ref*{#1}}}

\parskip1ex
\parindent0pt
\let\olditemize\itemize
\def\itemize{\olditemize\parskip0pt}

\begin{document}

\title{The \textsf{childdoc} Package}
\hypersetup{pdftitle={The childdoc Package}}
\author{Niklas Beisert\\[2ex]
  Institut f\"ur Theoretische Physik\\
  Eidgen\"ossische Technische Hochschule Z\"urich\\
  Wolfgang-Pauli-Strasse 27, 8093 Z\"urich, Switzerland\\[1ex]
  \href{mailto:nbeisert@itp.phys.ethz.ch}
  {\texttt{nbeisert@itp.phys.ethz.ch}}}
\hypersetup{pdfauthor={Niklas Beisert}}
\hypersetup{pdfsubject={Manual for the LaTeX2e Package childdoc}}
\date{30 December 2018, \textsf{v2.0}}
\maketitle

\begin{abstract}\noindent
\textsf{childdoc} is a \LaTeXe{} package
that enables the direct compilation
of document sections included by |\include|
to individual files.
\end{abstract}

\begingroup
\parskip0ex
\tableofcontents
\endgroup

%%%%%%%%%%%%%%%%%%%%%%%%%%%%%%%%%%%%%%%%%%%%%%%%%%%%%%%%%%%%%%%%%%%%%%%%%%%%%%%%
%%%%%%%%%%%%%%%%%%%%%%%%%%%%%%%%%%%%%%%%%%%%%%%%%%%%%%%%%%%%%%%%%%%%%%%%%%%%%%%%
\section{Introduction}

\LaTeX{} provides a mechanism to structure a large document (such as a book)
into a main file and several child files (containing the chapters)
using the |\include| command.
This mechanism is beneficial for documents
which span hundreds of pages in order to
make the source file(s) more manageable.
Moreover, compilation can be restricted to
selected child files by means of the |\includeonly| command.
The latter feature can be used to reduce the compilation time while editing
(this was significantly more useful in the earlier days of \LaTeX{})
or to generate a smaller document which is easier to navigate.
Another application of |\includeonly| is to generate
documents consisting of selected parts of the complete document.

However, there are a few drawbacks of the plain |\include| mechanism:
\begin{itemize}
\item
The child files cannot be compiled on their own,
they can only be compiled via the main file.
A naive editing environment
(such as a text editor with an option
to have the current file processed by \LaTeX)
may require one to switch to the main file before compiling;
attempting to compile the child file produces errors.
\item
The main file must be modified (each time)
to adjust the |\includeonly| command
to the present needs. This easily leaves the main file in a messy state.
\item
The generated document will always carry the filename
of the main document. This is inconvenient if
several child files are to be compiled and
to be kept for distribution.
\end{itemize}

The present package provides a simple interface
to make child files individually compilable by \LaTeX{}.
Compiling a child file then has the same effect as compiling
the main file with an |\includeonly| command
to select the appropriate child.
Moreover the generated document will carry the name of the child
rather than the main file.
This resolves all three above issues.

This feature is meant to make the editing of books,
thesis documents and lecture notes somewhat more convenient.
However, the package can also be used efficiently for
composing a series of documents (such as exercise sheets)
which are typically distributed individually.
It then assists the author in generating the individual documents
(potentially in different versions)
as well as a document containing the collected series.
Another application is in developing style files
or other kinds of included material
where compilation of the style file could redirect
to a sample or test file.

%%%%%%%%%%%%%%%%%%%%%%%%%%%%%%%%%%%%%%%%%%%%%%%%%%%%%%%%%%%%%%%%%%%%%%%%%%%%%%%%
%%%%%%%%%%%%%%%%%%%%%%%%%%%%%%%%%%%%%%%%%%%%%%%%%%%%%%%%%%%%%%%%%%%%%%%%%%%%%%%%
\section{Usage}

First of all, the package \textsf{childdoc} is \emph{not} a standard
\LaTeXe{} |.sty| style file! Therefore it needs to be invoked in
a non-standard way.

%%%%%%%%%%%%%%%%%%%%%%%%%%%%%%%%%%%%%%%%%%%%%%%%%%%%%%%%%%%%%%%%%%%%%%%%%%%%%%%%
\subsection{Included Files}
\label{sec:include}

%%%%%%%%%%%%%%%%%%%%%%%%%%%%%%%%%%%%%%%%
\DescribeMacro{\childdocmain}
To use the package, add the commands
\begin{center}
\begin{tabular}{l}
|\input{childdoc.def}|\\
|\childdocmain{}|\\
\end{tabular}
\end{center}
at the very top of the main \LaTeX{} file,
in particular \emph{before} the |\documentclass| statement!
The argument of |\childdocmain| should be left empty
(but it must be present).

%%%%%%%%%%%%%%%%%%%%%%%%%%%%%%%%%%%%%%%%
\DescribeMacro{\childdocof}
Furthermore, add the commands
\begin{center}
\begin{tabular}{l}
|\input{childdoc.def}|\\
|\childdocof{|\textit{main}|}|\\
\end{tabular}
\end{center}
at the top of every child file \textit{child}
which is included by |\include{|\textit{child}|}|
from within the main file
(or at least for those files to be compiled individually).
The argument \textit{main} must be the filename of the main file.

There are a couple of
considerations in setting up the main and child documents:

%%%%%%%%%%%%%%%%%%%%%%%%%%%%%%%%%%%%%%%%
\paragraph{Restrictions.}

Please note the following restrictions:
\begin{itemize}
\item
|\childdocmain| must be called with one argument \textit{main}
to ensure compatibility with earlier version of the package.
It must either be empty (|\childdocmain{}|)
or precisely match the filename of the main file in which it is specified.
See \secref{sec:detection} for further information.
\item
The filename \textit{main} must be specified without the |.tex| extension.
\item
The filename \textit{main} is case sensitive
(even in case-insensitive file systems)
due to internal string comparison.
\item
The argument \textit{main} should be fully expanded, it cannot be a macro.
\item
Subdirectories and special characters should be avoided in filenames.
\item
The command |\childdocmain{|\textit{main}|}| must be followed by a whitespace.
It should not be followed immediately by another command
or by a comment mark `|%|'.
This is because the \TeX{} parser reads the token immediately following
the argument of |\childdocmain| and puts it
at the beginning of every child section;
however, a white\-space is ignored.
\end{itemize}

%%%%%%%%%%%%%%%%%%%%%%%%%%%%%%%%%%%%%%%%
\paragraph{Content of Main File.}

It is advisable to place all content in the child files included by |\include|.
Any output contained in the main file will appear in all child documents
unless suppressed manually;
it cannot be suppressed automatically by the |\includeonly| directive
and thus should normally be avoided.
A method to include some content in the main file
by means of conditional processing is described in \secref{sec:conditional}.

%%%%%%%%%%%%%%%%%%%%%%%%%%%%%%%%%%%%%%%%
\paragraph{Page Numbering.}

When only a part of the document is compiled,
the appropriate numbering of pages
(as well as other status parameters)
is determined from the |.aux| files.
The latter contain information from previous passes.
However this information needs to propagate through
all intermediate child documents.
Therefore the page numbering in child documents may well
be inconsistent until the complete document is compiled at least once.

A useful (if unconventional) way to always ensure a consistent
page numbering is to restart the numbering in each child document
and denote the pages by `\textit{child}|.|\textit{page}'
where \textit{child} represents the chapter/section number of the child file.
This can be achieved by the command
|\numberwithin{page}{|\textit{child}|}|
of the \textsf{amsmath} package
where \textit{child} can be |chapter| or |section|
depending on the chosen structuring.
Alternatively, one can modify the macro |\thepage| appropriately
and reset the counter |page| at the start of each child file.

%%%%%%%%%%%%%%%%%%%%%%%%%%%%%%%%%%%%%%%%%%%%%%%%%%%%%%%%%%%%%%%%%%%%%%%%%%%%%%%%
\subsection{Conditional Processing}
\label{sec:conditional}

The package provides a mechanism to compile different versions
of a document. To customise the versions further some conditional processing
can come in handy to distinguish which version is being compiled.
The package provides two macros to describe the compilation context:

%%%%%%%%%%%%%%%%%%%%%%%%%%%%%%%%%%%%%%%%
\DescribeMacro{\ifchilddoc}
The conditional |\ifchilddoc| distinguishes between the compilation of
child documents and the main document:
%
\begin{center}
|\ifchilddoc |\textit{child-code}| |[|\||else |\textit{main-code}]| \||fi|
\end{center}

%%%%%%%%%%%%%%%%%%%%%%%%%%%%%%%%%%%%%%%%
\DescribeMacro{\childdocname}
\DescribeMacro{\childdocjob}
The macro |\childdocname| contains the filename (without extension)
of the main or child file being processed.
Note that |\childdocjob| will always contain the name of the main file.

%%%%%%%%%%%%%%%%%%%%%%%%%%%%%%%%%%%%%%%%
\paragraph{Title Page.}

Conditional processing can be used to include a title or banner page
in the main document when proper precautions are taken.
Importantly, the code in the main file should ensure that the page counter
(as well as other status parameters which are stored in the |.aux| files)
takes the same value after the conditional processing.
Otherwise the page numbers may take divergent values
depending on which part is compiled.

For example, a title page could be declared by:
%
\begin{center}
\begin{tabular}{l}
|\ifchilddoc\||else|\\
|\addtocounter{page}{-1}|\\
\textit{code for title page}\\
|\newpage|\\
|\||fi|
\end{tabular}
\end{center}
%
A banner page for the child documents can be generated by:
%
\begin{center}
\begin{tabular}{l}
|\ifchilddoc|\\
|\addtocounter{page}{-1}|\\
\textit{code for banner page}\\
|\newpage|\\
|\||fi|
\end{tabular}
\end{center}
%
Here one could write a message such as:
\begin{center}
|This is the part \childdocname{} of \childdocjob{}.|
\end{center}

%%%%%%%%%%%%%%%%%%%%%%%%%%%%%%%%%%%%%%%%%%%%%%%%%%%%%%%%%%%%%%%%%%%%%%%%%%%%%%%%
\subsection{Flags}
\label{sec:flags}

The package makes it easy to generate different versions
of the main or child documents.
To this end compilation flags can be defined
and assigned different default values.
They will be particularly useful in conjunction
with the forwarding mechanism described in \secref{sec:forward}.

For example, it may be useful to have a flag |\version|
which can be set to |draft| or |final|.
The document source will contain some conditional code
depending on the value of |\version|.
Suppose further, the flag should default to |final| for the main file
and to |draft| for child files
which is a natural assignment for editing the document.
This is achieved by placing the following code
in the preamble of the main document
(below the |\childdocmain| directive):
%
\begin{center}
\begin{tabular}{l}
|\ifchilddoc|\\
|\providecommand{\version}{draft}|\\
|\||else|\\
|\providecommand{\version}{final}|\\
|\||fi|
\end{tabular}
\end{center}
%
The definition by |\providecommand| makes sure
that previous definitions are not overwritten.
Further statements |\providecommand{\version}{...}|
can thus be added before the above code to override it.

For the main file, one might add a line
(between |\childdocmain| and the above block)
%
\begin{center}
|%\ifchilddoc\||else\providecommand{\version}{draft}\||fi|
\end{center}
%
which can be uncommented to produce a draft version.
Likewise one can add a line to the very top of a child file
(above the |\childdocof{|\textit{main}|}| directive)
%
\begin{center}
|%\providecommand{\version}{final}|
\end{center}
%
which can be uncommented to produce the final version of this child document.

%%%%%%%%%%%%%%%%%%%%%%%%%%%%%%%%%%%%%%%%%%%%%%%%%%%%%%%%%%%%%%%%%%%%%%%%%%%%%%%%
\subsection{Forwarding}
\label{sec:forward}

Different versions of the main or child documents
using compilation flags as described in \secref{sec:flags}
can be (permanently) stored in different files
for convenient compilation, viewing and distribution.
To this end, the package defines a command
to pass on compilation to a different file:

%%%%%%%%%%%%%%%%%%%%%%%%%%%%%%%%%%%%%%%%
\DescribeMacro{\childdocforward}
The command |\childdocforward| redirects processing to
another source file:
%
\begin{center}
\begin{tabular}{l}
|\input{childdoc.def}|\\
|\childdocforward[|\textit{main}|]{|\textit{dest}|}|\\
\end{tabular}
\end{center}
%
The argument \textit{dest} is the destination file
(without extension).
It should be the main file or one of the child files.
Note that further \textsf{childdoc} directives
such as |\childdocof| and |\childdocforward|
in the indicated file will be processed in this form.
The optional argument \textit{main}
passes on directly to the main file \textit{main}
while pretending to compile the child \textit{dest}.
This form behaves as if \textit{dest}
issues |\childdocof{|\textit{main}|}| right away,
and no further \textsf{childdoc} directives will be processed.

%%%%%%%%%%%%%%%%%%%%%%%%%%%%%%%%%%%%%%%%
\DescribeMacro{\...prefix}
In the alternative form |\childdocforwardprefix|,
%
\begin{center}
\begin{tabular}{l}
|\input{childdoc.def}|\\
|\childdocforwardprefix[|\textit{main}|]{|\textit{prefix}|}{|\textit{dest}|}|
\end{tabular}
\end{center}
%
the destination file is determined by a pattern
depending on the current file:
To make this work, the current file must be called
`{\textit{prefix}\hspace{0.2em}\textit{suffix}}'
with \textit{prefix} matching precisely the argument.
Processing is then passed on to the file
`{\textit{dest}\hspace{0.2em}\textit{suffix}}'.
Surely, the same effect is achieved by
directly specifying the
argument `{\textit{dest}\hspace{0.2em}\textit{suffix}}'
in the first form.
However, that requires to set up a different file
for each child. With the alternative form of the command
all these files can have exactly the same content
which simplifies setting them up and maintaining them.

For example, the following file |draft.tex|
with a compilation flag |\version| as described in \secref{sec:flags}
compiles the main document as a draft:
%
\begin{center}
\begin{tabular}{l}
|\def\version{draft}|\\
|\input{childdoc.def}|\\
|\childdocforward{|\textit{main}|}|
\end{tabular}
\end{center}
%
Likewise, the following files |final|\textit{nn}|.tex|
compile the final version of the child document
|child|\textit{nn}|.tex|:
%
\begin{center}
\begin{tabular}{l}
|\def\version{final}|\\
|\input{childdoc.def}|\\
|\childdocforwardprefix{final}{child}|
\end{tabular}
\end{center}
%

Note that when several versions of a main file and/or of each child file
are to be generated, it may be convenient to set up a |Makefile| or
shell script to automatise the process.

%%%%%%%%%%%%%%%%%%%%%%%%%%%%%%%%%%%%%%%%%%%%%%%%%%%%%%%%%%%%%%%%%%%%%%%%%%%%%%%%
\subsection{Command Line Processing}
\label{sec:commandline}

The effect of redirection files can also be achieved by invoking
the \LaTeX{} compiler with a more elaborate command line.
Most conveniently this should be done as part
of a shell script or a |Makefile|.

When using \textsf{childdoc} in the main file, the following
command lines effectively perform a redirection
(note that depending on the shell being used,
backslashes may have to be doubled: `|\|' $\to$ `|\\|'):
%
\begin{center}
|... -jobname "|\textit{target}|" |\\|"|[\textit{flags}]%
|\input{childdoc.def}\childdocforward[|\textit{main}|]{|\textit{dest}|}"|
\end{center}
%
Here \textit{target} is the name of the output file,
\textit{main} is the name of the main file
and \textit{dest} is the name of the main or child file to be processed
(all filenames without extensions).
The optional argument \textit{main} can be omitted
if \textit{main} matches \textit{dest}.
Optionally, compilation \textit{flags} can be defined via |\def| commands.
This command line makes the \TeX{} engine believe
it is compiling the file \textit{target}
whose content is specified as the latter parameter.
The provided code then forwards the processing to
\textit{main} or \textit{dest} as described in \secref{sec:forward}.

%%%%%%%%%%%%%%%%%%%%%%%%%%%%%%%%%%%%%%%%%%%%%%%%%%%%%%%%%%%%%%%%%%%%%%%%%%%%%%%%
\subsection{Include by Input}
\label{sec:input}

Including child documents by |\include| has some restrictions by design.
Most notably, the content of a child document always occupies
its own set of pages; pages cannot be shared between child documents.
Usually, this behaviour makes perfect sense
because each child document contain an essential part of the document.
However, in some situations it may be desirable to compose
a document from a collection of parts
without having mandatory page breaks between then.
For this case, the package
provides a mechanism to include parts
by |\input| which can also be processed individually.
However, by construction this mechanism
requires manual handling of the content to be output.

%%%%%%%%%%%%%%%%%%%%%%%%%%%%%%%%%%%%%%%%
\DescribeMacro{\ifchilddocmanual}
The main file should be prepared as usual, see \secref{sec:include}.
However, the document body must make a distinction
between processing of an individual part and of the main document, e.g.:
%
\begin{center}
\begin{tabular}{l}
|\ifchilddocmanual|\\
|\input{\childdocname}|\\
|\||else|\\
\textit{document body with }|\input{|\textit{part}|}|\\
|\||fi|
\end{tabular}
\end{center}
%
The conditional |\ifchilddocmanual| is true whenever
a part to be included by |\input| is being compiled,
and the name of the part is stored in |\childdocname|.

%%%%%%%%%%%%%%%%%%%%%%%%%%%%%%%%%%%%%%%%
\DescribeMacro{\childdocby}
Each part to be included by |\input| should start with:
%
\begin{center}
\begin{tabular}{l}
|\input{childdoc.def}|\\
|\childdocby{|\textit{main}|}|\\
\end{tabular}
\end{center}
%
The directive |\childdocby| is similar to |\childdocof|
described in \secref{sec:include},
but the subsequent selection of content must be done manually.
To that end, both |\ifchilddoc| and |\ifchilddocmanual|
will be true upon processing of a part,
and the name of the part is stored in |\childdocname|.
Note that |\jobname| will be set to the filename of the current part
so that each part receives an individual |.aux| file
that does not interfere with the |.aux| file(s) of the main document.
This behaviour can be altered by the alternative form
|\childdocby[*]{|\textit{main}|}| (with a non-empty optional argument)
which uses the |.aux| file of the main document
by setting |\jobname| to \textit{main}.

%%%%%%%%%%%%%%%%%%%%%%%%%%%%%%%%%%%%%%%%%%%%%%%%%%%%%%%%%%%%%%%%%%%%%%%%%%%%%%%%
\subsection{Driver Development}
\label{sec:driver}

The \textsf{childdoc} mechanism can also be use for the development
of definition files such as \LaTeX{} styles or classes.
This case differs from the above setup with multiple parts
included by |\include| in that no |\includeonly| should be invoked.
This can be achieved by starting the include file
(before |\ProvidesPackage|) with:
%
\begin{center}
\begin{tabular}{l}
|\input{childdoc.def}|\\
|\childdocforward{|\textit{main}|}|\\
\end{tabular}
\end{center}
%
or alternatively with:
%
\begin{center}
\begin{tabular}{l}
|\input{childdoc.def}|\\
|\childdocby{|\textit{main}|}|\\
\end{tabular}
\end{center}
%
Both forms have slightly different effects as described above.
The main file is prepared as usual, see \secref{sec:include}.

%%%%%%%%%%%%%%%%%%%%%%%%%%%%%%%%%%%%%%%%%%%%%%%%%%%%%%%%%%%%%%%%%%%%%%%%%%%%%%%%
\subsection{Legacy Detection}
\label{sec:detection}

The directive |\childdocmain| in the main file can detect
whether the complete document or merely a child is to be compiled
even without using the directive |\childdocof|.
This method is deprecated because it is less robust
and there is no compelling reason to use it;
it is merely provided for backward compatibility
and it may be removed in future versions.

If the detection mechanism is to be used,
it is mandatory to correctly specify
the filename of the main file as the argument of |\childdocmain|:
%
\begin{center}
\begin{tabular}{l}
|\input{childdoc.def}|\\
|\childdocmain{|\textit{main}|}|\\
\end{tabular}
\end{center}
%
If |\jobname| does not match the argument \textit{main} of |\childdocmain|,
it is assumed that |\jobname| points to the child file to be compiled.
When using |\childdocmain| with the main file specified as argument,
it suffices to start a child file
with just |\input{|\textit{main}|}|
without loading of the package and using |\childdocof|.
If instead all processing is done
with the appropriate \textsf{childdoc} directives,
the argument of \textit{main} of |\childdocmain| can be empty.

An alternative version of the command line processing described
in \secref{sec:commandline} using the detection mechanism reads:
%
\begin{center}
|... -jobname "|\textit{target}|" "|[\textit{flags}]%
[|\def\jobname{|\textit{dest}|}|]|\input{|\textit{main}|}"|
\end{center}

%%%%%%%%%%%%%%%%%%%%%%%%%%%%%%%%%%%%%%%%%%%%%%%%%%%%%%%%%%%%%%%%%%%%%%%%%%%%%%%%
\subsection{Manual Code}
\label{sec:manual}

In case one cannot be certain whether the definitions file |childdoc.def|
is installed on the target \TeX{} distribution
and one prefers not to ship it,
it is conceivable to paste a few relevant commands into the sources.

To that end, drop all statements |\input{childdoc.def}|
and perform the replacements as outlined below.
Instead of |\childdocmain{|\textit{main}|}| add the following code
to the top of the main file:
%
\begin{center}
\begin{tabular}{l}
|\||ifdefined\childdocname\endinput\||fi\newif\ifchilddoc|\\
|\edef\childdocname{\scantokens\expandafter{\jobname\noexpand}}|\\
|\def\childdocmain{|\textit{main}|}\||ifx\childdocmain\childdocname\||else|\\
|\childdoctrue\includeonly{\childdocname}\let\jobname\childdocmain\||fi|\\
\end{tabular}
\end{center}
%
Instead of |\childdocof{|\textit{main}|}| just include the main file
at the top of each child file:
%
\begin{center}
|\input{|\textit{main}|}|
\end{center}
%
A simple redirection |\childdocforward{|\textit{dest}|}| is achieved by:
%
\begin{center}
|\def\jobname{|\textit{dest}|}\input{\jobname}|
\end{center}
%
The redirection with prefix
|\childdocforwardprefix[|\textit{prefix}|]{|\textit{dest}|}|
is accomplished by:
%
\begin{center}
\begin{tabular}{l}
|{\edef\jobname{\scantokens\expandafter{\jobname\noexpand}}|\\
|\def\redirectjob |\textit{prefix}|#1~~~{\gdef\jobname{|\textit{dest}|#1}}|\\
|\expandafter\redirectjob\jobname~~~}\input{\jobname}|
\end{tabular}
\end{center}

In an alternative approach,
child documents can be compiled by a specific command line
without additional code or specific definitions:
%
\begin{center}
|... -jobname "|\textit{target}|" "|[\textit{flags}]%
|\includeonly{|\textit{dest}|}\input{|\textit{main}|}"|
\end{center}
%

%%%%%%%%%%%%%%%%%%%%%%%%%%%%%%%%%%%%%%%%%%%%%%%%%%%%%%%%%%%%%%%%%%%%%%%%%%%%%%%%
%%%%%%%%%%%%%%%%%%%%%%%%%%%%%%%%%%%%%%%%%%%%%%%%%%%%%%%%%%%%%%%%%%%%%%%%%%%%%%%%
\section{Information}

%%%%%%%%%%%%%%%%%%%%%%%%%%%%%%%%%%%%%%%%%%%%%%%%%%%%%%%%%%%%%%%%%%%%%%%%%%%%%%%%
\subsection{Copyright}

Copyright \copyright{} 2017--2018 Niklas Beisert

This work may be distributed and/or modified under the
conditions of the \LaTeX{} Project Public License, either version 1.3
of this license or (at your option) any later version.
The latest version of this license is in
  \url{http://www.latex-project.org/lppl.txt}
and version 1.3 or later is part of all distributions of \LaTeX{}
version 2005/12/01 or later.

This work has the LPPL maintenance status `maintained'.

The Current Maintainer of this work is Niklas Beisert.

This work consists of the files |README.txt|, |childdoc.ins| and |childdoc.dtx|
as well as the derived files |childdoc.def|, |cdocsamp.tex|
with |cdocsch1.tex|, |cdocsch2.tex|, |cdocspt3.tex|, |cdocspt4.tex|,
|cdocsdrf.tex|, |cdocsfn1.tex|, |cdocsfn2.tex|
as well as |childdoc.pdf|.

%%%%%%%%%%%%%%%%%%%%%%%%%%%%%%%%%%%%%%%%%%%%%%%%%%%%%%%%%%%%%%%%%%%%%%%%%%%%%%%%
\subsection{Files and Installation}

The package consists of the files:
%
\begin{center}
\begin{tabular}{ll}
    |README.txt|   & readme file \\
    |childdoc.ins| & installation file \\
    |childdoc.dtx| & source file \\
    |childdoc.def| & definition file \\
    |cdocsamp.tex| & sample main file \\
    |cdocsch1.tex| & sample include file \\
    |cdocsch2.tex| & sample include file \\
    |cdocspt3.tex| & sample part file \\
    |cdocspt4.tex| & sample part file \\
    |cdocsdrf.tex| & sample redirection file \\
    |cdocsfn1.tex| & sample redirection file \\
    |cdocsfn2.tex| & sample redirection file \\
    |childdoc.pdf| & manual
\end{tabular}
\end{center}
%
The distribution consists of the files
|README.txt|, |childdoc.ins| and |childdoc.dtx|.
%
\begin{itemize}
\item
Run (pdf)\LaTeX{} on |childdoc.dtx|
to compile the manual |childdoc.pdf| (this file).
\item
Run \LaTeX{} on |childdoc.ins| to create the definitions file |childdoc.def|
and the sample |cdocsamp.tex| with include files
|cdocsch1.tex|, |cdocsch2.tex|, |cdocspt3.tex|, |cdocspt4.tex|,
|cdocsdrf.tex|, |cdocsfn1.tex|, |cdocsfn2.tex|.
Then copy the file |childdoc.def| to an appropriate directory of your \LaTeX{}
distribution, e.g.\ \textit{texmf-root}|/tex/latex/childdoc|.
\end{itemize}

%%%%%%%%%%%%%%%%%%%%%%%%%%%%%%%%%%%%%%%%%%%%%%%%%%%%%%%%%%%%%%%%%%%%%%%%%%%%%%%%
\subsection{Related CTAN Packages}

There are several other packages which offer a similar functionality:
%
\begin{itemize}
\item
The packages
\href{http://ctan.org/pkg/docmute}{\textsf{docmute}},
\href{http://ctan.org/pkg/includex}{\textsf{includex}} and
\href{http://ctan.org/pkg/standalone}{\textsf{standalone}}
provide commands to include only the document body of
a child file thus allowing both files to be compiled individually.
\item
The packages \href{http://ctan.org/pkg/subdocs}{\textsf{subdocs}}
and \href{http://ctan.org/pkg/subfiles}{\textsf{subfiles}}
provide structures in which the main and child documents can be
encapsulated and allowing them to be compiled individually.
The inclusion mechanism is different from the conventional |\include|.
\item
The package \href{http://ctan.org/pkg/combine}{\textsf{combine}}
is an elaborate solution to combine several documents into one.
\end{itemize}
%
See also the CTAN topic \href{http://ctan.org/topic/subdocs}{\textsf{subdocs}}
for further related packages.
The present package differs from the above solutions in that
a document structure constructed with the conventional |\include| mechanism
just needs two extra commands at the top of every file
such that all constituent files can be compiled individually.

%%%%%%%%%%%%%%%%%%%%%%%%%%%%%%%%%%%%%%%%%%%%%%%%%%%%%%%%%%%%%%%%%%%%%%%%%%%%%%%%
%\subsection{Feature Suggestions}
%
%The following is a list of features which may be useful for future
%versions of this package:
%%
%\begin{itemize}
%\item
%\ldots
%\end{itemize}

%%%%%%%%%%%%%%%%%%%%%%%%%%%%%%%%%%%%%%%%%%%%%%%%%%%%%%%%%%%%%%%%%%%%%%%%%%%%%%%%
\subsection{Revision History}

%%%%%%%%%%%%%%%%%%%%%%%%%%%%%%%%%%%%%%%%
\paragraph{v2.0:} 2018/12/30

\begin{itemize}
\item
immediate forward processing
\item
added |\childdocby| mechanism
\item
manual restructured
\end{itemize}

%%%%%%%%%%%%%%%%%%%%%%%%%%%%%%%%%%%%%%%%
\paragraph{v1.6:} 2018/01/17

\begin{itemize}
\item
application for development of include files
\item
corrections to manual
\end{itemize}

%%%%%%%%%%%%%%%%%%%%%%%%%%%%%%%%%%%%%%%%
\paragraph{v1.5:} 2017/05/21

\begin{itemize}
\item
more complete structuring introduced
\item
|\childdocof| introduced
\item
|\childdoc| renamed to |\childdocmain|
\item
|\childredirect| renamed to |\childdocforward| and |\childdocforwardprefix|
and functionality expanded
\end{itemize}

%%%%%%%%%%%%%%%%%%%%%%%%%%%%%%%%%%%%%%%%
\paragraph{v1.0:} 2017/04/27

\begin{itemize}
\item
manual and install package
\item
first version published on CTAN
\end{itemize}

%%%%%%%%%%%%%%%%%%%%%%%%%%%%%%%%%%%%%%%%
\paragraph{v0.6:} 2017/04/26

\begin{itemize}
\item
redirection mechanism added
\end{itemize}

%%%%%%%%%%%%%%%%%%%%%%%%%%%%%%%%%%%%%%%%
\paragraph{v0.5:} 2017/04/26

\begin{itemize}
\item
functionality in definition file
\end{itemize}


%%%%%%%%%%%%%%%%%%%%%%%%%%%%%%%%%%%%%%%%%%%%%%%%%%%%%%%%%%%%%%%%%%%%%%%%%%%%%%%%
%%%%%%%%%%%%%%%%%%%%%%%%%%%%%%%%%%%%%%%%%%%%%%%%%%%%%%%%%%%%%%%%%%%%%%%%%%%%%%%%
%%%%%%%%%%%%%%%%%%%%%%%%%%%%%%%%%%%%%%%%%%%%%%%%%%%%%%%%%%%%%%%%%%%%%%%%%%%%%%%%
\appendix

\settowidth\MacroIndent{\rmfamily\scriptsize 000\ }

 \DocInput{childdoc.dtx}

\end{document}
%</driver>
% \fi
%
% %%%%%%%%%%%%%%%%%%%%%%%%%%%%%%%%%%%%%%%%%%%%%%%%%%%%%%%%%%%%%%%%%%%%%%%%%%%%%%
% %%%%%%%%%%%%%%%%%%%%%%%%%%%%%%%%%%%%%%%%%%%%%%%%%%%%%%%%%%%%%%%%%%%%%%%%%%%%%%
% \section{Sample}
%\iffalse
%<*samplemain>
%\fi
%
% The following presents a sample document
% with two chapters, two parts, a title page,
% a compile flag as well as three forwarding files to set the flag.
% It consists of eight |.tex| files:
% \begin{center}
% \begin{tabular}{ll}
% |cdocsamp.tex|&main file\\
% |cdocsch1.tex|&include file for chapter 1\\
% |cdocsch2.tex|&include file for chapter 2\\
% |cdocspt3.tex|&include file for part 3\\
% |cdocspt4.tex|&include file for part 4\\
% |cdocsdrf.tex|&forwarding file for main file in draft mode\\
% |cdocsfi1.tex|&forwarding file for final version of chapter 1\\
% |cdocsfi2.tex|&forwarding file for final version of chapter 2\\
% \end{tabular}
% \end{center}
% Each of the eight files can be compiled directly by the \LaTeX{} compiler.
%
% %%%%%%%%%%%%%%%%%%%%%%%%%%%%%%%%%%%%%%
% \paragraph{Main File.}
%
% The main file is called |cdocsamp.tex|.
%
% Load the \textsf{childdoc} definitions and
% declare the filename for the main document:
%    \begin{macrocode}
\input{childdoc.def}
\childdocmain{}
%    \end{macrocode}

% Optional override for |\version| flag:
%    \begin{macrocode}
%%\ifchilddoc\else\providecommand{\version}{draft}\fi
%    \end{macrocode}

% Define the default values for the |\version| flag
% (|final| for the main file and |draft| for childs):
%    \begin{macrocode}
\ifchilddoc
\providecommand{\version}{draft}
\else
\providecommand{\version}{final}
\fi
%    \end{macrocode}

% Load the standard document class:
%    \begin{macrocode}
\documentclass[12pt]{article}
%    \end{macrocode}

% Start the document body:
%    \begin{macrocode}
\begin{document}
%    \end{macrocode}

% Declare a title page.
% Print title, part of document being processed and version flag:
%    \begin{macrocode}
\addtocounter{page}{-1}
\begin{center}
{\LARGE\bfseries{}childdoc example\par}
\vspace{1cm}
\ifchilddoc
\ifchilddocmanual part\else chapter\fi:
`\childdocname' of `\childdocjob'\par
\else
main document: `\childdocjob'\par
\fi
version: \version\par
\end{center}
\newpage
%    \end{macrocode}

% Manually include selected file,
% otherwise process as usual:
%    \begin{macrocode}
\ifchilddocmanual
\section*{part `\childdocname'}
\input{\childdocname}
\else
%    \end{macrocode}

% Include the two chapters:
%    \begin{macrocode}
\include{cdocsch1}
\include{cdocsch2}
%    \end{macrocode}

% Include the two parts unless only chapters should be displayed:
%    \begin{macrocode}
\ifchilddoc\else
\section{part three}
\input{cdocspt3}
\section{part four}
\input{cdocspt4}
\fi
%    \end{macrocode}

% Process as usual until here:
%    \begin{macrocode}
\fi
%    \end{macrocode}

% End of document body:
%    \begin{macrocode}
\end{document}
%    \end{macrocode}
%\iffalse
%</samplemain>
%\fi
%
% %%%%%%%%%%%%%%%%%%%%%%%%%%%%%%%%%%%%%%
% \paragraph{Chapter Include Files.}
%
% The include files are called |cdocsch1.tex| and |cdocsch2.tex|.
%
%\iffalse
%<*samplechap1|samplechap2>
%\fi

% Optional override for |\version| flag:
%    \begin{macrocode}
%%\providecommand{\version}{final}
%    \end{macrocode}

% Include the main document:
%    \begin{macrocode}
\input{childdoc.def}
\childdocof{cdocsamp}
%    \end{macrocode}

%\iffalse
%</samplechap1|samplechap2>
%\fi
%
%\iffalse
%<*samplechap1>
%\fi
% Some text for chapter 1:
%    \begin{macrocode}
\section{one}
some text in chapter one
%    \end{macrocode}

%\iffalse
%</samplechap1>
%\fi
% Some text for chapter 2:
%\iffalse
%<*samplechap2>
%\fi
%    \begin{macrocode}
\section{two}
more text in chapter two
%    \end{macrocode}

%\iffalse
%</samplechap2>
%\fi
%
% %%%%%%%%%%%%%%%%%%%%%%%%%%%%%%%%%%%%%%
% \paragraph{Part Include Files.}
%
% The include files are called |cdocspt3.tex| and |cdocspt4.tex|.
%
%\iffalse
%<*samplepart3|samplepart4>
%\fi

% Optional override for |\version| flag:
%    \begin{macrocode}
%%\providecommand{\version}{final}
%    \end{macrocode}

% Include the main document:
%    \begin{macrocode}
\input{childdoc.def}
\childdocby{cdocsamp}
%    \end{macrocode}

%\iffalse
%</samplepart3|samplepart4>
%\fi
%
%\iffalse
%<*samplepart3>
%\fi
% Some text for part 3:
%    \begin{macrocode}
some text in part three
%    \end{macrocode}

%\iffalse
%</samplepart3>
%\fi
% Some text for part 4:
%\iffalse
%<*samplepart4>
%\fi
%    \begin{macrocode}
more text in part four
%    \end{macrocode}

%\iffalse
%</samplepart4>
%\fi
%
% %%%%%%%%%%%%%%%%%%%%%%%%%%%%%%%%%%%%%%
% \paragraph{Forwarding for a Complete Draft.}
%
% The following forwarding file |cdocsdrf.tex|
% compiles the main document in draft mode:
%\iffalse
%<*sampledraft>
%\fi
%    \begin{macrocode}
\def\version{draft}
\input{childdoc.def}
\childdocforward{cdocsamp}
%    \end{macrocode}

%\iffalse
%</sampledraft>
%\fi
%
% %%%%%%%%%%%%%%%%%%%%%%%%%%%%%%%%%%%%%%
% \paragraph{Forwarding for Final Version of the Chapters.}
%
% The following forwarding files |cdocsfn1.tex| and |cdocsfn2.tex|
% (with identical content)
% compile the final versions of the child documents
% |cdocsch1.tex| and |cdocsch2.tex|, respectively:
%\iffalse
%<*samplefinal>
%\fi
%    \begin{macrocode}
\def\version{final}
\input{childdoc.def}
\childdocforwardprefix[cdocsamp]{cdocsfn}{cdocsch}
%    \end{macrocode}

%\iffalse
%</samplefinal>
%\fi
%
% %%%%%%%%%%%%%%%%%%%%%%%%%%%%%%%%%%%%%%
% \paragraph{Command Line Processing.}
%
% The following three command lines generate the output files
% |cdocscld|, |cdocscl1| and |cdocscl2|
% which should be identical to
% |cdocsdrf|, |cdocsch1| and |cdocsfn2|, respectively:
% \begin{center}
% \begin{tabular}{l}
% |latex -jobname cdocscld \|\\
% |  "\def\version{draft}\input{childdoc.def}\childdocforward{cdocsamp}"|\\
% |latex -jobname cdocscl1 \|\\
% |  "\input{childdoc.def}\childdocforward[cdocsamp]{cdocsch1}"|\\
% |latex -jobname cdocscl2 \|\\
% |  "\def\version{final}\input{childdoc.def}\childdocforward{cdocsch2}"|
% \end{tabular}
% \end{center}
% Note that the trailing backslash on each first line
% merely continues the input to the second line
% (for convenient cut ant paste).
% Furthermore, the command |latex| can be replaced by any
% of its alternative versions such as |pdflatex|.
%
% %%%%%%%%%%%%%%%%%%%%%%%%%%%%%%%%%%%%%%%%%%%%%%%%%%%%%%%%%%%%%%%%%%%%%%%%%%%%%%
% %%%%%%%%%%%%%%%%%%%%%%%%%%%%%%%%%%%%%%%%%%%%%%%%%%%%%%%%%%%%%%%%%%%%%%%%%%%%%%
% \section{Implementation}
%\iffalse
%<*package>
%\fi
%
% This section describes the definitions file |childdoc.def|.

% The definitions cannot be loaded using |\usepackage| or |\RequirePackage|
% which has a mechanism to prevent loading a style file more than once.
% When loading the definitions by means of |\input|
% multiple instances have to be prevented manually:
%\iffalse
%This code needs to be before the `\ProvidesFile' directive
%which is defined at the beginning of this file.
%Therefore it is also placed there and commented out here.
%</package>
%<*discard>
%\fi
%    \begin{macrocode}
\ifdefined\childdocmain\endinput\fi
%    \end{macrocode}
%\iffalse
%</discard>
%<*package>
%\fi
%
% \macro{\ifchilddoc}
% \macro{\ifchilddocmanual}
% The conditional |\ifchilddoc| tells whether a
% child (true) or main (false) document is being compiled.
% The conditional |\ifchilddocmanual| tells whether
% the |\includeonly| mechanism is used (false) or
% the selection of child files must be performed manually (true).
% The definitions initialise to false:
%    \begin{macrocode}
\newif\ifchilddoc
\newif\ifchilddocmanual
%    \end{macrocode}

% \macro{\childdocname}
% \macro{\childdocjob}
% The macro |\childdocname| stores the name of the main document
% to be compiled. The macro |\childdocjob| stores the name of
% the document on which the \LaTeX{} compiler was originally invoked.
% The content of |\jobname| cannot be compared
% to filenames specified in the source due to different catcodes.
% The following code rescans |\jobname|, stores the result
% in |\childdocname| and saves a copy in |\childdocjob|:
%    \begin{macrocode}
\edef\childdocname{\scantokens\expandafter{\jobname\noexpand}}
\let\childdocjob\childdocname
%    \end{macrocode}

% \macro{\childdocdisable}
% The macro |\childdocdisable| prevents the main file
% from being processed more than once.
% At this stage, the main document command |\childdocmain|
% is assumed to be called once again where it should do nothing.
% Any subsequent call to it should prevent
% a secondary processing of the main document
% It overwrites the forwarding commands
% |\childdocof| and |\childdocforward|
% with empty macros to prevent further inclusions of the main document:
%    \begin{macrocode}
\newcommand{\childdocdisable}
{
  \renewcommand{\childdocmain}[1]{\renewcommand{\childdocmain}[1]{\endinput}}
  \renewcommand{\childdocof}[1]{}
  \renewcommand{\childdocby}[2][]{}
  \renewcommand{\childdocforward}[2][]{}
  \renewcommand{\childdocdisable}{}
}
%    \end{macrocode}

% \macro{\childdocmain}
% The macro |\childdocmain| is to be called at the top of the main file
% with nothing or the main filename (without extension) as argument.
% First, it breaks loops.
% If the argument is not empty and does not match |\childdocname|
% (which is set by the first inclusion of |childdoc.def|),
% |\ifchilddoc| is set to true, |\includeonly| is applied to the child file
% and |\jobname| is set to the main file
% (for proper handling of |.aux| files):
%    \begin{macrocode}
\newcommand{\childdocmain}[1]
{
  \childdocdisable\childdocmain{}
  \if?#1?\else
    \begingroup
      \def\childdoctmp{#1}
      \ifx\childdoctmp\childdocname
        \def\childdoctmp{}
      \else
        \def\childdoctmp
        {
          \childdoctrue
          \includeonly{\childdocname}
          \def\childdocjob{#1}
          \def\jobname{#1}
        }
      \fi
      \expandafter
    \endgroup
    \childdoctmp
  \fi
}
%    \end{macrocode}

% \macro{\childdocof}
% The command |\childdocof| redirects
% compilation to the main file |#1|.
%    \begin{macrocode}
\newcommand{\childdocof}[1]
{
  \childdocdisable
  \childdoctrue
  \includeonly{\childdocname}
  \def\jobname{#1}
  \def\childdocjob{#1}
  \input{#1}
}
%    \end{macrocode}

% \macro{\childdocby}
% The command |\childdocby| ....
%    \begin{macrocode}
\newcommand{\childdocby}[2][]
{
  \childdocdisable
  \childdoctrue
  \childdocmanualtrue
  \if?#1?\else
    \def\jobname{#2}
  \fi
  \def\childdocjob{#2}
  \input{#2}
  \endinput
}
%    \end{macrocode}

% \macro{\childdocforward}
% The command |\childdocforward| redirects
% compilation to the main file or
% (if the optional argument is given) a child file.
% Parameters are set as if the main file
% or a child file starting with |\childdocof| was compiled.
% Then compilation is handed over to the main file:
%    \begin{macrocode}
\newcommand{\childdocforward}[2][]
{
  \begingroup
    \if?#1?
      \def\childdoctmp
      {
        \def\childdocname{#2}
        \def\childdocjob{#2}
        \def\jobname{#2}
        \input{#2}
        \endinput
      }
    \else
      \def\childdoctmp
      {
        \childdocdisable
        \def\childdocname{#2}
        \childdoctrue
        \includeonly{#2}
        \def\childdocjob{#1}
        \def\jobname{#1}
        \input{#1}
        \endinput
      }
    \fi
    \expandafter
  \endgroup
  \childdoctmp
}
%    \end{macrocode}

% \macro{\childdocforwardprefix}
% The command |\childdocforwardprefix| redirects
% compilation to the main or a child file by means of a pattern.
% The prefix |#1| in the current filename is replaced by |#2|
% and the suffix of the current filename is kept
% (it is assumed that the filename does not contain the substring `|~~~|'
% which is used as a delimiter).
% Compilation is handed over to the new file by |\childdocforward|:
%    \begin{macrocode}
\newcommand{\childdocforwardprefix}[3][]
{
  \begingroup
    \def\childdocextract #2##1~~~{\def\childdoctmp{\childdocforward[#1]{#3##1}}}
    \expandafter\childdocextract\childdocname~~~
    \expandafter
  \endgroup
  \childdoctmp
}
%    \end{macrocode}

% \macro{\childdoc}
% The deprecated macro |\childdoc| is a legacy version of |\childdocmain|:
%    \begin{macrocode}
\newcommand{\childdoc}{\childdocmain}
%    \end{macrocode}

% \macro{\childdocredirect}
% The deprecated macro |\childdocredirect| is a legacy version
% of |\childdocforward| and |\childdocforwardprefix|:
%    \begin{macrocode}
\newcommand{\childdocredirect}[2][]
{
  \begingroup
    \if?#1?
      \def\childdoctmp{\childdocforward{#2}}
    \else
      \def\childdoctmp{\childdocforwardprefix{#1}{#2}}
    \fi
    \expandafter
  \endgroup
  \childdoctmp
}
%    \end{macrocode}

%\iffalse
%</package>
%\fi
%
\endinput

\childdocby{cdocsamp}
%    \end{macrocode}

%\iffalse
%</samplepart3|samplepart4>
%\fi
%
%\iffalse
%<*samplepart3>
%\fi
% Some text for part 3:
%    \begin{macrocode}
some text in part three
%    \end{macrocode}

%\iffalse
%</samplepart3>
%\fi
% Some text for part 4:
%\iffalse
%<*samplepart4>
%\fi
%    \begin{macrocode}
more text in part four
%    \end{macrocode}

%\iffalse
%</samplepart4>
%\fi
%
% %%%%%%%%%%%%%%%%%%%%%%%%%%%%%%%%%%%%%%
% \paragraph{Forwarding for a Complete Draft.}
%
% The following forwarding file |cdocsdrf.tex|
% compiles the main document in draft mode:
%\iffalse
%<*sampledraft>
%\fi
%    \begin{macrocode}
\def\version{draft}
% \iffalse
%
% childdoc.dtx Copyright (C) 2017-2018 Niklas Beisert
%
% This work may be distributed and/or modified under the
% conditions of the LaTeX Project Public License, either version 1.3
% of this license or (at your option) any later version.
% The latest version of this license is in
%   http://www.latex-project.org/lppl.txt
% and version 1.3 or later is part of all distributions of LaTeX
% version 2005/12/01 or later.
%
% This work has the LPPL maintenance status `maintained'.
%
% The Current Maintainer of this work is Niklas Beisert.
%
% This work consists of the files childdoc.dtx and childdoc.ins
% and the derived files childdoc.def and cdocsamp.tex with
% cdocsch1.tex, cdocsch2.tex, cdocsdrf.tex, cdocsfn1.tex, cdocsfn2.tex.
%
%<package>\ifdefined\childdocmain\endinput\fi
%<package>\ProvidesFile{childdoc.def}[2018/12/30 v2.0 child document driver]
%<samplemain>\ProvidesFile{cdocsamp.tex}[2018/12/30 v2.0 sample for childdoc]
%<*driver>
%\ProvidesFile{childdoc.drv}[2018/12/30 v2.0 childdoc reference manual file]
\PassOptionsToClass{10pt,a4paper}{article}
\documentclass{ltxdoc}

\usepackage[margin=35mm]{geometry}
\usepackage{hyperref}
\usepackage{hyperxmp}
\usepackage[usenames]{color}

\hypersetup{colorlinks=true}
\hypersetup{pdfstartview=FitH}
\hypersetup{pdfpagemode=UseNone}
\hypersetup{pdfsource={}}
\hypersetup{pdflang={en-UK}}
\hypersetup{pdfcopyright={Copyright 2017-2018 Niklas Beisert.
  This work may be distributed and/or modified under the
  conditions of the LaTeX Project Public License, either version 1.3
  of this license or (at your option) any later version.}}
\hypersetup{pdflicenseurl={http://www.latex-project.org/lppl.txt}}
\hypersetup{pdfcontactaddress={ETH Zurich, ITP, HIT K,
  Wolfgang-Pauli-Strasse 27}}
\hypersetup{pdfcontactpostcode={8093}}
\hypersetup{pdfcontactcity={Zurich}}
\hypersetup{pdfcontactcountry={Switzerland}}
\hypersetup{pdfcontactemail={nbeisert@itp.phys.ethz.ch}}
\hypersetup{pdfcontacturl={http://people.phys.ethz.ch/\xmptilde nbeisert/}}

\newcommand{\secref}[1]{\hyperref[#1]{section \ref*{#1}}}

\parskip1ex
\parindent0pt
\let\olditemize\itemize
\def\itemize{\olditemize\parskip0pt}

\begin{document}

\title{The \textsf{childdoc} Package}
\hypersetup{pdftitle={The childdoc Package}}
\author{Niklas Beisert\\[2ex]
  Institut f\"ur Theoretische Physik\\
  Eidgen\"ossische Technische Hochschule Z\"urich\\
  Wolfgang-Pauli-Strasse 27, 8093 Z\"urich, Switzerland\\[1ex]
  \href{mailto:nbeisert@itp.phys.ethz.ch}
  {\texttt{nbeisert@itp.phys.ethz.ch}}}
\hypersetup{pdfauthor={Niklas Beisert}}
\hypersetup{pdfsubject={Manual for the LaTeX2e Package childdoc}}
\date{30 December 2018, \textsf{v2.0}}
\maketitle

\begin{abstract}\noindent
\textsf{childdoc} is a \LaTeXe{} package
that enables the direct compilation
of document sections included by |\include|
to individual files.
\end{abstract}

\begingroup
\parskip0ex
\tableofcontents
\endgroup

%%%%%%%%%%%%%%%%%%%%%%%%%%%%%%%%%%%%%%%%%%%%%%%%%%%%%%%%%%%%%%%%%%%%%%%%%%%%%%%%
%%%%%%%%%%%%%%%%%%%%%%%%%%%%%%%%%%%%%%%%%%%%%%%%%%%%%%%%%%%%%%%%%%%%%%%%%%%%%%%%
\section{Introduction}

\LaTeX{} provides a mechanism to structure a large document (such as a book)
into a main file and several child files (containing the chapters)
using the |\include| command.
This mechanism is beneficial for documents
which span hundreds of pages in order to
make the source file(s) more manageable.
Moreover, compilation can be restricted to
selected child files by means of the |\includeonly| command.
The latter feature can be used to reduce the compilation time while editing
(this was significantly more useful in the earlier days of \LaTeX{})
or to generate a smaller document which is easier to navigate.
Another application of |\includeonly| is to generate
documents consisting of selected parts of the complete document.

However, there are a few drawbacks of the plain |\include| mechanism:
\begin{itemize}
\item
The child files cannot be compiled on their own,
they can only be compiled via the main file.
A naive editing environment
(such as a text editor with an option
to have the current file processed by \LaTeX)
may require one to switch to the main file before compiling;
attempting to compile the child file produces errors.
\item
The main file must be modified (each time)
to adjust the |\includeonly| command
to the present needs. This easily leaves the main file in a messy state.
\item
The generated document will always carry the filename
of the main document. This is inconvenient if
several child files are to be compiled and
to be kept for distribution.
\end{itemize}

The present package provides a simple interface
to make child files individually compilable by \LaTeX{}.
Compiling a child file then has the same effect as compiling
the main file with an |\includeonly| command
to select the appropriate child.
Moreover the generated document will carry the name of the child
rather than the main file.
This resolves all three above issues.

This feature is meant to make the editing of books,
thesis documents and lecture notes somewhat more convenient.
However, the package can also be used efficiently for
composing a series of documents (such as exercise sheets)
which are typically distributed individually.
It then assists the author in generating the individual documents
(potentially in different versions)
as well as a document containing the collected series.
Another application is in developing style files
or other kinds of included material
where compilation of the style file could redirect
to a sample or test file.

%%%%%%%%%%%%%%%%%%%%%%%%%%%%%%%%%%%%%%%%%%%%%%%%%%%%%%%%%%%%%%%%%%%%%%%%%%%%%%%%
%%%%%%%%%%%%%%%%%%%%%%%%%%%%%%%%%%%%%%%%%%%%%%%%%%%%%%%%%%%%%%%%%%%%%%%%%%%%%%%%
\section{Usage}

First of all, the package \textsf{childdoc} is \emph{not} a standard
\LaTeXe{} |.sty| style file! Therefore it needs to be invoked in
a non-standard way.

%%%%%%%%%%%%%%%%%%%%%%%%%%%%%%%%%%%%%%%%%%%%%%%%%%%%%%%%%%%%%%%%%%%%%%%%%%%%%%%%
\subsection{Included Files}
\label{sec:include}

%%%%%%%%%%%%%%%%%%%%%%%%%%%%%%%%%%%%%%%%
\DescribeMacro{\childdocmain}
To use the package, add the commands
\begin{center}
\begin{tabular}{l}
|\input{childdoc.def}|\\
|\childdocmain{}|\\
\end{tabular}
\end{center}
at the very top of the main \LaTeX{} file,
in particular \emph{before} the |\documentclass| statement!
The argument of |\childdocmain| should be left empty
(but it must be present).

%%%%%%%%%%%%%%%%%%%%%%%%%%%%%%%%%%%%%%%%
\DescribeMacro{\childdocof}
Furthermore, add the commands
\begin{center}
\begin{tabular}{l}
|\input{childdoc.def}|\\
|\childdocof{|\textit{main}|}|\\
\end{tabular}
\end{center}
at the top of every child file \textit{child}
which is included by |\include{|\textit{child}|}|
from within the main file
(or at least for those files to be compiled individually).
The argument \textit{main} must be the filename of the main file.

There are a couple of
considerations in setting up the main and child documents:

%%%%%%%%%%%%%%%%%%%%%%%%%%%%%%%%%%%%%%%%
\paragraph{Restrictions.}

Please note the following restrictions:
\begin{itemize}
\item
|\childdocmain| must be called with one argument \textit{main}
to ensure compatibility with earlier version of the package.
It must either be empty (|\childdocmain{}|)
or precisely match the filename of the main file in which it is specified.
See \secref{sec:detection} for further information.
\item
The filename \textit{main} must be specified without the |.tex| extension.
\item
The filename \textit{main} is case sensitive
(even in case-insensitive file systems)
due to internal string comparison.
\item
The argument \textit{main} should be fully expanded, it cannot be a macro.
\item
Subdirectories and special characters should be avoided in filenames.
\item
The command |\childdocmain{|\textit{main}|}| must be followed by a whitespace.
It should not be followed immediately by another command
or by a comment mark `|%|'.
This is because the \TeX{} parser reads the token immediately following
the argument of |\childdocmain| and puts it
at the beginning of every child section;
however, a white\-space is ignored.
\end{itemize}

%%%%%%%%%%%%%%%%%%%%%%%%%%%%%%%%%%%%%%%%
\paragraph{Content of Main File.}

It is advisable to place all content in the child files included by |\include|.
Any output contained in the main file will appear in all child documents
unless suppressed manually;
it cannot be suppressed automatically by the |\includeonly| directive
and thus should normally be avoided.
A method to include some content in the main file
by means of conditional processing is described in \secref{sec:conditional}.

%%%%%%%%%%%%%%%%%%%%%%%%%%%%%%%%%%%%%%%%
\paragraph{Page Numbering.}

When only a part of the document is compiled,
the appropriate numbering of pages
(as well as other status parameters)
is determined from the |.aux| files.
The latter contain information from previous passes.
However this information needs to propagate through
all intermediate child documents.
Therefore the page numbering in child documents may well
be inconsistent until the complete document is compiled at least once.

A useful (if unconventional) way to always ensure a consistent
page numbering is to restart the numbering in each child document
and denote the pages by `\textit{child}|.|\textit{page}'
where \textit{child} represents the chapter/section number of the child file.
This can be achieved by the command
|\numberwithin{page}{|\textit{child}|}|
of the \textsf{amsmath} package
where \textit{child} can be |chapter| or |section|
depending on the chosen structuring.
Alternatively, one can modify the macro |\thepage| appropriately
and reset the counter |page| at the start of each child file.

%%%%%%%%%%%%%%%%%%%%%%%%%%%%%%%%%%%%%%%%%%%%%%%%%%%%%%%%%%%%%%%%%%%%%%%%%%%%%%%%
\subsection{Conditional Processing}
\label{sec:conditional}

The package provides a mechanism to compile different versions
of a document. To customise the versions further some conditional processing
can come in handy to distinguish which version is being compiled.
The package provides two macros to describe the compilation context:

%%%%%%%%%%%%%%%%%%%%%%%%%%%%%%%%%%%%%%%%
\DescribeMacro{\ifchilddoc}
The conditional |\ifchilddoc| distinguishes between the compilation of
child documents and the main document:
%
\begin{center}
|\ifchilddoc |\textit{child-code}| |[|\||else |\textit{main-code}]| \||fi|
\end{center}

%%%%%%%%%%%%%%%%%%%%%%%%%%%%%%%%%%%%%%%%
\DescribeMacro{\childdocname}
\DescribeMacro{\childdocjob}
The macro |\childdocname| contains the filename (without extension)
of the main or child file being processed.
Note that |\childdocjob| will always contain the name of the main file.

%%%%%%%%%%%%%%%%%%%%%%%%%%%%%%%%%%%%%%%%
\paragraph{Title Page.}

Conditional processing can be used to include a title or banner page
in the main document when proper precautions are taken.
Importantly, the code in the main file should ensure that the page counter
(as well as other status parameters which are stored in the |.aux| files)
takes the same value after the conditional processing.
Otherwise the page numbers may take divergent values
depending on which part is compiled.

For example, a title page could be declared by:
%
\begin{center}
\begin{tabular}{l}
|\ifchilddoc\||else|\\
|\addtocounter{page}{-1}|\\
\textit{code for title page}\\
|\newpage|\\
|\||fi|
\end{tabular}
\end{center}
%
A banner page for the child documents can be generated by:
%
\begin{center}
\begin{tabular}{l}
|\ifchilddoc|\\
|\addtocounter{page}{-1}|\\
\textit{code for banner page}\\
|\newpage|\\
|\||fi|
\end{tabular}
\end{center}
%
Here one could write a message such as:
\begin{center}
|This is the part \childdocname{} of \childdocjob{}.|
\end{center}

%%%%%%%%%%%%%%%%%%%%%%%%%%%%%%%%%%%%%%%%%%%%%%%%%%%%%%%%%%%%%%%%%%%%%%%%%%%%%%%%
\subsection{Flags}
\label{sec:flags}

The package makes it easy to generate different versions
of the main or child documents.
To this end compilation flags can be defined
and assigned different default values.
They will be particularly useful in conjunction
with the forwarding mechanism described in \secref{sec:forward}.

For example, it may be useful to have a flag |\version|
which can be set to |draft| or |final|.
The document source will contain some conditional code
depending on the value of |\version|.
Suppose further, the flag should default to |final| for the main file
and to |draft| for child files
which is a natural assignment for editing the document.
This is achieved by placing the following code
in the preamble of the main document
(below the |\childdocmain| directive):
%
\begin{center}
\begin{tabular}{l}
|\ifchilddoc|\\
|\providecommand{\version}{draft}|\\
|\||else|\\
|\providecommand{\version}{final}|\\
|\||fi|
\end{tabular}
\end{center}
%
The definition by |\providecommand| makes sure
that previous definitions are not overwritten.
Further statements |\providecommand{\version}{...}|
can thus be added before the above code to override it.

For the main file, one might add a line
(between |\childdocmain| and the above block)
%
\begin{center}
|%\ifchilddoc\||else\providecommand{\version}{draft}\||fi|
\end{center}
%
which can be uncommented to produce a draft version.
Likewise one can add a line to the very top of a child file
(above the |\childdocof{|\textit{main}|}| directive)
%
\begin{center}
|%\providecommand{\version}{final}|
\end{center}
%
which can be uncommented to produce the final version of this child document.

%%%%%%%%%%%%%%%%%%%%%%%%%%%%%%%%%%%%%%%%%%%%%%%%%%%%%%%%%%%%%%%%%%%%%%%%%%%%%%%%
\subsection{Forwarding}
\label{sec:forward}

Different versions of the main or child documents
using compilation flags as described in \secref{sec:flags}
can be (permanently) stored in different files
for convenient compilation, viewing and distribution.
To this end, the package defines a command
to pass on compilation to a different file:

%%%%%%%%%%%%%%%%%%%%%%%%%%%%%%%%%%%%%%%%
\DescribeMacro{\childdocforward}
The command |\childdocforward| redirects processing to
another source file:
%
\begin{center}
\begin{tabular}{l}
|\input{childdoc.def}|\\
|\childdocforward[|\textit{main}|]{|\textit{dest}|}|\\
\end{tabular}
\end{center}
%
The argument \textit{dest} is the destination file
(without extension).
It should be the main file or one of the child files.
Note that further \textsf{childdoc} directives
such as |\childdocof| and |\childdocforward|
in the indicated file will be processed in this form.
The optional argument \textit{main}
passes on directly to the main file \textit{main}
while pretending to compile the child \textit{dest}.
This form behaves as if \textit{dest}
issues |\childdocof{|\textit{main}|}| right away,
and no further \textsf{childdoc} directives will be processed.

%%%%%%%%%%%%%%%%%%%%%%%%%%%%%%%%%%%%%%%%
\DescribeMacro{\...prefix}
In the alternative form |\childdocforwardprefix|,
%
\begin{center}
\begin{tabular}{l}
|\input{childdoc.def}|\\
|\childdocforwardprefix[|\textit{main}|]{|\textit{prefix}|}{|\textit{dest}|}|
\end{tabular}
\end{center}
%
the destination file is determined by a pattern
depending on the current file:
To make this work, the current file must be called
`{\textit{prefix}\hspace{0.2em}\textit{suffix}}'
with \textit{prefix} matching precisely the argument.
Processing is then passed on to the file
`{\textit{dest}\hspace{0.2em}\textit{suffix}}'.
Surely, the same effect is achieved by
directly specifying the
argument `{\textit{dest}\hspace{0.2em}\textit{suffix}}'
in the first form.
However, that requires to set up a different file
for each child. With the alternative form of the command
all these files can have exactly the same content
which simplifies setting them up and maintaining them.

For example, the following file |draft.tex|
with a compilation flag |\version| as described in \secref{sec:flags}
compiles the main document as a draft:
%
\begin{center}
\begin{tabular}{l}
|\def\version{draft}|\\
|\input{childdoc.def}|\\
|\childdocforward{|\textit{main}|}|
\end{tabular}
\end{center}
%
Likewise, the following files |final|\textit{nn}|.tex|
compile the final version of the child document
|child|\textit{nn}|.tex|:
%
\begin{center}
\begin{tabular}{l}
|\def\version{final}|\\
|\input{childdoc.def}|\\
|\childdocforwardprefix{final}{child}|
\end{tabular}
\end{center}
%

Note that when several versions of a main file and/or of each child file
are to be generated, it may be convenient to set up a |Makefile| or
shell script to automatise the process.

%%%%%%%%%%%%%%%%%%%%%%%%%%%%%%%%%%%%%%%%%%%%%%%%%%%%%%%%%%%%%%%%%%%%%%%%%%%%%%%%
\subsection{Command Line Processing}
\label{sec:commandline}

The effect of redirection files can also be achieved by invoking
the \LaTeX{} compiler with a more elaborate command line.
Most conveniently this should be done as part
of a shell script or a |Makefile|.

When using \textsf{childdoc} in the main file, the following
command lines effectively perform a redirection
(note that depending on the shell being used,
backslashes may have to be doubled: `|\|' $\to$ `|\\|'):
%
\begin{center}
|... -jobname "|\textit{target}|" |\\|"|[\textit{flags}]%
|\input{childdoc.def}\childdocforward[|\textit{main}|]{|\textit{dest}|}"|
\end{center}
%
Here \textit{target} is the name of the output file,
\textit{main} is the name of the main file
and \textit{dest} is the name of the main or child file to be processed
(all filenames without extensions).
The optional argument \textit{main} can be omitted
if \textit{main} matches \textit{dest}.
Optionally, compilation \textit{flags} can be defined via |\def| commands.
This command line makes the \TeX{} engine believe
it is compiling the file \textit{target}
whose content is specified as the latter parameter.
The provided code then forwards the processing to
\textit{main} or \textit{dest} as described in \secref{sec:forward}.

%%%%%%%%%%%%%%%%%%%%%%%%%%%%%%%%%%%%%%%%%%%%%%%%%%%%%%%%%%%%%%%%%%%%%%%%%%%%%%%%
\subsection{Include by Input}
\label{sec:input}

Including child documents by |\include| has some restrictions by design.
Most notably, the content of a child document always occupies
its own set of pages; pages cannot be shared between child documents.
Usually, this behaviour makes perfect sense
because each child document contain an essential part of the document.
However, in some situations it may be desirable to compose
a document from a collection of parts
without having mandatory page breaks between then.
For this case, the package
provides a mechanism to include parts
by |\input| which can also be processed individually.
However, by construction this mechanism
requires manual handling of the content to be output.

%%%%%%%%%%%%%%%%%%%%%%%%%%%%%%%%%%%%%%%%
\DescribeMacro{\ifchilddocmanual}
The main file should be prepared as usual, see \secref{sec:include}.
However, the document body must make a distinction
between processing of an individual part and of the main document, e.g.:
%
\begin{center}
\begin{tabular}{l}
|\ifchilddocmanual|\\
|\input{\childdocname}|\\
|\||else|\\
\textit{document body with }|\input{|\textit{part}|}|\\
|\||fi|
\end{tabular}
\end{center}
%
The conditional |\ifchilddocmanual| is true whenever
a part to be included by |\input| is being compiled,
and the name of the part is stored in |\childdocname|.

%%%%%%%%%%%%%%%%%%%%%%%%%%%%%%%%%%%%%%%%
\DescribeMacro{\childdocby}
Each part to be included by |\input| should start with:
%
\begin{center}
\begin{tabular}{l}
|\input{childdoc.def}|\\
|\childdocby{|\textit{main}|}|\\
\end{tabular}
\end{center}
%
The directive |\childdocby| is similar to |\childdocof|
described in \secref{sec:include},
but the subsequent selection of content must be done manually.
To that end, both |\ifchilddoc| and |\ifchilddocmanual|
will be true upon processing of a part,
and the name of the part is stored in |\childdocname|.
Note that |\jobname| will be set to the filename of the current part
so that each part receives an individual |.aux| file
that does not interfere with the |.aux| file(s) of the main document.
This behaviour can be altered by the alternative form
|\childdocby[*]{|\textit{main}|}| (with a non-empty optional argument)
which uses the |.aux| file of the main document
by setting |\jobname| to \textit{main}.

%%%%%%%%%%%%%%%%%%%%%%%%%%%%%%%%%%%%%%%%%%%%%%%%%%%%%%%%%%%%%%%%%%%%%%%%%%%%%%%%
\subsection{Driver Development}
\label{sec:driver}

The \textsf{childdoc} mechanism can also be use for the development
of definition files such as \LaTeX{} styles or classes.
This case differs from the above setup with multiple parts
included by |\include| in that no |\includeonly| should be invoked.
This can be achieved by starting the include file
(before |\ProvidesPackage|) with:
%
\begin{center}
\begin{tabular}{l}
|\input{childdoc.def}|\\
|\childdocforward{|\textit{main}|}|\\
\end{tabular}
\end{center}
%
or alternatively with:
%
\begin{center}
\begin{tabular}{l}
|\input{childdoc.def}|\\
|\childdocby{|\textit{main}|}|\\
\end{tabular}
\end{center}
%
Both forms have slightly different effects as described above.
The main file is prepared as usual, see \secref{sec:include}.

%%%%%%%%%%%%%%%%%%%%%%%%%%%%%%%%%%%%%%%%%%%%%%%%%%%%%%%%%%%%%%%%%%%%%%%%%%%%%%%%
\subsection{Legacy Detection}
\label{sec:detection}

The directive |\childdocmain| in the main file can detect
whether the complete document or merely a child is to be compiled
even without using the directive |\childdocof|.
This method is deprecated because it is less robust
and there is no compelling reason to use it;
it is merely provided for backward compatibility
and it may be removed in future versions.

If the detection mechanism is to be used,
it is mandatory to correctly specify
the filename of the main file as the argument of |\childdocmain|:
%
\begin{center}
\begin{tabular}{l}
|\input{childdoc.def}|\\
|\childdocmain{|\textit{main}|}|\\
\end{tabular}
\end{center}
%
If |\jobname| does not match the argument \textit{main} of |\childdocmain|,
it is assumed that |\jobname| points to the child file to be compiled.
When using |\childdocmain| with the main file specified as argument,
it suffices to start a child file
with just |\input{|\textit{main}|}|
without loading of the package and using |\childdocof|.
If instead all processing is done
with the appropriate \textsf{childdoc} directives,
the argument of \textit{main} of |\childdocmain| can be empty.

An alternative version of the command line processing described
in \secref{sec:commandline} using the detection mechanism reads:
%
\begin{center}
|... -jobname "|\textit{target}|" "|[\textit{flags}]%
[|\def\jobname{|\textit{dest}|}|]|\input{|\textit{main}|}"|
\end{center}

%%%%%%%%%%%%%%%%%%%%%%%%%%%%%%%%%%%%%%%%%%%%%%%%%%%%%%%%%%%%%%%%%%%%%%%%%%%%%%%%
\subsection{Manual Code}
\label{sec:manual}

In case one cannot be certain whether the definitions file |childdoc.def|
is installed on the target \TeX{} distribution
and one prefers not to ship it,
it is conceivable to paste a few relevant commands into the sources.

To that end, drop all statements |\input{childdoc.def}|
and perform the replacements as outlined below.
Instead of |\childdocmain{|\textit{main}|}| add the following code
to the top of the main file:
%
\begin{center}
\begin{tabular}{l}
|\||ifdefined\childdocname\endinput\||fi\newif\ifchilddoc|\\
|\edef\childdocname{\scantokens\expandafter{\jobname\noexpand}}|\\
|\def\childdocmain{|\textit{main}|}\||ifx\childdocmain\childdocname\||else|\\
|\childdoctrue\includeonly{\childdocname}\let\jobname\childdocmain\||fi|\\
\end{tabular}
\end{center}
%
Instead of |\childdocof{|\textit{main}|}| just include the main file
at the top of each child file:
%
\begin{center}
|\input{|\textit{main}|}|
\end{center}
%
A simple redirection |\childdocforward{|\textit{dest}|}| is achieved by:
%
\begin{center}
|\def\jobname{|\textit{dest}|}\input{\jobname}|
\end{center}
%
The redirection with prefix
|\childdocforwardprefix[|\textit{prefix}|]{|\textit{dest}|}|
is accomplished by:
%
\begin{center}
\begin{tabular}{l}
|{\edef\jobname{\scantokens\expandafter{\jobname\noexpand}}|\\
|\def\redirectjob |\textit{prefix}|#1~~~{\gdef\jobname{|\textit{dest}|#1}}|\\
|\expandafter\redirectjob\jobname~~~}\input{\jobname}|
\end{tabular}
\end{center}

In an alternative approach,
child documents can be compiled by a specific command line
without additional code or specific definitions:
%
\begin{center}
|... -jobname "|\textit{target}|" "|[\textit{flags}]%
|\includeonly{|\textit{dest}|}\input{|\textit{main}|}"|
\end{center}
%

%%%%%%%%%%%%%%%%%%%%%%%%%%%%%%%%%%%%%%%%%%%%%%%%%%%%%%%%%%%%%%%%%%%%%%%%%%%%%%%%
%%%%%%%%%%%%%%%%%%%%%%%%%%%%%%%%%%%%%%%%%%%%%%%%%%%%%%%%%%%%%%%%%%%%%%%%%%%%%%%%
\section{Information}

%%%%%%%%%%%%%%%%%%%%%%%%%%%%%%%%%%%%%%%%%%%%%%%%%%%%%%%%%%%%%%%%%%%%%%%%%%%%%%%%
\subsection{Copyright}

Copyright \copyright{} 2017--2018 Niklas Beisert

This work may be distributed and/or modified under the
conditions of the \LaTeX{} Project Public License, either version 1.3
of this license or (at your option) any later version.
The latest version of this license is in
  \url{http://www.latex-project.org/lppl.txt}
and version 1.3 or later is part of all distributions of \LaTeX{}
version 2005/12/01 or later.

This work has the LPPL maintenance status `maintained'.

The Current Maintainer of this work is Niklas Beisert.

This work consists of the files |README.txt|, |childdoc.ins| and |childdoc.dtx|
as well as the derived files |childdoc.def|, |cdocsamp.tex|
with |cdocsch1.tex|, |cdocsch2.tex|, |cdocspt3.tex|, |cdocspt4.tex|,
|cdocsdrf.tex|, |cdocsfn1.tex|, |cdocsfn2.tex|
as well as |childdoc.pdf|.

%%%%%%%%%%%%%%%%%%%%%%%%%%%%%%%%%%%%%%%%%%%%%%%%%%%%%%%%%%%%%%%%%%%%%%%%%%%%%%%%
\subsection{Files and Installation}

The package consists of the files:
%
\begin{center}
\begin{tabular}{ll}
    |README.txt|   & readme file \\
    |childdoc.ins| & installation file \\
    |childdoc.dtx| & source file \\
    |childdoc.def| & definition file \\
    |cdocsamp.tex| & sample main file \\
    |cdocsch1.tex| & sample include file \\
    |cdocsch2.tex| & sample include file \\
    |cdocspt3.tex| & sample part file \\
    |cdocspt4.tex| & sample part file \\
    |cdocsdrf.tex| & sample redirection file \\
    |cdocsfn1.tex| & sample redirection file \\
    |cdocsfn2.tex| & sample redirection file \\
    |childdoc.pdf| & manual
\end{tabular}
\end{center}
%
The distribution consists of the files
|README.txt|, |childdoc.ins| and |childdoc.dtx|.
%
\begin{itemize}
\item
Run (pdf)\LaTeX{} on |childdoc.dtx|
to compile the manual |childdoc.pdf| (this file).
\item
Run \LaTeX{} on |childdoc.ins| to create the definitions file |childdoc.def|
and the sample |cdocsamp.tex| with include files
|cdocsch1.tex|, |cdocsch2.tex|, |cdocspt3.tex|, |cdocspt4.tex|,
|cdocsdrf.tex|, |cdocsfn1.tex|, |cdocsfn2.tex|.
Then copy the file |childdoc.def| to an appropriate directory of your \LaTeX{}
distribution, e.g.\ \textit{texmf-root}|/tex/latex/childdoc|.
\end{itemize}

%%%%%%%%%%%%%%%%%%%%%%%%%%%%%%%%%%%%%%%%%%%%%%%%%%%%%%%%%%%%%%%%%%%%%%%%%%%%%%%%
\subsection{Related CTAN Packages}

There are several other packages which offer a similar functionality:
%
\begin{itemize}
\item
The packages
\href{http://ctan.org/pkg/docmute}{\textsf{docmute}},
\href{http://ctan.org/pkg/includex}{\textsf{includex}} and
\href{http://ctan.org/pkg/standalone}{\textsf{standalone}}
provide commands to include only the document body of
a child file thus allowing both files to be compiled individually.
\item
The packages \href{http://ctan.org/pkg/subdocs}{\textsf{subdocs}}
and \href{http://ctan.org/pkg/subfiles}{\textsf{subfiles}}
provide structures in which the main and child documents can be
encapsulated and allowing them to be compiled individually.
The inclusion mechanism is different from the conventional |\include|.
\item
The package \href{http://ctan.org/pkg/combine}{\textsf{combine}}
is an elaborate solution to combine several documents into one.
\end{itemize}
%
See also the CTAN topic \href{http://ctan.org/topic/subdocs}{\textsf{subdocs}}
for further related packages.
The present package differs from the above solutions in that
a document structure constructed with the conventional |\include| mechanism
just needs two extra commands at the top of every file
such that all constituent files can be compiled individually.

%%%%%%%%%%%%%%%%%%%%%%%%%%%%%%%%%%%%%%%%%%%%%%%%%%%%%%%%%%%%%%%%%%%%%%%%%%%%%%%%
%\subsection{Feature Suggestions}
%
%The following is a list of features which may be useful for future
%versions of this package:
%%
%\begin{itemize}
%\item
%\ldots
%\end{itemize}

%%%%%%%%%%%%%%%%%%%%%%%%%%%%%%%%%%%%%%%%%%%%%%%%%%%%%%%%%%%%%%%%%%%%%%%%%%%%%%%%
\subsection{Revision History}

%%%%%%%%%%%%%%%%%%%%%%%%%%%%%%%%%%%%%%%%
\paragraph{v2.0:} 2018/12/30

\begin{itemize}
\item
immediate forward processing
\item
added |\childdocby| mechanism
\item
manual restructured
\end{itemize}

%%%%%%%%%%%%%%%%%%%%%%%%%%%%%%%%%%%%%%%%
\paragraph{v1.6:} 2018/01/17

\begin{itemize}
\item
application for development of include files
\item
corrections to manual
\end{itemize}

%%%%%%%%%%%%%%%%%%%%%%%%%%%%%%%%%%%%%%%%
\paragraph{v1.5:} 2017/05/21

\begin{itemize}
\item
more complete structuring introduced
\item
|\childdocof| introduced
\item
|\childdoc| renamed to |\childdocmain|
\item
|\childredirect| renamed to |\childdocforward| and |\childdocforwardprefix|
and functionality expanded
\end{itemize}

%%%%%%%%%%%%%%%%%%%%%%%%%%%%%%%%%%%%%%%%
\paragraph{v1.0:} 2017/04/27

\begin{itemize}
\item
manual and install package
\item
first version published on CTAN
\end{itemize}

%%%%%%%%%%%%%%%%%%%%%%%%%%%%%%%%%%%%%%%%
\paragraph{v0.6:} 2017/04/26

\begin{itemize}
\item
redirection mechanism added
\end{itemize}

%%%%%%%%%%%%%%%%%%%%%%%%%%%%%%%%%%%%%%%%
\paragraph{v0.5:} 2017/04/26

\begin{itemize}
\item
functionality in definition file
\end{itemize}


%%%%%%%%%%%%%%%%%%%%%%%%%%%%%%%%%%%%%%%%%%%%%%%%%%%%%%%%%%%%%%%%%%%%%%%%%%%%%%%%
%%%%%%%%%%%%%%%%%%%%%%%%%%%%%%%%%%%%%%%%%%%%%%%%%%%%%%%%%%%%%%%%%%%%%%%%%%%%%%%%
%%%%%%%%%%%%%%%%%%%%%%%%%%%%%%%%%%%%%%%%%%%%%%%%%%%%%%%%%%%%%%%%%%%%%%%%%%%%%%%%
\appendix

\settowidth\MacroIndent{\rmfamily\scriptsize 000\ }

 \DocInput{childdoc.dtx}

\end{document}
%</driver>
% \fi
%
% %%%%%%%%%%%%%%%%%%%%%%%%%%%%%%%%%%%%%%%%%%%%%%%%%%%%%%%%%%%%%%%%%%%%%%%%%%%%%%
% %%%%%%%%%%%%%%%%%%%%%%%%%%%%%%%%%%%%%%%%%%%%%%%%%%%%%%%%%%%%%%%%%%%%%%%%%%%%%%
% \section{Sample}
%\iffalse
%<*samplemain>
%\fi
%
% The following presents a sample document
% with two chapters, two parts, a title page,
% a compile flag as well as three forwarding files to set the flag.
% It consists of eight |.tex| files:
% \begin{center}
% \begin{tabular}{ll}
% |cdocsamp.tex|&main file\\
% |cdocsch1.tex|&include file for chapter 1\\
% |cdocsch2.tex|&include file for chapter 2\\
% |cdocspt3.tex|&include file for part 3\\
% |cdocspt4.tex|&include file for part 4\\
% |cdocsdrf.tex|&forwarding file for main file in draft mode\\
% |cdocsfi1.tex|&forwarding file for final version of chapter 1\\
% |cdocsfi2.tex|&forwarding file for final version of chapter 2\\
% \end{tabular}
% \end{center}
% Each of the eight files can be compiled directly by the \LaTeX{} compiler.
%
% %%%%%%%%%%%%%%%%%%%%%%%%%%%%%%%%%%%%%%
% \paragraph{Main File.}
%
% The main file is called |cdocsamp.tex|.
%
% Load the \textsf{childdoc} definitions and
% declare the filename for the main document:
%    \begin{macrocode}
\input{childdoc.def}
\childdocmain{}
%    \end{macrocode}

% Optional override for |\version| flag:
%    \begin{macrocode}
%%\ifchilddoc\else\providecommand{\version}{draft}\fi
%    \end{macrocode}

% Define the default values for the |\version| flag
% (|final| for the main file and |draft| for childs):
%    \begin{macrocode}
\ifchilddoc
\providecommand{\version}{draft}
\else
\providecommand{\version}{final}
\fi
%    \end{macrocode}

% Load the standard document class:
%    \begin{macrocode}
\documentclass[12pt]{article}
%    \end{macrocode}

% Start the document body:
%    \begin{macrocode}
\begin{document}
%    \end{macrocode}

% Declare a title page.
% Print title, part of document being processed and version flag:
%    \begin{macrocode}
\addtocounter{page}{-1}
\begin{center}
{\LARGE\bfseries{}childdoc example\par}
\vspace{1cm}
\ifchilddoc
\ifchilddocmanual part\else chapter\fi:
`\childdocname' of `\childdocjob'\par
\else
main document: `\childdocjob'\par
\fi
version: \version\par
\end{center}
\newpage
%    \end{macrocode}

% Manually include selected file,
% otherwise process as usual:
%    \begin{macrocode}
\ifchilddocmanual
\section*{part `\childdocname'}
\input{\childdocname}
\else
%    \end{macrocode}

% Include the two chapters:
%    \begin{macrocode}
\include{cdocsch1}
\include{cdocsch2}
%    \end{macrocode}

% Include the two parts unless only chapters should be displayed:
%    \begin{macrocode}
\ifchilddoc\else
\section{part three}
\input{cdocspt3}
\section{part four}
\input{cdocspt4}
\fi
%    \end{macrocode}

% Process as usual until here:
%    \begin{macrocode}
\fi
%    \end{macrocode}

% End of document body:
%    \begin{macrocode}
\end{document}
%    \end{macrocode}
%\iffalse
%</samplemain>
%\fi
%
% %%%%%%%%%%%%%%%%%%%%%%%%%%%%%%%%%%%%%%
% \paragraph{Chapter Include Files.}
%
% The include files are called |cdocsch1.tex| and |cdocsch2.tex|.
%
%\iffalse
%<*samplechap1|samplechap2>
%\fi

% Optional override for |\version| flag:
%    \begin{macrocode}
%%\providecommand{\version}{final}
%    \end{macrocode}

% Include the main document:
%    \begin{macrocode}
\input{childdoc.def}
\childdocof{cdocsamp}
%    \end{macrocode}

%\iffalse
%</samplechap1|samplechap2>
%\fi
%
%\iffalse
%<*samplechap1>
%\fi
% Some text for chapter 1:
%    \begin{macrocode}
\section{one}
some text in chapter one
%    \end{macrocode}

%\iffalse
%</samplechap1>
%\fi
% Some text for chapter 2:
%\iffalse
%<*samplechap2>
%\fi
%    \begin{macrocode}
\section{two}
more text in chapter two
%    \end{macrocode}

%\iffalse
%</samplechap2>
%\fi
%
% %%%%%%%%%%%%%%%%%%%%%%%%%%%%%%%%%%%%%%
% \paragraph{Part Include Files.}
%
% The include files are called |cdocspt3.tex| and |cdocspt4.tex|.
%
%\iffalse
%<*samplepart3|samplepart4>
%\fi

% Optional override for |\version| flag:
%    \begin{macrocode}
%%\providecommand{\version}{final}
%    \end{macrocode}

% Include the main document:
%    \begin{macrocode}
\input{childdoc.def}
\childdocby{cdocsamp}
%    \end{macrocode}

%\iffalse
%</samplepart3|samplepart4>
%\fi
%
%\iffalse
%<*samplepart3>
%\fi
% Some text for part 3:
%    \begin{macrocode}
some text in part three
%    \end{macrocode}

%\iffalse
%</samplepart3>
%\fi
% Some text for part 4:
%\iffalse
%<*samplepart4>
%\fi
%    \begin{macrocode}
more text in part four
%    \end{macrocode}

%\iffalse
%</samplepart4>
%\fi
%
% %%%%%%%%%%%%%%%%%%%%%%%%%%%%%%%%%%%%%%
% \paragraph{Forwarding for a Complete Draft.}
%
% The following forwarding file |cdocsdrf.tex|
% compiles the main document in draft mode:
%\iffalse
%<*sampledraft>
%\fi
%    \begin{macrocode}
\def\version{draft}
\input{childdoc.def}
\childdocforward{cdocsamp}
%    \end{macrocode}

%\iffalse
%</sampledraft>
%\fi
%
% %%%%%%%%%%%%%%%%%%%%%%%%%%%%%%%%%%%%%%
% \paragraph{Forwarding for Final Version of the Chapters.}
%
% The following forwarding files |cdocsfn1.tex| and |cdocsfn2.tex|
% (with identical content)
% compile the final versions of the child documents
% |cdocsch1.tex| and |cdocsch2.tex|, respectively:
%\iffalse
%<*samplefinal>
%\fi
%    \begin{macrocode}
\def\version{final}
\input{childdoc.def}
\childdocforwardprefix[cdocsamp]{cdocsfn}{cdocsch}
%    \end{macrocode}

%\iffalse
%</samplefinal>
%\fi
%
% %%%%%%%%%%%%%%%%%%%%%%%%%%%%%%%%%%%%%%
% \paragraph{Command Line Processing.}
%
% The following three command lines generate the output files
% |cdocscld|, |cdocscl1| and |cdocscl2|
% which should be identical to
% |cdocsdrf|, |cdocsch1| and |cdocsfn2|, respectively:
% \begin{center}
% \begin{tabular}{l}
% |latex -jobname cdocscld \|\\
% |  "\def\version{draft}\input{childdoc.def}\childdocforward{cdocsamp}"|\\
% |latex -jobname cdocscl1 \|\\
% |  "\input{childdoc.def}\childdocforward[cdocsamp]{cdocsch1}"|\\
% |latex -jobname cdocscl2 \|\\
% |  "\def\version{final}\input{childdoc.def}\childdocforward{cdocsch2}"|
% \end{tabular}
% \end{center}
% Note that the trailing backslash on each first line
% merely continues the input to the second line
% (for convenient cut ant paste).
% Furthermore, the command |latex| can be replaced by any
% of its alternative versions such as |pdflatex|.
%
% %%%%%%%%%%%%%%%%%%%%%%%%%%%%%%%%%%%%%%%%%%%%%%%%%%%%%%%%%%%%%%%%%%%%%%%%%%%%%%
% %%%%%%%%%%%%%%%%%%%%%%%%%%%%%%%%%%%%%%%%%%%%%%%%%%%%%%%%%%%%%%%%%%%%%%%%%%%%%%
% \section{Implementation}
%\iffalse
%<*package>
%\fi
%
% This section describes the definitions file |childdoc.def|.

% The definitions cannot be loaded using |\usepackage| or |\RequirePackage|
% which has a mechanism to prevent loading a style file more than once.
% When loading the definitions by means of |\input|
% multiple instances have to be prevented manually:
%\iffalse
%This code needs to be before the `\ProvidesFile' directive
%which is defined at the beginning of this file.
%Therefore it is also placed there and commented out here.
%</package>
%<*discard>
%\fi
%    \begin{macrocode}
\ifdefined\childdocmain\endinput\fi
%    \end{macrocode}
%\iffalse
%</discard>
%<*package>
%\fi
%
% \macro{\ifchilddoc}
% \macro{\ifchilddocmanual}
% The conditional |\ifchilddoc| tells whether a
% child (true) or main (false) document is being compiled.
% The conditional |\ifchilddocmanual| tells whether
% the |\includeonly| mechanism is used (false) or
% the selection of child files must be performed manually (true).
% The definitions initialise to false:
%    \begin{macrocode}
\newif\ifchilddoc
\newif\ifchilddocmanual
%    \end{macrocode}

% \macro{\childdocname}
% \macro{\childdocjob}
% The macro |\childdocname| stores the name of the main document
% to be compiled. The macro |\childdocjob| stores the name of
% the document on which the \LaTeX{} compiler was originally invoked.
% The content of |\jobname| cannot be compared
% to filenames specified in the source due to different catcodes.
% The following code rescans |\jobname|, stores the result
% in |\childdocname| and saves a copy in |\childdocjob|:
%    \begin{macrocode}
\edef\childdocname{\scantokens\expandafter{\jobname\noexpand}}
\let\childdocjob\childdocname
%    \end{macrocode}

% \macro{\childdocdisable}
% The macro |\childdocdisable| prevents the main file
% from being processed more than once.
% At this stage, the main document command |\childdocmain|
% is assumed to be called once again where it should do nothing.
% Any subsequent call to it should prevent
% a secondary processing of the main document
% It overwrites the forwarding commands
% |\childdocof| and |\childdocforward|
% with empty macros to prevent further inclusions of the main document:
%    \begin{macrocode}
\newcommand{\childdocdisable}
{
  \renewcommand{\childdocmain}[1]{\renewcommand{\childdocmain}[1]{\endinput}}
  \renewcommand{\childdocof}[1]{}
  \renewcommand{\childdocby}[2][]{}
  \renewcommand{\childdocforward}[2][]{}
  \renewcommand{\childdocdisable}{}
}
%    \end{macrocode}

% \macro{\childdocmain}
% The macro |\childdocmain| is to be called at the top of the main file
% with nothing or the main filename (without extension) as argument.
% First, it breaks loops.
% If the argument is not empty and does not match |\childdocname|
% (which is set by the first inclusion of |childdoc.def|),
% |\ifchilddoc| is set to true, |\includeonly| is applied to the child file
% and |\jobname| is set to the main file
% (for proper handling of |.aux| files):
%    \begin{macrocode}
\newcommand{\childdocmain}[1]
{
  \childdocdisable\childdocmain{}
  \if?#1?\else
    \begingroup
      \def\childdoctmp{#1}
      \ifx\childdoctmp\childdocname
        \def\childdoctmp{}
      \else
        \def\childdoctmp
        {
          \childdoctrue
          \includeonly{\childdocname}
          \def\childdocjob{#1}
          \def\jobname{#1}
        }
      \fi
      \expandafter
    \endgroup
    \childdoctmp
  \fi
}
%    \end{macrocode}

% \macro{\childdocof}
% The command |\childdocof| redirects
% compilation to the main file |#1|.
%    \begin{macrocode}
\newcommand{\childdocof}[1]
{
  \childdocdisable
  \childdoctrue
  \includeonly{\childdocname}
  \def\jobname{#1}
  \def\childdocjob{#1}
  \input{#1}
}
%    \end{macrocode}

% \macro{\childdocby}
% The command |\childdocby| ....
%    \begin{macrocode}
\newcommand{\childdocby}[2][]
{
  \childdocdisable
  \childdoctrue
  \childdocmanualtrue
  \if?#1?\else
    \def\jobname{#2}
  \fi
  \def\childdocjob{#2}
  \input{#2}
  \endinput
}
%    \end{macrocode}

% \macro{\childdocforward}
% The command |\childdocforward| redirects
% compilation to the main file or
% (if the optional argument is given) a child file.
% Parameters are set as if the main file
% or a child file starting with |\childdocof| was compiled.
% Then compilation is handed over to the main file:
%    \begin{macrocode}
\newcommand{\childdocforward}[2][]
{
  \begingroup
    \if?#1?
      \def\childdoctmp
      {
        \def\childdocname{#2}
        \def\childdocjob{#2}
        \def\jobname{#2}
        \input{#2}
        \endinput
      }
    \else
      \def\childdoctmp
      {
        \childdocdisable
        \def\childdocname{#2}
        \childdoctrue
        \includeonly{#2}
        \def\childdocjob{#1}
        \def\jobname{#1}
        \input{#1}
        \endinput
      }
    \fi
    \expandafter
  \endgroup
  \childdoctmp
}
%    \end{macrocode}

% \macro{\childdocforwardprefix}
% The command |\childdocforwardprefix| redirects
% compilation to the main or a child file by means of a pattern.
% The prefix |#1| in the current filename is replaced by |#2|
% and the suffix of the current filename is kept
% (it is assumed that the filename does not contain the substring `|~~~|'
% which is used as a delimiter).
% Compilation is handed over to the new file by |\childdocforward|:
%    \begin{macrocode}
\newcommand{\childdocforwardprefix}[3][]
{
  \begingroup
    \def\childdocextract #2##1~~~{\def\childdoctmp{\childdocforward[#1]{#3##1}}}
    \expandafter\childdocextract\childdocname~~~
    \expandafter
  \endgroup
  \childdoctmp
}
%    \end{macrocode}

% \macro{\childdoc}
% The deprecated macro |\childdoc| is a legacy version of |\childdocmain|:
%    \begin{macrocode}
\newcommand{\childdoc}{\childdocmain}
%    \end{macrocode}

% \macro{\childdocredirect}
% The deprecated macro |\childdocredirect| is a legacy version
% of |\childdocforward| and |\childdocforwardprefix|:
%    \begin{macrocode}
\newcommand{\childdocredirect}[2][]
{
  \begingroup
    \if?#1?
      \def\childdoctmp{\childdocforward{#2}}
    \else
      \def\childdoctmp{\childdocforwardprefix{#1}{#2}}
    \fi
    \expandafter
  \endgroup
  \childdoctmp
}
%    \end{macrocode}

%\iffalse
%</package>
%\fi
%
\endinput

\childdocforward{cdocsamp}
%    \end{macrocode}

%\iffalse
%</sampledraft>
%\fi
%
% %%%%%%%%%%%%%%%%%%%%%%%%%%%%%%%%%%%%%%
% \paragraph{Forwarding for Final Version of the Chapters.}
%
% The following forwarding files |cdocsfn1.tex| and |cdocsfn2.tex|
% (with identical content)
% compile the final versions of the child documents
% |cdocsch1.tex| and |cdocsch2.tex|, respectively:
%\iffalse
%<*samplefinal>
%\fi
%    \begin{macrocode}
\def\version{final}
% \iffalse
%
% childdoc.dtx Copyright (C) 2017-2018 Niklas Beisert
%
% This work may be distributed and/or modified under the
% conditions of the LaTeX Project Public License, either version 1.3
% of this license or (at your option) any later version.
% The latest version of this license is in
%   http://www.latex-project.org/lppl.txt
% and version 1.3 or later is part of all distributions of LaTeX
% version 2005/12/01 or later.
%
% This work has the LPPL maintenance status `maintained'.
%
% The Current Maintainer of this work is Niklas Beisert.
%
% This work consists of the files childdoc.dtx and childdoc.ins
% and the derived files childdoc.def and cdocsamp.tex with
% cdocsch1.tex, cdocsch2.tex, cdocsdrf.tex, cdocsfn1.tex, cdocsfn2.tex.
%
%<package>\ifdefined\childdocmain\endinput\fi
%<package>\ProvidesFile{childdoc.def}[2018/12/30 v2.0 child document driver]
%<samplemain>\ProvidesFile{cdocsamp.tex}[2018/12/30 v2.0 sample for childdoc]
%<*driver>
%\ProvidesFile{childdoc.drv}[2018/12/30 v2.0 childdoc reference manual file]
\PassOptionsToClass{10pt,a4paper}{article}
\documentclass{ltxdoc}

\usepackage[margin=35mm]{geometry}
\usepackage{hyperref}
\usepackage{hyperxmp}
\usepackage[usenames]{color}

\hypersetup{colorlinks=true}
\hypersetup{pdfstartview=FitH}
\hypersetup{pdfpagemode=UseNone}
\hypersetup{pdfsource={}}
\hypersetup{pdflang={en-UK}}
\hypersetup{pdfcopyright={Copyright 2017-2018 Niklas Beisert.
  This work may be distributed and/or modified under the
  conditions of the LaTeX Project Public License, either version 1.3
  of this license or (at your option) any later version.}}
\hypersetup{pdflicenseurl={http://www.latex-project.org/lppl.txt}}
\hypersetup{pdfcontactaddress={ETH Zurich, ITP, HIT K,
  Wolfgang-Pauli-Strasse 27}}
\hypersetup{pdfcontactpostcode={8093}}
\hypersetup{pdfcontactcity={Zurich}}
\hypersetup{pdfcontactcountry={Switzerland}}
\hypersetup{pdfcontactemail={nbeisert@itp.phys.ethz.ch}}
\hypersetup{pdfcontacturl={http://people.phys.ethz.ch/\xmptilde nbeisert/}}

\newcommand{\secref}[1]{\hyperref[#1]{section \ref*{#1}}}

\parskip1ex
\parindent0pt
\let\olditemize\itemize
\def\itemize{\olditemize\parskip0pt}

\begin{document}

\title{The \textsf{childdoc} Package}
\hypersetup{pdftitle={The childdoc Package}}
\author{Niklas Beisert\\[2ex]
  Institut f\"ur Theoretische Physik\\
  Eidgen\"ossische Technische Hochschule Z\"urich\\
  Wolfgang-Pauli-Strasse 27, 8093 Z\"urich, Switzerland\\[1ex]
  \href{mailto:nbeisert@itp.phys.ethz.ch}
  {\texttt{nbeisert@itp.phys.ethz.ch}}}
\hypersetup{pdfauthor={Niklas Beisert}}
\hypersetup{pdfsubject={Manual for the LaTeX2e Package childdoc}}
\date{30 December 2018, \textsf{v2.0}}
\maketitle

\begin{abstract}\noindent
\textsf{childdoc} is a \LaTeXe{} package
that enables the direct compilation
of document sections included by |\include|
to individual files.
\end{abstract}

\begingroup
\parskip0ex
\tableofcontents
\endgroup

%%%%%%%%%%%%%%%%%%%%%%%%%%%%%%%%%%%%%%%%%%%%%%%%%%%%%%%%%%%%%%%%%%%%%%%%%%%%%%%%
%%%%%%%%%%%%%%%%%%%%%%%%%%%%%%%%%%%%%%%%%%%%%%%%%%%%%%%%%%%%%%%%%%%%%%%%%%%%%%%%
\section{Introduction}

\LaTeX{} provides a mechanism to structure a large document (such as a book)
into a main file and several child files (containing the chapters)
using the |\include| command.
This mechanism is beneficial for documents
which span hundreds of pages in order to
make the source file(s) more manageable.
Moreover, compilation can be restricted to
selected child files by means of the |\includeonly| command.
The latter feature can be used to reduce the compilation time while editing
(this was significantly more useful in the earlier days of \LaTeX{})
or to generate a smaller document which is easier to navigate.
Another application of |\includeonly| is to generate
documents consisting of selected parts of the complete document.

However, there are a few drawbacks of the plain |\include| mechanism:
\begin{itemize}
\item
The child files cannot be compiled on their own,
they can only be compiled via the main file.
A naive editing environment
(such as a text editor with an option
to have the current file processed by \LaTeX)
may require one to switch to the main file before compiling;
attempting to compile the child file produces errors.
\item
The main file must be modified (each time)
to adjust the |\includeonly| command
to the present needs. This easily leaves the main file in a messy state.
\item
The generated document will always carry the filename
of the main document. This is inconvenient if
several child files are to be compiled and
to be kept for distribution.
\end{itemize}

The present package provides a simple interface
to make child files individually compilable by \LaTeX{}.
Compiling a child file then has the same effect as compiling
the main file with an |\includeonly| command
to select the appropriate child.
Moreover the generated document will carry the name of the child
rather than the main file.
This resolves all three above issues.

This feature is meant to make the editing of books,
thesis documents and lecture notes somewhat more convenient.
However, the package can also be used efficiently for
composing a series of documents (such as exercise sheets)
which are typically distributed individually.
It then assists the author in generating the individual documents
(potentially in different versions)
as well as a document containing the collected series.
Another application is in developing style files
or other kinds of included material
where compilation of the style file could redirect
to a sample or test file.

%%%%%%%%%%%%%%%%%%%%%%%%%%%%%%%%%%%%%%%%%%%%%%%%%%%%%%%%%%%%%%%%%%%%%%%%%%%%%%%%
%%%%%%%%%%%%%%%%%%%%%%%%%%%%%%%%%%%%%%%%%%%%%%%%%%%%%%%%%%%%%%%%%%%%%%%%%%%%%%%%
\section{Usage}

First of all, the package \textsf{childdoc} is \emph{not} a standard
\LaTeXe{} |.sty| style file! Therefore it needs to be invoked in
a non-standard way.

%%%%%%%%%%%%%%%%%%%%%%%%%%%%%%%%%%%%%%%%%%%%%%%%%%%%%%%%%%%%%%%%%%%%%%%%%%%%%%%%
\subsection{Included Files}
\label{sec:include}

%%%%%%%%%%%%%%%%%%%%%%%%%%%%%%%%%%%%%%%%
\DescribeMacro{\childdocmain}
To use the package, add the commands
\begin{center}
\begin{tabular}{l}
|\input{childdoc.def}|\\
|\childdocmain{}|\\
\end{tabular}
\end{center}
at the very top of the main \LaTeX{} file,
in particular \emph{before} the |\documentclass| statement!
The argument of |\childdocmain| should be left empty
(but it must be present).

%%%%%%%%%%%%%%%%%%%%%%%%%%%%%%%%%%%%%%%%
\DescribeMacro{\childdocof}
Furthermore, add the commands
\begin{center}
\begin{tabular}{l}
|\input{childdoc.def}|\\
|\childdocof{|\textit{main}|}|\\
\end{tabular}
\end{center}
at the top of every child file \textit{child}
which is included by |\include{|\textit{child}|}|
from within the main file
(or at least for those files to be compiled individually).
The argument \textit{main} must be the filename of the main file.

There are a couple of
considerations in setting up the main and child documents:

%%%%%%%%%%%%%%%%%%%%%%%%%%%%%%%%%%%%%%%%
\paragraph{Restrictions.}

Please note the following restrictions:
\begin{itemize}
\item
|\childdocmain| must be called with one argument \textit{main}
to ensure compatibility with earlier version of the package.
It must either be empty (|\childdocmain{}|)
or precisely match the filename of the main file in which it is specified.
See \secref{sec:detection} for further information.
\item
The filename \textit{main} must be specified without the |.tex| extension.
\item
The filename \textit{main} is case sensitive
(even in case-insensitive file systems)
due to internal string comparison.
\item
The argument \textit{main} should be fully expanded, it cannot be a macro.
\item
Subdirectories and special characters should be avoided in filenames.
\item
The command |\childdocmain{|\textit{main}|}| must be followed by a whitespace.
It should not be followed immediately by another command
or by a comment mark `|%|'.
This is because the \TeX{} parser reads the token immediately following
the argument of |\childdocmain| and puts it
at the beginning of every child section;
however, a white\-space is ignored.
\end{itemize}

%%%%%%%%%%%%%%%%%%%%%%%%%%%%%%%%%%%%%%%%
\paragraph{Content of Main File.}

It is advisable to place all content in the child files included by |\include|.
Any output contained in the main file will appear in all child documents
unless suppressed manually;
it cannot be suppressed automatically by the |\includeonly| directive
and thus should normally be avoided.
A method to include some content in the main file
by means of conditional processing is described in \secref{sec:conditional}.

%%%%%%%%%%%%%%%%%%%%%%%%%%%%%%%%%%%%%%%%
\paragraph{Page Numbering.}

When only a part of the document is compiled,
the appropriate numbering of pages
(as well as other status parameters)
is determined from the |.aux| files.
The latter contain information from previous passes.
However this information needs to propagate through
all intermediate child documents.
Therefore the page numbering in child documents may well
be inconsistent until the complete document is compiled at least once.

A useful (if unconventional) way to always ensure a consistent
page numbering is to restart the numbering in each child document
and denote the pages by `\textit{child}|.|\textit{page}'
where \textit{child} represents the chapter/section number of the child file.
This can be achieved by the command
|\numberwithin{page}{|\textit{child}|}|
of the \textsf{amsmath} package
where \textit{child} can be |chapter| or |section|
depending on the chosen structuring.
Alternatively, one can modify the macro |\thepage| appropriately
and reset the counter |page| at the start of each child file.

%%%%%%%%%%%%%%%%%%%%%%%%%%%%%%%%%%%%%%%%%%%%%%%%%%%%%%%%%%%%%%%%%%%%%%%%%%%%%%%%
\subsection{Conditional Processing}
\label{sec:conditional}

The package provides a mechanism to compile different versions
of a document. To customise the versions further some conditional processing
can come in handy to distinguish which version is being compiled.
The package provides two macros to describe the compilation context:

%%%%%%%%%%%%%%%%%%%%%%%%%%%%%%%%%%%%%%%%
\DescribeMacro{\ifchilddoc}
The conditional |\ifchilddoc| distinguishes between the compilation of
child documents and the main document:
%
\begin{center}
|\ifchilddoc |\textit{child-code}| |[|\||else |\textit{main-code}]| \||fi|
\end{center}

%%%%%%%%%%%%%%%%%%%%%%%%%%%%%%%%%%%%%%%%
\DescribeMacro{\childdocname}
\DescribeMacro{\childdocjob}
The macro |\childdocname| contains the filename (without extension)
of the main or child file being processed.
Note that |\childdocjob| will always contain the name of the main file.

%%%%%%%%%%%%%%%%%%%%%%%%%%%%%%%%%%%%%%%%
\paragraph{Title Page.}

Conditional processing can be used to include a title or banner page
in the main document when proper precautions are taken.
Importantly, the code in the main file should ensure that the page counter
(as well as other status parameters which are stored in the |.aux| files)
takes the same value after the conditional processing.
Otherwise the page numbers may take divergent values
depending on which part is compiled.

For example, a title page could be declared by:
%
\begin{center}
\begin{tabular}{l}
|\ifchilddoc\||else|\\
|\addtocounter{page}{-1}|\\
\textit{code for title page}\\
|\newpage|\\
|\||fi|
\end{tabular}
\end{center}
%
A banner page for the child documents can be generated by:
%
\begin{center}
\begin{tabular}{l}
|\ifchilddoc|\\
|\addtocounter{page}{-1}|\\
\textit{code for banner page}\\
|\newpage|\\
|\||fi|
\end{tabular}
\end{center}
%
Here one could write a message such as:
\begin{center}
|This is the part \childdocname{} of \childdocjob{}.|
\end{center}

%%%%%%%%%%%%%%%%%%%%%%%%%%%%%%%%%%%%%%%%%%%%%%%%%%%%%%%%%%%%%%%%%%%%%%%%%%%%%%%%
\subsection{Flags}
\label{sec:flags}

The package makes it easy to generate different versions
of the main or child documents.
To this end compilation flags can be defined
and assigned different default values.
They will be particularly useful in conjunction
with the forwarding mechanism described in \secref{sec:forward}.

For example, it may be useful to have a flag |\version|
which can be set to |draft| or |final|.
The document source will contain some conditional code
depending on the value of |\version|.
Suppose further, the flag should default to |final| for the main file
and to |draft| for child files
which is a natural assignment for editing the document.
This is achieved by placing the following code
in the preamble of the main document
(below the |\childdocmain| directive):
%
\begin{center}
\begin{tabular}{l}
|\ifchilddoc|\\
|\providecommand{\version}{draft}|\\
|\||else|\\
|\providecommand{\version}{final}|\\
|\||fi|
\end{tabular}
\end{center}
%
The definition by |\providecommand| makes sure
that previous definitions are not overwritten.
Further statements |\providecommand{\version}{...}|
can thus be added before the above code to override it.

For the main file, one might add a line
(between |\childdocmain| and the above block)
%
\begin{center}
|%\ifchilddoc\||else\providecommand{\version}{draft}\||fi|
\end{center}
%
which can be uncommented to produce a draft version.
Likewise one can add a line to the very top of a child file
(above the |\childdocof{|\textit{main}|}| directive)
%
\begin{center}
|%\providecommand{\version}{final}|
\end{center}
%
which can be uncommented to produce the final version of this child document.

%%%%%%%%%%%%%%%%%%%%%%%%%%%%%%%%%%%%%%%%%%%%%%%%%%%%%%%%%%%%%%%%%%%%%%%%%%%%%%%%
\subsection{Forwarding}
\label{sec:forward}

Different versions of the main or child documents
using compilation flags as described in \secref{sec:flags}
can be (permanently) stored in different files
for convenient compilation, viewing and distribution.
To this end, the package defines a command
to pass on compilation to a different file:

%%%%%%%%%%%%%%%%%%%%%%%%%%%%%%%%%%%%%%%%
\DescribeMacro{\childdocforward}
The command |\childdocforward| redirects processing to
another source file:
%
\begin{center}
\begin{tabular}{l}
|\input{childdoc.def}|\\
|\childdocforward[|\textit{main}|]{|\textit{dest}|}|\\
\end{tabular}
\end{center}
%
The argument \textit{dest} is the destination file
(without extension).
It should be the main file or one of the child files.
Note that further \textsf{childdoc} directives
such as |\childdocof| and |\childdocforward|
in the indicated file will be processed in this form.
The optional argument \textit{main}
passes on directly to the main file \textit{main}
while pretending to compile the child \textit{dest}.
This form behaves as if \textit{dest}
issues |\childdocof{|\textit{main}|}| right away,
and no further \textsf{childdoc} directives will be processed.

%%%%%%%%%%%%%%%%%%%%%%%%%%%%%%%%%%%%%%%%
\DescribeMacro{\...prefix}
In the alternative form |\childdocforwardprefix|,
%
\begin{center}
\begin{tabular}{l}
|\input{childdoc.def}|\\
|\childdocforwardprefix[|\textit{main}|]{|\textit{prefix}|}{|\textit{dest}|}|
\end{tabular}
\end{center}
%
the destination file is determined by a pattern
depending on the current file:
To make this work, the current file must be called
`{\textit{prefix}\hspace{0.2em}\textit{suffix}}'
with \textit{prefix} matching precisely the argument.
Processing is then passed on to the file
`{\textit{dest}\hspace{0.2em}\textit{suffix}}'.
Surely, the same effect is achieved by
directly specifying the
argument `{\textit{dest}\hspace{0.2em}\textit{suffix}}'
in the first form.
However, that requires to set up a different file
for each child. With the alternative form of the command
all these files can have exactly the same content
which simplifies setting them up and maintaining them.

For example, the following file |draft.tex|
with a compilation flag |\version| as described in \secref{sec:flags}
compiles the main document as a draft:
%
\begin{center}
\begin{tabular}{l}
|\def\version{draft}|\\
|\input{childdoc.def}|\\
|\childdocforward{|\textit{main}|}|
\end{tabular}
\end{center}
%
Likewise, the following files |final|\textit{nn}|.tex|
compile the final version of the child document
|child|\textit{nn}|.tex|:
%
\begin{center}
\begin{tabular}{l}
|\def\version{final}|\\
|\input{childdoc.def}|\\
|\childdocforwardprefix{final}{child}|
\end{tabular}
\end{center}
%

Note that when several versions of a main file and/or of each child file
are to be generated, it may be convenient to set up a |Makefile| or
shell script to automatise the process.

%%%%%%%%%%%%%%%%%%%%%%%%%%%%%%%%%%%%%%%%%%%%%%%%%%%%%%%%%%%%%%%%%%%%%%%%%%%%%%%%
\subsection{Command Line Processing}
\label{sec:commandline}

The effect of redirection files can also be achieved by invoking
the \LaTeX{} compiler with a more elaborate command line.
Most conveniently this should be done as part
of a shell script or a |Makefile|.

When using \textsf{childdoc} in the main file, the following
command lines effectively perform a redirection
(note that depending on the shell being used,
backslashes may have to be doubled: `|\|' $\to$ `|\\|'):
%
\begin{center}
|... -jobname "|\textit{target}|" |\\|"|[\textit{flags}]%
|\input{childdoc.def}\childdocforward[|\textit{main}|]{|\textit{dest}|}"|
\end{center}
%
Here \textit{target} is the name of the output file,
\textit{main} is the name of the main file
and \textit{dest} is the name of the main or child file to be processed
(all filenames without extensions).
The optional argument \textit{main} can be omitted
if \textit{main} matches \textit{dest}.
Optionally, compilation \textit{flags} can be defined via |\def| commands.
This command line makes the \TeX{} engine believe
it is compiling the file \textit{target}
whose content is specified as the latter parameter.
The provided code then forwards the processing to
\textit{main} or \textit{dest} as described in \secref{sec:forward}.

%%%%%%%%%%%%%%%%%%%%%%%%%%%%%%%%%%%%%%%%%%%%%%%%%%%%%%%%%%%%%%%%%%%%%%%%%%%%%%%%
\subsection{Include by Input}
\label{sec:input}

Including child documents by |\include| has some restrictions by design.
Most notably, the content of a child document always occupies
its own set of pages; pages cannot be shared between child documents.
Usually, this behaviour makes perfect sense
because each child document contain an essential part of the document.
However, in some situations it may be desirable to compose
a document from a collection of parts
without having mandatory page breaks between then.
For this case, the package
provides a mechanism to include parts
by |\input| which can also be processed individually.
However, by construction this mechanism
requires manual handling of the content to be output.

%%%%%%%%%%%%%%%%%%%%%%%%%%%%%%%%%%%%%%%%
\DescribeMacro{\ifchilddocmanual}
The main file should be prepared as usual, see \secref{sec:include}.
However, the document body must make a distinction
between processing of an individual part and of the main document, e.g.:
%
\begin{center}
\begin{tabular}{l}
|\ifchilddocmanual|\\
|\input{\childdocname}|\\
|\||else|\\
\textit{document body with }|\input{|\textit{part}|}|\\
|\||fi|
\end{tabular}
\end{center}
%
The conditional |\ifchilddocmanual| is true whenever
a part to be included by |\input| is being compiled,
and the name of the part is stored in |\childdocname|.

%%%%%%%%%%%%%%%%%%%%%%%%%%%%%%%%%%%%%%%%
\DescribeMacro{\childdocby}
Each part to be included by |\input| should start with:
%
\begin{center}
\begin{tabular}{l}
|\input{childdoc.def}|\\
|\childdocby{|\textit{main}|}|\\
\end{tabular}
\end{center}
%
The directive |\childdocby| is similar to |\childdocof|
described in \secref{sec:include},
but the subsequent selection of content must be done manually.
To that end, both |\ifchilddoc| and |\ifchilddocmanual|
will be true upon processing of a part,
and the name of the part is stored in |\childdocname|.
Note that |\jobname| will be set to the filename of the current part
so that each part receives an individual |.aux| file
that does not interfere with the |.aux| file(s) of the main document.
This behaviour can be altered by the alternative form
|\childdocby[*]{|\textit{main}|}| (with a non-empty optional argument)
which uses the |.aux| file of the main document
by setting |\jobname| to \textit{main}.

%%%%%%%%%%%%%%%%%%%%%%%%%%%%%%%%%%%%%%%%%%%%%%%%%%%%%%%%%%%%%%%%%%%%%%%%%%%%%%%%
\subsection{Driver Development}
\label{sec:driver}

The \textsf{childdoc} mechanism can also be use for the development
of definition files such as \LaTeX{} styles or classes.
This case differs from the above setup with multiple parts
included by |\include| in that no |\includeonly| should be invoked.
This can be achieved by starting the include file
(before |\ProvidesPackage|) with:
%
\begin{center}
\begin{tabular}{l}
|\input{childdoc.def}|\\
|\childdocforward{|\textit{main}|}|\\
\end{tabular}
\end{center}
%
or alternatively with:
%
\begin{center}
\begin{tabular}{l}
|\input{childdoc.def}|\\
|\childdocby{|\textit{main}|}|\\
\end{tabular}
\end{center}
%
Both forms have slightly different effects as described above.
The main file is prepared as usual, see \secref{sec:include}.

%%%%%%%%%%%%%%%%%%%%%%%%%%%%%%%%%%%%%%%%%%%%%%%%%%%%%%%%%%%%%%%%%%%%%%%%%%%%%%%%
\subsection{Legacy Detection}
\label{sec:detection}

The directive |\childdocmain| in the main file can detect
whether the complete document or merely a child is to be compiled
even without using the directive |\childdocof|.
This method is deprecated because it is less robust
and there is no compelling reason to use it;
it is merely provided for backward compatibility
and it may be removed in future versions.

If the detection mechanism is to be used,
it is mandatory to correctly specify
the filename of the main file as the argument of |\childdocmain|:
%
\begin{center}
\begin{tabular}{l}
|\input{childdoc.def}|\\
|\childdocmain{|\textit{main}|}|\\
\end{tabular}
\end{center}
%
If |\jobname| does not match the argument \textit{main} of |\childdocmain|,
it is assumed that |\jobname| points to the child file to be compiled.
When using |\childdocmain| with the main file specified as argument,
it suffices to start a child file
with just |\input{|\textit{main}|}|
without loading of the package and using |\childdocof|.
If instead all processing is done
with the appropriate \textsf{childdoc} directives,
the argument of \textit{main} of |\childdocmain| can be empty.

An alternative version of the command line processing described
in \secref{sec:commandline} using the detection mechanism reads:
%
\begin{center}
|... -jobname "|\textit{target}|" "|[\textit{flags}]%
[|\def\jobname{|\textit{dest}|}|]|\input{|\textit{main}|}"|
\end{center}

%%%%%%%%%%%%%%%%%%%%%%%%%%%%%%%%%%%%%%%%%%%%%%%%%%%%%%%%%%%%%%%%%%%%%%%%%%%%%%%%
\subsection{Manual Code}
\label{sec:manual}

In case one cannot be certain whether the definitions file |childdoc.def|
is installed on the target \TeX{} distribution
and one prefers not to ship it,
it is conceivable to paste a few relevant commands into the sources.

To that end, drop all statements |\input{childdoc.def}|
and perform the replacements as outlined below.
Instead of |\childdocmain{|\textit{main}|}| add the following code
to the top of the main file:
%
\begin{center}
\begin{tabular}{l}
|\||ifdefined\childdocname\endinput\||fi\newif\ifchilddoc|\\
|\edef\childdocname{\scantokens\expandafter{\jobname\noexpand}}|\\
|\def\childdocmain{|\textit{main}|}\||ifx\childdocmain\childdocname\||else|\\
|\childdoctrue\includeonly{\childdocname}\let\jobname\childdocmain\||fi|\\
\end{tabular}
\end{center}
%
Instead of |\childdocof{|\textit{main}|}| just include the main file
at the top of each child file:
%
\begin{center}
|\input{|\textit{main}|}|
\end{center}
%
A simple redirection |\childdocforward{|\textit{dest}|}| is achieved by:
%
\begin{center}
|\def\jobname{|\textit{dest}|}\input{\jobname}|
\end{center}
%
The redirection with prefix
|\childdocforwardprefix[|\textit{prefix}|]{|\textit{dest}|}|
is accomplished by:
%
\begin{center}
\begin{tabular}{l}
|{\edef\jobname{\scantokens\expandafter{\jobname\noexpand}}|\\
|\def\redirectjob |\textit{prefix}|#1~~~{\gdef\jobname{|\textit{dest}|#1}}|\\
|\expandafter\redirectjob\jobname~~~}\input{\jobname}|
\end{tabular}
\end{center}

In an alternative approach,
child documents can be compiled by a specific command line
without additional code or specific definitions:
%
\begin{center}
|... -jobname "|\textit{target}|" "|[\textit{flags}]%
|\includeonly{|\textit{dest}|}\input{|\textit{main}|}"|
\end{center}
%

%%%%%%%%%%%%%%%%%%%%%%%%%%%%%%%%%%%%%%%%%%%%%%%%%%%%%%%%%%%%%%%%%%%%%%%%%%%%%%%%
%%%%%%%%%%%%%%%%%%%%%%%%%%%%%%%%%%%%%%%%%%%%%%%%%%%%%%%%%%%%%%%%%%%%%%%%%%%%%%%%
\section{Information}

%%%%%%%%%%%%%%%%%%%%%%%%%%%%%%%%%%%%%%%%%%%%%%%%%%%%%%%%%%%%%%%%%%%%%%%%%%%%%%%%
\subsection{Copyright}

Copyright \copyright{} 2017--2018 Niklas Beisert

This work may be distributed and/or modified under the
conditions of the \LaTeX{} Project Public License, either version 1.3
of this license or (at your option) any later version.
The latest version of this license is in
  \url{http://www.latex-project.org/lppl.txt}
and version 1.3 or later is part of all distributions of \LaTeX{}
version 2005/12/01 or later.

This work has the LPPL maintenance status `maintained'.

The Current Maintainer of this work is Niklas Beisert.

This work consists of the files |README.txt|, |childdoc.ins| and |childdoc.dtx|
as well as the derived files |childdoc.def|, |cdocsamp.tex|
with |cdocsch1.tex|, |cdocsch2.tex|, |cdocspt3.tex|, |cdocspt4.tex|,
|cdocsdrf.tex|, |cdocsfn1.tex|, |cdocsfn2.tex|
as well as |childdoc.pdf|.

%%%%%%%%%%%%%%%%%%%%%%%%%%%%%%%%%%%%%%%%%%%%%%%%%%%%%%%%%%%%%%%%%%%%%%%%%%%%%%%%
\subsection{Files and Installation}

The package consists of the files:
%
\begin{center}
\begin{tabular}{ll}
    |README.txt|   & readme file \\
    |childdoc.ins| & installation file \\
    |childdoc.dtx| & source file \\
    |childdoc.def| & definition file \\
    |cdocsamp.tex| & sample main file \\
    |cdocsch1.tex| & sample include file \\
    |cdocsch2.tex| & sample include file \\
    |cdocspt3.tex| & sample part file \\
    |cdocspt4.tex| & sample part file \\
    |cdocsdrf.tex| & sample redirection file \\
    |cdocsfn1.tex| & sample redirection file \\
    |cdocsfn2.tex| & sample redirection file \\
    |childdoc.pdf| & manual
\end{tabular}
\end{center}
%
The distribution consists of the files
|README.txt|, |childdoc.ins| and |childdoc.dtx|.
%
\begin{itemize}
\item
Run (pdf)\LaTeX{} on |childdoc.dtx|
to compile the manual |childdoc.pdf| (this file).
\item
Run \LaTeX{} on |childdoc.ins| to create the definitions file |childdoc.def|
and the sample |cdocsamp.tex| with include files
|cdocsch1.tex|, |cdocsch2.tex|, |cdocspt3.tex|, |cdocspt4.tex|,
|cdocsdrf.tex|, |cdocsfn1.tex|, |cdocsfn2.tex|.
Then copy the file |childdoc.def| to an appropriate directory of your \LaTeX{}
distribution, e.g.\ \textit{texmf-root}|/tex/latex/childdoc|.
\end{itemize}

%%%%%%%%%%%%%%%%%%%%%%%%%%%%%%%%%%%%%%%%%%%%%%%%%%%%%%%%%%%%%%%%%%%%%%%%%%%%%%%%
\subsection{Related CTAN Packages}

There are several other packages which offer a similar functionality:
%
\begin{itemize}
\item
The packages
\href{http://ctan.org/pkg/docmute}{\textsf{docmute}},
\href{http://ctan.org/pkg/includex}{\textsf{includex}} and
\href{http://ctan.org/pkg/standalone}{\textsf{standalone}}
provide commands to include only the document body of
a child file thus allowing both files to be compiled individually.
\item
The packages \href{http://ctan.org/pkg/subdocs}{\textsf{subdocs}}
and \href{http://ctan.org/pkg/subfiles}{\textsf{subfiles}}
provide structures in which the main and child documents can be
encapsulated and allowing them to be compiled individually.
The inclusion mechanism is different from the conventional |\include|.
\item
The package \href{http://ctan.org/pkg/combine}{\textsf{combine}}
is an elaborate solution to combine several documents into one.
\end{itemize}
%
See also the CTAN topic \href{http://ctan.org/topic/subdocs}{\textsf{subdocs}}
for further related packages.
The present package differs from the above solutions in that
a document structure constructed with the conventional |\include| mechanism
just needs two extra commands at the top of every file
such that all constituent files can be compiled individually.

%%%%%%%%%%%%%%%%%%%%%%%%%%%%%%%%%%%%%%%%%%%%%%%%%%%%%%%%%%%%%%%%%%%%%%%%%%%%%%%%
%\subsection{Feature Suggestions}
%
%The following is a list of features which may be useful for future
%versions of this package:
%%
%\begin{itemize}
%\item
%\ldots
%\end{itemize}

%%%%%%%%%%%%%%%%%%%%%%%%%%%%%%%%%%%%%%%%%%%%%%%%%%%%%%%%%%%%%%%%%%%%%%%%%%%%%%%%
\subsection{Revision History}

%%%%%%%%%%%%%%%%%%%%%%%%%%%%%%%%%%%%%%%%
\paragraph{v2.0:} 2018/12/30

\begin{itemize}
\item
immediate forward processing
\item
added |\childdocby| mechanism
\item
manual restructured
\end{itemize}

%%%%%%%%%%%%%%%%%%%%%%%%%%%%%%%%%%%%%%%%
\paragraph{v1.6:} 2018/01/17

\begin{itemize}
\item
application for development of include files
\item
corrections to manual
\end{itemize}

%%%%%%%%%%%%%%%%%%%%%%%%%%%%%%%%%%%%%%%%
\paragraph{v1.5:} 2017/05/21

\begin{itemize}
\item
more complete structuring introduced
\item
|\childdocof| introduced
\item
|\childdoc| renamed to |\childdocmain|
\item
|\childredirect| renamed to |\childdocforward| and |\childdocforwardprefix|
and functionality expanded
\end{itemize}

%%%%%%%%%%%%%%%%%%%%%%%%%%%%%%%%%%%%%%%%
\paragraph{v1.0:} 2017/04/27

\begin{itemize}
\item
manual and install package
\item
first version published on CTAN
\end{itemize}

%%%%%%%%%%%%%%%%%%%%%%%%%%%%%%%%%%%%%%%%
\paragraph{v0.6:} 2017/04/26

\begin{itemize}
\item
redirection mechanism added
\end{itemize}

%%%%%%%%%%%%%%%%%%%%%%%%%%%%%%%%%%%%%%%%
\paragraph{v0.5:} 2017/04/26

\begin{itemize}
\item
functionality in definition file
\end{itemize}


%%%%%%%%%%%%%%%%%%%%%%%%%%%%%%%%%%%%%%%%%%%%%%%%%%%%%%%%%%%%%%%%%%%%%%%%%%%%%%%%
%%%%%%%%%%%%%%%%%%%%%%%%%%%%%%%%%%%%%%%%%%%%%%%%%%%%%%%%%%%%%%%%%%%%%%%%%%%%%%%%
%%%%%%%%%%%%%%%%%%%%%%%%%%%%%%%%%%%%%%%%%%%%%%%%%%%%%%%%%%%%%%%%%%%%%%%%%%%%%%%%
\appendix

\settowidth\MacroIndent{\rmfamily\scriptsize 000\ }

 \DocInput{childdoc.dtx}

\end{document}
%</driver>
% \fi
%
% %%%%%%%%%%%%%%%%%%%%%%%%%%%%%%%%%%%%%%%%%%%%%%%%%%%%%%%%%%%%%%%%%%%%%%%%%%%%%%
% %%%%%%%%%%%%%%%%%%%%%%%%%%%%%%%%%%%%%%%%%%%%%%%%%%%%%%%%%%%%%%%%%%%%%%%%%%%%%%
% \section{Sample}
%\iffalse
%<*samplemain>
%\fi
%
% The following presents a sample document
% with two chapters, two parts, a title page,
% a compile flag as well as three forwarding files to set the flag.
% It consists of eight |.tex| files:
% \begin{center}
% \begin{tabular}{ll}
% |cdocsamp.tex|&main file\\
% |cdocsch1.tex|&include file for chapter 1\\
% |cdocsch2.tex|&include file for chapter 2\\
% |cdocspt3.tex|&include file for part 3\\
% |cdocspt4.tex|&include file for part 4\\
% |cdocsdrf.tex|&forwarding file for main file in draft mode\\
% |cdocsfi1.tex|&forwarding file for final version of chapter 1\\
% |cdocsfi2.tex|&forwarding file for final version of chapter 2\\
% \end{tabular}
% \end{center}
% Each of the eight files can be compiled directly by the \LaTeX{} compiler.
%
% %%%%%%%%%%%%%%%%%%%%%%%%%%%%%%%%%%%%%%
% \paragraph{Main File.}
%
% The main file is called |cdocsamp.tex|.
%
% Load the \textsf{childdoc} definitions and
% declare the filename for the main document:
%    \begin{macrocode}
\input{childdoc.def}
\childdocmain{}
%    \end{macrocode}

% Optional override for |\version| flag:
%    \begin{macrocode}
%%\ifchilddoc\else\providecommand{\version}{draft}\fi
%    \end{macrocode}

% Define the default values for the |\version| flag
% (|final| for the main file and |draft| for childs):
%    \begin{macrocode}
\ifchilddoc
\providecommand{\version}{draft}
\else
\providecommand{\version}{final}
\fi
%    \end{macrocode}

% Load the standard document class:
%    \begin{macrocode}
\documentclass[12pt]{article}
%    \end{macrocode}

% Start the document body:
%    \begin{macrocode}
\begin{document}
%    \end{macrocode}

% Declare a title page.
% Print title, part of document being processed and version flag:
%    \begin{macrocode}
\addtocounter{page}{-1}
\begin{center}
{\LARGE\bfseries{}childdoc example\par}
\vspace{1cm}
\ifchilddoc
\ifchilddocmanual part\else chapter\fi:
`\childdocname' of `\childdocjob'\par
\else
main document: `\childdocjob'\par
\fi
version: \version\par
\end{center}
\newpage
%    \end{macrocode}

% Manually include selected file,
% otherwise process as usual:
%    \begin{macrocode}
\ifchilddocmanual
\section*{part `\childdocname'}
\input{\childdocname}
\else
%    \end{macrocode}

% Include the two chapters:
%    \begin{macrocode}
\include{cdocsch1}
\include{cdocsch2}
%    \end{macrocode}

% Include the two parts unless only chapters should be displayed:
%    \begin{macrocode}
\ifchilddoc\else
\section{part three}
\input{cdocspt3}
\section{part four}
\input{cdocspt4}
\fi
%    \end{macrocode}

% Process as usual until here:
%    \begin{macrocode}
\fi
%    \end{macrocode}

% End of document body:
%    \begin{macrocode}
\end{document}
%    \end{macrocode}
%\iffalse
%</samplemain>
%\fi
%
% %%%%%%%%%%%%%%%%%%%%%%%%%%%%%%%%%%%%%%
% \paragraph{Chapter Include Files.}
%
% The include files are called |cdocsch1.tex| and |cdocsch2.tex|.
%
%\iffalse
%<*samplechap1|samplechap2>
%\fi

% Optional override for |\version| flag:
%    \begin{macrocode}
%%\providecommand{\version}{final}
%    \end{macrocode}

% Include the main document:
%    \begin{macrocode}
\input{childdoc.def}
\childdocof{cdocsamp}
%    \end{macrocode}

%\iffalse
%</samplechap1|samplechap2>
%\fi
%
%\iffalse
%<*samplechap1>
%\fi
% Some text for chapter 1:
%    \begin{macrocode}
\section{one}
some text in chapter one
%    \end{macrocode}

%\iffalse
%</samplechap1>
%\fi
% Some text for chapter 2:
%\iffalse
%<*samplechap2>
%\fi
%    \begin{macrocode}
\section{two}
more text in chapter two
%    \end{macrocode}

%\iffalse
%</samplechap2>
%\fi
%
% %%%%%%%%%%%%%%%%%%%%%%%%%%%%%%%%%%%%%%
% \paragraph{Part Include Files.}
%
% The include files are called |cdocspt3.tex| and |cdocspt4.tex|.
%
%\iffalse
%<*samplepart3|samplepart4>
%\fi

% Optional override for |\version| flag:
%    \begin{macrocode}
%%\providecommand{\version}{final}
%    \end{macrocode}

% Include the main document:
%    \begin{macrocode}
\input{childdoc.def}
\childdocby{cdocsamp}
%    \end{macrocode}

%\iffalse
%</samplepart3|samplepart4>
%\fi
%
%\iffalse
%<*samplepart3>
%\fi
% Some text for part 3:
%    \begin{macrocode}
some text in part three
%    \end{macrocode}

%\iffalse
%</samplepart3>
%\fi
% Some text for part 4:
%\iffalse
%<*samplepart4>
%\fi
%    \begin{macrocode}
more text in part four
%    \end{macrocode}

%\iffalse
%</samplepart4>
%\fi
%
% %%%%%%%%%%%%%%%%%%%%%%%%%%%%%%%%%%%%%%
% \paragraph{Forwarding for a Complete Draft.}
%
% The following forwarding file |cdocsdrf.tex|
% compiles the main document in draft mode:
%\iffalse
%<*sampledraft>
%\fi
%    \begin{macrocode}
\def\version{draft}
\input{childdoc.def}
\childdocforward{cdocsamp}
%    \end{macrocode}

%\iffalse
%</sampledraft>
%\fi
%
% %%%%%%%%%%%%%%%%%%%%%%%%%%%%%%%%%%%%%%
% \paragraph{Forwarding for Final Version of the Chapters.}
%
% The following forwarding files |cdocsfn1.tex| and |cdocsfn2.tex|
% (with identical content)
% compile the final versions of the child documents
% |cdocsch1.tex| and |cdocsch2.tex|, respectively:
%\iffalse
%<*samplefinal>
%\fi
%    \begin{macrocode}
\def\version{final}
\input{childdoc.def}
\childdocforwardprefix[cdocsamp]{cdocsfn}{cdocsch}
%    \end{macrocode}

%\iffalse
%</samplefinal>
%\fi
%
% %%%%%%%%%%%%%%%%%%%%%%%%%%%%%%%%%%%%%%
% \paragraph{Command Line Processing.}
%
% The following three command lines generate the output files
% |cdocscld|, |cdocscl1| and |cdocscl2|
% which should be identical to
% |cdocsdrf|, |cdocsch1| and |cdocsfn2|, respectively:
% \begin{center}
% \begin{tabular}{l}
% |latex -jobname cdocscld \|\\
% |  "\def\version{draft}\input{childdoc.def}\childdocforward{cdocsamp}"|\\
% |latex -jobname cdocscl1 \|\\
% |  "\input{childdoc.def}\childdocforward[cdocsamp]{cdocsch1}"|\\
% |latex -jobname cdocscl2 \|\\
% |  "\def\version{final}\input{childdoc.def}\childdocforward{cdocsch2}"|
% \end{tabular}
% \end{center}
% Note that the trailing backslash on each first line
% merely continues the input to the second line
% (for convenient cut ant paste).
% Furthermore, the command |latex| can be replaced by any
% of its alternative versions such as |pdflatex|.
%
% %%%%%%%%%%%%%%%%%%%%%%%%%%%%%%%%%%%%%%%%%%%%%%%%%%%%%%%%%%%%%%%%%%%%%%%%%%%%%%
% %%%%%%%%%%%%%%%%%%%%%%%%%%%%%%%%%%%%%%%%%%%%%%%%%%%%%%%%%%%%%%%%%%%%%%%%%%%%%%
% \section{Implementation}
%\iffalse
%<*package>
%\fi
%
% This section describes the definitions file |childdoc.def|.

% The definitions cannot be loaded using |\usepackage| or |\RequirePackage|
% which has a mechanism to prevent loading a style file more than once.
% When loading the definitions by means of |\input|
% multiple instances have to be prevented manually:
%\iffalse
%This code needs to be before the `\ProvidesFile' directive
%which is defined at the beginning of this file.
%Therefore it is also placed there and commented out here.
%</package>
%<*discard>
%\fi
%    \begin{macrocode}
\ifdefined\childdocmain\endinput\fi
%    \end{macrocode}
%\iffalse
%</discard>
%<*package>
%\fi
%
% \macro{\ifchilddoc}
% \macro{\ifchilddocmanual}
% The conditional |\ifchilddoc| tells whether a
% child (true) or main (false) document is being compiled.
% The conditional |\ifchilddocmanual| tells whether
% the |\includeonly| mechanism is used (false) or
% the selection of child files must be performed manually (true).
% The definitions initialise to false:
%    \begin{macrocode}
\newif\ifchilddoc
\newif\ifchilddocmanual
%    \end{macrocode}

% \macro{\childdocname}
% \macro{\childdocjob}
% The macro |\childdocname| stores the name of the main document
% to be compiled. The macro |\childdocjob| stores the name of
% the document on which the \LaTeX{} compiler was originally invoked.
% The content of |\jobname| cannot be compared
% to filenames specified in the source due to different catcodes.
% The following code rescans |\jobname|, stores the result
% in |\childdocname| and saves a copy in |\childdocjob|:
%    \begin{macrocode}
\edef\childdocname{\scantokens\expandafter{\jobname\noexpand}}
\let\childdocjob\childdocname
%    \end{macrocode}

% \macro{\childdocdisable}
% The macro |\childdocdisable| prevents the main file
% from being processed more than once.
% At this stage, the main document command |\childdocmain|
% is assumed to be called once again where it should do nothing.
% Any subsequent call to it should prevent
% a secondary processing of the main document
% It overwrites the forwarding commands
% |\childdocof| and |\childdocforward|
% with empty macros to prevent further inclusions of the main document:
%    \begin{macrocode}
\newcommand{\childdocdisable}
{
  \renewcommand{\childdocmain}[1]{\renewcommand{\childdocmain}[1]{\endinput}}
  \renewcommand{\childdocof}[1]{}
  \renewcommand{\childdocby}[2][]{}
  \renewcommand{\childdocforward}[2][]{}
  \renewcommand{\childdocdisable}{}
}
%    \end{macrocode}

% \macro{\childdocmain}
% The macro |\childdocmain| is to be called at the top of the main file
% with nothing or the main filename (without extension) as argument.
% First, it breaks loops.
% If the argument is not empty and does not match |\childdocname|
% (which is set by the first inclusion of |childdoc.def|),
% |\ifchilddoc| is set to true, |\includeonly| is applied to the child file
% and |\jobname| is set to the main file
% (for proper handling of |.aux| files):
%    \begin{macrocode}
\newcommand{\childdocmain}[1]
{
  \childdocdisable\childdocmain{}
  \if?#1?\else
    \begingroup
      \def\childdoctmp{#1}
      \ifx\childdoctmp\childdocname
        \def\childdoctmp{}
      \else
        \def\childdoctmp
        {
          \childdoctrue
          \includeonly{\childdocname}
          \def\childdocjob{#1}
          \def\jobname{#1}
        }
      \fi
      \expandafter
    \endgroup
    \childdoctmp
  \fi
}
%    \end{macrocode}

% \macro{\childdocof}
% The command |\childdocof| redirects
% compilation to the main file |#1|.
%    \begin{macrocode}
\newcommand{\childdocof}[1]
{
  \childdocdisable
  \childdoctrue
  \includeonly{\childdocname}
  \def\jobname{#1}
  \def\childdocjob{#1}
  \input{#1}
}
%    \end{macrocode}

% \macro{\childdocby}
% The command |\childdocby| ....
%    \begin{macrocode}
\newcommand{\childdocby}[2][]
{
  \childdocdisable
  \childdoctrue
  \childdocmanualtrue
  \if?#1?\else
    \def\jobname{#2}
  \fi
  \def\childdocjob{#2}
  \input{#2}
  \endinput
}
%    \end{macrocode}

% \macro{\childdocforward}
% The command |\childdocforward| redirects
% compilation to the main file or
% (if the optional argument is given) a child file.
% Parameters are set as if the main file
% or a child file starting with |\childdocof| was compiled.
% Then compilation is handed over to the main file:
%    \begin{macrocode}
\newcommand{\childdocforward}[2][]
{
  \begingroup
    \if?#1?
      \def\childdoctmp
      {
        \def\childdocname{#2}
        \def\childdocjob{#2}
        \def\jobname{#2}
        \input{#2}
        \endinput
      }
    \else
      \def\childdoctmp
      {
        \childdocdisable
        \def\childdocname{#2}
        \childdoctrue
        \includeonly{#2}
        \def\childdocjob{#1}
        \def\jobname{#1}
        \input{#1}
        \endinput
      }
    \fi
    \expandafter
  \endgroup
  \childdoctmp
}
%    \end{macrocode}

% \macro{\childdocforwardprefix}
% The command |\childdocforwardprefix| redirects
% compilation to the main or a child file by means of a pattern.
% The prefix |#1| in the current filename is replaced by |#2|
% and the suffix of the current filename is kept
% (it is assumed that the filename does not contain the substring `|~~~|'
% which is used as a delimiter).
% Compilation is handed over to the new file by |\childdocforward|:
%    \begin{macrocode}
\newcommand{\childdocforwardprefix}[3][]
{
  \begingroup
    \def\childdocextract #2##1~~~{\def\childdoctmp{\childdocforward[#1]{#3##1}}}
    \expandafter\childdocextract\childdocname~~~
    \expandafter
  \endgroup
  \childdoctmp
}
%    \end{macrocode}

% \macro{\childdoc}
% The deprecated macro |\childdoc| is a legacy version of |\childdocmain|:
%    \begin{macrocode}
\newcommand{\childdoc}{\childdocmain}
%    \end{macrocode}

% \macro{\childdocredirect}
% The deprecated macro |\childdocredirect| is a legacy version
% of |\childdocforward| and |\childdocforwardprefix|:
%    \begin{macrocode}
\newcommand{\childdocredirect}[2][]
{
  \begingroup
    \if?#1?
      \def\childdoctmp{\childdocforward{#2}}
    \else
      \def\childdoctmp{\childdocforwardprefix{#1}{#2}}
    \fi
    \expandafter
  \endgroup
  \childdoctmp
}
%    \end{macrocode}

%\iffalse
%</package>
%\fi
%
\endinput

\childdocforwardprefix[cdocsamp]{cdocsfn}{cdocsch}
%    \end{macrocode}

%\iffalse
%</samplefinal>
%\fi
%
% %%%%%%%%%%%%%%%%%%%%%%%%%%%%%%%%%%%%%%
% \paragraph{Command Line Processing.}
%
% The following three command lines generate the output files
% |cdocscld|, |cdocscl1| and |cdocscl2|
% which should be identical to
% |cdocsdrf|, |cdocsch1| and |cdocsfn2|, respectively:
% \begin{center}
% \begin{tabular}{l}
% |latex -jobname cdocscld \|\\
% |  "\def\version{draft}% \iffalse
%
% childdoc.dtx Copyright (C) 2017-2018 Niklas Beisert
%
% This work may be distributed and/or modified under the
% conditions of the LaTeX Project Public License, either version 1.3
% of this license or (at your option) any later version.
% The latest version of this license is in
%   http://www.latex-project.org/lppl.txt
% and version 1.3 or later is part of all distributions of LaTeX
% version 2005/12/01 or later.
%
% This work has the LPPL maintenance status `maintained'.
%
% The Current Maintainer of this work is Niklas Beisert.
%
% This work consists of the files childdoc.dtx and childdoc.ins
% and the derived files childdoc.def and cdocsamp.tex with
% cdocsch1.tex, cdocsch2.tex, cdocsdrf.tex, cdocsfn1.tex, cdocsfn2.tex.
%
%<package>\ifdefined\childdocmain\endinput\fi
%<package>\ProvidesFile{childdoc.def}[2018/12/30 v2.0 child document driver]
%<samplemain>\ProvidesFile{cdocsamp.tex}[2018/12/30 v2.0 sample for childdoc]
%<*driver>
%\ProvidesFile{childdoc.drv}[2018/12/30 v2.0 childdoc reference manual file]
\PassOptionsToClass{10pt,a4paper}{article}
\documentclass{ltxdoc}

\usepackage[margin=35mm]{geometry}
\usepackage{hyperref}
\usepackage{hyperxmp}
\usepackage[usenames]{color}

\hypersetup{colorlinks=true}
\hypersetup{pdfstartview=FitH}
\hypersetup{pdfpagemode=UseNone}
\hypersetup{pdfsource={}}
\hypersetup{pdflang={en-UK}}
\hypersetup{pdfcopyright={Copyright 2017-2018 Niklas Beisert.
  This work may be distributed and/or modified under the
  conditions of the LaTeX Project Public License, either version 1.3
  of this license or (at your option) any later version.}}
\hypersetup{pdflicenseurl={http://www.latex-project.org/lppl.txt}}
\hypersetup{pdfcontactaddress={ETH Zurich, ITP, HIT K,
  Wolfgang-Pauli-Strasse 27}}
\hypersetup{pdfcontactpostcode={8093}}
\hypersetup{pdfcontactcity={Zurich}}
\hypersetup{pdfcontactcountry={Switzerland}}
\hypersetup{pdfcontactemail={nbeisert@itp.phys.ethz.ch}}
\hypersetup{pdfcontacturl={http://people.phys.ethz.ch/\xmptilde nbeisert/}}

\newcommand{\secref}[1]{\hyperref[#1]{section \ref*{#1}}}

\parskip1ex
\parindent0pt
\let\olditemize\itemize
\def\itemize{\olditemize\parskip0pt}

\begin{document}

\title{The \textsf{childdoc} Package}
\hypersetup{pdftitle={The childdoc Package}}
\author{Niklas Beisert\\[2ex]
  Institut f\"ur Theoretische Physik\\
  Eidgen\"ossische Technische Hochschule Z\"urich\\
  Wolfgang-Pauli-Strasse 27, 8093 Z\"urich, Switzerland\\[1ex]
  \href{mailto:nbeisert@itp.phys.ethz.ch}
  {\texttt{nbeisert@itp.phys.ethz.ch}}}
\hypersetup{pdfauthor={Niklas Beisert}}
\hypersetup{pdfsubject={Manual for the LaTeX2e Package childdoc}}
\date{30 December 2018, \textsf{v2.0}}
\maketitle

\begin{abstract}\noindent
\textsf{childdoc} is a \LaTeXe{} package
that enables the direct compilation
of document sections included by |\include|
to individual files.
\end{abstract}

\begingroup
\parskip0ex
\tableofcontents
\endgroup

%%%%%%%%%%%%%%%%%%%%%%%%%%%%%%%%%%%%%%%%%%%%%%%%%%%%%%%%%%%%%%%%%%%%%%%%%%%%%%%%
%%%%%%%%%%%%%%%%%%%%%%%%%%%%%%%%%%%%%%%%%%%%%%%%%%%%%%%%%%%%%%%%%%%%%%%%%%%%%%%%
\section{Introduction}

\LaTeX{} provides a mechanism to structure a large document (such as a book)
into a main file and several child files (containing the chapters)
using the |\include| command.
This mechanism is beneficial for documents
which span hundreds of pages in order to
make the source file(s) more manageable.
Moreover, compilation can be restricted to
selected child files by means of the |\includeonly| command.
The latter feature can be used to reduce the compilation time while editing
(this was significantly more useful in the earlier days of \LaTeX{})
or to generate a smaller document which is easier to navigate.
Another application of |\includeonly| is to generate
documents consisting of selected parts of the complete document.

However, there are a few drawbacks of the plain |\include| mechanism:
\begin{itemize}
\item
The child files cannot be compiled on their own,
they can only be compiled via the main file.
A naive editing environment
(such as a text editor with an option
to have the current file processed by \LaTeX)
may require one to switch to the main file before compiling;
attempting to compile the child file produces errors.
\item
The main file must be modified (each time)
to adjust the |\includeonly| command
to the present needs. This easily leaves the main file in a messy state.
\item
The generated document will always carry the filename
of the main document. This is inconvenient if
several child files are to be compiled and
to be kept for distribution.
\end{itemize}

The present package provides a simple interface
to make child files individually compilable by \LaTeX{}.
Compiling a child file then has the same effect as compiling
the main file with an |\includeonly| command
to select the appropriate child.
Moreover the generated document will carry the name of the child
rather than the main file.
This resolves all three above issues.

This feature is meant to make the editing of books,
thesis documents and lecture notes somewhat more convenient.
However, the package can also be used efficiently for
composing a series of documents (such as exercise sheets)
which are typically distributed individually.
It then assists the author in generating the individual documents
(potentially in different versions)
as well as a document containing the collected series.
Another application is in developing style files
or other kinds of included material
where compilation of the style file could redirect
to a sample or test file.

%%%%%%%%%%%%%%%%%%%%%%%%%%%%%%%%%%%%%%%%%%%%%%%%%%%%%%%%%%%%%%%%%%%%%%%%%%%%%%%%
%%%%%%%%%%%%%%%%%%%%%%%%%%%%%%%%%%%%%%%%%%%%%%%%%%%%%%%%%%%%%%%%%%%%%%%%%%%%%%%%
\section{Usage}

First of all, the package \textsf{childdoc} is \emph{not} a standard
\LaTeXe{} |.sty| style file! Therefore it needs to be invoked in
a non-standard way.

%%%%%%%%%%%%%%%%%%%%%%%%%%%%%%%%%%%%%%%%%%%%%%%%%%%%%%%%%%%%%%%%%%%%%%%%%%%%%%%%
\subsection{Included Files}
\label{sec:include}

%%%%%%%%%%%%%%%%%%%%%%%%%%%%%%%%%%%%%%%%
\DescribeMacro{\childdocmain}
To use the package, add the commands
\begin{center}
\begin{tabular}{l}
|\input{childdoc.def}|\\
|\childdocmain{}|\\
\end{tabular}
\end{center}
at the very top of the main \LaTeX{} file,
in particular \emph{before} the |\documentclass| statement!
The argument of |\childdocmain| should be left empty
(but it must be present).

%%%%%%%%%%%%%%%%%%%%%%%%%%%%%%%%%%%%%%%%
\DescribeMacro{\childdocof}
Furthermore, add the commands
\begin{center}
\begin{tabular}{l}
|\input{childdoc.def}|\\
|\childdocof{|\textit{main}|}|\\
\end{tabular}
\end{center}
at the top of every child file \textit{child}
which is included by |\include{|\textit{child}|}|
from within the main file
(or at least for those files to be compiled individually).
The argument \textit{main} must be the filename of the main file.

There are a couple of
considerations in setting up the main and child documents:

%%%%%%%%%%%%%%%%%%%%%%%%%%%%%%%%%%%%%%%%
\paragraph{Restrictions.}

Please note the following restrictions:
\begin{itemize}
\item
|\childdocmain| must be called with one argument \textit{main}
to ensure compatibility with earlier version of the package.
It must either be empty (|\childdocmain{}|)
or precisely match the filename of the main file in which it is specified.
See \secref{sec:detection} for further information.
\item
The filename \textit{main} must be specified without the |.tex| extension.
\item
The filename \textit{main} is case sensitive
(even in case-insensitive file systems)
due to internal string comparison.
\item
The argument \textit{main} should be fully expanded, it cannot be a macro.
\item
Subdirectories and special characters should be avoided in filenames.
\item
The command |\childdocmain{|\textit{main}|}| must be followed by a whitespace.
It should not be followed immediately by another command
or by a comment mark `|%|'.
This is because the \TeX{} parser reads the token immediately following
the argument of |\childdocmain| and puts it
at the beginning of every child section;
however, a white\-space is ignored.
\end{itemize}

%%%%%%%%%%%%%%%%%%%%%%%%%%%%%%%%%%%%%%%%
\paragraph{Content of Main File.}

It is advisable to place all content in the child files included by |\include|.
Any output contained in the main file will appear in all child documents
unless suppressed manually;
it cannot be suppressed automatically by the |\includeonly| directive
and thus should normally be avoided.
A method to include some content in the main file
by means of conditional processing is described in \secref{sec:conditional}.

%%%%%%%%%%%%%%%%%%%%%%%%%%%%%%%%%%%%%%%%
\paragraph{Page Numbering.}

When only a part of the document is compiled,
the appropriate numbering of pages
(as well as other status parameters)
is determined from the |.aux| files.
The latter contain information from previous passes.
However this information needs to propagate through
all intermediate child documents.
Therefore the page numbering in child documents may well
be inconsistent until the complete document is compiled at least once.

A useful (if unconventional) way to always ensure a consistent
page numbering is to restart the numbering in each child document
and denote the pages by `\textit{child}|.|\textit{page}'
where \textit{child} represents the chapter/section number of the child file.
This can be achieved by the command
|\numberwithin{page}{|\textit{child}|}|
of the \textsf{amsmath} package
where \textit{child} can be |chapter| or |section|
depending on the chosen structuring.
Alternatively, one can modify the macro |\thepage| appropriately
and reset the counter |page| at the start of each child file.

%%%%%%%%%%%%%%%%%%%%%%%%%%%%%%%%%%%%%%%%%%%%%%%%%%%%%%%%%%%%%%%%%%%%%%%%%%%%%%%%
\subsection{Conditional Processing}
\label{sec:conditional}

The package provides a mechanism to compile different versions
of a document. To customise the versions further some conditional processing
can come in handy to distinguish which version is being compiled.
The package provides two macros to describe the compilation context:

%%%%%%%%%%%%%%%%%%%%%%%%%%%%%%%%%%%%%%%%
\DescribeMacro{\ifchilddoc}
The conditional |\ifchilddoc| distinguishes between the compilation of
child documents and the main document:
%
\begin{center}
|\ifchilddoc |\textit{child-code}| |[|\||else |\textit{main-code}]| \||fi|
\end{center}

%%%%%%%%%%%%%%%%%%%%%%%%%%%%%%%%%%%%%%%%
\DescribeMacro{\childdocname}
\DescribeMacro{\childdocjob}
The macro |\childdocname| contains the filename (without extension)
of the main or child file being processed.
Note that |\childdocjob| will always contain the name of the main file.

%%%%%%%%%%%%%%%%%%%%%%%%%%%%%%%%%%%%%%%%
\paragraph{Title Page.}

Conditional processing can be used to include a title or banner page
in the main document when proper precautions are taken.
Importantly, the code in the main file should ensure that the page counter
(as well as other status parameters which are stored in the |.aux| files)
takes the same value after the conditional processing.
Otherwise the page numbers may take divergent values
depending on which part is compiled.

For example, a title page could be declared by:
%
\begin{center}
\begin{tabular}{l}
|\ifchilddoc\||else|\\
|\addtocounter{page}{-1}|\\
\textit{code for title page}\\
|\newpage|\\
|\||fi|
\end{tabular}
\end{center}
%
A banner page for the child documents can be generated by:
%
\begin{center}
\begin{tabular}{l}
|\ifchilddoc|\\
|\addtocounter{page}{-1}|\\
\textit{code for banner page}\\
|\newpage|\\
|\||fi|
\end{tabular}
\end{center}
%
Here one could write a message such as:
\begin{center}
|This is the part \childdocname{} of \childdocjob{}.|
\end{center}

%%%%%%%%%%%%%%%%%%%%%%%%%%%%%%%%%%%%%%%%%%%%%%%%%%%%%%%%%%%%%%%%%%%%%%%%%%%%%%%%
\subsection{Flags}
\label{sec:flags}

The package makes it easy to generate different versions
of the main or child documents.
To this end compilation flags can be defined
and assigned different default values.
They will be particularly useful in conjunction
with the forwarding mechanism described in \secref{sec:forward}.

For example, it may be useful to have a flag |\version|
which can be set to |draft| or |final|.
The document source will contain some conditional code
depending on the value of |\version|.
Suppose further, the flag should default to |final| for the main file
and to |draft| for child files
which is a natural assignment for editing the document.
This is achieved by placing the following code
in the preamble of the main document
(below the |\childdocmain| directive):
%
\begin{center}
\begin{tabular}{l}
|\ifchilddoc|\\
|\providecommand{\version}{draft}|\\
|\||else|\\
|\providecommand{\version}{final}|\\
|\||fi|
\end{tabular}
\end{center}
%
The definition by |\providecommand| makes sure
that previous definitions are not overwritten.
Further statements |\providecommand{\version}{...}|
can thus be added before the above code to override it.

For the main file, one might add a line
(between |\childdocmain| and the above block)
%
\begin{center}
|%\ifchilddoc\||else\providecommand{\version}{draft}\||fi|
\end{center}
%
which can be uncommented to produce a draft version.
Likewise one can add a line to the very top of a child file
(above the |\childdocof{|\textit{main}|}| directive)
%
\begin{center}
|%\providecommand{\version}{final}|
\end{center}
%
which can be uncommented to produce the final version of this child document.

%%%%%%%%%%%%%%%%%%%%%%%%%%%%%%%%%%%%%%%%%%%%%%%%%%%%%%%%%%%%%%%%%%%%%%%%%%%%%%%%
\subsection{Forwarding}
\label{sec:forward}

Different versions of the main or child documents
using compilation flags as described in \secref{sec:flags}
can be (permanently) stored in different files
for convenient compilation, viewing and distribution.
To this end, the package defines a command
to pass on compilation to a different file:

%%%%%%%%%%%%%%%%%%%%%%%%%%%%%%%%%%%%%%%%
\DescribeMacro{\childdocforward}
The command |\childdocforward| redirects processing to
another source file:
%
\begin{center}
\begin{tabular}{l}
|\input{childdoc.def}|\\
|\childdocforward[|\textit{main}|]{|\textit{dest}|}|\\
\end{tabular}
\end{center}
%
The argument \textit{dest} is the destination file
(without extension).
It should be the main file or one of the child files.
Note that further \textsf{childdoc} directives
such as |\childdocof| and |\childdocforward|
in the indicated file will be processed in this form.
The optional argument \textit{main}
passes on directly to the main file \textit{main}
while pretending to compile the child \textit{dest}.
This form behaves as if \textit{dest}
issues |\childdocof{|\textit{main}|}| right away,
and no further \textsf{childdoc} directives will be processed.

%%%%%%%%%%%%%%%%%%%%%%%%%%%%%%%%%%%%%%%%
\DescribeMacro{\...prefix}
In the alternative form |\childdocforwardprefix|,
%
\begin{center}
\begin{tabular}{l}
|\input{childdoc.def}|\\
|\childdocforwardprefix[|\textit{main}|]{|\textit{prefix}|}{|\textit{dest}|}|
\end{tabular}
\end{center}
%
the destination file is determined by a pattern
depending on the current file:
To make this work, the current file must be called
`{\textit{prefix}\hspace{0.2em}\textit{suffix}}'
with \textit{prefix} matching precisely the argument.
Processing is then passed on to the file
`{\textit{dest}\hspace{0.2em}\textit{suffix}}'.
Surely, the same effect is achieved by
directly specifying the
argument `{\textit{dest}\hspace{0.2em}\textit{suffix}}'
in the first form.
However, that requires to set up a different file
for each child. With the alternative form of the command
all these files can have exactly the same content
which simplifies setting them up and maintaining them.

For example, the following file |draft.tex|
with a compilation flag |\version| as described in \secref{sec:flags}
compiles the main document as a draft:
%
\begin{center}
\begin{tabular}{l}
|\def\version{draft}|\\
|\input{childdoc.def}|\\
|\childdocforward{|\textit{main}|}|
\end{tabular}
\end{center}
%
Likewise, the following files |final|\textit{nn}|.tex|
compile the final version of the child document
|child|\textit{nn}|.tex|:
%
\begin{center}
\begin{tabular}{l}
|\def\version{final}|\\
|\input{childdoc.def}|\\
|\childdocforwardprefix{final}{child}|
\end{tabular}
\end{center}
%

Note that when several versions of a main file and/or of each child file
are to be generated, it may be convenient to set up a |Makefile| or
shell script to automatise the process.

%%%%%%%%%%%%%%%%%%%%%%%%%%%%%%%%%%%%%%%%%%%%%%%%%%%%%%%%%%%%%%%%%%%%%%%%%%%%%%%%
\subsection{Command Line Processing}
\label{sec:commandline}

The effect of redirection files can also be achieved by invoking
the \LaTeX{} compiler with a more elaborate command line.
Most conveniently this should be done as part
of a shell script or a |Makefile|.

When using \textsf{childdoc} in the main file, the following
command lines effectively perform a redirection
(note that depending on the shell being used,
backslashes may have to be doubled: `|\|' $\to$ `|\\|'):
%
\begin{center}
|... -jobname "|\textit{target}|" |\\|"|[\textit{flags}]%
|\input{childdoc.def}\childdocforward[|\textit{main}|]{|\textit{dest}|}"|
\end{center}
%
Here \textit{target} is the name of the output file,
\textit{main} is the name of the main file
and \textit{dest} is the name of the main or child file to be processed
(all filenames without extensions).
The optional argument \textit{main} can be omitted
if \textit{main} matches \textit{dest}.
Optionally, compilation \textit{flags} can be defined via |\def| commands.
This command line makes the \TeX{} engine believe
it is compiling the file \textit{target}
whose content is specified as the latter parameter.
The provided code then forwards the processing to
\textit{main} or \textit{dest} as described in \secref{sec:forward}.

%%%%%%%%%%%%%%%%%%%%%%%%%%%%%%%%%%%%%%%%%%%%%%%%%%%%%%%%%%%%%%%%%%%%%%%%%%%%%%%%
\subsection{Include by Input}
\label{sec:input}

Including child documents by |\include| has some restrictions by design.
Most notably, the content of a child document always occupies
its own set of pages; pages cannot be shared between child documents.
Usually, this behaviour makes perfect sense
because each child document contain an essential part of the document.
However, in some situations it may be desirable to compose
a document from a collection of parts
without having mandatory page breaks between then.
For this case, the package
provides a mechanism to include parts
by |\input| which can also be processed individually.
However, by construction this mechanism
requires manual handling of the content to be output.

%%%%%%%%%%%%%%%%%%%%%%%%%%%%%%%%%%%%%%%%
\DescribeMacro{\ifchilddocmanual}
The main file should be prepared as usual, see \secref{sec:include}.
However, the document body must make a distinction
between processing of an individual part and of the main document, e.g.:
%
\begin{center}
\begin{tabular}{l}
|\ifchilddocmanual|\\
|\input{\childdocname}|\\
|\||else|\\
\textit{document body with }|\input{|\textit{part}|}|\\
|\||fi|
\end{tabular}
\end{center}
%
The conditional |\ifchilddocmanual| is true whenever
a part to be included by |\input| is being compiled,
and the name of the part is stored in |\childdocname|.

%%%%%%%%%%%%%%%%%%%%%%%%%%%%%%%%%%%%%%%%
\DescribeMacro{\childdocby}
Each part to be included by |\input| should start with:
%
\begin{center}
\begin{tabular}{l}
|\input{childdoc.def}|\\
|\childdocby{|\textit{main}|}|\\
\end{tabular}
\end{center}
%
The directive |\childdocby| is similar to |\childdocof|
described in \secref{sec:include},
but the subsequent selection of content must be done manually.
To that end, both |\ifchilddoc| and |\ifchilddocmanual|
will be true upon processing of a part,
and the name of the part is stored in |\childdocname|.
Note that |\jobname| will be set to the filename of the current part
so that each part receives an individual |.aux| file
that does not interfere with the |.aux| file(s) of the main document.
This behaviour can be altered by the alternative form
|\childdocby[*]{|\textit{main}|}| (with a non-empty optional argument)
which uses the |.aux| file of the main document
by setting |\jobname| to \textit{main}.

%%%%%%%%%%%%%%%%%%%%%%%%%%%%%%%%%%%%%%%%%%%%%%%%%%%%%%%%%%%%%%%%%%%%%%%%%%%%%%%%
\subsection{Driver Development}
\label{sec:driver}

The \textsf{childdoc} mechanism can also be use for the development
of definition files such as \LaTeX{} styles or classes.
This case differs from the above setup with multiple parts
included by |\include| in that no |\includeonly| should be invoked.
This can be achieved by starting the include file
(before |\ProvidesPackage|) with:
%
\begin{center}
\begin{tabular}{l}
|\input{childdoc.def}|\\
|\childdocforward{|\textit{main}|}|\\
\end{tabular}
\end{center}
%
or alternatively with:
%
\begin{center}
\begin{tabular}{l}
|\input{childdoc.def}|\\
|\childdocby{|\textit{main}|}|\\
\end{tabular}
\end{center}
%
Both forms have slightly different effects as described above.
The main file is prepared as usual, see \secref{sec:include}.

%%%%%%%%%%%%%%%%%%%%%%%%%%%%%%%%%%%%%%%%%%%%%%%%%%%%%%%%%%%%%%%%%%%%%%%%%%%%%%%%
\subsection{Legacy Detection}
\label{sec:detection}

The directive |\childdocmain| in the main file can detect
whether the complete document or merely a child is to be compiled
even without using the directive |\childdocof|.
This method is deprecated because it is less robust
and there is no compelling reason to use it;
it is merely provided for backward compatibility
and it may be removed in future versions.

If the detection mechanism is to be used,
it is mandatory to correctly specify
the filename of the main file as the argument of |\childdocmain|:
%
\begin{center}
\begin{tabular}{l}
|\input{childdoc.def}|\\
|\childdocmain{|\textit{main}|}|\\
\end{tabular}
\end{center}
%
If |\jobname| does not match the argument \textit{main} of |\childdocmain|,
it is assumed that |\jobname| points to the child file to be compiled.
When using |\childdocmain| with the main file specified as argument,
it suffices to start a child file
with just |\input{|\textit{main}|}|
without loading of the package and using |\childdocof|.
If instead all processing is done
with the appropriate \textsf{childdoc} directives,
the argument of \textit{main} of |\childdocmain| can be empty.

An alternative version of the command line processing described
in \secref{sec:commandline} using the detection mechanism reads:
%
\begin{center}
|... -jobname "|\textit{target}|" "|[\textit{flags}]%
[|\def\jobname{|\textit{dest}|}|]|\input{|\textit{main}|}"|
\end{center}

%%%%%%%%%%%%%%%%%%%%%%%%%%%%%%%%%%%%%%%%%%%%%%%%%%%%%%%%%%%%%%%%%%%%%%%%%%%%%%%%
\subsection{Manual Code}
\label{sec:manual}

In case one cannot be certain whether the definitions file |childdoc.def|
is installed on the target \TeX{} distribution
and one prefers not to ship it,
it is conceivable to paste a few relevant commands into the sources.

To that end, drop all statements |\input{childdoc.def}|
and perform the replacements as outlined below.
Instead of |\childdocmain{|\textit{main}|}| add the following code
to the top of the main file:
%
\begin{center}
\begin{tabular}{l}
|\||ifdefined\childdocname\endinput\||fi\newif\ifchilddoc|\\
|\edef\childdocname{\scantokens\expandafter{\jobname\noexpand}}|\\
|\def\childdocmain{|\textit{main}|}\||ifx\childdocmain\childdocname\||else|\\
|\childdoctrue\includeonly{\childdocname}\let\jobname\childdocmain\||fi|\\
\end{tabular}
\end{center}
%
Instead of |\childdocof{|\textit{main}|}| just include the main file
at the top of each child file:
%
\begin{center}
|\input{|\textit{main}|}|
\end{center}
%
A simple redirection |\childdocforward{|\textit{dest}|}| is achieved by:
%
\begin{center}
|\def\jobname{|\textit{dest}|}\input{\jobname}|
\end{center}
%
The redirection with prefix
|\childdocforwardprefix[|\textit{prefix}|]{|\textit{dest}|}|
is accomplished by:
%
\begin{center}
\begin{tabular}{l}
|{\edef\jobname{\scantokens\expandafter{\jobname\noexpand}}|\\
|\def\redirectjob |\textit{prefix}|#1~~~{\gdef\jobname{|\textit{dest}|#1}}|\\
|\expandafter\redirectjob\jobname~~~}\input{\jobname}|
\end{tabular}
\end{center}

In an alternative approach,
child documents can be compiled by a specific command line
without additional code or specific definitions:
%
\begin{center}
|... -jobname "|\textit{target}|" "|[\textit{flags}]%
|\includeonly{|\textit{dest}|}\input{|\textit{main}|}"|
\end{center}
%

%%%%%%%%%%%%%%%%%%%%%%%%%%%%%%%%%%%%%%%%%%%%%%%%%%%%%%%%%%%%%%%%%%%%%%%%%%%%%%%%
%%%%%%%%%%%%%%%%%%%%%%%%%%%%%%%%%%%%%%%%%%%%%%%%%%%%%%%%%%%%%%%%%%%%%%%%%%%%%%%%
\section{Information}

%%%%%%%%%%%%%%%%%%%%%%%%%%%%%%%%%%%%%%%%%%%%%%%%%%%%%%%%%%%%%%%%%%%%%%%%%%%%%%%%
\subsection{Copyright}

Copyright \copyright{} 2017--2018 Niklas Beisert

This work may be distributed and/or modified under the
conditions of the \LaTeX{} Project Public License, either version 1.3
of this license or (at your option) any later version.
The latest version of this license is in
  \url{http://www.latex-project.org/lppl.txt}
and version 1.3 or later is part of all distributions of \LaTeX{}
version 2005/12/01 or later.

This work has the LPPL maintenance status `maintained'.

The Current Maintainer of this work is Niklas Beisert.

This work consists of the files |README.txt|, |childdoc.ins| and |childdoc.dtx|
as well as the derived files |childdoc.def|, |cdocsamp.tex|
with |cdocsch1.tex|, |cdocsch2.tex|, |cdocspt3.tex|, |cdocspt4.tex|,
|cdocsdrf.tex|, |cdocsfn1.tex|, |cdocsfn2.tex|
as well as |childdoc.pdf|.

%%%%%%%%%%%%%%%%%%%%%%%%%%%%%%%%%%%%%%%%%%%%%%%%%%%%%%%%%%%%%%%%%%%%%%%%%%%%%%%%
\subsection{Files and Installation}

The package consists of the files:
%
\begin{center}
\begin{tabular}{ll}
    |README.txt|   & readme file \\
    |childdoc.ins| & installation file \\
    |childdoc.dtx| & source file \\
    |childdoc.def| & definition file \\
    |cdocsamp.tex| & sample main file \\
    |cdocsch1.tex| & sample include file \\
    |cdocsch2.tex| & sample include file \\
    |cdocspt3.tex| & sample part file \\
    |cdocspt4.tex| & sample part file \\
    |cdocsdrf.tex| & sample redirection file \\
    |cdocsfn1.tex| & sample redirection file \\
    |cdocsfn2.tex| & sample redirection file \\
    |childdoc.pdf| & manual
\end{tabular}
\end{center}
%
The distribution consists of the files
|README.txt|, |childdoc.ins| and |childdoc.dtx|.
%
\begin{itemize}
\item
Run (pdf)\LaTeX{} on |childdoc.dtx|
to compile the manual |childdoc.pdf| (this file).
\item
Run \LaTeX{} on |childdoc.ins| to create the definitions file |childdoc.def|
and the sample |cdocsamp.tex| with include files
|cdocsch1.tex|, |cdocsch2.tex|, |cdocspt3.tex|, |cdocspt4.tex|,
|cdocsdrf.tex|, |cdocsfn1.tex|, |cdocsfn2.tex|.
Then copy the file |childdoc.def| to an appropriate directory of your \LaTeX{}
distribution, e.g.\ \textit{texmf-root}|/tex/latex/childdoc|.
\end{itemize}

%%%%%%%%%%%%%%%%%%%%%%%%%%%%%%%%%%%%%%%%%%%%%%%%%%%%%%%%%%%%%%%%%%%%%%%%%%%%%%%%
\subsection{Related CTAN Packages}

There are several other packages which offer a similar functionality:
%
\begin{itemize}
\item
The packages
\href{http://ctan.org/pkg/docmute}{\textsf{docmute}},
\href{http://ctan.org/pkg/includex}{\textsf{includex}} and
\href{http://ctan.org/pkg/standalone}{\textsf{standalone}}
provide commands to include only the document body of
a child file thus allowing both files to be compiled individually.
\item
The packages \href{http://ctan.org/pkg/subdocs}{\textsf{subdocs}}
and \href{http://ctan.org/pkg/subfiles}{\textsf{subfiles}}
provide structures in which the main and child documents can be
encapsulated and allowing them to be compiled individually.
The inclusion mechanism is different from the conventional |\include|.
\item
The package \href{http://ctan.org/pkg/combine}{\textsf{combine}}
is an elaborate solution to combine several documents into one.
\end{itemize}
%
See also the CTAN topic \href{http://ctan.org/topic/subdocs}{\textsf{subdocs}}
for further related packages.
The present package differs from the above solutions in that
a document structure constructed with the conventional |\include| mechanism
just needs two extra commands at the top of every file
such that all constituent files can be compiled individually.

%%%%%%%%%%%%%%%%%%%%%%%%%%%%%%%%%%%%%%%%%%%%%%%%%%%%%%%%%%%%%%%%%%%%%%%%%%%%%%%%
%\subsection{Feature Suggestions}
%
%The following is a list of features which may be useful for future
%versions of this package:
%%
%\begin{itemize}
%\item
%\ldots
%\end{itemize}

%%%%%%%%%%%%%%%%%%%%%%%%%%%%%%%%%%%%%%%%%%%%%%%%%%%%%%%%%%%%%%%%%%%%%%%%%%%%%%%%
\subsection{Revision History}

%%%%%%%%%%%%%%%%%%%%%%%%%%%%%%%%%%%%%%%%
\paragraph{v2.0:} 2018/12/30

\begin{itemize}
\item
immediate forward processing
\item
added |\childdocby| mechanism
\item
manual restructured
\end{itemize}

%%%%%%%%%%%%%%%%%%%%%%%%%%%%%%%%%%%%%%%%
\paragraph{v1.6:} 2018/01/17

\begin{itemize}
\item
application for development of include files
\item
corrections to manual
\end{itemize}

%%%%%%%%%%%%%%%%%%%%%%%%%%%%%%%%%%%%%%%%
\paragraph{v1.5:} 2017/05/21

\begin{itemize}
\item
more complete structuring introduced
\item
|\childdocof| introduced
\item
|\childdoc| renamed to |\childdocmain|
\item
|\childredirect| renamed to |\childdocforward| and |\childdocforwardprefix|
and functionality expanded
\end{itemize}

%%%%%%%%%%%%%%%%%%%%%%%%%%%%%%%%%%%%%%%%
\paragraph{v1.0:} 2017/04/27

\begin{itemize}
\item
manual and install package
\item
first version published on CTAN
\end{itemize}

%%%%%%%%%%%%%%%%%%%%%%%%%%%%%%%%%%%%%%%%
\paragraph{v0.6:} 2017/04/26

\begin{itemize}
\item
redirection mechanism added
\end{itemize}

%%%%%%%%%%%%%%%%%%%%%%%%%%%%%%%%%%%%%%%%
\paragraph{v0.5:} 2017/04/26

\begin{itemize}
\item
functionality in definition file
\end{itemize}


%%%%%%%%%%%%%%%%%%%%%%%%%%%%%%%%%%%%%%%%%%%%%%%%%%%%%%%%%%%%%%%%%%%%%%%%%%%%%%%%
%%%%%%%%%%%%%%%%%%%%%%%%%%%%%%%%%%%%%%%%%%%%%%%%%%%%%%%%%%%%%%%%%%%%%%%%%%%%%%%%
%%%%%%%%%%%%%%%%%%%%%%%%%%%%%%%%%%%%%%%%%%%%%%%%%%%%%%%%%%%%%%%%%%%%%%%%%%%%%%%%
\appendix

\settowidth\MacroIndent{\rmfamily\scriptsize 000\ }

 \DocInput{childdoc.dtx}

\end{document}
%</driver>
% \fi
%
% %%%%%%%%%%%%%%%%%%%%%%%%%%%%%%%%%%%%%%%%%%%%%%%%%%%%%%%%%%%%%%%%%%%%%%%%%%%%%%
% %%%%%%%%%%%%%%%%%%%%%%%%%%%%%%%%%%%%%%%%%%%%%%%%%%%%%%%%%%%%%%%%%%%%%%%%%%%%%%
% \section{Sample}
%\iffalse
%<*samplemain>
%\fi
%
% The following presents a sample document
% with two chapters, two parts, a title page,
% a compile flag as well as three forwarding files to set the flag.
% It consists of eight |.tex| files:
% \begin{center}
% \begin{tabular}{ll}
% |cdocsamp.tex|&main file\\
% |cdocsch1.tex|&include file for chapter 1\\
% |cdocsch2.tex|&include file for chapter 2\\
% |cdocspt3.tex|&include file for part 3\\
% |cdocspt4.tex|&include file for part 4\\
% |cdocsdrf.tex|&forwarding file for main file in draft mode\\
% |cdocsfi1.tex|&forwarding file for final version of chapter 1\\
% |cdocsfi2.tex|&forwarding file for final version of chapter 2\\
% \end{tabular}
% \end{center}
% Each of the eight files can be compiled directly by the \LaTeX{} compiler.
%
% %%%%%%%%%%%%%%%%%%%%%%%%%%%%%%%%%%%%%%
% \paragraph{Main File.}
%
% The main file is called |cdocsamp.tex|.
%
% Load the \textsf{childdoc} definitions and
% declare the filename for the main document:
%    \begin{macrocode}
\input{childdoc.def}
\childdocmain{}
%    \end{macrocode}

% Optional override for |\version| flag:
%    \begin{macrocode}
%%\ifchilddoc\else\providecommand{\version}{draft}\fi
%    \end{macrocode}

% Define the default values for the |\version| flag
% (|final| for the main file and |draft| for childs):
%    \begin{macrocode}
\ifchilddoc
\providecommand{\version}{draft}
\else
\providecommand{\version}{final}
\fi
%    \end{macrocode}

% Load the standard document class:
%    \begin{macrocode}
\documentclass[12pt]{article}
%    \end{macrocode}

% Start the document body:
%    \begin{macrocode}
\begin{document}
%    \end{macrocode}

% Declare a title page.
% Print title, part of document being processed and version flag:
%    \begin{macrocode}
\addtocounter{page}{-1}
\begin{center}
{\LARGE\bfseries{}childdoc example\par}
\vspace{1cm}
\ifchilddoc
\ifchilddocmanual part\else chapter\fi:
`\childdocname' of `\childdocjob'\par
\else
main document: `\childdocjob'\par
\fi
version: \version\par
\end{center}
\newpage
%    \end{macrocode}

% Manually include selected file,
% otherwise process as usual:
%    \begin{macrocode}
\ifchilddocmanual
\section*{part `\childdocname'}
\input{\childdocname}
\else
%    \end{macrocode}

% Include the two chapters:
%    \begin{macrocode}
\include{cdocsch1}
\include{cdocsch2}
%    \end{macrocode}

% Include the two parts unless only chapters should be displayed:
%    \begin{macrocode}
\ifchilddoc\else
\section{part three}
\input{cdocspt3}
\section{part four}
\input{cdocspt4}
\fi
%    \end{macrocode}

% Process as usual until here:
%    \begin{macrocode}
\fi
%    \end{macrocode}

% End of document body:
%    \begin{macrocode}
\end{document}
%    \end{macrocode}
%\iffalse
%</samplemain>
%\fi
%
% %%%%%%%%%%%%%%%%%%%%%%%%%%%%%%%%%%%%%%
% \paragraph{Chapter Include Files.}
%
% The include files are called |cdocsch1.tex| and |cdocsch2.tex|.
%
%\iffalse
%<*samplechap1|samplechap2>
%\fi

% Optional override for |\version| flag:
%    \begin{macrocode}
%%\providecommand{\version}{final}
%    \end{macrocode}

% Include the main document:
%    \begin{macrocode}
\input{childdoc.def}
\childdocof{cdocsamp}
%    \end{macrocode}

%\iffalse
%</samplechap1|samplechap2>
%\fi
%
%\iffalse
%<*samplechap1>
%\fi
% Some text for chapter 1:
%    \begin{macrocode}
\section{one}
some text in chapter one
%    \end{macrocode}

%\iffalse
%</samplechap1>
%\fi
% Some text for chapter 2:
%\iffalse
%<*samplechap2>
%\fi
%    \begin{macrocode}
\section{two}
more text in chapter two
%    \end{macrocode}

%\iffalse
%</samplechap2>
%\fi
%
% %%%%%%%%%%%%%%%%%%%%%%%%%%%%%%%%%%%%%%
% \paragraph{Part Include Files.}
%
% The include files are called |cdocspt3.tex| and |cdocspt4.tex|.
%
%\iffalse
%<*samplepart3|samplepart4>
%\fi

% Optional override for |\version| flag:
%    \begin{macrocode}
%%\providecommand{\version}{final}
%    \end{macrocode}

% Include the main document:
%    \begin{macrocode}
\input{childdoc.def}
\childdocby{cdocsamp}
%    \end{macrocode}

%\iffalse
%</samplepart3|samplepart4>
%\fi
%
%\iffalse
%<*samplepart3>
%\fi
% Some text for part 3:
%    \begin{macrocode}
some text in part three
%    \end{macrocode}

%\iffalse
%</samplepart3>
%\fi
% Some text for part 4:
%\iffalse
%<*samplepart4>
%\fi
%    \begin{macrocode}
more text in part four
%    \end{macrocode}

%\iffalse
%</samplepart4>
%\fi
%
% %%%%%%%%%%%%%%%%%%%%%%%%%%%%%%%%%%%%%%
% \paragraph{Forwarding for a Complete Draft.}
%
% The following forwarding file |cdocsdrf.tex|
% compiles the main document in draft mode:
%\iffalse
%<*sampledraft>
%\fi
%    \begin{macrocode}
\def\version{draft}
\input{childdoc.def}
\childdocforward{cdocsamp}
%    \end{macrocode}

%\iffalse
%</sampledraft>
%\fi
%
% %%%%%%%%%%%%%%%%%%%%%%%%%%%%%%%%%%%%%%
% \paragraph{Forwarding for Final Version of the Chapters.}
%
% The following forwarding files |cdocsfn1.tex| and |cdocsfn2.tex|
% (with identical content)
% compile the final versions of the child documents
% |cdocsch1.tex| and |cdocsch2.tex|, respectively:
%\iffalse
%<*samplefinal>
%\fi
%    \begin{macrocode}
\def\version{final}
\input{childdoc.def}
\childdocforwardprefix[cdocsamp]{cdocsfn}{cdocsch}
%    \end{macrocode}

%\iffalse
%</samplefinal>
%\fi
%
% %%%%%%%%%%%%%%%%%%%%%%%%%%%%%%%%%%%%%%
% \paragraph{Command Line Processing.}
%
% The following three command lines generate the output files
% |cdocscld|, |cdocscl1| and |cdocscl2|
% which should be identical to
% |cdocsdrf|, |cdocsch1| and |cdocsfn2|, respectively:
% \begin{center}
% \begin{tabular}{l}
% |latex -jobname cdocscld \|\\
% |  "\def\version{draft}\input{childdoc.def}\childdocforward{cdocsamp}"|\\
% |latex -jobname cdocscl1 \|\\
% |  "\input{childdoc.def}\childdocforward[cdocsamp]{cdocsch1}"|\\
% |latex -jobname cdocscl2 \|\\
% |  "\def\version{final}\input{childdoc.def}\childdocforward{cdocsch2}"|
% \end{tabular}
% \end{center}
% Note that the trailing backslash on each first line
% merely continues the input to the second line
% (for convenient cut ant paste).
% Furthermore, the command |latex| can be replaced by any
% of its alternative versions such as |pdflatex|.
%
% %%%%%%%%%%%%%%%%%%%%%%%%%%%%%%%%%%%%%%%%%%%%%%%%%%%%%%%%%%%%%%%%%%%%%%%%%%%%%%
% %%%%%%%%%%%%%%%%%%%%%%%%%%%%%%%%%%%%%%%%%%%%%%%%%%%%%%%%%%%%%%%%%%%%%%%%%%%%%%
% \section{Implementation}
%\iffalse
%<*package>
%\fi
%
% This section describes the definitions file |childdoc.def|.

% The definitions cannot be loaded using |\usepackage| or |\RequirePackage|
% which has a mechanism to prevent loading a style file more than once.
% When loading the definitions by means of |\input|
% multiple instances have to be prevented manually:
%\iffalse
%This code needs to be before the `\ProvidesFile' directive
%which is defined at the beginning of this file.
%Therefore it is also placed there and commented out here.
%</package>
%<*discard>
%\fi
%    \begin{macrocode}
\ifdefined\childdocmain\endinput\fi
%    \end{macrocode}
%\iffalse
%</discard>
%<*package>
%\fi
%
% \macro{\ifchilddoc}
% \macro{\ifchilddocmanual}
% The conditional |\ifchilddoc| tells whether a
% child (true) or main (false) document is being compiled.
% The conditional |\ifchilddocmanual| tells whether
% the |\includeonly| mechanism is used (false) or
% the selection of child files must be performed manually (true).
% The definitions initialise to false:
%    \begin{macrocode}
\newif\ifchilddoc
\newif\ifchilddocmanual
%    \end{macrocode}

% \macro{\childdocname}
% \macro{\childdocjob}
% The macro |\childdocname| stores the name of the main document
% to be compiled. The macro |\childdocjob| stores the name of
% the document on which the \LaTeX{} compiler was originally invoked.
% The content of |\jobname| cannot be compared
% to filenames specified in the source due to different catcodes.
% The following code rescans |\jobname|, stores the result
% in |\childdocname| and saves a copy in |\childdocjob|:
%    \begin{macrocode}
\edef\childdocname{\scantokens\expandafter{\jobname\noexpand}}
\let\childdocjob\childdocname
%    \end{macrocode}

% \macro{\childdocdisable}
% The macro |\childdocdisable| prevents the main file
% from being processed more than once.
% At this stage, the main document command |\childdocmain|
% is assumed to be called once again where it should do nothing.
% Any subsequent call to it should prevent
% a secondary processing of the main document
% It overwrites the forwarding commands
% |\childdocof| and |\childdocforward|
% with empty macros to prevent further inclusions of the main document:
%    \begin{macrocode}
\newcommand{\childdocdisable}
{
  \renewcommand{\childdocmain}[1]{\renewcommand{\childdocmain}[1]{\endinput}}
  \renewcommand{\childdocof}[1]{}
  \renewcommand{\childdocby}[2][]{}
  \renewcommand{\childdocforward}[2][]{}
  \renewcommand{\childdocdisable}{}
}
%    \end{macrocode}

% \macro{\childdocmain}
% The macro |\childdocmain| is to be called at the top of the main file
% with nothing or the main filename (without extension) as argument.
% First, it breaks loops.
% If the argument is not empty and does not match |\childdocname|
% (which is set by the first inclusion of |childdoc.def|),
% |\ifchilddoc| is set to true, |\includeonly| is applied to the child file
% and |\jobname| is set to the main file
% (for proper handling of |.aux| files):
%    \begin{macrocode}
\newcommand{\childdocmain}[1]
{
  \childdocdisable\childdocmain{}
  \if?#1?\else
    \begingroup
      \def\childdoctmp{#1}
      \ifx\childdoctmp\childdocname
        \def\childdoctmp{}
      \else
        \def\childdoctmp
        {
          \childdoctrue
          \includeonly{\childdocname}
          \def\childdocjob{#1}
          \def\jobname{#1}
        }
      \fi
      \expandafter
    \endgroup
    \childdoctmp
  \fi
}
%    \end{macrocode}

% \macro{\childdocof}
% The command |\childdocof| redirects
% compilation to the main file |#1|.
%    \begin{macrocode}
\newcommand{\childdocof}[1]
{
  \childdocdisable
  \childdoctrue
  \includeonly{\childdocname}
  \def\jobname{#1}
  \def\childdocjob{#1}
  \input{#1}
}
%    \end{macrocode}

% \macro{\childdocby}
% The command |\childdocby| ....
%    \begin{macrocode}
\newcommand{\childdocby}[2][]
{
  \childdocdisable
  \childdoctrue
  \childdocmanualtrue
  \if?#1?\else
    \def\jobname{#2}
  \fi
  \def\childdocjob{#2}
  \input{#2}
  \endinput
}
%    \end{macrocode}

% \macro{\childdocforward}
% The command |\childdocforward| redirects
% compilation to the main file or
% (if the optional argument is given) a child file.
% Parameters are set as if the main file
% or a child file starting with |\childdocof| was compiled.
% Then compilation is handed over to the main file:
%    \begin{macrocode}
\newcommand{\childdocforward}[2][]
{
  \begingroup
    \if?#1?
      \def\childdoctmp
      {
        \def\childdocname{#2}
        \def\childdocjob{#2}
        \def\jobname{#2}
        \input{#2}
        \endinput
      }
    \else
      \def\childdoctmp
      {
        \childdocdisable
        \def\childdocname{#2}
        \childdoctrue
        \includeonly{#2}
        \def\childdocjob{#1}
        \def\jobname{#1}
        \input{#1}
        \endinput
      }
    \fi
    \expandafter
  \endgroup
  \childdoctmp
}
%    \end{macrocode}

% \macro{\childdocforwardprefix}
% The command |\childdocforwardprefix| redirects
% compilation to the main or a child file by means of a pattern.
% The prefix |#1| in the current filename is replaced by |#2|
% and the suffix of the current filename is kept
% (it is assumed that the filename does not contain the substring `|~~~|'
% which is used as a delimiter).
% Compilation is handed over to the new file by |\childdocforward|:
%    \begin{macrocode}
\newcommand{\childdocforwardprefix}[3][]
{
  \begingroup
    \def\childdocextract #2##1~~~{\def\childdoctmp{\childdocforward[#1]{#3##1}}}
    \expandafter\childdocextract\childdocname~~~
    \expandafter
  \endgroup
  \childdoctmp
}
%    \end{macrocode}

% \macro{\childdoc}
% The deprecated macro |\childdoc| is a legacy version of |\childdocmain|:
%    \begin{macrocode}
\newcommand{\childdoc}{\childdocmain}
%    \end{macrocode}

% \macro{\childdocredirect}
% The deprecated macro |\childdocredirect| is a legacy version
% of |\childdocforward| and |\childdocforwardprefix|:
%    \begin{macrocode}
\newcommand{\childdocredirect}[2][]
{
  \begingroup
    \if?#1?
      \def\childdoctmp{\childdocforward{#2}}
    \else
      \def\childdoctmp{\childdocforwardprefix{#1}{#2}}
    \fi
    \expandafter
  \endgroup
  \childdoctmp
}
%    \end{macrocode}

%\iffalse
%</package>
%\fi
%
\endinput
\childdocforward{cdocsamp}"|\\
% |latex -jobname cdocscl1 \|\\
% |  "% \iffalse
%
% childdoc.dtx Copyright (C) 2017-2018 Niklas Beisert
%
% This work may be distributed and/or modified under the
% conditions of the LaTeX Project Public License, either version 1.3
% of this license or (at your option) any later version.
% The latest version of this license is in
%   http://www.latex-project.org/lppl.txt
% and version 1.3 or later is part of all distributions of LaTeX
% version 2005/12/01 or later.
%
% This work has the LPPL maintenance status `maintained'.
%
% The Current Maintainer of this work is Niklas Beisert.
%
% This work consists of the files childdoc.dtx and childdoc.ins
% and the derived files childdoc.def and cdocsamp.tex with
% cdocsch1.tex, cdocsch2.tex, cdocsdrf.tex, cdocsfn1.tex, cdocsfn2.tex.
%
%<package>\ifdefined\childdocmain\endinput\fi
%<package>\ProvidesFile{childdoc.def}[2018/12/30 v2.0 child document driver]
%<samplemain>\ProvidesFile{cdocsamp.tex}[2018/12/30 v2.0 sample for childdoc]
%<*driver>
%\ProvidesFile{childdoc.drv}[2018/12/30 v2.0 childdoc reference manual file]
\PassOptionsToClass{10pt,a4paper}{article}
\documentclass{ltxdoc}

\usepackage[margin=35mm]{geometry}
\usepackage{hyperref}
\usepackage{hyperxmp}
\usepackage[usenames]{color}

\hypersetup{colorlinks=true}
\hypersetup{pdfstartview=FitH}
\hypersetup{pdfpagemode=UseNone}
\hypersetup{pdfsource={}}
\hypersetup{pdflang={en-UK}}
\hypersetup{pdfcopyright={Copyright 2017-2018 Niklas Beisert.
  This work may be distributed and/or modified under the
  conditions of the LaTeX Project Public License, either version 1.3
  of this license or (at your option) any later version.}}
\hypersetup{pdflicenseurl={http://www.latex-project.org/lppl.txt}}
\hypersetup{pdfcontactaddress={ETH Zurich, ITP, HIT K,
  Wolfgang-Pauli-Strasse 27}}
\hypersetup{pdfcontactpostcode={8093}}
\hypersetup{pdfcontactcity={Zurich}}
\hypersetup{pdfcontactcountry={Switzerland}}
\hypersetup{pdfcontactemail={nbeisert@itp.phys.ethz.ch}}
\hypersetup{pdfcontacturl={http://people.phys.ethz.ch/\xmptilde nbeisert/}}

\newcommand{\secref}[1]{\hyperref[#1]{section \ref*{#1}}}

\parskip1ex
\parindent0pt
\let\olditemize\itemize
\def\itemize{\olditemize\parskip0pt}

\begin{document}

\title{The \textsf{childdoc} Package}
\hypersetup{pdftitle={The childdoc Package}}
\author{Niklas Beisert\\[2ex]
  Institut f\"ur Theoretische Physik\\
  Eidgen\"ossische Technische Hochschule Z\"urich\\
  Wolfgang-Pauli-Strasse 27, 8093 Z\"urich, Switzerland\\[1ex]
  \href{mailto:nbeisert@itp.phys.ethz.ch}
  {\texttt{nbeisert@itp.phys.ethz.ch}}}
\hypersetup{pdfauthor={Niklas Beisert}}
\hypersetup{pdfsubject={Manual for the LaTeX2e Package childdoc}}
\date{30 December 2018, \textsf{v2.0}}
\maketitle

\begin{abstract}\noindent
\textsf{childdoc} is a \LaTeXe{} package
that enables the direct compilation
of document sections included by |\include|
to individual files.
\end{abstract}

\begingroup
\parskip0ex
\tableofcontents
\endgroup

%%%%%%%%%%%%%%%%%%%%%%%%%%%%%%%%%%%%%%%%%%%%%%%%%%%%%%%%%%%%%%%%%%%%%%%%%%%%%%%%
%%%%%%%%%%%%%%%%%%%%%%%%%%%%%%%%%%%%%%%%%%%%%%%%%%%%%%%%%%%%%%%%%%%%%%%%%%%%%%%%
\section{Introduction}

\LaTeX{} provides a mechanism to structure a large document (such as a book)
into a main file and several child files (containing the chapters)
using the |\include| command.
This mechanism is beneficial for documents
which span hundreds of pages in order to
make the source file(s) more manageable.
Moreover, compilation can be restricted to
selected child files by means of the |\includeonly| command.
The latter feature can be used to reduce the compilation time while editing
(this was significantly more useful in the earlier days of \LaTeX{})
or to generate a smaller document which is easier to navigate.
Another application of |\includeonly| is to generate
documents consisting of selected parts of the complete document.

However, there are a few drawbacks of the plain |\include| mechanism:
\begin{itemize}
\item
The child files cannot be compiled on their own,
they can only be compiled via the main file.
A naive editing environment
(such as a text editor with an option
to have the current file processed by \LaTeX)
may require one to switch to the main file before compiling;
attempting to compile the child file produces errors.
\item
The main file must be modified (each time)
to adjust the |\includeonly| command
to the present needs. This easily leaves the main file in a messy state.
\item
The generated document will always carry the filename
of the main document. This is inconvenient if
several child files are to be compiled and
to be kept for distribution.
\end{itemize}

The present package provides a simple interface
to make child files individually compilable by \LaTeX{}.
Compiling a child file then has the same effect as compiling
the main file with an |\includeonly| command
to select the appropriate child.
Moreover the generated document will carry the name of the child
rather than the main file.
This resolves all three above issues.

This feature is meant to make the editing of books,
thesis documents and lecture notes somewhat more convenient.
However, the package can also be used efficiently for
composing a series of documents (such as exercise sheets)
which are typically distributed individually.
It then assists the author in generating the individual documents
(potentially in different versions)
as well as a document containing the collected series.
Another application is in developing style files
or other kinds of included material
where compilation of the style file could redirect
to a sample or test file.

%%%%%%%%%%%%%%%%%%%%%%%%%%%%%%%%%%%%%%%%%%%%%%%%%%%%%%%%%%%%%%%%%%%%%%%%%%%%%%%%
%%%%%%%%%%%%%%%%%%%%%%%%%%%%%%%%%%%%%%%%%%%%%%%%%%%%%%%%%%%%%%%%%%%%%%%%%%%%%%%%
\section{Usage}

First of all, the package \textsf{childdoc} is \emph{not} a standard
\LaTeXe{} |.sty| style file! Therefore it needs to be invoked in
a non-standard way.

%%%%%%%%%%%%%%%%%%%%%%%%%%%%%%%%%%%%%%%%%%%%%%%%%%%%%%%%%%%%%%%%%%%%%%%%%%%%%%%%
\subsection{Included Files}
\label{sec:include}

%%%%%%%%%%%%%%%%%%%%%%%%%%%%%%%%%%%%%%%%
\DescribeMacro{\childdocmain}
To use the package, add the commands
\begin{center}
\begin{tabular}{l}
|\input{childdoc.def}|\\
|\childdocmain{}|\\
\end{tabular}
\end{center}
at the very top of the main \LaTeX{} file,
in particular \emph{before} the |\documentclass| statement!
The argument of |\childdocmain| should be left empty
(but it must be present).

%%%%%%%%%%%%%%%%%%%%%%%%%%%%%%%%%%%%%%%%
\DescribeMacro{\childdocof}
Furthermore, add the commands
\begin{center}
\begin{tabular}{l}
|\input{childdoc.def}|\\
|\childdocof{|\textit{main}|}|\\
\end{tabular}
\end{center}
at the top of every child file \textit{child}
which is included by |\include{|\textit{child}|}|
from within the main file
(or at least for those files to be compiled individually).
The argument \textit{main} must be the filename of the main file.

There are a couple of
considerations in setting up the main and child documents:

%%%%%%%%%%%%%%%%%%%%%%%%%%%%%%%%%%%%%%%%
\paragraph{Restrictions.}

Please note the following restrictions:
\begin{itemize}
\item
|\childdocmain| must be called with one argument \textit{main}
to ensure compatibility with earlier version of the package.
It must either be empty (|\childdocmain{}|)
or precisely match the filename of the main file in which it is specified.
See \secref{sec:detection} for further information.
\item
The filename \textit{main} must be specified without the |.tex| extension.
\item
The filename \textit{main} is case sensitive
(even in case-insensitive file systems)
due to internal string comparison.
\item
The argument \textit{main} should be fully expanded, it cannot be a macro.
\item
Subdirectories and special characters should be avoided in filenames.
\item
The command |\childdocmain{|\textit{main}|}| must be followed by a whitespace.
It should not be followed immediately by another command
or by a comment mark `|%|'.
This is because the \TeX{} parser reads the token immediately following
the argument of |\childdocmain| and puts it
at the beginning of every child section;
however, a white\-space is ignored.
\end{itemize}

%%%%%%%%%%%%%%%%%%%%%%%%%%%%%%%%%%%%%%%%
\paragraph{Content of Main File.}

It is advisable to place all content in the child files included by |\include|.
Any output contained in the main file will appear in all child documents
unless suppressed manually;
it cannot be suppressed automatically by the |\includeonly| directive
and thus should normally be avoided.
A method to include some content in the main file
by means of conditional processing is described in \secref{sec:conditional}.

%%%%%%%%%%%%%%%%%%%%%%%%%%%%%%%%%%%%%%%%
\paragraph{Page Numbering.}

When only a part of the document is compiled,
the appropriate numbering of pages
(as well as other status parameters)
is determined from the |.aux| files.
The latter contain information from previous passes.
However this information needs to propagate through
all intermediate child documents.
Therefore the page numbering in child documents may well
be inconsistent until the complete document is compiled at least once.

A useful (if unconventional) way to always ensure a consistent
page numbering is to restart the numbering in each child document
and denote the pages by `\textit{child}|.|\textit{page}'
where \textit{child} represents the chapter/section number of the child file.
This can be achieved by the command
|\numberwithin{page}{|\textit{child}|}|
of the \textsf{amsmath} package
where \textit{child} can be |chapter| or |section|
depending on the chosen structuring.
Alternatively, one can modify the macro |\thepage| appropriately
and reset the counter |page| at the start of each child file.

%%%%%%%%%%%%%%%%%%%%%%%%%%%%%%%%%%%%%%%%%%%%%%%%%%%%%%%%%%%%%%%%%%%%%%%%%%%%%%%%
\subsection{Conditional Processing}
\label{sec:conditional}

The package provides a mechanism to compile different versions
of a document. To customise the versions further some conditional processing
can come in handy to distinguish which version is being compiled.
The package provides two macros to describe the compilation context:

%%%%%%%%%%%%%%%%%%%%%%%%%%%%%%%%%%%%%%%%
\DescribeMacro{\ifchilddoc}
The conditional |\ifchilddoc| distinguishes between the compilation of
child documents and the main document:
%
\begin{center}
|\ifchilddoc |\textit{child-code}| |[|\||else |\textit{main-code}]| \||fi|
\end{center}

%%%%%%%%%%%%%%%%%%%%%%%%%%%%%%%%%%%%%%%%
\DescribeMacro{\childdocname}
\DescribeMacro{\childdocjob}
The macro |\childdocname| contains the filename (without extension)
of the main or child file being processed.
Note that |\childdocjob| will always contain the name of the main file.

%%%%%%%%%%%%%%%%%%%%%%%%%%%%%%%%%%%%%%%%
\paragraph{Title Page.}

Conditional processing can be used to include a title or banner page
in the main document when proper precautions are taken.
Importantly, the code in the main file should ensure that the page counter
(as well as other status parameters which are stored in the |.aux| files)
takes the same value after the conditional processing.
Otherwise the page numbers may take divergent values
depending on which part is compiled.

For example, a title page could be declared by:
%
\begin{center}
\begin{tabular}{l}
|\ifchilddoc\||else|\\
|\addtocounter{page}{-1}|\\
\textit{code for title page}\\
|\newpage|\\
|\||fi|
\end{tabular}
\end{center}
%
A banner page for the child documents can be generated by:
%
\begin{center}
\begin{tabular}{l}
|\ifchilddoc|\\
|\addtocounter{page}{-1}|\\
\textit{code for banner page}\\
|\newpage|\\
|\||fi|
\end{tabular}
\end{center}
%
Here one could write a message such as:
\begin{center}
|This is the part \childdocname{} of \childdocjob{}.|
\end{center}

%%%%%%%%%%%%%%%%%%%%%%%%%%%%%%%%%%%%%%%%%%%%%%%%%%%%%%%%%%%%%%%%%%%%%%%%%%%%%%%%
\subsection{Flags}
\label{sec:flags}

The package makes it easy to generate different versions
of the main or child documents.
To this end compilation flags can be defined
and assigned different default values.
They will be particularly useful in conjunction
with the forwarding mechanism described in \secref{sec:forward}.

For example, it may be useful to have a flag |\version|
which can be set to |draft| or |final|.
The document source will contain some conditional code
depending on the value of |\version|.
Suppose further, the flag should default to |final| for the main file
and to |draft| for child files
which is a natural assignment for editing the document.
This is achieved by placing the following code
in the preamble of the main document
(below the |\childdocmain| directive):
%
\begin{center}
\begin{tabular}{l}
|\ifchilddoc|\\
|\providecommand{\version}{draft}|\\
|\||else|\\
|\providecommand{\version}{final}|\\
|\||fi|
\end{tabular}
\end{center}
%
The definition by |\providecommand| makes sure
that previous definitions are not overwritten.
Further statements |\providecommand{\version}{...}|
can thus be added before the above code to override it.

For the main file, one might add a line
(between |\childdocmain| and the above block)
%
\begin{center}
|%\ifchilddoc\||else\providecommand{\version}{draft}\||fi|
\end{center}
%
which can be uncommented to produce a draft version.
Likewise one can add a line to the very top of a child file
(above the |\childdocof{|\textit{main}|}| directive)
%
\begin{center}
|%\providecommand{\version}{final}|
\end{center}
%
which can be uncommented to produce the final version of this child document.

%%%%%%%%%%%%%%%%%%%%%%%%%%%%%%%%%%%%%%%%%%%%%%%%%%%%%%%%%%%%%%%%%%%%%%%%%%%%%%%%
\subsection{Forwarding}
\label{sec:forward}

Different versions of the main or child documents
using compilation flags as described in \secref{sec:flags}
can be (permanently) stored in different files
for convenient compilation, viewing and distribution.
To this end, the package defines a command
to pass on compilation to a different file:

%%%%%%%%%%%%%%%%%%%%%%%%%%%%%%%%%%%%%%%%
\DescribeMacro{\childdocforward}
The command |\childdocforward| redirects processing to
another source file:
%
\begin{center}
\begin{tabular}{l}
|\input{childdoc.def}|\\
|\childdocforward[|\textit{main}|]{|\textit{dest}|}|\\
\end{tabular}
\end{center}
%
The argument \textit{dest} is the destination file
(without extension).
It should be the main file or one of the child files.
Note that further \textsf{childdoc} directives
such as |\childdocof| and |\childdocforward|
in the indicated file will be processed in this form.
The optional argument \textit{main}
passes on directly to the main file \textit{main}
while pretending to compile the child \textit{dest}.
This form behaves as if \textit{dest}
issues |\childdocof{|\textit{main}|}| right away,
and no further \textsf{childdoc} directives will be processed.

%%%%%%%%%%%%%%%%%%%%%%%%%%%%%%%%%%%%%%%%
\DescribeMacro{\...prefix}
In the alternative form |\childdocforwardprefix|,
%
\begin{center}
\begin{tabular}{l}
|\input{childdoc.def}|\\
|\childdocforwardprefix[|\textit{main}|]{|\textit{prefix}|}{|\textit{dest}|}|
\end{tabular}
\end{center}
%
the destination file is determined by a pattern
depending on the current file:
To make this work, the current file must be called
`{\textit{prefix}\hspace{0.2em}\textit{suffix}}'
with \textit{prefix} matching precisely the argument.
Processing is then passed on to the file
`{\textit{dest}\hspace{0.2em}\textit{suffix}}'.
Surely, the same effect is achieved by
directly specifying the
argument `{\textit{dest}\hspace{0.2em}\textit{suffix}}'
in the first form.
However, that requires to set up a different file
for each child. With the alternative form of the command
all these files can have exactly the same content
which simplifies setting them up and maintaining them.

For example, the following file |draft.tex|
with a compilation flag |\version| as described in \secref{sec:flags}
compiles the main document as a draft:
%
\begin{center}
\begin{tabular}{l}
|\def\version{draft}|\\
|\input{childdoc.def}|\\
|\childdocforward{|\textit{main}|}|
\end{tabular}
\end{center}
%
Likewise, the following files |final|\textit{nn}|.tex|
compile the final version of the child document
|child|\textit{nn}|.tex|:
%
\begin{center}
\begin{tabular}{l}
|\def\version{final}|\\
|\input{childdoc.def}|\\
|\childdocforwardprefix{final}{child}|
\end{tabular}
\end{center}
%

Note that when several versions of a main file and/or of each child file
are to be generated, it may be convenient to set up a |Makefile| or
shell script to automatise the process.

%%%%%%%%%%%%%%%%%%%%%%%%%%%%%%%%%%%%%%%%%%%%%%%%%%%%%%%%%%%%%%%%%%%%%%%%%%%%%%%%
\subsection{Command Line Processing}
\label{sec:commandline}

The effect of redirection files can also be achieved by invoking
the \LaTeX{} compiler with a more elaborate command line.
Most conveniently this should be done as part
of a shell script or a |Makefile|.

When using \textsf{childdoc} in the main file, the following
command lines effectively perform a redirection
(note that depending on the shell being used,
backslashes may have to be doubled: `|\|' $\to$ `|\\|'):
%
\begin{center}
|... -jobname "|\textit{target}|" |\\|"|[\textit{flags}]%
|\input{childdoc.def}\childdocforward[|\textit{main}|]{|\textit{dest}|}"|
\end{center}
%
Here \textit{target} is the name of the output file,
\textit{main} is the name of the main file
and \textit{dest} is the name of the main or child file to be processed
(all filenames without extensions).
The optional argument \textit{main} can be omitted
if \textit{main} matches \textit{dest}.
Optionally, compilation \textit{flags} can be defined via |\def| commands.
This command line makes the \TeX{} engine believe
it is compiling the file \textit{target}
whose content is specified as the latter parameter.
The provided code then forwards the processing to
\textit{main} or \textit{dest} as described in \secref{sec:forward}.

%%%%%%%%%%%%%%%%%%%%%%%%%%%%%%%%%%%%%%%%%%%%%%%%%%%%%%%%%%%%%%%%%%%%%%%%%%%%%%%%
\subsection{Include by Input}
\label{sec:input}

Including child documents by |\include| has some restrictions by design.
Most notably, the content of a child document always occupies
its own set of pages; pages cannot be shared between child documents.
Usually, this behaviour makes perfect sense
because each child document contain an essential part of the document.
However, in some situations it may be desirable to compose
a document from a collection of parts
without having mandatory page breaks between then.
For this case, the package
provides a mechanism to include parts
by |\input| which can also be processed individually.
However, by construction this mechanism
requires manual handling of the content to be output.

%%%%%%%%%%%%%%%%%%%%%%%%%%%%%%%%%%%%%%%%
\DescribeMacro{\ifchilddocmanual}
The main file should be prepared as usual, see \secref{sec:include}.
However, the document body must make a distinction
between processing of an individual part and of the main document, e.g.:
%
\begin{center}
\begin{tabular}{l}
|\ifchilddocmanual|\\
|\input{\childdocname}|\\
|\||else|\\
\textit{document body with }|\input{|\textit{part}|}|\\
|\||fi|
\end{tabular}
\end{center}
%
The conditional |\ifchilddocmanual| is true whenever
a part to be included by |\input| is being compiled,
and the name of the part is stored in |\childdocname|.

%%%%%%%%%%%%%%%%%%%%%%%%%%%%%%%%%%%%%%%%
\DescribeMacro{\childdocby}
Each part to be included by |\input| should start with:
%
\begin{center}
\begin{tabular}{l}
|\input{childdoc.def}|\\
|\childdocby{|\textit{main}|}|\\
\end{tabular}
\end{center}
%
The directive |\childdocby| is similar to |\childdocof|
described in \secref{sec:include},
but the subsequent selection of content must be done manually.
To that end, both |\ifchilddoc| and |\ifchilddocmanual|
will be true upon processing of a part,
and the name of the part is stored in |\childdocname|.
Note that |\jobname| will be set to the filename of the current part
so that each part receives an individual |.aux| file
that does not interfere with the |.aux| file(s) of the main document.
This behaviour can be altered by the alternative form
|\childdocby[*]{|\textit{main}|}| (with a non-empty optional argument)
which uses the |.aux| file of the main document
by setting |\jobname| to \textit{main}.

%%%%%%%%%%%%%%%%%%%%%%%%%%%%%%%%%%%%%%%%%%%%%%%%%%%%%%%%%%%%%%%%%%%%%%%%%%%%%%%%
\subsection{Driver Development}
\label{sec:driver}

The \textsf{childdoc} mechanism can also be use for the development
of definition files such as \LaTeX{} styles or classes.
This case differs from the above setup with multiple parts
included by |\include| in that no |\includeonly| should be invoked.
This can be achieved by starting the include file
(before |\ProvidesPackage|) with:
%
\begin{center}
\begin{tabular}{l}
|\input{childdoc.def}|\\
|\childdocforward{|\textit{main}|}|\\
\end{tabular}
\end{center}
%
or alternatively with:
%
\begin{center}
\begin{tabular}{l}
|\input{childdoc.def}|\\
|\childdocby{|\textit{main}|}|\\
\end{tabular}
\end{center}
%
Both forms have slightly different effects as described above.
The main file is prepared as usual, see \secref{sec:include}.

%%%%%%%%%%%%%%%%%%%%%%%%%%%%%%%%%%%%%%%%%%%%%%%%%%%%%%%%%%%%%%%%%%%%%%%%%%%%%%%%
\subsection{Legacy Detection}
\label{sec:detection}

The directive |\childdocmain| in the main file can detect
whether the complete document or merely a child is to be compiled
even without using the directive |\childdocof|.
This method is deprecated because it is less robust
and there is no compelling reason to use it;
it is merely provided for backward compatibility
and it may be removed in future versions.

If the detection mechanism is to be used,
it is mandatory to correctly specify
the filename of the main file as the argument of |\childdocmain|:
%
\begin{center}
\begin{tabular}{l}
|\input{childdoc.def}|\\
|\childdocmain{|\textit{main}|}|\\
\end{tabular}
\end{center}
%
If |\jobname| does not match the argument \textit{main} of |\childdocmain|,
it is assumed that |\jobname| points to the child file to be compiled.
When using |\childdocmain| with the main file specified as argument,
it suffices to start a child file
with just |\input{|\textit{main}|}|
without loading of the package and using |\childdocof|.
If instead all processing is done
with the appropriate \textsf{childdoc} directives,
the argument of \textit{main} of |\childdocmain| can be empty.

An alternative version of the command line processing described
in \secref{sec:commandline} using the detection mechanism reads:
%
\begin{center}
|... -jobname "|\textit{target}|" "|[\textit{flags}]%
[|\def\jobname{|\textit{dest}|}|]|\input{|\textit{main}|}"|
\end{center}

%%%%%%%%%%%%%%%%%%%%%%%%%%%%%%%%%%%%%%%%%%%%%%%%%%%%%%%%%%%%%%%%%%%%%%%%%%%%%%%%
\subsection{Manual Code}
\label{sec:manual}

In case one cannot be certain whether the definitions file |childdoc.def|
is installed on the target \TeX{} distribution
and one prefers not to ship it,
it is conceivable to paste a few relevant commands into the sources.

To that end, drop all statements |\input{childdoc.def}|
and perform the replacements as outlined below.
Instead of |\childdocmain{|\textit{main}|}| add the following code
to the top of the main file:
%
\begin{center}
\begin{tabular}{l}
|\||ifdefined\childdocname\endinput\||fi\newif\ifchilddoc|\\
|\edef\childdocname{\scantokens\expandafter{\jobname\noexpand}}|\\
|\def\childdocmain{|\textit{main}|}\||ifx\childdocmain\childdocname\||else|\\
|\childdoctrue\includeonly{\childdocname}\let\jobname\childdocmain\||fi|\\
\end{tabular}
\end{center}
%
Instead of |\childdocof{|\textit{main}|}| just include the main file
at the top of each child file:
%
\begin{center}
|\input{|\textit{main}|}|
\end{center}
%
A simple redirection |\childdocforward{|\textit{dest}|}| is achieved by:
%
\begin{center}
|\def\jobname{|\textit{dest}|}\input{\jobname}|
\end{center}
%
The redirection with prefix
|\childdocforwardprefix[|\textit{prefix}|]{|\textit{dest}|}|
is accomplished by:
%
\begin{center}
\begin{tabular}{l}
|{\edef\jobname{\scantokens\expandafter{\jobname\noexpand}}|\\
|\def\redirectjob |\textit{prefix}|#1~~~{\gdef\jobname{|\textit{dest}|#1}}|\\
|\expandafter\redirectjob\jobname~~~}\input{\jobname}|
\end{tabular}
\end{center}

In an alternative approach,
child documents can be compiled by a specific command line
without additional code or specific definitions:
%
\begin{center}
|... -jobname "|\textit{target}|" "|[\textit{flags}]%
|\includeonly{|\textit{dest}|}\input{|\textit{main}|}"|
\end{center}
%

%%%%%%%%%%%%%%%%%%%%%%%%%%%%%%%%%%%%%%%%%%%%%%%%%%%%%%%%%%%%%%%%%%%%%%%%%%%%%%%%
%%%%%%%%%%%%%%%%%%%%%%%%%%%%%%%%%%%%%%%%%%%%%%%%%%%%%%%%%%%%%%%%%%%%%%%%%%%%%%%%
\section{Information}

%%%%%%%%%%%%%%%%%%%%%%%%%%%%%%%%%%%%%%%%%%%%%%%%%%%%%%%%%%%%%%%%%%%%%%%%%%%%%%%%
\subsection{Copyright}

Copyright \copyright{} 2017--2018 Niklas Beisert

This work may be distributed and/or modified under the
conditions of the \LaTeX{} Project Public License, either version 1.3
of this license or (at your option) any later version.
The latest version of this license is in
  \url{http://www.latex-project.org/lppl.txt}
and version 1.3 or later is part of all distributions of \LaTeX{}
version 2005/12/01 or later.

This work has the LPPL maintenance status `maintained'.

The Current Maintainer of this work is Niklas Beisert.

This work consists of the files |README.txt|, |childdoc.ins| and |childdoc.dtx|
as well as the derived files |childdoc.def|, |cdocsamp.tex|
with |cdocsch1.tex|, |cdocsch2.tex|, |cdocspt3.tex|, |cdocspt4.tex|,
|cdocsdrf.tex|, |cdocsfn1.tex|, |cdocsfn2.tex|
as well as |childdoc.pdf|.

%%%%%%%%%%%%%%%%%%%%%%%%%%%%%%%%%%%%%%%%%%%%%%%%%%%%%%%%%%%%%%%%%%%%%%%%%%%%%%%%
\subsection{Files and Installation}

The package consists of the files:
%
\begin{center}
\begin{tabular}{ll}
    |README.txt|   & readme file \\
    |childdoc.ins| & installation file \\
    |childdoc.dtx| & source file \\
    |childdoc.def| & definition file \\
    |cdocsamp.tex| & sample main file \\
    |cdocsch1.tex| & sample include file \\
    |cdocsch2.tex| & sample include file \\
    |cdocspt3.tex| & sample part file \\
    |cdocspt4.tex| & sample part file \\
    |cdocsdrf.tex| & sample redirection file \\
    |cdocsfn1.tex| & sample redirection file \\
    |cdocsfn2.tex| & sample redirection file \\
    |childdoc.pdf| & manual
\end{tabular}
\end{center}
%
The distribution consists of the files
|README.txt|, |childdoc.ins| and |childdoc.dtx|.
%
\begin{itemize}
\item
Run (pdf)\LaTeX{} on |childdoc.dtx|
to compile the manual |childdoc.pdf| (this file).
\item
Run \LaTeX{} on |childdoc.ins| to create the definitions file |childdoc.def|
and the sample |cdocsamp.tex| with include files
|cdocsch1.tex|, |cdocsch2.tex|, |cdocspt3.tex|, |cdocspt4.tex|,
|cdocsdrf.tex|, |cdocsfn1.tex|, |cdocsfn2.tex|.
Then copy the file |childdoc.def| to an appropriate directory of your \LaTeX{}
distribution, e.g.\ \textit{texmf-root}|/tex/latex/childdoc|.
\end{itemize}

%%%%%%%%%%%%%%%%%%%%%%%%%%%%%%%%%%%%%%%%%%%%%%%%%%%%%%%%%%%%%%%%%%%%%%%%%%%%%%%%
\subsection{Related CTAN Packages}

There are several other packages which offer a similar functionality:
%
\begin{itemize}
\item
The packages
\href{http://ctan.org/pkg/docmute}{\textsf{docmute}},
\href{http://ctan.org/pkg/includex}{\textsf{includex}} and
\href{http://ctan.org/pkg/standalone}{\textsf{standalone}}
provide commands to include only the document body of
a child file thus allowing both files to be compiled individually.
\item
The packages \href{http://ctan.org/pkg/subdocs}{\textsf{subdocs}}
and \href{http://ctan.org/pkg/subfiles}{\textsf{subfiles}}
provide structures in which the main and child documents can be
encapsulated and allowing them to be compiled individually.
The inclusion mechanism is different from the conventional |\include|.
\item
The package \href{http://ctan.org/pkg/combine}{\textsf{combine}}
is an elaborate solution to combine several documents into one.
\end{itemize}
%
See also the CTAN topic \href{http://ctan.org/topic/subdocs}{\textsf{subdocs}}
for further related packages.
The present package differs from the above solutions in that
a document structure constructed with the conventional |\include| mechanism
just needs two extra commands at the top of every file
such that all constituent files can be compiled individually.

%%%%%%%%%%%%%%%%%%%%%%%%%%%%%%%%%%%%%%%%%%%%%%%%%%%%%%%%%%%%%%%%%%%%%%%%%%%%%%%%
%\subsection{Feature Suggestions}
%
%The following is a list of features which may be useful for future
%versions of this package:
%%
%\begin{itemize}
%\item
%\ldots
%\end{itemize}

%%%%%%%%%%%%%%%%%%%%%%%%%%%%%%%%%%%%%%%%%%%%%%%%%%%%%%%%%%%%%%%%%%%%%%%%%%%%%%%%
\subsection{Revision History}

%%%%%%%%%%%%%%%%%%%%%%%%%%%%%%%%%%%%%%%%
\paragraph{v2.0:} 2018/12/30

\begin{itemize}
\item
immediate forward processing
\item
added |\childdocby| mechanism
\item
manual restructured
\end{itemize}

%%%%%%%%%%%%%%%%%%%%%%%%%%%%%%%%%%%%%%%%
\paragraph{v1.6:} 2018/01/17

\begin{itemize}
\item
application for development of include files
\item
corrections to manual
\end{itemize}

%%%%%%%%%%%%%%%%%%%%%%%%%%%%%%%%%%%%%%%%
\paragraph{v1.5:} 2017/05/21

\begin{itemize}
\item
more complete structuring introduced
\item
|\childdocof| introduced
\item
|\childdoc| renamed to |\childdocmain|
\item
|\childredirect| renamed to |\childdocforward| and |\childdocforwardprefix|
and functionality expanded
\end{itemize}

%%%%%%%%%%%%%%%%%%%%%%%%%%%%%%%%%%%%%%%%
\paragraph{v1.0:} 2017/04/27

\begin{itemize}
\item
manual and install package
\item
first version published on CTAN
\end{itemize}

%%%%%%%%%%%%%%%%%%%%%%%%%%%%%%%%%%%%%%%%
\paragraph{v0.6:} 2017/04/26

\begin{itemize}
\item
redirection mechanism added
\end{itemize}

%%%%%%%%%%%%%%%%%%%%%%%%%%%%%%%%%%%%%%%%
\paragraph{v0.5:} 2017/04/26

\begin{itemize}
\item
functionality in definition file
\end{itemize}


%%%%%%%%%%%%%%%%%%%%%%%%%%%%%%%%%%%%%%%%%%%%%%%%%%%%%%%%%%%%%%%%%%%%%%%%%%%%%%%%
%%%%%%%%%%%%%%%%%%%%%%%%%%%%%%%%%%%%%%%%%%%%%%%%%%%%%%%%%%%%%%%%%%%%%%%%%%%%%%%%
%%%%%%%%%%%%%%%%%%%%%%%%%%%%%%%%%%%%%%%%%%%%%%%%%%%%%%%%%%%%%%%%%%%%%%%%%%%%%%%%
\appendix

\settowidth\MacroIndent{\rmfamily\scriptsize 000\ }

 \DocInput{childdoc.dtx}

\end{document}
%</driver>
% \fi
%
% %%%%%%%%%%%%%%%%%%%%%%%%%%%%%%%%%%%%%%%%%%%%%%%%%%%%%%%%%%%%%%%%%%%%%%%%%%%%%%
% %%%%%%%%%%%%%%%%%%%%%%%%%%%%%%%%%%%%%%%%%%%%%%%%%%%%%%%%%%%%%%%%%%%%%%%%%%%%%%
% \section{Sample}
%\iffalse
%<*samplemain>
%\fi
%
% The following presents a sample document
% with two chapters, two parts, a title page,
% a compile flag as well as three forwarding files to set the flag.
% It consists of eight |.tex| files:
% \begin{center}
% \begin{tabular}{ll}
% |cdocsamp.tex|&main file\\
% |cdocsch1.tex|&include file for chapter 1\\
% |cdocsch2.tex|&include file for chapter 2\\
% |cdocspt3.tex|&include file for part 3\\
% |cdocspt4.tex|&include file for part 4\\
% |cdocsdrf.tex|&forwarding file for main file in draft mode\\
% |cdocsfi1.tex|&forwarding file for final version of chapter 1\\
% |cdocsfi2.tex|&forwarding file for final version of chapter 2\\
% \end{tabular}
% \end{center}
% Each of the eight files can be compiled directly by the \LaTeX{} compiler.
%
% %%%%%%%%%%%%%%%%%%%%%%%%%%%%%%%%%%%%%%
% \paragraph{Main File.}
%
% The main file is called |cdocsamp.tex|.
%
% Load the \textsf{childdoc} definitions and
% declare the filename for the main document:
%    \begin{macrocode}
\input{childdoc.def}
\childdocmain{}
%    \end{macrocode}

% Optional override for |\version| flag:
%    \begin{macrocode}
%%\ifchilddoc\else\providecommand{\version}{draft}\fi
%    \end{macrocode}

% Define the default values for the |\version| flag
% (|final| for the main file and |draft| for childs):
%    \begin{macrocode}
\ifchilddoc
\providecommand{\version}{draft}
\else
\providecommand{\version}{final}
\fi
%    \end{macrocode}

% Load the standard document class:
%    \begin{macrocode}
\documentclass[12pt]{article}
%    \end{macrocode}

% Start the document body:
%    \begin{macrocode}
\begin{document}
%    \end{macrocode}

% Declare a title page.
% Print title, part of document being processed and version flag:
%    \begin{macrocode}
\addtocounter{page}{-1}
\begin{center}
{\LARGE\bfseries{}childdoc example\par}
\vspace{1cm}
\ifchilddoc
\ifchilddocmanual part\else chapter\fi:
`\childdocname' of `\childdocjob'\par
\else
main document: `\childdocjob'\par
\fi
version: \version\par
\end{center}
\newpage
%    \end{macrocode}

% Manually include selected file,
% otherwise process as usual:
%    \begin{macrocode}
\ifchilddocmanual
\section*{part `\childdocname'}
\input{\childdocname}
\else
%    \end{macrocode}

% Include the two chapters:
%    \begin{macrocode}
\include{cdocsch1}
\include{cdocsch2}
%    \end{macrocode}

% Include the two parts unless only chapters should be displayed:
%    \begin{macrocode}
\ifchilddoc\else
\section{part three}
\input{cdocspt3}
\section{part four}
\input{cdocspt4}
\fi
%    \end{macrocode}

% Process as usual until here:
%    \begin{macrocode}
\fi
%    \end{macrocode}

% End of document body:
%    \begin{macrocode}
\end{document}
%    \end{macrocode}
%\iffalse
%</samplemain>
%\fi
%
% %%%%%%%%%%%%%%%%%%%%%%%%%%%%%%%%%%%%%%
% \paragraph{Chapter Include Files.}
%
% The include files are called |cdocsch1.tex| and |cdocsch2.tex|.
%
%\iffalse
%<*samplechap1|samplechap2>
%\fi

% Optional override for |\version| flag:
%    \begin{macrocode}
%%\providecommand{\version}{final}
%    \end{macrocode}

% Include the main document:
%    \begin{macrocode}
\input{childdoc.def}
\childdocof{cdocsamp}
%    \end{macrocode}

%\iffalse
%</samplechap1|samplechap2>
%\fi
%
%\iffalse
%<*samplechap1>
%\fi
% Some text for chapter 1:
%    \begin{macrocode}
\section{one}
some text in chapter one
%    \end{macrocode}

%\iffalse
%</samplechap1>
%\fi
% Some text for chapter 2:
%\iffalse
%<*samplechap2>
%\fi
%    \begin{macrocode}
\section{two}
more text in chapter two
%    \end{macrocode}

%\iffalse
%</samplechap2>
%\fi
%
% %%%%%%%%%%%%%%%%%%%%%%%%%%%%%%%%%%%%%%
% \paragraph{Part Include Files.}
%
% The include files are called |cdocspt3.tex| and |cdocspt4.tex|.
%
%\iffalse
%<*samplepart3|samplepart4>
%\fi

% Optional override for |\version| flag:
%    \begin{macrocode}
%%\providecommand{\version}{final}
%    \end{macrocode}

% Include the main document:
%    \begin{macrocode}
\input{childdoc.def}
\childdocby{cdocsamp}
%    \end{macrocode}

%\iffalse
%</samplepart3|samplepart4>
%\fi
%
%\iffalse
%<*samplepart3>
%\fi
% Some text for part 3:
%    \begin{macrocode}
some text in part three
%    \end{macrocode}

%\iffalse
%</samplepart3>
%\fi
% Some text for part 4:
%\iffalse
%<*samplepart4>
%\fi
%    \begin{macrocode}
more text in part four
%    \end{macrocode}

%\iffalse
%</samplepart4>
%\fi
%
% %%%%%%%%%%%%%%%%%%%%%%%%%%%%%%%%%%%%%%
% \paragraph{Forwarding for a Complete Draft.}
%
% The following forwarding file |cdocsdrf.tex|
% compiles the main document in draft mode:
%\iffalse
%<*sampledraft>
%\fi
%    \begin{macrocode}
\def\version{draft}
\input{childdoc.def}
\childdocforward{cdocsamp}
%    \end{macrocode}

%\iffalse
%</sampledraft>
%\fi
%
% %%%%%%%%%%%%%%%%%%%%%%%%%%%%%%%%%%%%%%
% \paragraph{Forwarding for Final Version of the Chapters.}
%
% The following forwarding files |cdocsfn1.tex| and |cdocsfn2.tex|
% (with identical content)
% compile the final versions of the child documents
% |cdocsch1.tex| and |cdocsch2.tex|, respectively:
%\iffalse
%<*samplefinal>
%\fi
%    \begin{macrocode}
\def\version{final}
\input{childdoc.def}
\childdocforwardprefix[cdocsamp]{cdocsfn}{cdocsch}
%    \end{macrocode}

%\iffalse
%</samplefinal>
%\fi
%
% %%%%%%%%%%%%%%%%%%%%%%%%%%%%%%%%%%%%%%
% \paragraph{Command Line Processing.}
%
% The following three command lines generate the output files
% |cdocscld|, |cdocscl1| and |cdocscl2|
% which should be identical to
% |cdocsdrf|, |cdocsch1| and |cdocsfn2|, respectively:
% \begin{center}
% \begin{tabular}{l}
% |latex -jobname cdocscld \|\\
% |  "\def\version{draft}\input{childdoc.def}\childdocforward{cdocsamp}"|\\
% |latex -jobname cdocscl1 \|\\
% |  "\input{childdoc.def}\childdocforward[cdocsamp]{cdocsch1}"|\\
% |latex -jobname cdocscl2 \|\\
% |  "\def\version{final}\input{childdoc.def}\childdocforward{cdocsch2}"|
% \end{tabular}
% \end{center}
% Note that the trailing backslash on each first line
% merely continues the input to the second line
% (for convenient cut ant paste).
% Furthermore, the command |latex| can be replaced by any
% of its alternative versions such as |pdflatex|.
%
% %%%%%%%%%%%%%%%%%%%%%%%%%%%%%%%%%%%%%%%%%%%%%%%%%%%%%%%%%%%%%%%%%%%%%%%%%%%%%%
% %%%%%%%%%%%%%%%%%%%%%%%%%%%%%%%%%%%%%%%%%%%%%%%%%%%%%%%%%%%%%%%%%%%%%%%%%%%%%%
% \section{Implementation}
%\iffalse
%<*package>
%\fi
%
% This section describes the definitions file |childdoc.def|.

% The definitions cannot be loaded using |\usepackage| or |\RequirePackage|
% which has a mechanism to prevent loading a style file more than once.
% When loading the definitions by means of |\input|
% multiple instances have to be prevented manually:
%\iffalse
%This code needs to be before the `\ProvidesFile' directive
%which is defined at the beginning of this file.
%Therefore it is also placed there and commented out here.
%</package>
%<*discard>
%\fi
%    \begin{macrocode}
\ifdefined\childdocmain\endinput\fi
%    \end{macrocode}
%\iffalse
%</discard>
%<*package>
%\fi
%
% \macro{\ifchilddoc}
% \macro{\ifchilddocmanual}
% The conditional |\ifchilddoc| tells whether a
% child (true) or main (false) document is being compiled.
% The conditional |\ifchilddocmanual| tells whether
% the |\includeonly| mechanism is used (false) or
% the selection of child files must be performed manually (true).
% The definitions initialise to false:
%    \begin{macrocode}
\newif\ifchilddoc
\newif\ifchilddocmanual
%    \end{macrocode}

% \macro{\childdocname}
% \macro{\childdocjob}
% The macro |\childdocname| stores the name of the main document
% to be compiled. The macro |\childdocjob| stores the name of
% the document on which the \LaTeX{} compiler was originally invoked.
% The content of |\jobname| cannot be compared
% to filenames specified in the source due to different catcodes.
% The following code rescans |\jobname|, stores the result
% in |\childdocname| and saves a copy in |\childdocjob|:
%    \begin{macrocode}
\edef\childdocname{\scantokens\expandafter{\jobname\noexpand}}
\let\childdocjob\childdocname
%    \end{macrocode}

% \macro{\childdocdisable}
% The macro |\childdocdisable| prevents the main file
% from being processed more than once.
% At this stage, the main document command |\childdocmain|
% is assumed to be called once again where it should do nothing.
% Any subsequent call to it should prevent
% a secondary processing of the main document
% It overwrites the forwarding commands
% |\childdocof| and |\childdocforward|
% with empty macros to prevent further inclusions of the main document:
%    \begin{macrocode}
\newcommand{\childdocdisable}
{
  \renewcommand{\childdocmain}[1]{\renewcommand{\childdocmain}[1]{\endinput}}
  \renewcommand{\childdocof}[1]{}
  \renewcommand{\childdocby}[2][]{}
  \renewcommand{\childdocforward}[2][]{}
  \renewcommand{\childdocdisable}{}
}
%    \end{macrocode}

% \macro{\childdocmain}
% The macro |\childdocmain| is to be called at the top of the main file
% with nothing or the main filename (without extension) as argument.
% First, it breaks loops.
% If the argument is not empty and does not match |\childdocname|
% (which is set by the first inclusion of |childdoc.def|),
% |\ifchilddoc| is set to true, |\includeonly| is applied to the child file
% and |\jobname| is set to the main file
% (for proper handling of |.aux| files):
%    \begin{macrocode}
\newcommand{\childdocmain}[1]
{
  \childdocdisable\childdocmain{}
  \if?#1?\else
    \begingroup
      \def\childdoctmp{#1}
      \ifx\childdoctmp\childdocname
        \def\childdoctmp{}
      \else
        \def\childdoctmp
        {
          \childdoctrue
          \includeonly{\childdocname}
          \def\childdocjob{#1}
          \def\jobname{#1}
        }
      \fi
      \expandafter
    \endgroup
    \childdoctmp
  \fi
}
%    \end{macrocode}

% \macro{\childdocof}
% The command |\childdocof| redirects
% compilation to the main file |#1|.
%    \begin{macrocode}
\newcommand{\childdocof}[1]
{
  \childdocdisable
  \childdoctrue
  \includeonly{\childdocname}
  \def\jobname{#1}
  \def\childdocjob{#1}
  \input{#1}
}
%    \end{macrocode}

% \macro{\childdocby}
% The command |\childdocby| ....
%    \begin{macrocode}
\newcommand{\childdocby}[2][]
{
  \childdocdisable
  \childdoctrue
  \childdocmanualtrue
  \if?#1?\else
    \def\jobname{#2}
  \fi
  \def\childdocjob{#2}
  \input{#2}
  \endinput
}
%    \end{macrocode}

% \macro{\childdocforward}
% The command |\childdocforward| redirects
% compilation to the main file or
% (if the optional argument is given) a child file.
% Parameters are set as if the main file
% or a child file starting with |\childdocof| was compiled.
% Then compilation is handed over to the main file:
%    \begin{macrocode}
\newcommand{\childdocforward}[2][]
{
  \begingroup
    \if?#1?
      \def\childdoctmp
      {
        \def\childdocname{#2}
        \def\childdocjob{#2}
        \def\jobname{#2}
        \input{#2}
        \endinput
      }
    \else
      \def\childdoctmp
      {
        \childdocdisable
        \def\childdocname{#2}
        \childdoctrue
        \includeonly{#2}
        \def\childdocjob{#1}
        \def\jobname{#1}
        \input{#1}
        \endinput
      }
    \fi
    \expandafter
  \endgroup
  \childdoctmp
}
%    \end{macrocode}

% \macro{\childdocforwardprefix}
% The command |\childdocforwardprefix| redirects
% compilation to the main or a child file by means of a pattern.
% The prefix |#1| in the current filename is replaced by |#2|
% and the suffix of the current filename is kept
% (it is assumed that the filename does not contain the substring `|~~~|'
% which is used as a delimiter).
% Compilation is handed over to the new file by |\childdocforward|:
%    \begin{macrocode}
\newcommand{\childdocforwardprefix}[3][]
{
  \begingroup
    \def\childdocextract #2##1~~~{\def\childdoctmp{\childdocforward[#1]{#3##1}}}
    \expandafter\childdocextract\childdocname~~~
    \expandafter
  \endgroup
  \childdoctmp
}
%    \end{macrocode}

% \macro{\childdoc}
% The deprecated macro |\childdoc| is a legacy version of |\childdocmain|:
%    \begin{macrocode}
\newcommand{\childdoc}{\childdocmain}
%    \end{macrocode}

% \macro{\childdocredirect}
% The deprecated macro |\childdocredirect| is a legacy version
% of |\childdocforward| and |\childdocforwardprefix|:
%    \begin{macrocode}
\newcommand{\childdocredirect}[2][]
{
  \begingroup
    \if?#1?
      \def\childdoctmp{\childdocforward{#2}}
    \else
      \def\childdoctmp{\childdocforwardprefix{#1}{#2}}
    \fi
    \expandafter
  \endgroup
  \childdoctmp
}
%    \end{macrocode}

%\iffalse
%</package>
%\fi
%
\endinput
\childdocforward[cdocsamp]{cdocsch1}"|\\
% |latex -jobname cdocscl2 \|\\
% |  "\def\version{final}% \iffalse
%
% childdoc.dtx Copyright (C) 2017-2018 Niklas Beisert
%
% This work may be distributed and/or modified under the
% conditions of the LaTeX Project Public License, either version 1.3
% of this license or (at your option) any later version.
% The latest version of this license is in
%   http://www.latex-project.org/lppl.txt
% and version 1.3 or later is part of all distributions of LaTeX
% version 2005/12/01 or later.
%
% This work has the LPPL maintenance status `maintained'.
%
% The Current Maintainer of this work is Niklas Beisert.
%
% This work consists of the files childdoc.dtx and childdoc.ins
% and the derived files childdoc.def and cdocsamp.tex with
% cdocsch1.tex, cdocsch2.tex, cdocsdrf.tex, cdocsfn1.tex, cdocsfn2.tex.
%
%<package>\ifdefined\childdocmain\endinput\fi
%<package>\ProvidesFile{childdoc.def}[2018/12/30 v2.0 child document driver]
%<samplemain>\ProvidesFile{cdocsamp.tex}[2018/12/30 v2.0 sample for childdoc]
%<*driver>
%\ProvidesFile{childdoc.drv}[2018/12/30 v2.0 childdoc reference manual file]
\PassOptionsToClass{10pt,a4paper}{article}
\documentclass{ltxdoc}

\usepackage[margin=35mm]{geometry}
\usepackage{hyperref}
\usepackage{hyperxmp}
\usepackage[usenames]{color}

\hypersetup{colorlinks=true}
\hypersetup{pdfstartview=FitH}
\hypersetup{pdfpagemode=UseNone}
\hypersetup{pdfsource={}}
\hypersetup{pdflang={en-UK}}
\hypersetup{pdfcopyright={Copyright 2017-2018 Niklas Beisert.
  This work may be distributed and/or modified under the
  conditions of the LaTeX Project Public License, either version 1.3
  of this license or (at your option) any later version.}}
\hypersetup{pdflicenseurl={http://www.latex-project.org/lppl.txt}}
\hypersetup{pdfcontactaddress={ETH Zurich, ITP, HIT K,
  Wolfgang-Pauli-Strasse 27}}
\hypersetup{pdfcontactpostcode={8093}}
\hypersetup{pdfcontactcity={Zurich}}
\hypersetup{pdfcontactcountry={Switzerland}}
\hypersetup{pdfcontactemail={nbeisert@itp.phys.ethz.ch}}
\hypersetup{pdfcontacturl={http://people.phys.ethz.ch/\xmptilde nbeisert/}}

\newcommand{\secref}[1]{\hyperref[#1]{section \ref*{#1}}}

\parskip1ex
\parindent0pt
\let\olditemize\itemize
\def\itemize{\olditemize\parskip0pt}

\begin{document}

\title{The \textsf{childdoc} Package}
\hypersetup{pdftitle={The childdoc Package}}
\author{Niklas Beisert\\[2ex]
  Institut f\"ur Theoretische Physik\\
  Eidgen\"ossische Technische Hochschule Z\"urich\\
  Wolfgang-Pauli-Strasse 27, 8093 Z\"urich, Switzerland\\[1ex]
  \href{mailto:nbeisert@itp.phys.ethz.ch}
  {\texttt{nbeisert@itp.phys.ethz.ch}}}
\hypersetup{pdfauthor={Niklas Beisert}}
\hypersetup{pdfsubject={Manual for the LaTeX2e Package childdoc}}
\date{30 December 2018, \textsf{v2.0}}
\maketitle

\begin{abstract}\noindent
\textsf{childdoc} is a \LaTeXe{} package
that enables the direct compilation
of document sections included by |\include|
to individual files.
\end{abstract}

\begingroup
\parskip0ex
\tableofcontents
\endgroup

%%%%%%%%%%%%%%%%%%%%%%%%%%%%%%%%%%%%%%%%%%%%%%%%%%%%%%%%%%%%%%%%%%%%%%%%%%%%%%%%
%%%%%%%%%%%%%%%%%%%%%%%%%%%%%%%%%%%%%%%%%%%%%%%%%%%%%%%%%%%%%%%%%%%%%%%%%%%%%%%%
\section{Introduction}

\LaTeX{} provides a mechanism to structure a large document (such as a book)
into a main file and several child files (containing the chapters)
using the |\include| command.
This mechanism is beneficial for documents
which span hundreds of pages in order to
make the source file(s) more manageable.
Moreover, compilation can be restricted to
selected child files by means of the |\includeonly| command.
The latter feature can be used to reduce the compilation time while editing
(this was significantly more useful in the earlier days of \LaTeX{})
or to generate a smaller document which is easier to navigate.
Another application of |\includeonly| is to generate
documents consisting of selected parts of the complete document.

However, there are a few drawbacks of the plain |\include| mechanism:
\begin{itemize}
\item
The child files cannot be compiled on their own,
they can only be compiled via the main file.
A naive editing environment
(such as a text editor with an option
to have the current file processed by \LaTeX)
may require one to switch to the main file before compiling;
attempting to compile the child file produces errors.
\item
The main file must be modified (each time)
to adjust the |\includeonly| command
to the present needs. This easily leaves the main file in a messy state.
\item
The generated document will always carry the filename
of the main document. This is inconvenient if
several child files are to be compiled and
to be kept for distribution.
\end{itemize}

The present package provides a simple interface
to make child files individually compilable by \LaTeX{}.
Compiling a child file then has the same effect as compiling
the main file with an |\includeonly| command
to select the appropriate child.
Moreover the generated document will carry the name of the child
rather than the main file.
This resolves all three above issues.

This feature is meant to make the editing of books,
thesis documents and lecture notes somewhat more convenient.
However, the package can also be used efficiently for
composing a series of documents (such as exercise sheets)
which are typically distributed individually.
It then assists the author in generating the individual documents
(potentially in different versions)
as well as a document containing the collected series.
Another application is in developing style files
or other kinds of included material
where compilation of the style file could redirect
to a sample or test file.

%%%%%%%%%%%%%%%%%%%%%%%%%%%%%%%%%%%%%%%%%%%%%%%%%%%%%%%%%%%%%%%%%%%%%%%%%%%%%%%%
%%%%%%%%%%%%%%%%%%%%%%%%%%%%%%%%%%%%%%%%%%%%%%%%%%%%%%%%%%%%%%%%%%%%%%%%%%%%%%%%
\section{Usage}

First of all, the package \textsf{childdoc} is \emph{not} a standard
\LaTeXe{} |.sty| style file! Therefore it needs to be invoked in
a non-standard way.

%%%%%%%%%%%%%%%%%%%%%%%%%%%%%%%%%%%%%%%%%%%%%%%%%%%%%%%%%%%%%%%%%%%%%%%%%%%%%%%%
\subsection{Included Files}
\label{sec:include}

%%%%%%%%%%%%%%%%%%%%%%%%%%%%%%%%%%%%%%%%
\DescribeMacro{\childdocmain}
To use the package, add the commands
\begin{center}
\begin{tabular}{l}
|\input{childdoc.def}|\\
|\childdocmain{}|\\
\end{tabular}
\end{center}
at the very top of the main \LaTeX{} file,
in particular \emph{before} the |\documentclass| statement!
The argument of |\childdocmain| should be left empty
(but it must be present).

%%%%%%%%%%%%%%%%%%%%%%%%%%%%%%%%%%%%%%%%
\DescribeMacro{\childdocof}
Furthermore, add the commands
\begin{center}
\begin{tabular}{l}
|\input{childdoc.def}|\\
|\childdocof{|\textit{main}|}|\\
\end{tabular}
\end{center}
at the top of every child file \textit{child}
which is included by |\include{|\textit{child}|}|
from within the main file
(or at least for those files to be compiled individually).
The argument \textit{main} must be the filename of the main file.

There are a couple of
considerations in setting up the main and child documents:

%%%%%%%%%%%%%%%%%%%%%%%%%%%%%%%%%%%%%%%%
\paragraph{Restrictions.}

Please note the following restrictions:
\begin{itemize}
\item
|\childdocmain| must be called with one argument \textit{main}
to ensure compatibility with earlier version of the package.
It must either be empty (|\childdocmain{}|)
or precisely match the filename of the main file in which it is specified.
See \secref{sec:detection} for further information.
\item
The filename \textit{main} must be specified without the |.tex| extension.
\item
The filename \textit{main} is case sensitive
(even in case-insensitive file systems)
due to internal string comparison.
\item
The argument \textit{main} should be fully expanded, it cannot be a macro.
\item
Subdirectories and special characters should be avoided in filenames.
\item
The command |\childdocmain{|\textit{main}|}| must be followed by a whitespace.
It should not be followed immediately by another command
or by a comment mark `|%|'.
This is because the \TeX{} parser reads the token immediately following
the argument of |\childdocmain| and puts it
at the beginning of every child section;
however, a white\-space is ignored.
\end{itemize}

%%%%%%%%%%%%%%%%%%%%%%%%%%%%%%%%%%%%%%%%
\paragraph{Content of Main File.}

It is advisable to place all content in the child files included by |\include|.
Any output contained in the main file will appear in all child documents
unless suppressed manually;
it cannot be suppressed automatically by the |\includeonly| directive
and thus should normally be avoided.
A method to include some content in the main file
by means of conditional processing is described in \secref{sec:conditional}.

%%%%%%%%%%%%%%%%%%%%%%%%%%%%%%%%%%%%%%%%
\paragraph{Page Numbering.}

When only a part of the document is compiled,
the appropriate numbering of pages
(as well as other status parameters)
is determined from the |.aux| files.
The latter contain information from previous passes.
However this information needs to propagate through
all intermediate child documents.
Therefore the page numbering in child documents may well
be inconsistent until the complete document is compiled at least once.

A useful (if unconventional) way to always ensure a consistent
page numbering is to restart the numbering in each child document
and denote the pages by `\textit{child}|.|\textit{page}'
where \textit{child} represents the chapter/section number of the child file.
This can be achieved by the command
|\numberwithin{page}{|\textit{child}|}|
of the \textsf{amsmath} package
where \textit{child} can be |chapter| or |section|
depending on the chosen structuring.
Alternatively, one can modify the macro |\thepage| appropriately
and reset the counter |page| at the start of each child file.

%%%%%%%%%%%%%%%%%%%%%%%%%%%%%%%%%%%%%%%%%%%%%%%%%%%%%%%%%%%%%%%%%%%%%%%%%%%%%%%%
\subsection{Conditional Processing}
\label{sec:conditional}

The package provides a mechanism to compile different versions
of a document. To customise the versions further some conditional processing
can come in handy to distinguish which version is being compiled.
The package provides two macros to describe the compilation context:

%%%%%%%%%%%%%%%%%%%%%%%%%%%%%%%%%%%%%%%%
\DescribeMacro{\ifchilddoc}
The conditional |\ifchilddoc| distinguishes between the compilation of
child documents and the main document:
%
\begin{center}
|\ifchilddoc |\textit{child-code}| |[|\||else |\textit{main-code}]| \||fi|
\end{center}

%%%%%%%%%%%%%%%%%%%%%%%%%%%%%%%%%%%%%%%%
\DescribeMacro{\childdocname}
\DescribeMacro{\childdocjob}
The macro |\childdocname| contains the filename (without extension)
of the main or child file being processed.
Note that |\childdocjob| will always contain the name of the main file.

%%%%%%%%%%%%%%%%%%%%%%%%%%%%%%%%%%%%%%%%
\paragraph{Title Page.}

Conditional processing can be used to include a title or banner page
in the main document when proper precautions are taken.
Importantly, the code in the main file should ensure that the page counter
(as well as other status parameters which are stored in the |.aux| files)
takes the same value after the conditional processing.
Otherwise the page numbers may take divergent values
depending on which part is compiled.

For example, a title page could be declared by:
%
\begin{center}
\begin{tabular}{l}
|\ifchilddoc\||else|\\
|\addtocounter{page}{-1}|\\
\textit{code for title page}\\
|\newpage|\\
|\||fi|
\end{tabular}
\end{center}
%
A banner page for the child documents can be generated by:
%
\begin{center}
\begin{tabular}{l}
|\ifchilddoc|\\
|\addtocounter{page}{-1}|\\
\textit{code for banner page}\\
|\newpage|\\
|\||fi|
\end{tabular}
\end{center}
%
Here one could write a message such as:
\begin{center}
|This is the part \childdocname{} of \childdocjob{}.|
\end{center}

%%%%%%%%%%%%%%%%%%%%%%%%%%%%%%%%%%%%%%%%%%%%%%%%%%%%%%%%%%%%%%%%%%%%%%%%%%%%%%%%
\subsection{Flags}
\label{sec:flags}

The package makes it easy to generate different versions
of the main or child documents.
To this end compilation flags can be defined
and assigned different default values.
They will be particularly useful in conjunction
with the forwarding mechanism described in \secref{sec:forward}.

For example, it may be useful to have a flag |\version|
which can be set to |draft| or |final|.
The document source will contain some conditional code
depending on the value of |\version|.
Suppose further, the flag should default to |final| for the main file
and to |draft| for child files
which is a natural assignment for editing the document.
This is achieved by placing the following code
in the preamble of the main document
(below the |\childdocmain| directive):
%
\begin{center}
\begin{tabular}{l}
|\ifchilddoc|\\
|\providecommand{\version}{draft}|\\
|\||else|\\
|\providecommand{\version}{final}|\\
|\||fi|
\end{tabular}
\end{center}
%
The definition by |\providecommand| makes sure
that previous definitions are not overwritten.
Further statements |\providecommand{\version}{...}|
can thus be added before the above code to override it.

For the main file, one might add a line
(between |\childdocmain| and the above block)
%
\begin{center}
|%\ifchilddoc\||else\providecommand{\version}{draft}\||fi|
\end{center}
%
which can be uncommented to produce a draft version.
Likewise one can add a line to the very top of a child file
(above the |\childdocof{|\textit{main}|}| directive)
%
\begin{center}
|%\providecommand{\version}{final}|
\end{center}
%
which can be uncommented to produce the final version of this child document.

%%%%%%%%%%%%%%%%%%%%%%%%%%%%%%%%%%%%%%%%%%%%%%%%%%%%%%%%%%%%%%%%%%%%%%%%%%%%%%%%
\subsection{Forwarding}
\label{sec:forward}

Different versions of the main or child documents
using compilation flags as described in \secref{sec:flags}
can be (permanently) stored in different files
for convenient compilation, viewing and distribution.
To this end, the package defines a command
to pass on compilation to a different file:

%%%%%%%%%%%%%%%%%%%%%%%%%%%%%%%%%%%%%%%%
\DescribeMacro{\childdocforward}
The command |\childdocforward| redirects processing to
another source file:
%
\begin{center}
\begin{tabular}{l}
|\input{childdoc.def}|\\
|\childdocforward[|\textit{main}|]{|\textit{dest}|}|\\
\end{tabular}
\end{center}
%
The argument \textit{dest} is the destination file
(without extension).
It should be the main file or one of the child files.
Note that further \textsf{childdoc} directives
such as |\childdocof| and |\childdocforward|
in the indicated file will be processed in this form.
The optional argument \textit{main}
passes on directly to the main file \textit{main}
while pretending to compile the child \textit{dest}.
This form behaves as if \textit{dest}
issues |\childdocof{|\textit{main}|}| right away,
and no further \textsf{childdoc} directives will be processed.

%%%%%%%%%%%%%%%%%%%%%%%%%%%%%%%%%%%%%%%%
\DescribeMacro{\...prefix}
In the alternative form |\childdocforwardprefix|,
%
\begin{center}
\begin{tabular}{l}
|\input{childdoc.def}|\\
|\childdocforwardprefix[|\textit{main}|]{|\textit{prefix}|}{|\textit{dest}|}|
\end{tabular}
\end{center}
%
the destination file is determined by a pattern
depending on the current file:
To make this work, the current file must be called
`{\textit{prefix}\hspace{0.2em}\textit{suffix}}'
with \textit{prefix} matching precisely the argument.
Processing is then passed on to the file
`{\textit{dest}\hspace{0.2em}\textit{suffix}}'.
Surely, the same effect is achieved by
directly specifying the
argument `{\textit{dest}\hspace{0.2em}\textit{suffix}}'
in the first form.
However, that requires to set up a different file
for each child. With the alternative form of the command
all these files can have exactly the same content
which simplifies setting them up and maintaining them.

For example, the following file |draft.tex|
with a compilation flag |\version| as described in \secref{sec:flags}
compiles the main document as a draft:
%
\begin{center}
\begin{tabular}{l}
|\def\version{draft}|\\
|\input{childdoc.def}|\\
|\childdocforward{|\textit{main}|}|
\end{tabular}
\end{center}
%
Likewise, the following files |final|\textit{nn}|.tex|
compile the final version of the child document
|child|\textit{nn}|.tex|:
%
\begin{center}
\begin{tabular}{l}
|\def\version{final}|\\
|\input{childdoc.def}|\\
|\childdocforwardprefix{final}{child}|
\end{tabular}
\end{center}
%

Note that when several versions of a main file and/or of each child file
are to be generated, it may be convenient to set up a |Makefile| or
shell script to automatise the process.

%%%%%%%%%%%%%%%%%%%%%%%%%%%%%%%%%%%%%%%%%%%%%%%%%%%%%%%%%%%%%%%%%%%%%%%%%%%%%%%%
\subsection{Command Line Processing}
\label{sec:commandline}

The effect of redirection files can also be achieved by invoking
the \LaTeX{} compiler with a more elaborate command line.
Most conveniently this should be done as part
of a shell script or a |Makefile|.

When using \textsf{childdoc} in the main file, the following
command lines effectively perform a redirection
(note that depending on the shell being used,
backslashes may have to be doubled: `|\|' $\to$ `|\\|'):
%
\begin{center}
|... -jobname "|\textit{target}|" |\\|"|[\textit{flags}]%
|\input{childdoc.def}\childdocforward[|\textit{main}|]{|\textit{dest}|}"|
\end{center}
%
Here \textit{target} is the name of the output file,
\textit{main} is the name of the main file
and \textit{dest} is the name of the main or child file to be processed
(all filenames without extensions).
The optional argument \textit{main} can be omitted
if \textit{main} matches \textit{dest}.
Optionally, compilation \textit{flags} can be defined via |\def| commands.
This command line makes the \TeX{} engine believe
it is compiling the file \textit{target}
whose content is specified as the latter parameter.
The provided code then forwards the processing to
\textit{main} or \textit{dest} as described in \secref{sec:forward}.

%%%%%%%%%%%%%%%%%%%%%%%%%%%%%%%%%%%%%%%%%%%%%%%%%%%%%%%%%%%%%%%%%%%%%%%%%%%%%%%%
\subsection{Include by Input}
\label{sec:input}

Including child documents by |\include| has some restrictions by design.
Most notably, the content of a child document always occupies
its own set of pages; pages cannot be shared between child documents.
Usually, this behaviour makes perfect sense
because each child document contain an essential part of the document.
However, in some situations it may be desirable to compose
a document from a collection of parts
without having mandatory page breaks between then.
For this case, the package
provides a mechanism to include parts
by |\input| which can also be processed individually.
However, by construction this mechanism
requires manual handling of the content to be output.

%%%%%%%%%%%%%%%%%%%%%%%%%%%%%%%%%%%%%%%%
\DescribeMacro{\ifchilddocmanual}
The main file should be prepared as usual, see \secref{sec:include}.
However, the document body must make a distinction
between processing of an individual part and of the main document, e.g.:
%
\begin{center}
\begin{tabular}{l}
|\ifchilddocmanual|\\
|\input{\childdocname}|\\
|\||else|\\
\textit{document body with }|\input{|\textit{part}|}|\\
|\||fi|
\end{tabular}
\end{center}
%
The conditional |\ifchilddocmanual| is true whenever
a part to be included by |\input| is being compiled,
and the name of the part is stored in |\childdocname|.

%%%%%%%%%%%%%%%%%%%%%%%%%%%%%%%%%%%%%%%%
\DescribeMacro{\childdocby}
Each part to be included by |\input| should start with:
%
\begin{center}
\begin{tabular}{l}
|\input{childdoc.def}|\\
|\childdocby{|\textit{main}|}|\\
\end{tabular}
\end{center}
%
The directive |\childdocby| is similar to |\childdocof|
described in \secref{sec:include},
but the subsequent selection of content must be done manually.
To that end, both |\ifchilddoc| and |\ifchilddocmanual|
will be true upon processing of a part,
and the name of the part is stored in |\childdocname|.
Note that |\jobname| will be set to the filename of the current part
so that each part receives an individual |.aux| file
that does not interfere with the |.aux| file(s) of the main document.
This behaviour can be altered by the alternative form
|\childdocby[*]{|\textit{main}|}| (with a non-empty optional argument)
which uses the |.aux| file of the main document
by setting |\jobname| to \textit{main}.

%%%%%%%%%%%%%%%%%%%%%%%%%%%%%%%%%%%%%%%%%%%%%%%%%%%%%%%%%%%%%%%%%%%%%%%%%%%%%%%%
\subsection{Driver Development}
\label{sec:driver}

The \textsf{childdoc} mechanism can also be use for the development
of definition files such as \LaTeX{} styles or classes.
This case differs from the above setup with multiple parts
included by |\include| in that no |\includeonly| should be invoked.
This can be achieved by starting the include file
(before |\ProvidesPackage|) with:
%
\begin{center}
\begin{tabular}{l}
|\input{childdoc.def}|\\
|\childdocforward{|\textit{main}|}|\\
\end{tabular}
\end{center}
%
or alternatively with:
%
\begin{center}
\begin{tabular}{l}
|\input{childdoc.def}|\\
|\childdocby{|\textit{main}|}|\\
\end{tabular}
\end{center}
%
Both forms have slightly different effects as described above.
The main file is prepared as usual, see \secref{sec:include}.

%%%%%%%%%%%%%%%%%%%%%%%%%%%%%%%%%%%%%%%%%%%%%%%%%%%%%%%%%%%%%%%%%%%%%%%%%%%%%%%%
\subsection{Legacy Detection}
\label{sec:detection}

The directive |\childdocmain| in the main file can detect
whether the complete document or merely a child is to be compiled
even without using the directive |\childdocof|.
This method is deprecated because it is less robust
and there is no compelling reason to use it;
it is merely provided for backward compatibility
and it may be removed in future versions.

If the detection mechanism is to be used,
it is mandatory to correctly specify
the filename of the main file as the argument of |\childdocmain|:
%
\begin{center}
\begin{tabular}{l}
|\input{childdoc.def}|\\
|\childdocmain{|\textit{main}|}|\\
\end{tabular}
\end{center}
%
If |\jobname| does not match the argument \textit{main} of |\childdocmain|,
it is assumed that |\jobname| points to the child file to be compiled.
When using |\childdocmain| with the main file specified as argument,
it suffices to start a child file
with just |\input{|\textit{main}|}|
without loading of the package and using |\childdocof|.
If instead all processing is done
with the appropriate \textsf{childdoc} directives,
the argument of \textit{main} of |\childdocmain| can be empty.

An alternative version of the command line processing described
in \secref{sec:commandline} using the detection mechanism reads:
%
\begin{center}
|... -jobname "|\textit{target}|" "|[\textit{flags}]%
[|\def\jobname{|\textit{dest}|}|]|\input{|\textit{main}|}"|
\end{center}

%%%%%%%%%%%%%%%%%%%%%%%%%%%%%%%%%%%%%%%%%%%%%%%%%%%%%%%%%%%%%%%%%%%%%%%%%%%%%%%%
\subsection{Manual Code}
\label{sec:manual}

In case one cannot be certain whether the definitions file |childdoc.def|
is installed on the target \TeX{} distribution
and one prefers not to ship it,
it is conceivable to paste a few relevant commands into the sources.

To that end, drop all statements |\input{childdoc.def}|
and perform the replacements as outlined below.
Instead of |\childdocmain{|\textit{main}|}| add the following code
to the top of the main file:
%
\begin{center}
\begin{tabular}{l}
|\||ifdefined\childdocname\endinput\||fi\newif\ifchilddoc|\\
|\edef\childdocname{\scantokens\expandafter{\jobname\noexpand}}|\\
|\def\childdocmain{|\textit{main}|}\||ifx\childdocmain\childdocname\||else|\\
|\childdoctrue\includeonly{\childdocname}\let\jobname\childdocmain\||fi|\\
\end{tabular}
\end{center}
%
Instead of |\childdocof{|\textit{main}|}| just include the main file
at the top of each child file:
%
\begin{center}
|\input{|\textit{main}|}|
\end{center}
%
A simple redirection |\childdocforward{|\textit{dest}|}| is achieved by:
%
\begin{center}
|\def\jobname{|\textit{dest}|}\input{\jobname}|
\end{center}
%
The redirection with prefix
|\childdocforwardprefix[|\textit{prefix}|]{|\textit{dest}|}|
is accomplished by:
%
\begin{center}
\begin{tabular}{l}
|{\edef\jobname{\scantokens\expandafter{\jobname\noexpand}}|\\
|\def\redirectjob |\textit{prefix}|#1~~~{\gdef\jobname{|\textit{dest}|#1}}|\\
|\expandafter\redirectjob\jobname~~~}\input{\jobname}|
\end{tabular}
\end{center}

In an alternative approach,
child documents can be compiled by a specific command line
without additional code or specific definitions:
%
\begin{center}
|... -jobname "|\textit{target}|" "|[\textit{flags}]%
|\includeonly{|\textit{dest}|}\input{|\textit{main}|}"|
\end{center}
%

%%%%%%%%%%%%%%%%%%%%%%%%%%%%%%%%%%%%%%%%%%%%%%%%%%%%%%%%%%%%%%%%%%%%%%%%%%%%%%%%
%%%%%%%%%%%%%%%%%%%%%%%%%%%%%%%%%%%%%%%%%%%%%%%%%%%%%%%%%%%%%%%%%%%%%%%%%%%%%%%%
\section{Information}

%%%%%%%%%%%%%%%%%%%%%%%%%%%%%%%%%%%%%%%%%%%%%%%%%%%%%%%%%%%%%%%%%%%%%%%%%%%%%%%%
\subsection{Copyright}

Copyright \copyright{} 2017--2018 Niklas Beisert

This work may be distributed and/or modified under the
conditions of the \LaTeX{} Project Public License, either version 1.3
of this license or (at your option) any later version.
The latest version of this license is in
  \url{http://www.latex-project.org/lppl.txt}
and version 1.3 or later is part of all distributions of \LaTeX{}
version 2005/12/01 or later.

This work has the LPPL maintenance status `maintained'.

The Current Maintainer of this work is Niklas Beisert.

This work consists of the files |README.txt|, |childdoc.ins| and |childdoc.dtx|
as well as the derived files |childdoc.def|, |cdocsamp.tex|
with |cdocsch1.tex|, |cdocsch2.tex|, |cdocspt3.tex|, |cdocspt4.tex|,
|cdocsdrf.tex|, |cdocsfn1.tex|, |cdocsfn2.tex|
as well as |childdoc.pdf|.

%%%%%%%%%%%%%%%%%%%%%%%%%%%%%%%%%%%%%%%%%%%%%%%%%%%%%%%%%%%%%%%%%%%%%%%%%%%%%%%%
\subsection{Files and Installation}

The package consists of the files:
%
\begin{center}
\begin{tabular}{ll}
    |README.txt|   & readme file \\
    |childdoc.ins| & installation file \\
    |childdoc.dtx| & source file \\
    |childdoc.def| & definition file \\
    |cdocsamp.tex| & sample main file \\
    |cdocsch1.tex| & sample include file \\
    |cdocsch2.tex| & sample include file \\
    |cdocspt3.tex| & sample part file \\
    |cdocspt4.tex| & sample part file \\
    |cdocsdrf.tex| & sample redirection file \\
    |cdocsfn1.tex| & sample redirection file \\
    |cdocsfn2.tex| & sample redirection file \\
    |childdoc.pdf| & manual
\end{tabular}
\end{center}
%
The distribution consists of the files
|README.txt|, |childdoc.ins| and |childdoc.dtx|.
%
\begin{itemize}
\item
Run (pdf)\LaTeX{} on |childdoc.dtx|
to compile the manual |childdoc.pdf| (this file).
\item
Run \LaTeX{} on |childdoc.ins| to create the definitions file |childdoc.def|
and the sample |cdocsamp.tex| with include files
|cdocsch1.tex|, |cdocsch2.tex|, |cdocspt3.tex|, |cdocspt4.tex|,
|cdocsdrf.tex|, |cdocsfn1.tex|, |cdocsfn2.tex|.
Then copy the file |childdoc.def| to an appropriate directory of your \LaTeX{}
distribution, e.g.\ \textit{texmf-root}|/tex/latex/childdoc|.
\end{itemize}

%%%%%%%%%%%%%%%%%%%%%%%%%%%%%%%%%%%%%%%%%%%%%%%%%%%%%%%%%%%%%%%%%%%%%%%%%%%%%%%%
\subsection{Related CTAN Packages}

There are several other packages which offer a similar functionality:
%
\begin{itemize}
\item
The packages
\href{http://ctan.org/pkg/docmute}{\textsf{docmute}},
\href{http://ctan.org/pkg/includex}{\textsf{includex}} and
\href{http://ctan.org/pkg/standalone}{\textsf{standalone}}
provide commands to include only the document body of
a child file thus allowing both files to be compiled individually.
\item
The packages \href{http://ctan.org/pkg/subdocs}{\textsf{subdocs}}
and \href{http://ctan.org/pkg/subfiles}{\textsf{subfiles}}
provide structures in which the main and child documents can be
encapsulated and allowing them to be compiled individually.
The inclusion mechanism is different from the conventional |\include|.
\item
The package \href{http://ctan.org/pkg/combine}{\textsf{combine}}
is an elaborate solution to combine several documents into one.
\end{itemize}
%
See also the CTAN topic \href{http://ctan.org/topic/subdocs}{\textsf{subdocs}}
for further related packages.
The present package differs from the above solutions in that
a document structure constructed with the conventional |\include| mechanism
just needs two extra commands at the top of every file
such that all constituent files can be compiled individually.

%%%%%%%%%%%%%%%%%%%%%%%%%%%%%%%%%%%%%%%%%%%%%%%%%%%%%%%%%%%%%%%%%%%%%%%%%%%%%%%%
%\subsection{Feature Suggestions}
%
%The following is a list of features which may be useful for future
%versions of this package:
%%
%\begin{itemize}
%\item
%\ldots
%\end{itemize}

%%%%%%%%%%%%%%%%%%%%%%%%%%%%%%%%%%%%%%%%%%%%%%%%%%%%%%%%%%%%%%%%%%%%%%%%%%%%%%%%
\subsection{Revision History}

%%%%%%%%%%%%%%%%%%%%%%%%%%%%%%%%%%%%%%%%
\paragraph{v2.0:} 2018/12/30

\begin{itemize}
\item
immediate forward processing
\item
added |\childdocby| mechanism
\item
manual restructured
\end{itemize}

%%%%%%%%%%%%%%%%%%%%%%%%%%%%%%%%%%%%%%%%
\paragraph{v1.6:} 2018/01/17

\begin{itemize}
\item
application for development of include files
\item
corrections to manual
\end{itemize}

%%%%%%%%%%%%%%%%%%%%%%%%%%%%%%%%%%%%%%%%
\paragraph{v1.5:} 2017/05/21

\begin{itemize}
\item
more complete structuring introduced
\item
|\childdocof| introduced
\item
|\childdoc| renamed to |\childdocmain|
\item
|\childredirect| renamed to |\childdocforward| and |\childdocforwardprefix|
and functionality expanded
\end{itemize}

%%%%%%%%%%%%%%%%%%%%%%%%%%%%%%%%%%%%%%%%
\paragraph{v1.0:} 2017/04/27

\begin{itemize}
\item
manual and install package
\item
first version published on CTAN
\end{itemize}

%%%%%%%%%%%%%%%%%%%%%%%%%%%%%%%%%%%%%%%%
\paragraph{v0.6:} 2017/04/26

\begin{itemize}
\item
redirection mechanism added
\end{itemize}

%%%%%%%%%%%%%%%%%%%%%%%%%%%%%%%%%%%%%%%%
\paragraph{v0.5:} 2017/04/26

\begin{itemize}
\item
functionality in definition file
\end{itemize}


%%%%%%%%%%%%%%%%%%%%%%%%%%%%%%%%%%%%%%%%%%%%%%%%%%%%%%%%%%%%%%%%%%%%%%%%%%%%%%%%
%%%%%%%%%%%%%%%%%%%%%%%%%%%%%%%%%%%%%%%%%%%%%%%%%%%%%%%%%%%%%%%%%%%%%%%%%%%%%%%%
%%%%%%%%%%%%%%%%%%%%%%%%%%%%%%%%%%%%%%%%%%%%%%%%%%%%%%%%%%%%%%%%%%%%%%%%%%%%%%%%
\appendix

\settowidth\MacroIndent{\rmfamily\scriptsize 000\ }

 \DocInput{childdoc.dtx}

\end{document}
%</driver>
% \fi
%
% %%%%%%%%%%%%%%%%%%%%%%%%%%%%%%%%%%%%%%%%%%%%%%%%%%%%%%%%%%%%%%%%%%%%%%%%%%%%%%
% %%%%%%%%%%%%%%%%%%%%%%%%%%%%%%%%%%%%%%%%%%%%%%%%%%%%%%%%%%%%%%%%%%%%%%%%%%%%%%
% \section{Sample}
%\iffalse
%<*samplemain>
%\fi
%
% The following presents a sample document
% with two chapters, two parts, a title page,
% a compile flag as well as three forwarding files to set the flag.
% It consists of eight |.tex| files:
% \begin{center}
% \begin{tabular}{ll}
% |cdocsamp.tex|&main file\\
% |cdocsch1.tex|&include file for chapter 1\\
% |cdocsch2.tex|&include file for chapter 2\\
% |cdocspt3.tex|&include file for part 3\\
% |cdocspt4.tex|&include file for part 4\\
% |cdocsdrf.tex|&forwarding file for main file in draft mode\\
% |cdocsfi1.tex|&forwarding file for final version of chapter 1\\
% |cdocsfi2.tex|&forwarding file for final version of chapter 2\\
% \end{tabular}
% \end{center}
% Each of the eight files can be compiled directly by the \LaTeX{} compiler.
%
% %%%%%%%%%%%%%%%%%%%%%%%%%%%%%%%%%%%%%%
% \paragraph{Main File.}
%
% The main file is called |cdocsamp.tex|.
%
% Load the \textsf{childdoc} definitions and
% declare the filename for the main document:
%    \begin{macrocode}
\input{childdoc.def}
\childdocmain{}
%    \end{macrocode}

% Optional override for |\version| flag:
%    \begin{macrocode}
%%\ifchilddoc\else\providecommand{\version}{draft}\fi
%    \end{macrocode}

% Define the default values for the |\version| flag
% (|final| for the main file and |draft| for childs):
%    \begin{macrocode}
\ifchilddoc
\providecommand{\version}{draft}
\else
\providecommand{\version}{final}
\fi
%    \end{macrocode}

% Load the standard document class:
%    \begin{macrocode}
\documentclass[12pt]{article}
%    \end{macrocode}

% Start the document body:
%    \begin{macrocode}
\begin{document}
%    \end{macrocode}

% Declare a title page.
% Print title, part of document being processed and version flag:
%    \begin{macrocode}
\addtocounter{page}{-1}
\begin{center}
{\LARGE\bfseries{}childdoc example\par}
\vspace{1cm}
\ifchilddoc
\ifchilddocmanual part\else chapter\fi:
`\childdocname' of `\childdocjob'\par
\else
main document: `\childdocjob'\par
\fi
version: \version\par
\end{center}
\newpage
%    \end{macrocode}

% Manually include selected file,
% otherwise process as usual:
%    \begin{macrocode}
\ifchilddocmanual
\section*{part `\childdocname'}
\input{\childdocname}
\else
%    \end{macrocode}

% Include the two chapters:
%    \begin{macrocode}
\include{cdocsch1}
\include{cdocsch2}
%    \end{macrocode}

% Include the two parts unless only chapters should be displayed:
%    \begin{macrocode}
\ifchilddoc\else
\section{part three}
\input{cdocspt3}
\section{part four}
\input{cdocspt4}
\fi
%    \end{macrocode}

% Process as usual until here:
%    \begin{macrocode}
\fi
%    \end{macrocode}

% End of document body:
%    \begin{macrocode}
\end{document}
%    \end{macrocode}
%\iffalse
%</samplemain>
%\fi
%
% %%%%%%%%%%%%%%%%%%%%%%%%%%%%%%%%%%%%%%
% \paragraph{Chapter Include Files.}
%
% The include files are called |cdocsch1.tex| and |cdocsch2.tex|.
%
%\iffalse
%<*samplechap1|samplechap2>
%\fi

% Optional override for |\version| flag:
%    \begin{macrocode}
%%\providecommand{\version}{final}
%    \end{macrocode}

% Include the main document:
%    \begin{macrocode}
\input{childdoc.def}
\childdocof{cdocsamp}
%    \end{macrocode}

%\iffalse
%</samplechap1|samplechap2>
%\fi
%
%\iffalse
%<*samplechap1>
%\fi
% Some text for chapter 1:
%    \begin{macrocode}
\section{one}
some text in chapter one
%    \end{macrocode}

%\iffalse
%</samplechap1>
%\fi
% Some text for chapter 2:
%\iffalse
%<*samplechap2>
%\fi
%    \begin{macrocode}
\section{two}
more text in chapter two
%    \end{macrocode}

%\iffalse
%</samplechap2>
%\fi
%
% %%%%%%%%%%%%%%%%%%%%%%%%%%%%%%%%%%%%%%
% \paragraph{Part Include Files.}
%
% The include files are called |cdocspt3.tex| and |cdocspt4.tex|.
%
%\iffalse
%<*samplepart3|samplepart4>
%\fi

% Optional override for |\version| flag:
%    \begin{macrocode}
%%\providecommand{\version}{final}
%    \end{macrocode}

% Include the main document:
%    \begin{macrocode}
\input{childdoc.def}
\childdocby{cdocsamp}
%    \end{macrocode}

%\iffalse
%</samplepart3|samplepart4>
%\fi
%
%\iffalse
%<*samplepart3>
%\fi
% Some text for part 3:
%    \begin{macrocode}
some text in part three
%    \end{macrocode}

%\iffalse
%</samplepart3>
%\fi
% Some text for part 4:
%\iffalse
%<*samplepart4>
%\fi
%    \begin{macrocode}
more text in part four
%    \end{macrocode}

%\iffalse
%</samplepart4>
%\fi
%
% %%%%%%%%%%%%%%%%%%%%%%%%%%%%%%%%%%%%%%
% \paragraph{Forwarding for a Complete Draft.}
%
% The following forwarding file |cdocsdrf.tex|
% compiles the main document in draft mode:
%\iffalse
%<*sampledraft>
%\fi
%    \begin{macrocode}
\def\version{draft}
\input{childdoc.def}
\childdocforward{cdocsamp}
%    \end{macrocode}

%\iffalse
%</sampledraft>
%\fi
%
% %%%%%%%%%%%%%%%%%%%%%%%%%%%%%%%%%%%%%%
% \paragraph{Forwarding for Final Version of the Chapters.}
%
% The following forwarding files |cdocsfn1.tex| and |cdocsfn2.tex|
% (with identical content)
% compile the final versions of the child documents
% |cdocsch1.tex| and |cdocsch2.tex|, respectively:
%\iffalse
%<*samplefinal>
%\fi
%    \begin{macrocode}
\def\version{final}
\input{childdoc.def}
\childdocforwardprefix[cdocsamp]{cdocsfn}{cdocsch}
%    \end{macrocode}

%\iffalse
%</samplefinal>
%\fi
%
% %%%%%%%%%%%%%%%%%%%%%%%%%%%%%%%%%%%%%%
% \paragraph{Command Line Processing.}
%
% The following three command lines generate the output files
% |cdocscld|, |cdocscl1| and |cdocscl2|
% which should be identical to
% |cdocsdrf|, |cdocsch1| and |cdocsfn2|, respectively:
% \begin{center}
% \begin{tabular}{l}
% |latex -jobname cdocscld \|\\
% |  "\def\version{draft}\input{childdoc.def}\childdocforward{cdocsamp}"|\\
% |latex -jobname cdocscl1 \|\\
% |  "\input{childdoc.def}\childdocforward[cdocsamp]{cdocsch1}"|\\
% |latex -jobname cdocscl2 \|\\
% |  "\def\version{final}\input{childdoc.def}\childdocforward{cdocsch2}"|
% \end{tabular}
% \end{center}
% Note that the trailing backslash on each first line
% merely continues the input to the second line
% (for convenient cut ant paste).
% Furthermore, the command |latex| can be replaced by any
% of its alternative versions such as |pdflatex|.
%
% %%%%%%%%%%%%%%%%%%%%%%%%%%%%%%%%%%%%%%%%%%%%%%%%%%%%%%%%%%%%%%%%%%%%%%%%%%%%%%
% %%%%%%%%%%%%%%%%%%%%%%%%%%%%%%%%%%%%%%%%%%%%%%%%%%%%%%%%%%%%%%%%%%%%%%%%%%%%%%
% \section{Implementation}
%\iffalse
%<*package>
%\fi
%
% This section describes the definitions file |childdoc.def|.

% The definitions cannot be loaded using |\usepackage| or |\RequirePackage|
% which has a mechanism to prevent loading a style file more than once.
% When loading the definitions by means of |\input|
% multiple instances have to be prevented manually:
%\iffalse
%This code needs to be before the `\ProvidesFile' directive
%which is defined at the beginning of this file.
%Therefore it is also placed there and commented out here.
%</package>
%<*discard>
%\fi
%    \begin{macrocode}
\ifdefined\childdocmain\endinput\fi
%    \end{macrocode}
%\iffalse
%</discard>
%<*package>
%\fi
%
% \macro{\ifchilddoc}
% \macro{\ifchilddocmanual}
% The conditional |\ifchilddoc| tells whether a
% child (true) or main (false) document is being compiled.
% The conditional |\ifchilddocmanual| tells whether
% the |\includeonly| mechanism is used (false) or
% the selection of child files must be performed manually (true).
% The definitions initialise to false:
%    \begin{macrocode}
\newif\ifchilddoc
\newif\ifchilddocmanual
%    \end{macrocode}

% \macro{\childdocname}
% \macro{\childdocjob}
% The macro |\childdocname| stores the name of the main document
% to be compiled. The macro |\childdocjob| stores the name of
% the document on which the \LaTeX{} compiler was originally invoked.
% The content of |\jobname| cannot be compared
% to filenames specified in the source due to different catcodes.
% The following code rescans |\jobname|, stores the result
% in |\childdocname| and saves a copy in |\childdocjob|:
%    \begin{macrocode}
\edef\childdocname{\scantokens\expandafter{\jobname\noexpand}}
\let\childdocjob\childdocname
%    \end{macrocode}

% \macro{\childdocdisable}
% The macro |\childdocdisable| prevents the main file
% from being processed more than once.
% At this stage, the main document command |\childdocmain|
% is assumed to be called once again where it should do nothing.
% Any subsequent call to it should prevent
% a secondary processing of the main document
% It overwrites the forwarding commands
% |\childdocof| and |\childdocforward|
% with empty macros to prevent further inclusions of the main document:
%    \begin{macrocode}
\newcommand{\childdocdisable}
{
  \renewcommand{\childdocmain}[1]{\renewcommand{\childdocmain}[1]{\endinput}}
  \renewcommand{\childdocof}[1]{}
  \renewcommand{\childdocby}[2][]{}
  \renewcommand{\childdocforward}[2][]{}
  \renewcommand{\childdocdisable}{}
}
%    \end{macrocode}

% \macro{\childdocmain}
% The macro |\childdocmain| is to be called at the top of the main file
% with nothing or the main filename (without extension) as argument.
% First, it breaks loops.
% If the argument is not empty and does not match |\childdocname|
% (which is set by the first inclusion of |childdoc.def|),
% |\ifchilddoc| is set to true, |\includeonly| is applied to the child file
% and |\jobname| is set to the main file
% (for proper handling of |.aux| files):
%    \begin{macrocode}
\newcommand{\childdocmain}[1]
{
  \childdocdisable\childdocmain{}
  \if?#1?\else
    \begingroup
      \def\childdoctmp{#1}
      \ifx\childdoctmp\childdocname
        \def\childdoctmp{}
      \else
        \def\childdoctmp
        {
          \childdoctrue
          \includeonly{\childdocname}
          \def\childdocjob{#1}
          \def\jobname{#1}
        }
      \fi
      \expandafter
    \endgroup
    \childdoctmp
  \fi
}
%    \end{macrocode}

% \macro{\childdocof}
% The command |\childdocof| redirects
% compilation to the main file |#1|.
%    \begin{macrocode}
\newcommand{\childdocof}[1]
{
  \childdocdisable
  \childdoctrue
  \includeonly{\childdocname}
  \def\jobname{#1}
  \def\childdocjob{#1}
  \input{#1}
}
%    \end{macrocode}

% \macro{\childdocby}
% The command |\childdocby| ....
%    \begin{macrocode}
\newcommand{\childdocby}[2][]
{
  \childdocdisable
  \childdoctrue
  \childdocmanualtrue
  \if?#1?\else
    \def\jobname{#2}
  \fi
  \def\childdocjob{#2}
  \input{#2}
  \endinput
}
%    \end{macrocode}

% \macro{\childdocforward}
% The command |\childdocforward| redirects
% compilation to the main file or
% (if the optional argument is given) a child file.
% Parameters are set as if the main file
% or a child file starting with |\childdocof| was compiled.
% Then compilation is handed over to the main file:
%    \begin{macrocode}
\newcommand{\childdocforward}[2][]
{
  \begingroup
    \if?#1?
      \def\childdoctmp
      {
        \def\childdocname{#2}
        \def\childdocjob{#2}
        \def\jobname{#2}
        \input{#2}
        \endinput
      }
    \else
      \def\childdoctmp
      {
        \childdocdisable
        \def\childdocname{#2}
        \childdoctrue
        \includeonly{#2}
        \def\childdocjob{#1}
        \def\jobname{#1}
        \input{#1}
        \endinput
      }
    \fi
    \expandafter
  \endgroup
  \childdoctmp
}
%    \end{macrocode}

% \macro{\childdocforwardprefix}
% The command |\childdocforwardprefix| redirects
% compilation to the main or a child file by means of a pattern.
% The prefix |#1| in the current filename is replaced by |#2|
% and the suffix of the current filename is kept
% (it is assumed that the filename does not contain the substring `|~~~|'
% which is used as a delimiter).
% Compilation is handed over to the new file by |\childdocforward|:
%    \begin{macrocode}
\newcommand{\childdocforwardprefix}[3][]
{
  \begingroup
    \def\childdocextract #2##1~~~{\def\childdoctmp{\childdocforward[#1]{#3##1}}}
    \expandafter\childdocextract\childdocname~~~
    \expandafter
  \endgroup
  \childdoctmp
}
%    \end{macrocode}

% \macro{\childdoc}
% The deprecated macro |\childdoc| is a legacy version of |\childdocmain|:
%    \begin{macrocode}
\newcommand{\childdoc}{\childdocmain}
%    \end{macrocode}

% \macro{\childdocredirect}
% The deprecated macro |\childdocredirect| is a legacy version
% of |\childdocforward| and |\childdocforwardprefix|:
%    \begin{macrocode}
\newcommand{\childdocredirect}[2][]
{
  \begingroup
    \if?#1?
      \def\childdoctmp{\childdocforward{#2}}
    \else
      \def\childdoctmp{\childdocforwardprefix{#1}{#2}}
    \fi
    \expandafter
  \endgroup
  \childdoctmp
}
%    \end{macrocode}

%\iffalse
%</package>
%\fi
%
\endinput
\childdocforward{cdocsch2}"|
% \end{tabular}
% \end{center}
% Note that the trailing backslash on each first line
% merely continues the input to the second line
% (for convenient cut ant paste).
% Furthermore, the command |latex| can be replaced by any
% of its alternative versions such as |pdflatex|.
%
% %%%%%%%%%%%%%%%%%%%%%%%%%%%%%%%%%%%%%%%%%%%%%%%%%%%%%%%%%%%%%%%%%%%%%%%%%%%%%%
% %%%%%%%%%%%%%%%%%%%%%%%%%%%%%%%%%%%%%%%%%%%%%%%%%%%%%%%%%%%%%%%%%%%%%%%%%%%%%%
% \section{Implementation}
%\iffalse
%<*package>
%\fi
%
% This section describes the definitions file |childdoc.def|.

% The definitions cannot be loaded using |\usepackage| or |\RequirePackage|
% which has a mechanism to prevent loading a style file more than once.
% When loading the definitions by means of |\input|
% multiple instances have to be prevented manually:
%\iffalse
%This code needs to be before the `\ProvidesFile' directive
%which is defined at the beginning of this file.
%Therefore it is also placed there and commented out here.
%</package>
%<*discard>
%\fi
%    \begin{macrocode}
\ifdefined\childdocmain\endinput\fi
%    \end{macrocode}
%\iffalse
%</discard>
%<*package>
%\fi
%
% \macro{\ifchilddoc}
% \macro{\ifchilddocmanual}
% The conditional |\ifchilddoc| tells whether a
% child (true) or main (false) document is being compiled.
% The conditional |\ifchilddocmanual| tells whether
% the |\includeonly| mechanism is used (false) or
% the selection of child files must be performed manually (true).
% The definitions initialise to false:
%    \begin{macrocode}
\newif\ifchilddoc
\newif\ifchilddocmanual
%    \end{macrocode}

% \macro{\childdocname}
% \macro{\childdocjob}
% The macro |\childdocname| stores the name of the main document
% to be compiled. The macro |\childdocjob| stores the name of
% the document on which the \LaTeX{} compiler was originally invoked.
% The content of |\jobname| cannot be compared
% to filenames specified in the source due to different catcodes.
% The following code rescans |\jobname|, stores the result
% in |\childdocname| and saves a copy in |\childdocjob|:
%    \begin{macrocode}
\edef\childdocname{\scantokens\expandafter{\jobname\noexpand}}
\let\childdocjob\childdocname
%    \end{macrocode}

% \macro{\childdocdisable}
% The macro |\childdocdisable| prevents the main file
% from being processed more than once.
% At this stage, the main document command |\childdocmain|
% is assumed to be called once again where it should do nothing.
% Any subsequent call to it should prevent
% a secondary processing of the main document
% It overwrites the forwarding commands
% |\childdocof| and |\childdocforward|
% with empty macros to prevent further inclusions of the main document:
%    \begin{macrocode}
\newcommand{\childdocdisable}
{
  \renewcommand{\childdocmain}[1]{\renewcommand{\childdocmain}[1]{\endinput}}
  \renewcommand{\childdocof}[1]{}
  \renewcommand{\childdocby}[2][]{}
  \renewcommand{\childdocforward}[2][]{}
  \renewcommand{\childdocdisable}{}
}
%    \end{macrocode}

% \macro{\childdocmain}
% The macro |\childdocmain| is to be called at the top of the main file
% with nothing or the main filename (without extension) as argument.
% First, it breaks loops.
% If the argument is not empty and does not match |\childdocname|
% (which is set by the first inclusion of |childdoc.def|),
% |\ifchilddoc| is set to true, |\includeonly| is applied to the child file
% and |\jobname| is set to the main file
% (for proper handling of |.aux| files):
%    \begin{macrocode}
\newcommand{\childdocmain}[1]
{
  \childdocdisable\childdocmain{}
  \if?#1?\else
    \begingroup
      \def\childdoctmp{#1}
      \ifx\childdoctmp\childdocname
        \def\childdoctmp{}
      \else
        \def\childdoctmp
        {
          \childdoctrue
          \includeonly{\childdocname}
          \def\childdocjob{#1}
          \def\jobname{#1}
        }
      \fi
      \expandafter
    \endgroup
    \childdoctmp
  \fi
}
%    \end{macrocode}

% \macro{\childdocof}
% The command |\childdocof| redirects
% compilation to the main file |#1|.
%    \begin{macrocode}
\newcommand{\childdocof}[1]
{
  \childdocdisable
  \childdoctrue
  \includeonly{\childdocname}
  \def\jobname{#1}
  \def\childdocjob{#1}
  \input{#1}
}
%    \end{macrocode}

% \macro{\childdocby}
% The command |\childdocby| ....
%    \begin{macrocode}
\newcommand{\childdocby}[2][]
{
  \childdocdisable
  \childdoctrue
  \childdocmanualtrue
  \if?#1?\else
    \def\jobname{#2}
  \fi
  \def\childdocjob{#2}
  \input{#2}
  \endinput
}
%    \end{macrocode}

% \macro{\childdocforward}
% The command |\childdocforward| redirects
% compilation to the main file or
% (if the optional argument is given) a child file.
% Parameters are set as if the main file
% or a child file starting with |\childdocof| was compiled.
% Then compilation is handed over to the main file:
%    \begin{macrocode}
\newcommand{\childdocforward}[2][]
{
  \begingroup
    \if?#1?
      \def\childdoctmp
      {
        \def\childdocname{#2}
        \def\childdocjob{#2}
        \def\jobname{#2}
        \input{#2}
        \endinput
      }
    \else
      \def\childdoctmp
      {
        \childdocdisable
        \def\childdocname{#2}
        \childdoctrue
        \includeonly{#2}
        \def\childdocjob{#1}
        \def\jobname{#1}
        \input{#1}
        \endinput
      }
    \fi
    \expandafter
  \endgroup
  \childdoctmp
}
%    \end{macrocode}

% \macro{\childdocforwardprefix}
% The command |\childdocforwardprefix| redirects
% compilation to the main or a child file by means of a pattern.
% The prefix |#1| in the current filename is replaced by |#2|
% and the suffix of the current filename is kept
% (it is assumed that the filename does not contain the substring `|~~~|'
% which is used as a delimiter).
% Compilation is handed over to the new file by |\childdocforward|:
%    \begin{macrocode}
\newcommand{\childdocforwardprefix}[3][]
{
  \begingroup
    \def\childdocextract #2##1~~~{\def\childdoctmp{\childdocforward[#1]{#3##1}}}
    \expandafter\childdocextract\childdocname~~~
    \expandafter
  \endgroup
  \childdoctmp
}
%    \end{macrocode}

% \macro{\childdoc}
% The deprecated macro |\childdoc| is a legacy version of |\childdocmain|:
%    \begin{macrocode}
\newcommand{\childdoc}{\childdocmain}
%    \end{macrocode}

% \macro{\childdocredirect}
% The deprecated macro |\childdocredirect| is a legacy version
% of |\childdocforward| and |\childdocforwardprefix|:
%    \begin{macrocode}
\newcommand{\childdocredirect}[2][]
{
  \begingroup
    \if?#1?
      \def\childdoctmp{\childdocforward{#2}}
    \else
      \def\childdoctmp{\childdocforwardprefix{#1}{#2}}
    \fi
    \expandafter
  \endgroup
  \childdoctmp
}
%    \end{macrocode}

%\iffalse
%</package>
%\fi
%
\endinput
|
and perform the replacements as outlined below.
Instead of |\childdocmain{|\textit{main}|}| add the following code
to the top of the main file:
%
\begin{center}
\begin{tabular}{l}
|\||ifdefined\childdocname\endinput\||fi\newif\ifchilddoc|\\
|\edef\childdocname{\scantokens\expandafter{\jobname\noexpand}}|\\
|\def\childdocmain{|\textit{main}|}\||ifx\childdocmain\childdocname\||else|\\
|\childdoctrue\includeonly{\childdocname}\let\jobname\childdocmain\||fi|\\
\end{tabular}
\end{center}
%
Instead of |\childdocof{|\textit{main}|}| just include the main file
at the top of each child file:
%
\begin{center}
|\input{|\textit{main}|}|
\end{center}
%
A simple redirection |\childdocforward{|\textit{dest}|}| is achieved by:
%
\begin{center}
|\def\jobname{|\textit{dest}|}\input{\jobname}|
\end{center}
%
The redirection with prefix
|\childdocforwardprefix[|\textit{prefix}|]{|\textit{dest}|}|
is accomplished by:
%
\begin{center}
\begin{tabular}{l}
|{\edef\jobname{\scantokens\expandafter{\jobname\noexpand}}|\\
|\def\redirectjob |\textit{prefix}|#1~~~{\gdef\jobname{|\textit{dest}|#1}}|\\
|\expandafter\redirectjob\jobname~~~}\input{\jobname}|
\end{tabular}
\end{center}

In an alternative approach,
child documents can be compiled by a specific command line
without additional code or specific definitions:
%
\begin{center}
|... -jobname "|\textit{target}|" "|[\textit{flags}]%
|\includeonly{|\textit{dest}|}\input{|\textit{main}|}"|
\end{center}
%

%%%%%%%%%%%%%%%%%%%%%%%%%%%%%%%%%%%%%%%%%%%%%%%%%%%%%%%%%%%%%%%%%%%%%%%%%%%%%%%%
%%%%%%%%%%%%%%%%%%%%%%%%%%%%%%%%%%%%%%%%%%%%%%%%%%%%%%%%%%%%%%%%%%%%%%%%%%%%%%%%
\section{Information}

%%%%%%%%%%%%%%%%%%%%%%%%%%%%%%%%%%%%%%%%%%%%%%%%%%%%%%%%%%%%%%%%%%%%%%%%%%%%%%%%
\subsection{Copyright}

Copyright \copyright{} 2017--2018 Niklas Beisert

This work may be distributed and/or modified under the
conditions of the \LaTeX{} Project Public License, either version 1.3
of this license or (at your option) any later version.
The latest version of this license is in
  \url{http://www.latex-project.org/lppl.txt}
and version 1.3 or later is part of all distributions of \LaTeX{}
version 2005/12/01 or later.

This work has the LPPL maintenance status `maintained'.

The Current Maintainer of this work is Niklas Beisert.

This work consists of the files |README.txt|, |childdoc.ins| and |childdoc.dtx|
as well as the derived files |childdoc.def|, |cdocsamp.tex|
with |cdocsch1.tex|, |cdocsch2.tex|, |cdocspt3.tex|, |cdocspt4.tex|,
|cdocsdrf.tex|, |cdocsfn1.tex|, |cdocsfn2.tex|
as well as |childdoc.pdf|.

%%%%%%%%%%%%%%%%%%%%%%%%%%%%%%%%%%%%%%%%%%%%%%%%%%%%%%%%%%%%%%%%%%%%%%%%%%%%%%%%
\subsection{Files and Installation}

The package consists of the files:
%
\begin{center}
\begin{tabular}{ll}
    |README.txt|   & readme file \\
    |childdoc.ins| & installation file \\
    |childdoc.dtx| & source file \\
    |childdoc.def| & definition file \\
    |cdocsamp.tex| & sample main file \\
    |cdocsch1.tex| & sample include file \\
    |cdocsch2.tex| & sample include file \\
    |cdocspt3.tex| & sample part file \\
    |cdocspt4.tex| & sample part file \\
    |cdocsdrf.tex| & sample redirection file \\
    |cdocsfn1.tex| & sample redirection file \\
    |cdocsfn2.tex| & sample redirection file \\
    |childdoc.pdf| & manual
\end{tabular}
\end{center}
%
The distribution consists of the files
|README.txt|, |childdoc.ins| and |childdoc.dtx|.
%
\begin{itemize}
\item
Run (pdf)\LaTeX{} on |childdoc.dtx|
to compile the manual |childdoc.pdf| (this file).
\item
Run \LaTeX{} on |childdoc.ins| to create the definitions file |childdoc.def|
and the sample |cdocsamp.tex| with include files
|cdocsch1.tex|, |cdocsch2.tex|, |cdocspt3.tex|, |cdocspt4.tex|,
|cdocsdrf.tex|, |cdocsfn1.tex|, |cdocsfn2.tex|.
Then copy the file |childdoc.def| to an appropriate directory of your \LaTeX{}
distribution, e.g.\ \textit{texmf-root}|/tex/latex/childdoc|.
\end{itemize}

%%%%%%%%%%%%%%%%%%%%%%%%%%%%%%%%%%%%%%%%%%%%%%%%%%%%%%%%%%%%%%%%%%%%%%%%%%%%%%%%
\subsection{Related CTAN Packages}

There are several other packages which offer a similar functionality:
%
\begin{itemize}
\item
The packages
\href{http://ctan.org/pkg/docmute}{\textsf{docmute}},
\href{http://ctan.org/pkg/includex}{\textsf{includex}} and
\href{http://ctan.org/pkg/standalone}{\textsf{standalone}}
provide commands to include only the document body of
a child file thus allowing both files to be compiled individually.
\item
The packages \href{http://ctan.org/pkg/subdocs}{\textsf{subdocs}}
and \href{http://ctan.org/pkg/subfiles}{\textsf{subfiles}}
provide structures in which the main and child documents can be
encapsulated and allowing them to be compiled individually.
The inclusion mechanism is different from the conventional |\include|.
\item
The package \href{http://ctan.org/pkg/combine}{\textsf{combine}}
is an elaborate solution to combine several documents into one.
\end{itemize}
%
See also the CTAN topic \href{http://ctan.org/topic/subdocs}{\textsf{subdocs}}
for further related packages.
The present package differs from the above solutions in that
a document structure constructed with the conventional |\include| mechanism
just needs two extra commands at the top of every file
such that all constituent files can be compiled individually.

%%%%%%%%%%%%%%%%%%%%%%%%%%%%%%%%%%%%%%%%%%%%%%%%%%%%%%%%%%%%%%%%%%%%%%%%%%%%%%%%
%\subsection{Feature Suggestions}
%
%The following is a list of features which may be useful for future
%versions of this package:
%%
%\begin{itemize}
%\item
%\ldots
%\end{itemize}

%%%%%%%%%%%%%%%%%%%%%%%%%%%%%%%%%%%%%%%%%%%%%%%%%%%%%%%%%%%%%%%%%%%%%%%%%%%%%%%%
\subsection{Revision History}

%%%%%%%%%%%%%%%%%%%%%%%%%%%%%%%%%%%%%%%%
\paragraph{v2.0:} 2018/12/30

\begin{itemize}
\item
immediate forward processing
\item
added |\childdocby| mechanism
\item
manual restructured
\end{itemize}

%%%%%%%%%%%%%%%%%%%%%%%%%%%%%%%%%%%%%%%%
\paragraph{v1.6:} 2018/01/17

\begin{itemize}
\item
application for development of include files
\item
corrections to manual
\end{itemize}

%%%%%%%%%%%%%%%%%%%%%%%%%%%%%%%%%%%%%%%%
\paragraph{v1.5:} 2017/05/21

\begin{itemize}
\item
more complete structuring introduced
\item
|\childdocof| introduced
\item
|\childdoc| renamed to |\childdocmain|
\item
|\childredirect| renamed to |\childdocforward| and |\childdocforwardprefix|
and functionality expanded
\end{itemize}

%%%%%%%%%%%%%%%%%%%%%%%%%%%%%%%%%%%%%%%%
\paragraph{v1.0:} 2017/04/27

\begin{itemize}
\item
manual and install package
\item
first version published on CTAN
\end{itemize}

%%%%%%%%%%%%%%%%%%%%%%%%%%%%%%%%%%%%%%%%
\paragraph{v0.6:} 2017/04/26

\begin{itemize}
\item
redirection mechanism added
\end{itemize}

%%%%%%%%%%%%%%%%%%%%%%%%%%%%%%%%%%%%%%%%
\paragraph{v0.5:} 2017/04/26

\begin{itemize}
\item
functionality in definition file
\end{itemize}


%%%%%%%%%%%%%%%%%%%%%%%%%%%%%%%%%%%%%%%%%%%%%%%%%%%%%%%%%%%%%%%%%%%%%%%%%%%%%%%%
%%%%%%%%%%%%%%%%%%%%%%%%%%%%%%%%%%%%%%%%%%%%%%%%%%%%%%%%%%%%%%%%%%%%%%%%%%%%%%%%
%%%%%%%%%%%%%%%%%%%%%%%%%%%%%%%%%%%%%%%%%%%%%%%%%%%%%%%%%%%%%%%%%%%%%%%%%%%%%%%%
\appendix

\settowidth\MacroIndent{\rmfamily\scriptsize 000\ }

 \DocInput{childdoc.dtx}

\end{document}
%</driver>
% \fi
%
% %%%%%%%%%%%%%%%%%%%%%%%%%%%%%%%%%%%%%%%%%%%%%%%%%%%%%%%%%%%%%%%%%%%%%%%%%%%%%%
% %%%%%%%%%%%%%%%%%%%%%%%%%%%%%%%%%%%%%%%%%%%%%%%%%%%%%%%%%%%%%%%%%%%%%%%%%%%%%%
% \section{Sample}
%\iffalse
%<*samplemain>
%\fi
%
% The following presents a sample document
% with two chapters, two parts, a title page,
% a compile flag as well as three forwarding files to set the flag.
% It consists of eight |.tex| files:
% \begin{center}
% \begin{tabular}{ll}
% |cdocsamp.tex|&main file\\
% |cdocsch1.tex|&include file for chapter 1\\
% |cdocsch2.tex|&include file for chapter 2\\
% |cdocspt3.tex|&include file for part 3\\
% |cdocspt4.tex|&include file for part 4\\
% |cdocsdrf.tex|&forwarding file for main file in draft mode\\
% |cdocsfi1.tex|&forwarding file for final version of chapter 1\\
% |cdocsfi2.tex|&forwarding file for final version of chapter 2\\
% \end{tabular}
% \end{center}
% Each of the eight files can be compiled directly by the \LaTeX{} compiler.
%
% %%%%%%%%%%%%%%%%%%%%%%%%%%%%%%%%%%%%%%
% \paragraph{Main File.}
%
% The main file is called |cdocsamp.tex|.
%
% Load the \textsf{childdoc} definitions and
% declare the filename for the main document:
%    \begin{macrocode}
% \iffalse
%
% childdoc.dtx Copyright (C) 2017-2018 Niklas Beisert
%
% This work may be distributed and/or modified under the
% conditions of the LaTeX Project Public License, either version 1.3
% of this license or (at your option) any later version.
% The latest version of this license is in
%   http://www.latex-project.org/lppl.txt
% and version 1.3 or later is part of all distributions of LaTeX
% version 2005/12/01 or later.
%
% This work has the LPPL maintenance status `maintained'.
%
% The Current Maintainer of this work is Niklas Beisert.
%
% This work consists of the files childdoc.dtx and childdoc.ins
% and the derived files childdoc.def and cdocsamp.tex with
% cdocsch1.tex, cdocsch2.tex, cdocsdrf.tex, cdocsfn1.tex, cdocsfn2.tex.
%
%<package>\ifdefined\childdocmain\endinput\fi
%<package>\ProvidesFile{childdoc.def}[2018/12/30 v2.0 child document driver]
%<samplemain>\ProvidesFile{cdocsamp.tex}[2018/12/30 v2.0 sample for childdoc]
%<*driver>
%\ProvidesFile{childdoc.drv}[2018/12/30 v2.0 childdoc reference manual file]
\PassOptionsToClass{10pt,a4paper}{article}
\documentclass{ltxdoc}

\usepackage[margin=35mm]{geometry}
\usepackage{hyperref}
\usepackage{hyperxmp}
\usepackage[usenames]{color}

\hypersetup{colorlinks=true}
\hypersetup{pdfstartview=FitH}
\hypersetup{pdfpagemode=UseNone}
\hypersetup{pdfsource={}}
\hypersetup{pdflang={en-UK}}
\hypersetup{pdfcopyright={Copyright 2017-2018 Niklas Beisert.
  This work may be distributed and/or modified under the
  conditions of the LaTeX Project Public License, either version 1.3
  of this license or (at your option) any later version.}}
\hypersetup{pdflicenseurl={http://www.latex-project.org/lppl.txt}}
\hypersetup{pdfcontactaddress={ETH Zurich, ITP, HIT K,
  Wolfgang-Pauli-Strasse 27}}
\hypersetup{pdfcontactpostcode={8093}}
\hypersetup{pdfcontactcity={Zurich}}
\hypersetup{pdfcontactcountry={Switzerland}}
\hypersetup{pdfcontactemail={nbeisert@itp.phys.ethz.ch}}
\hypersetup{pdfcontacturl={http://people.phys.ethz.ch/\xmptilde nbeisert/}}

\newcommand{\secref}[1]{\hyperref[#1]{section \ref*{#1}}}

\parskip1ex
\parindent0pt
\let\olditemize\itemize
\def\itemize{\olditemize\parskip0pt}

\begin{document}

\title{The \textsf{childdoc} Package}
\hypersetup{pdftitle={The childdoc Package}}
\author{Niklas Beisert\\[2ex]
  Institut f\"ur Theoretische Physik\\
  Eidgen\"ossische Technische Hochschule Z\"urich\\
  Wolfgang-Pauli-Strasse 27, 8093 Z\"urich, Switzerland\\[1ex]
  \href{mailto:nbeisert@itp.phys.ethz.ch}
  {\texttt{nbeisert@itp.phys.ethz.ch}}}
\hypersetup{pdfauthor={Niklas Beisert}}
\hypersetup{pdfsubject={Manual for the LaTeX2e Package childdoc}}
\date{30 December 2018, \textsf{v2.0}}
\maketitle

\begin{abstract}\noindent
\textsf{childdoc} is a \LaTeXe{} package
that enables the direct compilation
of document sections included by |\include|
to individual files.
\end{abstract}

\begingroup
\parskip0ex
\tableofcontents
\endgroup

%%%%%%%%%%%%%%%%%%%%%%%%%%%%%%%%%%%%%%%%%%%%%%%%%%%%%%%%%%%%%%%%%%%%%%%%%%%%%%%%
%%%%%%%%%%%%%%%%%%%%%%%%%%%%%%%%%%%%%%%%%%%%%%%%%%%%%%%%%%%%%%%%%%%%%%%%%%%%%%%%
\section{Introduction}

\LaTeX{} provides a mechanism to structure a large document (such as a book)
into a main file and several child files (containing the chapters)
using the |\include| command.
This mechanism is beneficial for documents
which span hundreds of pages in order to
make the source file(s) more manageable.
Moreover, compilation can be restricted to
selected child files by means of the |\includeonly| command.
The latter feature can be used to reduce the compilation time while editing
(this was significantly more useful in the earlier days of \LaTeX{})
or to generate a smaller document which is easier to navigate.
Another application of |\includeonly| is to generate
documents consisting of selected parts of the complete document.

However, there are a few drawbacks of the plain |\include| mechanism:
\begin{itemize}
\item
The child files cannot be compiled on their own,
they can only be compiled via the main file.
A naive editing environment
(such as a text editor with an option
to have the current file processed by \LaTeX)
may require one to switch to the main file before compiling;
attempting to compile the child file produces errors.
\item
The main file must be modified (each time)
to adjust the |\includeonly| command
to the present needs. This easily leaves the main file in a messy state.
\item
The generated document will always carry the filename
of the main document. This is inconvenient if
several child files are to be compiled and
to be kept for distribution.
\end{itemize}

The present package provides a simple interface
to make child files individually compilable by \LaTeX{}.
Compiling a child file then has the same effect as compiling
the main file with an |\includeonly| command
to select the appropriate child.
Moreover the generated document will carry the name of the child
rather than the main file.
This resolves all three above issues.

This feature is meant to make the editing of books,
thesis documents and lecture notes somewhat more convenient.
However, the package can also be used efficiently for
composing a series of documents (such as exercise sheets)
which are typically distributed individually.
It then assists the author in generating the individual documents
(potentially in different versions)
as well as a document containing the collected series.
Another application is in developing style files
or other kinds of included material
where compilation of the style file could redirect
to a sample or test file.

%%%%%%%%%%%%%%%%%%%%%%%%%%%%%%%%%%%%%%%%%%%%%%%%%%%%%%%%%%%%%%%%%%%%%%%%%%%%%%%%
%%%%%%%%%%%%%%%%%%%%%%%%%%%%%%%%%%%%%%%%%%%%%%%%%%%%%%%%%%%%%%%%%%%%%%%%%%%%%%%%
\section{Usage}

First of all, the package \textsf{childdoc} is \emph{not} a standard
\LaTeXe{} |.sty| style file! Therefore it needs to be invoked in
a non-standard way.

%%%%%%%%%%%%%%%%%%%%%%%%%%%%%%%%%%%%%%%%%%%%%%%%%%%%%%%%%%%%%%%%%%%%%%%%%%%%%%%%
\subsection{Included Files}
\label{sec:include}

%%%%%%%%%%%%%%%%%%%%%%%%%%%%%%%%%%%%%%%%
\DescribeMacro{\childdocmain}
To use the package, add the commands
\begin{center}
\begin{tabular}{l}
|% \iffalse
%
% childdoc.dtx Copyright (C) 2017-2018 Niklas Beisert
%
% This work may be distributed and/or modified under the
% conditions of the LaTeX Project Public License, either version 1.3
% of this license or (at your option) any later version.
% The latest version of this license is in
%   http://www.latex-project.org/lppl.txt
% and version 1.3 or later is part of all distributions of LaTeX
% version 2005/12/01 or later.
%
% This work has the LPPL maintenance status `maintained'.
%
% The Current Maintainer of this work is Niklas Beisert.
%
% This work consists of the files childdoc.dtx and childdoc.ins
% and the derived files childdoc.def and cdocsamp.tex with
% cdocsch1.tex, cdocsch2.tex, cdocsdrf.tex, cdocsfn1.tex, cdocsfn2.tex.
%
%<package>\ifdefined\childdocmain\endinput\fi
%<package>\ProvidesFile{childdoc.def}[2018/12/30 v2.0 child document driver]
%<samplemain>\ProvidesFile{cdocsamp.tex}[2018/12/30 v2.0 sample for childdoc]
%<*driver>
%\ProvidesFile{childdoc.drv}[2018/12/30 v2.0 childdoc reference manual file]
\PassOptionsToClass{10pt,a4paper}{article}
\documentclass{ltxdoc}

\usepackage[margin=35mm]{geometry}
\usepackage{hyperref}
\usepackage{hyperxmp}
\usepackage[usenames]{color}

\hypersetup{colorlinks=true}
\hypersetup{pdfstartview=FitH}
\hypersetup{pdfpagemode=UseNone}
\hypersetup{pdfsource={}}
\hypersetup{pdflang={en-UK}}
\hypersetup{pdfcopyright={Copyright 2017-2018 Niklas Beisert.
  This work may be distributed and/or modified under the
  conditions of the LaTeX Project Public License, either version 1.3
  of this license or (at your option) any later version.}}
\hypersetup{pdflicenseurl={http://www.latex-project.org/lppl.txt}}
\hypersetup{pdfcontactaddress={ETH Zurich, ITP, HIT K,
  Wolfgang-Pauli-Strasse 27}}
\hypersetup{pdfcontactpostcode={8093}}
\hypersetup{pdfcontactcity={Zurich}}
\hypersetup{pdfcontactcountry={Switzerland}}
\hypersetup{pdfcontactemail={nbeisert@itp.phys.ethz.ch}}
\hypersetup{pdfcontacturl={http://people.phys.ethz.ch/\xmptilde nbeisert/}}

\newcommand{\secref}[1]{\hyperref[#1]{section \ref*{#1}}}

\parskip1ex
\parindent0pt
\let\olditemize\itemize
\def\itemize{\olditemize\parskip0pt}

\begin{document}

\title{The \textsf{childdoc} Package}
\hypersetup{pdftitle={The childdoc Package}}
\author{Niklas Beisert\\[2ex]
  Institut f\"ur Theoretische Physik\\
  Eidgen\"ossische Technische Hochschule Z\"urich\\
  Wolfgang-Pauli-Strasse 27, 8093 Z\"urich, Switzerland\\[1ex]
  \href{mailto:nbeisert@itp.phys.ethz.ch}
  {\texttt{nbeisert@itp.phys.ethz.ch}}}
\hypersetup{pdfauthor={Niklas Beisert}}
\hypersetup{pdfsubject={Manual for the LaTeX2e Package childdoc}}
\date{30 December 2018, \textsf{v2.0}}
\maketitle

\begin{abstract}\noindent
\textsf{childdoc} is a \LaTeXe{} package
that enables the direct compilation
of document sections included by |\include|
to individual files.
\end{abstract}

\begingroup
\parskip0ex
\tableofcontents
\endgroup

%%%%%%%%%%%%%%%%%%%%%%%%%%%%%%%%%%%%%%%%%%%%%%%%%%%%%%%%%%%%%%%%%%%%%%%%%%%%%%%%
%%%%%%%%%%%%%%%%%%%%%%%%%%%%%%%%%%%%%%%%%%%%%%%%%%%%%%%%%%%%%%%%%%%%%%%%%%%%%%%%
\section{Introduction}

\LaTeX{} provides a mechanism to structure a large document (such as a book)
into a main file and several child files (containing the chapters)
using the |\include| command.
This mechanism is beneficial for documents
which span hundreds of pages in order to
make the source file(s) more manageable.
Moreover, compilation can be restricted to
selected child files by means of the |\includeonly| command.
The latter feature can be used to reduce the compilation time while editing
(this was significantly more useful in the earlier days of \LaTeX{})
or to generate a smaller document which is easier to navigate.
Another application of |\includeonly| is to generate
documents consisting of selected parts of the complete document.

However, there are a few drawbacks of the plain |\include| mechanism:
\begin{itemize}
\item
The child files cannot be compiled on their own,
they can only be compiled via the main file.
A naive editing environment
(such as a text editor with an option
to have the current file processed by \LaTeX)
may require one to switch to the main file before compiling;
attempting to compile the child file produces errors.
\item
The main file must be modified (each time)
to adjust the |\includeonly| command
to the present needs. This easily leaves the main file in a messy state.
\item
The generated document will always carry the filename
of the main document. This is inconvenient if
several child files are to be compiled and
to be kept for distribution.
\end{itemize}

The present package provides a simple interface
to make child files individually compilable by \LaTeX{}.
Compiling a child file then has the same effect as compiling
the main file with an |\includeonly| command
to select the appropriate child.
Moreover the generated document will carry the name of the child
rather than the main file.
This resolves all three above issues.

This feature is meant to make the editing of books,
thesis documents and lecture notes somewhat more convenient.
However, the package can also be used efficiently for
composing a series of documents (such as exercise sheets)
which are typically distributed individually.
It then assists the author in generating the individual documents
(potentially in different versions)
as well as a document containing the collected series.
Another application is in developing style files
or other kinds of included material
where compilation of the style file could redirect
to a sample or test file.

%%%%%%%%%%%%%%%%%%%%%%%%%%%%%%%%%%%%%%%%%%%%%%%%%%%%%%%%%%%%%%%%%%%%%%%%%%%%%%%%
%%%%%%%%%%%%%%%%%%%%%%%%%%%%%%%%%%%%%%%%%%%%%%%%%%%%%%%%%%%%%%%%%%%%%%%%%%%%%%%%
\section{Usage}

First of all, the package \textsf{childdoc} is \emph{not} a standard
\LaTeXe{} |.sty| style file! Therefore it needs to be invoked in
a non-standard way.

%%%%%%%%%%%%%%%%%%%%%%%%%%%%%%%%%%%%%%%%%%%%%%%%%%%%%%%%%%%%%%%%%%%%%%%%%%%%%%%%
\subsection{Included Files}
\label{sec:include}

%%%%%%%%%%%%%%%%%%%%%%%%%%%%%%%%%%%%%%%%
\DescribeMacro{\childdocmain}
To use the package, add the commands
\begin{center}
\begin{tabular}{l}
|\input{childdoc.def}|\\
|\childdocmain{}|\\
\end{tabular}
\end{center}
at the very top of the main \LaTeX{} file,
in particular \emph{before} the |\documentclass| statement!
The argument of |\childdocmain| should be left empty
(but it must be present).

%%%%%%%%%%%%%%%%%%%%%%%%%%%%%%%%%%%%%%%%
\DescribeMacro{\childdocof}
Furthermore, add the commands
\begin{center}
\begin{tabular}{l}
|\input{childdoc.def}|\\
|\childdocof{|\textit{main}|}|\\
\end{tabular}
\end{center}
at the top of every child file \textit{child}
which is included by |\include{|\textit{child}|}|
from within the main file
(or at least for those files to be compiled individually).
The argument \textit{main} must be the filename of the main file.

There are a couple of
considerations in setting up the main and child documents:

%%%%%%%%%%%%%%%%%%%%%%%%%%%%%%%%%%%%%%%%
\paragraph{Restrictions.}

Please note the following restrictions:
\begin{itemize}
\item
|\childdocmain| must be called with one argument \textit{main}
to ensure compatibility with earlier version of the package.
It must either be empty (|\childdocmain{}|)
or precisely match the filename of the main file in which it is specified.
See \secref{sec:detection} for further information.
\item
The filename \textit{main} must be specified without the |.tex| extension.
\item
The filename \textit{main} is case sensitive
(even in case-insensitive file systems)
due to internal string comparison.
\item
The argument \textit{main} should be fully expanded, it cannot be a macro.
\item
Subdirectories and special characters should be avoided in filenames.
\item
The command |\childdocmain{|\textit{main}|}| must be followed by a whitespace.
It should not be followed immediately by another command
or by a comment mark `|%|'.
This is because the \TeX{} parser reads the token immediately following
the argument of |\childdocmain| and puts it
at the beginning of every child section;
however, a white\-space is ignored.
\end{itemize}

%%%%%%%%%%%%%%%%%%%%%%%%%%%%%%%%%%%%%%%%
\paragraph{Content of Main File.}

It is advisable to place all content in the child files included by |\include|.
Any output contained in the main file will appear in all child documents
unless suppressed manually;
it cannot be suppressed automatically by the |\includeonly| directive
and thus should normally be avoided.
A method to include some content in the main file
by means of conditional processing is described in \secref{sec:conditional}.

%%%%%%%%%%%%%%%%%%%%%%%%%%%%%%%%%%%%%%%%
\paragraph{Page Numbering.}

When only a part of the document is compiled,
the appropriate numbering of pages
(as well as other status parameters)
is determined from the |.aux| files.
The latter contain information from previous passes.
However this information needs to propagate through
all intermediate child documents.
Therefore the page numbering in child documents may well
be inconsistent until the complete document is compiled at least once.

A useful (if unconventional) way to always ensure a consistent
page numbering is to restart the numbering in each child document
and denote the pages by `\textit{child}|.|\textit{page}'
where \textit{child} represents the chapter/section number of the child file.
This can be achieved by the command
|\numberwithin{page}{|\textit{child}|}|
of the \textsf{amsmath} package
where \textit{child} can be |chapter| or |section|
depending on the chosen structuring.
Alternatively, one can modify the macro |\thepage| appropriately
and reset the counter |page| at the start of each child file.

%%%%%%%%%%%%%%%%%%%%%%%%%%%%%%%%%%%%%%%%%%%%%%%%%%%%%%%%%%%%%%%%%%%%%%%%%%%%%%%%
\subsection{Conditional Processing}
\label{sec:conditional}

The package provides a mechanism to compile different versions
of a document. To customise the versions further some conditional processing
can come in handy to distinguish which version is being compiled.
The package provides two macros to describe the compilation context:

%%%%%%%%%%%%%%%%%%%%%%%%%%%%%%%%%%%%%%%%
\DescribeMacro{\ifchilddoc}
The conditional |\ifchilddoc| distinguishes between the compilation of
child documents and the main document:
%
\begin{center}
|\ifchilddoc |\textit{child-code}| |[|\||else |\textit{main-code}]| \||fi|
\end{center}

%%%%%%%%%%%%%%%%%%%%%%%%%%%%%%%%%%%%%%%%
\DescribeMacro{\childdocname}
\DescribeMacro{\childdocjob}
The macro |\childdocname| contains the filename (without extension)
of the main or child file being processed.
Note that |\childdocjob| will always contain the name of the main file.

%%%%%%%%%%%%%%%%%%%%%%%%%%%%%%%%%%%%%%%%
\paragraph{Title Page.}

Conditional processing can be used to include a title or banner page
in the main document when proper precautions are taken.
Importantly, the code in the main file should ensure that the page counter
(as well as other status parameters which are stored in the |.aux| files)
takes the same value after the conditional processing.
Otherwise the page numbers may take divergent values
depending on which part is compiled.

For example, a title page could be declared by:
%
\begin{center}
\begin{tabular}{l}
|\ifchilddoc\||else|\\
|\addtocounter{page}{-1}|\\
\textit{code for title page}\\
|\newpage|\\
|\||fi|
\end{tabular}
\end{center}
%
A banner page for the child documents can be generated by:
%
\begin{center}
\begin{tabular}{l}
|\ifchilddoc|\\
|\addtocounter{page}{-1}|\\
\textit{code for banner page}\\
|\newpage|\\
|\||fi|
\end{tabular}
\end{center}
%
Here one could write a message such as:
\begin{center}
|This is the part \childdocname{} of \childdocjob{}.|
\end{center}

%%%%%%%%%%%%%%%%%%%%%%%%%%%%%%%%%%%%%%%%%%%%%%%%%%%%%%%%%%%%%%%%%%%%%%%%%%%%%%%%
\subsection{Flags}
\label{sec:flags}

The package makes it easy to generate different versions
of the main or child documents.
To this end compilation flags can be defined
and assigned different default values.
They will be particularly useful in conjunction
with the forwarding mechanism described in \secref{sec:forward}.

For example, it may be useful to have a flag |\version|
which can be set to |draft| or |final|.
The document source will contain some conditional code
depending on the value of |\version|.
Suppose further, the flag should default to |final| for the main file
and to |draft| for child files
which is a natural assignment for editing the document.
This is achieved by placing the following code
in the preamble of the main document
(below the |\childdocmain| directive):
%
\begin{center}
\begin{tabular}{l}
|\ifchilddoc|\\
|\providecommand{\version}{draft}|\\
|\||else|\\
|\providecommand{\version}{final}|\\
|\||fi|
\end{tabular}
\end{center}
%
The definition by |\providecommand| makes sure
that previous definitions are not overwritten.
Further statements |\providecommand{\version}{...}|
can thus be added before the above code to override it.

For the main file, one might add a line
(between |\childdocmain| and the above block)
%
\begin{center}
|%\ifchilddoc\||else\providecommand{\version}{draft}\||fi|
\end{center}
%
which can be uncommented to produce a draft version.
Likewise one can add a line to the very top of a child file
(above the |\childdocof{|\textit{main}|}| directive)
%
\begin{center}
|%\providecommand{\version}{final}|
\end{center}
%
which can be uncommented to produce the final version of this child document.

%%%%%%%%%%%%%%%%%%%%%%%%%%%%%%%%%%%%%%%%%%%%%%%%%%%%%%%%%%%%%%%%%%%%%%%%%%%%%%%%
\subsection{Forwarding}
\label{sec:forward}

Different versions of the main or child documents
using compilation flags as described in \secref{sec:flags}
can be (permanently) stored in different files
for convenient compilation, viewing and distribution.
To this end, the package defines a command
to pass on compilation to a different file:

%%%%%%%%%%%%%%%%%%%%%%%%%%%%%%%%%%%%%%%%
\DescribeMacro{\childdocforward}
The command |\childdocforward| redirects processing to
another source file:
%
\begin{center}
\begin{tabular}{l}
|\input{childdoc.def}|\\
|\childdocforward[|\textit{main}|]{|\textit{dest}|}|\\
\end{tabular}
\end{center}
%
The argument \textit{dest} is the destination file
(without extension).
It should be the main file or one of the child files.
Note that further \textsf{childdoc} directives
such as |\childdocof| and |\childdocforward|
in the indicated file will be processed in this form.
The optional argument \textit{main}
passes on directly to the main file \textit{main}
while pretending to compile the child \textit{dest}.
This form behaves as if \textit{dest}
issues |\childdocof{|\textit{main}|}| right away,
and no further \textsf{childdoc} directives will be processed.

%%%%%%%%%%%%%%%%%%%%%%%%%%%%%%%%%%%%%%%%
\DescribeMacro{\...prefix}
In the alternative form |\childdocforwardprefix|,
%
\begin{center}
\begin{tabular}{l}
|\input{childdoc.def}|\\
|\childdocforwardprefix[|\textit{main}|]{|\textit{prefix}|}{|\textit{dest}|}|
\end{tabular}
\end{center}
%
the destination file is determined by a pattern
depending on the current file:
To make this work, the current file must be called
`{\textit{prefix}\hspace{0.2em}\textit{suffix}}'
with \textit{prefix} matching precisely the argument.
Processing is then passed on to the file
`{\textit{dest}\hspace{0.2em}\textit{suffix}}'.
Surely, the same effect is achieved by
directly specifying the
argument `{\textit{dest}\hspace{0.2em}\textit{suffix}}'
in the first form.
However, that requires to set up a different file
for each child. With the alternative form of the command
all these files can have exactly the same content
which simplifies setting them up and maintaining them.

For example, the following file |draft.tex|
with a compilation flag |\version| as described in \secref{sec:flags}
compiles the main document as a draft:
%
\begin{center}
\begin{tabular}{l}
|\def\version{draft}|\\
|\input{childdoc.def}|\\
|\childdocforward{|\textit{main}|}|
\end{tabular}
\end{center}
%
Likewise, the following files |final|\textit{nn}|.tex|
compile the final version of the child document
|child|\textit{nn}|.tex|:
%
\begin{center}
\begin{tabular}{l}
|\def\version{final}|\\
|\input{childdoc.def}|\\
|\childdocforwardprefix{final}{child}|
\end{tabular}
\end{center}
%

Note that when several versions of a main file and/or of each child file
are to be generated, it may be convenient to set up a |Makefile| or
shell script to automatise the process.

%%%%%%%%%%%%%%%%%%%%%%%%%%%%%%%%%%%%%%%%%%%%%%%%%%%%%%%%%%%%%%%%%%%%%%%%%%%%%%%%
\subsection{Command Line Processing}
\label{sec:commandline}

The effect of redirection files can also be achieved by invoking
the \LaTeX{} compiler with a more elaborate command line.
Most conveniently this should be done as part
of a shell script or a |Makefile|.

When using \textsf{childdoc} in the main file, the following
command lines effectively perform a redirection
(note that depending on the shell being used,
backslashes may have to be doubled: `|\|' $\to$ `|\\|'):
%
\begin{center}
|... -jobname "|\textit{target}|" |\\|"|[\textit{flags}]%
|\input{childdoc.def}\childdocforward[|\textit{main}|]{|\textit{dest}|}"|
\end{center}
%
Here \textit{target} is the name of the output file,
\textit{main} is the name of the main file
and \textit{dest} is the name of the main or child file to be processed
(all filenames without extensions).
The optional argument \textit{main} can be omitted
if \textit{main} matches \textit{dest}.
Optionally, compilation \textit{flags} can be defined via |\def| commands.
This command line makes the \TeX{} engine believe
it is compiling the file \textit{target}
whose content is specified as the latter parameter.
The provided code then forwards the processing to
\textit{main} or \textit{dest} as described in \secref{sec:forward}.

%%%%%%%%%%%%%%%%%%%%%%%%%%%%%%%%%%%%%%%%%%%%%%%%%%%%%%%%%%%%%%%%%%%%%%%%%%%%%%%%
\subsection{Include by Input}
\label{sec:input}

Including child documents by |\include| has some restrictions by design.
Most notably, the content of a child document always occupies
its own set of pages; pages cannot be shared between child documents.
Usually, this behaviour makes perfect sense
because each child document contain an essential part of the document.
However, in some situations it may be desirable to compose
a document from a collection of parts
without having mandatory page breaks between then.
For this case, the package
provides a mechanism to include parts
by |\input| which can also be processed individually.
However, by construction this mechanism
requires manual handling of the content to be output.

%%%%%%%%%%%%%%%%%%%%%%%%%%%%%%%%%%%%%%%%
\DescribeMacro{\ifchilddocmanual}
The main file should be prepared as usual, see \secref{sec:include}.
However, the document body must make a distinction
between processing of an individual part and of the main document, e.g.:
%
\begin{center}
\begin{tabular}{l}
|\ifchilddocmanual|\\
|\input{\childdocname}|\\
|\||else|\\
\textit{document body with }|\input{|\textit{part}|}|\\
|\||fi|
\end{tabular}
\end{center}
%
The conditional |\ifchilddocmanual| is true whenever
a part to be included by |\input| is being compiled,
and the name of the part is stored in |\childdocname|.

%%%%%%%%%%%%%%%%%%%%%%%%%%%%%%%%%%%%%%%%
\DescribeMacro{\childdocby}
Each part to be included by |\input| should start with:
%
\begin{center}
\begin{tabular}{l}
|\input{childdoc.def}|\\
|\childdocby{|\textit{main}|}|\\
\end{tabular}
\end{center}
%
The directive |\childdocby| is similar to |\childdocof|
described in \secref{sec:include},
but the subsequent selection of content must be done manually.
To that end, both |\ifchilddoc| and |\ifchilddocmanual|
will be true upon processing of a part,
and the name of the part is stored in |\childdocname|.
Note that |\jobname| will be set to the filename of the current part
so that each part receives an individual |.aux| file
that does not interfere with the |.aux| file(s) of the main document.
This behaviour can be altered by the alternative form
|\childdocby[*]{|\textit{main}|}| (with a non-empty optional argument)
which uses the |.aux| file of the main document
by setting |\jobname| to \textit{main}.

%%%%%%%%%%%%%%%%%%%%%%%%%%%%%%%%%%%%%%%%%%%%%%%%%%%%%%%%%%%%%%%%%%%%%%%%%%%%%%%%
\subsection{Driver Development}
\label{sec:driver}

The \textsf{childdoc} mechanism can also be use for the development
of definition files such as \LaTeX{} styles or classes.
This case differs from the above setup with multiple parts
included by |\include| in that no |\includeonly| should be invoked.
This can be achieved by starting the include file
(before |\ProvidesPackage|) with:
%
\begin{center}
\begin{tabular}{l}
|\input{childdoc.def}|\\
|\childdocforward{|\textit{main}|}|\\
\end{tabular}
\end{center}
%
or alternatively with:
%
\begin{center}
\begin{tabular}{l}
|\input{childdoc.def}|\\
|\childdocby{|\textit{main}|}|\\
\end{tabular}
\end{center}
%
Both forms have slightly different effects as described above.
The main file is prepared as usual, see \secref{sec:include}.

%%%%%%%%%%%%%%%%%%%%%%%%%%%%%%%%%%%%%%%%%%%%%%%%%%%%%%%%%%%%%%%%%%%%%%%%%%%%%%%%
\subsection{Legacy Detection}
\label{sec:detection}

The directive |\childdocmain| in the main file can detect
whether the complete document or merely a child is to be compiled
even without using the directive |\childdocof|.
This method is deprecated because it is less robust
and there is no compelling reason to use it;
it is merely provided for backward compatibility
and it may be removed in future versions.

If the detection mechanism is to be used,
it is mandatory to correctly specify
the filename of the main file as the argument of |\childdocmain|:
%
\begin{center}
\begin{tabular}{l}
|\input{childdoc.def}|\\
|\childdocmain{|\textit{main}|}|\\
\end{tabular}
\end{center}
%
If |\jobname| does not match the argument \textit{main} of |\childdocmain|,
it is assumed that |\jobname| points to the child file to be compiled.
When using |\childdocmain| with the main file specified as argument,
it suffices to start a child file
with just |\input{|\textit{main}|}|
without loading of the package and using |\childdocof|.
If instead all processing is done
with the appropriate \textsf{childdoc} directives,
the argument of \textit{main} of |\childdocmain| can be empty.

An alternative version of the command line processing described
in \secref{sec:commandline} using the detection mechanism reads:
%
\begin{center}
|... -jobname "|\textit{target}|" "|[\textit{flags}]%
[|\def\jobname{|\textit{dest}|}|]|\input{|\textit{main}|}"|
\end{center}

%%%%%%%%%%%%%%%%%%%%%%%%%%%%%%%%%%%%%%%%%%%%%%%%%%%%%%%%%%%%%%%%%%%%%%%%%%%%%%%%
\subsection{Manual Code}
\label{sec:manual}

In case one cannot be certain whether the definitions file |childdoc.def|
is installed on the target \TeX{} distribution
and one prefers not to ship it,
it is conceivable to paste a few relevant commands into the sources.

To that end, drop all statements |\input{childdoc.def}|
and perform the replacements as outlined below.
Instead of |\childdocmain{|\textit{main}|}| add the following code
to the top of the main file:
%
\begin{center}
\begin{tabular}{l}
|\||ifdefined\childdocname\endinput\||fi\newif\ifchilddoc|\\
|\edef\childdocname{\scantokens\expandafter{\jobname\noexpand}}|\\
|\def\childdocmain{|\textit{main}|}\||ifx\childdocmain\childdocname\||else|\\
|\childdoctrue\includeonly{\childdocname}\let\jobname\childdocmain\||fi|\\
\end{tabular}
\end{center}
%
Instead of |\childdocof{|\textit{main}|}| just include the main file
at the top of each child file:
%
\begin{center}
|\input{|\textit{main}|}|
\end{center}
%
A simple redirection |\childdocforward{|\textit{dest}|}| is achieved by:
%
\begin{center}
|\def\jobname{|\textit{dest}|}\input{\jobname}|
\end{center}
%
The redirection with prefix
|\childdocforwardprefix[|\textit{prefix}|]{|\textit{dest}|}|
is accomplished by:
%
\begin{center}
\begin{tabular}{l}
|{\edef\jobname{\scantokens\expandafter{\jobname\noexpand}}|\\
|\def\redirectjob |\textit{prefix}|#1~~~{\gdef\jobname{|\textit{dest}|#1}}|\\
|\expandafter\redirectjob\jobname~~~}\input{\jobname}|
\end{tabular}
\end{center}

In an alternative approach,
child documents can be compiled by a specific command line
without additional code or specific definitions:
%
\begin{center}
|... -jobname "|\textit{target}|" "|[\textit{flags}]%
|\includeonly{|\textit{dest}|}\input{|\textit{main}|}"|
\end{center}
%

%%%%%%%%%%%%%%%%%%%%%%%%%%%%%%%%%%%%%%%%%%%%%%%%%%%%%%%%%%%%%%%%%%%%%%%%%%%%%%%%
%%%%%%%%%%%%%%%%%%%%%%%%%%%%%%%%%%%%%%%%%%%%%%%%%%%%%%%%%%%%%%%%%%%%%%%%%%%%%%%%
\section{Information}

%%%%%%%%%%%%%%%%%%%%%%%%%%%%%%%%%%%%%%%%%%%%%%%%%%%%%%%%%%%%%%%%%%%%%%%%%%%%%%%%
\subsection{Copyright}

Copyright \copyright{} 2017--2018 Niklas Beisert

This work may be distributed and/or modified under the
conditions of the \LaTeX{} Project Public License, either version 1.3
of this license or (at your option) any later version.
The latest version of this license is in
  \url{http://www.latex-project.org/lppl.txt}
and version 1.3 or later is part of all distributions of \LaTeX{}
version 2005/12/01 or later.

This work has the LPPL maintenance status `maintained'.

The Current Maintainer of this work is Niklas Beisert.

This work consists of the files |README.txt|, |childdoc.ins| and |childdoc.dtx|
as well as the derived files |childdoc.def|, |cdocsamp.tex|
with |cdocsch1.tex|, |cdocsch2.tex|, |cdocspt3.tex|, |cdocspt4.tex|,
|cdocsdrf.tex|, |cdocsfn1.tex|, |cdocsfn2.tex|
as well as |childdoc.pdf|.

%%%%%%%%%%%%%%%%%%%%%%%%%%%%%%%%%%%%%%%%%%%%%%%%%%%%%%%%%%%%%%%%%%%%%%%%%%%%%%%%
\subsection{Files and Installation}

The package consists of the files:
%
\begin{center}
\begin{tabular}{ll}
    |README.txt|   & readme file \\
    |childdoc.ins| & installation file \\
    |childdoc.dtx| & source file \\
    |childdoc.def| & definition file \\
    |cdocsamp.tex| & sample main file \\
    |cdocsch1.tex| & sample include file \\
    |cdocsch2.tex| & sample include file \\
    |cdocspt3.tex| & sample part file \\
    |cdocspt4.tex| & sample part file \\
    |cdocsdrf.tex| & sample redirection file \\
    |cdocsfn1.tex| & sample redirection file \\
    |cdocsfn2.tex| & sample redirection file \\
    |childdoc.pdf| & manual
\end{tabular}
\end{center}
%
The distribution consists of the files
|README.txt|, |childdoc.ins| and |childdoc.dtx|.
%
\begin{itemize}
\item
Run (pdf)\LaTeX{} on |childdoc.dtx|
to compile the manual |childdoc.pdf| (this file).
\item
Run \LaTeX{} on |childdoc.ins| to create the definitions file |childdoc.def|
and the sample |cdocsamp.tex| with include files
|cdocsch1.tex|, |cdocsch2.tex|, |cdocspt3.tex|, |cdocspt4.tex|,
|cdocsdrf.tex|, |cdocsfn1.tex|, |cdocsfn2.tex|.
Then copy the file |childdoc.def| to an appropriate directory of your \LaTeX{}
distribution, e.g.\ \textit{texmf-root}|/tex/latex/childdoc|.
\end{itemize}

%%%%%%%%%%%%%%%%%%%%%%%%%%%%%%%%%%%%%%%%%%%%%%%%%%%%%%%%%%%%%%%%%%%%%%%%%%%%%%%%
\subsection{Related CTAN Packages}

There are several other packages which offer a similar functionality:
%
\begin{itemize}
\item
The packages
\href{http://ctan.org/pkg/docmute}{\textsf{docmute}},
\href{http://ctan.org/pkg/includex}{\textsf{includex}} and
\href{http://ctan.org/pkg/standalone}{\textsf{standalone}}
provide commands to include only the document body of
a child file thus allowing both files to be compiled individually.
\item
The packages \href{http://ctan.org/pkg/subdocs}{\textsf{subdocs}}
and \href{http://ctan.org/pkg/subfiles}{\textsf{subfiles}}
provide structures in which the main and child documents can be
encapsulated and allowing them to be compiled individually.
The inclusion mechanism is different from the conventional |\include|.
\item
The package \href{http://ctan.org/pkg/combine}{\textsf{combine}}
is an elaborate solution to combine several documents into one.
\end{itemize}
%
See also the CTAN topic \href{http://ctan.org/topic/subdocs}{\textsf{subdocs}}
for further related packages.
The present package differs from the above solutions in that
a document structure constructed with the conventional |\include| mechanism
just needs two extra commands at the top of every file
such that all constituent files can be compiled individually.

%%%%%%%%%%%%%%%%%%%%%%%%%%%%%%%%%%%%%%%%%%%%%%%%%%%%%%%%%%%%%%%%%%%%%%%%%%%%%%%%
%\subsection{Feature Suggestions}
%
%The following is a list of features which may be useful for future
%versions of this package:
%%
%\begin{itemize}
%\item
%\ldots
%\end{itemize}

%%%%%%%%%%%%%%%%%%%%%%%%%%%%%%%%%%%%%%%%%%%%%%%%%%%%%%%%%%%%%%%%%%%%%%%%%%%%%%%%
\subsection{Revision History}

%%%%%%%%%%%%%%%%%%%%%%%%%%%%%%%%%%%%%%%%
\paragraph{v2.0:} 2018/12/30

\begin{itemize}
\item
immediate forward processing
\item
added |\childdocby| mechanism
\item
manual restructured
\end{itemize}

%%%%%%%%%%%%%%%%%%%%%%%%%%%%%%%%%%%%%%%%
\paragraph{v1.6:} 2018/01/17

\begin{itemize}
\item
application for development of include files
\item
corrections to manual
\end{itemize}

%%%%%%%%%%%%%%%%%%%%%%%%%%%%%%%%%%%%%%%%
\paragraph{v1.5:} 2017/05/21

\begin{itemize}
\item
more complete structuring introduced
\item
|\childdocof| introduced
\item
|\childdoc| renamed to |\childdocmain|
\item
|\childredirect| renamed to |\childdocforward| and |\childdocforwardprefix|
and functionality expanded
\end{itemize}

%%%%%%%%%%%%%%%%%%%%%%%%%%%%%%%%%%%%%%%%
\paragraph{v1.0:} 2017/04/27

\begin{itemize}
\item
manual and install package
\item
first version published on CTAN
\end{itemize}

%%%%%%%%%%%%%%%%%%%%%%%%%%%%%%%%%%%%%%%%
\paragraph{v0.6:} 2017/04/26

\begin{itemize}
\item
redirection mechanism added
\end{itemize}

%%%%%%%%%%%%%%%%%%%%%%%%%%%%%%%%%%%%%%%%
\paragraph{v0.5:} 2017/04/26

\begin{itemize}
\item
functionality in definition file
\end{itemize}


%%%%%%%%%%%%%%%%%%%%%%%%%%%%%%%%%%%%%%%%%%%%%%%%%%%%%%%%%%%%%%%%%%%%%%%%%%%%%%%%
%%%%%%%%%%%%%%%%%%%%%%%%%%%%%%%%%%%%%%%%%%%%%%%%%%%%%%%%%%%%%%%%%%%%%%%%%%%%%%%%
%%%%%%%%%%%%%%%%%%%%%%%%%%%%%%%%%%%%%%%%%%%%%%%%%%%%%%%%%%%%%%%%%%%%%%%%%%%%%%%%
\appendix

\settowidth\MacroIndent{\rmfamily\scriptsize 000\ }

 \DocInput{childdoc.dtx}

\end{document}
%</driver>
% \fi
%
% %%%%%%%%%%%%%%%%%%%%%%%%%%%%%%%%%%%%%%%%%%%%%%%%%%%%%%%%%%%%%%%%%%%%%%%%%%%%%%
% %%%%%%%%%%%%%%%%%%%%%%%%%%%%%%%%%%%%%%%%%%%%%%%%%%%%%%%%%%%%%%%%%%%%%%%%%%%%%%
% \section{Sample}
%\iffalse
%<*samplemain>
%\fi
%
% The following presents a sample document
% with two chapters, two parts, a title page,
% a compile flag as well as three forwarding files to set the flag.
% It consists of eight |.tex| files:
% \begin{center}
% \begin{tabular}{ll}
% |cdocsamp.tex|&main file\\
% |cdocsch1.tex|&include file for chapter 1\\
% |cdocsch2.tex|&include file for chapter 2\\
% |cdocspt3.tex|&include file for part 3\\
% |cdocspt4.tex|&include file for part 4\\
% |cdocsdrf.tex|&forwarding file for main file in draft mode\\
% |cdocsfi1.tex|&forwarding file for final version of chapter 1\\
% |cdocsfi2.tex|&forwarding file for final version of chapter 2\\
% \end{tabular}
% \end{center}
% Each of the eight files can be compiled directly by the \LaTeX{} compiler.
%
% %%%%%%%%%%%%%%%%%%%%%%%%%%%%%%%%%%%%%%
% \paragraph{Main File.}
%
% The main file is called |cdocsamp.tex|.
%
% Load the \textsf{childdoc} definitions and
% declare the filename for the main document:
%    \begin{macrocode}
\input{childdoc.def}
\childdocmain{}
%    \end{macrocode}

% Optional override for |\version| flag:
%    \begin{macrocode}
%%\ifchilddoc\else\providecommand{\version}{draft}\fi
%    \end{macrocode}

% Define the default values for the |\version| flag
% (|final| for the main file and |draft| for childs):
%    \begin{macrocode}
\ifchilddoc
\providecommand{\version}{draft}
\else
\providecommand{\version}{final}
\fi
%    \end{macrocode}

% Load the standard document class:
%    \begin{macrocode}
\documentclass[12pt]{article}
%    \end{macrocode}

% Start the document body:
%    \begin{macrocode}
\begin{document}
%    \end{macrocode}

% Declare a title page.
% Print title, part of document being processed and version flag:
%    \begin{macrocode}
\addtocounter{page}{-1}
\begin{center}
{\LARGE\bfseries{}childdoc example\par}
\vspace{1cm}
\ifchilddoc
\ifchilddocmanual part\else chapter\fi:
`\childdocname' of `\childdocjob'\par
\else
main document: `\childdocjob'\par
\fi
version: \version\par
\end{center}
\newpage
%    \end{macrocode}

% Manually include selected file,
% otherwise process as usual:
%    \begin{macrocode}
\ifchilddocmanual
\section*{part `\childdocname'}
\input{\childdocname}
\else
%    \end{macrocode}

% Include the two chapters:
%    \begin{macrocode}
\include{cdocsch1}
\include{cdocsch2}
%    \end{macrocode}

% Include the two parts unless only chapters should be displayed:
%    \begin{macrocode}
\ifchilddoc\else
\section{part three}
\input{cdocspt3}
\section{part four}
\input{cdocspt4}
\fi
%    \end{macrocode}

% Process as usual until here:
%    \begin{macrocode}
\fi
%    \end{macrocode}

% End of document body:
%    \begin{macrocode}
\end{document}
%    \end{macrocode}
%\iffalse
%</samplemain>
%\fi
%
% %%%%%%%%%%%%%%%%%%%%%%%%%%%%%%%%%%%%%%
% \paragraph{Chapter Include Files.}
%
% The include files are called |cdocsch1.tex| and |cdocsch2.tex|.
%
%\iffalse
%<*samplechap1|samplechap2>
%\fi

% Optional override for |\version| flag:
%    \begin{macrocode}
%%\providecommand{\version}{final}
%    \end{macrocode}

% Include the main document:
%    \begin{macrocode}
\input{childdoc.def}
\childdocof{cdocsamp}
%    \end{macrocode}

%\iffalse
%</samplechap1|samplechap2>
%\fi
%
%\iffalse
%<*samplechap1>
%\fi
% Some text for chapter 1:
%    \begin{macrocode}
\section{one}
some text in chapter one
%    \end{macrocode}

%\iffalse
%</samplechap1>
%\fi
% Some text for chapter 2:
%\iffalse
%<*samplechap2>
%\fi
%    \begin{macrocode}
\section{two}
more text in chapter two
%    \end{macrocode}

%\iffalse
%</samplechap2>
%\fi
%
% %%%%%%%%%%%%%%%%%%%%%%%%%%%%%%%%%%%%%%
% \paragraph{Part Include Files.}
%
% The include files are called |cdocspt3.tex| and |cdocspt4.tex|.
%
%\iffalse
%<*samplepart3|samplepart4>
%\fi

% Optional override for |\version| flag:
%    \begin{macrocode}
%%\providecommand{\version}{final}
%    \end{macrocode}

% Include the main document:
%    \begin{macrocode}
\input{childdoc.def}
\childdocby{cdocsamp}
%    \end{macrocode}

%\iffalse
%</samplepart3|samplepart4>
%\fi
%
%\iffalse
%<*samplepart3>
%\fi
% Some text for part 3:
%    \begin{macrocode}
some text in part three
%    \end{macrocode}

%\iffalse
%</samplepart3>
%\fi
% Some text for part 4:
%\iffalse
%<*samplepart4>
%\fi
%    \begin{macrocode}
more text in part four
%    \end{macrocode}

%\iffalse
%</samplepart4>
%\fi
%
% %%%%%%%%%%%%%%%%%%%%%%%%%%%%%%%%%%%%%%
% \paragraph{Forwarding for a Complete Draft.}
%
% The following forwarding file |cdocsdrf.tex|
% compiles the main document in draft mode:
%\iffalse
%<*sampledraft>
%\fi
%    \begin{macrocode}
\def\version{draft}
\input{childdoc.def}
\childdocforward{cdocsamp}
%    \end{macrocode}

%\iffalse
%</sampledraft>
%\fi
%
% %%%%%%%%%%%%%%%%%%%%%%%%%%%%%%%%%%%%%%
% \paragraph{Forwarding for Final Version of the Chapters.}
%
% The following forwarding files |cdocsfn1.tex| and |cdocsfn2.tex|
% (with identical content)
% compile the final versions of the child documents
% |cdocsch1.tex| and |cdocsch2.tex|, respectively:
%\iffalse
%<*samplefinal>
%\fi
%    \begin{macrocode}
\def\version{final}
\input{childdoc.def}
\childdocforwardprefix[cdocsamp]{cdocsfn}{cdocsch}
%    \end{macrocode}

%\iffalse
%</samplefinal>
%\fi
%
% %%%%%%%%%%%%%%%%%%%%%%%%%%%%%%%%%%%%%%
% \paragraph{Command Line Processing.}
%
% The following three command lines generate the output files
% |cdocscld|, |cdocscl1| and |cdocscl2|
% which should be identical to
% |cdocsdrf|, |cdocsch1| and |cdocsfn2|, respectively:
% \begin{center}
% \begin{tabular}{l}
% |latex -jobname cdocscld \|\\
% |  "\def\version{draft}\input{childdoc.def}\childdocforward{cdocsamp}"|\\
% |latex -jobname cdocscl1 \|\\
% |  "\input{childdoc.def}\childdocforward[cdocsamp]{cdocsch1}"|\\
% |latex -jobname cdocscl2 \|\\
% |  "\def\version{final}\input{childdoc.def}\childdocforward{cdocsch2}"|
% \end{tabular}
% \end{center}
% Note that the trailing backslash on each first line
% merely continues the input to the second line
% (for convenient cut ant paste).
% Furthermore, the command |latex| can be replaced by any
% of its alternative versions such as |pdflatex|.
%
% %%%%%%%%%%%%%%%%%%%%%%%%%%%%%%%%%%%%%%%%%%%%%%%%%%%%%%%%%%%%%%%%%%%%%%%%%%%%%%
% %%%%%%%%%%%%%%%%%%%%%%%%%%%%%%%%%%%%%%%%%%%%%%%%%%%%%%%%%%%%%%%%%%%%%%%%%%%%%%
% \section{Implementation}
%\iffalse
%<*package>
%\fi
%
% This section describes the definitions file |childdoc.def|.

% The definitions cannot be loaded using |\usepackage| or |\RequirePackage|
% which has a mechanism to prevent loading a style file more than once.
% When loading the definitions by means of |\input|
% multiple instances have to be prevented manually:
%\iffalse
%This code needs to be before the `\ProvidesFile' directive
%which is defined at the beginning of this file.
%Therefore it is also placed there and commented out here.
%</package>
%<*discard>
%\fi
%    \begin{macrocode}
\ifdefined\childdocmain\endinput\fi
%    \end{macrocode}
%\iffalse
%</discard>
%<*package>
%\fi
%
% \macro{\ifchilddoc}
% \macro{\ifchilddocmanual}
% The conditional |\ifchilddoc| tells whether a
% child (true) or main (false) document is being compiled.
% The conditional |\ifchilddocmanual| tells whether
% the |\includeonly| mechanism is used (false) or
% the selection of child files must be performed manually (true).
% The definitions initialise to false:
%    \begin{macrocode}
\newif\ifchilddoc
\newif\ifchilddocmanual
%    \end{macrocode}

% \macro{\childdocname}
% \macro{\childdocjob}
% The macro |\childdocname| stores the name of the main document
% to be compiled. The macro |\childdocjob| stores the name of
% the document on which the \LaTeX{} compiler was originally invoked.
% The content of |\jobname| cannot be compared
% to filenames specified in the source due to different catcodes.
% The following code rescans |\jobname|, stores the result
% in |\childdocname| and saves a copy in |\childdocjob|:
%    \begin{macrocode}
\edef\childdocname{\scantokens\expandafter{\jobname\noexpand}}
\let\childdocjob\childdocname
%    \end{macrocode}

% \macro{\childdocdisable}
% The macro |\childdocdisable| prevents the main file
% from being processed more than once.
% At this stage, the main document command |\childdocmain|
% is assumed to be called once again where it should do nothing.
% Any subsequent call to it should prevent
% a secondary processing of the main document
% It overwrites the forwarding commands
% |\childdocof| and |\childdocforward|
% with empty macros to prevent further inclusions of the main document:
%    \begin{macrocode}
\newcommand{\childdocdisable}
{
  \renewcommand{\childdocmain}[1]{\renewcommand{\childdocmain}[1]{\endinput}}
  \renewcommand{\childdocof}[1]{}
  \renewcommand{\childdocby}[2][]{}
  \renewcommand{\childdocforward}[2][]{}
  \renewcommand{\childdocdisable}{}
}
%    \end{macrocode}

% \macro{\childdocmain}
% The macro |\childdocmain| is to be called at the top of the main file
% with nothing or the main filename (without extension) as argument.
% First, it breaks loops.
% If the argument is not empty and does not match |\childdocname|
% (which is set by the first inclusion of |childdoc.def|),
% |\ifchilddoc| is set to true, |\includeonly| is applied to the child file
% and |\jobname| is set to the main file
% (for proper handling of |.aux| files):
%    \begin{macrocode}
\newcommand{\childdocmain}[1]
{
  \childdocdisable\childdocmain{}
  \if?#1?\else
    \begingroup
      \def\childdoctmp{#1}
      \ifx\childdoctmp\childdocname
        \def\childdoctmp{}
      \else
        \def\childdoctmp
        {
          \childdoctrue
          \includeonly{\childdocname}
          \def\childdocjob{#1}
          \def\jobname{#1}
        }
      \fi
      \expandafter
    \endgroup
    \childdoctmp
  \fi
}
%    \end{macrocode}

% \macro{\childdocof}
% The command |\childdocof| redirects
% compilation to the main file |#1|.
%    \begin{macrocode}
\newcommand{\childdocof}[1]
{
  \childdocdisable
  \childdoctrue
  \includeonly{\childdocname}
  \def\jobname{#1}
  \def\childdocjob{#1}
  \input{#1}
}
%    \end{macrocode}

% \macro{\childdocby}
% The command |\childdocby| ....
%    \begin{macrocode}
\newcommand{\childdocby}[2][]
{
  \childdocdisable
  \childdoctrue
  \childdocmanualtrue
  \if?#1?\else
    \def\jobname{#2}
  \fi
  \def\childdocjob{#2}
  \input{#2}
  \endinput
}
%    \end{macrocode}

% \macro{\childdocforward}
% The command |\childdocforward| redirects
% compilation to the main file or
% (if the optional argument is given) a child file.
% Parameters are set as if the main file
% or a child file starting with |\childdocof| was compiled.
% Then compilation is handed over to the main file:
%    \begin{macrocode}
\newcommand{\childdocforward}[2][]
{
  \begingroup
    \if?#1?
      \def\childdoctmp
      {
        \def\childdocname{#2}
        \def\childdocjob{#2}
        \def\jobname{#2}
        \input{#2}
        \endinput
      }
    \else
      \def\childdoctmp
      {
        \childdocdisable
        \def\childdocname{#2}
        \childdoctrue
        \includeonly{#2}
        \def\childdocjob{#1}
        \def\jobname{#1}
        \input{#1}
        \endinput
      }
    \fi
    \expandafter
  \endgroup
  \childdoctmp
}
%    \end{macrocode}

% \macro{\childdocforwardprefix}
% The command |\childdocforwardprefix| redirects
% compilation to the main or a child file by means of a pattern.
% The prefix |#1| in the current filename is replaced by |#2|
% and the suffix of the current filename is kept
% (it is assumed that the filename does not contain the substring `|~~~|'
% which is used as a delimiter).
% Compilation is handed over to the new file by |\childdocforward|:
%    \begin{macrocode}
\newcommand{\childdocforwardprefix}[3][]
{
  \begingroup
    \def\childdocextract #2##1~~~{\def\childdoctmp{\childdocforward[#1]{#3##1}}}
    \expandafter\childdocextract\childdocname~~~
    \expandafter
  \endgroup
  \childdoctmp
}
%    \end{macrocode}

% \macro{\childdoc}
% The deprecated macro |\childdoc| is a legacy version of |\childdocmain|:
%    \begin{macrocode}
\newcommand{\childdoc}{\childdocmain}
%    \end{macrocode}

% \macro{\childdocredirect}
% The deprecated macro |\childdocredirect| is a legacy version
% of |\childdocforward| and |\childdocforwardprefix|:
%    \begin{macrocode}
\newcommand{\childdocredirect}[2][]
{
  \begingroup
    \if?#1?
      \def\childdoctmp{\childdocforward{#2}}
    \else
      \def\childdoctmp{\childdocforwardprefix{#1}{#2}}
    \fi
    \expandafter
  \endgroup
  \childdoctmp
}
%    \end{macrocode}

%\iffalse
%</package>
%\fi
%
\endinput
|\\
|\childdocmain{}|\\
\end{tabular}
\end{center}
at the very top of the main \LaTeX{} file,
in particular \emph{before} the |\documentclass| statement!
The argument of |\childdocmain| should be left empty
(but it must be present).

%%%%%%%%%%%%%%%%%%%%%%%%%%%%%%%%%%%%%%%%
\DescribeMacro{\childdocof}
Furthermore, add the commands
\begin{center}
\begin{tabular}{l}
|% \iffalse
%
% childdoc.dtx Copyright (C) 2017-2018 Niklas Beisert
%
% This work may be distributed and/or modified under the
% conditions of the LaTeX Project Public License, either version 1.3
% of this license or (at your option) any later version.
% The latest version of this license is in
%   http://www.latex-project.org/lppl.txt
% and version 1.3 or later is part of all distributions of LaTeX
% version 2005/12/01 or later.
%
% This work has the LPPL maintenance status `maintained'.
%
% The Current Maintainer of this work is Niklas Beisert.
%
% This work consists of the files childdoc.dtx and childdoc.ins
% and the derived files childdoc.def and cdocsamp.tex with
% cdocsch1.tex, cdocsch2.tex, cdocsdrf.tex, cdocsfn1.tex, cdocsfn2.tex.
%
%<package>\ifdefined\childdocmain\endinput\fi
%<package>\ProvidesFile{childdoc.def}[2018/12/30 v2.0 child document driver]
%<samplemain>\ProvidesFile{cdocsamp.tex}[2018/12/30 v2.0 sample for childdoc]
%<*driver>
%\ProvidesFile{childdoc.drv}[2018/12/30 v2.0 childdoc reference manual file]
\PassOptionsToClass{10pt,a4paper}{article}
\documentclass{ltxdoc}

\usepackage[margin=35mm]{geometry}
\usepackage{hyperref}
\usepackage{hyperxmp}
\usepackage[usenames]{color}

\hypersetup{colorlinks=true}
\hypersetup{pdfstartview=FitH}
\hypersetup{pdfpagemode=UseNone}
\hypersetup{pdfsource={}}
\hypersetup{pdflang={en-UK}}
\hypersetup{pdfcopyright={Copyright 2017-2018 Niklas Beisert.
  This work may be distributed and/or modified under the
  conditions of the LaTeX Project Public License, either version 1.3
  of this license or (at your option) any later version.}}
\hypersetup{pdflicenseurl={http://www.latex-project.org/lppl.txt}}
\hypersetup{pdfcontactaddress={ETH Zurich, ITP, HIT K,
  Wolfgang-Pauli-Strasse 27}}
\hypersetup{pdfcontactpostcode={8093}}
\hypersetup{pdfcontactcity={Zurich}}
\hypersetup{pdfcontactcountry={Switzerland}}
\hypersetup{pdfcontactemail={nbeisert@itp.phys.ethz.ch}}
\hypersetup{pdfcontacturl={http://people.phys.ethz.ch/\xmptilde nbeisert/}}

\newcommand{\secref}[1]{\hyperref[#1]{section \ref*{#1}}}

\parskip1ex
\parindent0pt
\let\olditemize\itemize
\def\itemize{\olditemize\parskip0pt}

\begin{document}

\title{The \textsf{childdoc} Package}
\hypersetup{pdftitle={The childdoc Package}}
\author{Niklas Beisert\\[2ex]
  Institut f\"ur Theoretische Physik\\
  Eidgen\"ossische Technische Hochschule Z\"urich\\
  Wolfgang-Pauli-Strasse 27, 8093 Z\"urich, Switzerland\\[1ex]
  \href{mailto:nbeisert@itp.phys.ethz.ch}
  {\texttt{nbeisert@itp.phys.ethz.ch}}}
\hypersetup{pdfauthor={Niklas Beisert}}
\hypersetup{pdfsubject={Manual for the LaTeX2e Package childdoc}}
\date{30 December 2018, \textsf{v2.0}}
\maketitle

\begin{abstract}\noindent
\textsf{childdoc} is a \LaTeXe{} package
that enables the direct compilation
of document sections included by |\include|
to individual files.
\end{abstract}

\begingroup
\parskip0ex
\tableofcontents
\endgroup

%%%%%%%%%%%%%%%%%%%%%%%%%%%%%%%%%%%%%%%%%%%%%%%%%%%%%%%%%%%%%%%%%%%%%%%%%%%%%%%%
%%%%%%%%%%%%%%%%%%%%%%%%%%%%%%%%%%%%%%%%%%%%%%%%%%%%%%%%%%%%%%%%%%%%%%%%%%%%%%%%
\section{Introduction}

\LaTeX{} provides a mechanism to structure a large document (such as a book)
into a main file and several child files (containing the chapters)
using the |\include| command.
This mechanism is beneficial for documents
which span hundreds of pages in order to
make the source file(s) more manageable.
Moreover, compilation can be restricted to
selected child files by means of the |\includeonly| command.
The latter feature can be used to reduce the compilation time while editing
(this was significantly more useful in the earlier days of \LaTeX{})
or to generate a smaller document which is easier to navigate.
Another application of |\includeonly| is to generate
documents consisting of selected parts of the complete document.

However, there are a few drawbacks of the plain |\include| mechanism:
\begin{itemize}
\item
The child files cannot be compiled on their own,
they can only be compiled via the main file.
A naive editing environment
(such as a text editor with an option
to have the current file processed by \LaTeX)
may require one to switch to the main file before compiling;
attempting to compile the child file produces errors.
\item
The main file must be modified (each time)
to adjust the |\includeonly| command
to the present needs. This easily leaves the main file in a messy state.
\item
The generated document will always carry the filename
of the main document. This is inconvenient if
several child files are to be compiled and
to be kept for distribution.
\end{itemize}

The present package provides a simple interface
to make child files individually compilable by \LaTeX{}.
Compiling a child file then has the same effect as compiling
the main file with an |\includeonly| command
to select the appropriate child.
Moreover the generated document will carry the name of the child
rather than the main file.
This resolves all three above issues.

This feature is meant to make the editing of books,
thesis documents and lecture notes somewhat more convenient.
However, the package can also be used efficiently for
composing a series of documents (such as exercise sheets)
which are typically distributed individually.
It then assists the author in generating the individual documents
(potentially in different versions)
as well as a document containing the collected series.
Another application is in developing style files
or other kinds of included material
where compilation of the style file could redirect
to a sample or test file.

%%%%%%%%%%%%%%%%%%%%%%%%%%%%%%%%%%%%%%%%%%%%%%%%%%%%%%%%%%%%%%%%%%%%%%%%%%%%%%%%
%%%%%%%%%%%%%%%%%%%%%%%%%%%%%%%%%%%%%%%%%%%%%%%%%%%%%%%%%%%%%%%%%%%%%%%%%%%%%%%%
\section{Usage}

First of all, the package \textsf{childdoc} is \emph{not} a standard
\LaTeXe{} |.sty| style file! Therefore it needs to be invoked in
a non-standard way.

%%%%%%%%%%%%%%%%%%%%%%%%%%%%%%%%%%%%%%%%%%%%%%%%%%%%%%%%%%%%%%%%%%%%%%%%%%%%%%%%
\subsection{Included Files}
\label{sec:include}

%%%%%%%%%%%%%%%%%%%%%%%%%%%%%%%%%%%%%%%%
\DescribeMacro{\childdocmain}
To use the package, add the commands
\begin{center}
\begin{tabular}{l}
|\input{childdoc.def}|\\
|\childdocmain{}|\\
\end{tabular}
\end{center}
at the very top of the main \LaTeX{} file,
in particular \emph{before} the |\documentclass| statement!
The argument of |\childdocmain| should be left empty
(but it must be present).

%%%%%%%%%%%%%%%%%%%%%%%%%%%%%%%%%%%%%%%%
\DescribeMacro{\childdocof}
Furthermore, add the commands
\begin{center}
\begin{tabular}{l}
|\input{childdoc.def}|\\
|\childdocof{|\textit{main}|}|\\
\end{tabular}
\end{center}
at the top of every child file \textit{child}
which is included by |\include{|\textit{child}|}|
from within the main file
(or at least for those files to be compiled individually).
The argument \textit{main} must be the filename of the main file.

There are a couple of
considerations in setting up the main and child documents:

%%%%%%%%%%%%%%%%%%%%%%%%%%%%%%%%%%%%%%%%
\paragraph{Restrictions.}

Please note the following restrictions:
\begin{itemize}
\item
|\childdocmain| must be called with one argument \textit{main}
to ensure compatibility with earlier version of the package.
It must either be empty (|\childdocmain{}|)
or precisely match the filename of the main file in which it is specified.
See \secref{sec:detection} for further information.
\item
The filename \textit{main} must be specified without the |.tex| extension.
\item
The filename \textit{main} is case sensitive
(even in case-insensitive file systems)
due to internal string comparison.
\item
The argument \textit{main} should be fully expanded, it cannot be a macro.
\item
Subdirectories and special characters should be avoided in filenames.
\item
The command |\childdocmain{|\textit{main}|}| must be followed by a whitespace.
It should not be followed immediately by another command
or by a comment mark `|%|'.
This is because the \TeX{} parser reads the token immediately following
the argument of |\childdocmain| and puts it
at the beginning of every child section;
however, a white\-space is ignored.
\end{itemize}

%%%%%%%%%%%%%%%%%%%%%%%%%%%%%%%%%%%%%%%%
\paragraph{Content of Main File.}

It is advisable to place all content in the child files included by |\include|.
Any output contained in the main file will appear in all child documents
unless suppressed manually;
it cannot be suppressed automatically by the |\includeonly| directive
and thus should normally be avoided.
A method to include some content in the main file
by means of conditional processing is described in \secref{sec:conditional}.

%%%%%%%%%%%%%%%%%%%%%%%%%%%%%%%%%%%%%%%%
\paragraph{Page Numbering.}

When only a part of the document is compiled,
the appropriate numbering of pages
(as well as other status parameters)
is determined from the |.aux| files.
The latter contain information from previous passes.
However this information needs to propagate through
all intermediate child documents.
Therefore the page numbering in child documents may well
be inconsistent until the complete document is compiled at least once.

A useful (if unconventional) way to always ensure a consistent
page numbering is to restart the numbering in each child document
and denote the pages by `\textit{child}|.|\textit{page}'
where \textit{child} represents the chapter/section number of the child file.
This can be achieved by the command
|\numberwithin{page}{|\textit{child}|}|
of the \textsf{amsmath} package
where \textit{child} can be |chapter| or |section|
depending on the chosen structuring.
Alternatively, one can modify the macro |\thepage| appropriately
and reset the counter |page| at the start of each child file.

%%%%%%%%%%%%%%%%%%%%%%%%%%%%%%%%%%%%%%%%%%%%%%%%%%%%%%%%%%%%%%%%%%%%%%%%%%%%%%%%
\subsection{Conditional Processing}
\label{sec:conditional}

The package provides a mechanism to compile different versions
of a document. To customise the versions further some conditional processing
can come in handy to distinguish which version is being compiled.
The package provides two macros to describe the compilation context:

%%%%%%%%%%%%%%%%%%%%%%%%%%%%%%%%%%%%%%%%
\DescribeMacro{\ifchilddoc}
The conditional |\ifchilddoc| distinguishes between the compilation of
child documents and the main document:
%
\begin{center}
|\ifchilddoc |\textit{child-code}| |[|\||else |\textit{main-code}]| \||fi|
\end{center}

%%%%%%%%%%%%%%%%%%%%%%%%%%%%%%%%%%%%%%%%
\DescribeMacro{\childdocname}
\DescribeMacro{\childdocjob}
The macro |\childdocname| contains the filename (without extension)
of the main or child file being processed.
Note that |\childdocjob| will always contain the name of the main file.

%%%%%%%%%%%%%%%%%%%%%%%%%%%%%%%%%%%%%%%%
\paragraph{Title Page.}

Conditional processing can be used to include a title or banner page
in the main document when proper precautions are taken.
Importantly, the code in the main file should ensure that the page counter
(as well as other status parameters which are stored in the |.aux| files)
takes the same value after the conditional processing.
Otherwise the page numbers may take divergent values
depending on which part is compiled.

For example, a title page could be declared by:
%
\begin{center}
\begin{tabular}{l}
|\ifchilddoc\||else|\\
|\addtocounter{page}{-1}|\\
\textit{code for title page}\\
|\newpage|\\
|\||fi|
\end{tabular}
\end{center}
%
A banner page for the child documents can be generated by:
%
\begin{center}
\begin{tabular}{l}
|\ifchilddoc|\\
|\addtocounter{page}{-1}|\\
\textit{code for banner page}\\
|\newpage|\\
|\||fi|
\end{tabular}
\end{center}
%
Here one could write a message such as:
\begin{center}
|This is the part \childdocname{} of \childdocjob{}.|
\end{center}

%%%%%%%%%%%%%%%%%%%%%%%%%%%%%%%%%%%%%%%%%%%%%%%%%%%%%%%%%%%%%%%%%%%%%%%%%%%%%%%%
\subsection{Flags}
\label{sec:flags}

The package makes it easy to generate different versions
of the main or child documents.
To this end compilation flags can be defined
and assigned different default values.
They will be particularly useful in conjunction
with the forwarding mechanism described in \secref{sec:forward}.

For example, it may be useful to have a flag |\version|
which can be set to |draft| or |final|.
The document source will contain some conditional code
depending on the value of |\version|.
Suppose further, the flag should default to |final| for the main file
and to |draft| for child files
which is a natural assignment for editing the document.
This is achieved by placing the following code
in the preamble of the main document
(below the |\childdocmain| directive):
%
\begin{center}
\begin{tabular}{l}
|\ifchilddoc|\\
|\providecommand{\version}{draft}|\\
|\||else|\\
|\providecommand{\version}{final}|\\
|\||fi|
\end{tabular}
\end{center}
%
The definition by |\providecommand| makes sure
that previous definitions are not overwritten.
Further statements |\providecommand{\version}{...}|
can thus be added before the above code to override it.

For the main file, one might add a line
(between |\childdocmain| and the above block)
%
\begin{center}
|%\ifchilddoc\||else\providecommand{\version}{draft}\||fi|
\end{center}
%
which can be uncommented to produce a draft version.
Likewise one can add a line to the very top of a child file
(above the |\childdocof{|\textit{main}|}| directive)
%
\begin{center}
|%\providecommand{\version}{final}|
\end{center}
%
which can be uncommented to produce the final version of this child document.

%%%%%%%%%%%%%%%%%%%%%%%%%%%%%%%%%%%%%%%%%%%%%%%%%%%%%%%%%%%%%%%%%%%%%%%%%%%%%%%%
\subsection{Forwarding}
\label{sec:forward}

Different versions of the main or child documents
using compilation flags as described in \secref{sec:flags}
can be (permanently) stored in different files
for convenient compilation, viewing and distribution.
To this end, the package defines a command
to pass on compilation to a different file:

%%%%%%%%%%%%%%%%%%%%%%%%%%%%%%%%%%%%%%%%
\DescribeMacro{\childdocforward}
The command |\childdocforward| redirects processing to
another source file:
%
\begin{center}
\begin{tabular}{l}
|\input{childdoc.def}|\\
|\childdocforward[|\textit{main}|]{|\textit{dest}|}|\\
\end{tabular}
\end{center}
%
The argument \textit{dest} is the destination file
(without extension).
It should be the main file or one of the child files.
Note that further \textsf{childdoc} directives
such as |\childdocof| and |\childdocforward|
in the indicated file will be processed in this form.
The optional argument \textit{main}
passes on directly to the main file \textit{main}
while pretending to compile the child \textit{dest}.
This form behaves as if \textit{dest}
issues |\childdocof{|\textit{main}|}| right away,
and no further \textsf{childdoc} directives will be processed.

%%%%%%%%%%%%%%%%%%%%%%%%%%%%%%%%%%%%%%%%
\DescribeMacro{\...prefix}
In the alternative form |\childdocforwardprefix|,
%
\begin{center}
\begin{tabular}{l}
|\input{childdoc.def}|\\
|\childdocforwardprefix[|\textit{main}|]{|\textit{prefix}|}{|\textit{dest}|}|
\end{tabular}
\end{center}
%
the destination file is determined by a pattern
depending on the current file:
To make this work, the current file must be called
`{\textit{prefix}\hspace{0.2em}\textit{suffix}}'
with \textit{prefix} matching precisely the argument.
Processing is then passed on to the file
`{\textit{dest}\hspace{0.2em}\textit{suffix}}'.
Surely, the same effect is achieved by
directly specifying the
argument `{\textit{dest}\hspace{0.2em}\textit{suffix}}'
in the first form.
However, that requires to set up a different file
for each child. With the alternative form of the command
all these files can have exactly the same content
which simplifies setting them up and maintaining them.

For example, the following file |draft.tex|
with a compilation flag |\version| as described in \secref{sec:flags}
compiles the main document as a draft:
%
\begin{center}
\begin{tabular}{l}
|\def\version{draft}|\\
|\input{childdoc.def}|\\
|\childdocforward{|\textit{main}|}|
\end{tabular}
\end{center}
%
Likewise, the following files |final|\textit{nn}|.tex|
compile the final version of the child document
|child|\textit{nn}|.tex|:
%
\begin{center}
\begin{tabular}{l}
|\def\version{final}|\\
|\input{childdoc.def}|\\
|\childdocforwardprefix{final}{child}|
\end{tabular}
\end{center}
%

Note that when several versions of a main file and/or of each child file
are to be generated, it may be convenient to set up a |Makefile| or
shell script to automatise the process.

%%%%%%%%%%%%%%%%%%%%%%%%%%%%%%%%%%%%%%%%%%%%%%%%%%%%%%%%%%%%%%%%%%%%%%%%%%%%%%%%
\subsection{Command Line Processing}
\label{sec:commandline}

The effect of redirection files can also be achieved by invoking
the \LaTeX{} compiler with a more elaborate command line.
Most conveniently this should be done as part
of a shell script or a |Makefile|.

When using \textsf{childdoc} in the main file, the following
command lines effectively perform a redirection
(note that depending on the shell being used,
backslashes may have to be doubled: `|\|' $\to$ `|\\|'):
%
\begin{center}
|... -jobname "|\textit{target}|" |\\|"|[\textit{flags}]%
|\input{childdoc.def}\childdocforward[|\textit{main}|]{|\textit{dest}|}"|
\end{center}
%
Here \textit{target} is the name of the output file,
\textit{main} is the name of the main file
and \textit{dest} is the name of the main or child file to be processed
(all filenames without extensions).
The optional argument \textit{main} can be omitted
if \textit{main} matches \textit{dest}.
Optionally, compilation \textit{flags} can be defined via |\def| commands.
This command line makes the \TeX{} engine believe
it is compiling the file \textit{target}
whose content is specified as the latter parameter.
The provided code then forwards the processing to
\textit{main} or \textit{dest} as described in \secref{sec:forward}.

%%%%%%%%%%%%%%%%%%%%%%%%%%%%%%%%%%%%%%%%%%%%%%%%%%%%%%%%%%%%%%%%%%%%%%%%%%%%%%%%
\subsection{Include by Input}
\label{sec:input}

Including child documents by |\include| has some restrictions by design.
Most notably, the content of a child document always occupies
its own set of pages; pages cannot be shared between child documents.
Usually, this behaviour makes perfect sense
because each child document contain an essential part of the document.
However, in some situations it may be desirable to compose
a document from a collection of parts
without having mandatory page breaks between then.
For this case, the package
provides a mechanism to include parts
by |\input| which can also be processed individually.
However, by construction this mechanism
requires manual handling of the content to be output.

%%%%%%%%%%%%%%%%%%%%%%%%%%%%%%%%%%%%%%%%
\DescribeMacro{\ifchilddocmanual}
The main file should be prepared as usual, see \secref{sec:include}.
However, the document body must make a distinction
between processing of an individual part and of the main document, e.g.:
%
\begin{center}
\begin{tabular}{l}
|\ifchilddocmanual|\\
|\input{\childdocname}|\\
|\||else|\\
\textit{document body with }|\input{|\textit{part}|}|\\
|\||fi|
\end{tabular}
\end{center}
%
The conditional |\ifchilddocmanual| is true whenever
a part to be included by |\input| is being compiled,
and the name of the part is stored in |\childdocname|.

%%%%%%%%%%%%%%%%%%%%%%%%%%%%%%%%%%%%%%%%
\DescribeMacro{\childdocby}
Each part to be included by |\input| should start with:
%
\begin{center}
\begin{tabular}{l}
|\input{childdoc.def}|\\
|\childdocby{|\textit{main}|}|\\
\end{tabular}
\end{center}
%
The directive |\childdocby| is similar to |\childdocof|
described in \secref{sec:include},
but the subsequent selection of content must be done manually.
To that end, both |\ifchilddoc| and |\ifchilddocmanual|
will be true upon processing of a part,
and the name of the part is stored in |\childdocname|.
Note that |\jobname| will be set to the filename of the current part
so that each part receives an individual |.aux| file
that does not interfere with the |.aux| file(s) of the main document.
This behaviour can be altered by the alternative form
|\childdocby[*]{|\textit{main}|}| (with a non-empty optional argument)
which uses the |.aux| file of the main document
by setting |\jobname| to \textit{main}.

%%%%%%%%%%%%%%%%%%%%%%%%%%%%%%%%%%%%%%%%%%%%%%%%%%%%%%%%%%%%%%%%%%%%%%%%%%%%%%%%
\subsection{Driver Development}
\label{sec:driver}

The \textsf{childdoc} mechanism can also be use for the development
of definition files such as \LaTeX{} styles or classes.
This case differs from the above setup with multiple parts
included by |\include| in that no |\includeonly| should be invoked.
This can be achieved by starting the include file
(before |\ProvidesPackage|) with:
%
\begin{center}
\begin{tabular}{l}
|\input{childdoc.def}|\\
|\childdocforward{|\textit{main}|}|\\
\end{tabular}
\end{center}
%
or alternatively with:
%
\begin{center}
\begin{tabular}{l}
|\input{childdoc.def}|\\
|\childdocby{|\textit{main}|}|\\
\end{tabular}
\end{center}
%
Both forms have slightly different effects as described above.
The main file is prepared as usual, see \secref{sec:include}.

%%%%%%%%%%%%%%%%%%%%%%%%%%%%%%%%%%%%%%%%%%%%%%%%%%%%%%%%%%%%%%%%%%%%%%%%%%%%%%%%
\subsection{Legacy Detection}
\label{sec:detection}

The directive |\childdocmain| in the main file can detect
whether the complete document or merely a child is to be compiled
even without using the directive |\childdocof|.
This method is deprecated because it is less robust
and there is no compelling reason to use it;
it is merely provided for backward compatibility
and it may be removed in future versions.

If the detection mechanism is to be used,
it is mandatory to correctly specify
the filename of the main file as the argument of |\childdocmain|:
%
\begin{center}
\begin{tabular}{l}
|\input{childdoc.def}|\\
|\childdocmain{|\textit{main}|}|\\
\end{tabular}
\end{center}
%
If |\jobname| does not match the argument \textit{main} of |\childdocmain|,
it is assumed that |\jobname| points to the child file to be compiled.
When using |\childdocmain| with the main file specified as argument,
it suffices to start a child file
with just |\input{|\textit{main}|}|
without loading of the package and using |\childdocof|.
If instead all processing is done
with the appropriate \textsf{childdoc} directives,
the argument of \textit{main} of |\childdocmain| can be empty.

An alternative version of the command line processing described
in \secref{sec:commandline} using the detection mechanism reads:
%
\begin{center}
|... -jobname "|\textit{target}|" "|[\textit{flags}]%
[|\def\jobname{|\textit{dest}|}|]|\input{|\textit{main}|}"|
\end{center}

%%%%%%%%%%%%%%%%%%%%%%%%%%%%%%%%%%%%%%%%%%%%%%%%%%%%%%%%%%%%%%%%%%%%%%%%%%%%%%%%
\subsection{Manual Code}
\label{sec:manual}

In case one cannot be certain whether the definitions file |childdoc.def|
is installed on the target \TeX{} distribution
and one prefers not to ship it,
it is conceivable to paste a few relevant commands into the sources.

To that end, drop all statements |\input{childdoc.def}|
and perform the replacements as outlined below.
Instead of |\childdocmain{|\textit{main}|}| add the following code
to the top of the main file:
%
\begin{center}
\begin{tabular}{l}
|\||ifdefined\childdocname\endinput\||fi\newif\ifchilddoc|\\
|\edef\childdocname{\scantokens\expandafter{\jobname\noexpand}}|\\
|\def\childdocmain{|\textit{main}|}\||ifx\childdocmain\childdocname\||else|\\
|\childdoctrue\includeonly{\childdocname}\let\jobname\childdocmain\||fi|\\
\end{tabular}
\end{center}
%
Instead of |\childdocof{|\textit{main}|}| just include the main file
at the top of each child file:
%
\begin{center}
|\input{|\textit{main}|}|
\end{center}
%
A simple redirection |\childdocforward{|\textit{dest}|}| is achieved by:
%
\begin{center}
|\def\jobname{|\textit{dest}|}\input{\jobname}|
\end{center}
%
The redirection with prefix
|\childdocforwardprefix[|\textit{prefix}|]{|\textit{dest}|}|
is accomplished by:
%
\begin{center}
\begin{tabular}{l}
|{\edef\jobname{\scantokens\expandafter{\jobname\noexpand}}|\\
|\def\redirectjob |\textit{prefix}|#1~~~{\gdef\jobname{|\textit{dest}|#1}}|\\
|\expandafter\redirectjob\jobname~~~}\input{\jobname}|
\end{tabular}
\end{center}

In an alternative approach,
child documents can be compiled by a specific command line
without additional code or specific definitions:
%
\begin{center}
|... -jobname "|\textit{target}|" "|[\textit{flags}]%
|\includeonly{|\textit{dest}|}\input{|\textit{main}|}"|
\end{center}
%

%%%%%%%%%%%%%%%%%%%%%%%%%%%%%%%%%%%%%%%%%%%%%%%%%%%%%%%%%%%%%%%%%%%%%%%%%%%%%%%%
%%%%%%%%%%%%%%%%%%%%%%%%%%%%%%%%%%%%%%%%%%%%%%%%%%%%%%%%%%%%%%%%%%%%%%%%%%%%%%%%
\section{Information}

%%%%%%%%%%%%%%%%%%%%%%%%%%%%%%%%%%%%%%%%%%%%%%%%%%%%%%%%%%%%%%%%%%%%%%%%%%%%%%%%
\subsection{Copyright}

Copyright \copyright{} 2017--2018 Niklas Beisert

This work may be distributed and/or modified under the
conditions of the \LaTeX{} Project Public License, either version 1.3
of this license or (at your option) any later version.
The latest version of this license is in
  \url{http://www.latex-project.org/lppl.txt}
and version 1.3 or later is part of all distributions of \LaTeX{}
version 2005/12/01 or later.

This work has the LPPL maintenance status `maintained'.

The Current Maintainer of this work is Niklas Beisert.

This work consists of the files |README.txt|, |childdoc.ins| and |childdoc.dtx|
as well as the derived files |childdoc.def|, |cdocsamp.tex|
with |cdocsch1.tex|, |cdocsch2.tex|, |cdocspt3.tex|, |cdocspt4.tex|,
|cdocsdrf.tex|, |cdocsfn1.tex|, |cdocsfn2.tex|
as well as |childdoc.pdf|.

%%%%%%%%%%%%%%%%%%%%%%%%%%%%%%%%%%%%%%%%%%%%%%%%%%%%%%%%%%%%%%%%%%%%%%%%%%%%%%%%
\subsection{Files and Installation}

The package consists of the files:
%
\begin{center}
\begin{tabular}{ll}
    |README.txt|   & readme file \\
    |childdoc.ins| & installation file \\
    |childdoc.dtx| & source file \\
    |childdoc.def| & definition file \\
    |cdocsamp.tex| & sample main file \\
    |cdocsch1.tex| & sample include file \\
    |cdocsch2.tex| & sample include file \\
    |cdocspt3.tex| & sample part file \\
    |cdocspt4.tex| & sample part file \\
    |cdocsdrf.tex| & sample redirection file \\
    |cdocsfn1.tex| & sample redirection file \\
    |cdocsfn2.tex| & sample redirection file \\
    |childdoc.pdf| & manual
\end{tabular}
\end{center}
%
The distribution consists of the files
|README.txt|, |childdoc.ins| and |childdoc.dtx|.
%
\begin{itemize}
\item
Run (pdf)\LaTeX{} on |childdoc.dtx|
to compile the manual |childdoc.pdf| (this file).
\item
Run \LaTeX{} on |childdoc.ins| to create the definitions file |childdoc.def|
and the sample |cdocsamp.tex| with include files
|cdocsch1.tex|, |cdocsch2.tex|, |cdocspt3.tex|, |cdocspt4.tex|,
|cdocsdrf.tex|, |cdocsfn1.tex|, |cdocsfn2.tex|.
Then copy the file |childdoc.def| to an appropriate directory of your \LaTeX{}
distribution, e.g.\ \textit{texmf-root}|/tex/latex/childdoc|.
\end{itemize}

%%%%%%%%%%%%%%%%%%%%%%%%%%%%%%%%%%%%%%%%%%%%%%%%%%%%%%%%%%%%%%%%%%%%%%%%%%%%%%%%
\subsection{Related CTAN Packages}

There are several other packages which offer a similar functionality:
%
\begin{itemize}
\item
The packages
\href{http://ctan.org/pkg/docmute}{\textsf{docmute}},
\href{http://ctan.org/pkg/includex}{\textsf{includex}} and
\href{http://ctan.org/pkg/standalone}{\textsf{standalone}}
provide commands to include only the document body of
a child file thus allowing both files to be compiled individually.
\item
The packages \href{http://ctan.org/pkg/subdocs}{\textsf{subdocs}}
and \href{http://ctan.org/pkg/subfiles}{\textsf{subfiles}}
provide structures in which the main and child documents can be
encapsulated and allowing them to be compiled individually.
The inclusion mechanism is different from the conventional |\include|.
\item
The package \href{http://ctan.org/pkg/combine}{\textsf{combine}}
is an elaborate solution to combine several documents into one.
\end{itemize}
%
See also the CTAN topic \href{http://ctan.org/topic/subdocs}{\textsf{subdocs}}
for further related packages.
The present package differs from the above solutions in that
a document structure constructed with the conventional |\include| mechanism
just needs two extra commands at the top of every file
such that all constituent files can be compiled individually.

%%%%%%%%%%%%%%%%%%%%%%%%%%%%%%%%%%%%%%%%%%%%%%%%%%%%%%%%%%%%%%%%%%%%%%%%%%%%%%%%
%\subsection{Feature Suggestions}
%
%The following is a list of features which may be useful for future
%versions of this package:
%%
%\begin{itemize}
%\item
%\ldots
%\end{itemize}

%%%%%%%%%%%%%%%%%%%%%%%%%%%%%%%%%%%%%%%%%%%%%%%%%%%%%%%%%%%%%%%%%%%%%%%%%%%%%%%%
\subsection{Revision History}

%%%%%%%%%%%%%%%%%%%%%%%%%%%%%%%%%%%%%%%%
\paragraph{v2.0:} 2018/12/30

\begin{itemize}
\item
immediate forward processing
\item
added |\childdocby| mechanism
\item
manual restructured
\end{itemize}

%%%%%%%%%%%%%%%%%%%%%%%%%%%%%%%%%%%%%%%%
\paragraph{v1.6:} 2018/01/17

\begin{itemize}
\item
application for development of include files
\item
corrections to manual
\end{itemize}

%%%%%%%%%%%%%%%%%%%%%%%%%%%%%%%%%%%%%%%%
\paragraph{v1.5:} 2017/05/21

\begin{itemize}
\item
more complete structuring introduced
\item
|\childdocof| introduced
\item
|\childdoc| renamed to |\childdocmain|
\item
|\childredirect| renamed to |\childdocforward| and |\childdocforwardprefix|
and functionality expanded
\end{itemize}

%%%%%%%%%%%%%%%%%%%%%%%%%%%%%%%%%%%%%%%%
\paragraph{v1.0:} 2017/04/27

\begin{itemize}
\item
manual and install package
\item
first version published on CTAN
\end{itemize}

%%%%%%%%%%%%%%%%%%%%%%%%%%%%%%%%%%%%%%%%
\paragraph{v0.6:} 2017/04/26

\begin{itemize}
\item
redirection mechanism added
\end{itemize}

%%%%%%%%%%%%%%%%%%%%%%%%%%%%%%%%%%%%%%%%
\paragraph{v0.5:} 2017/04/26

\begin{itemize}
\item
functionality in definition file
\end{itemize}


%%%%%%%%%%%%%%%%%%%%%%%%%%%%%%%%%%%%%%%%%%%%%%%%%%%%%%%%%%%%%%%%%%%%%%%%%%%%%%%%
%%%%%%%%%%%%%%%%%%%%%%%%%%%%%%%%%%%%%%%%%%%%%%%%%%%%%%%%%%%%%%%%%%%%%%%%%%%%%%%%
%%%%%%%%%%%%%%%%%%%%%%%%%%%%%%%%%%%%%%%%%%%%%%%%%%%%%%%%%%%%%%%%%%%%%%%%%%%%%%%%
\appendix

\settowidth\MacroIndent{\rmfamily\scriptsize 000\ }

 \DocInput{childdoc.dtx}

\end{document}
%</driver>
% \fi
%
% %%%%%%%%%%%%%%%%%%%%%%%%%%%%%%%%%%%%%%%%%%%%%%%%%%%%%%%%%%%%%%%%%%%%%%%%%%%%%%
% %%%%%%%%%%%%%%%%%%%%%%%%%%%%%%%%%%%%%%%%%%%%%%%%%%%%%%%%%%%%%%%%%%%%%%%%%%%%%%
% \section{Sample}
%\iffalse
%<*samplemain>
%\fi
%
% The following presents a sample document
% with two chapters, two parts, a title page,
% a compile flag as well as three forwarding files to set the flag.
% It consists of eight |.tex| files:
% \begin{center}
% \begin{tabular}{ll}
% |cdocsamp.tex|&main file\\
% |cdocsch1.tex|&include file for chapter 1\\
% |cdocsch2.tex|&include file for chapter 2\\
% |cdocspt3.tex|&include file for part 3\\
% |cdocspt4.tex|&include file for part 4\\
% |cdocsdrf.tex|&forwarding file for main file in draft mode\\
% |cdocsfi1.tex|&forwarding file for final version of chapter 1\\
% |cdocsfi2.tex|&forwarding file for final version of chapter 2\\
% \end{tabular}
% \end{center}
% Each of the eight files can be compiled directly by the \LaTeX{} compiler.
%
% %%%%%%%%%%%%%%%%%%%%%%%%%%%%%%%%%%%%%%
% \paragraph{Main File.}
%
% The main file is called |cdocsamp.tex|.
%
% Load the \textsf{childdoc} definitions and
% declare the filename for the main document:
%    \begin{macrocode}
\input{childdoc.def}
\childdocmain{}
%    \end{macrocode}

% Optional override for |\version| flag:
%    \begin{macrocode}
%%\ifchilddoc\else\providecommand{\version}{draft}\fi
%    \end{macrocode}

% Define the default values for the |\version| flag
% (|final| for the main file and |draft| for childs):
%    \begin{macrocode}
\ifchilddoc
\providecommand{\version}{draft}
\else
\providecommand{\version}{final}
\fi
%    \end{macrocode}

% Load the standard document class:
%    \begin{macrocode}
\documentclass[12pt]{article}
%    \end{macrocode}

% Start the document body:
%    \begin{macrocode}
\begin{document}
%    \end{macrocode}

% Declare a title page.
% Print title, part of document being processed and version flag:
%    \begin{macrocode}
\addtocounter{page}{-1}
\begin{center}
{\LARGE\bfseries{}childdoc example\par}
\vspace{1cm}
\ifchilddoc
\ifchilddocmanual part\else chapter\fi:
`\childdocname' of `\childdocjob'\par
\else
main document: `\childdocjob'\par
\fi
version: \version\par
\end{center}
\newpage
%    \end{macrocode}

% Manually include selected file,
% otherwise process as usual:
%    \begin{macrocode}
\ifchilddocmanual
\section*{part `\childdocname'}
\input{\childdocname}
\else
%    \end{macrocode}

% Include the two chapters:
%    \begin{macrocode}
\include{cdocsch1}
\include{cdocsch2}
%    \end{macrocode}

% Include the two parts unless only chapters should be displayed:
%    \begin{macrocode}
\ifchilddoc\else
\section{part three}
\input{cdocspt3}
\section{part four}
\input{cdocspt4}
\fi
%    \end{macrocode}

% Process as usual until here:
%    \begin{macrocode}
\fi
%    \end{macrocode}

% End of document body:
%    \begin{macrocode}
\end{document}
%    \end{macrocode}
%\iffalse
%</samplemain>
%\fi
%
% %%%%%%%%%%%%%%%%%%%%%%%%%%%%%%%%%%%%%%
% \paragraph{Chapter Include Files.}
%
% The include files are called |cdocsch1.tex| and |cdocsch2.tex|.
%
%\iffalse
%<*samplechap1|samplechap2>
%\fi

% Optional override for |\version| flag:
%    \begin{macrocode}
%%\providecommand{\version}{final}
%    \end{macrocode}

% Include the main document:
%    \begin{macrocode}
\input{childdoc.def}
\childdocof{cdocsamp}
%    \end{macrocode}

%\iffalse
%</samplechap1|samplechap2>
%\fi
%
%\iffalse
%<*samplechap1>
%\fi
% Some text for chapter 1:
%    \begin{macrocode}
\section{one}
some text in chapter one
%    \end{macrocode}

%\iffalse
%</samplechap1>
%\fi
% Some text for chapter 2:
%\iffalse
%<*samplechap2>
%\fi
%    \begin{macrocode}
\section{two}
more text in chapter two
%    \end{macrocode}

%\iffalse
%</samplechap2>
%\fi
%
% %%%%%%%%%%%%%%%%%%%%%%%%%%%%%%%%%%%%%%
% \paragraph{Part Include Files.}
%
% The include files are called |cdocspt3.tex| and |cdocspt4.tex|.
%
%\iffalse
%<*samplepart3|samplepart4>
%\fi

% Optional override for |\version| flag:
%    \begin{macrocode}
%%\providecommand{\version}{final}
%    \end{macrocode}

% Include the main document:
%    \begin{macrocode}
\input{childdoc.def}
\childdocby{cdocsamp}
%    \end{macrocode}

%\iffalse
%</samplepart3|samplepart4>
%\fi
%
%\iffalse
%<*samplepart3>
%\fi
% Some text for part 3:
%    \begin{macrocode}
some text in part three
%    \end{macrocode}

%\iffalse
%</samplepart3>
%\fi
% Some text for part 4:
%\iffalse
%<*samplepart4>
%\fi
%    \begin{macrocode}
more text in part four
%    \end{macrocode}

%\iffalse
%</samplepart4>
%\fi
%
% %%%%%%%%%%%%%%%%%%%%%%%%%%%%%%%%%%%%%%
% \paragraph{Forwarding for a Complete Draft.}
%
% The following forwarding file |cdocsdrf.tex|
% compiles the main document in draft mode:
%\iffalse
%<*sampledraft>
%\fi
%    \begin{macrocode}
\def\version{draft}
\input{childdoc.def}
\childdocforward{cdocsamp}
%    \end{macrocode}

%\iffalse
%</sampledraft>
%\fi
%
% %%%%%%%%%%%%%%%%%%%%%%%%%%%%%%%%%%%%%%
% \paragraph{Forwarding for Final Version of the Chapters.}
%
% The following forwarding files |cdocsfn1.tex| and |cdocsfn2.tex|
% (with identical content)
% compile the final versions of the child documents
% |cdocsch1.tex| and |cdocsch2.tex|, respectively:
%\iffalse
%<*samplefinal>
%\fi
%    \begin{macrocode}
\def\version{final}
\input{childdoc.def}
\childdocforwardprefix[cdocsamp]{cdocsfn}{cdocsch}
%    \end{macrocode}

%\iffalse
%</samplefinal>
%\fi
%
% %%%%%%%%%%%%%%%%%%%%%%%%%%%%%%%%%%%%%%
% \paragraph{Command Line Processing.}
%
% The following three command lines generate the output files
% |cdocscld|, |cdocscl1| and |cdocscl2|
% which should be identical to
% |cdocsdrf|, |cdocsch1| and |cdocsfn2|, respectively:
% \begin{center}
% \begin{tabular}{l}
% |latex -jobname cdocscld \|\\
% |  "\def\version{draft}\input{childdoc.def}\childdocforward{cdocsamp}"|\\
% |latex -jobname cdocscl1 \|\\
% |  "\input{childdoc.def}\childdocforward[cdocsamp]{cdocsch1}"|\\
% |latex -jobname cdocscl2 \|\\
% |  "\def\version{final}\input{childdoc.def}\childdocforward{cdocsch2}"|
% \end{tabular}
% \end{center}
% Note that the trailing backslash on each first line
% merely continues the input to the second line
% (for convenient cut ant paste).
% Furthermore, the command |latex| can be replaced by any
% of its alternative versions such as |pdflatex|.
%
% %%%%%%%%%%%%%%%%%%%%%%%%%%%%%%%%%%%%%%%%%%%%%%%%%%%%%%%%%%%%%%%%%%%%%%%%%%%%%%
% %%%%%%%%%%%%%%%%%%%%%%%%%%%%%%%%%%%%%%%%%%%%%%%%%%%%%%%%%%%%%%%%%%%%%%%%%%%%%%
% \section{Implementation}
%\iffalse
%<*package>
%\fi
%
% This section describes the definitions file |childdoc.def|.

% The definitions cannot be loaded using |\usepackage| or |\RequirePackage|
% which has a mechanism to prevent loading a style file more than once.
% When loading the definitions by means of |\input|
% multiple instances have to be prevented manually:
%\iffalse
%This code needs to be before the `\ProvidesFile' directive
%which is defined at the beginning of this file.
%Therefore it is also placed there and commented out here.
%</package>
%<*discard>
%\fi
%    \begin{macrocode}
\ifdefined\childdocmain\endinput\fi
%    \end{macrocode}
%\iffalse
%</discard>
%<*package>
%\fi
%
% \macro{\ifchilddoc}
% \macro{\ifchilddocmanual}
% The conditional |\ifchilddoc| tells whether a
% child (true) or main (false) document is being compiled.
% The conditional |\ifchilddocmanual| tells whether
% the |\includeonly| mechanism is used (false) or
% the selection of child files must be performed manually (true).
% The definitions initialise to false:
%    \begin{macrocode}
\newif\ifchilddoc
\newif\ifchilddocmanual
%    \end{macrocode}

% \macro{\childdocname}
% \macro{\childdocjob}
% The macro |\childdocname| stores the name of the main document
% to be compiled. The macro |\childdocjob| stores the name of
% the document on which the \LaTeX{} compiler was originally invoked.
% The content of |\jobname| cannot be compared
% to filenames specified in the source due to different catcodes.
% The following code rescans |\jobname|, stores the result
% in |\childdocname| and saves a copy in |\childdocjob|:
%    \begin{macrocode}
\edef\childdocname{\scantokens\expandafter{\jobname\noexpand}}
\let\childdocjob\childdocname
%    \end{macrocode}

% \macro{\childdocdisable}
% The macro |\childdocdisable| prevents the main file
% from being processed more than once.
% At this stage, the main document command |\childdocmain|
% is assumed to be called once again where it should do nothing.
% Any subsequent call to it should prevent
% a secondary processing of the main document
% It overwrites the forwarding commands
% |\childdocof| and |\childdocforward|
% with empty macros to prevent further inclusions of the main document:
%    \begin{macrocode}
\newcommand{\childdocdisable}
{
  \renewcommand{\childdocmain}[1]{\renewcommand{\childdocmain}[1]{\endinput}}
  \renewcommand{\childdocof}[1]{}
  \renewcommand{\childdocby}[2][]{}
  \renewcommand{\childdocforward}[2][]{}
  \renewcommand{\childdocdisable}{}
}
%    \end{macrocode}

% \macro{\childdocmain}
% The macro |\childdocmain| is to be called at the top of the main file
% with nothing or the main filename (without extension) as argument.
% First, it breaks loops.
% If the argument is not empty and does not match |\childdocname|
% (which is set by the first inclusion of |childdoc.def|),
% |\ifchilddoc| is set to true, |\includeonly| is applied to the child file
% and |\jobname| is set to the main file
% (for proper handling of |.aux| files):
%    \begin{macrocode}
\newcommand{\childdocmain}[1]
{
  \childdocdisable\childdocmain{}
  \if?#1?\else
    \begingroup
      \def\childdoctmp{#1}
      \ifx\childdoctmp\childdocname
        \def\childdoctmp{}
      \else
        \def\childdoctmp
        {
          \childdoctrue
          \includeonly{\childdocname}
          \def\childdocjob{#1}
          \def\jobname{#1}
        }
      \fi
      \expandafter
    \endgroup
    \childdoctmp
  \fi
}
%    \end{macrocode}

% \macro{\childdocof}
% The command |\childdocof| redirects
% compilation to the main file |#1|.
%    \begin{macrocode}
\newcommand{\childdocof}[1]
{
  \childdocdisable
  \childdoctrue
  \includeonly{\childdocname}
  \def\jobname{#1}
  \def\childdocjob{#1}
  \input{#1}
}
%    \end{macrocode}

% \macro{\childdocby}
% The command |\childdocby| ....
%    \begin{macrocode}
\newcommand{\childdocby}[2][]
{
  \childdocdisable
  \childdoctrue
  \childdocmanualtrue
  \if?#1?\else
    \def\jobname{#2}
  \fi
  \def\childdocjob{#2}
  \input{#2}
  \endinput
}
%    \end{macrocode}

% \macro{\childdocforward}
% The command |\childdocforward| redirects
% compilation to the main file or
% (if the optional argument is given) a child file.
% Parameters are set as if the main file
% or a child file starting with |\childdocof| was compiled.
% Then compilation is handed over to the main file:
%    \begin{macrocode}
\newcommand{\childdocforward}[2][]
{
  \begingroup
    \if?#1?
      \def\childdoctmp
      {
        \def\childdocname{#2}
        \def\childdocjob{#2}
        \def\jobname{#2}
        \input{#2}
        \endinput
      }
    \else
      \def\childdoctmp
      {
        \childdocdisable
        \def\childdocname{#2}
        \childdoctrue
        \includeonly{#2}
        \def\childdocjob{#1}
        \def\jobname{#1}
        \input{#1}
        \endinput
      }
    \fi
    \expandafter
  \endgroup
  \childdoctmp
}
%    \end{macrocode}

% \macro{\childdocforwardprefix}
% The command |\childdocforwardprefix| redirects
% compilation to the main or a child file by means of a pattern.
% The prefix |#1| in the current filename is replaced by |#2|
% and the suffix of the current filename is kept
% (it is assumed that the filename does not contain the substring `|~~~|'
% which is used as a delimiter).
% Compilation is handed over to the new file by |\childdocforward|:
%    \begin{macrocode}
\newcommand{\childdocforwardprefix}[3][]
{
  \begingroup
    \def\childdocextract #2##1~~~{\def\childdoctmp{\childdocforward[#1]{#3##1}}}
    \expandafter\childdocextract\childdocname~~~
    \expandafter
  \endgroup
  \childdoctmp
}
%    \end{macrocode}

% \macro{\childdoc}
% The deprecated macro |\childdoc| is a legacy version of |\childdocmain|:
%    \begin{macrocode}
\newcommand{\childdoc}{\childdocmain}
%    \end{macrocode}

% \macro{\childdocredirect}
% The deprecated macro |\childdocredirect| is a legacy version
% of |\childdocforward| and |\childdocforwardprefix|:
%    \begin{macrocode}
\newcommand{\childdocredirect}[2][]
{
  \begingroup
    \if?#1?
      \def\childdoctmp{\childdocforward{#2}}
    \else
      \def\childdoctmp{\childdocforwardprefix{#1}{#2}}
    \fi
    \expandafter
  \endgroup
  \childdoctmp
}
%    \end{macrocode}

%\iffalse
%</package>
%\fi
%
\endinput
|\\
|\childdocof{|\textit{main}|}|\\
\end{tabular}
\end{center}
at the top of every child file \textit{child}
which is included by |\include{|\textit{child}|}|
from within the main file
(or at least for those files to be compiled individually).
The argument \textit{main} must be the filename of the main file.

There are a couple of
considerations in setting up the main and child documents:

%%%%%%%%%%%%%%%%%%%%%%%%%%%%%%%%%%%%%%%%
\paragraph{Restrictions.}

Please note the following restrictions:
\begin{itemize}
\item
|\childdocmain| must be called with one argument \textit{main}
to ensure compatibility with earlier version of the package.
It must either be empty (|\childdocmain{}|)
or precisely match the filename of the main file in which it is specified.
See \secref{sec:detection} for further information.
\item
The filename \textit{main} must be specified without the |.tex| extension.
\item
The filename \textit{main} is case sensitive
(even in case-insensitive file systems)
due to internal string comparison.
\item
The argument \textit{main} should be fully expanded, it cannot be a macro.
\item
Subdirectories and special characters should be avoided in filenames.
\item
The command |\childdocmain{|\textit{main}|}| must be followed by a whitespace.
It should not be followed immediately by another command
or by a comment mark `|%|'.
This is because the \TeX{} parser reads the token immediately following
the argument of |\childdocmain| and puts it
at the beginning of every child section;
however, a white\-space is ignored.
\end{itemize}

%%%%%%%%%%%%%%%%%%%%%%%%%%%%%%%%%%%%%%%%
\paragraph{Content of Main File.}

It is advisable to place all content in the child files included by |\include|.
Any output contained in the main file will appear in all child documents
unless suppressed manually;
it cannot be suppressed automatically by the |\includeonly| directive
and thus should normally be avoided.
A method to include some content in the main file
by means of conditional processing is described in \secref{sec:conditional}.

%%%%%%%%%%%%%%%%%%%%%%%%%%%%%%%%%%%%%%%%
\paragraph{Page Numbering.}

When only a part of the document is compiled,
the appropriate numbering of pages
(as well as other status parameters)
is determined from the |.aux| files.
The latter contain information from previous passes.
However this information needs to propagate through
all intermediate child documents.
Therefore the page numbering in child documents may well
be inconsistent until the complete document is compiled at least once.

A useful (if unconventional) way to always ensure a consistent
page numbering is to restart the numbering in each child document
and denote the pages by `\textit{child}|.|\textit{page}'
where \textit{child} represents the chapter/section number of the child file.
This can be achieved by the command
|\numberwithin{page}{|\textit{child}|}|
of the \textsf{amsmath} package
where \textit{child} can be |chapter| or |section|
depending on the chosen structuring.
Alternatively, one can modify the macro |\thepage| appropriately
and reset the counter |page| at the start of each child file.

%%%%%%%%%%%%%%%%%%%%%%%%%%%%%%%%%%%%%%%%%%%%%%%%%%%%%%%%%%%%%%%%%%%%%%%%%%%%%%%%
\subsection{Conditional Processing}
\label{sec:conditional}

The package provides a mechanism to compile different versions
of a document. To customise the versions further some conditional processing
can come in handy to distinguish which version is being compiled.
The package provides two macros to describe the compilation context:

%%%%%%%%%%%%%%%%%%%%%%%%%%%%%%%%%%%%%%%%
\DescribeMacro{\ifchilddoc}
The conditional |\ifchilddoc| distinguishes between the compilation of
child documents and the main document:
%
\begin{center}
|\ifchilddoc |\textit{child-code}| |[|\||else |\textit{main-code}]| \||fi|
\end{center}

%%%%%%%%%%%%%%%%%%%%%%%%%%%%%%%%%%%%%%%%
\DescribeMacro{\childdocname}
\DescribeMacro{\childdocjob}
The macro |\childdocname| contains the filename (without extension)
of the main or child file being processed.
Note that |\childdocjob| will always contain the name of the main file.

%%%%%%%%%%%%%%%%%%%%%%%%%%%%%%%%%%%%%%%%
\paragraph{Title Page.}

Conditional processing can be used to include a title or banner page
in the main document when proper precautions are taken.
Importantly, the code in the main file should ensure that the page counter
(as well as other status parameters which are stored in the |.aux| files)
takes the same value after the conditional processing.
Otherwise the page numbers may take divergent values
depending on which part is compiled.

For example, a title page could be declared by:
%
\begin{center}
\begin{tabular}{l}
|\ifchilddoc\||else|\\
|\addtocounter{page}{-1}|\\
\textit{code for title page}\\
|\newpage|\\
|\||fi|
\end{tabular}
\end{center}
%
A banner page for the child documents can be generated by:
%
\begin{center}
\begin{tabular}{l}
|\ifchilddoc|\\
|\addtocounter{page}{-1}|\\
\textit{code for banner page}\\
|\newpage|\\
|\||fi|
\end{tabular}
\end{center}
%
Here one could write a message such as:
\begin{center}
|This is the part \childdocname{} of \childdocjob{}.|
\end{center}

%%%%%%%%%%%%%%%%%%%%%%%%%%%%%%%%%%%%%%%%%%%%%%%%%%%%%%%%%%%%%%%%%%%%%%%%%%%%%%%%
\subsection{Flags}
\label{sec:flags}

The package makes it easy to generate different versions
of the main or child documents.
To this end compilation flags can be defined
and assigned different default values.
They will be particularly useful in conjunction
with the forwarding mechanism described in \secref{sec:forward}.

For example, it may be useful to have a flag |\version|
which can be set to |draft| or |final|.
The document source will contain some conditional code
depending on the value of |\version|.
Suppose further, the flag should default to |final| for the main file
and to |draft| for child files
which is a natural assignment for editing the document.
This is achieved by placing the following code
in the preamble of the main document
(below the |\childdocmain| directive):
%
\begin{center}
\begin{tabular}{l}
|\ifchilddoc|\\
|\providecommand{\version}{draft}|\\
|\||else|\\
|\providecommand{\version}{final}|\\
|\||fi|
\end{tabular}
\end{center}
%
The definition by |\providecommand| makes sure
that previous definitions are not overwritten.
Further statements |\providecommand{\version}{...}|
can thus be added before the above code to override it.

For the main file, one might add a line
(between |\childdocmain| and the above block)
%
\begin{center}
|%\ifchilddoc\||else\providecommand{\version}{draft}\||fi|
\end{center}
%
which can be uncommented to produce a draft version.
Likewise one can add a line to the very top of a child file
(above the |\childdocof{|\textit{main}|}| directive)
%
\begin{center}
|%\providecommand{\version}{final}|
\end{center}
%
which can be uncommented to produce the final version of this child document.

%%%%%%%%%%%%%%%%%%%%%%%%%%%%%%%%%%%%%%%%%%%%%%%%%%%%%%%%%%%%%%%%%%%%%%%%%%%%%%%%
\subsection{Forwarding}
\label{sec:forward}

Different versions of the main or child documents
using compilation flags as described in \secref{sec:flags}
can be (permanently) stored in different files
for convenient compilation, viewing and distribution.
To this end, the package defines a command
to pass on compilation to a different file:

%%%%%%%%%%%%%%%%%%%%%%%%%%%%%%%%%%%%%%%%
\DescribeMacro{\childdocforward}
The command |\childdocforward| redirects processing to
another source file:
%
\begin{center}
\begin{tabular}{l}
|% \iffalse
%
% childdoc.dtx Copyright (C) 2017-2018 Niklas Beisert
%
% This work may be distributed and/or modified under the
% conditions of the LaTeX Project Public License, either version 1.3
% of this license or (at your option) any later version.
% The latest version of this license is in
%   http://www.latex-project.org/lppl.txt
% and version 1.3 or later is part of all distributions of LaTeX
% version 2005/12/01 or later.
%
% This work has the LPPL maintenance status `maintained'.
%
% The Current Maintainer of this work is Niklas Beisert.
%
% This work consists of the files childdoc.dtx and childdoc.ins
% and the derived files childdoc.def and cdocsamp.tex with
% cdocsch1.tex, cdocsch2.tex, cdocsdrf.tex, cdocsfn1.tex, cdocsfn2.tex.
%
%<package>\ifdefined\childdocmain\endinput\fi
%<package>\ProvidesFile{childdoc.def}[2018/12/30 v2.0 child document driver]
%<samplemain>\ProvidesFile{cdocsamp.tex}[2018/12/30 v2.0 sample for childdoc]
%<*driver>
%\ProvidesFile{childdoc.drv}[2018/12/30 v2.0 childdoc reference manual file]
\PassOptionsToClass{10pt,a4paper}{article}
\documentclass{ltxdoc}

\usepackage[margin=35mm]{geometry}
\usepackage{hyperref}
\usepackage{hyperxmp}
\usepackage[usenames]{color}

\hypersetup{colorlinks=true}
\hypersetup{pdfstartview=FitH}
\hypersetup{pdfpagemode=UseNone}
\hypersetup{pdfsource={}}
\hypersetup{pdflang={en-UK}}
\hypersetup{pdfcopyright={Copyright 2017-2018 Niklas Beisert.
  This work may be distributed and/or modified under the
  conditions of the LaTeX Project Public License, either version 1.3
  of this license or (at your option) any later version.}}
\hypersetup{pdflicenseurl={http://www.latex-project.org/lppl.txt}}
\hypersetup{pdfcontactaddress={ETH Zurich, ITP, HIT K,
  Wolfgang-Pauli-Strasse 27}}
\hypersetup{pdfcontactpostcode={8093}}
\hypersetup{pdfcontactcity={Zurich}}
\hypersetup{pdfcontactcountry={Switzerland}}
\hypersetup{pdfcontactemail={nbeisert@itp.phys.ethz.ch}}
\hypersetup{pdfcontacturl={http://people.phys.ethz.ch/\xmptilde nbeisert/}}

\newcommand{\secref}[1]{\hyperref[#1]{section \ref*{#1}}}

\parskip1ex
\parindent0pt
\let\olditemize\itemize
\def\itemize{\olditemize\parskip0pt}

\begin{document}

\title{The \textsf{childdoc} Package}
\hypersetup{pdftitle={The childdoc Package}}
\author{Niklas Beisert\\[2ex]
  Institut f\"ur Theoretische Physik\\
  Eidgen\"ossische Technische Hochschule Z\"urich\\
  Wolfgang-Pauli-Strasse 27, 8093 Z\"urich, Switzerland\\[1ex]
  \href{mailto:nbeisert@itp.phys.ethz.ch}
  {\texttt{nbeisert@itp.phys.ethz.ch}}}
\hypersetup{pdfauthor={Niklas Beisert}}
\hypersetup{pdfsubject={Manual for the LaTeX2e Package childdoc}}
\date{30 December 2018, \textsf{v2.0}}
\maketitle

\begin{abstract}\noindent
\textsf{childdoc} is a \LaTeXe{} package
that enables the direct compilation
of document sections included by |\include|
to individual files.
\end{abstract}

\begingroup
\parskip0ex
\tableofcontents
\endgroup

%%%%%%%%%%%%%%%%%%%%%%%%%%%%%%%%%%%%%%%%%%%%%%%%%%%%%%%%%%%%%%%%%%%%%%%%%%%%%%%%
%%%%%%%%%%%%%%%%%%%%%%%%%%%%%%%%%%%%%%%%%%%%%%%%%%%%%%%%%%%%%%%%%%%%%%%%%%%%%%%%
\section{Introduction}

\LaTeX{} provides a mechanism to structure a large document (such as a book)
into a main file and several child files (containing the chapters)
using the |\include| command.
This mechanism is beneficial for documents
which span hundreds of pages in order to
make the source file(s) more manageable.
Moreover, compilation can be restricted to
selected child files by means of the |\includeonly| command.
The latter feature can be used to reduce the compilation time while editing
(this was significantly more useful in the earlier days of \LaTeX{})
or to generate a smaller document which is easier to navigate.
Another application of |\includeonly| is to generate
documents consisting of selected parts of the complete document.

However, there are a few drawbacks of the plain |\include| mechanism:
\begin{itemize}
\item
The child files cannot be compiled on their own,
they can only be compiled via the main file.
A naive editing environment
(such as a text editor with an option
to have the current file processed by \LaTeX)
may require one to switch to the main file before compiling;
attempting to compile the child file produces errors.
\item
The main file must be modified (each time)
to adjust the |\includeonly| command
to the present needs. This easily leaves the main file in a messy state.
\item
The generated document will always carry the filename
of the main document. This is inconvenient if
several child files are to be compiled and
to be kept for distribution.
\end{itemize}

The present package provides a simple interface
to make child files individually compilable by \LaTeX{}.
Compiling a child file then has the same effect as compiling
the main file with an |\includeonly| command
to select the appropriate child.
Moreover the generated document will carry the name of the child
rather than the main file.
This resolves all three above issues.

This feature is meant to make the editing of books,
thesis documents and lecture notes somewhat more convenient.
However, the package can also be used efficiently for
composing a series of documents (such as exercise sheets)
which are typically distributed individually.
It then assists the author in generating the individual documents
(potentially in different versions)
as well as a document containing the collected series.
Another application is in developing style files
or other kinds of included material
where compilation of the style file could redirect
to a sample or test file.

%%%%%%%%%%%%%%%%%%%%%%%%%%%%%%%%%%%%%%%%%%%%%%%%%%%%%%%%%%%%%%%%%%%%%%%%%%%%%%%%
%%%%%%%%%%%%%%%%%%%%%%%%%%%%%%%%%%%%%%%%%%%%%%%%%%%%%%%%%%%%%%%%%%%%%%%%%%%%%%%%
\section{Usage}

First of all, the package \textsf{childdoc} is \emph{not} a standard
\LaTeXe{} |.sty| style file! Therefore it needs to be invoked in
a non-standard way.

%%%%%%%%%%%%%%%%%%%%%%%%%%%%%%%%%%%%%%%%%%%%%%%%%%%%%%%%%%%%%%%%%%%%%%%%%%%%%%%%
\subsection{Included Files}
\label{sec:include}

%%%%%%%%%%%%%%%%%%%%%%%%%%%%%%%%%%%%%%%%
\DescribeMacro{\childdocmain}
To use the package, add the commands
\begin{center}
\begin{tabular}{l}
|\input{childdoc.def}|\\
|\childdocmain{}|\\
\end{tabular}
\end{center}
at the very top of the main \LaTeX{} file,
in particular \emph{before} the |\documentclass| statement!
The argument of |\childdocmain| should be left empty
(but it must be present).

%%%%%%%%%%%%%%%%%%%%%%%%%%%%%%%%%%%%%%%%
\DescribeMacro{\childdocof}
Furthermore, add the commands
\begin{center}
\begin{tabular}{l}
|\input{childdoc.def}|\\
|\childdocof{|\textit{main}|}|\\
\end{tabular}
\end{center}
at the top of every child file \textit{child}
which is included by |\include{|\textit{child}|}|
from within the main file
(or at least for those files to be compiled individually).
The argument \textit{main} must be the filename of the main file.

There are a couple of
considerations in setting up the main and child documents:

%%%%%%%%%%%%%%%%%%%%%%%%%%%%%%%%%%%%%%%%
\paragraph{Restrictions.}

Please note the following restrictions:
\begin{itemize}
\item
|\childdocmain| must be called with one argument \textit{main}
to ensure compatibility with earlier version of the package.
It must either be empty (|\childdocmain{}|)
or precisely match the filename of the main file in which it is specified.
See \secref{sec:detection} for further information.
\item
The filename \textit{main} must be specified without the |.tex| extension.
\item
The filename \textit{main} is case sensitive
(even in case-insensitive file systems)
due to internal string comparison.
\item
The argument \textit{main} should be fully expanded, it cannot be a macro.
\item
Subdirectories and special characters should be avoided in filenames.
\item
The command |\childdocmain{|\textit{main}|}| must be followed by a whitespace.
It should not be followed immediately by another command
or by a comment mark `|%|'.
This is because the \TeX{} parser reads the token immediately following
the argument of |\childdocmain| and puts it
at the beginning of every child section;
however, a white\-space is ignored.
\end{itemize}

%%%%%%%%%%%%%%%%%%%%%%%%%%%%%%%%%%%%%%%%
\paragraph{Content of Main File.}

It is advisable to place all content in the child files included by |\include|.
Any output contained in the main file will appear in all child documents
unless suppressed manually;
it cannot be suppressed automatically by the |\includeonly| directive
and thus should normally be avoided.
A method to include some content in the main file
by means of conditional processing is described in \secref{sec:conditional}.

%%%%%%%%%%%%%%%%%%%%%%%%%%%%%%%%%%%%%%%%
\paragraph{Page Numbering.}

When only a part of the document is compiled,
the appropriate numbering of pages
(as well as other status parameters)
is determined from the |.aux| files.
The latter contain information from previous passes.
However this information needs to propagate through
all intermediate child documents.
Therefore the page numbering in child documents may well
be inconsistent until the complete document is compiled at least once.

A useful (if unconventional) way to always ensure a consistent
page numbering is to restart the numbering in each child document
and denote the pages by `\textit{child}|.|\textit{page}'
where \textit{child} represents the chapter/section number of the child file.
This can be achieved by the command
|\numberwithin{page}{|\textit{child}|}|
of the \textsf{amsmath} package
where \textit{child} can be |chapter| or |section|
depending on the chosen structuring.
Alternatively, one can modify the macro |\thepage| appropriately
and reset the counter |page| at the start of each child file.

%%%%%%%%%%%%%%%%%%%%%%%%%%%%%%%%%%%%%%%%%%%%%%%%%%%%%%%%%%%%%%%%%%%%%%%%%%%%%%%%
\subsection{Conditional Processing}
\label{sec:conditional}

The package provides a mechanism to compile different versions
of a document. To customise the versions further some conditional processing
can come in handy to distinguish which version is being compiled.
The package provides two macros to describe the compilation context:

%%%%%%%%%%%%%%%%%%%%%%%%%%%%%%%%%%%%%%%%
\DescribeMacro{\ifchilddoc}
The conditional |\ifchilddoc| distinguishes between the compilation of
child documents and the main document:
%
\begin{center}
|\ifchilddoc |\textit{child-code}| |[|\||else |\textit{main-code}]| \||fi|
\end{center}

%%%%%%%%%%%%%%%%%%%%%%%%%%%%%%%%%%%%%%%%
\DescribeMacro{\childdocname}
\DescribeMacro{\childdocjob}
The macro |\childdocname| contains the filename (without extension)
of the main or child file being processed.
Note that |\childdocjob| will always contain the name of the main file.

%%%%%%%%%%%%%%%%%%%%%%%%%%%%%%%%%%%%%%%%
\paragraph{Title Page.}

Conditional processing can be used to include a title or banner page
in the main document when proper precautions are taken.
Importantly, the code in the main file should ensure that the page counter
(as well as other status parameters which are stored in the |.aux| files)
takes the same value after the conditional processing.
Otherwise the page numbers may take divergent values
depending on which part is compiled.

For example, a title page could be declared by:
%
\begin{center}
\begin{tabular}{l}
|\ifchilddoc\||else|\\
|\addtocounter{page}{-1}|\\
\textit{code for title page}\\
|\newpage|\\
|\||fi|
\end{tabular}
\end{center}
%
A banner page for the child documents can be generated by:
%
\begin{center}
\begin{tabular}{l}
|\ifchilddoc|\\
|\addtocounter{page}{-1}|\\
\textit{code for banner page}\\
|\newpage|\\
|\||fi|
\end{tabular}
\end{center}
%
Here one could write a message such as:
\begin{center}
|This is the part \childdocname{} of \childdocjob{}.|
\end{center}

%%%%%%%%%%%%%%%%%%%%%%%%%%%%%%%%%%%%%%%%%%%%%%%%%%%%%%%%%%%%%%%%%%%%%%%%%%%%%%%%
\subsection{Flags}
\label{sec:flags}

The package makes it easy to generate different versions
of the main or child documents.
To this end compilation flags can be defined
and assigned different default values.
They will be particularly useful in conjunction
with the forwarding mechanism described in \secref{sec:forward}.

For example, it may be useful to have a flag |\version|
which can be set to |draft| or |final|.
The document source will contain some conditional code
depending on the value of |\version|.
Suppose further, the flag should default to |final| for the main file
and to |draft| for child files
which is a natural assignment for editing the document.
This is achieved by placing the following code
in the preamble of the main document
(below the |\childdocmain| directive):
%
\begin{center}
\begin{tabular}{l}
|\ifchilddoc|\\
|\providecommand{\version}{draft}|\\
|\||else|\\
|\providecommand{\version}{final}|\\
|\||fi|
\end{tabular}
\end{center}
%
The definition by |\providecommand| makes sure
that previous definitions are not overwritten.
Further statements |\providecommand{\version}{...}|
can thus be added before the above code to override it.

For the main file, one might add a line
(between |\childdocmain| and the above block)
%
\begin{center}
|%\ifchilddoc\||else\providecommand{\version}{draft}\||fi|
\end{center}
%
which can be uncommented to produce a draft version.
Likewise one can add a line to the very top of a child file
(above the |\childdocof{|\textit{main}|}| directive)
%
\begin{center}
|%\providecommand{\version}{final}|
\end{center}
%
which can be uncommented to produce the final version of this child document.

%%%%%%%%%%%%%%%%%%%%%%%%%%%%%%%%%%%%%%%%%%%%%%%%%%%%%%%%%%%%%%%%%%%%%%%%%%%%%%%%
\subsection{Forwarding}
\label{sec:forward}

Different versions of the main or child documents
using compilation flags as described in \secref{sec:flags}
can be (permanently) stored in different files
for convenient compilation, viewing and distribution.
To this end, the package defines a command
to pass on compilation to a different file:

%%%%%%%%%%%%%%%%%%%%%%%%%%%%%%%%%%%%%%%%
\DescribeMacro{\childdocforward}
The command |\childdocforward| redirects processing to
another source file:
%
\begin{center}
\begin{tabular}{l}
|\input{childdoc.def}|\\
|\childdocforward[|\textit{main}|]{|\textit{dest}|}|\\
\end{tabular}
\end{center}
%
The argument \textit{dest} is the destination file
(without extension).
It should be the main file or one of the child files.
Note that further \textsf{childdoc} directives
such as |\childdocof| and |\childdocforward|
in the indicated file will be processed in this form.
The optional argument \textit{main}
passes on directly to the main file \textit{main}
while pretending to compile the child \textit{dest}.
This form behaves as if \textit{dest}
issues |\childdocof{|\textit{main}|}| right away,
and no further \textsf{childdoc} directives will be processed.

%%%%%%%%%%%%%%%%%%%%%%%%%%%%%%%%%%%%%%%%
\DescribeMacro{\...prefix}
In the alternative form |\childdocforwardprefix|,
%
\begin{center}
\begin{tabular}{l}
|\input{childdoc.def}|\\
|\childdocforwardprefix[|\textit{main}|]{|\textit{prefix}|}{|\textit{dest}|}|
\end{tabular}
\end{center}
%
the destination file is determined by a pattern
depending on the current file:
To make this work, the current file must be called
`{\textit{prefix}\hspace{0.2em}\textit{suffix}}'
with \textit{prefix} matching precisely the argument.
Processing is then passed on to the file
`{\textit{dest}\hspace{0.2em}\textit{suffix}}'.
Surely, the same effect is achieved by
directly specifying the
argument `{\textit{dest}\hspace{0.2em}\textit{suffix}}'
in the first form.
However, that requires to set up a different file
for each child. With the alternative form of the command
all these files can have exactly the same content
which simplifies setting them up and maintaining them.

For example, the following file |draft.tex|
with a compilation flag |\version| as described in \secref{sec:flags}
compiles the main document as a draft:
%
\begin{center}
\begin{tabular}{l}
|\def\version{draft}|\\
|\input{childdoc.def}|\\
|\childdocforward{|\textit{main}|}|
\end{tabular}
\end{center}
%
Likewise, the following files |final|\textit{nn}|.tex|
compile the final version of the child document
|child|\textit{nn}|.tex|:
%
\begin{center}
\begin{tabular}{l}
|\def\version{final}|\\
|\input{childdoc.def}|\\
|\childdocforwardprefix{final}{child}|
\end{tabular}
\end{center}
%

Note that when several versions of a main file and/or of each child file
are to be generated, it may be convenient to set up a |Makefile| or
shell script to automatise the process.

%%%%%%%%%%%%%%%%%%%%%%%%%%%%%%%%%%%%%%%%%%%%%%%%%%%%%%%%%%%%%%%%%%%%%%%%%%%%%%%%
\subsection{Command Line Processing}
\label{sec:commandline}

The effect of redirection files can also be achieved by invoking
the \LaTeX{} compiler with a more elaborate command line.
Most conveniently this should be done as part
of a shell script or a |Makefile|.

When using \textsf{childdoc} in the main file, the following
command lines effectively perform a redirection
(note that depending on the shell being used,
backslashes may have to be doubled: `|\|' $\to$ `|\\|'):
%
\begin{center}
|... -jobname "|\textit{target}|" |\\|"|[\textit{flags}]%
|\input{childdoc.def}\childdocforward[|\textit{main}|]{|\textit{dest}|}"|
\end{center}
%
Here \textit{target} is the name of the output file,
\textit{main} is the name of the main file
and \textit{dest} is the name of the main or child file to be processed
(all filenames without extensions).
The optional argument \textit{main} can be omitted
if \textit{main} matches \textit{dest}.
Optionally, compilation \textit{flags} can be defined via |\def| commands.
This command line makes the \TeX{} engine believe
it is compiling the file \textit{target}
whose content is specified as the latter parameter.
The provided code then forwards the processing to
\textit{main} or \textit{dest} as described in \secref{sec:forward}.

%%%%%%%%%%%%%%%%%%%%%%%%%%%%%%%%%%%%%%%%%%%%%%%%%%%%%%%%%%%%%%%%%%%%%%%%%%%%%%%%
\subsection{Include by Input}
\label{sec:input}

Including child documents by |\include| has some restrictions by design.
Most notably, the content of a child document always occupies
its own set of pages; pages cannot be shared between child documents.
Usually, this behaviour makes perfect sense
because each child document contain an essential part of the document.
However, in some situations it may be desirable to compose
a document from a collection of parts
without having mandatory page breaks between then.
For this case, the package
provides a mechanism to include parts
by |\input| which can also be processed individually.
However, by construction this mechanism
requires manual handling of the content to be output.

%%%%%%%%%%%%%%%%%%%%%%%%%%%%%%%%%%%%%%%%
\DescribeMacro{\ifchilddocmanual}
The main file should be prepared as usual, see \secref{sec:include}.
However, the document body must make a distinction
between processing of an individual part and of the main document, e.g.:
%
\begin{center}
\begin{tabular}{l}
|\ifchilddocmanual|\\
|\input{\childdocname}|\\
|\||else|\\
\textit{document body with }|\input{|\textit{part}|}|\\
|\||fi|
\end{tabular}
\end{center}
%
The conditional |\ifchilddocmanual| is true whenever
a part to be included by |\input| is being compiled,
and the name of the part is stored in |\childdocname|.

%%%%%%%%%%%%%%%%%%%%%%%%%%%%%%%%%%%%%%%%
\DescribeMacro{\childdocby}
Each part to be included by |\input| should start with:
%
\begin{center}
\begin{tabular}{l}
|\input{childdoc.def}|\\
|\childdocby{|\textit{main}|}|\\
\end{tabular}
\end{center}
%
The directive |\childdocby| is similar to |\childdocof|
described in \secref{sec:include},
but the subsequent selection of content must be done manually.
To that end, both |\ifchilddoc| and |\ifchilddocmanual|
will be true upon processing of a part,
and the name of the part is stored in |\childdocname|.
Note that |\jobname| will be set to the filename of the current part
so that each part receives an individual |.aux| file
that does not interfere with the |.aux| file(s) of the main document.
This behaviour can be altered by the alternative form
|\childdocby[*]{|\textit{main}|}| (with a non-empty optional argument)
which uses the |.aux| file of the main document
by setting |\jobname| to \textit{main}.

%%%%%%%%%%%%%%%%%%%%%%%%%%%%%%%%%%%%%%%%%%%%%%%%%%%%%%%%%%%%%%%%%%%%%%%%%%%%%%%%
\subsection{Driver Development}
\label{sec:driver}

The \textsf{childdoc} mechanism can also be use for the development
of definition files such as \LaTeX{} styles or classes.
This case differs from the above setup with multiple parts
included by |\include| in that no |\includeonly| should be invoked.
This can be achieved by starting the include file
(before |\ProvidesPackage|) with:
%
\begin{center}
\begin{tabular}{l}
|\input{childdoc.def}|\\
|\childdocforward{|\textit{main}|}|\\
\end{tabular}
\end{center}
%
or alternatively with:
%
\begin{center}
\begin{tabular}{l}
|\input{childdoc.def}|\\
|\childdocby{|\textit{main}|}|\\
\end{tabular}
\end{center}
%
Both forms have slightly different effects as described above.
The main file is prepared as usual, see \secref{sec:include}.

%%%%%%%%%%%%%%%%%%%%%%%%%%%%%%%%%%%%%%%%%%%%%%%%%%%%%%%%%%%%%%%%%%%%%%%%%%%%%%%%
\subsection{Legacy Detection}
\label{sec:detection}

The directive |\childdocmain| in the main file can detect
whether the complete document or merely a child is to be compiled
even without using the directive |\childdocof|.
This method is deprecated because it is less robust
and there is no compelling reason to use it;
it is merely provided for backward compatibility
and it may be removed in future versions.

If the detection mechanism is to be used,
it is mandatory to correctly specify
the filename of the main file as the argument of |\childdocmain|:
%
\begin{center}
\begin{tabular}{l}
|\input{childdoc.def}|\\
|\childdocmain{|\textit{main}|}|\\
\end{tabular}
\end{center}
%
If |\jobname| does not match the argument \textit{main} of |\childdocmain|,
it is assumed that |\jobname| points to the child file to be compiled.
When using |\childdocmain| with the main file specified as argument,
it suffices to start a child file
with just |\input{|\textit{main}|}|
without loading of the package and using |\childdocof|.
If instead all processing is done
with the appropriate \textsf{childdoc} directives,
the argument of \textit{main} of |\childdocmain| can be empty.

An alternative version of the command line processing described
in \secref{sec:commandline} using the detection mechanism reads:
%
\begin{center}
|... -jobname "|\textit{target}|" "|[\textit{flags}]%
[|\def\jobname{|\textit{dest}|}|]|\input{|\textit{main}|}"|
\end{center}

%%%%%%%%%%%%%%%%%%%%%%%%%%%%%%%%%%%%%%%%%%%%%%%%%%%%%%%%%%%%%%%%%%%%%%%%%%%%%%%%
\subsection{Manual Code}
\label{sec:manual}

In case one cannot be certain whether the definitions file |childdoc.def|
is installed on the target \TeX{} distribution
and one prefers not to ship it,
it is conceivable to paste a few relevant commands into the sources.

To that end, drop all statements |\input{childdoc.def}|
and perform the replacements as outlined below.
Instead of |\childdocmain{|\textit{main}|}| add the following code
to the top of the main file:
%
\begin{center}
\begin{tabular}{l}
|\||ifdefined\childdocname\endinput\||fi\newif\ifchilddoc|\\
|\edef\childdocname{\scantokens\expandafter{\jobname\noexpand}}|\\
|\def\childdocmain{|\textit{main}|}\||ifx\childdocmain\childdocname\||else|\\
|\childdoctrue\includeonly{\childdocname}\let\jobname\childdocmain\||fi|\\
\end{tabular}
\end{center}
%
Instead of |\childdocof{|\textit{main}|}| just include the main file
at the top of each child file:
%
\begin{center}
|\input{|\textit{main}|}|
\end{center}
%
A simple redirection |\childdocforward{|\textit{dest}|}| is achieved by:
%
\begin{center}
|\def\jobname{|\textit{dest}|}\input{\jobname}|
\end{center}
%
The redirection with prefix
|\childdocforwardprefix[|\textit{prefix}|]{|\textit{dest}|}|
is accomplished by:
%
\begin{center}
\begin{tabular}{l}
|{\edef\jobname{\scantokens\expandafter{\jobname\noexpand}}|\\
|\def\redirectjob |\textit{prefix}|#1~~~{\gdef\jobname{|\textit{dest}|#1}}|\\
|\expandafter\redirectjob\jobname~~~}\input{\jobname}|
\end{tabular}
\end{center}

In an alternative approach,
child documents can be compiled by a specific command line
without additional code or specific definitions:
%
\begin{center}
|... -jobname "|\textit{target}|" "|[\textit{flags}]%
|\includeonly{|\textit{dest}|}\input{|\textit{main}|}"|
\end{center}
%

%%%%%%%%%%%%%%%%%%%%%%%%%%%%%%%%%%%%%%%%%%%%%%%%%%%%%%%%%%%%%%%%%%%%%%%%%%%%%%%%
%%%%%%%%%%%%%%%%%%%%%%%%%%%%%%%%%%%%%%%%%%%%%%%%%%%%%%%%%%%%%%%%%%%%%%%%%%%%%%%%
\section{Information}

%%%%%%%%%%%%%%%%%%%%%%%%%%%%%%%%%%%%%%%%%%%%%%%%%%%%%%%%%%%%%%%%%%%%%%%%%%%%%%%%
\subsection{Copyright}

Copyright \copyright{} 2017--2018 Niklas Beisert

This work may be distributed and/or modified under the
conditions of the \LaTeX{} Project Public License, either version 1.3
of this license or (at your option) any later version.
The latest version of this license is in
  \url{http://www.latex-project.org/lppl.txt}
and version 1.3 or later is part of all distributions of \LaTeX{}
version 2005/12/01 or later.

This work has the LPPL maintenance status `maintained'.

The Current Maintainer of this work is Niklas Beisert.

This work consists of the files |README.txt|, |childdoc.ins| and |childdoc.dtx|
as well as the derived files |childdoc.def|, |cdocsamp.tex|
with |cdocsch1.tex|, |cdocsch2.tex|, |cdocspt3.tex|, |cdocspt4.tex|,
|cdocsdrf.tex|, |cdocsfn1.tex|, |cdocsfn2.tex|
as well as |childdoc.pdf|.

%%%%%%%%%%%%%%%%%%%%%%%%%%%%%%%%%%%%%%%%%%%%%%%%%%%%%%%%%%%%%%%%%%%%%%%%%%%%%%%%
\subsection{Files and Installation}

The package consists of the files:
%
\begin{center}
\begin{tabular}{ll}
    |README.txt|   & readme file \\
    |childdoc.ins| & installation file \\
    |childdoc.dtx| & source file \\
    |childdoc.def| & definition file \\
    |cdocsamp.tex| & sample main file \\
    |cdocsch1.tex| & sample include file \\
    |cdocsch2.tex| & sample include file \\
    |cdocspt3.tex| & sample part file \\
    |cdocspt4.tex| & sample part file \\
    |cdocsdrf.tex| & sample redirection file \\
    |cdocsfn1.tex| & sample redirection file \\
    |cdocsfn2.tex| & sample redirection file \\
    |childdoc.pdf| & manual
\end{tabular}
\end{center}
%
The distribution consists of the files
|README.txt|, |childdoc.ins| and |childdoc.dtx|.
%
\begin{itemize}
\item
Run (pdf)\LaTeX{} on |childdoc.dtx|
to compile the manual |childdoc.pdf| (this file).
\item
Run \LaTeX{} on |childdoc.ins| to create the definitions file |childdoc.def|
and the sample |cdocsamp.tex| with include files
|cdocsch1.tex|, |cdocsch2.tex|, |cdocspt3.tex|, |cdocspt4.tex|,
|cdocsdrf.tex|, |cdocsfn1.tex|, |cdocsfn2.tex|.
Then copy the file |childdoc.def| to an appropriate directory of your \LaTeX{}
distribution, e.g.\ \textit{texmf-root}|/tex/latex/childdoc|.
\end{itemize}

%%%%%%%%%%%%%%%%%%%%%%%%%%%%%%%%%%%%%%%%%%%%%%%%%%%%%%%%%%%%%%%%%%%%%%%%%%%%%%%%
\subsection{Related CTAN Packages}

There are several other packages which offer a similar functionality:
%
\begin{itemize}
\item
The packages
\href{http://ctan.org/pkg/docmute}{\textsf{docmute}},
\href{http://ctan.org/pkg/includex}{\textsf{includex}} and
\href{http://ctan.org/pkg/standalone}{\textsf{standalone}}
provide commands to include only the document body of
a child file thus allowing both files to be compiled individually.
\item
The packages \href{http://ctan.org/pkg/subdocs}{\textsf{subdocs}}
and \href{http://ctan.org/pkg/subfiles}{\textsf{subfiles}}
provide structures in which the main and child documents can be
encapsulated and allowing them to be compiled individually.
The inclusion mechanism is different from the conventional |\include|.
\item
The package \href{http://ctan.org/pkg/combine}{\textsf{combine}}
is an elaborate solution to combine several documents into one.
\end{itemize}
%
See also the CTAN topic \href{http://ctan.org/topic/subdocs}{\textsf{subdocs}}
for further related packages.
The present package differs from the above solutions in that
a document structure constructed with the conventional |\include| mechanism
just needs two extra commands at the top of every file
such that all constituent files can be compiled individually.

%%%%%%%%%%%%%%%%%%%%%%%%%%%%%%%%%%%%%%%%%%%%%%%%%%%%%%%%%%%%%%%%%%%%%%%%%%%%%%%%
%\subsection{Feature Suggestions}
%
%The following is a list of features which may be useful for future
%versions of this package:
%%
%\begin{itemize}
%\item
%\ldots
%\end{itemize}

%%%%%%%%%%%%%%%%%%%%%%%%%%%%%%%%%%%%%%%%%%%%%%%%%%%%%%%%%%%%%%%%%%%%%%%%%%%%%%%%
\subsection{Revision History}

%%%%%%%%%%%%%%%%%%%%%%%%%%%%%%%%%%%%%%%%
\paragraph{v2.0:} 2018/12/30

\begin{itemize}
\item
immediate forward processing
\item
added |\childdocby| mechanism
\item
manual restructured
\end{itemize}

%%%%%%%%%%%%%%%%%%%%%%%%%%%%%%%%%%%%%%%%
\paragraph{v1.6:} 2018/01/17

\begin{itemize}
\item
application for development of include files
\item
corrections to manual
\end{itemize}

%%%%%%%%%%%%%%%%%%%%%%%%%%%%%%%%%%%%%%%%
\paragraph{v1.5:} 2017/05/21

\begin{itemize}
\item
more complete structuring introduced
\item
|\childdocof| introduced
\item
|\childdoc| renamed to |\childdocmain|
\item
|\childredirect| renamed to |\childdocforward| and |\childdocforwardprefix|
and functionality expanded
\end{itemize}

%%%%%%%%%%%%%%%%%%%%%%%%%%%%%%%%%%%%%%%%
\paragraph{v1.0:} 2017/04/27

\begin{itemize}
\item
manual and install package
\item
first version published on CTAN
\end{itemize}

%%%%%%%%%%%%%%%%%%%%%%%%%%%%%%%%%%%%%%%%
\paragraph{v0.6:} 2017/04/26

\begin{itemize}
\item
redirection mechanism added
\end{itemize}

%%%%%%%%%%%%%%%%%%%%%%%%%%%%%%%%%%%%%%%%
\paragraph{v0.5:} 2017/04/26

\begin{itemize}
\item
functionality in definition file
\end{itemize}


%%%%%%%%%%%%%%%%%%%%%%%%%%%%%%%%%%%%%%%%%%%%%%%%%%%%%%%%%%%%%%%%%%%%%%%%%%%%%%%%
%%%%%%%%%%%%%%%%%%%%%%%%%%%%%%%%%%%%%%%%%%%%%%%%%%%%%%%%%%%%%%%%%%%%%%%%%%%%%%%%
%%%%%%%%%%%%%%%%%%%%%%%%%%%%%%%%%%%%%%%%%%%%%%%%%%%%%%%%%%%%%%%%%%%%%%%%%%%%%%%%
\appendix

\settowidth\MacroIndent{\rmfamily\scriptsize 000\ }

 \DocInput{childdoc.dtx}

\end{document}
%</driver>
% \fi
%
% %%%%%%%%%%%%%%%%%%%%%%%%%%%%%%%%%%%%%%%%%%%%%%%%%%%%%%%%%%%%%%%%%%%%%%%%%%%%%%
% %%%%%%%%%%%%%%%%%%%%%%%%%%%%%%%%%%%%%%%%%%%%%%%%%%%%%%%%%%%%%%%%%%%%%%%%%%%%%%
% \section{Sample}
%\iffalse
%<*samplemain>
%\fi
%
% The following presents a sample document
% with two chapters, two parts, a title page,
% a compile flag as well as three forwarding files to set the flag.
% It consists of eight |.tex| files:
% \begin{center}
% \begin{tabular}{ll}
% |cdocsamp.tex|&main file\\
% |cdocsch1.tex|&include file for chapter 1\\
% |cdocsch2.tex|&include file for chapter 2\\
% |cdocspt3.tex|&include file for part 3\\
% |cdocspt4.tex|&include file for part 4\\
% |cdocsdrf.tex|&forwarding file for main file in draft mode\\
% |cdocsfi1.tex|&forwarding file for final version of chapter 1\\
% |cdocsfi2.tex|&forwarding file for final version of chapter 2\\
% \end{tabular}
% \end{center}
% Each of the eight files can be compiled directly by the \LaTeX{} compiler.
%
% %%%%%%%%%%%%%%%%%%%%%%%%%%%%%%%%%%%%%%
% \paragraph{Main File.}
%
% The main file is called |cdocsamp.tex|.
%
% Load the \textsf{childdoc} definitions and
% declare the filename for the main document:
%    \begin{macrocode}
\input{childdoc.def}
\childdocmain{}
%    \end{macrocode}

% Optional override for |\version| flag:
%    \begin{macrocode}
%%\ifchilddoc\else\providecommand{\version}{draft}\fi
%    \end{macrocode}

% Define the default values for the |\version| flag
% (|final| for the main file and |draft| for childs):
%    \begin{macrocode}
\ifchilddoc
\providecommand{\version}{draft}
\else
\providecommand{\version}{final}
\fi
%    \end{macrocode}

% Load the standard document class:
%    \begin{macrocode}
\documentclass[12pt]{article}
%    \end{macrocode}

% Start the document body:
%    \begin{macrocode}
\begin{document}
%    \end{macrocode}

% Declare a title page.
% Print title, part of document being processed and version flag:
%    \begin{macrocode}
\addtocounter{page}{-1}
\begin{center}
{\LARGE\bfseries{}childdoc example\par}
\vspace{1cm}
\ifchilddoc
\ifchilddocmanual part\else chapter\fi:
`\childdocname' of `\childdocjob'\par
\else
main document: `\childdocjob'\par
\fi
version: \version\par
\end{center}
\newpage
%    \end{macrocode}

% Manually include selected file,
% otherwise process as usual:
%    \begin{macrocode}
\ifchilddocmanual
\section*{part `\childdocname'}
\input{\childdocname}
\else
%    \end{macrocode}

% Include the two chapters:
%    \begin{macrocode}
\include{cdocsch1}
\include{cdocsch2}
%    \end{macrocode}

% Include the two parts unless only chapters should be displayed:
%    \begin{macrocode}
\ifchilddoc\else
\section{part three}
\input{cdocspt3}
\section{part four}
\input{cdocspt4}
\fi
%    \end{macrocode}

% Process as usual until here:
%    \begin{macrocode}
\fi
%    \end{macrocode}

% End of document body:
%    \begin{macrocode}
\end{document}
%    \end{macrocode}
%\iffalse
%</samplemain>
%\fi
%
% %%%%%%%%%%%%%%%%%%%%%%%%%%%%%%%%%%%%%%
% \paragraph{Chapter Include Files.}
%
% The include files are called |cdocsch1.tex| and |cdocsch2.tex|.
%
%\iffalse
%<*samplechap1|samplechap2>
%\fi

% Optional override for |\version| flag:
%    \begin{macrocode}
%%\providecommand{\version}{final}
%    \end{macrocode}

% Include the main document:
%    \begin{macrocode}
\input{childdoc.def}
\childdocof{cdocsamp}
%    \end{macrocode}

%\iffalse
%</samplechap1|samplechap2>
%\fi
%
%\iffalse
%<*samplechap1>
%\fi
% Some text for chapter 1:
%    \begin{macrocode}
\section{one}
some text in chapter one
%    \end{macrocode}

%\iffalse
%</samplechap1>
%\fi
% Some text for chapter 2:
%\iffalse
%<*samplechap2>
%\fi
%    \begin{macrocode}
\section{two}
more text in chapter two
%    \end{macrocode}

%\iffalse
%</samplechap2>
%\fi
%
% %%%%%%%%%%%%%%%%%%%%%%%%%%%%%%%%%%%%%%
% \paragraph{Part Include Files.}
%
% The include files are called |cdocspt3.tex| and |cdocspt4.tex|.
%
%\iffalse
%<*samplepart3|samplepart4>
%\fi

% Optional override for |\version| flag:
%    \begin{macrocode}
%%\providecommand{\version}{final}
%    \end{macrocode}

% Include the main document:
%    \begin{macrocode}
\input{childdoc.def}
\childdocby{cdocsamp}
%    \end{macrocode}

%\iffalse
%</samplepart3|samplepart4>
%\fi
%
%\iffalse
%<*samplepart3>
%\fi
% Some text for part 3:
%    \begin{macrocode}
some text in part three
%    \end{macrocode}

%\iffalse
%</samplepart3>
%\fi
% Some text for part 4:
%\iffalse
%<*samplepart4>
%\fi
%    \begin{macrocode}
more text in part four
%    \end{macrocode}

%\iffalse
%</samplepart4>
%\fi
%
% %%%%%%%%%%%%%%%%%%%%%%%%%%%%%%%%%%%%%%
% \paragraph{Forwarding for a Complete Draft.}
%
% The following forwarding file |cdocsdrf.tex|
% compiles the main document in draft mode:
%\iffalse
%<*sampledraft>
%\fi
%    \begin{macrocode}
\def\version{draft}
\input{childdoc.def}
\childdocforward{cdocsamp}
%    \end{macrocode}

%\iffalse
%</sampledraft>
%\fi
%
% %%%%%%%%%%%%%%%%%%%%%%%%%%%%%%%%%%%%%%
% \paragraph{Forwarding for Final Version of the Chapters.}
%
% The following forwarding files |cdocsfn1.tex| and |cdocsfn2.tex|
% (with identical content)
% compile the final versions of the child documents
% |cdocsch1.tex| and |cdocsch2.tex|, respectively:
%\iffalse
%<*samplefinal>
%\fi
%    \begin{macrocode}
\def\version{final}
\input{childdoc.def}
\childdocforwardprefix[cdocsamp]{cdocsfn}{cdocsch}
%    \end{macrocode}

%\iffalse
%</samplefinal>
%\fi
%
% %%%%%%%%%%%%%%%%%%%%%%%%%%%%%%%%%%%%%%
% \paragraph{Command Line Processing.}
%
% The following three command lines generate the output files
% |cdocscld|, |cdocscl1| and |cdocscl2|
% which should be identical to
% |cdocsdrf|, |cdocsch1| and |cdocsfn2|, respectively:
% \begin{center}
% \begin{tabular}{l}
% |latex -jobname cdocscld \|\\
% |  "\def\version{draft}\input{childdoc.def}\childdocforward{cdocsamp}"|\\
% |latex -jobname cdocscl1 \|\\
% |  "\input{childdoc.def}\childdocforward[cdocsamp]{cdocsch1}"|\\
% |latex -jobname cdocscl2 \|\\
% |  "\def\version{final}\input{childdoc.def}\childdocforward{cdocsch2}"|
% \end{tabular}
% \end{center}
% Note that the trailing backslash on each first line
% merely continues the input to the second line
% (for convenient cut ant paste).
% Furthermore, the command |latex| can be replaced by any
% of its alternative versions such as |pdflatex|.
%
% %%%%%%%%%%%%%%%%%%%%%%%%%%%%%%%%%%%%%%%%%%%%%%%%%%%%%%%%%%%%%%%%%%%%%%%%%%%%%%
% %%%%%%%%%%%%%%%%%%%%%%%%%%%%%%%%%%%%%%%%%%%%%%%%%%%%%%%%%%%%%%%%%%%%%%%%%%%%%%
% \section{Implementation}
%\iffalse
%<*package>
%\fi
%
% This section describes the definitions file |childdoc.def|.

% The definitions cannot be loaded using |\usepackage| or |\RequirePackage|
% which has a mechanism to prevent loading a style file more than once.
% When loading the definitions by means of |\input|
% multiple instances have to be prevented manually:
%\iffalse
%This code needs to be before the `\ProvidesFile' directive
%which is defined at the beginning of this file.
%Therefore it is also placed there and commented out here.
%</package>
%<*discard>
%\fi
%    \begin{macrocode}
\ifdefined\childdocmain\endinput\fi
%    \end{macrocode}
%\iffalse
%</discard>
%<*package>
%\fi
%
% \macro{\ifchilddoc}
% \macro{\ifchilddocmanual}
% The conditional |\ifchilddoc| tells whether a
% child (true) or main (false) document is being compiled.
% The conditional |\ifchilddocmanual| tells whether
% the |\includeonly| mechanism is used (false) or
% the selection of child files must be performed manually (true).
% The definitions initialise to false:
%    \begin{macrocode}
\newif\ifchilddoc
\newif\ifchilddocmanual
%    \end{macrocode}

% \macro{\childdocname}
% \macro{\childdocjob}
% The macro |\childdocname| stores the name of the main document
% to be compiled. The macro |\childdocjob| stores the name of
% the document on which the \LaTeX{} compiler was originally invoked.
% The content of |\jobname| cannot be compared
% to filenames specified in the source due to different catcodes.
% The following code rescans |\jobname|, stores the result
% in |\childdocname| and saves a copy in |\childdocjob|:
%    \begin{macrocode}
\edef\childdocname{\scantokens\expandafter{\jobname\noexpand}}
\let\childdocjob\childdocname
%    \end{macrocode}

% \macro{\childdocdisable}
% The macro |\childdocdisable| prevents the main file
% from being processed more than once.
% At this stage, the main document command |\childdocmain|
% is assumed to be called once again where it should do nothing.
% Any subsequent call to it should prevent
% a secondary processing of the main document
% It overwrites the forwarding commands
% |\childdocof| and |\childdocforward|
% with empty macros to prevent further inclusions of the main document:
%    \begin{macrocode}
\newcommand{\childdocdisable}
{
  \renewcommand{\childdocmain}[1]{\renewcommand{\childdocmain}[1]{\endinput}}
  \renewcommand{\childdocof}[1]{}
  \renewcommand{\childdocby}[2][]{}
  \renewcommand{\childdocforward}[2][]{}
  \renewcommand{\childdocdisable}{}
}
%    \end{macrocode}

% \macro{\childdocmain}
% The macro |\childdocmain| is to be called at the top of the main file
% with nothing or the main filename (without extension) as argument.
% First, it breaks loops.
% If the argument is not empty and does not match |\childdocname|
% (which is set by the first inclusion of |childdoc.def|),
% |\ifchilddoc| is set to true, |\includeonly| is applied to the child file
% and |\jobname| is set to the main file
% (for proper handling of |.aux| files):
%    \begin{macrocode}
\newcommand{\childdocmain}[1]
{
  \childdocdisable\childdocmain{}
  \if?#1?\else
    \begingroup
      \def\childdoctmp{#1}
      \ifx\childdoctmp\childdocname
        \def\childdoctmp{}
      \else
        \def\childdoctmp
        {
          \childdoctrue
          \includeonly{\childdocname}
          \def\childdocjob{#1}
          \def\jobname{#1}
        }
      \fi
      \expandafter
    \endgroup
    \childdoctmp
  \fi
}
%    \end{macrocode}

% \macro{\childdocof}
% The command |\childdocof| redirects
% compilation to the main file |#1|.
%    \begin{macrocode}
\newcommand{\childdocof}[1]
{
  \childdocdisable
  \childdoctrue
  \includeonly{\childdocname}
  \def\jobname{#1}
  \def\childdocjob{#1}
  \input{#1}
}
%    \end{macrocode}

% \macro{\childdocby}
% The command |\childdocby| ....
%    \begin{macrocode}
\newcommand{\childdocby}[2][]
{
  \childdocdisable
  \childdoctrue
  \childdocmanualtrue
  \if?#1?\else
    \def\jobname{#2}
  \fi
  \def\childdocjob{#2}
  \input{#2}
  \endinput
}
%    \end{macrocode}

% \macro{\childdocforward}
% The command |\childdocforward| redirects
% compilation to the main file or
% (if the optional argument is given) a child file.
% Parameters are set as if the main file
% or a child file starting with |\childdocof| was compiled.
% Then compilation is handed over to the main file:
%    \begin{macrocode}
\newcommand{\childdocforward}[2][]
{
  \begingroup
    \if?#1?
      \def\childdoctmp
      {
        \def\childdocname{#2}
        \def\childdocjob{#2}
        \def\jobname{#2}
        \input{#2}
        \endinput
      }
    \else
      \def\childdoctmp
      {
        \childdocdisable
        \def\childdocname{#2}
        \childdoctrue
        \includeonly{#2}
        \def\childdocjob{#1}
        \def\jobname{#1}
        \input{#1}
        \endinput
      }
    \fi
    \expandafter
  \endgroup
  \childdoctmp
}
%    \end{macrocode}

% \macro{\childdocforwardprefix}
% The command |\childdocforwardprefix| redirects
% compilation to the main or a child file by means of a pattern.
% The prefix |#1| in the current filename is replaced by |#2|
% and the suffix of the current filename is kept
% (it is assumed that the filename does not contain the substring `|~~~|'
% which is used as a delimiter).
% Compilation is handed over to the new file by |\childdocforward|:
%    \begin{macrocode}
\newcommand{\childdocforwardprefix}[3][]
{
  \begingroup
    \def\childdocextract #2##1~~~{\def\childdoctmp{\childdocforward[#1]{#3##1}}}
    \expandafter\childdocextract\childdocname~~~
    \expandafter
  \endgroup
  \childdoctmp
}
%    \end{macrocode}

% \macro{\childdoc}
% The deprecated macro |\childdoc| is a legacy version of |\childdocmain|:
%    \begin{macrocode}
\newcommand{\childdoc}{\childdocmain}
%    \end{macrocode}

% \macro{\childdocredirect}
% The deprecated macro |\childdocredirect| is a legacy version
% of |\childdocforward| and |\childdocforwardprefix|:
%    \begin{macrocode}
\newcommand{\childdocredirect}[2][]
{
  \begingroup
    \if?#1?
      \def\childdoctmp{\childdocforward{#2}}
    \else
      \def\childdoctmp{\childdocforwardprefix{#1}{#2}}
    \fi
    \expandafter
  \endgroup
  \childdoctmp
}
%    \end{macrocode}

%\iffalse
%</package>
%\fi
%
\endinput
|\\
|\childdocforward[|\textit{main}|]{|\textit{dest}|}|\\
\end{tabular}
\end{center}
%
The argument \textit{dest} is the destination file
(without extension).
It should be the main file or one of the child files.
Note that further \textsf{childdoc} directives
such as |\childdocof| and |\childdocforward|
in the indicated file will be processed in this form.
The optional argument \textit{main}
passes on directly to the main file \textit{main}
while pretending to compile the child \textit{dest}.
This form behaves as if \textit{dest}
issues |\childdocof{|\textit{main}|}| right away,
and no further \textsf{childdoc} directives will be processed.

%%%%%%%%%%%%%%%%%%%%%%%%%%%%%%%%%%%%%%%%
\DescribeMacro{\...prefix}
In the alternative form |\childdocforwardprefix|,
%
\begin{center}
\begin{tabular}{l}
|% \iffalse
%
% childdoc.dtx Copyright (C) 2017-2018 Niklas Beisert
%
% This work may be distributed and/or modified under the
% conditions of the LaTeX Project Public License, either version 1.3
% of this license or (at your option) any later version.
% The latest version of this license is in
%   http://www.latex-project.org/lppl.txt
% and version 1.3 or later is part of all distributions of LaTeX
% version 2005/12/01 or later.
%
% This work has the LPPL maintenance status `maintained'.
%
% The Current Maintainer of this work is Niklas Beisert.
%
% This work consists of the files childdoc.dtx and childdoc.ins
% and the derived files childdoc.def and cdocsamp.tex with
% cdocsch1.tex, cdocsch2.tex, cdocsdrf.tex, cdocsfn1.tex, cdocsfn2.tex.
%
%<package>\ifdefined\childdocmain\endinput\fi
%<package>\ProvidesFile{childdoc.def}[2018/12/30 v2.0 child document driver]
%<samplemain>\ProvidesFile{cdocsamp.tex}[2018/12/30 v2.0 sample for childdoc]
%<*driver>
%\ProvidesFile{childdoc.drv}[2018/12/30 v2.0 childdoc reference manual file]
\PassOptionsToClass{10pt,a4paper}{article}
\documentclass{ltxdoc}

\usepackage[margin=35mm]{geometry}
\usepackage{hyperref}
\usepackage{hyperxmp}
\usepackage[usenames]{color}

\hypersetup{colorlinks=true}
\hypersetup{pdfstartview=FitH}
\hypersetup{pdfpagemode=UseNone}
\hypersetup{pdfsource={}}
\hypersetup{pdflang={en-UK}}
\hypersetup{pdfcopyright={Copyright 2017-2018 Niklas Beisert.
  This work may be distributed and/or modified under the
  conditions of the LaTeX Project Public License, either version 1.3
  of this license or (at your option) any later version.}}
\hypersetup{pdflicenseurl={http://www.latex-project.org/lppl.txt}}
\hypersetup{pdfcontactaddress={ETH Zurich, ITP, HIT K,
  Wolfgang-Pauli-Strasse 27}}
\hypersetup{pdfcontactpostcode={8093}}
\hypersetup{pdfcontactcity={Zurich}}
\hypersetup{pdfcontactcountry={Switzerland}}
\hypersetup{pdfcontactemail={nbeisert@itp.phys.ethz.ch}}
\hypersetup{pdfcontacturl={http://people.phys.ethz.ch/\xmptilde nbeisert/}}

\newcommand{\secref}[1]{\hyperref[#1]{section \ref*{#1}}}

\parskip1ex
\parindent0pt
\let\olditemize\itemize
\def\itemize{\olditemize\parskip0pt}

\begin{document}

\title{The \textsf{childdoc} Package}
\hypersetup{pdftitle={The childdoc Package}}
\author{Niklas Beisert\\[2ex]
  Institut f\"ur Theoretische Physik\\
  Eidgen\"ossische Technische Hochschule Z\"urich\\
  Wolfgang-Pauli-Strasse 27, 8093 Z\"urich, Switzerland\\[1ex]
  \href{mailto:nbeisert@itp.phys.ethz.ch}
  {\texttt{nbeisert@itp.phys.ethz.ch}}}
\hypersetup{pdfauthor={Niklas Beisert}}
\hypersetup{pdfsubject={Manual for the LaTeX2e Package childdoc}}
\date{30 December 2018, \textsf{v2.0}}
\maketitle

\begin{abstract}\noindent
\textsf{childdoc} is a \LaTeXe{} package
that enables the direct compilation
of document sections included by |\include|
to individual files.
\end{abstract}

\begingroup
\parskip0ex
\tableofcontents
\endgroup

%%%%%%%%%%%%%%%%%%%%%%%%%%%%%%%%%%%%%%%%%%%%%%%%%%%%%%%%%%%%%%%%%%%%%%%%%%%%%%%%
%%%%%%%%%%%%%%%%%%%%%%%%%%%%%%%%%%%%%%%%%%%%%%%%%%%%%%%%%%%%%%%%%%%%%%%%%%%%%%%%
\section{Introduction}

\LaTeX{} provides a mechanism to structure a large document (such as a book)
into a main file and several child files (containing the chapters)
using the |\include| command.
This mechanism is beneficial for documents
which span hundreds of pages in order to
make the source file(s) more manageable.
Moreover, compilation can be restricted to
selected child files by means of the |\includeonly| command.
The latter feature can be used to reduce the compilation time while editing
(this was significantly more useful in the earlier days of \LaTeX{})
or to generate a smaller document which is easier to navigate.
Another application of |\includeonly| is to generate
documents consisting of selected parts of the complete document.

However, there are a few drawbacks of the plain |\include| mechanism:
\begin{itemize}
\item
The child files cannot be compiled on their own,
they can only be compiled via the main file.
A naive editing environment
(such as a text editor with an option
to have the current file processed by \LaTeX)
may require one to switch to the main file before compiling;
attempting to compile the child file produces errors.
\item
The main file must be modified (each time)
to adjust the |\includeonly| command
to the present needs. This easily leaves the main file in a messy state.
\item
The generated document will always carry the filename
of the main document. This is inconvenient if
several child files are to be compiled and
to be kept for distribution.
\end{itemize}

The present package provides a simple interface
to make child files individually compilable by \LaTeX{}.
Compiling a child file then has the same effect as compiling
the main file with an |\includeonly| command
to select the appropriate child.
Moreover the generated document will carry the name of the child
rather than the main file.
This resolves all three above issues.

This feature is meant to make the editing of books,
thesis documents and lecture notes somewhat more convenient.
However, the package can also be used efficiently for
composing a series of documents (such as exercise sheets)
which are typically distributed individually.
It then assists the author in generating the individual documents
(potentially in different versions)
as well as a document containing the collected series.
Another application is in developing style files
or other kinds of included material
where compilation of the style file could redirect
to a sample or test file.

%%%%%%%%%%%%%%%%%%%%%%%%%%%%%%%%%%%%%%%%%%%%%%%%%%%%%%%%%%%%%%%%%%%%%%%%%%%%%%%%
%%%%%%%%%%%%%%%%%%%%%%%%%%%%%%%%%%%%%%%%%%%%%%%%%%%%%%%%%%%%%%%%%%%%%%%%%%%%%%%%
\section{Usage}

First of all, the package \textsf{childdoc} is \emph{not} a standard
\LaTeXe{} |.sty| style file! Therefore it needs to be invoked in
a non-standard way.

%%%%%%%%%%%%%%%%%%%%%%%%%%%%%%%%%%%%%%%%%%%%%%%%%%%%%%%%%%%%%%%%%%%%%%%%%%%%%%%%
\subsection{Included Files}
\label{sec:include}

%%%%%%%%%%%%%%%%%%%%%%%%%%%%%%%%%%%%%%%%
\DescribeMacro{\childdocmain}
To use the package, add the commands
\begin{center}
\begin{tabular}{l}
|\input{childdoc.def}|\\
|\childdocmain{}|\\
\end{tabular}
\end{center}
at the very top of the main \LaTeX{} file,
in particular \emph{before} the |\documentclass| statement!
The argument of |\childdocmain| should be left empty
(but it must be present).

%%%%%%%%%%%%%%%%%%%%%%%%%%%%%%%%%%%%%%%%
\DescribeMacro{\childdocof}
Furthermore, add the commands
\begin{center}
\begin{tabular}{l}
|\input{childdoc.def}|\\
|\childdocof{|\textit{main}|}|\\
\end{tabular}
\end{center}
at the top of every child file \textit{child}
which is included by |\include{|\textit{child}|}|
from within the main file
(or at least for those files to be compiled individually).
The argument \textit{main} must be the filename of the main file.

There are a couple of
considerations in setting up the main and child documents:

%%%%%%%%%%%%%%%%%%%%%%%%%%%%%%%%%%%%%%%%
\paragraph{Restrictions.}

Please note the following restrictions:
\begin{itemize}
\item
|\childdocmain| must be called with one argument \textit{main}
to ensure compatibility with earlier version of the package.
It must either be empty (|\childdocmain{}|)
or precisely match the filename of the main file in which it is specified.
See \secref{sec:detection} for further information.
\item
The filename \textit{main} must be specified without the |.tex| extension.
\item
The filename \textit{main} is case sensitive
(even in case-insensitive file systems)
due to internal string comparison.
\item
The argument \textit{main} should be fully expanded, it cannot be a macro.
\item
Subdirectories and special characters should be avoided in filenames.
\item
The command |\childdocmain{|\textit{main}|}| must be followed by a whitespace.
It should not be followed immediately by another command
or by a comment mark `|%|'.
This is because the \TeX{} parser reads the token immediately following
the argument of |\childdocmain| and puts it
at the beginning of every child section;
however, a white\-space is ignored.
\end{itemize}

%%%%%%%%%%%%%%%%%%%%%%%%%%%%%%%%%%%%%%%%
\paragraph{Content of Main File.}

It is advisable to place all content in the child files included by |\include|.
Any output contained in the main file will appear in all child documents
unless suppressed manually;
it cannot be suppressed automatically by the |\includeonly| directive
and thus should normally be avoided.
A method to include some content in the main file
by means of conditional processing is described in \secref{sec:conditional}.

%%%%%%%%%%%%%%%%%%%%%%%%%%%%%%%%%%%%%%%%
\paragraph{Page Numbering.}

When only a part of the document is compiled,
the appropriate numbering of pages
(as well as other status parameters)
is determined from the |.aux| files.
The latter contain information from previous passes.
However this information needs to propagate through
all intermediate child documents.
Therefore the page numbering in child documents may well
be inconsistent until the complete document is compiled at least once.

A useful (if unconventional) way to always ensure a consistent
page numbering is to restart the numbering in each child document
and denote the pages by `\textit{child}|.|\textit{page}'
where \textit{child} represents the chapter/section number of the child file.
This can be achieved by the command
|\numberwithin{page}{|\textit{child}|}|
of the \textsf{amsmath} package
where \textit{child} can be |chapter| or |section|
depending on the chosen structuring.
Alternatively, one can modify the macro |\thepage| appropriately
and reset the counter |page| at the start of each child file.

%%%%%%%%%%%%%%%%%%%%%%%%%%%%%%%%%%%%%%%%%%%%%%%%%%%%%%%%%%%%%%%%%%%%%%%%%%%%%%%%
\subsection{Conditional Processing}
\label{sec:conditional}

The package provides a mechanism to compile different versions
of a document. To customise the versions further some conditional processing
can come in handy to distinguish which version is being compiled.
The package provides two macros to describe the compilation context:

%%%%%%%%%%%%%%%%%%%%%%%%%%%%%%%%%%%%%%%%
\DescribeMacro{\ifchilddoc}
The conditional |\ifchilddoc| distinguishes between the compilation of
child documents and the main document:
%
\begin{center}
|\ifchilddoc |\textit{child-code}| |[|\||else |\textit{main-code}]| \||fi|
\end{center}

%%%%%%%%%%%%%%%%%%%%%%%%%%%%%%%%%%%%%%%%
\DescribeMacro{\childdocname}
\DescribeMacro{\childdocjob}
The macro |\childdocname| contains the filename (without extension)
of the main or child file being processed.
Note that |\childdocjob| will always contain the name of the main file.

%%%%%%%%%%%%%%%%%%%%%%%%%%%%%%%%%%%%%%%%
\paragraph{Title Page.}

Conditional processing can be used to include a title or banner page
in the main document when proper precautions are taken.
Importantly, the code in the main file should ensure that the page counter
(as well as other status parameters which are stored in the |.aux| files)
takes the same value after the conditional processing.
Otherwise the page numbers may take divergent values
depending on which part is compiled.

For example, a title page could be declared by:
%
\begin{center}
\begin{tabular}{l}
|\ifchilddoc\||else|\\
|\addtocounter{page}{-1}|\\
\textit{code for title page}\\
|\newpage|\\
|\||fi|
\end{tabular}
\end{center}
%
A banner page for the child documents can be generated by:
%
\begin{center}
\begin{tabular}{l}
|\ifchilddoc|\\
|\addtocounter{page}{-1}|\\
\textit{code for banner page}\\
|\newpage|\\
|\||fi|
\end{tabular}
\end{center}
%
Here one could write a message such as:
\begin{center}
|This is the part \childdocname{} of \childdocjob{}.|
\end{center}

%%%%%%%%%%%%%%%%%%%%%%%%%%%%%%%%%%%%%%%%%%%%%%%%%%%%%%%%%%%%%%%%%%%%%%%%%%%%%%%%
\subsection{Flags}
\label{sec:flags}

The package makes it easy to generate different versions
of the main or child documents.
To this end compilation flags can be defined
and assigned different default values.
They will be particularly useful in conjunction
with the forwarding mechanism described in \secref{sec:forward}.

For example, it may be useful to have a flag |\version|
which can be set to |draft| or |final|.
The document source will contain some conditional code
depending on the value of |\version|.
Suppose further, the flag should default to |final| for the main file
and to |draft| for child files
which is a natural assignment for editing the document.
This is achieved by placing the following code
in the preamble of the main document
(below the |\childdocmain| directive):
%
\begin{center}
\begin{tabular}{l}
|\ifchilddoc|\\
|\providecommand{\version}{draft}|\\
|\||else|\\
|\providecommand{\version}{final}|\\
|\||fi|
\end{tabular}
\end{center}
%
The definition by |\providecommand| makes sure
that previous definitions are not overwritten.
Further statements |\providecommand{\version}{...}|
can thus be added before the above code to override it.

For the main file, one might add a line
(between |\childdocmain| and the above block)
%
\begin{center}
|%\ifchilddoc\||else\providecommand{\version}{draft}\||fi|
\end{center}
%
which can be uncommented to produce a draft version.
Likewise one can add a line to the very top of a child file
(above the |\childdocof{|\textit{main}|}| directive)
%
\begin{center}
|%\providecommand{\version}{final}|
\end{center}
%
which can be uncommented to produce the final version of this child document.

%%%%%%%%%%%%%%%%%%%%%%%%%%%%%%%%%%%%%%%%%%%%%%%%%%%%%%%%%%%%%%%%%%%%%%%%%%%%%%%%
\subsection{Forwarding}
\label{sec:forward}

Different versions of the main or child documents
using compilation flags as described in \secref{sec:flags}
can be (permanently) stored in different files
for convenient compilation, viewing and distribution.
To this end, the package defines a command
to pass on compilation to a different file:

%%%%%%%%%%%%%%%%%%%%%%%%%%%%%%%%%%%%%%%%
\DescribeMacro{\childdocforward}
The command |\childdocforward| redirects processing to
another source file:
%
\begin{center}
\begin{tabular}{l}
|\input{childdoc.def}|\\
|\childdocforward[|\textit{main}|]{|\textit{dest}|}|\\
\end{tabular}
\end{center}
%
The argument \textit{dest} is the destination file
(without extension).
It should be the main file or one of the child files.
Note that further \textsf{childdoc} directives
such as |\childdocof| and |\childdocforward|
in the indicated file will be processed in this form.
The optional argument \textit{main}
passes on directly to the main file \textit{main}
while pretending to compile the child \textit{dest}.
This form behaves as if \textit{dest}
issues |\childdocof{|\textit{main}|}| right away,
and no further \textsf{childdoc} directives will be processed.

%%%%%%%%%%%%%%%%%%%%%%%%%%%%%%%%%%%%%%%%
\DescribeMacro{\...prefix}
In the alternative form |\childdocforwardprefix|,
%
\begin{center}
\begin{tabular}{l}
|\input{childdoc.def}|\\
|\childdocforwardprefix[|\textit{main}|]{|\textit{prefix}|}{|\textit{dest}|}|
\end{tabular}
\end{center}
%
the destination file is determined by a pattern
depending on the current file:
To make this work, the current file must be called
`{\textit{prefix}\hspace{0.2em}\textit{suffix}}'
with \textit{prefix} matching precisely the argument.
Processing is then passed on to the file
`{\textit{dest}\hspace{0.2em}\textit{suffix}}'.
Surely, the same effect is achieved by
directly specifying the
argument `{\textit{dest}\hspace{0.2em}\textit{suffix}}'
in the first form.
However, that requires to set up a different file
for each child. With the alternative form of the command
all these files can have exactly the same content
which simplifies setting them up and maintaining them.

For example, the following file |draft.tex|
with a compilation flag |\version| as described in \secref{sec:flags}
compiles the main document as a draft:
%
\begin{center}
\begin{tabular}{l}
|\def\version{draft}|\\
|\input{childdoc.def}|\\
|\childdocforward{|\textit{main}|}|
\end{tabular}
\end{center}
%
Likewise, the following files |final|\textit{nn}|.tex|
compile the final version of the child document
|child|\textit{nn}|.tex|:
%
\begin{center}
\begin{tabular}{l}
|\def\version{final}|\\
|\input{childdoc.def}|\\
|\childdocforwardprefix{final}{child}|
\end{tabular}
\end{center}
%

Note that when several versions of a main file and/or of each child file
are to be generated, it may be convenient to set up a |Makefile| or
shell script to automatise the process.

%%%%%%%%%%%%%%%%%%%%%%%%%%%%%%%%%%%%%%%%%%%%%%%%%%%%%%%%%%%%%%%%%%%%%%%%%%%%%%%%
\subsection{Command Line Processing}
\label{sec:commandline}

The effect of redirection files can also be achieved by invoking
the \LaTeX{} compiler with a more elaborate command line.
Most conveniently this should be done as part
of a shell script or a |Makefile|.

When using \textsf{childdoc} in the main file, the following
command lines effectively perform a redirection
(note that depending on the shell being used,
backslashes may have to be doubled: `|\|' $\to$ `|\\|'):
%
\begin{center}
|... -jobname "|\textit{target}|" |\\|"|[\textit{flags}]%
|\input{childdoc.def}\childdocforward[|\textit{main}|]{|\textit{dest}|}"|
\end{center}
%
Here \textit{target} is the name of the output file,
\textit{main} is the name of the main file
and \textit{dest} is the name of the main or child file to be processed
(all filenames without extensions).
The optional argument \textit{main} can be omitted
if \textit{main} matches \textit{dest}.
Optionally, compilation \textit{flags} can be defined via |\def| commands.
This command line makes the \TeX{} engine believe
it is compiling the file \textit{target}
whose content is specified as the latter parameter.
The provided code then forwards the processing to
\textit{main} or \textit{dest} as described in \secref{sec:forward}.

%%%%%%%%%%%%%%%%%%%%%%%%%%%%%%%%%%%%%%%%%%%%%%%%%%%%%%%%%%%%%%%%%%%%%%%%%%%%%%%%
\subsection{Include by Input}
\label{sec:input}

Including child documents by |\include| has some restrictions by design.
Most notably, the content of a child document always occupies
its own set of pages; pages cannot be shared between child documents.
Usually, this behaviour makes perfect sense
because each child document contain an essential part of the document.
However, in some situations it may be desirable to compose
a document from a collection of parts
without having mandatory page breaks between then.
For this case, the package
provides a mechanism to include parts
by |\input| which can also be processed individually.
However, by construction this mechanism
requires manual handling of the content to be output.

%%%%%%%%%%%%%%%%%%%%%%%%%%%%%%%%%%%%%%%%
\DescribeMacro{\ifchilddocmanual}
The main file should be prepared as usual, see \secref{sec:include}.
However, the document body must make a distinction
between processing of an individual part and of the main document, e.g.:
%
\begin{center}
\begin{tabular}{l}
|\ifchilddocmanual|\\
|\input{\childdocname}|\\
|\||else|\\
\textit{document body with }|\input{|\textit{part}|}|\\
|\||fi|
\end{tabular}
\end{center}
%
The conditional |\ifchilddocmanual| is true whenever
a part to be included by |\input| is being compiled,
and the name of the part is stored in |\childdocname|.

%%%%%%%%%%%%%%%%%%%%%%%%%%%%%%%%%%%%%%%%
\DescribeMacro{\childdocby}
Each part to be included by |\input| should start with:
%
\begin{center}
\begin{tabular}{l}
|\input{childdoc.def}|\\
|\childdocby{|\textit{main}|}|\\
\end{tabular}
\end{center}
%
The directive |\childdocby| is similar to |\childdocof|
described in \secref{sec:include},
but the subsequent selection of content must be done manually.
To that end, both |\ifchilddoc| and |\ifchilddocmanual|
will be true upon processing of a part,
and the name of the part is stored in |\childdocname|.
Note that |\jobname| will be set to the filename of the current part
so that each part receives an individual |.aux| file
that does not interfere with the |.aux| file(s) of the main document.
This behaviour can be altered by the alternative form
|\childdocby[*]{|\textit{main}|}| (with a non-empty optional argument)
which uses the |.aux| file of the main document
by setting |\jobname| to \textit{main}.

%%%%%%%%%%%%%%%%%%%%%%%%%%%%%%%%%%%%%%%%%%%%%%%%%%%%%%%%%%%%%%%%%%%%%%%%%%%%%%%%
\subsection{Driver Development}
\label{sec:driver}

The \textsf{childdoc} mechanism can also be use for the development
of definition files such as \LaTeX{} styles or classes.
This case differs from the above setup with multiple parts
included by |\include| in that no |\includeonly| should be invoked.
This can be achieved by starting the include file
(before |\ProvidesPackage|) with:
%
\begin{center}
\begin{tabular}{l}
|\input{childdoc.def}|\\
|\childdocforward{|\textit{main}|}|\\
\end{tabular}
\end{center}
%
or alternatively with:
%
\begin{center}
\begin{tabular}{l}
|\input{childdoc.def}|\\
|\childdocby{|\textit{main}|}|\\
\end{tabular}
\end{center}
%
Both forms have slightly different effects as described above.
The main file is prepared as usual, see \secref{sec:include}.

%%%%%%%%%%%%%%%%%%%%%%%%%%%%%%%%%%%%%%%%%%%%%%%%%%%%%%%%%%%%%%%%%%%%%%%%%%%%%%%%
\subsection{Legacy Detection}
\label{sec:detection}

The directive |\childdocmain| in the main file can detect
whether the complete document or merely a child is to be compiled
even without using the directive |\childdocof|.
This method is deprecated because it is less robust
and there is no compelling reason to use it;
it is merely provided for backward compatibility
and it may be removed in future versions.

If the detection mechanism is to be used,
it is mandatory to correctly specify
the filename of the main file as the argument of |\childdocmain|:
%
\begin{center}
\begin{tabular}{l}
|\input{childdoc.def}|\\
|\childdocmain{|\textit{main}|}|\\
\end{tabular}
\end{center}
%
If |\jobname| does not match the argument \textit{main} of |\childdocmain|,
it is assumed that |\jobname| points to the child file to be compiled.
When using |\childdocmain| with the main file specified as argument,
it suffices to start a child file
with just |\input{|\textit{main}|}|
without loading of the package and using |\childdocof|.
If instead all processing is done
with the appropriate \textsf{childdoc} directives,
the argument of \textit{main} of |\childdocmain| can be empty.

An alternative version of the command line processing described
in \secref{sec:commandline} using the detection mechanism reads:
%
\begin{center}
|... -jobname "|\textit{target}|" "|[\textit{flags}]%
[|\def\jobname{|\textit{dest}|}|]|\input{|\textit{main}|}"|
\end{center}

%%%%%%%%%%%%%%%%%%%%%%%%%%%%%%%%%%%%%%%%%%%%%%%%%%%%%%%%%%%%%%%%%%%%%%%%%%%%%%%%
\subsection{Manual Code}
\label{sec:manual}

In case one cannot be certain whether the definitions file |childdoc.def|
is installed on the target \TeX{} distribution
and one prefers not to ship it,
it is conceivable to paste a few relevant commands into the sources.

To that end, drop all statements |\input{childdoc.def}|
and perform the replacements as outlined below.
Instead of |\childdocmain{|\textit{main}|}| add the following code
to the top of the main file:
%
\begin{center}
\begin{tabular}{l}
|\||ifdefined\childdocname\endinput\||fi\newif\ifchilddoc|\\
|\edef\childdocname{\scantokens\expandafter{\jobname\noexpand}}|\\
|\def\childdocmain{|\textit{main}|}\||ifx\childdocmain\childdocname\||else|\\
|\childdoctrue\includeonly{\childdocname}\let\jobname\childdocmain\||fi|\\
\end{tabular}
\end{center}
%
Instead of |\childdocof{|\textit{main}|}| just include the main file
at the top of each child file:
%
\begin{center}
|\input{|\textit{main}|}|
\end{center}
%
A simple redirection |\childdocforward{|\textit{dest}|}| is achieved by:
%
\begin{center}
|\def\jobname{|\textit{dest}|}\input{\jobname}|
\end{center}
%
The redirection with prefix
|\childdocforwardprefix[|\textit{prefix}|]{|\textit{dest}|}|
is accomplished by:
%
\begin{center}
\begin{tabular}{l}
|{\edef\jobname{\scantokens\expandafter{\jobname\noexpand}}|\\
|\def\redirectjob |\textit{prefix}|#1~~~{\gdef\jobname{|\textit{dest}|#1}}|\\
|\expandafter\redirectjob\jobname~~~}\input{\jobname}|
\end{tabular}
\end{center}

In an alternative approach,
child documents can be compiled by a specific command line
without additional code or specific definitions:
%
\begin{center}
|... -jobname "|\textit{target}|" "|[\textit{flags}]%
|\includeonly{|\textit{dest}|}\input{|\textit{main}|}"|
\end{center}
%

%%%%%%%%%%%%%%%%%%%%%%%%%%%%%%%%%%%%%%%%%%%%%%%%%%%%%%%%%%%%%%%%%%%%%%%%%%%%%%%%
%%%%%%%%%%%%%%%%%%%%%%%%%%%%%%%%%%%%%%%%%%%%%%%%%%%%%%%%%%%%%%%%%%%%%%%%%%%%%%%%
\section{Information}

%%%%%%%%%%%%%%%%%%%%%%%%%%%%%%%%%%%%%%%%%%%%%%%%%%%%%%%%%%%%%%%%%%%%%%%%%%%%%%%%
\subsection{Copyright}

Copyright \copyright{} 2017--2018 Niklas Beisert

This work may be distributed and/or modified under the
conditions of the \LaTeX{} Project Public License, either version 1.3
of this license or (at your option) any later version.
The latest version of this license is in
  \url{http://www.latex-project.org/lppl.txt}
and version 1.3 or later is part of all distributions of \LaTeX{}
version 2005/12/01 or later.

This work has the LPPL maintenance status `maintained'.

The Current Maintainer of this work is Niklas Beisert.

This work consists of the files |README.txt|, |childdoc.ins| and |childdoc.dtx|
as well as the derived files |childdoc.def|, |cdocsamp.tex|
with |cdocsch1.tex|, |cdocsch2.tex|, |cdocspt3.tex|, |cdocspt4.tex|,
|cdocsdrf.tex|, |cdocsfn1.tex|, |cdocsfn2.tex|
as well as |childdoc.pdf|.

%%%%%%%%%%%%%%%%%%%%%%%%%%%%%%%%%%%%%%%%%%%%%%%%%%%%%%%%%%%%%%%%%%%%%%%%%%%%%%%%
\subsection{Files and Installation}

The package consists of the files:
%
\begin{center}
\begin{tabular}{ll}
    |README.txt|   & readme file \\
    |childdoc.ins| & installation file \\
    |childdoc.dtx| & source file \\
    |childdoc.def| & definition file \\
    |cdocsamp.tex| & sample main file \\
    |cdocsch1.tex| & sample include file \\
    |cdocsch2.tex| & sample include file \\
    |cdocspt3.tex| & sample part file \\
    |cdocspt4.tex| & sample part file \\
    |cdocsdrf.tex| & sample redirection file \\
    |cdocsfn1.tex| & sample redirection file \\
    |cdocsfn2.tex| & sample redirection file \\
    |childdoc.pdf| & manual
\end{tabular}
\end{center}
%
The distribution consists of the files
|README.txt|, |childdoc.ins| and |childdoc.dtx|.
%
\begin{itemize}
\item
Run (pdf)\LaTeX{} on |childdoc.dtx|
to compile the manual |childdoc.pdf| (this file).
\item
Run \LaTeX{} on |childdoc.ins| to create the definitions file |childdoc.def|
and the sample |cdocsamp.tex| with include files
|cdocsch1.tex|, |cdocsch2.tex|, |cdocspt3.tex|, |cdocspt4.tex|,
|cdocsdrf.tex|, |cdocsfn1.tex|, |cdocsfn2.tex|.
Then copy the file |childdoc.def| to an appropriate directory of your \LaTeX{}
distribution, e.g.\ \textit{texmf-root}|/tex/latex/childdoc|.
\end{itemize}

%%%%%%%%%%%%%%%%%%%%%%%%%%%%%%%%%%%%%%%%%%%%%%%%%%%%%%%%%%%%%%%%%%%%%%%%%%%%%%%%
\subsection{Related CTAN Packages}

There are several other packages which offer a similar functionality:
%
\begin{itemize}
\item
The packages
\href{http://ctan.org/pkg/docmute}{\textsf{docmute}},
\href{http://ctan.org/pkg/includex}{\textsf{includex}} and
\href{http://ctan.org/pkg/standalone}{\textsf{standalone}}
provide commands to include only the document body of
a child file thus allowing both files to be compiled individually.
\item
The packages \href{http://ctan.org/pkg/subdocs}{\textsf{subdocs}}
and \href{http://ctan.org/pkg/subfiles}{\textsf{subfiles}}
provide structures in which the main and child documents can be
encapsulated and allowing them to be compiled individually.
The inclusion mechanism is different from the conventional |\include|.
\item
The package \href{http://ctan.org/pkg/combine}{\textsf{combine}}
is an elaborate solution to combine several documents into one.
\end{itemize}
%
See also the CTAN topic \href{http://ctan.org/topic/subdocs}{\textsf{subdocs}}
for further related packages.
The present package differs from the above solutions in that
a document structure constructed with the conventional |\include| mechanism
just needs two extra commands at the top of every file
such that all constituent files can be compiled individually.

%%%%%%%%%%%%%%%%%%%%%%%%%%%%%%%%%%%%%%%%%%%%%%%%%%%%%%%%%%%%%%%%%%%%%%%%%%%%%%%%
%\subsection{Feature Suggestions}
%
%The following is a list of features which may be useful for future
%versions of this package:
%%
%\begin{itemize}
%\item
%\ldots
%\end{itemize}

%%%%%%%%%%%%%%%%%%%%%%%%%%%%%%%%%%%%%%%%%%%%%%%%%%%%%%%%%%%%%%%%%%%%%%%%%%%%%%%%
\subsection{Revision History}

%%%%%%%%%%%%%%%%%%%%%%%%%%%%%%%%%%%%%%%%
\paragraph{v2.0:} 2018/12/30

\begin{itemize}
\item
immediate forward processing
\item
added |\childdocby| mechanism
\item
manual restructured
\end{itemize}

%%%%%%%%%%%%%%%%%%%%%%%%%%%%%%%%%%%%%%%%
\paragraph{v1.6:} 2018/01/17

\begin{itemize}
\item
application for development of include files
\item
corrections to manual
\end{itemize}

%%%%%%%%%%%%%%%%%%%%%%%%%%%%%%%%%%%%%%%%
\paragraph{v1.5:} 2017/05/21

\begin{itemize}
\item
more complete structuring introduced
\item
|\childdocof| introduced
\item
|\childdoc| renamed to |\childdocmain|
\item
|\childredirect| renamed to |\childdocforward| and |\childdocforwardprefix|
and functionality expanded
\end{itemize}

%%%%%%%%%%%%%%%%%%%%%%%%%%%%%%%%%%%%%%%%
\paragraph{v1.0:} 2017/04/27

\begin{itemize}
\item
manual and install package
\item
first version published on CTAN
\end{itemize}

%%%%%%%%%%%%%%%%%%%%%%%%%%%%%%%%%%%%%%%%
\paragraph{v0.6:} 2017/04/26

\begin{itemize}
\item
redirection mechanism added
\end{itemize}

%%%%%%%%%%%%%%%%%%%%%%%%%%%%%%%%%%%%%%%%
\paragraph{v0.5:} 2017/04/26

\begin{itemize}
\item
functionality in definition file
\end{itemize}


%%%%%%%%%%%%%%%%%%%%%%%%%%%%%%%%%%%%%%%%%%%%%%%%%%%%%%%%%%%%%%%%%%%%%%%%%%%%%%%%
%%%%%%%%%%%%%%%%%%%%%%%%%%%%%%%%%%%%%%%%%%%%%%%%%%%%%%%%%%%%%%%%%%%%%%%%%%%%%%%%
%%%%%%%%%%%%%%%%%%%%%%%%%%%%%%%%%%%%%%%%%%%%%%%%%%%%%%%%%%%%%%%%%%%%%%%%%%%%%%%%
\appendix

\settowidth\MacroIndent{\rmfamily\scriptsize 000\ }

 \DocInput{childdoc.dtx}

\end{document}
%</driver>
% \fi
%
% %%%%%%%%%%%%%%%%%%%%%%%%%%%%%%%%%%%%%%%%%%%%%%%%%%%%%%%%%%%%%%%%%%%%%%%%%%%%%%
% %%%%%%%%%%%%%%%%%%%%%%%%%%%%%%%%%%%%%%%%%%%%%%%%%%%%%%%%%%%%%%%%%%%%%%%%%%%%%%
% \section{Sample}
%\iffalse
%<*samplemain>
%\fi
%
% The following presents a sample document
% with two chapters, two parts, a title page,
% a compile flag as well as three forwarding files to set the flag.
% It consists of eight |.tex| files:
% \begin{center}
% \begin{tabular}{ll}
% |cdocsamp.tex|&main file\\
% |cdocsch1.tex|&include file for chapter 1\\
% |cdocsch2.tex|&include file for chapter 2\\
% |cdocspt3.tex|&include file for part 3\\
% |cdocspt4.tex|&include file for part 4\\
% |cdocsdrf.tex|&forwarding file for main file in draft mode\\
% |cdocsfi1.tex|&forwarding file for final version of chapter 1\\
% |cdocsfi2.tex|&forwarding file for final version of chapter 2\\
% \end{tabular}
% \end{center}
% Each of the eight files can be compiled directly by the \LaTeX{} compiler.
%
% %%%%%%%%%%%%%%%%%%%%%%%%%%%%%%%%%%%%%%
% \paragraph{Main File.}
%
% The main file is called |cdocsamp.tex|.
%
% Load the \textsf{childdoc} definitions and
% declare the filename for the main document:
%    \begin{macrocode}
\input{childdoc.def}
\childdocmain{}
%    \end{macrocode}

% Optional override for |\version| flag:
%    \begin{macrocode}
%%\ifchilddoc\else\providecommand{\version}{draft}\fi
%    \end{macrocode}

% Define the default values for the |\version| flag
% (|final| for the main file and |draft| for childs):
%    \begin{macrocode}
\ifchilddoc
\providecommand{\version}{draft}
\else
\providecommand{\version}{final}
\fi
%    \end{macrocode}

% Load the standard document class:
%    \begin{macrocode}
\documentclass[12pt]{article}
%    \end{macrocode}

% Start the document body:
%    \begin{macrocode}
\begin{document}
%    \end{macrocode}

% Declare a title page.
% Print title, part of document being processed and version flag:
%    \begin{macrocode}
\addtocounter{page}{-1}
\begin{center}
{\LARGE\bfseries{}childdoc example\par}
\vspace{1cm}
\ifchilddoc
\ifchilddocmanual part\else chapter\fi:
`\childdocname' of `\childdocjob'\par
\else
main document: `\childdocjob'\par
\fi
version: \version\par
\end{center}
\newpage
%    \end{macrocode}

% Manually include selected file,
% otherwise process as usual:
%    \begin{macrocode}
\ifchilddocmanual
\section*{part `\childdocname'}
\input{\childdocname}
\else
%    \end{macrocode}

% Include the two chapters:
%    \begin{macrocode}
\include{cdocsch1}
\include{cdocsch2}
%    \end{macrocode}

% Include the two parts unless only chapters should be displayed:
%    \begin{macrocode}
\ifchilddoc\else
\section{part three}
\input{cdocspt3}
\section{part four}
\input{cdocspt4}
\fi
%    \end{macrocode}

% Process as usual until here:
%    \begin{macrocode}
\fi
%    \end{macrocode}

% End of document body:
%    \begin{macrocode}
\end{document}
%    \end{macrocode}
%\iffalse
%</samplemain>
%\fi
%
% %%%%%%%%%%%%%%%%%%%%%%%%%%%%%%%%%%%%%%
% \paragraph{Chapter Include Files.}
%
% The include files are called |cdocsch1.tex| and |cdocsch2.tex|.
%
%\iffalse
%<*samplechap1|samplechap2>
%\fi

% Optional override for |\version| flag:
%    \begin{macrocode}
%%\providecommand{\version}{final}
%    \end{macrocode}

% Include the main document:
%    \begin{macrocode}
\input{childdoc.def}
\childdocof{cdocsamp}
%    \end{macrocode}

%\iffalse
%</samplechap1|samplechap2>
%\fi
%
%\iffalse
%<*samplechap1>
%\fi
% Some text for chapter 1:
%    \begin{macrocode}
\section{one}
some text in chapter one
%    \end{macrocode}

%\iffalse
%</samplechap1>
%\fi
% Some text for chapter 2:
%\iffalse
%<*samplechap2>
%\fi
%    \begin{macrocode}
\section{two}
more text in chapter two
%    \end{macrocode}

%\iffalse
%</samplechap2>
%\fi
%
% %%%%%%%%%%%%%%%%%%%%%%%%%%%%%%%%%%%%%%
% \paragraph{Part Include Files.}
%
% The include files are called |cdocspt3.tex| and |cdocspt4.tex|.
%
%\iffalse
%<*samplepart3|samplepart4>
%\fi

% Optional override for |\version| flag:
%    \begin{macrocode}
%%\providecommand{\version}{final}
%    \end{macrocode}

% Include the main document:
%    \begin{macrocode}
\input{childdoc.def}
\childdocby{cdocsamp}
%    \end{macrocode}

%\iffalse
%</samplepart3|samplepart4>
%\fi
%
%\iffalse
%<*samplepart3>
%\fi
% Some text for part 3:
%    \begin{macrocode}
some text in part three
%    \end{macrocode}

%\iffalse
%</samplepart3>
%\fi
% Some text for part 4:
%\iffalse
%<*samplepart4>
%\fi
%    \begin{macrocode}
more text in part four
%    \end{macrocode}

%\iffalse
%</samplepart4>
%\fi
%
% %%%%%%%%%%%%%%%%%%%%%%%%%%%%%%%%%%%%%%
% \paragraph{Forwarding for a Complete Draft.}
%
% The following forwarding file |cdocsdrf.tex|
% compiles the main document in draft mode:
%\iffalse
%<*sampledraft>
%\fi
%    \begin{macrocode}
\def\version{draft}
\input{childdoc.def}
\childdocforward{cdocsamp}
%    \end{macrocode}

%\iffalse
%</sampledraft>
%\fi
%
% %%%%%%%%%%%%%%%%%%%%%%%%%%%%%%%%%%%%%%
% \paragraph{Forwarding for Final Version of the Chapters.}
%
% The following forwarding files |cdocsfn1.tex| and |cdocsfn2.tex|
% (with identical content)
% compile the final versions of the child documents
% |cdocsch1.tex| and |cdocsch2.tex|, respectively:
%\iffalse
%<*samplefinal>
%\fi
%    \begin{macrocode}
\def\version{final}
\input{childdoc.def}
\childdocforwardprefix[cdocsamp]{cdocsfn}{cdocsch}
%    \end{macrocode}

%\iffalse
%</samplefinal>
%\fi
%
% %%%%%%%%%%%%%%%%%%%%%%%%%%%%%%%%%%%%%%
% \paragraph{Command Line Processing.}
%
% The following three command lines generate the output files
% |cdocscld|, |cdocscl1| and |cdocscl2|
% which should be identical to
% |cdocsdrf|, |cdocsch1| and |cdocsfn2|, respectively:
% \begin{center}
% \begin{tabular}{l}
% |latex -jobname cdocscld \|\\
% |  "\def\version{draft}\input{childdoc.def}\childdocforward{cdocsamp}"|\\
% |latex -jobname cdocscl1 \|\\
% |  "\input{childdoc.def}\childdocforward[cdocsamp]{cdocsch1}"|\\
% |latex -jobname cdocscl2 \|\\
% |  "\def\version{final}\input{childdoc.def}\childdocforward{cdocsch2}"|
% \end{tabular}
% \end{center}
% Note that the trailing backslash on each first line
% merely continues the input to the second line
% (for convenient cut ant paste).
% Furthermore, the command |latex| can be replaced by any
% of its alternative versions such as |pdflatex|.
%
% %%%%%%%%%%%%%%%%%%%%%%%%%%%%%%%%%%%%%%%%%%%%%%%%%%%%%%%%%%%%%%%%%%%%%%%%%%%%%%
% %%%%%%%%%%%%%%%%%%%%%%%%%%%%%%%%%%%%%%%%%%%%%%%%%%%%%%%%%%%%%%%%%%%%%%%%%%%%%%
% \section{Implementation}
%\iffalse
%<*package>
%\fi
%
% This section describes the definitions file |childdoc.def|.

% The definitions cannot be loaded using |\usepackage| or |\RequirePackage|
% which has a mechanism to prevent loading a style file more than once.
% When loading the definitions by means of |\input|
% multiple instances have to be prevented manually:
%\iffalse
%This code needs to be before the `\ProvidesFile' directive
%which is defined at the beginning of this file.
%Therefore it is also placed there and commented out here.
%</package>
%<*discard>
%\fi
%    \begin{macrocode}
\ifdefined\childdocmain\endinput\fi
%    \end{macrocode}
%\iffalse
%</discard>
%<*package>
%\fi
%
% \macro{\ifchilddoc}
% \macro{\ifchilddocmanual}
% The conditional |\ifchilddoc| tells whether a
% child (true) or main (false) document is being compiled.
% The conditional |\ifchilddocmanual| tells whether
% the |\includeonly| mechanism is used (false) or
% the selection of child files must be performed manually (true).
% The definitions initialise to false:
%    \begin{macrocode}
\newif\ifchilddoc
\newif\ifchilddocmanual
%    \end{macrocode}

% \macro{\childdocname}
% \macro{\childdocjob}
% The macro |\childdocname| stores the name of the main document
% to be compiled. The macro |\childdocjob| stores the name of
% the document on which the \LaTeX{} compiler was originally invoked.
% The content of |\jobname| cannot be compared
% to filenames specified in the source due to different catcodes.
% The following code rescans |\jobname|, stores the result
% in |\childdocname| and saves a copy in |\childdocjob|:
%    \begin{macrocode}
\edef\childdocname{\scantokens\expandafter{\jobname\noexpand}}
\let\childdocjob\childdocname
%    \end{macrocode}

% \macro{\childdocdisable}
% The macro |\childdocdisable| prevents the main file
% from being processed more than once.
% At this stage, the main document command |\childdocmain|
% is assumed to be called once again where it should do nothing.
% Any subsequent call to it should prevent
% a secondary processing of the main document
% It overwrites the forwarding commands
% |\childdocof| and |\childdocforward|
% with empty macros to prevent further inclusions of the main document:
%    \begin{macrocode}
\newcommand{\childdocdisable}
{
  \renewcommand{\childdocmain}[1]{\renewcommand{\childdocmain}[1]{\endinput}}
  \renewcommand{\childdocof}[1]{}
  \renewcommand{\childdocby}[2][]{}
  \renewcommand{\childdocforward}[2][]{}
  \renewcommand{\childdocdisable}{}
}
%    \end{macrocode}

% \macro{\childdocmain}
% The macro |\childdocmain| is to be called at the top of the main file
% with nothing or the main filename (without extension) as argument.
% First, it breaks loops.
% If the argument is not empty and does not match |\childdocname|
% (which is set by the first inclusion of |childdoc.def|),
% |\ifchilddoc| is set to true, |\includeonly| is applied to the child file
% and |\jobname| is set to the main file
% (for proper handling of |.aux| files):
%    \begin{macrocode}
\newcommand{\childdocmain}[1]
{
  \childdocdisable\childdocmain{}
  \if?#1?\else
    \begingroup
      \def\childdoctmp{#1}
      \ifx\childdoctmp\childdocname
        \def\childdoctmp{}
      \else
        \def\childdoctmp
        {
          \childdoctrue
          \includeonly{\childdocname}
          \def\childdocjob{#1}
          \def\jobname{#1}
        }
      \fi
      \expandafter
    \endgroup
    \childdoctmp
  \fi
}
%    \end{macrocode}

% \macro{\childdocof}
% The command |\childdocof| redirects
% compilation to the main file |#1|.
%    \begin{macrocode}
\newcommand{\childdocof}[1]
{
  \childdocdisable
  \childdoctrue
  \includeonly{\childdocname}
  \def\jobname{#1}
  \def\childdocjob{#1}
  \input{#1}
}
%    \end{macrocode}

% \macro{\childdocby}
% The command |\childdocby| ....
%    \begin{macrocode}
\newcommand{\childdocby}[2][]
{
  \childdocdisable
  \childdoctrue
  \childdocmanualtrue
  \if?#1?\else
    \def\jobname{#2}
  \fi
  \def\childdocjob{#2}
  \input{#2}
  \endinput
}
%    \end{macrocode}

% \macro{\childdocforward}
% The command |\childdocforward| redirects
% compilation to the main file or
% (if the optional argument is given) a child file.
% Parameters are set as if the main file
% or a child file starting with |\childdocof| was compiled.
% Then compilation is handed over to the main file:
%    \begin{macrocode}
\newcommand{\childdocforward}[2][]
{
  \begingroup
    \if?#1?
      \def\childdoctmp
      {
        \def\childdocname{#2}
        \def\childdocjob{#2}
        \def\jobname{#2}
        \input{#2}
        \endinput
      }
    \else
      \def\childdoctmp
      {
        \childdocdisable
        \def\childdocname{#2}
        \childdoctrue
        \includeonly{#2}
        \def\childdocjob{#1}
        \def\jobname{#1}
        \input{#1}
        \endinput
      }
    \fi
    \expandafter
  \endgroup
  \childdoctmp
}
%    \end{macrocode}

% \macro{\childdocforwardprefix}
% The command |\childdocforwardprefix| redirects
% compilation to the main or a child file by means of a pattern.
% The prefix |#1| in the current filename is replaced by |#2|
% and the suffix of the current filename is kept
% (it is assumed that the filename does not contain the substring `|~~~|'
% which is used as a delimiter).
% Compilation is handed over to the new file by |\childdocforward|:
%    \begin{macrocode}
\newcommand{\childdocforwardprefix}[3][]
{
  \begingroup
    \def\childdocextract #2##1~~~{\def\childdoctmp{\childdocforward[#1]{#3##1}}}
    \expandafter\childdocextract\childdocname~~~
    \expandafter
  \endgroup
  \childdoctmp
}
%    \end{macrocode}

% \macro{\childdoc}
% The deprecated macro |\childdoc| is a legacy version of |\childdocmain|:
%    \begin{macrocode}
\newcommand{\childdoc}{\childdocmain}
%    \end{macrocode}

% \macro{\childdocredirect}
% The deprecated macro |\childdocredirect| is a legacy version
% of |\childdocforward| and |\childdocforwardprefix|:
%    \begin{macrocode}
\newcommand{\childdocredirect}[2][]
{
  \begingroup
    \if?#1?
      \def\childdoctmp{\childdocforward{#2}}
    \else
      \def\childdoctmp{\childdocforwardprefix{#1}{#2}}
    \fi
    \expandafter
  \endgroup
  \childdoctmp
}
%    \end{macrocode}

%\iffalse
%</package>
%\fi
%
\endinput
|\\
|\childdocforwardprefix[|\textit{main}|]{|\textit{prefix}|}{|\textit{dest}|}|
\end{tabular}
\end{center}
%
the destination file is determined by a pattern
depending on the current file:
To make this work, the current file must be called
`{\textit{prefix}\hspace{0.2em}\textit{suffix}}'
with \textit{prefix} matching precisely the argument.
Processing is then passed on to the file
`{\textit{dest}\hspace{0.2em}\textit{suffix}}'.
Surely, the same effect is achieved by
directly specifying the
argument `{\textit{dest}\hspace{0.2em}\textit{suffix}}'
in the first form.
However, that requires to set up a different file
for each child. With the alternative form of the command
all these files can have exactly the same content
which simplifies setting them up and maintaining them.

For example, the following file |draft.tex|
with a compilation flag |\version| as described in \secref{sec:flags}
compiles the main document as a draft:
%
\begin{center}
\begin{tabular}{l}
|\def\version{draft}|\\
|% \iffalse
%
% childdoc.dtx Copyright (C) 2017-2018 Niklas Beisert
%
% This work may be distributed and/or modified under the
% conditions of the LaTeX Project Public License, either version 1.3
% of this license or (at your option) any later version.
% The latest version of this license is in
%   http://www.latex-project.org/lppl.txt
% and version 1.3 or later is part of all distributions of LaTeX
% version 2005/12/01 or later.
%
% This work has the LPPL maintenance status `maintained'.
%
% The Current Maintainer of this work is Niklas Beisert.
%
% This work consists of the files childdoc.dtx and childdoc.ins
% and the derived files childdoc.def and cdocsamp.tex with
% cdocsch1.tex, cdocsch2.tex, cdocsdrf.tex, cdocsfn1.tex, cdocsfn2.tex.
%
%<package>\ifdefined\childdocmain\endinput\fi
%<package>\ProvidesFile{childdoc.def}[2018/12/30 v2.0 child document driver]
%<samplemain>\ProvidesFile{cdocsamp.tex}[2018/12/30 v2.0 sample for childdoc]
%<*driver>
%\ProvidesFile{childdoc.drv}[2018/12/30 v2.0 childdoc reference manual file]
\PassOptionsToClass{10pt,a4paper}{article}
\documentclass{ltxdoc}

\usepackage[margin=35mm]{geometry}
\usepackage{hyperref}
\usepackage{hyperxmp}
\usepackage[usenames]{color}

\hypersetup{colorlinks=true}
\hypersetup{pdfstartview=FitH}
\hypersetup{pdfpagemode=UseNone}
\hypersetup{pdfsource={}}
\hypersetup{pdflang={en-UK}}
\hypersetup{pdfcopyright={Copyright 2017-2018 Niklas Beisert.
  This work may be distributed and/or modified under the
  conditions of the LaTeX Project Public License, either version 1.3
  of this license or (at your option) any later version.}}
\hypersetup{pdflicenseurl={http://www.latex-project.org/lppl.txt}}
\hypersetup{pdfcontactaddress={ETH Zurich, ITP, HIT K,
  Wolfgang-Pauli-Strasse 27}}
\hypersetup{pdfcontactpostcode={8093}}
\hypersetup{pdfcontactcity={Zurich}}
\hypersetup{pdfcontactcountry={Switzerland}}
\hypersetup{pdfcontactemail={nbeisert@itp.phys.ethz.ch}}
\hypersetup{pdfcontacturl={http://people.phys.ethz.ch/\xmptilde nbeisert/}}

\newcommand{\secref}[1]{\hyperref[#1]{section \ref*{#1}}}

\parskip1ex
\parindent0pt
\let\olditemize\itemize
\def\itemize{\olditemize\parskip0pt}

\begin{document}

\title{The \textsf{childdoc} Package}
\hypersetup{pdftitle={The childdoc Package}}
\author{Niklas Beisert\\[2ex]
  Institut f\"ur Theoretische Physik\\
  Eidgen\"ossische Technische Hochschule Z\"urich\\
  Wolfgang-Pauli-Strasse 27, 8093 Z\"urich, Switzerland\\[1ex]
  \href{mailto:nbeisert@itp.phys.ethz.ch}
  {\texttt{nbeisert@itp.phys.ethz.ch}}}
\hypersetup{pdfauthor={Niklas Beisert}}
\hypersetup{pdfsubject={Manual for the LaTeX2e Package childdoc}}
\date{30 December 2018, \textsf{v2.0}}
\maketitle

\begin{abstract}\noindent
\textsf{childdoc} is a \LaTeXe{} package
that enables the direct compilation
of document sections included by |\include|
to individual files.
\end{abstract}

\begingroup
\parskip0ex
\tableofcontents
\endgroup

%%%%%%%%%%%%%%%%%%%%%%%%%%%%%%%%%%%%%%%%%%%%%%%%%%%%%%%%%%%%%%%%%%%%%%%%%%%%%%%%
%%%%%%%%%%%%%%%%%%%%%%%%%%%%%%%%%%%%%%%%%%%%%%%%%%%%%%%%%%%%%%%%%%%%%%%%%%%%%%%%
\section{Introduction}

\LaTeX{} provides a mechanism to structure a large document (such as a book)
into a main file and several child files (containing the chapters)
using the |\include| command.
This mechanism is beneficial for documents
which span hundreds of pages in order to
make the source file(s) more manageable.
Moreover, compilation can be restricted to
selected child files by means of the |\includeonly| command.
The latter feature can be used to reduce the compilation time while editing
(this was significantly more useful in the earlier days of \LaTeX{})
or to generate a smaller document which is easier to navigate.
Another application of |\includeonly| is to generate
documents consisting of selected parts of the complete document.

However, there are a few drawbacks of the plain |\include| mechanism:
\begin{itemize}
\item
The child files cannot be compiled on their own,
they can only be compiled via the main file.
A naive editing environment
(such as a text editor with an option
to have the current file processed by \LaTeX)
may require one to switch to the main file before compiling;
attempting to compile the child file produces errors.
\item
The main file must be modified (each time)
to adjust the |\includeonly| command
to the present needs. This easily leaves the main file in a messy state.
\item
The generated document will always carry the filename
of the main document. This is inconvenient if
several child files are to be compiled and
to be kept for distribution.
\end{itemize}

The present package provides a simple interface
to make child files individually compilable by \LaTeX{}.
Compiling a child file then has the same effect as compiling
the main file with an |\includeonly| command
to select the appropriate child.
Moreover the generated document will carry the name of the child
rather than the main file.
This resolves all three above issues.

This feature is meant to make the editing of books,
thesis documents and lecture notes somewhat more convenient.
However, the package can also be used efficiently for
composing a series of documents (such as exercise sheets)
which are typically distributed individually.
It then assists the author in generating the individual documents
(potentially in different versions)
as well as a document containing the collected series.
Another application is in developing style files
or other kinds of included material
where compilation of the style file could redirect
to a sample or test file.

%%%%%%%%%%%%%%%%%%%%%%%%%%%%%%%%%%%%%%%%%%%%%%%%%%%%%%%%%%%%%%%%%%%%%%%%%%%%%%%%
%%%%%%%%%%%%%%%%%%%%%%%%%%%%%%%%%%%%%%%%%%%%%%%%%%%%%%%%%%%%%%%%%%%%%%%%%%%%%%%%
\section{Usage}

First of all, the package \textsf{childdoc} is \emph{not} a standard
\LaTeXe{} |.sty| style file! Therefore it needs to be invoked in
a non-standard way.

%%%%%%%%%%%%%%%%%%%%%%%%%%%%%%%%%%%%%%%%%%%%%%%%%%%%%%%%%%%%%%%%%%%%%%%%%%%%%%%%
\subsection{Included Files}
\label{sec:include}

%%%%%%%%%%%%%%%%%%%%%%%%%%%%%%%%%%%%%%%%
\DescribeMacro{\childdocmain}
To use the package, add the commands
\begin{center}
\begin{tabular}{l}
|\input{childdoc.def}|\\
|\childdocmain{}|\\
\end{tabular}
\end{center}
at the very top of the main \LaTeX{} file,
in particular \emph{before} the |\documentclass| statement!
The argument of |\childdocmain| should be left empty
(but it must be present).

%%%%%%%%%%%%%%%%%%%%%%%%%%%%%%%%%%%%%%%%
\DescribeMacro{\childdocof}
Furthermore, add the commands
\begin{center}
\begin{tabular}{l}
|\input{childdoc.def}|\\
|\childdocof{|\textit{main}|}|\\
\end{tabular}
\end{center}
at the top of every child file \textit{child}
which is included by |\include{|\textit{child}|}|
from within the main file
(or at least for those files to be compiled individually).
The argument \textit{main} must be the filename of the main file.

There are a couple of
considerations in setting up the main and child documents:

%%%%%%%%%%%%%%%%%%%%%%%%%%%%%%%%%%%%%%%%
\paragraph{Restrictions.}

Please note the following restrictions:
\begin{itemize}
\item
|\childdocmain| must be called with one argument \textit{main}
to ensure compatibility with earlier version of the package.
It must either be empty (|\childdocmain{}|)
or precisely match the filename of the main file in which it is specified.
See \secref{sec:detection} for further information.
\item
The filename \textit{main} must be specified without the |.tex| extension.
\item
The filename \textit{main} is case sensitive
(even in case-insensitive file systems)
due to internal string comparison.
\item
The argument \textit{main} should be fully expanded, it cannot be a macro.
\item
Subdirectories and special characters should be avoided in filenames.
\item
The command |\childdocmain{|\textit{main}|}| must be followed by a whitespace.
It should not be followed immediately by another command
or by a comment mark `|%|'.
This is because the \TeX{} parser reads the token immediately following
the argument of |\childdocmain| and puts it
at the beginning of every child section;
however, a white\-space is ignored.
\end{itemize}

%%%%%%%%%%%%%%%%%%%%%%%%%%%%%%%%%%%%%%%%
\paragraph{Content of Main File.}

It is advisable to place all content in the child files included by |\include|.
Any output contained in the main file will appear in all child documents
unless suppressed manually;
it cannot be suppressed automatically by the |\includeonly| directive
and thus should normally be avoided.
A method to include some content in the main file
by means of conditional processing is described in \secref{sec:conditional}.

%%%%%%%%%%%%%%%%%%%%%%%%%%%%%%%%%%%%%%%%
\paragraph{Page Numbering.}

When only a part of the document is compiled,
the appropriate numbering of pages
(as well as other status parameters)
is determined from the |.aux| files.
The latter contain information from previous passes.
However this information needs to propagate through
all intermediate child documents.
Therefore the page numbering in child documents may well
be inconsistent until the complete document is compiled at least once.

A useful (if unconventional) way to always ensure a consistent
page numbering is to restart the numbering in each child document
and denote the pages by `\textit{child}|.|\textit{page}'
where \textit{child} represents the chapter/section number of the child file.
This can be achieved by the command
|\numberwithin{page}{|\textit{child}|}|
of the \textsf{amsmath} package
where \textit{child} can be |chapter| or |section|
depending on the chosen structuring.
Alternatively, one can modify the macro |\thepage| appropriately
and reset the counter |page| at the start of each child file.

%%%%%%%%%%%%%%%%%%%%%%%%%%%%%%%%%%%%%%%%%%%%%%%%%%%%%%%%%%%%%%%%%%%%%%%%%%%%%%%%
\subsection{Conditional Processing}
\label{sec:conditional}

The package provides a mechanism to compile different versions
of a document. To customise the versions further some conditional processing
can come in handy to distinguish which version is being compiled.
The package provides two macros to describe the compilation context:

%%%%%%%%%%%%%%%%%%%%%%%%%%%%%%%%%%%%%%%%
\DescribeMacro{\ifchilddoc}
The conditional |\ifchilddoc| distinguishes between the compilation of
child documents and the main document:
%
\begin{center}
|\ifchilddoc |\textit{child-code}| |[|\||else |\textit{main-code}]| \||fi|
\end{center}

%%%%%%%%%%%%%%%%%%%%%%%%%%%%%%%%%%%%%%%%
\DescribeMacro{\childdocname}
\DescribeMacro{\childdocjob}
The macro |\childdocname| contains the filename (without extension)
of the main or child file being processed.
Note that |\childdocjob| will always contain the name of the main file.

%%%%%%%%%%%%%%%%%%%%%%%%%%%%%%%%%%%%%%%%
\paragraph{Title Page.}

Conditional processing can be used to include a title or banner page
in the main document when proper precautions are taken.
Importantly, the code in the main file should ensure that the page counter
(as well as other status parameters which are stored in the |.aux| files)
takes the same value after the conditional processing.
Otherwise the page numbers may take divergent values
depending on which part is compiled.

For example, a title page could be declared by:
%
\begin{center}
\begin{tabular}{l}
|\ifchilddoc\||else|\\
|\addtocounter{page}{-1}|\\
\textit{code for title page}\\
|\newpage|\\
|\||fi|
\end{tabular}
\end{center}
%
A banner page for the child documents can be generated by:
%
\begin{center}
\begin{tabular}{l}
|\ifchilddoc|\\
|\addtocounter{page}{-1}|\\
\textit{code for banner page}\\
|\newpage|\\
|\||fi|
\end{tabular}
\end{center}
%
Here one could write a message such as:
\begin{center}
|This is the part \childdocname{} of \childdocjob{}.|
\end{center}

%%%%%%%%%%%%%%%%%%%%%%%%%%%%%%%%%%%%%%%%%%%%%%%%%%%%%%%%%%%%%%%%%%%%%%%%%%%%%%%%
\subsection{Flags}
\label{sec:flags}

The package makes it easy to generate different versions
of the main or child documents.
To this end compilation flags can be defined
and assigned different default values.
They will be particularly useful in conjunction
with the forwarding mechanism described in \secref{sec:forward}.

For example, it may be useful to have a flag |\version|
which can be set to |draft| or |final|.
The document source will contain some conditional code
depending on the value of |\version|.
Suppose further, the flag should default to |final| for the main file
and to |draft| for child files
which is a natural assignment for editing the document.
This is achieved by placing the following code
in the preamble of the main document
(below the |\childdocmain| directive):
%
\begin{center}
\begin{tabular}{l}
|\ifchilddoc|\\
|\providecommand{\version}{draft}|\\
|\||else|\\
|\providecommand{\version}{final}|\\
|\||fi|
\end{tabular}
\end{center}
%
The definition by |\providecommand| makes sure
that previous definitions are not overwritten.
Further statements |\providecommand{\version}{...}|
can thus be added before the above code to override it.

For the main file, one might add a line
(between |\childdocmain| and the above block)
%
\begin{center}
|%\ifchilddoc\||else\providecommand{\version}{draft}\||fi|
\end{center}
%
which can be uncommented to produce a draft version.
Likewise one can add a line to the very top of a child file
(above the |\childdocof{|\textit{main}|}| directive)
%
\begin{center}
|%\providecommand{\version}{final}|
\end{center}
%
which can be uncommented to produce the final version of this child document.

%%%%%%%%%%%%%%%%%%%%%%%%%%%%%%%%%%%%%%%%%%%%%%%%%%%%%%%%%%%%%%%%%%%%%%%%%%%%%%%%
\subsection{Forwarding}
\label{sec:forward}

Different versions of the main or child documents
using compilation flags as described in \secref{sec:flags}
can be (permanently) stored in different files
for convenient compilation, viewing and distribution.
To this end, the package defines a command
to pass on compilation to a different file:

%%%%%%%%%%%%%%%%%%%%%%%%%%%%%%%%%%%%%%%%
\DescribeMacro{\childdocforward}
The command |\childdocforward| redirects processing to
another source file:
%
\begin{center}
\begin{tabular}{l}
|\input{childdoc.def}|\\
|\childdocforward[|\textit{main}|]{|\textit{dest}|}|\\
\end{tabular}
\end{center}
%
The argument \textit{dest} is the destination file
(without extension).
It should be the main file or one of the child files.
Note that further \textsf{childdoc} directives
such as |\childdocof| and |\childdocforward|
in the indicated file will be processed in this form.
The optional argument \textit{main}
passes on directly to the main file \textit{main}
while pretending to compile the child \textit{dest}.
This form behaves as if \textit{dest}
issues |\childdocof{|\textit{main}|}| right away,
and no further \textsf{childdoc} directives will be processed.

%%%%%%%%%%%%%%%%%%%%%%%%%%%%%%%%%%%%%%%%
\DescribeMacro{\...prefix}
In the alternative form |\childdocforwardprefix|,
%
\begin{center}
\begin{tabular}{l}
|\input{childdoc.def}|\\
|\childdocforwardprefix[|\textit{main}|]{|\textit{prefix}|}{|\textit{dest}|}|
\end{tabular}
\end{center}
%
the destination file is determined by a pattern
depending on the current file:
To make this work, the current file must be called
`{\textit{prefix}\hspace{0.2em}\textit{suffix}}'
with \textit{prefix} matching precisely the argument.
Processing is then passed on to the file
`{\textit{dest}\hspace{0.2em}\textit{suffix}}'.
Surely, the same effect is achieved by
directly specifying the
argument `{\textit{dest}\hspace{0.2em}\textit{suffix}}'
in the first form.
However, that requires to set up a different file
for each child. With the alternative form of the command
all these files can have exactly the same content
which simplifies setting them up and maintaining them.

For example, the following file |draft.tex|
with a compilation flag |\version| as described in \secref{sec:flags}
compiles the main document as a draft:
%
\begin{center}
\begin{tabular}{l}
|\def\version{draft}|\\
|\input{childdoc.def}|\\
|\childdocforward{|\textit{main}|}|
\end{tabular}
\end{center}
%
Likewise, the following files |final|\textit{nn}|.tex|
compile the final version of the child document
|child|\textit{nn}|.tex|:
%
\begin{center}
\begin{tabular}{l}
|\def\version{final}|\\
|\input{childdoc.def}|\\
|\childdocforwardprefix{final}{child}|
\end{tabular}
\end{center}
%

Note that when several versions of a main file and/or of each child file
are to be generated, it may be convenient to set up a |Makefile| or
shell script to automatise the process.

%%%%%%%%%%%%%%%%%%%%%%%%%%%%%%%%%%%%%%%%%%%%%%%%%%%%%%%%%%%%%%%%%%%%%%%%%%%%%%%%
\subsection{Command Line Processing}
\label{sec:commandline}

The effect of redirection files can also be achieved by invoking
the \LaTeX{} compiler with a more elaborate command line.
Most conveniently this should be done as part
of a shell script or a |Makefile|.

When using \textsf{childdoc} in the main file, the following
command lines effectively perform a redirection
(note that depending on the shell being used,
backslashes may have to be doubled: `|\|' $\to$ `|\\|'):
%
\begin{center}
|... -jobname "|\textit{target}|" |\\|"|[\textit{flags}]%
|\input{childdoc.def}\childdocforward[|\textit{main}|]{|\textit{dest}|}"|
\end{center}
%
Here \textit{target} is the name of the output file,
\textit{main} is the name of the main file
and \textit{dest} is the name of the main or child file to be processed
(all filenames without extensions).
The optional argument \textit{main} can be omitted
if \textit{main} matches \textit{dest}.
Optionally, compilation \textit{flags} can be defined via |\def| commands.
This command line makes the \TeX{} engine believe
it is compiling the file \textit{target}
whose content is specified as the latter parameter.
The provided code then forwards the processing to
\textit{main} or \textit{dest} as described in \secref{sec:forward}.

%%%%%%%%%%%%%%%%%%%%%%%%%%%%%%%%%%%%%%%%%%%%%%%%%%%%%%%%%%%%%%%%%%%%%%%%%%%%%%%%
\subsection{Include by Input}
\label{sec:input}

Including child documents by |\include| has some restrictions by design.
Most notably, the content of a child document always occupies
its own set of pages; pages cannot be shared between child documents.
Usually, this behaviour makes perfect sense
because each child document contain an essential part of the document.
However, in some situations it may be desirable to compose
a document from a collection of parts
without having mandatory page breaks between then.
For this case, the package
provides a mechanism to include parts
by |\input| which can also be processed individually.
However, by construction this mechanism
requires manual handling of the content to be output.

%%%%%%%%%%%%%%%%%%%%%%%%%%%%%%%%%%%%%%%%
\DescribeMacro{\ifchilddocmanual}
The main file should be prepared as usual, see \secref{sec:include}.
However, the document body must make a distinction
between processing of an individual part and of the main document, e.g.:
%
\begin{center}
\begin{tabular}{l}
|\ifchilddocmanual|\\
|\input{\childdocname}|\\
|\||else|\\
\textit{document body with }|\input{|\textit{part}|}|\\
|\||fi|
\end{tabular}
\end{center}
%
The conditional |\ifchilddocmanual| is true whenever
a part to be included by |\input| is being compiled,
and the name of the part is stored in |\childdocname|.

%%%%%%%%%%%%%%%%%%%%%%%%%%%%%%%%%%%%%%%%
\DescribeMacro{\childdocby}
Each part to be included by |\input| should start with:
%
\begin{center}
\begin{tabular}{l}
|\input{childdoc.def}|\\
|\childdocby{|\textit{main}|}|\\
\end{tabular}
\end{center}
%
The directive |\childdocby| is similar to |\childdocof|
described in \secref{sec:include},
but the subsequent selection of content must be done manually.
To that end, both |\ifchilddoc| and |\ifchilddocmanual|
will be true upon processing of a part,
and the name of the part is stored in |\childdocname|.
Note that |\jobname| will be set to the filename of the current part
so that each part receives an individual |.aux| file
that does not interfere with the |.aux| file(s) of the main document.
This behaviour can be altered by the alternative form
|\childdocby[*]{|\textit{main}|}| (with a non-empty optional argument)
which uses the |.aux| file of the main document
by setting |\jobname| to \textit{main}.

%%%%%%%%%%%%%%%%%%%%%%%%%%%%%%%%%%%%%%%%%%%%%%%%%%%%%%%%%%%%%%%%%%%%%%%%%%%%%%%%
\subsection{Driver Development}
\label{sec:driver}

The \textsf{childdoc} mechanism can also be use for the development
of definition files such as \LaTeX{} styles or classes.
This case differs from the above setup with multiple parts
included by |\include| in that no |\includeonly| should be invoked.
This can be achieved by starting the include file
(before |\ProvidesPackage|) with:
%
\begin{center}
\begin{tabular}{l}
|\input{childdoc.def}|\\
|\childdocforward{|\textit{main}|}|\\
\end{tabular}
\end{center}
%
or alternatively with:
%
\begin{center}
\begin{tabular}{l}
|\input{childdoc.def}|\\
|\childdocby{|\textit{main}|}|\\
\end{tabular}
\end{center}
%
Both forms have slightly different effects as described above.
The main file is prepared as usual, see \secref{sec:include}.

%%%%%%%%%%%%%%%%%%%%%%%%%%%%%%%%%%%%%%%%%%%%%%%%%%%%%%%%%%%%%%%%%%%%%%%%%%%%%%%%
\subsection{Legacy Detection}
\label{sec:detection}

The directive |\childdocmain| in the main file can detect
whether the complete document or merely a child is to be compiled
even without using the directive |\childdocof|.
This method is deprecated because it is less robust
and there is no compelling reason to use it;
it is merely provided for backward compatibility
and it may be removed in future versions.

If the detection mechanism is to be used,
it is mandatory to correctly specify
the filename of the main file as the argument of |\childdocmain|:
%
\begin{center}
\begin{tabular}{l}
|\input{childdoc.def}|\\
|\childdocmain{|\textit{main}|}|\\
\end{tabular}
\end{center}
%
If |\jobname| does not match the argument \textit{main} of |\childdocmain|,
it is assumed that |\jobname| points to the child file to be compiled.
When using |\childdocmain| with the main file specified as argument,
it suffices to start a child file
with just |\input{|\textit{main}|}|
without loading of the package and using |\childdocof|.
If instead all processing is done
with the appropriate \textsf{childdoc} directives,
the argument of \textit{main} of |\childdocmain| can be empty.

An alternative version of the command line processing described
in \secref{sec:commandline} using the detection mechanism reads:
%
\begin{center}
|... -jobname "|\textit{target}|" "|[\textit{flags}]%
[|\def\jobname{|\textit{dest}|}|]|\input{|\textit{main}|}"|
\end{center}

%%%%%%%%%%%%%%%%%%%%%%%%%%%%%%%%%%%%%%%%%%%%%%%%%%%%%%%%%%%%%%%%%%%%%%%%%%%%%%%%
\subsection{Manual Code}
\label{sec:manual}

In case one cannot be certain whether the definitions file |childdoc.def|
is installed on the target \TeX{} distribution
and one prefers not to ship it,
it is conceivable to paste a few relevant commands into the sources.

To that end, drop all statements |\input{childdoc.def}|
and perform the replacements as outlined below.
Instead of |\childdocmain{|\textit{main}|}| add the following code
to the top of the main file:
%
\begin{center}
\begin{tabular}{l}
|\||ifdefined\childdocname\endinput\||fi\newif\ifchilddoc|\\
|\edef\childdocname{\scantokens\expandafter{\jobname\noexpand}}|\\
|\def\childdocmain{|\textit{main}|}\||ifx\childdocmain\childdocname\||else|\\
|\childdoctrue\includeonly{\childdocname}\let\jobname\childdocmain\||fi|\\
\end{tabular}
\end{center}
%
Instead of |\childdocof{|\textit{main}|}| just include the main file
at the top of each child file:
%
\begin{center}
|\input{|\textit{main}|}|
\end{center}
%
A simple redirection |\childdocforward{|\textit{dest}|}| is achieved by:
%
\begin{center}
|\def\jobname{|\textit{dest}|}\input{\jobname}|
\end{center}
%
The redirection with prefix
|\childdocforwardprefix[|\textit{prefix}|]{|\textit{dest}|}|
is accomplished by:
%
\begin{center}
\begin{tabular}{l}
|{\edef\jobname{\scantokens\expandafter{\jobname\noexpand}}|\\
|\def\redirectjob |\textit{prefix}|#1~~~{\gdef\jobname{|\textit{dest}|#1}}|\\
|\expandafter\redirectjob\jobname~~~}\input{\jobname}|
\end{tabular}
\end{center}

In an alternative approach,
child documents can be compiled by a specific command line
without additional code or specific definitions:
%
\begin{center}
|... -jobname "|\textit{target}|" "|[\textit{flags}]%
|\includeonly{|\textit{dest}|}\input{|\textit{main}|}"|
\end{center}
%

%%%%%%%%%%%%%%%%%%%%%%%%%%%%%%%%%%%%%%%%%%%%%%%%%%%%%%%%%%%%%%%%%%%%%%%%%%%%%%%%
%%%%%%%%%%%%%%%%%%%%%%%%%%%%%%%%%%%%%%%%%%%%%%%%%%%%%%%%%%%%%%%%%%%%%%%%%%%%%%%%
\section{Information}

%%%%%%%%%%%%%%%%%%%%%%%%%%%%%%%%%%%%%%%%%%%%%%%%%%%%%%%%%%%%%%%%%%%%%%%%%%%%%%%%
\subsection{Copyright}

Copyright \copyright{} 2017--2018 Niklas Beisert

This work may be distributed and/or modified under the
conditions of the \LaTeX{} Project Public License, either version 1.3
of this license or (at your option) any later version.
The latest version of this license is in
  \url{http://www.latex-project.org/lppl.txt}
and version 1.3 or later is part of all distributions of \LaTeX{}
version 2005/12/01 or later.

This work has the LPPL maintenance status `maintained'.

The Current Maintainer of this work is Niklas Beisert.

This work consists of the files |README.txt|, |childdoc.ins| and |childdoc.dtx|
as well as the derived files |childdoc.def|, |cdocsamp.tex|
with |cdocsch1.tex|, |cdocsch2.tex|, |cdocspt3.tex|, |cdocspt4.tex|,
|cdocsdrf.tex|, |cdocsfn1.tex|, |cdocsfn2.tex|
as well as |childdoc.pdf|.

%%%%%%%%%%%%%%%%%%%%%%%%%%%%%%%%%%%%%%%%%%%%%%%%%%%%%%%%%%%%%%%%%%%%%%%%%%%%%%%%
\subsection{Files and Installation}

The package consists of the files:
%
\begin{center}
\begin{tabular}{ll}
    |README.txt|   & readme file \\
    |childdoc.ins| & installation file \\
    |childdoc.dtx| & source file \\
    |childdoc.def| & definition file \\
    |cdocsamp.tex| & sample main file \\
    |cdocsch1.tex| & sample include file \\
    |cdocsch2.tex| & sample include file \\
    |cdocspt3.tex| & sample part file \\
    |cdocspt4.tex| & sample part file \\
    |cdocsdrf.tex| & sample redirection file \\
    |cdocsfn1.tex| & sample redirection file \\
    |cdocsfn2.tex| & sample redirection file \\
    |childdoc.pdf| & manual
\end{tabular}
\end{center}
%
The distribution consists of the files
|README.txt|, |childdoc.ins| and |childdoc.dtx|.
%
\begin{itemize}
\item
Run (pdf)\LaTeX{} on |childdoc.dtx|
to compile the manual |childdoc.pdf| (this file).
\item
Run \LaTeX{} on |childdoc.ins| to create the definitions file |childdoc.def|
and the sample |cdocsamp.tex| with include files
|cdocsch1.tex|, |cdocsch2.tex|, |cdocspt3.tex|, |cdocspt4.tex|,
|cdocsdrf.tex|, |cdocsfn1.tex|, |cdocsfn2.tex|.
Then copy the file |childdoc.def| to an appropriate directory of your \LaTeX{}
distribution, e.g.\ \textit{texmf-root}|/tex/latex/childdoc|.
\end{itemize}

%%%%%%%%%%%%%%%%%%%%%%%%%%%%%%%%%%%%%%%%%%%%%%%%%%%%%%%%%%%%%%%%%%%%%%%%%%%%%%%%
\subsection{Related CTAN Packages}

There are several other packages which offer a similar functionality:
%
\begin{itemize}
\item
The packages
\href{http://ctan.org/pkg/docmute}{\textsf{docmute}},
\href{http://ctan.org/pkg/includex}{\textsf{includex}} and
\href{http://ctan.org/pkg/standalone}{\textsf{standalone}}
provide commands to include only the document body of
a child file thus allowing both files to be compiled individually.
\item
The packages \href{http://ctan.org/pkg/subdocs}{\textsf{subdocs}}
and \href{http://ctan.org/pkg/subfiles}{\textsf{subfiles}}
provide structures in which the main and child documents can be
encapsulated and allowing them to be compiled individually.
The inclusion mechanism is different from the conventional |\include|.
\item
The package \href{http://ctan.org/pkg/combine}{\textsf{combine}}
is an elaborate solution to combine several documents into one.
\end{itemize}
%
See also the CTAN topic \href{http://ctan.org/topic/subdocs}{\textsf{subdocs}}
for further related packages.
The present package differs from the above solutions in that
a document structure constructed with the conventional |\include| mechanism
just needs two extra commands at the top of every file
such that all constituent files can be compiled individually.

%%%%%%%%%%%%%%%%%%%%%%%%%%%%%%%%%%%%%%%%%%%%%%%%%%%%%%%%%%%%%%%%%%%%%%%%%%%%%%%%
%\subsection{Feature Suggestions}
%
%The following is a list of features which may be useful for future
%versions of this package:
%%
%\begin{itemize}
%\item
%\ldots
%\end{itemize}

%%%%%%%%%%%%%%%%%%%%%%%%%%%%%%%%%%%%%%%%%%%%%%%%%%%%%%%%%%%%%%%%%%%%%%%%%%%%%%%%
\subsection{Revision History}

%%%%%%%%%%%%%%%%%%%%%%%%%%%%%%%%%%%%%%%%
\paragraph{v2.0:} 2018/12/30

\begin{itemize}
\item
immediate forward processing
\item
added |\childdocby| mechanism
\item
manual restructured
\end{itemize}

%%%%%%%%%%%%%%%%%%%%%%%%%%%%%%%%%%%%%%%%
\paragraph{v1.6:} 2018/01/17

\begin{itemize}
\item
application for development of include files
\item
corrections to manual
\end{itemize}

%%%%%%%%%%%%%%%%%%%%%%%%%%%%%%%%%%%%%%%%
\paragraph{v1.5:} 2017/05/21

\begin{itemize}
\item
more complete structuring introduced
\item
|\childdocof| introduced
\item
|\childdoc| renamed to |\childdocmain|
\item
|\childredirect| renamed to |\childdocforward| and |\childdocforwardprefix|
and functionality expanded
\end{itemize}

%%%%%%%%%%%%%%%%%%%%%%%%%%%%%%%%%%%%%%%%
\paragraph{v1.0:} 2017/04/27

\begin{itemize}
\item
manual and install package
\item
first version published on CTAN
\end{itemize}

%%%%%%%%%%%%%%%%%%%%%%%%%%%%%%%%%%%%%%%%
\paragraph{v0.6:} 2017/04/26

\begin{itemize}
\item
redirection mechanism added
\end{itemize}

%%%%%%%%%%%%%%%%%%%%%%%%%%%%%%%%%%%%%%%%
\paragraph{v0.5:} 2017/04/26

\begin{itemize}
\item
functionality in definition file
\end{itemize}


%%%%%%%%%%%%%%%%%%%%%%%%%%%%%%%%%%%%%%%%%%%%%%%%%%%%%%%%%%%%%%%%%%%%%%%%%%%%%%%%
%%%%%%%%%%%%%%%%%%%%%%%%%%%%%%%%%%%%%%%%%%%%%%%%%%%%%%%%%%%%%%%%%%%%%%%%%%%%%%%%
%%%%%%%%%%%%%%%%%%%%%%%%%%%%%%%%%%%%%%%%%%%%%%%%%%%%%%%%%%%%%%%%%%%%%%%%%%%%%%%%
\appendix

\settowidth\MacroIndent{\rmfamily\scriptsize 000\ }

 \DocInput{childdoc.dtx}

\end{document}
%</driver>
% \fi
%
% %%%%%%%%%%%%%%%%%%%%%%%%%%%%%%%%%%%%%%%%%%%%%%%%%%%%%%%%%%%%%%%%%%%%%%%%%%%%%%
% %%%%%%%%%%%%%%%%%%%%%%%%%%%%%%%%%%%%%%%%%%%%%%%%%%%%%%%%%%%%%%%%%%%%%%%%%%%%%%
% \section{Sample}
%\iffalse
%<*samplemain>
%\fi
%
% The following presents a sample document
% with two chapters, two parts, a title page,
% a compile flag as well as three forwarding files to set the flag.
% It consists of eight |.tex| files:
% \begin{center}
% \begin{tabular}{ll}
% |cdocsamp.tex|&main file\\
% |cdocsch1.tex|&include file for chapter 1\\
% |cdocsch2.tex|&include file for chapter 2\\
% |cdocspt3.tex|&include file for part 3\\
% |cdocspt4.tex|&include file for part 4\\
% |cdocsdrf.tex|&forwarding file for main file in draft mode\\
% |cdocsfi1.tex|&forwarding file for final version of chapter 1\\
% |cdocsfi2.tex|&forwarding file for final version of chapter 2\\
% \end{tabular}
% \end{center}
% Each of the eight files can be compiled directly by the \LaTeX{} compiler.
%
% %%%%%%%%%%%%%%%%%%%%%%%%%%%%%%%%%%%%%%
% \paragraph{Main File.}
%
% The main file is called |cdocsamp.tex|.
%
% Load the \textsf{childdoc} definitions and
% declare the filename for the main document:
%    \begin{macrocode}
\input{childdoc.def}
\childdocmain{}
%    \end{macrocode}

% Optional override for |\version| flag:
%    \begin{macrocode}
%%\ifchilddoc\else\providecommand{\version}{draft}\fi
%    \end{macrocode}

% Define the default values for the |\version| flag
% (|final| for the main file and |draft| for childs):
%    \begin{macrocode}
\ifchilddoc
\providecommand{\version}{draft}
\else
\providecommand{\version}{final}
\fi
%    \end{macrocode}

% Load the standard document class:
%    \begin{macrocode}
\documentclass[12pt]{article}
%    \end{macrocode}

% Start the document body:
%    \begin{macrocode}
\begin{document}
%    \end{macrocode}

% Declare a title page.
% Print title, part of document being processed and version flag:
%    \begin{macrocode}
\addtocounter{page}{-1}
\begin{center}
{\LARGE\bfseries{}childdoc example\par}
\vspace{1cm}
\ifchilddoc
\ifchilddocmanual part\else chapter\fi:
`\childdocname' of `\childdocjob'\par
\else
main document: `\childdocjob'\par
\fi
version: \version\par
\end{center}
\newpage
%    \end{macrocode}

% Manually include selected file,
% otherwise process as usual:
%    \begin{macrocode}
\ifchilddocmanual
\section*{part `\childdocname'}
\input{\childdocname}
\else
%    \end{macrocode}

% Include the two chapters:
%    \begin{macrocode}
\include{cdocsch1}
\include{cdocsch2}
%    \end{macrocode}

% Include the two parts unless only chapters should be displayed:
%    \begin{macrocode}
\ifchilddoc\else
\section{part three}
\input{cdocspt3}
\section{part four}
\input{cdocspt4}
\fi
%    \end{macrocode}

% Process as usual until here:
%    \begin{macrocode}
\fi
%    \end{macrocode}

% End of document body:
%    \begin{macrocode}
\end{document}
%    \end{macrocode}
%\iffalse
%</samplemain>
%\fi
%
% %%%%%%%%%%%%%%%%%%%%%%%%%%%%%%%%%%%%%%
% \paragraph{Chapter Include Files.}
%
% The include files are called |cdocsch1.tex| and |cdocsch2.tex|.
%
%\iffalse
%<*samplechap1|samplechap2>
%\fi

% Optional override for |\version| flag:
%    \begin{macrocode}
%%\providecommand{\version}{final}
%    \end{macrocode}

% Include the main document:
%    \begin{macrocode}
\input{childdoc.def}
\childdocof{cdocsamp}
%    \end{macrocode}

%\iffalse
%</samplechap1|samplechap2>
%\fi
%
%\iffalse
%<*samplechap1>
%\fi
% Some text for chapter 1:
%    \begin{macrocode}
\section{one}
some text in chapter one
%    \end{macrocode}

%\iffalse
%</samplechap1>
%\fi
% Some text for chapter 2:
%\iffalse
%<*samplechap2>
%\fi
%    \begin{macrocode}
\section{two}
more text in chapter two
%    \end{macrocode}

%\iffalse
%</samplechap2>
%\fi
%
% %%%%%%%%%%%%%%%%%%%%%%%%%%%%%%%%%%%%%%
% \paragraph{Part Include Files.}
%
% The include files are called |cdocspt3.tex| and |cdocspt4.tex|.
%
%\iffalse
%<*samplepart3|samplepart4>
%\fi

% Optional override for |\version| flag:
%    \begin{macrocode}
%%\providecommand{\version}{final}
%    \end{macrocode}

% Include the main document:
%    \begin{macrocode}
\input{childdoc.def}
\childdocby{cdocsamp}
%    \end{macrocode}

%\iffalse
%</samplepart3|samplepart4>
%\fi
%
%\iffalse
%<*samplepart3>
%\fi
% Some text for part 3:
%    \begin{macrocode}
some text in part three
%    \end{macrocode}

%\iffalse
%</samplepart3>
%\fi
% Some text for part 4:
%\iffalse
%<*samplepart4>
%\fi
%    \begin{macrocode}
more text in part four
%    \end{macrocode}

%\iffalse
%</samplepart4>
%\fi
%
% %%%%%%%%%%%%%%%%%%%%%%%%%%%%%%%%%%%%%%
% \paragraph{Forwarding for a Complete Draft.}
%
% The following forwarding file |cdocsdrf.tex|
% compiles the main document in draft mode:
%\iffalse
%<*sampledraft>
%\fi
%    \begin{macrocode}
\def\version{draft}
\input{childdoc.def}
\childdocforward{cdocsamp}
%    \end{macrocode}

%\iffalse
%</sampledraft>
%\fi
%
% %%%%%%%%%%%%%%%%%%%%%%%%%%%%%%%%%%%%%%
% \paragraph{Forwarding for Final Version of the Chapters.}
%
% The following forwarding files |cdocsfn1.tex| and |cdocsfn2.tex|
% (with identical content)
% compile the final versions of the child documents
% |cdocsch1.tex| and |cdocsch2.tex|, respectively:
%\iffalse
%<*samplefinal>
%\fi
%    \begin{macrocode}
\def\version{final}
\input{childdoc.def}
\childdocforwardprefix[cdocsamp]{cdocsfn}{cdocsch}
%    \end{macrocode}

%\iffalse
%</samplefinal>
%\fi
%
% %%%%%%%%%%%%%%%%%%%%%%%%%%%%%%%%%%%%%%
% \paragraph{Command Line Processing.}
%
% The following three command lines generate the output files
% |cdocscld|, |cdocscl1| and |cdocscl2|
% which should be identical to
% |cdocsdrf|, |cdocsch1| and |cdocsfn2|, respectively:
% \begin{center}
% \begin{tabular}{l}
% |latex -jobname cdocscld \|\\
% |  "\def\version{draft}\input{childdoc.def}\childdocforward{cdocsamp}"|\\
% |latex -jobname cdocscl1 \|\\
% |  "\input{childdoc.def}\childdocforward[cdocsamp]{cdocsch1}"|\\
% |latex -jobname cdocscl2 \|\\
% |  "\def\version{final}\input{childdoc.def}\childdocforward{cdocsch2}"|
% \end{tabular}
% \end{center}
% Note that the trailing backslash on each first line
% merely continues the input to the second line
% (for convenient cut ant paste).
% Furthermore, the command |latex| can be replaced by any
% of its alternative versions such as |pdflatex|.
%
% %%%%%%%%%%%%%%%%%%%%%%%%%%%%%%%%%%%%%%%%%%%%%%%%%%%%%%%%%%%%%%%%%%%%%%%%%%%%%%
% %%%%%%%%%%%%%%%%%%%%%%%%%%%%%%%%%%%%%%%%%%%%%%%%%%%%%%%%%%%%%%%%%%%%%%%%%%%%%%
% \section{Implementation}
%\iffalse
%<*package>
%\fi
%
% This section describes the definitions file |childdoc.def|.

% The definitions cannot be loaded using |\usepackage| or |\RequirePackage|
% which has a mechanism to prevent loading a style file more than once.
% When loading the definitions by means of |\input|
% multiple instances have to be prevented manually:
%\iffalse
%This code needs to be before the `\ProvidesFile' directive
%which is defined at the beginning of this file.
%Therefore it is also placed there and commented out here.
%</package>
%<*discard>
%\fi
%    \begin{macrocode}
\ifdefined\childdocmain\endinput\fi
%    \end{macrocode}
%\iffalse
%</discard>
%<*package>
%\fi
%
% \macro{\ifchilddoc}
% \macro{\ifchilddocmanual}
% The conditional |\ifchilddoc| tells whether a
% child (true) or main (false) document is being compiled.
% The conditional |\ifchilddocmanual| tells whether
% the |\includeonly| mechanism is used (false) or
% the selection of child files must be performed manually (true).
% The definitions initialise to false:
%    \begin{macrocode}
\newif\ifchilddoc
\newif\ifchilddocmanual
%    \end{macrocode}

% \macro{\childdocname}
% \macro{\childdocjob}
% The macro |\childdocname| stores the name of the main document
% to be compiled. The macro |\childdocjob| stores the name of
% the document on which the \LaTeX{} compiler was originally invoked.
% The content of |\jobname| cannot be compared
% to filenames specified in the source due to different catcodes.
% The following code rescans |\jobname|, stores the result
% in |\childdocname| and saves a copy in |\childdocjob|:
%    \begin{macrocode}
\edef\childdocname{\scantokens\expandafter{\jobname\noexpand}}
\let\childdocjob\childdocname
%    \end{macrocode}

% \macro{\childdocdisable}
% The macro |\childdocdisable| prevents the main file
% from being processed more than once.
% At this stage, the main document command |\childdocmain|
% is assumed to be called once again where it should do nothing.
% Any subsequent call to it should prevent
% a secondary processing of the main document
% It overwrites the forwarding commands
% |\childdocof| and |\childdocforward|
% with empty macros to prevent further inclusions of the main document:
%    \begin{macrocode}
\newcommand{\childdocdisable}
{
  \renewcommand{\childdocmain}[1]{\renewcommand{\childdocmain}[1]{\endinput}}
  \renewcommand{\childdocof}[1]{}
  \renewcommand{\childdocby}[2][]{}
  \renewcommand{\childdocforward}[2][]{}
  \renewcommand{\childdocdisable}{}
}
%    \end{macrocode}

% \macro{\childdocmain}
% The macro |\childdocmain| is to be called at the top of the main file
% with nothing or the main filename (without extension) as argument.
% First, it breaks loops.
% If the argument is not empty and does not match |\childdocname|
% (which is set by the first inclusion of |childdoc.def|),
% |\ifchilddoc| is set to true, |\includeonly| is applied to the child file
% and |\jobname| is set to the main file
% (for proper handling of |.aux| files):
%    \begin{macrocode}
\newcommand{\childdocmain}[1]
{
  \childdocdisable\childdocmain{}
  \if?#1?\else
    \begingroup
      \def\childdoctmp{#1}
      \ifx\childdoctmp\childdocname
        \def\childdoctmp{}
      \else
        \def\childdoctmp
        {
          \childdoctrue
          \includeonly{\childdocname}
          \def\childdocjob{#1}
          \def\jobname{#1}
        }
      \fi
      \expandafter
    \endgroup
    \childdoctmp
  \fi
}
%    \end{macrocode}

% \macro{\childdocof}
% The command |\childdocof| redirects
% compilation to the main file |#1|.
%    \begin{macrocode}
\newcommand{\childdocof}[1]
{
  \childdocdisable
  \childdoctrue
  \includeonly{\childdocname}
  \def\jobname{#1}
  \def\childdocjob{#1}
  \input{#1}
}
%    \end{macrocode}

% \macro{\childdocby}
% The command |\childdocby| ....
%    \begin{macrocode}
\newcommand{\childdocby}[2][]
{
  \childdocdisable
  \childdoctrue
  \childdocmanualtrue
  \if?#1?\else
    \def\jobname{#2}
  \fi
  \def\childdocjob{#2}
  \input{#2}
  \endinput
}
%    \end{macrocode}

% \macro{\childdocforward}
% The command |\childdocforward| redirects
% compilation to the main file or
% (if the optional argument is given) a child file.
% Parameters are set as if the main file
% or a child file starting with |\childdocof| was compiled.
% Then compilation is handed over to the main file:
%    \begin{macrocode}
\newcommand{\childdocforward}[2][]
{
  \begingroup
    \if?#1?
      \def\childdoctmp
      {
        \def\childdocname{#2}
        \def\childdocjob{#2}
        \def\jobname{#2}
        \input{#2}
        \endinput
      }
    \else
      \def\childdoctmp
      {
        \childdocdisable
        \def\childdocname{#2}
        \childdoctrue
        \includeonly{#2}
        \def\childdocjob{#1}
        \def\jobname{#1}
        \input{#1}
        \endinput
      }
    \fi
    \expandafter
  \endgroup
  \childdoctmp
}
%    \end{macrocode}

% \macro{\childdocforwardprefix}
% The command |\childdocforwardprefix| redirects
% compilation to the main or a child file by means of a pattern.
% The prefix |#1| in the current filename is replaced by |#2|
% and the suffix of the current filename is kept
% (it is assumed that the filename does not contain the substring `|~~~|'
% which is used as a delimiter).
% Compilation is handed over to the new file by |\childdocforward|:
%    \begin{macrocode}
\newcommand{\childdocforwardprefix}[3][]
{
  \begingroup
    \def\childdocextract #2##1~~~{\def\childdoctmp{\childdocforward[#1]{#3##1}}}
    \expandafter\childdocextract\childdocname~~~
    \expandafter
  \endgroup
  \childdoctmp
}
%    \end{macrocode}

% \macro{\childdoc}
% The deprecated macro |\childdoc| is a legacy version of |\childdocmain|:
%    \begin{macrocode}
\newcommand{\childdoc}{\childdocmain}
%    \end{macrocode}

% \macro{\childdocredirect}
% The deprecated macro |\childdocredirect| is a legacy version
% of |\childdocforward| and |\childdocforwardprefix|:
%    \begin{macrocode}
\newcommand{\childdocredirect}[2][]
{
  \begingroup
    \if?#1?
      \def\childdoctmp{\childdocforward{#2}}
    \else
      \def\childdoctmp{\childdocforwardprefix{#1}{#2}}
    \fi
    \expandafter
  \endgroup
  \childdoctmp
}
%    \end{macrocode}

%\iffalse
%</package>
%\fi
%
\endinput
|\\
|\childdocforward{|\textit{main}|}|
\end{tabular}
\end{center}
%
Likewise, the following files |final|\textit{nn}|.tex|
compile the final version of the child document
|child|\textit{nn}|.tex|:
%
\begin{center}
\begin{tabular}{l}
|\def\version{final}|\\
|% \iffalse
%
% childdoc.dtx Copyright (C) 2017-2018 Niklas Beisert
%
% This work may be distributed and/or modified under the
% conditions of the LaTeX Project Public License, either version 1.3
% of this license or (at your option) any later version.
% The latest version of this license is in
%   http://www.latex-project.org/lppl.txt
% and version 1.3 or later is part of all distributions of LaTeX
% version 2005/12/01 or later.
%
% This work has the LPPL maintenance status `maintained'.
%
% The Current Maintainer of this work is Niklas Beisert.
%
% This work consists of the files childdoc.dtx and childdoc.ins
% and the derived files childdoc.def and cdocsamp.tex with
% cdocsch1.tex, cdocsch2.tex, cdocsdrf.tex, cdocsfn1.tex, cdocsfn2.tex.
%
%<package>\ifdefined\childdocmain\endinput\fi
%<package>\ProvidesFile{childdoc.def}[2018/12/30 v2.0 child document driver]
%<samplemain>\ProvidesFile{cdocsamp.tex}[2018/12/30 v2.0 sample for childdoc]
%<*driver>
%\ProvidesFile{childdoc.drv}[2018/12/30 v2.0 childdoc reference manual file]
\PassOptionsToClass{10pt,a4paper}{article}
\documentclass{ltxdoc}

\usepackage[margin=35mm]{geometry}
\usepackage{hyperref}
\usepackage{hyperxmp}
\usepackage[usenames]{color}

\hypersetup{colorlinks=true}
\hypersetup{pdfstartview=FitH}
\hypersetup{pdfpagemode=UseNone}
\hypersetup{pdfsource={}}
\hypersetup{pdflang={en-UK}}
\hypersetup{pdfcopyright={Copyright 2017-2018 Niklas Beisert.
  This work may be distributed and/or modified under the
  conditions of the LaTeX Project Public License, either version 1.3
  of this license or (at your option) any later version.}}
\hypersetup{pdflicenseurl={http://www.latex-project.org/lppl.txt}}
\hypersetup{pdfcontactaddress={ETH Zurich, ITP, HIT K,
  Wolfgang-Pauli-Strasse 27}}
\hypersetup{pdfcontactpostcode={8093}}
\hypersetup{pdfcontactcity={Zurich}}
\hypersetup{pdfcontactcountry={Switzerland}}
\hypersetup{pdfcontactemail={nbeisert@itp.phys.ethz.ch}}
\hypersetup{pdfcontacturl={http://people.phys.ethz.ch/\xmptilde nbeisert/}}

\newcommand{\secref}[1]{\hyperref[#1]{section \ref*{#1}}}

\parskip1ex
\parindent0pt
\let\olditemize\itemize
\def\itemize{\olditemize\parskip0pt}

\begin{document}

\title{The \textsf{childdoc} Package}
\hypersetup{pdftitle={The childdoc Package}}
\author{Niklas Beisert\\[2ex]
  Institut f\"ur Theoretische Physik\\
  Eidgen\"ossische Technische Hochschule Z\"urich\\
  Wolfgang-Pauli-Strasse 27, 8093 Z\"urich, Switzerland\\[1ex]
  \href{mailto:nbeisert@itp.phys.ethz.ch}
  {\texttt{nbeisert@itp.phys.ethz.ch}}}
\hypersetup{pdfauthor={Niklas Beisert}}
\hypersetup{pdfsubject={Manual for the LaTeX2e Package childdoc}}
\date{30 December 2018, \textsf{v2.0}}
\maketitle

\begin{abstract}\noindent
\textsf{childdoc} is a \LaTeXe{} package
that enables the direct compilation
of document sections included by |\include|
to individual files.
\end{abstract}

\begingroup
\parskip0ex
\tableofcontents
\endgroup

%%%%%%%%%%%%%%%%%%%%%%%%%%%%%%%%%%%%%%%%%%%%%%%%%%%%%%%%%%%%%%%%%%%%%%%%%%%%%%%%
%%%%%%%%%%%%%%%%%%%%%%%%%%%%%%%%%%%%%%%%%%%%%%%%%%%%%%%%%%%%%%%%%%%%%%%%%%%%%%%%
\section{Introduction}

\LaTeX{} provides a mechanism to structure a large document (such as a book)
into a main file and several child files (containing the chapters)
using the |\include| command.
This mechanism is beneficial for documents
which span hundreds of pages in order to
make the source file(s) more manageable.
Moreover, compilation can be restricted to
selected child files by means of the |\includeonly| command.
The latter feature can be used to reduce the compilation time while editing
(this was significantly more useful in the earlier days of \LaTeX{})
or to generate a smaller document which is easier to navigate.
Another application of |\includeonly| is to generate
documents consisting of selected parts of the complete document.

However, there are a few drawbacks of the plain |\include| mechanism:
\begin{itemize}
\item
The child files cannot be compiled on their own,
they can only be compiled via the main file.
A naive editing environment
(such as a text editor with an option
to have the current file processed by \LaTeX)
may require one to switch to the main file before compiling;
attempting to compile the child file produces errors.
\item
The main file must be modified (each time)
to adjust the |\includeonly| command
to the present needs. This easily leaves the main file in a messy state.
\item
The generated document will always carry the filename
of the main document. This is inconvenient if
several child files are to be compiled and
to be kept for distribution.
\end{itemize}

The present package provides a simple interface
to make child files individually compilable by \LaTeX{}.
Compiling a child file then has the same effect as compiling
the main file with an |\includeonly| command
to select the appropriate child.
Moreover the generated document will carry the name of the child
rather than the main file.
This resolves all three above issues.

This feature is meant to make the editing of books,
thesis documents and lecture notes somewhat more convenient.
However, the package can also be used efficiently for
composing a series of documents (such as exercise sheets)
which are typically distributed individually.
It then assists the author in generating the individual documents
(potentially in different versions)
as well as a document containing the collected series.
Another application is in developing style files
or other kinds of included material
where compilation of the style file could redirect
to a sample or test file.

%%%%%%%%%%%%%%%%%%%%%%%%%%%%%%%%%%%%%%%%%%%%%%%%%%%%%%%%%%%%%%%%%%%%%%%%%%%%%%%%
%%%%%%%%%%%%%%%%%%%%%%%%%%%%%%%%%%%%%%%%%%%%%%%%%%%%%%%%%%%%%%%%%%%%%%%%%%%%%%%%
\section{Usage}

First of all, the package \textsf{childdoc} is \emph{not} a standard
\LaTeXe{} |.sty| style file! Therefore it needs to be invoked in
a non-standard way.

%%%%%%%%%%%%%%%%%%%%%%%%%%%%%%%%%%%%%%%%%%%%%%%%%%%%%%%%%%%%%%%%%%%%%%%%%%%%%%%%
\subsection{Included Files}
\label{sec:include}

%%%%%%%%%%%%%%%%%%%%%%%%%%%%%%%%%%%%%%%%
\DescribeMacro{\childdocmain}
To use the package, add the commands
\begin{center}
\begin{tabular}{l}
|\input{childdoc.def}|\\
|\childdocmain{}|\\
\end{tabular}
\end{center}
at the very top of the main \LaTeX{} file,
in particular \emph{before} the |\documentclass| statement!
The argument of |\childdocmain| should be left empty
(but it must be present).

%%%%%%%%%%%%%%%%%%%%%%%%%%%%%%%%%%%%%%%%
\DescribeMacro{\childdocof}
Furthermore, add the commands
\begin{center}
\begin{tabular}{l}
|\input{childdoc.def}|\\
|\childdocof{|\textit{main}|}|\\
\end{tabular}
\end{center}
at the top of every child file \textit{child}
which is included by |\include{|\textit{child}|}|
from within the main file
(or at least for those files to be compiled individually).
The argument \textit{main} must be the filename of the main file.

There are a couple of
considerations in setting up the main and child documents:

%%%%%%%%%%%%%%%%%%%%%%%%%%%%%%%%%%%%%%%%
\paragraph{Restrictions.}

Please note the following restrictions:
\begin{itemize}
\item
|\childdocmain| must be called with one argument \textit{main}
to ensure compatibility with earlier version of the package.
It must either be empty (|\childdocmain{}|)
or precisely match the filename of the main file in which it is specified.
See \secref{sec:detection} for further information.
\item
The filename \textit{main} must be specified without the |.tex| extension.
\item
The filename \textit{main} is case sensitive
(even in case-insensitive file systems)
due to internal string comparison.
\item
The argument \textit{main} should be fully expanded, it cannot be a macro.
\item
Subdirectories and special characters should be avoided in filenames.
\item
The command |\childdocmain{|\textit{main}|}| must be followed by a whitespace.
It should not be followed immediately by another command
or by a comment mark `|%|'.
This is because the \TeX{} parser reads the token immediately following
the argument of |\childdocmain| and puts it
at the beginning of every child section;
however, a white\-space is ignored.
\end{itemize}

%%%%%%%%%%%%%%%%%%%%%%%%%%%%%%%%%%%%%%%%
\paragraph{Content of Main File.}

It is advisable to place all content in the child files included by |\include|.
Any output contained in the main file will appear in all child documents
unless suppressed manually;
it cannot be suppressed automatically by the |\includeonly| directive
and thus should normally be avoided.
A method to include some content in the main file
by means of conditional processing is described in \secref{sec:conditional}.

%%%%%%%%%%%%%%%%%%%%%%%%%%%%%%%%%%%%%%%%
\paragraph{Page Numbering.}

When only a part of the document is compiled,
the appropriate numbering of pages
(as well as other status parameters)
is determined from the |.aux| files.
The latter contain information from previous passes.
However this information needs to propagate through
all intermediate child documents.
Therefore the page numbering in child documents may well
be inconsistent until the complete document is compiled at least once.

A useful (if unconventional) way to always ensure a consistent
page numbering is to restart the numbering in each child document
and denote the pages by `\textit{child}|.|\textit{page}'
where \textit{child} represents the chapter/section number of the child file.
This can be achieved by the command
|\numberwithin{page}{|\textit{child}|}|
of the \textsf{amsmath} package
where \textit{child} can be |chapter| or |section|
depending on the chosen structuring.
Alternatively, one can modify the macro |\thepage| appropriately
and reset the counter |page| at the start of each child file.

%%%%%%%%%%%%%%%%%%%%%%%%%%%%%%%%%%%%%%%%%%%%%%%%%%%%%%%%%%%%%%%%%%%%%%%%%%%%%%%%
\subsection{Conditional Processing}
\label{sec:conditional}

The package provides a mechanism to compile different versions
of a document. To customise the versions further some conditional processing
can come in handy to distinguish which version is being compiled.
The package provides two macros to describe the compilation context:

%%%%%%%%%%%%%%%%%%%%%%%%%%%%%%%%%%%%%%%%
\DescribeMacro{\ifchilddoc}
The conditional |\ifchilddoc| distinguishes between the compilation of
child documents and the main document:
%
\begin{center}
|\ifchilddoc |\textit{child-code}| |[|\||else |\textit{main-code}]| \||fi|
\end{center}

%%%%%%%%%%%%%%%%%%%%%%%%%%%%%%%%%%%%%%%%
\DescribeMacro{\childdocname}
\DescribeMacro{\childdocjob}
The macro |\childdocname| contains the filename (without extension)
of the main or child file being processed.
Note that |\childdocjob| will always contain the name of the main file.

%%%%%%%%%%%%%%%%%%%%%%%%%%%%%%%%%%%%%%%%
\paragraph{Title Page.}

Conditional processing can be used to include a title or banner page
in the main document when proper precautions are taken.
Importantly, the code in the main file should ensure that the page counter
(as well as other status parameters which are stored in the |.aux| files)
takes the same value after the conditional processing.
Otherwise the page numbers may take divergent values
depending on which part is compiled.

For example, a title page could be declared by:
%
\begin{center}
\begin{tabular}{l}
|\ifchilddoc\||else|\\
|\addtocounter{page}{-1}|\\
\textit{code for title page}\\
|\newpage|\\
|\||fi|
\end{tabular}
\end{center}
%
A banner page for the child documents can be generated by:
%
\begin{center}
\begin{tabular}{l}
|\ifchilddoc|\\
|\addtocounter{page}{-1}|\\
\textit{code for banner page}\\
|\newpage|\\
|\||fi|
\end{tabular}
\end{center}
%
Here one could write a message such as:
\begin{center}
|This is the part \childdocname{} of \childdocjob{}.|
\end{center}

%%%%%%%%%%%%%%%%%%%%%%%%%%%%%%%%%%%%%%%%%%%%%%%%%%%%%%%%%%%%%%%%%%%%%%%%%%%%%%%%
\subsection{Flags}
\label{sec:flags}

The package makes it easy to generate different versions
of the main or child documents.
To this end compilation flags can be defined
and assigned different default values.
They will be particularly useful in conjunction
with the forwarding mechanism described in \secref{sec:forward}.

For example, it may be useful to have a flag |\version|
which can be set to |draft| or |final|.
The document source will contain some conditional code
depending on the value of |\version|.
Suppose further, the flag should default to |final| for the main file
and to |draft| for child files
which is a natural assignment for editing the document.
This is achieved by placing the following code
in the preamble of the main document
(below the |\childdocmain| directive):
%
\begin{center}
\begin{tabular}{l}
|\ifchilddoc|\\
|\providecommand{\version}{draft}|\\
|\||else|\\
|\providecommand{\version}{final}|\\
|\||fi|
\end{tabular}
\end{center}
%
The definition by |\providecommand| makes sure
that previous definitions are not overwritten.
Further statements |\providecommand{\version}{...}|
can thus be added before the above code to override it.

For the main file, one might add a line
(between |\childdocmain| and the above block)
%
\begin{center}
|%\ifchilddoc\||else\providecommand{\version}{draft}\||fi|
\end{center}
%
which can be uncommented to produce a draft version.
Likewise one can add a line to the very top of a child file
(above the |\childdocof{|\textit{main}|}| directive)
%
\begin{center}
|%\providecommand{\version}{final}|
\end{center}
%
which can be uncommented to produce the final version of this child document.

%%%%%%%%%%%%%%%%%%%%%%%%%%%%%%%%%%%%%%%%%%%%%%%%%%%%%%%%%%%%%%%%%%%%%%%%%%%%%%%%
\subsection{Forwarding}
\label{sec:forward}

Different versions of the main or child documents
using compilation flags as described in \secref{sec:flags}
can be (permanently) stored in different files
for convenient compilation, viewing and distribution.
To this end, the package defines a command
to pass on compilation to a different file:

%%%%%%%%%%%%%%%%%%%%%%%%%%%%%%%%%%%%%%%%
\DescribeMacro{\childdocforward}
The command |\childdocforward| redirects processing to
another source file:
%
\begin{center}
\begin{tabular}{l}
|\input{childdoc.def}|\\
|\childdocforward[|\textit{main}|]{|\textit{dest}|}|\\
\end{tabular}
\end{center}
%
The argument \textit{dest} is the destination file
(without extension).
It should be the main file or one of the child files.
Note that further \textsf{childdoc} directives
such as |\childdocof| and |\childdocforward|
in the indicated file will be processed in this form.
The optional argument \textit{main}
passes on directly to the main file \textit{main}
while pretending to compile the child \textit{dest}.
This form behaves as if \textit{dest}
issues |\childdocof{|\textit{main}|}| right away,
and no further \textsf{childdoc} directives will be processed.

%%%%%%%%%%%%%%%%%%%%%%%%%%%%%%%%%%%%%%%%
\DescribeMacro{\...prefix}
In the alternative form |\childdocforwardprefix|,
%
\begin{center}
\begin{tabular}{l}
|\input{childdoc.def}|\\
|\childdocforwardprefix[|\textit{main}|]{|\textit{prefix}|}{|\textit{dest}|}|
\end{tabular}
\end{center}
%
the destination file is determined by a pattern
depending on the current file:
To make this work, the current file must be called
`{\textit{prefix}\hspace{0.2em}\textit{suffix}}'
with \textit{prefix} matching precisely the argument.
Processing is then passed on to the file
`{\textit{dest}\hspace{0.2em}\textit{suffix}}'.
Surely, the same effect is achieved by
directly specifying the
argument `{\textit{dest}\hspace{0.2em}\textit{suffix}}'
in the first form.
However, that requires to set up a different file
for each child. With the alternative form of the command
all these files can have exactly the same content
which simplifies setting them up and maintaining them.

For example, the following file |draft.tex|
with a compilation flag |\version| as described in \secref{sec:flags}
compiles the main document as a draft:
%
\begin{center}
\begin{tabular}{l}
|\def\version{draft}|\\
|\input{childdoc.def}|\\
|\childdocforward{|\textit{main}|}|
\end{tabular}
\end{center}
%
Likewise, the following files |final|\textit{nn}|.tex|
compile the final version of the child document
|child|\textit{nn}|.tex|:
%
\begin{center}
\begin{tabular}{l}
|\def\version{final}|\\
|\input{childdoc.def}|\\
|\childdocforwardprefix{final}{child}|
\end{tabular}
\end{center}
%

Note that when several versions of a main file and/or of each child file
are to be generated, it may be convenient to set up a |Makefile| or
shell script to automatise the process.

%%%%%%%%%%%%%%%%%%%%%%%%%%%%%%%%%%%%%%%%%%%%%%%%%%%%%%%%%%%%%%%%%%%%%%%%%%%%%%%%
\subsection{Command Line Processing}
\label{sec:commandline}

The effect of redirection files can also be achieved by invoking
the \LaTeX{} compiler with a more elaborate command line.
Most conveniently this should be done as part
of a shell script or a |Makefile|.

When using \textsf{childdoc} in the main file, the following
command lines effectively perform a redirection
(note that depending on the shell being used,
backslashes may have to be doubled: `|\|' $\to$ `|\\|'):
%
\begin{center}
|... -jobname "|\textit{target}|" |\\|"|[\textit{flags}]%
|\input{childdoc.def}\childdocforward[|\textit{main}|]{|\textit{dest}|}"|
\end{center}
%
Here \textit{target} is the name of the output file,
\textit{main} is the name of the main file
and \textit{dest} is the name of the main or child file to be processed
(all filenames without extensions).
The optional argument \textit{main} can be omitted
if \textit{main} matches \textit{dest}.
Optionally, compilation \textit{flags} can be defined via |\def| commands.
This command line makes the \TeX{} engine believe
it is compiling the file \textit{target}
whose content is specified as the latter parameter.
The provided code then forwards the processing to
\textit{main} or \textit{dest} as described in \secref{sec:forward}.

%%%%%%%%%%%%%%%%%%%%%%%%%%%%%%%%%%%%%%%%%%%%%%%%%%%%%%%%%%%%%%%%%%%%%%%%%%%%%%%%
\subsection{Include by Input}
\label{sec:input}

Including child documents by |\include| has some restrictions by design.
Most notably, the content of a child document always occupies
its own set of pages; pages cannot be shared between child documents.
Usually, this behaviour makes perfect sense
because each child document contain an essential part of the document.
However, in some situations it may be desirable to compose
a document from a collection of parts
without having mandatory page breaks between then.
For this case, the package
provides a mechanism to include parts
by |\input| which can also be processed individually.
However, by construction this mechanism
requires manual handling of the content to be output.

%%%%%%%%%%%%%%%%%%%%%%%%%%%%%%%%%%%%%%%%
\DescribeMacro{\ifchilddocmanual}
The main file should be prepared as usual, see \secref{sec:include}.
However, the document body must make a distinction
between processing of an individual part and of the main document, e.g.:
%
\begin{center}
\begin{tabular}{l}
|\ifchilddocmanual|\\
|\input{\childdocname}|\\
|\||else|\\
\textit{document body with }|\input{|\textit{part}|}|\\
|\||fi|
\end{tabular}
\end{center}
%
The conditional |\ifchilddocmanual| is true whenever
a part to be included by |\input| is being compiled,
and the name of the part is stored in |\childdocname|.

%%%%%%%%%%%%%%%%%%%%%%%%%%%%%%%%%%%%%%%%
\DescribeMacro{\childdocby}
Each part to be included by |\input| should start with:
%
\begin{center}
\begin{tabular}{l}
|\input{childdoc.def}|\\
|\childdocby{|\textit{main}|}|\\
\end{tabular}
\end{center}
%
The directive |\childdocby| is similar to |\childdocof|
described in \secref{sec:include},
but the subsequent selection of content must be done manually.
To that end, both |\ifchilddoc| and |\ifchilddocmanual|
will be true upon processing of a part,
and the name of the part is stored in |\childdocname|.
Note that |\jobname| will be set to the filename of the current part
so that each part receives an individual |.aux| file
that does not interfere with the |.aux| file(s) of the main document.
This behaviour can be altered by the alternative form
|\childdocby[*]{|\textit{main}|}| (with a non-empty optional argument)
which uses the |.aux| file of the main document
by setting |\jobname| to \textit{main}.

%%%%%%%%%%%%%%%%%%%%%%%%%%%%%%%%%%%%%%%%%%%%%%%%%%%%%%%%%%%%%%%%%%%%%%%%%%%%%%%%
\subsection{Driver Development}
\label{sec:driver}

The \textsf{childdoc} mechanism can also be use for the development
of definition files such as \LaTeX{} styles or classes.
This case differs from the above setup with multiple parts
included by |\include| in that no |\includeonly| should be invoked.
This can be achieved by starting the include file
(before |\ProvidesPackage|) with:
%
\begin{center}
\begin{tabular}{l}
|\input{childdoc.def}|\\
|\childdocforward{|\textit{main}|}|\\
\end{tabular}
\end{center}
%
or alternatively with:
%
\begin{center}
\begin{tabular}{l}
|\input{childdoc.def}|\\
|\childdocby{|\textit{main}|}|\\
\end{tabular}
\end{center}
%
Both forms have slightly different effects as described above.
The main file is prepared as usual, see \secref{sec:include}.

%%%%%%%%%%%%%%%%%%%%%%%%%%%%%%%%%%%%%%%%%%%%%%%%%%%%%%%%%%%%%%%%%%%%%%%%%%%%%%%%
\subsection{Legacy Detection}
\label{sec:detection}

The directive |\childdocmain| in the main file can detect
whether the complete document or merely a child is to be compiled
even without using the directive |\childdocof|.
This method is deprecated because it is less robust
and there is no compelling reason to use it;
it is merely provided for backward compatibility
and it may be removed in future versions.

If the detection mechanism is to be used,
it is mandatory to correctly specify
the filename of the main file as the argument of |\childdocmain|:
%
\begin{center}
\begin{tabular}{l}
|\input{childdoc.def}|\\
|\childdocmain{|\textit{main}|}|\\
\end{tabular}
\end{center}
%
If |\jobname| does not match the argument \textit{main} of |\childdocmain|,
it is assumed that |\jobname| points to the child file to be compiled.
When using |\childdocmain| with the main file specified as argument,
it suffices to start a child file
with just |\input{|\textit{main}|}|
without loading of the package and using |\childdocof|.
If instead all processing is done
with the appropriate \textsf{childdoc} directives,
the argument of \textit{main} of |\childdocmain| can be empty.

An alternative version of the command line processing described
in \secref{sec:commandline} using the detection mechanism reads:
%
\begin{center}
|... -jobname "|\textit{target}|" "|[\textit{flags}]%
[|\def\jobname{|\textit{dest}|}|]|\input{|\textit{main}|}"|
\end{center}

%%%%%%%%%%%%%%%%%%%%%%%%%%%%%%%%%%%%%%%%%%%%%%%%%%%%%%%%%%%%%%%%%%%%%%%%%%%%%%%%
\subsection{Manual Code}
\label{sec:manual}

In case one cannot be certain whether the definitions file |childdoc.def|
is installed on the target \TeX{} distribution
and one prefers not to ship it,
it is conceivable to paste a few relevant commands into the sources.

To that end, drop all statements |\input{childdoc.def}|
and perform the replacements as outlined below.
Instead of |\childdocmain{|\textit{main}|}| add the following code
to the top of the main file:
%
\begin{center}
\begin{tabular}{l}
|\||ifdefined\childdocname\endinput\||fi\newif\ifchilddoc|\\
|\edef\childdocname{\scantokens\expandafter{\jobname\noexpand}}|\\
|\def\childdocmain{|\textit{main}|}\||ifx\childdocmain\childdocname\||else|\\
|\childdoctrue\includeonly{\childdocname}\let\jobname\childdocmain\||fi|\\
\end{tabular}
\end{center}
%
Instead of |\childdocof{|\textit{main}|}| just include the main file
at the top of each child file:
%
\begin{center}
|\input{|\textit{main}|}|
\end{center}
%
A simple redirection |\childdocforward{|\textit{dest}|}| is achieved by:
%
\begin{center}
|\def\jobname{|\textit{dest}|}\input{\jobname}|
\end{center}
%
The redirection with prefix
|\childdocforwardprefix[|\textit{prefix}|]{|\textit{dest}|}|
is accomplished by:
%
\begin{center}
\begin{tabular}{l}
|{\edef\jobname{\scantokens\expandafter{\jobname\noexpand}}|\\
|\def\redirectjob |\textit{prefix}|#1~~~{\gdef\jobname{|\textit{dest}|#1}}|\\
|\expandafter\redirectjob\jobname~~~}\input{\jobname}|
\end{tabular}
\end{center}

In an alternative approach,
child documents can be compiled by a specific command line
without additional code or specific definitions:
%
\begin{center}
|... -jobname "|\textit{target}|" "|[\textit{flags}]%
|\includeonly{|\textit{dest}|}\input{|\textit{main}|}"|
\end{center}
%

%%%%%%%%%%%%%%%%%%%%%%%%%%%%%%%%%%%%%%%%%%%%%%%%%%%%%%%%%%%%%%%%%%%%%%%%%%%%%%%%
%%%%%%%%%%%%%%%%%%%%%%%%%%%%%%%%%%%%%%%%%%%%%%%%%%%%%%%%%%%%%%%%%%%%%%%%%%%%%%%%
\section{Information}

%%%%%%%%%%%%%%%%%%%%%%%%%%%%%%%%%%%%%%%%%%%%%%%%%%%%%%%%%%%%%%%%%%%%%%%%%%%%%%%%
\subsection{Copyright}

Copyright \copyright{} 2017--2018 Niklas Beisert

This work may be distributed and/or modified under the
conditions of the \LaTeX{} Project Public License, either version 1.3
of this license or (at your option) any later version.
The latest version of this license is in
  \url{http://www.latex-project.org/lppl.txt}
and version 1.3 or later is part of all distributions of \LaTeX{}
version 2005/12/01 or later.

This work has the LPPL maintenance status `maintained'.

The Current Maintainer of this work is Niklas Beisert.

This work consists of the files |README.txt|, |childdoc.ins| and |childdoc.dtx|
as well as the derived files |childdoc.def|, |cdocsamp.tex|
with |cdocsch1.tex|, |cdocsch2.tex|, |cdocspt3.tex|, |cdocspt4.tex|,
|cdocsdrf.tex|, |cdocsfn1.tex|, |cdocsfn2.tex|
as well as |childdoc.pdf|.

%%%%%%%%%%%%%%%%%%%%%%%%%%%%%%%%%%%%%%%%%%%%%%%%%%%%%%%%%%%%%%%%%%%%%%%%%%%%%%%%
\subsection{Files and Installation}

The package consists of the files:
%
\begin{center}
\begin{tabular}{ll}
    |README.txt|   & readme file \\
    |childdoc.ins| & installation file \\
    |childdoc.dtx| & source file \\
    |childdoc.def| & definition file \\
    |cdocsamp.tex| & sample main file \\
    |cdocsch1.tex| & sample include file \\
    |cdocsch2.tex| & sample include file \\
    |cdocspt3.tex| & sample part file \\
    |cdocspt4.tex| & sample part file \\
    |cdocsdrf.tex| & sample redirection file \\
    |cdocsfn1.tex| & sample redirection file \\
    |cdocsfn2.tex| & sample redirection file \\
    |childdoc.pdf| & manual
\end{tabular}
\end{center}
%
The distribution consists of the files
|README.txt|, |childdoc.ins| and |childdoc.dtx|.
%
\begin{itemize}
\item
Run (pdf)\LaTeX{} on |childdoc.dtx|
to compile the manual |childdoc.pdf| (this file).
\item
Run \LaTeX{} on |childdoc.ins| to create the definitions file |childdoc.def|
and the sample |cdocsamp.tex| with include files
|cdocsch1.tex|, |cdocsch2.tex|, |cdocspt3.tex|, |cdocspt4.tex|,
|cdocsdrf.tex|, |cdocsfn1.tex|, |cdocsfn2.tex|.
Then copy the file |childdoc.def| to an appropriate directory of your \LaTeX{}
distribution, e.g.\ \textit{texmf-root}|/tex/latex/childdoc|.
\end{itemize}

%%%%%%%%%%%%%%%%%%%%%%%%%%%%%%%%%%%%%%%%%%%%%%%%%%%%%%%%%%%%%%%%%%%%%%%%%%%%%%%%
\subsection{Related CTAN Packages}

There are several other packages which offer a similar functionality:
%
\begin{itemize}
\item
The packages
\href{http://ctan.org/pkg/docmute}{\textsf{docmute}},
\href{http://ctan.org/pkg/includex}{\textsf{includex}} and
\href{http://ctan.org/pkg/standalone}{\textsf{standalone}}
provide commands to include only the document body of
a child file thus allowing both files to be compiled individually.
\item
The packages \href{http://ctan.org/pkg/subdocs}{\textsf{subdocs}}
and \href{http://ctan.org/pkg/subfiles}{\textsf{subfiles}}
provide structures in which the main and child documents can be
encapsulated and allowing them to be compiled individually.
The inclusion mechanism is different from the conventional |\include|.
\item
The package \href{http://ctan.org/pkg/combine}{\textsf{combine}}
is an elaborate solution to combine several documents into one.
\end{itemize}
%
See also the CTAN topic \href{http://ctan.org/topic/subdocs}{\textsf{subdocs}}
for further related packages.
The present package differs from the above solutions in that
a document structure constructed with the conventional |\include| mechanism
just needs two extra commands at the top of every file
such that all constituent files can be compiled individually.

%%%%%%%%%%%%%%%%%%%%%%%%%%%%%%%%%%%%%%%%%%%%%%%%%%%%%%%%%%%%%%%%%%%%%%%%%%%%%%%%
%\subsection{Feature Suggestions}
%
%The following is a list of features which may be useful for future
%versions of this package:
%%
%\begin{itemize}
%\item
%\ldots
%\end{itemize}

%%%%%%%%%%%%%%%%%%%%%%%%%%%%%%%%%%%%%%%%%%%%%%%%%%%%%%%%%%%%%%%%%%%%%%%%%%%%%%%%
\subsection{Revision History}

%%%%%%%%%%%%%%%%%%%%%%%%%%%%%%%%%%%%%%%%
\paragraph{v2.0:} 2018/12/30

\begin{itemize}
\item
immediate forward processing
\item
added |\childdocby| mechanism
\item
manual restructured
\end{itemize}

%%%%%%%%%%%%%%%%%%%%%%%%%%%%%%%%%%%%%%%%
\paragraph{v1.6:} 2018/01/17

\begin{itemize}
\item
application for development of include files
\item
corrections to manual
\end{itemize}

%%%%%%%%%%%%%%%%%%%%%%%%%%%%%%%%%%%%%%%%
\paragraph{v1.5:} 2017/05/21

\begin{itemize}
\item
more complete structuring introduced
\item
|\childdocof| introduced
\item
|\childdoc| renamed to |\childdocmain|
\item
|\childredirect| renamed to |\childdocforward| and |\childdocforwardprefix|
and functionality expanded
\end{itemize}

%%%%%%%%%%%%%%%%%%%%%%%%%%%%%%%%%%%%%%%%
\paragraph{v1.0:} 2017/04/27

\begin{itemize}
\item
manual and install package
\item
first version published on CTAN
\end{itemize}

%%%%%%%%%%%%%%%%%%%%%%%%%%%%%%%%%%%%%%%%
\paragraph{v0.6:} 2017/04/26

\begin{itemize}
\item
redirection mechanism added
\end{itemize}

%%%%%%%%%%%%%%%%%%%%%%%%%%%%%%%%%%%%%%%%
\paragraph{v0.5:} 2017/04/26

\begin{itemize}
\item
functionality in definition file
\end{itemize}


%%%%%%%%%%%%%%%%%%%%%%%%%%%%%%%%%%%%%%%%%%%%%%%%%%%%%%%%%%%%%%%%%%%%%%%%%%%%%%%%
%%%%%%%%%%%%%%%%%%%%%%%%%%%%%%%%%%%%%%%%%%%%%%%%%%%%%%%%%%%%%%%%%%%%%%%%%%%%%%%%
%%%%%%%%%%%%%%%%%%%%%%%%%%%%%%%%%%%%%%%%%%%%%%%%%%%%%%%%%%%%%%%%%%%%%%%%%%%%%%%%
\appendix

\settowidth\MacroIndent{\rmfamily\scriptsize 000\ }

 \DocInput{childdoc.dtx}

\end{document}
%</driver>
% \fi
%
% %%%%%%%%%%%%%%%%%%%%%%%%%%%%%%%%%%%%%%%%%%%%%%%%%%%%%%%%%%%%%%%%%%%%%%%%%%%%%%
% %%%%%%%%%%%%%%%%%%%%%%%%%%%%%%%%%%%%%%%%%%%%%%%%%%%%%%%%%%%%%%%%%%%%%%%%%%%%%%
% \section{Sample}
%\iffalse
%<*samplemain>
%\fi
%
% The following presents a sample document
% with two chapters, two parts, a title page,
% a compile flag as well as three forwarding files to set the flag.
% It consists of eight |.tex| files:
% \begin{center}
% \begin{tabular}{ll}
% |cdocsamp.tex|&main file\\
% |cdocsch1.tex|&include file for chapter 1\\
% |cdocsch2.tex|&include file for chapter 2\\
% |cdocspt3.tex|&include file for part 3\\
% |cdocspt4.tex|&include file for part 4\\
% |cdocsdrf.tex|&forwarding file for main file in draft mode\\
% |cdocsfi1.tex|&forwarding file for final version of chapter 1\\
% |cdocsfi2.tex|&forwarding file for final version of chapter 2\\
% \end{tabular}
% \end{center}
% Each of the eight files can be compiled directly by the \LaTeX{} compiler.
%
% %%%%%%%%%%%%%%%%%%%%%%%%%%%%%%%%%%%%%%
% \paragraph{Main File.}
%
% The main file is called |cdocsamp.tex|.
%
% Load the \textsf{childdoc} definitions and
% declare the filename for the main document:
%    \begin{macrocode}
\input{childdoc.def}
\childdocmain{}
%    \end{macrocode}

% Optional override for |\version| flag:
%    \begin{macrocode}
%%\ifchilddoc\else\providecommand{\version}{draft}\fi
%    \end{macrocode}

% Define the default values for the |\version| flag
% (|final| for the main file and |draft| for childs):
%    \begin{macrocode}
\ifchilddoc
\providecommand{\version}{draft}
\else
\providecommand{\version}{final}
\fi
%    \end{macrocode}

% Load the standard document class:
%    \begin{macrocode}
\documentclass[12pt]{article}
%    \end{macrocode}

% Start the document body:
%    \begin{macrocode}
\begin{document}
%    \end{macrocode}

% Declare a title page.
% Print title, part of document being processed and version flag:
%    \begin{macrocode}
\addtocounter{page}{-1}
\begin{center}
{\LARGE\bfseries{}childdoc example\par}
\vspace{1cm}
\ifchilddoc
\ifchilddocmanual part\else chapter\fi:
`\childdocname' of `\childdocjob'\par
\else
main document: `\childdocjob'\par
\fi
version: \version\par
\end{center}
\newpage
%    \end{macrocode}

% Manually include selected file,
% otherwise process as usual:
%    \begin{macrocode}
\ifchilddocmanual
\section*{part `\childdocname'}
\input{\childdocname}
\else
%    \end{macrocode}

% Include the two chapters:
%    \begin{macrocode}
\include{cdocsch1}
\include{cdocsch2}
%    \end{macrocode}

% Include the two parts unless only chapters should be displayed:
%    \begin{macrocode}
\ifchilddoc\else
\section{part three}
\input{cdocspt3}
\section{part four}
\input{cdocspt4}
\fi
%    \end{macrocode}

% Process as usual until here:
%    \begin{macrocode}
\fi
%    \end{macrocode}

% End of document body:
%    \begin{macrocode}
\end{document}
%    \end{macrocode}
%\iffalse
%</samplemain>
%\fi
%
% %%%%%%%%%%%%%%%%%%%%%%%%%%%%%%%%%%%%%%
% \paragraph{Chapter Include Files.}
%
% The include files are called |cdocsch1.tex| and |cdocsch2.tex|.
%
%\iffalse
%<*samplechap1|samplechap2>
%\fi

% Optional override for |\version| flag:
%    \begin{macrocode}
%%\providecommand{\version}{final}
%    \end{macrocode}

% Include the main document:
%    \begin{macrocode}
\input{childdoc.def}
\childdocof{cdocsamp}
%    \end{macrocode}

%\iffalse
%</samplechap1|samplechap2>
%\fi
%
%\iffalse
%<*samplechap1>
%\fi
% Some text for chapter 1:
%    \begin{macrocode}
\section{one}
some text in chapter one
%    \end{macrocode}

%\iffalse
%</samplechap1>
%\fi
% Some text for chapter 2:
%\iffalse
%<*samplechap2>
%\fi
%    \begin{macrocode}
\section{two}
more text in chapter two
%    \end{macrocode}

%\iffalse
%</samplechap2>
%\fi
%
% %%%%%%%%%%%%%%%%%%%%%%%%%%%%%%%%%%%%%%
% \paragraph{Part Include Files.}
%
% The include files are called |cdocspt3.tex| and |cdocspt4.tex|.
%
%\iffalse
%<*samplepart3|samplepart4>
%\fi

% Optional override for |\version| flag:
%    \begin{macrocode}
%%\providecommand{\version}{final}
%    \end{macrocode}

% Include the main document:
%    \begin{macrocode}
\input{childdoc.def}
\childdocby{cdocsamp}
%    \end{macrocode}

%\iffalse
%</samplepart3|samplepart4>
%\fi
%
%\iffalse
%<*samplepart3>
%\fi
% Some text for part 3:
%    \begin{macrocode}
some text in part three
%    \end{macrocode}

%\iffalse
%</samplepart3>
%\fi
% Some text for part 4:
%\iffalse
%<*samplepart4>
%\fi
%    \begin{macrocode}
more text in part four
%    \end{macrocode}

%\iffalse
%</samplepart4>
%\fi
%
% %%%%%%%%%%%%%%%%%%%%%%%%%%%%%%%%%%%%%%
% \paragraph{Forwarding for a Complete Draft.}
%
% The following forwarding file |cdocsdrf.tex|
% compiles the main document in draft mode:
%\iffalse
%<*sampledraft>
%\fi
%    \begin{macrocode}
\def\version{draft}
\input{childdoc.def}
\childdocforward{cdocsamp}
%    \end{macrocode}

%\iffalse
%</sampledraft>
%\fi
%
% %%%%%%%%%%%%%%%%%%%%%%%%%%%%%%%%%%%%%%
% \paragraph{Forwarding for Final Version of the Chapters.}
%
% The following forwarding files |cdocsfn1.tex| and |cdocsfn2.tex|
% (with identical content)
% compile the final versions of the child documents
% |cdocsch1.tex| and |cdocsch2.tex|, respectively:
%\iffalse
%<*samplefinal>
%\fi
%    \begin{macrocode}
\def\version{final}
\input{childdoc.def}
\childdocforwardprefix[cdocsamp]{cdocsfn}{cdocsch}
%    \end{macrocode}

%\iffalse
%</samplefinal>
%\fi
%
% %%%%%%%%%%%%%%%%%%%%%%%%%%%%%%%%%%%%%%
% \paragraph{Command Line Processing.}
%
% The following three command lines generate the output files
% |cdocscld|, |cdocscl1| and |cdocscl2|
% which should be identical to
% |cdocsdrf|, |cdocsch1| and |cdocsfn2|, respectively:
% \begin{center}
% \begin{tabular}{l}
% |latex -jobname cdocscld \|\\
% |  "\def\version{draft}\input{childdoc.def}\childdocforward{cdocsamp}"|\\
% |latex -jobname cdocscl1 \|\\
% |  "\input{childdoc.def}\childdocforward[cdocsamp]{cdocsch1}"|\\
% |latex -jobname cdocscl2 \|\\
% |  "\def\version{final}\input{childdoc.def}\childdocforward{cdocsch2}"|
% \end{tabular}
% \end{center}
% Note that the trailing backslash on each first line
% merely continues the input to the second line
% (for convenient cut ant paste).
% Furthermore, the command |latex| can be replaced by any
% of its alternative versions such as |pdflatex|.
%
% %%%%%%%%%%%%%%%%%%%%%%%%%%%%%%%%%%%%%%%%%%%%%%%%%%%%%%%%%%%%%%%%%%%%%%%%%%%%%%
% %%%%%%%%%%%%%%%%%%%%%%%%%%%%%%%%%%%%%%%%%%%%%%%%%%%%%%%%%%%%%%%%%%%%%%%%%%%%%%
% \section{Implementation}
%\iffalse
%<*package>
%\fi
%
% This section describes the definitions file |childdoc.def|.

% The definitions cannot be loaded using |\usepackage| or |\RequirePackage|
% which has a mechanism to prevent loading a style file more than once.
% When loading the definitions by means of |\input|
% multiple instances have to be prevented manually:
%\iffalse
%This code needs to be before the `\ProvidesFile' directive
%which is defined at the beginning of this file.
%Therefore it is also placed there and commented out here.
%</package>
%<*discard>
%\fi
%    \begin{macrocode}
\ifdefined\childdocmain\endinput\fi
%    \end{macrocode}
%\iffalse
%</discard>
%<*package>
%\fi
%
% \macro{\ifchilddoc}
% \macro{\ifchilddocmanual}
% The conditional |\ifchilddoc| tells whether a
% child (true) or main (false) document is being compiled.
% The conditional |\ifchilddocmanual| tells whether
% the |\includeonly| mechanism is used (false) or
% the selection of child files must be performed manually (true).
% The definitions initialise to false:
%    \begin{macrocode}
\newif\ifchilddoc
\newif\ifchilddocmanual
%    \end{macrocode}

% \macro{\childdocname}
% \macro{\childdocjob}
% The macro |\childdocname| stores the name of the main document
% to be compiled. The macro |\childdocjob| stores the name of
% the document on which the \LaTeX{} compiler was originally invoked.
% The content of |\jobname| cannot be compared
% to filenames specified in the source due to different catcodes.
% The following code rescans |\jobname|, stores the result
% in |\childdocname| and saves a copy in |\childdocjob|:
%    \begin{macrocode}
\edef\childdocname{\scantokens\expandafter{\jobname\noexpand}}
\let\childdocjob\childdocname
%    \end{macrocode}

% \macro{\childdocdisable}
% The macro |\childdocdisable| prevents the main file
% from being processed more than once.
% At this stage, the main document command |\childdocmain|
% is assumed to be called once again where it should do nothing.
% Any subsequent call to it should prevent
% a secondary processing of the main document
% It overwrites the forwarding commands
% |\childdocof| and |\childdocforward|
% with empty macros to prevent further inclusions of the main document:
%    \begin{macrocode}
\newcommand{\childdocdisable}
{
  \renewcommand{\childdocmain}[1]{\renewcommand{\childdocmain}[1]{\endinput}}
  \renewcommand{\childdocof}[1]{}
  \renewcommand{\childdocby}[2][]{}
  \renewcommand{\childdocforward}[2][]{}
  \renewcommand{\childdocdisable}{}
}
%    \end{macrocode}

% \macro{\childdocmain}
% The macro |\childdocmain| is to be called at the top of the main file
% with nothing or the main filename (without extension) as argument.
% First, it breaks loops.
% If the argument is not empty and does not match |\childdocname|
% (which is set by the first inclusion of |childdoc.def|),
% |\ifchilddoc| is set to true, |\includeonly| is applied to the child file
% and |\jobname| is set to the main file
% (for proper handling of |.aux| files):
%    \begin{macrocode}
\newcommand{\childdocmain}[1]
{
  \childdocdisable\childdocmain{}
  \if?#1?\else
    \begingroup
      \def\childdoctmp{#1}
      \ifx\childdoctmp\childdocname
        \def\childdoctmp{}
      \else
        \def\childdoctmp
        {
          \childdoctrue
          \includeonly{\childdocname}
          \def\childdocjob{#1}
          \def\jobname{#1}
        }
      \fi
      \expandafter
    \endgroup
    \childdoctmp
  \fi
}
%    \end{macrocode}

% \macro{\childdocof}
% The command |\childdocof| redirects
% compilation to the main file |#1|.
%    \begin{macrocode}
\newcommand{\childdocof}[1]
{
  \childdocdisable
  \childdoctrue
  \includeonly{\childdocname}
  \def\jobname{#1}
  \def\childdocjob{#1}
  \input{#1}
}
%    \end{macrocode}

% \macro{\childdocby}
% The command |\childdocby| ....
%    \begin{macrocode}
\newcommand{\childdocby}[2][]
{
  \childdocdisable
  \childdoctrue
  \childdocmanualtrue
  \if?#1?\else
    \def\jobname{#2}
  \fi
  \def\childdocjob{#2}
  \input{#2}
  \endinput
}
%    \end{macrocode}

% \macro{\childdocforward}
% The command |\childdocforward| redirects
% compilation to the main file or
% (if the optional argument is given) a child file.
% Parameters are set as if the main file
% or a child file starting with |\childdocof| was compiled.
% Then compilation is handed over to the main file:
%    \begin{macrocode}
\newcommand{\childdocforward}[2][]
{
  \begingroup
    \if?#1?
      \def\childdoctmp
      {
        \def\childdocname{#2}
        \def\childdocjob{#2}
        \def\jobname{#2}
        \input{#2}
        \endinput
      }
    \else
      \def\childdoctmp
      {
        \childdocdisable
        \def\childdocname{#2}
        \childdoctrue
        \includeonly{#2}
        \def\childdocjob{#1}
        \def\jobname{#1}
        \input{#1}
        \endinput
      }
    \fi
    \expandafter
  \endgroup
  \childdoctmp
}
%    \end{macrocode}

% \macro{\childdocforwardprefix}
% The command |\childdocforwardprefix| redirects
% compilation to the main or a child file by means of a pattern.
% The prefix |#1| in the current filename is replaced by |#2|
% and the suffix of the current filename is kept
% (it is assumed that the filename does not contain the substring `|~~~|'
% which is used as a delimiter).
% Compilation is handed over to the new file by |\childdocforward|:
%    \begin{macrocode}
\newcommand{\childdocforwardprefix}[3][]
{
  \begingroup
    \def\childdocextract #2##1~~~{\def\childdoctmp{\childdocforward[#1]{#3##1}}}
    \expandafter\childdocextract\childdocname~~~
    \expandafter
  \endgroup
  \childdoctmp
}
%    \end{macrocode}

% \macro{\childdoc}
% The deprecated macro |\childdoc| is a legacy version of |\childdocmain|:
%    \begin{macrocode}
\newcommand{\childdoc}{\childdocmain}
%    \end{macrocode}

% \macro{\childdocredirect}
% The deprecated macro |\childdocredirect| is a legacy version
% of |\childdocforward| and |\childdocforwardprefix|:
%    \begin{macrocode}
\newcommand{\childdocredirect}[2][]
{
  \begingroup
    \if?#1?
      \def\childdoctmp{\childdocforward{#2}}
    \else
      \def\childdoctmp{\childdocforwardprefix{#1}{#2}}
    \fi
    \expandafter
  \endgroup
  \childdoctmp
}
%    \end{macrocode}

%\iffalse
%</package>
%\fi
%
\endinput
|\\
|\childdocforwardprefix{final}{child}|
\end{tabular}
\end{center}
%

Note that when several versions of a main file and/or of each child file
are to be generated, it may be convenient to set up a |Makefile| or
shell script to automatise the process.

%%%%%%%%%%%%%%%%%%%%%%%%%%%%%%%%%%%%%%%%%%%%%%%%%%%%%%%%%%%%%%%%%%%%%%%%%%%%%%%%
\subsection{Command Line Processing}
\label{sec:commandline}

The effect of redirection files can also be achieved by invoking
the \LaTeX{} compiler with a more elaborate command line.
Most conveniently this should be done as part
of a shell script or a |Makefile|.

When using \textsf{childdoc} in the main file, the following
command lines effectively perform a redirection
(note that depending on the shell being used,
backslashes may have to be doubled: `|\|' $\to$ `|\\|'):
%
\begin{center}
|... -jobname "|\textit{target}|" |\\|"|[\textit{flags}]%
|% \iffalse
%
% childdoc.dtx Copyright (C) 2017-2018 Niklas Beisert
%
% This work may be distributed and/or modified under the
% conditions of the LaTeX Project Public License, either version 1.3
% of this license or (at your option) any later version.
% The latest version of this license is in
%   http://www.latex-project.org/lppl.txt
% and version 1.3 or later is part of all distributions of LaTeX
% version 2005/12/01 or later.
%
% This work has the LPPL maintenance status `maintained'.
%
% The Current Maintainer of this work is Niklas Beisert.
%
% This work consists of the files childdoc.dtx and childdoc.ins
% and the derived files childdoc.def and cdocsamp.tex with
% cdocsch1.tex, cdocsch2.tex, cdocsdrf.tex, cdocsfn1.tex, cdocsfn2.tex.
%
%<package>\ifdefined\childdocmain\endinput\fi
%<package>\ProvidesFile{childdoc.def}[2018/12/30 v2.0 child document driver]
%<samplemain>\ProvidesFile{cdocsamp.tex}[2018/12/30 v2.0 sample for childdoc]
%<*driver>
%\ProvidesFile{childdoc.drv}[2018/12/30 v2.0 childdoc reference manual file]
\PassOptionsToClass{10pt,a4paper}{article}
\documentclass{ltxdoc}

\usepackage[margin=35mm]{geometry}
\usepackage{hyperref}
\usepackage{hyperxmp}
\usepackage[usenames]{color}

\hypersetup{colorlinks=true}
\hypersetup{pdfstartview=FitH}
\hypersetup{pdfpagemode=UseNone}
\hypersetup{pdfsource={}}
\hypersetup{pdflang={en-UK}}
\hypersetup{pdfcopyright={Copyright 2017-2018 Niklas Beisert.
  This work may be distributed and/or modified under the
  conditions of the LaTeX Project Public License, either version 1.3
  of this license or (at your option) any later version.}}
\hypersetup{pdflicenseurl={http://www.latex-project.org/lppl.txt}}
\hypersetup{pdfcontactaddress={ETH Zurich, ITP, HIT K,
  Wolfgang-Pauli-Strasse 27}}
\hypersetup{pdfcontactpostcode={8093}}
\hypersetup{pdfcontactcity={Zurich}}
\hypersetup{pdfcontactcountry={Switzerland}}
\hypersetup{pdfcontactemail={nbeisert@itp.phys.ethz.ch}}
\hypersetup{pdfcontacturl={http://people.phys.ethz.ch/\xmptilde nbeisert/}}

\newcommand{\secref}[1]{\hyperref[#1]{section \ref*{#1}}}

\parskip1ex
\parindent0pt
\let\olditemize\itemize
\def\itemize{\olditemize\parskip0pt}

\begin{document}

\title{The \textsf{childdoc} Package}
\hypersetup{pdftitle={The childdoc Package}}
\author{Niklas Beisert\\[2ex]
  Institut f\"ur Theoretische Physik\\
  Eidgen\"ossische Technische Hochschule Z\"urich\\
  Wolfgang-Pauli-Strasse 27, 8093 Z\"urich, Switzerland\\[1ex]
  \href{mailto:nbeisert@itp.phys.ethz.ch}
  {\texttt{nbeisert@itp.phys.ethz.ch}}}
\hypersetup{pdfauthor={Niklas Beisert}}
\hypersetup{pdfsubject={Manual for the LaTeX2e Package childdoc}}
\date{30 December 2018, \textsf{v2.0}}
\maketitle

\begin{abstract}\noindent
\textsf{childdoc} is a \LaTeXe{} package
that enables the direct compilation
of document sections included by |\include|
to individual files.
\end{abstract}

\begingroup
\parskip0ex
\tableofcontents
\endgroup

%%%%%%%%%%%%%%%%%%%%%%%%%%%%%%%%%%%%%%%%%%%%%%%%%%%%%%%%%%%%%%%%%%%%%%%%%%%%%%%%
%%%%%%%%%%%%%%%%%%%%%%%%%%%%%%%%%%%%%%%%%%%%%%%%%%%%%%%%%%%%%%%%%%%%%%%%%%%%%%%%
\section{Introduction}

\LaTeX{} provides a mechanism to structure a large document (such as a book)
into a main file and several child files (containing the chapters)
using the |\include| command.
This mechanism is beneficial for documents
which span hundreds of pages in order to
make the source file(s) more manageable.
Moreover, compilation can be restricted to
selected child files by means of the |\includeonly| command.
The latter feature can be used to reduce the compilation time while editing
(this was significantly more useful in the earlier days of \LaTeX{})
or to generate a smaller document which is easier to navigate.
Another application of |\includeonly| is to generate
documents consisting of selected parts of the complete document.

However, there are a few drawbacks of the plain |\include| mechanism:
\begin{itemize}
\item
The child files cannot be compiled on their own,
they can only be compiled via the main file.
A naive editing environment
(such as a text editor with an option
to have the current file processed by \LaTeX)
may require one to switch to the main file before compiling;
attempting to compile the child file produces errors.
\item
The main file must be modified (each time)
to adjust the |\includeonly| command
to the present needs. This easily leaves the main file in a messy state.
\item
The generated document will always carry the filename
of the main document. This is inconvenient if
several child files are to be compiled and
to be kept for distribution.
\end{itemize}

The present package provides a simple interface
to make child files individually compilable by \LaTeX{}.
Compiling a child file then has the same effect as compiling
the main file with an |\includeonly| command
to select the appropriate child.
Moreover the generated document will carry the name of the child
rather than the main file.
This resolves all three above issues.

This feature is meant to make the editing of books,
thesis documents and lecture notes somewhat more convenient.
However, the package can also be used efficiently for
composing a series of documents (such as exercise sheets)
which are typically distributed individually.
It then assists the author in generating the individual documents
(potentially in different versions)
as well as a document containing the collected series.
Another application is in developing style files
or other kinds of included material
where compilation of the style file could redirect
to a sample or test file.

%%%%%%%%%%%%%%%%%%%%%%%%%%%%%%%%%%%%%%%%%%%%%%%%%%%%%%%%%%%%%%%%%%%%%%%%%%%%%%%%
%%%%%%%%%%%%%%%%%%%%%%%%%%%%%%%%%%%%%%%%%%%%%%%%%%%%%%%%%%%%%%%%%%%%%%%%%%%%%%%%
\section{Usage}

First of all, the package \textsf{childdoc} is \emph{not} a standard
\LaTeXe{} |.sty| style file! Therefore it needs to be invoked in
a non-standard way.

%%%%%%%%%%%%%%%%%%%%%%%%%%%%%%%%%%%%%%%%%%%%%%%%%%%%%%%%%%%%%%%%%%%%%%%%%%%%%%%%
\subsection{Included Files}
\label{sec:include}

%%%%%%%%%%%%%%%%%%%%%%%%%%%%%%%%%%%%%%%%
\DescribeMacro{\childdocmain}
To use the package, add the commands
\begin{center}
\begin{tabular}{l}
|\input{childdoc.def}|\\
|\childdocmain{}|\\
\end{tabular}
\end{center}
at the very top of the main \LaTeX{} file,
in particular \emph{before} the |\documentclass| statement!
The argument of |\childdocmain| should be left empty
(but it must be present).

%%%%%%%%%%%%%%%%%%%%%%%%%%%%%%%%%%%%%%%%
\DescribeMacro{\childdocof}
Furthermore, add the commands
\begin{center}
\begin{tabular}{l}
|\input{childdoc.def}|\\
|\childdocof{|\textit{main}|}|\\
\end{tabular}
\end{center}
at the top of every child file \textit{child}
which is included by |\include{|\textit{child}|}|
from within the main file
(or at least for those files to be compiled individually).
The argument \textit{main} must be the filename of the main file.

There are a couple of
considerations in setting up the main and child documents:

%%%%%%%%%%%%%%%%%%%%%%%%%%%%%%%%%%%%%%%%
\paragraph{Restrictions.}

Please note the following restrictions:
\begin{itemize}
\item
|\childdocmain| must be called with one argument \textit{main}
to ensure compatibility with earlier version of the package.
It must either be empty (|\childdocmain{}|)
or precisely match the filename of the main file in which it is specified.
See \secref{sec:detection} for further information.
\item
The filename \textit{main} must be specified without the |.tex| extension.
\item
The filename \textit{main} is case sensitive
(even in case-insensitive file systems)
due to internal string comparison.
\item
The argument \textit{main} should be fully expanded, it cannot be a macro.
\item
Subdirectories and special characters should be avoided in filenames.
\item
The command |\childdocmain{|\textit{main}|}| must be followed by a whitespace.
It should not be followed immediately by another command
or by a comment mark `|%|'.
This is because the \TeX{} parser reads the token immediately following
the argument of |\childdocmain| and puts it
at the beginning of every child section;
however, a white\-space is ignored.
\end{itemize}

%%%%%%%%%%%%%%%%%%%%%%%%%%%%%%%%%%%%%%%%
\paragraph{Content of Main File.}

It is advisable to place all content in the child files included by |\include|.
Any output contained in the main file will appear in all child documents
unless suppressed manually;
it cannot be suppressed automatically by the |\includeonly| directive
and thus should normally be avoided.
A method to include some content in the main file
by means of conditional processing is described in \secref{sec:conditional}.

%%%%%%%%%%%%%%%%%%%%%%%%%%%%%%%%%%%%%%%%
\paragraph{Page Numbering.}

When only a part of the document is compiled,
the appropriate numbering of pages
(as well as other status parameters)
is determined from the |.aux| files.
The latter contain information from previous passes.
However this information needs to propagate through
all intermediate child documents.
Therefore the page numbering in child documents may well
be inconsistent until the complete document is compiled at least once.

A useful (if unconventional) way to always ensure a consistent
page numbering is to restart the numbering in each child document
and denote the pages by `\textit{child}|.|\textit{page}'
where \textit{child} represents the chapter/section number of the child file.
This can be achieved by the command
|\numberwithin{page}{|\textit{child}|}|
of the \textsf{amsmath} package
where \textit{child} can be |chapter| or |section|
depending on the chosen structuring.
Alternatively, one can modify the macro |\thepage| appropriately
and reset the counter |page| at the start of each child file.

%%%%%%%%%%%%%%%%%%%%%%%%%%%%%%%%%%%%%%%%%%%%%%%%%%%%%%%%%%%%%%%%%%%%%%%%%%%%%%%%
\subsection{Conditional Processing}
\label{sec:conditional}

The package provides a mechanism to compile different versions
of a document. To customise the versions further some conditional processing
can come in handy to distinguish which version is being compiled.
The package provides two macros to describe the compilation context:

%%%%%%%%%%%%%%%%%%%%%%%%%%%%%%%%%%%%%%%%
\DescribeMacro{\ifchilddoc}
The conditional |\ifchilddoc| distinguishes between the compilation of
child documents and the main document:
%
\begin{center}
|\ifchilddoc |\textit{child-code}| |[|\||else |\textit{main-code}]| \||fi|
\end{center}

%%%%%%%%%%%%%%%%%%%%%%%%%%%%%%%%%%%%%%%%
\DescribeMacro{\childdocname}
\DescribeMacro{\childdocjob}
The macro |\childdocname| contains the filename (without extension)
of the main or child file being processed.
Note that |\childdocjob| will always contain the name of the main file.

%%%%%%%%%%%%%%%%%%%%%%%%%%%%%%%%%%%%%%%%
\paragraph{Title Page.}

Conditional processing can be used to include a title or banner page
in the main document when proper precautions are taken.
Importantly, the code in the main file should ensure that the page counter
(as well as other status parameters which are stored in the |.aux| files)
takes the same value after the conditional processing.
Otherwise the page numbers may take divergent values
depending on which part is compiled.

For example, a title page could be declared by:
%
\begin{center}
\begin{tabular}{l}
|\ifchilddoc\||else|\\
|\addtocounter{page}{-1}|\\
\textit{code for title page}\\
|\newpage|\\
|\||fi|
\end{tabular}
\end{center}
%
A banner page for the child documents can be generated by:
%
\begin{center}
\begin{tabular}{l}
|\ifchilddoc|\\
|\addtocounter{page}{-1}|\\
\textit{code for banner page}\\
|\newpage|\\
|\||fi|
\end{tabular}
\end{center}
%
Here one could write a message such as:
\begin{center}
|This is the part \childdocname{} of \childdocjob{}.|
\end{center}

%%%%%%%%%%%%%%%%%%%%%%%%%%%%%%%%%%%%%%%%%%%%%%%%%%%%%%%%%%%%%%%%%%%%%%%%%%%%%%%%
\subsection{Flags}
\label{sec:flags}

The package makes it easy to generate different versions
of the main or child documents.
To this end compilation flags can be defined
and assigned different default values.
They will be particularly useful in conjunction
with the forwarding mechanism described in \secref{sec:forward}.

For example, it may be useful to have a flag |\version|
which can be set to |draft| or |final|.
The document source will contain some conditional code
depending on the value of |\version|.
Suppose further, the flag should default to |final| for the main file
and to |draft| for child files
which is a natural assignment for editing the document.
This is achieved by placing the following code
in the preamble of the main document
(below the |\childdocmain| directive):
%
\begin{center}
\begin{tabular}{l}
|\ifchilddoc|\\
|\providecommand{\version}{draft}|\\
|\||else|\\
|\providecommand{\version}{final}|\\
|\||fi|
\end{tabular}
\end{center}
%
The definition by |\providecommand| makes sure
that previous definitions are not overwritten.
Further statements |\providecommand{\version}{...}|
can thus be added before the above code to override it.

For the main file, one might add a line
(between |\childdocmain| and the above block)
%
\begin{center}
|%\ifchilddoc\||else\providecommand{\version}{draft}\||fi|
\end{center}
%
which can be uncommented to produce a draft version.
Likewise one can add a line to the very top of a child file
(above the |\childdocof{|\textit{main}|}| directive)
%
\begin{center}
|%\providecommand{\version}{final}|
\end{center}
%
which can be uncommented to produce the final version of this child document.

%%%%%%%%%%%%%%%%%%%%%%%%%%%%%%%%%%%%%%%%%%%%%%%%%%%%%%%%%%%%%%%%%%%%%%%%%%%%%%%%
\subsection{Forwarding}
\label{sec:forward}

Different versions of the main or child documents
using compilation flags as described in \secref{sec:flags}
can be (permanently) stored in different files
for convenient compilation, viewing and distribution.
To this end, the package defines a command
to pass on compilation to a different file:

%%%%%%%%%%%%%%%%%%%%%%%%%%%%%%%%%%%%%%%%
\DescribeMacro{\childdocforward}
The command |\childdocforward| redirects processing to
another source file:
%
\begin{center}
\begin{tabular}{l}
|\input{childdoc.def}|\\
|\childdocforward[|\textit{main}|]{|\textit{dest}|}|\\
\end{tabular}
\end{center}
%
The argument \textit{dest} is the destination file
(without extension).
It should be the main file or one of the child files.
Note that further \textsf{childdoc} directives
such as |\childdocof| and |\childdocforward|
in the indicated file will be processed in this form.
The optional argument \textit{main}
passes on directly to the main file \textit{main}
while pretending to compile the child \textit{dest}.
This form behaves as if \textit{dest}
issues |\childdocof{|\textit{main}|}| right away,
and no further \textsf{childdoc} directives will be processed.

%%%%%%%%%%%%%%%%%%%%%%%%%%%%%%%%%%%%%%%%
\DescribeMacro{\...prefix}
In the alternative form |\childdocforwardprefix|,
%
\begin{center}
\begin{tabular}{l}
|\input{childdoc.def}|\\
|\childdocforwardprefix[|\textit{main}|]{|\textit{prefix}|}{|\textit{dest}|}|
\end{tabular}
\end{center}
%
the destination file is determined by a pattern
depending on the current file:
To make this work, the current file must be called
`{\textit{prefix}\hspace{0.2em}\textit{suffix}}'
with \textit{prefix} matching precisely the argument.
Processing is then passed on to the file
`{\textit{dest}\hspace{0.2em}\textit{suffix}}'.
Surely, the same effect is achieved by
directly specifying the
argument `{\textit{dest}\hspace{0.2em}\textit{suffix}}'
in the first form.
However, that requires to set up a different file
for each child. With the alternative form of the command
all these files can have exactly the same content
which simplifies setting them up and maintaining them.

For example, the following file |draft.tex|
with a compilation flag |\version| as described in \secref{sec:flags}
compiles the main document as a draft:
%
\begin{center}
\begin{tabular}{l}
|\def\version{draft}|\\
|\input{childdoc.def}|\\
|\childdocforward{|\textit{main}|}|
\end{tabular}
\end{center}
%
Likewise, the following files |final|\textit{nn}|.tex|
compile the final version of the child document
|child|\textit{nn}|.tex|:
%
\begin{center}
\begin{tabular}{l}
|\def\version{final}|\\
|\input{childdoc.def}|\\
|\childdocforwardprefix{final}{child}|
\end{tabular}
\end{center}
%

Note that when several versions of a main file and/or of each child file
are to be generated, it may be convenient to set up a |Makefile| or
shell script to automatise the process.

%%%%%%%%%%%%%%%%%%%%%%%%%%%%%%%%%%%%%%%%%%%%%%%%%%%%%%%%%%%%%%%%%%%%%%%%%%%%%%%%
\subsection{Command Line Processing}
\label{sec:commandline}

The effect of redirection files can also be achieved by invoking
the \LaTeX{} compiler with a more elaborate command line.
Most conveniently this should be done as part
of a shell script or a |Makefile|.

When using \textsf{childdoc} in the main file, the following
command lines effectively perform a redirection
(note that depending on the shell being used,
backslashes may have to be doubled: `|\|' $\to$ `|\\|'):
%
\begin{center}
|... -jobname "|\textit{target}|" |\\|"|[\textit{flags}]%
|\input{childdoc.def}\childdocforward[|\textit{main}|]{|\textit{dest}|}"|
\end{center}
%
Here \textit{target} is the name of the output file,
\textit{main} is the name of the main file
and \textit{dest} is the name of the main or child file to be processed
(all filenames without extensions).
The optional argument \textit{main} can be omitted
if \textit{main} matches \textit{dest}.
Optionally, compilation \textit{flags} can be defined via |\def| commands.
This command line makes the \TeX{} engine believe
it is compiling the file \textit{target}
whose content is specified as the latter parameter.
The provided code then forwards the processing to
\textit{main} or \textit{dest} as described in \secref{sec:forward}.

%%%%%%%%%%%%%%%%%%%%%%%%%%%%%%%%%%%%%%%%%%%%%%%%%%%%%%%%%%%%%%%%%%%%%%%%%%%%%%%%
\subsection{Include by Input}
\label{sec:input}

Including child documents by |\include| has some restrictions by design.
Most notably, the content of a child document always occupies
its own set of pages; pages cannot be shared between child documents.
Usually, this behaviour makes perfect sense
because each child document contain an essential part of the document.
However, in some situations it may be desirable to compose
a document from a collection of parts
without having mandatory page breaks between then.
For this case, the package
provides a mechanism to include parts
by |\input| which can also be processed individually.
However, by construction this mechanism
requires manual handling of the content to be output.

%%%%%%%%%%%%%%%%%%%%%%%%%%%%%%%%%%%%%%%%
\DescribeMacro{\ifchilddocmanual}
The main file should be prepared as usual, see \secref{sec:include}.
However, the document body must make a distinction
between processing of an individual part and of the main document, e.g.:
%
\begin{center}
\begin{tabular}{l}
|\ifchilddocmanual|\\
|\input{\childdocname}|\\
|\||else|\\
\textit{document body with }|\input{|\textit{part}|}|\\
|\||fi|
\end{tabular}
\end{center}
%
The conditional |\ifchilddocmanual| is true whenever
a part to be included by |\input| is being compiled,
and the name of the part is stored in |\childdocname|.

%%%%%%%%%%%%%%%%%%%%%%%%%%%%%%%%%%%%%%%%
\DescribeMacro{\childdocby}
Each part to be included by |\input| should start with:
%
\begin{center}
\begin{tabular}{l}
|\input{childdoc.def}|\\
|\childdocby{|\textit{main}|}|\\
\end{tabular}
\end{center}
%
The directive |\childdocby| is similar to |\childdocof|
described in \secref{sec:include},
but the subsequent selection of content must be done manually.
To that end, both |\ifchilddoc| and |\ifchilddocmanual|
will be true upon processing of a part,
and the name of the part is stored in |\childdocname|.
Note that |\jobname| will be set to the filename of the current part
so that each part receives an individual |.aux| file
that does not interfere with the |.aux| file(s) of the main document.
This behaviour can be altered by the alternative form
|\childdocby[*]{|\textit{main}|}| (with a non-empty optional argument)
which uses the |.aux| file of the main document
by setting |\jobname| to \textit{main}.

%%%%%%%%%%%%%%%%%%%%%%%%%%%%%%%%%%%%%%%%%%%%%%%%%%%%%%%%%%%%%%%%%%%%%%%%%%%%%%%%
\subsection{Driver Development}
\label{sec:driver}

The \textsf{childdoc} mechanism can also be use for the development
of definition files such as \LaTeX{} styles or classes.
This case differs from the above setup with multiple parts
included by |\include| in that no |\includeonly| should be invoked.
This can be achieved by starting the include file
(before |\ProvidesPackage|) with:
%
\begin{center}
\begin{tabular}{l}
|\input{childdoc.def}|\\
|\childdocforward{|\textit{main}|}|\\
\end{tabular}
\end{center}
%
or alternatively with:
%
\begin{center}
\begin{tabular}{l}
|\input{childdoc.def}|\\
|\childdocby{|\textit{main}|}|\\
\end{tabular}
\end{center}
%
Both forms have slightly different effects as described above.
The main file is prepared as usual, see \secref{sec:include}.

%%%%%%%%%%%%%%%%%%%%%%%%%%%%%%%%%%%%%%%%%%%%%%%%%%%%%%%%%%%%%%%%%%%%%%%%%%%%%%%%
\subsection{Legacy Detection}
\label{sec:detection}

The directive |\childdocmain| in the main file can detect
whether the complete document or merely a child is to be compiled
even without using the directive |\childdocof|.
This method is deprecated because it is less robust
and there is no compelling reason to use it;
it is merely provided for backward compatibility
and it may be removed in future versions.

If the detection mechanism is to be used,
it is mandatory to correctly specify
the filename of the main file as the argument of |\childdocmain|:
%
\begin{center}
\begin{tabular}{l}
|\input{childdoc.def}|\\
|\childdocmain{|\textit{main}|}|\\
\end{tabular}
\end{center}
%
If |\jobname| does not match the argument \textit{main} of |\childdocmain|,
it is assumed that |\jobname| points to the child file to be compiled.
When using |\childdocmain| with the main file specified as argument,
it suffices to start a child file
with just |\input{|\textit{main}|}|
without loading of the package and using |\childdocof|.
If instead all processing is done
with the appropriate \textsf{childdoc} directives,
the argument of \textit{main} of |\childdocmain| can be empty.

An alternative version of the command line processing described
in \secref{sec:commandline} using the detection mechanism reads:
%
\begin{center}
|... -jobname "|\textit{target}|" "|[\textit{flags}]%
[|\def\jobname{|\textit{dest}|}|]|\input{|\textit{main}|}"|
\end{center}

%%%%%%%%%%%%%%%%%%%%%%%%%%%%%%%%%%%%%%%%%%%%%%%%%%%%%%%%%%%%%%%%%%%%%%%%%%%%%%%%
\subsection{Manual Code}
\label{sec:manual}

In case one cannot be certain whether the definitions file |childdoc.def|
is installed on the target \TeX{} distribution
and one prefers not to ship it,
it is conceivable to paste a few relevant commands into the sources.

To that end, drop all statements |\input{childdoc.def}|
and perform the replacements as outlined below.
Instead of |\childdocmain{|\textit{main}|}| add the following code
to the top of the main file:
%
\begin{center}
\begin{tabular}{l}
|\||ifdefined\childdocname\endinput\||fi\newif\ifchilddoc|\\
|\edef\childdocname{\scantokens\expandafter{\jobname\noexpand}}|\\
|\def\childdocmain{|\textit{main}|}\||ifx\childdocmain\childdocname\||else|\\
|\childdoctrue\includeonly{\childdocname}\let\jobname\childdocmain\||fi|\\
\end{tabular}
\end{center}
%
Instead of |\childdocof{|\textit{main}|}| just include the main file
at the top of each child file:
%
\begin{center}
|\input{|\textit{main}|}|
\end{center}
%
A simple redirection |\childdocforward{|\textit{dest}|}| is achieved by:
%
\begin{center}
|\def\jobname{|\textit{dest}|}\input{\jobname}|
\end{center}
%
The redirection with prefix
|\childdocforwardprefix[|\textit{prefix}|]{|\textit{dest}|}|
is accomplished by:
%
\begin{center}
\begin{tabular}{l}
|{\edef\jobname{\scantokens\expandafter{\jobname\noexpand}}|\\
|\def\redirectjob |\textit{prefix}|#1~~~{\gdef\jobname{|\textit{dest}|#1}}|\\
|\expandafter\redirectjob\jobname~~~}\input{\jobname}|
\end{tabular}
\end{center}

In an alternative approach,
child documents can be compiled by a specific command line
without additional code or specific definitions:
%
\begin{center}
|... -jobname "|\textit{target}|" "|[\textit{flags}]%
|\includeonly{|\textit{dest}|}\input{|\textit{main}|}"|
\end{center}
%

%%%%%%%%%%%%%%%%%%%%%%%%%%%%%%%%%%%%%%%%%%%%%%%%%%%%%%%%%%%%%%%%%%%%%%%%%%%%%%%%
%%%%%%%%%%%%%%%%%%%%%%%%%%%%%%%%%%%%%%%%%%%%%%%%%%%%%%%%%%%%%%%%%%%%%%%%%%%%%%%%
\section{Information}

%%%%%%%%%%%%%%%%%%%%%%%%%%%%%%%%%%%%%%%%%%%%%%%%%%%%%%%%%%%%%%%%%%%%%%%%%%%%%%%%
\subsection{Copyright}

Copyright \copyright{} 2017--2018 Niklas Beisert

This work may be distributed and/or modified under the
conditions of the \LaTeX{} Project Public License, either version 1.3
of this license or (at your option) any later version.
The latest version of this license is in
  \url{http://www.latex-project.org/lppl.txt}
and version 1.3 or later is part of all distributions of \LaTeX{}
version 2005/12/01 or later.

This work has the LPPL maintenance status `maintained'.

The Current Maintainer of this work is Niklas Beisert.

This work consists of the files |README.txt|, |childdoc.ins| and |childdoc.dtx|
as well as the derived files |childdoc.def|, |cdocsamp.tex|
with |cdocsch1.tex|, |cdocsch2.tex|, |cdocspt3.tex|, |cdocspt4.tex|,
|cdocsdrf.tex|, |cdocsfn1.tex|, |cdocsfn2.tex|
as well as |childdoc.pdf|.

%%%%%%%%%%%%%%%%%%%%%%%%%%%%%%%%%%%%%%%%%%%%%%%%%%%%%%%%%%%%%%%%%%%%%%%%%%%%%%%%
\subsection{Files and Installation}

The package consists of the files:
%
\begin{center}
\begin{tabular}{ll}
    |README.txt|   & readme file \\
    |childdoc.ins| & installation file \\
    |childdoc.dtx| & source file \\
    |childdoc.def| & definition file \\
    |cdocsamp.tex| & sample main file \\
    |cdocsch1.tex| & sample include file \\
    |cdocsch2.tex| & sample include file \\
    |cdocspt3.tex| & sample part file \\
    |cdocspt4.tex| & sample part file \\
    |cdocsdrf.tex| & sample redirection file \\
    |cdocsfn1.tex| & sample redirection file \\
    |cdocsfn2.tex| & sample redirection file \\
    |childdoc.pdf| & manual
\end{tabular}
\end{center}
%
The distribution consists of the files
|README.txt|, |childdoc.ins| and |childdoc.dtx|.
%
\begin{itemize}
\item
Run (pdf)\LaTeX{} on |childdoc.dtx|
to compile the manual |childdoc.pdf| (this file).
\item
Run \LaTeX{} on |childdoc.ins| to create the definitions file |childdoc.def|
and the sample |cdocsamp.tex| with include files
|cdocsch1.tex|, |cdocsch2.tex|, |cdocspt3.tex|, |cdocspt4.tex|,
|cdocsdrf.tex|, |cdocsfn1.tex|, |cdocsfn2.tex|.
Then copy the file |childdoc.def| to an appropriate directory of your \LaTeX{}
distribution, e.g.\ \textit{texmf-root}|/tex/latex/childdoc|.
\end{itemize}

%%%%%%%%%%%%%%%%%%%%%%%%%%%%%%%%%%%%%%%%%%%%%%%%%%%%%%%%%%%%%%%%%%%%%%%%%%%%%%%%
\subsection{Related CTAN Packages}

There are several other packages which offer a similar functionality:
%
\begin{itemize}
\item
The packages
\href{http://ctan.org/pkg/docmute}{\textsf{docmute}},
\href{http://ctan.org/pkg/includex}{\textsf{includex}} and
\href{http://ctan.org/pkg/standalone}{\textsf{standalone}}
provide commands to include only the document body of
a child file thus allowing both files to be compiled individually.
\item
The packages \href{http://ctan.org/pkg/subdocs}{\textsf{subdocs}}
and \href{http://ctan.org/pkg/subfiles}{\textsf{subfiles}}
provide structures in which the main and child documents can be
encapsulated and allowing them to be compiled individually.
The inclusion mechanism is different from the conventional |\include|.
\item
The package \href{http://ctan.org/pkg/combine}{\textsf{combine}}
is an elaborate solution to combine several documents into one.
\end{itemize}
%
See also the CTAN topic \href{http://ctan.org/topic/subdocs}{\textsf{subdocs}}
for further related packages.
The present package differs from the above solutions in that
a document structure constructed with the conventional |\include| mechanism
just needs two extra commands at the top of every file
such that all constituent files can be compiled individually.

%%%%%%%%%%%%%%%%%%%%%%%%%%%%%%%%%%%%%%%%%%%%%%%%%%%%%%%%%%%%%%%%%%%%%%%%%%%%%%%%
%\subsection{Feature Suggestions}
%
%The following is a list of features which may be useful for future
%versions of this package:
%%
%\begin{itemize}
%\item
%\ldots
%\end{itemize}

%%%%%%%%%%%%%%%%%%%%%%%%%%%%%%%%%%%%%%%%%%%%%%%%%%%%%%%%%%%%%%%%%%%%%%%%%%%%%%%%
\subsection{Revision History}

%%%%%%%%%%%%%%%%%%%%%%%%%%%%%%%%%%%%%%%%
\paragraph{v2.0:} 2018/12/30

\begin{itemize}
\item
immediate forward processing
\item
added |\childdocby| mechanism
\item
manual restructured
\end{itemize}

%%%%%%%%%%%%%%%%%%%%%%%%%%%%%%%%%%%%%%%%
\paragraph{v1.6:} 2018/01/17

\begin{itemize}
\item
application for development of include files
\item
corrections to manual
\end{itemize}

%%%%%%%%%%%%%%%%%%%%%%%%%%%%%%%%%%%%%%%%
\paragraph{v1.5:} 2017/05/21

\begin{itemize}
\item
more complete structuring introduced
\item
|\childdocof| introduced
\item
|\childdoc| renamed to |\childdocmain|
\item
|\childredirect| renamed to |\childdocforward| and |\childdocforwardprefix|
and functionality expanded
\end{itemize}

%%%%%%%%%%%%%%%%%%%%%%%%%%%%%%%%%%%%%%%%
\paragraph{v1.0:} 2017/04/27

\begin{itemize}
\item
manual and install package
\item
first version published on CTAN
\end{itemize}

%%%%%%%%%%%%%%%%%%%%%%%%%%%%%%%%%%%%%%%%
\paragraph{v0.6:} 2017/04/26

\begin{itemize}
\item
redirection mechanism added
\end{itemize}

%%%%%%%%%%%%%%%%%%%%%%%%%%%%%%%%%%%%%%%%
\paragraph{v0.5:} 2017/04/26

\begin{itemize}
\item
functionality in definition file
\end{itemize}


%%%%%%%%%%%%%%%%%%%%%%%%%%%%%%%%%%%%%%%%%%%%%%%%%%%%%%%%%%%%%%%%%%%%%%%%%%%%%%%%
%%%%%%%%%%%%%%%%%%%%%%%%%%%%%%%%%%%%%%%%%%%%%%%%%%%%%%%%%%%%%%%%%%%%%%%%%%%%%%%%
%%%%%%%%%%%%%%%%%%%%%%%%%%%%%%%%%%%%%%%%%%%%%%%%%%%%%%%%%%%%%%%%%%%%%%%%%%%%%%%%
\appendix

\settowidth\MacroIndent{\rmfamily\scriptsize 000\ }

 \DocInput{childdoc.dtx}

\end{document}
%</driver>
% \fi
%
% %%%%%%%%%%%%%%%%%%%%%%%%%%%%%%%%%%%%%%%%%%%%%%%%%%%%%%%%%%%%%%%%%%%%%%%%%%%%%%
% %%%%%%%%%%%%%%%%%%%%%%%%%%%%%%%%%%%%%%%%%%%%%%%%%%%%%%%%%%%%%%%%%%%%%%%%%%%%%%
% \section{Sample}
%\iffalse
%<*samplemain>
%\fi
%
% The following presents a sample document
% with two chapters, two parts, a title page,
% a compile flag as well as three forwarding files to set the flag.
% It consists of eight |.tex| files:
% \begin{center}
% \begin{tabular}{ll}
% |cdocsamp.tex|&main file\\
% |cdocsch1.tex|&include file for chapter 1\\
% |cdocsch2.tex|&include file for chapter 2\\
% |cdocspt3.tex|&include file for part 3\\
% |cdocspt4.tex|&include file for part 4\\
% |cdocsdrf.tex|&forwarding file for main file in draft mode\\
% |cdocsfi1.tex|&forwarding file for final version of chapter 1\\
% |cdocsfi2.tex|&forwarding file for final version of chapter 2\\
% \end{tabular}
% \end{center}
% Each of the eight files can be compiled directly by the \LaTeX{} compiler.
%
% %%%%%%%%%%%%%%%%%%%%%%%%%%%%%%%%%%%%%%
% \paragraph{Main File.}
%
% The main file is called |cdocsamp.tex|.
%
% Load the \textsf{childdoc} definitions and
% declare the filename for the main document:
%    \begin{macrocode}
\input{childdoc.def}
\childdocmain{}
%    \end{macrocode}

% Optional override for |\version| flag:
%    \begin{macrocode}
%%\ifchilddoc\else\providecommand{\version}{draft}\fi
%    \end{macrocode}

% Define the default values for the |\version| flag
% (|final| for the main file and |draft| for childs):
%    \begin{macrocode}
\ifchilddoc
\providecommand{\version}{draft}
\else
\providecommand{\version}{final}
\fi
%    \end{macrocode}

% Load the standard document class:
%    \begin{macrocode}
\documentclass[12pt]{article}
%    \end{macrocode}

% Start the document body:
%    \begin{macrocode}
\begin{document}
%    \end{macrocode}

% Declare a title page.
% Print title, part of document being processed and version flag:
%    \begin{macrocode}
\addtocounter{page}{-1}
\begin{center}
{\LARGE\bfseries{}childdoc example\par}
\vspace{1cm}
\ifchilddoc
\ifchilddocmanual part\else chapter\fi:
`\childdocname' of `\childdocjob'\par
\else
main document: `\childdocjob'\par
\fi
version: \version\par
\end{center}
\newpage
%    \end{macrocode}

% Manually include selected file,
% otherwise process as usual:
%    \begin{macrocode}
\ifchilddocmanual
\section*{part `\childdocname'}
\input{\childdocname}
\else
%    \end{macrocode}

% Include the two chapters:
%    \begin{macrocode}
\include{cdocsch1}
\include{cdocsch2}
%    \end{macrocode}

% Include the two parts unless only chapters should be displayed:
%    \begin{macrocode}
\ifchilddoc\else
\section{part three}
\input{cdocspt3}
\section{part four}
\input{cdocspt4}
\fi
%    \end{macrocode}

% Process as usual until here:
%    \begin{macrocode}
\fi
%    \end{macrocode}

% End of document body:
%    \begin{macrocode}
\end{document}
%    \end{macrocode}
%\iffalse
%</samplemain>
%\fi
%
% %%%%%%%%%%%%%%%%%%%%%%%%%%%%%%%%%%%%%%
% \paragraph{Chapter Include Files.}
%
% The include files are called |cdocsch1.tex| and |cdocsch2.tex|.
%
%\iffalse
%<*samplechap1|samplechap2>
%\fi

% Optional override for |\version| flag:
%    \begin{macrocode}
%%\providecommand{\version}{final}
%    \end{macrocode}

% Include the main document:
%    \begin{macrocode}
\input{childdoc.def}
\childdocof{cdocsamp}
%    \end{macrocode}

%\iffalse
%</samplechap1|samplechap2>
%\fi
%
%\iffalse
%<*samplechap1>
%\fi
% Some text for chapter 1:
%    \begin{macrocode}
\section{one}
some text in chapter one
%    \end{macrocode}

%\iffalse
%</samplechap1>
%\fi
% Some text for chapter 2:
%\iffalse
%<*samplechap2>
%\fi
%    \begin{macrocode}
\section{two}
more text in chapter two
%    \end{macrocode}

%\iffalse
%</samplechap2>
%\fi
%
% %%%%%%%%%%%%%%%%%%%%%%%%%%%%%%%%%%%%%%
% \paragraph{Part Include Files.}
%
% The include files are called |cdocspt3.tex| and |cdocspt4.tex|.
%
%\iffalse
%<*samplepart3|samplepart4>
%\fi

% Optional override for |\version| flag:
%    \begin{macrocode}
%%\providecommand{\version}{final}
%    \end{macrocode}

% Include the main document:
%    \begin{macrocode}
\input{childdoc.def}
\childdocby{cdocsamp}
%    \end{macrocode}

%\iffalse
%</samplepart3|samplepart4>
%\fi
%
%\iffalse
%<*samplepart3>
%\fi
% Some text for part 3:
%    \begin{macrocode}
some text in part three
%    \end{macrocode}

%\iffalse
%</samplepart3>
%\fi
% Some text for part 4:
%\iffalse
%<*samplepart4>
%\fi
%    \begin{macrocode}
more text in part four
%    \end{macrocode}

%\iffalse
%</samplepart4>
%\fi
%
% %%%%%%%%%%%%%%%%%%%%%%%%%%%%%%%%%%%%%%
% \paragraph{Forwarding for a Complete Draft.}
%
% The following forwarding file |cdocsdrf.tex|
% compiles the main document in draft mode:
%\iffalse
%<*sampledraft>
%\fi
%    \begin{macrocode}
\def\version{draft}
\input{childdoc.def}
\childdocforward{cdocsamp}
%    \end{macrocode}

%\iffalse
%</sampledraft>
%\fi
%
% %%%%%%%%%%%%%%%%%%%%%%%%%%%%%%%%%%%%%%
% \paragraph{Forwarding for Final Version of the Chapters.}
%
% The following forwarding files |cdocsfn1.tex| and |cdocsfn2.tex|
% (with identical content)
% compile the final versions of the child documents
% |cdocsch1.tex| and |cdocsch2.tex|, respectively:
%\iffalse
%<*samplefinal>
%\fi
%    \begin{macrocode}
\def\version{final}
\input{childdoc.def}
\childdocforwardprefix[cdocsamp]{cdocsfn}{cdocsch}
%    \end{macrocode}

%\iffalse
%</samplefinal>
%\fi
%
% %%%%%%%%%%%%%%%%%%%%%%%%%%%%%%%%%%%%%%
% \paragraph{Command Line Processing.}
%
% The following three command lines generate the output files
% |cdocscld|, |cdocscl1| and |cdocscl2|
% which should be identical to
% |cdocsdrf|, |cdocsch1| and |cdocsfn2|, respectively:
% \begin{center}
% \begin{tabular}{l}
% |latex -jobname cdocscld \|\\
% |  "\def\version{draft}\input{childdoc.def}\childdocforward{cdocsamp}"|\\
% |latex -jobname cdocscl1 \|\\
% |  "\input{childdoc.def}\childdocforward[cdocsamp]{cdocsch1}"|\\
% |latex -jobname cdocscl2 \|\\
% |  "\def\version{final}\input{childdoc.def}\childdocforward{cdocsch2}"|
% \end{tabular}
% \end{center}
% Note that the trailing backslash on each first line
% merely continues the input to the second line
% (for convenient cut ant paste).
% Furthermore, the command |latex| can be replaced by any
% of its alternative versions such as |pdflatex|.
%
% %%%%%%%%%%%%%%%%%%%%%%%%%%%%%%%%%%%%%%%%%%%%%%%%%%%%%%%%%%%%%%%%%%%%%%%%%%%%%%
% %%%%%%%%%%%%%%%%%%%%%%%%%%%%%%%%%%%%%%%%%%%%%%%%%%%%%%%%%%%%%%%%%%%%%%%%%%%%%%
% \section{Implementation}
%\iffalse
%<*package>
%\fi
%
% This section describes the definitions file |childdoc.def|.

% The definitions cannot be loaded using |\usepackage| or |\RequirePackage|
% which has a mechanism to prevent loading a style file more than once.
% When loading the definitions by means of |\input|
% multiple instances have to be prevented manually:
%\iffalse
%This code needs to be before the `\ProvidesFile' directive
%which is defined at the beginning of this file.
%Therefore it is also placed there and commented out here.
%</package>
%<*discard>
%\fi
%    \begin{macrocode}
\ifdefined\childdocmain\endinput\fi
%    \end{macrocode}
%\iffalse
%</discard>
%<*package>
%\fi
%
% \macro{\ifchilddoc}
% \macro{\ifchilddocmanual}
% The conditional |\ifchilddoc| tells whether a
% child (true) or main (false) document is being compiled.
% The conditional |\ifchilddocmanual| tells whether
% the |\includeonly| mechanism is used (false) or
% the selection of child files must be performed manually (true).
% The definitions initialise to false:
%    \begin{macrocode}
\newif\ifchilddoc
\newif\ifchilddocmanual
%    \end{macrocode}

% \macro{\childdocname}
% \macro{\childdocjob}
% The macro |\childdocname| stores the name of the main document
% to be compiled. The macro |\childdocjob| stores the name of
% the document on which the \LaTeX{} compiler was originally invoked.
% The content of |\jobname| cannot be compared
% to filenames specified in the source due to different catcodes.
% The following code rescans |\jobname|, stores the result
% in |\childdocname| and saves a copy in |\childdocjob|:
%    \begin{macrocode}
\edef\childdocname{\scantokens\expandafter{\jobname\noexpand}}
\let\childdocjob\childdocname
%    \end{macrocode}

% \macro{\childdocdisable}
% The macro |\childdocdisable| prevents the main file
% from being processed more than once.
% At this stage, the main document command |\childdocmain|
% is assumed to be called once again where it should do nothing.
% Any subsequent call to it should prevent
% a secondary processing of the main document
% It overwrites the forwarding commands
% |\childdocof| and |\childdocforward|
% with empty macros to prevent further inclusions of the main document:
%    \begin{macrocode}
\newcommand{\childdocdisable}
{
  \renewcommand{\childdocmain}[1]{\renewcommand{\childdocmain}[1]{\endinput}}
  \renewcommand{\childdocof}[1]{}
  \renewcommand{\childdocby}[2][]{}
  \renewcommand{\childdocforward}[2][]{}
  \renewcommand{\childdocdisable}{}
}
%    \end{macrocode}

% \macro{\childdocmain}
% The macro |\childdocmain| is to be called at the top of the main file
% with nothing or the main filename (without extension) as argument.
% First, it breaks loops.
% If the argument is not empty and does not match |\childdocname|
% (which is set by the first inclusion of |childdoc.def|),
% |\ifchilddoc| is set to true, |\includeonly| is applied to the child file
% and |\jobname| is set to the main file
% (for proper handling of |.aux| files):
%    \begin{macrocode}
\newcommand{\childdocmain}[1]
{
  \childdocdisable\childdocmain{}
  \if?#1?\else
    \begingroup
      \def\childdoctmp{#1}
      \ifx\childdoctmp\childdocname
        \def\childdoctmp{}
      \else
        \def\childdoctmp
        {
          \childdoctrue
          \includeonly{\childdocname}
          \def\childdocjob{#1}
          \def\jobname{#1}
        }
      \fi
      \expandafter
    \endgroup
    \childdoctmp
  \fi
}
%    \end{macrocode}

% \macro{\childdocof}
% The command |\childdocof| redirects
% compilation to the main file |#1|.
%    \begin{macrocode}
\newcommand{\childdocof}[1]
{
  \childdocdisable
  \childdoctrue
  \includeonly{\childdocname}
  \def\jobname{#1}
  \def\childdocjob{#1}
  \input{#1}
}
%    \end{macrocode}

% \macro{\childdocby}
% The command |\childdocby| ....
%    \begin{macrocode}
\newcommand{\childdocby}[2][]
{
  \childdocdisable
  \childdoctrue
  \childdocmanualtrue
  \if?#1?\else
    \def\jobname{#2}
  \fi
  \def\childdocjob{#2}
  \input{#2}
  \endinput
}
%    \end{macrocode}

% \macro{\childdocforward}
% The command |\childdocforward| redirects
% compilation to the main file or
% (if the optional argument is given) a child file.
% Parameters are set as if the main file
% or a child file starting with |\childdocof| was compiled.
% Then compilation is handed over to the main file:
%    \begin{macrocode}
\newcommand{\childdocforward}[2][]
{
  \begingroup
    \if?#1?
      \def\childdoctmp
      {
        \def\childdocname{#2}
        \def\childdocjob{#2}
        \def\jobname{#2}
        \input{#2}
        \endinput
      }
    \else
      \def\childdoctmp
      {
        \childdocdisable
        \def\childdocname{#2}
        \childdoctrue
        \includeonly{#2}
        \def\childdocjob{#1}
        \def\jobname{#1}
        \input{#1}
        \endinput
      }
    \fi
    \expandafter
  \endgroup
  \childdoctmp
}
%    \end{macrocode}

% \macro{\childdocforwardprefix}
% The command |\childdocforwardprefix| redirects
% compilation to the main or a child file by means of a pattern.
% The prefix |#1| in the current filename is replaced by |#2|
% and the suffix of the current filename is kept
% (it is assumed that the filename does not contain the substring `|~~~|'
% which is used as a delimiter).
% Compilation is handed over to the new file by |\childdocforward|:
%    \begin{macrocode}
\newcommand{\childdocforwardprefix}[3][]
{
  \begingroup
    \def\childdocextract #2##1~~~{\def\childdoctmp{\childdocforward[#1]{#3##1}}}
    \expandafter\childdocextract\childdocname~~~
    \expandafter
  \endgroup
  \childdoctmp
}
%    \end{macrocode}

% \macro{\childdoc}
% The deprecated macro |\childdoc| is a legacy version of |\childdocmain|:
%    \begin{macrocode}
\newcommand{\childdoc}{\childdocmain}
%    \end{macrocode}

% \macro{\childdocredirect}
% The deprecated macro |\childdocredirect| is a legacy version
% of |\childdocforward| and |\childdocforwardprefix|:
%    \begin{macrocode}
\newcommand{\childdocredirect}[2][]
{
  \begingroup
    \if?#1?
      \def\childdoctmp{\childdocforward{#2}}
    \else
      \def\childdoctmp{\childdocforwardprefix{#1}{#2}}
    \fi
    \expandafter
  \endgroup
  \childdoctmp
}
%    \end{macrocode}

%\iffalse
%</package>
%\fi
%
\endinput
\childdocforward[|\textit{main}|]{|\textit{dest}|}"|
\end{center}
%
Here \textit{target} is the name of the output file,
\textit{main} is the name of the main file
and \textit{dest} is the name of the main or child file to be processed
(all filenames without extensions).
The optional argument \textit{main} can be omitted
if \textit{main} matches \textit{dest}.
Optionally, compilation \textit{flags} can be defined via |\def| commands.
This command line makes the \TeX{} engine believe
it is compiling the file \textit{target}
whose content is specified as the latter parameter.
The provided code then forwards the processing to
\textit{main} or \textit{dest} as described in \secref{sec:forward}.

%%%%%%%%%%%%%%%%%%%%%%%%%%%%%%%%%%%%%%%%%%%%%%%%%%%%%%%%%%%%%%%%%%%%%%%%%%%%%%%%
\subsection{Include by Input}
\label{sec:input}

Including child documents by |\include| has some restrictions by design.
Most notably, the content of a child document always occupies
its own set of pages; pages cannot be shared between child documents.
Usually, this behaviour makes perfect sense
because each child document contain an essential part of the document.
However, in some situations it may be desirable to compose
a document from a collection of parts
without having mandatory page breaks between then.
For this case, the package
provides a mechanism to include parts
by |\input| which can also be processed individually.
However, by construction this mechanism
requires manual handling of the content to be output.

%%%%%%%%%%%%%%%%%%%%%%%%%%%%%%%%%%%%%%%%
\DescribeMacro{\ifchilddocmanual}
The main file should be prepared as usual, see \secref{sec:include}.
However, the document body must make a distinction
between processing of an individual part and of the main document, e.g.:
%
\begin{center}
\begin{tabular}{l}
|\ifchilddocmanual|\\
|\input{\childdocname}|\\
|\||else|\\
\textit{document body with }|\input{|\textit{part}|}|\\
|\||fi|
\end{tabular}
\end{center}
%
The conditional |\ifchilddocmanual| is true whenever
a part to be included by |\input| is being compiled,
and the name of the part is stored in |\childdocname|.

%%%%%%%%%%%%%%%%%%%%%%%%%%%%%%%%%%%%%%%%
\DescribeMacro{\childdocby}
Each part to be included by |\input| should start with:
%
\begin{center}
\begin{tabular}{l}
|% \iffalse
%
% childdoc.dtx Copyright (C) 2017-2018 Niklas Beisert
%
% This work may be distributed and/or modified under the
% conditions of the LaTeX Project Public License, either version 1.3
% of this license or (at your option) any later version.
% The latest version of this license is in
%   http://www.latex-project.org/lppl.txt
% and version 1.3 or later is part of all distributions of LaTeX
% version 2005/12/01 or later.
%
% This work has the LPPL maintenance status `maintained'.
%
% The Current Maintainer of this work is Niklas Beisert.
%
% This work consists of the files childdoc.dtx and childdoc.ins
% and the derived files childdoc.def and cdocsamp.tex with
% cdocsch1.tex, cdocsch2.tex, cdocsdrf.tex, cdocsfn1.tex, cdocsfn2.tex.
%
%<package>\ifdefined\childdocmain\endinput\fi
%<package>\ProvidesFile{childdoc.def}[2018/12/30 v2.0 child document driver]
%<samplemain>\ProvidesFile{cdocsamp.tex}[2018/12/30 v2.0 sample for childdoc]
%<*driver>
%\ProvidesFile{childdoc.drv}[2018/12/30 v2.0 childdoc reference manual file]
\PassOptionsToClass{10pt,a4paper}{article}
\documentclass{ltxdoc}

\usepackage[margin=35mm]{geometry}
\usepackage{hyperref}
\usepackage{hyperxmp}
\usepackage[usenames]{color}

\hypersetup{colorlinks=true}
\hypersetup{pdfstartview=FitH}
\hypersetup{pdfpagemode=UseNone}
\hypersetup{pdfsource={}}
\hypersetup{pdflang={en-UK}}
\hypersetup{pdfcopyright={Copyright 2017-2018 Niklas Beisert.
  This work may be distributed and/or modified under the
  conditions of the LaTeX Project Public License, either version 1.3
  of this license or (at your option) any later version.}}
\hypersetup{pdflicenseurl={http://www.latex-project.org/lppl.txt}}
\hypersetup{pdfcontactaddress={ETH Zurich, ITP, HIT K,
  Wolfgang-Pauli-Strasse 27}}
\hypersetup{pdfcontactpostcode={8093}}
\hypersetup{pdfcontactcity={Zurich}}
\hypersetup{pdfcontactcountry={Switzerland}}
\hypersetup{pdfcontactemail={nbeisert@itp.phys.ethz.ch}}
\hypersetup{pdfcontacturl={http://people.phys.ethz.ch/\xmptilde nbeisert/}}

\newcommand{\secref}[1]{\hyperref[#1]{section \ref*{#1}}}

\parskip1ex
\parindent0pt
\let\olditemize\itemize
\def\itemize{\olditemize\parskip0pt}

\begin{document}

\title{The \textsf{childdoc} Package}
\hypersetup{pdftitle={The childdoc Package}}
\author{Niklas Beisert\\[2ex]
  Institut f\"ur Theoretische Physik\\
  Eidgen\"ossische Technische Hochschule Z\"urich\\
  Wolfgang-Pauli-Strasse 27, 8093 Z\"urich, Switzerland\\[1ex]
  \href{mailto:nbeisert@itp.phys.ethz.ch}
  {\texttt{nbeisert@itp.phys.ethz.ch}}}
\hypersetup{pdfauthor={Niklas Beisert}}
\hypersetup{pdfsubject={Manual for the LaTeX2e Package childdoc}}
\date{30 December 2018, \textsf{v2.0}}
\maketitle

\begin{abstract}\noindent
\textsf{childdoc} is a \LaTeXe{} package
that enables the direct compilation
of document sections included by |\include|
to individual files.
\end{abstract}

\begingroup
\parskip0ex
\tableofcontents
\endgroup

%%%%%%%%%%%%%%%%%%%%%%%%%%%%%%%%%%%%%%%%%%%%%%%%%%%%%%%%%%%%%%%%%%%%%%%%%%%%%%%%
%%%%%%%%%%%%%%%%%%%%%%%%%%%%%%%%%%%%%%%%%%%%%%%%%%%%%%%%%%%%%%%%%%%%%%%%%%%%%%%%
\section{Introduction}

\LaTeX{} provides a mechanism to structure a large document (such as a book)
into a main file and several child files (containing the chapters)
using the |\include| command.
This mechanism is beneficial for documents
which span hundreds of pages in order to
make the source file(s) more manageable.
Moreover, compilation can be restricted to
selected child files by means of the |\includeonly| command.
The latter feature can be used to reduce the compilation time while editing
(this was significantly more useful in the earlier days of \LaTeX{})
or to generate a smaller document which is easier to navigate.
Another application of |\includeonly| is to generate
documents consisting of selected parts of the complete document.

However, there are a few drawbacks of the plain |\include| mechanism:
\begin{itemize}
\item
The child files cannot be compiled on their own,
they can only be compiled via the main file.
A naive editing environment
(such as a text editor with an option
to have the current file processed by \LaTeX)
may require one to switch to the main file before compiling;
attempting to compile the child file produces errors.
\item
The main file must be modified (each time)
to adjust the |\includeonly| command
to the present needs. This easily leaves the main file in a messy state.
\item
The generated document will always carry the filename
of the main document. This is inconvenient if
several child files are to be compiled and
to be kept for distribution.
\end{itemize}

The present package provides a simple interface
to make child files individually compilable by \LaTeX{}.
Compiling a child file then has the same effect as compiling
the main file with an |\includeonly| command
to select the appropriate child.
Moreover the generated document will carry the name of the child
rather than the main file.
This resolves all three above issues.

This feature is meant to make the editing of books,
thesis documents and lecture notes somewhat more convenient.
However, the package can also be used efficiently for
composing a series of documents (such as exercise sheets)
which are typically distributed individually.
It then assists the author in generating the individual documents
(potentially in different versions)
as well as a document containing the collected series.
Another application is in developing style files
or other kinds of included material
where compilation of the style file could redirect
to a sample or test file.

%%%%%%%%%%%%%%%%%%%%%%%%%%%%%%%%%%%%%%%%%%%%%%%%%%%%%%%%%%%%%%%%%%%%%%%%%%%%%%%%
%%%%%%%%%%%%%%%%%%%%%%%%%%%%%%%%%%%%%%%%%%%%%%%%%%%%%%%%%%%%%%%%%%%%%%%%%%%%%%%%
\section{Usage}

First of all, the package \textsf{childdoc} is \emph{not} a standard
\LaTeXe{} |.sty| style file! Therefore it needs to be invoked in
a non-standard way.

%%%%%%%%%%%%%%%%%%%%%%%%%%%%%%%%%%%%%%%%%%%%%%%%%%%%%%%%%%%%%%%%%%%%%%%%%%%%%%%%
\subsection{Included Files}
\label{sec:include}

%%%%%%%%%%%%%%%%%%%%%%%%%%%%%%%%%%%%%%%%
\DescribeMacro{\childdocmain}
To use the package, add the commands
\begin{center}
\begin{tabular}{l}
|\input{childdoc.def}|\\
|\childdocmain{}|\\
\end{tabular}
\end{center}
at the very top of the main \LaTeX{} file,
in particular \emph{before} the |\documentclass| statement!
The argument of |\childdocmain| should be left empty
(but it must be present).

%%%%%%%%%%%%%%%%%%%%%%%%%%%%%%%%%%%%%%%%
\DescribeMacro{\childdocof}
Furthermore, add the commands
\begin{center}
\begin{tabular}{l}
|\input{childdoc.def}|\\
|\childdocof{|\textit{main}|}|\\
\end{tabular}
\end{center}
at the top of every child file \textit{child}
which is included by |\include{|\textit{child}|}|
from within the main file
(or at least for those files to be compiled individually).
The argument \textit{main} must be the filename of the main file.

There are a couple of
considerations in setting up the main and child documents:

%%%%%%%%%%%%%%%%%%%%%%%%%%%%%%%%%%%%%%%%
\paragraph{Restrictions.}

Please note the following restrictions:
\begin{itemize}
\item
|\childdocmain| must be called with one argument \textit{main}
to ensure compatibility with earlier version of the package.
It must either be empty (|\childdocmain{}|)
or precisely match the filename of the main file in which it is specified.
See \secref{sec:detection} for further information.
\item
The filename \textit{main} must be specified without the |.tex| extension.
\item
The filename \textit{main} is case sensitive
(even in case-insensitive file systems)
due to internal string comparison.
\item
The argument \textit{main} should be fully expanded, it cannot be a macro.
\item
Subdirectories and special characters should be avoided in filenames.
\item
The command |\childdocmain{|\textit{main}|}| must be followed by a whitespace.
It should not be followed immediately by another command
or by a comment mark `|%|'.
This is because the \TeX{} parser reads the token immediately following
the argument of |\childdocmain| and puts it
at the beginning of every child section;
however, a white\-space is ignored.
\end{itemize}

%%%%%%%%%%%%%%%%%%%%%%%%%%%%%%%%%%%%%%%%
\paragraph{Content of Main File.}

It is advisable to place all content in the child files included by |\include|.
Any output contained in the main file will appear in all child documents
unless suppressed manually;
it cannot be suppressed automatically by the |\includeonly| directive
and thus should normally be avoided.
A method to include some content in the main file
by means of conditional processing is described in \secref{sec:conditional}.

%%%%%%%%%%%%%%%%%%%%%%%%%%%%%%%%%%%%%%%%
\paragraph{Page Numbering.}

When only a part of the document is compiled,
the appropriate numbering of pages
(as well as other status parameters)
is determined from the |.aux| files.
The latter contain information from previous passes.
However this information needs to propagate through
all intermediate child documents.
Therefore the page numbering in child documents may well
be inconsistent until the complete document is compiled at least once.

A useful (if unconventional) way to always ensure a consistent
page numbering is to restart the numbering in each child document
and denote the pages by `\textit{child}|.|\textit{page}'
where \textit{child} represents the chapter/section number of the child file.
This can be achieved by the command
|\numberwithin{page}{|\textit{child}|}|
of the \textsf{amsmath} package
where \textit{child} can be |chapter| or |section|
depending on the chosen structuring.
Alternatively, one can modify the macro |\thepage| appropriately
and reset the counter |page| at the start of each child file.

%%%%%%%%%%%%%%%%%%%%%%%%%%%%%%%%%%%%%%%%%%%%%%%%%%%%%%%%%%%%%%%%%%%%%%%%%%%%%%%%
\subsection{Conditional Processing}
\label{sec:conditional}

The package provides a mechanism to compile different versions
of a document. To customise the versions further some conditional processing
can come in handy to distinguish which version is being compiled.
The package provides two macros to describe the compilation context:

%%%%%%%%%%%%%%%%%%%%%%%%%%%%%%%%%%%%%%%%
\DescribeMacro{\ifchilddoc}
The conditional |\ifchilddoc| distinguishes between the compilation of
child documents and the main document:
%
\begin{center}
|\ifchilddoc |\textit{child-code}| |[|\||else |\textit{main-code}]| \||fi|
\end{center}

%%%%%%%%%%%%%%%%%%%%%%%%%%%%%%%%%%%%%%%%
\DescribeMacro{\childdocname}
\DescribeMacro{\childdocjob}
The macro |\childdocname| contains the filename (without extension)
of the main or child file being processed.
Note that |\childdocjob| will always contain the name of the main file.

%%%%%%%%%%%%%%%%%%%%%%%%%%%%%%%%%%%%%%%%
\paragraph{Title Page.}

Conditional processing can be used to include a title or banner page
in the main document when proper precautions are taken.
Importantly, the code in the main file should ensure that the page counter
(as well as other status parameters which are stored in the |.aux| files)
takes the same value after the conditional processing.
Otherwise the page numbers may take divergent values
depending on which part is compiled.

For example, a title page could be declared by:
%
\begin{center}
\begin{tabular}{l}
|\ifchilddoc\||else|\\
|\addtocounter{page}{-1}|\\
\textit{code for title page}\\
|\newpage|\\
|\||fi|
\end{tabular}
\end{center}
%
A banner page for the child documents can be generated by:
%
\begin{center}
\begin{tabular}{l}
|\ifchilddoc|\\
|\addtocounter{page}{-1}|\\
\textit{code for banner page}\\
|\newpage|\\
|\||fi|
\end{tabular}
\end{center}
%
Here one could write a message such as:
\begin{center}
|This is the part \childdocname{} of \childdocjob{}.|
\end{center}

%%%%%%%%%%%%%%%%%%%%%%%%%%%%%%%%%%%%%%%%%%%%%%%%%%%%%%%%%%%%%%%%%%%%%%%%%%%%%%%%
\subsection{Flags}
\label{sec:flags}

The package makes it easy to generate different versions
of the main or child documents.
To this end compilation flags can be defined
and assigned different default values.
They will be particularly useful in conjunction
with the forwarding mechanism described in \secref{sec:forward}.

For example, it may be useful to have a flag |\version|
which can be set to |draft| or |final|.
The document source will contain some conditional code
depending on the value of |\version|.
Suppose further, the flag should default to |final| for the main file
and to |draft| for child files
which is a natural assignment for editing the document.
This is achieved by placing the following code
in the preamble of the main document
(below the |\childdocmain| directive):
%
\begin{center}
\begin{tabular}{l}
|\ifchilddoc|\\
|\providecommand{\version}{draft}|\\
|\||else|\\
|\providecommand{\version}{final}|\\
|\||fi|
\end{tabular}
\end{center}
%
The definition by |\providecommand| makes sure
that previous definitions are not overwritten.
Further statements |\providecommand{\version}{...}|
can thus be added before the above code to override it.

For the main file, one might add a line
(between |\childdocmain| and the above block)
%
\begin{center}
|%\ifchilddoc\||else\providecommand{\version}{draft}\||fi|
\end{center}
%
which can be uncommented to produce a draft version.
Likewise one can add a line to the very top of a child file
(above the |\childdocof{|\textit{main}|}| directive)
%
\begin{center}
|%\providecommand{\version}{final}|
\end{center}
%
which can be uncommented to produce the final version of this child document.

%%%%%%%%%%%%%%%%%%%%%%%%%%%%%%%%%%%%%%%%%%%%%%%%%%%%%%%%%%%%%%%%%%%%%%%%%%%%%%%%
\subsection{Forwarding}
\label{sec:forward}

Different versions of the main or child documents
using compilation flags as described in \secref{sec:flags}
can be (permanently) stored in different files
for convenient compilation, viewing and distribution.
To this end, the package defines a command
to pass on compilation to a different file:

%%%%%%%%%%%%%%%%%%%%%%%%%%%%%%%%%%%%%%%%
\DescribeMacro{\childdocforward}
The command |\childdocforward| redirects processing to
another source file:
%
\begin{center}
\begin{tabular}{l}
|\input{childdoc.def}|\\
|\childdocforward[|\textit{main}|]{|\textit{dest}|}|\\
\end{tabular}
\end{center}
%
The argument \textit{dest} is the destination file
(without extension).
It should be the main file or one of the child files.
Note that further \textsf{childdoc} directives
such as |\childdocof| and |\childdocforward|
in the indicated file will be processed in this form.
The optional argument \textit{main}
passes on directly to the main file \textit{main}
while pretending to compile the child \textit{dest}.
This form behaves as if \textit{dest}
issues |\childdocof{|\textit{main}|}| right away,
and no further \textsf{childdoc} directives will be processed.

%%%%%%%%%%%%%%%%%%%%%%%%%%%%%%%%%%%%%%%%
\DescribeMacro{\...prefix}
In the alternative form |\childdocforwardprefix|,
%
\begin{center}
\begin{tabular}{l}
|\input{childdoc.def}|\\
|\childdocforwardprefix[|\textit{main}|]{|\textit{prefix}|}{|\textit{dest}|}|
\end{tabular}
\end{center}
%
the destination file is determined by a pattern
depending on the current file:
To make this work, the current file must be called
`{\textit{prefix}\hspace{0.2em}\textit{suffix}}'
with \textit{prefix} matching precisely the argument.
Processing is then passed on to the file
`{\textit{dest}\hspace{0.2em}\textit{suffix}}'.
Surely, the same effect is achieved by
directly specifying the
argument `{\textit{dest}\hspace{0.2em}\textit{suffix}}'
in the first form.
However, that requires to set up a different file
for each child. With the alternative form of the command
all these files can have exactly the same content
which simplifies setting them up and maintaining them.

For example, the following file |draft.tex|
with a compilation flag |\version| as described in \secref{sec:flags}
compiles the main document as a draft:
%
\begin{center}
\begin{tabular}{l}
|\def\version{draft}|\\
|\input{childdoc.def}|\\
|\childdocforward{|\textit{main}|}|
\end{tabular}
\end{center}
%
Likewise, the following files |final|\textit{nn}|.tex|
compile the final version of the child document
|child|\textit{nn}|.tex|:
%
\begin{center}
\begin{tabular}{l}
|\def\version{final}|\\
|\input{childdoc.def}|\\
|\childdocforwardprefix{final}{child}|
\end{tabular}
\end{center}
%

Note that when several versions of a main file and/or of each child file
are to be generated, it may be convenient to set up a |Makefile| or
shell script to automatise the process.

%%%%%%%%%%%%%%%%%%%%%%%%%%%%%%%%%%%%%%%%%%%%%%%%%%%%%%%%%%%%%%%%%%%%%%%%%%%%%%%%
\subsection{Command Line Processing}
\label{sec:commandline}

The effect of redirection files can also be achieved by invoking
the \LaTeX{} compiler with a more elaborate command line.
Most conveniently this should be done as part
of a shell script or a |Makefile|.

When using \textsf{childdoc} in the main file, the following
command lines effectively perform a redirection
(note that depending on the shell being used,
backslashes may have to be doubled: `|\|' $\to$ `|\\|'):
%
\begin{center}
|... -jobname "|\textit{target}|" |\\|"|[\textit{flags}]%
|\input{childdoc.def}\childdocforward[|\textit{main}|]{|\textit{dest}|}"|
\end{center}
%
Here \textit{target} is the name of the output file,
\textit{main} is the name of the main file
and \textit{dest} is the name of the main or child file to be processed
(all filenames without extensions).
The optional argument \textit{main} can be omitted
if \textit{main} matches \textit{dest}.
Optionally, compilation \textit{flags} can be defined via |\def| commands.
This command line makes the \TeX{} engine believe
it is compiling the file \textit{target}
whose content is specified as the latter parameter.
The provided code then forwards the processing to
\textit{main} or \textit{dest} as described in \secref{sec:forward}.

%%%%%%%%%%%%%%%%%%%%%%%%%%%%%%%%%%%%%%%%%%%%%%%%%%%%%%%%%%%%%%%%%%%%%%%%%%%%%%%%
\subsection{Include by Input}
\label{sec:input}

Including child documents by |\include| has some restrictions by design.
Most notably, the content of a child document always occupies
its own set of pages; pages cannot be shared between child documents.
Usually, this behaviour makes perfect sense
because each child document contain an essential part of the document.
However, in some situations it may be desirable to compose
a document from a collection of parts
without having mandatory page breaks between then.
For this case, the package
provides a mechanism to include parts
by |\input| which can also be processed individually.
However, by construction this mechanism
requires manual handling of the content to be output.

%%%%%%%%%%%%%%%%%%%%%%%%%%%%%%%%%%%%%%%%
\DescribeMacro{\ifchilddocmanual}
The main file should be prepared as usual, see \secref{sec:include}.
However, the document body must make a distinction
between processing of an individual part and of the main document, e.g.:
%
\begin{center}
\begin{tabular}{l}
|\ifchilddocmanual|\\
|\input{\childdocname}|\\
|\||else|\\
\textit{document body with }|\input{|\textit{part}|}|\\
|\||fi|
\end{tabular}
\end{center}
%
The conditional |\ifchilddocmanual| is true whenever
a part to be included by |\input| is being compiled,
and the name of the part is stored in |\childdocname|.

%%%%%%%%%%%%%%%%%%%%%%%%%%%%%%%%%%%%%%%%
\DescribeMacro{\childdocby}
Each part to be included by |\input| should start with:
%
\begin{center}
\begin{tabular}{l}
|\input{childdoc.def}|\\
|\childdocby{|\textit{main}|}|\\
\end{tabular}
\end{center}
%
The directive |\childdocby| is similar to |\childdocof|
described in \secref{sec:include},
but the subsequent selection of content must be done manually.
To that end, both |\ifchilddoc| and |\ifchilddocmanual|
will be true upon processing of a part,
and the name of the part is stored in |\childdocname|.
Note that |\jobname| will be set to the filename of the current part
so that each part receives an individual |.aux| file
that does not interfere with the |.aux| file(s) of the main document.
This behaviour can be altered by the alternative form
|\childdocby[*]{|\textit{main}|}| (with a non-empty optional argument)
which uses the |.aux| file of the main document
by setting |\jobname| to \textit{main}.

%%%%%%%%%%%%%%%%%%%%%%%%%%%%%%%%%%%%%%%%%%%%%%%%%%%%%%%%%%%%%%%%%%%%%%%%%%%%%%%%
\subsection{Driver Development}
\label{sec:driver}

The \textsf{childdoc} mechanism can also be use for the development
of definition files such as \LaTeX{} styles or classes.
This case differs from the above setup with multiple parts
included by |\include| in that no |\includeonly| should be invoked.
This can be achieved by starting the include file
(before |\ProvidesPackage|) with:
%
\begin{center}
\begin{tabular}{l}
|\input{childdoc.def}|\\
|\childdocforward{|\textit{main}|}|\\
\end{tabular}
\end{center}
%
or alternatively with:
%
\begin{center}
\begin{tabular}{l}
|\input{childdoc.def}|\\
|\childdocby{|\textit{main}|}|\\
\end{tabular}
\end{center}
%
Both forms have slightly different effects as described above.
The main file is prepared as usual, see \secref{sec:include}.

%%%%%%%%%%%%%%%%%%%%%%%%%%%%%%%%%%%%%%%%%%%%%%%%%%%%%%%%%%%%%%%%%%%%%%%%%%%%%%%%
\subsection{Legacy Detection}
\label{sec:detection}

The directive |\childdocmain| in the main file can detect
whether the complete document or merely a child is to be compiled
even without using the directive |\childdocof|.
This method is deprecated because it is less robust
and there is no compelling reason to use it;
it is merely provided for backward compatibility
and it may be removed in future versions.

If the detection mechanism is to be used,
it is mandatory to correctly specify
the filename of the main file as the argument of |\childdocmain|:
%
\begin{center}
\begin{tabular}{l}
|\input{childdoc.def}|\\
|\childdocmain{|\textit{main}|}|\\
\end{tabular}
\end{center}
%
If |\jobname| does not match the argument \textit{main} of |\childdocmain|,
it is assumed that |\jobname| points to the child file to be compiled.
When using |\childdocmain| with the main file specified as argument,
it suffices to start a child file
with just |\input{|\textit{main}|}|
without loading of the package and using |\childdocof|.
If instead all processing is done
with the appropriate \textsf{childdoc} directives,
the argument of \textit{main} of |\childdocmain| can be empty.

An alternative version of the command line processing described
in \secref{sec:commandline} using the detection mechanism reads:
%
\begin{center}
|... -jobname "|\textit{target}|" "|[\textit{flags}]%
[|\def\jobname{|\textit{dest}|}|]|\input{|\textit{main}|}"|
\end{center}

%%%%%%%%%%%%%%%%%%%%%%%%%%%%%%%%%%%%%%%%%%%%%%%%%%%%%%%%%%%%%%%%%%%%%%%%%%%%%%%%
\subsection{Manual Code}
\label{sec:manual}

In case one cannot be certain whether the definitions file |childdoc.def|
is installed on the target \TeX{} distribution
and one prefers not to ship it,
it is conceivable to paste a few relevant commands into the sources.

To that end, drop all statements |\input{childdoc.def}|
and perform the replacements as outlined below.
Instead of |\childdocmain{|\textit{main}|}| add the following code
to the top of the main file:
%
\begin{center}
\begin{tabular}{l}
|\||ifdefined\childdocname\endinput\||fi\newif\ifchilddoc|\\
|\edef\childdocname{\scantokens\expandafter{\jobname\noexpand}}|\\
|\def\childdocmain{|\textit{main}|}\||ifx\childdocmain\childdocname\||else|\\
|\childdoctrue\includeonly{\childdocname}\let\jobname\childdocmain\||fi|\\
\end{tabular}
\end{center}
%
Instead of |\childdocof{|\textit{main}|}| just include the main file
at the top of each child file:
%
\begin{center}
|\input{|\textit{main}|}|
\end{center}
%
A simple redirection |\childdocforward{|\textit{dest}|}| is achieved by:
%
\begin{center}
|\def\jobname{|\textit{dest}|}\input{\jobname}|
\end{center}
%
The redirection with prefix
|\childdocforwardprefix[|\textit{prefix}|]{|\textit{dest}|}|
is accomplished by:
%
\begin{center}
\begin{tabular}{l}
|{\edef\jobname{\scantokens\expandafter{\jobname\noexpand}}|\\
|\def\redirectjob |\textit{prefix}|#1~~~{\gdef\jobname{|\textit{dest}|#1}}|\\
|\expandafter\redirectjob\jobname~~~}\input{\jobname}|
\end{tabular}
\end{center}

In an alternative approach,
child documents can be compiled by a specific command line
without additional code or specific definitions:
%
\begin{center}
|... -jobname "|\textit{target}|" "|[\textit{flags}]%
|\includeonly{|\textit{dest}|}\input{|\textit{main}|}"|
\end{center}
%

%%%%%%%%%%%%%%%%%%%%%%%%%%%%%%%%%%%%%%%%%%%%%%%%%%%%%%%%%%%%%%%%%%%%%%%%%%%%%%%%
%%%%%%%%%%%%%%%%%%%%%%%%%%%%%%%%%%%%%%%%%%%%%%%%%%%%%%%%%%%%%%%%%%%%%%%%%%%%%%%%
\section{Information}

%%%%%%%%%%%%%%%%%%%%%%%%%%%%%%%%%%%%%%%%%%%%%%%%%%%%%%%%%%%%%%%%%%%%%%%%%%%%%%%%
\subsection{Copyright}

Copyright \copyright{} 2017--2018 Niklas Beisert

This work may be distributed and/or modified under the
conditions of the \LaTeX{} Project Public License, either version 1.3
of this license or (at your option) any later version.
The latest version of this license is in
  \url{http://www.latex-project.org/lppl.txt}
and version 1.3 or later is part of all distributions of \LaTeX{}
version 2005/12/01 or later.

This work has the LPPL maintenance status `maintained'.

The Current Maintainer of this work is Niklas Beisert.

This work consists of the files |README.txt|, |childdoc.ins| and |childdoc.dtx|
as well as the derived files |childdoc.def|, |cdocsamp.tex|
with |cdocsch1.tex|, |cdocsch2.tex|, |cdocspt3.tex|, |cdocspt4.tex|,
|cdocsdrf.tex|, |cdocsfn1.tex|, |cdocsfn2.tex|
as well as |childdoc.pdf|.

%%%%%%%%%%%%%%%%%%%%%%%%%%%%%%%%%%%%%%%%%%%%%%%%%%%%%%%%%%%%%%%%%%%%%%%%%%%%%%%%
\subsection{Files and Installation}

The package consists of the files:
%
\begin{center}
\begin{tabular}{ll}
    |README.txt|   & readme file \\
    |childdoc.ins| & installation file \\
    |childdoc.dtx| & source file \\
    |childdoc.def| & definition file \\
    |cdocsamp.tex| & sample main file \\
    |cdocsch1.tex| & sample include file \\
    |cdocsch2.tex| & sample include file \\
    |cdocspt3.tex| & sample part file \\
    |cdocspt4.tex| & sample part file \\
    |cdocsdrf.tex| & sample redirection file \\
    |cdocsfn1.tex| & sample redirection file \\
    |cdocsfn2.tex| & sample redirection file \\
    |childdoc.pdf| & manual
\end{tabular}
\end{center}
%
The distribution consists of the files
|README.txt|, |childdoc.ins| and |childdoc.dtx|.
%
\begin{itemize}
\item
Run (pdf)\LaTeX{} on |childdoc.dtx|
to compile the manual |childdoc.pdf| (this file).
\item
Run \LaTeX{} on |childdoc.ins| to create the definitions file |childdoc.def|
and the sample |cdocsamp.tex| with include files
|cdocsch1.tex|, |cdocsch2.tex|, |cdocspt3.tex|, |cdocspt4.tex|,
|cdocsdrf.tex|, |cdocsfn1.tex|, |cdocsfn2.tex|.
Then copy the file |childdoc.def| to an appropriate directory of your \LaTeX{}
distribution, e.g.\ \textit{texmf-root}|/tex/latex/childdoc|.
\end{itemize}

%%%%%%%%%%%%%%%%%%%%%%%%%%%%%%%%%%%%%%%%%%%%%%%%%%%%%%%%%%%%%%%%%%%%%%%%%%%%%%%%
\subsection{Related CTAN Packages}

There are several other packages which offer a similar functionality:
%
\begin{itemize}
\item
The packages
\href{http://ctan.org/pkg/docmute}{\textsf{docmute}},
\href{http://ctan.org/pkg/includex}{\textsf{includex}} and
\href{http://ctan.org/pkg/standalone}{\textsf{standalone}}
provide commands to include only the document body of
a child file thus allowing both files to be compiled individually.
\item
The packages \href{http://ctan.org/pkg/subdocs}{\textsf{subdocs}}
and \href{http://ctan.org/pkg/subfiles}{\textsf{subfiles}}
provide structures in which the main and child documents can be
encapsulated and allowing them to be compiled individually.
The inclusion mechanism is different from the conventional |\include|.
\item
The package \href{http://ctan.org/pkg/combine}{\textsf{combine}}
is an elaborate solution to combine several documents into one.
\end{itemize}
%
See also the CTAN topic \href{http://ctan.org/topic/subdocs}{\textsf{subdocs}}
for further related packages.
The present package differs from the above solutions in that
a document structure constructed with the conventional |\include| mechanism
just needs two extra commands at the top of every file
such that all constituent files can be compiled individually.

%%%%%%%%%%%%%%%%%%%%%%%%%%%%%%%%%%%%%%%%%%%%%%%%%%%%%%%%%%%%%%%%%%%%%%%%%%%%%%%%
%\subsection{Feature Suggestions}
%
%The following is a list of features which may be useful for future
%versions of this package:
%%
%\begin{itemize}
%\item
%\ldots
%\end{itemize}

%%%%%%%%%%%%%%%%%%%%%%%%%%%%%%%%%%%%%%%%%%%%%%%%%%%%%%%%%%%%%%%%%%%%%%%%%%%%%%%%
\subsection{Revision History}

%%%%%%%%%%%%%%%%%%%%%%%%%%%%%%%%%%%%%%%%
\paragraph{v2.0:} 2018/12/30

\begin{itemize}
\item
immediate forward processing
\item
added |\childdocby| mechanism
\item
manual restructured
\end{itemize}

%%%%%%%%%%%%%%%%%%%%%%%%%%%%%%%%%%%%%%%%
\paragraph{v1.6:} 2018/01/17

\begin{itemize}
\item
application for development of include files
\item
corrections to manual
\end{itemize}

%%%%%%%%%%%%%%%%%%%%%%%%%%%%%%%%%%%%%%%%
\paragraph{v1.5:} 2017/05/21

\begin{itemize}
\item
more complete structuring introduced
\item
|\childdocof| introduced
\item
|\childdoc| renamed to |\childdocmain|
\item
|\childredirect| renamed to |\childdocforward| and |\childdocforwardprefix|
and functionality expanded
\end{itemize}

%%%%%%%%%%%%%%%%%%%%%%%%%%%%%%%%%%%%%%%%
\paragraph{v1.0:} 2017/04/27

\begin{itemize}
\item
manual and install package
\item
first version published on CTAN
\end{itemize}

%%%%%%%%%%%%%%%%%%%%%%%%%%%%%%%%%%%%%%%%
\paragraph{v0.6:} 2017/04/26

\begin{itemize}
\item
redirection mechanism added
\end{itemize}

%%%%%%%%%%%%%%%%%%%%%%%%%%%%%%%%%%%%%%%%
\paragraph{v0.5:} 2017/04/26

\begin{itemize}
\item
functionality in definition file
\end{itemize}


%%%%%%%%%%%%%%%%%%%%%%%%%%%%%%%%%%%%%%%%%%%%%%%%%%%%%%%%%%%%%%%%%%%%%%%%%%%%%%%%
%%%%%%%%%%%%%%%%%%%%%%%%%%%%%%%%%%%%%%%%%%%%%%%%%%%%%%%%%%%%%%%%%%%%%%%%%%%%%%%%
%%%%%%%%%%%%%%%%%%%%%%%%%%%%%%%%%%%%%%%%%%%%%%%%%%%%%%%%%%%%%%%%%%%%%%%%%%%%%%%%
\appendix

\settowidth\MacroIndent{\rmfamily\scriptsize 000\ }

 \DocInput{childdoc.dtx}

\end{document}
%</driver>
% \fi
%
% %%%%%%%%%%%%%%%%%%%%%%%%%%%%%%%%%%%%%%%%%%%%%%%%%%%%%%%%%%%%%%%%%%%%%%%%%%%%%%
% %%%%%%%%%%%%%%%%%%%%%%%%%%%%%%%%%%%%%%%%%%%%%%%%%%%%%%%%%%%%%%%%%%%%%%%%%%%%%%
% \section{Sample}
%\iffalse
%<*samplemain>
%\fi
%
% The following presents a sample document
% with two chapters, two parts, a title page,
% a compile flag as well as three forwarding files to set the flag.
% It consists of eight |.tex| files:
% \begin{center}
% \begin{tabular}{ll}
% |cdocsamp.tex|&main file\\
% |cdocsch1.tex|&include file for chapter 1\\
% |cdocsch2.tex|&include file for chapter 2\\
% |cdocspt3.tex|&include file for part 3\\
% |cdocspt4.tex|&include file for part 4\\
% |cdocsdrf.tex|&forwarding file for main file in draft mode\\
% |cdocsfi1.tex|&forwarding file for final version of chapter 1\\
% |cdocsfi2.tex|&forwarding file for final version of chapter 2\\
% \end{tabular}
% \end{center}
% Each of the eight files can be compiled directly by the \LaTeX{} compiler.
%
% %%%%%%%%%%%%%%%%%%%%%%%%%%%%%%%%%%%%%%
% \paragraph{Main File.}
%
% The main file is called |cdocsamp.tex|.
%
% Load the \textsf{childdoc} definitions and
% declare the filename for the main document:
%    \begin{macrocode}
\input{childdoc.def}
\childdocmain{}
%    \end{macrocode}

% Optional override for |\version| flag:
%    \begin{macrocode}
%%\ifchilddoc\else\providecommand{\version}{draft}\fi
%    \end{macrocode}

% Define the default values for the |\version| flag
% (|final| for the main file and |draft| for childs):
%    \begin{macrocode}
\ifchilddoc
\providecommand{\version}{draft}
\else
\providecommand{\version}{final}
\fi
%    \end{macrocode}

% Load the standard document class:
%    \begin{macrocode}
\documentclass[12pt]{article}
%    \end{macrocode}

% Start the document body:
%    \begin{macrocode}
\begin{document}
%    \end{macrocode}

% Declare a title page.
% Print title, part of document being processed and version flag:
%    \begin{macrocode}
\addtocounter{page}{-1}
\begin{center}
{\LARGE\bfseries{}childdoc example\par}
\vspace{1cm}
\ifchilddoc
\ifchilddocmanual part\else chapter\fi:
`\childdocname' of `\childdocjob'\par
\else
main document: `\childdocjob'\par
\fi
version: \version\par
\end{center}
\newpage
%    \end{macrocode}

% Manually include selected file,
% otherwise process as usual:
%    \begin{macrocode}
\ifchilddocmanual
\section*{part `\childdocname'}
\input{\childdocname}
\else
%    \end{macrocode}

% Include the two chapters:
%    \begin{macrocode}
\include{cdocsch1}
\include{cdocsch2}
%    \end{macrocode}

% Include the two parts unless only chapters should be displayed:
%    \begin{macrocode}
\ifchilddoc\else
\section{part three}
\input{cdocspt3}
\section{part four}
\input{cdocspt4}
\fi
%    \end{macrocode}

% Process as usual until here:
%    \begin{macrocode}
\fi
%    \end{macrocode}

% End of document body:
%    \begin{macrocode}
\end{document}
%    \end{macrocode}
%\iffalse
%</samplemain>
%\fi
%
% %%%%%%%%%%%%%%%%%%%%%%%%%%%%%%%%%%%%%%
% \paragraph{Chapter Include Files.}
%
% The include files are called |cdocsch1.tex| and |cdocsch2.tex|.
%
%\iffalse
%<*samplechap1|samplechap2>
%\fi

% Optional override for |\version| flag:
%    \begin{macrocode}
%%\providecommand{\version}{final}
%    \end{macrocode}

% Include the main document:
%    \begin{macrocode}
\input{childdoc.def}
\childdocof{cdocsamp}
%    \end{macrocode}

%\iffalse
%</samplechap1|samplechap2>
%\fi
%
%\iffalse
%<*samplechap1>
%\fi
% Some text for chapter 1:
%    \begin{macrocode}
\section{one}
some text in chapter one
%    \end{macrocode}

%\iffalse
%</samplechap1>
%\fi
% Some text for chapter 2:
%\iffalse
%<*samplechap2>
%\fi
%    \begin{macrocode}
\section{two}
more text in chapter two
%    \end{macrocode}

%\iffalse
%</samplechap2>
%\fi
%
% %%%%%%%%%%%%%%%%%%%%%%%%%%%%%%%%%%%%%%
% \paragraph{Part Include Files.}
%
% The include files are called |cdocspt3.tex| and |cdocspt4.tex|.
%
%\iffalse
%<*samplepart3|samplepart4>
%\fi

% Optional override for |\version| flag:
%    \begin{macrocode}
%%\providecommand{\version}{final}
%    \end{macrocode}

% Include the main document:
%    \begin{macrocode}
\input{childdoc.def}
\childdocby{cdocsamp}
%    \end{macrocode}

%\iffalse
%</samplepart3|samplepart4>
%\fi
%
%\iffalse
%<*samplepart3>
%\fi
% Some text for part 3:
%    \begin{macrocode}
some text in part three
%    \end{macrocode}

%\iffalse
%</samplepart3>
%\fi
% Some text for part 4:
%\iffalse
%<*samplepart4>
%\fi
%    \begin{macrocode}
more text in part four
%    \end{macrocode}

%\iffalse
%</samplepart4>
%\fi
%
% %%%%%%%%%%%%%%%%%%%%%%%%%%%%%%%%%%%%%%
% \paragraph{Forwarding for a Complete Draft.}
%
% The following forwarding file |cdocsdrf.tex|
% compiles the main document in draft mode:
%\iffalse
%<*sampledraft>
%\fi
%    \begin{macrocode}
\def\version{draft}
\input{childdoc.def}
\childdocforward{cdocsamp}
%    \end{macrocode}

%\iffalse
%</sampledraft>
%\fi
%
% %%%%%%%%%%%%%%%%%%%%%%%%%%%%%%%%%%%%%%
% \paragraph{Forwarding for Final Version of the Chapters.}
%
% The following forwarding files |cdocsfn1.tex| and |cdocsfn2.tex|
% (with identical content)
% compile the final versions of the child documents
% |cdocsch1.tex| and |cdocsch2.tex|, respectively:
%\iffalse
%<*samplefinal>
%\fi
%    \begin{macrocode}
\def\version{final}
\input{childdoc.def}
\childdocforwardprefix[cdocsamp]{cdocsfn}{cdocsch}
%    \end{macrocode}

%\iffalse
%</samplefinal>
%\fi
%
% %%%%%%%%%%%%%%%%%%%%%%%%%%%%%%%%%%%%%%
% \paragraph{Command Line Processing.}
%
% The following three command lines generate the output files
% |cdocscld|, |cdocscl1| and |cdocscl2|
% which should be identical to
% |cdocsdrf|, |cdocsch1| and |cdocsfn2|, respectively:
% \begin{center}
% \begin{tabular}{l}
% |latex -jobname cdocscld \|\\
% |  "\def\version{draft}\input{childdoc.def}\childdocforward{cdocsamp}"|\\
% |latex -jobname cdocscl1 \|\\
% |  "\input{childdoc.def}\childdocforward[cdocsamp]{cdocsch1}"|\\
% |latex -jobname cdocscl2 \|\\
% |  "\def\version{final}\input{childdoc.def}\childdocforward{cdocsch2}"|
% \end{tabular}
% \end{center}
% Note that the trailing backslash on each first line
% merely continues the input to the second line
% (for convenient cut ant paste).
% Furthermore, the command |latex| can be replaced by any
% of its alternative versions such as |pdflatex|.
%
% %%%%%%%%%%%%%%%%%%%%%%%%%%%%%%%%%%%%%%%%%%%%%%%%%%%%%%%%%%%%%%%%%%%%%%%%%%%%%%
% %%%%%%%%%%%%%%%%%%%%%%%%%%%%%%%%%%%%%%%%%%%%%%%%%%%%%%%%%%%%%%%%%%%%%%%%%%%%%%
% \section{Implementation}
%\iffalse
%<*package>
%\fi
%
% This section describes the definitions file |childdoc.def|.

% The definitions cannot be loaded using |\usepackage| or |\RequirePackage|
% which has a mechanism to prevent loading a style file more than once.
% When loading the definitions by means of |\input|
% multiple instances have to be prevented manually:
%\iffalse
%This code needs to be before the `\ProvidesFile' directive
%which is defined at the beginning of this file.
%Therefore it is also placed there and commented out here.
%</package>
%<*discard>
%\fi
%    \begin{macrocode}
\ifdefined\childdocmain\endinput\fi
%    \end{macrocode}
%\iffalse
%</discard>
%<*package>
%\fi
%
% \macro{\ifchilddoc}
% \macro{\ifchilddocmanual}
% The conditional |\ifchilddoc| tells whether a
% child (true) or main (false) document is being compiled.
% The conditional |\ifchilddocmanual| tells whether
% the |\includeonly| mechanism is used (false) or
% the selection of child files must be performed manually (true).
% The definitions initialise to false:
%    \begin{macrocode}
\newif\ifchilddoc
\newif\ifchilddocmanual
%    \end{macrocode}

% \macro{\childdocname}
% \macro{\childdocjob}
% The macro |\childdocname| stores the name of the main document
% to be compiled. The macro |\childdocjob| stores the name of
% the document on which the \LaTeX{} compiler was originally invoked.
% The content of |\jobname| cannot be compared
% to filenames specified in the source due to different catcodes.
% The following code rescans |\jobname|, stores the result
% in |\childdocname| and saves a copy in |\childdocjob|:
%    \begin{macrocode}
\edef\childdocname{\scantokens\expandafter{\jobname\noexpand}}
\let\childdocjob\childdocname
%    \end{macrocode}

% \macro{\childdocdisable}
% The macro |\childdocdisable| prevents the main file
% from being processed more than once.
% At this stage, the main document command |\childdocmain|
% is assumed to be called once again where it should do nothing.
% Any subsequent call to it should prevent
% a secondary processing of the main document
% It overwrites the forwarding commands
% |\childdocof| and |\childdocforward|
% with empty macros to prevent further inclusions of the main document:
%    \begin{macrocode}
\newcommand{\childdocdisable}
{
  \renewcommand{\childdocmain}[1]{\renewcommand{\childdocmain}[1]{\endinput}}
  \renewcommand{\childdocof}[1]{}
  \renewcommand{\childdocby}[2][]{}
  \renewcommand{\childdocforward}[2][]{}
  \renewcommand{\childdocdisable}{}
}
%    \end{macrocode}

% \macro{\childdocmain}
% The macro |\childdocmain| is to be called at the top of the main file
% with nothing or the main filename (without extension) as argument.
% First, it breaks loops.
% If the argument is not empty and does not match |\childdocname|
% (which is set by the first inclusion of |childdoc.def|),
% |\ifchilddoc| is set to true, |\includeonly| is applied to the child file
% and |\jobname| is set to the main file
% (for proper handling of |.aux| files):
%    \begin{macrocode}
\newcommand{\childdocmain}[1]
{
  \childdocdisable\childdocmain{}
  \if?#1?\else
    \begingroup
      \def\childdoctmp{#1}
      \ifx\childdoctmp\childdocname
        \def\childdoctmp{}
      \else
        \def\childdoctmp
        {
          \childdoctrue
          \includeonly{\childdocname}
          \def\childdocjob{#1}
          \def\jobname{#1}
        }
      \fi
      \expandafter
    \endgroup
    \childdoctmp
  \fi
}
%    \end{macrocode}

% \macro{\childdocof}
% The command |\childdocof| redirects
% compilation to the main file |#1|.
%    \begin{macrocode}
\newcommand{\childdocof}[1]
{
  \childdocdisable
  \childdoctrue
  \includeonly{\childdocname}
  \def\jobname{#1}
  \def\childdocjob{#1}
  \input{#1}
}
%    \end{macrocode}

% \macro{\childdocby}
% The command |\childdocby| ....
%    \begin{macrocode}
\newcommand{\childdocby}[2][]
{
  \childdocdisable
  \childdoctrue
  \childdocmanualtrue
  \if?#1?\else
    \def\jobname{#2}
  \fi
  \def\childdocjob{#2}
  \input{#2}
  \endinput
}
%    \end{macrocode}

% \macro{\childdocforward}
% The command |\childdocforward| redirects
% compilation to the main file or
% (if the optional argument is given) a child file.
% Parameters are set as if the main file
% or a child file starting with |\childdocof| was compiled.
% Then compilation is handed over to the main file:
%    \begin{macrocode}
\newcommand{\childdocforward}[2][]
{
  \begingroup
    \if?#1?
      \def\childdoctmp
      {
        \def\childdocname{#2}
        \def\childdocjob{#2}
        \def\jobname{#2}
        \input{#2}
        \endinput
      }
    \else
      \def\childdoctmp
      {
        \childdocdisable
        \def\childdocname{#2}
        \childdoctrue
        \includeonly{#2}
        \def\childdocjob{#1}
        \def\jobname{#1}
        \input{#1}
        \endinput
      }
    \fi
    \expandafter
  \endgroup
  \childdoctmp
}
%    \end{macrocode}

% \macro{\childdocforwardprefix}
% The command |\childdocforwardprefix| redirects
% compilation to the main or a child file by means of a pattern.
% The prefix |#1| in the current filename is replaced by |#2|
% and the suffix of the current filename is kept
% (it is assumed that the filename does not contain the substring `|~~~|'
% which is used as a delimiter).
% Compilation is handed over to the new file by |\childdocforward|:
%    \begin{macrocode}
\newcommand{\childdocforwardprefix}[3][]
{
  \begingroup
    \def\childdocextract #2##1~~~{\def\childdoctmp{\childdocforward[#1]{#3##1}}}
    \expandafter\childdocextract\childdocname~~~
    \expandafter
  \endgroup
  \childdoctmp
}
%    \end{macrocode}

% \macro{\childdoc}
% The deprecated macro |\childdoc| is a legacy version of |\childdocmain|:
%    \begin{macrocode}
\newcommand{\childdoc}{\childdocmain}
%    \end{macrocode}

% \macro{\childdocredirect}
% The deprecated macro |\childdocredirect| is a legacy version
% of |\childdocforward| and |\childdocforwardprefix|:
%    \begin{macrocode}
\newcommand{\childdocredirect}[2][]
{
  \begingroup
    \if?#1?
      \def\childdoctmp{\childdocforward{#2}}
    \else
      \def\childdoctmp{\childdocforwardprefix{#1}{#2}}
    \fi
    \expandafter
  \endgroup
  \childdoctmp
}
%    \end{macrocode}

%\iffalse
%</package>
%\fi
%
\endinput
|\\
|\childdocby{|\textit{main}|}|\\
\end{tabular}
\end{center}
%
The directive |\childdocby| is similar to |\childdocof|
described in \secref{sec:include},
but the subsequent selection of content must be done manually.
To that end, both |\ifchilddoc| and |\ifchilddocmanual|
will be true upon processing of a part,
and the name of the part is stored in |\childdocname|.
Note that |\jobname| will be set to the filename of the current part
so that each part receives an individual |.aux| file
that does not interfere with the |.aux| file(s) of the main document.
This behaviour can be altered by the alternative form
|\childdocby[*]{|\textit{main}|}| (with a non-empty optional argument)
which uses the |.aux| file of the main document
by setting |\jobname| to \textit{main}.

%%%%%%%%%%%%%%%%%%%%%%%%%%%%%%%%%%%%%%%%%%%%%%%%%%%%%%%%%%%%%%%%%%%%%%%%%%%%%%%%
\subsection{Driver Development}
\label{sec:driver}

The \textsf{childdoc} mechanism can also be use for the development
of definition files such as \LaTeX{} styles or classes.
This case differs from the above setup with multiple parts
included by |\include| in that no |\includeonly| should be invoked.
This can be achieved by starting the include file
(before |\ProvidesPackage|) with:
%
\begin{center}
\begin{tabular}{l}
|% \iffalse
%
% childdoc.dtx Copyright (C) 2017-2018 Niklas Beisert
%
% This work may be distributed and/or modified under the
% conditions of the LaTeX Project Public License, either version 1.3
% of this license or (at your option) any later version.
% The latest version of this license is in
%   http://www.latex-project.org/lppl.txt
% and version 1.3 or later is part of all distributions of LaTeX
% version 2005/12/01 or later.
%
% This work has the LPPL maintenance status `maintained'.
%
% The Current Maintainer of this work is Niklas Beisert.
%
% This work consists of the files childdoc.dtx and childdoc.ins
% and the derived files childdoc.def and cdocsamp.tex with
% cdocsch1.tex, cdocsch2.tex, cdocsdrf.tex, cdocsfn1.tex, cdocsfn2.tex.
%
%<package>\ifdefined\childdocmain\endinput\fi
%<package>\ProvidesFile{childdoc.def}[2018/12/30 v2.0 child document driver]
%<samplemain>\ProvidesFile{cdocsamp.tex}[2018/12/30 v2.0 sample for childdoc]
%<*driver>
%\ProvidesFile{childdoc.drv}[2018/12/30 v2.0 childdoc reference manual file]
\PassOptionsToClass{10pt,a4paper}{article}
\documentclass{ltxdoc}

\usepackage[margin=35mm]{geometry}
\usepackage{hyperref}
\usepackage{hyperxmp}
\usepackage[usenames]{color}

\hypersetup{colorlinks=true}
\hypersetup{pdfstartview=FitH}
\hypersetup{pdfpagemode=UseNone}
\hypersetup{pdfsource={}}
\hypersetup{pdflang={en-UK}}
\hypersetup{pdfcopyright={Copyright 2017-2018 Niklas Beisert.
  This work may be distributed and/or modified under the
  conditions of the LaTeX Project Public License, either version 1.3
  of this license or (at your option) any later version.}}
\hypersetup{pdflicenseurl={http://www.latex-project.org/lppl.txt}}
\hypersetup{pdfcontactaddress={ETH Zurich, ITP, HIT K,
  Wolfgang-Pauli-Strasse 27}}
\hypersetup{pdfcontactpostcode={8093}}
\hypersetup{pdfcontactcity={Zurich}}
\hypersetup{pdfcontactcountry={Switzerland}}
\hypersetup{pdfcontactemail={nbeisert@itp.phys.ethz.ch}}
\hypersetup{pdfcontacturl={http://people.phys.ethz.ch/\xmptilde nbeisert/}}

\newcommand{\secref}[1]{\hyperref[#1]{section \ref*{#1}}}

\parskip1ex
\parindent0pt
\let\olditemize\itemize
\def\itemize{\olditemize\parskip0pt}

\begin{document}

\title{The \textsf{childdoc} Package}
\hypersetup{pdftitle={The childdoc Package}}
\author{Niklas Beisert\\[2ex]
  Institut f\"ur Theoretische Physik\\
  Eidgen\"ossische Technische Hochschule Z\"urich\\
  Wolfgang-Pauli-Strasse 27, 8093 Z\"urich, Switzerland\\[1ex]
  \href{mailto:nbeisert@itp.phys.ethz.ch}
  {\texttt{nbeisert@itp.phys.ethz.ch}}}
\hypersetup{pdfauthor={Niklas Beisert}}
\hypersetup{pdfsubject={Manual for the LaTeX2e Package childdoc}}
\date{30 December 2018, \textsf{v2.0}}
\maketitle

\begin{abstract}\noindent
\textsf{childdoc} is a \LaTeXe{} package
that enables the direct compilation
of document sections included by |\include|
to individual files.
\end{abstract}

\begingroup
\parskip0ex
\tableofcontents
\endgroup

%%%%%%%%%%%%%%%%%%%%%%%%%%%%%%%%%%%%%%%%%%%%%%%%%%%%%%%%%%%%%%%%%%%%%%%%%%%%%%%%
%%%%%%%%%%%%%%%%%%%%%%%%%%%%%%%%%%%%%%%%%%%%%%%%%%%%%%%%%%%%%%%%%%%%%%%%%%%%%%%%
\section{Introduction}

\LaTeX{} provides a mechanism to structure a large document (such as a book)
into a main file and several child files (containing the chapters)
using the |\include| command.
This mechanism is beneficial for documents
which span hundreds of pages in order to
make the source file(s) more manageable.
Moreover, compilation can be restricted to
selected child files by means of the |\includeonly| command.
The latter feature can be used to reduce the compilation time while editing
(this was significantly more useful in the earlier days of \LaTeX{})
or to generate a smaller document which is easier to navigate.
Another application of |\includeonly| is to generate
documents consisting of selected parts of the complete document.

However, there are a few drawbacks of the plain |\include| mechanism:
\begin{itemize}
\item
The child files cannot be compiled on their own,
they can only be compiled via the main file.
A naive editing environment
(such as a text editor with an option
to have the current file processed by \LaTeX)
may require one to switch to the main file before compiling;
attempting to compile the child file produces errors.
\item
The main file must be modified (each time)
to adjust the |\includeonly| command
to the present needs. This easily leaves the main file in a messy state.
\item
The generated document will always carry the filename
of the main document. This is inconvenient if
several child files are to be compiled and
to be kept for distribution.
\end{itemize}

The present package provides a simple interface
to make child files individually compilable by \LaTeX{}.
Compiling a child file then has the same effect as compiling
the main file with an |\includeonly| command
to select the appropriate child.
Moreover the generated document will carry the name of the child
rather than the main file.
This resolves all three above issues.

This feature is meant to make the editing of books,
thesis documents and lecture notes somewhat more convenient.
However, the package can also be used efficiently for
composing a series of documents (such as exercise sheets)
which are typically distributed individually.
It then assists the author in generating the individual documents
(potentially in different versions)
as well as a document containing the collected series.
Another application is in developing style files
or other kinds of included material
where compilation of the style file could redirect
to a sample or test file.

%%%%%%%%%%%%%%%%%%%%%%%%%%%%%%%%%%%%%%%%%%%%%%%%%%%%%%%%%%%%%%%%%%%%%%%%%%%%%%%%
%%%%%%%%%%%%%%%%%%%%%%%%%%%%%%%%%%%%%%%%%%%%%%%%%%%%%%%%%%%%%%%%%%%%%%%%%%%%%%%%
\section{Usage}

First of all, the package \textsf{childdoc} is \emph{not} a standard
\LaTeXe{} |.sty| style file! Therefore it needs to be invoked in
a non-standard way.

%%%%%%%%%%%%%%%%%%%%%%%%%%%%%%%%%%%%%%%%%%%%%%%%%%%%%%%%%%%%%%%%%%%%%%%%%%%%%%%%
\subsection{Included Files}
\label{sec:include}

%%%%%%%%%%%%%%%%%%%%%%%%%%%%%%%%%%%%%%%%
\DescribeMacro{\childdocmain}
To use the package, add the commands
\begin{center}
\begin{tabular}{l}
|\input{childdoc.def}|\\
|\childdocmain{}|\\
\end{tabular}
\end{center}
at the very top of the main \LaTeX{} file,
in particular \emph{before} the |\documentclass| statement!
The argument of |\childdocmain| should be left empty
(but it must be present).

%%%%%%%%%%%%%%%%%%%%%%%%%%%%%%%%%%%%%%%%
\DescribeMacro{\childdocof}
Furthermore, add the commands
\begin{center}
\begin{tabular}{l}
|\input{childdoc.def}|\\
|\childdocof{|\textit{main}|}|\\
\end{tabular}
\end{center}
at the top of every child file \textit{child}
which is included by |\include{|\textit{child}|}|
from within the main file
(or at least for those files to be compiled individually).
The argument \textit{main} must be the filename of the main file.

There are a couple of
considerations in setting up the main and child documents:

%%%%%%%%%%%%%%%%%%%%%%%%%%%%%%%%%%%%%%%%
\paragraph{Restrictions.}

Please note the following restrictions:
\begin{itemize}
\item
|\childdocmain| must be called with one argument \textit{main}
to ensure compatibility with earlier version of the package.
It must either be empty (|\childdocmain{}|)
or precisely match the filename of the main file in which it is specified.
See \secref{sec:detection} for further information.
\item
The filename \textit{main} must be specified without the |.tex| extension.
\item
The filename \textit{main} is case sensitive
(even in case-insensitive file systems)
due to internal string comparison.
\item
The argument \textit{main} should be fully expanded, it cannot be a macro.
\item
Subdirectories and special characters should be avoided in filenames.
\item
The command |\childdocmain{|\textit{main}|}| must be followed by a whitespace.
It should not be followed immediately by another command
or by a comment mark `|%|'.
This is because the \TeX{} parser reads the token immediately following
the argument of |\childdocmain| and puts it
at the beginning of every child section;
however, a white\-space is ignored.
\end{itemize}

%%%%%%%%%%%%%%%%%%%%%%%%%%%%%%%%%%%%%%%%
\paragraph{Content of Main File.}

It is advisable to place all content in the child files included by |\include|.
Any output contained in the main file will appear in all child documents
unless suppressed manually;
it cannot be suppressed automatically by the |\includeonly| directive
and thus should normally be avoided.
A method to include some content in the main file
by means of conditional processing is described in \secref{sec:conditional}.

%%%%%%%%%%%%%%%%%%%%%%%%%%%%%%%%%%%%%%%%
\paragraph{Page Numbering.}

When only a part of the document is compiled,
the appropriate numbering of pages
(as well as other status parameters)
is determined from the |.aux| files.
The latter contain information from previous passes.
However this information needs to propagate through
all intermediate child documents.
Therefore the page numbering in child documents may well
be inconsistent until the complete document is compiled at least once.

A useful (if unconventional) way to always ensure a consistent
page numbering is to restart the numbering in each child document
and denote the pages by `\textit{child}|.|\textit{page}'
where \textit{child} represents the chapter/section number of the child file.
This can be achieved by the command
|\numberwithin{page}{|\textit{child}|}|
of the \textsf{amsmath} package
where \textit{child} can be |chapter| or |section|
depending on the chosen structuring.
Alternatively, one can modify the macro |\thepage| appropriately
and reset the counter |page| at the start of each child file.

%%%%%%%%%%%%%%%%%%%%%%%%%%%%%%%%%%%%%%%%%%%%%%%%%%%%%%%%%%%%%%%%%%%%%%%%%%%%%%%%
\subsection{Conditional Processing}
\label{sec:conditional}

The package provides a mechanism to compile different versions
of a document. To customise the versions further some conditional processing
can come in handy to distinguish which version is being compiled.
The package provides two macros to describe the compilation context:

%%%%%%%%%%%%%%%%%%%%%%%%%%%%%%%%%%%%%%%%
\DescribeMacro{\ifchilddoc}
The conditional |\ifchilddoc| distinguishes between the compilation of
child documents and the main document:
%
\begin{center}
|\ifchilddoc |\textit{child-code}| |[|\||else |\textit{main-code}]| \||fi|
\end{center}

%%%%%%%%%%%%%%%%%%%%%%%%%%%%%%%%%%%%%%%%
\DescribeMacro{\childdocname}
\DescribeMacro{\childdocjob}
The macro |\childdocname| contains the filename (without extension)
of the main or child file being processed.
Note that |\childdocjob| will always contain the name of the main file.

%%%%%%%%%%%%%%%%%%%%%%%%%%%%%%%%%%%%%%%%
\paragraph{Title Page.}

Conditional processing can be used to include a title or banner page
in the main document when proper precautions are taken.
Importantly, the code in the main file should ensure that the page counter
(as well as other status parameters which are stored in the |.aux| files)
takes the same value after the conditional processing.
Otherwise the page numbers may take divergent values
depending on which part is compiled.

For example, a title page could be declared by:
%
\begin{center}
\begin{tabular}{l}
|\ifchilddoc\||else|\\
|\addtocounter{page}{-1}|\\
\textit{code for title page}\\
|\newpage|\\
|\||fi|
\end{tabular}
\end{center}
%
A banner page for the child documents can be generated by:
%
\begin{center}
\begin{tabular}{l}
|\ifchilddoc|\\
|\addtocounter{page}{-1}|\\
\textit{code for banner page}\\
|\newpage|\\
|\||fi|
\end{tabular}
\end{center}
%
Here one could write a message such as:
\begin{center}
|This is the part \childdocname{} of \childdocjob{}.|
\end{center}

%%%%%%%%%%%%%%%%%%%%%%%%%%%%%%%%%%%%%%%%%%%%%%%%%%%%%%%%%%%%%%%%%%%%%%%%%%%%%%%%
\subsection{Flags}
\label{sec:flags}

The package makes it easy to generate different versions
of the main or child documents.
To this end compilation flags can be defined
and assigned different default values.
They will be particularly useful in conjunction
with the forwarding mechanism described in \secref{sec:forward}.

For example, it may be useful to have a flag |\version|
which can be set to |draft| or |final|.
The document source will contain some conditional code
depending on the value of |\version|.
Suppose further, the flag should default to |final| for the main file
and to |draft| for child files
which is a natural assignment for editing the document.
This is achieved by placing the following code
in the preamble of the main document
(below the |\childdocmain| directive):
%
\begin{center}
\begin{tabular}{l}
|\ifchilddoc|\\
|\providecommand{\version}{draft}|\\
|\||else|\\
|\providecommand{\version}{final}|\\
|\||fi|
\end{tabular}
\end{center}
%
The definition by |\providecommand| makes sure
that previous definitions are not overwritten.
Further statements |\providecommand{\version}{...}|
can thus be added before the above code to override it.

For the main file, one might add a line
(between |\childdocmain| and the above block)
%
\begin{center}
|%\ifchilddoc\||else\providecommand{\version}{draft}\||fi|
\end{center}
%
which can be uncommented to produce a draft version.
Likewise one can add a line to the very top of a child file
(above the |\childdocof{|\textit{main}|}| directive)
%
\begin{center}
|%\providecommand{\version}{final}|
\end{center}
%
which can be uncommented to produce the final version of this child document.

%%%%%%%%%%%%%%%%%%%%%%%%%%%%%%%%%%%%%%%%%%%%%%%%%%%%%%%%%%%%%%%%%%%%%%%%%%%%%%%%
\subsection{Forwarding}
\label{sec:forward}

Different versions of the main or child documents
using compilation flags as described in \secref{sec:flags}
can be (permanently) stored in different files
for convenient compilation, viewing and distribution.
To this end, the package defines a command
to pass on compilation to a different file:

%%%%%%%%%%%%%%%%%%%%%%%%%%%%%%%%%%%%%%%%
\DescribeMacro{\childdocforward}
The command |\childdocforward| redirects processing to
another source file:
%
\begin{center}
\begin{tabular}{l}
|\input{childdoc.def}|\\
|\childdocforward[|\textit{main}|]{|\textit{dest}|}|\\
\end{tabular}
\end{center}
%
The argument \textit{dest} is the destination file
(without extension).
It should be the main file or one of the child files.
Note that further \textsf{childdoc} directives
such as |\childdocof| and |\childdocforward|
in the indicated file will be processed in this form.
The optional argument \textit{main}
passes on directly to the main file \textit{main}
while pretending to compile the child \textit{dest}.
This form behaves as if \textit{dest}
issues |\childdocof{|\textit{main}|}| right away,
and no further \textsf{childdoc} directives will be processed.

%%%%%%%%%%%%%%%%%%%%%%%%%%%%%%%%%%%%%%%%
\DescribeMacro{\...prefix}
In the alternative form |\childdocforwardprefix|,
%
\begin{center}
\begin{tabular}{l}
|\input{childdoc.def}|\\
|\childdocforwardprefix[|\textit{main}|]{|\textit{prefix}|}{|\textit{dest}|}|
\end{tabular}
\end{center}
%
the destination file is determined by a pattern
depending on the current file:
To make this work, the current file must be called
`{\textit{prefix}\hspace{0.2em}\textit{suffix}}'
with \textit{prefix} matching precisely the argument.
Processing is then passed on to the file
`{\textit{dest}\hspace{0.2em}\textit{suffix}}'.
Surely, the same effect is achieved by
directly specifying the
argument `{\textit{dest}\hspace{0.2em}\textit{suffix}}'
in the first form.
However, that requires to set up a different file
for each child. With the alternative form of the command
all these files can have exactly the same content
which simplifies setting them up and maintaining them.

For example, the following file |draft.tex|
with a compilation flag |\version| as described in \secref{sec:flags}
compiles the main document as a draft:
%
\begin{center}
\begin{tabular}{l}
|\def\version{draft}|\\
|\input{childdoc.def}|\\
|\childdocforward{|\textit{main}|}|
\end{tabular}
\end{center}
%
Likewise, the following files |final|\textit{nn}|.tex|
compile the final version of the child document
|child|\textit{nn}|.tex|:
%
\begin{center}
\begin{tabular}{l}
|\def\version{final}|\\
|\input{childdoc.def}|\\
|\childdocforwardprefix{final}{child}|
\end{tabular}
\end{center}
%

Note that when several versions of a main file and/or of each child file
are to be generated, it may be convenient to set up a |Makefile| or
shell script to automatise the process.

%%%%%%%%%%%%%%%%%%%%%%%%%%%%%%%%%%%%%%%%%%%%%%%%%%%%%%%%%%%%%%%%%%%%%%%%%%%%%%%%
\subsection{Command Line Processing}
\label{sec:commandline}

The effect of redirection files can also be achieved by invoking
the \LaTeX{} compiler with a more elaborate command line.
Most conveniently this should be done as part
of a shell script or a |Makefile|.

When using \textsf{childdoc} in the main file, the following
command lines effectively perform a redirection
(note that depending on the shell being used,
backslashes may have to be doubled: `|\|' $\to$ `|\\|'):
%
\begin{center}
|... -jobname "|\textit{target}|" |\\|"|[\textit{flags}]%
|\input{childdoc.def}\childdocforward[|\textit{main}|]{|\textit{dest}|}"|
\end{center}
%
Here \textit{target} is the name of the output file,
\textit{main} is the name of the main file
and \textit{dest} is the name of the main or child file to be processed
(all filenames without extensions).
The optional argument \textit{main} can be omitted
if \textit{main} matches \textit{dest}.
Optionally, compilation \textit{flags} can be defined via |\def| commands.
This command line makes the \TeX{} engine believe
it is compiling the file \textit{target}
whose content is specified as the latter parameter.
The provided code then forwards the processing to
\textit{main} or \textit{dest} as described in \secref{sec:forward}.

%%%%%%%%%%%%%%%%%%%%%%%%%%%%%%%%%%%%%%%%%%%%%%%%%%%%%%%%%%%%%%%%%%%%%%%%%%%%%%%%
\subsection{Include by Input}
\label{sec:input}

Including child documents by |\include| has some restrictions by design.
Most notably, the content of a child document always occupies
its own set of pages; pages cannot be shared between child documents.
Usually, this behaviour makes perfect sense
because each child document contain an essential part of the document.
However, in some situations it may be desirable to compose
a document from a collection of parts
without having mandatory page breaks between then.
For this case, the package
provides a mechanism to include parts
by |\input| which can also be processed individually.
However, by construction this mechanism
requires manual handling of the content to be output.

%%%%%%%%%%%%%%%%%%%%%%%%%%%%%%%%%%%%%%%%
\DescribeMacro{\ifchilddocmanual}
The main file should be prepared as usual, see \secref{sec:include}.
However, the document body must make a distinction
between processing of an individual part and of the main document, e.g.:
%
\begin{center}
\begin{tabular}{l}
|\ifchilddocmanual|\\
|\input{\childdocname}|\\
|\||else|\\
\textit{document body with }|\input{|\textit{part}|}|\\
|\||fi|
\end{tabular}
\end{center}
%
The conditional |\ifchilddocmanual| is true whenever
a part to be included by |\input| is being compiled,
and the name of the part is stored in |\childdocname|.

%%%%%%%%%%%%%%%%%%%%%%%%%%%%%%%%%%%%%%%%
\DescribeMacro{\childdocby}
Each part to be included by |\input| should start with:
%
\begin{center}
\begin{tabular}{l}
|\input{childdoc.def}|\\
|\childdocby{|\textit{main}|}|\\
\end{tabular}
\end{center}
%
The directive |\childdocby| is similar to |\childdocof|
described in \secref{sec:include},
but the subsequent selection of content must be done manually.
To that end, both |\ifchilddoc| and |\ifchilddocmanual|
will be true upon processing of a part,
and the name of the part is stored in |\childdocname|.
Note that |\jobname| will be set to the filename of the current part
so that each part receives an individual |.aux| file
that does not interfere with the |.aux| file(s) of the main document.
This behaviour can be altered by the alternative form
|\childdocby[*]{|\textit{main}|}| (with a non-empty optional argument)
which uses the |.aux| file of the main document
by setting |\jobname| to \textit{main}.

%%%%%%%%%%%%%%%%%%%%%%%%%%%%%%%%%%%%%%%%%%%%%%%%%%%%%%%%%%%%%%%%%%%%%%%%%%%%%%%%
\subsection{Driver Development}
\label{sec:driver}

The \textsf{childdoc} mechanism can also be use for the development
of definition files such as \LaTeX{} styles or classes.
This case differs from the above setup with multiple parts
included by |\include| in that no |\includeonly| should be invoked.
This can be achieved by starting the include file
(before |\ProvidesPackage|) with:
%
\begin{center}
\begin{tabular}{l}
|\input{childdoc.def}|\\
|\childdocforward{|\textit{main}|}|\\
\end{tabular}
\end{center}
%
or alternatively with:
%
\begin{center}
\begin{tabular}{l}
|\input{childdoc.def}|\\
|\childdocby{|\textit{main}|}|\\
\end{tabular}
\end{center}
%
Both forms have slightly different effects as described above.
The main file is prepared as usual, see \secref{sec:include}.

%%%%%%%%%%%%%%%%%%%%%%%%%%%%%%%%%%%%%%%%%%%%%%%%%%%%%%%%%%%%%%%%%%%%%%%%%%%%%%%%
\subsection{Legacy Detection}
\label{sec:detection}

The directive |\childdocmain| in the main file can detect
whether the complete document or merely a child is to be compiled
even without using the directive |\childdocof|.
This method is deprecated because it is less robust
and there is no compelling reason to use it;
it is merely provided for backward compatibility
and it may be removed in future versions.

If the detection mechanism is to be used,
it is mandatory to correctly specify
the filename of the main file as the argument of |\childdocmain|:
%
\begin{center}
\begin{tabular}{l}
|\input{childdoc.def}|\\
|\childdocmain{|\textit{main}|}|\\
\end{tabular}
\end{center}
%
If |\jobname| does not match the argument \textit{main} of |\childdocmain|,
it is assumed that |\jobname| points to the child file to be compiled.
When using |\childdocmain| with the main file specified as argument,
it suffices to start a child file
with just |\input{|\textit{main}|}|
without loading of the package and using |\childdocof|.
If instead all processing is done
with the appropriate \textsf{childdoc} directives,
the argument of \textit{main} of |\childdocmain| can be empty.

An alternative version of the command line processing described
in \secref{sec:commandline} using the detection mechanism reads:
%
\begin{center}
|... -jobname "|\textit{target}|" "|[\textit{flags}]%
[|\def\jobname{|\textit{dest}|}|]|\input{|\textit{main}|}"|
\end{center}

%%%%%%%%%%%%%%%%%%%%%%%%%%%%%%%%%%%%%%%%%%%%%%%%%%%%%%%%%%%%%%%%%%%%%%%%%%%%%%%%
\subsection{Manual Code}
\label{sec:manual}

In case one cannot be certain whether the definitions file |childdoc.def|
is installed on the target \TeX{} distribution
and one prefers not to ship it,
it is conceivable to paste a few relevant commands into the sources.

To that end, drop all statements |\input{childdoc.def}|
and perform the replacements as outlined below.
Instead of |\childdocmain{|\textit{main}|}| add the following code
to the top of the main file:
%
\begin{center}
\begin{tabular}{l}
|\||ifdefined\childdocname\endinput\||fi\newif\ifchilddoc|\\
|\edef\childdocname{\scantokens\expandafter{\jobname\noexpand}}|\\
|\def\childdocmain{|\textit{main}|}\||ifx\childdocmain\childdocname\||else|\\
|\childdoctrue\includeonly{\childdocname}\let\jobname\childdocmain\||fi|\\
\end{tabular}
\end{center}
%
Instead of |\childdocof{|\textit{main}|}| just include the main file
at the top of each child file:
%
\begin{center}
|\input{|\textit{main}|}|
\end{center}
%
A simple redirection |\childdocforward{|\textit{dest}|}| is achieved by:
%
\begin{center}
|\def\jobname{|\textit{dest}|}\input{\jobname}|
\end{center}
%
The redirection with prefix
|\childdocforwardprefix[|\textit{prefix}|]{|\textit{dest}|}|
is accomplished by:
%
\begin{center}
\begin{tabular}{l}
|{\edef\jobname{\scantokens\expandafter{\jobname\noexpand}}|\\
|\def\redirectjob |\textit{prefix}|#1~~~{\gdef\jobname{|\textit{dest}|#1}}|\\
|\expandafter\redirectjob\jobname~~~}\input{\jobname}|
\end{tabular}
\end{center}

In an alternative approach,
child documents can be compiled by a specific command line
without additional code or specific definitions:
%
\begin{center}
|... -jobname "|\textit{target}|" "|[\textit{flags}]%
|\includeonly{|\textit{dest}|}\input{|\textit{main}|}"|
\end{center}
%

%%%%%%%%%%%%%%%%%%%%%%%%%%%%%%%%%%%%%%%%%%%%%%%%%%%%%%%%%%%%%%%%%%%%%%%%%%%%%%%%
%%%%%%%%%%%%%%%%%%%%%%%%%%%%%%%%%%%%%%%%%%%%%%%%%%%%%%%%%%%%%%%%%%%%%%%%%%%%%%%%
\section{Information}

%%%%%%%%%%%%%%%%%%%%%%%%%%%%%%%%%%%%%%%%%%%%%%%%%%%%%%%%%%%%%%%%%%%%%%%%%%%%%%%%
\subsection{Copyright}

Copyright \copyright{} 2017--2018 Niklas Beisert

This work may be distributed and/or modified under the
conditions of the \LaTeX{} Project Public License, either version 1.3
of this license or (at your option) any later version.
The latest version of this license is in
  \url{http://www.latex-project.org/lppl.txt}
and version 1.3 or later is part of all distributions of \LaTeX{}
version 2005/12/01 or later.

This work has the LPPL maintenance status `maintained'.

The Current Maintainer of this work is Niklas Beisert.

This work consists of the files |README.txt|, |childdoc.ins| and |childdoc.dtx|
as well as the derived files |childdoc.def|, |cdocsamp.tex|
with |cdocsch1.tex|, |cdocsch2.tex|, |cdocspt3.tex|, |cdocspt4.tex|,
|cdocsdrf.tex|, |cdocsfn1.tex|, |cdocsfn2.tex|
as well as |childdoc.pdf|.

%%%%%%%%%%%%%%%%%%%%%%%%%%%%%%%%%%%%%%%%%%%%%%%%%%%%%%%%%%%%%%%%%%%%%%%%%%%%%%%%
\subsection{Files and Installation}

The package consists of the files:
%
\begin{center}
\begin{tabular}{ll}
    |README.txt|   & readme file \\
    |childdoc.ins| & installation file \\
    |childdoc.dtx| & source file \\
    |childdoc.def| & definition file \\
    |cdocsamp.tex| & sample main file \\
    |cdocsch1.tex| & sample include file \\
    |cdocsch2.tex| & sample include file \\
    |cdocspt3.tex| & sample part file \\
    |cdocspt4.tex| & sample part file \\
    |cdocsdrf.tex| & sample redirection file \\
    |cdocsfn1.tex| & sample redirection file \\
    |cdocsfn2.tex| & sample redirection file \\
    |childdoc.pdf| & manual
\end{tabular}
\end{center}
%
The distribution consists of the files
|README.txt|, |childdoc.ins| and |childdoc.dtx|.
%
\begin{itemize}
\item
Run (pdf)\LaTeX{} on |childdoc.dtx|
to compile the manual |childdoc.pdf| (this file).
\item
Run \LaTeX{} on |childdoc.ins| to create the definitions file |childdoc.def|
and the sample |cdocsamp.tex| with include files
|cdocsch1.tex|, |cdocsch2.tex|, |cdocspt3.tex|, |cdocspt4.tex|,
|cdocsdrf.tex|, |cdocsfn1.tex|, |cdocsfn2.tex|.
Then copy the file |childdoc.def| to an appropriate directory of your \LaTeX{}
distribution, e.g.\ \textit{texmf-root}|/tex/latex/childdoc|.
\end{itemize}

%%%%%%%%%%%%%%%%%%%%%%%%%%%%%%%%%%%%%%%%%%%%%%%%%%%%%%%%%%%%%%%%%%%%%%%%%%%%%%%%
\subsection{Related CTAN Packages}

There are several other packages which offer a similar functionality:
%
\begin{itemize}
\item
The packages
\href{http://ctan.org/pkg/docmute}{\textsf{docmute}},
\href{http://ctan.org/pkg/includex}{\textsf{includex}} and
\href{http://ctan.org/pkg/standalone}{\textsf{standalone}}
provide commands to include only the document body of
a child file thus allowing both files to be compiled individually.
\item
The packages \href{http://ctan.org/pkg/subdocs}{\textsf{subdocs}}
and \href{http://ctan.org/pkg/subfiles}{\textsf{subfiles}}
provide structures in which the main and child documents can be
encapsulated and allowing them to be compiled individually.
The inclusion mechanism is different from the conventional |\include|.
\item
The package \href{http://ctan.org/pkg/combine}{\textsf{combine}}
is an elaborate solution to combine several documents into one.
\end{itemize}
%
See also the CTAN topic \href{http://ctan.org/topic/subdocs}{\textsf{subdocs}}
for further related packages.
The present package differs from the above solutions in that
a document structure constructed with the conventional |\include| mechanism
just needs two extra commands at the top of every file
such that all constituent files can be compiled individually.

%%%%%%%%%%%%%%%%%%%%%%%%%%%%%%%%%%%%%%%%%%%%%%%%%%%%%%%%%%%%%%%%%%%%%%%%%%%%%%%%
%\subsection{Feature Suggestions}
%
%The following is a list of features which may be useful for future
%versions of this package:
%%
%\begin{itemize}
%\item
%\ldots
%\end{itemize}

%%%%%%%%%%%%%%%%%%%%%%%%%%%%%%%%%%%%%%%%%%%%%%%%%%%%%%%%%%%%%%%%%%%%%%%%%%%%%%%%
\subsection{Revision History}

%%%%%%%%%%%%%%%%%%%%%%%%%%%%%%%%%%%%%%%%
\paragraph{v2.0:} 2018/12/30

\begin{itemize}
\item
immediate forward processing
\item
added |\childdocby| mechanism
\item
manual restructured
\end{itemize}

%%%%%%%%%%%%%%%%%%%%%%%%%%%%%%%%%%%%%%%%
\paragraph{v1.6:} 2018/01/17

\begin{itemize}
\item
application for development of include files
\item
corrections to manual
\end{itemize}

%%%%%%%%%%%%%%%%%%%%%%%%%%%%%%%%%%%%%%%%
\paragraph{v1.5:} 2017/05/21

\begin{itemize}
\item
more complete structuring introduced
\item
|\childdocof| introduced
\item
|\childdoc| renamed to |\childdocmain|
\item
|\childredirect| renamed to |\childdocforward| and |\childdocforwardprefix|
and functionality expanded
\end{itemize}

%%%%%%%%%%%%%%%%%%%%%%%%%%%%%%%%%%%%%%%%
\paragraph{v1.0:} 2017/04/27

\begin{itemize}
\item
manual and install package
\item
first version published on CTAN
\end{itemize}

%%%%%%%%%%%%%%%%%%%%%%%%%%%%%%%%%%%%%%%%
\paragraph{v0.6:} 2017/04/26

\begin{itemize}
\item
redirection mechanism added
\end{itemize}

%%%%%%%%%%%%%%%%%%%%%%%%%%%%%%%%%%%%%%%%
\paragraph{v0.5:} 2017/04/26

\begin{itemize}
\item
functionality in definition file
\end{itemize}


%%%%%%%%%%%%%%%%%%%%%%%%%%%%%%%%%%%%%%%%%%%%%%%%%%%%%%%%%%%%%%%%%%%%%%%%%%%%%%%%
%%%%%%%%%%%%%%%%%%%%%%%%%%%%%%%%%%%%%%%%%%%%%%%%%%%%%%%%%%%%%%%%%%%%%%%%%%%%%%%%
%%%%%%%%%%%%%%%%%%%%%%%%%%%%%%%%%%%%%%%%%%%%%%%%%%%%%%%%%%%%%%%%%%%%%%%%%%%%%%%%
\appendix

\settowidth\MacroIndent{\rmfamily\scriptsize 000\ }

 \DocInput{childdoc.dtx}

\end{document}
%</driver>
% \fi
%
% %%%%%%%%%%%%%%%%%%%%%%%%%%%%%%%%%%%%%%%%%%%%%%%%%%%%%%%%%%%%%%%%%%%%%%%%%%%%%%
% %%%%%%%%%%%%%%%%%%%%%%%%%%%%%%%%%%%%%%%%%%%%%%%%%%%%%%%%%%%%%%%%%%%%%%%%%%%%%%
% \section{Sample}
%\iffalse
%<*samplemain>
%\fi
%
% The following presents a sample document
% with two chapters, two parts, a title page,
% a compile flag as well as three forwarding files to set the flag.
% It consists of eight |.tex| files:
% \begin{center}
% \begin{tabular}{ll}
% |cdocsamp.tex|&main file\\
% |cdocsch1.tex|&include file for chapter 1\\
% |cdocsch2.tex|&include file for chapter 2\\
% |cdocspt3.tex|&include file for part 3\\
% |cdocspt4.tex|&include file for part 4\\
% |cdocsdrf.tex|&forwarding file for main file in draft mode\\
% |cdocsfi1.tex|&forwarding file for final version of chapter 1\\
% |cdocsfi2.tex|&forwarding file for final version of chapter 2\\
% \end{tabular}
% \end{center}
% Each of the eight files can be compiled directly by the \LaTeX{} compiler.
%
% %%%%%%%%%%%%%%%%%%%%%%%%%%%%%%%%%%%%%%
% \paragraph{Main File.}
%
% The main file is called |cdocsamp.tex|.
%
% Load the \textsf{childdoc} definitions and
% declare the filename for the main document:
%    \begin{macrocode}
\input{childdoc.def}
\childdocmain{}
%    \end{macrocode}

% Optional override for |\version| flag:
%    \begin{macrocode}
%%\ifchilddoc\else\providecommand{\version}{draft}\fi
%    \end{macrocode}

% Define the default values for the |\version| flag
% (|final| for the main file and |draft| for childs):
%    \begin{macrocode}
\ifchilddoc
\providecommand{\version}{draft}
\else
\providecommand{\version}{final}
\fi
%    \end{macrocode}

% Load the standard document class:
%    \begin{macrocode}
\documentclass[12pt]{article}
%    \end{macrocode}

% Start the document body:
%    \begin{macrocode}
\begin{document}
%    \end{macrocode}

% Declare a title page.
% Print title, part of document being processed and version flag:
%    \begin{macrocode}
\addtocounter{page}{-1}
\begin{center}
{\LARGE\bfseries{}childdoc example\par}
\vspace{1cm}
\ifchilddoc
\ifchilddocmanual part\else chapter\fi:
`\childdocname' of `\childdocjob'\par
\else
main document: `\childdocjob'\par
\fi
version: \version\par
\end{center}
\newpage
%    \end{macrocode}

% Manually include selected file,
% otherwise process as usual:
%    \begin{macrocode}
\ifchilddocmanual
\section*{part `\childdocname'}
\input{\childdocname}
\else
%    \end{macrocode}

% Include the two chapters:
%    \begin{macrocode}
\include{cdocsch1}
\include{cdocsch2}
%    \end{macrocode}

% Include the two parts unless only chapters should be displayed:
%    \begin{macrocode}
\ifchilddoc\else
\section{part three}
\input{cdocspt3}
\section{part four}
\input{cdocspt4}
\fi
%    \end{macrocode}

% Process as usual until here:
%    \begin{macrocode}
\fi
%    \end{macrocode}

% End of document body:
%    \begin{macrocode}
\end{document}
%    \end{macrocode}
%\iffalse
%</samplemain>
%\fi
%
% %%%%%%%%%%%%%%%%%%%%%%%%%%%%%%%%%%%%%%
% \paragraph{Chapter Include Files.}
%
% The include files are called |cdocsch1.tex| and |cdocsch2.tex|.
%
%\iffalse
%<*samplechap1|samplechap2>
%\fi

% Optional override for |\version| flag:
%    \begin{macrocode}
%%\providecommand{\version}{final}
%    \end{macrocode}

% Include the main document:
%    \begin{macrocode}
\input{childdoc.def}
\childdocof{cdocsamp}
%    \end{macrocode}

%\iffalse
%</samplechap1|samplechap2>
%\fi
%
%\iffalse
%<*samplechap1>
%\fi
% Some text for chapter 1:
%    \begin{macrocode}
\section{one}
some text in chapter one
%    \end{macrocode}

%\iffalse
%</samplechap1>
%\fi
% Some text for chapter 2:
%\iffalse
%<*samplechap2>
%\fi
%    \begin{macrocode}
\section{two}
more text in chapter two
%    \end{macrocode}

%\iffalse
%</samplechap2>
%\fi
%
% %%%%%%%%%%%%%%%%%%%%%%%%%%%%%%%%%%%%%%
% \paragraph{Part Include Files.}
%
% The include files are called |cdocspt3.tex| and |cdocspt4.tex|.
%
%\iffalse
%<*samplepart3|samplepart4>
%\fi

% Optional override for |\version| flag:
%    \begin{macrocode}
%%\providecommand{\version}{final}
%    \end{macrocode}

% Include the main document:
%    \begin{macrocode}
\input{childdoc.def}
\childdocby{cdocsamp}
%    \end{macrocode}

%\iffalse
%</samplepart3|samplepart4>
%\fi
%
%\iffalse
%<*samplepart3>
%\fi
% Some text for part 3:
%    \begin{macrocode}
some text in part three
%    \end{macrocode}

%\iffalse
%</samplepart3>
%\fi
% Some text for part 4:
%\iffalse
%<*samplepart4>
%\fi
%    \begin{macrocode}
more text in part four
%    \end{macrocode}

%\iffalse
%</samplepart4>
%\fi
%
% %%%%%%%%%%%%%%%%%%%%%%%%%%%%%%%%%%%%%%
% \paragraph{Forwarding for a Complete Draft.}
%
% The following forwarding file |cdocsdrf.tex|
% compiles the main document in draft mode:
%\iffalse
%<*sampledraft>
%\fi
%    \begin{macrocode}
\def\version{draft}
\input{childdoc.def}
\childdocforward{cdocsamp}
%    \end{macrocode}

%\iffalse
%</sampledraft>
%\fi
%
% %%%%%%%%%%%%%%%%%%%%%%%%%%%%%%%%%%%%%%
% \paragraph{Forwarding for Final Version of the Chapters.}
%
% The following forwarding files |cdocsfn1.tex| and |cdocsfn2.tex|
% (with identical content)
% compile the final versions of the child documents
% |cdocsch1.tex| and |cdocsch2.tex|, respectively:
%\iffalse
%<*samplefinal>
%\fi
%    \begin{macrocode}
\def\version{final}
\input{childdoc.def}
\childdocforwardprefix[cdocsamp]{cdocsfn}{cdocsch}
%    \end{macrocode}

%\iffalse
%</samplefinal>
%\fi
%
% %%%%%%%%%%%%%%%%%%%%%%%%%%%%%%%%%%%%%%
% \paragraph{Command Line Processing.}
%
% The following three command lines generate the output files
% |cdocscld|, |cdocscl1| and |cdocscl2|
% which should be identical to
% |cdocsdrf|, |cdocsch1| and |cdocsfn2|, respectively:
% \begin{center}
% \begin{tabular}{l}
% |latex -jobname cdocscld \|\\
% |  "\def\version{draft}\input{childdoc.def}\childdocforward{cdocsamp}"|\\
% |latex -jobname cdocscl1 \|\\
% |  "\input{childdoc.def}\childdocforward[cdocsamp]{cdocsch1}"|\\
% |latex -jobname cdocscl2 \|\\
% |  "\def\version{final}\input{childdoc.def}\childdocforward{cdocsch2}"|
% \end{tabular}
% \end{center}
% Note that the trailing backslash on each first line
% merely continues the input to the second line
% (for convenient cut ant paste).
% Furthermore, the command |latex| can be replaced by any
% of its alternative versions such as |pdflatex|.
%
% %%%%%%%%%%%%%%%%%%%%%%%%%%%%%%%%%%%%%%%%%%%%%%%%%%%%%%%%%%%%%%%%%%%%%%%%%%%%%%
% %%%%%%%%%%%%%%%%%%%%%%%%%%%%%%%%%%%%%%%%%%%%%%%%%%%%%%%%%%%%%%%%%%%%%%%%%%%%%%
% \section{Implementation}
%\iffalse
%<*package>
%\fi
%
% This section describes the definitions file |childdoc.def|.

% The definitions cannot be loaded using |\usepackage| or |\RequirePackage|
% which has a mechanism to prevent loading a style file more than once.
% When loading the definitions by means of |\input|
% multiple instances have to be prevented manually:
%\iffalse
%This code needs to be before the `\ProvidesFile' directive
%which is defined at the beginning of this file.
%Therefore it is also placed there and commented out here.
%</package>
%<*discard>
%\fi
%    \begin{macrocode}
\ifdefined\childdocmain\endinput\fi
%    \end{macrocode}
%\iffalse
%</discard>
%<*package>
%\fi
%
% \macro{\ifchilddoc}
% \macro{\ifchilddocmanual}
% The conditional |\ifchilddoc| tells whether a
% child (true) or main (false) document is being compiled.
% The conditional |\ifchilddocmanual| tells whether
% the |\includeonly| mechanism is used (false) or
% the selection of child files must be performed manually (true).
% The definitions initialise to false:
%    \begin{macrocode}
\newif\ifchilddoc
\newif\ifchilddocmanual
%    \end{macrocode}

% \macro{\childdocname}
% \macro{\childdocjob}
% The macro |\childdocname| stores the name of the main document
% to be compiled. The macro |\childdocjob| stores the name of
% the document on which the \LaTeX{} compiler was originally invoked.
% The content of |\jobname| cannot be compared
% to filenames specified in the source due to different catcodes.
% The following code rescans |\jobname|, stores the result
% in |\childdocname| and saves a copy in |\childdocjob|:
%    \begin{macrocode}
\edef\childdocname{\scantokens\expandafter{\jobname\noexpand}}
\let\childdocjob\childdocname
%    \end{macrocode}

% \macro{\childdocdisable}
% The macro |\childdocdisable| prevents the main file
% from being processed more than once.
% At this stage, the main document command |\childdocmain|
% is assumed to be called once again where it should do nothing.
% Any subsequent call to it should prevent
% a secondary processing of the main document
% It overwrites the forwarding commands
% |\childdocof| and |\childdocforward|
% with empty macros to prevent further inclusions of the main document:
%    \begin{macrocode}
\newcommand{\childdocdisable}
{
  \renewcommand{\childdocmain}[1]{\renewcommand{\childdocmain}[1]{\endinput}}
  \renewcommand{\childdocof}[1]{}
  \renewcommand{\childdocby}[2][]{}
  \renewcommand{\childdocforward}[2][]{}
  \renewcommand{\childdocdisable}{}
}
%    \end{macrocode}

% \macro{\childdocmain}
% The macro |\childdocmain| is to be called at the top of the main file
% with nothing or the main filename (without extension) as argument.
% First, it breaks loops.
% If the argument is not empty and does not match |\childdocname|
% (which is set by the first inclusion of |childdoc.def|),
% |\ifchilddoc| is set to true, |\includeonly| is applied to the child file
% and |\jobname| is set to the main file
% (for proper handling of |.aux| files):
%    \begin{macrocode}
\newcommand{\childdocmain}[1]
{
  \childdocdisable\childdocmain{}
  \if?#1?\else
    \begingroup
      \def\childdoctmp{#1}
      \ifx\childdoctmp\childdocname
        \def\childdoctmp{}
      \else
        \def\childdoctmp
        {
          \childdoctrue
          \includeonly{\childdocname}
          \def\childdocjob{#1}
          \def\jobname{#1}
        }
      \fi
      \expandafter
    \endgroup
    \childdoctmp
  \fi
}
%    \end{macrocode}

% \macro{\childdocof}
% The command |\childdocof| redirects
% compilation to the main file |#1|.
%    \begin{macrocode}
\newcommand{\childdocof}[1]
{
  \childdocdisable
  \childdoctrue
  \includeonly{\childdocname}
  \def\jobname{#1}
  \def\childdocjob{#1}
  \input{#1}
}
%    \end{macrocode}

% \macro{\childdocby}
% The command |\childdocby| ....
%    \begin{macrocode}
\newcommand{\childdocby}[2][]
{
  \childdocdisable
  \childdoctrue
  \childdocmanualtrue
  \if?#1?\else
    \def\jobname{#2}
  \fi
  \def\childdocjob{#2}
  \input{#2}
  \endinput
}
%    \end{macrocode}

% \macro{\childdocforward}
% The command |\childdocforward| redirects
% compilation to the main file or
% (if the optional argument is given) a child file.
% Parameters are set as if the main file
% or a child file starting with |\childdocof| was compiled.
% Then compilation is handed over to the main file:
%    \begin{macrocode}
\newcommand{\childdocforward}[2][]
{
  \begingroup
    \if?#1?
      \def\childdoctmp
      {
        \def\childdocname{#2}
        \def\childdocjob{#2}
        \def\jobname{#2}
        \input{#2}
        \endinput
      }
    \else
      \def\childdoctmp
      {
        \childdocdisable
        \def\childdocname{#2}
        \childdoctrue
        \includeonly{#2}
        \def\childdocjob{#1}
        \def\jobname{#1}
        \input{#1}
        \endinput
      }
    \fi
    \expandafter
  \endgroup
  \childdoctmp
}
%    \end{macrocode}

% \macro{\childdocforwardprefix}
% The command |\childdocforwardprefix| redirects
% compilation to the main or a child file by means of a pattern.
% The prefix |#1| in the current filename is replaced by |#2|
% and the suffix of the current filename is kept
% (it is assumed that the filename does not contain the substring `|~~~|'
% which is used as a delimiter).
% Compilation is handed over to the new file by |\childdocforward|:
%    \begin{macrocode}
\newcommand{\childdocforwardprefix}[3][]
{
  \begingroup
    \def\childdocextract #2##1~~~{\def\childdoctmp{\childdocforward[#1]{#3##1}}}
    \expandafter\childdocextract\childdocname~~~
    \expandafter
  \endgroup
  \childdoctmp
}
%    \end{macrocode}

% \macro{\childdoc}
% The deprecated macro |\childdoc| is a legacy version of |\childdocmain|:
%    \begin{macrocode}
\newcommand{\childdoc}{\childdocmain}
%    \end{macrocode}

% \macro{\childdocredirect}
% The deprecated macro |\childdocredirect| is a legacy version
% of |\childdocforward| and |\childdocforwardprefix|:
%    \begin{macrocode}
\newcommand{\childdocredirect}[2][]
{
  \begingroup
    \if?#1?
      \def\childdoctmp{\childdocforward{#2}}
    \else
      \def\childdoctmp{\childdocforwardprefix{#1}{#2}}
    \fi
    \expandafter
  \endgroup
  \childdoctmp
}
%    \end{macrocode}

%\iffalse
%</package>
%\fi
%
\endinput
|\\
|\childdocforward{|\textit{main}|}|\\
\end{tabular}
\end{center}
%
or alternatively with:
%
\begin{center}
\begin{tabular}{l}
|% \iffalse
%
% childdoc.dtx Copyright (C) 2017-2018 Niklas Beisert
%
% This work may be distributed and/or modified under the
% conditions of the LaTeX Project Public License, either version 1.3
% of this license or (at your option) any later version.
% The latest version of this license is in
%   http://www.latex-project.org/lppl.txt
% and version 1.3 or later is part of all distributions of LaTeX
% version 2005/12/01 or later.
%
% This work has the LPPL maintenance status `maintained'.
%
% The Current Maintainer of this work is Niklas Beisert.
%
% This work consists of the files childdoc.dtx and childdoc.ins
% and the derived files childdoc.def and cdocsamp.tex with
% cdocsch1.tex, cdocsch2.tex, cdocsdrf.tex, cdocsfn1.tex, cdocsfn2.tex.
%
%<package>\ifdefined\childdocmain\endinput\fi
%<package>\ProvidesFile{childdoc.def}[2018/12/30 v2.0 child document driver]
%<samplemain>\ProvidesFile{cdocsamp.tex}[2018/12/30 v2.0 sample for childdoc]
%<*driver>
%\ProvidesFile{childdoc.drv}[2018/12/30 v2.0 childdoc reference manual file]
\PassOptionsToClass{10pt,a4paper}{article}
\documentclass{ltxdoc}

\usepackage[margin=35mm]{geometry}
\usepackage{hyperref}
\usepackage{hyperxmp}
\usepackage[usenames]{color}

\hypersetup{colorlinks=true}
\hypersetup{pdfstartview=FitH}
\hypersetup{pdfpagemode=UseNone}
\hypersetup{pdfsource={}}
\hypersetup{pdflang={en-UK}}
\hypersetup{pdfcopyright={Copyright 2017-2018 Niklas Beisert.
  This work may be distributed and/or modified under the
  conditions of the LaTeX Project Public License, either version 1.3
  of this license or (at your option) any later version.}}
\hypersetup{pdflicenseurl={http://www.latex-project.org/lppl.txt}}
\hypersetup{pdfcontactaddress={ETH Zurich, ITP, HIT K,
  Wolfgang-Pauli-Strasse 27}}
\hypersetup{pdfcontactpostcode={8093}}
\hypersetup{pdfcontactcity={Zurich}}
\hypersetup{pdfcontactcountry={Switzerland}}
\hypersetup{pdfcontactemail={nbeisert@itp.phys.ethz.ch}}
\hypersetup{pdfcontacturl={http://people.phys.ethz.ch/\xmptilde nbeisert/}}

\newcommand{\secref}[1]{\hyperref[#1]{section \ref*{#1}}}

\parskip1ex
\parindent0pt
\let\olditemize\itemize
\def\itemize{\olditemize\parskip0pt}

\begin{document}

\title{The \textsf{childdoc} Package}
\hypersetup{pdftitle={The childdoc Package}}
\author{Niklas Beisert\\[2ex]
  Institut f\"ur Theoretische Physik\\
  Eidgen\"ossische Technische Hochschule Z\"urich\\
  Wolfgang-Pauli-Strasse 27, 8093 Z\"urich, Switzerland\\[1ex]
  \href{mailto:nbeisert@itp.phys.ethz.ch}
  {\texttt{nbeisert@itp.phys.ethz.ch}}}
\hypersetup{pdfauthor={Niklas Beisert}}
\hypersetup{pdfsubject={Manual for the LaTeX2e Package childdoc}}
\date{30 December 2018, \textsf{v2.0}}
\maketitle

\begin{abstract}\noindent
\textsf{childdoc} is a \LaTeXe{} package
that enables the direct compilation
of document sections included by |\include|
to individual files.
\end{abstract}

\begingroup
\parskip0ex
\tableofcontents
\endgroup

%%%%%%%%%%%%%%%%%%%%%%%%%%%%%%%%%%%%%%%%%%%%%%%%%%%%%%%%%%%%%%%%%%%%%%%%%%%%%%%%
%%%%%%%%%%%%%%%%%%%%%%%%%%%%%%%%%%%%%%%%%%%%%%%%%%%%%%%%%%%%%%%%%%%%%%%%%%%%%%%%
\section{Introduction}

\LaTeX{} provides a mechanism to structure a large document (such as a book)
into a main file and several child files (containing the chapters)
using the |\include| command.
This mechanism is beneficial for documents
which span hundreds of pages in order to
make the source file(s) more manageable.
Moreover, compilation can be restricted to
selected child files by means of the |\includeonly| command.
The latter feature can be used to reduce the compilation time while editing
(this was significantly more useful in the earlier days of \LaTeX{})
or to generate a smaller document which is easier to navigate.
Another application of |\includeonly| is to generate
documents consisting of selected parts of the complete document.

However, there are a few drawbacks of the plain |\include| mechanism:
\begin{itemize}
\item
The child files cannot be compiled on their own,
they can only be compiled via the main file.
A naive editing environment
(such as a text editor with an option
to have the current file processed by \LaTeX)
may require one to switch to the main file before compiling;
attempting to compile the child file produces errors.
\item
The main file must be modified (each time)
to adjust the |\includeonly| command
to the present needs. This easily leaves the main file in a messy state.
\item
The generated document will always carry the filename
of the main document. This is inconvenient if
several child files are to be compiled and
to be kept for distribution.
\end{itemize}

The present package provides a simple interface
to make child files individually compilable by \LaTeX{}.
Compiling a child file then has the same effect as compiling
the main file with an |\includeonly| command
to select the appropriate child.
Moreover the generated document will carry the name of the child
rather than the main file.
This resolves all three above issues.

This feature is meant to make the editing of books,
thesis documents and lecture notes somewhat more convenient.
However, the package can also be used efficiently for
composing a series of documents (such as exercise sheets)
which are typically distributed individually.
It then assists the author in generating the individual documents
(potentially in different versions)
as well as a document containing the collected series.
Another application is in developing style files
or other kinds of included material
where compilation of the style file could redirect
to a sample or test file.

%%%%%%%%%%%%%%%%%%%%%%%%%%%%%%%%%%%%%%%%%%%%%%%%%%%%%%%%%%%%%%%%%%%%%%%%%%%%%%%%
%%%%%%%%%%%%%%%%%%%%%%%%%%%%%%%%%%%%%%%%%%%%%%%%%%%%%%%%%%%%%%%%%%%%%%%%%%%%%%%%
\section{Usage}

First of all, the package \textsf{childdoc} is \emph{not} a standard
\LaTeXe{} |.sty| style file! Therefore it needs to be invoked in
a non-standard way.

%%%%%%%%%%%%%%%%%%%%%%%%%%%%%%%%%%%%%%%%%%%%%%%%%%%%%%%%%%%%%%%%%%%%%%%%%%%%%%%%
\subsection{Included Files}
\label{sec:include}

%%%%%%%%%%%%%%%%%%%%%%%%%%%%%%%%%%%%%%%%
\DescribeMacro{\childdocmain}
To use the package, add the commands
\begin{center}
\begin{tabular}{l}
|\input{childdoc.def}|\\
|\childdocmain{}|\\
\end{tabular}
\end{center}
at the very top of the main \LaTeX{} file,
in particular \emph{before} the |\documentclass| statement!
The argument of |\childdocmain| should be left empty
(but it must be present).

%%%%%%%%%%%%%%%%%%%%%%%%%%%%%%%%%%%%%%%%
\DescribeMacro{\childdocof}
Furthermore, add the commands
\begin{center}
\begin{tabular}{l}
|\input{childdoc.def}|\\
|\childdocof{|\textit{main}|}|\\
\end{tabular}
\end{center}
at the top of every child file \textit{child}
which is included by |\include{|\textit{child}|}|
from within the main file
(or at least for those files to be compiled individually).
The argument \textit{main} must be the filename of the main file.

There are a couple of
considerations in setting up the main and child documents:

%%%%%%%%%%%%%%%%%%%%%%%%%%%%%%%%%%%%%%%%
\paragraph{Restrictions.}

Please note the following restrictions:
\begin{itemize}
\item
|\childdocmain| must be called with one argument \textit{main}
to ensure compatibility with earlier version of the package.
It must either be empty (|\childdocmain{}|)
or precisely match the filename of the main file in which it is specified.
See \secref{sec:detection} for further information.
\item
The filename \textit{main} must be specified without the |.tex| extension.
\item
The filename \textit{main} is case sensitive
(even in case-insensitive file systems)
due to internal string comparison.
\item
The argument \textit{main} should be fully expanded, it cannot be a macro.
\item
Subdirectories and special characters should be avoided in filenames.
\item
The command |\childdocmain{|\textit{main}|}| must be followed by a whitespace.
It should not be followed immediately by another command
or by a comment mark `|%|'.
This is because the \TeX{} parser reads the token immediately following
the argument of |\childdocmain| and puts it
at the beginning of every child section;
however, a white\-space is ignored.
\end{itemize}

%%%%%%%%%%%%%%%%%%%%%%%%%%%%%%%%%%%%%%%%
\paragraph{Content of Main File.}

It is advisable to place all content in the child files included by |\include|.
Any output contained in the main file will appear in all child documents
unless suppressed manually;
it cannot be suppressed automatically by the |\includeonly| directive
and thus should normally be avoided.
A method to include some content in the main file
by means of conditional processing is described in \secref{sec:conditional}.

%%%%%%%%%%%%%%%%%%%%%%%%%%%%%%%%%%%%%%%%
\paragraph{Page Numbering.}

When only a part of the document is compiled,
the appropriate numbering of pages
(as well as other status parameters)
is determined from the |.aux| files.
The latter contain information from previous passes.
However this information needs to propagate through
all intermediate child documents.
Therefore the page numbering in child documents may well
be inconsistent until the complete document is compiled at least once.

A useful (if unconventional) way to always ensure a consistent
page numbering is to restart the numbering in each child document
and denote the pages by `\textit{child}|.|\textit{page}'
where \textit{child} represents the chapter/section number of the child file.
This can be achieved by the command
|\numberwithin{page}{|\textit{child}|}|
of the \textsf{amsmath} package
where \textit{child} can be |chapter| or |section|
depending on the chosen structuring.
Alternatively, one can modify the macro |\thepage| appropriately
and reset the counter |page| at the start of each child file.

%%%%%%%%%%%%%%%%%%%%%%%%%%%%%%%%%%%%%%%%%%%%%%%%%%%%%%%%%%%%%%%%%%%%%%%%%%%%%%%%
\subsection{Conditional Processing}
\label{sec:conditional}

The package provides a mechanism to compile different versions
of a document. To customise the versions further some conditional processing
can come in handy to distinguish which version is being compiled.
The package provides two macros to describe the compilation context:

%%%%%%%%%%%%%%%%%%%%%%%%%%%%%%%%%%%%%%%%
\DescribeMacro{\ifchilddoc}
The conditional |\ifchilddoc| distinguishes between the compilation of
child documents and the main document:
%
\begin{center}
|\ifchilddoc |\textit{child-code}| |[|\||else |\textit{main-code}]| \||fi|
\end{center}

%%%%%%%%%%%%%%%%%%%%%%%%%%%%%%%%%%%%%%%%
\DescribeMacro{\childdocname}
\DescribeMacro{\childdocjob}
The macro |\childdocname| contains the filename (without extension)
of the main or child file being processed.
Note that |\childdocjob| will always contain the name of the main file.

%%%%%%%%%%%%%%%%%%%%%%%%%%%%%%%%%%%%%%%%
\paragraph{Title Page.}

Conditional processing can be used to include a title or banner page
in the main document when proper precautions are taken.
Importantly, the code in the main file should ensure that the page counter
(as well as other status parameters which are stored in the |.aux| files)
takes the same value after the conditional processing.
Otherwise the page numbers may take divergent values
depending on which part is compiled.

For example, a title page could be declared by:
%
\begin{center}
\begin{tabular}{l}
|\ifchilddoc\||else|\\
|\addtocounter{page}{-1}|\\
\textit{code for title page}\\
|\newpage|\\
|\||fi|
\end{tabular}
\end{center}
%
A banner page for the child documents can be generated by:
%
\begin{center}
\begin{tabular}{l}
|\ifchilddoc|\\
|\addtocounter{page}{-1}|\\
\textit{code for banner page}\\
|\newpage|\\
|\||fi|
\end{tabular}
\end{center}
%
Here one could write a message such as:
\begin{center}
|This is the part \childdocname{} of \childdocjob{}.|
\end{center}

%%%%%%%%%%%%%%%%%%%%%%%%%%%%%%%%%%%%%%%%%%%%%%%%%%%%%%%%%%%%%%%%%%%%%%%%%%%%%%%%
\subsection{Flags}
\label{sec:flags}

The package makes it easy to generate different versions
of the main or child documents.
To this end compilation flags can be defined
and assigned different default values.
They will be particularly useful in conjunction
with the forwarding mechanism described in \secref{sec:forward}.

For example, it may be useful to have a flag |\version|
which can be set to |draft| or |final|.
The document source will contain some conditional code
depending on the value of |\version|.
Suppose further, the flag should default to |final| for the main file
and to |draft| for child files
which is a natural assignment for editing the document.
This is achieved by placing the following code
in the preamble of the main document
(below the |\childdocmain| directive):
%
\begin{center}
\begin{tabular}{l}
|\ifchilddoc|\\
|\providecommand{\version}{draft}|\\
|\||else|\\
|\providecommand{\version}{final}|\\
|\||fi|
\end{tabular}
\end{center}
%
The definition by |\providecommand| makes sure
that previous definitions are not overwritten.
Further statements |\providecommand{\version}{...}|
can thus be added before the above code to override it.

For the main file, one might add a line
(between |\childdocmain| and the above block)
%
\begin{center}
|%\ifchilddoc\||else\providecommand{\version}{draft}\||fi|
\end{center}
%
which can be uncommented to produce a draft version.
Likewise one can add a line to the very top of a child file
(above the |\childdocof{|\textit{main}|}| directive)
%
\begin{center}
|%\providecommand{\version}{final}|
\end{center}
%
which can be uncommented to produce the final version of this child document.

%%%%%%%%%%%%%%%%%%%%%%%%%%%%%%%%%%%%%%%%%%%%%%%%%%%%%%%%%%%%%%%%%%%%%%%%%%%%%%%%
\subsection{Forwarding}
\label{sec:forward}

Different versions of the main or child documents
using compilation flags as described in \secref{sec:flags}
can be (permanently) stored in different files
for convenient compilation, viewing and distribution.
To this end, the package defines a command
to pass on compilation to a different file:

%%%%%%%%%%%%%%%%%%%%%%%%%%%%%%%%%%%%%%%%
\DescribeMacro{\childdocforward}
The command |\childdocforward| redirects processing to
another source file:
%
\begin{center}
\begin{tabular}{l}
|\input{childdoc.def}|\\
|\childdocforward[|\textit{main}|]{|\textit{dest}|}|\\
\end{tabular}
\end{center}
%
The argument \textit{dest} is the destination file
(without extension).
It should be the main file or one of the child files.
Note that further \textsf{childdoc} directives
such as |\childdocof| and |\childdocforward|
in the indicated file will be processed in this form.
The optional argument \textit{main}
passes on directly to the main file \textit{main}
while pretending to compile the child \textit{dest}.
This form behaves as if \textit{dest}
issues |\childdocof{|\textit{main}|}| right away,
and no further \textsf{childdoc} directives will be processed.

%%%%%%%%%%%%%%%%%%%%%%%%%%%%%%%%%%%%%%%%
\DescribeMacro{\...prefix}
In the alternative form |\childdocforwardprefix|,
%
\begin{center}
\begin{tabular}{l}
|\input{childdoc.def}|\\
|\childdocforwardprefix[|\textit{main}|]{|\textit{prefix}|}{|\textit{dest}|}|
\end{tabular}
\end{center}
%
the destination file is determined by a pattern
depending on the current file:
To make this work, the current file must be called
`{\textit{prefix}\hspace{0.2em}\textit{suffix}}'
with \textit{prefix} matching precisely the argument.
Processing is then passed on to the file
`{\textit{dest}\hspace{0.2em}\textit{suffix}}'.
Surely, the same effect is achieved by
directly specifying the
argument `{\textit{dest}\hspace{0.2em}\textit{suffix}}'
in the first form.
However, that requires to set up a different file
for each child. With the alternative form of the command
all these files can have exactly the same content
which simplifies setting them up and maintaining them.

For example, the following file |draft.tex|
with a compilation flag |\version| as described in \secref{sec:flags}
compiles the main document as a draft:
%
\begin{center}
\begin{tabular}{l}
|\def\version{draft}|\\
|\input{childdoc.def}|\\
|\childdocforward{|\textit{main}|}|
\end{tabular}
\end{center}
%
Likewise, the following files |final|\textit{nn}|.tex|
compile the final version of the child document
|child|\textit{nn}|.tex|:
%
\begin{center}
\begin{tabular}{l}
|\def\version{final}|\\
|\input{childdoc.def}|\\
|\childdocforwardprefix{final}{child}|
\end{tabular}
\end{center}
%

Note that when several versions of a main file and/or of each child file
are to be generated, it may be convenient to set up a |Makefile| or
shell script to automatise the process.

%%%%%%%%%%%%%%%%%%%%%%%%%%%%%%%%%%%%%%%%%%%%%%%%%%%%%%%%%%%%%%%%%%%%%%%%%%%%%%%%
\subsection{Command Line Processing}
\label{sec:commandline}

The effect of redirection files can also be achieved by invoking
the \LaTeX{} compiler with a more elaborate command line.
Most conveniently this should be done as part
of a shell script or a |Makefile|.

When using \textsf{childdoc} in the main file, the following
command lines effectively perform a redirection
(note that depending on the shell being used,
backslashes may have to be doubled: `|\|' $\to$ `|\\|'):
%
\begin{center}
|... -jobname "|\textit{target}|" |\\|"|[\textit{flags}]%
|\input{childdoc.def}\childdocforward[|\textit{main}|]{|\textit{dest}|}"|
\end{center}
%
Here \textit{target} is the name of the output file,
\textit{main} is the name of the main file
and \textit{dest} is the name of the main or child file to be processed
(all filenames without extensions).
The optional argument \textit{main} can be omitted
if \textit{main} matches \textit{dest}.
Optionally, compilation \textit{flags} can be defined via |\def| commands.
This command line makes the \TeX{} engine believe
it is compiling the file \textit{target}
whose content is specified as the latter parameter.
The provided code then forwards the processing to
\textit{main} or \textit{dest} as described in \secref{sec:forward}.

%%%%%%%%%%%%%%%%%%%%%%%%%%%%%%%%%%%%%%%%%%%%%%%%%%%%%%%%%%%%%%%%%%%%%%%%%%%%%%%%
\subsection{Include by Input}
\label{sec:input}

Including child documents by |\include| has some restrictions by design.
Most notably, the content of a child document always occupies
its own set of pages; pages cannot be shared between child documents.
Usually, this behaviour makes perfect sense
because each child document contain an essential part of the document.
However, in some situations it may be desirable to compose
a document from a collection of parts
without having mandatory page breaks between then.
For this case, the package
provides a mechanism to include parts
by |\input| which can also be processed individually.
However, by construction this mechanism
requires manual handling of the content to be output.

%%%%%%%%%%%%%%%%%%%%%%%%%%%%%%%%%%%%%%%%
\DescribeMacro{\ifchilddocmanual}
The main file should be prepared as usual, see \secref{sec:include}.
However, the document body must make a distinction
between processing of an individual part and of the main document, e.g.:
%
\begin{center}
\begin{tabular}{l}
|\ifchilddocmanual|\\
|\input{\childdocname}|\\
|\||else|\\
\textit{document body with }|\input{|\textit{part}|}|\\
|\||fi|
\end{tabular}
\end{center}
%
The conditional |\ifchilddocmanual| is true whenever
a part to be included by |\input| is being compiled,
and the name of the part is stored in |\childdocname|.

%%%%%%%%%%%%%%%%%%%%%%%%%%%%%%%%%%%%%%%%
\DescribeMacro{\childdocby}
Each part to be included by |\input| should start with:
%
\begin{center}
\begin{tabular}{l}
|\input{childdoc.def}|\\
|\childdocby{|\textit{main}|}|\\
\end{tabular}
\end{center}
%
The directive |\childdocby| is similar to |\childdocof|
described in \secref{sec:include},
but the subsequent selection of content must be done manually.
To that end, both |\ifchilddoc| and |\ifchilddocmanual|
will be true upon processing of a part,
and the name of the part is stored in |\childdocname|.
Note that |\jobname| will be set to the filename of the current part
so that each part receives an individual |.aux| file
that does not interfere with the |.aux| file(s) of the main document.
This behaviour can be altered by the alternative form
|\childdocby[*]{|\textit{main}|}| (with a non-empty optional argument)
which uses the |.aux| file of the main document
by setting |\jobname| to \textit{main}.

%%%%%%%%%%%%%%%%%%%%%%%%%%%%%%%%%%%%%%%%%%%%%%%%%%%%%%%%%%%%%%%%%%%%%%%%%%%%%%%%
\subsection{Driver Development}
\label{sec:driver}

The \textsf{childdoc} mechanism can also be use for the development
of definition files such as \LaTeX{} styles or classes.
This case differs from the above setup with multiple parts
included by |\include| in that no |\includeonly| should be invoked.
This can be achieved by starting the include file
(before |\ProvidesPackage|) with:
%
\begin{center}
\begin{tabular}{l}
|\input{childdoc.def}|\\
|\childdocforward{|\textit{main}|}|\\
\end{tabular}
\end{center}
%
or alternatively with:
%
\begin{center}
\begin{tabular}{l}
|\input{childdoc.def}|\\
|\childdocby{|\textit{main}|}|\\
\end{tabular}
\end{center}
%
Both forms have slightly different effects as described above.
The main file is prepared as usual, see \secref{sec:include}.

%%%%%%%%%%%%%%%%%%%%%%%%%%%%%%%%%%%%%%%%%%%%%%%%%%%%%%%%%%%%%%%%%%%%%%%%%%%%%%%%
\subsection{Legacy Detection}
\label{sec:detection}

The directive |\childdocmain| in the main file can detect
whether the complete document or merely a child is to be compiled
even without using the directive |\childdocof|.
This method is deprecated because it is less robust
and there is no compelling reason to use it;
it is merely provided for backward compatibility
and it may be removed in future versions.

If the detection mechanism is to be used,
it is mandatory to correctly specify
the filename of the main file as the argument of |\childdocmain|:
%
\begin{center}
\begin{tabular}{l}
|\input{childdoc.def}|\\
|\childdocmain{|\textit{main}|}|\\
\end{tabular}
\end{center}
%
If |\jobname| does not match the argument \textit{main} of |\childdocmain|,
it is assumed that |\jobname| points to the child file to be compiled.
When using |\childdocmain| with the main file specified as argument,
it suffices to start a child file
with just |\input{|\textit{main}|}|
without loading of the package and using |\childdocof|.
If instead all processing is done
with the appropriate \textsf{childdoc} directives,
the argument of \textit{main} of |\childdocmain| can be empty.

An alternative version of the command line processing described
in \secref{sec:commandline} using the detection mechanism reads:
%
\begin{center}
|... -jobname "|\textit{target}|" "|[\textit{flags}]%
[|\def\jobname{|\textit{dest}|}|]|\input{|\textit{main}|}"|
\end{center}

%%%%%%%%%%%%%%%%%%%%%%%%%%%%%%%%%%%%%%%%%%%%%%%%%%%%%%%%%%%%%%%%%%%%%%%%%%%%%%%%
\subsection{Manual Code}
\label{sec:manual}

In case one cannot be certain whether the definitions file |childdoc.def|
is installed on the target \TeX{} distribution
and one prefers not to ship it,
it is conceivable to paste a few relevant commands into the sources.

To that end, drop all statements |\input{childdoc.def}|
and perform the replacements as outlined below.
Instead of |\childdocmain{|\textit{main}|}| add the following code
to the top of the main file:
%
\begin{center}
\begin{tabular}{l}
|\||ifdefined\childdocname\endinput\||fi\newif\ifchilddoc|\\
|\edef\childdocname{\scantokens\expandafter{\jobname\noexpand}}|\\
|\def\childdocmain{|\textit{main}|}\||ifx\childdocmain\childdocname\||else|\\
|\childdoctrue\includeonly{\childdocname}\let\jobname\childdocmain\||fi|\\
\end{tabular}
\end{center}
%
Instead of |\childdocof{|\textit{main}|}| just include the main file
at the top of each child file:
%
\begin{center}
|\input{|\textit{main}|}|
\end{center}
%
A simple redirection |\childdocforward{|\textit{dest}|}| is achieved by:
%
\begin{center}
|\def\jobname{|\textit{dest}|}\input{\jobname}|
\end{center}
%
The redirection with prefix
|\childdocforwardprefix[|\textit{prefix}|]{|\textit{dest}|}|
is accomplished by:
%
\begin{center}
\begin{tabular}{l}
|{\edef\jobname{\scantokens\expandafter{\jobname\noexpand}}|\\
|\def\redirectjob |\textit{prefix}|#1~~~{\gdef\jobname{|\textit{dest}|#1}}|\\
|\expandafter\redirectjob\jobname~~~}\input{\jobname}|
\end{tabular}
\end{center}

In an alternative approach,
child documents can be compiled by a specific command line
without additional code or specific definitions:
%
\begin{center}
|... -jobname "|\textit{target}|" "|[\textit{flags}]%
|\includeonly{|\textit{dest}|}\input{|\textit{main}|}"|
\end{center}
%

%%%%%%%%%%%%%%%%%%%%%%%%%%%%%%%%%%%%%%%%%%%%%%%%%%%%%%%%%%%%%%%%%%%%%%%%%%%%%%%%
%%%%%%%%%%%%%%%%%%%%%%%%%%%%%%%%%%%%%%%%%%%%%%%%%%%%%%%%%%%%%%%%%%%%%%%%%%%%%%%%
\section{Information}

%%%%%%%%%%%%%%%%%%%%%%%%%%%%%%%%%%%%%%%%%%%%%%%%%%%%%%%%%%%%%%%%%%%%%%%%%%%%%%%%
\subsection{Copyright}

Copyright \copyright{} 2017--2018 Niklas Beisert

This work may be distributed and/or modified under the
conditions of the \LaTeX{} Project Public License, either version 1.3
of this license or (at your option) any later version.
The latest version of this license is in
  \url{http://www.latex-project.org/lppl.txt}
and version 1.3 or later is part of all distributions of \LaTeX{}
version 2005/12/01 or later.

This work has the LPPL maintenance status `maintained'.

The Current Maintainer of this work is Niklas Beisert.

This work consists of the files |README.txt|, |childdoc.ins| and |childdoc.dtx|
as well as the derived files |childdoc.def|, |cdocsamp.tex|
with |cdocsch1.tex|, |cdocsch2.tex|, |cdocspt3.tex|, |cdocspt4.tex|,
|cdocsdrf.tex|, |cdocsfn1.tex|, |cdocsfn2.tex|
as well as |childdoc.pdf|.

%%%%%%%%%%%%%%%%%%%%%%%%%%%%%%%%%%%%%%%%%%%%%%%%%%%%%%%%%%%%%%%%%%%%%%%%%%%%%%%%
\subsection{Files and Installation}

The package consists of the files:
%
\begin{center}
\begin{tabular}{ll}
    |README.txt|   & readme file \\
    |childdoc.ins| & installation file \\
    |childdoc.dtx| & source file \\
    |childdoc.def| & definition file \\
    |cdocsamp.tex| & sample main file \\
    |cdocsch1.tex| & sample include file \\
    |cdocsch2.tex| & sample include file \\
    |cdocspt3.tex| & sample part file \\
    |cdocspt4.tex| & sample part file \\
    |cdocsdrf.tex| & sample redirection file \\
    |cdocsfn1.tex| & sample redirection file \\
    |cdocsfn2.tex| & sample redirection file \\
    |childdoc.pdf| & manual
\end{tabular}
\end{center}
%
The distribution consists of the files
|README.txt|, |childdoc.ins| and |childdoc.dtx|.
%
\begin{itemize}
\item
Run (pdf)\LaTeX{} on |childdoc.dtx|
to compile the manual |childdoc.pdf| (this file).
\item
Run \LaTeX{} on |childdoc.ins| to create the definitions file |childdoc.def|
and the sample |cdocsamp.tex| with include files
|cdocsch1.tex|, |cdocsch2.tex|, |cdocspt3.tex|, |cdocspt4.tex|,
|cdocsdrf.tex|, |cdocsfn1.tex|, |cdocsfn2.tex|.
Then copy the file |childdoc.def| to an appropriate directory of your \LaTeX{}
distribution, e.g.\ \textit{texmf-root}|/tex/latex/childdoc|.
\end{itemize}

%%%%%%%%%%%%%%%%%%%%%%%%%%%%%%%%%%%%%%%%%%%%%%%%%%%%%%%%%%%%%%%%%%%%%%%%%%%%%%%%
\subsection{Related CTAN Packages}

There are several other packages which offer a similar functionality:
%
\begin{itemize}
\item
The packages
\href{http://ctan.org/pkg/docmute}{\textsf{docmute}},
\href{http://ctan.org/pkg/includex}{\textsf{includex}} and
\href{http://ctan.org/pkg/standalone}{\textsf{standalone}}
provide commands to include only the document body of
a child file thus allowing both files to be compiled individually.
\item
The packages \href{http://ctan.org/pkg/subdocs}{\textsf{subdocs}}
and \href{http://ctan.org/pkg/subfiles}{\textsf{subfiles}}
provide structures in which the main and child documents can be
encapsulated and allowing them to be compiled individually.
The inclusion mechanism is different from the conventional |\include|.
\item
The package \href{http://ctan.org/pkg/combine}{\textsf{combine}}
is an elaborate solution to combine several documents into one.
\end{itemize}
%
See also the CTAN topic \href{http://ctan.org/topic/subdocs}{\textsf{subdocs}}
for further related packages.
The present package differs from the above solutions in that
a document structure constructed with the conventional |\include| mechanism
just needs two extra commands at the top of every file
such that all constituent files can be compiled individually.

%%%%%%%%%%%%%%%%%%%%%%%%%%%%%%%%%%%%%%%%%%%%%%%%%%%%%%%%%%%%%%%%%%%%%%%%%%%%%%%%
%\subsection{Feature Suggestions}
%
%The following is a list of features which may be useful for future
%versions of this package:
%%
%\begin{itemize}
%\item
%\ldots
%\end{itemize}

%%%%%%%%%%%%%%%%%%%%%%%%%%%%%%%%%%%%%%%%%%%%%%%%%%%%%%%%%%%%%%%%%%%%%%%%%%%%%%%%
\subsection{Revision History}

%%%%%%%%%%%%%%%%%%%%%%%%%%%%%%%%%%%%%%%%
\paragraph{v2.0:} 2018/12/30

\begin{itemize}
\item
immediate forward processing
\item
added |\childdocby| mechanism
\item
manual restructured
\end{itemize}

%%%%%%%%%%%%%%%%%%%%%%%%%%%%%%%%%%%%%%%%
\paragraph{v1.6:} 2018/01/17

\begin{itemize}
\item
application for development of include files
\item
corrections to manual
\end{itemize}

%%%%%%%%%%%%%%%%%%%%%%%%%%%%%%%%%%%%%%%%
\paragraph{v1.5:} 2017/05/21

\begin{itemize}
\item
more complete structuring introduced
\item
|\childdocof| introduced
\item
|\childdoc| renamed to |\childdocmain|
\item
|\childredirect| renamed to |\childdocforward| and |\childdocforwardprefix|
and functionality expanded
\end{itemize}

%%%%%%%%%%%%%%%%%%%%%%%%%%%%%%%%%%%%%%%%
\paragraph{v1.0:} 2017/04/27

\begin{itemize}
\item
manual and install package
\item
first version published on CTAN
\end{itemize}

%%%%%%%%%%%%%%%%%%%%%%%%%%%%%%%%%%%%%%%%
\paragraph{v0.6:} 2017/04/26

\begin{itemize}
\item
redirection mechanism added
\end{itemize}

%%%%%%%%%%%%%%%%%%%%%%%%%%%%%%%%%%%%%%%%
\paragraph{v0.5:} 2017/04/26

\begin{itemize}
\item
functionality in definition file
\end{itemize}


%%%%%%%%%%%%%%%%%%%%%%%%%%%%%%%%%%%%%%%%%%%%%%%%%%%%%%%%%%%%%%%%%%%%%%%%%%%%%%%%
%%%%%%%%%%%%%%%%%%%%%%%%%%%%%%%%%%%%%%%%%%%%%%%%%%%%%%%%%%%%%%%%%%%%%%%%%%%%%%%%
%%%%%%%%%%%%%%%%%%%%%%%%%%%%%%%%%%%%%%%%%%%%%%%%%%%%%%%%%%%%%%%%%%%%%%%%%%%%%%%%
\appendix

\settowidth\MacroIndent{\rmfamily\scriptsize 000\ }

 \DocInput{childdoc.dtx}

\end{document}
%</driver>
% \fi
%
% %%%%%%%%%%%%%%%%%%%%%%%%%%%%%%%%%%%%%%%%%%%%%%%%%%%%%%%%%%%%%%%%%%%%%%%%%%%%%%
% %%%%%%%%%%%%%%%%%%%%%%%%%%%%%%%%%%%%%%%%%%%%%%%%%%%%%%%%%%%%%%%%%%%%%%%%%%%%%%
% \section{Sample}
%\iffalse
%<*samplemain>
%\fi
%
% The following presents a sample document
% with two chapters, two parts, a title page,
% a compile flag as well as three forwarding files to set the flag.
% It consists of eight |.tex| files:
% \begin{center}
% \begin{tabular}{ll}
% |cdocsamp.tex|&main file\\
% |cdocsch1.tex|&include file for chapter 1\\
% |cdocsch2.tex|&include file for chapter 2\\
% |cdocspt3.tex|&include file for part 3\\
% |cdocspt4.tex|&include file for part 4\\
% |cdocsdrf.tex|&forwarding file for main file in draft mode\\
% |cdocsfi1.tex|&forwarding file for final version of chapter 1\\
% |cdocsfi2.tex|&forwarding file for final version of chapter 2\\
% \end{tabular}
% \end{center}
% Each of the eight files can be compiled directly by the \LaTeX{} compiler.
%
% %%%%%%%%%%%%%%%%%%%%%%%%%%%%%%%%%%%%%%
% \paragraph{Main File.}
%
% The main file is called |cdocsamp.tex|.
%
% Load the \textsf{childdoc} definitions and
% declare the filename for the main document:
%    \begin{macrocode}
\input{childdoc.def}
\childdocmain{}
%    \end{macrocode}

% Optional override for |\version| flag:
%    \begin{macrocode}
%%\ifchilddoc\else\providecommand{\version}{draft}\fi
%    \end{macrocode}

% Define the default values for the |\version| flag
% (|final| for the main file and |draft| for childs):
%    \begin{macrocode}
\ifchilddoc
\providecommand{\version}{draft}
\else
\providecommand{\version}{final}
\fi
%    \end{macrocode}

% Load the standard document class:
%    \begin{macrocode}
\documentclass[12pt]{article}
%    \end{macrocode}

% Start the document body:
%    \begin{macrocode}
\begin{document}
%    \end{macrocode}

% Declare a title page.
% Print title, part of document being processed and version flag:
%    \begin{macrocode}
\addtocounter{page}{-1}
\begin{center}
{\LARGE\bfseries{}childdoc example\par}
\vspace{1cm}
\ifchilddoc
\ifchilddocmanual part\else chapter\fi:
`\childdocname' of `\childdocjob'\par
\else
main document: `\childdocjob'\par
\fi
version: \version\par
\end{center}
\newpage
%    \end{macrocode}

% Manually include selected file,
% otherwise process as usual:
%    \begin{macrocode}
\ifchilddocmanual
\section*{part `\childdocname'}
\input{\childdocname}
\else
%    \end{macrocode}

% Include the two chapters:
%    \begin{macrocode}
\include{cdocsch1}
\include{cdocsch2}
%    \end{macrocode}

% Include the two parts unless only chapters should be displayed:
%    \begin{macrocode}
\ifchilddoc\else
\section{part three}
\input{cdocspt3}
\section{part four}
\input{cdocspt4}
\fi
%    \end{macrocode}

% Process as usual until here:
%    \begin{macrocode}
\fi
%    \end{macrocode}

% End of document body:
%    \begin{macrocode}
\end{document}
%    \end{macrocode}
%\iffalse
%</samplemain>
%\fi
%
% %%%%%%%%%%%%%%%%%%%%%%%%%%%%%%%%%%%%%%
% \paragraph{Chapter Include Files.}
%
% The include files are called |cdocsch1.tex| and |cdocsch2.tex|.
%
%\iffalse
%<*samplechap1|samplechap2>
%\fi

% Optional override for |\version| flag:
%    \begin{macrocode}
%%\providecommand{\version}{final}
%    \end{macrocode}

% Include the main document:
%    \begin{macrocode}
\input{childdoc.def}
\childdocof{cdocsamp}
%    \end{macrocode}

%\iffalse
%</samplechap1|samplechap2>
%\fi
%
%\iffalse
%<*samplechap1>
%\fi
% Some text for chapter 1:
%    \begin{macrocode}
\section{one}
some text in chapter one
%    \end{macrocode}

%\iffalse
%</samplechap1>
%\fi
% Some text for chapter 2:
%\iffalse
%<*samplechap2>
%\fi
%    \begin{macrocode}
\section{two}
more text in chapter two
%    \end{macrocode}

%\iffalse
%</samplechap2>
%\fi
%
% %%%%%%%%%%%%%%%%%%%%%%%%%%%%%%%%%%%%%%
% \paragraph{Part Include Files.}
%
% The include files are called |cdocspt3.tex| and |cdocspt4.tex|.
%
%\iffalse
%<*samplepart3|samplepart4>
%\fi

% Optional override for |\version| flag:
%    \begin{macrocode}
%%\providecommand{\version}{final}
%    \end{macrocode}

% Include the main document:
%    \begin{macrocode}
\input{childdoc.def}
\childdocby{cdocsamp}
%    \end{macrocode}

%\iffalse
%</samplepart3|samplepart4>
%\fi
%
%\iffalse
%<*samplepart3>
%\fi
% Some text for part 3:
%    \begin{macrocode}
some text in part three
%    \end{macrocode}

%\iffalse
%</samplepart3>
%\fi
% Some text for part 4:
%\iffalse
%<*samplepart4>
%\fi
%    \begin{macrocode}
more text in part four
%    \end{macrocode}

%\iffalse
%</samplepart4>
%\fi
%
% %%%%%%%%%%%%%%%%%%%%%%%%%%%%%%%%%%%%%%
% \paragraph{Forwarding for a Complete Draft.}
%
% The following forwarding file |cdocsdrf.tex|
% compiles the main document in draft mode:
%\iffalse
%<*sampledraft>
%\fi
%    \begin{macrocode}
\def\version{draft}
\input{childdoc.def}
\childdocforward{cdocsamp}
%    \end{macrocode}

%\iffalse
%</sampledraft>
%\fi
%
% %%%%%%%%%%%%%%%%%%%%%%%%%%%%%%%%%%%%%%
% \paragraph{Forwarding for Final Version of the Chapters.}
%
% The following forwarding files |cdocsfn1.tex| and |cdocsfn2.tex|
% (with identical content)
% compile the final versions of the child documents
% |cdocsch1.tex| and |cdocsch2.tex|, respectively:
%\iffalse
%<*samplefinal>
%\fi
%    \begin{macrocode}
\def\version{final}
\input{childdoc.def}
\childdocforwardprefix[cdocsamp]{cdocsfn}{cdocsch}
%    \end{macrocode}

%\iffalse
%</samplefinal>
%\fi
%
% %%%%%%%%%%%%%%%%%%%%%%%%%%%%%%%%%%%%%%
% \paragraph{Command Line Processing.}
%
% The following three command lines generate the output files
% |cdocscld|, |cdocscl1| and |cdocscl2|
% which should be identical to
% |cdocsdrf|, |cdocsch1| and |cdocsfn2|, respectively:
% \begin{center}
% \begin{tabular}{l}
% |latex -jobname cdocscld \|\\
% |  "\def\version{draft}\input{childdoc.def}\childdocforward{cdocsamp}"|\\
% |latex -jobname cdocscl1 \|\\
% |  "\input{childdoc.def}\childdocforward[cdocsamp]{cdocsch1}"|\\
% |latex -jobname cdocscl2 \|\\
% |  "\def\version{final}\input{childdoc.def}\childdocforward{cdocsch2}"|
% \end{tabular}
% \end{center}
% Note that the trailing backslash on each first line
% merely continues the input to the second line
% (for convenient cut ant paste).
% Furthermore, the command |latex| can be replaced by any
% of its alternative versions such as |pdflatex|.
%
% %%%%%%%%%%%%%%%%%%%%%%%%%%%%%%%%%%%%%%%%%%%%%%%%%%%%%%%%%%%%%%%%%%%%%%%%%%%%%%
% %%%%%%%%%%%%%%%%%%%%%%%%%%%%%%%%%%%%%%%%%%%%%%%%%%%%%%%%%%%%%%%%%%%%%%%%%%%%%%
% \section{Implementation}
%\iffalse
%<*package>
%\fi
%
% This section describes the definitions file |childdoc.def|.

% The definitions cannot be loaded using |\usepackage| or |\RequirePackage|
% which has a mechanism to prevent loading a style file more than once.
% When loading the definitions by means of |\input|
% multiple instances have to be prevented manually:
%\iffalse
%This code needs to be before the `\ProvidesFile' directive
%which is defined at the beginning of this file.
%Therefore it is also placed there and commented out here.
%</package>
%<*discard>
%\fi
%    \begin{macrocode}
\ifdefined\childdocmain\endinput\fi
%    \end{macrocode}
%\iffalse
%</discard>
%<*package>
%\fi
%
% \macro{\ifchilddoc}
% \macro{\ifchilddocmanual}
% The conditional |\ifchilddoc| tells whether a
% child (true) or main (false) document is being compiled.
% The conditional |\ifchilddocmanual| tells whether
% the |\includeonly| mechanism is used (false) or
% the selection of child files must be performed manually (true).
% The definitions initialise to false:
%    \begin{macrocode}
\newif\ifchilddoc
\newif\ifchilddocmanual
%    \end{macrocode}

% \macro{\childdocname}
% \macro{\childdocjob}
% The macro |\childdocname| stores the name of the main document
% to be compiled. The macro |\childdocjob| stores the name of
% the document on which the \LaTeX{} compiler was originally invoked.
% The content of |\jobname| cannot be compared
% to filenames specified in the source due to different catcodes.
% The following code rescans |\jobname|, stores the result
% in |\childdocname| and saves a copy in |\childdocjob|:
%    \begin{macrocode}
\edef\childdocname{\scantokens\expandafter{\jobname\noexpand}}
\let\childdocjob\childdocname
%    \end{macrocode}

% \macro{\childdocdisable}
% The macro |\childdocdisable| prevents the main file
% from being processed more than once.
% At this stage, the main document command |\childdocmain|
% is assumed to be called once again where it should do nothing.
% Any subsequent call to it should prevent
% a secondary processing of the main document
% It overwrites the forwarding commands
% |\childdocof| and |\childdocforward|
% with empty macros to prevent further inclusions of the main document:
%    \begin{macrocode}
\newcommand{\childdocdisable}
{
  \renewcommand{\childdocmain}[1]{\renewcommand{\childdocmain}[1]{\endinput}}
  \renewcommand{\childdocof}[1]{}
  \renewcommand{\childdocby}[2][]{}
  \renewcommand{\childdocforward}[2][]{}
  \renewcommand{\childdocdisable}{}
}
%    \end{macrocode}

% \macro{\childdocmain}
% The macro |\childdocmain| is to be called at the top of the main file
% with nothing or the main filename (without extension) as argument.
% First, it breaks loops.
% If the argument is not empty and does not match |\childdocname|
% (which is set by the first inclusion of |childdoc.def|),
% |\ifchilddoc| is set to true, |\includeonly| is applied to the child file
% and |\jobname| is set to the main file
% (for proper handling of |.aux| files):
%    \begin{macrocode}
\newcommand{\childdocmain}[1]
{
  \childdocdisable\childdocmain{}
  \if?#1?\else
    \begingroup
      \def\childdoctmp{#1}
      \ifx\childdoctmp\childdocname
        \def\childdoctmp{}
      \else
        \def\childdoctmp
        {
          \childdoctrue
          \includeonly{\childdocname}
          \def\childdocjob{#1}
          \def\jobname{#1}
        }
      \fi
      \expandafter
    \endgroup
    \childdoctmp
  \fi
}
%    \end{macrocode}

% \macro{\childdocof}
% The command |\childdocof| redirects
% compilation to the main file |#1|.
%    \begin{macrocode}
\newcommand{\childdocof}[1]
{
  \childdocdisable
  \childdoctrue
  \includeonly{\childdocname}
  \def\jobname{#1}
  \def\childdocjob{#1}
  \input{#1}
}
%    \end{macrocode}

% \macro{\childdocby}
% The command |\childdocby| ....
%    \begin{macrocode}
\newcommand{\childdocby}[2][]
{
  \childdocdisable
  \childdoctrue
  \childdocmanualtrue
  \if?#1?\else
    \def\jobname{#2}
  \fi
  \def\childdocjob{#2}
  \input{#2}
  \endinput
}
%    \end{macrocode}

% \macro{\childdocforward}
% The command |\childdocforward| redirects
% compilation to the main file or
% (if the optional argument is given) a child file.
% Parameters are set as if the main file
% or a child file starting with |\childdocof| was compiled.
% Then compilation is handed over to the main file:
%    \begin{macrocode}
\newcommand{\childdocforward}[2][]
{
  \begingroup
    \if?#1?
      \def\childdoctmp
      {
        \def\childdocname{#2}
        \def\childdocjob{#2}
        \def\jobname{#2}
        \input{#2}
        \endinput
      }
    \else
      \def\childdoctmp
      {
        \childdocdisable
        \def\childdocname{#2}
        \childdoctrue
        \includeonly{#2}
        \def\childdocjob{#1}
        \def\jobname{#1}
        \input{#1}
        \endinput
      }
    \fi
    \expandafter
  \endgroup
  \childdoctmp
}
%    \end{macrocode}

% \macro{\childdocforwardprefix}
% The command |\childdocforwardprefix| redirects
% compilation to the main or a child file by means of a pattern.
% The prefix |#1| in the current filename is replaced by |#2|
% and the suffix of the current filename is kept
% (it is assumed that the filename does not contain the substring `|~~~|'
% which is used as a delimiter).
% Compilation is handed over to the new file by |\childdocforward|:
%    \begin{macrocode}
\newcommand{\childdocforwardprefix}[3][]
{
  \begingroup
    \def\childdocextract #2##1~~~{\def\childdoctmp{\childdocforward[#1]{#3##1}}}
    \expandafter\childdocextract\childdocname~~~
    \expandafter
  \endgroup
  \childdoctmp
}
%    \end{macrocode}

% \macro{\childdoc}
% The deprecated macro |\childdoc| is a legacy version of |\childdocmain|:
%    \begin{macrocode}
\newcommand{\childdoc}{\childdocmain}
%    \end{macrocode}

% \macro{\childdocredirect}
% The deprecated macro |\childdocredirect| is a legacy version
% of |\childdocforward| and |\childdocforwardprefix|:
%    \begin{macrocode}
\newcommand{\childdocredirect}[2][]
{
  \begingroup
    \if?#1?
      \def\childdoctmp{\childdocforward{#2}}
    \else
      \def\childdoctmp{\childdocforwardprefix{#1}{#2}}
    \fi
    \expandafter
  \endgroup
  \childdoctmp
}
%    \end{macrocode}

%\iffalse
%</package>
%\fi
%
\endinput
|\\
|\childdocby{|\textit{main}|}|\\
\end{tabular}
\end{center}
%
Both forms have slightly different effects as described above.
The main file is prepared as usual, see \secref{sec:include}.

%%%%%%%%%%%%%%%%%%%%%%%%%%%%%%%%%%%%%%%%%%%%%%%%%%%%%%%%%%%%%%%%%%%%%%%%%%%%%%%%
\subsection{Legacy Detection}
\label{sec:detection}

The directive |\childdocmain| in the main file can detect
whether the complete document or merely a child is to be compiled
even without using the directive |\childdocof|.
This method is deprecated because it is less robust
and there is no compelling reason to use it;
it is merely provided for backward compatibility
and it may be removed in future versions.

If the detection mechanism is to be used,
it is mandatory to correctly specify
the filename of the main file as the argument of |\childdocmain|:
%
\begin{center}
\begin{tabular}{l}
|% \iffalse
%
% childdoc.dtx Copyright (C) 2017-2018 Niklas Beisert
%
% This work may be distributed and/or modified under the
% conditions of the LaTeX Project Public License, either version 1.3
% of this license or (at your option) any later version.
% The latest version of this license is in
%   http://www.latex-project.org/lppl.txt
% and version 1.3 or later is part of all distributions of LaTeX
% version 2005/12/01 or later.
%
% This work has the LPPL maintenance status `maintained'.
%
% The Current Maintainer of this work is Niklas Beisert.
%
% This work consists of the files childdoc.dtx and childdoc.ins
% and the derived files childdoc.def and cdocsamp.tex with
% cdocsch1.tex, cdocsch2.tex, cdocsdrf.tex, cdocsfn1.tex, cdocsfn2.tex.
%
%<package>\ifdefined\childdocmain\endinput\fi
%<package>\ProvidesFile{childdoc.def}[2018/12/30 v2.0 child document driver]
%<samplemain>\ProvidesFile{cdocsamp.tex}[2018/12/30 v2.0 sample for childdoc]
%<*driver>
%\ProvidesFile{childdoc.drv}[2018/12/30 v2.0 childdoc reference manual file]
\PassOptionsToClass{10pt,a4paper}{article}
\documentclass{ltxdoc}

\usepackage[margin=35mm]{geometry}
\usepackage{hyperref}
\usepackage{hyperxmp}
\usepackage[usenames]{color}

\hypersetup{colorlinks=true}
\hypersetup{pdfstartview=FitH}
\hypersetup{pdfpagemode=UseNone}
\hypersetup{pdfsource={}}
\hypersetup{pdflang={en-UK}}
\hypersetup{pdfcopyright={Copyright 2017-2018 Niklas Beisert.
  This work may be distributed and/or modified under the
  conditions of the LaTeX Project Public License, either version 1.3
  of this license or (at your option) any later version.}}
\hypersetup{pdflicenseurl={http://www.latex-project.org/lppl.txt}}
\hypersetup{pdfcontactaddress={ETH Zurich, ITP, HIT K,
  Wolfgang-Pauli-Strasse 27}}
\hypersetup{pdfcontactpostcode={8093}}
\hypersetup{pdfcontactcity={Zurich}}
\hypersetup{pdfcontactcountry={Switzerland}}
\hypersetup{pdfcontactemail={nbeisert@itp.phys.ethz.ch}}
\hypersetup{pdfcontacturl={http://people.phys.ethz.ch/\xmptilde nbeisert/}}

\newcommand{\secref}[1]{\hyperref[#1]{section \ref*{#1}}}

\parskip1ex
\parindent0pt
\let\olditemize\itemize
\def\itemize{\olditemize\parskip0pt}

\begin{document}

\title{The \textsf{childdoc} Package}
\hypersetup{pdftitle={The childdoc Package}}
\author{Niklas Beisert\\[2ex]
  Institut f\"ur Theoretische Physik\\
  Eidgen\"ossische Technische Hochschule Z\"urich\\
  Wolfgang-Pauli-Strasse 27, 8093 Z\"urich, Switzerland\\[1ex]
  \href{mailto:nbeisert@itp.phys.ethz.ch}
  {\texttt{nbeisert@itp.phys.ethz.ch}}}
\hypersetup{pdfauthor={Niklas Beisert}}
\hypersetup{pdfsubject={Manual for the LaTeX2e Package childdoc}}
\date{30 December 2018, \textsf{v2.0}}
\maketitle

\begin{abstract}\noindent
\textsf{childdoc} is a \LaTeXe{} package
that enables the direct compilation
of document sections included by |\include|
to individual files.
\end{abstract}

\begingroup
\parskip0ex
\tableofcontents
\endgroup

%%%%%%%%%%%%%%%%%%%%%%%%%%%%%%%%%%%%%%%%%%%%%%%%%%%%%%%%%%%%%%%%%%%%%%%%%%%%%%%%
%%%%%%%%%%%%%%%%%%%%%%%%%%%%%%%%%%%%%%%%%%%%%%%%%%%%%%%%%%%%%%%%%%%%%%%%%%%%%%%%
\section{Introduction}

\LaTeX{} provides a mechanism to structure a large document (such as a book)
into a main file and several child files (containing the chapters)
using the |\include| command.
This mechanism is beneficial for documents
which span hundreds of pages in order to
make the source file(s) more manageable.
Moreover, compilation can be restricted to
selected child files by means of the |\includeonly| command.
The latter feature can be used to reduce the compilation time while editing
(this was significantly more useful in the earlier days of \LaTeX{})
or to generate a smaller document which is easier to navigate.
Another application of |\includeonly| is to generate
documents consisting of selected parts of the complete document.

However, there are a few drawbacks of the plain |\include| mechanism:
\begin{itemize}
\item
The child files cannot be compiled on their own,
they can only be compiled via the main file.
A naive editing environment
(such as a text editor with an option
to have the current file processed by \LaTeX)
may require one to switch to the main file before compiling;
attempting to compile the child file produces errors.
\item
The main file must be modified (each time)
to adjust the |\includeonly| command
to the present needs. This easily leaves the main file in a messy state.
\item
The generated document will always carry the filename
of the main document. This is inconvenient if
several child files are to be compiled and
to be kept for distribution.
\end{itemize}

The present package provides a simple interface
to make child files individually compilable by \LaTeX{}.
Compiling a child file then has the same effect as compiling
the main file with an |\includeonly| command
to select the appropriate child.
Moreover the generated document will carry the name of the child
rather than the main file.
This resolves all three above issues.

This feature is meant to make the editing of books,
thesis documents and lecture notes somewhat more convenient.
However, the package can also be used efficiently for
composing a series of documents (such as exercise sheets)
which are typically distributed individually.
It then assists the author in generating the individual documents
(potentially in different versions)
as well as a document containing the collected series.
Another application is in developing style files
or other kinds of included material
where compilation of the style file could redirect
to a sample or test file.

%%%%%%%%%%%%%%%%%%%%%%%%%%%%%%%%%%%%%%%%%%%%%%%%%%%%%%%%%%%%%%%%%%%%%%%%%%%%%%%%
%%%%%%%%%%%%%%%%%%%%%%%%%%%%%%%%%%%%%%%%%%%%%%%%%%%%%%%%%%%%%%%%%%%%%%%%%%%%%%%%
\section{Usage}

First of all, the package \textsf{childdoc} is \emph{not} a standard
\LaTeXe{} |.sty| style file! Therefore it needs to be invoked in
a non-standard way.

%%%%%%%%%%%%%%%%%%%%%%%%%%%%%%%%%%%%%%%%%%%%%%%%%%%%%%%%%%%%%%%%%%%%%%%%%%%%%%%%
\subsection{Included Files}
\label{sec:include}

%%%%%%%%%%%%%%%%%%%%%%%%%%%%%%%%%%%%%%%%
\DescribeMacro{\childdocmain}
To use the package, add the commands
\begin{center}
\begin{tabular}{l}
|\input{childdoc.def}|\\
|\childdocmain{}|\\
\end{tabular}
\end{center}
at the very top of the main \LaTeX{} file,
in particular \emph{before} the |\documentclass| statement!
The argument of |\childdocmain| should be left empty
(but it must be present).

%%%%%%%%%%%%%%%%%%%%%%%%%%%%%%%%%%%%%%%%
\DescribeMacro{\childdocof}
Furthermore, add the commands
\begin{center}
\begin{tabular}{l}
|\input{childdoc.def}|\\
|\childdocof{|\textit{main}|}|\\
\end{tabular}
\end{center}
at the top of every child file \textit{child}
which is included by |\include{|\textit{child}|}|
from within the main file
(or at least for those files to be compiled individually).
The argument \textit{main} must be the filename of the main file.

There are a couple of
considerations in setting up the main and child documents:

%%%%%%%%%%%%%%%%%%%%%%%%%%%%%%%%%%%%%%%%
\paragraph{Restrictions.}

Please note the following restrictions:
\begin{itemize}
\item
|\childdocmain| must be called with one argument \textit{main}
to ensure compatibility with earlier version of the package.
It must either be empty (|\childdocmain{}|)
or precisely match the filename of the main file in which it is specified.
See \secref{sec:detection} for further information.
\item
The filename \textit{main} must be specified without the |.tex| extension.
\item
The filename \textit{main} is case sensitive
(even in case-insensitive file systems)
due to internal string comparison.
\item
The argument \textit{main} should be fully expanded, it cannot be a macro.
\item
Subdirectories and special characters should be avoided in filenames.
\item
The command |\childdocmain{|\textit{main}|}| must be followed by a whitespace.
It should not be followed immediately by another command
or by a comment mark `|%|'.
This is because the \TeX{} parser reads the token immediately following
the argument of |\childdocmain| and puts it
at the beginning of every child section;
however, a white\-space is ignored.
\end{itemize}

%%%%%%%%%%%%%%%%%%%%%%%%%%%%%%%%%%%%%%%%
\paragraph{Content of Main File.}

It is advisable to place all content in the child files included by |\include|.
Any output contained in the main file will appear in all child documents
unless suppressed manually;
it cannot be suppressed automatically by the |\includeonly| directive
and thus should normally be avoided.
A method to include some content in the main file
by means of conditional processing is described in \secref{sec:conditional}.

%%%%%%%%%%%%%%%%%%%%%%%%%%%%%%%%%%%%%%%%
\paragraph{Page Numbering.}

When only a part of the document is compiled,
the appropriate numbering of pages
(as well as other status parameters)
is determined from the |.aux| files.
The latter contain information from previous passes.
However this information needs to propagate through
all intermediate child documents.
Therefore the page numbering in child documents may well
be inconsistent until the complete document is compiled at least once.

A useful (if unconventional) way to always ensure a consistent
page numbering is to restart the numbering in each child document
and denote the pages by `\textit{child}|.|\textit{page}'
where \textit{child} represents the chapter/section number of the child file.
This can be achieved by the command
|\numberwithin{page}{|\textit{child}|}|
of the \textsf{amsmath} package
where \textit{child} can be |chapter| or |section|
depending on the chosen structuring.
Alternatively, one can modify the macro |\thepage| appropriately
and reset the counter |page| at the start of each child file.

%%%%%%%%%%%%%%%%%%%%%%%%%%%%%%%%%%%%%%%%%%%%%%%%%%%%%%%%%%%%%%%%%%%%%%%%%%%%%%%%
\subsection{Conditional Processing}
\label{sec:conditional}

The package provides a mechanism to compile different versions
of a document. To customise the versions further some conditional processing
can come in handy to distinguish which version is being compiled.
The package provides two macros to describe the compilation context:

%%%%%%%%%%%%%%%%%%%%%%%%%%%%%%%%%%%%%%%%
\DescribeMacro{\ifchilddoc}
The conditional |\ifchilddoc| distinguishes between the compilation of
child documents and the main document:
%
\begin{center}
|\ifchilddoc |\textit{child-code}| |[|\||else |\textit{main-code}]| \||fi|
\end{center}

%%%%%%%%%%%%%%%%%%%%%%%%%%%%%%%%%%%%%%%%
\DescribeMacro{\childdocname}
\DescribeMacro{\childdocjob}
The macro |\childdocname| contains the filename (without extension)
of the main or child file being processed.
Note that |\childdocjob| will always contain the name of the main file.

%%%%%%%%%%%%%%%%%%%%%%%%%%%%%%%%%%%%%%%%
\paragraph{Title Page.}

Conditional processing can be used to include a title or banner page
in the main document when proper precautions are taken.
Importantly, the code in the main file should ensure that the page counter
(as well as other status parameters which are stored in the |.aux| files)
takes the same value after the conditional processing.
Otherwise the page numbers may take divergent values
depending on which part is compiled.

For example, a title page could be declared by:
%
\begin{center}
\begin{tabular}{l}
|\ifchilddoc\||else|\\
|\addtocounter{page}{-1}|\\
\textit{code for title page}\\
|\newpage|\\
|\||fi|
\end{tabular}
\end{center}
%
A banner page for the child documents can be generated by:
%
\begin{center}
\begin{tabular}{l}
|\ifchilddoc|\\
|\addtocounter{page}{-1}|\\
\textit{code for banner page}\\
|\newpage|\\
|\||fi|
\end{tabular}
\end{center}
%
Here one could write a message such as:
\begin{center}
|This is the part \childdocname{} of \childdocjob{}.|
\end{center}

%%%%%%%%%%%%%%%%%%%%%%%%%%%%%%%%%%%%%%%%%%%%%%%%%%%%%%%%%%%%%%%%%%%%%%%%%%%%%%%%
\subsection{Flags}
\label{sec:flags}

The package makes it easy to generate different versions
of the main or child documents.
To this end compilation flags can be defined
and assigned different default values.
They will be particularly useful in conjunction
with the forwarding mechanism described in \secref{sec:forward}.

For example, it may be useful to have a flag |\version|
which can be set to |draft| or |final|.
The document source will contain some conditional code
depending on the value of |\version|.
Suppose further, the flag should default to |final| for the main file
and to |draft| for child files
which is a natural assignment for editing the document.
This is achieved by placing the following code
in the preamble of the main document
(below the |\childdocmain| directive):
%
\begin{center}
\begin{tabular}{l}
|\ifchilddoc|\\
|\providecommand{\version}{draft}|\\
|\||else|\\
|\providecommand{\version}{final}|\\
|\||fi|
\end{tabular}
\end{center}
%
The definition by |\providecommand| makes sure
that previous definitions are not overwritten.
Further statements |\providecommand{\version}{...}|
can thus be added before the above code to override it.

For the main file, one might add a line
(between |\childdocmain| and the above block)
%
\begin{center}
|%\ifchilddoc\||else\providecommand{\version}{draft}\||fi|
\end{center}
%
which can be uncommented to produce a draft version.
Likewise one can add a line to the very top of a child file
(above the |\childdocof{|\textit{main}|}| directive)
%
\begin{center}
|%\providecommand{\version}{final}|
\end{center}
%
which can be uncommented to produce the final version of this child document.

%%%%%%%%%%%%%%%%%%%%%%%%%%%%%%%%%%%%%%%%%%%%%%%%%%%%%%%%%%%%%%%%%%%%%%%%%%%%%%%%
\subsection{Forwarding}
\label{sec:forward}

Different versions of the main or child documents
using compilation flags as described in \secref{sec:flags}
can be (permanently) stored in different files
for convenient compilation, viewing and distribution.
To this end, the package defines a command
to pass on compilation to a different file:

%%%%%%%%%%%%%%%%%%%%%%%%%%%%%%%%%%%%%%%%
\DescribeMacro{\childdocforward}
The command |\childdocforward| redirects processing to
another source file:
%
\begin{center}
\begin{tabular}{l}
|\input{childdoc.def}|\\
|\childdocforward[|\textit{main}|]{|\textit{dest}|}|\\
\end{tabular}
\end{center}
%
The argument \textit{dest} is the destination file
(without extension).
It should be the main file or one of the child files.
Note that further \textsf{childdoc} directives
such as |\childdocof| and |\childdocforward|
in the indicated file will be processed in this form.
The optional argument \textit{main}
passes on directly to the main file \textit{main}
while pretending to compile the child \textit{dest}.
This form behaves as if \textit{dest}
issues |\childdocof{|\textit{main}|}| right away,
and no further \textsf{childdoc} directives will be processed.

%%%%%%%%%%%%%%%%%%%%%%%%%%%%%%%%%%%%%%%%
\DescribeMacro{\...prefix}
In the alternative form |\childdocforwardprefix|,
%
\begin{center}
\begin{tabular}{l}
|\input{childdoc.def}|\\
|\childdocforwardprefix[|\textit{main}|]{|\textit{prefix}|}{|\textit{dest}|}|
\end{tabular}
\end{center}
%
the destination file is determined by a pattern
depending on the current file:
To make this work, the current file must be called
`{\textit{prefix}\hspace{0.2em}\textit{suffix}}'
with \textit{prefix} matching precisely the argument.
Processing is then passed on to the file
`{\textit{dest}\hspace{0.2em}\textit{suffix}}'.
Surely, the same effect is achieved by
directly specifying the
argument `{\textit{dest}\hspace{0.2em}\textit{suffix}}'
in the first form.
However, that requires to set up a different file
for each child. With the alternative form of the command
all these files can have exactly the same content
which simplifies setting them up and maintaining them.

For example, the following file |draft.tex|
with a compilation flag |\version| as described in \secref{sec:flags}
compiles the main document as a draft:
%
\begin{center}
\begin{tabular}{l}
|\def\version{draft}|\\
|\input{childdoc.def}|\\
|\childdocforward{|\textit{main}|}|
\end{tabular}
\end{center}
%
Likewise, the following files |final|\textit{nn}|.tex|
compile the final version of the child document
|child|\textit{nn}|.tex|:
%
\begin{center}
\begin{tabular}{l}
|\def\version{final}|\\
|\input{childdoc.def}|\\
|\childdocforwardprefix{final}{child}|
\end{tabular}
\end{center}
%

Note that when several versions of a main file and/or of each child file
are to be generated, it may be convenient to set up a |Makefile| or
shell script to automatise the process.

%%%%%%%%%%%%%%%%%%%%%%%%%%%%%%%%%%%%%%%%%%%%%%%%%%%%%%%%%%%%%%%%%%%%%%%%%%%%%%%%
\subsection{Command Line Processing}
\label{sec:commandline}

The effect of redirection files can also be achieved by invoking
the \LaTeX{} compiler with a more elaborate command line.
Most conveniently this should be done as part
of a shell script or a |Makefile|.

When using \textsf{childdoc} in the main file, the following
command lines effectively perform a redirection
(note that depending on the shell being used,
backslashes may have to be doubled: `|\|' $\to$ `|\\|'):
%
\begin{center}
|... -jobname "|\textit{target}|" |\\|"|[\textit{flags}]%
|\input{childdoc.def}\childdocforward[|\textit{main}|]{|\textit{dest}|}"|
\end{center}
%
Here \textit{target} is the name of the output file,
\textit{main} is the name of the main file
and \textit{dest} is the name of the main or child file to be processed
(all filenames without extensions).
The optional argument \textit{main} can be omitted
if \textit{main} matches \textit{dest}.
Optionally, compilation \textit{flags} can be defined via |\def| commands.
This command line makes the \TeX{} engine believe
it is compiling the file \textit{target}
whose content is specified as the latter parameter.
The provided code then forwards the processing to
\textit{main} or \textit{dest} as described in \secref{sec:forward}.

%%%%%%%%%%%%%%%%%%%%%%%%%%%%%%%%%%%%%%%%%%%%%%%%%%%%%%%%%%%%%%%%%%%%%%%%%%%%%%%%
\subsection{Include by Input}
\label{sec:input}

Including child documents by |\include| has some restrictions by design.
Most notably, the content of a child document always occupies
its own set of pages; pages cannot be shared between child documents.
Usually, this behaviour makes perfect sense
because each child document contain an essential part of the document.
However, in some situations it may be desirable to compose
a document from a collection of parts
without having mandatory page breaks between then.
For this case, the package
provides a mechanism to include parts
by |\input| which can also be processed individually.
However, by construction this mechanism
requires manual handling of the content to be output.

%%%%%%%%%%%%%%%%%%%%%%%%%%%%%%%%%%%%%%%%
\DescribeMacro{\ifchilddocmanual}
The main file should be prepared as usual, see \secref{sec:include}.
However, the document body must make a distinction
between processing of an individual part and of the main document, e.g.:
%
\begin{center}
\begin{tabular}{l}
|\ifchilddocmanual|\\
|\input{\childdocname}|\\
|\||else|\\
\textit{document body with }|\input{|\textit{part}|}|\\
|\||fi|
\end{tabular}
\end{center}
%
The conditional |\ifchilddocmanual| is true whenever
a part to be included by |\input| is being compiled,
and the name of the part is stored in |\childdocname|.

%%%%%%%%%%%%%%%%%%%%%%%%%%%%%%%%%%%%%%%%
\DescribeMacro{\childdocby}
Each part to be included by |\input| should start with:
%
\begin{center}
\begin{tabular}{l}
|\input{childdoc.def}|\\
|\childdocby{|\textit{main}|}|\\
\end{tabular}
\end{center}
%
The directive |\childdocby| is similar to |\childdocof|
described in \secref{sec:include},
but the subsequent selection of content must be done manually.
To that end, both |\ifchilddoc| and |\ifchilddocmanual|
will be true upon processing of a part,
and the name of the part is stored in |\childdocname|.
Note that |\jobname| will be set to the filename of the current part
so that each part receives an individual |.aux| file
that does not interfere with the |.aux| file(s) of the main document.
This behaviour can be altered by the alternative form
|\childdocby[*]{|\textit{main}|}| (with a non-empty optional argument)
which uses the |.aux| file of the main document
by setting |\jobname| to \textit{main}.

%%%%%%%%%%%%%%%%%%%%%%%%%%%%%%%%%%%%%%%%%%%%%%%%%%%%%%%%%%%%%%%%%%%%%%%%%%%%%%%%
\subsection{Driver Development}
\label{sec:driver}

The \textsf{childdoc} mechanism can also be use for the development
of definition files such as \LaTeX{} styles or classes.
This case differs from the above setup with multiple parts
included by |\include| in that no |\includeonly| should be invoked.
This can be achieved by starting the include file
(before |\ProvidesPackage|) with:
%
\begin{center}
\begin{tabular}{l}
|\input{childdoc.def}|\\
|\childdocforward{|\textit{main}|}|\\
\end{tabular}
\end{center}
%
or alternatively with:
%
\begin{center}
\begin{tabular}{l}
|\input{childdoc.def}|\\
|\childdocby{|\textit{main}|}|\\
\end{tabular}
\end{center}
%
Both forms have slightly different effects as described above.
The main file is prepared as usual, see \secref{sec:include}.

%%%%%%%%%%%%%%%%%%%%%%%%%%%%%%%%%%%%%%%%%%%%%%%%%%%%%%%%%%%%%%%%%%%%%%%%%%%%%%%%
\subsection{Legacy Detection}
\label{sec:detection}

The directive |\childdocmain| in the main file can detect
whether the complete document or merely a child is to be compiled
even without using the directive |\childdocof|.
This method is deprecated because it is less robust
and there is no compelling reason to use it;
it is merely provided for backward compatibility
and it may be removed in future versions.

If the detection mechanism is to be used,
it is mandatory to correctly specify
the filename of the main file as the argument of |\childdocmain|:
%
\begin{center}
\begin{tabular}{l}
|\input{childdoc.def}|\\
|\childdocmain{|\textit{main}|}|\\
\end{tabular}
\end{center}
%
If |\jobname| does not match the argument \textit{main} of |\childdocmain|,
it is assumed that |\jobname| points to the child file to be compiled.
When using |\childdocmain| with the main file specified as argument,
it suffices to start a child file
with just |\input{|\textit{main}|}|
without loading of the package and using |\childdocof|.
If instead all processing is done
with the appropriate \textsf{childdoc} directives,
the argument of \textit{main} of |\childdocmain| can be empty.

An alternative version of the command line processing described
in \secref{sec:commandline} using the detection mechanism reads:
%
\begin{center}
|... -jobname "|\textit{target}|" "|[\textit{flags}]%
[|\def\jobname{|\textit{dest}|}|]|\input{|\textit{main}|}"|
\end{center}

%%%%%%%%%%%%%%%%%%%%%%%%%%%%%%%%%%%%%%%%%%%%%%%%%%%%%%%%%%%%%%%%%%%%%%%%%%%%%%%%
\subsection{Manual Code}
\label{sec:manual}

In case one cannot be certain whether the definitions file |childdoc.def|
is installed on the target \TeX{} distribution
and one prefers not to ship it,
it is conceivable to paste a few relevant commands into the sources.

To that end, drop all statements |\input{childdoc.def}|
and perform the replacements as outlined below.
Instead of |\childdocmain{|\textit{main}|}| add the following code
to the top of the main file:
%
\begin{center}
\begin{tabular}{l}
|\||ifdefined\childdocname\endinput\||fi\newif\ifchilddoc|\\
|\edef\childdocname{\scantokens\expandafter{\jobname\noexpand}}|\\
|\def\childdocmain{|\textit{main}|}\||ifx\childdocmain\childdocname\||else|\\
|\childdoctrue\includeonly{\childdocname}\let\jobname\childdocmain\||fi|\\
\end{tabular}
\end{center}
%
Instead of |\childdocof{|\textit{main}|}| just include the main file
at the top of each child file:
%
\begin{center}
|\input{|\textit{main}|}|
\end{center}
%
A simple redirection |\childdocforward{|\textit{dest}|}| is achieved by:
%
\begin{center}
|\def\jobname{|\textit{dest}|}\input{\jobname}|
\end{center}
%
The redirection with prefix
|\childdocforwardprefix[|\textit{prefix}|]{|\textit{dest}|}|
is accomplished by:
%
\begin{center}
\begin{tabular}{l}
|{\edef\jobname{\scantokens\expandafter{\jobname\noexpand}}|\\
|\def\redirectjob |\textit{prefix}|#1~~~{\gdef\jobname{|\textit{dest}|#1}}|\\
|\expandafter\redirectjob\jobname~~~}\input{\jobname}|
\end{tabular}
\end{center}

In an alternative approach,
child documents can be compiled by a specific command line
without additional code or specific definitions:
%
\begin{center}
|... -jobname "|\textit{target}|" "|[\textit{flags}]%
|\includeonly{|\textit{dest}|}\input{|\textit{main}|}"|
\end{center}
%

%%%%%%%%%%%%%%%%%%%%%%%%%%%%%%%%%%%%%%%%%%%%%%%%%%%%%%%%%%%%%%%%%%%%%%%%%%%%%%%%
%%%%%%%%%%%%%%%%%%%%%%%%%%%%%%%%%%%%%%%%%%%%%%%%%%%%%%%%%%%%%%%%%%%%%%%%%%%%%%%%
\section{Information}

%%%%%%%%%%%%%%%%%%%%%%%%%%%%%%%%%%%%%%%%%%%%%%%%%%%%%%%%%%%%%%%%%%%%%%%%%%%%%%%%
\subsection{Copyright}

Copyright \copyright{} 2017--2018 Niklas Beisert

This work may be distributed and/or modified under the
conditions of the \LaTeX{} Project Public License, either version 1.3
of this license or (at your option) any later version.
The latest version of this license is in
  \url{http://www.latex-project.org/lppl.txt}
and version 1.3 or later is part of all distributions of \LaTeX{}
version 2005/12/01 or later.

This work has the LPPL maintenance status `maintained'.

The Current Maintainer of this work is Niklas Beisert.

This work consists of the files |README.txt|, |childdoc.ins| and |childdoc.dtx|
as well as the derived files |childdoc.def|, |cdocsamp.tex|
with |cdocsch1.tex|, |cdocsch2.tex|, |cdocspt3.tex|, |cdocspt4.tex|,
|cdocsdrf.tex|, |cdocsfn1.tex|, |cdocsfn2.tex|
as well as |childdoc.pdf|.

%%%%%%%%%%%%%%%%%%%%%%%%%%%%%%%%%%%%%%%%%%%%%%%%%%%%%%%%%%%%%%%%%%%%%%%%%%%%%%%%
\subsection{Files and Installation}

The package consists of the files:
%
\begin{center}
\begin{tabular}{ll}
    |README.txt|   & readme file \\
    |childdoc.ins| & installation file \\
    |childdoc.dtx| & source file \\
    |childdoc.def| & definition file \\
    |cdocsamp.tex| & sample main file \\
    |cdocsch1.tex| & sample include file \\
    |cdocsch2.tex| & sample include file \\
    |cdocspt3.tex| & sample part file \\
    |cdocspt4.tex| & sample part file \\
    |cdocsdrf.tex| & sample redirection file \\
    |cdocsfn1.tex| & sample redirection file \\
    |cdocsfn2.tex| & sample redirection file \\
    |childdoc.pdf| & manual
\end{tabular}
\end{center}
%
The distribution consists of the files
|README.txt|, |childdoc.ins| and |childdoc.dtx|.
%
\begin{itemize}
\item
Run (pdf)\LaTeX{} on |childdoc.dtx|
to compile the manual |childdoc.pdf| (this file).
\item
Run \LaTeX{} on |childdoc.ins| to create the definitions file |childdoc.def|
and the sample |cdocsamp.tex| with include files
|cdocsch1.tex|, |cdocsch2.tex|, |cdocspt3.tex|, |cdocspt4.tex|,
|cdocsdrf.tex|, |cdocsfn1.tex|, |cdocsfn2.tex|.
Then copy the file |childdoc.def| to an appropriate directory of your \LaTeX{}
distribution, e.g.\ \textit{texmf-root}|/tex/latex/childdoc|.
\end{itemize}

%%%%%%%%%%%%%%%%%%%%%%%%%%%%%%%%%%%%%%%%%%%%%%%%%%%%%%%%%%%%%%%%%%%%%%%%%%%%%%%%
\subsection{Related CTAN Packages}

There are several other packages which offer a similar functionality:
%
\begin{itemize}
\item
The packages
\href{http://ctan.org/pkg/docmute}{\textsf{docmute}},
\href{http://ctan.org/pkg/includex}{\textsf{includex}} and
\href{http://ctan.org/pkg/standalone}{\textsf{standalone}}
provide commands to include only the document body of
a child file thus allowing both files to be compiled individually.
\item
The packages \href{http://ctan.org/pkg/subdocs}{\textsf{subdocs}}
and \href{http://ctan.org/pkg/subfiles}{\textsf{subfiles}}
provide structures in which the main and child documents can be
encapsulated and allowing them to be compiled individually.
The inclusion mechanism is different from the conventional |\include|.
\item
The package \href{http://ctan.org/pkg/combine}{\textsf{combine}}
is an elaborate solution to combine several documents into one.
\end{itemize}
%
See also the CTAN topic \href{http://ctan.org/topic/subdocs}{\textsf{subdocs}}
for further related packages.
The present package differs from the above solutions in that
a document structure constructed with the conventional |\include| mechanism
just needs two extra commands at the top of every file
such that all constituent files can be compiled individually.

%%%%%%%%%%%%%%%%%%%%%%%%%%%%%%%%%%%%%%%%%%%%%%%%%%%%%%%%%%%%%%%%%%%%%%%%%%%%%%%%
%\subsection{Feature Suggestions}
%
%The following is a list of features which may be useful for future
%versions of this package:
%%
%\begin{itemize}
%\item
%\ldots
%\end{itemize}

%%%%%%%%%%%%%%%%%%%%%%%%%%%%%%%%%%%%%%%%%%%%%%%%%%%%%%%%%%%%%%%%%%%%%%%%%%%%%%%%
\subsection{Revision History}

%%%%%%%%%%%%%%%%%%%%%%%%%%%%%%%%%%%%%%%%
\paragraph{v2.0:} 2018/12/30

\begin{itemize}
\item
immediate forward processing
\item
added |\childdocby| mechanism
\item
manual restructured
\end{itemize}

%%%%%%%%%%%%%%%%%%%%%%%%%%%%%%%%%%%%%%%%
\paragraph{v1.6:} 2018/01/17

\begin{itemize}
\item
application for development of include files
\item
corrections to manual
\end{itemize}

%%%%%%%%%%%%%%%%%%%%%%%%%%%%%%%%%%%%%%%%
\paragraph{v1.5:} 2017/05/21

\begin{itemize}
\item
more complete structuring introduced
\item
|\childdocof| introduced
\item
|\childdoc| renamed to |\childdocmain|
\item
|\childredirect| renamed to |\childdocforward| and |\childdocforwardprefix|
and functionality expanded
\end{itemize}

%%%%%%%%%%%%%%%%%%%%%%%%%%%%%%%%%%%%%%%%
\paragraph{v1.0:} 2017/04/27

\begin{itemize}
\item
manual and install package
\item
first version published on CTAN
\end{itemize}

%%%%%%%%%%%%%%%%%%%%%%%%%%%%%%%%%%%%%%%%
\paragraph{v0.6:} 2017/04/26

\begin{itemize}
\item
redirection mechanism added
\end{itemize}

%%%%%%%%%%%%%%%%%%%%%%%%%%%%%%%%%%%%%%%%
\paragraph{v0.5:} 2017/04/26

\begin{itemize}
\item
functionality in definition file
\end{itemize}


%%%%%%%%%%%%%%%%%%%%%%%%%%%%%%%%%%%%%%%%%%%%%%%%%%%%%%%%%%%%%%%%%%%%%%%%%%%%%%%%
%%%%%%%%%%%%%%%%%%%%%%%%%%%%%%%%%%%%%%%%%%%%%%%%%%%%%%%%%%%%%%%%%%%%%%%%%%%%%%%%
%%%%%%%%%%%%%%%%%%%%%%%%%%%%%%%%%%%%%%%%%%%%%%%%%%%%%%%%%%%%%%%%%%%%%%%%%%%%%%%%
\appendix

\settowidth\MacroIndent{\rmfamily\scriptsize 000\ }

 \DocInput{childdoc.dtx}

\end{document}
%</driver>
% \fi
%
% %%%%%%%%%%%%%%%%%%%%%%%%%%%%%%%%%%%%%%%%%%%%%%%%%%%%%%%%%%%%%%%%%%%%%%%%%%%%%%
% %%%%%%%%%%%%%%%%%%%%%%%%%%%%%%%%%%%%%%%%%%%%%%%%%%%%%%%%%%%%%%%%%%%%%%%%%%%%%%
% \section{Sample}
%\iffalse
%<*samplemain>
%\fi
%
% The following presents a sample document
% with two chapters, two parts, a title page,
% a compile flag as well as three forwarding files to set the flag.
% It consists of eight |.tex| files:
% \begin{center}
% \begin{tabular}{ll}
% |cdocsamp.tex|&main file\\
% |cdocsch1.tex|&include file for chapter 1\\
% |cdocsch2.tex|&include file for chapter 2\\
% |cdocspt3.tex|&include file for part 3\\
% |cdocspt4.tex|&include file for part 4\\
% |cdocsdrf.tex|&forwarding file for main file in draft mode\\
% |cdocsfi1.tex|&forwarding file for final version of chapter 1\\
% |cdocsfi2.tex|&forwarding file for final version of chapter 2\\
% \end{tabular}
% \end{center}
% Each of the eight files can be compiled directly by the \LaTeX{} compiler.
%
% %%%%%%%%%%%%%%%%%%%%%%%%%%%%%%%%%%%%%%
% \paragraph{Main File.}
%
% The main file is called |cdocsamp.tex|.
%
% Load the \textsf{childdoc} definitions and
% declare the filename for the main document:
%    \begin{macrocode}
\input{childdoc.def}
\childdocmain{}
%    \end{macrocode}

% Optional override for |\version| flag:
%    \begin{macrocode}
%%\ifchilddoc\else\providecommand{\version}{draft}\fi
%    \end{macrocode}

% Define the default values for the |\version| flag
% (|final| for the main file and |draft| for childs):
%    \begin{macrocode}
\ifchilddoc
\providecommand{\version}{draft}
\else
\providecommand{\version}{final}
\fi
%    \end{macrocode}

% Load the standard document class:
%    \begin{macrocode}
\documentclass[12pt]{article}
%    \end{macrocode}

% Start the document body:
%    \begin{macrocode}
\begin{document}
%    \end{macrocode}

% Declare a title page.
% Print title, part of document being processed and version flag:
%    \begin{macrocode}
\addtocounter{page}{-1}
\begin{center}
{\LARGE\bfseries{}childdoc example\par}
\vspace{1cm}
\ifchilddoc
\ifchilddocmanual part\else chapter\fi:
`\childdocname' of `\childdocjob'\par
\else
main document: `\childdocjob'\par
\fi
version: \version\par
\end{center}
\newpage
%    \end{macrocode}

% Manually include selected file,
% otherwise process as usual:
%    \begin{macrocode}
\ifchilddocmanual
\section*{part `\childdocname'}
\input{\childdocname}
\else
%    \end{macrocode}

% Include the two chapters:
%    \begin{macrocode}
\include{cdocsch1}
\include{cdocsch2}
%    \end{macrocode}

% Include the two parts unless only chapters should be displayed:
%    \begin{macrocode}
\ifchilddoc\else
\section{part three}
\input{cdocspt3}
\section{part four}
\input{cdocspt4}
\fi
%    \end{macrocode}

% Process as usual until here:
%    \begin{macrocode}
\fi
%    \end{macrocode}

% End of document body:
%    \begin{macrocode}
\end{document}
%    \end{macrocode}
%\iffalse
%</samplemain>
%\fi
%
% %%%%%%%%%%%%%%%%%%%%%%%%%%%%%%%%%%%%%%
% \paragraph{Chapter Include Files.}
%
% The include files are called |cdocsch1.tex| and |cdocsch2.tex|.
%
%\iffalse
%<*samplechap1|samplechap2>
%\fi

% Optional override for |\version| flag:
%    \begin{macrocode}
%%\providecommand{\version}{final}
%    \end{macrocode}

% Include the main document:
%    \begin{macrocode}
\input{childdoc.def}
\childdocof{cdocsamp}
%    \end{macrocode}

%\iffalse
%</samplechap1|samplechap2>
%\fi
%
%\iffalse
%<*samplechap1>
%\fi
% Some text for chapter 1:
%    \begin{macrocode}
\section{one}
some text in chapter one
%    \end{macrocode}

%\iffalse
%</samplechap1>
%\fi
% Some text for chapter 2:
%\iffalse
%<*samplechap2>
%\fi
%    \begin{macrocode}
\section{two}
more text in chapter two
%    \end{macrocode}

%\iffalse
%</samplechap2>
%\fi
%
% %%%%%%%%%%%%%%%%%%%%%%%%%%%%%%%%%%%%%%
% \paragraph{Part Include Files.}
%
% The include files are called |cdocspt3.tex| and |cdocspt4.tex|.
%
%\iffalse
%<*samplepart3|samplepart4>
%\fi

% Optional override for |\version| flag:
%    \begin{macrocode}
%%\providecommand{\version}{final}
%    \end{macrocode}

% Include the main document:
%    \begin{macrocode}
\input{childdoc.def}
\childdocby{cdocsamp}
%    \end{macrocode}

%\iffalse
%</samplepart3|samplepart4>
%\fi
%
%\iffalse
%<*samplepart3>
%\fi
% Some text for part 3:
%    \begin{macrocode}
some text in part three
%    \end{macrocode}

%\iffalse
%</samplepart3>
%\fi
% Some text for part 4:
%\iffalse
%<*samplepart4>
%\fi
%    \begin{macrocode}
more text in part four
%    \end{macrocode}

%\iffalse
%</samplepart4>
%\fi
%
% %%%%%%%%%%%%%%%%%%%%%%%%%%%%%%%%%%%%%%
% \paragraph{Forwarding for a Complete Draft.}
%
% The following forwarding file |cdocsdrf.tex|
% compiles the main document in draft mode:
%\iffalse
%<*sampledraft>
%\fi
%    \begin{macrocode}
\def\version{draft}
\input{childdoc.def}
\childdocforward{cdocsamp}
%    \end{macrocode}

%\iffalse
%</sampledraft>
%\fi
%
% %%%%%%%%%%%%%%%%%%%%%%%%%%%%%%%%%%%%%%
% \paragraph{Forwarding for Final Version of the Chapters.}
%
% The following forwarding files |cdocsfn1.tex| and |cdocsfn2.tex|
% (with identical content)
% compile the final versions of the child documents
% |cdocsch1.tex| and |cdocsch2.tex|, respectively:
%\iffalse
%<*samplefinal>
%\fi
%    \begin{macrocode}
\def\version{final}
\input{childdoc.def}
\childdocforwardprefix[cdocsamp]{cdocsfn}{cdocsch}
%    \end{macrocode}

%\iffalse
%</samplefinal>
%\fi
%
% %%%%%%%%%%%%%%%%%%%%%%%%%%%%%%%%%%%%%%
% \paragraph{Command Line Processing.}
%
% The following three command lines generate the output files
% |cdocscld|, |cdocscl1| and |cdocscl2|
% which should be identical to
% |cdocsdrf|, |cdocsch1| and |cdocsfn2|, respectively:
% \begin{center}
% \begin{tabular}{l}
% |latex -jobname cdocscld \|\\
% |  "\def\version{draft}\input{childdoc.def}\childdocforward{cdocsamp}"|\\
% |latex -jobname cdocscl1 \|\\
% |  "\input{childdoc.def}\childdocforward[cdocsamp]{cdocsch1}"|\\
% |latex -jobname cdocscl2 \|\\
% |  "\def\version{final}\input{childdoc.def}\childdocforward{cdocsch2}"|
% \end{tabular}
% \end{center}
% Note that the trailing backslash on each first line
% merely continues the input to the second line
% (for convenient cut ant paste).
% Furthermore, the command |latex| can be replaced by any
% of its alternative versions such as |pdflatex|.
%
% %%%%%%%%%%%%%%%%%%%%%%%%%%%%%%%%%%%%%%%%%%%%%%%%%%%%%%%%%%%%%%%%%%%%%%%%%%%%%%
% %%%%%%%%%%%%%%%%%%%%%%%%%%%%%%%%%%%%%%%%%%%%%%%%%%%%%%%%%%%%%%%%%%%%%%%%%%%%%%
% \section{Implementation}
%\iffalse
%<*package>
%\fi
%
% This section describes the definitions file |childdoc.def|.

% The definitions cannot be loaded using |\usepackage| or |\RequirePackage|
% which has a mechanism to prevent loading a style file more than once.
% When loading the definitions by means of |\input|
% multiple instances have to be prevented manually:
%\iffalse
%This code needs to be before the `\ProvidesFile' directive
%which is defined at the beginning of this file.
%Therefore it is also placed there and commented out here.
%</package>
%<*discard>
%\fi
%    \begin{macrocode}
\ifdefined\childdocmain\endinput\fi
%    \end{macrocode}
%\iffalse
%</discard>
%<*package>
%\fi
%
% \macro{\ifchilddoc}
% \macro{\ifchilddocmanual}
% The conditional |\ifchilddoc| tells whether a
% child (true) or main (false) document is being compiled.
% The conditional |\ifchilddocmanual| tells whether
% the |\includeonly| mechanism is used (false) or
% the selection of child files must be performed manually (true).
% The definitions initialise to false:
%    \begin{macrocode}
\newif\ifchilddoc
\newif\ifchilddocmanual
%    \end{macrocode}

% \macro{\childdocname}
% \macro{\childdocjob}
% The macro |\childdocname| stores the name of the main document
% to be compiled. The macro |\childdocjob| stores the name of
% the document on which the \LaTeX{} compiler was originally invoked.
% The content of |\jobname| cannot be compared
% to filenames specified in the source due to different catcodes.
% The following code rescans |\jobname|, stores the result
% in |\childdocname| and saves a copy in |\childdocjob|:
%    \begin{macrocode}
\edef\childdocname{\scantokens\expandafter{\jobname\noexpand}}
\let\childdocjob\childdocname
%    \end{macrocode}

% \macro{\childdocdisable}
% The macro |\childdocdisable| prevents the main file
% from being processed more than once.
% At this stage, the main document command |\childdocmain|
% is assumed to be called once again where it should do nothing.
% Any subsequent call to it should prevent
% a secondary processing of the main document
% It overwrites the forwarding commands
% |\childdocof| and |\childdocforward|
% with empty macros to prevent further inclusions of the main document:
%    \begin{macrocode}
\newcommand{\childdocdisable}
{
  \renewcommand{\childdocmain}[1]{\renewcommand{\childdocmain}[1]{\endinput}}
  \renewcommand{\childdocof}[1]{}
  \renewcommand{\childdocby}[2][]{}
  \renewcommand{\childdocforward}[2][]{}
  \renewcommand{\childdocdisable}{}
}
%    \end{macrocode}

% \macro{\childdocmain}
% The macro |\childdocmain| is to be called at the top of the main file
% with nothing or the main filename (without extension) as argument.
% First, it breaks loops.
% If the argument is not empty and does not match |\childdocname|
% (which is set by the first inclusion of |childdoc.def|),
% |\ifchilddoc| is set to true, |\includeonly| is applied to the child file
% and |\jobname| is set to the main file
% (for proper handling of |.aux| files):
%    \begin{macrocode}
\newcommand{\childdocmain}[1]
{
  \childdocdisable\childdocmain{}
  \if?#1?\else
    \begingroup
      \def\childdoctmp{#1}
      \ifx\childdoctmp\childdocname
        \def\childdoctmp{}
      \else
        \def\childdoctmp
        {
          \childdoctrue
          \includeonly{\childdocname}
          \def\childdocjob{#1}
          \def\jobname{#1}
        }
      \fi
      \expandafter
    \endgroup
    \childdoctmp
  \fi
}
%    \end{macrocode}

% \macro{\childdocof}
% The command |\childdocof| redirects
% compilation to the main file |#1|.
%    \begin{macrocode}
\newcommand{\childdocof}[1]
{
  \childdocdisable
  \childdoctrue
  \includeonly{\childdocname}
  \def\jobname{#1}
  \def\childdocjob{#1}
  \input{#1}
}
%    \end{macrocode}

% \macro{\childdocby}
% The command |\childdocby| ....
%    \begin{macrocode}
\newcommand{\childdocby}[2][]
{
  \childdocdisable
  \childdoctrue
  \childdocmanualtrue
  \if?#1?\else
    \def\jobname{#2}
  \fi
  \def\childdocjob{#2}
  \input{#2}
  \endinput
}
%    \end{macrocode}

% \macro{\childdocforward}
% The command |\childdocforward| redirects
% compilation to the main file or
% (if the optional argument is given) a child file.
% Parameters are set as if the main file
% or a child file starting with |\childdocof| was compiled.
% Then compilation is handed over to the main file:
%    \begin{macrocode}
\newcommand{\childdocforward}[2][]
{
  \begingroup
    \if?#1?
      \def\childdoctmp
      {
        \def\childdocname{#2}
        \def\childdocjob{#2}
        \def\jobname{#2}
        \input{#2}
        \endinput
      }
    \else
      \def\childdoctmp
      {
        \childdocdisable
        \def\childdocname{#2}
        \childdoctrue
        \includeonly{#2}
        \def\childdocjob{#1}
        \def\jobname{#1}
        \input{#1}
        \endinput
      }
    \fi
    \expandafter
  \endgroup
  \childdoctmp
}
%    \end{macrocode}

% \macro{\childdocforwardprefix}
% The command |\childdocforwardprefix| redirects
% compilation to the main or a child file by means of a pattern.
% The prefix |#1| in the current filename is replaced by |#2|
% and the suffix of the current filename is kept
% (it is assumed that the filename does not contain the substring `|~~~|'
% which is used as a delimiter).
% Compilation is handed over to the new file by |\childdocforward|:
%    \begin{macrocode}
\newcommand{\childdocforwardprefix}[3][]
{
  \begingroup
    \def\childdocextract #2##1~~~{\def\childdoctmp{\childdocforward[#1]{#3##1}}}
    \expandafter\childdocextract\childdocname~~~
    \expandafter
  \endgroup
  \childdoctmp
}
%    \end{macrocode}

% \macro{\childdoc}
% The deprecated macro |\childdoc| is a legacy version of |\childdocmain|:
%    \begin{macrocode}
\newcommand{\childdoc}{\childdocmain}
%    \end{macrocode}

% \macro{\childdocredirect}
% The deprecated macro |\childdocredirect| is a legacy version
% of |\childdocforward| and |\childdocforwardprefix|:
%    \begin{macrocode}
\newcommand{\childdocredirect}[2][]
{
  \begingroup
    \if?#1?
      \def\childdoctmp{\childdocforward{#2}}
    \else
      \def\childdoctmp{\childdocforwardprefix{#1}{#2}}
    \fi
    \expandafter
  \endgroup
  \childdoctmp
}
%    \end{macrocode}

%\iffalse
%</package>
%\fi
%
\endinput
|\\
|\childdocmain{|\textit{main}|}|\\
\end{tabular}
\end{center}
%
If |\jobname| does not match the argument \textit{main} of |\childdocmain|,
it is assumed that |\jobname| points to the child file to be compiled.
When using |\childdocmain| with the main file specified as argument,
it suffices to start a child file
with just |\input{|\textit{main}|}|
without loading of the package and using |\childdocof|.
If instead all processing is done
with the appropriate \textsf{childdoc} directives,
the argument of \textit{main} of |\childdocmain| can be empty.

An alternative version of the command line processing described
in \secref{sec:commandline} using the detection mechanism reads:
%
\begin{center}
|... -jobname "|\textit{target}|" "|[\textit{flags}]%
[|\def\jobname{|\textit{dest}|}|]|\input{|\textit{main}|}"|
\end{center}

%%%%%%%%%%%%%%%%%%%%%%%%%%%%%%%%%%%%%%%%%%%%%%%%%%%%%%%%%%%%%%%%%%%%%%%%%%%%%%%%
\subsection{Manual Code}
\label{sec:manual}

In case one cannot be certain whether the definitions file |childdoc.def|
is installed on the target \TeX{} distribution
and one prefers not to ship it,
it is conceivable to paste a few relevant commands into the sources.

To that end, drop all statements |% \iffalse
%
% childdoc.dtx Copyright (C) 2017-2018 Niklas Beisert
%
% This work may be distributed and/or modified under the
% conditions of the LaTeX Project Public License, either version 1.3
% of this license or (at your option) any later version.
% The latest version of this license is in
%   http://www.latex-project.org/lppl.txt
% and version 1.3 or later is part of all distributions of LaTeX
% version 2005/12/01 or later.
%
% This work has the LPPL maintenance status `maintained'.
%
% The Current Maintainer of this work is Niklas Beisert.
%
% This work consists of the files childdoc.dtx and childdoc.ins
% and the derived files childdoc.def and cdocsamp.tex with
% cdocsch1.tex, cdocsch2.tex, cdocsdrf.tex, cdocsfn1.tex, cdocsfn2.tex.
%
%<package>\ifdefined\childdocmain\endinput\fi
%<package>\ProvidesFile{childdoc.def}[2018/12/30 v2.0 child document driver]
%<samplemain>\ProvidesFile{cdocsamp.tex}[2018/12/30 v2.0 sample for childdoc]
%<*driver>
%\ProvidesFile{childdoc.drv}[2018/12/30 v2.0 childdoc reference manual file]
\PassOptionsToClass{10pt,a4paper}{article}
\documentclass{ltxdoc}

\usepackage[margin=35mm]{geometry}
\usepackage{hyperref}
\usepackage{hyperxmp}
\usepackage[usenames]{color}

\hypersetup{colorlinks=true}
\hypersetup{pdfstartview=FitH}
\hypersetup{pdfpagemode=UseNone}
\hypersetup{pdfsource={}}
\hypersetup{pdflang={en-UK}}
\hypersetup{pdfcopyright={Copyright 2017-2018 Niklas Beisert.
  This work may be distributed and/or modified under the
  conditions of the LaTeX Project Public License, either version 1.3
  of this license or (at your option) any later version.}}
\hypersetup{pdflicenseurl={http://www.latex-project.org/lppl.txt}}
\hypersetup{pdfcontactaddress={ETH Zurich, ITP, HIT K,
  Wolfgang-Pauli-Strasse 27}}
\hypersetup{pdfcontactpostcode={8093}}
\hypersetup{pdfcontactcity={Zurich}}
\hypersetup{pdfcontactcountry={Switzerland}}
\hypersetup{pdfcontactemail={nbeisert@itp.phys.ethz.ch}}
\hypersetup{pdfcontacturl={http://people.phys.ethz.ch/\xmptilde nbeisert/}}

\newcommand{\secref}[1]{\hyperref[#1]{section \ref*{#1}}}

\parskip1ex
\parindent0pt
\let\olditemize\itemize
\def\itemize{\olditemize\parskip0pt}

\begin{document}

\title{The \textsf{childdoc} Package}
\hypersetup{pdftitle={The childdoc Package}}
\author{Niklas Beisert\\[2ex]
  Institut f\"ur Theoretische Physik\\
  Eidgen\"ossische Technische Hochschule Z\"urich\\
  Wolfgang-Pauli-Strasse 27, 8093 Z\"urich, Switzerland\\[1ex]
  \href{mailto:nbeisert@itp.phys.ethz.ch}
  {\texttt{nbeisert@itp.phys.ethz.ch}}}
\hypersetup{pdfauthor={Niklas Beisert}}
\hypersetup{pdfsubject={Manual for the LaTeX2e Package childdoc}}
\date{30 December 2018, \textsf{v2.0}}
\maketitle

\begin{abstract}\noindent
\textsf{childdoc} is a \LaTeXe{} package
that enables the direct compilation
of document sections included by |\include|
to individual files.
\end{abstract}

\begingroup
\parskip0ex
\tableofcontents
\endgroup

%%%%%%%%%%%%%%%%%%%%%%%%%%%%%%%%%%%%%%%%%%%%%%%%%%%%%%%%%%%%%%%%%%%%%%%%%%%%%%%%
%%%%%%%%%%%%%%%%%%%%%%%%%%%%%%%%%%%%%%%%%%%%%%%%%%%%%%%%%%%%%%%%%%%%%%%%%%%%%%%%
\section{Introduction}

\LaTeX{} provides a mechanism to structure a large document (such as a book)
into a main file and several child files (containing the chapters)
using the |\include| command.
This mechanism is beneficial for documents
which span hundreds of pages in order to
make the source file(s) more manageable.
Moreover, compilation can be restricted to
selected child files by means of the |\includeonly| command.
The latter feature can be used to reduce the compilation time while editing
(this was significantly more useful in the earlier days of \LaTeX{})
or to generate a smaller document which is easier to navigate.
Another application of |\includeonly| is to generate
documents consisting of selected parts of the complete document.

However, there are a few drawbacks of the plain |\include| mechanism:
\begin{itemize}
\item
The child files cannot be compiled on their own,
they can only be compiled via the main file.
A naive editing environment
(such as a text editor with an option
to have the current file processed by \LaTeX)
may require one to switch to the main file before compiling;
attempting to compile the child file produces errors.
\item
The main file must be modified (each time)
to adjust the |\includeonly| command
to the present needs. This easily leaves the main file in a messy state.
\item
The generated document will always carry the filename
of the main document. This is inconvenient if
several child files are to be compiled and
to be kept for distribution.
\end{itemize}

The present package provides a simple interface
to make child files individually compilable by \LaTeX{}.
Compiling a child file then has the same effect as compiling
the main file with an |\includeonly| command
to select the appropriate child.
Moreover the generated document will carry the name of the child
rather than the main file.
This resolves all three above issues.

This feature is meant to make the editing of books,
thesis documents and lecture notes somewhat more convenient.
However, the package can also be used efficiently for
composing a series of documents (such as exercise sheets)
which are typically distributed individually.
It then assists the author in generating the individual documents
(potentially in different versions)
as well as a document containing the collected series.
Another application is in developing style files
or other kinds of included material
where compilation of the style file could redirect
to a sample or test file.

%%%%%%%%%%%%%%%%%%%%%%%%%%%%%%%%%%%%%%%%%%%%%%%%%%%%%%%%%%%%%%%%%%%%%%%%%%%%%%%%
%%%%%%%%%%%%%%%%%%%%%%%%%%%%%%%%%%%%%%%%%%%%%%%%%%%%%%%%%%%%%%%%%%%%%%%%%%%%%%%%
\section{Usage}

First of all, the package \textsf{childdoc} is \emph{not} a standard
\LaTeXe{} |.sty| style file! Therefore it needs to be invoked in
a non-standard way.

%%%%%%%%%%%%%%%%%%%%%%%%%%%%%%%%%%%%%%%%%%%%%%%%%%%%%%%%%%%%%%%%%%%%%%%%%%%%%%%%
\subsection{Included Files}
\label{sec:include}

%%%%%%%%%%%%%%%%%%%%%%%%%%%%%%%%%%%%%%%%
\DescribeMacro{\childdocmain}
To use the package, add the commands
\begin{center}
\begin{tabular}{l}
|\input{childdoc.def}|\\
|\childdocmain{}|\\
\end{tabular}
\end{center}
at the very top of the main \LaTeX{} file,
in particular \emph{before} the |\documentclass| statement!
The argument of |\childdocmain| should be left empty
(but it must be present).

%%%%%%%%%%%%%%%%%%%%%%%%%%%%%%%%%%%%%%%%
\DescribeMacro{\childdocof}
Furthermore, add the commands
\begin{center}
\begin{tabular}{l}
|\input{childdoc.def}|\\
|\childdocof{|\textit{main}|}|\\
\end{tabular}
\end{center}
at the top of every child file \textit{child}
which is included by |\include{|\textit{child}|}|
from within the main file
(or at least for those files to be compiled individually).
The argument \textit{main} must be the filename of the main file.

There are a couple of
considerations in setting up the main and child documents:

%%%%%%%%%%%%%%%%%%%%%%%%%%%%%%%%%%%%%%%%
\paragraph{Restrictions.}

Please note the following restrictions:
\begin{itemize}
\item
|\childdocmain| must be called with one argument \textit{main}
to ensure compatibility with earlier version of the package.
It must either be empty (|\childdocmain{}|)
or precisely match the filename of the main file in which it is specified.
See \secref{sec:detection} for further information.
\item
The filename \textit{main} must be specified without the |.tex| extension.
\item
The filename \textit{main} is case sensitive
(even in case-insensitive file systems)
due to internal string comparison.
\item
The argument \textit{main} should be fully expanded, it cannot be a macro.
\item
Subdirectories and special characters should be avoided in filenames.
\item
The command |\childdocmain{|\textit{main}|}| must be followed by a whitespace.
It should not be followed immediately by another command
or by a comment mark `|%|'.
This is because the \TeX{} parser reads the token immediately following
the argument of |\childdocmain| and puts it
at the beginning of every child section;
however, a white\-space is ignored.
\end{itemize}

%%%%%%%%%%%%%%%%%%%%%%%%%%%%%%%%%%%%%%%%
\paragraph{Content of Main File.}

It is advisable to place all content in the child files included by |\include|.
Any output contained in the main file will appear in all child documents
unless suppressed manually;
it cannot be suppressed automatically by the |\includeonly| directive
and thus should normally be avoided.
A method to include some content in the main file
by means of conditional processing is described in \secref{sec:conditional}.

%%%%%%%%%%%%%%%%%%%%%%%%%%%%%%%%%%%%%%%%
\paragraph{Page Numbering.}

When only a part of the document is compiled,
the appropriate numbering of pages
(as well as other status parameters)
is determined from the |.aux| files.
The latter contain information from previous passes.
However this information needs to propagate through
all intermediate child documents.
Therefore the page numbering in child documents may well
be inconsistent until the complete document is compiled at least once.

A useful (if unconventional) way to always ensure a consistent
page numbering is to restart the numbering in each child document
and denote the pages by `\textit{child}|.|\textit{page}'
where \textit{child} represents the chapter/section number of the child file.
This can be achieved by the command
|\numberwithin{page}{|\textit{child}|}|
of the \textsf{amsmath} package
where \textit{child} can be |chapter| or |section|
depending on the chosen structuring.
Alternatively, one can modify the macro |\thepage| appropriately
and reset the counter |page| at the start of each child file.

%%%%%%%%%%%%%%%%%%%%%%%%%%%%%%%%%%%%%%%%%%%%%%%%%%%%%%%%%%%%%%%%%%%%%%%%%%%%%%%%
\subsection{Conditional Processing}
\label{sec:conditional}

The package provides a mechanism to compile different versions
of a document. To customise the versions further some conditional processing
can come in handy to distinguish which version is being compiled.
The package provides two macros to describe the compilation context:

%%%%%%%%%%%%%%%%%%%%%%%%%%%%%%%%%%%%%%%%
\DescribeMacro{\ifchilddoc}
The conditional |\ifchilddoc| distinguishes between the compilation of
child documents and the main document:
%
\begin{center}
|\ifchilddoc |\textit{child-code}| |[|\||else |\textit{main-code}]| \||fi|
\end{center}

%%%%%%%%%%%%%%%%%%%%%%%%%%%%%%%%%%%%%%%%
\DescribeMacro{\childdocname}
\DescribeMacro{\childdocjob}
The macro |\childdocname| contains the filename (without extension)
of the main or child file being processed.
Note that |\childdocjob| will always contain the name of the main file.

%%%%%%%%%%%%%%%%%%%%%%%%%%%%%%%%%%%%%%%%
\paragraph{Title Page.}

Conditional processing can be used to include a title or banner page
in the main document when proper precautions are taken.
Importantly, the code in the main file should ensure that the page counter
(as well as other status parameters which are stored in the |.aux| files)
takes the same value after the conditional processing.
Otherwise the page numbers may take divergent values
depending on which part is compiled.

For example, a title page could be declared by:
%
\begin{center}
\begin{tabular}{l}
|\ifchilddoc\||else|\\
|\addtocounter{page}{-1}|\\
\textit{code for title page}\\
|\newpage|\\
|\||fi|
\end{tabular}
\end{center}
%
A banner page for the child documents can be generated by:
%
\begin{center}
\begin{tabular}{l}
|\ifchilddoc|\\
|\addtocounter{page}{-1}|\\
\textit{code for banner page}\\
|\newpage|\\
|\||fi|
\end{tabular}
\end{center}
%
Here one could write a message such as:
\begin{center}
|This is the part \childdocname{} of \childdocjob{}.|
\end{center}

%%%%%%%%%%%%%%%%%%%%%%%%%%%%%%%%%%%%%%%%%%%%%%%%%%%%%%%%%%%%%%%%%%%%%%%%%%%%%%%%
\subsection{Flags}
\label{sec:flags}

The package makes it easy to generate different versions
of the main or child documents.
To this end compilation flags can be defined
and assigned different default values.
They will be particularly useful in conjunction
with the forwarding mechanism described in \secref{sec:forward}.

For example, it may be useful to have a flag |\version|
which can be set to |draft| or |final|.
The document source will contain some conditional code
depending on the value of |\version|.
Suppose further, the flag should default to |final| for the main file
and to |draft| for child files
which is a natural assignment for editing the document.
This is achieved by placing the following code
in the preamble of the main document
(below the |\childdocmain| directive):
%
\begin{center}
\begin{tabular}{l}
|\ifchilddoc|\\
|\providecommand{\version}{draft}|\\
|\||else|\\
|\providecommand{\version}{final}|\\
|\||fi|
\end{tabular}
\end{center}
%
The definition by |\providecommand| makes sure
that previous definitions are not overwritten.
Further statements |\providecommand{\version}{...}|
can thus be added before the above code to override it.

For the main file, one might add a line
(between |\childdocmain| and the above block)
%
\begin{center}
|%\ifchilddoc\||else\providecommand{\version}{draft}\||fi|
\end{center}
%
which can be uncommented to produce a draft version.
Likewise one can add a line to the very top of a child file
(above the |\childdocof{|\textit{main}|}| directive)
%
\begin{center}
|%\providecommand{\version}{final}|
\end{center}
%
which can be uncommented to produce the final version of this child document.

%%%%%%%%%%%%%%%%%%%%%%%%%%%%%%%%%%%%%%%%%%%%%%%%%%%%%%%%%%%%%%%%%%%%%%%%%%%%%%%%
\subsection{Forwarding}
\label{sec:forward}

Different versions of the main or child documents
using compilation flags as described in \secref{sec:flags}
can be (permanently) stored in different files
for convenient compilation, viewing and distribution.
To this end, the package defines a command
to pass on compilation to a different file:

%%%%%%%%%%%%%%%%%%%%%%%%%%%%%%%%%%%%%%%%
\DescribeMacro{\childdocforward}
The command |\childdocforward| redirects processing to
another source file:
%
\begin{center}
\begin{tabular}{l}
|\input{childdoc.def}|\\
|\childdocforward[|\textit{main}|]{|\textit{dest}|}|\\
\end{tabular}
\end{center}
%
The argument \textit{dest} is the destination file
(without extension).
It should be the main file or one of the child files.
Note that further \textsf{childdoc} directives
such as |\childdocof| and |\childdocforward|
in the indicated file will be processed in this form.
The optional argument \textit{main}
passes on directly to the main file \textit{main}
while pretending to compile the child \textit{dest}.
This form behaves as if \textit{dest}
issues |\childdocof{|\textit{main}|}| right away,
and no further \textsf{childdoc} directives will be processed.

%%%%%%%%%%%%%%%%%%%%%%%%%%%%%%%%%%%%%%%%
\DescribeMacro{\...prefix}
In the alternative form |\childdocforwardprefix|,
%
\begin{center}
\begin{tabular}{l}
|\input{childdoc.def}|\\
|\childdocforwardprefix[|\textit{main}|]{|\textit{prefix}|}{|\textit{dest}|}|
\end{tabular}
\end{center}
%
the destination file is determined by a pattern
depending on the current file:
To make this work, the current file must be called
`{\textit{prefix}\hspace{0.2em}\textit{suffix}}'
with \textit{prefix} matching precisely the argument.
Processing is then passed on to the file
`{\textit{dest}\hspace{0.2em}\textit{suffix}}'.
Surely, the same effect is achieved by
directly specifying the
argument `{\textit{dest}\hspace{0.2em}\textit{suffix}}'
in the first form.
However, that requires to set up a different file
for each child. With the alternative form of the command
all these files can have exactly the same content
which simplifies setting them up and maintaining them.

For example, the following file |draft.tex|
with a compilation flag |\version| as described in \secref{sec:flags}
compiles the main document as a draft:
%
\begin{center}
\begin{tabular}{l}
|\def\version{draft}|\\
|\input{childdoc.def}|\\
|\childdocforward{|\textit{main}|}|
\end{tabular}
\end{center}
%
Likewise, the following files |final|\textit{nn}|.tex|
compile the final version of the child document
|child|\textit{nn}|.tex|:
%
\begin{center}
\begin{tabular}{l}
|\def\version{final}|\\
|\input{childdoc.def}|\\
|\childdocforwardprefix{final}{child}|
\end{tabular}
\end{center}
%

Note that when several versions of a main file and/or of each child file
are to be generated, it may be convenient to set up a |Makefile| or
shell script to automatise the process.

%%%%%%%%%%%%%%%%%%%%%%%%%%%%%%%%%%%%%%%%%%%%%%%%%%%%%%%%%%%%%%%%%%%%%%%%%%%%%%%%
\subsection{Command Line Processing}
\label{sec:commandline}

The effect of redirection files can also be achieved by invoking
the \LaTeX{} compiler with a more elaborate command line.
Most conveniently this should be done as part
of a shell script or a |Makefile|.

When using \textsf{childdoc} in the main file, the following
command lines effectively perform a redirection
(note that depending on the shell being used,
backslashes may have to be doubled: `|\|' $\to$ `|\\|'):
%
\begin{center}
|... -jobname "|\textit{target}|" |\\|"|[\textit{flags}]%
|\input{childdoc.def}\childdocforward[|\textit{main}|]{|\textit{dest}|}"|
\end{center}
%
Here \textit{target} is the name of the output file,
\textit{main} is the name of the main file
and \textit{dest} is the name of the main or child file to be processed
(all filenames without extensions).
The optional argument \textit{main} can be omitted
if \textit{main} matches \textit{dest}.
Optionally, compilation \textit{flags} can be defined via |\def| commands.
This command line makes the \TeX{} engine believe
it is compiling the file \textit{target}
whose content is specified as the latter parameter.
The provided code then forwards the processing to
\textit{main} or \textit{dest} as described in \secref{sec:forward}.

%%%%%%%%%%%%%%%%%%%%%%%%%%%%%%%%%%%%%%%%%%%%%%%%%%%%%%%%%%%%%%%%%%%%%%%%%%%%%%%%
\subsection{Include by Input}
\label{sec:input}

Including child documents by |\include| has some restrictions by design.
Most notably, the content of a child document always occupies
its own set of pages; pages cannot be shared between child documents.
Usually, this behaviour makes perfect sense
because each child document contain an essential part of the document.
However, in some situations it may be desirable to compose
a document from a collection of parts
without having mandatory page breaks between then.
For this case, the package
provides a mechanism to include parts
by |\input| which can also be processed individually.
However, by construction this mechanism
requires manual handling of the content to be output.

%%%%%%%%%%%%%%%%%%%%%%%%%%%%%%%%%%%%%%%%
\DescribeMacro{\ifchilddocmanual}
The main file should be prepared as usual, see \secref{sec:include}.
However, the document body must make a distinction
between processing of an individual part and of the main document, e.g.:
%
\begin{center}
\begin{tabular}{l}
|\ifchilddocmanual|\\
|\input{\childdocname}|\\
|\||else|\\
\textit{document body with }|\input{|\textit{part}|}|\\
|\||fi|
\end{tabular}
\end{center}
%
The conditional |\ifchilddocmanual| is true whenever
a part to be included by |\input| is being compiled,
and the name of the part is stored in |\childdocname|.

%%%%%%%%%%%%%%%%%%%%%%%%%%%%%%%%%%%%%%%%
\DescribeMacro{\childdocby}
Each part to be included by |\input| should start with:
%
\begin{center}
\begin{tabular}{l}
|\input{childdoc.def}|\\
|\childdocby{|\textit{main}|}|\\
\end{tabular}
\end{center}
%
The directive |\childdocby| is similar to |\childdocof|
described in \secref{sec:include},
but the subsequent selection of content must be done manually.
To that end, both |\ifchilddoc| and |\ifchilddocmanual|
will be true upon processing of a part,
and the name of the part is stored in |\childdocname|.
Note that |\jobname| will be set to the filename of the current part
so that each part receives an individual |.aux| file
that does not interfere with the |.aux| file(s) of the main document.
This behaviour can be altered by the alternative form
|\childdocby[*]{|\textit{main}|}| (with a non-empty optional argument)
which uses the |.aux| file of the main document
by setting |\jobname| to \textit{main}.

%%%%%%%%%%%%%%%%%%%%%%%%%%%%%%%%%%%%%%%%%%%%%%%%%%%%%%%%%%%%%%%%%%%%%%%%%%%%%%%%
\subsection{Driver Development}
\label{sec:driver}

The \textsf{childdoc} mechanism can also be use for the development
of definition files such as \LaTeX{} styles or classes.
This case differs from the above setup with multiple parts
included by |\include| in that no |\includeonly| should be invoked.
This can be achieved by starting the include file
(before |\ProvidesPackage|) with:
%
\begin{center}
\begin{tabular}{l}
|\input{childdoc.def}|\\
|\childdocforward{|\textit{main}|}|\\
\end{tabular}
\end{center}
%
or alternatively with:
%
\begin{center}
\begin{tabular}{l}
|\input{childdoc.def}|\\
|\childdocby{|\textit{main}|}|\\
\end{tabular}
\end{center}
%
Both forms have slightly different effects as described above.
The main file is prepared as usual, see \secref{sec:include}.

%%%%%%%%%%%%%%%%%%%%%%%%%%%%%%%%%%%%%%%%%%%%%%%%%%%%%%%%%%%%%%%%%%%%%%%%%%%%%%%%
\subsection{Legacy Detection}
\label{sec:detection}

The directive |\childdocmain| in the main file can detect
whether the complete document or merely a child is to be compiled
even without using the directive |\childdocof|.
This method is deprecated because it is less robust
and there is no compelling reason to use it;
it is merely provided for backward compatibility
and it may be removed in future versions.

If the detection mechanism is to be used,
it is mandatory to correctly specify
the filename of the main file as the argument of |\childdocmain|:
%
\begin{center}
\begin{tabular}{l}
|\input{childdoc.def}|\\
|\childdocmain{|\textit{main}|}|\\
\end{tabular}
\end{center}
%
If |\jobname| does not match the argument \textit{main} of |\childdocmain|,
it is assumed that |\jobname| points to the child file to be compiled.
When using |\childdocmain| with the main file specified as argument,
it suffices to start a child file
with just |\input{|\textit{main}|}|
without loading of the package and using |\childdocof|.
If instead all processing is done
with the appropriate \textsf{childdoc} directives,
the argument of \textit{main} of |\childdocmain| can be empty.

An alternative version of the command line processing described
in \secref{sec:commandline} using the detection mechanism reads:
%
\begin{center}
|... -jobname "|\textit{target}|" "|[\textit{flags}]%
[|\def\jobname{|\textit{dest}|}|]|\input{|\textit{main}|}"|
\end{center}

%%%%%%%%%%%%%%%%%%%%%%%%%%%%%%%%%%%%%%%%%%%%%%%%%%%%%%%%%%%%%%%%%%%%%%%%%%%%%%%%
\subsection{Manual Code}
\label{sec:manual}

In case one cannot be certain whether the definitions file |childdoc.def|
is installed on the target \TeX{} distribution
and one prefers not to ship it,
it is conceivable to paste a few relevant commands into the sources.

To that end, drop all statements |\input{childdoc.def}|
and perform the replacements as outlined below.
Instead of |\childdocmain{|\textit{main}|}| add the following code
to the top of the main file:
%
\begin{center}
\begin{tabular}{l}
|\||ifdefined\childdocname\endinput\||fi\newif\ifchilddoc|\\
|\edef\childdocname{\scantokens\expandafter{\jobname\noexpand}}|\\
|\def\childdocmain{|\textit{main}|}\||ifx\childdocmain\childdocname\||else|\\
|\childdoctrue\includeonly{\childdocname}\let\jobname\childdocmain\||fi|\\
\end{tabular}
\end{center}
%
Instead of |\childdocof{|\textit{main}|}| just include the main file
at the top of each child file:
%
\begin{center}
|\input{|\textit{main}|}|
\end{center}
%
A simple redirection |\childdocforward{|\textit{dest}|}| is achieved by:
%
\begin{center}
|\def\jobname{|\textit{dest}|}\input{\jobname}|
\end{center}
%
The redirection with prefix
|\childdocforwardprefix[|\textit{prefix}|]{|\textit{dest}|}|
is accomplished by:
%
\begin{center}
\begin{tabular}{l}
|{\edef\jobname{\scantokens\expandafter{\jobname\noexpand}}|\\
|\def\redirectjob |\textit{prefix}|#1~~~{\gdef\jobname{|\textit{dest}|#1}}|\\
|\expandafter\redirectjob\jobname~~~}\input{\jobname}|
\end{tabular}
\end{center}

In an alternative approach,
child documents can be compiled by a specific command line
without additional code or specific definitions:
%
\begin{center}
|... -jobname "|\textit{target}|" "|[\textit{flags}]%
|\includeonly{|\textit{dest}|}\input{|\textit{main}|}"|
\end{center}
%

%%%%%%%%%%%%%%%%%%%%%%%%%%%%%%%%%%%%%%%%%%%%%%%%%%%%%%%%%%%%%%%%%%%%%%%%%%%%%%%%
%%%%%%%%%%%%%%%%%%%%%%%%%%%%%%%%%%%%%%%%%%%%%%%%%%%%%%%%%%%%%%%%%%%%%%%%%%%%%%%%
\section{Information}

%%%%%%%%%%%%%%%%%%%%%%%%%%%%%%%%%%%%%%%%%%%%%%%%%%%%%%%%%%%%%%%%%%%%%%%%%%%%%%%%
\subsection{Copyright}

Copyright \copyright{} 2017--2018 Niklas Beisert

This work may be distributed and/or modified under the
conditions of the \LaTeX{} Project Public License, either version 1.3
of this license or (at your option) any later version.
The latest version of this license is in
  \url{http://www.latex-project.org/lppl.txt}
and version 1.3 or later is part of all distributions of \LaTeX{}
version 2005/12/01 or later.

This work has the LPPL maintenance status `maintained'.

The Current Maintainer of this work is Niklas Beisert.

This work consists of the files |README.txt|, |childdoc.ins| and |childdoc.dtx|
as well as the derived files |childdoc.def|, |cdocsamp.tex|
with |cdocsch1.tex|, |cdocsch2.tex|, |cdocspt3.tex|, |cdocspt4.tex|,
|cdocsdrf.tex|, |cdocsfn1.tex|, |cdocsfn2.tex|
as well as |childdoc.pdf|.

%%%%%%%%%%%%%%%%%%%%%%%%%%%%%%%%%%%%%%%%%%%%%%%%%%%%%%%%%%%%%%%%%%%%%%%%%%%%%%%%
\subsection{Files and Installation}

The package consists of the files:
%
\begin{center}
\begin{tabular}{ll}
    |README.txt|   & readme file \\
    |childdoc.ins| & installation file \\
    |childdoc.dtx| & source file \\
    |childdoc.def| & definition file \\
    |cdocsamp.tex| & sample main file \\
    |cdocsch1.tex| & sample include file \\
    |cdocsch2.tex| & sample include file \\
    |cdocspt3.tex| & sample part file \\
    |cdocspt4.tex| & sample part file \\
    |cdocsdrf.tex| & sample redirection file \\
    |cdocsfn1.tex| & sample redirection file \\
    |cdocsfn2.tex| & sample redirection file \\
    |childdoc.pdf| & manual
\end{tabular}
\end{center}
%
The distribution consists of the files
|README.txt|, |childdoc.ins| and |childdoc.dtx|.
%
\begin{itemize}
\item
Run (pdf)\LaTeX{} on |childdoc.dtx|
to compile the manual |childdoc.pdf| (this file).
\item
Run \LaTeX{} on |childdoc.ins| to create the definitions file |childdoc.def|
and the sample |cdocsamp.tex| with include files
|cdocsch1.tex|, |cdocsch2.tex|, |cdocspt3.tex|, |cdocspt4.tex|,
|cdocsdrf.tex|, |cdocsfn1.tex|, |cdocsfn2.tex|.
Then copy the file |childdoc.def| to an appropriate directory of your \LaTeX{}
distribution, e.g.\ \textit{texmf-root}|/tex/latex/childdoc|.
\end{itemize}

%%%%%%%%%%%%%%%%%%%%%%%%%%%%%%%%%%%%%%%%%%%%%%%%%%%%%%%%%%%%%%%%%%%%%%%%%%%%%%%%
\subsection{Related CTAN Packages}

There are several other packages which offer a similar functionality:
%
\begin{itemize}
\item
The packages
\href{http://ctan.org/pkg/docmute}{\textsf{docmute}},
\href{http://ctan.org/pkg/includex}{\textsf{includex}} and
\href{http://ctan.org/pkg/standalone}{\textsf{standalone}}
provide commands to include only the document body of
a child file thus allowing both files to be compiled individually.
\item
The packages \href{http://ctan.org/pkg/subdocs}{\textsf{subdocs}}
and \href{http://ctan.org/pkg/subfiles}{\textsf{subfiles}}
provide structures in which the main and child documents can be
encapsulated and allowing them to be compiled individually.
The inclusion mechanism is different from the conventional |\include|.
\item
The package \href{http://ctan.org/pkg/combine}{\textsf{combine}}
is an elaborate solution to combine several documents into one.
\end{itemize}
%
See also the CTAN topic \href{http://ctan.org/topic/subdocs}{\textsf{subdocs}}
for further related packages.
The present package differs from the above solutions in that
a document structure constructed with the conventional |\include| mechanism
just needs two extra commands at the top of every file
such that all constituent files can be compiled individually.

%%%%%%%%%%%%%%%%%%%%%%%%%%%%%%%%%%%%%%%%%%%%%%%%%%%%%%%%%%%%%%%%%%%%%%%%%%%%%%%%
%\subsection{Feature Suggestions}
%
%The following is a list of features which may be useful for future
%versions of this package:
%%
%\begin{itemize}
%\item
%\ldots
%\end{itemize}

%%%%%%%%%%%%%%%%%%%%%%%%%%%%%%%%%%%%%%%%%%%%%%%%%%%%%%%%%%%%%%%%%%%%%%%%%%%%%%%%
\subsection{Revision History}

%%%%%%%%%%%%%%%%%%%%%%%%%%%%%%%%%%%%%%%%
\paragraph{v2.0:} 2018/12/30

\begin{itemize}
\item
immediate forward processing
\item
added |\childdocby| mechanism
\item
manual restructured
\end{itemize}

%%%%%%%%%%%%%%%%%%%%%%%%%%%%%%%%%%%%%%%%
\paragraph{v1.6:} 2018/01/17

\begin{itemize}
\item
application for development of include files
\item
corrections to manual
\end{itemize}

%%%%%%%%%%%%%%%%%%%%%%%%%%%%%%%%%%%%%%%%
\paragraph{v1.5:} 2017/05/21

\begin{itemize}
\item
more complete structuring introduced
\item
|\childdocof| introduced
\item
|\childdoc| renamed to |\childdocmain|
\item
|\childredirect| renamed to |\childdocforward| and |\childdocforwardprefix|
and functionality expanded
\end{itemize}

%%%%%%%%%%%%%%%%%%%%%%%%%%%%%%%%%%%%%%%%
\paragraph{v1.0:} 2017/04/27

\begin{itemize}
\item
manual and install package
\item
first version published on CTAN
\end{itemize}

%%%%%%%%%%%%%%%%%%%%%%%%%%%%%%%%%%%%%%%%
\paragraph{v0.6:} 2017/04/26

\begin{itemize}
\item
redirection mechanism added
\end{itemize}

%%%%%%%%%%%%%%%%%%%%%%%%%%%%%%%%%%%%%%%%
\paragraph{v0.5:} 2017/04/26

\begin{itemize}
\item
functionality in definition file
\end{itemize}


%%%%%%%%%%%%%%%%%%%%%%%%%%%%%%%%%%%%%%%%%%%%%%%%%%%%%%%%%%%%%%%%%%%%%%%%%%%%%%%%
%%%%%%%%%%%%%%%%%%%%%%%%%%%%%%%%%%%%%%%%%%%%%%%%%%%%%%%%%%%%%%%%%%%%%%%%%%%%%%%%
%%%%%%%%%%%%%%%%%%%%%%%%%%%%%%%%%%%%%%%%%%%%%%%%%%%%%%%%%%%%%%%%%%%%%%%%%%%%%%%%
\appendix

\settowidth\MacroIndent{\rmfamily\scriptsize 000\ }

 \DocInput{childdoc.dtx}

\end{document}
%</driver>
% \fi
%
% %%%%%%%%%%%%%%%%%%%%%%%%%%%%%%%%%%%%%%%%%%%%%%%%%%%%%%%%%%%%%%%%%%%%%%%%%%%%%%
% %%%%%%%%%%%%%%%%%%%%%%%%%%%%%%%%%%%%%%%%%%%%%%%%%%%%%%%%%%%%%%%%%%%%%%%%%%%%%%
% \section{Sample}
%\iffalse
%<*samplemain>
%\fi
%
% The following presents a sample document
% with two chapters, two parts, a title page,
% a compile flag as well as three forwarding files to set the flag.
% It consists of eight |.tex| files:
% \begin{center}
% \begin{tabular}{ll}
% |cdocsamp.tex|&main file\\
% |cdocsch1.tex|&include file for chapter 1\\
% |cdocsch2.tex|&include file for chapter 2\\
% |cdocspt3.tex|&include file for part 3\\
% |cdocspt4.tex|&include file for part 4\\
% |cdocsdrf.tex|&forwarding file for main file in draft mode\\
% |cdocsfi1.tex|&forwarding file for final version of chapter 1\\
% |cdocsfi2.tex|&forwarding file for final version of chapter 2\\
% \end{tabular}
% \end{center}
% Each of the eight files can be compiled directly by the \LaTeX{} compiler.
%
% %%%%%%%%%%%%%%%%%%%%%%%%%%%%%%%%%%%%%%
% \paragraph{Main File.}
%
% The main file is called |cdocsamp.tex|.
%
% Load the \textsf{childdoc} definitions and
% declare the filename for the main document:
%    \begin{macrocode}
\input{childdoc.def}
\childdocmain{}
%    \end{macrocode}

% Optional override for |\version| flag:
%    \begin{macrocode}
%%\ifchilddoc\else\providecommand{\version}{draft}\fi
%    \end{macrocode}

% Define the default values for the |\version| flag
% (|final| for the main file and |draft| for childs):
%    \begin{macrocode}
\ifchilddoc
\providecommand{\version}{draft}
\else
\providecommand{\version}{final}
\fi
%    \end{macrocode}

% Load the standard document class:
%    \begin{macrocode}
\documentclass[12pt]{article}
%    \end{macrocode}

% Start the document body:
%    \begin{macrocode}
\begin{document}
%    \end{macrocode}

% Declare a title page.
% Print title, part of document being processed and version flag:
%    \begin{macrocode}
\addtocounter{page}{-1}
\begin{center}
{\LARGE\bfseries{}childdoc example\par}
\vspace{1cm}
\ifchilddoc
\ifchilddocmanual part\else chapter\fi:
`\childdocname' of `\childdocjob'\par
\else
main document: `\childdocjob'\par
\fi
version: \version\par
\end{center}
\newpage
%    \end{macrocode}

% Manually include selected file,
% otherwise process as usual:
%    \begin{macrocode}
\ifchilddocmanual
\section*{part `\childdocname'}
\input{\childdocname}
\else
%    \end{macrocode}

% Include the two chapters:
%    \begin{macrocode}
\include{cdocsch1}
\include{cdocsch2}
%    \end{macrocode}

% Include the two parts unless only chapters should be displayed:
%    \begin{macrocode}
\ifchilddoc\else
\section{part three}
\input{cdocspt3}
\section{part four}
\input{cdocspt4}
\fi
%    \end{macrocode}

% Process as usual until here:
%    \begin{macrocode}
\fi
%    \end{macrocode}

% End of document body:
%    \begin{macrocode}
\end{document}
%    \end{macrocode}
%\iffalse
%</samplemain>
%\fi
%
% %%%%%%%%%%%%%%%%%%%%%%%%%%%%%%%%%%%%%%
% \paragraph{Chapter Include Files.}
%
% The include files are called |cdocsch1.tex| and |cdocsch2.tex|.
%
%\iffalse
%<*samplechap1|samplechap2>
%\fi

% Optional override for |\version| flag:
%    \begin{macrocode}
%%\providecommand{\version}{final}
%    \end{macrocode}

% Include the main document:
%    \begin{macrocode}
\input{childdoc.def}
\childdocof{cdocsamp}
%    \end{macrocode}

%\iffalse
%</samplechap1|samplechap2>
%\fi
%
%\iffalse
%<*samplechap1>
%\fi
% Some text for chapter 1:
%    \begin{macrocode}
\section{one}
some text in chapter one
%    \end{macrocode}

%\iffalse
%</samplechap1>
%\fi
% Some text for chapter 2:
%\iffalse
%<*samplechap2>
%\fi
%    \begin{macrocode}
\section{two}
more text in chapter two
%    \end{macrocode}

%\iffalse
%</samplechap2>
%\fi
%
% %%%%%%%%%%%%%%%%%%%%%%%%%%%%%%%%%%%%%%
% \paragraph{Part Include Files.}
%
% The include files are called |cdocspt3.tex| and |cdocspt4.tex|.
%
%\iffalse
%<*samplepart3|samplepart4>
%\fi

% Optional override for |\version| flag:
%    \begin{macrocode}
%%\providecommand{\version}{final}
%    \end{macrocode}

% Include the main document:
%    \begin{macrocode}
\input{childdoc.def}
\childdocby{cdocsamp}
%    \end{macrocode}

%\iffalse
%</samplepart3|samplepart4>
%\fi
%
%\iffalse
%<*samplepart3>
%\fi
% Some text for part 3:
%    \begin{macrocode}
some text in part three
%    \end{macrocode}

%\iffalse
%</samplepart3>
%\fi
% Some text for part 4:
%\iffalse
%<*samplepart4>
%\fi
%    \begin{macrocode}
more text in part four
%    \end{macrocode}

%\iffalse
%</samplepart4>
%\fi
%
% %%%%%%%%%%%%%%%%%%%%%%%%%%%%%%%%%%%%%%
% \paragraph{Forwarding for a Complete Draft.}
%
% The following forwarding file |cdocsdrf.tex|
% compiles the main document in draft mode:
%\iffalse
%<*sampledraft>
%\fi
%    \begin{macrocode}
\def\version{draft}
\input{childdoc.def}
\childdocforward{cdocsamp}
%    \end{macrocode}

%\iffalse
%</sampledraft>
%\fi
%
% %%%%%%%%%%%%%%%%%%%%%%%%%%%%%%%%%%%%%%
% \paragraph{Forwarding for Final Version of the Chapters.}
%
% The following forwarding files |cdocsfn1.tex| and |cdocsfn2.tex|
% (with identical content)
% compile the final versions of the child documents
% |cdocsch1.tex| and |cdocsch2.tex|, respectively:
%\iffalse
%<*samplefinal>
%\fi
%    \begin{macrocode}
\def\version{final}
\input{childdoc.def}
\childdocforwardprefix[cdocsamp]{cdocsfn}{cdocsch}
%    \end{macrocode}

%\iffalse
%</samplefinal>
%\fi
%
% %%%%%%%%%%%%%%%%%%%%%%%%%%%%%%%%%%%%%%
% \paragraph{Command Line Processing.}
%
% The following three command lines generate the output files
% |cdocscld|, |cdocscl1| and |cdocscl2|
% which should be identical to
% |cdocsdrf|, |cdocsch1| and |cdocsfn2|, respectively:
% \begin{center}
% \begin{tabular}{l}
% |latex -jobname cdocscld \|\\
% |  "\def\version{draft}\input{childdoc.def}\childdocforward{cdocsamp}"|\\
% |latex -jobname cdocscl1 \|\\
% |  "\input{childdoc.def}\childdocforward[cdocsamp]{cdocsch1}"|\\
% |latex -jobname cdocscl2 \|\\
% |  "\def\version{final}\input{childdoc.def}\childdocforward{cdocsch2}"|
% \end{tabular}
% \end{center}
% Note that the trailing backslash on each first line
% merely continues the input to the second line
% (for convenient cut ant paste).
% Furthermore, the command |latex| can be replaced by any
% of its alternative versions such as |pdflatex|.
%
% %%%%%%%%%%%%%%%%%%%%%%%%%%%%%%%%%%%%%%%%%%%%%%%%%%%%%%%%%%%%%%%%%%%%%%%%%%%%%%
% %%%%%%%%%%%%%%%%%%%%%%%%%%%%%%%%%%%%%%%%%%%%%%%%%%%%%%%%%%%%%%%%%%%%%%%%%%%%%%
% \section{Implementation}
%\iffalse
%<*package>
%\fi
%
% This section describes the definitions file |childdoc.def|.

% The definitions cannot be loaded using |\usepackage| or |\RequirePackage|
% which has a mechanism to prevent loading a style file more than once.
% When loading the definitions by means of |\input|
% multiple instances have to be prevented manually:
%\iffalse
%This code needs to be before the `\ProvidesFile' directive
%which is defined at the beginning of this file.
%Therefore it is also placed there and commented out here.
%</package>
%<*discard>
%\fi
%    \begin{macrocode}
\ifdefined\childdocmain\endinput\fi
%    \end{macrocode}
%\iffalse
%</discard>
%<*package>
%\fi
%
% \macro{\ifchilddoc}
% \macro{\ifchilddocmanual}
% The conditional |\ifchilddoc| tells whether a
% child (true) or main (false) document is being compiled.
% The conditional |\ifchilddocmanual| tells whether
% the |\includeonly| mechanism is used (false) or
% the selection of child files must be performed manually (true).
% The definitions initialise to false:
%    \begin{macrocode}
\newif\ifchilddoc
\newif\ifchilddocmanual
%    \end{macrocode}

% \macro{\childdocname}
% \macro{\childdocjob}
% The macro |\childdocname| stores the name of the main document
% to be compiled. The macro |\childdocjob| stores the name of
% the document on which the \LaTeX{} compiler was originally invoked.
% The content of |\jobname| cannot be compared
% to filenames specified in the source due to different catcodes.
% The following code rescans |\jobname|, stores the result
% in |\childdocname| and saves a copy in |\childdocjob|:
%    \begin{macrocode}
\edef\childdocname{\scantokens\expandafter{\jobname\noexpand}}
\let\childdocjob\childdocname
%    \end{macrocode}

% \macro{\childdocdisable}
% The macro |\childdocdisable| prevents the main file
% from being processed more than once.
% At this stage, the main document command |\childdocmain|
% is assumed to be called once again where it should do nothing.
% Any subsequent call to it should prevent
% a secondary processing of the main document
% It overwrites the forwarding commands
% |\childdocof| and |\childdocforward|
% with empty macros to prevent further inclusions of the main document:
%    \begin{macrocode}
\newcommand{\childdocdisable}
{
  \renewcommand{\childdocmain}[1]{\renewcommand{\childdocmain}[1]{\endinput}}
  \renewcommand{\childdocof}[1]{}
  \renewcommand{\childdocby}[2][]{}
  \renewcommand{\childdocforward}[2][]{}
  \renewcommand{\childdocdisable}{}
}
%    \end{macrocode}

% \macro{\childdocmain}
% The macro |\childdocmain| is to be called at the top of the main file
% with nothing or the main filename (without extension) as argument.
% First, it breaks loops.
% If the argument is not empty and does not match |\childdocname|
% (which is set by the first inclusion of |childdoc.def|),
% |\ifchilddoc| is set to true, |\includeonly| is applied to the child file
% and |\jobname| is set to the main file
% (for proper handling of |.aux| files):
%    \begin{macrocode}
\newcommand{\childdocmain}[1]
{
  \childdocdisable\childdocmain{}
  \if?#1?\else
    \begingroup
      \def\childdoctmp{#1}
      \ifx\childdoctmp\childdocname
        \def\childdoctmp{}
      \else
        \def\childdoctmp
        {
          \childdoctrue
          \includeonly{\childdocname}
          \def\childdocjob{#1}
          \def\jobname{#1}
        }
      \fi
      \expandafter
    \endgroup
    \childdoctmp
  \fi
}
%    \end{macrocode}

% \macro{\childdocof}
% The command |\childdocof| redirects
% compilation to the main file |#1|.
%    \begin{macrocode}
\newcommand{\childdocof}[1]
{
  \childdocdisable
  \childdoctrue
  \includeonly{\childdocname}
  \def\jobname{#1}
  \def\childdocjob{#1}
  \input{#1}
}
%    \end{macrocode}

% \macro{\childdocby}
% The command |\childdocby| ....
%    \begin{macrocode}
\newcommand{\childdocby}[2][]
{
  \childdocdisable
  \childdoctrue
  \childdocmanualtrue
  \if?#1?\else
    \def\jobname{#2}
  \fi
  \def\childdocjob{#2}
  \input{#2}
  \endinput
}
%    \end{macrocode}

% \macro{\childdocforward}
% The command |\childdocforward| redirects
% compilation to the main file or
% (if the optional argument is given) a child file.
% Parameters are set as if the main file
% or a child file starting with |\childdocof| was compiled.
% Then compilation is handed over to the main file:
%    \begin{macrocode}
\newcommand{\childdocforward}[2][]
{
  \begingroup
    \if?#1?
      \def\childdoctmp
      {
        \def\childdocname{#2}
        \def\childdocjob{#2}
        \def\jobname{#2}
        \input{#2}
        \endinput
      }
    \else
      \def\childdoctmp
      {
        \childdocdisable
        \def\childdocname{#2}
        \childdoctrue
        \includeonly{#2}
        \def\childdocjob{#1}
        \def\jobname{#1}
        \input{#1}
        \endinput
      }
    \fi
    \expandafter
  \endgroup
  \childdoctmp
}
%    \end{macrocode}

% \macro{\childdocforwardprefix}
% The command |\childdocforwardprefix| redirects
% compilation to the main or a child file by means of a pattern.
% The prefix |#1| in the current filename is replaced by |#2|
% and the suffix of the current filename is kept
% (it is assumed that the filename does not contain the substring `|~~~|'
% which is used as a delimiter).
% Compilation is handed over to the new file by |\childdocforward|:
%    \begin{macrocode}
\newcommand{\childdocforwardprefix}[3][]
{
  \begingroup
    \def\childdocextract #2##1~~~{\def\childdoctmp{\childdocforward[#1]{#3##1}}}
    \expandafter\childdocextract\childdocname~~~
    \expandafter
  \endgroup
  \childdoctmp
}
%    \end{macrocode}

% \macro{\childdoc}
% The deprecated macro |\childdoc| is a legacy version of |\childdocmain|:
%    \begin{macrocode}
\newcommand{\childdoc}{\childdocmain}
%    \end{macrocode}

% \macro{\childdocredirect}
% The deprecated macro |\childdocredirect| is a legacy version
% of |\childdocforward| and |\childdocforwardprefix|:
%    \begin{macrocode}
\newcommand{\childdocredirect}[2][]
{
  \begingroup
    \if?#1?
      \def\childdoctmp{\childdocforward{#2}}
    \else
      \def\childdoctmp{\childdocforwardprefix{#1}{#2}}
    \fi
    \expandafter
  \endgroup
  \childdoctmp
}
%    \end{macrocode}

%\iffalse
%</package>
%\fi
%
\endinput
|
and perform the replacements as outlined below.
Instead of |\childdocmain{|\textit{main}|}| add the following code
to the top of the main file:
%
\begin{center}
\begin{tabular}{l}
|\||ifdefined\childdocname\endinput\||fi\newif\ifchilddoc|\\
|\edef\childdocname{\scantokens\expandafter{\jobname\noexpand}}|\\
|\def\childdocmain{|\textit{main}|}\||ifx\childdocmain\childdocname\||else|\\
|\childdoctrue\includeonly{\childdocname}\let\jobname\childdocmain\||fi|\\
\end{tabular}
\end{center}
%
Instead of |\childdocof{|\textit{main}|}| just include the main file
at the top of each child file:
%
\begin{center}
|\input{|\textit{main}|}|
\end{center}
%
A simple redirection |\childdocforward{|\textit{dest}|}| is achieved by:
%
\begin{center}
|\def\jobname{|\textit{dest}|}\input{\jobname}|
\end{center}
%
The redirection with prefix
|\childdocforwardprefix[|\textit{prefix}|]{|\textit{dest}|}|
is accomplished by:
%
\begin{center}
\begin{tabular}{l}
|{\edef\jobname{\scantokens\expandafter{\jobname\noexpand}}|\\
|\def\redirectjob |\textit{prefix}|#1~~~{\gdef\jobname{|\textit{dest}|#1}}|\\
|\expandafter\redirectjob\jobname~~~}\input{\jobname}|
\end{tabular}
\end{center}

In an alternative approach,
child documents can be compiled by a specific command line
without additional code or specific definitions:
%
\begin{center}
|... -jobname "|\textit{target}|" "|[\textit{flags}]%
|\includeonly{|\textit{dest}|}\input{|\textit{main}|}"|
\end{center}
%

%%%%%%%%%%%%%%%%%%%%%%%%%%%%%%%%%%%%%%%%%%%%%%%%%%%%%%%%%%%%%%%%%%%%%%%%%%%%%%%%
%%%%%%%%%%%%%%%%%%%%%%%%%%%%%%%%%%%%%%%%%%%%%%%%%%%%%%%%%%%%%%%%%%%%%%%%%%%%%%%%
\section{Information}

%%%%%%%%%%%%%%%%%%%%%%%%%%%%%%%%%%%%%%%%%%%%%%%%%%%%%%%%%%%%%%%%%%%%%%%%%%%%%%%%
\subsection{Copyright}

Copyright \copyright{} 2017--2018 Niklas Beisert

This work may be distributed and/or modified under the
conditions of the \LaTeX{} Project Public License, either version 1.3
of this license or (at your option) any later version.
The latest version of this license is in
  \url{http://www.latex-project.org/lppl.txt}
and version 1.3 or later is part of all distributions of \LaTeX{}
version 2005/12/01 or later.

This work has the LPPL maintenance status `maintained'.

The Current Maintainer of this work is Niklas Beisert.

This work consists of the files |README.txt|, |childdoc.ins| and |childdoc.dtx|
as well as the derived files |childdoc.def|, |cdocsamp.tex|
with |cdocsch1.tex|, |cdocsch2.tex|, |cdocspt3.tex|, |cdocspt4.tex|,
|cdocsdrf.tex|, |cdocsfn1.tex|, |cdocsfn2.tex|
as well as |childdoc.pdf|.

%%%%%%%%%%%%%%%%%%%%%%%%%%%%%%%%%%%%%%%%%%%%%%%%%%%%%%%%%%%%%%%%%%%%%%%%%%%%%%%%
\subsection{Files and Installation}

The package consists of the files:
%
\begin{center}
\begin{tabular}{ll}
    |README.txt|   & readme file \\
    |childdoc.ins| & installation file \\
    |childdoc.dtx| & source file \\
    |childdoc.def| & definition file \\
    |cdocsamp.tex| & sample main file \\
    |cdocsch1.tex| & sample include file \\
    |cdocsch2.tex| & sample include file \\
    |cdocspt3.tex| & sample part file \\
    |cdocspt4.tex| & sample part file \\
    |cdocsdrf.tex| & sample redirection file \\
    |cdocsfn1.tex| & sample redirection file \\
    |cdocsfn2.tex| & sample redirection file \\
    |childdoc.pdf| & manual
\end{tabular}
\end{center}
%
The distribution consists of the files
|README.txt|, |childdoc.ins| and |childdoc.dtx|.
%
\begin{itemize}
\item
Run (pdf)\LaTeX{} on |childdoc.dtx|
to compile the manual |childdoc.pdf| (this file).
\item
Run \LaTeX{} on |childdoc.ins| to create the definitions file |childdoc.def|
and the sample |cdocsamp.tex| with include files
|cdocsch1.tex|, |cdocsch2.tex|, |cdocspt3.tex|, |cdocspt4.tex|,
|cdocsdrf.tex|, |cdocsfn1.tex|, |cdocsfn2.tex|.
Then copy the file |childdoc.def| to an appropriate directory of your \LaTeX{}
distribution, e.g.\ \textit{texmf-root}|/tex/latex/childdoc|.
\end{itemize}

%%%%%%%%%%%%%%%%%%%%%%%%%%%%%%%%%%%%%%%%%%%%%%%%%%%%%%%%%%%%%%%%%%%%%%%%%%%%%%%%
\subsection{Related CTAN Packages}

There are several other packages which offer a similar functionality:
%
\begin{itemize}
\item
The packages
\href{http://ctan.org/pkg/docmute}{\textsf{docmute}},
\href{http://ctan.org/pkg/includex}{\textsf{includex}} and
\href{http://ctan.org/pkg/standalone}{\textsf{standalone}}
provide commands to include only the document body of
a child file thus allowing both files to be compiled individually.
\item
The packages \href{http://ctan.org/pkg/subdocs}{\textsf{subdocs}}
and \href{http://ctan.org/pkg/subfiles}{\textsf{subfiles}}
provide structures in which the main and child documents can be
encapsulated and allowing them to be compiled individually.
The inclusion mechanism is different from the conventional |\include|.
\item
The package \href{http://ctan.org/pkg/combine}{\textsf{combine}}
is an elaborate solution to combine several documents into one.
\end{itemize}
%
See also the CTAN topic \href{http://ctan.org/topic/subdocs}{\textsf{subdocs}}
for further related packages.
The present package differs from the above solutions in that
a document structure constructed with the conventional |\include| mechanism
just needs two extra commands at the top of every file
such that all constituent files can be compiled individually.

%%%%%%%%%%%%%%%%%%%%%%%%%%%%%%%%%%%%%%%%%%%%%%%%%%%%%%%%%%%%%%%%%%%%%%%%%%%%%%%%
%\subsection{Feature Suggestions}
%
%The following is a list of features which may be useful for future
%versions of this package:
%%
%\begin{itemize}
%\item
%\ldots
%\end{itemize}

%%%%%%%%%%%%%%%%%%%%%%%%%%%%%%%%%%%%%%%%%%%%%%%%%%%%%%%%%%%%%%%%%%%%%%%%%%%%%%%%
\subsection{Revision History}

%%%%%%%%%%%%%%%%%%%%%%%%%%%%%%%%%%%%%%%%
\paragraph{v2.0:} 2018/12/30

\begin{itemize}
\item
immediate forward processing
\item
added |\childdocby| mechanism
\item
manual restructured
\end{itemize}

%%%%%%%%%%%%%%%%%%%%%%%%%%%%%%%%%%%%%%%%
\paragraph{v1.6:} 2018/01/17

\begin{itemize}
\item
application for development of include files
\item
corrections to manual
\end{itemize}

%%%%%%%%%%%%%%%%%%%%%%%%%%%%%%%%%%%%%%%%
\paragraph{v1.5:} 2017/05/21

\begin{itemize}
\item
more complete structuring introduced
\item
|\childdocof| introduced
\item
|\childdoc| renamed to |\childdocmain|
\item
|\childredirect| renamed to |\childdocforward| and |\childdocforwardprefix|
and functionality expanded
\end{itemize}

%%%%%%%%%%%%%%%%%%%%%%%%%%%%%%%%%%%%%%%%
\paragraph{v1.0:} 2017/04/27

\begin{itemize}
\item
manual and install package
\item
first version published on CTAN
\end{itemize}

%%%%%%%%%%%%%%%%%%%%%%%%%%%%%%%%%%%%%%%%
\paragraph{v0.6:} 2017/04/26

\begin{itemize}
\item
redirection mechanism added
\end{itemize}

%%%%%%%%%%%%%%%%%%%%%%%%%%%%%%%%%%%%%%%%
\paragraph{v0.5:} 2017/04/26

\begin{itemize}
\item
functionality in definition file
\end{itemize}


%%%%%%%%%%%%%%%%%%%%%%%%%%%%%%%%%%%%%%%%%%%%%%%%%%%%%%%%%%%%%%%%%%%%%%%%%%%%%%%%
%%%%%%%%%%%%%%%%%%%%%%%%%%%%%%%%%%%%%%%%%%%%%%%%%%%%%%%%%%%%%%%%%%%%%%%%%%%%%%%%
%%%%%%%%%%%%%%%%%%%%%%%%%%%%%%%%%%%%%%%%%%%%%%%%%%%%%%%%%%%%%%%%%%%%%%%%%%%%%%%%
\appendix

\settowidth\MacroIndent{\rmfamily\scriptsize 000\ }

 \DocInput{childdoc.dtx}

\end{document}
%</driver>
% \fi
%
% %%%%%%%%%%%%%%%%%%%%%%%%%%%%%%%%%%%%%%%%%%%%%%%%%%%%%%%%%%%%%%%%%%%%%%%%%%%%%%
% %%%%%%%%%%%%%%%%%%%%%%%%%%%%%%%%%%%%%%%%%%%%%%%%%%%%%%%%%%%%%%%%%%%%%%%%%%%%%%
% \section{Sample}
%\iffalse
%<*samplemain>
%\fi
%
% The following presents a sample document
% with two chapters, two parts, a title page,
% a compile flag as well as three forwarding files to set the flag.
% It consists of eight |.tex| files:
% \begin{center}
% \begin{tabular}{ll}
% |cdocsamp.tex|&main file\\
% |cdocsch1.tex|&include file for chapter 1\\
% |cdocsch2.tex|&include file for chapter 2\\
% |cdocspt3.tex|&include file for part 3\\
% |cdocspt4.tex|&include file for part 4\\
% |cdocsdrf.tex|&forwarding file for main file in draft mode\\
% |cdocsfi1.tex|&forwarding file for final version of chapter 1\\
% |cdocsfi2.tex|&forwarding file for final version of chapter 2\\
% \end{tabular}
% \end{center}
% Each of the eight files can be compiled directly by the \LaTeX{} compiler.
%
% %%%%%%%%%%%%%%%%%%%%%%%%%%%%%%%%%%%%%%
% \paragraph{Main File.}
%
% The main file is called |cdocsamp.tex|.
%
% Load the \textsf{childdoc} definitions and
% declare the filename for the main document:
%    \begin{macrocode}
% \iffalse
%
% childdoc.dtx Copyright (C) 2017-2018 Niklas Beisert
%
% This work may be distributed and/or modified under the
% conditions of the LaTeX Project Public License, either version 1.3
% of this license or (at your option) any later version.
% The latest version of this license is in
%   http://www.latex-project.org/lppl.txt
% and version 1.3 or later is part of all distributions of LaTeX
% version 2005/12/01 or later.
%
% This work has the LPPL maintenance status `maintained'.
%
% The Current Maintainer of this work is Niklas Beisert.
%
% This work consists of the files childdoc.dtx and childdoc.ins
% and the derived files childdoc.def and cdocsamp.tex with
% cdocsch1.tex, cdocsch2.tex, cdocsdrf.tex, cdocsfn1.tex, cdocsfn2.tex.
%
%<package>\ifdefined\childdocmain\endinput\fi
%<package>\ProvidesFile{childdoc.def}[2018/12/30 v2.0 child document driver]
%<samplemain>\ProvidesFile{cdocsamp.tex}[2018/12/30 v2.0 sample for childdoc]
%<*driver>
%\ProvidesFile{childdoc.drv}[2018/12/30 v2.0 childdoc reference manual file]
\PassOptionsToClass{10pt,a4paper}{article}
\documentclass{ltxdoc}

\usepackage[margin=35mm]{geometry}
\usepackage{hyperref}
\usepackage{hyperxmp}
\usepackage[usenames]{color}

\hypersetup{colorlinks=true}
\hypersetup{pdfstartview=FitH}
\hypersetup{pdfpagemode=UseNone}
\hypersetup{pdfsource={}}
\hypersetup{pdflang={en-UK}}
\hypersetup{pdfcopyright={Copyright 2017-2018 Niklas Beisert.
  This work may be distributed and/or modified under the
  conditions of the LaTeX Project Public License, either version 1.3
  of this license or (at your option) any later version.}}
\hypersetup{pdflicenseurl={http://www.latex-project.org/lppl.txt}}
\hypersetup{pdfcontactaddress={ETH Zurich, ITP, HIT K,
  Wolfgang-Pauli-Strasse 27}}
\hypersetup{pdfcontactpostcode={8093}}
\hypersetup{pdfcontactcity={Zurich}}
\hypersetup{pdfcontactcountry={Switzerland}}
\hypersetup{pdfcontactemail={nbeisert@itp.phys.ethz.ch}}
\hypersetup{pdfcontacturl={http://people.phys.ethz.ch/\xmptilde nbeisert/}}

\newcommand{\secref}[1]{\hyperref[#1]{section \ref*{#1}}}

\parskip1ex
\parindent0pt
\let\olditemize\itemize
\def\itemize{\olditemize\parskip0pt}

\begin{document}

\title{The \textsf{childdoc} Package}
\hypersetup{pdftitle={The childdoc Package}}
\author{Niklas Beisert\\[2ex]
  Institut f\"ur Theoretische Physik\\
  Eidgen\"ossische Technische Hochschule Z\"urich\\
  Wolfgang-Pauli-Strasse 27, 8093 Z\"urich, Switzerland\\[1ex]
  \href{mailto:nbeisert@itp.phys.ethz.ch}
  {\texttt{nbeisert@itp.phys.ethz.ch}}}
\hypersetup{pdfauthor={Niklas Beisert}}
\hypersetup{pdfsubject={Manual for the LaTeX2e Package childdoc}}
\date{30 December 2018, \textsf{v2.0}}
\maketitle

\begin{abstract}\noindent
\textsf{childdoc} is a \LaTeXe{} package
that enables the direct compilation
of document sections included by |\include|
to individual files.
\end{abstract}

\begingroup
\parskip0ex
\tableofcontents
\endgroup

%%%%%%%%%%%%%%%%%%%%%%%%%%%%%%%%%%%%%%%%%%%%%%%%%%%%%%%%%%%%%%%%%%%%%%%%%%%%%%%%
%%%%%%%%%%%%%%%%%%%%%%%%%%%%%%%%%%%%%%%%%%%%%%%%%%%%%%%%%%%%%%%%%%%%%%%%%%%%%%%%
\section{Introduction}

\LaTeX{} provides a mechanism to structure a large document (such as a book)
into a main file and several child files (containing the chapters)
using the |\include| command.
This mechanism is beneficial for documents
which span hundreds of pages in order to
make the source file(s) more manageable.
Moreover, compilation can be restricted to
selected child files by means of the |\includeonly| command.
The latter feature can be used to reduce the compilation time while editing
(this was significantly more useful in the earlier days of \LaTeX{})
or to generate a smaller document which is easier to navigate.
Another application of |\includeonly| is to generate
documents consisting of selected parts of the complete document.

However, there are a few drawbacks of the plain |\include| mechanism:
\begin{itemize}
\item
The child files cannot be compiled on their own,
they can only be compiled via the main file.
A naive editing environment
(such as a text editor with an option
to have the current file processed by \LaTeX)
may require one to switch to the main file before compiling;
attempting to compile the child file produces errors.
\item
The main file must be modified (each time)
to adjust the |\includeonly| command
to the present needs. This easily leaves the main file in a messy state.
\item
The generated document will always carry the filename
of the main document. This is inconvenient if
several child files are to be compiled and
to be kept for distribution.
\end{itemize}

The present package provides a simple interface
to make child files individually compilable by \LaTeX{}.
Compiling a child file then has the same effect as compiling
the main file with an |\includeonly| command
to select the appropriate child.
Moreover the generated document will carry the name of the child
rather than the main file.
This resolves all three above issues.

This feature is meant to make the editing of books,
thesis documents and lecture notes somewhat more convenient.
However, the package can also be used efficiently for
composing a series of documents (such as exercise sheets)
which are typically distributed individually.
It then assists the author in generating the individual documents
(potentially in different versions)
as well as a document containing the collected series.
Another application is in developing style files
or other kinds of included material
where compilation of the style file could redirect
to a sample or test file.

%%%%%%%%%%%%%%%%%%%%%%%%%%%%%%%%%%%%%%%%%%%%%%%%%%%%%%%%%%%%%%%%%%%%%%%%%%%%%%%%
%%%%%%%%%%%%%%%%%%%%%%%%%%%%%%%%%%%%%%%%%%%%%%%%%%%%%%%%%%%%%%%%%%%%%%%%%%%%%%%%
\section{Usage}

First of all, the package \textsf{childdoc} is \emph{not} a standard
\LaTeXe{} |.sty| style file! Therefore it needs to be invoked in
a non-standard way.

%%%%%%%%%%%%%%%%%%%%%%%%%%%%%%%%%%%%%%%%%%%%%%%%%%%%%%%%%%%%%%%%%%%%%%%%%%%%%%%%
\subsection{Included Files}
\label{sec:include}

%%%%%%%%%%%%%%%%%%%%%%%%%%%%%%%%%%%%%%%%
\DescribeMacro{\childdocmain}
To use the package, add the commands
\begin{center}
\begin{tabular}{l}
|\input{childdoc.def}|\\
|\childdocmain{}|\\
\end{tabular}
\end{center}
at the very top of the main \LaTeX{} file,
in particular \emph{before} the |\documentclass| statement!
The argument of |\childdocmain| should be left empty
(but it must be present).

%%%%%%%%%%%%%%%%%%%%%%%%%%%%%%%%%%%%%%%%
\DescribeMacro{\childdocof}
Furthermore, add the commands
\begin{center}
\begin{tabular}{l}
|\input{childdoc.def}|\\
|\childdocof{|\textit{main}|}|\\
\end{tabular}
\end{center}
at the top of every child file \textit{child}
which is included by |\include{|\textit{child}|}|
from within the main file
(or at least for those files to be compiled individually).
The argument \textit{main} must be the filename of the main file.

There are a couple of
considerations in setting up the main and child documents:

%%%%%%%%%%%%%%%%%%%%%%%%%%%%%%%%%%%%%%%%
\paragraph{Restrictions.}

Please note the following restrictions:
\begin{itemize}
\item
|\childdocmain| must be called with one argument \textit{main}
to ensure compatibility with earlier version of the package.
It must either be empty (|\childdocmain{}|)
or precisely match the filename of the main file in which it is specified.
See \secref{sec:detection} for further information.
\item
The filename \textit{main} must be specified without the |.tex| extension.
\item
The filename \textit{main} is case sensitive
(even in case-insensitive file systems)
due to internal string comparison.
\item
The argument \textit{main} should be fully expanded, it cannot be a macro.
\item
Subdirectories and special characters should be avoided in filenames.
\item
The command |\childdocmain{|\textit{main}|}| must be followed by a whitespace.
It should not be followed immediately by another command
or by a comment mark `|%|'.
This is because the \TeX{} parser reads the token immediately following
the argument of |\childdocmain| and puts it
at the beginning of every child section;
however, a white\-space is ignored.
\end{itemize}

%%%%%%%%%%%%%%%%%%%%%%%%%%%%%%%%%%%%%%%%
\paragraph{Content of Main File.}

It is advisable to place all content in the child files included by |\include|.
Any output contained in the main file will appear in all child documents
unless suppressed manually;
it cannot be suppressed automatically by the |\includeonly| directive
and thus should normally be avoided.
A method to include some content in the main file
by means of conditional processing is described in \secref{sec:conditional}.

%%%%%%%%%%%%%%%%%%%%%%%%%%%%%%%%%%%%%%%%
\paragraph{Page Numbering.}

When only a part of the document is compiled,
the appropriate numbering of pages
(as well as other status parameters)
is determined from the |.aux| files.
The latter contain information from previous passes.
However this information needs to propagate through
all intermediate child documents.
Therefore the page numbering in child documents may well
be inconsistent until the complete document is compiled at least once.

A useful (if unconventional) way to always ensure a consistent
page numbering is to restart the numbering in each child document
and denote the pages by `\textit{child}|.|\textit{page}'
where \textit{child} represents the chapter/section number of the child file.
This can be achieved by the command
|\numberwithin{page}{|\textit{child}|}|
of the \textsf{amsmath} package
where \textit{child} can be |chapter| or |section|
depending on the chosen structuring.
Alternatively, one can modify the macro |\thepage| appropriately
and reset the counter |page| at the start of each child file.

%%%%%%%%%%%%%%%%%%%%%%%%%%%%%%%%%%%%%%%%%%%%%%%%%%%%%%%%%%%%%%%%%%%%%%%%%%%%%%%%
\subsection{Conditional Processing}
\label{sec:conditional}

The package provides a mechanism to compile different versions
of a document. To customise the versions further some conditional processing
can come in handy to distinguish which version is being compiled.
The package provides two macros to describe the compilation context:

%%%%%%%%%%%%%%%%%%%%%%%%%%%%%%%%%%%%%%%%
\DescribeMacro{\ifchilddoc}
The conditional |\ifchilddoc| distinguishes between the compilation of
child documents and the main document:
%
\begin{center}
|\ifchilddoc |\textit{child-code}| |[|\||else |\textit{main-code}]| \||fi|
\end{center}

%%%%%%%%%%%%%%%%%%%%%%%%%%%%%%%%%%%%%%%%
\DescribeMacro{\childdocname}
\DescribeMacro{\childdocjob}
The macro |\childdocname| contains the filename (without extension)
of the main or child file being processed.
Note that |\childdocjob| will always contain the name of the main file.

%%%%%%%%%%%%%%%%%%%%%%%%%%%%%%%%%%%%%%%%
\paragraph{Title Page.}

Conditional processing can be used to include a title or banner page
in the main document when proper precautions are taken.
Importantly, the code in the main file should ensure that the page counter
(as well as other status parameters which are stored in the |.aux| files)
takes the same value after the conditional processing.
Otherwise the page numbers may take divergent values
depending on which part is compiled.

For example, a title page could be declared by:
%
\begin{center}
\begin{tabular}{l}
|\ifchilddoc\||else|\\
|\addtocounter{page}{-1}|\\
\textit{code for title page}\\
|\newpage|\\
|\||fi|
\end{tabular}
\end{center}
%
A banner page for the child documents can be generated by:
%
\begin{center}
\begin{tabular}{l}
|\ifchilddoc|\\
|\addtocounter{page}{-1}|\\
\textit{code for banner page}\\
|\newpage|\\
|\||fi|
\end{tabular}
\end{center}
%
Here one could write a message such as:
\begin{center}
|This is the part \childdocname{} of \childdocjob{}.|
\end{center}

%%%%%%%%%%%%%%%%%%%%%%%%%%%%%%%%%%%%%%%%%%%%%%%%%%%%%%%%%%%%%%%%%%%%%%%%%%%%%%%%
\subsection{Flags}
\label{sec:flags}

The package makes it easy to generate different versions
of the main or child documents.
To this end compilation flags can be defined
and assigned different default values.
They will be particularly useful in conjunction
with the forwarding mechanism described in \secref{sec:forward}.

For example, it may be useful to have a flag |\version|
which can be set to |draft| or |final|.
The document source will contain some conditional code
depending on the value of |\version|.
Suppose further, the flag should default to |final| for the main file
and to |draft| for child files
which is a natural assignment for editing the document.
This is achieved by placing the following code
in the preamble of the main document
(below the |\childdocmain| directive):
%
\begin{center}
\begin{tabular}{l}
|\ifchilddoc|\\
|\providecommand{\version}{draft}|\\
|\||else|\\
|\providecommand{\version}{final}|\\
|\||fi|
\end{tabular}
\end{center}
%
The definition by |\providecommand| makes sure
that previous definitions are not overwritten.
Further statements |\providecommand{\version}{...}|
can thus be added before the above code to override it.

For the main file, one might add a line
(between |\childdocmain| and the above block)
%
\begin{center}
|%\ifchilddoc\||else\providecommand{\version}{draft}\||fi|
\end{center}
%
which can be uncommented to produce a draft version.
Likewise one can add a line to the very top of a child file
(above the |\childdocof{|\textit{main}|}| directive)
%
\begin{center}
|%\providecommand{\version}{final}|
\end{center}
%
which can be uncommented to produce the final version of this child document.

%%%%%%%%%%%%%%%%%%%%%%%%%%%%%%%%%%%%%%%%%%%%%%%%%%%%%%%%%%%%%%%%%%%%%%%%%%%%%%%%
\subsection{Forwarding}
\label{sec:forward}

Different versions of the main or child documents
using compilation flags as described in \secref{sec:flags}
can be (permanently) stored in different files
for convenient compilation, viewing and distribution.
To this end, the package defines a command
to pass on compilation to a different file:

%%%%%%%%%%%%%%%%%%%%%%%%%%%%%%%%%%%%%%%%
\DescribeMacro{\childdocforward}
The command |\childdocforward| redirects processing to
another source file:
%
\begin{center}
\begin{tabular}{l}
|\input{childdoc.def}|\\
|\childdocforward[|\textit{main}|]{|\textit{dest}|}|\\
\end{tabular}
\end{center}
%
The argument \textit{dest} is the destination file
(without extension).
It should be the main file or one of the child files.
Note that further \textsf{childdoc} directives
such as |\childdocof| and |\childdocforward|
in the indicated file will be processed in this form.
The optional argument \textit{main}
passes on directly to the main file \textit{main}
while pretending to compile the child \textit{dest}.
This form behaves as if \textit{dest}
issues |\childdocof{|\textit{main}|}| right away,
and no further \textsf{childdoc} directives will be processed.

%%%%%%%%%%%%%%%%%%%%%%%%%%%%%%%%%%%%%%%%
\DescribeMacro{\...prefix}
In the alternative form |\childdocforwardprefix|,
%
\begin{center}
\begin{tabular}{l}
|\input{childdoc.def}|\\
|\childdocforwardprefix[|\textit{main}|]{|\textit{prefix}|}{|\textit{dest}|}|
\end{tabular}
\end{center}
%
the destination file is determined by a pattern
depending on the current file:
To make this work, the current file must be called
`{\textit{prefix}\hspace{0.2em}\textit{suffix}}'
with \textit{prefix} matching precisely the argument.
Processing is then passed on to the file
`{\textit{dest}\hspace{0.2em}\textit{suffix}}'.
Surely, the same effect is achieved by
directly specifying the
argument `{\textit{dest}\hspace{0.2em}\textit{suffix}}'
in the first form.
However, that requires to set up a different file
for each child. With the alternative form of the command
all these files can have exactly the same content
which simplifies setting them up and maintaining them.

For example, the following file |draft.tex|
with a compilation flag |\version| as described in \secref{sec:flags}
compiles the main document as a draft:
%
\begin{center}
\begin{tabular}{l}
|\def\version{draft}|\\
|\input{childdoc.def}|\\
|\childdocforward{|\textit{main}|}|
\end{tabular}
\end{center}
%
Likewise, the following files |final|\textit{nn}|.tex|
compile the final version of the child document
|child|\textit{nn}|.tex|:
%
\begin{center}
\begin{tabular}{l}
|\def\version{final}|\\
|\input{childdoc.def}|\\
|\childdocforwardprefix{final}{child}|
\end{tabular}
\end{center}
%

Note that when several versions of a main file and/or of each child file
are to be generated, it may be convenient to set up a |Makefile| or
shell script to automatise the process.

%%%%%%%%%%%%%%%%%%%%%%%%%%%%%%%%%%%%%%%%%%%%%%%%%%%%%%%%%%%%%%%%%%%%%%%%%%%%%%%%
\subsection{Command Line Processing}
\label{sec:commandline}

The effect of redirection files can also be achieved by invoking
the \LaTeX{} compiler with a more elaborate command line.
Most conveniently this should be done as part
of a shell script or a |Makefile|.

When using \textsf{childdoc} in the main file, the following
command lines effectively perform a redirection
(note that depending on the shell being used,
backslashes may have to be doubled: `|\|' $\to$ `|\\|'):
%
\begin{center}
|... -jobname "|\textit{target}|" |\\|"|[\textit{flags}]%
|\input{childdoc.def}\childdocforward[|\textit{main}|]{|\textit{dest}|}"|
\end{center}
%
Here \textit{target} is the name of the output file,
\textit{main} is the name of the main file
and \textit{dest} is the name of the main or child file to be processed
(all filenames without extensions).
The optional argument \textit{main} can be omitted
if \textit{main} matches \textit{dest}.
Optionally, compilation \textit{flags} can be defined via |\def| commands.
This command line makes the \TeX{} engine believe
it is compiling the file \textit{target}
whose content is specified as the latter parameter.
The provided code then forwards the processing to
\textit{main} or \textit{dest} as described in \secref{sec:forward}.

%%%%%%%%%%%%%%%%%%%%%%%%%%%%%%%%%%%%%%%%%%%%%%%%%%%%%%%%%%%%%%%%%%%%%%%%%%%%%%%%
\subsection{Include by Input}
\label{sec:input}

Including child documents by |\include| has some restrictions by design.
Most notably, the content of a child document always occupies
its own set of pages; pages cannot be shared between child documents.
Usually, this behaviour makes perfect sense
because each child document contain an essential part of the document.
However, in some situations it may be desirable to compose
a document from a collection of parts
without having mandatory page breaks between then.
For this case, the package
provides a mechanism to include parts
by |\input| which can also be processed individually.
However, by construction this mechanism
requires manual handling of the content to be output.

%%%%%%%%%%%%%%%%%%%%%%%%%%%%%%%%%%%%%%%%
\DescribeMacro{\ifchilddocmanual}
The main file should be prepared as usual, see \secref{sec:include}.
However, the document body must make a distinction
between processing of an individual part and of the main document, e.g.:
%
\begin{center}
\begin{tabular}{l}
|\ifchilddocmanual|\\
|\input{\childdocname}|\\
|\||else|\\
\textit{document body with }|\input{|\textit{part}|}|\\
|\||fi|
\end{tabular}
\end{center}
%
The conditional |\ifchilddocmanual| is true whenever
a part to be included by |\input| is being compiled,
and the name of the part is stored in |\childdocname|.

%%%%%%%%%%%%%%%%%%%%%%%%%%%%%%%%%%%%%%%%
\DescribeMacro{\childdocby}
Each part to be included by |\input| should start with:
%
\begin{center}
\begin{tabular}{l}
|\input{childdoc.def}|\\
|\childdocby{|\textit{main}|}|\\
\end{tabular}
\end{center}
%
The directive |\childdocby| is similar to |\childdocof|
described in \secref{sec:include},
but the subsequent selection of content must be done manually.
To that end, both |\ifchilddoc| and |\ifchilddocmanual|
will be true upon processing of a part,
and the name of the part is stored in |\childdocname|.
Note that |\jobname| will be set to the filename of the current part
so that each part receives an individual |.aux| file
that does not interfere with the |.aux| file(s) of the main document.
This behaviour can be altered by the alternative form
|\childdocby[*]{|\textit{main}|}| (with a non-empty optional argument)
which uses the |.aux| file of the main document
by setting |\jobname| to \textit{main}.

%%%%%%%%%%%%%%%%%%%%%%%%%%%%%%%%%%%%%%%%%%%%%%%%%%%%%%%%%%%%%%%%%%%%%%%%%%%%%%%%
\subsection{Driver Development}
\label{sec:driver}

The \textsf{childdoc} mechanism can also be use for the development
of definition files such as \LaTeX{} styles or classes.
This case differs from the above setup with multiple parts
included by |\include| in that no |\includeonly| should be invoked.
This can be achieved by starting the include file
(before |\ProvidesPackage|) with:
%
\begin{center}
\begin{tabular}{l}
|\input{childdoc.def}|\\
|\childdocforward{|\textit{main}|}|\\
\end{tabular}
\end{center}
%
or alternatively with:
%
\begin{center}
\begin{tabular}{l}
|\input{childdoc.def}|\\
|\childdocby{|\textit{main}|}|\\
\end{tabular}
\end{center}
%
Both forms have slightly different effects as described above.
The main file is prepared as usual, see \secref{sec:include}.

%%%%%%%%%%%%%%%%%%%%%%%%%%%%%%%%%%%%%%%%%%%%%%%%%%%%%%%%%%%%%%%%%%%%%%%%%%%%%%%%
\subsection{Legacy Detection}
\label{sec:detection}

The directive |\childdocmain| in the main file can detect
whether the complete document or merely a child is to be compiled
even without using the directive |\childdocof|.
This method is deprecated because it is less robust
and there is no compelling reason to use it;
it is merely provided for backward compatibility
and it may be removed in future versions.

If the detection mechanism is to be used,
it is mandatory to correctly specify
the filename of the main file as the argument of |\childdocmain|:
%
\begin{center}
\begin{tabular}{l}
|\input{childdoc.def}|\\
|\childdocmain{|\textit{main}|}|\\
\end{tabular}
\end{center}
%
If |\jobname| does not match the argument \textit{main} of |\childdocmain|,
it is assumed that |\jobname| points to the child file to be compiled.
When using |\childdocmain| with the main file specified as argument,
it suffices to start a child file
with just |\input{|\textit{main}|}|
without loading of the package and using |\childdocof|.
If instead all processing is done
with the appropriate \textsf{childdoc} directives,
the argument of \textit{main} of |\childdocmain| can be empty.

An alternative version of the command line processing described
in \secref{sec:commandline} using the detection mechanism reads:
%
\begin{center}
|... -jobname "|\textit{target}|" "|[\textit{flags}]%
[|\def\jobname{|\textit{dest}|}|]|\input{|\textit{main}|}"|
\end{center}

%%%%%%%%%%%%%%%%%%%%%%%%%%%%%%%%%%%%%%%%%%%%%%%%%%%%%%%%%%%%%%%%%%%%%%%%%%%%%%%%
\subsection{Manual Code}
\label{sec:manual}

In case one cannot be certain whether the definitions file |childdoc.def|
is installed on the target \TeX{} distribution
and one prefers not to ship it,
it is conceivable to paste a few relevant commands into the sources.

To that end, drop all statements |\input{childdoc.def}|
and perform the replacements as outlined below.
Instead of |\childdocmain{|\textit{main}|}| add the following code
to the top of the main file:
%
\begin{center}
\begin{tabular}{l}
|\||ifdefined\childdocname\endinput\||fi\newif\ifchilddoc|\\
|\edef\childdocname{\scantokens\expandafter{\jobname\noexpand}}|\\
|\def\childdocmain{|\textit{main}|}\||ifx\childdocmain\childdocname\||else|\\
|\childdoctrue\includeonly{\childdocname}\let\jobname\childdocmain\||fi|\\
\end{tabular}
\end{center}
%
Instead of |\childdocof{|\textit{main}|}| just include the main file
at the top of each child file:
%
\begin{center}
|\input{|\textit{main}|}|
\end{center}
%
A simple redirection |\childdocforward{|\textit{dest}|}| is achieved by:
%
\begin{center}
|\def\jobname{|\textit{dest}|}\input{\jobname}|
\end{center}
%
The redirection with prefix
|\childdocforwardprefix[|\textit{prefix}|]{|\textit{dest}|}|
is accomplished by:
%
\begin{center}
\begin{tabular}{l}
|{\edef\jobname{\scantokens\expandafter{\jobname\noexpand}}|\\
|\def\redirectjob |\textit{prefix}|#1~~~{\gdef\jobname{|\textit{dest}|#1}}|\\
|\expandafter\redirectjob\jobname~~~}\input{\jobname}|
\end{tabular}
\end{center}

In an alternative approach,
child documents can be compiled by a specific command line
without additional code or specific definitions:
%
\begin{center}
|... -jobname "|\textit{target}|" "|[\textit{flags}]%
|\includeonly{|\textit{dest}|}\input{|\textit{main}|}"|
\end{center}
%

%%%%%%%%%%%%%%%%%%%%%%%%%%%%%%%%%%%%%%%%%%%%%%%%%%%%%%%%%%%%%%%%%%%%%%%%%%%%%%%%
%%%%%%%%%%%%%%%%%%%%%%%%%%%%%%%%%%%%%%%%%%%%%%%%%%%%%%%%%%%%%%%%%%%%%%%%%%%%%%%%
\section{Information}

%%%%%%%%%%%%%%%%%%%%%%%%%%%%%%%%%%%%%%%%%%%%%%%%%%%%%%%%%%%%%%%%%%%%%%%%%%%%%%%%
\subsection{Copyright}

Copyright \copyright{} 2017--2018 Niklas Beisert

This work may be distributed and/or modified under the
conditions of the \LaTeX{} Project Public License, either version 1.3
of this license or (at your option) any later version.
The latest version of this license is in
  \url{http://www.latex-project.org/lppl.txt}
and version 1.3 or later is part of all distributions of \LaTeX{}
version 2005/12/01 or later.

This work has the LPPL maintenance status `maintained'.

The Current Maintainer of this work is Niklas Beisert.

This work consists of the files |README.txt|, |childdoc.ins| and |childdoc.dtx|
as well as the derived files |childdoc.def|, |cdocsamp.tex|
with |cdocsch1.tex|, |cdocsch2.tex|, |cdocspt3.tex|, |cdocspt4.tex|,
|cdocsdrf.tex|, |cdocsfn1.tex|, |cdocsfn2.tex|
as well as |childdoc.pdf|.

%%%%%%%%%%%%%%%%%%%%%%%%%%%%%%%%%%%%%%%%%%%%%%%%%%%%%%%%%%%%%%%%%%%%%%%%%%%%%%%%
\subsection{Files and Installation}

The package consists of the files:
%
\begin{center}
\begin{tabular}{ll}
    |README.txt|   & readme file \\
    |childdoc.ins| & installation file \\
    |childdoc.dtx| & source file \\
    |childdoc.def| & definition file \\
    |cdocsamp.tex| & sample main file \\
    |cdocsch1.tex| & sample include file \\
    |cdocsch2.tex| & sample include file \\
    |cdocspt3.tex| & sample part file \\
    |cdocspt4.tex| & sample part file \\
    |cdocsdrf.tex| & sample redirection file \\
    |cdocsfn1.tex| & sample redirection file \\
    |cdocsfn2.tex| & sample redirection file \\
    |childdoc.pdf| & manual
\end{tabular}
\end{center}
%
The distribution consists of the files
|README.txt|, |childdoc.ins| and |childdoc.dtx|.
%
\begin{itemize}
\item
Run (pdf)\LaTeX{} on |childdoc.dtx|
to compile the manual |childdoc.pdf| (this file).
\item
Run \LaTeX{} on |childdoc.ins| to create the definitions file |childdoc.def|
and the sample |cdocsamp.tex| with include files
|cdocsch1.tex|, |cdocsch2.tex|, |cdocspt3.tex|, |cdocspt4.tex|,
|cdocsdrf.tex|, |cdocsfn1.tex|, |cdocsfn2.tex|.
Then copy the file |childdoc.def| to an appropriate directory of your \LaTeX{}
distribution, e.g.\ \textit{texmf-root}|/tex/latex/childdoc|.
\end{itemize}

%%%%%%%%%%%%%%%%%%%%%%%%%%%%%%%%%%%%%%%%%%%%%%%%%%%%%%%%%%%%%%%%%%%%%%%%%%%%%%%%
\subsection{Related CTAN Packages}

There are several other packages which offer a similar functionality:
%
\begin{itemize}
\item
The packages
\href{http://ctan.org/pkg/docmute}{\textsf{docmute}},
\href{http://ctan.org/pkg/includex}{\textsf{includex}} and
\href{http://ctan.org/pkg/standalone}{\textsf{standalone}}
provide commands to include only the document body of
a child file thus allowing both files to be compiled individually.
\item
The packages \href{http://ctan.org/pkg/subdocs}{\textsf{subdocs}}
and \href{http://ctan.org/pkg/subfiles}{\textsf{subfiles}}
provide structures in which the main and child documents can be
encapsulated and allowing them to be compiled individually.
The inclusion mechanism is different from the conventional |\include|.
\item
The package \href{http://ctan.org/pkg/combine}{\textsf{combine}}
is an elaborate solution to combine several documents into one.
\end{itemize}
%
See also the CTAN topic \href{http://ctan.org/topic/subdocs}{\textsf{subdocs}}
for further related packages.
The present package differs from the above solutions in that
a document structure constructed with the conventional |\include| mechanism
just needs two extra commands at the top of every file
such that all constituent files can be compiled individually.

%%%%%%%%%%%%%%%%%%%%%%%%%%%%%%%%%%%%%%%%%%%%%%%%%%%%%%%%%%%%%%%%%%%%%%%%%%%%%%%%
%\subsection{Feature Suggestions}
%
%The following is a list of features which may be useful for future
%versions of this package:
%%
%\begin{itemize}
%\item
%\ldots
%\end{itemize}

%%%%%%%%%%%%%%%%%%%%%%%%%%%%%%%%%%%%%%%%%%%%%%%%%%%%%%%%%%%%%%%%%%%%%%%%%%%%%%%%
\subsection{Revision History}

%%%%%%%%%%%%%%%%%%%%%%%%%%%%%%%%%%%%%%%%
\paragraph{v2.0:} 2018/12/30

\begin{itemize}
\item
immediate forward processing
\item
added |\childdocby| mechanism
\item
manual restructured
\end{itemize}

%%%%%%%%%%%%%%%%%%%%%%%%%%%%%%%%%%%%%%%%
\paragraph{v1.6:} 2018/01/17

\begin{itemize}
\item
application for development of include files
\item
corrections to manual
\end{itemize}

%%%%%%%%%%%%%%%%%%%%%%%%%%%%%%%%%%%%%%%%
\paragraph{v1.5:} 2017/05/21

\begin{itemize}
\item
more complete structuring introduced
\item
|\childdocof| introduced
\item
|\childdoc| renamed to |\childdocmain|
\item
|\childredirect| renamed to |\childdocforward| and |\childdocforwardprefix|
and functionality expanded
\end{itemize}

%%%%%%%%%%%%%%%%%%%%%%%%%%%%%%%%%%%%%%%%
\paragraph{v1.0:} 2017/04/27

\begin{itemize}
\item
manual and install package
\item
first version published on CTAN
\end{itemize}

%%%%%%%%%%%%%%%%%%%%%%%%%%%%%%%%%%%%%%%%
\paragraph{v0.6:} 2017/04/26

\begin{itemize}
\item
redirection mechanism added
\end{itemize}

%%%%%%%%%%%%%%%%%%%%%%%%%%%%%%%%%%%%%%%%
\paragraph{v0.5:} 2017/04/26

\begin{itemize}
\item
functionality in definition file
\end{itemize}


%%%%%%%%%%%%%%%%%%%%%%%%%%%%%%%%%%%%%%%%%%%%%%%%%%%%%%%%%%%%%%%%%%%%%%%%%%%%%%%%
%%%%%%%%%%%%%%%%%%%%%%%%%%%%%%%%%%%%%%%%%%%%%%%%%%%%%%%%%%%%%%%%%%%%%%%%%%%%%%%%
%%%%%%%%%%%%%%%%%%%%%%%%%%%%%%%%%%%%%%%%%%%%%%%%%%%%%%%%%%%%%%%%%%%%%%%%%%%%%%%%
\appendix

\settowidth\MacroIndent{\rmfamily\scriptsize 000\ }

 \DocInput{childdoc.dtx}

\end{document}
%</driver>
% \fi
%
% %%%%%%%%%%%%%%%%%%%%%%%%%%%%%%%%%%%%%%%%%%%%%%%%%%%%%%%%%%%%%%%%%%%%%%%%%%%%%%
% %%%%%%%%%%%%%%%%%%%%%%%%%%%%%%%%%%%%%%%%%%%%%%%%%%%%%%%%%%%%%%%%%%%%%%%%%%%%%%
% \section{Sample}
%\iffalse
%<*samplemain>
%\fi
%
% The following presents a sample document
% with two chapters, two parts, a title page,
% a compile flag as well as three forwarding files to set the flag.
% It consists of eight |.tex| files:
% \begin{center}
% \begin{tabular}{ll}
% |cdocsamp.tex|&main file\\
% |cdocsch1.tex|&include file for chapter 1\\
% |cdocsch2.tex|&include file for chapter 2\\
% |cdocspt3.tex|&include file for part 3\\
% |cdocspt4.tex|&include file for part 4\\
% |cdocsdrf.tex|&forwarding file for main file in draft mode\\
% |cdocsfi1.tex|&forwarding file for final version of chapter 1\\
% |cdocsfi2.tex|&forwarding file for final version of chapter 2\\
% \end{tabular}
% \end{center}
% Each of the eight files can be compiled directly by the \LaTeX{} compiler.
%
% %%%%%%%%%%%%%%%%%%%%%%%%%%%%%%%%%%%%%%
% \paragraph{Main File.}
%
% The main file is called |cdocsamp.tex|.
%
% Load the \textsf{childdoc} definitions and
% declare the filename for the main document:
%    \begin{macrocode}
\input{childdoc.def}
\childdocmain{}
%    \end{macrocode}

% Optional override for |\version| flag:
%    \begin{macrocode}
%%\ifchilddoc\else\providecommand{\version}{draft}\fi
%    \end{macrocode}

% Define the default values for the |\version| flag
% (|final| for the main file and |draft| for childs):
%    \begin{macrocode}
\ifchilddoc
\providecommand{\version}{draft}
\else
\providecommand{\version}{final}
\fi
%    \end{macrocode}

% Load the standard document class:
%    \begin{macrocode}
\documentclass[12pt]{article}
%    \end{macrocode}

% Start the document body:
%    \begin{macrocode}
\begin{document}
%    \end{macrocode}

% Declare a title page.
% Print title, part of document being processed and version flag:
%    \begin{macrocode}
\addtocounter{page}{-1}
\begin{center}
{\LARGE\bfseries{}childdoc example\par}
\vspace{1cm}
\ifchilddoc
\ifchilddocmanual part\else chapter\fi:
`\childdocname' of `\childdocjob'\par
\else
main document: `\childdocjob'\par
\fi
version: \version\par
\end{center}
\newpage
%    \end{macrocode}

% Manually include selected file,
% otherwise process as usual:
%    \begin{macrocode}
\ifchilddocmanual
\section*{part `\childdocname'}
\input{\childdocname}
\else
%    \end{macrocode}

% Include the two chapters:
%    \begin{macrocode}
\include{cdocsch1}
\include{cdocsch2}
%    \end{macrocode}

% Include the two parts unless only chapters should be displayed:
%    \begin{macrocode}
\ifchilddoc\else
\section{part three}
\input{cdocspt3}
\section{part four}
\input{cdocspt4}
\fi
%    \end{macrocode}

% Process as usual until here:
%    \begin{macrocode}
\fi
%    \end{macrocode}

% End of document body:
%    \begin{macrocode}
\end{document}
%    \end{macrocode}
%\iffalse
%</samplemain>
%\fi
%
% %%%%%%%%%%%%%%%%%%%%%%%%%%%%%%%%%%%%%%
% \paragraph{Chapter Include Files.}
%
% The include files are called |cdocsch1.tex| and |cdocsch2.tex|.
%
%\iffalse
%<*samplechap1|samplechap2>
%\fi

% Optional override for |\version| flag:
%    \begin{macrocode}
%%\providecommand{\version}{final}
%    \end{macrocode}

% Include the main document:
%    \begin{macrocode}
\input{childdoc.def}
\childdocof{cdocsamp}
%    \end{macrocode}

%\iffalse
%</samplechap1|samplechap2>
%\fi
%
%\iffalse
%<*samplechap1>
%\fi
% Some text for chapter 1:
%    \begin{macrocode}
\section{one}
some text in chapter one
%    \end{macrocode}

%\iffalse
%</samplechap1>
%\fi
% Some text for chapter 2:
%\iffalse
%<*samplechap2>
%\fi
%    \begin{macrocode}
\section{two}
more text in chapter two
%    \end{macrocode}

%\iffalse
%</samplechap2>
%\fi
%
% %%%%%%%%%%%%%%%%%%%%%%%%%%%%%%%%%%%%%%
% \paragraph{Part Include Files.}
%
% The include files are called |cdocspt3.tex| and |cdocspt4.tex|.
%
%\iffalse
%<*samplepart3|samplepart4>
%\fi

% Optional override for |\version| flag:
%    \begin{macrocode}
%%\providecommand{\version}{final}
%    \end{macrocode}

% Include the main document:
%    \begin{macrocode}
\input{childdoc.def}
\childdocby{cdocsamp}
%    \end{macrocode}

%\iffalse
%</samplepart3|samplepart4>
%\fi
%
%\iffalse
%<*samplepart3>
%\fi
% Some text for part 3:
%    \begin{macrocode}
some text in part three
%    \end{macrocode}

%\iffalse
%</samplepart3>
%\fi
% Some text for part 4:
%\iffalse
%<*samplepart4>
%\fi
%    \begin{macrocode}
more text in part four
%    \end{macrocode}

%\iffalse
%</samplepart4>
%\fi
%
% %%%%%%%%%%%%%%%%%%%%%%%%%%%%%%%%%%%%%%
% \paragraph{Forwarding for a Complete Draft.}
%
% The following forwarding file |cdocsdrf.tex|
% compiles the main document in draft mode:
%\iffalse
%<*sampledraft>
%\fi
%    \begin{macrocode}
\def\version{draft}
\input{childdoc.def}
\childdocforward{cdocsamp}
%    \end{macrocode}

%\iffalse
%</sampledraft>
%\fi
%
% %%%%%%%%%%%%%%%%%%%%%%%%%%%%%%%%%%%%%%
% \paragraph{Forwarding for Final Version of the Chapters.}
%
% The following forwarding files |cdocsfn1.tex| and |cdocsfn2.tex|
% (with identical content)
% compile the final versions of the child documents
% |cdocsch1.tex| and |cdocsch2.tex|, respectively:
%\iffalse
%<*samplefinal>
%\fi
%    \begin{macrocode}
\def\version{final}
\input{childdoc.def}
\childdocforwardprefix[cdocsamp]{cdocsfn}{cdocsch}
%    \end{macrocode}

%\iffalse
%</samplefinal>
%\fi
%
% %%%%%%%%%%%%%%%%%%%%%%%%%%%%%%%%%%%%%%
% \paragraph{Command Line Processing.}
%
% The following three command lines generate the output files
% |cdocscld|, |cdocscl1| and |cdocscl2|
% which should be identical to
% |cdocsdrf|, |cdocsch1| and |cdocsfn2|, respectively:
% \begin{center}
% \begin{tabular}{l}
% |latex -jobname cdocscld \|\\
% |  "\def\version{draft}\input{childdoc.def}\childdocforward{cdocsamp}"|\\
% |latex -jobname cdocscl1 \|\\
% |  "\input{childdoc.def}\childdocforward[cdocsamp]{cdocsch1}"|\\
% |latex -jobname cdocscl2 \|\\
% |  "\def\version{final}\input{childdoc.def}\childdocforward{cdocsch2}"|
% \end{tabular}
% \end{center}
% Note that the trailing backslash on each first line
% merely continues the input to the second line
% (for convenient cut ant paste).
% Furthermore, the command |latex| can be replaced by any
% of its alternative versions such as |pdflatex|.
%
% %%%%%%%%%%%%%%%%%%%%%%%%%%%%%%%%%%%%%%%%%%%%%%%%%%%%%%%%%%%%%%%%%%%%%%%%%%%%%%
% %%%%%%%%%%%%%%%%%%%%%%%%%%%%%%%%%%%%%%%%%%%%%%%%%%%%%%%%%%%%%%%%%%%%%%%%%%%%%%
% \section{Implementation}
%\iffalse
%<*package>
%\fi
%
% This section describes the definitions file |childdoc.def|.

% The definitions cannot be loaded using |\usepackage| or |\RequirePackage|
% which has a mechanism to prevent loading a style file more than once.
% When loading the definitions by means of |\input|
% multiple instances have to be prevented manually:
%\iffalse
%This code needs to be before the `\ProvidesFile' directive
%which is defined at the beginning of this file.
%Therefore it is also placed there and commented out here.
%</package>
%<*discard>
%\fi
%    \begin{macrocode}
\ifdefined\childdocmain\endinput\fi
%    \end{macrocode}
%\iffalse
%</discard>
%<*package>
%\fi
%
% \macro{\ifchilddoc}
% \macro{\ifchilddocmanual}
% The conditional |\ifchilddoc| tells whether a
% child (true) or main (false) document is being compiled.
% The conditional |\ifchilddocmanual| tells whether
% the |\includeonly| mechanism is used (false) or
% the selection of child files must be performed manually (true).
% The definitions initialise to false:
%    \begin{macrocode}
\newif\ifchilddoc
\newif\ifchilddocmanual
%    \end{macrocode}

% \macro{\childdocname}
% \macro{\childdocjob}
% The macro |\childdocname| stores the name of the main document
% to be compiled. The macro |\childdocjob| stores the name of
% the document on which the \LaTeX{} compiler was originally invoked.
% The content of |\jobname| cannot be compared
% to filenames specified in the source due to different catcodes.
% The following code rescans |\jobname|, stores the result
% in |\childdocname| and saves a copy in |\childdocjob|:
%    \begin{macrocode}
\edef\childdocname{\scantokens\expandafter{\jobname\noexpand}}
\let\childdocjob\childdocname
%    \end{macrocode}

% \macro{\childdocdisable}
% The macro |\childdocdisable| prevents the main file
% from being processed more than once.
% At this stage, the main document command |\childdocmain|
% is assumed to be called once again where it should do nothing.
% Any subsequent call to it should prevent
% a secondary processing of the main document
% It overwrites the forwarding commands
% |\childdocof| and |\childdocforward|
% with empty macros to prevent further inclusions of the main document:
%    \begin{macrocode}
\newcommand{\childdocdisable}
{
  \renewcommand{\childdocmain}[1]{\renewcommand{\childdocmain}[1]{\endinput}}
  \renewcommand{\childdocof}[1]{}
  \renewcommand{\childdocby}[2][]{}
  \renewcommand{\childdocforward}[2][]{}
  \renewcommand{\childdocdisable}{}
}
%    \end{macrocode}

% \macro{\childdocmain}
% The macro |\childdocmain| is to be called at the top of the main file
% with nothing or the main filename (without extension) as argument.
% First, it breaks loops.
% If the argument is not empty and does not match |\childdocname|
% (which is set by the first inclusion of |childdoc.def|),
% |\ifchilddoc| is set to true, |\includeonly| is applied to the child file
% and |\jobname| is set to the main file
% (for proper handling of |.aux| files):
%    \begin{macrocode}
\newcommand{\childdocmain}[1]
{
  \childdocdisable\childdocmain{}
  \if?#1?\else
    \begingroup
      \def\childdoctmp{#1}
      \ifx\childdoctmp\childdocname
        \def\childdoctmp{}
      \else
        \def\childdoctmp
        {
          \childdoctrue
          \includeonly{\childdocname}
          \def\childdocjob{#1}
          \def\jobname{#1}
        }
      \fi
      \expandafter
    \endgroup
    \childdoctmp
  \fi
}
%    \end{macrocode}

% \macro{\childdocof}
% The command |\childdocof| redirects
% compilation to the main file |#1|.
%    \begin{macrocode}
\newcommand{\childdocof}[1]
{
  \childdocdisable
  \childdoctrue
  \includeonly{\childdocname}
  \def\jobname{#1}
  \def\childdocjob{#1}
  \input{#1}
}
%    \end{macrocode}

% \macro{\childdocby}
% The command |\childdocby| ....
%    \begin{macrocode}
\newcommand{\childdocby}[2][]
{
  \childdocdisable
  \childdoctrue
  \childdocmanualtrue
  \if?#1?\else
    \def\jobname{#2}
  \fi
  \def\childdocjob{#2}
  \input{#2}
  \endinput
}
%    \end{macrocode}

% \macro{\childdocforward}
% The command |\childdocforward| redirects
% compilation to the main file or
% (if the optional argument is given) a child file.
% Parameters are set as if the main file
% or a child file starting with |\childdocof| was compiled.
% Then compilation is handed over to the main file:
%    \begin{macrocode}
\newcommand{\childdocforward}[2][]
{
  \begingroup
    \if?#1?
      \def\childdoctmp
      {
        \def\childdocname{#2}
        \def\childdocjob{#2}
        \def\jobname{#2}
        \input{#2}
        \endinput
      }
    \else
      \def\childdoctmp
      {
        \childdocdisable
        \def\childdocname{#2}
        \childdoctrue
        \includeonly{#2}
        \def\childdocjob{#1}
        \def\jobname{#1}
        \input{#1}
        \endinput
      }
    \fi
    \expandafter
  \endgroup
  \childdoctmp
}
%    \end{macrocode}

% \macro{\childdocforwardprefix}
% The command |\childdocforwardprefix| redirects
% compilation to the main or a child file by means of a pattern.
% The prefix |#1| in the current filename is replaced by |#2|
% and the suffix of the current filename is kept
% (it is assumed that the filename does not contain the substring `|~~~|'
% which is used as a delimiter).
% Compilation is handed over to the new file by |\childdocforward|:
%    \begin{macrocode}
\newcommand{\childdocforwardprefix}[3][]
{
  \begingroup
    \def\childdocextract #2##1~~~{\def\childdoctmp{\childdocforward[#1]{#3##1}}}
    \expandafter\childdocextract\childdocname~~~
    \expandafter
  \endgroup
  \childdoctmp
}
%    \end{macrocode}

% \macro{\childdoc}
% The deprecated macro |\childdoc| is a legacy version of |\childdocmain|:
%    \begin{macrocode}
\newcommand{\childdoc}{\childdocmain}
%    \end{macrocode}

% \macro{\childdocredirect}
% The deprecated macro |\childdocredirect| is a legacy version
% of |\childdocforward| and |\childdocforwardprefix|:
%    \begin{macrocode}
\newcommand{\childdocredirect}[2][]
{
  \begingroup
    \if?#1?
      \def\childdoctmp{\childdocforward{#2}}
    \else
      \def\childdoctmp{\childdocforwardprefix{#1}{#2}}
    \fi
    \expandafter
  \endgroup
  \childdoctmp
}
%    \end{macrocode}

%\iffalse
%</package>
%\fi
%
\endinput

\childdocmain{}
%    \end{macrocode}

% Optional override for |\version| flag:
%    \begin{macrocode}
%%\ifchilddoc\else\providecommand{\version}{draft}\fi
%    \end{macrocode}

% Define the default values for the |\version| flag
% (|final| for the main file and |draft| for childs):
%    \begin{macrocode}
\ifchilddoc
\providecommand{\version}{draft}
\else
\providecommand{\version}{final}
\fi
%    \end{macrocode}

% Load the standard document class:
%    \begin{macrocode}
\documentclass[12pt]{article}
%    \end{macrocode}

% Start the document body:
%    \begin{macrocode}
\begin{document}
%    \end{macrocode}

% Declare a title page.
% Print title, part of document being processed and version flag:
%    \begin{macrocode}
\addtocounter{page}{-1}
\begin{center}
{\LARGE\bfseries{}childdoc example\par}
\vspace{1cm}
\ifchilddoc
\ifchilddocmanual part\else chapter\fi:
`\childdocname' of `\childdocjob'\par
\else
main document: `\childdocjob'\par
\fi
version: \version\par
\end{center}
\newpage
%    \end{macrocode}

% Manually include selected file,
% otherwise process as usual:
%    \begin{macrocode}
\ifchilddocmanual
\section*{part `\childdocname'}
\input{\childdocname}
\else
%    \end{macrocode}

% Include the two chapters:
%    \begin{macrocode}
\include{cdocsch1}
\include{cdocsch2}
%    \end{macrocode}

% Include the two parts unless only chapters should be displayed:
%    \begin{macrocode}
\ifchilddoc\else
\section{part three}
\input{cdocspt3}
\section{part four}
\input{cdocspt4}
\fi
%    \end{macrocode}

% Process as usual until here:
%    \begin{macrocode}
\fi
%    \end{macrocode}

% End of document body:
%    \begin{macrocode}
\end{document}
%    \end{macrocode}
%\iffalse
%</samplemain>
%\fi
%
% %%%%%%%%%%%%%%%%%%%%%%%%%%%%%%%%%%%%%%
% \paragraph{Chapter Include Files.}
%
% The include files are called |cdocsch1.tex| and |cdocsch2.tex|.
%
%\iffalse
%<*samplechap1|samplechap2>
%\fi

% Optional override for |\version| flag:
%    \begin{macrocode}
%%\providecommand{\version}{final}
%    \end{macrocode}

% Include the main document:
%    \begin{macrocode}
% \iffalse
%
% childdoc.dtx Copyright (C) 2017-2018 Niklas Beisert
%
% This work may be distributed and/or modified under the
% conditions of the LaTeX Project Public License, either version 1.3
% of this license or (at your option) any later version.
% The latest version of this license is in
%   http://www.latex-project.org/lppl.txt
% and version 1.3 or later is part of all distributions of LaTeX
% version 2005/12/01 or later.
%
% This work has the LPPL maintenance status `maintained'.
%
% The Current Maintainer of this work is Niklas Beisert.
%
% This work consists of the files childdoc.dtx and childdoc.ins
% and the derived files childdoc.def and cdocsamp.tex with
% cdocsch1.tex, cdocsch2.tex, cdocsdrf.tex, cdocsfn1.tex, cdocsfn2.tex.
%
%<package>\ifdefined\childdocmain\endinput\fi
%<package>\ProvidesFile{childdoc.def}[2018/12/30 v2.0 child document driver]
%<samplemain>\ProvidesFile{cdocsamp.tex}[2018/12/30 v2.0 sample for childdoc]
%<*driver>
%\ProvidesFile{childdoc.drv}[2018/12/30 v2.0 childdoc reference manual file]
\PassOptionsToClass{10pt,a4paper}{article}
\documentclass{ltxdoc}

\usepackage[margin=35mm]{geometry}
\usepackage{hyperref}
\usepackage{hyperxmp}
\usepackage[usenames]{color}

\hypersetup{colorlinks=true}
\hypersetup{pdfstartview=FitH}
\hypersetup{pdfpagemode=UseNone}
\hypersetup{pdfsource={}}
\hypersetup{pdflang={en-UK}}
\hypersetup{pdfcopyright={Copyright 2017-2018 Niklas Beisert.
  This work may be distributed and/or modified under the
  conditions of the LaTeX Project Public License, either version 1.3
  of this license or (at your option) any later version.}}
\hypersetup{pdflicenseurl={http://www.latex-project.org/lppl.txt}}
\hypersetup{pdfcontactaddress={ETH Zurich, ITP, HIT K,
  Wolfgang-Pauli-Strasse 27}}
\hypersetup{pdfcontactpostcode={8093}}
\hypersetup{pdfcontactcity={Zurich}}
\hypersetup{pdfcontactcountry={Switzerland}}
\hypersetup{pdfcontactemail={nbeisert@itp.phys.ethz.ch}}
\hypersetup{pdfcontacturl={http://people.phys.ethz.ch/\xmptilde nbeisert/}}

\newcommand{\secref}[1]{\hyperref[#1]{section \ref*{#1}}}

\parskip1ex
\parindent0pt
\let\olditemize\itemize
\def\itemize{\olditemize\parskip0pt}

\begin{document}

\title{The \textsf{childdoc} Package}
\hypersetup{pdftitle={The childdoc Package}}
\author{Niklas Beisert\\[2ex]
  Institut f\"ur Theoretische Physik\\
  Eidgen\"ossische Technische Hochschule Z\"urich\\
  Wolfgang-Pauli-Strasse 27, 8093 Z\"urich, Switzerland\\[1ex]
  \href{mailto:nbeisert@itp.phys.ethz.ch}
  {\texttt{nbeisert@itp.phys.ethz.ch}}}
\hypersetup{pdfauthor={Niklas Beisert}}
\hypersetup{pdfsubject={Manual for the LaTeX2e Package childdoc}}
\date{30 December 2018, \textsf{v2.0}}
\maketitle

\begin{abstract}\noindent
\textsf{childdoc} is a \LaTeXe{} package
that enables the direct compilation
of document sections included by |\include|
to individual files.
\end{abstract}

\begingroup
\parskip0ex
\tableofcontents
\endgroup

%%%%%%%%%%%%%%%%%%%%%%%%%%%%%%%%%%%%%%%%%%%%%%%%%%%%%%%%%%%%%%%%%%%%%%%%%%%%%%%%
%%%%%%%%%%%%%%%%%%%%%%%%%%%%%%%%%%%%%%%%%%%%%%%%%%%%%%%%%%%%%%%%%%%%%%%%%%%%%%%%
\section{Introduction}

\LaTeX{} provides a mechanism to structure a large document (such as a book)
into a main file and several child files (containing the chapters)
using the |\include| command.
This mechanism is beneficial for documents
which span hundreds of pages in order to
make the source file(s) more manageable.
Moreover, compilation can be restricted to
selected child files by means of the |\includeonly| command.
The latter feature can be used to reduce the compilation time while editing
(this was significantly more useful in the earlier days of \LaTeX{})
or to generate a smaller document which is easier to navigate.
Another application of |\includeonly| is to generate
documents consisting of selected parts of the complete document.

However, there are a few drawbacks of the plain |\include| mechanism:
\begin{itemize}
\item
The child files cannot be compiled on their own,
they can only be compiled via the main file.
A naive editing environment
(such as a text editor with an option
to have the current file processed by \LaTeX)
may require one to switch to the main file before compiling;
attempting to compile the child file produces errors.
\item
The main file must be modified (each time)
to adjust the |\includeonly| command
to the present needs. This easily leaves the main file in a messy state.
\item
The generated document will always carry the filename
of the main document. This is inconvenient if
several child files are to be compiled and
to be kept for distribution.
\end{itemize}

The present package provides a simple interface
to make child files individually compilable by \LaTeX{}.
Compiling a child file then has the same effect as compiling
the main file with an |\includeonly| command
to select the appropriate child.
Moreover the generated document will carry the name of the child
rather than the main file.
This resolves all three above issues.

This feature is meant to make the editing of books,
thesis documents and lecture notes somewhat more convenient.
However, the package can also be used efficiently for
composing a series of documents (such as exercise sheets)
which are typically distributed individually.
It then assists the author in generating the individual documents
(potentially in different versions)
as well as a document containing the collected series.
Another application is in developing style files
or other kinds of included material
where compilation of the style file could redirect
to a sample or test file.

%%%%%%%%%%%%%%%%%%%%%%%%%%%%%%%%%%%%%%%%%%%%%%%%%%%%%%%%%%%%%%%%%%%%%%%%%%%%%%%%
%%%%%%%%%%%%%%%%%%%%%%%%%%%%%%%%%%%%%%%%%%%%%%%%%%%%%%%%%%%%%%%%%%%%%%%%%%%%%%%%
\section{Usage}

First of all, the package \textsf{childdoc} is \emph{not} a standard
\LaTeXe{} |.sty| style file! Therefore it needs to be invoked in
a non-standard way.

%%%%%%%%%%%%%%%%%%%%%%%%%%%%%%%%%%%%%%%%%%%%%%%%%%%%%%%%%%%%%%%%%%%%%%%%%%%%%%%%
\subsection{Included Files}
\label{sec:include}

%%%%%%%%%%%%%%%%%%%%%%%%%%%%%%%%%%%%%%%%
\DescribeMacro{\childdocmain}
To use the package, add the commands
\begin{center}
\begin{tabular}{l}
|\input{childdoc.def}|\\
|\childdocmain{}|\\
\end{tabular}
\end{center}
at the very top of the main \LaTeX{} file,
in particular \emph{before} the |\documentclass| statement!
The argument of |\childdocmain| should be left empty
(but it must be present).

%%%%%%%%%%%%%%%%%%%%%%%%%%%%%%%%%%%%%%%%
\DescribeMacro{\childdocof}
Furthermore, add the commands
\begin{center}
\begin{tabular}{l}
|\input{childdoc.def}|\\
|\childdocof{|\textit{main}|}|\\
\end{tabular}
\end{center}
at the top of every child file \textit{child}
which is included by |\include{|\textit{child}|}|
from within the main file
(or at least for those files to be compiled individually).
The argument \textit{main} must be the filename of the main file.

There are a couple of
considerations in setting up the main and child documents:

%%%%%%%%%%%%%%%%%%%%%%%%%%%%%%%%%%%%%%%%
\paragraph{Restrictions.}

Please note the following restrictions:
\begin{itemize}
\item
|\childdocmain| must be called with one argument \textit{main}
to ensure compatibility with earlier version of the package.
It must either be empty (|\childdocmain{}|)
or precisely match the filename of the main file in which it is specified.
See \secref{sec:detection} for further information.
\item
The filename \textit{main} must be specified without the |.tex| extension.
\item
The filename \textit{main} is case sensitive
(even in case-insensitive file systems)
due to internal string comparison.
\item
The argument \textit{main} should be fully expanded, it cannot be a macro.
\item
Subdirectories and special characters should be avoided in filenames.
\item
The command |\childdocmain{|\textit{main}|}| must be followed by a whitespace.
It should not be followed immediately by another command
or by a comment mark `|%|'.
This is because the \TeX{} parser reads the token immediately following
the argument of |\childdocmain| and puts it
at the beginning of every child section;
however, a white\-space is ignored.
\end{itemize}

%%%%%%%%%%%%%%%%%%%%%%%%%%%%%%%%%%%%%%%%
\paragraph{Content of Main File.}

It is advisable to place all content in the child files included by |\include|.
Any output contained in the main file will appear in all child documents
unless suppressed manually;
it cannot be suppressed automatically by the |\includeonly| directive
and thus should normally be avoided.
A method to include some content in the main file
by means of conditional processing is described in \secref{sec:conditional}.

%%%%%%%%%%%%%%%%%%%%%%%%%%%%%%%%%%%%%%%%
\paragraph{Page Numbering.}

When only a part of the document is compiled,
the appropriate numbering of pages
(as well as other status parameters)
is determined from the |.aux| files.
The latter contain information from previous passes.
However this information needs to propagate through
all intermediate child documents.
Therefore the page numbering in child documents may well
be inconsistent until the complete document is compiled at least once.

A useful (if unconventional) way to always ensure a consistent
page numbering is to restart the numbering in each child document
and denote the pages by `\textit{child}|.|\textit{page}'
where \textit{child} represents the chapter/section number of the child file.
This can be achieved by the command
|\numberwithin{page}{|\textit{child}|}|
of the \textsf{amsmath} package
where \textit{child} can be |chapter| or |section|
depending on the chosen structuring.
Alternatively, one can modify the macro |\thepage| appropriately
and reset the counter |page| at the start of each child file.

%%%%%%%%%%%%%%%%%%%%%%%%%%%%%%%%%%%%%%%%%%%%%%%%%%%%%%%%%%%%%%%%%%%%%%%%%%%%%%%%
\subsection{Conditional Processing}
\label{sec:conditional}

The package provides a mechanism to compile different versions
of a document. To customise the versions further some conditional processing
can come in handy to distinguish which version is being compiled.
The package provides two macros to describe the compilation context:

%%%%%%%%%%%%%%%%%%%%%%%%%%%%%%%%%%%%%%%%
\DescribeMacro{\ifchilddoc}
The conditional |\ifchilddoc| distinguishes between the compilation of
child documents and the main document:
%
\begin{center}
|\ifchilddoc |\textit{child-code}| |[|\||else |\textit{main-code}]| \||fi|
\end{center}

%%%%%%%%%%%%%%%%%%%%%%%%%%%%%%%%%%%%%%%%
\DescribeMacro{\childdocname}
\DescribeMacro{\childdocjob}
The macro |\childdocname| contains the filename (without extension)
of the main or child file being processed.
Note that |\childdocjob| will always contain the name of the main file.

%%%%%%%%%%%%%%%%%%%%%%%%%%%%%%%%%%%%%%%%
\paragraph{Title Page.}

Conditional processing can be used to include a title or banner page
in the main document when proper precautions are taken.
Importantly, the code in the main file should ensure that the page counter
(as well as other status parameters which are stored in the |.aux| files)
takes the same value after the conditional processing.
Otherwise the page numbers may take divergent values
depending on which part is compiled.

For example, a title page could be declared by:
%
\begin{center}
\begin{tabular}{l}
|\ifchilddoc\||else|\\
|\addtocounter{page}{-1}|\\
\textit{code for title page}\\
|\newpage|\\
|\||fi|
\end{tabular}
\end{center}
%
A banner page for the child documents can be generated by:
%
\begin{center}
\begin{tabular}{l}
|\ifchilddoc|\\
|\addtocounter{page}{-1}|\\
\textit{code for banner page}\\
|\newpage|\\
|\||fi|
\end{tabular}
\end{center}
%
Here one could write a message such as:
\begin{center}
|This is the part \childdocname{} of \childdocjob{}.|
\end{center}

%%%%%%%%%%%%%%%%%%%%%%%%%%%%%%%%%%%%%%%%%%%%%%%%%%%%%%%%%%%%%%%%%%%%%%%%%%%%%%%%
\subsection{Flags}
\label{sec:flags}

The package makes it easy to generate different versions
of the main or child documents.
To this end compilation flags can be defined
and assigned different default values.
They will be particularly useful in conjunction
with the forwarding mechanism described in \secref{sec:forward}.

For example, it may be useful to have a flag |\version|
which can be set to |draft| or |final|.
The document source will contain some conditional code
depending on the value of |\version|.
Suppose further, the flag should default to |final| for the main file
and to |draft| for child files
which is a natural assignment for editing the document.
This is achieved by placing the following code
in the preamble of the main document
(below the |\childdocmain| directive):
%
\begin{center}
\begin{tabular}{l}
|\ifchilddoc|\\
|\providecommand{\version}{draft}|\\
|\||else|\\
|\providecommand{\version}{final}|\\
|\||fi|
\end{tabular}
\end{center}
%
The definition by |\providecommand| makes sure
that previous definitions are not overwritten.
Further statements |\providecommand{\version}{...}|
can thus be added before the above code to override it.

For the main file, one might add a line
(between |\childdocmain| and the above block)
%
\begin{center}
|%\ifchilddoc\||else\providecommand{\version}{draft}\||fi|
\end{center}
%
which can be uncommented to produce a draft version.
Likewise one can add a line to the very top of a child file
(above the |\childdocof{|\textit{main}|}| directive)
%
\begin{center}
|%\providecommand{\version}{final}|
\end{center}
%
which can be uncommented to produce the final version of this child document.

%%%%%%%%%%%%%%%%%%%%%%%%%%%%%%%%%%%%%%%%%%%%%%%%%%%%%%%%%%%%%%%%%%%%%%%%%%%%%%%%
\subsection{Forwarding}
\label{sec:forward}

Different versions of the main or child documents
using compilation flags as described in \secref{sec:flags}
can be (permanently) stored in different files
for convenient compilation, viewing and distribution.
To this end, the package defines a command
to pass on compilation to a different file:

%%%%%%%%%%%%%%%%%%%%%%%%%%%%%%%%%%%%%%%%
\DescribeMacro{\childdocforward}
The command |\childdocforward| redirects processing to
another source file:
%
\begin{center}
\begin{tabular}{l}
|\input{childdoc.def}|\\
|\childdocforward[|\textit{main}|]{|\textit{dest}|}|\\
\end{tabular}
\end{center}
%
The argument \textit{dest} is the destination file
(without extension).
It should be the main file or one of the child files.
Note that further \textsf{childdoc} directives
such as |\childdocof| and |\childdocforward|
in the indicated file will be processed in this form.
The optional argument \textit{main}
passes on directly to the main file \textit{main}
while pretending to compile the child \textit{dest}.
This form behaves as if \textit{dest}
issues |\childdocof{|\textit{main}|}| right away,
and no further \textsf{childdoc} directives will be processed.

%%%%%%%%%%%%%%%%%%%%%%%%%%%%%%%%%%%%%%%%
\DescribeMacro{\...prefix}
In the alternative form |\childdocforwardprefix|,
%
\begin{center}
\begin{tabular}{l}
|\input{childdoc.def}|\\
|\childdocforwardprefix[|\textit{main}|]{|\textit{prefix}|}{|\textit{dest}|}|
\end{tabular}
\end{center}
%
the destination file is determined by a pattern
depending on the current file:
To make this work, the current file must be called
`{\textit{prefix}\hspace{0.2em}\textit{suffix}}'
with \textit{prefix} matching precisely the argument.
Processing is then passed on to the file
`{\textit{dest}\hspace{0.2em}\textit{suffix}}'.
Surely, the same effect is achieved by
directly specifying the
argument `{\textit{dest}\hspace{0.2em}\textit{suffix}}'
in the first form.
However, that requires to set up a different file
for each child. With the alternative form of the command
all these files can have exactly the same content
which simplifies setting them up and maintaining them.

For example, the following file |draft.tex|
with a compilation flag |\version| as described in \secref{sec:flags}
compiles the main document as a draft:
%
\begin{center}
\begin{tabular}{l}
|\def\version{draft}|\\
|\input{childdoc.def}|\\
|\childdocforward{|\textit{main}|}|
\end{tabular}
\end{center}
%
Likewise, the following files |final|\textit{nn}|.tex|
compile the final version of the child document
|child|\textit{nn}|.tex|:
%
\begin{center}
\begin{tabular}{l}
|\def\version{final}|\\
|\input{childdoc.def}|\\
|\childdocforwardprefix{final}{child}|
\end{tabular}
\end{center}
%

Note that when several versions of a main file and/or of each child file
are to be generated, it may be convenient to set up a |Makefile| or
shell script to automatise the process.

%%%%%%%%%%%%%%%%%%%%%%%%%%%%%%%%%%%%%%%%%%%%%%%%%%%%%%%%%%%%%%%%%%%%%%%%%%%%%%%%
\subsection{Command Line Processing}
\label{sec:commandline}

The effect of redirection files can also be achieved by invoking
the \LaTeX{} compiler with a more elaborate command line.
Most conveniently this should be done as part
of a shell script or a |Makefile|.

When using \textsf{childdoc} in the main file, the following
command lines effectively perform a redirection
(note that depending on the shell being used,
backslashes may have to be doubled: `|\|' $\to$ `|\\|'):
%
\begin{center}
|... -jobname "|\textit{target}|" |\\|"|[\textit{flags}]%
|\input{childdoc.def}\childdocforward[|\textit{main}|]{|\textit{dest}|}"|
\end{center}
%
Here \textit{target} is the name of the output file,
\textit{main} is the name of the main file
and \textit{dest} is the name of the main or child file to be processed
(all filenames without extensions).
The optional argument \textit{main} can be omitted
if \textit{main} matches \textit{dest}.
Optionally, compilation \textit{flags} can be defined via |\def| commands.
This command line makes the \TeX{} engine believe
it is compiling the file \textit{target}
whose content is specified as the latter parameter.
The provided code then forwards the processing to
\textit{main} or \textit{dest} as described in \secref{sec:forward}.

%%%%%%%%%%%%%%%%%%%%%%%%%%%%%%%%%%%%%%%%%%%%%%%%%%%%%%%%%%%%%%%%%%%%%%%%%%%%%%%%
\subsection{Include by Input}
\label{sec:input}

Including child documents by |\include| has some restrictions by design.
Most notably, the content of a child document always occupies
its own set of pages; pages cannot be shared between child documents.
Usually, this behaviour makes perfect sense
because each child document contain an essential part of the document.
However, in some situations it may be desirable to compose
a document from a collection of parts
without having mandatory page breaks between then.
For this case, the package
provides a mechanism to include parts
by |\input| which can also be processed individually.
However, by construction this mechanism
requires manual handling of the content to be output.

%%%%%%%%%%%%%%%%%%%%%%%%%%%%%%%%%%%%%%%%
\DescribeMacro{\ifchilddocmanual}
The main file should be prepared as usual, see \secref{sec:include}.
However, the document body must make a distinction
between processing of an individual part and of the main document, e.g.:
%
\begin{center}
\begin{tabular}{l}
|\ifchilddocmanual|\\
|\input{\childdocname}|\\
|\||else|\\
\textit{document body with }|\input{|\textit{part}|}|\\
|\||fi|
\end{tabular}
\end{center}
%
The conditional |\ifchilddocmanual| is true whenever
a part to be included by |\input| is being compiled,
and the name of the part is stored in |\childdocname|.

%%%%%%%%%%%%%%%%%%%%%%%%%%%%%%%%%%%%%%%%
\DescribeMacro{\childdocby}
Each part to be included by |\input| should start with:
%
\begin{center}
\begin{tabular}{l}
|\input{childdoc.def}|\\
|\childdocby{|\textit{main}|}|\\
\end{tabular}
\end{center}
%
The directive |\childdocby| is similar to |\childdocof|
described in \secref{sec:include},
but the subsequent selection of content must be done manually.
To that end, both |\ifchilddoc| and |\ifchilddocmanual|
will be true upon processing of a part,
and the name of the part is stored in |\childdocname|.
Note that |\jobname| will be set to the filename of the current part
so that each part receives an individual |.aux| file
that does not interfere with the |.aux| file(s) of the main document.
This behaviour can be altered by the alternative form
|\childdocby[*]{|\textit{main}|}| (with a non-empty optional argument)
which uses the |.aux| file of the main document
by setting |\jobname| to \textit{main}.

%%%%%%%%%%%%%%%%%%%%%%%%%%%%%%%%%%%%%%%%%%%%%%%%%%%%%%%%%%%%%%%%%%%%%%%%%%%%%%%%
\subsection{Driver Development}
\label{sec:driver}

The \textsf{childdoc} mechanism can also be use for the development
of definition files such as \LaTeX{} styles or classes.
This case differs from the above setup with multiple parts
included by |\include| in that no |\includeonly| should be invoked.
This can be achieved by starting the include file
(before |\ProvidesPackage|) with:
%
\begin{center}
\begin{tabular}{l}
|\input{childdoc.def}|\\
|\childdocforward{|\textit{main}|}|\\
\end{tabular}
\end{center}
%
or alternatively with:
%
\begin{center}
\begin{tabular}{l}
|\input{childdoc.def}|\\
|\childdocby{|\textit{main}|}|\\
\end{tabular}
\end{center}
%
Both forms have slightly different effects as described above.
The main file is prepared as usual, see \secref{sec:include}.

%%%%%%%%%%%%%%%%%%%%%%%%%%%%%%%%%%%%%%%%%%%%%%%%%%%%%%%%%%%%%%%%%%%%%%%%%%%%%%%%
\subsection{Legacy Detection}
\label{sec:detection}

The directive |\childdocmain| in the main file can detect
whether the complete document or merely a child is to be compiled
even without using the directive |\childdocof|.
This method is deprecated because it is less robust
and there is no compelling reason to use it;
it is merely provided for backward compatibility
and it may be removed in future versions.

If the detection mechanism is to be used,
it is mandatory to correctly specify
the filename of the main file as the argument of |\childdocmain|:
%
\begin{center}
\begin{tabular}{l}
|\input{childdoc.def}|\\
|\childdocmain{|\textit{main}|}|\\
\end{tabular}
\end{center}
%
If |\jobname| does not match the argument \textit{main} of |\childdocmain|,
it is assumed that |\jobname| points to the child file to be compiled.
When using |\childdocmain| with the main file specified as argument,
it suffices to start a child file
with just |\input{|\textit{main}|}|
without loading of the package and using |\childdocof|.
If instead all processing is done
with the appropriate \textsf{childdoc} directives,
the argument of \textit{main} of |\childdocmain| can be empty.

An alternative version of the command line processing described
in \secref{sec:commandline} using the detection mechanism reads:
%
\begin{center}
|... -jobname "|\textit{target}|" "|[\textit{flags}]%
[|\def\jobname{|\textit{dest}|}|]|\input{|\textit{main}|}"|
\end{center}

%%%%%%%%%%%%%%%%%%%%%%%%%%%%%%%%%%%%%%%%%%%%%%%%%%%%%%%%%%%%%%%%%%%%%%%%%%%%%%%%
\subsection{Manual Code}
\label{sec:manual}

In case one cannot be certain whether the definitions file |childdoc.def|
is installed on the target \TeX{} distribution
and one prefers not to ship it,
it is conceivable to paste a few relevant commands into the sources.

To that end, drop all statements |\input{childdoc.def}|
and perform the replacements as outlined below.
Instead of |\childdocmain{|\textit{main}|}| add the following code
to the top of the main file:
%
\begin{center}
\begin{tabular}{l}
|\||ifdefined\childdocname\endinput\||fi\newif\ifchilddoc|\\
|\edef\childdocname{\scantokens\expandafter{\jobname\noexpand}}|\\
|\def\childdocmain{|\textit{main}|}\||ifx\childdocmain\childdocname\||else|\\
|\childdoctrue\includeonly{\childdocname}\let\jobname\childdocmain\||fi|\\
\end{tabular}
\end{center}
%
Instead of |\childdocof{|\textit{main}|}| just include the main file
at the top of each child file:
%
\begin{center}
|\input{|\textit{main}|}|
\end{center}
%
A simple redirection |\childdocforward{|\textit{dest}|}| is achieved by:
%
\begin{center}
|\def\jobname{|\textit{dest}|}\input{\jobname}|
\end{center}
%
The redirection with prefix
|\childdocforwardprefix[|\textit{prefix}|]{|\textit{dest}|}|
is accomplished by:
%
\begin{center}
\begin{tabular}{l}
|{\edef\jobname{\scantokens\expandafter{\jobname\noexpand}}|\\
|\def\redirectjob |\textit{prefix}|#1~~~{\gdef\jobname{|\textit{dest}|#1}}|\\
|\expandafter\redirectjob\jobname~~~}\input{\jobname}|
\end{tabular}
\end{center}

In an alternative approach,
child documents can be compiled by a specific command line
without additional code or specific definitions:
%
\begin{center}
|... -jobname "|\textit{target}|" "|[\textit{flags}]%
|\includeonly{|\textit{dest}|}\input{|\textit{main}|}"|
\end{center}
%

%%%%%%%%%%%%%%%%%%%%%%%%%%%%%%%%%%%%%%%%%%%%%%%%%%%%%%%%%%%%%%%%%%%%%%%%%%%%%%%%
%%%%%%%%%%%%%%%%%%%%%%%%%%%%%%%%%%%%%%%%%%%%%%%%%%%%%%%%%%%%%%%%%%%%%%%%%%%%%%%%
\section{Information}

%%%%%%%%%%%%%%%%%%%%%%%%%%%%%%%%%%%%%%%%%%%%%%%%%%%%%%%%%%%%%%%%%%%%%%%%%%%%%%%%
\subsection{Copyright}

Copyright \copyright{} 2017--2018 Niklas Beisert

This work may be distributed and/or modified under the
conditions of the \LaTeX{} Project Public License, either version 1.3
of this license or (at your option) any later version.
The latest version of this license is in
  \url{http://www.latex-project.org/lppl.txt}
and version 1.3 or later is part of all distributions of \LaTeX{}
version 2005/12/01 or later.

This work has the LPPL maintenance status `maintained'.

The Current Maintainer of this work is Niklas Beisert.

This work consists of the files |README.txt|, |childdoc.ins| and |childdoc.dtx|
as well as the derived files |childdoc.def|, |cdocsamp.tex|
with |cdocsch1.tex|, |cdocsch2.tex|, |cdocspt3.tex|, |cdocspt4.tex|,
|cdocsdrf.tex|, |cdocsfn1.tex|, |cdocsfn2.tex|
as well as |childdoc.pdf|.

%%%%%%%%%%%%%%%%%%%%%%%%%%%%%%%%%%%%%%%%%%%%%%%%%%%%%%%%%%%%%%%%%%%%%%%%%%%%%%%%
\subsection{Files and Installation}

The package consists of the files:
%
\begin{center}
\begin{tabular}{ll}
    |README.txt|   & readme file \\
    |childdoc.ins| & installation file \\
    |childdoc.dtx| & source file \\
    |childdoc.def| & definition file \\
    |cdocsamp.tex| & sample main file \\
    |cdocsch1.tex| & sample include file \\
    |cdocsch2.tex| & sample include file \\
    |cdocspt3.tex| & sample part file \\
    |cdocspt4.tex| & sample part file \\
    |cdocsdrf.tex| & sample redirection file \\
    |cdocsfn1.tex| & sample redirection file \\
    |cdocsfn2.tex| & sample redirection file \\
    |childdoc.pdf| & manual
\end{tabular}
\end{center}
%
The distribution consists of the files
|README.txt|, |childdoc.ins| and |childdoc.dtx|.
%
\begin{itemize}
\item
Run (pdf)\LaTeX{} on |childdoc.dtx|
to compile the manual |childdoc.pdf| (this file).
\item
Run \LaTeX{} on |childdoc.ins| to create the definitions file |childdoc.def|
and the sample |cdocsamp.tex| with include files
|cdocsch1.tex|, |cdocsch2.tex|, |cdocspt3.tex|, |cdocspt4.tex|,
|cdocsdrf.tex|, |cdocsfn1.tex|, |cdocsfn2.tex|.
Then copy the file |childdoc.def| to an appropriate directory of your \LaTeX{}
distribution, e.g.\ \textit{texmf-root}|/tex/latex/childdoc|.
\end{itemize}

%%%%%%%%%%%%%%%%%%%%%%%%%%%%%%%%%%%%%%%%%%%%%%%%%%%%%%%%%%%%%%%%%%%%%%%%%%%%%%%%
\subsection{Related CTAN Packages}

There are several other packages which offer a similar functionality:
%
\begin{itemize}
\item
The packages
\href{http://ctan.org/pkg/docmute}{\textsf{docmute}},
\href{http://ctan.org/pkg/includex}{\textsf{includex}} and
\href{http://ctan.org/pkg/standalone}{\textsf{standalone}}
provide commands to include only the document body of
a child file thus allowing both files to be compiled individually.
\item
The packages \href{http://ctan.org/pkg/subdocs}{\textsf{subdocs}}
and \href{http://ctan.org/pkg/subfiles}{\textsf{subfiles}}
provide structures in which the main and child documents can be
encapsulated and allowing them to be compiled individually.
The inclusion mechanism is different from the conventional |\include|.
\item
The package \href{http://ctan.org/pkg/combine}{\textsf{combine}}
is an elaborate solution to combine several documents into one.
\end{itemize}
%
See also the CTAN topic \href{http://ctan.org/topic/subdocs}{\textsf{subdocs}}
for further related packages.
The present package differs from the above solutions in that
a document structure constructed with the conventional |\include| mechanism
just needs two extra commands at the top of every file
such that all constituent files can be compiled individually.

%%%%%%%%%%%%%%%%%%%%%%%%%%%%%%%%%%%%%%%%%%%%%%%%%%%%%%%%%%%%%%%%%%%%%%%%%%%%%%%%
%\subsection{Feature Suggestions}
%
%The following is a list of features which may be useful for future
%versions of this package:
%%
%\begin{itemize}
%\item
%\ldots
%\end{itemize}

%%%%%%%%%%%%%%%%%%%%%%%%%%%%%%%%%%%%%%%%%%%%%%%%%%%%%%%%%%%%%%%%%%%%%%%%%%%%%%%%
\subsection{Revision History}

%%%%%%%%%%%%%%%%%%%%%%%%%%%%%%%%%%%%%%%%
\paragraph{v2.0:} 2018/12/30

\begin{itemize}
\item
immediate forward processing
\item
added |\childdocby| mechanism
\item
manual restructured
\end{itemize}

%%%%%%%%%%%%%%%%%%%%%%%%%%%%%%%%%%%%%%%%
\paragraph{v1.6:} 2018/01/17

\begin{itemize}
\item
application for development of include files
\item
corrections to manual
\end{itemize}

%%%%%%%%%%%%%%%%%%%%%%%%%%%%%%%%%%%%%%%%
\paragraph{v1.5:} 2017/05/21

\begin{itemize}
\item
more complete structuring introduced
\item
|\childdocof| introduced
\item
|\childdoc| renamed to |\childdocmain|
\item
|\childredirect| renamed to |\childdocforward| and |\childdocforwardprefix|
and functionality expanded
\end{itemize}

%%%%%%%%%%%%%%%%%%%%%%%%%%%%%%%%%%%%%%%%
\paragraph{v1.0:} 2017/04/27

\begin{itemize}
\item
manual and install package
\item
first version published on CTAN
\end{itemize}

%%%%%%%%%%%%%%%%%%%%%%%%%%%%%%%%%%%%%%%%
\paragraph{v0.6:} 2017/04/26

\begin{itemize}
\item
redirection mechanism added
\end{itemize}

%%%%%%%%%%%%%%%%%%%%%%%%%%%%%%%%%%%%%%%%
\paragraph{v0.5:} 2017/04/26

\begin{itemize}
\item
functionality in definition file
\end{itemize}


%%%%%%%%%%%%%%%%%%%%%%%%%%%%%%%%%%%%%%%%%%%%%%%%%%%%%%%%%%%%%%%%%%%%%%%%%%%%%%%%
%%%%%%%%%%%%%%%%%%%%%%%%%%%%%%%%%%%%%%%%%%%%%%%%%%%%%%%%%%%%%%%%%%%%%%%%%%%%%%%%
%%%%%%%%%%%%%%%%%%%%%%%%%%%%%%%%%%%%%%%%%%%%%%%%%%%%%%%%%%%%%%%%%%%%%%%%%%%%%%%%
\appendix

\settowidth\MacroIndent{\rmfamily\scriptsize 000\ }

 \DocInput{childdoc.dtx}

\end{document}
%</driver>
% \fi
%
% %%%%%%%%%%%%%%%%%%%%%%%%%%%%%%%%%%%%%%%%%%%%%%%%%%%%%%%%%%%%%%%%%%%%%%%%%%%%%%
% %%%%%%%%%%%%%%%%%%%%%%%%%%%%%%%%%%%%%%%%%%%%%%%%%%%%%%%%%%%%%%%%%%%%%%%%%%%%%%
% \section{Sample}
%\iffalse
%<*samplemain>
%\fi
%
% The following presents a sample document
% with two chapters, two parts, a title page,
% a compile flag as well as three forwarding files to set the flag.
% It consists of eight |.tex| files:
% \begin{center}
% \begin{tabular}{ll}
% |cdocsamp.tex|&main file\\
% |cdocsch1.tex|&include file for chapter 1\\
% |cdocsch2.tex|&include file for chapter 2\\
% |cdocspt3.tex|&include file for part 3\\
% |cdocspt4.tex|&include file for part 4\\
% |cdocsdrf.tex|&forwarding file for main file in draft mode\\
% |cdocsfi1.tex|&forwarding file for final version of chapter 1\\
% |cdocsfi2.tex|&forwarding file for final version of chapter 2\\
% \end{tabular}
% \end{center}
% Each of the eight files can be compiled directly by the \LaTeX{} compiler.
%
% %%%%%%%%%%%%%%%%%%%%%%%%%%%%%%%%%%%%%%
% \paragraph{Main File.}
%
% The main file is called |cdocsamp.tex|.
%
% Load the \textsf{childdoc} definitions and
% declare the filename for the main document:
%    \begin{macrocode}
\input{childdoc.def}
\childdocmain{}
%    \end{macrocode}

% Optional override for |\version| flag:
%    \begin{macrocode}
%%\ifchilddoc\else\providecommand{\version}{draft}\fi
%    \end{macrocode}

% Define the default values for the |\version| flag
% (|final| for the main file and |draft| for childs):
%    \begin{macrocode}
\ifchilddoc
\providecommand{\version}{draft}
\else
\providecommand{\version}{final}
\fi
%    \end{macrocode}

% Load the standard document class:
%    \begin{macrocode}
\documentclass[12pt]{article}
%    \end{macrocode}

% Start the document body:
%    \begin{macrocode}
\begin{document}
%    \end{macrocode}

% Declare a title page.
% Print title, part of document being processed and version flag:
%    \begin{macrocode}
\addtocounter{page}{-1}
\begin{center}
{\LARGE\bfseries{}childdoc example\par}
\vspace{1cm}
\ifchilddoc
\ifchilddocmanual part\else chapter\fi:
`\childdocname' of `\childdocjob'\par
\else
main document: `\childdocjob'\par
\fi
version: \version\par
\end{center}
\newpage
%    \end{macrocode}

% Manually include selected file,
% otherwise process as usual:
%    \begin{macrocode}
\ifchilddocmanual
\section*{part `\childdocname'}
\input{\childdocname}
\else
%    \end{macrocode}

% Include the two chapters:
%    \begin{macrocode}
\include{cdocsch1}
\include{cdocsch2}
%    \end{macrocode}

% Include the two parts unless only chapters should be displayed:
%    \begin{macrocode}
\ifchilddoc\else
\section{part three}
\input{cdocspt3}
\section{part four}
\input{cdocspt4}
\fi
%    \end{macrocode}

% Process as usual until here:
%    \begin{macrocode}
\fi
%    \end{macrocode}

% End of document body:
%    \begin{macrocode}
\end{document}
%    \end{macrocode}
%\iffalse
%</samplemain>
%\fi
%
% %%%%%%%%%%%%%%%%%%%%%%%%%%%%%%%%%%%%%%
% \paragraph{Chapter Include Files.}
%
% The include files are called |cdocsch1.tex| and |cdocsch2.tex|.
%
%\iffalse
%<*samplechap1|samplechap2>
%\fi

% Optional override for |\version| flag:
%    \begin{macrocode}
%%\providecommand{\version}{final}
%    \end{macrocode}

% Include the main document:
%    \begin{macrocode}
\input{childdoc.def}
\childdocof{cdocsamp}
%    \end{macrocode}

%\iffalse
%</samplechap1|samplechap2>
%\fi
%
%\iffalse
%<*samplechap1>
%\fi
% Some text for chapter 1:
%    \begin{macrocode}
\section{one}
some text in chapter one
%    \end{macrocode}

%\iffalse
%</samplechap1>
%\fi
% Some text for chapter 2:
%\iffalse
%<*samplechap2>
%\fi
%    \begin{macrocode}
\section{two}
more text in chapter two
%    \end{macrocode}

%\iffalse
%</samplechap2>
%\fi
%
% %%%%%%%%%%%%%%%%%%%%%%%%%%%%%%%%%%%%%%
% \paragraph{Part Include Files.}
%
% The include files are called |cdocspt3.tex| and |cdocspt4.tex|.
%
%\iffalse
%<*samplepart3|samplepart4>
%\fi

% Optional override for |\version| flag:
%    \begin{macrocode}
%%\providecommand{\version}{final}
%    \end{macrocode}

% Include the main document:
%    \begin{macrocode}
\input{childdoc.def}
\childdocby{cdocsamp}
%    \end{macrocode}

%\iffalse
%</samplepart3|samplepart4>
%\fi
%
%\iffalse
%<*samplepart3>
%\fi
% Some text for part 3:
%    \begin{macrocode}
some text in part three
%    \end{macrocode}

%\iffalse
%</samplepart3>
%\fi
% Some text for part 4:
%\iffalse
%<*samplepart4>
%\fi
%    \begin{macrocode}
more text in part four
%    \end{macrocode}

%\iffalse
%</samplepart4>
%\fi
%
% %%%%%%%%%%%%%%%%%%%%%%%%%%%%%%%%%%%%%%
% \paragraph{Forwarding for a Complete Draft.}
%
% The following forwarding file |cdocsdrf.tex|
% compiles the main document in draft mode:
%\iffalse
%<*sampledraft>
%\fi
%    \begin{macrocode}
\def\version{draft}
\input{childdoc.def}
\childdocforward{cdocsamp}
%    \end{macrocode}

%\iffalse
%</sampledraft>
%\fi
%
% %%%%%%%%%%%%%%%%%%%%%%%%%%%%%%%%%%%%%%
% \paragraph{Forwarding for Final Version of the Chapters.}
%
% The following forwarding files |cdocsfn1.tex| and |cdocsfn2.tex|
% (with identical content)
% compile the final versions of the child documents
% |cdocsch1.tex| and |cdocsch2.tex|, respectively:
%\iffalse
%<*samplefinal>
%\fi
%    \begin{macrocode}
\def\version{final}
\input{childdoc.def}
\childdocforwardprefix[cdocsamp]{cdocsfn}{cdocsch}
%    \end{macrocode}

%\iffalse
%</samplefinal>
%\fi
%
% %%%%%%%%%%%%%%%%%%%%%%%%%%%%%%%%%%%%%%
% \paragraph{Command Line Processing.}
%
% The following three command lines generate the output files
% |cdocscld|, |cdocscl1| and |cdocscl2|
% which should be identical to
% |cdocsdrf|, |cdocsch1| and |cdocsfn2|, respectively:
% \begin{center}
% \begin{tabular}{l}
% |latex -jobname cdocscld \|\\
% |  "\def\version{draft}\input{childdoc.def}\childdocforward{cdocsamp}"|\\
% |latex -jobname cdocscl1 \|\\
% |  "\input{childdoc.def}\childdocforward[cdocsamp]{cdocsch1}"|\\
% |latex -jobname cdocscl2 \|\\
% |  "\def\version{final}\input{childdoc.def}\childdocforward{cdocsch2}"|
% \end{tabular}
% \end{center}
% Note that the trailing backslash on each first line
% merely continues the input to the second line
% (for convenient cut ant paste).
% Furthermore, the command |latex| can be replaced by any
% of its alternative versions such as |pdflatex|.
%
% %%%%%%%%%%%%%%%%%%%%%%%%%%%%%%%%%%%%%%%%%%%%%%%%%%%%%%%%%%%%%%%%%%%%%%%%%%%%%%
% %%%%%%%%%%%%%%%%%%%%%%%%%%%%%%%%%%%%%%%%%%%%%%%%%%%%%%%%%%%%%%%%%%%%%%%%%%%%%%
% \section{Implementation}
%\iffalse
%<*package>
%\fi
%
% This section describes the definitions file |childdoc.def|.

% The definitions cannot be loaded using |\usepackage| or |\RequirePackage|
% which has a mechanism to prevent loading a style file more than once.
% When loading the definitions by means of |\input|
% multiple instances have to be prevented manually:
%\iffalse
%This code needs to be before the `\ProvidesFile' directive
%which is defined at the beginning of this file.
%Therefore it is also placed there and commented out here.
%</package>
%<*discard>
%\fi
%    \begin{macrocode}
\ifdefined\childdocmain\endinput\fi
%    \end{macrocode}
%\iffalse
%</discard>
%<*package>
%\fi
%
% \macro{\ifchilddoc}
% \macro{\ifchilddocmanual}
% The conditional |\ifchilddoc| tells whether a
% child (true) or main (false) document is being compiled.
% The conditional |\ifchilddocmanual| tells whether
% the |\includeonly| mechanism is used (false) or
% the selection of child files must be performed manually (true).
% The definitions initialise to false:
%    \begin{macrocode}
\newif\ifchilddoc
\newif\ifchilddocmanual
%    \end{macrocode}

% \macro{\childdocname}
% \macro{\childdocjob}
% The macro |\childdocname| stores the name of the main document
% to be compiled. The macro |\childdocjob| stores the name of
% the document on which the \LaTeX{} compiler was originally invoked.
% The content of |\jobname| cannot be compared
% to filenames specified in the source due to different catcodes.
% The following code rescans |\jobname|, stores the result
% in |\childdocname| and saves a copy in |\childdocjob|:
%    \begin{macrocode}
\edef\childdocname{\scantokens\expandafter{\jobname\noexpand}}
\let\childdocjob\childdocname
%    \end{macrocode}

% \macro{\childdocdisable}
% The macro |\childdocdisable| prevents the main file
% from being processed more than once.
% At this stage, the main document command |\childdocmain|
% is assumed to be called once again where it should do nothing.
% Any subsequent call to it should prevent
% a secondary processing of the main document
% It overwrites the forwarding commands
% |\childdocof| and |\childdocforward|
% with empty macros to prevent further inclusions of the main document:
%    \begin{macrocode}
\newcommand{\childdocdisable}
{
  \renewcommand{\childdocmain}[1]{\renewcommand{\childdocmain}[1]{\endinput}}
  \renewcommand{\childdocof}[1]{}
  \renewcommand{\childdocby}[2][]{}
  \renewcommand{\childdocforward}[2][]{}
  \renewcommand{\childdocdisable}{}
}
%    \end{macrocode}

% \macro{\childdocmain}
% The macro |\childdocmain| is to be called at the top of the main file
% with nothing or the main filename (without extension) as argument.
% First, it breaks loops.
% If the argument is not empty and does not match |\childdocname|
% (which is set by the first inclusion of |childdoc.def|),
% |\ifchilddoc| is set to true, |\includeonly| is applied to the child file
% and |\jobname| is set to the main file
% (for proper handling of |.aux| files):
%    \begin{macrocode}
\newcommand{\childdocmain}[1]
{
  \childdocdisable\childdocmain{}
  \if?#1?\else
    \begingroup
      \def\childdoctmp{#1}
      \ifx\childdoctmp\childdocname
        \def\childdoctmp{}
      \else
        \def\childdoctmp
        {
          \childdoctrue
          \includeonly{\childdocname}
          \def\childdocjob{#1}
          \def\jobname{#1}
        }
      \fi
      \expandafter
    \endgroup
    \childdoctmp
  \fi
}
%    \end{macrocode}

% \macro{\childdocof}
% The command |\childdocof| redirects
% compilation to the main file |#1|.
%    \begin{macrocode}
\newcommand{\childdocof}[1]
{
  \childdocdisable
  \childdoctrue
  \includeonly{\childdocname}
  \def\jobname{#1}
  \def\childdocjob{#1}
  \input{#1}
}
%    \end{macrocode}

% \macro{\childdocby}
% The command |\childdocby| ....
%    \begin{macrocode}
\newcommand{\childdocby}[2][]
{
  \childdocdisable
  \childdoctrue
  \childdocmanualtrue
  \if?#1?\else
    \def\jobname{#2}
  \fi
  \def\childdocjob{#2}
  \input{#2}
  \endinput
}
%    \end{macrocode}

% \macro{\childdocforward}
% The command |\childdocforward| redirects
% compilation to the main file or
% (if the optional argument is given) a child file.
% Parameters are set as if the main file
% or a child file starting with |\childdocof| was compiled.
% Then compilation is handed over to the main file:
%    \begin{macrocode}
\newcommand{\childdocforward}[2][]
{
  \begingroup
    \if?#1?
      \def\childdoctmp
      {
        \def\childdocname{#2}
        \def\childdocjob{#2}
        \def\jobname{#2}
        \input{#2}
        \endinput
      }
    \else
      \def\childdoctmp
      {
        \childdocdisable
        \def\childdocname{#2}
        \childdoctrue
        \includeonly{#2}
        \def\childdocjob{#1}
        \def\jobname{#1}
        \input{#1}
        \endinput
      }
    \fi
    \expandafter
  \endgroup
  \childdoctmp
}
%    \end{macrocode}

% \macro{\childdocforwardprefix}
% The command |\childdocforwardprefix| redirects
% compilation to the main or a child file by means of a pattern.
% The prefix |#1| in the current filename is replaced by |#2|
% and the suffix of the current filename is kept
% (it is assumed that the filename does not contain the substring `|~~~|'
% which is used as a delimiter).
% Compilation is handed over to the new file by |\childdocforward|:
%    \begin{macrocode}
\newcommand{\childdocforwardprefix}[3][]
{
  \begingroup
    \def\childdocextract #2##1~~~{\def\childdoctmp{\childdocforward[#1]{#3##1}}}
    \expandafter\childdocextract\childdocname~~~
    \expandafter
  \endgroup
  \childdoctmp
}
%    \end{macrocode}

% \macro{\childdoc}
% The deprecated macro |\childdoc| is a legacy version of |\childdocmain|:
%    \begin{macrocode}
\newcommand{\childdoc}{\childdocmain}
%    \end{macrocode}

% \macro{\childdocredirect}
% The deprecated macro |\childdocredirect| is a legacy version
% of |\childdocforward| and |\childdocforwardprefix|:
%    \begin{macrocode}
\newcommand{\childdocredirect}[2][]
{
  \begingroup
    \if?#1?
      \def\childdoctmp{\childdocforward{#2}}
    \else
      \def\childdoctmp{\childdocforwardprefix{#1}{#2}}
    \fi
    \expandafter
  \endgroup
  \childdoctmp
}
%    \end{macrocode}

%\iffalse
%</package>
%\fi
%
\endinput

\childdocof{cdocsamp}
%    \end{macrocode}

%\iffalse
%</samplechap1|samplechap2>
%\fi
%
%\iffalse
%<*samplechap1>
%\fi
% Some text for chapter 1:
%    \begin{macrocode}
\section{one}
some text in chapter one
%    \end{macrocode}

%\iffalse
%</samplechap1>
%\fi
% Some text for chapter 2:
%\iffalse
%<*samplechap2>
%\fi
%    \begin{macrocode}
\section{two}
more text in chapter two
%    \end{macrocode}

%\iffalse
%</samplechap2>
%\fi
%
% %%%%%%%%%%%%%%%%%%%%%%%%%%%%%%%%%%%%%%
% \paragraph{Part Include Files.}
%
% The include files are called |cdocspt3.tex| and |cdocspt4.tex|.
%
%\iffalse
%<*samplepart3|samplepart4>
%\fi

% Optional override for |\version| flag:
%    \begin{macrocode}
%%\providecommand{\version}{final}
%    \end{macrocode}

% Include the main document:
%    \begin{macrocode}
% \iffalse
%
% childdoc.dtx Copyright (C) 2017-2018 Niklas Beisert
%
% This work may be distributed and/or modified under the
% conditions of the LaTeX Project Public License, either version 1.3
% of this license or (at your option) any later version.
% The latest version of this license is in
%   http://www.latex-project.org/lppl.txt
% and version 1.3 or later is part of all distributions of LaTeX
% version 2005/12/01 or later.
%
% This work has the LPPL maintenance status `maintained'.
%
% The Current Maintainer of this work is Niklas Beisert.
%
% This work consists of the files childdoc.dtx and childdoc.ins
% and the derived files childdoc.def and cdocsamp.tex with
% cdocsch1.tex, cdocsch2.tex, cdocsdrf.tex, cdocsfn1.tex, cdocsfn2.tex.
%
%<package>\ifdefined\childdocmain\endinput\fi
%<package>\ProvidesFile{childdoc.def}[2018/12/30 v2.0 child document driver]
%<samplemain>\ProvidesFile{cdocsamp.tex}[2018/12/30 v2.0 sample for childdoc]
%<*driver>
%\ProvidesFile{childdoc.drv}[2018/12/30 v2.0 childdoc reference manual file]
\PassOptionsToClass{10pt,a4paper}{article}
\documentclass{ltxdoc}

\usepackage[margin=35mm]{geometry}
\usepackage{hyperref}
\usepackage{hyperxmp}
\usepackage[usenames]{color}

\hypersetup{colorlinks=true}
\hypersetup{pdfstartview=FitH}
\hypersetup{pdfpagemode=UseNone}
\hypersetup{pdfsource={}}
\hypersetup{pdflang={en-UK}}
\hypersetup{pdfcopyright={Copyright 2017-2018 Niklas Beisert.
  This work may be distributed and/or modified under the
  conditions of the LaTeX Project Public License, either version 1.3
  of this license or (at your option) any later version.}}
\hypersetup{pdflicenseurl={http://www.latex-project.org/lppl.txt}}
\hypersetup{pdfcontactaddress={ETH Zurich, ITP, HIT K,
  Wolfgang-Pauli-Strasse 27}}
\hypersetup{pdfcontactpostcode={8093}}
\hypersetup{pdfcontactcity={Zurich}}
\hypersetup{pdfcontactcountry={Switzerland}}
\hypersetup{pdfcontactemail={nbeisert@itp.phys.ethz.ch}}
\hypersetup{pdfcontacturl={http://people.phys.ethz.ch/\xmptilde nbeisert/}}

\newcommand{\secref}[1]{\hyperref[#1]{section \ref*{#1}}}

\parskip1ex
\parindent0pt
\let\olditemize\itemize
\def\itemize{\olditemize\parskip0pt}

\begin{document}

\title{The \textsf{childdoc} Package}
\hypersetup{pdftitle={The childdoc Package}}
\author{Niklas Beisert\\[2ex]
  Institut f\"ur Theoretische Physik\\
  Eidgen\"ossische Technische Hochschule Z\"urich\\
  Wolfgang-Pauli-Strasse 27, 8093 Z\"urich, Switzerland\\[1ex]
  \href{mailto:nbeisert@itp.phys.ethz.ch}
  {\texttt{nbeisert@itp.phys.ethz.ch}}}
\hypersetup{pdfauthor={Niklas Beisert}}
\hypersetup{pdfsubject={Manual for the LaTeX2e Package childdoc}}
\date{30 December 2018, \textsf{v2.0}}
\maketitle

\begin{abstract}\noindent
\textsf{childdoc} is a \LaTeXe{} package
that enables the direct compilation
of document sections included by |\include|
to individual files.
\end{abstract}

\begingroup
\parskip0ex
\tableofcontents
\endgroup

%%%%%%%%%%%%%%%%%%%%%%%%%%%%%%%%%%%%%%%%%%%%%%%%%%%%%%%%%%%%%%%%%%%%%%%%%%%%%%%%
%%%%%%%%%%%%%%%%%%%%%%%%%%%%%%%%%%%%%%%%%%%%%%%%%%%%%%%%%%%%%%%%%%%%%%%%%%%%%%%%
\section{Introduction}

\LaTeX{} provides a mechanism to structure a large document (such as a book)
into a main file and several child files (containing the chapters)
using the |\include| command.
This mechanism is beneficial for documents
which span hundreds of pages in order to
make the source file(s) more manageable.
Moreover, compilation can be restricted to
selected child files by means of the |\includeonly| command.
The latter feature can be used to reduce the compilation time while editing
(this was significantly more useful in the earlier days of \LaTeX{})
or to generate a smaller document which is easier to navigate.
Another application of |\includeonly| is to generate
documents consisting of selected parts of the complete document.

However, there are a few drawbacks of the plain |\include| mechanism:
\begin{itemize}
\item
The child files cannot be compiled on their own,
they can only be compiled via the main file.
A naive editing environment
(such as a text editor with an option
to have the current file processed by \LaTeX)
may require one to switch to the main file before compiling;
attempting to compile the child file produces errors.
\item
The main file must be modified (each time)
to adjust the |\includeonly| command
to the present needs. This easily leaves the main file in a messy state.
\item
The generated document will always carry the filename
of the main document. This is inconvenient if
several child files are to be compiled and
to be kept for distribution.
\end{itemize}

The present package provides a simple interface
to make child files individually compilable by \LaTeX{}.
Compiling a child file then has the same effect as compiling
the main file with an |\includeonly| command
to select the appropriate child.
Moreover the generated document will carry the name of the child
rather than the main file.
This resolves all three above issues.

This feature is meant to make the editing of books,
thesis documents and lecture notes somewhat more convenient.
However, the package can also be used efficiently for
composing a series of documents (such as exercise sheets)
which are typically distributed individually.
It then assists the author in generating the individual documents
(potentially in different versions)
as well as a document containing the collected series.
Another application is in developing style files
or other kinds of included material
where compilation of the style file could redirect
to a sample or test file.

%%%%%%%%%%%%%%%%%%%%%%%%%%%%%%%%%%%%%%%%%%%%%%%%%%%%%%%%%%%%%%%%%%%%%%%%%%%%%%%%
%%%%%%%%%%%%%%%%%%%%%%%%%%%%%%%%%%%%%%%%%%%%%%%%%%%%%%%%%%%%%%%%%%%%%%%%%%%%%%%%
\section{Usage}

First of all, the package \textsf{childdoc} is \emph{not} a standard
\LaTeXe{} |.sty| style file! Therefore it needs to be invoked in
a non-standard way.

%%%%%%%%%%%%%%%%%%%%%%%%%%%%%%%%%%%%%%%%%%%%%%%%%%%%%%%%%%%%%%%%%%%%%%%%%%%%%%%%
\subsection{Included Files}
\label{sec:include}

%%%%%%%%%%%%%%%%%%%%%%%%%%%%%%%%%%%%%%%%
\DescribeMacro{\childdocmain}
To use the package, add the commands
\begin{center}
\begin{tabular}{l}
|\input{childdoc.def}|\\
|\childdocmain{}|\\
\end{tabular}
\end{center}
at the very top of the main \LaTeX{} file,
in particular \emph{before} the |\documentclass| statement!
The argument of |\childdocmain| should be left empty
(but it must be present).

%%%%%%%%%%%%%%%%%%%%%%%%%%%%%%%%%%%%%%%%
\DescribeMacro{\childdocof}
Furthermore, add the commands
\begin{center}
\begin{tabular}{l}
|\input{childdoc.def}|\\
|\childdocof{|\textit{main}|}|\\
\end{tabular}
\end{center}
at the top of every child file \textit{child}
which is included by |\include{|\textit{child}|}|
from within the main file
(or at least for those files to be compiled individually).
The argument \textit{main} must be the filename of the main file.

There are a couple of
considerations in setting up the main and child documents:

%%%%%%%%%%%%%%%%%%%%%%%%%%%%%%%%%%%%%%%%
\paragraph{Restrictions.}

Please note the following restrictions:
\begin{itemize}
\item
|\childdocmain| must be called with one argument \textit{main}
to ensure compatibility with earlier version of the package.
It must either be empty (|\childdocmain{}|)
or precisely match the filename of the main file in which it is specified.
See \secref{sec:detection} for further information.
\item
The filename \textit{main} must be specified without the |.tex| extension.
\item
The filename \textit{main} is case sensitive
(even in case-insensitive file systems)
due to internal string comparison.
\item
The argument \textit{main} should be fully expanded, it cannot be a macro.
\item
Subdirectories and special characters should be avoided in filenames.
\item
The command |\childdocmain{|\textit{main}|}| must be followed by a whitespace.
It should not be followed immediately by another command
or by a comment mark `|%|'.
This is because the \TeX{} parser reads the token immediately following
the argument of |\childdocmain| and puts it
at the beginning of every child section;
however, a white\-space is ignored.
\end{itemize}

%%%%%%%%%%%%%%%%%%%%%%%%%%%%%%%%%%%%%%%%
\paragraph{Content of Main File.}

It is advisable to place all content in the child files included by |\include|.
Any output contained in the main file will appear in all child documents
unless suppressed manually;
it cannot be suppressed automatically by the |\includeonly| directive
and thus should normally be avoided.
A method to include some content in the main file
by means of conditional processing is described in \secref{sec:conditional}.

%%%%%%%%%%%%%%%%%%%%%%%%%%%%%%%%%%%%%%%%
\paragraph{Page Numbering.}

When only a part of the document is compiled,
the appropriate numbering of pages
(as well as other status parameters)
is determined from the |.aux| files.
The latter contain information from previous passes.
However this information needs to propagate through
all intermediate child documents.
Therefore the page numbering in child documents may well
be inconsistent until the complete document is compiled at least once.

A useful (if unconventional) way to always ensure a consistent
page numbering is to restart the numbering in each child document
and denote the pages by `\textit{child}|.|\textit{page}'
where \textit{child} represents the chapter/section number of the child file.
This can be achieved by the command
|\numberwithin{page}{|\textit{child}|}|
of the \textsf{amsmath} package
where \textit{child} can be |chapter| or |section|
depending on the chosen structuring.
Alternatively, one can modify the macro |\thepage| appropriately
and reset the counter |page| at the start of each child file.

%%%%%%%%%%%%%%%%%%%%%%%%%%%%%%%%%%%%%%%%%%%%%%%%%%%%%%%%%%%%%%%%%%%%%%%%%%%%%%%%
\subsection{Conditional Processing}
\label{sec:conditional}

The package provides a mechanism to compile different versions
of a document. To customise the versions further some conditional processing
can come in handy to distinguish which version is being compiled.
The package provides two macros to describe the compilation context:

%%%%%%%%%%%%%%%%%%%%%%%%%%%%%%%%%%%%%%%%
\DescribeMacro{\ifchilddoc}
The conditional |\ifchilddoc| distinguishes between the compilation of
child documents and the main document:
%
\begin{center}
|\ifchilddoc |\textit{child-code}| |[|\||else |\textit{main-code}]| \||fi|
\end{center}

%%%%%%%%%%%%%%%%%%%%%%%%%%%%%%%%%%%%%%%%
\DescribeMacro{\childdocname}
\DescribeMacro{\childdocjob}
The macro |\childdocname| contains the filename (without extension)
of the main or child file being processed.
Note that |\childdocjob| will always contain the name of the main file.

%%%%%%%%%%%%%%%%%%%%%%%%%%%%%%%%%%%%%%%%
\paragraph{Title Page.}

Conditional processing can be used to include a title or banner page
in the main document when proper precautions are taken.
Importantly, the code in the main file should ensure that the page counter
(as well as other status parameters which are stored in the |.aux| files)
takes the same value after the conditional processing.
Otherwise the page numbers may take divergent values
depending on which part is compiled.

For example, a title page could be declared by:
%
\begin{center}
\begin{tabular}{l}
|\ifchilddoc\||else|\\
|\addtocounter{page}{-1}|\\
\textit{code for title page}\\
|\newpage|\\
|\||fi|
\end{tabular}
\end{center}
%
A banner page for the child documents can be generated by:
%
\begin{center}
\begin{tabular}{l}
|\ifchilddoc|\\
|\addtocounter{page}{-1}|\\
\textit{code for banner page}\\
|\newpage|\\
|\||fi|
\end{tabular}
\end{center}
%
Here one could write a message such as:
\begin{center}
|This is the part \childdocname{} of \childdocjob{}.|
\end{center}

%%%%%%%%%%%%%%%%%%%%%%%%%%%%%%%%%%%%%%%%%%%%%%%%%%%%%%%%%%%%%%%%%%%%%%%%%%%%%%%%
\subsection{Flags}
\label{sec:flags}

The package makes it easy to generate different versions
of the main or child documents.
To this end compilation flags can be defined
and assigned different default values.
They will be particularly useful in conjunction
with the forwarding mechanism described in \secref{sec:forward}.

For example, it may be useful to have a flag |\version|
which can be set to |draft| or |final|.
The document source will contain some conditional code
depending on the value of |\version|.
Suppose further, the flag should default to |final| for the main file
and to |draft| for child files
which is a natural assignment for editing the document.
This is achieved by placing the following code
in the preamble of the main document
(below the |\childdocmain| directive):
%
\begin{center}
\begin{tabular}{l}
|\ifchilddoc|\\
|\providecommand{\version}{draft}|\\
|\||else|\\
|\providecommand{\version}{final}|\\
|\||fi|
\end{tabular}
\end{center}
%
The definition by |\providecommand| makes sure
that previous definitions are not overwritten.
Further statements |\providecommand{\version}{...}|
can thus be added before the above code to override it.

For the main file, one might add a line
(between |\childdocmain| and the above block)
%
\begin{center}
|%\ifchilddoc\||else\providecommand{\version}{draft}\||fi|
\end{center}
%
which can be uncommented to produce a draft version.
Likewise one can add a line to the very top of a child file
(above the |\childdocof{|\textit{main}|}| directive)
%
\begin{center}
|%\providecommand{\version}{final}|
\end{center}
%
which can be uncommented to produce the final version of this child document.

%%%%%%%%%%%%%%%%%%%%%%%%%%%%%%%%%%%%%%%%%%%%%%%%%%%%%%%%%%%%%%%%%%%%%%%%%%%%%%%%
\subsection{Forwarding}
\label{sec:forward}

Different versions of the main or child documents
using compilation flags as described in \secref{sec:flags}
can be (permanently) stored in different files
for convenient compilation, viewing and distribution.
To this end, the package defines a command
to pass on compilation to a different file:

%%%%%%%%%%%%%%%%%%%%%%%%%%%%%%%%%%%%%%%%
\DescribeMacro{\childdocforward}
The command |\childdocforward| redirects processing to
another source file:
%
\begin{center}
\begin{tabular}{l}
|\input{childdoc.def}|\\
|\childdocforward[|\textit{main}|]{|\textit{dest}|}|\\
\end{tabular}
\end{center}
%
The argument \textit{dest} is the destination file
(without extension).
It should be the main file or one of the child files.
Note that further \textsf{childdoc} directives
such as |\childdocof| and |\childdocforward|
in the indicated file will be processed in this form.
The optional argument \textit{main}
passes on directly to the main file \textit{main}
while pretending to compile the child \textit{dest}.
This form behaves as if \textit{dest}
issues |\childdocof{|\textit{main}|}| right away,
and no further \textsf{childdoc} directives will be processed.

%%%%%%%%%%%%%%%%%%%%%%%%%%%%%%%%%%%%%%%%
\DescribeMacro{\...prefix}
In the alternative form |\childdocforwardprefix|,
%
\begin{center}
\begin{tabular}{l}
|\input{childdoc.def}|\\
|\childdocforwardprefix[|\textit{main}|]{|\textit{prefix}|}{|\textit{dest}|}|
\end{tabular}
\end{center}
%
the destination file is determined by a pattern
depending on the current file:
To make this work, the current file must be called
`{\textit{prefix}\hspace{0.2em}\textit{suffix}}'
with \textit{prefix} matching precisely the argument.
Processing is then passed on to the file
`{\textit{dest}\hspace{0.2em}\textit{suffix}}'.
Surely, the same effect is achieved by
directly specifying the
argument `{\textit{dest}\hspace{0.2em}\textit{suffix}}'
in the first form.
However, that requires to set up a different file
for each child. With the alternative form of the command
all these files can have exactly the same content
which simplifies setting them up and maintaining them.

For example, the following file |draft.tex|
with a compilation flag |\version| as described in \secref{sec:flags}
compiles the main document as a draft:
%
\begin{center}
\begin{tabular}{l}
|\def\version{draft}|\\
|\input{childdoc.def}|\\
|\childdocforward{|\textit{main}|}|
\end{tabular}
\end{center}
%
Likewise, the following files |final|\textit{nn}|.tex|
compile the final version of the child document
|child|\textit{nn}|.tex|:
%
\begin{center}
\begin{tabular}{l}
|\def\version{final}|\\
|\input{childdoc.def}|\\
|\childdocforwardprefix{final}{child}|
\end{tabular}
\end{center}
%

Note that when several versions of a main file and/or of each child file
are to be generated, it may be convenient to set up a |Makefile| or
shell script to automatise the process.

%%%%%%%%%%%%%%%%%%%%%%%%%%%%%%%%%%%%%%%%%%%%%%%%%%%%%%%%%%%%%%%%%%%%%%%%%%%%%%%%
\subsection{Command Line Processing}
\label{sec:commandline}

The effect of redirection files can also be achieved by invoking
the \LaTeX{} compiler with a more elaborate command line.
Most conveniently this should be done as part
of a shell script or a |Makefile|.

When using \textsf{childdoc} in the main file, the following
command lines effectively perform a redirection
(note that depending on the shell being used,
backslashes may have to be doubled: `|\|' $\to$ `|\\|'):
%
\begin{center}
|... -jobname "|\textit{target}|" |\\|"|[\textit{flags}]%
|\input{childdoc.def}\childdocforward[|\textit{main}|]{|\textit{dest}|}"|
\end{center}
%
Here \textit{target} is the name of the output file,
\textit{main} is the name of the main file
and \textit{dest} is the name of the main or child file to be processed
(all filenames without extensions).
The optional argument \textit{main} can be omitted
if \textit{main} matches \textit{dest}.
Optionally, compilation \textit{flags} can be defined via |\def| commands.
This command line makes the \TeX{} engine believe
it is compiling the file \textit{target}
whose content is specified as the latter parameter.
The provided code then forwards the processing to
\textit{main} or \textit{dest} as described in \secref{sec:forward}.

%%%%%%%%%%%%%%%%%%%%%%%%%%%%%%%%%%%%%%%%%%%%%%%%%%%%%%%%%%%%%%%%%%%%%%%%%%%%%%%%
\subsection{Include by Input}
\label{sec:input}

Including child documents by |\include| has some restrictions by design.
Most notably, the content of a child document always occupies
its own set of pages; pages cannot be shared between child documents.
Usually, this behaviour makes perfect sense
because each child document contain an essential part of the document.
However, in some situations it may be desirable to compose
a document from a collection of parts
without having mandatory page breaks between then.
For this case, the package
provides a mechanism to include parts
by |\input| which can also be processed individually.
However, by construction this mechanism
requires manual handling of the content to be output.

%%%%%%%%%%%%%%%%%%%%%%%%%%%%%%%%%%%%%%%%
\DescribeMacro{\ifchilddocmanual}
The main file should be prepared as usual, see \secref{sec:include}.
However, the document body must make a distinction
between processing of an individual part and of the main document, e.g.:
%
\begin{center}
\begin{tabular}{l}
|\ifchilddocmanual|\\
|\input{\childdocname}|\\
|\||else|\\
\textit{document body with }|\input{|\textit{part}|}|\\
|\||fi|
\end{tabular}
\end{center}
%
The conditional |\ifchilddocmanual| is true whenever
a part to be included by |\input| is being compiled,
and the name of the part is stored in |\childdocname|.

%%%%%%%%%%%%%%%%%%%%%%%%%%%%%%%%%%%%%%%%
\DescribeMacro{\childdocby}
Each part to be included by |\input| should start with:
%
\begin{center}
\begin{tabular}{l}
|\input{childdoc.def}|\\
|\childdocby{|\textit{main}|}|\\
\end{tabular}
\end{center}
%
The directive |\childdocby| is similar to |\childdocof|
described in \secref{sec:include},
but the subsequent selection of content must be done manually.
To that end, both |\ifchilddoc| and |\ifchilddocmanual|
will be true upon processing of a part,
and the name of the part is stored in |\childdocname|.
Note that |\jobname| will be set to the filename of the current part
so that each part receives an individual |.aux| file
that does not interfere with the |.aux| file(s) of the main document.
This behaviour can be altered by the alternative form
|\childdocby[*]{|\textit{main}|}| (with a non-empty optional argument)
which uses the |.aux| file of the main document
by setting |\jobname| to \textit{main}.

%%%%%%%%%%%%%%%%%%%%%%%%%%%%%%%%%%%%%%%%%%%%%%%%%%%%%%%%%%%%%%%%%%%%%%%%%%%%%%%%
\subsection{Driver Development}
\label{sec:driver}

The \textsf{childdoc} mechanism can also be use for the development
of definition files such as \LaTeX{} styles or classes.
This case differs from the above setup with multiple parts
included by |\include| in that no |\includeonly| should be invoked.
This can be achieved by starting the include file
(before |\ProvidesPackage|) with:
%
\begin{center}
\begin{tabular}{l}
|\input{childdoc.def}|\\
|\childdocforward{|\textit{main}|}|\\
\end{tabular}
\end{center}
%
or alternatively with:
%
\begin{center}
\begin{tabular}{l}
|\input{childdoc.def}|\\
|\childdocby{|\textit{main}|}|\\
\end{tabular}
\end{center}
%
Both forms have slightly different effects as described above.
The main file is prepared as usual, see \secref{sec:include}.

%%%%%%%%%%%%%%%%%%%%%%%%%%%%%%%%%%%%%%%%%%%%%%%%%%%%%%%%%%%%%%%%%%%%%%%%%%%%%%%%
\subsection{Legacy Detection}
\label{sec:detection}

The directive |\childdocmain| in the main file can detect
whether the complete document or merely a child is to be compiled
even without using the directive |\childdocof|.
This method is deprecated because it is less robust
and there is no compelling reason to use it;
it is merely provided for backward compatibility
and it may be removed in future versions.

If the detection mechanism is to be used,
it is mandatory to correctly specify
the filename of the main file as the argument of |\childdocmain|:
%
\begin{center}
\begin{tabular}{l}
|\input{childdoc.def}|\\
|\childdocmain{|\textit{main}|}|\\
\end{tabular}
\end{center}
%
If |\jobname| does not match the argument \textit{main} of |\childdocmain|,
it is assumed that |\jobname| points to the child file to be compiled.
When using |\childdocmain| with the main file specified as argument,
it suffices to start a child file
with just |\input{|\textit{main}|}|
without loading of the package and using |\childdocof|.
If instead all processing is done
with the appropriate \textsf{childdoc} directives,
the argument of \textit{main} of |\childdocmain| can be empty.

An alternative version of the command line processing described
in \secref{sec:commandline} using the detection mechanism reads:
%
\begin{center}
|... -jobname "|\textit{target}|" "|[\textit{flags}]%
[|\def\jobname{|\textit{dest}|}|]|\input{|\textit{main}|}"|
\end{center}

%%%%%%%%%%%%%%%%%%%%%%%%%%%%%%%%%%%%%%%%%%%%%%%%%%%%%%%%%%%%%%%%%%%%%%%%%%%%%%%%
\subsection{Manual Code}
\label{sec:manual}

In case one cannot be certain whether the definitions file |childdoc.def|
is installed on the target \TeX{} distribution
and one prefers not to ship it,
it is conceivable to paste a few relevant commands into the sources.

To that end, drop all statements |\input{childdoc.def}|
and perform the replacements as outlined below.
Instead of |\childdocmain{|\textit{main}|}| add the following code
to the top of the main file:
%
\begin{center}
\begin{tabular}{l}
|\||ifdefined\childdocname\endinput\||fi\newif\ifchilddoc|\\
|\edef\childdocname{\scantokens\expandafter{\jobname\noexpand}}|\\
|\def\childdocmain{|\textit{main}|}\||ifx\childdocmain\childdocname\||else|\\
|\childdoctrue\includeonly{\childdocname}\let\jobname\childdocmain\||fi|\\
\end{tabular}
\end{center}
%
Instead of |\childdocof{|\textit{main}|}| just include the main file
at the top of each child file:
%
\begin{center}
|\input{|\textit{main}|}|
\end{center}
%
A simple redirection |\childdocforward{|\textit{dest}|}| is achieved by:
%
\begin{center}
|\def\jobname{|\textit{dest}|}\input{\jobname}|
\end{center}
%
The redirection with prefix
|\childdocforwardprefix[|\textit{prefix}|]{|\textit{dest}|}|
is accomplished by:
%
\begin{center}
\begin{tabular}{l}
|{\edef\jobname{\scantokens\expandafter{\jobname\noexpand}}|\\
|\def\redirectjob |\textit{prefix}|#1~~~{\gdef\jobname{|\textit{dest}|#1}}|\\
|\expandafter\redirectjob\jobname~~~}\input{\jobname}|
\end{tabular}
\end{center}

In an alternative approach,
child documents can be compiled by a specific command line
without additional code or specific definitions:
%
\begin{center}
|... -jobname "|\textit{target}|" "|[\textit{flags}]%
|\includeonly{|\textit{dest}|}\input{|\textit{main}|}"|
\end{center}
%

%%%%%%%%%%%%%%%%%%%%%%%%%%%%%%%%%%%%%%%%%%%%%%%%%%%%%%%%%%%%%%%%%%%%%%%%%%%%%%%%
%%%%%%%%%%%%%%%%%%%%%%%%%%%%%%%%%%%%%%%%%%%%%%%%%%%%%%%%%%%%%%%%%%%%%%%%%%%%%%%%
\section{Information}

%%%%%%%%%%%%%%%%%%%%%%%%%%%%%%%%%%%%%%%%%%%%%%%%%%%%%%%%%%%%%%%%%%%%%%%%%%%%%%%%
\subsection{Copyright}

Copyright \copyright{} 2017--2018 Niklas Beisert

This work may be distributed and/or modified under the
conditions of the \LaTeX{} Project Public License, either version 1.3
of this license or (at your option) any later version.
The latest version of this license is in
  \url{http://www.latex-project.org/lppl.txt}
and version 1.3 or later is part of all distributions of \LaTeX{}
version 2005/12/01 or later.

This work has the LPPL maintenance status `maintained'.

The Current Maintainer of this work is Niklas Beisert.

This work consists of the files |README.txt|, |childdoc.ins| and |childdoc.dtx|
as well as the derived files |childdoc.def|, |cdocsamp.tex|
with |cdocsch1.tex|, |cdocsch2.tex|, |cdocspt3.tex|, |cdocspt4.tex|,
|cdocsdrf.tex|, |cdocsfn1.tex|, |cdocsfn2.tex|
as well as |childdoc.pdf|.

%%%%%%%%%%%%%%%%%%%%%%%%%%%%%%%%%%%%%%%%%%%%%%%%%%%%%%%%%%%%%%%%%%%%%%%%%%%%%%%%
\subsection{Files and Installation}

The package consists of the files:
%
\begin{center}
\begin{tabular}{ll}
    |README.txt|   & readme file \\
    |childdoc.ins| & installation file \\
    |childdoc.dtx| & source file \\
    |childdoc.def| & definition file \\
    |cdocsamp.tex| & sample main file \\
    |cdocsch1.tex| & sample include file \\
    |cdocsch2.tex| & sample include file \\
    |cdocspt3.tex| & sample part file \\
    |cdocspt4.tex| & sample part file \\
    |cdocsdrf.tex| & sample redirection file \\
    |cdocsfn1.tex| & sample redirection file \\
    |cdocsfn2.tex| & sample redirection file \\
    |childdoc.pdf| & manual
\end{tabular}
\end{center}
%
The distribution consists of the files
|README.txt|, |childdoc.ins| and |childdoc.dtx|.
%
\begin{itemize}
\item
Run (pdf)\LaTeX{} on |childdoc.dtx|
to compile the manual |childdoc.pdf| (this file).
\item
Run \LaTeX{} on |childdoc.ins| to create the definitions file |childdoc.def|
and the sample |cdocsamp.tex| with include files
|cdocsch1.tex|, |cdocsch2.tex|, |cdocspt3.tex|, |cdocspt4.tex|,
|cdocsdrf.tex|, |cdocsfn1.tex|, |cdocsfn2.tex|.
Then copy the file |childdoc.def| to an appropriate directory of your \LaTeX{}
distribution, e.g.\ \textit{texmf-root}|/tex/latex/childdoc|.
\end{itemize}

%%%%%%%%%%%%%%%%%%%%%%%%%%%%%%%%%%%%%%%%%%%%%%%%%%%%%%%%%%%%%%%%%%%%%%%%%%%%%%%%
\subsection{Related CTAN Packages}

There are several other packages which offer a similar functionality:
%
\begin{itemize}
\item
The packages
\href{http://ctan.org/pkg/docmute}{\textsf{docmute}},
\href{http://ctan.org/pkg/includex}{\textsf{includex}} and
\href{http://ctan.org/pkg/standalone}{\textsf{standalone}}
provide commands to include only the document body of
a child file thus allowing both files to be compiled individually.
\item
The packages \href{http://ctan.org/pkg/subdocs}{\textsf{subdocs}}
and \href{http://ctan.org/pkg/subfiles}{\textsf{subfiles}}
provide structures in which the main and child documents can be
encapsulated and allowing them to be compiled individually.
The inclusion mechanism is different from the conventional |\include|.
\item
The package \href{http://ctan.org/pkg/combine}{\textsf{combine}}
is an elaborate solution to combine several documents into one.
\end{itemize}
%
See also the CTAN topic \href{http://ctan.org/topic/subdocs}{\textsf{subdocs}}
for further related packages.
The present package differs from the above solutions in that
a document structure constructed with the conventional |\include| mechanism
just needs two extra commands at the top of every file
such that all constituent files can be compiled individually.

%%%%%%%%%%%%%%%%%%%%%%%%%%%%%%%%%%%%%%%%%%%%%%%%%%%%%%%%%%%%%%%%%%%%%%%%%%%%%%%%
%\subsection{Feature Suggestions}
%
%The following is a list of features which may be useful for future
%versions of this package:
%%
%\begin{itemize}
%\item
%\ldots
%\end{itemize}

%%%%%%%%%%%%%%%%%%%%%%%%%%%%%%%%%%%%%%%%%%%%%%%%%%%%%%%%%%%%%%%%%%%%%%%%%%%%%%%%
\subsection{Revision History}

%%%%%%%%%%%%%%%%%%%%%%%%%%%%%%%%%%%%%%%%
\paragraph{v2.0:} 2018/12/30

\begin{itemize}
\item
immediate forward processing
\item
added |\childdocby| mechanism
\item
manual restructured
\end{itemize}

%%%%%%%%%%%%%%%%%%%%%%%%%%%%%%%%%%%%%%%%
\paragraph{v1.6:} 2018/01/17

\begin{itemize}
\item
application for development of include files
\item
corrections to manual
\end{itemize}

%%%%%%%%%%%%%%%%%%%%%%%%%%%%%%%%%%%%%%%%
\paragraph{v1.5:} 2017/05/21

\begin{itemize}
\item
more complete structuring introduced
\item
|\childdocof| introduced
\item
|\childdoc| renamed to |\childdocmain|
\item
|\childredirect| renamed to |\childdocforward| and |\childdocforwardprefix|
and functionality expanded
\end{itemize}

%%%%%%%%%%%%%%%%%%%%%%%%%%%%%%%%%%%%%%%%
\paragraph{v1.0:} 2017/04/27

\begin{itemize}
\item
manual and install package
\item
first version published on CTAN
\end{itemize}

%%%%%%%%%%%%%%%%%%%%%%%%%%%%%%%%%%%%%%%%
\paragraph{v0.6:} 2017/04/26

\begin{itemize}
\item
redirection mechanism added
\end{itemize}

%%%%%%%%%%%%%%%%%%%%%%%%%%%%%%%%%%%%%%%%
\paragraph{v0.5:} 2017/04/26

\begin{itemize}
\item
functionality in definition file
\end{itemize}


%%%%%%%%%%%%%%%%%%%%%%%%%%%%%%%%%%%%%%%%%%%%%%%%%%%%%%%%%%%%%%%%%%%%%%%%%%%%%%%%
%%%%%%%%%%%%%%%%%%%%%%%%%%%%%%%%%%%%%%%%%%%%%%%%%%%%%%%%%%%%%%%%%%%%%%%%%%%%%%%%
%%%%%%%%%%%%%%%%%%%%%%%%%%%%%%%%%%%%%%%%%%%%%%%%%%%%%%%%%%%%%%%%%%%%%%%%%%%%%%%%
\appendix

\settowidth\MacroIndent{\rmfamily\scriptsize 000\ }

 \DocInput{childdoc.dtx}

\end{document}
%</driver>
% \fi
%
% %%%%%%%%%%%%%%%%%%%%%%%%%%%%%%%%%%%%%%%%%%%%%%%%%%%%%%%%%%%%%%%%%%%%%%%%%%%%%%
% %%%%%%%%%%%%%%%%%%%%%%%%%%%%%%%%%%%%%%%%%%%%%%%%%%%%%%%%%%%%%%%%%%%%%%%%%%%%%%
% \section{Sample}
%\iffalse
%<*samplemain>
%\fi
%
% The following presents a sample document
% with two chapters, two parts, a title page,
% a compile flag as well as three forwarding files to set the flag.
% It consists of eight |.tex| files:
% \begin{center}
% \begin{tabular}{ll}
% |cdocsamp.tex|&main file\\
% |cdocsch1.tex|&include file for chapter 1\\
% |cdocsch2.tex|&include file for chapter 2\\
% |cdocspt3.tex|&include file for part 3\\
% |cdocspt4.tex|&include file for part 4\\
% |cdocsdrf.tex|&forwarding file for main file in draft mode\\
% |cdocsfi1.tex|&forwarding file for final version of chapter 1\\
% |cdocsfi2.tex|&forwarding file for final version of chapter 2\\
% \end{tabular}
% \end{center}
% Each of the eight files can be compiled directly by the \LaTeX{} compiler.
%
% %%%%%%%%%%%%%%%%%%%%%%%%%%%%%%%%%%%%%%
% \paragraph{Main File.}
%
% The main file is called |cdocsamp.tex|.
%
% Load the \textsf{childdoc} definitions and
% declare the filename for the main document:
%    \begin{macrocode}
\input{childdoc.def}
\childdocmain{}
%    \end{macrocode}

% Optional override for |\version| flag:
%    \begin{macrocode}
%%\ifchilddoc\else\providecommand{\version}{draft}\fi
%    \end{macrocode}

% Define the default values for the |\version| flag
% (|final| for the main file and |draft| for childs):
%    \begin{macrocode}
\ifchilddoc
\providecommand{\version}{draft}
\else
\providecommand{\version}{final}
\fi
%    \end{macrocode}

% Load the standard document class:
%    \begin{macrocode}
\documentclass[12pt]{article}
%    \end{macrocode}

% Start the document body:
%    \begin{macrocode}
\begin{document}
%    \end{macrocode}

% Declare a title page.
% Print title, part of document being processed and version flag:
%    \begin{macrocode}
\addtocounter{page}{-1}
\begin{center}
{\LARGE\bfseries{}childdoc example\par}
\vspace{1cm}
\ifchilddoc
\ifchilddocmanual part\else chapter\fi:
`\childdocname' of `\childdocjob'\par
\else
main document: `\childdocjob'\par
\fi
version: \version\par
\end{center}
\newpage
%    \end{macrocode}

% Manually include selected file,
% otherwise process as usual:
%    \begin{macrocode}
\ifchilddocmanual
\section*{part `\childdocname'}
\input{\childdocname}
\else
%    \end{macrocode}

% Include the two chapters:
%    \begin{macrocode}
\include{cdocsch1}
\include{cdocsch2}
%    \end{macrocode}

% Include the two parts unless only chapters should be displayed:
%    \begin{macrocode}
\ifchilddoc\else
\section{part three}
\input{cdocspt3}
\section{part four}
\input{cdocspt4}
\fi
%    \end{macrocode}

% Process as usual until here:
%    \begin{macrocode}
\fi
%    \end{macrocode}

% End of document body:
%    \begin{macrocode}
\end{document}
%    \end{macrocode}
%\iffalse
%</samplemain>
%\fi
%
% %%%%%%%%%%%%%%%%%%%%%%%%%%%%%%%%%%%%%%
% \paragraph{Chapter Include Files.}
%
% The include files are called |cdocsch1.tex| and |cdocsch2.tex|.
%
%\iffalse
%<*samplechap1|samplechap2>
%\fi

% Optional override for |\version| flag:
%    \begin{macrocode}
%%\providecommand{\version}{final}
%    \end{macrocode}

% Include the main document:
%    \begin{macrocode}
\input{childdoc.def}
\childdocof{cdocsamp}
%    \end{macrocode}

%\iffalse
%</samplechap1|samplechap2>
%\fi
%
%\iffalse
%<*samplechap1>
%\fi
% Some text for chapter 1:
%    \begin{macrocode}
\section{one}
some text in chapter one
%    \end{macrocode}

%\iffalse
%</samplechap1>
%\fi
% Some text for chapter 2:
%\iffalse
%<*samplechap2>
%\fi
%    \begin{macrocode}
\section{two}
more text in chapter two
%    \end{macrocode}

%\iffalse
%</samplechap2>
%\fi
%
% %%%%%%%%%%%%%%%%%%%%%%%%%%%%%%%%%%%%%%
% \paragraph{Part Include Files.}
%
% The include files are called |cdocspt3.tex| and |cdocspt4.tex|.
%
%\iffalse
%<*samplepart3|samplepart4>
%\fi

% Optional override for |\version| flag:
%    \begin{macrocode}
%%\providecommand{\version}{final}
%    \end{macrocode}

% Include the main document:
%    \begin{macrocode}
\input{childdoc.def}
\childdocby{cdocsamp}
%    \end{macrocode}

%\iffalse
%</samplepart3|samplepart4>
%\fi
%
%\iffalse
%<*samplepart3>
%\fi
% Some text for part 3:
%    \begin{macrocode}
some text in part three
%    \end{macrocode}

%\iffalse
%</samplepart3>
%\fi
% Some text for part 4:
%\iffalse
%<*samplepart4>
%\fi
%    \begin{macrocode}
more text in part four
%    \end{macrocode}

%\iffalse
%</samplepart4>
%\fi
%
% %%%%%%%%%%%%%%%%%%%%%%%%%%%%%%%%%%%%%%
% \paragraph{Forwarding for a Complete Draft.}
%
% The following forwarding file |cdocsdrf.tex|
% compiles the main document in draft mode:
%\iffalse
%<*sampledraft>
%\fi
%    \begin{macrocode}
\def\version{draft}
\input{childdoc.def}
\childdocforward{cdocsamp}
%    \end{macrocode}

%\iffalse
%</sampledraft>
%\fi
%
% %%%%%%%%%%%%%%%%%%%%%%%%%%%%%%%%%%%%%%
% \paragraph{Forwarding for Final Version of the Chapters.}
%
% The following forwarding files |cdocsfn1.tex| and |cdocsfn2.tex|
% (with identical content)
% compile the final versions of the child documents
% |cdocsch1.tex| and |cdocsch2.tex|, respectively:
%\iffalse
%<*samplefinal>
%\fi
%    \begin{macrocode}
\def\version{final}
\input{childdoc.def}
\childdocforwardprefix[cdocsamp]{cdocsfn}{cdocsch}
%    \end{macrocode}

%\iffalse
%</samplefinal>
%\fi
%
% %%%%%%%%%%%%%%%%%%%%%%%%%%%%%%%%%%%%%%
% \paragraph{Command Line Processing.}
%
% The following three command lines generate the output files
% |cdocscld|, |cdocscl1| and |cdocscl2|
% which should be identical to
% |cdocsdrf|, |cdocsch1| and |cdocsfn2|, respectively:
% \begin{center}
% \begin{tabular}{l}
% |latex -jobname cdocscld \|\\
% |  "\def\version{draft}\input{childdoc.def}\childdocforward{cdocsamp}"|\\
% |latex -jobname cdocscl1 \|\\
% |  "\input{childdoc.def}\childdocforward[cdocsamp]{cdocsch1}"|\\
% |latex -jobname cdocscl2 \|\\
% |  "\def\version{final}\input{childdoc.def}\childdocforward{cdocsch2}"|
% \end{tabular}
% \end{center}
% Note that the trailing backslash on each first line
% merely continues the input to the second line
% (for convenient cut ant paste).
% Furthermore, the command |latex| can be replaced by any
% of its alternative versions such as |pdflatex|.
%
% %%%%%%%%%%%%%%%%%%%%%%%%%%%%%%%%%%%%%%%%%%%%%%%%%%%%%%%%%%%%%%%%%%%%%%%%%%%%%%
% %%%%%%%%%%%%%%%%%%%%%%%%%%%%%%%%%%%%%%%%%%%%%%%%%%%%%%%%%%%%%%%%%%%%%%%%%%%%%%
% \section{Implementation}
%\iffalse
%<*package>
%\fi
%
% This section describes the definitions file |childdoc.def|.

% The definitions cannot be loaded using |\usepackage| or |\RequirePackage|
% which has a mechanism to prevent loading a style file more than once.
% When loading the definitions by means of |\input|
% multiple instances have to be prevented manually:
%\iffalse
%This code needs to be before the `\ProvidesFile' directive
%which is defined at the beginning of this file.
%Therefore it is also placed there and commented out here.
%</package>
%<*discard>
%\fi
%    \begin{macrocode}
\ifdefined\childdocmain\endinput\fi
%    \end{macrocode}
%\iffalse
%</discard>
%<*package>
%\fi
%
% \macro{\ifchilddoc}
% \macro{\ifchilddocmanual}
% The conditional |\ifchilddoc| tells whether a
% child (true) or main (false) document is being compiled.
% The conditional |\ifchilddocmanual| tells whether
% the |\includeonly| mechanism is used (false) or
% the selection of child files must be performed manually (true).
% The definitions initialise to false:
%    \begin{macrocode}
\newif\ifchilddoc
\newif\ifchilddocmanual
%    \end{macrocode}

% \macro{\childdocname}
% \macro{\childdocjob}
% The macro |\childdocname| stores the name of the main document
% to be compiled. The macro |\childdocjob| stores the name of
% the document on which the \LaTeX{} compiler was originally invoked.
% The content of |\jobname| cannot be compared
% to filenames specified in the source due to different catcodes.
% The following code rescans |\jobname|, stores the result
% in |\childdocname| and saves a copy in |\childdocjob|:
%    \begin{macrocode}
\edef\childdocname{\scantokens\expandafter{\jobname\noexpand}}
\let\childdocjob\childdocname
%    \end{macrocode}

% \macro{\childdocdisable}
% The macro |\childdocdisable| prevents the main file
% from being processed more than once.
% At this stage, the main document command |\childdocmain|
% is assumed to be called once again where it should do nothing.
% Any subsequent call to it should prevent
% a secondary processing of the main document
% It overwrites the forwarding commands
% |\childdocof| and |\childdocforward|
% with empty macros to prevent further inclusions of the main document:
%    \begin{macrocode}
\newcommand{\childdocdisable}
{
  \renewcommand{\childdocmain}[1]{\renewcommand{\childdocmain}[1]{\endinput}}
  \renewcommand{\childdocof}[1]{}
  \renewcommand{\childdocby}[2][]{}
  \renewcommand{\childdocforward}[2][]{}
  \renewcommand{\childdocdisable}{}
}
%    \end{macrocode}

% \macro{\childdocmain}
% The macro |\childdocmain| is to be called at the top of the main file
% with nothing or the main filename (without extension) as argument.
% First, it breaks loops.
% If the argument is not empty and does not match |\childdocname|
% (which is set by the first inclusion of |childdoc.def|),
% |\ifchilddoc| is set to true, |\includeonly| is applied to the child file
% and |\jobname| is set to the main file
% (for proper handling of |.aux| files):
%    \begin{macrocode}
\newcommand{\childdocmain}[1]
{
  \childdocdisable\childdocmain{}
  \if?#1?\else
    \begingroup
      \def\childdoctmp{#1}
      \ifx\childdoctmp\childdocname
        \def\childdoctmp{}
      \else
        \def\childdoctmp
        {
          \childdoctrue
          \includeonly{\childdocname}
          \def\childdocjob{#1}
          \def\jobname{#1}
        }
      \fi
      \expandafter
    \endgroup
    \childdoctmp
  \fi
}
%    \end{macrocode}

% \macro{\childdocof}
% The command |\childdocof| redirects
% compilation to the main file |#1|.
%    \begin{macrocode}
\newcommand{\childdocof}[1]
{
  \childdocdisable
  \childdoctrue
  \includeonly{\childdocname}
  \def\jobname{#1}
  \def\childdocjob{#1}
  \input{#1}
}
%    \end{macrocode}

% \macro{\childdocby}
% The command |\childdocby| ....
%    \begin{macrocode}
\newcommand{\childdocby}[2][]
{
  \childdocdisable
  \childdoctrue
  \childdocmanualtrue
  \if?#1?\else
    \def\jobname{#2}
  \fi
  \def\childdocjob{#2}
  \input{#2}
  \endinput
}
%    \end{macrocode}

% \macro{\childdocforward}
% The command |\childdocforward| redirects
% compilation to the main file or
% (if the optional argument is given) a child file.
% Parameters are set as if the main file
% or a child file starting with |\childdocof| was compiled.
% Then compilation is handed over to the main file:
%    \begin{macrocode}
\newcommand{\childdocforward}[2][]
{
  \begingroup
    \if?#1?
      \def\childdoctmp
      {
        \def\childdocname{#2}
        \def\childdocjob{#2}
        \def\jobname{#2}
        \input{#2}
        \endinput
      }
    \else
      \def\childdoctmp
      {
        \childdocdisable
        \def\childdocname{#2}
        \childdoctrue
        \includeonly{#2}
        \def\childdocjob{#1}
        \def\jobname{#1}
        \input{#1}
        \endinput
      }
    \fi
    \expandafter
  \endgroup
  \childdoctmp
}
%    \end{macrocode}

% \macro{\childdocforwardprefix}
% The command |\childdocforwardprefix| redirects
% compilation to the main or a child file by means of a pattern.
% The prefix |#1| in the current filename is replaced by |#2|
% and the suffix of the current filename is kept
% (it is assumed that the filename does not contain the substring `|~~~|'
% which is used as a delimiter).
% Compilation is handed over to the new file by |\childdocforward|:
%    \begin{macrocode}
\newcommand{\childdocforwardprefix}[3][]
{
  \begingroup
    \def\childdocextract #2##1~~~{\def\childdoctmp{\childdocforward[#1]{#3##1}}}
    \expandafter\childdocextract\childdocname~~~
    \expandafter
  \endgroup
  \childdoctmp
}
%    \end{macrocode}

% \macro{\childdoc}
% The deprecated macro |\childdoc| is a legacy version of |\childdocmain|:
%    \begin{macrocode}
\newcommand{\childdoc}{\childdocmain}
%    \end{macrocode}

% \macro{\childdocredirect}
% The deprecated macro |\childdocredirect| is a legacy version
% of |\childdocforward| and |\childdocforwardprefix|:
%    \begin{macrocode}
\newcommand{\childdocredirect}[2][]
{
  \begingroup
    \if?#1?
      \def\childdoctmp{\childdocforward{#2}}
    \else
      \def\childdoctmp{\childdocforwardprefix{#1}{#2}}
    \fi
    \expandafter
  \endgroup
  \childdoctmp
}
%    \end{macrocode}

%\iffalse
%</package>
%\fi
%
\endinput

\childdocby{cdocsamp}
%    \end{macrocode}

%\iffalse
%</samplepart3|samplepart4>
%\fi
%
%\iffalse
%<*samplepart3>
%\fi
% Some text for part 3:
%    \begin{macrocode}
some text in part three
%    \end{macrocode}

%\iffalse
%</samplepart3>
%\fi
% Some text for part 4:
%\iffalse
%<*samplepart4>
%\fi
%    \begin{macrocode}
more text in part four
%    \end{macrocode}

%\iffalse
%</samplepart4>
%\fi
%
% %%%%%%%%%%%%%%%%%%%%%%%%%%%%%%%%%%%%%%
% \paragraph{Forwarding for a Complete Draft.}
%
% The following forwarding file |cdocsdrf.tex|
% compiles the main document in draft mode:
%\iffalse
%<*sampledraft>
%\fi
%    \begin{macrocode}
\def\version{draft}
% \iffalse
%
% childdoc.dtx Copyright (C) 2017-2018 Niklas Beisert
%
% This work may be distributed and/or modified under the
% conditions of the LaTeX Project Public License, either version 1.3
% of this license or (at your option) any later version.
% The latest version of this license is in
%   http://www.latex-project.org/lppl.txt
% and version 1.3 or later is part of all distributions of LaTeX
% version 2005/12/01 or later.
%
% This work has the LPPL maintenance status `maintained'.
%
% The Current Maintainer of this work is Niklas Beisert.
%
% This work consists of the files childdoc.dtx and childdoc.ins
% and the derived files childdoc.def and cdocsamp.tex with
% cdocsch1.tex, cdocsch2.tex, cdocsdrf.tex, cdocsfn1.tex, cdocsfn2.tex.
%
%<package>\ifdefined\childdocmain\endinput\fi
%<package>\ProvidesFile{childdoc.def}[2018/12/30 v2.0 child document driver]
%<samplemain>\ProvidesFile{cdocsamp.tex}[2018/12/30 v2.0 sample for childdoc]
%<*driver>
%\ProvidesFile{childdoc.drv}[2018/12/30 v2.0 childdoc reference manual file]
\PassOptionsToClass{10pt,a4paper}{article}
\documentclass{ltxdoc}

\usepackage[margin=35mm]{geometry}
\usepackage{hyperref}
\usepackage{hyperxmp}
\usepackage[usenames]{color}

\hypersetup{colorlinks=true}
\hypersetup{pdfstartview=FitH}
\hypersetup{pdfpagemode=UseNone}
\hypersetup{pdfsource={}}
\hypersetup{pdflang={en-UK}}
\hypersetup{pdfcopyright={Copyright 2017-2018 Niklas Beisert.
  This work may be distributed and/or modified under the
  conditions of the LaTeX Project Public License, either version 1.3
  of this license or (at your option) any later version.}}
\hypersetup{pdflicenseurl={http://www.latex-project.org/lppl.txt}}
\hypersetup{pdfcontactaddress={ETH Zurich, ITP, HIT K,
  Wolfgang-Pauli-Strasse 27}}
\hypersetup{pdfcontactpostcode={8093}}
\hypersetup{pdfcontactcity={Zurich}}
\hypersetup{pdfcontactcountry={Switzerland}}
\hypersetup{pdfcontactemail={nbeisert@itp.phys.ethz.ch}}
\hypersetup{pdfcontacturl={http://people.phys.ethz.ch/\xmptilde nbeisert/}}

\newcommand{\secref}[1]{\hyperref[#1]{section \ref*{#1}}}

\parskip1ex
\parindent0pt
\let\olditemize\itemize
\def\itemize{\olditemize\parskip0pt}

\begin{document}

\title{The \textsf{childdoc} Package}
\hypersetup{pdftitle={The childdoc Package}}
\author{Niklas Beisert\\[2ex]
  Institut f\"ur Theoretische Physik\\
  Eidgen\"ossische Technische Hochschule Z\"urich\\
  Wolfgang-Pauli-Strasse 27, 8093 Z\"urich, Switzerland\\[1ex]
  \href{mailto:nbeisert@itp.phys.ethz.ch}
  {\texttt{nbeisert@itp.phys.ethz.ch}}}
\hypersetup{pdfauthor={Niklas Beisert}}
\hypersetup{pdfsubject={Manual for the LaTeX2e Package childdoc}}
\date{30 December 2018, \textsf{v2.0}}
\maketitle

\begin{abstract}\noindent
\textsf{childdoc} is a \LaTeXe{} package
that enables the direct compilation
of document sections included by |\include|
to individual files.
\end{abstract}

\begingroup
\parskip0ex
\tableofcontents
\endgroup

%%%%%%%%%%%%%%%%%%%%%%%%%%%%%%%%%%%%%%%%%%%%%%%%%%%%%%%%%%%%%%%%%%%%%%%%%%%%%%%%
%%%%%%%%%%%%%%%%%%%%%%%%%%%%%%%%%%%%%%%%%%%%%%%%%%%%%%%%%%%%%%%%%%%%%%%%%%%%%%%%
\section{Introduction}

\LaTeX{} provides a mechanism to structure a large document (such as a book)
into a main file and several child files (containing the chapters)
using the |\include| command.
This mechanism is beneficial for documents
which span hundreds of pages in order to
make the source file(s) more manageable.
Moreover, compilation can be restricted to
selected child files by means of the |\includeonly| command.
The latter feature can be used to reduce the compilation time while editing
(this was significantly more useful in the earlier days of \LaTeX{})
or to generate a smaller document which is easier to navigate.
Another application of |\includeonly| is to generate
documents consisting of selected parts of the complete document.

However, there are a few drawbacks of the plain |\include| mechanism:
\begin{itemize}
\item
The child files cannot be compiled on their own,
they can only be compiled via the main file.
A naive editing environment
(such as a text editor with an option
to have the current file processed by \LaTeX)
may require one to switch to the main file before compiling;
attempting to compile the child file produces errors.
\item
The main file must be modified (each time)
to adjust the |\includeonly| command
to the present needs. This easily leaves the main file in a messy state.
\item
The generated document will always carry the filename
of the main document. This is inconvenient if
several child files are to be compiled and
to be kept for distribution.
\end{itemize}

The present package provides a simple interface
to make child files individually compilable by \LaTeX{}.
Compiling a child file then has the same effect as compiling
the main file with an |\includeonly| command
to select the appropriate child.
Moreover the generated document will carry the name of the child
rather than the main file.
This resolves all three above issues.

This feature is meant to make the editing of books,
thesis documents and lecture notes somewhat more convenient.
However, the package can also be used efficiently for
composing a series of documents (such as exercise sheets)
which are typically distributed individually.
It then assists the author in generating the individual documents
(potentially in different versions)
as well as a document containing the collected series.
Another application is in developing style files
or other kinds of included material
where compilation of the style file could redirect
to a sample or test file.

%%%%%%%%%%%%%%%%%%%%%%%%%%%%%%%%%%%%%%%%%%%%%%%%%%%%%%%%%%%%%%%%%%%%%%%%%%%%%%%%
%%%%%%%%%%%%%%%%%%%%%%%%%%%%%%%%%%%%%%%%%%%%%%%%%%%%%%%%%%%%%%%%%%%%%%%%%%%%%%%%
\section{Usage}

First of all, the package \textsf{childdoc} is \emph{not} a standard
\LaTeXe{} |.sty| style file! Therefore it needs to be invoked in
a non-standard way.

%%%%%%%%%%%%%%%%%%%%%%%%%%%%%%%%%%%%%%%%%%%%%%%%%%%%%%%%%%%%%%%%%%%%%%%%%%%%%%%%
\subsection{Included Files}
\label{sec:include}

%%%%%%%%%%%%%%%%%%%%%%%%%%%%%%%%%%%%%%%%
\DescribeMacro{\childdocmain}
To use the package, add the commands
\begin{center}
\begin{tabular}{l}
|\input{childdoc.def}|\\
|\childdocmain{}|\\
\end{tabular}
\end{center}
at the very top of the main \LaTeX{} file,
in particular \emph{before} the |\documentclass| statement!
The argument of |\childdocmain| should be left empty
(but it must be present).

%%%%%%%%%%%%%%%%%%%%%%%%%%%%%%%%%%%%%%%%
\DescribeMacro{\childdocof}
Furthermore, add the commands
\begin{center}
\begin{tabular}{l}
|\input{childdoc.def}|\\
|\childdocof{|\textit{main}|}|\\
\end{tabular}
\end{center}
at the top of every child file \textit{child}
which is included by |\include{|\textit{child}|}|
from within the main file
(or at least for those files to be compiled individually).
The argument \textit{main} must be the filename of the main file.

There are a couple of
considerations in setting up the main and child documents:

%%%%%%%%%%%%%%%%%%%%%%%%%%%%%%%%%%%%%%%%
\paragraph{Restrictions.}

Please note the following restrictions:
\begin{itemize}
\item
|\childdocmain| must be called with one argument \textit{main}
to ensure compatibility with earlier version of the package.
It must either be empty (|\childdocmain{}|)
or precisely match the filename of the main file in which it is specified.
See \secref{sec:detection} for further information.
\item
The filename \textit{main} must be specified without the |.tex| extension.
\item
The filename \textit{main} is case sensitive
(even in case-insensitive file systems)
due to internal string comparison.
\item
The argument \textit{main} should be fully expanded, it cannot be a macro.
\item
Subdirectories and special characters should be avoided in filenames.
\item
The command |\childdocmain{|\textit{main}|}| must be followed by a whitespace.
It should not be followed immediately by another command
or by a comment mark `|%|'.
This is because the \TeX{} parser reads the token immediately following
the argument of |\childdocmain| and puts it
at the beginning of every child section;
however, a white\-space is ignored.
\end{itemize}

%%%%%%%%%%%%%%%%%%%%%%%%%%%%%%%%%%%%%%%%
\paragraph{Content of Main File.}

It is advisable to place all content in the child files included by |\include|.
Any output contained in the main file will appear in all child documents
unless suppressed manually;
it cannot be suppressed automatically by the |\includeonly| directive
and thus should normally be avoided.
A method to include some content in the main file
by means of conditional processing is described in \secref{sec:conditional}.

%%%%%%%%%%%%%%%%%%%%%%%%%%%%%%%%%%%%%%%%
\paragraph{Page Numbering.}

When only a part of the document is compiled,
the appropriate numbering of pages
(as well as other status parameters)
is determined from the |.aux| files.
The latter contain information from previous passes.
However this information needs to propagate through
all intermediate child documents.
Therefore the page numbering in child documents may well
be inconsistent until the complete document is compiled at least once.

A useful (if unconventional) way to always ensure a consistent
page numbering is to restart the numbering in each child document
and denote the pages by `\textit{child}|.|\textit{page}'
where \textit{child} represents the chapter/section number of the child file.
This can be achieved by the command
|\numberwithin{page}{|\textit{child}|}|
of the \textsf{amsmath} package
where \textit{child} can be |chapter| or |section|
depending on the chosen structuring.
Alternatively, one can modify the macro |\thepage| appropriately
and reset the counter |page| at the start of each child file.

%%%%%%%%%%%%%%%%%%%%%%%%%%%%%%%%%%%%%%%%%%%%%%%%%%%%%%%%%%%%%%%%%%%%%%%%%%%%%%%%
\subsection{Conditional Processing}
\label{sec:conditional}

The package provides a mechanism to compile different versions
of a document. To customise the versions further some conditional processing
can come in handy to distinguish which version is being compiled.
The package provides two macros to describe the compilation context:

%%%%%%%%%%%%%%%%%%%%%%%%%%%%%%%%%%%%%%%%
\DescribeMacro{\ifchilddoc}
The conditional |\ifchilddoc| distinguishes between the compilation of
child documents and the main document:
%
\begin{center}
|\ifchilddoc |\textit{child-code}| |[|\||else |\textit{main-code}]| \||fi|
\end{center}

%%%%%%%%%%%%%%%%%%%%%%%%%%%%%%%%%%%%%%%%
\DescribeMacro{\childdocname}
\DescribeMacro{\childdocjob}
The macro |\childdocname| contains the filename (without extension)
of the main or child file being processed.
Note that |\childdocjob| will always contain the name of the main file.

%%%%%%%%%%%%%%%%%%%%%%%%%%%%%%%%%%%%%%%%
\paragraph{Title Page.}

Conditional processing can be used to include a title or banner page
in the main document when proper precautions are taken.
Importantly, the code in the main file should ensure that the page counter
(as well as other status parameters which are stored in the |.aux| files)
takes the same value after the conditional processing.
Otherwise the page numbers may take divergent values
depending on which part is compiled.

For example, a title page could be declared by:
%
\begin{center}
\begin{tabular}{l}
|\ifchilddoc\||else|\\
|\addtocounter{page}{-1}|\\
\textit{code for title page}\\
|\newpage|\\
|\||fi|
\end{tabular}
\end{center}
%
A banner page for the child documents can be generated by:
%
\begin{center}
\begin{tabular}{l}
|\ifchilddoc|\\
|\addtocounter{page}{-1}|\\
\textit{code for banner page}\\
|\newpage|\\
|\||fi|
\end{tabular}
\end{center}
%
Here one could write a message such as:
\begin{center}
|This is the part \childdocname{} of \childdocjob{}.|
\end{center}

%%%%%%%%%%%%%%%%%%%%%%%%%%%%%%%%%%%%%%%%%%%%%%%%%%%%%%%%%%%%%%%%%%%%%%%%%%%%%%%%
\subsection{Flags}
\label{sec:flags}

The package makes it easy to generate different versions
of the main or child documents.
To this end compilation flags can be defined
and assigned different default values.
They will be particularly useful in conjunction
with the forwarding mechanism described in \secref{sec:forward}.

For example, it may be useful to have a flag |\version|
which can be set to |draft| or |final|.
The document source will contain some conditional code
depending on the value of |\version|.
Suppose further, the flag should default to |final| for the main file
and to |draft| for child files
which is a natural assignment for editing the document.
This is achieved by placing the following code
in the preamble of the main document
(below the |\childdocmain| directive):
%
\begin{center}
\begin{tabular}{l}
|\ifchilddoc|\\
|\providecommand{\version}{draft}|\\
|\||else|\\
|\providecommand{\version}{final}|\\
|\||fi|
\end{tabular}
\end{center}
%
The definition by |\providecommand| makes sure
that previous definitions are not overwritten.
Further statements |\providecommand{\version}{...}|
can thus be added before the above code to override it.

For the main file, one might add a line
(between |\childdocmain| and the above block)
%
\begin{center}
|%\ifchilddoc\||else\providecommand{\version}{draft}\||fi|
\end{center}
%
which can be uncommented to produce a draft version.
Likewise one can add a line to the very top of a child file
(above the |\childdocof{|\textit{main}|}| directive)
%
\begin{center}
|%\providecommand{\version}{final}|
\end{center}
%
which can be uncommented to produce the final version of this child document.

%%%%%%%%%%%%%%%%%%%%%%%%%%%%%%%%%%%%%%%%%%%%%%%%%%%%%%%%%%%%%%%%%%%%%%%%%%%%%%%%
\subsection{Forwarding}
\label{sec:forward}

Different versions of the main or child documents
using compilation flags as described in \secref{sec:flags}
can be (permanently) stored in different files
for convenient compilation, viewing and distribution.
To this end, the package defines a command
to pass on compilation to a different file:

%%%%%%%%%%%%%%%%%%%%%%%%%%%%%%%%%%%%%%%%
\DescribeMacro{\childdocforward}
The command |\childdocforward| redirects processing to
another source file:
%
\begin{center}
\begin{tabular}{l}
|\input{childdoc.def}|\\
|\childdocforward[|\textit{main}|]{|\textit{dest}|}|\\
\end{tabular}
\end{center}
%
The argument \textit{dest} is the destination file
(without extension).
It should be the main file or one of the child files.
Note that further \textsf{childdoc} directives
such as |\childdocof| and |\childdocforward|
in the indicated file will be processed in this form.
The optional argument \textit{main}
passes on directly to the main file \textit{main}
while pretending to compile the child \textit{dest}.
This form behaves as if \textit{dest}
issues |\childdocof{|\textit{main}|}| right away,
and no further \textsf{childdoc} directives will be processed.

%%%%%%%%%%%%%%%%%%%%%%%%%%%%%%%%%%%%%%%%
\DescribeMacro{\...prefix}
In the alternative form |\childdocforwardprefix|,
%
\begin{center}
\begin{tabular}{l}
|\input{childdoc.def}|\\
|\childdocforwardprefix[|\textit{main}|]{|\textit{prefix}|}{|\textit{dest}|}|
\end{tabular}
\end{center}
%
the destination file is determined by a pattern
depending on the current file:
To make this work, the current file must be called
`{\textit{prefix}\hspace{0.2em}\textit{suffix}}'
with \textit{prefix} matching precisely the argument.
Processing is then passed on to the file
`{\textit{dest}\hspace{0.2em}\textit{suffix}}'.
Surely, the same effect is achieved by
directly specifying the
argument `{\textit{dest}\hspace{0.2em}\textit{suffix}}'
in the first form.
However, that requires to set up a different file
for each child. With the alternative form of the command
all these files can have exactly the same content
which simplifies setting them up and maintaining them.

For example, the following file |draft.tex|
with a compilation flag |\version| as described in \secref{sec:flags}
compiles the main document as a draft:
%
\begin{center}
\begin{tabular}{l}
|\def\version{draft}|\\
|\input{childdoc.def}|\\
|\childdocforward{|\textit{main}|}|
\end{tabular}
\end{center}
%
Likewise, the following files |final|\textit{nn}|.tex|
compile the final version of the child document
|child|\textit{nn}|.tex|:
%
\begin{center}
\begin{tabular}{l}
|\def\version{final}|\\
|\input{childdoc.def}|\\
|\childdocforwardprefix{final}{child}|
\end{tabular}
\end{center}
%

Note that when several versions of a main file and/or of each child file
are to be generated, it may be convenient to set up a |Makefile| or
shell script to automatise the process.

%%%%%%%%%%%%%%%%%%%%%%%%%%%%%%%%%%%%%%%%%%%%%%%%%%%%%%%%%%%%%%%%%%%%%%%%%%%%%%%%
\subsection{Command Line Processing}
\label{sec:commandline}

The effect of redirection files can also be achieved by invoking
the \LaTeX{} compiler with a more elaborate command line.
Most conveniently this should be done as part
of a shell script or a |Makefile|.

When using \textsf{childdoc} in the main file, the following
command lines effectively perform a redirection
(note that depending on the shell being used,
backslashes may have to be doubled: `|\|' $\to$ `|\\|'):
%
\begin{center}
|... -jobname "|\textit{target}|" |\\|"|[\textit{flags}]%
|\input{childdoc.def}\childdocforward[|\textit{main}|]{|\textit{dest}|}"|
\end{center}
%
Here \textit{target} is the name of the output file,
\textit{main} is the name of the main file
and \textit{dest} is the name of the main or child file to be processed
(all filenames without extensions).
The optional argument \textit{main} can be omitted
if \textit{main} matches \textit{dest}.
Optionally, compilation \textit{flags} can be defined via |\def| commands.
This command line makes the \TeX{} engine believe
it is compiling the file \textit{target}
whose content is specified as the latter parameter.
The provided code then forwards the processing to
\textit{main} or \textit{dest} as described in \secref{sec:forward}.

%%%%%%%%%%%%%%%%%%%%%%%%%%%%%%%%%%%%%%%%%%%%%%%%%%%%%%%%%%%%%%%%%%%%%%%%%%%%%%%%
\subsection{Include by Input}
\label{sec:input}

Including child documents by |\include| has some restrictions by design.
Most notably, the content of a child document always occupies
its own set of pages; pages cannot be shared between child documents.
Usually, this behaviour makes perfect sense
because each child document contain an essential part of the document.
However, in some situations it may be desirable to compose
a document from a collection of parts
without having mandatory page breaks between then.
For this case, the package
provides a mechanism to include parts
by |\input| which can also be processed individually.
However, by construction this mechanism
requires manual handling of the content to be output.

%%%%%%%%%%%%%%%%%%%%%%%%%%%%%%%%%%%%%%%%
\DescribeMacro{\ifchilddocmanual}
The main file should be prepared as usual, see \secref{sec:include}.
However, the document body must make a distinction
between processing of an individual part and of the main document, e.g.:
%
\begin{center}
\begin{tabular}{l}
|\ifchilddocmanual|\\
|\input{\childdocname}|\\
|\||else|\\
\textit{document body with }|\input{|\textit{part}|}|\\
|\||fi|
\end{tabular}
\end{center}
%
The conditional |\ifchilddocmanual| is true whenever
a part to be included by |\input| is being compiled,
and the name of the part is stored in |\childdocname|.

%%%%%%%%%%%%%%%%%%%%%%%%%%%%%%%%%%%%%%%%
\DescribeMacro{\childdocby}
Each part to be included by |\input| should start with:
%
\begin{center}
\begin{tabular}{l}
|\input{childdoc.def}|\\
|\childdocby{|\textit{main}|}|\\
\end{tabular}
\end{center}
%
The directive |\childdocby| is similar to |\childdocof|
described in \secref{sec:include},
but the subsequent selection of content must be done manually.
To that end, both |\ifchilddoc| and |\ifchilddocmanual|
will be true upon processing of a part,
and the name of the part is stored in |\childdocname|.
Note that |\jobname| will be set to the filename of the current part
so that each part receives an individual |.aux| file
that does not interfere with the |.aux| file(s) of the main document.
This behaviour can be altered by the alternative form
|\childdocby[*]{|\textit{main}|}| (with a non-empty optional argument)
which uses the |.aux| file of the main document
by setting |\jobname| to \textit{main}.

%%%%%%%%%%%%%%%%%%%%%%%%%%%%%%%%%%%%%%%%%%%%%%%%%%%%%%%%%%%%%%%%%%%%%%%%%%%%%%%%
\subsection{Driver Development}
\label{sec:driver}

The \textsf{childdoc} mechanism can also be use for the development
of definition files such as \LaTeX{} styles or classes.
This case differs from the above setup with multiple parts
included by |\include| in that no |\includeonly| should be invoked.
This can be achieved by starting the include file
(before |\ProvidesPackage|) with:
%
\begin{center}
\begin{tabular}{l}
|\input{childdoc.def}|\\
|\childdocforward{|\textit{main}|}|\\
\end{tabular}
\end{center}
%
or alternatively with:
%
\begin{center}
\begin{tabular}{l}
|\input{childdoc.def}|\\
|\childdocby{|\textit{main}|}|\\
\end{tabular}
\end{center}
%
Both forms have slightly different effects as described above.
The main file is prepared as usual, see \secref{sec:include}.

%%%%%%%%%%%%%%%%%%%%%%%%%%%%%%%%%%%%%%%%%%%%%%%%%%%%%%%%%%%%%%%%%%%%%%%%%%%%%%%%
\subsection{Legacy Detection}
\label{sec:detection}

The directive |\childdocmain| in the main file can detect
whether the complete document or merely a child is to be compiled
even without using the directive |\childdocof|.
This method is deprecated because it is less robust
and there is no compelling reason to use it;
it is merely provided for backward compatibility
and it may be removed in future versions.

If the detection mechanism is to be used,
it is mandatory to correctly specify
the filename of the main file as the argument of |\childdocmain|:
%
\begin{center}
\begin{tabular}{l}
|\input{childdoc.def}|\\
|\childdocmain{|\textit{main}|}|\\
\end{tabular}
\end{center}
%
If |\jobname| does not match the argument \textit{main} of |\childdocmain|,
it is assumed that |\jobname| points to the child file to be compiled.
When using |\childdocmain| with the main file specified as argument,
it suffices to start a child file
with just |\input{|\textit{main}|}|
without loading of the package and using |\childdocof|.
If instead all processing is done
with the appropriate \textsf{childdoc} directives,
the argument of \textit{main} of |\childdocmain| can be empty.

An alternative version of the command line processing described
in \secref{sec:commandline} using the detection mechanism reads:
%
\begin{center}
|... -jobname "|\textit{target}|" "|[\textit{flags}]%
[|\def\jobname{|\textit{dest}|}|]|\input{|\textit{main}|}"|
\end{center}

%%%%%%%%%%%%%%%%%%%%%%%%%%%%%%%%%%%%%%%%%%%%%%%%%%%%%%%%%%%%%%%%%%%%%%%%%%%%%%%%
\subsection{Manual Code}
\label{sec:manual}

In case one cannot be certain whether the definitions file |childdoc.def|
is installed on the target \TeX{} distribution
and one prefers not to ship it,
it is conceivable to paste a few relevant commands into the sources.

To that end, drop all statements |\input{childdoc.def}|
and perform the replacements as outlined below.
Instead of |\childdocmain{|\textit{main}|}| add the following code
to the top of the main file:
%
\begin{center}
\begin{tabular}{l}
|\||ifdefined\childdocname\endinput\||fi\newif\ifchilddoc|\\
|\edef\childdocname{\scantokens\expandafter{\jobname\noexpand}}|\\
|\def\childdocmain{|\textit{main}|}\||ifx\childdocmain\childdocname\||else|\\
|\childdoctrue\includeonly{\childdocname}\let\jobname\childdocmain\||fi|\\
\end{tabular}
\end{center}
%
Instead of |\childdocof{|\textit{main}|}| just include the main file
at the top of each child file:
%
\begin{center}
|\input{|\textit{main}|}|
\end{center}
%
A simple redirection |\childdocforward{|\textit{dest}|}| is achieved by:
%
\begin{center}
|\def\jobname{|\textit{dest}|}\input{\jobname}|
\end{center}
%
The redirection with prefix
|\childdocforwardprefix[|\textit{prefix}|]{|\textit{dest}|}|
is accomplished by:
%
\begin{center}
\begin{tabular}{l}
|{\edef\jobname{\scantokens\expandafter{\jobname\noexpand}}|\\
|\def\redirectjob |\textit{prefix}|#1~~~{\gdef\jobname{|\textit{dest}|#1}}|\\
|\expandafter\redirectjob\jobname~~~}\input{\jobname}|
\end{tabular}
\end{center}

In an alternative approach,
child documents can be compiled by a specific command line
without additional code or specific definitions:
%
\begin{center}
|... -jobname "|\textit{target}|" "|[\textit{flags}]%
|\includeonly{|\textit{dest}|}\input{|\textit{main}|}"|
\end{center}
%

%%%%%%%%%%%%%%%%%%%%%%%%%%%%%%%%%%%%%%%%%%%%%%%%%%%%%%%%%%%%%%%%%%%%%%%%%%%%%%%%
%%%%%%%%%%%%%%%%%%%%%%%%%%%%%%%%%%%%%%%%%%%%%%%%%%%%%%%%%%%%%%%%%%%%%%%%%%%%%%%%
\section{Information}

%%%%%%%%%%%%%%%%%%%%%%%%%%%%%%%%%%%%%%%%%%%%%%%%%%%%%%%%%%%%%%%%%%%%%%%%%%%%%%%%
\subsection{Copyright}

Copyright \copyright{} 2017--2018 Niklas Beisert

This work may be distributed and/or modified under the
conditions of the \LaTeX{} Project Public License, either version 1.3
of this license or (at your option) any later version.
The latest version of this license is in
  \url{http://www.latex-project.org/lppl.txt}
and version 1.3 or later is part of all distributions of \LaTeX{}
version 2005/12/01 or later.

This work has the LPPL maintenance status `maintained'.

The Current Maintainer of this work is Niklas Beisert.

This work consists of the files |README.txt|, |childdoc.ins| and |childdoc.dtx|
as well as the derived files |childdoc.def|, |cdocsamp.tex|
with |cdocsch1.tex|, |cdocsch2.tex|, |cdocspt3.tex|, |cdocspt4.tex|,
|cdocsdrf.tex|, |cdocsfn1.tex|, |cdocsfn2.tex|
as well as |childdoc.pdf|.

%%%%%%%%%%%%%%%%%%%%%%%%%%%%%%%%%%%%%%%%%%%%%%%%%%%%%%%%%%%%%%%%%%%%%%%%%%%%%%%%
\subsection{Files and Installation}

The package consists of the files:
%
\begin{center}
\begin{tabular}{ll}
    |README.txt|   & readme file \\
    |childdoc.ins| & installation file \\
    |childdoc.dtx| & source file \\
    |childdoc.def| & definition file \\
    |cdocsamp.tex| & sample main file \\
    |cdocsch1.tex| & sample include file \\
    |cdocsch2.tex| & sample include file \\
    |cdocspt3.tex| & sample part file \\
    |cdocspt4.tex| & sample part file \\
    |cdocsdrf.tex| & sample redirection file \\
    |cdocsfn1.tex| & sample redirection file \\
    |cdocsfn2.tex| & sample redirection file \\
    |childdoc.pdf| & manual
\end{tabular}
\end{center}
%
The distribution consists of the files
|README.txt|, |childdoc.ins| and |childdoc.dtx|.
%
\begin{itemize}
\item
Run (pdf)\LaTeX{} on |childdoc.dtx|
to compile the manual |childdoc.pdf| (this file).
\item
Run \LaTeX{} on |childdoc.ins| to create the definitions file |childdoc.def|
and the sample |cdocsamp.tex| with include files
|cdocsch1.tex|, |cdocsch2.tex|, |cdocspt3.tex|, |cdocspt4.tex|,
|cdocsdrf.tex|, |cdocsfn1.tex|, |cdocsfn2.tex|.
Then copy the file |childdoc.def| to an appropriate directory of your \LaTeX{}
distribution, e.g.\ \textit{texmf-root}|/tex/latex/childdoc|.
\end{itemize}

%%%%%%%%%%%%%%%%%%%%%%%%%%%%%%%%%%%%%%%%%%%%%%%%%%%%%%%%%%%%%%%%%%%%%%%%%%%%%%%%
\subsection{Related CTAN Packages}

There are several other packages which offer a similar functionality:
%
\begin{itemize}
\item
The packages
\href{http://ctan.org/pkg/docmute}{\textsf{docmute}},
\href{http://ctan.org/pkg/includex}{\textsf{includex}} and
\href{http://ctan.org/pkg/standalone}{\textsf{standalone}}
provide commands to include only the document body of
a child file thus allowing both files to be compiled individually.
\item
The packages \href{http://ctan.org/pkg/subdocs}{\textsf{subdocs}}
and \href{http://ctan.org/pkg/subfiles}{\textsf{subfiles}}
provide structures in which the main and child documents can be
encapsulated and allowing them to be compiled individually.
The inclusion mechanism is different from the conventional |\include|.
\item
The package \href{http://ctan.org/pkg/combine}{\textsf{combine}}
is an elaborate solution to combine several documents into one.
\end{itemize}
%
See also the CTAN topic \href{http://ctan.org/topic/subdocs}{\textsf{subdocs}}
for further related packages.
The present package differs from the above solutions in that
a document structure constructed with the conventional |\include| mechanism
just needs two extra commands at the top of every file
such that all constituent files can be compiled individually.

%%%%%%%%%%%%%%%%%%%%%%%%%%%%%%%%%%%%%%%%%%%%%%%%%%%%%%%%%%%%%%%%%%%%%%%%%%%%%%%%
%\subsection{Feature Suggestions}
%
%The following is a list of features which may be useful for future
%versions of this package:
%%
%\begin{itemize}
%\item
%\ldots
%\end{itemize}

%%%%%%%%%%%%%%%%%%%%%%%%%%%%%%%%%%%%%%%%%%%%%%%%%%%%%%%%%%%%%%%%%%%%%%%%%%%%%%%%
\subsection{Revision History}

%%%%%%%%%%%%%%%%%%%%%%%%%%%%%%%%%%%%%%%%
\paragraph{v2.0:} 2018/12/30

\begin{itemize}
\item
immediate forward processing
\item
added |\childdocby| mechanism
\item
manual restructured
\end{itemize}

%%%%%%%%%%%%%%%%%%%%%%%%%%%%%%%%%%%%%%%%
\paragraph{v1.6:} 2018/01/17

\begin{itemize}
\item
application for development of include files
\item
corrections to manual
\end{itemize}

%%%%%%%%%%%%%%%%%%%%%%%%%%%%%%%%%%%%%%%%
\paragraph{v1.5:} 2017/05/21

\begin{itemize}
\item
more complete structuring introduced
\item
|\childdocof| introduced
\item
|\childdoc| renamed to |\childdocmain|
\item
|\childredirect| renamed to |\childdocforward| and |\childdocforwardprefix|
and functionality expanded
\end{itemize}

%%%%%%%%%%%%%%%%%%%%%%%%%%%%%%%%%%%%%%%%
\paragraph{v1.0:} 2017/04/27

\begin{itemize}
\item
manual and install package
\item
first version published on CTAN
\end{itemize}

%%%%%%%%%%%%%%%%%%%%%%%%%%%%%%%%%%%%%%%%
\paragraph{v0.6:} 2017/04/26

\begin{itemize}
\item
redirection mechanism added
\end{itemize}

%%%%%%%%%%%%%%%%%%%%%%%%%%%%%%%%%%%%%%%%
\paragraph{v0.5:} 2017/04/26

\begin{itemize}
\item
functionality in definition file
\end{itemize}


%%%%%%%%%%%%%%%%%%%%%%%%%%%%%%%%%%%%%%%%%%%%%%%%%%%%%%%%%%%%%%%%%%%%%%%%%%%%%%%%
%%%%%%%%%%%%%%%%%%%%%%%%%%%%%%%%%%%%%%%%%%%%%%%%%%%%%%%%%%%%%%%%%%%%%%%%%%%%%%%%
%%%%%%%%%%%%%%%%%%%%%%%%%%%%%%%%%%%%%%%%%%%%%%%%%%%%%%%%%%%%%%%%%%%%%%%%%%%%%%%%
\appendix

\settowidth\MacroIndent{\rmfamily\scriptsize 000\ }

 \DocInput{childdoc.dtx}

\end{document}
%</driver>
% \fi
%
% %%%%%%%%%%%%%%%%%%%%%%%%%%%%%%%%%%%%%%%%%%%%%%%%%%%%%%%%%%%%%%%%%%%%%%%%%%%%%%
% %%%%%%%%%%%%%%%%%%%%%%%%%%%%%%%%%%%%%%%%%%%%%%%%%%%%%%%%%%%%%%%%%%%%%%%%%%%%%%
% \section{Sample}
%\iffalse
%<*samplemain>
%\fi
%
% The following presents a sample document
% with two chapters, two parts, a title page,
% a compile flag as well as three forwarding files to set the flag.
% It consists of eight |.tex| files:
% \begin{center}
% \begin{tabular}{ll}
% |cdocsamp.tex|&main file\\
% |cdocsch1.tex|&include file for chapter 1\\
% |cdocsch2.tex|&include file for chapter 2\\
% |cdocspt3.tex|&include file for part 3\\
% |cdocspt4.tex|&include file for part 4\\
% |cdocsdrf.tex|&forwarding file for main file in draft mode\\
% |cdocsfi1.tex|&forwarding file for final version of chapter 1\\
% |cdocsfi2.tex|&forwarding file for final version of chapter 2\\
% \end{tabular}
% \end{center}
% Each of the eight files can be compiled directly by the \LaTeX{} compiler.
%
% %%%%%%%%%%%%%%%%%%%%%%%%%%%%%%%%%%%%%%
% \paragraph{Main File.}
%
% The main file is called |cdocsamp.tex|.
%
% Load the \textsf{childdoc} definitions and
% declare the filename for the main document:
%    \begin{macrocode}
\input{childdoc.def}
\childdocmain{}
%    \end{macrocode}

% Optional override for |\version| flag:
%    \begin{macrocode}
%%\ifchilddoc\else\providecommand{\version}{draft}\fi
%    \end{macrocode}

% Define the default values for the |\version| flag
% (|final| for the main file and |draft| for childs):
%    \begin{macrocode}
\ifchilddoc
\providecommand{\version}{draft}
\else
\providecommand{\version}{final}
\fi
%    \end{macrocode}

% Load the standard document class:
%    \begin{macrocode}
\documentclass[12pt]{article}
%    \end{macrocode}

% Start the document body:
%    \begin{macrocode}
\begin{document}
%    \end{macrocode}

% Declare a title page.
% Print title, part of document being processed and version flag:
%    \begin{macrocode}
\addtocounter{page}{-1}
\begin{center}
{\LARGE\bfseries{}childdoc example\par}
\vspace{1cm}
\ifchilddoc
\ifchilddocmanual part\else chapter\fi:
`\childdocname' of `\childdocjob'\par
\else
main document: `\childdocjob'\par
\fi
version: \version\par
\end{center}
\newpage
%    \end{macrocode}

% Manually include selected file,
% otherwise process as usual:
%    \begin{macrocode}
\ifchilddocmanual
\section*{part `\childdocname'}
\input{\childdocname}
\else
%    \end{macrocode}

% Include the two chapters:
%    \begin{macrocode}
\include{cdocsch1}
\include{cdocsch2}
%    \end{macrocode}

% Include the two parts unless only chapters should be displayed:
%    \begin{macrocode}
\ifchilddoc\else
\section{part three}
\input{cdocspt3}
\section{part four}
\input{cdocspt4}
\fi
%    \end{macrocode}

% Process as usual until here:
%    \begin{macrocode}
\fi
%    \end{macrocode}

% End of document body:
%    \begin{macrocode}
\end{document}
%    \end{macrocode}
%\iffalse
%</samplemain>
%\fi
%
% %%%%%%%%%%%%%%%%%%%%%%%%%%%%%%%%%%%%%%
% \paragraph{Chapter Include Files.}
%
% The include files are called |cdocsch1.tex| and |cdocsch2.tex|.
%
%\iffalse
%<*samplechap1|samplechap2>
%\fi

% Optional override for |\version| flag:
%    \begin{macrocode}
%%\providecommand{\version}{final}
%    \end{macrocode}

% Include the main document:
%    \begin{macrocode}
\input{childdoc.def}
\childdocof{cdocsamp}
%    \end{macrocode}

%\iffalse
%</samplechap1|samplechap2>
%\fi
%
%\iffalse
%<*samplechap1>
%\fi
% Some text for chapter 1:
%    \begin{macrocode}
\section{one}
some text in chapter one
%    \end{macrocode}

%\iffalse
%</samplechap1>
%\fi
% Some text for chapter 2:
%\iffalse
%<*samplechap2>
%\fi
%    \begin{macrocode}
\section{two}
more text in chapter two
%    \end{macrocode}

%\iffalse
%</samplechap2>
%\fi
%
% %%%%%%%%%%%%%%%%%%%%%%%%%%%%%%%%%%%%%%
% \paragraph{Part Include Files.}
%
% The include files are called |cdocspt3.tex| and |cdocspt4.tex|.
%
%\iffalse
%<*samplepart3|samplepart4>
%\fi

% Optional override for |\version| flag:
%    \begin{macrocode}
%%\providecommand{\version}{final}
%    \end{macrocode}

% Include the main document:
%    \begin{macrocode}
\input{childdoc.def}
\childdocby{cdocsamp}
%    \end{macrocode}

%\iffalse
%</samplepart3|samplepart4>
%\fi
%
%\iffalse
%<*samplepart3>
%\fi
% Some text for part 3:
%    \begin{macrocode}
some text in part three
%    \end{macrocode}

%\iffalse
%</samplepart3>
%\fi
% Some text for part 4:
%\iffalse
%<*samplepart4>
%\fi
%    \begin{macrocode}
more text in part four
%    \end{macrocode}

%\iffalse
%</samplepart4>
%\fi
%
% %%%%%%%%%%%%%%%%%%%%%%%%%%%%%%%%%%%%%%
% \paragraph{Forwarding for a Complete Draft.}
%
% The following forwarding file |cdocsdrf.tex|
% compiles the main document in draft mode:
%\iffalse
%<*sampledraft>
%\fi
%    \begin{macrocode}
\def\version{draft}
\input{childdoc.def}
\childdocforward{cdocsamp}
%    \end{macrocode}

%\iffalse
%</sampledraft>
%\fi
%
% %%%%%%%%%%%%%%%%%%%%%%%%%%%%%%%%%%%%%%
% \paragraph{Forwarding for Final Version of the Chapters.}
%
% The following forwarding files |cdocsfn1.tex| and |cdocsfn2.tex|
% (with identical content)
% compile the final versions of the child documents
% |cdocsch1.tex| and |cdocsch2.tex|, respectively:
%\iffalse
%<*samplefinal>
%\fi
%    \begin{macrocode}
\def\version{final}
\input{childdoc.def}
\childdocforwardprefix[cdocsamp]{cdocsfn}{cdocsch}
%    \end{macrocode}

%\iffalse
%</samplefinal>
%\fi
%
% %%%%%%%%%%%%%%%%%%%%%%%%%%%%%%%%%%%%%%
% \paragraph{Command Line Processing.}
%
% The following three command lines generate the output files
% |cdocscld|, |cdocscl1| and |cdocscl2|
% which should be identical to
% |cdocsdrf|, |cdocsch1| and |cdocsfn2|, respectively:
% \begin{center}
% \begin{tabular}{l}
% |latex -jobname cdocscld \|\\
% |  "\def\version{draft}\input{childdoc.def}\childdocforward{cdocsamp}"|\\
% |latex -jobname cdocscl1 \|\\
% |  "\input{childdoc.def}\childdocforward[cdocsamp]{cdocsch1}"|\\
% |latex -jobname cdocscl2 \|\\
% |  "\def\version{final}\input{childdoc.def}\childdocforward{cdocsch2}"|
% \end{tabular}
% \end{center}
% Note that the trailing backslash on each first line
% merely continues the input to the second line
% (for convenient cut ant paste).
% Furthermore, the command |latex| can be replaced by any
% of its alternative versions such as |pdflatex|.
%
% %%%%%%%%%%%%%%%%%%%%%%%%%%%%%%%%%%%%%%%%%%%%%%%%%%%%%%%%%%%%%%%%%%%%%%%%%%%%%%
% %%%%%%%%%%%%%%%%%%%%%%%%%%%%%%%%%%%%%%%%%%%%%%%%%%%%%%%%%%%%%%%%%%%%%%%%%%%%%%
% \section{Implementation}
%\iffalse
%<*package>
%\fi
%
% This section describes the definitions file |childdoc.def|.

% The definitions cannot be loaded using |\usepackage| or |\RequirePackage|
% which has a mechanism to prevent loading a style file more than once.
% When loading the definitions by means of |\input|
% multiple instances have to be prevented manually:
%\iffalse
%This code needs to be before the `\ProvidesFile' directive
%which is defined at the beginning of this file.
%Therefore it is also placed there and commented out here.
%</package>
%<*discard>
%\fi
%    \begin{macrocode}
\ifdefined\childdocmain\endinput\fi
%    \end{macrocode}
%\iffalse
%</discard>
%<*package>
%\fi
%
% \macro{\ifchilddoc}
% \macro{\ifchilddocmanual}
% The conditional |\ifchilddoc| tells whether a
% child (true) or main (false) document is being compiled.
% The conditional |\ifchilddocmanual| tells whether
% the |\includeonly| mechanism is used (false) or
% the selection of child files must be performed manually (true).
% The definitions initialise to false:
%    \begin{macrocode}
\newif\ifchilddoc
\newif\ifchilddocmanual
%    \end{macrocode}

% \macro{\childdocname}
% \macro{\childdocjob}
% The macro |\childdocname| stores the name of the main document
% to be compiled. The macro |\childdocjob| stores the name of
% the document on which the \LaTeX{} compiler was originally invoked.
% The content of |\jobname| cannot be compared
% to filenames specified in the source due to different catcodes.
% The following code rescans |\jobname|, stores the result
% in |\childdocname| and saves a copy in |\childdocjob|:
%    \begin{macrocode}
\edef\childdocname{\scantokens\expandafter{\jobname\noexpand}}
\let\childdocjob\childdocname
%    \end{macrocode}

% \macro{\childdocdisable}
% The macro |\childdocdisable| prevents the main file
% from being processed more than once.
% At this stage, the main document command |\childdocmain|
% is assumed to be called once again where it should do nothing.
% Any subsequent call to it should prevent
% a secondary processing of the main document
% It overwrites the forwarding commands
% |\childdocof| and |\childdocforward|
% with empty macros to prevent further inclusions of the main document:
%    \begin{macrocode}
\newcommand{\childdocdisable}
{
  \renewcommand{\childdocmain}[1]{\renewcommand{\childdocmain}[1]{\endinput}}
  \renewcommand{\childdocof}[1]{}
  \renewcommand{\childdocby}[2][]{}
  \renewcommand{\childdocforward}[2][]{}
  \renewcommand{\childdocdisable}{}
}
%    \end{macrocode}

% \macro{\childdocmain}
% The macro |\childdocmain| is to be called at the top of the main file
% with nothing or the main filename (without extension) as argument.
% First, it breaks loops.
% If the argument is not empty and does not match |\childdocname|
% (which is set by the first inclusion of |childdoc.def|),
% |\ifchilddoc| is set to true, |\includeonly| is applied to the child file
% and |\jobname| is set to the main file
% (for proper handling of |.aux| files):
%    \begin{macrocode}
\newcommand{\childdocmain}[1]
{
  \childdocdisable\childdocmain{}
  \if?#1?\else
    \begingroup
      \def\childdoctmp{#1}
      \ifx\childdoctmp\childdocname
        \def\childdoctmp{}
      \else
        \def\childdoctmp
        {
          \childdoctrue
          \includeonly{\childdocname}
          \def\childdocjob{#1}
          \def\jobname{#1}
        }
      \fi
      \expandafter
    \endgroup
    \childdoctmp
  \fi
}
%    \end{macrocode}

% \macro{\childdocof}
% The command |\childdocof| redirects
% compilation to the main file |#1|.
%    \begin{macrocode}
\newcommand{\childdocof}[1]
{
  \childdocdisable
  \childdoctrue
  \includeonly{\childdocname}
  \def\jobname{#1}
  \def\childdocjob{#1}
  \input{#1}
}
%    \end{macrocode}

% \macro{\childdocby}
% The command |\childdocby| ....
%    \begin{macrocode}
\newcommand{\childdocby}[2][]
{
  \childdocdisable
  \childdoctrue
  \childdocmanualtrue
  \if?#1?\else
    \def\jobname{#2}
  \fi
  \def\childdocjob{#2}
  \input{#2}
  \endinput
}
%    \end{macrocode}

% \macro{\childdocforward}
% The command |\childdocforward| redirects
% compilation to the main file or
% (if the optional argument is given) a child file.
% Parameters are set as if the main file
% or a child file starting with |\childdocof| was compiled.
% Then compilation is handed over to the main file:
%    \begin{macrocode}
\newcommand{\childdocforward}[2][]
{
  \begingroup
    \if?#1?
      \def\childdoctmp
      {
        \def\childdocname{#2}
        \def\childdocjob{#2}
        \def\jobname{#2}
        \input{#2}
        \endinput
      }
    \else
      \def\childdoctmp
      {
        \childdocdisable
        \def\childdocname{#2}
        \childdoctrue
        \includeonly{#2}
        \def\childdocjob{#1}
        \def\jobname{#1}
        \input{#1}
        \endinput
      }
    \fi
    \expandafter
  \endgroup
  \childdoctmp
}
%    \end{macrocode}

% \macro{\childdocforwardprefix}
% The command |\childdocforwardprefix| redirects
% compilation to the main or a child file by means of a pattern.
% The prefix |#1| in the current filename is replaced by |#2|
% and the suffix of the current filename is kept
% (it is assumed that the filename does not contain the substring `|~~~|'
% which is used as a delimiter).
% Compilation is handed over to the new file by |\childdocforward|:
%    \begin{macrocode}
\newcommand{\childdocforwardprefix}[3][]
{
  \begingroup
    \def\childdocextract #2##1~~~{\def\childdoctmp{\childdocforward[#1]{#3##1}}}
    \expandafter\childdocextract\childdocname~~~
    \expandafter
  \endgroup
  \childdoctmp
}
%    \end{macrocode}

% \macro{\childdoc}
% The deprecated macro |\childdoc| is a legacy version of |\childdocmain|:
%    \begin{macrocode}
\newcommand{\childdoc}{\childdocmain}
%    \end{macrocode}

% \macro{\childdocredirect}
% The deprecated macro |\childdocredirect| is a legacy version
% of |\childdocforward| and |\childdocforwardprefix|:
%    \begin{macrocode}
\newcommand{\childdocredirect}[2][]
{
  \begingroup
    \if?#1?
      \def\childdoctmp{\childdocforward{#2}}
    \else
      \def\childdoctmp{\childdocforwardprefix{#1}{#2}}
    \fi
    \expandafter
  \endgroup
  \childdoctmp
}
%    \end{macrocode}

%\iffalse
%</package>
%\fi
%
\endinput

\childdocforward{cdocsamp}
%    \end{macrocode}

%\iffalse
%</sampledraft>
%\fi
%
% %%%%%%%%%%%%%%%%%%%%%%%%%%%%%%%%%%%%%%
% \paragraph{Forwarding for Final Version of the Chapters.}
%
% The following forwarding files |cdocsfn1.tex| and |cdocsfn2.tex|
% (with identical content)
% compile the final versions of the child documents
% |cdocsch1.tex| and |cdocsch2.tex|, respectively:
%\iffalse
%<*samplefinal>
%\fi
%    \begin{macrocode}
\def\version{final}
% \iffalse
%
% childdoc.dtx Copyright (C) 2017-2018 Niklas Beisert
%
% This work may be distributed and/or modified under the
% conditions of the LaTeX Project Public License, either version 1.3
% of this license or (at your option) any later version.
% The latest version of this license is in
%   http://www.latex-project.org/lppl.txt
% and version 1.3 or later is part of all distributions of LaTeX
% version 2005/12/01 or later.
%
% This work has the LPPL maintenance status `maintained'.
%
% The Current Maintainer of this work is Niklas Beisert.
%
% This work consists of the files childdoc.dtx and childdoc.ins
% and the derived files childdoc.def and cdocsamp.tex with
% cdocsch1.tex, cdocsch2.tex, cdocsdrf.tex, cdocsfn1.tex, cdocsfn2.tex.
%
%<package>\ifdefined\childdocmain\endinput\fi
%<package>\ProvidesFile{childdoc.def}[2018/12/30 v2.0 child document driver]
%<samplemain>\ProvidesFile{cdocsamp.tex}[2018/12/30 v2.0 sample for childdoc]
%<*driver>
%\ProvidesFile{childdoc.drv}[2018/12/30 v2.0 childdoc reference manual file]
\PassOptionsToClass{10pt,a4paper}{article}
\documentclass{ltxdoc}

\usepackage[margin=35mm]{geometry}
\usepackage{hyperref}
\usepackage{hyperxmp}
\usepackage[usenames]{color}

\hypersetup{colorlinks=true}
\hypersetup{pdfstartview=FitH}
\hypersetup{pdfpagemode=UseNone}
\hypersetup{pdfsource={}}
\hypersetup{pdflang={en-UK}}
\hypersetup{pdfcopyright={Copyright 2017-2018 Niklas Beisert.
  This work may be distributed and/or modified under the
  conditions of the LaTeX Project Public License, either version 1.3
  of this license or (at your option) any later version.}}
\hypersetup{pdflicenseurl={http://www.latex-project.org/lppl.txt}}
\hypersetup{pdfcontactaddress={ETH Zurich, ITP, HIT K,
  Wolfgang-Pauli-Strasse 27}}
\hypersetup{pdfcontactpostcode={8093}}
\hypersetup{pdfcontactcity={Zurich}}
\hypersetup{pdfcontactcountry={Switzerland}}
\hypersetup{pdfcontactemail={nbeisert@itp.phys.ethz.ch}}
\hypersetup{pdfcontacturl={http://people.phys.ethz.ch/\xmptilde nbeisert/}}

\newcommand{\secref}[1]{\hyperref[#1]{section \ref*{#1}}}

\parskip1ex
\parindent0pt
\let\olditemize\itemize
\def\itemize{\olditemize\parskip0pt}

\begin{document}

\title{The \textsf{childdoc} Package}
\hypersetup{pdftitle={The childdoc Package}}
\author{Niklas Beisert\\[2ex]
  Institut f\"ur Theoretische Physik\\
  Eidgen\"ossische Technische Hochschule Z\"urich\\
  Wolfgang-Pauli-Strasse 27, 8093 Z\"urich, Switzerland\\[1ex]
  \href{mailto:nbeisert@itp.phys.ethz.ch}
  {\texttt{nbeisert@itp.phys.ethz.ch}}}
\hypersetup{pdfauthor={Niklas Beisert}}
\hypersetup{pdfsubject={Manual for the LaTeX2e Package childdoc}}
\date{30 December 2018, \textsf{v2.0}}
\maketitle

\begin{abstract}\noindent
\textsf{childdoc} is a \LaTeXe{} package
that enables the direct compilation
of document sections included by |\include|
to individual files.
\end{abstract}

\begingroup
\parskip0ex
\tableofcontents
\endgroup

%%%%%%%%%%%%%%%%%%%%%%%%%%%%%%%%%%%%%%%%%%%%%%%%%%%%%%%%%%%%%%%%%%%%%%%%%%%%%%%%
%%%%%%%%%%%%%%%%%%%%%%%%%%%%%%%%%%%%%%%%%%%%%%%%%%%%%%%%%%%%%%%%%%%%%%%%%%%%%%%%
\section{Introduction}

\LaTeX{} provides a mechanism to structure a large document (such as a book)
into a main file and several child files (containing the chapters)
using the |\include| command.
This mechanism is beneficial for documents
which span hundreds of pages in order to
make the source file(s) more manageable.
Moreover, compilation can be restricted to
selected child files by means of the |\includeonly| command.
The latter feature can be used to reduce the compilation time while editing
(this was significantly more useful in the earlier days of \LaTeX{})
or to generate a smaller document which is easier to navigate.
Another application of |\includeonly| is to generate
documents consisting of selected parts of the complete document.

However, there are a few drawbacks of the plain |\include| mechanism:
\begin{itemize}
\item
The child files cannot be compiled on their own,
they can only be compiled via the main file.
A naive editing environment
(such as a text editor with an option
to have the current file processed by \LaTeX)
may require one to switch to the main file before compiling;
attempting to compile the child file produces errors.
\item
The main file must be modified (each time)
to adjust the |\includeonly| command
to the present needs. This easily leaves the main file in a messy state.
\item
The generated document will always carry the filename
of the main document. This is inconvenient if
several child files are to be compiled and
to be kept for distribution.
\end{itemize}

The present package provides a simple interface
to make child files individually compilable by \LaTeX{}.
Compiling a child file then has the same effect as compiling
the main file with an |\includeonly| command
to select the appropriate child.
Moreover the generated document will carry the name of the child
rather than the main file.
This resolves all three above issues.

This feature is meant to make the editing of books,
thesis documents and lecture notes somewhat more convenient.
However, the package can also be used efficiently for
composing a series of documents (such as exercise sheets)
which are typically distributed individually.
It then assists the author in generating the individual documents
(potentially in different versions)
as well as a document containing the collected series.
Another application is in developing style files
or other kinds of included material
where compilation of the style file could redirect
to a sample or test file.

%%%%%%%%%%%%%%%%%%%%%%%%%%%%%%%%%%%%%%%%%%%%%%%%%%%%%%%%%%%%%%%%%%%%%%%%%%%%%%%%
%%%%%%%%%%%%%%%%%%%%%%%%%%%%%%%%%%%%%%%%%%%%%%%%%%%%%%%%%%%%%%%%%%%%%%%%%%%%%%%%
\section{Usage}

First of all, the package \textsf{childdoc} is \emph{not} a standard
\LaTeXe{} |.sty| style file! Therefore it needs to be invoked in
a non-standard way.

%%%%%%%%%%%%%%%%%%%%%%%%%%%%%%%%%%%%%%%%%%%%%%%%%%%%%%%%%%%%%%%%%%%%%%%%%%%%%%%%
\subsection{Included Files}
\label{sec:include}

%%%%%%%%%%%%%%%%%%%%%%%%%%%%%%%%%%%%%%%%
\DescribeMacro{\childdocmain}
To use the package, add the commands
\begin{center}
\begin{tabular}{l}
|\input{childdoc.def}|\\
|\childdocmain{}|\\
\end{tabular}
\end{center}
at the very top of the main \LaTeX{} file,
in particular \emph{before} the |\documentclass| statement!
The argument of |\childdocmain| should be left empty
(but it must be present).

%%%%%%%%%%%%%%%%%%%%%%%%%%%%%%%%%%%%%%%%
\DescribeMacro{\childdocof}
Furthermore, add the commands
\begin{center}
\begin{tabular}{l}
|\input{childdoc.def}|\\
|\childdocof{|\textit{main}|}|\\
\end{tabular}
\end{center}
at the top of every child file \textit{child}
which is included by |\include{|\textit{child}|}|
from within the main file
(or at least for those files to be compiled individually).
The argument \textit{main} must be the filename of the main file.

There are a couple of
considerations in setting up the main and child documents:

%%%%%%%%%%%%%%%%%%%%%%%%%%%%%%%%%%%%%%%%
\paragraph{Restrictions.}

Please note the following restrictions:
\begin{itemize}
\item
|\childdocmain| must be called with one argument \textit{main}
to ensure compatibility with earlier version of the package.
It must either be empty (|\childdocmain{}|)
or precisely match the filename of the main file in which it is specified.
See \secref{sec:detection} for further information.
\item
The filename \textit{main} must be specified without the |.tex| extension.
\item
The filename \textit{main} is case sensitive
(even in case-insensitive file systems)
due to internal string comparison.
\item
The argument \textit{main} should be fully expanded, it cannot be a macro.
\item
Subdirectories and special characters should be avoided in filenames.
\item
The command |\childdocmain{|\textit{main}|}| must be followed by a whitespace.
It should not be followed immediately by another command
or by a comment mark `|%|'.
This is because the \TeX{} parser reads the token immediately following
the argument of |\childdocmain| and puts it
at the beginning of every child section;
however, a white\-space is ignored.
\end{itemize}

%%%%%%%%%%%%%%%%%%%%%%%%%%%%%%%%%%%%%%%%
\paragraph{Content of Main File.}

It is advisable to place all content in the child files included by |\include|.
Any output contained in the main file will appear in all child documents
unless suppressed manually;
it cannot be suppressed automatically by the |\includeonly| directive
and thus should normally be avoided.
A method to include some content in the main file
by means of conditional processing is described in \secref{sec:conditional}.

%%%%%%%%%%%%%%%%%%%%%%%%%%%%%%%%%%%%%%%%
\paragraph{Page Numbering.}

When only a part of the document is compiled,
the appropriate numbering of pages
(as well as other status parameters)
is determined from the |.aux| files.
The latter contain information from previous passes.
However this information needs to propagate through
all intermediate child documents.
Therefore the page numbering in child documents may well
be inconsistent until the complete document is compiled at least once.

A useful (if unconventional) way to always ensure a consistent
page numbering is to restart the numbering in each child document
and denote the pages by `\textit{child}|.|\textit{page}'
where \textit{child} represents the chapter/section number of the child file.
This can be achieved by the command
|\numberwithin{page}{|\textit{child}|}|
of the \textsf{amsmath} package
where \textit{child} can be |chapter| or |section|
depending on the chosen structuring.
Alternatively, one can modify the macro |\thepage| appropriately
and reset the counter |page| at the start of each child file.

%%%%%%%%%%%%%%%%%%%%%%%%%%%%%%%%%%%%%%%%%%%%%%%%%%%%%%%%%%%%%%%%%%%%%%%%%%%%%%%%
\subsection{Conditional Processing}
\label{sec:conditional}

The package provides a mechanism to compile different versions
of a document. To customise the versions further some conditional processing
can come in handy to distinguish which version is being compiled.
The package provides two macros to describe the compilation context:

%%%%%%%%%%%%%%%%%%%%%%%%%%%%%%%%%%%%%%%%
\DescribeMacro{\ifchilddoc}
The conditional |\ifchilddoc| distinguishes between the compilation of
child documents and the main document:
%
\begin{center}
|\ifchilddoc |\textit{child-code}| |[|\||else |\textit{main-code}]| \||fi|
\end{center}

%%%%%%%%%%%%%%%%%%%%%%%%%%%%%%%%%%%%%%%%
\DescribeMacro{\childdocname}
\DescribeMacro{\childdocjob}
The macro |\childdocname| contains the filename (without extension)
of the main or child file being processed.
Note that |\childdocjob| will always contain the name of the main file.

%%%%%%%%%%%%%%%%%%%%%%%%%%%%%%%%%%%%%%%%
\paragraph{Title Page.}

Conditional processing can be used to include a title or banner page
in the main document when proper precautions are taken.
Importantly, the code in the main file should ensure that the page counter
(as well as other status parameters which are stored in the |.aux| files)
takes the same value after the conditional processing.
Otherwise the page numbers may take divergent values
depending on which part is compiled.

For example, a title page could be declared by:
%
\begin{center}
\begin{tabular}{l}
|\ifchilddoc\||else|\\
|\addtocounter{page}{-1}|\\
\textit{code for title page}\\
|\newpage|\\
|\||fi|
\end{tabular}
\end{center}
%
A banner page for the child documents can be generated by:
%
\begin{center}
\begin{tabular}{l}
|\ifchilddoc|\\
|\addtocounter{page}{-1}|\\
\textit{code for banner page}\\
|\newpage|\\
|\||fi|
\end{tabular}
\end{center}
%
Here one could write a message such as:
\begin{center}
|This is the part \childdocname{} of \childdocjob{}.|
\end{center}

%%%%%%%%%%%%%%%%%%%%%%%%%%%%%%%%%%%%%%%%%%%%%%%%%%%%%%%%%%%%%%%%%%%%%%%%%%%%%%%%
\subsection{Flags}
\label{sec:flags}

The package makes it easy to generate different versions
of the main or child documents.
To this end compilation flags can be defined
and assigned different default values.
They will be particularly useful in conjunction
with the forwarding mechanism described in \secref{sec:forward}.

For example, it may be useful to have a flag |\version|
which can be set to |draft| or |final|.
The document source will contain some conditional code
depending on the value of |\version|.
Suppose further, the flag should default to |final| for the main file
and to |draft| for child files
which is a natural assignment for editing the document.
This is achieved by placing the following code
in the preamble of the main document
(below the |\childdocmain| directive):
%
\begin{center}
\begin{tabular}{l}
|\ifchilddoc|\\
|\providecommand{\version}{draft}|\\
|\||else|\\
|\providecommand{\version}{final}|\\
|\||fi|
\end{tabular}
\end{center}
%
The definition by |\providecommand| makes sure
that previous definitions are not overwritten.
Further statements |\providecommand{\version}{...}|
can thus be added before the above code to override it.

For the main file, one might add a line
(between |\childdocmain| and the above block)
%
\begin{center}
|%\ifchilddoc\||else\providecommand{\version}{draft}\||fi|
\end{center}
%
which can be uncommented to produce a draft version.
Likewise one can add a line to the very top of a child file
(above the |\childdocof{|\textit{main}|}| directive)
%
\begin{center}
|%\providecommand{\version}{final}|
\end{center}
%
which can be uncommented to produce the final version of this child document.

%%%%%%%%%%%%%%%%%%%%%%%%%%%%%%%%%%%%%%%%%%%%%%%%%%%%%%%%%%%%%%%%%%%%%%%%%%%%%%%%
\subsection{Forwarding}
\label{sec:forward}

Different versions of the main or child documents
using compilation flags as described in \secref{sec:flags}
can be (permanently) stored in different files
for convenient compilation, viewing and distribution.
To this end, the package defines a command
to pass on compilation to a different file:

%%%%%%%%%%%%%%%%%%%%%%%%%%%%%%%%%%%%%%%%
\DescribeMacro{\childdocforward}
The command |\childdocforward| redirects processing to
another source file:
%
\begin{center}
\begin{tabular}{l}
|\input{childdoc.def}|\\
|\childdocforward[|\textit{main}|]{|\textit{dest}|}|\\
\end{tabular}
\end{center}
%
The argument \textit{dest} is the destination file
(without extension).
It should be the main file or one of the child files.
Note that further \textsf{childdoc} directives
such as |\childdocof| and |\childdocforward|
in the indicated file will be processed in this form.
The optional argument \textit{main}
passes on directly to the main file \textit{main}
while pretending to compile the child \textit{dest}.
This form behaves as if \textit{dest}
issues |\childdocof{|\textit{main}|}| right away,
and no further \textsf{childdoc} directives will be processed.

%%%%%%%%%%%%%%%%%%%%%%%%%%%%%%%%%%%%%%%%
\DescribeMacro{\...prefix}
In the alternative form |\childdocforwardprefix|,
%
\begin{center}
\begin{tabular}{l}
|\input{childdoc.def}|\\
|\childdocforwardprefix[|\textit{main}|]{|\textit{prefix}|}{|\textit{dest}|}|
\end{tabular}
\end{center}
%
the destination file is determined by a pattern
depending on the current file:
To make this work, the current file must be called
`{\textit{prefix}\hspace{0.2em}\textit{suffix}}'
with \textit{prefix} matching precisely the argument.
Processing is then passed on to the file
`{\textit{dest}\hspace{0.2em}\textit{suffix}}'.
Surely, the same effect is achieved by
directly specifying the
argument `{\textit{dest}\hspace{0.2em}\textit{suffix}}'
in the first form.
However, that requires to set up a different file
for each child. With the alternative form of the command
all these files can have exactly the same content
which simplifies setting them up and maintaining them.

For example, the following file |draft.tex|
with a compilation flag |\version| as described in \secref{sec:flags}
compiles the main document as a draft:
%
\begin{center}
\begin{tabular}{l}
|\def\version{draft}|\\
|\input{childdoc.def}|\\
|\childdocforward{|\textit{main}|}|
\end{tabular}
\end{center}
%
Likewise, the following files |final|\textit{nn}|.tex|
compile the final version of the child document
|child|\textit{nn}|.tex|:
%
\begin{center}
\begin{tabular}{l}
|\def\version{final}|\\
|\input{childdoc.def}|\\
|\childdocforwardprefix{final}{child}|
\end{tabular}
\end{center}
%

Note that when several versions of a main file and/or of each child file
are to be generated, it may be convenient to set up a |Makefile| or
shell script to automatise the process.

%%%%%%%%%%%%%%%%%%%%%%%%%%%%%%%%%%%%%%%%%%%%%%%%%%%%%%%%%%%%%%%%%%%%%%%%%%%%%%%%
\subsection{Command Line Processing}
\label{sec:commandline}

The effect of redirection files can also be achieved by invoking
the \LaTeX{} compiler with a more elaborate command line.
Most conveniently this should be done as part
of a shell script or a |Makefile|.

When using \textsf{childdoc} in the main file, the following
command lines effectively perform a redirection
(note that depending on the shell being used,
backslashes may have to be doubled: `|\|' $\to$ `|\\|'):
%
\begin{center}
|... -jobname "|\textit{target}|" |\\|"|[\textit{flags}]%
|\input{childdoc.def}\childdocforward[|\textit{main}|]{|\textit{dest}|}"|
\end{center}
%
Here \textit{target} is the name of the output file,
\textit{main} is the name of the main file
and \textit{dest} is the name of the main or child file to be processed
(all filenames without extensions).
The optional argument \textit{main} can be omitted
if \textit{main} matches \textit{dest}.
Optionally, compilation \textit{flags} can be defined via |\def| commands.
This command line makes the \TeX{} engine believe
it is compiling the file \textit{target}
whose content is specified as the latter parameter.
The provided code then forwards the processing to
\textit{main} or \textit{dest} as described in \secref{sec:forward}.

%%%%%%%%%%%%%%%%%%%%%%%%%%%%%%%%%%%%%%%%%%%%%%%%%%%%%%%%%%%%%%%%%%%%%%%%%%%%%%%%
\subsection{Include by Input}
\label{sec:input}

Including child documents by |\include| has some restrictions by design.
Most notably, the content of a child document always occupies
its own set of pages; pages cannot be shared between child documents.
Usually, this behaviour makes perfect sense
because each child document contain an essential part of the document.
However, in some situations it may be desirable to compose
a document from a collection of parts
without having mandatory page breaks between then.
For this case, the package
provides a mechanism to include parts
by |\input| which can also be processed individually.
However, by construction this mechanism
requires manual handling of the content to be output.

%%%%%%%%%%%%%%%%%%%%%%%%%%%%%%%%%%%%%%%%
\DescribeMacro{\ifchilddocmanual}
The main file should be prepared as usual, see \secref{sec:include}.
However, the document body must make a distinction
between processing of an individual part and of the main document, e.g.:
%
\begin{center}
\begin{tabular}{l}
|\ifchilddocmanual|\\
|\input{\childdocname}|\\
|\||else|\\
\textit{document body with }|\input{|\textit{part}|}|\\
|\||fi|
\end{tabular}
\end{center}
%
The conditional |\ifchilddocmanual| is true whenever
a part to be included by |\input| is being compiled,
and the name of the part is stored in |\childdocname|.

%%%%%%%%%%%%%%%%%%%%%%%%%%%%%%%%%%%%%%%%
\DescribeMacro{\childdocby}
Each part to be included by |\input| should start with:
%
\begin{center}
\begin{tabular}{l}
|\input{childdoc.def}|\\
|\childdocby{|\textit{main}|}|\\
\end{tabular}
\end{center}
%
The directive |\childdocby| is similar to |\childdocof|
described in \secref{sec:include},
but the subsequent selection of content must be done manually.
To that end, both |\ifchilddoc| and |\ifchilddocmanual|
will be true upon processing of a part,
and the name of the part is stored in |\childdocname|.
Note that |\jobname| will be set to the filename of the current part
so that each part receives an individual |.aux| file
that does not interfere with the |.aux| file(s) of the main document.
This behaviour can be altered by the alternative form
|\childdocby[*]{|\textit{main}|}| (with a non-empty optional argument)
which uses the |.aux| file of the main document
by setting |\jobname| to \textit{main}.

%%%%%%%%%%%%%%%%%%%%%%%%%%%%%%%%%%%%%%%%%%%%%%%%%%%%%%%%%%%%%%%%%%%%%%%%%%%%%%%%
\subsection{Driver Development}
\label{sec:driver}

The \textsf{childdoc} mechanism can also be use for the development
of definition files such as \LaTeX{} styles or classes.
This case differs from the above setup with multiple parts
included by |\include| in that no |\includeonly| should be invoked.
This can be achieved by starting the include file
(before |\ProvidesPackage|) with:
%
\begin{center}
\begin{tabular}{l}
|\input{childdoc.def}|\\
|\childdocforward{|\textit{main}|}|\\
\end{tabular}
\end{center}
%
or alternatively with:
%
\begin{center}
\begin{tabular}{l}
|\input{childdoc.def}|\\
|\childdocby{|\textit{main}|}|\\
\end{tabular}
\end{center}
%
Both forms have slightly different effects as described above.
The main file is prepared as usual, see \secref{sec:include}.

%%%%%%%%%%%%%%%%%%%%%%%%%%%%%%%%%%%%%%%%%%%%%%%%%%%%%%%%%%%%%%%%%%%%%%%%%%%%%%%%
\subsection{Legacy Detection}
\label{sec:detection}

The directive |\childdocmain| in the main file can detect
whether the complete document or merely a child is to be compiled
even without using the directive |\childdocof|.
This method is deprecated because it is less robust
and there is no compelling reason to use it;
it is merely provided for backward compatibility
and it may be removed in future versions.

If the detection mechanism is to be used,
it is mandatory to correctly specify
the filename of the main file as the argument of |\childdocmain|:
%
\begin{center}
\begin{tabular}{l}
|\input{childdoc.def}|\\
|\childdocmain{|\textit{main}|}|\\
\end{tabular}
\end{center}
%
If |\jobname| does not match the argument \textit{main} of |\childdocmain|,
it is assumed that |\jobname| points to the child file to be compiled.
When using |\childdocmain| with the main file specified as argument,
it suffices to start a child file
with just |\input{|\textit{main}|}|
without loading of the package and using |\childdocof|.
If instead all processing is done
with the appropriate \textsf{childdoc} directives,
the argument of \textit{main} of |\childdocmain| can be empty.

An alternative version of the command line processing described
in \secref{sec:commandline} using the detection mechanism reads:
%
\begin{center}
|... -jobname "|\textit{target}|" "|[\textit{flags}]%
[|\def\jobname{|\textit{dest}|}|]|\input{|\textit{main}|}"|
\end{center}

%%%%%%%%%%%%%%%%%%%%%%%%%%%%%%%%%%%%%%%%%%%%%%%%%%%%%%%%%%%%%%%%%%%%%%%%%%%%%%%%
\subsection{Manual Code}
\label{sec:manual}

In case one cannot be certain whether the definitions file |childdoc.def|
is installed on the target \TeX{} distribution
and one prefers not to ship it,
it is conceivable to paste a few relevant commands into the sources.

To that end, drop all statements |\input{childdoc.def}|
and perform the replacements as outlined below.
Instead of |\childdocmain{|\textit{main}|}| add the following code
to the top of the main file:
%
\begin{center}
\begin{tabular}{l}
|\||ifdefined\childdocname\endinput\||fi\newif\ifchilddoc|\\
|\edef\childdocname{\scantokens\expandafter{\jobname\noexpand}}|\\
|\def\childdocmain{|\textit{main}|}\||ifx\childdocmain\childdocname\||else|\\
|\childdoctrue\includeonly{\childdocname}\let\jobname\childdocmain\||fi|\\
\end{tabular}
\end{center}
%
Instead of |\childdocof{|\textit{main}|}| just include the main file
at the top of each child file:
%
\begin{center}
|\input{|\textit{main}|}|
\end{center}
%
A simple redirection |\childdocforward{|\textit{dest}|}| is achieved by:
%
\begin{center}
|\def\jobname{|\textit{dest}|}\input{\jobname}|
\end{center}
%
The redirection with prefix
|\childdocforwardprefix[|\textit{prefix}|]{|\textit{dest}|}|
is accomplished by:
%
\begin{center}
\begin{tabular}{l}
|{\edef\jobname{\scantokens\expandafter{\jobname\noexpand}}|\\
|\def\redirectjob |\textit{prefix}|#1~~~{\gdef\jobname{|\textit{dest}|#1}}|\\
|\expandafter\redirectjob\jobname~~~}\input{\jobname}|
\end{tabular}
\end{center}

In an alternative approach,
child documents can be compiled by a specific command line
without additional code or specific definitions:
%
\begin{center}
|... -jobname "|\textit{target}|" "|[\textit{flags}]%
|\includeonly{|\textit{dest}|}\input{|\textit{main}|}"|
\end{center}
%

%%%%%%%%%%%%%%%%%%%%%%%%%%%%%%%%%%%%%%%%%%%%%%%%%%%%%%%%%%%%%%%%%%%%%%%%%%%%%%%%
%%%%%%%%%%%%%%%%%%%%%%%%%%%%%%%%%%%%%%%%%%%%%%%%%%%%%%%%%%%%%%%%%%%%%%%%%%%%%%%%
\section{Information}

%%%%%%%%%%%%%%%%%%%%%%%%%%%%%%%%%%%%%%%%%%%%%%%%%%%%%%%%%%%%%%%%%%%%%%%%%%%%%%%%
\subsection{Copyright}

Copyright \copyright{} 2017--2018 Niklas Beisert

This work may be distributed and/or modified under the
conditions of the \LaTeX{} Project Public License, either version 1.3
of this license or (at your option) any later version.
The latest version of this license is in
  \url{http://www.latex-project.org/lppl.txt}
and version 1.3 or later is part of all distributions of \LaTeX{}
version 2005/12/01 or later.

This work has the LPPL maintenance status `maintained'.

The Current Maintainer of this work is Niklas Beisert.

This work consists of the files |README.txt|, |childdoc.ins| and |childdoc.dtx|
as well as the derived files |childdoc.def|, |cdocsamp.tex|
with |cdocsch1.tex|, |cdocsch2.tex|, |cdocspt3.tex|, |cdocspt4.tex|,
|cdocsdrf.tex|, |cdocsfn1.tex|, |cdocsfn2.tex|
as well as |childdoc.pdf|.

%%%%%%%%%%%%%%%%%%%%%%%%%%%%%%%%%%%%%%%%%%%%%%%%%%%%%%%%%%%%%%%%%%%%%%%%%%%%%%%%
\subsection{Files and Installation}

The package consists of the files:
%
\begin{center}
\begin{tabular}{ll}
    |README.txt|   & readme file \\
    |childdoc.ins| & installation file \\
    |childdoc.dtx| & source file \\
    |childdoc.def| & definition file \\
    |cdocsamp.tex| & sample main file \\
    |cdocsch1.tex| & sample include file \\
    |cdocsch2.tex| & sample include file \\
    |cdocspt3.tex| & sample part file \\
    |cdocspt4.tex| & sample part file \\
    |cdocsdrf.tex| & sample redirection file \\
    |cdocsfn1.tex| & sample redirection file \\
    |cdocsfn2.tex| & sample redirection file \\
    |childdoc.pdf| & manual
\end{tabular}
\end{center}
%
The distribution consists of the files
|README.txt|, |childdoc.ins| and |childdoc.dtx|.
%
\begin{itemize}
\item
Run (pdf)\LaTeX{} on |childdoc.dtx|
to compile the manual |childdoc.pdf| (this file).
\item
Run \LaTeX{} on |childdoc.ins| to create the definitions file |childdoc.def|
and the sample |cdocsamp.tex| with include files
|cdocsch1.tex|, |cdocsch2.tex|, |cdocspt3.tex|, |cdocspt4.tex|,
|cdocsdrf.tex|, |cdocsfn1.tex|, |cdocsfn2.tex|.
Then copy the file |childdoc.def| to an appropriate directory of your \LaTeX{}
distribution, e.g.\ \textit{texmf-root}|/tex/latex/childdoc|.
\end{itemize}

%%%%%%%%%%%%%%%%%%%%%%%%%%%%%%%%%%%%%%%%%%%%%%%%%%%%%%%%%%%%%%%%%%%%%%%%%%%%%%%%
\subsection{Related CTAN Packages}

There are several other packages which offer a similar functionality:
%
\begin{itemize}
\item
The packages
\href{http://ctan.org/pkg/docmute}{\textsf{docmute}},
\href{http://ctan.org/pkg/includex}{\textsf{includex}} and
\href{http://ctan.org/pkg/standalone}{\textsf{standalone}}
provide commands to include only the document body of
a child file thus allowing both files to be compiled individually.
\item
The packages \href{http://ctan.org/pkg/subdocs}{\textsf{subdocs}}
and \href{http://ctan.org/pkg/subfiles}{\textsf{subfiles}}
provide structures in which the main and child documents can be
encapsulated and allowing them to be compiled individually.
The inclusion mechanism is different from the conventional |\include|.
\item
The package \href{http://ctan.org/pkg/combine}{\textsf{combine}}
is an elaborate solution to combine several documents into one.
\end{itemize}
%
See also the CTAN topic \href{http://ctan.org/topic/subdocs}{\textsf{subdocs}}
for further related packages.
The present package differs from the above solutions in that
a document structure constructed with the conventional |\include| mechanism
just needs two extra commands at the top of every file
such that all constituent files can be compiled individually.

%%%%%%%%%%%%%%%%%%%%%%%%%%%%%%%%%%%%%%%%%%%%%%%%%%%%%%%%%%%%%%%%%%%%%%%%%%%%%%%%
%\subsection{Feature Suggestions}
%
%The following is a list of features which may be useful for future
%versions of this package:
%%
%\begin{itemize}
%\item
%\ldots
%\end{itemize}

%%%%%%%%%%%%%%%%%%%%%%%%%%%%%%%%%%%%%%%%%%%%%%%%%%%%%%%%%%%%%%%%%%%%%%%%%%%%%%%%
\subsection{Revision History}

%%%%%%%%%%%%%%%%%%%%%%%%%%%%%%%%%%%%%%%%
\paragraph{v2.0:} 2018/12/30

\begin{itemize}
\item
immediate forward processing
\item
added |\childdocby| mechanism
\item
manual restructured
\end{itemize}

%%%%%%%%%%%%%%%%%%%%%%%%%%%%%%%%%%%%%%%%
\paragraph{v1.6:} 2018/01/17

\begin{itemize}
\item
application for development of include files
\item
corrections to manual
\end{itemize}

%%%%%%%%%%%%%%%%%%%%%%%%%%%%%%%%%%%%%%%%
\paragraph{v1.5:} 2017/05/21

\begin{itemize}
\item
more complete structuring introduced
\item
|\childdocof| introduced
\item
|\childdoc| renamed to |\childdocmain|
\item
|\childredirect| renamed to |\childdocforward| and |\childdocforwardprefix|
and functionality expanded
\end{itemize}

%%%%%%%%%%%%%%%%%%%%%%%%%%%%%%%%%%%%%%%%
\paragraph{v1.0:} 2017/04/27

\begin{itemize}
\item
manual and install package
\item
first version published on CTAN
\end{itemize}

%%%%%%%%%%%%%%%%%%%%%%%%%%%%%%%%%%%%%%%%
\paragraph{v0.6:} 2017/04/26

\begin{itemize}
\item
redirection mechanism added
\end{itemize}

%%%%%%%%%%%%%%%%%%%%%%%%%%%%%%%%%%%%%%%%
\paragraph{v0.5:} 2017/04/26

\begin{itemize}
\item
functionality in definition file
\end{itemize}


%%%%%%%%%%%%%%%%%%%%%%%%%%%%%%%%%%%%%%%%%%%%%%%%%%%%%%%%%%%%%%%%%%%%%%%%%%%%%%%%
%%%%%%%%%%%%%%%%%%%%%%%%%%%%%%%%%%%%%%%%%%%%%%%%%%%%%%%%%%%%%%%%%%%%%%%%%%%%%%%%
%%%%%%%%%%%%%%%%%%%%%%%%%%%%%%%%%%%%%%%%%%%%%%%%%%%%%%%%%%%%%%%%%%%%%%%%%%%%%%%%
\appendix

\settowidth\MacroIndent{\rmfamily\scriptsize 000\ }

 \DocInput{childdoc.dtx}

\end{document}
%</driver>
% \fi
%
% %%%%%%%%%%%%%%%%%%%%%%%%%%%%%%%%%%%%%%%%%%%%%%%%%%%%%%%%%%%%%%%%%%%%%%%%%%%%%%
% %%%%%%%%%%%%%%%%%%%%%%%%%%%%%%%%%%%%%%%%%%%%%%%%%%%%%%%%%%%%%%%%%%%%%%%%%%%%%%
% \section{Sample}
%\iffalse
%<*samplemain>
%\fi
%
% The following presents a sample document
% with two chapters, two parts, a title page,
% a compile flag as well as three forwarding files to set the flag.
% It consists of eight |.tex| files:
% \begin{center}
% \begin{tabular}{ll}
% |cdocsamp.tex|&main file\\
% |cdocsch1.tex|&include file for chapter 1\\
% |cdocsch2.tex|&include file for chapter 2\\
% |cdocspt3.tex|&include file for part 3\\
% |cdocspt4.tex|&include file for part 4\\
% |cdocsdrf.tex|&forwarding file for main file in draft mode\\
% |cdocsfi1.tex|&forwarding file for final version of chapter 1\\
% |cdocsfi2.tex|&forwarding file for final version of chapter 2\\
% \end{tabular}
% \end{center}
% Each of the eight files can be compiled directly by the \LaTeX{} compiler.
%
% %%%%%%%%%%%%%%%%%%%%%%%%%%%%%%%%%%%%%%
% \paragraph{Main File.}
%
% The main file is called |cdocsamp.tex|.
%
% Load the \textsf{childdoc} definitions and
% declare the filename for the main document:
%    \begin{macrocode}
\input{childdoc.def}
\childdocmain{}
%    \end{macrocode}

% Optional override for |\version| flag:
%    \begin{macrocode}
%%\ifchilddoc\else\providecommand{\version}{draft}\fi
%    \end{macrocode}

% Define the default values for the |\version| flag
% (|final| for the main file and |draft| for childs):
%    \begin{macrocode}
\ifchilddoc
\providecommand{\version}{draft}
\else
\providecommand{\version}{final}
\fi
%    \end{macrocode}

% Load the standard document class:
%    \begin{macrocode}
\documentclass[12pt]{article}
%    \end{macrocode}

% Start the document body:
%    \begin{macrocode}
\begin{document}
%    \end{macrocode}

% Declare a title page.
% Print title, part of document being processed and version flag:
%    \begin{macrocode}
\addtocounter{page}{-1}
\begin{center}
{\LARGE\bfseries{}childdoc example\par}
\vspace{1cm}
\ifchilddoc
\ifchilddocmanual part\else chapter\fi:
`\childdocname' of `\childdocjob'\par
\else
main document: `\childdocjob'\par
\fi
version: \version\par
\end{center}
\newpage
%    \end{macrocode}

% Manually include selected file,
% otherwise process as usual:
%    \begin{macrocode}
\ifchilddocmanual
\section*{part `\childdocname'}
\input{\childdocname}
\else
%    \end{macrocode}

% Include the two chapters:
%    \begin{macrocode}
\include{cdocsch1}
\include{cdocsch2}
%    \end{macrocode}

% Include the two parts unless only chapters should be displayed:
%    \begin{macrocode}
\ifchilddoc\else
\section{part three}
\input{cdocspt3}
\section{part four}
\input{cdocspt4}
\fi
%    \end{macrocode}

% Process as usual until here:
%    \begin{macrocode}
\fi
%    \end{macrocode}

% End of document body:
%    \begin{macrocode}
\end{document}
%    \end{macrocode}
%\iffalse
%</samplemain>
%\fi
%
% %%%%%%%%%%%%%%%%%%%%%%%%%%%%%%%%%%%%%%
% \paragraph{Chapter Include Files.}
%
% The include files are called |cdocsch1.tex| and |cdocsch2.tex|.
%
%\iffalse
%<*samplechap1|samplechap2>
%\fi

% Optional override for |\version| flag:
%    \begin{macrocode}
%%\providecommand{\version}{final}
%    \end{macrocode}

% Include the main document:
%    \begin{macrocode}
\input{childdoc.def}
\childdocof{cdocsamp}
%    \end{macrocode}

%\iffalse
%</samplechap1|samplechap2>
%\fi
%
%\iffalse
%<*samplechap1>
%\fi
% Some text for chapter 1:
%    \begin{macrocode}
\section{one}
some text in chapter one
%    \end{macrocode}

%\iffalse
%</samplechap1>
%\fi
% Some text for chapter 2:
%\iffalse
%<*samplechap2>
%\fi
%    \begin{macrocode}
\section{two}
more text in chapter two
%    \end{macrocode}

%\iffalse
%</samplechap2>
%\fi
%
% %%%%%%%%%%%%%%%%%%%%%%%%%%%%%%%%%%%%%%
% \paragraph{Part Include Files.}
%
% The include files are called |cdocspt3.tex| and |cdocspt4.tex|.
%
%\iffalse
%<*samplepart3|samplepart4>
%\fi

% Optional override for |\version| flag:
%    \begin{macrocode}
%%\providecommand{\version}{final}
%    \end{macrocode}

% Include the main document:
%    \begin{macrocode}
\input{childdoc.def}
\childdocby{cdocsamp}
%    \end{macrocode}

%\iffalse
%</samplepart3|samplepart4>
%\fi
%
%\iffalse
%<*samplepart3>
%\fi
% Some text for part 3:
%    \begin{macrocode}
some text in part three
%    \end{macrocode}

%\iffalse
%</samplepart3>
%\fi
% Some text for part 4:
%\iffalse
%<*samplepart4>
%\fi
%    \begin{macrocode}
more text in part four
%    \end{macrocode}

%\iffalse
%</samplepart4>
%\fi
%
% %%%%%%%%%%%%%%%%%%%%%%%%%%%%%%%%%%%%%%
% \paragraph{Forwarding for a Complete Draft.}
%
% The following forwarding file |cdocsdrf.tex|
% compiles the main document in draft mode:
%\iffalse
%<*sampledraft>
%\fi
%    \begin{macrocode}
\def\version{draft}
\input{childdoc.def}
\childdocforward{cdocsamp}
%    \end{macrocode}

%\iffalse
%</sampledraft>
%\fi
%
% %%%%%%%%%%%%%%%%%%%%%%%%%%%%%%%%%%%%%%
% \paragraph{Forwarding for Final Version of the Chapters.}
%
% The following forwarding files |cdocsfn1.tex| and |cdocsfn2.tex|
% (with identical content)
% compile the final versions of the child documents
% |cdocsch1.tex| and |cdocsch2.tex|, respectively:
%\iffalse
%<*samplefinal>
%\fi
%    \begin{macrocode}
\def\version{final}
\input{childdoc.def}
\childdocforwardprefix[cdocsamp]{cdocsfn}{cdocsch}
%    \end{macrocode}

%\iffalse
%</samplefinal>
%\fi
%
% %%%%%%%%%%%%%%%%%%%%%%%%%%%%%%%%%%%%%%
% \paragraph{Command Line Processing.}
%
% The following three command lines generate the output files
% |cdocscld|, |cdocscl1| and |cdocscl2|
% which should be identical to
% |cdocsdrf|, |cdocsch1| and |cdocsfn2|, respectively:
% \begin{center}
% \begin{tabular}{l}
% |latex -jobname cdocscld \|\\
% |  "\def\version{draft}\input{childdoc.def}\childdocforward{cdocsamp}"|\\
% |latex -jobname cdocscl1 \|\\
% |  "\input{childdoc.def}\childdocforward[cdocsamp]{cdocsch1}"|\\
% |latex -jobname cdocscl2 \|\\
% |  "\def\version{final}\input{childdoc.def}\childdocforward{cdocsch2}"|
% \end{tabular}
% \end{center}
% Note that the trailing backslash on each first line
% merely continues the input to the second line
% (for convenient cut ant paste).
% Furthermore, the command |latex| can be replaced by any
% of its alternative versions such as |pdflatex|.
%
% %%%%%%%%%%%%%%%%%%%%%%%%%%%%%%%%%%%%%%%%%%%%%%%%%%%%%%%%%%%%%%%%%%%%%%%%%%%%%%
% %%%%%%%%%%%%%%%%%%%%%%%%%%%%%%%%%%%%%%%%%%%%%%%%%%%%%%%%%%%%%%%%%%%%%%%%%%%%%%
% \section{Implementation}
%\iffalse
%<*package>
%\fi
%
% This section describes the definitions file |childdoc.def|.

% The definitions cannot be loaded using |\usepackage| or |\RequirePackage|
% which has a mechanism to prevent loading a style file more than once.
% When loading the definitions by means of |\input|
% multiple instances have to be prevented manually:
%\iffalse
%This code needs to be before the `\ProvidesFile' directive
%which is defined at the beginning of this file.
%Therefore it is also placed there and commented out here.
%</package>
%<*discard>
%\fi
%    \begin{macrocode}
\ifdefined\childdocmain\endinput\fi
%    \end{macrocode}
%\iffalse
%</discard>
%<*package>
%\fi
%
% \macro{\ifchilddoc}
% \macro{\ifchilddocmanual}
% The conditional |\ifchilddoc| tells whether a
% child (true) or main (false) document is being compiled.
% The conditional |\ifchilddocmanual| tells whether
% the |\includeonly| mechanism is used (false) or
% the selection of child files must be performed manually (true).
% The definitions initialise to false:
%    \begin{macrocode}
\newif\ifchilddoc
\newif\ifchilddocmanual
%    \end{macrocode}

% \macro{\childdocname}
% \macro{\childdocjob}
% The macro |\childdocname| stores the name of the main document
% to be compiled. The macro |\childdocjob| stores the name of
% the document on which the \LaTeX{} compiler was originally invoked.
% The content of |\jobname| cannot be compared
% to filenames specified in the source due to different catcodes.
% The following code rescans |\jobname|, stores the result
% in |\childdocname| and saves a copy in |\childdocjob|:
%    \begin{macrocode}
\edef\childdocname{\scantokens\expandafter{\jobname\noexpand}}
\let\childdocjob\childdocname
%    \end{macrocode}

% \macro{\childdocdisable}
% The macro |\childdocdisable| prevents the main file
% from being processed more than once.
% At this stage, the main document command |\childdocmain|
% is assumed to be called once again where it should do nothing.
% Any subsequent call to it should prevent
% a secondary processing of the main document
% It overwrites the forwarding commands
% |\childdocof| and |\childdocforward|
% with empty macros to prevent further inclusions of the main document:
%    \begin{macrocode}
\newcommand{\childdocdisable}
{
  \renewcommand{\childdocmain}[1]{\renewcommand{\childdocmain}[1]{\endinput}}
  \renewcommand{\childdocof}[1]{}
  \renewcommand{\childdocby}[2][]{}
  \renewcommand{\childdocforward}[2][]{}
  \renewcommand{\childdocdisable}{}
}
%    \end{macrocode}

% \macro{\childdocmain}
% The macro |\childdocmain| is to be called at the top of the main file
% with nothing or the main filename (without extension) as argument.
% First, it breaks loops.
% If the argument is not empty and does not match |\childdocname|
% (which is set by the first inclusion of |childdoc.def|),
% |\ifchilddoc| is set to true, |\includeonly| is applied to the child file
% and |\jobname| is set to the main file
% (for proper handling of |.aux| files):
%    \begin{macrocode}
\newcommand{\childdocmain}[1]
{
  \childdocdisable\childdocmain{}
  \if?#1?\else
    \begingroup
      \def\childdoctmp{#1}
      \ifx\childdoctmp\childdocname
        \def\childdoctmp{}
      \else
        \def\childdoctmp
        {
          \childdoctrue
          \includeonly{\childdocname}
          \def\childdocjob{#1}
          \def\jobname{#1}
        }
      \fi
      \expandafter
    \endgroup
    \childdoctmp
  \fi
}
%    \end{macrocode}

% \macro{\childdocof}
% The command |\childdocof| redirects
% compilation to the main file |#1|.
%    \begin{macrocode}
\newcommand{\childdocof}[1]
{
  \childdocdisable
  \childdoctrue
  \includeonly{\childdocname}
  \def\jobname{#1}
  \def\childdocjob{#1}
  \input{#1}
}
%    \end{macrocode}

% \macro{\childdocby}
% The command |\childdocby| ....
%    \begin{macrocode}
\newcommand{\childdocby}[2][]
{
  \childdocdisable
  \childdoctrue
  \childdocmanualtrue
  \if?#1?\else
    \def\jobname{#2}
  \fi
  \def\childdocjob{#2}
  \input{#2}
  \endinput
}
%    \end{macrocode}

% \macro{\childdocforward}
% The command |\childdocforward| redirects
% compilation to the main file or
% (if the optional argument is given) a child file.
% Parameters are set as if the main file
% or a child file starting with |\childdocof| was compiled.
% Then compilation is handed over to the main file:
%    \begin{macrocode}
\newcommand{\childdocforward}[2][]
{
  \begingroup
    \if?#1?
      \def\childdoctmp
      {
        \def\childdocname{#2}
        \def\childdocjob{#2}
        \def\jobname{#2}
        \input{#2}
        \endinput
      }
    \else
      \def\childdoctmp
      {
        \childdocdisable
        \def\childdocname{#2}
        \childdoctrue
        \includeonly{#2}
        \def\childdocjob{#1}
        \def\jobname{#1}
        \input{#1}
        \endinput
      }
    \fi
    \expandafter
  \endgroup
  \childdoctmp
}
%    \end{macrocode}

% \macro{\childdocforwardprefix}
% The command |\childdocforwardprefix| redirects
% compilation to the main or a child file by means of a pattern.
% The prefix |#1| in the current filename is replaced by |#2|
% and the suffix of the current filename is kept
% (it is assumed that the filename does not contain the substring `|~~~|'
% which is used as a delimiter).
% Compilation is handed over to the new file by |\childdocforward|:
%    \begin{macrocode}
\newcommand{\childdocforwardprefix}[3][]
{
  \begingroup
    \def\childdocextract #2##1~~~{\def\childdoctmp{\childdocforward[#1]{#3##1}}}
    \expandafter\childdocextract\childdocname~~~
    \expandafter
  \endgroup
  \childdoctmp
}
%    \end{macrocode}

% \macro{\childdoc}
% The deprecated macro |\childdoc| is a legacy version of |\childdocmain|:
%    \begin{macrocode}
\newcommand{\childdoc}{\childdocmain}
%    \end{macrocode}

% \macro{\childdocredirect}
% The deprecated macro |\childdocredirect| is a legacy version
% of |\childdocforward| and |\childdocforwardprefix|:
%    \begin{macrocode}
\newcommand{\childdocredirect}[2][]
{
  \begingroup
    \if?#1?
      \def\childdoctmp{\childdocforward{#2}}
    \else
      \def\childdoctmp{\childdocforwardprefix{#1}{#2}}
    \fi
    \expandafter
  \endgroup
  \childdoctmp
}
%    \end{macrocode}

%\iffalse
%</package>
%\fi
%
\endinput

\childdocforwardprefix[cdocsamp]{cdocsfn}{cdocsch}
%    \end{macrocode}

%\iffalse
%</samplefinal>
%\fi
%
% %%%%%%%%%%%%%%%%%%%%%%%%%%%%%%%%%%%%%%
% \paragraph{Command Line Processing.}
%
% The following three command lines generate the output files
% |cdocscld|, |cdocscl1| and |cdocscl2|
% which should be identical to
% |cdocsdrf|, |cdocsch1| and |cdocsfn2|, respectively:
% \begin{center}
% \begin{tabular}{l}
% |latex -jobname cdocscld \|\\
% |  "\def\version{draft}% \iffalse
%
% childdoc.dtx Copyright (C) 2017-2018 Niklas Beisert
%
% This work may be distributed and/or modified under the
% conditions of the LaTeX Project Public License, either version 1.3
% of this license or (at your option) any later version.
% The latest version of this license is in
%   http://www.latex-project.org/lppl.txt
% and version 1.3 or later is part of all distributions of LaTeX
% version 2005/12/01 or later.
%
% This work has the LPPL maintenance status `maintained'.
%
% The Current Maintainer of this work is Niklas Beisert.
%
% This work consists of the files childdoc.dtx and childdoc.ins
% and the derived files childdoc.def and cdocsamp.tex with
% cdocsch1.tex, cdocsch2.tex, cdocsdrf.tex, cdocsfn1.tex, cdocsfn2.tex.
%
%<package>\ifdefined\childdocmain\endinput\fi
%<package>\ProvidesFile{childdoc.def}[2018/12/30 v2.0 child document driver]
%<samplemain>\ProvidesFile{cdocsamp.tex}[2018/12/30 v2.0 sample for childdoc]
%<*driver>
%\ProvidesFile{childdoc.drv}[2018/12/30 v2.0 childdoc reference manual file]
\PassOptionsToClass{10pt,a4paper}{article}
\documentclass{ltxdoc}

\usepackage[margin=35mm]{geometry}
\usepackage{hyperref}
\usepackage{hyperxmp}
\usepackage[usenames]{color}

\hypersetup{colorlinks=true}
\hypersetup{pdfstartview=FitH}
\hypersetup{pdfpagemode=UseNone}
\hypersetup{pdfsource={}}
\hypersetup{pdflang={en-UK}}
\hypersetup{pdfcopyright={Copyright 2017-2018 Niklas Beisert.
  This work may be distributed and/or modified under the
  conditions of the LaTeX Project Public License, either version 1.3
  of this license or (at your option) any later version.}}
\hypersetup{pdflicenseurl={http://www.latex-project.org/lppl.txt}}
\hypersetup{pdfcontactaddress={ETH Zurich, ITP, HIT K,
  Wolfgang-Pauli-Strasse 27}}
\hypersetup{pdfcontactpostcode={8093}}
\hypersetup{pdfcontactcity={Zurich}}
\hypersetup{pdfcontactcountry={Switzerland}}
\hypersetup{pdfcontactemail={nbeisert@itp.phys.ethz.ch}}
\hypersetup{pdfcontacturl={http://people.phys.ethz.ch/\xmptilde nbeisert/}}

\newcommand{\secref}[1]{\hyperref[#1]{section \ref*{#1}}}

\parskip1ex
\parindent0pt
\let\olditemize\itemize
\def\itemize{\olditemize\parskip0pt}

\begin{document}

\title{The \textsf{childdoc} Package}
\hypersetup{pdftitle={The childdoc Package}}
\author{Niklas Beisert\\[2ex]
  Institut f\"ur Theoretische Physik\\
  Eidgen\"ossische Technische Hochschule Z\"urich\\
  Wolfgang-Pauli-Strasse 27, 8093 Z\"urich, Switzerland\\[1ex]
  \href{mailto:nbeisert@itp.phys.ethz.ch}
  {\texttt{nbeisert@itp.phys.ethz.ch}}}
\hypersetup{pdfauthor={Niklas Beisert}}
\hypersetup{pdfsubject={Manual for the LaTeX2e Package childdoc}}
\date{30 December 2018, \textsf{v2.0}}
\maketitle

\begin{abstract}\noindent
\textsf{childdoc} is a \LaTeXe{} package
that enables the direct compilation
of document sections included by |\include|
to individual files.
\end{abstract}

\begingroup
\parskip0ex
\tableofcontents
\endgroup

%%%%%%%%%%%%%%%%%%%%%%%%%%%%%%%%%%%%%%%%%%%%%%%%%%%%%%%%%%%%%%%%%%%%%%%%%%%%%%%%
%%%%%%%%%%%%%%%%%%%%%%%%%%%%%%%%%%%%%%%%%%%%%%%%%%%%%%%%%%%%%%%%%%%%%%%%%%%%%%%%
\section{Introduction}

\LaTeX{} provides a mechanism to structure a large document (such as a book)
into a main file and several child files (containing the chapters)
using the |\include| command.
This mechanism is beneficial for documents
which span hundreds of pages in order to
make the source file(s) more manageable.
Moreover, compilation can be restricted to
selected child files by means of the |\includeonly| command.
The latter feature can be used to reduce the compilation time while editing
(this was significantly more useful in the earlier days of \LaTeX{})
or to generate a smaller document which is easier to navigate.
Another application of |\includeonly| is to generate
documents consisting of selected parts of the complete document.

However, there are a few drawbacks of the plain |\include| mechanism:
\begin{itemize}
\item
The child files cannot be compiled on their own,
they can only be compiled via the main file.
A naive editing environment
(such as a text editor with an option
to have the current file processed by \LaTeX)
may require one to switch to the main file before compiling;
attempting to compile the child file produces errors.
\item
The main file must be modified (each time)
to adjust the |\includeonly| command
to the present needs. This easily leaves the main file in a messy state.
\item
The generated document will always carry the filename
of the main document. This is inconvenient if
several child files are to be compiled and
to be kept for distribution.
\end{itemize}

The present package provides a simple interface
to make child files individually compilable by \LaTeX{}.
Compiling a child file then has the same effect as compiling
the main file with an |\includeonly| command
to select the appropriate child.
Moreover the generated document will carry the name of the child
rather than the main file.
This resolves all three above issues.

This feature is meant to make the editing of books,
thesis documents and lecture notes somewhat more convenient.
However, the package can also be used efficiently for
composing a series of documents (such as exercise sheets)
which are typically distributed individually.
It then assists the author in generating the individual documents
(potentially in different versions)
as well as a document containing the collected series.
Another application is in developing style files
or other kinds of included material
where compilation of the style file could redirect
to a sample or test file.

%%%%%%%%%%%%%%%%%%%%%%%%%%%%%%%%%%%%%%%%%%%%%%%%%%%%%%%%%%%%%%%%%%%%%%%%%%%%%%%%
%%%%%%%%%%%%%%%%%%%%%%%%%%%%%%%%%%%%%%%%%%%%%%%%%%%%%%%%%%%%%%%%%%%%%%%%%%%%%%%%
\section{Usage}

First of all, the package \textsf{childdoc} is \emph{not} a standard
\LaTeXe{} |.sty| style file! Therefore it needs to be invoked in
a non-standard way.

%%%%%%%%%%%%%%%%%%%%%%%%%%%%%%%%%%%%%%%%%%%%%%%%%%%%%%%%%%%%%%%%%%%%%%%%%%%%%%%%
\subsection{Included Files}
\label{sec:include}

%%%%%%%%%%%%%%%%%%%%%%%%%%%%%%%%%%%%%%%%
\DescribeMacro{\childdocmain}
To use the package, add the commands
\begin{center}
\begin{tabular}{l}
|\input{childdoc.def}|\\
|\childdocmain{}|\\
\end{tabular}
\end{center}
at the very top of the main \LaTeX{} file,
in particular \emph{before} the |\documentclass| statement!
The argument of |\childdocmain| should be left empty
(but it must be present).

%%%%%%%%%%%%%%%%%%%%%%%%%%%%%%%%%%%%%%%%
\DescribeMacro{\childdocof}
Furthermore, add the commands
\begin{center}
\begin{tabular}{l}
|\input{childdoc.def}|\\
|\childdocof{|\textit{main}|}|\\
\end{tabular}
\end{center}
at the top of every child file \textit{child}
which is included by |\include{|\textit{child}|}|
from within the main file
(or at least for those files to be compiled individually).
The argument \textit{main} must be the filename of the main file.

There are a couple of
considerations in setting up the main and child documents:

%%%%%%%%%%%%%%%%%%%%%%%%%%%%%%%%%%%%%%%%
\paragraph{Restrictions.}

Please note the following restrictions:
\begin{itemize}
\item
|\childdocmain| must be called with one argument \textit{main}
to ensure compatibility with earlier version of the package.
It must either be empty (|\childdocmain{}|)
or precisely match the filename of the main file in which it is specified.
See \secref{sec:detection} for further information.
\item
The filename \textit{main} must be specified without the |.tex| extension.
\item
The filename \textit{main} is case sensitive
(even in case-insensitive file systems)
due to internal string comparison.
\item
The argument \textit{main} should be fully expanded, it cannot be a macro.
\item
Subdirectories and special characters should be avoided in filenames.
\item
The command |\childdocmain{|\textit{main}|}| must be followed by a whitespace.
It should not be followed immediately by another command
or by a comment mark `|%|'.
This is because the \TeX{} parser reads the token immediately following
the argument of |\childdocmain| and puts it
at the beginning of every child section;
however, a white\-space is ignored.
\end{itemize}

%%%%%%%%%%%%%%%%%%%%%%%%%%%%%%%%%%%%%%%%
\paragraph{Content of Main File.}

It is advisable to place all content in the child files included by |\include|.
Any output contained in the main file will appear in all child documents
unless suppressed manually;
it cannot be suppressed automatically by the |\includeonly| directive
and thus should normally be avoided.
A method to include some content in the main file
by means of conditional processing is described in \secref{sec:conditional}.

%%%%%%%%%%%%%%%%%%%%%%%%%%%%%%%%%%%%%%%%
\paragraph{Page Numbering.}

When only a part of the document is compiled,
the appropriate numbering of pages
(as well as other status parameters)
is determined from the |.aux| files.
The latter contain information from previous passes.
However this information needs to propagate through
all intermediate child documents.
Therefore the page numbering in child documents may well
be inconsistent until the complete document is compiled at least once.

A useful (if unconventional) way to always ensure a consistent
page numbering is to restart the numbering in each child document
and denote the pages by `\textit{child}|.|\textit{page}'
where \textit{child} represents the chapter/section number of the child file.
This can be achieved by the command
|\numberwithin{page}{|\textit{child}|}|
of the \textsf{amsmath} package
where \textit{child} can be |chapter| or |section|
depending on the chosen structuring.
Alternatively, one can modify the macro |\thepage| appropriately
and reset the counter |page| at the start of each child file.

%%%%%%%%%%%%%%%%%%%%%%%%%%%%%%%%%%%%%%%%%%%%%%%%%%%%%%%%%%%%%%%%%%%%%%%%%%%%%%%%
\subsection{Conditional Processing}
\label{sec:conditional}

The package provides a mechanism to compile different versions
of a document. To customise the versions further some conditional processing
can come in handy to distinguish which version is being compiled.
The package provides two macros to describe the compilation context:

%%%%%%%%%%%%%%%%%%%%%%%%%%%%%%%%%%%%%%%%
\DescribeMacro{\ifchilddoc}
The conditional |\ifchilddoc| distinguishes between the compilation of
child documents and the main document:
%
\begin{center}
|\ifchilddoc |\textit{child-code}| |[|\||else |\textit{main-code}]| \||fi|
\end{center}

%%%%%%%%%%%%%%%%%%%%%%%%%%%%%%%%%%%%%%%%
\DescribeMacro{\childdocname}
\DescribeMacro{\childdocjob}
The macro |\childdocname| contains the filename (without extension)
of the main or child file being processed.
Note that |\childdocjob| will always contain the name of the main file.

%%%%%%%%%%%%%%%%%%%%%%%%%%%%%%%%%%%%%%%%
\paragraph{Title Page.}

Conditional processing can be used to include a title or banner page
in the main document when proper precautions are taken.
Importantly, the code in the main file should ensure that the page counter
(as well as other status parameters which are stored in the |.aux| files)
takes the same value after the conditional processing.
Otherwise the page numbers may take divergent values
depending on which part is compiled.

For example, a title page could be declared by:
%
\begin{center}
\begin{tabular}{l}
|\ifchilddoc\||else|\\
|\addtocounter{page}{-1}|\\
\textit{code for title page}\\
|\newpage|\\
|\||fi|
\end{tabular}
\end{center}
%
A banner page for the child documents can be generated by:
%
\begin{center}
\begin{tabular}{l}
|\ifchilddoc|\\
|\addtocounter{page}{-1}|\\
\textit{code for banner page}\\
|\newpage|\\
|\||fi|
\end{tabular}
\end{center}
%
Here one could write a message such as:
\begin{center}
|This is the part \childdocname{} of \childdocjob{}.|
\end{center}

%%%%%%%%%%%%%%%%%%%%%%%%%%%%%%%%%%%%%%%%%%%%%%%%%%%%%%%%%%%%%%%%%%%%%%%%%%%%%%%%
\subsection{Flags}
\label{sec:flags}

The package makes it easy to generate different versions
of the main or child documents.
To this end compilation flags can be defined
and assigned different default values.
They will be particularly useful in conjunction
with the forwarding mechanism described in \secref{sec:forward}.

For example, it may be useful to have a flag |\version|
which can be set to |draft| or |final|.
The document source will contain some conditional code
depending on the value of |\version|.
Suppose further, the flag should default to |final| for the main file
and to |draft| for child files
which is a natural assignment for editing the document.
This is achieved by placing the following code
in the preamble of the main document
(below the |\childdocmain| directive):
%
\begin{center}
\begin{tabular}{l}
|\ifchilddoc|\\
|\providecommand{\version}{draft}|\\
|\||else|\\
|\providecommand{\version}{final}|\\
|\||fi|
\end{tabular}
\end{center}
%
The definition by |\providecommand| makes sure
that previous definitions are not overwritten.
Further statements |\providecommand{\version}{...}|
can thus be added before the above code to override it.

For the main file, one might add a line
(between |\childdocmain| and the above block)
%
\begin{center}
|%\ifchilddoc\||else\providecommand{\version}{draft}\||fi|
\end{center}
%
which can be uncommented to produce a draft version.
Likewise one can add a line to the very top of a child file
(above the |\childdocof{|\textit{main}|}| directive)
%
\begin{center}
|%\providecommand{\version}{final}|
\end{center}
%
which can be uncommented to produce the final version of this child document.

%%%%%%%%%%%%%%%%%%%%%%%%%%%%%%%%%%%%%%%%%%%%%%%%%%%%%%%%%%%%%%%%%%%%%%%%%%%%%%%%
\subsection{Forwarding}
\label{sec:forward}

Different versions of the main or child documents
using compilation flags as described in \secref{sec:flags}
can be (permanently) stored in different files
for convenient compilation, viewing and distribution.
To this end, the package defines a command
to pass on compilation to a different file:

%%%%%%%%%%%%%%%%%%%%%%%%%%%%%%%%%%%%%%%%
\DescribeMacro{\childdocforward}
The command |\childdocforward| redirects processing to
another source file:
%
\begin{center}
\begin{tabular}{l}
|\input{childdoc.def}|\\
|\childdocforward[|\textit{main}|]{|\textit{dest}|}|\\
\end{tabular}
\end{center}
%
The argument \textit{dest} is the destination file
(without extension).
It should be the main file or one of the child files.
Note that further \textsf{childdoc} directives
such as |\childdocof| and |\childdocforward|
in the indicated file will be processed in this form.
The optional argument \textit{main}
passes on directly to the main file \textit{main}
while pretending to compile the child \textit{dest}.
This form behaves as if \textit{dest}
issues |\childdocof{|\textit{main}|}| right away,
and no further \textsf{childdoc} directives will be processed.

%%%%%%%%%%%%%%%%%%%%%%%%%%%%%%%%%%%%%%%%
\DescribeMacro{\...prefix}
In the alternative form |\childdocforwardprefix|,
%
\begin{center}
\begin{tabular}{l}
|\input{childdoc.def}|\\
|\childdocforwardprefix[|\textit{main}|]{|\textit{prefix}|}{|\textit{dest}|}|
\end{tabular}
\end{center}
%
the destination file is determined by a pattern
depending on the current file:
To make this work, the current file must be called
`{\textit{prefix}\hspace{0.2em}\textit{suffix}}'
with \textit{prefix} matching precisely the argument.
Processing is then passed on to the file
`{\textit{dest}\hspace{0.2em}\textit{suffix}}'.
Surely, the same effect is achieved by
directly specifying the
argument `{\textit{dest}\hspace{0.2em}\textit{suffix}}'
in the first form.
However, that requires to set up a different file
for each child. With the alternative form of the command
all these files can have exactly the same content
which simplifies setting them up and maintaining them.

For example, the following file |draft.tex|
with a compilation flag |\version| as described in \secref{sec:flags}
compiles the main document as a draft:
%
\begin{center}
\begin{tabular}{l}
|\def\version{draft}|\\
|\input{childdoc.def}|\\
|\childdocforward{|\textit{main}|}|
\end{tabular}
\end{center}
%
Likewise, the following files |final|\textit{nn}|.tex|
compile the final version of the child document
|child|\textit{nn}|.tex|:
%
\begin{center}
\begin{tabular}{l}
|\def\version{final}|\\
|\input{childdoc.def}|\\
|\childdocforwardprefix{final}{child}|
\end{tabular}
\end{center}
%

Note that when several versions of a main file and/or of each child file
are to be generated, it may be convenient to set up a |Makefile| or
shell script to automatise the process.

%%%%%%%%%%%%%%%%%%%%%%%%%%%%%%%%%%%%%%%%%%%%%%%%%%%%%%%%%%%%%%%%%%%%%%%%%%%%%%%%
\subsection{Command Line Processing}
\label{sec:commandline}

The effect of redirection files can also be achieved by invoking
the \LaTeX{} compiler with a more elaborate command line.
Most conveniently this should be done as part
of a shell script or a |Makefile|.

When using \textsf{childdoc} in the main file, the following
command lines effectively perform a redirection
(note that depending on the shell being used,
backslashes may have to be doubled: `|\|' $\to$ `|\\|'):
%
\begin{center}
|... -jobname "|\textit{target}|" |\\|"|[\textit{flags}]%
|\input{childdoc.def}\childdocforward[|\textit{main}|]{|\textit{dest}|}"|
\end{center}
%
Here \textit{target} is the name of the output file,
\textit{main} is the name of the main file
and \textit{dest} is the name of the main or child file to be processed
(all filenames without extensions).
The optional argument \textit{main} can be omitted
if \textit{main} matches \textit{dest}.
Optionally, compilation \textit{flags} can be defined via |\def| commands.
This command line makes the \TeX{} engine believe
it is compiling the file \textit{target}
whose content is specified as the latter parameter.
The provided code then forwards the processing to
\textit{main} or \textit{dest} as described in \secref{sec:forward}.

%%%%%%%%%%%%%%%%%%%%%%%%%%%%%%%%%%%%%%%%%%%%%%%%%%%%%%%%%%%%%%%%%%%%%%%%%%%%%%%%
\subsection{Include by Input}
\label{sec:input}

Including child documents by |\include| has some restrictions by design.
Most notably, the content of a child document always occupies
its own set of pages; pages cannot be shared between child documents.
Usually, this behaviour makes perfect sense
because each child document contain an essential part of the document.
However, in some situations it may be desirable to compose
a document from a collection of parts
without having mandatory page breaks between then.
For this case, the package
provides a mechanism to include parts
by |\input| which can also be processed individually.
However, by construction this mechanism
requires manual handling of the content to be output.

%%%%%%%%%%%%%%%%%%%%%%%%%%%%%%%%%%%%%%%%
\DescribeMacro{\ifchilddocmanual}
The main file should be prepared as usual, see \secref{sec:include}.
However, the document body must make a distinction
between processing of an individual part and of the main document, e.g.:
%
\begin{center}
\begin{tabular}{l}
|\ifchilddocmanual|\\
|\input{\childdocname}|\\
|\||else|\\
\textit{document body with }|\input{|\textit{part}|}|\\
|\||fi|
\end{tabular}
\end{center}
%
The conditional |\ifchilddocmanual| is true whenever
a part to be included by |\input| is being compiled,
and the name of the part is stored in |\childdocname|.

%%%%%%%%%%%%%%%%%%%%%%%%%%%%%%%%%%%%%%%%
\DescribeMacro{\childdocby}
Each part to be included by |\input| should start with:
%
\begin{center}
\begin{tabular}{l}
|\input{childdoc.def}|\\
|\childdocby{|\textit{main}|}|\\
\end{tabular}
\end{center}
%
The directive |\childdocby| is similar to |\childdocof|
described in \secref{sec:include},
but the subsequent selection of content must be done manually.
To that end, both |\ifchilddoc| and |\ifchilddocmanual|
will be true upon processing of a part,
and the name of the part is stored in |\childdocname|.
Note that |\jobname| will be set to the filename of the current part
so that each part receives an individual |.aux| file
that does not interfere with the |.aux| file(s) of the main document.
This behaviour can be altered by the alternative form
|\childdocby[*]{|\textit{main}|}| (with a non-empty optional argument)
which uses the |.aux| file of the main document
by setting |\jobname| to \textit{main}.

%%%%%%%%%%%%%%%%%%%%%%%%%%%%%%%%%%%%%%%%%%%%%%%%%%%%%%%%%%%%%%%%%%%%%%%%%%%%%%%%
\subsection{Driver Development}
\label{sec:driver}

The \textsf{childdoc} mechanism can also be use for the development
of definition files such as \LaTeX{} styles or classes.
This case differs from the above setup with multiple parts
included by |\include| in that no |\includeonly| should be invoked.
This can be achieved by starting the include file
(before |\ProvidesPackage|) with:
%
\begin{center}
\begin{tabular}{l}
|\input{childdoc.def}|\\
|\childdocforward{|\textit{main}|}|\\
\end{tabular}
\end{center}
%
or alternatively with:
%
\begin{center}
\begin{tabular}{l}
|\input{childdoc.def}|\\
|\childdocby{|\textit{main}|}|\\
\end{tabular}
\end{center}
%
Both forms have slightly different effects as described above.
The main file is prepared as usual, see \secref{sec:include}.

%%%%%%%%%%%%%%%%%%%%%%%%%%%%%%%%%%%%%%%%%%%%%%%%%%%%%%%%%%%%%%%%%%%%%%%%%%%%%%%%
\subsection{Legacy Detection}
\label{sec:detection}

The directive |\childdocmain| in the main file can detect
whether the complete document or merely a child is to be compiled
even without using the directive |\childdocof|.
This method is deprecated because it is less robust
and there is no compelling reason to use it;
it is merely provided for backward compatibility
and it may be removed in future versions.

If the detection mechanism is to be used,
it is mandatory to correctly specify
the filename of the main file as the argument of |\childdocmain|:
%
\begin{center}
\begin{tabular}{l}
|\input{childdoc.def}|\\
|\childdocmain{|\textit{main}|}|\\
\end{tabular}
\end{center}
%
If |\jobname| does not match the argument \textit{main} of |\childdocmain|,
it is assumed that |\jobname| points to the child file to be compiled.
When using |\childdocmain| with the main file specified as argument,
it suffices to start a child file
with just |\input{|\textit{main}|}|
without loading of the package and using |\childdocof|.
If instead all processing is done
with the appropriate \textsf{childdoc} directives,
the argument of \textit{main} of |\childdocmain| can be empty.

An alternative version of the command line processing described
in \secref{sec:commandline} using the detection mechanism reads:
%
\begin{center}
|... -jobname "|\textit{target}|" "|[\textit{flags}]%
[|\def\jobname{|\textit{dest}|}|]|\input{|\textit{main}|}"|
\end{center}

%%%%%%%%%%%%%%%%%%%%%%%%%%%%%%%%%%%%%%%%%%%%%%%%%%%%%%%%%%%%%%%%%%%%%%%%%%%%%%%%
\subsection{Manual Code}
\label{sec:manual}

In case one cannot be certain whether the definitions file |childdoc.def|
is installed on the target \TeX{} distribution
and one prefers not to ship it,
it is conceivable to paste a few relevant commands into the sources.

To that end, drop all statements |\input{childdoc.def}|
and perform the replacements as outlined below.
Instead of |\childdocmain{|\textit{main}|}| add the following code
to the top of the main file:
%
\begin{center}
\begin{tabular}{l}
|\||ifdefined\childdocname\endinput\||fi\newif\ifchilddoc|\\
|\edef\childdocname{\scantokens\expandafter{\jobname\noexpand}}|\\
|\def\childdocmain{|\textit{main}|}\||ifx\childdocmain\childdocname\||else|\\
|\childdoctrue\includeonly{\childdocname}\let\jobname\childdocmain\||fi|\\
\end{tabular}
\end{center}
%
Instead of |\childdocof{|\textit{main}|}| just include the main file
at the top of each child file:
%
\begin{center}
|\input{|\textit{main}|}|
\end{center}
%
A simple redirection |\childdocforward{|\textit{dest}|}| is achieved by:
%
\begin{center}
|\def\jobname{|\textit{dest}|}\input{\jobname}|
\end{center}
%
The redirection with prefix
|\childdocforwardprefix[|\textit{prefix}|]{|\textit{dest}|}|
is accomplished by:
%
\begin{center}
\begin{tabular}{l}
|{\edef\jobname{\scantokens\expandafter{\jobname\noexpand}}|\\
|\def\redirectjob |\textit{prefix}|#1~~~{\gdef\jobname{|\textit{dest}|#1}}|\\
|\expandafter\redirectjob\jobname~~~}\input{\jobname}|
\end{tabular}
\end{center}

In an alternative approach,
child documents can be compiled by a specific command line
without additional code or specific definitions:
%
\begin{center}
|... -jobname "|\textit{target}|" "|[\textit{flags}]%
|\includeonly{|\textit{dest}|}\input{|\textit{main}|}"|
\end{center}
%

%%%%%%%%%%%%%%%%%%%%%%%%%%%%%%%%%%%%%%%%%%%%%%%%%%%%%%%%%%%%%%%%%%%%%%%%%%%%%%%%
%%%%%%%%%%%%%%%%%%%%%%%%%%%%%%%%%%%%%%%%%%%%%%%%%%%%%%%%%%%%%%%%%%%%%%%%%%%%%%%%
\section{Information}

%%%%%%%%%%%%%%%%%%%%%%%%%%%%%%%%%%%%%%%%%%%%%%%%%%%%%%%%%%%%%%%%%%%%%%%%%%%%%%%%
\subsection{Copyright}

Copyright \copyright{} 2017--2018 Niklas Beisert

This work may be distributed and/or modified under the
conditions of the \LaTeX{} Project Public License, either version 1.3
of this license or (at your option) any later version.
The latest version of this license is in
  \url{http://www.latex-project.org/lppl.txt}
and version 1.3 or later is part of all distributions of \LaTeX{}
version 2005/12/01 or later.

This work has the LPPL maintenance status `maintained'.

The Current Maintainer of this work is Niklas Beisert.

This work consists of the files |README.txt|, |childdoc.ins| and |childdoc.dtx|
as well as the derived files |childdoc.def|, |cdocsamp.tex|
with |cdocsch1.tex|, |cdocsch2.tex|, |cdocspt3.tex|, |cdocspt4.tex|,
|cdocsdrf.tex|, |cdocsfn1.tex|, |cdocsfn2.tex|
as well as |childdoc.pdf|.

%%%%%%%%%%%%%%%%%%%%%%%%%%%%%%%%%%%%%%%%%%%%%%%%%%%%%%%%%%%%%%%%%%%%%%%%%%%%%%%%
\subsection{Files and Installation}

The package consists of the files:
%
\begin{center}
\begin{tabular}{ll}
    |README.txt|   & readme file \\
    |childdoc.ins| & installation file \\
    |childdoc.dtx| & source file \\
    |childdoc.def| & definition file \\
    |cdocsamp.tex| & sample main file \\
    |cdocsch1.tex| & sample include file \\
    |cdocsch2.tex| & sample include file \\
    |cdocspt3.tex| & sample part file \\
    |cdocspt4.tex| & sample part file \\
    |cdocsdrf.tex| & sample redirection file \\
    |cdocsfn1.tex| & sample redirection file \\
    |cdocsfn2.tex| & sample redirection file \\
    |childdoc.pdf| & manual
\end{tabular}
\end{center}
%
The distribution consists of the files
|README.txt|, |childdoc.ins| and |childdoc.dtx|.
%
\begin{itemize}
\item
Run (pdf)\LaTeX{} on |childdoc.dtx|
to compile the manual |childdoc.pdf| (this file).
\item
Run \LaTeX{} on |childdoc.ins| to create the definitions file |childdoc.def|
and the sample |cdocsamp.tex| with include files
|cdocsch1.tex|, |cdocsch2.tex|, |cdocspt3.tex|, |cdocspt4.tex|,
|cdocsdrf.tex|, |cdocsfn1.tex|, |cdocsfn2.tex|.
Then copy the file |childdoc.def| to an appropriate directory of your \LaTeX{}
distribution, e.g.\ \textit{texmf-root}|/tex/latex/childdoc|.
\end{itemize}

%%%%%%%%%%%%%%%%%%%%%%%%%%%%%%%%%%%%%%%%%%%%%%%%%%%%%%%%%%%%%%%%%%%%%%%%%%%%%%%%
\subsection{Related CTAN Packages}

There are several other packages which offer a similar functionality:
%
\begin{itemize}
\item
The packages
\href{http://ctan.org/pkg/docmute}{\textsf{docmute}},
\href{http://ctan.org/pkg/includex}{\textsf{includex}} and
\href{http://ctan.org/pkg/standalone}{\textsf{standalone}}
provide commands to include only the document body of
a child file thus allowing both files to be compiled individually.
\item
The packages \href{http://ctan.org/pkg/subdocs}{\textsf{subdocs}}
and \href{http://ctan.org/pkg/subfiles}{\textsf{subfiles}}
provide structures in which the main and child documents can be
encapsulated and allowing them to be compiled individually.
The inclusion mechanism is different from the conventional |\include|.
\item
The package \href{http://ctan.org/pkg/combine}{\textsf{combine}}
is an elaborate solution to combine several documents into one.
\end{itemize}
%
See also the CTAN topic \href{http://ctan.org/topic/subdocs}{\textsf{subdocs}}
for further related packages.
The present package differs from the above solutions in that
a document structure constructed with the conventional |\include| mechanism
just needs two extra commands at the top of every file
such that all constituent files can be compiled individually.

%%%%%%%%%%%%%%%%%%%%%%%%%%%%%%%%%%%%%%%%%%%%%%%%%%%%%%%%%%%%%%%%%%%%%%%%%%%%%%%%
%\subsection{Feature Suggestions}
%
%The following is a list of features which may be useful for future
%versions of this package:
%%
%\begin{itemize}
%\item
%\ldots
%\end{itemize}

%%%%%%%%%%%%%%%%%%%%%%%%%%%%%%%%%%%%%%%%%%%%%%%%%%%%%%%%%%%%%%%%%%%%%%%%%%%%%%%%
\subsection{Revision History}

%%%%%%%%%%%%%%%%%%%%%%%%%%%%%%%%%%%%%%%%
\paragraph{v2.0:} 2018/12/30

\begin{itemize}
\item
immediate forward processing
\item
added |\childdocby| mechanism
\item
manual restructured
\end{itemize}

%%%%%%%%%%%%%%%%%%%%%%%%%%%%%%%%%%%%%%%%
\paragraph{v1.6:} 2018/01/17

\begin{itemize}
\item
application for development of include files
\item
corrections to manual
\end{itemize}

%%%%%%%%%%%%%%%%%%%%%%%%%%%%%%%%%%%%%%%%
\paragraph{v1.5:} 2017/05/21

\begin{itemize}
\item
more complete structuring introduced
\item
|\childdocof| introduced
\item
|\childdoc| renamed to |\childdocmain|
\item
|\childredirect| renamed to |\childdocforward| and |\childdocforwardprefix|
and functionality expanded
\end{itemize}

%%%%%%%%%%%%%%%%%%%%%%%%%%%%%%%%%%%%%%%%
\paragraph{v1.0:} 2017/04/27

\begin{itemize}
\item
manual and install package
\item
first version published on CTAN
\end{itemize}

%%%%%%%%%%%%%%%%%%%%%%%%%%%%%%%%%%%%%%%%
\paragraph{v0.6:} 2017/04/26

\begin{itemize}
\item
redirection mechanism added
\end{itemize}

%%%%%%%%%%%%%%%%%%%%%%%%%%%%%%%%%%%%%%%%
\paragraph{v0.5:} 2017/04/26

\begin{itemize}
\item
functionality in definition file
\end{itemize}


%%%%%%%%%%%%%%%%%%%%%%%%%%%%%%%%%%%%%%%%%%%%%%%%%%%%%%%%%%%%%%%%%%%%%%%%%%%%%%%%
%%%%%%%%%%%%%%%%%%%%%%%%%%%%%%%%%%%%%%%%%%%%%%%%%%%%%%%%%%%%%%%%%%%%%%%%%%%%%%%%
%%%%%%%%%%%%%%%%%%%%%%%%%%%%%%%%%%%%%%%%%%%%%%%%%%%%%%%%%%%%%%%%%%%%%%%%%%%%%%%%
\appendix

\settowidth\MacroIndent{\rmfamily\scriptsize 000\ }

 \DocInput{childdoc.dtx}

\end{document}
%</driver>
% \fi
%
% %%%%%%%%%%%%%%%%%%%%%%%%%%%%%%%%%%%%%%%%%%%%%%%%%%%%%%%%%%%%%%%%%%%%%%%%%%%%%%
% %%%%%%%%%%%%%%%%%%%%%%%%%%%%%%%%%%%%%%%%%%%%%%%%%%%%%%%%%%%%%%%%%%%%%%%%%%%%%%
% \section{Sample}
%\iffalse
%<*samplemain>
%\fi
%
% The following presents a sample document
% with two chapters, two parts, a title page,
% a compile flag as well as three forwarding files to set the flag.
% It consists of eight |.tex| files:
% \begin{center}
% \begin{tabular}{ll}
% |cdocsamp.tex|&main file\\
% |cdocsch1.tex|&include file for chapter 1\\
% |cdocsch2.tex|&include file for chapter 2\\
% |cdocspt3.tex|&include file for part 3\\
% |cdocspt4.tex|&include file for part 4\\
% |cdocsdrf.tex|&forwarding file for main file in draft mode\\
% |cdocsfi1.tex|&forwarding file for final version of chapter 1\\
% |cdocsfi2.tex|&forwarding file for final version of chapter 2\\
% \end{tabular}
% \end{center}
% Each of the eight files can be compiled directly by the \LaTeX{} compiler.
%
% %%%%%%%%%%%%%%%%%%%%%%%%%%%%%%%%%%%%%%
% \paragraph{Main File.}
%
% The main file is called |cdocsamp.tex|.
%
% Load the \textsf{childdoc} definitions and
% declare the filename for the main document:
%    \begin{macrocode}
\input{childdoc.def}
\childdocmain{}
%    \end{macrocode}

% Optional override for |\version| flag:
%    \begin{macrocode}
%%\ifchilddoc\else\providecommand{\version}{draft}\fi
%    \end{macrocode}

% Define the default values for the |\version| flag
% (|final| for the main file and |draft| for childs):
%    \begin{macrocode}
\ifchilddoc
\providecommand{\version}{draft}
\else
\providecommand{\version}{final}
\fi
%    \end{macrocode}

% Load the standard document class:
%    \begin{macrocode}
\documentclass[12pt]{article}
%    \end{macrocode}

% Start the document body:
%    \begin{macrocode}
\begin{document}
%    \end{macrocode}

% Declare a title page.
% Print title, part of document being processed and version flag:
%    \begin{macrocode}
\addtocounter{page}{-1}
\begin{center}
{\LARGE\bfseries{}childdoc example\par}
\vspace{1cm}
\ifchilddoc
\ifchilddocmanual part\else chapter\fi:
`\childdocname' of `\childdocjob'\par
\else
main document: `\childdocjob'\par
\fi
version: \version\par
\end{center}
\newpage
%    \end{macrocode}

% Manually include selected file,
% otherwise process as usual:
%    \begin{macrocode}
\ifchilddocmanual
\section*{part `\childdocname'}
\input{\childdocname}
\else
%    \end{macrocode}

% Include the two chapters:
%    \begin{macrocode}
\include{cdocsch1}
\include{cdocsch2}
%    \end{macrocode}

% Include the two parts unless only chapters should be displayed:
%    \begin{macrocode}
\ifchilddoc\else
\section{part three}
\input{cdocspt3}
\section{part four}
\input{cdocspt4}
\fi
%    \end{macrocode}

% Process as usual until here:
%    \begin{macrocode}
\fi
%    \end{macrocode}

% End of document body:
%    \begin{macrocode}
\end{document}
%    \end{macrocode}
%\iffalse
%</samplemain>
%\fi
%
% %%%%%%%%%%%%%%%%%%%%%%%%%%%%%%%%%%%%%%
% \paragraph{Chapter Include Files.}
%
% The include files are called |cdocsch1.tex| and |cdocsch2.tex|.
%
%\iffalse
%<*samplechap1|samplechap2>
%\fi

% Optional override for |\version| flag:
%    \begin{macrocode}
%%\providecommand{\version}{final}
%    \end{macrocode}

% Include the main document:
%    \begin{macrocode}
\input{childdoc.def}
\childdocof{cdocsamp}
%    \end{macrocode}

%\iffalse
%</samplechap1|samplechap2>
%\fi
%
%\iffalse
%<*samplechap1>
%\fi
% Some text for chapter 1:
%    \begin{macrocode}
\section{one}
some text in chapter one
%    \end{macrocode}

%\iffalse
%</samplechap1>
%\fi
% Some text for chapter 2:
%\iffalse
%<*samplechap2>
%\fi
%    \begin{macrocode}
\section{two}
more text in chapter two
%    \end{macrocode}

%\iffalse
%</samplechap2>
%\fi
%
% %%%%%%%%%%%%%%%%%%%%%%%%%%%%%%%%%%%%%%
% \paragraph{Part Include Files.}
%
% The include files are called |cdocspt3.tex| and |cdocspt4.tex|.
%
%\iffalse
%<*samplepart3|samplepart4>
%\fi

% Optional override for |\version| flag:
%    \begin{macrocode}
%%\providecommand{\version}{final}
%    \end{macrocode}

% Include the main document:
%    \begin{macrocode}
\input{childdoc.def}
\childdocby{cdocsamp}
%    \end{macrocode}

%\iffalse
%</samplepart3|samplepart4>
%\fi
%
%\iffalse
%<*samplepart3>
%\fi
% Some text for part 3:
%    \begin{macrocode}
some text in part three
%    \end{macrocode}

%\iffalse
%</samplepart3>
%\fi
% Some text for part 4:
%\iffalse
%<*samplepart4>
%\fi
%    \begin{macrocode}
more text in part four
%    \end{macrocode}

%\iffalse
%</samplepart4>
%\fi
%
% %%%%%%%%%%%%%%%%%%%%%%%%%%%%%%%%%%%%%%
% \paragraph{Forwarding for a Complete Draft.}
%
% The following forwarding file |cdocsdrf.tex|
% compiles the main document in draft mode:
%\iffalse
%<*sampledraft>
%\fi
%    \begin{macrocode}
\def\version{draft}
\input{childdoc.def}
\childdocforward{cdocsamp}
%    \end{macrocode}

%\iffalse
%</sampledraft>
%\fi
%
% %%%%%%%%%%%%%%%%%%%%%%%%%%%%%%%%%%%%%%
% \paragraph{Forwarding for Final Version of the Chapters.}
%
% The following forwarding files |cdocsfn1.tex| and |cdocsfn2.tex|
% (with identical content)
% compile the final versions of the child documents
% |cdocsch1.tex| and |cdocsch2.tex|, respectively:
%\iffalse
%<*samplefinal>
%\fi
%    \begin{macrocode}
\def\version{final}
\input{childdoc.def}
\childdocforwardprefix[cdocsamp]{cdocsfn}{cdocsch}
%    \end{macrocode}

%\iffalse
%</samplefinal>
%\fi
%
% %%%%%%%%%%%%%%%%%%%%%%%%%%%%%%%%%%%%%%
% \paragraph{Command Line Processing.}
%
% The following three command lines generate the output files
% |cdocscld|, |cdocscl1| and |cdocscl2|
% which should be identical to
% |cdocsdrf|, |cdocsch1| and |cdocsfn2|, respectively:
% \begin{center}
% \begin{tabular}{l}
% |latex -jobname cdocscld \|\\
% |  "\def\version{draft}\input{childdoc.def}\childdocforward{cdocsamp}"|\\
% |latex -jobname cdocscl1 \|\\
% |  "\input{childdoc.def}\childdocforward[cdocsamp]{cdocsch1}"|\\
% |latex -jobname cdocscl2 \|\\
% |  "\def\version{final}\input{childdoc.def}\childdocforward{cdocsch2}"|
% \end{tabular}
% \end{center}
% Note that the trailing backslash on each first line
% merely continues the input to the second line
% (for convenient cut ant paste).
% Furthermore, the command |latex| can be replaced by any
% of its alternative versions such as |pdflatex|.
%
% %%%%%%%%%%%%%%%%%%%%%%%%%%%%%%%%%%%%%%%%%%%%%%%%%%%%%%%%%%%%%%%%%%%%%%%%%%%%%%
% %%%%%%%%%%%%%%%%%%%%%%%%%%%%%%%%%%%%%%%%%%%%%%%%%%%%%%%%%%%%%%%%%%%%%%%%%%%%%%
% \section{Implementation}
%\iffalse
%<*package>
%\fi
%
% This section describes the definitions file |childdoc.def|.

% The definitions cannot be loaded using |\usepackage| or |\RequirePackage|
% which has a mechanism to prevent loading a style file more than once.
% When loading the definitions by means of |\input|
% multiple instances have to be prevented manually:
%\iffalse
%This code needs to be before the `\ProvidesFile' directive
%which is defined at the beginning of this file.
%Therefore it is also placed there and commented out here.
%</package>
%<*discard>
%\fi
%    \begin{macrocode}
\ifdefined\childdocmain\endinput\fi
%    \end{macrocode}
%\iffalse
%</discard>
%<*package>
%\fi
%
% \macro{\ifchilddoc}
% \macro{\ifchilddocmanual}
% The conditional |\ifchilddoc| tells whether a
% child (true) or main (false) document is being compiled.
% The conditional |\ifchilddocmanual| tells whether
% the |\includeonly| mechanism is used (false) or
% the selection of child files must be performed manually (true).
% The definitions initialise to false:
%    \begin{macrocode}
\newif\ifchilddoc
\newif\ifchilddocmanual
%    \end{macrocode}

% \macro{\childdocname}
% \macro{\childdocjob}
% The macro |\childdocname| stores the name of the main document
% to be compiled. The macro |\childdocjob| stores the name of
% the document on which the \LaTeX{} compiler was originally invoked.
% The content of |\jobname| cannot be compared
% to filenames specified in the source due to different catcodes.
% The following code rescans |\jobname|, stores the result
% in |\childdocname| and saves a copy in |\childdocjob|:
%    \begin{macrocode}
\edef\childdocname{\scantokens\expandafter{\jobname\noexpand}}
\let\childdocjob\childdocname
%    \end{macrocode}

% \macro{\childdocdisable}
% The macro |\childdocdisable| prevents the main file
% from being processed more than once.
% At this stage, the main document command |\childdocmain|
% is assumed to be called once again where it should do nothing.
% Any subsequent call to it should prevent
% a secondary processing of the main document
% It overwrites the forwarding commands
% |\childdocof| and |\childdocforward|
% with empty macros to prevent further inclusions of the main document:
%    \begin{macrocode}
\newcommand{\childdocdisable}
{
  \renewcommand{\childdocmain}[1]{\renewcommand{\childdocmain}[1]{\endinput}}
  \renewcommand{\childdocof}[1]{}
  \renewcommand{\childdocby}[2][]{}
  \renewcommand{\childdocforward}[2][]{}
  \renewcommand{\childdocdisable}{}
}
%    \end{macrocode}

% \macro{\childdocmain}
% The macro |\childdocmain| is to be called at the top of the main file
% with nothing or the main filename (without extension) as argument.
% First, it breaks loops.
% If the argument is not empty and does not match |\childdocname|
% (which is set by the first inclusion of |childdoc.def|),
% |\ifchilddoc| is set to true, |\includeonly| is applied to the child file
% and |\jobname| is set to the main file
% (for proper handling of |.aux| files):
%    \begin{macrocode}
\newcommand{\childdocmain}[1]
{
  \childdocdisable\childdocmain{}
  \if?#1?\else
    \begingroup
      \def\childdoctmp{#1}
      \ifx\childdoctmp\childdocname
        \def\childdoctmp{}
      \else
        \def\childdoctmp
        {
          \childdoctrue
          \includeonly{\childdocname}
          \def\childdocjob{#1}
          \def\jobname{#1}
        }
      \fi
      \expandafter
    \endgroup
    \childdoctmp
  \fi
}
%    \end{macrocode}

% \macro{\childdocof}
% The command |\childdocof| redirects
% compilation to the main file |#1|.
%    \begin{macrocode}
\newcommand{\childdocof}[1]
{
  \childdocdisable
  \childdoctrue
  \includeonly{\childdocname}
  \def\jobname{#1}
  \def\childdocjob{#1}
  \input{#1}
}
%    \end{macrocode}

% \macro{\childdocby}
% The command |\childdocby| ....
%    \begin{macrocode}
\newcommand{\childdocby}[2][]
{
  \childdocdisable
  \childdoctrue
  \childdocmanualtrue
  \if?#1?\else
    \def\jobname{#2}
  \fi
  \def\childdocjob{#2}
  \input{#2}
  \endinput
}
%    \end{macrocode}

% \macro{\childdocforward}
% The command |\childdocforward| redirects
% compilation to the main file or
% (if the optional argument is given) a child file.
% Parameters are set as if the main file
% or a child file starting with |\childdocof| was compiled.
% Then compilation is handed over to the main file:
%    \begin{macrocode}
\newcommand{\childdocforward}[2][]
{
  \begingroup
    \if?#1?
      \def\childdoctmp
      {
        \def\childdocname{#2}
        \def\childdocjob{#2}
        \def\jobname{#2}
        \input{#2}
        \endinput
      }
    \else
      \def\childdoctmp
      {
        \childdocdisable
        \def\childdocname{#2}
        \childdoctrue
        \includeonly{#2}
        \def\childdocjob{#1}
        \def\jobname{#1}
        \input{#1}
        \endinput
      }
    \fi
    \expandafter
  \endgroup
  \childdoctmp
}
%    \end{macrocode}

% \macro{\childdocforwardprefix}
% The command |\childdocforwardprefix| redirects
% compilation to the main or a child file by means of a pattern.
% The prefix |#1| in the current filename is replaced by |#2|
% and the suffix of the current filename is kept
% (it is assumed that the filename does not contain the substring `|~~~|'
% which is used as a delimiter).
% Compilation is handed over to the new file by |\childdocforward|:
%    \begin{macrocode}
\newcommand{\childdocforwardprefix}[3][]
{
  \begingroup
    \def\childdocextract #2##1~~~{\def\childdoctmp{\childdocforward[#1]{#3##1}}}
    \expandafter\childdocextract\childdocname~~~
    \expandafter
  \endgroup
  \childdoctmp
}
%    \end{macrocode}

% \macro{\childdoc}
% The deprecated macro |\childdoc| is a legacy version of |\childdocmain|:
%    \begin{macrocode}
\newcommand{\childdoc}{\childdocmain}
%    \end{macrocode}

% \macro{\childdocredirect}
% The deprecated macro |\childdocredirect| is a legacy version
% of |\childdocforward| and |\childdocforwardprefix|:
%    \begin{macrocode}
\newcommand{\childdocredirect}[2][]
{
  \begingroup
    \if?#1?
      \def\childdoctmp{\childdocforward{#2}}
    \else
      \def\childdoctmp{\childdocforwardprefix{#1}{#2}}
    \fi
    \expandafter
  \endgroup
  \childdoctmp
}
%    \end{macrocode}

%\iffalse
%</package>
%\fi
%
\endinput
\childdocforward{cdocsamp}"|\\
% |latex -jobname cdocscl1 \|\\
% |  "% \iffalse
%
% childdoc.dtx Copyright (C) 2017-2018 Niklas Beisert
%
% This work may be distributed and/or modified under the
% conditions of the LaTeX Project Public License, either version 1.3
% of this license or (at your option) any later version.
% The latest version of this license is in
%   http://www.latex-project.org/lppl.txt
% and version 1.3 or later is part of all distributions of LaTeX
% version 2005/12/01 or later.
%
% This work has the LPPL maintenance status `maintained'.
%
% The Current Maintainer of this work is Niklas Beisert.
%
% This work consists of the files childdoc.dtx and childdoc.ins
% and the derived files childdoc.def and cdocsamp.tex with
% cdocsch1.tex, cdocsch2.tex, cdocsdrf.tex, cdocsfn1.tex, cdocsfn2.tex.
%
%<package>\ifdefined\childdocmain\endinput\fi
%<package>\ProvidesFile{childdoc.def}[2018/12/30 v2.0 child document driver]
%<samplemain>\ProvidesFile{cdocsamp.tex}[2018/12/30 v2.0 sample for childdoc]
%<*driver>
%\ProvidesFile{childdoc.drv}[2018/12/30 v2.0 childdoc reference manual file]
\PassOptionsToClass{10pt,a4paper}{article}
\documentclass{ltxdoc}

\usepackage[margin=35mm]{geometry}
\usepackage{hyperref}
\usepackage{hyperxmp}
\usepackage[usenames]{color}

\hypersetup{colorlinks=true}
\hypersetup{pdfstartview=FitH}
\hypersetup{pdfpagemode=UseNone}
\hypersetup{pdfsource={}}
\hypersetup{pdflang={en-UK}}
\hypersetup{pdfcopyright={Copyright 2017-2018 Niklas Beisert.
  This work may be distributed and/or modified under the
  conditions of the LaTeX Project Public License, either version 1.3
  of this license or (at your option) any later version.}}
\hypersetup{pdflicenseurl={http://www.latex-project.org/lppl.txt}}
\hypersetup{pdfcontactaddress={ETH Zurich, ITP, HIT K,
  Wolfgang-Pauli-Strasse 27}}
\hypersetup{pdfcontactpostcode={8093}}
\hypersetup{pdfcontactcity={Zurich}}
\hypersetup{pdfcontactcountry={Switzerland}}
\hypersetup{pdfcontactemail={nbeisert@itp.phys.ethz.ch}}
\hypersetup{pdfcontacturl={http://people.phys.ethz.ch/\xmptilde nbeisert/}}

\newcommand{\secref}[1]{\hyperref[#1]{section \ref*{#1}}}

\parskip1ex
\parindent0pt
\let\olditemize\itemize
\def\itemize{\olditemize\parskip0pt}

\begin{document}

\title{The \textsf{childdoc} Package}
\hypersetup{pdftitle={The childdoc Package}}
\author{Niklas Beisert\\[2ex]
  Institut f\"ur Theoretische Physik\\
  Eidgen\"ossische Technische Hochschule Z\"urich\\
  Wolfgang-Pauli-Strasse 27, 8093 Z\"urich, Switzerland\\[1ex]
  \href{mailto:nbeisert@itp.phys.ethz.ch}
  {\texttt{nbeisert@itp.phys.ethz.ch}}}
\hypersetup{pdfauthor={Niklas Beisert}}
\hypersetup{pdfsubject={Manual for the LaTeX2e Package childdoc}}
\date{30 December 2018, \textsf{v2.0}}
\maketitle

\begin{abstract}\noindent
\textsf{childdoc} is a \LaTeXe{} package
that enables the direct compilation
of document sections included by |\include|
to individual files.
\end{abstract}

\begingroup
\parskip0ex
\tableofcontents
\endgroup

%%%%%%%%%%%%%%%%%%%%%%%%%%%%%%%%%%%%%%%%%%%%%%%%%%%%%%%%%%%%%%%%%%%%%%%%%%%%%%%%
%%%%%%%%%%%%%%%%%%%%%%%%%%%%%%%%%%%%%%%%%%%%%%%%%%%%%%%%%%%%%%%%%%%%%%%%%%%%%%%%
\section{Introduction}

\LaTeX{} provides a mechanism to structure a large document (such as a book)
into a main file and several child files (containing the chapters)
using the |\include| command.
This mechanism is beneficial for documents
which span hundreds of pages in order to
make the source file(s) more manageable.
Moreover, compilation can be restricted to
selected child files by means of the |\includeonly| command.
The latter feature can be used to reduce the compilation time while editing
(this was significantly more useful in the earlier days of \LaTeX{})
or to generate a smaller document which is easier to navigate.
Another application of |\includeonly| is to generate
documents consisting of selected parts of the complete document.

However, there are a few drawbacks of the plain |\include| mechanism:
\begin{itemize}
\item
The child files cannot be compiled on their own,
they can only be compiled via the main file.
A naive editing environment
(such as a text editor with an option
to have the current file processed by \LaTeX)
may require one to switch to the main file before compiling;
attempting to compile the child file produces errors.
\item
The main file must be modified (each time)
to adjust the |\includeonly| command
to the present needs. This easily leaves the main file in a messy state.
\item
The generated document will always carry the filename
of the main document. This is inconvenient if
several child files are to be compiled and
to be kept for distribution.
\end{itemize}

The present package provides a simple interface
to make child files individually compilable by \LaTeX{}.
Compiling a child file then has the same effect as compiling
the main file with an |\includeonly| command
to select the appropriate child.
Moreover the generated document will carry the name of the child
rather than the main file.
This resolves all three above issues.

This feature is meant to make the editing of books,
thesis documents and lecture notes somewhat more convenient.
However, the package can also be used efficiently for
composing a series of documents (such as exercise sheets)
which are typically distributed individually.
It then assists the author in generating the individual documents
(potentially in different versions)
as well as a document containing the collected series.
Another application is in developing style files
or other kinds of included material
where compilation of the style file could redirect
to a sample or test file.

%%%%%%%%%%%%%%%%%%%%%%%%%%%%%%%%%%%%%%%%%%%%%%%%%%%%%%%%%%%%%%%%%%%%%%%%%%%%%%%%
%%%%%%%%%%%%%%%%%%%%%%%%%%%%%%%%%%%%%%%%%%%%%%%%%%%%%%%%%%%%%%%%%%%%%%%%%%%%%%%%
\section{Usage}

First of all, the package \textsf{childdoc} is \emph{not} a standard
\LaTeXe{} |.sty| style file! Therefore it needs to be invoked in
a non-standard way.

%%%%%%%%%%%%%%%%%%%%%%%%%%%%%%%%%%%%%%%%%%%%%%%%%%%%%%%%%%%%%%%%%%%%%%%%%%%%%%%%
\subsection{Included Files}
\label{sec:include}

%%%%%%%%%%%%%%%%%%%%%%%%%%%%%%%%%%%%%%%%
\DescribeMacro{\childdocmain}
To use the package, add the commands
\begin{center}
\begin{tabular}{l}
|\input{childdoc.def}|\\
|\childdocmain{}|\\
\end{tabular}
\end{center}
at the very top of the main \LaTeX{} file,
in particular \emph{before} the |\documentclass| statement!
The argument of |\childdocmain| should be left empty
(but it must be present).

%%%%%%%%%%%%%%%%%%%%%%%%%%%%%%%%%%%%%%%%
\DescribeMacro{\childdocof}
Furthermore, add the commands
\begin{center}
\begin{tabular}{l}
|\input{childdoc.def}|\\
|\childdocof{|\textit{main}|}|\\
\end{tabular}
\end{center}
at the top of every child file \textit{child}
which is included by |\include{|\textit{child}|}|
from within the main file
(or at least for those files to be compiled individually).
The argument \textit{main} must be the filename of the main file.

There are a couple of
considerations in setting up the main and child documents:

%%%%%%%%%%%%%%%%%%%%%%%%%%%%%%%%%%%%%%%%
\paragraph{Restrictions.}

Please note the following restrictions:
\begin{itemize}
\item
|\childdocmain| must be called with one argument \textit{main}
to ensure compatibility with earlier version of the package.
It must either be empty (|\childdocmain{}|)
or precisely match the filename of the main file in which it is specified.
See \secref{sec:detection} for further information.
\item
The filename \textit{main} must be specified without the |.tex| extension.
\item
The filename \textit{main} is case sensitive
(even in case-insensitive file systems)
due to internal string comparison.
\item
The argument \textit{main} should be fully expanded, it cannot be a macro.
\item
Subdirectories and special characters should be avoided in filenames.
\item
The command |\childdocmain{|\textit{main}|}| must be followed by a whitespace.
It should not be followed immediately by another command
or by a comment mark `|%|'.
This is because the \TeX{} parser reads the token immediately following
the argument of |\childdocmain| and puts it
at the beginning of every child section;
however, a white\-space is ignored.
\end{itemize}

%%%%%%%%%%%%%%%%%%%%%%%%%%%%%%%%%%%%%%%%
\paragraph{Content of Main File.}

It is advisable to place all content in the child files included by |\include|.
Any output contained in the main file will appear in all child documents
unless suppressed manually;
it cannot be suppressed automatically by the |\includeonly| directive
and thus should normally be avoided.
A method to include some content in the main file
by means of conditional processing is described in \secref{sec:conditional}.

%%%%%%%%%%%%%%%%%%%%%%%%%%%%%%%%%%%%%%%%
\paragraph{Page Numbering.}

When only a part of the document is compiled,
the appropriate numbering of pages
(as well as other status parameters)
is determined from the |.aux| files.
The latter contain information from previous passes.
However this information needs to propagate through
all intermediate child documents.
Therefore the page numbering in child documents may well
be inconsistent until the complete document is compiled at least once.

A useful (if unconventional) way to always ensure a consistent
page numbering is to restart the numbering in each child document
and denote the pages by `\textit{child}|.|\textit{page}'
where \textit{child} represents the chapter/section number of the child file.
This can be achieved by the command
|\numberwithin{page}{|\textit{child}|}|
of the \textsf{amsmath} package
where \textit{child} can be |chapter| or |section|
depending on the chosen structuring.
Alternatively, one can modify the macro |\thepage| appropriately
and reset the counter |page| at the start of each child file.

%%%%%%%%%%%%%%%%%%%%%%%%%%%%%%%%%%%%%%%%%%%%%%%%%%%%%%%%%%%%%%%%%%%%%%%%%%%%%%%%
\subsection{Conditional Processing}
\label{sec:conditional}

The package provides a mechanism to compile different versions
of a document. To customise the versions further some conditional processing
can come in handy to distinguish which version is being compiled.
The package provides two macros to describe the compilation context:

%%%%%%%%%%%%%%%%%%%%%%%%%%%%%%%%%%%%%%%%
\DescribeMacro{\ifchilddoc}
The conditional |\ifchilddoc| distinguishes between the compilation of
child documents and the main document:
%
\begin{center}
|\ifchilddoc |\textit{child-code}| |[|\||else |\textit{main-code}]| \||fi|
\end{center}

%%%%%%%%%%%%%%%%%%%%%%%%%%%%%%%%%%%%%%%%
\DescribeMacro{\childdocname}
\DescribeMacro{\childdocjob}
The macro |\childdocname| contains the filename (without extension)
of the main or child file being processed.
Note that |\childdocjob| will always contain the name of the main file.

%%%%%%%%%%%%%%%%%%%%%%%%%%%%%%%%%%%%%%%%
\paragraph{Title Page.}

Conditional processing can be used to include a title or banner page
in the main document when proper precautions are taken.
Importantly, the code in the main file should ensure that the page counter
(as well as other status parameters which are stored in the |.aux| files)
takes the same value after the conditional processing.
Otherwise the page numbers may take divergent values
depending on which part is compiled.

For example, a title page could be declared by:
%
\begin{center}
\begin{tabular}{l}
|\ifchilddoc\||else|\\
|\addtocounter{page}{-1}|\\
\textit{code for title page}\\
|\newpage|\\
|\||fi|
\end{tabular}
\end{center}
%
A banner page for the child documents can be generated by:
%
\begin{center}
\begin{tabular}{l}
|\ifchilddoc|\\
|\addtocounter{page}{-1}|\\
\textit{code for banner page}\\
|\newpage|\\
|\||fi|
\end{tabular}
\end{center}
%
Here one could write a message such as:
\begin{center}
|This is the part \childdocname{} of \childdocjob{}.|
\end{center}

%%%%%%%%%%%%%%%%%%%%%%%%%%%%%%%%%%%%%%%%%%%%%%%%%%%%%%%%%%%%%%%%%%%%%%%%%%%%%%%%
\subsection{Flags}
\label{sec:flags}

The package makes it easy to generate different versions
of the main or child documents.
To this end compilation flags can be defined
and assigned different default values.
They will be particularly useful in conjunction
with the forwarding mechanism described in \secref{sec:forward}.

For example, it may be useful to have a flag |\version|
which can be set to |draft| or |final|.
The document source will contain some conditional code
depending on the value of |\version|.
Suppose further, the flag should default to |final| for the main file
and to |draft| for child files
which is a natural assignment for editing the document.
This is achieved by placing the following code
in the preamble of the main document
(below the |\childdocmain| directive):
%
\begin{center}
\begin{tabular}{l}
|\ifchilddoc|\\
|\providecommand{\version}{draft}|\\
|\||else|\\
|\providecommand{\version}{final}|\\
|\||fi|
\end{tabular}
\end{center}
%
The definition by |\providecommand| makes sure
that previous definitions are not overwritten.
Further statements |\providecommand{\version}{...}|
can thus be added before the above code to override it.

For the main file, one might add a line
(between |\childdocmain| and the above block)
%
\begin{center}
|%\ifchilddoc\||else\providecommand{\version}{draft}\||fi|
\end{center}
%
which can be uncommented to produce a draft version.
Likewise one can add a line to the very top of a child file
(above the |\childdocof{|\textit{main}|}| directive)
%
\begin{center}
|%\providecommand{\version}{final}|
\end{center}
%
which can be uncommented to produce the final version of this child document.

%%%%%%%%%%%%%%%%%%%%%%%%%%%%%%%%%%%%%%%%%%%%%%%%%%%%%%%%%%%%%%%%%%%%%%%%%%%%%%%%
\subsection{Forwarding}
\label{sec:forward}

Different versions of the main or child documents
using compilation flags as described in \secref{sec:flags}
can be (permanently) stored in different files
for convenient compilation, viewing and distribution.
To this end, the package defines a command
to pass on compilation to a different file:

%%%%%%%%%%%%%%%%%%%%%%%%%%%%%%%%%%%%%%%%
\DescribeMacro{\childdocforward}
The command |\childdocforward| redirects processing to
another source file:
%
\begin{center}
\begin{tabular}{l}
|\input{childdoc.def}|\\
|\childdocforward[|\textit{main}|]{|\textit{dest}|}|\\
\end{tabular}
\end{center}
%
The argument \textit{dest} is the destination file
(without extension).
It should be the main file or one of the child files.
Note that further \textsf{childdoc} directives
such as |\childdocof| and |\childdocforward|
in the indicated file will be processed in this form.
The optional argument \textit{main}
passes on directly to the main file \textit{main}
while pretending to compile the child \textit{dest}.
This form behaves as if \textit{dest}
issues |\childdocof{|\textit{main}|}| right away,
and no further \textsf{childdoc} directives will be processed.

%%%%%%%%%%%%%%%%%%%%%%%%%%%%%%%%%%%%%%%%
\DescribeMacro{\...prefix}
In the alternative form |\childdocforwardprefix|,
%
\begin{center}
\begin{tabular}{l}
|\input{childdoc.def}|\\
|\childdocforwardprefix[|\textit{main}|]{|\textit{prefix}|}{|\textit{dest}|}|
\end{tabular}
\end{center}
%
the destination file is determined by a pattern
depending on the current file:
To make this work, the current file must be called
`{\textit{prefix}\hspace{0.2em}\textit{suffix}}'
with \textit{prefix} matching precisely the argument.
Processing is then passed on to the file
`{\textit{dest}\hspace{0.2em}\textit{suffix}}'.
Surely, the same effect is achieved by
directly specifying the
argument `{\textit{dest}\hspace{0.2em}\textit{suffix}}'
in the first form.
However, that requires to set up a different file
for each child. With the alternative form of the command
all these files can have exactly the same content
which simplifies setting them up and maintaining them.

For example, the following file |draft.tex|
with a compilation flag |\version| as described in \secref{sec:flags}
compiles the main document as a draft:
%
\begin{center}
\begin{tabular}{l}
|\def\version{draft}|\\
|\input{childdoc.def}|\\
|\childdocforward{|\textit{main}|}|
\end{tabular}
\end{center}
%
Likewise, the following files |final|\textit{nn}|.tex|
compile the final version of the child document
|child|\textit{nn}|.tex|:
%
\begin{center}
\begin{tabular}{l}
|\def\version{final}|\\
|\input{childdoc.def}|\\
|\childdocforwardprefix{final}{child}|
\end{tabular}
\end{center}
%

Note that when several versions of a main file and/or of each child file
are to be generated, it may be convenient to set up a |Makefile| or
shell script to automatise the process.

%%%%%%%%%%%%%%%%%%%%%%%%%%%%%%%%%%%%%%%%%%%%%%%%%%%%%%%%%%%%%%%%%%%%%%%%%%%%%%%%
\subsection{Command Line Processing}
\label{sec:commandline}

The effect of redirection files can also be achieved by invoking
the \LaTeX{} compiler with a more elaborate command line.
Most conveniently this should be done as part
of a shell script or a |Makefile|.

When using \textsf{childdoc} in the main file, the following
command lines effectively perform a redirection
(note that depending on the shell being used,
backslashes may have to be doubled: `|\|' $\to$ `|\\|'):
%
\begin{center}
|... -jobname "|\textit{target}|" |\\|"|[\textit{flags}]%
|\input{childdoc.def}\childdocforward[|\textit{main}|]{|\textit{dest}|}"|
\end{center}
%
Here \textit{target} is the name of the output file,
\textit{main} is the name of the main file
and \textit{dest} is the name of the main or child file to be processed
(all filenames without extensions).
The optional argument \textit{main} can be omitted
if \textit{main} matches \textit{dest}.
Optionally, compilation \textit{flags} can be defined via |\def| commands.
This command line makes the \TeX{} engine believe
it is compiling the file \textit{target}
whose content is specified as the latter parameter.
The provided code then forwards the processing to
\textit{main} or \textit{dest} as described in \secref{sec:forward}.

%%%%%%%%%%%%%%%%%%%%%%%%%%%%%%%%%%%%%%%%%%%%%%%%%%%%%%%%%%%%%%%%%%%%%%%%%%%%%%%%
\subsection{Include by Input}
\label{sec:input}

Including child documents by |\include| has some restrictions by design.
Most notably, the content of a child document always occupies
its own set of pages; pages cannot be shared between child documents.
Usually, this behaviour makes perfect sense
because each child document contain an essential part of the document.
However, in some situations it may be desirable to compose
a document from a collection of parts
without having mandatory page breaks between then.
For this case, the package
provides a mechanism to include parts
by |\input| which can also be processed individually.
However, by construction this mechanism
requires manual handling of the content to be output.

%%%%%%%%%%%%%%%%%%%%%%%%%%%%%%%%%%%%%%%%
\DescribeMacro{\ifchilddocmanual}
The main file should be prepared as usual, see \secref{sec:include}.
However, the document body must make a distinction
between processing of an individual part and of the main document, e.g.:
%
\begin{center}
\begin{tabular}{l}
|\ifchilddocmanual|\\
|\input{\childdocname}|\\
|\||else|\\
\textit{document body with }|\input{|\textit{part}|}|\\
|\||fi|
\end{tabular}
\end{center}
%
The conditional |\ifchilddocmanual| is true whenever
a part to be included by |\input| is being compiled,
and the name of the part is stored in |\childdocname|.

%%%%%%%%%%%%%%%%%%%%%%%%%%%%%%%%%%%%%%%%
\DescribeMacro{\childdocby}
Each part to be included by |\input| should start with:
%
\begin{center}
\begin{tabular}{l}
|\input{childdoc.def}|\\
|\childdocby{|\textit{main}|}|\\
\end{tabular}
\end{center}
%
The directive |\childdocby| is similar to |\childdocof|
described in \secref{sec:include},
but the subsequent selection of content must be done manually.
To that end, both |\ifchilddoc| and |\ifchilddocmanual|
will be true upon processing of a part,
and the name of the part is stored in |\childdocname|.
Note that |\jobname| will be set to the filename of the current part
so that each part receives an individual |.aux| file
that does not interfere with the |.aux| file(s) of the main document.
This behaviour can be altered by the alternative form
|\childdocby[*]{|\textit{main}|}| (with a non-empty optional argument)
which uses the |.aux| file of the main document
by setting |\jobname| to \textit{main}.

%%%%%%%%%%%%%%%%%%%%%%%%%%%%%%%%%%%%%%%%%%%%%%%%%%%%%%%%%%%%%%%%%%%%%%%%%%%%%%%%
\subsection{Driver Development}
\label{sec:driver}

The \textsf{childdoc} mechanism can also be use for the development
of definition files such as \LaTeX{} styles or classes.
This case differs from the above setup with multiple parts
included by |\include| in that no |\includeonly| should be invoked.
This can be achieved by starting the include file
(before |\ProvidesPackage|) with:
%
\begin{center}
\begin{tabular}{l}
|\input{childdoc.def}|\\
|\childdocforward{|\textit{main}|}|\\
\end{tabular}
\end{center}
%
or alternatively with:
%
\begin{center}
\begin{tabular}{l}
|\input{childdoc.def}|\\
|\childdocby{|\textit{main}|}|\\
\end{tabular}
\end{center}
%
Both forms have slightly different effects as described above.
The main file is prepared as usual, see \secref{sec:include}.

%%%%%%%%%%%%%%%%%%%%%%%%%%%%%%%%%%%%%%%%%%%%%%%%%%%%%%%%%%%%%%%%%%%%%%%%%%%%%%%%
\subsection{Legacy Detection}
\label{sec:detection}

The directive |\childdocmain| in the main file can detect
whether the complete document or merely a child is to be compiled
even without using the directive |\childdocof|.
This method is deprecated because it is less robust
and there is no compelling reason to use it;
it is merely provided for backward compatibility
and it may be removed in future versions.

If the detection mechanism is to be used,
it is mandatory to correctly specify
the filename of the main file as the argument of |\childdocmain|:
%
\begin{center}
\begin{tabular}{l}
|\input{childdoc.def}|\\
|\childdocmain{|\textit{main}|}|\\
\end{tabular}
\end{center}
%
If |\jobname| does not match the argument \textit{main} of |\childdocmain|,
it is assumed that |\jobname| points to the child file to be compiled.
When using |\childdocmain| with the main file specified as argument,
it suffices to start a child file
with just |\input{|\textit{main}|}|
without loading of the package and using |\childdocof|.
If instead all processing is done
with the appropriate \textsf{childdoc} directives,
the argument of \textit{main} of |\childdocmain| can be empty.

An alternative version of the command line processing described
in \secref{sec:commandline} using the detection mechanism reads:
%
\begin{center}
|... -jobname "|\textit{target}|" "|[\textit{flags}]%
[|\def\jobname{|\textit{dest}|}|]|\input{|\textit{main}|}"|
\end{center}

%%%%%%%%%%%%%%%%%%%%%%%%%%%%%%%%%%%%%%%%%%%%%%%%%%%%%%%%%%%%%%%%%%%%%%%%%%%%%%%%
\subsection{Manual Code}
\label{sec:manual}

In case one cannot be certain whether the definitions file |childdoc.def|
is installed on the target \TeX{} distribution
and one prefers not to ship it,
it is conceivable to paste a few relevant commands into the sources.

To that end, drop all statements |\input{childdoc.def}|
and perform the replacements as outlined below.
Instead of |\childdocmain{|\textit{main}|}| add the following code
to the top of the main file:
%
\begin{center}
\begin{tabular}{l}
|\||ifdefined\childdocname\endinput\||fi\newif\ifchilddoc|\\
|\edef\childdocname{\scantokens\expandafter{\jobname\noexpand}}|\\
|\def\childdocmain{|\textit{main}|}\||ifx\childdocmain\childdocname\||else|\\
|\childdoctrue\includeonly{\childdocname}\let\jobname\childdocmain\||fi|\\
\end{tabular}
\end{center}
%
Instead of |\childdocof{|\textit{main}|}| just include the main file
at the top of each child file:
%
\begin{center}
|\input{|\textit{main}|}|
\end{center}
%
A simple redirection |\childdocforward{|\textit{dest}|}| is achieved by:
%
\begin{center}
|\def\jobname{|\textit{dest}|}\input{\jobname}|
\end{center}
%
The redirection with prefix
|\childdocforwardprefix[|\textit{prefix}|]{|\textit{dest}|}|
is accomplished by:
%
\begin{center}
\begin{tabular}{l}
|{\edef\jobname{\scantokens\expandafter{\jobname\noexpand}}|\\
|\def\redirectjob |\textit{prefix}|#1~~~{\gdef\jobname{|\textit{dest}|#1}}|\\
|\expandafter\redirectjob\jobname~~~}\input{\jobname}|
\end{tabular}
\end{center}

In an alternative approach,
child documents can be compiled by a specific command line
without additional code or specific definitions:
%
\begin{center}
|... -jobname "|\textit{target}|" "|[\textit{flags}]%
|\includeonly{|\textit{dest}|}\input{|\textit{main}|}"|
\end{center}
%

%%%%%%%%%%%%%%%%%%%%%%%%%%%%%%%%%%%%%%%%%%%%%%%%%%%%%%%%%%%%%%%%%%%%%%%%%%%%%%%%
%%%%%%%%%%%%%%%%%%%%%%%%%%%%%%%%%%%%%%%%%%%%%%%%%%%%%%%%%%%%%%%%%%%%%%%%%%%%%%%%
\section{Information}

%%%%%%%%%%%%%%%%%%%%%%%%%%%%%%%%%%%%%%%%%%%%%%%%%%%%%%%%%%%%%%%%%%%%%%%%%%%%%%%%
\subsection{Copyright}

Copyright \copyright{} 2017--2018 Niklas Beisert

This work may be distributed and/or modified under the
conditions of the \LaTeX{} Project Public License, either version 1.3
of this license or (at your option) any later version.
The latest version of this license is in
  \url{http://www.latex-project.org/lppl.txt}
and version 1.3 or later is part of all distributions of \LaTeX{}
version 2005/12/01 or later.

This work has the LPPL maintenance status `maintained'.

The Current Maintainer of this work is Niklas Beisert.

This work consists of the files |README.txt|, |childdoc.ins| and |childdoc.dtx|
as well as the derived files |childdoc.def|, |cdocsamp.tex|
with |cdocsch1.tex|, |cdocsch2.tex|, |cdocspt3.tex|, |cdocspt4.tex|,
|cdocsdrf.tex|, |cdocsfn1.tex|, |cdocsfn2.tex|
as well as |childdoc.pdf|.

%%%%%%%%%%%%%%%%%%%%%%%%%%%%%%%%%%%%%%%%%%%%%%%%%%%%%%%%%%%%%%%%%%%%%%%%%%%%%%%%
\subsection{Files and Installation}

The package consists of the files:
%
\begin{center}
\begin{tabular}{ll}
    |README.txt|   & readme file \\
    |childdoc.ins| & installation file \\
    |childdoc.dtx| & source file \\
    |childdoc.def| & definition file \\
    |cdocsamp.tex| & sample main file \\
    |cdocsch1.tex| & sample include file \\
    |cdocsch2.tex| & sample include file \\
    |cdocspt3.tex| & sample part file \\
    |cdocspt4.tex| & sample part file \\
    |cdocsdrf.tex| & sample redirection file \\
    |cdocsfn1.tex| & sample redirection file \\
    |cdocsfn2.tex| & sample redirection file \\
    |childdoc.pdf| & manual
\end{tabular}
\end{center}
%
The distribution consists of the files
|README.txt|, |childdoc.ins| and |childdoc.dtx|.
%
\begin{itemize}
\item
Run (pdf)\LaTeX{} on |childdoc.dtx|
to compile the manual |childdoc.pdf| (this file).
\item
Run \LaTeX{} on |childdoc.ins| to create the definitions file |childdoc.def|
and the sample |cdocsamp.tex| with include files
|cdocsch1.tex|, |cdocsch2.tex|, |cdocspt3.tex|, |cdocspt4.tex|,
|cdocsdrf.tex|, |cdocsfn1.tex|, |cdocsfn2.tex|.
Then copy the file |childdoc.def| to an appropriate directory of your \LaTeX{}
distribution, e.g.\ \textit{texmf-root}|/tex/latex/childdoc|.
\end{itemize}

%%%%%%%%%%%%%%%%%%%%%%%%%%%%%%%%%%%%%%%%%%%%%%%%%%%%%%%%%%%%%%%%%%%%%%%%%%%%%%%%
\subsection{Related CTAN Packages}

There are several other packages which offer a similar functionality:
%
\begin{itemize}
\item
The packages
\href{http://ctan.org/pkg/docmute}{\textsf{docmute}},
\href{http://ctan.org/pkg/includex}{\textsf{includex}} and
\href{http://ctan.org/pkg/standalone}{\textsf{standalone}}
provide commands to include only the document body of
a child file thus allowing both files to be compiled individually.
\item
The packages \href{http://ctan.org/pkg/subdocs}{\textsf{subdocs}}
and \href{http://ctan.org/pkg/subfiles}{\textsf{subfiles}}
provide structures in which the main and child documents can be
encapsulated and allowing them to be compiled individually.
The inclusion mechanism is different from the conventional |\include|.
\item
The package \href{http://ctan.org/pkg/combine}{\textsf{combine}}
is an elaborate solution to combine several documents into one.
\end{itemize}
%
See also the CTAN topic \href{http://ctan.org/topic/subdocs}{\textsf{subdocs}}
for further related packages.
The present package differs from the above solutions in that
a document structure constructed with the conventional |\include| mechanism
just needs two extra commands at the top of every file
such that all constituent files can be compiled individually.

%%%%%%%%%%%%%%%%%%%%%%%%%%%%%%%%%%%%%%%%%%%%%%%%%%%%%%%%%%%%%%%%%%%%%%%%%%%%%%%%
%\subsection{Feature Suggestions}
%
%The following is a list of features which may be useful for future
%versions of this package:
%%
%\begin{itemize}
%\item
%\ldots
%\end{itemize}

%%%%%%%%%%%%%%%%%%%%%%%%%%%%%%%%%%%%%%%%%%%%%%%%%%%%%%%%%%%%%%%%%%%%%%%%%%%%%%%%
\subsection{Revision History}

%%%%%%%%%%%%%%%%%%%%%%%%%%%%%%%%%%%%%%%%
\paragraph{v2.0:} 2018/12/30

\begin{itemize}
\item
immediate forward processing
\item
added |\childdocby| mechanism
\item
manual restructured
\end{itemize}

%%%%%%%%%%%%%%%%%%%%%%%%%%%%%%%%%%%%%%%%
\paragraph{v1.6:} 2018/01/17

\begin{itemize}
\item
application for development of include files
\item
corrections to manual
\end{itemize}

%%%%%%%%%%%%%%%%%%%%%%%%%%%%%%%%%%%%%%%%
\paragraph{v1.5:} 2017/05/21

\begin{itemize}
\item
more complete structuring introduced
\item
|\childdocof| introduced
\item
|\childdoc| renamed to |\childdocmain|
\item
|\childredirect| renamed to |\childdocforward| and |\childdocforwardprefix|
and functionality expanded
\end{itemize}

%%%%%%%%%%%%%%%%%%%%%%%%%%%%%%%%%%%%%%%%
\paragraph{v1.0:} 2017/04/27

\begin{itemize}
\item
manual and install package
\item
first version published on CTAN
\end{itemize}

%%%%%%%%%%%%%%%%%%%%%%%%%%%%%%%%%%%%%%%%
\paragraph{v0.6:} 2017/04/26

\begin{itemize}
\item
redirection mechanism added
\end{itemize}

%%%%%%%%%%%%%%%%%%%%%%%%%%%%%%%%%%%%%%%%
\paragraph{v0.5:} 2017/04/26

\begin{itemize}
\item
functionality in definition file
\end{itemize}


%%%%%%%%%%%%%%%%%%%%%%%%%%%%%%%%%%%%%%%%%%%%%%%%%%%%%%%%%%%%%%%%%%%%%%%%%%%%%%%%
%%%%%%%%%%%%%%%%%%%%%%%%%%%%%%%%%%%%%%%%%%%%%%%%%%%%%%%%%%%%%%%%%%%%%%%%%%%%%%%%
%%%%%%%%%%%%%%%%%%%%%%%%%%%%%%%%%%%%%%%%%%%%%%%%%%%%%%%%%%%%%%%%%%%%%%%%%%%%%%%%
\appendix

\settowidth\MacroIndent{\rmfamily\scriptsize 000\ }

 \DocInput{childdoc.dtx}

\end{document}
%</driver>
% \fi
%
% %%%%%%%%%%%%%%%%%%%%%%%%%%%%%%%%%%%%%%%%%%%%%%%%%%%%%%%%%%%%%%%%%%%%%%%%%%%%%%
% %%%%%%%%%%%%%%%%%%%%%%%%%%%%%%%%%%%%%%%%%%%%%%%%%%%%%%%%%%%%%%%%%%%%%%%%%%%%%%
% \section{Sample}
%\iffalse
%<*samplemain>
%\fi
%
% The following presents a sample document
% with two chapters, two parts, a title page,
% a compile flag as well as three forwarding files to set the flag.
% It consists of eight |.tex| files:
% \begin{center}
% \begin{tabular}{ll}
% |cdocsamp.tex|&main file\\
% |cdocsch1.tex|&include file for chapter 1\\
% |cdocsch2.tex|&include file for chapter 2\\
% |cdocspt3.tex|&include file for part 3\\
% |cdocspt4.tex|&include file for part 4\\
% |cdocsdrf.tex|&forwarding file for main file in draft mode\\
% |cdocsfi1.tex|&forwarding file for final version of chapter 1\\
% |cdocsfi2.tex|&forwarding file for final version of chapter 2\\
% \end{tabular}
% \end{center}
% Each of the eight files can be compiled directly by the \LaTeX{} compiler.
%
% %%%%%%%%%%%%%%%%%%%%%%%%%%%%%%%%%%%%%%
% \paragraph{Main File.}
%
% The main file is called |cdocsamp.tex|.
%
% Load the \textsf{childdoc} definitions and
% declare the filename for the main document:
%    \begin{macrocode}
\input{childdoc.def}
\childdocmain{}
%    \end{macrocode}

% Optional override for |\version| flag:
%    \begin{macrocode}
%%\ifchilddoc\else\providecommand{\version}{draft}\fi
%    \end{macrocode}

% Define the default values for the |\version| flag
% (|final| for the main file and |draft| for childs):
%    \begin{macrocode}
\ifchilddoc
\providecommand{\version}{draft}
\else
\providecommand{\version}{final}
\fi
%    \end{macrocode}

% Load the standard document class:
%    \begin{macrocode}
\documentclass[12pt]{article}
%    \end{macrocode}

% Start the document body:
%    \begin{macrocode}
\begin{document}
%    \end{macrocode}

% Declare a title page.
% Print title, part of document being processed and version flag:
%    \begin{macrocode}
\addtocounter{page}{-1}
\begin{center}
{\LARGE\bfseries{}childdoc example\par}
\vspace{1cm}
\ifchilddoc
\ifchilddocmanual part\else chapter\fi:
`\childdocname' of `\childdocjob'\par
\else
main document: `\childdocjob'\par
\fi
version: \version\par
\end{center}
\newpage
%    \end{macrocode}

% Manually include selected file,
% otherwise process as usual:
%    \begin{macrocode}
\ifchilddocmanual
\section*{part `\childdocname'}
\input{\childdocname}
\else
%    \end{macrocode}

% Include the two chapters:
%    \begin{macrocode}
\include{cdocsch1}
\include{cdocsch2}
%    \end{macrocode}

% Include the two parts unless only chapters should be displayed:
%    \begin{macrocode}
\ifchilddoc\else
\section{part three}
\input{cdocspt3}
\section{part four}
\input{cdocspt4}
\fi
%    \end{macrocode}

% Process as usual until here:
%    \begin{macrocode}
\fi
%    \end{macrocode}

% End of document body:
%    \begin{macrocode}
\end{document}
%    \end{macrocode}
%\iffalse
%</samplemain>
%\fi
%
% %%%%%%%%%%%%%%%%%%%%%%%%%%%%%%%%%%%%%%
% \paragraph{Chapter Include Files.}
%
% The include files are called |cdocsch1.tex| and |cdocsch2.tex|.
%
%\iffalse
%<*samplechap1|samplechap2>
%\fi

% Optional override for |\version| flag:
%    \begin{macrocode}
%%\providecommand{\version}{final}
%    \end{macrocode}

% Include the main document:
%    \begin{macrocode}
\input{childdoc.def}
\childdocof{cdocsamp}
%    \end{macrocode}

%\iffalse
%</samplechap1|samplechap2>
%\fi
%
%\iffalse
%<*samplechap1>
%\fi
% Some text for chapter 1:
%    \begin{macrocode}
\section{one}
some text in chapter one
%    \end{macrocode}

%\iffalse
%</samplechap1>
%\fi
% Some text for chapter 2:
%\iffalse
%<*samplechap2>
%\fi
%    \begin{macrocode}
\section{two}
more text in chapter two
%    \end{macrocode}

%\iffalse
%</samplechap2>
%\fi
%
% %%%%%%%%%%%%%%%%%%%%%%%%%%%%%%%%%%%%%%
% \paragraph{Part Include Files.}
%
% The include files are called |cdocspt3.tex| and |cdocspt4.tex|.
%
%\iffalse
%<*samplepart3|samplepart4>
%\fi

% Optional override for |\version| flag:
%    \begin{macrocode}
%%\providecommand{\version}{final}
%    \end{macrocode}

% Include the main document:
%    \begin{macrocode}
\input{childdoc.def}
\childdocby{cdocsamp}
%    \end{macrocode}

%\iffalse
%</samplepart3|samplepart4>
%\fi
%
%\iffalse
%<*samplepart3>
%\fi
% Some text for part 3:
%    \begin{macrocode}
some text in part three
%    \end{macrocode}

%\iffalse
%</samplepart3>
%\fi
% Some text for part 4:
%\iffalse
%<*samplepart4>
%\fi
%    \begin{macrocode}
more text in part four
%    \end{macrocode}

%\iffalse
%</samplepart4>
%\fi
%
% %%%%%%%%%%%%%%%%%%%%%%%%%%%%%%%%%%%%%%
% \paragraph{Forwarding for a Complete Draft.}
%
% The following forwarding file |cdocsdrf.tex|
% compiles the main document in draft mode:
%\iffalse
%<*sampledraft>
%\fi
%    \begin{macrocode}
\def\version{draft}
\input{childdoc.def}
\childdocforward{cdocsamp}
%    \end{macrocode}

%\iffalse
%</sampledraft>
%\fi
%
% %%%%%%%%%%%%%%%%%%%%%%%%%%%%%%%%%%%%%%
% \paragraph{Forwarding for Final Version of the Chapters.}
%
% The following forwarding files |cdocsfn1.tex| and |cdocsfn2.tex|
% (with identical content)
% compile the final versions of the child documents
% |cdocsch1.tex| and |cdocsch2.tex|, respectively:
%\iffalse
%<*samplefinal>
%\fi
%    \begin{macrocode}
\def\version{final}
\input{childdoc.def}
\childdocforwardprefix[cdocsamp]{cdocsfn}{cdocsch}
%    \end{macrocode}

%\iffalse
%</samplefinal>
%\fi
%
% %%%%%%%%%%%%%%%%%%%%%%%%%%%%%%%%%%%%%%
% \paragraph{Command Line Processing.}
%
% The following three command lines generate the output files
% |cdocscld|, |cdocscl1| and |cdocscl2|
% which should be identical to
% |cdocsdrf|, |cdocsch1| and |cdocsfn2|, respectively:
% \begin{center}
% \begin{tabular}{l}
% |latex -jobname cdocscld \|\\
% |  "\def\version{draft}\input{childdoc.def}\childdocforward{cdocsamp}"|\\
% |latex -jobname cdocscl1 \|\\
% |  "\input{childdoc.def}\childdocforward[cdocsamp]{cdocsch1}"|\\
% |latex -jobname cdocscl2 \|\\
% |  "\def\version{final}\input{childdoc.def}\childdocforward{cdocsch2}"|
% \end{tabular}
% \end{center}
% Note that the trailing backslash on each first line
% merely continues the input to the second line
% (for convenient cut ant paste).
% Furthermore, the command |latex| can be replaced by any
% of its alternative versions such as |pdflatex|.
%
% %%%%%%%%%%%%%%%%%%%%%%%%%%%%%%%%%%%%%%%%%%%%%%%%%%%%%%%%%%%%%%%%%%%%%%%%%%%%%%
% %%%%%%%%%%%%%%%%%%%%%%%%%%%%%%%%%%%%%%%%%%%%%%%%%%%%%%%%%%%%%%%%%%%%%%%%%%%%%%
% \section{Implementation}
%\iffalse
%<*package>
%\fi
%
% This section describes the definitions file |childdoc.def|.

% The definitions cannot be loaded using |\usepackage| or |\RequirePackage|
% which has a mechanism to prevent loading a style file more than once.
% When loading the definitions by means of |\input|
% multiple instances have to be prevented manually:
%\iffalse
%This code needs to be before the `\ProvidesFile' directive
%which is defined at the beginning of this file.
%Therefore it is also placed there and commented out here.
%</package>
%<*discard>
%\fi
%    \begin{macrocode}
\ifdefined\childdocmain\endinput\fi
%    \end{macrocode}
%\iffalse
%</discard>
%<*package>
%\fi
%
% \macro{\ifchilddoc}
% \macro{\ifchilddocmanual}
% The conditional |\ifchilddoc| tells whether a
% child (true) or main (false) document is being compiled.
% The conditional |\ifchilddocmanual| tells whether
% the |\includeonly| mechanism is used (false) or
% the selection of child files must be performed manually (true).
% The definitions initialise to false:
%    \begin{macrocode}
\newif\ifchilddoc
\newif\ifchilddocmanual
%    \end{macrocode}

% \macro{\childdocname}
% \macro{\childdocjob}
% The macro |\childdocname| stores the name of the main document
% to be compiled. The macro |\childdocjob| stores the name of
% the document on which the \LaTeX{} compiler was originally invoked.
% The content of |\jobname| cannot be compared
% to filenames specified in the source due to different catcodes.
% The following code rescans |\jobname|, stores the result
% in |\childdocname| and saves a copy in |\childdocjob|:
%    \begin{macrocode}
\edef\childdocname{\scantokens\expandafter{\jobname\noexpand}}
\let\childdocjob\childdocname
%    \end{macrocode}

% \macro{\childdocdisable}
% The macro |\childdocdisable| prevents the main file
% from being processed more than once.
% At this stage, the main document command |\childdocmain|
% is assumed to be called once again where it should do nothing.
% Any subsequent call to it should prevent
% a secondary processing of the main document
% It overwrites the forwarding commands
% |\childdocof| and |\childdocforward|
% with empty macros to prevent further inclusions of the main document:
%    \begin{macrocode}
\newcommand{\childdocdisable}
{
  \renewcommand{\childdocmain}[1]{\renewcommand{\childdocmain}[1]{\endinput}}
  \renewcommand{\childdocof}[1]{}
  \renewcommand{\childdocby}[2][]{}
  \renewcommand{\childdocforward}[2][]{}
  \renewcommand{\childdocdisable}{}
}
%    \end{macrocode}

% \macro{\childdocmain}
% The macro |\childdocmain| is to be called at the top of the main file
% with nothing or the main filename (without extension) as argument.
% First, it breaks loops.
% If the argument is not empty and does not match |\childdocname|
% (which is set by the first inclusion of |childdoc.def|),
% |\ifchilddoc| is set to true, |\includeonly| is applied to the child file
% and |\jobname| is set to the main file
% (for proper handling of |.aux| files):
%    \begin{macrocode}
\newcommand{\childdocmain}[1]
{
  \childdocdisable\childdocmain{}
  \if?#1?\else
    \begingroup
      \def\childdoctmp{#1}
      \ifx\childdoctmp\childdocname
        \def\childdoctmp{}
      \else
        \def\childdoctmp
        {
          \childdoctrue
          \includeonly{\childdocname}
          \def\childdocjob{#1}
          \def\jobname{#1}
        }
      \fi
      \expandafter
    \endgroup
    \childdoctmp
  \fi
}
%    \end{macrocode}

% \macro{\childdocof}
% The command |\childdocof| redirects
% compilation to the main file |#1|.
%    \begin{macrocode}
\newcommand{\childdocof}[1]
{
  \childdocdisable
  \childdoctrue
  \includeonly{\childdocname}
  \def\jobname{#1}
  \def\childdocjob{#1}
  \input{#1}
}
%    \end{macrocode}

% \macro{\childdocby}
% The command |\childdocby| ....
%    \begin{macrocode}
\newcommand{\childdocby}[2][]
{
  \childdocdisable
  \childdoctrue
  \childdocmanualtrue
  \if?#1?\else
    \def\jobname{#2}
  \fi
  \def\childdocjob{#2}
  \input{#2}
  \endinput
}
%    \end{macrocode}

% \macro{\childdocforward}
% The command |\childdocforward| redirects
% compilation to the main file or
% (if the optional argument is given) a child file.
% Parameters are set as if the main file
% or a child file starting with |\childdocof| was compiled.
% Then compilation is handed over to the main file:
%    \begin{macrocode}
\newcommand{\childdocforward}[2][]
{
  \begingroup
    \if?#1?
      \def\childdoctmp
      {
        \def\childdocname{#2}
        \def\childdocjob{#2}
        \def\jobname{#2}
        \input{#2}
        \endinput
      }
    \else
      \def\childdoctmp
      {
        \childdocdisable
        \def\childdocname{#2}
        \childdoctrue
        \includeonly{#2}
        \def\childdocjob{#1}
        \def\jobname{#1}
        \input{#1}
        \endinput
      }
    \fi
    \expandafter
  \endgroup
  \childdoctmp
}
%    \end{macrocode}

% \macro{\childdocforwardprefix}
% The command |\childdocforwardprefix| redirects
% compilation to the main or a child file by means of a pattern.
% The prefix |#1| in the current filename is replaced by |#2|
% and the suffix of the current filename is kept
% (it is assumed that the filename does not contain the substring `|~~~|'
% which is used as a delimiter).
% Compilation is handed over to the new file by |\childdocforward|:
%    \begin{macrocode}
\newcommand{\childdocforwardprefix}[3][]
{
  \begingroup
    \def\childdocextract #2##1~~~{\def\childdoctmp{\childdocforward[#1]{#3##1}}}
    \expandafter\childdocextract\childdocname~~~
    \expandafter
  \endgroup
  \childdoctmp
}
%    \end{macrocode}

% \macro{\childdoc}
% The deprecated macro |\childdoc| is a legacy version of |\childdocmain|:
%    \begin{macrocode}
\newcommand{\childdoc}{\childdocmain}
%    \end{macrocode}

% \macro{\childdocredirect}
% The deprecated macro |\childdocredirect| is a legacy version
% of |\childdocforward| and |\childdocforwardprefix|:
%    \begin{macrocode}
\newcommand{\childdocredirect}[2][]
{
  \begingroup
    \if?#1?
      \def\childdoctmp{\childdocforward{#2}}
    \else
      \def\childdoctmp{\childdocforwardprefix{#1}{#2}}
    \fi
    \expandafter
  \endgroup
  \childdoctmp
}
%    \end{macrocode}

%\iffalse
%</package>
%\fi
%
\endinput
\childdocforward[cdocsamp]{cdocsch1}"|\\
% |latex -jobname cdocscl2 \|\\
% |  "\def\version{final}% \iffalse
%
% childdoc.dtx Copyright (C) 2017-2018 Niklas Beisert
%
% This work may be distributed and/or modified under the
% conditions of the LaTeX Project Public License, either version 1.3
% of this license or (at your option) any later version.
% The latest version of this license is in
%   http://www.latex-project.org/lppl.txt
% and version 1.3 or later is part of all distributions of LaTeX
% version 2005/12/01 or later.
%
% This work has the LPPL maintenance status `maintained'.
%
% The Current Maintainer of this work is Niklas Beisert.
%
% This work consists of the files childdoc.dtx and childdoc.ins
% and the derived files childdoc.def and cdocsamp.tex with
% cdocsch1.tex, cdocsch2.tex, cdocsdrf.tex, cdocsfn1.tex, cdocsfn2.tex.
%
%<package>\ifdefined\childdocmain\endinput\fi
%<package>\ProvidesFile{childdoc.def}[2018/12/30 v2.0 child document driver]
%<samplemain>\ProvidesFile{cdocsamp.tex}[2018/12/30 v2.0 sample for childdoc]
%<*driver>
%\ProvidesFile{childdoc.drv}[2018/12/30 v2.0 childdoc reference manual file]
\PassOptionsToClass{10pt,a4paper}{article}
\documentclass{ltxdoc}

\usepackage[margin=35mm]{geometry}
\usepackage{hyperref}
\usepackage{hyperxmp}
\usepackage[usenames]{color}

\hypersetup{colorlinks=true}
\hypersetup{pdfstartview=FitH}
\hypersetup{pdfpagemode=UseNone}
\hypersetup{pdfsource={}}
\hypersetup{pdflang={en-UK}}
\hypersetup{pdfcopyright={Copyright 2017-2018 Niklas Beisert.
  This work may be distributed and/or modified under the
  conditions of the LaTeX Project Public License, either version 1.3
  of this license or (at your option) any later version.}}
\hypersetup{pdflicenseurl={http://www.latex-project.org/lppl.txt}}
\hypersetup{pdfcontactaddress={ETH Zurich, ITP, HIT K,
  Wolfgang-Pauli-Strasse 27}}
\hypersetup{pdfcontactpostcode={8093}}
\hypersetup{pdfcontactcity={Zurich}}
\hypersetup{pdfcontactcountry={Switzerland}}
\hypersetup{pdfcontactemail={nbeisert@itp.phys.ethz.ch}}
\hypersetup{pdfcontacturl={http://people.phys.ethz.ch/\xmptilde nbeisert/}}

\newcommand{\secref}[1]{\hyperref[#1]{section \ref*{#1}}}

\parskip1ex
\parindent0pt
\let\olditemize\itemize
\def\itemize{\olditemize\parskip0pt}

\begin{document}

\title{The \textsf{childdoc} Package}
\hypersetup{pdftitle={The childdoc Package}}
\author{Niklas Beisert\\[2ex]
  Institut f\"ur Theoretische Physik\\
  Eidgen\"ossische Technische Hochschule Z\"urich\\
  Wolfgang-Pauli-Strasse 27, 8093 Z\"urich, Switzerland\\[1ex]
  \href{mailto:nbeisert@itp.phys.ethz.ch}
  {\texttt{nbeisert@itp.phys.ethz.ch}}}
\hypersetup{pdfauthor={Niklas Beisert}}
\hypersetup{pdfsubject={Manual for the LaTeX2e Package childdoc}}
\date{30 December 2018, \textsf{v2.0}}
\maketitle

\begin{abstract}\noindent
\textsf{childdoc} is a \LaTeXe{} package
that enables the direct compilation
of document sections included by |\include|
to individual files.
\end{abstract}

\begingroup
\parskip0ex
\tableofcontents
\endgroup

%%%%%%%%%%%%%%%%%%%%%%%%%%%%%%%%%%%%%%%%%%%%%%%%%%%%%%%%%%%%%%%%%%%%%%%%%%%%%%%%
%%%%%%%%%%%%%%%%%%%%%%%%%%%%%%%%%%%%%%%%%%%%%%%%%%%%%%%%%%%%%%%%%%%%%%%%%%%%%%%%
\section{Introduction}

\LaTeX{} provides a mechanism to structure a large document (such as a book)
into a main file and several child files (containing the chapters)
using the |\include| command.
This mechanism is beneficial for documents
which span hundreds of pages in order to
make the source file(s) more manageable.
Moreover, compilation can be restricted to
selected child files by means of the |\includeonly| command.
The latter feature can be used to reduce the compilation time while editing
(this was significantly more useful in the earlier days of \LaTeX{})
or to generate a smaller document which is easier to navigate.
Another application of |\includeonly| is to generate
documents consisting of selected parts of the complete document.

However, there are a few drawbacks of the plain |\include| mechanism:
\begin{itemize}
\item
The child files cannot be compiled on their own,
they can only be compiled via the main file.
A naive editing environment
(such as a text editor with an option
to have the current file processed by \LaTeX)
may require one to switch to the main file before compiling;
attempting to compile the child file produces errors.
\item
The main file must be modified (each time)
to adjust the |\includeonly| command
to the present needs. This easily leaves the main file in a messy state.
\item
The generated document will always carry the filename
of the main document. This is inconvenient if
several child files are to be compiled and
to be kept for distribution.
\end{itemize}

The present package provides a simple interface
to make child files individually compilable by \LaTeX{}.
Compiling a child file then has the same effect as compiling
the main file with an |\includeonly| command
to select the appropriate child.
Moreover the generated document will carry the name of the child
rather than the main file.
This resolves all three above issues.

This feature is meant to make the editing of books,
thesis documents and lecture notes somewhat more convenient.
However, the package can also be used efficiently for
composing a series of documents (such as exercise sheets)
which are typically distributed individually.
It then assists the author in generating the individual documents
(potentially in different versions)
as well as a document containing the collected series.
Another application is in developing style files
or other kinds of included material
where compilation of the style file could redirect
to a sample or test file.

%%%%%%%%%%%%%%%%%%%%%%%%%%%%%%%%%%%%%%%%%%%%%%%%%%%%%%%%%%%%%%%%%%%%%%%%%%%%%%%%
%%%%%%%%%%%%%%%%%%%%%%%%%%%%%%%%%%%%%%%%%%%%%%%%%%%%%%%%%%%%%%%%%%%%%%%%%%%%%%%%
\section{Usage}

First of all, the package \textsf{childdoc} is \emph{not} a standard
\LaTeXe{} |.sty| style file! Therefore it needs to be invoked in
a non-standard way.

%%%%%%%%%%%%%%%%%%%%%%%%%%%%%%%%%%%%%%%%%%%%%%%%%%%%%%%%%%%%%%%%%%%%%%%%%%%%%%%%
\subsection{Included Files}
\label{sec:include}

%%%%%%%%%%%%%%%%%%%%%%%%%%%%%%%%%%%%%%%%
\DescribeMacro{\childdocmain}
To use the package, add the commands
\begin{center}
\begin{tabular}{l}
|\input{childdoc.def}|\\
|\childdocmain{}|\\
\end{tabular}
\end{center}
at the very top of the main \LaTeX{} file,
in particular \emph{before} the |\documentclass| statement!
The argument of |\childdocmain| should be left empty
(but it must be present).

%%%%%%%%%%%%%%%%%%%%%%%%%%%%%%%%%%%%%%%%
\DescribeMacro{\childdocof}
Furthermore, add the commands
\begin{center}
\begin{tabular}{l}
|\input{childdoc.def}|\\
|\childdocof{|\textit{main}|}|\\
\end{tabular}
\end{center}
at the top of every child file \textit{child}
which is included by |\include{|\textit{child}|}|
from within the main file
(or at least for those files to be compiled individually).
The argument \textit{main} must be the filename of the main file.

There are a couple of
considerations in setting up the main and child documents:

%%%%%%%%%%%%%%%%%%%%%%%%%%%%%%%%%%%%%%%%
\paragraph{Restrictions.}

Please note the following restrictions:
\begin{itemize}
\item
|\childdocmain| must be called with one argument \textit{main}
to ensure compatibility with earlier version of the package.
It must either be empty (|\childdocmain{}|)
or precisely match the filename of the main file in which it is specified.
See \secref{sec:detection} for further information.
\item
The filename \textit{main} must be specified without the |.tex| extension.
\item
The filename \textit{main} is case sensitive
(even in case-insensitive file systems)
due to internal string comparison.
\item
The argument \textit{main} should be fully expanded, it cannot be a macro.
\item
Subdirectories and special characters should be avoided in filenames.
\item
The command |\childdocmain{|\textit{main}|}| must be followed by a whitespace.
It should not be followed immediately by another command
or by a comment mark `|%|'.
This is because the \TeX{} parser reads the token immediately following
the argument of |\childdocmain| and puts it
at the beginning of every child section;
however, a white\-space is ignored.
\end{itemize}

%%%%%%%%%%%%%%%%%%%%%%%%%%%%%%%%%%%%%%%%
\paragraph{Content of Main File.}

It is advisable to place all content in the child files included by |\include|.
Any output contained in the main file will appear in all child documents
unless suppressed manually;
it cannot be suppressed automatically by the |\includeonly| directive
and thus should normally be avoided.
A method to include some content in the main file
by means of conditional processing is described in \secref{sec:conditional}.

%%%%%%%%%%%%%%%%%%%%%%%%%%%%%%%%%%%%%%%%
\paragraph{Page Numbering.}

When only a part of the document is compiled,
the appropriate numbering of pages
(as well as other status parameters)
is determined from the |.aux| files.
The latter contain information from previous passes.
However this information needs to propagate through
all intermediate child documents.
Therefore the page numbering in child documents may well
be inconsistent until the complete document is compiled at least once.

A useful (if unconventional) way to always ensure a consistent
page numbering is to restart the numbering in each child document
and denote the pages by `\textit{child}|.|\textit{page}'
where \textit{child} represents the chapter/section number of the child file.
This can be achieved by the command
|\numberwithin{page}{|\textit{child}|}|
of the \textsf{amsmath} package
where \textit{child} can be |chapter| or |section|
depending on the chosen structuring.
Alternatively, one can modify the macro |\thepage| appropriately
and reset the counter |page| at the start of each child file.

%%%%%%%%%%%%%%%%%%%%%%%%%%%%%%%%%%%%%%%%%%%%%%%%%%%%%%%%%%%%%%%%%%%%%%%%%%%%%%%%
\subsection{Conditional Processing}
\label{sec:conditional}

The package provides a mechanism to compile different versions
of a document. To customise the versions further some conditional processing
can come in handy to distinguish which version is being compiled.
The package provides two macros to describe the compilation context:

%%%%%%%%%%%%%%%%%%%%%%%%%%%%%%%%%%%%%%%%
\DescribeMacro{\ifchilddoc}
The conditional |\ifchilddoc| distinguishes between the compilation of
child documents and the main document:
%
\begin{center}
|\ifchilddoc |\textit{child-code}| |[|\||else |\textit{main-code}]| \||fi|
\end{center}

%%%%%%%%%%%%%%%%%%%%%%%%%%%%%%%%%%%%%%%%
\DescribeMacro{\childdocname}
\DescribeMacro{\childdocjob}
The macro |\childdocname| contains the filename (without extension)
of the main or child file being processed.
Note that |\childdocjob| will always contain the name of the main file.

%%%%%%%%%%%%%%%%%%%%%%%%%%%%%%%%%%%%%%%%
\paragraph{Title Page.}

Conditional processing can be used to include a title or banner page
in the main document when proper precautions are taken.
Importantly, the code in the main file should ensure that the page counter
(as well as other status parameters which are stored in the |.aux| files)
takes the same value after the conditional processing.
Otherwise the page numbers may take divergent values
depending on which part is compiled.

For example, a title page could be declared by:
%
\begin{center}
\begin{tabular}{l}
|\ifchilddoc\||else|\\
|\addtocounter{page}{-1}|\\
\textit{code for title page}\\
|\newpage|\\
|\||fi|
\end{tabular}
\end{center}
%
A banner page for the child documents can be generated by:
%
\begin{center}
\begin{tabular}{l}
|\ifchilddoc|\\
|\addtocounter{page}{-1}|\\
\textit{code for banner page}\\
|\newpage|\\
|\||fi|
\end{tabular}
\end{center}
%
Here one could write a message such as:
\begin{center}
|This is the part \childdocname{} of \childdocjob{}.|
\end{center}

%%%%%%%%%%%%%%%%%%%%%%%%%%%%%%%%%%%%%%%%%%%%%%%%%%%%%%%%%%%%%%%%%%%%%%%%%%%%%%%%
\subsection{Flags}
\label{sec:flags}

The package makes it easy to generate different versions
of the main or child documents.
To this end compilation flags can be defined
and assigned different default values.
They will be particularly useful in conjunction
with the forwarding mechanism described in \secref{sec:forward}.

For example, it may be useful to have a flag |\version|
which can be set to |draft| or |final|.
The document source will contain some conditional code
depending on the value of |\version|.
Suppose further, the flag should default to |final| for the main file
and to |draft| for child files
which is a natural assignment for editing the document.
This is achieved by placing the following code
in the preamble of the main document
(below the |\childdocmain| directive):
%
\begin{center}
\begin{tabular}{l}
|\ifchilddoc|\\
|\providecommand{\version}{draft}|\\
|\||else|\\
|\providecommand{\version}{final}|\\
|\||fi|
\end{tabular}
\end{center}
%
The definition by |\providecommand| makes sure
that previous definitions are not overwritten.
Further statements |\providecommand{\version}{...}|
can thus be added before the above code to override it.

For the main file, one might add a line
(between |\childdocmain| and the above block)
%
\begin{center}
|%\ifchilddoc\||else\providecommand{\version}{draft}\||fi|
\end{center}
%
which can be uncommented to produce a draft version.
Likewise one can add a line to the very top of a child file
(above the |\childdocof{|\textit{main}|}| directive)
%
\begin{center}
|%\providecommand{\version}{final}|
\end{center}
%
which can be uncommented to produce the final version of this child document.

%%%%%%%%%%%%%%%%%%%%%%%%%%%%%%%%%%%%%%%%%%%%%%%%%%%%%%%%%%%%%%%%%%%%%%%%%%%%%%%%
\subsection{Forwarding}
\label{sec:forward}

Different versions of the main or child documents
using compilation flags as described in \secref{sec:flags}
can be (permanently) stored in different files
for convenient compilation, viewing and distribution.
To this end, the package defines a command
to pass on compilation to a different file:

%%%%%%%%%%%%%%%%%%%%%%%%%%%%%%%%%%%%%%%%
\DescribeMacro{\childdocforward}
The command |\childdocforward| redirects processing to
another source file:
%
\begin{center}
\begin{tabular}{l}
|\input{childdoc.def}|\\
|\childdocforward[|\textit{main}|]{|\textit{dest}|}|\\
\end{tabular}
\end{center}
%
The argument \textit{dest} is the destination file
(without extension).
It should be the main file or one of the child files.
Note that further \textsf{childdoc} directives
such as |\childdocof| and |\childdocforward|
in the indicated file will be processed in this form.
The optional argument \textit{main}
passes on directly to the main file \textit{main}
while pretending to compile the child \textit{dest}.
This form behaves as if \textit{dest}
issues |\childdocof{|\textit{main}|}| right away,
and no further \textsf{childdoc} directives will be processed.

%%%%%%%%%%%%%%%%%%%%%%%%%%%%%%%%%%%%%%%%
\DescribeMacro{\...prefix}
In the alternative form |\childdocforwardprefix|,
%
\begin{center}
\begin{tabular}{l}
|\input{childdoc.def}|\\
|\childdocforwardprefix[|\textit{main}|]{|\textit{prefix}|}{|\textit{dest}|}|
\end{tabular}
\end{center}
%
the destination file is determined by a pattern
depending on the current file:
To make this work, the current file must be called
`{\textit{prefix}\hspace{0.2em}\textit{suffix}}'
with \textit{prefix} matching precisely the argument.
Processing is then passed on to the file
`{\textit{dest}\hspace{0.2em}\textit{suffix}}'.
Surely, the same effect is achieved by
directly specifying the
argument `{\textit{dest}\hspace{0.2em}\textit{suffix}}'
in the first form.
However, that requires to set up a different file
for each child. With the alternative form of the command
all these files can have exactly the same content
which simplifies setting them up and maintaining them.

For example, the following file |draft.tex|
with a compilation flag |\version| as described in \secref{sec:flags}
compiles the main document as a draft:
%
\begin{center}
\begin{tabular}{l}
|\def\version{draft}|\\
|\input{childdoc.def}|\\
|\childdocforward{|\textit{main}|}|
\end{tabular}
\end{center}
%
Likewise, the following files |final|\textit{nn}|.tex|
compile the final version of the child document
|child|\textit{nn}|.tex|:
%
\begin{center}
\begin{tabular}{l}
|\def\version{final}|\\
|\input{childdoc.def}|\\
|\childdocforwardprefix{final}{child}|
\end{tabular}
\end{center}
%

Note that when several versions of a main file and/or of each child file
are to be generated, it may be convenient to set up a |Makefile| or
shell script to automatise the process.

%%%%%%%%%%%%%%%%%%%%%%%%%%%%%%%%%%%%%%%%%%%%%%%%%%%%%%%%%%%%%%%%%%%%%%%%%%%%%%%%
\subsection{Command Line Processing}
\label{sec:commandline}

The effect of redirection files can also be achieved by invoking
the \LaTeX{} compiler with a more elaborate command line.
Most conveniently this should be done as part
of a shell script or a |Makefile|.

When using \textsf{childdoc} in the main file, the following
command lines effectively perform a redirection
(note that depending on the shell being used,
backslashes may have to be doubled: `|\|' $\to$ `|\\|'):
%
\begin{center}
|... -jobname "|\textit{target}|" |\\|"|[\textit{flags}]%
|\input{childdoc.def}\childdocforward[|\textit{main}|]{|\textit{dest}|}"|
\end{center}
%
Here \textit{target} is the name of the output file,
\textit{main} is the name of the main file
and \textit{dest} is the name of the main or child file to be processed
(all filenames without extensions).
The optional argument \textit{main} can be omitted
if \textit{main} matches \textit{dest}.
Optionally, compilation \textit{flags} can be defined via |\def| commands.
This command line makes the \TeX{} engine believe
it is compiling the file \textit{target}
whose content is specified as the latter parameter.
The provided code then forwards the processing to
\textit{main} or \textit{dest} as described in \secref{sec:forward}.

%%%%%%%%%%%%%%%%%%%%%%%%%%%%%%%%%%%%%%%%%%%%%%%%%%%%%%%%%%%%%%%%%%%%%%%%%%%%%%%%
\subsection{Include by Input}
\label{sec:input}

Including child documents by |\include| has some restrictions by design.
Most notably, the content of a child document always occupies
its own set of pages; pages cannot be shared between child documents.
Usually, this behaviour makes perfect sense
because each child document contain an essential part of the document.
However, in some situations it may be desirable to compose
a document from a collection of parts
without having mandatory page breaks between then.
For this case, the package
provides a mechanism to include parts
by |\input| which can also be processed individually.
However, by construction this mechanism
requires manual handling of the content to be output.

%%%%%%%%%%%%%%%%%%%%%%%%%%%%%%%%%%%%%%%%
\DescribeMacro{\ifchilddocmanual}
The main file should be prepared as usual, see \secref{sec:include}.
However, the document body must make a distinction
between processing of an individual part and of the main document, e.g.:
%
\begin{center}
\begin{tabular}{l}
|\ifchilddocmanual|\\
|\input{\childdocname}|\\
|\||else|\\
\textit{document body with }|\input{|\textit{part}|}|\\
|\||fi|
\end{tabular}
\end{center}
%
The conditional |\ifchilddocmanual| is true whenever
a part to be included by |\input| is being compiled,
and the name of the part is stored in |\childdocname|.

%%%%%%%%%%%%%%%%%%%%%%%%%%%%%%%%%%%%%%%%
\DescribeMacro{\childdocby}
Each part to be included by |\input| should start with:
%
\begin{center}
\begin{tabular}{l}
|\input{childdoc.def}|\\
|\childdocby{|\textit{main}|}|\\
\end{tabular}
\end{center}
%
The directive |\childdocby| is similar to |\childdocof|
described in \secref{sec:include},
but the subsequent selection of content must be done manually.
To that end, both |\ifchilddoc| and |\ifchilddocmanual|
will be true upon processing of a part,
and the name of the part is stored in |\childdocname|.
Note that |\jobname| will be set to the filename of the current part
so that each part receives an individual |.aux| file
that does not interfere with the |.aux| file(s) of the main document.
This behaviour can be altered by the alternative form
|\childdocby[*]{|\textit{main}|}| (with a non-empty optional argument)
which uses the |.aux| file of the main document
by setting |\jobname| to \textit{main}.

%%%%%%%%%%%%%%%%%%%%%%%%%%%%%%%%%%%%%%%%%%%%%%%%%%%%%%%%%%%%%%%%%%%%%%%%%%%%%%%%
\subsection{Driver Development}
\label{sec:driver}

The \textsf{childdoc} mechanism can also be use for the development
of definition files such as \LaTeX{} styles or classes.
This case differs from the above setup with multiple parts
included by |\include| in that no |\includeonly| should be invoked.
This can be achieved by starting the include file
(before |\ProvidesPackage|) with:
%
\begin{center}
\begin{tabular}{l}
|\input{childdoc.def}|\\
|\childdocforward{|\textit{main}|}|\\
\end{tabular}
\end{center}
%
or alternatively with:
%
\begin{center}
\begin{tabular}{l}
|\input{childdoc.def}|\\
|\childdocby{|\textit{main}|}|\\
\end{tabular}
\end{center}
%
Both forms have slightly different effects as described above.
The main file is prepared as usual, see \secref{sec:include}.

%%%%%%%%%%%%%%%%%%%%%%%%%%%%%%%%%%%%%%%%%%%%%%%%%%%%%%%%%%%%%%%%%%%%%%%%%%%%%%%%
\subsection{Legacy Detection}
\label{sec:detection}

The directive |\childdocmain| in the main file can detect
whether the complete document or merely a child is to be compiled
even without using the directive |\childdocof|.
This method is deprecated because it is less robust
and there is no compelling reason to use it;
it is merely provided for backward compatibility
and it may be removed in future versions.

If the detection mechanism is to be used,
it is mandatory to correctly specify
the filename of the main file as the argument of |\childdocmain|:
%
\begin{center}
\begin{tabular}{l}
|\input{childdoc.def}|\\
|\childdocmain{|\textit{main}|}|\\
\end{tabular}
\end{center}
%
If |\jobname| does not match the argument \textit{main} of |\childdocmain|,
it is assumed that |\jobname| points to the child file to be compiled.
When using |\childdocmain| with the main file specified as argument,
it suffices to start a child file
with just |\input{|\textit{main}|}|
without loading of the package and using |\childdocof|.
If instead all processing is done
with the appropriate \textsf{childdoc} directives,
the argument of \textit{main} of |\childdocmain| can be empty.

An alternative version of the command line processing described
in \secref{sec:commandline} using the detection mechanism reads:
%
\begin{center}
|... -jobname "|\textit{target}|" "|[\textit{flags}]%
[|\def\jobname{|\textit{dest}|}|]|\input{|\textit{main}|}"|
\end{center}

%%%%%%%%%%%%%%%%%%%%%%%%%%%%%%%%%%%%%%%%%%%%%%%%%%%%%%%%%%%%%%%%%%%%%%%%%%%%%%%%
\subsection{Manual Code}
\label{sec:manual}

In case one cannot be certain whether the definitions file |childdoc.def|
is installed on the target \TeX{} distribution
and one prefers not to ship it,
it is conceivable to paste a few relevant commands into the sources.

To that end, drop all statements |\input{childdoc.def}|
and perform the replacements as outlined below.
Instead of |\childdocmain{|\textit{main}|}| add the following code
to the top of the main file:
%
\begin{center}
\begin{tabular}{l}
|\||ifdefined\childdocname\endinput\||fi\newif\ifchilddoc|\\
|\edef\childdocname{\scantokens\expandafter{\jobname\noexpand}}|\\
|\def\childdocmain{|\textit{main}|}\||ifx\childdocmain\childdocname\||else|\\
|\childdoctrue\includeonly{\childdocname}\let\jobname\childdocmain\||fi|\\
\end{tabular}
\end{center}
%
Instead of |\childdocof{|\textit{main}|}| just include the main file
at the top of each child file:
%
\begin{center}
|\input{|\textit{main}|}|
\end{center}
%
A simple redirection |\childdocforward{|\textit{dest}|}| is achieved by:
%
\begin{center}
|\def\jobname{|\textit{dest}|}\input{\jobname}|
\end{center}
%
The redirection with prefix
|\childdocforwardprefix[|\textit{prefix}|]{|\textit{dest}|}|
is accomplished by:
%
\begin{center}
\begin{tabular}{l}
|{\edef\jobname{\scantokens\expandafter{\jobname\noexpand}}|\\
|\def\redirectjob |\textit{prefix}|#1~~~{\gdef\jobname{|\textit{dest}|#1}}|\\
|\expandafter\redirectjob\jobname~~~}\input{\jobname}|
\end{tabular}
\end{center}

In an alternative approach,
child documents can be compiled by a specific command line
without additional code or specific definitions:
%
\begin{center}
|... -jobname "|\textit{target}|" "|[\textit{flags}]%
|\includeonly{|\textit{dest}|}\input{|\textit{main}|}"|
\end{center}
%

%%%%%%%%%%%%%%%%%%%%%%%%%%%%%%%%%%%%%%%%%%%%%%%%%%%%%%%%%%%%%%%%%%%%%%%%%%%%%%%%
%%%%%%%%%%%%%%%%%%%%%%%%%%%%%%%%%%%%%%%%%%%%%%%%%%%%%%%%%%%%%%%%%%%%%%%%%%%%%%%%
\section{Information}

%%%%%%%%%%%%%%%%%%%%%%%%%%%%%%%%%%%%%%%%%%%%%%%%%%%%%%%%%%%%%%%%%%%%%%%%%%%%%%%%
\subsection{Copyright}

Copyright \copyright{} 2017--2018 Niklas Beisert

This work may be distributed and/or modified under the
conditions of the \LaTeX{} Project Public License, either version 1.3
of this license or (at your option) any later version.
The latest version of this license is in
  \url{http://www.latex-project.org/lppl.txt}
and version 1.3 or later is part of all distributions of \LaTeX{}
version 2005/12/01 or later.

This work has the LPPL maintenance status `maintained'.

The Current Maintainer of this work is Niklas Beisert.

This work consists of the files |README.txt|, |childdoc.ins| and |childdoc.dtx|
as well as the derived files |childdoc.def|, |cdocsamp.tex|
with |cdocsch1.tex|, |cdocsch2.tex|, |cdocspt3.tex|, |cdocspt4.tex|,
|cdocsdrf.tex|, |cdocsfn1.tex|, |cdocsfn2.tex|
as well as |childdoc.pdf|.

%%%%%%%%%%%%%%%%%%%%%%%%%%%%%%%%%%%%%%%%%%%%%%%%%%%%%%%%%%%%%%%%%%%%%%%%%%%%%%%%
\subsection{Files and Installation}

The package consists of the files:
%
\begin{center}
\begin{tabular}{ll}
    |README.txt|   & readme file \\
    |childdoc.ins| & installation file \\
    |childdoc.dtx| & source file \\
    |childdoc.def| & definition file \\
    |cdocsamp.tex| & sample main file \\
    |cdocsch1.tex| & sample include file \\
    |cdocsch2.tex| & sample include file \\
    |cdocspt3.tex| & sample part file \\
    |cdocspt4.tex| & sample part file \\
    |cdocsdrf.tex| & sample redirection file \\
    |cdocsfn1.tex| & sample redirection file \\
    |cdocsfn2.tex| & sample redirection file \\
    |childdoc.pdf| & manual
\end{tabular}
\end{center}
%
The distribution consists of the files
|README.txt|, |childdoc.ins| and |childdoc.dtx|.
%
\begin{itemize}
\item
Run (pdf)\LaTeX{} on |childdoc.dtx|
to compile the manual |childdoc.pdf| (this file).
\item
Run \LaTeX{} on |childdoc.ins| to create the definitions file |childdoc.def|
and the sample |cdocsamp.tex| with include files
|cdocsch1.tex|, |cdocsch2.tex|, |cdocspt3.tex|, |cdocspt4.tex|,
|cdocsdrf.tex|, |cdocsfn1.tex|, |cdocsfn2.tex|.
Then copy the file |childdoc.def| to an appropriate directory of your \LaTeX{}
distribution, e.g.\ \textit{texmf-root}|/tex/latex/childdoc|.
\end{itemize}

%%%%%%%%%%%%%%%%%%%%%%%%%%%%%%%%%%%%%%%%%%%%%%%%%%%%%%%%%%%%%%%%%%%%%%%%%%%%%%%%
\subsection{Related CTAN Packages}

There are several other packages which offer a similar functionality:
%
\begin{itemize}
\item
The packages
\href{http://ctan.org/pkg/docmute}{\textsf{docmute}},
\href{http://ctan.org/pkg/includex}{\textsf{includex}} and
\href{http://ctan.org/pkg/standalone}{\textsf{standalone}}
provide commands to include only the document body of
a child file thus allowing both files to be compiled individually.
\item
The packages \href{http://ctan.org/pkg/subdocs}{\textsf{subdocs}}
and \href{http://ctan.org/pkg/subfiles}{\textsf{subfiles}}
provide structures in which the main and child documents can be
encapsulated and allowing them to be compiled individually.
The inclusion mechanism is different from the conventional |\include|.
\item
The package \href{http://ctan.org/pkg/combine}{\textsf{combine}}
is an elaborate solution to combine several documents into one.
\end{itemize}
%
See also the CTAN topic \href{http://ctan.org/topic/subdocs}{\textsf{subdocs}}
for further related packages.
The present package differs from the above solutions in that
a document structure constructed with the conventional |\include| mechanism
just needs two extra commands at the top of every file
such that all constituent files can be compiled individually.

%%%%%%%%%%%%%%%%%%%%%%%%%%%%%%%%%%%%%%%%%%%%%%%%%%%%%%%%%%%%%%%%%%%%%%%%%%%%%%%%
%\subsection{Feature Suggestions}
%
%The following is a list of features which may be useful for future
%versions of this package:
%%
%\begin{itemize}
%\item
%\ldots
%\end{itemize}

%%%%%%%%%%%%%%%%%%%%%%%%%%%%%%%%%%%%%%%%%%%%%%%%%%%%%%%%%%%%%%%%%%%%%%%%%%%%%%%%
\subsection{Revision History}

%%%%%%%%%%%%%%%%%%%%%%%%%%%%%%%%%%%%%%%%
\paragraph{v2.0:} 2018/12/30

\begin{itemize}
\item
immediate forward processing
\item
added |\childdocby| mechanism
\item
manual restructured
\end{itemize}

%%%%%%%%%%%%%%%%%%%%%%%%%%%%%%%%%%%%%%%%
\paragraph{v1.6:} 2018/01/17

\begin{itemize}
\item
application for development of include files
\item
corrections to manual
\end{itemize}

%%%%%%%%%%%%%%%%%%%%%%%%%%%%%%%%%%%%%%%%
\paragraph{v1.5:} 2017/05/21

\begin{itemize}
\item
more complete structuring introduced
\item
|\childdocof| introduced
\item
|\childdoc| renamed to |\childdocmain|
\item
|\childredirect| renamed to |\childdocforward| and |\childdocforwardprefix|
and functionality expanded
\end{itemize}

%%%%%%%%%%%%%%%%%%%%%%%%%%%%%%%%%%%%%%%%
\paragraph{v1.0:} 2017/04/27

\begin{itemize}
\item
manual and install package
\item
first version published on CTAN
\end{itemize}

%%%%%%%%%%%%%%%%%%%%%%%%%%%%%%%%%%%%%%%%
\paragraph{v0.6:} 2017/04/26

\begin{itemize}
\item
redirection mechanism added
\end{itemize}

%%%%%%%%%%%%%%%%%%%%%%%%%%%%%%%%%%%%%%%%
\paragraph{v0.5:} 2017/04/26

\begin{itemize}
\item
functionality in definition file
\end{itemize}


%%%%%%%%%%%%%%%%%%%%%%%%%%%%%%%%%%%%%%%%%%%%%%%%%%%%%%%%%%%%%%%%%%%%%%%%%%%%%%%%
%%%%%%%%%%%%%%%%%%%%%%%%%%%%%%%%%%%%%%%%%%%%%%%%%%%%%%%%%%%%%%%%%%%%%%%%%%%%%%%%
%%%%%%%%%%%%%%%%%%%%%%%%%%%%%%%%%%%%%%%%%%%%%%%%%%%%%%%%%%%%%%%%%%%%%%%%%%%%%%%%
\appendix

\settowidth\MacroIndent{\rmfamily\scriptsize 000\ }

 \DocInput{childdoc.dtx}

\end{document}
%</driver>
% \fi
%
% %%%%%%%%%%%%%%%%%%%%%%%%%%%%%%%%%%%%%%%%%%%%%%%%%%%%%%%%%%%%%%%%%%%%%%%%%%%%%%
% %%%%%%%%%%%%%%%%%%%%%%%%%%%%%%%%%%%%%%%%%%%%%%%%%%%%%%%%%%%%%%%%%%%%%%%%%%%%%%
% \section{Sample}
%\iffalse
%<*samplemain>
%\fi
%
% The following presents a sample document
% with two chapters, two parts, a title page,
% a compile flag as well as three forwarding files to set the flag.
% It consists of eight |.tex| files:
% \begin{center}
% \begin{tabular}{ll}
% |cdocsamp.tex|&main file\\
% |cdocsch1.tex|&include file for chapter 1\\
% |cdocsch2.tex|&include file for chapter 2\\
% |cdocspt3.tex|&include file for part 3\\
% |cdocspt4.tex|&include file for part 4\\
% |cdocsdrf.tex|&forwarding file for main file in draft mode\\
% |cdocsfi1.tex|&forwarding file for final version of chapter 1\\
% |cdocsfi2.tex|&forwarding file for final version of chapter 2\\
% \end{tabular}
% \end{center}
% Each of the eight files can be compiled directly by the \LaTeX{} compiler.
%
% %%%%%%%%%%%%%%%%%%%%%%%%%%%%%%%%%%%%%%
% \paragraph{Main File.}
%
% The main file is called |cdocsamp.tex|.
%
% Load the \textsf{childdoc} definitions and
% declare the filename for the main document:
%    \begin{macrocode}
\input{childdoc.def}
\childdocmain{}
%    \end{macrocode}

% Optional override for |\version| flag:
%    \begin{macrocode}
%%\ifchilddoc\else\providecommand{\version}{draft}\fi
%    \end{macrocode}

% Define the default values for the |\version| flag
% (|final| for the main file and |draft| for childs):
%    \begin{macrocode}
\ifchilddoc
\providecommand{\version}{draft}
\else
\providecommand{\version}{final}
\fi
%    \end{macrocode}

% Load the standard document class:
%    \begin{macrocode}
\documentclass[12pt]{article}
%    \end{macrocode}

% Start the document body:
%    \begin{macrocode}
\begin{document}
%    \end{macrocode}

% Declare a title page.
% Print title, part of document being processed and version flag:
%    \begin{macrocode}
\addtocounter{page}{-1}
\begin{center}
{\LARGE\bfseries{}childdoc example\par}
\vspace{1cm}
\ifchilddoc
\ifchilddocmanual part\else chapter\fi:
`\childdocname' of `\childdocjob'\par
\else
main document: `\childdocjob'\par
\fi
version: \version\par
\end{center}
\newpage
%    \end{macrocode}

% Manually include selected file,
% otherwise process as usual:
%    \begin{macrocode}
\ifchilddocmanual
\section*{part `\childdocname'}
\input{\childdocname}
\else
%    \end{macrocode}

% Include the two chapters:
%    \begin{macrocode}
\include{cdocsch1}
\include{cdocsch2}
%    \end{macrocode}

% Include the two parts unless only chapters should be displayed:
%    \begin{macrocode}
\ifchilddoc\else
\section{part three}
\input{cdocspt3}
\section{part four}
\input{cdocspt4}
\fi
%    \end{macrocode}

% Process as usual until here:
%    \begin{macrocode}
\fi
%    \end{macrocode}

% End of document body:
%    \begin{macrocode}
\end{document}
%    \end{macrocode}
%\iffalse
%</samplemain>
%\fi
%
% %%%%%%%%%%%%%%%%%%%%%%%%%%%%%%%%%%%%%%
% \paragraph{Chapter Include Files.}
%
% The include files are called |cdocsch1.tex| and |cdocsch2.tex|.
%
%\iffalse
%<*samplechap1|samplechap2>
%\fi

% Optional override for |\version| flag:
%    \begin{macrocode}
%%\providecommand{\version}{final}
%    \end{macrocode}

% Include the main document:
%    \begin{macrocode}
\input{childdoc.def}
\childdocof{cdocsamp}
%    \end{macrocode}

%\iffalse
%</samplechap1|samplechap2>
%\fi
%
%\iffalse
%<*samplechap1>
%\fi
% Some text for chapter 1:
%    \begin{macrocode}
\section{one}
some text in chapter one
%    \end{macrocode}

%\iffalse
%</samplechap1>
%\fi
% Some text for chapter 2:
%\iffalse
%<*samplechap2>
%\fi
%    \begin{macrocode}
\section{two}
more text in chapter two
%    \end{macrocode}

%\iffalse
%</samplechap2>
%\fi
%
% %%%%%%%%%%%%%%%%%%%%%%%%%%%%%%%%%%%%%%
% \paragraph{Part Include Files.}
%
% The include files are called |cdocspt3.tex| and |cdocspt4.tex|.
%
%\iffalse
%<*samplepart3|samplepart4>
%\fi

% Optional override for |\version| flag:
%    \begin{macrocode}
%%\providecommand{\version}{final}
%    \end{macrocode}

% Include the main document:
%    \begin{macrocode}
\input{childdoc.def}
\childdocby{cdocsamp}
%    \end{macrocode}

%\iffalse
%</samplepart3|samplepart4>
%\fi
%
%\iffalse
%<*samplepart3>
%\fi
% Some text for part 3:
%    \begin{macrocode}
some text in part three
%    \end{macrocode}

%\iffalse
%</samplepart3>
%\fi
% Some text for part 4:
%\iffalse
%<*samplepart4>
%\fi
%    \begin{macrocode}
more text in part four
%    \end{macrocode}

%\iffalse
%</samplepart4>
%\fi
%
% %%%%%%%%%%%%%%%%%%%%%%%%%%%%%%%%%%%%%%
% \paragraph{Forwarding for a Complete Draft.}
%
% The following forwarding file |cdocsdrf.tex|
% compiles the main document in draft mode:
%\iffalse
%<*sampledraft>
%\fi
%    \begin{macrocode}
\def\version{draft}
\input{childdoc.def}
\childdocforward{cdocsamp}
%    \end{macrocode}

%\iffalse
%</sampledraft>
%\fi
%
% %%%%%%%%%%%%%%%%%%%%%%%%%%%%%%%%%%%%%%
% \paragraph{Forwarding for Final Version of the Chapters.}
%
% The following forwarding files |cdocsfn1.tex| and |cdocsfn2.tex|
% (with identical content)
% compile the final versions of the child documents
% |cdocsch1.tex| and |cdocsch2.tex|, respectively:
%\iffalse
%<*samplefinal>
%\fi
%    \begin{macrocode}
\def\version{final}
\input{childdoc.def}
\childdocforwardprefix[cdocsamp]{cdocsfn}{cdocsch}
%    \end{macrocode}

%\iffalse
%</samplefinal>
%\fi
%
% %%%%%%%%%%%%%%%%%%%%%%%%%%%%%%%%%%%%%%
% \paragraph{Command Line Processing.}
%
% The following three command lines generate the output files
% |cdocscld|, |cdocscl1| and |cdocscl2|
% which should be identical to
% |cdocsdrf|, |cdocsch1| and |cdocsfn2|, respectively:
% \begin{center}
% \begin{tabular}{l}
% |latex -jobname cdocscld \|\\
% |  "\def\version{draft}\input{childdoc.def}\childdocforward{cdocsamp}"|\\
% |latex -jobname cdocscl1 \|\\
% |  "\input{childdoc.def}\childdocforward[cdocsamp]{cdocsch1}"|\\
% |latex -jobname cdocscl2 \|\\
% |  "\def\version{final}\input{childdoc.def}\childdocforward{cdocsch2}"|
% \end{tabular}
% \end{center}
% Note that the trailing backslash on each first line
% merely continues the input to the second line
% (for convenient cut ant paste).
% Furthermore, the command |latex| can be replaced by any
% of its alternative versions such as |pdflatex|.
%
% %%%%%%%%%%%%%%%%%%%%%%%%%%%%%%%%%%%%%%%%%%%%%%%%%%%%%%%%%%%%%%%%%%%%%%%%%%%%%%
% %%%%%%%%%%%%%%%%%%%%%%%%%%%%%%%%%%%%%%%%%%%%%%%%%%%%%%%%%%%%%%%%%%%%%%%%%%%%%%
% \section{Implementation}
%\iffalse
%<*package>
%\fi
%
% This section describes the definitions file |childdoc.def|.

% The definitions cannot be loaded using |\usepackage| or |\RequirePackage|
% which has a mechanism to prevent loading a style file more than once.
% When loading the definitions by means of |\input|
% multiple instances have to be prevented manually:
%\iffalse
%This code needs to be before the `\ProvidesFile' directive
%which is defined at the beginning of this file.
%Therefore it is also placed there and commented out here.
%</package>
%<*discard>
%\fi
%    \begin{macrocode}
\ifdefined\childdocmain\endinput\fi
%    \end{macrocode}
%\iffalse
%</discard>
%<*package>
%\fi
%
% \macro{\ifchilddoc}
% \macro{\ifchilddocmanual}
% The conditional |\ifchilddoc| tells whether a
% child (true) or main (false) document is being compiled.
% The conditional |\ifchilddocmanual| tells whether
% the |\includeonly| mechanism is used (false) or
% the selection of child files must be performed manually (true).
% The definitions initialise to false:
%    \begin{macrocode}
\newif\ifchilddoc
\newif\ifchilddocmanual
%    \end{macrocode}

% \macro{\childdocname}
% \macro{\childdocjob}
% The macro |\childdocname| stores the name of the main document
% to be compiled. The macro |\childdocjob| stores the name of
% the document on which the \LaTeX{} compiler was originally invoked.
% The content of |\jobname| cannot be compared
% to filenames specified in the source due to different catcodes.
% The following code rescans |\jobname|, stores the result
% in |\childdocname| and saves a copy in |\childdocjob|:
%    \begin{macrocode}
\edef\childdocname{\scantokens\expandafter{\jobname\noexpand}}
\let\childdocjob\childdocname
%    \end{macrocode}

% \macro{\childdocdisable}
% The macro |\childdocdisable| prevents the main file
% from being processed more than once.
% At this stage, the main document command |\childdocmain|
% is assumed to be called once again where it should do nothing.
% Any subsequent call to it should prevent
% a secondary processing of the main document
% It overwrites the forwarding commands
% |\childdocof| and |\childdocforward|
% with empty macros to prevent further inclusions of the main document:
%    \begin{macrocode}
\newcommand{\childdocdisable}
{
  \renewcommand{\childdocmain}[1]{\renewcommand{\childdocmain}[1]{\endinput}}
  \renewcommand{\childdocof}[1]{}
  \renewcommand{\childdocby}[2][]{}
  \renewcommand{\childdocforward}[2][]{}
  \renewcommand{\childdocdisable}{}
}
%    \end{macrocode}

% \macro{\childdocmain}
% The macro |\childdocmain| is to be called at the top of the main file
% with nothing or the main filename (without extension) as argument.
% First, it breaks loops.
% If the argument is not empty and does not match |\childdocname|
% (which is set by the first inclusion of |childdoc.def|),
% |\ifchilddoc| is set to true, |\includeonly| is applied to the child file
% and |\jobname| is set to the main file
% (for proper handling of |.aux| files):
%    \begin{macrocode}
\newcommand{\childdocmain}[1]
{
  \childdocdisable\childdocmain{}
  \if?#1?\else
    \begingroup
      \def\childdoctmp{#1}
      \ifx\childdoctmp\childdocname
        \def\childdoctmp{}
      \else
        \def\childdoctmp
        {
          \childdoctrue
          \includeonly{\childdocname}
          \def\childdocjob{#1}
          \def\jobname{#1}
        }
      \fi
      \expandafter
    \endgroup
    \childdoctmp
  \fi
}
%    \end{macrocode}

% \macro{\childdocof}
% The command |\childdocof| redirects
% compilation to the main file |#1|.
%    \begin{macrocode}
\newcommand{\childdocof}[1]
{
  \childdocdisable
  \childdoctrue
  \includeonly{\childdocname}
  \def\jobname{#1}
  \def\childdocjob{#1}
  \input{#1}
}
%    \end{macrocode}

% \macro{\childdocby}
% The command |\childdocby| ....
%    \begin{macrocode}
\newcommand{\childdocby}[2][]
{
  \childdocdisable
  \childdoctrue
  \childdocmanualtrue
  \if?#1?\else
    \def\jobname{#2}
  \fi
  \def\childdocjob{#2}
  \input{#2}
  \endinput
}
%    \end{macrocode}

% \macro{\childdocforward}
% The command |\childdocforward| redirects
% compilation to the main file or
% (if the optional argument is given) a child file.
% Parameters are set as if the main file
% or a child file starting with |\childdocof| was compiled.
% Then compilation is handed over to the main file:
%    \begin{macrocode}
\newcommand{\childdocforward}[2][]
{
  \begingroup
    \if?#1?
      \def\childdoctmp
      {
        \def\childdocname{#2}
        \def\childdocjob{#2}
        \def\jobname{#2}
        \input{#2}
        \endinput
      }
    \else
      \def\childdoctmp
      {
        \childdocdisable
        \def\childdocname{#2}
        \childdoctrue
        \includeonly{#2}
        \def\childdocjob{#1}
        \def\jobname{#1}
        \input{#1}
        \endinput
      }
    \fi
    \expandafter
  \endgroup
  \childdoctmp
}
%    \end{macrocode}

% \macro{\childdocforwardprefix}
% The command |\childdocforwardprefix| redirects
% compilation to the main or a child file by means of a pattern.
% The prefix |#1| in the current filename is replaced by |#2|
% and the suffix of the current filename is kept
% (it is assumed that the filename does not contain the substring `|~~~|'
% which is used as a delimiter).
% Compilation is handed over to the new file by |\childdocforward|:
%    \begin{macrocode}
\newcommand{\childdocforwardprefix}[3][]
{
  \begingroup
    \def\childdocextract #2##1~~~{\def\childdoctmp{\childdocforward[#1]{#3##1}}}
    \expandafter\childdocextract\childdocname~~~
    \expandafter
  \endgroup
  \childdoctmp
}
%    \end{macrocode}

% \macro{\childdoc}
% The deprecated macro |\childdoc| is a legacy version of |\childdocmain|:
%    \begin{macrocode}
\newcommand{\childdoc}{\childdocmain}
%    \end{macrocode}

% \macro{\childdocredirect}
% The deprecated macro |\childdocredirect| is a legacy version
% of |\childdocforward| and |\childdocforwardprefix|:
%    \begin{macrocode}
\newcommand{\childdocredirect}[2][]
{
  \begingroup
    \if?#1?
      \def\childdoctmp{\childdocforward{#2}}
    \else
      \def\childdoctmp{\childdocforwardprefix{#1}{#2}}
    \fi
    \expandafter
  \endgroup
  \childdoctmp
}
%    \end{macrocode}

%\iffalse
%</package>
%\fi
%
\endinput
\childdocforward{cdocsch2}"|
% \end{tabular}
% \end{center}
% Note that the trailing backslash on each first line
% merely continues the input to the second line
% (for convenient cut ant paste).
% Furthermore, the command |latex| can be replaced by any
% of its alternative versions such as |pdflatex|.
%
% %%%%%%%%%%%%%%%%%%%%%%%%%%%%%%%%%%%%%%%%%%%%%%%%%%%%%%%%%%%%%%%%%%%%%%%%%%%%%%
% %%%%%%%%%%%%%%%%%%%%%%%%%%%%%%%%%%%%%%%%%%%%%%%%%%%%%%%%%%%%%%%%%%%%%%%%%%%%%%
% \section{Implementation}
%\iffalse
%<*package>
%\fi
%
% This section describes the definitions file |childdoc.def|.

% The definitions cannot be loaded using |\usepackage| or |\RequirePackage|
% which has a mechanism to prevent loading a style file more than once.
% When loading the definitions by means of |\input|
% multiple instances have to be prevented manually:
%\iffalse
%This code needs to be before the `\ProvidesFile' directive
%which is defined at the beginning of this file.
%Therefore it is also placed there and commented out here.
%</package>
%<*discard>
%\fi
%    \begin{macrocode}
\ifdefined\childdocmain\endinput\fi
%    \end{macrocode}
%\iffalse
%</discard>
%<*package>
%\fi
%
% \macro{\ifchilddoc}
% \macro{\ifchilddocmanual}
% The conditional |\ifchilddoc| tells whether a
% child (true) or main (false) document is being compiled.
% The conditional |\ifchilddocmanual| tells whether
% the |\includeonly| mechanism is used (false) or
% the selection of child files must be performed manually (true).
% The definitions initialise to false:
%    \begin{macrocode}
\newif\ifchilddoc
\newif\ifchilddocmanual
%    \end{macrocode}

% \macro{\childdocname}
% \macro{\childdocjob}
% The macro |\childdocname| stores the name of the main document
% to be compiled. The macro |\childdocjob| stores the name of
% the document on which the \LaTeX{} compiler was originally invoked.
% The content of |\jobname| cannot be compared
% to filenames specified in the source due to different catcodes.
% The following code rescans |\jobname|, stores the result
% in |\childdocname| and saves a copy in |\childdocjob|:
%    \begin{macrocode}
\edef\childdocname{\scantokens\expandafter{\jobname\noexpand}}
\let\childdocjob\childdocname
%    \end{macrocode}

% \macro{\childdocdisable}
% The macro |\childdocdisable| prevents the main file
% from being processed more than once.
% At this stage, the main document command |\childdocmain|
% is assumed to be called once again where it should do nothing.
% Any subsequent call to it should prevent
% a secondary processing of the main document
% It overwrites the forwarding commands
% |\childdocof| and |\childdocforward|
% with empty macros to prevent further inclusions of the main document:
%    \begin{macrocode}
\newcommand{\childdocdisable}
{
  \renewcommand{\childdocmain}[1]{\renewcommand{\childdocmain}[1]{\endinput}}
  \renewcommand{\childdocof}[1]{}
  \renewcommand{\childdocby}[2][]{}
  \renewcommand{\childdocforward}[2][]{}
  \renewcommand{\childdocdisable}{}
}
%    \end{macrocode}

% \macro{\childdocmain}
% The macro |\childdocmain| is to be called at the top of the main file
% with nothing or the main filename (without extension) as argument.
% First, it breaks loops.
% If the argument is not empty and does not match |\childdocname|
% (which is set by the first inclusion of |childdoc.def|),
% |\ifchilddoc| is set to true, |\includeonly| is applied to the child file
% and |\jobname| is set to the main file
% (for proper handling of |.aux| files):
%    \begin{macrocode}
\newcommand{\childdocmain}[1]
{
  \childdocdisable\childdocmain{}
  \if?#1?\else
    \begingroup
      \def\childdoctmp{#1}
      \ifx\childdoctmp\childdocname
        \def\childdoctmp{}
      \else
        \def\childdoctmp
        {
          \childdoctrue
          \includeonly{\childdocname}
          \def\childdocjob{#1}
          \def\jobname{#1}
        }
      \fi
      \expandafter
    \endgroup
    \childdoctmp
  \fi
}
%    \end{macrocode}

% \macro{\childdocof}
% The command |\childdocof| redirects
% compilation to the main file |#1|.
%    \begin{macrocode}
\newcommand{\childdocof}[1]
{
  \childdocdisable
  \childdoctrue
  \includeonly{\childdocname}
  \def\jobname{#1}
  \def\childdocjob{#1}
  \input{#1}
}
%    \end{macrocode}

% \macro{\childdocby}
% The command |\childdocby| ....
%    \begin{macrocode}
\newcommand{\childdocby}[2][]
{
  \childdocdisable
  \childdoctrue
  \childdocmanualtrue
  \if?#1?\else
    \def\jobname{#2}
  \fi
  \def\childdocjob{#2}
  \input{#2}
  \endinput
}
%    \end{macrocode}

% \macro{\childdocforward}
% The command |\childdocforward| redirects
% compilation to the main file or
% (if the optional argument is given) a child file.
% Parameters are set as if the main file
% or a child file starting with |\childdocof| was compiled.
% Then compilation is handed over to the main file:
%    \begin{macrocode}
\newcommand{\childdocforward}[2][]
{
  \begingroup
    \if?#1?
      \def\childdoctmp
      {
        \def\childdocname{#2}
        \def\childdocjob{#2}
        \def\jobname{#2}
        \input{#2}
        \endinput
      }
    \else
      \def\childdoctmp
      {
        \childdocdisable
        \def\childdocname{#2}
        \childdoctrue
        \includeonly{#2}
        \def\childdocjob{#1}
        \def\jobname{#1}
        \input{#1}
        \endinput
      }
    \fi
    \expandafter
  \endgroup
  \childdoctmp
}
%    \end{macrocode}

% \macro{\childdocforwardprefix}
% The command |\childdocforwardprefix| redirects
% compilation to the main or a child file by means of a pattern.
% The prefix |#1| in the current filename is replaced by |#2|
% and the suffix of the current filename is kept
% (it is assumed that the filename does not contain the substring `|~~~|'
% which is used as a delimiter).
% Compilation is handed over to the new file by |\childdocforward|:
%    \begin{macrocode}
\newcommand{\childdocforwardprefix}[3][]
{
  \begingroup
    \def\childdocextract #2##1~~~{\def\childdoctmp{\childdocforward[#1]{#3##1}}}
    \expandafter\childdocextract\childdocname~~~
    \expandafter
  \endgroup
  \childdoctmp
}
%    \end{macrocode}

% \macro{\childdoc}
% The deprecated macro |\childdoc| is a legacy version of |\childdocmain|:
%    \begin{macrocode}
\newcommand{\childdoc}{\childdocmain}
%    \end{macrocode}

% \macro{\childdocredirect}
% The deprecated macro |\childdocredirect| is a legacy version
% of |\childdocforward| and |\childdocforwardprefix|:
%    \begin{macrocode}
\newcommand{\childdocredirect}[2][]
{
  \begingroup
    \if?#1?
      \def\childdoctmp{\childdocforward{#2}}
    \else
      \def\childdoctmp{\childdocforwardprefix{#1}{#2}}
    \fi
    \expandafter
  \endgroup
  \childdoctmp
}
%    \end{macrocode}

%\iffalse
%</package>
%\fi
%
\endinput

\childdocmain{}
%    \end{macrocode}

% Optional override for |\version| flag:
%    \begin{macrocode}
%%\ifchilddoc\else\providecommand{\version}{draft}\fi
%    \end{macrocode}

% Define the default values for the |\version| flag
% (|final| for the main file and |draft| for childs):
%    \begin{macrocode}
\ifchilddoc
\providecommand{\version}{draft}
\else
\providecommand{\version}{final}
\fi
%    \end{macrocode}

% Load the standard document class:
%    \begin{macrocode}
\documentclass[12pt]{article}
%    \end{macrocode}

% Start the document body:
%    \begin{macrocode}
\begin{document}
%    \end{macrocode}

% Declare a title page.
% Print title, part of document being processed and version flag:
%    \begin{macrocode}
\addtocounter{page}{-1}
\begin{center}
{\LARGE\bfseries{}childdoc example\par}
\vspace{1cm}
\ifchilddoc
\ifchilddocmanual part\else chapter\fi:
`\childdocname' of `\childdocjob'\par
\else
main document: `\childdocjob'\par
\fi
version: \version\par
\end{center}
\newpage
%    \end{macrocode}

% Manually include selected file,
% otherwise process as usual:
%    \begin{macrocode}
\ifchilddocmanual
\section*{part `\childdocname'}
\input{\childdocname}
\else
%    \end{macrocode}

% Include the two chapters:
%    \begin{macrocode}
\include{cdocsch1}
\include{cdocsch2}
%    \end{macrocode}

% Include the two parts unless only chapters should be displayed:
%    \begin{macrocode}
\ifchilddoc\else
\section{part three}
\input{cdocspt3}
\section{part four}
\input{cdocspt4}
\fi
%    \end{macrocode}

% Process as usual until here:
%    \begin{macrocode}
\fi
%    \end{macrocode}

% End of document body:
%    \begin{macrocode}
\end{document}
%    \end{macrocode}
%\iffalse
%</samplemain>
%\fi
%
% %%%%%%%%%%%%%%%%%%%%%%%%%%%%%%%%%%%%%%
% \paragraph{Chapter Include Files.}
%
% The include files are called |cdocsch1.tex| and |cdocsch2.tex|.
%
%\iffalse
%<*samplechap1|samplechap2>
%\fi

% Optional override for |\version| flag:
%    \begin{macrocode}
%%\providecommand{\version}{final}
%    \end{macrocode}

% Include the main document:
%    \begin{macrocode}
% \iffalse
%
% childdoc.dtx Copyright (C) 2017-2018 Niklas Beisert
%
% This work may be distributed and/or modified under the
% conditions of the LaTeX Project Public License, either version 1.3
% of this license or (at your option) any later version.
% The latest version of this license is in
%   http://www.latex-project.org/lppl.txt
% and version 1.3 or later is part of all distributions of LaTeX
% version 2005/12/01 or later.
%
% This work has the LPPL maintenance status `maintained'.
%
% The Current Maintainer of this work is Niklas Beisert.
%
% This work consists of the files childdoc.dtx and childdoc.ins
% and the derived files childdoc.def and cdocsamp.tex with
% cdocsch1.tex, cdocsch2.tex, cdocsdrf.tex, cdocsfn1.tex, cdocsfn2.tex.
%
%<package>\ifdefined\childdocmain\endinput\fi
%<package>\ProvidesFile{childdoc.def}[2018/12/30 v2.0 child document driver]
%<samplemain>\ProvidesFile{cdocsamp.tex}[2018/12/30 v2.0 sample for childdoc]
%<*driver>
%\ProvidesFile{childdoc.drv}[2018/12/30 v2.0 childdoc reference manual file]
\PassOptionsToClass{10pt,a4paper}{article}
\documentclass{ltxdoc}

\usepackage[margin=35mm]{geometry}
\usepackage{hyperref}
\usepackage{hyperxmp}
\usepackage[usenames]{color}

\hypersetup{colorlinks=true}
\hypersetup{pdfstartview=FitH}
\hypersetup{pdfpagemode=UseNone}
\hypersetup{pdfsource={}}
\hypersetup{pdflang={en-UK}}
\hypersetup{pdfcopyright={Copyright 2017-2018 Niklas Beisert.
  This work may be distributed and/or modified under the
  conditions of the LaTeX Project Public License, either version 1.3
  of this license or (at your option) any later version.}}
\hypersetup{pdflicenseurl={http://www.latex-project.org/lppl.txt}}
\hypersetup{pdfcontactaddress={ETH Zurich, ITP, HIT K,
  Wolfgang-Pauli-Strasse 27}}
\hypersetup{pdfcontactpostcode={8093}}
\hypersetup{pdfcontactcity={Zurich}}
\hypersetup{pdfcontactcountry={Switzerland}}
\hypersetup{pdfcontactemail={nbeisert@itp.phys.ethz.ch}}
\hypersetup{pdfcontacturl={http://people.phys.ethz.ch/\xmptilde nbeisert/}}

\newcommand{\secref}[1]{\hyperref[#1]{section \ref*{#1}}}

\parskip1ex
\parindent0pt
\let\olditemize\itemize
\def\itemize{\olditemize\parskip0pt}

\begin{document}

\title{The \textsf{childdoc} Package}
\hypersetup{pdftitle={The childdoc Package}}
\author{Niklas Beisert\\[2ex]
  Institut f\"ur Theoretische Physik\\
  Eidgen\"ossische Technische Hochschule Z\"urich\\
  Wolfgang-Pauli-Strasse 27, 8093 Z\"urich, Switzerland\\[1ex]
  \href{mailto:nbeisert@itp.phys.ethz.ch}
  {\texttt{nbeisert@itp.phys.ethz.ch}}}
\hypersetup{pdfauthor={Niklas Beisert}}
\hypersetup{pdfsubject={Manual for the LaTeX2e Package childdoc}}
\date{30 December 2018, \textsf{v2.0}}
\maketitle

\begin{abstract}\noindent
\textsf{childdoc} is a \LaTeXe{} package
that enables the direct compilation
of document sections included by |\include|
to individual files.
\end{abstract}

\begingroup
\parskip0ex
\tableofcontents
\endgroup

%%%%%%%%%%%%%%%%%%%%%%%%%%%%%%%%%%%%%%%%%%%%%%%%%%%%%%%%%%%%%%%%%%%%%%%%%%%%%%%%
%%%%%%%%%%%%%%%%%%%%%%%%%%%%%%%%%%%%%%%%%%%%%%%%%%%%%%%%%%%%%%%%%%%%%%%%%%%%%%%%
\section{Introduction}

\LaTeX{} provides a mechanism to structure a large document (such as a book)
into a main file and several child files (containing the chapters)
using the |\include| command.
This mechanism is beneficial for documents
which span hundreds of pages in order to
make the source file(s) more manageable.
Moreover, compilation can be restricted to
selected child files by means of the |\includeonly| command.
The latter feature can be used to reduce the compilation time while editing
(this was significantly more useful in the earlier days of \LaTeX{})
or to generate a smaller document which is easier to navigate.
Another application of |\includeonly| is to generate
documents consisting of selected parts of the complete document.

However, there are a few drawbacks of the plain |\include| mechanism:
\begin{itemize}
\item
The child files cannot be compiled on their own,
they can only be compiled via the main file.
A naive editing environment
(such as a text editor with an option
to have the current file processed by \LaTeX)
may require one to switch to the main file before compiling;
attempting to compile the child file produces errors.
\item
The main file must be modified (each time)
to adjust the |\includeonly| command
to the present needs. This easily leaves the main file in a messy state.
\item
The generated document will always carry the filename
of the main document. This is inconvenient if
several child files are to be compiled and
to be kept for distribution.
\end{itemize}

The present package provides a simple interface
to make child files individually compilable by \LaTeX{}.
Compiling a child file then has the same effect as compiling
the main file with an |\includeonly| command
to select the appropriate child.
Moreover the generated document will carry the name of the child
rather than the main file.
This resolves all three above issues.

This feature is meant to make the editing of books,
thesis documents and lecture notes somewhat more convenient.
However, the package can also be used efficiently for
composing a series of documents (such as exercise sheets)
which are typically distributed individually.
It then assists the author in generating the individual documents
(potentially in different versions)
as well as a document containing the collected series.
Another application is in developing style files
or other kinds of included material
where compilation of the style file could redirect
to a sample or test file.

%%%%%%%%%%%%%%%%%%%%%%%%%%%%%%%%%%%%%%%%%%%%%%%%%%%%%%%%%%%%%%%%%%%%%%%%%%%%%%%%
%%%%%%%%%%%%%%%%%%%%%%%%%%%%%%%%%%%%%%%%%%%%%%%%%%%%%%%%%%%%%%%%%%%%%%%%%%%%%%%%
\section{Usage}

First of all, the package \textsf{childdoc} is \emph{not} a standard
\LaTeXe{} |.sty| style file! Therefore it needs to be invoked in
a non-standard way.

%%%%%%%%%%%%%%%%%%%%%%%%%%%%%%%%%%%%%%%%%%%%%%%%%%%%%%%%%%%%%%%%%%%%%%%%%%%%%%%%
\subsection{Included Files}
\label{sec:include}

%%%%%%%%%%%%%%%%%%%%%%%%%%%%%%%%%%%%%%%%
\DescribeMacro{\childdocmain}
To use the package, add the commands
\begin{center}
\begin{tabular}{l}
|% \iffalse
%
% childdoc.dtx Copyright (C) 2017-2018 Niklas Beisert
%
% This work may be distributed and/or modified under the
% conditions of the LaTeX Project Public License, either version 1.3
% of this license or (at your option) any later version.
% The latest version of this license is in
%   http://www.latex-project.org/lppl.txt
% and version 1.3 or later is part of all distributions of LaTeX
% version 2005/12/01 or later.
%
% This work has the LPPL maintenance status `maintained'.
%
% The Current Maintainer of this work is Niklas Beisert.
%
% This work consists of the files childdoc.dtx and childdoc.ins
% and the derived files childdoc.def and cdocsamp.tex with
% cdocsch1.tex, cdocsch2.tex, cdocsdrf.tex, cdocsfn1.tex, cdocsfn2.tex.
%
%<package>\ifdefined\childdocmain\endinput\fi
%<package>\ProvidesFile{childdoc.def}[2018/12/30 v2.0 child document driver]
%<samplemain>\ProvidesFile{cdocsamp.tex}[2018/12/30 v2.0 sample for childdoc]
%<*driver>
%\ProvidesFile{childdoc.drv}[2018/12/30 v2.0 childdoc reference manual file]
\PassOptionsToClass{10pt,a4paper}{article}
\documentclass{ltxdoc}

\usepackage[margin=35mm]{geometry}
\usepackage{hyperref}
\usepackage{hyperxmp}
\usepackage[usenames]{color}

\hypersetup{colorlinks=true}
\hypersetup{pdfstartview=FitH}
\hypersetup{pdfpagemode=UseNone}
\hypersetup{pdfsource={}}
\hypersetup{pdflang={en-UK}}
\hypersetup{pdfcopyright={Copyright 2017-2018 Niklas Beisert.
  This work may be distributed and/or modified under the
  conditions of the LaTeX Project Public License, either version 1.3
  of this license or (at your option) any later version.}}
\hypersetup{pdflicenseurl={http://www.latex-project.org/lppl.txt}}
\hypersetup{pdfcontactaddress={ETH Zurich, ITP, HIT K,
  Wolfgang-Pauli-Strasse 27}}
\hypersetup{pdfcontactpostcode={8093}}
\hypersetup{pdfcontactcity={Zurich}}
\hypersetup{pdfcontactcountry={Switzerland}}
\hypersetup{pdfcontactemail={nbeisert@itp.phys.ethz.ch}}
\hypersetup{pdfcontacturl={http://people.phys.ethz.ch/\xmptilde nbeisert/}}

\newcommand{\secref}[1]{\hyperref[#1]{section \ref*{#1}}}

\parskip1ex
\parindent0pt
\let\olditemize\itemize
\def\itemize{\olditemize\parskip0pt}

\begin{document}

\title{The \textsf{childdoc} Package}
\hypersetup{pdftitle={The childdoc Package}}
\author{Niklas Beisert\\[2ex]
  Institut f\"ur Theoretische Physik\\
  Eidgen\"ossische Technische Hochschule Z\"urich\\
  Wolfgang-Pauli-Strasse 27, 8093 Z\"urich, Switzerland\\[1ex]
  \href{mailto:nbeisert@itp.phys.ethz.ch}
  {\texttt{nbeisert@itp.phys.ethz.ch}}}
\hypersetup{pdfauthor={Niklas Beisert}}
\hypersetup{pdfsubject={Manual for the LaTeX2e Package childdoc}}
\date{30 December 2018, \textsf{v2.0}}
\maketitle

\begin{abstract}\noindent
\textsf{childdoc} is a \LaTeXe{} package
that enables the direct compilation
of document sections included by |\include|
to individual files.
\end{abstract}

\begingroup
\parskip0ex
\tableofcontents
\endgroup

%%%%%%%%%%%%%%%%%%%%%%%%%%%%%%%%%%%%%%%%%%%%%%%%%%%%%%%%%%%%%%%%%%%%%%%%%%%%%%%%
%%%%%%%%%%%%%%%%%%%%%%%%%%%%%%%%%%%%%%%%%%%%%%%%%%%%%%%%%%%%%%%%%%%%%%%%%%%%%%%%
\section{Introduction}

\LaTeX{} provides a mechanism to structure a large document (such as a book)
into a main file and several child files (containing the chapters)
using the |\include| command.
This mechanism is beneficial for documents
which span hundreds of pages in order to
make the source file(s) more manageable.
Moreover, compilation can be restricted to
selected child files by means of the |\includeonly| command.
The latter feature can be used to reduce the compilation time while editing
(this was significantly more useful in the earlier days of \LaTeX{})
or to generate a smaller document which is easier to navigate.
Another application of |\includeonly| is to generate
documents consisting of selected parts of the complete document.

However, there are a few drawbacks of the plain |\include| mechanism:
\begin{itemize}
\item
The child files cannot be compiled on their own,
they can only be compiled via the main file.
A naive editing environment
(such as a text editor with an option
to have the current file processed by \LaTeX)
may require one to switch to the main file before compiling;
attempting to compile the child file produces errors.
\item
The main file must be modified (each time)
to adjust the |\includeonly| command
to the present needs. This easily leaves the main file in a messy state.
\item
The generated document will always carry the filename
of the main document. This is inconvenient if
several child files are to be compiled and
to be kept for distribution.
\end{itemize}

The present package provides a simple interface
to make child files individually compilable by \LaTeX{}.
Compiling a child file then has the same effect as compiling
the main file with an |\includeonly| command
to select the appropriate child.
Moreover the generated document will carry the name of the child
rather than the main file.
This resolves all three above issues.

This feature is meant to make the editing of books,
thesis documents and lecture notes somewhat more convenient.
However, the package can also be used efficiently for
composing a series of documents (such as exercise sheets)
which are typically distributed individually.
It then assists the author in generating the individual documents
(potentially in different versions)
as well as a document containing the collected series.
Another application is in developing style files
or other kinds of included material
where compilation of the style file could redirect
to a sample or test file.

%%%%%%%%%%%%%%%%%%%%%%%%%%%%%%%%%%%%%%%%%%%%%%%%%%%%%%%%%%%%%%%%%%%%%%%%%%%%%%%%
%%%%%%%%%%%%%%%%%%%%%%%%%%%%%%%%%%%%%%%%%%%%%%%%%%%%%%%%%%%%%%%%%%%%%%%%%%%%%%%%
\section{Usage}

First of all, the package \textsf{childdoc} is \emph{not} a standard
\LaTeXe{} |.sty| style file! Therefore it needs to be invoked in
a non-standard way.

%%%%%%%%%%%%%%%%%%%%%%%%%%%%%%%%%%%%%%%%%%%%%%%%%%%%%%%%%%%%%%%%%%%%%%%%%%%%%%%%
\subsection{Included Files}
\label{sec:include}

%%%%%%%%%%%%%%%%%%%%%%%%%%%%%%%%%%%%%%%%
\DescribeMacro{\childdocmain}
To use the package, add the commands
\begin{center}
\begin{tabular}{l}
|\input{childdoc.def}|\\
|\childdocmain{}|\\
\end{tabular}
\end{center}
at the very top of the main \LaTeX{} file,
in particular \emph{before} the |\documentclass| statement!
The argument of |\childdocmain| should be left empty
(but it must be present).

%%%%%%%%%%%%%%%%%%%%%%%%%%%%%%%%%%%%%%%%
\DescribeMacro{\childdocof}
Furthermore, add the commands
\begin{center}
\begin{tabular}{l}
|\input{childdoc.def}|\\
|\childdocof{|\textit{main}|}|\\
\end{tabular}
\end{center}
at the top of every child file \textit{child}
which is included by |\include{|\textit{child}|}|
from within the main file
(or at least for those files to be compiled individually).
The argument \textit{main} must be the filename of the main file.

There are a couple of
considerations in setting up the main and child documents:

%%%%%%%%%%%%%%%%%%%%%%%%%%%%%%%%%%%%%%%%
\paragraph{Restrictions.}

Please note the following restrictions:
\begin{itemize}
\item
|\childdocmain| must be called with one argument \textit{main}
to ensure compatibility with earlier version of the package.
It must either be empty (|\childdocmain{}|)
or precisely match the filename of the main file in which it is specified.
See \secref{sec:detection} for further information.
\item
The filename \textit{main} must be specified without the |.tex| extension.
\item
The filename \textit{main} is case sensitive
(even in case-insensitive file systems)
due to internal string comparison.
\item
The argument \textit{main} should be fully expanded, it cannot be a macro.
\item
Subdirectories and special characters should be avoided in filenames.
\item
The command |\childdocmain{|\textit{main}|}| must be followed by a whitespace.
It should not be followed immediately by another command
or by a comment mark `|%|'.
This is because the \TeX{} parser reads the token immediately following
the argument of |\childdocmain| and puts it
at the beginning of every child section;
however, a white\-space is ignored.
\end{itemize}

%%%%%%%%%%%%%%%%%%%%%%%%%%%%%%%%%%%%%%%%
\paragraph{Content of Main File.}

It is advisable to place all content in the child files included by |\include|.
Any output contained in the main file will appear in all child documents
unless suppressed manually;
it cannot be suppressed automatically by the |\includeonly| directive
and thus should normally be avoided.
A method to include some content in the main file
by means of conditional processing is described in \secref{sec:conditional}.

%%%%%%%%%%%%%%%%%%%%%%%%%%%%%%%%%%%%%%%%
\paragraph{Page Numbering.}

When only a part of the document is compiled,
the appropriate numbering of pages
(as well as other status parameters)
is determined from the |.aux| files.
The latter contain information from previous passes.
However this information needs to propagate through
all intermediate child documents.
Therefore the page numbering in child documents may well
be inconsistent until the complete document is compiled at least once.

A useful (if unconventional) way to always ensure a consistent
page numbering is to restart the numbering in each child document
and denote the pages by `\textit{child}|.|\textit{page}'
where \textit{child} represents the chapter/section number of the child file.
This can be achieved by the command
|\numberwithin{page}{|\textit{child}|}|
of the \textsf{amsmath} package
where \textit{child} can be |chapter| or |section|
depending on the chosen structuring.
Alternatively, one can modify the macro |\thepage| appropriately
and reset the counter |page| at the start of each child file.

%%%%%%%%%%%%%%%%%%%%%%%%%%%%%%%%%%%%%%%%%%%%%%%%%%%%%%%%%%%%%%%%%%%%%%%%%%%%%%%%
\subsection{Conditional Processing}
\label{sec:conditional}

The package provides a mechanism to compile different versions
of a document. To customise the versions further some conditional processing
can come in handy to distinguish which version is being compiled.
The package provides two macros to describe the compilation context:

%%%%%%%%%%%%%%%%%%%%%%%%%%%%%%%%%%%%%%%%
\DescribeMacro{\ifchilddoc}
The conditional |\ifchilddoc| distinguishes between the compilation of
child documents and the main document:
%
\begin{center}
|\ifchilddoc |\textit{child-code}| |[|\||else |\textit{main-code}]| \||fi|
\end{center}

%%%%%%%%%%%%%%%%%%%%%%%%%%%%%%%%%%%%%%%%
\DescribeMacro{\childdocname}
\DescribeMacro{\childdocjob}
The macro |\childdocname| contains the filename (without extension)
of the main or child file being processed.
Note that |\childdocjob| will always contain the name of the main file.

%%%%%%%%%%%%%%%%%%%%%%%%%%%%%%%%%%%%%%%%
\paragraph{Title Page.}

Conditional processing can be used to include a title or banner page
in the main document when proper precautions are taken.
Importantly, the code in the main file should ensure that the page counter
(as well as other status parameters which are stored in the |.aux| files)
takes the same value after the conditional processing.
Otherwise the page numbers may take divergent values
depending on which part is compiled.

For example, a title page could be declared by:
%
\begin{center}
\begin{tabular}{l}
|\ifchilddoc\||else|\\
|\addtocounter{page}{-1}|\\
\textit{code for title page}\\
|\newpage|\\
|\||fi|
\end{tabular}
\end{center}
%
A banner page for the child documents can be generated by:
%
\begin{center}
\begin{tabular}{l}
|\ifchilddoc|\\
|\addtocounter{page}{-1}|\\
\textit{code for banner page}\\
|\newpage|\\
|\||fi|
\end{tabular}
\end{center}
%
Here one could write a message such as:
\begin{center}
|This is the part \childdocname{} of \childdocjob{}.|
\end{center}

%%%%%%%%%%%%%%%%%%%%%%%%%%%%%%%%%%%%%%%%%%%%%%%%%%%%%%%%%%%%%%%%%%%%%%%%%%%%%%%%
\subsection{Flags}
\label{sec:flags}

The package makes it easy to generate different versions
of the main or child documents.
To this end compilation flags can be defined
and assigned different default values.
They will be particularly useful in conjunction
with the forwarding mechanism described in \secref{sec:forward}.

For example, it may be useful to have a flag |\version|
which can be set to |draft| or |final|.
The document source will contain some conditional code
depending on the value of |\version|.
Suppose further, the flag should default to |final| for the main file
and to |draft| for child files
which is a natural assignment for editing the document.
This is achieved by placing the following code
in the preamble of the main document
(below the |\childdocmain| directive):
%
\begin{center}
\begin{tabular}{l}
|\ifchilddoc|\\
|\providecommand{\version}{draft}|\\
|\||else|\\
|\providecommand{\version}{final}|\\
|\||fi|
\end{tabular}
\end{center}
%
The definition by |\providecommand| makes sure
that previous definitions are not overwritten.
Further statements |\providecommand{\version}{...}|
can thus be added before the above code to override it.

For the main file, one might add a line
(between |\childdocmain| and the above block)
%
\begin{center}
|%\ifchilddoc\||else\providecommand{\version}{draft}\||fi|
\end{center}
%
which can be uncommented to produce a draft version.
Likewise one can add a line to the very top of a child file
(above the |\childdocof{|\textit{main}|}| directive)
%
\begin{center}
|%\providecommand{\version}{final}|
\end{center}
%
which can be uncommented to produce the final version of this child document.

%%%%%%%%%%%%%%%%%%%%%%%%%%%%%%%%%%%%%%%%%%%%%%%%%%%%%%%%%%%%%%%%%%%%%%%%%%%%%%%%
\subsection{Forwarding}
\label{sec:forward}

Different versions of the main or child documents
using compilation flags as described in \secref{sec:flags}
can be (permanently) stored in different files
for convenient compilation, viewing and distribution.
To this end, the package defines a command
to pass on compilation to a different file:

%%%%%%%%%%%%%%%%%%%%%%%%%%%%%%%%%%%%%%%%
\DescribeMacro{\childdocforward}
The command |\childdocforward| redirects processing to
another source file:
%
\begin{center}
\begin{tabular}{l}
|\input{childdoc.def}|\\
|\childdocforward[|\textit{main}|]{|\textit{dest}|}|\\
\end{tabular}
\end{center}
%
The argument \textit{dest} is the destination file
(without extension).
It should be the main file or one of the child files.
Note that further \textsf{childdoc} directives
such as |\childdocof| and |\childdocforward|
in the indicated file will be processed in this form.
The optional argument \textit{main}
passes on directly to the main file \textit{main}
while pretending to compile the child \textit{dest}.
This form behaves as if \textit{dest}
issues |\childdocof{|\textit{main}|}| right away,
and no further \textsf{childdoc} directives will be processed.

%%%%%%%%%%%%%%%%%%%%%%%%%%%%%%%%%%%%%%%%
\DescribeMacro{\...prefix}
In the alternative form |\childdocforwardprefix|,
%
\begin{center}
\begin{tabular}{l}
|\input{childdoc.def}|\\
|\childdocforwardprefix[|\textit{main}|]{|\textit{prefix}|}{|\textit{dest}|}|
\end{tabular}
\end{center}
%
the destination file is determined by a pattern
depending on the current file:
To make this work, the current file must be called
`{\textit{prefix}\hspace{0.2em}\textit{suffix}}'
with \textit{prefix} matching precisely the argument.
Processing is then passed on to the file
`{\textit{dest}\hspace{0.2em}\textit{suffix}}'.
Surely, the same effect is achieved by
directly specifying the
argument `{\textit{dest}\hspace{0.2em}\textit{suffix}}'
in the first form.
However, that requires to set up a different file
for each child. With the alternative form of the command
all these files can have exactly the same content
which simplifies setting them up and maintaining them.

For example, the following file |draft.tex|
with a compilation flag |\version| as described in \secref{sec:flags}
compiles the main document as a draft:
%
\begin{center}
\begin{tabular}{l}
|\def\version{draft}|\\
|\input{childdoc.def}|\\
|\childdocforward{|\textit{main}|}|
\end{tabular}
\end{center}
%
Likewise, the following files |final|\textit{nn}|.tex|
compile the final version of the child document
|child|\textit{nn}|.tex|:
%
\begin{center}
\begin{tabular}{l}
|\def\version{final}|\\
|\input{childdoc.def}|\\
|\childdocforwardprefix{final}{child}|
\end{tabular}
\end{center}
%

Note that when several versions of a main file and/or of each child file
are to be generated, it may be convenient to set up a |Makefile| or
shell script to automatise the process.

%%%%%%%%%%%%%%%%%%%%%%%%%%%%%%%%%%%%%%%%%%%%%%%%%%%%%%%%%%%%%%%%%%%%%%%%%%%%%%%%
\subsection{Command Line Processing}
\label{sec:commandline}

The effect of redirection files can also be achieved by invoking
the \LaTeX{} compiler with a more elaborate command line.
Most conveniently this should be done as part
of a shell script or a |Makefile|.

When using \textsf{childdoc} in the main file, the following
command lines effectively perform a redirection
(note that depending on the shell being used,
backslashes may have to be doubled: `|\|' $\to$ `|\\|'):
%
\begin{center}
|... -jobname "|\textit{target}|" |\\|"|[\textit{flags}]%
|\input{childdoc.def}\childdocforward[|\textit{main}|]{|\textit{dest}|}"|
\end{center}
%
Here \textit{target} is the name of the output file,
\textit{main} is the name of the main file
and \textit{dest} is the name of the main or child file to be processed
(all filenames without extensions).
The optional argument \textit{main} can be omitted
if \textit{main} matches \textit{dest}.
Optionally, compilation \textit{flags} can be defined via |\def| commands.
This command line makes the \TeX{} engine believe
it is compiling the file \textit{target}
whose content is specified as the latter parameter.
The provided code then forwards the processing to
\textit{main} or \textit{dest} as described in \secref{sec:forward}.

%%%%%%%%%%%%%%%%%%%%%%%%%%%%%%%%%%%%%%%%%%%%%%%%%%%%%%%%%%%%%%%%%%%%%%%%%%%%%%%%
\subsection{Include by Input}
\label{sec:input}

Including child documents by |\include| has some restrictions by design.
Most notably, the content of a child document always occupies
its own set of pages; pages cannot be shared between child documents.
Usually, this behaviour makes perfect sense
because each child document contain an essential part of the document.
However, in some situations it may be desirable to compose
a document from a collection of parts
without having mandatory page breaks between then.
For this case, the package
provides a mechanism to include parts
by |\input| which can also be processed individually.
However, by construction this mechanism
requires manual handling of the content to be output.

%%%%%%%%%%%%%%%%%%%%%%%%%%%%%%%%%%%%%%%%
\DescribeMacro{\ifchilddocmanual}
The main file should be prepared as usual, see \secref{sec:include}.
However, the document body must make a distinction
between processing of an individual part and of the main document, e.g.:
%
\begin{center}
\begin{tabular}{l}
|\ifchilddocmanual|\\
|\input{\childdocname}|\\
|\||else|\\
\textit{document body with }|\input{|\textit{part}|}|\\
|\||fi|
\end{tabular}
\end{center}
%
The conditional |\ifchilddocmanual| is true whenever
a part to be included by |\input| is being compiled,
and the name of the part is stored in |\childdocname|.

%%%%%%%%%%%%%%%%%%%%%%%%%%%%%%%%%%%%%%%%
\DescribeMacro{\childdocby}
Each part to be included by |\input| should start with:
%
\begin{center}
\begin{tabular}{l}
|\input{childdoc.def}|\\
|\childdocby{|\textit{main}|}|\\
\end{tabular}
\end{center}
%
The directive |\childdocby| is similar to |\childdocof|
described in \secref{sec:include},
but the subsequent selection of content must be done manually.
To that end, both |\ifchilddoc| and |\ifchilddocmanual|
will be true upon processing of a part,
and the name of the part is stored in |\childdocname|.
Note that |\jobname| will be set to the filename of the current part
so that each part receives an individual |.aux| file
that does not interfere with the |.aux| file(s) of the main document.
This behaviour can be altered by the alternative form
|\childdocby[*]{|\textit{main}|}| (with a non-empty optional argument)
which uses the |.aux| file of the main document
by setting |\jobname| to \textit{main}.

%%%%%%%%%%%%%%%%%%%%%%%%%%%%%%%%%%%%%%%%%%%%%%%%%%%%%%%%%%%%%%%%%%%%%%%%%%%%%%%%
\subsection{Driver Development}
\label{sec:driver}

The \textsf{childdoc} mechanism can also be use for the development
of definition files such as \LaTeX{} styles or classes.
This case differs from the above setup with multiple parts
included by |\include| in that no |\includeonly| should be invoked.
This can be achieved by starting the include file
(before |\ProvidesPackage|) with:
%
\begin{center}
\begin{tabular}{l}
|\input{childdoc.def}|\\
|\childdocforward{|\textit{main}|}|\\
\end{tabular}
\end{center}
%
or alternatively with:
%
\begin{center}
\begin{tabular}{l}
|\input{childdoc.def}|\\
|\childdocby{|\textit{main}|}|\\
\end{tabular}
\end{center}
%
Both forms have slightly different effects as described above.
The main file is prepared as usual, see \secref{sec:include}.

%%%%%%%%%%%%%%%%%%%%%%%%%%%%%%%%%%%%%%%%%%%%%%%%%%%%%%%%%%%%%%%%%%%%%%%%%%%%%%%%
\subsection{Legacy Detection}
\label{sec:detection}

The directive |\childdocmain| in the main file can detect
whether the complete document or merely a child is to be compiled
even without using the directive |\childdocof|.
This method is deprecated because it is less robust
and there is no compelling reason to use it;
it is merely provided for backward compatibility
and it may be removed in future versions.

If the detection mechanism is to be used,
it is mandatory to correctly specify
the filename of the main file as the argument of |\childdocmain|:
%
\begin{center}
\begin{tabular}{l}
|\input{childdoc.def}|\\
|\childdocmain{|\textit{main}|}|\\
\end{tabular}
\end{center}
%
If |\jobname| does not match the argument \textit{main} of |\childdocmain|,
it is assumed that |\jobname| points to the child file to be compiled.
When using |\childdocmain| with the main file specified as argument,
it suffices to start a child file
with just |\input{|\textit{main}|}|
without loading of the package and using |\childdocof|.
If instead all processing is done
with the appropriate \textsf{childdoc} directives,
the argument of \textit{main} of |\childdocmain| can be empty.

An alternative version of the command line processing described
in \secref{sec:commandline} using the detection mechanism reads:
%
\begin{center}
|... -jobname "|\textit{target}|" "|[\textit{flags}]%
[|\def\jobname{|\textit{dest}|}|]|\input{|\textit{main}|}"|
\end{center}

%%%%%%%%%%%%%%%%%%%%%%%%%%%%%%%%%%%%%%%%%%%%%%%%%%%%%%%%%%%%%%%%%%%%%%%%%%%%%%%%
\subsection{Manual Code}
\label{sec:manual}

In case one cannot be certain whether the definitions file |childdoc.def|
is installed on the target \TeX{} distribution
and one prefers not to ship it,
it is conceivable to paste a few relevant commands into the sources.

To that end, drop all statements |\input{childdoc.def}|
and perform the replacements as outlined below.
Instead of |\childdocmain{|\textit{main}|}| add the following code
to the top of the main file:
%
\begin{center}
\begin{tabular}{l}
|\||ifdefined\childdocname\endinput\||fi\newif\ifchilddoc|\\
|\edef\childdocname{\scantokens\expandafter{\jobname\noexpand}}|\\
|\def\childdocmain{|\textit{main}|}\||ifx\childdocmain\childdocname\||else|\\
|\childdoctrue\includeonly{\childdocname}\let\jobname\childdocmain\||fi|\\
\end{tabular}
\end{center}
%
Instead of |\childdocof{|\textit{main}|}| just include the main file
at the top of each child file:
%
\begin{center}
|\input{|\textit{main}|}|
\end{center}
%
A simple redirection |\childdocforward{|\textit{dest}|}| is achieved by:
%
\begin{center}
|\def\jobname{|\textit{dest}|}\input{\jobname}|
\end{center}
%
The redirection with prefix
|\childdocforwardprefix[|\textit{prefix}|]{|\textit{dest}|}|
is accomplished by:
%
\begin{center}
\begin{tabular}{l}
|{\edef\jobname{\scantokens\expandafter{\jobname\noexpand}}|\\
|\def\redirectjob |\textit{prefix}|#1~~~{\gdef\jobname{|\textit{dest}|#1}}|\\
|\expandafter\redirectjob\jobname~~~}\input{\jobname}|
\end{tabular}
\end{center}

In an alternative approach,
child documents can be compiled by a specific command line
without additional code or specific definitions:
%
\begin{center}
|... -jobname "|\textit{target}|" "|[\textit{flags}]%
|\includeonly{|\textit{dest}|}\input{|\textit{main}|}"|
\end{center}
%

%%%%%%%%%%%%%%%%%%%%%%%%%%%%%%%%%%%%%%%%%%%%%%%%%%%%%%%%%%%%%%%%%%%%%%%%%%%%%%%%
%%%%%%%%%%%%%%%%%%%%%%%%%%%%%%%%%%%%%%%%%%%%%%%%%%%%%%%%%%%%%%%%%%%%%%%%%%%%%%%%
\section{Information}

%%%%%%%%%%%%%%%%%%%%%%%%%%%%%%%%%%%%%%%%%%%%%%%%%%%%%%%%%%%%%%%%%%%%%%%%%%%%%%%%
\subsection{Copyright}

Copyright \copyright{} 2017--2018 Niklas Beisert

This work may be distributed and/or modified under the
conditions of the \LaTeX{} Project Public License, either version 1.3
of this license or (at your option) any later version.
The latest version of this license is in
  \url{http://www.latex-project.org/lppl.txt}
and version 1.3 or later is part of all distributions of \LaTeX{}
version 2005/12/01 or later.

This work has the LPPL maintenance status `maintained'.

The Current Maintainer of this work is Niklas Beisert.

This work consists of the files |README.txt|, |childdoc.ins| and |childdoc.dtx|
as well as the derived files |childdoc.def|, |cdocsamp.tex|
with |cdocsch1.tex|, |cdocsch2.tex|, |cdocspt3.tex|, |cdocspt4.tex|,
|cdocsdrf.tex|, |cdocsfn1.tex|, |cdocsfn2.tex|
as well as |childdoc.pdf|.

%%%%%%%%%%%%%%%%%%%%%%%%%%%%%%%%%%%%%%%%%%%%%%%%%%%%%%%%%%%%%%%%%%%%%%%%%%%%%%%%
\subsection{Files and Installation}

The package consists of the files:
%
\begin{center}
\begin{tabular}{ll}
    |README.txt|   & readme file \\
    |childdoc.ins| & installation file \\
    |childdoc.dtx| & source file \\
    |childdoc.def| & definition file \\
    |cdocsamp.tex| & sample main file \\
    |cdocsch1.tex| & sample include file \\
    |cdocsch2.tex| & sample include file \\
    |cdocspt3.tex| & sample part file \\
    |cdocspt4.tex| & sample part file \\
    |cdocsdrf.tex| & sample redirection file \\
    |cdocsfn1.tex| & sample redirection file \\
    |cdocsfn2.tex| & sample redirection file \\
    |childdoc.pdf| & manual
\end{tabular}
\end{center}
%
The distribution consists of the files
|README.txt|, |childdoc.ins| and |childdoc.dtx|.
%
\begin{itemize}
\item
Run (pdf)\LaTeX{} on |childdoc.dtx|
to compile the manual |childdoc.pdf| (this file).
\item
Run \LaTeX{} on |childdoc.ins| to create the definitions file |childdoc.def|
and the sample |cdocsamp.tex| with include files
|cdocsch1.tex|, |cdocsch2.tex|, |cdocspt3.tex|, |cdocspt4.tex|,
|cdocsdrf.tex|, |cdocsfn1.tex|, |cdocsfn2.tex|.
Then copy the file |childdoc.def| to an appropriate directory of your \LaTeX{}
distribution, e.g.\ \textit{texmf-root}|/tex/latex/childdoc|.
\end{itemize}

%%%%%%%%%%%%%%%%%%%%%%%%%%%%%%%%%%%%%%%%%%%%%%%%%%%%%%%%%%%%%%%%%%%%%%%%%%%%%%%%
\subsection{Related CTAN Packages}

There are several other packages which offer a similar functionality:
%
\begin{itemize}
\item
The packages
\href{http://ctan.org/pkg/docmute}{\textsf{docmute}},
\href{http://ctan.org/pkg/includex}{\textsf{includex}} and
\href{http://ctan.org/pkg/standalone}{\textsf{standalone}}
provide commands to include only the document body of
a child file thus allowing both files to be compiled individually.
\item
The packages \href{http://ctan.org/pkg/subdocs}{\textsf{subdocs}}
and \href{http://ctan.org/pkg/subfiles}{\textsf{subfiles}}
provide structures in which the main and child documents can be
encapsulated and allowing them to be compiled individually.
The inclusion mechanism is different from the conventional |\include|.
\item
The package \href{http://ctan.org/pkg/combine}{\textsf{combine}}
is an elaborate solution to combine several documents into one.
\end{itemize}
%
See also the CTAN topic \href{http://ctan.org/topic/subdocs}{\textsf{subdocs}}
for further related packages.
The present package differs from the above solutions in that
a document structure constructed with the conventional |\include| mechanism
just needs two extra commands at the top of every file
such that all constituent files can be compiled individually.

%%%%%%%%%%%%%%%%%%%%%%%%%%%%%%%%%%%%%%%%%%%%%%%%%%%%%%%%%%%%%%%%%%%%%%%%%%%%%%%%
%\subsection{Feature Suggestions}
%
%The following is a list of features which may be useful for future
%versions of this package:
%%
%\begin{itemize}
%\item
%\ldots
%\end{itemize}

%%%%%%%%%%%%%%%%%%%%%%%%%%%%%%%%%%%%%%%%%%%%%%%%%%%%%%%%%%%%%%%%%%%%%%%%%%%%%%%%
\subsection{Revision History}

%%%%%%%%%%%%%%%%%%%%%%%%%%%%%%%%%%%%%%%%
\paragraph{v2.0:} 2018/12/30

\begin{itemize}
\item
immediate forward processing
\item
added |\childdocby| mechanism
\item
manual restructured
\end{itemize}

%%%%%%%%%%%%%%%%%%%%%%%%%%%%%%%%%%%%%%%%
\paragraph{v1.6:} 2018/01/17

\begin{itemize}
\item
application for development of include files
\item
corrections to manual
\end{itemize}

%%%%%%%%%%%%%%%%%%%%%%%%%%%%%%%%%%%%%%%%
\paragraph{v1.5:} 2017/05/21

\begin{itemize}
\item
more complete structuring introduced
\item
|\childdocof| introduced
\item
|\childdoc| renamed to |\childdocmain|
\item
|\childredirect| renamed to |\childdocforward| and |\childdocforwardprefix|
and functionality expanded
\end{itemize}

%%%%%%%%%%%%%%%%%%%%%%%%%%%%%%%%%%%%%%%%
\paragraph{v1.0:} 2017/04/27

\begin{itemize}
\item
manual and install package
\item
first version published on CTAN
\end{itemize}

%%%%%%%%%%%%%%%%%%%%%%%%%%%%%%%%%%%%%%%%
\paragraph{v0.6:} 2017/04/26

\begin{itemize}
\item
redirection mechanism added
\end{itemize}

%%%%%%%%%%%%%%%%%%%%%%%%%%%%%%%%%%%%%%%%
\paragraph{v0.5:} 2017/04/26

\begin{itemize}
\item
functionality in definition file
\end{itemize}


%%%%%%%%%%%%%%%%%%%%%%%%%%%%%%%%%%%%%%%%%%%%%%%%%%%%%%%%%%%%%%%%%%%%%%%%%%%%%%%%
%%%%%%%%%%%%%%%%%%%%%%%%%%%%%%%%%%%%%%%%%%%%%%%%%%%%%%%%%%%%%%%%%%%%%%%%%%%%%%%%
%%%%%%%%%%%%%%%%%%%%%%%%%%%%%%%%%%%%%%%%%%%%%%%%%%%%%%%%%%%%%%%%%%%%%%%%%%%%%%%%
\appendix

\settowidth\MacroIndent{\rmfamily\scriptsize 000\ }

 \DocInput{childdoc.dtx}

\end{document}
%</driver>
% \fi
%
% %%%%%%%%%%%%%%%%%%%%%%%%%%%%%%%%%%%%%%%%%%%%%%%%%%%%%%%%%%%%%%%%%%%%%%%%%%%%%%
% %%%%%%%%%%%%%%%%%%%%%%%%%%%%%%%%%%%%%%%%%%%%%%%%%%%%%%%%%%%%%%%%%%%%%%%%%%%%%%
% \section{Sample}
%\iffalse
%<*samplemain>
%\fi
%
% The following presents a sample document
% with two chapters, two parts, a title page,
% a compile flag as well as three forwarding files to set the flag.
% It consists of eight |.tex| files:
% \begin{center}
% \begin{tabular}{ll}
% |cdocsamp.tex|&main file\\
% |cdocsch1.tex|&include file for chapter 1\\
% |cdocsch2.tex|&include file for chapter 2\\
% |cdocspt3.tex|&include file for part 3\\
% |cdocspt4.tex|&include file for part 4\\
% |cdocsdrf.tex|&forwarding file for main file in draft mode\\
% |cdocsfi1.tex|&forwarding file for final version of chapter 1\\
% |cdocsfi2.tex|&forwarding file for final version of chapter 2\\
% \end{tabular}
% \end{center}
% Each of the eight files can be compiled directly by the \LaTeX{} compiler.
%
% %%%%%%%%%%%%%%%%%%%%%%%%%%%%%%%%%%%%%%
% \paragraph{Main File.}
%
% The main file is called |cdocsamp.tex|.
%
% Load the \textsf{childdoc} definitions and
% declare the filename for the main document:
%    \begin{macrocode}
\input{childdoc.def}
\childdocmain{}
%    \end{macrocode}

% Optional override for |\version| flag:
%    \begin{macrocode}
%%\ifchilddoc\else\providecommand{\version}{draft}\fi
%    \end{macrocode}

% Define the default values for the |\version| flag
% (|final| for the main file and |draft| for childs):
%    \begin{macrocode}
\ifchilddoc
\providecommand{\version}{draft}
\else
\providecommand{\version}{final}
\fi
%    \end{macrocode}

% Load the standard document class:
%    \begin{macrocode}
\documentclass[12pt]{article}
%    \end{macrocode}

% Start the document body:
%    \begin{macrocode}
\begin{document}
%    \end{macrocode}

% Declare a title page.
% Print title, part of document being processed and version flag:
%    \begin{macrocode}
\addtocounter{page}{-1}
\begin{center}
{\LARGE\bfseries{}childdoc example\par}
\vspace{1cm}
\ifchilddoc
\ifchilddocmanual part\else chapter\fi:
`\childdocname' of `\childdocjob'\par
\else
main document: `\childdocjob'\par
\fi
version: \version\par
\end{center}
\newpage
%    \end{macrocode}

% Manually include selected file,
% otherwise process as usual:
%    \begin{macrocode}
\ifchilddocmanual
\section*{part `\childdocname'}
\input{\childdocname}
\else
%    \end{macrocode}

% Include the two chapters:
%    \begin{macrocode}
\include{cdocsch1}
\include{cdocsch2}
%    \end{macrocode}

% Include the two parts unless only chapters should be displayed:
%    \begin{macrocode}
\ifchilddoc\else
\section{part three}
\input{cdocspt3}
\section{part four}
\input{cdocspt4}
\fi
%    \end{macrocode}

% Process as usual until here:
%    \begin{macrocode}
\fi
%    \end{macrocode}

% End of document body:
%    \begin{macrocode}
\end{document}
%    \end{macrocode}
%\iffalse
%</samplemain>
%\fi
%
% %%%%%%%%%%%%%%%%%%%%%%%%%%%%%%%%%%%%%%
% \paragraph{Chapter Include Files.}
%
% The include files are called |cdocsch1.tex| and |cdocsch2.tex|.
%
%\iffalse
%<*samplechap1|samplechap2>
%\fi

% Optional override for |\version| flag:
%    \begin{macrocode}
%%\providecommand{\version}{final}
%    \end{macrocode}

% Include the main document:
%    \begin{macrocode}
\input{childdoc.def}
\childdocof{cdocsamp}
%    \end{macrocode}

%\iffalse
%</samplechap1|samplechap2>
%\fi
%
%\iffalse
%<*samplechap1>
%\fi
% Some text for chapter 1:
%    \begin{macrocode}
\section{one}
some text in chapter one
%    \end{macrocode}

%\iffalse
%</samplechap1>
%\fi
% Some text for chapter 2:
%\iffalse
%<*samplechap2>
%\fi
%    \begin{macrocode}
\section{two}
more text in chapter two
%    \end{macrocode}

%\iffalse
%</samplechap2>
%\fi
%
% %%%%%%%%%%%%%%%%%%%%%%%%%%%%%%%%%%%%%%
% \paragraph{Part Include Files.}
%
% The include files are called |cdocspt3.tex| and |cdocspt4.tex|.
%
%\iffalse
%<*samplepart3|samplepart4>
%\fi

% Optional override for |\version| flag:
%    \begin{macrocode}
%%\providecommand{\version}{final}
%    \end{macrocode}

% Include the main document:
%    \begin{macrocode}
\input{childdoc.def}
\childdocby{cdocsamp}
%    \end{macrocode}

%\iffalse
%</samplepart3|samplepart4>
%\fi
%
%\iffalse
%<*samplepart3>
%\fi
% Some text for part 3:
%    \begin{macrocode}
some text in part three
%    \end{macrocode}

%\iffalse
%</samplepart3>
%\fi
% Some text for part 4:
%\iffalse
%<*samplepart4>
%\fi
%    \begin{macrocode}
more text in part four
%    \end{macrocode}

%\iffalse
%</samplepart4>
%\fi
%
% %%%%%%%%%%%%%%%%%%%%%%%%%%%%%%%%%%%%%%
% \paragraph{Forwarding for a Complete Draft.}
%
% The following forwarding file |cdocsdrf.tex|
% compiles the main document in draft mode:
%\iffalse
%<*sampledraft>
%\fi
%    \begin{macrocode}
\def\version{draft}
\input{childdoc.def}
\childdocforward{cdocsamp}
%    \end{macrocode}

%\iffalse
%</sampledraft>
%\fi
%
% %%%%%%%%%%%%%%%%%%%%%%%%%%%%%%%%%%%%%%
% \paragraph{Forwarding for Final Version of the Chapters.}
%
% The following forwarding files |cdocsfn1.tex| and |cdocsfn2.tex|
% (with identical content)
% compile the final versions of the child documents
% |cdocsch1.tex| and |cdocsch2.tex|, respectively:
%\iffalse
%<*samplefinal>
%\fi
%    \begin{macrocode}
\def\version{final}
\input{childdoc.def}
\childdocforwardprefix[cdocsamp]{cdocsfn}{cdocsch}
%    \end{macrocode}

%\iffalse
%</samplefinal>
%\fi
%
% %%%%%%%%%%%%%%%%%%%%%%%%%%%%%%%%%%%%%%
% \paragraph{Command Line Processing.}
%
% The following three command lines generate the output files
% |cdocscld|, |cdocscl1| and |cdocscl2|
% which should be identical to
% |cdocsdrf|, |cdocsch1| and |cdocsfn2|, respectively:
% \begin{center}
% \begin{tabular}{l}
% |latex -jobname cdocscld \|\\
% |  "\def\version{draft}\input{childdoc.def}\childdocforward{cdocsamp}"|\\
% |latex -jobname cdocscl1 \|\\
% |  "\input{childdoc.def}\childdocforward[cdocsamp]{cdocsch1}"|\\
% |latex -jobname cdocscl2 \|\\
% |  "\def\version{final}\input{childdoc.def}\childdocforward{cdocsch2}"|
% \end{tabular}
% \end{center}
% Note that the trailing backslash on each first line
% merely continues the input to the second line
% (for convenient cut ant paste).
% Furthermore, the command |latex| can be replaced by any
% of its alternative versions such as |pdflatex|.
%
% %%%%%%%%%%%%%%%%%%%%%%%%%%%%%%%%%%%%%%%%%%%%%%%%%%%%%%%%%%%%%%%%%%%%%%%%%%%%%%
% %%%%%%%%%%%%%%%%%%%%%%%%%%%%%%%%%%%%%%%%%%%%%%%%%%%%%%%%%%%%%%%%%%%%%%%%%%%%%%
% \section{Implementation}
%\iffalse
%<*package>
%\fi
%
% This section describes the definitions file |childdoc.def|.

% The definitions cannot be loaded using |\usepackage| or |\RequirePackage|
% which has a mechanism to prevent loading a style file more than once.
% When loading the definitions by means of |\input|
% multiple instances have to be prevented manually:
%\iffalse
%This code needs to be before the `\ProvidesFile' directive
%which is defined at the beginning of this file.
%Therefore it is also placed there and commented out here.
%</package>
%<*discard>
%\fi
%    \begin{macrocode}
\ifdefined\childdocmain\endinput\fi
%    \end{macrocode}
%\iffalse
%</discard>
%<*package>
%\fi
%
% \macro{\ifchilddoc}
% \macro{\ifchilddocmanual}
% The conditional |\ifchilddoc| tells whether a
% child (true) or main (false) document is being compiled.
% The conditional |\ifchilddocmanual| tells whether
% the |\includeonly| mechanism is used (false) or
% the selection of child files must be performed manually (true).
% The definitions initialise to false:
%    \begin{macrocode}
\newif\ifchilddoc
\newif\ifchilddocmanual
%    \end{macrocode}

% \macro{\childdocname}
% \macro{\childdocjob}
% The macro |\childdocname| stores the name of the main document
% to be compiled. The macro |\childdocjob| stores the name of
% the document on which the \LaTeX{} compiler was originally invoked.
% The content of |\jobname| cannot be compared
% to filenames specified in the source due to different catcodes.
% The following code rescans |\jobname|, stores the result
% in |\childdocname| and saves a copy in |\childdocjob|:
%    \begin{macrocode}
\edef\childdocname{\scantokens\expandafter{\jobname\noexpand}}
\let\childdocjob\childdocname
%    \end{macrocode}

% \macro{\childdocdisable}
% The macro |\childdocdisable| prevents the main file
% from being processed more than once.
% At this stage, the main document command |\childdocmain|
% is assumed to be called once again where it should do nothing.
% Any subsequent call to it should prevent
% a secondary processing of the main document
% It overwrites the forwarding commands
% |\childdocof| and |\childdocforward|
% with empty macros to prevent further inclusions of the main document:
%    \begin{macrocode}
\newcommand{\childdocdisable}
{
  \renewcommand{\childdocmain}[1]{\renewcommand{\childdocmain}[1]{\endinput}}
  \renewcommand{\childdocof}[1]{}
  \renewcommand{\childdocby}[2][]{}
  \renewcommand{\childdocforward}[2][]{}
  \renewcommand{\childdocdisable}{}
}
%    \end{macrocode}

% \macro{\childdocmain}
% The macro |\childdocmain| is to be called at the top of the main file
% with nothing or the main filename (without extension) as argument.
% First, it breaks loops.
% If the argument is not empty and does not match |\childdocname|
% (which is set by the first inclusion of |childdoc.def|),
% |\ifchilddoc| is set to true, |\includeonly| is applied to the child file
% and |\jobname| is set to the main file
% (for proper handling of |.aux| files):
%    \begin{macrocode}
\newcommand{\childdocmain}[1]
{
  \childdocdisable\childdocmain{}
  \if?#1?\else
    \begingroup
      \def\childdoctmp{#1}
      \ifx\childdoctmp\childdocname
        \def\childdoctmp{}
      \else
        \def\childdoctmp
        {
          \childdoctrue
          \includeonly{\childdocname}
          \def\childdocjob{#1}
          \def\jobname{#1}
        }
      \fi
      \expandafter
    \endgroup
    \childdoctmp
  \fi
}
%    \end{macrocode}

% \macro{\childdocof}
% The command |\childdocof| redirects
% compilation to the main file |#1|.
%    \begin{macrocode}
\newcommand{\childdocof}[1]
{
  \childdocdisable
  \childdoctrue
  \includeonly{\childdocname}
  \def\jobname{#1}
  \def\childdocjob{#1}
  \input{#1}
}
%    \end{macrocode}

% \macro{\childdocby}
% The command |\childdocby| ....
%    \begin{macrocode}
\newcommand{\childdocby}[2][]
{
  \childdocdisable
  \childdoctrue
  \childdocmanualtrue
  \if?#1?\else
    \def\jobname{#2}
  \fi
  \def\childdocjob{#2}
  \input{#2}
  \endinput
}
%    \end{macrocode}

% \macro{\childdocforward}
% The command |\childdocforward| redirects
% compilation to the main file or
% (if the optional argument is given) a child file.
% Parameters are set as if the main file
% or a child file starting with |\childdocof| was compiled.
% Then compilation is handed over to the main file:
%    \begin{macrocode}
\newcommand{\childdocforward}[2][]
{
  \begingroup
    \if?#1?
      \def\childdoctmp
      {
        \def\childdocname{#2}
        \def\childdocjob{#2}
        \def\jobname{#2}
        \input{#2}
        \endinput
      }
    \else
      \def\childdoctmp
      {
        \childdocdisable
        \def\childdocname{#2}
        \childdoctrue
        \includeonly{#2}
        \def\childdocjob{#1}
        \def\jobname{#1}
        \input{#1}
        \endinput
      }
    \fi
    \expandafter
  \endgroup
  \childdoctmp
}
%    \end{macrocode}

% \macro{\childdocforwardprefix}
% The command |\childdocforwardprefix| redirects
% compilation to the main or a child file by means of a pattern.
% The prefix |#1| in the current filename is replaced by |#2|
% and the suffix of the current filename is kept
% (it is assumed that the filename does not contain the substring `|~~~|'
% which is used as a delimiter).
% Compilation is handed over to the new file by |\childdocforward|:
%    \begin{macrocode}
\newcommand{\childdocforwardprefix}[3][]
{
  \begingroup
    \def\childdocextract #2##1~~~{\def\childdoctmp{\childdocforward[#1]{#3##1}}}
    \expandafter\childdocextract\childdocname~~~
    \expandafter
  \endgroup
  \childdoctmp
}
%    \end{macrocode}

% \macro{\childdoc}
% The deprecated macro |\childdoc| is a legacy version of |\childdocmain|:
%    \begin{macrocode}
\newcommand{\childdoc}{\childdocmain}
%    \end{macrocode}

% \macro{\childdocredirect}
% The deprecated macro |\childdocredirect| is a legacy version
% of |\childdocforward| and |\childdocforwardprefix|:
%    \begin{macrocode}
\newcommand{\childdocredirect}[2][]
{
  \begingroup
    \if?#1?
      \def\childdoctmp{\childdocforward{#2}}
    \else
      \def\childdoctmp{\childdocforwardprefix{#1}{#2}}
    \fi
    \expandafter
  \endgroup
  \childdoctmp
}
%    \end{macrocode}

%\iffalse
%</package>
%\fi
%
\endinput
|\\
|\childdocmain{}|\\
\end{tabular}
\end{center}
at the very top of the main \LaTeX{} file,
in particular \emph{before} the |\documentclass| statement!
The argument of |\childdocmain| should be left empty
(but it must be present).

%%%%%%%%%%%%%%%%%%%%%%%%%%%%%%%%%%%%%%%%
\DescribeMacro{\childdocof}
Furthermore, add the commands
\begin{center}
\begin{tabular}{l}
|% \iffalse
%
% childdoc.dtx Copyright (C) 2017-2018 Niklas Beisert
%
% This work may be distributed and/or modified under the
% conditions of the LaTeX Project Public License, either version 1.3
% of this license or (at your option) any later version.
% The latest version of this license is in
%   http://www.latex-project.org/lppl.txt
% and version 1.3 or later is part of all distributions of LaTeX
% version 2005/12/01 or later.
%
% This work has the LPPL maintenance status `maintained'.
%
% The Current Maintainer of this work is Niklas Beisert.
%
% This work consists of the files childdoc.dtx and childdoc.ins
% and the derived files childdoc.def and cdocsamp.tex with
% cdocsch1.tex, cdocsch2.tex, cdocsdrf.tex, cdocsfn1.tex, cdocsfn2.tex.
%
%<package>\ifdefined\childdocmain\endinput\fi
%<package>\ProvidesFile{childdoc.def}[2018/12/30 v2.0 child document driver]
%<samplemain>\ProvidesFile{cdocsamp.tex}[2018/12/30 v2.0 sample for childdoc]
%<*driver>
%\ProvidesFile{childdoc.drv}[2018/12/30 v2.0 childdoc reference manual file]
\PassOptionsToClass{10pt,a4paper}{article}
\documentclass{ltxdoc}

\usepackage[margin=35mm]{geometry}
\usepackage{hyperref}
\usepackage{hyperxmp}
\usepackage[usenames]{color}

\hypersetup{colorlinks=true}
\hypersetup{pdfstartview=FitH}
\hypersetup{pdfpagemode=UseNone}
\hypersetup{pdfsource={}}
\hypersetup{pdflang={en-UK}}
\hypersetup{pdfcopyright={Copyright 2017-2018 Niklas Beisert.
  This work may be distributed and/or modified under the
  conditions of the LaTeX Project Public License, either version 1.3
  of this license or (at your option) any later version.}}
\hypersetup{pdflicenseurl={http://www.latex-project.org/lppl.txt}}
\hypersetup{pdfcontactaddress={ETH Zurich, ITP, HIT K,
  Wolfgang-Pauli-Strasse 27}}
\hypersetup{pdfcontactpostcode={8093}}
\hypersetup{pdfcontactcity={Zurich}}
\hypersetup{pdfcontactcountry={Switzerland}}
\hypersetup{pdfcontactemail={nbeisert@itp.phys.ethz.ch}}
\hypersetup{pdfcontacturl={http://people.phys.ethz.ch/\xmptilde nbeisert/}}

\newcommand{\secref}[1]{\hyperref[#1]{section \ref*{#1}}}

\parskip1ex
\parindent0pt
\let\olditemize\itemize
\def\itemize{\olditemize\parskip0pt}

\begin{document}

\title{The \textsf{childdoc} Package}
\hypersetup{pdftitle={The childdoc Package}}
\author{Niklas Beisert\\[2ex]
  Institut f\"ur Theoretische Physik\\
  Eidgen\"ossische Technische Hochschule Z\"urich\\
  Wolfgang-Pauli-Strasse 27, 8093 Z\"urich, Switzerland\\[1ex]
  \href{mailto:nbeisert@itp.phys.ethz.ch}
  {\texttt{nbeisert@itp.phys.ethz.ch}}}
\hypersetup{pdfauthor={Niklas Beisert}}
\hypersetup{pdfsubject={Manual for the LaTeX2e Package childdoc}}
\date{30 December 2018, \textsf{v2.0}}
\maketitle

\begin{abstract}\noindent
\textsf{childdoc} is a \LaTeXe{} package
that enables the direct compilation
of document sections included by |\include|
to individual files.
\end{abstract}

\begingroup
\parskip0ex
\tableofcontents
\endgroup

%%%%%%%%%%%%%%%%%%%%%%%%%%%%%%%%%%%%%%%%%%%%%%%%%%%%%%%%%%%%%%%%%%%%%%%%%%%%%%%%
%%%%%%%%%%%%%%%%%%%%%%%%%%%%%%%%%%%%%%%%%%%%%%%%%%%%%%%%%%%%%%%%%%%%%%%%%%%%%%%%
\section{Introduction}

\LaTeX{} provides a mechanism to structure a large document (such as a book)
into a main file and several child files (containing the chapters)
using the |\include| command.
This mechanism is beneficial for documents
which span hundreds of pages in order to
make the source file(s) more manageable.
Moreover, compilation can be restricted to
selected child files by means of the |\includeonly| command.
The latter feature can be used to reduce the compilation time while editing
(this was significantly more useful in the earlier days of \LaTeX{})
or to generate a smaller document which is easier to navigate.
Another application of |\includeonly| is to generate
documents consisting of selected parts of the complete document.

However, there are a few drawbacks of the plain |\include| mechanism:
\begin{itemize}
\item
The child files cannot be compiled on their own,
they can only be compiled via the main file.
A naive editing environment
(such as a text editor with an option
to have the current file processed by \LaTeX)
may require one to switch to the main file before compiling;
attempting to compile the child file produces errors.
\item
The main file must be modified (each time)
to adjust the |\includeonly| command
to the present needs. This easily leaves the main file in a messy state.
\item
The generated document will always carry the filename
of the main document. This is inconvenient if
several child files are to be compiled and
to be kept for distribution.
\end{itemize}

The present package provides a simple interface
to make child files individually compilable by \LaTeX{}.
Compiling a child file then has the same effect as compiling
the main file with an |\includeonly| command
to select the appropriate child.
Moreover the generated document will carry the name of the child
rather than the main file.
This resolves all three above issues.

This feature is meant to make the editing of books,
thesis documents and lecture notes somewhat more convenient.
However, the package can also be used efficiently for
composing a series of documents (such as exercise sheets)
which are typically distributed individually.
It then assists the author in generating the individual documents
(potentially in different versions)
as well as a document containing the collected series.
Another application is in developing style files
or other kinds of included material
where compilation of the style file could redirect
to a sample or test file.

%%%%%%%%%%%%%%%%%%%%%%%%%%%%%%%%%%%%%%%%%%%%%%%%%%%%%%%%%%%%%%%%%%%%%%%%%%%%%%%%
%%%%%%%%%%%%%%%%%%%%%%%%%%%%%%%%%%%%%%%%%%%%%%%%%%%%%%%%%%%%%%%%%%%%%%%%%%%%%%%%
\section{Usage}

First of all, the package \textsf{childdoc} is \emph{not} a standard
\LaTeXe{} |.sty| style file! Therefore it needs to be invoked in
a non-standard way.

%%%%%%%%%%%%%%%%%%%%%%%%%%%%%%%%%%%%%%%%%%%%%%%%%%%%%%%%%%%%%%%%%%%%%%%%%%%%%%%%
\subsection{Included Files}
\label{sec:include}

%%%%%%%%%%%%%%%%%%%%%%%%%%%%%%%%%%%%%%%%
\DescribeMacro{\childdocmain}
To use the package, add the commands
\begin{center}
\begin{tabular}{l}
|\input{childdoc.def}|\\
|\childdocmain{}|\\
\end{tabular}
\end{center}
at the very top of the main \LaTeX{} file,
in particular \emph{before} the |\documentclass| statement!
The argument of |\childdocmain| should be left empty
(but it must be present).

%%%%%%%%%%%%%%%%%%%%%%%%%%%%%%%%%%%%%%%%
\DescribeMacro{\childdocof}
Furthermore, add the commands
\begin{center}
\begin{tabular}{l}
|\input{childdoc.def}|\\
|\childdocof{|\textit{main}|}|\\
\end{tabular}
\end{center}
at the top of every child file \textit{child}
which is included by |\include{|\textit{child}|}|
from within the main file
(or at least for those files to be compiled individually).
The argument \textit{main} must be the filename of the main file.

There are a couple of
considerations in setting up the main and child documents:

%%%%%%%%%%%%%%%%%%%%%%%%%%%%%%%%%%%%%%%%
\paragraph{Restrictions.}

Please note the following restrictions:
\begin{itemize}
\item
|\childdocmain| must be called with one argument \textit{main}
to ensure compatibility with earlier version of the package.
It must either be empty (|\childdocmain{}|)
or precisely match the filename of the main file in which it is specified.
See \secref{sec:detection} for further information.
\item
The filename \textit{main} must be specified without the |.tex| extension.
\item
The filename \textit{main} is case sensitive
(even in case-insensitive file systems)
due to internal string comparison.
\item
The argument \textit{main} should be fully expanded, it cannot be a macro.
\item
Subdirectories and special characters should be avoided in filenames.
\item
The command |\childdocmain{|\textit{main}|}| must be followed by a whitespace.
It should not be followed immediately by another command
or by a comment mark `|%|'.
This is because the \TeX{} parser reads the token immediately following
the argument of |\childdocmain| and puts it
at the beginning of every child section;
however, a white\-space is ignored.
\end{itemize}

%%%%%%%%%%%%%%%%%%%%%%%%%%%%%%%%%%%%%%%%
\paragraph{Content of Main File.}

It is advisable to place all content in the child files included by |\include|.
Any output contained in the main file will appear in all child documents
unless suppressed manually;
it cannot be suppressed automatically by the |\includeonly| directive
and thus should normally be avoided.
A method to include some content in the main file
by means of conditional processing is described in \secref{sec:conditional}.

%%%%%%%%%%%%%%%%%%%%%%%%%%%%%%%%%%%%%%%%
\paragraph{Page Numbering.}

When only a part of the document is compiled,
the appropriate numbering of pages
(as well as other status parameters)
is determined from the |.aux| files.
The latter contain information from previous passes.
However this information needs to propagate through
all intermediate child documents.
Therefore the page numbering in child documents may well
be inconsistent until the complete document is compiled at least once.

A useful (if unconventional) way to always ensure a consistent
page numbering is to restart the numbering in each child document
and denote the pages by `\textit{child}|.|\textit{page}'
where \textit{child} represents the chapter/section number of the child file.
This can be achieved by the command
|\numberwithin{page}{|\textit{child}|}|
of the \textsf{amsmath} package
where \textit{child} can be |chapter| or |section|
depending on the chosen structuring.
Alternatively, one can modify the macro |\thepage| appropriately
and reset the counter |page| at the start of each child file.

%%%%%%%%%%%%%%%%%%%%%%%%%%%%%%%%%%%%%%%%%%%%%%%%%%%%%%%%%%%%%%%%%%%%%%%%%%%%%%%%
\subsection{Conditional Processing}
\label{sec:conditional}

The package provides a mechanism to compile different versions
of a document. To customise the versions further some conditional processing
can come in handy to distinguish which version is being compiled.
The package provides two macros to describe the compilation context:

%%%%%%%%%%%%%%%%%%%%%%%%%%%%%%%%%%%%%%%%
\DescribeMacro{\ifchilddoc}
The conditional |\ifchilddoc| distinguishes between the compilation of
child documents and the main document:
%
\begin{center}
|\ifchilddoc |\textit{child-code}| |[|\||else |\textit{main-code}]| \||fi|
\end{center}

%%%%%%%%%%%%%%%%%%%%%%%%%%%%%%%%%%%%%%%%
\DescribeMacro{\childdocname}
\DescribeMacro{\childdocjob}
The macro |\childdocname| contains the filename (without extension)
of the main or child file being processed.
Note that |\childdocjob| will always contain the name of the main file.

%%%%%%%%%%%%%%%%%%%%%%%%%%%%%%%%%%%%%%%%
\paragraph{Title Page.}

Conditional processing can be used to include a title or banner page
in the main document when proper precautions are taken.
Importantly, the code in the main file should ensure that the page counter
(as well as other status parameters which are stored in the |.aux| files)
takes the same value after the conditional processing.
Otherwise the page numbers may take divergent values
depending on which part is compiled.

For example, a title page could be declared by:
%
\begin{center}
\begin{tabular}{l}
|\ifchilddoc\||else|\\
|\addtocounter{page}{-1}|\\
\textit{code for title page}\\
|\newpage|\\
|\||fi|
\end{tabular}
\end{center}
%
A banner page for the child documents can be generated by:
%
\begin{center}
\begin{tabular}{l}
|\ifchilddoc|\\
|\addtocounter{page}{-1}|\\
\textit{code for banner page}\\
|\newpage|\\
|\||fi|
\end{tabular}
\end{center}
%
Here one could write a message such as:
\begin{center}
|This is the part \childdocname{} of \childdocjob{}.|
\end{center}

%%%%%%%%%%%%%%%%%%%%%%%%%%%%%%%%%%%%%%%%%%%%%%%%%%%%%%%%%%%%%%%%%%%%%%%%%%%%%%%%
\subsection{Flags}
\label{sec:flags}

The package makes it easy to generate different versions
of the main or child documents.
To this end compilation flags can be defined
and assigned different default values.
They will be particularly useful in conjunction
with the forwarding mechanism described in \secref{sec:forward}.

For example, it may be useful to have a flag |\version|
which can be set to |draft| or |final|.
The document source will contain some conditional code
depending on the value of |\version|.
Suppose further, the flag should default to |final| for the main file
and to |draft| for child files
which is a natural assignment for editing the document.
This is achieved by placing the following code
in the preamble of the main document
(below the |\childdocmain| directive):
%
\begin{center}
\begin{tabular}{l}
|\ifchilddoc|\\
|\providecommand{\version}{draft}|\\
|\||else|\\
|\providecommand{\version}{final}|\\
|\||fi|
\end{tabular}
\end{center}
%
The definition by |\providecommand| makes sure
that previous definitions are not overwritten.
Further statements |\providecommand{\version}{...}|
can thus be added before the above code to override it.

For the main file, one might add a line
(between |\childdocmain| and the above block)
%
\begin{center}
|%\ifchilddoc\||else\providecommand{\version}{draft}\||fi|
\end{center}
%
which can be uncommented to produce a draft version.
Likewise one can add a line to the very top of a child file
(above the |\childdocof{|\textit{main}|}| directive)
%
\begin{center}
|%\providecommand{\version}{final}|
\end{center}
%
which can be uncommented to produce the final version of this child document.

%%%%%%%%%%%%%%%%%%%%%%%%%%%%%%%%%%%%%%%%%%%%%%%%%%%%%%%%%%%%%%%%%%%%%%%%%%%%%%%%
\subsection{Forwarding}
\label{sec:forward}

Different versions of the main or child documents
using compilation flags as described in \secref{sec:flags}
can be (permanently) stored in different files
for convenient compilation, viewing and distribution.
To this end, the package defines a command
to pass on compilation to a different file:

%%%%%%%%%%%%%%%%%%%%%%%%%%%%%%%%%%%%%%%%
\DescribeMacro{\childdocforward}
The command |\childdocforward| redirects processing to
another source file:
%
\begin{center}
\begin{tabular}{l}
|\input{childdoc.def}|\\
|\childdocforward[|\textit{main}|]{|\textit{dest}|}|\\
\end{tabular}
\end{center}
%
The argument \textit{dest} is the destination file
(without extension).
It should be the main file or one of the child files.
Note that further \textsf{childdoc} directives
such as |\childdocof| and |\childdocforward|
in the indicated file will be processed in this form.
The optional argument \textit{main}
passes on directly to the main file \textit{main}
while pretending to compile the child \textit{dest}.
This form behaves as if \textit{dest}
issues |\childdocof{|\textit{main}|}| right away,
and no further \textsf{childdoc} directives will be processed.

%%%%%%%%%%%%%%%%%%%%%%%%%%%%%%%%%%%%%%%%
\DescribeMacro{\...prefix}
In the alternative form |\childdocforwardprefix|,
%
\begin{center}
\begin{tabular}{l}
|\input{childdoc.def}|\\
|\childdocforwardprefix[|\textit{main}|]{|\textit{prefix}|}{|\textit{dest}|}|
\end{tabular}
\end{center}
%
the destination file is determined by a pattern
depending on the current file:
To make this work, the current file must be called
`{\textit{prefix}\hspace{0.2em}\textit{suffix}}'
with \textit{prefix} matching precisely the argument.
Processing is then passed on to the file
`{\textit{dest}\hspace{0.2em}\textit{suffix}}'.
Surely, the same effect is achieved by
directly specifying the
argument `{\textit{dest}\hspace{0.2em}\textit{suffix}}'
in the first form.
However, that requires to set up a different file
for each child. With the alternative form of the command
all these files can have exactly the same content
which simplifies setting them up and maintaining them.

For example, the following file |draft.tex|
with a compilation flag |\version| as described in \secref{sec:flags}
compiles the main document as a draft:
%
\begin{center}
\begin{tabular}{l}
|\def\version{draft}|\\
|\input{childdoc.def}|\\
|\childdocforward{|\textit{main}|}|
\end{tabular}
\end{center}
%
Likewise, the following files |final|\textit{nn}|.tex|
compile the final version of the child document
|child|\textit{nn}|.tex|:
%
\begin{center}
\begin{tabular}{l}
|\def\version{final}|\\
|\input{childdoc.def}|\\
|\childdocforwardprefix{final}{child}|
\end{tabular}
\end{center}
%

Note that when several versions of a main file and/or of each child file
are to be generated, it may be convenient to set up a |Makefile| or
shell script to automatise the process.

%%%%%%%%%%%%%%%%%%%%%%%%%%%%%%%%%%%%%%%%%%%%%%%%%%%%%%%%%%%%%%%%%%%%%%%%%%%%%%%%
\subsection{Command Line Processing}
\label{sec:commandline}

The effect of redirection files can also be achieved by invoking
the \LaTeX{} compiler with a more elaborate command line.
Most conveniently this should be done as part
of a shell script or a |Makefile|.

When using \textsf{childdoc} in the main file, the following
command lines effectively perform a redirection
(note that depending on the shell being used,
backslashes may have to be doubled: `|\|' $\to$ `|\\|'):
%
\begin{center}
|... -jobname "|\textit{target}|" |\\|"|[\textit{flags}]%
|\input{childdoc.def}\childdocforward[|\textit{main}|]{|\textit{dest}|}"|
\end{center}
%
Here \textit{target} is the name of the output file,
\textit{main} is the name of the main file
and \textit{dest} is the name of the main or child file to be processed
(all filenames without extensions).
The optional argument \textit{main} can be omitted
if \textit{main} matches \textit{dest}.
Optionally, compilation \textit{flags} can be defined via |\def| commands.
This command line makes the \TeX{} engine believe
it is compiling the file \textit{target}
whose content is specified as the latter parameter.
The provided code then forwards the processing to
\textit{main} or \textit{dest} as described in \secref{sec:forward}.

%%%%%%%%%%%%%%%%%%%%%%%%%%%%%%%%%%%%%%%%%%%%%%%%%%%%%%%%%%%%%%%%%%%%%%%%%%%%%%%%
\subsection{Include by Input}
\label{sec:input}

Including child documents by |\include| has some restrictions by design.
Most notably, the content of a child document always occupies
its own set of pages; pages cannot be shared between child documents.
Usually, this behaviour makes perfect sense
because each child document contain an essential part of the document.
However, in some situations it may be desirable to compose
a document from a collection of parts
without having mandatory page breaks between then.
For this case, the package
provides a mechanism to include parts
by |\input| which can also be processed individually.
However, by construction this mechanism
requires manual handling of the content to be output.

%%%%%%%%%%%%%%%%%%%%%%%%%%%%%%%%%%%%%%%%
\DescribeMacro{\ifchilddocmanual}
The main file should be prepared as usual, see \secref{sec:include}.
However, the document body must make a distinction
between processing of an individual part and of the main document, e.g.:
%
\begin{center}
\begin{tabular}{l}
|\ifchilddocmanual|\\
|\input{\childdocname}|\\
|\||else|\\
\textit{document body with }|\input{|\textit{part}|}|\\
|\||fi|
\end{tabular}
\end{center}
%
The conditional |\ifchilddocmanual| is true whenever
a part to be included by |\input| is being compiled,
and the name of the part is stored in |\childdocname|.

%%%%%%%%%%%%%%%%%%%%%%%%%%%%%%%%%%%%%%%%
\DescribeMacro{\childdocby}
Each part to be included by |\input| should start with:
%
\begin{center}
\begin{tabular}{l}
|\input{childdoc.def}|\\
|\childdocby{|\textit{main}|}|\\
\end{tabular}
\end{center}
%
The directive |\childdocby| is similar to |\childdocof|
described in \secref{sec:include},
but the subsequent selection of content must be done manually.
To that end, both |\ifchilddoc| and |\ifchilddocmanual|
will be true upon processing of a part,
and the name of the part is stored in |\childdocname|.
Note that |\jobname| will be set to the filename of the current part
so that each part receives an individual |.aux| file
that does not interfere with the |.aux| file(s) of the main document.
This behaviour can be altered by the alternative form
|\childdocby[*]{|\textit{main}|}| (with a non-empty optional argument)
which uses the |.aux| file of the main document
by setting |\jobname| to \textit{main}.

%%%%%%%%%%%%%%%%%%%%%%%%%%%%%%%%%%%%%%%%%%%%%%%%%%%%%%%%%%%%%%%%%%%%%%%%%%%%%%%%
\subsection{Driver Development}
\label{sec:driver}

The \textsf{childdoc} mechanism can also be use for the development
of definition files such as \LaTeX{} styles or classes.
This case differs from the above setup with multiple parts
included by |\include| in that no |\includeonly| should be invoked.
This can be achieved by starting the include file
(before |\ProvidesPackage|) with:
%
\begin{center}
\begin{tabular}{l}
|\input{childdoc.def}|\\
|\childdocforward{|\textit{main}|}|\\
\end{tabular}
\end{center}
%
or alternatively with:
%
\begin{center}
\begin{tabular}{l}
|\input{childdoc.def}|\\
|\childdocby{|\textit{main}|}|\\
\end{tabular}
\end{center}
%
Both forms have slightly different effects as described above.
The main file is prepared as usual, see \secref{sec:include}.

%%%%%%%%%%%%%%%%%%%%%%%%%%%%%%%%%%%%%%%%%%%%%%%%%%%%%%%%%%%%%%%%%%%%%%%%%%%%%%%%
\subsection{Legacy Detection}
\label{sec:detection}

The directive |\childdocmain| in the main file can detect
whether the complete document or merely a child is to be compiled
even without using the directive |\childdocof|.
This method is deprecated because it is less robust
and there is no compelling reason to use it;
it is merely provided for backward compatibility
and it may be removed in future versions.

If the detection mechanism is to be used,
it is mandatory to correctly specify
the filename of the main file as the argument of |\childdocmain|:
%
\begin{center}
\begin{tabular}{l}
|\input{childdoc.def}|\\
|\childdocmain{|\textit{main}|}|\\
\end{tabular}
\end{center}
%
If |\jobname| does not match the argument \textit{main} of |\childdocmain|,
it is assumed that |\jobname| points to the child file to be compiled.
When using |\childdocmain| with the main file specified as argument,
it suffices to start a child file
with just |\input{|\textit{main}|}|
without loading of the package and using |\childdocof|.
If instead all processing is done
with the appropriate \textsf{childdoc} directives,
the argument of \textit{main} of |\childdocmain| can be empty.

An alternative version of the command line processing described
in \secref{sec:commandline} using the detection mechanism reads:
%
\begin{center}
|... -jobname "|\textit{target}|" "|[\textit{flags}]%
[|\def\jobname{|\textit{dest}|}|]|\input{|\textit{main}|}"|
\end{center}

%%%%%%%%%%%%%%%%%%%%%%%%%%%%%%%%%%%%%%%%%%%%%%%%%%%%%%%%%%%%%%%%%%%%%%%%%%%%%%%%
\subsection{Manual Code}
\label{sec:manual}

In case one cannot be certain whether the definitions file |childdoc.def|
is installed on the target \TeX{} distribution
and one prefers not to ship it,
it is conceivable to paste a few relevant commands into the sources.

To that end, drop all statements |\input{childdoc.def}|
and perform the replacements as outlined below.
Instead of |\childdocmain{|\textit{main}|}| add the following code
to the top of the main file:
%
\begin{center}
\begin{tabular}{l}
|\||ifdefined\childdocname\endinput\||fi\newif\ifchilddoc|\\
|\edef\childdocname{\scantokens\expandafter{\jobname\noexpand}}|\\
|\def\childdocmain{|\textit{main}|}\||ifx\childdocmain\childdocname\||else|\\
|\childdoctrue\includeonly{\childdocname}\let\jobname\childdocmain\||fi|\\
\end{tabular}
\end{center}
%
Instead of |\childdocof{|\textit{main}|}| just include the main file
at the top of each child file:
%
\begin{center}
|\input{|\textit{main}|}|
\end{center}
%
A simple redirection |\childdocforward{|\textit{dest}|}| is achieved by:
%
\begin{center}
|\def\jobname{|\textit{dest}|}\input{\jobname}|
\end{center}
%
The redirection with prefix
|\childdocforwardprefix[|\textit{prefix}|]{|\textit{dest}|}|
is accomplished by:
%
\begin{center}
\begin{tabular}{l}
|{\edef\jobname{\scantokens\expandafter{\jobname\noexpand}}|\\
|\def\redirectjob |\textit{prefix}|#1~~~{\gdef\jobname{|\textit{dest}|#1}}|\\
|\expandafter\redirectjob\jobname~~~}\input{\jobname}|
\end{tabular}
\end{center}

In an alternative approach,
child documents can be compiled by a specific command line
without additional code or specific definitions:
%
\begin{center}
|... -jobname "|\textit{target}|" "|[\textit{flags}]%
|\includeonly{|\textit{dest}|}\input{|\textit{main}|}"|
\end{center}
%

%%%%%%%%%%%%%%%%%%%%%%%%%%%%%%%%%%%%%%%%%%%%%%%%%%%%%%%%%%%%%%%%%%%%%%%%%%%%%%%%
%%%%%%%%%%%%%%%%%%%%%%%%%%%%%%%%%%%%%%%%%%%%%%%%%%%%%%%%%%%%%%%%%%%%%%%%%%%%%%%%
\section{Information}

%%%%%%%%%%%%%%%%%%%%%%%%%%%%%%%%%%%%%%%%%%%%%%%%%%%%%%%%%%%%%%%%%%%%%%%%%%%%%%%%
\subsection{Copyright}

Copyright \copyright{} 2017--2018 Niklas Beisert

This work may be distributed and/or modified under the
conditions of the \LaTeX{} Project Public License, either version 1.3
of this license or (at your option) any later version.
The latest version of this license is in
  \url{http://www.latex-project.org/lppl.txt}
and version 1.3 or later is part of all distributions of \LaTeX{}
version 2005/12/01 or later.

This work has the LPPL maintenance status `maintained'.

The Current Maintainer of this work is Niklas Beisert.

This work consists of the files |README.txt|, |childdoc.ins| and |childdoc.dtx|
as well as the derived files |childdoc.def|, |cdocsamp.tex|
with |cdocsch1.tex|, |cdocsch2.tex|, |cdocspt3.tex|, |cdocspt4.tex|,
|cdocsdrf.tex|, |cdocsfn1.tex|, |cdocsfn2.tex|
as well as |childdoc.pdf|.

%%%%%%%%%%%%%%%%%%%%%%%%%%%%%%%%%%%%%%%%%%%%%%%%%%%%%%%%%%%%%%%%%%%%%%%%%%%%%%%%
\subsection{Files and Installation}

The package consists of the files:
%
\begin{center}
\begin{tabular}{ll}
    |README.txt|   & readme file \\
    |childdoc.ins| & installation file \\
    |childdoc.dtx| & source file \\
    |childdoc.def| & definition file \\
    |cdocsamp.tex| & sample main file \\
    |cdocsch1.tex| & sample include file \\
    |cdocsch2.tex| & sample include file \\
    |cdocspt3.tex| & sample part file \\
    |cdocspt4.tex| & sample part file \\
    |cdocsdrf.tex| & sample redirection file \\
    |cdocsfn1.tex| & sample redirection file \\
    |cdocsfn2.tex| & sample redirection file \\
    |childdoc.pdf| & manual
\end{tabular}
\end{center}
%
The distribution consists of the files
|README.txt|, |childdoc.ins| and |childdoc.dtx|.
%
\begin{itemize}
\item
Run (pdf)\LaTeX{} on |childdoc.dtx|
to compile the manual |childdoc.pdf| (this file).
\item
Run \LaTeX{} on |childdoc.ins| to create the definitions file |childdoc.def|
and the sample |cdocsamp.tex| with include files
|cdocsch1.tex|, |cdocsch2.tex|, |cdocspt3.tex|, |cdocspt4.tex|,
|cdocsdrf.tex|, |cdocsfn1.tex|, |cdocsfn2.tex|.
Then copy the file |childdoc.def| to an appropriate directory of your \LaTeX{}
distribution, e.g.\ \textit{texmf-root}|/tex/latex/childdoc|.
\end{itemize}

%%%%%%%%%%%%%%%%%%%%%%%%%%%%%%%%%%%%%%%%%%%%%%%%%%%%%%%%%%%%%%%%%%%%%%%%%%%%%%%%
\subsection{Related CTAN Packages}

There are several other packages which offer a similar functionality:
%
\begin{itemize}
\item
The packages
\href{http://ctan.org/pkg/docmute}{\textsf{docmute}},
\href{http://ctan.org/pkg/includex}{\textsf{includex}} and
\href{http://ctan.org/pkg/standalone}{\textsf{standalone}}
provide commands to include only the document body of
a child file thus allowing both files to be compiled individually.
\item
The packages \href{http://ctan.org/pkg/subdocs}{\textsf{subdocs}}
and \href{http://ctan.org/pkg/subfiles}{\textsf{subfiles}}
provide structures in which the main and child documents can be
encapsulated and allowing them to be compiled individually.
The inclusion mechanism is different from the conventional |\include|.
\item
The package \href{http://ctan.org/pkg/combine}{\textsf{combine}}
is an elaborate solution to combine several documents into one.
\end{itemize}
%
See also the CTAN topic \href{http://ctan.org/topic/subdocs}{\textsf{subdocs}}
for further related packages.
The present package differs from the above solutions in that
a document structure constructed with the conventional |\include| mechanism
just needs two extra commands at the top of every file
such that all constituent files can be compiled individually.

%%%%%%%%%%%%%%%%%%%%%%%%%%%%%%%%%%%%%%%%%%%%%%%%%%%%%%%%%%%%%%%%%%%%%%%%%%%%%%%%
%\subsection{Feature Suggestions}
%
%The following is a list of features which may be useful for future
%versions of this package:
%%
%\begin{itemize}
%\item
%\ldots
%\end{itemize}

%%%%%%%%%%%%%%%%%%%%%%%%%%%%%%%%%%%%%%%%%%%%%%%%%%%%%%%%%%%%%%%%%%%%%%%%%%%%%%%%
\subsection{Revision History}

%%%%%%%%%%%%%%%%%%%%%%%%%%%%%%%%%%%%%%%%
\paragraph{v2.0:} 2018/12/30

\begin{itemize}
\item
immediate forward processing
\item
added |\childdocby| mechanism
\item
manual restructured
\end{itemize}

%%%%%%%%%%%%%%%%%%%%%%%%%%%%%%%%%%%%%%%%
\paragraph{v1.6:} 2018/01/17

\begin{itemize}
\item
application for development of include files
\item
corrections to manual
\end{itemize}

%%%%%%%%%%%%%%%%%%%%%%%%%%%%%%%%%%%%%%%%
\paragraph{v1.5:} 2017/05/21

\begin{itemize}
\item
more complete structuring introduced
\item
|\childdocof| introduced
\item
|\childdoc| renamed to |\childdocmain|
\item
|\childredirect| renamed to |\childdocforward| and |\childdocforwardprefix|
and functionality expanded
\end{itemize}

%%%%%%%%%%%%%%%%%%%%%%%%%%%%%%%%%%%%%%%%
\paragraph{v1.0:} 2017/04/27

\begin{itemize}
\item
manual and install package
\item
first version published on CTAN
\end{itemize}

%%%%%%%%%%%%%%%%%%%%%%%%%%%%%%%%%%%%%%%%
\paragraph{v0.6:} 2017/04/26

\begin{itemize}
\item
redirection mechanism added
\end{itemize}

%%%%%%%%%%%%%%%%%%%%%%%%%%%%%%%%%%%%%%%%
\paragraph{v0.5:} 2017/04/26

\begin{itemize}
\item
functionality in definition file
\end{itemize}


%%%%%%%%%%%%%%%%%%%%%%%%%%%%%%%%%%%%%%%%%%%%%%%%%%%%%%%%%%%%%%%%%%%%%%%%%%%%%%%%
%%%%%%%%%%%%%%%%%%%%%%%%%%%%%%%%%%%%%%%%%%%%%%%%%%%%%%%%%%%%%%%%%%%%%%%%%%%%%%%%
%%%%%%%%%%%%%%%%%%%%%%%%%%%%%%%%%%%%%%%%%%%%%%%%%%%%%%%%%%%%%%%%%%%%%%%%%%%%%%%%
\appendix

\settowidth\MacroIndent{\rmfamily\scriptsize 000\ }

 \DocInput{childdoc.dtx}

\end{document}
%</driver>
% \fi
%
% %%%%%%%%%%%%%%%%%%%%%%%%%%%%%%%%%%%%%%%%%%%%%%%%%%%%%%%%%%%%%%%%%%%%%%%%%%%%%%
% %%%%%%%%%%%%%%%%%%%%%%%%%%%%%%%%%%%%%%%%%%%%%%%%%%%%%%%%%%%%%%%%%%%%%%%%%%%%%%
% \section{Sample}
%\iffalse
%<*samplemain>
%\fi
%
% The following presents a sample document
% with two chapters, two parts, a title page,
% a compile flag as well as three forwarding files to set the flag.
% It consists of eight |.tex| files:
% \begin{center}
% \begin{tabular}{ll}
% |cdocsamp.tex|&main file\\
% |cdocsch1.tex|&include file for chapter 1\\
% |cdocsch2.tex|&include file for chapter 2\\
% |cdocspt3.tex|&include file for part 3\\
% |cdocspt4.tex|&include file for part 4\\
% |cdocsdrf.tex|&forwarding file for main file in draft mode\\
% |cdocsfi1.tex|&forwarding file for final version of chapter 1\\
% |cdocsfi2.tex|&forwarding file for final version of chapter 2\\
% \end{tabular}
% \end{center}
% Each of the eight files can be compiled directly by the \LaTeX{} compiler.
%
% %%%%%%%%%%%%%%%%%%%%%%%%%%%%%%%%%%%%%%
% \paragraph{Main File.}
%
% The main file is called |cdocsamp.tex|.
%
% Load the \textsf{childdoc} definitions and
% declare the filename for the main document:
%    \begin{macrocode}
\input{childdoc.def}
\childdocmain{}
%    \end{macrocode}

% Optional override for |\version| flag:
%    \begin{macrocode}
%%\ifchilddoc\else\providecommand{\version}{draft}\fi
%    \end{macrocode}

% Define the default values for the |\version| flag
% (|final| for the main file and |draft| for childs):
%    \begin{macrocode}
\ifchilddoc
\providecommand{\version}{draft}
\else
\providecommand{\version}{final}
\fi
%    \end{macrocode}

% Load the standard document class:
%    \begin{macrocode}
\documentclass[12pt]{article}
%    \end{macrocode}

% Start the document body:
%    \begin{macrocode}
\begin{document}
%    \end{macrocode}

% Declare a title page.
% Print title, part of document being processed and version flag:
%    \begin{macrocode}
\addtocounter{page}{-1}
\begin{center}
{\LARGE\bfseries{}childdoc example\par}
\vspace{1cm}
\ifchilddoc
\ifchilddocmanual part\else chapter\fi:
`\childdocname' of `\childdocjob'\par
\else
main document: `\childdocjob'\par
\fi
version: \version\par
\end{center}
\newpage
%    \end{macrocode}

% Manually include selected file,
% otherwise process as usual:
%    \begin{macrocode}
\ifchilddocmanual
\section*{part `\childdocname'}
\input{\childdocname}
\else
%    \end{macrocode}

% Include the two chapters:
%    \begin{macrocode}
\include{cdocsch1}
\include{cdocsch2}
%    \end{macrocode}

% Include the two parts unless only chapters should be displayed:
%    \begin{macrocode}
\ifchilddoc\else
\section{part three}
\input{cdocspt3}
\section{part four}
\input{cdocspt4}
\fi
%    \end{macrocode}

% Process as usual until here:
%    \begin{macrocode}
\fi
%    \end{macrocode}

% End of document body:
%    \begin{macrocode}
\end{document}
%    \end{macrocode}
%\iffalse
%</samplemain>
%\fi
%
% %%%%%%%%%%%%%%%%%%%%%%%%%%%%%%%%%%%%%%
% \paragraph{Chapter Include Files.}
%
% The include files are called |cdocsch1.tex| and |cdocsch2.tex|.
%
%\iffalse
%<*samplechap1|samplechap2>
%\fi

% Optional override for |\version| flag:
%    \begin{macrocode}
%%\providecommand{\version}{final}
%    \end{macrocode}

% Include the main document:
%    \begin{macrocode}
\input{childdoc.def}
\childdocof{cdocsamp}
%    \end{macrocode}

%\iffalse
%</samplechap1|samplechap2>
%\fi
%
%\iffalse
%<*samplechap1>
%\fi
% Some text for chapter 1:
%    \begin{macrocode}
\section{one}
some text in chapter one
%    \end{macrocode}

%\iffalse
%</samplechap1>
%\fi
% Some text for chapter 2:
%\iffalse
%<*samplechap2>
%\fi
%    \begin{macrocode}
\section{two}
more text in chapter two
%    \end{macrocode}

%\iffalse
%</samplechap2>
%\fi
%
% %%%%%%%%%%%%%%%%%%%%%%%%%%%%%%%%%%%%%%
% \paragraph{Part Include Files.}
%
% The include files are called |cdocspt3.tex| and |cdocspt4.tex|.
%
%\iffalse
%<*samplepart3|samplepart4>
%\fi

% Optional override for |\version| flag:
%    \begin{macrocode}
%%\providecommand{\version}{final}
%    \end{macrocode}

% Include the main document:
%    \begin{macrocode}
\input{childdoc.def}
\childdocby{cdocsamp}
%    \end{macrocode}

%\iffalse
%</samplepart3|samplepart4>
%\fi
%
%\iffalse
%<*samplepart3>
%\fi
% Some text for part 3:
%    \begin{macrocode}
some text in part three
%    \end{macrocode}

%\iffalse
%</samplepart3>
%\fi
% Some text for part 4:
%\iffalse
%<*samplepart4>
%\fi
%    \begin{macrocode}
more text in part four
%    \end{macrocode}

%\iffalse
%</samplepart4>
%\fi
%
% %%%%%%%%%%%%%%%%%%%%%%%%%%%%%%%%%%%%%%
% \paragraph{Forwarding for a Complete Draft.}
%
% The following forwarding file |cdocsdrf.tex|
% compiles the main document in draft mode:
%\iffalse
%<*sampledraft>
%\fi
%    \begin{macrocode}
\def\version{draft}
\input{childdoc.def}
\childdocforward{cdocsamp}
%    \end{macrocode}

%\iffalse
%</sampledraft>
%\fi
%
% %%%%%%%%%%%%%%%%%%%%%%%%%%%%%%%%%%%%%%
% \paragraph{Forwarding for Final Version of the Chapters.}
%
% The following forwarding files |cdocsfn1.tex| and |cdocsfn2.tex|
% (with identical content)
% compile the final versions of the child documents
% |cdocsch1.tex| and |cdocsch2.tex|, respectively:
%\iffalse
%<*samplefinal>
%\fi
%    \begin{macrocode}
\def\version{final}
\input{childdoc.def}
\childdocforwardprefix[cdocsamp]{cdocsfn}{cdocsch}
%    \end{macrocode}

%\iffalse
%</samplefinal>
%\fi
%
% %%%%%%%%%%%%%%%%%%%%%%%%%%%%%%%%%%%%%%
% \paragraph{Command Line Processing.}
%
% The following three command lines generate the output files
% |cdocscld|, |cdocscl1| and |cdocscl2|
% which should be identical to
% |cdocsdrf|, |cdocsch1| and |cdocsfn2|, respectively:
% \begin{center}
% \begin{tabular}{l}
% |latex -jobname cdocscld \|\\
% |  "\def\version{draft}\input{childdoc.def}\childdocforward{cdocsamp}"|\\
% |latex -jobname cdocscl1 \|\\
% |  "\input{childdoc.def}\childdocforward[cdocsamp]{cdocsch1}"|\\
% |latex -jobname cdocscl2 \|\\
% |  "\def\version{final}\input{childdoc.def}\childdocforward{cdocsch2}"|
% \end{tabular}
% \end{center}
% Note that the trailing backslash on each first line
% merely continues the input to the second line
% (for convenient cut ant paste).
% Furthermore, the command |latex| can be replaced by any
% of its alternative versions such as |pdflatex|.
%
% %%%%%%%%%%%%%%%%%%%%%%%%%%%%%%%%%%%%%%%%%%%%%%%%%%%%%%%%%%%%%%%%%%%%%%%%%%%%%%
% %%%%%%%%%%%%%%%%%%%%%%%%%%%%%%%%%%%%%%%%%%%%%%%%%%%%%%%%%%%%%%%%%%%%%%%%%%%%%%
% \section{Implementation}
%\iffalse
%<*package>
%\fi
%
% This section describes the definitions file |childdoc.def|.

% The definitions cannot be loaded using |\usepackage| or |\RequirePackage|
% which has a mechanism to prevent loading a style file more than once.
% When loading the definitions by means of |\input|
% multiple instances have to be prevented manually:
%\iffalse
%This code needs to be before the `\ProvidesFile' directive
%which is defined at the beginning of this file.
%Therefore it is also placed there and commented out here.
%</package>
%<*discard>
%\fi
%    \begin{macrocode}
\ifdefined\childdocmain\endinput\fi
%    \end{macrocode}
%\iffalse
%</discard>
%<*package>
%\fi
%
% \macro{\ifchilddoc}
% \macro{\ifchilddocmanual}
% The conditional |\ifchilddoc| tells whether a
% child (true) or main (false) document is being compiled.
% The conditional |\ifchilddocmanual| tells whether
% the |\includeonly| mechanism is used (false) or
% the selection of child files must be performed manually (true).
% The definitions initialise to false:
%    \begin{macrocode}
\newif\ifchilddoc
\newif\ifchilddocmanual
%    \end{macrocode}

% \macro{\childdocname}
% \macro{\childdocjob}
% The macro |\childdocname| stores the name of the main document
% to be compiled. The macro |\childdocjob| stores the name of
% the document on which the \LaTeX{} compiler was originally invoked.
% The content of |\jobname| cannot be compared
% to filenames specified in the source due to different catcodes.
% The following code rescans |\jobname|, stores the result
% in |\childdocname| and saves a copy in |\childdocjob|:
%    \begin{macrocode}
\edef\childdocname{\scantokens\expandafter{\jobname\noexpand}}
\let\childdocjob\childdocname
%    \end{macrocode}

% \macro{\childdocdisable}
% The macro |\childdocdisable| prevents the main file
% from being processed more than once.
% At this stage, the main document command |\childdocmain|
% is assumed to be called once again where it should do nothing.
% Any subsequent call to it should prevent
% a secondary processing of the main document
% It overwrites the forwarding commands
% |\childdocof| and |\childdocforward|
% with empty macros to prevent further inclusions of the main document:
%    \begin{macrocode}
\newcommand{\childdocdisable}
{
  \renewcommand{\childdocmain}[1]{\renewcommand{\childdocmain}[1]{\endinput}}
  \renewcommand{\childdocof}[1]{}
  \renewcommand{\childdocby}[2][]{}
  \renewcommand{\childdocforward}[2][]{}
  \renewcommand{\childdocdisable}{}
}
%    \end{macrocode}

% \macro{\childdocmain}
% The macro |\childdocmain| is to be called at the top of the main file
% with nothing or the main filename (without extension) as argument.
% First, it breaks loops.
% If the argument is not empty and does not match |\childdocname|
% (which is set by the first inclusion of |childdoc.def|),
% |\ifchilddoc| is set to true, |\includeonly| is applied to the child file
% and |\jobname| is set to the main file
% (for proper handling of |.aux| files):
%    \begin{macrocode}
\newcommand{\childdocmain}[1]
{
  \childdocdisable\childdocmain{}
  \if?#1?\else
    \begingroup
      \def\childdoctmp{#1}
      \ifx\childdoctmp\childdocname
        \def\childdoctmp{}
      \else
        \def\childdoctmp
        {
          \childdoctrue
          \includeonly{\childdocname}
          \def\childdocjob{#1}
          \def\jobname{#1}
        }
      \fi
      \expandafter
    \endgroup
    \childdoctmp
  \fi
}
%    \end{macrocode}

% \macro{\childdocof}
% The command |\childdocof| redirects
% compilation to the main file |#1|.
%    \begin{macrocode}
\newcommand{\childdocof}[1]
{
  \childdocdisable
  \childdoctrue
  \includeonly{\childdocname}
  \def\jobname{#1}
  \def\childdocjob{#1}
  \input{#1}
}
%    \end{macrocode}

% \macro{\childdocby}
% The command |\childdocby| ....
%    \begin{macrocode}
\newcommand{\childdocby}[2][]
{
  \childdocdisable
  \childdoctrue
  \childdocmanualtrue
  \if?#1?\else
    \def\jobname{#2}
  \fi
  \def\childdocjob{#2}
  \input{#2}
  \endinput
}
%    \end{macrocode}

% \macro{\childdocforward}
% The command |\childdocforward| redirects
% compilation to the main file or
% (if the optional argument is given) a child file.
% Parameters are set as if the main file
% or a child file starting with |\childdocof| was compiled.
% Then compilation is handed over to the main file:
%    \begin{macrocode}
\newcommand{\childdocforward}[2][]
{
  \begingroup
    \if?#1?
      \def\childdoctmp
      {
        \def\childdocname{#2}
        \def\childdocjob{#2}
        \def\jobname{#2}
        \input{#2}
        \endinput
      }
    \else
      \def\childdoctmp
      {
        \childdocdisable
        \def\childdocname{#2}
        \childdoctrue
        \includeonly{#2}
        \def\childdocjob{#1}
        \def\jobname{#1}
        \input{#1}
        \endinput
      }
    \fi
    \expandafter
  \endgroup
  \childdoctmp
}
%    \end{macrocode}

% \macro{\childdocforwardprefix}
% The command |\childdocforwardprefix| redirects
% compilation to the main or a child file by means of a pattern.
% The prefix |#1| in the current filename is replaced by |#2|
% and the suffix of the current filename is kept
% (it is assumed that the filename does not contain the substring `|~~~|'
% which is used as a delimiter).
% Compilation is handed over to the new file by |\childdocforward|:
%    \begin{macrocode}
\newcommand{\childdocforwardprefix}[3][]
{
  \begingroup
    \def\childdocextract #2##1~~~{\def\childdoctmp{\childdocforward[#1]{#3##1}}}
    \expandafter\childdocextract\childdocname~~~
    \expandafter
  \endgroup
  \childdoctmp
}
%    \end{macrocode}

% \macro{\childdoc}
% The deprecated macro |\childdoc| is a legacy version of |\childdocmain|:
%    \begin{macrocode}
\newcommand{\childdoc}{\childdocmain}
%    \end{macrocode}

% \macro{\childdocredirect}
% The deprecated macro |\childdocredirect| is a legacy version
% of |\childdocforward| and |\childdocforwardprefix|:
%    \begin{macrocode}
\newcommand{\childdocredirect}[2][]
{
  \begingroup
    \if?#1?
      \def\childdoctmp{\childdocforward{#2}}
    \else
      \def\childdoctmp{\childdocforwardprefix{#1}{#2}}
    \fi
    \expandafter
  \endgroup
  \childdoctmp
}
%    \end{macrocode}

%\iffalse
%</package>
%\fi
%
\endinput
|\\
|\childdocof{|\textit{main}|}|\\
\end{tabular}
\end{center}
at the top of every child file \textit{child}
which is included by |\include{|\textit{child}|}|
from within the main file
(or at least for those files to be compiled individually).
The argument \textit{main} must be the filename of the main file.

There are a couple of
considerations in setting up the main and child documents:

%%%%%%%%%%%%%%%%%%%%%%%%%%%%%%%%%%%%%%%%
\paragraph{Restrictions.}

Please note the following restrictions:
\begin{itemize}
\item
|\childdocmain| must be called with one argument \textit{main}
to ensure compatibility with earlier version of the package.
It must either be empty (|\childdocmain{}|)
or precisely match the filename of the main file in which it is specified.
See \secref{sec:detection} for further information.
\item
The filename \textit{main} must be specified without the |.tex| extension.
\item
The filename \textit{main} is case sensitive
(even in case-insensitive file systems)
due to internal string comparison.
\item
The argument \textit{main} should be fully expanded, it cannot be a macro.
\item
Subdirectories and special characters should be avoided in filenames.
\item
The command |\childdocmain{|\textit{main}|}| must be followed by a whitespace.
It should not be followed immediately by another command
or by a comment mark `|%|'.
This is because the \TeX{} parser reads the token immediately following
the argument of |\childdocmain| and puts it
at the beginning of every child section;
however, a white\-space is ignored.
\end{itemize}

%%%%%%%%%%%%%%%%%%%%%%%%%%%%%%%%%%%%%%%%
\paragraph{Content of Main File.}

It is advisable to place all content in the child files included by |\include|.
Any output contained in the main file will appear in all child documents
unless suppressed manually;
it cannot be suppressed automatically by the |\includeonly| directive
and thus should normally be avoided.
A method to include some content in the main file
by means of conditional processing is described in \secref{sec:conditional}.

%%%%%%%%%%%%%%%%%%%%%%%%%%%%%%%%%%%%%%%%
\paragraph{Page Numbering.}

When only a part of the document is compiled,
the appropriate numbering of pages
(as well as other status parameters)
is determined from the |.aux| files.
The latter contain information from previous passes.
However this information needs to propagate through
all intermediate child documents.
Therefore the page numbering in child documents may well
be inconsistent until the complete document is compiled at least once.

A useful (if unconventional) way to always ensure a consistent
page numbering is to restart the numbering in each child document
and denote the pages by `\textit{child}|.|\textit{page}'
where \textit{child} represents the chapter/section number of the child file.
This can be achieved by the command
|\numberwithin{page}{|\textit{child}|}|
of the \textsf{amsmath} package
where \textit{child} can be |chapter| or |section|
depending on the chosen structuring.
Alternatively, one can modify the macro |\thepage| appropriately
and reset the counter |page| at the start of each child file.

%%%%%%%%%%%%%%%%%%%%%%%%%%%%%%%%%%%%%%%%%%%%%%%%%%%%%%%%%%%%%%%%%%%%%%%%%%%%%%%%
\subsection{Conditional Processing}
\label{sec:conditional}

The package provides a mechanism to compile different versions
of a document. To customise the versions further some conditional processing
can come in handy to distinguish which version is being compiled.
The package provides two macros to describe the compilation context:

%%%%%%%%%%%%%%%%%%%%%%%%%%%%%%%%%%%%%%%%
\DescribeMacro{\ifchilddoc}
The conditional |\ifchilddoc| distinguishes between the compilation of
child documents and the main document:
%
\begin{center}
|\ifchilddoc |\textit{child-code}| |[|\||else |\textit{main-code}]| \||fi|
\end{center}

%%%%%%%%%%%%%%%%%%%%%%%%%%%%%%%%%%%%%%%%
\DescribeMacro{\childdocname}
\DescribeMacro{\childdocjob}
The macro |\childdocname| contains the filename (without extension)
of the main or child file being processed.
Note that |\childdocjob| will always contain the name of the main file.

%%%%%%%%%%%%%%%%%%%%%%%%%%%%%%%%%%%%%%%%
\paragraph{Title Page.}

Conditional processing can be used to include a title or banner page
in the main document when proper precautions are taken.
Importantly, the code in the main file should ensure that the page counter
(as well as other status parameters which are stored in the |.aux| files)
takes the same value after the conditional processing.
Otherwise the page numbers may take divergent values
depending on which part is compiled.

For example, a title page could be declared by:
%
\begin{center}
\begin{tabular}{l}
|\ifchilddoc\||else|\\
|\addtocounter{page}{-1}|\\
\textit{code for title page}\\
|\newpage|\\
|\||fi|
\end{tabular}
\end{center}
%
A banner page for the child documents can be generated by:
%
\begin{center}
\begin{tabular}{l}
|\ifchilddoc|\\
|\addtocounter{page}{-1}|\\
\textit{code for banner page}\\
|\newpage|\\
|\||fi|
\end{tabular}
\end{center}
%
Here one could write a message such as:
\begin{center}
|This is the part \childdocname{} of \childdocjob{}.|
\end{center}

%%%%%%%%%%%%%%%%%%%%%%%%%%%%%%%%%%%%%%%%%%%%%%%%%%%%%%%%%%%%%%%%%%%%%%%%%%%%%%%%
\subsection{Flags}
\label{sec:flags}

The package makes it easy to generate different versions
of the main or child documents.
To this end compilation flags can be defined
and assigned different default values.
They will be particularly useful in conjunction
with the forwarding mechanism described in \secref{sec:forward}.

For example, it may be useful to have a flag |\version|
which can be set to |draft| or |final|.
The document source will contain some conditional code
depending on the value of |\version|.
Suppose further, the flag should default to |final| for the main file
and to |draft| for child files
which is a natural assignment for editing the document.
This is achieved by placing the following code
in the preamble of the main document
(below the |\childdocmain| directive):
%
\begin{center}
\begin{tabular}{l}
|\ifchilddoc|\\
|\providecommand{\version}{draft}|\\
|\||else|\\
|\providecommand{\version}{final}|\\
|\||fi|
\end{tabular}
\end{center}
%
The definition by |\providecommand| makes sure
that previous definitions are not overwritten.
Further statements |\providecommand{\version}{...}|
can thus be added before the above code to override it.

For the main file, one might add a line
(between |\childdocmain| and the above block)
%
\begin{center}
|%\ifchilddoc\||else\providecommand{\version}{draft}\||fi|
\end{center}
%
which can be uncommented to produce a draft version.
Likewise one can add a line to the very top of a child file
(above the |\childdocof{|\textit{main}|}| directive)
%
\begin{center}
|%\providecommand{\version}{final}|
\end{center}
%
which can be uncommented to produce the final version of this child document.

%%%%%%%%%%%%%%%%%%%%%%%%%%%%%%%%%%%%%%%%%%%%%%%%%%%%%%%%%%%%%%%%%%%%%%%%%%%%%%%%
\subsection{Forwarding}
\label{sec:forward}

Different versions of the main or child documents
using compilation flags as described in \secref{sec:flags}
can be (permanently) stored in different files
for convenient compilation, viewing and distribution.
To this end, the package defines a command
to pass on compilation to a different file:

%%%%%%%%%%%%%%%%%%%%%%%%%%%%%%%%%%%%%%%%
\DescribeMacro{\childdocforward}
The command |\childdocforward| redirects processing to
another source file:
%
\begin{center}
\begin{tabular}{l}
|% \iffalse
%
% childdoc.dtx Copyright (C) 2017-2018 Niklas Beisert
%
% This work may be distributed and/or modified under the
% conditions of the LaTeX Project Public License, either version 1.3
% of this license or (at your option) any later version.
% The latest version of this license is in
%   http://www.latex-project.org/lppl.txt
% and version 1.3 or later is part of all distributions of LaTeX
% version 2005/12/01 or later.
%
% This work has the LPPL maintenance status `maintained'.
%
% The Current Maintainer of this work is Niklas Beisert.
%
% This work consists of the files childdoc.dtx and childdoc.ins
% and the derived files childdoc.def and cdocsamp.tex with
% cdocsch1.tex, cdocsch2.tex, cdocsdrf.tex, cdocsfn1.tex, cdocsfn2.tex.
%
%<package>\ifdefined\childdocmain\endinput\fi
%<package>\ProvidesFile{childdoc.def}[2018/12/30 v2.0 child document driver]
%<samplemain>\ProvidesFile{cdocsamp.tex}[2018/12/30 v2.0 sample for childdoc]
%<*driver>
%\ProvidesFile{childdoc.drv}[2018/12/30 v2.0 childdoc reference manual file]
\PassOptionsToClass{10pt,a4paper}{article}
\documentclass{ltxdoc}

\usepackage[margin=35mm]{geometry}
\usepackage{hyperref}
\usepackage{hyperxmp}
\usepackage[usenames]{color}

\hypersetup{colorlinks=true}
\hypersetup{pdfstartview=FitH}
\hypersetup{pdfpagemode=UseNone}
\hypersetup{pdfsource={}}
\hypersetup{pdflang={en-UK}}
\hypersetup{pdfcopyright={Copyright 2017-2018 Niklas Beisert.
  This work may be distributed and/or modified under the
  conditions of the LaTeX Project Public License, either version 1.3
  of this license or (at your option) any later version.}}
\hypersetup{pdflicenseurl={http://www.latex-project.org/lppl.txt}}
\hypersetup{pdfcontactaddress={ETH Zurich, ITP, HIT K,
  Wolfgang-Pauli-Strasse 27}}
\hypersetup{pdfcontactpostcode={8093}}
\hypersetup{pdfcontactcity={Zurich}}
\hypersetup{pdfcontactcountry={Switzerland}}
\hypersetup{pdfcontactemail={nbeisert@itp.phys.ethz.ch}}
\hypersetup{pdfcontacturl={http://people.phys.ethz.ch/\xmptilde nbeisert/}}

\newcommand{\secref}[1]{\hyperref[#1]{section \ref*{#1}}}

\parskip1ex
\parindent0pt
\let\olditemize\itemize
\def\itemize{\olditemize\parskip0pt}

\begin{document}

\title{The \textsf{childdoc} Package}
\hypersetup{pdftitle={The childdoc Package}}
\author{Niklas Beisert\\[2ex]
  Institut f\"ur Theoretische Physik\\
  Eidgen\"ossische Technische Hochschule Z\"urich\\
  Wolfgang-Pauli-Strasse 27, 8093 Z\"urich, Switzerland\\[1ex]
  \href{mailto:nbeisert@itp.phys.ethz.ch}
  {\texttt{nbeisert@itp.phys.ethz.ch}}}
\hypersetup{pdfauthor={Niklas Beisert}}
\hypersetup{pdfsubject={Manual for the LaTeX2e Package childdoc}}
\date{30 December 2018, \textsf{v2.0}}
\maketitle

\begin{abstract}\noindent
\textsf{childdoc} is a \LaTeXe{} package
that enables the direct compilation
of document sections included by |\include|
to individual files.
\end{abstract}

\begingroup
\parskip0ex
\tableofcontents
\endgroup

%%%%%%%%%%%%%%%%%%%%%%%%%%%%%%%%%%%%%%%%%%%%%%%%%%%%%%%%%%%%%%%%%%%%%%%%%%%%%%%%
%%%%%%%%%%%%%%%%%%%%%%%%%%%%%%%%%%%%%%%%%%%%%%%%%%%%%%%%%%%%%%%%%%%%%%%%%%%%%%%%
\section{Introduction}

\LaTeX{} provides a mechanism to structure a large document (such as a book)
into a main file and several child files (containing the chapters)
using the |\include| command.
This mechanism is beneficial for documents
which span hundreds of pages in order to
make the source file(s) more manageable.
Moreover, compilation can be restricted to
selected child files by means of the |\includeonly| command.
The latter feature can be used to reduce the compilation time while editing
(this was significantly more useful in the earlier days of \LaTeX{})
or to generate a smaller document which is easier to navigate.
Another application of |\includeonly| is to generate
documents consisting of selected parts of the complete document.

However, there are a few drawbacks of the plain |\include| mechanism:
\begin{itemize}
\item
The child files cannot be compiled on their own,
they can only be compiled via the main file.
A naive editing environment
(such as a text editor with an option
to have the current file processed by \LaTeX)
may require one to switch to the main file before compiling;
attempting to compile the child file produces errors.
\item
The main file must be modified (each time)
to adjust the |\includeonly| command
to the present needs. This easily leaves the main file in a messy state.
\item
The generated document will always carry the filename
of the main document. This is inconvenient if
several child files are to be compiled and
to be kept for distribution.
\end{itemize}

The present package provides a simple interface
to make child files individually compilable by \LaTeX{}.
Compiling a child file then has the same effect as compiling
the main file with an |\includeonly| command
to select the appropriate child.
Moreover the generated document will carry the name of the child
rather than the main file.
This resolves all three above issues.

This feature is meant to make the editing of books,
thesis documents and lecture notes somewhat more convenient.
However, the package can also be used efficiently for
composing a series of documents (such as exercise sheets)
which are typically distributed individually.
It then assists the author in generating the individual documents
(potentially in different versions)
as well as a document containing the collected series.
Another application is in developing style files
or other kinds of included material
where compilation of the style file could redirect
to a sample or test file.

%%%%%%%%%%%%%%%%%%%%%%%%%%%%%%%%%%%%%%%%%%%%%%%%%%%%%%%%%%%%%%%%%%%%%%%%%%%%%%%%
%%%%%%%%%%%%%%%%%%%%%%%%%%%%%%%%%%%%%%%%%%%%%%%%%%%%%%%%%%%%%%%%%%%%%%%%%%%%%%%%
\section{Usage}

First of all, the package \textsf{childdoc} is \emph{not} a standard
\LaTeXe{} |.sty| style file! Therefore it needs to be invoked in
a non-standard way.

%%%%%%%%%%%%%%%%%%%%%%%%%%%%%%%%%%%%%%%%%%%%%%%%%%%%%%%%%%%%%%%%%%%%%%%%%%%%%%%%
\subsection{Included Files}
\label{sec:include}

%%%%%%%%%%%%%%%%%%%%%%%%%%%%%%%%%%%%%%%%
\DescribeMacro{\childdocmain}
To use the package, add the commands
\begin{center}
\begin{tabular}{l}
|\input{childdoc.def}|\\
|\childdocmain{}|\\
\end{tabular}
\end{center}
at the very top of the main \LaTeX{} file,
in particular \emph{before} the |\documentclass| statement!
The argument of |\childdocmain| should be left empty
(but it must be present).

%%%%%%%%%%%%%%%%%%%%%%%%%%%%%%%%%%%%%%%%
\DescribeMacro{\childdocof}
Furthermore, add the commands
\begin{center}
\begin{tabular}{l}
|\input{childdoc.def}|\\
|\childdocof{|\textit{main}|}|\\
\end{tabular}
\end{center}
at the top of every child file \textit{child}
which is included by |\include{|\textit{child}|}|
from within the main file
(or at least for those files to be compiled individually).
The argument \textit{main} must be the filename of the main file.

There are a couple of
considerations in setting up the main and child documents:

%%%%%%%%%%%%%%%%%%%%%%%%%%%%%%%%%%%%%%%%
\paragraph{Restrictions.}

Please note the following restrictions:
\begin{itemize}
\item
|\childdocmain| must be called with one argument \textit{main}
to ensure compatibility with earlier version of the package.
It must either be empty (|\childdocmain{}|)
or precisely match the filename of the main file in which it is specified.
See \secref{sec:detection} for further information.
\item
The filename \textit{main} must be specified without the |.tex| extension.
\item
The filename \textit{main} is case sensitive
(even in case-insensitive file systems)
due to internal string comparison.
\item
The argument \textit{main} should be fully expanded, it cannot be a macro.
\item
Subdirectories and special characters should be avoided in filenames.
\item
The command |\childdocmain{|\textit{main}|}| must be followed by a whitespace.
It should not be followed immediately by another command
or by a comment mark `|%|'.
This is because the \TeX{} parser reads the token immediately following
the argument of |\childdocmain| and puts it
at the beginning of every child section;
however, a white\-space is ignored.
\end{itemize}

%%%%%%%%%%%%%%%%%%%%%%%%%%%%%%%%%%%%%%%%
\paragraph{Content of Main File.}

It is advisable to place all content in the child files included by |\include|.
Any output contained in the main file will appear in all child documents
unless suppressed manually;
it cannot be suppressed automatically by the |\includeonly| directive
and thus should normally be avoided.
A method to include some content in the main file
by means of conditional processing is described in \secref{sec:conditional}.

%%%%%%%%%%%%%%%%%%%%%%%%%%%%%%%%%%%%%%%%
\paragraph{Page Numbering.}

When only a part of the document is compiled,
the appropriate numbering of pages
(as well as other status parameters)
is determined from the |.aux| files.
The latter contain information from previous passes.
However this information needs to propagate through
all intermediate child documents.
Therefore the page numbering in child documents may well
be inconsistent until the complete document is compiled at least once.

A useful (if unconventional) way to always ensure a consistent
page numbering is to restart the numbering in each child document
and denote the pages by `\textit{child}|.|\textit{page}'
where \textit{child} represents the chapter/section number of the child file.
This can be achieved by the command
|\numberwithin{page}{|\textit{child}|}|
of the \textsf{amsmath} package
where \textit{child} can be |chapter| or |section|
depending on the chosen structuring.
Alternatively, one can modify the macro |\thepage| appropriately
and reset the counter |page| at the start of each child file.

%%%%%%%%%%%%%%%%%%%%%%%%%%%%%%%%%%%%%%%%%%%%%%%%%%%%%%%%%%%%%%%%%%%%%%%%%%%%%%%%
\subsection{Conditional Processing}
\label{sec:conditional}

The package provides a mechanism to compile different versions
of a document. To customise the versions further some conditional processing
can come in handy to distinguish which version is being compiled.
The package provides two macros to describe the compilation context:

%%%%%%%%%%%%%%%%%%%%%%%%%%%%%%%%%%%%%%%%
\DescribeMacro{\ifchilddoc}
The conditional |\ifchilddoc| distinguishes between the compilation of
child documents and the main document:
%
\begin{center}
|\ifchilddoc |\textit{child-code}| |[|\||else |\textit{main-code}]| \||fi|
\end{center}

%%%%%%%%%%%%%%%%%%%%%%%%%%%%%%%%%%%%%%%%
\DescribeMacro{\childdocname}
\DescribeMacro{\childdocjob}
The macro |\childdocname| contains the filename (without extension)
of the main or child file being processed.
Note that |\childdocjob| will always contain the name of the main file.

%%%%%%%%%%%%%%%%%%%%%%%%%%%%%%%%%%%%%%%%
\paragraph{Title Page.}

Conditional processing can be used to include a title or banner page
in the main document when proper precautions are taken.
Importantly, the code in the main file should ensure that the page counter
(as well as other status parameters which are stored in the |.aux| files)
takes the same value after the conditional processing.
Otherwise the page numbers may take divergent values
depending on which part is compiled.

For example, a title page could be declared by:
%
\begin{center}
\begin{tabular}{l}
|\ifchilddoc\||else|\\
|\addtocounter{page}{-1}|\\
\textit{code for title page}\\
|\newpage|\\
|\||fi|
\end{tabular}
\end{center}
%
A banner page for the child documents can be generated by:
%
\begin{center}
\begin{tabular}{l}
|\ifchilddoc|\\
|\addtocounter{page}{-1}|\\
\textit{code for banner page}\\
|\newpage|\\
|\||fi|
\end{tabular}
\end{center}
%
Here one could write a message such as:
\begin{center}
|This is the part \childdocname{} of \childdocjob{}.|
\end{center}

%%%%%%%%%%%%%%%%%%%%%%%%%%%%%%%%%%%%%%%%%%%%%%%%%%%%%%%%%%%%%%%%%%%%%%%%%%%%%%%%
\subsection{Flags}
\label{sec:flags}

The package makes it easy to generate different versions
of the main or child documents.
To this end compilation flags can be defined
and assigned different default values.
They will be particularly useful in conjunction
with the forwarding mechanism described in \secref{sec:forward}.

For example, it may be useful to have a flag |\version|
which can be set to |draft| or |final|.
The document source will contain some conditional code
depending on the value of |\version|.
Suppose further, the flag should default to |final| for the main file
and to |draft| for child files
which is a natural assignment for editing the document.
This is achieved by placing the following code
in the preamble of the main document
(below the |\childdocmain| directive):
%
\begin{center}
\begin{tabular}{l}
|\ifchilddoc|\\
|\providecommand{\version}{draft}|\\
|\||else|\\
|\providecommand{\version}{final}|\\
|\||fi|
\end{tabular}
\end{center}
%
The definition by |\providecommand| makes sure
that previous definitions are not overwritten.
Further statements |\providecommand{\version}{...}|
can thus be added before the above code to override it.

For the main file, one might add a line
(between |\childdocmain| and the above block)
%
\begin{center}
|%\ifchilddoc\||else\providecommand{\version}{draft}\||fi|
\end{center}
%
which can be uncommented to produce a draft version.
Likewise one can add a line to the very top of a child file
(above the |\childdocof{|\textit{main}|}| directive)
%
\begin{center}
|%\providecommand{\version}{final}|
\end{center}
%
which can be uncommented to produce the final version of this child document.

%%%%%%%%%%%%%%%%%%%%%%%%%%%%%%%%%%%%%%%%%%%%%%%%%%%%%%%%%%%%%%%%%%%%%%%%%%%%%%%%
\subsection{Forwarding}
\label{sec:forward}

Different versions of the main or child documents
using compilation flags as described in \secref{sec:flags}
can be (permanently) stored in different files
for convenient compilation, viewing and distribution.
To this end, the package defines a command
to pass on compilation to a different file:

%%%%%%%%%%%%%%%%%%%%%%%%%%%%%%%%%%%%%%%%
\DescribeMacro{\childdocforward}
The command |\childdocforward| redirects processing to
another source file:
%
\begin{center}
\begin{tabular}{l}
|\input{childdoc.def}|\\
|\childdocforward[|\textit{main}|]{|\textit{dest}|}|\\
\end{tabular}
\end{center}
%
The argument \textit{dest} is the destination file
(without extension).
It should be the main file or one of the child files.
Note that further \textsf{childdoc} directives
such as |\childdocof| and |\childdocforward|
in the indicated file will be processed in this form.
The optional argument \textit{main}
passes on directly to the main file \textit{main}
while pretending to compile the child \textit{dest}.
This form behaves as if \textit{dest}
issues |\childdocof{|\textit{main}|}| right away,
and no further \textsf{childdoc} directives will be processed.

%%%%%%%%%%%%%%%%%%%%%%%%%%%%%%%%%%%%%%%%
\DescribeMacro{\...prefix}
In the alternative form |\childdocforwardprefix|,
%
\begin{center}
\begin{tabular}{l}
|\input{childdoc.def}|\\
|\childdocforwardprefix[|\textit{main}|]{|\textit{prefix}|}{|\textit{dest}|}|
\end{tabular}
\end{center}
%
the destination file is determined by a pattern
depending on the current file:
To make this work, the current file must be called
`{\textit{prefix}\hspace{0.2em}\textit{suffix}}'
with \textit{prefix} matching precisely the argument.
Processing is then passed on to the file
`{\textit{dest}\hspace{0.2em}\textit{suffix}}'.
Surely, the same effect is achieved by
directly specifying the
argument `{\textit{dest}\hspace{0.2em}\textit{suffix}}'
in the first form.
However, that requires to set up a different file
for each child. With the alternative form of the command
all these files can have exactly the same content
which simplifies setting them up and maintaining them.

For example, the following file |draft.tex|
with a compilation flag |\version| as described in \secref{sec:flags}
compiles the main document as a draft:
%
\begin{center}
\begin{tabular}{l}
|\def\version{draft}|\\
|\input{childdoc.def}|\\
|\childdocforward{|\textit{main}|}|
\end{tabular}
\end{center}
%
Likewise, the following files |final|\textit{nn}|.tex|
compile the final version of the child document
|child|\textit{nn}|.tex|:
%
\begin{center}
\begin{tabular}{l}
|\def\version{final}|\\
|\input{childdoc.def}|\\
|\childdocforwardprefix{final}{child}|
\end{tabular}
\end{center}
%

Note that when several versions of a main file and/or of each child file
are to be generated, it may be convenient to set up a |Makefile| or
shell script to automatise the process.

%%%%%%%%%%%%%%%%%%%%%%%%%%%%%%%%%%%%%%%%%%%%%%%%%%%%%%%%%%%%%%%%%%%%%%%%%%%%%%%%
\subsection{Command Line Processing}
\label{sec:commandline}

The effect of redirection files can also be achieved by invoking
the \LaTeX{} compiler with a more elaborate command line.
Most conveniently this should be done as part
of a shell script or a |Makefile|.

When using \textsf{childdoc} in the main file, the following
command lines effectively perform a redirection
(note that depending on the shell being used,
backslashes may have to be doubled: `|\|' $\to$ `|\\|'):
%
\begin{center}
|... -jobname "|\textit{target}|" |\\|"|[\textit{flags}]%
|\input{childdoc.def}\childdocforward[|\textit{main}|]{|\textit{dest}|}"|
\end{center}
%
Here \textit{target} is the name of the output file,
\textit{main} is the name of the main file
and \textit{dest} is the name of the main or child file to be processed
(all filenames without extensions).
The optional argument \textit{main} can be omitted
if \textit{main} matches \textit{dest}.
Optionally, compilation \textit{flags} can be defined via |\def| commands.
This command line makes the \TeX{} engine believe
it is compiling the file \textit{target}
whose content is specified as the latter parameter.
The provided code then forwards the processing to
\textit{main} or \textit{dest} as described in \secref{sec:forward}.

%%%%%%%%%%%%%%%%%%%%%%%%%%%%%%%%%%%%%%%%%%%%%%%%%%%%%%%%%%%%%%%%%%%%%%%%%%%%%%%%
\subsection{Include by Input}
\label{sec:input}

Including child documents by |\include| has some restrictions by design.
Most notably, the content of a child document always occupies
its own set of pages; pages cannot be shared between child documents.
Usually, this behaviour makes perfect sense
because each child document contain an essential part of the document.
However, in some situations it may be desirable to compose
a document from a collection of parts
without having mandatory page breaks between then.
For this case, the package
provides a mechanism to include parts
by |\input| which can also be processed individually.
However, by construction this mechanism
requires manual handling of the content to be output.

%%%%%%%%%%%%%%%%%%%%%%%%%%%%%%%%%%%%%%%%
\DescribeMacro{\ifchilddocmanual}
The main file should be prepared as usual, see \secref{sec:include}.
However, the document body must make a distinction
between processing of an individual part and of the main document, e.g.:
%
\begin{center}
\begin{tabular}{l}
|\ifchilddocmanual|\\
|\input{\childdocname}|\\
|\||else|\\
\textit{document body with }|\input{|\textit{part}|}|\\
|\||fi|
\end{tabular}
\end{center}
%
The conditional |\ifchilddocmanual| is true whenever
a part to be included by |\input| is being compiled,
and the name of the part is stored in |\childdocname|.

%%%%%%%%%%%%%%%%%%%%%%%%%%%%%%%%%%%%%%%%
\DescribeMacro{\childdocby}
Each part to be included by |\input| should start with:
%
\begin{center}
\begin{tabular}{l}
|\input{childdoc.def}|\\
|\childdocby{|\textit{main}|}|\\
\end{tabular}
\end{center}
%
The directive |\childdocby| is similar to |\childdocof|
described in \secref{sec:include},
but the subsequent selection of content must be done manually.
To that end, both |\ifchilddoc| and |\ifchilddocmanual|
will be true upon processing of a part,
and the name of the part is stored in |\childdocname|.
Note that |\jobname| will be set to the filename of the current part
so that each part receives an individual |.aux| file
that does not interfere with the |.aux| file(s) of the main document.
This behaviour can be altered by the alternative form
|\childdocby[*]{|\textit{main}|}| (with a non-empty optional argument)
which uses the |.aux| file of the main document
by setting |\jobname| to \textit{main}.

%%%%%%%%%%%%%%%%%%%%%%%%%%%%%%%%%%%%%%%%%%%%%%%%%%%%%%%%%%%%%%%%%%%%%%%%%%%%%%%%
\subsection{Driver Development}
\label{sec:driver}

The \textsf{childdoc} mechanism can also be use for the development
of definition files such as \LaTeX{} styles or classes.
This case differs from the above setup with multiple parts
included by |\include| in that no |\includeonly| should be invoked.
This can be achieved by starting the include file
(before |\ProvidesPackage|) with:
%
\begin{center}
\begin{tabular}{l}
|\input{childdoc.def}|\\
|\childdocforward{|\textit{main}|}|\\
\end{tabular}
\end{center}
%
or alternatively with:
%
\begin{center}
\begin{tabular}{l}
|\input{childdoc.def}|\\
|\childdocby{|\textit{main}|}|\\
\end{tabular}
\end{center}
%
Both forms have slightly different effects as described above.
The main file is prepared as usual, see \secref{sec:include}.

%%%%%%%%%%%%%%%%%%%%%%%%%%%%%%%%%%%%%%%%%%%%%%%%%%%%%%%%%%%%%%%%%%%%%%%%%%%%%%%%
\subsection{Legacy Detection}
\label{sec:detection}

The directive |\childdocmain| in the main file can detect
whether the complete document or merely a child is to be compiled
even without using the directive |\childdocof|.
This method is deprecated because it is less robust
and there is no compelling reason to use it;
it is merely provided for backward compatibility
and it may be removed in future versions.

If the detection mechanism is to be used,
it is mandatory to correctly specify
the filename of the main file as the argument of |\childdocmain|:
%
\begin{center}
\begin{tabular}{l}
|\input{childdoc.def}|\\
|\childdocmain{|\textit{main}|}|\\
\end{tabular}
\end{center}
%
If |\jobname| does not match the argument \textit{main} of |\childdocmain|,
it is assumed that |\jobname| points to the child file to be compiled.
When using |\childdocmain| with the main file specified as argument,
it suffices to start a child file
with just |\input{|\textit{main}|}|
without loading of the package and using |\childdocof|.
If instead all processing is done
with the appropriate \textsf{childdoc} directives,
the argument of \textit{main} of |\childdocmain| can be empty.

An alternative version of the command line processing described
in \secref{sec:commandline} using the detection mechanism reads:
%
\begin{center}
|... -jobname "|\textit{target}|" "|[\textit{flags}]%
[|\def\jobname{|\textit{dest}|}|]|\input{|\textit{main}|}"|
\end{center}

%%%%%%%%%%%%%%%%%%%%%%%%%%%%%%%%%%%%%%%%%%%%%%%%%%%%%%%%%%%%%%%%%%%%%%%%%%%%%%%%
\subsection{Manual Code}
\label{sec:manual}

In case one cannot be certain whether the definitions file |childdoc.def|
is installed on the target \TeX{} distribution
and one prefers not to ship it,
it is conceivable to paste a few relevant commands into the sources.

To that end, drop all statements |\input{childdoc.def}|
and perform the replacements as outlined below.
Instead of |\childdocmain{|\textit{main}|}| add the following code
to the top of the main file:
%
\begin{center}
\begin{tabular}{l}
|\||ifdefined\childdocname\endinput\||fi\newif\ifchilddoc|\\
|\edef\childdocname{\scantokens\expandafter{\jobname\noexpand}}|\\
|\def\childdocmain{|\textit{main}|}\||ifx\childdocmain\childdocname\||else|\\
|\childdoctrue\includeonly{\childdocname}\let\jobname\childdocmain\||fi|\\
\end{tabular}
\end{center}
%
Instead of |\childdocof{|\textit{main}|}| just include the main file
at the top of each child file:
%
\begin{center}
|\input{|\textit{main}|}|
\end{center}
%
A simple redirection |\childdocforward{|\textit{dest}|}| is achieved by:
%
\begin{center}
|\def\jobname{|\textit{dest}|}\input{\jobname}|
\end{center}
%
The redirection with prefix
|\childdocforwardprefix[|\textit{prefix}|]{|\textit{dest}|}|
is accomplished by:
%
\begin{center}
\begin{tabular}{l}
|{\edef\jobname{\scantokens\expandafter{\jobname\noexpand}}|\\
|\def\redirectjob |\textit{prefix}|#1~~~{\gdef\jobname{|\textit{dest}|#1}}|\\
|\expandafter\redirectjob\jobname~~~}\input{\jobname}|
\end{tabular}
\end{center}

In an alternative approach,
child documents can be compiled by a specific command line
without additional code or specific definitions:
%
\begin{center}
|... -jobname "|\textit{target}|" "|[\textit{flags}]%
|\includeonly{|\textit{dest}|}\input{|\textit{main}|}"|
\end{center}
%

%%%%%%%%%%%%%%%%%%%%%%%%%%%%%%%%%%%%%%%%%%%%%%%%%%%%%%%%%%%%%%%%%%%%%%%%%%%%%%%%
%%%%%%%%%%%%%%%%%%%%%%%%%%%%%%%%%%%%%%%%%%%%%%%%%%%%%%%%%%%%%%%%%%%%%%%%%%%%%%%%
\section{Information}

%%%%%%%%%%%%%%%%%%%%%%%%%%%%%%%%%%%%%%%%%%%%%%%%%%%%%%%%%%%%%%%%%%%%%%%%%%%%%%%%
\subsection{Copyright}

Copyright \copyright{} 2017--2018 Niklas Beisert

This work may be distributed and/or modified under the
conditions of the \LaTeX{} Project Public License, either version 1.3
of this license or (at your option) any later version.
The latest version of this license is in
  \url{http://www.latex-project.org/lppl.txt}
and version 1.3 or later is part of all distributions of \LaTeX{}
version 2005/12/01 or later.

This work has the LPPL maintenance status `maintained'.

The Current Maintainer of this work is Niklas Beisert.

This work consists of the files |README.txt|, |childdoc.ins| and |childdoc.dtx|
as well as the derived files |childdoc.def|, |cdocsamp.tex|
with |cdocsch1.tex|, |cdocsch2.tex|, |cdocspt3.tex|, |cdocspt4.tex|,
|cdocsdrf.tex|, |cdocsfn1.tex|, |cdocsfn2.tex|
as well as |childdoc.pdf|.

%%%%%%%%%%%%%%%%%%%%%%%%%%%%%%%%%%%%%%%%%%%%%%%%%%%%%%%%%%%%%%%%%%%%%%%%%%%%%%%%
\subsection{Files and Installation}

The package consists of the files:
%
\begin{center}
\begin{tabular}{ll}
    |README.txt|   & readme file \\
    |childdoc.ins| & installation file \\
    |childdoc.dtx| & source file \\
    |childdoc.def| & definition file \\
    |cdocsamp.tex| & sample main file \\
    |cdocsch1.tex| & sample include file \\
    |cdocsch2.tex| & sample include file \\
    |cdocspt3.tex| & sample part file \\
    |cdocspt4.tex| & sample part file \\
    |cdocsdrf.tex| & sample redirection file \\
    |cdocsfn1.tex| & sample redirection file \\
    |cdocsfn2.tex| & sample redirection file \\
    |childdoc.pdf| & manual
\end{tabular}
\end{center}
%
The distribution consists of the files
|README.txt|, |childdoc.ins| and |childdoc.dtx|.
%
\begin{itemize}
\item
Run (pdf)\LaTeX{} on |childdoc.dtx|
to compile the manual |childdoc.pdf| (this file).
\item
Run \LaTeX{} on |childdoc.ins| to create the definitions file |childdoc.def|
and the sample |cdocsamp.tex| with include files
|cdocsch1.tex|, |cdocsch2.tex|, |cdocspt3.tex|, |cdocspt4.tex|,
|cdocsdrf.tex|, |cdocsfn1.tex|, |cdocsfn2.tex|.
Then copy the file |childdoc.def| to an appropriate directory of your \LaTeX{}
distribution, e.g.\ \textit{texmf-root}|/tex/latex/childdoc|.
\end{itemize}

%%%%%%%%%%%%%%%%%%%%%%%%%%%%%%%%%%%%%%%%%%%%%%%%%%%%%%%%%%%%%%%%%%%%%%%%%%%%%%%%
\subsection{Related CTAN Packages}

There are several other packages which offer a similar functionality:
%
\begin{itemize}
\item
The packages
\href{http://ctan.org/pkg/docmute}{\textsf{docmute}},
\href{http://ctan.org/pkg/includex}{\textsf{includex}} and
\href{http://ctan.org/pkg/standalone}{\textsf{standalone}}
provide commands to include only the document body of
a child file thus allowing both files to be compiled individually.
\item
The packages \href{http://ctan.org/pkg/subdocs}{\textsf{subdocs}}
and \href{http://ctan.org/pkg/subfiles}{\textsf{subfiles}}
provide structures in which the main and child documents can be
encapsulated and allowing them to be compiled individually.
The inclusion mechanism is different from the conventional |\include|.
\item
The package \href{http://ctan.org/pkg/combine}{\textsf{combine}}
is an elaborate solution to combine several documents into one.
\end{itemize}
%
See also the CTAN topic \href{http://ctan.org/topic/subdocs}{\textsf{subdocs}}
for further related packages.
The present package differs from the above solutions in that
a document structure constructed with the conventional |\include| mechanism
just needs two extra commands at the top of every file
such that all constituent files can be compiled individually.

%%%%%%%%%%%%%%%%%%%%%%%%%%%%%%%%%%%%%%%%%%%%%%%%%%%%%%%%%%%%%%%%%%%%%%%%%%%%%%%%
%\subsection{Feature Suggestions}
%
%The following is a list of features which may be useful for future
%versions of this package:
%%
%\begin{itemize}
%\item
%\ldots
%\end{itemize}

%%%%%%%%%%%%%%%%%%%%%%%%%%%%%%%%%%%%%%%%%%%%%%%%%%%%%%%%%%%%%%%%%%%%%%%%%%%%%%%%
\subsection{Revision History}

%%%%%%%%%%%%%%%%%%%%%%%%%%%%%%%%%%%%%%%%
\paragraph{v2.0:} 2018/12/30

\begin{itemize}
\item
immediate forward processing
\item
added |\childdocby| mechanism
\item
manual restructured
\end{itemize}

%%%%%%%%%%%%%%%%%%%%%%%%%%%%%%%%%%%%%%%%
\paragraph{v1.6:} 2018/01/17

\begin{itemize}
\item
application for development of include files
\item
corrections to manual
\end{itemize}

%%%%%%%%%%%%%%%%%%%%%%%%%%%%%%%%%%%%%%%%
\paragraph{v1.5:} 2017/05/21

\begin{itemize}
\item
more complete structuring introduced
\item
|\childdocof| introduced
\item
|\childdoc| renamed to |\childdocmain|
\item
|\childredirect| renamed to |\childdocforward| and |\childdocforwardprefix|
and functionality expanded
\end{itemize}

%%%%%%%%%%%%%%%%%%%%%%%%%%%%%%%%%%%%%%%%
\paragraph{v1.0:} 2017/04/27

\begin{itemize}
\item
manual and install package
\item
first version published on CTAN
\end{itemize}

%%%%%%%%%%%%%%%%%%%%%%%%%%%%%%%%%%%%%%%%
\paragraph{v0.6:} 2017/04/26

\begin{itemize}
\item
redirection mechanism added
\end{itemize}

%%%%%%%%%%%%%%%%%%%%%%%%%%%%%%%%%%%%%%%%
\paragraph{v0.5:} 2017/04/26

\begin{itemize}
\item
functionality in definition file
\end{itemize}


%%%%%%%%%%%%%%%%%%%%%%%%%%%%%%%%%%%%%%%%%%%%%%%%%%%%%%%%%%%%%%%%%%%%%%%%%%%%%%%%
%%%%%%%%%%%%%%%%%%%%%%%%%%%%%%%%%%%%%%%%%%%%%%%%%%%%%%%%%%%%%%%%%%%%%%%%%%%%%%%%
%%%%%%%%%%%%%%%%%%%%%%%%%%%%%%%%%%%%%%%%%%%%%%%%%%%%%%%%%%%%%%%%%%%%%%%%%%%%%%%%
\appendix

\settowidth\MacroIndent{\rmfamily\scriptsize 000\ }

 \DocInput{childdoc.dtx}

\end{document}
%</driver>
% \fi
%
% %%%%%%%%%%%%%%%%%%%%%%%%%%%%%%%%%%%%%%%%%%%%%%%%%%%%%%%%%%%%%%%%%%%%%%%%%%%%%%
% %%%%%%%%%%%%%%%%%%%%%%%%%%%%%%%%%%%%%%%%%%%%%%%%%%%%%%%%%%%%%%%%%%%%%%%%%%%%%%
% \section{Sample}
%\iffalse
%<*samplemain>
%\fi
%
% The following presents a sample document
% with two chapters, two parts, a title page,
% a compile flag as well as three forwarding files to set the flag.
% It consists of eight |.tex| files:
% \begin{center}
% \begin{tabular}{ll}
% |cdocsamp.tex|&main file\\
% |cdocsch1.tex|&include file for chapter 1\\
% |cdocsch2.tex|&include file for chapter 2\\
% |cdocspt3.tex|&include file for part 3\\
% |cdocspt4.tex|&include file for part 4\\
% |cdocsdrf.tex|&forwarding file for main file in draft mode\\
% |cdocsfi1.tex|&forwarding file for final version of chapter 1\\
% |cdocsfi2.tex|&forwarding file for final version of chapter 2\\
% \end{tabular}
% \end{center}
% Each of the eight files can be compiled directly by the \LaTeX{} compiler.
%
% %%%%%%%%%%%%%%%%%%%%%%%%%%%%%%%%%%%%%%
% \paragraph{Main File.}
%
% The main file is called |cdocsamp.tex|.
%
% Load the \textsf{childdoc} definitions and
% declare the filename for the main document:
%    \begin{macrocode}
\input{childdoc.def}
\childdocmain{}
%    \end{macrocode}

% Optional override for |\version| flag:
%    \begin{macrocode}
%%\ifchilddoc\else\providecommand{\version}{draft}\fi
%    \end{macrocode}

% Define the default values for the |\version| flag
% (|final| for the main file and |draft| for childs):
%    \begin{macrocode}
\ifchilddoc
\providecommand{\version}{draft}
\else
\providecommand{\version}{final}
\fi
%    \end{macrocode}

% Load the standard document class:
%    \begin{macrocode}
\documentclass[12pt]{article}
%    \end{macrocode}

% Start the document body:
%    \begin{macrocode}
\begin{document}
%    \end{macrocode}

% Declare a title page.
% Print title, part of document being processed and version flag:
%    \begin{macrocode}
\addtocounter{page}{-1}
\begin{center}
{\LARGE\bfseries{}childdoc example\par}
\vspace{1cm}
\ifchilddoc
\ifchilddocmanual part\else chapter\fi:
`\childdocname' of `\childdocjob'\par
\else
main document: `\childdocjob'\par
\fi
version: \version\par
\end{center}
\newpage
%    \end{macrocode}

% Manually include selected file,
% otherwise process as usual:
%    \begin{macrocode}
\ifchilddocmanual
\section*{part `\childdocname'}
\input{\childdocname}
\else
%    \end{macrocode}

% Include the two chapters:
%    \begin{macrocode}
\include{cdocsch1}
\include{cdocsch2}
%    \end{macrocode}

% Include the two parts unless only chapters should be displayed:
%    \begin{macrocode}
\ifchilddoc\else
\section{part three}
\input{cdocspt3}
\section{part four}
\input{cdocspt4}
\fi
%    \end{macrocode}

% Process as usual until here:
%    \begin{macrocode}
\fi
%    \end{macrocode}

% End of document body:
%    \begin{macrocode}
\end{document}
%    \end{macrocode}
%\iffalse
%</samplemain>
%\fi
%
% %%%%%%%%%%%%%%%%%%%%%%%%%%%%%%%%%%%%%%
% \paragraph{Chapter Include Files.}
%
% The include files are called |cdocsch1.tex| and |cdocsch2.tex|.
%
%\iffalse
%<*samplechap1|samplechap2>
%\fi

% Optional override for |\version| flag:
%    \begin{macrocode}
%%\providecommand{\version}{final}
%    \end{macrocode}

% Include the main document:
%    \begin{macrocode}
\input{childdoc.def}
\childdocof{cdocsamp}
%    \end{macrocode}

%\iffalse
%</samplechap1|samplechap2>
%\fi
%
%\iffalse
%<*samplechap1>
%\fi
% Some text for chapter 1:
%    \begin{macrocode}
\section{one}
some text in chapter one
%    \end{macrocode}

%\iffalse
%</samplechap1>
%\fi
% Some text for chapter 2:
%\iffalse
%<*samplechap2>
%\fi
%    \begin{macrocode}
\section{two}
more text in chapter two
%    \end{macrocode}

%\iffalse
%</samplechap2>
%\fi
%
% %%%%%%%%%%%%%%%%%%%%%%%%%%%%%%%%%%%%%%
% \paragraph{Part Include Files.}
%
% The include files are called |cdocspt3.tex| and |cdocspt4.tex|.
%
%\iffalse
%<*samplepart3|samplepart4>
%\fi

% Optional override for |\version| flag:
%    \begin{macrocode}
%%\providecommand{\version}{final}
%    \end{macrocode}

% Include the main document:
%    \begin{macrocode}
\input{childdoc.def}
\childdocby{cdocsamp}
%    \end{macrocode}

%\iffalse
%</samplepart3|samplepart4>
%\fi
%
%\iffalse
%<*samplepart3>
%\fi
% Some text for part 3:
%    \begin{macrocode}
some text in part three
%    \end{macrocode}

%\iffalse
%</samplepart3>
%\fi
% Some text for part 4:
%\iffalse
%<*samplepart4>
%\fi
%    \begin{macrocode}
more text in part four
%    \end{macrocode}

%\iffalse
%</samplepart4>
%\fi
%
% %%%%%%%%%%%%%%%%%%%%%%%%%%%%%%%%%%%%%%
% \paragraph{Forwarding for a Complete Draft.}
%
% The following forwarding file |cdocsdrf.tex|
% compiles the main document in draft mode:
%\iffalse
%<*sampledraft>
%\fi
%    \begin{macrocode}
\def\version{draft}
\input{childdoc.def}
\childdocforward{cdocsamp}
%    \end{macrocode}

%\iffalse
%</sampledraft>
%\fi
%
% %%%%%%%%%%%%%%%%%%%%%%%%%%%%%%%%%%%%%%
% \paragraph{Forwarding for Final Version of the Chapters.}
%
% The following forwarding files |cdocsfn1.tex| and |cdocsfn2.tex|
% (with identical content)
% compile the final versions of the child documents
% |cdocsch1.tex| and |cdocsch2.tex|, respectively:
%\iffalse
%<*samplefinal>
%\fi
%    \begin{macrocode}
\def\version{final}
\input{childdoc.def}
\childdocforwardprefix[cdocsamp]{cdocsfn}{cdocsch}
%    \end{macrocode}

%\iffalse
%</samplefinal>
%\fi
%
% %%%%%%%%%%%%%%%%%%%%%%%%%%%%%%%%%%%%%%
% \paragraph{Command Line Processing.}
%
% The following three command lines generate the output files
% |cdocscld|, |cdocscl1| and |cdocscl2|
% which should be identical to
% |cdocsdrf|, |cdocsch1| and |cdocsfn2|, respectively:
% \begin{center}
% \begin{tabular}{l}
% |latex -jobname cdocscld \|\\
% |  "\def\version{draft}\input{childdoc.def}\childdocforward{cdocsamp}"|\\
% |latex -jobname cdocscl1 \|\\
% |  "\input{childdoc.def}\childdocforward[cdocsamp]{cdocsch1}"|\\
% |latex -jobname cdocscl2 \|\\
% |  "\def\version{final}\input{childdoc.def}\childdocforward{cdocsch2}"|
% \end{tabular}
% \end{center}
% Note that the trailing backslash on each first line
% merely continues the input to the second line
% (for convenient cut ant paste).
% Furthermore, the command |latex| can be replaced by any
% of its alternative versions such as |pdflatex|.
%
% %%%%%%%%%%%%%%%%%%%%%%%%%%%%%%%%%%%%%%%%%%%%%%%%%%%%%%%%%%%%%%%%%%%%%%%%%%%%%%
% %%%%%%%%%%%%%%%%%%%%%%%%%%%%%%%%%%%%%%%%%%%%%%%%%%%%%%%%%%%%%%%%%%%%%%%%%%%%%%
% \section{Implementation}
%\iffalse
%<*package>
%\fi
%
% This section describes the definitions file |childdoc.def|.

% The definitions cannot be loaded using |\usepackage| or |\RequirePackage|
% which has a mechanism to prevent loading a style file more than once.
% When loading the definitions by means of |\input|
% multiple instances have to be prevented manually:
%\iffalse
%This code needs to be before the `\ProvidesFile' directive
%which is defined at the beginning of this file.
%Therefore it is also placed there and commented out here.
%</package>
%<*discard>
%\fi
%    \begin{macrocode}
\ifdefined\childdocmain\endinput\fi
%    \end{macrocode}
%\iffalse
%</discard>
%<*package>
%\fi
%
% \macro{\ifchilddoc}
% \macro{\ifchilddocmanual}
% The conditional |\ifchilddoc| tells whether a
% child (true) or main (false) document is being compiled.
% The conditional |\ifchilddocmanual| tells whether
% the |\includeonly| mechanism is used (false) or
% the selection of child files must be performed manually (true).
% The definitions initialise to false:
%    \begin{macrocode}
\newif\ifchilddoc
\newif\ifchilddocmanual
%    \end{macrocode}

% \macro{\childdocname}
% \macro{\childdocjob}
% The macro |\childdocname| stores the name of the main document
% to be compiled. The macro |\childdocjob| stores the name of
% the document on which the \LaTeX{} compiler was originally invoked.
% The content of |\jobname| cannot be compared
% to filenames specified in the source due to different catcodes.
% The following code rescans |\jobname|, stores the result
% in |\childdocname| and saves a copy in |\childdocjob|:
%    \begin{macrocode}
\edef\childdocname{\scantokens\expandafter{\jobname\noexpand}}
\let\childdocjob\childdocname
%    \end{macrocode}

% \macro{\childdocdisable}
% The macro |\childdocdisable| prevents the main file
% from being processed more than once.
% At this stage, the main document command |\childdocmain|
% is assumed to be called once again where it should do nothing.
% Any subsequent call to it should prevent
% a secondary processing of the main document
% It overwrites the forwarding commands
% |\childdocof| and |\childdocforward|
% with empty macros to prevent further inclusions of the main document:
%    \begin{macrocode}
\newcommand{\childdocdisable}
{
  \renewcommand{\childdocmain}[1]{\renewcommand{\childdocmain}[1]{\endinput}}
  \renewcommand{\childdocof}[1]{}
  \renewcommand{\childdocby}[2][]{}
  \renewcommand{\childdocforward}[2][]{}
  \renewcommand{\childdocdisable}{}
}
%    \end{macrocode}

% \macro{\childdocmain}
% The macro |\childdocmain| is to be called at the top of the main file
% with nothing or the main filename (without extension) as argument.
% First, it breaks loops.
% If the argument is not empty and does not match |\childdocname|
% (which is set by the first inclusion of |childdoc.def|),
% |\ifchilddoc| is set to true, |\includeonly| is applied to the child file
% and |\jobname| is set to the main file
% (for proper handling of |.aux| files):
%    \begin{macrocode}
\newcommand{\childdocmain}[1]
{
  \childdocdisable\childdocmain{}
  \if?#1?\else
    \begingroup
      \def\childdoctmp{#1}
      \ifx\childdoctmp\childdocname
        \def\childdoctmp{}
      \else
        \def\childdoctmp
        {
          \childdoctrue
          \includeonly{\childdocname}
          \def\childdocjob{#1}
          \def\jobname{#1}
        }
      \fi
      \expandafter
    \endgroup
    \childdoctmp
  \fi
}
%    \end{macrocode}

% \macro{\childdocof}
% The command |\childdocof| redirects
% compilation to the main file |#1|.
%    \begin{macrocode}
\newcommand{\childdocof}[1]
{
  \childdocdisable
  \childdoctrue
  \includeonly{\childdocname}
  \def\jobname{#1}
  \def\childdocjob{#1}
  \input{#1}
}
%    \end{macrocode}

% \macro{\childdocby}
% The command |\childdocby| ....
%    \begin{macrocode}
\newcommand{\childdocby}[2][]
{
  \childdocdisable
  \childdoctrue
  \childdocmanualtrue
  \if?#1?\else
    \def\jobname{#2}
  \fi
  \def\childdocjob{#2}
  \input{#2}
  \endinput
}
%    \end{macrocode}

% \macro{\childdocforward}
% The command |\childdocforward| redirects
% compilation to the main file or
% (if the optional argument is given) a child file.
% Parameters are set as if the main file
% or a child file starting with |\childdocof| was compiled.
% Then compilation is handed over to the main file:
%    \begin{macrocode}
\newcommand{\childdocforward}[2][]
{
  \begingroup
    \if?#1?
      \def\childdoctmp
      {
        \def\childdocname{#2}
        \def\childdocjob{#2}
        \def\jobname{#2}
        \input{#2}
        \endinput
      }
    \else
      \def\childdoctmp
      {
        \childdocdisable
        \def\childdocname{#2}
        \childdoctrue
        \includeonly{#2}
        \def\childdocjob{#1}
        \def\jobname{#1}
        \input{#1}
        \endinput
      }
    \fi
    \expandafter
  \endgroup
  \childdoctmp
}
%    \end{macrocode}

% \macro{\childdocforwardprefix}
% The command |\childdocforwardprefix| redirects
% compilation to the main or a child file by means of a pattern.
% The prefix |#1| in the current filename is replaced by |#2|
% and the suffix of the current filename is kept
% (it is assumed that the filename does not contain the substring `|~~~|'
% which is used as a delimiter).
% Compilation is handed over to the new file by |\childdocforward|:
%    \begin{macrocode}
\newcommand{\childdocforwardprefix}[3][]
{
  \begingroup
    \def\childdocextract #2##1~~~{\def\childdoctmp{\childdocforward[#1]{#3##1}}}
    \expandafter\childdocextract\childdocname~~~
    \expandafter
  \endgroup
  \childdoctmp
}
%    \end{macrocode}

% \macro{\childdoc}
% The deprecated macro |\childdoc| is a legacy version of |\childdocmain|:
%    \begin{macrocode}
\newcommand{\childdoc}{\childdocmain}
%    \end{macrocode}

% \macro{\childdocredirect}
% The deprecated macro |\childdocredirect| is a legacy version
% of |\childdocforward| and |\childdocforwardprefix|:
%    \begin{macrocode}
\newcommand{\childdocredirect}[2][]
{
  \begingroup
    \if?#1?
      \def\childdoctmp{\childdocforward{#2}}
    \else
      \def\childdoctmp{\childdocforwardprefix{#1}{#2}}
    \fi
    \expandafter
  \endgroup
  \childdoctmp
}
%    \end{macrocode}

%\iffalse
%</package>
%\fi
%
\endinput
|\\
|\childdocforward[|\textit{main}|]{|\textit{dest}|}|\\
\end{tabular}
\end{center}
%
The argument \textit{dest} is the destination file
(without extension).
It should be the main file or one of the child files.
Note that further \textsf{childdoc} directives
such as |\childdocof| and |\childdocforward|
in the indicated file will be processed in this form.
The optional argument \textit{main}
passes on directly to the main file \textit{main}
while pretending to compile the child \textit{dest}.
This form behaves as if \textit{dest}
issues |\childdocof{|\textit{main}|}| right away,
and no further \textsf{childdoc} directives will be processed.

%%%%%%%%%%%%%%%%%%%%%%%%%%%%%%%%%%%%%%%%
\DescribeMacro{\...prefix}
In the alternative form |\childdocforwardprefix|,
%
\begin{center}
\begin{tabular}{l}
|% \iffalse
%
% childdoc.dtx Copyright (C) 2017-2018 Niklas Beisert
%
% This work may be distributed and/or modified under the
% conditions of the LaTeX Project Public License, either version 1.3
% of this license or (at your option) any later version.
% The latest version of this license is in
%   http://www.latex-project.org/lppl.txt
% and version 1.3 or later is part of all distributions of LaTeX
% version 2005/12/01 or later.
%
% This work has the LPPL maintenance status `maintained'.
%
% The Current Maintainer of this work is Niklas Beisert.
%
% This work consists of the files childdoc.dtx and childdoc.ins
% and the derived files childdoc.def and cdocsamp.tex with
% cdocsch1.tex, cdocsch2.tex, cdocsdrf.tex, cdocsfn1.tex, cdocsfn2.tex.
%
%<package>\ifdefined\childdocmain\endinput\fi
%<package>\ProvidesFile{childdoc.def}[2018/12/30 v2.0 child document driver]
%<samplemain>\ProvidesFile{cdocsamp.tex}[2018/12/30 v2.0 sample for childdoc]
%<*driver>
%\ProvidesFile{childdoc.drv}[2018/12/30 v2.0 childdoc reference manual file]
\PassOptionsToClass{10pt,a4paper}{article}
\documentclass{ltxdoc}

\usepackage[margin=35mm]{geometry}
\usepackage{hyperref}
\usepackage{hyperxmp}
\usepackage[usenames]{color}

\hypersetup{colorlinks=true}
\hypersetup{pdfstartview=FitH}
\hypersetup{pdfpagemode=UseNone}
\hypersetup{pdfsource={}}
\hypersetup{pdflang={en-UK}}
\hypersetup{pdfcopyright={Copyright 2017-2018 Niklas Beisert.
  This work may be distributed and/or modified under the
  conditions of the LaTeX Project Public License, either version 1.3
  of this license or (at your option) any later version.}}
\hypersetup{pdflicenseurl={http://www.latex-project.org/lppl.txt}}
\hypersetup{pdfcontactaddress={ETH Zurich, ITP, HIT K,
  Wolfgang-Pauli-Strasse 27}}
\hypersetup{pdfcontactpostcode={8093}}
\hypersetup{pdfcontactcity={Zurich}}
\hypersetup{pdfcontactcountry={Switzerland}}
\hypersetup{pdfcontactemail={nbeisert@itp.phys.ethz.ch}}
\hypersetup{pdfcontacturl={http://people.phys.ethz.ch/\xmptilde nbeisert/}}

\newcommand{\secref}[1]{\hyperref[#1]{section \ref*{#1}}}

\parskip1ex
\parindent0pt
\let\olditemize\itemize
\def\itemize{\olditemize\parskip0pt}

\begin{document}

\title{The \textsf{childdoc} Package}
\hypersetup{pdftitle={The childdoc Package}}
\author{Niklas Beisert\\[2ex]
  Institut f\"ur Theoretische Physik\\
  Eidgen\"ossische Technische Hochschule Z\"urich\\
  Wolfgang-Pauli-Strasse 27, 8093 Z\"urich, Switzerland\\[1ex]
  \href{mailto:nbeisert@itp.phys.ethz.ch}
  {\texttt{nbeisert@itp.phys.ethz.ch}}}
\hypersetup{pdfauthor={Niklas Beisert}}
\hypersetup{pdfsubject={Manual for the LaTeX2e Package childdoc}}
\date{30 December 2018, \textsf{v2.0}}
\maketitle

\begin{abstract}\noindent
\textsf{childdoc} is a \LaTeXe{} package
that enables the direct compilation
of document sections included by |\include|
to individual files.
\end{abstract}

\begingroup
\parskip0ex
\tableofcontents
\endgroup

%%%%%%%%%%%%%%%%%%%%%%%%%%%%%%%%%%%%%%%%%%%%%%%%%%%%%%%%%%%%%%%%%%%%%%%%%%%%%%%%
%%%%%%%%%%%%%%%%%%%%%%%%%%%%%%%%%%%%%%%%%%%%%%%%%%%%%%%%%%%%%%%%%%%%%%%%%%%%%%%%
\section{Introduction}

\LaTeX{} provides a mechanism to structure a large document (such as a book)
into a main file and several child files (containing the chapters)
using the |\include| command.
This mechanism is beneficial for documents
which span hundreds of pages in order to
make the source file(s) more manageable.
Moreover, compilation can be restricted to
selected child files by means of the |\includeonly| command.
The latter feature can be used to reduce the compilation time while editing
(this was significantly more useful in the earlier days of \LaTeX{})
or to generate a smaller document which is easier to navigate.
Another application of |\includeonly| is to generate
documents consisting of selected parts of the complete document.

However, there are a few drawbacks of the plain |\include| mechanism:
\begin{itemize}
\item
The child files cannot be compiled on their own,
they can only be compiled via the main file.
A naive editing environment
(such as a text editor with an option
to have the current file processed by \LaTeX)
may require one to switch to the main file before compiling;
attempting to compile the child file produces errors.
\item
The main file must be modified (each time)
to adjust the |\includeonly| command
to the present needs. This easily leaves the main file in a messy state.
\item
The generated document will always carry the filename
of the main document. This is inconvenient if
several child files are to be compiled and
to be kept for distribution.
\end{itemize}

The present package provides a simple interface
to make child files individually compilable by \LaTeX{}.
Compiling a child file then has the same effect as compiling
the main file with an |\includeonly| command
to select the appropriate child.
Moreover the generated document will carry the name of the child
rather than the main file.
This resolves all three above issues.

This feature is meant to make the editing of books,
thesis documents and lecture notes somewhat more convenient.
However, the package can also be used efficiently for
composing a series of documents (such as exercise sheets)
which are typically distributed individually.
It then assists the author in generating the individual documents
(potentially in different versions)
as well as a document containing the collected series.
Another application is in developing style files
or other kinds of included material
where compilation of the style file could redirect
to a sample or test file.

%%%%%%%%%%%%%%%%%%%%%%%%%%%%%%%%%%%%%%%%%%%%%%%%%%%%%%%%%%%%%%%%%%%%%%%%%%%%%%%%
%%%%%%%%%%%%%%%%%%%%%%%%%%%%%%%%%%%%%%%%%%%%%%%%%%%%%%%%%%%%%%%%%%%%%%%%%%%%%%%%
\section{Usage}

First of all, the package \textsf{childdoc} is \emph{not} a standard
\LaTeXe{} |.sty| style file! Therefore it needs to be invoked in
a non-standard way.

%%%%%%%%%%%%%%%%%%%%%%%%%%%%%%%%%%%%%%%%%%%%%%%%%%%%%%%%%%%%%%%%%%%%%%%%%%%%%%%%
\subsection{Included Files}
\label{sec:include}

%%%%%%%%%%%%%%%%%%%%%%%%%%%%%%%%%%%%%%%%
\DescribeMacro{\childdocmain}
To use the package, add the commands
\begin{center}
\begin{tabular}{l}
|\input{childdoc.def}|\\
|\childdocmain{}|\\
\end{tabular}
\end{center}
at the very top of the main \LaTeX{} file,
in particular \emph{before} the |\documentclass| statement!
The argument of |\childdocmain| should be left empty
(but it must be present).

%%%%%%%%%%%%%%%%%%%%%%%%%%%%%%%%%%%%%%%%
\DescribeMacro{\childdocof}
Furthermore, add the commands
\begin{center}
\begin{tabular}{l}
|\input{childdoc.def}|\\
|\childdocof{|\textit{main}|}|\\
\end{tabular}
\end{center}
at the top of every child file \textit{child}
which is included by |\include{|\textit{child}|}|
from within the main file
(or at least for those files to be compiled individually).
The argument \textit{main} must be the filename of the main file.

There are a couple of
considerations in setting up the main and child documents:

%%%%%%%%%%%%%%%%%%%%%%%%%%%%%%%%%%%%%%%%
\paragraph{Restrictions.}

Please note the following restrictions:
\begin{itemize}
\item
|\childdocmain| must be called with one argument \textit{main}
to ensure compatibility with earlier version of the package.
It must either be empty (|\childdocmain{}|)
or precisely match the filename of the main file in which it is specified.
See \secref{sec:detection} for further information.
\item
The filename \textit{main} must be specified without the |.tex| extension.
\item
The filename \textit{main} is case sensitive
(even in case-insensitive file systems)
due to internal string comparison.
\item
The argument \textit{main} should be fully expanded, it cannot be a macro.
\item
Subdirectories and special characters should be avoided in filenames.
\item
The command |\childdocmain{|\textit{main}|}| must be followed by a whitespace.
It should not be followed immediately by another command
or by a comment mark `|%|'.
This is because the \TeX{} parser reads the token immediately following
the argument of |\childdocmain| and puts it
at the beginning of every child section;
however, a white\-space is ignored.
\end{itemize}

%%%%%%%%%%%%%%%%%%%%%%%%%%%%%%%%%%%%%%%%
\paragraph{Content of Main File.}

It is advisable to place all content in the child files included by |\include|.
Any output contained in the main file will appear in all child documents
unless suppressed manually;
it cannot be suppressed automatically by the |\includeonly| directive
and thus should normally be avoided.
A method to include some content in the main file
by means of conditional processing is described in \secref{sec:conditional}.

%%%%%%%%%%%%%%%%%%%%%%%%%%%%%%%%%%%%%%%%
\paragraph{Page Numbering.}

When only a part of the document is compiled,
the appropriate numbering of pages
(as well as other status parameters)
is determined from the |.aux| files.
The latter contain information from previous passes.
However this information needs to propagate through
all intermediate child documents.
Therefore the page numbering in child documents may well
be inconsistent until the complete document is compiled at least once.

A useful (if unconventional) way to always ensure a consistent
page numbering is to restart the numbering in each child document
and denote the pages by `\textit{child}|.|\textit{page}'
where \textit{child} represents the chapter/section number of the child file.
This can be achieved by the command
|\numberwithin{page}{|\textit{child}|}|
of the \textsf{amsmath} package
where \textit{child} can be |chapter| or |section|
depending on the chosen structuring.
Alternatively, one can modify the macro |\thepage| appropriately
and reset the counter |page| at the start of each child file.

%%%%%%%%%%%%%%%%%%%%%%%%%%%%%%%%%%%%%%%%%%%%%%%%%%%%%%%%%%%%%%%%%%%%%%%%%%%%%%%%
\subsection{Conditional Processing}
\label{sec:conditional}

The package provides a mechanism to compile different versions
of a document. To customise the versions further some conditional processing
can come in handy to distinguish which version is being compiled.
The package provides two macros to describe the compilation context:

%%%%%%%%%%%%%%%%%%%%%%%%%%%%%%%%%%%%%%%%
\DescribeMacro{\ifchilddoc}
The conditional |\ifchilddoc| distinguishes between the compilation of
child documents and the main document:
%
\begin{center}
|\ifchilddoc |\textit{child-code}| |[|\||else |\textit{main-code}]| \||fi|
\end{center}

%%%%%%%%%%%%%%%%%%%%%%%%%%%%%%%%%%%%%%%%
\DescribeMacro{\childdocname}
\DescribeMacro{\childdocjob}
The macro |\childdocname| contains the filename (without extension)
of the main or child file being processed.
Note that |\childdocjob| will always contain the name of the main file.

%%%%%%%%%%%%%%%%%%%%%%%%%%%%%%%%%%%%%%%%
\paragraph{Title Page.}

Conditional processing can be used to include a title or banner page
in the main document when proper precautions are taken.
Importantly, the code in the main file should ensure that the page counter
(as well as other status parameters which are stored in the |.aux| files)
takes the same value after the conditional processing.
Otherwise the page numbers may take divergent values
depending on which part is compiled.

For example, a title page could be declared by:
%
\begin{center}
\begin{tabular}{l}
|\ifchilddoc\||else|\\
|\addtocounter{page}{-1}|\\
\textit{code for title page}\\
|\newpage|\\
|\||fi|
\end{tabular}
\end{center}
%
A banner page for the child documents can be generated by:
%
\begin{center}
\begin{tabular}{l}
|\ifchilddoc|\\
|\addtocounter{page}{-1}|\\
\textit{code for banner page}\\
|\newpage|\\
|\||fi|
\end{tabular}
\end{center}
%
Here one could write a message such as:
\begin{center}
|This is the part \childdocname{} of \childdocjob{}.|
\end{center}

%%%%%%%%%%%%%%%%%%%%%%%%%%%%%%%%%%%%%%%%%%%%%%%%%%%%%%%%%%%%%%%%%%%%%%%%%%%%%%%%
\subsection{Flags}
\label{sec:flags}

The package makes it easy to generate different versions
of the main or child documents.
To this end compilation flags can be defined
and assigned different default values.
They will be particularly useful in conjunction
with the forwarding mechanism described in \secref{sec:forward}.

For example, it may be useful to have a flag |\version|
which can be set to |draft| or |final|.
The document source will contain some conditional code
depending on the value of |\version|.
Suppose further, the flag should default to |final| for the main file
and to |draft| for child files
which is a natural assignment for editing the document.
This is achieved by placing the following code
in the preamble of the main document
(below the |\childdocmain| directive):
%
\begin{center}
\begin{tabular}{l}
|\ifchilddoc|\\
|\providecommand{\version}{draft}|\\
|\||else|\\
|\providecommand{\version}{final}|\\
|\||fi|
\end{tabular}
\end{center}
%
The definition by |\providecommand| makes sure
that previous definitions are not overwritten.
Further statements |\providecommand{\version}{...}|
can thus be added before the above code to override it.

For the main file, one might add a line
(between |\childdocmain| and the above block)
%
\begin{center}
|%\ifchilddoc\||else\providecommand{\version}{draft}\||fi|
\end{center}
%
which can be uncommented to produce a draft version.
Likewise one can add a line to the very top of a child file
(above the |\childdocof{|\textit{main}|}| directive)
%
\begin{center}
|%\providecommand{\version}{final}|
\end{center}
%
which can be uncommented to produce the final version of this child document.

%%%%%%%%%%%%%%%%%%%%%%%%%%%%%%%%%%%%%%%%%%%%%%%%%%%%%%%%%%%%%%%%%%%%%%%%%%%%%%%%
\subsection{Forwarding}
\label{sec:forward}

Different versions of the main or child documents
using compilation flags as described in \secref{sec:flags}
can be (permanently) stored in different files
for convenient compilation, viewing and distribution.
To this end, the package defines a command
to pass on compilation to a different file:

%%%%%%%%%%%%%%%%%%%%%%%%%%%%%%%%%%%%%%%%
\DescribeMacro{\childdocforward}
The command |\childdocforward| redirects processing to
another source file:
%
\begin{center}
\begin{tabular}{l}
|\input{childdoc.def}|\\
|\childdocforward[|\textit{main}|]{|\textit{dest}|}|\\
\end{tabular}
\end{center}
%
The argument \textit{dest} is the destination file
(without extension).
It should be the main file or one of the child files.
Note that further \textsf{childdoc} directives
such as |\childdocof| and |\childdocforward|
in the indicated file will be processed in this form.
The optional argument \textit{main}
passes on directly to the main file \textit{main}
while pretending to compile the child \textit{dest}.
This form behaves as if \textit{dest}
issues |\childdocof{|\textit{main}|}| right away,
and no further \textsf{childdoc} directives will be processed.

%%%%%%%%%%%%%%%%%%%%%%%%%%%%%%%%%%%%%%%%
\DescribeMacro{\...prefix}
In the alternative form |\childdocforwardprefix|,
%
\begin{center}
\begin{tabular}{l}
|\input{childdoc.def}|\\
|\childdocforwardprefix[|\textit{main}|]{|\textit{prefix}|}{|\textit{dest}|}|
\end{tabular}
\end{center}
%
the destination file is determined by a pattern
depending on the current file:
To make this work, the current file must be called
`{\textit{prefix}\hspace{0.2em}\textit{suffix}}'
with \textit{prefix} matching precisely the argument.
Processing is then passed on to the file
`{\textit{dest}\hspace{0.2em}\textit{suffix}}'.
Surely, the same effect is achieved by
directly specifying the
argument `{\textit{dest}\hspace{0.2em}\textit{suffix}}'
in the first form.
However, that requires to set up a different file
for each child. With the alternative form of the command
all these files can have exactly the same content
which simplifies setting them up and maintaining them.

For example, the following file |draft.tex|
with a compilation flag |\version| as described in \secref{sec:flags}
compiles the main document as a draft:
%
\begin{center}
\begin{tabular}{l}
|\def\version{draft}|\\
|\input{childdoc.def}|\\
|\childdocforward{|\textit{main}|}|
\end{tabular}
\end{center}
%
Likewise, the following files |final|\textit{nn}|.tex|
compile the final version of the child document
|child|\textit{nn}|.tex|:
%
\begin{center}
\begin{tabular}{l}
|\def\version{final}|\\
|\input{childdoc.def}|\\
|\childdocforwardprefix{final}{child}|
\end{tabular}
\end{center}
%

Note that when several versions of a main file and/or of each child file
are to be generated, it may be convenient to set up a |Makefile| or
shell script to automatise the process.

%%%%%%%%%%%%%%%%%%%%%%%%%%%%%%%%%%%%%%%%%%%%%%%%%%%%%%%%%%%%%%%%%%%%%%%%%%%%%%%%
\subsection{Command Line Processing}
\label{sec:commandline}

The effect of redirection files can also be achieved by invoking
the \LaTeX{} compiler with a more elaborate command line.
Most conveniently this should be done as part
of a shell script or a |Makefile|.

When using \textsf{childdoc} in the main file, the following
command lines effectively perform a redirection
(note that depending on the shell being used,
backslashes may have to be doubled: `|\|' $\to$ `|\\|'):
%
\begin{center}
|... -jobname "|\textit{target}|" |\\|"|[\textit{flags}]%
|\input{childdoc.def}\childdocforward[|\textit{main}|]{|\textit{dest}|}"|
\end{center}
%
Here \textit{target} is the name of the output file,
\textit{main} is the name of the main file
and \textit{dest} is the name of the main or child file to be processed
(all filenames without extensions).
The optional argument \textit{main} can be omitted
if \textit{main} matches \textit{dest}.
Optionally, compilation \textit{flags} can be defined via |\def| commands.
This command line makes the \TeX{} engine believe
it is compiling the file \textit{target}
whose content is specified as the latter parameter.
The provided code then forwards the processing to
\textit{main} or \textit{dest} as described in \secref{sec:forward}.

%%%%%%%%%%%%%%%%%%%%%%%%%%%%%%%%%%%%%%%%%%%%%%%%%%%%%%%%%%%%%%%%%%%%%%%%%%%%%%%%
\subsection{Include by Input}
\label{sec:input}

Including child documents by |\include| has some restrictions by design.
Most notably, the content of a child document always occupies
its own set of pages; pages cannot be shared between child documents.
Usually, this behaviour makes perfect sense
because each child document contain an essential part of the document.
However, in some situations it may be desirable to compose
a document from a collection of parts
without having mandatory page breaks between then.
For this case, the package
provides a mechanism to include parts
by |\input| which can also be processed individually.
However, by construction this mechanism
requires manual handling of the content to be output.

%%%%%%%%%%%%%%%%%%%%%%%%%%%%%%%%%%%%%%%%
\DescribeMacro{\ifchilddocmanual}
The main file should be prepared as usual, see \secref{sec:include}.
However, the document body must make a distinction
between processing of an individual part and of the main document, e.g.:
%
\begin{center}
\begin{tabular}{l}
|\ifchilddocmanual|\\
|\input{\childdocname}|\\
|\||else|\\
\textit{document body with }|\input{|\textit{part}|}|\\
|\||fi|
\end{tabular}
\end{center}
%
The conditional |\ifchilddocmanual| is true whenever
a part to be included by |\input| is being compiled,
and the name of the part is stored in |\childdocname|.

%%%%%%%%%%%%%%%%%%%%%%%%%%%%%%%%%%%%%%%%
\DescribeMacro{\childdocby}
Each part to be included by |\input| should start with:
%
\begin{center}
\begin{tabular}{l}
|\input{childdoc.def}|\\
|\childdocby{|\textit{main}|}|\\
\end{tabular}
\end{center}
%
The directive |\childdocby| is similar to |\childdocof|
described in \secref{sec:include},
but the subsequent selection of content must be done manually.
To that end, both |\ifchilddoc| and |\ifchilddocmanual|
will be true upon processing of a part,
and the name of the part is stored in |\childdocname|.
Note that |\jobname| will be set to the filename of the current part
so that each part receives an individual |.aux| file
that does not interfere with the |.aux| file(s) of the main document.
This behaviour can be altered by the alternative form
|\childdocby[*]{|\textit{main}|}| (with a non-empty optional argument)
which uses the |.aux| file of the main document
by setting |\jobname| to \textit{main}.

%%%%%%%%%%%%%%%%%%%%%%%%%%%%%%%%%%%%%%%%%%%%%%%%%%%%%%%%%%%%%%%%%%%%%%%%%%%%%%%%
\subsection{Driver Development}
\label{sec:driver}

The \textsf{childdoc} mechanism can also be use for the development
of definition files such as \LaTeX{} styles or classes.
This case differs from the above setup with multiple parts
included by |\include| in that no |\includeonly| should be invoked.
This can be achieved by starting the include file
(before |\ProvidesPackage|) with:
%
\begin{center}
\begin{tabular}{l}
|\input{childdoc.def}|\\
|\childdocforward{|\textit{main}|}|\\
\end{tabular}
\end{center}
%
or alternatively with:
%
\begin{center}
\begin{tabular}{l}
|\input{childdoc.def}|\\
|\childdocby{|\textit{main}|}|\\
\end{tabular}
\end{center}
%
Both forms have slightly different effects as described above.
The main file is prepared as usual, see \secref{sec:include}.

%%%%%%%%%%%%%%%%%%%%%%%%%%%%%%%%%%%%%%%%%%%%%%%%%%%%%%%%%%%%%%%%%%%%%%%%%%%%%%%%
\subsection{Legacy Detection}
\label{sec:detection}

The directive |\childdocmain| in the main file can detect
whether the complete document or merely a child is to be compiled
even without using the directive |\childdocof|.
This method is deprecated because it is less robust
and there is no compelling reason to use it;
it is merely provided for backward compatibility
and it may be removed in future versions.

If the detection mechanism is to be used,
it is mandatory to correctly specify
the filename of the main file as the argument of |\childdocmain|:
%
\begin{center}
\begin{tabular}{l}
|\input{childdoc.def}|\\
|\childdocmain{|\textit{main}|}|\\
\end{tabular}
\end{center}
%
If |\jobname| does not match the argument \textit{main} of |\childdocmain|,
it is assumed that |\jobname| points to the child file to be compiled.
When using |\childdocmain| with the main file specified as argument,
it suffices to start a child file
with just |\input{|\textit{main}|}|
without loading of the package and using |\childdocof|.
If instead all processing is done
with the appropriate \textsf{childdoc} directives,
the argument of \textit{main} of |\childdocmain| can be empty.

An alternative version of the command line processing described
in \secref{sec:commandline} using the detection mechanism reads:
%
\begin{center}
|... -jobname "|\textit{target}|" "|[\textit{flags}]%
[|\def\jobname{|\textit{dest}|}|]|\input{|\textit{main}|}"|
\end{center}

%%%%%%%%%%%%%%%%%%%%%%%%%%%%%%%%%%%%%%%%%%%%%%%%%%%%%%%%%%%%%%%%%%%%%%%%%%%%%%%%
\subsection{Manual Code}
\label{sec:manual}

In case one cannot be certain whether the definitions file |childdoc.def|
is installed on the target \TeX{} distribution
and one prefers not to ship it,
it is conceivable to paste a few relevant commands into the sources.

To that end, drop all statements |\input{childdoc.def}|
and perform the replacements as outlined below.
Instead of |\childdocmain{|\textit{main}|}| add the following code
to the top of the main file:
%
\begin{center}
\begin{tabular}{l}
|\||ifdefined\childdocname\endinput\||fi\newif\ifchilddoc|\\
|\edef\childdocname{\scantokens\expandafter{\jobname\noexpand}}|\\
|\def\childdocmain{|\textit{main}|}\||ifx\childdocmain\childdocname\||else|\\
|\childdoctrue\includeonly{\childdocname}\let\jobname\childdocmain\||fi|\\
\end{tabular}
\end{center}
%
Instead of |\childdocof{|\textit{main}|}| just include the main file
at the top of each child file:
%
\begin{center}
|\input{|\textit{main}|}|
\end{center}
%
A simple redirection |\childdocforward{|\textit{dest}|}| is achieved by:
%
\begin{center}
|\def\jobname{|\textit{dest}|}\input{\jobname}|
\end{center}
%
The redirection with prefix
|\childdocforwardprefix[|\textit{prefix}|]{|\textit{dest}|}|
is accomplished by:
%
\begin{center}
\begin{tabular}{l}
|{\edef\jobname{\scantokens\expandafter{\jobname\noexpand}}|\\
|\def\redirectjob |\textit{prefix}|#1~~~{\gdef\jobname{|\textit{dest}|#1}}|\\
|\expandafter\redirectjob\jobname~~~}\input{\jobname}|
\end{tabular}
\end{center}

In an alternative approach,
child documents can be compiled by a specific command line
without additional code or specific definitions:
%
\begin{center}
|... -jobname "|\textit{target}|" "|[\textit{flags}]%
|\includeonly{|\textit{dest}|}\input{|\textit{main}|}"|
\end{center}
%

%%%%%%%%%%%%%%%%%%%%%%%%%%%%%%%%%%%%%%%%%%%%%%%%%%%%%%%%%%%%%%%%%%%%%%%%%%%%%%%%
%%%%%%%%%%%%%%%%%%%%%%%%%%%%%%%%%%%%%%%%%%%%%%%%%%%%%%%%%%%%%%%%%%%%%%%%%%%%%%%%
\section{Information}

%%%%%%%%%%%%%%%%%%%%%%%%%%%%%%%%%%%%%%%%%%%%%%%%%%%%%%%%%%%%%%%%%%%%%%%%%%%%%%%%
\subsection{Copyright}

Copyright \copyright{} 2017--2018 Niklas Beisert

This work may be distributed and/or modified under the
conditions of the \LaTeX{} Project Public License, either version 1.3
of this license or (at your option) any later version.
The latest version of this license is in
  \url{http://www.latex-project.org/lppl.txt}
and version 1.3 or later is part of all distributions of \LaTeX{}
version 2005/12/01 or later.

This work has the LPPL maintenance status `maintained'.

The Current Maintainer of this work is Niklas Beisert.

This work consists of the files |README.txt|, |childdoc.ins| and |childdoc.dtx|
as well as the derived files |childdoc.def|, |cdocsamp.tex|
with |cdocsch1.tex|, |cdocsch2.tex|, |cdocspt3.tex|, |cdocspt4.tex|,
|cdocsdrf.tex|, |cdocsfn1.tex|, |cdocsfn2.tex|
as well as |childdoc.pdf|.

%%%%%%%%%%%%%%%%%%%%%%%%%%%%%%%%%%%%%%%%%%%%%%%%%%%%%%%%%%%%%%%%%%%%%%%%%%%%%%%%
\subsection{Files and Installation}

The package consists of the files:
%
\begin{center}
\begin{tabular}{ll}
    |README.txt|   & readme file \\
    |childdoc.ins| & installation file \\
    |childdoc.dtx| & source file \\
    |childdoc.def| & definition file \\
    |cdocsamp.tex| & sample main file \\
    |cdocsch1.tex| & sample include file \\
    |cdocsch2.tex| & sample include file \\
    |cdocspt3.tex| & sample part file \\
    |cdocspt4.tex| & sample part file \\
    |cdocsdrf.tex| & sample redirection file \\
    |cdocsfn1.tex| & sample redirection file \\
    |cdocsfn2.tex| & sample redirection file \\
    |childdoc.pdf| & manual
\end{tabular}
\end{center}
%
The distribution consists of the files
|README.txt|, |childdoc.ins| and |childdoc.dtx|.
%
\begin{itemize}
\item
Run (pdf)\LaTeX{} on |childdoc.dtx|
to compile the manual |childdoc.pdf| (this file).
\item
Run \LaTeX{} on |childdoc.ins| to create the definitions file |childdoc.def|
and the sample |cdocsamp.tex| with include files
|cdocsch1.tex|, |cdocsch2.tex|, |cdocspt3.tex|, |cdocspt4.tex|,
|cdocsdrf.tex|, |cdocsfn1.tex|, |cdocsfn2.tex|.
Then copy the file |childdoc.def| to an appropriate directory of your \LaTeX{}
distribution, e.g.\ \textit{texmf-root}|/tex/latex/childdoc|.
\end{itemize}

%%%%%%%%%%%%%%%%%%%%%%%%%%%%%%%%%%%%%%%%%%%%%%%%%%%%%%%%%%%%%%%%%%%%%%%%%%%%%%%%
\subsection{Related CTAN Packages}

There are several other packages which offer a similar functionality:
%
\begin{itemize}
\item
The packages
\href{http://ctan.org/pkg/docmute}{\textsf{docmute}},
\href{http://ctan.org/pkg/includex}{\textsf{includex}} and
\href{http://ctan.org/pkg/standalone}{\textsf{standalone}}
provide commands to include only the document body of
a child file thus allowing both files to be compiled individually.
\item
The packages \href{http://ctan.org/pkg/subdocs}{\textsf{subdocs}}
and \href{http://ctan.org/pkg/subfiles}{\textsf{subfiles}}
provide structures in which the main and child documents can be
encapsulated and allowing them to be compiled individually.
The inclusion mechanism is different from the conventional |\include|.
\item
The package \href{http://ctan.org/pkg/combine}{\textsf{combine}}
is an elaborate solution to combine several documents into one.
\end{itemize}
%
See also the CTAN topic \href{http://ctan.org/topic/subdocs}{\textsf{subdocs}}
for further related packages.
The present package differs from the above solutions in that
a document structure constructed with the conventional |\include| mechanism
just needs two extra commands at the top of every file
such that all constituent files can be compiled individually.

%%%%%%%%%%%%%%%%%%%%%%%%%%%%%%%%%%%%%%%%%%%%%%%%%%%%%%%%%%%%%%%%%%%%%%%%%%%%%%%%
%\subsection{Feature Suggestions}
%
%The following is a list of features which may be useful for future
%versions of this package:
%%
%\begin{itemize}
%\item
%\ldots
%\end{itemize}

%%%%%%%%%%%%%%%%%%%%%%%%%%%%%%%%%%%%%%%%%%%%%%%%%%%%%%%%%%%%%%%%%%%%%%%%%%%%%%%%
\subsection{Revision History}

%%%%%%%%%%%%%%%%%%%%%%%%%%%%%%%%%%%%%%%%
\paragraph{v2.0:} 2018/12/30

\begin{itemize}
\item
immediate forward processing
\item
added |\childdocby| mechanism
\item
manual restructured
\end{itemize}

%%%%%%%%%%%%%%%%%%%%%%%%%%%%%%%%%%%%%%%%
\paragraph{v1.6:} 2018/01/17

\begin{itemize}
\item
application for development of include files
\item
corrections to manual
\end{itemize}

%%%%%%%%%%%%%%%%%%%%%%%%%%%%%%%%%%%%%%%%
\paragraph{v1.5:} 2017/05/21

\begin{itemize}
\item
more complete structuring introduced
\item
|\childdocof| introduced
\item
|\childdoc| renamed to |\childdocmain|
\item
|\childredirect| renamed to |\childdocforward| and |\childdocforwardprefix|
and functionality expanded
\end{itemize}

%%%%%%%%%%%%%%%%%%%%%%%%%%%%%%%%%%%%%%%%
\paragraph{v1.0:} 2017/04/27

\begin{itemize}
\item
manual and install package
\item
first version published on CTAN
\end{itemize}

%%%%%%%%%%%%%%%%%%%%%%%%%%%%%%%%%%%%%%%%
\paragraph{v0.6:} 2017/04/26

\begin{itemize}
\item
redirection mechanism added
\end{itemize}

%%%%%%%%%%%%%%%%%%%%%%%%%%%%%%%%%%%%%%%%
\paragraph{v0.5:} 2017/04/26

\begin{itemize}
\item
functionality in definition file
\end{itemize}


%%%%%%%%%%%%%%%%%%%%%%%%%%%%%%%%%%%%%%%%%%%%%%%%%%%%%%%%%%%%%%%%%%%%%%%%%%%%%%%%
%%%%%%%%%%%%%%%%%%%%%%%%%%%%%%%%%%%%%%%%%%%%%%%%%%%%%%%%%%%%%%%%%%%%%%%%%%%%%%%%
%%%%%%%%%%%%%%%%%%%%%%%%%%%%%%%%%%%%%%%%%%%%%%%%%%%%%%%%%%%%%%%%%%%%%%%%%%%%%%%%
\appendix

\settowidth\MacroIndent{\rmfamily\scriptsize 000\ }

 \DocInput{childdoc.dtx}

\end{document}
%</driver>
% \fi
%
% %%%%%%%%%%%%%%%%%%%%%%%%%%%%%%%%%%%%%%%%%%%%%%%%%%%%%%%%%%%%%%%%%%%%%%%%%%%%%%
% %%%%%%%%%%%%%%%%%%%%%%%%%%%%%%%%%%%%%%%%%%%%%%%%%%%%%%%%%%%%%%%%%%%%%%%%%%%%%%
% \section{Sample}
%\iffalse
%<*samplemain>
%\fi
%
% The following presents a sample document
% with two chapters, two parts, a title page,
% a compile flag as well as three forwarding files to set the flag.
% It consists of eight |.tex| files:
% \begin{center}
% \begin{tabular}{ll}
% |cdocsamp.tex|&main file\\
% |cdocsch1.tex|&include file for chapter 1\\
% |cdocsch2.tex|&include file for chapter 2\\
% |cdocspt3.tex|&include file for part 3\\
% |cdocspt4.tex|&include file for part 4\\
% |cdocsdrf.tex|&forwarding file for main file in draft mode\\
% |cdocsfi1.tex|&forwarding file for final version of chapter 1\\
% |cdocsfi2.tex|&forwarding file for final version of chapter 2\\
% \end{tabular}
% \end{center}
% Each of the eight files can be compiled directly by the \LaTeX{} compiler.
%
% %%%%%%%%%%%%%%%%%%%%%%%%%%%%%%%%%%%%%%
% \paragraph{Main File.}
%
% The main file is called |cdocsamp.tex|.
%
% Load the \textsf{childdoc} definitions and
% declare the filename for the main document:
%    \begin{macrocode}
\input{childdoc.def}
\childdocmain{}
%    \end{macrocode}

% Optional override for |\version| flag:
%    \begin{macrocode}
%%\ifchilddoc\else\providecommand{\version}{draft}\fi
%    \end{macrocode}

% Define the default values for the |\version| flag
% (|final| for the main file and |draft| for childs):
%    \begin{macrocode}
\ifchilddoc
\providecommand{\version}{draft}
\else
\providecommand{\version}{final}
\fi
%    \end{macrocode}

% Load the standard document class:
%    \begin{macrocode}
\documentclass[12pt]{article}
%    \end{macrocode}

% Start the document body:
%    \begin{macrocode}
\begin{document}
%    \end{macrocode}

% Declare a title page.
% Print title, part of document being processed and version flag:
%    \begin{macrocode}
\addtocounter{page}{-1}
\begin{center}
{\LARGE\bfseries{}childdoc example\par}
\vspace{1cm}
\ifchilddoc
\ifchilddocmanual part\else chapter\fi:
`\childdocname' of `\childdocjob'\par
\else
main document: `\childdocjob'\par
\fi
version: \version\par
\end{center}
\newpage
%    \end{macrocode}

% Manually include selected file,
% otherwise process as usual:
%    \begin{macrocode}
\ifchilddocmanual
\section*{part `\childdocname'}
\input{\childdocname}
\else
%    \end{macrocode}

% Include the two chapters:
%    \begin{macrocode}
\include{cdocsch1}
\include{cdocsch2}
%    \end{macrocode}

% Include the two parts unless only chapters should be displayed:
%    \begin{macrocode}
\ifchilddoc\else
\section{part three}
\input{cdocspt3}
\section{part four}
\input{cdocspt4}
\fi
%    \end{macrocode}

% Process as usual until here:
%    \begin{macrocode}
\fi
%    \end{macrocode}

% End of document body:
%    \begin{macrocode}
\end{document}
%    \end{macrocode}
%\iffalse
%</samplemain>
%\fi
%
% %%%%%%%%%%%%%%%%%%%%%%%%%%%%%%%%%%%%%%
% \paragraph{Chapter Include Files.}
%
% The include files are called |cdocsch1.tex| and |cdocsch2.tex|.
%
%\iffalse
%<*samplechap1|samplechap2>
%\fi

% Optional override for |\version| flag:
%    \begin{macrocode}
%%\providecommand{\version}{final}
%    \end{macrocode}

% Include the main document:
%    \begin{macrocode}
\input{childdoc.def}
\childdocof{cdocsamp}
%    \end{macrocode}

%\iffalse
%</samplechap1|samplechap2>
%\fi
%
%\iffalse
%<*samplechap1>
%\fi
% Some text for chapter 1:
%    \begin{macrocode}
\section{one}
some text in chapter one
%    \end{macrocode}

%\iffalse
%</samplechap1>
%\fi
% Some text for chapter 2:
%\iffalse
%<*samplechap2>
%\fi
%    \begin{macrocode}
\section{two}
more text in chapter two
%    \end{macrocode}

%\iffalse
%</samplechap2>
%\fi
%
% %%%%%%%%%%%%%%%%%%%%%%%%%%%%%%%%%%%%%%
% \paragraph{Part Include Files.}
%
% The include files are called |cdocspt3.tex| and |cdocspt4.tex|.
%
%\iffalse
%<*samplepart3|samplepart4>
%\fi

% Optional override for |\version| flag:
%    \begin{macrocode}
%%\providecommand{\version}{final}
%    \end{macrocode}

% Include the main document:
%    \begin{macrocode}
\input{childdoc.def}
\childdocby{cdocsamp}
%    \end{macrocode}

%\iffalse
%</samplepart3|samplepart4>
%\fi
%
%\iffalse
%<*samplepart3>
%\fi
% Some text for part 3:
%    \begin{macrocode}
some text in part three
%    \end{macrocode}

%\iffalse
%</samplepart3>
%\fi
% Some text for part 4:
%\iffalse
%<*samplepart4>
%\fi
%    \begin{macrocode}
more text in part four
%    \end{macrocode}

%\iffalse
%</samplepart4>
%\fi
%
% %%%%%%%%%%%%%%%%%%%%%%%%%%%%%%%%%%%%%%
% \paragraph{Forwarding for a Complete Draft.}
%
% The following forwarding file |cdocsdrf.tex|
% compiles the main document in draft mode:
%\iffalse
%<*sampledraft>
%\fi
%    \begin{macrocode}
\def\version{draft}
\input{childdoc.def}
\childdocforward{cdocsamp}
%    \end{macrocode}

%\iffalse
%</sampledraft>
%\fi
%
% %%%%%%%%%%%%%%%%%%%%%%%%%%%%%%%%%%%%%%
% \paragraph{Forwarding for Final Version of the Chapters.}
%
% The following forwarding files |cdocsfn1.tex| and |cdocsfn2.tex|
% (with identical content)
% compile the final versions of the child documents
% |cdocsch1.tex| and |cdocsch2.tex|, respectively:
%\iffalse
%<*samplefinal>
%\fi
%    \begin{macrocode}
\def\version{final}
\input{childdoc.def}
\childdocforwardprefix[cdocsamp]{cdocsfn}{cdocsch}
%    \end{macrocode}

%\iffalse
%</samplefinal>
%\fi
%
% %%%%%%%%%%%%%%%%%%%%%%%%%%%%%%%%%%%%%%
% \paragraph{Command Line Processing.}
%
% The following three command lines generate the output files
% |cdocscld|, |cdocscl1| and |cdocscl2|
% which should be identical to
% |cdocsdrf|, |cdocsch1| and |cdocsfn2|, respectively:
% \begin{center}
% \begin{tabular}{l}
% |latex -jobname cdocscld \|\\
% |  "\def\version{draft}\input{childdoc.def}\childdocforward{cdocsamp}"|\\
% |latex -jobname cdocscl1 \|\\
% |  "\input{childdoc.def}\childdocforward[cdocsamp]{cdocsch1}"|\\
% |latex -jobname cdocscl2 \|\\
% |  "\def\version{final}\input{childdoc.def}\childdocforward{cdocsch2}"|
% \end{tabular}
% \end{center}
% Note that the trailing backslash on each first line
% merely continues the input to the second line
% (for convenient cut ant paste).
% Furthermore, the command |latex| can be replaced by any
% of its alternative versions such as |pdflatex|.
%
% %%%%%%%%%%%%%%%%%%%%%%%%%%%%%%%%%%%%%%%%%%%%%%%%%%%%%%%%%%%%%%%%%%%%%%%%%%%%%%
% %%%%%%%%%%%%%%%%%%%%%%%%%%%%%%%%%%%%%%%%%%%%%%%%%%%%%%%%%%%%%%%%%%%%%%%%%%%%%%
% \section{Implementation}
%\iffalse
%<*package>
%\fi
%
% This section describes the definitions file |childdoc.def|.

% The definitions cannot be loaded using |\usepackage| or |\RequirePackage|
% which has a mechanism to prevent loading a style file more than once.
% When loading the definitions by means of |\input|
% multiple instances have to be prevented manually:
%\iffalse
%This code needs to be before the `\ProvidesFile' directive
%which is defined at the beginning of this file.
%Therefore it is also placed there and commented out here.
%</package>
%<*discard>
%\fi
%    \begin{macrocode}
\ifdefined\childdocmain\endinput\fi
%    \end{macrocode}
%\iffalse
%</discard>
%<*package>
%\fi
%
% \macro{\ifchilddoc}
% \macro{\ifchilddocmanual}
% The conditional |\ifchilddoc| tells whether a
% child (true) or main (false) document is being compiled.
% The conditional |\ifchilddocmanual| tells whether
% the |\includeonly| mechanism is used (false) or
% the selection of child files must be performed manually (true).
% The definitions initialise to false:
%    \begin{macrocode}
\newif\ifchilddoc
\newif\ifchilddocmanual
%    \end{macrocode}

% \macro{\childdocname}
% \macro{\childdocjob}
% The macro |\childdocname| stores the name of the main document
% to be compiled. The macro |\childdocjob| stores the name of
% the document on which the \LaTeX{} compiler was originally invoked.
% The content of |\jobname| cannot be compared
% to filenames specified in the source due to different catcodes.
% The following code rescans |\jobname|, stores the result
% in |\childdocname| and saves a copy in |\childdocjob|:
%    \begin{macrocode}
\edef\childdocname{\scantokens\expandafter{\jobname\noexpand}}
\let\childdocjob\childdocname
%    \end{macrocode}

% \macro{\childdocdisable}
% The macro |\childdocdisable| prevents the main file
% from being processed more than once.
% At this stage, the main document command |\childdocmain|
% is assumed to be called once again where it should do nothing.
% Any subsequent call to it should prevent
% a secondary processing of the main document
% It overwrites the forwarding commands
% |\childdocof| and |\childdocforward|
% with empty macros to prevent further inclusions of the main document:
%    \begin{macrocode}
\newcommand{\childdocdisable}
{
  \renewcommand{\childdocmain}[1]{\renewcommand{\childdocmain}[1]{\endinput}}
  \renewcommand{\childdocof}[1]{}
  \renewcommand{\childdocby}[2][]{}
  \renewcommand{\childdocforward}[2][]{}
  \renewcommand{\childdocdisable}{}
}
%    \end{macrocode}

% \macro{\childdocmain}
% The macro |\childdocmain| is to be called at the top of the main file
% with nothing or the main filename (without extension) as argument.
% First, it breaks loops.
% If the argument is not empty and does not match |\childdocname|
% (which is set by the first inclusion of |childdoc.def|),
% |\ifchilddoc| is set to true, |\includeonly| is applied to the child file
% and |\jobname| is set to the main file
% (for proper handling of |.aux| files):
%    \begin{macrocode}
\newcommand{\childdocmain}[1]
{
  \childdocdisable\childdocmain{}
  \if?#1?\else
    \begingroup
      \def\childdoctmp{#1}
      \ifx\childdoctmp\childdocname
        \def\childdoctmp{}
      \else
        \def\childdoctmp
        {
          \childdoctrue
          \includeonly{\childdocname}
          \def\childdocjob{#1}
          \def\jobname{#1}
        }
      \fi
      \expandafter
    \endgroup
    \childdoctmp
  \fi
}
%    \end{macrocode}

% \macro{\childdocof}
% The command |\childdocof| redirects
% compilation to the main file |#1|.
%    \begin{macrocode}
\newcommand{\childdocof}[1]
{
  \childdocdisable
  \childdoctrue
  \includeonly{\childdocname}
  \def\jobname{#1}
  \def\childdocjob{#1}
  \input{#1}
}
%    \end{macrocode}

% \macro{\childdocby}
% The command |\childdocby| ....
%    \begin{macrocode}
\newcommand{\childdocby}[2][]
{
  \childdocdisable
  \childdoctrue
  \childdocmanualtrue
  \if?#1?\else
    \def\jobname{#2}
  \fi
  \def\childdocjob{#2}
  \input{#2}
  \endinput
}
%    \end{macrocode}

% \macro{\childdocforward}
% The command |\childdocforward| redirects
% compilation to the main file or
% (if the optional argument is given) a child file.
% Parameters are set as if the main file
% or a child file starting with |\childdocof| was compiled.
% Then compilation is handed over to the main file:
%    \begin{macrocode}
\newcommand{\childdocforward}[2][]
{
  \begingroup
    \if?#1?
      \def\childdoctmp
      {
        \def\childdocname{#2}
        \def\childdocjob{#2}
        \def\jobname{#2}
        \input{#2}
        \endinput
      }
    \else
      \def\childdoctmp
      {
        \childdocdisable
        \def\childdocname{#2}
        \childdoctrue
        \includeonly{#2}
        \def\childdocjob{#1}
        \def\jobname{#1}
        \input{#1}
        \endinput
      }
    \fi
    \expandafter
  \endgroup
  \childdoctmp
}
%    \end{macrocode}

% \macro{\childdocforwardprefix}
% The command |\childdocforwardprefix| redirects
% compilation to the main or a child file by means of a pattern.
% The prefix |#1| in the current filename is replaced by |#2|
% and the suffix of the current filename is kept
% (it is assumed that the filename does not contain the substring `|~~~|'
% which is used as a delimiter).
% Compilation is handed over to the new file by |\childdocforward|:
%    \begin{macrocode}
\newcommand{\childdocforwardprefix}[3][]
{
  \begingroup
    \def\childdocextract #2##1~~~{\def\childdoctmp{\childdocforward[#1]{#3##1}}}
    \expandafter\childdocextract\childdocname~~~
    \expandafter
  \endgroup
  \childdoctmp
}
%    \end{macrocode}

% \macro{\childdoc}
% The deprecated macro |\childdoc| is a legacy version of |\childdocmain|:
%    \begin{macrocode}
\newcommand{\childdoc}{\childdocmain}
%    \end{macrocode}

% \macro{\childdocredirect}
% The deprecated macro |\childdocredirect| is a legacy version
% of |\childdocforward| and |\childdocforwardprefix|:
%    \begin{macrocode}
\newcommand{\childdocredirect}[2][]
{
  \begingroup
    \if?#1?
      \def\childdoctmp{\childdocforward{#2}}
    \else
      \def\childdoctmp{\childdocforwardprefix{#1}{#2}}
    \fi
    \expandafter
  \endgroup
  \childdoctmp
}
%    \end{macrocode}

%\iffalse
%</package>
%\fi
%
\endinput
|\\
|\childdocforwardprefix[|\textit{main}|]{|\textit{prefix}|}{|\textit{dest}|}|
\end{tabular}
\end{center}
%
the destination file is determined by a pattern
depending on the current file:
To make this work, the current file must be called
`{\textit{prefix}\hspace{0.2em}\textit{suffix}}'
with \textit{prefix} matching precisely the argument.
Processing is then passed on to the file
`{\textit{dest}\hspace{0.2em}\textit{suffix}}'.
Surely, the same effect is achieved by
directly specifying the
argument `{\textit{dest}\hspace{0.2em}\textit{suffix}}'
in the first form.
However, that requires to set up a different file
for each child. With the alternative form of the command
all these files can have exactly the same content
which simplifies setting them up and maintaining them.

For example, the following file |draft.tex|
with a compilation flag |\version| as described in \secref{sec:flags}
compiles the main document as a draft:
%
\begin{center}
\begin{tabular}{l}
|\def\version{draft}|\\
|% \iffalse
%
% childdoc.dtx Copyright (C) 2017-2018 Niklas Beisert
%
% This work may be distributed and/or modified under the
% conditions of the LaTeX Project Public License, either version 1.3
% of this license or (at your option) any later version.
% The latest version of this license is in
%   http://www.latex-project.org/lppl.txt
% and version 1.3 or later is part of all distributions of LaTeX
% version 2005/12/01 or later.
%
% This work has the LPPL maintenance status `maintained'.
%
% The Current Maintainer of this work is Niklas Beisert.
%
% This work consists of the files childdoc.dtx and childdoc.ins
% and the derived files childdoc.def and cdocsamp.tex with
% cdocsch1.tex, cdocsch2.tex, cdocsdrf.tex, cdocsfn1.tex, cdocsfn2.tex.
%
%<package>\ifdefined\childdocmain\endinput\fi
%<package>\ProvidesFile{childdoc.def}[2018/12/30 v2.0 child document driver]
%<samplemain>\ProvidesFile{cdocsamp.tex}[2018/12/30 v2.0 sample for childdoc]
%<*driver>
%\ProvidesFile{childdoc.drv}[2018/12/30 v2.0 childdoc reference manual file]
\PassOptionsToClass{10pt,a4paper}{article}
\documentclass{ltxdoc}

\usepackage[margin=35mm]{geometry}
\usepackage{hyperref}
\usepackage{hyperxmp}
\usepackage[usenames]{color}

\hypersetup{colorlinks=true}
\hypersetup{pdfstartview=FitH}
\hypersetup{pdfpagemode=UseNone}
\hypersetup{pdfsource={}}
\hypersetup{pdflang={en-UK}}
\hypersetup{pdfcopyright={Copyright 2017-2018 Niklas Beisert.
  This work may be distributed and/or modified under the
  conditions of the LaTeX Project Public License, either version 1.3
  of this license or (at your option) any later version.}}
\hypersetup{pdflicenseurl={http://www.latex-project.org/lppl.txt}}
\hypersetup{pdfcontactaddress={ETH Zurich, ITP, HIT K,
  Wolfgang-Pauli-Strasse 27}}
\hypersetup{pdfcontactpostcode={8093}}
\hypersetup{pdfcontactcity={Zurich}}
\hypersetup{pdfcontactcountry={Switzerland}}
\hypersetup{pdfcontactemail={nbeisert@itp.phys.ethz.ch}}
\hypersetup{pdfcontacturl={http://people.phys.ethz.ch/\xmptilde nbeisert/}}

\newcommand{\secref}[1]{\hyperref[#1]{section \ref*{#1}}}

\parskip1ex
\parindent0pt
\let\olditemize\itemize
\def\itemize{\olditemize\parskip0pt}

\begin{document}

\title{The \textsf{childdoc} Package}
\hypersetup{pdftitle={The childdoc Package}}
\author{Niklas Beisert\\[2ex]
  Institut f\"ur Theoretische Physik\\
  Eidgen\"ossische Technische Hochschule Z\"urich\\
  Wolfgang-Pauli-Strasse 27, 8093 Z\"urich, Switzerland\\[1ex]
  \href{mailto:nbeisert@itp.phys.ethz.ch}
  {\texttt{nbeisert@itp.phys.ethz.ch}}}
\hypersetup{pdfauthor={Niklas Beisert}}
\hypersetup{pdfsubject={Manual for the LaTeX2e Package childdoc}}
\date{30 December 2018, \textsf{v2.0}}
\maketitle

\begin{abstract}\noindent
\textsf{childdoc} is a \LaTeXe{} package
that enables the direct compilation
of document sections included by |\include|
to individual files.
\end{abstract}

\begingroup
\parskip0ex
\tableofcontents
\endgroup

%%%%%%%%%%%%%%%%%%%%%%%%%%%%%%%%%%%%%%%%%%%%%%%%%%%%%%%%%%%%%%%%%%%%%%%%%%%%%%%%
%%%%%%%%%%%%%%%%%%%%%%%%%%%%%%%%%%%%%%%%%%%%%%%%%%%%%%%%%%%%%%%%%%%%%%%%%%%%%%%%
\section{Introduction}

\LaTeX{} provides a mechanism to structure a large document (such as a book)
into a main file and several child files (containing the chapters)
using the |\include| command.
This mechanism is beneficial for documents
which span hundreds of pages in order to
make the source file(s) more manageable.
Moreover, compilation can be restricted to
selected child files by means of the |\includeonly| command.
The latter feature can be used to reduce the compilation time while editing
(this was significantly more useful in the earlier days of \LaTeX{})
or to generate a smaller document which is easier to navigate.
Another application of |\includeonly| is to generate
documents consisting of selected parts of the complete document.

However, there are a few drawbacks of the plain |\include| mechanism:
\begin{itemize}
\item
The child files cannot be compiled on their own,
they can only be compiled via the main file.
A naive editing environment
(such as a text editor with an option
to have the current file processed by \LaTeX)
may require one to switch to the main file before compiling;
attempting to compile the child file produces errors.
\item
The main file must be modified (each time)
to adjust the |\includeonly| command
to the present needs. This easily leaves the main file in a messy state.
\item
The generated document will always carry the filename
of the main document. This is inconvenient if
several child files are to be compiled and
to be kept for distribution.
\end{itemize}

The present package provides a simple interface
to make child files individually compilable by \LaTeX{}.
Compiling a child file then has the same effect as compiling
the main file with an |\includeonly| command
to select the appropriate child.
Moreover the generated document will carry the name of the child
rather than the main file.
This resolves all three above issues.

This feature is meant to make the editing of books,
thesis documents and lecture notes somewhat more convenient.
However, the package can also be used efficiently for
composing a series of documents (such as exercise sheets)
which are typically distributed individually.
It then assists the author in generating the individual documents
(potentially in different versions)
as well as a document containing the collected series.
Another application is in developing style files
or other kinds of included material
where compilation of the style file could redirect
to a sample or test file.

%%%%%%%%%%%%%%%%%%%%%%%%%%%%%%%%%%%%%%%%%%%%%%%%%%%%%%%%%%%%%%%%%%%%%%%%%%%%%%%%
%%%%%%%%%%%%%%%%%%%%%%%%%%%%%%%%%%%%%%%%%%%%%%%%%%%%%%%%%%%%%%%%%%%%%%%%%%%%%%%%
\section{Usage}

First of all, the package \textsf{childdoc} is \emph{not} a standard
\LaTeXe{} |.sty| style file! Therefore it needs to be invoked in
a non-standard way.

%%%%%%%%%%%%%%%%%%%%%%%%%%%%%%%%%%%%%%%%%%%%%%%%%%%%%%%%%%%%%%%%%%%%%%%%%%%%%%%%
\subsection{Included Files}
\label{sec:include}

%%%%%%%%%%%%%%%%%%%%%%%%%%%%%%%%%%%%%%%%
\DescribeMacro{\childdocmain}
To use the package, add the commands
\begin{center}
\begin{tabular}{l}
|\input{childdoc.def}|\\
|\childdocmain{}|\\
\end{tabular}
\end{center}
at the very top of the main \LaTeX{} file,
in particular \emph{before} the |\documentclass| statement!
The argument of |\childdocmain| should be left empty
(but it must be present).

%%%%%%%%%%%%%%%%%%%%%%%%%%%%%%%%%%%%%%%%
\DescribeMacro{\childdocof}
Furthermore, add the commands
\begin{center}
\begin{tabular}{l}
|\input{childdoc.def}|\\
|\childdocof{|\textit{main}|}|\\
\end{tabular}
\end{center}
at the top of every child file \textit{child}
which is included by |\include{|\textit{child}|}|
from within the main file
(or at least for those files to be compiled individually).
The argument \textit{main} must be the filename of the main file.

There are a couple of
considerations in setting up the main and child documents:

%%%%%%%%%%%%%%%%%%%%%%%%%%%%%%%%%%%%%%%%
\paragraph{Restrictions.}

Please note the following restrictions:
\begin{itemize}
\item
|\childdocmain| must be called with one argument \textit{main}
to ensure compatibility with earlier version of the package.
It must either be empty (|\childdocmain{}|)
or precisely match the filename of the main file in which it is specified.
See \secref{sec:detection} for further information.
\item
The filename \textit{main} must be specified without the |.tex| extension.
\item
The filename \textit{main} is case sensitive
(even in case-insensitive file systems)
due to internal string comparison.
\item
The argument \textit{main} should be fully expanded, it cannot be a macro.
\item
Subdirectories and special characters should be avoided in filenames.
\item
The command |\childdocmain{|\textit{main}|}| must be followed by a whitespace.
It should not be followed immediately by another command
or by a comment mark `|%|'.
This is because the \TeX{} parser reads the token immediately following
the argument of |\childdocmain| and puts it
at the beginning of every child section;
however, a white\-space is ignored.
\end{itemize}

%%%%%%%%%%%%%%%%%%%%%%%%%%%%%%%%%%%%%%%%
\paragraph{Content of Main File.}

It is advisable to place all content in the child files included by |\include|.
Any output contained in the main file will appear in all child documents
unless suppressed manually;
it cannot be suppressed automatically by the |\includeonly| directive
and thus should normally be avoided.
A method to include some content in the main file
by means of conditional processing is described in \secref{sec:conditional}.

%%%%%%%%%%%%%%%%%%%%%%%%%%%%%%%%%%%%%%%%
\paragraph{Page Numbering.}

When only a part of the document is compiled,
the appropriate numbering of pages
(as well as other status parameters)
is determined from the |.aux| files.
The latter contain information from previous passes.
However this information needs to propagate through
all intermediate child documents.
Therefore the page numbering in child documents may well
be inconsistent until the complete document is compiled at least once.

A useful (if unconventional) way to always ensure a consistent
page numbering is to restart the numbering in each child document
and denote the pages by `\textit{child}|.|\textit{page}'
where \textit{child} represents the chapter/section number of the child file.
This can be achieved by the command
|\numberwithin{page}{|\textit{child}|}|
of the \textsf{amsmath} package
where \textit{child} can be |chapter| or |section|
depending on the chosen structuring.
Alternatively, one can modify the macro |\thepage| appropriately
and reset the counter |page| at the start of each child file.

%%%%%%%%%%%%%%%%%%%%%%%%%%%%%%%%%%%%%%%%%%%%%%%%%%%%%%%%%%%%%%%%%%%%%%%%%%%%%%%%
\subsection{Conditional Processing}
\label{sec:conditional}

The package provides a mechanism to compile different versions
of a document. To customise the versions further some conditional processing
can come in handy to distinguish which version is being compiled.
The package provides two macros to describe the compilation context:

%%%%%%%%%%%%%%%%%%%%%%%%%%%%%%%%%%%%%%%%
\DescribeMacro{\ifchilddoc}
The conditional |\ifchilddoc| distinguishes between the compilation of
child documents and the main document:
%
\begin{center}
|\ifchilddoc |\textit{child-code}| |[|\||else |\textit{main-code}]| \||fi|
\end{center}

%%%%%%%%%%%%%%%%%%%%%%%%%%%%%%%%%%%%%%%%
\DescribeMacro{\childdocname}
\DescribeMacro{\childdocjob}
The macro |\childdocname| contains the filename (without extension)
of the main or child file being processed.
Note that |\childdocjob| will always contain the name of the main file.

%%%%%%%%%%%%%%%%%%%%%%%%%%%%%%%%%%%%%%%%
\paragraph{Title Page.}

Conditional processing can be used to include a title or banner page
in the main document when proper precautions are taken.
Importantly, the code in the main file should ensure that the page counter
(as well as other status parameters which are stored in the |.aux| files)
takes the same value after the conditional processing.
Otherwise the page numbers may take divergent values
depending on which part is compiled.

For example, a title page could be declared by:
%
\begin{center}
\begin{tabular}{l}
|\ifchilddoc\||else|\\
|\addtocounter{page}{-1}|\\
\textit{code for title page}\\
|\newpage|\\
|\||fi|
\end{tabular}
\end{center}
%
A banner page for the child documents can be generated by:
%
\begin{center}
\begin{tabular}{l}
|\ifchilddoc|\\
|\addtocounter{page}{-1}|\\
\textit{code for banner page}\\
|\newpage|\\
|\||fi|
\end{tabular}
\end{center}
%
Here one could write a message such as:
\begin{center}
|This is the part \childdocname{} of \childdocjob{}.|
\end{center}

%%%%%%%%%%%%%%%%%%%%%%%%%%%%%%%%%%%%%%%%%%%%%%%%%%%%%%%%%%%%%%%%%%%%%%%%%%%%%%%%
\subsection{Flags}
\label{sec:flags}

The package makes it easy to generate different versions
of the main or child documents.
To this end compilation flags can be defined
and assigned different default values.
They will be particularly useful in conjunction
with the forwarding mechanism described in \secref{sec:forward}.

For example, it may be useful to have a flag |\version|
which can be set to |draft| or |final|.
The document source will contain some conditional code
depending on the value of |\version|.
Suppose further, the flag should default to |final| for the main file
and to |draft| for child files
which is a natural assignment for editing the document.
This is achieved by placing the following code
in the preamble of the main document
(below the |\childdocmain| directive):
%
\begin{center}
\begin{tabular}{l}
|\ifchilddoc|\\
|\providecommand{\version}{draft}|\\
|\||else|\\
|\providecommand{\version}{final}|\\
|\||fi|
\end{tabular}
\end{center}
%
The definition by |\providecommand| makes sure
that previous definitions are not overwritten.
Further statements |\providecommand{\version}{...}|
can thus be added before the above code to override it.

For the main file, one might add a line
(between |\childdocmain| and the above block)
%
\begin{center}
|%\ifchilddoc\||else\providecommand{\version}{draft}\||fi|
\end{center}
%
which can be uncommented to produce a draft version.
Likewise one can add a line to the very top of a child file
(above the |\childdocof{|\textit{main}|}| directive)
%
\begin{center}
|%\providecommand{\version}{final}|
\end{center}
%
which can be uncommented to produce the final version of this child document.

%%%%%%%%%%%%%%%%%%%%%%%%%%%%%%%%%%%%%%%%%%%%%%%%%%%%%%%%%%%%%%%%%%%%%%%%%%%%%%%%
\subsection{Forwarding}
\label{sec:forward}

Different versions of the main or child documents
using compilation flags as described in \secref{sec:flags}
can be (permanently) stored in different files
for convenient compilation, viewing and distribution.
To this end, the package defines a command
to pass on compilation to a different file:

%%%%%%%%%%%%%%%%%%%%%%%%%%%%%%%%%%%%%%%%
\DescribeMacro{\childdocforward}
The command |\childdocforward| redirects processing to
another source file:
%
\begin{center}
\begin{tabular}{l}
|\input{childdoc.def}|\\
|\childdocforward[|\textit{main}|]{|\textit{dest}|}|\\
\end{tabular}
\end{center}
%
The argument \textit{dest} is the destination file
(without extension).
It should be the main file or one of the child files.
Note that further \textsf{childdoc} directives
such as |\childdocof| and |\childdocforward|
in the indicated file will be processed in this form.
The optional argument \textit{main}
passes on directly to the main file \textit{main}
while pretending to compile the child \textit{dest}.
This form behaves as if \textit{dest}
issues |\childdocof{|\textit{main}|}| right away,
and no further \textsf{childdoc} directives will be processed.

%%%%%%%%%%%%%%%%%%%%%%%%%%%%%%%%%%%%%%%%
\DescribeMacro{\...prefix}
In the alternative form |\childdocforwardprefix|,
%
\begin{center}
\begin{tabular}{l}
|\input{childdoc.def}|\\
|\childdocforwardprefix[|\textit{main}|]{|\textit{prefix}|}{|\textit{dest}|}|
\end{tabular}
\end{center}
%
the destination file is determined by a pattern
depending on the current file:
To make this work, the current file must be called
`{\textit{prefix}\hspace{0.2em}\textit{suffix}}'
with \textit{prefix} matching precisely the argument.
Processing is then passed on to the file
`{\textit{dest}\hspace{0.2em}\textit{suffix}}'.
Surely, the same effect is achieved by
directly specifying the
argument `{\textit{dest}\hspace{0.2em}\textit{suffix}}'
in the first form.
However, that requires to set up a different file
for each child. With the alternative form of the command
all these files can have exactly the same content
which simplifies setting them up and maintaining them.

For example, the following file |draft.tex|
with a compilation flag |\version| as described in \secref{sec:flags}
compiles the main document as a draft:
%
\begin{center}
\begin{tabular}{l}
|\def\version{draft}|\\
|\input{childdoc.def}|\\
|\childdocforward{|\textit{main}|}|
\end{tabular}
\end{center}
%
Likewise, the following files |final|\textit{nn}|.tex|
compile the final version of the child document
|child|\textit{nn}|.tex|:
%
\begin{center}
\begin{tabular}{l}
|\def\version{final}|\\
|\input{childdoc.def}|\\
|\childdocforwardprefix{final}{child}|
\end{tabular}
\end{center}
%

Note that when several versions of a main file and/or of each child file
are to be generated, it may be convenient to set up a |Makefile| or
shell script to automatise the process.

%%%%%%%%%%%%%%%%%%%%%%%%%%%%%%%%%%%%%%%%%%%%%%%%%%%%%%%%%%%%%%%%%%%%%%%%%%%%%%%%
\subsection{Command Line Processing}
\label{sec:commandline}

The effect of redirection files can also be achieved by invoking
the \LaTeX{} compiler with a more elaborate command line.
Most conveniently this should be done as part
of a shell script or a |Makefile|.

When using \textsf{childdoc} in the main file, the following
command lines effectively perform a redirection
(note that depending on the shell being used,
backslashes may have to be doubled: `|\|' $\to$ `|\\|'):
%
\begin{center}
|... -jobname "|\textit{target}|" |\\|"|[\textit{flags}]%
|\input{childdoc.def}\childdocforward[|\textit{main}|]{|\textit{dest}|}"|
\end{center}
%
Here \textit{target} is the name of the output file,
\textit{main} is the name of the main file
and \textit{dest} is the name of the main or child file to be processed
(all filenames without extensions).
The optional argument \textit{main} can be omitted
if \textit{main} matches \textit{dest}.
Optionally, compilation \textit{flags} can be defined via |\def| commands.
This command line makes the \TeX{} engine believe
it is compiling the file \textit{target}
whose content is specified as the latter parameter.
The provided code then forwards the processing to
\textit{main} or \textit{dest} as described in \secref{sec:forward}.

%%%%%%%%%%%%%%%%%%%%%%%%%%%%%%%%%%%%%%%%%%%%%%%%%%%%%%%%%%%%%%%%%%%%%%%%%%%%%%%%
\subsection{Include by Input}
\label{sec:input}

Including child documents by |\include| has some restrictions by design.
Most notably, the content of a child document always occupies
its own set of pages; pages cannot be shared between child documents.
Usually, this behaviour makes perfect sense
because each child document contain an essential part of the document.
However, in some situations it may be desirable to compose
a document from a collection of parts
without having mandatory page breaks between then.
For this case, the package
provides a mechanism to include parts
by |\input| which can also be processed individually.
However, by construction this mechanism
requires manual handling of the content to be output.

%%%%%%%%%%%%%%%%%%%%%%%%%%%%%%%%%%%%%%%%
\DescribeMacro{\ifchilddocmanual}
The main file should be prepared as usual, see \secref{sec:include}.
However, the document body must make a distinction
between processing of an individual part and of the main document, e.g.:
%
\begin{center}
\begin{tabular}{l}
|\ifchilddocmanual|\\
|\input{\childdocname}|\\
|\||else|\\
\textit{document body with }|\input{|\textit{part}|}|\\
|\||fi|
\end{tabular}
\end{center}
%
The conditional |\ifchilddocmanual| is true whenever
a part to be included by |\input| is being compiled,
and the name of the part is stored in |\childdocname|.

%%%%%%%%%%%%%%%%%%%%%%%%%%%%%%%%%%%%%%%%
\DescribeMacro{\childdocby}
Each part to be included by |\input| should start with:
%
\begin{center}
\begin{tabular}{l}
|\input{childdoc.def}|\\
|\childdocby{|\textit{main}|}|\\
\end{tabular}
\end{center}
%
The directive |\childdocby| is similar to |\childdocof|
described in \secref{sec:include},
but the subsequent selection of content must be done manually.
To that end, both |\ifchilddoc| and |\ifchilddocmanual|
will be true upon processing of a part,
and the name of the part is stored in |\childdocname|.
Note that |\jobname| will be set to the filename of the current part
so that each part receives an individual |.aux| file
that does not interfere with the |.aux| file(s) of the main document.
This behaviour can be altered by the alternative form
|\childdocby[*]{|\textit{main}|}| (with a non-empty optional argument)
which uses the |.aux| file of the main document
by setting |\jobname| to \textit{main}.

%%%%%%%%%%%%%%%%%%%%%%%%%%%%%%%%%%%%%%%%%%%%%%%%%%%%%%%%%%%%%%%%%%%%%%%%%%%%%%%%
\subsection{Driver Development}
\label{sec:driver}

The \textsf{childdoc} mechanism can also be use for the development
of definition files such as \LaTeX{} styles or classes.
This case differs from the above setup with multiple parts
included by |\include| in that no |\includeonly| should be invoked.
This can be achieved by starting the include file
(before |\ProvidesPackage|) with:
%
\begin{center}
\begin{tabular}{l}
|\input{childdoc.def}|\\
|\childdocforward{|\textit{main}|}|\\
\end{tabular}
\end{center}
%
or alternatively with:
%
\begin{center}
\begin{tabular}{l}
|\input{childdoc.def}|\\
|\childdocby{|\textit{main}|}|\\
\end{tabular}
\end{center}
%
Both forms have slightly different effects as described above.
The main file is prepared as usual, see \secref{sec:include}.

%%%%%%%%%%%%%%%%%%%%%%%%%%%%%%%%%%%%%%%%%%%%%%%%%%%%%%%%%%%%%%%%%%%%%%%%%%%%%%%%
\subsection{Legacy Detection}
\label{sec:detection}

The directive |\childdocmain| in the main file can detect
whether the complete document or merely a child is to be compiled
even without using the directive |\childdocof|.
This method is deprecated because it is less robust
and there is no compelling reason to use it;
it is merely provided for backward compatibility
and it may be removed in future versions.

If the detection mechanism is to be used,
it is mandatory to correctly specify
the filename of the main file as the argument of |\childdocmain|:
%
\begin{center}
\begin{tabular}{l}
|\input{childdoc.def}|\\
|\childdocmain{|\textit{main}|}|\\
\end{tabular}
\end{center}
%
If |\jobname| does not match the argument \textit{main} of |\childdocmain|,
it is assumed that |\jobname| points to the child file to be compiled.
When using |\childdocmain| with the main file specified as argument,
it suffices to start a child file
with just |\input{|\textit{main}|}|
without loading of the package and using |\childdocof|.
If instead all processing is done
with the appropriate \textsf{childdoc} directives,
the argument of \textit{main} of |\childdocmain| can be empty.

An alternative version of the command line processing described
in \secref{sec:commandline} using the detection mechanism reads:
%
\begin{center}
|... -jobname "|\textit{target}|" "|[\textit{flags}]%
[|\def\jobname{|\textit{dest}|}|]|\input{|\textit{main}|}"|
\end{center}

%%%%%%%%%%%%%%%%%%%%%%%%%%%%%%%%%%%%%%%%%%%%%%%%%%%%%%%%%%%%%%%%%%%%%%%%%%%%%%%%
\subsection{Manual Code}
\label{sec:manual}

In case one cannot be certain whether the definitions file |childdoc.def|
is installed on the target \TeX{} distribution
and one prefers not to ship it,
it is conceivable to paste a few relevant commands into the sources.

To that end, drop all statements |\input{childdoc.def}|
and perform the replacements as outlined below.
Instead of |\childdocmain{|\textit{main}|}| add the following code
to the top of the main file:
%
\begin{center}
\begin{tabular}{l}
|\||ifdefined\childdocname\endinput\||fi\newif\ifchilddoc|\\
|\edef\childdocname{\scantokens\expandafter{\jobname\noexpand}}|\\
|\def\childdocmain{|\textit{main}|}\||ifx\childdocmain\childdocname\||else|\\
|\childdoctrue\includeonly{\childdocname}\let\jobname\childdocmain\||fi|\\
\end{tabular}
\end{center}
%
Instead of |\childdocof{|\textit{main}|}| just include the main file
at the top of each child file:
%
\begin{center}
|\input{|\textit{main}|}|
\end{center}
%
A simple redirection |\childdocforward{|\textit{dest}|}| is achieved by:
%
\begin{center}
|\def\jobname{|\textit{dest}|}\input{\jobname}|
\end{center}
%
The redirection with prefix
|\childdocforwardprefix[|\textit{prefix}|]{|\textit{dest}|}|
is accomplished by:
%
\begin{center}
\begin{tabular}{l}
|{\edef\jobname{\scantokens\expandafter{\jobname\noexpand}}|\\
|\def\redirectjob |\textit{prefix}|#1~~~{\gdef\jobname{|\textit{dest}|#1}}|\\
|\expandafter\redirectjob\jobname~~~}\input{\jobname}|
\end{tabular}
\end{center}

In an alternative approach,
child documents can be compiled by a specific command line
without additional code or specific definitions:
%
\begin{center}
|... -jobname "|\textit{target}|" "|[\textit{flags}]%
|\includeonly{|\textit{dest}|}\input{|\textit{main}|}"|
\end{center}
%

%%%%%%%%%%%%%%%%%%%%%%%%%%%%%%%%%%%%%%%%%%%%%%%%%%%%%%%%%%%%%%%%%%%%%%%%%%%%%%%%
%%%%%%%%%%%%%%%%%%%%%%%%%%%%%%%%%%%%%%%%%%%%%%%%%%%%%%%%%%%%%%%%%%%%%%%%%%%%%%%%
\section{Information}

%%%%%%%%%%%%%%%%%%%%%%%%%%%%%%%%%%%%%%%%%%%%%%%%%%%%%%%%%%%%%%%%%%%%%%%%%%%%%%%%
\subsection{Copyright}

Copyright \copyright{} 2017--2018 Niklas Beisert

This work may be distributed and/or modified under the
conditions of the \LaTeX{} Project Public License, either version 1.3
of this license or (at your option) any later version.
The latest version of this license is in
  \url{http://www.latex-project.org/lppl.txt}
and version 1.3 or later is part of all distributions of \LaTeX{}
version 2005/12/01 or later.

This work has the LPPL maintenance status `maintained'.

The Current Maintainer of this work is Niklas Beisert.

This work consists of the files |README.txt|, |childdoc.ins| and |childdoc.dtx|
as well as the derived files |childdoc.def|, |cdocsamp.tex|
with |cdocsch1.tex|, |cdocsch2.tex|, |cdocspt3.tex|, |cdocspt4.tex|,
|cdocsdrf.tex|, |cdocsfn1.tex|, |cdocsfn2.tex|
as well as |childdoc.pdf|.

%%%%%%%%%%%%%%%%%%%%%%%%%%%%%%%%%%%%%%%%%%%%%%%%%%%%%%%%%%%%%%%%%%%%%%%%%%%%%%%%
\subsection{Files and Installation}

The package consists of the files:
%
\begin{center}
\begin{tabular}{ll}
    |README.txt|   & readme file \\
    |childdoc.ins| & installation file \\
    |childdoc.dtx| & source file \\
    |childdoc.def| & definition file \\
    |cdocsamp.tex| & sample main file \\
    |cdocsch1.tex| & sample include file \\
    |cdocsch2.tex| & sample include file \\
    |cdocspt3.tex| & sample part file \\
    |cdocspt4.tex| & sample part file \\
    |cdocsdrf.tex| & sample redirection file \\
    |cdocsfn1.tex| & sample redirection file \\
    |cdocsfn2.tex| & sample redirection file \\
    |childdoc.pdf| & manual
\end{tabular}
\end{center}
%
The distribution consists of the files
|README.txt|, |childdoc.ins| and |childdoc.dtx|.
%
\begin{itemize}
\item
Run (pdf)\LaTeX{} on |childdoc.dtx|
to compile the manual |childdoc.pdf| (this file).
\item
Run \LaTeX{} on |childdoc.ins| to create the definitions file |childdoc.def|
and the sample |cdocsamp.tex| with include files
|cdocsch1.tex|, |cdocsch2.tex|, |cdocspt3.tex|, |cdocspt4.tex|,
|cdocsdrf.tex|, |cdocsfn1.tex|, |cdocsfn2.tex|.
Then copy the file |childdoc.def| to an appropriate directory of your \LaTeX{}
distribution, e.g.\ \textit{texmf-root}|/tex/latex/childdoc|.
\end{itemize}

%%%%%%%%%%%%%%%%%%%%%%%%%%%%%%%%%%%%%%%%%%%%%%%%%%%%%%%%%%%%%%%%%%%%%%%%%%%%%%%%
\subsection{Related CTAN Packages}

There are several other packages which offer a similar functionality:
%
\begin{itemize}
\item
The packages
\href{http://ctan.org/pkg/docmute}{\textsf{docmute}},
\href{http://ctan.org/pkg/includex}{\textsf{includex}} and
\href{http://ctan.org/pkg/standalone}{\textsf{standalone}}
provide commands to include only the document body of
a child file thus allowing both files to be compiled individually.
\item
The packages \href{http://ctan.org/pkg/subdocs}{\textsf{subdocs}}
and \href{http://ctan.org/pkg/subfiles}{\textsf{subfiles}}
provide structures in which the main and child documents can be
encapsulated and allowing them to be compiled individually.
The inclusion mechanism is different from the conventional |\include|.
\item
The package \href{http://ctan.org/pkg/combine}{\textsf{combine}}
is an elaborate solution to combine several documents into one.
\end{itemize}
%
See also the CTAN topic \href{http://ctan.org/topic/subdocs}{\textsf{subdocs}}
for further related packages.
The present package differs from the above solutions in that
a document structure constructed with the conventional |\include| mechanism
just needs two extra commands at the top of every file
such that all constituent files can be compiled individually.

%%%%%%%%%%%%%%%%%%%%%%%%%%%%%%%%%%%%%%%%%%%%%%%%%%%%%%%%%%%%%%%%%%%%%%%%%%%%%%%%
%\subsection{Feature Suggestions}
%
%The following is a list of features which may be useful for future
%versions of this package:
%%
%\begin{itemize}
%\item
%\ldots
%\end{itemize}

%%%%%%%%%%%%%%%%%%%%%%%%%%%%%%%%%%%%%%%%%%%%%%%%%%%%%%%%%%%%%%%%%%%%%%%%%%%%%%%%
\subsection{Revision History}

%%%%%%%%%%%%%%%%%%%%%%%%%%%%%%%%%%%%%%%%
\paragraph{v2.0:} 2018/12/30

\begin{itemize}
\item
immediate forward processing
\item
added |\childdocby| mechanism
\item
manual restructured
\end{itemize}

%%%%%%%%%%%%%%%%%%%%%%%%%%%%%%%%%%%%%%%%
\paragraph{v1.6:} 2018/01/17

\begin{itemize}
\item
application for development of include files
\item
corrections to manual
\end{itemize}

%%%%%%%%%%%%%%%%%%%%%%%%%%%%%%%%%%%%%%%%
\paragraph{v1.5:} 2017/05/21

\begin{itemize}
\item
more complete structuring introduced
\item
|\childdocof| introduced
\item
|\childdoc| renamed to |\childdocmain|
\item
|\childredirect| renamed to |\childdocforward| and |\childdocforwardprefix|
and functionality expanded
\end{itemize}

%%%%%%%%%%%%%%%%%%%%%%%%%%%%%%%%%%%%%%%%
\paragraph{v1.0:} 2017/04/27

\begin{itemize}
\item
manual and install package
\item
first version published on CTAN
\end{itemize}

%%%%%%%%%%%%%%%%%%%%%%%%%%%%%%%%%%%%%%%%
\paragraph{v0.6:} 2017/04/26

\begin{itemize}
\item
redirection mechanism added
\end{itemize}

%%%%%%%%%%%%%%%%%%%%%%%%%%%%%%%%%%%%%%%%
\paragraph{v0.5:} 2017/04/26

\begin{itemize}
\item
functionality in definition file
\end{itemize}


%%%%%%%%%%%%%%%%%%%%%%%%%%%%%%%%%%%%%%%%%%%%%%%%%%%%%%%%%%%%%%%%%%%%%%%%%%%%%%%%
%%%%%%%%%%%%%%%%%%%%%%%%%%%%%%%%%%%%%%%%%%%%%%%%%%%%%%%%%%%%%%%%%%%%%%%%%%%%%%%%
%%%%%%%%%%%%%%%%%%%%%%%%%%%%%%%%%%%%%%%%%%%%%%%%%%%%%%%%%%%%%%%%%%%%%%%%%%%%%%%%
\appendix

\settowidth\MacroIndent{\rmfamily\scriptsize 000\ }

 \DocInput{childdoc.dtx}

\end{document}
%</driver>
% \fi
%
% %%%%%%%%%%%%%%%%%%%%%%%%%%%%%%%%%%%%%%%%%%%%%%%%%%%%%%%%%%%%%%%%%%%%%%%%%%%%%%
% %%%%%%%%%%%%%%%%%%%%%%%%%%%%%%%%%%%%%%%%%%%%%%%%%%%%%%%%%%%%%%%%%%%%%%%%%%%%%%
% \section{Sample}
%\iffalse
%<*samplemain>
%\fi
%
% The following presents a sample document
% with two chapters, two parts, a title page,
% a compile flag as well as three forwarding files to set the flag.
% It consists of eight |.tex| files:
% \begin{center}
% \begin{tabular}{ll}
% |cdocsamp.tex|&main file\\
% |cdocsch1.tex|&include file for chapter 1\\
% |cdocsch2.tex|&include file for chapter 2\\
% |cdocspt3.tex|&include file for part 3\\
% |cdocspt4.tex|&include file for part 4\\
% |cdocsdrf.tex|&forwarding file for main file in draft mode\\
% |cdocsfi1.tex|&forwarding file for final version of chapter 1\\
% |cdocsfi2.tex|&forwarding file for final version of chapter 2\\
% \end{tabular}
% \end{center}
% Each of the eight files can be compiled directly by the \LaTeX{} compiler.
%
% %%%%%%%%%%%%%%%%%%%%%%%%%%%%%%%%%%%%%%
% \paragraph{Main File.}
%
% The main file is called |cdocsamp.tex|.
%
% Load the \textsf{childdoc} definitions and
% declare the filename for the main document:
%    \begin{macrocode}
\input{childdoc.def}
\childdocmain{}
%    \end{macrocode}

% Optional override for |\version| flag:
%    \begin{macrocode}
%%\ifchilddoc\else\providecommand{\version}{draft}\fi
%    \end{macrocode}

% Define the default values for the |\version| flag
% (|final| for the main file and |draft| for childs):
%    \begin{macrocode}
\ifchilddoc
\providecommand{\version}{draft}
\else
\providecommand{\version}{final}
\fi
%    \end{macrocode}

% Load the standard document class:
%    \begin{macrocode}
\documentclass[12pt]{article}
%    \end{macrocode}

% Start the document body:
%    \begin{macrocode}
\begin{document}
%    \end{macrocode}

% Declare a title page.
% Print title, part of document being processed and version flag:
%    \begin{macrocode}
\addtocounter{page}{-1}
\begin{center}
{\LARGE\bfseries{}childdoc example\par}
\vspace{1cm}
\ifchilddoc
\ifchilddocmanual part\else chapter\fi:
`\childdocname' of `\childdocjob'\par
\else
main document: `\childdocjob'\par
\fi
version: \version\par
\end{center}
\newpage
%    \end{macrocode}

% Manually include selected file,
% otherwise process as usual:
%    \begin{macrocode}
\ifchilddocmanual
\section*{part `\childdocname'}
\input{\childdocname}
\else
%    \end{macrocode}

% Include the two chapters:
%    \begin{macrocode}
\include{cdocsch1}
\include{cdocsch2}
%    \end{macrocode}

% Include the two parts unless only chapters should be displayed:
%    \begin{macrocode}
\ifchilddoc\else
\section{part three}
\input{cdocspt3}
\section{part four}
\input{cdocspt4}
\fi
%    \end{macrocode}

% Process as usual until here:
%    \begin{macrocode}
\fi
%    \end{macrocode}

% End of document body:
%    \begin{macrocode}
\end{document}
%    \end{macrocode}
%\iffalse
%</samplemain>
%\fi
%
% %%%%%%%%%%%%%%%%%%%%%%%%%%%%%%%%%%%%%%
% \paragraph{Chapter Include Files.}
%
% The include files are called |cdocsch1.tex| and |cdocsch2.tex|.
%
%\iffalse
%<*samplechap1|samplechap2>
%\fi

% Optional override for |\version| flag:
%    \begin{macrocode}
%%\providecommand{\version}{final}
%    \end{macrocode}

% Include the main document:
%    \begin{macrocode}
\input{childdoc.def}
\childdocof{cdocsamp}
%    \end{macrocode}

%\iffalse
%</samplechap1|samplechap2>
%\fi
%
%\iffalse
%<*samplechap1>
%\fi
% Some text for chapter 1:
%    \begin{macrocode}
\section{one}
some text in chapter one
%    \end{macrocode}

%\iffalse
%</samplechap1>
%\fi
% Some text for chapter 2:
%\iffalse
%<*samplechap2>
%\fi
%    \begin{macrocode}
\section{two}
more text in chapter two
%    \end{macrocode}

%\iffalse
%</samplechap2>
%\fi
%
% %%%%%%%%%%%%%%%%%%%%%%%%%%%%%%%%%%%%%%
% \paragraph{Part Include Files.}
%
% The include files are called |cdocspt3.tex| and |cdocspt4.tex|.
%
%\iffalse
%<*samplepart3|samplepart4>
%\fi

% Optional override for |\version| flag:
%    \begin{macrocode}
%%\providecommand{\version}{final}
%    \end{macrocode}

% Include the main document:
%    \begin{macrocode}
\input{childdoc.def}
\childdocby{cdocsamp}
%    \end{macrocode}

%\iffalse
%</samplepart3|samplepart4>
%\fi
%
%\iffalse
%<*samplepart3>
%\fi
% Some text for part 3:
%    \begin{macrocode}
some text in part three
%    \end{macrocode}

%\iffalse
%</samplepart3>
%\fi
% Some text for part 4:
%\iffalse
%<*samplepart4>
%\fi
%    \begin{macrocode}
more text in part four
%    \end{macrocode}

%\iffalse
%</samplepart4>
%\fi
%
% %%%%%%%%%%%%%%%%%%%%%%%%%%%%%%%%%%%%%%
% \paragraph{Forwarding for a Complete Draft.}
%
% The following forwarding file |cdocsdrf.tex|
% compiles the main document in draft mode:
%\iffalse
%<*sampledraft>
%\fi
%    \begin{macrocode}
\def\version{draft}
\input{childdoc.def}
\childdocforward{cdocsamp}
%    \end{macrocode}

%\iffalse
%</sampledraft>
%\fi
%
% %%%%%%%%%%%%%%%%%%%%%%%%%%%%%%%%%%%%%%
% \paragraph{Forwarding for Final Version of the Chapters.}
%
% The following forwarding files |cdocsfn1.tex| and |cdocsfn2.tex|
% (with identical content)
% compile the final versions of the child documents
% |cdocsch1.tex| and |cdocsch2.tex|, respectively:
%\iffalse
%<*samplefinal>
%\fi
%    \begin{macrocode}
\def\version{final}
\input{childdoc.def}
\childdocforwardprefix[cdocsamp]{cdocsfn}{cdocsch}
%    \end{macrocode}

%\iffalse
%</samplefinal>
%\fi
%
% %%%%%%%%%%%%%%%%%%%%%%%%%%%%%%%%%%%%%%
% \paragraph{Command Line Processing.}
%
% The following three command lines generate the output files
% |cdocscld|, |cdocscl1| and |cdocscl2|
% which should be identical to
% |cdocsdrf|, |cdocsch1| and |cdocsfn2|, respectively:
% \begin{center}
% \begin{tabular}{l}
% |latex -jobname cdocscld \|\\
% |  "\def\version{draft}\input{childdoc.def}\childdocforward{cdocsamp}"|\\
% |latex -jobname cdocscl1 \|\\
% |  "\input{childdoc.def}\childdocforward[cdocsamp]{cdocsch1}"|\\
% |latex -jobname cdocscl2 \|\\
% |  "\def\version{final}\input{childdoc.def}\childdocforward{cdocsch2}"|
% \end{tabular}
% \end{center}
% Note that the trailing backslash on each first line
% merely continues the input to the second line
% (for convenient cut ant paste).
% Furthermore, the command |latex| can be replaced by any
% of its alternative versions such as |pdflatex|.
%
% %%%%%%%%%%%%%%%%%%%%%%%%%%%%%%%%%%%%%%%%%%%%%%%%%%%%%%%%%%%%%%%%%%%%%%%%%%%%%%
% %%%%%%%%%%%%%%%%%%%%%%%%%%%%%%%%%%%%%%%%%%%%%%%%%%%%%%%%%%%%%%%%%%%%%%%%%%%%%%
% \section{Implementation}
%\iffalse
%<*package>
%\fi
%
% This section describes the definitions file |childdoc.def|.

% The definitions cannot be loaded using |\usepackage| or |\RequirePackage|
% which has a mechanism to prevent loading a style file more than once.
% When loading the definitions by means of |\input|
% multiple instances have to be prevented manually:
%\iffalse
%This code needs to be before the `\ProvidesFile' directive
%which is defined at the beginning of this file.
%Therefore it is also placed there and commented out here.
%</package>
%<*discard>
%\fi
%    \begin{macrocode}
\ifdefined\childdocmain\endinput\fi
%    \end{macrocode}
%\iffalse
%</discard>
%<*package>
%\fi
%
% \macro{\ifchilddoc}
% \macro{\ifchilddocmanual}
% The conditional |\ifchilddoc| tells whether a
% child (true) or main (false) document is being compiled.
% The conditional |\ifchilddocmanual| tells whether
% the |\includeonly| mechanism is used (false) or
% the selection of child files must be performed manually (true).
% The definitions initialise to false:
%    \begin{macrocode}
\newif\ifchilddoc
\newif\ifchilddocmanual
%    \end{macrocode}

% \macro{\childdocname}
% \macro{\childdocjob}
% The macro |\childdocname| stores the name of the main document
% to be compiled. The macro |\childdocjob| stores the name of
% the document on which the \LaTeX{} compiler was originally invoked.
% The content of |\jobname| cannot be compared
% to filenames specified in the source due to different catcodes.
% The following code rescans |\jobname|, stores the result
% in |\childdocname| and saves a copy in |\childdocjob|:
%    \begin{macrocode}
\edef\childdocname{\scantokens\expandafter{\jobname\noexpand}}
\let\childdocjob\childdocname
%    \end{macrocode}

% \macro{\childdocdisable}
% The macro |\childdocdisable| prevents the main file
% from being processed more than once.
% At this stage, the main document command |\childdocmain|
% is assumed to be called once again where it should do nothing.
% Any subsequent call to it should prevent
% a secondary processing of the main document
% It overwrites the forwarding commands
% |\childdocof| and |\childdocforward|
% with empty macros to prevent further inclusions of the main document:
%    \begin{macrocode}
\newcommand{\childdocdisable}
{
  \renewcommand{\childdocmain}[1]{\renewcommand{\childdocmain}[1]{\endinput}}
  \renewcommand{\childdocof}[1]{}
  \renewcommand{\childdocby}[2][]{}
  \renewcommand{\childdocforward}[2][]{}
  \renewcommand{\childdocdisable}{}
}
%    \end{macrocode}

% \macro{\childdocmain}
% The macro |\childdocmain| is to be called at the top of the main file
% with nothing or the main filename (without extension) as argument.
% First, it breaks loops.
% If the argument is not empty and does not match |\childdocname|
% (which is set by the first inclusion of |childdoc.def|),
% |\ifchilddoc| is set to true, |\includeonly| is applied to the child file
% and |\jobname| is set to the main file
% (for proper handling of |.aux| files):
%    \begin{macrocode}
\newcommand{\childdocmain}[1]
{
  \childdocdisable\childdocmain{}
  \if?#1?\else
    \begingroup
      \def\childdoctmp{#1}
      \ifx\childdoctmp\childdocname
        \def\childdoctmp{}
      \else
        \def\childdoctmp
        {
          \childdoctrue
          \includeonly{\childdocname}
          \def\childdocjob{#1}
          \def\jobname{#1}
        }
      \fi
      \expandafter
    \endgroup
    \childdoctmp
  \fi
}
%    \end{macrocode}

% \macro{\childdocof}
% The command |\childdocof| redirects
% compilation to the main file |#1|.
%    \begin{macrocode}
\newcommand{\childdocof}[1]
{
  \childdocdisable
  \childdoctrue
  \includeonly{\childdocname}
  \def\jobname{#1}
  \def\childdocjob{#1}
  \input{#1}
}
%    \end{macrocode}

% \macro{\childdocby}
% The command |\childdocby| ....
%    \begin{macrocode}
\newcommand{\childdocby}[2][]
{
  \childdocdisable
  \childdoctrue
  \childdocmanualtrue
  \if?#1?\else
    \def\jobname{#2}
  \fi
  \def\childdocjob{#2}
  \input{#2}
  \endinput
}
%    \end{macrocode}

% \macro{\childdocforward}
% The command |\childdocforward| redirects
% compilation to the main file or
% (if the optional argument is given) a child file.
% Parameters are set as if the main file
% or a child file starting with |\childdocof| was compiled.
% Then compilation is handed over to the main file:
%    \begin{macrocode}
\newcommand{\childdocforward}[2][]
{
  \begingroup
    \if?#1?
      \def\childdoctmp
      {
        \def\childdocname{#2}
        \def\childdocjob{#2}
        \def\jobname{#2}
        \input{#2}
        \endinput
      }
    \else
      \def\childdoctmp
      {
        \childdocdisable
        \def\childdocname{#2}
        \childdoctrue
        \includeonly{#2}
        \def\childdocjob{#1}
        \def\jobname{#1}
        \input{#1}
        \endinput
      }
    \fi
    \expandafter
  \endgroup
  \childdoctmp
}
%    \end{macrocode}

% \macro{\childdocforwardprefix}
% The command |\childdocforwardprefix| redirects
% compilation to the main or a child file by means of a pattern.
% The prefix |#1| in the current filename is replaced by |#2|
% and the suffix of the current filename is kept
% (it is assumed that the filename does not contain the substring `|~~~|'
% which is used as a delimiter).
% Compilation is handed over to the new file by |\childdocforward|:
%    \begin{macrocode}
\newcommand{\childdocforwardprefix}[3][]
{
  \begingroup
    \def\childdocextract #2##1~~~{\def\childdoctmp{\childdocforward[#1]{#3##1}}}
    \expandafter\childdocextract\childdocname~~~
    \expandafter
  \endgroup
  \childdoctmp
}
%    \end{macrocode}

% \macro{\childdoc}
% The deprecated macro |\childdoc| is a legacy version of |\childdocmain|:
%    \begin{macrocode}
\newcommand{\childdoc}{\childdocmain}
%    \end{macrocode}

% \macro{\childdocredirect}
% The deprecated macro |\childdocredirect| is a legacy version
% of |\childdocforward| and |\childdocforwardprefix|:
%    \begin{macrocode}
\newcommand{\childdocredirect}[2][]
{
  \begingroup
    \if?#1?
      \def\childdoctmp{\childdocforward{#2}}
    \else
      \def\childdoctmp{\childdocforwardprefix{#1}{#2}}
    \fi
    \expandafter
  \endgroup
  \childdoctmp
}
%    \end{macrocode}

%\iffalse
%</package>
%\fi
%
\endinput
|\\
|\childdocforward{|\textit{main}|}|
\end{tabular}
\end{center}
%
Likewise, the following files |final|\textit{nn}|.tex|
compile the final version of the child document
|child|\textit{nn}|.tex|:
%
\begin{center}
\begin{tabular}{l}
|\def\version{final}|\\
|% \iffalse
%
% childdoc.dtx Copyright (C) 2017-2018 Niklas Beisert
%
% This work may be distributed and/or modified under the
% conditions of the LaTeX Project Public License, either version 1.3
% of this license or (at your option) any later version.
% The latest version of this license is in
%   http://www.latex-project.org/lppl.txt
% and version 1.3 or later is part of all distributions of LaTeX
% version 2005/12/01 or later.
%
% This work has the LPPL maintenance status `maintained'.
%
% The Current Maintainer of this work is Niklas Beisert.
%
% This work consists of the files childdoc.dtx and childdoc.ins
% and the derived files childdoc.def and cdocsamp.tex with
% cdocsch1.tex, cdocsch2.tex, cdocsdrf.tex, cdocsfn1.tex, cdocsfn2.tex.
%
%<package>\ifdefined\childdocmain\endinput\fi
%<package>\ProvidesFile{childdoc.def}[2018/12/30 v2.0 child document driver]
%<samplemain>\ProvidesFile{cdocsamp.tex}[2018/12/30 v2.0 sample for childdoc]
%<*driver>
%\ProvidesFile{childdoc.drv}[2018/12/30 v2.0 childdoc reference manual file]
\PassOptionsToClass{10pt,a4paper}{article}
\documentclass{ltxdoc}

\usepackage[margin=35mm]{geometry}
\usepackage{hyperref}
\usepackage{hyperxmp}
\usepackage[usenames]{color}

\hypersetup{colorlinks=true}
\hypersetup{pdfstartview=FitH}
\hypersetup{pdfpagemode=UseNone}
\hypersetup{pdfsource={}}
\hypersetup{pdflang={en-UK}}
\hypersetup{pdfcopyright={Copyright 2017-2018 Niklas Beisert.
  This work may be distributed and/or modified under the
  conditions of the LaTeX Project Public License, either version 1.3
  of this license or (at your option) any later version.}}
\hypersetup{pdflicenseurl={http://www.latex-project.org/lppl.txt}}
\hypersetup{pdfcontactaddress={ETH Zurich, ITP, HIT K,
  Wolfgang-Pauli-Strasse 27}}
\hypersetup{pdfcontactpostcode={8093}}
\hypersetup{pdfcontactcity={Zurich}}
\hypersetup{pdfcontactcountry={Switzerland}}
\hypersetup{pdfcontactemail={nbeisert@itp.phys.ethz.ch}}
\hypersetup{pdfcontacturl={http://people.phys.ethz.ch/\xmptilde nbeisert/}}

\newcommand{\secref}[1]{\hyperref[#1]{section \ref*{#1}}}

\parskip1ex
\parindent0pt
\let\olditemize\itemize
\def\itemize{\olditemize\parskip0pt}

\begin{document}

\title{The \textsf{childdoc} Package}
\hypersetup{pdftitle={The childdoc Package}}
\author{Niklas Beisert\\[2ex]
  Institut f\"ur Theoretische Physik\\
  Eidgen\"ossische Technische Hochschule Z\"urich\\
  Wolfgang-Pauli-Strasse 27, 8093 Z\"urich, Switzerland\\[1ex]
  \href{mailto:nbeisert@itp.phys.ethz.ch}
  {\texttt{nbeisert@itp.phys.ethz.ch}}}
\hypersetup{pdfauthor={Niklas Beisert}}
\hypersetup{pdfsubject={Manual for the LaTeX2e Package childdoc}}
\date{30 December 2018, \textsf{v2.0}}
\maketitle

\begin{abstract}\noindent
\textsf{childdoc} is a \LaTeXe{} package
that enables the direct compilation
of document sections included by |\include|
to individual files.
\end{abstract}

\begingroup
\parskip0ex
\tableofcontents
\endgroup

%%%%%%%%%%%%%%%%%%%%%%%%%%%%%%%%%%%%%%%%%%%%%%%%%%%%%%%%%%%%%%%%%%%%%%%%%%%%%%%%
%%%%%%%%%%%%%%%%%%%%%%%%%%%%%%%%%%%%%%%%%%%%%%%%%%%%%%%%%%%%%%%%%%%%%%%%%%%%%%%%
\section{Introduction}

\LaTeX{} provides a mechanism to structure a large document (such as a book)
into a main file and several child files (containing the chapters)
using the |\include| command.
This mechanism is beneficial for documents
which span hundreds of pages in order to
make the source file(s) more manageable.
Moreover, compilation can be restricted to
selected child files by means of the |\includeonly| command.
The latter feature can be used to reduce the compilation time while editing
(this was significantly more useful in the earlier days of \LaTeX{})
or to generate a smaller document which is easier to navigate.
Another application of |\includeonly| is to generate
documents consisting of selected parts of the complete document.

However, there are a few drawbacks of the plain |\include| mechanism:
\begin{itemize}
\item
The child files cannot be compiled on their own,
they can only be compiled via the main file.
A naive editing environment
(such as a text editor with an option
to have the current file processed by \LaTeX)
may require one to switch to the main file before compiling;
attempting to compile the child file produces errors.
\item
The main file must be modified (each time)
to adjust the |\includeonly| command
to the present needs. This easily leaves the main file in a messy state.
\item
The generated document will always carry the filename
of the main document. This is inconvenient if
several child files are to be compiled and
to be kept for distribution.
\end{itemize}

The present package provides a simple interface
to make child files individually compilable by \LaTeX{}.
Compiling a child file then has the same effect as compiling
the main file with an |\includeonly| command
to select the appropriate child.
Moreover the generated document will carry the name of the child
rather than the main file.
This resolves all three above issues.

This feature is meant to make the editing of books,
thesis documents and lecture notes somewhat more convenient.
However, the package can also be used efficiently for
composing a series of documents (such as exercise sheets)
which are typically distributed individually.
It then assists the author in generating the individual documents
(potentially in different versions)
as well as a document containing the collected series.
Another application is in developing style files
or other kinds of included material
where compilation of the style file could redirect
to a sample or test file.

%%%%%%%%%%%%%%%%%%%%%%%%%%%%%%%%%%%%%%%%%%%%%%%%%%%%%%%%%%%%%%%%%%%%%%%%%%%%%%%%
%%%%%%%%%%%%%%%%%%%%%%%%%%%%%%%%%%%%%%%%%%%%%%%%%%%%%%%%%%%%%%%%%%%%%%%%%%%%%%%%
\section{Usage}

First of all, the package \textsf{childdoc} is \emph{not} a standard
\LaTeXe{} |.sty| style file! Therefore it needs to be invoked in
a non-standard way.

%%%%%%%%%%%%%%%%%%%%%%%%%%%%%%%%%%%%%%%%%%%%%%%%%%%%%%%%%%%%%%%%%%%%%%%%%%%%%%%%
\subsection{Included Files}
\label{sec:include}

%%%%%%%%%%%%%%%%%%%%%%%%%%%%%%%%%%%%%%%%
\DescribeMacro{\childdocmain}
To use the package, add the commands
\begin{center}
\begin{tabular}{l}
|\input{childdoc.def}|\\
|\childdocmain{}|\\
\end{tabular}
\end{center}
at the very top of the main \LaTeX{} file,
in particular \emph{before} the |\documentclass| statement!
The argument of |\childdocmain| should be left empty
(but it must be present).

%%%%%%%%%%%%%%%%%%%%%%%%%%%%%%%%%%%%%%%%
\DescribeMacro{\childdocof}
Furthermore, add the commands
\begin{center}
\begin{tabular}{l}
|\input{childdoc.def}|\\
|\childdocof{|\textit{main}|}|\\
\end{tabular}
\end{center}
at the top of every child file \textit{child}
which is included by |\include{|\textit{child}|}|
from within the main file
(or at least for those files to be compiled individually).
The argument \textit{main} must be the filename of the main file.

There are a couple of
considerations in setting up the main and child documents:

%%%%%%%%%%%%%%%%%%%%%%%%%%%%%%%%%%%%%%%%
\paragraph{Restrictions.}

Please note the following restrictions:
\begin{itemize}
\item
|\childdocmain| must be called with one argument \textit{main}
to ensure compatibility with earlier version of the package.
It must either be empty (|\childdocmain{}|)
or precisely match the filename of the main file in which it is specified.
See \secref{sec:detection} for further information.
\item
The filename \textit{main} must be specified without the |.tex| extension.
\item
The filename \textit{main} is case sensitive
(even in case-insensitive file systems)
due to internal string comparison.
\item
The argument \textit{main} should be fully expanded, it cannot be a macro.
\item
Subdirectories and special characters should be avoided in filenames.
\item
The command |\childdocmain{|\textit{main}|}| must be followed by a whitespace.
It should not be followed immediately by another command
or by a comment mark `|%|'.
This is because the \TeX{} parser reads the token immediately following
the argument of |\childdocmain| and puts it
at the beginning of every child section;
however, a white\-space is ignored.
\end{itemize}

%%%%%%%%%%%%%%%%%%%%%%%%%%%%%%%%%%%%%%%%
\paragraph{Content of Main File.}

It is advisable to place all content in the child files included by |\include|.
Any output contained in the main file will appear in all child documents
unless suppressed manually;
it cannot be suppressed automatically by the |\includeonly| directive
and thus should normally be avoided.
A method to include some content in the main file
by means of conditional processing is described in \secref{sec:conditional}.

%%%%%%%%%%%%%%%%%%%%%%%%%%%%%%%%%%%%%%%%
\paragraph{Page Numbering.}

When only a part of the document is compiled,
the appropriate numbering of pages
(as well as other status parameters)
is determined from the |.aux| files.
The latter contain information from previous passes.
However this information needs to propagate through
all intermediate child documents.
Therefore the page numbering in child documents may well
be inconsistent until the complete document is compiled at least once.

A useful (if unconventional) way to always ensure a consistent
page numbering is to restart the numbering in each child document
and denote the pages by `\textit{child}|.|\textit{page}'
where \textit{child} represents the chapter/section number of the child file.
This can be achieved by the command
|\numberwithin{page}{|\textit{child}|}|
of the \textsf{amsmath} package
where \textit{child} can be |chapter| or |section|
depending on the chosen structuring.
Alternatively, one can modify the macro |\thepage| appropriately
and reset the counter |page| at the start of each child file.

%%%%%%%%%%%%%%%%%%%%%%%%%%%%%%%%%%%%%%%%%%%%%%%%%%%%%%%%%%%%%%%%%%%%%%%%%%%%%%%%
\subsection{Conditional Processing}
\label{sec:conditional}

The package provides a mechanism to compile different versions
of a document. To customise the versions further some conditional processing
can come in handy to distinguish which version is being compiled.
The package provides two macros to describe the compilation context:

%%%%%%%%%%%%%%%%%%%%%%%%%%%%%%%%%%%%%%%%
\DescribeMacro{\ifchilddoc}
The conditional |\ifchilddoc| distinguishes between the compilation of
child documents and the main document:
%
\begin{center}
|\ifchilddoc |\textit{child-code}| |[|\||else |\textit{main-code}]| \||fi|
\end{center}

%%%%%%%%%%%%%%%%%%%%%%%%%%%%%%%%%%%%%%%%
\DescribeMacro{\childdocname}
\DescribeMacro{\childdocjob}
The macro |\childdocname| contains the filename (without extension)
of the main or child file being processed.
Note that |\childdocjob| will always contain the name of the main file.

%%%%%%%%%%%%%%%%%%%%%%%%%%%%%%%%%%%%%%%%
\paragraph{Title Page.}

Conditional processing can be used to include a title or banner page
in the main document when proper precautions are taken.
Importantly, the code in the main file should ensure that the page counter
(as well as other status parameters which are stored in the |.aux| files)
takes the same value after the conditional processing.
Otherwise the page numbers may take divergent values
depending on which part is compiled.

For example, a title page could be declared by:
%
\begin{center}
\begin{tabular}{l}
|\ifchilddoc\||else|\\
|\addtocounter{page}{-1}|\\
\textit{code for title page}\\
|\newpage|\\
|\||fi|
\end{tabular}
\end{center}
%
A banner page for the child documents can be generated by:
%
\begin{center}
\begin{tabular}{l}
|\ifchilddoc|\\
|\addtocounter{page}{-1}|\\
\textit{code for banner page}\\
|\newpage|\\
|\||fi|
\end{tabular}
\end{center}
%
Here one could write a message such as:
\begin{center}
|This is the part \childdocname{} of \childdocjob{}.|
\end{center}

%%%%%%%%%%%%%%%%%%%%%%%%%%%%%%%%%%%%%%%%%%%%%%%%%%%%%%%%%%%%%%%%%%%%%%%%%%%%%%%%
\subsection{Flags}
\label{sec:flags}

The package makes it easy to generate different versions
of the main or child documents.
To this end compilation flags can be defined
and assigned different default values.
They will be particularly useful in conjunction
with the forwarding mechanism described in \secref{sec:forward}.

For example, it may be useful to have a flag |\version|
which can be set to |draft| or |final|.
The document source will contain some conditional code
depending on the value of |\version|.
Suppose further, the flag should default to |final| for the main file
and to |draft| for child files
which is a natural assignment for editing the document.
This is achieved by placing the following code
in the preamble of the main document
(below the |\childdocmain| directive):
%
\begin{center}
\begin{tabular}{l}
|\ifchilddoc|\\
|\providecommand{\version}{draft}|\\
|\||else|\\
|\providecommand{\version}{final}|\\
|\||fi|
\end{tabular}
\end{center}
%
The definition by |\providecommand| makes sure
that previous definitions are not overwritten.
Further statements |\providecommand{\version}{...}|
can thus be added before the above code to override it.

For the main file, one might add a line
(between |\childdocmain| and the above block)
%
\begin{center}
|%\ifchilddoc\||else\providecommand{\version}{draft}\||fi|
\end{center}
%
which can be uncommented to produce a draft version.
Likewise one can add a line to the very top of a child file
(above the |\childdocof{|\textit{main}|}| directive)
%
\begin{center}
|%\providecommand{\version}{final}|
\end{center}
%
which can be uncommented to produce the final version of this child document.

%%%%%%%%%%%%%%%%%%%%%%%%%%%%%%%%%%%%%%%%%%%%%%%%%%%%%%%%%%%%%%%%%%%%%%%%%%%%%%%%
\subsection{Forwarding}
\label{sec:forward}

Different versions of the main or child documents
using compilation flags as described in \secref{sec:flags}
can be (permanently) stored in different files
for convenient compilation, viewing and distribution.
To this end, the package defines a command
to pass on compilation to a different file:

%%%%%%%%%%%%%%%%%%%%%%%%%%%%%%%%%%%%%%%%
\DescribeMacro{\childdocforward}
The command |\childdocforward| redirects processing to
another source file:
%
\begin{center}
\begin{tabular}{l}
|\input{childdoc.def}|\\
|\childdocforward[|\textit{main}|]{|\textit{dest}|}|\\
\end{tabular}
\end{center}
%
The argument \textit{dest} is the destination file
(without extension).
It should be the main file or one of the child files.
Note that further \textsf{childdoc} directives
such as |\childdocof| and |\childdocforward|
in the indicated file will be processed in this form.
The optional argument \textit{main}
passes on directly to the main file \textit{main}
while pretending to compile the child \textit{dest}.
This form behaves as if \textit{dest}
issues |\childdocof{|\textit{main}|}| right away,
and no further \textsf{childdoc} directives will be processed.

%%%%%%%%%%%%%%%%%%%%%%%%%%%%%%%%%%%%%%%%
\DescribeMacro{\...prefix}
In the alternative form |\childdocforwardprefix|,
%
\begin{center}
\begin{tabular}{l}
|\input{childdoc.def}|\\
|\childdocforwardprefix[|\textit{main}|]{|\textit{prefix}|}{|\textit{dest}|}|
\end{tabular}
\end{center}
%
the destination file is determined by a pattern
depending on the current file:
To make this work, the current file must be called
`{\textit{prefix}\hspace{0.2em}\textit{suffix}}'
with \textit{prefix} matching precisely the argument.
Processing is then passed on to the file
`{\textit{dest}\hspace{0.2em}\textit{suffix}}'.
Surely, the same effect is achieved by
directly specifying the
argument `{\textit{dest}\hspace{0.2em}\textit{suffix}}'
in the first form.
However, that requires to set up a different file
for each child. With the alternative form of the command
all these files can have exactly the same content
which simplifies setting them up and maintaining them.

For example, the following file |draft.tex|
with a compilation flag |\version| as described in \secref{sec:flags}
compiles the main document as a draft:
%
\begin{center}
\begin{tabular}{l}
|\def\version{draft}|\\
|\input{childdoc.def}|\\
|\childdocforward{|\textit{main}|}|
\end{tabular}
\end{center}
%
Likewise, the following files |final|\textit{nn}|.tex|
compile the final version of the child document
|child|\textit{nn}|.tex|:
%
\begin{center}
\begin{tabular}{l}
|\def\version{final}|\\
|\input{childdoc.def}|\\
|\childdocforwardprefix{final}{child}|
\end{tabular}
\end{center}
%

Note that when several versions of a main file and/or of each child file
are to be generated, it may be convenient to set up a |Makefile| or
shell script to automatise the process.

%%%%%%%%%%%%%%%%%%%%%%%%%%%%%%%%%%%%%%%%%%%%%%%%%%%%%%%%%%%%%%%%%%%%%%%%%%%%%%%%
\subsection{Command Line Processing}
\label{sec:commandline}

The effect of redirection files can also be achieved by invoking
the \LaTeX{} compiler with a more elaborate command line.
Most conveniently this should be done as part
of a shell script or a |Makefile|.

When using \textsf{childdoc} in the main file, the following
command lines effectively perform a redirection
(note that depending on the shell being used,
backslashes may have to be doubled: `|\|' $\to$ `|\\|'):
%
\begin{center}
|... -jobname "|\textit{target}|" |\\|"|[\textit{flags}]%
|\input{childdoc.def}\childdocforward[|\textit{main}|]{|\textit{dest}|}"|
\end{center}
%
Here \textit{target} is the name of the output file,
\textit{main} is the name of the main file
and \textit{dest} is the name of the main or child file to be processed
(all filenames without extensions).
The optional argument \textit{main} can be omitted
if \textit{main} matches \textit{dest}.
Optionally, compilation \textit{flags} can be defined via |\def| commands.
This command line makes the \TeX{} engine believe
it is compiling the file \textit{target}
whose content is specified as the latter parameter.
The provided code then forwards the processing to
\textit{main} or \textit{dest} as described in \secref{sec:forward}.

%%%%%%%%%%%%%%%%%%%%%%%%%%%%%%%%%%%%%%%%%%%%%%%%%%%%%%%%%%%%%%%%%%%%%%%%%%%%%%%%
\subsection{Include by Input}
\label{sec:input}

Including child documents by |\include| has some restrictions by design.
Most notably, the content of a child document always occupies
its own set of pages; pages cannot be shared between child documents.
Usually, this behaviour makes perfect sense
because each child document contain an essential part of the document.
However, in some situations it may be desirable to compose
a document from a collection of parts
without having mandatory page breaks between then.
For this case, the package
provides a mechanism to include parts
by |\input| which can also be processed individually.
However, by construction this mechanism
requires manual handling of the content to be output.

%%%%%%%%%%%%%%%%%%%%%%%%%%%%%%%%%%%%%%%%
\DescribeMacro{\ifchilddocmanual}
The main file should be prepared as usual, see \secref{sec:include}.
However, the document body must make a distinction
between processing of an individual part and of the main document, e.g.:
%
\begin{center}
\begin{tabular}{l}
|\ifchilddocmanual|\\
|\input{\childdocname}|\\
|\||else|\\
\textit{document body with }|\input{|\textit{part}|}|\\
|\||fi|
\end{tabular}
\end{center}
%
The conditional |\ifchilddocmanual| is true whenever
a part to be included by |\input| is being compiled,
and the name of the part is stored in |\childdocname|.

%%%%%%%%%%%%%%%%%%%%%%%%%%%%%%%%%%%%%%%%
\DescribeMacro{\childdocby}
Each part to be included by |\input| should start with:
%
\begin{center}
\begin{tabular}{l}
|\input{childdoc.def}|\\
|\childdocby{|\textit{main}|}|\\
\end{tabular}
\end{center}
%
The directive |\childdocby| is similar to |\childdocof|
described in \secref{sec:include},
but the subsequent selection of content must be done manually.
To that end, both |\ifchilddoc| and |\ifchilddocmanual|
will be true upon processing of a part,
and the name of the part is stored in |\childdocname|.
Note that |\jobname| will be set to the filename of the current part
so that each part receives an individual |.aux| file
that does not interfere with the |.aux| file(s) of the main document.
This behaviour can be altered by the alternative form
|\childdocby[*]{|\textit{main}|}| (with a non-empty optional argument)
which uses the |.aux| file of the main document
by setting |\jobname| to \textit{main}.

%%%%%%%%%%%%%%%%%%%%%%%%%%%%%%%%%%%%%%%%%%%%%%%%%%%%%%%%%%%%%%%%%%%%%%%%%%%%%%%%
\subsection{Driver Development}
\label{sec:driver}

The \textsf{childdoc} mechanism can also be use for the development
of definition files such as \LaTeX{} styles or classes.
This case differs from the above setup with multiple parts
included by |\include| in that no |\includeonly| should be invoked.
This can be achieved by starting the include file
(before |\ProvidesPackage|) with:
%
\begin{center}
\begin{tabular}{l}
|\input{childdoc.def}|\\
|\childdocforward{|\textit{main}|}|\\
\end{tabular}
\end{center}
%
or alternatively with:
%
\begin{center}
\begin{tabular}{l}
|\input{childdoc.def}|\\
|\childdocby{|\textit{main}|}|\\
\end{tabular}
\end{center}
%
Both forms have slightly different effects as described above.
The main file is prepared as usual, see \secref{sec:include}.

%%%%%%%%%%%%%%%%%%%%%%%%%%%%%%%%%%%%%%%%%%%%%%%%%%%%%%%%%%%%%%%%%%%%%%%%%%%%%%%%
\subsection{Legacy Detection}
\label{sec:detection}

The directive |\childdocmain| in the main file can detect
whether the complete document or merely a child is to be compiled
even without using the directive |\childdocof|.
This method is deprecated because it is less robust
and there is no compelling reason to use it;
it is merely provided for backward compatibility
and it may be removed in future versions.

If the detection mechanism is to be used,
it is mandatory to correctly specify
the filename of the main file as the argument of |\childdocmain|:
%
\begin{center}
\begin{tabular}{l}
|\input{childdoc.def}|\\
|\childdocmain{|\textit{main}|}|\\
\end{tabular}
\end{center}
%
If |\jobname| does not match the argument \textit{main} of |\childdocmain|,
it is assumed that |\jobname| points to the child file to be compiled.
When using |\childdocmain| with the main file specified as argument,
it suffices to start a child file
with just |\input{|\textit{main}|}|
without loading of the package and using |\childdocof|.
If instead all processing is done
with the appropriate \textsf{childdoc} directives,
the argument of \textit{main} of |\childdocmain| can be empty.

An alternative version of the command line processing described
in \secref{sec:commandline} using the detection mechanism reads:
%
\begin{center}
|... -jobname "|\textit{target}|" "|[\textit{flags}]%
[|\def\jobname{|\textit{dest}|}|]|\input{|\textit{main}|}"|
\end{center}

%%%%%%%%%%%%%%%%%%%%%%%%%%%%%%%%%%%%%%%%%%%%%%%%%%%%%%%%%%%%%%%%%%%%%%%%%%%%%%%%
\subsection{Manual Code}
\label{sec:manual}

In case one cannot be certain whether the definitions file |childdoc.def|
is installed on the target \TeX{} distribution
and one prefers not to ship it,
it is conceivable to paste a few relevant commands into the sources.

To that end, drop all statements |\input{childdoc.def}|
and perform the replacements as outlined below.
Instead of |\childdocmain{|\textit{main}|}| add the following code
to the top of the main file:
%
\begin{center}
\begin{tabular}{l}
|\||ifdefined\childdocname\endinput\||fi\newif\ifchilddoc|\\
|\edef\childdocname{\scantokens\expandafter{\jobname\noexpand}}|\\
|\def\childdocmain{|\textit{main}|}\||ifx\childdocmain\childdocname\||else|\\
|\childdoctrue\includeonly{\childdocname}\let\jobname\childdocmain\||fi|\\
\end{tabular}
\end{center}
%
Instead of |\childdocof{|\textit{main}|}| just include the main file
at the top of each child file:
%
\begin{center}
|\input{|\textit{main}|}|
\end{center}
%
A simple redirection |\childdocforward{|\textit{dest}|}| is achieved by:
%
\begin{center}
|\def\jobname{|\textit{dest}|}\input{\jobname}|
\end{center}
%
The redirection with prefix
|\childdocforwardprefix[|\textit{prefix}|]{|\textit{dest}|}|
is accomplished by:
%
\begin{center}
\begin{tabular}{l}
|{\edef\jobname{\scantokens\expandafter{\jobname\noexpand}}|\\
|\def\redirectjob |\textit{prefix}|#1~~~{\gdef\jobname{|\textit{dest}|#1}}|\\
|\expandafter\redirectjob\jobname~~~}\input{\jobname}|
\end{tabular}
\end{center}

In an alternative approach,
child documents can be compiled by a specific command line
without additional code or specific definitions:
%
\begin{center}
|... -jobname "|\textit{target}|" "|[\textit{flags}]%
|\includeonly{|\textit{dest}|}\input{|\textit{main}|}"|
\end{center}
%

%%%%%%%%%%%%%%%%%%%%%%%%%%%%%%%%%%%%%%%%%%%%%%%%%%%%%%%%%%%%%%%%%%%%%%%%%%%%%%%%
%%%%%%%%%%%%%%%%%%%%%%%%%%%%%%%%%%%%%%%%%%%%%%%%%%%%%%%%%%%%%%%%%%%%%%%%%%%%%%%%
\section{Information}

%%%%%%%%%%%%%%%%%%%%%%%%%%%%%%%%%%%%%%%%%%%%%%%%%%%%%%%%%%%%%%%%%%%%%%%%%%%%%%%%
\subsection{Copyright}

Copyright \copyright{} 2017--2018 Niklas Beisert

This work may be distributed and/or modified under the
conditions of the \LaTeX{} Project Public License, either version 1.3
of this license or (at your option) any later version.
The latest version of this license is in
  \url{http://www.latex-project.org/lppl.txt}
and version 1.3 or later is part of all distributions of \LaTeX{}
version 2005/12/01 or later.

This work has the LPPL maintenance status `maintained'.

The Current Maintainer of this work is Niklas Beisert.

This work consists of the files |README.txt|, |childdoc.ins| and |childdoc.dtx|
as well as the derived files |childdoc.def|, |cdocsamp.tex|
with |cdocsch1.tex|, |cdocsch2.tex|, |cdocspt3.tex|, |cdocspt4.tex|,
|cdocsdrf.tex|, |cdocsfn1.tex|, |cdocsfn2.tex|
as well as |childdoc.pdf|.

%%%%%%%%%%%%%%%%%%%%%%%%%%%%%%%%%%%%%%%%%%%%%%%%%%%%%%%%%%%%%%%%%%%%%%%%%%%%%%%%
\subsection{Files and Installation}

The package consists of the files:
%
\begin{center}
\begin{tabular}{ll}
    |README.txt|   & readme file \\
    |childdoc.ins| & installation file \\
    |childdoc.dtx| & source file \\
    |childdoc.def| & definition file \\
    |cdocsamp.tex| & sample main file \\
    |cdocsch1.tex| & sample include file \\
    |cdocsch2.tex| & sample include file \\
    |cdocspt3.tex| & sample part file \\
    |cdocspt4.tex| & sample part file \\
    |cdocsdrf.tex| & sample redirection file \\
    |cdocsfn1.tex| & sample redirection file \\
    |cdocsfn2.tex| & sample redirection file \\
    |childdoc.pdf| & manual
\end{tabular}
\end{center}
%
The distribution consists of the files
|README.txt|, |childdoc.ins| and |childdoc.dtx|.
%
\begin{itemize}
\item
Run (pdf)\LaTeX{} on |childdoc.dtx|
to compile the manual |childdoc.pdf| (this file).
\item
Run \LaTeX{} on |childdoc.ins| to create the definitions file |childdoc.def|
and the sample |cdocsamp.tex| with include files
|cdocsch1.tex|, |cdocsch2.tex|, |cdocspt3.tex|, |cdocspt4.tex|,
|cdocsdrf.tex|, |cdocsfn1.tex|, |cdocsfn2.tex|.
Then copy the file |childdoc.def| to an appropriate directory of your \LaTeX{}
distribution, e.g.\ \textit{texmf-root}|/tex/latex/childdoc|.
\end{itemize}

%%%%%%%%%%%%%%%%%%%%%%%%%%%%%%%%%%%%%%%%%%%%%%%%%%%%%%%%%%%%%%%%%%%%%%%%%%%%%%%%
\subsection{Related CTAN Packages}

There are several other packages which offer a similar functionality:
%
\begin{itemize}
\item
The packages
\href{http://ctan.org/pkg/docmute}{\textsf{docmute}},
\href{http://ctan.org/pkg/includex}{\textsf{includex}} and
\href{http://ctan.org/pkg/standalone}{\textsf{standalone}}
provide commands to include only the document body of
a child file thus allowing both files to be compiled individually.
\item
The packages \href{http://ctan.org/pkg/subdocs}{\textsf{subdocs}}
and \href{http://ctan.org/pkg/subfiles}{\textsf{subfiles}}
provide structures in which the main and child documents can be
encapsulated and allowing them to be compiled individually.
The inclusion mechanism is different from the conventional |\include|.
\item
The package \href{http://ctan.org/pkg/combine}{\textsf{combine}}
is an elaborate solution to combine several documents into one.
\end{itemize}
%
See also the CTAN topic \href{http://ctan.org/topic/subdocs}{\textsf{subdocs}}
for further related packages.
The present package differs from the above solutions in that
a document structure constructed with the conventional |\include| mechanism
just needs two extra commands at the top of every file
such that all constituent files can be compiled individually.

%%%%%%%%%%%%%%%%%%%%%%%%%%%%%%%%%%%%%%%%%%%%%%%%%%%%%%%%%%%%%%%%%%%%%%%%%%%%%%%%
%\subsection{Feature Suggestions}
%
%The following is a list of features which may be useful for future
%versions of this package:
%%
%\begin{itemize}
%\item
%\ldots
%\end{itemize}

%%%%%%%%%%%%%%%%%%%%%%%%%%%%%%%%%%%%%%%%%%%%%%%%%%%%%%%%%%%%%%%%%%%%%%%%%%%%%%%%
\subsection{Revision History}

%%%%%%%%%%%%%%%%%%%%%%%%%%%%%%%%%%%%%%%%
\paragraph{v2.0:} 2018/12/30

\begin{itemize}
\item
immediate forward processing
\item
added |\childdocby| mechanism
\item
manual restructured
\end{itemize}

%%%%%%%%%%%%%%%%%%%%%%%%%%%%%%%%%%%%%%%%
\paragraph{v1.6:} 2018/01/17

\begin{itemize}
\item
application for development of include files
\item
corrections to manual
\end{itemize}

%%%%%%%%%%%%%%%%%%%%%%%%%%%%%%%%%%%%%%%%
\paragraph{v1.5:} 2017/05/21

\begin{itemize}
\item
more complete structuring introduced
\item
|\childdocof| introduced
\item
|\childdoc| renamed to |\childdocmain|
\item
|\childredirect| renamed to |\childdocforward| and |\childdocforwardprefix|
and functionality expanded
\end{itemize}

%%%%%%%%%%%%%%%%%%%%%%%%%%%%%%%%%%%%%%%%
\paragraph{v1.0:} 2017/04/27

\begin{itemize}
\item
manual and install package
\item
first version published on CTAN
\end{itemize}

%%%%%%%%%%%%%%%%%%%%%%%%%%%%%%%%%%%%%%%%
\paragraph{v0.6:} 2017/04/26

\begin{itemize}
\item
redirection mechanism added
\end{itemize}

%%%%%%%%%%%%%%%%%%%%%%%%%%%%%%%%%%%%%%%%
\paragraph{v0.5:} 2017/04/26

\begin{itemize}
\item
functionality in definition file
\end{itemize}


%%%%%%%%%%%%%%%%%%%%%%%%%%%%%%%%%%%%%%%%%%%%%%%%%%%%%%%%%%%%%%%%%%%%%%%%%%%%%%%%
%%%%%%%%%%%%%%%%%%%%%%%%%%%%%%%%%%%%%%%%%%%%%%%%%%%%%%%%%%%%%%%%%%%%%%%%%%%%%%%%
%%%%%%%%%%%%%%%%%%%%%%%%%%%%%%%%%%%%%%%%%%%%%%%%%%%%%%%%%%%%%%%%%%%%%%%%%%%%%%%%
\appendix

\settowidth\MacroIndent{\rmfamily\scriptsize 000\ }

 \DocInput{childdoc.dtx}

\end{document}
%</driver>
% \fi
%
% %%%%%%%%%%%%%%%%%%%%%%%%%%%%%%%%%%%%%%%%%%%%%%%%%%%%%%%%%%%%%%%%%%%%%%%%%%%%%%
% %%%%%%%%%%%%%%%%%%%%%%%%%%%%%%%%%%%%%%%%%%%%%%%%%%%%%%%%%%%%%%%%%%%%%%%%%%%%%%
% \section{Sample}
%\iffalse
%<*samplemain>
%\fi
%
% The following presents a sample document
% with two chapters, two parts, a title page,
% a compile flag as well as three forwarding files to set the flag.
% It consists of eight |.tex| files:
% \begin{center}
% \begin{tabular}{ll}
% |cdocsamp.tex|&main file\\
% |cdocsch1.tex|&include file for chapter 1\\
% |cdocsch2.tex|&include file for chapter 2\\
% |cdocspt3.tex|&include file for part 3\\
% |cdocspt4.tex|&include file for part 4\\
% |cdocsdrf.tex|&forwarding file for main file in draft mode\\
% |cdocsfi1.tex|&forwarding file for final version of chapter 1\\
% |cdocsfi2.tex|&forwarding file for final version of chapter 2\\
% \end{tabular}
% \end{center}
% Each of the eight files can be compiled directly by the \LaTeX{} compiler.
%
% %%%%%%%%%%%%%%%%%%%%%%%%%%%%%%%%%%%%%%
% \paragraph{Main File.}
%
% The main file is called |cdocsamp.tex|.
%
% Load the \textsf{childdoc} definitions and
% declare the filename for the main document:
%    \begin{macrocode}
\input{childdoc.def}
\childdocmain{}
%    \end{macrocode}

% Optional override for |\version| flag:
%    \begin{macrocode}
%%\ifchilddoc\else\providecommand{\version}{draft}\fi
%    \end{macrocode}

% Define the default values for the |\version| flag
% (|final| for the main file and |draft| for childs):
%    \begin{macrocode}
\ifchilddoc
\providecommand{\version}{draft}
\else
\providecommand{\version}{final}
\fi
%    \end{macrocode}

% Load the standard document class:
%    \begin{macrocode}
\documentclass[12pt]{article}
%    \end{macrocode}

% Start the document body:
%    \begin{macrocode}
\begin{document}
%    \end{macrocode}

% Declare a title page.
% Print title, part of document being processed and version flag:
%    \begin{macrocode}
\addtocounter{page}{-1}
\begin{center}
{\LARGE\bfseries{}childdoc example\par}
\vspace{1cm}
\ifchilddoc
\ifchilddocmanual part\else chapter\fi:
`\childdocname' of `\childdocjob'\par
\else
main document: `\childdocjob'\par
\fi
version: \version\par
\end{center}
\newpage
%    \end{macrocode}

% Manually include selected file,
% otherwise process as usual:
%    \begin{macrocode}
\ifchilddocmanual
\section*{part `\childdocname'}
\input{\childdocname}
\else
%    \end{macrocode}

% Include the two chapters:
%    \begin{macrocode}
\include{cdocsch1}
\include{cdocsch2}
%    \end{macrocode}

% Include the two parts unless only chapters should be displayed:
%    \begin{macrocode}
\ifchilddoc\else
\section{part three}
\input{cdocspt3}
\section{part four}
\input{cdocspt4}
\fi
%    \end{macrocode}

% Process as usual until here:
%    \begin{macrocode}
\fi
%    \end{macrocode}

% End of document body:
%    \begin{macrocode}
\end{document}
%    \end{macrocode}
%\iffalse
%</samplemain>
%\fi
%
% %%%%%%%%%%%%%%%%%%%%%%%%%%%%%%%%%%%%%%
% \paragraph{Chapter Include Files.}
%
% The include files are called |cdocsch1.tex| and |cdocsch2.tex|.
%
%\iffalse
%<*samplechap1|samplechap2>
%\fi

% Optional override for |\version| flag:
%    \begin{macrocode}
%%\providecommand{\version}{final}
%    \end{macrocode}

% Include the main document:
%    \begin{macrocode}
\input{childdoc.def}
\childdocof{cdocsamp}
%    \end{macrocode}

%\iffalse
%</samplechap1|samplechap2>
%\fi
%
%\iffalse
%<*samplechap1>
%\fi
% Some text for chapter 1:
%    \begin{macrocode}
\section{one}
some text in chapter one
%    \end{macrocode}

%\iffalse
%</samplechap1>
%\fi
% Some text for chapter 2:
%\iffalse
%<*samplechap2>
%\fi
%    \begin{macrocode}
\section{two}
more text in chapter two
%    \end{macrocode}

%\iffalse
%</samplechap2>
%\fi
%
% %%%%%%%%%%%%%%%%%%%%%%%%%%%%%%%%%%%%%%
% \paragraph{Part Include Files.}
%
% The include files are called |cdocspt3.tex| and |cdocspt4.tex|.
%
%\iffalse
%<*samplepart3|samplepart4>
%\fi

% Optional override for |\version| flag:
%    \begin{macrocode}
%%\providecommand{\version}{final}
%    \end{macrocode}

% Include the main document:
%    \begin{macrocode}
\input{childdoc.def}
\childdocby{cdocsamp}
%    \end{macrocode}

%\iffalse
%</samplepart3|samplepart4>
%\fi
%
%\iffalse
%<*samplepart3>
%\fi
% Some text for part 3:
%    \begin{macrocode}
some text in part three
%    \end{macrocode}

%\iffalse
%</samplepart3>
%\fi
% Some text for part 4:
%\iffalse
%<*samplepart4>
%\fi
%    \begin{macrocode}
more text in part four
%    \end{macrocode}

%\iffalse
%</samplepart4>
%\fi
%
% %%%%%%%%%%%%%%%%%%%%%%%%%%%%%%%%%%%%%%
% \paragraph{Forwarding for a Complete Draft.}
%
% The following forwarding file |cdocsdrf.tex|
% compiles the main document in draft mode:
%\iffalse
%<*sampledraft>
%\fi
%    \begin{macrocode}
\def\version{draft}
\input{childdoc.def}
\childdocforward{cdocsamp}
%    \end{macrocode}

%\iffalse
%</sampledraft>
%\fi
%
% %%%%%%%%%%%%%%%%%%%%%%%%%%%%%%%%%%%%%%
% \paragraph{Forwarding for Final Version of the Chapters.}
%
% The following forwarding files |cdocsfn1.tex| and |cdocsfn2.tex|
% (with identical content)
% compile the final versions of the child documents
% |cdocsch1.tex| and |cdocsch2.tex|, respectively:
%\iffalse
%<*samplefinal>
%\fi
%    \begin{macrocode}
\def\version{final}
\input{childdoc.def}
\childdocforwardprefix[cdocsamp]{cdocsfn}{cdocsch}
%    \end{macrocode}

%\iffalse
%</samplefinal>
%\fi
%
% %%%%%%%%%%%%%%%%%%%%%%%%%%%%%%%%%%%%%%
% \paragraph{Command Line Processing.}
%
% The following three command lines generate the output files
% |cdocscld|, |cdocscl1| and |cdocscl2|
% which should be identical to
% |cdocsdrf|, |cdocsch1| and |cdocsfn2|, respectively:
% \begin{center}
% \begin{tabular}{l}
% |latex -jobname cdocscld \|\\
% |  "\def\version{draft}\input{childdoc.def}\childdocforward{cdocsamp}"|\\
% |latex -jobname cdocscl1 \|\\
% |  "\input{childdoc.def}\childdocforward[cdocsamp]{cdocsch1}"|\\
% |latex -jobname cdocscl2 \|\\
% |  "\def\version{final}\input{childdoc.def}\childdocforward{cdocsch2}"|
% \end{tabular}
% \end{center}
% Note that the trailing backslash on each first line
% merely continues the input to the second line
% (for convenient cut ant paste).
% Furthermore, the command |latex| can be replaced by any
% of its alternative versions such as |pdflatex|.
%
% %%%%%%%%%%%%%%%%%%%%%%%%%%%%%%%%%%%%%%%%%%%%%%%%%%%%%%%%%%%%%%%%%%%%%%%%%%%%%%
% %%%%%%%%%%%%%%%%%%%%%%%%%%%%%%%%%%%%%%%%%%%%%%%%%%%%%%%%%%%%%%%%%%%%%%%%%%%%%%
% \section{Implementation}
%\iffalse
%<*package>
%\fi
%
% This section describes the definitions file |childdoc.def|.

% The definitions cannot be loaded using |\usepackage| or |\RequirePackage|
% which has a mechanism to prevent loading a style file more than once.
% When loading the definitions by means of |\input|
% multiple instances have to be prevented manually:
%\iffalse
%This code needs to be before the `\ProvidesFile' directive
%which is defined at the beginning of this file.
%Therefore it is also placed there and commented out here.
%</package>
%<*discard>
%\fi
%    \begin{macrocode}
\ifdefined\childdocmain\endinput\fi
%    \end{macrocode}
%\iffalse
%</discard>
%<*package>
%\fi
%
% \macro{\ifchilddoc}
% \macro{\ifchilddocmanual}
% The conditional |\ifchilddoc| tells whether a
% child (true) or main (false) document is being compiled.
% The conditional |\ifchilddocmanual| tells whether
% the |\includeonly| mechanism is used (false) or
% the selection of child files must be performed manually (true).
% The definitions initialise to false:
%    \begin{macrocode}
\newif\ifchilddoc
\newif\ifchilddocmanual
%    \end{macrocode}

% \macro{\childdocname}
% \macro{\childdocjob}
% The macro |\childdocname| stores the name of the main document
% to be compiled. The macro |\childdocjob| stores the name of
% the document on which the \LaTeX{} compiler was originally invoked.
% The content of |\jobname| cannot be compared
% to filenames specified in the source due to different catcodes.
% The following code rescans |\jobname|, stores the result
% in |\childdocname| and saves a copy in |\childdocjob|:
%    \begin{macrocode}
\edef\childdocname{\scantokens\expandafter{\jobname\noexpand}}
\let\childdocjob\childdocname
%    \end{macrocode}

% \macro{\childdocdisable}
% The macro |\childdocdisable| prevents the main file
% from being processed more than once.
% At this stage, the main document command |\childdocmain|
% is assumed to be called once again where it should do nothing.
% Any subsequent call to it should prevent
% a secondary processing of the main document
% It overwrites the forwarding commands
% |\childdocof| and |\childdocforward|
% with empty macros to prevent further inclusions of the main document:
%    \begin{macrocode}
\newcommand{\childdocdisable}
{
  \renewcommand{\childdocmain}[1]{\renewcommand{\childdocmain}[1]{\endinput}}
  \renewcommand{\childdocof}[1]{}
  \renewcommand{\childdocby}[2][]{}
  \renewcommand{\childdocforward}[2][]{}
  \renewcommand{\childdocdisable}{}
}
%    \end{macrocode}

% \macro{\childdocmain}
% The macro |\childdocmain| is to be called at the top of the main file
% with nothing or the main filename (without extension) as argument.
% First, it breaks loops.
% If the argument is not empty and does not match |\childdocname|
% (which is set by the first inclusion of |childdoc.def|),
% |\ifchilddoc| is set to true, |\includeonly| is applied to the child file
% and |\jobname| is set to the main file
% (for proper handling of |.aux| files):
%    \begin{macrocode}
\newcommand{\childdocmain}[1]
{
  \childdocdisable\childdocmain{}
  \if?#1?\else
    \begingroup
      \def\childdoctmp{#1}
      \ifx\childdoctmp\childdocname
        \def\childdoctmp{}
      \else
        \def\childdoctmp
        {
          \childdoctrue
          \includeonly{\childdocname}
          \def\childdocjob{#1}
          \def\jobname{#1}
        }
      \fi
      \expandafter
    \endgroup
    \childdoctmp
  \fi
}
%    \end{macrocode}

% \macro{\childdocof}
% The command |\childdocof| redirects
% compilation to the main file |#1|.
%    \begin{macrocode}
\newcommand{\childdocof}[1]
{
  \childdocdisable
  \childdoctrue
  \includeonly{\childdocname}
  \def\jobname{#1}
  \def\childdocjob{#1}
  \input{#1}
}
%    \end{macrocode}

% \macro{\childdocby}
% The command |\childdocby| ....
%    \begin{macrocode}
\newcommand{\childdocby}[2][]
{
  \childdocdisable
  \childdoctrue
  \childdocmanualtrue
  \if?#1?\else
    \def\jobname{#2}
  \fi
  \def\childdocjob{#2}
  \input{#2}
  \endinput
}
%    \end{macrocode}

% \macro{\childdocforward}
% The command |\childdocforward| redirects
% compilation to the main file or
% (if the optional argument is given) a child file.
% Parameters are set as if the main file
% or a child file starting with |\childdocof| was compiled.
% Then compilation is handed over to the main file:
%    \begin{macrocode}
\newcommand{\childdocforward}[2][]
{
  \begingroup
    \if?#1?
      \def\childdoctmp
      {
        \def\childdocname{#2}
        \def\childdocjob{#2}
        \def\jobname{#2}
        \input{#2}
        \endinput
      }
    \else
      \def\childdoctmp
      {
        \childdocdisable
        \def\childdocname{#2}
        \childdoctrue
        \includeonly{#2}
        \def\childdocjob{#1}
        \def\jobname{#1}
        \input{#1}
        \endinput
      }
    \fi
    \expandafter
  \endgroup
  \childdoctmp
}
%    \end{macrocode}

% \macro{\childdocforwardprefix}
% The command |\childdocforwardprefix| redirects
% compilation to the main or a child file by means of a pattern.
% The prefix |#1| in the current filename is replaced by |#2|
% and the suffix of the current filename is kept
% (it is assumed that the filename does not contain the substring `|~~~|'
% which is used as a delimiter).
% Compilation is handed over to the new file by |\childdocforward|:
%    \begin{macrocode}
\newcommand{\childdocforwardprefix}[3][]
{
  \begingroup
    \def\childdocextract #2##1~~~{\def\childdoctmp{\childdocforward[#1]{#3##1}}}
    \expandafter\childdocextract\childdocname~~~
    \expandafter
  \endgroup
  \childdoctmp
}
%    \end{macrocode}

% \macro{\childdoc}
% The deprecated macro |\childdoc| is a legacy version of |\childdocmain|:
%    \begin{macrocode}
\newcommand{\childdoc}{\childdocmain}
%    \end{macrocode}

% \macro{\childdocredirect}
% The deprecated macro |\childdocredirect| is a legacy version
% of |\childdocforward| and |\childdocforwardprefix|:
%    \begin{macrocode}
\newcommand{\childdocredirect}[2][]
{
  \begingroup
    \if?#1?
      \def\childdoctmp{\childdocforward{#2}}
    \else
      \def\childdoctmp{\childdocforwardprefix{#1}{#2}}
    \fi
    \expandafter
  \endgroup
  \childdoctmp
}
%    \end{macrocode}

%\iffalse
%</package>
%\fi
%
\endinput
|\\
|\childdocforwardprefix{final}{child}|
\end{tabular}
\end{center}
%

Note that when several versions of a main file and/or of each child file
are to be generated, it may be convenient to set up a |Makefile| or
shell script to automatise the process.

%%%%%%%%%%%%%%%%%%%%%%%%%%%%%%%%%%%%%%%%%%%%%%%%%%%%%%%%%%%%%%%%%%%%%%%%%%%%%%%%
\subsection{Command Line Processing}
\label{sec:commandline}

The effect of redirection files can also be achieved by invoking
the \LaTeX{} compiler with a more elaborate command line.
Most conveniently this should be done as part
of a shell script or a |Makefile|.

When using \textsf{childdoc} in the main file, the following
command lines effectively perform a redirection
(note that depending on the shell being used,
backslashes may have to be doubled: `|\|' $\to$ `|\\|'):
%
\begin{center}
|... -jobname "|\textit{target}|" |\\|"|[\textit{flags}]%
|% \iffalse
%
% childdoc.dtx Copyright (C) 2017-2018 Niklas Beisert
%
% This work may be distributed and/or modified under the
% conditions of the LaTeX Project Public License, either version 1.3
% of this license or (at your option) any later version.
% The latest version of this license is in
%   http://www.latex-project.org/lppl.txt
% and version 1.3 or later is part of all distributions of LaTeX
% version 2005/12/01 or later.
%
% This work has the LPPL maintenance status `maintained'.
%
% The Current Maintainer of this work is Niklas Beisert.
%
% This work consists of the files childdoc.dtx and childdoc.ins
% and the derived files childdoc.def and cdocsamp.tex with
% cdocsch1.tex, cdocsch2.tex, cdocsdrf.tex, cdocsfn1.tex, cdocsfn2.tex.
%
%<package>\ifdefined\childdocmain\endinput\fi
%<package>\ProvidesFile{childdoc.def}[2018/12/30 v2.0 child document driver]
%<samplemain>\ProvidesFile{cdocsamp.tex}[2018/12/30 v2.0 sample for childdoc]
%<*driver>
%\ProvidesFile{childdoc.drv}[2018/12/30 v2.0 childdoc reference manual file]
\PassOptionsToClass{10pt,a4paper}{article}
\documentclass{ltxdoc}

\usepackage[margin=35mm]{geometry}
\usepackage{hyperref}
\usepackage{hyperxmp}
\usepackage[usenames]{color}

\hypersetup{colorlinks=true}
\hypersetup{pdfstartview=FitH}
\hypersetup{pdfpagemode=UseNone}
\hypersetup{pdfsource={}}
\hypersetup{pdflang={en-UK}}
\hypersetup{pdfcopyright={Copyright 2017-2018 Niklas Beisert.
  This work may be distributed and/or modified under the
  conditions of the LaTeX Project Public License, either version 1.3
  of this license or (at your option) any later version.}}
\hypersetup{pdflicenseurl={http://www.latex-project.org/lppl.txt}}
\hypersetup{pdfcontactaddress={ETH Zurich, ITP, HIT K,
  Wolfgang-Pauli-Strasse 27}}
\hypersetup{pdfcontactpostcode={8093}}
\hypersetup{pdfcontactcity={Zurich}}
\hypersetup{pdfcontactcountry={Switzerland}}
\hypersetup{pdfcontactemail={nbeisert@itp.phys.ethz.ch}}
\hypersetup{pdfcontacturl={http://people.phys.ethz.ch/\xmptilde nbeisert/}}

\newcommand{\secref}[1]{\hyperref[#1]{section \ref*{#1}}}

\parskip1ex
\parindent0pt
\let\olditemize\itemize
\def\itemize{\olditemize\parskip0pt}

\begin{document}

\title{The \textsf{childdoc} Package}
\hypersetup{pdftitle={The childdoc Package}}
\author{Niklas Beisert\\[2ex]
  Institut f\"ur Theoretische Physik\\
  Eidgen\"ossische Technische Hochschule Z\"urich\\
  Wolfgang-Pauli-Strasse 27, 8093 Z\"urich, Switzerland\\[1ex]
  \href{mailto:nbeisert@itp.phys.ethz.ch}
  {\texttt{nbeisert@itp.phys.ethz.ch}}}
\hypersetup{pdfauthor={Niklas Beisert}}
\hypersetup{pdfsubject={Manual for the LaTeX2e Package childdoc}}
\date{30 December 2018, \textsf{v2.0}}
\maketitle

\begin{abstract}\noindent
\textsf{childdoc} is a \LaTeXe{} package
that enables the direct compilation
of document sections included by |\include|
to individual files.
\end{abstract}

\begingroup
\parskip0ex
\tableofcontents
\endgroup

%%%%%%%%%%%%%%%%%%%%%%%%%%%%%%%%%%%%%%%%%%%%%%%%%%%%%%%%%%%%%%%%%%%%%%%%%%%%%%%%
%%%%%%%%%%%%%%%%%%%%%%%%%%%%%%%%%%%%%%%%%%%%%%%%%%%%%%%%%%%%%%%%%%%%%%%%%%%%%%%%
\section{Introduction}

\LaTeX{} provides a mechanism to structure a large document (such as a book)
into a main file and several child files (containing the chapters)
using the |\include| command.
This mechanism is beneficial for documents
which span hundreds of pages in order to
make the source file(s) more manageable.
Moreover, compilation can be restricted to
selected child files by means of the |\includeonly| command.
The latter feature can be used to reduce the compilation time while editing
(this was significantly more useful in the earlier days of \LaTeX{})
or to generate a smaller document which is easier to navigate.
Another application of |\includeonly| is to generate
documents consisting of selected parts of the complete document.

However, there are a few drawbacks of the plain |\include| mechanism:
\begin{itemize}
\item
The child files cannot be compiled on their own,
they can only be compiled via the main file.
A naive editing environment
(such as a text editor with an option
to have the current file processed by \LaTeX)
may require one to switch to the main file before compiling;
attempting to compile the child file produces errors.
\item
The main file must be modified (each time)
to adjust the |\includeonly| command
to the present needs. This easily leaves the main file in a messy state.
\item
The generated document will always carry the filename
of the main document. This is inconvenient if
several child files are to be compiled and
to be kept for distribution.
\end{itemize}

The present package provides a simple interface
to make child files individually compilable by \LaTeX{}.
Compiling a child file then has the same effect as compiling
the main file with an |\includeonly| command
to select the appropriate child.
Moreover the generated document will carry the name of the child
rather than the main file.
This resolves all three above issues.

This feature is meant to make the editing of books,
thesis documents and lecture notes somewhat more convenient.
However, the package can also be used efficiently for
composing a series of documents (such as exercise sheets)
which are typically distributed individually.
It then assists the author in generating the individual documents
(potentially in different versions)
as well as a document containing the collected series.
Another application is in developing style files
or other kinds of included material
where compilation of the style file could redirect
to a sample or test file.

%%%%%%%%%%%%%%%%%%%%%%%%%%%%%%%%%%%%%%%%%%%%%%%%%%%%%%%%%%%%%%%%%%%%%%%%%%%%%%%%
%%%%%%%%%%%%%%%%%%%%%%%%%%%%%%%%%%%%%%%%%%%%%%%%%%%%%%%%%%%%%%%%%%%%%%%%%%%%%%%%
\section{Usage}

First of all, the package \textsf{childdoc} is \emph{not} a standard
\LaTeXe{} |.sty| style file! Therefore it needs to be invoked in
a non-standard way.

%%%%%%%%%%%%%%%%%%%%%%%%%%%%%%%%%%%%%%%%%%%%%%%%%%%%%%%%%%%%%%%%%%%%%%%%%%%%%%%%
\subsection{Included Files}
\label{sec:include}

%%%%%%%%%%%%%%%%%%%%%%%%%%%%%%%%%%%%%%%%
\DescribeMacro{\childdocmain}
To use the package, add the commands
\begin{center}
\begin{tabular}{l}
|\input{childdoc.def}|\\
|\childdocmain{}|\\
\end{tabular}
\end{center}
at the very top of the main \LaTeX{} file,
in particular \emph{before} the |\documentclass| statement!
The argument of |\childdocmain| should be left empty
(but it must be present).

%%%%%%%%%%%%%%%%%%%%%%%%%%%%%%%%%%%%%%%%
\DescribeMacro{\childdocof}
Furthermore, add the commands
\begin{center}
\begin{tabular}{l}
|\input{childdoc.def}|\\
|\childdocof{|\textit{main}|}|\\
\end{tabular}
\end{center}
at the top of every child file \textit{child}
which is included by |\include{|\textit{child}|}|
from within the main file
(or at least for those files to be compiled individually).
The argument \textit{main} must be the filename of the main file.

There are a couple of
considerations in setting up the main and child documents:

%%%%%%%%%%%%%%%%%%%%%%%%%%%%%%%%%%%%%%%%
\paragraph{Restrictions.}

Please note the following restrictions:
\begin{itemize}
\item
|\childdocmain| must be called with one argument \textit{main}
to ensure compatibility with earlier version of the package.
It must either be empty (|\childdocmain{}|)
or precisely match the filename of the main file in which it is specified.
See \secref{sec:detection} for further information.
\item
The filename \textit{main} must be specified without the |.tex| extension.
\item
The filename \textit{main} is case sensitive
(even in case-insensitive file systems)
due to internal string comparison.
\item
The argument \textit{main} should be fully expanded, it cannot be a macro.
\item
Subdirectories and special characters should be avoided in filenames.
\item
The command |\childdocmain{|\textit{main}|}| must be followed by a whitespace.
It should not be followed immediately by another command
or by a comment mark `|%|'.
This is because the \TeX{} parser reads the token immediately following
the argument of |\childdocmain| and puts it
at the beginning of every child section;
however, a white\-space is ignored.
\end{itemize}

%%%%%%%%%%%%%%%%%%%%%%%%%%%%%%%%%%%%%%%%
\paragraph{Content of Main File.}

It is advisable to place all content in the child files included by |\include|.
Any output contained in the main file will appear in all child documents
unless suppressed manually;
it cannot be suppressed automatically by the |\includeonly| directive
and thus should normally be avoided.
A method to include some content in the main file
by means of conditional processing is described in \secref{sec:conditional}.

%%%%%%%%%%%%%%%%%%%%%%%%%%%%%%%%%%%%%%%%
\paragraph{Page Numbering.}

When only a part of the document is compiled,
the appropriate numbering of pages
(as well as other status parameters)
is determined from the |.aux| files.
The latter contain information from previous passes.
However this information needs to propagate through
all intermediate child documents.
Therefore the page numbering in child documents may well
be inconsistent until the complete document is compiled at least once.

A useful (if unconventional) way to always ensure a consistent
page numbering is to restart the numbering in each child document
and denote the pages by `\textit{child}|.|\textit{page}'
where \textit{child} represents the chapter/section number of the child file.
This can be achieved by the command
|\numberwithin{page}{|\textit{child}|}|
of the \textsf{amsmath} package
where \textit{child} can be |chapter| or |section|
depending on the chosen structuring.
Alternatively, one can modify the macro |\thepage| appropriately
and reset the counter |page| at the start of each child file.

%%%%%%%%%%%%%%%%%%%%%%%%%%%%%%%%%%%%%%%%%%%%%%%%%%%%%%%%%%%%%%%%%%%%%%%%%%%%%%%%
\subsection{Conditional Processing}
\label{sec:conditional}

The package provides a mechanism to compile different versions
of a document. To customise the versions further some conditional processing
can come in handy to distinguish which version is being compiled.
The package provides two macros to describe the compilation context:

%%%%%%%%%%%%%%%%%%%%%%%%%%%%%%%%%%%%%%%%
\DescribeMacro{\ifchilddoc}
The conditional |\ifchilddoc| distinguishes between the compilation of
child documents and the main document:
%
\begin{center}
|\ifchilddoc |\textit{child-code}| |[|\||else |\textit{main-code}]| \||fi|
\end{center}

%%%%%%%%%%%%%%%%%%%%%%%%%%%%%%%%%%%%%%%%
\DescribeMacro{\childdocname}
\DescribeMacro{\childdocjob}
The macro |\childdocname| contains the filename (without extension)
of the main or child file being processed.
Note that |\childdocjob| will always contain the name of the main file.

%%%%%%%%%%%%%%%%%%%%%%%%%%%%%%%%%%%%%%%%
\paragraph{Title Page.}

Conditional processing can be used to include a title or banner page
in the main document when proper precautions are taken.
Importantly, the code in the main file should ensure that the page counter
(as well as other status parameters which are stored in the |.aux| files)
takes the same value after the conditional processing.
Otherwise the page numbers may take divergent values
depending on which part is compiled.

For example, a title page could be declared by:
%
\begin{center}
\begin{tabular}{l}
|\ifchilddoc\||else|\\
|\addtocounter{page}{-1}|\\
\textit{code for title page}\\
|\newpage|\\
|\||fi|
\end{tabular}
\end{center}
%
A banner page for the child documents can be generated by:
%
\begin{center}
\begin{tabular}{l}
|\ifchilddoc|\\
|\addtocounter{page}{-1}|\\
\textit{code for banner page}\\
|\newpage|\\
|\||fi|
\end{tabular}
\end{center}
%
Here one could write a message such as:
\begin{center}
|This is the part \childdocname{} of \childdocjob{}.|
\end{center}

%%%%%%%%%%%%%%%%%%%%%%%%%%%%%%%%%%%%%%%%%%%%%%%%%%%%%%%%%%%%%%%%%%%%%%%%%%%%%%%%
\subsection{Flags}
\label{sec:flags}

The package makes it easy to generate different versions
of the main or child documents.
To this end compilation flags can be defined
and assigned different default values.
They will be particularly useful in conjunction
with the forwarding mechanism described in \secref{sec:forward}.

For example, it may be useful to have a flag |\version|
which can be set to |draft| or |final|.
The document source will contain some conditional code
depending on the value of |\version|.
Suppose further, the flag should default to |final| for the main file
and to |draft| for child files
which is a natural assignment for editing the document.
This is achieved by placing the following code
in the preamble of the main document
(below the |\childdocmain| directive):
%
\begin{center}
\begin{tabular}{l}
|\ifchilddoc|\\
|\providecommand{\version}{draft}|\\
|\||else|\\
|\providecommand{\version}{final}|\\
|\||fi|
\end{tabular}
\end{center}
%
The definition by |\providecommand| makes sure
that previous definitions are not overwritten.
Further statements |\providecommand{\version}{...}|
can thus be added before the above code to override it.

For the main file, one might add a line
(between |\childdocmain| and the above block)
%
\begin{center}
|%\ifchilddoc\||else\providecommand{\version}{draft}\||fi|
\end{center}
%
which can be uncommented to produce a draft version.
Likewise one can add a line to the very top of a child file
(above the |\childdocof{|\textit{main}|}| directive)
%
\begin{center}
|%\providecommand{\version}{final}|
\end{center}
%
which can be uncommented to produce the final version of this child document.

%%%%%%%%%%%%%%%%%%%%%%%%%%%%%%%%%%%%%%%%%%%%%%%%%%%%%%%%%%%%%%%%%%%%%%%%%%%%%%%%
\subsection{Forwarding}
\label{sec:forward}

Different versions of the main or child documents
using compilation flags as described in \secref{sec:flags}
can be (permanently) stored in different files
for convenient compilation, viewing and distribution.
To this end, the package defines a command
to pass on compilation to a different file:

%%%%%%%%%%%%%%%%%%%%%%%%%%%%%%%%%%%%%%%%
\DescribeMacro{\childdocforward}
The command |\childdocforward| redirects processing to
another source file:
%
\begin{center}
\begin{tabular}{l}
|\input{childdoc.def}|\\
|\childdocforward[|\textit{main}|]{|\textit{dest}|}|\\
\end{tabular}
\end{center}
%
The argument \textit{dest} is the destination file
(without extension).
It should be the main file or one of the child files.
Note that further \textsf{childdoc} directives
such as |\childdocof| and |\childdocforward|
in the indicated file will be processed in this form.
The optional argument \textit{main}
passes on directly to the main file \textit{main}
while pretending to compile the child \textit{dest}.
This form behaves as if \textit{dest}
issues |\childdocof{|\textit{main}|}| right away,
and no further \textsf{childdoc} directives will be processed.

%%%%%%%%%%%%%%%%%%%%%%%%%%%%%%%%%%%%%%%%
\DescribeMacro{\...prefix}
In the alternative form |\childdocforwardprefix|,
%
\begin{center}
\begin{tabular}{l}
|\input{childdoc.def}|\\
|\childdocforwardprefix[|\textit{main}|]{|\textit{prefix}|}{|\textit{dest}|}|
\end{tabular}
\end{center}
%
the destination file is determined by a pattern
depending on the current file:
To make this work, the current file must be called
`{\textit{prefix}\hspace{0.2em}\textit{suffix}}'
with \textit{prefix} matching precisely the argument.
Processing is then passed on to the file
`{\textit{dest}\hspace{0.2em}\textit{suffix}}'.
Surely, the same effect is achieved by
directly specifying the
argument `{\textit{dest}\hspace{0.2em}\textit{suffix}}'
in the first form.
However, that requires to set up a different file
for each child. With the alternative form of the command
all these files can have exactly the same content
which simplifies setting them up and maintaining them.

For example, the following file |draft.tex|
with a compilation flag |\version| as described in \secref{sec:flags}
compiles the main document as a draft:
%
\begin{center}
\begin{tabular}{l}
|\def\version{draft}|\\
|\input{childdoc.def}|\\
|\childdocforward{|\textit{main}|}|
\end{tabular}
\end{center}
%
Likewise, the following files |final|\textit{nn}|.tex|
compile the final version of the child document
|child|\textit{nn}|.tex|:
%
\begin{center}
\begin{tabular}{l}
|\def\version{final}|\\
|\input{childdoc.def}|\\
|\childdocforwardprefix{final}{child}|
\end{tabular}
\end{center}
%

Note that when several versions of a main file and/or of each child file
are to be generated, it may be convenient to set up a |Makefile| or
shell script to automatise the process.

%%%%%%%%%%%%%%%%%%%%%%%%%%%%%%%%%%%%%%%%%%%%%%%%%%%%%%%%%%%%%%%%%%%%%%%%%%%%%%%%
\subsection{Command Line Processing}
\label{sec:commandline}

The effect of redirection files can also be achieved by invoking
the \LaTeX{} compiler with a more elaborate command line.
Most conveniently this should be done as part
of a shell script or a |Makefile|.

When using \textsf{childdoc} in the main file, the following
command lines effectively perform a redirection
(note that depending on the shell being used,
backslashes may have to be doubled: `|\|' $\to$ `|\\|'):
%
\begin{center}
|... -jobname "|\textit{target}|" |\\|"|[\textit{flags}]%
|\input{childdoc.def}\childdocforward[|\textit{main}|]{|\textit{dest}|}"|
\end{center}
%
Here \textit{target} is the name of the output file,
\textit{main} is the name of the main file
and \textit{dest} is the name of the main or child file to be processed
(all filenames without extensions).
The optional argument \textit{main} can be omitted
if \textit{main} matches \textit{dest}.
Optionally, compilation \textit{flags} can be defined via |\def| commands.
This command line makes the \TeX{} engine believe
it is compiling the file \textit{target}
whose content is specified as the latter parameter.
The provided code then forwards the processing to
\textit{main} or \textit{dest} as described in \secref{sec:forward}.

%%%%%%%%%%%%%%%%%%%%%%%%%%%%%%%%%%%%%%%%%%%%%%%%%%%%%%%%%%%%%%%%%%%%%%%%%%%%%%%%
\subsection{Include by Input}
\label{sec:input}

Including child documents by |\include| has some restrictions by design.
Most notably, the content of a child document always occupies
its own set of pages; pages cannot be shared between child documents.
Usually, this behaviour makes perfect sense
because each child document contain an essential part of the document.
However, in some situations it may be desirable to compose
a document from a collection of parts
without having mandatory page breaks between then.
For this case, the package
provides a mechanism to include parts
by |\input| which can also be processed individually.
However, by construction this mechanism
requires manual handling of the content to be output.

%%%%%%%%%%%%%%%%%%%%%%%%%%%%%%%%%%%%%%%%
\DescribeMacro{\ifchilddocmanual}
The main file should be prepared as usual, see \secref{sec:include}.
However, the document body must make a distinction
between processing of an individual part and of the main document, e.g.:
%
\begin{center}
\begin{tabular}{l}
|\ifchilddocmanual|\\
|\input{\childdocname}|\\
|\||else|\\
\textit{document body with }|\input{|\textit{part}|}|\\
|\||fi|
\end{tabular}
\end{center}
%
The conditional |\ifchilddocmanual| is true whenever
a part to be included by |\input| is being compiled,
and the name of the part is stored in |\childdocname|.

%%%%%%%%%%%%%%%%%%%%%%%%%%%%%%%%%%%%%%%%
\DescribeMacro{\childdocby}
Each part to be included by |\input| should start with:
%
\begin{center}
\begin{tabular}{l}
|\input{childdoc.def}|\\
|\childdocby{|\textit{main}|}|\\
\end{tabular}
\end{center}
%
The directive |\childdocby| is similar to |\childdocof|
described in \secref{sec:include},
but the subsequent selection of content must be done manually.
To that end, both |\ifchilddoc| and |\ifchilddocmanual|
will be true upon processing of a part,
and the name of the part is stored in |\childdocname|.
Note that |\jobname| will be set to the filename of the current part
so that each part receives an individual |.aux| file
that does not interfere with the |.aux| file(s) of the main document.
This behaviour can be altered by the alternative form
|\childdocby[*]{|\textit{main}|}| (with a non-empty optional argument)
which uses the |.aux| file of the main document
by setting |\jobname| to \textit{main}.

%%%%%%%%%%%%%%%%%%%%%%%%%%%%%%%%%%%%%%%%%%%%%%%%%%%%%%%%%%%%%%%%%%%%%%%%%%%%%%%%
\subsection{Driver Development}
\label{sec:driver}

The \textsf{childdoc} mechanism can also be use for the development
of definition files such as \LaTeX{} styles or classes.
This case differs from the above setup with multiple parts
included by |\include| in that no |\includeonly| should be invoked.
This can be achieved by starting the include file
(before |\ProvidesPackage|) with:
%
\begin{center}
\begin{tabular}{l}
|\input{childdoc.def}|\\
|\childdocforward{|\textit{main}|}|\\
\end{tabular}
\end{center}
%
or alternatively with:
%
\begin{center}
\begin{tabular}{l}
|\input{childdoc.def}|\\
|\childdocby{|\textit{main}|}|\\
\end{tabular}
\end{center}
%
Both forms have slightly different effects as described above.
The main file is prepared as usual, see \secref{sec:include}.

%%%%%%%%%%%%%%%%%%%%%%%%%%%%%%%%%%%%%%%%%%%%%%%%%%%%%%%%%%%%%%%%%%%%%%%%%%%%%%%%
\subsection{Legacy Detection}
\label{sec:detection}

The directive |\childdocmain| in the main file can detect
whether the complete document or merely a child is to be compiled
even without using the directive |\childdocof|.
This method is deprecated because it is less robust
and there is no compelling reason to use it;
it is merely provided for backward compatibility
and it may be removed in future versions.

If the detection mechanism is to be used,
it is mandatory to correctly specify
the filename of the main file as the argument of |\childdocmain|:
%
\begin{center}
\begin{tabular}{l}
|\input{childdoc.def}|\\
|\childdocmain{|\textit{main}|}|\\
\end{tabular}
\end{center}
%
If |\jobname| does not match the argument \textit{main} of |\childdocmain|,
it is assumed that |\jobname| points to the child file to be compiled.
When using |\childdocmain| with the main file specified as argument,
it suffices to start a child file
with just |\input{|\textit{main}|}|
without loading of the package and using |\childdocof|.
If instead all processing is done
with the appropriate \textsf{childdoc} directives,
the argument of \textit{main} of |\childdocmain| can be empty.

An alternative version of the command line processing described
in \secref{sec:commandline} using the detection mechanism reads:
%
\begin{center}
|... -jobname "|\textit{target}|" "|[\textit{flags}]%
[|\def\jobname{|\textit{dest}|}|]|\input{|\textit{main}|}"|
\end{center}

%%%%%%%%%%%%%%%%%%%%%%%%%%%%%%%%%%%%%%%%%%%%%%%%%%%%%%%%%%%%%%%%%%%%%%%%%%%%%%%%
\subsection{Manual Code}
\label{sec:manual}

In case one cannot be certain whether the definitions file |childdoc.def|
is installed on the target \TeX{} distribution
and one prefers not to ship it,
it is conceivable to paste a few relevant commands into the sources.

To that end, drop all statements |\input{childdoc.def}|
and perform the replacements as outlined below.
Instead of |\childdocmain{|\textit{main}|}| add the following code
to the top of the main file:
%
\begin{center}
\begin{tabular}{l}
|\||ifdefined\childdocname\endinput\||fi\newif\ifchilddoc|\\
|\edef\childdocname{\scantokens\expandafter{\jobname\noexpand}}|\\
|\def\childdocmain{|\textit{main}|}\||ifx\childdocmain\childdocname\||else|\\
|\childdoctrue\includeonly{\childdocname}\let\jobname\childdocmain\||fi|\\
\end{tabular}
\end{center}
%
Instead of |\childdocof{|\textit{main}|}| just include the main file
at the top of each child file:
%
\begin{center}
|\input{|\textit{main}|}|
\end{center}
%
A simple redirection |\childdocforward{|\textit{dest}|}| is achieved by:
%
\begin{center}
|\def\jobname{|\textit{dest}|}\input{\jobname}|
\end{center}
%
The redirection with prefix
|\childdocforwardprefix[|\textit{prefix}|]{|\textit{dest}|}|
is accomplished by:
%
\begin{center}
\begin{tabular}{l}
|{\edef\jobname{\scantokens\expandafter{\jobname\noexpand}}|\\
|\def\redirectjob |\textit{prefix}|#1~~~{\gdef\jobname{|\textit{dest}|#1}}|\\
|\expandafter\redirectjob\jobname~~~}\input{\jobname}|
\end{tabular}
\end{center}

In an alternative approach,
child documents can be compiled by a specific command line
without additional code or specific definitions:
%
\begin{center}
|... -jobname "|\textit{target}|" "|[\textit{flags}]%
|\includeonly{|\textit{dest}|}\input{|\textit{main}|}"|
\end{center}
%

%%%%%%%%%%%%%%%%%%%%%%%%%%%%%%%%%%%%%%%%%%%%%%%%%%%%%%%%%%%%%%%%%%%%%%%%%%%%%%%%
%%%%%%%%%%%%%%%%%%%%%%%%%%%%%%%%%%%%%%%%%%%%%%%%%%%%%%%%%%%%%%%%%%%%%%%%%%%%%%%%
\section{Information}

%%%%%%%%%%%%%%%%%%%%%%%%%%%%%%%%%%%%%%%%%%%%%%%%%%%%%%%%%%%%%%%%%%%%%%%%%%%%%%%%
\subsection{Copyright}

Copyright \copyright{} 2017--2018 Niklas Beisert

This work may be distributed and/or modified under the
conditions of the \LaTeX{} Project Public License, either version 1.3
of this license or (at your option) any later version.
The latest version of this license is in
  \url{http://www.latex-project.org/lppl.txt}
and version 1.3 or later is part of all distributions of \LaTeX{}
version 2005/12/01 or later.

This work has the LPPL maintenance status `maintained'.

The Current Maintainer of this work is Niklas Beisert.

This work consists of the files |README.txt|, |childdoc.ins| and |childdoc.dtx|
as well as the derived files |childdoc.def|, |cdocsamp.tex|
with |cdocsch1.tex|, |cdocsch2.tex|, |cdocspt3.tex|, |cdocspt4.tex|,
|cdocsdrf.tex|, |cdocsfn1.tex|, |cdocsfn2.tex|
as well as |childdoc.pdf|.

%%%%%%%%%%%%%%%%%%%%%%%%%%%%%%%%%%%%%%%%%%%%%%%%%%%%%%%%%%%%%%%%%%%%%%%%%%%%%%%%
\subsection{Files and Installation}

The package consists of the files:
%
\begin{center}
\begin{tabular}{ll}
    |README.txt|   & readme file \\
    |childdoc.ins| & installation file \\
    |childdoc.dtx| & source file \\
    |childdoc.def| & definition file \\
    |cdocsamp.tex| & sample main file \\
    |cdocsch1.tex| & sample include file \\
    |cdocsch2.tex| & sample include file \\
    |cdocspt3.tex| & sample part file \\
    |cdocspt4.tex| & sample part file \\
    |cdocsdrf.tex| & sample redirection file \\
    |cdocsfn1.tex| & sample redirection file \\
    |cdocsfn2.tex| & sample redirection file \\
    |childdoc.pdf| & manual
\end{tabular}
\end{center}
%
The distribution consists of the files
|README.txt|, |childdoc.ins| and |childdoc.dtx|.
%
\begin{itemize}
\item
Run (pdf)\LaTeX{} on |childdoc.dtx|
to compile the manual |childdoc.pdf| (this file).
\item
Run \LaTeX{} on |childdoc.ins| to create the definitions file |childdoc.def|
and the sample |cdocsamp.tex| with include files
|cdocsch1.tex|, |cdocsch2.tex|, |cdocspt3.tex|, |cdocspt4.tex|,
|cdocsdrf.tex|, |cdocsfn1.tex|, |cdocsfn2.tex|.
Then copy the file |childdoc.def| to an appropriate directory of your \LaTeX{}
distribution, e.g.\ \textit{texmf-root}|/tex/latex/childdoc|.
\end{itemize}

%%%%%%%%%%%%%%%%%%%%%%%%%%%%%%%%%%%%%%%%%%%%%%%%%%%%%%%%%%%%%%%%%%%%%%%%%%%%%%%%
\subsection{Related CTAN Packages}

There are several other packages which offer a similar functionality:
%
\begin{itemize}
\item
The packages
\href{http://ctan.org/pkg/docmute}{\textsf{docmute}},
\href{http://ctan.org/pkg/includex}{\textsf{includex}} and
\href{http://ctan.org/pkg/standalone}{\textsf{standalone}}
provide commands to include only the document body of
a child file thus allowing both files to be compiled individually.
\item
The packages \href{http://ctan.org/pkg/subdocs}{\textsf{subdocs}}
and \href{http://ctan.org/pkg/subfiles}{\textsf{subfiles}}
provide structures in which the main and child documents can be
encapsulated and allowing them to be compiled individually.
The inclusion mechanism is different from the conventional |\include|.
\item
The package \href{http://ctan.org/pkg/combine}{\textsf{combine}}
is an elaborate solution to combine several documents into one.
\end{itemize}
%
See also the CTAN topic \href{http://ctan.org/topic/subdocs}{\textsf{subdocs}}
for further related packages.
The present package differs from the above solutions in that
a document structure constructed with the conventional |\include| mechanism
just needs two extra commands at the top of every file
such that all constituent files can be compiled individually.

%%%%%%%%%%%%%%%%%%%%%%%%%%%%%%%%%%%%%%%%%%%%%%%%%%%%%%%%%%%%%%%%%%%%%%%%%%%%%%%%
%\subsection{Feature Suggestions}
%
%The following is a list of features which may be useful for future
%versions of this package:
%%
%\begin{itemize}
%\item
%\ldots
%\end{itemize}

%%%%%%%%%%%%%%%%%%%%%%%%%%%%%%%%%%%%%%%%%%%%%%%%%%%%%%%%%%%%%%%%%%%%%%%%%%%%%%%%
\subsection{Revision History}

%%%%%%%%%%%%%%%%%%%%%%%%%%%%%%%%%%%%%%%%
\paragraph{v2.0:} 2018/12/30

\begin{itemize}
\item
immediate forward processing
\item
added |\childdocby| mechanism
\item
manual restructured
\end{itemize}

%%%%%%%%%%%%%%%%%%%%%%%%%%%%%%%%%%%%%%%%
\paragraph{v1.6:} 2018/01/17

\begin{itemize}
\item
application for development of include files
\item
corrections to manual
\end{itemize}

%%%%%%%%%%%%%%%%%%%%%%%%%%%%%%%%%%%%%%%%
\paragraph{v1.5:} 2017/05/21

\begin{itemize}
\item
more complete structuring introduced
\item
|\childdocof| introduced
\item
|\childdoc| renamed to |\childdocmain|
\item
|\childredirect| renamed to |\childdocforward| and |\childdocforwardprefix|
and functionality expanded
\end{itemize}

%%%%%%%%%%%%%%%%%%%%%%%%%%%%%%%%%%%%%%%%
\paragraph{v1.0:} 2017/04/27

\begin{itemize}
\item
manual and install package
\item
first version published on CTAN
\end{itemize}

%%%%%%%%%%%%%%%%%%%%%%%%%%%%%%%%%%%%%%%%
\paragraph{v0.6:} 2017/04/26

\begin{itemize}
\item
redirection mechanism added
\end{itemize}

%%%%%%%%%%%%%%%%%%%%%%%%%%%%%%%%%%%%%%%%
\paragraph{v0.5:} 2017/04/26

\begin{itemize}
\item
functionality in definition file
\end{itemize}


%%%%%%%%%%%%%%%%%%%%%%%%%%%%%%%%%%%%%%%%%%%%%%%%%%%%%%%%%%%%%%%%%%%%%%%%%%%%%%%%
%%%%%%%%%%%%%%%%%%%%%%%%%%%%%%%%%%%%%%%%%%%%%%%%%%%%%%%%%%%%%%%%%%%%%%%%%%%%%%%%
%%%%%%%%%%%%%%%%%%%%%%%%%%%%%%%%%%%%%%%%%%%%%%%%%%%%%%%%%%%%%%%%%%%%%%%%%%%%%%%%
\appendix

\settowidth\MacroIndent{\rmfamily\scriptsize 000\ }

 \DocInput{childdoc.dtx}

\end{document}
%</driver>
% \fi
%
% %%%%%%%%%%%%%%%%%%%%%%%%%%%%%%%%%%%%%%%%%%%%%%%%%%%%%%%%%%%%%%%%%%%%%%%%%%%%%%
% %%%%%%%%%%%%%%%%%%%%%%%%%%%%%%%%%%%%%%%%%%%%%%%%%%%%%%%%%%%%%%%%%%%%%%%%%%%%%%
% \section{Sample}
%\iffalse
%<*samplemain>
%\fi
%
% The following presents a sample document
% with two chapters, two parts, a title page,
% a compile flag as well as three forwarding files to set the flag.
% It consists of eight |.tex| files:
% \begin{center}
% \begin{tabular}{ll}
% |cdocsamp.tex|&main file\\
% |cdocsch1.tex|&include file for chapter 1\\
% |cdocsch2.tex|&include file for chapter 2\\
% |cdocspt3.tex|&include file for part 3\\
% |cdocspt4.tex|&include file for part 4\\
% |cdocsdrf.tex|&forwarding file for main file in draft mode\\
% |cdocsfi1.tex|&forwarding file for final version of chapter 1\\
% |cdocsfi2.tex|&forwarding file for final version of chapter 2\\
% \end{tabular}
% \end{center}
% Each of the eight files can be compiled directly by the \LaTeX{} compiler.
%
% %%%%%%%%%%%%%%%%%%%%%%%%%%%%%%%%%%%%%%
% \paragraph{Main File.}
%
% The main file is called |cdocsamp.tex|.
%
% Load the \textsf{childdoc} definitions and
% declare the filename for the main document:
%    \begin{macrocode}
\input{childdoc.def}
\childdocmain{}
%    \end{macrocode}

% Optional override for |\version| flag:
%    \begin{macrocode}
%%\ifchilddoc\else\providecommand{\version}{draft}\fi
%    \end{macrocode}

% Define the default values for the |\version| flag
% (|final| for the main file and |draft| for childs):
%    \begin{macrocode}
\ifchilddoc
\providecommand{\version}{draft}
\else
\providecommand{\version}{final}
\fi
%    \end{macrocode}

% Load the standard document class:
%    \begin{macrocode}
\documentclass[12pt]{article}
%    \end{macrocode}

% Start the document body:
%    \begin{macrocode}
\begin{document}
%    \end{macrocode}

% Declare a title page.
% Print title, part of document being processed and version flag:
%    \begin{macrocode}
\addtocounter{page}{-1}
\begin{center}
{\LARGE\bfseries{}childdoc example\par}
\vspace{1cm}
\ifchilddoc
\ifchilddocmanual part\else chapter\fi:
`\childdocname' of `\childdocjob'\par
\else
main document: `\childdocjob'\par
\fi
version: \version\par
\end{center}
\newpage
%    \end{macrocode}

% Manually include selected file,
% otherwise process as usual:
%    \begin{macrocode}
\ifchilddocmanual
\section*{part `\childdocname'}
\input{\childdocname}
\else
%    \end{macrocode}

% Include the two chapters:
%    \begin{macrocode}
\include{cdocsch1}
\include{cdocsch2}
%    \end{macrocode}

% Include the two parts unless only chapters should be displayed:
%    \begin{macrocode}
\ifchilddoc\else
\section{part three}
\input{cdocspt3}
\section{part four}
\input{cdocspt4}
\fi
%    \end{macrocode}

% Process as usual until here:
%    \begin{macrocode}
\fi
%    \end{macrocode}

% End of document body:
%    \begin{macrocode}
\end{document}
%    \end{macrocode}
%\iffalse
%</samplemain>
%\fi
%
% %%%%%%%%%%%%%%%%%%%%%%%%%%%%%%%%%%%%%%
% \paragraph{Chapter Include Files.}
%
% The include files are called |cdocsch1.tex| and |cdocsch2.tex|.
%
%\iffalse
%<*samplechap1|samplechap2>
%\fi

% Optional override for |\version| flag:
%    \begin{macrocode}
%%\providecommand{\version}{final}
%    \end{macrocode}

% Include the main document:
%    \begin{macrocode}
\input{childdoc.def}
\childdocof{cdocsamp}
%    \end{macrocode}

%\iffalse
%</samplechap1|samplechap2>
%\fi
%
%\iffalse
%<*samplechap1>
%\fi
% Some text for chapter 1:
%    \begin{macrocode}
\section{one}
some text in chapter one
%    \end{macrocode}

%\iffalse
%</samplechap1>
%\fi
% Some text for chapter 2:
%\iffalse
%<*samplechap2>
%\fi
%    \begin{macrocode}
\section{two}
more text in chapter two
%    \end{macrocode}

%\iffalse
%</samplechap2>
%\fi
%
% %%%%%%%%%%%%%%%%%%%%%%%%%%%%%%%%%%%%%%
% \paragraph{Part Include Files.}
%
% The include files are called |cdocspt3.tex| and |cdocspt4.tex|.
%
%\iffalse
%<*samplepart3|samplepart4>
%\fi

% Optional override for |\version| flag:
%    \begin{macrocode}
%%\providecommand{\version}{final}
%    \end{macrocode}

% Include the main document:
%    \begin{macrocode}
\input{childdoc.def}
\childdocby{cdocsamp}
%    \end{macrocode}

%\iffalse
%</samplepart3|samplepart4>
%\fi
%
%\iffalse
%<*samplepart3>
%\fi
% Some text for part 3:
%    \begin{macrocode}
some text in part three
%    \end{macrocode}

%\iffalse
%</samplepart3>
%\fi
% Some text for part 4:
%\iffalse
%<*samplepart4>
%\fi
%    \begin{macrocode}
more text in part four
%    \end{macrocode}

%\iffalse
%</samplepart4>
%\fi
%
% %%%%%%%%%%%%%%%%%%%%%%%%%%%%%%%%%%%%%%
% \paragraph{Forwarding for a Complete Draft.}
%
% The following forwarding file |cdocsdrf.tex|
% compiles the main document in draft mode:
%\iffalse
%<*sampledraft>
%\fi
%    \begin{macrocode}
\def\version{draft}
\input{childdoc.def}
\childdocforward{cdocsamp}
%    \end{macrocode}

%\iffalse
%</sampledraft>
%\fi
%
% %%%%%%%%%%%%%%%%%%%%%%%%%%%%%%%%%%%%%%
% \paragraph{Forwarding for Final Version of the Chapters.}
%
% The following forwarding files |cdocsfn1.tex| and |cdocsfn2.tex|
% (with identical content)
% compile the final versions of the child documents
% |cdocsch1.tex| and |cdocsch2.tex|, respectively:
%\iffalse
%<*samplefinal>
%\fi
%    \begin{macrocode}
\def\version{final}
\input{childdoc.def}
\childdocforwardprefix[cdocsamp]{cdocsfn}{cdocsch}
%    \end{macrocode}

%\iffalse
%</samplefinal>
%\fi
%
% %%%%%%%%%%%%%%%%%%%%%%%%%%%%%%%%%%%%%%
% \paragraph{Command Line Processing.}
%
% The following three command lines generate the output files
% |cdocscld|, |cdocscl1| and |cdocscl2|
% which should be identical to
% |cdocsdrf|, |cdocsch1| and |cdocsfn2|, respectively:
% \begin{center}
% \begin{tabular}{l}
% |latex -jobname cdocscld \|\\
% |  "\def\version{draft}\input{childdoc.def}\childdocforward{cdocsamp}"|\\
% |latex -jobname cdocscl1 \|\\
% |  "\input{childdoc.def}\childdocforward[cdocsamp]{cdocsch1}"|\\
% |latex -jobname cdocscl2 \|\\
% |  "\def\version{final}\input{childdoc.def}\childdocforward{cdocsch2}"|
% \end{tabular}
% \end{center}
% Note that the trailing backslash on each first line
% merely continues the input to the second line
% (for convenient cut ant paste).
% Furthermore, the command |latex| can be replaced by any
% of its alternative versions such as |pdflatex|.
%
% %%%%%%%%%%%%%%%%%%%%%%%%%%%%%%%%%%%%%%%%%%%%%%%%%%%%%%%%%%%%%%%%%%%%%%%%%%%%%%
% %%%%%%%%%%%%%%%%%%%%%%%%%%%%%%%%%%%%%%%%%%%%%%%%%%%%%%%%%%%%%%%%%%%%%%%%%%%%%%
% \section{Implementation}
%\iffalse
%<*package>
%\fi
%
% This section describes the definitions file |childdoc.def|.

% The definitions cannot be loaded using |\usepackage| or |\RequirePackage|
% which has a mechanism to prevent loading a style file more than once.
% When loading the definitions by means of |\input|
% multiple instances have to be prevented manually:
%\iffalse
%This code needs to be before the `\ProvidesFile' directive
%which is defined at the beginning of this file.
%Therefore it is also placed there and commented out here.
%</package>
%<*discard>
%\fi
%    \begin{macrocode}
\ifdefined\childdocmain\endinput\fi
%    \end{macrocode}
%\iffalse
%</discard>
%<*package>
%\fi
%
% \macro{\ifchilddoc}
% \macro{\ifchilddocmanual}
% The conditional |\ifchilddoc| tells whether a
% child (true) or main (false) document is being compiled.
% The conditional |\ifchilddocmanual| tells whether
% the |\includeonly| mechanism is used (false) or
% the selection of child files must be performed manually (true).
% The definitions initialise to false:
%    \begin{macrocode}
\newif\ifchilddoc
\newif\ifchilddocmanual
%    \end{macrocode}

% \macro{\childdocname}
% \macro{\childdocjob}
% The macro |\childdocname| stores the name of the main document
% to be compiled. The macro |\childdocjob| stores the name of
% the document on which the \LaTeX{} compiler was originally invoked.
% The content of |\jobname| cannot be compared
% to filenames specified in the source due to different catcodes.
% The following code rescans |\jobname|, stores the result
% in |\childdocname| and saves a copy in |\childdocjob|:
%    \begin{macrocode}
\edef\childdocname{\scantokens\expandafter{\jobname\noexpand}}
\let\childdocjob\childdocname
%    \end{macrocode}

% \macro{\childdocdisable}
% The macro |\childdocdisable| prevents the main file
% from being processed more than once.
% At this stage, the main document command |\childdocmain|
% is assumed to be called once again where it should do nothing.
% Any subsequent call to it should prevent
% a secondary processing of the main document
% It overwrites the forwarding commands
% |\childdocof| and |\childdocforward|
% with empty macros to prevent further inclusions of the main document:
%    \begin{macrocode}
\newcommand{\childdocdisable}
{
  \renewcommand{\childdocmain}[1]{\renewcommand{\childdocmain}[1]{\endinput}}
  \renewcommand{\childdocof}[1]{}
  \renewcommand{\childdocby}[2][]{}
  \renewcommand{\childdocforward}[2][]{}
  \renewcommand{\childdocdisable}{}
}
%    \end{macrocode}

% \macro{\childdocmain}
% The macro |\childdocmain| is to be called at the top of the main file
% with nothing or the main filename (without extension) as argument.
% First, it breaks loops.
% If the argument is not empty and does not match |\childdocname|
% (which is set by the first inclusion of |childdoc.def|),
% |\ifchilddoc| is set to true, |\includeonly| is applied to the child file
% and |\jobname| is set to the main file
% (for proper handling of |.aux| files):
%    \begin{macrocode}
\newcommand{\childdocmain}[1]
{
  \childdocdisable\childdocmain{}
  \if?#1?\else
    \begingroup
      \def\childdoctmp{#1}
      \ifx\childdoctmp\childdocname
        \def\childdoctmp{}
      \else
        \def\childdoctmp
        {
          \childdoctrue
          \includeonly{\childdocname}
          \def\childdocjob{#1}
          \def\jobname{#1}
        }
      \fi
      \expandafter
    \endgroup
    \childdoctmp
  \fi
}
%    \end{macrocode}

% \macro{\childdocof}
% The command |\childdocof| redirects
% compilation to the main file |#1|.
%    \begin{macrocode}
\newcommand{\childdocof}[1]
{
  \childdocdisable
  \childdoctrue
  \includeonly{\childdocname}
  \def\jobname{#1}
  \def\childdocjob{#1}
  \input{#1}
}
%    \end{macrocode}

% \macro{\childdocby}
% The command |\childdocby| ....
%    \begin{macrocode}
\newcommand{\childdocby}[2][]
{
  \childdocdisable
  \childdoctrue
  \childdocmanualtrue
  \if?#1?\else
    \def\jobname{#2}
  \fi
  \def\childdocjob{#2}
  \input{#2}
  \endinput
}
%    \end{macrocode}

% \macro{\childdocforward}
% The command |\childdocforward| redirects
% compilation to the main file or
% (if the optional argument is given) a child file.
% Parameters are set as if the main file
% or a child file starting with |\childdocof| was compiled.
% Then compilation is handed over to the main file:
%    \begin{macrocode}
\newcommand{\childdocforward}[2][]
{
  \begingroup
    \if?#1?
      \def\childdoctmp
      {
        \def\childdocname{#2}
        \def\childdocjob{#2}
        \def\jobname{#2}
        \input{#2}
        \endinput
      }
    \else
      \def\childdoctmp
      {
        \childdocdisable
        \def\childdocname{#2}
        \childdoctrue
        \includeonly{#2}
        \def\childdocjob{#1}
        \def\jobname{#1}
        \input{#1}
        \endinput
      }
    \fi
    \expandafter
  \endgroup
  \childdoctmp
}
%    \end{macrocode}

% \macro{\childdocforwardprefix}
% The command |\childdocforwardprefix| redirects
% compilation to the main or a child file by means of a pattern.
% The prefix |#1| in the current filename is replaced by |#2|
% and the suffix of the current filename is kept
% (it is assumed that the filename does not contain the substring `|~~~|'
% which is used as a delimiter).
% Compilation is handed over to the new file by |\childdocforward|:
%    \begin{macrocode}
\newcommand{\childdocforwardprefix}[3][]
{
  \begingroup
    \def\childdocextract #2##1~~~{\def\childdoctmp{\childdocforward[#1]{#3##1}}}
    \expandafter\childdocextract\childdocname~~~
    \expandafter
  \endgroup
  \childdoctmp
}
%    \end{macrocode}

% \macro{\childdoc}
% The deprecated macro |\childdoc| is a legacy version of |\childdocmain|:
%    \begin{macrocode}
\newcommand{\childdoc}{\childdocmain}
%    \end{macrocode}

% \macro{\childdocredirect}
% The deprecated macro |\childdocredirect| is a legacy version
% of |\childdocforward| and |\childdocforwardprefix|:
%    \begin{macrocode}
\newcommand{\childdocredirect}[2][]
{
  \begingroup
    \if?#1?
      \def\childdoctmp{\childdocforward{#2}}
    \else
      \def\childdoctmp{\childdocforwardprefix{#1}{#2}}
    \fi
    \expandafter
  \endgroup
  \childdoctmp
}
%    \end{macrocode}

%\iffalse
%</package>
%\fi
%
\endinput
\childdocforward[|\textit{main}|]{|\textit{dest}|}"|
\end{center}
%
Here \textit{target} is the name of the output file,
\textit{main} is the name of the main file
and \textit{dest} is the name of the main or child file to be processed
(all filenames without extensions).
The optional argument \textit{main} can be omitted
if \textit{main} matches \textit{dest}.
Optionally, compilation \textit{flags} can be defined via |\def| commands.
This command line makes the \TeX{} engine believe
it is compiling the file \textit{target}
whose content is specified as the latter parameter.
The provided code then forwards the processing to
\textit{main} or \textit{dest} as described in \secref{sec:forward}.

%%%%%%%%%%%%%%%%%%%%%%%%%%%%%%%%%%%%%%%%%%%%%%%%%%%%%%%%%%%%%%%%%%%%%%%%%%%%%%%%
\subsection{Include by Input}
\label{sec:input}

Including child documents by |\include| has some restrictions by design.
Most notably, the content of a child document always occupies
its own set of pages; pages cannot be shared between child documents.
Usually, this behaviour makes perfect sense
because each child document contain an essential part of the document.
However, in some situations it may be desirable to compose
a document from a collection of parts
without having mandatory page breaks between then.
For this case, the package
provides a mechanism to include parts
by |\input| which can also be processed individually.
However, by construction this mechanism
requires manual handling of the content to be output.

%%%%%%%%%%%%%%%%%%%%%%%%%%%%%%%%%%%%%%%%
\DescribeMacro{\ifchilddocmanual}
The main file should be prepared as usual, see \secref{sec:include}.
However, the document body must make a distinction
between processing of an individual part and of the main document, e.g.:
%
\begin{center}
\begin{tabular}{l}
|\ifchilddocmanual|\\
|\input{\childdocname}|\\
|\||else|\\
\textit{document body with }|\input{|\textit{part}|}|\\
|\||fi|
\end{tabular}
\end{center}
%
The conditional |\ifchilddocmanual| is true whenever
a part to be included by |\input| is being compiled,
and the name of the part is stored in |\childdocname|.

%%%%%%%%%%%%%%%%%%%%%%%%%%%%%%%%%%%%%%%%
\DescribeMacro{\childdocby}
Each part to be included by |\input| should start with:
%
\begin{center}
\begin{tabular}{l}
|% \iffalse
%
% childdoc.dtx Copyright (C) 2017-2018 Niklas Beisert
%
% This work may be distributed and/or modified under the
% conditions of the LaTeX Project Public License, either version 1.3
% of this license or (at your option) any later version.
% The latest version of this license is in
%   http://www.latex-project.org/lppl.txt
% and version 1.3 or later is part of all distributions of LaTeX
% version 2005/12/01 or later.
%
% This work has the LPPL maintenance status `maintained'.
%
% The Current Maintainer of this work is Niklas Beisert.
%
% This work consists of the files childdoc.dtx and childdoc.ins
% and the derived files childdoc.def and cdocsamp.tex with
% cdocsch1.tex, cdocsch2.tex, cdocsdrf.tex, cdocsfn1.tex, cdocsfn2.tex.
%
%<package>\ifdefined\childdocmain\endinput\fi
%<package>\ProvidesFile{childdoc.def}[2018/12/30 v2.0 child document driver]
%<samplemain>\ProvidesFile{cdocsamp.tex}[2018/12/30 v2.0 sample for childdoc]
%<*driver>
%\ProvidesFile{childdoc.drv}[2018/12/30 v2.0 childdoc reference manual file]
\PassOptionsToClass{10pt,a4paper}{article}
\documentclass{ltxdoc}

\usepackage[margin=35mm]{geometry}
\usepackage{hyperref}
\usepackage{hyperxmp}
\usepackage[usenames]{color}

\hypersetup{colorlinks=true}
\hypersetup{pdfstartview=FitH}
\hypersetup{pdfpagemode=UseNone}
\hypersetup{pdfsource={}}
\hypersetup{pdflang={en-UK}}
\hypersetup{pdfcopyright={Copyright 2017-2018 Niklas Beisert.
  This work may be distributed and/or modified under the
  conditions of the LaTeX Project Public License, either version 1.3
  of this license or (at your option) any later version.}}
\hypersetup{pdflicenseurl={http://www.latex-project.org/lppl.txt}}
\hypersetup{pdfcontactaddress={ETH Zurich, ITP, HIT K,
  Wolfgang-Pauli-Strasse 27}}
\hypersetup{pdfcontactpostcode={8093}}
\hypersetup{pdfcontactcity={Zurich}}
\hypersetup{pdfcontactcountry={Switzerland}}
\hypersetup{pdfcontactemail={nbeisert@itp.phys.ethz.ch}}
\hypersetup{pdfcontacturl={http://people.phys.ethz.ch/\xmptilde nbeisert/}}

\newcommand{\secref}[1]{\hyperref[#1]{section \ref*{#1}}}

\parskip1ex
\parindent0pt
\let\olditemize\itemize
\def\itemize{\olditemize\parskip0pt}

\begin{document}

\title{The \textsf{childdoc} Package}
\hypersetup{pdftitle={The childdoc Package}}
\author{Niklas Beisert\\[2ex]
  Institut f\"ur Theoretische Physik\\
  Eidgen\"ossische Technische Hochschule Z\"urich\\
  Wolfgang-Pauli-Strasse 27, 8093 Z\"urich, Switzerland\\[1ex]
  \href{mailto:nbeisert@itp.phys.ethz.ch}
  {\texttt{nbeisert@itp.phys.ethz.ch}}}
\hypersetup{pdfauthor={Niklas Beisert}}
\hypersetup{pdfsubject={Manual for the LaTeX2e Package childdoc}}
\date{30 December 2018, \textsf{v2.0}}
\maketitle

\begin{abstract}\noindent
\textsf{childdoc} is a \LaTeXe{} package
that enables the direct compilation
of document sections included by |\include|
to individual files.
\end{abstract}

\begingroup
\parskip0ex
\tableofcontents
\endgroup

%%%%%%%%%%%%%%%%%%%%%%%%%%%%%%%%%%%%%%%%%%%%%%%%%%%%%%%%%%%%%%%%%%%%%%%%%%%%%%%%
%%%%%%%%%%%%%%%%%%%%%%%%%%%%%%%%%%%%%%%%%%%%%%%%%%%%%%%%%%%%%%%%%%%%%%%%%%%%%%%%
\section{Introduction}

\LaTeX{} provides a mechanism to structure a large document (such as a book)
into a main file and several child files (containing the chapters)
using the |\include| command.
This mechanism is beneficial for documents
which span hundreds of pages in order to
make the source file(s) more manageable.
Moreover, compilation can be restricted to
selected child files by means of the |\includeonly| command.
The latter feature can be used to reduce the compilation time while editing
(this was significantly more useful in the earlier days of \LaTeX{})
or to generate a smaller document which is easier to navigate.
Another application of |\includeonly| is to generate
documents consisting of selected parts of the complete document.

However, there are a few drawbacks of the plain |\include| mechanism:
\begin{itemize}
\item
The child files cannot be compiled on their own,
they can only be compiled via the main file.
A naive editing environment
(such as a text editor with an option
to have the current file processed by \LaTeX)
may require one to switch to the main file before compiling;
attempting to compile the child file produces errors.
\item
The main file must be modified (each time)
to adjust the |\includeonly| command
to the present needs. This easily leaves the main file in a messy state.
\item
The generated document will always carry the filename
of the main document. This is inconvenient if
several child files are to be compiled and
to be kept for distribution.
\end{itemize}

The present package provides a simple interface
to make child files individually compilable by \LaTeX{}.
Compiling a child file then has the same effect as compiling
the main file with an |\includeonly| command
to select the appropriate child.
Moreover the generated document will carry the name of the child
rather than the main file.
This resolves all three above issues.

This feature is meant to make the editing of books,
thesis documents and lecture notes somewhat more convenient.
However, the package can also be used efficiently for
composing a series of documents (such as exercise sheets)
which are typically distributed individually.
It then assists the author in generating the individual documents
(potentially in different versions)
as well as a document containing the collected series.
Another application is in developing style files
or other kinds of included material
where compilation of the style file could redirect
to a sample or test file.

%%%%%%%%%%%%%%%%%%%%%%%%%%%%%%%%%%%%%%%%%%%%%%%%%%%%%%%%%%%%%%%%%%%%%%%%%%%%%%%%
%%%%%%%%%%%%%%%%%%%%%%%%%%%%%%%%%%%%%%%%%%%%%%%%%%%%%%%%%%%%%%%%%%%%%%%%%%%%%%%%
\section{Usage}

First of all, the package \textsf{childdoc} is \emph{not} a standard
\LaTeXe{} |.sty| style file! Therefore it needs to be invoked in
a non-standard way.

%%%%%%%%%%%%%%%%%%%%%%%%%%%%%%%%%%%%%%%%%%%%%%%%%%%%%%%%%%%%%%%%%%%%%%%%%%%%%%%%
\subsection{Included Files}
\label{sec:include}

%%%%%%%%%%%%%%%%%%%%%%%%%%%%%%%%%%%%%%%%
\DescribeMacro{\childdocmain}
To use the package, add the commands
\begin{center}
\begin{tabular}{l}
|\input{childdoc.def}|\\
|\childdocmain{}|\\
\end{tabular}
\end{center}
at the very top of the main \LaTeX{} file,
in particular \emph{before} the |\documentclass| statement!
The argument of |\childdocmain| should be left empty
(but it must be present).

%%%%%%%%%%%%%%%%%%%%%%%%%%%%%%%%%%%%%%%%
\DescribeMacro{\childdocof}
Furthermore, add the commands
\begin{center}
\begin{tabular}{l}
|\input{childdoc.def}|\\
|\childdocof{|\textit{main}|}|\\
\end{tabular}
\end{center}
at the top of every child file \textit{child}
which is included by |\include{|\textit{child}|}|
from within the main file
(or at least for those files to be compiled individually).
The argument \textit{main} must be the filename of the main file.

There are a couple of
considerations in setting up the main and child documents:

%%%%%%%%%%%%%%%%%%%%%%%%%%%%%%%%%%%%%%%%
\paragraph{Restrictions.}

Please note the following restrictions:
\begin{itemize}
\item
|\childdocmain| must be called with one argument \textit{main}
to ensure compatibility with earlier version of the package.
It must either be empty (|\childdocmain{}|)
or precisely match the filename of the main file in which it is specified.
See \secref{sec:detection} for further information.
\item
The filename \textit{main} must be specified without the |.tex| extension.
\item
The filename \textit{main} is case sensitive
(even in case-insensitive file systems)
due to internal string comparison.
\item
The argument \textit{main} should be fully expanded, it cannot be a macro.
\item
Subdirectories and special characters should be avoided in filenames.
\item
The command |\childdocmain{|\textit{main}|}| must be followed by a whitespace.
It should not be followed immediately by another command
or by a comment mark `|%|'.
This is because the \TeX{} parser reads the token immediately following
the argument of |\childdocmain| and puts it
at the beginning of every child section;
however, a white\-space is ignored.
\end{itemize}

%%%%%%%%%%%%%%%%%%%%%%%%%%%%%%%%%%%%%%%%
\paragraph{Content of Main File.}

It is advisable to place all content in the child files included by |\include|.
Any output contained in the main file will appear in all child documents
unless suppressed manually;
it cannot be suppressed automatically by the |\includeonly| directive
and thus should normally be avoided.
A method to include some content in the main file
by means of conditional processing is described in \secref{sec:conditional}.

%%%%%%%%%%%%%%%%%%%%%%%%%%%%%%%%%%%%%%%%
\paragraph{Page Numbering.}

When only a part of the document is compiled,
the appropriate numbering of pages
(as well as other status parameters)
is determined from the |.aux| files.
The latter contain information from previous passes.
However this information needs to propagate through
all intermediate child documents.
Therefore the page numbering in child documents may well
be inconsistent until the complete document is compiled at least once.

A useful (if unconventional) way to always ensure a consistent
page numbering is to restart the numbering in each child document
and denote the pages by `\textit{child}|.|\textit{page}'
where \textit{child} represents the chapter/section number of the child file.
This can be achieved by the command
|\numberwithin{page}{|\textit{child}|}|
of the \textsf{amsmath} package
where \textit{child} can be |chapter| or |section|
depending on the chosen structuring.
Alternatively, one can modify the macro |\thepage| appropriately
and reset the counter |page| at the start of each child file.

%%%%%%%%%%%%%%%%%%%%%%%%%%%%%%%%%%%%%%%%%%%%%%%%%%%%%%%%%%%%%%%%%%%%%%%%%%%%%%%%
\subsection{Conditional Processing}
\label{sec:conditional}

The package provides a mechanism to compile different versions
of a document. To customise the versions further some conditional processing
can come in handy to distinguish which version is being compiled.
The package provides two macros to describe the compilation context:

%%%%%%%%%%%%%%%%%%%%%%%%%%%%%%%%%%%%%%%%
\DescribeMacro{\ifchilddoc}
The conditional |\ifchilddoc| distinguishes between the compilation of
child documents and the main document:
%
\begin{center}
|\ifchilddoc |\textit{child-code}| |[|\||else |\textit{main-code}]| \||fi|
\end{center}

%%%%%%%%%%%%%%%%%%%%%%%%%%%%%%%%%%%%%%%%
\DescribeMacro{\childdocname}
\DescribeMacro{\childdocjob}
The macro |\childdocname| contains the filename (without extension)
of the main or child file being processed.
Note that |\childdocjob| will always contain the name of the main file.

%%%%%%%%%%%%%%%%%%%%%%%%%%%%%%%%%%%%%%%%
\paragraph{Title Page.}

Conditional processing can be used to include a title or banner page
in the main document when proper precautions are taken.
Importantly, the code in the main file should ensure that the page counter
(as well as other status parameters which are stored in the |.aux| files)
takes the same value after the conditional processing.
Otherwise the page numbers may take divergent values
depending on which part is compiled.

For example, a title page could be declared by:
%
\begin{center}
\begin{tabular}{l}
|\ifchilddoc\||else|\\
|\addtocounter{page}{-1}|\\
\textit{code for title page}\\
|\newpage|\\
|\||fi|
\end{tabular}
\end{center}
%
A banner page for the child documents can be generated by:
%
\begin{center}
\begin{tabular}{l}
|\ifchilddoc|\\
|\addtocounter{page}{-1}|\\
\textit{code for banner page}\\
|\newpage|\\
|\||fi|
\end{tabular}
\end{center}
%
Here one could write a message such as:
\begin{center}
|This is the part \childdocname{} of \childdocjob{}.|
\end{center}

%%%%%%%%%%%%%%%%%%%%%%%%%%%%%%%%%%%%%%%%%%%%%%%%%%%%%%%%%%%%%%%%%%%%%%%%%%%%%%%%
\subsection{Flags}
\label{sec:flags}

The package makes it easy to generate different versions
of the main or child documents.
To this end compilation flags can be defined
and assigned different default values.
They will be particularly useful in conjunction
with the forwarding mechanism described in \secref{sec:forward}.

For example, it may be useful to have a flag |\version|
which can be set to |draft| or |final|.
The document source will contain some conditional code
depending on the value of |\version|.
Suppose further, the flag should default to |final| for the main file
and to |draft| for child files
which is a natural assignment for editing the document.
This is achieved by placing the following code
in the preamble of the main document
(below the |\childdocmain| directive):
%
\begin{center}
\begin{tabular}{l}
|\ifchilddoc|\\
|\providecommand{\version}{draft}|\\
|\||else|\\
|\providecommand{\version}{final}|\\
|\||fi|
\end{tabular}
\end{center}
%
The definition by |\providecommand| makes sure
that previous definitions are not overwritten.
Further statements |\providecommand{\version}{...}|
can thus be added before the above code to override it.

For the main file, one might add a line
(between |\childdocmain| and the above block)
%
\begin{center}
|%\ifchilddoc\||else\providecommand{\version}{draft}\||fi|
\end{center}
%
which can be uncommented to produce a draft version.
Likewise one can add a line to the very top of a child file
(above the |\childdocof{|\textit{main}|}| directive)
%
\begin{center}
|%\providecommand{\version}{final}|
\end{center}
%
which can be uncommented to produce the final version of this child document.

%%%%%%%%%%%%%%%%%%%%%%%%%%%%%%%%%%%%%%%%%%%%%%%%%%%%%%%%%%%%%%%%%%%%%%%%%%%%%%%%
\subsection{Forwarding}
\label{sec:forward}

Different versions of the main or child documents
using compilation flags as described in \secref{sec:flags}
can be (permanently) stored in different files
for convenient compilation, viewing and distribution.
To this end, the package defines a command
to pass on compilation to a different file:

%%%%%%%%%%%%%%%%%%%%%%%%%%%%%%%%%%%%%%%%
\DescribeMacro{\childdocforward}
The command |\childdocforward| redirects processing to
another source file:
%
\begin{center}
\begin{tabular}{l}
|\input{childdoc.def}|\\
|\childdocforward[|\textit{main}|]{|\textit{dest}|}|\\
\end{tabular}
\end{center}
%
The argument \textit{dest} is the destination file
(without extension).
It should be the main file or one of the child files.
Note that further \textsf{childdoc} directives
such as |\childdocof| and |\childdocforward|
in the indicated file will be processed in this form.
The optional argument \textit{main}
passes on directly to the main file \textit{main}
while pretending to compile the child \textit{dest}.
This form behaves as if \textit{dest}
issues |\childdocof{|\textit{main}|}| right away,
and no further \textsf{childdoc} directives will be processed.

%%%%%%%%%%%%%%%%%%%%%%%%%%%%%%%%%%%%%%%%
\DescribeMacro{\...prefix}
In the alternative form |\childdocforwardprefix|,
%
\begin{center}
\begin{tabular}{l}
|\input{childdoc.def}|\\
|\childdocforwardprefix[|\textit{main}|]{|\textit{prefix}|}{|\textit{dest}|}|
\end{tabular}
\end{center}
%
the destination file is determined by a pattern
depending on the current file:
To make this work, the current file must be called
`{\textit{prefix}\hspace{0.2em}\textit{suffix}}'
with \textit{prefix} matching precisely the argument.
Processing is then passed on to the file
`{\textit{dest}\hspace{0.2em}\textit{suffix}}'.
Surely, the same effect is achieved by
directly specifying the
argument `{\textit{dest}\hspace{0.2em}\textit{suffix}}'
in the first form.
However, that requires to set up a different file
for each child. With the alternative form of the command
all these files can have exactly the same content
which simplifies setting them up and maintaining them.

For example, the following file |draft.tex|
with a compilation flag |\version| as described in \secref{sec:flags}
compiles the main document as a draft:
%
\begin{center}
\begin{tabular}{l}
|\def\version{draft}|\\
|\input{childdoc.def}|\\
|\childdocforward{|\textit{main}|}|
\end{tabular}
\end{center}
%
Likewise, the following files |final|\textit{nn}|.tex|
compile the final version of the child document
|child|\textit{nn}|.tex|:
%
\begin{center}
\begin{tabular}{l}
|\def\version{final}|\\
|\input{childdoc.def}|\\
|\childdocforwardprefix{final}{child}|
\end{tabular}
\end{center}
%

Note that when several versions of a main file and/or of each child file
are to be generated, it may be convenient to set up a |Makefile| or
shell script to automatise the process.

%%%%%%%%%%%%%%%%%%%%%%%%%%%%%%%%%%%%%%%%%%%%%%%%%%%%%%%%%%%%%%%%%%%%%%%%%%%%%%%%
\subsection{Command Line Processing}
\label{sec:commandline}

The effect of redirection files can also be achieved by invoking
the \LaTeX{} compiler with a more elaborate command line.
Most conveniently this should be done as part
of a shell script or a |Makefile|.

When using \textsf{childdoc} in the main file, the following
command lines effectively perform a redirection
(note that depending on the shell being used,
backslashes may have to be doubled: `|\|' $\to$ `|\\|'):
%
\begin{center}
|... -jobname "|\textit{target}|" |\\|"|[\textit{flags}]%
|\input{childdoc.def}\childdocforward[|\textit{main}|]{|\textit{dest}|}"|
\end{center}
%
Here \textit{target} is the name of the output file,
\textit{main} is the name of the main file
and \textit{dest} is the name of the main or child file to be processed
(all filenames without extensions).
The optional argument \textit{main} can be omitted
if \textit{main} matches \textit{dest}.
Optionally, compilation \textit{flags} can be defined via |\def| commands.
This command line makes the \TeX{} engine believe
it is compiling the file \textit{target}
whose content is specified as the latter parameter.
The provided code then forwards the processing to
\textit{main} or \textit{dest} as described in \secref{sec:forward}.

%%%%%%%%%%%%%%%%%%%%%%%%%%%%%%%%%%%%%%%%%%%%%%%%%%%%%%%%%%%%%%%%%%%%%%%%%%%%%%%%
\subsection{Include by Input}
\label{sec:input}

Including child documents by |\include| has some restrictions by design.
Most notably, the content of a child document always occupies
its own set of pages; pages cannot be shared between child documents.
Usually, this behaviour makes perfect sense
because each child document contain an essential part of the document.
However, in some situations it may be desirable to compose
a document from a collection of parts
without having mandatory page breaks between then.
For this case, the package
provides a mechanism to include parts
by |\input| which can also be processed individually.
However, by construction this mechanism
requires manual handling of the content to be output.

%%%%%%%%%%%%%%%%%%%%%%%%%%%%%%%%%%%%%%%%
\DescribeMacro{\ifchilddocmanual}
The main file should be prepared as usual, see \secref{sec:include}.
However, the document body must make a distinction
between processing of an individual part and of the main document, e.g.:
%
\begin{center}
\begin{tabular}{l}
|\ifchilddocmanual|\\
|\input{\childdocname}|\\
|\||else|\\
\textit{document body with }|\input{|\textit{part}|}|\\
|\||fi|
\end{tabular}
\end{center}
%
The conditional |\ifchilddocmanual| is true whenever
a part to be included by |\input| is being compiled,
and the name of the part is stored in |\childdocname|.

%%%%%%%%%%%%%%%%%%%%%%%%%%%%%%%%%%%%%%%%
\DescribeMacro{\childdocby}
Each part to be included by |\input| should start with:
%
\begin{center}
\begin{tabular}{l}
|\input{childdoc.def}|\\
|\childdocby{|\textit{main}|}|\\
\end{tabular}
\end{center}
%
The directive |\childdocby| is similar to |\childdocof|
described in \secref{sec:include},
but the subsequent selection of content must be done manually.
To that end, both |\ifchilddoc| and |\ifchilddocmanual|
will be true upon processing of a part,
and the name of the part is stored in |\childdocname|.
Note that |\jobname| will be set to the filename of the current part
so that each part receives an individual |.aux| file
that does not interfere with the |.aux| file(s) of the main document.
This behaviour can be altered by the alternative form
|\childdocby[*]{|\textit{main}|}| (with a non-empty optional argument)
which uses the |.aux| file of the main document
by setting |\jobname| to \textit{main}.

%%%%%%%%%%%%%%%%%%%%%%%%%%%%%%%%%%%%%%%%%%%%%%%%%%%%%%%%%%%%%%%%%%%%%%%%%%%%%%%%
\subsection{Driver Development}
\label{sec:driver}

The \textsf{childdoc} mechanism can also be use for the development
of definition files such as \LaTeX{} styles or classes.
This case differs from the above setup with multiple parts
included by |\include| in that no |\includeonly| should be invoked.
This can be achieved by starting the include file
(before |\ProvidesPackage|) with:
%
\begin{center}
\begin{tabular}{l}
|\input{childdoc.def}|\\
|\childdocforward{|\textit{main}|}|\\
\end{tabular}
\end{center}
%
or alternatively with:
%
\begin{center}
\begin{tabular}{l}
|\input{childdoc.def}|\\
|\childdocby{|\textit{main}|}|\\
\end{tabular}
\end{center}
%
Both forms have slightly different effects as described above.
The main file is prepared as usual, see \secref{sec:include}.

%%%%%%%%%%%%%%%%%%%%%%%%%%%%%%%%%%%%%%%%%%%%%%%%%%%%%%%%%%%%%%%%%%%%%%%%%%%%%%%%
\subsection{Legacy Detection}
\label{sec:detection}

The directive |\childdocmain| in the main file can detect
whether the complete document or merely a child is to be compiled
even without using the directive |\childdocof|.
This method is deprecated because it is less robust
and there is no compelling reason to use it;
it is merely provided for backward compatibility
and it may be removed in future versions.

If the detection mechanism is to be used,
it is mandatory to correctly specify
the filename of the main file as the argument of |\childdocmain|:
%
\begin{center}
\begin{tabular}{l}
|\input{childdoc.def}|\\
|\childdocmain{|\textit{main}|}|\\
\end{tabular}
\end{center}
%
If |\jobname| does not match the argument \textit{main} of |\childdocmain|,
it is assumed that |\jobname| points to the child file to be compiled.
When using |\childdocmain| with the main file specified as argument,
it suffices to start a child file
with just |\input{|\textit{main}|}|
without loading of the package and using |\childdocof|.
If instead all processing is done
with the appropriate \textsf{childdoc} directives,
the argument of \textit{main} of |\childdocmain| can be empty.

An alternative version of the command line processing described
in \secref{sec:commandline} using the detection mechanism reads:
%
\begin{center}
|... -jobname "|\textit{target}|" "|[\textit{flags}]%
[|\def\jobname{|\textit{dest}|}|]|\input{|\textit{main}|}"|
\end{center}

%%%%%%%%%%%%%%%%%%%%%%%%%%%%%%%%%%%%%%%%%%%%%%%%%%%%%%%%%%%%%%%%%%%%%%%%%%%%%%%%
\subsection{Manual Code}
\label{sec:manual}

In case one cannot be certain whether the definitions file |childdoc.def|
is installed on the target \TeX{} distribution
and one prefers not to ship it,
it is conceivable to paste a few relevant commands into the sources.

To that end, drop all statements |\input{childdoc.def}|
and perform the replacements as outlined below.
Instead of |\childdocmain{|\textit{main}|}| add the following code
to the top of the main file:
%
\begin{center}
\begin{tabular}{l}
|\||ifdefined\childdocname\endinput\||fi\newif\ifchilddoc|\\
|\edef\childdocname{\scantokens\expandafter{\jobname\noexpand}}|\\
|\def\childdocmain{|\textit{main}|}\||ifx\childdocmain\childdocname\||else|\\
|\childdoctrue\includeonly{\childdocname}\let\jobname\childdocmain\||fi|\\
\end{tabular}
\end{center}
%
Instead of |\childdocof{|\textit{main}|}| just include the main file
at the top of each child file:
%
\begin{center}
|\input{|\textit{main}|}|
\end{center}
%
A simple redirection |\childdocforward{|\textit{dest}|}| is achieved by:
%
\begin{center}
|\def\jobname{|\textit{dest}|}\input{\jobname}|
\end{center}
%
The redirection with prefix
|\childdocforwardprefix[|\textit{prefix}|]{|\textit{dest}|}|
is accomplished by:
%
\begin{center}
\begin{tabular}{l}
|{\edef\jobname{\scantokens\expandafter{\jobname\noexpand}}|\\
|\def\redirectjob |\textit{prefix}|#1~~~{\gdef\jobname{|\textit{dest}|#1}}|\\
|\expandafter\redirectjob\jobname~~~}\input{\jobname}|
\end{tabular}
\end{center}

In an alternative approach,
child documents can be compiled by a specific command line
without additional code or specific definitions:
%
\begin{center}
|... -jobname "|\textit{target}|" "|[\textit{flags}]%
|\includeonly{|\textit{dest}|}\input{|\textit{main}|}"|
\end{center}
%

%%%%%%%%%%%%%%%%%%%%%%%%%%%%%%%%%%%%%%%%%%%%%%%%%%%%%%%%%%%%%%%%%%%%%%%%%%%%%%%%
%%%%%%%%%%%%%%%%%%%%%%%%%%%%%%%%%%%%%%%%%%%%%%%%%%%%%%%%%%%%%%%%%%%%%%%%%%%%%%%%
\section{Information}

%%%%%%%%%%%%%%%%%%%%%%%%%%%%%%%%%%%%%%%%%%%%%%%%%%%%%%%%%%%%%%%%%%%%%%%%%%%%%%%%
\subsection{Copyright}

Copyright \copyright{} 2017--2018 Niklas Beisert

This work may be distributed and/or modified under the
conditions of the \LaTeX{} Project Public License, either version 1.3
of this license or (at your option) any later version.
The latest version of this license is in
  \url{http://www.latex-project.org/lppl.txt}
and version 1.3 or later is part of all distributions of \LaTeX{}
version 2005/12/01 or later.

This work has the LPPL maintenance status `maintained'.

The Current Maintainer of this work is Niklas Beisert.

This work consists of the files |README.txt|, |childdoc.ins| and |childdoc.dtx|
as well as the derived files |childdoc.def|, |cdocsamp.tex|
with |cdocsch1.tex|, |cdocsch2.tex|, |cdocspt3.tex|, |cdocspt4.tex|,
|cdocsdrf.tex|, |cdocsfn1.tex|, |cdocsfn2.tex|
as well as |childdoc.pdf|.

%%%%%%%%%%%%%%%%%%%%%%%%%%%%%%%%%%%%%%%%%%%%%%%%%%%%%%%%%%%%%%%%%%%%%%%%%%%%%%%%
\subsection{Files and Installation}

The package consists of the files:
%
\begin{center}
\begin{tabular}{ll}
    |README.txt|   & readme file \\
    |childdoc.ins| & installation file \\
    |childdoc.dtx| & source file \\
    |childdoc.def| & definition file \\
    |cdocsamp.tex| & sample main file \\
    |cdocsch1.tex| & sample include file \\
    |cdocsch2.tex| & sample include file \\
    |cdocspt3.tex| & sample part file \\
    |cdocspt4.tex| & sample part file \\
    |cdocsdrf.tex| & sample redirection file \\
    |cdocsfn1.tex| & sample redirection file \\
    |cdocsfn2.tex| & sample redirection file \\
    |childdoc.pdf| & manual
\end{tabular}
\end{center}
%
The distribution consists of the files
|README.txt|, |childdoc.ins| and |childdoc.dtx|.
%
\begin{itemize}
\item
Run (pdf)\LaTeX{} on |childdoc.dtx|
to compile the manual |childdoc.pdf| (this file).
\item
Run \LaTeX{} on |childdoc.ins| to create the definitions file |childdoc.def|
and the sample |cdocsamp.tex| with include files
|cdocsch1.tex|, |cdocsch2.tex|, |cdocspt3.tex|, |cdocspt4.tex|,
|cdocsdrf.tex|, |cdocsfn1.tex|, |cdocsfn2.tex|.
Then copy the file |childdoc.def| to an appropriate directory of your \LaTeX{}
distribution, e.g.\ \textit{texmf-root}|/tex/latex/childdoc|.
\end{itemize}

%%%%%%%%%%%%%%%%%%%%%%%%%%%%%%%%%%%%%%%%%%%%%%%%%%%%%%%%%%%%%%%%%%%%%%%%%%%%%%%%
\subsection{Related CTAN Packages}

There are several other packages which offer a similar functionality:
%
\begin{itemize}
\item
The packages
\href{http://ctan.org/pkg/docmute}{\textsf{docmute}},
\href{http://ctan.org/pkg/includex}{\textsf{includex}} and
\href{http://ctan.org/pkg/standalone}{\textsf{standalone}}
provide commands to include only the document body of
a child file thus allowing both files to be compiled individually.
\item
The packages \href{http://ctan.org/pkg/subdocs}{\textsf{subdocs}}
and \href{http://ctan.org/pkg/subfiles}{\textsf{subfiles}}
provide structures in which the main and child documents can be
encapsulated and allowing them to be compiled individually.
The inclusion mechanism is different from the conventional |\include|.
\item
The package \href{http://ctan.org/pkg/combine}{\textsf{combine}}
is an elaborate solution to combine several documents into one.
\end{itemize}
%
See also the CTAN topic \href{http://ctan.org/topic/subdocs}{\textsf{subdocs}}
for further related packages.
The present package differs from the above solutions in that
a document structure constructed with the conventional |\include| mechanism
just needs two extra commands at the top of every file
such that all constituent files can be compiled individually.

%%%%%%%%%%%%%%%%%%%%%%%%%%%%%%%%%%%%%%%%%%%%%%%%%%%%%%%%%%%%%%%%%%%%%%%%%%%%%%%%
%\subsection{Feature Suggestions}
%
%The following is a list of features which may be useful for future
%versions of this package:
%%
%\begin{itemize}
%\item
%\ldots
%\end{itemize}

%%%%%%%%%%%%%%%%%%%%%%%%%%%%%%%%%%%%%%%%%%%%%%%%%%%%%%%%%%%%%%%%%%%%%%%%%%%%%%%%
\subsection{Revision History}

%%%%%%%%%%%%%%%%%%%%%%%%%%%%%%%%%%%%%%%%
\paragraph{v2.0:} 2018/12/30

\begin{itemize}
\item
immediate forward processing
\item
added |\childdocby| mechanism
\item
manual restructured
\end{itemize}

%%%%%%%%%%%%%%%%%%%%%%%%%%%%%%%%%%%%%%%%
\paragraph{v1.6:} 2018/01/17

\begin{itemize}
\item
application for development of include files
\item
corrections to manual
\end{itemize}

%%%%%%%%%%%%%%%%%%%%%%%%%%%%%%%%%%%%%%%%
\paragraph{v1.5:} 2017/05/21

\begin{itemize}
\item
more complete structuring introduced
\item
|\childdocof| introduced
\item
|\childdoc| renamed to |\childdocmain|
\item
|\childredirect| renamed to |\childdocforward| and |\childdocforwardprefix|
and functionality expanded
\end{itemize}

%%%%%%%%%%%%%%%%%%%%%%%%%%%%%%%%%%%%%%%%
\paragraph{v1.0:} 2017/04/27

\begin{itemize}
\item
manual and install package
\item
first version published on CTAN
\end{itemize}

%%%%%%%%%%%%%%%%%%%%%%%%%%%%%%%%%%%%%%%%
\paragraph{v0.6:} 2017/04/26

\begin{itemize}
\item
redirection mechanism added
\end{itemize}

%%%%%%%%%%%%%%%%%%%%%%%%%%%%%%%%%%%%%%%%
\paragraph{v0.5:} 2017/04/26

\begin{itemize}
\item
functionality in definition file
\end{itemize}


%%%%%%%%%%%%%%%%%%%%%%%%%%%%%%%%%%%%%%%%%%%%%%%%%%%%%%%%%%%%%%%%%%%%%%%%%%%%%%%%
%%%%%%%%%%%%%%%%%%%%%%%%%%%%%%%%%%%%%%%%%%%%%%%%%%%%%%%%%%%%%%%%%%%%%%%%%%%%%%%%
%%%%%%%%%%%%%%%%%%%%%%%%%%%%%%%%%%%%%%%%%%%%%%%%%%%%%%%%%%%%%%%%%%%%%%%%%%%%%%%%
\appendix

\settowidth\MacroIndent{\rmfamily\scriptsize 000\ }

 \DocInput{childdoc.dtx}

\end{document}
%</driver>
% \fi
%
% %%%%%%%%%%%%%%%%%%%%%%%%%%%%%%%%%%%%%%%%%%%%%%%%%%%%%%%%%%%%%%%%%%%%%%%%%%%%%%
% %%%%%%%%%%%%%%%%%%%%%%%%%%%%%%%%%%%%%%%%%%%%%%%%%%%%%%%%%%%%%%%%%%%%%%%%%%%%%%
% \section{Sample}
%\iffalse
%<*samplemain>
%\fi
%
% The following presents a sample document
% with two chapters, two parts, a title page,
% a compile flag as well as three forwarding files to set the flag.
% It consists of eight |.tex| files:
% \begin{center}
% \begin{tabular}{ll}
% |cdocsamp.tex|&main file\\
% |cdocsch1.tex|&include file for chapter 1\\
% |cdocsch2.tex|&include file for chapter 2\\
% |cdocspt3.tex|&include file for part 3\\
% |cdocspt4.tex|&include file for part 4\\
% |cdocsdrf.tex|&forwarding file for main file in draft mode\\
% |cdocsfi1.tex|&forwarding file for final version of chapter 1\\
% |cdocsfi2.tex|&forwarding file for final version of chapter 2\\
% \end{tabular}
% \end{center}
% Each of the eight files can be compiled directly by the \LaTeX{} compiler.
%
% %%%%%%%%%%%%%%%%%%%%%%%%%%%%%%%%%%%%%%
% \paragraph{Main File.}
%
% The main file is called |cdocsamp.tex|.
%
% Load the \textsf{childdoc} definitions and
% declare the filename for the main document:
%    \begin{macrocode}
\input{childdoc.def}
\childdocmain{}
%    \end{macrocode}

% Optional override for |\version| flag:
%    \begin{macrocode}
%%\ifchilddoc\else\providecommand{\version}{draft}\fi
%    \end{macrocode}

% Define the default values for the |\version| flag
% (|final| for the main file and |draft| for childs):
%    \begin{macrocode}
\ifchilddoc
\providecommand{\version}{draft}
\else
\providecommand{\version}{final}
\fi
%    \end{macrocode}

% Load the standard document class:
%    \begin{macrocode}
\documentclass[12pt]{article}
%    \end{macrocode}

% Start the document body:
%    \begin{macrocode}
\begin{document}
%    \end{macrocode}

% Declare a title page.
% Print title, part of document being processed and version flag:
%    \begin{macrocode}
\addtocounter{page}{-1}
\begin{center}
{\LARGE\bfseries{}childdoc example\par}
\vspace{1cm}
\ifchilddoc
\ifchilddocmanual part\else chapter\fi:
`\childdocname' of `\childdocjob'\par
\else
main document: `\childdocjob'\par
\fi
version: \version\par
\end{center}
\newpage
%    \end{macrocode}

% Manually include selected file,
% otherwise process as usual:
%    \begin{macrocode}
\ifchilddocmanual
\section*{part `\childdocname'}
\input{\childdocname}
\else
%    \end{macrocode}

% Include the two chapters:
%    \begin{macrocode}
\include{cdocsch1}
\include{cdocsch2}
%    \end{macrocode}

% Include the two parts unless only chapters should be displayed:
%    \begin{macrocode}
\ifchilddoc\else
\section{part three}
\input{cdocspt3}
\section{part four}
\input{cdocspt4}
\fi
%    \end{macrocode}

% Process as usual until here:
%    \begin{macrocode}
\fi
%    \end{macrocode}

% End of document body:
%    \begin{macrocode}
\end{document}
%    \end{macrocode}
%\iffalse
%</samplemain>
%\fi
%
% %%%%%%%%%%%%%%%%%%%%%%%%%%%%%%%%%%%%%%
% \paragraph{Chapter Include Files.}
%
% The include files are called |cdocsch1.tex| and |cdocsch2.tex|.
%
%\iffalse
%<*samplechap1|samplechap2>
%\fi

% Optional override for |\version| flag:
%    \begin{macrocode}
%%\providecommand{\version}{final}
%    \end{macrocode}

% Include the main document:
%    \begin{macrocode}
\input{childdoc.def}
\childdocof{cdocsamp}
%    \end{macrocode}

%\iffalse
%</samplechap1|samplechap2>
%\fi
%
%\iffalse
%<*samplechap1>
%\fi
% Some text for chapter 1:
%    \begin{macrocode}
\section{one}
some text in chapter one
%    \end{macrocode}

%\iffalse
%</samplechap1>
%\fi
% Some text for chapter 2:
%\iffalse
%<*samplechap2>
%\fi
%    \begin{macrocode}
\section{two}
more text in chapter two
%    \end{macrocode}

%\iffalse
%</samplechap2>
%\fi
%
% %%%%%%%%%%%%%%%%%%%%%%%%%%%%%%%%%%%%%%
% \paragraph{Part Include Files.}
%
% The include files are called |cdocspt3.tex| and |cdocspt4.tex|.
%
%\iffalse
%<*samplepart3|samplepart4>
%\fi

% Optional override for |\version| flag:
%    \begin{macrocode}
%%\providecommand{\version}{final}
%    \end{macrocode}

% Include the main document:
%    \begin{macrocode}
\input{childdoc.def}
\childdocby{cdocsamp}
%    \end{macrocode}

%\iffalse
%</samplepart3|samplepart4>
%\fi
%
%\iffalse
%<*samplepart3>
%\fi
% Some text for part 3:
%    \begin{macrocode}
some text in part three
%    \end{macrocode}

%\iffalse
%</samplepart3>
%\fi
% Some text for part 4:
%\iffalse
%<*samplepart4>
%\fi
%    \begin{macrocode}
more text in part four
%    \end{macrocode}

%\iffalse
%</samplepart4>
%\fi
%
% %%%%%%%%%%%%%%%%%%%%%%%%%%%%%%%%%%%%%%
% \paragraph{Forwarding for a Complete Draft.}
%
% The following forwarding file |cdocsdrf.tex|
% compiles the main document in draft mode:
%\iffalse
%<*sampledraft>
%\fi
%    \begin{macrocode}
\def\version{draft}
\input{childdoc.def}
\childdocforward{cdocsamp}
%    \end{macrocode}

%\iffalse
%</sampledraft>
%\fi
%
% %%%%%%%%%%%%%%%%%%%%%%%%%%%%%%%%%%%%%%
% \paragraph{Forwarding for Final Version of the Chapters.}
%
% The following forwarding files |cdocsfn1.tex| and |cdocsfn2.tex|
% (with identical content)
% compile the final versions of the child documents
% |cdocsch1.tex| and |cdocsch2.tex|, respectively:
%\iffalse
%<*samplefinal>
%\fi
%    \begin{macrocode}
\def\version{final}
\input{childdoc.def}
\childdocforwardprefix[cdocsamp]{cdocsfn}{cdocsch}
%    \end{macrocode}

%\iffalse
%</samplefinal>
%\fi
%
% %%%%%%%%%%%%%%%%%%%%%%%%%%%%%%%%%%%%%%
% \paragraph{Command Line Processing.}
%
% The following three command lines generate the output files
% |cdocscld|, |cdocscl1| and |cdocscl2|
% which should be identical to
% |cdocsdrf|, |cdocsch1| and |cdocsfn2|, respectively:
% \begin{center}
% \begin{tabular}{l}
% |latex -jobname cdocscld \|\\
% |  "\def\version{draft}\input{childdoc.def}\childdocforward{cdocsamp}"|\\
% |latex -jobname cdocscl1 \|\\
% |  "\input{childdoc.def}\childdocforward[cdocsamp]{cdocsch1}"|\\
% |latex -jobname cdocscl2 \|\\
% |  "\def\version{final}\input{childdoc.def}\childdocforward{cdocsch2}"|
% \end{tabular}
% \end{center}
% Note that the trailing backslash on each first line
% merely continues the input to the second line
% (for convenient cut ant paste).
% Furthermore, the command |latex| can be replaced by any
% of its alternative versions such as |pdflatex|.
%
% %%%%%%%%%%%%%%%%%%%%%%%%%%%%%%%%%%%%%%%%%%%%%%%%%%%%%%%%%%%%%%%%%%%%%%%%%%%%%%
% %%%%%%%%%%%%%%%%%%%%%%%%%%%%%%%%%%%%%%%%%%%%%%%%%%%%%%%%%%%%%%%%%%%%%%%%%%%%%%
% \section{Implementation}
%\iffalse
%<*package>
%\fi
%
% This section describes the definitions file |childdoc.def|.

% The definitions cannot be loaded using |\usepackage| or |\RequirePackage|
% which has a mechanism to prevent loading a style file more than once.
% When loading the definitions by means of |\input|
% multiple instances have to be prevented manually:
%\iffalse
%This code needs to be before the `\ProvidesFile' directive
%which is defined at the beginning of this file.
%Therefore it is also placed there and commented out here.
%</package>
%<*discard>
%\fi
%    \begin{macrocode}
\ifdefined\childdocmain\endinput\fi
%    \end{macrocode}
%\iffalse
%</discard>
%<*package>
%\fi
%
% \macro{\ifchilddoc}
% \macro{\ifchilddocmanual}
% The conditional |\ifchilddoc| tells whether a
% child (true) or main (false) document is being compiled.
% The conditional |\ifchilddocmanual| tells whether
% the |\includeonly| mechanism is used (false) or
% the selection of child files must be performed manually (true).
% The definitions initialise to false:
%    \begin{macrocode}
\newif\ifchilddoc
\newif\ifchilddocmanual
%    \end{macrocode}

% \macro{\childdocname}
% \macro{\childdocjob}
% The macro |\childdocname| stores the name of the main document
% to be compiled. The macro |\childdocjob| stores the name of
% the document on which the \LaTeX{} compiler was originally invoked.
% The content of |\jobname| cannot be compared
% to filenames specified in the source due to different catcodes.
% The following code rescans |\jobname|, stores the result
% in |\childdocname| and saves a copy in |\childdocjob|:
%    \begin{macrocode}
\edef\childdocname{\scantokens\expandafter{\jobname\noexpand}}
\let\childdocjob\childdocname
%    \end{macrocode}

% \macro{\childdocdisable}
% The macro |\childdocdisable| prevents the main file
% from being processed more than once.
% At this stage, the main document command |\childdocmain|
% is assumed to be called once again where it should do nothing.
% Any subsequent call to it should prevent
% a secondary processing of the main document
% It overwrites the forwarding commands
% |\childdocof| and |\childdocforward|
% with empty macros to prevent further inclusions of the main document:
%    \begin{macrocode}
\newcommand{\childdocdisable}
{
  \renewcommand{\childdocmain}[1]{\renewcommand{\childdocmain}[1]{\endinput}}
  \renewcommand{\childdocof}[1]{}
  \renewcommand{\childdocby}[2][]{}
  \renewcommand{\childdocforward}[2][]{}
  \renewcommand{\childdocdisable}{}
}
%    \end{macrocode}

% \macro{\childdocmain}
% The macro |\childdocmain| is to be called at the top of the main file
% with nothing or the main filename (without extension) as argument.
% First, it breaks loops.
% If the argument is not empty and does not match |\childdocname|
% (which is set by the first inclusion of |childdoc.def|),
% |\ifchilddoc| is set to true, |\includeonly| is applied to the child file
% and |\jobname| is set to the main file
% (for proper handling of |.aux| files):
%    \begin{macrocode}
\newcommand{\childdocmain}[1]
{
  \childdocdisable\childdocmain{}
  \if?#1?\else
    \begingroup
      \def\childdoctmp{#1}
      \ifx\childdoctmp\childdocname
        \def\childdoctmp{}
      \else
        \def\childdoctmp
        {
          \childdoctrue
          \includeonly{\childdocname}
          \def\childdocjob{#1}
          \def\jobname{#1}
        }
      \fi
      \expandafter
    \endgroup
    \childdoctmp
  \fi
}
%    \end{macrocode}

% \macro{\childdocof}
% The command |\childdocof| redirects
% compilation to the main file |#1|.
%    \begin{macrocode}
\newcommand{\childdocof}[1]
{
  \childdocdisable
  \childdoctrue
  \includeonly{\childdocname}
  \def\jobname{#1}
  \def\childdocjob{#1}
  \input{#1}
}
%    \end{macrocode}

% \macro{\childdocby}
% The command |\childdocby| ....
%    \begin{macrocode}
\newcommand{\childdocby}[2][]
{
  \childdocdisable
  \childdoctrue
  \childdocmanualtrue
  \if?#1?\else
    \def\jobname{#2}
  \fi
  \def\childdocjob{#2}
  \input{#2}
  \endinput
}
%    \end{macrocode}

% \macro{\childdocforward}
% The command |\childdocforward| redirects
% compilation to the main file or
% (if the optional argument is given) a child file.
% Parameters are set as if the main file
% or a child file starting with |\childdocof| was compiled.
% Then compilation is handed over to the main file:
%    \begin{macrocode}
\newcommand{\childdocforward}[2][]
{
  \begingroup
    \if?#1?
      \def\childdoctmp
      {
        \def\childdocname{#2}
        \def\childdocjob{#2}
        \def\jobname{#2}
        \input{#2}
        \endinput
      }
    \else
      \def\childdoctmp
      {
        \childdocdisable
        \def\childdocname{#2}
        \childdoctrue
        \includeonly{#2}
        \def\childdocjob{#1}
        \def\jobname{#1}
        \input{#1}
        \endinput
      }
    \fi
    \expandafter
  \endgroup
  \childdoctmp
}
%    \end{macrocode}

% \macro{\childdocforwardprefix}
% The command |\childdocforwardprefix| redirects
% compilation to the main or a child file by means of a pattern.
% The prefix |#1| in the current filename is replaced by |#2|
% and the suffix of the current filename is kept
% (it is assumed that the filename does not contain the substring `|~~~|'
% which is used as a delimiter).
% Compilation is handed over to the new file by |\childdocforward|:
%    \begin{macrocode}
\newcommand{\childdocforwardprefix}[3][]
{
  \begingroup
    \def\childdocextract #2##1~~~{\def\childdoctmp{\childdocforward[#1]{#3##1}}}
    \expandafter\childdocextract\childdocname~~~
    \expandafter
  \endgroup
  \childdoctmp
}
%    \end{macrocode}

% \macro{\childdoc}
% The deprecated macro |\childdoc| is a legacy version of |\childdocmain|:
%    \begin{macrocode}
\newcommand{\childdoc}{\childdocmain}
%    \end{macrocode}

% \macro{\childdocredirect}
% The deprecated macro |\childdocredirect| is a legacy version
% of |\childdocforward| and |\childdocforwardprefix|:
%    \begin{macrocode}
\newcommand{\childdocredirect}[2][]
{
  \begingroup
    \if?#1?
      \def\childdoctmp{\childdocforward{#2}}
    \else
      \def\childdoctmp{\childdocforwardprefix{#1}{#2}}
    \fi
    \expandafter
  \endgroup
  \childdoctmp
}
%    \end{macrocode}

%\iffalse
%</package>
%\fi
%
\endinput
|\\
|\childdocby{|\textit{main}|}|\\
\end{tabular}
\end{center}
%
The directive |\childdocby| is similar to |\childdocof|
described in \secref{sec:include},
but the subsequent selection of content must be done manually.
To that end, both |\ifchilddoc| and |\ifchilddocmanual|
will be true upon processing of a part,
and the name of the part is stored in |\childdocname|.
Note that |\jobname| will be set to the filename of the current part
so that each part receives an individual |.aux| file
that does not interfere with the |.aux| file(s) of the main document.
This behaviour can be altered by the alternative form
|\childdocby[*]{|\textit{main}|}| (with a non-empty optional argument)
which uses the |.aux| file of the main document
by setting |\jobname| to \textit{main}.

%%%%%%%%%%%%%%%%%%%%%%%%%%%%%%%%%%%%%%%%%%%%%%%%%%%%%%%%%%%%%%%%%%%%%%%%%%%%%%%%
\subsection{Driver Development}
\label{sec:driver}

The \textsf{childdoc} mechanism can also be use for the development
of definition files such as \LaTeX{} styles or classes.
This case differs from the above setup with multiple parts
included by |\include| in that no |\includeonly| should be invoked.
This can be achieved by starting the include file
(before |\ProvidesPackage|) with:
%
\begin{center}
\begin{tabular}{l}
|% \iffalse
%
% childdoc.dtx Copyright (C) 2017-2018 Niklas Beisert
%
% This work may be distributed and/or modified under the
% conditions of the LaTeX Project Public License, either version 1.3
% of this license or (at your option) any later version.
% The latest version of this license is in
%   http://www.latex-project.org/lppl.txt
% and version 1.3 or later is part of all distributions of LaTeX
% version 2005/12/01 or later.
%
% This work has the LPPL maintenance status `maintained'.
%
% The Current Maintainer of this work is Niklas Beisert.
%
% This work consists of the files childdoc.dtx and childdoc.ins
% and the derived files childdoc.def and cdocsamp.tex with
% cdocsch1.tex, cdocsch2.tex, cdocsdrf.tex, cdocsfn1.tex, cdocsfn2.tex.
%
%<package>\ifdefined\childdocmain\endinput\fi
%<package>\ProvidesFile{childdoc.def}[2018/12/30 v2.0 child document driver]
%<samplemain>\ProvidesFile{cdocsamp.tex}[2018/12/30 v2.0 sample for childdoc]
%<*driver>
%\ProvidesFile{childdoc.drv}[2018/12/30 v2.0 childdoc reference manual file]
\PassOptionsToClass{10pt,a4paper}{article}
\documentclass{ltxdoc}

\usepackage[margin=35mm]{geometry}
\usepackage{hyperref}
\usepackage{hyperxmp}
\usepackage[usenames]{color}

\hypersetup{colorlinks=true}
\hypersetup{pdfstartview=FitH}
\hypersetup{pdfpagemode=UseNone}
\hypersetup{pdfsource={}}
\hypersetup{pdflang={en-UK}}
\hypersetup{pdfcopyright={Copyright 2017-2018 Niklas Beisert.
  This work may be distributed and/or modified under the
  conditions of the LaTeX Project Public License, either version 1.3
  of this license or (at your option) any later version.}}
\hypersetup{pdflicenseurl={http://www.latex-project.org/lppl.txt}}
\hypersetup{pdfcontactaddress={ETH Zurich, ITP, HIT K,
  Wolfgang-Pauli-Strasse 27}}
\hypersetup{pdfcontactpostcode={8093}}
\hypersetup{pdfcontactcity={Zurich}}
\hypersetup{pdfcontactcountry={Switzerland}}
\hypersetup{pdfcontactemail={nbeisert@itp.phys.ethz.ch}}
\hypersetup{pdfcontacturl={http://people.phys.ethz.ch/\xmptilde nbeisert/}}

\newcommand{\secref}[1]{\hyperref[#1]{section \ref*{#1}}}

\parskip1ex
\parindent0pt
\let\olditemize\itemize
\def\itemize{\olditemize\parskip0pt}

\begin{document}

\title{The \textsf{childdoc} Package}
\hypersetup{pdftitle={The childdoc Package}}
\author{Niklas Beisert\\[2ex]
  Institut f\"ur Theoretische Physik\\
  Eidgen\"ossische Technische Hochschule Z\"urich\\
  Wolfgang-Pauli-Strasse 27, 8093 Z\"urich, Switzerland\\[1ex]
  \href{mailto:nbeisert@itp.phys.ethz.ch}
  {\texttt{nbeisert@itp.phys.ethz.ch}}}
\hypersetup{pdfauthor={Niklas Beisert}}
\hypersetup{pdfsubject={Manual for the LaTeX2e Package childdoc}}
\date{30 December 2018, \textsf{v2.0}}
\maketitle

\begin{abstract}\noindent
\textsf{childdoc} is a \LaTeXe{} package
that enables the direct compilation
of document sections included by |\include|
to individual files.
\end{abstract}

\begingroup
\parskip0ex
\tableofcontents
\endgroup

%%%%%%%%%%%%%%%%%%%%%%%%%%%%%%%%%%%%%%%%%%%%%%%%%%%%%%%%%%%%%%%%%%%%%%%%%%%%%%%%
%%%%%%%%%%%%%%%%%%%%%%%%%%%%%%%%%%%%%%%%%%%%%%%%%%%%%%%%%%%%%%%%%%%%%%%%%%%%%%%%
\section{Introduction}

\LaTeX{} provides a mechanism to structure a large document (such as a book)
into a main file and several child files (containing the chapters)
using the |\include| command.
This mechanism is beneficial for documents
which span hundreds of pages in order to
make the source file(s) more manageable.
Moreover, compilation can be restricted to
selected child files by means of the |\includeonly| command.
The latter feature can be used to reduce the compilation time while editing
(this was significantly more useful in the earlier days of \LaTeX{})
or to generate a smaller document which is easier to navigate.
Another application of |\includeonly| is to generate
documents consisting of selected parts of the complete document.

However, there are a few drawbacks of the plain |\include| mechanism:
\begin{itemize}
\item
The child files cannot be compiled on their own,
they can only be compiled via the main file.
A naive editing environment
(such as a text editor with an option
to have the current file processed by \LaTeX)
may require one to switch to the main file before compiling;
attempting to compile the child file produces errors.
\item
The main file must be modified (each time)
to adjust the |\includeonly| command
to the present needs. This easily leaves the main file in a messy state.
\item
The generated document will always carry the filename
of the main document. This is inconvenient if
several child files are to be compiled and
to be kept for distribution.
\end{itemize}

The present package provides a simple interface
to make child files individually compilable by \LaTeX{}.
Compiling a child file then has the same effect as compiling
the main file with an |\includeonly| command
to select the appropriate child.
Moreover the generated document will carry the name of the child
rather than the main file.
This resolves all three above issues.

This feature is meant to make the editing of books,
thesis documents and lecture notes somewhat more convenient.
However, the package can also be used efficiently for
composing a series of documents (such as exercise sheets)
which are typically distributed individually.
It then assists the author in generating the individual documents
(potentially in different versions)
as well as a document containing the collected series.
Another application is in developing style files
or other kinds of included material
where compilation of the style file could redirect
to a sample or test file.

%%%%%%%%%%%%%%%%%%%%%%%%%%%%%%%%%%%%%%%%%%%%%%%%%%%%%%%%%%%%%%%%%%%%%%%%%%%%%%%%
%%%%%%%%%%%%%%%%%%%%%%%%%%%%%%%%%%%%%%%%%%%%%%%%%%%%%%%%%%%%%%%%%%%%%%%%%%%%%%%%
\section{Usage}

First of all, the package \textsf{childdoc} is \emph{not} a standard
\LaTeXe{} |.sty| style file! Therefore it needs to be invoked in
a non-standard way.

%%%%%%%%%%%%%%%%%%%%%%%%%%%%%%%%%%%%%%%%%%%%%%%%%%%%%%%%%%%%%%%%%%%%%%%%%%%%%%%%
\subsection{Included Files}
\label{sec:include}

%%%%%%%%%%%%%%%%%%%%%%%%%%%%%%%%%%%%%%%%
\DescribeMacro{\childdocmain}
To use the package, add the commands
\begin{center}
\begin{tabular}{l}
|\input{childdoc.def}|\\
|\childdocmain{}|\\
\end{tabular}
\end{center}
at the very top of the main \LaTeX{} file,
in particular \emph{before} the |\documentclass| statement!
The argument of |\childdocmain| should be left empty
(but it must be present).

%%%%%%%%%%%%%%%%%%%%%%%%%%%%%%%%%%%%%%%%
\DescribeMacro{\childdocof}
Furthermore, add the commands
\begin{center}
\begin{tabular}{l}
|\input{childdoc.def}|\\
|\childdocof{|\textit{main}|}|\\
\end{tabular}
\end{center}
at the top of every child file \textit{child}
which is included by |\include{|\textit{child}|}|
from within the main file
(or at least for those files to be compiled individually).
The argument \textit{main} must be the filename of the main file.

There are a couple of
considerations in setting up the main and child documents:

%%%%%%%%%%%%%%%%%%%%%%%%%%%%%%%%%%%%%%%%
\paragraph{Restrictions.}

Please note the following restrictions:
\begin{itemize}
\item
|\childdocmain| must be called with one argument \textit{main}
to ensure compatibility with earlier version of the package.
It must either be empty (|\childdocmain{}|)
or precisely match the filename of the main file in which it is specified.
See \secref{sec:detection} for further information.
\item
The filename \textit{main} must be specified without the |.tex| extension.
\item
The filename \textit{main} is case sensitive
(even in case-insensitive file systems)
due to internal string comparison.
\item
The argument \textit{main} should be fully expanded, it cannot be a macro.
\item
Subdirectories and special characters should be avoided in filenames.
\item
The command |\childdocmain{|\textit{main}|}| must be followed by a whitespace.
It should not be followed immediately by another command
or by a comment mark `|%|'.
This is because the \TeX{} parser reads the token immediately following
the argument of |\childdocmain| and puts it
at the beginning of every child section;
however, a white\-space is ignored.
\end{itemize}

%%%%%%%%%%%%%%%%%%%%%%%%%%%%%%%%%%%%%%%%
\paragraph{Content of Main File.}

It is advisable to place all content in the child files included by |\include|.
Any output contained in the main file will appear in all child documents
unless suppressed manually;
it cannot be suppressed automatically by the |\includeonly| directive
and thus should normally be avoided.
A method to include some content in the main file
by means of conditional processing is described in \secref{sec:conditional}.

%%%%%%%%%%%%%%%%%%%%%%%%%%%%%%%%%%%%%%%%
\paragraph{Page Numbering.}

When only a part of the document is compiled,
the appropriate numbering of pages
(as well as other status parameters)
is determined from the |.aux| files.
The latter contain information from previous passes.
However this information needs to propagate through
all intermediate child documents.
Therefore the page numbering in child documents may well
be inconsistent until the complete document is compiled at least once.

A useful (if unconventional) way to always ensure a consistent
page numbering is to restart the numbering in each child document
and denote the pages by `\textit{child}|.|\textit{page}'
where \textit{child} represents the chapter/section number of the child file.
This can be achieved by the command
|\numberwithin{page}{|\textit{child}|}|
of the \textsf{amsmath} package
where \textit{child} can be |chapter| or |section|
depending on the chosen structuring.
Alternatively, one can modify the macro |\thepage| appropriately
and reset the counter |page| at the start of each child file.

%%%%%%%%%%%%%%%%%%%%%%%%%%%%%%%%%%%%%%%%%%%%%%%%%%%%%%%%%%%%%%%%%%%%%%%%%%%%%%%%
\subsection{Conditional Processing}
\label{sec:conditional}

The package provides a mechanism to compile different versions
of a document. To customise the versions further some conditional processing
can come in handy to distinguish which version is being compiled.
The package provides two macros to describe the compilation context:

%%%%%%%%%%%%%%%%%%%%%%%%%%%%%%%%%%%%%%%%
\DescribeMacro{\ifchilddoc}
The conditional |\ifchilddoc| distinguishes between the compilation of
child documents and the main document:
%
\begin{center}
|\ifchilddoc |\textit{child-code}| |[|\||else |\textit{main-code}]| \||fi|
\end{center}

%%%%%%%%%%%%%%%%%%%%%%%%%%%%%%%%%%%%%%%%
\DescribeMacro{\childdocname}
\DescribeMacro{\childdocjob}
The macro |\childdocname| contains the filename (without extension)
of the main or child file being processed.
Note that |\childdocjob| will always contain the name of the main file.

%%%%%%%%%%%%%%%%%%%%%%%%%%%%%%%%%%%%%%%%
\paragraph{Title Page.}

Conditional processing can be used to include a title or banner page
in the main document when proper precautions are taken.
Importantly, the code in the main file should ensure that the page counter
(as well as other status parameters which are stored in the |.aux| files)
takes the same value after the conditional processing.
Otherwise the page numbers may take divergent values
depending on which part is compiled.

For example, a title page could be declared by:
%
\begin{center}
\begin{tabular}{l}
|\ifchilddoc\||else|\\
|\addtocounter{page}{-1}|\\
\textit{code for title page}\\
|\newpage|\\
|\||fi|
\end{tabular}
\end{center}
%
A banner page for the child documents can be generated by:
%
\begin{center}
\begin{tabular}{l}
|\ifchilddoc|\\
|\addtocounter{page}{-1}|\\
\textit{code for banner page}\\
|\newpage|\\
|\||fi|
\end{tabular}
\end{center}
%
Here one could write a message such as:
\begin{center}
|This is the part \childdocname{} of \childdocjob{}.|
\end{center}

%%%%%%%%%%%%%%%%%%%%%%%%%%%%%%%%%%%%%%%%%%%%%%%%%%%%%%%%%%%%%%%%%%%%%%%%%%%%%%%%
\subsection{Flags}
\label{sec:flags}

The package makes it easy to generate different versions
of the main or child documents.
To this end compilation flags can be defined
and assigned different default values.
They will be particularly useful in conjunction
with the forwarding mechanism described in \secref{sec:forward}.

For example, it may be useful to have a flag |\version|
which can be set to |draft| or |final|.
The document source will contain some conditional code
depending on the value of |\version|.
Suppose further, the flag should default to |final| for the main file
and to |draft| for child files
which is a natural assignment for editing the document.
This is achieved by placing the following code
in the preamble of the main document
(below the |\childdocmain| directive):
%
\begin{center}
\begin{tabular}{l}
|\ifchilddoc|\\
|\providecommand{\version}{draft}|\\
|\||else|\\
|\providecommand{\version}{final}|\\
|\||fi|
\end{tabular}
\end{center}
%
The definition by |\providecommand| makes sure
that previous definitions are not overwritten.
Further statements |\providecommand{\version}{...}|
can thus be added before the above code to override it.

For the main file, one might add a line
(between |\childdocmain| and the above block)
%
\begin{center}
|%\ifchilddoc\||else\providecommand{\version}{draft}\||fi|
\end{center}
%
which can be uncommented to produce a draft version.
Likewise one can add a line to the very top of a child file
(above the |\childdocof{|\textit{main}|}| directive)
%
\begin{center}
|%\providecommand{\version}{final}|
\end{center}
%
which can be uncommented to produce the final version of this child document.

%%%%%%%%%%%%%%%%%%%%%%%%%%%%%%%%%%%%%%%%%%%%%%%%%%%%%%%%%%%%%%%%%%%%%%%%%%%%%%%%
\subsection{Forwarding}
\label{sec:forward}

Different versions of the main or child documents
using compilation flags as described in \secref{sec:flags}
can be (permanently) stored in different files
for convenient compilation, viewing and distribution.
To this end, the package defines a command
to pass on compilation to a different file:

%%%%%%%%%%%%%%%%%%%%%%%%%%%%%%%%%%%%%%%%
\DescribeMacro{\childdocforward}
The command |\childdocforward| redirects processing to
another source file:
%
\begin{center}
\begin{tabular}{l}
|\input{childdoc.def}|\\
|\childdocforward[|\textit{main}|]{|\textit{dest}|}|\\
\end{tabular}
\end{center}
%
The argument \textit{dest} is the destination file
(without extension).
It should be the main file or one of the child files.
Note that further \textsf{childdoc} directives
such as |\childdocof| and |\childdocforward|
in the indicated file will be processed in this form.
The optional argument \textit{main}
passes on directly to the main file \textit{main}
while pretending to compile the child \textit{dest}.
This form behaves as if \textit{dest}
issues |\childdocof{|\textit{main}|}| right away,
and no further \textsf{childdoc} directives will be processed.

%%%%%%%%%%%%%%%%%%%%%%%%%%%%%%%%%%%%%%%%
\DescribeMacro{\...prefix}
In the alternative form |\childdocforwardprefix|,
%
\begin{center}
\begin{tabular}{l}
|\input{childdoc.def}|\\
|\childdocforwardprefix[|\textit{main}|]{|\textit{prefix}|}{|\textit{dest}|}|
\end{tabular}
\end{center}
%
the destination file is determined by a pattern
depending on the current file:
To make this work, the current file must be called
`{\textit{prefix}\hspace{0.2em}\textit{suffix}}'
with \textit{prefix} matching precisely the argument.
Processing is then passed on to the file
`{\textit{dest}\hspace{0.2em}\textit{suffix}}'.
Surely, the same effect is achieved by
directly specifying the
argument `{\textit{dest}\hspace{0.2em}\textit{suffix}}'
in the first form.
However, that requires to set up a different file
for each child. With the alternative form of the command
all these files can have exactly the same content
which simplifies setting them up and maintaining them.

For example, the following file |draft.tex|
with a compilation flag |\version| as described in \secref{sec:flags}
compiles the main document as a draft:
%
\begin{center}
\begin{tabular}{l}
|\def\version{draft}|\\
|\input{childdoc.def}|\\
|\childdocforward{|\textit{main}|}|
\end{tabular}
\end{center}
%
Likewise, the following files |final|\textit{nn}|.tex|
compile the final version of the child document
|child|\textit{nn}|.tex|:
%
\begin{center}
\begin{tabular}{l}
|\def\version{final}|\\
|\input{childdoc.def}|\\
|\childdocforwardprefix{final}{child}|
\end{tabular}
\end{center}
%

Note that when several versions of a main file and/or of each child file
are to be generated, it may be convenient to set up a |Makefile| or
shell script to automatise the process.

%%%%%%%%%%%%%%%%%%%%%%%%%%%%%%%%%%%%%%%%%%%%%%%%%%%%%%%%%%%%%%%%%%%%%%%%%%%%%%%%
\subsection{Command Line Processing}
\label{sec:commandline}

The effect of redirection files can also be achieved by invoking
the \LaTeX{} compiler with a more elaborate command line.
Most conveniently this should be done as part
of a shell script or a |Makefile|.

When using \textsf{childdoc} in the main file, the following
command lines effectively perform a redirection
(note that depending on the shell being used,
backslashes may have to be doubled: `|\|' $\to$ `|\\|'):
%
\begin{center}
|... -jobname "|\textit{target}|" |\\|"|[\textit{flags}]%
|\input{childdoc.def}\childdocforward[|\textit{main}|]{|\textit{dest}|}"|
\end{center}
%
Here \textit{target} is the name of the output file,
\textit{main} is the name of the main file
and \textit{dest} is the name of the main or child file to be processed
(all filenames without extensions).
The optional argument \textit{main} can be omitted
if \textit{main} matches \textit{dest}.
Optionally, compilation \textit{flags} can be defined via |\def| commands.
This command line makes the \TeX{} engine believe
it is compiling the file \textit{target}
whose content is specified as the latter parameter.
The provided code then forwards the processing to
\textit{main} or \textit{dest} as described in \secref{sec:forward}.

%%%%%%%%%%%%%%%%%%%%%%%%%%%%%%%%%%%%%%%%%%%%%%%%%%%%%%%%%%%%%%%%%%%%%%%%%%%%%%%%
\subsection{Include by Input}
\label{sec:input}

Including child documents by |\include| has some restrictions by design.
Most notably, the content of a child document always occupies
its own set of pages; pages cannot be shared between child documents.
Usually, this behaviour makes perfect sense
because each child document contain an essential part of the document.
However, in some situations it may be desirable to compose
a document from a collection of parts
without having mandatory page breaks between then.
For this case, the package
provides a mechanism to include parts
by |\input| which can also be processed individually.
However, by construction this mechanism
requires manual handling of the content to be output.

%%%%%%%%%%%%%%%%%%%%%%%%%%%%%%%%%%%%%%%%
\DescribeMacro{\ifchilddocmanual}
The main file should be prepared as usual, see \secref{sec:include}.
However, the document body must make a distinction
between processing of an individual part and of the main document, e.g.:
%
\begin{center}
\begin{tabular}{l}
|\ifchilddocmanual|\\
|\input{\childdocname}|\\
|\||else|\\
\textit{document body with }|\input{|\textit{part}|}|\\
|\||fi|
\end{tabular}
\end{center}
%
The conditional |\ifchilddocmanual| is true whenever
a part to be included by |\input| is being compiled,
and the name of the part is stored in |\childdocname|.

%%%%%%%%%%%%%%%%%%%%%%%%%%%%%%%%%%%%%%%%
\DescribeMacro{\childdocby}
Each part to be included by |\input| should start with:
%
\begin{center}
\begin{tabular}{l}
|\input{childdoc.def}|\\
|\childdocby{|\textit{main}|}|\\
\end{tabular}
\end{center}
%
The directive |\childdocby| is similar to |\childdocof|
described in \secref{sec:include},
but the subsequent selection of content must be done manually.
To that end, both |\ifchilddoc| and |\ifchilddocmanual|
will be true upon processing of a part,
and the name of the part is stored in |\childdocname|.
Note that |\jobname| will be set to the filename of the current part
so that each part receives an individual |.aux| file
that does not interfere with the |.aux| file(s) of the main document.
This behaviour can be altered by the alternative form
|\childdocby[*]{|\textit{main}|}| (with a non-empty optional argument)
which uses the |.aux| file of the main document
by setting |\jobname| to \textit{main}.

%%%%%%%%%%%%%%%%%%%%%%%%%%%%%%%%%%%%%%%%%%%%%%%%%%%%%%%%%%%%%%%%%%%%%%%%%%%%%%%%
\subsection{Driver Development}
\label{sec:driver}

The \textsf{childdoc} mechanism can also be use for the development
of definition files such as \LaTeX{} styles or classes.
This case differs from the above setup with multiple parts
included by |\include| in that no |\includeonly| should be invoked.
This can be achieved by starting the include file
(before |\ProvidesPackage|) with:
%
\begin{center}
\begin{tabular}{l}
|\input{childdoc.def}|\\
|\childdocforward{|\textit{main}|}|\\
\end{tabular}
\end{center}
%
or alternatively with:
%
\begin{center}
\begin{tabular}{l}
|\input{childdoc.def}|\\
|\childdocby{|\textit{main}|}|\\
\end{tabular}
\end{center}
%
Both forms have slightly different effects as described above.
The main file is prepared as usual, see \secref{sec:include}.

%%%%%%%%%%%%%%%%%%%%%%%%%%%%%%%%%%%%%%%%%%%%%%%%%%%%%%%%%%%%%%%%%%%%%%%%%%%%%%%%
\subsection{Legacy Detection}
\label{sec:detection}

The directive |\childdocmain| in the main file can detect
whether the complete document or merely a child is to be compiled
even without using the directive |\childdocof|.
This method is deprecated because it is less robust
and there is no compelling reason to use it;
it is merely provided for backward compatibility
and it may be removed in future versions.

If the detection mechanism is to be used,
it is mandatory to correctly specify
the filename of the main file as the argument of |\childdocmain|:
%
\begin{center}
\begin{tabular}{l}
|\input{childdoc.def}|\\
|\childdocmain{|\textit{main}|}|\\
\end{tabular}
\end{center}
%
If |\jobname| does not match the argument \textit{main} of |\childdocmain|,
it is assumed that |\jobname| points to the child file to be compiled.
When using |\childdocmain| with the main file specified as argument,
it suffices to start a child file
with just |\input{|\textit{main}|}|
without loading of the package and using |\childdocof|.
If instead all processing is done
with the appropriate \textsf{childdoc} directives,
the argument of \textit{main} of |\childdocmain| can be empty.

An alternative version of the command line processing described
in \secref{sec:commandline} using the detection mechanism reads:
%
\begin{center}
|... -jobname "|\textit{target}|" "|[\textit{flags}]%
[|\def\jobname{|\textit{dest}|}|]|\input{|\textit{main}|}"|
\end{center}

%%%%%%%%%%%%%%%%%%%%%%%%%%%%%%%%%%%%%%%%%%%%%%%%%%%%%%%%%%%%%%%%%%%%%%%%%%%%%%%%
\subsection{Manual Code}
\label{sec:manual}

In case one cannot be certain whether the definitions file |childdoc.def|
is installed on the target \TeX{} distribution
and one prefers not to ship it,
it is conceivable to paste a few relevant commands into the sources.

To that end, drop all statements |\input{childdoc.def}|
and perform the replacements as outlined below.
Instead of |\childdocmain{|\textit{main}|}| add the following code
to the top of the main file:
%
\begin{center}
\begin{tabular}{l}
|\||ifdefined\childdocname\endinput\||fi\newif\ifchilddoc|\\
|\edef\childdocname{\scantokens\expandafter{\jobname\noexpand}}|\\
|\def\childdocmain{|\textit{main}|}\||ifx\childdocmain\childdocname\||else|\\
|\childdoctrue\includeonly{\childdocname}\let\jobname\childdocmain\||fi|\\
\end{tabular}
\end{center}
%
Instead of |\childdocof{|\textit{main}|}| just include the main file
at the top of each child file:
%
\begin{center}
|\input{|\textit{main}|}|
\end{center}
%
A simple redirection |\childdocforward{|\textit{dest}|}| is achieved by:
%
\begin{center}
|\def\jobname{|\textit{dest}|}\input{\jobname}|
\end{center}
%
The redirection with prefix
|\childdocforwardprefix[|\textit{prefix}|]{|\textit{dest}|}|
is accomplished by:
%
\begin{center}
\begin{tabular}{l}
|{\edef\jobname{\scantokens\expandafter{\jobname\noexpand}}|\\
|\def\redirectjob |\textit{prefix}|#1~~~{\gdef\jobname{|\textit{dest}|#1}}|\\
|\expandafter\redirectjob\jobname~~~}\input{\jobname}|
\end{tabular}
\end{center}

In an alternative approach,
child documents can be compiled by a specific command line
without additional code or specific definitions:
%
\begin{center}
|... -jobname "|\textit{target}|" "|[\textit{flags}]%
|\includeonly{|\textit{dest}|}\input{|\textit{main}|}"|
\end{center}
%

%%%%%%%%%%%%%%%%%%%%%%%%%%%%%%%%%%%%%%%%%%%%%%%%%%%%%%%%%%%%%%%%%%%%%%%%%%%%%%%%
%%%%%%%%%%%%%%%%%%%%%%%%%%%%%%%%%%%%%%%%%%%%%%%%%%%%%%%%%%%%%%%%%%%%%%%%%%%%%%%%
\section{Information}

%%%%%%%%%%%%%%%%%%%%%%%%%%%%%%%%%%%%%%%%%%%%%%%%%%%%%%%%%%%%%%%%%%%%%%%%%%%%%%%%
\subsection{Copyright}

Copyright \copyright{} 2017--2018 Niklas Beisert

This work may be distributed and/or modified under the
conditions of the \LaTeX{} Project Public License, either version 1.3
of this license or (at your option) any later version.
The latest version of this license is in
  \url{http://www.latex-project.org/lppl.txt}
and version 1.3 or later is part of all distributions of \LaTeX{}
version 2005/12/01 or later.

This work has the LPPL maintenance status `maintained'.

The Current Maintainer of this work is Niklas Beisert.

This work consists of the files |README.txt|, |childdoc.ins| and |childdoc.dtx|
as well as the derived files |childdoc.def|, |cdocsamp.tex|
with |cdocsch1.tex|, |cdocsch2.tex|, |cdocspt3.tex|, |cdocspt4.tex|,
|cdocsdrf.tex|, |cdocsfn1.tex|, |cdocsfn2.tex|
as well as |childdoc.pdf|.

%%%%%%%%%%%%%%%%%%%%%%%%%%%%%%%%%%%%%%%%%%%%%%%%%%%%%%%%%%%%%%%%%%%%%%%%%%%%%%%%
\subsection{Files and Installation}

The package consists of the files:
%
\begin{center}
\begin{tabular}{ll}
    |README.txt|   & readme file \\
    |childdoc.ins| & installation file \\
    |childdoc.dtx| & source file \\
    |childdoc.def| & definition file \\
    |cdocsamp.tex| & sample main file \\
    |cdocsch1.tex| & sample include file \\
    |cdocsch2.tex| & sample include file \\
    |cdocspt3.tex| & sample part file \\
    |cdocspt4.tex| & sample part file \\
    |cdocsdrf.tex| & sample redirection file \\
    |cdocsfn1.tex| & sample redirection file \\
    |cdocsfn2.tex| & sample redirection file \\
    |childdoc.pdf| & manual
\end{tabular}
\end{center}
%
The distribution consists of the files
|README.txt|, |childdoc.ins| and |childdoc.dtx|.
%
\begin{itemize}
\item
Run (pdf)\LaTeX{} on |childdoc.dtx|
to compile the manual |childdoc.pdf| (this file).
\item
Run \LaTeX{} on |childdoc.ins| to create the definitions file |childdoc.def|
and the sample |cdocsamp.tex| with include files
|cdocsch1.tex|, |cdocsch2.tex|, |cdocspt3.tex|, |cdocspt4.tex|,
|cdocsdrf.tex|, |cdocsfn1.tex|, |cdocsfn2.tex|.
Then copy the file |childdoc.def| to an appropriate directory of your \LaTeX{}
distribution, e.g.\ \textit{texmf-root}|/tex/latex/childdoc|.
\end{itemize}

%%%%%%%%%%%%%%%%%%%%%%%%%%%%%%%%%%%%%%%%%%%%%%%%%%%%%%%%%%%%%%%%%%%%%%%%%%%%%%%%
\subsection{Related CTAN Packages}

There are several other packages which offer a similar functionality:
%
\begin{itemize}
\item
The packages
\href{http://ctan.org/pkg/docmute}{\textsf{docmute}},
\href{http://ctan.org/pkg/includex}{\textsf{includex}} and
\href{http://ctan.org/pkg/standalone}{\textsf{standalone}}
provide commands to include only the document body of
a child file thus allowing both files to be compiled individually.
\item
The packages \href{http://ctan.org/pkg/subdocs}{\textsf{subdocs}}
and \href{http://ctan.org/pkg/subfiles}{\textsf{subfiles}}
provide structures in which the main and child documents can be
encapsulated and allowing them to be compiled individually.
The inclusion mechanism is different from the conventional |\include|.
\item
The package \href{http://ctan.org/pkg/combine}{\textsf{combine}}
is an elaborate solution to combine several documents into one.
\end{itemize}
%
See also the CTAN topic \href{http://ctan.org/topic/subdocs}{\textsf{subdocs}}
for further related packages.
The present package differs from the above solutions in that
a document structure constructed with the conventional |\include| mechanism
just needs two extra commands at the top of every file
such that all constituent files can be compiled individually.

%%%%%%%%%%%%%%%%%%%%%%%%%%%%%%%%%%%%%%%%%%%%%%%%%%%%%%%%%%%%%%%%%%%%%%%%%%%%%%%%
%\subsection{Feature Suggestions}
%
%The following is a list of features which may be useful for future
%versions of this package:
%%
%\begin{itemize}
%\item
%\ldots
%\end{itemize}

%%%%%%%%%%%%%%%%%%%%%%%%%%%%%%%%%%%%%%%%%%%%%%%%%%%%%%%%%%%%%%%%%%%%%%%%%%%%%%%%
\subsection{Revision History}

%%%%%%%%%%%%%%%%%%%%%%%%%%%%%%%%%%%%%%%%
\paragraph{v2.0:} 2018/12/30

\begin{itemize}
\item
immediate forward processing
\item
added |\childdocby| mechanism
\item
manual restructured
\end{itemize}

%%%%%%%%%%%%%%%%%%%%%%%%%%%%%%%%%%%%%%%%
\paragraph{v1.6:} 2018/01/17

\begin{itemize}
\item
application for development of include files
\item
corrections to manual
\end{itemize}

%%%%%%%%%%%%%%%%%%%%%%%%%%%%%%%%%%%%%%%%
\paragraph{v1.5:} 2017/05/21

\begin{itemize}
\item
more complete structuring introduced
\item
|\childdocof| introduced
\item
|\childdoc| renamed to |\childdocmain|
\item
|\childredirect| renamed to |\childdocforward| and |\childdocforwardprefix|
and functionality expanded
\end{itemize}

%%%%%%%%%%%%%%%%%%%%%%%%%%%%%%%%%%%%%%%%
\paragraph{v1.0:} 2017/04/27

\begin{itemize}
\item
manual and install package
\item
first version published on CTAN
\end{itemize}

%%%%%%%%%%%%%%%%%%%%%%%%%%%%%%%%%%%%%%%%
\paragraph{v0.6:} 2017/04/26

\begin{itemize}
\item
redirection mechanism added
\end{itemize}

%%%%%%%%%%%%%%%%%%%%%%%%%%%%%%%%%%%%%%%%
\paragraph{v0.5:} 2017/04/26

\begin{itemize}
\item
functionality in definition file
\end{itemize}


%%%%%%%%%%%%%%%%%%%%%%%%%%%%%%%%%%%%%%%%%%%%%%%%%%%%%%%%%%%%%%%%%%%%%%%%%%%%%%%%
%%%%%%%%%%%%%%%%%%%%%%%%%%%%%%%%%%%%%%%%%%%%%%%%%%%%%%%%%%%%%%%%%%%%%%%%%%%%%%%%
%%%%%%%%%%%%%%%%%%%%%%%%%%%%%%%%%%%%%%%%%%%%%%%%%%%%%%%%%%%%%%%%%%%%%%%%%%%%%%%%
\appendix

\settowidth\MacroIndent{\rmfamily\scriptsize 000\ }

 \DocInput{childdoc.dtx}

\end{document}
%</driver>
% \fi
%
% %%%%%%%%%%%%%%%%%%%%%%%%%%%%%%%%%%%%%%%%%%%%%%%%%%%%%%%%%%%%%%%%%%%%%%%%%%%%%%
% %%%%%%%%%%%%%%%%%%%%%%%%%%%%%%%%%%%%%%%%%%%%%%%%%%%%%%%%%%%%%%%%%%%%%%%%%%%%%%
% \section{Sample}
%\iffalse
%<*samplemain>
%\fi
%
% The following presents a sample document
% with two chapters, two parts, a title page,
% a compile flag as well as three forwarding files to set the flag.
% It consists of eight |.tex| files:
% \begin{center}
% \begin{tabular}{ll}
% |cdocsamp.tex|&main file\\
% |cdocsch1.tex|&include file for chapter 1\\
% |cdocsch2.tex|&include file for chapter 2\\
% |cdocspt3.tex|&include file for part 3\\
% |cdocspt4.tex|&include file for part 4\\
% |cdocsdrf.tex|&forwarding file for main file in draft mode\\
% |cdocsfi1.tex|&forwarding file for final version of chapter 1\\
% |cdocsfi2.tex|&forwarding file for final version of chapter 2\\
% \end{tabular}
% \end{center}
% Each of the eight files can be compiled directly by the \LaTeX{} compiler.
%
% %%%%%%%%%%%%%%%%%%%%%%%%%%%%%%%%%%%%%%
% \paragraph{Main File.}
%
% The main file is called |cdocsamp.tex|.
%
% Load the \textsf{childdoc} definitions and
% declare the filename for the main document:
%    \begin{macrocode}
\input{childdoc.def}
\childdocmain{}
%    \end{macrocode}

% Optional override for |\version| flag:
%    \begin{macrocode}
%%\ifchilddoc\else\providecommand{\version}{draft}\fi
%    \end{macrocode}

% Define the default values for the |\version| flag
% (|final| for the main file and |draft| for childs):
%    \begin{macrocode}
\ifchilddoc
\providecommand{\version}{draft}
\else
\providecommand{\version}{final}
\fi
%    \end{macrocode}

% Load the standard document class:
%    \begin{macrocode}
\documentclass[12pt]{article}
%    \end{macrocode}

% Start the document body:
%    \begin{macrocode}
\begin{document}
%    \end{macrocode}

% Declare a title page.
% Print title, part of document being processed and version flag:
%    \begin{macrocode}
\addtocounter{page}{-1}
\begin{center}
{\LARGE\bfseries{}childdoc example\par}
\vspace{1cm}
\ifchilddoc
\ifchilddocmanual part\else chapter\fi:
`\childdocname' of `\childdocjob'\par
\else
main document: `\childdocjob'\par
\fi
version: \version\par
\end{center}
\newpage
%    \end{macrocode}

% Manually include selected file,
% otherwise process as usual:
%    \begin{macrocode}
\ifchilddocmanual
\section*{part `\childdocname'}
\input{\childdocname}
\else
%    \end{macrocode}

% Include the two chapters:
%    \begin{macrocode}
\include{cdocsch1}
\include{cdocsch2}
%    \end{macrocode}

% Include the two parts unless only chapters should be displayed:
%    \begin{macrocode}
\ifchilddoc\else
\section{part three}
\input{cdocspt3}
\section{part four}
\input{cdocspt4}
\fi
%    \end{macrocode}

% Process as usual until here:
%    \begin{macrocode}
\fi
%    \end{macrocode}

% End of document body:
%    \begin{macrocode}
\end{document}
%    \end{macrocode}
%\iffalse
%</samplemain>
%\fi
%
% %%%%%%%%%%%%%%%%%%%%%%%%%%%%%%%%%%%%%%
% \paragraph{Chapter Include Files.}
%
% The include files are called |cdocsch1.tex| and |cdocsch2.tex|.
%
%\iffalse
%<*samplechap1|samplechap2>
%\fi

% Optional override for |\version| flag:
%    \begin{macrocode}
%%\providecommand{\version}{final}
%    \end{macrocode}

% Include the main document:
%    \begin{macrocode}
\input{childdoc.def}
\childdocof{cdocsamp}
%    \end{macrocode}

%\iffalse
%</samplechap1|samplechap2>
%\fi
%
%\iffalse
%<*samplechap1>
%\fi
% Some text for chapter 1:
%    \begin{macrocode}
\section{one}
some text in chapter one
%    \end{macrocode}

%\iffalse
%</samplechap1>
%\fi
% Some text for chapter 2:
%\iffalse
%<*samplechap2>
%\fi
%    \begin{macrocode}
\section{two}
more text in chapter two
%    \end{macrocode}

%\iffalse
%</samplechap2>
%\fi
%
% %%%%%%%%%%%%%%%%%%%%%%%%%%%%%%%%%%%%%%
% \paragraph{Part Include Files.}
%
% The include files are called |cdocspt3.tex| and |cdocspt4.tex|.
%
%\iffalse
%<*samplepart3|samplepart4>
%\fi

% Optional override for |\version| flag:
%    \begin{macrocode}
%%\providecommand{\version}{final}
%    \end{macrocode}

% Include the main document:
%    \begin{macrocode}
\input{childdoc.def}
\childdocby{cdocsamp}
%    \end{macrocode}

%\iffalse
%</samplepart3|samplepart4>
%\fi
%
%\iffalse
%<*samplepart3>
%\fi
% Some text for part 3:
%    \begin{macrocode}
some text in part three
%    \end{macrocode}

%\iffalse
%</samplepart3>
%\fi
% Some text for part 4:
%\iffalse
%<*samplepart4>
%\fi
%    \begin{macrocode}
more text in part four
%    \end{macrocode}

%\iffalse
%</samplepart4>
%\fi
%
% %%%%%%%%%%%%%%%%%%%%%%%%%%%%%%%%%%%%%%
% \paragraph{Forwarding for a Complete Draft.}
%
% The following forwarding file |cdocsdrf.tex|
% compiles the main document in draft mode:
%\iffalse
%<*sampledraft>
%\fi
%    \begin{macrocode}
\def\version{draft}
\input{childdoc.def}
\childdocforward{cdocsamp}
%    \end{macrocode}

%\iffalse
%</sampledraft>
%\fi
%
% %%%%%%%%%%%%%%%%%%%%%%%%%%%%%%%%%%%%%%
% \paragraph{Forwarding for Final Version of the Chapters.}
%
% The following forwarding files |cdocsfn1.tex| and |cdocsfn2.tex|
% (with identical content)
% compile the final versions of the child documents
% |cdocsch1.tex| and |cdocsch2.tex|, respectively:
%\iffalse
%<*samplefinal>
%\fi
%    \begin{macrocode}
\def\version{final}
\input{childdoc.def}
\childdocforwardprefix[cdocsamp]{cdocsfn}{cdocsch}
%    \end{macrocode}

%\iffalse
%</samplefinal>
%\fi
%
% %%%%%%%%%%%%%%%%%%%%%%%%%%%%%%%%%%%%%%
% \paragraph{Command Line Processing.}
%
% The following three command lines generate the output files
% |cdocscld|, |cdocscl1| and |cdocscl2|
% which should be identical to
% |cdocsdrf|, |cdocsch1| and |cdocsfn2|, respectively:
% \begin{center}
% \begin{tabular}{l}
% |latex -jobname cdocscld \|\\
% |  "\def\version{draft}\input{childdoc.def}\childdocforward{cdocsamp}"|\\
% |latex -jobname cdocscl1 \|\\
% |  "\input{childdoc.def}\childdocforward[cdocsamp]{cdocsch1}"|\\
% |latex -jobname cdocscl2 \|\\
% |  "\def\version{final}\input{childdoc.def}\childdocforward{cdocsch2}"|
% \end{tabular}
% \end{center}
% Note that the trailing backslash on each first line
% merely continues the input to the second line
% (for convenient cut ant paste).
% Furthermore, the command |latex| can be replaced by any
% of its alternative versions such as |pdflatex|.
%
% %%%%%%%%%%%%%%%%%%%%%%%%%%%%%%%%%%%%%%%%%%%%%%%%%%%%%%%%%%%%%%%%%%%%%%%%%%%%%%
% %%%%%%%%%%%%%%%%%%%%%%%%%%%%%%%%%%%%%%%%%%%%%%%%%%%%%%%%%%%%%%%%%%%%%%%%%%%%%%
% \section{Implementation}
%\iffalse
%<*package>
%\fi
%
% This section describes the definitions file |childdoc.def|.

% The definitions cannot be loaded using |\usepackage| or |\RequirePackage|
% which has a mechanism to prevent loading a style file more than once.
% When loading the definitions by means of |\input|
% multiple instances have to be prevented manually:
%\iffalse
%This code needs to be before the `\ProvidesFile' directive
%which is defined at the beginning of this file.
%Therefore it is also placed there and commented out here.
%</package>
%<*discard>
%\fi
%    \begin{macrocode}
\ifdefined\childdocmain\endinput\fi
%    \end{macrocode}
%\iffalse
%</discard>
%<*package>
%\fi
%
% \macro{\ifchilddoc}
% \macro{\ifchilddocmanual}
% The conditional |\ifchilddoc| tells whether a
% child (true) or main (false) document is being compiled.
% The conditional |\ifchilddocmanual| tells whether
% the |\includeonly| mechanism is used (false) or
% the selection of child files must be performed manually (true).
% The definitions initialise to false:
%    \begin{macrocode}
\newif\ifchilddoc
\newif\ifchilddocmanual
%    \end{macrocode}

% \macro{\childdocname}
% \macro{\childdocjob}
% The macro |\childdocname| stores the name of the main document
% to be compiled. The macro |\childdocjob| stores the name of
% the document on which the \LaTeX{} compiler was originally invoked.
% The content of |\jobname| cannot be compared
% to filenames specified in the source due to different catcodes.
% The following code rescans |\jobname|, stores the result
% in |\childdocname| and saves a copy in |\childdocjob|:
%    \begin{macrocode}
\edef\childdocname{\scantokens\expandafter{\jobname\noexpand}}
\let\childdocjob\childdocname
%    \end{macrocode}

% \macro{\childdocdisable}
% The macro |\childdocdisable| prevents the main file
% from being processed more than once.
% At this stage, the main document command |\childdocmain|
% is assumed to be called once again where it should do nothing.
% Any subsequent call to it should prevent
% a secondary processing of the main document
% It overwrites the forwarding commands
% |\childdocof| and |\childdocforward|
% with empty macros to prevent further inclusions of the main document:
%    \begin{macrocode}
\newcommand{\childdocdisable}
{
  \renewcommand{\childdocmain}[1]{\renewcommand{\childdocmain}[1]{\endinput}}
  \renewcommand{\childdocof}[1]{}
  \renewcommand{\childdocby}[2][]{}
  \renewcommand{\childdocforward}[2][]{}
  \renewcommand{\childdocdisable}{}
}
%    \end{macrocode}

% \macro{\childdocmain}
% The macro |\childdocmain| is to be called at the top of the main file
% with nothing or the main filename (without extension) as argument.
% First, it breaks loops.
% If the argument is not empty and does not match |\childdocname|
% (which is set by the first inclusion of |childdoc.def|),
% |\ifchilddoc| is set to true, |\includeonly| is applied to the child file
% and |\jobname| is set to the main file
% (for proper handling of |.aux| files):
%    \begin{macrocode}
\newcommand{\childdocmain}[1]
{
  \childdocdisable\childdocmain{}
  \if?#1?\else
    \begingroup
      \def\childdoctmp{#1}
      \ifx\childdoctmp\childdocname
        \def\childdoctmp{}
      \else
        \def\childdoctmp
        {
          \childdoctrue
          \includeonly{\childdocname}
          \def\childdocjob{#1}
          \def\jobname{#1}
        }
      \fi
      \expandafter
    \endgroup
    \childdoctmp
  \fi
}
%    \end{macrocode}

% \macro{\childdocof}
% The command |\childdocof| redirects
% compilation to the main file |#1|.
%    \begin{macrocode}
\newcommand{\childdocof}[1]
{
  \childdocdisable
  \childdoctrue
  \includeonly{\childdocname}
  \def\jobname{#1}
  \def\childdocjob{#1}
  \input{#1}
}
%    \end{macrocode}

% \macro{\childdocby}
% The command |\childdocby| ....
%    \begin{macrocode}
\newcommand{\childdocby}[2][]
{
  \childdocdisable
  \childdoctrue
  \childdocmanualtrue
  \if?#1?\else
    \def\jobname{#2}
  \fi
  \def\childdocjob{#2}
  \input{#2}
  \endinput
}
%    \end{macrocode}

% \macro{\childdocforward}
% The command |\childdocforward| redirects
% compilation to the main file or
% (if the optional argument is given) a child file.
% Parameters are set as if the main file
% or a child file starting with |\childdocof| was compiled.
% Then compilation is handed over to the main file:
%    \begin{macrocode}
\newcommand{\childdocforward}[2][]
{
  \begingroup
    \if?#1?
      \def\childdoctmp
      {
        \def\childdocname{#2}
        \def\childdocjob{#2}
        \def\jobname{#2}
        \input{#2}
        \endinput
      }
    \else
      \def\childdoctmp
      {
        \childdocdisable
        \def\childdocname{#2}
        \childdoctrue
        \includeonly{#2}
        \def\childdocjob{#1}
        \def\jobname{#1}
        \input{#1}
        \endinput
      }
    \fi
    \expandafter
  \endgroup
  \childdoctmp
}
%    \end{macrocode}

% \macro{\childdocforwardprefix}
% The command |\childdocforwardprefix| redirects
% compilation to the main or a child file by means of a pattern.
% The prefix |#1| in the current filename is replaced by |#2|
% and the suffix of the current filename is kept
% (it is assumed that the filename does not contain the substring `|~~~|'
% which is used as a delimiter).
% Compilation is handed over to the new file by |\childdocforward|:
%    \begin{macrocode}
\newcommand{\childdocforwardprefix}[3][]
{
  \begingroup
    \def\childdocextract #2##1~~~{\def\childdoctmp{\childdocforward[#1]{#3##1}}}
    \expandafter\childdocextract\childdocname~~~
    \expandafter
  \endgroup
  \childdoctmp
}
%    \end{macrocode}

% \macro{\childdoc}
% The deprecated macro |\childdoc| is a legacy version of |\childdocmain|:
%    \begin{macrocode}
\newcommand{\childdoc}{\childdocmain}
%    \end{macrocode}

% \macro{\childdocredirect}
% The deprecated macro |\childdocredirect| is a legacy version
% of |\childdocforward| and |\childdocforwardprefix|:
%    \begin{macrocode}
\newcommand{\childdocredirect}[2][]
{
  \begingroup
    \if?#1?
      \def\childdoctmp{\childdocforward{#2}}
    \else
      \def\childdoctmp{\childdocforwardprefix{#1}{#2}}
    \fi
    \expandafter
  \endgroup
  \childdoctmp
}
%    \end{macrocode}

%\iffalse
%</package>
%\fi
%
\endinput
|\\
|\childdocforward{|\textit{main}|}|\\
\end{tabular}
\end{center}
%
or alternatively with:
%
\begin{center}
\begin{tabular}{l}
|% \iffalse
%
% childdoc.dtx Copyright (C) 2017-2018 Niklas Beisert
%
% This work may be distributed and/or modified under the
% conditions of the LaTeX Project Public License, either version 1.3
% of this license or (at your option) any later version.
% The latest version of this license is in
%   http://www.latex-project.org/lppl.txt
% and version 1.3 or later is part of all distributions of LaTeX
% version 2005/12/01 or later.
%
% This work has the LPPL maintenance status `maintained'.
%
% The Current Maintainer of this work is Niklas Beisert.
%
% This work consists of the files childdoc.dtx and childdoc.ins
% and the derived files childdoc.def and cdocsamp.tex with
% cdocsch1.tex, cdocsch2.tex, cdocsdrf.tex, cdocsfn1.tex, cdocsfn2.tex.
%
%<package>\ifdefined\childdocmain\endinput\fi
%<package>\ProvidesFile{childdoc.def}[2018/12/30 v2.0 child document driver]
%<samplemain>\ProvidesFile{cdocsamp.tex}[2018/12/30 v2.0 sample for childdoc]
%<*driver>
%\ProvidesFile{childdoc.drv}[2018/12/30 v2.0 childdoc reference manual file]
\PassOptionsToClass{10pt,a4paper}{article}
\documentclass{ltxdoc}

\usepackage[margin=35mm]{geometry}
\usepackage{hyperref}
\usepackage{hyperxmp}
\usepackage[usenames]{color}

\hypersetup{colorlinks=true}
\hypersetup{pdfstartview=FitH}
\hypersetup{pdfpagemode=UseNone}
\hypersetup{pdfsource={}}
\hypersetup{pdflang={en-UK}}
\hypersetup{pdfcopyright={Copyright 2017-2018 Niklas Beisert.
  This work may be distributed and/or modified under the
  conditions of the LaTeX Project Public License, either version 1.3
  of this license or (at your option) any later version.}}
\hypersetup{pdflicenseurl={http://www.latex-project.org/lppl.txt}}
\hypersetup{pdfcontactaddress={ETH Zurich, ITP, HIT K,
  Wolfgang-Pauli-Strasse 27}}
\hypersetup{pdfcontactpostcode={8093}}
\hypersetup{pdfcontactcity={Zurich}}
\hypersetup{pdfcontactcountry={Switzerland}}
\hypersetup{pdfcontactemail={nbeisert@itp.phys.ethz.ch}}
\hypersetup{pdfcontacturl={http://people.phys.ethz.ch/\xmptilde nbeisert/}}

\newcommand{\secref}[1]{\hyperref[#1]{section \ref*{#1}}}

\parskip1ex
\parindent0pt
\let\olditemize\itemize
\def\itemize{\olditemize\parskip0pt}

\begin{document}

\title{The \textsf{childdoc} Package}
\hypersetup{pdftitle={The childdoc Package}}
\author{Niklas Beisert\\[2ex]
  Institut f\"ur Theoretische Physik\\
  Eidgen\"ossische Technische Hochschule Z\"urich\\
  Wolfgang-Pauli-Strasse 27, 8093 Z\"urich, Switzerland\\[1ex]
  \href{mailto:nbeisert@itp.phys.ethz.ch}
  {\texttt{nbeisert@itp.phys.ethz.ch}}}
\hypersetup{pdfauthor={Niklas Beisert}}
\hypersetup{pdfsubject={Manual for the LaTeX2e Package childdoc}}
\date{30 December 2018, \textsf{v2.0}}
\maketitle

\begin{abstract}\noindent
\textsf{childdoc} is a \LaTeXe{} package
that enables the direct compilation
of document sections included by |\include|
to individual files.
\end{abstract}

\begingroup
\parskip0ex
\tableofcontents
\endgroup

%%%%%%%%%%%%%%%%%%%%%%%%%%%%%%%%%%%%%%%%%%%%%%%%%%%%%%%%%%%%%%%%%%%%%%%%%%%%%%%%
%%%%%%%%%%%%%%%%%%%%%%%%%%%%%%%%%%%%%%%%%%%%%%%%%%%%%%%%%%%%%%%%%%%%%%%%%%%%%%%%
\section{Introduction}

\LaTeX{} provides a mechanism to structure a large document (such as a book)
into a main file and several child files (containing the chapters)
using the |\include| command.
This mechanism is beneficial for documents
which span hundreds of pages in order to
make the source file(s) more manageable.
Moreover, compilation can be restricted to
selected child files by means of the |\includeonly| command.
The latter feature can be used to reduce the compilation time while editing
(this was significantly more useful in the earlier days of \LaTeX{})
or to generate a smaller document which is easier to navigate.
Another application of |\includeonly| is to generate
documents consisting of selected parts of the complete document.

However, there are a few drawbacks of the plain |\include| mechanism:
\begin{itemize}
\item
The child files cannot be compiled on their own,
they can only be compiled via the main file.
A naive editing environment
(such as a text editor with an option
to have the current file processed by \LaTeX)
may require one to switch to the main file before compiling;
attempting to compile the child file produces errors.
\item
The main file must be modified (each time)
to adjust the |\includeonly| command
to the present needs. This easily leaves the main file in a messy state.
\item
The generated document will always carry the filename
of the main document. This is inconvenient if
several child files are to be compiled and
to be kept for distribution.
\end{itemize}

The present package provides a simple interface
to make child files individually compilable by \LaTeX{}.
Compiling a child file then has the same effect as compiling
the main file with an |\includeonly| command
to select the appropriate child.
Moreover the generated document will carry the name of the child
rather than the main file.
This resolves all three above issues.

This feature is meant to make the editing of books,
thesis documents and lecture notes somewhat more convenient.
However, the package can also be used efficiently for
composing a series of documents (such as exercise sheets)
which are typically distributed individually.
It then assists the author in generating the individual documents
(potentially in different versions)
as well as a document containing the collected series.
Another application is in developing style files
or other kinds of included material
where compilation of the style file could redirect
to a sample or test file.

%%%%%%%%%%%%%%%%%%%%%%%%%%%%%%%%%%%%%%%%%%%%%%%%%%%%%%%%%%%%%%%%%%%%%%%%%%%%%%%%
%%%%%%%%%%%%%%%%%%%%%%%%%%%%%%%%%%%%%%%%%%%%%%%%%%%%%%%%%%%%%%%%%%%%%%%%%%%%%%%%
\section{Usage}

First of all, the package \textsf{childdoc} is \emph{not} a standard
\LaTeXe{} |.sty| style file! Therefore it needs to be invoked in
a non-standard way.

%%%%%%%%%%%%%%%%%%%%%%%%%%%%%%%%%%%%%%%%%%%%%%%%%%%%%%%%%%%%%%%%%%%%%%%%%%%%%%%%
\subsection{Included Files}
\label{sec:include}

%%%%%%%%%%%%%%%%%%%%%%%%%%%%%%%%%%%%%%%%
\DescribeMacro{\childdocmain}
To use the package, add the commands
\begin{center}
\begin{tabular}{l}
|\input{childdoc.def}|\\
|\childdocmain{}|\\
\end{tabular}
\end{center}
at the very top of the main \LaTeX{} file,
in particular \emph{before} the |\documentclass| statement!
The argument of |\childdocmain| should be left empty
(but it must be present).

%%%%%%%%%%%%%%%%%%%%%%%%%%%%%%%%%%%%%%%%
\DescribeMacro{\childdocof}
Furthermore, add the commands
\begin{center}
\begin{tabular}{l}
|\input{childdoc.def}|\\
|\childdocof{|\textit{main}|}|\\
\end{tabular}
\end{center}
at the top of every child file \textit{child}
which is included by |\include{|\textit{child}|}|
from within the main file
(or at least for those files to be compiled individually).
The argument \textit{main} must be the filename of the main file.

There are a couple of
considerations in setting up the main and child documents:

%%%%%%%%%%%%%%%%%%%%%%%%%%%%%%%%%%%%%%%%
\paragraph{Restrictions.}

Please note the following restrictions:
\begin{itemize}
\item
|\childdocmain| must be called with one argument \textit{main}
to ensure compatibility with earlier version of the package.
It must either be empty (|\childdocmain{}|)
or precisely match the filename of the main file in which it is specified.
See \secref{sec:detection} for further information.
\item
The filename \textit{main} must be specified without the |.tex| extension.
\item
The filename \textit{main} is case sensitive
(even in case-insensitive file systems)
due to internal string comparison.
\item
The argument \textit{main} should be fully expanded, it cannot be a macro.
\item
Subdirectories and special characters should be avoided in filenames.
\item
The command |\childdocmain{|\textit{main}|}| must be followed by a whitespace.
It should not be followed immediately by another command
or by a comment mark `|%|'.
This is because the \TeX{} parser reads the token immediately following
the argument of |\childdocmain| and puts it
at the beginning of every child section;
however, a white\-space is ignored.
\end{itemize}

%%%%%%%%%%%%%%%%%%%%%%%%%%%%%%%%%%%%%%%%
\paragraph{Content of Main File.}

It is advisable to place all content in the child files included by |\include|.
Any output contained in the main file will appear in all child documents
unless suppressed manually;
it cannot be suppressed automatically by the |\includeonly| directive
and thus should normally be avoided.
A method to include some content in the main file
by means of conditional processing is described in \secref{sec:conditional}.

%%%%%%%%%%%%%%%%%%%%%%%%%%%%%%%%%%%%%%%%
\paragraph{Page Numbering.}

When only a part of the document is compiled,
the appropriate numbering of pages
(as well as other status parameters)
is determined from the |.aux| files.
The latter contain information from previous passes.
However this information needs to propagate through
all intermediate child documents.
Therefore the page numbering in child documents may well
be inconsistent until the complete document is compiled at least once.

A useful (if unconventional) way to always ensure a consistent
page numbering is to restart the numbering in each child document
and denote the pages by `\textit{child}|.|\textit{page}'
where \textit{child} represents the chapter/section number of the child file.
This can be achieved by the command
|\numberwithin{page}{|\textit{child}|}|
of the \textsf{amsmath} package
where \textit{child} can be |chapter| or |section|
depending on the chosen structuring.
Alternatively, one can modify the macro |\thepage| appropriately
and reset the counter |page| at the start of each child file.

%%%%%%%%%%%%%%%%%%%%%%%%%%%%%%%%%%%%%%%%%%%%%%%%%%%%%%%%%%%%%%%%%%%%%%%%%%%%%%%%
\subsection{Conditional Processing}
\label{sec:conditional}

The package provides a mechanism to compile different versions
of a document. To customise the versions further some conditional processing
can come in handy to distinguish which version is being compiled.
The package provides two macros to describe the compilation context:

%%%%%%%%%%%%%%%%%%%%%%%%%%%%%%%%%%%%%%%%
\DescribeMacro{\ifchilddoc}
The conditional |\ifchilddoc| distinguishes between the compilation of
child documents and the main document:
%
\begin{center}
|\ifchilddoc |\textit{child-code}| |[|\||else |\textit{main-code}]| \||fi|
\end{center}

%%%%%%%%%%%%%%%%%%%%%%%%%%%%%%%%%%%%%%%%
\DescribeMacro{\childdocname}
\DescribeMacro{\childdocjob}
The macro |\childdocname| contains the filename (without extension)
of the main or child file being processed.
Note that |\childdocjob| will always contain the name of the main file.

%%%%%%%%%%%%%%%%%%%%%%%%%%%%%%%%%%%%%%%%
\paragraph{Title Page.}

Conditional processing can be used to include a title or banner page
in the main document when proper precautions are taken.
Importantly, the code in the main file should ensure that the page counter
(as well as other status parameters which are stored in the |.aux| files)
takes the same value after the conditional processing.
Otherwise the page numbers may take divergent values
depending on which part is compiled.

For example, a title page could be declared by:
%
\begin{center}
\begin{tabular}{l}
|\ifchilddoc\||else|\\
|\addtocounter{page}{-1}|\\
\textit{code for title page}\\
|\newpage|\\
|\||fi|
\end{tabular}
\end{center}
%
A banner page for the child documents can be generated by:
%
\begin{center}
\begin{tabular}{l}
|\ifchilddoc|\\
|\addtocounter{page}{-1}|\\
\textit{code for banner page}\\
|\newpage|\\
|\||fi|
\end{tabular}
\end{center}
%
Here one could write a message such as:
\begin{center}
|This is the part \childdocname{} of \childdocjob{}.|
\end{center}

%%%%%%%%%%%%%%%%%%%%%%%%%%%%%%%%%%%%%%%%%%%%%%%%%%%%%%%%%%%%%%%%%%%%%%%%%%%%%%%%
\subsection{Flags}
\label{sec:flags}

The package makes it easy to generate different versions
of the main or child documents.
To this end compilation flags can be defined
and assigned different default values.
They will be particularly useful in conjunction
with the forwarding mechanism described in \secref{sec:forward}.

For example, it may be useful to have a flag |\version|
which can be set to |draft| or |final|.
The document source will contain some conditional code
depending on the value of |\version|.
Suppose further, the flag should default to |final| for the main file
and to |draft| for child files
which is a natural assignment for editing the document.
This is achieved by placing the following code
in the preamble of the main document
(below the |\childdocmain| directive):
%
\begin{center}
\begin{tabular}{l}
|\ifchilddoc|\\
|\providecommand{\version}{draft}|\\
|\||else|\\
|\providecommand{\version}{final}|\\
|\||fi|
\end{tabular}
\end{center}
%
The definition by |\providecommand| makes sure
that previous definitions are not overwritten.
Further statements |\providecommand{\version}{...}|
can thus be added before the above code to override it.

For the main file, one might add a line
(between |\childdocmain| and the above block)
%
\begin{center}
|%\ifchilddoc\||else\providecommand{\version}{draft}\||fi|
\end{center}
%
which can be uncommented to produce a draft version.
Likewise one can add a line to the very top of a child file
(above the |\childdocof{|\textit{main}|}| directive)
%
\begin{center}
|%\providecommand{\version}{final}|
\end{center}
%
which can be uncommented to produce the final version of this child document.

%%%%%%%%%%%%%%%%%%%%%%%%%%%%%%%%%%%%%%%%%%%%%%%%%%%%%%%%%%%%%%%%%%%%%%%%%%%%%%%%
\subsection{Forwarding}
\label{sec:forward}

Different versions of the main or child documents
using compilation flags as described in \secref{sec:flags}
can be (permanently) stored in different files
for convenient compilation, viewing and distribution.
To this end, the package defines a command
to pass on compilation to a different file:

%%%%%%%%%%%%%%%%%%%%%%%%%%%%%%%%%%%%%%%%
\DescribeMacro{\childdocforward}
The command |\childdocforward| redirects processing to
another source file:
%
\begin{center}
\begin{tabular}{l}
|\input{childdoc.def}|\\
|\childdocforward[|\textit{main}|]{|\textit{dest}|}|\\
\end{tabular}
\end{center}
%
The argument \textit{dest} is the destination file
(without extension).
It should be the main file or one of the child files.
Note that further \textsf{childdoc} directives
such as |\childdocof| and |\childdocforward|
in the indicated file will be processed in this form.
The optional argument \textit{main}
passes on directly to the main file \textit{main}
while pretending to compile the child \textit{dest}.
This form behaves as if \textit{dest}
issues |\childdocof{|\textit{main}|}| right away,
and no further \textsf{childdoc} directives will be processed.

%%%%%%%%%%%%%%%%%%%%%%%%%%%%%%%%%%%%%%%%
\DescribeMacro{\...prefix}
In the alternative form |\childdocforwardprefix|,
%
\begin{center}
\begin{tabular}{l}
|\input{childdoc.def}|\\
|\childdocforwardprefix[|\textit{main}|]{|\textit{prefix}|}{|\textit{dest}|}|
\end{tabular}
\end{center}
%
the destination file is determined by a pattern
depending on the current file:
To make this work, the current file must be called
`{\textit{prefix}\hspace{0.2em}\textit{suffix}}'
with \textit{prefix} matching precisely the argument.
Processing is then passed on to the file
`{\textit{dest}\hspace{0.2em}\textit{suffix}}'.
Surely, the same effect is achieved by
directly specifying the
argument `{\textit{dest}\hspace{0.2em}\textit{suffix}}'
in the first form.
However, that requires to set up a different file
for each child. With the alternative form of the command
all these files can have exactly the same content
which simplifies setting them up and maintaining them.

For example, the following file |draft.tex|
with a compilation flag |\version| as described in \secref{sec:flags}
compiles the main document as a draft:
%
\begin{center}
\begin{tabular}{l}
|\def\version{draft}|\\
|\input{childdoc.def}|\\
|\childdocforward{|\textit{main}|}|
\end{tabular}
\end{center}
%
Likewise, the following files |final|\textit{nn}|.tex|
compile the final version of the child document
|child|\textit{nn}|.tex|:
%
\begin{center}
\begin{tabular}{l}
|\def\version{final}|\\
|\input{childdoc.def}|\\
|\childdocforwardprefix{final}{child}|
\end{tabular}
\end{center}
%

Note that when several versions of a main file and/or of each child file
are to be generated, it may be convenient to set up a |Makefile| or
shell script to automatise the process.

%%%%%%%%%%%%%%%%%%%%%%%%%%%%%%%%%%%%%%%%%%%%%%%%%%%%%%%%%%%%%%%%%%%%%%%%%%%%%%%%
\subsection{Command Line Processing}
\label{sec:commandline}

The effect of redirection files can also be achieved by invoking
the \LaTeX{} compiler with a more elaborate command line.
Most conveniently this should be done as part
of a shell script or a |Makefile|.

When using \textsf{childdoc} in the main file, the following
command lines effectively perform a redirection
(note that depending on the shell being used,
backslashes may have to be doubled: `|\|' $\to$ `|\\|'):
%
\begin{center}
|... -jobname "|\textit{target}|" |\\|"|[\textit{flags}]%
|\input{childdoc.def}\childdocforward[|\textit{main}|]{|\textit{dest}|}"|
\end{center}
%
Here \textit{target} is the name of the output file,
\textit{main} is the name of the main file
and \textit{dest} is the name of the main or child file to be processed
(all filenames without extensions).
The optional argument \textit{main} can be omitted
if \textit{main} matches \textit{dest}.
Optionally, compilation \textit{flags} can be defined via |\def| commands.
This command line makes the \TeX{} engine believe
it is compiling the file \textit{target}
whose content is specified as the latter parameter.
The provided code then forwards the processing to
\textit{main} or \textit{dest} as described in \secref{sec:forward}.

%%%%%%%%%%%%%%%%%%%%%%%%%%%%%%%%%%%%%%%%%%%%%%%%%%%%%%%%%%%%%%%%%%%%%%%%%%%%%%%%
\subsection{Include by Input}
\label{sec:input}

Including child documents by |\include| has some restrictions by design.
Most notably, the content of a child document always occupies
its own set of pages; pages cannot be shared between child documents.
Usually, this behaviour makes perfect sense
because each child document contain an essential part of the document.
However, in some situations it may be desirable to compose
a document from a collection of parts
without having mandatory page breaks between then.
For this case, the package
provides a mechanism to include parts
by |\input| which can also be processed individually.
However, by construction this mechanism
requires manual handling of the content to be output.

%%%%%%%%%%%%%%%%%%%%%%%%%%%%%%%%%%%%%%%%
\DescribeMacro{\ifchilddocmanual}
The main file should be prepared as usual, see \secref{sec:include}.
However, the document body must make a distinction
between processing of an individual part and of the main document, e.g.:
%
\begin{center}
\begin{tabular}{l}
|\ifchilddocmanual|\\
|\input{\childdocname}|\\
|\||else|\\
\textit{document body with }|\input{|\textit{part}|}|\\
|\||fi|
\end{tabular}
\end{center}
%
The conditional |\ifchilddocmanual| is true whenever
a part to be included by |\input| is being compiled,
and the name of the part is stored in |\childdocname|.

%%%%%%%%%%%%%%%%%%%%%%%%%%%%%%%%%%%%%%%%
\DescribeMacro{\childdocby}
Each part to be included by |\input| should start with:
%
\begin{center}
\begin{tabular}{l}
|\input{childdoc.def}|\\
|\childdocby{|\textit{main}|}|\\
\end{tabular}
\end{center}
%
The directive |\childdocby| is similar to |\childdocof|
described in \secref{sec:include},
but the subsequent selection of content must be done manually.
To that end, both |\ifchilddoc| and |\ifchilddocmanual|
will be true upon processing of a part,
and the name of the part is stored in |\childdocname|.
Note that |\jobname| will be set to the filename of the current part
so that each part receives an individual |.aux| file
that does not interfere with the |.aux| file(s) of the main document.
This behaviour can be altered by the alternative form
|\childdocby[*]{|\textit{main}|}| (with a non-empty optional argument)
which uses the |.aux| file of the main document
by setting |\jobname| to \textit{main}.

%%%%%%%%%%%%%%%%%%%%%%%%%%%%%%%%%%%%%%%%%%%%%%%%%%%%%%%%%%%%%%%%%%%%%%%%%%%%%%%%
\subsection{Driver Development}
\label{sec:driver}

The \textsf{childdoc} mechanism can also be use for the development
of definition files such as \LaTeX{} styles or classes.
This case differs from the above setup with multiple parts
included by |\include| in that no |\includeonly| should be invoked.
This can be achieved by starting the include file
(before |\ProvidesPackage|) with:
%
\begin{center}
\begin{tabular}{l}
|\input{childdoc.def}|\\
|\childdocforward{|\textit{main}|}|\\
\end{tabular}
\end{center}
%
or alternatively with:
%
\begin{center}
\begin{tabular}{l}
|\input{childdoc.def}|\\
|\childdocby{|\textit{main}|}|\\
\end{tabular}
\end{center}
%
Both forms have slightly different effects as described above.
The main file is prepared as usual, see \secref{sec:include}.

%%%%%%%%%%%%%%%%%%%%%%%%%%%%%%%%%%%%%%%%%%%%%%%%%%%%%%%%%%%%%%%%%%%%%%%%%%%%%%%%
\subsection{Legacy Detection}
\label{sec:detection}

The directive |\childdocmain| in the main file can detect
whether the complete document or merely a child is to be compiled
even without using the directive |\childdocof|.
This method is deprecated because it is less robust
and there is no compelling reason to use it;
it is merely provided for backward compatibility
and it may be removed in future versions.

If the detection mechanism is to be used,
it is mandatory to correctly specify
the filename of the main file as the argument of |\childdocmain|:
%
\begin{center}
\begin{tabular}{l}
|\input{childdoc.def}|\\
|\childdocmain{|\textit{main}|}|\\
\end{tabular}
\end{center}
%
If |\jobname| does not match the argument \textit{main} of |\childdocmain|,
it is assumed that |\jobname| points to the child file to be compiled.
When using |\childdocmain| with the main file specified as argument,
it suffices to start a child file
with just |\input{|\textit{main}|}|
without loading of the package and using |\childdocof|.
If instead all processing is done
with the appropriate \textsf{childdoc} directives,
the argument of \textit{main} of |\childdocmain| can be empty.

An alternative version of the command line processing described
in \secref{sec:commandline} using the detection mechanism reads:
%
\begin{center}
|... -jobname "|\textit{target}|" "|[\textit{flags}]%
[|\def\jobname{|\textit{dest}|}|]|\input{|\textit{main}|}"|
\end{center}

%%%%%%%%%%%%%%%%%%%%%%%%%%%%%%%%%%%%%%%%%%%%%%%%%%%%%%%%%%%%%%%%%%%%%%%%%%%%%%%%
\subsection{Manual Code}
\label{sec:manual}

In case one cannot be certain whether the definitions file |childdoc.def|
is installed on the target \TeX{} distribution
and one prefers not to ship it,
it is conceivable to paste a few relevant commands into the sources.

To that end, drop all statements |\input{childdoc.def}|
and perform the replacements as outlined below.
Instead of |\childdocmain{|\textit{main}|}| add the following code
to the top of the main file:
%
\begin{center}
\begin{tabular}{l}
|\||ifdefined\childdocname\endinput\||fi\newif\ifchilddoc|\\
|\edef\childdocname{\scantokens\expandafter{\jobname\noexpand}}|\\
|\def\childdocmain{|\textit{main}|}\||ifx\childdocmain\childdocname\||else|\\
|\childdoctrue\includeonly{\childdocname}\let\jobname\childdocmain\||fi|\\
\end{tabular}
\end{center}
%
Instead of |\childdocof{|\textit{main}|}| just include the main file
at the top of each child file:
%
\begin{center}
|\input{|\textit{main}|}|
\end{center}
%
A simple redirection |\childdocforward{|\textit{dest}|}| is achieved by:
%
\begin{center}
|\def\jobname{|\textit{dest}|}\input{\jobname}|
\end{center}
%
The redirection with prefix
|\childdocforwardprefix[|\textit{prefix}|]{|\textit{dest}|}|
is accomplished by:
%
\begin{center}
\begin{tabular}{l}
|{\edef\jobname{\scantokens\expandafter{\jobname\noexpand}}|\\
|\def\redirectjob |\textit{prefix}|#1~~~{\gdef\jobname{|\textit{dest}|#1}}|\\
|\expandafter\redirectjob\jobname~~~}\input{\jobname}|
\end{tabular}
\end{center}

In an alternative approach,
child documents can be compiled by a specific command line
without additional code or specific definitions:
%
\begin{center}
|... -jobname "|\textit{target}|" "|[\textit{flags}]%
|\includeonly{|\textit{dest}|}\input{|\textit{main}|}"|
\end{center}
%

%%%%%%%%%%%%%%%%%%%%%%%%%%%%%%%%%%%%%%%%%%%%%%%%%%%%%%%%%%%%%%%%%%%%%%%%%%%%%%%%
%%%%%%%%%%%%%%%%%%%%%%%%%%%%%%%%%%%%%%%%%%%%%%%%%%%%%%%%%%%%%%%%%%%%%%%%%%%%%%%%
\section{Information}

%%%%%%%%%%%%%%%%%%%%%%%%%%%%%%%%%%%%%%%%%%%%%%%%%%%%%%%%%%%%%%%%%%%%%%%%%%%%%%%%
\subsection{Copyright}

Copyright \copyright{} 2017--2018 Niklas Beisert

This work may be distributed and/or modified under the
conditions of the \LaTeX{} Project Public License, either version 1.3
of this license or (at your option) any later version.
The latest version of this license is in
  \url{http://www.latex-project.org/lppl.txt}
and version 1.3 or later is part of all distributions of \LaTeX{}
version 2005/12/01 or later.

This work has the LPPL maintenance status `maintained'.

The Current Maintainer of this work is Niklas Beisert.

This work consists of the files |README.txt|, |childdoc.ins| and |childdoc.dtx|
as well as the derived files |childdoc.def|, |cdocsamp.tex|
with |cdocsch1.tex|, |cdocsch2.tex|, |cdocspt3.tex|, |cdocspt4.tex|,
|cdocsdrf.tex|, |cdocsfn1.tex|, |cdocsfn2.tex|
as well as |childdoc.pdf|.

%%%%%%%%%%%%%%%%%%%%%%%%%%%%%%%%%%%%%%%%%%%%%%%%%%%%%%%%%%%%%%%%%%%%%%%%%%%%%%%%
\subsection{Files and Installation}

The package consists of the files:
%
\begin{center}
\begin{tabular}{ll}
    |README.txt|   & readme file \\
    |childdoc.ins| & installation file \\
    |childdoc.dtx| & source file \\
    |childdoc.def| & definition file \\
    |cdocsamp.tex| & sample main file \\
    |cdocsch1.tex| & sample include file \\
    |cdocsch2.tex| & sample include file \\
    |cdocspt3.tex| & sample part file \\
    |cdocspt4.tex| & sample part file \\
    |cdocsdrf.tex| & sample redirection file \\
    |cdocsfn1.tex| & sample redirection file \\
    |cdocsfn2.tex| & sample redirection file \\
    |childdoc.pdf| & manual
\end{tabular}
\end{center}
%
The distribution consists of the files
|README.txt|, |childdoc.ins| and |childdoc.dtx|.
%
\begin{itemize}
\item
Run (pdf)\LaTeX{} on |childdoc.dtx|
to compile the manual |childdoc.pdf| (this file).
\item
Run \LaTeX{} on |childdoc.ins| to create the definitions file |childdoc.def|
and the sample |cdocsamp.tex| with include files
|cdocsch1.tex|, |cdocsch2.tex|, |cdocspt3.tex|, |cdocspt4.tex|,
|cdocsdrf.tex|, |cdocsfn1.tex|, |cdocsfn2.tex|.
Then copy the file |childdoc.def| to an appropriate directory of your \LaTeX{}
distribution, e.g.\ \textit{texmf-root}|/tex/latex/childdoc|.
\end{itemize}

%%%%%%%%%%%%%%%%%%%%%%%%%%%%%%%%%%%%%%%%%%%%%%%%%%%%%%%%%%%%%%%%%%%%%%%%%%%%%%%%
\subsection{Related CTAN Packages}

There are several other packages which offer a similar functionality:
%
\begin{itemize}
\item
The packages
\href{http://ctan.org/pkg/docmute}{\textsf{docmute}},
\href{http://ctan.org/pkg/includex}{\textsf{includex}} and
\href{http://ctan.org/pkg/standalone}{\textsf{standalone}}
provide commands to include only the document body of
a child file thus allowing both files to be compiled individually.
\item
The packages \href{http://ctan.org/pkg/subdocs}{\textsf{subdocs}}
and \href{http://ctan.org/pkg/subfiles}{\textsf{subfiles}}
provide structures in which the main and child documents can be
encapsulated and allowing them to be compiled individually.
The inclusion mechanism is different from the conventional |\include|.
\item
The package \href{http://ctan.org/pkg/combine}{\textsf{combine}}
is an elaborate solution to combine several documents into one.
\end{itemize}
%
See also the CTAN topic \href{http://ctan.org/topic/subdocs}{\textsf{subdocs}}
for further related packages.
The present package differs from the above solutions in that
a document structure constructed with the conventional |\include| mechanism
just needs two extra commands at the top of every file
such that all constituent files can be compiled individually.

%%%%%%%%%%%%%%%%%%%%%%%%%%%%%%%%%%%%%%%%%%%%%%%%%%%%%%%%%%%%%%%%%%%%%%%%%%%%%%%%
%\subsection{Feature Suggestions}
%
%The following is a list of features which may be useful for future
%versions of this package:
%%
%\begin{itemize}
%\item
%\ldots
%\end{itemize}

%%%%%%%%%%%%%%%%%%%%%%%%%%%%%%%%%%%%%%%%%%%%%%%%%%%%%%%%%%%%%%%%%%%%%%%%%%%%%%%%
\subsection{Revision History}

%%%%%%%%%%%%%%%%%%%%%%%%%%%%%%%%%%%%%%%%
\paragraph{v2.0:} 2018/12/30

\begin{itemize}
\item
immediate forward processing
\item
added |\childdocby| mechanism
\item
manual restructured
\end{itemize}

%%%%%%%%%%%%%%%%%%%%%%%%%%%%%%%%%%%%%%%%
\paragraph{v1.6:} 2018/01/17

\begin{itemize}
\item
application for development of include files
\item
corrections to manual
\end{itemize}

%%%%%%%%%%%%%%%%%%%%%%%%%%%%%%%%%%%%%%%%
\paragraph{v1.5:} 2017/05/21

\begin{itemize}
\item
more complete structuring introduced
\item
|\childdocof| introduced
\item
|\childdoc| renamed to |\childdocmain|
\item
|\childredirect| renamed to |\childdocforward| and |\childdocforwardprefix|
and functionality expanded
\end{itemize}

%%%%%%%%%%%%%%%%%%%%%%%%%%%%%%%%%%%%%%%%
\paragraph{v1.0:} 2017/04/27

\begin{itemize}
\item
manual and install package
\item
first version published on CTAN
\end{itemize}

%%%%%%%%%%%%%%%%%%%%%%%%%%%%%%%%%%%%%%%%
\paragraph{v0.6:} 2017/04/26

\begin{itemize}
\item
redirection mechanism added
\end{itemize}

%%%%%%%%%%%%%%%%%%%%%%%%%%%%%%%%%%%%%%%%
\paragraph{v0.5:} 2017/04/26

\begin{itemize}
\item
functionality in definition file
\end{itemize}


%%%%%%%%%%%%%%%%%%%%%%%%%%%%%%%%%%%%%%%%%%%%%%%%%%%%%%%%%%%%%%%%%%%%%%%%%%%%%%%%
%%%%%%%%%%%%%%%%%%%%%%%%%%%%%%%%%%%%%%%%%%%%%%%%%%%%%%%%%%%%%%%%%%%%%%%%%%%%%%%%
%%%%%%%%%%%%%%%%%%%%%%%%%%%%%%%%%%%%%%%%%%%%%%%%%%%%%%%%%%%%%%%%%%%%%%%%%%%%%%%%
\appendix

\settowidth\MacroIndent{\rmfamily\scriptsize 000\ }

 \DocInput{childdoc.dtx}

\end{document}
%</driver>
% \fi
%
% %%%%%%%%%%%%%%%%%%%%%%%%%%%%%%%%%%%%%%%%%%%%%%%%%%%%%%%%%%%%%%%%%%%%%%%%%%%%%%
% %%%%%%%%%%%%%%%%%%%%%%%%%%%%%%%%%%%%%%%%%%%%%%%%%%%%%%%%%%%%%%%%%%%%%%%%%%%%%%
% \section{Sample}
%\iffalse
%<*samplemain>
%\fi
%
% The following presents a sample document
% with two chapters, two parts, a title page,
% a compile flag as well as three forwarding files to set the flag.
% It consists of eight |.tex| files:
% \begin{center}
% \begin{tabular}{ll}
% |cdocsamp.tex|&main file\\
% |cdocsch1.tex|&include file for chapter 1\\
% |cdocsch2.tex|&include file for chapter 2\\
% |cdocspt3.tex|&include file for part 3\\
% |cdocspt4.tex|&include file for part 4\\
% |cdocsdrf.tex|&forwarding file for main file in draft mode\\
% |cdocsfi1.tex|&forwarding file for final version of chapter 1\\
% |cdocsfi2.tex|&forwarding file for final version of chapter 2\\
% \end{tabular}
% \end{center}
% Each of the eight files can be compiled directly by the \LaTeX{} compiler.
%
% %%%%%%%%%%%%%%%%%%%%%%%%%%%%%%%%%%%%%%
% \paragraph{Main File.}
%
% The main file is called |cdocsamp.tex|.
%
% Load the \textsf{childdoc} definitions and
% declare the filename for the main document:
%    \begin{macrocode}
\input{childdoc.def}
\childdocmain{}
%    \end{macrocode}

% Optional override for |\version| flag:
%    \begin{macrocode}
%%\ifchilddoc\else\providecommand{\version}{draft}\fi
%    \end{macrocode}

% Define the default values for the |\version| flag
% (|final| for the main file and |draft| for childs):
%    \begin{macrocode}
\ifchilddoc
\providecommand{\version}{draft}
\else
\providecommand{\version}{final}
\fi
%    \end{macrocode}

% Load the standard document class:
%    \begin{macrocode}
\documentclass[12pt]{article}
%    \end{macrocode}

% Start the document body:
%    \begin{macrocode}
\begin{document}
%    \end{macrocode}

% Declare a title page.
% Print title, part of document being processed and version flag:
%    \begin{macrocode}
\addtocounter{page}{-1}
\begin{center}
{\LARGE\bfseries{}childdoc example\par}
\vspace{1cm}
\ifchilddoc
\ifchilddocmanual part\else chapter\fi:
`\childdocname' of `\childdocjob'\par
\else
main document: `\childdocjob'\par
\fi
version: \version\par
\end{center}
\newpage
%    \end{macrocode}

% Manually include selected file,
% otherwise process as usual:
%    \begin{macrocode}
\ifchilddocmanual
\section*{part `\childdocname'}
\input{\childdocname}
\else
%    \end{macrocode}

% Include the two chapters:
%    \begin{macrocode}
\include{cdocsch1}
\include{cdocsch2}
%    \end{macrocode}

% Include the two parts unless only chapters should be displayed:
%    \begin{macrocode}
\ifchilddoc\else
\section{part three}
\input{cdocspt3}
\section{part four}
\input{cdocspt4}
\fi
%    \end{macrocode}

% Process as usual until here:
%    \begin{macrocode}
\fi
%    \end{macrocode}

% End of document body:
%    \begin{macrocode}
\end{document}
%    \end{macrocode}
%\iffalse
%</samplemain>
%\fi
%
% %%%%%%%%%%%%%%%%%%%%%%%%%%%%%%%%%%%%%%
% \paragraph{Chapter Include Files.}
%
% The include files are called |cdocsch1.tex| and |cdocsch2.tex|.
%
%\iffalse
%<*samplechap1|samplechap2>
%\fi

% Optional override for |\version| flag:
%    \begin{macrocode}
%%\providecommand{\version}{final}
%    \end{macrocode}

% Include the main document:
%    \begin{macrocode}
\input{childdoc.def}
\childdocof{cdocsamp}
%    \end{macrocode}

%\iffalse
%</samplechap1|samplechap2>
%\fi
%
%\iffalse
%<*samplechap1>
%\fi
% Some text for chapter 1:
%    \begin{macrocode}
\section{one}
some text in chapter one
%    \end{macrocode}

%\iffalse
%</samplechap1>
%\fi
% Some text for chapter 2:
%\iffalse
%<*samplechap2>
%\fi
%    \begin{macrocode}
\section{two}
more text in chapter two
%    \end{macrocode}

%\iffalse
%</samplechap2>
%\fi
%
% %%%%%%%%%%%%%%%%%%%%%%%%%%%%%%%%%%%%%%
% \paragraph{Part Include Files.}
%
% The include files are called |cdocspt3.tex| and |cdocspt4.tex|.
%
%\iffalse
%<*samplepart3|samplepart4>
%\fi

% Optional override for |\version| flag:
%    \begin{macrocode}
%%\providecommand{\version}{final}
%    \end{macrocode}

% Include the main document:
%    \begin{macrocode}
\input{childdoc.def}
\childdocby{cdocsamp}
%    \end{macrocode}

%\iffalse
%</samplepart3|samplepart4>
%\fi
%
%\iffalse
%<*samplepart3>
%\fi
% Some text for part 3:
%    \begin{macrocode}
some text in part three
%    \end{macrocode}

%\iffalse
%</samplepart3>
%\fi
% Some text for part 4:
%\iffalse
%<*samplepart4>
%\fi
%    \begin{macrocode}
more text in part four
%    \end{macrocode}

%\iffalse
%</samplepart4>
%\fi
%
% %%%%%%%%%%%%%%%%%%%%%%%%%%%%%%%%%%%%%%
% \paragraph{Forwarding for a Complete Draft.}
%
% The following forwarding file |cdocsdrf.tex|
% compiles the main document in draft mode:
%\iffalse
%<*sampledraft>
%\fi
%    \begin{macrocode}
\def\version{draft}
\input{childdoc.def}
\childdocforward{cdocsamp}
%    \end{macrocode}

%\iffalse
%</sampledraft>
%\fi
%
% %%%%%%%%%%%%%%%%%%%%%%%%%%%%%%%%%%%%%%
% \paragraph{Forwarding for Final Version of the Chapters.}
%
% The following forwarding files |cdocsfn1.tex| and |cdocsfn2.tex|
% (with identical content)
% compile the final versions of the child documents
% |cdocsch1.tex| and |cdocsch2.tex|, respectively:
%\iffalse
%<*samplefinal>
%\fi
%    \begin{macrocode}
\def\version{final}
\input{childdoc.def}
\childdocforwardprefix[cdocsamp]{cdocsfn}{cdocsch}
%    \end{macrocode}

%\iffalse
%</samplefinal>
%\fi
%
% %%%%%%%%%%%%%%%%%%%%%%%%%%%%%%%%%%%%%%
% \paragraph{Command Line Processing.}
%
% The following three command lines generate the output files
% |cdocscld|, |cdocscl1| and |cdocscl2|
% which should be identical to
% |cdocsdrf|, |cdocsch1| and |cdocsfn2|, respectively:
% \begin{center}
% \begin{tabular}{l}
% |latex -jobname cdocscld \|\\
% |  "\def\version{draft}\input{childdoc.def}\childdocforward{cdocsamp}"|\\
% |latex -jobname cdocscl1 \|\\
% |  "\input{childdoc.def}\childdocforward[cdocsamp]{cdocsch1}"|\\
% |latex -jobname cdocscl2 \|\\
% |  "\def\version{final}\input{childdoc.def}\childdocforward{cdocsch2}"|
% \end{tabular}
% \end{center}
% Note that the trailing backslash on each first line
% merely continues the input to the second line
% (for convenient cut ant paste).
% Furthermore, the command |latex| can be replaced by any
% of its alternative versions such as |pdflatex|.
%
% %%%%%%%%%%%%%%%%%%%%%%%%%%%%%%%%%%%%%%%%%%%%%%%%%%%%%%%%%%%%%%%%%%%%%%%%%%%%%%
% %%%%%%%%%%%%%%%%%%%%%%%%%%%%%%%%%%%%%%%%%%%%%%%%%%%%%%%%%%%%%%%%%%%%%%%%%%%%%%
% \section{Implementation}
%\iffalse
%<*package>
%\fi
%
% This section describes the definitions file |childdoc.def|.

% The definitions cannot be loaded using |\usepackage| or |\RequirePackage|
% which has a mechanism to prevent loading a style file more than once.
% When loading the definitions by means of |\input|
% multiple instances have to be prevented manually:
%\iffalse
%This code needs to be before the `\ProvidesFile' directive
%which is defined at the beginning of this file.
%Therefore it is also placed there and commented out here.
%</package>
%<*discard>
%\fi
%    \begin{macrocode}
\ifdefined\childdocmain\endinput\fi
%    \end{macrocode}
%\iffalse
%</discard>
%<*package>
%\fi
%
% \macro{\ifchilddoc}
% \macro{\ifchilddocmanual}
% The conditional |\ifchilddoc| tells whether a
% child (true) or main (false) document is being compiled.
% The conditional |\ifchilddocmanual| tells whether
% the |\includeonly| mechanism is used (false) or
% the selection of child files must be performed manually (true).
% The definitions initialise to false:
%    \begin{macrocode}
\newif\ifchilddoc
\newif\ifchilddocmanual
%    \end{macrocode}

% \macro{\childdocname}
% \macro{\childdocjob}
% The macro |\childdocname| stores the name of the main document
% to be compiled. The macro |\childdocjob| stores the name of
% the document on which the \LaTeX{} compiler was originally invoked.
% The content of |\jobname| cannot be compared
% to filenames specified in the source due to different catcodes.
% The following code rescans |\jobname|, stores the result
% in |\childdocname| and saves a copy in |\childdocjob|:
%    \begin{macrocode}
\edef\childdocname{\scantokens\expandafter{\jobname\noexpand}}
\let\childdocjob\childdocname
%    \end{macrocode}

% \macro{\childdocdisable}
% The macro |\childdocdisable| prevents the main file
% from being processed more than once.
% At this stage, the main document command |\childdocmain|
% is assumed to be called once again where it should do nothing.
% Any subsequent call to it should prevent
% a secondary processing of the main document
% It overwrites the forwarding commands
% |\childdocof| and |\childdocforward|
% with empty macros to prevent further inclusions of the main document:
%    \begin{macrocode}
\newcommand{\childdocdisable}
{
  \renewcommand{\childdocmain}[1]{\renewcommand{\childdocmain}[1]{\endinput}}
  \renewcommand{\childdocof}[1]{}
  \renewcommand{\childdocby}[2][]{}
  \renewcommand{\childdocforward}[2][]{}
  \renewcommand{\childdocdisable}{}
}
%    \end{macrocode}

% \macro{\childdocmain}
% The macro |\childdocmain| is to be called at the top of the main file
% with nothing or the main filename (without extension) as argument.
% First, it breaks loops.
% If the argument is not empty and does not match |\childdocname|
% (which is set by the first inclusion of |childdoc.def|),
% |\ifchilddoc| is set to true, |\includeonly| is applied to the child file
% and |\jobname| is set to the main file
% (for proper handling of |.aux| files):
%    \begin{macrocode}
\newcommand{\childdocmain}[1]
{
  \childdocdisable\childdocmain{}
  \if?#1?\else
    \begingroup
      \def\childdoctmp{#1}
      \ifx\childdoctmp\childdocname
        \def\childdoctmp{}
      \else
        \def\childdoctmp
        {
          \childdoctrue
          \includeonly{\childdocname}
          \def\childdocjob{#1}
          \def\jobname{#1}
        }
      \fi
      \expandafter
    \endgroup
    \childdoctmp
  \fi
}
%    \end{macrocode}

% \macro{\childdocof}
% The command |\childdocof| redirects
% compilation to the main file |#1|.
%    \begin{macrocode}
\newcommand{\childdocof}[1]
{
  \childdocdisable
  \childdoctrue
  \includeonly{\childdocname}
  \def\jobname{#1}
  \def\childdocjob{#1}
  \input{#1}
}
%    \end{macrocode}

% \macro{\childdocby}
% The command |\childdocby| ....
%    \begin{macrocode}
\newcommand{\childdocby}[2][]
{
  \childdocdisable
  \childdoctrue
  \childdocmanualtrue
  \if?#1?\else
    \def\jobname{#2}
  \fi
  \def\childdocjob{#2}
  \input{#2}
  \endinput
}
%    \end{macrocode}

% \macro{\childdocforward}
% The command |\childdocforward| redirects
% compilation to the main file or
% (if the optional argument is given) a child file.
% Parameters are set as if the main file
% or a child file starting with |\childdocof| was compiled.
% Then compilation is handed over to the main file:
%    \begin{macrocode}
\newcommand{\childdocforward}[2][]
{
  \begingroup
    \if?#1?
      \def\childdoctmp
      {
        \def\childdocname{#2}
        \def\childdocjob{#2}
        \def\jobname{#2}
        \input{#2}
        \endinput
      }
    \else
      \def\childdoctmp
      {
        \childdocdisable
        \def\childdocname{#2}
        \childdoctrue
        \includeonly{#2}
        \def\childdocjob{#1}
        \def\jobname{#1}
        \input{#1}
        \endinput
      }
    \fi
    \expandafter
  \endgroup
  \childdoctmp
}
%    \end{macrocode}

% \macro{\childdocforwardprefix}
% The command |\childdocforwardprefix| redirects
% compilation to the main or a child file by means of a pattern.
% The prefix |#1| in the current filename is replaced by |#2|
% and the suffix of the current filename is kept
% (it is assumed that the filename does not contain the substring `|~~~|'
% which is used as a delimiter).
% Compilation is handed over to the new file by |\childdocforward|:
%    \begin{macrocode}
\newcommand{\childdocforwardprefix}[3][]
{
  \begingroup
    \def\childdocextract #2##1~~~{\def\childdoctmp{\childdocforward[#1]{#3##1}}}
    \expandafter\childdocextract\childdocname~~~
    \expandafter
  \endgroup
  \childdoctmp
}
%    \end{macrocode}

% \macro{\childdoc}
% The deprecated macro |\childdoc| is a legacy version of |\childdocmain|:
%    \begin{macrocode}
\newcommand{\childdoc}{\childdocmain}
%    \end{macrocode}

% \macro{\childdocredirect}
% The deprecated macro |\childdocredirect| is a legacy version
% of |\childdocforward| and |\childdocforwardprefix|:
%    \begin{macrocode}
\newcommand{\childdocredirect}[2][]
{
  \begingroup
    \if?#1?
      \def\childdoctmp{\childdocforward{#2}}
    \else
      \def\childdoctmp{\childdocforwardprefix{#1}{#2}}
    \fi
    \expandafter
  \endgroup
  \childdoctmp
}
%    \end{macrocode}

%\iffalse
%</package>
%\fi
%
\endinput
|\\
|\childdocby{|\textit{main}|}|\\
\end{tabular}
\end{center}
%
Both forms have slightly different effects as described above.
The main file is prepared as usual, see \secref{sec:include}.

%%%%%%%%%%%%%%%%%%%%%%%%%%%%%%%%%%%%%%%%%%%%%%%%%%%%%%%%%%%%%%%%%%%%%%%%%%%%%%%%
\subsection{Legacy Detection}
\label{sec:detection}

The directive |\childdocmain| in the main file can detect
whether the complete document or merely a child is to be compiled
even without using the directive |\childdocof|.
This method is deprecated because it is less robust
and there is no compelling reason to use it;
it is merely provided for backward compatibility
and it may be removed in future versions.

If the detection mechanism is to be used,
it is mandatory to correctly specify
the filename of the main file as the argument of |\childdocmain|:
%
\begin{center}
\begin{tabular}{l}
|% \iffalse
%
% childdoc.dtx Copyright (C) 2017-2018 Niklas Beisert
%
% This work may be distributed and/or modified under the
% conditions of the LaTeX Project Public License, either version 1.3
% of this license or (at your option) any later version.
% The latest version of this license is in
%   http://www.latex-project.org/lppl.txt
% and version 1.3 or later is part of all distributions of LaTeX
% version 2005/12/01 or later.
%
% This work has the LPPL maintenance status `maintained'.
%
% The Current Maintainer of this work is Niklas Beisert.
%
% This work consists of the files childdoc.dtx and childdoc.ins
% and the derived files childdoc.def and cdocsamp.tex with
% cdocsch1.tex, cdocsch2.tex, cdocsdrf.tex, cdocsfn1.tex, cdocsfn2.tex.
%
%<package>\ifdefined\childdocmain\endinput\fi
%<package>\ProvidesFile{childdoc.def}[2018/12/30 v2.0 child document driver]
%<samplemain>\ProvidesFile{cdocsamp.tex}[2018/12/30 v2.0 sample for childdoc]
%<*driver>
%\ProvidesFile{childdoc.drv}[2018/12/30 v2.0 childdoc reference manual file]
\PassOptionsToClass{10pt,a4paper}{article}
\documentclass{ltxdoc}

\usepackage[margin=35mm]{geometry}
\usepackage{hyperref}
\usepackage{hyperxmp}
\usepackage[usenames]{color}

\hypersetup{colorlinks=true}
\hypersetup{pdfstartview=FitH}
\hypersetup{pdfpagemode=UseNone}
\hypersetup{pdfsource={}}
\hypersetup{pdflang={en-UK}}
\hypersetup{pdfcopyright={Copyright 2017-2018 Niklas Beisert.
  This work may be distributed and/or modified under the
  conditions of the LaTeX Project Public License, either version 1.3
  of this license or (at your option) any later version.}}
\hypersetup{pdflicenseurl={http://www.latex-project.org/lppl.txt}}
\hypersetup{pdfcontactaddress={ETH Zurich, ITP, HIT K,
  Wolfgang-Pauli-Strasse 27}}
\hypersetup{pdfcontactpostcode={8093}}
\hypersetup{pdfcontactcity={Zurich}}
\hypersetup{pdfcontactcountry={Switzerland}}
\hypersetup{pdfcontactemail={nbeisert@itp.phys.ethz.ch}}
\hypersetup{pdfcontacturl={http://people.phys.ethz.ch/\xmptilde nbeisert/}}

\newcommand{\secref}[1]{\hyperref[#1]{section \ref*{#1}}}

\parskip1ex
\parindent0pt
\let\olditemize\itemize
\def\itemize{\olditemize\parskip0pt}

\begin{document}

\title{The \textsf{childdoc} Package}
\hypersetup{pdftitle={The childdoc Package}}
\author{Niklas Beisert\\[2ex]
  Institut f\"ur Theoretische Physik\\
  Eidgen\"ossische Technische Hochschule Z\"urich\\
  Wolfgang-Pauli-Strasse 27, 8093 Z\"urich, Switzerland\\[1ex]
  \href{mailto:nbeisert@itp.phys.ethz.ch}
  {\texttt{nbeisert@itp.phys.ethz.ch}}}
\hypersetup{pdfauthor={Niklas Beisert}}
\hypersetup{pdfsubject={Manual for the LaTeX2e Package childdoc}}
\date{30 December 2018, \textsf{v2.0}}
\maketitle

\begin{abstract}\noindent
\textsf{childdoc} is a \LaTeXe{} package
that enables the direct compilation
of document sections included by |\include|
to individual files.
\end{abstract}

\begingroup
\parskip0ex
\tableofcontents
\endgroup

%%%%%%%%%%%%%%%%%%%%%%%%%%%%%%%%%%%%%%%%%%%%%%%%%%%%%%%%%%%%%%%%%%%%%%%%%%%%%%%%
%%%%%%%%%%%%%%%%%%%%%%%%%%%%%%%%%%%%%%%%%%%%%%%%%%%%%%%%%%%%%%%%%%%%%%%%%%%%%%%%
\section{Introduction}

\LaTeX{} provides a mechanism to structure a large document (such as a book)
into a main file and several child files (containing the chapters)
using the |\include| command.
This mechanism is beneficial for documents
which span hundreds of pages in order to
make the source file(s) more manageable.
Moreover, compilation can be restricted to
selected child files by means of the |\includeonly| command.
The latter feature can be used to reduce the compilation time while editing
(this was significantly more useful in the earlier days of \LaTeX{})
or to generate a smaller document which is easier to navigate.
Another application of |\includeonly| is to generate
documents consisting of selected parts of the complete document.

However, there are a few drawbacks of the plain |\include| mechanism:
\begin{itemize}
\item
The child files cannot be compiled on their own,
they can only be compiled via the main file.
A naive editing environment
(such as a text editor with an option
to have the current file processed by \LaTeX)
may require one to switch to the main file before compiling;
attempting to compile the child file produces errors.
\item
The main file must be modified (each time)
to adjust the |\includeonly| command
to the present needs. This easily leaves the main file in a messy state.
\item
The generated document will always carry the filename
of the main document. This is inconvenient if
several child files are to be compiled and
to be kept for distribution.
\end{itemize}

The present package provides a simple interface
to make child files individually compilable by \LaTeX{}.
Compiling a child file then has the same effect as compiling
the main file with an |\includeonly| command
to select the appropriate child.
Moreover the generated document will carry the name of the child
rather than the main file.
This resolves all three above issues.

This feature is meant to make the editing of books,
thesis documents and lecture notes somewhat more convenient.
However, the package can also be used efficiently for
composing a series of documents (such as exercise sheets)
which are typically distributed individually.
It then assists the author in generating the individual documents
(potentially in different versions)
as well as a document containing the collected series.
Another application is in developing style files
or other kinds of included material
where compilation of the style file could redirect
to a sample or test file.

%%%%%%%%%%%%%%%%%%%%%%%%%%%%%%%%%%%%%%%%%%%%%%%%%%%%%%%%%%%%%%%%%%%%%%%%%%%%%%%%
%%%%%%%%%%%%%%%%%%%%%%%%%%%%%%%%%%%%%%%%%%%%%%%%%%%%%%%%%%%%%%%%%%%%%%%%%%%%%%%%
\section{Usage}

First of all, the package \textsf{childdoc} is \emph{not} a standard
\LaTeXe{} |.sty| style file! Therefore it needs to be invoked in
a non-standard way.

%%%%%%%%%%%%%%%%%%%%%%%%%%%%%%%%%%%%%%%%%%%%%%%%%%%%%%%%%%%%%%%%%%%%%%%%%%%%%%%%
\subsection{Included Files}
\label{sec:include}

%%%%%%%%%%%%%%%%%%%%%%%%%%%%%%%%%%%%%%%%
\DescribeMacro{\childdocmain}
To use the package, add the commands
\begin{center}
\begin{tabular}{l}
|\input{childdoc.def}|\\
|\childdocmain{}|\\
\end{tabular}
\end{center}
at the very top of the main \LaTeX{} file,
in particular \emph{before} the |\documentclass| statement!
The argument of |\childdocmain| should be left empty
(but it must be present).

%%%%%%%%%%%%%%%%%%%%%%%%%%%%%%%%%%%%%%%%
\DescribeMacro{\childdocof}
Furthermore, add the commands
\begin{center}
\begin{tabular}{l}
|\input{childdoc.def}|\\
|\childdocof{|\textit{main}|}|\\
\end{tabular}
\end{center}
at the top of every child file \textit{child}
which is included by |\include{|\textit{child}|}|
from within the main file
(or at least for those files to be compiled individually).
The argument \textit{main} must be the filename of the main file.

There are a couple of
considerations in setting up the main and child documents:

%%%%%%%%%%%%%%%%%%%%%%%%%%%%%%%%%%%%%%%%
\paragraph{Restrictions.}

Please note the following restrictions:
\begin{itemize}
\item
|\childdocmain| must be called with one argument \textit{main}
to ensure compatibility with earlier version of the package.
It must either be empty (|\childdocmain{}|)
or precisely match the filename of the main file in which it is specified.
See \secref{sec:detection} for further information.
\item
The filename \textit{main} must be specified without the |.tex| extension.
\item
The filename \textit{main} is case sensitive
(even in case-insensitive file systems)
due to internal string comparison.
\item
The argument \textit{main} should be fully expanded, it cannot be a macro.
\item
Subdirectories and special characters should be avoided in filenames.
\item
The command |\childdocmain{|\textit{main}|}| must be followed by a whitespace.
It should not be followed immediately by another command
or by a comment mark `|%|'.
This is because the \TeX{} parser reads the token immediately following
the argument of |\childdocmain| and puts it
at the beginning of every child section;
however, a white\-space is ignored.
\end{itemize}

%%%%%%%%%%%%%%%%%%%%%%%%%%%%%%%%%%%%%%%%
\paragraph{Content of Main File.}

It is advisable to place all content in the child files included by |\include|.
Any output contained in the main file will appear in all child documents
unless suppressed manually;
it cannot be suppressed automatically by the |\includeonly| directive
and thus should normally be avoided.
A method to include some content in the main file
by means of conditional processing is described in \secref{sec:conditional}.

%%%%%%%%%%%%%%%%%%%%%%%%%%%%%%%%%%%%%%%%
\paragraph{Page Numbering.}

When only a part of the document is compiled,
the appropriate numbering of pages
(as well as other status parameters)
is determined from the |.aux| files.
The latter contain information from previous passes.
However this information needs to propagate through
all intermediate child documents.
Therefore the page numbering in child documents may well
be inconsistent until the complete document is compiled at least once.

A useful (if unconventional) way to always ensure a consistent
page numbering is to restart the numbering in each child document
and denote the pages by `\textit{child}|.|\textit{page}'
where \textit{child} represents the chapter/section number of the child file.
This can be achieved by the command
|\numberwithin{page}{|\textit{child}|}|
of the \textsf{amsmath} package
where \textit{child} can be |chapter| or |section|
depending on the chosen structuring.
Alternatively, one can modify the macro |\thepage| appropriately
and reset the counter |page| at the start of each child file.

%%%%%%%%%%%%%%%%%%%%%%%%%%%%%%%%%%%%%%%%%%%%%%%%%%%%%%%%%%%%%%%%%%%%%%%%%%%%%%%%
\subsection{Conditional Processing}
\label{sec:conditional}

The package provides a mechanism to compile different versions
of a document. To customise the versions further some conditional processing
can come in handy to distinguish which version is being compiled.
The package provides two macros to describe the compilation context:

%%%%%%%%%%%%%%%%%%%%%%%%%%%%%%%%%%%%%%%%
\DescribeMacro{\ifchilddoc}
The conditional |\ifchilddoc| distinguishes between the compilation of
child documents and the main document:
%
\begin{center}
|\ifchilddoc |\textit{child-code}| |[|\||else |\textit{main-code}]| \||fi|
\end{center}

%%%%%%%%%%%%%%%%%%%%%%%%%%%%%%%%%%%%%%%%
\DescribeMacro{\childdocname}
\DescribeMacro{\childdocjob}
The macro |\childdocname| contains the filename (without extension)
of the main or child file being processed.
Note that |\childdocjob| will always contain the name of the main file.

%%%%%%%%%%%%%%%%%%%%%%%%%%%%%%%%%%%%%%%%
\paragraph{Title Page.}

Conditional processing can be used to include a title or banner page
in the main document when proper precautions are taken.
Importantly, the code in the main file should ensure that the page counter
(as well as other status parameters which are stored in the |.aux| files)
takes the same value after the conditional processing.
Otherwise the page numbers may take divergent values
depending on which part is compiled.

For example, a title page could be declared by:
%
\begin{center}
\begin{tabular}{l}
|\ifchilddoc\||else|\\
|\addtocounter{page}{-1}|\\
\textit{code for title page}\\
|\newpage|\\
|\||fi|
\end{tabular}
\end{center}
%
A banner page for the child documents can be generated by:
%
\begin{center}
\begin{tabular}{l}
|\ifchilddoc|\\
|\addtocounter{page}{-1}|\\
\textit{code for banner page}\\
|\newpage|\\
|\||fi|
\end{tabular}
\end{center}
%
Here one could write a message such as:
\begin{center}
|This is the part \childdocname{} of \childdocjob{}.|
\end{center}

%%%%%%%%%%%%%%%%%%%%%%%%%%%%%%%%%%%%%%%%%%%%%%%%%%%%%%%%%%%%%%%%%%%%%%%%%%%%%%%%
\subsection{Flags}
\label{sec:flags}

The package makes it easy to generate different versions
of the main or child documents.
To this end compilation flags can be defined
and assigned different default values.
They will be particularly useful in conjunction
with the forwarding mechanism described in \secref{sec:forward}.

For example, it may be useful to have a flag |\version|
which can be set to |draft| or |final|.
The document source will contain some conditional code
depending on the value of |\version|.
Suppose further, the flag should default to |final| for the main file
and to |draft| for child files
which is a natural assignment for editing the document.
This is achieved by placing the following code
in the preamble of the main document
(below the |\childdocmain| directive):
%
\begin{center}
\begin{tabular}{l}
|\ifchilddoc|\\
|\providecommand{\version}{draft}|\\
|\||else|\\
|\providecommand{\version}{final}|\\
|\||fi|
\end{tabular}
\end{center}
%
The definition by |\providecommand| makes sure
that previous definitions are not overwritten.
Further statements |\providecommand{\version}{...}|
can thus be added before the above code to override it.

For the main file, one might add a line
(between |\childdocmain| and the above block)
%
\begin{center}
|%\ifchilddoc\||else\providecommand{\version}{draft}\||fi|
\end{center}
%
which can be uncommented to produce a draft version.
Likewise one can add a line to the very top of a child file
(above the |\childdocof{|\textit{main}|}| directive)
%
\begin{center}
|%\providecommand{\version}{final}|
\end{center}
%
which can be uncommented to produce the final version of this child document.

%%%%%%%%%%%%%%%%%%%%%%%%%%%%%%%%%%%%%%%%%%%%%%%%%%%%%%%%%%%%%%%%%%%%%%%%%%%%%%%%
\subsection{Forwarding}
\label{sec:forward}

Different versions of the main or child documents
using compilation flags as described in \secref{sec:flags}
can be (permanently) stored in different files
for convenient compilation, viewing and distribution.
To this end, the package defines a command
to pass on compilation to a different file:

%%%%%%%%%%%%%%%%%%%%%%%%%%%%%%%%%%%%%%%%
\DescribeMacro{\childdocforward}
The command |\childdocforward| redirects processing to
another source file:
%
\begin{center}
\begin{tabular}{l}
|\input{childdoc.def}|\\
|\childdocforward[|\textit{main}|]{|\textit{dest}|}|\\
\end{tabular}
\end{center}
%
The argument \textit{dest} is the destination file
(without extension).
It should be the main file or one of the child files.
Note that further \textsf{childdoc} directives
such as |\childdocof| and |\childdocforward|
in the indicated file will be processed in this form.
The optional argument \textit{main}
passes on directly to the main file \textit{main}
while pretending to compile the child \textit{dest}.
This form behaves as if \textit{dest}
issues |\childdocof{|\textit{main}|}| right away,
and no further \textsf{childdoc} directives will be processed.

%%%%%%%%%%%%%%%%%%%%%%%%%%%%%%%%%%%%%%%%
\DescribeMacro{\...prefix}
In the alternative form |\childdocforwardprefix|,
%
\begin{center}
\begin{tabular}{l}
|\input{childdoc.def}|\\
|\childdocforwardprefix[|\textit{main}|]{|\textit{prefix}|}{|\textit{dest}|}|
\end{tabular}
\end{center}
%
the destination file is determined by a pattern
depending on the current file:
To make this work, the current file must be called
`{\textit{prefix}\hspace{0.2em}\textit{suffix}}'
with \textit{prefix} matching precisely the argument.
Processing is then passed on to the file
`{\textit{dest}\hspace{0.2em}\textit{suffix}}'.
Surely, the same effect is achieved by
directly specifying the
argument `{\textit{dest}\hspace{0.2em}\textit{suffix}}'
in the first form.
However, that requires to set up a different file
for each child. With the alternative form of the command
all these files can have exactly the same content
which simplifies setting them up and maintaining them.

For example, the following file |draft.tex|
with a compilation flag |\version| as described in \secref{sec:flags}
compiles the main document as a draft:
%
\begin{center}
\begin{tabular}{l}
|\def\version{draft}|\\
|\input{childdoc.def}|\\
|\childdocforward{|\textit{main}|}|
\end{tabular}
\end{center}
%
Likewise, the following files |final|\textit{nn}|.tex|
compile the final version of the child document
|child|\textit{nn}|.tex|:
%
\begin{center}
\begin{tabular}{l}
|\def\version{final}|\\
|\input{childdoc.def}|\\
|\childdocforwardprefix{final}{child}|
\end{tabular}
\end{center}
%

Note that when several versions of a main file and/or of each child file
are to be generated, it may be convenient to set up a |Makefile| or
shell script to automatise the process.

%%%%%%%%%%%%%%%%%%%%%%%%%%%%%%%%%%%%%%%%%%%%%%%%%%%%%%%%%%%%%%%%%%%%%%%%%%%%%%%%
\subsection{Command Line Processing}
\label{sec:commandline}

The effect of redirection files can also be achieved by invoking
the \LaTeX{} compiler with a more elaborate command line.
Most conveniently this should be done as part
of a shell script or a |Makefile|.

When using \textsf{childdoc} in the main file, the following
command lines effectively perform a redirection
(note that depending on the shell being used,
backslashes may have to be doubled: `|\|' $\to$ `|\\|'):
%
\begin{center}
|... -jobname "|\textit{target}|" |\\|"|[\textit{flags}]%
|\input{childdoc.def}\childdocforward[|\textit{main}|]{|\textit{dest}|}"|
\end{center}
%
Here \textit{target} is the name of the output file,
\textit{main} is the name of the main file
and \textit{dest} is the name of the main or child file to be processed
(all filenames without extensions).
The optional argument \textit{main} can be omitted
if \textit{main} matches \textit{dest}.
Optionally, compilation \textit{flags} can be defined via |\def| commands.
This command line makes the \TeX{} engine believe
it is compiling the file \textit{target}
whose content is specified as the latter parameter.
The provided code then forwards the processing to
\textit{main} or \textit{dest} as described in \secref{sec:forward}.

%%%%%%%%%%%%%%%%%%%%%%%%%%%%%%%%%%%%%%%%%%%%%%%%%%%%%%%%%%%%%%%%%%%%%%%%%%%%%%%%
\subsection{Include by Input}
\label{sec:input}

Including child documents by |\include| has some restrictions by design.
Most notably, the content of a child document always occupies
its own set of pages; pages cannot be shared between child documents.
Usually, this behaviour makes perfect sense
because each child document contain an essential part of the document.
However, in some situations it may be desirable to compose
a document from a collection of parts
without having mandatory page breaks between then.
For this case, the package
provides a mechanism to include parts
by |\input| which can also be processed individually.
However, by construction this mechanism
requires manual handling of the content to be output.

%%%%%%%%%%%%%%%%%%%%%%%%%%%%%%%%%%%%%%%%
\DescribeMacro{\ifchilddocmanual}
The main file should be prepared as usual, see \secref{sec:include}.
However, the document body must make a distinction
between processing of an individual part and of the main document, e.g.:
%
\begin{center}
\begin{tabular}{l}
|\ifchilddocmanual|\\
|\input{\childdocname}|\\
|\||else|\\
\textit{document body with }|\input{|\textit{part}|}|\\
|\||fi|
\end{tabular}
\end{center}
%
The conditional |\ifchilddocmanual| is true whenever
a part to be included by |\input| is being compiled,
and the name of the part is stored in |\childdocname|.

%%%%%%%%%%%%%%%%%%%%%%%%%%%%%%%%%%%%%%%%
\DescribeMacro{\childdocby}
Each part to be included by |\input| should start with:
%
\begin{center}
\begin{tabular}{l}
|\input{childdoc.def}|\\
|\childdocby{|\textit{main}|}|\\
\end{tabular}
\end{center}
%
The directive |\childdocby| is similar to |\childdocof|
described in \secref{sec:include},
but the subsequent selection of content must be done manually.
To that end, both |\ifchilddoc| and |\ifchilddocmanual|
will be true upon processing of a part,
and the name of the part is stored in |\childdocname|.
Note that |\jobname| will be set to the filename of the current part
so that each part receives an individual |.aux| file
that does not interfere with the |.aux| file(s) of the main document.
This behaviour can be altered by the alternative form
|\childdocby[*]{|\textit{main}|}| (with a non-empty optional argument)
which uses the |.aux| file of the main document
by setting |\jobname| to \textit{main}.

%%%%%%%%%%%%%%%%%%%%%%%%%%%%%%%%%%%%%%%%%%%%%%%%%%%%%%%%%%%%%%%%%%%%%%%%%%%%%%%%
\subsection{Driver Development}
\label{sec:driver}

The \textsf{childdoc} mechanism can also be use for the development
of definition files such as \LaTeX{} styles or classes.
This case differs from the above setup with multiple parts
included by |\include| in that no |\includeonly| should be invoked.
This can be achieved by starting the include file
(before |\ProvidesPackage|) with:
%
\begin{center}
\begin{tabular}{l}
|\input{childdoc.def}|\\
|\childdocforward{|\textit{main}|}|\\
\end{tabular}
\end{center}
%
or alternatively with:
%
\begin{center}
\begin{tabular}{l}
|\input{childdoc.def}|\\
|\childdocby{|\textit{main}|}|\\
\end{tabular}
\end{center}
%
Both forms have slightly different effects as described above.
The main file is prepared as usual, see \secref{sec:include}.

%%%%%%%%%%%%%%%%%%%%%%%%%%%%%%%%%%%%%%%%%%%%%%%%%%%%%%%%%%%%%%%%%%%%%%%%%%%%%%%%
\subsection{Legacy Detection}
\label{sec:detection}

The directive |\childdocmain| in the main file can detect
whether the complete document or merely a child is to be compiled
even without using the directive |\childdocof|.
This method is deprecated because it is less robust
and there is no compelling reason to use it;
it is merely provided for backward compatibility
and it may be removed in future versions.

If the detection mechanism is to be used,
it is mandatory to correctly specify
the filename of the main file as the argument of |\childdocmain|:
%
\begin{center}
\begin{tabular}{l}
|\input{childdoc.def}|\\
|\childdocmain{|\textit{main}|}|\\
\end{tabular}
\end{center}
%
If |\jobname| does not match the argument \textit{main} of |\childdocmain|,
it is assumed that |\jobname| points to the child file to be compiled.
When using |\childdocmain| with the main file specified as argument,
it suffices to start a child file
with just |\input{|\textit{main}|}|
without loading of the package and using |\childdocof|.
If instead all processing is done
with the appropriate \textsf{childdoc} directives,
the argument of \textit{main} of |\childdocmain| can be empty.

An alternative version of the command line processing described
in \secref{sec:commandline} using the detection mechanism reads:
%
\begin{center}
|... -jobname "|\textit{target}|" "|[\textit{flags}]%
[|\def\jobname{|\textit{dest}|}|]|\input{|\textit{main}|}"|
\end{center}

%%%%%%%%%%%%%%%%%%%%%%%%%%%%%%%%%%%%%%%%%%%%%%%%%%%%%%%%%%%%%%%%%%%%%%%%%%%%%%%%
\subsection{Manual Code}
\label{sec:manual}

In case one cannot be certain whether the definitions file |childdoc.def|
is installed on the target \TeX{} distribution
and one prefers not to ship it,
it is conceivable to paste a few relevant commands into the sources.

To that end, drop all statements |\input{childdoc.def}|
and perform the replacements as outlined below.
Instead of |\childdocmain{|\textit{main}|}| add the following code
to the top of the main file:
%
\begin{center}
\begin{tabular}{l}
|\||ifdefined\childdocname\endinput\||fi\newif\ifchilddoc|\\
|\edef\childdocname{\scantokens\expandafter{\jobname\noexpand}}|\\
|\def\childdocmain{|\textit{main}|}\||ifx\childdocmain\childdocname\||else|\\
|\childdoctrue\includeonly{\childdocname}\let\jobname\childdocmain\||fi|\\
\end{tabular}
\end{center}
%
Instead of |\childdocof{|\textit{main}|}| just include the main file
at the top of each child file:
%
\begin{center}
|\input{|\textit{main}|}|
\end{center}
%
A simple redirection |\childdocforward{|\textit{dest}|}| is achieved by:
%
\begin{center}
|\def\jobname{|\textit{dest}|}\input{\jobname}|
\end{center}
%
The redirection with prefix
|\childdocforwardprefix[|\textit{prefix}|]{|\textit{dest}|}|
is accomplished by:
%
\begin{center}
\begin{tabular}{l}
|{\edef\jobname{\scantokens\expandafter{\jobname\noexpand}}|\\
|\def\redirectjob |\textit{prefix}|#1~~~{\gdef\jobname{|\textit{dest}|#1}}|\\
|\expandafter\redirectjob\jobname~~~}\input{\jobname}|
\end{tabular}
\end{center}

In an alternative approach,
child documents can be compiled by a specific command line
without additional code or specific definitions:
%
\begin{center}
|... -jobname "|\textit{target}|" "|[\textit{flags}]%
|\includeonly{|\textit{dest}|}\input{|\textit{main}|}"|
\end{center}
%

%%%%%%%%%%%%%%%%%%%%%%%%%%%%%%%%%%%%%%%%%%%%%%%%%%%%%%%%%%%%%%%%%%%%%%%%%%%%%%%%
%%%%%%%%%%%%%%%%%%%%%%%%%%%%%%%%%%%%%%%%%%%%%%%%%%%%%%%%%%%%%%%%%%%%%%%%%%%%%%%%
\section{Information}

%%%%%%%%%%%%%%%%%%%%%%%%%%%%%%%%%%%%%%%%%%%%%%%%%%%%%%%%%%%%%%%%%%%%%%%%%%%%%%%%
\subsection{Copyright}

Copyright \copyright{} 2017--2018 Niklas Beisert

This work may be distributed and/or modified under the
conditions of the \LaTeX{} Project Public License, either version 1.3
of this license or (at your option) any later version.
The latest version of this license is in
  \url{http://www.latex-project.org/lppl.txt}
and version 1.3 or later is part of all distributions of \LaTeX{}
version 2005/12/01 or later.

This work has the LPPL maintenance status `maintained'.

The Current Maintainer of this work is Niklas Beisert.

This work consists of the files |README.txt|, |childdoc.ins| and |childdoc.dtx|
as well as the derived files |childdoc.def|, |cdocsamp.tex|
with |cdocsch1.tex|, |cdocsch2.tex|, |cdocspt3.tex|, |cdocspt4.tex|,
|cdocsdrf.tex|, |cdocsfn1.tex|, |cdocsfn2.tex|
as well as |childdoc.pdf|.

%%%%%%%%%%%%%%%%%%%%%%%%%%%%%%%%%%%%%%%%%%%%%%%%%%%%%%%%%%%%%%%%%%%%%%%%%%%%%%%%
\subsection{Files and Installation}

The package consists of the files:
%
\begin{center}
\begin{tabular}{ll}
    |README.txt|   & readme file \\
    |childdoc.ins| & installation file \\
    |childdoc.dtx| & source file \\
    |childdoc.def| & definition file \\
    |cdocsamp.tex| & sample main file \\
    |cdocsch1.tex| & sample include file \\
    |cdocsch2.tex| & sample include file \\
    |cdocspt3.tex| & sample part file \\
    |cdocspt4.tex| & sample part file \\
    |cdocsdrf.tex| & sample redirection file \\
    |cdocsfn1.tex| & sample redirection file \\
    |cdocsfn2.tex| & sample redirection file \\
    |childdoc.pdf| & manual
\end{tabular}
\end{center}
%
The distribution consists of the files
|README.txt|, |childdoc.ins| and |childdoc.dtx|.
%
\begin{itemize}
\item
Run (pdf)\LaTeX{} on |childdoc.dtx|
to compile the manual |childdoc.pdf| (this file).
\item
Run \LaTeX{} on |childdoc.ins| to create the definitions file |childdoc.def|
and the sample |cdocsamp.tex| with include files
|cdocsch1.tex|, |cdocsch2.tex|, |cdocspt3.tex|, |cdocspt4.tex|,
|cdocsdrf.tex|, |cdocsfn1.tex|, |cdocsfn2.tex|.
Then copy the file |childdoc.def| to an appropriate directory of your \LaTeX{}
distribution, e.g.\ \textit{texmf-root}|/tex/latex/childdoc|.
\end{itemize}

%%%%%%%%%%%%%%%%%%%%%%%%%%%%%%%%%%%%%%%%%%%%%%%%%%%%%%%%%%%%%%%%%%%%%%%%%%%%%%%%
\subsection{Related CTAN Packages}

There are several other packages which offer a similar functionality:
%
\begin{itemize}
\item
The packages
\href{http://ctan.org/pkg/docmute}{\textsf{docmute}},
\href{http://ctan.org/pkg/includex}{\textsf{includex}} and
\href{http://ctan.org/pkg/standalone}{\textsf{standalone}}
provide commands to include only the document body of
a child file thus allowing both files to be compiled individually.
\item
The packages \href{http://ctan.org/pkg/subdocs}{\textsf{subdocs}}
and \href{http://ctan.org/pkg/subfiles}{\textsf{subfiles}}
provide structures in which the main and child documents can be
encapsulated and allowing them to be compiled individually.
The inclusion mechanism is different from the conventional |\include|.
\item
The package \href{http://ctan.org/pkg/combine}{\textsf{combine}}
is an elaborate solution to combine several documents into one.
\end{itemize}
%
See also the CTAN topic \href{http://ctan.org/topic/subdocs}{\textsf{subdocs}}
for further related packages.
The present package differs from the above solutions in that
a document structure constructed with the conventional |\include| mechanism
just needs two extra commands at the top of every file
such that all constituent files can be compiled individually.

%%%%%%%%%%%%%%%%%%%%%%%%%%%%%%%%%%%%%%%%%%%%%%%%%%%%%%%%%%%%%%%%%%%%%%%%%%%%%%%%
%\subsection{Feature Suggestions}
%
%The following is a list of features which may be useful for future
%versions of this package:
%%
%\begin{itemize}
%\item
%\ldots
%\end{itemize}

%%%%%%%%%%%%%%%%%%%%%%%%%%%%%%%%%%%%%%%%%%%%%%%%%%%%%%%%%%%%%%%%%%%%%%%%%%%%%%%%
\subsection{Revision History}

%%%%%%%%%%%%%%%%%%%%%%%%%%%%%%%%%%%%%%%%
\paragraph{v2.0:} 2018/12/30

\begin{itemize}
\item
immediate forward processing
\item
added |\childdocby| mechanism
\item
manual restructured
\end{itemize}

%%%%%%%%%%%%%%%%%%%%%%%%%%%%%%%%%%%%%%%%
\paragraph{v1.6:} 2018/01/17

\begin{itemize}
\item
application for development of include files
\item
corrections to manual
\end{itemize}

%%%%%%%%%%%%%%%%%%%%%%%%%%%%%%%%%%%%%%%%
\paragraph{v1.5:} 2017/05/21

\begin{itemize}
\item
more complete structuring introduced
\item
|\childdocof| introduced
\item
|\childdoc| renamed to |\childdocmain|
\item
|\childredirect| renamed to |\childdocforward| and |\childdocforwardprefix|
and functionality expanded
\end{itemize}

%%%%%%%%%%%%%%%%%%%%%%%%%%%%%%%%%%%%%%%%
\paragraph{v1.0:} 2017/04/27

\begin{itemize}
\item
manual and install package
\item
first version published on CTAN
\end{itemize}

%%%%%%%%%%%%%%%%%%%%%%%%%%%%%%%%%%%%%%%%
\paragraph{v0.6:} 2017/04/26

\begin{itemize}
\item
redirection mechanism added
\end{itemize}

%%%%%%%%%%%%%%%%%%%%%%%%%%%%%%%%%%%%%%%%
\paragraph{v0.5:} 2017/04/26

\begin{itemize}
\item
functionality in definition file
\end{itemize}


%%%%%%%%%%%%%%%%%%%%%%%%%%%%%%%%%%%%%%%%%%%%%%%%%%%%%%%%%%%%%%%%%%%%%%%%%%%%%%%%
%%%%%%%%%%%%%%%%%%%%%%%%%%%%%%%%%%%%%%%%%%%%%%%%%%%%%%%%%%%%%%%%%%%%%%%%%%%%%%%%
%%%%%%%%%%%%%%%%%%%%%%%%%%%%%%%%%%%%%%%%%%%%%%%%%%%%%%%%%%%%%%%%%%%%%%%%%%%%%%%%
\appendix

\settowidth\MacroIndent{\rmfamily\scriptsize 000\ }

 \DocInput{childdoc.dtx}

\end{document}
%</driver>
% \fi
%
% %%%%%%%%%%%%%%%%%%%%%%%%%%%%%%%%%%%%%%%%%%%%%%%%%%%%%%%%%%%%%%%%%%%%%%%%%%%%%%
% %%%%%%%%%%%%%%%%%%%%%%%%%%%%%%%%%%%%%%%%%%%%%%%%%%%%%%%%%%%%%%%%%%%%%%%%%%%%%%
% \section{Sample}
%\iffalse
%<*samplemain>
%\fi
%
% The following presents a sample document
% with two chapters, two parts, a title page,
% a compile flag as well as three forwarding files to set the flag.
% It consists of eight |.tex| files:
% \begin{center}
% \begin{tabular}{ll}
% |cdocsamp.tex|&main file\\
% |cdocsch1.tex|&include file for chapter 1\\
% |cdocsch2.tex|&include file for chapter 2\\
% |cdocspt3.tex|&include file for part 3\\
% |cdocspt4.tex|&include file for part 4\\
% |cdocsdrf.tex|&forwarding file for main file in draft mode\\
% |cdocsfi1.tex|&forwarding file for final version of chapter 1\\
% |cdocsfi2.tex|&forwarding file for final version of chapter 2\\
% \end{tabular}
% \end{center}
% Each of the eight files can be compiled directly by the \LaTeX{} compiler.
%
% %%%%%%%%%%%%%%%%%%%%%%%%%%%%%%%%%%%%%%
% \paragraph{Main File.}
%
% The main file is called |cdocsamp.tex|.
%
% Load the \textsf{childdoc} definitions and
% declare the filename for the main document:
%    \begin{macrocode}
\input{childdoc.def}
\childdocmain{}
%    \end{macrocode}

% Optional override for |\version| flag:
%    \begin{macrocode}
%%\ifchilddoc\else\providecommand{\version}{draft}\fi
%    \end{macrocode}

% Define the default values for the |\version| flag
% (|final| for the main file and |draft| for childs):
%    \begin{macrocode}
\ifchilddoc
\providecommand{\version}{draft}
\else
\providecommand{\version}{final}
\fi
%    \end{macrocode}

% Load the standard document class:
%    \begin{macrocode}
\documentclass[12pt]{article}
%    \end{macrocode}

% Start the document body:
%    \begin{macrocode}
\begin{document}
%    \end{macrocode}

% Declare a title page.
% Print title, part of document being processed and version flag:
%    \begin{macrocode}
\addtocounter{page}{-1}
\begin{center}
{\LARGE\bfseries{}childdoc example\par}
\vspace{1cm}
\ifchilddoc
\ifchilddocmanual part\else chapter\fi:
`\childdocname' of `\childdocjob'\par
\else
main document: `\childdocjob'\par
\fi
version: \version\par
\end{center}
\newpage
%    \end{macrocode}

% Manually include selected file,
% otherwise process as usual:
%    \begin{macrocode}
\ifchilddocmanual
\section*{part `\childdocname'}
\input{\childdocname}
\else
%    \end{macrocode}

% Include the two chapters:
%    \begin{macrocode}
\include{cdocsch1}
\include{cdocsch2}
%    \end{macrocode}

% Include the two parts unless only chapters should be displayed:
%    \begin{macrocode}
\ifchilddoc\else
\section{part three}
\input{cdocspt3}
\section{part four}
\input{cdocspt4}
\fi
%    \end{macrocode}

% Process as usual until here:
%    \begin{macrocode}
\fi
%    \end{macrocode}

% End of document body:
%    \begin{macrocode}
\end{document}
%    \end{macrocode}
%\iffalse
%</samplemain>
%\fi
%
% %%%%%%%%%%%%%%%%%%%%%%%%%%%%%%%%%%%%%%
% \paragraph{Chapter Include Files.}
%
% The include files are called |cdocsch1.tex| and |cdocsch2.tex|.
%
%\iffalse
%<*samplechap1|samplechap2>
%\fi

% Optional override for |\version| flag:
%    \begin{macrocode}
%%\providecommand{\version}{final}
%    \end{macrocode}

% Include the main document:
%    \begin{macrocode}
\input{childdoc.def}
\childdocof{cdocsamp}
%    \end{macrocode}

%\iffalse
%</samplechap1|samplechap2>
%\fi
%
%\iffalse
%<*samplechap1>
%\fi
% Some text for chapter 1:
%    \begin{macrocode}
\section{one}
some text in chapter one
%    \end{macrocode}

%\iffalse
%</samplechap1>
%\fi
% Some text for chapter 2:
%\iffalse
%<*samplechap2>
%\fi
%    \begin{macrocode}
\section{two}
more text in chapter two
%    \end{macrocode}

%\iffalse
%</samplechap2>
%\fi
%
% %%%%%%%%%%%%%%%%%%%%%%%%%%%%%%%%%%%%%%
% \paragraph{Part Include Files.}
%
% The include files are called |cdocspt3.tex| and |cdocspt4.tex|.
%
%\iffalse
%<*samplepart3|samplepart4>
%\fi

% Optional override for |\version| flag:
%    \begin{macrocode}
%%\providecommand{\version}{final}
%    \end{macrocode}

% Include the main document:
%    \begin{macrocode}
\input{childdoc.def}
\childdocby{cdocsamp}
%    \end{macrocode}

%\iffalse
%</samplepart3|samplepart4>
%\fi
%
%\iffalse
%<*samplepart3>
%\fi
% Some text for part 3:
%    \begin{macrocode}
some text in part three
%    \end{macrocode}

%\iffalse
%</samplepart3>
%\fi
% Some text for part 4:
%\iffalse
%<*samplepart4>
%\fi
%    \begin{macrocode}
more text in part four
%    \end{macrocode}

%\iffalse
%</samplepart4>
%\fi
%
% %%%%%%%%%%%%%%%%%%%%%%%%%%%%%%%%%%%%%%
% \paragraph{Forwarding for a Complete Draft.}
%
% The following forwarding file |cdocsdrf.tex|
% compiles the main document in draft mode:
%\iffalse
%<*sampledraft>
%\fi
%    \begin{macrocode}
\def\version{draft}
\input{childdoc.def}
\childdocforward{cdocsamp}
%    \end{macrocode}

%\iffalse
%</sampledraft>
%\fi
%
% %%%%%%%%%%%%%%%%%%%%%%%%%%%%%%%%%%%%%%
% \paragraph{Forwarding for Final Version of the Chapters.}
%
% The following forwarding files |cdocsfn1.tex| and |cdocsfn2.tex|
% (with identical content)
% compile the final versions of the child documents
% |cdocsch1.tex| and |cdocsch2.tex|, respectively:
%\iffalse
%<*samplefinal>
%\fi
%    \begin{macrocode}
\def\version{final}
\input{childdoc.def}
\childdocforwardprefix[cdocsamp]{cdocsfn}{cdocsch}
%    \end{macrocode}

%\iffalse
%</samplefinal>
%\fi
%
% %%%%%%%%%%%%%%%%%%%%%%%%%%%%%%%%%%%%%%
% \paragraph{Command Line Processing.}
%
% The following three command lines generate the output files
% |cdocscld|, |cdocscl1| and |cdocscl2|
% which should be identical to
% |cdocsdrf|, |cdocsch1| and |cdocsfn2|, respectively:
% \begin{center}
% \begin{tabular}{l}
% |latex -jobname cdocscld \|\\
% |  "\def\version{draft}\input{childdoc.def}\childdocforward{cdocsamp}"|\\
% |latex -jobname cdocscl1 \|\\
% |  "\input{childdoc.def}\childdocforward[cdocsamp]{cdocsch1}"|\\
% |latex -jobname cdocscl2 \|\\
% |  "\def\version{final}\input{childdoc.def}\childdocforward{cdocsch2}"|
% \end{tabular}
% \end{center}
% Note that the trailing backslash on each first line
% merely continues the input to the second line
% (for convenient cut ant paste).
% Furthermore, the command |latex| can be replaced by any
% of its alternative versions such as |pdflatex|.
%
% %%%%%%%%%%%%%%%%%%%%%%%%%%%%%%%%%%%%%%%%%%%%%%%%%%%%%%%%%%%%%%%%%%%%%%%%%%%%%%
% %%%%%%%%%%%%%%%%%%%%%%%%%%%%%%%%%%%%%%%%%%%%%%%%%%%%%%%%%%%%%%%%%%%%%%%%%%%%%%
% \section{Implementation}
%\iffalse
%<*package>
%\fi
%
% This section describes the definitions file |childdoc.def|.

% The definitions cannot be loaded using |\usepackage| or |\RequirePackage|
% which has a mechanism to prevent loading a style file more than once.
% When loading the definitions by means of |\input|
% multiple instances have to be prevented manually:
%\iffalse
%This code needs to be before the `\ProvidesFile' directive
%which is defined at the beginning of this file.
%Therefore it is also placed there and commented out here.
%</package>
%<*discard>
%\fi
%    \begin{macrocode}
\ifdefined\childdocmain\endinput\fi
%    \end{macrocode}
%\iffalse
%</discard>
%<*package>
%\fi
%
% \macro{\ifchilddoc}
% \macro{\ifchilddocmanual}
% The conditional |\ifchilddoc| tells whether a
% child (true) or main (false) document is being compiled.
% The conditional |\ifchilddocmanual| tells whether
% the |\includeonly| mechanism is used (false) or
% the selection of child files must be performed manually (true).
% The definitions initialise to false:
%    \begin{macrocode}
\newif\ifchilddoc
\newif\ifchilddocmanual
%    \end{macrocode}

% \macro{\childdocname}
% \macro{\childdocjob}
% The macro |\childdocname| stores the name of the main document
% to be compiled. The macro |\childdocjob| stores the name of
% the document on which the \LaTeX{} compiler was originally invoked.
% The content of |\jobname| cannot be compared
% to filenames specified in the source due to different catcodes.
% The following code rescans |\jobname|, stores the result
% in |\childdocname| and saves a copy in |\childdocjob|:
%    \begin{macrocode}
\edef\childdocname{\scantokens\expandafter{\jobname\noexpand}}
\let\childdocjob\childdocname
%    \end{macrocode}

% \macro{\childdocdisable}
% The macro |\childdocdisable| prevents the main file
% from being processed more than once.
% At this stage, the main document command |\childdocmain|
% is assumed to be called once again where it should do nothing.
% Any subsequent call to it should prevent
% a secondary processing of the main document
% It overwrites the forwarding commands
% |\childdocof| and |\childdocforward|
% with empty macros to prevent further inclusions of the main document:
%    \begin{macrocode}
\newcommand{\childdocdisable}
{
  \renewcommand{\childdocmain}[1]{\renewcommand{\childdocmain}[1]{\endinput}}
  \renewcommand{\childdocof}[1]{}
  \renewcommand{\childdocby}[2][]{}
  \renewcommand{\childdocforward}[2][]{}
  \renewcommand{\childdocdisable}{}
}
%    \end{macrocode}

% \macro{\childdocmain}
% The macro |\childdocmain| is to be called at the top of the main file
% with nothing or the main filename (without extension) as argument.
% First, it breaks loops.
% If the argument is not empty and does not match |\childdocname|
% (which is set by the first inclusion of |childdoc.def|),
% |\ifchilddoc| is set to true, |\includeonly| is applied to the child file
% and |\jobname| is set to the main file
% (for proper handling of |.aux| files):
%    \begin{macrocode}
\newcommand{\childdocmain}[1]
{
  \childdocdisable\childdocmain{}
  \if?#1?\else
    \begingroup
      \def\childdoctmp{#1}
      \ifx\childdoctmp\childdocname
        \def\childdoctmp{}
      \else
        \def\childdoctmp
        {
          \childdoctrue
          \includeonly{\childdocname}
          \def\childdocjob{#1}
          \def\jobname{#1}
        }
      \fi
      \expandafter
    \endgroup
    \childdoctmp
  \fi
}
%    \end{macrocode}

% \macro{\childdocof}
% The command |\childdocof| redirects
% compilation to the main file |#1|.
%    \begin{macrocode}
\newcommand{\childdocof}[1]
{
  \childdocdisable
  \childdoctrue
  \includeonly{\childdocname}
  \def\jobname{#1}
  \def\childdocjob{#1}
  \input{#1}
}
%    \end{macrocode}

% \macro{\childdocby}
% The command |\childdocby| ....
%    \begin{macrocode}
\newcommand{\childdocby}[2][]
{
  \childdocdisable
  \childdoctrue
  \childdocmanualtrue
  \if?#1?\else
    \def\jobname{#2}
  \fi
  \def\childdocjob{#2}
  \input{#2}
  \endinput
}
%    \end{macrocode}

% \macro{\childdocforward}
% The command |\childdocforward| redirects
% compilation to the main file or
% (if the optional argument is given) a child file.
% Parameters are set as if the main file
% or a child file starting with |\childdocof| was compiled.
% Then compilation is handed over to the main file:
%    \begin{macrocode}
\newcommand{\childdocforward}[2][]
{
  \begingroup
    \if?#1?
      \def\childdoctmp
      {
        \def\childdocname{#2}
        \def\childdocjob{#2}
        \def\jobname{#2}
        \input{#2}
        \endinput
      }
    \else
      \def\childdoctmp
      {
        \childdocdisable
        \def\childdocname{#2}
        \childdoctrue
        \includeonly{#2}
        \def\childdocjob{#1}
        \def\jobname{#1}
        \input{#1}
        \endinput
      }
    \fi
    \expandafter
  \endgroup
  \childdoctmp
}
%    \end{macrocode}

% \macro{\childdocforwardprefix}
% The command |\childdocforwardprefix| redirects
% compilation to the main or a child file by means of a pattern.
% The prefix |#1| in the current filename is replaced by |#2|
% and the suffix of the current filename is kept
% (it is assumed that the filename does not contain the substring `|~~~|'
% which is used as a delimiter).
% Compilation is handed over to the new file by |\childdocforward|:
%    \begin{macrocode}
\newcommand{\childdocforwardprefix}[3][]
{
  \begingroup
    \def\childdocextract #2##1~~~{\def\childdoctmp{\childdocforward[#1]{#3##1}}}
    \expandafter\childdocextract\childdocname~~~
    \expandafter
  \endgroup
  \childdoctmp
}
%    \end{macrocode}

% \macro{\childdoc}
% The deprecated macro |\childdoc| is a legacy version of |\childdocmain|:
%    \begin{macrocode}
\newcommand{\childdoc}{\childdocmain}
%    \end{macrocode}

% \macro{\childdocredirect}
% The deprecated macro |\childdocredirect| is a legacy version
% of |\childdocforward| and |\childdocforwardprefix|:
%    \begin{macrocode}
\newcommand{\childdocredirect}[2][]
{
  \begingroup
    \if?#1?
      \def\childdoctmp{\childdocforward{#2}}
    \else
      \def\childdoctmp{\childdocforwardprefix{#1}{#2}}
    \fi
    \expandafter
  \endgroup
  \childdoctmp
}
%    \end{macrocode}

%\iffalse
%</package>
%\fi
%
\endinput
|\\
|\childdocmain{|\textit{main}|}|\\
\end{tabular}
\end{center}
%
If |\jobname| does not match the argument \textit{main} of |\childdocmain|,
it is assumed that |\jobname| points to the child file to be compiled.
When using |\childdocmain| with the main file specified as argument,
it suffices to start a child file
with just |\input{|\textit{main}|}|
without loading of the package and using |\childdocof|.
If instead all processing is done
with the appropriate \textsf{childdoc} directives,
the argument of \textit{main} of |\childdocmain| can be empty.

An alternative version of the command line processing described
in \secref{sec:commandline} using the detection mechanism reads:
%
\begin{center}
|... -jobname "|\textit{target}|" "|[\textit{flags}]%
[|\def\jobname{|\textit{dest}|}|]|\input{|\textit{main}|}"|
\end{center}

%%%%%%%%%%%%%%%%%%%%%%%%%%%%%%%%%%%%%%%%%%%%%%%%%%%%%%%%%%%%%%%%%%%%%%%%%%%%%%%%
\subsection{Manual Code}
\label{sec:manual}

In case one cannot be certain whether the definitions file |childdoc.def|
is installed on the target \TeX{} distribution
and one prefers not to ship it,
it is conceivable to paste a few relevant commands into the sources.

To that end, drop all statements |% \iffalse
%
% childdoc.dtx Copyright (C) 2017-2018 Niklas Beisert
%
% This work may be distributed and/or modified under the
% conditions of the LaTeX Project Public License, either version 1.3
% of this license or (at your option) any later version.
% The latest version of this license is in
%   http://www.latex-project.org/lppl.txt
% and version 1.3 or later is part of all distributions of LaTeX
% version 2005/12/01 or later.
%
% This work has the LPPL maintenance status `maintained'.
%
% The Current Maintainer of this work is Niklas Beisert.
%
% This work consists of the files childdoc.dtx and childdoc.ins
% and the derived files childdoc.def and cdocsamp.tex with
% cdocsch1.tex, cdocsch2.tex, cdocsdrf.tex, cdocsfn1.tex, cdocsfn2.tex.
%
%<package>\ifdefined\childdocmain\endinput\fi
%<package>\ProvidesFile{childdoc.def}[2018/12/30 v2.0 child document driver]
%<samplemain>\ProvidesFile{cdocsamp.tex}[2018/12/30 v2.0 sample for childdoc]
%<*driver>
%\ProvidesFile{childdoc.drv}[2018/12/30 v2.0 childdoc reference manual file]
\PassOptionsToClass{10pt,a4paper}{article}
\documentclass{ltxdoc}

\usepackage[margin=35mm]{geometry}
\usepackage{hyperref}
\usepackage{hyperxmp}
\usepackage[usenames]{color}

\hypersetup{colorlinks=true}
\hypersetup{pdfstartview=FitH}
\hypersetup{pdfpagemode=UseNone}
\hypersetup{pdfsource={}}
\hypersetup{pdflang={en-UK}}
\hypersetup{pdfcopyright={Copyright 2017-2018 Niklas Beisert.
  This work may be distributed and/or modified under the
  conditions of the LaTeX Project Public License, either version 1.3
  of this license or (at your option) any later version.}}
\hypersetup{pdflicenseurl={http://www.latex-project.org/lppl.txt}}
\hypersetup{pdfcontactaddress={ETH Zurich, ITP, HIT K,
  Wolfgang-Pauli-Strasse 27}}
\hypersetup{pdfcontactpostcode={8093}}
\hypersetup{pdfcontactcity={Zurich}}
\hypersetup{pdfcontactcountry={Switzerland}}
\hypersetup{pdfcontactemail={nbeisert@itp.phys.ethz.ch}}
\hypersetup{pdfcontacturl={http://people.phys.ethz.ch/\xmptilde nbeisert/}}

\newcommand{\secref}[1]{\hyperref[#1]{section \ref*{#1}}}

\parskip1ex
\parindent0pt
\let\olditemize\itemize
\def\itemize{\olditemize\parskip0pt}

\begin{document}

\title{The \textsf{childdoc} Package}
\hypersetup{pdftitle={The childdoc Package}}
\author{Niklas Beisert\\[2ex]
  Institut f\"ur Theoretische Physik\\
  Eidgen\"ossische Technische Hochschule Z\"urich\\
  Wolfgang-Pauli-Strasse 27, 8093 Z\"urich, Switzerland\\[1ex]
  \href{mailto:nbeisert@itp.phys.ethz.ch}
  {\texttt{nbeisert@itp.phys.ethz.ch}}}
\hypersetup{pdfauthor={Niklas Beisert}}
\hypersetup{pdfsubject={Manual for the LaTeX2e Package childdoc}}
\date{30 December 2018, \textsf{v2.0}}
\maketitle

\begin{abstract}\noindent
\textsf{childdoc} is a \LaTeXe{} package
that enables the direct compilation
of document sections included by |\include|
to individual files.
\end{abstract}

\begingroup
\parskip0ex
\tableofcontents
\endgroup

%%%%%%%%%%%%%%%%%%%%%%%%%%%%%%%%%%%%%%%%%%%%%%%%%%%%%%%%%%%%%%%%%%%%%%%%%%%%%%%%
%%%%%%%%%%%%%%%%%%%%%%%%%%%%%%%%%%%%%%%%%%%%%%%%%%%%%%%%%%%%%%%%%%%%%%%%%%%%%%%%
\section{Introduction}

\LaTeX{} provides a mechanism to structure a large document (such as a book)
into a main file and several child files (containing the chapters)
using the |\include| command.
This mechanism is beneficial for documents
which span hundreds of pages in order to
make the source file(s) more manageable.
Moreover, compilation can be restricted to
selected child files by means of the |\includeonly| command.
The latter feature can be used to reduce the compilation time while editing
(this was significantly more useful in the earlier days of \LaTeX{})
or to generate a smaller document which is easier to navigate.
Another application of |\includeonly| is to generate
documents consisting of selected parts of the complete document.

However, there are a few drawbacks of the plain |\include| mechanism:
\begin{itemize}
\item
The child files cannot be compiled on their own,
they can only be compiled via the main file.
A naive editing environment
(such as a text editor with an option
to have the current file processed by \LaTeX)
may require one to switch to the main file before compiling;
attempting to compile the child file produces errors.
\item
The main file must be modified (each time)
to adjust the |\includeonly| command
to the present needs. This easily leaves the main file in a messy state.
\item
The generated document will always carry the filename
of the main document. This is inconvenient if
several child files are to be compiled and
to be kept for distribution.
\end{itemize}

The present package provides a simple interface
to make child files individually compilable by \LaTeX{}.
Compiling a child file then has the same effect as compiling
the main file with an |\includeonly| command
to select the appropriate child.
Moreover the generated document will carry the name of the child
rather than the main file.
This resolves all three above issues.

This feature is meant to make the editing of books,
thesis documents and lecture notes somewhat more convenient.
However, the package can also be used efficiently for
composing a series of documents (such as exercise sheets)
which are typically distributed individually.
It then assists the author in generating the individual documents
(potentially in different versions)
as well as a document containing the collected series.
Another application is in developing style files
or other kinds of included material
where compilation of the style file could redirect
to a sample or test file.

%%%%%%%%%%%%%%%%%%%%%%%%%%%%%%%%%%%%%%%%%%%%%%%%%%%%%%%%%%%%%%%%%%%%%%%%%%%%%%%%
%%%%%%%%%%%%%%%%%%%%%%%%%%%%%%%%%%%%%%%%%%%%%%%%%%%%%%%%%%%%%%%%%%%%%%%%%%%%%%%%
\section{Usage}

First of all, the package \textsf{childdoc} is \emph{not} a standard
\LaTeXe{} |.sty| style file! Therefore it needs to be invoked in
a non-standard way.

%%%%%%%%%%%%%%%%%%%%%%%%%%%%%%%%%%%%%%%%%%%%%%%%%%%%%%%%%%%%%%%%%%%%%%%%%%%%%%%%
\subsection{Included Files}
\label{sec:include}

%%%%%%%%%%%%%%%%%%%%%%%%%%%%%%%%%%%%%%%%
\DescribeMacro{\childdocmain}
To use the package, add the commands
\begin{center}
\begin{tabular}{l}
|\input{childdoc.def}|\\
|\childdocmain{}|\\
\end{tabular}
\end{center}
at the very top of the main \LaTeX{} file,
in particular \emph{before} the |\documentclass| statement!
The argument of |\childdocmain| should be left empty
(but it must be present).

%%%%%%%%%%%%%%%%%%%%%%%%%%%%%%%%%%%%%%%%
\DescribeMacro{\childdocof}
Furthermore, add the commands
\begin{center}
\begin{tabular}{l}
|\input{childdoc.def}|\\
|\childdocof{|\textit{main}|}|\\
\end{tabular}
\end{center}
at the top of every child file \textit{child}
which is included by |\include{|\textit{child}|}|
from within the main file
(or at least for those files to be compiled individually).
The argument \textit{main} must be the filename of the main file.

There are a couple of
considerations in setting up the main and child documents:

%%%%%%%%%%%%%%%%%%%%%%%%%%%%%%%%%%%%%%%%
\paragraph{Restrictions.}

Please note the following restrictions:
\begin{itemize}
\item
|\childdocmain| must be called with one argument \textit{main}
to ensure compatibility with earlier version of the package.
It must either be empty (|\childdocmain{}|)
or precisely match the filename of the main file in which it is specified.
See \secref{sec:detection} for further information.
\item
The filename \textit{main} must be specified without the |.tex| extension.
\item
The filename \textit{main} is case sensitive
(even in case-insensitive file systems)
due to internal string comparison.
\item
The argument \textit{main} should be fully expanded, it cannot be a macro.
\item
Subdirectories and special characters should be avoided in filenames.
\item
The command |\childdocmain{|\textit{main}|}| must be followed by a whitespace.
It should not be followed immediately by another command
or by a comment mark `|%|'.
This is because the \TeX{} parser reads the token immediately following
the argument of |\childdocmain| and puts it
at the beginning of every child section;
however, a white\-space is ignored.
\end{itemize}

%%%%%%%%%%%%%%%%%%%%%%%%%%%%%%%%%%%%%%%%
\paragraph{Content of Main File.}

It is advisable to place all content in the child files included by |\include|.
Any output contained in the main file will appear in all child documents
unless suppressed manually;
it cannot be suppressed automatically by the |\includeonly| directive
and thus should normally be avoided.
A method to include some content in the main file
by means of conditional processing is described in \secref{sec:conditional}.

%%%%%%%%%%%%%%%%%%%%%%%%%%%%%%%%%%%%%%%%
\paragraph{Page Numbering.}

When only a part of the document is compiled,
the appropriate numbering of pages
(as well as other status parameters)
is determined from the |.aux| files.
The latter contain information from previous passes.
However this information needs to propagate through
all intermediate child documents.
Therefore the page numbering in child documents may well
be inconsistent until the complete document is compiled at least once.

A useful (if unconventional) way to always ensure a consistent
page numbering is to restart the numbering in each child document
and denote the pages by `\textit{child}|.|\textit{page}'
where \textit{child} represents the chapter/section number of the child file.
This can be achieved by the command
|\numberwithin{page}{|\textit{child}|}|
of the \textsf{amsmath} package
where \textit{child} can be |chapter| or |section|
depending on the chosen structuring.
Alternatively, one can modify the macro |\thepage| appropriately
and reset the counter |page| at the start of each child file.

%%%%%%%%%%%%%%%%%%%%%%%%%%%%%%%%%%%%%%%%%%%%%%%%%%%%%%%%%%%%%%%%%%%%%%%%%%%%%%%%
\subsection{Conditional Processing}
\label{sec:conditional}

The package provides a mechanism to compile different versions
of a document. To customise the versions further some conditional processing
can come in handy to distinguish which version is being compiled.
The package provides two macros to describe the compilation context:

%%%%%%%%%%%%%%%%%%%%%%%%%%%%%%%%%%%%%%%%
\DescribeMacro{\ifchilddoc}
The conditional |\ifchilddoc| distinguishes between the compilation of
child documents and the main document:
%
\begin{center}
|\ifchilddoc |\textit{child-code}| |[|\||else |\textit{main-code}]| \||fi|
\end{center}

%%%%%%%%%%%%%%%%%%%%%%%%%%%%%%%%%%%%%%%%
\DescribeMacro{\childdocname}
\DescribeMacro{\childdocjob}
The macro |\childdocname| contains the filename (without extension)
of the main or child file being processed.
Note that |\childdocjob| will always contain the name of the main file.

%%%%%%%%%%%%%%%%%%%%%%%%%%%%%%%%%%%%%%%%
\paragraph{Title Page.}

Conditional processing can be used to include a title or banner page
in the main document when proper precautions are taken.
Importantly, the code in the main file should ensure that the page counter
(as well as other status parameters which are stored in the |.aux| files)
takes the same value after the conditional processing.
Otherwise the page numbers may take divergent values
depending on which part is compiled.

For example, a title page could be declared by:
%
\begin{center}
\begin{tabular}{l}
|\ifchilddoc\||else|\\
|\addtocounter{page}{-1}|\\
\textit{code for title page}\\
|\newpage|\\
|\||fi|
\end{tabular}
\end{center}
%
A banner page for the child documents can be generated by:
%
\begin{center}
\begin{tabular}{l}
|\ifchilddoc|\\
|\addtocounter{page}{-1}|\\
\textit{code for banner page}\\
|\newpage|\\
|\||fi|
\end{tabular}
\end{center}
%
Here one could write a message such as:
\begin{center}
|This is the part \childdocname{} of \childdocjob{}.|
\end{center}

%%%%%%%%%%%%%%%%%%%%%%%%%%%%%%%%%%%%%%%%%%%%%%%%%%%%%%%%%%%%%%%%%%%%%%%%%%%%%%%%
\subsection{Flags}
\label{sec:flags}

The package makes it easy to generate different versions
of the main or child documents.
To this end compilation flags can be defined
and assigned different default values.
They will be particularly useful in conjunction
with the forwarding mechanism described in \secref{sec:forward}.

For example, it may be useful to have a flag |\version|
which can be set to |draft| or |final|.
The document source will contain some conditional code
depending on the value of |\version|.
Suppose further, the flag should default to |final| for the main file
and to |draft| for child files
which is a natural assignment for editing the document.
This is achieved by placing the following code
in the preamble of the main document
(below the |\childdocmain| directive):
%
\begin{center}
\begin{tabular}{l}
|\ifchilddoc|\\
|\providecommand{\version}{draft}|\\
|\||else|\\
|\providecommand{\version}{final}|\\
|\||fi|
\end{tabular}
\end{center}
%
The definition by |\providecommand| makes sure
that previous definitions are not overwritten.
Further statements |\providecommand{\version}{...}|
can thus be added before the above code to override it.

For the main file, one might add a line
(between |\childdocmain| and the above block)
%
\begin{center}
|%\ifchilddoc\||else\providecommand{\version}{draft}\||fi|
\end{center}
%
which can be uncommented to produce a draft version.
Likewise one can add a line to the very top of a child file
(above the |\childdocof{|\textit{main}|}| directive)
%
\begin{center}
|%\providecommand{\version}{final}|
\end{center}
%
which can be uncommented to produce the final version of this child document.

%%%%%%%%%%%%%%%%%%%%%%%%%%%%%%%%%%%%%%%%%%%%%%%%%%%%%%%%%%%%%%%%%%%%%%%%%%%%%%%%
\subsection{Forwarding}
\label{sec:forward}

Different versions of the main or child documents
using compilation flags as described in \secref{sec:flags}
can be (permanently) stored in different files
for convenient compilation, viewing and distribution.
To this end, the package defines a command
to pass on compilation to a different file:

%%%%%%%%%%%%%%%%%%%%%%%%%%%%%%%%%%%%%%%%
\DescribeMacro{\childdocforward}
The command |\childdocforward| redirects processing to
another source file:
%
\begin{center}
\begin{tabular}{l}
|\input{childdoc.def}|\\
|\childdocforward[|\textit{main}|]{|\textit{dest}|}|\\
\end{tabular}
\end{center}
%
The argument \textit{dest} is the destination file
(without extension).
It should be the main file or one of the child files.
Note that further \textsf{childdoc} directives
such as |\childdocof| and |\childdocforward|
in the indicated file will be processed in this form.
The optional argument \textit{main}
passes on directly to the main file \textit{main}
while pretending to compile the child \textit{dest}.
This form behaves as if \textit{dest}
issues |\childdocof{|\textit{main}|}| right away,
and no further \textsf{childdoc} directives will be processed.

%%%%%%%%%%%%%%%%%%%%%%%%%%%%%%%%%%%%%%%%
\DescribeMacro{\...prefix}
In the alternative form |\childdocforwardprefix|,
%
\begin{center}
\begin{tabular}{l}
|\input{childdoc.def}|\\
|\childdocforwardprefix[|\textit{main}|]{|\textit{prefix}|}{|\textit{dest}|}|
\end{tabular}
\end{center}
%
the destination file is determined by a pattern
depending on the current file:
To make this work, the current file must be called
`{\textit{prefix}\hspace{0.2em}\textit{suffix}}'
with \textit{prefix} matching precisely the argument.
Processing is then passed on to the file
`{\textit{dest}\hspace{0.2em}\textit{suffix}}'.
Surely, the same effect is achieved by
directly specifying the
argument `{\textit{dest}\hspace{0.2em}\textit{suffix}}'
in the first form.
However, that requires to set up a different file
for each child. With the alternative form of the command
all these files can have exactly the same content
which simplifies setting them up and maintaining them.

For example, the following file |draft.tex|
with a compilation flag |\version| as described in \secref{sec:flags}
compiles the main document as a draft:
%
\begin{center}
\begin{tabular}{l}
|\def\version{draft}|\\
|\input{childdoc.def}|\\
|\childdocforward{|\textit{main}|}|
\end{tabular}
\end{center}
%
Likewise, the following files |final|\textit{nn}|.tex|
compile the final version of the child document
|child|\textit{nn}|.tex|:
%
\begin{center}
\begin{tabular}{l}
|\def\version{final}|\\
|\input{childdoc.def}|\\
|\childdocforwardprefix{final}{child}|
\end{tabular}
\end{center}
%

Note that when several versions of a main file and/or of each child file
are to be generated, it may be convenient to set up a |Makefile| or
shell script to automatise the process.

%%%%%%%%%%%%%%%%%%%%%%%%%%%%%%%%%%%%%%%%%%%%%%%%%%%%%%%%%%%%%%%%%%%%%%%%%%%%%%%%
\subsection{Command Line Processing}
\label{sec:commandline}

The effect of redirection files can also be achieved by invoking
the \LaTeX{} compiler with a more elaborate command line.
Most conveniently this should be done as part
of a shell script or a |Makefile|.

When using \textsf{childdoc} in the main file, the following
command lines effectively perform a redirection
(note that depending on the shell being used,
backslashes may have to be doubled: `|\|' $\to$ `|\\|'):
%
\begin{center}
|... -jobname "|\textit{target}|" |\\|"|[\textit{flags}]%
|\input{childdoc.def}\childdocforward[|\textit{main}|]{|\textit{dest}|}"|
\end{center}
%
Here \textit{target} is the name of the output file,
\textit{main} is the name of the main file
and \textit{dest} is the name of the main or child file to be processed
(all filenames without extensions).
The optional argument \textit{main} can be omitted
if \textit{main} matches \textit{dest}.
Optionally, compilation \textit{flags} can be defined via |\def| commands.
This command line makes the \TeX{} engine believe
it is compiling the file \textit{target}
whose content is specified as the latter parameter.
The provided code then forwards the processing to
\textit{main} or \textit{dest} as described in \secref{sec:forward}.

%%%%%%%%%%%%%%%%%%%%%%%%%%%%%%%%%%%%%%%%%%%%%%%%%%%%%%%%%%%%%%%%%%%%%%%%%%%%%%%%
\subsection{Include by Input}
\label{sec:input}

Including child documents by |\include| has some restrictions by design.
Most notably, the content of a child document always occupies
its own set of pages; pages cannot be shared between child documents.
Usually, this behaviour makes perfect sense
because each child document contain an essential part of the document.
However, in some situations it may be desirable to compose
a document from a collection of parts
without having mandatory page breaks between then.
For this case, the package
provides a mechanism to include parts
by |\input| which can also be processed individually.
However, by construction this mechanism
requires manual handling of the content to be output.

%%%%%%%%%%%%%%%%%%%%%%%%%%%%%%%%%%%%%%%%
\DescribeMacro{\ifchilddocmanual}
The main file should be prepared as usual, see \secref{sec:include}.
However, the document body must make a distinction
between processing of an individual part and of the main document, e.g.:
%
\begin{center}
\begin{tabular}{l}
|\ifchilddocmanual|\\
|\input{\childdocname}|\\
|\||else|\\
\textit{document body with }|\input{|\textit{part}|}|\\
|\||fi|
\end{tabular}
\end{center}
%
The conditional |\ifchilddocmanual| is true whenever
a part to be included by |\input| is being compiled,
and the name of the part is stored in |\childdocname|.

%%%%%%%%%%%%%%%%%%%%%%%%%%%%%%%%%%%%%%%%
\DescribeMacro{\childdocby}
Each part to be included by |\input| should start with:
%
\begin{center}
\begin{tabular}{l}
|\input{childdoc.def}|\\
|\childdocby{|\textit{main}|}|\\
\end{tabular}
\end{center}
%
The directive |\childdocby| is similar to |\childdocof|
described in \secref{sec:include},
but the subsequent selection of content must be done manually.
To that end, both |\ifchilddoc| and |\ifchilddocmanual|
will be true upon processing of a part,
and the name of the part is stored in |\childdocname|.
Note that |\jobname| will be set to the filename of the current part
so that each part receives an individual |.aux| file
that does not interfere with the |.aux| file(s) of the main document.
This behaviour can be altered by the alternative form
|\childdocby[*]{|\textit{main}|}| (with a non-empty optional argument)
which uses the |.aux| file of the main document
by setting |\jobname| to \textit{main}.

%%%%%%%%%%%%%%%%%%%%%%%%%%%%%%%%%%%%%%%%%%%%%%%%%%%%%%%%%%%%%%%%%%%%%%%%%%%%%%%%
\subsection{Driver Development}
\label{sec:driver}

The \textsf{childdoc} mechanism can also be use for the development
of definition files such as \LaTeX{} styles or classes.
This case differs from the above setup with multiple parts
included by |\include| in that no |\includeonly| should be invoked.
This can be achieved by starting the include file
(before |\ProvidesPackage|) with:
%
\begin{center}
\begin{tabular}{l}
|\input{childdoc.def}|\\
|\childdocforward{|\textit{main}|}|\\
\end{tabular}
\end{center}
%
or alternatively with:
%
\begin{center}
\begin{tabular}{l}
|\input{childdoc.def}|\\
|\childdocby{|\textit{main}|}|\\
\end{tabular}
\end{center}
%
Both forms have slightly different effects as described above.
The main file is prepared as usual, see \secref{sec:include}.

%%%%%%%%%%%%%%%%%%%%%%%%%%%%%%%%%%%%%%%%%%%%%%%%%%%%%%%%%%%%%%%%%%%%%%%%%%%%%%%%
\subsection{Legacy Detection}
\label{sec:detection}

The directive |\childdocmain| in the main file can detect
whether the complete document or merely a child is to be compiled
even without using the directive |\childdocof|.
This method is deprecated because it is less robust
and there is no compelling reason to use it;
it is merely provided for backward compatibility
and it may be removed in future versions.

If the detection mechanism is to be used,
it is mandatory to correctly specify
the filename of the main file as the argument of |\childdocmain|:
%
\begin{center}
\begin{tabular}{l}
|\input{childdoc.def}|\\
|\childdocmain{|\textit{main}|}|\\
\end{tabular}
\end{center}
%
If |\jobname| does not match the argument \textit{main} of |\childdocmain|,
it is assumed that |\jobname| points to the child file to be compiled.
When using |\childdocmain| with the main file specified as argument,
it suffices to start a child file
with just |\input{|\textit{main}|}|
without loading of the package and using |\childdocof|.
If instead all processing is done
with the appropriate \textsf{childdoc} directives,
the argument of \textit{main} of |\childdocmain| can be empty.

An alternative version of the command line processing described
in \secref{sec:commandline} using the detection mechanism reads:
%
\begin{center}
|... -jobname "|\textit{target}|" "|[\textit{flags}]%
[|\def\jobname{|\textit{dest}|}|]|\input{|\textit{main}|}"|
\end{center}

%%%%%%%%%%%%%%%%%%%%%%%%%%%%%%%%%%%%%%%%%%%%%%%%%%%%%%%%%%%%%%%%%%%%%%%%%%%%%%%%
\subsection{Manual Code}
\label{sec:manual}

In case one cannot be certain whether the definitions file |childdoc.def|
is installed on the target \TeX{} distribution
and one prefers not to ship it,
it is conceivable to paste a few relevant commands into the sources.

To that end, drop all statements |\input{childdoc.def}|
and perform the replacements as outlined below.
Instead of |\childdocmain{|\textit{main}|}| add the following code
to the top of the main file:
%
\begin{center}
\begin{tabular}{l}
|\||ifdefined\childdocname\endinput\||fi\newif\ifchilddoc|\\
|\edef\childdocname{\scantokens\expandafter{\jobname\noexpand}}|\\
|\def\childdocmain{|\textit{main}|}\||ifx\childdocmain\childdocname\||else|\\
|\childdoctrue\includeonly{\childdocname}\let\jobname\childdocmain\||fi|\\
\end{tabular}
\end{center}
%
Instead of |\childdocof{|\textit{main}|}| just include the main file
at the top of each child file:
%
\begin{center}
|\input{|\textit{main}|}|
\end{center}
%
A simple redirection |\childdocforward{|\textit{dest}|}| is achieved by:
%
\begin{center}
|\def\jobname{|\textit{dest}|}\input{\jobname}|
\end{center}
%
The redirection with prefix
|\childdocforwardprefix[|\textit{prefix}|]{|\textit{dest}|}|
is accomplished by:
%
\begin{center}
\begin{tabular}{l}
|{\edef\jobname{\scantokens\expandafter{\jobname\noexpand}}|\\
|\def\redirectjob |\textit{prefix}|#1~~~{\gdef\jobname{|\textit{dest}|#1}}|\\
|\expandafter\redirectjob\jobname~~~}\input{\jobname}|
\end{tabular}
\end{center}

In an alternative approach,
child documents can be compiled by a specific command line
without additional code or specific definitions:
%
\begin{center}
|... -jobname "|\textit{target}|" "|[\textit{flags}]%
|\includeonly{|\textit{dest}|}\input{|\textit{main}|}"|
\end{center}
%

%%%%%%%%%%%%%%%%%%%%%%%%%%%%%%%%%%%%%%%%%%%%%%%%%%%%%%%%%%%%%%%%%%%%%%%%%%%%%%%%
%%%%%%%%%%%%%%%%%%%%%%%%%%%%%%%%%%%%%%%%%%%%%%%%%%%%%%%%%%%%%%%%%%%%%%%%%%%%%%%%
\section{Information}

%%%%%%%%%%%%%%%%%%%%%%%%%%%%%%%%%%%%%%%%%%%%%%%%%%%%%%%%%%%%%%%%%%%%%%%%%%%%%%%%
\subsection{Copyright}

Copyright \copyright{} 2017--2018 Niklas Beisert

This work may be distributed and/or modified under the
conditions of the \LaTeX{} Project Public License, either version 1.3
of this license or (at your option) any later version.
The latest version of this license is in
  \url{http://www.latex-project.org/lppl.txt}
and version 1.3 or later is part of all distributions of \LaTeX{}
version 2005/12/01 or later.

This work has the LPPL maintenance status `maintained'.

The Current Maintainer of this work is Niklas Beisert.

This work consists of the files |README.txt|, |childdoc.ins| and |childdoc.dtx|
as well as the derived files |childdoc.def|, |cdocsamp.tex|
with |cdocsch1.tex|, |cdocsch2.tex|, |cdocspt3.tex|, |cdocspt4.tex|,
|cdocsdrf.tex|, |cdocsfn1.tex|, |cdocsfn2.tex|
as well as |childdoc.pdf|.

%%%%%%%%%%%%%%%%%%%%%%%%%%%%%%%%%%%%%%%%%%%%%%%%%%%%%%%%%%%%%%%%%%%%%%%%%%%%%%%%
\subsection{Files and Installation}

The package consists of the files:
%
\begin{center}
\begin{tabular}{ll}
    |README.txt|   & readme file \\
    |childdoc.ins| & installation file \\
    |childdoc.dtx| & source file \\
    |childdoc.def| & definition file \\
    |cdocsamp.tex| & sample main file \\
    |cdocsch1.tex| & sample include file \\
    |cdocsch2.tex| & sample include file \\
    |cdocspt3.tex| & sample part file \\
    |cdocspt4.tex| & sample part file \\
    |cdocsdrf.tex| & sample redirection file \\
    |cdocsfn1.tex| & sample redirection file \\
    |cdocsfn2.tex| & sample redirection file \\
    |childdoc.pdf| & manual
\end{tabular}
\end{center}
%
The distribution consists of the files
|README.txt|, |childdoc.ins| and |childdoc.dtx|.
%
\begin{itemize}
\item
Run (pdf)\LaTeX{} on |childdoc.dtx|
to compile the manual |childdoc.pdf| (this file).
\item
Run \LaTeX{} on |childdoc.ins| to create the definitions file |childdoc.def|
and the sample |cdocsamp.tex| with include files
|cdocsch1.tex|, |cdocsch2.tex|, |cdocspt3.tex|, |cdocspt4.tex|,
|cdocsdrf.tex|, |cdocsfn1.tex|, |cdocsfn2.tex|.
Then copy the file |childdoc.def| to an appropriate directory of your \LaTeX{}
distribution, e.g.\ \textit{texmf-root}|/tex/latex/childdoc|.
\end{itemize}

%%%%%%%%%%%%%%%%%%%%%%%%%%%%%%%%%%%%%%%%%%%%%%%%%%%%%%%%%%%%%%%%%%%%%%%%%%%%%%%%
\subsection{Related CTAN Packages}

There are several other packages which offer a similar functionality:
%
\begin{itemize}
\item
The packages
\href{http://ctan.org/pkg/docmute}{\textsf{docmute}},
\href{http://ctan.org/pkg/includex}{\textsf{includex}} and
\href{http://ctan.org/pkg/standalone}{\textsf{standalone}}
provide commands to include only the document body of
a child file thus allowing both files to be compiled individually.
\item
The packages \href{http://ctan.org/pkg/subdocs}{\textsf{subdocs}}
and \href{http://ctan.org/pkg/subfiles}{\textsf{subfiles}}
provide structures in which the main and child documents can be
encapsulated and allowing them to be compiled individually.
The inclusion mechanism is different from the conventional |\include|.
\item
The package \href{http://ctan.org/pkg/combine}{\textsf{combine}}
is an elaborate solution to combine several documents into one.
\end{itemize}
%
See also the CTAN topic \href{http://ctan.org/topic/subdocs}{\textsf{subdocs}}
for further related packages.
The present package differs from the above solutions in that
a document structure constructed with the conventional |\include| mechanism
just needs two extra commands at the top of every file
such that all constituent files can be compiled individually.

%%%%%%%%%%%%%%%%%%%%%%%%%%%%%%%%%%%%%%%%%%%%%%%%%%%%%%%%%%%%%%%%%%%%%%%%%%%%%%%%
%\subsection{Feature Suggestions}
%
%The following is a list of features which may be useful for future
%versions of this package:
%%
%\begin{itemize}
%\item
%\ldots
%\end{itemize}

%%%%%%%%%%%%%%%%%%%%%%%%%%%%%%%%%%%%%%%%%%%%%%%%%%%%%%%%%%%%%%%%%%%%%%%%%%%%%%%%
\subsection{Revision History}

%%%%%%%%%%%%%%%%%%%%%%%%%%%%%%%%%%%%%%%%
\paragraph{v2.0:} 2018/12/30

\begin{itemize}
\item
immediate forward processing
\item
added |\childdocby| mechanism
\item
manual restructured
\end{itemize}

%%%%%%%%%%%%%%%%%%%%%%%%%%%%%%%%%%%%%%%%
\paragraph{v1.6:} 2018/01/17

\begin{itemize}
\item
application for development of include files
\item
corrections to manual
\end{itemize}

%%%%%%%%%%%%%%%%%%%%%%%%%%%%%%%%%%%%%%%%
\paragraph{v1.5:} 2017/05/21

\begin{itemize}
\item
more complete structuring introduced
\item
|\childdocof| introduced
\item
|\childdoc| renamed to |\childdocmain|
\item
|\childredirect| renamed to |\childdocforward| and |\childdocforwardprefix|
and functionality expanded
\end{itemize}

%%%%%%%%%%%%%%%%%%%%%%%%%%%%%%%%%%%%%%%%
\paragraph{v1.0:} 2017/04/27

\begin{itemize}
\item
manual and install package
\item
first version published on CTAN
\end{itemize}

%%%%%%%%%%%%%%%%%%%%%%%%%%%%%%%%%%%%%%%%
\paragraph{v0.6:} 2017/04/26

\begin{itemize}
\item
redirection mechanism added
\end{itemize}

%%%%%%%%%%%%%%%%%%%%%%%%%%%%%%%%%%%%%%%%
\paragraph{v0.5:} 2017/04/26

\begin{itemize}
\item
functionality in definition file
\end{itemize}


%%%%%%%%%%%%%%%%%%%%%%%%%%%%%%%%%%%%%%%%%%%%%%%%%%%%%%%%%%%%%%%%%%%%%%%%%%%%%%%%
%%%%%%%%%%%%%%%%%%%%%%%%%%%%%%%%%%%%%%%%%%%%%%%%%%%%%%%%%%%%%%%%%%%%%%%%%%%%%%%%
%%%%%%%%%%%%%%%%%%%%%%%%%%%%%%%%%%%%%%%%%%%%%%%%%%%%%%%%%%%%%%%%%%%%%%%%%%%%%%%%
\appendix

\settowidth\MacroIndent{\rmfamily\scriptsize 000\ }

 \DocInput{childdoc.dtx}

\end{document}
%</driver>
% \fi
%
% %%%%%%%%%%%%%%%%%%%%%%%%%%%%%%%%%%%%%%%%%%%%%%%%%%%%%%%%%%%%%%%%%%%%%%%%%%%%%%
% %%%%%%%%%%%%%%%%%%%%%%%%%%%%%%%%%%%%%%%%%%%%%%%%%%%%%%%%%%%%%%%%%%%%%%%%%%%%%%
% \section{Sample}
%\iffalse
%<*samplemain>
%\fi
%
% The following presents a sample document
% with two chapters, two parts, a title page,
% a compile flag as well as three forwarding files to set the flag.
% It consists of eight |.tex| files:
% \begin{center}
% \begin{tabular}{ll}
% |cdocsamp.tex|&main file\\
% |cdocsch1.tex|&include file for chapter 1\\
% |cdocsch2.tex|&include file for chapter 2\\
% |cdocspt3.tex|&include file for part 3\\
% |cdocspt4.tex|&include file for part 4\\
% |cdocsdrf.tex|&forwarding file for main file in draft mode\\
% |cdocsfi1.tex|&forwarding file for final version of chapter 1\\
% |cdocsfi2.tex|&forwarding file for final version of chapter 2\\
% \end{tabular}
% \end{center}
% Each of the eight files can be compiled directly by the \LaTeX{} compiler.
%
% %%%%%%%%%%%%%%%%%%%%%%%%%%%%%%%%%%%%%%
% \paragraph{Main File.}
%
% The main file is called |cdocsamp.tex|.
%
% Load the \textsf{childdoc} definitions and
% declare the filename for the main document:
%    \begin{macrocode}
\input{childdoc.def}
\childdocmain{}
%    \end{macrocode}

% Optional override for |\version| flag:
%    \begin{macrocode}
%%\ifchilddoc\else\providecommand{\version}{draft}\fi
%    \end{macrocode}

% Define the default values for the |\version| flag
% (|final| for the main file and |draft| for childs):
%    \begin{macrocode}
\ifchilddoc
\providecommand{\version}{draft}
\else
\providecommand{\version}{final}
\fi
%    \end{macrocode}

% Load the standard document class:
%    \begin{macrocode}
\documentclass[12pt]{article}
%    \end{macrocode}

% Start the document body:
%    \begin{macrocode}
\begin{document}
%    \end{macrocode}

% Declare a title page.
% Print title, part of document being processed and version flag:
%    \begin{macrocode}
\addtocounter{page}{-1}
\begin{center}
{\LARGE\bfseries{}childdoc example\par}
\vspace{1cm}
\ifchilddoc
\ifchilddocmanual part\else chapter\fi:
`\childdocname' of `\childdocjob'\par
\else
main document: `\childdocjob'\par
\fi
version: \version\par
\end{center}
\newpage
%    \end{macrocode}

% Manually include selected file,
% otherwise process as usual:
%    \begin{macrocode}
\ifchilddocmanual
\section*{part `\childdocname'}
\input{\childdocname}
\else
%    \end{macrocode}

% Include the two chapters:
%    \begin{macrocode}
\include{cdocsch1}
\include{cdocsch2}
%    \end{macrocode}

% Include the two parts unless only chapters should be displayed:
%    \begin{macrocode}
\ifchilddoc\else
\section{part three}
\input{cdocspt3}
\section{part four}
\input{cdocspt4}
\fi
%    \end{macrocode}

% Process as usual until here:
%    \begin{macrocode}
\fi
%    \end{macrocode}

% End of document body:
%    \begin{macrocode}
\end{document}
%    \end{macrocode}
%\iffalse
%</samplemain>
%\fi
%
% %%%%%%%%%%%%%%%%%%%%%%%%%%%%%%%%%%%%%%
% \paragraph{Chapter Include Files.}
%
% The include files are called |cdocsch1.tex| and |cdocsch2.tex|.
%
%\iffalse
%<*samplechap1|samplechap2>
%\fi

% Optional override for |\version| flag:
%    \begin{macrocode}
%%\providecommand{\version}{final}
%    \end{macrocode}

% Include the main document:
%    \begin{macrocode}
\input{childdoc.def}
\childdocof{cdocsamp}
%    \end{macrocode}

%\iffalse
%</samplechap1|samplechap2>
%\fi
%
%\iffalse
%<*samplechap1>
%\fi
% Some text for chapter 1:
%    \begin{macrocode}
\section{one}
some text in chapter one
%    \end{macrocode}

%\iffalse
%</samplechap1>
%\fi
% Some text for chapter 2:
%\iffalse
%<*samplechap2>
%\fi
%    \begin{macrocode}
\section{two}
more text in chapter two
%    \end{macrocode}

%\iffalse
%</samplechap2>
%\fi
%
% %%%%%%%%%%%%%%%%%%%%%%%%%%%%%%%%%%%%%%
% \paragraph{Part Include Files.}
%
% The include files are called |cdocspt3.tex| and |cdocspt4.tex|.
%
%\iffalse
%<*samplepart3|samplepart4>
%\fi

% Optional override for |\version| flag:
%    \begin{macrocode}
%%\providecommand{\version}{final}
%    \end{macrocode}

% Include the main document:
%    \begin{macrocode}
\input{childdoc.def}
\childdocby{cdocsamp}
%    \end{macrocode}

%\iffalse
%</samplepart3|samplepart4>
%\fi
%
%\iffalse
%<*samplepart3>
%\fi
% Some text for part 3:
%    \begin{macrocode}
some text in part three
%    \end{macrocode}

%\iffalse
%</samplepart3>
%\fi
% Some text for part 4:
%\iffalse
%<*samplepart4>
%\fi
%    \begin{macrocode}
more text in part four
%    \end{macrocode}

%\iffalse
%</samplepart4>
%\fi
%
% %%%%%%%%%%%%%%%%%%%%%%%%%%%%%%%%%%%%%%
% \paragraph{Forwarding for a Complete Draft.}
%
% The following forwarding file |cdocsdrf.tex|
% compiles the main document in draft mode:
%\iffalse
%<*sampledraft>
%\fi
%    \begin{macrocode}
\def\version{draft}
\input{childdoc.def}
\childdocforward{cdocsamp}
%    \end{macrocode}

%\iffalse
%</sampledraft>
%\fi
%
% %%%%%%%%%%%%%%%%%%%%%%%%%%%%%%%%%%%%%%
% \paragraph{Forwarding for Final Version of the Chapters.}
%
% The following forwarding files |cdocsfn1.tex| and |cdocsfn2.tex|
% (with identical content)
% compile the final versions of the child documents
% |cdocsch1.tex| and |cdocsch2.tex|, respectively:
%\iffalse
%<*samplefinal>
%\fi
%    \begin{macrocode}
\def\version{final}
\input{childdoc.def}
\childdocforwardprefix[cdocsamp]{cdocsfn}{cdocsch}
%    \end{macrocode}

%\iffalse
%</samplefinal>
%\fi
%
% %%%%%%%%%%%%%%%%%%%%%%%%%%%%%%%%%%%%%%
% \paragraph{Command Line Processing.}
%
% The following three command lines generate the output files
% |cdocscld|, |cdocscl1| and |cdocscl2|
% which should be identical to
% |cdocsdrf|, |cdocsch1| and |cdocsfn2|, respectively:
% \begin{center}
% \begin{tabular}{l}
% |latex -jobname cdocscld \|\\
% |  "\def\version{draft}\input{childdoc.def}\childdocforward{cdocsamp}"|\\
% |latex -jobname cdocscl1 \|\\
% |  "\input{childdoc.def}\childdocforward[cdocsamp]{cdocsch1}"|\\
% |latex -jobname cdocscl2 \|\\
% |  "\def\version{final}\input{childdoc.def}\childdocforward{cdocsch2}"|
% \end{tabular}
% \end{center}
% Note that the trailing backslash on each first line
% merely continues the input to the second line
% (for convenient cut ant paste).
% Furthermore, the command |latex| can be replaced by any
% of its alternative versions such as |pdflatex|.
%
% %%%%%%%%%%%%%%%%%%%%%%%%%%%%%%%%%%%%%%%%%%%%%%%%%%%%%%%%%%%%%%%%%%%%%%%%%%%%%%
% %%%%%%%%%%%%%%%%%%%%%%%%%%%%%%%%%%%%%%%%%%%%%%%%%%%%%%%%%%%%%%%%%%%%%%%%%%%%%%
% \section{Implementation}
%\iffalse
%<*package>
%\fi
%
% This section describes the definitions file |childdoc.def|.

% The definitions cannot be loaded using |\usepackage| or |\RequirePackage|
% which has a mechanism to prevent loading a style file more than once.
% When loading the definitions by means of |\input|
% multiple instances have to be prevented manually:
%\iffalse
%This code needs to be before the `\ProvidesFile' directive
%which is defined at the beginning of this file.
%Therefore it is also placed there and commented out here.
%</package>
%<*discard>
%\fi
%    \begin{macrocode}
\ifdefined\childdocmain\endinput\fi
%    \end{macrocode}
%\iffalse
%</discard>
%<*package>
%\fi
%
% \macro{\ifchilddoc}
% \macro{\ifchilddocmanual}
% The conditional |\ifchilddoc| tells whether a
% child (true) or main (false) document is being compiled.
% The conditional |\ifchilddocmanual| tells whether
% the |\includeonly| mechanism is used (false) or
% the selection of child files must be performed manually (true).
% The definitions initialise to false:
%    \begin{macrocode}
\newif\ifchilddoc
\newif\ifchilddocmanual
%    \end{macrocode}

% \macro{\childdocname}
% \macro{\childdocjob}
% The macro |\childdocname| stores the name of the main document
% to be compiled. The macro |\childdocjob| stores the name of
% the document on which the \LaTeX{} compiler was originally invoked.
% The content of |\jobname| cannot be compared
% to filenames specified in the source due to different catcodes.
% The following code rescans |\jobname|, stores the result
% in |\childdocname| and saves a copy in |\childdocjob|:
%    \begin{macrocode}
\edef\childdocname{\scantokens\expandafter{\jobname\noexpand}}
\let\childdocjob\childdocname
%    \end{macrocode}

% \macro{\childdocdisable}
% The macro |\childdocdisable| prevents the main file
% from being processed more than once.
% At this stage, the main document command |\childdocmain|
% is assumed to be called once again where it should do nothing.
% Any subsequent call to it should prevent
% a secondary processing of the main document
% It overwrites the forwarding commands
% |\childdocof| and |\childdocforward|
% with empty macros to prevent further inclusions of the main document:
%    \begin{macrocode}
\newcommand{\childdocdisable}
{
  \renewcommand{\childdocmain}[1]{\renewcommand{\childdocmain}[1]{\endinput}}
  \renewcommand{\childdocof}[1]{}
  \renewcommand{\childdocby}[2][]{}
  \renewcommand{\childdocforward}[2][]{}
  \renewcommand{\childdocdisable}{}
}
%    \end{macrocode}

% \macro{\childdocmain}
% The macro |\childdocmain| is to be called at the top of the main file
% with nothing or the main filename (without extension) as argument.
% First, it breaks loops.
% If the argument is not empty and does not match |\childdocname|
% (which is set by the first inclusion of |childdoc.def|),
% |\ifchilddoc| is set to true, |\includeonly| is applied to the child file
% and |\jobname| is set to the main file
% (for proper handling of |.aux| files):
%    \begin{macrocode}
\newcommand{\childdocmain}[1]
{
  \childdocdisable\childdocmain{}
  \if?#1?\else
    \begingroup
      \def\childdoctmp{#1}
      \ifx\childdoctmp\childdocname
        \def\childdoctmp{}
      \else
        \def\childdoctmp
        {
          \childdoctrue
          \includeonly{\childdocname}
          \def\childdocjob{#1}
          \def\jobname{#1}
        }
      \fi
      \expandafter
    \endgroup
    \childdoctmp
  \fi
}
%    \end{macrocode}

% \macro{\childdocof}
% The command |\childdocof| redirects
% compilation to the main file |#1|.
%    \begin{macrocode}
\newcommand{\childdocof}[1]
{
  \childdocdisable
  \childdoctrue
  \includeonly{\childdocname}
  \def\jobname{#1}
  \def\childdocjob{#1}
  \input{#1}
}
%    \end{macrocode}

% \macro{\childdocby}
% The command |\childdocby| ....
%    \begin{macrocode}
\newcommand{\childdocby}[2][]
{
  \childdocdisable
  \childdoctrue
  \childdocmanualtrue
  \if?#1?\else
    \def\jobname{#2}
  \fi
  \def\childdocjob{#2}
  \input{#2}
  \endinput
}
%    \end{macrocode}

% \macro{\childdocforward}
% The command |\childdocforward| redirects
% compilation to the main file or
% (if the optional argument is given) a child file.
% Parameters are set as if the main file
% or a child file starting with |\childdocof| was compiled.
% Then compilation is handed over to the main file:
%    \begin{macrocode}
\newcommand{\childdocforward}[2][]
{
  \begingroup
    \if?#1?
      \def\childdoctmp
      {
        \def\childdocname{#2}
        \def\childdocjob{#2}
        \def\jobname{#2}
        \input{#2}
        \endinput
      }
    \else
      \def\childdoctmp
      {
        \childdocdisable
        \def\childdocname{#2}
        \childdoctrue
        \includeonly{#2}
        \def\childdocjob{#1}
        \def\jobname{#1}
        \input{#1}
        \endinput
      }
    \fi
    \expandafter
  \endgroup
  \childdoctmp
}
%    \end{macrocode}

% \macro{\childdocforwardprefix}
% The command |\childdocforwardprefix| redirects
% compilation to the main or a child file by means of a pattern.
% The prefix |#1| in the current filename is replaced by |#2|
% and the suffix of the current filename is kept
% (it is assumed that the filename does not contain the substring `|~~~|'
% which is used as a delimiter).
% Compilation is handed over to the new file by |\childdocforward|:
%    \begin{macrocode}
\newcommand{\childdocforwardprefix}[3][]
{
  \begingroup
    \def\childdocextract #2##1~~~{\def\childdoctmp{\childdocforward[#1]{#3##1}}}
    \expandafter\childdocextract\childdocname~~~
    \expandafter
  \endgroup
  \childdoctmp
}
%    \end{macrocode}

% \macro{\childdoc}
% The deprecated macro |\childdoc| is a legacy version of |\childdocmain|:
%    \begin{macrocode}
\newcommand{\childdoc}{\childdocmain}
%    \end{macrocode}

% \macro{\childdocredirect}
% The deprecated macro |\childdocredirect| is a legacy version
% of |\childdocforward| and |\childdocforwardprefix|:
%    \begin{macrocode}
\newcommand{\childdocredirect}[2][]
{
  \begingroup
    \if?#1?
      \def\childdoctmp{\childdocforward{#2}}
    \else
      \def\childdoctmp{\childdocforwardprefix{#1}{#2}}
    \fi
    \expandafter
  \endgroup
  \childdoctmp
}
%    \end{macrocode}

%\iffalse
%</package>
%\fi
%
\endinput
|
and perform the replacements as outlined below.
Instead of |\childdocmain{|\textit{main}|}| add the following code
to the top of the main file:
%
\begin{center}
\begin{tabular}{l}
|\||ifdefined\childdocname\endinput\||fi\newif\ifchilddoc|\\
|\edef\childdocname{\scantokens\expandafter{\jobname\noexpand}}|\\
|\def\childdocmain{|\textit{main}|}\||ifx\childdocmain\childdocname\||else|\\
|\childdoctrue\includeonly{\childdocname}\let\jobname\childdocmain\||fi|\\
\end{tabular}
\end{center}
%
Instead of |\childdocof{|\textit{main}|}| just include the main file
at the top of each child file:
%
\begin{center}
|\input{|\textit{main}|}|
\end{center}
%
A simple redirection |\childdocforward{|\textit{dest}|}| is achieved by:
%
\begin{center}
|\def\jobname{|\textit{dest}|}\input{\jobname}|
\end{center}
%
The redirection with prefix
|\childdocforwardprefix[|\textit{prefix}|]{|\textit{dest}|}|
is accomplished by:
%
\begin{center}
\begin{tabular}{l}
|{\edef\jobname{\scantokens\expandafter{\jobname\noexpand}}|\\
|\def\redirectjob |\textit{prefix}|#1~~~{\gdef\jobname{|\textit{dest}|#1}}|\\
|\expandafter\redirectjob\jobname~~~}\input{\jobname}|
\end{tabular}
\end{center}

In an alternative approach,
child documents can be compiled by a specific command line
without additional code or specific definitions:
%
\begin{center}
|... -jobname "|\textit{target}|" "|[\textit{flags}]%
|\includeonly{|\textit{dest}|}\input{|\textit{main}|}"|
\end{center}
%

%%%%%%%%%%%%%%%%%%%%%%%%%%%%%%%%%%%%%%%%%%%%%%%%%%%%%%%%%%%%%%%%%%%%%%%%%%%%%%%%
%%%%%%%%%%%%%%%%%%%%%%%%%%%%%%%%%%%%%%%%%%%%%%%%%%%%%%%%%%%%%%%%%%%%%%%%%%%%%%%%
\section{Information}

%%%%%%%%%%%%%%%%%%%%%%%%%%%%%%%%%%%%%%%%%%%%%%%%%%%%%%%%%%%%%%%%%%%%%%%%%%%%%%%%
\subsection{Copyright}

Copyright \copyright{} 2017--2018 Niklas Beisert

This work may be distributed and/or modified under the
conditions of the \LaTeX{} Project Public License, either version 1.3
of this license or (at your option) any later version.
The latest version of this license is in
  \url{http://www.latex-project.org/lppl.txt}
and version 1.3 or later is part of all distributions of \LaTeX{}
version 2005/12/01 or later.

This work has the LPPL maintenance status `maintained'.

The Current Maintainer of this work is Niklas Beisert.

This work consists of the files |README.txt|, |childdoc.ins| and |childdoc.dtx|
as well as the derived files |childdoc.def|, |cdocsamp.tex|
with |cdocsch1.tex|, |cdocsch2.tex|, |cdocspt3.tex|, |cdocspt4.tex|,
|cdocsdrf.tex|, |cdocsfn1.tex|, |cdocsfn2.tex|
as well as |childdoc.pdf|.

%%%%%%%%%%%%%%%%%%%%%%%%%%%%%%%%%%%%%%%%%%%%%%%%%%%%%%%%%%%%%%%%%%%%%%%%%%%%%%%%
\subsection{Files and Installation}

The package consists of the files:
%
\begin{center}
\begin{tabular}{ll}
    |README.txt|   & readme file \\
    |childdoc.ins| & installation file \\
    |childdoc.dtx| & source file \\
    |childdoc.def| & definition file \\
    |cdocsamp.tex| & sample main file \\
    |cdocsch1.tex| & sample include file \\
    |cdocsch2.tex| & sample include file \\
    |cdocspt3.tex| & sample part file \\
    |cdocspt4.tex| & sample part file \\
    |cdocsdrf.tex| & sample redirection file \\
    |cdocsfn1.tex| & sample redirection file \\
    |cdocsfn2.tex| & sample redirection file \\
    |childdoc.pdf| & manual
\end{tabular}
\end{center}
%
The distribution consists of the files
|README.txt|, |childdoc.ins| and |childdoc.dtx|.
%
\begin{itemize}
\item
Run (pdf)\LaTeX{} on |childdoc.dtx|
to compile the manual |childdoc.pdf| (this file).
\item
Run \LaTeX{} on |childdoc.ins| to create the definitions file |childdoc.def|
and the sample |cdocsamp.tex| with include files
|cdocsch1.tex|, |cdocsch2.tex|, |cdocspt3.tex|, |cdocspt4.tex|,
|cdocsdrf.tex|, |cdocsfn1.tex|, |cdocsfn2.tex|.
Then copy the file |childdoc.def| to an appropriate directory of your \LaTeX{}
distribution, e.g.\ \textit{texmf-root}|/tex/latex/childdoc|.
\end{itemize}

%%%%%%%%%%%%%%%%%%%%%%%%%%%%%%%%%%%%%%%%%%%%%%%%%%%%%%%%%%%%%%%%%%%%%%%%%%%%%%%%
\subsection{Related CTAN Packages}

There are several other packages which offer a similar functionality:
%
\begin{itemize}
\item
The packages
\href{http://ctan.org/pkg/docmute}{\textsf{docmute}},
\href{http://ctan.org/pkg/includex}{\textsf{includex}} and
\href{http://ctan.org/pkg/standalone}{\textsf{standalone}}
provide commands to include only the document body of
a child file thus allowing both files to be compiled individually.
\item
The packages \href{http://ctan.org/pkg/subdocs}{\textsf{subdocs}}
and \href{http://ctan.org/pkg/subfiles}{\textsf{subfiles}}
provide structures in which the main and child documents can be
encapsulated and allowing them to be compiled individually.
The inclusion mechanism is different from the conventional |\include|.
\item
The package \href{http://ctan.org/pkg/combine}{\textsf{combine}}
is an elaborate solution to combine several documents into one.
\end{itemize}
%
See also the CTAN topic \href{http://ctan.org/topic/subdocs}{\textsf{subdocs}}
for further related packages.
The present package differs from the above solutions in that
a document structure constructed with the conventional |\include| mechanism
just needs two extra commands at the top of every file
such that all constituent files can be compiled individually.

%%%%%%%%%%%%%%%%%%%%%%%%%%%%%%%%%%%%%%%%%%%%%%%%%%%%%%%%%%%%%%%%%%%%%%%%%%%%%%%%
%\subsection{Feature Suggestions}
%
%The following is a list of features which may be useful for future
%versions of this package:
%%
%\begin{itemize}
%\item
%\ldots
%\end{itemize}

%%%%%%%%%%%%%%%%%%%%%%%%%%%%%%%%%%%%%%%%%%%%%%%%%%%%%%%%%%%%%%%%%%%%%%%%%%%%%%%%
\subsection{Revision History}

%%%%%%%%%%%%%%%%%%%%%%%%%%%%%%%%%%%%%%%%
\paragraph{v2.0:} 2018/12/30

\begin{itemize}
\item
immediate forward processing
\item
added |\childdocby| mechanism
\item
manual restructured
\end{itemize}

%%%%%%%%%%%%%%%%%%%%%%%%%%%%%%%%%%%%%%%%
\paragraph{v1.6:} 2018/01/17

\begin{itemize}
\item
application for development of include files
\item
corrections to manual
\end{itemize}

%%%%%%%%%%%%%%%%%%%%%%%%%%%%%%%%%%%%%%%%
\paragraph{v1.5:} 2017/05/21

\begin{itemize}
\item
more complete structuring introduced
\item
|\childdocof| introduced
\item
|\childdoc| renamed to |\childdocmain|
\item
|\childredirect| renamed to |\childdocforward| and |\childdocforwardprefix|
and functionality expanded
\end{itemize}

%%%%%%%%%%%%%%%%%%%%%%%%%%%%%%%%%%%%%%%%
\paragraph{v1.0:} 2017/04/27

\begin{itemize}
\item
manual and install package
\item
first version published on CTAN
\end{itemize}

%%%%%%%%%%%%%%%%%%%%%%%%%%%%%%%%%%%%%%%%
\paragraph{v0.6:} 2017/04/26

\begin{itemize}
\item
redirection mechanism added
\end{itemize}

%%%%%%%%%%%%%%%%%%%%%%%%%%%%%%%%%%%%%%%%
\paragraph{v0.5:} 2017/04/26

\begin{itemize}
\item
functionality in definition file
\end{itemize}


%%%%%%%%%%%%%%%%%%%%%%%%%%%%%%%%%%%%%%%%%%%%%%%%%%%%%%%%%%%%%%%%%%%%%%%%%%%%%%%%
%%%%%%%%%%%%%%%%%%%%%%%%%%%%%%%%%%%%%%%%%%%%%%%%%%%%%%%%%%%%%%%%%%%%%%%%%%%%%%%%
%%%%%%%%%%%%%%%%%%%%%%%%%%%%%%%%%%%%%%%%%%%%%%%%%%%%%%%%%%%%%%%%%%%%%%%%%%%%%%%%
\appendix

\settowidth\MacroIndent{\rmfamily\scriptsize 000\ }

 \DocInput{childdoc.dtx}

\end{document}
%</driver>
% \fi
%
% %%%%%%%%%%%%%%%%%%%%%%%%%%%%%%%%%%%%%%%%%%%%%%%%%%%%%%%%%%%%%%%%%%%%%%%%%%%%%%
% %%%%%%%%%%%%%%%%%%%%%%%%%%%%%%%%%%%%%%%%%%%%%%%%%%%%%%%%%%%%%%%%%%%%%%%%%%%%%%
% \section{Sample}
%\iffalse
%<*samplemain>
%\fi
%
% The following presents a sample document
% with two chapters, two parts, a title page,
% a compile flag as well as three forwarding files to set the flag.
% It consists of eight |.tex| files:
% \begin{center}
% \begin{tabular}{ll}
% |cdocsamp.tex|&main file\\
% |cdocsch1.tex|&include file for chapter 1\\
% |cdocsch2.tex|&include file for chapter 2\\
% |cdocspt3.tex|&include file for part 3\\
% |cdocspt4.tex|&include file for part 4\\
% |cdocsdrf.tex|&forwarding file for main file in draft mode\\
% |cdocsfi1.tex|&forwarding file for final version of chapter 1\\
% |cdocsfi2.tex|&forwarding file for final version of chapter 2\\
% \end{tabular}
% \end{center}
% Each of the eight files can be compiled directly by the \LaTeX{} compiler.
%
% %%%%%%%%%%%%%%%%%%%%%%%%%%%%%%%%%%%%%%
% \paragraph{Main File.}
%
% The main file is called |cdocsamp.tex|.
%
% Load the \textsf{childdoc} definitions and
% declare the filename for the main document:
%    \begin{macrocode}
% \iffalse
%
% childdoc.dtx Copyright (C) 2017-2018 Niklas Beisert
%
% This work may be distributed and/or modified under the
% conditions of the LaTeX Project Public License, either version 1.3
% of this license or (at your option) any later version.
% The latest version of this license is in
%   http://www.latex-project.org/lppl.txt
% and version 1.3 or later is part of all distributions of LaTeX
% version 2005/12/01 or later.
%
% This work has the LPPL maintenance status `maintained'.
%
% The Current Maintainer of this work is Niklas Beisert.
%
% This work consists of the files childdoc.dtx and childdoc.ins
% and the derived files childdoc.def and cdocsamp.tex with
% cdocsch1.tex, cdocsch2.tex, cdocsdrf.tex, cdocsfn1.tex, cdocsfn2.tex.
%
%<package>\ifdefined\childdocmain\endinput\fi
%<package>\ProvidesFile{childdoc.def}[2018/12/30 v2.0 child document driver]
%<samplemain>\ProvidesFile{cdocsamp.tex}[2018/12/30 v2.0 sample for childdoc]
%<*driver>
%\ProvidesFile{childdoc.drv}[2018/12/30 v2.0 childdoc reference manual file]
\PassOptionsToClass{10pt,a4paper}{article}
\documentclass{ltxdoc}

\usepackage[margin=35mm]{geometry}
\usepackage{hyperref}
\usepackage{hyperxmp}
\usepackage[usenames]{color}

\hypersetup{colorlinks=true}
\hypersetup{pdfstartview=FitH}
\hypersetup{pdfpagemode=UseNone}
\hypersetup{pdfsource={}}
\hypersetup{pdflang={en-UK}}
\hypersetup{pdfcopyright={Copyright 2017-2018 Niklas Beisert.
  This work may be distributed and/or modified under the
  conditions of the LaTeX Project Public License, either version 1.3
  of this license or (at your option) any later version.}}
\hypersetup{pdflicenseurl={http://www.latex-project.org/lppl.txt}}
\hypersetup{pdfcontactaddress={ETH Zurich, ITP, HIT K,
  Wolfgang-Pauli-Strasse 27}}
\hypersetup{pdfcontactpostcode={8093}}
\hypersetup{pdfcontactcity={Zurich}}
\hypersetup{pdfcontactcountry={Switzerland}}
\hypersetup{pdfcontactemail={nbeisert@itp.phys.ethz.ch}}
\hypersetup{pdfcontacturl={http://people.phys.ethz.ch/\xmptilde nbeisert/}}

\newcommand{\secref}[1]{\hyperref[#1]{section \ref*{#1}}}

\parskip1ex
\parindent0pt
\let\olditemize\itemize
\def\itemize{\olditemize\parskip0pt}

\begin{document}

\title{The \textsf{childdoc} Package}
\hypersetup{pdftitle={The childdoc Package}}
\author{Niklas Beisert\\[2ex]
  Institut f\"ur Theoretische Physik\\
  Eidgen\"ossische Technische Hochschule Z\"urich\\
  Wolfgang-Pauli-Strasse 27, 8093 Z\"urich, Switzerland\\[1ex]
  \href{mailto:nbeisert@itp.phys.ethz.ch}
  {\texttt{nbeisert@itp.phys.ethz.ch}}}
\hypersetup{pdfauthor={Niklas Beisert}}
\hypersetup{pdfsubject={Manual for the LaTeX2e Package childdoc}}
\date{30 December 2018, \textsf{v2.0}}
\maketitle

\begin{abstract}\noindent
\textsf{childdoc} is a \LaTeXe{} package
that enables the direct compilation
of document sections included by |\include|
to individual files.
\end{abstract}

\begingroup
\parskip0ex
\tableofcontents
\endgroup

%%%%%%%%%%%%%%%%%%%%%%%%%%%%%%%%%%%%%%%%%%%%%%%%%%%%%%%%%%%%%%%%%%%%%%%%%%%%%%%%
%%%%%%%%%%%%%%%%%%%%%%%%%%%%%%%%%%%%%%%%%%%%%%%%%%%%%%%%%%%%%%%%%%%%%%%%%%%%%%%%
\section{Introduction}

\LaTeX{} provides a mechanism to structure a large document (such as a book)
into a main file and several child files (containing the chapters)
using the |\include| command.
This mechanism is beneficial for documents
which span hundreds of pages in order to
make the source file(s) more manageable.
Moreover, compilation can be restricted to
selected child files by means of the |\includeonly| command.
The latter feature can be used to reduce the compilation time while editing
(this was significantly more useful in the earlier days of \LaTeX{})
or to generate a smaller document which is easier to navigate.
Another application of |\includeonly| is to generate
documents consisting of selected parts of the complete document.

However, there are a few drawbacks of the plain |\include| mechanism:
\begin{itemize}
\item
The child files cannot be compiled on their own,
they can only be compiled via the main file.
A naive editing environment
(such as a text editor with an option
to have the current file processed by \LaTeX)
may require one to switch to the main file before compiling;
attempting to compile the child file produces errors.
\item
The main file must be modified (each time)
to adjust the |\includeonly| command
to the present needs. This easily leaves the main file in a messy state.
\item
The generated document will always carry the filename
of the main document. This is inconvenient if
several child files are to be compiled and
to be kept for distribution.
\end{itemize}

The present package provides a simple interface
to make child files individually compilable by \LaTeX{}.
Compiling a child file then has the same effect as compiling
the main file with an |\includeonly| command
to select the appropriate child.
Moreover the generated document will carry the name of the child
rather than the main file.
This resolves all three above issues.

This feature is meant to make the editing of books,
thesis documents and lecture notes somewhat more convenient.
However, the package can also be used efficiently for
composing a series of documents (such as exercise sheets)
which are typically distributed individually.
It then assists the author in generating the individual documents
(potentially in different versions)
as well as a document containing the collected series.
Another application is in developing style files
or other kinds of included material
where compilation of the style file could redirect
to a sample or test file.

%%%%%%%%%%%%%%%%%%%%%%%%%%%%%%%%%%%%%%%%%%%%%%%%%%%%%%%%%%%%%%%%%%%%%%%%%%%%%%%%
%%%%%%%%%%%%%%%%%%%%%%%%%%%%%%%%%%%%%%%%%%%%%%%%%%%%%%%%%%%%%%%%%%%%%%%%%%%%%%%%
\section{Usage}

First of all, the package \textsf{childdoc} is \emph{not} a standard
\LaTeXe{} |.sty| style file! Therefore it needs to be invoked in
a non-standard way.

%%%%%%%%%%%%%%%%%%%%%%%%%%%%%%%%%%%%%%%%%%%%%%%%%%%%%%%%%%%%%%%%%%%%%%%%%%%%%%%%
\subsection{Included Files}
\label{sec:include}

%%%%%%%%%%%%%%%%%%%%%%%%%%%%%%%%%%%%%%%%
\DescribeMacro{\childdocmain}
To use the package, add the commands
\begin{center}
\begin{tabular}{l}
|\input{childdoc.def}|\\
|\childdocmain{}|\\
\end{tabular}
\end{center}
at the very top of the main \LaTeX{} file,
in particular \emph{before} the |\documentclass| statement!
The argument of |\childdocmain| should be left empty
(but it must be present).

%%%%%%%%%%%%%%%%%%%%%%%%%%%%%%%%%%%%%%%%
\DescribeMacro{\childdocof}
Furthermore, add the commands
\begin{center}
\begin{tabular}{l}
|\input{childdoc.def}|\\
|\childdocof{|\textit{main}|}|\\
\end{tabular}
\end{center}
at the top of every child file \textit{child}
which is included by |\include{|\textit{child}|}|
from within the main file
(or at least for those files to be compiled individually).
The argument \textit{main} must be the filename of the main file.

There are a couple of
considerations in setting up the main and child documents:

%%%%%%%%%%%%%%%%%%%%%%%%%%%%%%%%%%%%%%%%
\paragraph{Restrictions.}

Please note the following restrictions:
\begin{itemize}
\item
|\childdocmain| must be called with one argument \textit{main}
to ensure compatibility with earlier version of the package.
It must either be empty (|\childdocmain{}|)
or precisely match the filename of the main file in which it is specified.
See \secref{sec:detection} for further information.
\item
The filename \textit{main} must be specified without the |.tex| extension.
\item
The filename \textit{main} is case sensitive
(even in case-insensitive file systems)
due to internal string comparison.
\item
The argument \textit{main} should be fully expanded, it cannot be a macro.
\item
Subdirectories and special characters should be avoided in filenames.
\item
The command |\childdocmain{|\textit{main}|}| must be followed by a whitespace.
It should not be followed immediately by another command
or by a comment mark `|%|'.
This is because the \TeX{} parser reads the token immediately following
the argument of |\childdocmain| and puts it
at the beginning of every child section;
however, a white\-space is ignored.
\end{itemize}

%%%%%%%%%%%%%%%%%%%%%%%%%%%%%%%%%%%%%%%%
\paragraph{Content of Main File.}

It is advisable to place all content in the child files included by |\include|.
Any output contained in the main file will appear in all child documents
unless suppressed manually;
it cannot be suppressed automatically by the |\includeonly| directive
and thus should normally be avoided.
A method to include some content in the main file
by means of conditional processing is described in \secref{sec:conditional}.

%%%%%%%%%%%%%%%%%%%%%%%%%%%%%%%%%%%%%%%%
\paragraph{Page Numbering.}

When only a part of the document is compiled,
the appropriate numbering of pages
(as well as other status parameters)
is determined from the |.aux| files.
The latter contain information from previous passes.
However this information needs to propagate through
all intermediate child documents.
Therefore the page numbering in child documents may well
be inconsistent until the complete document is compiled at least once.

A useful (if unconventional) way to always ensure a consistent
page numbering is to restart the numbering in each child document
and denote the pages by `\textit{child}|.|\textit{page}'
where \textit{child} represents the chapter/section number of the child file.
This can be achieved by the command
|\numberwithin{page}{|\textit{child}|}|
of the \textsf{amsmath} package
where \textit{child} can be |chapter| or |section|
depending on the chosen structuring.
Alternatively, one can modify the macro |\thepage| appropriately
and reset the counter |page| at the start of each child file.

%%%%%%%%%%%%%%%%%%%%%%%%%%%%%%%%%%%%%%%%%%%%%%%%%%%%%%%%%%%%%%%%%%%%%%%%%%%%%%%%
\subsection{Conditional Processing}
\label{sec:conditional}

The package provides a mechanism to compile different versions
of a document. To customise the versions further some conditional processing
can come in handy to distinguish which version is being compiled.
The package provides two macros to describe the compilation context:

%%%%%%%%%%%%%%%%%%%%%%%%%%%%%%%%%%%%%%%%
\DescribeMacro{\ifchilddoc}
The conditional |\ifchilddoc| distinguishes between the compilation of
child documents and the main document:
%
\begin{center}
|\ifchilddoc |\textit{child-code}| |[|\||else |\textit{main-code}]| \||fi|
\end{center}

%%%%%%%%%%%%%%%%%%%%%%%%%%%%%%%%%%%%%%%%
\DescribeMacro{\childdocname}
\DescribeMacro{\childdocjob}
The macro |\childdocname| contains the filename (without extension)
of the main or child file being processed.
Note that |\childdocjob| will always contain the name of the main file.

%%%%%%%%%%%%%%%%%%%%%%%%%%%%%%%%%%%%%%%%
\paragraph{Title Page.}

Conditional processing can be used to include a title or banner page
in the main document when proper precautions are taken.
Importantly, the code in the main file should ensure that the page counter
(as well as other status parameters which are stored in the |.aux| files)
takes the same value after the conditional processing.
Otherwise the page numbers may take divergent values
depending on which part is compiled.

For example, a title page could be declared by:
%
\begin{center}
\begin{tabular}{l}
|\ifchilddoc\||else|\\
|\addtocounter{page}{-1}|\\
\textit{code for title page}\\
|\newpage|\\
|\||fi|
\end{tabular}
\end{center}
%
A banner page for the child documents can be generated by:
%
\begin{center}
\begin{tabular}{l}
|\ifchilddoc|\\
|\addtocounter{page}{-1}|\\
\textit{code for banner page}\\
|\newpage|\\
|\||fi|
\end{tabular}
\end{center}
%
Here one could write a message such as:
\begin{center}
|This is the part \childdocname{} of \childdocjob{}.|
\end{center}

%%%%%%%%%%%%%%%%%%%%%%%%%%%%%%%%%%%%%%%%%%%%%%%%%%%%%%%%%%%%%%%%%%%%%%%%%%%%%%%%
\subsection{Flags}
\label{sec:flags}

The package makes it easy to generate different versions
of the main or child documents.
To this end compilation flags can be defined
and assigned different default values.
They will be particularly useful in conjunction
with the forwarding mechanism described in \secref{sec:forward}.

For example, it may be useful to have a flag |\version|
which can be set to |draft| or |final|.
The document source will contain some conditional code
depending on the value of |\version|.
Suppose further, the flag should default to |final| for the main file
and to |draft| for child files
which is a natural assignment for editing the document.
This is achieved by placing the following code
in the preamble of the main document
(below the |\childdocmain| directive):
%
\begin{center}
\begin{tabular}{l}
|\ifchilddoc|\\
|\providecommand{\version}{draft}|\\
|\||else|\\
|\providecommand{\version}{final}|\\
|\||fi|
\end{tabular}
\end{center}
%
The definition by |\providecommand| makes sure
that previous definitions are not overwritten.
Further statements |\providecommand{\version}{...}|
can thus be added before the above code to override it.

For the main file, one might add a line
(between |\childdocmain| and the above block)
%
\begin{center}
|%\ifchilddoc\||else\providecommand{\version}{draft}\||fi|
\end{center}
%
which can be uncommented to produce a draft version.
Likewise one can add a line to the very top of a child file
(above the |\childdocof{|\textit{main}|}| directive)
%
\begin{center}
|%\providecommand{\version}{final}|
\end{center}
%
which can be uncommented to produce the final version of this child document.

%%%%%%%%%%%%%%%%%%%%%%%%%%%%%%%%%%%%%%%%%%%%%%%%%%%%%%%%%%%%%%%%%%%%%%%%%%%%%%%%
\subsection{Forwarding}
\label{sec:forward}

Different versions of the main or child documents
using compilation flags as described in \secref{sec:flags}
can be (permanently) stored in different files
for convenient compilation, viewing and distribution.
To this end, the package defines a command
to pass on compilation to a different file:

%%%%%%%%%%%%%%%%%%%%%%%%%%%%%%%%%%%%%%%%
\DescribeMacro{\childdocforward}
The command |\childdocforward| redirects processing to
another source file:
%
\begin{center}
\begin{tabular}{l}
|\input{childdoc.def}|\\
|\childdocforward[|\textit{main}|]{|\textit{dest}|}|\\
\end{tabular}
\end{center}
%
The argument \textit{dest} is the destination file
(without extension).
It should be the main file or one of the child files.
Note that further \textsf{childdoc} directives
such as |\childdocof| and |\childdocforward|
in the indicated file will be processed in this form.
The optional argument \textit{main}
passes on directly to the main file \textit{main}
while pretending to compile the child \textit{dest}.
This form behaves as if \textit{dest}
issues |\childdocof{|\textit{main}|}| right away,
and no further \textsf{childdoc} directives will be processed.

%%%%%%%%%%%%%%%%%%%%%%%%%%%%%%%%%%%%%%%%
\DescribeMacro{\...prefix}
In the alternative form |\childdocforwardprefix|,
%
\begin{center}
\begin{tabular}{l}
|\input{childdoc.def}|\\
|\childdocforwardprefix[|\textit{main}|]{|\textit{prefix}|}{|\textit{dest}|}|
\end{tabular}
\end{center}
%
the destination file is determined by a pattern
depending on the current file:
To make this work, the current file must be called
`{\textit{prefix}\hspace{0.2em}\textit{suffix}}'
with \textit{prefix} matching precisely the argument.
Processing is then passed on to the file
`{\textit{dest}\hspace{0.2em}\textit{suffix}}'.
Surely, the same effect is achieved by
directly specifying the
argument `{\textit{dest}\hspace{0.2em}\textit{suffix}}'
in the first form.
However, that requires to set up a different file
for each child. With the alternative form of the command
all these files can have exactly the same content
which simplifies setting them up and maintaining them.

For example, the following file |draft.tex|
with a compilation flag |\version| as described in \secref{sec:flags}
compiles the main document as a draft:
%
\begin{center}
\begin{tabular}{l}
|\def\version{draft}|\\
|\input{childdoc.def}|\\
|\childdocforward{|\textit{main}|}|
\end{tabular}
\end{center}
%
Likewise, the following files |final|\textit{nn}|.tex|
compile the final version of the child document
|child|\textit{nn}|.tex|:
%
\begin{center}
\begin{tabular}{l}
|\def\version{final}|\\
|\input{childdoc.def}|\\
|\childdocforwardprefix{final}{child}|
\end{tabular}
\end{center}
%

Note that when several versions of a main file and/or of each child file
are to be generated, it may be convenient to set up a |Makefile| or
shell script to automatise the process.

%%%%%%%%%%%%%%%%%%%%%%%%%%%%%%%%%%%%%%%%%%%%%%%%%%%%%%%%%%%%%%%%%%%%%%%%%%%%%%%%
\subsection{Command Line Processing}
\label{sec:commandline}

The effect of redirection files can also be achieved by invoking
the \LaTeX{} compiler with a more elaborate command line.
Most conveniently this should be done as part
of a shell script or a |Makefile|.

When using \textsf{childdoc} in the main file, the following
command lines effectively perform a redirection
(note that depending on the shell being used,
backslashes may have to be doubled: `|\|' $\to$ `|\\|'):
%
\begin{center}
|... -jobname "|\textit{target}|" |\\|"|[\textit{flags}]%
|\input{childdoc.def}\childdocforward[|\textit{main}|]{|\textit{dest}|}"|
\end{center}
%
Here \textit{target} is the name of the output file,
\textit{main} is the name of the main file
and \textit{dest} is the name of the main or child file to be processed
(all filenames without extensions).
The optional argument \textit{main} can be omitted
if \textit{main} matches \textit{dest}.
Optionally, compilation \textit{flags} can be defined via |\def| commands.
This command line makes the \TeX{} engine believe
it is compiling the file \textit{target}
whose content is specified as the latter parameter.
The provided code then forwards the processing to
\textit{main} or \textit{dest} as described in \secref{sec:forward}.

%%%%%%%%%%%%%%%%%%%%%%%%%%%%%%%%%%%%%%%%%%%%%%%%%%%%%%%%%%%%%%%%%%%%%%%%%%%%%%%%
\subsection{Include by Input}
\label{sec:input}

Including child documents by |\include| has some restrictions by design.
Most notably, the content of a child document always occupies
its own set of pages; pages cannot be shared between child documents.
Usually, this behaviour makes perfect sense
because each child document contain an essential part of the document.
However, in some situations it may be desirable to compose
a document from a collection of parts
without having mandatory page breaks between then.
For this case, the package
provides a mechanism to include parts
by |\input| which can also be processed individually.
However, by construction this mechanism
requires manual handling of the content to be output.

%%%%%%%%%%%%%%%%%%%%%%%%%%%%%%%%%%%%%%%%
\DescribeMacro{\ifchilddocmanual}
The main file should be prepared as usual, see \secref{sec:include}.
However, the document body must make a distinction
between processing of an individual part and of the main document, e.g.:
%
\begin{center}
\begin{tabular}{l}
|\ifchilddocmanual|\\
|\input{\childdocname}|\\
|\||else|\\
\textit{document body with }|\input{|\textit{part}|}|\\
|\||fi|
\end{tabular}
\end{center}
%
The conditional |\ifchilddocmanual| is true whenever
a part to be included by |\input| is being compiled,
and the name of the part is stored in |\childdocname|.

%%%%%%%%%%%%%%%%%%%%%%%%%%%%%%%%%%%%%%%%
\DescribeMacro{\childdocby}
Each part to be included by |\input| should start with:
%
\begin{center}
\begin{tabular}{l}
|\input{childdoc.def}|\\
|\childdocby{|\textit{main}|}|\\
\end{tabular}
\end{center}
%
The directive |\childdocby| is similar to |\childdocof|
described in \secref{sec:include},
but the subsequent selection of content must be done manually.
To that end, both |\ifchilddoc| and |\ifchilddocmanual|
will be true upon processing of a part,
and the name of the part is stored in |\childdocname|.
Note that |\jobname| will be set to the filename of the current part
so that each part receives an individual |.aux| file
that does not interfere with the |.aux| file(s) of the main document.
This behaviour can be altered by the alternative form
|\childdocby[*]{|\textit{main}|}| (with a non-empty optional argument)
which uses the |.aux| file of the main document
by setting |\jobname| to \textit{main}.

%%%%%%%%%%%%%%%%%%%%%%%%%%%%%%%%%%%%%%%%%%%%%%%%%%%%%%%%%%%%%%%%%%%%%%%%%%%%%%%%
\subsection{Driver Development}
\label{sec:driver}

The \textsf{childdoc} mechanism can also be use for the development
of definition files such as \LaTeX{} styles or classes.
This case differs from the above setup with multiple parts
included by |\include| in that no |\includeonly| should be invoked.
This can be achieved by starting the include file
(before |\ProvidesPackage|) with:
%
\begin{center}
\begin{tabular}{l}
|\input{childdoc.def}|\\
|\childdocforward{|\textit{main}|}|\\
\end{tabular}
\end{center}
%
or alternatively with:
%
\begin{center}
\begin{tabular}{l}
|\input{childdoc.def}|\\
|\childdocby{|\textit{main}|}|\\
\end{tabular}
\end{center}
%
Both forms have slightly different effects as described above.
The main file is prepared as usual, see \secref{sec:include}.

%%%%%%%%%%%%%%%%%%%%%%%%%%%%%%%%%%%%%%%%%%%%%%%%%%%%%%%%%%%%%%%%%%%%%%%%%%%%%%%%
\subsection{Legacy Detection}
\label{sec:detection}

The directive |\childdocmain| in the main file can detect
whether the complete document or merely a child is to be compiled
even without using the directive |\childdocof|.
This method is deprecated because it is less robust
and there is no compelling reason to use it;
it is merely provided for backward compatibility
and it may be removed in future versions.

If the detection mechanism is to be used,
it is mandatory to correctly specify
the filename of the main file as the argument of |\childdocmain|:
%
\begin{center}
\begin{tabular}{l}
|\input{childdoc.def}|\\
|\childdocmain{|\textit{main}|}|\\
\end{tabular}
\end{center}
%
If |\jobname| does not match the argument \textit{main} of |\childdocmain|,
it is assumed that |\jobname| points to the child file to be compiled.
When using |\childdocmain| with the main file specified as argument,
it suffices to start a child file
with just |\input{|\textit{main}|}|
without loading of the package and using |\childdocof|.
If instead all processing is done
with the appropriate \textsf{childdoc} directives,
the argument of \textit{main} of |\childdocmain| can be empty.

An alternative version of the command line processing described
in \secref{sec:commandline} using the detection mechanism reads:
%
\begin{center}
|... -jobname "|\textit{target}|" "|[\textit{flags}]%
[|\def\jobname{|\textit{dest}|}|]|\input{|\textit{main}|}"|
\end{center}

%%%%%%%%%%%%%%%%%%%%%%%%%%%%%%%%%%%%%%%%%%%%%%%%%%%%%%%%%%%%%%%%%%%%%%%%%%%%%%%%
\subsection{Manual Code}
\label{sec:manual}

In case one cannot be certain whether the definitions file |childdoc.def|
is installed on the target \TeX{} distribution
and one prefers not to ship it,
it is conceivable to paste a few relevant commands into the sources.

To that end, drop all statements |\input{childdoc.def}|
and perform the replacements as outlined below.
Instead of |\childdocmain{|\textit{main}|}| add the following code
to the top of the main file:
%
\begin{center}
\begin{tabular}{l}
|\||ifdefined\childdocname\endinput\||fi\newif\ifchilddoc|\\
|\edef\childdocname{\scantokens\expandafter{\jobname\noexpand}}|\\
|\def\childdocmain{|\textit{main}|}\||ifx\childdocmain\childdocname\||else|\\
|\childdoctrue\includeonly{\childdocname}\let\jobname\childdocmain\||fi|\\
\end{tabular}
\end{center}
%
Instead of |\childdocof{|\textit{main}|}| just include the main file
at the top of each child file:
%
\begin{center}
|\input{|\textit{main}|}|
\end{center}
%
A simple redirection |\childdocforward{|\textit{dest}|}| is achieved by:
%
\begin{center}
|\def\jobname{|\textit{dest}|}\input{\jobname}|
\end{center}
%
The redirection with prefix
|\childdocforwardprefix[|\textit{prefix}|]{|\textit{dest}|}|
is accomplished by:
%
\begin{center}
\begin{tabular}{l}
|{\edef\jobname{\scantokens\expandafter{\jobname\noexpand}}|\\
|\def\redirectjob |\textit{prefix}|#1~~~{\gdef\jobname{|\textit{dest}|#1}}|\\
|\expandafter\redirectjob\jobname~~~}\input{\jobname}|
\end{tabular}
\end{center}

In an alternative approach,
child documents can be compiled by a specific command line
without additional code or specific definitions:
%
\begin{center}
|... -jobname "|\textit{target}|" "|[\textit{flags}]%
|\includeonly{|\textit{dest}|}\input{|\textit{main}|}"|
\end{center}
%

%%%%%%%%%%%%%%%%%%%%%%%%%%%%%%%%%%%%%%%%%%%%%%%%%%%%%%%%%%%%%%%%%%%%%%%%%%%%%%%%
%%%%%%%%%%%%%%%%%%%%%%%%%%%%%%%%%%%%%%%%%%%%%%%%%%%%%%%%%%%%%%%%%%%%%%%%%%%%%%%%
\section{Information}

%%%%%%%%%%%%%%%%%%%%%%%%%%%%%%%%%%%%%%%%%%%%%%%%%%%%%%%%%%%%%%%%%%%%%%%%%%%%%%%%
\subsection{Copyright}

Copyright \copyright{} 2017--2018 Niklas Beisert

This work may be distributed and/or modified under the
conditions of the \LaTeX{} Project Public License, either version 1.3
of this license or (at your option) any later version.
The latest version of this license is in
  \url{http://www.latex-project.org/lppl.txt}
and version 1.3 or later is part of all distributions of \LaTeX{}
version 2005/12/01 or later.

This work has the LPPL maintenance status `maintained'.

The Current Maintainer of this work is Niklas Beisert.

This work consists of the files |README.txt|, |childdoc.ins| and |childdoc.dtx|
as well as the derived files |childdoc.def|, |cdocsamp.tex|
with |cdocsch1.tex|, |cdocsch2.tex|, |cdocspt3.tex|, |cdocspt4.tex|,
|cdocsdrf.tex|, |cdocsfn1.tex|, |cdocsfn2.tex|
as well as |childdoc.pdf|.

%%%%%%%%%%%%%%%%%%%%%%%%%%%%%%%%%%%%%%%%%%%%%%%%%%%%%%%%%%%%%%%%%%%%%%%%%%%%%%%%
\subsection{Files and Installation}

The package consists of the files:
%
\begin{center}
\begin{tabular}{ll}
    |README.txt|   & readme file \\
    |childdoc.ins| & installation file \\
    |childdoc.dtx| & source file \\
    |childdoc.def| & definition file \\
    |cdocsamp.tex| & sample main file \\
    |cdocsch1.tex| & sample include file \\
    |cdocsch2.tex| & sample include file \\
    |cdocspt3.tex| & sample part file \\
    |cdocspt4.tex| & sample part file \\
    |cdocsdrf.tex| & sample redirection file \\
    |cdocsfn1.tex| & sample redirection file \\
    |cdocsfn2.tex| & sample redirection file \\
    |childdoc.pdf| & manual
\end{tabular}
\end{center}
%
The distribution consists of the files
|README.txt|, |childdoc.ins| and |childdoc.dtx|.
%
\begin{itemize}
\item
Run (pdf)\LaTeX{} on |childdoc.dtx|
to compile the manual |childdoc.pdf| (this file).
\item
Run \LaTeX{} on |childdoc.ins| to create the definitions file |childdoc.def|
and the sample |cdocsamp.tex| with include files
|cdocsch1.tex|, |cdocsch2.tex|, |cdocspt3.tex|, |cdocspt4.tex|,
|cdocsdrf.tex|, |cdocsfn1.tex|, |cdocsfn2.tex|.
Then copy the file |childdoc.def| to an appropriate directory of your \LaTeX{}
distribution, e.g.\ \textit{texmf-root}|/tex/latex/childdoc|.
\end{itemize}

%%%%%%%%%%%%%%%%%%%%%%%%%%%%%%%%%%%%%%%%%%%%%%%%%%%%%%%%%%%%%%%%%%%%%%%%%%%%%%%%
\subsection{Related CTAN Packages}

There are several other packages which offer a similar functionality:
%
\begin{itemize}
\item
The packages
\href{http://ctan.org/pkg/docmute}{\textsf{docmute}},
\href{http://ctan.org/pkg/includex}{\textsf{includex}} and
\href{http://ctan.org/pkg/standalone}{\textsf{standalone}}
provide commands to include only the document body of
a child file thus allowing both files to be compiled individually.
\item
The packages \href{http://ctan.org/pkg/subdocs}{\textsf{subdocs}}
and \href{http://ctan.org/pkg/subfiles}{\textsf{subfiles}}
provide structures in which the main and child documents can be
encapsulated and allowing them to be compiled individually.
The inclusion mechanism is different from the conventional |\include|.
\item
The package \href{http://ctan.org/pkg/combine}{\textsf{combine}}
is an elaborate solution to combine several documents into one.
\end{itemize}
%
See also the CTAN topic \href{http://ctan.org/topic/subdocs}{\textsf{subdocs}}
for further related packages.
The present package differs from the above solutions in that
a document structure constructed with the conventional |\include| mechanism
just needs two extra commands at the top of every file
such that all constituent files can be compiled individually.

%%%%%%%%%%%%%%%%%%%%%%%%%%%%%%%%%%%%%%%%%%%%%%%%%%%%%%%%%%%%%%%%%%%%%%%%%%%%%%%%
%\subsection{Feature Suggestions}
%
%The following is a list of features which may be useful for future
%versions of this package:
%%
%\begin{itemize}
%\item
%\ldots
%\end{itemize}

%%%%%%%%%%%%%%%%%%%%%%%%%%%%%%%%%%%%%%%%%%%%%%%%%%%%%%%%%%%%%%%%%%%%%%%%%%%%%%%%
\subsection{Revision History}

%%%%%%%%%%%%%%%%%%%%%%%%%%%%%%%%%%%%%%%%
\paragraph{v2.0:} 2018/12/30

\begin{itemize}
\item
immediate forward processing
\item
added |\childdocby| mechanism
\item
manual restructured
\end{itemize}

%%%%%%%%%%%%%%%%%%%%%%%%%%%%%%%%%%%%%%%%
\paragraph{v1.6:} 2018/01/17

\begin{itemize}
\item
application for development of include files
\item
corrections to manual
\end{itemize}

%%%%%%%%%%%%%%%%%%%%%%%%%%%%%%%%%%%%%%%%
\paragraph{v1.5:} 2017/05/21

\begin{itemize}
\item
more complete structuring introduced
\item
|\childdocof| introduced
\item
|\childdoc| renamed to |\childdocmain|
\item
|\childredirect| renamed to |\childdocforward| and |\childdocforwardprefix|
and functionality expanded
\end{itemize}

%%%%%%%%%%%%%%%%%%%%%%%%%%%%%%%%%%%%%%%%
\paragraph{v1.0:} 2017/04/27

\begin{itemize}
\item
manual and install package
\item
first version published on CTAN
\end{itemize}

%%%%%%%%%%%%%%%%%%%%%%%%%%%%%%%%%%%%%%%%
\paragraph{v0.6:} 2017/04/26

\begin{itemize}
\item
redirection mechanism added
\end{itemize}

%%%%%%%%%%%%%%%%%%%%%%%%%%%%%%%%%%%%%%%%
\paragraph{v0.5:} 2017/04/26

\begin{itemize}
\item
functionality in definition file
\end{itemize}


%%%%%%%%%%%%%%%%%%%%%%%%%%%%%%%%%%%%%%%%%%%%%%%%%%%%%%%%%%%%%%%%%%%%%%%%%%%%%%%%
%%%%%%%%%%%%%%%%%%%%%%%%%%%%%%%%%%%%%%%%%%%%%%%%%%%%%%%%%%%%%%%%%%%%%%%%%%%%%%%%
%%%%%%%%%%%%%%%%%%%%%%%%%%%%%%%%%%%%%%%%%%%%%%%%%%%%%%%%%%%%%%%%%%%%%%%%%%%%%%%%
\appendix

\settowidth\MacroIndent{\rmfamily\scriptsize 000\ }

 \DocInput{childdoc.dtx}

\end{document}
%</driver>
% \fi
%
% %%%%%%%%%%%%%%%%%%%%%%%%%%%%%%%%%%%%%%%%%%%%%%%%%%%%%%%%%%%%%%%%%%%%%%%%%%%%%%
% %%%%%%%%%%%%%%%%%%%%%%%%%%%%%%%%%%%%%%%%%%%%%%%%%%%%%%%%%%%%%%%%%%%%%%%%%%%%%%
% \section{Sample}
%\iffalse
%<*samplemain>
%\fi
%
% The following presents a sample document
% with two chapters, two parts, a title page,
% a compile flag as well as three forwarding files to set the flag.
% It consists of eight |.tex| files:
% \begin{center}
% \begin{tabular}{ll}
% |cdocsamp.tex|&main file\\
% |cdocsch1.tex|&include file for chapter 1\\
% |cdocsch2.tex|&include file for chapter 2\\
% |cdocspt3.tex|&include file for part 3\\
% |cdocspt4.tex|&include file for part 4\\
% |cdocsdrf.tex|&forwarding file for main file in draft mode\\
% |cdocsfi1.tex|&forwarding file for final version of chapter 1\\
% |cdocsfi2.tex|&forwarding file for final version of chapter 2\\
% \end{tabular}
% \end{center}
% Each of the eight files can be compiled directly by the \LaTeX{} compiler.
%
% %%%%%%%%%%%%%%%%%%%%%%%%%%%%%%%%%%%%%%
% \paragraph{Main File.}
%
% The main file is called |cdocsamp.tex|.
%
% Load the \textsf{childdoc} definitions and
% declare the filename for the main document:
%    \begin{macrocode}
\input{childdoc.def}
\childdocmain{}
%    \end{macrocode}

% Optional override for |\version| flag:
%    \begin{macrocode}
%%\ifchilddoc\else\providecommand{\version}{draft}\fi
%    \end{macrocode}

% Define the default values for the |\version| flag
% (|final| for the main file and |draft| for childs):
%    \begin{macrocode}
\ifchilddoc
\providecommand{\version}{draft}
\else
\providecommand{\version}{final}
\fi
%    \end{macrocode}

% Load the standard document class:
%    \begin{macrocode}
\documentclass[12pt]{article}
%    \end{macrocode}

% Start the document body:
%    \begin{macrocode}
\begin{document}
%    \end{macrocode}

% Declare a title page.
% Print title, part of document being processed and version flag:
%    \begin{macrocode}
\addtocounter{page}{-1}
\begin{center}
{\LARGE\bfseries{}childdoc example\par}
\vspace{1cm}
\ifchilddoc
\ifchilddocmanual part\else chapter\fi:
`\childdocname' of `\childdocjob'\par
\else
main document: `\childdocjob'\par
\fi
version: \version\par
\end{center}
\newpage
%    \end{macrocode}

% Manually include selected file,
% otherwise process as usual:
%    \begin{macrocode}
\ifchilddocmanual
\section*{part `\childdocname'}
\input{\childdocname}
\else
%    \end{macrocode}

% Include the two chapters:
%    \begin{macrocode}
\include{cdocsch1}
\include{cdocsch2}
%    \end{macrocode}

% Include the two parts unless only chapters should be displayed:
%    \begin{macrocode}
\ifchilddoc\else
\section{part three}
\input{cdocspt3}
\section{part four}
\input{cdocspt4}
\fi
%    \end{macrocode}

% Process as usual until here:
%    \begin{macrocode}
\fi
%    \end{macrocode}

% End of document body:
%    \begin{macrocode}
\end{document}
%    \end{macrocode}
%\iffalse
%</samplemain>
%\fi
%
% %%%%%%%%%%%%%%%%%%%%%%%%%%%%%%%%%%%%%%
% \paragraph{Chapter Include Files.}
%
% The include files are called |cdocsch1.tex| and |cdocsch2.tex|.
%
%\iffalse
%<*samplechap1|samplechap2>
%\fi

% Optional override for |\version| flag:
%    \begin{macrocode}
%%\providecommand{\version}{final}
%    \end{macrocode}

% Include the main document:
%    \begin{macrocode}
\input{childdoc.def}
\childdocof{cdocsamp}
%    \end{macrocode}

%\iffalse
%</samplechap1|samplechap2>
%\fi
%
%\iffalse
%<*samplechap1>
%\fi
% Some text for chapter 1:
%    \begin{macrocode}
\section{one}
some text in chapter one
%    \end{macrocode}

%\iffalse
%</samplechap1>
%\fi
% Some text for chapter 2:
%\iffalse
%<*samplechap2>
%\fi
%    \begin{macrocode}
\section{two}
more text in chapter two
%    \end{macrocode}

%\iffalse
%</samplechap2>
%\fi
%
% %%%%%%%%%%%%%%%%%%%%%%%%%%%%%%%%%%%%%%
% \paragraph{Part Include Files.}
%
% The include files are called |cdocspt3.tex| and |cdocspt4.tex|.
%
%\iffalse
%<*samplepart3|samplepart4>
%\fi

% Optional override for |\version| flag:
%    \begin{macrocode}
%%\providecommand{\version}{final}
%    \end{macrocode}

% Include the main document:
%    \begin{macrocode}
\input{childdoc.def}
\childdocby{cdocsamp}
%    \end{macrocode}

%\iffalse
%</samplepart3|samplepart4>
%\fi
%
%\iffalse
%<*samplepart3>
%\fi
% Some text for part 3:
%    \begin{macrocode}
some text in part three
%    \end{macrocode}

%\iffalse
%</samplepart3>
%\fi
% Some text for part 4:
%\iffalse
%<*samplepart4>
%\fi
%    \begin{macrocode}
more text in part four
%    \end{macrocode}

%\iffalse
%</samplepart4>
%\fi
%
% %%%%%%%%%%%%%%%%%%%%%%%%%%%%%%%%%%%%%%
% \paragraph{Forwarding for a Complete Draft.}
%
% The following forwarding file |cdocsdrf.tex|
% compiles the main document in draft mode:
%\iffalse
%<*sampledraft>
%\fi
%    \begin{macrocode}
\def\version{draft}
\input{childdoc.def}
\childdocforward{cdocsamp}
%    \end{macrocode}

%\iffalse
%</sampledraft>
%\fi
%
% %%%%%%%%%%%%%%%%%%%%%%%%%%%%%%%%%%%%%%
% \paragraph{Forwarding for Final Version of the Chapters.}
%
% The following forwarding files |cdocsfn1.tex| and |cdocsfn2.tex|
% (with identical content)
% compile the final versions of the child documents
% |cdocsch1.tex| and |cdocsch2.tex|, respectively:
%\iffalse
%<*samplefinal>
%\fi
%    \begin{macrocode}
\def\version{final}
\input{childdoc.def}
\childdocforwardprefix[cdocsamp]{cdocsfn}{cdocsch}
%    \end{macrocode}

%\iffalse
%</samplefinal>
%\fi
%
% %%%%%%%%%%%%%%%%%%%%%%%%%%%%%%%%%%%%%%
% \paragraph{Command Line Processing.}
%
% The following three command lines generate the output files
% |cdocscld|, |cdocscl1| and |cdocscl2|
% which should be identical to
% |cdocsdrf|, |cdocsch1| and |cdocsfn2|, respectively:
% \begin{center}
% \begin{tabular}{l}
% |latex -jobname cdocscld \|\\
% |  "\def\version{draft}\input{childdoc.def}\childdocforward{cdocsamp}"|\\
% |latex -jobname cdocscl1 \|\\
% |  "\input{childdoc.def}\childdocforward[cdocsamp]{cdocsch1}"|\\
% |latex -jobname cdocscl2 \|\\
% |  "\def\version{final}\input{childdoc.def}\childdocforward{cdocsch2}"|
% \end{tabular}
% \end{center}
% Note that the trailing backslash on each first line
% merely continues the input to the second line
% (for convenient cut ant paste).
% Furthermore, the command |latex| can be replaced by any
% of its alternative versions such as |pdflatex|.
%
% %%%%%%%%%%%%%%%%%%%%%%%%%%%%%%%%%%%%%%%%%%%%%%%%%%%%%%%%%%%%%%%%%%%%%%%%%%%%%%
% %%%%%%%%%%%%%%%%%%%%%%%%%%%%%%%%%%%%%%%%%%%%%%%%%%%%%%%%%%%%%%%%%%%%%%%%%%%%%%
% \section{Implementation}
%\iffalse
%<*package>
%\fi
%
% This section describes the definitions file |childdoc.def|.

% The definitions cannot be loaded using |\usepackage| or |\RequirePackage|
% which has a mechanism to prevent loading a style file more than once.
% When loading the definitions by means of |\input|
% multiple instances have to be prevented manually:
%\iffalse
%This code needs to be before the `\ProvidesFile' directive
%which is defined at the beginning of this file.
%Therefore it is also placed there and commented out here.
%</package>
%<*discard>
%\fi
%    \begin{macrocode}
\ifdefined\childdocmain\endinput\fi
%    \end{macrocode}
%\iffalse
%</discard>
%<*package>
%\fi
%
% \macro{\ifchilddoc}
% \macro{\ifchilddocmanual}
% The conditional |\ifchilddoc| tells whether a
% child (true) or main (false) document is being compiled.
% The conditional |\ifchilddocmanual| tells whether
% the |\includeonly| mechanism is used (false) or
% the selection of child files must be performed manually (true).
% The definitions initialise to false:
%    \begin{macrocode}
\newif\ifchilddoc
\newif\ifchilddocmanual
%    \end{macrocode}

% \macro{\childdocname}
% \macro{\childdocjob}
% The macro |\childdocname| stores the name of the main document
% to be compiled. The macro |\childdocjob| stores the name of
% the document on which the \LaTeX{} compiler was originally invoked.
% The content of |\jobname| cannot be compared
% to filenames specified in the source due to different catcodes.
% The following code rescans |\jobname|, stores the result
% in |\childdocname| and saves a copy in |\childdocjob|:
%    \begin{macrocode}
\edef\childdocname{\scantokens\expandafter{\jobname\noexpand}}
\let\childdocjob\childdocname
%    \end{macrocode}

% \macro{\childdocdisable}
% The macro |\childdocdisable| prevents the main file
% from being processed more than once.
% At this stage, the main document command |\childdocmain|
% is assumed to be called once again where it should do nothing.
% Any subsequent call to it should prevent
% a secondary processing of the main document
% It overwrites the forwarding commands
% |\childdocof| and |\childdocforward|
% with empty macros to prevent further inclusions of the main document:
%    \begin{macrocode}
\newcommand{\childdocdisable}
{
  \renewcommand{\childdocmain}[1]{\renewcommand{\childdocmain}[1]{\endinput}}
  \renewcommand{\childdocof}[1]{}
  \renewcommand{\childdocby}[2][]{}
  \renewcommand{\childdocforward}[2][]{}
  \renewcommand{\childdocdisable}{}
}
%    \end{macrocode}

% \macro{\childdocmain}
% The macro |\childdocmain| is to be called at the top of the main file
% with nothing or the main filename (without extension) as argument.
% First, it breaks loops.
% If the argument is not empty and does not match |\childdocname|
% (which is set by the first inclusion of |childdoc.def|),
% |\ifchilddoc| is set to true, |\includeonly| is applied to the child file
% and |\jobname| is set to the main file
% (for proper handling of |.aux| files):
%    \begin{macrocode}
\newcommand{\childdocmain}[1]
{
  \childdocdisable\childdocmain{}
  \if?#1?\else
    \begingroup
      \def\childdoctmp{#1}
      \ifx\childdoctmp\childdocname
        \def\childdoctmp{}
      \else
        \def\childdoctmp
        {
          \childdoctrue
          \includeonly{\childdocname}
          \def\childdocjob{#1}
          \def\jobname{#1}
        }
      \fi
      \expandafter
    \endgroup
    \childdoctmp
  \fi
}
%    \end{macrocode}

% \macro{\childdocof}
% The command |\childdocof| redirects
% compilation to the main file |#1|.
%    \begin{macrocode}
\newcommand{\childdocof}[1]
{
  \childdocdisable
  \childdoctrue
  \includeonly{\childdocname}
  \def\jobname{#1}
  \def\childdocjob{#1}
  \input{#1}
}
%    \end{macrocode}

% \macro{\childdocby}
% The command |\childdocby| ....
%    \begin{macrocode}
\newcommand{\childdocby}[2][]
{
  \childdocdisable
  \childdoctrue
  \childdocmanualtrue
  \if?#1?\else
    \def\jobname{#2}
  \fi
  \def\childdocjob{#2}
  \input{#2}
  \endinput
}
%    \end{macrocode}

% \macro{\childdocforward}
% The command |\childdocforward| redirects
% compilation to the main file or
% (if the optional argument is given) a child file.
% Parameters are set as if the main file
% or a child file starting with |\childdocof| was compiled.
% Then compilation is handed over to the main file:
%    \begin{macrocode}
\newcommand{\childdocforward}[2][]
{
  \begingroup
    \if?#1?
      \def\childdoctmp
      {
        \def\childdocname{#2}
        \def\childdocjob{#2}
        \def\jobname{#2}
        \input{#2}
        \endinput
      }
    \else
      \def\childdoctmp
      {
        \childdocdisable
        \def\childdocname{#2}
        \childdoctrue
        \includeonly{#2}
        \def\childdocjob{#1}
        \def\jobname{#1}
        \input{#1}
        \endinput
      }
    \fi
    \expandafter
  \endgroup
  \childdoctmp
}
%    \end{macrocode}

% \macro{\childdocforwardprefix}
% The command |\childdocforwardprefix| redirects
% compilation to the main or a child file by means of a pattern.
% The prefix |#1| in the current filename is replaced by |#2|
% and the suffix of the current filename is kept
% (it is assumed that the filename does not contain the substring `|~~~|'
% which is used as a delimiter).
% Compilation is handed over to the new file by |\childdocforward|:
%    \begin{macrocode}
\newcommand{\childdocforwardprefix}[3][]
{
  \begingroup
    \def\childdocextract #2##1~~~{\def\childdoctmp{\childdocforward[#1]{#3##1}}}
    \expandafter\childdocextract\childdocname~~~
    \expandafter
  \endgroup
  \childdoctmp
}
%    \end{macrocode}

% \macro{\childdoc}
% The deprecated macro |\childdoc| is a legacy version of |\childdocmain|:
%    \begin{macrocode}
\newcommand{\childdoc}{\childdocmain}
%    \end{macrocode}

% \macro{\childdocredirect}
% The deprecated macro |\childdocredirect| is a legacy version
% of |\childdocforward| and |\childdocforwardprefix|:
%    \begin{macrocode}
\newcommand{\childdocredirect}[2][]
{
  \begingroup
    \if?#1?
      \def\childdoctmp{\childdocforward{#2}}
    \else
      \def\childdoctmp{\childdocforwardprefix{#1}{#2}}
    \fi
    \expandafter
  \endgroup
  \childdoctmp
}
%    \end{macrocode}

%\iffalse
%</package>
%\fi
%
\endinput

\childdocmain{}
%    \end{macrocode}

% Optional override for |\version| flag:
%    \begin{macrocode}
%%\ifchilddoc\else\providecommand{\version}{draft}\fi
%    \end{macrocode}

% Define the default values for the |\version| flag
% (|final| for the main file and |draft| for childs):
%    \begin{macrocode}
\ifchilddoc
\providecommand{\version}{draft}
\else
\providecommand{\version}{final}
\fi
%    \end{macrocode}

% Load the standard document class:
%    \begin{macrocode}
\documentclass[12pt]{article}
%    \end{macrocode}

% Start the document body:
%    \begin{macrocode}
\begin{document}
%    \end{macrocode}

% Declare a title page.
% Print title, part of document being processed and version flag:
%    \begin{macrocode}
\addtocounter{page}{-1}
\begin{center}
{\LARGE\bfseries{}childdoc example\par}
\vspace{1cm}
\ifchilddoc
\ifchilddocmanual part\else chapter\fi:
`\childdocname' of `\childdocjob'\par
\else
main document: `\childdocjob'\par
\fi
version: \version\par
\end{center}
\newpage
%    \end{macrocode}

% Manually include selected file,
% otherwise process as usual:
%    \begin{macrocode}
\ifchilddocmanual
\section*{part `\childdocname'}
\input{\childdocname}
\else
%    \end{macrocode}

% Include the two chapters:
%    \begin{macrocode}
\include{cdocsch1}
\include{cdocsch2}
%    \end{macrocode}

% Include the two parts unless only chapters should be displayed:
%    \begin{macrocode}
\ifchilddoc\else
\section{part three}
\input{cdocspt3}
\section{part four}
\input{cdocspt4}
\fi
%    \end{macrocode}

% Process as usual until here:
%    \begin{macrocode}
\fi
%    \end{macrocode}

% End of document body:
%    \begin{macrocode}
\end{document}
%    \end{macrocode}
%\iffalse
%</samplemain>
%\fi
%
% %%%%%%%%%%%%%%%%%%%%%%%%%%%%%%%%%%%%%%
% \paragraph{Chapter Include Files.}
%
% The include files are called |cdocsch1.tex| and |cdocsch2.tex|.
%
%\iffalse
%<*samplechap1|samplechap2>
%\fi

% Optional override for |\version| flag:
%    \begin{macrocode}
%%\providecommand{\version}{final}
%    \end{macrocode}

% Include the main document:
%    \begin{macrocode}
% \iffalse
%
% childdoc.dtx Copyright (C) 2017-2018 Niklas Beisert
%
% This work may be distributed and/or modified under the
% conditions of the LaTeX Project Public License, either version 1.3
% of this license or (at your option) any later version.
% The latest version of this license is in
%   http://www.latex-project.org/lppl.txt
% and version 1.3 or later is part of all distributions of LaTeX
% version 2005/12/01 or later.
%
% This work has the LPPL maintenance status `maintained'.
%
% The Current Maintainer of this work is Niklas Beisert.
%
% This work consists of the files childdoc.dtx and childdoc.ins
% and the derived files childdoc.def and cdocsamp.tex with
% cdocsch1.tex, cdocsch2.tex, cdocsdrf.tex, cdocsfn1.tex, cdocsfn2.tex.
%
%<package>\ifdefined\childdocmain\endinput\fi
%<package>\ProvidesFile{childdoc.def}[2018/12/30 v2.0 child document driver]
%<samplemain>\ProvidesFile{cdocsamp.tex}[2018/12/30 v2.0 sample for childdoc]
%<*driver>
%\ProvidesFile{childdoc.drv}[2018/12/30 v2.0 childdoc reference manual file]
\PassOptionsToClass{10pt,a4paper}{article}
\documentclass{ltxdoc}

\usepackage[margin=35mm]{geometry}
\usepackage{hyperref}
\usepackage{hyperxmp}
\usepackage[usenames]{color}

\hypersetup{colorlinks=true}
\hypersetup{pdfstartview=FitH}
\hypersetup{pdfpagemode=UseNone}
\hypersetup{pdfsource={}}
\hypersetup{pdflang={en-UK}}
\hypersetup{pdfcopyright={Copyright 2017-2018 Niklas Beisert.
  This work may be distributed and/or modified under the
  conditions of the LaTeX Project Public License, either version 1.3
  of this license or (at your option) any later version.}}
\hypersetup{pdflicenseurl={http://www.latex-project.org/lppl.txt}}
\hypersetup{pdfcontactaddress={ETH Zurich, ITP, HIT K,
  Wolfgang-Pauli-Strasse 27}}
\hypersetup{pdfcontactpostcode={8093}}
\hypersetup{pdfcontactcity={Zurich}}
\hypersetup{pdfcontactcountry={Switzerland}}
\hypersetup{pdfcontactemail={nbeisert@itp.phys.ethz.ch}}
\hypersetup{pdfcontacturl={http://people.phys.ethz.ch/\xmptilde nbeisert/}}

\newcommand{\secref}[1]{\hyperref[#1]{section \ref*{#1}}}

\parskip1ex
\parindent0pt
\let\olditemize\itemize
\def\itemize{\olditemize\parskip0pt}

\begin{document}

\title{The \textsf{childdoc} Package}
\hypersetup{pdftitle={The childdoc Package}}
\author{Niklas Beisert\\[2ex]
  Institut f\"ur Theoretische Physik\\
  Eidgen\"ossische Technische Hochschule Z\"urich\\
  Wolfgang-Pauli-Strasse 27, 8093 Z\"urich, Switzerland\\[1ex]
  \href{mailto:nbeisert@itp.phys.ethz.ch}
  {\texttt{nbeisert@itp.phys.ethz.ch}}}
\hypersetup{pdfauthor={Niklas Beisert}}
\hypersetup{pdfsubject={Manual for the LaTeX2e Package childdoc}}
\date{30 December 2018, \textsf{v2.0}}
\maketitle

\begin{abstract}\noindent
\textsf{childdoc} is a \LaTeXe{} package
that enables the direct compilation
of document sections included by |\include|
to individual files.
\end{abstract}

\begingroup
\parskip0ex
\tableofcontents
\endgroup

%%%%%%%%%%%%%%%%%%%%%%%%%%%%%%%%%%%%%%%%%%%%%%%%%%%%%%%%%%%%%%%%%%%%%%%%%%%%%%%%
%%%%%%%%%%%%%%%%%%%%%%%%%%%%%%%%%%%%%%%%%%%%%%%%%%%%%%%%%%%%%%%%%%%%%%%%%%%%%%%%
\section{Introduction}

\LaTeX{} provides a mechanism to structure a large document (such as a book)
into a main file and several child files (containing the chapters)
using the |\include| command.
This mechanism is beneficial for documents
which span hundreds of pages in order to
make the source file(s) more manageable.
Moreover, compilation can be restricted to
selected child files by means of the |\includeonly| command.
The latter feature can be used to reduce the compilation time while editing
(this was significantly more useful in the earlier days of \LaTeX{})
or to generate a smaller document which is easier to navigate.
Another application of |\includeonly| is to generate
documents consisting of selected parts of the complete document.

However, there are a few drawbacks of the plain |\include| mechanism:
\begin{itemize}
\item
The child files cannot be compiled on their own,
they can only be compiled via the main file.
A naive editing environment
(such as a text editor with an option
to have the current file processed by \LaTeX)
may require one to switch to the main file before compiling;
attempting to compile the child file produces errors.
\item
The main file must be modified (each time)
to adjust the |\includeonly| command
to the present needs. This easily leaves the main file in a messy state.
\item
The generated document will always carry the filename
of the main document. This is inconvenient if
several child files are to be compiled and
to be kept for distribution.
\end{itemize}

The present package provides a simple interface
to make child files individually compilable by \LaTeX{}.
Compiling a child file then has the same effect as compiling
the main file with an |\includeonly| command
to select the appropriate child.
Moreover the generated document will carry the name of the child
rather than the main file.
This resolves all three above issues.

This feature is meant to make the editing of books,
thesis documents and lecture notes somewhat more convenient.
However, the package can also be used efficiently for
composing a series of documents (such as exercise sheets)
which are typically distributed individually.
It then assists the author in generating the individual documents
(potentially in different versions)
as well as a document containing the collected series.
Another application is in developing style files
or other kinds of included material
where compilation of the style file could redirect
to a sample or test file.

%%%%%%%%%%%%%%%%%%%%%%%%%%%%%%%%%%%%%%%%%%%%%%%%%%%%%%%%%%%%%%%%%%%%%%%%%%%%%%%%
%%%%%%%%%%%%%%%%%%%%%%%%%%%%%%%%%%%%%%%%%%%%%%%%%%%%%%%%%%%%%%%%%%%%%%%%%%%%%%%%
\section{Usage}

First of all, the package \textsf{childdoc} is \emph{not} a standard
\LaTeXe{} |.sty| style file! Therefore it needs to be invoked in
a non-standard way.

%%%%%%%%%%%%%%%%%%%%%%%%%%%%%%%%%%%%%%%%%%%%%%%%%%%%%%%%%%%%%%%%%%%%%%%%%%%%%%%%
\subsection{Included Files}
\label{sec:include}

%%%%%%%%%%%%%%%%%%%%%%%%%%%%%%%%%%%%%%%%
\DescribeMacro{\childdocmain}
To use the package, add the commands
\begin{center}
\begin{tabular}{l}
|\input{childdoc.def}|\\
|\childdocmain{}|\\
\end{tabular}
\end{center}
at the very top of the main \LaTeX{} file,
in particular \emph{before} the |\documentclass| statement!
The argument of |\childdocmain| should be left empty
(but it must be present).

%%%%%%%%%%%%%%%%%%%%%%%%%%%%%%%%%%%%%%%%
\DescribeMacro{\childdocof}
Furthermore, add the commands
\begin{center}
\begin{tabular}{l}
|\input{childdoc.def}|\\
|\childdocof{|\textit{main}|}|\\
\end{tabular}
\end{center}
at the top of every child file \textit{child}
which is included by |\include{|\textit{child}|}|
from within the main file
(or at least for those files to be compiled individually).
The argument \textit{main} must be the filename of the main file.

There are a couple of
considerations in setting up the main and child documents:

%%%%%%%%%%%%%%%%%%%%%%%%%%%%%%%%%%%%%%%%
\paragraph{Restrictions.}

Please note the following restrictions:
\begin{itemize}
\item
|\childdocmain| must be called with one argument \textit{main}
to ensure compatibility with earlier version of the package.
It must either be empty (|\childdocmain{}|)
or precisely match the filename of the main file in which it is specified.
See \secref{sec:detection} for further information.
\item
The filename \textit{main} must be specified without the |.tex| extension.
\item
The filename \textit{main} is case sensitive
(even in case-insensitive file systems)
due to internal string comparison.
\item
The argument \textit{main} should be fully expanded, it cannot be a macro.
\item
Subdirectories and special characters should be avoided in filenames.
\item
The command |\childdocmain{|\textit{main}|}| must be followed by a whitespace.
It should not be followed immediately by another command
or by a comment mark `|%|'.
This is because the \TeX{} parser reads the token immediately following
the argument of |\childdocmain| and puts it
at the beginning of every child section;
however, a white\-space is ignored.
\end{itemize}

%%%%%%%%%%%%%%%%%%%%%%%%%%%%%%%%%%%%%%%%
\paragraph{Content of Main File.}

It is advisable to place all content in the child files included by |\include|.
Any output contained in the main file will appear in all child documents
unless suppressed manually;
it cannot be suppressed automatically by the |\includeonly| directive
and thus should normally be avoided.
A method to include some content in the main file
by means of conditional processing is described in \secref{sec:conditional}.

%%%%%%%%%%%%%%%%%%%%%%%%%%%%%%%%%%%%%%%%
\paragraph{Page Numbering.}

When only a part of the document is compiled,
the appropriate numbering of pages
(as well as other status parameters)
is determined from the |.aux| files.
The latter contain information from previous passes.
However this information needs to propagate through
all intermediate child documents.
Therefore the page numbering in child documents may well
be inconsistent until the complete document is compiled at least once.

A useful (if unconventional) way to always ensure a consistent
page numbering is to restart the numbering in each child document
and denote the pages by `\textit{child}|.|\textit{page}'
where \textit{child} represents the chapter/section number of the child file.
This can be achieved by the command
|\numberwithin{page}{|\textit{child}|}|
of the \textsf{amsmath} package
where \textit{child} can be |chapter| or |section|
depending on the chosen structuring.
Alternatively, one can modify the macro |\thepage| appropriately
and reset the counter |page| at the start of each child file.

%%%%%%%%%%%%%%%%%%%%%%%%%%%%%%%%%%%%%%%%%%%%%%%%%%%%%%%%%%%%%%%%%%%%%%%%%%%%%%%%
\subsection{Conditional Processing}
\label{sec:conditional}

The package provides a mechanism to compile different versions
of a document. To customise the versions further some conditional processing
can come in handy to distinguish which version is being compiled.
The package provides two macros to describe the compilation context:

%%%%%%%%%%%%%%%%%%%%%%%%%%%%%%%%%%%%%%%%
\DescribeMacro{\ifchilddoc}
The conditional |\ifchilddoc| distinguishes between the compilation of
child documents and the main document:
%
\begin{center}
|\ifchilddoc |\textit{child-code}| |[|\||else |\textit{main-code}]| \||fi|
\end{center}

%%%%%%%%%%%%%%%%%%%%%%%%%%%%%%%%%%%%%%%%
\DescribeMacro{\childdocname}
\DescribeMacro{\childdocjob}
The macro |\childdocname| contains the filename (without extension)
of the main or child file being processed.
Note that |\childdocjob| will always contain the name of the main file.

%%%%%%%%%%%%%%%%%%%%%%%%%%%%%%%%%%%%%%%%
\paragraph{Title Page.}

Conditional processing can be used to include a title or banner page
in the main document when proper precautions are taken.
Importantly, the code in the main file should ensure that the page counter
(as well as other status parameters which are stored in the |.aux| files)
takes the same value after the conditional processing.
Otherwise the page numbers may take divergent values
depending on which part is compiled.

For example, a title page could be declared by:
%
\begin{center}
\begin{tabular}{l}
|\ifchilddoc\||else|\\
|\addtocounter{page}{-1}|\\
\textit{code for title page}\\
|\newpage|\\
|\||fi|
\end{tabular}
\end{center}
%
A banner page for the child documents can be generated by:
%
\begin{center}
\begin{tabular}{l}
|\ifchilddoc|\\
|\addtocounter{page}{-1}|\\
\textit{code for banner page}\\
|\newpage|\\
|\||fi|
\end{tabular}
\end{center}
%
Here one could write a message such as:
\begin{center}
|This is the part \childdocname{} of \childdocjob{}.|
\end{center}

%%%%%%%%%%%%%%%%%%%%%%%%%%%%%%%%%%%%%%%%%%%%%%%%%%%%%%%%%%%%%%%%%%%%%%%%%%%%%%%%
\subsection{Flags}
\label{sec:flags}

The package makes it easy to generate different versions
of the main or child documents.
To this end compilation flags can be defined
and assigned different default values.
They will be particularly useful in conjunction
with the forwarding mechanism described in \secref{sec:forward}.

For example, it may be useful to have a flag |\version|
which can be set to |draft| or |final|.
The document source will contain some conditional code
depending on the value of |\version|.
Suppose further, the flag should default to |final| for the main file
and to |draft| for child files
which is a natural assignment for editing the document.
This is achieved by placing the following code
in the preamble of the main document
(below the |\childdocmain| directive):
%
\begin{center}
\begin{tabular}{l}
|\ifchilddoc|\\
|\providecommand{\version}{draft}|\\
|\||else|\\
|\providecommand{\version}{final}|\\
|\||fi|
\end{tabular}
\end{center}
%
The definition by |\providecommand| makes sure
that previous definitions are not overwritten.
Further statements |\providecommand{\version}{...}|
can thus be added before the above code to override it.

For the main file, one might add a line
(between |\childdocmain| and the above block)
%
\begin{center}
|%\ifchilddoc\||else\providecommand{\version}{draft}\||fi|
\end{center}
%
which can be uncommented to produce a draft version.
Likewise one can add a line to the very top of a child file
(above the |\childdocof{|\textit{main}|}| directive)
%
\begin{center}
|%\providecommand{\version}{final}|
\end{center}
%
which can be uncommented to produce the final version of this child document.

%%%%%%%%%%%%%%%%%%%%%%%%%%%%%%%%%%%%%%%%%%%%%%%%%%%%%%%%%%%%%%%%%%%%%%%%%%%%%%%%
\subsection{Forwarding}
\label{sec:forward}

Different versions of the main or child documents
using compilation flags as described in \secref{sec:flags}
can be (permanently) stored in different files
for convenient compilation, viewing and distribution.
To this end, the package defines a command
to pass on compilation to a different file:

%%%%%%%%%%%%%%%%%%%%%%%%%%%%%%%%%%%%%%%%
\DescribeMacro{\childdocforward}
The command |\childdocforward| redirects processing to
another source file:
%
\begin{center}
\begin{tabular}{l}
|\input{childdoc.def}|\\
|\childdocforward[|\textit{main}|]{|\textit{dest}|}|\\
\end{tabular}
\end{center}
%
The argument \textit{dest} is the destination file
(without extension).
It should be the main file or one of the child files.
Note that further \textsf{childdoc} directives
such as |\childdocof| and |\childdocforward|
in the indicated file will be processed in this form.
The optional argument \textit{main}
passes on directly to the main file \textit{main}
while pretending to compile the child \textit{dest}.
This form behaves as if \textit{dest}
issues |\childdocof{|\textit{main}|}| right away,
and no further \textsf{childdoc} directives will be processed.

%%%%%%%%%%%%%%%%%%%%%%%%%%%%%%%%%%%%%%%%
\DescribeMacro{\...prefix}
In the alternative form |\childdocforwardprefix|,
%
\begin{center}
\begin{tabular}{l}
|\input{childdoc.def}|\\
|\childdocforwardprefix[|\textit{main}|]{|\textit{prefix}|}{|\textit{dest}|}|
\end{tabular}
\end{center}
%
the destination file is determined by a pattern
depending on the current file:
To make this work, the current file must be called
`{\textit{prefix}\hspace{0.2em}\textit{suffix}}'
with \textit{prefix} matching precisely the argument.
Processing is then passed on to the file
`{\textit{dest}\hspace{0.2em}\textit{suffix}}'.
Surely, the same effect is achieved by
directly specifying the
argument `{\textit{dest}\hspace{0.2em}\textit{suffix}}'
in the first form.
However, that requires to set up a different file
for each child. With the alternative form of the command
all these files can have exactly the same content
which simplifies setting them up and maintaining them.

For example, the following file |draft.tex|
with a compilation flag |\version| as described in \secref{sec:flags}
compiles the main document as a draft:
%
\begin{center}
\begin{tabular}{l}
|\def\version{draft}|\\
|\input{childdoc.def}|\\
|\childdocforward{|\textit{main}|}|
\end{tabular}
\end{center}
%
Likewise, the following files |final|\textit{nn}|.tex|
compile the final version of the child document
|child|\textit{nn}|.tex|:
%
\begin{center}
\begin{tabular}{l}
|\def\version{final}|\\
|\input{childdoc.def}|\\
|\childdocforwardprefix{final}{child}|
\end{tabular}
\end{center}
%

Note that when several versions of a main file and/or of each child file
are to be generated, it may be convenient to set up a |Makefile| or
shell script to automatise the process.

%%%%%%%%%%%%%%%%%%%%%%%%%%%%%%%%%%%%%%%%%%%%%%%%%%%%%%%%%%%%%%%%%%%%%%%%%%%%%%%%
\subsection{Command Line Processing}
\label{sec:commandline}

The effect of redirection files can also be achieved by invoking
the \LaTeX{} compiler with a more elaborate command line.
Most conveniently this should be done as part
of a shell script or a |Makefile|.

When using \textsf{childdoc} in the main file, the following
command lines effectively perform a redirection
(note that depending on the shell being used,
backslashes may have to be doubled: `|\|' $\to$ `|\\|'):
%
\begin{center}
|... -jobname "|\textit{target}|" |\\|"|[\textit{flags}]%
|\input{childdoc.def}\childdocforward[|\textit{main}|]{|\textit{dest}|}"|
\end{center}
%
Here \textit{target} is the name of the output file,
\textit{main} is the name of the main file
and \textit{dest} is the name of the main or child file to be processed
(all filenames without extensions).
The optional argument \textit{main} can be omitted
if \textit{main} matches \textit{dest}.
Optionally, compilation \textit{flags} can be defined via |\def| commands.
This command line makes the \TeX{} engine believe
it is compiling the file \textit{target}
whose content is specified as the latter parameter.
The provided code then forwards the processing to
\textit{main} or \textit{dest} as described in \secref{sec:forward}.

%%%%%%%%%%%%%%%%%%%%%%%%%%%%%%%%%%%%%%%%%%%%%%%%%%%%%%%%%%%%%%%%%%%%%%%%%%%%%%%%
\subsection{Include by Input}
\label{sec:input}

Including child documents by |\include| has some restrictions by design.
Most notably, the content of a child document always occupies
its own set of pages; pages cannot be shared between child documents.
Usually, this behaviour makes perfect sense
because each child document contain an essential part of the document.
However, in some situations it may be desirable to compose
a document from a collection of parts
without having mandatory page breaks between then.
For this case, the package
provides a mechanism to include parts
by |\input| which can also be processed individually.
However, by construction this mechanism
requires manual handling of the content to be output.

%%%%%%%%%%%%%%%%%%%%%%%%%%%%%%%%%%%%%%%%
\DescribeMacro{\ifchilddocmanual}
The main file should be prepared as usual, see \secref{sec:include}.
However, the document body must make a distinction
between processing of an individual part and of the main document, e.g.:
%
\begin{center}
\begin{tabular}{l}
|\ifchilddocmanual|\\
|\input{\childdocname}|\\
|\||else|\\
\textit{document body with }|\input{|\textit{part}|}|\\
|\||fi|
\end{tabular}
\end{center}
%
The conditional |\ifchilddocmanual| is true whenever
a part to be included by |\input| is being compiled,
and the name of the part is stored in |\childdocname|.

%%%%%%%%%%%%%%%%%%%%%%%%%%%%%%%%%%%%%%%%
\DescribeMacro{\childdocby}
Each part to be included by |\input| should start with:
%
\begin{center}
\begin{tabular}{l}
|\input{childdoc.def}|\\
|\childdocby{|\textit{main}|}|\\
\end{tabular}
\end{center}
%
The directive |\childdocby| is similar to |\childdocof|
described in \secref{sec:include},
but the subsequent selection of content must be done manually.
To that end, both |\ifchilddoc| and |\ifchilddocmanual|
will be true upon processing of a part,
and the name of the part is stored in |\childdocname|.
Note that |\jobname| will be set to the filename of the current part
so that each part receives an individual |.aux| file
that does not interfere with the |.aux| file(s) of the main document.
This behaviour can be altered by the alternative form
|\childdocby[*]{|\textit{main}|}| (with a non-empty optional argument)
which uses the |.aux| file of the main document
by setting |\jobname| to \textit{main}.

%%%%%%%%%%%%%%%%%%%%%%%%%%%%%%%%%%%%%%%%%%%%%%%%%%%%%%%%%%%%%%%%%%%%%%%%%%%%%%%%
\subsection{Driver Development}
\label{sec:driver}

The \textsf{childdoc} mechanism can also be use for the development
of definition files such as \LaTeX{} styles or classes.
This case differs from the above setup with multiple parts
included by |\include| in that no |\includeonly| should be invoked.
This can be achieved by starting the include file
(before |\ProvidesPackage|) with:
%
\begin{center}
\begin{tabular}{l}
|\input{childdoc.def}|\\
|\childdocforward{|\textit{main}|}|\\
\end{tabular}
\end{center}
%
or alternatively with:
%
\begin{center}
\begin{tabular}{l}
|\input{childdoc.def}|\\
|\childdocby{|\textit{main}|}|\\
\end{tabular}
\end{center}
%
Both forms have slightly different effects as described above.
The main file is prepared as usual, see \secref{sec:include}.

%%%%%%%%%%%%%%%%%%%%%%%%%%%%%%%%%%%%%%%%%%%%%%%%%%%%%%%%%%%%%%%%%%%%%%%%%%%%%%%%
\subsection{Legacy Detection}
\label{sec:detection}

The directive |\childdocmain| in the main file can detect
whether the complete document or merely a child is to be compiled
even without using the directive |\childdocof|.
This method is deprecated because it is less robust
and there is no compelling reason to use it;
it is merely provided for backward compatibility
and it may be removed in future versions.

If the detection mechanism is to be used,
it is mandatory to correctly specify
the filename of the main file as the argument of |\childdocmain|:
%
\begin{center}
\begin{tabular}{l}
|\input{childdoc.def}|\\
|\childdocmain{|\textit{main}|}|\\
\end{tabular}
\end{center}
%
If |\jobname| does not match the argument \textit{main} of |\childdocmain|,
it is assumed that |\jobname| points to the child file to be compiled.
When using |\childdocmain| with the main file specified as argument,
it suffices to start a child file
with just |\input{|\textit{main}|}|
without loading of the package and using |\childdocof|.
If instead all processing is done
with the appropriate \textsf{childdoc} directives,
the argument of \textit{main} of |\childdocmain| can be empty.

An alternative version of the command line processing described
in \secref{sec:commandline} using the detection mechanism reads:
%
\begin{center}
|... -jobname "|\textit{target}|" "|[\textit{flags}]%
[|\def\jobname{|\textit{dest}|}|]|\input{|\textit{main}|}"|
\end{center}

%%%%%%%%%%%%%%%%%%%%%%%%%%%%%%%%%%%%%%%%%%%%%%%%%%%%%%%%%%%%%%%%%%%%%%%%%%%%%%%%
\subsection{Manual Code}
\label{sec:manual}

In case one cannot be certain whether the definitions file |childdoc.def|
is installed on the target \TeX{} distribution
and one prefers not to ship it,
it is conceivable to paste a few relevant commands into the sources.

To that end, drop all statements |\input{childdoc.def}|
and perform the replacements as outlined below.
Instead of |\childdocmain{|\textit{main}|}| add the following code
to the top of the main file:
%
\begin{center}
\begin{tabular}{l}
|\||ifdefined\childdocname\endinput\||fi\newif\ifchilddoc|\\
|\edef\childdocname{\scantokens\expandafter{\jobname\noexpand}}|\\
|\def\childdocmain{|\textit{main}|}\||ifx\childdocmain\childdocname\||else|\\
|\childdoctrue\includeonly{\childdocname}\let\jobname\childdocmain\||fi|\\
\end{tabular}
\end{center}
%
Instead of |\childdocof{|\textit{main}|}| just include the main file
at the top of each child file:
%
\begin{center}
|\input{|\textit{main}|}|
\end{center}
%
A simple redirection |\childdocforward{|\textit{dest}|}| is achieved by:
%
\begin{center}
|\def\jobname{|\textit{dest}|}\input{\jobname}|
\end{center}
%
The redirection with prefix
|\childdocforwardprefix[|\textit{prefix}|]{|\textit{dest}|}|
is accomplished by:
%
\begin{center}
\begin{tabular}{l}
|{\edef\jobname{\scantokens\expandafter{\jobname\noexpand}}|\\
|\def\redirectjob |\textit{prefix}|#1~~~{\gdef\jobname{|\textit{dest}|#1}}|\\
|\expandafter\redirectjob\jobname~~~}\input{\jobname}|
\end{tabular}
\end{center}

In an alternative approach,
child documents can be compiled by a specific command line
without additional code or specific definitions:
%
\begin{center}
|... -jobname "|\textit{target}|" "|[\textit{flags}]%
|\includeonly{|\textit{dest}|}\input{|\textit{main}|}"|
\end{center}
%

%%%%%%%%%%%%%%%%%%%%%%%%%%%%%%%%%%%%%%%%%%%%%%%%%%%%%%%%%%%%%%%%%%%%%%%%%%%%%%%%
%%%%%%%%%%%%%%%%%%%%%%%%%%%%%%%%%%%%%%%%%%%%%%%%%%%%%%%%%%%%%%%%%%%%%%%%%%%%%%%%
\section{Information}

%%%%%%%%%%%%%%%%%%%%%%%%%%%%%%%%%%%%%%%%%%%%%%%%%%%%%%%%%%%%%%%%%%%%%%%%%%%%%%%%
\subsection{Copyright}

Copyright \copyright{} 2017--2018 Niklas Beisert

This work may be distributed and/or modified under the
conditions of the \LaTeX{} Project Public License, either version 1.3
of this license or (at your option) any later version.
The latest version of this license is in
  \url{http://www.latex-project.org/lppl.txt}
and version 1.3 or later is part of all distributions of \LaTeX{}
version 2005/12/01 or later.

This work has the LPPL maintenance status `maintained'.

The Current Maintainer of this work is Niklas Beisert.

This work consists of the files |README.txt|, |childdoc.ins| and |childdoc.dtx|
as well as the derived files |childdoc.def|, |cdocsamp.tex|
with |cdocsch1.tex|, |cdocsch2.tex|, |cdocspt3.tex|, |cdocspt4.tex|,
|cdocsdrf.tex|, |cdocsfn1.tex|, |cdocsfn2.tex|
as well as |childdoc.pdf|.

%%%%%%%%%%%%%%%%%%%%%%%%%%%%%%%%%%%%%%%%%%%%%%%%%%%%%%%%%%%%%%%%%%%%%%%%%%%%%%%%
\subsection{Files and Installation}

The package consists of the files:
%
\begin{center}
\begin{tabular}{ll}
    |README.txt|   & readme file \\
    |childdoc.ins| & installation file \\
    |childdoc.dtx| & source file \\
    |childdoc.def| & definition file \\
    |cdocsamp.tex| & sample main file \\
    |cdocsch1.tex| & sample include file \\
    |cdocsch2.tex| & sample include file \\
    |cdocspt3.tex| & sample part file \\
    |cdocspt4.tex| & sample part file \\
    |cdocsdrf.tex| & sample redirection file \\
    |cdocsfn1.tex| & sample redirection file \\
    |cdocsfn2.tex| & sample redirection file \\
    |childdoc.pdf| & manual
\end{tabular}
\end{center}
%
The distribution consists of the files
|README.txt|, |childdoc.ins| and |childdoc.dtx|.
%
\begin{itemize}
\item
Run (pdf)\LaTeX{} on |childdoc.dtx|
to compile the manual |childdoc.pdf| (this file).
\item
Run \LaTeX{} on |childdoc.ins| to create the definitions file |childdoc.def|
and the sample |cdocsamp.tex| with include files
|cdocsch1.tex|, |cdocsch2.tex|, |cdocspt3.tex|, |cdocspt4.tex|,
|cdocsdrf.tex|, |cdocsfn1.tex|, |cdocsfn2.tex|.
Then copy the file |childdoc.def| to an appropriate directory of your \LaTeX{}
distribution, e.g.\ \textit{texmf-root}|/tex/latex/childdoc|.
\end{itemize}

%%%%%%%%%%%%%%%%%%%%%%%%%%%%%%%%%%%%%%%%%%%%%%%%%%%%%%%%%%%%%%%%%%%%%%%%%%%%%%%%
\subsection{Related CTAN Packages}

There are several other packages which offer a similar functionality:
%
\begin{itemize}
\item
The packages
\href{http://ctan.org/pkg/docmute}{\textsf{docmute}},
\href{http://ctan.org/pkg/includex}{\textsf{includex}} and
\href{http://ctan.org/pkg/standalone}{\textsf{standalone}}
provide commands to include only the document body of
a child file thus allowing both files to be compiled individually.
\item
The packages \href{http://ctan.org/pkg/subdocs}{\textsf{subdocs}}
and \href{http://ctan.org/pkg/subfiles}{\textsf{subfiles}}
provide structures in which the main and child documents can be
encapsulated and allowing them to be compiled individually.
The inclusion mechanism is different from the conventional |\include|.
\item
The package \href{http://ctan.org/pkg/combine}{\textsf{combine}}
is an elaborate solution to combine several documents into one.
\end{itemize}
%
See also the CTAN topic \href{http://ctan.org/topic/subdocs}{\textsf{subdocs}}
for further related packages.
The present package differs from the above solutions in that
a document structure constructed with the conventional |\include| mechanism
just needs two extra commands at the top of every file
such that all constituent files can be compiled individually.

%%%%%%%%%%%%%%%%%%%%%%%%%%%%%%%%%%%%%%%%%%%%%%%%%%%%%%%%%%%%%%%%%%%%%%%%%%%%%%%%
%\subsection{Feature Suggestions}
%
%The following is a list of features which may be useful for future
%versions of this package:
%%
%\begin{itemize}
%\item
%\ldots
%\end{itemize}

%%%%%%%%%%%%%%%%%%%%%%%%%%%%%%%%%%%%%%%%%%%%%%%%%%%%%%%%%%%%%%%%%%%%%%%%%%%%%%%%
\subsection{Revision History}

%%%%%%%%%%%%%%%%%%%%%%%%%%%%%%%%%%%%%%%%
\paragraph{v2.0:} 2018/12/30

\begin{itemize}
\item
immediate forward processing
\item
added |\childdocby| mechanism
\item
manual restructured
\end{itemize}

%%%%%%%%%%%%%%%%%%%%%%%%%%%%%%%%%%%%%%%%
\paragraph{v1.6:} 2018/01/17

\begin{itemize}
\item
application for development of include files
\item
corrections to manual
\end{itemize}

%%%%%%%%%%%%%%%%%%%%%%%%%%%%%%%%%%%%%%%%
\paragraph{v1.5:} 2017/05/21

\begin{itemize}
\item
more complete structuring introduced
\item
|\childdocof| introduced
\item
|\childdoc| renamed to |\childdocmain|
\item
|\childredirect| renamed to |\childdocforward| and |\childdocforwardprefix|
and functionality expanded
\end{itemize}

%%%%%%%%%%%%%%%%%%%%%%%%%%%%%%%%%%%%%%%%
\paragraph{v1.0:} 2017/04/27

\begin{itemize}
\item
manual and install package
\item
first version published on CTAN
\end{itemize}

%%%%%%%%%%%%%%%%%%%%%%%%%%%%%%%%%%%%%%%%
\paragraph{v0.6:} 2017/04/26

\begin{itemize}
\item
redirection mechanism added
\end{itemize}

%%%%%%%%%%%%%%%%%%%%%%%%%%%%%%%%%%%%%%%%
\paragraph{v0.5:} 2017/04/26

\begin{itemize}
\item
functionality in definition file
\end{itemize}


%%%%%%%%%%%%%%%%%%%%%%%%%%%%%%%%%%%%%%%%%%%%%%%%%%%%%%%%%%%%%%%%%%%%%%%%%%%%%%%%
%%%%%%%%%%%%%%%%%%%%%%%%%%%%%%%%%%%%%%%%%%%%%%%%%%%%%%%%%%%%%%%%%%%%%%%%%%%%%%%%
%%%%%%%%%%%%%%%%%%%%%%%%%%%%%%%%%%%%%%%%%%%%%%%%%%%%%%%%%%%%%%%%%%%%%%%%%%%%%%%%
\appendix

\settowidth\MacroIndent{\rmfamily\scriptsize 000\ }

 \DocInput{childdoc.dtx}

\end{document}
%</driver>
% \fi
%
% %%%%%%%%%%%%%%%%%%%%%%%%%%%%%%%%%%%%%%%%%%%%%%%%%%%%%%%%%%%%%%%%%%%%%%%%%%%%%%
% %%%%%%%%%%%%%%%%%%%%%%%%%%%%%%%%%%%%%%%%%%%%%%%%%%%%%%%%%%%%%%%%%%%%%%%%%%%%%%
% \section{Sample}
%\iffalse
%<*samplemain>
%\fi
%
% The following presents a sample document
% with two chapters, two parts, a title page,
% a compile flag as well as three forwarding files to set the flag.
% It consists of eight |.tex| files:
% \begin{center}
% \begin{tabular}{ll}
% |cdocsamp.tex|&main file\\
% |cdocsch1.tex|&include file for chapter 1\\
% |cdocsch2.tex|&include file for chapter 2\\
% |cdocspt3.tex|&include file for part 3\\
% |cdocspt4.tex|&include file for part 4\\
% |cdocsdrf.tex|&forwarding file for main file in draft mode\\
% |cdocsfi1.tex|&forwarding file for final version of chapter 1\\
% |cdocsfi2.tex|&forwarding file for final version of chapter 2\\
% \end{tabular}
% \end{center}
% Each of the eight files can be compiled directly by the \LaTeX{} compiler.
%
% %%%%%%%%%%%%%%%%%%%%%%%%%%%%%%%%%%%%%%
% \paragraph{Main File.}
%
% The main file is called |cdocsamp.tex|.
%
% Load the \textsf{childdoc} definitions and
% declare the filename for the main document:
%    \begin{macrocode}
\input{childdoc.def}
\childdocmain{}
%    \end{macrocode}

% Optional override for |\version| flag:
%    \begin{macrocode}
%%\ifchilddoc\else\providecommand{\version}{draft}\fi
%    \end{macrocode}

% Define the default values for the |\version| flag
% (|final| for the main file and |draft| for childs):
%    \begin{macrocode}
\ifchilddoc
\providecommand{\version}{draft}
\else
\providecommand{\version}{final}
\fi
%    \end{macrocode}

% Load the standard document class:
%    \begin{macrocode}
\documentclass[12pt]{article}
%    \end{macrocode}

% Start the document body:
%    \begin{macrocode}
\begin{document}
%    \end{macrocode}

% Declare a title page.
% Print title, part of document being processed and version flag:
%    \begin{macrocode}
\addtocounter{page}{-1}
\begin{center}
{\LARGE\bfseries{}childdoc example\par}
\vspace{1cm}
\ifchilddoc
\ifchilddocmanual part\else chapter\fi:
`\childdocname' of `\childdocjob'\par
\else
main document: `\childdocjob'\par
\fi
version: \version\par
\end{center}
\newpage
%    \end{macrocode}

% Manually include selected file,
% otherwise process as usual:
%    \begin{macrocode}
\ifchilddocmanual
\section*{part `\childdocname'}
\input{\childdocname}
\else
%    \end{macrocode}

% Include the two chapters:
%    \begin{macrocode}
\include{cdocsch1}
\include{cdocsch2}
%    \end{macrocode}

% Include the two parts unless only chapters should be displayed:
%    \begin{macrocode}
\ifchilddoc\else
\section{part three}
\input{cdocspt3}
\section{part four}
\input{cdocspt4}
\fi
%    \end{macrocode}

% Process as usual until here:
%    \begin{macrocode}
\fi
%    \end{macrocode}

% End of document body:
%    \begin{macrocode}
\end{document}
%    \end{macrocode}
%\iffalse
%</samplemain>
%\fi
%
% %%%%%%%%%%%%%%%%%%%%%%%%%%%%%%%%%%%%%%
% \paragraph{Chapter Include Files.}
%
% The include files are called |cdocsch1.tex| and |cdocsch2.tex|.
%
%\iffalse
%<*samplechap1|samplechap2>
%\fi

% Optional override for |\version| flag:
%    \begin{macrocode}
%%\providecommand{\version}{final}
%    \end{macrocode}

% Include the main document:
%    \begin{macrocode}
\input{childdoc.def}
\childdocof{cdocsamp}
%    \end{macrocode}

%\iffalse
%</samplechap1|samplechap2>
%\fi
%
%\iffalse
%<*samplechap1>
%\fi
% Some text for chapter 1:
%    \begin{macrocode}
\section{one}
some text in chapter one
%    \end{macrocode}

%\iffalse
%</samplechap1>
%\fi
% Some text for chapter 2:
%\iffalse
%<*samplechap2>
%\fi
%    \begin{macrocode}
\section{two}
more text in chapter two
%    \end{macrocode}

%\iffalse
%</samplechap2>
%\fi
%
% %%%%%%%%%%%%%%%%%%%%%%%%%%%%%%%%%%%%%%
% \paragraph{Part Include Files.}
%
% The include files are called |cdocspt3.tex| and |cdocspt4.tex|.
%
%\iffalse
%<*samplepart3|samplepart4>
%\fi

% Optional override for |\version| flag:
%    \begin{macrocode}
%%\providecommand{\version}{final}
%    \end{macrocode}

% Include the main document:
%    \begin{macrocode}
\input{childdoc.def}
\childdocby{cdocsamp}
%    \end{macrocode}

%\iffalse
%</samplepart3|samplepart4>
%\fi
%
%\iffalse
%<*samplepart3>
%\fi
% Some text for part 3:
%    \begin{macrocode}
some text in part three
%    \end{macrocode}

%\iffalse
%</samplepart3>
%\fi
% Some text for part 4:
%\iffalse
%<*samplepart4>
%\fi
%    \begin{macrocode}
more text in part four
%    \end{macrocode}

%\iffalse
%</samplepart4>
%\fi
%
% %%%%%%%%%%%%%%%%%%%%%%%%%%%%%%%%%%%%%%
% \paragraph{Forwarding for a Complete Draft.}
%
% The following forwarding file |cdocsdrf.tex|
% compiles the main document in draft mode:
%\iffalse
%<*sampledraft>
%\fi
%    \begin{macrocode}
\def\version{draft}
\input{childdoc.def}
\childdocforward{cdocsamp}
%    \end{macrocode}

%\iffalse
%</sampledraft>
%\fi
%
% %%%%%%%%%%%%%%%%%%%%%%%%%%%%%%%%%%%%%%
% \paragraph{Forwarding for Final Version of the Chapters.}
%
% The following forwarding files |cdocsfn1.tex| and |cdocsfn2.tex|
% (with identical content)
% compile the final versions of the child documents
% |cdocsch1.tex| and |cdocsch2.tex|, respectively:
%\iffalse
%<*samplefinal>
%\fi
%    \begin{macrocode}
\def\version{final}
\input{childdoc.def}
\childdocforwardprefix[cdocsamp]{cdocsfn}{cdocsch}
%    \end{macrocode}

%\iffalse
%</samplefinal>
%\fi
%
% %%%%%%%%%%%%%%%%%%%%%%%%%%%%%%%%%%%%%%
% \paragraph{Command Line Processing.}
%
% The following three command lines generate the output files
% |cdocscld|, |cdocscl1| and |cdocscl2|
% which should be identical to
% |cdocsdrf|, |cdocsch1| and |cdocsfn2|, respectively:
% \begin{center}
% \begin{tabular}{l}
% |latex -jobname cdocscld \|\\
% |  "\def\version{draft}\input{childdoc.def}\childdocforward{cdocsamp}"|\\
% |latex -jobname cdocscl1 \|\\
% |  "\input{childdoc.def}\childdocforward[cdocsamp]{cdocsch1}"|\\
% |latex -jobname cdocscl2 \|\\
% |  "\def\version{final}\input{childdoc.def}\childdocforward{cdocsch2}"|
% \end{tabular}
% \end{center}
% Note that the trailing backslash on each first line
% merely continues the input to the second line
% (for convenient cut ant paste).
% Furthermore, the command |latex| can be replaced by any
% of its alternative versions such as |pdflatex|.
%
% %%%%%%%%%%%%%%%%%%%%%%%%%%%%%%%%%%%%%%%%%%%%%%%%%%%%%%%%%%%%%%%%%%%%%%%%%%%%%%
% %%%%%%%%%%%%%%%%%%%%%%%%%%%%%%%%%%%%%%%%%%%%%%%%%%%%%%%%%%%%%%%%%%%%%%%%%%%%%%
% \section{Implementation}
%\iffalse
%<*package>
%\fi
%
% This section describes the definitions file |childdoc.def|.

% The definitions cannot be loaded using |\usepackage| or |\RequirePackage|
% which has a mechanism to prevent loading a style file more than once.
% When loading the definitions by means of |\input|
% multiple instances have to be prevented manually:
%\iffalse
%This code needs to be before the `\ProvidesFile' directive
%which is defined at the beginning of this file.
%Therefore it is also placed there and commented out here.
%</package>
%<*discard>
%\fi
%    \begin{macrocode}
\ifdefined\childdocmain\endinput\fi
%    \end{macrocode}
%\iffalse
%</discard>
%<*package>
%\fi
%
% \macro{\ifchilddoc}
% \macro{\ifchilddocmanual}
% The conditional |\ifchilddoc| tells whether a
% child (true) or main (false) document is being compiled.
% The conditional |\ifchilddocmanual| tells whether
% the |\includeonly| mechanism is used (false) or
% the selection of child files must be performed manually (true).
% The definitions initialise to false:
%    \begin{macrocode}
\newif\ifchilddoc
\newif\ifchilddocmanual
%    \end{macrocode}

% \macro{\childdocname}
% \macro{\childdocjob}
% The macro |\childdocname| stores the name of the main document
% to be compiled. The macro |\childdocjob| stores the name of
% the document on which the \LaTeX{} compiler was originally invoked.
% The content of |\jobname| cannot be compared
% to filenames specified in the source due to different catcodes.
% The following code rescans |\jobname|, stores the result
% in |\childdocname| and saves a copy in |\childdocjob|:
%    \begin{macrocode}
\edef\childdocname{\scantokens\expandafter{\jobname\noexpand}}
\let\childdocjob\childdocname
%    \end{macrocode}

% \macro{\childdocdisable}
% The macro |\childdocdisable| prevents the main file
% from being processed more than once.
% At this stage, the main document command |\childdocmain|
% is assumed to be called once again where it should do nothing.
% Any subsequent call to it should prevent
% a secondary processing of the main document
% It overwrites the forwarding commands
% |\childdocof| and |\childdocforward|
% with empty macros to prevent further inclusions of the main document:
%    \begin{macrocode}
\newcommand{\childdocdisable}
{
  \renewcommand{\childdocmain}[1]{\renewcommand{\childdocmain}[1]{\endinput}}
  \renewcommand{\childdocof}[1]{}
  \renewcommand{\childdocby}[2][]{}
  \renewcommand{\childdocforward}[2][]{}
  \renewcommand{\childdocdisable}{}
}
%    \end{macrocode}

% \macro{\childdocmain}
% The macro |\childdocmain| is to be called at the top of the main file
% with nothing or the main filename (without extension) as argument.
% First, it breaks loops.
% If the argument is not empty and does not match |\childdocname|
% (which is set by the first inclusion of |childdoc.def|),
% |\ifchilddoc| is set to true, |\includeonly| is applied to the child file
% and |\jobname| is set to the main file
% (for proper handling of |.aux| files):
%    \begin{macrocode}
\newcommand{\childdocmain}[1]
{
  \childdocdisable\childdocmain{}
  \if?#1?\else
    \begingroup
      \def\childdoctmp{#1}
      \ifx\childdoctmp\childdocname
        \def\childdoctmp{}
      \else
        \def\childdoctmp
        {
          \childdoctrue
          \includeonly{\childdocname}
          \def\childdocjob{#1}
          \def\jobname{#1}
        }
      \fi
      \expandafter
    \endgroup
    \childdoctmp
  \fi
}
%    \end{macrocode}

% \macro{\childdocof}
% The command |\childdocof| redirects
% compilation to the main file |#1|.
%    \begin{macrocode}
\newcommand{\childdocof}[1]
{
  \childdocdisable
  \childdoctrue
  \includeonly{\childdocname}
  \def\jobname{#1}
  \def\childdocjob{#1}
  \input{#1}
}
%    \end{macrocode}

% \macro{\childdocby}
% The command |\childdocby| ....
%    \begin{macrocode}
\newcommand{\childdocby}[2][]
{
  \childdocdisable
  \childdoctrue
  \childdocmanualtrue
  \if?#1?\else
    \def\jobname{#2}
  \fi
  \def\childdocjob{#2}
  \input{#2}
  \endinput
}
%    \end{macrocode}

% \macro{\childdocforward}
% The command |\childdocforward| redirects
% compilation to the main file or
% (if the optional argument is given) a child file.
% Parameters are set as if the main file
% or a child file starting with |\childdocof| was compiled.
% Then compilation is handed over to the main file:
%    \begin{macrocode}
\newcommand{\childdocforward}[2][]
{
  \begingroup
    \if?#1?
      \def\childdoctmp
      {
        \def\childdocname{#2}
        \def\childdocjob{#2}
        \def\jobname{#2}
        \input{#2}
        \endinput
      }
    \else
      \def\childdoctmp
      {
        \childdocdisable
        \def\childdocname{#2}
        \childdoctrue
        \includeonly{#2}
        \def\childdocjob{#1}
        \def\jobname{#1}
        \input{#1}
        \endinput
      }
    \fi
    \expandafter
  \endgroup
  \childdoctmp
}
%    \end{macrocode}

% \macro{\childdocforwardprefix}
% The command |\childdocforwardprefix| redirects
% compilation to the main or a child file by means of a pattern.
% The prefix |#1| in the current filename is replaced by |#2|
% and the suffix of the current filename is kept
% (it is assumed that the filename does not contain the substring `|~~~|'
% which is used as a delimiter).
% Compilation is handed over to the new file by |\childdocforward|:
%    \begin{macrocode}
\newcommand{\childdocforwardprefix}[3][]
{
  \begingroup
    \def\childdocextract #2##1~~~{\def\childdoctmp{\childdocforward[#1]{#3##1}}}
    \expandafter\childdocextract\childdocname~~~
    \expandafter
  \endgroup
  \childdoctmp
}
%    \end{macrocode}

% \macro{\childdoc}
% The deprecated macro |\childdoc| is a legacy version of |\childdocmain|:
%    \begin{macrocode}
\newcommand{\childdoc}{\childdocmain}
%    \end{macrocode}

% \macro{\childdocredirect}
% The deprecated macro |\childdocredirect| is a legacy version
% of |\childdocforward| and |\childdocforwardprefix|:
%    \begin{macrocode}
\newcommand{\childdocredirect}[2][]
{
  \begingroup
    \if?#1?
      \def\childdoctmp{\childdocforward{#2}}
    \else
      \def\childdoctmp{\childdocforwardprefix{#1}{#2}}
    \fi
    \expandafter
  \endgroup
  \childdoctmp
}
%    \end{macrocode}

%\iffalse
%</package>
%\fi
%
\endinput

\childdocof{cdocsamp}
%    \end{macrocode}

%\iffalse
%</samplechap1|samplechap2>
%\fi
%
%\iffalse
%<*samplechap1>
%\fi
% Some text for chapter 1:
%    \begin{macrocode}
\section{one}
some text in chapter one
%    \end{macrocode}

%\iffalse
%</samplechap1>
%\fi
% Some text for chapter 2:
%\iffalse
%<*samplechap2>
%\fi
%    \begin{macrocode}
\section{two}
more text in chapter two
%    \end{macrocode}

%\iffalse
%</samplechap2>
%\fi
%
% %%%%%%%%%%%%%%%%%%%%%%%%%%%%%%%%%%%%%%
% \paragraph{Part Include Files.}
%
% The include files are called |cdocspt3.tex| and |cdocspt4.tex|.
%
%\iffalse
%<*samplepart3|samplepart4>
%\fi

% Optional override for |\version| flag:
%    \begin{macrocode}
%%\providecommand{\version}{final}
%    \end{macrocode}

% Include the main document:
%    \begin{macrocode}
% \iffalse
%
% childdoc.dtx Copyright (C) 2017-2018 Niklas Beisert
%
% This work may be distributed and/or modified under the
% conditions of the LaTeX Project Public License, either version 1.3
% of this license or (at your option) any later version.
% The latest version of this license is in
%   http://www.latex-project.org/lppl.txt
% and version 1.3 or later is part of all distributions of LaTeX
% version 2005/12/01 or later.
%
% This work has the LPPL maintenance status `maintained'.
%
% The Current Maintainer of this work is Niklas Beisert.
%
% This work consists of the files childdoc.dtx and childdoc.ins
% and the derived files childdoc.def and cdocsamp.tex with
% cdocsch1.tex, cdocsch2.tex, cdocsdrf.tex, cdocsfn1.tex, cdocsfn2.tex.
%
%<package>\ifdefined\childdocmain\endinput\fi
%<package>\ProvidesFile{childdoc.def}[2018/12/30 v2.0 child document driver]
%<samplemain>\ProvidesFile{cdocsamp.tex}[2018/12/30 v2.0 sample for childdoc]
%<*driver>
%\ProvidesFile{childdoc.drv}[2018/12/30 v2.0 childdoc reference manual file]
\PassOptionsToClass{10pt,a4paper}{article}
\documentclass{ltxdoc}

\usepackage[margin=35mm]{geometry}
\usepackage{hyperref}
\usepackage{hyperxmp}
\usepackage[usenames]{color}

\hypersetup{colorlinks=true}
\hypersetup{pdfstartview=FitH}
\hypersetup{pdfpagemode=UseNone}
\hypersetup{pdfsource={}}
\hypersetup{pdflang={en-UK}}
\hypersetup{pdfcopyright={Copyright 2017-2018 Niklas Beisert.
  This work may be distributed and/or modified under the
  conditions of the LaTeX Project Public License, either version 1.3
  of this license or (at your option) any later version.}}
\hypersetup{pdflicenseurl={http://www.latex-project.org/lppl.txt}}
\hypersetup{pdfcontactaddress={ETH Zurich, ITP, HIT K,
  Wolfgang-Pauli-Strasse 27}}
\hypersetup{pdfcontactpostcode={8093}}
\hypersetup{pdfcontactcity={Zurich}}
\hypersetup{pdfcontactcountry={Switzerland}}
\hypersetup{pdfcontactemail={nbeisert@itp.phys.ethz.ch}}
\hypersetup{pdfcontacturl={http://people.phys.ethz.ch/\xmptilde nbeisert/}}

\newcommand{\secref}[1]{\hyperref[#1]{section \ref*{#1}}}

\parskip1ex
\parindent0pt
\let\olditemize\itemize
\def\itemize{\olditemize\parskip0pt}

\begin{document}

\title{The \textsf{childdoc} Package}
\hypersetup{pdftitle={The childdoc Package}}
\author{Niklas Beisert\\[2ex]
  Institut f\"ur Theoretische Physik\\
  Eidgen\"ossische Technische Hochschule Z\"urich\\
  Wolfgang-Pauli-Strasse 27, 8093 Z\"urich, Switzerland\\[1ex]
  \href{mailto:nbeisert@itp.phys.ethz.ch}
  {\texttt{nbeisert@itp.phys.ethz.ch}}}
\hypersetup{pdfauthor={Niklas Beisert}}
\hypersetup{pdfsubject={Manual for the LaTeX2e Package childdoc}}
\date{30 December 2018, \textsf{v2.0}}
\maketitle

\begin{abstract}\noindent
\textsf{childdoc} is a \LaTeXe{} package
that enables the direct compilation
of document sections included by |\include|
to individual files.
\end{abstract}

\begingroup
\parskip0ex
\tableofcontents
\endgroup

%%%%%%%%%%%%%%%%%%%%%%%%%%%%%%%%%%%%%%%%%%%%%%%%%%%%%%%%%%%%%%%%%%%%%%%%%%%%%%%%
%%%%%%%%%%%%%%%%%%%%%%%%%%%%%%%%%%%%%%%%%%%%%%%%%%%%%%%%%%%%%%%%%%%%%%%%%%%%%%%%
\section{Introduction}

\LaTeX{} provides a mechanism to structure a large document (such as a book)
into a main file and several child files (containing the chapters)
using the |\include| command.
This mechanism is beneficial for documents
which span hundreds of pages in order to
make the source file(s) more manageable.
Moreover, compilation can be restricted to
selected child files by means of the |\includeonly| command.
The latter feature can be used to reduce the compilation time while editing
(this was significantly more useful in the earlier days of \LaTeX{})
or to generate a smaller document which is easier to navigate.
Another application of |\includeonly| is to generate
documents consisting of selected parts of the complete document.

However, there are a few drawbacks of the plain |\include| mechanism:
\begin{itemize}
\item
The child files cannot be compiled on their own,
they can only be compiled via the main file.
A naive editing environment
(such as a text editor with an option
to have the current file processed by \LaTeX)
may require one to switch to the main file before compiling;
attempting to compile the child file produces errors.
\item
The main file must be modified (each time)
to adjust the |\includeonly| command
to the present needs. This easily leaves the main file in a messy state.
\item
The generated document will always carry the filename
of the main document. This is inconvenient if
several child files are to be compiled and
to be kept for distribution.
\end{itemize}

The present package provides a simple interface
to make child files individually compilable by \LaTeX{}.
Compiling a child file then has the same effect as compiling
the main file with an |\includeonly| command
to select the appropriate child.
Moreover the generated document will carry the name of the child
rather than the main file.
This resolves all three above issues.

This feature is meant to make the editing of books,
thesis documents and lecture notes somewhat more convenient.
However, the package can also be used efficiently for
composing a series of documents (such as exercise sheets)
which are typically distributed individually.
It then assists the author in generating the individual documents
(potentially in different versions)
as well as a document containing the collected series.
Another application is in developing style files
or other kinds of included material
where compilation of the style file could redirect
to a sample or test file.

%%%%%%%%%%%%%%%%%%%%%%%%%%%%%%%%%%%%%%%%%%%%%%%%%%%%%%%%%%%%%%%%%%%%%%%%%%%%%%%%
%%%%%%%%%%%%%%%%%%%%%%%%%%%%%%%%%%%%%%%%%%%%%%%%%%%%%%%%%%%%%%%%%%%%%%%%%%%%%%%%
\section{Usage}

First of all, the package \textsf{childdoc} is \emph{not} a standard
\LaTeXe{} |.sty| style file! Therefore it needs to be invoked in
a non-standard way.

%%%%%%%%%%%%%%%%%%%%%%%%%%%%%%%%%%%%%%%%%%%%%%%%%%%%%%%%%%%%%%%%%%%%%%%%%%%%%%%%
\subsection{Included Files}
\label{sec:include}

%%%%%%%%%%%%%%%%%%%%%%%%%%%%%%%%%%%%%%%%
\DescribeMacro{\childdocmain}
To use the package, add the commands
\begin{center}
\begin{tabular}{l}
|\input{childdoc.def}|\\
|\childdocmain{}|\\
\end{tabular}
\end{center}
at the very top of the main \LaTeX{} file,
in particular \emph{before} the |\documentclass| statement!
The argument of |\childdocmain| should be left empty
(but it must be present).

%%%%%%%%%%%%%%%%%%%%%%%%%%%%%%%%%%%%%%%%
\DescribeMacro{\childdocof}
Furthermore, add the commands
\begin{center}
\begin{tabular}{l}
|\input{childdoc.def}|\\
|\childdocof{|\textit{main}|}|\\
\end{tabular}
\end{center}
at the top of every child file \textit{child}
which is included by |\include{|\textit{child}|}|
from within the main file
(or at least for those files to be compiled individually).
The argument \textit{main} must be the filename of the main file.

There are a couple of
considerations in setting up the main and child documents:

%%%%%%%%%%%%%%%%%%%%%%%%%%%%%%%%%%%%%%%%
\paragraph{Restrictions.}

Please note the following restrictions:
\begin{itemize}
\item
|\childdocmain| must be called with one argument \textit{main}
to ensure compatibility with earlier version of the package.
It must either be empty (|\childdocmain{}|)
or precisely match the filename of the main file in which it is specified.
See \secref{sec:detection} for further information.
\item
The filename \textit{main} must be specified without the |.tex| extension.
\item
The filename \textit{main} is case sensitive
(even in case-insensitive file systems)
due to internal string comparison.
\item
The argument \textit{main} should be fully expanded, it cannot be a macro.
\item
Subdirectories and special characters should be avoided in filenames.
\item
The command |\childdocmain{|\textit{main}|}| must be followed by a whitespace.
It should not be followed immediately by another command
or by a comment mark `|%|'.
This is because the \TeX{} parser reads the token immediately following
the argument of |\childdocmain| and puts it
at the beginning of every child section;
however, a white\-space is ignored.
\end{itemize}

%%%%%%%%%%%%%%%%%%%%%%%%%%%%%%%%%%%%%%%%
\paragraph{Content of Main File.}

It is advisable to place all content in the child files included by |\include|.
Any output contained in the main file will appear in all child documents
unless suppressed manually;
it cannot be suppressed automatically by the |\includeonly| directive
and thus should normally be avoided.
A method to include some content in the main file
by means of conditional processing is described in \secref{sec:conditional}.

%%%%%%%%%%%%%%%%%%%%%%%%%%%%%%%%%%%%%%%%
\paragraph{Page Numbering.}

When only a part of the document is compiled,
the appropriate numbering of pages
(as well as other status parameters)
is determined from the |.aux| files.
The latter contain information from previous passes.
However this information needs to propagate through
all intermediate child documents.
Therefore the page numbering in child documents may well
be inconsistent until the complete document is compiled at least once.

A useful (if unconventional) way to always ensure a consistent
page numbering is to restart the numbering in each child document
and denote the pages by `\textit{child}|.|\textit{page}'
where \textit{child} represents the chapter/section number of the child file.
This can be achieved by the command
|\numberwithin{page}{|\textit{child}|}|
of the \textsf{amsmath} package
where \textit{child} can be |chapter| or |section|
depending on the chosen structuring.
Alternatively, one can modify the macro |\thepage| appropriately
and reset the counter |page| at the start of each child file.

%%%%%%%%%%%%%%%%%%%%%%%%%%%%%%%%%%%%%%%%%%%%%%%%%%%%%%%%%%%%%%%%%%%%%%%%%%%%%%%%
\subsection{Conditional Processing}
\label{sec:conditional}

The package provides a mechanism to compile different versions
of a document. To customise the versions further some conditional processing
can come in handy to distinguish which version is being compiled.
The package provides two macros to describe the compilation context:

%%%%%%%%%%%%%%%%%%%%%%%%%%%%%%%%%%%%%%%%
\DescribeMacro{\ifchilddoc}
The conditional |\ifchilddoc| distinguishes between the compilation of
child documents and the main document:
%
\begin{center}
|\ifchilddoc |\textit{child-code}| |[|\||else |\textit{main-code}]| \||fi|
\end{center}

%%%%%%%%%%%%%%%%%%%%%%%%%%%%%%%%%%%%%%%%
\DescribeMacro{\childdocname}
\DescribeMacro{\childdocjob}
The macro |\childdocname| contains the filename (without extension)
of the main or child file being processed.
Note that |\childdocjob| will always contain the name of the main file.

%%%%%%%%%%%%%%%%%%%%%%%%%%%%%%%%%%%%%%%%
\paragraph{Title Page.}

Conditional processing can be used to include a title or banner page
in the main document when proper precautions are taken.
Importantly, the code in the main file should ensure that the page counter
(as well as other status parameters which are stored in the |.aux| files)
takes the same value after the conditional processing.
Otherwise the page numbers may take divergent values
depending on which part is compiled.

For example, a title page could be declared by:
%
\begin{center}
\begin{tabular}{l}
|\ifchilddoc\||else|\\
|\addtocounter{page}{-1}|\\
\textit{code for title page}\\
|\newpage|\\
|\||fi|
\end{tabular}
\end{center}
%
A banner page for the child documents can be generated by:
%
\begin{center}
\begin{tabular}{l}
|\ifchilddoc|\\
|\addtocounter{page}{-1}|\\
\textit{code for banner page}\\
|\newpage|\\
|\||fi|
\end{tabular}
\end{center}
%
Here one could write a message such as:
\begin{center}
|This is the part \childdocname{} of \childdocjob{}.|
\end{center}

%%%%%%%%%%%%%%%%%%%%%%%%%%%%%%%%%%%%%%%%%%%%%%%%%%%%%%%%%%%%%%%%%%%%%%%%%%%%%%%%
\subsection{Flags}
\label{sec:flags}

The package makes it easy to generate different versions
of the main or child documents.
To this end compilation flags can be defined
and assigned different default values.
They will be particularly useful in conjunction
with the forwarding mechanism described in \secref{sec:forward}.

For example, it may be useful to have a flag |\version|
which can be set to |draft| or |final|.
The document source will contain some conditional code
depending on the value of |\version|.
Suppose further, the flag should default to |final| for the main file
and to |draft| for child files
which is a natural assignment for editing the document.
This is achieved by placing the following code
in the preamble of the main document
(below the |\childdocmain| directive):
%
\begin{center}
\begin{tabular}{l}
|\ifchilddoc|\\
|\providecommand{\version}{draft}|\\
|\||else|\\
|\providecommand{\version}{final}|\\
|\||fi|
\end{tabular}
\end{center}
%
The definition by |\providecommand| makes sure
that previous definitions are not overwritten.
Further statements |\providecommand{\version}{...}|
can thus be added before the above code to override it.

For the main file, one might add a line
(between |\childdocmain| and the above block)
%
\begin{center}
|%\ifchilddoc\||else\providecommand{\version}{draft}\||fi|
\end{center}
%
which can be uncommented to produce a draft version.
Likewise one can add a line to the very top of a child file
(above the |\childdocof{|\textit{main}|}| directive)
%
\begin{center}
|%\providecommand{\version}{final}|
\end{center}
%
which can be uncommented to produce the final version of this child document.

%%%%%%%%%%%%%%%%%%%%%%%%%%%%%%%%%%%%%%%%%%%%%%%%%%%%%%%%%%%%%%%%%%%%%%%%%%%%%%%%
\subsection{Forwarding}
\label{sec:forward}

Different versions of the main or child documents
using compilation flags as described in \secref{sec:flags}
can be (permanently) stored in different files
for convenient compilation, viewing and distribution.
To this end, the package defines a command
to pass on compilation to a different file:

%%%%%%%%%%%%%%%%%%%%%%%%%%%%%%%%%%%%%%%%
\DescribeMacro{\childdocforward}
The command |\childdocforward| redirects processing to
another source file:
%
\begin{center}
\begin{tabular}{l}
|\input{childdoc.def}|\\
|\childdocforward[|\textit{main}|]{|\textit{dest}|}|\\
\end{tabular}
\end{center}
%
The argument \textit{dest} is the destination file
(without extension).
It should be the main file or one of the child files.
Note that further \textsf{childdoc} directives
such as |\childdocof| and |\childdocforward|
in the indicated file will be processed in this form.
The optional argument \textit{main}
passes on directly to the main file \textit{main}
while pretending to compile the child \textit{dest}.
This form behaves as if \textit{dest}
issues |\childdocof{|\textit{main}|}| right away,
and no further \textsf{childdoc} directives will be processed.

%%%%%%%%%%%%%%%%%%%%%%%%%%%%%%%%%%%%%%%%
\DescribeMacro{\...prefix}
In the alternative form |\childdocforwardprefix|,
%
\begin{center}
\begin{tabular}{l}
|\input{childdoc.def}|\\
|\childdocforwardprefix[|\textit{main}|]{|\textit{prefix}|}{|\textit{dest}|}|
\end{tabular}
\end{center}
%
the destination file is determined by a pattern
depending on the current file:
To make this work, the current file must be called
`{\textit{prefix}\hspace{0.2em}\textit{suffix}}'
with \textit{prefix} matching precisely the argument.
Processing is then passed on to the file
`{\textit{dest}\hspace{0.2em}\textit{suffix}}'.
Surely, the same effect is achieved by
directly specifying the
argument `{\textit{dest}\hspace{0.2em}\textit{suffix}}'
in the first form.
However, that requires to set up a different file
for each child. With the alternative form of the command
all these files can have exactly the same content
which simplifies setting them up and maintaining them.

For example, the following file |draft.tex|
with a compilation flag |\version| as described in \secref{sec:flags}
compiles the main document as a draft:
%
\begin{center}
\begin{tabular}{l}
|\def\version{draft}|\\
|\input{childdoc.def}|\\
|\childdocforward{|\textit{main}|}|
\end{tabular}
\end{center}
%
Likewise, the following files |final|\textit{nn}|.tex|
compile the final version of the child document
|child|\textit{nn}|.tex|:
%
\begin{center}
\begin{tabular}{l}
|\def\version{final}|\\
|\input{childdoc.def}|\\
|\childdocforwardprefix{final}{child}|
\end{tabular}
\end{center}
%

Note that when several versions of a main file and/or of each child file
are to be generated, it may be convenient to set up a |Makefile| or
shell script to automatise the process.

%%%%%%%%%%%%%%%%%%%%%%%%%%%%%%%%%%%%%%%%%%%%%%%%%%%%%%%%%%%%%%%%%%%%%%%%%%%%%%%%
\subsection{Command Line Processing}
\label{sec:commandline}

The effect of redirection files can also be achieved by invoking
the \LaTeX{} compiler with a more elaborate command line.
Most conveniently this should be done as part
of a shell script or a |Makefile|.

When using \textsf{childdoc} in the main file, the following
command lines effectively perform a redirection
(note that depending on the shell being used,
backslashes may have to be doubled: `|\|' $\to$ `|\\|'):
%
\begin{center}
|... -jobname "|\textit{target}|" |\\|"|[\textit{flags}]%
|\input{childdoc.def}\childdocforward[|\textit{main}|]{|\textit{dest}|}"|
\end{center}
%
Here \textit{target} is the name of the output file,
\textit{main} is the name of the main file
and \textit{dest} is the name of the main or child file to be processed
(all filenames without extensions).
The optional argument \textit{main} can be omitted
if \textit{main} matches \textit{dest}.
Optionally, compilation \textit{flags} can be defined via |\def| commands.
This command line makes the \TeX{} engine believe
it is compiling the file \textit{target}
whose content is specified as the latter parameter.
The provided code then forwards the processing to
\textit{main} or \textit{dest} as described in \secref{sec:forward}.

%%%%%%%%%%%%%%%%%%%%%%%%%%%%%%%%%%%%%%%%%%%%%%%%%%%%%%%%%%%%%%%%%%%%%%%%%%%%%%%%
\subsection{Include by Input}
\label{sec:input}

Including child documents by |\include| has some restrictions by design.
Most notably, the content of a child document always occupies
its own set of pages; pages cannot be shared between child documents.
Usually, this behaviour makes perfect sense
because each child document contain an essential part of the document.
However, in some situations it may be desirable to compose
a document from a collection of parts
without having mandatory page breaks between then.
For this case, the package
provides a mechanism to include parts
by |\input| which can also be processed individually.
However, by construction this mechanism
requires manual handling of the content to be output.

%%%%%%%%%%%%%%%%%%%%%%%%%%%%%%%%%%%%%%%%
\DescribeMacro{\ifchilddocmanual}
The main file should be prepared as usual, see \secref{sec:include}.
However, the document body must make a distinction
between processing of an individual part and of the main document, e.g.:
%
\begin{center}
\begin{tabular}{l}
|\ifchilddocmanual|\\
|\input{\childdocname}|\\
|\||else|\\
\textit{document body with }|\input{|\textit{part}|}|\\
|\||fi|
\end{tabular}
\end{center}
%
The conditional |\ifchilddocmanual| is true whenever
a part to be included by |\input| is being compiled,
and the name of the part is stored in |\childdocname|.

%%%%%%%%%%%%%%%%%%%%%%%%%%%%%%%%%%%%%%%%
\DescribeMacro{\childdocby}
Each part to be included by |\input| should start with:
%
\begin{center}
\begin{tabular}{l}
|\input{childdoc.def}|\\
|\childdocby{|\textit{main}|}|\\
\end{tabular}
\end{center}
%
The directive |\childdocby| is similar to |\childdocof|
described in \secref{sec:include},
but the subsequent selection of content must be done manually.
To that end, both |\ifchilddoc| and |\ifchilddocmanual|
will be true upon processing of a part,
and the name of the part is stored in |\childdocname|.
Note that |\jobname| will be set to the filename of the current part
so that each part receives an individual |.aux| file
that does not interfere with the |.aux| file(s) of the main document.
This behaviour can be altered by the alternative form
|\childdocby[*]{|\textit{main}|}| (with a non-empty optional argument)
which uses the |.aux| file of the main document
by setting |\jobname| to \textit{main}.

%%%%%%%%%%%%%%%%%%%%%%%%%%%%%%%%%%%%%%%%%%%%%%%%%%%%%%%%%%%%%%%%%%%%%%%%%%%%%%%%
\subsection{Driver Development}
\label{sec:driver}

The \textsf{childdoc} mechanism can also be use for the development
of definition files such as \LaTeX{} styles or classes.
This case differs from the above setup with multiple parts
included by |\include| in that no |\includeonly| should be invoked.
This can be achieved by starting the include file
(before |\ProvidesPackage|) with:
%
\begin{center}
\begin{tabular}{l}
|\input{childdoc.def}|\\
|\childdocforward{|\textit{main}|}|\\
\end{tabular}
\end{center}
%
or alternatively with:
%
\begin{center}
\begin{tabular}{l}
|\input{childdoc.def}|\\
|\childdocby{|\textit{main}|}|\\
\end{tabular}
\end{center}
%
Both forms have slightly different effects as described above.
The main file is prepared as usual, see \secref{sec:include}.

%%%%%%%%%%%%%%%%%%%%%%%%%%%%%%%%%%%%%%%%%%%%%%%%%%%%%%%%%%%%%%%%%%%%%%%%%%%%%%%%
\subsection{Legacy Detection}
\label{sec:detection}

The directive |\childdocmain| in the main file can detect
whether the complete document or merely a child is to be compiled
even without using the directive |\childdocof|.
This method is deprecated because it is less robust
and there is no compelling reason to use it;
it is merely provided for backward compatibility
and it may be removed in future versions.

If the detection mechanism is to be used,
it is mandatory to correctly specify
the filename of the main file as the argument of |\childdocmain|:
%
\begin{center}
\begin{tabular}{l}
|\input{childdoc.def}|\\
|\childdocmain{|\textit{main}|}|\\
\end{tabular}
\end{center}
%
If |\jobname| does not match the argument \textit{main} of |\childdocmain|,
it is assumed that |\jobname| points to the child file to be compiled.
When using |\childdocmain| with the main file specified as argument,
it suffices to start a child file
with just |\input{|\textit{main}|}|
without loading of the package and using |\childdocof|.
If instead all processing is done
with the appropriate \textsf{childdoc} directives,
the argument of \textit{main} of |\childdocmain| can be empty.

An alternative version of the command line processing described
in \secref{sec:commandline} using the detection mechanism reads:
%
\begin{center}
|... -jobname "|\textit{target}|" "|[\textit{flags}]%
[|\def\jobname{|\textit{dest}|}|]|\input{|\textit{main}|}"|
\end{center}

%%%%%%%%%%%%%%%%%%%%%%%%%%%%%%%%%%%%%%%%%%%%%%%%%%%%%%%%%%%%%%%%%%%%%%%%%%%%%%%%
\subsection{Manual Code}
\label{sec:manual}

In case one cannot be certain whether the definitions file |childdoc.def|
is installed on the target \TeX{} distribution
and one prefers not to ship it,
it is conceivable to paste a few relevant commands into the sources.

To that end, drop all statements |\input{childdoc.def}|
and perform the replacements as outlined below.
Instead of |\childdocmain{|\textit{main}|}| add the following code
to the top of the main file:
%
\begin{center}
\begin{tabular}{l}
|\||ifdefined\childdocname\endinput\||fi\newif\ifchilddoc|\\
|\edef\childdocname{\scantokens\expandafter{\jobname\noexpand}}|\\
|\def\childdocmain{|\textit{main}|}\||ifx\childdocmain\childdocname\||else|\\
|\childdoctrue\includeonly{\childdocname}\let\jobname\childdocmain\||fi|\\
\end{tabular}
\end{center}
%
Instead of |\childdocof{|\textit{main}|}| just include the main file
at the top of each child file:
%
\begin{center}
|\input{|\textit{main}|}|
\end{center}
%
A simple redirection |\childdocforward{|\textit{dest}|}| is achieved by:
%
\begin{center}
|\def\jobname{|\textit{dest}|}\input{\jobname}|
\end{center}
%
The redirection with prefix
|\childdocforwardprefix[|\textit{prefix}|]{|\textit{dest}|}|
is accomplished by:
%
\begin{center}
\begin{tabular}{l}
|{\edef\jobname{\scantokens\expandafter{\jobname\noexpand}}|\\
|\def\redirectjob |\textit{prefix}|#1~~~{\gdef\jobname{|\textit{dest}|#1}}|\\
|\expandafter\redirectjob\jobname~~~}\input{\jobname}|
\end{tabular}
\end{center}

In an alternative approach,
child documents can be compiled by a specific command line
without additional code or specific definitions:
%
\begin{center}
|... -jobname "|\textit{target}|" "|[\textit{flags}]%
|\includeonly{|\textit{dest}|}\input{|\textit{main}|}"|
\end{center}
%

%%%%%%%%%%%%%%%%%%%%%%%%%%%%%%%%%%%%%%%%%%%%%%%%%%%%%%%%%%%%%%%%%%%%%%%%%%%%%%%%
%%%%%%%%%%%%%%%%%%%%%%%%%%%%%%%%%%%%%%%%%%%%%%%%%%%%%%%%%%%%%%%%%%%%%%%%%%%%%%%%
\section{Information}

%%%%%%%%%%%%%%%%%%%%%%%%%%%%%%%%%%%%%%%%%%%%%%%%%%%%%%%%%%%%%%%%%%%%%%%%%%%%%%%%
\subsection{Copyright}

Copyright \copyright{} 2017--2018 Niklas Beisert

This work may be distributed and/or modified under the
conditions of the \LaTeX{} Project Public License, either version 1.3
of this license or (at your option) any later version.
The latest version of this license is in
  \url{http://www.latex-project.org/lppl.txt}
and version 1.3 or later is part of all distributions of \LaTeX{}
version 2005/12/01 or later.

This work has the LPPL maintenance status `maintained'.

The Current Maintainer of this work is Niklas Beisert.

This work consists of the files |README.txt|, |childdoc.ins| and |childdoc.dtx|
as well as the derived files |childdoc.def|, |cdocsamp.tex|
with |cdocsch1.tex|, |cdocsch2.tex|, |cdocspt3.tex|, |cdocspt4.tex|,
|cdocsdrf.tex|, |cdocsfn1.tex|, |cdocsfn2.tex|
as well as |childdoc.pdf|.

%%%%%%%%%%%%%%%%%%%%%%%%%%%%%%%%%%%%%%%%%%%%%%%%%%%%%%%%%%%%%%%%%%%%%%%%%%%%%%%%
\subsection{Files and Installation}

The package consists of the files:
%
\begin{center}
\begin{tabular}{ll}
    |README.txt|   & readme file \\
    |childdoc.ins| & installation file \\
    |childdoc.dtx| & source file \\
    |childdoc.def| & definition file \\
    |cdocsamp.tex| & sample main file \\
    |cdocsch1.tex| & sample include file \\
    |cdocsch2.tex| & sample include file \\
    |cdocspt3.tex| & sample part file \\
    |cdocspt4.tex| & sample part file \\
    |cdocsdrf.tex| & sample redirection file \\
    |cdocsfn1.tex| & sample redirection file \\
    |cdocsfn2.tex| & sample redirection file \\
    |childdoc.pdf| & manual
\end{tabular}
\end{center}
%
The distribution consists of the files
|README.txt|, |childdoc.ins| and |childdoc.dtx|.
%
\begin{itemize}
\item
Run (pdf)\LaTeX{} on |childdoc.dtx|
to compile the manual |childdoc.pdf| (this file).
\item
Run \LaTeX{} on |childdoc.ins| to create the definitions file |childdoc.def|
and the sample |cdocsamp.tex| with include files
|cdocsch1.tex|, |cdocsch2.tex|, |cdocspt3.tex|, |cdocspt4.tex|,
|cdocsdrf.tex|, |cdocsfn1.tex|, |cdocsfn2.tex|.
Then copy the file |childdoc.def| to an appropriate directory of your \LaTeX{}
distribution, e.g.\ \textit{texmf-root}|/tex/latex/childdoc|.
\end{itemize}

%%%%%%%%%%%%%%%%%%%%%%%%%%%%%%%%%%%%%%%%%%%%%%%%%%%%%%%%%%%%%%%%%%%%%%%%%%%%%%%%
\subsection{Related CTAN Packages}

There are several other packages which offer a similar functionality:
%
\begin{itemize}
\item
The packages
\href{http://ctan.org/pkg/docmute}{\textsf{docmute}},
\href{http://ctan.org/pkg/includex}{\textsf{includex}} and
\href{http://ctan.org/pkg/standalone}{\textsf{standalone}}
provide commands to include only the document body of
a child file thus allowing both files to be compiled individually.
\item
The packages \href{http://ctan.org/pkg/subdocs}{\textsf{subdocs}}
and \href{http://ctan.org/pkg/subfiles}{\textsf{subfiles}}
provide structures in which the main and child documents can be
encapsulated and allowing them to be compiled individually.
The inclusion mechanism is different from the conventional |\include|.
\item
The package \href{http://ctan.org/pkg/combine}{\textsf{combine}}
is an elaborate solution to combine several documents into one.
\end{itemize}
%
See also the CTAN topic \href{http://ctan.org/topic/subdocs}{\textsf{subdocs}}
for further related packages.
The present package differs from the above solutions in that
a document structure constructed with the conventional |\include| mechanism
just needs two extra commands at the top of every file
such that all constituent files can be compiled individually.

%%%%%%%%%%%%%%%%%%%%%%%%%%%%%%%%%%%%%%%%%%%%%%%%%%%%%%%%%%%%%%%%%%%%%%%%%%%%%%%%
%\subsection{Feature Suggestions}
%
%The following is a list of features which may be useful for future
%versions of this package:
%%
%\begin{itemize}
%\item
%\ldots
%\end{itemize}

%%%%%%%%%%%%%%%%%%%%%%%%%%%%%%%%%%%%%%%%%%%%%%%%%%%%%%%%%%%%%%%%%%%%%%%%%%%%%%%%
\subsection{Revision History}

%%%%%%%%%%%%%%%%%%%%%%%%%%%%%%%%%%%%%%%%
\paragraph{v2.0:} 2018/12/30

\begin{itemize}
\item
immediate forward processing
\item
added |\childdocby| mechanism
\item
manual restructured
\end{itemize}

%%%%%%%%%%%%%%%%%%%%%%%%%%%%%%%%%%%%%%%%
\paragraph{v1.6:} 2018/01/17

\begin{itemize}
\item
application for development of include files
\item
corrections to manual
\end{itemize}

%%%%%%%%%%%%%%%%%%%%%%%%%%%%%%%%%%%%%%%%
\paragraph{v1.5:} 2017/05/21

\begin{itemize}
\item
more complete structuring introduced
\item
|\childdocof| introduced
\item
|\childdoc| renamed to |\childdocmain|
\item
|\childredirect| renamed to |\childdocforward| and |\childdocforwardprefix|
and functionality expanded
\end{itemize}

%%%%%%%%%%%%%%%%%%%%%%%%%%%%%%%%%%%%%%%%
\paragraph{v1.0:} 2017/04/27

\begin{itemize}
\item
manual and install package
\item
first version published on CTAN
\end{itemize}

%%%%%%%%%%%%%%%%%%%%%%%%%%%%%%%%%%%%%%%%
\paragraph{v0.6:} 2017/04/26

\begin{itemize}
\item
redirection mechanism added
\end{itemize}

%%%%%%%%%%%%%%%%%%%%%%%%%%%%%%%%%%%%%%%%
\paragraph{v0.5:} 2017/04/26

\begin{itemize}
\item
functionality in definition file
\end{itemize}


%%%%%%%%%%%%%%%%%%%%%%%%%%%%%%%%%%%%%%%%%%%%%%%%%%%%%%%%%%%%%%%%%%%%%%%%%%%%%%%%
%%%%%%%%%%%%%%%%%%%%%%%%%%%%%%%%%%%%%%%%%%%%%%%%%%%%%%%%%%%%%%%%%%%%%%%%%%%%%%%%
%%%%%%%%%%%%%%%%%%%%%%%%%%%%%%%%%%%%%%%%%%%%%%%%%%%%%%%%%%%%%%%%%%%%%%%%%%%%%%%%
\appendix

\settowidth\MacroIndent{\rmfamily\scriptsize 000\ }

 \DocInput{childdoc.dtx}

\end{document}
%</driver>
% \fi
%
% %%%%%%%%%%%%%%%%%%%%%%%%%%%%%%%%%%%%%%%%%%%%%%%%%%%%%%%%%%%%%%%%%%%%%%%%%%%%%%
% %%%%%%%%%%%%%%%%%%%%%%%%%%%%%%%%%%%%%%%%%%%%%%%%%%%%%%%%%%%%%%%%%%%%%%%%%%%%%%
% \section{Sample}
%\iffalse
%<*samplemain>
%\fi
%
% The following presents a sample document
% with two chapters, two parts, a title page,
% a compile flag as well as three forwarding files to set the flag.
% It consists of eight |.tex| files:
% \begin{center}
% \begin{tabular}{ll}
% |cdocsamp.tex|&main file\\
% |cdocsch1.tex|&include file for chapter 1\\
% |cdocsch2.tex|&include file for chapter 2\\
% |cdocspt3.tex|&include file for part 3\\
% |cdocspt4.tex|&include file for part 4\\
% |cdocsdrf.tex|&forwarding file for main file in draft mode\\
% |cdocsfi1.tex|&forwarding file for final version of chapter 1\\
% |cdocsfi2.tex|&forwarding file for final version of chapter 2\\
% \end{tabular}
% \end{center}
% Each of the eight files can be compiled directly by the \LaTeX{} compiler.
%
% %%%%%%%%%%%%%%%%%%%%%%%%%%%%%%%%%%%%%%
% \paragraph{Main File.}
%
% The main file is called |cdocsamp.tex|.
%
% Load the \textsf{childdoc} definitions and
% declare the filename for the main document:
%    \begin{macrocode}
\input{childdoc.def}
\childdocmain{}
%    \end{macrocode}

% Optional override for |\version| flag:
%    \begin{macrocode}
%%\ifchilddoc\else\providecommand{\version}{draft}\fi
%    \end{macrocode}

% Define the default values for the |\version| flag
% (|final| for the main file and |draft| for childs):
%    \begin{macrocode}
\ifchilddoc
\providecommand{\version}{draft}
\else
\providecommand{\version}{final}
\fi
%    \end{macrocode}

% Load the standard document class:
%    \begin{macrocode}
\documentclass[12pt]{article}
%    \end{macrocode}

% Start the document body:
%    \begin{macrocode}
\begin{document}
%    \end{macrocode}

% Declare a title page.
% Print title, part of document being processed and version flag:
%    \begin{macrocode}
\addtocounter{page}{-1}
\begin{center}
{\LARGE\bfseries{}childdoc example\par}
\vspace{1cm}
\ifchilddoc
\ifchilddocmanual part\else chapter\fi:
`\childdocname' of `\childdocjob'\par
\else
main document: `\childdocjob'\par
\fi
version: \version\par
\end{center}
\newpage
%    \end{macrocode}

% Manually include selected file,
% otherwise process as usual:
%    \begin{macrocode}
\ifchilddocmanual
\section*{part `\childdocname'}
\input{\childdocname}
\else
%    \end{macrocode}

% Include the two chapters:
%    \begin{macrocode}
\include{cdocsch1}
\include{cdocsch2}
%    \end{macrocode}

% Include the two parts unless only chapters should be displayed:
%    \begin{macrocode}
\ifchilddoc\else
\section{part three}
\input{cdocspt3}
\section{part four}
\input{cdocspt4}
\fi
%    \end{macrocode}

% Process as usual until here:
%    \begin{macrocode}
\fi
%    \end{macrocode}

% End of document body:
%    \begin{macrocode}
\end{document}
%    \end{macrocode}
%\iffalse
%</samplemain>
%\fi
%
% %%%%%%%%%%%%%%%%%%%%%%%%%%%%%%%%%%%%%%
% \paragraph{Chapter Include Files.}
%
% The include files are called |cdocsch1.tex| and |cdocsch2.tex|.
%
%\iffalse
%<*samplechap1|samplechap2>
%\fi

% Optional override for |\version| flag:
%    \begin{macrocode}
%%\providecommand{\version}{final}
%    \end{macrocode}

% Include the main document:
%    \begin{macrocode}
\input{childdoc.def}
\childdocof{cdocsamp}
%    \end{macrocode}

%\iffalse
%</samplechap1|samplechap2>
%\fi
%
%\iffalse
%<*samplechap1>
%\fi
% Some text for chapter 1:
%    \begin{macrocode}
\section{one}
some text in chapter one
%    \end{macrocode}

%\iffalse
%</samplechap1>
%\fi
% Some text for chapter 2:
%\iffalse
%<*samplechap2>
%\fi
%    \begin{macrocode}
\section{two}
more text in chapter two
%    \end{macrocode}

%\iffalse
%</samplechap2>
%\fi
%
% %%%%%%%%%%%%%%%%%%%%%%%%%%%%%%%%%%%%%%
% \paragraph{Part Include Files.}
%
% The include files are called |cdocspt3.tex| and |cdocspt4.tex|.
%
%\iffalse
%<*samplepart3|samplepart4>
%\fi

% Optional override for |\version| flag:
%    \begin{macrocode}
%%\providecommand{\version}{final}
%    \end{macrocode}

% Include the main document:
%    \begin{macrocode}
\input{childdoc.def}
\childdocby{cdocsamp}
%    \end{macrocode}

%\iffalse
%</samplepart3|samplepart4>
%\fi
%
%\iffalse
%<*samplepart3>
%\fi
% Some text for part 3:
%    \begin{macrocode}
some text in part three
%    \end{macrocode}

%\iffalse
%</samplepart3>
%\fi
% Some text for part 4:
%\iffalse
%<*samplepart4>
%\fi
%    \begin{macrocode}
more text in part four
%    \end{macrocode}

%\iffalse
%</samplepart4>
%\fi
%
% %%%%%%%%%%%%%%%%%%%%%%%%%%%%%%%%%%%%%%
% \paragraph{Forwarding for a Complete Draft.}
%
% The following forwarding file |cdocsdrf.tex|
% compiles the main document in draft mode:
%\iffalse
%<*sampledraft>
%\fi
%    \begin{macrocode}
\def\version{draft}
\input{childdoc.def}
\childdocforward{cdocsamp}
%    \end{macrocode}

%\iffalse
%</sampledraft>
%\fi
%
% %%%%%%%%%%%%%%%%%%%%%%%%%%%%%%%%%%%%%%
% \paragraph{Forwarding for Final Version of the Chapters.}
%
% The following forwarding files |cdocsfn1.tex| and |cdocsfn2.tex|
% (with identical content)
% compile the final versions of the child documents
% |cdocsch1.tex| and |cdocsch2.tex|, respectively:
%\iffalse
%<*samplefinal>
%\fi
%    \begin{macrocode}
\def\version{final}
\input{childdoc.def}
\childdocforwardprefix[cdocsamp]{cdocsfn}{cdocsch}
%    \end{macrocode}

%\iffalse
%</samplefinal>
%\fi
%
% %%%%%%%%%%%%%%%%%%%%%%%%%%%%%%%%%%%%%%
% \paragraph{Command Line Processing.}
%
% The following three command lines generate the output files
% |cdocscld|, |cdocscl1| and |cdocscl2|
% which should be identical to
% |cdocsdrf|, |cdocsch1| and |cdocsfn2|, respectively:
% \begin{center}
% \begin{tabular}{l}
% |latex -jobname cdocscld \|\\
% |  "\def\version{draft}\input{childdoc.def}\childdocforward{cdocsamp}"|\\
% |latex -jobname cdocscl1 \|\\
% |  "\input{childdoc.def}\childdocforward[cdocsamp]{cdocsch1}"|\\
% |latex -jobname cdocscl2 \|\\
% |  "\def\version{final}\input{childdoc.def}\childdocforward{cdocsch2}"|
% \end{tabular}
% \end{center}
% Note that the trailing backslash on each first line
% merely continues the input to the second line
% (for convenient cut ant paste).
% Furthermore, the command |latex| can be replaced by any
% of its alternative versions such as |pdflatex|.
%
% %%%%%%%%%%%%%%%%%%%%%%%%%%%%%%%%%%%%%%%%%%%%%%%%%%%%%%%%%%%%%%%%%%%%%%%%%%%%%%
% %%%%%%%%%%%%%%%%%%%%%%%%%%%%%%%%%%%%%%%%%%%%%%%%%%%%%%%%%%%%%%%%%%%%%%%%%%%%%%
% \section{Implementation}
%\iffalse
%<*package>
%\fi
%
% This section describes the definitions file |childdoc.def|.

% The definitions cannot be loaded using |\usepackage| or |\RequirePackage|
% which has a mechanism to prevent loading a style file more than once.
% When loading the definitions by means of |\input|
% multiple instances have to be prevented manually:
%\iffalse
%This code needs to be before the `\ProvidesFile' directive
%which is defined at the beginning of this file.
%Therefore it is also placed there and commented out here.
%</package>
%<*discard>
%\fi
%    \begin{macrocode}
\ifdefined\childdocmain\endinput\fi
%    \end{macrocode}
%\iffalse
%</discard>
%<*package>
%\fi
%
% \macro{\ifchilddoc}
% \macro{\ifchilddocmanual}
% The conditional |\ifchilddoc| tells whether a
% child (true) or main (false) document is being compiled.
% The conditional |\ifchilddocmanual| tells whether
% the |\includeonly| mechanism is used (false) or
% the selection of child files must be performed manually (true).
% The definitions initialise to false:
%    \begin{macrocode}
\newif\ifchilddoc
\newif\ifchilddocmanual
%    \end{macrocode}

% \macro{\childdocname}
% \macro{\childdocjob}
% The macro |\childdocname| stores the name of the main document
% to be compiled. The macro |\childdocjob| stores the name of
% the document on which the \LaTeX{} compiler was originally invoked.
% The content of |\jobname| cannot be compared
% to filenames specified in the source due to different catcodes.
% The following code rescans |\jobname|, stores the result
% in |\childdocname| and saves a copy in |\childdocjob|:
%    \begin{macrocode}
\edef\childdocname{\scantokens\expandafter{\jobname\noexpand}}
\let\childdocjob\childdocname
%    \end{macrocode}

% \macro{\childdocdisable}
% The macro |\childdocdisable| prevents the main file
% from being processed more than once.
% At this stage, the main document command |\childdocmain|
% is assumed to be called once again where it should do nothing.
% Any subsequent call to it should prevent
% a secondary processing of the main document
% It overwrites the forwarding commands
% |\childdocof| and |\childdocforward|
% with empty macros to prevent further inclusions of the main document:
%    \begin{macrocode}
\newcommand{\childdocdisable}
{
  \renewcommand{\childdocmain}[1]{\renewcommand{\childdocmain}[1]{\endinput}}
  \renewcommand{\childdocof}[1]{}
  \renewcommand{\childdocby}[2][]{}
  \renewcommand{\childdocforward}[2][]{}
  \renewcommand{\childdocdisable}{}
}
%    \end{macrocode}

% \macro{\childdocmain}
% The macro |\childdocmain| is to be called at the top of the main file
% with nothing or the main filename (without extension) as argument.
% First, it breaks loops.
% If the argument is not empty and does not match |\childdocname|
% (which is set by the first inclusion of |childdoc.def|),
% |\ifchilddoc| is set to true, |\includeonly| is applied to the child file
% and |\jobname| is set to the main file
% (for proper handling of |.aux| files):
%    \begin{macrocode}
\newcommand{\childdocmain}[1]
{
  \childdocdisable\childdocmain{}
  \if?#1?\else
    \begingroup
      \def\childdoctmp{#1}
      \ifx\childdoctmp\childdocname
        \def\childdoctmp{}
      \else
        \def\childdoctmp
        {
          \childdoctrue
          \includeonly{\childdocname}
          \def\childdocjob{#1}
          \def\jobname{#1}
        }
      \fi
      \expandafter
    \endgroup
    \childdoctmp
  \fi
}
%    \end{macrocode}

% \macro{\childdocof}
% The command |\childdocof| redirects
% compilation to the main file |#1|.
%    \begin{macrocode}
\newcommand{\childdocof}[1]
{
  \childdocdisable
  \childdoctrue
  \includeonly{\childdocname}
  \def\jobname{#1}
  \def\childdocjob{#1}
  \input{#1}
}
%    \end{macrocode}

% \macro{\childdocby}
% The command |\childdocby| ....
%    \begin{macrocode}
\newcommand{\childdocby}[2][]
{
  \childdocdisable
  \childdoctrue
  \childdocmanualtrue
  \if?#1?\else
    \def\jobname{#2}
  \fi
  \def\childdocjob{#2}
  \input{#2}
  \endinput
}
%    \end{macrocode}

% \macro{\childdocforward}
% The command |\childdocforward| redirects
% compilation to the main file or
% (if the optional argument is given) a child file.
% Parameters are set as if the main file
% or a child file starting with |\childdocof| was compiled.
% Then compilation is handed over to the main file:
%    \begin{macrocode}
\newcommand{\childdocforward}[2][]
{
  \begingroup
    \if?#1?
      \def\childdoctmp
      {
        \def\childdocname{#2}
        \def\childdocjob{#2}
        \def\jobname{#2}
        \input{#2}
        \endinput
      }
    \else
      \def\childdoctmp
      {
        \childdocdisable
        \def\childdocname{#2}
        \childdoctrue
        \includeonly{#2}
        \def\childdocjob{#1}
        \def\jobname{#1}
        \input{#1}
        \endinput
      }
    \fi
    \expandafter
  \endgroup
  \childdoctmp
}
%    \end{macrocode}

% \macro{\childdocforwardprefix}
% The command |\childdocforwardprefix| redirects
% compilation to the main or a child file by means of a pattern.
% The prefix |#1| in the current filename is replaced by |#2|
% and the suffix of the current filename is kept
% (it is assumed that the filename does not contain the substring `|~~~|'
% which is used as a delimiter).
% Compilation is handed over to the new file by |\childdocforward|:
%    \begin{macrocode}
\newcommand{\childdocforwardprefix}[3][]
{
  \begingroup
    \def\childdocextract #2##1~~~{\def\childdoctmp{\childdocforward[#1]{#3##1}}}
    \expandafter\childdocextract\childdocname~~~
    \expandafter
  \endgroup
  \childdoctmp
}
%    \end{macrocode}

% \macro{\childdoc}
% The deprecated macro |\childdoc| is a legacy version of |\childdocmain|:
%    \begin{macrocode}
\newcommand{\childdoc}{\childdocmain}
%    \end{macrocode}

% \macro{\childdocredirect}
% The deprecated macro |\childdocredirect| is a legacy version
% of |\childdocforward| and |\childdocforwardprefix|:
%    \begin{macrocode}
\newcommand{\childdocredirect}[2][]
{
  \begingroup
    \if?#1?
      \def\childdoctmp{\childdocforward{#2}}
    \else
      \def\childdoctmp{\childdocforwardprefix{#1}{#2}}
    \fi
    \expandafter
  \endgroup
  \childdoctmp
}
%    \end{macrocode}

%\iffalse
%</package>
%\fi
%
\endinput

\childdocby{cdocsamp}
%    \end{macrocode}

%\iffalse
%</samplepart3|samplepart4>
%\fi
%
%\iffalse
%<*samplepart3>
%\fi
% Some text for part 3:
%    \begin{macrocode}
some text in part three
%    \end{macrocode}

%\iffalse
%</samplepart3>
%\fi
% Some text for part 4:
%\iffalse
%<*samplepart4>
%\fi
%    \begin{macrocode}
more text in part four
%    \end{macrocode}

%\iffalse
%</samplepart4>
%\fi
%
% %%%%%%%%%%%%%%%%%%%%%%%%%%%%%%%%%%%%%%
% \paragraph{Forwarding for a Complete Draft.}
%
% The following forwarding file |cdocsdrf.tex|
% compiles the main document in draft mode:
%\iffalse
%<*sampledraft>
%\fi
%    \begin{macrocode}
\def\version{draft}
% \iffalse
%
% childdoc.dtx Copyright (C) 2017-2018 Niklas Beisert
%
% This work may be distributed and/or modified under the
% conditions of the LaTeX Project Public License, either version 1.3
% of this license or (at your option) any later version.
% The latest version of this license is in
%   http://www.latex-project.org/lppl.txt
% and version 1.3 or later is part of all distributions of LaTeX
% version 2005/12/01 or later.
%
% This work has the LPPL maintenance status `maintained'.
%
% The Current Maintainer of this work is Niklas Beisert.
%
% This work consists of the files childdoc.dtx and childdoc.ins
% and the derived files childdoc.def and cdocsamp.tex with
% cdocsch1.tex, cdocsch2.tex, cdocsdrf.tex, cdocsfn1.tex, cdocsfn2.tex.
%
%<package>\ifdefined\childdocmain\endinput\fi
%<package>\ProvidesFile{childdoc.def}[2018/12/30 v2.0 child document driver]
%<samplemain>\ProvidesFile{cdocsamp.tex}[2018/12/30 v2.0 sample for childdoc]
%<*driver>
%\ProvidesFile{childdoc.drv}[2018/12/30 v2.0 childdoc reference manual file]
\PassOptionsToClass{10pt,a4paper}{article}
\documentclass{ltxdoc}

\usepackage[margin=35mm]{geometry}
\usepackage{hyperref}
\usepackage{hyperxmp}
\usepackage[usenames]{color}

\hypersetup{colorlinks=true}
\hypersetup{pdfstartview=FitH}
\hypersetup{pdfpagemode=UseNone}
\hypersetup{pdfsource={}}
\hypersetup{pdflang={en-UK}}
\hypersetup{pdfcopyright={Copyright 2017-2018 Niklas Beisert.
  This work may be distributed and/or modified under the
  conditions of the LaTeX Project Public License, either version 1.3
  of this license or (at your option) any later version.}}
\hypersetup{pdflicenseurl={http://www.latex-project.org/lppl.txt}}
\hypersetup{pdfcontactaddress={ETH Zurich, ITP, HIT K,
  Wolfgang-Pauli-Strasse 27}}
\hypersetup{pdfcontactpostcode={8093}}
\hypersetup{pdfcontactcity={Zurich}}
\hypersetup{pdfcontactcountry={Switzerland}}
\hypersetup{pdfcontactemail={nbeisert@itp.phys.ethz.ch}}
\hypersetup{pdfcontacturl={http://people.phys.ethz.ch/\xmptilde nbeisert/}}

\newcommand{\secref}[1]{\hyperref[#1]{section \ref*{#1}}}

\parskip1ex
\parindent0pt
\let\olditemize\itemize
\def\itemize{\olditemize\parskip0pt}

\begin{document}

\title{The \textsf{childdoc} Package}
\hypersetup{pdftitle={The childdoc Package}}
\author{Niklas Beisert\\[2ex]
  Institut f\"ur Theoretische Physik\\
  Eidgen\"ossische Technische Hochschule Z\"urich\\
  Wolfgang-Pauli-Strasse 27, 8093 Z\"urich, Switzerland\\[1ex]
  \href{mailto:nbeisert@itp.phys.ethz.ch}
  {\texttt{nbeisert@itp.phys.ethz.ch}}}
\hypersetup{pdfauthor={Niklas Beisert}}
\hypersetup{pdfsubject={Manual for the LaTeX2e Package childdoc}}
\date{30 December 2018, \textsf{v2.0}}
\maketitle

\begin{abstract}\noindent
\textsf{childdoc} is a \LaTeXe{} package
that enables the direct compilation
of document sections included by |\include|
to individual files.
\end{abstract}

\begingroup
\parskip0ex
\tableofcontents
\endgroup

%%%%%%%%%%%%%%%%%%%%%%%%%%%%%%%%%%%%%%%%%%%%%%%%%%%%%%%%%%%%%%%%%%%%%%%%%%%%%%%%
%%%%%%%%%%%%%%%%%%%%%%%%%%%%%%%%%%%%%%%%%%%%%%%%%%%%%%%%%%%%%%%%%%%%%%%%%%%%%%%%
\section{Introduction}

\LaTeX{} provides a mechanism to structure a large document (such as a book)
into a main file and several child files (containing the chapters)
using the |\include| command.
This mechanism is beneficial for documents
which span hundreds of pages in order to
make the source file(s) more manageable.
Moreover, compilation can be restricted to
selected child files by means of the |\includeonly| command.
The latter feature can be used to reduce the compilation time while editing
(this was significantly more useful in the earlier days of \LaTeX{})
or to generate a smaller document which is easier to navigate.
Another application of |\includeonly| is to generate
documents consisting of selected parts of the complete document.

However, there are a few drawbacks of the plain |\include| mechanism:
\begin{itemize}
\item
The child files cannot be compiled on their own,
they can only be compiled via the main file.
A naive editing environment
(such as a text editor with an option
to have the current file processed by \LaTeX)
may require one to switch to the main file before compiling;
attempting to compile the child file produces errors.
\item
The main file must be modified (each time)
to adjust the |\includeonly| command
to the present needs. This easily leaves the main file in a messy state.
\item
The generated document will always carry the filename
of the main document. This is inconvenient if
several child files are to be compiled and
to be kept for distribution.
\end{itemize}

The present package provides a simple interface
to make child files individually compilable by \LaTeX{}.
Compiling a child file then has the same effect as compiling
the main file with an |\includeonly| command
to select the appropriate child.
Moreover the generated document will carry the name of the child
rather than the main file.
This resolves all three above issues.

This feature is meant to make the editing of books,
thesis documents and lecture notes somewhat more convenient.
However, the package can also be used efficiently for
composing a series of documents (such as exercise sheets)
which are typically distributed individually.
It then assists the author in generating the individual documents
(potentially in different versions)
as well as a document containing the collected series.
Another application is in developing style files
or other kinds of included material
where compilation of the style file could redirect
to a sample or test file.

%%%%%%%%%%%%%%%%%%%%%%%%%%%%%%%%%%%%%%%%%%%%%%%%%%%%%%%%%%%%%%%%%%%%%%%%%%%%%%%%
%%%%%%%%%%%%%%%%%%%%%%%%%%%%%%%%%%%%%%%%%%%%%%%%%%%%%%%%%%%%%%%%%%%%%%%%%%%%%%%%
\section{Usage}

First of all, the package \textsf{childdoc} is \emph{not} a standard
\LaTeXe{} |.sty| style file! Therefore it needs to be invoked in
a non-standard way.

%%%%%%%%%%%%%%%%%%%%%%%%%%%%%%%%%%%%%%%%%%%%%%%%%%%%%%%%%%%%%%%%%%%%%%%%%%%%%%%%
\subsection{Included Files}
\label{sec:include}

%%%%%%%%%%%%%%%%%%%%%%%%%%%%%%%%%%%%%%%%
\DescribeMacro{\childdocmain}
To use the package, add the commands
\begin{center}
\begin{tabular}{l}
|\input{childdoc.def}|\\
|\childdocmain{}|\\
\end{tabular}
\end{center}
at the very top of the main \LaTeX{} file,
in particular \emph{before} the |\documentclass| statement!
The argument of |\childdocmain| should be left empty
(but it must be present).

%%%%%%%%%%%%%%%%%%%%%%%%%%%%%%%%%%%%%%%%
\DescribeMacro{\childdocof}
Furthermore, add the commands
\begin{center}
\begin{tabular}{l}
|\input{childdoc.def}|\\
|\childdocof{|\textit{main}|}|\\
\end{tabular}
\end{center}
at the top of every child file \textit{child}
which is included by |\include{|\textit{child}|}|
from within the main file
(or at least for those files to be compiled individually).
The argument \textit{main} must be the filename of the main file.

There are a couple of
considerations in setting up the main and child documents:

%%%%%%%%%%%%%%%%%%%%%%%%%%%%%%%%%%%%%%%%
\paragraph{Restrictions.}

Please note the following restrictions:
\begin{itemize}
\item
|\childdocmain| must be called with one argument \textit{main}
to ensure compatibility with earlier version of the package.
It must either be empty (|\childdocmain{}|)
or precisely match the filename of the main file in which it is specified.
See \secref{sec:detection} for further information.
\item
The filename \textit{main} must be specified without the |.tex| extension.
\item
The filename \textit{main} is case sensitive
(even in case-insensitive file systems)
due to internal string comparison.
\item
The argument \textit{main} should be fully expanded, it cannot be a macro.
\item
Subdirectories and special characters should be avoided in filenames.
\item
The command |\childdocmain{|\textit{main}|}| must be followed by a whitespace.
It should not be followed immediately by another command
or by a comment mark `|%|'.
This is because the \TeX{} parser reads the token immediately following
the argument of |\childdocmain| and puts it
at the beginning of every child section;
however, a white\-space is ignored.
\end{itemize}

%%%%%%%%%%%%%%%%%%%%%%%%%%%%%%%%%%%%%%%%
\paragraph{Content of Main File.}

It is advisable to place all content in the child files included by |\include|.
Any output contained in the main file will appear in all child documents
unless suppressed manually;
it cannot be suppressed automatically by the |\includeonly| directive
and thus should normally be avoided.
A method to include some content in the main file
by means of conditional processing is described in \secref{sec:conditional}.

%%%%%%%%%%%%%%%%%%%%%%%%%%%%%%%%%%%%%%%%
\paragraph{Page Numbering.}

When only a part of the document is compiled,
the appropriate numbering of pages
(as well as other status parameters)
is determined from the |.aux| files.
The latter contain information from previous passes.
However this information needs to propagate through
all intermediate child documents.
Therefore the page numbering in child documents may well
be inconsistent until the complete document is compiled at least once.

A useful (if unconventional) way to always ensure a consistent
page numbering is to restart the numbering in each child document
and denote the pages by `\textit{child}|.|\textit{page}'
where \textit{child} represents the chapter/section number of the child file.
This can be achieved by the command
|\numberwithin{page}{|\textit{child}|}|
of the \textsf{amsmath} package
where \textit{child} can be |chapter| or |section|
depending on the chosen structuring.
Alternatively, one can modify the macro |\thepage| appropriately
and reset the counter |page| at the start of each child file.

%%%%%%%%%%%%%%%%%%%%%%%%%%%%%%%%%%%%%%%%%%%%%%%%%%%%%%%%%%%%%%%%%%%%%%%%%%%%%%%%
\subsection{Conditional Processing}
\label{sec:conditional}

The package provides a mechanism to compile different versions
of a document. To customise the versions further some conditional processing
can come in handy to distinguish which version is being compiled.
The package provides two macros to describe the compilation context:

%%%%%%%%%%%%%%%%%%%%%%%%%%%%%%%%%%%%%%%%
\DescribeMacro{\ifchilddoc}
The conditional |\ifchilddoc| distinguishes between the compilation of
child documents and the main document:
%
\begin{center}
|\ifchilddoc |\textit{child-code}| |[|\||else |\textit{main-code}]| \||fi|
\end{center}

%%%%%%%%%%%%%%%%%%%%%%%%%%%%%%%%%%%%%%%%
\DescribeMacro{\childdocname}
\DescribeMacro{\childdocjob}
The macro |\childdocname| contains the filename (without extension)
of the main or child file being processed.
Note that |\childdocjob| will always contain the name of the main file.

%%%%%%%%%%%%%%%%%%%%%%%%%%%%%%%%%%%%%%%%
\paragraph{Title Page.}

Conditional processing can be used to include a title or banner page
in the main document when proper precautions are taken.
Importantly, the code in the main file should ensure that the page counter
(as well as other status parameters which are stored in the |.aux| files)
takes the same value after the conditional processing.
Otherwise the page numbers may take divergent values
depending on which part is compiled.

For example, a title page could be declared by:
%
\begin{center}
\begin{tabular}{l}
|\ifchilddoc\||else|\\
|\addtocounter{page}{-1}|\\
\textit{code for title page}\\
|\newpage|\\
|\||fi|
\end{tabular}
\end{center}
%
A banner page for the child documents can be generated by:
%
\begin{center}
\begin{tabular}{l}
|\ifchilddoc|\\
|\addtocounter{page}{-1}|\\
\textit{code for banner page}\\
|\newpage|\\
|\||fi|
\end{tabular}
\end{center}
%
Here one could write a message such as:
\begin{center}
|This is the part \childdocname{} of \childdocjob{}.|
\end{center}

%%%%%%%%%%%%%%%%%%%%%%%%%%%%%%%%%%%%%%%%%%%%%%%%%%%%%%%%%%%%%%%%%%%%%%%%%%%%%%%%
\subsection{Flags}
\label{sec:flags}

The package makes it easy to generate different versions
of the main or child documents.
To this end compilation flags can be defined
and assigned different default values.
They will be particularly useful in conjunction
with the forwarding mechanism described in \secref{sec:forward}.

For example, it may be useful to have a flag |\version|
which can be set to |draft| or |final|.
The document source will contain some conditional code
depending on the value of |\version|.
Suppose further, the flag should default to |final| for the main file
and to |draft| for child files
which is a natural assignment for editing the document.
This is achieved by placing the following code
in the preamble of the main document
(below the |\childdocmain| directive):
%
\begin{center}
\begin{tabular}{l}
|\ifchilddoc|\\
|\providecommand{\version}{draft}|\\
|\||else|\\
|\providecommand{\version}{final}|\\
|\||fi|
\end{tabular}
\end{center}
%
The definition by |\providecommand| makes sure
that previous definitions are not overwritten.
Further statements |\providecommand{\version}{...}|
can thus be added before the above code to override it.

For the main file, one might add a line
(between |\childdocmain| and the above block)
%
\begin{center}
|%\ifchilddoc\||else\providecommand{\version}{draft}\||fi|
\end{center}
%
which can be uncommented to produce a draft version.
Likewise one can add a line to the very top of a child file
(above the |\childdocof{|\textit{main}|}| directive)
%
\begin{center}
|%\providecommand{\version}{final}|
\end{center}
%
which can be uncommented to produce the final version of this child document.

%%%%%%%%%%%%%%%%%%%%%%%%%%%%%%%%%%%%%%%%%%%%%%%%%%%%%%%%%%%%%%%%%%%%%%%%%%%%%%%%
\subsection{Forwarding}
\label{sec:forward}

Different versions of the main or child documents
using compilation flags as described in \secref{sec:flags}
can be (permanently) stored in different files
for convenient compilation, viewing and distribution.
To this end, the package defines a command
to pass on compilation to a different file:

%%%%%%%%%%%%%%%%%%%%%%%%%%%%%%%%%%%%%%%%
\DescribeMacro{\childdocforward}
The command |\childdocforward| redirects processing to
another source file:
%
\begin{center}
\begin{tabular}{l}
|\input{childdoc.def}|\\
|\childdocforward[|\textit{main}|]{|\textit{dest}|}|\\
\end{tabular}
\end{center}
%
The argument \textit{dest} is the destination file
(without extension).
It should be the main file or one of the child files.
Note that further \textsf{childdoc} directives
such as |\childdocof| and |\childdocforward|
in the indicated file will be processed in this form.
The optional argument \textit{main}
passes on directly to the main file \textit{main}
while pretending to compile the child \textit{dest}.
This form behaves as if \textit{dest}
issues |\childdocof{|\textit{main}|}| right away,
and no further \textsf{childdoc} directives will be processed.

%%%%%%%%%%%%%%%%%%%%%%%%%%%%%%%%%%%%%%%%
\DescribeMacro{\...prefix}
In the alternative form |\childdocforwardprefix|,
%
\begin{center}
\begin{tabular}{l}
|\input{childdoc.def}|\\
|\childdocforwardprefix[|\textit{main}|]{|\textit{prefix}|}{|\textit{dest}|}|
\end{tabular}
\end{center}
%
the destination file is determined by a pattern
depending on the current file:
To make this work, the current file must be called
`{\textit{prefix}\hspace{0.2em}\textit{suffix}}'
with \textit{prefix} matching precisely the argument.
Processing is then passed on to the file
`{\textit{dest}\hspace{0.2em}\textit{suffix}}'.
Surely, the same effect is achieved by
directly specifying the
argument `{\textit{dest}\hspace{0.2em}\textit{suffix}}'
in the first form.
However, that requires to set up a different file
for each child. With the alternative form of the command
all these files can have exactly the same content
which simplifies setting them up and maintaining them.

For example, the following file |draft.tex|
with a compilation flag |\version| as described in \secref{sec:flags}
compiles the main document as a draft:
%
\begin{center}
\begin{tabular}{l}
|\def\version{draft}|\\
|\input{childdoc.def}|\\
|\childdocforward{|\textit{main}|}|
\end{tabular}
\end{center}
%
Likewise, the following files |final|\textit{nn}|.tex|
compile the final version of the child document
|child|\textit{nn}|.tex|:
%
\begin{center}
\begin{tabular}{l}
|\def\version{final}|\\
|\input{childdoc.def}|\\
|\childdocforwardprefix{final}{child}|
\end{tabular}
\end{center}
%

Note that when several versions of a main file and/or of each child file
are to be generated, it may be convenient to set up a |Makefile| or
shell script to automatise the process.

%%%%%%%%%%%%%%%%%%%%%%%%%%%%%%%%%%%%%%%%%%%%%%%%%%%%%%%%%%%%%%%%%%%%%%%%%%%%%%%%
\subsection{Command Line Processing}
\label{sec:commandline}

The effect of redirection files can also be achieved by invoking
the \LaTeX{} compiler with a more elaborate command line.
Most conveniently this should be done as part
of a shell script or a |Makefile|.

When using \textsf{childdoc} in the main file, the following
command lines effectively perform a redirection
(note that depending on the shell being used,
backslashes may have to be doubled: `|\|' $\to$ `|\\|'):
%
\begin{center}
|... -jobname "|\textit{target}|" |\\|"|[\textit{flags}]%
|\input{childdoc.def}\childdocforward[|\textit{main}|]{|\textit{dest}|}"|
\end{center}
%
Here \textit{target} is the name of the output file,
\textit{main} is the name of the main file
and \textit{dest} is the name of the main or child file to be processed
(all filenames without extensions).
The optional argument \textit{main} can be omitted
if \textit{main} matches \textit{dest}.
Optionally, compilation \textit{flags} can be defined via |\def| commands.
This command line makes the \TeX{} engine believe
it is compiling the file \textit{target}
whose content is specified as the latter parameter.
The provided code then forwards the processing to
\textit{main} or \textit{dest} as described in \secref{sec:forward}.

%%%%%%%%%%%%%%%%%%%%%%%%%%%%%%%%%%%%%%%%%%%%%%%%%%%%%%%%%%%%%%%%%%%%%%%%%%%%%%%%
\subsection{Include by Input}
\label{sec:input}

Including child documents by |\include| has some restrictions by design.
Most notably, the content of a child document always occupies
its own set of pages; pages cannot be shared between child documents.
Usually, this behaviour makes perfect sense
because each child document contain an essential part of the document.
However, in some situations it may be desirable to compose
a document from a collection of parts
without having mandatory page breaks between then.
For this case, the package
provides a mechanism to include parts
by |\input| which can also be processed individually.
However, by construction this mechanism
requires manual handling of the content to be output.

%%%%%%%%%%%%%%%%%%%%%%%%%%%%%%%%%%%%%%%%
\DescribeMacro{\ifchilddocmanual}
The main file should be prepared as usual, see \secref{sec:include}.
However, the document body must make a distinction
between processing of an individual part and of the main document, e.g.:
%
\begin{center}
\begin{tabular}{l}
|\ifchilddocmanual|\\
|\input{\childdocname}|\\
|\||else|\\
\textit{document body with }|\input{|\textit{part}|}|\\
|\||fi|
\end{tabular}
\end{center}
%
The conditional |\ifchilddocmanual| is true whenever
a part to be included by |\input| is being compiled,
and the name of the part is stored in |\childdocname|.

%%%%%%%%%%%%%%%%%%%%%%%%%%%%%%%%%%%%%%%%
\DescribeMacro{\childdocby}
Each part to be included by |\input| should start with:
%
\begin{center}
\begin{tabular}{l}
|\input{childdoc.def}|\\
|\childdocby{|\textit{main}|}|\\
\end{tabular}
\end{center}
%
The directive |\childdocby| is similar to |\childdocof|
described in \secref{sec:include},
but the subsequent selection of content must be done manually.
To that end, both |\ifchilddoc| and |\ifchilddocmanual|
will be true upon processing of a part,
and the name of the part is stored in |\childdocname|.
Note that |\jobname| will be set to the filename of the current part
so that each part receives an individual |.aux| file
that does not interfere with the |.aux| file(s) of the main document.
This behaviour can be altered by the alternative form
|\childdocby[*]{|\textit{main}|}| (with a non-empty optional argument)
which uses the |.aux| file of the main document
by setting |\jobname| to \textit{main}.

%%%%%%%%%%%%%%%%%%%%%%%%%%%%%%%%%%%%%%%%%%%%%%%%%%%%%%%%%%%%%%%%%%%%%%%%%%%%%%%%
\subsection{Driver Development}
\label{sec:driver}

The \textsf{childdoc} mechanism can also be use for the development
of definition files such as \LaTeX{} styles or classes.
This case differs from the above setup with multiple parts
included by |\include| in that no |\includeonly| should be invoked.
This can be achieved by starting the include file
(before |\ProvidesPackage|) with:
%
\begin{center}
\begin{tabular}{l}
|\input{childdoc.def}|\\
|\childdocforward{|\textit{main}|}|\\
\end{tabular}
\end{center}
%
or alternatively with:
%
\begin{center}
\begin{tabular}{l}
|\input{childdoc.def}|\\
|\childdocby{|\textit{main}|}|\\
\end{tabular}
\end{center}
%
Both forms have slightly different effects as described above.
The main file is prepared as usual, see \secref{sec:include}.

%%%%%%%%%%%%%%%%%%%%%%%%%%%%%%%%%%%%%%%%%%%%%%%%%%%%%%%%%%%%%%%%%%%%%%%%%%%%%%%%
\subsection{Legacy Detection}
\label{sec:detection}

The directive |\childdocmain| in the main file can detect
whether the complete document or merely a child is to be compiled
even without using the directive |\childdocof|.
This method is deprecated because it is less robust
and there is no compelling reason to use it;
it is merely provided for backward compatibility
and it may be removed in future versions.

If the detection mechanism is to be used,
it is mandatory to correctly specify
the filename of the main file as the argument of |\childdocmain|:
%
\begin{center}
\begin{tabular}{l}
|\input{childdoc.def}|\\
|\childdocmain{|\textit{main}|}|\\
\end{tabular}
\end{center}
%
If |\jobname| does not match the argument \textit{main} of |\childdocmain|,
it is assumed that |\jobname| points to the child file to be compiled.
When using |\childdocmain| with the main file specified as argument,
it suffices to start a child file
with just |\input{|\textit{main}|}|
without loading of the package and using |\childdocof|.
If instead all processing is done
with the appropriate \textsf{childdoc} directives,
the argument of \textit{main} of |\childdocmain| can be empty.

An alternative version of the command line processing described
in \secref{sec:commandline} using the detection mechanism reads:
%
\begin{center}
|... -jobname "|\textit{target}|" "|[\textit{flags}]%
[|\def\jobname{|\textit{dest}|}|]|\input{|\textit{main}|}"|
\end{center}

%%%%%%%%%%%%%%%%%%%%%%%%%%%%%%%%%%%%%%%%%%%%%%%%%%%%%%%%%%%%%%%%%%%%%%%%%%%%%%%%
\subsection{Manual Code}
\label{sec:manual}

In case one cannot be certain whether the definitions file |childdoc.def|
is installed on the target \TeX{} distribution
and one prefers not to ship it,
it is conceivable to paste a few relevant commands into the sources.

To that end, drop all statements |\input{childdoc.def}|
and perform the replacements as outlined below.
Instead of |\childdocmain{|\textit{main}|}| add the following code
to the top of the main file:
%
\begin{center}
\begin{tabular}{l}
|\||ifdefined\childdocname\endinput\||fi\newif\ifchilddoc|\\
|\edef\childdocname{\scantokens\expandafter{\jobname\noexpand}}|\\
|\def\childdocmain{|\textit{main}|}\||ifx\childdocmain\childdocname\||else|\\
|\childdoctrue\includeonly{\childdocname}\let\jobname\childdocmain\||fi|\\
\end{tabular}
\end{center}
%
Instead of |\childdocof{|\textit{main}|}| just include the main file
at the top of each child file:
%
\begin{center}
|\input{|\textit{main}|}|
\end{center}
%
A simple redirection |\childdocforward{|\textit{dest}|}| is achieved by:
%
\begin{center}
|\def\jobname{|\textit{dest}|}\input{\jobname}|
\end{center}
%
The redirection with prefix
|\childdocforwardprefix[|\textit{prefix}|]{|\textit{dest}|}|
is accomplished by:
%
\begin{center}
\begin{tabular}{l}
|{\edef\jobname{\scantokens\expandafter{\jobname\noexpand}}|\\
|\def\redirectjob |\textit{prefix}|#1~~~{\gdef\jobname{|\textit{dest}|#1}}|\\
|\expandafter\redirectjob\jobname~~~}\input{\jobname}|
\end{tabular}
\end{center}

In an alternative approach,
child documents can be compiled by a specific command line
without additional code or specific definitions:
%
\begin{center}
|... -jobname "|\textit{target}|" "|[\textit{flags}]%
|\includeonly{|\textit{dest}|}\input{|\textit{main}|}"|
\end{center}
%

%%%%%%%%%%%%%%%%%%%%%%%%%%%%%%%%%%%%%%%%%%%%%%%%%%%%%%%%%%%%%%%%%%%%%%%%%%%%%%%%
%%%%%%%%%%%%%%%%%%%%%%%%%%%%%%%%%%%%%%%%%%%%%%%%%%%%%%%%%%%%%%%%%%%%%%%%%%%%%%%%
\section{Information}

%%%%%%%%%%%%%%%%%%%%%%%%%%%%%%%%%%%%%%%%%%%%%%%%%%%%%%%%%%%%%%%%%%%%%%%%%%%%%%%%
\subsection{Copyright}

Copyright \copyright{} 2017--2018 Niklas Beisert

This work may be distributed and/or modified under the
conditions of the \LaTeX{} Project Public License, either version 1.3
of this license or (at your option) any later version.
The latest version of this license is in
  \url{http://www.latex-project.org/lppl.txt}
and version 1.3 or later is part of all distributions of \LaTeX{}
version 2005/12/01 or later.

This work has the LPPL maintenance status `maintained'.

The Current Maintainer of this work is Niklas Beisert.

This work consists of the files |README.txt|, |childdoc.ins| and |childdoc.dtx|
as well as the derived files |childdoc.def|, |cdocsamp.tex|
with |cdocsch1.tex|, |cdocsch2.tex|, |cdocspt3.tex|, |cdocspt4.tex|,
|cdocsdrf.tex|, |cdocsfn1.tex|, |cdocsfn2.tex|
as well as |childdoc.pdf|.

%%%%%%%%%%%%%%%%%%%%%%%%%%%%%%%%%%%%%%%%%%%%%%%%%%%%%%%%%%%%%%%%%%%%%%%%%%%%%%%%
\subsection{Files and Installation}

The package consists of the files:
%
\begin{center}
\begin{tabular}{ll}
    |README.txt|   & readme file \\
    |childdoc.ins| & installation file \\
    |childdoc.dtx| & source file \\
    |childdoc.def| & definition file \\
    |cdocsamp.tex| & sample main file \\
    |cdocsch1.tex| & sample include file \\
    |cdocsch2.tex| & sample include file \\
    |cdocspt3.tex| & sample part file \\
    |cdocspt4.tex| & sample part file \\
    |cdocsdrf.tex| & sample redirection file \\
    |cdocsfn1.tex| & sample redirection file \\
    |cdocsfn2.tex| & sample redirection file \\
    |childdoc.pdf| & manual
\end{tabular}
\end{center}
%
The distribution consists of the files
|README.txt|, |childdoc.ins| and |childdoc.dtx|.
%
\begin{itemize}
\item
Run (pdf)\LaTeX{} on |childdoc.dtx|
to compile the manual |childdoc.pdf| (this file).
\item
Run \LaTeX{} on |childdoc.ins| to create the definitions file |childdoc.def|
and the sample |cdocsamp.tex| with include files
|cdocsch1.tex|, |cdocsch2.tex|, |cdocspt3.tex|, |cdocspt4.tex|,
|cdocsdrf.tex|, |cdocsfn1.tex|, |cdocsfn2.tex|.
Then copy the file |childdoc.def| to an appropriate directory of your \LaTeX{}
distribution, e.g.\ \textit{texmf-root}|/tex/latex/childdoc|.
\end{itemize}

%%%%%%%%%%%%%%%%%%%%%%%%%%%%%%%%%%%%%%%%%%%%%%%%%%%%%%%%%%%%%%%%%%%%%%%%%%%%%%%%
\subsection{Related CTAN Packages}

There are several other packages which offer a similar functionality:
%
\begin{itemize}
\item
The packages
\href{http://ctan.org/pkg/docmute}{\textsf{docmute}},
\href{http://ctan.org/pkg/includex}{\textsf{includex}} and
\href{http://ctan.org/pkg/standalone}{\textsf{standalone}}
provide commands to include only the document body of
a child file thus allowing both files to be compiled individually.
\item
The packages \href{http://ctan.org/pkg/subdocs}{\textsf{subdocs}}
and \href{http://ctan.org/pkg/subfiles}{\textsf{subfiles}}
provide structures in which the main and child documents can be
encapsulated and allowing them to be compiled individually.
The inclusion mechanism is different from the conventional |\include|.
\item
The package \href{http://ctan.org/pkg/combine}{\textsf{combine}}
is an elaborate solution to combine several documents into one.
\end{itemize}
%
See also the CTAN topic \href{http://ctan.org/topic/subdocs}{\textsf{subdocs}}
for further related packages.
The present package differs from the above solutions in that
a document structure constructed with the conventional |\include| mechanism
just needs two extra commands at the top of every file
such that all constituent files can be compiled individually.

%%%%%%%%%%%%%%%%%%%%%%%%%%%%%%%%%%%%%%%%%%%%%%%%%%%%%%%%%%%%%%%%%%%%%%%%%%%%%%%%
%\subsection{Feature Suggestions}
%
%The following is a list of features which may be useful for future
%versions of this package:
%%
%\begin{itemize}
%\item
%\ldots
%\end{itemize}

%%%%%%%%%%%%%%%%%%%%%%%%%%%%%%%%%%%%%%%%%%%%%%%%%%%%%%%%%%%%%%%%%%%%%%%%%%%%%%%%
\subsection{Revision History}

%%%%%%%%%%%%%%%%%%%%%%%%%%%%%%%%%%%%%%%%
\paragraph{v2.0:} 2018/12/30

\begin{itemize}
\item
immediate forward processing
\item
added |\childdocby| mechanism
\item
manual restructured
\end{itemize}

%%%%%%%%%%%%%%%%%%%%%%%%%%%%%%%%%%%%%%%%
\paragraph{v1.6:} 2018/01/17

\begin{itemize}
\item
application for development of include files
\item
corrections to manual
\end{itemize}

%%%%%%%%%%%%%%%%%%%%%%%%%%%%%%%%%%%%%%%%
\paragraph{v1.5:} 2017/05/21

\begin{itemize}
\item
more complete structuring introduced
\item
|\childdocof| introduced
\item
|\childdoc| renamed to |\childdocmain|
\item
|\childredirect| renamed to |\childdocforward| and |\childdocforwardprefix|
and functionality expanded
\end{itemize}

%%%%%%%%%%%%%%%%%%%%%%%%%%%%%%%%%%%%%%%%
\paragraph{v1.0:} 2017/04/27

\begin{itemize}
\item
manual and install package
\item
first version published on CTAN
\end{itemize}

%%%%%%%%%%%%%%%%%%%%%%%%%%%%%%%%%%%%%%%%
\paragraph{v0.6:} 2017/04/26

\begin{itemize}
\item
redirection mechanism added
\end{itemize}

%%%%%%%%%%%%%%%%%%%%%%%%%%%%%%%%%%%%%%%%
\paragraph{v0.5:} 2017/04/26

\begin{itemize}
\item
functionality in definition file
\end{itemize}


%%%%%%%%%%%%%%%%%%%%%%%%%%%%%%%%%%%%%%%%%%%%%%%%%%%%%%%%%%%%%%%%%%%%%%%%%%%%%%%%
%%%%%%%%%%%%%%%%%%%%%%%%%%%%%%%%%%%%%%%%%%%%%%%%%%%%%%%%%%%%%%%%%%%%%%%%%%%%%%%%
%%%%%%%%%%%%%%%%%%%%%%%%%%%%%%%%%%%%%%%%%%%%%%%%%%%%%%%%%%%%%%%%%%%%%%%%%%%%%%%%
\appendix

\settowidth\MacroIndent{\rmfamily\scriptsize 000\ }

 \DocInput{childdoc.dtx}

\end{document}
%</driver>
% \fi
%
% %%%%%%%%%%%%%%%%%%%%%%%%%%%%%%%%%%%%%%%%%%%%%%%%%%%%%%%%%%%%%%%%%%%%%%%%%%%%%%
% %%%%%%%%%%%%%%%%%%%%%%%%%%%%%%%%%%%%%%%%%%%%%%%%%%%%%%%%%%%%%%%%%%%%%%%%%%%%%%
% \section{Sample}
%\iffalse
%<*samplemain>
%\fi
%
% The following presents a sample document
% with two chapters, two parts, a title page,
% a compile flag as well as three forwarding files to set the flag.
% It consists of eight |.tex| files:
% \begin{center}
% \begin{tabular}{ll}
% |cdocsamp.tex|&main file\\
% |cdocsch1.tex|&include file for chapter 1\\
% |cdocsch2.tex|&include file for chapter 2\\
% |cdocspt3.tex|&include file for part 3\\
% |cdocspt4.tex|&include file for part 4\\
% |cdocsdrf.tex|&forwarding file for main file in draft mode\\
% |cdocsfi1.tex|&forwarding file for final version of chapter 1\\
% |cdocsfi2.tex|&forwarding file for final version of chapter 2\\
% \end{tabular}
% \end{center}
% Each of the eight files can be compiled directly by the \LaTeX{} compiler.
%
% %%%%%%%%%%%%%%%%%%%%%%%%%%%%%%%%%%%%%%
% \paragraph{Main File.}
%
% The main file is called |cdocsamp.tex|.
%
% Load the \textsf{childdoc} definitions and
% declare the filename for the main document:
%    \begin{macrocode}
\input{childdoc.def}
\childdocmain{}
%    \end{macrocode}

% Optional override for |\version| flag:
%    \begin{macrocode}
%%\ifchilddoc\else\providecommand{\version}{draft}\fi
%    \end{macrocode}

% Define the default values for the |\version| flag
% (|final| for the main file and |draft| for childs):
%    \begin{macrocode}
\ifchilddoc
\providecommand{\version}{draft}
\else
\providecommand{\version}{final}
\fi
%    \end{macrocode}

% Load the standard document class:
%    \begin{macrocode}
\documentclass[12pt]{article}
%    \end{macrocode}

% Start the document body:
%    \begin{macrocode}
\begin{document}
%    \end{macrocode}

% Declare a title page.
% Print title, part of document being processed and version flag:
%    \begin{macrocode}
\addtocounter{page}{-1}
\begin{center}
{\LARGE\bfseries{}childdoc example\par}
\vspace{1cm}
\ifchilddoc
\ifchilddocmanual part\else chapter\fi:
`\childdocname' of `\childdocjob'\par
\else
main document: `\childdocjob'\par
\fi
version: \version\par
\end{center}
\newpage
%    \end{macrocode}

% Manually include selected file,
% otherwise process as usual:
%    \begin{macrocode}
\ifchilddocmanual
\section*{part `\childdocname'}
\input{\childdocname}
\else
%    \end{macrocode}

% Include the two chapters:
%    \begin{macrocode}
\include{cdocsch1}
\include{cdocsch2}
%    \end{macrocode}

% Include the two parts unless only chapters should be displayed:
%    \begin{macrocode}
\ifchilddoc\else
\section{part three}
\input{cdocspt3}
\section{part four}
\input{cdocspt4}
\fi
%    \end{macrocode}

% Process as usual until here:
%    \begin{macrocode}
\fi
%    \end{macrocode}

% End of document body:
%    \begin{macrocode}
\end{document}
%    \end{macrocode}
%\iffalse
%</samplemain>
%\fi
%
% %%%%%%%%%%%%%%%%%%%%%%%%%%%%%%%%%%%%%%
% \paragraph{Chapter Include Files.}
%
% The include files are called |cdocsch1.tex| and |cdocsch2.tex|.
%
%\iffalse
%<*samplechap1|samplechap2>
%\fi

% Optional override for |\version| flag:
%    \begin{macrocode}
%%\providecommand{\version}{final}
%    \end{macrocode}

% Include the main document:
%    \begin{macrocode}
\input{childdoc.def}
\childdocof{cdocsamp}
%    \end{macrocode}

%\iffalse
%</samplechap1|samplechap2>
%\fi
%
%\iffalse
%<*samplechap1>
%\fi
% Some text for chapter 1:
%    \begin{macrocode}
\section{one}
some text in chapter one
%    \end{macrocode}

%\iffalse
%</samplechap1>
%\fi
% Some text for chapter 2:
%\iffalse
%<*samplechap2>
%\fi
%    \begin{macrocode}
\section{two}
more text in chapter two
%    \end{macrocode}

%\iffalse
%</samplechap2>
%\fi
%
% %%%%%%%%%%%%%%%%%%%%%%%%%%%%%%%%%%%%%%
% \paragraph{Part Include Files.}
%
% The include files are called |cdocspt3.tex| and |cdocspt4.tex|.
%
%\iffalse
%<*samplepart3|samplepart4>
%\fi

% Optional override for |\version| flag:
%    \begin{macrocode}
%%\providecommand{\version}{final}
%    \end{macrocode}

% Include the main document:
%    \begin{macrocode}
\input{childdoc.def}
\childdocby{cdocsamp}
%    \end{macrocode}

%\iffalse
%</samplepart3|samplepart4>
%\fi
%
%\iffalse
%<*samplepart3>
%\fi
% Some text for part 3:
%    \begin{macrocode}
some text in part three
%    \end{macrocode}

%\iffalse
%</samplepart3>
%\fi
% Some text for part 4:
%\iffalse
%<*samplepart4>
%\fi
%    \begin{macrocode}
more text in part four
%    \end{macrocode}

%\iffalse
%</samplepart4>
%\fi
%
% %%%%%%%%%%%%%%%%%%%%%%%%%%%%%%%%%%%%%%
% \paragraph{Forwarding for a Complete Draft.}
%
% The following forwarding file |cdocsdrf.tex|
% compiles the main document in draft mode:
%\iffalse
%<*sampledraft>
%\fi
%    \begin{macrocode}
\def\version{draft}
\input{childdoc.def}
\childdocforward{cdocsamp}
%    \end{macrocode}

%\iffalse
%</sampledraft>
%\fi
%
% %%%%%%%%%%%%%%%%%%%%%%%%%%%%%%%%%%%%%%
% \paragraph{Forwarding for Final Version of the Chapters.}
%
% The following forwarding files |cdocsfn1.tex| and |cdocsfn2.tex|
% (with identical content)
% compile the final versions of the child documents
% |cdocsch1.tex| and |cdocsch2.tex|, respectively:
%\iffalse
%<*samplefinal>
%\fi
%    \begin{macrocode}
\def\version{final}
\input{childdoc.def}
\childdocforwardprefix[cdocsamp]{cdocsfn}{cdocsch}
%    \end{macrocode}

%\iffalse
%</samplefinal>
%\fi
%
% %%%%%%%%%%%%%%%%%%%%%%%%%%%%%%%%%%%%%%
% \paragraph{Command Line Processing.}
%
% The following three command lines generate the output files
% |cdocscld|, |cdocscl1| and |cdocscl2|
% which should be identical to
% |cdocsdrf|, |cdocsch1| and |cdocsfn2|, respectively:
% \begin{center}
% \begin{tabular}{l}
% |latex -jobname cdocscld \|\\
% |  "\def\version{draft}\input{childdoc.def}\childdocforward{cdocsamp}"|\\
% |latex -jobname cdocscl1 \|\\
% |  "\input{childdoc.def}\childdocforward[cdocsamp]{cdocsch1}"|\\
% |latex -jobname cdocscl2 \|\\
% |  "\def\version{final}\input{childdoc.def}\childdocforward{cdocsch2}"|
% \end{tabular}
% \end{center}
% Note that the trailing backslash on each first line
% merely continues the input to the second line
% (for convenient cut ant paste).
% Furthermore, the command |latex| can be replaced by any
% of its alternative versions such as |pdflatex|.
%
% %%%%%%%%%%%%%%%%%%%%%%%%%%%%%%%%%%%%%%%%%%%%%%%%%%%%%%%%%%%%%%%%%%%%%%%%%%%%%%
% %%%%%%%%%%%%%%%%%%%%%%%%%%%%%%%%%%%%%%%%%%%%%%%%%%%%%%%%%%%%%%%%%%%%%%%%%%%%%%
% \section{Implementation}
%\iffalse
%<*package>
%\fi
%
% This section describes the definitions file |childdoc.def|.

% The definitions cannot be loaded using |\usepackage| or |\RequirePackage|
% which has a mechanism to prevent loading a style file more than once.
% When loading the definitions by means of |\input|
% multiple instances have to be prevented manually:
%\iffalse
%This code needs to be before the `\ProvidesFile' directive
%which is defined at the beginning of this file.
%Therefore it is also placed there and commented out here.
%</package>
%<*discard>
%\fi
%    \begin{macrocode}
\ifdefined\childdocmain\endinput\fi
%    \end{macrocode}
%\iffalse
%</discard>
%<*package>
%\fi
%
% \macro{\ifchilddoc}
% \macro{\ifchilddocmanual}
% The conditional |\ifchilddoc| tells whether a
% child (true) or main (false) document is being compiled.
% The conditional |\ifchilddocmanual| tells whether
% the |\includeonly| mechanism is used (false) or
% the selection of child files must be performed manually (true).
% The definitions initialise to false:
%    \begin{macrocode}
\newif\ifchilddoc
\newif\ifchilddocmanual
%    \end{macrocode}

% \macro{\childdocname}
% \macro{\childdocjob}
% The macro |\childdocname| stores the name of the main document
% to be compiled. The macro |\childdocjob| stores the name of
% the document on which the \LaTeX{} compiler was originally invoked.
% The content of |\jobname| cannot be compared
% to filenames specified in the source due to different catcodes.
% The following code rescans |\jobname|, stores the result
% in |\childdocname| and saves a copy in |\childdocjob|:
%    \begin{macrocode}
\edef\childdocname{\scantokens\expandafter{\jobname\noexpand}}
\let\childdocjob\childdocname
%    \end{macrocode}

% \macro{\childdocdisable}
% The macro |\childdocdisable| prevents the main file
% from being processed more than once.
% At this stage, the main document command |\childdocmain|
% is assumed to be called once again where it should do nothing.
% Any subsequent call to it should prevent
% a secondary processing of the main document
% It overwrites the forwarding commands
% |\childdocof| and |\childdocforward|
% with empty macros to prevent further inclusions of the main document:
%    \begin{macrocode}
\newcommand{\childdocdisable}
{
  \renewcommand{\childdocmain}[1]{\renewcommand{\childdocmain}[1]{\endinput}}
  \renewcommand{\childdocof}[1]{}
  \renewcommand{\childdocby}[2][]{}
  \renewcommand{\childdocforward}[2][]{}
  \renewcommand{\childdocdisable}{}
}
%    \end{macrocode}

% \macro{\childdocmain}
% The macro |\childdocmain| is to be called at the top of the main file
% with nothing or the main filename (without extension) as argument.
% First, it breaks loops.
% If the argument is not empty and does not match |\childdocname|
% (which is set by the first inclusion of |childdoc.def|),
% |\ifchilddoc| is set to true, |\includeonly| is applied to the child file
% and |\jobname| is set to the main file
% (for proper handling of |.aux| files):
%    \begin{macrocode}
\newcommand{\childdocmain}[1]
{
  \childdocdisable\childdocmain{}
  \if?#1?\else
    \begingroup
      \def\childdoctmp{#1}
      \ifx\childdoctmp\childdocname
        \def\childdoctmp{}
      \else
        \def\childdoctmp
        {
          \childdoctrue
          \includeonly{\childdocname}
          \def\childdocjob{#1}
          \def\jobname{#1}
        }
      \fi
      \expandafter
    \endgroup
    \childdoctmp
  \fi
}
%    \end{macrocode}

% \macro{\childdocof}
% The command |\childdocof| redirects
% compilation to the main file |#1|.
%    \begin{macrocode}
\newcommand{\childdocof}[1]
{
  \childdocdisable
  \childdoctrue
  \includeonly{\childdocname}
  \def\jobname{#1}
  \def\childdocjob{#1}
  \input{#1}
}
%    \end{macrocode}

% \macro{\childdocby}
% The command |\childdocby| ....
%    \begin{macrocode}
\newcommand{\childdocby}[2][]
{
  \childdocdisable
  \childdoctrue
  \childdocmanualtrue
  \if?#1?\else
    \def\jobname{#2}
  \fi
  \def\childdocjob{#2}
  \input{#2}
  \endinput
}
%    \end{macrocode}

% \macro{\childdocforward}
% The command |\childdocforward| redirects
% compilation to the main file or
% (if the optional argument is given) a child file.
% Parameters are set as if the main file
% or a child file starting with |\childdocof| was compiled.
% Then compilation is handed over to the main file:
%    \begin{macrocode}
\newcommand{\childdocforward}[2][]
{
  \begingroup
    \if?#1?
      \def\childdoctmp
      {
        \def\childdocname{#2}
        \def\childdocjob{#2}
        \def\jobname{#2}
        \input{#2}
        \endinput
      }
    \else
      \def\childdoctmp
      {
        \childdocdisable
        \def\childdocname{#2}
        \childdoctrue
        \includeonly{#2}
        \def\childdocjob{#1}
        \def\jobname{#1}
        \input{#1}
        \endinput
      }
    \fi
    \expandafter
  \endgroup
  \childdoctmp
}
%    \end{macrocode}

% \macro{\childdocforwardprefix}
% The command |\childdocforwardprefix| redirects
% compilation to the main or a child file by means of a pattern.
% The prefix |#1| in the current filename is replaced by |#2|
% and the suffix of the current filename is kept
% (it is assumed that the filename does not contain the substring `|~~~|'
% which is used as a delimiter).
% Compilation is handed over to the new file by |\childdocforward|:
%    \begin{macrocode}
\newcommand{\childdocforwardprefix}[3][]
{
  \begingroup
    \def\childdocextract #2##1~~~{\def\childdoctmp{\childdocforward[#1]{#3##1}}}
    \expandafter\childdocextract\childdocname~~~
    \expandafter
  \endgroup
  \childdoctmp
}
%    \end{macrocode}

% \macro{\childdoc}
% The deprecated macro |\childdoc| is a legacy version of |\childdocmain|:
%    \begin{macrocode}
\newcommand{\childdoc}{\childdocmain}
%    \end{macrocode}

% \macro{\childdocredirect}
% The deprecated macro |\childdocredirect| is a legacy version
% of |\childdocforward| and |\childdocforwardprefix|:
%    \begin{macrocode}
\newcommand{\childdocredirect}[2][]
{
  \begingroup
    \if?#1?
      \def\childdoctmp{\childdocforward{#2}}
    \else
      \def\childdoctmp{\childdocforwardprefix{#1}{#2}}
    \fi
    \expandafter
  \endgroup
  \childdoctmp
}
%    \end{macrocode}

%\iffalse
%</package>
%\fi
%
\endinput

\childdocforward{cdocsamp}
%    \end{macrocode}

%\iffalse
%</sampledraft>
%\fi
%
% %%%%%%%%%%%%%%%%%%%%%%%%%%%%%%%%%%%%%%
% \paragraph{Forwarding for Final Version of the Chapters.}
%
% The following forwarding files |cdocsfn1.tex| and |cdocsfn2.tex|
% (with identical content)
% compile the final versions of the child documents
% |cdocsch1.tex| and |cdocsch2.tex|, respectively:
%\iffalse
%<*samplefinal>
%\fi
%    \begin{macrocode}
\def\version{final}
% \iffalse
%
% childdoc.dtx Copyright (C) 2017-2018 Niklas Beisert
%
% This work may be distributed and/or modified under the
% conditions of the LaTeX Project Public License, either version 1.3
% of this license or (at your option) any later version.
% The latest version of this license is in
%   http://www.latex-project.org/lppl.txt
% and version 1.3 or later is part of all distributions of LaTeX
% version 2005/12/01 or later.
%
% This work has the LPPL maintenance status `maintained'.
%
% The Current Maintainer of this work is Niklas Beisert.
%
% This work consists of the files childdoc.dtx and childdoc.ins
% and the derived files childdoc.def and cdocsamp.tex with
% cdocsch1.tex, cdocsch2.tex, cdocsdrf.tex, cdocsfn1.tex, cdocsfn2.tex.
%
%<package>\ifdefined\childdocmain\endinput\fi
%<package>\ProvidesFile{childdoc.def}[2018/12/30 v2.0 child document driver]
%<samplemain>\ProvidesFile{cdocsamp.tex}[2018/12/30 v2.0 sample for childdoc]
%<*driver>
%\ProvidesFile{childdoc.drv}[2018/12/30 v2.0 childdoc reference manual file]
\PassOptionsToClass{10pt,a4paper}{article}
\documentclass{ltxdoc}

\usepackage[margin=35mm]{geometry}
\usepackage{hyperref}
\usepackage{hyperxmp}
\usepackage[usenames]{color}

\hypersetup{colorlinks=true}
\hypersetup{pdfstartview=FitH}
\hypersetup{pdfpagemode=UseNone}
\hypersetup{pdfsource={}}
\hypersetup{pdflang={en-UK}}
\hypersetup{pdfcopyright={Copyright 2017-2018 Niklas Beisert.
  This work may be distributed and/or modified under the
  conditions of the LaTeX Project Public License, either version 1.3
  of this license or (at your option) any later version.}}
\hypersetup{pdflicenseurl={http://www.latex-project.org/lppl.txt}}
\hypersetup{pdfcontactaddress={ETH Zurich, ITP, HIT K,
  Wolfgang-Pauli-Strasse 27}}
\hypersetup{pdfcontactpostcode={8093}}
\hypersetup{pdfcontactcity={Zurich}}
\hypersetup{pdfcontactcountry={Switzerland}}
\hypersetup{pdfcontactemail={nbeisert@itp.phys.ethz.ch}}
\hypersetup{pdfcontacturl={http://people.phys.ethz.ch/\xmptilde nbeisert/}}

\newcommand{\secref}[1]{\hyperref[#1]{section \ref*{#1}}}

\parskip1ex
\parindent0pt
\let\olditemize\itemize
\def\itemize{\olditemize\parskip0pt}

\begin{document}

\title{The \textsf{childdoc} Package}
\hypersetup{pdftitle={The childdoc Package}}
\author{Niklas Beisert\\[2ex]
  Institut f\"ur Theoretische Physik\\
  Eidgen\"ossische Technische Hochschule Z\"urich\\
  Wolfgang-Pauli-Strasse 27, 8093 Z\"urich, Switzerland\\[1ex]
  \href{mailto:nbeisert@itp.phys.ethz.ch}
  {\texttt{nbeisert@itp.phys.ethz.ch}}}
\hypersetup{pdfauthor={Niklas Beisert}}
\hypersetup{pdfsubject={Manual for the LaTeX2e Package childdoc}}
\date{30 December 2018, \textsf{v2.0}}
\maketitle

\begin{abstract}\noindent
\textsf{childdoc} is a \LaTeXe{} package
that enables the direct compilation
of document sections included by |\include|
to individual files.
\end{abstract}

\begingroup
\parskip0ex
\tableofcontents
\endgroup

%%%%%%%%%%%%%%%%%%%%%%%%%%%%%%%%%%%%%%%%%%%%%%%%%%%%%%%%%%%%%%%%%%%%%%%%%%%%%%%%
%%%%%%%%%%%%%%%%%%%%%%%%%%%%%%%%%%%%%%%%%%%%%%%%%%%%%%%%%%%%%%%%%%%%%%%%%%%%%%%%
\section{Introduction}

\LaTeX{} provides a mechanism to structure a large document (such as a book)
into a main file and several child files (containing the chapters)
using the |\include| command.
This mechanism is beneficial for documents
which span hundreds of pages in order to
make the source file(s) more manageable.
Moreover, compilation can be restricted to
selected child files by means of the |\includeonly| command.
The latter feature can be used to reduce the compilation time while editing
(this was significantly more useful in the earlier days of \LaTeX{})
or to generate a smaller document which is easier to navigate.
Another application of |\includeonly| is to generate
documents consisting of selected parts of the complete document.

However, there are a few drawbacks of the plain |\include| mechanism:
\begin{itemize}
\item
The child files cannot be compiled on their own,
they can only be compiled via the main file.
A naive editing environment
(such as a text editor with an option
to have the current file processed by \LaTeX)
may require one to switch to the main file before compiling;
attempting to compile the child file produces errors.
\item
The main file must be modified (each time)
to adjust the |\includeonly| command
to the present needs. This easily leaves the main file in a messy state.
\item
The generated document will always carry the filename
of the main document. This is inconvenient if
several child files are to be compiled and
to be kept for distribution.
\end{itemize}

The present package provides a simple interface
to make child files individually compilable by \LaTeX{}.
Compiling a child file then has the same effect as compiling
the main file with an |\includeonly| command
to select the appropriate child.
Moreover the generated document will carry the name of the child
rather than the main file.
This resolves all three above issues.

This feature is meant to make the editing of books,
thesis documents and lecture notes somewhat more convenient.
However, the package can also be used efficiently for
composing a series of documents (such as exercise sheets)
which are typically distributed individually.
It then assists the author in generating the individual documents
(potentially in different versions)
as well as a document containing the collected series.
Another application is in developing style files
or other kinds of included material
where compilation of the style file could redirect
to a sample or test file.

%%%%%%%%%%%%%%%%%%%%%%%%%%%%%%%%%%%%%%%%%%%%%%%%%%%%%%%%%%%%%%%%%%%%%%%%%%%%%%%%
%%%%%%%%%%%%%%%%%%%%%%%%%%%%%%%%%%%%%%%%%%%%%%%%%%%%%%%%%%%%%%%%%%%%%%%%%%%%%%%%
\section{Usage}

First of all, the package \textsf{childdoc} is \emph{not} a standard
\LaTeXe{} |.sty| style file! Therefore it needs to be invoked in
a non-standard way.

%%%%%%%%%%%%%%%%%%%%%%%%%%%%%%%%%%%%%%%%%%%%%%%%%%%%%%%%%%%%%%%%%%%%%%%%%%%%%%%%
\subsection{Included Files}
\label{sec:include}

%%%%%%%%%%%%%%%%%%%%%%%%%%%%%%%%%%%%%%%%
\DescribeMacro{\childdocmain}
To use the package, add the commands
\begin{center}
\begin{tabular}{l}
|\input{childdoc.def}|\\
|\childdocmain{}|\\
\end{tabular}
\end{center}
at the very top of the main \LaTeX{} file,
in particular \emph{before} the |\documentclass| statement!
The argument of |\childdocmain| should be left empty
(but it must be present).

%%%%%%%%%%%%%%%%%%%%%%%%%%%%%%%%%%%%%%%%
\DescribeMacro{\childdocof}
Furthermore, add the commands
\begin{center}
\begin{tabular}{l}
|\input{childdoc.def}|\\
|\childdocof{|\textit{main}|}|\\
\end{tabular}
\end{center}
at the top of every child file \textit{child}
which is included by |\include{|\textit{child}|}|
from within the main file
(or at least for those files to be compiled individually).
The argument \textit{main} must be the filename of the main file.

There are a couple of
considerations in setting up the main and child documents:

%%%%%%%%%%%%%%%%%%%%%%%%%%%%%%%%%%%%%%%%
\paragraph{Restrictions.}

Please note the following restrictions:
\begin{itemize}
\item
|\childdocmain| must be called with one argument \textit{main}
to ensure compatibility with earlier version of the package.
It must either be empty (|\childdocmain{}|)
or precisely match the filename of the main file in which it is specified.
See \secref{sec:detection} for further information.
\item
The filename \textit{main} must be specified without the |.tex| extension.
\item
The filename \textit{main} is case sensitive
(even in case-insensitive file systems)
due to internal string comparison.
\item
The argument \textit{main} should be fully expanded, it cannot be a macro.
\item
Subdirectories and special characters should be avoided in filenames.
\item
The command |\childdocmain{|\textit{main}|}| must be followed by a whitespace.
It should not be followed immediately by another command
or by a comment mark `|%|'.
This is because the \TeX{} parser reads the token immediately following
the argument of |\childdocmain| and puts it
at the beginning of every child section;
however, a white\-space is ignored.
\end{itemize}

%%%%%%%%%%%%%%%%%%%%%%%%%%%%%%%%%%%%%%%%
\paragraph{Content of Main File.}

It is advisable to place all content in the child files included by |\include|.
Any output contained in the main file will appear in all child documents
unless suppressed manually;
it cannot be suppressed automatically by the |\includeonly| directive
and thus should normally be avoided.
A method to include some content in the main file
by means of conditional processing is described in \secref{sec:conditional}.

%%%%%%%%%%%%%%%%%%%%%%%%%%%%%%%%%%%%%%%%
\paragraph{Page Numbering.}

When only a part of the document is compiled,
the appropriate numbering of pages
(as well as other status parameters)
is determined from the |.aux| files.
The latter contain information from previous passes.
However this information needs to propagate through
all intermediate child documents.
Therefore the page numbering in child documents may well
be inconsistent until the complete document is compiled at least once.

A useful (if unconventional) way to always ensure a consistent
page numbering is to restart the numbering in each child document
and denote the pages by `\textit{child}|.|\textit{page}'
where \textit{child} represents the chapter/section number of the child file.
This can be achieved by the command
|\numberwithin{page}{|\textit{child}|}|
of the \textsf{amsmath} package
where \textit{child} can be |chapter| or |section|
depending on the chosen structuring.
Alternatively, one can modify the macro |\thepage| appropriately
and reset the counter |page| at the start of each child file.

%%%%%%%%%%%%%%%%%%%%%%%%%%%%%%%%%%%%%%%%%%%%%%%%%%%%%%%%%%%%%%%%%%%%%%%%%%%%%%%%
\subsection{Conditional Processing}
\label{sec:conditional}

The package provides a mechanism to compile different versions
of a document. To customise the versions further some conditional processing
can come in handy to distinguish which version is being compiled.
The package provides two macros to describe the compilation context:

%%%%%%%%%%%%%%%%%%%%%%%%%%%%%%%%%%%%%%%%
\DescribeMacro{\ifchilddoc}
The conditional |\ifchilddoc| distinguishes between the compilation of
child documents and the main document:
%
\begin{center}
|\ifchilddoc |\textit{child-code}| |[|\||else |\textit{main-code}]| \||fi|
\end{center}

%%%%%%%%%%%%%%%%%%%%%%%%%%%%%%%%%%%%%%%%
\DescribeMacro{\childdocname}
\DescribeMacro{\childdocjob}
The macro |\childdocname| contains the filename (without extension)
of the main or child file being processed.
Note that |\childdocjob| will always contain the name of the main file.

%%%%%%%%%%%%%%%%%%%%%%%%%%%%%%%%%%%%%%%%
\paragraph{Title Page.}

Conditional processing can be used to include a title or banner page
in the main document when proper precautions are taken.
Importantly, the code in the main file should ensure that the page counter
(as well as other status parameters which are stored in the |.aux| files)
takes the same value after the conditional processing.
Otherwise the page numbers may take divergent values
depending on which part is compiled.

For example, a title page could be declared by:
%
\begin{center}
\begin{tabular}{l}
|\ifchilddoc\||else|\\
|\addtocounter{page}{-1}|\\
\textit{code for title page}\\
|\newpage|\\
|\||fi|
\end{tabular}
\end{center}
%
A banner page for the child documents can be generated by:
%
\begin{center}
\begin{tabular}{l}
|\ifchilddoc|\\
|\addtocounter{page}{-1}|\\
\textit{code for banner page}\\
|\newpage|\\
|\||fi|
\end{tabular}
\end{center}
%
Here one could write a message such as:
\begin{center}
|This is the part \childdocname{} of \childdocjob{}.|
\end{center}

%%%%%%%%%%%%%%%%%%%%%%%%%%%%%%%%%%%%%%%%%%%%%%%%%%%%%%%%%%%%%%%%%%%%%%%%%%%%%%%%
\subsection{Flags}
\label{sec:flags}

The package makes it easy to generate different versions
of the main or child documents.
To this end compilation flags can be defined
and assigned different default values.
They will be particularly useful in conjunction
with the forwarding mechanism described in \secref{sec:forward}.

For example, it may be useful to have a flag |\version|
which can be set to |draft| or |final|.
The document source will contain some conditional code
depending on the value of |\version|.
Suppose further, the flag should default to |final| for the main file
and to |draft| for child files
which is a natural assignment for editing the document.
This is achieved by placing the following code
in the preamble of the main document
(below the |\childdocmain| directive):
%
\begin{center}
\begin{tabular}{l}
|\ifchilddoc|\\
|\providecommand{\version}{draft}|\\
|\||else|\\
|\providecommand{\version}{final}|\\
|\||fi|
\end{tabular}
\end{center}
%
The definition by |\providecommand| makes sure
that previous definitions are not overwritten.
Further statements |\providecommand{\version}{...}|
can thus be added before the above code to override it.

For the main file, one might add a line
(between |\childdocmain| and the above block)
%
\begin{center}
|%\ifchilddoc\||else\providecommand{\version}{draft}\||fi|
\end{center}
%
which can be uncommented to produce a draft version.
Likewise one can add a line to the very top of a child file
(above the |\childdocof{|\textit{main}|}| directive)
%
\begin{center}
|%\providecommand{\version}{final}|
\end{center}
%
which can be uncommented to produce the final version of this child document.

%%%%%%%%%%%%%%%%%%%%%%%%%%%%%%%%%%%%%%%%%%%%%%%%%%%%%%%%%%%%%%%%%%%%%%%%%%%%%%%%
\subsection{Forwarding}
\label{sec:forward}

Different versions of the main or child documents
using compilation flags as described in \secref{sec:flags}
can be (permanently) stored in different files
for convenient compilation, viewing and distribution.
To this end, the package defines a command
to pass on compilation to a different file:

%%%%%%%%%%%%%%%%%%%%%%%%%%%%%%%%%%%%%%%%
\DescribeMacro{\childdocforward}
The command |\childdocforward| redirects processing to
another source file:
%
\begin{center}
\begin{tabular}{l}
|\input{childdoc.def}|\\
|\childdocforward[|\textit{main}|]{|\textit{dest}|}|\\
\end{tabular}
\end{center}
%
The argument \textit{dest} is the destination file
(without extension).
It should be the main file or one of the child files.
Note that further \textsf{childdoc} directives
such as |\childdocof| and |\childdocforward|
in the indicated file will be processed in this form.
The optional argument \textit{main}
passes on directly to the main file \textit{main}
while pretending to compile the child \textit{dest}.
This form behaves as if \textit{dest}
issues |\childdocof{|\textit{main}|}| right away,
and no further \textsf{childdoc} directives will be processed.

%%%%%%%%%%%%%%%%%%%%%%%%%%%%%%%%%%%%%%%%
\DescribeMacro{\...prefix}
In the alternative form |\childdocforwardprefix|,
%
\begin{center}
\begin{tabular}{l}
|\input{childdoc.def}|\\
|\childdocforwardprefix[|\textit{main}|]{|\textit{prefix}|}{|\textit{dest}|}|
\end{tabular}
\end{center}
%
the destination file is determined by a pattern
depending on the current file:
To make this work, the current file must be called
`{\textit{prefix}\hspace{0.2em}\textit{suffix}}'
with \textit{prefix} matching precisely the argument.
Processing is then passed on to the file
`{\textit{dest}\hspace{0.2em}\textit{suffix}}'.
Surely, the same effect is achieved by
directly specifying the
argument `{\textit{dest}\hspace{0.2em}\textit{suffix}}'
in the first form.
However, that requires to set up a different file
for each child. With the alternative form of the command
all these files can have exactly the same content
which simplifies setting them up and maintaining them.

For example, the following file |draft.tex|
with a compilation flag |\version| as described in \secref{sec:flags}
compiles the main document as a draft:
%
\begin{center}
\begin{tabular}{l}
|\def\version{draft}|\\
|\input{childdoc.def}|\\
|\childdocforward{|\textit{main}|}|
\end{tabular}
\end{center}
%
Likewise, the following files |final|\textit{nn}|.tex|
compile the final version of the child document
|child|\textit{nn}|.tex|:
%
\begin{center}
\begin{tabular}{l}
|\def\version{final}|\\
|\input{childdoc.def}|\\
|\childdocforwardprefix{final}{child}|
\end{tabular}
\end{center}
%

Note that when several versions of a main file and/or of each child file
are to be generated, it may be convenient to set up a |Makefile| or
shell script to automatise the process.

%%%%%%%%%%%%%%%%%%%%%%%%%%%%%%%%%%%%%%%%%%%%%%%%%%%%%%%%%%%%%%%%%%%%%%%%%%%%%%%%
\subsection{Command Line Processing}
\label{sec:commandline}

The effect of redirection files can also be achieved by invoking
the \LaTeX{} compiler with a more elaborate command line.
Most conveniently this should be done as part
of a shell script or a |Makefile|.

When using \textsf{childdoc} in the main file, the following
command lines effectively perform a redirection
(note that depending on the shell being used,
backslashes may have to be doubled: `|\|' $\to$ `|\\|'):
%
\begin{center}
|... -jobname "|\textit{target}|" |\\|"|[\textit{flags}]%
|\input{childdoc.def}\childdocforward[|\textit{main}|]{|\textit{dest}|}"|
\end{center}
%
Here \textit{target} is the name of the output file,
\textit{main} is the name of the main file
and \textit{dest} is the name of the main or child file to be processed
(all filenames without extensions).
The optional argument \textit{main} can be omitted
if \textit{main} matches \textit{dest}.
Optionally, compilation \textit{flags} can be defined via |\def| commands.
This command line makes the \TeX{} engine believe
it is compiling the file \textit{target}
whose content is specified as the latter parameter.
The provided code then forwards the processing to
\textit{main} or \textit{dest} as described in \secref{sec:forward}.

%%%%%%%%%%%%%%%%%%%%%%%%%%%%%%%%%%%%%%%%%%%%%%%%%%%%%%%%%%%%%%%%%%%%%%%%%%%%%%%%
\subsection{Include by Input}
\label{sec:input}

Including child documents by |\include| has some restrictions by design.
Most notably, the content of a child document always occupies
its own set of pages; pages cannot be shared between child documents.
Usually, this behaviour makes perfect sense
because each child document contain an essential part of the document.
However, in some situations it may be desirable to compose
a document from a collection of parts
without having mandatory page breaks between then.
For this case, the package
provides a mechanism to include parts
by |\input| which can also be processed individually.
However, by construction this mechanism
requires manual handling of the content to be output.

%%%%%%%%%%%%%%%%%%%%%%%%%%%%%%%%%%%%%%%%
\DescribeMacro{\ifchilddocmanual}
The main file should be prepared as usual, see \secref{sec:include}.
However, the document body must make a distinction
between processing of an individual part and of the main document, e.g.:
%
\begin{center}
\begin{tabular}{l}
|\ifchilddocmanual|\\
|\input{\childdocname}|\\
|\||else|\\
\textit{document body with }|\input{|\textit{part}|}|\\
|\||fi|
\end{tabular}
\end{center}
%
The conditional |\ifchilddocmanual| is true whenever
a part to be included by |\input| is being compiled,
and the name of the part is stored in |\childdocname|.

%%%%%%%%%%%%%%%%%%%%%%%%%%%%%%%%%%%%%%%%
\DescribeMacro{\childdocby}
Each part to be included by |\input| should start with:
%
\begin{center}
\begin{tabular}{l}
|\input{childdoc.def}|\\
|\childdocby{|\textit{main}|}|\\
\end{tabular}
\end{center}
%
The directive |\childdocby| is similar to |\childdocof|
described in \secref{sec:include},
but the subsequent selection of content must be done manually.
To that end, both |\ifchilddoc| and |\ifchilddocmanual|
will be true upon processing of a part,
and the name of the part is stored in |\childdocname|.
Note that |\jobname| will be set to the filename of the current part
so that each part receives an individual |.aux| file
that does not interfere with the |.aux| file(s) of the main document.
This behaviour can be altered by the alternative form
|\childdocby[*]{|\textit{main}|}| (with a non-empty optional argument)
which uses the |.aux| file of the main document
by setting |\jobname| to \textit{main}.

%%%%%%%%%%%%%%%%%%%%%%%%%%%%%%%%%%%%%%%%%%%%%%%%%%%%%%%%%%%%%%%%%%%%%%%%%%%%%%%%
\subsection{Driver Development}
\label{sec:driver}

The \textsf{childdoc} mechanism can also be use for the development
of definition files such as \LaTeX{} styles or classes.
This case differs from the above setup with multiple parts
included by |\include| in that no |\includeonly| should be invoked.
This can be achieved by starting the include file
(before |\ProvidesPackage|) with:
%
\begin{center}
\begin{tabular}{l}
|\input{childdoc.def}|\\
|\childdocforward{|\textit{main}|}|\\
\end{tabular}
\end{center}
%
or alternatively with:
%
\begin{center}
\begin{tabular}{l}
|\input{childdoc.def}|\\
|\childdocby{|\textit{main}|}|\\
\end{tabular}
\end{center}
%
Both forms have slightly different effects as described above.
The main file is prepared as usual, see \secref{sec:include}.

%%%%%%%%%%%%%%%%%%%%%%%%%%%%%%%%%%%%%%%%%%%%%%%%%%%%%%%%%%%%%%%%%%%%%%%%%%%%%%%%
\subsection{Legacy Detection}
\label{sec:detection}

The directive |\childdocmain| in the main file can detect
whether the complete document or merely a child is to be compiled
even without using the directive |\childdocof|.
This method is deprecated because it is less robust
and there is no compelling reason to use it;
it is merely provided for backward compatibility
and it may be removed in future versions.

If the detection mechanism is to be used,
it is mandatory to correctly specify
the filename of the main file as the argument of |\childdocmain|:
%
\begin{center}
\begin{tabular}{l}
|\input{childdoc.def}|\\
|\childdocmain{|\textit{main}|}|\\
\end{tabular}
\end{center}
%
If |\jobname| does not match the argument \textit{main} of |\childdocmain|,
it is assumed that |\jobname| points to the child file to be compiled.
When using |\childdocmain| with the main file specified as argument,
it suffices to start a child file
with just |\input{|\textit{main}|}|
without loading of the package and using |\childdocof|.
If instead all processing is done
with the appropriate \textsf{childdoc} directives,
the argument of \textit{main} of |\childdocmain| can be empty.

An alternative version of the command line processing described
in \secref{sec:commandline} using the detection mechanism reads:
%
\begin{center}
|... -jobname "|\textit{target}|" "|[\textit{flags}]%
[|\def\jobname{|\textit{dest}|}|]|\input{|\textit{main}|}"|
\end{center}

%%%%%%%%%%%%%%%%%%%%%%%%%%%%%%%%%%%%%%%%%%%%%%%%%%%%%%%%%%%%%%%%%%%%%%%%%%%%%%%%
\subsection{Manual Code}
\label{sec:manual}

In case one cannot be certain whether the definitions file |childdoc.def|
is installed on the target \TeX{} distribution
and one prefers not to ship it,
it is conceivable to paste a few relevant commands into the sources.

To that end, drop all statements |\input{childdoc.def}|
and perform the replacements as outlined below.
Instead of |\childdocmain{|\textit{main}|}| add the following code
to the top of the main file:
%
\begin{center}
\begin{tabular}{l}
|\||ifdefined\childdocname\endinput\||fi\newif\ifchilddoc|\\
|\edef\childdocname{\scantokens\expandafter{\jobname\noexpand}}|\\
|\def\childdocmain{|\textit{main}|}\||ifx\childdocmain\childdocname\||else|\\
|\childdoctrue\includeonly{\childdocname}\let\jobname\childdocmain\||fi|\\
\end{tabular}
\end{center}
%
Instead of |\childdocof{|\textit{main}|}| just include the main file
at the top of each child file:
%
\begin{center}
|\input{|\textit{main}|}|
\end{center}
%
A simple redirection |\childdocforward{|\textit{dest}|}| is achieved by:
%
\begin{center}
|\def\jobname{|\textit{dest}|}\input{\jobname}|
\end{center}
%
The redirection with prefix
|\childdocforwardprefix[|\textit{prefix}|]{|\textit{dest}|}|
is accomplished by:
%
\begin{center}
\begin{tabular}{l}
|{\edef\jobname{\scantokens\expandafter{\jobname\noexpand}}|\\
|\def\redirectjob |\textit{prefix}|#1~~~{\gdef\jobname{|\textit{dest}|#1}}|\\
|\expandafter\redirectjob\jobname~~~}\input{\jobname}|
\end{tabular}
\end{center}

In an alternative approach,
child documents can be compiled by a specific command line
without additional code or specific definitions:
%
\begin{center}
|... -jobname "|\textit{target}|" "|[\textit{flags}]%
|\includeonly{|\textit{dest}|}\input{|\textit{main}|}"|
\end{center}
%

%%%%%%%%%%%%%%%%%%%%%%%%%%%%%%%%%%%%%%%%%%%%%%%%%%%%%%%%%%%%%%%%%%%%%%%%%%%%%%%%
%%%%%%%%%%%%%%%%%%%%%%%%%%%%%%%%%%%%%%%%%%%%%%%%%%%%%%%%%%%%%%%%%%%%%%%%%%%%%%%%
\section{Information}

%%%%%%%%%%%%%%%%%%%%%%%%%%%%%%%%%%%%%%%%%%%%%%%%%%%%%%%%%%%%%%%%%%%%%%%%%%%%%%%%
\subsection{Copyright}

Copyright \copyright{} 2017--2018 Niklas Beisert

This work may be distributed and/or modified under the
conditions of the \LaTeX{} Project Public License, either version 1.3
of this license or (at your option) any later version.
The latest version of this license is in
  \url{http://www.latex-project.org/lppl.txt}
and version 1.3 or later is part of all distributions of \LaTeX{}
version 2005/12/01 or later.

This work has the LPPL maintenance status `maintained'.

The Current Maintainer of this work is Niklas Beisert.

This work consists of the files |README.txt|, |childdoc.ins| and |childdoc.dtx|
as well as the derived files |childdoc.def|, |cdocsamp.tex|
with |cdocsch1.tex|, |cdocsch2.tex|, |cdocspt3.tex|, |cdocspt4.tex|,
|cdocsdrf.tex|, |cdocsfn1.tex|, |cdocsfn2.tex|
as well as |childdoc.pdf|.

%%%%%%%%%%%%%%%%%%%%%%%%%%%%%%%%%%%%%%%%%%%%%%%%%%%%%%%%%%%%%%%%%%%%%%%%%%%%%%%%
\subsection{Files and Installation}

The package consists of the files:
%
\begin{center}
\begin{tabular}{ll}
    |README.txt|   & readme file \\
    |childdoc.ins| & installation file \\
    |childdoc.dtx| & source file \\
    |childdoc.def| & definition file \\
    |cdocsamp.tex| & sample main file \\
    |cdocsch1.tex| & sample include file \\
    |cdocsch2.tex| & sample include file \\
    |cdocspt3.tex| & sample part file \\
    |cdocspt4.tex| & sample part file \\
    |cdocsdrf.tex| & sample redirection file \\
    |cdocsfn1.tex| & sample redirection file \\
    |cdocsfn2.tex| & sample redirection file \\
    |childdoc.pdf| & manual
\end{tabular}
\end{center}
%
The distribution consists of the files
|README.txt|, |childdoc.ins| and |childdoc.dtx|.
%
\begin{itemize}
\item
Run (pdf)\LaTeX{} on |childdoc.dtx|
to compile the manual |childdoc.pdf| (this file).
\item
Run \LaTeX{} on |childdoc.ins| to create the definitions file |childdoc.def|
and the sample |cdocsamp.tex| with include files
|cdocsch1.tex|, |cdocsch2.tex|, |cdocspt3.tex|, |cdocspt4.tex|,
|cdocsdrf.tex|, |cdocsfn1.tex|, |cdocsfn2.tex|.
Then copy the file |childdoc.def| to an appropriate directory of your \LaTeX{}
distribution, e.g.\ \textit{texmf-root}|/tex/latex/childdoc|.
\end{itemize}

%%%%%%%%%%%%%%%%%%%%%%%%%%%%%%%%%%%%%%%%%%%%%%%%%%%%%%%%%%%%%%%%%%%%%%%%%%%%%%%%
\subsection{Related CTAN Packages}

There are several other packages which offer a similar functionality:
%
\begin{itemize}
\item
The packages
\href{http://ctan.org/pkg/docmute}{\textsf{docmute}},
\href{http://ctan.org/pkg/includex}{\textsf{includex}} and
\href{http://ctan.org/pkg/standalone}{\textsf{standalone}}
provide commands to include only the document body of
a child file thus allowing both files to be compiled individually.
\item
The packages \href{http://ctan.org/pkg/subdocs}{\textsf{subdocs}}
and \href{http://ctan.org/pkg/subfiles}{\textsf{subfiles}}
provide structures in which the main and child documents can be
encapsulated and allowing them to be compiled individually.
The inclusion mechanism is different from the conventional |\include|.
\item
The package \href{http://ctan.org/pkg/combine}{\textsf{combine}}
is an elaborate solution to combine several documents into one.
\end{itemize}
%
See also the CTAN topic \href{http://ctan.org/topic/subdocs}{\textsf{subdocs}}
for further related packages.
The present package differs from the above solutions in that
a document structure constructed with the conventional |\include| mechanism
just needs two extra commands at the top of every file
such that all constituent files can be compiled individually.

%%%%%%%%%%%%%%%%%%%%%%%%%%%%%%%%%%%%%%%%%%%%%%%%%%%%%%%%%%%%%%%%%%%%%%%%%%%%%%%%
%\subsection{Feature Suggestions}
%
%The following is a list of features which may be useful for future
%versions of this package:
%%
%\begin{itemize}
%\item
%\ldots
%\end{itemize}

%%%%%%%%%%%%%%%%%%%%%%%%%%%%%%%%%%%%%%%%%%%%%%%%%%%%%%%%%%%%%%%%%%%%%%%%%%%%%%%%
\subsection{Revision History}

%%%%%%%%%%%%%%%%%%%%%%%%%%%%%%%%%%%%%%%%
\paragraph{v2.0:} 2018/12/30

\begin{itemize}
\item
immediate forward processing
\item
added |\childdocby| mechanism
\item
manual restructured
\end{itemize}

%%%%%%%%%%%%%%%%%%%%%%%%%%%%%%%%%%%%%%%%
\paragraph{v1.6:} 2018/01/17

\begin{itemize}
\item
application for development of include files
\item
corrections to manual
\end{itemize}

%%%%%%%%%%%%%%%%%%%%%%%%%%%%%%%%%%%%%%%%
\paragraph{v1.5:} 2017/05/21

\begin{itemize}
\item
more complete structuring introduced
\item
|\childdocof| introduced
\item
|\childdoc| renamed to |\childdocmain|
\item
|\childredirect| renamed to |\childdocforward| and |\childdocforwardprefix|
and functionality expanded
\end{itemize}

%%%%%%%%%%%%%%%%%%%%%%%%%%%%%%%%%%%%%%%%
\paragraph{v1.0:} 2017/04/27

\begin{itemize}
\item
manual and install package
\item
first version published on CTAN
\end{itemize}

%%%%%%%%%%%%%%%%%%%%%%%%%%%%%%%%%%%%%%%%
\paragraph{v0.6:} 2017/04/26

\begin{itemize}
\item
redirection mechanism added
\end{itemize}

%%%%%%%%%%%%%%%%%%%%%%%%%%%%%%%%%%%%%%%%
\paragraph{v0.5:} 2017/04/26

\begin{itemize}
\item
functionality in definition file
\end{itemize}


%%%%%%%%%%%%%%%%%%%%%%%%%%%%%%%%%%%%%%%%%%%%%%%%%%%%%%%%%%%%%%%%%%%%%%%%%%%%%%%%
%%%%%%%%%%%%%%%%%%%%%%%%%%%%%%%%%%%%%%%%%%%%%%%%%%%%%%%%%%%%%%%%%%%%%%%%%%%%%%%%
%%%%%%%%%%%%%%%%%%%%%%%%%%%%%%%%%%%%%%%%%%%%%%%%%%%%%%%%%%%%%%%%%%%%%%%%%%%%%%%%
\appendix

\settowidth\MacroIndent{\rmfamily\scriptsize 000\ }

 \DocInput{childdoc.dtx}

\end{document}
%</driver>
% \fi
%
% %%%%%%%%%%%%%%%%%%%%%%%%%%%%%%%%%%%%%%%%%%%%%%%%%%%%%%%%%%%%%%%%%%%%%%%%%%%%%%
% %%%%%%%%%%%%%%%%%%%%%%%%%%%%%%%%%%%%%%%%%%%%%%%%%%%%%%%%%%%%%%%%%%%%%%%%%%%%%%
% \section{Sample}
%\iffalse
%<*samplemain>
%\fi
%
% The following presents a sample document
% with two chapters, two parts, a title page,
% a compile flag as well as three forwarding files to set the flag.
% It consists of eight |.tex| files:
% \begin{center}
% \begin{tabular}{ll}
% |cdocsamp.tex|&main file\\
% |cdocsch1.tex|&include file for chapter 1\\
% |cdocsch2.tex|&include file for chapter 2\\
% |cdocspt3.tex|&include file for part 3\\
% |cdocspt4.tex|&include file for part 4\\
% |cdocsdrf.tex|&forwarding file for main file in draft mode\\
% |cdocsfi1.tex|&forwarding file for final version of chapter 1\\
% |cdocsfi2.tex|&forwarding file for final version of chapter 2\\
% \end{tabular}
% \end{center}
% Each of the eight files can be compiled directly by the \LaTeX{} compiler.
%
% %%%%%%%%%%%%%%%%%%%%%%%%%%%%%%%%%%%%%%
% \paragraph{Main File.}
%
% The main file is called |cdocsamp.tex|.
%
% Load the \textsf{childdoc} definitions and
% declare the filename for the main document:
%    \begin{macrocode}
\input{childdoc.def}
\childdocmain{}
%    \end{macrocode}

% Optional override for |\version| flag:
%    \begin{macrocode}
%%\ifchilddoc\else\providecommand{\version}{draft}\fi
%    \end{macrocode}

% Define the default values for the |\version| flag
% (|final| for the main file and |draft| for childs):
%    \begin{macrocode}
\ifchilddoc
\providecommand{\version}{draft}
\else
\providecommand{\version}{final}
\fi
%    \end{macrocode}

% Load the standard document class:
%    \begin{macrocode}
\documentclass[12pt]{article}
%    \end{macrocode}

% Start the document body:
%    \begin{macrocode}
\begin{document}
%    \end{macrocode}

% Declare a title page.
% Print title, part of document being processed and version flag:
%    \begin{macrocode}
\addtocounter{page}{-1}
\begin{center}
{\LARGE\bfseries{}childdoc example\par}
\vspace{1cm}
\ifchilddoc
\ifchilddocmanual part\else chapter\fi:
`\childdocname' of `\childdocjob'\par
\else
main document: `\childdocjob'\par
\fi
version: \version\par
\end{center}
\newpage
%    \end{macrocode}

% Manually include selected file,
% otherwise process as usual:
%    \begin{macrocode}
\ifchilddocmanual
\section*{part `\childdocname'}
\input{\childdocname}
\else
%    \end{macrocode}

% Include the two chapters:
%    \begin{macrocode}
\include{cdocsch1}
\include{cdocsch2}
%    \end{macrocode}

% Include the two parts unless only chapters should be displayed:
%    \begin{macrocode}
\ifchilddoc\else
\section{part three}
\input{cdocspt3}
\section{part four}
\input{cdocspt4}
\fi
%    \end{macrocode}

% Process as usual until here:
%    \begin{macrocode}
\fi
%    \end{macrocode}

% End of document body:
%    \begin{macrocode}
\end{document}
%    \end{macrocode}
%\iffalse
%</samplemain>
%\fi
%
% %%%%%%%%%%%%%%%%%%%%%%%%%%%%%%%%%%%%%%
% \paragraph{Chapter Include Files.}
%
% The include files are called |cdocsch1.tex| and |cdocsch2.tex|.
%
%\iffalse
%<*samplechap1|samplechap2>
%\fi

% Optional override for |\version| flag:
%    \begin{macrocode}
%%\providecommand{\version}{final}
%    \end{macrocode}

% Include the main document:
%    \begin{macrocode}
\input{childdoc.def}
\childdocof{cdocsamp}
%    \end{macrocode}

%\iffalse
%</samplechap1|samplechap2>
%\fi
%
%\iffalse
%<*samplechap1>
%\fi
% Some text for chapter 1:
%    \begin{macrocode}
\section{one}
some text in chapter one
%    \end{macrocode}

%\iffalse
%</samplechap1>
%\fi
% Some text for chapter 2:
%\iffalse
%<*samplechap2>
%\fi
%    \begin{macrocode}
\section{two}
more text in chapter two
%    \end{macrocode}

%\iffalse
%</samplechap2>
%\fi
%
% %%%%%%%%%%%%%%%%%%%%%%%%%%%%%%%%%%%%%%
% \paragraph{Part Include Files.}
%
% The include files are called |cdocspt3.tex| and |cdocspt4.tex|.
%
%\iffalse
%<*samplepart3|samplepart4>
%\fi

% Optional override for |\version| flag:
%    \begin{macrocode}
%%\providecommand{\version}{final}
%    \end{macrocode}

% Include the main document:
%    \begin{macrocode}
\input{childdoc.def}
\childdocby{cdocsamp}
%    \end{macrocode}

%\iffalse
%</samplepart3|samplepart4>
%\fi
%
%\iffalse
%<*samplepart3>
%\fi
% Some text for part 3:
%    \begin{macrocode}
some text in part three
%    \end{macrocode}

%\iffalse
%</samplepart3>
%\fi
% Some text for part 4:
%\iffalse
%<*samplepart4>
%\fi
%    \begin{macrocode}
more text in part four
%    \end{macrocode}

%\iffalse
%</samplepart4>
%\fi
%
% %%%%%%%%%%%%%%%%%%%%%%%%%%%%%%%%%%%%%%
% \paragraph{Forwarding for a Complete Draft.}
%
% The following forwarding file |cdocsdrf.tex|
% compiles the main document in draft mode:
%\iffalse
%<*sampledraft>
%\fi
%    \begin{macrocode}
\def\version{draft}
\input{childdoc.def}
\childdocforward{cdocsamp}
%    \end{macrocode}

%\iffalse
%</sampledraft>
%\fi
%
% %%%%%%%%%%%%%%%%%%%%%%%%%%%%%%%%%%%%%%
% \paragraph{Forwarding for Final Version of the Chapters.}
%
% The following forwarding files |cdocsfn1.tex| and |cdocsfn2.tex|
% (with identical content)
% compile the final versions of the child documents
% |cdocsch1.tex| and |cdocsch2.tex|, respectively:
%\iffalse
%<*samplefinal>
%\fi
%    \begin{macrocode}
\def\version{final}
\input{childdoc.def}
\childdocforwardprefix[cdocsamp]{cdocsfn}{cdocsch}
%    \end{macrocode}

%\iffalse
%</samplefinal>
%\fi
%
% %%%%%%%%%%%%%%%%%%%%%%%%%%%%%%%%%%%%%%
% \paragraph{Command Line Processing.}
%
% The following three command lines generate the output files
% |cdocscld|, |cdocscl1| and |cdocscl2|
% which should be identical to
% |cdocsdrf|, |cdocsch1| and |cdocsfn2|, respectively:
% \begin{center}
% \begin{tabular}{l}
% |latex -jobname cdocscld \|\\
% |  "\def\version{draft}\input{childdoc.def}\childdocforward{cdocsamp}"|\\
% |latex -jobname cdocscl1 \|\\
% |  "\input{childdoc.def}\childdocforward[cdocsamp]{cdocsch1}"|\\
% |latex -jobname cdocscl2 \|\\
% |  "\def\version{final}\input{childdoc.def}\childdocforward{cdocsch2}"|
% \end{tabular}
% \end{center}
% Note that the trailing backslash on each first line
% merely continues the input to the second line
% (for convenient cut ant paste).
% Furthermore, the command |latex| can be replaced by any
% of its alternative versions such as |pdflatex|.
%
% %%%%%%%%%%%%%%%%%%%%%%%%%%%%%%%%%%%%%%%%%%%%%%%%%%%%%%%%%%%%%%%%%%%%%%%%%%%%%%
% %%%%%%%%%%%%%%%%%%%%%%%%%%%%%%%%%%%%%%%%%%%%%%%%%%%%%%%%%%%%%%%%%%%%%%%%%%%%%%
% \section{Implementation}
%\iffalse
%<*package>
%\fi
%
% This section describes the definitions file |childdoc.def|.

% The definitions cannot be loaded using |\usepackage| or |\RequirePackage|
% which has a mechanism to prevent loading a style file more than once.
% When loading the definitions by means of |\input|
% multiple instances have to be prevented manually:
%\iffalse
%This code needs to be before the `\ProvidesFile' directive
%which is defined at the beginning of this file.
%Therefore it is also placed there and commented out here.
%</package>
%<*discard>
%\fi
%    \begin{macrocode}
\ifdefined\childdocmain\endinput\fi
%    \end{macrocode}
%\iffalse
%</discard>
%<*package>
%\fi
%
% \macro{\ifchilddoc}
% \macro{\ifchilddocmanual}
% The conditional |\ifchilddoc| tells whether a
% child (true) or main (false) document is being compiled.
% The conditional |\ifchilddocmanual| tells whether
% the |\includeonly| mechanism is used (false) or
% the selection of child files must be performed manually (true).
% The definitions initialise to false:
%    \begin{macrocode}
\newif\ifchilddoc
\newif\ifchilddocmanual
%    \end{macrocode}

% \macro{\childdocname}
% \macro{\childdocjob}
% The macro |\childdocname| stores the name of the main document
% to be compiled. The macro |\childdocjob| stores the name of
% the document on which the \LaTeX{} compiler was originally invoked.
% The content of |\jobname| cannot be compared
% to filenames specified in the source due to different catcodes.
% The following code rescans |\jobname|, stores the result
% in |\childdocname| and saves a copy in |\childdocjob|:
%    \begin{macrocode}
\edef\childdocname{\scantokens\expandafter{\jobname\noexpand}}
\let\childdocjob\childdocname
%    \end{macrocode}

% \macro{\childdocdisable}
% The macro |\childdocdisable| prevents the main file
% from being processed more than once.
% At this stage, the main document command |\childdocmain|
% is assumed to be called once again where it should do nothing.
% Any subsequent call to it should prevent
% a secondary processing of the main document
% It overwrites the forwarding commands
% |\childdocof| and |\childdocforward|
% with empty macros to prevent further inclusions of the main document:
%    \begin{macrocode}
\newcommand{\childdocdisable}
{
  \renewcommand{\childdocmain}[1]{\renewcommand{\childdocmain}[1]{\endinput}}
  \renewcommand{\childdocof}[1]{}
  \renewcommand{\childdocby}[2][]{}
  \renewcommand{\childdocforward}[2][]{}
  \renewcommand{\childdocdisable}{}
}
%    \end{macrocode}

% \macro{\childdocmain}
% The macro |\childdocmain| is to be called at the top of the main file
% with nothing or the main filename (without extension) as argument.
% First, it breaks loops.
% If the argument is not empty and does not match |\childdocname|
% (which is set by the first inclusion of |childdoc.def|),
% |\ifchilddoc| is set to true, |\includeonly| is applied to the child file
% and |\jobname| is set to the main file
% (for proper handling of |.aux| files):
%    \begin{macrocode}
\newcommand{\childdocmain}[1]
{
  \childdocdisable\childdocmain{}
  \if?#1?\else
    \begingroup
      \def\childdoctmp{#1}
      \ifx\childdoctmp\childdocname
        \def\childdoctmp{}
      \else
        \def\childdoctmp
        {
          \childdoctrue
          \includeonly{\childdocname}
          \def\childdocjob{#1}
          \def\jobname{#1}
        }
      \fi
      \expandafter
    \endgroup
    \childdoctmp
  \fi
}
%    \end{macrocode}

% \macro{\childdocof}
% The command |\childdocof| redirects
% compilation to the main file |#1|.
%    \begin{macrocode}
\newcommand{\childdocof}[1]
{
  \childdocdisable
  \childdoctrue
  \includeonly{\childdocname}
  \def\jobname{#1}
  \def\childdocjob{#1}
  \input{#1}
}
%    \end{macrocode}

% \macro{\childdocby}
% The command |\childdocby| ....
%    \begin{macrocode}
\newcommand{\childdocby}[2][]
{
  \childdocdisable
  \childdoctrue
  \childdocmanualtrue
  \if?#1?\else
    \def\jobname{#2}
  \fi
  \def\childdocjob{#2}
  \input{#2}
  \endinput
}
%    \end{macrocode}

% \macro{\childdocforward}
% The command |\childdocforward| redirects
% compilation to the main file or
% (if the optional argument is given) a child file.
% Parameters are set as if the main file
% or a child file starting with |\childdocof| was compiled.
% Then compilation is handed over to the main file:
%    \begin{macrocode}
\newcommand{\childdocforward}[2][]
{
  \begingroup
    \if?#1?
      \def\childdoctmp
      {
        \def\childdocname{#2}
        \def\childdocjob{#2}
        \def\jobname{#2}
        \input{#2}
        \endinput
      }
    \else
      \def\childdoctmp
      {
        \childdocdisable
        \def\childdocname{#2}
        \childdoctrue
        \includeonly{#2}
        \def\childdocjob{#1}
        \def\jobname{#1}
        \input{#1}
        \endinput
      }
    \fi
    \expandafter
  \endgroup
  \childdoctmp
}
%    \end{macrocode}

% \macro{\childdocforwardprefix}
% The command |\childdocforwardprefix| redirects
% compilation to the main or a child file by means of a pattern.
% The prefix |#1| in the current filename is replaced by |#2|
% and the suffix of the current filename is kept
% (it is assumed that the filename does not contain the substring `|~~~|'
% which is used as a delimiter).
% Compilation is handed over to the new file by |\childdocforward|:
%    \begin{macrocode}
\newcommand{\childdocforwardprefix}[3][]
{
  \begingroup
    \def\childdocextract #2##1~~~{\def\childdoctmp{\childdocforward[#1]{#3##1}}}
    \expandafter\childdocextract\childdocname~~~
    \expandafter
  \endgroup
  \childdoctmp
}
%    \end{macrocode}

% \macro{\childdoc}
% The deprecated macro |\childdoc| is a legacy version of |\childdocmain|:
%    \begin{macrocode}
\newcommand{\childdoc}{\childdocmain}
%    \end{macrocode}

% \macro{\childdocredirect}
% The deprecated macro |\childdocredirect| is a legacy version
% of |\childdocforward| and |\childdocforwardprefix|:
%    \begin{macrocode}
\newcommand{\childdocredirect}[2][]
{
  \begingroup
    \if?#1?
      \def\childdoctmp{\childdocforward{#2}}
    \else
      \def\childdoctmp{\childdocforwardprefix{#1}{#2}}
    \fi
    \expandafter
  \endgroup
  \childdoctmp
}
%    \end{macrocode}

%\iffalse
%</package>
%\fi
%
\endinput

\childdocforwardprefix[cdocsamp]{cdocsfn}{cdocsch}
%    \end{macrocode}

%\iffalse
%</samplefinal>
%\fi
%
% %%%%%%%%%%%%%%%%%%%%%%%%%%%%%%%%%%%%%%
% \paragraph{Command Line Processing.}
%
% The following three command lines generate the output files
% |cdocscld|, |cdocscl1| and |cdocscl2|
% which should be identical to
% |cdocsdrf|, |cdocsch1| and |cdocsfn2|, respectively:
% \begin{center}
% \begin{tabular}{l}
% |latex -jobname cdocscld \|\\
% |  "\def\version{draft}% \iffalse
%
% childdoc.dtx Copyright (C) 2017-2018 Niklas Beisert
%
% This work may be distributed and/or modified under the
% conditions of the LaTeX Project Public License, either version 1.3
% of this license or (at your option) any later version.
% The latest version of this license is in
%   http://www.latex-project.org/lppl.txt
% and version 1.3 or later is part of all distributions of LaTeX
% version 2005/12/01 or later.
%
% This work has the LPPL maintenance status `maintained'.
%
% The Current Maintainer of this work is Niklas Beisert.
%
% This work consists of the files childdoc.dtx and childdoc.ins
% and the derived files childdoc.def and cdocsamp.tex with
% cdocsch1.tex, cdocsch2.tex, cdocsdrf.tex, cdocsfn1.tex, cdocsfn2.tex.
%
%<package>\ifdefined\childdocmain\endinput\fi
%<package>\ProvidesFile{childdoc.def}[2018/12/30 v2.0 child document driver]
%<samplemain>\ProvidesFile{cdocsamp.tex}[2018/12/30 v2.0 sample for childdoc]
%<*driver>
%\ProvidesFile{childdoc.drv}[2018/12/30 v2.0 childdoc reference manual file]
\PassOptionsToClass{10pt,a4paper}{article}
\documentclass{ltxdoc}

\usepackage[margin=35mm]{geometry}
\usepackage{hyperref}
\usepackage{hyperxmp}
\usepackage[usenames]{color}

\hypersetup{colorlinks=true}
\hypersetup{pdfstartview=FitH}
\hypersetup{pdfpagemode=UseNone}
\hypersetup{pdfsource={}}
\hypersetup{pdflang={en-UK}}
\hypersetup{pdfcopyright={Copyright 2017-2018 Niklas Beisert.
  This work may be distributed and/or modified under the
  conditions of the LaTeX Project Public License, either version 1.3
  of this license or (at your option) any later version.}}
\hypersetup{pdflicenseurl={http://www.latex-project.org/lppl.txt}}
\hypersetup{pdfcontactaddress={ETH Zurich, ITP, HIT K,
  Wolfgang-Pauli-Strasse 27}}
\hypersetup{pdfcontactpostcode={8093}}
\hypersetup{pdfcontactcity={Zurich}}
\hypersetup{pdfcontactcountry={Switzerland}}
\hypersetup{pdfcontactemail={nbeisert@itp.phys.ethz.ch}}
\hypersetup{pdfcontacturl={http://people.phys.ethz.ch/\xmptilde nbeisert/}}

\newcommand{\secref}[1]{\hyperref[#1]{section \ref*{#1}}}

\parskip1ex
\parindent0pt
\let\olditemize\itemize
\def\itemize{\olditemize\parskip0pt}

\begin{document}

\title{The \textsf{childdoc} Package}
\hypersetup{pdftitle={The childdoc Package}}
\author{Niklas Beisert\\[2ex]
  Institut f\"ur Theoretische Physik\\
  Eidgen\"ossische Technische Hochschule Z\"urich\\
  Wolfgang-Pauli-Strasse 27, 8093 Z\"urich, Switzerland\\[1ex]
  \href{mailto:nbeisert@itp.phys.ethz.ch}
  {\texttt{nbeisert@itp.phys.ethz.ch}}}
\hypersetup{pdfauthor={Niklas Beisert}}
\hypersetup{pdfsubject={Manual for the LaTeX2e Package childdoc}}
\date{30 December 2018, \textsf{v2.0}}
\maketitle

\begin{abstract}\noindent
\textsf{childdoc} is a \LaTeXe{} package
that enables the direct compilation
of document sections included by |\include|
to individual files.
\end{abstract}

\begingroup
\parskip0ex
\tableofcontents
\endgroup

%%%%%%%%%%%%%%%%%%%%%%%%%%%%%%%%%%%%%%%%%%%%%%%%%%%%%%%%%%%%%%%%%%%%%%%%%%%%%%%%
%%%%%%%%%%%%%%%%%%%%%%%%%%%%%%%%%%%%%%%%%%%%%%%%%%%%%%%%%%%%%%%%%%%%%%%%%%%%%%%%
\section{Introduction}

\LaTeX{} provides a mechanism to structure a large document (such as a book)
into a main file and several child files (containing the chapters)
using the |\include| command.
This mechanism is beneficial for documents
which span hundreds of pages in order to
make the source file(s) more manageable.
Moreover, compilation can be restricted to
selected child files by means of the |\includeonly| command.
The latter feature can be used to reduce the compilation time while editing
(this was significantly more useful in the earlier days of \LaTeX{})
or to generate a smaller document which is easier to navigate.
Another application of |\includeonly| is to generate
documents consisting of selected parts of the complete document.

However, there are a few drawbacks of the plain |\include| mechanism:
\begin{itemize}
\item
The child files cannot be compiled on their own,
they can only be compiled via the main file.
A naive editing environment
(such as a text editor with an option
to have the current file processed by \LaTeX)
may require one to switch to the main file before compiling;
attempting to compile the child file produces errors.
\item
The main file must be modified (each time)
to adjust the |\includeonly| command
to the present needs. This easily leaves the main file in a messy state.
\item
The generated document will always carry the filename
of the main document. This is inconvenient if
several child files are to be compiled and
to be kept for distribution.
\end{itemize}

The present package provides a simple interface
to make child files individually compilable by \LaTeX{}.
Compiling a child file then has the same effect as compiling
the main file with an |\includeonly| command
to select the appropriate child.
Moreover the generated document will carry the name of the child
rather than the main file.
This resolves all three above issues.

This feature is meant to make the editing of books,
thesis documents and lecture notes somewhat more convenient.
However, the package can also be used efficiently for
composing a series of documents (such as exercise sheets)
which are typically distributed individually.
It then assists the author in generating the individual documents
(potentially in different versions)
as well as a document containing the collected series.
Another application is in developing style files
or other kinds of included material
where compilation of the style file could redirect
to a sample or test file.

%%%%%%%%%%%%%%%%%%%%%%%%%%%%%%%%%%%%%%%%%%%%%%%%%%%%%%%%%%%%%%%%%%%%%%%%%%%%%%%%
%%%%%%%%%%%%%%%%%%%%%%%%%%%%%%%%%%%%%%%%%%%%%%%%%%%%%%%%%%%%%%%%%%%%%%%%%%%%%%%%
\section{Usage}

First of all, the package \textsf{childdoc} is \emph{not} a standard
\LaTeXe{} |.sty| style file! Therefore it needs to be invoked in
a non-standard way.

%%%%%%%%%%%%%%%%%%%%%%%%%%%%%%%%%%%%%%%%%%%%%%%%%%%%%%%%%%%%%%%%%%%%%%%%%%%%%%%%
\subsection{Included Files}
\label{sec:include}

%%%%%%%%%%%%%%%%%%%%%%%%%%%%%%%%%%%%%%%%
\DescribeMacro{\childdocmain}
To use the package, add the commands
\begin{center}
\begin{tabular}{l}
|\input{childdoc.def}|\\
|\childdocmain{}|\\
\end{tabular}
\end{center}
at the very top of the main \LaTeX{} file,
in particular \emph{before} the |\documentclass| statement!
The argument of |\childdocmain| should be left empty
(but it must be present).

%%%%%%%%%%%%%%%%%%%%%%%%%%%%%%%%%%%%%%%%
\DescribeMacro{\childdocof}
Furthermore, add the commands
\begin{center}
\begin{tabular}{l}
|\input{childdoc.def}|\\
|\childdocof{|\textit{main}|}|\\
\end{tabular}
\end{center}
at the top of every child file \textit{child}
which is included by |\include{|\textit{child}|}|
from within the main file
(or at least for those files to be compiled individually).
The argument \textit{main} must be the filename of the main file.

There are a couple of
considerations in setting up the main and child documents:

%%%%%%%%%%%%%%%%%%%%%%%%%%%%%%%%%%%%%%%%
\paragraph{Restrictions.}

Please note the following restrictions:
\begin{itemize}
\item
|\childdocmain| must be called with one argument \textit{main}
to ensure compatibility with earlier version of the package.
It must either be empty (|\childdocmain{}|)
or precisely match the filename of the main file in which it is specified.
See \secref{sec:detection} for further information.
\item
The filename \textit{main} must be specified without the |.tex| extension.
\item
The filename \textit{main} is case sensitive
(even in case-insensitive file systems)
due to internal string comparison.
\item
The argument \textit{main} should be fully expanded, it cannot be a macro.
\item
Subdirectories and special characters should be avoided in filenames.
\item
The command |\childdocmain{|\textit{main}|}| must be followed by a whitespace.
It should not be followed immediately by another command
or by a comment mark `|%|'.
This is because the \TeX{} parser reads the token immediately following
the argument of |\childdocmain| and puts it
at the beginning of every child section;
however, a white\-space is ignored.
\end{itemize}

%%%%%%%%%%%%%%%%%%%%%%%%%%%%%%%%%%%%%%%%
\paragraph{Content of Main File.}

It is advisable to place all content in the child files included by |\include|.
Any output contained in the main file will appear in all child documents
unless suppressed manually;
it cannot be suppressed automatically by the |\includeonly| directive
and thus should normally be avoided.
A method to include some content in the main file
by means of conditional processing is described in \secref{sec:conditional}.

%%%%%%%%%%%%%%%%%%%%%%%%%%%%%%%%%%%%%%%%
\paragraph{Page Numbering.}

When only a part of the document is compiled,
the appropriate numbering of pages
(as well as other status parameters)
is determined from the |.aux| files.
The latter contain information from previous passes.
However this information needs to propagate through
all intermediate child documents.
Therefore the page numbering in child documents may well
be inconsistent until the complete document is compiled at least once.

A useful (if unconventional) way to always ensure a consistent
page numbering is to restart the numbering in each child document
and denote the pages by `\textit{child}|.|\textit{page}'
where \textit{child} represents the chapter/section number of the child file.
This can be achieved by the command
|\numberwithin{page}{|\textit{child}|}|
of the \textsf{amsmath} package
where \textit{child} can be |chapter| or |section|
depending on the chosen structuring.
Alternatively, one can modify the macro |\thepage| appropriately
and reset the counter |page| at the start of each child file.

%%%%%%%%%%%%%%%%%%%%%%%%%%%%%%%%%%%%%%%%%%%%%%%%%%%%%%%%%%%%%%%%%%%%%%%%%%%%%%%%
\subsection{Conditional Processing}
\label{sec:conditional}

The package provides a mechanism to compile different versions
of a document. To customise the versions further some conditional processing
can come in handy to distinguish which version is being compiled.
The package provides two macros to describe the compilation context:

%%%%%%%%%%%%%%%%%%%%%%%%%%%%%%%%%%%%%%%%
\DescribeMacro{\ifchilddoc}
The conditional |\ifchilddoc| distinguishes between the compilation of
child documents and the main document:
%
\begin{center}
|\ifchilddoc |\textit{child-code}| |[|\||else |\textit{main-code}]| \||fi|
\end{center}

%%%%%%%%%%%%%%%%%%%%%%%%%%%%%%%%%%%%%%%%
\DescribeMacro{\childdocname}
\DescribeMacro{\childdocjob}
The macro |\childdocname| contains the filename (without extension)
of the main or child file being processed.
Note that |\childdocjob| will always contain the name of the main file.

%%%%%%%%%%%%%%%%%%%%%%%%%%%%%%%%%%%%%%%%
\paragraph{Title Page.}

Conditional processing can be used to include a title or banner page
in the main document when proper precautions are taken.
Importantly, the code in the main file should ensure that the page counter
(as well as other status parameters which are stored in the |.aux| files)
takes the same value after the conditional processing.
Otherwise the page numbers may take divergent values
depending on which part is compiled.

For example, a title page could be declared by:
%
\begin{center}
\begin{tabular}{l}
|\ifchilddoc\||else|\\
|\addtocounter{page}{-1}|\\
\textit{code for title page}\\
|\newpage|\\
|\||fi|
\end{tabular}
\end{center}
%
A banner page for the child documents can be generated by:
%
\begin{center}
\begin{tabular}{l}
|\ifchilddoc|\\
|\addtocounter{page}{-1}|\\
\textit{code for banner page}\\
|\newpage|\\
|\||fi|
\end{tabular}
\end{center}
%
Here one could write a message such as:
\begin{center}
|This is the part \childdocname{} of \childdocjob{}.|
\end{center}

%%%%%%%%%%%%%%%%%%%%%%%%%%%%%%%%%%%%%%%%%%%%%%%%%%%%%%%%%%%%%%%%%%%%%%%%%%%%%%%%
\subsection{Flags}
\label{sec:flags}

The package makes it easy to generate different versions
of the main or child documents.
To this end compilation flags can be defined
and assigned different default values.
They will be particularly useful in conjunction
with the forwarding mechanism described in \secref{sec:forward}.

For example, it may be useful to have a flag |\version|
which can be set to |draft| or |final|.
The document source will contain some conditional code
depending on the value of |\version|.
Suppose further, the flag should default to |final| for the main file
and to |draft| for child files
which is a natural assignment for editing the document.
This is achieved by placing the following code
in the preamble of the main document
(below the |\childdocmain| directive):
%
\begin{center}
\begin{tabular}{l}
|\ifchilddoc|\\
|\providecommand{\version}{draft}|\\
|\||else|\\
|\providecommand{\version}{final}|\\
|\||fi|
\end{tabular}
\end{center}
%
The definition by |\providecommand| makes sure
that previous definitions are not overwritten.
Further statements |\providecommand{\version}{...}|
can thus be added before the above code to override it.

For the main file, one might add a line
(between |\childdocmain| and the above block)
%
\begin{center}
|%\ifchilddoc\||else\providecommand{\version}{draft}\||fi|
\end{center}
%
which can be uncommented to produce a draft version.
Likewise one can add a line to the very top of a child file
(above the |\childdocof{|\textit{main}|}| directive)
%
\begin{center}
|%\providecommand{\version}{final}|
\end{center}
%
which can be uncommented to produce the final version of this child document.

%%%%%%%%%%%%%%%%%%%%%%%%%%%%%%%%%%%%%%%%%%%%%%%%%%%%%%%%%%%%%%%%%%%%%%%%%%%%%%%%
\subsection{Forwarding}
\label{sec:forward}

Different versions of the main or child documents
using compilation flags as described in \secref{sec:flags}
can be (permanently) stored in different files
for convenient compilation, viewing and distribution.
To this end, the package defines a command
to pass on compilation to a different file:

%%%%%%%%%%%%%%%%%%%%%%%%%%%%%%%%%%%%%%%%
\DescribeMacro{\childdocforward}
The command |\childdocforward| redirects processing to
another source file:
%
\begin{center}
\begin{tabular}{l}
|\input{childdoc.def}|\\
|\childdocforward[|\textit{main}|]{|\textit{dest}|}|\\
\end{tabular}
\end{center}
%
The argument \textit{dest} is the destination file
(without extension).
It should be the main file or one of the child files.
Note that further \textsf{childdoc} directives
such as |\childdocof| and |\childdocforward|
in the indicated file will be processed in this form.
The optional argument \textit{main}
passes on directly to the main file \textit{main}
while pretending to compile the child \textit{dest}.
This form behaves as if \textit{dest}
issues |\childdocof{|\textit{main}|}| right away,
and no further \textsf{childdoc} directives will be processed.

%%%%%%%%%%%%%%%%%%%%%%%%%%%%%%%%%%%%%%%%
\DescribeMacro{\...prefix}
In the alternative form |\childdocforwardprefix|,
%
\begin{center}
\begin{tabular}{l}
|\input{childdoc.def}|\\
|\childdocforwardprefix[|\textit{main}|]{|\textit{prefix}|}{|\textit{dest}|}|
\end{tabular}
\end{center}
%
the destination file is determined by a pattern
depending on the current file:
To make this work, the current file must be called
`{\textit{prefix}\hspace{0.2em}\textit{suffix}}'
with \textit{prefix} matching precisely the argument.
Processing is then passed on to the file
`{\textit{dest}\hspace{0.2em}\textit{suffix}}'.
Surely, the same effect is achieved by
directly specifying the
argument `{\textit{dest}\hspace{0.2em}\textit{suffix}}'
in the first form.
However, that requires to set up a different file
for each child. With the alternative form of the command
all these files can have exactly the same content
which simplifies setting them up and maintaining them.

For example, the following file |draft.tex|
with a compilation flag |\version| as described in \secref{sec:flags}
compiles the main document as a draft:
%
\begin{center}
\begin{tabular}{l}
|\def\version{draft}|\\
|\input{childdoc.def}|\\
|\childdocforward{|\textit{main}|}|
\end{tabular}
\end{center}
%
Likewise, the following files |final|\textit{nn}|.tex|
compile the final version of the child document
|child|\textit{nn}|.tex|:
%
\begin{center}
\begin{tabular}{l}
|\def\version{final}|\\
|\input{childdoc.def}|\\
|\childdocforwardprefix{final}{child}|
\end{tabular}
\end{center}
%

Note that when several versions of a main file and/or of each child file
are to be generated, it may be convenient to set up a |Makefile| or
shell script to automatise the process.

%%%%%%%%%%%%%%%%%%%%%%%%%%%%%%%%%%%%%%%%%%%%%%%%%%%%%%%%%%%%%%%%%%%%%%%%%%%%%%%%
\subsection{Command Line Processing}
\label{sec:commandline}

The effect of redirection files can also be achieved by invoking
the \LaTeX{} compiler with a more elaborate command line.
Most conveniently this should be done as part
of a shell script or a |Makefile|.

When using \textsf{childdoc} in the main file, the following
command lines effectively perform a redirection
(note that depending on the shell being used,
backslashes may have to be doubled: `|\|' $\to$ `|\\|'):
%
\begin{center}
|... -jobname "|\textit{target}|" |\\|"|[\textit{flags}]%
|\input{childdoc.def}\childdocforward[|\textit{main}|]{|\textit{dest}|}"|
\end{center}
%
Here \textit{target} is the name of the output file,
\textit{main} is the name of the main file
and \textit{dest} is the name of the main or child file to be processed
(all filenames without extensions).
The optional argument \textit{main} can be omitted
if \textit{main} matches \textit{dest}.
Optionally, compilation \textit{flags} can be defined via |\def| commands.
This command line makes the \TeX{} engine believe
it is compiling the file \textit{target}
whose content is specified as the latter parameter.
The provided code then forwards the processing to
\textit{main} or \textit{dest} as described in \secref{sec:forward}.

%%%%%%%%%%%%%%%%%%%%%%%%%%%%%%%%%%%%%%%%%%%%%%%%%%%%%%%%%%%%%%%%%%%%%%%%%%%%%%%%
\subsection{Include by Input}
\label{sec:input}

Including child documents by |\include| has some restrictions by design.
Most notably, the content of a child document always occupies
its own set of pages; pages cannot be shared between child documents.
Usually, this behaviour makes perfect sense
because each child document contain an essential part of the document.
However, in some situations it may be desirable to compose
a document from a collection of parts
without having mandatory page breaks between then.
For this case, the package
provides a mechanism to include parts
by |\input| which can also be processed individually.
However, by construction this mechanism
requires manual handling of the content to be output.

%%%%%%%%%%%%%%%%%%%%%%%%%%%%%%%%%%%%%%%%
\DescribeMacro{\ifchilddocmanual}
The main file should be prepared as usual, see \secref{sec:include}.
However, the document body must make a distinction
between processing of an individual part and of the main document, e.g.:
%
\begin{center}
\begin{tabular}{l}
|\ifchilddocmanual|\\
|\input{\childdocname}|\\
|\||else|\\
\textit{document body with }|\input{|\textit{part}|}|\\
|\||fi|
\end{tabular}
\end{center}
%
The conditional |\ifchilddocmanual| is true whenever
a part to be included by |\input| is being compiled,
and the name of the part is stored in |\childdocname|.

%%%%%%%%%%%%%%%%%%%%%%%%%%%%%%%%%%%%%%%%
\DescribeMacro{\childdocby}
Each part to be included by |\input| should start with:
%
\begin{center}
\begin{tabular}{l}
|\input{childdoc.def}|\\
|\childdocby{|\textit{main}|}|\\
\end{tabular}
\end{center}
%
The directive |\childdocby| is similar to |\childdocof|
described in \secref{sec:include},
but the subsequent selection of content must be done manually.
To that end, both |\ifchilddoc| and |\ifchilddocmanual|
will be true upon processing of a part,
and the name of the part is stored in |\childdocname|.
Note that |\jobname| will be set to the filename of the current part
so that each part receives an individual |.aux| file
that does not interfere with the |.aux| file(s) of the main document.
This behaviour can be altered by the alternative form
|\childdocby[*]{|\textit{main}|}| (with a non-empty optional argument)
which uses the |.aux| file of the main document
by setting |\jobname| to \textit{main}.

%%%%%%%%%%%%%%%%%%%%%%%%%%%%%%%%%%%%%%%%%%%%%%%%%%%%%%%%%%%%%%%%%%%%%%%%%%%%%%%%
\subsection{Driver Development}
\label{sec:driver}

The \textsf{childdoc} mechanism can also be use for the development
of definition files such as \LaTeX{} styles or classes.
This case differs from the above setup with multiple parts
included by |\include| in that no |\includeonly| should be invoked.
This can be achieved by starting the include file
(before |\ProvidesPackage|) with:
%
\begin{center}
\begin{tabular}{l}
|\input{childdoc.def}|\\
|\childdocforward{|\textit{main}|}|\\
\end{tabular}
\end{center}
%
or alternatively with:
%
\begin{center}
\begin{tabular}{l}
|\input{childdoc.def}|\\
|\childdocby{|\textit{main}|}|\\
\end{tabular}
\end{center}
%
Both forms have slightly different effects as described above.
The main file is prepared as usual, see \secref{sec:include}.

%%%%%%%%%%%%%%%%%%%%%%%%%%%%%%%%%%%%%%%%%%%%%%%%%%%%%%%%%%%%%%%%%%%%%%%%%%%%%%%%
\subsection{Legacy Detection}
\label{sec:detection}

The directive |\childdocmain| in the main file can detect
whether the complete document or merely a child is to be compiled
even without using the directive |\childdocof|.
This method is deprecated because it is less robust
and there is no compelling reason to use it;
it is merely provided for backward compatibility
and it may be removed in future versions.

If the detection mechanism is to be used,
it is mandatory to correctly specify
the filename of the main file as the argument of |\childdocmain|:
%
\begin{center}
\begin{tabular}{l}
|\input{childdoc.def}|\\
|\childdocmain{|\textit{main}|}|\\
\end{tabular}
\end{center}
%
If |\jobname| does not match the argument \textit{main} of |\childdocmain|,
it is assumed that |\jobname| points to the child file to be compiled.
When using |\childdocmain| with the main file specified as argument,
it suffices to start a child file
with just |\input{|\textit{main}|}|
without loading of the package and using |\childdocof|.
If instead all processing is done
with the appropriate \textsf{childdoc} directives,
the argument of \textit{main} of |\childdocmain| can be empty.

An alternative version of the command line processing described
in \secref{sec:commandline} using the detection mechanism reads:
%
\begin{center}
|... -jobname "|\textit{target}|" "|[\textit{flags}]%
[|\def\jobname{|\textit{dest}|}|]|\input{|\textit{main}|}"|
\end{center}

%%%%%%%%%%%%%%%%%%%%%%%%%%%%%%%%%%%%%%%%%%%%%%%%%%%%%%%%%%%%%%%%%%%%%%%%%%%%%%%%
\subsection{Manual Code}
\label{sec:manual}

In case one cannot be certain whether the definitions file |childdoc.def|
is installed on the target \TeX{} distribution
and one prefers not to ship it,
it is conceivable to paste a few relevant commands into the sources.

To that end, drop all statements |\input{childdoc.def}|
and perform the replacements as outlined below.
Instead of |\childdocmain{|\textit{main}|}| add the following code
to the top of the main file:
%
\begin{center}
\begin{tabular}{l}
|\||ifdefined\childdocname\endinput\||fi\newif\ifchilddoc|\\
|\edef\childdocname{\scantokens\expandafter{\jobname\noexpand}}|\\
|\def\childdocmain{|\textit{main}|}\||ifx\childdocmain\childdocname\||else|\\
|\childdoctrue\includeonly{\childdocname}\let\jobname\childdocmain\||fi|\\
\end{tabular}
\end{center}
%
Instead of |\childdocof{|\textit{main}|}| just include the main file
at the top of each child file:
%
\begin{center}
|\input{|\textit{main}|}|
\end{center}
%
A simple redirection |\childdocforward{|\textit{dest}|}| is achieved by:
%
\begin{center}
|\def\jobname{|\textit{dest}|}\input{\jobname}|
\end{center}
%
The redirection with prefix
|\childdocforwardprefix[|\textit{prefix}|]{|\textit{dest}|}|
is accomplished by:
%
\begin{center}
\begin{tabular}{l}
|{\edef\jobname{\scantokens\expandafter{\jobname\noexpand}}|\\
|\def\redirectjob |\textit{prefix}|#1~~~{\gdef\jobname{|\textit{dest}|#1}}|\\
|\expandafter\redirectjob\jobname~~~}\input{\jobname}|
\end{tabular}
\end{center}

In an alternative approach,
child documents can be compiled by a specific command line
without additional code or specific definitions:
%
\begin{center}
|... -jobname "|\textit{target}|" "|[\textit{flags}]%
|\includeonly{|\textit{dest}|}\input{|\textit{main}|}"|
\end{center}
%

%%%%%%%%%%%%%%%%%%%%%%%%%%%%%%%%%%%%%%%%%%%%%%%%%%%%%%%%%%%%%%%%%%%%%%%%%%%%%%%%
%%%%%%%%%%%%%%%%%%%%%%%%%%%%%%%%%%%%%%%%%%%%%%%%%%%%%%%%%%%%%%%%%%%%%%%%%%%%%%%%
\section{Information}

%%%%%%%%%%%%%%%%%%%%%%%%%%%%%%%%%%%%%%%%%%%%%%%%%%%%%%%%%%%%%%%%%%%%%%%%%%%%%%%%
\subsection{Copyright}

Copyright \copyright{} 2017--2018 Niklas Beisert

This work may be distributed and/or modified under the
conditions of the \LaTeX{} Project Public License, either version 1.3
of this license or (at your option) any later version.
The latest version of this license is in
  \url{http://www.latex-project.org/lppl.txt}
and version 1.3 or later is part of all distributions of \LaTeX{}
version 2005/12/01 or later.

This work has the LPPL maintenance status `maintained'.

The Current Maintainer of this work is Niklas Beisert.

This work consists of the files |README.txt|, |childdoc.ins| and |childdoc.dtx|
as well as the derived files |childdoc.def|, |cdocsamp.tex|
with |cdocsch1.tex|, |cdocsch2.tex|, |cdocspt3.tex|, |cdocspt4.tex|,
|cdocsdrf.tex|, |cdocsfn1.tex|, |cdocsfn2.tex|
as well as |childdoc.pdf|.

%%%%%%%%%%%%%%%%%%%%%%%%%%%%%%%%%%%%%%%%%%%%%%%%%%%%%%%%%%%%%%%%%%%%%%%%%%%%%%%%
\subsection{Files and Installation}

The package consists of the files:
%
\begin{center}
\begin{tabular}{ll}
    |README.txt|   & readme file \\
    |childdoc.ins| & installation file \\
    |childdoc.dtx| & source file \\
    |childdoc.def| & definition file \\
    |cdocsamp.tex| & sample main file \\
    |cdocsch1.tex| & sample include file \\
    |cdocsch2.tex| & sample include file \\
    |cdocspt3.tex| & sample part file \\
    |cdocspt4.tex| & sample part file \\
    |cdocsdrf.tex| & sample redirection file \\
    |cdocsfn1.tex| & sample redirection file \\
    |cdocsfn2.tex| & sample redirection file \\
    |childdoc.pdf| & manual
\end{tabular}
\end{center}
%
The distribution consists of the files
|README.txt|, |childdoc.ins| and |childdoc.dtx|.
%
\begin{itemize}
\item
Run (pdf)\LaTeX{} on |childdoc.dtx|
to compile the manual |childdoc.pdf| (this file).
\item
Run \LaTeX{} on |childdoc.ins| to create the definitions file |childdoc.def|
and the sample |cdocsamp.tex| with include files
|cdocsch1.tex|, |cdocsch2.tex|, |cdocspt3.tex|, |cdocspt4.tex|,
|cdocsdrf.tex|, |cdocsfn1.tex|, |cdocsfn2.tex|.
Then copy the file |childdoc.def| to an appropriate directory of your \LaTeX{}
distribution, e.g.\ \textit{texmf-root}|/tex/latex/childdoc|.
\end{itemize}

%%%%%%%%%%%%%%%%%%%%%%%%%%%%%%%%%%%%%%%%%%%%%%%%%%%%%%%%%%%%%%%%%%%%%%%%%%%%%%%%
\subsection{Related CTAN Packages}

There are several other packages which offer a similar functionality:
%
\begin{itemize}
\item
The packages
\href{http://ctan.org/pkg/docmute}{\textsf{docmute}},
\href{http://ctan.org/pkg/includex}{\textsf{includex}} and
\href{http://ctan.org/pkg/standalone}{\textsf{standalone}}
provide commands to include only the document body of
a child file thus allowing both files to be compiled individually.
\item
The packages \href{http://ctan.org/pkg/subdocs}{\textsf{subdocs}}
and \href{http://ctan.org/pkg/subfiles}{\textsf{subfiles}}
provide structures in which the main and child documents can be
encapsulated and allowing them to be compiled individually.
The inclusion mechanism is different from the conventional |\include|.
\item
The package \href{http://ctan.org/pkg/combine}{\textsf{combine}}
is an elaborate solution to combine several documents into one.
\end{itemize}
%
See also the CTAN topic \href{http://ctan.org/topic/subdocs}{\textsf{subdocs}}
for further related packages.
The present package differs from the above solutions in that
a document structure constructed with the conventional |\include| mechanism
just needs two extra commands at the top of every file
such that all constituent files can be compiled individually.

%%%%%%%%%%%%%%%%%%%%%%%%%%%%%%%%%%%%%%%%%%%%%%%%%%%%%%%%%%%%%%%%%%%%%%%%%%%%%%%%
%\subsection{Feature Suggestions}
%
%The following is a list of features which may be useful for future
%versions of this package:
%%
%\begin{itemize}
%\item
%\ldots
%\end{itemize}

%%%%%%%%%%%%%%%%%%%%%%%%%%%%%%%%%%%%%%%%%%%%%%%%%%%%%%%%%%%%%%%%%%%%%%%%%%%%%%%%
\subsection{Revision History}

%%%%%%%%%%%%%%%%%%%%%%%%%%%%%%%%%%%%%%%%
\paragraph{v2.0:} 2018/12/30

\begin{itemize}
\item
immediate forward processing
\item
added |\childdocby| mechanism
\item
manual restructured
\end{itemize}

%%%%%%%%%%%%%%%%%%%%%%%%%%%%%%%%%%%%%%%%
\paragraph{v1.6:} 2018/01/17

\begin{itemize}
\item
application for development of include files
\item
corrections to manual
\end{itemize}

%%%%%%%%%%%%%%%%%%%%%%%%%%%%%%%%%%%%%%%%
\paragraph{v1.5:} 2017/05/21

\begin{itemize}
\item
more complete structuring introduced
\item
|\childdocof| introduced
\item
|\childdoc| renamed to |\childdocmain|
\item
|\childredirect| renamed to |\childdocforward| and |\childdocforwardprefix|
and functionality expanded
\end{itemize}

%%%%%%%%%%%%%%%%%%%%%%%%%%%%%%%%%%%%%%%%
\paragraph{v1.0:} 2017/04/27

\begin{itemize}
\item
manual and install package
\item
first version published on CTAN
\end{itemize}

%%%%%%%%%%%%%%%%%%%%%%%%%%%%%%%%%%%%%%%%
\paragraph{v0.6:} 2017/04/26

\begin{itemize}
\item
redirection mechanism added
\end{itemize}

%%%%%%%%%%%%%%%%%%%%%%%%%%%%%%%%%%%%%%%%
\paragraph{v0.5:} 2017/04/26

\begin{itemize}
\item
functionality in definition file
\end{itemize}


%%%%%%%%%%%%%%%%%%%%%%%%%%%%%%%%%%%%%%%%%%%%%%%%%%%%%%%%%%%%%%%%%%%%%%%%%%%%%%%%
%%%%%%%%%%%%%%%%%%%%%%%%%%%%%%%%%%%%%%%%%%%%%%%%%%%%%%%%%%%%%%%%%%%%%%%%%%%%%%%%
%%%%%%%%%%%%%%%%%%%%%%%%%%%%%%%%%%%%%%%%%%%%%%%%%%%%%%%%%%%%%%%%%%%%%%%%%%%%%%%%
\appendix

\settowidth\MacroIndent{\rmfamily\scriptsize 000\ }

 \DocInput{childdoc.dtx}

\end{document}
%</driver>
% \fi
%
% %%%%%%%%%%%%%%%%%%%%%%%%%%%%%%%%%%%%%%%%%%%%%%%%%%%%%%%%%%%%%%%%%%%%%%%%%%%%%%
% %%%%%%%%%%%%%%%%%%%%%%%%%%%%%%%%%%%%%%%%%%%%%%%%%%%%%%%%%%%%%%%%%%%%%%%%%%%%%%
% \section{Sample}
%\iffalse
%<*samplemain>
%\fi
%
% The following presents a sample document
% with two chapters, two parts, a title page,
% a compile flag as well as three forwarding files to set the flag.
% It consists of eight |.tex| files:
% \begin{center}
% \begin{tabular}{ll}
% |cdocsamp.tex|&main file\\
% |cdocsch1.tex|&include file for chapter 1\\
% |cdocsch2.tex|&include file for chapter 2\\
% |cdocspt3.tex|&include file for part 3\\
% |cdocspt4.tex|&include file for part 4\\
% |cdocsdrf.tex|&forwarding file for main file in draft mode\\
% |cdocsfi1.tex|&forwarding file for final version of chapter 1\\
% |cdocsfi2.tex|&forwarding file for final version of chapter 2\\
% \end{tabular}
% \end{center}
% Each of the eight files can be compiled directly by the \LaTeX{} compiler.
%
% %%%%%%%%%%%%%%%%%%%%%%%%%%%%%%%%%%%%%%
% \paragraph{Main File.}
%
% The main file is called |cdocsamp.tex|.
%
% Load the \textsf{childdoc} definitions and
% declare the filename for the main document:
%    \begin{macrocode}
\input{childdoc.def}
\childdocmain{}
%    \end{macrocode}

% Optional override for |\version| flag:
%    \begin{macrocode}
%%\ifchilddoc\else\providecommand{\version}{draft}\fi
%    \end{macrocode}

% Define the default values for the |\version| flag
% (|final| for the main file and |draft| for childs):
%    \begin{macrocode}
\ifchilddoc
\providecommand{\version}{draft}
\else
\providecommand{\version}{final}
\fi
%    \end{macrocode}

% Load the standard document class:
%    \begin{macrocode}
\documentclass[12pt]{article}
%    \end{macrocode}

% Start the document body:
%    \begin{macrocode}
\begin{document}
%    \end{macrocode}

% Declare a title page.
% Print title, part of document being processed and version flag:
%    \begin{macrocode}
\addtocounter{page}{-1}
\begin{center}
{\LARGE\bfseries{}childdoc example\par}
\vspace{1cm}
\ifchilddoc
\ifchilddocmanual part\else chapter\fi:
`\childdocname' of `\childdocjob'\par
\else
main document: `\childdocjob'\par
\fi
version: \version\par
\end{center}
\newpage
%    \end{macrocode}

% Manually include selected file,
% otherwise process as usual:
%    \begin{macrocode}
\ifchilddocmanual
\section*{part `\childdocname'}
\input{\childdocname}
\else
%    \end{macrocode}

% Include the two chapters:
%    \begin{macrocode}
\include{cdocsch1}
\include{cdocsch2}
%    \end{macrocode}

% Include the two parts unless only chapters should be displayed:
%    \begin{macrocode}
\ifchilddoc\else
\section{part three}
\input{cdocspt3}
\section{part four}
\input{cdocspt4}
\fi
%    \end{macrocode}

% Process as usual until here:
%    \begin{macrocode}
\fi
%    \end{macrocode}

% End of document body:
%    \begin{macrocode}
\end{document}
%    \end{macrocode}
%\iffalse
%</samplemain>
%\fi
%
% %%%%%%%%%%%%%%%%%%%%%%%%%%%%%%%%%%%%%%
% \paragraph{Chapter Include Files.}
%
% The include files are called |cdocsch1.tex| and |cdocsch2.tex|.
%
%\iffalse
%<*samplechap1|samplechap2>
%\fi

% Optional override for |\version| flag:
%    \begin{macrocode}
%%\providecommand{\version}{final}
%    \end{macrocode}

% Include the main document:
%    \begin{macrocode}
\input{childdoc.def}
\childdocof{cdocsamp}
%    \end{macrocode}

%\iffalse
%</samplechap1|samplechap2>
%\fi
%
%\iffalse
%<*samplechap1>
%\fi
% Some text for chapter 1:
%    \begin{macrocode}
\section{one}
some text in chapter one
%    \end{macrocode}

%\iffalse
%</samplechap1>
%\fi
% Some text for chapter 2:
%\iffalse
%<*samplechap2>
%\fi
%    \begin{macrocode}
\section{two}
more text in chapter two
%    \end{macrocode}

%\iffalse
%</samplechap2>
%\fi
%
% %%%%%%%%%%%%%%%%%%%%%%%%%%%%%%%%%%%%%%
% \paragraph{Part Include Files.}
%
% The include files are called |cdocspt3.tex| and |cdocspt4.tex|.
%
%\iffalse
%<*samplepart3|samplepart4>
%\fi

% Optional override for |\version| flag:
%    \begin{macrocode}
%%\providecommand{\version}{final}
%    \end{macrocode}

% Include the main document:
%    \begin{macrocode}
\input{childdoc.def}
\childdocby{cdocsamp}
%    \end{macrocode}

%\iffalse
%</samplepart3|samplepart4>
%\fi
%
%\iffalse
%<*samplepart3>
%\fi
% Some text for part 3:
%    \begin{macrocode}
some text in part three
%    \end{macrocode}

%\iffalse
%</samplepart3>
%\fi
% Some text for part 4:
%\iffalse
%<*samplepart4>
%\fi
%    \begin{macrocode}
more text in part four
%    \end{macrocode}

%\iffalse
%</samplepart4>
%\fi
%
% %%%%%%%%%%%%%%%%%%%%%%%%%%%%%%%%%%%%%%
% \paragraph{Forwarding for a Complete Draft.}
%
% The following forwarding file |cdocsdrf.tex|
% compiles the main document in draft mode:
%\iffalse
%<*sampledraft>
%\fi
%    \begin{macrocode}
\def\version{draft}
\input{childdoc.def}
\childdocforward{cdocsamp}
%    \end{macrocode}

%\iffalse
%</sampledraft>
%\fi
%
% %%%%%%%%%%%%%%%%%%%%%%%%%%%%%%%%%%%%%%
% \paragraph{Forwarding for Final Version of the Chapters.}
%
% The following forwarding files |cdocsfn1.tex| and |cdocsfn2.tex|
% (with identical content)
% compile the final versions of the child documents
% |cdocsch1.tex| and |cdocsch2.tex|, respectively:
%\iffalse
%<*samplefinal>
%\fi
%    \begin{macrocode}
\def\version{final}
\input{childdoc.def}
\childdocforwardprefix[cdocsamp]{cdocsfn}{cdocsch}
%    \end{macrocode}

%\iffalse
%</samplefinal>
%\fi
%
% %%%%%%%%%%%%%%%%%%%%%%%%%%%%%%%%%%%%%%
% \paragraph{Command Line Processing.}
%
% The following three command lines generate the output files
% |cdocscld|, |cdocscl1| and |cdocscl2|
% which should be identical to
% |cdocsdrf|, |cdocsch1| and |cdocsfn2|, respectively:
% \begin{center}
% \begin{tabular}{l}
% |latex -jobname cdocscld \|\\
% |  "\def\version{draft}\input{childdoc.def}\childdocforward{cdocsamp}"|\\
% |latex -jobname cdocscl1 \|\\
% |  "\input{childdoc.def}\childdocforward[cdocsamp]{cdocsch1}"|\\
% |latex -jobname cdocscl2 \|\\
% |  "\def\version{final}\input{childdoc.def}\childdocforward{cdocsch2}"|
% \end{tabular}
% \end{center}
% Note that the trailing backslash on each first line
% merely continues the input to the second line
% (for convenient cut ant paste).
% Furthermore, the command |latex| can be replaced by any
% of its alternative versions such as |pdflatex|.
%
% %%%%%%%%%%%%%%%%%%%%%%%%%%%%%%%%%%%%%%%%%%%%%%%%%%%%%%%%%%%%%%%%%%%%%%%%%%%%%%
% %%%%%%%%%%%%%%%%%%%%%%%%%%%%%%%%%%%%%%%%%%%%%%%%%%%%%%%%%%%%%%%%%%%%%%%%%%%%%%
% \section{Implementation}
%\iffalse
%<*package>
%\fi
%
% This section describes the definitions file |childdoc.def|.

% The definitions cannot be loaded using |\usepackage| or |\RequirePackage|
% which has a mechanism to prevent loading a style file more than once.
% When loading the definitions by means of |\input|
% multiple instances have to be prevented manually:
%\iffalse
%This code needs to be before the `\ProvidesFile' directive
%which is defined at the beginning of this file.
%Therefore it is also placed there and commented out here.
%</package>
%<*discard>
%\fi
%    \begin{macrocode}
\ifdefined\childdocmain\endinput\fi
%    \end{macrocode}
%\iffalse
%</discard>
%<*package>
%\fi
%
% \macro{\ifchilddoc}
% \macro{\ifchilddocmanual}
% The conditional |\ifchilddoc| tells whether a
% child (true) or main (false) document is being compiled.
% The conditional |\ifchilddocmanual| tells whether
% the |\includeonly| mechanism is used (false) or
% the selection of child files must be performed manually (true).
% The definitions initialise to false:
%    \begin{macrocode}
\newif\ifchilddoc
\newif\ifchilddocmanual
%    \end{macrocode}

% \macro{\childdocname}
% \macro{\childdocjob}
% The macro |\childdocname| stores the name of the main document
% to be compiled. The macro |\childdocjob| stores the name of
% the document on which the \LaTeX{} compiler was originally invoked.
% The content of |\jobname| cannot be compared
% to filenames specified in the source due to different catcodes.
% The following code rescans |\jobname|, stores the result
% in |\childdocname| and saves a copy in |\childdocjob|:
%    \begin{macrocode}
\edef\childdocname{\scantokens\expandafter{\jobname\noexpand}}
\let\childdocjob\childdocname
%    \end{macrocode}

% \macro{\childdocdisable}
% The macro |\childdocdisable| prevents the main file
% from being processed more than once.
% At this stage, the main document command |\childdocmain|
% is assumed to be called once again where it should do nothing.
% Any subsequent call to it should prevent
% a secondary processing of the main document
% It overwrites the forwarding commands
% |\childdocof| and |\childdocforward|
% with empty macros to prevent further inclusions of the main document:
%    \begin{macrocode}
\newcommand{\childdocdisable}
{
  \renewcommand{\childdocmain}[1]{\renewcommand{\childdocmain}[1]{\endinput}}
  \renewcommand{\childdocof}[1]{}
  \renewcommand{\childdocby}[2][]{}
  \renewcommand{\childdocforward}[2][]{}
  \renewcommand{\childdocdisable}{}
}
%    \end{macrocode}

% \macro{\childdocmain}
% The macro |\childdocmain| is to be called at the top of the main file
% with nothing or the main filename (without extension) as argument.
% First, it breaks loops.
% If the argument is not empty and does not match |\childdocname|
% (which is set by the first inclusion of |childdoc.def|),
% |\ifchilddoc| is set to true, |\includeonly| is applied to the child file
% and |\jobname| is set to the main file
% (for proper handling of |.aux| files):
%    \begin{macrocode}
\newcommand{\childdocmain}[1]
{
  \childdocdisable\childdocmain{}
  \if?#1?\else
    \begingroup
      \def\childdoctmp{#1}
      \ifx\childdoctmp\childdocname
        \def\childdoctmp{}
      \else
        \def\childdoctmp
        {
          \childdoctrue
          \includeonly{\childdocname}
          \def\childdocjob{#1}
          \def\jobname{#1}
        }
      \fi
      \expandafter
    \endgroup
    \childdoctmp
  \fi
}
%    \end{macrocode}

% \macro{\childdocof}
% The command |\childdocof| redirects
% compilation to the main file |#1|.
%    \begin{macrocode}
\newcommand{\childdocof}[1]
{
  \childdocdisable
  \childdoctrue
  \includeonly{\childdocname}
  \def\jobname{#1}
  \def\childdocjob{#1}
  \input{#1}
}
%    \end{macrocode}

% \macro{\childdocby}
% The command |\childdocby| ....
%    \begin{macrocode}
\newcommand{\childdocby}[2][]
{
  \childdocdisable
  \childdoctrue
  \childdocmanualtrue
  \if?#1?\else
    \def\jobname{#2}
  \fi
  \def\childdocjob{#2}
  \input{#2}
  \endinput
}
%    \end{macrocode}

% \macro{\childdocforward}
% The command |\childdocforward| redirects
% compilation to the main file or
% (if the optional argument is given) a child file.
% Parameters are set as if the main file
% or a child file starting with |\childdocof| was compiled.
% Then compilation is handed over to the main file:
%    \begin{macrocode}
\newcommand{\childdocforward}[2][]
{
  \begingroup
    \if?#1?
      \def\childdoctmp
      {
        \def\childdocname{#2}
        \def\childdocjob{#2}
        \def\jobname{#2}
        \input{#2}
        \endinput
      }
    \else
      \def\childdoctmp
      {
        \childdocdisable
        \def\childdocname{#2}
        \childdoctrue
        \includeonly{#2}
        \def\childdocjob{#1}
        \def\jobname{#1}
        \input{#1}
        \endinput
      }
    \fi
    \expandafter
  \endgroup
  \childdoctmp
}
%    \end{macrocode}

% \macro{\childdocforwardprefix}
% The command |\childdocforwardprefix| redirects
% compilation to the main or a child file by means of a pattern.
% The prefix |#1| in the current filename is replaced by |#2|
% and the suffix of the current filename is kept
% (it is assumed that the filename does not contain the substring `|~~~|'
% which is used as a delimiter).
% Compilation is handed over to the new file by |\childdocforward|:
%    \begin{macrocode}
\newcommand{\childdocforwardprefix}[3][]
{
  \begingroup
    \def\childdocextract #2##1~~~{\def\childdoctmp{\childdocforward[#1]{#3##1}}}
    \expandafter\childdocextract\childdocname~~~
    \expandafter
  \endgroup
  \childdoctmp
}
%    \end{macrocode}

% \macro{\childdoc}
% The deprecated macro |\childdoc| is a legacy version of |\childdocmain|:
%    \begin{macrocode}
\newcommand{\childdoc}{\childdocmain}
%    \end{macrocode}

% \macro{\childdocredirect}
% The deprecated macro |\childdocredirect| is a legacy version
% of |\childdocforward| and |\childdocforwardprefix|:
%    \begin{macrocode}
\newcommand{\childdocredirect}[2][]
{
  \begingroup
    \if?#1?
      \def\childdoctmp{\childdocforward{#2}}
    \else
      \def\childdoctmp{\childdocforwardprefix{#1}{#2}}
    \fi
    \expandafter
  \endgroup
  \childdoctmp
}
%    \end{macrocode}

%\iffalse
%</package>
%\fi
%
\endinput
\childdocforward{cdocsamp}"|\\
% |latex -jobname cdocscl1 \|\\
% |  "% \iffalse
%
% childdoc.dtx Copyright (C) 2017-2018 Niklas Beisert
%
% This work may be distributed and/or modified under the
% conditions of the LaTeX Project Public License, either version 1.3
% of this license or (at your option) any later version.
% The latest version of this license is in
%   http://www.latex-project.org/lppl.txt
% and version 1.3 or later is part of all distributions of LaTeX
% version 2005/12/01 or later.
%
% This work has the LPPL maintenance status `maintained'.
%
% The Current Maintainer of this work is Niklas Beisert.
%
% This work consists of the files childdoc.dtx and childdoc.ins
% and the derived files childdoc.def and cdocsamp.tex with
% cdocsch1.tex, cdocsch2.tex, cdocsdrf.tex, cdocsfn1.tex, cdocsfn2.tex.
%
%<package>\ifdefined\childdocmain\endinput\fi
%<package>\ProvidesFile{childdoc.def}[2018/12/30 v2.0 child document driver]
%<samplemain>\ProvidesFile{cdocsamp.tex}[2018/12/30 v2.0 sample for childdoc]
%<*driver>
%\ProvidesFile{childdoc.drv}[2018/12/30 v2.0 childdoc reference manual file]
\PassOptionsToClass{10pt,a4paper}{article}
\documentclass{ltxdoc}

\usepackage[margin=35mm]{geometry}
\usepackage{hyperref}
\usepackage{hyperxmp}
\usepackage[usenames]{color}

\hypersetup{colorlinks=true}
\hypersetup{pdfstartview=FitH}
\hypersetup{pdfpagemode=UseNone}
\hypersetup{pdfsource={}}
\hypersetup{pdflang={en-UK}}
\hypersetup{pdfcopyright={Copyright 2017-2018 Niklas Beisert.
  This work may be distributed and/or modified under the
  conditions of the LaTeX Project Public License, either version 1.3
  of this license or (at your option) any later version.}}
\hypersetup{pdflicenseurl={http://www.latex-project.org/lppl.txt}}
\hypersetup{pdfcontactaddress={ETH Zurich, ITP, HIT K,
  Wolfgang-Pauli-Strasse 27}}
\hypersetup{pdfcontactpostcode={8093}}
\hypersetup{pdfcontactcity={Zurich}}
\hypersetup{pdfcontactcountry={Switzerland}}
\hypersetup{pdfcontactemail={nbeisert@itp.phys.ethz.ch}}
\hypersetup{pdfcontacturl={http://people.phys.ethz.ch/\xmptilde nbeisert/}}

\newcommand{\secref}[1]{\hyperref[#1]{section \ref*{#1}}}

\parskip1ex
\parindent0pt
\let\olditemize\itemize
\def\itemize{\olditemize\parskip0pt}

\begin{document}

\title{The \textsf{childdoc} Package}
\hypersetup{pdftitle={The childdoc Package}}
\author{Niklas Beisert\\[2ex]
  Institut f\"ur Theoretische Physik\\
  Eidgen\"ossische Technische Hochschule Z\"urich\\
  Wolfgang-Pauli-Strasse 27, 8093 Z\"urich, Switzerland\\[1ex]
  \href{mailto:nbeisert@itp.phys.ethz.ch}
  {\texttt{nbeisert@itp.phys.ethz.ch}}}
\hypersetup{pdfauthor={Niklas Beisert}}
\hypersetup{pdfsubject={Manual for the LaTeX2e Package childdoc}}
\date{30 December 2018, \textsf{v2.0}}
\maketitle

\begin{abstract}\noindent
\textsf{childdoc} is a \LaTeXe{} package
that enables the direct compilation
of document sections included by |\include|
to individual files.
\end{abstract}

\begingroup
\parskip0ex
\tableofcontents
\endgroup

%%%%%%%%%%%%%%%%%%%%%%%%%%%%%%%%%%%%%%%%%%%%%%%%%%%%%%%%%%%%%%%%%%%%%%%%%%%%%%%%
%%%%%%%%%%%%%%%%%%%%%%%%%%%%%%%%%%%%%%%%%%%%%%%%%%%%%%%%%%%%%%%%%%%%%%%%%%%%%%%%
\section{Introduction}

\LaTeX{} provides a mechanism to structure a large document (such as a book)
into a main file and several child files (containing the chapters)
using the |\include| command.
This mechanism is beneficial for documents
which span hundreds of pages in order to
make the source file(s) more manageable.
Moreover, compilation can be restricted to
selected child files by means of the |\includeonly| command.
The latter feature can be used to reduce the compilation time while editing
(this was significantly more useful in the earlier days of \LaTeX{})
or to generate a smaller document which is easier to navigate.
Another application of |\includeonly| is to generate
documents consisting of selected parts of the complete document.

However, there are a few drawbacks of the plain |\include| mechanism:
\begin{itemize}
\item
The child files cannot be compiled on their own,
they can only be compiled via the main file.
A naive editing environment
(such as a text editor with an option
to have the current file processed by \LaTeX)
may require one to switch to the main file before compiling;
attempting to compile the child file produces errors.
\item
The main file must be modified (each time)
to adjust the |\includeonly| command
to the present needs. This easily leaves the main file in a messy state.
\item
The generated document will always carry the filename
of the main document. This is inconvenient if
several child files are to be compiled and
to be kept for distribution.
\end{itemize}

The present package provides a simple interface
to make child files individually compilable by \LaTeX{}.
Compiling a child file then has the same effect as compiling
the main file with an |\includeonly| command
to select the appropriate child.
Moreover the generated document will carry the name of the child
rather than the main file.
This resolves all three above issues.

This feature is meant to make the editing of books,
thesis documents and lecture notes somewhat more convenient.
However, the package can also be used efficiently for
composing a series of documents (such as exercise sheets)
which are typically distributed individually.
It then assists the author in generating the individual documents
(potentially in different versions)
as well as a document containing the collected series.
Another application is in developing style files
or other kinds of included material
where compilation of the style file could redirect
to a sample or test file.

%%%%%%%%%%%%%%%%%%%%%%%%%%%%%%%%%%%%%%%%%%%%%%%%%%%%%%%%%%%%%%%%%%%%%%%%%%%%%%%%
%%%%%%%%%%%%%%%%%%%%%%%%%%%%%%%%%%%%%%%%%%%%%%%%%%%%%%%%%%%%%%%%%%%%%%%%%%%%%%%%
\section{Usage}

First of all, the package \textsf{childdoc} is \emph{not} a standard
\LaTeXe{} |.sty| style file! Therefore it needs to be invoked in
a non-standard way.

%%%%%%%%%%%%%%%%%%%%%%%%%%%%%%%%%%%%%%%%%%%%%%%%%%%%%%%%%%%%%%%%%%%%%%%%%%%%%%%%
\subsection{Included Files}
\label{sec:include}

%%%%%%%%%%%%%%%%%%%%%%%%%%%%%%%%%%%%%%%%
\DescribeMacro{\childdocmain}
To use the package, add the commands
\begin{center}
\begin{tabular}{l}
|\input{childdoc.def}|\\
|\childdocmain{}|\\
\end{tabular}
\end{center}
at the very top of the main \LaTeX{} file,
in particular \emph{before} the |\documentclass| statement!
The argument of |\childdocmain| should be left empty
(but it must be present).

%%%%%%%%%%%%%%%%%%%%%%%%%%%%%%%%%%%%%%%%
\DescribeMacro{\childdocof}
Furthermore, add the commands
\begin{center}
\begin{tabular}{l}
|\input{childdoc.def}|\\
|\childdocof{|\textit{main}|}|\\
\end{tabular}
\end{center}
at the top of every child file \textit{child}
which is included by |\include{|\textit{child}|}|
from within the main file
(or at least for those files to be compiled individually).
The argument \textit{main} must be the filename of the main file.

There are a couple of
considerations in setting up the main and child documents:

%%%%%%%%%%%%%%%%%%%%%%%%%%%%%%%%%%%%%%%%
\paragraph{Restrictions.}

Please note the following restrictions:
\begin{itemize}
\item
|\childdocmain| must be called with one argument \textit{main}
to ensure compatibility with earlier version of the package.
It must either be empty (|\childdocmain{}|)
or precisely match the filename of the main file in which it is specified.
See \secref{sec:detection} for further information.
\item
The filename \textit{main} must be specified without the |.tex| extension.
\item
The filename \textit{main} is case sensitive
(even in case-insensitive file systems)
due to internal string comparison.
\item
The argument \textit{main} should be fully expanded, it cannot be a macro.
\item
Subdirectories and special characters should be avoided in filenames.
\item
The command |\childdocmain{|\textit{main}|}| must be followed by a whitespace.
It should not be followed immediately by another command
or by a comment mark `|%|'.
This is because the \TeX{} parser reads the token immediately following
the argument of |\childdocmain| and puts it
at the beginning of every child section;
however, a white\-space is ignored.
\end{itemize}

%%%%%%%%%%%%%%%%%%%%%%%%%%%%%%%%%%%%%%%%
\paragraph{Content of Main File.}

It is advisable to place all content in the child files included by |\include|.
Any output contained in the main file will appear in all child documents
unless suppressed manually;
it cannot be suppressed automatically by the |\includeonly| directive
and thus should normally be avoided.
A method to include some content in the main file
by means of conditional processing is described in \secref{sec:conditional}.

%%%%%%%%%%%%%%%%%%%%%%%%%%%%%%%%%%%%%%%%
\paragraph{Page Numbering.}

When only a part of the document is compiled,
the appropriate numbering of pages
(as well as other status parameters)
is determined from the |.aux| files.
The latter contain information from previous passes.
However this information needs to propagate through
all intermediate child documents.
Therefore the page numbering in child documents may well
be inconsistent until the complete document is compiled at least once.

A useful (if unconventional) way to always ensure a consistent
page numbering is to restart the numbering in each child document
and denote the pages by `\textit{child}|.|\textit{page}'
where \textit{child} represents the chapter/section number of the child file.
This can be achieved by the command
|\numberwithin{page}{|\textit{child}|}|
of the \textsf{amsmath} package
where \textit{child} can be |chapter| or |section|
depending on the chosen structuring.
Alternatively, one can modify the macro |\thepage| appropriately
and reset the counter |page| at the start of each child file.

%%%%%%%%%%%%%%%%%%%%%%%%%%%%%%%%%%%%%%%%%%%%%%%%%%%%%%%%%%%%%%%%%%%%%%%%%%%%%%%%
\subsection{Conditional Processing}
\label{sec:conditional}

The package provides a mechanism to compile different versions
of a document. To customise the versions further some conditional processing
can come in handy to distinguish which version is being compiled.
The package provides two macros to describe the compilation context:

%%%%%%%%%%%%%%%%%%%%%%%%%%%%%%%%%%%%%%%%
\DescribeMacro{\ifchilddoc}
The conditional |\ifchilddoc| distinguishes between the compilation of
child documents and the main document:
%
\begin{center}
|\ifchilddoc |\textit{child-code}| |[|\||else |\textit{main-code}]| \||fi|
\end{center}

%%%%%%%%%%%%%%%%%%%%%%%%%%%%%%%%%%%%%%%%
\DescribeMacro{\childdocname}
\DescribeMacro{\childdocjob}
The macro |\childdocname| contains the filename (without extension)
of the main or child file being processed.
Note that |\childdocjob| will always contain the name of the main file.

%%%%%%%%%%%%%%%%%%%%%%%%%%%%%%%%%%%%%%%%
\paragraph{Title Page.}

Conditional processing can be used to include a title or banner page
in the main document when proper precautions are taken.
Importantly, the code in the main file should ensure that the page counter
(as well as other status parameters which are stored in the |.aux| files)
takes the same value after the conditional processing.
Otherwise the page numbers may take divergent values
depending on which part is compiled.

For example, a title page could be declared by:
%
\begin{center}
\begin{tabular}{l}
|\ifchilddoc\||else|\\
|\addtocounter{page}{-1}|\\
\textit{code for title page}\\
|\newpage|\\
|\||fi|
\end{tabular}
\end{center}
%
A banner page for the child documents can be generated by:
%
\begin{center}
\begin{tabular}{l}
|\ifchilddoc|\\
|\addtocounter{page}{-1}|\\
\textit{code for banner page}\\
|\newpage|\\
|\||fi|
\end{tabular}
\end{center}
%
Here one could write a message such as:
\begin{center}
|This is the part \childdocname{} of \childdocjob{}.|
\end{center}

%%%%%%%%%%%%%%%%%%%%%%%%%%%%%%%%%%%%%%%%%%%%%%%%%%%%%%%%%%%%%%%%%%%%%%%%%%%%%%%%
\subsection{Flags}
\label{sec:flags}

The package makes it easy to generate different versions
of the main or child documents.
To this end compilation flags can be defined
and assigned different default values.
They will be particularly useful in conjunction
with the forwarding mechanism described in \secref{sec:forward}.

For example, it may be useful to have a flag |\version|
which can be set to |draft| or |final|.
The document source will contain some conditional code
depending on the value of |\version|.
Suppose further, the flag should default to |final| for the main file
and to |draft| for child files
which is a natural assignment for editing the document.
This is achieved by placing the following code
in the preamble of the main document
(below the |\childdocmain| directive):
%
\begin{center}
\begin{tabular}{l}
|\ifchilddoc|\\
|\providecommand{\version}{draft}|\\
|\||else|\\
|\providecommand{\version}{final}|\\
|\||fi|
\end{tabular}
\end{center}
%
The definition by |\providecommand| makes sure
that previous definitions are not overwritten.
Further statements |\providecommand{\version}{...}|
can thus be added before the above code to override it.

For the main file, one might add a line
(between |\childdocmain| and the above block)
%
\begin{center}
|%\ifchilddoc\||else\providecommand{\version}{draft}\||fi|
\end{center}
%
which can be uncommented to produce a draft version.
Likewise one can add a line to the very top of a child file
(above the |\childdocof{|\textit{main}|}| directive)
%
\begin{center}
|%\providecommand{\version}{final}|
\end{center}
%
which can be uncommented to produce the final version of this child document.

%%%%%%%%%%%%%%%%%%%%%%%%%%%%%%%%%%%%%%%%%%%%%%%%%%%%%%%%%%%%%%%%%%%%%%%%%%%%%%%%
\subsection{Forwarding}
\label{sec:forward}

Different versions of the main or child documents
using compilation flags as described in \secref{sec:flags}
can be (permanently) stored in different files
for convenient compilation, viewing and distribution.
To this end, the package defines a command
to pass on compilation to a different file:

%%%%%%%%%%%%%%%%%%%%%%%%%%%%%%%%%%%%%%%%
\DescribeMacro{\childdocforward}
The command |\childdocforward| redirects processing to
another source file:
%
\begin{center}
\begin{tabular}{l}
|\input{childdoc.def}|\\
|\childdocforward[|\textit{main}|]{|\textit{dest}|}|\\
\end{tabular}
\end{center}
%
The argument \textit{dest} is the destination file
(without extension).
It should be the main file or one of the child files.
Note that further \textsf{childdoc} directives
such as |\childdocof| and |\childdocforward|
in the indicated file will be processed in this form.
The optional argument \textit{main}
passes on directly to the main file \textit{main}
while pretending to compile the child \textit{dest}.
This form behaves as if \textit{dest}
issues |\childdocof{|\textit{main}|}| right away,
and no further \textsf{childdoc} directives will be processed.

%%%%%%%%%%%%%%%%%%%%%%%%%%%%%%%%%%%%%%%%
\DescribeMacro{\...prefix}
In the alternative form |\childdocforwardprefix|,
%
\begin{center}
\begin{tabular}{l}
|\input{childdoc.def}|\\
|\childdocforwardprefix[|\textit{main}|]{|\textit{prefix}|}{|\textit{dest}|}|
\end{tabular}
\end{center}
%
the destination file is determined by a pattern
depending on the current file:
To make this work, the current file must be called
`{\textit{prefix}\hspace{0.2em}\textit{suffix}}'
with \textit{prefix} matching precisely the argument.
Processing is then passed on to the file
`{\textit{dest}\hspace{0.2em}\textit{suffix}}'.
Surely, the same effect is achieved by
directly specifying the
argument `{\textit{dest}\hspace{0.2em}\textit{suffix}}'
in the first form.
However, that requires to set up a different file
for each child. With the alternative form of the command
all these files can have exactly the same content
which simplifies setting them up and maintaining them.

For example, the following file |draft.tex|
with a compilation flag |\version| as described in \secref{sec:flags}
compiles the main document as a draft:
%
\begin{center}
\begin{tabular}{l}
|\def\version{draft}|\\
|\input{childdoc.def}|\\
|\childdocforward{|\textit{main}|}|
\end{tabular}
\end{center}
%
Likewise, the following files |final|\textit{nn}|.tex|
compile the final version of the child document
|child|\textit{nn}|.tex|:
%
\begin{center}
\begin{tabular}{l}
|\def\version{final}|\\
|\input{childdoc.def}|\\
|\childdocforwardprefix{final}{child}|
\end{tabular}
\end{center}
%

Note that when several versions of a main file and/or of each child file
are to be generated, it may be convenient to set up a |Makefile| or
shell script to automatise the process.

%%%%%%%%%%%%%%%%%%%%%%%%%%%%%%%%%%%%%%%%%%%%%%%%%%%%%%%%%%%%%%%%%%%%%%%%%%%%%%%%
\subsection{Command Line Processing}
\label{sec:commandline}

The effect of redirection files can also be achieved by invoking
the \LaTeX{} compiler with a more elaborate command line.
Most conveniently this should be done as part
of a shell script or a |Makefile|.

When using \textsf{childdoc} in the main file, the following
command lines effectively perform a redirection
(note that depending on the shell being used,
backslashes may have to be doubled: `|\|' $\to$ `|\\|'):
%
\begin{center}
|... -jobname "|\textit{target}|" |\\|"|[\textit{flags}]%
|\input{childdoc.def}\childdocforward[|\textit{main}|]{|\textit{dest}|}"|
\end{center}
%
Here \textit{target} is the name of the output file,
\textit{main} is the name of the main file
and \textit{dest} is the name of the main or child file to be processed
(all filenames without extensions).
The optional argument \textit{main} can be omitted
if \textit{main} matches \textit{dest}.
Optionally, compilation \textit{flags} can be defined via |\def| commands.
This command line makes the \TeX{} engine believe
it is compiling the file \textit{target}
whose content is specified as the latter parameter.
The provided code then forwards the processing to
\textit{main} or \textit{dest} as described in \secref{sec:forward}.

%%%%%%%%%%%%%%%%%%%%%%%%%%%%%%%%%%%%%%%%%%%%%%%%%%%%%%%%%%%%%%%%%%%%%%%%%%%%%%%%
\subsection{Include by Input}
\label{sec:input}

Including child documents by |\include| has some restrictions by design.
Most notably, the content of a child document always occupies
its own set of pages; pages cannot be shared between child documents.
Usually, this behaviour makes perfect sense
because each child document contain an essential part of the document.
However, in some situations it may be desirable to compose
a document from a collection of parts
without having mandatory page breaks between then.
For this case, the package
provides a mechanism to include parts
by |\input| which can also be processed individually.
However, by construction this mechanism
requires manual handling of the content to be output.

%%%%%%%%%%%%%%%%%%%%%%%%%%%%%%%%%%%%%%%%
\DescribeMacro{\ifchilddocmanual}
The main file should be prepared as usual, see \secref{sec:include}.
However, the document body must make a distinction
between processing of an individual part and of the main document, e.g.:
%
\begin{center}
\begin{tabular}{l}
|\ifchilddocmanual|\\
|\input{\childdocname}|\\
|\||else|\\
\textit{document body with }|\input{|\textit{part}|}|\\
|\||fi|
\end{tabular}
\end{center}
%
The conditional |\ifchilddocmanual| is true whenever
a part to be included by |\input| is being compiled,
and the name of the part is stored in |\childdocname|.

%%%%%%%%%%%%%%%%%%%%%%%%%%%%%%%%%%%%%%%%
\DescribeMacro{\childdocby}
Each part to be included by |\input| should start with:
%
\begin{center}
\begin{tabular}{l}
|\input{childdoc.def}|\\
|\childdocby{|\textit{main}|}|\\
\end{tabular}
\end{center}
%
The directive |\childdocby| is similar to |\childdocof|
described in \secref{sec:include},
but the subsequent selection of content must be done manually.
To that end, both |\ifchilddoc| and |\ifchilddocmanual|
will be true upon processing of a part,
and the name of the part is stored in |\childdocname|.
Note that |\jobname| will be set to the filename of the current part
so that each part receives an individual |.aux| file
that does not interfere with the |.aux| file(s) of the main document.
This behaviour can be altered by the alternative form
|\childdocby[*]{|\textit{main}|}| (with a non-empty optional argument)
which uses the |.aux| file of the main document
by setting |\jobname| to \textit{main}.

%%%%%%%%%%%%%%%%%%%%%%%%%%%%%%%%%%%%%%%%%%%%%%%%%%%%%%%%%%%%%%%%%%%%%%%%%%%%%%%%
\subsection{Driver Development}
\label{sec:driver}

The \textsf{childdoc} mechanism can also be use for the development
of definition files such as \LaTeX{} styles or classes.
This case differs from the above setup with multiple parts
included by |\include| in that no |\includeonly| should be invoked.
This can be achieved by starting the include file
(before |\ProvidesPackage|) with:
%
\begin{center}
\begin{tabular}{l}
|\input{childdoc.def}|\\
|\childdocforward{|\textit{main}|}|\\
\end{tabular}
\end{center}
%
or alternatively with:
%
\begin{center}
\begin{tabular}{l}
|\input{childdoc.def}|\\
|\childdocby{|\textit{main}|}|\\
\end{tabular}
\end{center}
%
Both forms have slightly different effects as described above.
The main file is prepared as usual, see \secref{sec:include}.

%%%%%%%%%%%%%%%%%%%%%%%%%%%%%%%%%%%%%%%%%%%%%%%%%%%%%%%%%%%%%%%%%%%%%%%%%%%%%%%%
\subsection{Legacy Detection}
\label{sec:detection}

The directive |\childdocmain| in the main file can detect
whether the complete document or merely a child is to be compiled
even without using the directive |\childdocof|.
This method is deprecated because it is less robust
and there is no compelling reason to use it;
it is merely provided for backward compatibility
and it may be removed in future versions.

If the detection mechanism is to be used,
it is mandatory to correctly specify
the filename of the main file as the argument of |\childdocmain|:
%
\begin{center}
\begin{tabular}{l}
|\input{childdoc.def}|\\
|\childdocmain{|\textit{main}|}|\\
\end{tabular}
\end{center}
%
If |\jobname| does not match the argument \textit{main} of |\childdocmain|,
it is assumed that |\jobname| points to the child file to be compiled.
When using |\childdocmain| with the main file specified as argument,
it suffices to start a child file
with just |\input{|\textit{main}|}|
without loading of the package and using |\childdocof|.
If instead all processing is done
with the appropriate \textsf{childdoc} directives,
the argument of \textit{main} of |\childdocmain| can be empty.

An alternative version of the command line processing described
in \secref{sec:commandline} using the detection mechanism reads:
%
\begin{center}
|... -jobname "|\textit{target}|" "|[\textit{flags}]%
[|\def\jobname{|\textit{dest}|}|]|\input{|\textit{main}|}"|
\end{center}

%%%%%%%%%%%%%%%%%%%%%%%%%%%%%%%%%%%%%%%%%%%%%%%%%%%%%%%%%%%%%%%%%%%%%%%%%%%%%%%%
\subsection{Manual Code}
\label{sec:manual}

In case one cannot be certain whether the definitions file |childdoc.def|
is installed on the target \TeX{} distribution
and one prefers not to ship it,
it is conceivable to paste a few relevant commands into the sources.

To that end, drop all statements |\input{childdoc.def}|
and perform the replacements as outlined below.
Instead of |\childdocmain{|\textit{main}|}| add the following code
to the top of the main file:
%
\begin{center}
\begin{tabular}{l}
|\||ifdefined\childdocname\endinput\||fi\newif\ifchilddoc|\\
|\edef\childdocname{\scantokens\expandafter{\jobname\noexpand}}|\\
|\def\childdocmain{|\textit{main}|}\||ifx\childdocmain\childdocname\||else|\\
|\childdoctrue\includeonly{\childdocname}\let\jobname\childdocmain\||fi|\\
\end{tabular}
\end{center}
%
Instead of |\childdocof{|\textit{main}|}| just include the main file
at the top of each child file:
%
\begin{center}
|\input{|\textit{main}|}|
\end{center}
%
A simple redirection |\childdocforward{|\textit{dest}|}| is achieved by:
%
\begin{center}
|\def\jobname{|\textit{dest}|}\input{\jobname}|
\end{center}
%
The redirection with prefix
|\childdocforwardprefix[|\textit{prefix}|]{|\textit{dest}|}|
is accomplished by:
%
\begin{center}
\begin{tabular}{l}
|{\edef\jobname{\scantokens\expandafter{\jobname\noexpand}}|\\
|\def\redirectjob |\textit{prefix}|#1~~~{\gdef\jobname{|\textit{dest}|#1}}|\\
|\expandafter\redirectjob\jobname~~~}\input{\jobname}|
\end{tabular}
\end{center}

In an alternative approach,
child documents can be compiled by a specific command line
without additional code or specific definitions:
%
\begin{center}
|... -jobname "|\textit{target}|" "|[\textit{flags}]%
|\includeonly{|\textit{dest}|}\input{|\textit{main}|}"|
\end{center}
%

%%%%%%%%%%%%%%%%%%%%%%%%%%%%%%%%%%%%%%%%%%%%%%%%%%%%%%%%%%%%%%%%%%%%%%%%%%%%%%%%
%%%%%%%%%%%%%%%%%%%%%%%%%%%%%%%%%%%%%%%%%%%%%%%%%%%%%%%%%%%%%%%%%%%%%%%%%%%%%%%%
\section{Information}

%%%%%%%%%%%%%%%%%%%%%%%%%%%%%%%%%%%%%%%%%%%%%%%%%%%%%%%%%%%%%%%%%%%%%%%%%%%%%%%%
\subsection{Copyright}

Copyright \copyright{} 2017--2018 Niklas Beisert

This work may be distributed and/or modified under the
conditions of the \LaTeX{} Project Public License, either version 1.3
of this license or (at your option) any later version.
The latest version of this license is in
  \url{http://www.latex-project.org/lppl.txt}
and version 1.3 or later is part of all distributions of \LaTeX{}
version 2005/12/01 or later.

This work has the LPPL maintenance status `maintained'.

The Current Maintainer of this work is Niklas Beisert.

This work consists of the files |README.txt|, |childdoc.ins| and |childdoc.dtx|
as well as the derived files |childdoc.def|, |cdocsamp.tex|
with |cdocsch1.tex|, |cdocsch2.tex|, |cdocspt3.tex|, |cdocspt4.tex|,
|cdocsdrf.tex|, |cdocsfn1.tex|, |cdocsfn2.tex|
as well as |childdoc.pdf|.

%%%%%%%%%%%%%%%%%%%%%%%%%%%%%%%%%%%%%%%%%%%%%%%%%%%%%%%%%%%%%%%%%%%%%%%%%%%%%%%%
\subsection{Files and Installation}

The package consists of the files:
%
\begin{center}
\begin{tabular}{ll}
    |README.txt|   & readme file \\
    |childdoc.ins| & installation file \\
    |childdoc.dtx| & source file \\
    |childdoc.def| & definition file \\
    |cdocsamp.tex| & sample main file \\
    |cdocsch1.tex| & sample include file \\
    |cdocsch2.tex| & sample include file \\
    |cdocspt3.tex| & sample part file \\
    |cdocspt4.tex| & sample part file \\
    |cdocsdrf.tex| & sample redirection file \\
    |cdocsfn1.tex| & sample redirection file \\
    |cdocsfn2.tex| & sample redirection file \\
    |childdoc.pdf| & manual
\end{tabular}
\end{center}
%
The distribution consists of the files
|README.txt|, |childdoc.ins| and |childdoc.dtx|.
%
\begin{itemize}
\item
Run (pdf)\LaTeX{} on |childdoc.dtx|
to compile the manual |childdoc.pdf| (this file).
\item
Run \LaTeX{} on |childdoc.ins| to create the definitions file |childdoc.def|
and the sample |cdocsamp.tex| with include files
|cdocsch1.tex|, |cdocsch2.tex|, |cdocspt3.tex|, |cdocspt4.tex|,
|cdocsdrf.tex|, |cdocsfn1.tex|, |cdocsfn2.tex|.
Then copy the file |childdoc.def| to an appropriate directory of your \LaTeX{}
distribution, e.g.\ \textit{texmf-root}|/tex/latex/childdoc|.
\end{itemize}

%%%%%%%%%%%%%%%%%%%%%%%%%%%%%%%%%%%%%%%%%%%%%%%%%%%%%%%%%%%%%%%%%%%%%%%%%%%%%%%%
\subsection{Related CTAN Packages}

There are several other packages which offer a similar functionality:
%
\begin{itemize}
\item
The packages
\href{http://ctan.org/pkg/docmute}{\textsf{docmute}},
\href{http://ctan.org/pkg/includex}{\textsf{includex}} and
\href{http://ctan.org/pkg/standalone}{\textsf{standalone}}
provide commands to include only the document body of
a child file thus allowing both files to be compiled individually.
\item
The packages \href{http://ctan.org/pkg/subdocs}{\textsf{subdocs}}
and \href{http://ctan.org/pkg/subfiles}{\textsf{subfiles}}
provide structures in which the main and child documents can be
encapsulated and allowing them to be compiled individually.
The inclusion mechanism is different from the conventional |\include|.
\item
The package \href{http://ctan.org/pkg/combine}{\textsf{combine}}
is an elaborate solution to combine several documents into one.
\end{itemize}
%
See also the CTAN topic \href{http://ctan.org/topic/subdocs}{\textsf{subdocs}}
for further related packages.
The present package differs from the above solutions in that
a document structure constructed with the conventional |\include| mechanism
just needs two extra commands at the top of every file
such that all constituent files can be compiled individually.

%%%%%%%%%%%%%%%%%%%%%%%%%%%%%%%%%%%%%%%%%%%%%%%%%%%%%%%%%%%%%%%%%%%%%%%%%%%%%%%%
%\subsection{Feature Suggestions}
%
%The following is a list of features which may be useful for future
%versions of this package:
%%
%\begin{itemize}
%\item
%\ldots
%\end{itemize}

%%%%%%%%%%%%%%%%%%%%%%%%%%%%%%%%%%%%%%%%%%%%%%%%%%%%%%%%%%%%%%%%%%%%%%%%%%%%%%%%
\subsection{Revision History}

%%%%%%%%%%%%%%%%%%%%%%%%%%%%%%%%%%%%%%%%
\paragraph{v2.0:} 2018/12/30

\begin{itemize}
\item
immediate forward processing
\item
added |\childdocby| mechanism
\item
manual restructured
\end{itemize}

%%%%%%%%%%%%%%%%%%%%%%%%%%%%%%%%%%%%%%%%
\paragraph{v1.6:} 2018/01/17

\begin{itemize}
\item
application for development of include files
\item
corrections to manual
\end{itemize}

%%%%%%%%%%%%%%%%%%%%%%%%%%%%%%%%%%%%%%%%
\paragraph{v1.5:} 2017/05/21

\begin{itemize}
\item
more complete structuring introduced
\item
|\childdocof| introduced
\item
|\childdoc| renamed to |\childdocmain|
\item
|\childredirect| renamed to |\childdocforward| and |\childdocforwardprefix|
and functionality expanded
\end{itemize}

%%%%%%%%%%%%%%%%%%%%%%%%%%%%%%%%%%%%%%%%
\paragraph{v1.0:} 2017/04/27

\begin{itemize}
\item
manual and install package
\item
first version published on CTAN
\end{itemize}

%%%%%%%%%%%%%%%%%%%%%%%%%%%%%%%%%%%%%%%%
\paragraph{v0.6:} 2017/04/26

\begin{itemize}
\item
redirection mechanism added
\end{itemize}

%%%%%%%%%%%%%%%%%%%%%%%%%%%%%%%%%%%%%%%%
\paragraph{v0.5:} 2017/04/26

\begin{itemize}
\item
functionality in definition file
\end{itemize}


%%%%%%%%%%%%%%%%%%%%%%%%%%%%%%%%%%%%%%%%%%%%%%%%%%%%%%%%%%%%%%%%%%%%%%%%%%%%%%%%
%%%%%%%%%%%%%%%%%%%%%%%%%%%%%%%%%%%%%%%%%%%%%%%%%%%%%%%%%%%%%%%%%%%%%%%%%%%%%%%%
%%%%%%%%%%%%%%%%%%%%%%%%%%%%%%%%%%%%%%%%%%%%%%%%%%%%%%%%%%%%%%%%%%%%%%%%%%%%%%%%
\appendix

\settowidth\MacroIndent{\rmfamily\scriptsize 000\ }

 \DocInput{childdoc.dtx}

\end{document}
%</driver>
% \fi
%
% %%%%%%%%%%%%%%%%%%%%%%%%%%%%%%%%%%%%%%%%%%%%%%%%%%%%%%%%%%%%%%%%%%%%%%%%%%%%%%
% %%%%%%%%%%%%%%%%%%%%%%%%%%%%%%%%%%%%%%%%%%%%%%%%%%%%%%%%%%%%%%%%%%%%%%%%%%%%%%
% \section{Sample}
%\iffalse
%<*samplemain>
%\fi
%
% The following presents a sample document
% with two chapters, two parts, a title page,
% a compile flag as well as three forwarding files to set the flag.
% It consists of eight |.tex| files:
% \begin{center}
% \begin{tabular}{ll}
% |cdocsamp.tex|&main file\\
% |cdocsch1.tex|&include file for chapter 1\\
% |cdocsch2.tex|&include file for chapter 2\\
% |cdocspt3.tex|&include file for part 3\\
% |cdocspt4.tex|&include file for part 4\\
% |cdocsdrf.tex|&forwarding file for main file in draft mode\\
% |cdocsfi1.tex|&forwarding file for final version of chapter 1\\
% |cdocsfi2.tex|&forwarding file for final version of chapter 2\\
% \end{tabular}
% \end{center}
% Each of the eight files can be compiled directly by the \LaTeX{} compiler.
%
% %%%%%%%%%%%%%%%%%%%%%%%%%%%%%%%%%%%%%%
% \paragraph{Main File.}
%
% The main file is called |cdocsamp.tex|.
%
% Load the \textsf{childdoc} definitions and
% declare the filename for the main document:
%    \begin{macrocode}
\input{childdoc.def}
\childdocmain{}
%    \end{macrocode}

% Optional override for |\version| flag:
%    \begin{macrocode}
%%\ifchilddoc\else\providecommand{\version}{draft}\fi
%    \end{macrocode}

% Define the default values for the |\version| flag
% (|final| for the main file and |draft| for childs):
%    \begin{macrocode}
\ifchilddoc
\providecommand{\version}{draft}
\else
\providecommand{\version}{final}
\fi
%    \end{macrocode}

% Load the standard document class:
%    \begin{macrocode}
\documentclass[12pt]{article}
%    \end{macrocode}

% Start the document body:
%    \begin{macrocode}
\begin{document}
%    \end{macrocode}

% Declare a title page.
% Print title, part of document being processed and version flag:
%    \begin{macrocode}
\addtocounter{page}{-1}
\begin{center}
{\LARGE\bfseries{}childdoc example\par}
\vspace{1cm}
\ifchilddoc
\ifchilddocmanual part\else chapter\fi:
`\childdocname' of `\childdocjob'\par
\else
main document: `\childdocjob'\par
\fi
version: \version\par
\end{center}
\newpage
%    \end{macrocode}

% Manually include selected file,
% otherwise process as usual:
%    \begin{macrocode}
\ifchilddocmanual
\section*{part `\childdocname'}
\input{\childdocname}
\else
%    \end{macrocode}

% Include the two chapters:
%    \begin{macrocode}
\include{cdocsch1}
\include{cdocsch2}
%    \end{macrocode}

% Include the two parts unless only chapters should be displayed:
%    \begin{macrocode}
\ifchilddoc\else
\section{part three}
\input{cdocspt3}
\section{part four}
\input{cdocspt4}
\fi
%    \end{macrocode}

% Process as usual until here:
%    \begin{macrocode}
\fi
%    \end{macrocode}

% End of document body:
%    \begin{macrocode}
\end{document}
%    \end{macrocode}
%\iffalse
%</samplemain>
%\fi
%
% %%%%%%%%%%%%%%%%%%%%%%%%%%%%%%%%%%%%%%
% \paragraph{Chapter Include Files.}
%
% The include files are called |cdocsch1.tex| and |cdocsch2.tex|.
%
%\iffalse
%<*samplechap1|samplechap2>
%\fi

% Optional override for |\version| flag:
%    \begin{macrocode}
%%\providecommand{\version}{final}
%    \end{macrocode}

% Include the main document:
%    \begin{macrocode}
\input{childdoc.def}
\childdocof{cdocsamp}
%    \end{macrocode}

%\iffalse
%</samplechap1|samplechap2>
%\fi
%
%\iffalse
%<*samplechap1>
%\fi
% Some text for chapter 1:
%    \begin{macrocode}
\section{one}
some text in chapter one
%    \end{macrocode}

%\iffalse
%</samplechap1>
%\fi
% Some text for chapter 2:
%\iffalse
%<*samplechap2>
%\fi
%    \begin{macrocode}
\section{two}
more text in chapter two
%    \end{macrocode}

%\iffalse
%</samplechap2>
%\fi
%
% %%%%%%%%%%%%%%%%%%%%%%%%%%%%%%%%%%%%%%
% \paragraph{Part Include Files.}
%
% The include files are called |cdocspt3.tex| and |cdocspt4.tex|.
%
%\iffalse
%<*samplepart3|samplepart4>
%\fi

% Optional override for |\version| flag:
%    \begin{macrocode}
%%\providecommand{\version}{final}
%    \end{macrocode}

% Include the main document:
%    \begin{macrocode}
\input{childdoc.def}
\childdocby{cdocsamp}
%    \end{macrocode}

%\iffalse
%</samplepart3|samplepart4>
%\fi
%
%\iffalse
%<*samplepart3>
%\fi
% Some text for part 3:
%    \begin{macrocode}
some text in part three
%    \end{macrocode}

%\iffalse
%</samplepart3>
%\fi
% Some text for part 4:
%\iffalse
%<*samplepart4>
%\fi
%    \begin{macrocode}
more text in part four
%    \end{macrocode}

%\iffalse
%</samplepart4>
%\fi
%
% %%%%%%%%%%%%%%%%%%%%%%%%%%%%%%%%%%%%%%
% \paragraph{Forwarding for a Complete Draft.}
%
% The following forwarding file |cdocsdrf.tex|
% compiles the main document in draft mode:
%\iffalse
%<*sampledraft>
%\fi
%    \begin{macrocode}
\def\version{draft}
\input{childdoc.def}
\childdocforward{cdocsamp}
%    \end{macrocode}

%\iffalse
%</sampledraft>
%\fi
%
% %%%%%%%%%%%%%%%%%%%%%%%%%%%%%%%%%%%%%%
% \paragraph{Forwarding for Final Version of the Chapters.}
%
% The following forwarding files |cdocsfn1.tex| and |cdocsfn2.tex|
% (with identical content)
% compile the final versions of the child documents
% |cdocsch1.tex| and |cdocsch2.tex|, respectively:
%\iffalse
%<*samplefinal>
%\fi
%    \begin{macrocode}
\def\version{final}
\input{childdoc.def}
\childdocforwardprefix[cdocsamp]{cdocsfn}{cdocsch}
%    \end{macrocode}

%\iffalse
%</samplefinal>
%\fi
%
% %%%%%%%%%%%%%%%%%%%%%%%%%%%%%%%%%%%%%%
% \paragraph{Command Line Processing.}
%
% The following three command lines generate the output files
% |cdocscld|, |cdocscl1| and |cdocscl2|
% which should be identical to
% |cdocsdrf|, |cdocsch1| and |cdocsfn2|, respectively:
% \begin{center}
% \begin{tabular}{l}
% |latex -jobname cdocscld \|\\
% |  "\def\version{draft}\input{childdoc.def}\childdocforward{cdocsamp}"|\\
% |latex -jobname cdocscl1 \|\\
% |  "\input{childdoc.def}\childdocforward[cdocsamp]{cdocsch1}"|\\
% |latex -jobname cdocscl2 \|\\
% |  "\def\version{final}\input{childdoc.def}\childdocforward{cdocsch2}"|
% \end{tabular}
% \end{center}
% Note that the trailing backslash on each first line
% merely continues the input to the second line
% (for convenient cut ant paste).
% Furthermore, the command |latex| can be replaced by any
% of its alternative versions such as |pdflatex|.
%
% %%%%%%%%%%%%%%%%%%%%%%%%%%%%%%%%%%%%%%%%%%%%%%%%%%%%%%%%%%%%%%%%%%%%%%%%%%%%%%
% %%%%%%%%%%%%%%%%%%%%%%%%%%%%%%%%%%%%%%%%%%%%%%%%%%%%%%%%%%%%%%%%%%%%%%%%%%%%%%
% \section{Implementation}
%\iffalse
%<*package>
%\fi
%
% This section describes the definitions file |childdoc.def|.

% The definitions cannot be loaded using |\usepackage| or |\RequirePackage|
% which has a mechanism to prevent loading a style file more than once.
% When loading the definitions by means of |\input|
% multiple instances have to be prevented manually:
%\iffalse
%This code needs to be before the `\ProvidesFile' directive
%which is defined at the beginning of this file.
%Therefore it is also placed there and commented out here.
%</package>
%<*discard>
%\fi
%    \begin{macrocode}
\ifdefined\childdocmain\endinput\fi
%    \end{macrocode}
%\iffalse
%</discard>
%<*package>
%\fi
%
% \macro{\ifchilddoc}
% \macro{\ifchilddocmanual}
% The conditional |\ifchilddoc| tells whether a
% child (true) or main (false) document is being compiled.
% The conditional |\ifchilddocmanual| tells whether
% the |\includeonly| mechanism is used (false) or
% the selection of child files must be performed manually (true).
% The definitions initialise to false:
%    \begin{macrocode}
\newif\ifchilddoc
\newif\ifchilddocmanual
%    \end{macrocode}

% \macro{\childdocname}
% \macro{\childdocjob}
% The macro |\childdocname| stores the name of the main document
% to be compiled. The macro |\childdocjob| stores the name of
% the document on which the \LaTeX{} compiler was originally invoked.
% The content of |\jobname| cannot be compared
% to filenames specified in the source due to different catcodes.
% The following code rescans |\jobname|, stores the result
% in |\childdocname| and saves a copy in |\childdocjob|:
%    \begin{macrocode}
\edef\childdocname{\scantokens\expandafter{\jobname\noexpand}}
\let\childdocjob\childdocname
%    \end{macrocode}

% \macro{\childdocdisable}
% The macro |\childdocdisable| prevents the main file
% from being processed more than once.
% At this stage, the main document command |\childdocmain|
% is assumed to be called once again where it should do nothing.
% Any subsequent call to it should prevent
% a secondary processing of the main document
% It overwrites the forwarding commands
% |\childdocof| and |\childdocforward|
% with empty macros to prevent further inclusions of the main document:
%    \begin{macrocode}
\newcommand{\childdocdisable}
{
  \renewcommand{\childdocmain}[1]{\renewcommand{\childdocmain}[1]{\endinput}}
  \renewcommand{\childdocof}[1]{}
  \renewcommand{\childdocby}[2][]{}
  \renewcommand{\childdocforward}[2][]{}
  \renewcommand{\childdocdisable}{}
}
%    \end{macrocode}

% \macro{\childdocmain}
% The macro |\childdocmain| is to be called at the top of the main file
% with nothing or the main filename (without extension) as argument.
% First, it breaks loops.
% If the argument is not empty and does not match |\childdocname|
% (which is set by the first inclusion of |childdoc.def|),
% |\ifchilddoc| is set to true, |\includeonly| is applied to the child file
% and |\jobname| is set to the main file
% (for proper handling of |.aux| files):
%    \begin{macrocode}
\newcommand{\childdocmain}[1]
{
  \childdocdisable\childdocmain{}
  \if?#1?\else
    \begingroup
      \def\childdoctmp{#1}
      \ifx\childdoctmp\childdocname
        \def\childdoctmp{}
      \else
        \def\childdoctmp
        {
          \childdoctrue
          \includeonly{\childdocname}
          \def\childdocjob{#1}
          \def\jobname{#1}
        }
      \fi
      \expandafter
    \endgroup
    \childdoctmp
  \fi
}
%    \end{macrocode}

% \macro{\childdocof}
% The command |\childdocof| redirects
% compilation to the main file |#1|.
%    \begin{macrocode}
\newcommand{\childdocof}[1]
{
  \childdocdisable
  \childdoctrue
  \includeonly{\childdocname}
  \def\jobname{#1}
  \def\childdocjob{#1}
  \input{#1}
}
%    \end{macrocode}

% \macro{\childdocby}
% The command |\childdocby| ....
%    \begin{macrocode}
\newcommand{\childdocby}[2][]
{
  \childdocdisable
  \childdoctrue
  \childdocmanualtrue
  \if?#1?\else
    \def\jobname{#2}
  \fi
  \def\childdocjob{#2}
  \input{#2}
  \endinput
}
%    \end{macrocode}

% \macro{\childdocforward}
% The command |\childdocforward| redirects
% compilation to the main file or
% (if the optional argument is given) a child file.
% Parameters are set as if the main file
% or a child file starting with |\childdocof| was compiled.
% Then compilation is handed over to the main file:
%    \begin{macrocode}
\newcommand{\childdocforward}[2][]
{
  \begingroup
    \if?#1?
      \def\childdoctmp
      {
        \def\childdocname{#2}
        \def\childdocjob{#2}
        \def\jobname{#2}
        \input{#2}
        \endinput
      }
    \else
      \def\childdoctmp
      {
        \childdocdisable
        \def\childdocname{#2}
        \childdoctrue
        \includeonly{#2}
        \def\childdocjob{#1}
        \def\jobname{#1}
        \input{#1}
        \endinput
      }
    \fi
    \expandafter
  \endgroup
  \childdoctmp
}
%    \end{macrocode}

% \macro{\childdocforwardprefix}
% The command |\childdocforwardprefix| redirects
% compilation to the main or a child file by means of a pattern.
% The prefix |#1| in the current filename is replaced by |#2|
% and the suffix of the current filename is kept
% (it is assumed that the filename does not contain the substring `|~~~|'
% which is used as a delimiter).
% Compilation is handed over to the new file by |\childdocforward|:
%    \begin{macrocode}
\newcommand{\childdocforwardprefix}[3][]
{
  \begingroup
    \def\childdocextract #2##1~~~{\def\childdoctmp{\childdocforward[#1]{#3##1}}}
    \expandafter\childdocextract\childdocname~~~
    \expandafter
  \endgroup
  \childdoctmp
}
%    \end{macrocode}

% \macro{\childdoc}
% The deprecated macro |\childdoc| is a legacy version of |\childdocmain|:
%    \begin{macrocode}
\newcommand{\childdoc}{\childdocmain}
%    \end{macrocode}

% \macro{\childdocredirect}
% The deprecated macro |\childdocredirect| is a legacy version
% of |\childdocforward| and |\childdocforwardprefix|:
%    \begin{macrocode}
\newcommand{\childdocredirect}[2][]
{
  \begingroup
    \if?#1?
      \def\childdoctmp{\childdocforward{#2}}
    \else
      \def\childdoctmp{\childdocforwardprefix{#1}{#2}}
    \fi
    \expandafter
  \endgroup
  \childdoctmp
}
%    \end{macrocode}

%\iffalse
%</package>
%\fi
%
\endinput
\childdocforward[cdocsamp]{cdocsch1}"|\\
% |latex -jobname cdocscl2 \|\\
% |  "\def\version{final}% \iffalse
%
% childdoc.dtx Copyright (C) 2017-2018 Niklas Beisert
%
% This work may be distributed and/or modified under the
% conditions of the LaTeX Project Public License, either version 1.3
% of this license or (at your option) any later version.
% The latest version of this license is in
%   http://www.latex-project.org/lppl.txt
% and version 1.3 or later is part of all distributions of LaTeX
% version 2005/12/01 or later.
%
% This work has the LPPL maintenance status `maintained'.
%
% The Current Maintainer of this work is Niklas Beisert.
%
% This work consists of the files childdoc.dtx and childdoc.ins
% and the derived files childdoc.def and cdocsamp.tex with
% cdocsch1.tex, cdocsch2.tex, cdocsdrf.tex, cdocsfn1.tex, cdocsfn2.tex.
%
%<package>\ifdefined\childdocmain\endinput\fi
%<package>\ProvidesFile{childdoc.def}[2018/12/30 v2.0 child document driver]
%<samplemain>\ProvidesFile{cdocsamp.tex}[2018/12/30 v2.0 sample for childdoc]
%<*driver>
%\ProvidesFile{childdoc.drv}[2018/12/30 v2.0 childdoc reference manual file]
\PassOptionsToClass{10pt,a4paper}{article}
\documentclass{ltxdoc}

\usepackage[margin=35mm]{geometry}
\usepackage{hyperref}
\usepackage{hyperxmp}
\usepackage[usenames]{color}

\hypersetup{colorlinks=true}
\hypersetup{pdfstartview=FitH}
\hypersetup{pdfpagemode=UseNone}
\hypersetup{pdfsource={}}
\hypersetup{pdflang={en-UK}}
\hypersetup{pdfcopyright={Copyright 2017-2018 Niklas Beisert.
  This work may be distributed and/or modified under the
  conditions of the LaTeX Project Public License, either version 1.3
  of this license or (at your option) any later version.}}
\hypersetup{pdflicenseurl={http://www.latex-project.org/lppl.txt}}
\hypersetup{pdfcontactaddress={ETH Zurich, ITP, HIT K,
  Wolfgang-Pauli-Strasse 27}}
\hypersetup{pdfcontactpostcode={8093}}
\hypersetup{pdfcontactcity={Zurich}}
\hypersetup{pdfcontactcountry={Switzerland}}
\hypersetup{pdfcontactemail={nbeisert@itp.phys.ethz.ch}}
\hypersetup{pdfcontacturl={http://people.phys.ethz.ch/\xmptilde nbeisert/}}

\newcommand{\secref}[1]{\hyperref[#1]{section \ref*{#1}}}

\parskip1ex
\parindent0pt
\let\olditemize\itemize
\def\itemize{\olditemize\parskip0pt}

\begin{document}

\title{The \textsf{childdoc} Package}
\hypersetup{pdftitle={The childdoc Package}}
\author{Niklas Beisert\\[2ex]
  Institut f\"ur Theoretische Physik\\
  Eidgen\"ossische Technische Hochschule Z\"urich\\
  Wolfgang-Pauli-Strasse 27, 8093 Z\"urich, Switzerland\\[1ex]
  \href{mailto:nbeisert@itp.phys.ethz.ch}
  {\texttt{nbeisert@itp.phys.ethz.ch}}}
\hypersetup{pdfauthor={Niklas Beisert}}
\hypersetup{pdfsubject={Manual for the LaTeX2e Package childdoc}}
\date{30 December 2018, \textsf{v2.0}}
\maketitle

\begin{abstract}\noindent
\textsf{childdoc} is a \LaTeXe{} package
that enables the direct compilation
of document sections included by |\include|
to individual files.
\end{abstract}

\begingroup
\parskip0ex
\tableofcontents
\endgroup

%%%%%%%%%%%%%%%%%%%%%%%%%%%%%%%%%%%%%%%%%%%%%%%%%%%%%%%%%%%%%%%%%%%%%%%%%%%%%%%%
%%%%%%%%%%%%%%%%%%%%%%%%%%%%%%%%%%%%%%%%%%%%%%%%%%%%%%%%%%%%%%%%%%%%%%%%%%%%%%%%
\section{Introduction}

\LaTeX{} provides a mechanism to structure a large document (such as a book)
into a main file and several child files (containing the chapters)
using the |\include| command.
This mechanism is beneficial for documents
which span hundreds of pages in order to
make the source file(s) more manageable.
Moreover, compilation can be restricted to
selected child files by means of the |\includeonly| command.
The latter feature can be used to reduce the compilation time while editing
(this was significantly more useful in the earlier days of \LaTeX{})
or to generate a smaller document which is easier to navigate.
Another application of |\includeonly| is to generate
documents consisting of selected parts of the complete document.

However, there are a few drawbacks of the plain |\include| mechanism:
\begin{itemize}
\item
The child files cannot be compiled on their own,
they can only be compiled via the main file.
A naive editing environment
(such as a text editor with an option
to have the current file processed by \LaTeX)
may require one to switch to the main file before compiling;
attempting to compile the child file produces errors.
\item
The main file must be modified (each time)
to adjust the |\includeonly| command
to the present needs. This easily leaves the main file in a messy state.
\item
The generated document will always carry the filename
of the main document. This is inconvenient if
several child files are to be compiled and
to be kept for distribution.
\end{itemize}

The present package provides a simple interface
to make child files individually compilable by \LaTeX{}.
Compiling a child file then has the same effect as compiling
the main file with an |\includeonly| command
to select the appropriate child.
Moreover the generated document will carry the name of the child
rather than the main file.
This resolves all three above issues.

This feature is meant to make the editing of books,
thesis documents and lecture notes somewhat more convenient.
However, the package can also be used efficiently for
composing a series of documents (such as exercise sheets)
which are typically distributed individually.
It then assists the author in generating the individual documents
(potentially in different versions)
as well as a document containing the collected series.
Another application is in developing style files
or other kinds of included material
where compilation of the style file could redirect
to a sample or test file.

%%%%%%%%%%%%%%%%%%%%%%%%%%%%%%%%%%%%%%%%%%%%%%%%%%%%%%%%%%%%%%%%%%%%%%%%%%%%%%%%
%%%%%%%%%%%%%%%%%%%%%%%%%%%%%%%%%%%%%%%%%%%%%%%%%%%%%%%%%%%%%%%%%%%%%%%%%%%%%%%%
\section{Usage}

First of all, the package \textsf{childdoc} is \emph{not} a standard
\LaTeXe{} |.sty| style file! Therefore it needs to be invoked in
a non-standard way.

%%%%%%%%%%%%%%%%%%%%%%%%%%%%%%%%%%%%%%%%%%%%%%%%%%%%%%%%%%%%%%%%%%%%%%%%%%%%%%%%
\subsection{Included Files}
\label{sec:include}

%%%%%%%%%%%%%%%%%%%%%%%%%%%%%%%%%%%%%%%%
\DescribeMacro{\childdocmain}
To use the package, add the commands
\begin{center}
\begin{tabular}{l}
|\input{childdoc.def}|\\
|\childdocmain{}|\\
\end{tabular}
\end{center}
at the very top of the main \LaTeX{} file,
in particular \emph{before} the |\documentclass| statement!
The argument of |\childdocmain| should be left empty
(but it must be present).

%%%%%%%%%%%%%%%%%%%%%%%%%%%%%%%%%%%%%%%%
\DescribeMacro{\childdocof}
Furthermore, add the commands
\begin{center}
\begin{tabular}{l}
|\input{childdoc.def}|\\
|\childdocof{|\textit{main}|}|\\
\end{tabular}
\end{center}
at the top of every child file \textit{child}
which is included by |\include{|\textit{child}|}|
from within the main file
(or at least for those files to be compiled individually).
The argument \textit{main} must be the filename of the main file.

There are a couple of
considerations in setting up the main and child documents:

%%%%%%%%%%%%%%%%%%%%%%%%%%%%%%%%%%%%%%%%
\paragraph{Restrictions.}

Please note the following restrictions:
\begin{itemize}
\item
|\childdocmain| must be called with one argument \textit{main}
to ensure compatibility with earlier version of the package.
It must either be empty (|\childdocmain{}|)
or precisely match the filename of the main file in which it is specified.
See \secref{sec:detection} for further information.
\item
The filename \textit{main} must be specified without the |.tex| extension.
\item
The filename \textit{main} is case sensitive
(even in case-insensitive file systems)
due to internal string comparison.
\item
The argument \textit{main} should be fully expanded, it cannot be a macro.
\item
Subdirectories and special characters should be avoided in filenames.
\item
The command |\childdocmain{|\textit{main}|}| must be followed by a whitespace.
It should not be followed immediately by another command
or by a comment mark `|%|'.
This is because the \TeX{} parser reads the token immediately following
the argument of |\childdocmain| and puts it
at the beginning of every child section;
however, a white\-space is ignored.
\end{itemize}

%%%%%%%%%%%%%%%%%%%%%%%%%%%%%%%%%%%%%%%%
\paragraph{Content of Main File.}

It is advisable to place all content in the child files included by |\include|.
Any output contained in the main file will appear in all child documents
unless suppressed manually;
it cannot be suppressed automatically by the |\includeonly| directive
and thus should normally be avoided.
A method to include some content in the main file
by means of conditional processing is described in \secref{sec:conditional}.

%%%%%%%%%%%%%%%%%%%%%%%%%%%%%%%%%%%%%%%%
\paragraph{Page Numbering.}

When only a part of the document is compiled,
the appropriate numbering of pages
(as well as other status parameters)
is determined from the |.aux| files.
The latter contain information from previous passes.
However this information needs to propagate through
all intermediate child documents.
Therefore the page numbering in child documents may well
be inconsistent until the complete document is compiled at least once.

A useful (if unconventional) way to always ensure a consistent
page numbering is to restart the numbering in each child document
and denote the pages by `\textit{child}|.|\textit{page}'
where \textit{child} represents the chapter/section number of the child file.
This can be achieved by the command
|\numberwithin{page}{|\textit{child}|}|
of the \textsf{amsmath} package
where \textit{child} can be |chapter| or |section|
depending on the chosen structuring.
Alternatively, one can modify the macro |\thepage| appropriately
and reset the counter |page| at the start of each child file.

%%%%%%%%%%%%%%%%%%%%%%%%%%%%%%%%%%%%%%%%%%%%%%%%%%%%%%%%%%%%%%%%%%%%%%%%%%%%%%%%
\subsection{Conditional Processing}
\label{sec:conditional}

The package provides a mechanism to compile different versions
of a document. To customise the versions further some conditional processing
can come in handy to distinguish which version is being compiled.
The package provides two macros to describe the compilation context:

%%%%%%%%%%%%%%%%%%%%%%%%%%%%%%%%%%%%%%%%
\DescribeMacro{\ifchilddoc}
The conditional |\ifchilddoc| distinguishes between the compilation of
child documents and the main document:
%
\begin{center}
|\ifchilddoc |\textit{child-code}| |[|\||else |\textit{main-code}]| \||fi|
\end{center}

%%%%%%%%%%%%%%%%%%%%%%%%%%%%%%%%%%%%%%%%
\DescribeMacro{\childdocname}
\DescribeMacro{\childdocjob}
The macro |\childdocname| contains the filename (without extension)
of the main or child file being processed.
Note that |\childdocjob| will always contain the name of the main file.

%%%%%%%%%%%%%%%%%%%%%%%%%%%%%%%%%%%%%%%%
\paragraph{Title Page.}

Conditional processing can be used to include a title or banner page
in the main document when proper precautions are taken.
Importantly, the code in the main file should ensure that the page counter
(as well as other status parameters which are stored in the |.aux| files)
takes the same value after the conditional processing.
Otherwise the page numbers may take divergent values
depending on which part is compiled.

For example, a title page could be declared by:
%
\begin{center}
\begin{tabular}{l}
|\ifchilddoc\||else|\\
|\addtocounter{page}{-1}|\\
\textit{code for title page}\\
|\newpage|\\
|\||fi|
\end{tabular}
\end{center}
%
A banner page for the child documents can be generated by:
%
\begin{center}
\begin{tabular}{l}
|\ifchilddoc|\\
|\addtocounter{page}{-1}|\\
\textit{code for banner page}\\
|\newpage|\\
|\||fi|
\end{tabular}
\end{center}
%
Here one could write a message such as:
\begin{center}
|This is the part \childdocname{} of \childdocjob{}.|
\end{center}

%%%%%%%%%%%%%%%%%%%%%%%%%%%%%%%%%%%%%%%%%%%%%%%%%%%%%%%%%%%%%%%%%%%%%%%%%%%%%%%%
\subsection{Flags}
\label{sec:flags}

The package makes it easy to generate different versions
of the main or child documents.
To this end compilation flags can be defined
and assigned different default values.
They will be particularly useful in conjunction
with the forwarding mechanism described in \secref{sec:forward}.

For example, it may be useful to have a flag |\version|
which can be set to |draft| or |final|.
The document source will contain some conditional code
depending on the value of |\version|.
Suppose further, the flag should default to |final| for the main file
and to |draft| for child files
which is a natural assignment for editing the document.
This is achieved by placing the following code
in the preamble of the main document
(below the |\childdocmain| directive):
%
\begin{center}
\begin{tabular}{l}
|\ifchilddoc|\\
|\providecommand{\version}{draft}|\\
|\||else|\\
|\providecommand{\version}{final}|\\
|\||fi|
\end{tabular}
\end{center}
%
The definition by |\providecommand| makes sure
that previous definitions are not overwritten.
Further statements |\providecommand{\version}{...}|
can thus be added before the above code to override it.

For the main file, one might add a line
(between |\childdocmain| and the above block)
%
\begin{center}
|%\ifchilddoc\||else\providecommand{\version}{draft}\||fi|
\end{center}
%
which can be uncommented to produce a draft version.
Likewise one can add a line to the very top of a child file
(above the |\childdocof{|\textit{main}|}| directive)
%
\begin{center}
|%\providecommand{\version}{final}|
\end{center}
%
which can be uncommented to produce the final version of this child document.

%%%%%%%%%%%%%%%%%%%%%%%%%%%%%%%%%%%%%%%%%%%%%%%%%%%%%%%%%%%%%%%%%%%%%%%%%%%%%%%%
\subsection{Forwarding}
\label{sec:forward}

Different versions of the main or child documents
using compilation flags as described in \secref{sec:flags}
can be (permanently) stored in different files
for convenient compilation, viewing and distribution.
To this end, the package defines a command
to pass on compilation to a different file:

%%%%%%%%%%%%%%%%%%%%%%%%%%%%%%%%%%%%%%%%
\DescribeMacro{\childdocforward}
The command |\childdocforward| redirects processing to
another source file:
%
\begin{center}
\begin{tabular}{l}
|\input{childdoc.def}|\\
|\childdocforward[|\textit{main}|]{|\textit{dest}|}|\\
\end{tabular}
\end{center}
%
The argument \textit{dest} is the destination file
(without extension).
It should be the main file or one of the child files.
Note that further \textsf{childdoc} directives
such as |\childdocof| and |\childdocforward|
in the indicated file will be processed in this form.
The optional argument \textit{main}
passes on directly to the main file \textit{main}
while pretending to compile the child \textit{dest}.
This form behaves as if \textit{dest}
issues |\childdocof{|\textit{main}|}| right away,
and no further \textsf{childdoc} directives will be processed.

%%%%%%%%%%%%%%%%%%%%%%%%%%%%%%%%%%%%%%%%
\DescribeMacro{\...prefix}
In the alternative form |\childdocforwardprefix|,
%
\begin{center}
\begin{tabular}{l}
|\input{childdoc.def}|\\
|\childdocforwardprefix[|\textit{main}|]{|\textit{prefix}|}{|\textit{dest}|}|
\end{tabular}
\end{center}
%
the destination file is determined by a pattern
depending on the current file:
To make this work, the current file must be called
`{\textit{prefix}\hspace{0.2em}\textit{suffix}}'
with \textit{prefix} matching precisely the argument.
Processing is then passed on to the file
`{\textit{dest}\hspace{0.2em}\textit{suffix}}'.
Surely, the same effect is achieved by
directly specifying the
argument `{\textit{dest}\hspace{0.2em}\textit{suffix}}'
in the first form.
However, that requires to set up a different file
for each child. With the alternative form of the command
all these files can have exactly the same content
which simplifies setting them up and maintaining them.

For example, the following file |draft.tex|
with a compilation flag |\version| as described in \secref{sec:flags}
compiles the main document as a draft:
%
\begin{center}
\begin{tabular}{l}
|\def\version{draft}|\\
|\input{childdoc.def}|\\
|\childdocforward{|\textit{main}|}|
\end{tabular}
\end{center}
%
Likewise, the following files |final|\textit{nn}|.tex|
compile the final version of the child document
|child|\textit{nn}|.tex|:
%
\begin{center}
\begin{tabular}{l}
|\def\version{final}|\\
|\input{childdoc.def}|\\
|\childdocforwardprefix{final}{child}|
\end{tabular}
\end{center}
%

Note that when several versions of a main file and/or of each child file
are to be generated, it may be convenient to set up a |Makefile| or
shell script to automatise the process.

%%%%%%%%%%%%%%%%%%%%%%%%%%%%%%%%%%%%%%%%%%%%%%%%%%%%%%%%%%%%%%%%%%%%%%%%%%%%%%%%
\subsection{Command Line Processing}
\label{sec:commandline}

The effect of redirection files can also be achieved by invoking
the \LaTeX{} compiler with a more elaborate command line.
Most conveniently this should be done as part
of a shell script or a |Makefile|.

When using \textsf{childdoc} in the main file, the following
command lines effectively perform a redirection
(note that depending on the shell being used,
backslashes may have to be doubled: `|\|' $\to$ `|\\|'):
%
\begin{center}
|... -jobname "|\textit{target}|" |\\|"|[\textit{flags}]%
|\input{childdoc.def}\childdocforward[|\textit{main}|]{|\textit{dest}|}"|
\end{center}
%
Here \textit{target} is the name of the output file,
\textit{main} is the name of the main file
and \textit{dest} is the name of the main or child file to be processed
(all filenames without extensions).
The optional argument \textit{main} can be omitted
if \textit{main} matches \textit{dest}.
Optionally, compilation \textit{flags} can be defined via |\def| commands.
This command line makes the \TeX{} engine believe
it is compiling the file \textit{target}
whose content is specified as the latter parameter.
The provided code then forwards the processing to
\textit{main} or \textit{dest} as described in \secref{sec:forward}.

%%%%%%%%%%%%%%%%%%%%%%%%%%%%%%%%%%%%%%%%%%%%%%%%%%%%%%%%%%%%%%%%%%%%%%%%%%%%%%%%
\subsection{Include by Input}
\label{sec:input}

Including child documents by |\include| has some restrictions by design.
Most notably, the content of a child document always occupies
its own set of pages; pages cannot be shared between child documents.
Usually, this behaviour makes perfect sense
because each child document contain an essential part of the document.
However, in some situations it may be desirable to compose
a document from a collection of parts
without having mandatory page breaks between then.
For this case, the package
provides a mechanism to include parts
by |\input| which can also be processed individually.
However, by construction this mechanism
requires manual handling of the content to be output.

%%%%%%%%%%%%%%%%%%%%%%%%%%%%%%%%%%%%%%%%
\DescribeMacro{\ifchilddocmanual}
The main file should be prepared as usual, see \secref{sec:include}.
However, the document body must make a distinction
between processing of an individual part and of the main document, e.g.:
%
\begin{center}
\begin{tabular}{l}
|\ifchilddocmanual|\\
|\input{\childdocname}|\\
|\||else|\\
\textit{document body with }|\input{|\textit{part}|}|\\
|\||fi|
\end{tabular}
\end{center}
%
The conditional |\ifchilddocmanual| is true whenever
a part to be included by |\input| is being compiled,
and the name of the part is stored in |\childdocname|.

%%%%%%%%%%%%%%%%%%%%%%%%%%%%%%%%%%%%%%%%
\DescribeMacro{\childdocby}
Each part to be included by |\input| should start with:
%
\begin{center}
\begin{tabular}{l}
|\input{childdoc.def}|\\
|\childdocby{|\textit{main}|}|\\
\end{tabular}
\end{center}
%
The directive |\childdocby| is similar to |\childdocof|
described in \secref{sec:include},
but the subsequent selection of content must be done manually.
To that end, both |\ifchilddoc| and |\ifchilddocmanual|
will be true upon processing of a part,
and the name of the part is stored in |\childdocname|.
Note that |\jobname| will be set to the filename of the current part
so that each part receives an individual |.aux| file
that does not interfere with the |.aux| file(s) of the main document.
This behaviour can be altered by the alternative form
|\childdocby[*]{|\textit{main}|}| (with a non-empty optional argument)
which uses the |.aux| file of the main document
by setting |\jobname| to \textit{main}.

%%%%%%%%%%%%%%%%%%%%%%%%%%%%%%%%%%%%%%%%%%%%%%%%%%%%%%%%%%%%%%%%%%%%%%%%%%%%%%%%
\subsection{Driver Development}
\label{sec:driver}

The \textsf{childdoc} mechanism can also be use for the development
of definition files such as \LaTeX{} styles or classes.
This case differs from the above setup with multiple parts
included by |\include| in that no |\includeonly| should be invoked.
This can be achieved by starting the include file
(before |\ProvidesPackage|) with:
%
\begin{center}
\begin{tabular}{l}
|\input{childdoc.def}|\\
|\childdocforward{|\textit{main}|}|\\
\end{tabular}
\end{center}
%
or alternatively with:
%
\begin{center}
\begin{tabular}{l}
|\input{childdoc.def}|\\
|\childdocby{|\textit{main}|}|\\
\end{tabular}
\end{center}
%
Both forms have slightly different effects as described above.
The main file is prepared as usual, see \secref{sec:include}.

%%%%%%%%%%%%%%%%%%%%%%%%%%%%%%%%%%%%%%%%%%%%%%%%%%%%%%%%%%%%%%%%%%%%%%%%%%%%%%%%
\subsection{Legacy Detection}
\label{sec:detection}

The directive |\childdocmain| in the main file can detect
whether the complete document or merely a child is to be compiled
even without using the directive |\childdocof|.
This method is deprecated because it is less robust
and there is no compelling reason to use it;
it is merely provided for backward compatibility
and it may be removed in future versions.

If the detection mechanism is to be used,
it is mandatory to correctly specify
the filename of the main file as the argument of |\childdocmain|:
%
\begin{center}
\begin{tabular}{l}
|\input{childdoc.def}|\\
|\childdocmain{|\textit{main}|}|\\
\end{tabular}
\end{center}
%
If |\jobname| does not match the argument \textit{main} of |\childdocmain|,
it is assumed that |\jobname| points to the child file to be compiled.
When using |\childdocmain| with the main file specified as argument,
it suffices to start a child file
with just |\input{|\textit{main}|}|
without loading of the package and using |\childdocof|.
If instead all processing is done
with the appropriate \textsf{childdoc} directives,
the argument of \textit{main} of |\childdocmain| can be empty.

An alternative version of the command line processing described
in \secref{sec:commandline} using the detection mechanism reads:
%
\begin{center}
|... -jobname "|\textit{target}|" "|[\textit{flags}]%
[|\def\jobname{|\textit{dest}|}|]|\input{|\textit{main}|}"|
\end{center}

%%%%%%%%%%%%%%%%%%%%%%%%%%%%%%%%%%%%%%%%%%%%%%%%%%%%%%%%%%%%%%%%%%%%%%%%%%%%%%%%
\subsection{Manual Code}
\label{sec:manual}

In case one cannot be certain whether the definitions file |childdoc.def|
is installed on the target \TeX{} distribution
and one prefers not to ship it,
it is conceivable to paste a few relevant commands into the sources.

To that end, drop all statements |\input{childdoc.def}|
and perform the replacements as outlined below.
Instead of |\childdocmain{|\textit{main}|}| add the following code
to the top of the main file:
%
\begin{center}
\begin{tabular}{l}
|\||ifdefined\childdocname\endinput\||fi\newif\ifchilddoc|\\
|\edef\childdocname{\scantokens\expandafter{\jobname\noexpand}}|\\
|\def\childdocmain{|\textit{main}|}\||ifx\childdocmain\childdocname\||else|\\
|\childdoctrue\includeonly{\childdocname}\let\jobname\childdocmain\||fi|\\
\end{tabular}
\end{center}
%
Instead of |\childdocof{|\textit{main}|}| just include the main file
at the top of each child file:
%
\begin{center}
|\input{|\textit{main}|}|
\end{center}
%
A simple redirection |\childdocforward{|\textit{dest}|}| is achieved by:
%
\begin{center}
|\def\jobname{|\textit{dest}|}\input{\jobname}|
\end{center}
%
The redirection with prefix
|\childdocforwardprefix[|\textit{prefix}|]{|\textit{dest}|}|
is accomplished by:
%
\begin{center}
\begin{tabular}{l}
|{\edef\jobname{\scantokens\expandafter{\jobname\noexpand}}|\\
|\def\redirectjob |\textit{prefix}|#1~~~{\gdef\jobname{|\textit{dest}|#1}}|\\
|\expandafter\redirectjob\jobname~~~}\input{\jobname}|
\end{tabular}
\end{center}

In an alternative approach,
child documents can be compiled by a specific command line
without additional code or specific definitions:
%
\begin{center}
|... -jobname "|\textit{target}|" "|[\textit{flags}]%
|\includeonly{|\textit{dest}|}\input{|\textit{main}|}"|
\end{center}
%

%%%%%%%%%%%%%%%%%%%%%%%%%%%%%%%%%%%%%%%%%%%%%%%%%%%%%%%%%%%%%%%%%%%%%%%%%%%%%%%%
%%%%%%%%%%%%%%%%%%%%%%%%%%%%%%%%%%%%%%%%%%%%%%%%%%%%%%%%%%%%%%%%%%%%%%%%%%%%%%%%
\section{Information}

%%%%%%%%%%%%%%%%%%%%%%%%%%%%%%%%%%%%%%%%%%%%%%%%%%%%%%%%%%%%%%%%%%%%%%%%%%%%%%%%
\subsection{Copyright}

Copyright \copyright{} 2017--2018 Niklas Beisert

This work may be distributed and/or modified under the
conditions of the \LaTeX{} Project Public License, either version 1.3
of this license or (at your option) any later version.
The latest version of this license is in
  \url{http://www.latex-project.org/lppl.txt}
and version 1.3 or later is part of all distributions of \LaTeX{}
version 2005/12/01 or later.

This work has the LPPL maintenance status `maintained'.

The Current Maintainer of this work is Niklas Beisert.

This work consists of the files |README.txt|, |childdoc.ins| and |childdoc.dtx|
as well as the derived files |childdoc.def|, |cdocsamp.tex|
with |cdocsch1.tex|, |cdocsch2.tex|, |cdocspt3.tex|, |cdocspt4.tex|,
|cdocsdrf.tex|, |cdocsfn1.tex|, |cdocsfn2.tex|
as well as |childdoc.pdf|.

%%%%%%%%%%%%%%%%%%%%%%%%%%%%%%%%%%%%%%%%%%%%%%%%%%%%%%%%%%%%%%%%%%%%%%%%%%%%%%%%
\subsection{Files and Installation}

The package consists of the files:
%
\begin{center}
\begin{tabular}{ll}
    |README.txt|   & readme file \\
    |childdoc.ins| & installation file \\
    |childdoc.dtx| & source file \\
    |childdoc.def| & definition file \\
    |cdocsamp.tex| & sample main file \\
    |cdocsch1.tex| & sample include file \\
    |cdocsch2.tex| & sample include file \\
    |cdocspt3.tex| & sample part file \\
    |cdocspt4.tex| & sample part file \\
    |cdocsdrf.tex| & sample redirection file \\
    |cdocsfn1.tex| & sample redirection file \\
    |cdocsfn2.tex| & sample redirection file \\
    |childdoc.pdf| & manual
\end{tabular}
\end{center}
%
The distribution consists of the files
|README.txt|, |childdoc.ins| and |childdoc.dtx|.
%
\begin{itemize}
\item
Run (pdf)\LaTeX{} on |childdoc.dtx|
to compile the manual |childdoc.pdf| (this file).
\item
Run \LaTeX{} on |childdoc.ins| to create the definitions file |childdoc.def|
and the sample |cdocsamp.tex| with include files
|cdocsch1.tex|, |cdocsch2.tex|, |cdocspt3.tex|, |cdocspt4.tex|,
|cdocsdrf.tex|, |cdocsfn1.tex|, |cdocsfn2.tex|.
Then copy the file |childdoc.def| to an appropriate directory of your \LaTeX{}
distribution, e.g.\ \textit{texmf-root}|/tex/latex/childdoc|.
\end{itemize}

%%%%%%%%%%%%%%%%%%%%%%%%%%%%%%%%%%%%%%%%%%%%%%%%%%%%%%%%%%%%%%%%%%%%%%%%%%%%%%%%
\subsection{Related CTAN Packages}

There are several other packages which offer a similar functionality:
%
\begin{itemize}
\item
The packages
\href{http://ctan.org/pkg/docmute}{\textsf{docmute}},
\href{http://ctan.org/pkg/includex}{\textsf{includex}} and
\href{http://ctan.org/pkg/standalone}{\textsf{standalone}}
provide commands to include only the document body of
a child file thus allowing both files to be compiled individually.
\item
The packages \href{http://ctan.org/pkg/subdocs}{\textsf{subdocs}}
and \href{http://ctan.org/pkg/subfiles}{\textsf{subfiles}}
provide structures in which the main and child documents can be
encapsulated and allowing them to be compiled individually.
The inclusion mechanism is different from the conventional |\include|.
\item
The package \href{http://ctan.org/pkg/combine}{\textsf{combine}}
is an elaborate solution to combine several documents into one.
\end{itemize}
%
See also the CTAN topic \href{http://ctan.org/topic/subdocs}{\textsf{subdocs}}
for further related packages.
The present package differs from the above solutions in that
a document structure constructed with the conventional |\include| mechanism
just needs two extra commands at the top of every file
such that all constituent files can be compiled individually.

%%%%%%%%%%%%%%%%%%%%%%%%%%%%%%%%%%%%%%%%%%%%%%%%%%%%%%%%%%%%%%%%%%%%%%%%%%%%%%%%
%\subsection{Feature Suggestions}
%
%The following is a list of features which may be useful for future
%versions of this package:
%%
%\begin{itemize}
%\item
%\ldots
%\end{itemize}

%%%%%%%%%%%%%%%%%%%%%%%%%%%%%%%%%%%%%%%%%%%%%%%%%%%%%%%%%%%%%%%%%%%%%%%%%%%%%%%%
\subsection{Revision History}

%%%%%%%%%%%%%%%%%%%%%%%%%%%%%%%%%%%%%%%%
\paragraph{v2.0:} 2018/12/30

\begin{itemize}
\item
immediate forward processing
\item
added |\childdocby| mechanism
\item
manual restructured
\end{itemize}

%%%%%%%%%%%%%%%%%%%%%%%%%%%%%%%%%%%%%%%%
\paragraph{v1.6:} 2018/01/17

\begin{itemize}
\item
application for development of include files
\item
corrections to manual
\end{itemize}

%%%%%%%%%%%%%%%%%%%%%%%%%%%%%%%%%%%%%%%%
\paragraph{v1.5:} 2017/05/21

\begin{itemize}
\item
more complete structuring introduced
\item
|\childdocof| introduced
\item
|\childdoc| renamed to |\childdocmain|
\item
|\childredirect| renamed to |\childdocforward| and |\childdocforwardprefix|
and functionality expanded
\end{itemize}

%%%%%%%%%%%%%%%%%%%%%%%%%%%%%%%%%%%%%%%%
\paragraph{v1.0:} 2017/04/27

\begin{itemize}
\item
manual and install package
\item
first version published on CTAN
\end{itemize}

%%%%%%%%%%%%%%%%%%%%%%%%%%%%%%%%%%%%%%%%
\paragraph{v0.6:} 2017/04/26

\begin{itemize}
\item
redirection mechanism added
\end{itemize}

%%%%%%%%%%%%%%%%%%%%%%%%%%%%%%%%%%%%%%%%
\paragraph{v0.5:} 2017/04/26

\begin{itemize}
\item
functionality in definition file
\end{itemize}


%%%%%%%%%%%%%%%%%%%%%%%%%%%%%%%%%%%%%%%%%%%%%%%%%%%%%%%%%%%%%%%%%%%%%%%%%%%%%%%%
%%%%%%%%%%%%%%%%%%%%%%%%%%%%%%%%%%%%%%%%%%%%%%%%%%%%%%%%%%%%%%%%%%%%%%%%%%%%%%%%
%%%%%%%%%%%%%%%%%%%%%%%%%%%%%%%%%%%%%%%%%%%%%%%%%%%%%%%%%%%%%%%%%%%%%%%%%%%%%%%%
\appendix

\settowidth\MacroIndent{\rmfamily\scriptsize 000\ }

 \DocInput{childdoc.dtx}

\end{document}
%</driver>
% \fi
%
% %%%%%%%%%%%%%%%%%%%%%%%%%%%%%%%%%%%%%%%%%%%%%%%%%%%%%%%%%%%%%%%%%%%%%%%%%%%%%%
% %%%%%%%%%%%%%%%%%%%%%%%%%%%%%%%%%%%%%%%%%%%%%%%%%%%%%%%%%%%%%%%%%%%%%%%%%%%%%%
% \section{Sample}
%\iffalse
%<*samplemain>
%\fi
%
% The following presents a sample document
% with two chapters, two parts, a title page,
% a compile flag as well as three forwarding files to set the flag.
% It consists of eight |.tex| files:
% \begin{center}
% \begin{tabular}{ll}
% |cdocsamp.tex|&main file\\
% |cdocsch1.tex|&include file for chapter 1\\
% |cdocsch2.tex|&include file for chapter 2\\
% |cdocspt3.tex|&include file for part 3\\
% |cdocspt4.tex|&include file for part 4\\
% |cdocsdrf.tex|&forwarding file for main file in draft mode\\
% |cdocsfi1.tex|&forwarding file for final version of chapter 1\\
% |cdocsfi2.tex|&forwarding file for final version of chapter 2\\
% \end{tabular}
% \end{center}
% Each of the eight files can be compiled directly by the \LaTeX{} compiler.
%
% %%%%%%%%%%%%%%%%%%%%%%%%%%%%%%%%%%%%%%
% \paragraph{Main File.}
%
% The main file is called |cdocsamp.tex|.
%
% Load the \textsf{childdoc} definitions and
% declare the filename for the main document:
%    \begin{macrocode}
\input{childdoc.def}
\childdocmain{}
%    \end{macrocode}

% Optional override for |\version| flag:
%    \begin{macrocode}
%%\ifchilddoc\else\providecommand{\version}{draft}\fi
%    \end{macrocode}

% Define the default values for the |\version| flag
% (|final| for the main file and |draft| for childs):
%    \begin{macrocode}
\ifchilddoc
\providecommand{\version}{draft}
\else
\providecommand{\version}{final}
\fi
%    \end{macrocode}

% Load the standard document class:
%    \begin{macrocode}
\documentclass[12pt]{article}
%    \end{macrocode}

% Start the document body:
%    \begin{macrocode}
\begin{document}
%    \end{macrocode}

% Declare a title page.
% Print title, part of document being processed and version flag:
%    \begin{macrocode}
\addtocounter{page}{-1}
\begin{center}
{\LARGE\bfseries{}childdoc example\par}
\vspace{1cm}
\ifchilddoc
\ifchilddocmanual part\else chapter\fi:
`\childdocname' of `\childdocjob'\par
\else
main document: `\childdocjob'\par
\fi
version: \version\par
\end{center}
\newpage
%    \end{macrocode}

% Manually include selected file,
% otherwise process as usual:
%    \begin{macrocode}
\ifchilddocmanual
\section*{part `\childdocname'}
\input{\childdocname}
\else
%    \end{macrocode}

% Include the two chapters:
%    \begin{macrocode}
\include{cdocsch1}
\include{cdocsch2}
%    \end{macrocode}

% Include the two parts unless only chapters should be displayed:
%    \begin{macrocode}
\ifchilddoc\else
\section{part three}
\input{cdocspt3}
\section{part four}
\input{cdocspt4}
\fi
%    \end{macrocode}

% Process as usual until here:
%    \begin{macrocode}
\fi
%    \end{macrocode}

% End of document body:
%    \begin{macrocode}
\end{document}
%    \end{macrocode}
%\iffalse
%</samplemain>
%\fi
%
% %%%%%%%%%%%%%%%%%%%%%%%%%%%%%%%%%%%%%%
% \paragraph{Chapter Include Files.}
%
% The include files are called |cdocsch1.tex| and |cdocsch2.tex|.
%
%\iffalse
%<*samplechap1|samplechap2>
%\fi

% Optional override for |\version| flag:
%    \begin{macrocode}
%%\providecommand{\version}{final}
%    \end{macrocode}

% Include the main document:
%    \begin{macrocode}
\input{childdoc.def}
\childdocof{cdocsamp}
%    \end{macrocode}

%\iffalse
%</samplechap1|samplechap2>
%\fi
%
%\iffalse
%<*samplechap1>
%\fi
% Some text for chapter 1:
%    \begin{macrocode}
\section{one}
some text in chapter one
%    \end{macrocode}

%\iffalse
%</samplechap1>
%\fi
% Some text for chapter 2:
%\iffalse
%<*samplechap2>
%\fi
%    \begin{macrocode}
\section{two}
more text in chapter two
%    \end{macrocode}

%\iffalse
%</samplechap2>
%\fi
%
% %%%%%%%%%%%%%%%%%%%%%%%%%%%%%%%%%%%%%%
% \paragraph{Part Include Files.}
%
% The include files are called |cdocspt3.tex| and |cdocspt4.tex|.
%
%\iffalse
%<*samplepart3|samplepart4>
%\fi

% Optional override for |\version| flag:
%    \begin{macrocode}
%%\providecommand{\version}{final}
%    \end{macrocode}

% Include the main document:
%    \begin{macrocode}
\input{childdoc.def}
\childdocby{cdocsamp}
%    \end{macrocode}

%\iffalse
%</samplepart3|samplepart4>
%\fi
%
%\iffalse
%<*samplepart3>
%\fi
% Some text for part 3:
%    \begin{macrocode}
some text in part three
%    \end{macrocode}

%\iffalse
%</samplepart3>
%\fi
% Some text for part 4:
%\iffalse
%<*samplepart4>
%\fi
%    \begin{macrocode}
more text in part four
%    \end{macrocode}

%\iffalse
%</samplepart4>
%\fi
%
% %%%%%%%%%%%%%%%%%%%%%%%%%%%%%%%%%%%%%%
% \paragraph{Forwarding for a Complete Draft.}
%
% The following forwarding file |cdocsdrf.tex|
% compiles the main document in draft mode:
%\iffalse
%<*sampledraft>
%\fi
%    \begin{macrocode}
\def\version{draft}
\input{childdoc.def}
\childdocforward{cdocsamp}
%    \end{macrocode}

%\iffalse
%</sampledraft>
%\fi
%
% %%%%%%%%%%%%%%%%%%%%%%%%%%%%%%%%%%%%%%
% \paragraph{Forwarding for Final Version of the Chapters.}
%
% The following forwarding files |cdocsfn1.tex| and |cdocsfn2.tex|
% (with identical content)
% compile the final versions of the child documents
% |cdocsch1.tex| and |cdocsch2.tex|, respectively:
%\iffalse
%<*samplefinal>
%\fi
%    \begin{macrocode}
\def\version{final}
\input{childdoc.def}
\childdocforwardprefix[cdocsamp]{cdocsfn}{cdocsch}
%    \end{macrocode}

%\iffalse
%</samplefinal>
%\fi
%
% %%%%%%%%%%%%%%%%%%%%%%%%%%%%%%%%%%%%%%
% \paragraph{Command Line Processing.}
%
% The following three command lines generate the output files
% |cdocscld|, |cdocscl1| and |cdocscl2|
% which should be identical to
% |cdocsdrf|, |cdocsch1| and |cdocsfn2|, respectively:
% \begin{center}
% \begin{tabular}{l}
% |latex -jobname cdocscld \|\\
% |  "\def\version{draft}\input{childdoc.def}\childdocforward{cdocsamp}"|\\
% |latex -jobname cdocscl1 \|\\
% |  "\input{childdoc.def}\childdocforward[cdocsamp]{cdocsch1}"|\\
% |latex -jobname cdocscl2 \|\\
% |  "\def\version{final}\input{childdoc.def}\childdocforward{cdocsch2}"|
% \end{tabular}
% \end{center}
% Note that the trailing backslash on each first line
% merely continues the input to the second line
% (for convenient cut ant paste).
% Furthermore, the command |latex| can be replaced by any
% of its alternative versions such as |pdflatex|.
%
% %%%%%%%%%%%%%%%%%%%%%%%%%%%%%%%%%%%%%%%%%%%%%%%%%%%%%%%%%%%%%%%%%%%%%%%%%%%%%%
% %%%%%%%%%%%%%%%%%%%%%%%%%%%%%%%%%%%%%%%%%%%%%%%%%%%%%%%%%%%%%%%%%%%%%%%%%%%%%%
% \section{Implementation}
%\iffalse
%<*package>
%\fi
%
% This section describes the definitions file |childdoc.def|.

% The definitions cannot be loaded using |\usepackage| or |\RequirePackage|
% which has a mechanism to prevent loading a style file more than once.
% When loading the definitions by means of |\input|
% multiple instances have to be prevented manually:
%\iffalse
%This code needs to be before the `\ProvidesFile' directive
%which is defined at the beginning of this file.
%Therefore it is also placed there and commented out here.
%</package>
%<*discard>
%\fi
%    \begin{macrocode}
\ifdefined\childdocmain\endinput\fi
%    \end{macrocode}
%\iffalse
%</discard>
%<*package>
%\fi
%
% \macro{\ifchilddoc}
% \macro{\ifchilddocmanual}
% The conditional |\ifchilddoc| tells whether a
% child (true) or main (false) document is being compiled.
% The conditional |\ifchilddocmanual| tells whether
% the |\includeonly| mechanism is used (false) or
% the selection of child files must be performed manually (true).
% The definitions initialise to false:
%    \begin{macrocode}
\newif\ifchilddoc
\newif\ifchilddocmanual
%    \end{macrocode}

% \macro{\childdocname}
% \macro{\childdocjob}
% The macro |\childdocname| stores the name of the main document
% to be compiled. The macro |\childdocjob| stores the name of
% the document on which the \LaTeX{} compiler was originally invoked.
% The content of |\jobname| cannot be compared
% to filenames specified in the source due to different catcodes.
% The following code rescans |\jobname|, stores the result
% in |\childdocname| and saves a copy in |\childdocjob|:
%    \begin{macrocode}
\edef\childdocname{\scantokens\expandafter{\jobname\noexpand}}
\let\childdocjob\childdocname
%    \end{macrocode}

% \macro{\childdocdisable}
% The macro |\childdocdisable| prevents the main file
% from being processed more than once.
% At this stage, the main document command |\childdocmain|
% is assumed to be called once again where it should do nothing.
% Any subsequent call to it should prevent
% a secondary processing of the main document
% It overwrites the forwarding commands
% |\childdocof| and |\childdocforward|
% with empty macros to prevent further inclusions of the main document:
%    \begin{macrocode}
\newcommand{\childdocdisable}
{
  \renewcommand{\childdocmain}[1]{\renewcommand{\childdocmain}[1]{\endinput}}
  \renewcommand{\childdocof}[1]{}
  \renewcommand{\childdocby}[2][]{}
  \renewcommand{\childdocforward}[2][]{}
  \renewcommand{\childdocdisable}{}
}
%    \end{macrocode}

% \macro{\childdocmain}
% The macro |\childdocmain| is to be called at the top of the main file
% with nothing or the main filename (without extension) as argument.
% First, it breaks loops.
% If the argument is not empty and does not match |\childdocname|
% (which is set by the first inclusion of |childdoc.def|),
% |\ifchilddoc| is set to true, |\includeonly| is applied to the child file
% and |\jobname| is set to the main file
% (for proper handling of |.aux| files):
%    \begin{macrocode}
\newcommand{\childdocmain}[1]
{
  \childdocdisable\childdocmain{}
  \if?#1?\else
    \begingroup
      \def\childdoctmp{#1}
      \ifx\childdoctmp\childdocname
        \def\childdoctmp{}
      \else
        \def\childdoctmp
        {
          \childdoctrue
          \includeonly{\childdocname}
          \def\childdocjob{#1}
          \def\jobname{#1}
        }
      \fi
      \expandafter
    \endgroup
    \childdoctmp
  \fi
}
%    \end{macrocode}

% \macro{\childdocof}
% The command |\childdocof| redirects
% compilation to the main file |#1|.
%    \begin{macrocode}
\newcommand{\childdocof}[1]
{
  \childdocdisable
  \childdoctrue
  \includeonly{\childdocname}
  \def\jobname{#1}
  \def\childdocjob{#1}
  \input{#1}
}
%    \end{macrocode}

% \macro{\childdocby}
% The command |\childdocby| ....
%    \begin{macrocode}
\newcommand{\childdocby}[2][]
{
  \childdocdisable
  \childdoctrue
  \childdocmanualtrue
  \if?#1?\else
    \def\jobname{#2}
  \fi
  \def\childdocjob{#2}
  \input{#2}
  \endinput
}
%    \end{macrocode}

% \macro{\childdocforward}
% The command |\childdocforward| redirects
% compilation to the main file or
% (if the optional argument is given) a child file.
% Parameters are set as if the main file
% or a child file starting with |\childdocof| was compiled.
% Then compilation is handed over to the main file:
%    \begin{macrocode}
\newcommand{\childdocforward}[2][]
{
  \begingroup
    \if?#1?
      \def\childdoctmp
      {
        \def\childdocname{#2}
        \def\childdocjob{#2}
        \def\jobname{#2}
        \input{#2}
        \endinput
      }
    \else
      \def\childdoctmp
      {
        \childdocdisable
        \def\childdocname{#2}
        \childdoctrue
        \includeonly{#2}
        \def\childdocjob{#1}
        \def\jobname{#1}
        \input{#1}
        \endinput
      }
    \fi
    \expandafter
  \endgroup
  \childdoctmp
}
%    \end{macrocode}

% \macro{\childdocforwardprefix}
% The command |\childdocforwardprefix| redirects
% compilation to the main or a child file by means of a pattern.
% The prefix |#1| in the current filename is replaced by |#2|
% and the suffix of the current filename is kept
% (it is assumed that the filename does not contain the substring `|~~~|'
% which is used as a delimiter).
% Compilation is handed over to the new file by |\childdocforward|:
%    \begin{macrocode}
\newcommand{\childdocforwardprefix}[3][]
{
  \begingroup
    \def\childdocextract #2##1~~~{\def\childdoctmp{\childdocforward[#1]{#3##1}}}
    \expandafter\childdocextract\childdocname~~~
    \expandafter
  \endgroup
  \childdoctmp
}
%    \end{macrocode}

% \macro{\childdoc}
% The deprecated macro |\childdoc| is a legacy version of |\childdocmain|:
%    \begin{macrocode}
\newcommand{\childdoc}{\childdocmain}
%    \end{macrocode}

% \macro{\childdocredirect}
% The deprecated macro |\childdocredirect| is a legacy version
% of |\childdocforward| and |\childdocforwardprefix|:
%    \begin{macrocode}
\newcommand{\childdocredirect}[2][]
{
  \begingroup
    \if?#1?
      \def\childdoctmp{\childdocforward{#2}}
    \else
      \def\childdoctmp{\childdocforwardprefix{#1}{#2}}
    \fi
    \expandafter
  \endgroup
  \childdoctmp
}
%    \end{macrocode}

%\iffalse
%</package>
%\fi
%
\endinput
\childdocforward{cdocsch2}"|
% \end{tabular}
% \end{center}
% Note that the trailing backslash on each first line
% merely continues the input to the second line
% (for convenient cut ant paste).
% Furthermore, the command |latex| can be replaced by any
% of its alternative versions such as |pdflatex|.
%
% %%%%%%%%%%%%%%%%%%%%%%%%%%%%%%%%%%%%%%%%%%%%%%%%%%%%%%%%%%%%%%%%%%%%%%%%%%%%%%
% %%%%%%%%%%%%%%%%%%%%%%%%%%%%%%%%%%%%%%%%%%%%%%%%%%%%%%%%%%%%%%%%%%%%%%%%%%%%%%
% \section{Implementation}
%\iffalse
%<*package>
%\fi
%
% This section describes the definitions file |childdoc.def|.

% The definitions cannot be loaded using |\usepackage| or |\RequirePackage|
% which has a mechanism to prevent loading a style file more than once.
% When loading the definitions by means of |\input|
% multiple instances have to be prevented manually:
%\iffalse
%This code needs to be before the `\ProvidesFile' directive
%which is defined at the beginning of this file.
%Therefore it is also placed there and commented out here.
%</package>
%<*discard>
%\fi
%    \begin{macrocode}
\ifdefined\childdocmain\endinput\fi
%    \end{macrocode}
%\iffalse
%</discard>
%<*package>
%\fi
%
% \macro{\ifchilddoc}
% \macro{\ifchilddocmanual}
% The conditional |\ifchilddoc| tells whether a
% child (true) or main (false) document is being compiled.
% The conditional |\ifchilddocmanual| tells whether
% the |\includeonly| mechanism is used (false) or
% the selection of child files must be performed manually (true).
% The definitions initialise to false:
%    \begin{macrocode}
\newif\ifchilddoc
\newif\ifchilddocmanual
%    \end{macrocode}

% \macro{\childdocname}
% \macro{\childdocjob}
% The macro |\childdocname| stores the name of the main document
% to be compiled. The macro |\childdocjob| stores the name of
% the document on which the \LaTeX{} compiler was originally invoked.
% The content of |\jobname| cannot be compared
% to filenames specified in the source due to different catcodes.
% The following code rescans |\jobname|, stores the result
% in |\childdocname| and saves a copy in |\childdocjob|:
%    \begin{macrocode}
\edef\childdocname{\scantokens\expandafter{\jobname\noexpand}}
\let\childdocjob\childdocname
%    \end{macrocode}

% \macro{\childdocdisable}
% The macro |\childdocdisable| prevents the main file
% from being processed more than once.
% At this stage, the main document command |\childdocmain|
% is assumed to be called once again where it should do nothing.
% Any subsequent call to it should prevent
% a secondary processing of the main document
% It overwrites the forwarding commands
% |\childdocof| and |\childdocforward|
% with empty macros to prevent further inclusions of the main document:
%    \begin{macrocode}
\newcommand{\childdocdisable}
{
  \renewcommand{\childdocmain}[1]{\renewcommand{\childdocmain}[1]{\endinput}}
  \renewcommand{\childdocof}[1]{}
  \renewcommand{\childdocby}[2][]{}
  \renewcommand{\childdocforward}[2][]{}
  \renewcommand{\childdocdisable}{}
}
%    \end{macrocode}

% \macro{\childdocmain}
% The macro |\childdocmain| is to be called at the top of the main file
% with nothing or the main filename (without extension) as argument.
% First, it breaks loops.
% If the argument is not empty and does not match |\childdocname|
% (which is set by the first inclusion of |childdoc.def|),
% |\ifchilddoc| is set to true, |\includeonly| is applied to the child file
% and |\jobname| is set to the main file
% (for proper handling of |.aux| files):
%    \begin{macrocode}
\newcommand{\childdocmain}[1]
{
  \childdocdisable\childdocmain{}
  \if?#1?\else
    \begingroup
      \def\childdoctmp{#1}
      \ifx\childdoctmp\childdocname
        \def\childdoctmp{}
      \else
        \def\childdoctmp
        {
          \childdoctrue
          \includeonly{\childdocname}
          \def\childdocjob{#1}
          \def\jobname{#1}
        }
      \fi
      \expandafter
    \endgroup
    \childdoctmp
  \fi
}
%    \end{macrocode}

% \macro{\childdocof}
% The command |\childdocof| redirects
% compilation to the main file |#1|.
%    \begin{macrocode}
\newcommand{\childdocof}[1]
{
  \childdocdisable
  \childdoctrue
  \includeonly{\childdocname}
  \def\jobname{#1}
  \def\childdocjob{#1}
  \input{#1}
}
%    \end{macrocode}

% \macro{\childdocby}
% The command |\childdocby| ....
%    \begin{macrocode}
\newcommand{\childdocby}[2][]
{
  \childdocdisable
  \childdoctrue
  \childdocmanualtrue
  \if?#1?\else
    \def\jobname{#2}
  \fi
  \def\childdocjob{#2}
  \input{#2}
  \endinput
}
%    \end{macrocode}

% \macro{\childdocforward}
% The command |\childdocforward| redirects
% compilation to the main file or
% (if the optional argument is given) a child file.
% Parameters are set as if the main file
% or a child file starting with |\childdocof| was compiled.
% Then compilation is handed over to the main file:
%    \begin{macrocode}
\newcommand{\childdocforward}[2][]
{
  \begingroup
    \if?#1?
      \def\childdoctmp
      {
        \def\childdocname{#2}
        \def\childdocjob{#2}
        \def\jobname{#2}
        \input{#2}
        \endinput
      }
    \else
      \def\childdoctmp
      {
        \childdocdisable
        \def\childdocname{#2}
        \childdoctrue
        \includeonly{#2}
        \def\childdocjob{#1}
        \def\jobname{#1}
        \input{#1}
        \endinput
      }
    \fi
    \expandafter
  \endgroup
  \childdoctmp
}
%    \end{macrocode}

% \macro{\childdocforwardprefix}
% The command |\childdocforwardprefix| redirects
% compilation to the main or a child file by means of a pattern.
% The prefix |#1| in the current filename is replaced by |#2|
% and the suffix of the current filename is kept
% (it is assumed that the filename does not contain the substring `|~~~|'
% which is used as a delimiter).
% Compilation is handed over to the new file by |\childdocforward|:
%    \begin{macrocode}
\newcommand{\childdocforwardprefix}[3][]
{
  \begingroup
    \def\childdocextract #2##1~~~{\def\childdoctmp{\childdocforward[#1]{#3##1}}}
    \expandafter\childdocextract\childdocname~~~
    \expandafter
  \endgroup
  \childdoctmp
}
%    \end{macrocode}

% \macro{\childdoc}
% The deprecated macro |\childdoc| is a legacy version of |\childdocmain|:
%    \begin{macrocode}
\newcommand{\childdoc}{\childdocmain}
%    \end{macrocode}

% \macro{\childdocredirect}
% The deprecated macro |\childdocredirect| is a legacy version
% of |\childdocforward| and |\childdocforwardprefix|:
%    \begin{macrocode}
\newcommand{\childdocredirect}[2][]
{
  \begingroup
    \if?#1?
      \def\childdoctmp{\childdocforward{#2}}
    \else
      \def\childdoctmp{\childdocforwardprefix{#1}{#2}}
    \fi
    \expandafter
  \endgroup
  \childdoctmp
}
%    \end{macrocode}

%\iffalse
%</package>
%\fi
%
\endinput

\childdocof{cdocsamp}
%    \end{macrocode}

%\iffalse
%</samplechap1|samplechap2>
%\fi
%
%\iffalse
%<*samplechap1>
%\fi
% Some text for chapter 1:
%    \begin{macrocode}
\section{one}
some text in chapter one
%    \end{macrocode}

%\iffalse
%</samplechap1>
%\fi
% Some text for chapter 2:
%\iffalse
%<*samplechap2>
%\fi
%    \begin{macrocode}
\section{two}
more text in chapter two
%    \end{macrocode}

%\iffalse
%</samplechap2>
%\fi
%
% %%%%%%%%%%%%%%%%%%%%%%%%%%%%%%%%%%%%%%
% \paragraph{Part Include Files.}
%
% The include files are called |cdocspt3.tex| and |cdocspt4.tex|.
%
%\iffalse
%<*samplepart3|samplepart4>
%\fi

% Optional override for |\version| flag:
%    \begin{macrocode}
%%\providecommand{\version}{final}
%    \end{macrocode}

% Include the main document:
%    \begin{macrocode}
% \iffalse
%
% childdoc.dtx Copyright (C) 2017-2018 Niklas Beisert
%
% This work may be distributed and/or modified under the
% conditions of the LaTeX Project Public License, either version 1.3
% of this license or (at your option) any later version.
% The latest version of this license is in
%   http://www.latex-project.org/lppl.txt
% and version 1.3 or later is part of all distributions of LaTeX
% version 2005/12/01 or later.
%
% This work has the LPPL maintenance status `maintained'.
%
% The Current Maintainer of this work is Niklas Beisert.
%
% This work consists of the files childdoc.dtx and childdoc.ins
% and the derived files childdoc.def and cdocsamp.tex with
% cdocsch1.tex, cdocsch2.tex, cdocsdrf.tex, cdocsfn1.tex, cdocsfn2.tex.
%
%<package>\ifdefined\childdocmain\endinput\fi
%<package>\ProvidesFile{childdoc.def}[2018/12/30 v2.0 child document driver]
%<samplemain>\ProvidesFile{cdocsamp.tex}[2018/12/30 v2.0 sample for childdoc]
%<*driver>
%\ProvidesFile{childdoc.drv}[2018/12/30 v2.0 childdoc reference manual file]
\PassOptionsToClass{10pt,a4paper}{article}
\documentclass{ltxdoc}

\usepackage[margin=35mm]{geometry}
\usepackage{hyperref}
\usepackage{hyperxmp}
\usepackage[usenames]{color}

\hypersetup{colorlinks=true}
\hypersetup{pdfstartview=FitH}
\hypersetup{pdfpagemode=UseNone}
\hypersetup{pdfsource={}}
\hypersetup{pdflang={en-UK}}
\hypersetup{pdfcopyright={Copyright 2017-2018 Niklas Beisert.
  This work may be distributed and/or modified under the
  conditions of the LaTeX Project Public License, either version 1.3
  of this license or (at your option) any later version.}}
\hypersetup{pdflicenseurl={http://www.latex-project.org/lppl.txt}}
\hypersetup{pdfcontactaddress={ETH Zurich, ITP, HIT K,
  Wolfgang-Pauli-Strasse 27}}
\hypersetup{pdfcontactpostcode={8093}}
\hypersetup{pdfcontactcity={Zurich}}
\hypersetup{pdfcontactcountry={Switzerland}}
\hypersetup{pdfcontactemail={nbeisert@itp.phys.ethz.ch}}
\hypersetup{pdfcontacturl={http://people.phys.ethz.ch/\xmptilde nbeisert/}}

\newcommand{\secref}[1]{\hyperref[#1]{section \ref*{#1}}}

\parskip1ex
\parindent0pt
\let\olditemize\itemize
\def\itemize{\olditemize\parskip0pt}

\begin{document}

\title{The \textsf{childdoc} Package}
\hypersetup{pdftitle={The childdoc Package}}
\author{Niklas Beisert\\[2ex]
  Institut f\"ur Theoretische Physik\\
  Eidgen\"ossische Technische Hochschule Z\"urich\\
  Wolfgang-Pauli-Strasse 27, 8093 Z\"urich, Switzerland\\[1ex]
  \href{mailto:nbeisert@itp.phys.ethz.ch}
  {\texttt{nbeisert@itp.phys.ethz.ch}}}
\hypersetup{pdfauthor={Niklas Beisert}}
\hypersetup{pdfsubject={Manual for the LaTeX2e Package childdoc}}
\date{30 December 2018, \textsf{v2.0}}
\maketitle

\begin{abstract}\noindent
\textsf{childdoc} is a \LaTeXe{} package
that enables the direct compilation
of document sections included by |\include|
to individual files.
\end{abstract}

\begingroup
\parskip0ex
\tableofcontents
\endgroup

%%%%%%%%%%%%%%%%%%%%%%%%%%%%%%%%%%%%%%%%%%%%%%%%%%%%%%%%%%%%%%%%%%%%%%%%%%%%%%%%
%%%%%%%%%%%%%%%%%%%%%%%%%%%%%%%%%%%%%%%%%%%%%%%%%%%%%%%%%%%%%%%%%%%%%%%%%%%%%%%%
\section{Introduction}

\LaTeX{} provides a mechanism to structure a large document (such as a book)
into a main file and several child files (containing the chapters)
using the |\include| command.
This mechanism is beneficial for documents
which span hundreds of pages in order to
make the source file(s) more manageable.
Moreover, compilation can be restricted to
selected child files by means of the |\includeonly| command.
The latter feature can be used to reduce the compilation time while editing
(this was significantly more useful in the earlier days of \LaTeX{})
or to generate a smaller document which is easier to navigate.
Another application of |\includeonly| is to generate
documents consisting of selected parts of the complete document.

However, there are a few drawbacks of the plain |\include| mechanism:
\begin{itemize}
\item
The child files cannot be compiled on their own,
they can only be compiled via the main file.
A naive editing environment
(such as a text editor with an option
to have the current file processed by \LaTeX)
may require one to switch to the main file before compiling;
attempting to compile the child file produces errors.
\item
The main file must be modified (each time)
to adjust the |\includeonly| command
to the present needs. This easily leaves the main file in a messy state.
\item
The generated document will always carry the filename
of the main document. This is inconvenient if
several child files are to be compiled and
to be kept for distribution.
\end{itemize}

The present package provides a simple interface
to make child files individually compilable by \LaTeX{}.
Compiling a child file then has the same effect as compiling
the main file with an |\includeonly| command
to select the appropriate child.
Moreover the generated document will carry the name of the child
rather than the main file.
This resolves all three above issues.

This feature is meant to make the editing of books,
thesis documents and lecture notes somewhat more convenient.
However, the package can also be used efficiently for
composing a series of documents (such as exercise sheets)
which are typically distributed individually.
It then assists the author in generating the individual documents
(potentially in different versions)
as well as a document containing the collected series.
Another application is in developing style files
or other kinds of included material
where compilation of the style file could redirect
to a sample or test file.

%%%%%%%%%%%%%%%%%%%%%%%%%%%%%%%%%%%%%%%%%%%%%%%%%%%%%%%%%%%%%%%%%%%%%%%%%%%%%%%%
%%%%%%%%%%%%%%%%%%%%%%%%%%%%%%%%%%%%%%%%%%%%%%%%%%%%%%%%%%%%%%%%%%%%%%%%%%%%%%%%
\section{Usage}

First of all, the package \textsf{childdoc} is \emph{not} a standard
\LaTeXe{} |.sty| style file! Therefore it needs to be invoked in
a non-standard way.

%%%%%%%%%%%%%%%%%%%%%%%%%%%%%%%%%%%%%%%%%%%%%%%%%%%%%%%%%%%%%%%%%%%%%%%%%%%%%%%%
\subsection{Included Files}
\label{sec:include}

%%%%%%%%%%%%%%%%%%%%%%%%%%%%%%%%%%%%%%%%
\DescribeMacro{\childdocmain}
To use the package, add the commands
\begin{center}
\begin{tabular}{l}
|% \iffalse
%
% childdoc.dtx Copyright (C) 2017-2018 Niklas Beisert
%
% This work may be distributed and/or modified under the
% conditions of the LaTeX Project Public License, either version 1.3
% of this license or (at your option) any later version.
% The latest version of this license is in
%   http://www.latex-project.org/lppl.txt
% and version 1.3 or later is part of all distributions of LaTeX
% version 2005/12/01 or later.
%
% This work has the LPPL maintenance status `maintained'.
%
% The Current Maintainer of this work is Niklas Beisert.
%
% This work consists of the files childdoc.dtx and childdoc.ins
% and the derived files childdoc.def and cdocsamp.tex with
% cdocsch1.tex, cdocsch2.tex, cdocsdrf.tex, cdocsfn1.tex, cdocsfn2.tex.
%
%<package>\ifdefined\childdocmain\endinput\fi
%<package>\ProvidesFile{childdoc.def}[2018/12/30 v2.0 child document driver]
%<samplemain>\ProvidesFile{cdocsamp.tex}[2018/12/30 v2.0 sample for childdoc]
%<*driver>
%\ProvidesFile{childdoc.drv}[2018/12/30 v2.0 childdoc reference manual file]
\PassOptionsToClass{10pt,a4paper}{article}
\documentclass{ltxdoc}

\usepackage[margin=35mm]{geometry}
\usepackage{hyperref}
\usepackage{hyperxmp}
\usepackage[usenames]{color}

\hypersetup{colorlinks=true}
\hypersetup{pdfstartview=FitH}
\hypersetup{pdfpagemode=UseNone}
\hypersetup{pdfsource={}}
\hypersetup{pdflang={en-UK}}
\hypersetup{pdfcopyright={Copyright 2017-2018 Niklas Beisert.
  This work may be distributed and/or modified under the
  conditions of the LaTeX Project Public License, either version 1.3
  of this license or (at your option) any later version.}}
\hypersetup{pdflicenseurl={http://www.latex-project.org/lppl.txt}}
\hypersetup{pdfcontactaddress={ETH Zurich, ITP, HIT K,
  Wolfgang-Pauli-Strasse 27}}
\hypersetup{pdfcontactpostcode={8093}}
\hypersetup{pdfcontactcity={Zurich}}
\hypersetup{pdfcontactcountry={Switzerland}}
\hypersetup{pdfcontactemail={nbeisert@itp.phys.ethz.ch}}
\hypersetup{pdfcontacturl={http://people.phys.ethz.ch/\xmptilde nbeisert/}}

\newcommand{\secref}[1]{\hyperref[#1]{section \ref*{#1}}}

\parskip1ex
\parindent0pt
\let\olditemize\itemize
\def\itemize{\olditemize\parskip0pt}

\begin{document}

\title{The \textsf{childdoc} Package}
\hypersetup{pdftitle={The childdoc Package}}
\author{Niklas Beisert\\[2ex]
  Institut f\"ur Theoretische Physik\\
  Eidgen\"ossische Technische Hochschule Z\"urich\\
  Wolfgang-Pauli-Strasse 27, 8093 Z\"urich, Switzerland\\[1ex]
  \href{mailto:nbeisert@itp.phys.ethz.ch}
  {\texttt{nbeisert@itp.phys.ethz.ch}}}
\hypersetup{pdfauthor={Niklas Beisert}}
\hypersetup{pdfsubject={Manual for the LaTeX2e Package childdoc}}
\date{30 December 2018, \textsf{v2.0}}
\maketitle

\begin{abstract}\noindent
\textsf{childdoc} is a \LaTeXe{} package
that enables the direct compilation
of document sections included by |\include|
to individual files.
\end{abstract}

\begingroup
\parskip0ex
\tableofcontents
\endgroup

%%%%%%%%%%%%%%%%%%%%%%%%%%%%%%%%%%%%%%%%%%%%%%%%%%%%%%%%%%%%%%%%%%%%%%%%%%%%%%%%
%%%%%%%%%%%%%%%%%%%%%%%%%%%%%%%%%%%%%%%%%%%%%%%%%%%%%%%%%%%%%%%%%%%%%%%%%%%%%%%%
\section{Introduction}

\LaTeX{} provides a mechanism to structure a large document (such as a book)
into a main file and several child files (containing the chapters)
using the |\include| command.
This mechanism is beneficial for documents
which span hundreds of pages in order to
make the source file(s) more manageable.
Moreover, compilation can be restricted to
selected child files by means of the |\includeonly| command.
The latter feature can be used to reduce the compilation time while editing
(this was significantly more useful in the earlier days of \LaTeX{})
or to generate a smaller document which is easier to navigate.
Another application of |\includeonly| is to generate
documents consisting of selected parts of the complete document.

However, there are a few drawbacks of the plain |\include| mechanism:
\begin{itemize}
\item
The child files cannot be compiled on their own,
they can only be compiled via the main file.
A naive editing environment
(such as a text editor with an option
to have the current file processed by \LaTeX)
may require one to switch to the main file before compiling;
attempting to compile the child file produces errors.
\item
The main file must be modified (each time)
to adjust the |\includeonly| command
to the present needs. This easily leaves the main file in a messy state.
\item
The generated document will always carry the filename
of the main document. This is inconvenient if
several child files are to be compiled and
to be kept for distribution.
\end{itemize}

The present package provides a simple interface
to make child files individually compilable by \LaTeX{}.
Compiling a child file then has the same effect as compiling
the main file with an |\includeonly| command
to select the appropriate child.
Moreover the generated document will carry the name of the child
rather than the main file.
This resolves all three above issues.

This feature is meant to make the editing of books,
thesis documents and lecture notes somewhat more convenient.
However, the package can also be used efficiently for
composing a series of documents (such as exercise sheets)
which are typically distributed individually.
It then assists the author in generating the individual documents
(potentially in different versions)
as well as a document containing the collected series.
Another application is in developing style files
or other kinds of included material
where compilation of the style file could redirect
to a sample or test file.

%%%%%%%%%%%%%%%%%%%%%%%%%%%%%%%%%%%%%%%%%%%%%%%%%%%%%%%%%%%%%%%%%%%%%%%%%%%%%%%%
%%%%%%%%%%%%%%%%%%%%%%%%%%%%%%%%%%%%%%%%%%%%%%%%%%%%%%%%%%%%%%%%%%%%%%%%%%%%%%%%
\section{Usage}

First of all, the package \textsf{childdoc} is \emph{not} a standard
\LaTeXe{} |.sty| style file! Therefore it needs to be invoked in
a non-standard way.

%%%%%%%%%%%%%%%%%%%%%%%%%%%%%%%%%%%%%%%%%%%%%%%%%%%%%%%%%%%%%%%%%%%%%%%%%%%%%%%%
\subsection{Included Files}
\label{sec:include}

%%%%%%%%%%%%%%%%%%%%%%%%%%%%%%%%%%%%%%%%
\DescribeMacro{\childdocmain}
To use the package, add the commands
\begin{center}
\begin{tabular}{l}
|\input{childdoc.def}|\\
|\childdocmain{}|\\
\end{tabular}
\end{center}
at the very top of the main \LaTeX{} file,
in particular \emph{before} the |\documentclass| statement!
The argument of |\childdocmain| should be left empty
(but it must be present).

%%%%%%%%%%%%%%%%%%%%%%%%%%%%%%%%%%%%%%%%
\DescribeMacro{\childdocof}
Furthermore, add the commands
\begin{center}
\begin{tabular}{l}
|\input{childdoc.def}|\\
|\childdocof{|\textit{main}|}|\\
\end{tabular}
\end{center}
at the top of every child file \textit{child}
which is included by |\include{|\textit{child}|}|
from within the main file
(or at least for those files to be compiled individually).
The argument \textit{main} must be the filename of the main file.

There are a couple of
considerations in setting up the main and child documents:

%%%%%%%%%%%%%%%%%%%%%%%%%%%%%%%%%%%%%%%%
\paragraph{Restrictions.}

Please note the following restrictions:
\begin{itemize}
\item
|\childdocmain| must be called with one argument \textit{main}
to ensure compatibility with earlier version of the package.
It must either be empty (|\childdocmain{}|)
or precisely match the filename of the main file in which it is specified.
See \secref{sec:detection} for further information.
\item
The filename \textit{main} must be specified without the |.tex| extension.
\item
The filename \textit{main} is case sensitive
(even in case-insensitive file systems)
due to internal string comparison.
\item
The argument \textit{main} should be fully expanded, it cannot be a macro.
\item
Subdirectories and special characters should be avoided in filenames.
\item
The command |\childdocmain{|\textit{main}|}| must be followed by a whitespace.
It should not be followed immediately by another command
or by a comment mark `|%|'.
This is because the \TeX{} parser reads the token immediately following
the argument of |\childdocmain| and puts it
at the beginning of every child section;
however, a white\-space is ignored.
\end{itemize}

%%%%%%%%%%%%%%%%%%%%%%%%%%%%%%%%%%%%%%%%
\paragraph{Content of Main File.}

It is advisable to place all content in the child files included by |\include|.
Any output contained in the main file will appear in all child documents
unless suppressed manually;
it cannot be suppressed automatically by the |\includeonly| directive
and thus should normally be avoided.
A method to include some content in the main file
by means of conditional processing is described in \secref{sec:conditional}.

%%%%%%%%%%%%%%%%%%%%%%%%%%%%%%%%%%%%%%%%
\paragraph{Page Numbering.}

When only a part of the document is compiled,
the appropriate numbering of pages
(as well as other status parameters)
is determined from the |.aux| files.
The latter contain information from previous passes.
However this information needs to propagate through
all intermediate child documents.
Therefore the page numbering in child documents may well
be inconsistent until the complete document is compiled at least once.

A useful (if unconventional) way to always ensure a consistent
page numbering is to restart the numbering in each child document
and denote the pages by `\textit{child}|.|\textit{page}'
where \textit{child} represents the chapter/section number of the child file.
This can be achieved by the command
|\numberwithin{page}{|\textit{child}|}|
of the \textsf{amsmath} package
where \textit{child} can be |chapter| or |section|
depending on the chosen structuring.
Alternatively, one can modify the macro |\thepage| appropriately
and reset the counter |page| at the start of each child file.

%%%%%%%%%%%%%%%%%%%%%%%%%%%%%%%%%%%%%%%%%%%%%%%%%%%%%%%%%%%%%%%%%%%%%%%%%%%%%%%%
\subsection{Conditional Processing}
\label{sec:conditional}

The package provides a mechanism to compile different versions
of a document. To customise the versions further some conditional processing
can come in handy to distinguish which version is being compiled.
The package provides two macros to describe the compilation context:

%%%%%%%%%%%%%%%%%%%%%%%%%%%%%%%%%%%%%%%%
\DescribeMacro{\ifchilddoc}
The conditional |\ifchilddoc| distinguishes between the compilation of
child documents and the main document:
%
\begin{center}
|\ifchilddoc |\textit{child-code}| |[|\||else |\textit{main-code}]| \||fi|
\end{center}

%%%%%%%%%%%%%%%%%%%%%%%%%%%%%%%%%%%%%%%%
\DescribeMacro{\childdocname}
\DescribeMacro{\childdocjob}
The macro |\childdocname| contains the filename (without extension)
of the main or child file being processed.
Note that |\childdocjob| will always contain the name of the main file.

%%%%%%%%%%%%%%%%%%%%%%%%%%%%%%%%%%%%%%%%
\paragraph{Title Page.}

Conditional processing can be used to include a title or banner page
in the main document when proper precautions are taken.
Importantly, the code in the main file should ensure that the page counter
(as well as other status parameters which are stored in the |.aux| files)
takes the same value after the conditional processing.
Otherwise the page numbers may take divergent values
depending on which part is compiled.

For example, a title page could be declared by:
%
\begin{center}
\begin{tabular}{l}
|\ifchilddoc\||else|\\
|\addtocounter{page}{-1}|\\
\textit{code for title page}\\
|\newpage|\\
|\||fi|
\end{tabular}
\end{center}
%
A banner page for the child documents can be generated by:
%
\begin{center}
\begin{tabular}{l}
|\ifchilddoc|\\
|\addtocounter{page}{-1}|\\
\textit{code for banner page}\\
|\newpage|\\
|\||fi|
\end{tabular}
\end{center}
%
Here one could write a message such as:
\begin{center}
|This is the part \childdocname{} of \childdocjob{}.|
\end{center}

%%%%%%%%%%%%%%%%%%%%%%%%%%%%%%%%%%%%%%%%%%%%%%%%%%%%%%%%%%%%%%%%%%%%%%%%%%%%%%%%
\subsection{Flags}
\label{sec:flags}

The package makes it easy to generate different versions
of the main or child documents.
To this end compilation flags can be defined
and assigned different default values.
They will be particularly useful in conjunction
with the forwarding mechanism described in \secref{sec:forward}.

For example, it may be useful to have a flag |\version|
which can be set to |draft| or |final|.
The document source will contain some conditional code
depending on the value of |\version|.
Suppose further, the flag should default to |final| for the main file
and to |draft| for child files
which is a natural assignment for editing the document.
This is achieved by placing the following code
in the preamble of the main document
(below the |\childdocmain| directive):
%
\begin{center}
\begin{tabular}{l}
|\ifchilddoc|\\
|\providecommand{\version}{draft}|\\
|\||else|\\
|\providecommand{\version}{final}|\\
|\||fi|
\end{tabular}
\end{center}
%
The definition by |\providecommand| makes sure
that previous definitions are not overwritten.
Further statements |\providecommand{\version}{...}|
can thus be added before the above code to override it.

For the main file, one might add a line
(between |\childdocmain| and the above block)
%
\begin{center}
|%\ifchilddoc\||else\providecommand{\version}{draft}\||fi|
\end{center}
%
which can be uncommented to produce a draft version.
Likewise one can add a line to the very top of a child file
(above the |\childdocof{|\textit{main}|}| directive)
%
\begin{center}
|%\providecommand{\version}{final}|
\end{center}
%
which can be uncommented to produce the final version of this child document.

%%%%%%%%%%%%%%%%%%%%%%%%%%%%%%%%%%%%%%%%%%%%%%%%%%%%%%%%%%%%%%%%%%%%%%%%%%%%%%%%
\subsection{Forwarding}
\label{sec:forward}

Different versions of the main or child documents
using compilation flags as described in \secref{sec:flags}
can be (permanently) stored in different files
for convenient compilation, viewing and distribution.
To this end, the package defines a command
to pass on compilation to a different file:

%%%%%%%%%%%%%%%%%%%%%%%%%%%%%%%%%%%%%%%%
\DescribeMacro{\childdocforward}
The command |\childdocforward| redirects processing to
another source file:
%
\begin{center}
\begin{tabular}{l}
|\input{childdoc.def}|\\
|\childdocforward[|\textit{main}|]{|\textit{dest}|}|\\
\end{tabular}
\end{center}
%
The argument \textit{dest} is the destination file
(without extension).
It should be the main file or one of the child files.
Note that further \textsf{childdoc} directives
such as |\childdocof| and |\childdocforward|
in the indicated file will be processed in this form.
The optional argument \textit{main}
passes on directly to the main file \textit{main}
while pretending to compile the child \textit{dest}.
This form behaves as if \textit{dest}
issues |\childdocof{|\textit{main}|}| right away,
and no further \textsf{childdoc} directives will be processed.

%%%%%%%%%%%%%%%%%%%%%%%%%%%%%%%%%%%%%%%%
\DescribeMacro{\...prefix}
In the alternative form |\childdocforwardprefix|,
%
\begin{center}
\begin{tabular}{l}
|\input{childdoc.def}|\\
|\childdocforwardprefix[|\textit{main}|]{|\textit{prefix}|}{|\textit{dest}|}|
\end{tabular}
\end{center}
%
the destination file is determined by a pattern
depending on the current file:
To make this work, the current file must be called
`{\textit{prefix}\hspace{0.2em}\textit{suffix}}'
with \textit{prefix} matching precisely the argument.
Processing is then passed on to the file
`{\textit{dest}\hspace{0.2em}\textit{suffix}}'.
Surely, the same effect is achieved by
directly specifying the
argument `{\textit{dest}\hspace{0.2em}\textit{suffix}}'
in the first form.
However, that requires to set up a different file
for each child. With the alternative form of the command
all these files can have exactly the same content
which simplifies setting them up and maintaining them.

For example, the following file |draft.tex|
with a compilation flag |\version| as described in \secref{sec:flags}
compiles the main document as a draft:
%
\begin{center}
\begin{tabular}{l}
|\def\version{draft}|\\
|\input{childdoc.def}|\\
|\childdocforward{|\textit{main}|}|
\end{tabular}
\end{center}
%
Likewise, the following files |final|\textit{nn}|.tex|
compile the final version of the child document
|child|\textit{nn}|.tex|:
%
\begin{center}
\begin{tabular}{l}
|\def\version{final}|\\
|\input{childdoc.def}|\\
|\childdocforwardprefix{final}{child}|
\end{tabular}
\end{center}
%

Note that when several versions of a main file and/or of each child file
are to be generated, it may be convenient to set up a |Makefile| or
shell script to automatise the process.

%%%%%%%%%%%%%%%%%%%%%%%%%%%%%%%%%%%%%%%%%%%%%%%%%%%%%%%%%%%%%%%%%%%%%%%%%%%%%%%%
\subsection{Command Line Processing}
\label{sec:commandline}

The effect of redirection files can also be achieved by invoking
the \LaTeX{} compiler with a more elaborate command line.
Most conveniently this should be done as part
of a shell script or a |Makefile|.

When using \textsf{childdoc} in the main file, the following
command lines effectively perform a redirection
(note that depending on the shell being used,
backslashes may have to be doubled: `|\|' $\to$ `|\\|'):
%
\begin{center}
|... -jobname "|\textit{target}|" |\\|"|[\textit{flags}]%
|\input{childdoc.def}\childdocforward[|\textit{main}|]{|\textit{dest}|}"|
\end{center}
%
Here \textit{target} is the name of the output file,
\textit{main} is the name of the main file
and \textit{dest} is the name of the main or child file to be processed
(all filenames without extensions).
The optional argument \textit{main} can be omitted
if \textit{main} matches \textit{dest}.
Optionally, compilation \textit{flags} can be defined via |\def| commands.
This command line makes the \TeX{} engine believe
it is compiling the file \textit{target}
whose content is specified as the latter parameter.
The provided code then forwards the processing to
\textit{main} or \textit{dest} as described in \secref{sec:forward}.

%%%%%%%%%%%%%%%%%%%%%%%%%%%%%%%%%%%%%%%%%%%%%%%%%%%%%%%%%%%%%%%%%%%%%%%%%%%%%%%%
\subsection{Include by Input}
\label{sec:input}

Including child documents by |\include| has some restrictions by design.
Most notably, the content of a child document always occupies
its own set of pages; pages cannot be shared between child documents.
Usually, this behaviour makes perfect sense
because each child document contain an essential part of the document.
However, in some situations it may be desirable to compose
a document from a collection of parts
without having mandatory page breaks between then.
For this case, the package
provides a mechanism to include parts
by |\input| which can also be processed individually.
However, by construction this mechanism
requires manual handling of the content to be output.

%%%%%%%%%%%%%%%%%%%%%%%%%%%%%%%%%%%%%%%%
\DescribeMacro{\ifchilddocmanual}
The main file should be prepared as usual, see \secref{sec:include}.
However, the document body must make a distinction
between processing of an individual part and of the main document, e.g.:
%
\begin{center}
\begin{tabular}{l}
|\ifchilddocmanual|\\
|\input{\childdocname}|\\
|\||else|\\
\textit{document body with }|\input{|\textit{part}|}|\\
|\||fi|
\end{tabular}
\end{center}
%
The conditional |\ifchilddocmanual| is true whenever
a part to be included by |\input| is being compiled,
and the name of the part is stored in |\childdocname|.

%%%%%%%%%%%%%%%%%%%%%%%%%%%%%%%%%%%%%%%%
\DescribeMacro{\childdocby}
Each part to be included by |\input| should start with:
%
\begin{center}
\begin{tabular}{l}
|\input{childdoc.def}|\\
|\childdocby{|\textit{main}|}|\\
\end{tabular}
\end{center}
%
The directive |\childdocby| is similar to |\childdocof|
described in \secref{sec:include},
but the subsequent selection of content must be done manually.
To that end, both |\ifchilddoc| and |\ifchilddocmanual|
will be true upon processing of a part,
and the name of the part is stored in |\childdocname|.
Note that |\jobname| will be set to the filename of the current part
so that each part receives an individual |.aux| file
that does not interfere with the |.aux| file(s) of the main document.
This behaviour can be altered by the alternative form
|\childdocby[*]{|\textit{main}|}| (with a non-empty optional argument)
which uses the |.aux| file of the main document
by setting |\jobname| to \textit{main}.

%%%%%%%%%%%%%%%%%%%%%%%%%%%%%%%%%%%%%%%%%%%%%%%%%%%%%%%%%%%%%%%%%%%%%%%%%%%%%%%%
\subsection{Driver Development}
\label{sec:driver}

The \textsf{childdoc} mechanism can also be use for the development
of definition files such as \LaTeX{} styles or classes.
This case differs from the above setup with multiple parts
included by |\include| in that no |\includeonly| should be invoked.
This can be achieved by starting the include file
(before |\ProvidesPackage|) with:
%
\begin{center}
\begin{tabular}{l}
|\input{childdoc.def}|\\
|\childdocforward{|\textit{main}|}|\\
\end{tabular}
\end{center}
%
or alternatively with:
%
\begin{center}
\begin{tabular}{l}
|\input{childdoc.def}|\\
|\childdocby{|\textit{main}|}|\\
\end{tabular}
\end{center}
%
Both forms have slightly different effects as described above.
The main file is prepared as usual, see \secref{sec:include}.

%%%%%%%%%%%%%%%%%%%%%%%%%%%%%%%%%%%%%%%%%%%%%%%%%%%%%%%%%%%%%%%%%%%%%%%%%%%%%%%%
\subsection{Legacy Detection}
\label{sec:detection}

The directive |\childdocmain| in the main file can detect
whether the complete document or merely a child is to be compiled
even without using the directive |\childdocof|.
This method is deprecated because it is less robust
and there is no compelling reason to use it;
it is merely provided for backward compatibility
and it may be removed in future versions.

If the detection mechanism is to be used,
it is mandatory to correctly specify
the filename of the main file as the argument of |\childdocmain|:
%
\begin{center}
\begin{tabular}{l}
|\input{childdoc.def}|\\
|\childdocmain{|\textit{main}|}|\\
\end{tabular}
\end{center}
%
If |\jobname| does not match the argument \textit{main} of |\childdocmain|,
it is assumed that |\jobname| points to the child file to be compiled.
When using |\childdocmain| with the main file specified as argument,
it suffices to start a child file
with just |\input{|\textit{main}|}|
without loading of the package and using |\childdocof|.
If instead all processing is done
with the appropriate \textsf{childdoc} directives,
the argument of \textit{main} of |\childdocmain| can be empty.

An alternative version of the command line processing described
in \secref{sec:commandline} using the detection mechanism reads:
%
\begin{center}
|... -jobname "|\textit{target}|" "|[\textit{flags}]%
[|\def\jobname{|\textit{dest}|}|]|\input{|\textit{main}|}"|
\end{center}

%%%%%%%%%%%%%%%%%%%%%%%%%%%%%%%%%%%%%%%%%%%%%%%%%%%%%%%%%%%%%%%%%%%%%%%%%%%%%%%%
\subsection{Manual Code}
\label{sec:manual}

In case one cannot be certain whether the definitions file |childdoc.def|
is installed on the target \TeX{} distribution
and one prefers not to ship it,
it is conceivable to paste a few relevant commands into the sources.

To that end, drop all statements |\input{childdoc.def}|
and perform the replacements as outlined below.
Instead of |\childdocmain{|\textit{main}|}| add the following code
to the top of the main file:
%
\begin{center}
\begin{tabular}{l}
|\||ifdefined\childdocname\endinput\||fi\newif\ifchilddoc|\\
|\edef\childdocname{\scantokens\expandafter{\jobname\noexpand}}|\\
|\def\childdocmain{|\textit{main}|}\||ifx\childdocmain\childdocname\||else|\\
|\childdoctrue\includeonly{\childdocname}\let\jobname\childdocmain\||fi|\\
\end{tabular}
\end{center}
%
Instead of |\childdocof{|\textit{main}|}| just include the main file
at the top of each child file:
%
\begin{center}
|\input{|\textit{main}|}|
\end{center}
%
A simple redirection |\childdocforward{|\textit{dest}|}| is achieved by:
%
\begin{center}
|\def\jobname{|\textit{dest}|}\input{\jobname}|
\end{center}
%
The redirection with prefix
|\childdocforwardprefix[|\textit{prefix}|]{|\textit{dest}|}|
is accomplished by:
%
\begin{center}
\begin{tabular}{l}
|{\edef\jobname{\scantokens\expandafter{\jobname\noexpand}}|\\
|\def\redirectjob |\textit{prefix}|#1~~~{\gdef\jobname{|\textit{dest}|#1}}|\\
|\expandafter\redirectjob\jobname~~~}\input{\jobname}|
\end{tabular}
\end{center}

In an alternative approach,
child documents can be compiled by a specific command line
without additional code or specific definitions:
%
\begin{center}
|... -jobname "|\textit{target}|" "|[\textit{flags}]%
|\includeonly{|\textit{dest}|}\input{|\textit{main}|}"|
\end{center}
%

%%%%%%%%%%%%%%%%%%%%%%%%%%%%%%%%%%%%%%%%%%%%%%%%%%%%%%%%%%%%%%%%%%%%%%%%%%%%%%%%
%%%%%%%%%%%%%%%%%%%%%%%%%%%%%%%%%%%%%%%%%%%%%%%%%%%%%%%%%%%%%%%%%%%%%%%%%%%%%%%%
\section{Information}

%%%%%%%%%%%%%%%%%%%%%%%%%%%%%%%%%%%%%%%%%%%%%%%%%%%%%%%%%%%%%%%%%%%%%%%%%%%%%%%%
\subsection{Copyright}

Copyright \copyright{} 2017--2018 Niklas Beisert

This work may be distributed and/or modified under the
conditions of the \LaTeX{} Project Public License, either version 1.3
of this license or (at your option) any later version.
The latest version of this license is in
  \url{http://www.latex-project.org/lppl.txt}
and version 1.3 or later is part of all distributions of \LaTeX{}
version 2005/12/01 or later.

This work has the LPPL maintenance status `maintained'.

The Current Maintainer of this work is Niklas Beisert.

This work consists of the files |README.txt|, |childdoc.ins| and |childdoc.dtx|
as well as the derived files |childdoc.def|, |cdocsamp.tex|
with |cdocsch1.tex|, |cdocsch2.tex|, |cdocspt3.tex|, |cdocspt4.tex|,
|cdocsdrf.tex|, |cdocsfn1.tex|, |cdocsfn2.tex|
as well as |childdoc.pdf|.

%%%%%%%%%%%%%%%%%%%%%%%%%%%%%%%%%%%%%%%%%%%%%%%%%%%%%%%%%%%%%%%%%%%%%%%%%%%%%%%%
\subsection{Files and Installation}

The package consists of the files:
%
\begin{center}
\begin{tabular}{ll}
    |README.txt|   & readme file \\
    |childdoc.ins| & installation file \\
    |childdoc.dtx| & source file \\
    |childdoc.def| & definition file \\
    |cdocsamp.tex| & sample main file \\
    |cdocsch1.tex| & sample include file \\
    |cdocsch2.tex| & sample include file \\
    |cdocspt3.tex| & sample part file \\
    |cdocspt4.tex| & sample part file \\
    |cdocsdrf.tex| & sample redirection file \\
    |cdocsfn1.tex| & sample redirection file \\
    |cdocsfn2.tex| & sample redirection file \\
    |childdoc.pdf| & manual
\end{tabular}
\end{center}
%
The distribution consists of the files
|README.txt|, |childdoc.ins| and |childdoc.dtx|.
%
\begin{itemize}
\item
Run (pdf)\LaTeX{} on |childdoc.dtx|
to compile the manual |childdoc.pdf| (this file).
\item
Run \LaTeX{} on |childdoc.ins| to create the definitions file |childdoc.def|
and the sample |cdocsamp.tex| with include files
|cdocsch1.tex|, |cdocsch2.tex|, |cdocspt3.tex|, |cdocspt4.tex|,
|cdocsdrf.tex|, |cdocsfn1.tex|, |cdocsfn2.tex|.
Then copy the file |childdoc.def| to an appropriate directory of your \LaTeX{}
distribution, e.g.\ \textit{texmf-root}|/tex/latex/childdoc|.
\end{itemize}

%%%%%%%%%%%%%%%%%%%%%%%%%%%%%%%%%%%%%%%%%%%%%%%%%%%%%%%%%%%%%%%%%%%%%%%%%%%%%%%%
\subsection{Related CTAN Packages}

There are several other packages which offer a similar functionality:
%
\begin{itemize}
\item
The packages
\href{http://ctan.org/pkg/docmute}{\textsf{docmute}},
\href{http://ctan.org/pkg/includex}{\textsf{includex}} and
\href{http://ctan.org/pkg/standalone}{\textsf{standalone}}
provide commands to include only the document body of
a child file thus allowing both files to be compiled individually.
\item
The packages \href{http://ctan.org/pkg/subdocs}{\textsf{subdocs}}
and \href{http://ctan.org/pkg/subfiles}{\textsf{subfiles}}
provide structures in which the main and child documents can be
encapsulated and allowing them to be compiled individually.
The inclusion mechanism is different from the conventional |\include|.
\item
The package \href{http://ctan.org/pkg/combine}{\textsf{combine}}
is an elaborate solution to combine several documents into one.
\end{itemize}
%
See also the CTAN topic \href{http://ctan.org/topic/subdocs}{\textsf{subdocs}}
for further related packages.
The present package differs from the above solutions in that
a document structure constructed with the conventional |\include| mechanism
just needs two extra commands at the top of every file
such that all constituent files can be compiled individually.

%%%%%%%%%%%%%%%%%%%%%%%%%%%%%%%%%%%%%%%%%%%%%%%%%%%%%%%%%%%%%%%%%%%%%%%%%%%%%%%%
%\subsection{Feature Suggestions}
%
%The following is a list of features which may be useful for future
%versions of this package:
%%
%\begin{itemize}
%\item
%\ldots
%\end{itemize}

%%%%%%%%%%%%%%%%%%%%%%%%%%%%%%%%%%%%%%%%%%%%%%%%%%%%%%%%%%%%%%%%%%%%%%%%%%%%%%%%
\subsection{Revision History}

%%%%%%%%%%%%%%%%%%%%%%%%%%%%%%%%%%%%%%%%
\paragraph{v2.0:} 2018/12/30

\begin{itemize}
\item
immediate forward processing
\item
added |\childdocby| mechanism
\item
manual restructured
\end{itemize}

%%%%%%%%%%%%%%%%%%%%%%%%%%%%%%%%%%%%%%%%
\paragraph{v1.6:} 2018/01/17

\begin{itemize}
\item
application for development of include files
\item
corrections to manual
\end{itemize}

%%%%%%%%%%%%%%%%%%%%%%%%%%%%%%%%%%%%%%%%
\paragraph{v1.5:} 2017/05/21

\begin{itemize}
\item
more complete structuring introduced
\item
|\childdocof| introduced
\item
|\childdoc| renamed to |\childdocmain|
\item
|\childredirect| renamed to |\childdocforward| and |\childdocforwardprefix|
and functionality expanded
\end{itemize}

%%%%%%%%%%%%%%%%%%%%%%%%%%%%%%%%%%%%%%%%
\paragraph{v1.0:} 2017/04/27

\begin{itemize}
\item
manual and install package
\item
first version published on CTAN
\end{itemize}

%%%%%%%%%%%%%%%%%%%%%%%%%%%%%%%%%%%%%%%%
\paragraph{v0.6:} 2017/04/26

\begin{itemize}
\item
redirection mechanism added
\end{itemize}

%%%%%%%%%%%%%%%%%%%%%%%%%%%%%%%%%%%%%%%%
\paragraph{v0.5:} 2017/04/26

\begin{itemize}
\item
functionality in definition file
\end{itemize}


%%%%%%%%%%%%%%%%%%%%%%%%%%%%%%%%%%%%%%%%%%%%%%%%%%%%%%%%%%%%%%%%%%%%%%%%%%%%%%%%
%%%%%%%%%%%%%%%%%%%%%%%%%%%%%%%%%%%%%%%%%%%%%%%%%%%%%%%%%%%%%%%%%%%%%%%%%%%%%%%%
%%%%%%%%%%%%%%%%%%%%%%%%%%%%%%%%%%%%%%%%%%%%%%%%%%%%%%%%%%%%%%%%%%%%%%%%%%%%%%%%
\appendix

\settowidth\MacroIndent{\rmfamily\scriptsize 000\ }

 \DocInput{childdoc.dtx}

\end{document}
%</driver>
% \fi
%
% %%%%%%%%%%%%%%%%%%%%%%%%%%%%%%%%%%%%%%%%%%%%%%%%%%%%%%%%%%%%%%%%%%%%%%%%%%%%%%
% %%%%%%%%%%%%%%%%%%%%%%%%%%%%%%%%%%%%%%%%%%%%%%%%%%%%%%%%%%%%%%%%%%%%%%%%%%%%%%
% \section{Sample}
%\iffalse
%<*samplemain>
%\fi
%
% The following presents a sample document
% with two chapters, two parts, a title page,
% a compile flag as well as three forwarding files to set the flag.
% It consists of eight |.tex| files:
% \begin{center}
% \begin{tabular}{ll}
% |cdocsamp.tex|&main file\\
% |cdocsch1.tex|&include file for chapter 1\\
% |cdocsch2.tex|&include file for chapter 2\\
% |cdocspt3.tex|&include file for part 3\\
% |cdocspt4.tex|&include file for part 4\\
% |cdocsdrf.tex|&forwarding file for main file in draft mode\\
% |cdocsfi1.tex|&forwarding file for final version of chapter 1\\
% |cdocsfi2.tex|&forwarding file for final version of chapter 2\\
% \end{tabular}
% \end{center}
% Each of the eight files can be compiled directly by the \LaTeX{} compiler.
%
% %%%%%%%%%%%%%%%%%%%%%%%%%%%%%%%%%%%%%%
% \paragraph{Main File.}
%
% The main file is called |cdocsamp.tex|.
%
% Load the \textsf{childdoc} definitions and
% declare the filename for the main document:
%    \begin{macrocode}
\input{childdoc.def}
\childdocmain{}
%    \end{macrocode}

% Optional override for |\version| flag:
%    \begin{macrocode}
%%\ifchilddoc\else\providecommand{\version}{draft}\fi
%    \end{macrocode}

% Define the default values for the |\version| flag
% (|final| for the main file and |draft| for childs):
%    \begin{macrocode}
\ifchilddoc
\providecommand{\version}{draft}
\else
\providecommand{\version}{final}
\fi
%    \end{macrocode}

% Load the standard document class:
%    \begin{macrocode}
\documentclass[12pt]{article}
%    \end{macrocode}

% Start the document body:
%    \begin{macrocode}
\begin{document}
%    \end{macrocode}

% Declare a title page.
% Print title, part of document being processed and version flag:
%    \begin{macrocode}
\addtocounter{page}{-1}
\begin{center}
{\LARGE\bfseries{}childdoc example\par}
\vspace{1cm}
\ifchilddoc
\ifchilddocmanual part\else chapter\fi:
`\childdocname' of `\childdocjob'\par
\else
main document: `\childdocjob'\par
\fi
version: \version\par
\end{center}
\newpage
%    \end{macrocode}

% Manually include selected file,
% otherwise process as usual:
%    \begin{macrocode}
\ifchilddocmanual
\section*{part `\childdocname'}
\input{\childdocname}
\else
%    \end{macrocode}

% Include the two chapters:
%    \begin{macrocode}
\include{cdocsch1}
\include{cdocsch2}
%    \end{macrocode}

% Include the two parts unless only chapters should be displayed:
%    \begin{macrocode}
\ifchilddoc\else
\section{part three}
\input{cdocspt3}
\section{part four}
\input{cdocspt4}
\fi
%    \end{macrocode}

% Process as usual until here:
%    \begin{macrocode}
\fi
%    \end{macrocode}

% End of document body:
%    \begin{macrocode}
\end{document}
%    \end{macrocode}
%\iffalse
%</samplemain>
%\fi
%
% %%%%%%%%%%%%%%%%%%%%%%%%%%%%%%%%%%%%%%
% \paragraph{Chapter Include Files.}
%
% The include files are called |cdocsch1.tex| and |cdocsch2.tex|.
%
%\iffalse
%<*samplechap1|samplechap2>
%\fi

% Optional override for |\version| flag:
%    \begin{macrocode}
%%\providecommand{\version}{final}
%    \end{macrocode}

% Include the main document:
%    \begin{macrocode}
\input{childdoc.def}
\childdocof{cdocsamp}
%    \end{macrocode}

%\iffalse
%</samplechap1|samplechap2>
%\fi
%
%\iffalse
%<*samplechap1>
%\fi
% Some text for chapter 1:
%    \begin{macrocode}
\section{one}
some text in chapter one
%    \end{macrocode}

%\iffalse
%</samplechap1>
%\fi
% Some text for chapter 2:
%\iffalse
%<*samplechap2>
%\fi
%    \begin{macrocode}
\section{two}
more text in chapter two
%    \end{macrocode}

%\iffalse
%</samplechap2>
%\fi
%
% %%%%%%%%%%%%%%%%%%%%%%%%%%%%%%%%%%%%%%
% \paragraph{Part Include Files.}
%
% The include files are called |cdocspt3.tex| and |cdocspt4.tex|.
%
%\iffalse
%<*samplepart3|samplepart4>
%\fi

% Optional override for |\version| flag:
%    \begin{macrocode}
%%\providecommand{\version}{final}
%    \end{macrocode}

% Include the main document:
%    \begin{macrocode}
\input{childdoc.def}
\childdocby{cdocsamp}
%    \end{macrocode}

%\iffalse
%</samplepart3|samplepart4>
%\fi
%
%\iffalse
%<*samplepart3>
%\fi
% Some text for part 3:
%    \begin{macrocode}
some text in part three
%    \end{macrocode}

%\iffalse
%</samplepart3>
%\fi
% Some text for part 4:
%\iffalse
%<*samplepart4>
%\fi
%    \begin{macrocode}
more text in part four
%    \end{macrocode}

%\iffalse
%</samplepart4>
%\fi
%
% %%%%%%%%%%%%%%%%%%%%%%%%%%%%%%%%%%%%%%
% \paragraph{Forwarding for a Complete Draft.}
%
% The following forwarding file |cdocsdrf.tex|
% compiles the main document in draft mode:
%\iffalse
%<*sampledraft>
%\fi
%    \begin{macrocode}
\def\version{draft}
\input{childdoc.def}
\childdocforward{cdocsamp}
%    \end{macrocode}

%\iffalse
%</sampledraft>
%\fi
%
% %%%%%%%%%%%%%%%%%%%%%%%%%%%%%%%%%%%%%%
% \paragraph{Forwarding for Final Version of the Chapters.}
%
% The following forwarding files |cdocsfn1.tex| and |cdocsfn2.tex|
% (with identical content)
% compile the final versions of the child documents
% |cdocsch1.tex| and |cdocsch2.tex|, respectively:
%\iffalse
%<*samplefinal>
%\fi
%    \begin{macrocode}
\def\version{final}
\input{childdoc.def}
\childdocforwardprefix[cdocsamp]{cdocsfn}{cdocsch}
%    \end{macrocode}

%\iffalse
%</samplefinal>
%\fi
%
% %%%%%%%%%%%%%%%%%%%%%%%%%%%%%%%%%%%%%%
% \paragraph{Command Line Processing.}
%
% The following three command lines generate the output files
% |cdocscld|, |cdocscl1| and |cdocscl2|
% which should be identical to
% |cdocsdrf|, |cdocsch1| and |cdocsfn2|, respectively:
% \begin{center}
% \begin{tabular}{l}
% |latex -jobname cdocscld \|\\
% |  "\def\version{draft}\input{childdoc.def}\childdocforward{cdocsamp}"|\\
% |latex -jobname cdocscl1 \|\\
% |  "\input{childdoc.def}\childdocforward[cdocsamp]{cdocsch1}"|\\
% |latex -jobname cdocscl2 \|\\
% |  "\def\version{final}\input{childdoc.def}\childdocforward{cdocsch2}"|
% \end{tabular}
% \end{center}
% Note that the trailing backslash on each first line
% merely continues the input to the second line
% (for convenient cut ant paste).
% Furthermore, the command |latex| can be replaced by any
% of its alternative versions such as |pdflatex|.
%
% %%%%%%%%%%%%%%%%%%%%%%%%%%%%%%%%%%%%%%%%%%%%%%%%%%%%%%%%%%%%%%%%%%%%%%%%%%%%%%
% %%%%%%%%%%%%%%%%%%%%%%%%%%%%%%%%%%%%%%%%%%%%%%%%%%%%%%%%%%%%%%%%%%%%%%%%%%%%%%
% \section{Implementation}
%\iffalse
%<*package>
%\fi
%
% This section describes the definitions file |childdoc.def|.

% The definitions cannot be loaded using |\usepackage| or |\RequirePackage|
% which has a mechanism to prevent loading a style file more than once.
% When loading the definitions by means of |\input|
% multiple instances have to be prevented manually:
%\iffalse
%This code needs to be before the `\ProvidesFile' directive
%which is defined at the beginning of this file.
%Therefore it is also placed there and commented out here.
%</package>
%<*discard>
%\fi
%    \begin{macrocode}
\ifdefined\childdocmain\endinput\fi
%    \end{macrocode}
%\iffalse
%</discard>
%<*package>
%\fi
%
% \macro{\ifchilddoc}
% \macro{\ifchilddocmanual}
% The conditional |\ifchilddoc| tells whether a
% child (true) or main (false) document is being compiled.
% The conditional |\ifchilddocmanual| tells whether
% the |\includeonly| mechanism is used (false) or
% the selection of child files must be performed manually (true).
% The definitions initialise to false:
%    \begin{macrocode}
\newif\ifchilddoc
\newif\ifchilddocmanual
%    \end{macrocode}

% \macro{\childdocname}
% \macro{\childdocjob}
% The macro |\childdocname| stores the name of the main document
% to be compiled. The macro |\childdocjob| stores the name of
% the document on which the \LaTeX{} compiler was originally invoked.
% The content of |\jobname| cannot be compared
% to filenames specified in the source due to different catcodes.
% The following code rescans |\jobname|, stores the result
% in |\childdocname| and saves a copy in |\childdocjob|:
%    \begin{macrocode}
\edef\childdocname{\scantokens\expandafter{\jobname\noexpand}}
\let\childdocjob\childdocname
%    \end{macrocode}

% \macro{\childdocdisable}
% The macro |\childdocdisable| prevents the main file
% from being processed more than once.
% At this stage, the main document command |\childdocmain|
% is assumed to be called once again where it should do nothing.
% Any subsequent call to it should prevent
% a secondary processing of the main document
% It overwrites the forwarding commands
% |\childdocof| and |\childdocforward|
% with empty macros to prevent further inclusions of the main document:
%    \begin{macrocode}
\newcommand{\childdocdisable}
{
  \renewcommand{\childdocmain}[1]{\renewcommand{\childdocmain}[1]{\endinput}}
  \renewcommand{\childdocof}[1]{}
  \renewcommand{\childdocby}[2][]{}
  \renewcommand{\childdocforward}[2][]{}
  \renewcommand{\childdocdisable}{}
}
%    \end{macrocode}

% \macro{\childdocmain}
% The macro |\childdocmain| is to be called at the top of the main file
% with nothing or the main filename (without extension) as argument.
% First, it breaks loops.
% If the argument is not empty and does not match |\childdocname|
% (which is set by the first inclusion of |childdoc.def|),
% |\ifchilddoc| is set to true, |\includeonly| is applied to the child file
% and |\jobname| is set to the main file
% (for proper handling of |.aux| files):
%    \begin{macrocode}
\newcommand{\childdocmain}[1]
{
  \childdocdisable\childdocmain{}
  \if?#1?\else
    \begingroup
      \def\childdoctmp{#1}
      \ifx\childdoctmp\childdocname
        \def\childdoctmp{}
      \else
        \def\childdoctmp
        {
          \childdoctrue
          \includeonly{\childdocname}
          \def\childdocjob{#1}
          \def\jobname{#1}
        }
      \fi
      \expandafter
    \endgroup
    \childdoctmp
  \fi
}
%    \end{macrocode}

% \macro{\childdocof}
% The command |\childdocof| redirects
% compilation to the main file |#1|.
%    \begin{macrocode}
\newcommand{\childdocof}[1]
{
  \childdocdisable
  \childdoctrue
  \includeonly{\childdocname}
  \def\jobname{#1}
  \def\childdocjob{#1}
  \input{#1}
}
%    \end{macrocode}

% \macro{\childdocby}
% The command |\childdocby| ....
%    \begin{macrocode}
\newcommand{\childdocby}[2][]
{
  \childdocdisable
  \childdoctrue
  \childdocmanualtrue
  \if?#1?\else
    \def\jobname{#2}
  \fi
  \def\childdocjob{#2}
  \input{#2}
  \endinput
}
%    \end{macrocode}

% \macro{\childdocforward}
% The command |\childdocforward| redirects
% compilation to the main file or
% (if the optional argument is given) a child file.
% Parameters are set as if the main file
% or a child file starting with |\childdocof| was compiled.
% Then compilation is handed over to the main file:
%    \begin{macrocode}
\newcommand{\childdocforward}[2][]
{
  \begingroup
    \if?#1?
      \def\childdoctmp
      {
        \def\childdocname{#2}
        \def\childdocjob{#2}
        \def\jobname{#2}
        \input{#2}
        \endinput
      }
    \else
      \def\childdoctmp
      {
        \childdocdisable
        \def\childdocname{#2}
        \childdoctrue
        \includeonly{#2}
        \def\childdocjob{#1}
        \def\jobname{#1}
        \input{#1}
        \endinput
      }
    \fi
    \expandafter
  \endgroup
  \childdoctmp
}
%    \end{macrocode}

% \macro{\childdocforwardprefix}
% The command |\childdocforwardprefix| redirects
% compilation to the main or a child file by means of a pattern.
% The prefix |#1| in the current filename is replaced by |#2|
% and the suffix of the current filename is kept
% (it is assumed that the filename does not contain the substring `|~~~|'
% which is used as a delimiter).
% Compilation is handed over to the new file by |\childdocforward|:
%    \begin{macrocode}
\newcommand{\childdocforwardprefix}[3][]
{
  \begingroup
    \def\childdocextract #2##1~~~{\def\childdoctmp{\childdocforward[#1]{#3##1}}}
    \expandafter\childdocextract\childdocname~~~
    \expandafter
  \endgroup
  \childdoctmp
}
%    \end{macrocode}

% \macro{\childdoc}
% The deprecated macro |\childdoc| is a legacy version of |\childdocmain|:
%    \begin{macrocode}
\newcommand{\childdoc}{\childdocmain}
%    \end{macrocode}

% \macro{\childdocredirect}
% The deprecated macro |\childdocredirect| is a legacy version
% of |\childdocforward| and |\childdocforwardprefix|:
%    \begin{macrocode}
\newcommand{\childdocredirect}[2][]
{
  \begingroup
    \if?#1?
      \def\childdoctmp{\childdocforward{#2}}
    \else
      \def\childdoctmp{\childdocforwardprefix{#1}{#2}}
    \fi
    \expandafter
  \endgroup
  \childdoctmp
}
%    \end{macrocode}

%\iffalse
%</package>
%\fi
%
\endinput
|\\
|\childdocmain{}|\\
\end{tabular}
\end{center}
at the very top of the main \LaTeX{} file,
in particular \emph{before} the |\documentclass| statement!
The argument of |\childdocmain| should be left empty
(but it must be present).

%%%%%%%%%%%%%%%%%%%%%%%%%%%%%%%%%%%%%%%%
\DescribeMacro{\childdocof}
Furthermore, add the commands
\begin{center}
\begin{tabular}{l}
|% \iffalse
%
% childdoc.dtx Copyright (C) 2017-2018 Niklas Beisert
%
% This work may be distributed and/or modified under the
% conditions of the LaTeX Project Public License, either version 1.3
% of this license or (at your option) any later version.
% The latest version of this license is in
%   http://www.latex-project.org/lppl.txt
% and version 1.3 or later is part of all distributions of LaTeX
% version 2005/12/01 or later.
%
% This work has the LPPL maintenance status `maintained'.
%
% The Current Maintainer of this work is Niklas Beisert.
%
% This work consists of the files childdoc.dtx and childdoc.ins
% and the derived files childdoc.def and cdocsamp.tex with
% cdocsch1.tex, cdocsch2.tex, cdocsdrf.tex, cdocsfn1.tex, cdocsfn2.tex.
%
%<package>\ifdefined\childdocmain\endinput\fi
%<package>\ProvidesFile{childdoc.def}[2018/12/30 v2.0 child document driver]
%<samplemain>\ProvidesFile{cdocsamp.tex}[2018/12/30 v2.0 sample for childdoc]
%<*driver>
%\ProvidesFile{childdoc.drv}[2018/12/30 v2.0 childdoc reference manual file]
\PassOptionsToClass{10pt,a4paper}{article}
\documentclass{ltxdoc}

\usepackage[margin=35mm]{geometry}
\usepackage{hyperref}
\usepackage{hyperxmp}
\usepackage[usenames]{color}

\hypersetup{colorlinks=true}
\hypersetup{pdfstartview=FitH}
\hypersetup{pdfpagemode=UseNone}
\hypersetup{pdfsource={}}
\hypersetup{pdflang={en-UK}}
\hypersetup{pdfcopyright={Copyright 2017-2018 Niklas Beisert.
  This work may be distributed and/or modified under the
  conditions of the LaTeX Project Public License, either version 1.3
  of this license or (at your option) any later version.}}
\hypersetup{pdflicenseurl={http://www.latex-project.org/lppl.txt}}
\hypersetup{pdfcontactaddress={ETH Zurich, ITP, HIT K,
  Wolfgang-Pauli-Strasse 27}}
\hypersetup{pdfcontactpostcode={8093}}
\hypersetup{pdfcontactcity={Zurich}}
\hypersetup{pdfcontactcountry={Switzerland}}
\hypersetup{pdfcontactemail={nbeisert@itp.phys.ethz.ch}}
\hypersetup{pdfcontacturl={http://people.phys.ethz.ch/\xmptilde nbeisert/}}

\newcommand{\secref}[1]{\hyperref[#1]{section \ref*{#1}}}

\parskip1ex
\parindent0pt
\let\olditemize\itemize
\def\itemize{\olditemize\parskip0pt}

\begin{document}

\title{The \textsf{childdoc} Package}
\hypersetup{pdftitle={The childdoc Package}}
\author{Niklas Beisert\\[2ex]
  Institut f\"ur Theoretische Physik\\
  Eidgen\"ossische Technische Hochschule Z\"urich\\
  Wolfgang-Pauli-Strasse 27, 8093 Z\"urich, Switzerland\\[1ex]
  \href{mailto:nbeisert@itp.phys.ethz.ch}
  {\texttt{nbeisert@itp.phys.ethz.ch}}}
\hypersetup{pdfauthor={Niklas Beisert}}
\hypersetup{pdfsubject={Manual for the LaTeX2e Package childdoc}}
\date{30 December 2018, \textsf{v2.0}}
\maketitle

\begin{abstract}\noindent
\textsf{childdoc} is a \LaTeXe{} package
that enables the direct compilation
of document sections included by |\include|
to individual files.
\end{abstract}

\begingroup
\parskip0ex
\tableofcontents
\endgroup

%%%%%%%%%%%%%%%%%%%%%%%%%%%%%%%%%%%%%%%%%%%%%%%%%%%%%%%%%%%%%%%%%%%%%%%%%%%%%%%%
%%%%%%%%%%%%%%%%%%%%%%%%%%%%%%%%%%%%%%%%%%%%%%%%%%%%%%%%%%%%%%%%%%%%%%%%%%%%%%%%
\section{Introduction}

\LaTeX{} provides a mechanism to structure a large document (such as a book)
into a main file and several child files (containing the chapters)
using the |\include| command.
This mechanism is beneficial for documents
which span hundreds of pages in order to
make the source file(s) more manageable.
Moreover, compilation can be restricted to
selected child files by means of the |\includeonly| command.
The latter feature can be used to reduce the compilation time while editing
(this was significantly more useful in the earlier days of \LaTeX{})
or to generate a smaller document which is easier to navigate.
Another application of |\includeonly| is to generate
documents consisting of selected parts of the complete document.

However, there are a few drawbacks of the plain |\include| mechanism:
\begin{itemize}
\item
The child files cannot be compiled on their own,
they can only be compiled via the main file.
A naive editing environment
(such as a text editor with an option
to have the current file processed by \LaTeX)
may require one to switch to the main file before compiling;
attempting to compile the child file produces errors.
\item
The main file must be modified (each time)
to adjust the |\includeonly| command
to the present needs. This easily leaves the main file in a messy state.
\item
The generated document will always carry the filename
of the main document. This is inconvenient if
several child files are to be compiled and
to be kept for distribution.
\end{itemize}

The present package provides a simple interface
to make child files individually compilable by \LaTeX{}.
Compiling a child file then has the same effect as compiling
the main file with an |\includeonly| command
to select the appropriate child.
Moreover the generated document will carry the name of the child
rather than the main file.
This resolves all three above issues.

This feature is meant to make the editing of books,
thesis documents and lecture notes somewhat more convenient.
However, the package can also be used efficiently for
composing a series of documents (such as exercise sheets)
which are typically distributed individually.
It then assists the author in generating the individual documents
(potentially in different versions)
as well as a document containing the collected series.
Another application is in developing style files
or other kinds of included material
where compilation of the style file could redirect
to a sample or test file.

%%%%%%%%%%%%%%%%%%%%%%%%%%%%%%%%%%%%%%%%%%%%%%%%%%%%%%%%%%%%%%%%%%%%%%%%%%%%%%%%
%%%%%%%%%%%%%%%%%%%%%%%%%%%%%%%%%%%%%%%%%%%%%%%%%%%%%%%%%%%%%%%%%%%%%%%%%%%%%%%%
\section{Usage}

First of all, the package \textsf{childdoc} is \emph{not} a standard
\LaTeXe{} |.sty| style file! Therefore it needs to be invoked in
a non-standard way.

%%%%%%%%%%%%%%%%%%%%%%%%%%%%%%%%%%%%%%%%%%%%%%%%%%%%%%%%%%%%%%%%%%%%%%%%%%%%%%%%
\subsection{Included Files}
\label{sec:include}

%%%%%%%%%%%%%%%%%%%%%%%%%%%%%%%%%%%%%%%%
\DescribeMacro{\childdocmain}
To use the package, add the commands
\begin{center}
\begin{tabular}{l}
|\input{childdoc.def}|\\
|\childdocmain{}|\\
\end{tabular}
\end{center}
at the very top of the main \LaTeX{} file,
in particular \emph{before} the |\documentclass| statement!
The argument of |\childdocmain| should be left empty
(but it must be present).

%%%%%%%%%%%%%%%%%%%%%%%%%%%%%%%%%%%%%%%%
\DescribeMacro{\childdocof}
Furthermore, add the commands
\begin{center}
\begin{tabular}{l}
|\input{childdoc.def}|\\
|\childdocof{|\textit{main}|}|\\
\end{tabular}
\end{center}
at the top of every child file \textit{child}
which is included by |\include{|\textit{child}|}|
from within the main file
(or at least for those files to be compiled individually).
The argument \textit{main} must be the filename of the main file.

There are a couple of
considerations in setting up the main and child documents:

%%%%%%%%%%%%%%%%%%%%%%%%%%%%%%%%%%%%%%%%
\paragraph{Restrictions.}

Please note the following restrictions:
\begin{itemize}
\item
|\childdocmain| must be called with one argument \textit{main}
to ensure compatibility with earlier version of the package.
It must either be empty (|\childdocmain{}|)
or precisely match the filename of the main file in which it is specified.
See \secref{sec:detection} for further information.
\item
The filename \textit{main} must be specified without the |.tex| extension.
\item
The filename \textit{main} is case sensitive
(even in case-insensitive file systems)
due to internal string comparison.
\item
The argument \textit{main} should be fully expanded, it cannot be a macro.
\item
Subdirectories and special characters should be avoided in filenames.
\item
The command |\childdocmain{|\textit{main}|}| must be followed by a whitespace.
It should not be followed immediately by another command
or by a comment mark `|%|'.
This is because the \TeX{} parser reads the token immediately following
the argument of |\childdocmain| and puts it
at the beginning of every child section;
however, a white\-space is ignored.
\end{itemize}

%%%%%%%%%%%%%%%%%%%%%%%%%%%%%%%%%%%%%%%%
\paragraph{Content of Main File.}

It is advisable to place all content in the child files included by |\include|.
Any output contained in the main file will appear in all child documents
unless suppressed manually;
it cannot be suppressed automatically by the |\includeonly| directive
and thus should normally be avoided.
A method to include some content in the main file
by means of conditional processing is described in \secref{sec:conditional}.

%%%%%%%%%%%%%%%%%%%%%%%%%%%%%%%%%%%%%%%%
\paragraph{Page Numbering.}

When only a part of the document is compiled,
the appropriate numbering of pages
(as well as other status parameters)
is determined from the |.aux| files.
The latter contain information from previous passes.
However this information needs to propagate through
all intermediate child documents.
Therefore the page numbering in child documents may well
be inconsistent until the complete document is compiled at least once.

A useful (if unconventional) way to always ensure a consistent
page numbering is to restart the numbering in each child document
and denote the pages by `\textit{child}|.|\textit{page}'
where \textit{child} represents the chapter/section number of the child file.
This can be achieved by the command
|\numberwithin{page}{|\textit{child}|}|
of the \textsf{amsmath} package
where \textit{child} can be |chapter| or |section|
depending on the chosen structuring.
Alternatively, one can modify the macro |\thepage| appropriately
and reset the counter |page| at the start of each child file.

%%%%%%%%%%%%%%%%%%%%%%%%%%%%%%%%%%%%%%%%%%%%%%%%%%%%%%%%%%%%%%%%%%%%%%%%%%%%%%%%
\subsection{Conditional Processing}
\label{sec:conditional}

The package provides a mechanism to compile different versions
of a document. To customise the versions further some conditional processing
can come in handy to distinguish which version is being compiled.
The package provides two macros to describe the compilation context:

%%%%%%%%%%%%%%%%%%%%%%%%%%%%%%%%%%%%%%%%
\DescribeMacro{\ifchilddoc}
The conditional |\ifchilddoc| distinguishes between the compilation of
child documents and the main document:
%
\begin{center}
|\ifchilddoc |\textit{child-code}| |[|\||else |\textit{main-code}]| \||fi|
\end{center}

%%%%%%%%%%%%%%%%%%%%%%%%%%%%%%%%%%%%%%%%
\DescribeMacro{\childdocname}
\DescribeMacro{\childdocjob}
The macro |\childdocname| contains the filename (without extension)
of the main or child file being processed.
Note that |\childdocjob| will always contain the name of the main file.

%%%%%%%%%%%%%%%%%%%%%%%%%%%%%%%%%%%%%%%%
\paragraph{Title Page.}

Conditional processing can be used to include a title or banner page
in the main document when proper precautions are taken.
Importantly, the code in the main file should ensure that the page counter
(as well as other status parameters which are stored in the |.aux| files)
takes the same value after the conditional processing.
Otherwise the page numbers may take divergent values
depending on which part is compiled.

For example, a title page could be declared by:
%
\begin{center}
\begin{tabular}{l}
|\ifchilddoc\||else|\\
|\addtocounter{page}{-1}|\\
\textit{code for title page}\\
|\newpage|\\
|\||fi|
\end{tabular}
\end{center}
%
A banner page for the child documents can be generated by:
%
\begin{center}
\begin{tabular}{l}
|\ifchilddoc|\\
|\addtocounter{page}{-1}|\\
\textit{code for banner page}\\
|\newpage|\\
|\||fi|
\end{tabular}
\end{center}
%
Here one could write a message such as:
\begin{center}
|This is the part \childdocname{} of \childdocjob{}.|
\end{center}

%%%%%%%%%%%%%%%%%%%%%%%%%%%%%%%%%%%%%%%%%%%%%%%%%%%%%%%%%%%%%%%%%%%%%%%%%%%%%%%%
\subsection{Flags}
\label{sec:flags}

The package makes it easy to generate different versions
of the main or child documents.
To this end compilation flags can be defined
and assigned different default values.
They will be particularly useful in conjunction
with the forwarding mechanism described in \secref{sec:forward}.

For example, it may be useful to have a flag |\version|
which can be set to |draft| or |final|.
The document source will contain some conditional code
depending on the value of |\version|.
Suppose further, the flag should default to |final| for the main file
and to |draft| for child files
which is a natural assignment for editing the document.
This is achieved by placing the following code
in the preamble of the main document
(below the |\childdocmain| directive):
%
\begin{center}
\begin{tabular}{l}
|\ifchilddoc|\\
|\providecommand{\version}{draft}|\\
|\||else|\\
|\providecommand{\version}{final}|\\
|\||fi|
\end{tabular}
\end{center}
%
The definition by |\providecommand| makes sure
that previous definitions are not overwritten.
Further statements |\providecommand{\version}{...}|
can thus be added before the above code to override it.

For the main file, one might add a line
(between |\childdocmain| and the above block)
%
\begin{center}
|%\ifchilddoc\||else\providecommand{\version}{draft}\||fi|
\end{center}
%
which can be uncommented to produce a draft version.
Likewise one can add a line to the very top of a child file
(above the |\childdocof{|\textit{main}|}| directive)
%
\begin{center}
|%\providecommand{\version}{final}|
\end{center}
%
which can be uncommented to produce the final version of this child document.

%%%%%%%%%%%%%%%%%%%%%%%%%%%%%%%%%%%%%%%%%%%%%%%%%%%%%%%%%%%%%%%%%%%%%%%%%%%%%%%%
\subsection{Forwarding}
\label{sec:forward}

Different versions of the main or child documents
using compilation flags as described in \secref{sec:flags}
can be (permanently) stored in different files
for convenient compilation, viewing and distribution.
To this end, the package defines a command
to pass on compilation to a different file:

%%%%%%%%%%%%%%%%%%%%%%%%%%%%%%%%%%%%%%%%
\DescribeMacro{\childdocforward}
The command |\childdocforward| redirects processing to
another source file:
%
\begin{center}
\begin{tabular}{l}
|\input{childdoc.def}|\\
|\childdocforward[|\textit{main}|]{|\textit{dest}|}|\\
\end{tabular}
\end{center}
%
The argument \textit{dest} is the destination file
(without extension).
It should be the main file or one of the child files.
Note that further \textsf{childdoc} directives
such as |\childdocof| and |\childdocforward|
in the indicated file will be processed in this form.
The optional argument \textit{main}
passes on directly to the main file \textit{main}
while pretending to compile the child \textit{dest}.
This form behaves as if \textit{dest}
issues |\childdocof{|\textit{main}|}| right away,
and no further \textsf{childdoc} directives will be processed.

%%%%%%%%%%%%%%%%%%%%%%%%%%%%%%%%%%%%%%%%
\DescribeMacro{\...prefix}
In the alternative form |\childdocforwardprefix|,
%
\begin{center}
\begin{tabular}{l}
|\input{childdoc.def}|\\
|\childdocforwardprefix[|\textit{main}|]{|\textit{prefix}|}{|\textit{dest}|}|
\end{tabular}
\end{center}
%
the destination file is determined by a pattern
depending on the current file:
To make this work, the current file must be called
`{\textit{prefix}\hspace{0.2em}\textit{suffix}}'
with \textit{prefix} matching precisely the argument.
Processing is then passed on to the file
`{\textit{dest}\hspace{0.2em}\textit{suffix}}'.
Surely, the same effect is achieved by
directly specifying the
argument `{\textit{dest}\hspace{0.2em}\textit{suffix}}'
in the first form.
However, that requires to set up a different file
for each child. With the alternative form of the command
all these files can have exactly the same content
which simplifies setting them up and maintaining them.

For example, the following file |draft.tex|
with a compilation flag |\version| as described in \secref{sec:flags}
compiles the main document as a draft:
%
\begin{center}
\begin{tabular}{l}
|\def\version{draft}|\\
|\input{childdoc.def}|\\
|\childdocforward{|\textit{main}|}|
\end{tabular}
\end{center}
%
Likewise, the following files |final|\textit{nn}|.tex|
compile the final version of the child document
|child|\textit{nn}|.tex|:
%
\begin{center}
\begin{tabular}{l}
|\def\version{final}|\\
|\input{childdoc.def}|\\
|\childdocforwardprefix{final}{child}|
\end{tabular}
\end{center}
%

Note that when several versions of a main file and/or of each child file
are to be generated, it may be convenient to set up a |Makefile| or
shell script to automatise the process.

%%%%%%%%%%%%%%%%%%%%%%%%%%%%%%%%%%%%%%%%%%%%%%%%%%%%%%%%%%%%%%%%%%%%%%%%%%%%%%%%
\subsection{Command Line Processing}
\label{sec:commandline}

The effect of redirection files can also be achieved by invoking
the \LaTeX{} compiler with a more elaborate command line.
Most conveniently this should be done as part
of a shell script or a |Makefile|.

When using \textsf{childdoc} in the main file, the following
command lines effectively perform a redirection
(note that depending on the shell being used,
backslashes may have to be doubled: `|\|' $\to$ `|\\|'):
%
\begin{center}
|... -jobname "|\textit{target}|" |\\|"|[\textit{flags}]%
|\input{childdoc.def}\childdocforward[|\textit{main}|]{|\textit{dest}|}"|
\end{center}
%
Here \textit{target} is the name of the output file,
\textit{main} is the name of the main file
and \textit{dest} is the name of the main or child file to be processed
(all filenames without extensions).
The optional argument \textit{main} can be omitted
if \textit{main} matches \textit{dest}.
Optionally, compilation \textit{flags} can be defined via |\def| commands.
This command line makes the \TeX{} engine believe
it is compiling the file \textit{target}
whose content is specified as the latter parameter.
The provided code then forwards the processing to
\textit{main} or \textit{dest} as described in \secref{sec:forward}.

%%%%%%%%%%%%%%%%%%%%%%%%%%%%%%%%%%%%%%%%%%%%%%%%%%%%%%%%%%%%%%%%%%%%%%%%%%%%%%%%
\subsection{Include by Input}
\label{sec:input}

Including child documents by |\include| has some restrictions by design.
Most notably, the content of a child document always occupies
its own set of pages; pages cannot be shared between child documents.
Usually, this behaviour makes perfect sense
because each child document contain an essential part of the document.
However, in some situations it may be desirable to compose
a document from a collection of parts
without having mandatory page breaks between then.
For this case, the package
provides a mechanism to include parts
by |\input| which can also be processed individually.
However, by construction this mechanism
requires manual handling of the content to be output.

%%%%%%%%%%%%%%%%%%%%%%%%%%%%%%%%%%%%%%%%
\DescribeMacro{\ifchilddocmanual}
The main file should be prepared as usual, see \secref{sec:include}.
However, the document body must make a distinction
between processing of an individual part and of the main document, e.g.:
%
\begin{center}
\begin{tabular}{l}
|\ifchilddocmanual|\\
|\input{\childdocname}|\\
|\||else|\\
\textit{document body with }|\input{|\textit{part}|}|\\
|\||fi|
\end{tabular}
\end{center}
%
The conditional |\ifchilddocmanual| is true whenever
a part to be included by |\input| is being compiled,
and the name of the part is stored in |\childdocname|.

%%%%%%%%%%%%%%%%%%%%%%%%%%%%%%%%%%%%%%%%
\DescribeMacro{\childdocby}
Each part to be included by |\input| should start with:
%
\begin{center}
\begin{tabular}{l}
|\input{childdoc.def}|\\
|\childdocby{|\textit{main}|}|\\
\end{tabular}
\end{center}
%
The directive |\childdocby| is similar to |\childdocof|
described in \secref{sec:include},
but the subsequent selection of content must be done manually.
To that end, both |\ifchilddoc| and |\ifchilddocmanual|
will be true upon processing of a part,
and the name of the part is stored in |\childdocname|.
Note that |\jobname| will be set to the filename of the current part
so that each part receives an individual |.aux| file
that does not interfere with the |.aux| file(s) of the main document.
This behaviour can be altered by the alternative form
|\childdocby[*]{|\textit{main}|}| (with a non-empty optional argument)
which uses the |.aux| file of the main document
by setting |\jobname| to \textit{main}.

%%%%%%%%%%%%%%%%%%%%%%%%%%%%%%%%%%%%%%%%%%%%%%%%%%%%%%%%%%%%%%%%%%%%%%%%%%%%%%%%
\subsection{Driver Development}
\label{sec:driver}

The \textsf{childdoc} mechanism can also be use for the development
of definition files such as \LaTeX{} styles or classes.
This case differs from the above setup with multiple parts
included by |\include| in that no |\includeonly| should be invoked.
This can be achieved by starting the include file
(before |\ProvidesPackage|) with:
%
\begin{center}
\begin{tabular}{l}
|\input{childdoc.def}|\\
|\childdocforward{|\textit{main}|}|\\
\end{tabular}
\end{center}
%
or alternatively with:
%
\begin{center}
\begin{tabular}{l}
|\input{childdoc.def}|\\
|\childdocby{|\textit{main}|}|\\
\end{tabular}
\end{center}
%
Both forms have slightly different effects as described above.
The main file is prepared as usual, see \secref{sec:include}.

%%%%%%%%%%%%%%%%%%%%%%%%%%%%%%%%%%%%%%%%%%%%%%%%%%%%%%%%%%%%%%%%%%%%%%%%%%%%%%%%
\subsection{Legacy Detection}
\label{sec:detection}

The directive |\childdocmain| in the main file can detect
whether the complete document or merely a child is to be compiled
even without using the directive |\childdocof|.
This method is deprecated because it is less robust
and there is no compelling reason to use it;
it is merely provided for backward compatibility
and it may be removed in future versions.

If the detection mechanism is to be used,
it is mandatory to correctly specify
the filename of the main file as the argument of |\childdocmain|:
%
\begin{center}
\begin{tabular}{l}
|\input{childdoc.def}|\\
|\childdocmain{|\textit{main}|}|\\
\end{tabular}
\end{center}
%
If |\jobname| does not match the argument \textit{main} of |\childdocmain|,
it is assumed that |\jobname| points to the child file to be compiled.
When using |\childdocmain| with the main file specified as argument,
it suffices to start a child file
with just |\input{|\textit{main}|}|
without loading of the package and using |\childdocof|.
If instead all processing is done
with the appropriate \textsf{childdoc} directives,
the argument of \textit{main} of |\childdocmain| can be empty.

An alternative version of the command line processing described
in \secref{sec:commandline} using the detection mechanism reads:
%
\begin{center}
|... -jobname "|\textit{target}|" "|[\textit{flags}]%
[|\def\jobname{|\textit{dest}|}|]|\input{|\textit{main}|}"|
\end{center}

%%%%%%%%%%%%%%%%%%%%%%%%%%%%%%%%%%%%%%%%%%%%%%%%%%%%%%%%%%%%%%%%%%%%%%%%%%%%%%%%
\subsection{Manual Code}
\label{sec:manual}

In case one cannot be certain whether the definitions file |childdoc.def|
is installed on the target \TeX{} distribution
and one prefers not to ship it,
it is conceivable to paste a few relevant commands into the sources.

To that end, drop all statements |\input{childdoc.def}|
and perform the replacements as outlined below.
Instead of |\childdocmain{|\textit{main}|}| add the following code
to the top of the main file:
%
\begin{center}
\begin{tabular}{l}
|\||ifdefined\childdocname\endinput\||fi\newif\ifchilddoc|\\
|\edef\childdocname{\scantokens\expandafter{\jobname\noexpand}}|\\
|\def\childdocmain{|\textit{main}|}\||ifx\childdocmain\childdocname\||else|\\
|\childdoctrue\includeonly{\childdocname}\let\jobname\childdocmain\||fi|\\
\end{tabular}
\end{center}
%
Instead of |\childdocof{|\textit{main}|}| just include the main file
at the top of each child file:
%
\begin{center}
|\input{|\textit{main}|}|
\end{center}
%
A simple redirection |\childdocforward{|\textit{dest}|}| is achieved by:
%
\begin{center}
|\def\jobname{|\textit{dest}|}\input{\jobname}|
\end{center}
%
The redirection with prefix
|\childdocforwardprefix[|\textit{prefix}|]{|\textit{dest}|}|
is accomplished by:
%
\begin{center}
\begin{tabular}{l}
|{\edef\jobname{\scantokens\expandafter{\jobname\noexpand}}|\\
|\def\redirectjob |\textit{prefix}|#1~~~{\gdef\jobname{|\textit{dest}|#1}}|\\
|\expandafter\redirectjob\jobname~~~}\input{\jobname}|
\end{tabular}
\end{center}

In an alternative approach,
child documents can be compiled by a specific command line
without additional code or specific definitions:
%
\begin{center}
|... -jobname "|\textit{target}|" "|[\textit{flags}]%
|\includeonly{|\textit{dest}|}\input{|\textit{main}|}"|
\end{center}
%

%%%%%%%%%%%%%%%%%%%%%%%%%%%%%%%%%%%%%%%%%%%%%%%%%%%%%%%%%%%%%%%%%%%%%%%%%%%%%%%%
%%%%%%%%%%%%%%%%%%%%%%%%%%%%%%%%%%%%%%%%%%%%%%%%%%%%%%%%%%%%%%%%%%%%%%%%%%%%%%%%
\section{Information}

%%%%%%%%%%%%%%%%%%%%%%%%%%%%%%%%%%%%%%%%%%%%%%%%%%%%%%%%%%%%%%%%%%%%%%%%%%%%%%%%
\subsection{Copyright}

Copyright \copyright{} 2017--2018 Niklas Beisert

This work may be distributed and/or modified under the
conditions of the \LaTeX{} Project Public License, either version 1.3
of this license or (at your option) any later version.
The latest version of this license is in
  \url{http://www.latex-project.org/lppl.txt}
and version 1.3 or later is part of all distributions of \LaTeX{}
version 2005/12/01 or later.

This work has the LPPL maintenance status `maintained'.

The Current Maintainer of this work is Niklas Beisert.

This work consists of the files |README.txt|, |childdoc.ins| and |childdoc.dtx|
as well as the derived files |childdoc.def|, |cdocsamp.tex|
with |cdocsch1.tex|, |cdocsch2.tex|, |cdocspt3.tex|, |cdocspt4.tex|,
|cdocsdrf.tex|, |cdocsfn1.tex|, |cdocsfn2.tex|
as well as |childdoc.pdf|.

%%%%%%%%%%%%%%%%%%%%%%%%%%%%%%%%%%%%%%%%%%%%%%%%%%%%%%%%%%%%%%%%%%%%%%%%%%%%%%%%
\subsection{Files and Installation}

The package consists of the files:
%
\begin{center}
\begin{tabular}{ll}
    |README.txt|   & readme file \\
    |childdoc.ins| & installation file \\
    |childdoc.dtx| & source file \\
    |childdoc.def| & definition file \\
    |cdocsamp.tex| & sample main file \\
    |cdocsch1.tex| & sample include file \\
    |cdocsch2.tex| & sample include file \\
    |cdocspt3.tex| & sample part file \\
    |cdocspt4.tex| & sample part file \\
    |cdocsdrf.tex| & sample redirection file \\
    |cdocsfn1.tex| & sample redirection file \\
    |cdocsfn2.tex| & sample redirection file \\
    |childdoc.pdf| & manual
\end{tabular}
\end{center}
%
The distribution consists of the files
|README.txt|, |childdoc.ins| and |childdoc.dtx|.
%
\begin{itemize}
\item
Run (pdf)\LaTeX{} on |childdoc.dtx|
to compile the manual |childdoc.pdf| (this file).
\item
Run \LaTeX{} on |childdoc.ins| to create the definitions file |childdoc.def|
and the sample |cdocsamp.tex| with include files
|cdocsch1.tex|, |cdocsch2.tex|, |cdocspt3.tex|, |cdocspt4.tex|,
|cdocsdrf.tex|, |cdocsfn1.tex|, |cdocsfn2.tex|.
Then copy the file |childdoc.def| to an appropriate directory of your \LaTeX{}
distribution, e.g.\ \textit{texmf-root}|/tex/latex/childdoc|.
\end{itemize}

%%%%%%%%%%%%%%%%%%%%%%%%%%%%%%%%%%%%%%%%%%%%%%%%%%%%%%%%%%%%%%%%%%%%%%%%%%%%%%%%
\subsection{Related CTAN Packages}

There are several other packages which offer a similar functionality:
%
\begin{itemize}
\item
The packages
\href{http://ctan.org/pkg/docmute}{\textsf{docmute}},
\href{http://ctan.org/pkg/includex}{\textsf{includex}} and
\href{http://ctan.org/pkg/standalone}{\textsf{standalone}}
provide commands to include only the document body of
a child file thus allowing both files to be compiled individually.
\item
The packages \href{http://ctan.org/pkg/subdocs}{\textsf{subdocs}}
and \href{http://ctan.org/pkg/subfiles}{\textsf{subfiles}}
provide structures in which the main and child documents can be
encapsulated and allowing them to be compiled individually.
The inclusion mechanism is different from the conventional |\include|.
\item
The package \href{http://ctan.org/pkg/combine}{\textsf{combine}}
is an elaborate solution to combine several documents into one.
\end{itemize}
%
See also the CTAN topic \href{http://ctan.org/topic/subdocs}{\textsf{subdocs}}
for further related packages.
The present package differs from the above solutions in that
a document structure constructed with the conventional |\include| mechanism
just needs two extra commands at the top of every file
such that all constituent files can be compiled individually.

%%%%%%%%%%%%%%%%%%%%%%%%%%%%%%%%%%%%%%%%%%%%%%%%%%%%%%%%%%%%%%%%%%%%%%%%%%%%%%%%
%\subsection{Feature Suggestions}
%
%The following is a list of features which may be useful for future
%versions of this package:
%%
%\begin{itemize}
%\item
%\ldots
%\end{itemize}

%%%%%%%%%%%%%%%%%%%%%%%%%%%%%%%%%%%%%%%%%%%%%%%%%%%%%%%%%%%%%%%%%%%%%%%%%%%%%%%%
\subsection{Revision History}

%%%%%%%%%%%%%%%%%%%%%%%%%%%%%%%%%%%%%%%%
\paragraph{v2.0:} 2018/12/30

\begin{itemize}
\item
immediate forward processing
\item
added |\childdocby| mechanism
\item
manual restructured
\end{itemize}

%%%%%%%%%%%%%%%%%%%%%%%%%%%%%%%%%%%%%%%%
\paragraph{v1.6:} 2018/01/17

\begin{itemize}
\item
application for development of include files
\item
corrections to manual
\end{itemize}

%%%%%%%%%%%%%%%%%%%%%%%%%%%%%%%%%%%%%%%%
\paragraph{v1.5:} 2017/05/21

\begin{itemize}
\item
more complete structuring introduced
\item
|\childdocof| introduced
\item
|\childdoc| renamed to |\childdocmain|
\item
|\childredirect| renamed to |\childdocforward| and |\childdocforwardprefix|
and functionality expanded
\end{itemize}

%%%%%%%%%%%%%%%%%%%%%%%%%%%%%%%%%%%%%%%%
\paragraph{v1.0:} 2017/04/27

\begin{itemize}
\item
manual and install package
\item
first version published on CTAN
\end{itemize}

%%%%%%%%%%%%%%%%%%%%%%%%%%%%%%%%%%%%%%%%
\paragraph{v0.6:} 2017/04/26

\begin{itemize}
\item
redirection mechanism added
\end{itemize}

%%%%%%%%%%%%%%%%%%%%%%%%%%%%%%%%%%%%%%%%
\paragraph{v0.5:} 2017/04/26

\begin{itemize}
\item
functionality in definition file
\end{itemize}


%%%%%%%%%%%%%%%%%%%%%%%%%%%%%%%%%%%%%%%%%%%%%%%%%%%%%%%%%%%%%%%%%%%%%%%%%%%%%%%%
%%%%%%%%%%%%%%%%%%%%%%%%%%%%%%%%%%%%%%%%%%%%%%%%%%%%%%%%%%%%%%%%%%%%%%%%%%%%%%%%
%%%%%%%%%%%%%%%%%%%%%%%%%%%%%%%%%%%%%%%%%%%%%%%%%%%%%%%%%%%%%%%%%%%%%%%%%%%%%%%%
\appendix

\settowidth\MacroIndent{\rmfamily\scriptsize 000\ }

 \DocInput{childdoc.dtx}

\end{document}
%</driver>
% \fi
%
% %%%%%%%%%%%%%%%%%%%%%%%%%%%%%%%%%%%%%%%%%%%%%%%%%%%%%%%%%%%%%%%%%%%%%%%%%%%%%%
% %%%%%%%%%%%%%%%%%%%%%%%%%%%%%%%%%%%%%%%%%%%%%%%%%%%%%%%%%%%%%%%%%%%%%%%%%%%%%%
% \section{Sample}
%\iffalse
%<*samplemain>
%\fi
%
% The following presents a sample document
% with two chapters, two parts, a title page,
% a compile flag as well as three forwarding files to set the flag.
% It consists of eight |.tex| files:
% \begin{center}
% \begin{tabular}{ll}
% |cdocsamp.tex|&main file\\
% |cdocsch1.tex|&include file for chapter 1\\
% |cdocsch2.tex|&include file for chapter 2\\
% |cdocspt3.tex|&include file for part 3\\
% |cdocspt4.tex|&include file for part 4\\
% |cdocsdrf.tex|&forwarding file for main file in draft mode\\
% |cdocsfi1.tex|&forwarding file for final version of chapter 1\\
% |cdocsfi2.tex|&forwarding file for final version of chapter 2\\
% \end{tabular}
% \end{center}
% Each of the eight files can be compiled directly by the \LaTeX{} compiler.
%
% %%%%%%%%%%%%%%%%%%%%%%%%%%%%%%%%%%%%%%
% \paragraph{Main File.}
%
% The main file is called |cdocsamp.tex|.
%
% Load the \textsf{childdoc} definitions and
% declare the filename for the main document:
%    \begin{macrocode}
\input{childdoc.def}
\childdocmain{}
%    \end{macrocode}

% Optional override for |\version| flag:
%    \begin{macrocode}
%%\ifchilddoc\else\providecommand{\version}{draft}\fi
%    \end{macrocode}

% Define the default values for the |\version| flag
% (|final| for the main file and |draft| for childs):
%    \begin{macrocode}
\ifchilddoc
\providecommand{\version}{draft}
\else
\providecommand{\version}{final}
\fi
%    \end{macrocode}

% Load the standard document class:
%    \begin{macrocode}
\documentclass[12pt]{article}
%    \end{macrocode}

% Start the document body:
%    \begin{macrocode}
\begin{document}
%    \end{macrocode}

% Declare a title page.
% Print title, part of document being processed and version flag:
%    \begin{macrocode}
\addtocounter{page}{-1}
\begin{center}
{\LARGE\bfseries{}childdoc example\par}
\vspace{1cm}
\ifchilddoc
\ifchilddocmanual part\else chapter\fi:
`\childdocname' of `\childdocjob'\par
\else
main document: `\childdocjob'\par
\fi
version: \version\par
\end{center}
\newpage
%    \end{macrocode}

% Manually include selected file,
% otherwise process as usual:
%    \begin{macrocode}
\ifchilddocmanual
\section*{part `\childdocname'}
\input{\childdocname}
\else
%    \end{macrocode}

% Include the two chapters:
%    \begin{macrocode}
\include{cdocsch1}
\include{cdocsch2}
%    \end{macrocode}

% Include the two parts unless only chapters should be displayed:
%    \begin{macrocode}
\ifchilddoc\else
\section{part three}
\input{cdocspt3}
\section{part four}
\input{cdocspt4}
\fi
%    \end{macrocode}

% Process as usual until here:
%    \begin{macrocode}
\fi
%    \end{macrocode}

% End of document body:
%    \begin{macrocode}
\end{document}
%    \end{macrocode}
%\iffalse
%</samplemain>
%\fi
%
% %%%%%%%%%%%%%%%%%%%%%%%%%%%%%%%%%%%%%%
% \paragraph{Chapter Include Files.}
%
% The include files are called |cdocsch1.tex| and |cdocsch2.tex|.
%
%\iffalse
%<*samplechap1|samplechap2>
%\fi

% Optional override for |\version| flag:
%    \begin{macrocode}
%%\providecommand{\version}{final}
%    \end{macrocode}

% Include the main document:
%    \begin{macrocode}
\input{childdoc.def}
\childdocof{cdocsamp}
%    \end{macrocode}

%\iffalse
%</samplechap1|samplechap2>
%\fi
%
%\iffalse
%<*samplechap1>
%\fi
% Some text for chapter 1:
%    \begin{macrocode}
\section{one}
some text in chapter one
%    \end{macrocode}

%\iffalse
%</samplechap1>
%\fi
% Some text for chapter 2:
%\iffalse
%<*samplechap2>
%\fi
%    \begin{macrocode}
\section{two}
more text in chapter two
%    \end{macrocode}

%\iffalse
%</samplechap2>
%\fi
%
% %%%%%%%%%%%%%%%%%%%%%%%%%%%%%%%%%%%%%%
% \paragraph{Part Include Files.}
%
% The include files are called |cdocspt3.tex| and |cdocspt4.tex|.
%
%\iffalse
%<*samplepart3|samplepart4>
%\fi

% Optional override for |\version| flag:
%    \begin{macrocode}
%%\providecommand{\version}{final}
%    \end{macrocode}

% Include the main document:
%    \begin{macrocode}
\input{childdoc.def}
\childdocby{cdocsamp}
%    \end{macrocode}

%\iffalse
%</samplepart3|samplepart4>
%\fi
%
%\iffalse
%<*samplepart3>
%\fi
% Some text for part 3:
%    \begin{macrocode}
some text in part three
%    \end{macrocode}

%\iffalse
%</samplepart3>
%\fi
% Some text for part 4:
%\iffalse
%<*samplepart4>
%\fi
%    \begin{macrocode}
more text in part four
%    \end{macrocode}

%\iffalse
%</samplepart4>
%\fi
%
% %%%%%%%%%%%%%%%%%%%%%%%%%%%%%%%%%%%%%%
% \paragraph{Forwarding for a Complete Draft.}
%
% The following forwarding file |cdocsdrf.tex|
% compiles the main document in draft mode:
%\iffalse
%<*sampledraft>
%\fi
%    \begin{macrocode}
\def\version{draft}
\input{childdoc.def}
\childdocforward{cdocsamp}
%    \end{macrocode}

%\iffalse
%</sampledraft>
%\fi
%
% %%%%%%%%%%%%%%%%%%%%%%%%%%%%%%%%%%%%%%
% \paragraph{Forwarding for Final Version of the Chapters.}
%
% The following forwarding files |cdocsfn1.tex| and |cdocsfn2.tex|
% (with identical content)
% compile the final versions of the child documents
% |cdocsch1.tex| and |cdocsch2.tex|, respectively:
%\iffalse
%<*samplefinal>
%\fi
%    \begin{macrocode}
\def\version{final}
\input{childdoc.def}
\childdocforwardprefix[cdocsamp]{cdocsfn}{cdocsch}
%    \end{macrocode}

%\iffalse
%</samplefinal>
%\fi
%
% %%%%%%%%%%%%%%%%%%%%%%%%%%%%%%%%%%%%%%
% \paragraph{Command Line Processing.}
%
% The following three command lines generate the output files
% |cdocscld|, |cdocscl1| and |cdocscl2|
% which should be identical to
% |cdocsdrf|, |cdocsch1| and |cdocsfn2|, respectively:
% \begin{center}
% \begin{tabular}{l}
% |latex -jobname cdocscld \|\\
% |  "\def\version{draft}\input{childdoc.def}\childdocforward{cdocsamp}"|\\
% |latex -jobname cdocscl1 \|\\
% |  "\input{childdoc.def}\childdocforward[cdocsamp]{cdocsch1}"|\\
% |latex -jobname cdocscl2 \|\\
% |  "\def\version{final}\input{childdoc.def}\childdocforward{cdocsch2}"|
% \end{tabular}
% \end{center}
% Note that the trailing backslash on each first line
% merely continues the input to the second line
% (for convenient cut ant paste).
% Furthermore, the command |latex| can be replaced by any
% of its alternative versions such as |pdflatex|.
%
% %%%%%%%%%%%%%%%%%%%%%%%%%%%%%%%%%%%%%%%%%%%%%%%%%%%%%%%%%%%%%%%%%%%%%%%%%%%%%%
% %%%%%%%%%%%%%%%%%%%%%%%%%%%%%%%%%%%%%%%%%%%%%%%%%%%%%%%%%%%%%%%%%%%%%%%%%%%%%%
% \section{Implementation}
%\iffalse
%<*package>
%\fi
%
% This section describes the definitions file |childdoc.def|.

% The definitions cannot be loaded using |\usepackage| or |\RequirePackage|
% which has a mechanism to prevent loading a style file more than once.
% When loading the definitions by means of |\input|
% multiple instances have to be prevented manually:
%\iffalse
%This code needs to be before the `\ProvidesFile' directive
%which is defined at the beginning of this file.
%Therefore it is also placed there and commented out here.
%</package>
%<*discard>
%\fi
%    \begin{macrocode}
\ifdefined\childdocmain\endinput\fi
%    \end{macrocode}
%\iffalse
%</discard>
%<*package>
%\fi
%
% \macro{\ifchilddoc}
% \macro{\ifchilddocmanual}
% The conditional |\ifchilddoc| tells whether a
% child (true) or main (false) document is being compiled.
% The conditional |\ifchilddocmanual| tells whether
% the |\includeonly| mechanism is used (false) or
% the selection of child files must be performed manually (true).
% The definitions initialise to false:
%    \begin{macrocode}
\newif\ifchilddoc
\newif\ifchilddocmanual
%    \end{macrocode}

% \macro{\childdocname}
% \macro{\childdocjob}
% The macro |\childdocname| stores the name of the main document
% to be compiled. The macro |\childdocjob| stores the name of
% the document on which the \LaTeX{} compiler was originally invoked.
% The content of |\jobname| cannot be compared
% to filenames specified in the source due to different catcodes.
% The following code rescans |\jobname|, stores the result
% in |\childdocname| and saves a copy in |\childdocjob|:
%    \begin{macrocode}
\edef\childdocname{\scantokens\expandafter{\jobname\noexpand}}
\let\childdocjob\childdocname
%    \end{macrocode}

% \macro{\childdocdisable}
% The macro |\childdocdisable| prevents the main file
% from being processed more than once.
% At this stage, the main document command |\childdocmain|
% is assumed to be called once again where it should do nothing.
% Any subsequent call to it should prevent
% a secondary processing of the main document
% It overwrites the forwarding commands
% |\childdocof| and |\childdocforward|
% with empty macros to prevent further inclusions of the main document:
%    \begin{macrocode}
\newcommand{\childdocdisable}
{
  \renewcommand{\childdocmain}[1]{\renewcommand{\childdocmain}[1]{\endinput}}
  \renewcommand{\childdocof}[1]{}
  \renewcommand{\childdocby}[2][]{}
  \renewcommand{\childdocforward}[2][]{}
  \renewcommand{\childdocdisable}{}
}
%    \end{macrocode}

% \macro{\childdocmain}
% The macro |\childdocmain| is to be called at the top of the main file
% with nothing or the main filename (without extension) as argument.
% First, it breaks loops.
% If the argument is not empty and does not match |\childdocname|
% (which is set by the first inclusion of |childdoc.def|),
% |\ifchilddoc| is set to true, |\includeonly| is applied to the child file
% and |\jobname| is set to the main file
% (for proper handling of |.aux| files):
%    \begin{macrocode}
\newcommand{\childdocmain}[1]
{
  \childdocdisable\childdocmain{}
  \if?#1?\else
    \begingroup
      \def\childdoctmp{#1}
      \ifx\childdoctmp\childdocname
        \def\childdoctmp{}
      \else
        \def\childdoctmp
        {
          \childdoctrue
          \includeonly{\childdocname}
          \def\childdocjob{#1}
          \def\jobname{#1}
        }
      \fi
      \expandafter
    \endgroup
    \childdoctmp
  \fi
}
%    \end{macrocode}

% \macro{\childdocof}
% The command |\childdocof| redirects
% compilation to the main file |#1|.
%    \begin{macrocode}
\newcommand{\childdocof}[1]
{
  \childdocdisable
  \childdoctrue
  \includeonly{\childdocname}
  \def\jobname{#1}
  \def\childdocjob{#1}
  \input{#1}
}
%    \end{macrocode}

% \macro{\childdocby}
% The command |\childdocby| ....
%    \begin{macrocode}
\newcommand{\childdocby}[2][]
{
  \childdocdisable
  \childdoctrue
  \childdocmanualtrue
  \if?#1?\else
    \def\jobname{#2}
  \fi
  \def\childdocjob{#2}
  \input{#2}
  \endinput
}
%    \end{macrocode}

% \macro{\childdocforward}
% The command |\childdocforward| redirects
% compilation to the main file or
% (if the optional argument is given) a child file.
% Parameters are set as if the main file
% or a child file starting with |\childdocof| was compiled.
% Then compilation is handed over to the main file:
%    \begin{macrocode}
\newcommand{\childdocforward}[2][]
{
  \begingroup
    \if?#1?
      \def\childdoctmp
      {
        \def\childdocname{#2}
        \def\childdocjob{#2}
        \def\jobname{#2}
        \input{#2}
        \endinput
      }
    \else
      \def\childdoctmp
      {
        \childdocdisable
        \def\childdocname{#2}
        \childdoctrue
        \includeonly{#2}
        \def\childdocjob{#1}
        \def\jobname{#1}
        \input{#1}
        \endinput
      }
    \fi
    \expandafter
  \endgroup
  \childdoctmp
}
%    \end{macrocode}

% \macro{\childdocforwardprefix}
% The command |\childdocforwardprefix| redirects
% compilation to the main or a child file by means of a pattern.
% The prefix |#1| in the current filename is replaced by |#2|
% and the suffix of the current filename is kept
% (it is assumed that the filename does not contain the substring `|~~~|'
% which is used as a delimiter).
% Compilation is handed over to the new file by |\childdocforward|:
%    \begin{macrocode}
\newcommand{\childdocforwardprefix}[3][]
{
  \begingroup
    \def\childdocextract #2##1~~~{\def\childdoctmp{\childdocforward[#1]{#3##1}}}
    \expandafter\childdocextract\childdocname~~~
    \expandafter
  \endgroup
  \childdoctmp
}
%    \end{macrocode}

% \macro{\childdoc}
% The deprecated macro |\childdoc| is a legacy version of |\childdocmain|:
%    \begin{macrocode}
\newcommand{\childdoc}{\childdocmain}
%    \end{macrocode}

% \macro{\childdocredirect}
% The deprecated macro |\childdocredirect| is a legacy version
% of |\childdocforward| and |\childdocforwardprefix|:
%    \begin{macrocode}
\newcommand{\childdocredirect}[2][]
{
  \begingroup
    \if?#1?
      \def\childdoctmp{\childdocforward{#2}}
    \else
      \def\childdoctmp{\childdocforwardprefix{#1}{#2}}
    \fi
    \expandafter
  \endgroup
  \childdoctmp
}
%    \end{macrocode}

%\iffalse
%</package>
%\fi
%
\endinput
|\\
|\childdocof{|\textit{main}|}|\\
\end{tabular}
\end{center}
at the top of every child file \textit{child}
which is included by |\include{|\textit{child}|}|
from within the main file
(or at least for those files to be compiled individually).
The argument \textit{main} must be the filename of the main file.

There are a couple of
considerations in setting up the main and child documents:

%%%%%%%%%%%%%%%%%%%%%%%%%%%%%%%%%%%%%%%%
\paragraph{Restrictions.}

Please note the following restrictions:
\begin{itemize}
\item
|\childdocmain| must be called with one argument \textit{main}
to ensure compatibility with earlier version of the package.
It must either be empty (|\childdocmain{}|)
or precisely match the filename of the main file in which it is specified.
See \secref{sec:detection} for further information.
\item
The filename \textit{main} must be specified without the |.tex| extension.
\item
The filename \textit{main} is case sensitive
(even in case-insensitive file systems)
due to internal string comparison.
\item
The argument \textit{main} should be fully expanded, it cannot be a macro.
\item
Subdirectories and special characters should be avoided in filenames.
\item
The command |\childdocmain{|\textit{main}|}| must be followed by a whitespace.
It should not be followed immediately by another command
or by a comment mark `|%|'.
This is because the \TeX{} parser reads the token immediately following
the argument of |\childdocmain| and puts it
at the beginning of every child section;
however, a white\-space is ignored.
\end{itemize}

%%%%%%%%%%%%%%%%%%%%%%%%%%%%%%%%%%%%%%%%
\paragraph{Content of Main File.}

It is advisable to place all content in the child files included by |\include|.
Any output contained in the main file will appear in all child documents
unless suppressed manually;
it cannot be suppressed automatically by the |\includeonly| directive
and thus should normally be avoided.
A method to include some content in the main file
by means of conditional processing is described in \secref{sec:conditional}.

%%%%%%%%%%%%%%%%%%%%%%%%%%%%%%%%%%%%%%%%
\paragraph{Page Numbering.}

When only a part of the document is compiled,
the appropriate numbering of pages
(as well as other status parameters)
is determined from the |.aux| files.
The latter contain information from previous passes.
However this information needs to propagate through
all intermediate child documents.
Therefore the page numbering in child documents may well
be inconsistent until the complete document is compiled at least once.

A useful (if unconventional) way to always ensure a consistent
page numbering is to restart the numbering in each child document
and denote the pages by `\textit{child}|.|\textit{page}'
where \textit{child} represents the chapter/section number of the child file.
This can be achieved by the command
|\numberwithin{page}{|\textit{child}|}|
of the \textsf{amsmath} package
where \textit{child} can be |chapter| or |section|
depending on the chosen structuring.
Alternatively, one can modify the macro |\thepage| appropriately
and reset the counter |page| at the start of each child file.

%%%%%%%%%%%%%%%%%%%%%%%%%%%%%%%%%%%%%%%%%%%%%%%%%%%%%%%%%%%%%%%%%%%%%%%%%%%%%%%%
\subsection{Conditional Processing}
\label{sec:conditional}

The package provides a mechanism to compile different versions
of a document. To customise the versions further some conditional processing
can come in handy to distinguish which version is being compiled.
The package provides two macros to describe the compilation context:

%%%%%%%%%%%%%%%%%%%%%%%%%%%%%%%%%%%%%%%%
\DescribeMacro{\ifchilddoc}
The conditional |\ifchilddoc| distinguishes between the compilation of
child documents and the main document:
%
\begin{center}
|\ifchilddoc |\textit{child-code}| |[|\||else |\textit{main-code}]| \||fi|
\end{center}

%%%%%%%%%%%%%%%%%%%%%%%%%%%%%%%%%%%%%%%%
\DescribeMacro{\childdocname}
\DescribeMacro{\childdocjob}
The macro |\childdocname| contains the filename (without extension)
of the main or child file being processed.
Note that |\childdocjob| will always contain the name of the main file.

%%%%%%%%%%%%%%%%%%%%%%%%%%%%%%%%%%%%%%%%
\paragraph{Title Page.}

Conditional processing can be used to include a title or banner page
in the main document when proper precautions are taken.
Importantly, the code in the main file should ensure that the page counter
(as well as other status parameters which are stored in the |.aux| files)
takes the same value after the conditional processing.
Otherwise the page numbers may take divergent values
depending on which part is compiled.

For example, a title page could be declared by:
%
\begin{center}
\begin{tabular}{l}
|\ifchilddoc\||else|\\
|\addtocounter{page}{-1}|\\
\textit{code for title page}\\
|\newpage|\\
|\||fi|
\end{tabular}
\end{center}
%
A banner page for the child documents can be generated by:
%
\begin{center}
\begin{tabular}{l}
|\ifchilddoc|\\
|\addtocounter{page}{-1}|\\
\textit{code for banner page}\\
|\newpage|\\
|\||fi|
\end{tabular}
\end{center}
%
Here one could write a message such as:
\begin{center}
|This is the part \childdocname{} of \childdocjob{}.|
\end{center}

%%%%%%%%%%%%%%%%%%%%%%%%%%%%%%%%%%%%%%%%%%%%%%%%%%%%%%%%%%%%%%%%%%%%%%%%%%%%%%%%
\subsection{Flags}
\label{sec:flags}

The package makes it easy to generate different versions
of the main or child documents.
To this end compilation flags can be defined
and assigned different default values.
They will be particularly useful in conjunction
with the forwarding mechanism described in \secref{sec:forward}.

For example, it may be useful to have a flag |\version|
which can be set to |draft| or |final|.
The document source will contain some conditional code
depending on the value of |\version|.
Suppose further, the flag should default to |final| for the main file
and to |draft| for child files
which is a natural assignment for editing the document.
This is achieved by placing the following code
in the preamble of the main document
(below the |\childdocmain| directive):
%
\begin{center}
\begin{tabular}{l}
|\ifchilddoc|\\
|\providecommand{\version}{draft}|\\
|\||else|\\
|\providecommand{\version}{final}|\\
|\||fi|
\end{tabular}
\end{center}
%
The definition by |\providecommand| makes sure
that previous definitions are not overwritten.
Further statements |\providecommand{\version}{...}|
can thus be added before the above code to override it.

For the main file, one might add a line
(between |\childdocmain| and the above block)
%
\begin{center}
|%\ifchilddoc\||else\providecommand{\version}{draft}\||fi|
\end{center}
%
which can be uncommented to produce a draft version.
Likewise one can add a line to the very top of a child file
(above the |\childdocof{|\textit{main}|}| directive)
%
\begin{center}
|%\providecommand{\version}{final}|
\end{center}
%
which can be uncommented to produce the final version of this child document.

%%%%%%%%%%%%%%%%%%%%%%%%%%%%%%%%%%%%%%%%%%%%%%%%%%%%%%%%%%%%%%%%%%%%%%%%%%%%%%%%
\subsection{Forwarding}
\label{sec:forward}

Different versions of the main or child documents
using compilation flags as described in \secref{sec:flags}
can be (permanently) stored in different files
for convenient compilation, viewing and distribution.
To this end, the package defines a command
to pass on compilation to a different file:

%%%%%%%%%%%%%%%%%%%%%%%%%%%%%%%%%%%%%%%%
\DescribeMacro{\childdocforward}
The command |\childdocforward| redirects processing to
another source file:
%
\begin{center}
\begin{tabular}{l}
|% \iffalse
%
% childdoc.dtx Copyright (C) 2017-2018 Niklas Beisert
%
% This work may be distributed and/or modified under the
% conditions of the LaTeX Project Public License, either version 1.3
% of this license or (at your option) any later version.
% The latest version of this license is in
%   http://www.latex-project.org/lppl.txt
% and version 1.3 or later is part of all distributions of LaTeX
% version 2005/12/01 or later.
%
% This work has the LPPL maintenance status `maintained'.
%
% The Current Maintainer of this work is Niklas Beisert.
%
% This work consists of the files childdoc.dtx and childdoc.ins
% and the derived files childdoc.def and cdocsamp.tex with
% cdocsch1.tex, cdocsch2.tex, cdocsdrf.tex, cdocsfn1.tex, cdocsfn2.tex.
%
%<package>\ifdefined\childdocmain\endinput\fi
%<package>\ProvidesFile{childdoc.def}[2018/12/30 v2.0 child document driver]
%<samplemain>\ProvidesFile{cdocsamp.tex}[2018/12/30 v2.0 sample for childdoc]
%<*driver>
%\ProvidesFile{childdoc.drv}[2018/12/30 v2.0 childdoc reference manual file]
\PassOptionsToClass{10pt,a4paper}{article}
\documentclass{ltxdoc}

\usepackage[margin=35mm]{geometry}
\usepackage{hyperref}
\usepackage{hyperxmp}
\usepackage[usenames]{color}

\hypersetup{colorlinks=true}
\hypersetup{pdfstartview=FitH}
\hypersetup{pdfpagemode=UseNone}
\hypersetup{pdfsource={}}
\hypersetup{pdflang={en-UK}}
\hypersetup{pdfcopyright={Copyright 2017-2018 Niklas Beisert.
  This work may be distributed and/or modified under the
  conditions of the LaTeX Project Public License, either version 1.3
  of this license or (at your option) any later version.}}
\hypersetup{pdflicenseurl={http://www.latex-project.org/lppl.txt}}
\hypersetup{pdfcontactaddress={ETH Zurich, ITP, HIT K,
  Wolfgang-Pauli-Strasse 27}}
\hypersetup{pdfcontactpostcode={8093}}
\hypersetup{pdfcontactcity={Zurich}}
\hypersetup{pdfcontactcountry={Switzerland}}
\hypersetup{pdfcontactemail={nbeisert@itp.phys.ethz.ch}}
\hypersetup{pdfcontacturl={http://people.phys.ethz.ch/\xmptilde nbeisert/}}

\newcommand{\secref}[1]{\hyperref[#1]{section \ref*{#1}}}

\parskip1ex
\parindent0pt
\let\olditemize\itemize
\def\itemize{\olditemize\parskip0pt}

\begin{document}

\title{The \textsf{childdoc} Package}
\hypersetup{pdftitle={The childdoc Package}}
\author{Niklas Beisert\\[2ex]
  Institut f\"ur Theoretische Physik\\
  Eidgen\"ossische Technische Hochschule Z\"urich\\
  Wolfgang-Pauli-Strasse 27, 8093 Z\"urich, Switzerland\\[1ex]
  \href{mailto:nbeisert@itp.phys.ethz.ch}
  {\texttt{nbeisert@itp.phys.ethz.ch}}}
\hypersetup{pdfauthor={Niklas Beisert}}
\hypersetup{pdfsubject={Manual for the LaTeX2e Package childdoc}}
\date{30 December 2018, \textsf{v2.0}}
\maketitle

\begin{abstract}\noindent
\textsf{childdoc} is a \LaTeXe{} package
that enables the direct compilation
of document sections included by |\include|
to individual files.
\end{abstract}

\begingroup
\parskip0ex
\tableofcontents
\endgroup

%%%%%%%%%%%%%%%%%%%%%%%%%%%%%%%%%%%%%%%%%%%%%%%%%%%%%%%%%%%%%%%%%%%%%%%%%%%%%%%%
%%%%%%%%%%%%%%%%%%%%%%%%%%%%%%%%%%%%%%%%%%%%%%%%%%%%%%%%%%%%%%%%%%%%%%%%%%%%%%%%
\section{Introduction}

\LaTeX{} provides a mechanism to structure a large document (such as a book)
into a main file and several child files (containing the chapters)
using the |\include| command.
This mechanism is beneficial for documents
which span hundreds of pages in order to
make the source file(s) more manageable.
Moreover, compilation can be restricted to
selected child files by means of the |\includeonly| command.
The latter feature can be used to reduce the compilation time while editing
(this was significantly more useful in the earlier days of \LaTeX{})
or to generate a smaller document which is easier to navigate.
Another application of |\includeonly| is to generate
documents consisting of selected parts of the complete document.

However, there are a few drawbacks of the plain |\include| mechanism:
\begin{itemize}
\item
The child files cannot be compiled on their own,
they can only be compiled via the main file.
A naive editing environment
(such as a text editor with an option
to have the current file processed by \LaTeX)
may require one to switch to the main file before compiling;
attempting to compile the child file produces errors.
\item
The main file must be modified (each time)
to adjust the |\includeonly| command
to the present needs. This easily leaves the main file in a messy state.
\item
The generated document will always carry the filename
of the main document. This is inconvenient if
several child files are to be compiled and
to be kept for distribution.
\end{itemize}

The present package provides a simple interface
to make child files individually compilable by \LaTeX{}.
Compiling a child file then has the same effect as compiling
the main file with an |\includeonly| command
to select the appropriate child.
Moreover the generated document will carry the name of the child
rather than the main file.
This resolves all three above issues.

This feature is meant to make the editing of books,
thesis documents and lecture notes somewhat more convenient.
However, the package can also be used efficiently for
composing a series of documents (such as exercise sheets)
which are typically distributed individually.
It then assists the author in generating the individual documents
(potentially in different versions)
as well as a document containing the collected series.
Another application is in developing style files
or other kinds of included material
where compilation of the style file could redirect
to a sample or test file.

%%%%%%%%%%%%%%%%%%%%%%%%%%%%%%%%%%%%%%%%%%%%%%%%%%%%%%%%%%%%%%%%%%%%%%%%%%%%%%%%
%%%%%%%%%%%%%%%%%%%%%%%%%%%%%%%%%%%%%%%%%%%%%%%%%%%%%%%%%%%%%%%%%%%%%%%%%%%%%%%%
\section{Usage}

First of all, the package \textsf{childdoc} is \emph{not} a standard
\LaTeXe{} |.sty| style file! Therefore it needs to be invoked in
a non-standard way.

%%%%%%%%%%%%%%%%%%%%%%%%%%%%%%%%%%%%%%%%%%%%%%%%%%%%%%%%%%%%%%%%%%%%%%%%%%%%%%%%
\subsection{Included Files}
\label{sec:include}

%%%%%%%%%%%%%%%%%%%%%%%%%%%%%%%%%%%%%%%%
\DescribeMacro{\childdocmain}
To use the package, add the commands
\begin{center}
\begin{tabular}{l}
|\input{childdoc.def}|\\
|\childdocmain{}|\\
\end{tabular}
\end{center}
at the very top of the main \LaTeX{} file,
in particular \emph{before} the |\documentclass| statement!
The argument of |\childdocmain| should be left empty
(but it must be present).

%%%%%%%%%%%%%%%%%%%%%%%%%%%%%%%%%%%%%%%%
\DescribeMacro{\childdocof}
Furthermore, add the commands
\begin{center}
\begin{tabular}{l}
|\input{childdoc.def}|\\
|\childdocof{|\textit{main}|}|\\
\end{tabular}
\end{center}
at the top of every child file \textit{child}
which is included by |\include{|\textit{child}|}|
from within the main file
(or at least for those files to be compiled individually).
The argument \textit{main} must be the filename of the main file.

There are a couple of
considerations in setting up the main and child documents:

%%%%%%%%%%%%%%%%%%%%%%%%%%%%%%%%%%%%%%%%
\paragraph{Restrictions.}

Please note the following restrictions:
\begin{itemize}
\item
|\childdocmain| must be called with one argument \textit{main}
to ensure compatibility with earlier version of the package.
It must either be empty (|\childdocmain{}|)
or precisely match the filename of the main file in which it is specified.
See \secref{sec:detection} for further information.
\item
The filename \textit{main} must be specified without the |.tex| extension.
\item
The filename \textit{main} is case sensitive
(even in case-insensitive file systems)
due to internal string comparison.
\item
The argument \textit{main} should be fully expanded, it cannot be a macro.
\item
Subdirectories and special characters should be avoided in filenames.
\item
The command |\childdocmain{|\textit{main}|}| must be followed by a whitespace.
It should not be followed immediately by another command
or by a comment mark `|%|'.
This is because the \TeX{} parser reads the token immediately following
the argument of |\childdocmain| and puts it
at the beginning of every child section;
however, a white\-space is ignored.
\end{itemize}

%%%%%%%%%%%%%%%%%%%%%%%%%%%%%%%%%%%%%%%%
\paragraph{Content of Main File.}

It is advisable to place all content in the child files included by |\include|.
Any output contained in the main file will appear in all child documents
unless suppressed manually;
it cannot be suppressed automatically by the |\includeonly| directive
and thus should normally be avoided.
A method to include some content in the main file
by means of conditional processing is described in \secref{sec:conditional}.

%%%%%%%%%%%%%%%%%%%%%%%%%%%%%%%%%%%%%%%%
\paragraph{Page Numbering.}

When only a part of the document is compiled,
the appropriate numbering of pages
(as well as other status parameters)
is determined from the |.aux| files.
The latter contain information from previous passes.
However this information needs to propagate through
all intermediate child documents.
Therefore the page numbering in child documents may well
be inconsistent until the complete document is compiled at least once.

A useful (if unconventional) way to always ensure a consistent
page numbering is to restart the numbering in each child document
and denote the pages by `\textit{child}|.|\textit{page}'
where \textit{child} represents the chapter/section number of the child file.
This can be achieved by the command
|\numberwithin{page}{|\textit{child}|}|
of the \textsf{amsmath} package
where \textit{child} can be |chapter| or |section|
depending on the chosen structuring.
Alternatively, one can modify the macro |\thepage| appropriately
and reset the counter |page| at the start of each child file.

%%%%%%%%%%%%%%%%%%%%%%%%%%%%%%%%%%%%%%%%%%%%%%%%%%%%%%%%%%%%%%%%%%%%%%%%%%%%%%%%
\subsection{Conditional Processing}
\label{sec:conditional}

The package provides a mechanism to compile different versions
of a document. To customise the versions further some conditional processing
can come in handy to distinguish which version is being compiled.
The package provides two macros to describe the compilation context:

%%%%%%%%%%%%%%%%%%%%%%%%%%%%%%%%%%%%%%%%
\DescribeMacro{\ifchilddoc}
The conditional |\ifchilddoc| distinguishes between the compilation of
child documents and the main document:
%
\begin{center}
|\ifchilddoc |\textit{child-code}| |[|\||else |\textit{main-code}]| \||fi|
\end{center}

%%%%%%%%%%%%%%%%%%%%%%%%%%%%%%%%%%%%%%%%
\DescribeMacro{\childdocname}
\DescribeMacro{\childdocjob}
The macro |\childdocname| contains the filename (without extension)
of the main or child file being processed.
Note that |\childdocjob| will always contain the name of the main file.

%%%%%%%%%%%%%%%%%%%%%%%%%%%%%%%%%%%%%%%%
\paragraph{Title Page.}

Conditional processing can be used to include a title or banner page
in the main document when proper precautions are taken.
Importantly, the code in the main file should ensure that the page counter
(as well as other status parameters which are stored in the |.aux| files)
takes the same value after the conditional processing.
Otherwise the page numbers may take divergent values
depending on which part is compiled.

For example, a title page could be declared by:
%
\begin{center}
\begin{tabular}{l}
|\ifchilddoc\||else|\\
|\addtocounter{page}{-1}|\\
\textit{code for title page}\\
|\newpage|\\
|\||fi|
\end{tabular}
\end{center}
%
A banner page for the child documents can be generated by:
%
\begin{center}
\begin{tabular}{l}
|\ifchilddoc|\\
|\addtocounter{page}{-1}|\\
\textit{code for banner page}\\
|\newpage|\\
|\||fi|
\end{tabular}
\end{center}
%
Here one could write a message such as:
\begin{center}
|This is the part \childdocname{} of \childdocjob{}.|
\end{center}

%%%%%%%%%%%%%%%%%%%%%%%%%%%%%%%%%%%%%%%%%%%%%%%%%%%%%%%%%%%%%%%%%%%%%%%%%%%%%%%%
\subsection{Flags}
\label{sec:flags}

The package makes it easy to generate different versions
of the main or child documents.
To this end compilation flags can be defined
and assigned different default values.
They will be particularly useful in conjunction
with the forwarding mechanism described in \secref{sec:forward}.

For example, it may be useful to have a flag |\version|
which can be set to |draft| or |final|.
The document source will contain some conditional code
depending on the value of |\version|.
Suppose further, the flag should default to |final| for the main file
and to |draft| for child files
which is a natural assignment for editing the document.
This is achieved by placing the following code
in the preamble of the main document
(below the |\childdocmain| directive):
%
\begin{center}
\begin{tabular}{l}
|\ifchilddoc|\\
|\providecommand{\version}{draft}|\\
|\||else|\\
|\providecommand{\version}{final}|\\
|\||fi|
\end{tabular}
\end{center}
%
The definition by |\providecommand| makes sure
that previous definitions are not overwritten.
Further statements |\providecommand{\version}{...}|
can thus be added before the above code to override it.

For the main file, one might add a line
(between |\childdocmain| and the above block)
%
\begin{center}
|%\ifchilddoc\||else\providecommand{\version}{draft}\||fi|
\end{center}
%
which can be uncommented to produce a draft version.
Likewise one can add a line to the very top of a child file
(above the |\childdocof{|\textit{main}|}| directive)
%
\begin{center}
|%\providecommand{\version}{final}|
\end{center}
%
which can be uncommented to produce the final version of this child document.

%%%%%%%%%%%%%%%%%%%%%%%%%%%%%%%%%%%%%%%%%%%%%%%%%%%%%%%%%%%%%%%%%%%%%%%%%%%%%%%%
\subsection{Forwarding}
\label{sec:forward}

Different versions of the main or child documents
using compilation flags as described in \secref{sec:flags}
can be (permanently) stored in different files
for convenient compilation, viewing and distribution.
To this end, the package defines a command
to pass on compilation to a different file:

%%%%%%%%%%%%%%%%%%%%%%%%%%%%%%%%%%%%%%%%
\DescribeMacro{\childdocforward}
The command |\childdocforward| redirects processing to
another source file:
%
\begin{center}
\begin{tabular}{l}
|\input{childdoc.def}|\\
|\childdocforward[|\textit{main}|]{|\textit{dest}|}|\\
\end{tabular}
\end{center}
%
The argument \textit{dest} is the destination file
(without extension).
It should be the main file or one of the child files.
Note that further \textsf{childdoc} directives
such as |\childdocof| and |\childdocforward|
in the indicated file will be processed in this form.
The optional argument \textit{main}
passes on directly to the main file \textit{main}
while pretending to compile the child \textit{dest}.
This form behaves as if \textit{dest}
issues |\childdocof{|\textit{main}|}| right away,
and no further \textsf{childdoc} directives will be processed.

%%%%%%%%%%%%%%%%%%%%%%%%%%%%%%%%%%%%%%%%
\DescribeMacro{\...prefix}
In the alternative form |\childdocforwardprefix|,
%
\begin{center}
\begin{tabular}{l}
|\input{childdoc.def}|\\
|\childdocforwardprefix[|\textit{main}|]{|\textit{prefix}|}{|\textit{dest}|}|
\end{tabular}
\end{center}
%
the destination file is determined by a pattern
depending on the current file:
To make this work, the current file must be called
`{\textit{prefix}\hspace{0.2em}\textit{suffix}}'
with \textit{prefix} matching precisely the argument.
Processing is then passed on to the file
`{\textit{dest}\hspace{0.2em}\textit{suffix}}'.
Surely, the same effect is achieved by
directly specifying the
argument `{\textit{dest}\hspace{0.2em}\textit{suffix}}'
in the first form.
However, that requires to set up a different file
for each child. With the alternative form of the command
all these files can have exactly the same content
which simplifies setting them up and maintaining them.

For example, the following file |draft.tex|
with a compilation flag |\version| as described in \secref{sec:flags}
compiles the main document as a draft:
%
\begin{center}
\begin{tabular}{l}
|\def\version{draft}|\\
|\input{childdoc.def}|\\
|\childdocforward{|\textit{main}|}|
\end{tabular}
\end{center}
%
Likewise, the following files |final|\textit{nn}|.tex|
compile the final version of the child document
|child|\textit{nn}|.tex|:
%
\begin{center}
\begin{tabular}{l}
|\def\version{final}|\\
|\input{childdoc.def}|\\
|\childdocforwardprefix{final}{child}|
\end{tabular}
\end{center}
%

Note that when several versions of a main file and/or of each child file
are to be generated, it may be convenient to set up a |Makefile| or
shell script to automatise the process.

%%%%%%%%%%%%%%%%%%%%%%%%%%%%%%%%%%%%%%%%%%%%%%%%%%%%%%%%%%%%%%%%%%%%%%%%%%%%%%%%
\subsection{Command Line Processing}
\label{sec:commandline}

The effect of redirection files can also be achieved by invoking
the \LaTeX{} compiler with a more elaborate command line.
Most conveniently this should be done as part
of a shell script or a |Makefile|.

When using \textsf{childdoc} in the main file, the following
command lines effectively perform a redirection
(note that depending on the shell being used,
backslashes may have to be doubled: `|\|' $\to$ `|\\|'):
%
\begin{center}
|... -jobname "|\textit{target}|" |\\|"|[\textit{flags}]%
|\input{childdoc.def}\childdocforward[|\textit{main}|]{|\textit{dest}|}"|
\end{center}
%
Here \textit{target} is the name of the output file,
\textit{main} is the name of the main file
and \textit{dest} is the name of the main or child file to be processed
(all filenames without extensions).
The optional argument \textit{main} can be omitted
if \textit{main} matches \textit{dest}.
Optionally, compilation \textit{flags} can be defined via |\def| commands.
This command line makes the \TeX{} engine believe
it is compiling the file \textit{target}
whose content is specified as the latter parameter.
The provided code then forwards the processing to
\textit{main} or \textit{dest} as described in \secref{sec:forward}.

%%%%%%%%%%%%%%%%%%%%%%%%%%%%%%%%%%%%%%%%%%%%%%%%%%%%%%%%%%%%%%%%%%%%%%%%%%%%%%%%
\subsection{Include by Input}
\label{sec:input}

Including child documents by |\include| has some restrictions by design.
Most notably, the content of a child document always occupies
its own set of pages; pages cannot be shared between child documents.
Usually, this behaviour makes perfect sense
because each child document contain an essential part of the document.
However, in some situations it may be desirable to compose
a document from a collection of parts
without having mandatory page breaks between then.
For this case, the package
provides a mechanism to include parts
by |\input| which can also be processed individually.
However, by construction this mechanism
requires manual handling of the content to be output.

%%%%%%%%%%%%%%%%%%%%%%%%%%%%%%%%%%%%%%%%
\DescribeMacro{\ifchilddocmanual}
The main file should be prepared as usual, see \secref{sec:include}.
However, the document body must make a distinction
between processing of an individual part and of the main document, e.g.:
%
\begin{center}
\begin{tabular}{l}
|\ifchilddocmanual|\\
|\input{\childdocname}|\\
|\||else|\\
\textit{document body with }|\input{|\textit{part}|}|\\
|\||fi|
\end{tabular}
\end{center}
%
The conditional |\ifchilddocmanual| is true whenever
a part to be included by |\input| is being compiled,
and the name of the part is stored in |\childdocname|.

%%%%%%%%%%%%%%%%%%%%%%%%%%%%%%%%%%%%%%%%
\DescribeMacro{\childdocby}
Each part to be included by |\input| should start with:
%
\begin{center}
\begin{tabular}{l}
|\input{childdoc.def}|\\
|\childdocby{|\textit{main}|}|\\
\end{tabular}
\end{center}
%
The directive |\childdocby| is similar to |\childdocof|
described in \secref{sec:include},
but the subsequent selection of content must be done manually.
To that end, both |\ifchilddoc| and |\ifchilddocmanual|
will be true upon processing of a part,
and the name of the part is stored in |\childdocname|.
Note that |\jobname| will be set to the filename of the current part
so that each part receives an individual |.aux| file
that does not interfere with the |.aux| file(s) of the main document.
This behaviour can be altered by the alternative form
|\childdocby[*]{|\textit{main}|}| (with a non-empty optional argument)
which uses the |.aux| file of the main document
by setting |\jobname| to \textit{main}.

%%%%%%%%%%%%%%%%%%%%%%%%%%%%%%%%%%%%%%%%%%%%%%%%%%%%%%%%%%%%%%%%%%%%%%%%%%%%%%%%
\subsection{Driver Development}
\label{sec:driver}

The \textsf{childdoc} mechanism can also be use for the development
of definition files such as \LaTeX{} styles or classes.
This case differs from the above setup with multiple parts
included by |\include| in that no |\includeonly| should be invoked.
This can be achieved by starting the include file
(before |\ProvidesPackage|) with:
%
\begin{center}
\begin{tabular}{l}
|\input{childdoc.def}|\\
|\childdocforward{|\textit{main}|}|\\
\end{tabular}
\end{center}
%
or alternatively with:
%
\begin{center}
\begin{tabular}{l}
|\input{childdoc.def}|\\
|\childdocby{|\textit{main}|}|\\
\end{tabular}
\end{center}
%
Both forms have slightly different effects as described above.
The main file is prepared as usual, see \secref{sec:include}.

%%%%%%%%%%%%%%%%%%%%%%%%%%%%%%%%%%%%%%%%%%%%%%%%%%%%%%%%%%%%%%%%%%%%%%%%%%%%%%%%
\subsection{Legacy Detection}
\label{sec:detection}

The directive |\childdocmain| in the main file can detect
whether the complete document or merely a child is to be compiled
even without using the directive |\childdocof|.
This method is deprecated because it is less robust
and there is no compelling reason to use it;
it is merely provided for backward compatibility
and it may be removed in future versions.

If the detection mechanism is to be used,
it is mandatory to correctly specify
the filename of the main file as the argument of |\childdocmain|:
%
\begin{center}
\begin{tabular}{l}
|\input{childdoc.def}|\\
|\childdocmain{|\textit{main}|}|\\
\end{tabular}
\end{center}
%
If |\jobname| does not match the argument \textit{main} of |\childdocmain|,
it is assumed that |\jobname| points to the child file to be compiled.
When using |\childdocmain| with the main file specified as argument,
it suffices to start a child file
with just |\input{|\textit{main}|}|
without loading of the package and using |\childdocof|.
If instead all processing is done
with the appropriate \textsf{childdoc} directives,
the argument of \textit{main} of |\childdocmain| can be empty.

An alternative version of the command line processing described
in \secref{sec:commandline} using the detection mechanism reads:
%
\begin{center}
|... -jobname "|\textit{target}|" "|[\textit{flags}]%
[|\def\jobname{|\textit{dest}|}|]|\input{|\textit{main}|}"|
\end{center}

%%%%%%%%%%%%%%%%%%%%%%%%%%%%%%%%%%%%%%%%%%%%%%%%%%%%%%%%%%%%%%%%%%%%%%%%%%%%%%%%
\subsection{Manual Code}
\label{sec:manual}

In case one cannot be certain whether the definitions file |childdoc.def|
is installed on the target \TeX{} distribution
and one prefers not to ship it,
it is conceivable to paste a few relevant commands into the sources.

To that end, drop all statements |\input{childdoc.def}|
and perform the replacements as outlined below.
Instead of |\childdocmain{|\textit{main}|}| add the following code
to the top of the main file:
%
\begin{center}
\begin{tabular}{l}
|\||ifdefined\childdocname\endinput\||fi\newif\ifchilddoc|\\
|\edef\childdocname{\scantokens\expandafter{\jobname\noexpand}}|\\
|\def\childdocmain{|\textit{main}|}\||ifx\childdocmain\childdocname\||else|\\
|\childdoctrue\includeonly{\childdocname}\let\jobname\childdocmain\||fi|\\
\end{tabular}
\end{center}
%
Instead of |\childdocof{|\textit{main}|}| just include the main file
at the top of each child file:
%
\begin{center}
|\input{|\textit{main}|}|
\end{center}
%
A simple redirection |\childdocforward{|\textit{dest}|}| is achieved by:
%
\begin{center}
|\def\jobname{|\textit{dest}|}\input{\jobname}|
\end{center}
%
The redirection with prefix
|\childdocforwardprefix[|\textit{prefix}|]{|\textit{dest}|}|
is accomplished by:
%
\begin{center}
\begin{tabular}{l}
|{\edef\jobname{\scantokens\expandafter{\jobname\noexpand}}|\\
|\def\redirectjob |\textit{prefix}|#1~~~{\gdef\jobname{|\textit{dest}|#1}}|\\
|\expandafter\redirectjob\jobname~~~}\input{\jobname}|
\end{tabular}
\end{center}

In an alternative approach,
child documents can be compiled by a specific command line
without additional code or specific definitions:
%
\begin{center}
|... -jobname "|\textit{target}|" "|[\textit{flags}]%
|\includeonly{|\textit{dest}|}\input{|\textit{main}|}"|
\end{center}
%

%%%%%%%%%%%%%%%%%%%%%%%%%%%%%%%%%%%%%%%%%%%%%%%%%%%%%%%%%%%%%%%%%%%%%%%%%%%%%%%%
%%%%%%%%%%%%%%%%%%%%%%%%%%%%%%%%%%%%%%%%%%%%%%%%%%%%%%%%%%%%%%%%%%%%%%%%%%%%%%%%
\section{Information}

%%%%%%%%%%%%%%%%%%%%%%%%%%%%%%%%%%%%%%%%%%%%%%%%%%%%%%%%%%%%%%%%%%%%%%%%%%%%%%%%
\subsection{Copyright}

Copyright \copyright{} 2017--2018 Niklas Beisert

This work may be distributed and/or modified under the
conditions of the \LaTeX{} Project Public License, either version 1.3
of this license or (at your option) any later version.
The latest version of this license is in
  \url{http://www.latex-project.org/lppl.txt}
and version 1.3 or later is part of all distributions of \LaTeX{}
version 2005/12/01 or later.

This work has the LPPL maintenance status `maintained'.

The Current Maintainer of this work is Niklas Beisert.

This work consists of the files |README.txt|, |childdoc.ins| and |childdoc.dtx|
as well as the derived files |childdoc.def|, |cdocsamp.tex|
with |cdocsch1.tex|, |cdocsch2.tex|, |cdocspt3.tex|, |cdocspt4.tex|,
|cdocsdrf.tex|, |cdocsfn1.tex|, |cdocsfn2.tex|
as well as |childdoc.pdf|.

%%%%%%%%%%%%%%%%%%%%%%%%%%%%%%%%%%%%%%%%%%%%%%%%%%%%%%%%%%%%%%%%%%%%%%%%%%%%%%%%
\subsection{Files and Installation}

The package consists of the files:
%
\begin{center}
\begin{tabular}{ll}
    |README.txt|   & readme file \\
    |childdoc.ins| & installation file \\
    |childdoc.dtx| & source file \\
    |childdoc.def| & definition file \\
    |cdocsamp.tex| & sample main file \\
    |cdocsch1.tex| & sample include file \\
    |cdocsch2.tex| & sample include file \\
    |cdocspt3.tex| & sample part file \\
    |cdocspt4.tex| & sample part file \\
    |cdocsdrf.tex| & sample redirection file \\
    |cdocsfn1.tex| & sample redirection file \\
    |cdocsfn2.tex| & sample redirection file \\
    |childdoc.pdf| & manual
\end{tabular}
\end{center}
%
The distribution consists of the files
|README.txt|, |childdoc.ins| and |childdoc.dtx|.
%
\begin{itemize}
\item
Run (pdf)\LaTeX{} on |childdoc.dtx|
to compile the manual |childdoc.pdf| (this file).
\item
Run \LaTeX{} on |childdoc.ins| to create the definitions file |childdoc.def|
and the sample |cdocsamp.tex| with include files
|cdocsch1.tex|, |cdocsch2.tex|, |cdocspt3.tex|, |cdocspt4.tex|,
|cdocsdrf.tex|, |cdocsfn1.tex|, |cdocsfn2.tex|.
Then copy the file |childdoc.def| to an appropriate directory of your \LaTeX{}
distribution, e.g.\ \textit{texmf-root}|/tex/latex/childdoc|.
\end{itemize}

%%%%%%%%%%%%%%%%%%%%%%%%%%%%%%%%%%%%%%%%%%%%%%%%%%%%%%%%%%%%%%%%%%%%%%%%%%%%%%%%
\subsection{Related CTAN Packages}

There are several other packages which offer a similar functionality:
%
\begin{itemize}
\item
The packages
\href{http://ctan.org/pkg/docmute}{\textsf{docmute}},
\href{http://ctan.org/pkg/includex}{\textsf{includex}} and
\href{http://ctan.org/pkg/standalone}{\textsf{standalone}}
provide commands to include only the document body of
a child file thus allowing both files to be compiled individually.
\item
The packages \href{http://ctan.org/pkg/subdocs}{\textsf{subdocs}}
and \href{http://ctan.org/pkg/subfiles}{\textsf{subfiles}}
provide structures in which the main and child documents can be
encapsulated and allowing them to be compiled individually.
The inclusion mechanism is different from the conventional |\include|.
\item
The package \href{http://ctan.org/pkg/combine}{\textsf{combine}}
is an elaborate solution to combine several documents into one.
\end{itemize}
%
See also the CTAN topic \href{http://ctan.org/topic/subdocs}{\textsf{subdocs}}
for further related packages.
The present package differs from the above solutions in that
a document structure constructed with the conventional |\include| mechanism
just needs two extra commands at the top of every file
such that all constituent files can be compiled individually.

%%%%%%%%%%%%%%%%%%%%%%%%%%%%%%%%%%%%%%%%%%%%%%%%%%%%%%%%%%%%%%%%%%%%%%%%%%%%%%%%
%\subsection{Feature Suggestions}
%
%The following is a list of features which may be useful for future
%versions of this package:
%%
%\begin{itemize}
%\item
%\ldots
%\end{itemize}

%%%%%%%%%%%%%%%%%%%%%%%%%%%%%%%%%%%%%%%%%%%%%%%%%%%%%%%%%%%%%%%%%%%%%%%%%%%%%%%%
\subsection{Revision History}

%%%%%%%%%%%%%%%%%%%%%%%%%%%%%%%%%%%%%%%%
\paragraph{v2.0:} 2018/12/30

\begin{itemize}
\item
immediate forward processing
\item
added |\childdocby| mechanism
\item
manual restructured
\end{itemize}

%%%%%%%%%%%%%%%%%%%%%%%%%%%%%%%%%%%%%%%%
\paragraph{v1.6:} 2018/01/17

\begin{itemize}
\item
application for development of include files
\item
corrections to manual
\end{itemize}

%%%%%%%%%%%%%%%%%%%%%%%%%%%%%%%%%%%%%%%%
\paragraph{v1.5:} 2017/05/21

\begin{itemize}
\item
more complete structuring introduced
\item
|\childdocof| introduced
\item
|\childdoc| renamed to |\childdocmain|
\item
|\childredirect| renamed to |\childdocforward| and |\childdocforwardprefix|
and functionality expanded
\end{itemize}

%%%%%%%%%%%%%%%%%%%%%%%%%%%%%%%%%%%%%%%%
\paragraph{v1.0:} 2017/04/27

\begin{itemize}
\item
manual and install package
\item
first version published on CTAN
\end{itemize}

%%%%%%%%%%%%%%%%%%%%%%%%%%%%%%%%%%%%%%%%
\paragraph{v0.6:} 2017/04/26

\begin{itemize}
\item
redirection mechanism added
\end{itemize}

%%%%%%%%%%%%%%%%%%%%%%%%%%%%%%%%%%%%%%%%
\paragraph{v0.5:} 2017/04/26

\begin{itemize}
\item
functionality in definition file
\end{itemize}


%%%%%%%%%%%%%%%%%%%%%%%%%%%%%%%%%%%%%%%%%%%%%%%%%%%%%%%%%%%%%%%%%%%%%%%%%%%%%%%%
%%%%%%%%%%%%%%%%%%%%%%%%%%%%%%%%%%%%%%%%%%%%%%%%%%%%%%%%%%%%%%%%%%%%%%%%%%%%%%%%
%%%%%%%%%%%%%%%%%%%%%%%%%%%%%%%%%%%%%%%%%%%%%%%%%%%%%%%%%%%%%%%%%%%%%%%%%%%%%%%%
\appendix

\settowidth\MacroIndent{\rmfamily\scriptsize 000\ }

 \DocInput{childdoc.dtx}

\end{document}
%</driver>
% \fi
%
% %%%%%%%%%%%%%%%%%%%%%%%%%%%%%%%%%%%%%%%%%%%%%%%%%%%%%%%%%%%%%%%%%%%%%%%%%%%%%%
% %%%%%%%%%%%%%%%%%%%%%%%%%%%%%%%%%%%%%%%%%%%%%%%%%%%%%%%%%%%%%%%%%%%%%%%%%%%%%%
% \section{Sample}
%\iffalse
%<*samplemain>
%\fi
%
% The following presents a sample document
% with two chapters, two parts, a title page,
% a compile flag as well as three forwarding files to set the flag.
% It consists of eight |.tex| files:
% \begin{center}
% \begin{tabular}{ll}
% |cdocsamp.tex|&main file\\
% |cdocsch1.tex|&include file for chapter 1\\
% |cdocsch2.tex|&include file for chapter 2\\
% |cdocspt3.tex|&include file for part 3\\
% |cdocspt4.tex|&include file for part 4\\
% |cdocsdrf.tex|&forwarding file for main file in draft mode\\
% |cdocsfi1.tex|&forwarding file for final version of chapter 1\\
% |cdocsfi2.tex|&forwarding file for final version of chapter 2\\
% \end{tabular}
% \end{center}
% Each of the eight files can be compiled directly by the \LaTeX{} compiler.
%
% %%%%%%%%%%%%%%%%%%%%%%%%%%%%%%%%%%%%%%
% \paragraph{Main File.}
%
% The main file is called |cdocsamp.tex|.
%
% Load the \textsf{childdoc} definitions and
% declare the filename for the main document:
%    \begin{macrocode}
\input{childdoc.def}
\childdocmain{}
%    \end{macrocode}

% Optional override for |\version| flag:
%    \begin{macrocode}
%%\ifchilddoc\else\providecommand{\version}{draft}\fi
%    \end{macrocode}

% Define the default values for the |\version| flag
% (|final| for the main file and |draft| for childs):
%    \begin{macrocode}
\ifchilddoc
\providecommand{\version}{draft}
\else
\providecommand{\version}{final}
\fi
%    \end{macrocode}

% Load the standard document class:
%    \begin{macrocode}
\documentclass[12pt]{article}
%    \end{macrocode}

% Start the document body:
%    \begin{macrocode}
\begin{document}
%    \end{macrocode}

% Declare a title page.
% Print title, part of document being processed and version flag:
%    \begin{macrocode}
\addtocounter{page}{-1}
\begin{center}
{\LARGE\bfseries{}childdoc example\par}
\vspace{1cm}
\ifchilddoc
\ifchilddocmanual part\else chapter\fi:
`\childdocname' of `\childdocjob'\par
\else
main document: `\childdocjob'\par
\fi
version: \version\par
\end{center}
\newpage
%    \end{macrocode}

% Manually include selected file,
% otherwise process as usual:
%    \begin{macrocode}
\ifchilddocmanual
\section*{part `\childdocname'}
\input{\childdocname}
\else
%    \end{macrocode}

% Include the two chapters:
%    \begin{macrocode}
\include{cdocsch1}
\include{cdocsch2}
%    \end{macrocode}

% Include the two parts unless only chapters should be displayed:
%    \begin{macrocode}
\ifchilddoc\else
\section{part three}
\input{cdocspt3}
\section{part four}
\input{cdocspt4}
\fi
%    \end{macrocode}

% Process as usual until here:
%    \begin{macrocode}
\fi
%    \end{macrocode}

% End of document body:
%    \begin{macrocode}
\end{document}
%    \end{macrocode}
%\iffalse
%</samplemain>
%\fi
%
% %%%%%%%%%%%%%%%%%%%%%%%%%%%%%%%%%%%%%%
% \paragraph{Chapter Include Files.}
%
% The include files are called |cdocsch1.tex| and |cdocsch2.tex|.
%
%\iffalse
%<*samplechap1|samplechap2>
%\fi

% Optional override for |\version| flag:
%    \begin{macrocode}
%%\providecommand{\version}{final}
%    \end{macrocode}

% Include the main document:
%    \begin{macrocode}
\input{childdoc.def}
\childdocof{cdocsamp}
%    \end{macrocode}

%\iffalse
%</samplechap1|samplechap2>
%\fi
%
%\iffalse
%<*samplechap1>
%\fi
% Some text for chapter 1:
%    \begin{macrocode}
\section{one}
some text in chapter one
%    \end{macrocode}

%\iffalse
%</samplechap1>
%\fi
% Some text for chapter 2:
%\iffalse
%<*samplechap2>
%\fi
%    \begin{macrocode}
\section{two}
more text in chapter two
%    \end{macrocode}

%\iffalse
%</samplechap2>
%\fi
%
% %%%%%%%%%%%%%%%%%%%%%%%%%%%%%%%%%%%%%%
% \paragraph{Part Include Files.}
%
% The include files are called |cdocspt3.tex| and |cdocspt4.tex|.
%
%\iffalse
%<*samplepart3|samplepart4>
%\fi

% Optional override for |\version| flag:
%    \begin{macrocode}
%%\providecommand{\version}{final}
%    \end{macrocode}

% Include the main document:
%    \begin{macrocode}
\input{childdoc.def}
\childdocby{cdocsamp}
%    \end{macrocode}

%\iffalse
%</samplepart3|samplepart4>
%\fi
%
%\iffalse
%<*samplepart3>
%\fi
% Some text for part 3:
%    \begin{macrocode}
some text in part three
%    \end{macrocode}

%\iffalse
%</samplepart3>
%\fi
% Some text for part 4:
%\iffalse
%<*samplepart4>
%\fi
%    \begin{macrocode}
more text in part four
%    \end{macrocode}

%\iffalse
%</samplepart4>
%\fi
%
% %%%%%%%%%%%%%%%%%%%%%%%%%%%%%%%%%%%%%%
% \paragraph{Forwarding for a Complete Draft.}
%
% The following forwarding file |cdocsdrf.tex|
% compiles the main document in draft mode:
%\iffalse
%<*sampledraft>
%\fi
%    \begin{macrocode}
\def\version{draft}
\input{childdoc.def}
\childdocforward{cdocsamp}
%    \end{macrocode}

%\iffalse
%</sampledraft>
%\fi
%
% %%%%%%%%%%%%%%%%%%%%%%%%%%%%%%%%%%%%%%
% \paragraph{Forwarding for Final Version of the Chapters.}
%
% The following forwarding files |cdocsfn1.tex| and |cdocsfn2.tex|
% (with identical content)
% compile the final versions of the child documents
% |cdocsch1.tex| and |cdocsch2.tex|, respectively:
%\iffalse
%<*samplefinal>
%\fi
%    \begin{macrocode}
\def\version{final}
\input{childdoc.def}
\childdocforwardprefix[cdocsamp]{cdocsfn}{cdocsch}
%    \end{macrocode}

%\iffalse
%</samplefinal>
%\fi
%
% %%%%%%%%%%%%%%%%%%%%%%%%%%%%%%%%%%%%%%
% \paragraph{Command Line Processing.}
%
% The following three command lines generate the output files
% |cdocscld|, |cdocscl1| and |cdocscl2|
% which should be identical to
% |cdocsdrf|, |cdocsch1| and |cdocsfn2|, respectively:
% \begin{center}
% \begin{tabular}{l}
% |latex -jobname cdocscld \|\\
% |  "\def\version{draft}\input{childdoc.def}\childdocforward{cdocsamp}"|\\
% |latex -jobname cdocscl1 \|\\
% |  "\input{childdoc.def}\childdocforward[cdocsamp]{cdocsch1}"|\\
% |latex -jobname cdocscl2 \|\\
% |  "\def\version{final}\input{childdoc.def}\childdocforward{cdocsch2}"|
% \end{tabular}
% \end{center}
% Note that the trailing backslash on each first line
% merely continues the input to the second line
% (for convenient cut ant paste).
% Furthermore, the command |latex| can be replaced by any
% of its alternative versions such as |pdflatex|.
%
% %%%%%%%%%%%%%%%%%%%%%%%%%%%%%%%%%%%%%%%%%%%%%%%%%%%%%%%%%%%%%%%%%%%%%%%%%%%%%%
% %%%%%%%%%%%%%%%%%%%%%%%%%%%%%%%%%%%%%%%%%%%%%%%%%%%%%%%%%%%%%%%%%%%%%%%%%%%%%%
% \section{Implementation}
%\iffalse
%<*package>
%\fi
%
% This section describes the definitions file |childdoc.def|.

% The definitions cannot be loaded using |\usepackage| or |\RequirePackage|
% which has a mechanism to prevent loading a style file more than once.
% When loading the definitions by means of |\input|
% multiple instances have to be prevented manually:
%\iffalse
%This code needs to be before the `\ProvidesFile' directive
%which is defined at the beginning of this file.
%Therefore it is also placed there and commented out here.
%</package>
%<*discard>
%\fi
%    \begin{macrocode}
\ifdefined\childdocmain\endinput\fi
%    \end{macrocode}
%\iffalse
%</discard>
%<*package>
%\fi
%
% \macro{\ifchilddoc}
% \macro{\ifchilddocmanual}
% The conditional |\ifchilddoc| tells whether a
% child (true) or main (false) document is being compiled.
% The conditional |\ifchilddocmanual| tells whether
% the |\includeonly| mechanism is used (false) or
% the selection of child files must be performed manually (true).
% The definitions initialise to false:
%    \begin{macrocode}
\newif\ifchilddoc
\newif\ifchilddocmanual
%    \end{macrocode}

% \macro{\childdocname}
% \macro{\childdocjob}
% The macro |\childdocname| stores the name of the main document
% to be compiled. The macro |\childdocjob| stores the name of
% the document on which the \LaTeX{} compiler was originally invoked.
% The content of |\jobname| cannot be compared
% to filenames specified in the source due to different catcodes.
% The following code rescans |\jobname|, stores the result
% in |\childdocname| and saves a copy in |\childdocjob|:
%    \begin{macrocode}
\edef\childdocname{\scantokens\expandafter{\jobname\noexpand}}
\let\childdocjob\childdocname
%    \end{macrocode}

% \macro{\childdocdisable}
% The macro |\childdocdisable| prevents the main file
% from being processed more than once.
% At this stage, the main document command |\childdocmain|
% is assumed to be called once again where it should do nothing.
% Any subsequent call to it should prevent
% a secondary processing of the main document
% It overwrites the forwarding commands
% |\childdocof| and |\childdocforward|
% with empty macros to prevent further inclusions of the main document:
%    \begin{macrocode}
\newcommand{\childdocdisable}
{
  \renewcommand{\childdocmain}[1]{\renewcommand{\childdocmain}[1]{\endinput}}
  \renewcommand{\childdocof}[1]{}
  \renewcommand{\childdocby}[2][]{}
  \renewcommand{\childdocforward}[2][]{}
  \renewcommand{\childdocdisable}{}
}
%    \end{macrocode}

% \macro{\childdocmain}
% The macro |\childdocmain| is to be called at the top of the main file
% with nothing or the main filename (without extension) as argument.
% First, it breaks loops.
% If the argument is not empty and does not match |\childdocname|
% (which is set by the first inclusion of |childdoc.def|),
% |\ifchilddoc| is set to true, |\includeonly| is applied to the child file
% and |\jobname| is set to the main file
% (for proper handling of |.aux| files):
%    \begin{macrocode}
\newcommand{\childdocmain}[1]
{
  \childdocdisable\childdocmain{}
  \if?#1?\else
    \begingroup
      \def\childdoctmp{#1}
      \ifx\childdoctmp\childdocname
        \def\childdoctmp{}
      \else
        \def\childdoctmp
        {
          \childdoctrue
          \includeonly{\childdocname}
          \def\childdocjob{#1}
          \def\jobname{#1}
        }
      \fi
      \expandafter
    \endgroup
    \childdoctmp
  \fi
}
%    \end{macrocode}

% \macro{\childdocof}
% The command |\childdocof| redirects
% compilation to the main file |#1|.
%    \begin{macrocode}
\newcommand{\childdocof}[1]
{
  \childdocdisable
  \childdoctrue
  \includeonly{\childdocname}
  \def\jobname{#1}
  \def\childdocjob{#1}
  \input{#1}
}
%    \end{macrocode}

% \macro{\childdocby}
% The command |\childdocby| ....
%    \begin{macrocode}
\newcommand{\childdocby}[2][]
{
  \childdocdisable
  \childdoctrue
  \childdocmanualtrue
  \if?#1?\else
    \def\jobname{#2}
  \fi
  \def\childdocjob{#2}
  \input{#2}
  \endinput
}
%    \end{macrocode}

% \macro{\childdocforward}
% The command |\childdocforward| redirects
% compilation to the main file or
% (if the optional argument is given) a child file.
% Parameters are set as if the main file
% or a child file starting with |\childdocof| was compiled.
% Then compilation is handed over to the main file:
%    \begin{macrocode}
\newcommand{\childdocforward}[2][]
{
  \begingroup
    \if?#1?
      \def\childdoctmp
      {
        \def\childdocname{#2}
        \def\childdocjob{#2}
        \def\jobname{#2}
        \input{#2}
        \endinput
      }
    \else
      \def\childdoctmp
      {
        \childdocdisable
        \def\childdocname{#2}
        \childdoctrue
        \includeonly{#2}
        \def\childdocjob{#1}
        \def\jobname{#1}
        \input{#1}
        \endinput
      }
    \fi
    \expandafter
  \endgroup
  \childdoctmp
}
%    \end{macrocode}

% \macro{\childdocforwardprefix}
% The command |\childdocforwardprefix| redirects
% compilation to the main or a child file by means of a pattern.
% The prefix |#1| in the current filename is replaced by |#2|
% and the suffix of the current filename is kept
% (it is assumed that the filename does not contain the substring `|~~~|'
% which is used as a delimiter).
% Compilation is handed over to the new file by |\childdocforward|:
%    \begin{macrocode}
\newcommand{\childdocforwardprefix}[3][]
{
  \begingroup
    \def\childdocextract #2##1~~~{\def\childdoctmp{\childdocforward[#1]{#3##1}}}
    \expandafter\childdocextract\childdocname~~~
    \expandafter
  \endgroup
  \childdoctmp
}
%    \end{macrocode}

% \macro{\childdoc}
% The deprecated macro |\childdoc| is a legacy version of |\childdocmain|:
%    \begin{macrocode}
\newcommand{\childdoc}{\childdocmain}
%    \end{macrocode}

% \macro{\childdocredirect}
% The deprecated macro |\childdocredirect| is a legacy version
% of |\childdocforward| and |\childdocforwardprefix|:
%    \begin{macrocode}
\newcommand{\childdocredirect}[2][]
{
  \begingroup
    \if?#1?
      \def\childdoctmp{\childdocforward{#2}}
    \else
      \def\childdoctmp{\childdocforwardprefix{#1}{#2}}
    \fi
    \expandafter
  \endgroup
  \childdoctmp
}
%    \end{macrocode}

%\iffalse
%</package>
%\fi
%
\endinput
|\\
|\childdocforward[|\textit{main}|]{|\textit{dest}|}|\\
\end{tabular}
\end{center}
%
The argument \textit{dest} is the destination file
(without extension).
It should be the main file or one of the child files.
Note that further \textsf{childdoc} directives
such as |\childdocof| and |\childdocforward|
in the indicated file will be processed in this form.
The optional argument \textit{main}
passes on directly to the main file \textit{main}
while pretending to compile the child \textit{dest}.
This form behaves as if \textit{dest}
issues |\childdocof{|\textit{main}|}| right away,
and no further \textsf{childdoc} directives will be processed.

%%%%%%%%%%%%%%%%%%%%%%%%%%%%%%%%%%%%%%%%
\DescribeMacro{\...prefix}
In the alternative form |\childdocforwardprefix|,
%
\begin{center}
\begin{tabular}{l}
|% \iffalse
%
% childdoc.dtx Copyright (C) 2017-2018 Niklas Beisert
%
% This work may be distributed and/or modified under the
% conditions of the LaTeX Project Public License, either version 1.3
% of this license or (at your option) any later version.
% The latest version of this license is in
%   http://www.latex-project.org/lppl.txt
% and version 1.3 or later is part of all distributions of LaTeX
% version 2005/12/01 or later.
%
% This work has the LPPL maintenance status `maintained'.
%
% The Current Maintainer of this work is Niklas Beisert.
%
% This work consists of the files childdoc.dtx and childdoc.ins
% and the derived files childdoc.def and cdocsamp.tex with
% cdocsch1.tex, cdocsch2.tex, cdocsdrf.tex, cdocsfn1.tex, cdocsfn2.tex.
%
%<package>\ifdefined\childdocmain\endinput\fi
%<package>\ProvidesFile{childdoc.def}[2018/12/30 v2.0 child document driver]
%<samplemain>\ProvidesFile{cdocsamp.tex}[2018/12/30 v2.0 sample for childdoc]
%<*driver>
%\ProvidesFile{childdoc.drv}[2018/12/30 v2.0 childdoc reference manual file]
\PassOptionsToClass{10pt,a4paper}{article}
\documentclass{ltxdoc}

\usepackage[margin=35mm]{geometry}
\usepackage{hyperref}
\usepackage{hyperxmp}
\usepackage[usenames]{color}

\hypersetup{colorlinks=true}
\hypersetup{pdfstartview=FitH}
\hypersetup{pdfpagemode=UseNone}
\hypersetup{pdfsource={}}
\hypersetup{pdflang={en-UK}}
\hypersetup{pdfcopyright={Copyright 2017-2018 Niklas Beisert.
  This work may be distributed and/or modified under the
  conditions of the LaTeX Project Public License, either version 1.3
  of this license or (at your option) any later version.}}
\hypersetup{pdflicenseurl={http://www.latex-project.org/lppl.txt}}
\hypersetup{pdfcontactaddress={ETH Zurich, ITP, HIT K,
  Wolfgang-Pauli-Strasse 27}}
\hypersetup{pdfcontactpostcode={8093}}
\hypersetup{pdfcontactcity={Zurich}}
\hypersetup{pdfcontactcountry={Switzerland}}
\hypersetup{pdfcontactemail={nbeisert@itp.phys.ethz.ch}}
\hypersetup{pdfcontacturl={http://people.phys.ethz.ch/\xmptilde nbeisert/}}

\newcommand{\secref}[1]{\hyperref[#1]{section \ref*{#1}}}

\parskip1ex
\parindent0pt
\let\olditemize\itemize
\def\itemize{\olditemize\parskip0pt}

\begin{document}

\title{The \textsf{childdoc} Package}
\hypersetup{pdftitle={The childdoc Package}}
\author{Niklas Beisert\\[2ex]
  Institut f\"ur Theoretische Physik\\
  Eidgen\"ossische Technische Hochschule Z\"urich\\
  Wolfgang-Pauli-Strasse 27, 8093 Z\"urich, Switzerland\\[1ex]
  \href{mailto:nbeisert@itp.phys.ethz.ch}
  {\texttt{nbeisert@itp.phys.ethz.ch}}}
\hypersetup{pdfauthor={Niklas Beisert}}
\hypersetup{pdfsubject={Manual for the LaTeX2e Package childdoc}}
\date{30 December 2018, \textsf{v2.0}}
\maketitle

\begin{abstract}\noindent
\textsf{childdoc} is a \LaTeXe{} package
that enables the direct compilation
of document sections included by |\include|
to individual files.
\end{abstract}

\begingroup
\parskip0ex
\tableofcontents
\endgroup

%%%%%%%%%%%%%%%%%%%%%%%%%%%%%%%%%%%%%%%%%%%%%%%%%%%%%%%%%%%%%%%%%%%%%%%%%%%%%%%%
%%%%%%%%%%%%%%%%%%%%%%%%%%%%%%%%%%%%%%%%%%%%%%%%%%%%%%%%%%%%%%%%%%%%%%%%%%%%%%%%
\section{Introduction}

\LaTeX{} provides a mechanism to structure a large document (such as a book)
into a main file and several child files (containing the chapters)
using the |\include| command.
This mechanism is beneficial for documents
which span hundreds of pages in order to
make the source file(s) more manageable.
Moreover, compilation can be restricted to
selected child files by means of the |\includeonly| command.
The latter feature can be used to reduce the compilation time while editing
(this was significantly more useful in the earlier days of \LaTeX{})
or to generate a smaller document which is easier to navigate.
Another application of |\includeonly| is to generate
documents consisting of selected parts of the complete document.

However, there are a few drawbacks of the plain |\include| mechanism:
\begin{itemize}
\item
The child files cannot be compiled on their own,
they can only be compiled via the main file.
A naive editing environment
(such as a text editor with an option
to have the current file processed by \LaTeX)
may require one to switch to the main file before compiling;
attempting to compile the child file produces errors.
\item
The main file must be modified (each time)
to adjust the |\includeonly| command
to the present needs. This easily leaves the main file in a messy state.
\item
The generated document will always carry the filename
of the main document. This is inconvenient if
several child files are to be compiled and
to be kept for distribution.
\end{itemize}

The present package provides a simple interface
to make child files individually compilable by \LaTeX{}.
Compiling a child file then has the same effect as compiling
the main file with an |\includeonly| command
to select the appropriate child.
Moreover the generated document will carry the name of the child
rather than the main file.
This resolves all three above issues.

This feature is meant to make the editing of books,
thesis documents and lecture notes somewhat more convenient.
However, the package can also be used efficiently for
composing a series of documents (such as exercise sheets)
which are typically distributed individually.
It then assists the author in generating the individual documents
(potentially in different versions)
as well as a document containing the collected series.
Another application is in developing style files
or other kinds of included material
where compilation of the style file could redirect
to a sample or test file.

%%%%%%%%%%%%%%%%%%%%%%%%%%%%%%%%%%%%%%%%%%%%%%%%%%%%%%%%%%%%%%%%%%%%%%%%%%%%%%%%
%%%%%%%%%%%%%%%%%%%%%%%%%%%%%%%%%%%%%%%%%%%%%%%%%%%%%%%%%%%%%%%%%%%%%%%%%%%%%%%%
\section{Usage}

First of all, the package \textsf{childdoc} is \emph{not} a standard
\LaTeXe{} |.sty| style file! Therefore it needs to be invoked in
a non-standard way.

%%%%%%%%%%%%%%%%%%%%%%%%%%%%%%%%%%%%%%%%%%%%%%%%%%%%%%%%%%%%%%%%%%%%%%%%%%%%%%%%
\subsection{Included Files}
\label{sec:include}

%%%%%%%%%%%%%%%%%%%%%%%%%%%%%%%%%%%%%%%%
\DescribeMacro{\childdocmain}
To use the package, add the commands
\begin{center}
\begin{tabular}{l}
|\input{childdoc.def}|\\
|\childdocmain{}|\\
\end{tabular}
\end{center}
at the very top of the main \LaTeX{} file,
in particular \emph{before} the |\documentclass| statement!
The argument of |\childdocmain| should be left empty
(but it must be present).

%%%%%%%%%%%%%%%%%%%%%%%%%%%%%%%%%%%%%%%%
\DescribeMacro{\childdocof}
Furthermore, add the commands
\begin{center}
\begin{tabular}{l}
|\input{childdoc.def}|\\
|\childdocof{|\textit{main}|}|\\
\end{tabular}
\end{center}
at the top of every child file \textit{child}
which is included by |\include{|\textit{child}|}|
from within the main file
(or at least for those files to be compiled individually).
The argument \textit{main} must be the filename of the main file.

There are a couple of
considerations in setting up the main and child documents:

%%%%%%%%%%%%%%%%%%%%%%%%%%%%%%%%%%%%%%%%
\paragraph{Restrictions.}

Please note the following restrictions:
\begin{itemize}
\item
|\childdocmain| must be called with one argument \textit{main}
to ensure compatibility with earlier version of the package.
It must either be empty (|\childdocmain{}|)
or precisely match the filename of the main file in which it is specified.
See \secref{sec:detection} for further information.
\item
The filename \textit{main} must be specified without the |.tex| extension.
\item
The filename \textit{main} is case sensitive
(even in case-insensitive file systems)
due to internal string comparison.
\item
The argument \textit{main} should be fully expanded, it cannot be a macro.
\item
Subdirectories and special characters should be avoided in filenames.
\item
The command |\childdocmain{|\textit{main}|}| must be followed by a whitespace.
It should not be followed immediately by another command
or by a comment mark `|%|'.
This is because the \TeX{} parser reads the token immediately following
the argument of |\childdocmain| and puts it
at the beginning of every child section;
however, a white\-space is ignored.
\end{itemize}

%%%%%%%%%%%%%%%%%%%%%%%%%%%%%%%%%%%%%%%%
\paragraph{Content of Main File.}

It is advisable to place all content in the child files included by |\include|.
Any output contained in the main file will appear in all child documents
unless suppressed manually;
it cannot be suppressed automatically by the |\includeonly| directive
and thus should normally be avoided.
A method to include some content in the main file
by means of conditional processing is described in \secref{sec:conditional}.

%%%%%%%%%%%%%%%%%%%%%%%%%%%%%%%%%%%%%%%%
\paragraph{Page Numbering.}

When only a part of the document is compiled,
the appropriate numbering of pages
(as well as other status parameters)
is determined from the |.aux| files.
The latter contain information from previous passes.
However this information needs to propagate through
all intermediate child documents.
Therefore the page numbering in child documents may well
be inconsistent until the complete document is compiled at least once.

A useful (if unconventional) way to always ensure a consistent
page numbering is to restart the numbering in each child document
and denote the pages by `\textit{child}|.|\textit{page}'
where \textit{child} represents the chapter/section number of the child file.
This can be achieved by the command
|\numberwithin{page}{|\textit{child}|}|
of the \textsf{amsmath} package
where \textit{child} can be |chapter| or |section|
depending on the chosen structuring.
Alternatively, one can modify the macro |\thepage| appropriately
and reset the counter |page| at the start of each child file.

%%%%%%%%%%%%%%%%%%%%%%%%%%%%%%%%%%%%%%%%%%%%%%%%%%%%%%%%%%%%%%%%%%%%%%%%%%%%%%%%
\subsection{Conditional Processing}
\label{sec:conditional}

The package provides a mechanism to compile different versions
of a document. To customise the versions further some conditional processing
can come in handy to distinguish which version is being compiled.
The package provides two macros to describe the compilation context:

%%%%%%%%%%%%%%%%%%%%%%%%%%%%%%%%%%%%%%%%
\DescribeMacro{\ifchilddoc}
The conditional |\ifchilddoc| distinguishes between the compilation of
child documents and the main document:
%
\begin{center}
|\ifchilddoc |\textit{child-code}| |[|\||else |\textit{main-code}]| \||fi|
\end{center}

%%%%%%%%%%%%%%%%%%%%%%%%%%%%%%%%%%%%%%%%
\DescribeMacro{\childdocname}
\DescribeMacro{\childdocjob}
The macro |\childdocname| contains the filename (without extension)
of the main or child file being processed.
Note that |\childdocjob| will always contain the name of the main file.

%%%%%%%%%%%%%%%%%%%%%%%%%%%%%%%%%%%%%%%%
\paragraph{Title Page.}

Conditional processing can be used to include a title or banner page
in the main document when proper precautions are taken.
Importantly, the code in the main file should ensure that the page counter
(as well as other status parameters which are stored in the |.aux| files)
takes the same value after the conditional processing.
Otherwise the page numbers may take divergent values
depending on which part is compiled.

For example, a title page could be declared by:
%
\begin{center}
\begin{tabular}{l}
|\ifchilddoc\||else|\\
|\addtocounter{page}{-1}|\\
\textit{code for title page}\\
|\newpage|\\
|\||fi|
\end{tabular}
\end{center}
%
A banner page for the child documents can be generated by:
%
\begin{center}
\begin{tabular}{l}
|\ifchilddoc|\\
|\addtocounter{page}{-1}|\\
\textit{code for banner page}\\
|\newpage|\\
|\||fi|
\end{tabular}
\end{center}
%
Here one could write a message such as:
\begin{center}
|This is the part \childdocname{} of \childdocjob{}.|
\end{center}

%%%%%%%%%%%%%%%%%%%%%%%%%%%%%%%%%%%%%%%%%%%%%%%%%%%%%%%%%%%%%%%%%%%%%%%%%%%%%%%%
\subsection{Flags}
\label{sec:flags}

The package makes it easy to generate different versions
of the main or child documents.
To this end compilation flags can be defined
and assigned different default values.
They will be particularly useful in conjunction
with the forwarding mechanism described in \secref{sec:forward}.

For example, it may be useful to have a flag |\version|
which can be set to |draft| or |final|.
The document source will contain some conditional code
depending on the value of |\version|.
Suppose further, the flag should default to |final| for the main file
and to |draft| for child files
which is a natural assignment for editing the document.
This is achieved by placing the following code
in the preamble of the main document
(below the |\childdocmain| directive):
%
\begin{center}
\begin{tabular}{l}
|\ifchilddoc|\\
|\providecommand{\version}{draft}|\\
|\||else|\\
|\providecommand{\version}{final}|\\
|\||fi|
\end{tabular}
\end{center}
%
The definition by |\providecommand| makes sure
that previous definitions are not overwritten.
Further statements |\providecommand{\version}{...}|
can thus be added before the above code to override it.

For the main file, one might add a line
(between |\childdocmain| and the above block)
%
\begin{center}
|%\ifchilddoc\||else\providecommand{\version}{draft}\||fi|
\end{center}
%
which can be uncommented to produce a draft version.
Likewise one can add a line to the very top of a child file
(above the |\childdocof{|\textit{main}|}| directive)
%
\begin{center}
|%\providecommand{\version}{final}|
\end{center}
%
which can be uncommented to produce the final version of this child document.

%%%%%%%%%%%%%%%%%%%%%%%%%%%%%%%%%%%%%%%%%%%%%%%%%%%%%%%%%%%%%%%%%%%%%%%%%%%%%%%%
\subsection{Forwarding}
\label{sec:forward}

Different versions of the main or child documents
using compilation flags as described in \secref{sec:flags}
can be (permanently) stored in different files
for convenient compilation, viewing and distribution.
To this end, the package defines a command
to pass on compilation to a different file:

%%%%%%%%%%%%%%%%%%%%%%%%%%%%%%%%%%%%%%%%
\DescribeMacro{\childdocforward}
The command |\childdocforward| redirects processing to
another source file:
%
\begin{center}
\begin{tabular}{l}
|\input{childdoc.def}|\\
|\childdocforward[|\textit{main}|]{|\textit{dest}|}|\\
\end{tabular}
\end{center}
%
The argument \textit{dest} is the destination file
(without extension).
It should be the main file or one of the child files.
Note that further \textsf{childdoc} directives
such as |\childdocof| and |\childdocforward|
in the indicated file will be processed in this form.
The optional argument \textit{main}
passes on directly to the main file \textit{main}
while pretending to compile the child \textit{dest}.
This form behaves as if \textit{dest}
issues |\childdocof{|\textit{main}|}| right away,
and no further \textsf{childdoc} directives will be processed.

%%%%%%%%%%%%%%%%%%%%%%%%%%%%%%%%%%%%%%%%
\DescribeMacro{\...prefix}
In the alternative form |\childdocforwardprefix|,
%
\begin{center}
\begin{tabular}{l}
|\input{childdoc.def}|\\
|\childdocforwardprefix[|\textit{main}|]{|\textit{prefix}|}{|\textit{dest}|}|
\end{tabular}
\end{center}
%
the destination file is determined by a pattern
depending on the current file:
To make this work, the current file must be called
`{\textit{prefix}\hspace{0.2em}\textit{suffix}}'
with \textit{prefix} matching precisely the argument.
Processing is then passed on to the file
`{\textit{dest}\hspace{0.2em}\textit{suffix}}'.
Surely, the same effect is achieved by
directly specifying the
argument `{\textit{dest}\hspace{0.2em}\textit{suffix}}'
in the first form.
However, that requires to set up a different file
for each child. With the alternative form of the command
all these files can have exactly the same content
which simplifies setting them up and maintaining them.

For example, the following file |draft.tex|
with a compilation flag |\version| as described in \secref{sec:flags}
compiles the main document as a draft:
%
\begin{center}
\begin{tabular}{l}
|\def\version{draft}|\\
|\input{childdoc.def}|\\
|\childdocforward{|\textit{main}|}|
\end{tabular}
\end{center}
%
Likewise, the following files |final|\textit{nn}|.tex|
compile the final version of the child document
|child|\textit{nn}|.tex|:
%
\begin{center}
\begin{tabular}{l}
|\def\version{final}|\\
|\input{childdoc.def}|\\
|\childdocforwardprefix{final}{child}|
\end{tabular}
\end{center}
%

Note that when several versions of a main file and/or of each child file
are to be generated, it may be convenient to set up a |Makefile| or
shell script to automatise the process.

%%%%%%%%%%%%%%%%%%%%%%%%%%%%%%%%%%%%%%%%%%%%%%%%%%%%%%%%%%%%%%%%%%%%%%%%%%%%%%%%
\subsection{Command Line Processing}
\label{sec:commandline}

The effect of redirection files can also be achieved by invoking
the \LaTeX{} compiler with a more elaborate command line.
Most conveniently this should be done as part
of a shell script or a |Makefile|.

When using \textsf{childdoc} in the main file, the following
command lines effectively perform a redirection
(note that depending on the shell being used,
backslashes may have to be doubled: `|\|' $\to$ `|\\|'):
%
\begin{center}
|... -jobname "|\textit{target}|" |\\|"|[\textit{flags}]%
|\input{childdoc.def}\childdocforward[|\textit{main}|]{|\textit{dest}|}"|
\end{center}
%
Here \textit{target} is the name of the output file,
\textit{main} is the name of the main file
and \textit{dest} is the name of the main or child file to be processed
(all filenames without extensions).
The optional argument \textit{main} can be omitted
if \textit{main} matches \textit{dest}.
Optionally, compilation \textit{flags} can be defined via |\def| commands.
This command line makes the \TeX{} engine believe
it is compiling the file \textit{target}
whose content is specified as the latter parameter.
The provided code then forwards the processing to
\textit{main} or \textit{dest} as described in \secref{sec:forward}.

%%%%%%%%%%%%%%%%%%%%%%%%%%%%%%%%%%%%%%%%%%%%%%%%%%%%%%%%%%%%%%%%%%%%%%%%%%%%%%%%
\subsection{Include by Input}
\label{sec:input}

Including child documents by |\include| has some restrictions by design.
Most notably, the content of a child document always occupies
its own set of pages; pages cannot be shared between child documents.
Usually, this behaviour makes perfect sense
because each child document contain an essential part of the document.
However, in some situations it may be desirable to compose
a document from a collection of parts
without having mandatory page breaks between then.
For this case, the package
provides a mechanism to include parts
by |\input| which can also be processed individually.
However, by construction this mechanism
requires manual handling of the content to be output.

%%%%%%%%%%%%%%%%%%%%%%%%%%%%%%%%%%%%%%%%
\DescribeMacro{\ifchilddocmanual}
The main file should be prepared as usual, see \secref{sec:include}.
However, the document body must make a distinction
between processing of an individual part and of the main document, e.g.:
%
\begin{center}
\begin{tabular}{l}
|\ifchilddocmanual|\\
|\input{\childdocname}|\\
|\||else|\\
\textit{document body with }|\input{|\textit{part}|}|\\
|\||fi|
\end{tabular}
\end{center}
%
The conditional |\ifchilddocmanual| is true whenever
a part to be included by |\input| is being compiled,
and the name of the part is stored in |\childdocname|.

%%%%%%%%%%%%%%%%%%%%%%%%%%%%%%%%%%%%%%%%
\DescribeMacro{\childdocby}
Each part to be included by |\input| should start with:
%
\begin{center}
\begin{tabular}{l}
|\input{childdoc.def}|\\
|\childdocby{|\textit{main}|}|\\
\end{tabular}
\end{center}
%
The directive |\childdocby| is similar to |\childdocof|
described in \secref{sec:include},
but the subsequent selection of content must be done manually.
To that end, both |\ifchilddoc| and |\ifchilddocmanual|
will be true upon processing of a part,
and the name of the part is stored in |\childdocname|.
Note that |\jobname| will be set to the filename of the current part
so that each part receives an individual |.aux| file
that does not interfere with the |.aux| file(s) of the main document.
This behaviour can be altered by the alternative form
|\childdocby[*]{|\textit{main}|}| (with a non-empty optional argument)
which uses the |.aux| file of the main document
by setting |\jobname| to \textit{main}.

%%%%%%%%%%%%%%%%%%%%%%%%%%%%%%%%%%%%%%%%%%%%%%%%%%%%%%%%%%%%%%%%%%%%%%%%%%%%%%%%
\subsection{Driver Development}
\label{sec:driver}

The \textsf{childdoc} mechanism can also be use for the development
of definition files such as \LaTeX{} styles or classes.
This case differs from the above setup with multiple parts
included by |\include| in that no |\includeonly| should be invoked.
This can be achieved by starting the include file
(before |\ProvidesPackage|) with:
%
\begin{center}
\begin{tabular}{l}
|\input{childdoc.def}|\\
|\childdocforward{|\textit{main}|}|\\
\end{tabular}
\end{center}
%
or alternatively with:
%
\begin{center}
\begin{tabular}{l}
|\input{childdoc.def}|\\
|\childdocby{|\textit{main}|}|\\
\end{tabular}
\end{center}
%
Both forms have slightly different effects as described above.
The main file is prepared as usual, see \secref{sec:include}.

%%%%%%%%%%%%%%%%%%%%%%%%%%%%%%%%%%%%%%%%%%%%%%%%%%%%%%%%%%%%%%%%%%%%%%%%%%%%%%%%
\subsection{Legacy Detection}
\label{sec:detection}

The directive |\childdocmain| in the main file can detect
whether the complete document or merely a child is to be compiled
even without using the directive |\childdocof|.
This method is deprecated because it is less robust
and there is no compelling reason to use it;
it is merely provided for backward compatibility
and it may be removed in future versions.

If the detection mechanism is to be used,
it is mandatory to correctly specify
the filename of the main file as the argument of |\childdocmain|:
%
\begin{center}
\begin{tabular}{l}
|\input{childdoc.def}|\\
|\childdocmain{|\textit{main}|}|\\
\end{tabular}
\end{center}
%
If |\jobname| does not match the argument \textit{main} of |\childdocmain|,
it is assumed that |\jobname| points to the child file to be compiled.
When using |\childdocmain| with the main file specified as argument,
it suffices to start a child file
with just |\input{|\textit{main}|}|
without loading of the package and using |\childdocof|.
If instead all processing is done
with the appropriate \textsf{childdoc} directives,
the argument of \textit{main} of |\childdocmain| can be empty.

An alternative version of the command line processing described
in \secref{sec:commandline} using the detection mechanism reads:
%
\begin{center}
|... -jobname "|\textit{target}|" "|[\textit{flags}]%
[|\def\jobname{|\textit{dest}|}|]|\input{|\textit{main}|}"|
\end{center}

%%%%%%%%%%%%%%%%%%%%%%%%%%%%%%%%%%%%%%%%%%%%%%%%%%%%%%%%%%%%%%%%%%%%%%%%%%%%%%%%
\subsection{Manual Code}
\label{sec:manual}

In case one cannot be certain whether the definitions file |childdoc.def|
is installed on the target \TeX{} distribution
and one prefers not to ship it,
it is conceivable to paste a few relevant commands into the sources.

To that end, drop all statements |\input{childdoc.def}|
and perform the replacements as outlined below.
Instead of |\childdocmain{|\textit{main}|}| add the following code
to the top of the main file:
%
\begin{center}
\begin{tabular}{l}
|\||ifdefined\childdocname\endinput\||fi\newif\ifchilddoc|\\
|\edef\childdocname{\scantokens\expandafter{\jobname\noexpand}}|\\
|\def\childdocmain{|\textit{main}|}\||ifx\childdocmain\childdocname\||else|\\
|\childdoctrue\includeonly{\childdocname}\let\jobname\childdocmain\||fi|\\
\end{tabular}
\end{center}
%
Instead of |\childdocof{|\textit{main}|}| just include the main file
at the top of each child file:
%
\begin{center}
|\input{|\textit{main}|}|
\end{center}
%
A simple redirection |\childdocforward{|\textit{dest}|}| is achieved by:
%
\begin{center}
|\def\jobname{|\textit{dest}|}\input{\jobname}|
\end{center}
%
The redirection with prefix
|\childdocforwardprefix[|\textit{prefix}|]{|\textit{dest}|}|
is accomplished by:
%
\begin{center}
\begin{tabular}{l}
|{\edef\jobname{\scantokens\expandafter{\jobname\noexpand}}|\\
|\def\redirectjob |\textit{prefix}|#1~~~{\gdef\jobname{|\textit{dest}|#1}}|\\
|\expandafter\redirectjob\jobname~~~}\input{\jobname}|
\end{tabular}
\end{center}

In an alternative approach,
child documents can be compiled by a specific command line
without additional code or specific definitions:
%
\begin{center}
|... -jobname "|\textit{target}|" "|[\textit{flags}]%
|\includeonly{|\textit{dest}|}\input{|\textit{main}|}"|
\end{center}
%

%%%%%%%%%%%%%%%%%%%%%%%%%%%%%%%%%%%%%%%%%%%%%%%%%%%%%%%%%%%%%%%%%%%%%%%%%%%%%%%%
%%%%%%%%%%%%%%%%%%%%%%%%%%%%%%%%%%%%%%%%%%%%%%%%%%%%%%%%%%%%%%%%%%%%%%%%%%%%%%%%
\section{Information}

%%%%%%%%%%%%%%%%%%%%%%%%%%%%%%%%%%%%%%%%%%%%%%%%%%%%%%%%%%%%%%%%%%%%%%%%%%%%%%%%
\subsection{Copyright}

Copyright \copyright{} 2017--2018 Niklas Beisert

This work may be distributed and/or modified under the
conditions of the \LaTeX{} Project Public License, either version 1.3
of this license or (at your option) any later version.
The latest version of this license is in
  \url{http://www.latex-project.org/lppl.txt}
and version 1.3 or later is part of all distributions of \LaTeX{}
version 2005/12/01 or later.

This work has the LPPL maintenance status `maintained'.

The Current Maintainer of this work is Niklas Beisert.

This work consists of the files |README.txt|, |childdoc.ins| and |childdoc.dtx|
as well as the derived files |childdoc.def|, |cdocsamp.tex|
with |cdocsch1.tex|, |cdocsch2.tex|, |cdocspt3.tex|, |cdocspt4.tex|,
|cdocsdrf.tex|, |cdocsfn1.tex|, |cdocsfn2.tex|
as well as |childdoc.pdf|.

%%%%%%%%%%%%%%%%%%%%%%%%%%%%%%%%%%%%%%%%%%%%%%%%%%%%%%%%%%%%%%%%%%%%%%%%%%%%%%%%
\subsection{Files and Installation}

The package consists of the files:
%
\begin{center}
\begin{tabular}{ll}
    |README.txt|   & readme file \\
    |childdoc.ins| & installation file \\
    |childdoc.dtx| & source file \\
    |childdoc.def| & definition file \\
    |cdocsamp.tex| & sample main file \\
    |cdocsch1.tex| & sample include file \\
    |cdocsch2.tex| & sample include file \\
    |cdocspt3.tex| & sample part file \\
    |cdocspt4.tex| & sample part file \\
    |cdocsdrf.tex| & sample redirection file \\
    |cdocsfn1.tex| & sample redirection file \\
    |cdocsfn2.tex| & sample redirection file \\
    |childdoc.pdf| & manual
\end{tabular}
\end{center}
%
The distribution consists of the files
|README.txt|, |childdoc.ins| and |childdoc.dtx|.
%
\begin{itemize}
\item
Run (pdf)\LaTeX{} on |childdoc.dtx|
to compile the manual |childdoc.pdf| (this file).
\item
Run \LaTeX{} on |childdoc.ins| to create the definitions file |childdoc.def|
and the sample |cdocsamp.tex| with include files
|cdocsch1.tex|, |cdocsch2.tex|, |cdocspt3.tex|, |cdocspt4.tex|,
|cdocsdrf.tex|, |cdocsfn1.tex|, |cdocsfn2.tex|.
Then copy the file |childdoc.def| to an appropriate directory of your \LaTeX{}
distribution, e.g.\ \textit{texmf-root}|/tex/latex/childdoc|.
\end{itemize}

%%%%%%%%%%%%%%%%%%%%%%%%%%%%%%%%%%%%%%%%%%%%%%%%%%%%%%%%%%%%%%%%%%%%%%%%%%%%%%%%
\subsection{Related CTAN Packages}

There are several other packages which offer a similar functionality:
%
\begin{itemize}
\item
The packages
\href{http://ctan.org/pkg/docmute}{\textsf{docmute}},
\href{http://ctan.org/pkg/includex}{\textsf{includex}} and
\href{http://ctan.org/pkg/standalone}{\textsf{standalone}}
provide commands to include only the document body of
a child file thus allowing both files to be compiled individually.
\item
The packages \href{http://ctan.org/pkg/subdocs}{\textsf{subdocs}}
and \href{http://ctan.org/pkg/subfiles}{\textsf{subfiles}}
provide structures in which the main and child documents can be
encapsulated and allowing them to be compiled individually.
The inclusion mechanism is different from the conventional |\include|.
\item
The package \href{http://ctan.org/pkg/combine}{\textsf{combine}}
is an elaborate solution to combine several documents into one.
\end{itemize}
%
See also the CTAN topic \href{http://ctan.org/topic/subdocs}{\textsf{subdocs}}
for further related packages.
The present package differs from the above solutions in that
a document structure constructed with the conventional |\include| mechanism
just needs two extra commands at the top of every file
such that all constituent files can be compiled individually.

%%%%%%%%%%%%%%%%%%%%%%%%%%%%%%%%%%%%%%%%%%%%%%%%%%%%%%%%%%%%%%%%%%%%%%%%%%%%%%%%
%\subsection{Feature Suggestions}
%
%The following is a list of features which may be useful for future
%versions of this package:
%%
%\begin{itemize}
%\item
%\ldots
%\end{itemize}

%%%%%%%%%%%%%%%%%%%%%%%%%%%%%%%%%%%%%%%%%%%%%%%%%%%%%%%%%%%%%%%%%%%%%%%%%%%%%%%%
\subsection{Revision History}

%%%%%%%%%%%%%%%%%%%%%%%%%%%%%%%%%%%%%%%%
\paragraph{v2.0:} 2018/12/30

\begin{itemize}
\item
immediate forward processing
\item
added |\childdocby| mechanism
\item
manual restructured
\end{itemize}

%%%%%%%%%%%%%%%%%%%%%%%%%%%%%%%%%%%%%%%%
\paragraph{v1.6:} 2018/01/17

\begin{itemize}
\item
application for development of include files
\item
corrections to manual
\end{itemize}

%%%%%%%%%%%%%%%%%%%%%%%%%%%%%%%%%%%%%%%%
\paragraph{v1.5:} 2017/05/21

\begin{itemize}
\item
more complete structuring introduced
\item
|\childdocof| introduced
\item
|\childdoc| renamed to |\childdocmain|
\item
|\childredirect| renamed to |\childdocforward| and |\childdocforwardprefix|
and functionality expanded
\end{itemize}

%%%%%%%%%%%%%%%%%%%%%%%%%%%%%%%%%%%%%%%%
\paragraph{v1.0:} 2017/04/27

\begin{itemize}
\item
manual and install package
\item
first version published on CTAN
\end{itemize}

%%%%%%%%%%%%%%%%%%%%%%%%%%%%%%%%%%%%%%%%
\paragraph{v0.6:} 2017/04/26

\begin{itemize}
\item
redirection mechanism added
\end{itemize}

%%%%%%%%%%%%%%%%%%%%%%%%%%%%%%%%%%%%%%%%
\paragraph{v0.5:} 2017/04/26

\begin{itemize}
\item
functionality in definition file
\end{itemize}


%%%%%%%%%%%%%%%%%%%%%%%%%%%%%%%%%%%%%%%%%%%%%%%%%%%%%%%%%%%%%%%%%%%%%%%%%%%%%%%%
%%%%%%%%%%%%%%%%%%%%%%%%%%%%%%%%%%%%%%%%%%%%%%%%%%%%%%%%%%%%%%%%%%%%%%%%%%%%%%%%
%%%%%%%%%%%%%%%%%%%%%%%%%%%%%%%%%%%%%%%%%%%%%%%%%%%%%%%%%%%%%%%%%%%%%%%%%%%%%%%%
\appendix

\settowidth\MacroIndent{\rmfamily\scriptsize 000\ }

 \DocInput{childdoc.dtx}

\end{document}
%</driver>
% \fi
%
% %%%%%%%%%%%%%%%%%%%%%%%%%%%%%%%%%%%%%%%%%%%%%%%%%%%%%%%%%%%%%%%%%%%%%%%%%%%%%%
% %%%%%%%%%%%%%%%%%%%%%%%%%%%%%%%%%%%%%%%%%%%%%%%%%%%%%%%%%%%%%%%%%%%%%%%%%%%%%%
% \section{Sample}
%\iffalse
%<*samplemain>
%\fi
%
% The following presents a sample document
% with two chapters, two parts, a title page,
% a compile flag as well as three forwarding files to set the flag.
% It consists of eight |.tex| files:
% \begin{center}
% \begin{tabular}{ll}
% |cdocsamp.tex|&main file\\
% |cdocsch1.tex|&include file for chapter 1\\
% |cdocsch2.tex|&include file for chapter 2\\
% |cdocspt3.tex|&include file for part 3\\
% |cdocspt4.tex|&include file for part 4\\
% |cdocsdrf.tex|&forwarding file for main file in draft mode\\
% |cdocsfi1.tex|&forwarding file for final version of chapter 1\\
% |cdocsfi2.tex|&forwarding file for final version of chapter 2\\
% \end{tabular}
% \end{center}
% Each of the eight files can be compiled directly by the \LaTeX{} compiler.
%
% %%%%%%%%%%%%%%%%%%%%%%%%%%%%%%%%%%%%%%
% \paragraph{Main File.}
%
% The main file is called |cdocsamp.tex|.
%
% Load the \textsf{childdoc} definitions and
% declare the filename for the main document:
%    \begin{macrocode}
\input{childdoc.def}
\childdocmain{}
%    \end{macrocode}

% Optional override for |\version| flag:
%    \begin{macrocode}
%%\ifchilddoc\else\providecommand{\version}{draft}\fi
%    \end{macrocode}

% Define the default values for the |\version| flag
% (|final| for the main file and |draft| for childs):
%    \begin{macrocode}
\ifchilddoc
\providecommand{\version}{draft}
\else
\providecommand{\version}{final}
\fi
%    \end{macrocode}

% Load the standard document class:
%    \begin{macrocode}
\documentclass[12pt]{article}
%    \end{macrocode}

% Start the document body:
%    \begin{macrocode}
\begin{document}
%    \end{macrocode}

% Declare a title page.
% Print title, part of document being processed and version flag:
%    \begin{macrocode}
\addtocounter{page}{-1}
\begin{center}
{\LARGE\bfseries{}childdoc example\par}
\vspace{1cm}
\ifchilddoc
\ifchilddocmanual part\else chapter\fi:
`\childdocname' of `\childdocjob'\par
\else
main document: `\childdocjob'\par
\fi
version: \version\par
\end{center}
\newpage
%    \end{macrocode}

% Manually include selected file,
% otherwise process as usual:
%    \begin{macrocode}
\ifchilddocmanual
\section*{part `\childdocname'}
\input{\childdocname}
\else
%    \end{macrocode}

% Include the two chapters:
%    \begin{macrocode}
\include{cdocsch1}
\include{cdocsch2}
%    \end{macrocode}

% Include the two parts unless only chapters should be displayed:
%    \begin{macrocode}
\ifchilddoc\else
\section{part three}
\input{cdocspt3}
\section{part four}
\input{cdocspt4}
\fi
%    \end{macrocode}

% Process as usual until here:
%    \begin{macrocode}
\fi
%    \end{macrocode}

% End of document body:
%    \begin{macrocode}
\end{document}
%    \end{macrocode}
%\iffalse
%</samplemain>
%\fi
%
% %%%%%%%%%%%%%%%%%%%%%%%%%%%%%%%%%%%%%%
% \paragraph{Chapter Include Files.}
%
% The include files are called |cdocsch1.tex| and |cdocsch2.tex|.
%
%\iffalse
%<*samplechap1|samplechap2>
%\fi

% Optional override for |\version| flag:
%    \begin{macrocode}
%%\providecommand{\version}{final}
%    \end{macrocode}

% Include the main document:
%    \begin{macrocode}
\input{childdoc.def}
\childdocof{cdocsamp}
%    \end{macrocode}

%\iffalse
%</samplechap1|samplechap2>
%\fi
%
%\iffalse
%<*samplechap1>
%\fi
% Some text for chapter 1:
%    \begin{macrocode}
\section{one}
some text in chapter one
%    \end{macrocode}

%\iffalse
%</samplechap1>
%\fi
% Some text for chapter 2:
%\iffalse
%<*samplechap2>
%\fi
%    \begin{macrocode}
\section{two}
more text in chapter two
%    \end{macrocode}

%\iffalse
%</samplechap2>
%\fi
%
% %%%%%%%%%%%%%%%%%%%%%%%%%%%%%%%%%%%%%%
% \paragraph{Part Include Files.}
%
% The include files are called |cdocspt3.tex| and |cdocspt4.tex|.
%
%\iffalse
%<*samplepart3|samplepart4>
%\fi

% Optional override for |\version| flag:
%    \begin{macrocode}
%%\providecommand{\version}{final}
%    \end{macrocode}

% Include the main document:
%    \begin{macrocode}
\input{childdoc.def}
\childdocby{cdocsamp}
%    \end{macrocode}

%\iffalse
%</samplepart3|samplepart4>
%\fi
%
%\iffalse
%<*samplepart3>
%\fi
% Some text for part 3:
%    \begin{macrocode}
some text in part three
%    \end{macrocode}

%\iffalse
%</samplepart3>
%\fi
% Some text for part 4:
%\iffalse
%<*samplepart4>
%\fi
%    \begin{macrocode}
more text in part four
%    \end{macrocode}

%\iffalse
%</samplepart4>
%\fi
%
% %%%%%%%%%%%%%%%%%%%%%%%%%%%%%%%%%%%%%%
% \paragraph{Forwarding for a Complete Draft.}
%
% The following forwarding file |cdocsdrf.tex|
% compiles the main document in draft mode:
%\iffalse
%<*sampledraft>
%\fi
%    \begin{macrocode}
\def\version{draft}
\input{childdoc.def}
\childdocforward{cdocsamp}
%    \end{macrocode}

%\iffalse
%</sampledraft>
%\fi
%
% %%%%%%%%%%%%%%%%%%%%%%%%%%%%%%%%%%%%%%
% \paragraph{Forwarding for Final Version of the Chapters.}
%
% The following forwarding files |cdocsfn1.tex| and |cdocsfn2.tex|
% (with identical content)
% compile the final versions of the child documents
% |cdocsch1.tex| and |cdocsch2.tex|, respectively:
%\iffalse
%<*samplefinal>
%\fi
%    \begin{macrocode}
\def\version{final}
\input{childdoc.def}
\childdocforwardprefix[cdocsamp]{cdocsfn}{cdocsch}
%    \end{macrocode}

%\iffalse
%</samplefinal>
%\fi
%
% %%%%%%%%%%%%%%%%%%%%%%%%%%%%%%%%%%%%%%
% \paragraph{Command Line Processing.}
%
% The following three command lines generate the output files
% |cdocscld|, |cdocscl1| and |cdocscl2|
% which should be identical to
% |cdocsdrf|, |cdocsch1| and |cdocsfn2|, respectively:
% \begin{center}
% \begin{tabular}{l}
% |latex -jobname cdocscld \|\\
% |  "\def\version{draft}\input{childdoc.def}\childdocforward{cdocsamp}"|\\
% |latex -jobname cdocscl1 \|\\
% |  "\input{childdoc.def}\childdocforward[cdocsamp]{cdocsch1}"|\\
% |latex -jobname cdocscl2 \|\\
% |  "\def\version{final}\input{childdoc.def}\childdocforward{cdocsch2}"|
% \end{tabular}
% \end{center}
% Note that the trailing backslash on each first line
% merely continues the input to the second line
% (for convenient cut ant paste).
% Furthermore, the command |latex| can be replaced by any
% of its alternative versions such as |pdflatex|.
%
% %%%%%%%%%%%%%%%%%%%%%%%%%%%%%%%%%%%%%%%%%%%%%%%%%%%%%%%%%%%%%%%%%%%%%%%%%%%%%%
% %%%%%%%%%%%%%%%%%%%%%%%%%%%%%%%%%%%%%%%%%%%%%%%%%%%%%%%%%%%%%%%%%%%%%%%%%%%%%%
% \section{Implementation}
%\iffalse
%<*package>
%\fi
%
% This section describes the definitions file |childdoc.def|.

% The definitions cannot be loaded using |\usepackage| or |\RequirePackage|
% which has a mechanism to prevent loading a style file more than once.
% When loading the definitions by means of |\input|
% multiple instances have to be prevented manually:
%\iffalse
%This code needs to be before the `\ProvidesFile' directive
%which is defined at the beginning of this file.
%Therefore it is also placed there and commented out here.
%</package>
%<*discard>
%\fi
%    \begin{macrocode}
\ifdefined\childdocmain\endinput\fi
%    \end{macrocode}
%\iffalse
%</discard>
%<*package>
%\fi
%
% \macro{\ifchilddoc}
% \macro{\ifchilddocmanual}
% The conditional |\ifchilddoc| tells whether a
% child (true) or main (false) document is being compiled.
% The conditional |\ifchilddocmanual| tells whether
% the |\includeonly| mechanism is used (false) or
% the selection of child files must be performed manually (true).
% The definitions initialise to false:
%    \begin{macrocode}
\newif\ifchilddoc
\newif\ifchilddocmanual
%    \end{macrocode}

% \macro{\childdocname}
% \macro{\childdocjob}
% The macro |\childdocname| stores the name of the main document
% to be compiled. The macro |\childdocjob| stores the name of
% the document on which the \LaTeX{} compiler was originally invoked.
% The content of |\jobname| cannot be compared
% to filenames specified in the source due to different catcodes.
% The following code rescans |\jobname|, stores the result
% in |\childdocname| and saves a copy in |\childdocjob|:
%    \begin{macrocode}
\edef\childdocname{\scantokens\expandafter{\jobname\noexpand}}
\let\childdocjob\childdocname
%    \end{macrocode}

% \macro{\childdocdisable}
% The macro |\childdocdisable| prevents the main file
% from being processed more than once.
% At this stage, the main document command |\childdocmain|
% is assumed to be called once again where it should do nothing.
% Any subsequent call to it should prevent
% a secondary processing of the main document
% It overwrites the forwarding commands
% |\childdocof| and |\childdocforward|
% with empty macros to prevent further inclusions of the main document:
%    \begin{macrocode}
\newcommand{\childdocdisable}
{
  \renewcommand{\childdocmain}[1]{\renewcommand{\childdocmain}[1]{\endinput}}
  \renewcommand{\childdocof}[1]{}
  \renewcommand{\childdocby}[2][]{}
  \renewcommand{\childdocforward}[2][]{}
  \renewcommand{\childdocdisable}{}
}
%    \end{macrocode}

% \macro{\childdocmain}
% The macro |\childdocmain| is to be called at the top of the main file
% with nothing or the main filename (without extension) as argument.
% First, it breaks loops.
% If the argument is not empty and does not match |\childdocname|
% (which is set by the first inclusion of |childdoc.def|),
% |\ifchilddoc| is set to true, |\includeonly| is applied to the child file
% and |\jobname| is set to the main file
% (for proper handling of |.aux| files):
%    \begin{macrocode}
\newcommand{\childdocmain}[1]
{
  \childdocdisable\childdocmain{}
  \if?#1?\else
    \begingroup
      \def\childdoctmp{#1}
      \ifx\childdoctmp\childdocname
        \def\childdoctmp{}
      \else
        \def\childdoctmp
        {
          \childdoctrue
          \includeonly{\childdocname}
          \def\childdocjob{#1}
          \def\jobname{#1}
        }
      \fi
      \expandafter
    \endgroup
    \childdoctmp
  \fi
}
%    \end{macrocode}

% \macro{\childdocof}
% The command |\childdocof| redirects
% compilation to the main file |#1|.
%    \begin{macrocode}
\newcommand{\childdocof}[1]
{
  \childdocdisable
  \childdoctrue
  \includeonly{\childdocname}
  \def\jobname{#1}
  \def\childdocjob{#1}
  \input{#1}
}
%    \end{macrocode}

% \macro{\childdocby}
% The command |\childdocby| ....
%    \begin{macrocode}
\newcommand{\childdocby}[2][]
{
  \childdocdisable
  \childdoctrue
  \childdocmanualtrue
  \if?#1?\else
    \def\jobname{#2}
  \fi
  \def\childdocjob{#2}
  \input{#2}
  \endinput
}
%    \end{macrocode}

% \macro{\childdocforward}
% The command |\childdocforward| redirects
% compilation to the main file or
% (if the optional argument is given) a child file.
% Parameters are set as if the main file
% or a child file starting with |\childdocof| was compiled.
% Then compilation is handed over to the main file:
%    \begin{macrocode}
\newcommand{\childdocforward}[2][]
{
  \begingroup
    \if?#1?
      \def\childdoctmp
      {
        \def\childdocname{#2}
        \def\childdocjob{#2}
        \def\jobname{#2}
        \input{#2}
        \endinput
      }
    \else
      \def\childdoctmp
      {
        \childdocdisable
        \def\childdocname{#2}
        \childdoctrue
        \includeonly{#2}
        \def\childdocjob{#1}
        \def\jobname{#1}
        \input{#1}
        \endinput
      }
    \fi
    \expandafter
  \endgroup
  \childdoctmp
}
%    \end{macrocode}

% \macro{\childdocforwardprefix}
% The command |\childdocforwardprefix| redirects
% compilation to the main or a child file by means of a pattern.
% The prefix |#1| in the current filename is replaced by |#2|
% and the suffix of the current filename is kept
% (it is assumed that the filename does not contain the substring `|~~~|'
% which is used as a delimiter).
% Compilation is handed over to the new file by |\childdocforward|:
%    \begin{macrocode}
\newcommand{\childdocforwardprefix}[3][]
{
  \begingroup
    \def\childdocextract #2##1~~~{\def\childdoctmp{\childdocforward[#1]{#3##1}}}
    \expandafter\childdocextract\childdocname~~~
    \expandafter
  \endgroup
  \childdoctmp
}
%    \end{macrocode}

% \macro{\childdoc}
% The deprecated macro |\childdoc| is a legacy version of |\childdocmain|:
%    \begin{macrocode}
\newcommand{\childdoc}{\childdocmain}
%    \end{macrocode}

% \macro{\childdocredirect}
% The deprecated macro |\childdocredirect| is a legacy version
% of |\childdocforward| and |\childdocforwardprefix|:
%    \begin{macrocode}
\newcommand{\childdocredirect}[2][]
{
  \begingroup
    \if?#1?
      \def\childdoctmp{\childdocforward{#2}}
    \else
      \def\childdoctmp{\childdocforwardprefix{#1}{#2}}
    \fi
    \expandafter
  \endgroup
  \childdoctmp
}
%    \end{macrocode}

%\iffalse
%</package>
%\fi
%
\endinput
|\\
|\childdocforwardprefix[|\textit{main}|]{|\textit{prefix}|}{|\textit{dest}|}|
\end{tabular}
\end{center}
%
the destination file is determined by a pattern
depending on the current file:
To make this work, the current file must be called
`{\textit{prefix}\hspace{0.2em}\textit{suffix}}'
with \textit{prefix} matching precisely the argument.
Processing is then passed on to the file
`{\textit{dest}\hspace{0.2em}\textit{suffix}}'.
Surely, the same effect is achieved by
directly specifying the
argument `{\textit{dest}\hspace{0.2em}\textit{suffix}}'
in the first form.
However, that requires to set up a different file
for each child. With the alternative form of the command
all these files can have exactly the same content
which simplifies setting them up and maintaining them.

For example, the following file |draft.tex|
with a compilation flag |\version| as described in \secref{sec:flags}
compiles the main document as a draft:
%
\begin{center}
\begin{tabular}{l}
|\def\version{draft}|\\
|% \iffalse
%
% childdoc.dtx Copyright (C) 2017-2018 Niklas Beisert
%
% This work may be distributed and/or modified under the
% conditions of the LaTeX Project Public License, either version 1.3
% of this license or (at your option) any later version.
% The latest version of this license is in
%   http://www.latex-project.org/lppl.txt
% and version 1.3 or later is part of all distributions of LaTeX
% version 2005/12/01 or later.
%
% This work has the LPPL maintenance status `maintained'.
%
% The Current Maintainer of this work is Niklas Beisert.
%
% This work consists of the files childdoc.dtx and childdoc.ins
% and the derived files childdoc.def and cdocsamp.tex with
% cdocsch1.tex, cdocsch2.tex, cdocsdrf.tex, cdocsfn1.tex, cdocsfn2.tex.
%
%<package>\ifdefined\childdocmain\endinput\fi
%<package>\ProvidesFile{childdoc.def}[2018/12/30 v2.0 child document driver]
%<samplemain>\ProvidesFile{cdocsamp.tex}[2018/12/30 v2.0 sample for childdoc]
%<*driver>
%\ProvidesFile{childdoc.drv}[2018/12/30 v2.0 childdoc reference manual file]
\PassOptionsToClass{10pt,a4paper}{article}
\documentclass{ltxdoc}

\usepackage[margin=35mm]{geometry}
\usepackage{hyperref}
\usepackage{hyperxmp}
\usepackage[usenames]{color}

\hypersetup{colorlinks=true}
\hypersetup{pdfstartview=FitH}
\hypersetup{pdfpagemode=UseNone}
\hypersetup{pdfsource={}}
\hypersetup{pdflang={en-UK}}
\hypersetup{pdfcopyright={Copyright 2017-2018 Niklas Beisert.
  This work may be distributed and/or modified under the
  conditions of the LaTeX Project Public License, either version 1.3
  of this license or (at your option) any later version.}}
\hypersetup{pdflicenseurl={http://www.latex-project.org/lppl.txt}}
\hypersetup{pdfcontactaddress={ETH Zurich, ITP, HIT K,
  Wolfgang-Pauli-Strasse 27}}
\hypersetup{pdfcontactpostcode={8093}}
\hypersetup{pdfcontactcity={Zurich}}
\hypersetup{pdfcontactcountry={Switzerland}}
\hypersetup{pdfcontactemail={nbeisert@itp.phys.ethz.ch}}
\hypersetup{pdfcontacturl={http://people.phys.ethz.ch/\xmptilde nbeisert/}}

\newcommand{\secref}[1]{\hyperref[#1]{section \ref*{#1}}}

\parskip1ex
\parindent0pt
\let\olditemize\itemize
\def\itemize{\olditemize\parskip0pt}

\begin{document}

\title{The \textsf{childdoc} Package}
\hypersetup{pdftitle={The childdoc Package}}
\author{Niklas Beisert\\[2ex]
  Institut f\"ur Theoretische Physik\\
  Eidgen\"ossische Technische Hochschule Z\"urich\\
  Wolfgang-Pauli-Strasse 27, 8093 Z\"urich, Switzerland\\[1ex]
  \href{mailto:nbeisert@itp.phys.ethz.ch}
  {\texttt{nbeisert@itp.phys.ethz.ch}}}
\hypersetup{pdfauthor={Niklas Beisert}}
\hypersetup{pdfsubject={Manual for the LaTeX2e Package childdoc}}
\date{30 December 2018, \textsf{v2.0}}
\maketitle

\begin{abstract}\noindent
\textsf{childdoc} is a \LaTeXe{} package
that enables the direct compilation
of document sections included by |\include|
to individual files.
\end{abstract}

\begingroup
\parskip0ex
\tableofcontents
\endgroup

%%%%%%%%%%%%%%%%%%%%%%%%%%%%%%%%%%%%%%%%%%%%%%%%%%%%%%%%%%%%%%%%%%%%%%%%%%%%%%%%
%%%%%%%%%%%%%%%%%%%%%%%%%%%%%%%%%%%%%%%%%%%%%%%%%%%%%%%%%%%%%%%%%%%%%%%%%%%%%%%%
\section{Introduction}

\LaTeX{} provides a mechanism to structure a large document (such as a book)
into a main file and several child files (containing the chapters)
using the |\include| command.
This mechanism is beneficial for documents
which span hundreds of pages in order to
make the source file(s) more manageable.
Moreover, compilation can be restricted to
selected child files by means of the |\includeonly| command.
The latter feature can be used to reduce the compilation time while editing
(this was significantly more useful in the earlier days of \LaTeX{})
or to generate a smaller document which is easier to navigate.
Another application of |\includeonly| is to generate
documents consisting of selected parts of the complete document.

However, there are a few drawbacks of the plain |\include| mechanism:
\begin{itemize}
\item
The child files cannot be compiled on their own,
they can only be compiled via the main file.
A naive editing environment
(such as a text editor with an option
to have the current file processed by \LaTeX)
may require one to switch to the main file before compiling;
attempting to compile the child file produces errors.
\item
The main file must be modified (each time)
to adjust the |\includeonly| command
to the present needs. This easily leaves the main file in a messy state.
\item
The generated document will always carry the filename
of the main document. This is inconvenient if
several child files are to be compiled and
to be kept for distribution.
\end{itemize}

The present package provides a simple interface
to make child files individually compilable by \LaTeX{}.
Compiling a child file then has the same effect as compiling
the main file with an |\includeonly| command
to select the appropriate child.
Moreover the generated document will carry the name of the child
rather than the main file.
This resolves all three above issues.

This feature is meant to make the editing of books,
thesis documents and lecture notes somewhat more convenient.
However, the package can also be used efficiently for
composing a series of documents (such as exercise sheets)
which are typically distributed individually.
It then assists the author in generating the individual documents
(potentially in different versions)
as well as a document containing the collected series.
Another application is in developing style files
or other kinds of included material
where compilation of the style file could redirect
to a sample or test file.

%%%%%%%%%%%%%%%%%%%%%%%%%%%%%%%%%%%%%%%%%%%%%%%%%%%%%%%%%%%%%%%%%%%%%%%%%%%%%%%%
%%%%%%%%%%%%%%%%%%%%%%%%%%%%%%%%%%%%%%%%%%%%%%%%%%%%%%%%%%%%%%%%%%%%%%%%%%%%%%%%
\section{Usage}

First of all, the package \textsf{childdoc} is \emph{not} a standard
\LaTeXe{} |.sty| style file! Therefore it needs to be invoked in
a non-standard way.

%%%%%%%%%%%%%%%%%%%%%%%%%%%%%%%%%%%%%%%%%%%%%%%%%%%%%%%%%%%%%%%%%%%%%%%%%%%%%%%%
\subsection{Included Files}
\label{sec:include}

%%%%%%%%%%%%%%%%%%%%%%%%%%%%%%%%%%%%%%%%
\DescribeMacro{\childdocmain}
To use the package, add the commands
\begin{center}
\begin{tabular}{l}
|\input{childdoc.def}|\\
|\childdocmain{}|\\
\end{tabular}
\end{center}
at the very top of the main \LaTeX{} file,
in particular \emph{before} the |\documentclass| statement!
The argument of |\childdocmain| should be left empty
(but it must be present).

%%%%%%%%%%%%%%%%%%%%%%%%%%%%%%%%%%%%%%%%
\DescribeMacro{\childdocof}
Furthermore, add the commands
\begin{center}
\begin{tabular}{l}
|\input{childdoc.def}|\\
|\childdocof{|\textit{main}|}|\\
\end{tabular}
\end{center}
at the top of every child file \textit{child}
which is included by |\include{|\textit{child}|}|
from within the main file
(or at least for those files to be compiled individually).
The argument \textit{main} must be the filename of the main file.

There are a couple of
considerations in setting up the main and child documents:

%%%%%%%%%%%%%%%%%%%%%%%%%%%%%%%%%%%%%%%%
\paragraph{Restrictions.}

Please note the following restrictions:
\begin{itemize}
\item
|\childdocmain| must be called with one argument \textit{main}
to ensure compatibility with earlier version of the package.
It must either be empty (|\childdocmain{}|)
or precisely match the filename of the main file in which it is specified.
See \secref{sec:detection} for further information.
\item
The filename \textit{main} must be specified without the |.tex| extension.
\item
The filename \textit{main} is case sensitive
(even in case-insensitive file systems)
due to internal string comparison.
\item
The argument \textit{main} should be fully expanded, it cannot be a macro.
\item
Subdirectories and special characters should be avoided in filenames.
\item
The command |\childdocmain{|\textit{main}|}| must be followed by a whitespace.
It should not be followed immediately by another command
or by a comment mark `|%|'.
This is because the \TeX{} parser reads the token immediately following
the argument of |\childdocmain| and puts it
at the beginning of every child section;
however, a white\-space is ignored.
\end{itemize}

%%%%%%%%%%%%%%%%%%%%%%%%%%%%%%%%%%%%%%%%
\paragraph{Content of Main File.}

It is advisable to place all content in the child files included by |\include|.
Any output contained in the main file will appear in all child documents
unless suppressed manually;
it cannot be suppressed automatically by the |\includeonly| directive
and thus should normally be avoided.
A method to include some content in the main file
by means of conditional processing is described in \secref{sec:conditional}.

%%%%%%%%%%%%%%%%%%%%%%%%%%%%%%%%%%%%%%%%
\paragraph{Page Numbering.}

When only a part of the document is compiled,
the appropriate numbering of pages
(as well as other status parameters)
is determined from the |.aux| files.
The latter contain information from previous passes.
However this information needs to propagate through
all intermediate child documents.
Therefore the page numbering in child documents may well
be inconsistent until the complete document is compiled at least once.

A useful (if unconventional) way to always ensure a consistent
page numbering is to restart the numbering in each child document
and denote the pages by `\textit{child}|.|\textit{page}'
where \textit{child} represents the chapter/section number of the child file.
This can be achieved by the command
|\numberwithin{page}{|\textit{child}|}|
of the \textsf{amsmath} package
where \textit{child} can be |chapter| or |section|
depending on the chosen structuring.
Alternatively, one can modify the macro |\thepage| appropriately
and reset the counter |page| at the start of each child file.

%%%%%%%%%%%%%%%%%%%%%%%%%%%%%%%%%%%%%%%%%%%%%%%%%%%%%%%%%%%%%%%%%%%%%%%%%%%%%%%%
\subsection{Conditional Processing}
\label{sec:conditional}

The package provides a mechanism to compile different versions
of a document. To customise the versions further some conditional processing
can come in handy to distinguish which version is being compiled.
The package provides two macros to describe the compilation context:

%%%%%%%%%%%%%%%%%%%%%%%%%%%%%%%%%%%%%%%%
\DescribeMacro{\ifchilddoc}
The conditional |\ifchilddoc| distinguishes between the compilation of
child documents and the main document:
%
\begin{center}
|\ifchilddoc |\textit{child-code}| |[|\||else |\textit{main-code}]| \||fi|
\end{center}

%%%%%%%%%%%%%%%%%%%%%%%%%%%%%%%%%%%%%%%%
\DescribeMacro{\childdocname}
\DescribeMacro{\childdocjob}
The macro |\childdocname| contains the filename (without extension)
of the main or child file being processed.
Note that |\childdocjob| will always contain the name of the main file.

%%%%%%%%%%%%%%%%%%%%%%%%%%%%%%%%%%%%%%%%
\paragraph{Title Page.}

Conditional processing can be used to include a title or banner page
in the main document when proper precautions are taken.
Importantly, the code in the main file should ensure that the page counter
(as well as other status parameters which are stored in the |.aux| files)
takes the same value after the conditional processing.
Otherwise the page numbers may take divergent values
depending on which part is compiled.

For example, a title page could be declared by:
%
\begin{center}
\begin{tabular}{l}
|\ifchilddoc\||else|\\
|\addtocounter{page}{-1}|\\
\textit{code for title page}\\
|\newpage|\\
|\||fi|
\end{tabular}
\end{center}
%
A banner page for the child documents can be generated by:
%
\begin{center}
\begin{tabular}{l}
|\ifchilddoc|\\
|\addtocounter{page}{-1}|\\
\textit{code for banner page}\\
|\newpage|\\
|\||fi|
\end{tabular}
\end{center}
%
Here one could write a message such as:
\begin{center}
|This is the part \childdocname{} of \childdocjob{}.|
\end{center}

%%%%%%%%%%%%%%%%%%%%%%%%%%%%%%%%%%%%%%%%%%%%%%%%%%%%%%%%%%%%%%%%%%%%%%%%%%%%%%%%
\subsection{Flags}
\label{sec:flags}

The package makes it easy to generate different versions
of the main or child documents.
To this end compilation flags can be defined
and assigned different default values.
They will be particularly useful in conjunction
with the forwarding mechanism described in \secref{sec:forward}.

For example, it may be useful to have a flag |\version|
which can be set to |draft| or |final|.
The document source will contain some conditional code
depending on the value of |\version|.
Suppose further, the flag should default to |final| for the main file
and to |draft| for child files
which is a natural assignment for editing the document.
This is achieved by placing the following code
in the preamble of the main document
(below the |\childdocmain| directive):
%
\begin{center}
\begin{tabular}{l}
|\ifchilddoc|\\
|\providecommand{\version}{draft}|\\
|\||else|\\
|\providecommand{\version}{final}|\\
|\||fi|
\end{tabular}
\end{center}
%
The definition by |\providecommand| makes sure
that previous definitions are not overwritten.
Further statements |\providecommand{\version}{...}|
can thus be added before the above code to override it.

For the main file, one might add a line
(between |\childdocmain| and the above block)
%
\begin{center}
|%\ifchilddoc\||else\providecommand{\version}{draft}\||fi|
\end{center}
%
which can be uncommented to produce a draft version.
Likewise one can add a line to the very top of a child file
(above the |\childdocof{|\textit{main}|}| directive)
%
\begin{center}
|%\providecommand{\version}{final}|
\end{center}
%
which can be uncommented to produce the final version of this child document.

%%%%%%%%%%%%%%%%%%%%%%%%%%%%%%%%%%%%%%%%%%%%%%%%%%%%%%%%%%%%%%%%%%%%%%%%%%%%%%%%
\subsection{Forwarding}
\label{sec:forward}

Different versions of the main or child documents
using compilation flags as described in \secref{sec:flags}
can be (permanently) stored in different files
for convenient compilation, viewing and distribution.
To this end, the package defines a command
to pass on compilation to a different file:

%%%%%%%%%%%%%%%%%%%%%%%%%%%%%%%%%%%%%%%%
\DescribeMacro{\childdocforward}
The command |\childdocforward| redirects processing to
another source file:
%
\begin{center}
\begin{tabular}{l}
|\input{childdoc.def}|\\
|\childdocforward[|\textit{main}|]{|\textit{dest}|}|\\
\end{tabular}
\end{center}
%
The argument \textit{dest} is the destination file
(without extension).
It should be the main file or one of the child files.
Note that further \textsf{childdoc} directives
such as |\childdocof| and |\childdocforward|
in the indicated file will be processed in this form.
The optional argument \textit{main}
passes on directly to the main file \textit{main}
while pretending to compile the child \textit{dest}.
This form behaves as if \textit{dest}
issues |\childdocof{|\textit{main}|}| right away,
and no further \textsf{childdoc} directives will be processed.

%%%%%%%%%%%%%%%%%%%%%%%%%%%%%%%%%%%%%%%%
\DescribeMacro{\...prefix}
In the alternative form |\childdocforwardprefix|,
%
\begin{center}
\begin{tabular}{l}
|\input{childdoc.def}|\\
|\childdocforwardprefix[|\textit{main}|]{|\textit{prefix}|}{|\textit{dest}|}|
\end{tabular}
\end{center}
%
the destination file is determined by a pattern
depending on the current file:
To make this work, the current file must be called
`{\textit{prefix}\hspace{0.2em}\textit{suffix}}'
with \textit{prefix} matching precisely the argument.
Processing is then passed on to the file
`{\textit{dest}\hspace{0.2em}\textit{suffix}}'.
Surely, the same effect is achieved by
directly specifying the
argument `{\textit{dest}\hspace{0.2em}\textit{suffix}}'
in the first form.
However, that requires to set up a different file
for each child. With the alternative form of the command
all these files can have exactly the same content
which simplifies setting them up and maintaining them.

For example, the following file |draft.tex|
with a compilation flag |\version| as described in \secref{sec:flags}
compiles the main document as a draft:
%
\begin{center}
\begin{tabular}{l}
|\def\version{draft}|\\
|\input{childdoc.def}|\\
|\childdocforward{|\textit{main}|}|
\end{tabular}
\end{center}
%
Likewise, the following files |final|\textit{nn}|.tex|
compile the final version of the child document
|child|\textit{nn}|.tex|:
%
\begin{center}
\begin{tabular}{l}
|\def\version{final}|\\
|\input{childdoc.def}|\\
|\childdocforwardprefix{final}{child}|
\end{tabular}
\end{center}
%

Note that when several versions of a main file and/or of each child file
are to be generated, it may be convenient to set up a |Makefile| or
shell script to automatise the process.

%%%%%%%%%%%%%%%%%%%%%%%%%%%%%%%%%%%%%%%%%%%%%%%%%%%%%%%%%%%%%%%%%%%%%%%%%%%%%%%%
\subsection{Command Line Processing}
\label{sec:commandline}

The effect of redirection files can also be achieved by invoking
the \LaTeX{} compiler with a more elaborate command line.
Most conveniently this should be done as part
of a shell script or a |Makefile|.

When using \textsf{childdoc} in the main file, the following
command lines effectively perform a redirection
(note that depending on the shell being used,
backslashes may have to be doubled: `|\|' $\to$ `|\\|'):
%
\begin{center}
|... -jobname "|\textit{target}|" |\\|"|[\textit{flags}]%
|\input{childdoc.def}\childdocforward[|\textit{main}|]{|\textit{dest}|}"|
\end{center}
%
Here \textit{target} is the name of the output file,
\textit{main} is the name of the main file
and \textit{dest} is the name of the main or child file to be processed
(all filenames without extensions).
The optional argument \textit{main} can be omitted
if \textit{main} matches \textit{dest}.
Optionally, compilation \textit{flags} can be defined via |\def| commands.
This command line makes the \TeX{} engine believe
it is compiling the file \textit{target}
whose content is specified as the latter parameter.
The provided code then forwards the processing to
\textit{main} or \textit{dest} as described in \secref{sec:forward}.

%%%%%%%%%%%%%%%%%%%%%%%%%%%%%%%%%%%%%%%%%%%%%%%%%%%%%%%%%%%%%%%%%%%%%%%%%%%%%%%%
\subsection{Include by Input}
\label{sec:input}

Including child documents by |\include| has some restrictions by design.
Most notably, the content of a child document always occupies
its own set of pages; pages cannot be shared between child documents.
Usually, this behaviour makes perfect sense
because each child document contain an essential part of the document.
However, in some situations it may be desirable to compose
a document from a collection of parts
without having mandatory page breaks between then.
For this case, the package
provides a mechanism to include parts
by |\input| which can also be processed individually.
However, by construction this mechanism
requires manual handling of the content to be output.

%%%%%%%%%%%%%%%%%%%%%%%%%%%%%%%%%%%%%%%%
\DescribeMacro{\ifchilddocmanual}
The main file should be prepared as usual, see \secref{sec:include}.
However, the document body must make a distinction
between processing of an individual part and of the main document, e.g.:
%
\begin{center}
\begin{tabular}{l}
|\ifchilddocmanual|\\
|\input{\childdocname}|\\
|\||else|\\
\textit{document body with }|\input{|\textit{part}|}|\\
|\||fi|
\end{tabular}
\end{center}
%
The conditional |\ifchilddocmanual| is true whenever
a part to be included by |\input| is being compiled,
and the name of the part is stored in |\childdocname|.

%%%%%%%%%%%%%%%%%%%%%%%%%%%%%%%%%%%%%%%%
\DescribeMacro{\childdocby}
Each part to be included by |\input| should start with:
%
\begin{center}
\begin{tabular}{l}
|\input{childdoc.def}|\\
|\childdocby{|\textit{main}|}|\\
\end{tabular}
\end{center}
%
The directive |\childdocby| is similar to |\childdocof|
described in \secref{sec:include},
but the subsequent selection of content must be done manually.
To that end, both |\ifchilddoc| and |\ifchilddocmanual|
will be true upon processing of a part,
and the name of the part is stored in |\childdocname|.
Note that |\jobname| will be set to the filename of the current part
so that each part receives an individual |.aux| file
that does not interfere with the |.aux| file(s) of the main document.
This behaviour can be altered by the alternative form
|\childdocby[*]{|\textit{main}|}| (with a non-empty optional argument)
which uses the |.aux| file of the main document
by setting |\jobname| to \textit{main}.

%%%%%%%%%%%%%%%%%%%%%%%%%%%%%%%%%%%%%%%%%%%%%%%%%%%%%%%%%%%%%%%%%%%%%%%%%%%%%%%%
\subsection{Driver Development}
\label{sec:driver}

The \textsf{childdoc} mechanism can also be use for the development
of definition files such as \LaTeX{} styles or classes.
This case differs from the above setup with multiple parts
included by |\include| in that no |\includeonly| should be invoked.
This can be achieved by starting the include file
(before |\ProvidesPackage|) with:
%
\begin{center}
\begin{tabular}{l}
|\input{childdoc.def}|\\
|\childdocforward{|\textit{main}|}|\\
\end{tabular}
\end{center}
%
or alternatively with:
%
\begin{center}
\begin{tabular}{l}
|\input{childdoc.def}|\\
|\childdocby{|\textit{main}|}|\\
\end{tabular}
\end{center}
%
Both forms have slightly different effects as described above.
The main file is prepared as usual, see \secref{sec:include}.

%%%%%%%%%%%%%%%%%%%%%%%%%%%%%%%%%%%%%%%%%%%%%%%%%%%%%%%%%%%%%%%%%%%%%%%%%%%%%%%%
\subsection{Legacy Detection}
\label{sec:detection}

The directive |\childdocmain| in the main file can detect
whether the complete document or merely a child is to be compiled
even without using the directive |\childdocof|.
This method is deprecated because it is less robust
and there is no compelling reason to use it;
it is merely provided for backward compatibility
and it may be removed in future versions.

If the detection mechanism is to be used,
it is mandatory to correctly specify
the filename of the main file as the argument of |\childdocmain|:
%
\begin{center}
\begin{tabular}{l}
|\input{childdoc.def}|\\
|\childdocmain{|\textit{main}|}|\\
\end{tabular}
\end{center}
%
If |\jobname| does not match the argument \textit{main} of |\childdocmain|,
it is assumed that |\jobname| points to the child file to be compiled.
When using |\childdocmain| with the main file specified as argument,
it suffices to start a child file
with just |\input{|\textit{main}|}|
without loading of the package and using |\childdocof|.
If instead all processing is done
with the appropriate \textsf{childdoc} directives,
the argument of \textit{main} of |\childdocmain| can be empty.

An alternative version of the command line processing described
in \secref{sec:commandline} using the detection mechanism reads:
%
\begin{center}
|... -jobname "|\textit{target}|" "|[\textit{flags}]%
[|\def\jobname{|\textit{dest}|}|]|\input{|\textit{main}|}"|
\end{center}

%%%%%%%%%%%%%%%%%%%%%%%%%%%%%%%%%%%%%%%%%%%%%%%%%%%%%%%%%%%%%%%%%%%%%%%%%%%%%%%%
\subsection{Manual Code}
\label{sec:manual}

In case one cannot be certain whether the definitions file |childdoc.def|
is installed on the target \TeX{} distribution
and one prefers not to ship it,
it is conceivable to paste a few relevant commands into the sources.

To that end, drop all statements |\input{childdoc.def}|
and perform the replacements as outlined below.
Instead of |\childdocmain{|\textit{main}|}| add the following code
to the top of the main file:
%
\begin{center}
\begin{tabular}{l}
|\||ifdefined\childdocname\endinput\||fi\newif\ifchilddoc|\\
|\edef\childdocname{\scantokens\expandafter{\jobname\noexpand}}|\\
|\def\childdocmain{|\textit{main}|}\||ifx\childdocmain\childdocname\||else|\\
|\childdoctrue\includeonly{\childdocname}\let\jobname\childdocmain\||fi|\\
\end{tabular}
\end{center}
%
Instead of |\childdocof{|\textit{main}|}| just include the main file
at the top of each child file:
%
\begin{center}
|\input{|\textit{main}|}|
\end{center}
%
A simple redirection |\childdocforward{|\textit{dest}|}| is achieved by:
%
\begin{center}
|\def\jobname{|\textit{dest}|}\input{\jobname}|
\end{center}
%
The redirection with prefix
|\childdocforwardprefix[|\textit{prefix}|]{|\textit{dest}|}|
is accomplished by:
%
\begin{center}
\begin{tabular}{l}
|{\edef\jobname{\scantokens\expandafter{\jobname\noexpand}}|\\
|\def\redirectjob |\textit{prefix}|#1~~~{\gdef\jobname{|\textit{dest}|#1}}|\\
|\expandafter\redirectjob\jobname~~~}\input{\jobname}|
\end{tabular}
\end{center}

In an alternative approach,
child documents can be compiled by a specific command line
without additional code or specific definitions:
%
\begin{center}
|... -jobname "|\textit{target}|" "|[\textit{flags}]%
|\includeonly{|\textit{dest}|}\input{|\textit{main}|}"|
\end{center}
%

%%%%%%%%%%%%%%%%%%%%%%%%%%%%%%%%%%%%%%%%%%%%%%%%%%%%%%%%%%%%%%%%%%%%%%%%%%%%%%%%
%%%%%%%%%%%%%%%%%%%%%%%%%%%%%%%%%%%%%%%%%%%%%%%%%%%%%%%%%%%%%%%%%%%%%%%%%%%%%%%%
\section{Information}

%%%%%%%%%%%%%%%%%%%%%%%%%%%%%%%%%%%%%%%%%%%%%%%%%%%%%%%%%%%%%%%%%%%%%%%%%%%%%%%%
\subsection{Copyright}

Copyright \copyright{} 2017--2018 Niklas Beisert

This work may be distributed and/or modified under the
conditions of the \LaTeX{} Project Public License, either version 1.3
of this license or (at your option) any later version.
The latest version of this license is in
  \url{http://www.latex-project.org/lppl.txt}
and version 1.3 or later is part of all distributions of \LaTeX{}
version 2005/12/01 or later.

This work has the LPPL maintenance status `maintained'.

The Current Maintainer of this work is Niklas Beisert.

This work consists of the files |README.txt|, |childdoc.ins| and |childdoc.dtx|
as well as the derived files |childdoc.def|, |cdocsamp.tex|
with |cdocsch1.tex|, |cdocsch2.tex|, |cdocspt3.tex|, |cdocspt4.tex|,
|cdocsdrf.tex|, |cdocsfn1.tex|, |cdocsfn2.tex|
as well as |childdoc.pdf|.

%%%%%%%%%%%%%%%%%%%%%%%%%%%%%%%%%%%%%%%%%%%%%%%%%%%%%%%%%%%%%%%%%%%%%%%%%%%%%%%%
\subsection{Files and Installation}

The package consists of the files:
%
\begin{center}
\begin{tabular}{ll}
    |README.txt|   & readme file \\
    |childdoc.ins| & installation file \\
    |childdoc.dtx| & source file \\
    |childdoc.def| & definition file \\
    |cdocsamp.tex| & sample main file \\
    |cdocsch1.tex| & sample include file \\
    |cdocsch2.tex| & sample include file \\
    |cdocspt3.tex| & sample part file \\
    |cdocspt4.tex| & sample part file \\
    |cdocsdrf.tex| & sample redirection file \\
    |cdocsfn1.tex| & sample redirection file \\
    |cdocsfn2.tex| & sample redirection file \\
    |childdoc.pdf| & manual
\end{tabular}
\end{center}
%
The distribution consists of the files
|README.txt|, |childdoc.ins| and |childdoc.dtx|.
%
\begin{itemize}
\item
Run (pdf)\LaTeX{} on |childdoc.dtx|
to compile the manual |childdoc.pdf| (this file).
\item
Run \LaTeX{} on |childdoc.ins| to create the definitions file |childdoc.def|
and the sample |cdocsamp.tex| with include files
|cdocsch1.tex|, |cdocsch2.tex|, |cdocspt3.tex|, |cdocspt4.tex|,
|cdocsdrf.tex|, |cdocsfn1.tex|, |cdocsfn2.tex|.
Then copy the file |childdoc.def| to an appropriate directory of your \LaTeX{}
distribution, e.g.\ \textit{texmf-root}|/tex/latex/childdoc|.
\end{itemize}

%%%%%%%%%%%%%%%%%%%%%%%%%%%%%%%%%%%%%%%%%%%%%%%%%%%%%%%%%%%%%%%%%%%%%%%%%%%%%%%%
\subsection{Related CTAN Packages}

There are several other packages which offer a similar functionality:
%
\begin{itemize}
\item
The packages
\href{http://ctan.org/pkg/docmute}{\textsf{docmute}},
\href{http://ctan.org/pkg/includex}{\textsf{includex}} and
\href{http://ctan.org/pkg/standalone}{\textsf{standalone}}
provide commands to include only the document body of
a child file thus allowing both files to be compiled individually.
\item
The packages \href{http://ctan.org/pkg/subdocs}{\textsf{subdocs}}
and \href{http://ctan.org/pkg/subfiles}{\textsf{subfiles}}
provide structures in which the main and child documents can be
encapsulated and allowing them to be compiled individually.
The inclusion mechanism is different from the conventional |\include|.
\item
The package \href{http://ctan.org/pkg/combine}{\textsf{combine}}
is an elaborate solution to combine several documents into one.
\end{itemize}
%
See also the CTAN topic \href{http://ctan.org/topic/subdocs}{\textsf{subdocs}}
for further related packages.
The present package differs from the above solutions in that
a document structure constructed with the conventional |\include| mechanism
just needs two extra commands at the top of every file
such that all constituent files can be compiled individually.

%%%%%%%%%%%%%%%%%%%%%%%%%%%%%%%%%%%%%%%%%%%%%%%%%%%%%%%%%%%%%%%%%%%%%%%%%%%%%%%%
%\subsection{Feature Suggestions}
%
%The following is a list of features which may be useful for future
%versions of this package:
%%
%\begin{itemize}
%\item
%\ldots
%\end{itemize}

%%%%%%%%%%%%%%%%%%%%%%%%%%%%%%%%%%%%%%%%%%%%%%%%%%%%%%%%%%%%%%%%%%%%%%%%%%%%%%%%
\subsection{Revision History}

%%%%%%%%%%%%%%%%%%%%%%%%%%%%%%%%%%%%%%%%
\paragraph{v2.0:} 2018/12/30

\begin{itemize}
\item
immediate forward processing
\item
added |\childdocby| mechanism
\item
manual restructured
\end{itemize}

%%%%%%%%%%%%%%%%%%%%%%%%%%%%%%%%%%%%%%%%
\paragraph{v1.6:} 2018/01/17

\begin{itemize}
\item
application for development of include files
\item
corrections to manual
\end{itemize}

%%%%%%%%%%%%%%%%%%%%%%%%%%%%%%%%%%%%%%%%
\paragraph{v1.5:} 2017/05/21

\begin{itemize}
\item
more complete structuring introduced
\item
|\childdocof| introduced
\item
|\childdoc| renamed to |\childdocmain|
\item
|\childredirect| renamed to |\childdocforward| and |\childdocforwardprefix|
and functionality expanded
\end{itemize}

%%%%%%%%%%%%%%%%%%%%%%%%%%%%%%%%%%%%%%%%
\paragraph{v1.0:} 2017/04/27

\begin{itemize}
\item
manual and install package
\item
first version published on CTAN
\end{itemize}

%%%%%%%%%%%%%%%%%%%%%%%%%%%%%%%%%%%%%%%%
\paragraph{v0.6:} 2017/04/26

\begin{itemize}
\item
redirection mechanism added
\end{itemize}

%%%%%%%%%%%%%%%%%%%%%%%%%%%%%%%%%%%%%%%%
\paragraph{v0.5:} 2017/04/26

\begin{itemize}
\item
functionality in definition file
\end{itemize}


%%%%%%%%%%%%%%%%%%%%%%%%%%%%%%%%%%%%%%%%%%%%%%%%%%%%%%%%%%%%%%%%%%%%%%%%%%%%%%%%
%%%%%%%%%%%%%%%%%%%%%%%%%%%%%%%%%%%%%%%%%%%%%%%%%%%%%%%%%%%%%%%%%%%%%%%%%%%%%%%%
%%%%%%%%%%%%%%%%%%%%%%%%%%%%%%%%%%%%%%%%%%%%%%%%%%%%%%%%%%%%%%%%%%%%%%%%%%%%%%%%
\appendix

\settowidth\MacroIndent{\rmfamily\scriptsize 000\ }

 \DocInput{childdoc.dtx}

\end{document}
%</driver>
% \fi
%
% %%%%%%%%%%%%%%%%%%%%%%%%%%%%%%%%%%%%%%%%%%%%%%%%%%%%%%%%%%%%%%%%%%%%%%%%%%%%%%
% %%%%%%%%%%%%%%%%%%%%%%%%%%%%%%%%%%%%%%%%%%%%%%%%%%%%%%%%%%%%%%%%%%%%%%%%%%%%%%
% \section{Sample}
%\iffalse
%<*samplemain>
%\fi
%
% The following presents a sample document
% with two chapters, two parts, a title page,
% a compile flag as well as three forwarding files to set the flag.
% It consists of eight |.tex| files:
% \begin{center}
% \begin{tabular}{ll}
% |cdocsamp.tex|&main file\\
% |cdocsch1.tex|&include file for chapter 1\\
% |cdocsch2.tex|&include file for chapter 2\\
% |cdocspt3.tex|&include file for part 3\\
% |cdocspt4.tex|&include file for part 4\\
% |cdocsdrf.tex|&forwarding file for main file in draft mode\\
% |cdocsfi1.tex|&forwarding file for final version of chapter 1\\
% |cdocsfi2.tex|&forwarding file for final version of chapter 2\\
% \end{tabular}
% \end{center}
% Each of the eight files can be compiled directly by the \LaTeX{} compiler.
%
% %%%%%%%%%%%%%%%%%%%%%%%%%%%%%%%%%%%%%%
% \paragraph{Main File.}
%
% The main file is called |cdocsamp.tex|.
%
% Load the \textsf{childdoc} definitions and
% declare the filename for the main document:
%    \begin{macrocode}
\input{childdoc.def}
\childdocmain{}
%    \end{macrocode}

% Optional override for |\version| flag:
%    \begin{macrocode}
%%\ifchilddoc\else\providecommand{\version}{draft}\fi
%    \end{macrocode}

% Define the default values for the |\version| flag
% (|final| for the main file and |draft| for childs):
%    \begin{macrocode}
\ifchilddoc
\providecommand{\version}{draft}
\else
\providecommand{\version}{final}
\fi
%    \end{macrocode}

% Load the standard document class:
%    \begin{macrocode}
\documentclass[12pt]{article}
%    \end{macrocode}

% Start the document body:
%    \begin{macrocode}
\begin{document}
%    \end{macrocode}

% Declare a title page.
% Print title, part of document being processed and version flag:
%    \begin{macrocode}
\addtocounter{page}{-1}
\begin{center}
{\LARGE\bfseries{}childdoc example\par}
\vspace{1cm}
\ifchilddoc
\ifchilddocmanual part\else chapter\fi:
`\childdocname' of `\childdocjob'\par
\else
main document: `\childdocjob'\par
\fi
version: \version\par
\end{center}
\newpage
%    \end{macrocode}

% Manually include selected file,
% otherwise process as usual:
%    \begin{macrocode}
\ifchilddocmanual
\section*{part `\childdocname'}
\input{\childdocname}
\else
%    \end{macrocode}

% Include the two chapters:
%    \begin{macrocode}
\include{cdocsch1}
\include{cdocsch2}
%    \end{macrocode}

% Include the two parts unless only chapters should be displayed:
%    \begin{macrocode}
\ifchilddoc\else
\section{part three}
\input{cdocspt3}
\section{part four}
\input{cdocspt4}
\fi
%    \end{macrocode}

% Process as usual until here:
%    \begin{macrocode}
\fi
%    \end{macrocode}

% End of document body:
%    \begin{macrocode}
\end{document}
%    \end{macrocode}
%\iffalse
%</samplemain>
%\fi
%
% %%%%%%%%%%%%%%%%%%%%%%%%%%%%%%%%%%%%%%
% \paragraph{Chapter Include Files.}
%
% The include files are called |cdocsch1.tex| and |cdocsch2.tex|.
%
%\iffalse
%<*samplechap1|samplechap2>
%\fi

% Optional override for |\version| flag:
%    \begin{macrocode}
%%\providecommand{\version}{final}
%    \end{macrocode}

% Include the main document:
%    \begin{macrocode}
\input{childdoc.def}
\childdocof{cdocsamp}
%    \end{macrocode}

%\iffalse
%</samplechap1|samplechap2>
%\fi
%
%\iffalse
%<*samplechap1>
%\fi
% Some text for chapter 1:
%    \begin{macrocode}
\section{one}
some text in chapter one
%    \end{macrocode}

%\iffalse
%</samplechap1>
%\fi
% Some text for chapter 2:
%\iffalse
%<*samplechap2>
%\fi
%    \begin{macrocode}
\section{two}
more text in chapter two
%    \end{macrocode}

%\iffalse
%</samplechap2>
%\fi
%
% %%%%%%%%%%%%%%%%%%%%%%%%%%%%%%%%%%%%%%
% \paragraph{Part Include Files.}
%
% The include files are called |cdocspt3.tex| and |cdocspt4.tex|.
%
%\iffalse
%<*samplepart3|samplepart4>
%\fi

% Optional override for |\version| flag:
%    \begin{macrocode}
%%\providecommand{\version}{final}
%    \end{macrocode}

% Include the main document:
%    \begin{macrocode}
\input{childdoc.def}
\childdocby{cdocsamp}
%    \end{macrocode}

%\iffalse
%</samplepart3|samplepart4>
%\fi
%
%\iffalse
%<*samplepart3>
%\fi
% Some text for part 3:
%    \begin{macrocode}
some text in part three
%    \end{macrocode}

%\iffalse
%</samplepart3>
%\fi
% Some text for part 4:
%\iffalse
%<*samplepart4>
%\fi
%    \begin{macrocode}
more text in part four
%    \end{macrocode}

%\iffalse
%</samplepart4>
%\fi
%
% %%%%%%%%%%%%%%%%%%%%%%%%%%%%%%%%%%%%%%
% \paragraph{Forwarding for a Complete Draft.}
%
% The following forwarding file |cdocsdrf.tex|
% compiles the main document in draft mode:
%\iffalse
%<*sampledraft>
%\fi
%    \begin{macrocode}
\def\version{draft}
\input{childdoc.def}
\childdocforward{cdocsamp}
%    \end{macrocode}

%\iffalse
%</sampledraft>
%\fi
%
% %%%%%%%%%%%%%%%%%%%%%%%%%%%%%%%%%%%%%%
% \paragraph{Forwarding for Final Version of the Chapters.}
%
% The following forwarding files |cdocsfn1.tex| and |cdocsfn2.tex|
% (with identical content)
% compile the final versions of the child documents
% |cdocsch1.tex| and |cdocsch2.tex|, respectively:
%\iffalse
%<*samplefinal>
%\fi
%    \begin{macrocode}
\def\version{final}
\input{childdoc.def}
\childdocforwardprefix[cdocsamp]{cdocsfn}{cdocsch}
%    \end{macrocode}

%\iffalse
%</samplefinal>
%\fi
%
% %%%%%%%%%%%%%%%%%%%%%%%%%%%%%%%%%%%%%%
% \paragraph{Command Line Processing.}
%
% The following three command lines generate the output files
% |cdocscld|, |cdocscl1| and |cdocscl2|
% which should be identical to
% |cdocsdrf|, |cdocsch1| and |cdocsfn2|, respectively:
% \begin{center}
% \begin{tabular}{l}
% |latex -jobname cdocscld \|\\
% |  "\def\version{draft}\input{childdoc.def}\childdocforward{cdocsamp}"|\\
% |latex -jobname cdocscl1 \|\\
% |  "\input{childdoc.def}\childdocforward[cdocsamp]{cdocsch1}"|\\
% |latex -jobname cdocscl2 \|\\
% |  "\def\version{final}\input{childdoc.def}\childdocforward{cdocsch2}"|
% \end{tabular}
% \end{center}
% Note that the trailing backslash on each first line
% merely continues the input to the second line
% (for convenient cut ant paste).
% Furthermore, the command |latex| can be replaced by any
% of its alternative versions such as |pdflatex|.
%
% %%%%%%%%%%%%%%%%%%%%%%%%%%%%%%%%%%%%%%%%%%%%%%%%%%%%%%%%%%%%%%%%%%%%%%%%%%%%%%
% %%%%%%%%%%%%%%%%%%%%%%%%%%%%%%%%%%%%%%%%%%%%%%%%%%%%%%%%%%%%%%%%%%%%%%%%%%%%%%
% \section{Implementation}
%\iffalse
%<*package>
%\fi
%
% This section describes the definitions file |childdoc.def|.

% The definitions cannot be loaded using |\usepackage| or |\RequirePackage|
% which has a mechanism to prevent loading a style file more than once.
% When loading the definitions by means of |\input|
% multiple instances have to be prevented manually:
%\iffalse
%This code needs to be before the `\ProvidesFile' directive
%which is defined at the beginning of this file.
%Therefore it is also placed there and commented out here.
%</package>
%<*discard>
%\fi
%    \begin{macrocode}
\ifdefined\childdocmain\endinput\fi
%    \end{macrocode}
%\iffalse
%</discard>
%<*package>
%\fi
%
% \macro{\ifchilddoc}
% \macro{\ifchilddocmanual}
% The conditional |\ifchilddoc| tells whether a
% child (true) or main (false) document is being compiled.
% The conditional |\ifchilddocmanual| tells whether
% the |\includeonly| mechanism is used (false) or
% the selection of child files must be performed manually (true).
% The definitions initialise to false:
%    \begin{macrocode}
\newif\ifchilddoc
\newif\ifchilddocmanual
%    \end{macrocode}

% \macro{\childdocname}
% \macro{\childdocjob}
% The macro |\childdocname| stores the name of the main document
% to be compiled. The macro |\childdocjob| stores the name of
% the document on which the \LaTeX{} compiler was originally invoked.
% The content of |\jobname| cannot be compared
% to filenames specified in the source due to different catcodes.
% The following code rescans |\jobname|, stores the result
% in |\childdocname| and saves a copy in |\childdocjob|:
%    \begin{macrocode}
\edef\childdocname{\scantokens\expandafter{\jobname\noexpand}}
\let\childdocjob\childdocname
%    \end{macrocode}

% \macro{\childdocdisable}
% The macro |\childdocdisable| prevents the main file
% from being processed more than once.
% At this stage, the main document command |\childdocmain|
% is assumed to be called once again where it should do nothing.
% Any subsequent call to it should prevent
% a secondary processing of the main document
% It overwrites the forwarding commands
% |\childdocof| and |\childdocforward|
% with empty macros to prevent further inclusions of the main document:
%    \begin{macrocode}
\newcommand{\childdocdisable}
{
  \renewcommand{\childdocmain}[1]{\renewcommand{\childdocmain}[1]{\endinput}}
  \renewcommand{\childdocof}[1]{}
  \renewcommand{\childdocby}[2][]{}
  \renewcommand{\childdocforward}[2][]{}
  \renewcommand{\childdocdisable}{}
}
%    \end{macrocode}

% \macro{\childdocmain}
% The macro |\childdocmain| is to be called at the top of the main file
% with nothing or the main filename (without extension) as argument.
% First, it breaks loops.
% If the argument is not empty and does not match |\childdocname|
% (which is set by the first inclusion of |childdoc.def|),
% |\ifchilddoc| is set to true, |\includeonly| is applied to the child file
% and |\jobname| is set to the main file
% (for proper handling of |.aux| files):
%    \begin{macrocode}
\newcommand{\childdocmain}[1]
{
  \childdocdisable\childdocmain{}
  \if?#1?\else
    \begingroup
      \def\childdoctmp{#1}
      \ifx\childdoctmp\childdocname
        \def\childdoctmp{}
      \else
        \def\childdoctmp
        {
          \childdoctrue
          \includeonly{\childdocname}
          \def\childdocjob{#1}
          \def\jobname{#1}
        }
      \fi
      \expandafter
    \endgroup
    \childdoctmp
  \fi
}
%    \end{macrocode}

% \macro{\childdocof}
% The command |\childdocof| redirects
% compilation to the main file |#1|.
%    \begin{macrocode}
\newcommand{\childdocof}[1]
{
  \childdocdisable
  \childdoctrue
  \includeonly{\childdocname}
  \def\jobname{#1}
  \def\childdocjob{#1}
  \input{#1}
}
%    \end{macrocode}

% \macro{\childdocby}
% The command |\childdocby| ....
%    \begin{macrocode}
\newcommand{\childdocby}[2][]
{
  \childdocdisable
  \childdoctrue
  \childdocmanualtrue
  \if?#1?\else
    \def\jobname{#2}
  \fi
  \def\childdocjob{#2}
  \input{#2}
  \endinput
}
%    \end{macrocode}

% \macro{\childdocforward}
% The command |\childdocforward| redirects
% compilation to the main file or
% (if the optional argument is given) a child file.
% Parameters are set as if the main file
% or a child file starting with |\childdocof| was compiled.
% Then compilation is handed over to the main file:
%    \begin{macrocode}
\newcommand{\childdocforward}[2][]
{
  \begingroup
    \if?#1?
      \def\childdoctmp
      {
        \def\childdocname{#2}
        \def\childdocjob{#2}
        \def\jobname{#2}
        \input{#2}
        \endinput
      }
    \else
      \def\childdoctmp
      {
        \childdocdisable
        \def\childdocname{#2}
        \childdoctrue
        \includeonly{#2}
        \def\childdocjob{#1}
        \def\jobname{#1}
        \input{#1}
        \endinput
      }
    \fi
    \expandafter
  \endgroup
  \childdoctmp
}
%    \end{macrocode}

% \macro{\childdocforwardprefix}
% The command |\childdocforwardprefix| redirects
% compilation to the main or a child file by means of a pattern.
% The prefix |#1| in the current filename is replaced by |#2|
% and the suffix of the current filename is kept
% (it is assumed that the filename does not contain the substring `|~~~|'
% which is used as a delimiter).
% Compilation is handed over to the new file by |\childdocforward|:
%    \begin{macrocode}
\newcommand{\childdocforwardprefix}[3][]
{
  \begingroup
    \def\childdocextract #2##1~~~{\def\childdoctmp{\childdocforward[#1]{#3##1}}}
    \expandafter\childdocextract\childdocname~~~
    \expandafter
  \endgroup
  \childdoctmp
}
%    \end{macrocode}

% \macro{\childdoc}
% The deprecated macro |\childdoc| is a legacy version of |\childdocmain|:
%    \begin{macrocode}
\newcommand{\childdoc}{\childdocmain}
%    \end{macrocode}

% \macro{\childdocredirect}
% The deprecated macro |\childdocredirect| is a legacy version
% of |\childdocforward| and |\childdocforwardprefix|:
%    \begin{macrocode}
\newcommand{\childdocredirect}[2][]
{
  \begingroup
    \if?#1?
      \def\childdoctmp{\childdocforward{#2}}
    \else
      \def\childdoctmp{\childdocforwardprefix{#1}{#2}}
    \fi
    \expandafter
  \endgroup
  \childdoctmp
}
%    \end{macrocode}

%\iffalse
%</package>
%\fi
%
\endinput
|\\
|\childdocforward{|\textit{main}|}|
\end{tabular}
\end{center}
%
Likewise, the following files |final|\textit{nn}|.tex|
compile the final version of the child document
|child|\textit{nn}|.tex|:
%
\begin{center}
\begin{tabular}{l}
|\def\version{final}|\\
|% \iffalse
%
% childdoc.dtx Copyright (C) 2017-2018 Niklas Beisert
%
% This work may be distributed and/or modified under the
% conditions of the LaTeX Project Public License, either version 1.3
% of this license or (at your option) any later version.
% The latest version of this license is in
%   http://www.latex-project.org/lppl.txt
% and version 1.3 or later is part of all distributions of LaTeX
% version 2005/12/01 or later.
%
% This work has the LPPL maintenance status `maintained'.
%
% The Current Maintainer of this work is Niklas Beisert.
%
% This work consists of the files childdoc.dtx and childdoc.ins
% and the derived files childdoc.def and cdocsamp.tex with
% cdocsch1.tex, cdocsch2.tex, cdocsdrf.tex, cdocsfn1.tex, cdocsfn2.tex.
%
%<package>\ifdefined\childdocmain\endinput\fi
%<package>\ProvidesFile{childdoc.def}[2018/12/30 v2.0 child document driver]
%<samplemain>\ProvidesFile{cdocsamp.tex}[2018/12/30 v2.0 sample for childdoc]
%<*driver>
%\ProvidesFile{childdoc.drv}[2018/12/30 v2.0 childdoc reference manual file]
\PassOptionsToClass{10pt,a4paper}{article}
\documentclass{ltxdoc}

\usepackage[margin=35mm]{geometry}
\usepackage{hyperref}
\usepackage{hyperxmp}
\usepackage[usenames]{color}

\hypersetup{colorlinks=true}
\hypersetup{pdfstartview=FitH}
\hypersetup{pdfpagemode=UseNone}
\hypersetup{pdfsource={}}
\hypersetup{pdflang={en-UK}}
\hypersetup{pdfcopyright={Copyright 2017-2018 Niklas Beisert.
  This work may be distributed and/or modified under the
  conditions of the LaTeX Project Public License, either version 1.3
  of this license or (at your option) any later version.}}
\hypersetup{pdflicenseurl={http://www.latex-project.org/lppl.txt}}
\hypersetup{pdfcontactaddress={ETH Zurich, ITP, HIT K,
  Wolfgang-Pauli-Strasse 27}}
\hypersetup{pdfcontactpostcode={8093}}
\hypersetup{pdfcontactcity={Zurich}}
\hypersetup{pdfcontactcountry={Switzerland}}
\hypersetup{pdfcontactemail={nbeisert@itp.phys.ethz.ch}}
\hypersetup{pdfcontacturl={http://people.phys.ethz.ch/\xmptilde nbeisert/}}

\newcommand{\secref}[1]{\hyperref[#1]{section \ref*{#1}}}

\parskip1ex
\parindent0pt
\let\olditemize\itemize
\def\itemize{\olditemize\parskip0pt}

\begin{document}

\title{The \textsf{childdoc} Package}
\hypersetup{pdftitle={The childdoc Package}}
\author{Niklas Beisert\\[2ex]
  Institut f\"ur Theoretische Physik\\
  Eidgen\"ossische Technische Hochschule Z\"urich\\
  Wolfgang-Pauli-Strasse 27, 8093 Z\"urich, Switzerland\\[1ex]
  \href{mailto:nbeisert@itp.phys.ethz.ch}
  {\texttt{nbeisert@itp.phys.ethz.ch}}}
\hypersetup{pdfauthor={Niklas Beisert}}
\hypersetup{pdfsubject={Manual for the LaTeX2e Package childdoc}}
\date{30 December 2018, \textsf{v2.0}}
\maketitle

\begin{abstract}\noindent
\textsf{childdoc} is a \LaTeXe{} package
that enables the direct compilation
of document sections included by |\include|
to individual files.
\end{abstract}

\begingroup
\parskip0ex
\tableofcontents
\endgroup

%%%%%%%%%%%%%%%%%%%%%%%%%%%%%%%%%%%%%%%%%%%%%%%%%%%%%%%%%%%%%%%%%%%%%%%%%%%%%%%%
%%%%%%%%%%%%%%%%%%%%%%%%%%%%%%%%%%%%%%%%%%%%%%%%%%%%%%%%%%%%%%%%%%%%%%%%%%%%%%%%
\section{Introduction}

\LaTeX{} provides a mechanism to structure a large document (such as a book)
into a main file and several child files (containing the chapters)
using the |\include| command.
This mechanism is beneficial for documents
which span hundreds of pages in order to
make the source file(s) more manageable.
Moreover, compilation can be restricted to
selected child files by means of the |\includeonly| command.
The latter feature can be used to reduce the compilation time while editing
(this was significantly more useful in the earlier days of \LaTeX{})
or to generate a smaller document which is easier to navigate.
Another application of |\includeonly| is to generate
documents consisting of selected parts of the complete document.

However, there are a few drawbacks of the plain |\include| mechanism:
\begin{itemize}
\item
The child files cannot be compiled on their own,
they can only be compiled via the main file.
A naive editing environment
(such as a text editor with an option
to have the current file processed by \LaTeX)
may require one to switch to the main file before compiling;
attempting to compile the child file produces errors.
\item
The main file must be modified (each time)
to adjust the |\includeonly| command
to the present needs. This easily leaves the main file in a messy state.
\item
The generated document will always carry the filename
of the main document. This is inconvenient if
several child files are to be compiled and
to be kept for distribution.
\end{itemize}

The present package provides a simple interface
to make child files individually compilable by \LaTeX{}.
Compiling a child file then has the same effect as compiling
the main file with an |\includeonly| command
to select the appropriate child.
Moreover the generated document will carry the name of the child
rather than the main file.
This resolves all three above issues.

This feature is meant to make the editing of books,
thesis documents and lecture notes somewhat more convenient.
However, the package can also be used efficiently for
composing a series of documents (such as exercise sheets)
which are typically distributed individually.
It then assists the author in generating the individual documents
(potentially in different versions)
as well as a document containing the collected series.
Another application is in developing style files
or other kinds of included material
where compilation of the style file could redirect
to a sample or test file.

%%%%%%%%%%%%%%%%%%%%%%%%%%%%%%%%%%%%%%%%%%%%%%%%%%%%%%%%%%%%%%%%%%%%%%%%%%%%%%%%
%%%%%%%%%%%%%%%%%%%%%%%%%%%%%%%%%%%%%%%%%%%%%%%%%%%%%%%%%%%%%%%%%%%%%%%%%%%%%%%%
\section{Usage}

First of all, the package \textsf{childdoc} is \emph{not} a standard
\LaTeXe{} |.sty| style file! Therefore it needs to be invoked in
a non-standard way.

%%%%%%%%%%%%%%%%%%%%%%%%%%%%%%%%%%%%%%%%%%%%%%%%%%%%%%%%%%%%%%%%%%%%%%%%%%%%%%%%
\subsection{Included Files}
\label{sec:include}

%%%%%%%%%%%%%%%%%%%%%%%%%%%%%%%%%%%%%%%%
\DescribeMacro{\childdocmain}
To use the package, add the commands
\begin{center}
\begin{tabular}{l}
|\input{childdoc.def}|\\
|\childdocmain{}|\\
\end{tabular}
\end{center}
at the very top of the main \LaTeX{} file,
in particular \emph{before} the |\documentclass| statement!
The argument of |\childdocmain| should be left empty
(but it must be present).

%%%%%%%%%%%%%%%%%%%%%%%%%%%%%%%%%%%%%%%%
\DescribeMacro{\childdocof}
Furthermore, add the commands
\begin{center}
\begin{tabular}{l}
|\input{childdoc.def}|\\
|\childdocof{|\textit{main}|}|\\
\end{tabular}
\end{center}
at the top of every child file \textit{child}
which is included by |\include{|\textit{child}|}|
from within the main file
(or at least for those files to be compiled individually).
The argument \textit{main} must be the filename of the main file.

There are a couple of
considerations in setting up the main and child documents:

%%%%%%%%%%%%%%%%%%%%%%%%%%%%%%%%%%%%%%%%
\paragraph{Restrictions.}

Please note the following restrictions:
\begin{itemize}
\item
|\childdocmain| must be called with one argument \textit{main}
to ensure compatibility with earlier version of the package.
It must either be empty (|\childdocmain{}|)
or precisely match the filename of the main file in which it is specified.
See \secref{sec:detection} for further information.
\item
The filename \textit{main} must be specified without the |.tex| extension.
\item
The filename \textit{main} is case sensitive
(even in case-insensitive file systems)
due to internal string comparison.
\item
The argument \textit{main} should be fully expanded, it cannot be a macro.
\item
Subdirectories and special characters should be avoided in filenames.
\item
The command |\childdocmain{|\textit{main}|}| must be followed by a whitespace.
It should not be followed immediately by another command
or by a comment mark `|%|'.
This is because the \TeX{} parser reads the token immediately following
the argument of |\childdocmain| and puts it
at the beginning of every child section;
however, a white\-space is ignored.
\end{itemize}

%%%%%%%%%%%%%%%%%%%%%%%%%%%%%%%%%%%%%%%%
\paragraph{Content of Main File.}

It is advisable to place all content in the child files included by |\include|.
Any output contained in the main file will appear in all child documents
unless suppressed manually;
it cannot be suppressed automatically by the |\includeonly| directive
and thus should normally be avoided.
A method to include some content in the main file
by means of conditional processing is described in \secref{sec:conditional}.

%%%%%%%%%%%%%%%%%%%%%%%%%%%%%%%%%%%%%%%%
\paragraph{Page Numbering.}

When only a part of the document is compiled,
the appropriate numbering of pages
(as well as other status parameters)
is determined from the |.aux| files.
The latter contain information from previous passes.
However this information needs to propagate through
all intermediate child documents.
Therefore the page numbering in child documents may well
be inconsistent until the complete document is compiled at least once.

A useful (if unconventional) way to always ensure a consistent
page numbering is to restart the numbering in each child document
and denote the pages by `\textit{child}|.|\textit{page}'
where \textit{child} represents the chapter/section number of the child file.
This can be achieved by the command
|\numberwithin{page}{|\textit{child}|}|
of the \textsf{amsmath} package
where \textit{child} can be |chapter| or |section|
depending on the chosen structuring.
Alternatively, one can modify the macro |\thepage| appropriately
and reset the counter |page| at the start of each child file.

%%%%%%%%%%%%%%%%%%%%%%%%%%%%%%%%%%%%%%%%%%%%%%%%%%%%%%%%%%%%%%%%%%%%%%%%%%%%%%%%
\subsection{Conditional Processing}
\label{sec:conditional}

The package provides a mechanism to compile different versions
of a document. To customise the versions further some conditional processing
can come in handy to distinguish which version is being compiled.
The package provides two macros to describe the compilation context:

%%%%%%%%%%%%%%%%%%%%%%%%%%%%%%%%%%%%%%%%
\DescribeMacro{\ifchilddoc}
The conditional |\ifchilddoc| distinguishes between the compilation of
child documents and the main document:
%
\begin{center}
|\ifchilddoc |\textit{child-code}| |[|\||else |\textit{main-code}]| \||fi|
\end{center}

%%%%%%%%%%%%%%%%%%%%%%%%%%%%%%%%%%%%%%%%
\DescribeMacro{\childdocname}
\DescribeMacro{\childdocjob}
The macro |\childdocname| contains the filename (without extension)
of the main or child file being processed.
Note that |\childdocjob| will always contain the name of the main file.

%%%%%%%%%%%%%%%%%%%%%%%%%%%%%%%%%%%%%%%%
\paragraph{Title Page.}

Conditional processing can be used to include a title or banner page
in the main document when proper precautions are taken.
Importantly, the code in the main file should ensure that the page counter
(as well as other status parameters which are stored in the |.aux| files)
takes the same value after the conditional processing.
Otherwise the page numbers may take divergent values
depending on which part is compiled.

For example, a title page could be declared by:
%
\begin{center}
\begin{tabular}{l}
|\ifchilddoc\||else|\\
|\addtocounter{page}{-1}|\\
\textit{code for title page}\\
|\newpage|\\
|\||fi|
\end{tabular}
\end{center}
%
A banner page for the child documents can be generated by:
%
\begin{center}
\begin{tabular}{l}
|\ifchilddoc|\\
|\addtocounter{page}{-1}|\\
\textit{code for banner page}\\
|\newpage|\\
|\||fi|
\end{tabular}
\end{center}
%
Here one could write a message such as:
\begin{center}
|This is the part \childdocname{} of \childdocjob{}.|
\end{center}

%%%%%%%%%%%%%%%%%%%%%%%%%%%%%%%%%%%%%%%%%%%%%%%%%%%%%%%%%%%%%%%%%%%%%%%%%%%%%%%%
\subsection{Flags}
\label{sec:flags}

The package makes it easy to generate different versions
of the main or child documents.
To this end compilation flags can be defined
and assigned different default values.
They will be particularly useful in conjunction
with the forwarding mechanism described in \secref{sec:forward}.

For example, it may be useful to have a flag |\version|
which can be set to |draft| or |final|.
The document source will contain some conditional code
depending on the value of |\version|.
Suppose further, the flag should default to |final| for the main file
and to |draft| for child files
which is a natural assignment for editing the document.
This is achieved by placing the following code
in the preamble of the main document
(below the |\childdocmain| directive):
%
\begin{center}
\begin{tabular}{l}
|\ifchilddoc|\\
|\providecommand{\version}{draft}|\\
|\||else|\\
|\providecommand{\version}{final}|\\
|\||fi|
\end{tabular}
\end{center}
%
The definition by |\providecommand| makes sure
that previous definitions are not overwritten.
Further statements |\providecommand{\version}{...}|
can thus be added before the above code to override it.

For the main file, one might add a line
(between |\childdocmain| and the above block)
%
\begin{center}
|%\ifchilddoc\||else\providecommand{\version}{draft}\||fi|
\end{center}
%
which can be uncommented to produce a draft version.
Likewise one can add a line to the very top of a child file
(above the |\childdocof{|\textit{main}|}| directive)
%
\begin{center}
|%\providecommand{\version}{final}|
\end{center}
%
which can be uncommented to produce the final version of this child document.

%%%%%%%%%%%%%%%%%%%%%%%%%%%%%%%%%%%%%%%%%%%%%%%%%%%%%%%%%%%%%%%%%%%%%%%%%%%%%%%%
\subsection{Forwarding}
\label{sec:forward}

Different versions of the main or child documents
using compilation flags as described in \secref{sec:flags}
can be (permanently) stored in different files
for convenient compilation, viewing and distribution.
To this end, the package defines a command
to pass on compilation to a different file:

%%%%%%%%%%%%%%%%%%%%%%%%%%%%%%%%%%%%%%%%
\DescribeMacro{\childdocforward}
The command |\childdocforward| redirects processing to
another source file:
%
\begin{center}
\begin{tabular}{l}
|\input{childdoc.def}|\\
|\childdocforward[|\textit{main}|]{|\textit{dest}|}|\\
\end{tabular}
\end{center}
%
The argument \textit{dest} is the destination file
(without extension).
It should be the main file or one of the child files.
Note that further \textsf{childdoc} directives
such as |\childdocof| and |\childdocforward|
in the indicated file will be processed in this form.
The optional argument \textit{main}
passes on directly to the main file \textit{main}
while pretending to compile the child \textit{dest}.
This form behaves as if \textit{dest}
issues |\childdocof{|\textit{main}|}| right away,
and no further \textsf{childdoc} directives will be processed.

%%%%%%%%%%%%%%%%%%%%%%%%%%%%%%%%%%%%%%%%
\DescribeMacro{\...prefix}
In the alternative form |\childdocforwardprefix|,
%
\begin{center}
\begin{tabular}{l}
|\input{childdoc.def}|\\
|\childdocforwardprefix[|\textit{main}|]{|\textit{prefix}|}{|\textit{dest}|}|
\end{tabular}
\end{center}
%
the destination file is determined by a pattern
depending on the current file:
To make this work, the current file must be called
`{\textit{prefix}\hspace{0.2em}\textit{suffix}}'
with \textit{prefix} matching precisely the argument.
Processing is then passed on to the file
`{\textit{dest}\hspace{0.2em}\textit{suffix}}'.
Surely, the same effect is achieved by
directly specifying the
argument `{\textit{dest}\hspace{0.2em}\textit{suffix}}'
in the first form.
However, that requires to set up a different file
for each child. With the alternative form of the command
all these files can have exactly the same content
which simplifies setting them up and maintaining them.

For example, the following file |draft.tex|
with a compilation flag |\version| as described in \secref{sec:flags}
compiles the main document as a draft:
%
\begin{center}
\begin{tabular}{l}
|\def\version{draft}|\\
|\input{childdoc.def}|\\
|\childdocforward{|\textit{main}|}|
\end{tabular}
\end{center}
%
Likewise, the following files |final|\textit{nn}|.tex|
compile the final version of the child document
|child|\textit{nn}|.tex|:
%
\begin{center}
\begin{tabular}{l}
|\def\version{final}|\\
|\input{childdoc.def}|\\
|\childdocforwardprefix{final}{child}|
\end{tabular}
\end{center}
%

Note that when several versions of a main file and/or of each child file
are to be generated, it may be convenient to set up a |Makefile| or
shell script to automatise the process.

%%%%%%%%%%%%%%%%%%%%%%%%%%%%%%%%%%%%%%%%%%%%%%%%%%%%%%%%%%%%%%%%%%%%%%%%%%%%%%%%
\subsection{Command Line Processing}
\label{sec:commandline}

The effect of redirection files can also be achieved by invoking
the \LaTeX{} compiler with a more elaborate command line.
Most conveniently this should be done as part
of a shell script or a |Makefile|.

When using \textsf{childdoc} in the main file, the following
command lines effectively perform a redirection
(note that depending on the shell being used,
backslashes may have to be doubled: `|\|' $\to$ `|\\|'):
%
\begin{center}
|... -jobname "|\textit{target}|" |\\|"|[\textit{flags}]%
|\input{childdoc.def}\childdocforward[|\textit{main}|]{|\textit{dest}|}"|
\end{center}
%
Here \textit{target} is the name of the output file,
\textit{main} is the name of the main file
and \textit{dest} is the name of the main or child file to be processed
(all filenames without extensions).
The optional argument \textit{main} can be omitted
if \textit{main} matches \textit{dest}.
Optionally, compilation \textit{flags} can be defined via |\def| commands.
This command line makes the \TeX{} engine believe
it is compiling the file \textit{target}
whose content is specified as the latter parameter.
The provided code then forwards the processing to
\textit{main} or \textit{dest} as described in \secref{sec:forward}.

%%%%%%%%%%%%%%%%%%%%%%%%%%%%%%%%%%%%%%%%%%%%%%%%%%%%%%%%%%%%%%%%%%%%%%%%%%%%%%%%
\subsection{Include by Input}
\label{sec:input}

Including child documents by |\include| has some restrictions by design.
Most notably, the content of a child document always occupies
its own set of pages; pages cannot be shared between child documents.
Usually, this behaviour makes perfect sense
because each child document contain an essential part of the document.
However, in some situations it may be desirable to compose
a document from a collection of parts
without having mandatory page breaks between then.
For this case, the package
provides a mechanism to include parts
by |\input| which can also be processed individually.
However, by construction this mechanism
requires manual handling of the content to be output.

%%%%%%%%%%%%%%%%%%%%%%%%%%%%%%%%%%%%%%%%
\DescribeMacro{\ifchilddocmanual}
The main file should be prepared as usual, see \secref{sec:include}.
However, the document body must make a distinction
between processing of an individual part and of the main document, e.g.:
%
\begin{center}
\begin{tabular}{l}
|\ifchilddocmanual|\\
|\input{\childdocname}|\\
|\||else|\\
\textit{document body with }|\input{|\textit{part}|}|\\
|\||fi|
\end{tabular}
\end{center}
%
The conditional |\ifchilddocmanual| is true whenever
a part to be included by |\input| is being compiled,
and the name of the part is stored in |\childdocname|.

%%%%%%%%%%%%%%%%%%%%%%%%%%%%%%%%%%%%%%%%
\DescribeMacro{\childdocby}
Each part to be included by |\input| should start with:
%
\begin{center}
\begin{tabular}{l}
|\input{childdoc.def}|\\
|\childdocby{|\textit{main}|}|\\
\end{tabular}
\end{center}
%
The directive |\childdocby| is similar to |\childdocof|
described in \secref{sec:include},
but the subsequent selection of content must be done manually.
To that end, both |\ifchilddoc| and |\ifchilddocmanual|
will be true upon processing of a part,
and the name of the part is stored in |\childdocname|.
Note that |\jobname| will be set to the filename of the current part
so that each part receives an individual |.aux| file
that does not interfere with the |.aux| file(s) of the main document.
This behaviour can be altered by the alternative form
|\childdocby[*]{|\textit{main}|}| (with a non-empty optional argument)
which uses the |.aux| file of the main document
by setting |\jobname| to \textit{main}.

%%%%%%%%%%%%%%%%%%%%%%%%%%%%%%%%%%%%%%%%%%%%%%%%%%%%%%%%%%%%%%%%%%%%%%%%%%%%%%%%
\subsection{Driver Development}
\label{sec:driver}

The \textsf{childdoc} mechanism can also be use for the development
of definition files such as \LaTeX{} styles or classes.
This case differs from the above setup with multiple parts
included by |\include| in that no |\includeonly| should be invoked.
This can be achieved by starting the include file
(before |\ProvidesPackage|) with:
%
\begin{center}
\begin{tabular}{l}
|\input{childdoc.def}|\\
|\childdocforward{|\textit{main}|}|\\
\end{tabular}
\end{center}
%
or alternatively with:
%
\begin{center}
\begin{tabular}{l}
|\input{childdoc.def}|\\
|\childdocby{|\textit{main}|}|\\
\end{tabular}
\end{center}
%
Both forms have slightly different effects as described above.
The main file is prepared as usual, see \secref{sec:include}.

%%%%%%%%%%%%%%%%%%%%%%%%%%%%%%%%%%%%%%%%%%%%%%%%%%%%%%%%%%%%%%%%%%%%%%%%%%%%%%%%
\subsection{Legacy Detection}
\label{sec:detection}

The directive |\childdocmain| in the main file can detect
whether the complete document or merely a child is to be compiled
even without using the directive |\childdocof|.
This method is deprecated because it is less robust
and there is no compelling reason to use it;
it is merely provided for backward compatibility
and it may be removed in future versions.

If the detection mechanism is to be used,
it is mandatory to correctly specify
the filename of the main file as the argument of |\childdocmain|:
%
\begin{center}
\begin{tabular}{l}
|\input{childdoc.def}|\\
|\childdocmain{|\textit{main}|}|\\
\end{tabular}
\end{center}
%
If |\jobname| does not match the argument \textit{main} of |\childdocmain|,
it is assumed that |\jobname| points to the child file to be compiled.
When using |\childdocmain| with the main file specified as argument,
it suffices to start a child file
with just |\input{|\textit{main}|}|
without loading of the package and using |\childdocof|.
If instead all processing is done
with the appropriate \textsf{childdoc} directives,
the argument of \textit{main} of |\childdocmain| can be empty.

An alternative version of the command line processing described
in \secref{sec:commandline} using the detection mechanism reads:
%
\begin{center}
|... -jobname "|\textit{target}|" "|[\textit{flags}]%
[|\def\jobname{|\textit{dest}|}|]|\input{|\textit{main}|}"|
\end{center}

%%%%%%%%%%%%%%%%%%%%%%%%%%%%%%%%%%%%%%%%%%%%%%%%%%%%%%%%%%%%%%%%%%%%%%%%%%%%%%%%
\subsection{Manual Code}
\label{sec:manual}

In case one cannot be certain whether the definitions file |childdoc.def|
is installed on the target \TeX{} distribution
and one prefers not to ship it,
it is conceivable to paste a few relevant commands into the sources.

To that end, drop all statements |\input{childdoc.def}|
and perform the replacements as outlined below.
Instead of |\childdocmain{|\textit{main}|}| add the following code
to the top of the main file:
%
\begin{center}
\begin{tabular}{l}
|\||ifdefined\childdocname\endinput\||fi\newif\ifchilddoc|\\
|\edef\childdocname{\scantokens\expandafter{\jobname\noexpand}}|\\
|\def\childdocmain{|\textit{main}|}\||ifx\childdocmain\childdocname\||else|\\
|\childdoctrue\includeonly{\childdocname}\let\jobname\childdocmain\||fi|\\
\end{tabular}
\end{center}
%
Instead of |\childdocof{|\textit{main}|}| just include the main file
at the top of each child file:
%
\begin{center}
|\input{|\textit{main}|}|
\end{center}
%
A simple redirection |\childdocforward{|\textit{dest}|}| is achieved by:
%
\begin{center}
|\def\jobname{|\textit{dest}|}\input{\jobname}|
\end{center}
%
The redirection with prefix
|\childdocforwardprefix[|\textit{prefix}|]{|\textit{dest}|}|
is accomplished by:
%
\begin{center}
\begin{tabular}{l}
|{\edef\jobname{\scantokens\expandafter{\jobname\noexpand}}|\\
|\def\redirectjob |\textit{prefix}|#1~~~{\gdef\jobname{|\textit{dest}|#1}}|\\
|\expandafter\redirectjob\jobname~~~}\input{\jobname}|
\end{tabular}
\end{center}

In an alternative approach,
child documents can be compiled by a specific command line
without additional code or specific definitions:
%
\begin{center}
|... -jobname "|\textit{target}|" "|[\textit{flags}]%
|\includeonly{|\textit{dest}|}\input{|\textit{main}|}"|
\end{center}
%

%%%%%%%%%%%%%%%%%%%%%%%%%%%%%%%%%%%%%%%%%%%%%%%%%%%%%%%%%%%%%%%%%%%%%%%%%%%%%%%%
%%%%%%%%%%%%%%%%%%%%%%%%%%%%%%%%%%%%%%%%%%%%%%%%%%%%%%%%%%%%%%%%%%%%%%%%%%%%%%%%
\section{Information}

%%%%%%%%%%%%%%%%%%%%%%%%%%%%%%%%%%%%%%%%%%%%%%%%%%%%%%%%%%%%%%%%%%%%%%%%%%%%%%%%
\subsection{Copyright}

Copyright \copyright{} 2017--2018 Niklas Beisert

This work may be distributed and/or modified under the
conditions of the \LaTeX{} Project Public License, either version 1.3
of this license or (at your option) any later version.
The latest version of this license is in
  \url{http://www.latex-project.org/lppl.txt}
and version 1.3 or later is part of all distributions of \LaTeX{}
version 2005/12/01 or later.

This work has the LPPL maintenance status `maintained'.

The Current Maintainer of this work is Niklas Beisert.

This work consists of the files |README.txt|, |childdoc.ins| and |childdoc.dtx|
as well as the derived files |childdoc.def|, |cdocsamp.tex|
with |cdocsch1.tex|, |cdocsch2.tex|, |cdocspt3.tex|, |cdocspt4.tex|,
|cdocsdrf.tex|, |cdocsfn1.tex|, |cdocsfn2.tex|
as well as |childdoc.pdf|.

%%%%%%%%%%%%%%%%%%%%%%%%%%%%%%%%%%%%%%%%%%%%%%%%%%%%%%%%%%%%%%%%%%%%%%%%%%%%%%%%
\subsection{Files and Installation}

The package consists of the files:
%
\begin{center}
\begin{tabular}{ll}
    |README.txt|   & readme file \\
    |childdoc.ins| & installation file \\
    |childdoc.dtx| & source file \\
    |childdoc.def| & definition file \\
    |cdocsamp.tex| & sample main file \\
    |cdocsch1.tex| & sample include file \\
    |cdocsch2.tex| & sample include file \\
    |cdocspt3.tex| & sample part file \\
    |cdocspt4.tex| & sample part file \\
    |cdocsdrf.tex| & sample redirection file \\
    |cdocsfn1.tex| & sample redirection file \\
    |cdocsfn2.tex| & sample redirection file \\
    |childdoc.pdf| & manual
\end{tabular}
\end{center}
%
The distribution consists of the files
|README.txt|, |childdoc.ins| and |childdoc.dtx|.
%
\begin{itemize}
\item
Run (pdf)\LaTeX{} on |childdoc.dtx|
to compile the manual |childdoc.pdf| (this file).
\item
Run \LaTeX{} on |childdoc.ins| to create the definitions file |childdoc.def|
and the sample |cdocsamp.tex| with include files
|cdocsch1.tex|, |cdocsch2.tex|, |cdocspt3.tex|, |cdocspt4.tex|,
|cdocsdrf.tex|, |cdocsfn1.tex|, |cdocsfn2.tex|.
Then copy the file |childdoc.def| to an appropriate directory of your \LaTeX{}
distribution, e.g.\ \textit{texmf-root}|/tex/latex/childdoc|.
\end{itemize}

%%%%%%%%%%%%%%%%%%%%%%%%%%%%%%%%%%%%%%%%%%%%%%%%%%%%%%%%%%%%%%%%%%%%%%%%%%%%%%%%
\subsection{Related CTAN Packages}

There are several other packages which offer a similar functionality:
%
\begin{itemize}
\item
The packages
\href{http://ctan.org/pkg/docmute}{\textsf{docmute}},
\href{http://ctan.org/pkg/includex}{\textsf{includex}} and
\href{http://ctan.org/pkg/standalone}{\textsf{standalone}}
provide commands to include only the document body of
a child file thus allowing both files to be compiled individually.
\item
The packages \href{http://ctan.org/pkg/subdocs}{\textsf{subdocs}}
and \href{http://ctan.org/pkg/subfiles}{\textsf{subfiles}}
provide structures in which the main and child documents can be
encapsulated and allowing them to be compiled individually.
The inclusion mechanism is different from the conventional |\include|.
\item
The package \href{http://ctan.org/pkg/combine}{\textsf{combine}}
is an elaborate solution to combine several documents into one.
\end{itemize}
%
See also the CTAN topic \href{http://ctan.org/topic/subdocs}{\textsf{subdocs}}
for further related packages.
The present package differs from the above solutions in that
a document structure constructed with the conventional |\include| mechanism
just needs two extra commands at the top of every file
such that all constituent files can be compiled individually.

%%%%%%%%%%%%%%%%%%%%%%%%%%%%%%%%%%%%%%%%%%%%%%%%%%%%%%%%%%%%%%%%%%%%%%%%%%%%%%%%
%\subsection{Feature Suggestions}
%
%The following is a list of features which may be useful for future
%versions of this package:
%%
%\begin{itemize}
%\item
%\ldots
%\end{itemize}

%%%%%%%%%%%%%%%%%%%%%%%%%%%%%%%%%%%%%%%%%%%%%%%%%%%%%%%%%%%%%%%%%%%%%%%%%%%%%%%%
\subsection{Revision History}

%%%%%%%%%%%%%%%%%%%%%%%%%%%%%%%%%%%%%%%%
\paragraph{v2.0:} 2018/12/30

\begin{itemize}
\item
immediate forward processing
\item
added |\childdocby| mechanism
\item
manual restructured
\end{itemize}

%%%%%%%%%%%%%%%%%%%%%%%%%%%%%%%%%%%%%%%%
\paragraph{v1.6:} 2018/01/17

\begin{itemize}
\item
application for development of include files
\item
corrections to manual
\end{itemize}

%%%%%%%%%%%%%%%%%%%%%%%%%%%%%%%%%%%%%%%%
\paragraph{v1.5:} 2017/05/21

\begin{itemize}
\item
more complete structuring introduced
\item
|\childdocof| introduced
\item
|\childdoc| renamed to |\childdocmain|
\item
|\childredirect| renamed to |\childdocforward| and |\childdocforwardprefix|
and functionality expanded
\end{itemize}

%%%%%%%%%%%%%%%%%%%%%%%%%%%%%%%%%%%%%%%%
\paragraph{v1.0:} 2017/04/27

\begin{itemize}
\item
manual and install package
\item
first version published on CTAN
\end{itemize}

%%%%%%%%%%%%%%%%%%%%%%%%%%%%%%%%%%%%%%%%
\paragraph{v0.6:} 2017/04/26

\begin{itemize}
\item
redirection mechanism added
\end{itemize}

%%%%%%%%%%%%%%%%%%%%%%%%%%%%%%%%%%%%%%%%
\paragraph{v0.5:} 2017/04/26

\begin{itemize}
\item
functionality in definition file
\end{itemize}


%%%%%%%%%%%%%%%%%%%%%%%%%%%%%%%%%%%%%%%%%%%%%%%%%%%%%%%%%%%%%%%%%%%%%%%%%%%%%%%%
%%%%%%%%%%%%%%%%%%%%%%%%%%%%%%%%%%%%%%%%%%%%%%%%%%%%%%%%%%%%%%%%%%%%%%%%%%%%%%%%
%%%%%%%%%%%%%%%%%%%%%%%%%%%%%%%%%%%%%%%%%%%%%%%%%%%%%%%%%%%%%%%%%%%%%%%%%%%%%%%%
\appendix

\settowidth\MacroIndent{\rmfamily\scriptsize 000\ }

 \DocInput{childdoc.dtx}

\end{document}
%</driver>
% \fi
%
% %%%%%%%%%%%%%%%%%%%%%%%%%%%%%%%%%%%%%%%%%%%%%%%%%%%%%%%%%%%%%%%%%%%%%%%%%%%%%%
% %%%%%%%%%%%%%%%%%%%%%%%%%%%%%%%%%%%%%%%%%%%%%%%%%%%%%%%%%%%%%%%%%%%%%%%%%%%%%%
% \section{Sample}
%\iffalse
%<*samplemain>
%\fi
%
% The following presents a sample document
% with two chapters, two parts, a title page,
% a compile flag as well as three forwarding files to set the flag.
% It consists of eight |.tex| files:
% \begin{center}
% \begin{tabular}{ll}
% |cdocsamp.tex|&main file\\
% |cdocsch1.tex|&include file for chapter 1\\
% |cdocsch2.tex|&include file for chapter 2\\
% |cdocspt3.tex|&include file for part 3\\
% |cdocspt4.tex|&include file for part 4\\
% |cdocsdrf.tex|&forwarding file for main file in draft mode\\
% |cdocsfi1.tex|&forwarding file for final version of chapter 1\\
% |cdocsfi2.tex|&forwarding file for final version of chapter 2\\
% \end{tabular}
% \end{center}
% Each of the eight files can be compiled directly by the \LaTeX{} compiler.
%
% %%%%%%%%%%%%%%%%%%%%%%%%%%%%%%%%%%%%%%
% \paragraph{Main File.}
%
% The main file is called |cdocsamp.tex|.
%
% Load the \textsf{childdoc} definitions and
% declare the filename for the main document:
%    \begin{macrocode}
\input{childdoc.def}
\childdocmain{}
%    \end{macrocode}

% Optional override for |\version| flag:
%    \begin{macrocode}
%%\ifchilddoc\else\providecommand{\version}{draft}\fi
%    \end{macrocode}

% Define the default values for the |\version| flag
% (|final| for the main file and |draft| for childs):
%    \begin{macrocode}
\ifchilddoc
\providecommand{\version}{draft}
\else
\providecommand{\version}{final}
\fi
%    \end{macrocode}

% Load the standard document class:
%    \begin{macrocode}
\documentclass[12pt]{article}
%    \end{macrocode}

% Start the document body:
%    \begin{macrocode}
\begin{document}
%    \end{macrocode}

% Declare a title page.
% Print title, part of document being processed and version flag:
%    \begin{macrocode}
\addtocounter{page}{-1}
\begin{center}
{\LARGE\bfseries{}childdoc example\par}
\vspace{1cm}
\ifchilddoc
\ifchilddocmanual part\else chapter\fi:
`\childdocname' of `\childdocjob'\par
\else
main document: `\childdocjob'\par
\fi
version: \version\par
\end{center}
\newpage
%    \end{macrocode}

% Manually include selected file,
% otherwise process as usual:
%    \begin{macrocode}
\ifchilddocmanual
\section*{part `\childdocname'}
\input{\childdocname}
\else
%    \end{macrocode}

% Include the two chapters:
%    \begin{macrocode}
\include{cdocsch1}
\include{cdocsch2}
%    \end{macrocode}

% Include the two parts unless only chapters should be displayed:
%    \begin{macrocode}
\ifchilddoc\else
\section{part three}
\input{cdocspt3}
\section{part four}
\input{cdocspt4}
\fi
%    \end{macrocode}

% Process as usual until here:
%    \begin{macrocode}
\fi
%    \end{macrocode}

% End of document body:
%    \begin{macrocode}
\end{document}
%    \end{macrocode}
%\iffalse
%</samplemain>
%\fi
%
% %%%%%%%%%%%%%%%%%%%%%%%%%%%%%%%%%%%%%%
% \paragraph{Chapter Include Files.}
%
% The include files are called |cdocsch1.tex| and |cdocsch2.tex|.
%
%\iffalse
%<*samplechap1|samplechap2>
%\fi

% Optional override for |\version| flag:
%    \begin{macrocode}
%%\providecommand{\version}{final}
%    \end{macrocode}

% Include the main document:
%    \begin{macrocode}
\input{childdoc.def}
\childdocof{cdocsamp}
%    \end{macrocode}

%\iffalse
%</samplechap1|samplechap2>
%\fi
%
%\iffalse
%<*samplechap1>
%\fi
% Some text for chapter 1:
%    \begin{macrocode}
\section{one}
some text in chapter one
%    \end{macrocode}

%\iffalse
%</samplechap1>
%\fi
% Some text for chapter 2:
%\iffalse
%<*samplechap2>
%\fi
%    \begin{macrocode}
\section{two}
more text in chapter two
%    \end{macrocode}

%\iffalse
%</samplechap2>
%\fi
%
% %%%%%%%%%%%%%%%%%%%%%%%%%%%%%%%%%%%%%%
% \paragraph{Part Include Files.}
%
% The include files are called |cdocspt3.tex| and |cdocspt4.tex|.
%
%\iffalse
%<*samplepart3|samplepart4>
%\fi

% Optional override for |\version| flag:
%    \begin{macrocode}
%%\providecommand{\version}{final}
%    \end{macrocode}

% Include the main document:
%    \begin{macrocode}
\input{childdoc.def}
\childdocby{cdocsamp}
%    \end{macrocode}

%\iffalse
%</samplepart3|samplepart4>
%\fi
%
%\iffalse
%<*samplepart3>
%\fi
% Some text for part 3:
%    \begin{macrocode}
some text in part three
%    \end{macrocode}

%\iffalse
%</samplepart3>
%\fi
% Some text for part 4:
%\iffalse
%<*samplepart4>
%\fi
%    \begin{macrocode}
more text in part four
%    \end{macrocode}

%\iffalse
%</samplepart4>
%\fi
%
% %%%%%%%%%%%%%%%%%%%%%%%%%%%%%%%%%%%%%%
% \paragraph{Forwarding for a Complete Draft.}
%
% The following forwarding file |cdocsdrf.tex|
% compiles the main document in draft mode:
%\iffalse
%<*sampledraft>
%\fi
%    \begin{macrocode}
\def\version{draft}
\input{childdoc.def}
\childdocforward{cdocsamp}
%    \end{macrocode}

%\iffalse
%</sampledraft>
%\fi
%
% %%%%%%%%%%%%%%%%%%%%%%%%%%%%%%%%%%%%%%
% \paragraph{Forwarding for Final Version of the Chapters.}
%
% The following forwarding files |cdocsfn1.tex| and |cdocsfn2.tex|
% (with identical content)
% compile the final versions of the child documents
% |cdocsch1.tex| and |cdocsch2.tex|, respectively:
%\iffalse
%<*samplefinal>
%\fi
%    \begin{macrocode}
\def\version{final}
\input{childdoc.def}
\childdocforwardprefix[cdocsamp]{cdocsfn}{cdocsch}
%    \end{macrocode}

%\iffalse
%</samplefinal>
%\fi
%
% %%%%%%%%%%%%%%%%%%%%%%%%%%%%%%%%%%%%%%
% \paragraph{Command Line Processing.}
%
% The following three command lines generate the output files
% |cdocscld|, |cdocscl1| and |cdocscl2|
% which should be identical to
% |cdocsdrf|, |cdocsch1| and |cdocsfn2|, respectively:
% \begin{center}
% \begin{tabular}{l}
% |latex -jobname cdocscld \|\\
% |  "\def\version{draft}\input{childdoc.def}\childdocforward{cdocsamp}"|\\
% |latex -jobname cdocscl1 \|\\
% |  "\input{childdoc.def}\childdocforward[cdocsamp]{cdocsch1}"|\\
% |latex -jobname cdocscl2 \|\\
% |  "\def\version{final}\input{childdoc.def}\childdocforward{cdocsch2}"|
% \end{tabular}
% \end{center}
% Note that the trailing backslash on each first line
% merely continues the input to the second line
% (for convenient cut ant paste).
% Furthermore, the command |latex| can be replaced by any
% of its alternative versions such as |pdflatex|.
%
% %%%%%%%%%%%%%%%%%%%%%%%%%%%%%%%%%%%%%%%%%%%%%%%%%%%%%%%%%%%%%%%%%%%%%%%%%%%%%%
% %%%%%%%%%%%%%%%%%%%%%%%%%%%%%%%%%%%%%%%%%%%%%%%%%%%%%%%%%%%%%%%%%%%%%%%%%%%%%%
% \section{Implementation}
%\iffalse
%<*package>
%\fi
%
% This section describes the definitions file |childdoc.def|.

% The definitions cannot be loaded using |\usepackage| or |\RequirePackage|
% which has a mechanism to prevent loading a style file more than once.
% When loading the definitions by means of |\input|
% multiple instances have to be prevented manually:
%\iffalse
%This code needs to be before the `\ProvidesFile' directive
%which is defined at the beginning of this file.
%Therefore it is also placed there and commented out here.
%</package>
%<*discard>
%\fi
%    \begin{macrocode}
\ifdefined\childdocmain\endinput\fi
%    \end{macrocode}
%\iffalse
%</discard>
%<*package>
%\fi
%
% \macro{\ifchilddoc}
% \macro{\ifchilddocmanual}
% The conditional |\ifchilddoc| tells whether a
% child (true) or main (false) document is being compiled.
% The conditional |\ifchilddocmanual| tells whether
% the |\includeonly| mechanism is used (false) or
% the selection of child files must be performed manually (true).
% The definitions initialise to false:
%    \begin{macrocode}
\newif\ifchilddoc
\newif\ifchilddocmanual
%    \end{macrocode}

% \macro{\childdocname}
% \macro{\childdocjob}
% The macro |\childdocname| stores the name of the main document
% to be compiled. The macro |\childdocjob| stores the name of
% the document on which the \LaTeX{} compiler was originally invoked.
% The content of |\jobname| cannot be compared
% to filenames specified in the source due to different catcodes.
% The following code rescans |\jobname|, stores the result
% in |\childdocname| and saves a copy in |\childdocjob|:
%    \begin{macrocode}
\edef\childdocname{\scantokens\expandafter{\jobname\noexpand}}
\let\childdocjob\childdocname
%    \end{macrocode}

% \macro{\childdocdisable}
% The macro |\childdocdisable| prevents the main file
% from being processed more than once.
% At this stage, the main document command |\childdocmain|
% is assumed to be called once again where it should do nothing.
% Any subsequent call to it should prevent
% a secondary processing of the main document
% It overwrites the forwarding commands
% |\childdocof| and |\childdocforward|
% with empty macros to prevent further inclusions of the main document:
%    \begin{macrocode}
\newcommand{\childdocdisable}
{
  \renewcommand{\childdocmain}[1]{\renewcommand{\childdocmain}[1]{\endinput}}
  \renewcommand{\childdocof}[1]{}
  \renewcommand{\childdocby}[2][]{}
  \renewcommand{\childdocforward}[2][]{}
  \renewcommand{\childdocdisable}{}
}
%    \end{macrocode}

% \macro{\childdocmain}
% The macro |\childdocmain| is to be called at the top of the main file
% with nothing or the main filename (without extension) as argument.
% First, it breaks loops.
% If the argument is not empty and does not match |\childdocname|
% (which is set by the first inclusion of |childdoc.def|),
% |\ifchilddoc| is set to true, |\includeonly| is applied to the child file
% and |\jobname| is set to the main file
% (for proper handling of |.aux| files):
%    \begin{macrocode}
\newcommand{\childdocmain}[1]
{
  \childdocdisable\childdocmain{}
  \if?#1?\else
    \begingroup
      \def\childdoctmp{#1}
      \ifx\childdoctmp\childdocname
        \def\childdoctmp{}
      \else
        \def\childdoctmp
        {
          \childdoctrue
          \includeonly{\childdocname}
          \def\childdocjob{#1}
          \def\jobname{#1}
        }
      \fi
      \expandafter
    \endgroup
    \childdoctmp
  \fi
}
%    \end{macrocode}

% \macro{\childdocof}
% The command |\childdocof| redirects
% compilation to the main file |#1|.
%    \begin{macrocode}
\newcommand{\childdocof}[1]
{
  \childdocdisable
  \childdoctrue
  \includeonly{\childdocname}
  \def\jobname{#1}
  \def\childdocjob{#1}
  \input{#1}
}
%    \end{macrocode}

% \macro{\childdocby}
% The command |\childdocby| ....
%    \begin{macrocode}
\newcommand{\childdocby}[2][]
{
  \childdocdisable
  \childdoctrue
  \childdocmanualtrue
  \if?#1?\else
    \def\jobname{#2}
  \fi
  \def\childdocjob{#2}
  \input{#2}
  \endinput
}
%    \end{macrocode}

% \macro{\childdocforward}
% The command |\childdocforward| redirects
% compilation to the main file or
% (if the optional argument is given) a child file.
% Parameters are set as if the main file
% or a child file starting with |\childdocof| was compiled.
% Then compilation is handed over to the main file:
%    \begin{macrocode}
\newcommand{\childdocforward}[2][]
{
  \begingroup
    \if?#1?
      \def\childdoctmp
      {
        \def\childdocname{#2}
        \def\childdocjob{#2}
        \def\jobname{#2}
        \input{#2}
        \endinput
      }
    \else
      \def\childdoctmp
      {
        \childdocdisable
        \def\childdocname{#2}
        \childdoctrue
        \includeonly{#2}
        \def\childdocjob{#1}
        \def\jobname{#1}
        \input{#1}
        \endinput
      }
    \fi
    \expandafter
  \endgroup
  \childdoctmp
}
%    \end{macrocode}

% \macro{\childdocforwardprefix}
% The command |\childdocforwardprefix| redirects
% compilation to the main or a child file by means of a pattern.
% The prefix |#1| in the current filename is replaced by |#2|
% and the suffix of the current filename is kept
% (it is assumed that the filename does not contain the substring `|~~~|'
% which is used as a delimiter).
% Compilation is handed over to the new file by |\childdocforward|:
%    \begin{macrocode}
\newcommand{\childdocforwardprefix}[3][]
{
  \begingroup
    \def\childdocextract #2##1~~~{\def\childdoctmp{\childdocforward[#1]{#3##1}}}
    \expandafter\childdocextract\childdocname~~~
    \expandafter
  \endgroup
  \childdoctmp
}
%    \end{macrocode}

% \macro{\childdoc}
% The deprecated macro |\childdoc| is a legacy version of |\childdocmain|:
%    \begin{macrocode}
\newcommand{\childdoc}{\childdocmain}
%    \end{macrocode}

% \macro{\childdocredirect}
% The deprecated macro |\childdocredirect| is a legacy version
% of |\childdocforward| and |\childdocforwardprefix|:
%    \begin{macrocode}
\newcommand{\childdocredirect}[2][]
{
  \begingroup
    \if?#1?
      \def\childdoctmp{\childdocforward{#2}}
    \else
      \def\childdoctmp{\childdocforwardprefix{#1}{#2}}
    \fi
    \expandafter
  \endgroup
  \childdoctmp
}
%    \end{macrocode}

%\iffalse
%</package>
%\fi
%
\endinput
|\\
|\childdocforwardprefix{final}{child}|
\end{tabular}
\end{center}
%

Note that when several versions of a main file and/or of each child file
are to be generated, it may be convenient to set up a |Makefile| or
shell script to automatise the process.

%%%%%%%%%%%%%%%%%%%%%%%%%%%%%%%%%%%%%%%%%%%%%%%%%%%%%%%%%%%%%%%%%%%%%%%%%%%%%%%%
\subsection{Command Line Processing}
\label{sec:commandline}

The effect of redirection files can also be achieved by invoking
the \LaTeX{} compiler with a more elaborate command line.
Most conveniently this should be done as part
of a shell script or a |Makefile|.

When using \textsf{childdoc} in the main file, the following
command lines effectively perform a redirection
(note that depending on the shell being used,
backslashes may have to be doubled: `|\|' $\to$ `|\\|'):
%
\begin{center}
|... -jobname "|\textit{target}|" |\\|"|[\textit{flags}]%
|% \iffalse
%
% childdoc.dtx Copyright (C) 2017-2018 Niklas Beisert
%
% This work may be distributed and/or modified under the
% conditions of the LaTeX Project Public License, either version 1.3
% of this license or (at your option) any later version.
% The latest version of this license is in
%   http://www.latex-project.org/lppl.txt
% and version 1.3 or later is part of all distributions of LaTeX
% version 2005/12/01 or later.
%
% This work has the LPPL maintenance status `maintained'.
%
% The Current Maintainer of this work is Niklas Beisert.
%
% This work consists of the files childdoc.dtx and childdoc.ins
% and the derived files childdoc.def and cdocsamp.tex with
% cdocsch1.tex, cdocsch2.tex, cdocsdrf.tex, cdocsfn1.tex, cdocsfn2.tex.
%
%<package>\ifdefined\childdocmain\endinput\fi
%<package>\ProvidesFile{childdoc.def}[2018/12/30 v2.0 child document driver]
%<samplemain>\ProvidesFile{cdocsamp.tex}[2018/12/30 v2.0 sample for childdoc]
%<*driver>
%\ProvidesFile{childdoc.drv}[2018/12/30 v2.0 childdoc reference manual file]
\PassOptionsToClass{10pt,a4paper}{article}
\documentclass{ltxdoc}

\usepackage[margin=35mm]{geometry}
\usepackage{hyperref}
\usepackage{hyperxmp}
\usepackage[usenames]{color}

\hypersetup{colorlinks=true}
\hypersetup{pdfstartview=FitH}
\hypersetup{pdfpagemode=UseNone}
\hypersetup{pdfsource={}}
\hypersetup{pdflang={en-UK}}
\hypersetup{pdfcopyright={Copyright 2017-2018 Niklas Beisert.
  This work may be distributed and/or modified under the
  conditions of the LaTeX Project Public License, either version 1.3
  of this license or (at your option) any later version.}}
\hypersetup{pdflicenseurl={http://www.latex-project.org/lppl.txt}}
\hypersetup{pdfcontactaddress={ETH Zurich, ITP, HIT K,
  Wolfgang-Pauli-Strasse 27}}
\hypersetup{pdfcontactpostcode={8093}}
\hypersetup{pdfcontactcity={Zurich}}
\hypersetup{pdfcontactcountry={Switzerland}}
\hypersetup{pdfcontactemail={nbeisert@itp.phys.ethz.ch}}
\hypersetup{pdfcontacturl={http://people.phys.ethz.ch/\xmptilde nbeisert/}}

\newcommand{\secref}[1]{\hyperref[#1]{section \ref*{#1}}}

\parskip1ex
\parindent0pt
\let\olditemize\itemize
\def\itemize{\olditemize\parskip0pt}

\begin{document}

\title{The \textsf{childdoc} Package}
\hypersetup{pdftitle={The childdoc Package}}
\author{Niklas Beisert\\[2ex]
  Institut f\"ur Theoretische Physik\\
  Eidgen\"ossische Technische Hochschule Z\"urich\\
  Wolfgang-Pauli-Strasse 27, 8093 Z\"urich, Switzerland\\[1ex]
  \href{mailto:nbeisert@itp.phys.ethz.ch}
  {\texttt{nbeisert@itp.phys.ethz.ch}}}
\hypersetup{pdfauthor={Niklas Beisert}}
\hypersetup{pdfsubject={Manual for the LaTeX2e Package childdoc}}
\date{30 December 2018, \textsf{v2.0}}
\maketitle

\begin{abstract}\noindent
\textsf{childdoc} is a \LaTeXe{} package
that enables the direct compilation
of document sections included by |\include|
to individual files.
\end{abstract}

\begingroup
\parskip0ex
\tableofcontents
\endgroup

%%%%%%%%%%%%%%%%%%%%%%%%%%%%%%%%%%%%%%%%%%%%%%%%%%%%%%%%%%%%%%%%%%%%%%%%%%%%%%%%
%%%%%%%%%%%%%%%%%%%%%%%%%%%%%%%%%%%%%%%%%%%%%%%%%%%%%%%%%%%%%%%%%%%%%%%%%%%%%%%%
\section{Introduction}

\LaTeX{} provides a mechanism to structure a large document (such as a book)
into a main file and several child files (containing the chapters)
using the |\include| command.
This mechanism is beneficial for documents
which span hundreds of pages in order to
make the source file(s) more manageable.
Moreover, compilation can be restricted to
selected child files by means of the |\includeonly| command.
The latter feature can be used to reduce the compilation time while editing
(this was significantly more useful in the earlier days of \LaTeX{})
or to generate a smaller document which is easier to navigate.
Another application of |\includeonly| is to generate
documents consisting of selected parts of the complete document.

However, there are a few drawbacks of the plain |\include| mechanism:
\begin{itemize}
\item
The child files cannot be compiled on their own,
they can only be compiled via the main file.
A naive editing environment
(such as a text editor with an option
to have the current file processed by \LaTeX)
may require one to switch to the main file before compiling;
attempting to compile the child file produces errors.
\item
The main file must be modified (each time)
to adjust the |\includeonly| command
to the present needs. This easily leaves the main file in a messy state.
\item
The generated document will always carry the filename
of the main document. This is inconvenient if
several child files are to be compiled and
to be kept for distribution.
\end{itemize}

The present package provides a simple interface
to make child files individually compilable by \LaTeX{}.
Compiling a child file then has the same effect as compiling
the main file with an |\includeonly| command
to select the appropriate child.
Moreover the generated document will carry the name of the child
rather than the main file.
This resolves all three above issues.

This feature is meant to make the editing of books,
thesis documents and lecture notes somewhat more convenient.
However, the package can also be used efficiently for
composing a series of documents (such as exercise sheets)
which are typically distributed individually.
It then assists the author in generating the individual documents
(potentially in different versions)
as well as a document containing the collected series.
Another application is in developing style files
or other kinds of included material
where compilation of the style file could redirect
to a sample or test file.

%%%%%%%%%%%%%%%%%%%%%%%%%%%%%%%%%%%%%%%%%%%%%%%%%%%%%%%%%%%%%%%%%%%%%%%%%%%%%%%%
%%%%%%%%%%%%%%%%%%%%%%%%%%%%%%%%%%%%%%%%%%%%%%%%%%%%%%%%%%%%%%%%%%%%%%%%%%%%%%%%
\section{Usage}

First of all, the package \textsf{childdoc} is \emph{not} a standard
\LaTeXe{} |.sty| style file! Therefore it needs to be invoked in
a non-standard way.

%%%%%%%%%%%%%%%%%%%%%%%%%%%%%%%%%%%%%%%%%%%%%%%%%%%%%%%%%%%%%%%%%%%%%%%%%%%%%%%%
\subsection{Included Files}
\label{sec:include}

%%%%%%%%%%%%%%%%%%%%%%%%%%%%%%%%%%%%%%%%
\DescribeMacro{\childdocmain}
To use the package, add the commands
\begin{center}
\begin{tabular}{l}
|\input{childdoc.def}|\\
|\childdocmain{}|\\
\end{tabular}
\end{center}
at the very top of the main \LaTeX{} file,
in particular \emph{before} the |\documentclass| statement!
The argument of |\childdocmain| should be left empty
(but it must be present).

%%%%%%%%%%%%%%%%%%%%%%%%%%%%%%%%%%%%%%%%
\DescribeMacro{\childdocof}
Furthermore, add the commands
\begin{center}
\begin{tabular}{l}
|\input{childdoc.def}|\\
|\childdocof{|\textit{main}|}|\\
\end{tabular}
\end{center}
at the top of every child file \textit{child}
which is included by |\include{|\textit{child}|}|
from within the main file
(or at least for those files to be compiled individually).
The argument \textit{main} must be the filename of the main file.

There are a couple of
considerations in setting up the main and child documents:

%%%%%%%%%%%%%%%%%%%%%%%%%%%%%%%%%%%%%%%%
\paragraph{Restrictions.}

Please note the following restrictions:
\begin{itemize}
\item
|\childdocmain| must be called with one argument \textit{main}
to ensure compatibility with earlier version of the package.
It must either be empty (|\childdocmain{}|)
or precisely match the filename of the main file in which it is specified.
See \secref{sec:detection} for further information.
\item
The filename \textit{main} must be specified without the |.tex| extension.
\item
The filename \textit{main} is case sensitive
(even in case-insensitive file systems)
due to internal string comparison.
\item
The argument \textit{main} should be fully expanded, it cannot be a macro.
\item
Subdirectories and special characters should be avoided in filenames.
\item
The command |\childdocmain{|\textit{main}|}| must be followed by a whitespace.
It should not be followed immediately by another command
or by a comment mark `|%|'.
This is because the \TeX{} parser reads the token immediately following
the argument of |\childdocmain| and puts it
at the beginning of every child section;
however, a white\-space is ignored.
\end{itemize}

%%%%%%%%%%%%%%%%%%%%%%%%%%%%%%%%%%%%%%%%
\paragraph{Content of Main File.}

It is advisable to place all content in the child files included by |\include|.
Any output contained in the main file will appear in all child documents
unless suppressed manually;
it cannot be suppressed automatically by the |\includeonly| directive
and thus should normally be avoided.
A method to include some content in the main file
by means of conditional processing is described in \secref{sec:conditional}.

%%%%%%%%%%%%%%%%%%%%%%%%%%%%%%%%%%%%%%%%
\paragraph{Page Numbering.}

When only a part of the document is compiled,
the appropriate numbering of pages
(as well as other status parameters)
is determined from the |.aux| files.
The latter contain information from previous passes.
However this information needs to propagate through
all intermediate child documents.
Therefore the page numbering in child documents may well
be inconsistent until the complete document is compiled at least once.

A useful (if unconventional) way to always ensure a consistent
page numbering is to restart the numbering in each child document
and denote the pages by `\textit{child}|.|\textit{page}'
where \textit{child} represents the chapter/section number of the child file.
This can be achieved by the command
|\numberwithin{page}{|\textit{child}|}|
of the \textsf{amsmath} package
where \textit{child} can be |chapter| or |section|
depending on the chosen structuring.
Alternatively, one can modify the macro |\thepage| appropriately
and reset the counter |page| at the start of each child file.

%%%%%%%%%%%%%%%%%%%%%%%%%%%%%%%%%%%%%%%%%%%%%%%%%%%%%%%%%%%%%%%%%%%%%%%%%%%%%%%%
\subsection{Conditional Processing}
\label{sec:conditional}

The package provides a mechanism to compile different versions
of a document. To customise the versions further some conditional processing
can come in handy to distinguish which version is being compiled.
The package provides two macros to describe the compilation context:

%%%%%%%%%%%%%%%%%%%%%%%%%%%%%%%%%%%%%%%%
\DescribeMacro{\ifchilddoc}
The conditional |\ifchilddoc| distinguishes between the compilation of
child documents and the main document:
%
\begin{center}
|\ifchilddoc |\textit{child-code}| |[|\||else |\textit{main-code}]| \||fi|
\end{center}

%%%%%%%%%%%%%%%%%%%%%%%%%%%%%%%%%%%%%%%%
\DescribeMacro{\childdocname}
\DescribeMacro{\childdocjob}
The macro |\childdocname| contains the filename (without extension)
of the main or child file being processed.
Note that |\childdocjob| will always contain the name of the main file.

%%%%%%%%%%%%%%%%%%%%%%%%%%%%%%%%%%%%%%%%
\paragraph{Title Page.}

Conditional processing can be used to include a title or banner page
in the main document when proper precautions are taken.
Importantly, the code in the main file should ensure that the page counter
(as well as other status parameters which are stored in the |.aux| files)
takes the same value after the conditional processing.
Otherwise the page numbers may take divergent values
depending on which part is compiled.

For example, a title page could be declared by:
%
\begin{center}
\begin{tabular}{l}
|\ifchilddoc\||else|\\
|\addtocounter{page}{-1}|\\
\textit{code for title page}\\
|\newpage|\\
|\||fi|
\end{tabular}
\end{center}
%
A banner page for the child documents can be generated by:
%
\begin{center}
\begin{tabular}{l}
|\ifchilddoc|\\
|\addtocounter{page}{-1}|\\
\textit{code for banner page}\\
|\newpage|\\
|\||fi|
\end{tabular}
\end{center}
%
Here one could write a message such as:
\begin{center}
|This is the part \childdocname{} of \childdocjob{}.|
\end{center}

%%%%%%%%%%%%%%%%%%%%%%%%%%%%%%%%%%%%%%%%%%%%%%%%%%%%%%%%%%%%%%%%%%%%%%%%%%%%%%%%
\subsection{Flags}
\label{sec:flags}

The package makes it easy to generate different versions
of the main or child documents.
To this end compilation flags can be defined
and assigned different default values.
They will be particularly useful in conjunction
with the forwarding mechanism described in \secref{sec:forward}.

For example, it may be useful to have a flag |\version|
which can be set to |draft| or |final|.
The document source will contain some conditional code
depending on the value of |\version|.
Suppose further, the flag should default to |final| for the main file
and to |draft| for child files
which is a natural assignment for editing the document.
This is achieved by placing the following code
in the preamble of the main document
(below the |\childdocmain| directive):
%
\begin{center}
\begin{tabular}{l}
|\ifchilddoc|\\
|\providecommand{\version}{draft}|\\
|\||else|\\
|\providecommand{\version}{final}|\\
|\||fi|
\end{tabular}
\end{center}
%
The definition by |\providecommand| makes sure
that previous definitions are not overwritten.
Further statements |\providecommand{\version}{...}|
can thus be added before the above code to override it.

For the main file, one might add a line
(between |\childdocmain| and the above block)
%
\begin{center}
|%\ifchilddoc\||else\providecommand{\version}{draft}\||fi|
\end{center}
%
which can be uncommented to produce a draft version.
Likewise one can add a line to the very top of a child file
(above the |\childdocof{|\textit{main}|}| directive)
%
\begin{center}
|%\providecommand{\version}{final}|
\end{center}
%
which can be uncommented to produce the final version of this child document.

%%%%%%%%%%%%%%%%%%%%%%%%%%%%%%%%%%%%%%%%%%%%%%%%%%%%%%%%%%%%%%%%%%%%%%%%%%%%%%%%
\subsection{Forwarding}
\label{sec:forward}

Different versions of the main or child documents
using compilation flags as described in \secref{sec:flags}
can be (permanently) stored in different files
for convenient compilation, viewing and distribution.
To this end, the package defines a command
to pass on compilation to a different file:

%%%%%%%%%%%%%%%%%%%%%%%%%%%%%%%%%%%%%%%%
\DescribeMacro{\childdocforward}
The command |\childdocforward| redirects processing to
another source file:
%
\begin{center}
\begin{tabular}{l}
|\input{childdoc.def}|\\
|\childdocforward[|\textit{main}|]{|\textit{dest}|}|\\
\end{tabular}
\end{center}
%
The argument \textit{dest} is the destination file
(without extension).
It should be the main file or one of the child files.
Note that further \textsf{childdoc} directives
such as |\childdocof| and |\childdocforward|
in the indicated file will be processed in this form.
The optional argument \textit{main}
passes on directly to the main file \textit{main}
while pretending to compile the child \textit{dest}.
This form behaves as if \textit{dest}
issues |\childdocof{|\textit{main}|}| right away,
and no further \textsf{childdoc} directives will be processed.

%%%%%%%%%%%%%%%%%%%%%%%%%%%%%%%%%%%%%%%%
\DescribeMacro{\...prefix}
In the alternative form |\childdocforwardprefix|,
%
\begin{center}
\begin{tabular}{l}
|\input{childdoc.def}|\\
|\childdocforwardprefix[|\textit{main}|]{|\textit{prefix}|}{|\textit{dest}|}|
\end{tabular}
\end{center}
%
the destination file is determined by a pattern
depending on the current file:
To make this work, the current file must be called
`{\textit{prefix}\hspace{0.2em}\textit{suffix}}'
with \textit{prefix} matching precisely the argument.
Processing is then passed on to the file
`{\textit{dest}\hspace{0.2em}\textit{suffix}}'.
Surely, the same effect is achieved by
directly specifying the
argument `{\textit{dest}\hspace{0.2em}\textit{suffix}}'
in the first form.
However, that requires to set up a different file
for each child. With the alternative form of the command
all these files can have exactly the same content
which simplifies setting them up and maintaining them.

For example, the following file |draft.tex|
with a compilation flag |\version| as described in \secref{sec:flags}
compiles the main document as a draft:
%
\begin{center}
\begin{tabular}{l}
|\def\version{draft}|\\
|\input{childdoc.def}|\\
|\childdocforward{|\textit{main}|}|
\end{tabular}
\end{center}
%
Likewise, the following files |final|\textit{nn}|.tex|
compile the final version of the child document
|child|\textit{nn}|.tex|:
%
\begin{center}
\begin{tabular}{l}
|\def\version{final}|\\
|\input{childdoc.def}|\\
|\childdocforwardprefix{final}{child}|
\end{tabular}
\end{center}
%

Note that when several versions of a main file and/or of each child file
are to be generated, it may be convenient to set up a |Makefile| or
shell script to automatise the process.

%%%%%%%%%%%%%%%%%%%%%%%%%%%%%%%%%%%%%%%%%%%%%%%%%%%%%%%%%%%%%%%%%%%%%%%%%%%%%%%%
\subsection{Command Line Processing}
\label{sec:commandline}

The effect of redirection files can also be achieved by invoking
the \LaTeX{} compiler with a more elaborate command line.
Most conveniently this should be done as part
of a shell script or a |Makefile|.

When using \textsf{childdoc} in the main file, the following
command lines effectively perform a redirection
(note that depending on the shell being used,
backslashes may have to be doubled: `|\|' $\to$ `|\\|'):
%
\begin{center}
|... -jobname "|\textit{target}|" |\\|"|[\textit{flags}]%
|\input{childdoc.def}\childdocforward[|\textit{main}|]{|\textit{dest}|}"|
\end{center}
%
Here \textit{target} is the name of the output file,
\textit{main} is the name of the main file
and \textit{dest} is the name of the main or child file to be processed
(all filenames without extensions).
The optional argument \textit{main} can be omitted
if \textit{main} matches \textit{dest}.
Optionally, compilation \textit{flags} can be defined via |\def| commands.
This command line makes the \TeX{} engine believe
it is compiling the file \textit{target}
whose content is specified as the latter parameter.
The provided code then forwards the processing to
\textit{main} or \textit{dest} as described in \secref{sec:forward}.

%%%%%%%%%%%%%%%%%%%%%%%%%%%%%%%%%%%%%%%%%%%%%%%%%%%%%%%%%%%%%%%%%%%%%%%%%%%%%%%%
\subsection{Include by Input}
\label{sec:input}

Including child documents by |\include| has some restrictions by design.
Most notably, the content of a child document always occupies
its own set of pages; pages cannot be shared between child documents.
Usually, this behaviour makes perfect sense
because each child document contain an essential part of the document.
However, in some situations it may be desirable to compose
a document from a collection of parts
without having mandatory page breaks between then.
For this case, the package
provides a mechanism to include parts
by |\input| which can also be processed individually.
However, by construction this mechanism
requires manual handling of the content to be output.

%%%%%%%%%%%%%%%%%%%%%%%%%%%%%%%%%%%%%%%%
\DescribeMacro{\ifchilddocmanual}
The main file should be prepared as usual, see \secref{sec:include}.
However, the document body must make a distinction
between processing of an individual part and of the main document, e.g.:
%
\begin{center}
\begin{tabular}{l}
|\ifchilddocmanual|\\
|\input{\childdocname}|\\
|\||else|\\
\textit{document body with }|\input{|\textit{part}|}|\\
|\||fi|
\end{tabular}
\end{center}
%
The conditional |\ifchilddocmanual| is true whenever
a part to be included by |\input| is being compiled,
and the name of the part is stored in |\childdocname|.

%%%%%%%%%%%%%%%%%%%%%%%%%%%%%%%%%%%%%%%%
\DescribeMacro{\childdocby}
Each part to be included by |\input| should start with:
%
\begin{center}
\begin{tabular}{l}
|\input{childdoc.def}|\\
|\childdocby{|\textit{main}|}|\\
\end{tabular}
\end{center}
%
The directive |\childdocby| is similar to |\childdocof|
described in \secref{sec:include},
but the subsequent selection of content must be done manually.
To that end, both |\ifchilddoc| and |\ifchilddocmanual|
will be true upon processing of a part,
and the name of the part is stored in |\childdocname|.
Note that |\jobname| will be set to the filename of the current part
so that each part receives an individual |.aux| file
that does not interfere with the |.aux| file(s) of the main document.
This behaviour can be altered by the alternative form
|\childdocby[*]{|\textit{main}|}| (with a non-empty optional argument)
which uses the |.aux| file of the main document
by setting |\jobname| to \textit{main}.

%%%%%%%%%%%%%%%%%%%%%%%%%%%%%%%%%%%%%%%%%%%%%%%%%%%%%%%%%%%%%%%%%%%%%%%%%%%%%%%%
\subsection{Driver Development}
\label{sec:driver}

The \textsf{childdoc} mechanism can also be use for the development
of definition files such as \LaTeX{} styles or classes.
This case differs from the above setup with multiple parts
included by |\include| in that no |\includeonly| should be invoked.
This can be achieved by starting the include file
(before |\ProvidesPackage|) with:
%
\begin{center}
\begin{tabular}{l}
|\input{childdoc.def}|\\
|\childdocforward{|\textit{main}|}|\\
\end{tabular}
\end{center}
%
or alternatively with:
%
\begin{center}
\begin{tabular}{l}
|\input{childdoc.def}|\\
|\childdocby{|\textit{main}|}|\\
\end{tabular}
\end{center}
%
Both forms have slightly different effects as described above.
The main file is prepared as usual, see \secref{sec:include}.

%%%%%%%%%%%%%%%%%%%%%%%%%%%%%%%%%%%%%%%%%%%%%%%%%%%%%%%%%%%%%%%%%%%%%%%%%%%%%%%%
\subsection{Legacy Detection}
\label{sec:detection}

The directive |\childdocmain| in the main file can detect
whether the complete document or merely a child is to be compiled
even without using the directive |\childdocof|.
This method is deprecated because it is less robust
and there is no compelling reason to use it;
it is merely provided for backward compatibility
and it may be removed in future versions.

If the detection mechanism is to be used,
it is mandatory to correctly specify
the filename of the main file as the argument of |\childdocmain|:
%
\begin{center}
\begin{tabular}{l}
|\input{childdoc.def}|\\
|\childdocmain{|\textit{main}|}|\\
\end{tabular}
\end{center}
%
If |\jobname| does not match the argument \textit{main} of |\childdocmain|,
it is assumed that |\jobname| points to the child file to be compiled.
When using |\childdocmain| with the main file specified as argument,
it suffices to start a child file
with just |\input{|\textit{main}|}|
without loading of the package and using |\childdocof|.
If instead all processing is done
with the appropriate \textsf{childdoc} directives,
the argument of \textit{main} of |\childdocmain| can be empty.

An alternative version of the command line processing described
in \secref{sec:commandline} using the detection mechanism reads:
%
\begin{center}
|... -jobname "|\textit{target}|" "|[\textit{flags}]%
[|\def\jobname{|\textit{dest}|}|]|\input{|\textit{main}|}"|
\end{center}

%%%%%%%%%%%%%%%%%%%%%%%%%%%%%%%%%%%%%%%%%%%%%%%%%%%%%%%%%%%%%%%%%%%%%%%%%%%%%%%%
\subsection{Manual Code}
\label{sec:manual}

In case one cannot be certain whether the definitions file |childdoc.def|
is installed on the target \TeX{} distribution
and one prefers not to ship it,
it is conceivable to paste a few relevant commands into the sources.

To that end, drop all statements |\input{childdoc.def}|
and perform the replacements as outlined below.
Instead of |\childdocmain{|\textit{main}|}| add the following code
to the top of the main file:
%
\begin{center}
\begin{tabular}{l}
|\||ifdefined\childdocname\endinput\||fi\newif\ifchilddoc|\\
|\edef\childdocname{\scantokens\expandafter{\jobname\noexpand}}|\\
|\def\childdocmain{|\textit{main}|}\||ifx\childdocmain\childdocname\||else|\\
|\childdoctrue\includeonly{\childdocname}\let\jobname\childdocmain\||fi|\\
\end{tabular}
\end{center}
%
Instead of |\childdocof{|\textit{main}|}| just include the main file
at the top of each child file:
%
\begin{center}
|\input{|\textit{main}|}|
\end{center}
%
A simple redirection |\childdocforward{|\textit{dest}|}| is achieved by:
%
\begin{center}
|\def\jobname{|\textit{dest}|}\input{\jobname}|
\end{center}
%
The redirection with prefix
|\childdocforwardprefix[|\textit{prefix}|]{|\textit{dest}|}|
is accomplished by:
%
\begin{center}
\begin{tabular}{l}
|{\edef\jobname{\scantokens\expandafter{\jobname\noexpand}}|\\
|\def\redirectjob |\textit{prefix}|#1~~~{\gdef\jobname{|\textit{dest}|#1}}|\\
|\expandafter\redirectjob\jobname~~~}\input{\jobname}|
\end{tabular}
\end{center}

In an alternative approach,
child documents can be compiled by a specific command line
without additional code or specific definitions:
%
\begin{center}
|... -jobname "|\textit{target}|" "|[\textit{flags}]%
|\includeonly{|\textit{dest}|}\input{|\textit{main}|}"|
\end{center}
%

%%%%%%%%%%%%%%%%%%%%%%%%%%%%%%%%%%%%%%%%%%%%%%%%%%%%%%%%%%%%%%%%%%%%%%%%%%%%%%%%
%%%%%%%%%%%%%%%%%%%%%%%%%%%%%%%%%%%%%%%%%%%%%%%%%%%%%%%%%%%%%%%%%%%%%%%%%%%%%%%%
\section{Information}

%%%%%%%%%%%%%%%%%%%%%%%%%%%%%%%%%%%%%%%%%%%%%%%%%%%%%%%%%%%%%%%%%%%%%%%%%%%%%%%%
\subsection{Copyright}

Copyright \copyright{} 2017--2018 Niklas Beisert

This work may be distributed and/or modified under the
conditions of the \LaTeX{} Project Public License, either version 1.3
of this license or (at your option) any later version.
The latest version of this license is in
  \url{http://www.latex-project.org/lppl.txt}
and version 1.3 or later is part of all distributions of \LaTeX{}
version 2005/12/01 or later.

This work has the LPPL maintenance status `maintained'.

The Current Maintainer of this work is Niklas Beisert.

This work consists of the files |README.txt|, |childdoc.ins| and |childdoc.dtx|
as well as the derived files |childdoc.def|, |cdocsamp.tex|
with |cdocsch1.tex|, |cdocsch2.tex|, |cdocspt3.tex|, |cdocspt4.tex|,
|cdocsdrf.tex|, |cdocsfn1.tex|, |cdocsfn2.tex|
as well as |childdoc.pdf|.

%%%%%%%%%%%%%%%%%%%%%%%%%%%%%%%%%%%%%%%%%%%%%%%%%%%%%%%%%%%%%%%%%%%%%%%%%%%%%%%%
\subsection{Files and Installation}

The package consists of the files:
%
\begin{center}
\begin{tabular}{ll}
    |README.txt|   & readme file \\
    |childdoc.ins| & installation file \\
    |childdoc.dtx| & source file \\
    |childdoc.def| & definition file \\
    |cdocsamp.tex| & sample main file \\
    |cdocsch1.tex| & sample include file \\
    |cdocsch2.tex| & sample include file \\
    |cdocspt3.tex| & sample part file \\
    |cdocspt4.tex| & sample part file \\
    |cdocsdrf.tex| & sample redirection file \\
    |cdocsfn1.tex| & sample redirection file \\
    |cdocsfn2.tex| & sample redirection file \\
    |childdoc.pdf| & manual
\end{tabular}
\end{center}
%
The distribution consists of the files
|README.txt|, |childdoc.ins| and |childdoc.dtx|.
%
\begin{itemize}
\item
Run (pdf)\LaTeX{} on |childdoc.dtx|
to compile the manual |childdoc.pdf| (this file).
\item
Run \LaTeX{} on |childdoc.ins| to create the definitions file |childdoc.def|
and the sample |cdocsamp.tex| with include files
|cdocsch1.tex|, |cdocsch2.tex|, |cdocspt3.tex|, |cdocspt4.tex|,
|cdocsdrf.tex|, |cdocsfn1.tex|, |cdocsfn2.tex|.
Then copy the file |childdoc.def| to an appropriate directory of your \LaTeX{}
distribution, e.g.\ \textit{texmf-root}|/tex/latex/childdoc|.
\end{itemize}

%%%%%%%%%%%%%%%%%%%%%%%%%%%%%%%%%%%%%%%%%%%%%%%%%%%%%%%%%%%%%%%%%%%%%%%%%%%%%%%%
\subsection{Related CTAN Packages}

There are several other packages which offer a similar functionality:
%
\begin{itemize}
\item
The packages
\href{http://ctan.org/pkg/docmute}{\textsf{docmute}},
\href{http://ctan.org/pkg/includex}{\textsf{includex}} and
\href{http://ctan.org/pkg/standalone}{\textsf{standalone}}
provide commands to include only the document body of
a child file thus allowing both files to be compiled individually.
\item
The packages \href{http://ctan.org/pkg/subdocs}{\textsf{subdocs}}
and \href{http://ctan.org/pkg/subfiles}{\textsf{subfiles}}
provide structures in which the main and child documents can be
encapsulated and allowing them to be compiled individually.
The inclusion mechanism is different from the conventional |\include|.
\item
The package \href{http://ctan.org/pkg/combine}{\textsf{combine}}
is an elaborate solution to combine several documents into one.
\end{itemize}
%
See also the CTAN topic \href{http://ctan.org/topic/subdocs}{\textsf{subdocs}}
for further related packages.
The present package differs from the above solutions in that
a document structure constructed with the conventional |\include| mechanism
just needs two extra commands at the top of every file
such that all constituent files can be compiled individually.

%%%%%%%%%%%%%%%%%%%%%%%%%%%%%%%%%%%%%%%%%%%%%%%%%%%%%%%%%%%%%%%%%%%%%%%%%%%%%%%%
%\subsection{Feature Suggestions}
%
%The following is a list of features which may be useful for future
%versions of this package:
%%
%\begin{itemize}
%\item
%\ldots
%\end{itemize}

%%%%%%%%%%%%%%%%%%%%%%%%%%%%%%%%%%%%%%%%%%%%%%%%%%%%%%%%%%%%%%%%%%%%%%%%%%%%%%%%
\subsection{Revision History}

%%%%%%%%%%%%%%%%%%%%%%%%%%%%%%%%%%%%%%%%
\paragraph{v2.0:} 2018/12/30

\begin{itemize}
\item
immediate forward processing
\item
added |\childdocby| mechanism
\item
manual restructured
\end{itemize}

%%%%%%%%%%%%%%%%%%%%%%%%%%%%%%%%%%%%%%%%
\paragraph{v1.6:} 2018/01/17

\begin{itemize}
\item
application for development of include files
\item
corrections to manual
\end{itemize}

%%%%%%%%%%%%%%%%%%%%%%%%%%%%%%%%%%%%%%%%
\paragraph{v1.5:} 2017/05/21

\begin{itemize}
\item
more complete structuring introduced
\item
|\childdocof| introduced
\item
|\childdoc| renamed to |\childdocmain|
\item
|\childredirect| renamed to |\childdocforward| and |\childdocforwardprefix|
and functionality expanded
\end{itemize}

%%%%%%%%%%%%%%%%%%%%%%%%%%%%%%%%%%%%%%%%
\paragraph{v1.0:} 2017/04/27

\begin{itemize}
\item
manual and install package
\item
first version published on CTAN
\end{itemize}

%%%%%%%%%%%%%%%%%%%%%%%%%%%%%%%%%%%%%%%%
\paragraph{v0.6:} 2017/04/26

\begin{itemize}
\item
redirection mechanism added
\end{itemize}

%%%%%%%%%%%%%%%%%%%%%%%%%%%%%%%%%%%%%%%%
\paragraph{v0.5:} 2017/04/26

\begin{itemize}
\item
functionality in definition file
\end{itemize}


%%%%%%%%%%%%%%%%%%%%%%%%%%%%%%%%%%%%%%%%%%%%%%%%%%%%%%%%%%%%%%%%%%%%%%%%%%%%%%%%
%%%%%%%%%%%%%%%%%%%%%%%%%%%%%%%%%%%%%%%%%%%%%%%%%%%%%%%%%%%%%%%%%%%%%%%%%%%%%%%%
%%%%%%%%%%%%%%%%%%%%%%%%%%%%%%%%%%%%%%%%%%%%%%%%%%%%%%%%%%%%%%%%%%%%%%%%%%%%%%%%
\appendix

\settowidth\MacroIndent{\rmfamily\scriptsize 000\ }

 \DocInput{childdoc.dtx}

\end{document}
%</driver>
% \fi
%
% %%%%%%%%%%%%%%%%%%%%%%%%%%%%%%%%%%%%%%%%%%%%%%%%%%%%%%%%%%%%%%%%%%%%%%%%%%%%%%
% %%%%%%%%%%%%%%%%%%%%%%%%%%%%%%%%%%%%%%%%%%%%%%%%%%%%%%%%%%%%%%%%%%%%%%%%%%%%%%
% \section{Sample}
%\iffalse
%<*samplemain>
%\fi
%
% The following presents a sample document
% with two chapters, two parts, a title page,
% a compile flag as well as three forwarding files to set the flag.
% It consists of eight |.tex| files:
% \begin{center}
% \begin{tabular}{ll}
% |cdocsamp.tex|&main file\\
% |cdocsch1.tex|&include file for chapter 1\\
% |cdocsch2.tex|&include file for chapter 2\\
% |cdocspt3.tex|&include file for part 3\\
% |cdocspt4.tex|&include file for part 4\\
% |cdocsdrf.tex|&forwarding file for main file in draft mode\\
% |cdocsfi1.tex|&forwarding file for final version of chapter 1\\
% |cdocsfi2.tex|&forwarding file for final version of chapter 2\\
% \end{tabular}
% \end{center}
% Each of the eight files can be compiled directly by the \LaTeX{} compiler.
%
% %%%%%%%%%%%%%%%%%%%%%%%%%%%%%%%%%%%%%%
% \paragraph{Main File.}
%
% The main file is called |cdocsamp.tex|.
%
% Load the \textsf{childdoc} definitions and
% declare the filename for the main document:
%    \begin{macrocode}
\input{childdoc.def}
\childdocmain{}
%    \end{macrocode}

% Optional override for |\version| flag:
%    \begin{macrocode}
%%\ifchilddoc\else\providecommand{\version}{draft}\fi
%    \end{macrocode}

% Define the default values for the |\version| flag
% (|final| for the main file and |draft| for childs):
%    \begin{macrocode}
\ifchilddoc
\providecommand{\version}{draft}
\else
\providecommand{\version}{final}
\fi
%    \end{macrocode}

% Load the standard document class:
%    \begin{macrocode}
\documentclass[12pt]{article}
%    \end{macrocode}

% Start the document body:
%    \begin{macrocode}
\begin{document}
%    \end{macrocode}

% Declare a title page.
% Print title, part of document being processed and version flag:
%    \begin{macrocode}
\addtocounter{page}{-1}
\begin{center}
{\LARGE\bfseries{}childdoc example\par}
\vspace{1cm}
\ifchilddoc
\ifchilddocmanual part\else chapter\fi:
`\childdocname' of `\childdocjob'\par
\else
main document: `\childdocjob'\par
\fi
version: \version\par
\end{center}
\newpage
%    \end{macrocode}

% Manually include selected file,
% otherwise process as usual:
%    \begin{macrocode}
\ifchilddocmanual
\section*{part `\childdocname'}
\input{\childdocname}
\else
%    \end{macrocode}

% Include the two chapters:
%    \begin{macrocode}
\include{cdocsch1}
\include{cdocsch2}
%    \end{macrocode}

% Include the two parts unless only chapters should be displayed:
%    \begin{macrocode}
\ifchilddoc\else
\section{part three}
\input{cdocspt3}
\section{part four}
\input{cdocspt4}
\fi
%    \end{macrocode}

% Process as usual until here:
%    \begin{macrocode}
\fi
%    \end{macrocode}

% End of document body:
%    \begin{macrocode}
\end{document}
%    \end{macrocode}
%\iffalse
%</samplemain>
%\fi
%
% %%%%%%%%%%%%%%%%%%%%%%%%%%%%%%%%%%%%%%
% \paragraph{Chapter Include Files.}
%
% The include files are called |cdocsch1.tex| and |cdocsch2.tex|.
%
%\iffalse
%<*samplechap1|samplechap2>
%\fi

% Optional override for |\version| flag:
%    \begin{macrocode}
%%\providecommand{\version}{final}
%    \end{macrocode}

% Include the main document:
%    \begin{macrocode}
\input{childdoc.def}
\childdocof{cdocsamp}
%    \end{macrocode}

%\iffalse
%</samplechap1|samplechap2>
%\fi
%
%\iffalse
%<*samplechap1>
%\fi
% Some text for chapter 1:
%    \begin{macrocode}
\section{one}
some text in chapter one
%    \end{macrocode}

%\iffalse
%</samplechap1>
%\fi
% Some text for chapter 2:
%\iffalse
%<*samplechap2>
%\fi
%    \begin{macrocode}
\section{two}
more text in chapter two
%    \end{macrocode}

%\iffalse
%</samplechap2>
%\fi
%
% %%%%%%%%%%%%%%%%%%%%%%%%%%%%%%%%%%%%%%
% \paragraph{Part Include Files.}
%
% The include files are called |cdocspt3.tex| and |cdocspt4.tex|.
%
%\iffalse
%<*samplepart3|samplepart4>
%\fi

% Optional override for |\version| flag:
%    \begin{macrocode}
%%\providecommand{\version}{final}
%    \end{macrocode}

% Include the main document:
%    \begin{macrocode}
\input{childdoc.def}
\childdocby{cdocsamp}
%    \end{macrocode}

%\iffalse
%</samplepart3|samplepart4>
%\fi
%
%\iffalse
%<*samplepart3>
%\fi
% Some text for part 3:
%    \begin{macrocode}
some text in part three
%    \end{macrocode}

%\iffalse
%</samplepart3>
%\fi
% Some text for part 4:
%\iffalse
%<*samplepart4>
%\fi
%    \begin{macrocode}
more text in part four
%    \end{macrocode}

%\iffalse
%</samplepart4>
%\fi
%
% %%%%%%%%%%%%%%%%%%%%%%%%%%%%%%%%%%%%%%
% \paragraph{Forwarding for a Complete Draft.}
%
% The following forwarding file |cdocsdrf.tex|
% compiles the main document in draft mode:
%\iffalse
%<*sampledraft>
%\fi
%    \begin{macrocode}
\def\version{draft}
\input{childdoc.def}
\childdocforward{cdocsamp}
%    \end{macrocode}

%\iffalse
%</sampledraft>
%\fi
%
% %%%%%%%%%%%%%%%%%%%%%%%%%%%%%%%%%%%%%%
% \paragraph{Forwarding for Final Version of the Chapters.}
%
% The following forwarding files |cdocsfn1.tex| and |cdocsfn2.tex|
% (with identical content)
% compile the final versions of the child documents
% |cdocsch1.tex| and |cdocsch2.tex|, respectively:
%\iffalse
%<*samplefinal>
%\fi
%    \begin{macrocode}
\def\version{final}
\input{childdoc.def}
\childdocforwardprefix[cdocsamp]{cdocsfn}{cdocsch}
%    \end{macrocode}

%\iffalse
%</samplefinal>
%\fi
%
% %%%%%%%%%%%%%%%%%%%%%%%%%%%%%%%%%%%%%%
% \paragraph{Command Line Processing.}
%
% The following three command lines generate the output files
% |cdocscld|, |cdocscl1| and |cdocscl2|
% which should be identical to
% |cdocsdrf|, |cdocsch1| and |cdocsfn2|, respectively:
% \begin{center}
% \begin{tabular}{l}
% |latex -jobname cdocscld \|\\
% |  "\def\version{draft}\input{childdoc.def}\childdocforward{cdocsamp}"|\\
% |latex -jobname cdocscl1 \|\\
% |  "\input{childdoc.def}\childdocforward[cdocsamp]{cdocsch1}"|\\
% |latex -jobname cdocscl2 \|\\
% |  "\def\version{final}\input{childdoc.def}\childdocforward{cdocsch2}"|
% \end{tabular}
% \end{center}
% Note that the trailing backslash on each first line
% merely continues the input to the second line
% (for convenient cut ant paste).
% Furthermore, the command |latex| can be replaced by any
% of its alternative versions such as |pdflatex|.
%
% %%%%%%%%%%%%%%%%%%%%%%%%%%%%%%%%%%%%%%%%%%%%%%%%%%%%%%%%%%%%%%%%%%%%%%%%%%%%%%
% %%%%%%%%%%%%%%%%%%%%%%%%%%%%%%%%%%%%%%%%%%%%%%%%%%%%%%%%%%%%%%%%%%%%%%%%%%%%%%
% \section{Implementation}
%\iffalse
%<*package>
%\fi
%
% This section describes the definitions file |childdoc.def|.

% The definitions cannot be loaded using |\usepackage| or |\RequirePackage|
% which has a mechanism to prevent loading a style file more than once.
% When loading the definitions by means of |\input|
% multiple instances have to be prevented manually:
%\iffalse
%This code needs to be before the `\ProvidesFile' directive
%which is defined at the beginning of this file.
%Therefore it is also placed there and commented out here.
%</package>
%<*discard>
%\fi
%    \begin{macrocode}
\ifdefined\childdocmain\endinput\fi
%    \end{macrocode}
%\iffalse
%</discard>
%<*package>
%\fi
%
% \macro{\ifchilddoc}
% \macro{\ifchilddocmanual}
% The conditional |\ifchilddoc| tells whether a
% child (true) or main (false) document is being compiled.
% The conditional |\ifchilddocmanual| tells whether
% the |\includeonly| mechanism is used (false) or
% the selection of child files must be performed manually (true).
% The definitions initialise to false:
%    \begin{macrocode}
\newif\ifchilddoc
\newif\ifchilddocmanual
%    \end{macrocode}

% \macro{\childdocname}
% \macro{\childdocjob}
% The macro |\childdocname| stores the name of the main document
% to be compiled. The macro |\childdocjob| stores the name of
% the document on which the \LaTeX{} compiler was originally invoked.
% The content of |\jobname| cannot be compared
% to filenames specified in the source due to different catcodes.
% The following code rescans |\jobname|, stores the result
% in |\childdocname| and saves a copy in |\childdocjob|:
%    \begin{macrocode}
\edef\childdocname{\scantokens\expandafter{\jobname\noexpand}}
\let\childdocjob\childdocname
%    \end{macrocode}

% \macro{\childdocdisable}
% The macro |\childdocdisable| prevents the main file
% from being processed more than once.
% At this stage, the main document command |\childdocmain|
% is assumed to be called once again where it should do nothing.
% Any subsequent call to it should prevent
% a secondary processing of the main document
% It overwrites the forwarding commands
% |\childdocof| and |\childdocforward|
% with empty macros to prevent further inclusions of the main document:
%    \begin{macrocode}
\newcommand{\childdocdisable}
{
  \renewcommand{\childdocmain}[1]{\renewcommand{\childdocmain}[1]{\endinput}}
  \renewcommand{\childdocof}[1]{}
  \renewcommand{\childdocby}[2][]{}
  \renewcommand{\childdocforward}[2][]{}
  \renewcommand{\childdocdisable}{}
}
%    \end{macrocode}

% \macro{\childdocmain}
% The macro |\childdocmain| is to be called at the top of the main file
% with nothing or the main filename (without extension) as argument.
% First, it breaks loops.
% If the argument is not empty and does not match |\childdocname|
% (which is set by the first inclusion of |childdoc.def|),
% |\ifchilddoc| is set to true, |\includeonly| is applied to the child file
% and |\jobname| is set to the main file
% (for proper handling of |.aux| files):
%    \begin{macrocode}
\newcommand{\childdocmain}[1]
{
  \childdocdisable\childdocmain{}
  \if?#1?\else
    \begingroup
      \def\childdoctmp{#1}
      \ifx\childdoctmp\childdocname
        \def\childdoctmp{}
      \else
        \def\childdoctmp
        {
          \childdoctrue
          \includeonly{\childdocname}
          \def\childdocjob{#1}
          \def\jobname{#1}
        }
      \fi
      \expandafter
    \endgroup
    \childdoctmp
  \fi
}
%    \end{macrocode}

% \macro{\childdocof}
% The command |\childdocof| redirects
% compilation to the main file |#1|.
%    \begin{macrocode}
\newcommand{\childdocof}[1]
{
  \childdocdisable
  \childdoctrue
  \includeonly{\childdocname}
  \def\jobname{#1}
  \def\childdocjob{#1}
  \input{#1}
}
%    \end{macrocode}

% \macro{\childdocby}
% The command |\childdocby| ....
%    \begin{macrocode}
\newcommand{\childdocby}[2][]
{
  \childdocdisable
  \childdoctrue
  \childdocmanualtrue
  \if?#1?\else
    \def\jobname{#2}
  \fi
  \def\childdocjob{#2}
  \input{#2}
  \endinput
}
%    \end{macrocode}

% \macro{\childdocforward}
% The command |\childdocforward| redirects
% compilation to the main file or
% (if the optional argument is given) a child file.
% Parameters are set as if the main file
% or a child file starting with |\childdocof| was compiled.
% Then compilation is handed over to the main file:
%    \begin{macrocode}
\newcommand{\childdocforward}[2][]
{
  \begingroup
    \if?#1?
      \def\childdoctmp
      {
        \def\childdocname{#2}
        \def\childdocjob{#2}
        \def\jobname{#2}
        \input{#2}
        \endinput
      }
    \else
      \def\childdoctmp
      {
        \childdocdisable
        \def\childdocname{#2}
        \childdoctrue
        \includeonly{#2}
        \def\childdocjob{#1}
        \def\jobname{#1}
        \input{#1}
        \endinput
      }
    \fi
    \expandafter
  \endgroup
  \childdoctmp
}
%    \end{macrocode}

% \macro{\childdocforwardprefix}
% The command |\childdocforwardprefix| redirects
% compilation to the main or a child file by means of a pattern.
% The prefix |#1| in the current filename is replaced by |#2|
% and the suffix of the current filename is kept
% (it is assumed that the filename does not contain the substring `|~~~|'
% which is used as a delimiter).
% Compilation is handed over to the new file by |\childdocforward|:
%    \begin{macrocode}
\newcommand{\childdocforwardprefix}[3][]
{
  \begingroup
    \def\childdocextract #2##1~~~{\def\childdoctmp{\childdocforward[#1]{#3##1}}}
    \expandafter\childdocextract\childdocname~~~
    \expandafter
  \endgroup
  \childdoctmp
}
%    \end{macrocode}

% \macro{\childdoc}
% The deprecated macro |\childdoc| is a legacy version of |\childdocmain|:
%    \begin{macrocode}
\newcommand{\childdoc}{\childdocmain}
%    \end{macrocode}

% \macro{\childdocredirect}
% The deprecated macro |\childdocredirect| is a legacy version
% of |\childdocforward| and |\childdocforwardprefix|:
%    \begin{macrocode}
\newcommand{\childdocredirect}[2][]
{
  \begingroup
    \if?#1?
      \def\childdoctmp{\childdocforward{#2}}
    \else
      \def\childdoctmp{\childdocforwardprefix{#1}{#2}}
    \fi
    \expandafter
  \endgroup
  \childdoctmp
}
%    \end{macrocode}

%\iffalse
%</package>
%\fi
%
\endinput
\childdocforward[|\textit{main}|]{|\textit{dest}|}"|
\end{center}
%
Here \textit{target} is the name of the output file,
\textit{main} is the name of the main file
and \textit{dest} is the name of the main or child file to be processed
(all filenames without extensions).
The optional argument \textit{main} can be omitted
if \textit{main} matches \textit{dest}.
Optionally, compilation \textit{flags} can be defined via |\def| commands.
This command line makes the \TeX{} engine believe
it is compiling the file \textit{target}
whose content is specified as the latter parameter.
The provided code then forwards the processing to
\textit{main} or \textit{dest} as described in \secref{sec:forward}.

%%%%%%%%%%%%%%%%%%%%%%%%%%%%%%%%%%%%%%%%%%%%%%%%%%%%%%%%%%%%%%%%%%%%%%%%%%%%%%%%
\subsection{Include by Input}
\label{sec:input}

Including child documents by |\include| has some restrictions by design.
Most notably, the content of a child document always occupies
its own set of pages; pages cannot be shared between child documents.
Usually, this behaviour makes perfect sense
because each child document contain an essential part of the document.
However, in some situations it may be desirable to compose
a document from a collection of parts
without having mandatory page breaks between then.
For this case, the package
provides a mechanism to include parts
by |\input| which can also be processed individually.
However, by construction this mechanism
requires manual handling of the content to be output.

%%%%%%%%%%%%%%%%%%%%%%%%%%%%%%%%%%%%%%%%
\DescribeMacro{\ifchilddocmanual}
The main file should be prepared as usual, see \secref{sec:include}.
However, the document body must make a distinction
between processing of an individual part and of the main document, e.g.:
%
\begin{center}
\begin{tabular}{l}
|\ifchilddocmanual|\\
|\input{\childdocname}|\\
|\||else|\\
\textit{document body with }|\input{|\textit{part}|}|\\
|\||fi|
\end{tabular}
\end{center}
%
The conditional |\ifchilddocmanual| is true whenever
a part to be included by |\input| is being compiled,
and the name of the part is stored in |\childdocname|.

%%%%%%%%%%%%%%%%%%%%%%%%%%%%%%%%%%%%%%%%
\DescribeMacro{\childdocby}
Each part to be included by |\input| should start with:
%
\begin{center}
\begin{tabular}{l}
|% \iffalse
%
% childdoc.dtx Copyright (C) 2017-2018 Niklas Beisert
%
% This work may be distributed and/or modified under the
% conditions of the LaTeX Project Public License, either version 1.3
% of this license or (at your option) any later version.
% The latest version of this license is in
%   http://www.latex-project.org/lppl.txt
% and version 1.3 or later is part of all distributions of LaTeX
% version 2005/12/01 or later.
%
% This work has the LPPL maintenance status `maintained'.
%
% The Current Maintainer of this work is Niklas Beisert.
%
% This work consists of the files childdoc.dtx and childdoc.ins
% and the derived files childdoc.def and cdocsamp.tex with
% cdocsch1.tex, cdocsch2.tex, cdocsdrf.tex, cdocsfn1.tex, cdocsfn2.tex.
%
%<package>\ifdefined\childdocmain\endinput\fi
%<package>\ProvidesFile{childdoc.def}[2018/12/30 v2.0 child document driver]
%<samplemain>\ProvidesFile{cdocsamp.tex}[2018/12/30 v2.0 sample for childdoc]
%<*driver>
%\ProvidesFile{childdoc.drv}[2018/12/30 v2.0 childdoc reference manual file]
\PassOptionsToClass{10pt,a4paper}{article}
\documentclass{ltxdoc}

\usepackage[margin=35mm]{geometry}
\usepackage{hyperref}
\usepackage{hyperxmp}
\usepackage[usenames]{color}

\hypersetup{colorlinks=true}
\hypersetup{pdfstartview=FitH}
\hypersetup{pdfpagemode=UseNone}
\hypersetup{pdfsource={}}
\hypersetup{pdflang={en-UK}}
\hypersetup{pdfcopyright={Copyright 2017-2018 Niklas Beisert.
  This work may be distributed and/or modified under the
  conditions of the LaTeX Project Public License, either version 1.3
  of this license or (at your option) any later version.}}
\hypersetup{pdflicenseurl={http://www.latex-project.org/lppl.txt}}
\hypersetup{pdfcontactaddress={ETH Zurich, ITP, HIT K,
  Wolfgang-Pauli-Strasse 27}}
\hypersetup{pdfcontactpostcode={8093}}
\hypersetup{pdfcontactcity={Zurich}}
\hypersetup{pdfcontactcountry={Switzerland}}
\hypersetup{pdfcontactemail={nbeisert@itp.phys.ethz.ch}}
\hypersetup{pdfcontacturl={http://people.phys.ethz.ch/\xmptilde nbeisert/}}

\newcommand{\secref}[1]{\hyperref[#1]{section \ref*{#1}}}

\parskip1ex
\parindent0pt
\let\olditemize\itemize
\def\itemize{\olditemize\parskip0pt}

\begin{document}

\title{The \textsf{childdoc} Package}
\hypersetup{pdftitle={The childdoc Package}}
\author{Niklas Beisert\\[2ex]
  Institut f\"ur Theoretische Physik\\
  Eidgen\"ossische Technische Hochschule Z\"urich\\
  Wolfgang-Pauli-Strasse 27, 8093 Z\"urich, Switzerland\\[1ex]
  \href{mailto:nbeisert@itp.phys.ethz.ch}
  {\texttt{nbeisert@itp.phys.ethz.ch}}}
\hypersetup{pdfauthor={Niklas Beisert}}
\hypersetup{pdfsubject={Manual for the LaTeX2e Package childdoc}}
\date{30 December 2018, \textsf{v2.0}}
\maketitle

\begin{abstract}\noindent
\textsf{childdoc} is a \LaTeXe{} package
that enables the direct compilation
of document sections included by |\include|
to individual files.
\end{abstract}

\begingroup
\parskip0ex
\tableofcontents
\endgroup

%%%%%%%%%%%%%%%%%%%%%%%%%%%%%%%%%%%%%%%%%%%%%%%%%%%%%%%%%%%%%%%%%%%%%%%%%%%%%%%%
%%%%%%%%%%%%%%%%%%%%%%%%%%%%%%%%%%%%%%%%%%%%%%%%%%%%%%%%%%%%%%%%%%%%%%%%%%%%%%%%
\section{Introduction}

\LaTeX{} provides a mechanism to structure a large document (such as a book)
into a main file and several child files (containing the chapters)
using the |\include| command.
This mechanism is beneficial for documents
which span hundreds of pages in order to
make the source file(s) more manageable.
Moreover, compilation can be restricted to
selected child files by means of the |\includeonly| command.
The latter feature can be used to reduce the compilation time while editing
(this was significantly more useful in the earlier days of \LaTeX{})
or to generate a smaller document which is easier to navigate.
Another application of |\includeonly| is to generate
documents consisting of selected parts of the complete document.

However, there are a few drawbacks of the plain |\include| mechanism:
\begin{itemize}
\item
The child files cannot be compiled on their own,
they can only be compiled via the main file.
A naive editing environment
(such as a text editor with an option
to have the current file processed by \LaTeX)
may require one to switch to the main file before compiling;
attempting to compile the child file produces errors.
\item
The main file must be modified (each time)
to adjust the |\includeonly| command
to the present needs. This easily leaves the main file in a messy state.
\item
The generated document will always carry the filename
of the main document. This is inconvenient if
several child files are to be compiled and
to be kept for distribution.
\end{itemize}

The present package provides a simple interface
to make child files individually compilable by \LaTeX{}.
Compiling a child file then has the same effect as compiling
the main file with an |\includeonly| command
to select the appropriate child.
Moreover the generated document will carry the name of the child
rather than the main file.
This resolves all three above issues.

This feature is meant to make the editing of books,
thesis documents and lecture notes somewhat more convenient.
However, the package can also be used efficiently for
composing a series of documents (such as exercise sheets)
which are typically distributed individually.
It then assists the author in generating the individual documents
(potentially in different versions)
as well as a document containing the collected series.
Another application is in developing style files
or other kinds of included material
where compilation of the style file could redirect
to a sample or test file.

%%%%%%%%%%%%%%%%%%%%%%%%%%%%%%%%%%%%%%%%%%%%%%%%%%%%%%%%%%%%%%%%%%%%%%%%%%%%%%%%
%%%%%%%%%%%%%%%%%%%%%%%%%%%%%%%%%%%%%%%%%%%%%%%%%%%%%%%%%%%%%%%%%%%%%%%%%%%%%%%%
\section{Usage}

First of all, the package \textsf{childdoc} is \emph{not} a standard
\LaTeXe{} |.sty| style file! Therefore it needs to be invoked in
a non-standard way.

%%%%%%%%%%%%%%%%%%%%%%%%%%%%%%%%%%%%%%%%%%%%%%%%%%%%%%%%%%%%%%%%%%%%%%%%%%%%%%%%
\subsection{Included Files}
\label{sec:include}

%%%%%%%%%%%%%%%%%%%%%%%%%%%%%%%%%%%%%%%%
\DescribeMacro{\childdocmain}
To use the package, add the commands
\begin{center}
\begin{tabular}{l}
|\input{childdoc.def}|\\
|\childdocmain{}|\\
\end{tabular}
\end{center}
at the very top of the main \LaTeX{} file,
in particular \emph{before} the |\documentclass| statement!
The argument of |\childdocmain| should be left empty
(but it must be present).

%%%%%%%%%%%%%%%%%%%%%%%%%%%%%%%%%%%%%%%%
\DescribeMacro{\childdocof}
Furthermore, add the commands
\begin{center}
\begin{tabular}{l}
|\input{childdoc.def}|\\
|\childdocof{|\textit{main}|}|\\
\end{tabular}
\end{center}
at the top of every child file \textit{child}
which is included by |\include{|\textit{child}|}|
from within the main file
(or at least for those files to be compiled individually).
The argument \textit{main} must be the filename of the main file.

There are a couple of
considerations in setting up the main and child documents:

%%%%%%%%%%%%%%%%%%%%%%%%%%%%%%%%%%%%%%%%
\paragraph{Restrictions.}

Please note the following restrictions:
\begin{itemize}
\item
|\childdocmain| must be called with one argument \textit{main}
to ensure compatibility with earlier version of the package.
It must either be empty (|\childdocmain{}|)
or precisely match the filename of the main file in which it is specified.
See \secref{sec:detection} for further information.
\item
The filename \textit{main} must be specified without the |.tex| extension.
\item
The filename \textit{main} is case sensitive
(even in case-insensitive file systems)
due to internal string comparison.
\item
The argument \textit{main} should be fully expanded, it cannot be a macro.
\item
Subdirectories and special characters should be avoided in filenames.
\item
The command |\childdocmain{|\textit{main}|}| must be followed by a whitespace.
It should not be followed immediately by another command
or by a comment mark `|%|'.
This is because the \TeX{} parser reads the token immediately following
the argument of |\childdocmain| and puts it
at the beginning of every child section;
however, a white\-space is ignored.
\end{itemize}

%%%%%%%%%%%%%%%%%%%%%%%%%%%%%%%%%%%%%%%%
\paragraph{Content of Main File.}

It is advisable to place all content in the child files included by |\include|.
Any output contained in the main file will appear in all child documents
unless suppressed manually;
it cannot be suppressed automatically by the |\includeonly| directive
and thus should normally be avoided.
A method to include some content in the main file
by means of conditional processing is described in \secref{sec:conditional}.

%%%%%%%%%%%%%%%%%%%%%%%%%%%%%%%%%%%%%%%%
\paragraph{Page Numbering.}

When only a part of the document is compiled,
the appropriate numbering of pages
(as well as other status parameters)
is determined from the |.aux| files.
The latter contain information from previous passes.
However this information needs to propagate through
all intermediate child documents.
Therefore the page numbering in child documents may well
be inconsistent until the complete document is compiled at least once.

A useful (if unconventional) way to always ensure a consistent
page numbering is to restart the numbering in each child document
and denote the pages by `\textit{child}|.|\textit{page}'
where \textit{child} represents the chapter/section number of the child file.
This can be achieved by the command
|\numberwithin{page}{|\textit{child}|}|
of the \textsf{amsmath} package
where \textit{child} can be |chapter| or |section|
depending on the chosen structuring.
Alternatively, one can modify the macro |\thepage| appropriately
and reset the counter |page| at the start of each child file.

%%%%%%%%%%%%%%%%%%%%%%%%%%%%%%%%%%%%%%%%%%%%%%%%%%%%%%%%%%%%%%%%%%%%%%%%%%%%%%%%
\subsection{Conditional Processing}
\label{sec:conditional}

The package provides a mechanism to compile different versions
of a document. To customise the versions further some conditional processing
can come in handy to distinguish which version is being compiled.
The package provides two macros to describe the compilation context:

%%%%%%%%%%%%%%%%%%%%%%%%%%%%%%%%%%%%%%%%
\DescribeMacro{\ifchilddoc}
The conditional |\ifchilddoc| distinguishes between the compilation of
child documents and the main document:
%
\begin{center}
|\ifchilddoc |\textit{child-code}| |[|\||else |\textit{main-code}]| \||fi|
\end{center}

%%%%%%%%%%%%%%%%%%%%%%%%%%%%%%%%%%%%%%%%
\DescribeMacro{\childdocname}
\DescribeMacro{\childdocjob}
The macro |\childdocname| contains the filename (without extension)
of the main or child file being processed.
Note that |\childdocjob| will always contain the name of the main file.

%%%%%%%%%%%%%%%%%%%%%%%%%%%%%%%%%%%%%%%%
\paragraph{Title Page.}

Conditional processing can be used to include a title or banner page
in the main document when proper precautions are taken.
Importantly, the code in the main file should ensure that the page counter
(as well as other status parameters which are stored in the |.aux| files)
takes the same value after the conditional processing.
Otherwise the page numbers may take divergent values
depending on which part is compiled.

For example, a title page could be declared by:
%
\begin{center}
\begin{tabular}{l}
|\ifchilddoc\||else|\\
|\addtocounter{page}{-1}|\\
\textit{code for title page}\\
|\newpage|\\
|\||fi|
\end{tabular}
\end{center}
%
A banner page for the child documents can be generated by:
%
\begin{center}
\begin{tabular}{l}
|\ifchilddoc|\\
|\addtocounter{page}{-1}|\\
\textit{code for banner page}\\
|\newpage|\\
|\||fi|
\end{tabular}
\end{center}
%
Here one could write a message such as:
\begin{center}
|This is the part \childdocname{} of \childdocjob{}.|
\end{center}

%%%%%%%%%%%%%%%%%%%%%%%%%%%%%%%%%%%%%%%%%%%%%%%%%%%%%%%%%%%%%%%%%%%%%%%%%%%%%%%%
\subsection{Flags}
\label{sec:flags}

The package makes it easy to generate different versions
of the main or child documents.
To this end compilation flags can be defined
and assigned different default values.
They will be particularly useful in conjunction
with the forwarding mechanism described in \secref{sec:forward}.

For example, it may be useful to have a flag |\version|
which can be set to |draft| or |final|.
The document source will contain some conditional code
depending on the value of |\version|.
Suppose further, the flag should default to |final| for the main file
and to |draft| for child files
which is a natural assignment for editing the document.
This is achieved by placing the following code
in the preamble of the main document
(below the |\childdocmain| directive):
%
\begin{center}
\begin{tabular}{l}
|\ifchilddoc|\\
|\providecommand{\version}{draft}|\\
|\||else|\\
|\providecommand{\version}{final}|\\
|\||fi|
\end{tabular}
\end{center}
%
The definition by |\providecommand| makes sure
that previous definitions are not overwritten.
Further statements |\providecommand{\version}{...}|
can thus be added before the above code to override it.

For the main file, one might add a line
(between |\childdocmain| and the above block)
%
\begin{center}
|%\ifchilddoc\||else\providecommand{\version}{draft}\||fi|
\end{center}
%
which can be uncommented to produce a draft version.
Likewise one can add a line to the very top of a child file
(above the |\childdocof{|\textit{main}|}| directive)
%
\begin{center}
|%\providecommand{\version}{final}|
\end{center}
%
which can be uncommented to produce the final version of this child document.

%%%%%%%%%%%%%%%%%%%%%%%%%%%%%%%%%%%%%%%%%%%%%%%%%%%%%%%%%%%%%%%%%%%%%%%%%%%%%%%%
\subsection{Forwarding}
\label{sec:forward}

Different versions of the main or child documents
using compilation flags as described in \secref{sec:flags}
can be (permanently) stored in different files
for convenient compilation, viewing and distribution.
To this end, the package defines a command
to pass on compilation to a different file:

%%%%%%%%%%%%%%%%%%%%%%%%%%%%%%%%%%%%%%%%
\DescribeMacro{\childdocforward}
The command |\childdocforward| redirects processing to
another source file:
%
\begin{center}
\begin{tabular}{l}
|\input{childdoc.def}|\\
|\childdocforward[|\textit{main}|]{|\textit{dest}|}|\\
\end{tabular}
\end{center}
%
The argument \textit{dest} is the destination file
(without extension).
It should be the main file or one of the child files.
Note that further \textsf{childdoc} directives
such as |\childdocof| and |\childdocforward|
in the indicated file will be processed in this form.
The optional argument \textit{main}
passes on directly to the main file \textit{main}
while pretending to compile the child \textit{dest}.
This form behaves as if \textit{dest}
issues |\childdocof{|\textit{main}|}| right away,
and no further \textsf{childdoc} directives will be processed.

%%%%%%%%%%%%%%%%%%%%%%%%%%%%%%%%%%%%%%%%
\DescribeMacro{\...prefix}
In the alternative form |\childdocforwardprefix|,
%
\begin{center}
\begin{tabular}{l}
|\input{childdoc.def}|\\
|\childdocforwardprefix[|\textit{main}|]{|\textit{prefix}|}{|\textit{dest}|}|
\end{tabular}
\end{center}
%
the destination file is determined by a pattern
depending on the current file:
To make this work, the current file must be called
`{\textit{prefix}\hspace{0.2em}\textit{suffix}}'
with \textit{prefix} matching precisely the argument.
Processing is then passed on to the file
`{\textit{dest}\hspace{0.2em}\textit{suffix}}'.
Surely, the same effect is achieved by
directly specifying the
argument `{\textit{dest}\hspace{0.2em}\textit{suffix}}'
in the first form.
However, that requires to set up a different file
for each child. With the alternative form of the command
all these files can have exactly the same content
which simplifies setting them up and maintaining them.

For example, the following file |draft.tex|
with a compilation flag |\version| as described in \secref{sec:flags}
compiles the main document as a draft:
%
\begin{center}
\begin{tabular}{l}
|\def\version{draft}|\\
|\input{childdoc.def}|\\
|\childdocforward{|\textit{main}|}|
\end{tabular}
\end{center}
%
Likewise, the following files |final|\textit{nn}|.tex|
compile the final version of the child document
|child|\textit{nn}|.tex|:
%
\begin{center}
\begin{tabular}{l}
|\def\version{final}|\\
|\input{childdoc.def}|\\
|\childdocforwardprefix{final}{child}|
\end{tabular}
\end{center}
%

Note that when several versions of a main file and/or of each child file
are to be generated, it may be convenient to set up a |Makefile| or
shell script to automatise the process.

%%%%%%%%%%%%%%%%%%%%%%%%%%%%%%%%%%%%%%%%%%%%%%%%%%%%%%%%%%%%%%%%%%%%%%%%%%%%%%%%
\subsection{Command Line Processing}
\label{sec:commandline}

The effect of redirection files can also be achieved by invoking
the \LaTeX{} compiler with a more elaborate command line.
Most conveniently this should be done as part
of a shell script or a |Makefile|.

When using \textsf{childdoc} in the main file, the following
command lines effectively perform a redirection
(note that depending on the shell being used,
backslashes may have to be doubled: `|\|' $\to$ `|\\|'):
%
\begin{center}
|... -jobname "|\textit{target}|" |\\|"|[\textit{flags}]%
|\input{childdoc.def}\childdocforward[|\textit{main}|]{|\textit{dest}|}"|
\end{center}
%
Here \textit{target} is the name of the output file,
\textit{main} is the name of the main file
and \textit{dest} is the name of the main or child file to be processed
(all filenames without extensions).
The optional argument \textit{main} can be omitted
if \textit{main} matches \textit{dest}.
Optionally, compilation \textit{flags} can be defined via |\def| commands.
This command line makes the \TeX{} engine believe
it is compiling the file \textit{target}
whose content is specified as the latter parameter.
The provided code then forwards the processing to
\textit{main} or \textit{dest} as described in \secref{sec:forward}.

%%%%%%%%%%%%%%%%%%%%%%%%%%%%%%%%%%%%%%%%%%%%%%%%%%%%%%%%%%%%%%%%%%%%%%%%%%%%%%%%
\subsection{Include by Input}
\label{sec:input}

Including child documents by |\include| has some restrictions by design.
Most notably, the content of a child document always occupies
its own set of pages; pages cannot be shared between child documents.
Usually, this behaviour makes perfect sense
because each child document contain an essential part of the document.
However, in some situations it may be desirable to compose
a document from a collection of parts
without having mandatory page breaks between then.
For this case, the package
provides a mechanism to include parts
by |\input| which can also be processed individually.
However, by construction this mechanism
requires manual handling of the content to be output.

%%%%%%%%%%%%%%%%%%%%%%%%%%%%%%%%%%%%%%%%
\DescribeMacro{\ifchilddocmanual}
The main file should be prepared as usual, see \secref{sec:include}.
However, the document body must make a distinction
between processing of an individual part and of the main document, e.g.:
%
\begin{center}
\begin{tabular}{l}
|\ifchilddocmanual|\\
|\input{\childdocname}|\\
|\||else|\\
\textit{document body with }|\input{|\textit{part}|}|\\
|\||fi|
\end{tabular}
\end{center}
%
The conditional |\ifchilddocmanual| is true whenever
a part to be included by |\input| is being compiled,
and the name of the part is stored in |\childdocname|.

%%%%%%%%%%%%%%%%%%%%%%%%%%%%%%%%%%%%%%%%
\DescribeMacro{\childdocby}
Each part to be included by |\input| should start with:
%
\begin{center}
\begin{tabular}{l}
|\input{childdoc.def}|\\
|\childdocby{|\textit{main}|}|\\
\end{tabular}
\end{center}
%
The directive |\childdocby| is similar to |\childdocof|
described in \secref{sec:include},
but the subsequent selection of content must be done manually.
To that end, both |\ifchilddoc| and |\ifchilddocmanual|
will be true upon processing of a part,
and the name of the part is stored in |\childdocname|.
Note that |\jobname| will be set to the filename of the current part
so that each part receives an individual |.aux| file
that does not interfere with the |.aux| file(s) of the main document.
This behaviour can be altered by the alternative form
|\childdocby[*]{|\textit{main}|}| (with a non-empty optional argument)
which uses the |.aux| file of the main document
by setting |\jobname| to \textit{main}.

%%%%%%%%%%%%%%%%%%%%%%%%%%%%%%%%%%%%%%%%%%%%%%%%%%%%%%%%%%%%%%%%%%%%%%%%%%%%%%%%
\subsection{Driver Development}
\label{sec:driver}

The \textsf{childdoc} mechanism can also be use for the development
of definition files such as \LaTeX{} styles or classes.
This case differs from the above setup with multiple parts
included by |\include| in that no |\includeonly| should be invoked.
This can be achieved by starting the include file
(before |\ProvidesPackage|) with:
%
\begin{center}
\begin{tabular}{l}
|\input{childdoc.def}|\\
|\childdocforward{|\textit{main}|}|\\
\end{tabular}
\end{center}
%
or alternatively with:
%
\begin{center}
\begin{tabular}{l}
|\input{childdoc.def}|\\
|\childdocby{|\textit{main}|}|\\
\end{tabular}
\end{center}
%
Both forms have slightly different effects as described above.
The main file is prepared as usual, see \secref{sec:include}.

%%%%%%%%%%%%%%%%%%%%%%%%%%%%%%%%%%%%%%%%%%%%%%%%%%%%%%%%%%%%%%%%%%%%%%%%%%%%%%%%
\subsection{Legacy Detection}
\label{sec:detection}

The directive |\childdocmain| in the main file can detect
whether the complete document or merely a child is to be compiled
even without using the directive |\childdocof|.
This method is deprecated because it is less robust
and there is no compelling reason to use it;
it is merely provided for backward compatibility
and it may be removed in future versions.

If the detection mechanism is to be used,
it is mandatory to correctly specify
the filename of the main file as the argument of |\childdocmain|:
%
\begin{center}
\begin{tabular}{l}
|\input{childdoc.def}|\\
|\childdocmain{|\textit{main}|}|\\
\end{tabular}
\end{center}
%
If |\jobname| does not match the argument \textit{main} of |\childdocmain|,
it is assumed that |\jobname| points to the child file to be compiled.
When using |\childdocmain| with the main file specified as argument,
it suffices to start a child file
with just |\input{|\textit{main}|}|
without loading of the package and using |\childdocof|.
If instead all processing is done
with the appropriate \textsf{childdoc} directives,
the argument of \textit{main} of |\childdocmain| can be empty.

An alternative version of the command line processing described
in \secref{sec:commandline} using the detection mechanism reads:
%
\begin{center}
|... -jobname "|\textit{target}|" "|[\textit{flags}]%
[|\def\jobname{|\textit{dest}|}|]|\input{|\textit{main}|}"|
\end{center}

%%%%%%%%%%%%%%%%%%%%%%%%%%%%%%%%%%%%%%%%%%%%%%%%%%%%%%%%%%%%%%%%%%%%%%%%%%%%%%%%
\subsection{Manual Code}
\label{sec:manual}

In case one cannot be certain whether the definitions file |childdoc.def|
is installed on the target \TeX{} distribution
and one prefers not to ship it,
it is conceivable to paste a few relevant commands into the sources.

To that end, drop all statements |\input{childdoc.def}|
and perform the replacements as outlined below.
Instead of |\childdocmain{|\textit{main}|}| add the following code
to the top of the main file:
%
\begin{center}
\begin{tabular}{l}
|\||ifdefined\childdocname\endinput\||fi\newif\ifchilddoc|\\
|\edef\childdocname{\scantokens\expandafter{\jobname\noexpand}}|\\
|\def\childdocmain{|\textit{main}|}\||ifx\childdocmain\childdocname\||else|\\
|\childdoctrue\includeonly{\childdocname}\let\jobname\childdocmain\||fi|\\
\end{tabular}
\end{center}
%
Instead of |\childdocof{|\textit{main}|}| just include the main file
at the top of each child file:
%
\begin{center}
|\input{|\textit{main}|}|
\end{center}
%
A simple redirection |\childdocforward{|\textit{dest}|}| is achieved by:
%
\begin{center}
|\def\jobname{|\textit{dest}|}\input{\jobname}|
\end{center}
%
The redirection with prefix
|\childdocforwardprefix[|\textit{prefix}|]{|\textit{dest}|}|
is accomplished by:
%
\begin{center}
\begin{tabular}{l}
|{\edef\jobname{\scantokens\expandafter{\jobname\noexpand}}|\\
|\def\redirectjob |\textit{prefix}|#1~~~{\gdef\jobname{|\textit{dest}|#1}}|\\
|\expandafter\redirectjob\jobname~~~}\input{\jobname}|
\end{tabular}
\end{center}

In an alternative approach,
child documents can be compiled by a specific command line
without additional code or specific definitions:
%
\begin{center}
|... -jobname "|\textit{target}|" "|[\textit{flags}]%
|\includeonly{|\textit{dest}|}\input{|\textit{main}|}"|
\end{center}
%

%%%%%%%%%%%%%%%%%%%%%%%%%%%%%%%%%%%%%%%%%%%%%%%%%%%%%%%%%%%%%%%%%%%%%%%%%%%%%%%%
%%%%%%%%%%%%%%%%%%%%%%%%%%%%%%%%%%%%%%%%%%%%%%%%%%%%%%%%%%%%%%%%%%%%%%%%%%%%%%%%
\section{Information}

%%%%%%%%%%%%%%%%%%%%%%%%%%%%%%%%%%%%%%%%%%%%%%%%%%%%%%%%%%%%%%%%%%%%%%%%%%%%%%%%
\subsection{Copyright}

Copyright \copyright{} 2017--2018 Niklas Beisert

This work may be distributed and/or modified under the
conditions of the \LaTeX{} Project Public License, either version 1.3
of this license or (at your option) any later version.
The latest version of this license is in
  \url{http://www.latex-project.org/lppl.txt}
and version 1.3 or later is part of all distributions of \LaTeX{}
version 2005/12/01 or later.

This work has the LPPL maintenance status `maintained'.

The Current Maintainer of this work is Niklas Beisert.

This work consists of the files |README.txt|, |childdoc.ins| and |childdoc.dtx|
as well as the derived files |childdoc.def|, |cdocsamp.tex|
with |cdocsch1.tex|, |cdocsch2.tex|, |cdocspt3.tex|, |cdocspt4.tex|,
|cdocsdrf.tex|, |cdocsfn1.tex|, |cdocsfn2.tex|
as well as |childdoc.pdf|.

%%%%%%%%%%%%%%%%%%%%%%%%%%%%%%%%%%%%%%%%%%%%%%%%%%%%%%%%%%%%%%%%%%%%%%%%%%%%%%%%
\subsection{Files and Installation}

The package consists of the files:
%
\begin{center}
\begin{tabular}{ll}
    |README.txt|   & readme file \\
    |childdoc.ins| & installation file \\
    |childdoc.dtx| & source file \\
    |childdoc.def| & definition file \\
    |cdocsamp.tex| & sample main file \\
    |cdocsch1.tex| & sample include file \\
    |cdocsch2.tex| & sample include file \\
    |cdocspt3.tex| & sample part file \\
    |cdocspt4.tex| & sample part file \\
    |cdocsdrf.tex| & sample redirection file \\
    |cdocsfn1.tex| & sample redirection file \\
    |cdocsfn2.tex| & sample redirection file \\
    |childdoc.pdf| & manual
\end{tabular}
\end{center}
%
The distribution consists of the files
|README.txt|, |childdoc.ins| and |childdoc.dtx|.
%
\begin{itemize}
\item
Run (pdf)\LaTeX{} on |childdoc.dtx|
to compile the manual |childdoc.pdf| (this file).
\item
Run \LaTeX{} on |childdoc.ins| to create the definitions file |childdoc.def|
and the sample |cdocsamp.tex| with include files
|cdocsch1.tex|, |cdocsch2.tex|, |cdocspt3.tex|, |cdocspt4.tex|,
|cdocsdrf.tex|, |cdocsfn1.tex|, |cdocsfn2.tex|.
Then copy the file |childdoc.def| to an appropriate directory of your \LaTeX{}
distribution, e.g.\ \textit{texmf-root}|/tex/latex/childdoc|.
\end{itemize}

%%%%%%%%%%%%%%%%%%%%%%%%%%%%%%%%%%%%%%%%%%%%%%%%%%%%%%%%%%%%%%%%%%%%%%%%%%%%%%%%
\subsection{Related CTAN Packages}

There are several other packages which offer a similar functionality:
%
\begin{itemize}
\item
The packages
\href{http://ctan.org/pkg/docmute}{\textsf{docmute}},
\href{http://ctan.org/pkg/includex}{\textsf{includex}} and
\href{http://ctan.org/pkg/standalone}{\textsf{standalone}}
provide commands to include only the document body of
a child file thus allowing both files to be compiled individually.
\item
The packages \href{http://ctan.org/pkg/subdocs}{\textsf{subdocs}}
and \href{http://ctan.org/pkg/subfiles}{\textsf{subfiles}}
provide structures in which the main and child documents can be
encapsulated and allowing them to be compiled individually.
The inclusion mechanism is different from the conventional |\include|.
\item
The package \href{http://ctan.org/pkg/combine}{\textsf{combine}}
is an elaborate solution to combine several documents into one.
\end{itemize}
%
See also the CTAN topic \href{http://ctan.org/topic/subdocs}{\textsf{subdocs}}
for further related packages.
The present package differs from the above solutions in that
a document structure constructed with the conventional |\include| mechanism
just needs two extra commands at the top of every file
such that all constituent files can be compiled individually.

%%%%%%%%%%%%%%%%%%%%%%%%%%%%%%%%%%%%%%%%%%%%%%%%%%%%%%%%%%%%%%%%%%%%%%%%%%%%%%%%
%\subsection{Feature Suggestions}
%
%The following is a list of features which may be useful for future
%versions of this package:
%%
%\begin{itemize}
%\item
%\ldots
%\end{itemize}

%%%%%%%%%%%%%%%%%%%%%%%%%%%%%%%%%%%%%%%%%%%%%%%%%%%%%%%%%%%%%%%%%%%%%%%%%%%%%%%%
\subsection{Revision History}

%%%%%%%%%%%%%%%%%%%%%%%%%%%%%%%%%%%%%%%%
\paragraph{v2.0:} 2018/12/30

\begin{itemize}
\item
immediate forward processing
\item
added |\childdocby| mechanism
\item
manual restructured
\end{itemize}

%%%%%%%%%%%%%%%%%%%%%%%%%%%%%%%%%%%%%%%%
\paragraph{v1.6:} 2018/01/17

\begin{itemize}
\item
application for development of include files
\item
corrections to manual
\end{itemize}

%%%%%%%%%%%%%%%%%%%%%%%%%%%%%%%%%%%%%%%%
\paragraph{v1.5:} 2017/05/21

\begin{itemize}
\item
more complete structuring introduced
\item
|\childdocof| introduced
\item
|\childdoc| renamed to |\childdocmain|
\item
|\childredirect| renamed to |\childdocforward| and |\childdocforwardprefix|
and functionality expanded
\end{itemize}

%%%%%%%%%%%%%%%%%%%%%%%%%%%%%%%%%%%%%%%%
\paragraph{v1.0:} 2017/04/27

\begin{itemize}
\item
manual and install package
\item
first version published on CTAN
\end{itemize}

%%%%%%%%%%%%%%%%%%%%%%%%%%%%%%%%%%%%%%%%
\paragraph{v0.6:} 2017/04/26

\begin{itemize}
\item
redirection mechanism added
\end{itemize}

%%%%%%%%%%%%%%%%%%%%%%%%%%%%%%%%%%%%%%%%
\paragraph{v0.5:} 2017/04/26

\begin{itemize}
\item
functionality in definition file
\end{itemize}


%%%%%%%%%%%%%%%%%%%%%%%%%%%%%%%%%%%%%%%%%%%%%%%%%%%%%%%%%%%%%%%%%%%%%%%%%%%%%%%%
%%%%%%%%%%%%%%%%%%%%%%%%%%%%%%%%%%%%%%%%%%%%%%%%%%%%%%%%%%%%%%%%%%%%%%%%%%%%%%%%
%%%%%%%%%%%%%%%%%%%%%%%%%%%%%%%%%%%%%%%%%%%%%%%%%%%%%%%%%%%%%%%%%%%%%%%%%%%%%%%%
\appendix

\settowidth\MacroIndent{\rmfamily\scriptsize 000\ }

 \DocInput{childdoc.dtx}

\end{document}
%</driver>
% \fi
%
% %%%%%%%%%%%%%%%%%%%%%%%%%%%%%%%%%%%%%%%%%%%%%%%%%%%%%%%%%%%%%%%%%%%%%%%%%%%%%%
% %%%%%%%%%%%%%%%%%%%%%%%%%%%%%%%%%%%%%%%%%%%%%%%%%%%%%%%%%%%%%%%%%%%%%%%%%%%%%%
% \section{Sample}
%\iffalse
%<*samplemain>
%\fi
%
% The following presents a sample document
% with two chapters, two parts, a title page,
% a compile flag as well as three forwarding files to set the flag.
% It consists of eight |.tex| files:
% \begin{center}
% \begin{tabular}{ll}
% |cdocsamp.tex|&main file\\
% |cdocsch1.tex|&include file for chapter 1\\
% |cdocsch2.tex|&include file for chapter 2\\
% |cdocspt3.tex|&include file for part 3\\
% |cdocspt4.tex|&include file for part 4\\
% |cdocsdrf.tex|&forwarding file for main file in draft mode\\
% |cdocsfi1.tex|&forwarding file for final version of chapter 1\\
% |cdocsfi2.tex|&forwarding file for final version of chapter 2\\
% \end{tabular}
% \end{center}
% Each of the eight files can be compiled directly by the \LaTeX{} compiler.
%
% %%%%%%%%%%%%%%%%%%%%%%%%%%%%%%%%%%%%%%
% \paragraph{Main File.}
%
% The main file is called |cdocsamp.tex|.
%
% Load the \textsf{childdoc} definitions and
% declare the filename for the main document:
%    \begin{macrocode}
\input{childdoc.def}
\childdocmain{}
%    \end{macrocode}

% Optional override for |\version| flag:
%    \begin{macrocode}
%%\ifchilddoc\else\providecommand{\version}{draft}\fi
%    \end{macrocode}

% Define the default values for the |\version| flag
% (|final| for the main file and |draft| for childs):
%    \begin{macrocode}
\ifchilddoc
\providecommand{\version}{draft}
\else
\providecommand{\version}{final}
\fi
%    \end{macrocode}

% Load the standard document class:
%    \begin{macrocode}
\documentclass[12pt]{article}
%    \end{macrocode}

% Start the document body:
%    \begin{macrocode}
\begin{document}
%    \end{macrocode}

% Declare a title page.
% Print title, part of document being processed and version flag:
%    \begin{macrocode}
\addtocounter{page}{-1}
\begin{center}
{\LARGE\bfseries{}childdoc example\par}
\vspace{1cm}
\ifchilddoc
\ifchilddocmanual part\else chapter\fi:
`\childdocname' of `\childdocjob'\par
\else
main document: `\childdocjob'\par
\fi
version: \version\par
\end{center}
\newpage
%    \end{macrocode}

% Manually include selected file,
% otherwise process as usual:
%    \begin{macrocode}
\ifchilddocmanual
\section*{part `\childdocname'}
\input{\childdocname}
\else
%    \end{macrocode}

% Include the two chapters:
%    \begin{macrocode}
\include{cdocsch1}
\include{cdocsch2}
%    \end{macrocode}

% Include the two parts unless only chapters should be displayed:
%    \begin{macrocode}
\ifchilddoc\else
\section{part three}
\input{cdocspt3}
\section{part four}
\input{cdocspt4}
\fi
%    \end{macrocode}

% Process as usual until here:
%    \begin{macrocode}
\fi
%    \end{macrocode}

% End of document body:
%    \begin{macrocode}
\end{document}
%    \end{macrocode}
%\iffalse
%</samplemain>
%\fi
%
% %%%%%%%%%%%%%%%%%%%%%%%%%%%%%%%%%%%%%%
% \paragraph{Chapter Include Files.}
%
% The include files are called |cdocsch1.tex| and |cdocsch2.tex|.
%
%\iffalse
%<*samplechap1|samplechap2>
%\fi

% Optional override for |\version| flag:
%    \begin{macrocode}
%%\providecommand{\version}{final}
%    \end{macrocode}

% Include the main document:
%    \begin{macrocode}
\input{childdoc.def}
\childdocof{cdocsamp}
%    \end{macrocode}

%\iffalse
%</samplechap1|samplechap2>
%\fi
%
%\iffalse
%<*samplechap1>
%\fi
% Some text for chapter 1:
%    \begin{macrocode}
\section{one}
some text in chapter one
%    \end{macrocode}

%\iffalse
%</samplechap1>
%\fi
% Some text for chapter 2:
%\iffalse
%<*samplechap2>
%\fi
%    \begin{macrocode}
\section{two}
more text in chapter two
%    \end{macrocode}

%\iffalse
%</samplechap2>
%\fi
%
% %%%%%%%%%%%%%%%%%%%%%%%%%%%%%%%%%%%%%%
% \paragraph{Part Include Files.}
%
% The include files are called |cdocspt3.tex| and |cdocspt4.tex|.
%
%\iffalse
%<*samplepart3|samplepart4>
%\fi

% Optional override for |\version| flag:
%    \begin{macrocode}
%%\providecommand{\version}{final}
%    \end{macrocode}

% Include the main document:
%    \begin{macrocode}
\input{childdoc.def}
\childdocby{cdocsamp}
%    \end{macrocode}

%\iffalse
%</samplepart3|samplepart4>
%\fi
%
%\iffalse
%<*samplepart3>
%\fi
% Some text for part 3:
%    \begin{macrocode}
some text in part three
%    \end{macrocode}

%\iffalse
%</samplepart3>
%\fi
% Some text for part 4:
%\iffalse
%<*samplepart4>
%\fi
%    \begin{macrocode}
more text in part four
%    \end{macrocode}

%\iffalse
%</samplepart4>
%\fi
%
% %%%%%%%%%%%%%%%%%%%%%%%%%%%%%%%%%%%%%%
% \paragraph{Forwarding for a Complete Draft.}
%
% The following forwarding file |cdocsdrf.tex|
% compiles the main document in draft mode:
%\iffalse
%<*sampledraft>
%\fi
%    \begin{macrocode}
\def\version{draft}
\input{childdoc.def}
\childdocforward{cdocsamp}
%    \end{macrocode}

%\iffalse
%</sampledraft>
%\fi
%
% %%%%%%%%%%%%%%%%%%%%%%%%%%%%%%%%%%%%%%
% \paragraph{Forwarding for Final Version of the Chapters.}
%
% The following forwarding files |cdocsfn1.tex| and |cdocsfn2.tex|
% (with identical content)
% compile the final versions of the child documents
% |cdocsch1.tex| and |cdocsch2.tex|, respectively:
%\iffalse
%<*samplefinal>
%\fi
%    \begin{macrocode}
\def\version{final}
\input{childdoc.def}
\childdocforwardprefix[cdocsamp]{cdocsfn}{cdocsch}
%    \end{macrocode}

%\iffalse
%</samplefinal>
%\fi
%
% %%%%%%%%%%%%%%%%%%%%%%%%%%%%%%%%%%%%%%
% \paragraph{Command Line Processing.}
%
% The following three command lines generate the output files
% |cdocscld|, |cdocscl1| and |cdocscl2|
% which should be identical to
% |cdocsdrf|, |cdocsch1| and |cdocsfn2|, respectively:
% \begin{center}
% \begin{tabular}{l}
% |latex -jobname cdocscld \|\\
% |  "\def\version{draft}\input{childdoc.def}\childdocforward{cdocsamp}"|\\
% |latex -jobname cdocscl1 \|\\
% |  "\input{childdoc.def}\childdocforward[cdocsamp]{cdocsch1}"|\\
% |latex -jobname cdocscl2 \|\\
% |  "\def\version{final}\input{childdoc.def}\childdocforward{cdocsch2}"|
% \end{tabular}
% \end{center}
% Note that the trailing backslash on each first line
% merely continues the input to the second line
% (for convenient cut ant paste).
% Furthermore, the command |latex| can be replaced by any
% of its alternative versions such as |pdflatex|.
%
% %%%%%%%%%%%%%%%%%%%%%%%%%%%%%%%%%%%%%%%%%%%%%%%%%%%%%%%%%%%%%%%%%%%%%%%%%%%%%%
% %%%%%%%%%%%%%%%%%%%%%%%%%%%%%%%%%%%%%%%%%%%%%%%%%%%%%%%%%%%%%%%%%%%%%%%%%%%%%%
% \section{Implementation}
%\iffalse
%<*package>
%\fi
%
% This section describes the definitions file |childdoc.def|.

% The definitions cannot be loaded using |\usepackage| or |\RequirePackage|
% which has a mechanism to prevent loading a style file more than once.
% When loading the definitions by means of |\input|
% multiple instances have to be prevented manually:
%\iffalse
%This code needs to be before the `\ProvidesFile' directive
%which is defined at the beginning of this file.
%Therefore it is also placed there and commented out here.
%</package>
%<*discard>
%\fi
%    \begin{macrocode}
\ifdefined\childdocmain\endinput\fi
%    \end{macrocode}
%\iffalse
%</discard>
%<*package>
%\fi
%
% \macro{\ifchilddoc}
% \macro{\ifchilddocmanual}
% The conditional |\ifchilddoc| tells whether a
% child (true) or main (false) document is being compiled.
% The conditional |\ifchilddocmanual| tells whether
% the |\includeonly| mechanism is used (false) or
% the selection of child files must be performed manually (true).
% The definitions initialise to false:
%    \begin{macrocode}
\newif\ifchilddoc
\newif\ifchilddocmanual
%    \end{macrocode}

% \macro{\childdocname}
% \macro{\childdocjob}
% The macro |\childdocname| stores the name of the main document
% to be compiled. The macro |\childdocjob| stores the name of
% the document on which the \LaTeX{} compiler was originally invoked.
% The content of |\jobname| cannot be compared
% to filenames specified in the source due to different catcodes.
% The following code rescans |\jobname|, stores the result
% in |\childdocname| and saves a copy in |\childdocjob|:
%    \begin{macrocode}
\edef\childdocname{\scantokens\expandafter{\jobname\noexpand}}
\let\childdocjob\childdocname
%    \end{macrocode}

% \macro{\childdocdisable}
% The macro |\childdocdisable| prevents the main file
% from being processed more than once.
% At this stage, the main document command |\childdocmain|
% is assumed to be called once again where it should do nothing.
% Any subsequent call to it should prevent
% a secondary processing of the main document
% It overwrites the forwarding commands
% |\childdocof| and |\childdocforward|
% with empty macros to prevent further inclusions of the main document:
%    \begin{macrocode}
\newcommand{\childdocdisable}
{
  \renewcommand{\childdocmain}[1]{\renewcommand{\childdocmain}[1]{\endinput}}
  \renewcommand{\childdocof}[1]{}
  \renewcommand{\childdocby}[2][]{}
  \renewcommand{\childdocforward}[2][]{}
  \renewcommand{\childdocdisable}{}
}
%    \end{macrocode}

% \macro{\childdocmain}
% The macro |\childdocmain| is to be called at the top of the main file
% with nothing or the main filename (without extension) as argument.
% First, it breaks loops.
% If the argument is not empty and does not match |\childdocname|
% (which is set by the first inclusion of |childdoc.def|),
% |\ifchilddoc| is set to true, |\includeonly| is applied to the child file
% and |\jobname| is set to the main file
% (for proper handling of |.aux| files):
%    \begin{macrocode}
\newcommand{\childdocmain}[1]
{
  \childdocdisable\childdocmain{}
  \if?#1?\else
    \begingroup
      \def\childdoctmp{#1}
      \ifx\childdoctmp\childdocname
        \def\childdoctmp{}
      \else
        \def\childdoctmp
        {
          \childdoctrue
          \includeonly{\childdocname}
          \def\childdocjob{#1}
          \def\jobname{#1}
        }
      \fi
      \expandafter
    \endgroup
    \childdoctmp
  \fi
}
%    \end{macrocode}

% \macro{\childdocof}
% The command |\childdocof| redirects
% compilation to the main file |#1|.
%    \begin{macrocode}
\newcommand{\childdocof}[1]
{
  \childdocdisable
  \childdoctrue
  \includeonly{\childdocname}
  \def\jobname{#1}
  \def\childdocjob{#1}
  \input{#1}
}
%    \end{macrocode}

% \macro{\childdocby}
% The command |\childdocby| ....
%    \begin{macrocode}
\newcommand{\childdocby}[2][]
{
  \childdocdisable
  \childdoctrue
  \childdocmanualtrue
  \if?#1?\else
    \def\jobname{#2}
  \fi
  \def\childdocjob{#2}
  \input{#2}
  \endinput
}
%    \end{macrocode}

% \macro{\childdocforward}
% The command |\childdocforward| redirects
% compilation to the main file or
% (if the optional argument is given) a child file.
% Parameters are set as if the main file
% or a child file starting with |\childdocof| was compiled.
% Then compilation is handed over to the main file:
%    \begin{macrocode}
\newcommand{\childdocforward}[2][]
{
  \begingroup
    \if?#1?
      \def\childdoctmp
      {
        \def\childdocname{#2}
        \def\childdocjob{#2}
        \def\jobname{#2}
        \input{#2}
        \endinput
      }
    \else
      \def\childdoctmp
      {
        \childdocdisable
        \def\childdocname{#2}
        \childdoctrue
        \includeonly{#2}
        \def\childdocjob{#1}
        \def\jobname{#1}
        \input{#1}
        \endinput
      }
    \fi
    \expandafter
  \endgroup
  \childdoctmp
}
%    \end{macrocode}

% \macro{\childdocforwardprefix}
% The command |\childdocforwardprefix| redirects
% compilation to the main or a child file by means of a pattern.
% The prefix |#1| in the current filename is replaced by |#2|
% and the suffix of the current filename is kept
% (it is assumed that the filename does not contain the substring `|~~~|'
% which is used as a delimiter).
% Compilation is handed over to the new file by |\childdocforward|:
%    \begin{macrocode}
\newcommand{\childdocforwardprefix}[3][]
{
  \begingroup
    \def\childdocextract #2##1~~~{\def\childdoctmp{\childdocforward[#1]{#3##1}}}
    \expandafter\childdocextract\childdocname~~~
    \expandafter
  \endgroup
  \childdoctmp
}
%    \end{macrocode}

% \macro{\childdoc}
% The deprecated macro |\childdoc| is a legacy version of |\childdocmain|:
%    \begin{macrocode}
\newcommand{\childdoc}{\childdocmain}
%    \end{macrocode}

% \macro{\childdocredirect}
% The deprecated macro |\childdocredirect| is a legacy version
% of |\childdocforward| and |\childdocforwardprefix|:
%    \begin{macrocode}
\newcommand{\childdocredirect}[2][]
{
  \begingroup
    \if?#1?
      \def\childdoctmp{\childdocforward{#2}}
    \else
      \def\childdoctmp{\childdocforwardprefix{#1}{#2}}
    \fi
    \expandafter
  \endgroup
  \childdoctmp
}
%    \end{macrocode}

%\iffalse
%</package>
%\fi
%
\endinput
|\\
|\childdocby{|\textit{main}|}|\\
\end{tabular}
\end{center}
%
The directive |\childdocby| is similar to |\childdocof|
described in \secref{sec:include},
but the subsequent selection of content must be done manually.
To that end, both |\ifchilddoc| and |\ifchilddocmanual|
will be true upon processing of a part,
and the name of the part is stored in |\childdocname|.
Note that |\jobname| will be set to the filename of the current part
so that each part receives an individual |.aux| file
that does not interfere with the |.aux| file(s) of the main document.
This behaviour can be altered by the alternative form
|\childdocby[*]{|\textit{main}|}| (with a non-empty optional argument)
which uses the |.aux| file of the main document
by setting |\jobname| to \textit{main}.

%%%%%%%%%%%%%%%%%%%%%%%%%%%%%%%%%%%%%%%%%%%%%%%%%%%%%%%%%%%%%%%%%%%%%%%%%%%%%%%%
\subsection{Driver Development}
\label{sec:driver}

The \textsf{childdoc} mechanism can also be use for the development
of definition files such as \LaTeX{} styles or classes.
This case differs from the above setup with multiple parts
included by |\include| in that no |\includeonly| should be invoked.
This can be achieved by starting the include file
(before |\ProvidesPackage|) with:
%
\begin{center}
\begin{tabular}{l}
|% \iffalse
%
% childdoc.dtx Copyright (C) 2017-2018 Niklas Beisert
%
% This work may be distributed and/or modified under the
% conditions of the LaTeX Project Public License, either version 1.3
% of this license or (at your option) any later version.
% The latest version of this license is in
%   http://www.latex-project.org/lppl.txt
% and version 1.3 or later is part of all distributions of LaTeX
% version 2005/12/01 or later.
%
% This work has the LPPL maintenance status `maintained'.
%
% The Current Maintainer of this work is Niklas Beisert.
%
% This work consists of the files childdoc.dtx and childdoc.ins
% and the derived files childdoc.def and cdocsamp.tex with
% cdocsch1.tex, cdocsch2.tex, cdocsdrf.tex, cdocsfn1.tex, cdocsfn2.tex.
%
%<package>\ifdefined\childdocmain\endinput\fi
%<package>\ProvidesFile{childdoc.def}[2018/12/30 v2.0 child document driver]
%<samplemain>\ProvidesFile{cdocsamp.tex}[2018/12/30 v2.0 sample for childdoc]
%<*driver>
%\ProvidesFile{childdoc.drv}[2018/12/30 v2.0 childdoc reference manual file]
\PassOptionsToClass{10pt,a4paper}{article}
\documentclass{ltxdoc}

\usepackage[margin=35mm]{geometry}
\usepackage{hyperref}
\usepackage{hyperxmp}
\usepackage[usenames]{color}

\hypersetup{colorlinks=true}
\hypersetup{pdfstartview=FitH}
\hypersetup{pdfpagemode=UseNone}
\hypersetup{pdfsource={}}
\hypersetup{pdflang={en-UK}}
\hypersetup{pdfcopyright={Copyright 2017-2018 Niklas Beisert.
  This work may be distributed and/or modified under the
  conditions of the LaTeX Project Public License, either version 1.3
  of this license or (at your option) any later version.}}
\hypersetup{pdflicenseurl={http://www.latex-project.org/lppl.txt}}
\hypersetup{pdfcontactaddress={ETH Zurich, ITP, HIT K,
  Wolfgang-Pauli-Strasse 27}}
\hypersetup{pdfcontactpostcode={8093}}
\hypersetup{pdfcontactcity={Zurich}}
\hypersetup{pdfcontactcountry={Switzerland}}
\hypersetup{pdfcontactemail={nbeisert@itp.phys.ethz.ch}}
\hypersetup{pdfcontacturl={http://people.phys.ethz.ch/\xmptilde nbeisert/}}

\newcommand{\secref}[1]{\hyperref[#1]{section \ref*{#1}}}

\parskip1ex
\parindent0pt
\let\olditemize\itemize
\def\itemize{\olditemize\parskip0pt}

\begin{document}

\title{The \textsf{childdoc} Package}
\hypersetup{pdftitle={The childdoc Package}}
\author{Niklas Beisert\\[2ex]
  Institut f\"ur Theoretische Physik\\
  Eidgen\"ossische Technische Hochschule Z\"urich\\
  Wolfgang-Pauli-Strasse 27, 8093 Z\"urich, Switzerland\\[1ex]
  \href{mailto:nbeisert@itp.phys.ethz.ch}
  {\texttt{nbeisert@itp.phys.ethz.ch}}}
\hypersetup{pdfauthor={Niklas Beisert}}
\hypersetup{pdfsubject={Manual for the LaTeX2e Package childdoc}}
\date{30 December 2018, \textsf{v2.0}}
\maketitle

\begin{abstract}\noindent
\textsf{childdoc} is a \LaTeXe{} package
that enables the direct compilation
of document sections included by |\include|
to individual files.
\end{abstract}

\begingroup
\parskip0ex
\tableofcontents
\endgroup

%%%%%%%%%%%%%%%%%%%%%%%%%%%%%%%%%%%%%%%%%%%%%%%%%%%%%%%%%%%%%%%%%%%%%%%%%%%%%%%%
%%%%%%%%%%%%%%%%%%%%%%%%%%%%%%%%%%%%%%%%%%%%%%%%%%%%%%%%%%%%%%%%%%%%%%%%%%%%%%%%
\section{Introduction}

\LaTeX{} provides a mechanism to structure a large document (such as a book)
into a main file and several child files (containing the chapters)
using the |\include| command.
This mechanism is beneficial for documents
which span hundreds of pages in order to
make the source file(s) more manageable.
Moreover, compilation can be restricted to
selected child files by means of the |\includeonly| command.
The latter feature can be used to reduce the compilation time while editing
(this was significantly more useful in the earlier days of \LaTeX{})
or to generate a smaller document which is easier to navigate.
Another application of |\includeonly| is to generate
documents consisting of selected parts of the complete document.

However, there are a few drawbacks of the plain |\include| mechanism:
\begin{itemize}
\item
The child files cannot be compiled on their own,
they can only be compiled via the main file.
A naive editing environment
(such as a text editor with an option
to have the current file processed by \LaTeX)
may require one to switch to the main file before compiling;
attempting to compile the child file produces errors.
\item
The main file must be modified (each time)
to adjust the |\includeonly| command
to the present needs. This easily leaves the main file in a messy state.
\item
The generated document will always carry the filename
of the main document. This is inconvenient if
several child files are to be compiled and
to be kept for distribution.
\end{itemize}

The present package provides a simple interface
to make child files individually compilable by \LaTeX{}.
Compiling a child file then has the same effect as compiling
the main file with an |\includeonly| command
to select the appropriate child.
Moreover the generated document will carry the name of the child
rather than the main file.
This resolves all three above issues.

This feature is meant to make the editing of books,
thesis documents and lecture notes somewhat more convenient.
However, the package can also be used efficiently for
composing a series of documents (such as exercise sheets)
which are typically distributed individually.
It then assists the author in generating the individual documents
(potentially in different versions)
as well as a document containing the collected series.
Another application is in developing style files
or other kinds of included material
where compilation of the style file could redirect
to a sample or test file.

%%%%%%%%%%%%%%%%%%%%%%%%%%%%%%%%%%%%%%%%%%%%%%%%%%%%%%%%%%%%%%%%%%%%%%%%%%%%%%%%
%%%%%%%%%%%%%%%%%%%%%%%%%%%%%%%%%%%%%%%%%%%%%%%%%%%%%%%%%%%%%%%%%%%%%%%%%%%%%%%%
\section{Usage}

First of all, the package \textsf{childdoc} is \emph{not} a standard
\LaTeXe{} |.sty| style file! Therefore it needs to be invoked in
a non-standard way.

%%%%%%%%%%%%%%%%%%%%%%%%%%%%%%%%%%%%%%%%%%%%%%%%%%%%%%%%%%%%%%%%%%%%%%%%%%%%%%%%
\subsection{Included Files}
\label{sec:include}

%%%%%%%%%%%%%%%%%%%%%%%%%%%%%%%%%%%%%%%%
\DescribeMacro{\childdocmain}
To use the package, add the commands
\begin{center}
\begin{tabular}{l}
|\input{childdoc.def}|\\
|\childdocmain{}|\\
\end{tabular}
\end{center}
at the very top of the main \LaTeX{} file,
in particular \emph{before} the |\documentclass| statement!
The argument of |\childdocmain| should be left empty
(but it must be present).

%%%%%%%%%%%%%%%%%%%%%%%%%%%%%%%%%%%%%%%%
\DescribeMacro{\childdocof}
Furthermore, add the commands
\begin{center}
\begin{tabular}{l}
|\input{childdoc.def}|\\
|\childdocof{|\textit{main}|}|\\
\end{tabular}
\end{center}
at the top of every child file \textit{child}
which is included by |\include{|\textit{child}|}|
from within the main file
(or at least for those files to be compiled individually).
The argument \textit{main} must be the filename of the main file.

There are a couple of
considerations in setting up the main and child documents:

%%%%%%%%%%%%%%%%%%%%%%%%%%%%%%%%%%%%%%%%
\paragraph{Restrictions.}

Please note the following restrictions:
\begin{itemize}
\item
|\childdocmain| must be called with one argument \textit{main}
to ensure compatibility with earlier version of the package.
It must either be empty (|\childdocmain{}|)
or precisely match the filename of the main file in which it is specified.
See \secref{sec:detection} for further information.
\item
The filename \textit{main} must be specified without the |.tex| extension.
\item
The filename \textit{main} is case sensitive
(even in case-insensitive file systems)
due to internal string comparison.
\item
The argument \textit{main} should be fully expanded, it cannot be a macro.
\item
Subdirectories and special characters should be avoided in filenames.
\item
The command |\childdocmain{|\textit{main}|}| must be followed by a whitespace.
It should not be followed immediately by another command
or by a comment mark `|%|'.
This is because the \TeX{} parser reads the token immediately following
the argument of |\childdocmain| and puts it
at the beginning of every child section;
however, a white\-space is ignored.
\end{itemize}

%%%%%%%%%%%%%%%%%%%%%%%%%%%%%%%%%%%%%%%%
\paragraph{Content of Main File.}

It is advisable to place all content in the child files included by |\include|.
Any output contained in the main file will appear in all child documents
unless suppressed manually;
it cannot be suppressed automatically by the |\includeonly| directive
and thus should normally be avoided.
A method to include some content in the main file
by means of conditional processing is described in \secref{sec:conditional}.

%%%%%%%%%%%%%%%%%%%%%%%%%%%%%%%%%%%%%%%%
\paragraph{Page Numbering.}

When only a part of the document is compiled,
the appropriate numbering of pages
(as well as other status parameters)
is determined from the |.aux| files.
The latter contain information from previous passes.
However this information needs to propagate through
all intermediate child documents.
Therefore the page numbering in child documents may well
be inconsistent until the complete document is compiled at least once.

A useful (if unconventional) way to always ensure a consistent
page numbering is to restart the numbering in each child document
and denote the pages by `\textit{child}|.|\textit{page}'
where \textit{child} represents the chapter/section number of the child file.
This can be achieved by the command
|\numberwithin{page}{|\textit{child}|}|
of the \textsf{amsmath} package
where \textit{child} can be |chapter| or |section|
depending on the chosen structuring.
Alternatively, one can modify the macro |\thepage| appropriately
and reset the counter |page| at the start of each child file.

%%%%%%%%%%%%%%%%%%%%%%%%%%%%%%%%%%%%%%%%%%%%%%%%%%%%%%%%%%%%%%%%%%%%%%%%%%%%%%%%
\subsection{Conditional Processing}
\label{sec:conditional}

The package provides a mechanism to compile different versions
of a document. To customise the versions further some conditional processing
can come in handy to distinguish which version is being compiled.
The package provides two macros to describe the compilation context:

%%%%%%%%%%%%%%%%%%%%%%%%%%%%%%%%%%%%%%%%
\DescribeMacro{\ifchilddoc}
The conditional |\ifchilddoc| distinguishes between the compilation of
child documents and the main document:
%
\begin{center}
|\ifchilddoc |\textit{child-code}| |[|\||else |\textit{main-code}]| \||fi|
\end{center}

%%%%%%%%%%%%%%%%%%%%%%%%%%%%%%%%%%%%%%%%
\DescribeMacro{\childdocname}
\DescribeMacro{\childdocjob}
The macro |\childdocname| contains the filename (without extension)
of the main or child file being processed.
Note that |\childdocjob| will always contain the name of the main file.

%%%%%%%%%%%%%%%%%%%%%%%%%%%%%%%%%%%%%%%%
\paragraph{Title Page.}

Conditional processing can be used to include a title or banner page
in the main document when proper precautions are taken.
Importantly, the code in the main file should ensure that the page counter
(as well as other status parameters which are stored in the |.aux| files)
takes the same value after the conditional processing.
Otherwise the page numbers may take divergent values
depending on which part is compiled.

For example, a title page could be declared by:
%
\begin{center}
\begin{tabular}{l}
|\ifchilddoc\||else|\\
|\addtocounter{page}{-1}|\\
\textit{code for title page}\\
|\newpage|\\
|\||fi|
\end{tabular}
\end{center}
%
A banner page for the child documents can be generated by:
%
\begin{center}
\begin{tabular}{l}
|\ifchilddoc|\\
|\addtocounter{page}{-1}|\\
\textit{code for banner page}\\
|\newpage|\\
|\||fi|
\end{tabular}
\end{center}
%
Here one could write a message such as:
\begin{center}
|This is the part \childdocname{} of \childdocjob{}.|
\end{center}

%%%%%%%%%%%%%%%%%%%%%%%%%%%%%%%%%%%%%%%%%%%%%%%%%%%%%%%%%%%%%%%%%%%%%%%%%%%%%%%%
\subsection{Flags}
\label{sec:flags}

The package makes it easy to generate different versions
of the main or child documents.
To this end compilation flags can be defined
and assigned different default values.
They will be particularly useful in conjunction
with the forwarding mechanism described in \secref{sec:forward}.

For example, it may be useful to have a flag |\version|
which can be set to |draft| or |final|.
The document source will contain some conditional code
depending on the value of |\version|.
Suppose further, the flag should default to |final| for the main file
and to |draft| for child files
which is a natural assignment for editing the document.
This is achieved by placing the following code
in the preamble of the main document
(below the |\childdocmain| directive):
%
\begin{center}
\begin{tabular}{l}
|\ifchilddoc|\\
|\providecommand{\version}{draft}|\\
|\||else|\\
|\providecommand{\version}{final}|\\
|\||fi|
\end{tabular}
\end{center}
%
The definition by |\providecommand| makes sure
that previous definitions are not overwritten.
Further statements |\providecommand{\version}{...}|
can thus be added before the above code to override it.

For the main file, one might add a line
(between |\childdocmain| and the above block)
%
\begin{center}
|%\ifchilddoc\||else\providecommand{\version}{draft}\||fi|
\end{center}
%
which can be uncommented to produce a draft version.
Likewise one can add a line to the very top of a child file
(above the |\childdocof{|\textit{main}|}| directive)
%
\begin{center}
|%\providecommand{\version}{final}|
\end{center}
%
which can be uncommented to produce the final version of this child document.

%%%%%%%%%%%%%%%%%%%%%%%%%%%%%%%%%%%%%%%%%%%%%%%%%%%%%%%%%%%%%%%%%%%%%%%%%%%%%%%%
\subsection{Forwarding}
\label{sec:forward}

Different versions of the main or child documents
using compilation flags as described in \secref{sec:flags}
can be (permanently) stored in different files
for convenient compilation, viewing and distribution.
To this end, the package defines a command
to pass on compilation to a different file:

%%%%%%%%%%%%%%%%%%%%%%%%%%%%%%%%%%%%%%%%
\DescribeMacro{\childdocforward}
The command |\childdocforward| redirects processing to
another source file:
%
\begin{center}
\begin{tabular}{l}
|\input{childdoc.def}|\\
|\childdocforward[|\textit{main}|]{|\textit{dest}|}|\\
\end{tabular}
\end{center}
%
The argument \textit{dest} is the destination file
(without extension).
It should be the main file or one of the child files.
Note that further \textsf{childdoc} directives
such as |\childdocof| and |\childdocforward|
in the indicated file will be processed in this form.
The optional argument \textit{main}
passes on directly to the main file \textit{main}
while pretending to compile the child \textit{dest}.
This form behaves as if \textit{dest}
issues |\childdocof{|\textit{main}|}| right away,
and no further \textsf{childdoc} directives will be processed.

%%%%%%%%%%%%%%%%%%%%%%%%%%%%%%%%%%%%%%%%
\DescribeMacro{\...prefix}
In the alternative form |\childdocforwardprefix|,
%
\begin{center}
\begin{tabular}{l}
|\input{childdoc.def}|\\
|\childdocforwardprefix[|\textit{main}|]{|\textit{prefix}|}{|\textit{dest}|}|
\end{tabular}
\end{center}
%
the destination file is determined by a pattern
depending on the current file:
To make this work, the current file must be called
`{\textit{prefix}\hspace{0.2em}\textit{suffix}}'
with \textit{prefix} matching precisely the argument.
Processing is then passed on to the file
`{\textit{dest}\hspace{0.2em}\textit{suffix}}'.
Surely, the same effect is achieved by
directly specifying the
argument `{\textit{dest}\hspace{0.2em}\textit{suffix}}'
in the first form.
However, that requires to set up a different file
for each child. With the alternative form of the command
all these files can have exactly the same content
which simplifies setting them up and maintaining them.

For example, the following file |draft.tex|
with a compilation flag |\version| as described in \secref{sec:flags}
compiles the main document as a draft:
%
\begin{center}
\begin{tabular}{l}
|\def\version{draft}|\\
|\input{childdoc.def}|\\
|\childdocforward{|\textit{main}|}|
\end{tabular}
\end{center}
%
Likewise, the following files |final|\textit{nn}|.tex|
compile the final version of the child document
|child|\textit{nn}|.tex|:
%
\begin{center}
\begin{tabular}{l}
|\def\version{final}|\\
|\input{childdoc.def}|\\
|\childdocforwardprefix{final}{child}|
\end{tabular}
\end{center}
%

Note that when several versions of a main file and/or of each child file
are to be generated, it may be convenient to set up a |Makefile| or
shell script to automatise the process.

%%%%%%%%%%%%%%%%%%%%%%%%%%%%%%%%%%%%%%%%%%%%%%%%%%%%%%%%%%%%%%%%%%%%%%%%%%%%%%%%
\subsection{Command Line Processing}
\label{sec:commandline}

The effect of redirection files can also be achieved by invoking
the \LaTeX{} compiler with a more elaborate command line.
Most conveniently this should be done as part
of a shell script or a |Makefile|.

When using \textsf{childdoc} in the main file, the following
command lines effectively perform a redirection
(note that depending on the shell being used,
backslashes may have to be doubled: `|\|' $\to$ `|\\|'):
%
\begin{center}
|... -jobname "|\textit{target}|" |\\|"|[\textit{flags}]%
|\input{childdoc.def}\childdocforward[|\textit{main}|]{|\textit{dest}|}"|
\end{center}
%
Here \textit{target} is the name of the output file,
\textit{main} is the name of the main file
and \textit{dest} is the name of the main or child file to be processed
(all filenames without extensions).
The optional argument \textit{main} can be omitted
if \textit{main} matches \textit{dest}.
Optionally, compilation \textit{flags} can be defined via |\def| commands.
This command line makes the \TeX{} engine believe
it is compiling the file \textit{target}
whose content is specified as the latter parameter.
The provided code then forwards the processing to
\textit{main} or \textit{dest} as described in \secref{sec:forward}.

%%%%%%%%%%%%%%%%%%%%%%%%%%%%%%%%%%%%%%%%%%%%%%%%%%%%%%%%%%%%%%%%%%%%%%%%%%%%%%%%
\subsection{Include by Input}
\label{sec:input}

Including child documents by |\include| has some restrictions by design.
Most notably, the content of a child document always occupies
its own set of pages; pages cannot be shared between child documents.
Usually, this behaviour makes perfect sense
because each child document contain an essential part of the document.
However, in some situations it may be desirable to compose
a document from a collection of parts
without having mandatory page breaks between then.
For this case, the package
provides a mechanism to include parts
by |\input| which can also be processed individually.
However, by construction this mechanism
requires manual handling of the content to be output.

%%%%%%%%%%%%%%%%%%%%%%%%%%%%%%%%%%%%%%%%
\DescribeMacro{\ifchilddocmanual}
The main file should be prepared as usual, see \secref{sec:include}.
However, the document body must make a distinction
between processing of an individual part and of the main document, e.g.:
%
\begin{center}
\begin{tabular}{l}
|\ifchilddocmanual|\\
|\input{\childdocname}|\\
|\||else|\\
\textit{document body with }|\input{|\textit{part}|}|\\
|\||fi|
\end{tabular}
\end{center}
%
The conditional |\ifchilddocmanual| is true whenever
a part to be included by |\input| is being compiled,
and the name of the part is stored in |\childdocname|.

%%%%%%%%%%%%%%%%%%%%%%%%%%%%%%%%%%%%%%%%
\DescribeMacro{\childdocby}
Each part to be included by |\input| should start with:
%
\begin{center}
\begin{tabular}{l}
|\input{childdoc.def}|\\
|\childdocby{|\textit{main}|}|\\
\end{tabular}
\end{center}
%
The directive |\childdocby| is similar to |\childdocof|
described in \secref{sec:include},
but the subsequent selection of content must be done manually.
To that end, both |\ifchilddoc| and |\ifchilddocmanual|
will be true upon processing of a part,
and the name of the part is stored in |\childdocname|.
Note that |\jobname| will be set to the filename of the current part
so that each part receives an individual |.aux| file
that does not interfere with the |.aux| file(s) of the main document.
This behaviour can be altered by the alternative form
|\childdocby[*]{|\textit{main}|}| (with a non-empty optional argument)
which uses the |.aux| file of the main document
by setting |\jobname| to \textit{main}.

%%%%%%%%%%%%%%%%%%%%%%%%%%%%%%%%%%%%%%%%%%%%%%%%%%%%%%%%%%%%%%%%%%%%%%%%%%%%%%%%
\subsection{Driver Development}
\label{sec:driver}

The \textsf{childdoc} mechanism can also be use for the development
of definition files such as \LaTeX{} styles or classes.
This case differs from the above setup with multiple parts
included by |\include| in that no |\includeonly| should be invoked.
This can be achieved by starting the include file
(before |\ProvidesPackage|) with:
%
\begin{center}
\begin{tabular}{l}
|\input{childdoc.def}|\\
|\childdocforward{|\textit{main}|}|\\
\end{tabular}
\end{center}
%
or alternatively with:
%
\begin{center}
\begin{tabular}{l}
|\input{childdoc.def}|\\
|\childdocby{|\textit{main}|}|\\
\end{tabular}
\end{center}
%
Both forms have slightly different effects as described above.
The main file is prepared as usual, see \secref{sec:include}.

%%%%%%%%%%%%%%%%%%%%%%%%%%%%%%%%%%%%%%%%%%%%%%%%%%%%%%%%%%%%%%%%%%%%%%%%%%%%%%%%
\subsection{Legacy Detection}
\label{sec:detection}

The directive |\childdocmain| in the main file can detect
whether the complete document or merely a child is to be compiled
even without using the directive |\childdocof|.
This method is deprecated because it is less robust
and there is no compelling reason to use it;
it is merely provided for backward compatibility
and it may be removed in future versions.

If the detection mechanism is to be used,
it is mandatory to correctly specify
the filename of the main file as the argument of |\childdocmain|:
%
\begin{center}
\begin{tabular}{l}
|\input{childdoc.def}|\\
|\childdocmain{|\textit{main}|}|\\
\end{tabular}
\end{center}
%
If |\jobname| does not match the argument \textit{main} of |\childdocmain|,
it is assumed that |\jobname| points to the child file to be compiled.
When using |\childdocmain| with the main file specified as argument,
it suffices to start a child file
with just |\input{|\textit{main}|}|
without loading of the package and using |\childdocof|.
If instead all processing is done
with the appropriate \textsf{childdoc} directives,
the argument of \textit{main} of |\childdocmain| can be empty.

An alternative version of the command line processing described
in \secref{sec:commandline} using the detection mechanism reads:
%
\begin{center}
|... -jobname "|\textit{target}|" "|[\textit{flags}]%
[|\def\jobname{|\textit{dest}|}|]|\input{|\textit{main}|}"|
\end{center}

%%%%%%%%%%%%%%%%%%%%%%%%%%%%%%%%%%%%%%%%%%%%%%%%%%%%%%%%%%%%%%%%%%%%%%%%%%%%%%%%
\subsection{Manual Code}
\label{sec:manual}

In case one cannot be certain whether the definitions file |childdoc.def|
is installed on the target \TeX{} distribution
and one prefers not to ship it,
it is conceivable to paste a few relevant commands into the sources.

To that end, drop all statements |\input{childdoc.def}|
and perform the replacements as outlined below.
Instead of |\childdocmain{|\textit{main}|}| add the following code
to the top of the main file:
%
\begin{center}
\begin{tabular}{l}
|\||ifdefined\childdocname\endinput\||fi\newif\ifchilddoc|\\
|\edef\childdocname{\scantokens\expandafter{\jobname\noexpand}}|\\
|\def\childdocmain{|\textit{main}|}\||ifx\childdocmain\childdocname\||else|\\
|\childdoctrue\includeonly{\childdocname}\let\jobname\childdocmain\||fi|\\
\end{tabular}
\end{center}
%
Instead of |\childdocof{|\textit{main}|}| just include the main file
at the top of each child file:
%
\begin{center}
|\input{|\textit{main}|}|
\end{center}
%
A simple redirection |\childdocforward{|\textit{dest}|}| is achieved by:
%
\begin{center}
|\def\jobname{|\textit{dest}|}\input{\jobname}|
\end{center}
%
The redirection with prefix
|\childdocforwardprefix[|\textit{prefix}|]{|\textit{dest}|}|
is accomplished by:
%
\begin{center}
\begin{tabular}{l}
|{\edef\jobname{\scantokens\expandafter{\jobname\noexpand}}|\\
|\def\redirectjob |\textit{prefix}|#1~~~{\gdef\jobname{|\textit{dest}|#1}}|\\
|\expandafter\redirectjob\jobname~~~}\input{\jobname}|
\end{tabular}
\end{center}

In an alternative approach,
child documents can be compiled by a specific command line
without additional code or specific definitions:
%
\begin{center}
|... -jobname "|\textit{target}|" "|[\textit{flags}]%
|\includeonly{|\textit{dest}|}\input{|\textit{main}|}"|
\end{center}
%

%%%%%%%%%%%%%%%%%%%%%%%%%%%%%%%%%%%%%%%%%%%%%%%%%%%%%%%%%%%%%%%%%%%%%%%%%%%%%%%%
%%%%%%%%%%%%%%%%%%%%%%%%%%%%%%%%%%%%%%%%%%%%%%%%%%%%%%%%%%%%%%%%%%%%%%%%%%%%%%%%
\section{Information}

%%%%%%%%%%%%%%%%%%%%%%%%%%%%%%%%%%%%%%%%%%%%%%%%%%%%%%%%%%%%%%%%%%%%%%%%%%%%%%%%
\subsection{Copyright}

Copyright \copyright{} 2017--2018 Niklas Beisert

This work may be distributed and/or modified under the
conditions of the \LaTeX{} Project Public License, either version 1.3
of this license or (at your option) any later version.
The latest version of this license is in
  \url{http://www.latex-project.org/lppl.txt}
and version 1.3 or later is part of all distributions of \LaTeX{}
version 2005/12/01 or later.

This work has the LPPL maintenance status `maintained'.

The Current Maintainer of this work is Niklas Beisert.

This work consists of the files |README.txt|, |childdoc.ins| and |childdoc.dtx|
as well as the derived files |childdoc.def|, |cdocsamp.tex|
with |cdocsch1.tex|, |cdocsch2.tex|, |cdocspt3.tex|, |cdocspt4.tex|,
|cdocsdrf.tex|, |cdocsfn1.tex|, |cdocsfn2.tex|
as well as |childdoc.pdf|.

%%%%%%%%%%%%%%%%%%%%%%%%%%%%%%%%%%%%%%%%%%%%%%%%%%%%%%%%%%%%%%%%%%%%%%%%%%%%%%%%
\subsection{Files and Installation}

The package consists of the files:
%
\begin{center}
\begin{tabular}{ll}
    |README.txt|   & readme file \\
    |childdoc.ins| & installation file \\
    |childdoc.dtx| & source file \\
    |childdoc.def| & definition file \\
    |cdocsamp.tex| & sample main file \\
    |cdocsch1.tex| & sample include file \\
    |cdocsch2.tex| & sample include file \\
    |cdocspt3.tex| & sample part file \\
    |cdocspt4.tex| & sample part file \\
    |cdocsdrf.tex| & sample redirection file \\
    |cdocsfn1.tex| & sample redirection file \\
    |cdocsfn2.tex| & sample redirection file \\
    |childdoc.pdf| & manual
\end{tabular}
\end{center}
%
The distribution consists of the files
|README.txt|, |childdoc.ins| and |childdoc.dtx|.
%
\begin{itemize}
\item
Run (pdf)\LaTeX{} on |childdoc.dtx|
to compile the manual |childdoc.pdf| (this file).
\item
Run \LaTeX{} on |childdoc.ins| to create the definitions file |childdoc.def|
and the sample |cdocsamp.tex| with include files
|cdocsch1.tex|, |cdocsch2.tex|, |cdocspt3.tex|, |cdocspt4.tex|,
|cdocsdrf.tex|, |cdocsfn1.tex|, |cdocsfn2.tex|.
Then copy the file |childdoc.def| to an appropriate directory of your \LaTeX{}
distribution, e.g.\ \textit{texmf-root}|/tex/latex/childdoc|.
\end{itemize}

%%%%%%%%%%%%%%%%%%%%%%%%%%%%%%%%%%%%%%%%%%%%%%%%%%%%%%%%%%%%%%%%%%%%%%%%%%%%%%%%
\subsection{Related CTAN Packages}

There are several other packages which offer a similar functionality:
%
\begin{itemize}
\item
The packages
\href{http://ctan.org/pkg/docmute}{\textsf{docmute}},
\href{http://ctan.org/pkg/includex}{\textsf{includex}} and
\href{http://ctan.org/pkg/standalone}{\textsf{standalone}}
provide commands to include only the document body of
a child file thus allowing both files to be compiled individually.
\item
The packages \href{http://ctan.org/pkg/subdocs}{\textsf{subdocs}}
and \href{http://ctan.org/pkg/subfiles}{\textsf{subfiles}}
provide structures in which the main and child documents can be
encapsulated and allowing them to be compiled individually.
The inclusion mechanism is different from the conventional |\include|.
\item
The package \href{http://ctan.org/pkg/combine}{\textsf{combine}}
is an elaborate solution to combine several documents into one.
\end{itemize}
%
See also the CTAN topic \href{http://ctan.org/topic/subdocs}{\textsf{subdocs}}
for further related packages.
The present package differs from the above solutions in that
a document structure constructed with the conventional |\include| mechanism
just needs two extra commands at the top of every file
such that all constituent files can be compiled individually.

%%%%%%%%%%%%%%%%%%%%%%%%%%%%%%%%%%%%%%%%%%%%%%%%%%%%%%%%%%%%%%%%%%%%%%%%%%%%%%%%
%\subsection{Feature Suggestions}
%
%The following is a list of features which may be useful for future
%versions of this package:
%%
%\begin{itemize}
%\item
%\ldots
%\end{itemize}

%%%%%%%%%%%%%%%%%%%%%%%%%%%%%%%%%%%%%%%%%%%%%%%%%%%%%%%%%%%%%%%%%%%%%%%%%%%%%%%%
\subsection{Revision History}

%%%%%%%%%%%%%%%%%%%%%%%%%%%%%%%%%%%%%%%%
\paragraph{v2.0:} 2018/12/30

\begin{itemize}
\item
immediate forward processing
\item
added |\childdocby| mechanism
\item
manual restructured
\end{itemize}

%%%%%%%%%%%%%%%%%%%%%%%%%%%%%%%%%%%%%%%%
\paragraph{v1.6:} 2018/01/17

\begin{itemize}
\item
application for development of include files
\item
corrections to manual
\end{itemize}

%%%%%%%%%%%%%%%%%%%%%%%%%%%%%%%%%%%%%%%%
\paragraph{v1.5:} 2017/05/21

\begin{itemize}
\item
more complete structuring introduced
\item
|\childdocof| introduced
\item
|\childdoc| renamed to |\childdocmain|
\item
|\childredirect| renamed to |\childdocforward| and |\childdocforwardprefix|
and functionality expanded
\end{itemize}

%%%%%%%%%%%%%%%%%%%%%%%%%%%%%%%%%%%%%%%%
\paragraph{v1.0:} 2017/04/27

\begin{itemize}
\item
manual and install package
\item
first version published on CTAN
\end{itemize}

%%%%%%%%%%%%%%%%%%%%%%%%%%%%%%%%%%%%%%%%
\paragraph{v0.6:} 2017/04/26

\begin{itemize}
\item
redirection mechanism added
\end{itemize}

%%%%%%%%%%%%%%%%%%%%%%%%%%%%%%%%%%%%%%%%
\paragraph{v0.5:} 2017/04/26

\begin{itemize}
\item
functionality in definition file
\end{itemize}


%%%%%%%%%%%%%%%%%%%%%%%%%%%%%%%%%%%%%%%%%%%%%%%%%%%%%%%%%%%%%%%%%%%%%%%%%%%%%%%%
%%%%%%%%%%%%%%%%%%%%%%%%%%%%%%%%%%%%%%%%%%%%%%%%%%%%%%%%%%%%%%%%%%%%%%%%%%%%%%%%
%%%%%%%%%%%%%%%%%%%%%%%%%%%%%%%%%%%%%%%%%%%%%%%%%%%%%%%%%%%%%%%%%%%%%%%%%%%%%%%%
\appendix

\settowidth\MacroIndent{\rmfamily\scriptsize 000\ }

 \DocInput{childdoc.dtx}

\end{document}
%</driver>
% \fi
%
% %%%%%%%%%%%%%%%%%%%%%%%%%%%%%%%%%%%%%%%%%%%%%%%%%%%%%%%%%%%%%%%%%%%%%%%%%%%%%%
% %%%%%%%%%%%%%%%%%%%%%%%%%%%%%%%%%%%%%%%%%%%%%%%%%%%%%%%%%%%%%%%%%%%%%%%%%%%%%%
% \section{Sample}
%\iffalse
%<*samplemain>
%\fi
%
% The following presents a sample document
% with two chapters, two parts, a title page,
% a compile flag as well as three forwarding files to set the flag.
% It consists of eight |.tex| files:
% \begin{center}
% \begin{tabular}{ll}
% |cdocsamp.tex|&main file\\
% |cdocsch1.tex|&include file for chapter 1\\
% |cdocsch2.tex|&include file for chapter 2\\
% |cdocspt3.tex|&include file for part 3\\
% |cdocspt4.tex|&include file for part 4\\
% |cdocsdrf.tex|&forwarding file for main file in draft mode\\
% |cdocsfi1.tex|&forwarding file for final version of chapter 1\\
% |cdocsfi2.tex|&forwarding file for final version of chapter 2\\
% \end{tabular}
% \end{center}
% Each of the eight files can be compiled directly by the \LaTeX{} compiler.
%
% %%%%%%%%%%%%%%%%%%%%%%%%%%%%%%%%%%%%%%
% \paragraph{Main File.}
%
% The main file is called |cdocsamp.tex|.
%
% Load the \textsf{childdoc} definitions and
% declare the filename for the main document:
%    \begin{macrocode}
\input{childdoc.def}
\childdocmain{}
%    \end{macrocode}

% Optional override for |\version| flag:
%    \begin{macrocode}
%%\ifchilddoc\else\providecommand{\version}{draft}\fi
%    \end{macrocode}

% Define the default values for the |\version| flag
% (|final| for the main file and |draft| for childs):
%    \begin{macrocode}
\ifchilddoc
\providecommand{\version}{draft}
\else
\providecommand{\version}{final}
\fi
%    \end{macrocode}

% Load the standard document class:
%    \begin{macrocode}
\documentclass[12pt]{article}
%    \end{macrocode}

% Start the document body:
%    \begin{macrocode}
\begin{document}
%    \end{macrocode}

% Declare a title page.
% Print title, part of document being processed and version flag:
%    \begin{macrocode}
\addtocounter{page}{-1}
\begin{center}
{\LARGE\bfseries{}childdoc example\par}
\vspace{1cm}
\ifchilddoc
\ifchilddocmanual part\else chapter\fi:
`\childdocname' of `\childdocjob'\par
\else
main document: `\childdocjob'\par
\fi
version: \version\par
\end{center}
\newpage
%    \end{macrocode}

% Manually include selected file,
% otherwise process as usual:
%    \begin{macrocode}
\ifchilddocmanual
\section*{part `\childdocname'}
\input{\childdocname}
\else
%    \end{macrocode}

% Include the two chapters:
%    \begin{macrocode}
\include{cdocsch1}
\include{cdocsch2}
%    \end{macrocode}

% Include the two parts unless only chapters should be displayed:
%    \begin{macrocode}
\ifchilddoc\else
\section{part three}
\input{cdocspt3}
\section{part four}
\input{cdocspt4}
\fi
%    \end{macrocode}

% Process as usual until here:
%    \begin{macrocode}
\fi
%    \end{macrocode}

% End of document body:
%    \begin{macrocode}
\end{document}
%    \end{macrocode}
%\iffalse
%</samplemain>
%\fi
%
% %%%%%%%%%%%%%%%%%%%%%%%%%%%%%%%%%%%%%%
% \paragraph{Chapter Include Files.}
%
% The include files are called |cdocsch1.tex| and |cdocsch2.tex|.
%
%\iffalse
%<*samplechap1|samplechap2>
%\fi

% Optional override for |\version| flag:
%    \begin{macrocode}
%%\providecommand{\version}{final}
%    \end{macrocode}

% Include the main document:
%    \begin{macrocode}
\input{childdoc.def}
\childdocof{cdocsamp}
%    \end{macrocode}

%\iffalse
%</samplechap1|samplechap2>
%\fi
%
%\iffalse
%<*samplechap1>
%\fi
% Some text for chapter 1:
%    \begin{macrocode}
\section{one}
some text in chapter one
%    \end{macrocode}

%\iffalse
%</samplechap1>
%\fi
% Some text for chapter 2:
%\iffalse
%<*samplechap2>
%\fi
%    \begin{macrocode}
\section{two}
more text in chapter two
%    \end{macrocode}

%\iffalse
%</samplechap2>
%\fi
%
% %%%%%%%%%%%%%%%%%%%%%%%%%%%%%%%%%%%%%%
% \paragraph{Part Include Files.}
%
% The include files are called |cdocspt3.tex| and |cdocspt4.tex|.
%
%\iffalse
%<*samplepart3|samplepart4>
%\fi

% Optional override for |\version| flag:
%    \begin{macrocode}
%%\providecommand{\version}{final}
%    \end{macrocode}

% Include the main document:
%    \begin{macrocode}
\input{childdoc.def}
\childdocby{cdocsamp}
%    \end{macrocode}

%\iffalse
%</samplepart3|samplepart4>
%\fi
%
%\iffalse
%<*samplepart3>
%\fi
% Some text for part 3:
%    \begin{macrocode}
some text in part three
%    \end{macrocode}

%\iffalse
%</samplepart3>
%\fi
% Some text for part 4:
%\iffalse
%<*samplepart4>
%\fi
%    \begin{macrocode}
more text in part four
%    \end{macrocode}

%\iffalse
%</samplepart4>
%\fi
%
% %%%%%%%%%%%%%%%%%%%%%%%%%%%%%%%%%%%%%%
% \paragraph{Forwarding for a Complete Draft.}
%
% The following forwarding file |cdocsdrf.tex|
% compiles the main document in draft mode:
%\iffalse
%<*sampledraft>
%\fi
%    \begin{macrocode}
\def\version{draft}
\input{childdoc.def}
\childdocforward{cdocsamp}
%    \end{macrocode}

%\iffalse
%</sampledraft>
%\fi
%
% %%%%%%%%%%%%%%%%%%%%%%%%%%%%%%%%%%%%%%
% \paragraph{Forwarding for Final Version of the Chapters.}
%
% The following forwarding files |cdocsfn1.tex| and |cdocsfn2.tex|
% (with identical content)
% compile the final versions of the child documents
% |cdocsch1.tex| and |cdocsch2.tex|, respectively:
%\iffalse
%<*samplefinal>
%\fi
%    \begin{macrocode}
\def\version{final}
\input{childdoc.def}
\childdocforwardprefix[cdocsamp]{cdocsfn}{cdocsch}
%    \end{macrocode}

%\iffalse
%</samplefinal>
%\fi
%
% %%%%%%%%%%%%%%%%%%%%%%%%%%%%%%%%%%%%%%
% \paragraph{Command Line Processing.}
%
% The following three command lines generate the output files
% |cdocscld|, |cdocscl1| and |cdocscl2|
% which should be identical to
% |cdocsdrf|, |cdocsch1| and |cdocsfn2|, respectively:
% \begin{center}
% \begin{tabular}{l}
% |latex -jobname cdocscld \|\\
% |  "\def\version{draft}\input{childdoc.def}\childdocforward{cdocsamp}"|\\
% |latex -jobname cdocscl1 \|\\
% |  "\input{childdoc.def}\childdocforward[cdocsamp]{cdocsch1}"|\\
% |latex -jobname cdocscl2 \|\\
% |  "\def\version{final}\input{childdoc.def}\childdocforward{cdocsch2}"|
% \end{tabular}
% \end{center}
% Note that the trailing backslash on each first line
% merely continues the input to the second line
% (for convenient cut ant paste).
% Furthermore, the command |latex| can be replaced by any
% of its alternative versions such as |pdflatex|.
%
% %%%%%%%%%%%%%%%%%%%%%%%%%%%%%%%%%%%%%%%%%%%%%%%%%%%%%%%%%%%%%%%%%%%%%%%%%%%%%%
% %%%%%%%%%%%%%%%%%%%%%%%%%%%%%%%%%%%%%%%%%%%%%%%%%%%%%%%%%%%%%%%%%%%%%%%%%%%%%%
% \section{Implementation}
%\iffalse
%<*package>
%\fi
%
% This section describes the definitions file |childdoc.def|.

% The definitions cannot be loaded using |\usepackage| or |\RequirePackage|
% which has a mechanism to prevent loading a style file more than once.
% When loading the definitions by means of |\input|
% multiple instances have to be prevented manually:
%\iffalse
%This code needs to be before the `\ProvidesFile' directive
%which is defined at the beginning of this file.
%Therefore it is also placed there and commented out here.
%</package>
%<*discard>
%\fi
%    \begin{macrocode}
\ifdefined\childdocmain\endinput\fi
%    \end{macrocode}
%\iffalse
%</discard>
%<*package>
%\fi
%
% \macro{\ifchilddoc}
% \macro{\ifchilddocmanual}
% The conditional |\ifchilddoc| tells whether a
% child (true) or main (false) document is being compiled.
% The conditional |\ifchilddocmanual| tells whether
% the |\includeonly| mechanism is used (false) or
% the selection of child files must be performed manually (true).
% The definitions initialise to false:
%    \begin{macrocode}
\newif\ifchilddoc
\newif\ifchilddocmanual
%    \end{macrocode}

% \macro{\childdocname}
% \macro{\childdocjob}
% The macro |\childdocname| stores the name of the main document
% to be compiled. The macro |\childdocjob| stores the name of
% the document on which the \LaTeX{} compiler was originally invoked.
% The content of |\jobname| cannot be compared
% to filenames specified in the source due to different catcodes.
% The following code rescans |\jobname|, stores the result
% in |\childdocname| and saves a copy in |\childdocjob|:
%    \begin{macrocode}
\edef\childdocname{\scantokens\expandafter{\jobname\noexpand}}
\let\childdocjob\childdocname
%    \end{macrocode}

% \macro{\childdocdisable}
% The macro |\childdocdisable| prevents the main file
% from being processed more than once.
% At this stage, the main document command |\childdocmain|
% is assumed to be called once again where it should do nothing.
% Any subsequent call to it should prevent
% a secondary processing of the main document
% It overwrites the forwarding commands
% |\childdocof| and |\childdocforward|
% with empty macros to prevent further inclusions of the main document:
%    \begin{macrocode}
\newcommand{\childdocdisable}
{
  \renewcommand{\childdocmain}[1]{\renewcommand{\childdocmain}[1]{\endinput}}
  \renewcommand{\childdocof}[1]{}
  \renewcommand{\childdocby}[2][]{}
  \renewcommand{\childdocforward}[2][]{}
  \renewcommand{\childdocdisable}{}
}
%    \end{macrocode}

% \macro{\childdocmain}
% The macro |\childdocmain| is to be called at the top of the main file
% with nothing or the main filename (without extension) as argument.
% First, it breaks loops.
% If the argument is not empty and does not match |\childdocname|
% (which is set by the first inclusion of |childdoc.def|),
% |\ifchilddoc| is set to true, |\includeonly| is applied to the child file
% and |\jobname| is set to the main file
% (for proper handling of |.aux| files):
%    \begin{macrocode}
\newcommand{\childdocmain}[1]
{
  \childdocdisable\childdocmain{}
  \if?#1?\else
    \begingroup
      \def\childdoctmp{#1}
      \ifx\childdoctmp\childdocname
        \def\childdoctmp{}
      \else
        \def\childdoctmp
        {
          \childdoctrue
          \includeonly{\childdocname}
          \def\childdocjob{#1}
          \def\jobname{#1}
        }
      \fi
      \expandafter
    \endgroup
    \childdoctmp
  \fi
}
%    \end{macrocode}

% \macro{\childdocof}
% The command |\childdocof| redirects
% compilation to the main file |#1|.
%    \begin{macrocode}
\newcommand{\childdocof}[1]
{
  \childdocdisable
  \childdoctrue
  \includeonly{\childdocname}
  \def\jobname{#1}
  \def\childdocjob{#1}
  \input{#1}
}
%    \end{macrocode}

% \macro{\childdocby}
% The command |\childdocby| ....
%    \begin{macrocode}
\newcommand{\childdocby}[2][]
{
  \childdocdisable
  \childdoctrue
  \childdocmanualtrue
  \if?#1?\else
    \def\jobname{#2}
  \fi
  \def\childdocjob{#2}
  \input{#2}
  \endinput
}
%    \end{macrocode}

% \macro{\childdocforward}
% The command |\childdocforward| redirects
% compilation to the main file or
% (if the optional argument is given) a child file.
% Parameters are set as if the main file
% or a child file starting with |\childdocof| was compiled.
% Then compilation is handed over to the main file:
%    \begin{macrocode}
\newcommand{\childdocforward}[2][]
{
  \begingroup
    \if?#1?
      \def\childdoctmp
      {
        \def\childdocname{#2}
        \def\childdocjob{#2}
        \def\jobname{#2}
        \input{#2}
        \endinput
      }
    \else
      \def\childdoctmp
      {
        \childdocdisable
        \def\childdocname{#2}
        \childdoctrue
        \includeonly{#2}
        \def\childdocjob{#1}
        \def\jobname{#1}
        \input{#1}
        \endinput
      }
    \fi
    \expandafter
  \endgroup
  \childdoctmp
}
%    \end{macrocode}

% \macro{\childdocforwardprefix}
% The command |\childdocforwardprefix| redirects
% compilation to the main or a child file by means of a pattern.
% The prefix |#1| in the current filename is replaced by |#2|
% and the suffix of the current filename is kept
% (it is assumed that the filename does not contain the substring `|~~~|'
% which is used as a delimiter).
% Compilation is handed over to the new file by |\childdocforward|:
%    \begin{macrocode}
\newcommand{\childdocforwardprefix}[3][]
{
  \begingroup
    \def\childdocextract #2##1~~~{\def\childdoctmp{\childdocforward[#1]{#3##1}}}
    \expandafter\childdocextract\childdocname~~~
    \expandafter
  \endgroup
  \childdoctmp
}
%    \end{macrocode}

% \macro{\childdoc}
% The deprecated macro |\childdoc| is a legacy version of |\childdocmain|:
%    \begin{macrocode}
\newcommand{\childdoc}{\childdocmain}
%    \end{macrocode}

% \macro{\childdocredirect}
% The deprecated macro |\childdocredirect| is a legacy version
% of |\childdocforward| and |\childdocforwardprefix|:
%    \begin{macrocode}
\newcommand{\childdocredirect}[2][]
{
  \begingroup
    \if?#1?
      \def\childdoctmp{\childdocforward{#2}}
    \else
      \def\childdoctmp{\childdocforwardprefix{#1}{#2}}
    \fi
    \expandafter
  \endgroup
  \childdoctmp
}
%    \end{macrocode}

%\iffalse
%</package>
%\fi
%
\endinput
|\\
|\childdocforward{|\textit{main}|}|\\
\end{tabular}
\end{center}
%
or alternatively with:
%
\begin{center}
\begin{tabular}{l}
|% \iffalse
%
% childdoc.dtx Copyright (C) 2017-2018 Niklas Beisert
%
% This work may be distributed and/or modified under the
% conditions of the LaTeX Project Public License, either version 1.3
% of this license or (at your option) any later version.
% The latest version of this license is in
%   http://www.latex-project.org/lppl.txt
% and version 1.3 or later is part of all distributions of LaTeX
% version 2005/12/01 or later.
%
% This work has the LPPL maintenance status `maintained'.
%
% The Current Maintainer of this work is Niklas Beisert.
%
% This work consists of the files childdoc.dtx and childdoc.ins
% and the derived files childdoc.def and cdocsamp.tex with
% cdocsch1.tex, cdocsch2.tex, cdocsdrf.tex, cdocsfn1.tex, cdocsfn2.tex.
%
%<package>\ifdefined\childdocmain\endinput\fi
%<package>\ProvidesFile{childdoc.def}[2018/12/30 v2.0 child document driver]
%<samplemain>\ProvidesFile{cdocsamp.tex}[2018/12/30 v2.0 sample for childdoc]
%<*driver>
%\ProvidesFile{childdoc.drv}[2018/12/30 v2.0 childdoc reference manual file]
\PassOptionsToClass{10pt,a4paper}{article}
\documentclass{ltxdoc}

\usepackage[margin=35mm]{geometry}
\usepackage{hyperref}
\usepackage{hyperxmp}
\usepackage[usenames]{color}

\hypersetup{colorlinks=true}
\hypersetup{pdfstartview=FitH}
\hypersetup{pdfpagemode=UseNone}
\hypersetup{pdfsource={}}
\hypersetup{pdflang={en-UK}}
\hypersetup{pdfcopyright={Copyright 2017-2018 Niklas Beisert.
  This work may be distributed and/or modified under the
  conditions of the LaTeX Project Public License, either version 1.3
  of this license or (at your option) any later version.}}
\hypersetup{pdflicenseurl={http://www.latex-project.org/lppl.txt}}
\hypersetup{pdfcontactaddress={ETH Zurich, ITP, HIT K,
  Wolfgang-Pauli-Strasse 27}}
\hypersetup{pdfcontactpostcode={8093}}
\hypersetup{pdfcontactcity={Zurich}}
\hypersetup{pdfcontactcountry={Switzerland}}
\hypersetup{pdfcontactemail={nbeisert@itp.phys.ethz.ch}}
\hypersetup{pdfcontacturl={http://people.phys.ethz.ch/\xmptilde nbeisert/}}

\newcommand{\secref}[1]{\hyperref[#1]{section \ref*{#1}}}

\parskip1ex
\parindent0pt
\let\olditemize\itemize
\def\itemize{\olditemize\parskip0pt}

\begin{document}

\title{The \textsf{childdoc} Package}
\hypersetup{pdftitle={The childdoc Package}}
\author{Niklas Beisert\\[2ex]
  Institut f\"ur Theoretische Physik\\
  Eidgen\"ossische Technische Hochschule Z\"urich\\
  Wolfgang-Pauli-Strasse 27, 8093 Z\"urich, Switzerland\\[1ex]
  \href{mailto:nbeisert@itp.phys.ethz.ch}
  {\texttt{nbeisert@itp.phys.ethz.ch}}}
\hypersetup{pdfauthor={Niklas Beisert}}
\hypersetup{pdfsubject={Manual for the LaTeX2e Package childdoc}}
\date{30 December 2018, \textsf{v2.0}}
\maketitle

\begin{abstract}\noindent
\textsf{childdoc} is a \LaTeXe{} package
that enables the direct compilation
of document sections included by |\include|
to individual files.
\end{abstract}

\begingroup
\parskip0ex
\tableofcontents
\endgroup

%%%%%%%%%%%%%%%%%%%%%%%%%%%%%%%%%%%%%%%%%%%%%%%%%%%%%%%%%%%%%%%%%%%%%%%%%%%%%%%%
%%%%%%%%%%%%%%%%%%%%%%%%%%%%%%%%%%%%%%%%%%%%%%%%%%%%%%%%%%%%%%%%%%%%%%%%%%%%%%%%
\section{Introduction}

\LaTeX{} provides a mechanism to structure a large document (such as a book)
into a main file and several child files (containing the chapters)
using the |\include| command.
This mechanism is beneficial for documents
which span hundreds of pages in order to
make the source file(s) more manageable.
Moreover, compilation can be restricted to
selected child files by means of the |\includeonly| command.
The latter feature can be used to reduce the compilation time while editing
(this was significantly more useful in the earlier days of \LaTeX{})
or to generate a smaller document which is easier to navigate.
Another application of |\includeonly| is to generate
documents consisting of selected parts of the complete document.

However, there are a few drawbacks of the plain |\include| mechanism:
\begin{itemize}
\item
The child files cannot be compiled on their own,
they can only be compiled via the main file.
A naive editing environment
(such as a text editor with an option
to have the current file processed by \LaTeX)
may require one to switch to the main file before compiling;
attempting to compile the child file produces errors.
\item
The main file must be modified (each time)
to adjust the |\includeonly| command
to the present needs. This easily leaves the main file in a messy state.
\item
The generated document will always carry the filename
of the main document. This is inconvenient if
several child files are to be compiled and
to be kept for distribution.
\end{itemize}

The present package provides a simple interface
to make child files individually compilable by \LaTeX{}.
Compiling a child file then has the same effect as compiling
the main file with an |\includeonly| command
to select the appropriate child.
Moreover the generated document will carry the name of the child
rather than the main file.
This resolves all three above issues.

This feature is meant to make the editing of books,
thesis documents and lecture notes somewhat more convenient.
However, the package can also be used efficiently for
composing a series of documents (such as exercise sheets)
which are typically distributed individually.
It then assists the author in generating the individual documents
(potentially in different versions)
as well as a document containing the collected series.
Another application is in developing style files
or other kinds of included material
where compilation of the style file could redirect
to a sample or test file.

%%%%%%%%%%%%%%%%%%%%%%%%%%%%%%%%%%%%%%%%%%%%%%%%%%%%%%%%%%%%%%%%%%%%%%%%%%%%%%%%
%%%%%%%%%%%%%%%%%%%%%%%%%%%%%%%%%%%%%%%%%%%%%%%%%%%%%%%%%%%%%%%%%%%%%%%%%%%%%%%%
\section{Usage}

First of all, the package \textsf{childdoc} is \emph{not} a standard
\LaTeXe{} |.sty| style file! Therefore it needs to be invoked in
a non-standard way.

%%%%%%%%%%%%%%%%%%%%%%%%%%%%%%%%%%%%%%%%%%%%%%%%%%%%%%%%%%%%%%%%%%%%%%%%%%%%%%%%
\subsection{Included Files}
\label{sec:include}

%%%%%%%%%%%%%%%%%%%%%%%%%%%%%%%%%%%%%%%%
\DescribeMacro{\childdocmain}
To use the package, add the commands
\begin{center}
\begin{tabular}{l}
|\input{childdoc.def}|\\
|\childdocmain{}|\\
\end{tabular}
\end{center}
at the very top of the main \LaTeX{} file,
in particular \emph{before} the |\documentclass| statement!
The argument of |\childdocmain| should be left empty
(but it must be present).

%%%%%%%%%%%%%%%%%%%%%%%%%%%%%%%%%%%%%%%%
\DescribeMacro{\childdocof}
Furthermore, add the commands
\begin{center}
\begin{tabular}{l}
|\input{childdoc.def}|\\
|\childdocof{|\textit{main}|}|\\
\end{tabular}
\end{center}
at the top of every child file \textit{child}
which is included by |\include{|\textit{child}|}|
from within the main file
(or at least for those files to be compiled individually).
The argument \textit{main} must be the filename of the main file.

There are a couple of
considerations in setting up the main and child documents:

%%%%%%%%%%%%%%%%%%%%%%%%%%%%%%%%%%%%%%%%
\paragraph{Restrictions.}

Please note the following restrictions:
\begin{itemize}
\item
|\childdocmain| must be called with one argument \textit{main}
to ensure compatibility with earlier version of the package.
It must either be empty (|\childdocmain{}|)
or precisely match the filename of the main file in which it is specified.
See \secref{sec:detection} for further information.
\item
The filename \textit{main} must be specified without the |.tex| extension.
\item
The filename \textit{main} is case sensitive
(even in case-insensitive file systems)
due to internal string comparison.
\item
The argument \textit{main} should be fully expanded, it cannot be a macro.
\item
Subdirectories and special characters should be avoided in filenames.
\item
The command |\childdocmain{|\textit{main}|}| must be followed by a whitespace.
It should not be followed immediately by another command
or by a comment mark `|%|'.
This is because the \TeX{} parser reads the token immediately following
the argument of |\childdocmain| and puts it
at the beginning of every child section;
however, a white\-space is ignored.
\end{itemize}

%%%%%%%%%%%%%%%%%%%%%%%%%%%%%%%%%%%%%%%%
\paragraph{Content of Main File.}

It is advisable to place all content in the child files included by |\include|.
Any output contained in the main file will appear in all child documents
unless suppressed manually;
it cannot be suppressed automatically by the |\includeonly| directive
and thus should normally be avoided.
A method to include some content in the main file
by means of conditional processing is described in \secref{sec:conditional}.

%%%%%%%%%%%%%%%%%%%%%%%%%%%%%%%%%%%%%%%%
\paragraph{Page Numbering.}

When only a part of the document is compiled,
the appropriate numbering of pages
(as well as other status parameters)
is determined from the |.aux| files.
The latter contain information from previous passes.
However this information needs to propagate through
all intermediate child documents.
Therefore the page numbering in child documents may well
be inconsistent until the complete document is compiled at least once.

A useful (if unconventional) way to always ensure a consistent
page numbering is to restart the numbering in each child document
and denote the pages by `\textit{child}|.|\textit{page}'
where \textit{child} represents the chapter/section number of the child file.
This can be achieved by the command
|\numberwithin{page}{|\textit{child}|}|
of the \textsf{amsmath} package
where \textit{child} can be |chapter| or |section|
depending on the chosen structuring.
Alternatively, one can modify the macro |\thepage| appropriately
and reset the counter |page| at the start of each child file.

%%%%%%%%%%%%%%%%%%%%%%%%%%%%%%%%%%%%%%%%%%%%%%%%%%%%%%%%%%%%%%%%%%%%%%%%%%%%%%%%
\subsection{Conditional Processing}
\label{sec:conditional}

The package provides a mechanism to compile different versions
of a document. To customise the versions further some conditional processing
can come in handy to distinguish which version is being compiled.
The package provides two macros to describe the compilation context:

%%%%%%%%%%%%%%%%%%%%%%%%%%%%%%%%%%%%%%%%
\DescribeMacro{\ifchilddoc}
The conditional |\ifchilddoc| distinguishes between the compilation of
child documents and the main document:
%
\begin{center}
|\ifchilddoc |\textit{child-code}| |[|\||else |\textit{main-code}]| \||fi|
\end{center}

%%%%%%%%%%%%%%%%%%%%%%%%%%%%%%%%%%%%%%%%
\DescribeMacro{\childdocname}
\DescribeMacro{\childdocjob}
The macro |\childdocname| contains the filename (without extension)
of the main or child file being processed.
Note that |\childdocjob| will always contain the name of the main file.

%%%%%%%%%%%%%%%%%%%%%%%%%%%%%%%%%%%%%%%%
\paragraph{Title Page.}

Conditional processing can be used to include a title or banner page
in the main document when proper precautions are taken.
Importantly, the code in the main file should ensure that the page counter
(as well as other status parameters which are stored in the |.aux| files)
takes the same value after the conditional processing.
Otherwise the page numbers may take divergent values
depending on which part is compiled.

For example, a title page could be declared by:
%
\begin{center}
\begin{tabular}{l}
|\ifchilddoc\||else|\\
|\addtocounter{page}{-1}|\\
\textit{code for title page}\\
|\newpage|\\
|\||fi|
\end{tabular}
\end{center}
%
A banner page for the child documents can be generated by:
%
\begin{center}
\begin{tabular}{l}
|\ifchilddoc|\\
|\addtocounter{page}{-1}|\\
\textit{code for banner page}\\
|\newpage|\\
|\||fi|
\end{tabular}
\end{center}
%
Here one could write a message such as:
\begin{center}
|This is the part \childdocname{} of \childdocjob{}.|
\end{center}

%%%%%%%%%%%%%%%%%%%%%%%%%%%%%%%%%%%%%%%%%%%%%%%%%%%%%%%%%%%%%%%%%%%%%%%%%%%%%%%%
\subsection{Flags}
\label{sec:flags}

The package makes it easy to generate different versions
of the main or child documents.
To this end compilation flags can be defined
and assigned different default values.
They will be particularly useful in conjunction
with the forwarding mechanism described in \secref{sec:forward}.

For example, it may be useful to have a flag |\version|
which can be set to |draft| or |final|.
The document source will contain some conditional code
depending on the value of |\version|.
Suppose further, the flag should default to |final| for the main file
and to |draft| for child files
which is a natural assignment for editing the document.
This is achieved by placing the following code
in the preamble of the main document
(below the |\childdocmain| directive):
%
\begin{center}
\begin{tabular}{l}
|\ifchilddoc|\\
|\providecommand{\version}{draft}|\\
|\||else|\\
|\providecommand{\version}{final}|\\
|\||fi|
\end{tabular}
\end{center}
%
The definition by |\providecommand| makes sure
that previous definitions are not overwritten.
Further statements |\providecommand{\version}{...}|
can thus be added before the above code to override it.

For the main file, one might add a line
(between |\childdocmain| and the above block)
%
\begin{center}
|%\ifchilddoc\||else\providecommand{\version}{draft}\||fi|
\end{center}
%
which can be uncommented to produce a draft version.
Likewise one can add a line to the very top of a child file
(above the |\childdocof{|\textit{main}|}| directive)
%
\begin{center}
|%\providecommand{\version}{final}|
\end{center}
%
which can be uncommented to produce the final version of this child document.

%%%%%%%%%%%%%%%%%%%%%%%%%%%%%%%%%%%%%%%%%%%%%%%%%%%%%%%%%%%%%%%%%%%%%%%%%%%%%%%%
\subsection{Forwarding}
\label{sec:forward}

Different versions of the main or child documents
using compilation flags as described in \secref{sec:flags}
can be (permanently) stored in different files
for convenient compilation, viewing and distribution.
To this end, the package defines a command
to pass on compilation to a different file:

%%%%%%%%%%%%%%%%%%%%%%%%%%%%%%%%%%%%%%%%
\DescribeMacro{\childdocforward}
The command |\childdocforward| redirects processing to
another source file:
%
\begin{center}
\begin{tabular}{l}
|\input{childdoc.def}|\\
|\childdocforward[|\textit{main}|]{|\textit{dest}|}|\\
\end{tabular}
\end{center}
%
The argument \textit{dest} is the destination file
(without extension).
It should be the main file or one of the child files.
Note that further \textsf{childdoc} directives
such as |\childdocof| and |\childdocforward|
in the indicated file will be processed in this form.
The optional argument \textit{main}
passes on directly to the main file \textit{main}
while pretending to compile the child \textit{dest}.
This form behaves as if \textit{dest}
issues |\childdocof{|\textit{main}|}| right away,
and no further \textsf{childdoc} directives will be processed.

%%%%%%%%%%%%%%%%%%%%%%%%%%%%%%%%%%%%%%%%
\DescribeMacro{\...prefix}
In the alternative form |\childdocforwardprefix|,
%
\begin{center}
\begin{tabular}{l}
|\input{childdoc.def}|\\
|\childdocforwardprefix[|\textit{main}|]{|\textit{prefix}|}{|\textit{dest}|}|
\end{tabular}
\end{center}
%
the destination file is determined by a pattern
depending on the current file:
To make this work, the current file must be called
`{\textit{prefix}\hspace{0.2em}\textit{suffix}}'
with \textit{prefix} matching precisely the argument.
Processing is then passed on to the file
`{\textit{dest}\hspace{0.2em}\textit{suffix}}'.
Surely, the same effect is achieved by
directly specifying the
argument `{\textit{dest}\hspace{0.2em}\textit{suffix}}'
in the first form.
However, that requires to set up a different file
for each child. With the alternative form of the command
all these files can have exactly the same content
which simplifies setting them up and maintaining them.

For example, the following file |draft.tex|
with a compilation flag |\version| as described in \secref{sec:flags}
compiles the main document as a draft:
%
\begin{center}
\begin{tabular}{l}
|\def\version{draft}|\\
|\input{childdoc.def}|\\
|\childdocforward{|\textit{main}|}|
\end{tabular}
\end{center}
%
Likewise, the following files |final|\textit{nn}|.tex|
compile the final version of the child document
|child|\textit{nn}|.tex|:
%
\begin{center}
\begin{tabular}{l}
|\def\version{final}|\\
|\input{childdoc.def}|\\
|\childdocforwardprefix{final}{child}|
\end{tabular}
\end{center}
%

Note that when several versions of a main file and/or of each child file
are to be generated, it may be convenient to set up a |Makefile| or
shell script to automatise the process.

%%%%%%%%%%%%%%%%%%%%%%%%%%%%%%%%%%%%%%%%%%%%%%%%%%%%%%%%%%%%%%%%%%%%%%%%%%%%%%%%
\subsection{Command Line Processing}
\label{sec:commandline}

The effect of redirection files can also be achieved by invoking
the \LaTeX{} compiler with a more elaborate command line.
Most conveniently this should be done as part
of a shell script or a |Makefile|.

When using \textsf{childdoc} in the main file, the following
command lines effectively perform a redirection
(note that depending on the shell being used,
backslashes may have to be doubled: `|\|' $\to$ `|\\|'):
%
\begin{center}
|... -jobname "|\textit{target}|" |\\|"|[\textit{flags}]%
|\input{childdoc.def}\childdocforward[|\textit{main}|]{|\textit{dest}|}"|
\end{center}
%
Here \textit{target} is the name of the output file,
\textit{main} is the name of the main file
and \textit{dest} is the name of the main or child file to be processed
(all filenames without extensions).
The optional argument \textit{main} can be omitted
if \textit{main} matches \textit{dest}.
Optionally, compilation \textit{flags} can be defined via |\def| commands.
This command line makes the \TeX{} engine believe
it is compiling the file \textit{target}
whose content is specified as the latter parameter.
The provided code then forwards the processing to
\textit{main} or \textit{dest} as described in \secref{sec:forward}.

%%%%%%%%%%%%%%%%%%%%%%%%%%%%%%%%%%%%%%%%%%%%%%%%%%%%%%%%%%%%%%%%%%%%%%%%%%%%%%%%
\subsection{Include by Input}
\label{sec:input}

Including child documents by |\include| has some restrictions by design.
Most notably, the content of a child document always occupies
its own set of pages; pages cannot be shared between child documents.
Usually, this behaviour makes perfect sense
because each child document contain an essential part of the document.
However, in some situations it may be desirable to compose
a document from a collection of parts
without having mandatory page breaks between then.
For this case, the package
provides a mechanism to include parts
by |\input| which can also be processed individually.
However, by construction this mechanism
requires manual handling of the content to be output.

%%%%%%%%%%%%%%%%%%%%%%%%%%%%%%%%%%%%%%%%
\DescribeMacro{\ifchilddocmanual}
The main file should be prepared as usual, see \secref{sec:include}.
However, the document body must make a distinction
between processing of an individual part and of the main document, e.g.:
%
\begin{center}
\begin{tabular}{l}
|\ifchilddocmanual|\\
|\input{\childdocname}|\\
|\||else|\\
\textit{document body with }|\input{|\textit{part}|}|\\
|\||fi|
\end{tabular}
\end{center}
%
The conditional |\ifchilddocmanual| is true whenever
a part to be included by |\input| is being compiled,
and the name of the part is stored in |\childdocname|.

%%%%%%%%%%%%%%%%%%%%%%%%%%%%%%%%%%%%%%%%
\DescribeMacro{\childdocby}
Each part to be included by |\input| should start with:
%
\begin{center}
\begin{tabular}{l}
|\input{childdoc.def}|\\
|\childdocby{|\textit{main}|}|\\
\end{tabular}
\end{center}
%
The directive |\childdocby| is similar to |\childdocof|
described in \secref{sec:include},
but the subsequent selection of content must be done manually.
To that end, both |\ifchilddoc| and |\ifchilddocmanual|
will be true upon processing of a part,
and the name of the part is stored in |\childdocname|.
Note that |\jobname| will be set to the filename of the current part
so that each part receives an individual |.aux| file
that does not interfere with the |.aux| file(s) of the main document.
This behaviour can be altered by the alternative form
|\childdocby[*]{|\textit{main}|}| (with a non-empty optional argument)
which uses the |.aux| file of the main document
by setting |\jobname| to \textit{main}.

%%%%%%%%%%%%%%%%%%%%%%%%%%%%%%%%%%%%%%%%%%%%%%%%%%%%%%%%%%%%%%%%%%%%%%%%%%%%%%%%
\subsection{Driver Development}
\label{sec:driver}

The \textsf{childdoc} mechanism can also be use for the development
of definition files such as \LaTeX{} styles or classes.
This case differs from the above setup with multiple parts
included by |\include| in that no |\includeonly| should be invoked.
This can be achieved by starting the include file
(before |\ProvidesPackage|) with:
%
\begin{center}
\begin{tabular}{l}
|\input{childdoc.def}|\\
|\childdocforward{|\textit{main}|}|\\
\end{tabular}
\end{center}
%
or alternatively with:
%
\begin{center}
\begin{tabular}{l}
|\input{childdoc.def}|\\
|\childdocby{|\textit{main}|}|\\
\end{tabular}
\end{center}
%
Both forms have slightly different effects as described above.
The main file is prepared as usual, see \secref{sec:include}.

%%%%%%%%%%%%%%%%%%%%%%%%%%%%%%%%%%%%%%%%%%%%%%%%%%%%%%%%%%%%%%%%%%%%%%%%%%%%%%%%
\subsection{Legacy Detection}
\label{sec:detection}

The directive |\childdocmain| in the main file can detect
whether the complete document or merely a child is to be compiled
even without using the directive |\childdocof|.
This method is deprecated because it is less robust
and there is no compelling reason to use it;
it is merely provided for backward compatibility
and it may be removed in future versions.

If the detection mechanism is to be used,
it is mandatory to correctly specify
the filename of the main file as the argument of |\childdocmain|:
%
\begin{center}
\begin{tabular}{l}
|\input{childdoc.def}|\\
|\childdocmain{|\textit{main}|}|\\
\end{tabular}
\end{center}
%
If |\jobname| does not match the argument \textit{main} of |\childdocmain|,
it is assumed that |\jobname| points to the child file to be compiled.
When using |\childdocmain| with the main file specified as argument,
it suffices to start a child file
with just |\input{|\textit{main}|}|
without loading of the package and using |\childdocof|.
If instead all processing is done
with the appropriate \textsf{childdoc} directives,
the argument of \textit{main} of |\childdocmain| can be empty.

An alternative version of the command line processing described
in \secref{sec:commandline} using the detection mechanism reads:
%
\begin{center}
|... -jobname "|\textit{target}|" "|[\textit{flags}]%
[|\def\jobname{|\textit{dest}|}|]|\input{|\textit{main}|}"|
\end{center}

%%%%%%%%%%%%%%%%%%%%%%%%%%%%%%%%%%%%%%%%%%%%%%%%%%%%%%%%%%%%%%%%%%%%%%%%%%%%%%%%
\subsection{Manual Code}
\label{sec:manual}

In case one cannot be certain whether the definitions file |childdoc.def|
is installed on the target \TeX{} distribution
and one prefers not to ship it,
it is conceivable to paste a few relevant commands into the sources.

To that end, drop all statements |\input{childdoc.def}|
and perform the replacements as outlined below.
Instead of |\childdocmain{|\textit{main}|}| add the following code
to the top of the main file:
%
\begin{center}
\begin{tabular}{l}
|\||ifdefined\childdocname\endinput\||fi\newif\ifchilddoc|\\
|\edef\childdocname{\scantokens\expandafter{\jobname\noexpand}}|\\
|\def\childdocmain{|\textit{main}|}\||ifx\childdocmain\childdocname\||else|\\
|\childdoctrue\includeonly{\childdocname}\let\jobname\childdocmain\||fi|\\
\end{tabular}
\end{center}
%
Instead of |\childdocof{|\textit{main}|}| just include the main file
at the top of each child file:
%
\begin{center}
|\input{|\textit{main}|}|
\end{center}
%
A simple redirection |\childdocforward{|\textit{dest}|}| is achieved by:
%
\begin{center}
|\def\jobname{|\textit{dest}|}\input{\jobname}|
\end{center}
%
The redirection with prefix
|\childdocforwardprefix[|\textit{prefix}|]{|\textit{dest}|}|
is accomplished by:
%
\begin{center}
\begin{tabular}{l}
|{\edef\jobname{\scantokens\expandafter{\jobname\noexpand}}|\\
|\def\redirectjob |\textit{prefix}|#1~~~{\gdef\jobname{|\textit{dest}|#1}}|\\
|\expandafter\redirectjob\jobname~~~}\input{\jobname}|
\end{tabular}
\end{center}

In an alternative approach,
child documents can be compiled by a specific command line
without additional code or specific definitions:
%
\begin{center}
|... -jobname "|\textit{target}|" "|[\textit{flags}]%
|\includeonly{|\textit{dest}|}\input{|\textit{main}|}"|
\end{center}
%

%%%%%%%%%%%%%%%%%%%%%%%%%%%%%%%%%%%%%%%%%%%%%%%%%%%%%%%%%%%%%%%%%%%%%%%%%%%%%%%%
%%%%%%%%%%%%%%%%%%%%%%%%%%%%%%%%%%%%%%%%%%%%%%%%%%%%%%%%%%%%%%%%%%%%%%%%%%%%%%%%
\section{Information}

%%%%%%%%%%%%%%%%%%%%%%%%%%%%%%%%%%%%%%%%%%%%%%%%%%%%%%%%%%%%%%%%%%%%%%%%%%%%%%%%
\subsection{Copyright}

Copyright \copyright{} 2017--2018 Niklas Beisert

This work may be distributed and/or modified under the
conditions of the \LaTeX{} Project Public License, either version 1.3
of this license or (at your option) any later version.
The latest version of this license is in
  \url{http://www.latex-project.org/lppl.txt}
and version 1.3 or later is part of all distributions of \LaTeX{}
version 2005/12/01 or later.

This work has the LPPL maintenance status `maintained'.

The Current Maintainer of this work is Niklas Beisert.

This work consists of the files |README.txt|, |childdoc.ins| and |childdoc.dtx|
as well as the derived files |childdoc.def|, |cdocsamp.tex|
with |cdocsch1.tex|, |cdocsch2.tex|, |cdocspt3.tex|, |cdocspt4.tex|,
|cdocsdrf.tex|, |cdocsfn1.tex|, |cdocsfn2.tex|
as well as |childdoc.pdf|.

%%%%%%%%%%%%%%%%%%%%%%%%%%%%%%%%%%%%%%%%%%%%%%%%%%%%%%%%%%%%%%%%%%%%%%%%%%%%%%%%
\subsection{Files and Installation}

The package consists of the files:
%
\begin{center}
\begin{tabular}{ll}
    |README.txt|   & readme file \\
    |childdoc.ins| & installation file \\
    |childdoc.dtx| & source file \\
    |childdoc.def| & definition file \\
    |cdocsamp.tex| & sample main file \\
    |cdocsch1.tex| & sample include file \\
    |cdocsch2.tex| & sample include file \\
    |cdocspt3.tex| & sample part file \\
    |cdocspt4.tex| & sample part file \\
    |cdocsdrf.tex| & sample redirection file \\
    |cdocsfn1.tex| & sample redirection file \\
    |cdocsfn2.tex| & sample redirection file \\
    |childdoc.pdf| & manual
\end{tabular}
\end{center}
%
The distribution consists of the files
|README.txt|, |childdoc.ins| and |childdoc.dtx|.
%
\begin{itemize}
\item
Run (pdf)\LaTeX{} on |childdoc.dtx|
to compile the manual |childdoc.pdf| (this file).
\item
Run \LaTeX{} on |childdoc.ins| to create the definitions file |childdoc.def|
and the sample |cdocsamp.tex| with include files
|cdocsch1.tex|, |cdocsch2.tex|, |cdocspt3.tex|, |cdocspt4.tex|,
|cdocsdrf.tex|, |cdocsfn1.tex|, |cdocsfn2.tex|.
Then copy the file |childdoc.def| to an appropriate directory of your \LaTeX{}
distribution, e.g.\ \textit{texmf-root}|/tex/latex/childdoc|.
\end{itemize}

%%%%%%%%%%%%%%%%%%%%%%%%%%%%%%%%%%%%%%%%%%%%%%%%%%%%%%%%%%%%%%%%%%%%%%%%%%%%%%%%
\subsection{Related CTAN Packages}

There are several other packages which offer a similar functionality:
%
\begin{itemize}
\item
The packages
\href{http://ctan.org/pkg/docmute}{\textsf{docmute}},
\href{http://ctan.org/pkg/includex}{\textsf{includex}} and
\href{http://ctan.org/pkg/standalone}{\textsf{standalone}}
provide commands to include only the document body of
a child file thus allowing both files to be compiled individually.
\item
The packages \href{http://ctan.org/pkg/subdocs}{\textsf{subdocs}}
and \href{http://ctan.org/pkg/subfiles}{\textsf{subfiles}}
provide structures in which the main and child documents can be
encapsulated and allowing them to be compiled individually.
The inclusion mechanism is different from the conventional |\include|.
\item
The package \href{http://ctan.org/pkg/combine}{\textsf{combine}}
is an elaborate solution to combine several documents into one.
\end{itemize}
%
See also the CTAN topic \href{http://ctan.org/topic/subdocs}{\textsf{subdocs}}
for further related packages.
The present package differs from the above solutions in that
a document structure constructed with the conventional |\include| mechanism
just needs two extra commands at the top of every file
such that all constituent files can be compiled individually.

%%%%%%%%%%%%%%%%%%%%%%%%%%%%%%%%%%%%%%%%%%%%%%%%%%%%%%%%%%%%%%%%%%%%%%%%%%%%%%%%
%\subsection{Feature Suggestions}
%
%The following is a list of features which may be useful for future
%versions of this package:
%%
%\begin{itemize}
%\item
%\ldots
%\end{itemize}

%%%%%%%%%%%%%%%%%%%%%%%%%%%%%%%%%%%%%%%%%%%%%%%%%%%%%%%%%%%%%%%%%%%%%%%%%%%%%%%%
\subsection{Revision History}

%%%%%%%%%%%%%%%%%%%%%%%%%%%%%%%%%%%%%%%%
\paragraph{v2.0:} 2018/12/30

\begin{itemize}
\item
immediate forward processing
\item
added |\childdocby| mechanism
\item
manual restructured
\end{itemize}

%%%%%%%%%%%%%%%%%%%%%%%%%%%%%%%%%%%%%%%%
\paragraph{v1.6:} 2018/01/17

\begin{itemize}
\item
application for development of include files
\item
corrections to manual
\end{itemize}

%%%%%%%%%%%%%%%%%%%%%%%%%%%%%%%%%%%%%%%%
\paragraph{v1.5:} 2017/05/21

\begin{itemize}
\item
more complete structuring introduced
\item
|\childdocof| introduced
\item
|\childdoc| renamed to |\childdocmain|
\item
|\childredirect| renamed to |\childdocforward| and |\childdocforwardprefix|
and functionality expanded
\end{itemize}

%%%%%%%%%%%%%%%%%%%%%%%%%%%%%%%%%%%%%%%%
\paragraph{v1.0:} 2017/04/27

\begin{itemize}
\item
manual and install package
\item
first version published on CTAN
\end{itemize}

%%%%%%%%%%%%%%%%%%%%%%%%%%%%%%%%%%%%%%%%
\paragraph{v0.6:} 2017/04/26

\begin{itemize}
\item
redirection mechanism added
\end{itemize}

%%%%%%%%%%%%%%%%%%%%%%%%%%%%%%%%%%%%%%%%
\paragraph{v0.5:} 2017/04/26

\begin{itemize}
\item
functionality in definition file
\end{itemize}


%%%%%%%%%%%%%%%%%%%%%%%%%%%%%%%%%%%%%%%%%%%%%%%%%%%%%%%%%%%%%%%%%%%%%%%%%%%%%%%%
%%%%%%%%%%%%%%%%%%%%%%%%%%%%%%%%%%%%%%%%%%%%%%%%%%%%%%%%%%%%%%%%%%%%%%%%%%%%%%%%
%%%%%%%%%%%%%%%%%%%%%%%%%%%%%%%%%%%%%%%%%%%%%%%%%%%%%%%%%%%%%%%%%%%%%%%%%%%%%%%%
\appendix

\settowidth\MacroIndent{\rmfamily\scriptsize 000\ }

 \DocInput{childdoc.dtx}

\end{document}
%</driver>
% \fi
%
% %%%%%%%%%%%%%%%%%%%%%%%%%%%%%%%%%%%%%%%%%%%%%%%%%%%%%%%%%%%%%%%%%%%%%%%%%%%%%%
% %%%%%%%%%%%%%%%%%%%%%%%%%%%%%%%%%%%%%%%%%%%%%%%%%%%%%%%%%%%%%%%%%%%%%%%%%%%%%%
% \section{Sample}
%\iffalse
%<*samplemain>
%\fi
%
% The following presents a sample document
% with two chapters, two parts, a title page,
% a compile flag as well as three forwarding files to set the flag.
% It consists of eight |.tex| files:
% \begin{center}
% \begin{tabular}{ll}
% |cdocsamp.tex|&main file\\
% |cdocsch1.tex|&include file for chapter 1\\
% |cdocsch2.tex|&include file for chapter 2\\
% |cdocspt3.tex|&include file for part 3\\
% |cdocspt4.tex|&include file for part 4\\
% |cdocsdrf.tex|&forwarding file for main file in draft mode\\
% |cdocsfi1.tex|&forwarding file for final version of chapter 1\\
% |cdocsfi2.tex|&forwarding file for final version of chapter 2\\
% \end{tabular}
% \end{center}
% Each of the eight files can be compiled directly by the \LaTeX{} compiler.
%
% %%%%%%%%%%%%%%%%%%%%%%%%%%%%%%%%%%%%%%
% \paragraph{Main File.}
%
% The main file is called |cdocsamp.tex|.
%
% Load the \textsf{childdoc} definitions and
% declare the filename for the main document:
%    \begin{macrocode}
\input{childdoc.def}
\childdocmain{}
%    \end{macrocode}

% Optional override for |\version| flag:
%    \begin{macrocode}
%%\ifchilddoc\else\providecommand{\version}{draft}\fi
%    \end{macrocode}

% Define the default values for the |\version| flag
% (|final| for the main file and |draft| for childs):
%    \begin{macrocode}
\ifchilddoc
\providecommand{\version}{draft}
\else
\providecommand{\version}{final}
\fi
%    \end{macrocode}

% Load the standard document class:
%    \begin{macrocode}
\documentclass[12pt]{article}
%    \end{macrocode}

% Start the document body:
%    \begin{macrocode}
\begin{document}
%    \end{macrocode}

% Declare a title page.
% Print title, part of document being processed and version flag:
%    \begin{macrocode}
\addtocounter{page}{-1}
\begin{center}
{\LARGE\bfseries{}childdoc example\par}
\vspace{1cm}
\ifchilddoc
\ifchilddocmanual part\else chapter\fi:
`\childdocname' of `\childdocjob'\par
\else
main document: `\childdocjob'\par
\fi
version: \version\par
\end{center}
\newpage
%    \end{macrocode}

% Manually include selected file,
% otherwise process as usual:
%    \begin{macrocode}
\ifchilddocmanual
\section*{part `\childdocname'}
\input{\childdocname}
\else
%    \end{macrocode}

% Include the two chapters:
%    \begin{macrocode}
\include{cdocsch1}
\include{cdocsch2}
%    \end{macrocode}

% Include the two parts unless only chapters should be displayed:
%    \begin{macrocode}
\ifchilddoc\else
\section{part three}
\input{cdocspt3}
\section{part four}
\input{cdocspt4}
\fi
%    \end{macrocode}

% Process as usual until here:
%    \begin{macrocode}
\fi
%    \end{macrocode}

% End of document body:
%    \begin{macrocode}
\end{document}
%    \end{macrocode}
%\iffalse
%</samplemain>
%\fi
%
% %%%%%%%%%%%%%%%%%%%%%%%%%%%%%%%%%%%%%%
% \paragraph{Chapter Include Files.}
%
% The include files are called |cdocsch1.tex| and |cdocsch2.tex|.
%
%\iffalse
%<*samplechap1|samplechap2>
%\fi

% Optional override for |\version| flag:
%    \begin{macrocode}
%%\providecommand{\version}{final}
%    \end{macrocode}

% Include the main document:
%    \begin{macrocode}
\input{childdoc.def}
\childdocof{cdocsamp}
%    \end{macrocode}

%\iffalse
%</samplechap1|samplechap2>
%\fi
%
%\iffalse
%<*samplechap1>
%\fi
% Some text for chapter 1:
%    \begin{macrocode}
\section{one}
some text in chapter one
%    \end{macrocode}

%\iffalse
%</samplechap1>
%\fi
% Some text for chapter 2:
%\iffalse
%<*samplechap2>
%\fi
%    \begin{macrocode}
\section{two}
more text in chapter two
%    \end{macrocode}

%\iffalse
%</samplechap2>
%\fi
%
% %%%%%%%%%%%%%%%%%%%%%%%%%%%%%%%%%%%%%%
% \paragraph{Part Include Files.}
%
% The include files are called |cdocspt3.tex| and |cdocspt4.tex|.
%
%\iffalse
%<*samplepart3|samplepart4>
%\fi

% Optional override for |\version| flag:
%    \begin{macrocode}
%%\providecommand{\version}{final}
%    \end{macrocode}

% Include the main document:
%    \begin{macrocode}
\input{childdoc.def}
\childdocby{cdocsamp}
%    \end{macrocode}

%\iffalse
%</samplepart3|samplepart4>
%\fi
%
%\iffalse
%<*samplepart3>
%\fi
% Some text for part 3:
%    \begin{macrocode}
some text in part three
%    \end{macrocode}

%\iffalse
%</samplepart3>
%\fi
% Some text for part 4:
%\iffalse
%<*samplepart4>
%\fi
%    \begin{macrocode}
more text in part four
%    \end{macrocode}

%\iffalse
%</samplepart4>
%\fi
%
% %%%%%%%%%%%%%%%%%%%%%%%%%%%%%%%%%%%%%%
% \paragraph{Forwarding for a Complete Draft.}
%
% The following forwarding file |cdocsdrf.tex|
% compiles the main document in draft mode:
%\iffalse
%<*sampledraft>
%\fi
%    \begin{macrocode}
\def\version{draft}
\input{childdoc.def}
\childdocforward{cdocsamp}
%    \end{macrocode}

%\iffalse
%</sampledraft>
%\fi
%
% %%%%%%%%%%%%%%%%%%%%%%%%%%%%%%%%%%%%%%
% \paragraph{Forwarding for Final Version of the Chapters.}
%
% The following forwarding files |cdocsfn1.tex| and |cdocsfn2.tex|
% (with identical content)
% compile the final versions of the child documents
% |cdocsch1.tex| and |cdocsch2.tex|, respectively:
%\iffalse
%<*samplefinal>
%\fi
%    \begin{macrocode}
\def\version{final}
\input{childdoc.def}
\childdocforwardprefix[cdocsamp]{cdocsfn}{cdocsch}
%    \end{macrocode}

%\iffalse
%</samplefinal>
%\fi
%
% %%%%%%%%%%%%%%%%%%%%%%%%%%%%%%%%%%%%%%
% \paragraph{Command Line Processing.}
%
% The following three command lines generate the output files
% |cdocscld|, |cdocscl1| and |cdocscl2|
% which should be identical to
% |cdocsdrf|, |cdocsch1| and |cdocsfn2|, respectively:
% \begin{center}
% \begin{tabular}{l}
% |latex -jobname cdocscld \|\\
% |  "\def\version{draft}\input{childdoc.def}\childdocforward{cdocsamp}"|\\
% |latex -jobname cdocscl1 \|\\
% |  "\input{childdoc.def}\childdocforward[cdocsamp]{cdocsch1}"|\\
% |latex -jobname cdocscl2 \|\\
% |  "\def\version{final}\input{childdoc.def}\childdocforward{cdocsch2}"|
% \end{tabular}
% \end{center}
% Note that the trailing backslash on each first line
% merely continues the input to the second line
% (for convenient cut ant paste).
% Furthermore, the command |latex| can be replaced by any
% of its alternative versions such as |pdflatex|.
%
% %%%%%%%%%%%%%%%%%%%%%%%%%%%%%%%%%%%%%%%%%%%%%%%%%%%%%%%%%%%%%%%%%%%%%%%%%%%%%%
% %%%%%%%%%%%%%%%%%%%%%%%%%%%%%%%%%%%%%%%%%%%%%%%%%%%%%%%%%%%%%%%%%%%%%%%%%%%%%%
% \section{Implementation}
%\iffalse
%<*package>
%\fi
%
% This section describes the definitions file |childdoc.def|.

% The definitions cannot be loaded using |\usepackage| or |\RequirePackage|
% which has a mechanism to prevent loading a style file more than once.
% When loading the definitions by means of |\input|
% multiple instances have to be prevented manually:
%\iffalse
%This code needs to be before the `\ProvidesFile' directive
%which is defined at the beginning of this file.
%Therefore it is also placed there and commented out here.
%</package>
%<*discard>
%\fi
%    \begin{macrocode}
\ifdefined\childdocmain\endinput\fi
%    \end{macrocode}
%\iffalse
%</discard>
%<*package>
%\fi
%
% \macro{\ifchilddoc}
% \macro{\ifchilddocmanual}
% The conditional |\ifchilddoc| tells whether a
% child (true) or main (false) document is being compiled.
% The conditional |\ifchilddocmanual| tells whether
% the |\includeonly| mechanism is used (false) or
% the selection of child files must be performed manually (true).
% The definitions initialise to false:
%    \begin{macrocode}
\newif\ifchilddoc
\newif\ifchilddocmanual
%    \end{macrocode}

% \macro{\childdocname}
% \macro{\childdocjob}
% The macro |\childdocname| stores the name of the main document
% to be compiled. The macro |\childdocjob| stores the name of
% the document on which the \LaTeX{} compiler was originally invoked.
% The content of |\jobname| cannot be compared
% to filenames specified in the source due to different catcodes.
% The following code rescans |\jobname|, stores the result
% in |\childdocname| and saves a copy in |\childdocjob|:
%    \begin{macrocode}
\edef\childdocname{\scantokens\expandafter{\jobname\noexpand}}
\let\childdocjob\childdocname
%    \end{macrocode}

% \macro{\childdocdisable}
% The macro |\childdocdisable| prevents the main file
% from being processed more than once.
% At this stage, the main document command |\childdocmain|
% is assumed to be called once again where it should do nothing.
% Any subsequent call to it should prevent
% a secondary processing of the main document
% It overwrites the forwarding commands
% |\childdocof| and |\childdocforward|
% with empty macros to prevent further inclusions of the main document:
%    \begin{macrocode}
\newcommand{\childdocdisable}
{
  \renewcommand{\childdocmain}[1]{\renewcommand{\childdocmain}[1]{\endinput}}
  \renewcommand{\childdocof}[1]{}
  \renewcommand{\childdocby}[2][]{}
  \renewcommand{\childdocforward}[2][]{}
  \renewcommand{\childdocdisable}{}
}
%    \end{macrocode}

% \macro{\childdocmain}
% The macro |\childdocmain| is to be called at the top of the main file
% with nothing or the main filename (without extension) as argument.
% First, it breaks loops.
% If the argument is not empty and does not match |\childdocname|
% (which is set by the first inclusion of |childdoc.def|),
% |\ifchilddoc| is set to true, |\includeonly| is applied to the child file
% and |\jobname| is set to the main file
% (for proper handling of |.aux| files):
%    \begin{macrocode}
\newcommand{\childdocmain}[1]
{
  \childdocdisable\childdocmain{}
  \if?#1?\else
    \begingroup
      \def\childdoctmp{#1}
      \ifx\childdoctmp\childdocname
        \def\childdoctmp{}
      \else
        \def\childdoctmp
        {
          \childdoctrue
          \includeonly{\childdocname}
          \def\childdocjob{#1}
          \def\jobname{#1}
        }
      \fi
      \expandafter
    \endgroup
    \childdoctmp
  \fi
}
%    \end{macrocode}

% \macro{\childdocof}
% The command |\childdocof| redirects
% compilation to the main file |#1|.
%    \begin{macrocode}
\newcommand{\childdocof}[1]
{
  \childdocdisable
  \childdoctrue
  \includeonly{\childdocname}
  \def\jobname{#1}
  \def\childdocjob{#1}
  \input{#1}
}
%    \end{macrocode}

% \macro{\childdocby}
% The command |\childdocby| ....
%    \begin{macrocode}
\newcommand{\childdocby}[2][]
{
  \childdocdisable
  \childdoctrue
  \childdocmanualtrue
  \if?#1?\else
    \def\jobname{#2}
  \fi
  \def\childdocjob{#2}
  \input{#2}
  \endinput
}
%    \end{macrocode}

% \macro{\childdocforward}
% The command |\childdocforward| redirects
% compilation to the main file or
% (if the optional argument is given) a child file.
% Parameters are set as if the main file
% or a child file starting with |\childdocof| was compiled.
% Then compilation is handed over to the main file:
%    \begin{macrocode}
\newcommand{\childdocforward}[2][]
{
  \begingroup
    \if?#1?
      \def\childdoctmp
      {
        \def\childdocname{#2}
        \def\childdocjob{#2}
        \def\jobname{#2}
        \input{#2}
        \endinput
      }
    \else
      \def\childdoctmp
      {
        \childdocdisable
        \def\childdocname{#2}
        \childdoctrue
        \includeonly{#2}
        \def\childdocjob{#1}
        \def\jobname{#1}
        \input{#1}
        \endinput
      }
    \fi
    \expandafter
  \endgroup
  \childdoctmp
}
%    \end{macrocode}

% \macro{\childdocforwardprefix}
% The command |\childdocforwardprefix| redirects
% compilation to the main or a child file by means of a pattern.
% The prefix |#1| in the current filename is replaced by |#2|
% and the suffix of the current filename is kept
% (it is assumed that the filename does not contain the substring `|~~~|'
% which is used as a delimiter).
% Compilation is handed over to the new file by |\childdocforward|:
%    \begin{macrocode}
\newcommand{\childdocforwardprefix}[3][]
{
  \begingroup
    \def\childdocextract #2##1~~~{\def\childdoctmp{\childdocforward[#1]{#3##1}}}
    \expandafter\childdocextract\childdocname~~~
    \expandafter
  \endgroup
  \childdoctmp
}
%    \end{macrocode}

% \macro{\childdoc}
% The deprecated macro |\childdoc| is a legacy version of |\childdocmain|:
%    \begin{macrocode}
\newcommand{\childdoc}{\childdocmain}
%    \end{macrocode}

% \macro{\childdocredirect}
% The deprecated macro |\childdocredirect| is a legacy version
% of |\childdocforward| and |\childdocforwardprefix|:
%    \begin{macrocode}
\newcommand{\childdocredirect}[2][]
{
  \begingroup
    \if?#1?
      \def\childdoctmp{\childdocforward{#2}}
    \else
      \def\childdoctmp{\childdocforwardprefix{#1}{#2}}
    \fi
    \expandafter
  \endgroup
  \childdoctmp
}
%    \end{macrocode}

%\iffalse
%</package>
%\fi
%
\endinput
|\\
|\childdocby{|\textit{main}|}|\\
\end{tabular}
\end{center}
%
Both forms have slightly different effects as described above.
The main file is prepared as usual, see \secref{sec:include}.

%%%%%%%%%%%%%%%%%%%%%%%%%%%%%%%%%%%%%%%%%%%%%%%%%%%%%%%%%%%%%%%%%%%%%%%%%%%%%%%%
\subsection{Legacy Detection}
\label{sec:detection}

The directive |\childdocmain| in the main file can detect
whether the complete document or merely a child is to be compiled
even without using the directive |\childdocof|.
This method is deprecated because it is less robust
and there is no compelling reason to use it;
it is merely provided for backward compatibility
and it may be removed in future versions.

If the detection mechanism is to be used,
it is mandatory to correctly specify
the filename of the main file as the argument of |\childdocmain|:
%
\begin{center}
\begin{tabular}{l}
|% \iffalse
%
% childdoc.dtx Copyright (C) 2017-2018 Niklas Beisert
%
% This work may be distributed and/or modified under the
% conditions of the LaTeX Project Public License, either version 1.3
% of this license or (at your option) any later version.
% The latest version of this license is in
%   http://www.latex-project.org/lppl.txt
% and version 1.3 or later is part of all distributions of LaTeX
% version 2005/12/01 or later.
%
% This work has the LPPL maintenance status `maintained'.
%
% The Current Maintainer of this work is Niklas Beisert.
%
% This work consists of the files childdoc.dtx and childdoc.ins
% and the derived files childdoc.def and cdocsamp.tex with
% cdocsch1.tex, cdocsch2.tex, cdocsdrf.tex, cdocsfn1.tex, cdocsfn2.tex.
%
%<package>\ifdefined\childdocmain\endinput\fi
%<package>\ProvidesFile{childdoc.def}[2018/12/30 v2.0 child document driver]
%<samplemain>\ProvidesFile{cdocsamp.tex}[2018/12/30 v2.0 sample for childdoc]
%<*driver>
%\ProvidesFile{childdoc.drv}[2018/12/30 v2.0 childdoc reference manual file]
\PassOptionsToClass{10pt,a4paper}{article}
\documentclass{ltxdoc}

\usepackage[margin=35mm]{geometry}
\usepackage{hyperref}
\usepackage{hyperxmp}
\usepackage[usenames]{color}

\hypersetup{colorlinks=true}
\hypersetup{pdfstartview=FitH}
\hypersetup{pdfpagemode=UseNone}
\hypersetup{pdfsource={}}
\hypersetup{pdflang={en-UK}}
\hypersetup{pdfcopyright={Copyright 2017-2018 Niklas Beisert.
  This work may be distributed and/or modified under the
  conditions of the LaTeX Project Public License, either version 1.3
  of this license or (at your option) any later version.}}
\hypersetup{pdflicenseurl={http://www.latex-project.org/lppl.txt}}
\hypersetup{pdfcontactaddress={ETH Zurich, ITP, HIT K,
  Wolfgang-Pauli-Strasse 27}}
\hypersetup{pdfcontactpostcode={8093}}
\hypersetup{pdfcontactcity={Zurich}}
\hypersetup{pdfcontactcountry={Switzerland}}
\hypersetup{pdfcontactemail={nbeisert@itp.phys.ethz.ch}}
\hypersetup{pdfcontacturl={http://people.phys.ethz.ch/\xmptilde nbeisert/}}

\newcommand{\secref}[1]{\hyperref[#1]{section \ref*{#1}}}

\parskip1ex
\parindent0pt
\let\olditemize\itemize
\def\itemize{\olditemize\parskip0pt}

\begin{document}

\title{The \textsf{childdoc} Package}
\hypersetup{pdftitle={The childdoc Package}}
\author{Niklas Beisert\\[2ex]
  Institut f\"ur Theoretische Physik\\
  Eidgen\"ossische Technische Hochschule Z\"urich\\
  Wolfgang-Pauli-Strasse 27, 8093 Z\"urich, Switzerland\\[1ex]
  \href{mailto:nbeisert@itp.phys.ethz.ch}
  {\texttt{nbeisert@itp.phys.ethz.ch}}}
\hypersetup{pdfauthor={Niklas Beisert}}
\hypersetup{pdfsubject={Manual for the LaTeX2e Package childdoc}}
\date{30 December 2018, \textsf{v2.0}}
\maketitle

\begin{abstract}\noindent
\textsf{childdoc} is a \LaTeXe{} package
that enables the direct compilation
of document sections included by |\include|
to individual files.
\end{abstract}

\begingroup
\parskip0ex
\tableofcontents
\endgroup

%%%%%%%%%%%%%%%%%%%%%%%%%%%%%%%%%%%%%%%%%%%%%%%%%%%%%%%%%%%%%%%%%%%%%%%%%%%%%%%%
%%%%%%%%%%%%%%%%%%%%%%%%%%%%%%%%%%%%%%%%%%%%%%%%%%%%%%%%%%%%%%%%%%%%%%%%%%%%%%%%
\section{Introduction}

\LaTeX{} provides a mechanism to structure a large document (such as a book)
into a main file and several child files (containing the chapters)
using the |\include| command.
This mechanism is beneficial for documents
which span hundreds of pages in order to
make the source file(s) more manageable.
Moreover, compilation can be restricted to
selected child files by means of the |\includeonly| command.
The latter feature can be used to reduce the compilation time while editing
(this was significantly more useful in the earlier days of \LaTeX{})
or to generate a smaller document which is easier to navigate.
Another application of |\includeonly| is to generate
documents consisting of selected parts of the complete document.

However, there are a few drawbacks of the plain |\include| mechanism:
\begin{itemize}
\item
The child files cannot be compiled on their own,
they can only be compiled via the main file.
A naive editing environment
(such as a text editor with an option
to have the current file processed by \LaTeX)
may require one to switch to the main file before compiling;
attempting to compile the child file produces errors.
\item
The main file must be modified (each time)
to adjust the |\includeonly| command
to the present needs. This easily leaves the main file in a messy state.
\item
The generated document will always carry the filename
of the main document. This is inconvenient if
several child files are to be compiled and
to be kept for distribution.
\end{itemize}

The present package provides a simple interface
to make child files individually compilable by \LaTeX{}.
Compiling a child file then has the same effect as compiling
the main file with an |\includeonly| command
to select the appropriate child.
Moreover the generated document will carry the name of the child
rather than the main file.
This resolves all three above issues.

This feature is meant to make the editing of books,
thesis documents and lecture notes somewhat more convenient.
However, the package can also be used efficiently for
composing a series of documents (such as exercise sheets)
which are typically distributed individually.
It then assists the author in generating the individual documents
(potentially in different versions)
as well as a document containing the collected series.
Another application is in developing style files
or other kinds of included material
where compilation of the style file could redirect
to a sample or test file.

%%%%%%%%%%%%%%%%%%%%%%%%%%%%%%%%%%%%%%%%%%%%%%%%%%%%%%%%%%%%%%%%%%%%%%%%%%%%%%%%
%%%%%%%%%%%%%%%%%%%%%%%%%%%%%%%%%%%%%%%%%%%%%%%%%%%%%%%%%%%%%%%%%%%%%%%%%%%%%%%%
\section{Usage}

First of all, the package \textsf{childdoc} is \emph{not} a standard
\LaTeXe{} |.sty| style file! Therefore it needs to be invoked in
a non-standard way.

%%%%%%%%%%%%%%%%%%%%%%%%%%%%%%%%%%%%%%%%%%%%%%%%%%%%%%%%%%%%%%%%%%%%%%%%%%%%%%%%
\subsection{Included Files}
\label{sec:include}

%%%%%%%%%%%%%%%%%%%%%%%%%%%%%%%%%%%%%%%%
\DescribeMacro{\childdocmain}
To use the package, add the commands
\begin{center}
\begin{tabular}{l}
|\input{childdoc.def}|\\
|\childdocmain{}|\\
\end{tabular}
\end{center}
at the very top of the main \LaTeX{} file,
in particular \emph{before} the |\documentclass| statement!
The argument of |\childdocmain| should be left empty
(but it must be present).

%%%%%%%%%%%%%%%%%%%%%%%%%%%%%%%%%%%%%%%%
\DescribeMacro{\childdocof}
Furthermore, add the commands
\begin{center}
\begin{tabular}{l}
|\input{childdoc.def}|\\
|\childdocof{|\textit{main}|}|\\
\end{tabular}
\end{center}
at the top of every child file \textit{child}
which is included by |\include{|\textit{child}|}|
from within the main file
(or at least for those files to be compiled individually).
The argument \textit{main} must be the filename of the main file.

There are a couple of
considerations in setting up the main and child documents:

%%%%%%%%%%%%%%%%%%%%%%%%%%%%%%%%%%%%%%%%
\paragraph{Restrictions.}

Please note the following restrictions:
\begin{itemize}
\item
|\childdocmain| must be called with one argument \textit{main}
to ensure compatibility with earlier version of the package.
It must either be empty (|\childdocmain{}|)
or precisely match the filename of the main file in which it is specified.
See \secref{sec:detection} for further information.
\item
The filename \textit{main} must be specified without the |.tex| extension.
\item
The filename \textit{main} is case sensitive
(even in case-insensitive file systems)
due to internal string comparison.
\item
The argument \textit{main} should be fully expanded, it cannot be a macro.
\item
Subdirectories and special characters should be avoided in filenames.
\item
The command |\childdocmain{|\textit{main}|}| must be followed by a whitespace.
It should not be followed immediately by another command
or by a comment mark `|%|'.
This is because the \TeX{} parser reads the token immediately following
the argument of |\childdocmain| and puts it
at the beginning of every child section;
however, a white\-space is ignored.
\end{itemize}

%%%%%%%%%%%%%%%%%%%%%%%%%%%%%%%%%%%%%%%%
\paragraph{Content of Main File.}

It is advisable to place all content in the child files included by |\include|.
Any output contained in the main file will appear in all child documents
unless suppressed manually;
it cannot be suppressed automatically by the |\includeonly| directive
and thus should normally be avoided.
A method to include some content in the main file
by means of conditional processing is described in \secref{sec:conditional}.

%%%%%%%%%%%%%%%%%%%%%%%%%%%%%%%%%%%%%%%%
\paragraph{Page Numbering.}

When only a part of the document is compiled,
the appropriate numbering of pages
(as well as other status parameters)
is determined from the |.aux| files.
The latter contain information from previous passes.
However this information needs to propagate through
all intermediate child documents.
Therefore the page numbering in child documents may well
be inconsistent until the complete document is compiled at least once.

A useful (if unconventional) way to always ensure a consistent
page numbering is to restart the numbering in each child document
and denote the pages by `\textit{child}|.|\textit{page}'
where \textit{child} represents the chapter/section number of the child file.
This can be achieved by the command
|\numberwithin{page}{|\textit{child}|}|
of the \textsf{amsmath} package
where \textit{child} can be |chapter| or |section|
depending on the chosen structuring.
Alternatively, one can modify the macro |\thepage| appropriately
and reset the counter |page| at the start of each child file.

%%%%%%%%%%%%%%%%%%%%%%%%%%%%%%%%%%%%%%%%%%%%%%%%%%%%%%%%%%%%%%%%%%%%%%%%%%%%%%%%
\subsection{Conditional Processing}
\label{sec:conditional}

The package provides a mechanism to compile different versions
of a document. To customise the versions further some conditional processing
can come in handy to distinguish which version is being compiled.
The package provides two macros to describe the compilation context:

%%%%%%%%%%%%%%%%%%%%%%%%%%%%%%%%%%%%%%%%
\DescribeMacro{\ifchilddoc}
The conditional |\ifchilddoc| distinguishes between the compilation of
child documents and the main document:
%
\begin{center}
|\ifchilddoc |\textit{child-code}| |[|\||else |\textit{main-code}]| \||fi|
\end{center}

%%%%%%%%%%%%%%%%%%%%%%%%%%%%%%%%%%%%%%%%
\DescribeMacro{\childdocname}
\DescribeMacro{\childdocjob}
The macro |\childdocname| contains the filename (without extension)
of the main or child file being processed.
Note that |\childdocjob| will always contain the name of the main file.

%%%%%%%%%%%%%%%%%%%%%%%%%%%%%%%%%%%%%%%%
\paragraph{Title Page.}

Conditional processing can be used to include a title or banner page
in the main document when proper precautions are taken.
Importantly, the code in the main file should ensure that the page counter
(as well as other status parameters which are stored in the |.aux| files)
takes the same value after the conditional processing.
Otherwise the page numbers may take divergent values
depending on which part is compiled.

For example, a title page could be declared by:
%
\begin{center}
\begin{tabular}{l}
|\ifchilddoc\||else|\\
|\addtocounter{page}{-1}|\\
\textit{code for title page}\\
|\newpage|\\
|\||fi|
\end{tabular}
\end{center}
%
A banner page for the child documents can be generated by:
%
\begin{center}
\begin{tabular}{l}
|\ifchilddoc|\\
|\addtocounter{page}{-1}|\\
\textit{code for banner page}\\
|\newpage|\\
|\||fi|
\end{tabular}
\end{center}
%
Here one could write a message such as:
\begin{center}
|This is the part \childdocname{} of \childdocjob{}.|
\end{center}

%%%%%%%%%%%%%%%%%%%%%%%%%%%%%%%%%%%%%%%%%%%%%%%%%%%%%%%%%%%%%%%%%%%%%%%%%%%%%%%%
\subsection{Flags}
\label{sec:flags}

The package makes it easy to generate different versions
of the main or child documents.
To this end compilation flags can be defined
and assigned different default values.
They will be particularly useful in conjunction
with the forwarding mechanism described in \secref{sec:forward}.

For example, it may be useful to have a flag |\version|
which can be set to |draft| or |final|.
The document source will contain some conditional code
depending on the value of |\version|.
Suppose further, the flag should default to |final| for the main file
and to |draft| for child files
which is a natural assignment for editing the document.
This is achieved by placing the following code
in the preamble of the main document
(below the |\childdocmain| directive):
%
\begin{center}
\begin{tabular}{l}
|\ifchilddoc|\\
|\providecommand{\version}{draft}|\\
|\||else|\\
|\providecommand{\version}{final}|\\
|\||fi|
\end{tabular}
\end{center}
%
The definition by |\providecommand| makes sure
that previous definitions are not overwritten.
Further statements |\providecommand{\version}{...}|
can thus be added before the above code to override it.

For the main file, one might add a line
(between |\childdocmain| and the above block)
%
\begin{center}
|%\ifchilddoc\||else\providecommand{\version}{draft}\||fi|
\end{center}
%
which can be uncommented to produce a draft version.
Likewise one can add a line to the very top of a child file
(above the |\childdocof{|\textit{main}|}| directive)
%
\begin{center}
|%\providecommand{\version}{final}|
\end{center}
%
which can be uncommented to produce the final version of this child document.

%%%%%%%%%%%%%%%%%%%%%%%%%%%%%%%%%%%%%%%%%%%%%%%%%%%%%%%%%%%%%%%%%%%%%%%%%%%%%%%%
\subsection{Forwarding}
\label{sec:forward}

Different versions of the main or child documents
using compilation flags as described in \secref{sec:flags}
can be (permanently) stored in different files
for convenient compilation, viewing and distribution.
To this end, the package defines a command
to pass on compilation to a different file:

%%%%%%%%%%%%%%%%%%%%%%%%%%%%%%%%%%%%%%%%
\DescribeMacro{\childdocforward}
The command |\childdocforward| redirects processing to
another source file:
%
\begin{center}
\begin{tabular}{l}
|\input{childdoc.def}|\\
|\childdocforward[|\textit{main}|]{|\textit{dest}|}|\\
\end{tabular}
\end{center}
%
The argument \textit{dest} is the destination file
(without extension).
It should be the main file or one of the child files.
Note that further \textsf{childdoc} directives
such as |\childdocof| and |\childdocforward|
in the indicated file will be processed in this form.
The optional argument \textit{main}
passes on directly to the main file \textit{main}
while pretending to compile the child \textit{dest}.
This form behaves as if \textit{dest}
issues |\childdocof{|\textit{main}|}| right away,
and no further \textsf{childdoc} directives will be processed.

%%%%%%%%%%%%%%%%%%%%%%%%%%%%%%%%%%%%%%%%
\DescribeMacro{\...prefix}
In the alternative form |\childdocforwardprefix|,
%
\begin{center}
\begin{tabular}{l}
|\input{childdoc.def}|\\
|\childdocforwardprefix[|\textit{main}|]{|\textit{prefix}|}{|\textit{dest}|}|
\end{tabular}
\end{center}
%
the destination file is determined by a pattern
depending on the current file:
To make this work, the current file must be called
`{\textit{prefix}\hspace{0.2em}\textit{suffix}}'
with \textit{prefix} matching precisely the argument.
Processing is then passed on to the file
`{\textit{dest}\hspace{0.2em}\textit{suffix}}'.
Surely, the same effect is achieved by
directly specifying the
argument `{\textit{dest}\hspace{0.2em}\textit{suffix}}'
in the first form.
However, that requires to set up a different file
for each child. With the alternative form of the command
all these files can have exactly the same content
which simplifies setting them up and maintaining them.

For example, the following file |draft.tex|
with a compilation flag |\version| as described in \secref{sec:flags}
compiles the main document as a draft:
%
\begin{center}
\begin{tabular}{l}
|\def\version{draft}|\\
|\input{childdoc.def}|\\
|\childdocforward{|\textit{main}|}|
\end{tabular}
\end{center}
%
Likewise, the following files |final|\textit{nn}|.tex|
compile the final version of the child document
|child|\textit{nn}|.tex|:
%
\begin{center}
\begin{tabular}{l}
|\def\version{final}|\\
|\input{childdoc.def}|\\
|\childdocforwardprefix{final}{child}|
\end{tabular}
\end{center}
%

Note that when several versions of a main file and/or of each child file
are to be generated, it may be convenient to set up a |Makefile| or
shell script to automatise the process.

%%%%%%%%%%%%%%%%%%%%%%%%%%%%%%%%%%%%%%%%%%%%%%%%%%%%%%%%%%%%%%%%%%%%%%%%%%%%%%%%
\subsection{Command Line Processing}
\label{sec:commandline}

The effect of redirection files can also be achieved by invoking
the \LaTeX{} compiler with a more elaborate command line.
Most conveniently this should be done as part
of a shell script or a |Makefile|.

When using \textsf{childdoc} in the main file, the following
command lines effectively perform a redirection
(note that depending on the shell being used,
backslashes may have to be doubled: `|\|' $\to$ `|\\|'):
%
\begin{center}
|... -jobname "|\textit{target}|" |\\|"|[\textit{flags}]%
|\input{childdoc.def}\childdocforward[|\textit{main}|]{|\textit{dest}|}"|
\end{center}
%
Here \textit{target} is the name of the output file,
\textit{main} is the name of the main file
and \textit{dest} is the name of the main or child file to be processed
(all filenames without extensions).
The optional argument \textit{main} can be omitted
if \textit{main} matches \textit{dest}.
Optionally, compilation \textit{flags} can be defined via |\def| commands.
This command line makes the \TeX{} engine believe
it is compiling the file \textit{target}
whose content is specified as the latter parameter.
The provided code then forwards the processing to
\textit{main} or \textit{dest} as described in \secref{sec:forward}.

%%%%%%%%%%%%%%%%%%%%%%%%%%%%%%%%%%%%%%%%%%%%%%%%%%%%%%%%%%%%%%%%%%%%%%%%%%%%%%%%
\subsection{Include by Input}
\label{sec:input}

Including child documents by |\include| has some restrictions by design.
Most notably, the content of a child document always occupies
its own set of pages; pages cannot be shared between child documents.
Usually, this behaviour makes perfect sense
because each child document contain an essential part of the document.
However, in some situations it may be desirable to compose
a document from a collection of parts
without having mandatory page breaks between then.
For this case, the package
provides a mechanism to include parts
by |\input| which can also be processed individually.
However, by construction this mechanism
requires manual handling of the content to be output.

%%%%%%%%%%%%%%%%%%%%%%%%%%%%%%%%%%%%%%%%
\DescribeMacro{\ifchilddocmanual}
The main file should be prepared as usual, see \secref{sec:include}.
However, the document body must make a distinction
between processing of an individual part and of the main document, e.g.:
%
\begin{center}
\begin{tabular}{l}
|\ifchilddocmanual|\\
|\input{\childdocname}|\\
|\||else|\\
\textit{document body with }|\input{|\textit{part}|}|\\
|\||fi|
\end{tabular}
\end{center}
%
The conditional |\ifchilddocmanual| is true whenever
a part to be included by |\input| is being compiled,
and the name of the part is stored in |\childdocname|.

%%%%%%%%%%%%%%%%%%%%%%%%%%%%%%%%%%%%%%%%
\DescribeMacro{\childdocby}
Each part to be included by |\input| should start with:
%
\begin{center}
\begin{tabular}{l}
|\input{childdoc.def}|\\
|\childdocby{|\textit{main}|}|\\
\end{tabular}
\end{center}
%
The directive |\childdocby| is similar to |\childdocof|
described in \secref{sec:include},
but the subsequent selection of content must be done manually.
To that end, both |\ifchilddoc| and |\ifchilddocmanual|
will be true upon processing of a part,
and the name of the part is stored in |\childdocname|.
Note that |\jobname| will be set to the filename of the current part
so that each part receives an individual |.aux| file
that does not interfere with the |.aux| file(s) of the main document.
This behaviour can be altered by the alternative form
|\childdocby[*]{|\textit{main}|}| (with a non-empty optional argument)
which uses the |.aux| file of the main document
by setting |\jobname| to \textit{main}.

%%%%%%%%%%%%%%%%%%%%%%%%%%%%%%%%%%%%%%%%%%%%%%%%%%%%%%%%%%%%%%%%%%%%%%%%%%%%%%%%
\subsection{Driver Development}
\label{sec:driver}

The \textsf{childdoc} mechanism can also be use for the development
of definition files such as \LaTeX{} styles or classes.
This case differs from the above setup with multiple parts
included by |\include| in that no |\includeonly| should be invoked.
This can be achieved by starting the include file
(before |\ProvidesPackage|) with:
%
\begin{center}
\begin{tabular}{l}
|\input{childdoc.def}|\\
|\childdocforward{|\textit{main}|}|\\
\end{tabular}
\end{center}
%
or alternatively with:
%
\begin{center}
\begin{tabular}{l}
|\input{childdoc.def}|\\
|\childdocby{|\textit{main}|}|\\
\end{tabular}
\end{center}
%
Both forms have slightly different effects as described above.
The main file is prepared as usual, see \secref{sec:include}.

%%%%%%%%%%%%%%%%%%%%%%%%%%%%%%%%%%%%%%%%%%%%%%%%%%%%%%%%%%%%%%%%%%%%%%%%%%%%%%%%
\subsection{Legacy Detection}
\label{sec:detection}

The directive |\childdocmain| in the main file can detect
whether the complete document or merely a child is to be compiled
even without using the directive |\childdocof|.
This method is deprecated because it is less robust
and there is no compelling reason to use it;
it is merely provided for backward compatibility
and it may be removed in future versions.

If the detection mechanism is to be used,
it is mandatory to correctly specify
the filename of the main file as the argument of |\childdocmain|:
%
\begin{center}
\begin{tabular}{l}
|\input{childdoc.def}|\\
|\childdocmain{|\textit{main}|}|\\
\end{tabular}
\end{center}
%
If |\jobname| does not match the argument \textit{main} of |\childdocmain|,
it is assumed that |\jobname| points to the child file to be compiled.
When using |\childdocmain| with the main file specified as argument,
it suffices to start a child file
with just |\input{|\textit{main}|}|
without loading of the package and using |\childdocof|.
If instead all processing is done
with the appropriate \textsf{childdoc} directives,
the argument of \textit{main} of |\childdocmain| can be empty.

An alternative version of the command line processing described
in \secref{sec:commandline} using the detection mechanism reads:
%
\begin{center}
|... -jobname "|\textit{target}|" "|[\textit{flags}]%
[|\def\jobname{|\textit{dest}|}|]|\input{|\textit{main}|}"|
\end{center}

%%%%%%%%%%%%%%%%%%%%%%%%%%%%%%%%%%%%%%%%%%%%%%%%%%%%%%%%%%%%%%%%%%%%%%%%%%%%%%%%
\subsection{Manual Code}
\label{sec:manual}

In case one cannot be certain whether the definitions file |childdoc.def|
is installed on the target \TeX{} distribution
and one prefers not to ship it,
it is conceivable to paste a few relevant commands into the sources.

To that end, drop all statements |\input{childdoc.def}|
and perform the replacements as outlined below.
Instead of |\childdocmain{|\textit{main}|}| add the following code
to the top of the main file:
%
\begin{center}
\begin{tabular}{l}
|\||ifdefined\childdocname\endinput\||fi\newif\ifchilddoc|\\
|\edef\childdocname{\scantokens\expandafter{\jobname\noexpand}}|\\
|\def\childdocmain{|\textit{main}|}\||ifx\childdocmain\childdocname\||else|\\
|\childdoctrue\includeonly{\childdocname}\let\jobname\childdocmain\||fi|\\
\end{tabular}
\end{center}
%
Instead of |\childdocof{|\textit{main}|}| just include the main file
at the top of each child file:
%
\begin{center}
|\input{|\textit{main}|}|
\end{center}
%
A simple redirection |\childdocforward{|\textit{dest}|}| is achieved by:
%
\begin{center}
|\def\jobname{|\textit{dest}|}\input{\jobname}|
\end{center}
%
The redirection with prefix
|\childdocforwardprefix[|\textit{prefix}|]{|\textit{dest}|}|
is accomplished by:
%
\begin{center}
\begin{tabular}{l}
|{\edef\jobname{\scantokens\expandafter{\jobname\noexpand}}|\\
|\def\redirectjob |\textit{prefix}|#1~~~{\gdef\jobname{|\textit{dest}|#1}}|\\
|\expandafter\redirectjob\jobname~~~}\input{\jobname}|
\end{tabular}
\end{center}

In an alternative approach,
child documents can be compiled by a specific command line
without additional code or specific definitions:
%
\begin{center}
|... -jobname "|\textit{target}|" "|[\textit{flags}]%
|\includeonly{|\textit{dest}|}\input{|\textit{main}|}"|
\end{center}
%

%%%%%%%%%%%%%%%%%%%%%%%%%%%%%%%%%%%%%%%%%%%%%%%%%%%%%%%%%%%%%%%%%%%%%%%%%%%%%%%%
%%%%%%%%%%%%%%%%%%%%%%%%%%%%%%%%%%%%%%%%%%%%%%%%%%%%%%%%%%%%%%%%%%%%%%%%%%%%%%%%
\section{Information}

%%%%%%%%%%%%%%%%%%%%%%%%%%%%%%%%%%%%%%%%%%%%%%%%%%%%%%%%%%%%%%%%%%%%%%%%%%%%%%%%
\subsection{Copyright}

Copyright \copyright{} 2017--2018 Niklas Beisert

This work may be distributed and/or modified under the
conditions of the \LaTeX{} Project Public License, either version 1.3
of this license or (at your option) any later version.
The latest version of this license is in
  \url{http://www.latex-project.org/lppl.txt}
and version 1.3 or later is part of all distributions of \LaTeX{}
version 2005/12/01 or later.

This work has the LPPL maintenance status `maintained'.

The Current Maintainer of this work is Niklas Beisert.

This work consists of the files |README.txt|, |childdoc.ins| and |childdoc.dtx|
as well as the derived files |childdoc.def|, |cdocsamp.tex|
with |cdocsch1.tex|, |cdocsch2.tex|, |cdocspt3.tex|, |cdocspt4.tex|,
|cdocsdrf.tex|, |cdocsfn1.tex|, |cdocsfn2.tex|
as well as |childdoc.pdf|.

%%%%%%%%%%%%%%%%%%%%%%%%%%%%%%%%%%%%%%%%%%%%%%%%%%%%%%%%%%%%%%%%%%%%%%%%%%%%%%%%
\subsection{Files and Installation}

The package consists of the files:
%
\begin{center}
\begin{tabular}{ll}
    |README.txt|   & readme file \\
    |childdoc.ins| & installation file \\
    |childdoc.dtx| & source file \\
    |childdoc.def| & definition file \\
    |cdocsamp.tex| & sample main file \\
    |cdocsch1.tex| & sample include file \\
    |cdocsch2.tex| & sample include file \\
    |cdocspt3.tex| & sample part file \\
    |cdocspt4.tex| & sample part file \\
    |cdocsdrf.tex| & sample redirection file \\
    |cdocsfn1.tex| & sample redirection file \\
    |cdocsfn2.tex| & sample redirection file \\
    |childdoc.pdf| & manual
\end{tabular}
\end{center}
%
The distribution consists of the files
|README.txt|, |childdoc.ins| and |childdoc.dtx|.
%
\begin{itemize}
\item
Run (pdf)\LaTeX{} on |childdoc.dtx|
to compile the manual |childdoc.pdf| (this file).
\item
Run \LaTeX{} on |childdoc.ins| to create the definitions file |childdoc.def|
and the sample |cdocsamp.tex| with include files
|cdocsch1.tex|, |cdocsch2.tex|, |cdocspt3.tex|, |cdocspt4.tex|,
|cdocsdrf.tex|, |cdocsfn1.tex|, |cdocsfn2.tex|.
Then copy the file |childdoc.def| to an appropriate directory of your \LaTeX{}
distribution, e.g.\ \textit{texmf-root}|/tex/latex/childdoc|.
\end{itemize}

%%%%%%%%%%%%%%%%%%%%%%%%%%%%%%%%%%%%%%%%%%%%%%%%%%%%%%%%%%%%%%%%%%%%%%%%%%%%%%%%
\subsection{Related CTAN Packages}

There are several other packages which offer a similar functionality:
%
\begin{itemize}
\item
The packages
\href{http://ctan.org/pkg/docmute}{\textsf{docmute}},
\href{http://ctan.org/pkg/includex}{\textsf{includex}} and
\href{http://ctan.org/pkg/standalone}{\textsf{standalone}}
provide commands to include only the document body of
a child file thus allowing both files to be compiled individually.
\item
The packages \href{http://ctan.org/pkg/subdocs}{\textsf{subdocs}}
and \href{http://ctan.org/pkg/subfiles}{\textsf{subfiles}}
provide structures in which the main and child documents can be
encapsulated and allowing them to be compiled individually.
The inclusion mechanism is different from the conventional |\include|.
\item
The package \href{http://ctan.org/pkg/combine}{\textsf{combine}}
is an elaborate solution to combine several documents into one.
\end{itemize}
%
See also the CTAN topic \href{http://ctan.org/topic/subdocs}{\textsf{subdocs}}
for further related packages.
The present package differs from the above solutions in that
a document structure constructed with the conventional |\include| mechanism
just needs two extra commands at the top of every file
such that all constituent files can be compiled individually.

%%%%%%%%%%%%%%%%%%%%%%%%%%%%%%%%%%%%%%%%%%%%%%%%%%%%%%%%%%%%%%%%%%%%%%%%%%%%%%%%
%\subsection{Feature Suggestions}
%
%The following is a list of features which may be useful for future
%versions of this package:
%%
%\begin{itemize}
%\item
%\ldots
%\end{itemize}

%%%%%%%%%%%%%%%%%%%%%%%%%%%%%%%%%%%%%%%%%%%%%%%%%%%%%%%%%%%%%%%%%%%%%%%%%%%%%%%%
\subsection{Revision History}

%%%%%%%%%%%%%%%%%%%%%%%%%%%%%%%%%%%%%%%%
\paragraph{v2.0:} 2018/12/30

\begin{itemize}
\item
immediate forward processing
\item
added |\childdocby| mechanism
\item
manual restructured
\end{itemize}

%%%%%%%%%%%%%%%%%%%%%%%%%%%%%%%%%%%%%%%%
\paragraph{v1.6:} 2018/01/17

\begin{itemize}
\item
application for development of include files
\item
corrections to manual
\end{itemize}

%%%%%%%%%%%%%%%%%%%%%%%%%%%%%%%%%%%%%%%%
\paragraph{v1.5:} 2017/05/21

\begin{itemize}
\item
more complete structuring introduced
\item
|\childdocof| introduced
\item
|\childdoc| renamed to |\childdocmain|
\item
|\childredirect| renamed to |\childdocforward| and |\childdocforwardprefix|
and functionality expanded
\end{itemize}

%%%%%%%%%%%%%%%%%%%%%%%%%%%%%%%%%%%%%%%%
\paragraph{v1.0:} 2017/04/27

\begin{itemize}
\item
manual and install package
\item
first version published on CTAN
\end{itemize}

%%%%%%%%%%%%%%%%%%%%%%%%%%%%%%%%%%%%%%%%
\paragraph{v0.6:} 2017/04/26

\begin{itemize}
\item
redirection mechanism added
\end{itemize}

%%%%%%%%%%%%%%%%%%%%%%%%%%%%%%%%%%%%%%%%
\paragraph{v0.5:} 2017/04/26

\begin{itemize}
\item
functionality in definition file
\end{itemize}


%%%%%%%%%%%%%%%%%%%%%%%%%%%%%%%%%%%%%%%%%%%%%%%%%%%%%%%%%%%%%%%%%%%%%%%%%%%%%%%%
%%%%%%%%%%%%%%%%%%%%%%%%%%%%%%%%%%%%%%%%%%%%%%%%%%%%%%%%%%%%%%%%%%%%%%%%%%%%%%%%
%%%%%%%%%%%%%%%%%%%%%%%%%%%%%%%%%%%%%%%%%%%%%%%%%%%%%%%%%%%%%%%%%%%%%%%%%%%%%%%%
\appendix

\settowidth\MacroIndent{\rmfamily\scriptsize 000\ }

 \DocInput{childdoc.dtx}

\end{document}
%</driver>
% \fi
%
% %%%%%%%%%%%%%%%%%%%%%%%%%%%%%%%%%%%%%%%%%%%%%%%%%%%%%%%%%%%%%%%%%%%%%%%%%%%%%%
% %%%%%%%%%%%%%%%%%%%%%%%%%%%%%%%%%%%%%%%%%%%%%%%%%%%%%%%%%%%%%%%%%%%%%%%%%%%%%%
% \section{Sample}
%\iffalse
%<*samplemain>
%\fi
%
% The following presents a sample document
% with two chapters, two parts, a title page,
% a compile flag as well as three forwarding files to set the flag.
% It consists of eight |.tex| files:
% \begin{center}
% \begin{tabular}{ll}
% |cdocsamp.tex|&main file\\
% |cdocsch1.tex|&include file for chapter 1\\
% |cdocsch2.tex|&include file for chapter 2\\
% |cdocspt3.tex|&include file for part 3\\
% |cdocspt4.tex|&include file for part 4\\
% |cdocsdrf.tex|&forwarding file for main file in draft mode\\
% |cdocsfi1.tex|&forwarding file for final version of chapter 1\\
% |cdocsfi2.tex|&forwarding file for final version of chapter 2\\
% \end{tabular}
% \end{center}
% Each of the eight files can be compiled directly by the \LaTeX{} compiler.
%
% %%%%%%%%%%%%%%%%%%%%%%%%%%%%%%%%%%%%%%
% \paragraph{Main File.}
%
% The main file is called |cdocsamp.tex|.
%
% Load the \textsf{childdoc} definitions and
% declare the filename for the main document:
%    \begin{macrocode}
\input{childdoc.def}
\childdocmain{}
%    \end{macrocode}

% Optional override for |\version| flag:
%    \begin{macrocode}
%%\ifchilddoc\else\providecommand{\version}{draft}\fi
%    \end{macrocode}

% Define the default values for the |\version| flag
% (|final| for the main file and |draft| for childs):
%    \begin{macrocode}
\ifchilddoc
\providecommand{\version}{draft}
\else
\providecommand{\version}{final}
\fi
%    \end{macrocode}

% Load the standard document class:
%    \begin{macrocode}
\documentclass[12pt]{article}
%    \end{macrocode}

% Start the document body:
%    \begin{macrocode}
\begin{document}
%    \end{macrocode}

% Declare a title page.
% Print title, part of document being processed and version flag:
%    \begin{macrocode}
\addtocounter{page}{-1}
\begin{center}
{\LARGE\bfseries{}childdoc example\par}
\vspace{1cm}
\ifchilddoc
\ifchilddocmanual part\else chapter\fi:
`\childdocname' of `\childdocjob'\par
\else
main document: `\childdocjob'\par
\fi
version: \version\par
\end{center}
\newpage
%    \end{macrocode}

% Manually include selected file,
% otherwise process as usual:
%    \begin{macrocode}
\ifchilddocmanual
\section*{part `\childdocname'}
\input{\childdocname}
\else
%    \end{macrocode}

% Include the two chapters:
%    \begin{macrocode}
\include{cdocsch1}
\include{cdocsch2}
%    \end{macrocode}

% Include the two parts unless only chapters should be displayed:
%    \begin{macrocode}
\ifchilddoc\else
\section{part three}
\input{cdocspt3}
\section{part four}
\input{cdocspt4}
\fi
%    \end{macrocode}

% Process as usual until here:
%    \begin{macrocode}
\fi
%    \end{macrocode}

% End of document body:
%    \begin{macrocode}
\end{document}
%    \end{macrocode}
%\iffalse
%</samplemain>
%\fi
%
% %%%%%%%%%%%%%%%%%%%%%%%%%%%%%%%%%%%%%%
% \paragraph{Chapter Include Files.}
%
% The include files are called |cdocsch1.tex| and |cdocsch2.tex|.
%
%\iffalse
%<*samplechap1|samplechap2>
%\fi

% Optional override for |\version| flag:
%    \begin{macrocode}
%%\providecommand{\version}{final}
%    \end{macrocode}

% Include the main document:
%    \begin{macrocode}
\input{childdoc.def}
\childdocof{cdocsamp}
%    \end{macrocode}

%\iffalse
%</samplechap1|samplechap2>
%\fi
%
%\iffalse
%<*samplechap1>
%\fi
% Some text for chapter 1:
%    \begin{macrocode}
\section{one}
some text in chapter one
%    \end{macrocode}

%\iffalse
%</samplechap1>
%\fi
% Some text for chapter 2:
%\iffalse
%<*samplechap2>
%\fi
%    \begin{macrocode}
\section{two}
more text in chapter two
%    \end{macrocode}

%\iffalse
%</samplechap2>
%\fi
%
% %%%%%%%%%%%%%%%%%%%%%%%%%%%%%%%%%%%%%%
% \paragraph{Part Include Files.}
%
% The include files are called |cdocspt3.tex| and |cdocspt4.tex|.
%
%\iffalse
%<*samplepart3|samplepart4>
%\fi

% Optional override for |\version| flag:
%    \begin{macrocode}
%%\providecommand{\version}{final}
%    \end{macrocode}

% Include the main document:
%    \begin{macrocode}
\input{childdoc.def}
\childdocby{cdocsamp}
%    \end{macrocode}

%\iffalse
%</samplepart3|samplepart4>
%\fi
%
%\iffalse
%<*samplepart3>
%\fi
% Some text for part 3:
%    \begin{macrocode}
some text in part three
%    \end{macrocode}

%\iffalse
%</samplepart3>
%\fi
% Some text for part 4:
%\iffalse
%<*samplepart4>
%\fi
%    \begin{macrocode}
more text in part four
%    \end{macrocode}

%\iffalse
%</samplepart4>
%\fi
%
% %%%%%%%%%%%%%%%%%%%%%%%%%%%%%%%%%%%%%%
% \paragraph{Forwarding for a Complete Draft.}
%
% The following forwarding file |cdocsdrf.tex|
% compiles the main document in draft mode:
%\iffalse
%<*sampledraft>
%\fi
%    \begin{macrocode}
\def\version{draft}
\input{childdoc.def}
\childdocforward{cdocsamp}
%    \end{macrocode}

%\iffalse
%</sampledraft>
%\fi
%
% %%%%%%%%%%%%%%%%%%%%%%%%%%%%%%%%%%%%%%
% \paragraph{Forwarding for Final Version of the Chapters.}
%
% The following forwarding files |cdocsfn1.tex| and |cdocsfn2.tex|
% (with identical content)
% compile the final versions of the child documents
% |cdocsch1.tex| and |cdocsch2.tex|, respectively:
%\iffalse
%<*samplefinal>
%\fi
%    \begin{macrocode}
\def\version{final}
\input{childdoc.def}
\childdocforwardprefix[cdocsamp]{cdocsfn}{cdocsch}
%    \end{macrocode}

%\iffalse
%</samplefinal>
%\fi
%
% %%%%%%%%%%%%%%%%%%%%%%%%%%%%%%%%%%%%%%
% \paragraph{Command Line Processing.}
%
% The following three command lines generate the output files
% |cdocscld|, |cdocscl1| and |cdocscl2|
% which should be identical to
% |cdocsdrf|, |cdocsch1| and |cdocsfn2|, respectively:
% \begin{center}
% \begin{tabular}{l}
% |latex -jobname cdocscld \|\\
% |  "\def\version{draft}\input{childdoc.def}\childdocforward{cdocsamp}"|\\
% |latex -jobname cdocscl1 \|\\
% |  "\input{childdoc.def}\childdocforward[cdocsamp]{cdocsch1}"|\\
% |latex -jobname cdocscl2 \|\\
% |  "\def\version{final}\input{childdoc.def}\childdocforward{cdocsch2}"|
% \end{tabular}
% \end{center}
% Note that the trailing backslash on each first line
% merely continues the input to the second line
% (for convenient cut ant paste).
% Furthermore, the command |latex| can be replaced by any
% of its alternative versions such as |pdflatex|.
%
% %%%%%%%%%%%%%%%%%%%%%%%%%%%%%%%%%%%%%%%%%%%%%%%%%%%%%%%%%%%%%%%%%%%%%%%%%%%%%%
% %%%%%%%%%%%%%%%%%%%%%%%%%%%%%%%%%%%%%%%%%%%%%%%%%%%%%%%%%%%%%%%%%%%%%%%%%%%%%%
% \section{Implementation}
%\iffalse
%<*package>
%\fi
%
% This section describes the definitions file |childdoc.def|.

% The definitions cannot be loaded using |\usepackage| or |\RequirePackage|
% which has a mechanism to prevent loading a style file more than once.
% When loading the definitions by means of |\input|
% multiple instances have to be prevented manually:
%\iffalse
%This code needs to be before the `\ProvidesFile' directive
%which is defined at the beginning of this file.
%Therefore it is also placed there and commented out here.
%</package>
%<*discard>
%\fi
%    \begin{macrocode}
\ifdefined\childdocmain\endinput\fi
%    \end{macrocode}
%\iffalse
%</discard>
%<*package>
%\fi
%
% \macro{\ifchilddoc}
% \macro{\ifchilddocmanual}
% The conditional |\ifchilddoc| tells whether a
% child (true) or main (false) document is being compiled.
% The conditional |\ifchilddocmanual| tells whether
% the |\includeonly| mechanism is used (false) or
% the selection of child files must be performed manually (true).
% The definitions initialise to false:
%    \begin{macrocode}
\newif\ifchilddoc
\newif\ifchilddocmanual
%    \end{macrocode}

% \macro{\childdocname}
% \macro{\childdocjob}
% The macro |\childdocname| stores the name of the main document
% to be compiled. The macro |\childdocjob| stores the name of
% the document on which the \LaTeX{} compiler was originally invoked.
% The content of |\jobname| cannot be compared
% to filenames specified in the source due to different catcodes.
% The following code rescans |\jobname|, stores the result
% in |\childdocname| and saves a copy in |\childdocjob|:
%    \begin{macrocode}
\edef\childdocname{\scantokens\expandafter{\jobname\noexpand}}
\let\childdocjob\childdocname
%    \end{macrocode}

% \macro{\childdocdisable}
% The macro |\childdocdisable| prevents the main file
% from being processed more than once.
% At this stage, the main document command |\childdocmain|
% is assumed to be called once again where it should do nothing.
% Any subsequent call to it should prevent
% a secondary processing of the main document
% It overwrites the forwarding commands
% |\childdocof| and |\childdocforward|
% with empty macros to prevent further inclusions of the main document:
%    \begin{macrocode}
\newcommand{\childdocdisable}
{
  \renewcommand{\childdocmain}[1]{\renewcommand{\childdocmain}[1]{\endinput}}
  \renewcommand{\childdocof}[1]{}
  \renewcommand{\childdocby}[2][]{}
  \renewcommand{\childdocforward}[2][]{}
  \renewcommand{\childdocdisable}{}
}
%    \end{macrocode}

% \macro{\childdocmain}
% The macro |\childdocmain| is to be called at the top of the main file
% with nothing or the main filename (without extension) as argument.
% First, it breaks loops.
% If the argument is not empty and does not match |\childdocname|
% (which is set by the first inclusion of |childdoc.def|),
% |\ifchilddoc| is set to true, |\includeonly| is applied to the child file
% and |\jobname| is set to the main file
% (for proper handling of |.aux| files):
%    \begin{macrocode}
\newcommand{\childdocmain}[1]
{
  \childdocdisable\childdocmain{}
  \if?#1?\else
    \begingroup
      \def\childdoctmp{#1}
      \ifx\childdoctmp\childdocname
        \def\childdoctmp{}
      \else
        \def\childdoctmp
        {
          \childdoctrue
          \includeonly{\childdocname}
          \def\childdocjob{#1}
          \def\jobname{#1}
        }
      \fi
      \expandafter
    \endgroup
    \childdoctmp
  \fi
}
%    \end{macrocode}

% \macro{\childdocof}
% The command |\childdocof| redirects
% compilation to the main file |#1|.
%    \begin{macrocode}
\newcommand{\childdocof}[1]
{
  \childdocdisable
  \childdoctrue
  \includeonly{\childdocname}
  \def\jobname{#1}
  \def\childdocjob{#1}
  \input{#1}
}
%    \end{macrocode}

% \macro{\childdocby}
% The command |\childdocby| ....
%    \begin{macrocode}
\newcommand{\childdocby}[2][]
{
  \childdocdisable
  \childdoctrue
  \childdocmanualtrue
  \if?#1?\else
    \def\jobname{#2}
  \fi
  \def\childdocjob{#2}
  \input{#2}
  \endinput
}
%    \end{macrocode}

% \macro{\childdocforward}
% The command |\childdocforward| redirects
% compilation to the main file or
% (if the optional argument is given) a child file.
% Parameters are set as if the main file
% or a child file starting with |\childdocof| was compiled.
% Then compilation is handed over to the main file:
%    \begin{macrocode}
\newcommand{\childdocforward}[2][]
{
  \begingroup
    \if?#1?
      \def\childdoctmp
      {
        \def\childdocname{#2}
        \def\childdocjob{#2}
        \def\jobname{#2}
        \input{#2}
        \endinput
      }
    \else
      \def\childdoctmp
      {
        \childdocdisable
        \def\childdocname{#2}
        \childdoctrue
        \includeonly{#2}
        \def\childdocjob{#1}
        \def\jobname{#1}
        \input{#1}
        \endinput
      }
    \fi
    \expandafter
  \endgroup
  \childdoctmp
}
%    \end{macrocode}

% \macro{\childdocforwardprefix}
% The command |\childdocforwardprefix| redirects
% compilation to the main or a child file by means of a pattern.
% The prefix |#1| in the current filename is replaced by |#2|
% and the suffix of the current filename is kept
% (it is assumed that the filename does not contain the substring `|~~~|'
% which is used as a delimiter).
% Compilation is handed over to the new file by |\childdocforward|:
%    \begin{macrocode}
\newcommand{\childdocforwardprefix}[3][]
{
  \begingroup
    \def\childdocextract #2##1~~~{\def\childdoctmp{\childdocforward[#1]{#3##1}}}
    \expandafter\childdocextract\childdocname~~~
    \expandafter
  \endgroup
  \childdoctmp
}
%    \end{macrocode}

% \macro{\childdoc}
% The deprecated macro |\childdoc| is a legacy version of |\childdocmain|:
%    \begin{macrocode}
\newcommand{\childdoc}{\childdocmain}
%    \end{macrocode}

% \macro{\childdocredirect}
% The deprecated macro |\childdocredirect| is a legacy version
% of |\childdocforward| and |\childdocforwardprefix|:
%    \begin{macrocode}
\newcommand{\childdocredirect}[2][]
{
  \begingroup
    \if?#1?
      \def\childdoctmp{\childdocforward{#2}}
    \else
      \def\childdoctmp{\childdocforwardprefix{#1}{#2}}
    \fi
    \expandafter
  \endgroup
  \childdoctmp
}
%    \end{macrocode}

%\iffalse
%</package>
%\fi
%
\endinput
|\\
|\childdocmain{|\textit{main}|}|\\
\end{tabular}
\end{center}
%
If |\jobname| does not match the argument \textit{main} of |\childdocmain|,
it is assumed that |\jobname| points to the child file to be compiled.
When using |\childdocmain| with the main file specified as argument,
it suffices to start a child file
with just |\input{|\textit{main}|}|
without loading of the package and using |\childdocof|.
If instead all processing is done
with the appropriate \textsf{childdoc} directives,
the argument of \textit{main} of |\childdocmain| can be empty.

An alternative version of the command line processing described
in \secref{sec:commandline} using the detection mechanism reads:
%
\begin{center}
|... -jobname "|\textit{target}|" "|[\textit{flags}]%
[|\def\jobname{|\textit{dest}|}|]|\input{|\textit{main}|}"|
\end{center}

%%%%%%%%%%%%%%%%%%%%%%%%%%%%%%%%%%%%%%%%%%%%%%%%%%%%%%%%%%%%%%%%%%%%%%%%%%%%%%%%
\subsection{Manual Code}
\label{sec:manual}

In case one cannot be certain whether the definitions file |childdoc.def|
is installed on the target \TeX{} distribution
and one prefers not to ship it,
it is conceivable to paste a few relevant commands into the sources.

To that end, drop all statements |% \iffalse
%
% childdoc.dtx Copyright (C) 2017-2018 Niklas Beisert
%
% This work may be distributed and/or modified under the
% conditions of the LaTeX Project Public License, either version 1.3
% of this license or (at your option) any later version.
% The latest version of this license is in
%   http://www.latex-project.org/lppl.txt
% and version 1.3 or later is part of all distributions of LaTeX
% version 2005/12/01 or later.
%
% This work has the LPPL maintenance status `maintained'.
%
% The Current Maintainer of this work is Niklas Beisert.
%
% This work consists of the files childdoc.dtx and childdoc.ins
% and the derived files childdoc.def and cdocsamp.tex with
% cdocsch1.tex, cdocsch2.tex, cdocsdrf.tex, cdocsfn1.tex, cdocsfn2.tex.
%
%<package>\ifdefined\childdocmain\endinput\fi
%<package>\ProvidesFile{childdoc.def}[2018/12/30 v2.0 child document driver]
%<samplemain>\ProvidesFile{cdocsamp.tex}[2018/12/30 v2.0 sample for childdoc]
%<*driver>
%\ProvidesFile{childdoc.drv}[2018/12/30 v2.0 childdoc reference manual file]
\PassOptionsToClass{10pt,a4paper}{article}
\documentclass{ltxdoc}

\usepackage[margin=35mm]{geometry}
\usepackage{hyperref}
\usepackage{hyperxmp}
\usepackage[usenames]{color}

\hypersetup{colorlinks=true}
\hypersetup{pdfstartview=FitH}
\hypersetup{pdfpagemode=UseNone}
\hypersetup{pdfsource={}}
\hypersetup{pdflang={en-UK}}
\hypersetup{pdfcopyright={Copyright 2017-2018 Niklas Beisert.
  This work may be distributed and/or modified under the
  conditions of the LaTeX Project Public License, either version 1.3
  of this license or (at your option) any later version.}}
\hypersetup{pdflicenseurl={http://www.latex-project.org/lppl.txt}}
\hypersetup{pdfcontactaddress={ETH Zurich, ITP, HIT K,
  Wolfgang-Pauli-Strasse 27}}
\hypersetup{pdfcontactpostcode={8093}}
\hypersetup{pdfcontactcity={Zurich}}
\hypersetup{pdfcontactcountry={Switzerland}}
\hypersetup{pdfcontactemail={nbeisert@itp.phys.ethz.ch}}
\hypersetup{pdfcontacturl={http://people.phys.ethz.ch/\xmptilde nbeisert/}}

\newcommand{\secref}[1]{\hyperref[#1]{section \ref*{#1}}}

\parskip1ex
\parindent0pt
\let\olditemize\itemize
\def\itemize{\olditemize\parskip0pt}

\begin{document}

\title{The \textsf{childdoc} Package}
\hypersetup{pdftitle={The childdoc Package}}
\author{Niklas Beisert\\[2ex]
  Institut f\"ur Theoretische Physik\\
  Eidgen\"ossische Technische Hochschule Z\"urich\\
  Wolfgang-Pauli-Strasse 27, 8093 Z\"urich, Switzerland\\[1ex]
  \href{mailto:nbeisert@itp.phys.ethz.ch}
  {\texttt{nbeisert@itp.phys.ethz.ch}}}
\hypersetup{pdfauthor={Niklas Beisert}}
\hypersetup{pdfsubject={Manual for the LaTeX2e Package childdoc}}
\date{30 December 2018, \textsf{v2.0}}
\maketitle

\begin{abstract}\noindent
\textsf{childdoc} is a \LaTeXe{} package
that enables the direct compilation
of document sections included by |\include|
to individual files.
\end{abstract}

\begingroup
\parskip0ex
\tableofcontents
\endgroup

%%%%%%%%%%%%%%%%%%%%%%%%%%%%%%%%%%%%%%%%%%%%%%%%%%%%%%%%%%%%%%%%%%%%%%%%%%%%%%%%
%%%%%%%%%%%%%%%%%%%%%%%%%%%%%%%%%%%%%%%%%%%%%%%%%%%%%%%%%%%%%%%%%%%%%%%%%%%%%%%%
\section{Introduction}

\LaTeX{} provides a mechanism to structure a large document (such as a book)
into a main file and several child files (containing the chapters)
using the |\include| command.
This mechanism is beneficial for documents
which span hundreds of pages in order to
make the source file(s) more manageable.
Moreover, compilation can be restricted to
selected child files by means of the |\includeonly| command.
The latter feature can be used to reduce the compilation time while editing
(this was significantly more useful in the earlier days of \LaTeX{})
or to generate a smaller document which is easier to navigate.
Another application of |\includeonly| is to generate
documents consisting of selected parts of the complete document.

However, there are a few drawbacks of the plain |\include| mechanism:
\begin{itemize}
\item
The child files cannot be compiled on their own,
they can only be compiled via the main file.
A naive editing environment
(such as a text editor with an option
to have the current file processed by \LaTeX)
may require one to switch to the main file before compiling;
attempting to compile the child file produces errors.
\item
The main file must be modified (each time)
to adjust the |\includeonly| command
to the present needs. This easily leaves the main file in a messy state.
\item
The generated document will always carry the filename
of the main document. This is inconvenient if
several child files are to be compiled and
to be kept for distribution.
\end{itemize}

The present package provides a simple interface
to make child files individually compilable by \LaTeX{}.
Compiling a child file then has the same effect as compiling
the main file with an |\includeonly| command
to select the appropriate child.
Moreover the generated document will carry the name of the child
rather than the main file.
This resolves all three above issues.

This feature is meant to make the editing of books,
thesis documents and lecture notes somewhat more convenient.
However, the package can also be used efficiently for
composing a series of documents (such as exercise sheets)
which are typically distributed individually.
It then assists the author in generating the individual documents
(potentially in different versions)
as well as a document containing the collected series.
Another application is in developing style files
or other kinds of included material
where compilation of the style file could redirect
to a sample or test file.

%%%%%%%%%%%%%%%%%%%%%%%%%%%%%%%%%%%%%%%%%%%%%%%%%%%%%%%%%%%%%%%%%%%%%%%%%%%%%%%%
%%%%%%%%%%%%%%%%%%%%%%%%%%%%%%%%%%%%%%%%%%%%%%%%%%%%%%%%%%%%%%%%%%%%%%%%%%%%%%%%
\section{Usage}

First of all, the package \textsf{childdoc} is \emph{not} a standard
\LaTeXe{} |.sty| style file! Therefore it needs to be invoked in
a non-standard way.

%%%%%%%%%%%%%%%%%%%%%%%%%%%%%%%%%%%%%%%%%%%%%%%%%%%%%%%%%%%%%%%%%%%%%%%%%%%%%%%%
\subsection{Included Files}
\label{sec:include}

%%%%%%%%%%%%%%%%%%%%%%%%%%%%%%%%%%%%%%%%
\DescribeMacro{\childdocmain}
To use the package, add the commands
\begin{center}
\begin{tabular}{l}
|\input{childdoc.def}|\\
|\childdocmain{}|\\
\end{tabular}
\end{center}
at the very top of the main \LaTeX{} file,
in particular \emph{before} the |\documentclass| statement!
The argument of |\childdocmain| should be left empty
(but it must be present).

%%%%%%%%%%%%%%%%%%%%%%%%%%%%%%%%%%%%%%%%
\DescribeMacro{\childdocof}
Furthermore, add the commands
\begin{center}
\begin{tabular}{l}
|\input{childdoc.def}|\\
|\childdocof{|\textit{main}|}|\\
\end{tabular}
\end{center}
at the top of every child file \textit{child}
which is included by |\include{|\textit{child}|}|
from within the main file
(or at least for those files to be compiled individually).
The argument \textit{main} must be the filename of the main file.

There are a couple of
considerations in setting up the main and child documents:

%%%%%%%%%%%%%%%%%%%%%%%%%%%%%%%%%%%%%%%%
\paragraph{Restrictions.}

Please note the following restrictions:
\begin{itemize}
\item
|\childdocmain| must be called with one argument \textit{main}
to ensure compatibility with earlier version of the package.
It must either be empty (|\childdocmain{}|)
or precisely match the filename of the main file in which it is specified.
See \secref{sec:detection} for further information.
\item
The filename \textit{main} must be specified without the |.tex| extension.
\item
The filename \textit{main} is case sensitive
(even in case-insensitive file systems)
due to internal string comparison.
\item
The argument \textit{main} should be fully expanded, it cannot be a macro.
\item
Subdirectories and special characters should be avoided in filenames.
\item
The command |\childdocmain{|\textit{main}|}| must be followed by a whitespace.
It should not be followed immediately by another command
or by a comment mark `|%|'.
This is because the \TeX{} parser reads the token immediately following
the argument of |\childdocmain| and puts it
at the beginning of every child section;
however, a white\-space is ignored.
\end{itemize}

%%%%%%%%%%%%%%%%%%%%%%%%%%%%%%%%%%%%%%%%
\paragraph{Content of Main File.}

It is advisable to place all content in the child files included by |\include|.
Any output contained in the main file will appear in all child documents
unless suppressed manually;
it cannot be suppressed automatically by the |\includeonly| directive
and thus should normally be avoided.
A method to include some content in the main file
by means of conditional processing is described in \secref{sec:conditional}.

%%%%%%%%%%%%%%%%%%%%%%%%%%%%%%%%%%%%%%%%
\paragraph{Page Numbering.}

When only a part of the document is compiled,
the appropriate numbering of pages
(as well as other status parameters)
is determined from the |.aux| files.
The latter contain information from previous passes.
However this information needs to propagate through
all intermediate child documents.
Therefore the page numbering in child documents may well
be inconsistent until the complete document is compiled at least once.

A useful (if unconventional) way to always ensure a consistent
page numbering is to restart the numbering in each child document
and denote the pages by `\textit{child}|.|\textit{page}'
where \textit{child} represents the chapter/section number of the child file.
This can be achieved by the command
|\numberwithin{page}{|\textit{child}|}|
of the \textsf{amsmath} package
where \textit{child} can be |chapter| or |section|
depending on the chosen structuring.
Alternatively, one can modify the macro |\thepage| appropriately
and reset the counter |page| at the start of each child file.

%%%%%%%%%%%%%%%%%%%%%%%%%%%%%%%%%%%%%%%%%%%%%%%%%%%%%%%%%%%%%%%%%%%%%%%%%%%%%%%%
\subsection{Conditional Processing}
\label{sec:conditional}

The package provides a mechanism to compile different versions
of a document. To customise the versions further some conditional processing
can come in handy to distinguish which version is being compiled.
The package provides two macros to describe the compilation context:

%%%%%%%%%%%%%%%%%%%%%%%%%%%%%%%%%%%%%%%%
\DescribeMacro{\ifchilddoc}
The conditional |\ifchilddoc| distinguishes between the compilation of
child documents and the main document:
%
\begin{center}
|\ifchilddoc |\textit{child-code}| |[|\||else |\textit{main-code}]| \||fi|
\end{center}

%%%%%%%%%%%%%%%%%%%%%%%%%%%%%%%%%%%%%%%%
\DescribeMacro{\childdocname}
\DescribeMacro{\childdocjob}
The macro |\childdocname| contains the filename (without extension)
of the main or child file being processed.
Note that |\childdocjob| will always contain the name of the main file.

%%%%%%%%%%%%%%%%%%%%%%%%%%%%%%%%%%%%%%%%
\paragraph{Title Page.}

Conditional processing can be used to include a title or banner page
in the main document when proper precautions are taken.
Importantly, the code in the main file should ensure that the page counter
(as well as other status parameters which are stored in the |.aux| files)
takes the same value after the conditional processing.
Otherwise the page numbers may take divergent values
depending on which part is compiled.

For example, a title page could be declared by:
%
\begin{center}
\begin{tabular}{l}
|\ifchilddoc\||else|\\
|\addtocounter{page}{-1}|\\
\textit{code for title page}\\
|\newpage|\\
|\||fi|
\end{tabular}
\end{center}
%
A banner page for the child documents can be generated by:
%
\begin{center}
\begin{tabular}{l}
|\ifchilddoc|\\
|\addtocounter{page}{-1}|\\
\textit{code for banner page}\\
|\newpage|\\
|\||fi|
\end{tabular}
\end{center}
%
Here one could write a message such as:
\begin{center}
|This is the part \childdocname{} of \childdocjob{}.|
\end{center}

%%%%%%%%%%%%%%%%%%%%%%%%%%%%%%%%%%%%%%%%%%%%%%%%%%%%%%%%%%%%%%%%%%%%%%%%%%%%%%%%
\subsection{Flags}
\label{sec:flags}

The package makes it easy to generate different versions
of the main or child documents.
To this end compilation flags can be defined
and assigned different default values.
They will be particularly useful in conjunction
with the forwarding mechanism described in \secref{sec:forward}.

For example, it may be useful to have a flag |\version|
which can be set to |draft| or |final|.
The document source will contain some conditional code
depending on the value of |\version|.
Suppose further, the flag should default to |final| for the main file
and to |draft| for child files
which is a natural assignment for editing the document.
This is achieved by placing the following code
in the preamble of the main document
(below the |\childdocmain| directive):
%
\begin{center}
\begin{tabular}{l}
|\ifchilddoc|\\
|\providecommand{\version}{draft}|\\
|\||else|\\
|\providecommand{\version}{final}|\\
|\||fi|
\end{tabular}
\end{center}
%
The definition by |\providecommand| makes sure
that previous definitions are not overwritten.
Further statements |\providecommand{\version}{...}|
can thus be added before the above code to override it.

For the main file, one might add a line
(between |\childdocmain| and the above block)
%
\begin{center}
|%\ifchilddoc\||else\providecommand{\version}{draft}\||fi|
\end{center}
%
which can be uncommented to produce a draft version.
Likewise one can add a line to the very top of a child file
(above the |\childdocof{|\textit{main}|}| directive)
%
\begin{center}
|%\providecommand{\version}{final}|
\end{center}
%
which can be uncommented to produce the final version of this child document.

%%%%%%%%%%%%%%%%%%%%%%%%%%%%%%%%%%%%%%%%%%%%%%%%%%%%%%%%%%%%%%%%%%%%%%%%%%%%%%%%
\subsection{Forwarding}
\label{sec:forward}

Different versions of the main or child documents
using compilation flags as described in \secref{sec:flags}
can be (permanently) stored in different files
for convenient compilation, viewing and distribution.
To this end, the package defines a command
to pass on compilation to a different file:

%%%%%%%%%%%%%%%%%%%%%%%%%%%%%%%%%%%%%%%%
\DescribeMacro{\childdocforward}
The command |\childdocforward| redirects processing to
another source file:
%
\begin{center}
\begin{tabular}{l}
|\input{childdoc.def}|\\
|\childdocforward[|\textit{main}|]{|\textit{dest}|}|\\
\end{tabular}
\end{center}
%
The argument \textit{dest} is the destination file
(without extension).
It should be the main file or one of the child files.
Note that further \textsf{childdoc} directives
such as |\childdocof| and |\childdocforward|
in the indicated file will be processed in this form.
The optional argument \textit{main}
passes on directly to the main file \textit{main}
while pretending to compile the child \textit{dest}.
This form behaves as if \textit{dest}
issues |\childdocof{|\textit{main}|}| right away,
and no further \textsf{childdoc} directives will be processed.

%%%%%%%%%%%%%%%%%%%%%%%%%%%%%%%%%%%%%%%%
\DescribeMacro{\...prefix}
In the alternative form |\childdocforwardprefix|,
%
\begin{center}
\begin{tabular}{l}
|\input{childdoc.def}|\\
|\childdocforwardprefix[|\textit{main}|]{|\textit{prefix}|}{|\textit{dest}|}|
\end{tabular}
\end{center}
%
the destination file is determined by a pattern
depending on the current file:
To make this work, the current file must be called
`{\textit{prefix}\hspace{0.2em}\textit{suffix}}'
with \textit{prefix} matching precisely the argument.
Processing is then passed on to the file
`{\textit{dest}\hspace{0.2em}\textit{suffix}}'.
Surely, the same effect is achieved by
directly specifying the
argument `{\textit{dest}\hspace{0.2em}\textit{suffix}}'
in the first form.
However, that requires to set up a different file
for each child. With the alternative form of the command
all these files can have exactly the same content
which simplifies setting them up and maintaining them.

For example, the following file |draft.tex|
with a compilation flag |\version| as described in \secref{sec:flags}
compiles the main document as a draft:
%
\begin{center}
\begin{tabular}{l}
|\def\version{draft}|\\
|\input{childdoc.def}|\\
|\childdocforward{|\textit{main}|}|
\end{tabular}
\end{center}
%
Likewise, the following files |final|\textit{nn}|.tex|
compile the final version of the child document
|child|\textit{nn}|.tex|:
%
\begin{center}
\begin{tabular}{l}
|\def\version{final}|\\
|\input{childdoc.def}|\\
|\childdocforwardprefix{final}{child}|
\end{tabular}
\end{center}
%

Note that when several versions of a main file and/or of each child file
are to be generated, it may be convenient to set up a |Makefile| or
shell script to automatise the process.

%%%%%%%%%%%%%%%%%%%%%%%%%%%%%%%%%%%%%%%%%%%%%%%%%%%%%%%%%%%%%%%%%%%%%%%%%%%%%%%%
\subsection{Command Line Processing}
\label{sec:commandline}

The effect of redirection files can also be achieved by invoking
the \LaTeX{} compiler with a more elaborate command line.
Most conveniently this should be done as part
of a shell script or a |Makefile|.

When using \textsf{childdoc} in the main file, the following
command lines effectively perform a redirection
(note that depending on the shell being used,
backslashes may have to be doubled: `|\|' $\to$ `|\\|'):
%
\begin{center}
|... -jobname "|\textit{target}|" |\\|"|[\textit{flags}]%
|\input{childdoc.def}\childdocforward[|\textit{main}|]{|\textit{dest}|}"|
\end{center}
%
Here \textit{target} is the name of the output file,
\textit{main} is the name of the main file
and \textit{dest} is the name of the main or child file to be processed
(all filenames without extensions).
The optional argument \textit{main} can be omitted
if \textit{main} matches \textit{dest}.
Optionally, compilation \textit{flags} can be defined via |\def| commands.
This command line makes the \TeX{} engine believe
it is compiling the file \textit{target}
whose content is specified as the latter parameter.
The provided code then forwards the processing to
\textit{main} or \textit{dest} as described in \secref{sec:forward}.

%%%%%%%%%%%%%%%%%%%%%%%%%%%%%%%%%%%%%%%%%%%%%%%%%%%%%%%%%%%%%%%%%%%%%%%%%%%%%%%%
\subsection{Include by Input}
\label{sec:input}

Including child documents by |\include| has some restrictions by design.
Most notably, the content of a child document always occupies
its own set of pages; pages cannot be shared between child documents.
Usually, this behaviour makes perfect sense
because each child document contain an essential part of the document.
However, in some situations it may be desirable to compose
a document from a collection of parts
without having mandatory page breaks between then.
For this case, the package
provides a mechanism to include parts
by |\input| which can also be processed individually.
However, by construction this mechanism
requires manual handling of the content to be output.

%%%%%%%%%%%%%%%%%%%%%%%%%%%%%%%%%%%%%%%%
\DescribeMacro{\ifchilddocmanual}
The main file should be prepared as usual, see \secref{sec:include}.
However, the document body must make a distinction
between processing of an individual part and of the main document, e.g.:
%
\begin{center}
\begin{tabular}{l}
|\ifchilddocmanual|\\
|\input{\childdocname}|\\
|\||else|\\
\textit{document body with }|\input{|\textit{part}|}|\\
|\||fi|
\end{tabular}
\end{center}
%
The conditional |\ifchilddocmanual| is true whenever
a part to be included by |\input| is being compiled,
and the name of the part is stored in |\childdocname|.

%%%%%%%%%%%%%%%%%%%%%%%%%%%%%%%%%%%%%%%%
\DescribeMacro{\childdocby}
Each part to be included by |\input| should start with:
%
\begin{center}
\begin{tabular}{l}
|\input{childdoc.def}|\\
|\childdocby{|\textit{main}|}|\\
\end{tabular}
\end{center}
%
The directive |\childdocby| is similar to |\childdocof|
described in \secref{sec:include},
but the subsequent selection of content must be done manually.
To that end, both |\ifchilddoc| and |\ifchilddocmanual|
will be true upon processing of a part,
and the name of the part is stored in |\childdocname|.
Note that |\jobname| will be set to the filename of the current part
so that each part receives an individual |.aux| file
that does not interfere with the |.aux| file(s) of the main document.
This behaviour can be altered by the alternative form
|\childdocby[*]{|\textit{main}|}| (with a non-empty optional argument)
which uses the |.aux| file of the main document
by setting |\jobname| to \textit{main}.

%%%%%%%%%%%%%%%%%%%%%%%%%%%%%%%%%%%%%%%%%%%%%%%%%%%%%%%%%%%%%%%%%%%%%%%%%%%%%%%%
\subsection{Driver Development}
\label{sec:driver}

The \textsf{childdoc} mechanism can also be use for the development
of definition files such as \LaTeX{} styles or classes.
This case differs from the above setup with multiple parts
included by |\include| in that no |\includeonly| should be invoked.
This can be achieved by starting the include file
(before |\ProvidesPackage|) with:
%
\begin{center}
\begin{tabular}{l}
|\input{childdoc.def}|\\
|\childdocforward{|\textit{main}|}|\\
\end{tabular}
\end{center}
%
or alternatively with:
%
\begin{center}
\begin{tabular}{l}
|\input{childdoc.def}|\\
|\childdocby{|\textit{main}|}|\\
\end{tabular}
\end{center}
%
Both forms have slightly different effects as described above.
The main file is prepared as usual, see \secref{sec:include}.

%%%%%%%%%%%%%%%%%%%%%%%%%%%%%%%%%%%%%%%%%%%%%%%%%%%%%%%%%%%%%%%%%%%%%%%%%%%%%%%%
\subsection{Legacy Detection}
\label{sec:detection}

The directive |\childdocmain| in the main file can detect
whether the complete document or merely a child is to be compiled
even without using the directive |\childdocof|.
This method is deprecated because it is less robust
and there is no compelling reason to use it;
it is merely provided for backward compatibility
and it may be removed in future versions.

If the detection mechanism is to be used,
it is mandatory to correctly specify
the filename of the main file as the argument of |\childdocmain|:
%
\begin{center}
\begin{tabular}{l}
|\input{childdoc.def}|\\
|\childdocmain{|\textit{main}|}|\\
\end{tabular}
\end{center}
%
If |\jobname| does not match the argument \textit{main} of |\childdocmain|,
it is assumed that |\jobname| points to the child file to be compiled.
When using |\childdocmain| with the main file specified as argument,
it suffices to start a child file
with just |\input{|\textit{main}|}|
without loading of the package and using |\childdocof|.
If instead all processing is done
with the appropriate \textsf{childdoc} directives,
the argument of \textit{main} of |\childdocmain| can be empty.

An alternative version of the command line processing described
in \secref{sec:commandline} using the detection mechanism reads:
%
\begin{center}
|... -jobname "|\textit{target}|" "|[\textit{flags}]%
[|\def\jobname{|\textit{dest}|}|]|\input{|\textit{main}|}"|
\end{center}

%%%%%%%%%%%%%%%%%%%%%%%%%%%%%%%%%%%%%%%%%%%%%%%%%%%%%%%%%%%%%%%%%%%%%%%%%%%%%%%%
\subsection{Manual Code}
\label{sec:manual}

In case one cannot be certain whether the definitions file |childdoc.def|
is installed on the target \TeX{} distribution
and one prefers not to ship it,
it is conceivable to paste a few relevant commands into the sources.

To that end, drop all statements |\input{childdoc.def}|
and perform the replacements as outlined below.
Instead of |\childdocmain{|\textit{main}|}| add the following code
to the top of the main file:
%
\begin{center}
\begin{tabular}{l}
|\||ifdefined\childdocname\endinput\||fi\newif\ifchilddoc|\\
|\edef\childdocname{\scantokens\expandafter{\jobname\noexpand}}|\\
|\def\childdocmain{|\textit{main}|}\||ifx\childdocmain\childdocname\||else|\\
|\childdoctrue\includeonly{\childdocname}\let\jobname\childdocmain\||fi|\\
\end{tabular}
\end{center}
%
Instead of |\childdocof{|\textit{main}|}| just include the main file
at the top of each child file:
%
\begin{center}
|\input{|\textit{main}|}|
\end{center}
%
A simple redirection |\childdocforward{|\textit{dest}|}| is achieved by:
%
\begin{center}
|\def\jobname{|\textit{dest}|}\input{\jobname}|
\end{center}
%
The redirection with prefix
|\childdocforwardprefix[|\textit{prefix}|]{|\textit{dest}|}|
is accomplished by:
%
\begin{center}
\begin{tabular}{l}
|{\edef\jobname{\scantokens\expandafter{\jobname\noexpand}}|\\
|\def\redirectjob |\textit{prefix}|#1~~~{\gdef\jobname{|\textit{dest}|#1}}|\\
|\expandafter\redirectjob\jobname~~~}\input{\jobname}|
\end{tabular}
\end{center}

In an alternative approach,
child documents can be compiled by a specific command line
without additional code or specific definitions:
%
\begin{center}
|... -jobname "|\textit{target}|" "|[\textit{flags}]%
|\includeonly{|\textit{dest}|}\input{|\textit{main}|}"|
\end{center}
%

%%%%%%%%%%%%%%%%%%%%%%%%%%%%%%%%%%%%%%%%%%%%%%%%%%%%%%%%%%%%%%%%%%%%%%%%%%%%%%%%
%%%%%%%%%%%%%%%%%%%%%%%%%%%%%%%%%%%%%%%%%%%%%%%%%%%%%%%%%%%%%%%%%%%%%%%%%%%%%%%%
\section{Information}

%%%%%%%%%%%%%%%%%%%%%%%%%%%%%%%%%%%%%%%%%%%%%%%%%%%%%%%%%%%%%%%%%%%%%%%%%%%%%%%%
\subsection{Copyright}

Copyright \copyright{} 2017--2018 Niklas Beisert

This work may be distributed and/or modified under the
conditions of the \LaTeX{} Project Public License, either version 1.3
of this license or (at your option) any later version.
The latest version of this license is in
  \url{http://www.latex-project.org/lppl.txt}
and version 1.3 or later is part of all distributions of \LaTeX{}
version 2005/12/01 or later.

This work has the LPPL maintenance status `maintained'.

The Current Maintainer of this work is Niklas Beisert.

This work consists of the files |README.txt|, |childdoc.ins| and |childdoc.dtx|
as well as the derived files |childdoc.def|, |cdocsamp.tex|
with |cdocsch1.tex|, |cdocsch2.tex|, |cdocspt3.tex|, |cdocspt4.tex|,
|cdocsdrf.tex|, |cdocsfn1.tex|, |cdocsfn2.tex|
as well as |childdoc.pdf|.

%%%%%%%%%%%%%%%%%%%%%%%%%%%%%%%%%%%%%%%%%%%%%%%%%%%%%%%%%%%%%%%%%%%%%%%%%%%%%%%%
\subsection{Files and Installation}

The package consists of the files:
%
\begin{center}
\begin{tabular}{ll}
    |README.txt|   & readme file \\
    |childdoc.ins| & installation file \\
    |childdoc.dtx| & source file \\
    |childdoc.def| & definition file \\
    |cdocsamp.tex| & sample main file \\
    |cdocsch1.tex| & sample include file \\
    |cdocsch2.tex| & sample include file \\
    |cdocspt3.tex| & sample part file \\
    |cdocspt4.tex| & sample part file \\
    |cdocsdrf.tex| & sample redirection file \\
    |cdocsfn1.tex| & sample redirection file \\
    |cdocsfn2.tex| & sample redirection file \\
    |childdoc.pdf| & manual
\end{tabular}
\end{center}
%
The distribution consists of the files
|README.txt|, |childdoc.ins| and |childdoc.dtx|.
%
\begin{itemize}
\item
Run (pdf)\LaTeX{} on |childdoc.dtx|
to compile the manual |childdoc.pdf| (this file).
\item
Run \LaTeX{} on |childdoc.ins| to create the definitions file |childdoc.def|
and the sample |cdocsamp.tex| with include files
|cdocsch1.tex|, |cdocsch2.tex|, |cdocspt3.tex|, |cdocspt4.tex|,
|cdocsdrf.tex|, |cdocsfn1.tex|, |cdocsfn2.tex|.
Then copy the file |childdoc.def| to an appropriate directory of your \LaTeX{}
distribution, e.g.\ \textit{texmf-root}|/tex/latex/childdoc|.
\end{itemize}

%%%%%%%%%%%%%%%%%%%%%%%%%%%%%%%%%%%%%%%%%%%%%%%%%%%%%%%%%%%%%%%%%%%%%%%%%%%%%%%%
\subsection{Related CTAN Packages}

There are several other packages which offer a similar functionality:
%
\begin{itemize}
\item
The packages
\href{http://ctan.org/pkg/docmute}{\textsf{docmute}},
\href{http://ctan.org/pkg/includex}{\textsf{includex}} and
\href{http://ctan.org/pkg/standalone}{\textsf{standalone}}
provide commands to include only the document body of
a child file thus allowing both files to be compiled individually.
\item
The packages \href{http://ctan.org/pkg/subdocs}{\textsf{subdocs}}
and \href{http://ctan.org/pkg/subfiles}{\textsf{subfiles}}
provide structures in which the main and child documents can be
encapsulated and allowing them to be compiled individually.
The inclusion mechanism is different from the conventional |\include|.
\item
The package \href{http://ctan.org/pkg/combine}{\textsf{combine}}
is an elaborate solution to combine several documents into one.
\end{itemize}
%
See also the CTAN topic \href{http://ctan.org/topic/subdocs}{\textsf{subdocs}}
for further related packages.
The present package differs from the above solutions in that
a document structure constructed with the conventional |\include| mechanism
just needs two extra commands at the top of every file
such that all constituent files can be compiled individually.

%%%%%%%%%%%%%%%%%%%%%%%%%%%%%%%%%%%%%%%%%%%%%%%%%%%%%%%%%%%%%%%%%%%%%%%%%%%%%%%%
%\subsection{Feature Suggestions}
%
%The following is a list of features which may be useful for future
%versions of this package:
%%
%\begin{itemize}
%\item
%\ldots
%\end{itemize}

%%%%%%%%%%%%%%%%%%%%%%%%%%%%%%%%%%%%%%%%%%%%%%%%%%%%%%%%%%%%%%%%%%%%%%%%%%%%%%%%
\subsection{Revision History}

%%%%%%%%%%%%%%%%%%%%%%%%%%%%%%%%%%%%%%%%
\paragraph{v2.0:} 2018/12/30

\begin{itemize}
\item
immediate forward processing
\item
added |\childdocby| mechanism
\item
manual restructured
\end{itemize}

%%%%%%%%%%%%%%%%%%%%%%%%%%%%%%%%%%%%%%%%
\paragraph{v1.6:} 2018/01/17

\begin{itemize}
\item
application for development of include files
\item
corrections to manual
\end{itemize}

%%%%%%%%%%%%%%%%%%%%%%%%%%%%%%%%%%%%%%%%
\paragraph{v1.5:} 2017/05/21

\begin{itemize}
\item
more complete structuring introduced
\item
|\childdocof| introduced
\item
|\childdoc| renamed to |\childdocmain|
\item
|\childredirect| renamed to |\childdocforward| and |\childdocforwardprefix|
and functionality expanded
\end{itemize}

%%%%%%%%%%%%%%%%%%%%%%%%%%%%%%%%%%%%%%%%
\paragraph{v1.0:} 2017/04/27

\begin{itemize}
\item
manual and install package
\item
first version published on CTAN
\end{itemize}

%%%%%%%%%%%%%%%%%%%%%%%%%%%%%%%%%%%%%%%%
\paragraph{v0.6:} 2017/04/26

\begin{itemize}
\item
redirection mechanism added
\end{itemize}

%%%%%%%%%%%%%%%%%%%%%%%%%%%%%%%%%%%%%%%%
\paragraph{v0.5:} 2017/04/26

\begin{itemize}
\item
functionality in definition file
\end{itemize}


%%%%%%%%%%%%%%%%%%%%%%%%%%%%%%%%%%%%%%%%%%%%%%%%%%%%%%%%%%%%%%%%%%%%%%%%%%%%%%%%
%%%%%%%%%%%%%%%%%%%%%%%%%%%%%%%%%%%%%%%%%%%%%%%%%%%%%%%%%%%%%%%%%%%%%%%%%%%%%%%%
%%%%%%%%%%%%%%%%%%%%%%%%%%%%%%%%%%%%%%%%%%%%%%%%%%%%%%%%%%%%%%%%%%%%%%%%%%%%%%%%
\appendix

\settowidth\MacroIndent{\rmfamily\scriptsize 000\ }

 \DocInput{childdoc.dtx}

\end{document}
%</driver>
% \fi
%
% %%%%%%%%%%%%%%%%%%%%%%%%%%%%%%%%%%%%%%%%%%%%%%%%%%%%%%%%%%%%%%%%%%%%%%%%%%%%%%
% %%%%%%%%%%%%%%%%%%%%%%%%%%%%%%%%%%%%%%%%%%%%%%%%%%%%%%%%%%%%%%%%%%%%%%%%%%%%%%
% \section{Sample}
%\iffalse
%<*samplemain>
%\fi
%
% The following presents a sample document
% with two chapters, two parts, a title page,
% a compile flag as well as three forwarding files to set the flag.
% It consists of eight |.tex| files:
% \begin{center}
% \begin{tabular}{ll}
% |cdocsamp.tex|&main file\\
% |cdocsch1.tex|&include file for chapter 1\\
% |cdocsch2.tex|&include file for chapter 2\\
% |cdocspt3.tex|&include file for part 3\\
% |cdocspt4.tex|&include file for part 4\\
% |cdocsdrf.tex|&forwarding file for main file in draft mode\\
% |cdocsfi1.tex|&forwarding file for final version of chapter 1\\
% |cdocsfi2.tex|&forwarding file for final version of chapter 2\\
% \end{tabular}
% \end{center}
% Each of the eight files can be compiled directly by the \LaTeX{} compiler.
%
% %%%%%%%%%%%%%%%%%%%%%%%%%%%%%%%%%%%%%%
% \paragraph{Main File.}
%
% The main file is called |cdocsamp.tex|.
%
% Load the \textsf{childdoc} definitions and
% declare the filename for the main document:
%    \begin{macrocode}
\input{childdoc.def}
\childdocmain{}
%    \end{macrocode}

% Optional override for |\version| flag:
%    \begin{macrocode}
%%\ifchilddoc\else\providecommand{\version}{draft}\fi
%    \end{macrocode}

% Define the default values for the |\version| flag
% (|final| for the main file and |draft| for childs):
%    \begin{macrocode}
\ifchilddoc
\providecommand{\version}{draft}
\else
\providecommand{\version}{final}
\fi
%    \end{macrocode}

% Load the standard document class:
%    \begin{macrocode}
\documentclass[12pt]{article}
%    \end{macrocode}

% Start the document body:
%    \begin{macrocode}
\begin{document}
%    \end{macrocode}

% Declare a title page.
% Print title, part of document being processed and version flag:
%    \begin{macrocode}
\addtocounter{page}{-1}
\begin{center}
{\LARGE\bfseries{}childdoc example\par}
\vspace{1cm}
\ifchilddoc
\ifchilddocmanual part\else chapter\fi:
`\childdocname' of `\childdocjob'\par
\else
main document: `\childdocjob'\par
\fi
version: \version\par
\end{center}
\newpage
%    \end{macrocode}

% Manually include selected file,
% otherwise process as usual:
%    \begin{macrocode}
\ifchilddocmanual
\section*{part `\childdocname'}
\input{\childdocname}
\else
%    \end{macrocode}

% Include the two chapters:
%    \begin{macrocode}
\include{cdocsch1}
\include{cdocsch2}
%    \end{macrocode}

% Include the two parts unless only chapters should be displayed:
%    \begin{macrocode}
\ifchilddoc\else
\section{part three}
\input{cdocspt3}
\section{part four}
\input{cdocspt4}
\fi
%    \end{macrocode}

% Process as usual until here:
%    \begin{macrocode}
\fi
%    \end{macrocode}

% End of document body:
%    \begin{macrocode}
\end{document}
%    \end{macrocode}
%\iffalse
%</samplemain>
%\fi
%
% %%%%%%%%%%%%%%%%%%%%%%%%%%%%%%%%%%%%%%
% \paragraph{Chapter Include Files.}
%
% The include files are called |cdocsch1.tex| and |cdocsch2.tex|.
%
%\iffalse
%<*samplechap1|samplechap2>
%\fi

% Optional override for |\version| flag:
%    \begin{macrocode}
%%\providecommand{\version}{final}
%    \end{macrocode}

% Include the main document:
%    \begin{macrocode}
\input{childdoc.def}
\childdocof{cdocsamp}
%    \end{macrocode}

%\iffalse
%</samplechap1|samplechap2>
%\fi
%
%\iffalse
%<*samplechap1>
%\fi
% Some text for chapter 1:
%    \begin{macrocode}
\section{one}
some text in chapter one
%    \end{macrocode}

%\iffalse
%</samplechap1>
%\fi
% Some text for chapter 2:
%\iffalse
%<*samplechap2>
%\fi
%    \begin{macrocode}
\section{two}
more text in chapter two
%    \end{macrocode}

%\iffalse
%</samplechap2>
%\fi
%
% %%%%%%%%%%%%%%%%%%%%%%%%%%%%%%%%%%%%%%
% \paragraph{Part Include Files.}
%
% The include files are called |cdocspt3.tex| and |cdocspt4.tex|.
%
%\iffalse
%<*samplepart3|samplepart4>
%\fi

% Optional override for |\version| flag:
%    \begin{macrocode}
%%\providecommand{\version}{final}
%    \end{macrocode}

% Include the main document:
%    \begin{macrocode}
\input{childdoc.def}
\childdocby{cdocsamp}
%    \end{macrocode}

%\iffalse
%</samplepart3|samplepart4>
%\fi
%
%\iffalse
%<*samplepart3>
%\fi
% Some text for part 3:
%    \begin{macrocode}
some text in part three
%    \end{macrocode}

%\iffalse
%</samplepart3>
%\fi
% Some text for part 4:
%\iffalse
%<*samplepart4>
%\fi
%    \begin{macrocode}
more text in part four
%    \end{macrocode}

%\iffalse
%</samplepart4>
%\fi
%
% %%%%%%%%%%%%%%%%%%%%%%%%%%%%%%%%%%%%%%
% \paragraph{Forwarding for a Complete Draft.}
%
% The following forwarding file |cdocsdrf.tex|
% compiles the main document in draft mode:
%\iffalse
%<*sampledraft>
%\fi
%    \begin{macrocode}
\def\version{draft}
\input{childdoc.def}
\childdocforward{cdocsamp}
%    \end{macrocode}

%\iffalse
%</sampledraft>
%\fi
%
% %%%%%%%%%%%%%%%%%%%%%%%%%%%%%%%%%%%%%%
% \paragraph{Forwarding for Final Version of the Chapters.}
%
% The following forwarding files |cdocsfn1.tex| and |cdocsfn2.tex|
% (with identical content)
% compile the final versions of the child documents
% |cdocsch1.tex| and |cdocsch2.tex|, respectively:
%\iffalse
%<*samplefinal>
%\fi
%    \begin{macrocode}
\def\version{final}
\input{childdoc.def}
\childdocforwardprefix[cdocsamp]{cdocsfn}{cdocsch}
%    \end{macrocode}

%\iffalse
%</samplefinal>
%\fi
%
% %%%%%%%%%%%%%%%%%%%%%%%%%%%%%%%%%%%%%%
% \paragraph{Command Line Processing.}
%
% The following three command lines generate the output files
% |cdocscld|, |cdocscl1| and |cdocscl2|
% which should be identical to
% |cdocsdrf|, |cdocsch1| and |cdocsfn2|, respectively:
% \begin{center}
% \begin{tabular}{l}
% |latex -jobname cdocscld \|\\
% |  "\def\version{draft}\input{childdoc.def}\childdocforward{cdocsamp}"|\\
% |latex -jobname cdocscl1 \|\\
% |  "\input{childdoc.def}\childdocforward[cdocsamp]{cdocsch1}"|\\
% |latex -jobname cdocscl2 \|\\
% |  "\def\version{final}\input{childdoc.def}\childdocforward{cdocsch2}"|
% \end{tabular}
% \end{center}
% Note that the trailing backslash on each first line
% merely continues the input to the second line
% (for convenient cut ant paste).
% Furthermore, the command |latex| can be replaced by any
% of its alternative versions such as |pdflatex|.
%
% %%%%%%%%%%%%%%%%%%%%%%%%%%%%%%%%%%%%%%%%%%%%%%%%%%%%%%%%%%%%%%%%%%%%%%%%%%%%%%
% %%%%%%%%%%%%%%%%%%%%%%%%%%%%%%%%%%%%%%%%%%%%%%%%%%%%%%%%%%%%%%%%%%%%%%%%%%%%%%
% \section{Implementation}
%\iffalse
%<*package>
%\fi
%
% This section describes the definitions file |childdoc.def|.

% The definitions cannot be loaded using |\usepackage| or |\RequirePackage|
% which has a mechanism to prevent loading a style file more than once.
% When loading the definitions by means of |\input|
% multiple instances have to be prevented manually:
%\iffalse
%This code needs to be before the `\ProvidesFile' directive
%which is defined at the beginning of this file.
%Therefore it is also placed there and commented out here.
%</package>
%<*discard>
%\fi
%    \begin{macrocode}
\ifdefined\childdocmain\endinput\fi
%    \end{macrocode}
%\iffalse
%</discard>
%<*package>
%\fi
%
% \macro{\ifchilddoc}
% \macro{\ifchilddocmanual}
% The conditional |\ifchilddoc| tells whether a
% child (true) or main (false) document is being compiled.
% The conditional |\ifchilddocmanual| tells whether
% the |\includeonly| mechanism is used (false) or
% the selection of child files must be performed manually (true).
% The definitions initialise to false:
%    \begin{macrocode}
\newif\ifchilddoc
\newif\ifchilddocmanual
%    \end{macrocode}

% \macro{\childdocname}
% \macro{\childdocjob}
% The macro |\childdocname| stores the name of the main document
% to be compiled. The macro |\childdocjob| stores the name of
% the document on which the \LaTeX{} compiler was originally invoked.
% The content of |\jobname| cannot be compared
% to filenames specified in the source due to different catcodes.
% The following code rescans |\jobname|, stores the result
% in |\childdocname| and saves a copy in |\childdocjob|:
%    \begin{macrocode}
\edef\childdocname{\scantokens\expandafter{\jobname\noexpand}}
\let\childdocjob\childdocname
%    \end{macrocode}

% \macro{\childdocdisable}
% The macro |\childdocdisable| prevents the main file
% from being processed more than once.
% At this stage, the main document command |\childdocmain|
% is assumed to be called once again where it should do nothing.
% Any subsequent call to it should prevent
% a secondary processing of the main document
% It overwrites the forwarding commands
% |\childdocof| and |\childdocforward|
% with empty macros to prevent further inclusions of the main document:
%    \begin{macrocode}
\newcommand{\childdocdisable}
{
  \renewcommand{\childdocmain}[1]{\renewcommand{\childdocmain}[1]{\endinput}}
  \renewcommand{\childdocof}[1]{}
  \renewcommand{\childdocby}[2][]{}
  \renewcommand{\childdocforward}[2][]{}
  \renewcommand{\childdocdisable}{}
}
%    \end{macrocode}

% \macro{\childdocmain}
% The macro |\childdocmain| is to be called at the top of the main file
% with nothing or the main filename (without extension) as argument.
% First, it breaks loops.
% If the argument is not empty and does not match |\childdocname|
% (which is set by the first inclusion of |childdoc.def|),
% |\ifchilddoc| is set to true, |\includeonly| is applied to the child file
% and |\jobname| is set to the main file
% (for proper handling of |.aux| files):
%    \begin{macrocode}
\newcommand{\childdocmain}[1]
{
  \childdocdisable\childdocmain{}
  \if?#1?\else
    \begingroup
      \def\childdoctmp{#1}
      \ifx\childdoctmp\childdocname
        \def\childdoctmp{}
      \else
        \def\childdoctmp
        {
          \childdoctrue
          \includeonly{\childdocname}
          \def\childdocjob{#1}
          \def\jobname{#1}
        }
      \fi
      \expandafter
    \endgroup
    \childdoctmp
  \fi
}
%    \end{macrocode}

% \macro{\childdocof}
% The command |\childdocof| redirects
% compilation to the main file |#1|.
%    \begin{macrocode}
\newcommand{\childdocof}[1]
{
  \childdocdisable
  \childdoctrue
  \includeonly{\childdocname}
  \def\jobname{#1}
  \def\childdocjob{#1}
  \input{#1}
}
%    \end{macrocode}

% \macro{\childdocby}
% The command |\childdocby| ....
%    \begin{macrocode}
\newcommand{\childdocby}[2][]
{
  \childdocdisable
  \childdoctrue
  \childdocmanualtrue
  \if?#1?\else
    \def\jobname{#2}
  \fi
  \def\childdocjob{#2}
  \input{#2}
  \endinput
}
%    \end{macrocode}

% \macro{\childdocforward}
% The command |\childdocforward| redirects
% compilation to the main file or
% (if the optional argument is given) a child file.
% Parameters are set as if the main file
% or a child file starting with |\childdocof| was compiled.
% Then compilation is handed over to the main file:
%    \begin{macrocode}
\newcommand{\childdocforward}[2][]
{
  \begingroup
    \if?#1?
      \def\childdoctmp
      {
        \def\childdocname{#2}
        \def\childdocjob{#2}
        \def\jobname{#2}
        \input{#2}
        \endinput
      }
    \else
      \def\childdoctmp
      {
        \childdocdisable
        \def\childdocname{#2}
        \childdoctrue
        \includeonly{#2}
        \def\childdocjob{#1}
        \def\jobname{#1}
        \input{#1}
        \endinput
      }
    \fi
    \expandafter
  \endgroup
  \childdoctmp
}
%    \end{macrocode}

% \macro{\childdocforwardprefix}
% The command |\childdocforwardprefix| redirects
% compilation to the main or a child file by means of a pattern.
% The prefix |#1| in the current filename is replaced by |#2|
% and the suffix of the current filename is kept
% (it is assumed that the filename does not contain the substring `|~~~|'
% which is used as a delimiter).
% Compilation is handed over to the new file by |\childdocforward|:
%    \begin{macrocode}
\newcommand{\childdocforwardprefix}[3][]
{
  \begingroup
    \def\childdocextract #2##1~~~{\def\childdoctmp{\childdocforward[#1]{#3##1}}}
    \expandafter\childdocextract\childdocname~~~
    \expandafter
  \endgroup
  \childdoctmp
}
%    \end{macrocode}

% \macro{\childdoc}
% The deprecated macro |\childdoc| is a legacy version of |\childdocmain|:
%    \begin{macrocode}
\newcommand{\childdoc}{\childdocmain}
%    \end{macrocode}

% \macro{\childdocredirect}
% The deprecated macro |\childdocredirect| is a legacy version
% of |\childdocforward| and |\childdocforwardprefix|:
%    \begin{macrocode}
\newcommand{\childdocredirect}[2][]
{
  \begingroup
    \if?#1?
      \def\childdoctmp{\childdocforward{#2}}
    \else
      \def\childdoctmp{\childdocforwardprefix{#1}{#2}}
    \fi
    \expandafter
  \endgroup
  \childdoctmp
}
%    \end{macrocode}

%\iffalse
%</package>
%\fi
%
\endinput
|
and perform the replacements as outlined below.
Instead of |\childdocmain{|\textit{main}|}| add the following code
to the top of the main file:
%
\begin{center}
\begin{tabular}{l}
|\||ifdefined\childdocname\endinput\||fi\newif\ifchilddoc|\\
|\edef\childdocname{\scantokens\expandafter{\jobname\noexpand}}|\\
|\def\childdocmain{|\textit{main}|}\||ifx\childdocmain\childdocname\||else|\\
|\childdoctrue\includeonly{\childdocname}\let\jobname\childdocmain\||fi|\\
\end{tabular}
\end{center}
%
Instead of |\childdocof{|\textit{main}|}| just include the main file
at the top of each child file:
%
\begin{center}
|\input{|\textit{main}|}|
\end{center}
%
A simple redirection |\childdocforward{|\textit{dest}|}| is achieved by:
%
\begin{center}
|\def\jobname{|\textit{dest}|}\input{\jobname}|
\end{center}
%
The redirection with prefix
|\childdocforwardprefix[|\textit{prefix}|]{|\textit{dest}|}|
is accomplished by:
%
\begin{center}
\begin{tabular}{l}
|{\edef\jobname{\scantokens\expandafter{\jobname\noexpand}}|\\
|\def\redirectjob |\textit{prefix}|#1~~~{\gdef\jobname{|\textit{dest}|#1}}|\\
|\expandafter\redirectjob\jobname~~~}\input{\jobname}|
\end{tabular}
\end{center}

In an alternative approach,
child documents can be compiled by a specific command line
without additional code or specific definitions:
%
\begin{center}
|... -jobname "|\textit{target}|" "|[\textit{flags}]%
|\includeonly{|\textit{dest}|}\input{|\textit{main}|}"|
\end{center}
%

%%%%%%%%%%%%%%%%%%%%%%%%%%%%%%%%%%%%%%%%%%%%%%%%%%%%%%%%%%%%%%%%%%%%%%%%%%%%%%%%
%%%%%%%%%%%%%%%%%%%%%%%%%%%%%%%%%%%%%%%%%%%%%%%%%%%%%%%%%%%%%%%%%%%%%%%%%%%%%%%%
\section{Information}

%%%%%%%%%%%%%%%%%%%%%%%%%%%%%%%%%%%%%%%%%%%%%%%%%%%%%%%%%%%%%%%%%%%%%%%%%%%%%%%%
\subsection{Copyright}

Copyright \copyright{} 2017--2018 Niklas Beisert

This work may be distributed and/or modified under the
conditions of the \LaTeX{} Project Public License, either version 1.3
of this license or (at your option) any later version.
The latest version of this license is in
  \url{http://www.latex-project.org/lppl.txt}
and version 1.3 or later is part of all distributions of \LaTeX{}
version 2005/12/01 or later.

This work has the LPPL maintenance status `maintained'.

The Current Maintainer of this work is Niklas Beisert.

This work consists of the files |README.txt|, |childdoc.ins| and |childdoc.dtx|
as well as the derived files |childdoc.def|, |cdocsamp.tex|
with |cdocsch1.tex|, |cdocsch2.tex|, |cdocspt3.tex|, |cdocspt4.tex|,
|cdocsdrf.tex|, |cdocsfn1.tex|, |cdocsfn2.tex|
as well as |childdoc.pdf|.

%%%%%%%%%%%%%%%%%%%%%%%%%%%%%%%%%%%%%%%%%%%%%%%%%%%%%%%%%%%%%%%%%%%%%%%%%%%%%%%%
\subsection{Files and Installation}

The package consists of the files:
%
\begin{center}
\begin{tabular}{ll}
    |README.txt|   & readme file \\
    |childdoc.ins| & installation file \\
    |childdoc.dtx| & source file \\
    |childdoc.def| & definition file \\
    |cdocsamp.tex| & sample main file \\
    |cdocsch1.tex| & sample include file \\
    |cdocsch2.tex| & sample include file \\
    |cdocspt3.tex| & sample part file \\
    |cdocspt4.tex| & sample part file \\
    |cdocsdrf.tex| & sample redirection file \\
    |cdocsfn1.tex| & sample redirection file \\
    |cdocsfn2.tex| & sample redirection file \\
    |childdoc.pdf| & manual
\end{tabular}
\end{center}
%
The distribution consists of the files
|README.txt|, |childdoc.ins| and |childdoc.dtx|.
%
\begin{itemize}
\item
Run (pdf)\LaTeX{} on |childdoc.dtx|
to compile the manual |childdoc.pdf| (this file).
\item
Run \LaTeX{} on |childdoc.ins| to create the definitions file |childdoc.def|
and the sample |cdocsamp.tex| with include files
|cdocsch1.tex|, |cdocsch2.tex|, |cdocspt3.tex|, |cdocspt4.tex|,
|cdocsdrf.tex|, |cdocsfn1.tex|, |cdocsfn2.tex|.
Then copy the file |childdoc.def| to an appropriate directory of your \LaTeX{}
distribution, e.g.\ \textit{texmf-root}|/tex/latex/childdoc|.
\end{itemize}

%%%%%%%%%%%%%%%%%%%%%%%%%%%%%%%%%%%%%%%%%%%%%%%%%%%%%%%%%%%%%%%%%%%%%%%%%%%%%%%%
\subsection{Related CTAN Packages}

There are several other packages which offer a similar functionality:
%
\begin{itemize}
\item
The packages
\href{http://ctan.org/pkg/docmute}{\textsf{docmute}},
\href{http://ctan.org/pkg/includex}{\textsf{includex}} and
\href{http://ctan.org/pkg/standalone}{\textsf{standalone}}
provide commands to include only the document body of
a child file thus allowing both files to be compiled individually.
\item
The packages \href{http://ctan.org/pkg/subdocs}{\textsf{subdocs}}
and \href{http://ctan.org/pkg/subfiles}{\textsf{subfiles}}
provide structures in which the main and child documents can be
encapsulated and allowing them to be compiled individually.
The inclusion mechanism is different from the conventional |\include|.
\item
The package \href{http://ctan.org/pkg/combine}{\textsf{combine}}
is an elaborate solution to combine several documents into one.
\end{itemize}
%
See also the CTAN topic \href{http://ctan.org/topic/subdocs}{\textsf{subdocs}}
for further related packages.
The present package differs from the above solutions in that
a document structure constructed with the conventional |\include| mechanism
just needs two extra commands at the top of every file
such that all constituent files can be compiled individually.

%%%%%%%%%%%%%%%%%%%%%%%%%%%%%%%%%%%%%%%%%%%%%%%%%%%%%%%%%%%%%%%%%%%%%%%%%%%%%%%%
%\subsection{Feature Suggestions}
%
%The following is a list of features which may be useful for future
%versions of this package:
%%
%\begin{itemize}
%\item
%\ldots
%\end{itemize}

%%%%%%%%%%%%%%%%%%%%%%%%%%%%%%%%%%%%%%%%%%%%%%%%%%%%%%%%%%%%%%%%%%%%%%%%%%%%%%%%
\subsection{Revision History}

%%%%%%%%%%%%%%%%%%%%%%%%%%%%%%%%%%%%%%%%
\paragraph{v2.0:} 2018/12/30

\begin{itemize}
\item
immediate forward processing
\item
added |\childdocby| mechanism
\item
manual restructured
\end{itemize}

%%%%%%%%%%%%%%%%%%%%%%%%%%%%%%%%%%%%%%%%
\paragraph{v1.6:} 2018/01/17

\begin{itemize}
\item
application for development of include files
\item
corrections to manual
\end{itemize}

%%%%%%%%%%%%%%%%%%%%%%%%%%%%%%%%%%%%%%%%
\paragraph{v1.5:} 2017/05/21

\begin{itemize}
\item
more complete structuring introduced
\item
|\childdocof| introduced
\item
|\childdoc| renamed to |\childdocmain|
\item
|\childredirect| renamed to |\childdocforward| and |\childdocforwardprefix|
and functionality expanded
\end{itemize}

%%%%%%%%%%%%%%%%%%%%%%%%%%%%%%%%%%%%%%%%
\paragraph{v1.0:} 2017/04/27

\begin{itemize}
\item
manual and install package
\item
first version published on CTAN
\end{itemize}

%%%%%%%%%%%%%%%%%%%%%%%%%%%%%%%%%%%%%%%%
\paragraph{v0.6:} 2017/04/26

\begin{itemize}
\item
redirection mechanism added
\end{itemize}

%%%%%%%%%%%%%%%%%%%%%%%%%%%%%%%%%%%%%%%%
\paragraph{v0.5:} 2017/04/26

\begin{itemize}
\item
functionality in definition file
\end{itemize}


%%%%%%%%%%%%%%%%%%%%%%%%%%%%%%%%%%%%%%%%%%%%%%%%%%%%%%%%%%%%%%%%%%%%%%%%%%%%%%%%
%%%%%%%%%%%%%%%%%%%%%%%%%%%%%%%%%%%%%%%%%%%%%%%%%%%%%%%%%%%%%%%%%%%%%%%%%%%%%%%%
%%%%%%%%%%%%%%%%%%%%%%%%%%%%%%%%%%%%%%%%%%%%%%%%%%%%%%%%%%%%%%%%%%%%%%%%%%%%%%%%
\appendix

\settowidth\MacroIndent{\rmfamily\scriptsize 000\ }

 \DocInput{childdoc.dtx}

\end{document}
%</driver>
% \fi
%
% %%%%%%%%%%%%%%%%%%%%%%%%%%%%%%%%%%%%%%%%%%%%%%%%%%%%%%%%%%%%%%%%%%%%%%%%%%%%%%
% %%%%%%%%%%%%%%%%%%%%%%%%%%%%%%%%%%%%%%%%%%%%%%%%%%%%%%%%%%%%%%%%%%%%%%%%%%%%%%
% \section{Sample}
%\iffalse
%<*samplemain>
%\fi
%
% The following presents a sample document
% with two chapters, two parts, a title page,
% a compile flag as well as three forwarding files to set the flag.
% It consists of eight |.tex| files:
% \begin{center}
% \begin{tabular}{ll}
% |cdocsamp.tex|&main file\\
% |cdocsch1.tex|&include file for chapter 1\\
% |cdocsch2.tex|&include file for chapter 2\\
% |cdocspt3.tex|&include file for part 3\\
% |cdocspt4.tex|&include file for part 4\\
% |cdocsdrf.tex|&forwarding file for main file in draft mode\\
% |cdocsfi1.tex|&forwarding file for final version of chapter 1\\
% |cdocsfi2.tex|&forwarding file for final version of chapter 2\\
% \end{tabular}
% \end{center}
% Each of the eight files can be compiled directly by the \LaTeX{} compiler.
%
% %%%%%%%%%%%%%%%%%%%%%%%%%%%%%%%%%%%%%%
% \paragraph{Main File.}
%
% The main file is called |cdocsamp.tex|.
%
% Load the \textsf{childdoc} definitions and
% declare the filename for the main document:
%    \begin{macrocode}
% \iffalse
%
% childdoc.dtx Copyright (C) 2017-2018 Niklas Beisert
%
% This work may be distributed and/or modified under the
% conditions of the LaTeX Project Public License, either version 1.3
% of this license or (at your option) any later version.
% The latest version of this license is in
%   http://www.latex-project.org/lppl.txt
% and version 1.3 or later is part of all distributions of LaTeX
% version 2005/12/01 or later.
%
% This work has the LPPL maintenance status `maintained'.
%
% The Current Maintainer of this work is Niklas Beisert.
%
% This work consists of the files childdoc.dtx and childdoc.ins
% and the derived files childdoc.def and cdocsamp.tex with
% cdocsch1.tex, cdocsch2.tex, cdocsdrf.tex, cdocsfn1.tex, cdocsfn2.tex.
%
%<package>\ifdefined\childdocmain\endinput\fi
%<package>\ProvidesFile{childdoc.def}[2018/12/30 v2.0 child document driver]
%<samplemain>\ProvidesFile{cdocsamp.tex}[2018/12/30 v2.0 sample for childdoc]
%<*driver>
%\ProvidesFile{childdoc.drv}[2018/12/30 v2.0 childdoc reference manual file]
\PassOptionsToClass{10pt,a4paper}{article}
\documentclass{ltxdoc}

\usepackage[margin=35mm]{geometry}
\usepackage{hyperref}
\usepackage{hyperxmp}
\usepackage[usenames]{color}

\hypersetup{colorlinks=true}
\hypersetup{pdfstartview=FitH}
\hypersetup{pdfpagemode=UseNone}
\hypersetup{pdfsource={}}
\hypersetup{pdflang={en-UK}}
\hypersetup{pdfcopyright={Copyright 2017-2018 Niklas Beisert.
  This work may be distributed and/or modified under the
  conditions of the LaTeX Project Public License, either version 1.3
  of this license or (at your option) any later version.}}
\hypersetup{pdflicenseurl={http://www.latex-project.org/lppl.txt}}
\hypersetup{pdfcontactaddress={ETH Zurich, ITP, HIT K,
  Wolfgang-Pauli-Strasse 27}}
\hypersetup{pdfcontactpostcode={8093}}
\hypersetup{pdfcontactcity={Zurich}}
\hypersetup{pdfcontactcountry={Switzerland}}
\hypersetup{pdfcontactemail={nbeisert@itp.phys.ethz.ch}}
\hypersetup{pdfcontacturl={http://people.phys.ethz.ch/\xmptilde nbeisert/}}

\newcommand{\secref}[1]{\hyperref[#1]{section \ref*{#1}}}

\parskip1ex
\parindent0pt
\let\olditemize\itemize
\def\itemize{\olditemize\parskip0pt}

\begin{document}

\title{The \textsf{childdoc} Package}
\hypersetup{pdftitle={The childdoc Package}}
\author{Niklas Beisert\\[2ex]
  Institut f\"ur Theoretische Physik\\
  Eidgen\"ossische Technische Hochschule Z\"urich\\
  Wolfgang-Pauli-Strasse 27, 8093 Z\"urich, Switzerland\\[1ex]
  \href{mailto:nbeisert@itp.phys.ethz.ch}
  {\texttt{nbeisert@itp.phys.ethz.ch}}}
\hypersetup{pdfauthor={Niklas Beisert}}
\hypersetup{pdfsubject={Manual for the LaTeX2e Package childdoc}}
\date{30 December 2018, \textsf{v2.0}}
\maketitle

\begin{abstract}\noindent
\textsf{childdoc} is a \LaTeXe{} package
that enables the direct compilation
of document sections included by |\include|
to individual files.
\end{abstract}

\begingroup
\parskip0ex
\tableofcontents
\endgroup

%%%%%%%%%%%%%%%%%%%%%%%%%%%%%%%%%%%%%%%%%%%%%%%%%%%%%%%%%%%%%%%%%%%%%%%%%%%%%%%%
%%%%%%%%%%%%%%%%%%%%%%%%%%%%%%%%%%%%%%%%%%%%%%%%%%%%%%%%%%%%%%%%%%%%%%%%%%%%%%%%
\section{Introduction}

\LaTeX{} provides a mechanism to structure a large document (such as a book)
into a main file and several child files (containing the chapters)
using the |\include| command.
This mechanism is beneficial for documents
which span hundreds of pages in order to
make the source file(s) more manageable.
Moreover, compilation can be restricted to
selected child files by means of the |\includeonly| command.
The latter feature can be used to reduce the compilation time while editing
(this was significantly more useful in the earlier days of \LaTeX{})
or to generate a smaller document which is easier to navigate.
Another application of |\includeonly| is to generate
documents consisting of selected parts of the complete document.

However, there are a few drawbacks of the plain |\include| mechanism:
\begin{itemize}
\item
The child files cannot be compiled on their own,
they can only be compiled via the main file.
A naive editing environment
(such as a text editor with an option
to have the current file processed by \LaTeX)
may require one to switch to the main file before compiling;
attempting to compile the child file produces errors.
\item
The main file must be modified (each time)
to adjust the |\includeonly| command
to the present needs. This easily leaves the main file in a messy state.
\item
The generated document will always carry the filename
of the main document. This is inconvenient if
several child files are to be compiled and
to be kept for distribution.
\end{itemize}

The present package provides a simple interface
to make child files individually compilable by \LaTeX{}.
Compiling a child file then has the same effect as compiling
the main file with an |\includeonly| command
to select the appropriate child.
Moreover the generated document will carry the name of the child
rather than the main file.
This resolves all three above issues.

This feature is meant to make the editing of books,
thesis documents and lecture notes somewhat more convenient.
However, the package can also be used efficiently for
composing a series of documents (such as exercise sheets)
which are typically distributed individually.
It then assists the author in generating the individual documents
(potentially in different versions)
as well as a document containing the collected series.
Another application is in developing style files
or other kinds of included material
where compilation of the style file could redirect
to a sample or test file.

%%%%%%%%%%%%%%%%%%%%%%%%%%%%%%%%%%%%%%%%%%%%%%%%%%%%%%%%%%%%%%%%%%%%%%%%%%%%%%%%
%%%%%%%%%%%%%%%%%%%%%%%%%%%%%%%%%%%%%%%%%%%%%%%%%%%%%%%%%%%%%%%%%%%%%%%%%%%%%%%%
\section{Usage}

First of all, the package \textsf{childdoc} is \emph{not} a standard
\LaTeXe{} |.sty| style file! Therefore it needs to be invoked in
a non-standard way.

%%%%%%%%%%%%%%%%%%%%%%%%%%%%%%%%%%%%%%%%%%%%%%%%%%%%%%%%%%%%%%%%%%%%%%%%%%%%%%%%
\subsection{Included Files}
\label{sec:include}

%%%%%%%%%%%%%%%%%%%%%%%%%%%%%%%%%%%%%%%%
\DescribeMacro{\childdocmain}
To use the package, add the commands
\begin{center}
\begin{tabular}{l}
|\input{childdoc.def}|\\
|\childdocmain{}|\\
\end{tabular}
\end{center}
at the very top of the main \LaTeX{} file,
in particular \emph{before} the |\documentclass| statement!
The argument of |\childdocmain| should be left empty
(but it must be present).

%%%%%%%%%%%%%%%%%%%%%%%%%%%%%%%%%%%%%%%%
\DescribeMacro{\childdocof}
Furthermore, add the commands
\begin{center}
\begin{tabular}{l}
|\input{childdoc.def}|\\
|\childdocof{|\textit{main}|}|\\
\end{tabular}
\end{center}
at the top of every child file \textit{child}
which is included by |\include{|\textit{child}|}|
from within the main file
(or at least for those files to be compiled individually).
The argument \textit{main} must be the filename of the main file.

There are a couple of
considerations in setting up the main and child documents:

%%%%%%%%%%%%%%%%%%%%%%%%%%%%%%%%%%%%%%%%
\paragraph{Restrictions.}

Please note the following restrictions:
\begin{itemize}
\item
|\childdocmain| must be called with one argument \textit{main}
to ensure compatibility with earlier version of the package.
It must either be empty (|\childdocmain{}|)
or precisely match the filename of the main file in which it is specified.
See \secref{sec:detection} for further information.
\item
The filename \textit{main} must be specified without the |.tex| extension.
\item
The filename \textit{main} is case sensitive
(even in case-insensitive file systems)
due to internal string comparison.
\item
The argument \textit{main} should be fully expanded, it cannot be a macro.
\item
Subdirectories and special characters should be avoided in filenames.
\item
The command |\childdocmain{|\textit{main}|}| must be followed by a whitespace.
It should not be followed immediately by another command
or by a comment mark `|%|'.
This is because the \TeX{} parser reads the token immediately following
the argument of |\childdocmain| and puts it
at the beginning of every child section;
however, a white\-space is ignored.
\end{itemize}

%%%%%%%%%%%%%%%%%%%%%%%%%%%%%%%%%%%%%%%%
\paragraph{Content of Main File.}

It is advisable to place all content in the child files included by |\include|.
Any output contained in the main file will appear in all child documents
unless suppressed manually;
it cannot be suppressed automatically by the |\includeonly| directive
and thus should normally be avoided.
A method to include some content in the main file
by means of conditional processing is described in \secref{sec:conditional}.

%%%%%%%%%%%%%%%%%%%%%%%%%%%%%%%%%%%%%%%%
\paragraph{Page Numbering.}

When only a part of the document is compiled,
the appropriate numbering of pages
(as well as other status parameters)
is determined from the |.aux| files.
The latter contain information from previous passes.
However this information needs to propagate through
all intermediate child documents.
Therefore the page numbering in child documents may well
be inconsistent until the complete document is compiled at least once.

A useful (if unconventional) way to always ensure a consistent
page numbering is to restart the numbering in each child document
and denote the pages by `\textit{child}|.|\textit{page}'
where \textit{child} represents the chapter/section number of the child file.
This can be achieved by the command
|\numberwithin{page}{|\textit{child}|}|
of the \textsf{amsmath} package
where \textit{child} can be |chapter| or |section|
depending on the chosen structuring.
Alternatively, one can modify the macro |\thepage| appropriately
and reset the counter |page| at the start of each child file.

%%%%%%%%%%%%%%%%%%%%%%%%%%%%%%%%%%%%%%%%%%%%%%%%%%%%%%%%%%%%%%%%%%%%%%%%%%%%%%%%
\subsection{Conditional Processing}
\label{sec:conditional}

The package provides a mechanism to compile different versions
of a document. To customise the versions further some conditional processing
can come in handy to distinguish which version is being compiled.
The package provides two macros to describe the compilation context:

%%%%%%%%%%%%%%%%%%%%%%%%%%%%%%%%%%%%%%%%
\DescribeMacro{\ifchilddoc}
The conditional |\ifchilddoc| distinguishes between the compilation of
child documents and the main document:
%
\begin{center}
|\ifchilddoc |\textit{child-code}| |[|\||else |\textit{main-code}]| \||fi|
\end{center}

%%%%%%%%%%%%%%%%%%%%%%%%%%%%%%%%%%%%%%%%
\DescribeMacro{\childdocname}
\DescribeMacro{\childdocjob}
The macro |\childdocname| contains the filename (without extension)
of the main or child file being processed.
Note that |\childdocjob| will always contain the name of the main file.

%%%%%%%%%%%%%%%%%%%%%%%%%%%%%%%%%%%%%%%%
\paragraph{Title Page.}

Conditional processing can be used to include a title or banner page
in the main document when proper precautions are taken.
Importantly, the code in the main file should ensure that the page counter
(as well as other status parameters which are stored in the |.aux| files)
takes the same value after the conditional processing.
Otherwise the page numbers may take divergent values
depending on which part is compiled.

For example, a title page could be declared by:
%
\begin{center}
\begin{tabular}{l}
|\ifchilddoc\||else|\\
|\addtocounter{page}{-1}|\\
\textit{code for title page}\\
|\newpage|\\
|\||fi|
\end{tabular}
\end{center}
%
A banner page for the child documents can be generated by:
%
\begin{center}
\begin{tabular}{l}
|\ifchilddoc|\\
|\addtocounter{page}{-1}|\\
\textit{code for banner page}\\
|\newpage|\\
|\||fi|
\end{tabular}
\end{center}
%
Here one could write a message such as:
\begin{center}
|This is the part \childdocname{} of \childdocjob{}.|
\end{center}

%%%%%%%%%%%%%%%%%%%%%%%%%%%%%%%%%%%%%%%%%%%%%%%%%%%%%%%%%%%%%%%%%%%%%%%%%%%%%%%%
\subsection{Flags}
\label{sec:flags}

The package makes it easy to generate different versions
of the main or child documents.
To this end compilation flags can be defined
and assigned different default values.
They will be particularly useful in conjunction
with the forwarding mechanism described in \secref{sec:forward}.

For example, it may be useful to have a flag |\version|
which can be set to |draft| or |final|.
The document source will contain some conditional code
depending on the value of |\version|.
Suppose further, the flag should default to |final| for the main file
and to |draft| for child files
which is a natural assignment for editing the document.
This is achieved by placing the following code
in the preamble of the main document
(below the |\childdocmain| directive):
%
\begin{center}
\begin{tabular}{l}
|\ifchilddoc|\\
|\providecommand{\version}{draft}|\\
|\||else|\\
|\providecommand{\version}{final}|\\
|\||fi|
\end{tabular}
\end{center}
%
The definition by |\providecommand| makes sure
that previous definitions are not overwritten.
Further statements |\providecommand{\version}{...}|
can thus be added before the above code to override it.

For the main file, one might add a line
(between |\childdocmain| and the above block)
%
\begin{center}
|%\ifchilddoc\||else\providecommand{\version}{draft}\||fi|
\end{center}
%
which can be uncommented to produce a draft version.
Likewise one can add a line to the very top of a child file
(above the |\childdocof{|\textit{main}|}| directive)
%
\begin{center}
|%\providecommand{\version}{final}|
\end{center}
%
which can be uncommented to produce the final version of this child document.

%%%%%%%%%%%%%%%%%%%%%%%%%%%%%%%%%%%%%%%%%%%%%%%%%%%%%%%%%%%%%%%%%%%%%%%%%%%%%%%%
\subsection{Forwarding}
\label{sec:forward}

Different versions of the main or child documents
using compilation flags as described in \secref{sec:flags}
can be (permanently) stored in different files
for convenient compilation, viewing and distribution.
To this end, the package defines a command
to pass on compilation to a different file:

%%%%%%%%%%%%%%%%%%%%%%%%%%%%%%%%%%%%%%%%
\DescribeMacro{\childdocforward}
The command |\childdocforward| redirects processing to
another source file:
%
\begin{center}
\begin{tabular}{l}
|\input{childdoc.def}|\\
|\childdocforward[|\textit{main}|]{|\textit{dest}|}|\\
\end{tabular}
\end{center}
%
The argument \textit{dest} is the destination file
(without extension).
It should be the main file or one of the child files.
Note that further \textsf{childdoc} directives
such as |\childdocof| and |\childdocforward|
in the indicated file will be processed in this form.
The optional argument \textit{main}
passes on directly to the main file \textit{main}
while pretending to compile the child \textit{dest}.
This form behaves as if \textit{dest}
issues |\childdocof{|\textit{main}|}| right away,
and no further \textsf{childdoc} directives will be processed.

%%%%%%%%%%%%%%%%%%%%%%%%%%%%%%%%%%%%%%%%
\DescribeMacro{\...prefix}
In the alternative form |\childdocforwardprefix|,
%
\begin{center}
\begin{tabular}{l}
|\input{childdoc.def}|\\
|\childdocforwardprefix[|\textit{main}|]{|\textit{prefix}|}{|\textit{dest}|}|
\end{tabular}
\end{center}
%
the destination file is determined by a pattern
depending on the current file:
To make this work, the current file must be called
`{\textit{prefix}\hspace{0.2em}\textit{suffix}}'
with \textit{prefix} matching precisely the argument.
Processing is then passed on to the file
`{\textit{dest}\hspace{0.2em}\textit{suffix}}'.
Surely, the same effect is achieved by
directly specifying the
argument `{\textit{dest}\hspace{0.2em}\textit{suffix}}'
in the first form.
However, that requires to set up a different file
for each child. With the alternative form of the command
all these files can have exactly the same content
which simplifies setting them up and maintaining them.

For example, the following file |draft.tex|
with a compilation flag |\version| as described in \secref{sec:flags}
compiles the main document as a draft:
%
\begin{center}
\begin{tabular}{l}
|\def\version{draft}|\\
|\input{childdoc.def}|\\
|\childdocforward{|\textit{main}|}|
\end{tabular}
\end{center}
%
Likewise, the following files |final|\textit{nn}|.tex|
compile the final version of the child document
|child|\textit{nn}|.tex|:
%
\begin{center}
\begin{tabular}{l}
|\def\version{final}|\\
|\input{childdoc.def}|\\
|\childdocforwardprefix{final}{child}|
\end{tabular}
\end{center}
%

Note that when several versions of a main file and/or of each child file
are to be generated, it may be convenient to set up a |Makefile| or
shell script to automatise the process.

%%%%%%%%%%%%%%%%%%%%%%%%%%%%%%%%%%%%%%%%%%%%%%%%%%%%%%%%%%%%%%%%%%%%%%%%%%%%%%%%
\subsection{Command Line Processing}
\label{sec:commandline}

The effect of redirection files can also be achieved by invoking
the \LaTeX{} compiler with a more elaborate command line.
Most conveniently this should be done as part
of a shell script or a |Makefile|.

When using \textsf{childdoc} in the main file, the following
command lines effectively perform a redirection
(note that depending on the shell being used,
backslashes may have to be doubled: `|\|' $\to$ `|\\|'):
%
\begin{center}
|... -jobname "|\textit{target}|" |\\|"|[\textit{flags}]%
|\input{childdoc.def}\childdocforward[|\textit{main}|]{|\textit{dest}|}"|
\end{center}
%
Here \textit{target} is the name of the output file,
\textit{main} is the name of the main file
and \textit{dest} is the name of the main or child file to be processed
(all filenames without extensions).
The optional argument \textit{main} can be omitted
if \textit{main} matches \textit{dest}.
Optionally, compilation \textit{flags} can be defined via |\def| commands.
This command line makes the \TeX{} engine believe
it is compiling the file \textit{target}
whose content is specified as the latter parameter.
The provided code then forwards the processing to
\textit{main} or \textit{dest} as described in \secref{sec:forward}.

%%%%%%%%%%%%%%%%%%%%%%%%%%%%%%%%%%%%%%%%%%%%%%%%%%%%%%%%%%%%%%%%%%%%%%%%%%%%%%%%
\subsection{Include by Input}
\label{sec:input}

Including child documents by |\include| has some restrictions by design.
Most notably, the content of a child document always occupies
its own set of pages; pages cannot be shared between child documents.
Usually, this behaviour makes perfect sense
because each child document contain an essential part of the document.
However, in some situations it may be desirable to compose
a document from a collection of parts
without having mandatory page breaks between then.
For this case, the package
provides a mechanism to include parts
by |\input| which can also be processed individually.
However, by construction this mechanism
requires manual handling of the content to be output.

%%%%%%%%%%%%%%%%%%%%%%%%%%%%%%%%%%%%%%%%
\DescribeMacro{\ifchilddocmanual}
The main file should be prepared as usual, see \secref{sec:include}.
However, the document body must make a distinction
between processing of an individual part and of the main document, e.g.:
%
\begin{center}
\begin{tabular}{l}
|\ifchilddocmanual|\\
|\input{\childdocname}|\\
|\||else|\\
\textit{document body with }|\input{|\textit{part}|}|\\
|\||fi|
\end{tabular}
\end{center}
%
The conditional |\ifchilddocmanual| is true whenever
a part to be included by |\input| is being compiled,
and the name of the part is stored in |\childdocname|.

%%%%%%%%%%%%%%%%%%%%%%%%%%%%%%%%%%%%%%%%
\DescribeMacro{\childdocby}
Each part to be included by |\input| should start with:
%
\begin{center}
\begin{tabular}{l}
|\input{childdoc.def}|\\
|\childdocby{|\textit{main}|}|\\
\end{tabular}
\end{center}
%
The directive |\childdocby| is similar to |\childdocof|
described in \secref{sec:include},
but the subsequent selection of content must be done manually.
To that end, both |\ifchilddoc| and |\ifchilddocmanual|
will be true upon processing of a part,
and the name of the part is stored in |\childdocname|.
Note that |\jobname| will be set to the filename of the current part
so that each part receives an individual |.aux| file
that does not interfere with the |.aux| file(s) of the main document.
This behaviour can be altered by the alternative form
|\childdocby[*]{|\textit{main}|}| (with a non-empty optional argument)
which uses the |.aux| file of the main document
by setting |\jobname| to \textit{main}.

%%%%%%%%%%%%%%%%%%%%%%%%%%%%%%%%%%%%%%%%%%%%%%%%%%%%%%%%%%%%%%%%%%%%%%%%%%%%%%%%
\subsection{Driver Development}
\label{sec:driver}

The \textsf{childdoc} mechanism can also be use for the development
of definition files such as \LaTeX{} styles or classes.
This case differs from the above setup with multiple parts
included by |\include| in that no |\includeonly| should be invoked.
This can be achieved by starting the include file
(before |\ProvidesPackage|) with:
%
\begin{center}
\begin{tabular}{l}
|\input{childdoc.def}|\\
|\childdocforward{|\textit{main}|}|\\
\end{tabular}
\end{center}
%
or alternatively with:
%
\begin{center}
\begin{tabular}{l}
|\input{childdoc.def}|\\
|\childdocby{|\textit{main}|}|\\
\end{tabular}
\end{center}
%
Both forms have slightly different effects as described above.
The main file is prepared as usual, see \secref{sec:include}.

%%%%%%%%%%%%%%%%%%%%%%%%%%%%%%%%%%%%%%%%%%%%%%%%%%%%%%%%%%%%%%%%%%%%%%%%%%%%%%%%
\subsection{Legacy Detection}
\label{sec:detection}

The directive |\childdocmain| in the main file can detect
whether the complete document or merely a child is to be compiled
even without using the directive |\childdocof|.
This method is deprecated because it is less robust
and there is no compelling reason to use it;
it is merely provided for backward compatibility
and it may be removed in future versions.

If the detection mechanism is to be used,
it is mandatory to correctly specify
the filename of the main file as the argument of |\childdocmain|:
%
\begin{center}
\begin{tabular}{l}
|\input{childdoc.def}|\\
|\childdocmain{|\textit{main}|}|\\
\end{tabular}
\end{center}
%
If |\jobname| does not match the argument \textit{main} of |\childdocmain|,
it is assumed that |\jobname| points to the child file to be compiled.
When using |\childdocmain| with the main file specified as argument,
it suffices to start a child file
with just |\input{|\textit{main}|}|
without loading of the package and using |\childdocof|.
If instead all processing is done
with the appropriate \textsf{childdoc} directives,
the argument of \textit{main} of |\childdocmain| can be empty.

An alternative version of the command line processing described
in \secref{sec:commandline} using the detection mechanism reads:
%
\begin{center}
|... -jobname "|\textit{target}|" "|[\textit{flags}]%
[|\def\jobname{|\textit{dest}|}|]|\input{|\textit{main}|}"|
\end{center}

%%%%%%%%%%%%%%%%%%%%%%%%%%%%%%%%%%%%%%%%%%%%%%%%%%%%%%%%%%%%%%%%%%%%%%%%%%%%%%%%
\subsection{Manual Code}
\label{sec:manual}

In case one cannot be certain whether the definitions file |childdoc.def|
is installed on the target \TeX{} distribution
and one prefers not to ship it,
it is conceivable to paste a few relevant commands into the sources.

To that end, drop all statements |\input{childdoc.def}|
and perform the replacements as outlined below.
Instead of |\childdocmain{|\textit{main}|}| add the following code
to the top of the main file:
%
\begin{center}
\begin{tabular}{l}
|\||ifdefined\childdocname\endinput\||fi\newif\ifchilddoc|\\
|\edef\childdocname{\scantokens\expandafter{\jobname\noexpand}}|\\
|\def\childdocmain{|\textit{main}|}\||ifx\childdocmain\childdocname\||else|\\
|\childdoctrue\includeonly{\childdocname}\let\jobname\childdocmain\||fi|\\
\end{tabular}
\end{center}
%
Instead of |\childdocof{|\textit{main}|}| just include the main file
at the top of each child file:
%
\begin{center}
|\input{|\textit{main}|}|
\end{center}
%
A simple redirection |\childdocforward{|\textit{dest}|}| is achieved by:
%
\begin{center}
|\def\jobname{|\textit{dest}|}\input{\jobname}|
\end{center}
%
The redirection with prefix
|\childdocforwardprefix[|\textit{prefix}|]{|\textit{dest}|}|
is accomplished by:
%
\begin{center}
\begin{tabular}{l}
|{\edef\jobname{\scantokens\expandafter{\jobname\noexpand}}|\\
|\def\redirectjob |\textit{prefix}|#1~~~{\gdef\jobname{|\textit{dest}|#1}}|\\
|\expandafter\redirectjob\jobname~~~}\input{\jobname}|
\end{tabular}
\end{center}

In an alternative approach,
child documents can be compiled by a specific command line
without additional code or specific definitions:
%
\begin{center}
|... -jobname "|\textit{target}|" "|[\textit{flags}]%
|\includeonly{|\textit{dest}|}\input{|\textit{main}|}"|
\end{center}
%

%%%%%%%%%%%%%%%%%%%%%%%%%%%%%%%%%%%%%%%%%%%%%%%%%%%%%%%%%%%%%%%%%%%%%%%%%%%%%%%%
%%%%%%%%%%%%%%%%%%%%%%%%%%%%%%%%%%%%%%%%%%%%%%%%%%%%%%%%%%%%%%%%%%%%%%%%%%%%%%%%
\section{Information}

%%%%%%%%%%%%%%%%%%%%%%%%%%%%%%%%%%%%%%%%%%%%%%%%%%%%%%%%%%%%%%%%%%%%%%%%%%%%%%%%
\subsection{Copyright}

Copyright \copyright{} 2017--2018 Niklas Beisert

This work may be distributed and/or modified under the
conditions of the \LaTeX{} Project Public License, either version 1.3
of this license or (at your option) any later version.
The latest version of this license is in
  \url{http://www.latex-project.org/lppl.txt}
and version 1.3 or later is part of all distributions of \LaTeX{}
version 2005/12/01 or later.

This work has the LPPL maintenance status `maintained'.

The Current Maintainer of this work is Niklas Beisert.

This work consists of the files |README.txt|, |childdoc.ins| and |childdoc.dtx|
as well as the derived files |childdoc.def|, |cdocsamp.tex|
with |cdocsch1.tex|, |cdocsch2.tex|, |cdocspt3.tex|, |cdocspt4.tex|,
|cdocsdrf.tex|, |cdocsfn1.tex|, |cdocsfn2.tex|
as well as |childdoc.pdf|.

%%%%%%%%%%%%%%%%%%%%%%%%%%%%%%%%%%%%%%%%%%%%%%%%%%%%%%%%%%%%%%%%%%%%%%%%%%%%%%%%
\subsection{Files and Installation}

The package consists of the files:
%
\begin{center}
\begin{tabular}{ll}
    |README.txt|   & readme file \\
    |childdoc.ins| & installation file \\
    |childdoc.dtx| & source file \\
    |childdoc.def| & definition file \\
    |cdocsamp.tex| & sample main file \\
    |cdocsch1.tex| & sample include file \\
    |cdocsch2.tex| & sample include file \\
    |cdocspt3.tex| & sample part file \\
    |cdocspt4.tex| & sample part file \\
    |cdocsdrf.tex| & sample redirection file \\
    |cdocsfn1.tex| & sample redirection file \\
    |cdocsfn2.tex| & sample redirection file \\
    |childdoc.pdf| & manual
\end{tabular}
\end{center}
%
The distribution consists of the files
|README.txt|, |childdoc.ins| and |childdoc.dtx|.
%
\begin{itemize}
\item
Run (pdf)\LaTeX{} on |childdoc.dtx|
to compile the manual |childdoc.pdf| (this file).
\item
Run \LaTeX{} on |childdoc.ins| to create the definitions file |childdoc.def|
and the sample |cdocsamp.tex| with include files
|cdocsch1.tex|, |cdocsch2.tex|, |cdocspt3.tex|, |cdocspt4.tex|,
|cdocsdrf.tex|, |cdocsfn1.tex|, |cdocsfn2.tex|.
Then copy the file |childdoc.def| to an appropriate directory of your \LaTeX{}
distribution, e.g.\ \textit{texmf-root}|/tex/latex/childdoc|.
\end{itemize}

%%%%%%%%%%%%%%%%%%%%%%%%%%%%%%%%%%%%%%%%%%%%%%%%%%%%%%%%%%%%%%%%%%%%%%%%%%%%%%%%
\subsection{Related CTAN Packages}

There are several other packages which offer a similar functionality:
%
\begin{itemize}
\item
The packages
\href{http://ctan.org/pkg/docmute}{\textsf{docmute}},
\href{http://ctan.org/pkg/includex}{\textsf{includex}} and
\href{http://ctan.org/pkg/standalone}{\textsf{standalone}}
provide commands to include only the document body of
a child file thus allowing both files to be compiled individually.
\item
The packages \href{http://ctan.org/pkg/subdocs}{\textsf{subdocs}}
and \href{http://ctan.org/pkg/subfiles}{\textsf{subfiles}}
provide structures in which the main and child documents can be
encapsulated and allowing them to be compiled individually.
The inclusion mechanism is different from the conventional |\include|.
\item
The package \href{http://ctan.org/pkg/combine}{\textsf{combine}}
is an elaborate solution to combine several documents into one.
\end{itemize}
%
See also the CTAN topic \href{http://ctan.org/topic/subdocs}{\textsf{subdocs}}
for further related packages.
The present package differs from the above solutions in that
a document structure constructed with the conventional |\include| mechanism
just needs two extra commands at the top of every file
such that all constituent files can be compiled individually.

%%%%%%%%%%%%%%%%%%%%%%%%%%%%%%%%%%%%%%%%%%%%%%%%%%%%%%%%%%%%%%%%%%%%%%%%%%%%%%%%
%\subsection{Feature Suggestions}
%
%The following is a list of features which may be useful for future
%versions of this package:
%%
%\begin{itemize}
%\item
%\ldots
%\end{itemize}

%%%%%%%%%%%%%%%%%%%%%%%%%%%%%%%%%%%%%%%%%%%%%%%%%%%%%%%%%%%%%%%%%%%%%%%%%%%%%%%%
\subsection{Revision History}

%%%%%%%%%%%%%%%%%%%%%%%%%%%%%%%%%%%%%%%%
\paragraph{v2.0:} 2018/12/30

\begin{itemize}
\item
immediate forward processing
\item
added |\childdocby| mechanism
\item
manual restructured
\end{itemize}

%%%%%%%%%%%%%%%%%%%%%%%%%%%%%%%%%%%%%%%%
\paragraph{v1.6:} 2018/01/17

\begin{itemize}
\item
application for development of include files
\item
corrections to manual
\end{itemize}

%%%%%%%%%%%%%%%%%%%%%%%%%%%%%%%%%%%%%%%%
\paragraph{v1.5:} 2017/05/21

\begin{itemize}
\item
more complete structuring introduced
\item
|\childdocof| introduced
\item
|\childdoc| renamed to |\childdocmain|
\item
|\childredirect| renamed to |\childdocforward| and |\childdocforwardprefix|
and functionality expanded
\end{itemize}

%%%%%%%%%%%%%%%%%%%%%%%%%%%%%%%%%%%%%%%%
\paragraph{v1.0:} 2017/04/27

\begin{itemize}
\item
manual and install package
\item
first version published on CTAN
\end{itemize}

%%%%%%%%%%%%%%%%%%%%%%%%%%%%%%%%%%%%%%%%
\paragraph{v0.6:} 2017/04/26

\begin{itemize}
\item
redirection mechanism added
\end{itemize}

%%%%%%%%%%%%%%%%%%%%%%%%%%%%%%%%%%%%%%%%
\paragraph{v0.5:} 2017/04/26

\begin{itemize}
\item
functionality in definition file
\end{itemize}


%%%%%%%%%%%%%%%%%%%%%%%%%%%%%%%%%%%%%%%%%%%%%%%%%%%%%%%%%%%%%%%%%%%%%%%%%%%%%%%%
%%%%%%%%%%%%%%%%%%%%%%%%%%%%%%%%%%%%%%%%%%%%%%%%%%%%%%%%%%%%%%%%%%%%%%%%%%%%%%%%
%%%%%%%%%%%%%%%%%%%%%%%%%%%%%%%%%%%%%%%%%%%%%%%%%%%%%%%%%%%%%%%%%%%%%%%%%%%%%%%%
\appendix

\settowidth\MacroIndent{\rmfamily\scriptsize 000\ }

 \DocInput{childdoc.dtx}

\end{document}
%</driver>
% \fi
%
% %%%%%%%%%%%%%%%%%%%%%%%%%%%%%%%%%%%%%%%%%%%%%%%%%%%%%%%%%%%%%%%%%%%%%%%%%%%%%%
% %%%%%%%%%%%%%%%%%%%%%%%%%%%%%%%%%%%%%%%%%%%%%%%%%%%%%%%%%%%%%%%%%%%%%%%%%%%%%%
% \section{Sample}
%\iffalse
%<*samplemain>
%\fi
%
% The following presents a sample document
% with two chapters, two parts, a title page,
% a compile flag as well as three forwarding files to set the flag.
% It consists of eight |.tex| files:
% \begin{center}
% \begin{tabular}{ll}
% |cdocsamp.tex|&main file\\
% |cdocsch1.tex|&include file for chapter 1\\
% |cdocsch2.tex|&include file for chapter 2\\
% |cdocspt3.tex|&include file for part 3\\
% |cdocspt4.tex|&include file for part 4\\
% |cdocsdrf.tex|&forwarding file for main file in draft mode\\
% |cdocsfi1.tex|&forwarding file for final version of chapter 1\\
% |cdocsfi2.tex|&forwarding file for final version of chapter 2\\
% \end{tabular}
% \end{center}
% Each of the eight files can be compiled directly by the \LaTeX{} compiler.
%
% %%%%%%%%%%%%%%%%%%%%%%%%%%%%%%%%%%%%%%
% \paragraph{Main File.}
%
% The main file is called |cdocsamp.tex|.
%
% Load the \textsf{childdoc} definitions and
% declare the filename for the main document:
%    \begin{macrocode}
\input{childdoc.def}
\childdocmain{}
%    \end{macrocode}

% Optional override for |\version| flag:
%    \begin{macrocode}
%%\ifchilddoc\else\providecommand{\version}{draft}\fi
%    \end{macrocode}

% Define the default values for the |\version| flag
% (|final| for the main file and |draft| for childs):
%    \begin{macrocode}
\ifchilddoc
\providecommand{\version}{draft}
\else
\providecommand{\version}{final}
\fi
%    \end{macrocode}

% Load the standard document class:
%    \begin{macrocode}
\documentclass[12pt]{article}
%    \end{macrocode}

% Start the document body:
%    \begin{macrocode}
\begin{document}
%    \end{macrocode}

% Declare a title page.
% Print title, part of document being processed and version flag:
%    \begin{macrocode}
\addtocounter{page}{-1}
\begin{center}
{\LARGE\bfseries{}childdoc example\par}
\vspace{1cm}
\ifchilddoc
\ifchilddocmanual part\else chapter\fi:
`\childdocname' of `\childdocjob'\par
\else
main document: `\childdocjob'\par
\fi
version: \version\par
\end{center}
\newpage
%    \end{macrocode}

% Manually include selected file,
% otherwise process as usual:
%    \begin{macrocode}
\ifchilddocmanual
\section*{part `\childdocname'}
\input{\childdocname}
\else
%    \end{macrocode}

% Include the two chapters:
%    \begin{macrocode}
\include{cdocsch1}
\include{cdocsch2}
%    \end{macrocode}

% Include the two parts unless only chapters should be displayed:
%    \begin{macrocode}
\ifchilddoc\else
\section{part three}
\input{cdocspt3}
\section{part four}
\input{cdocspt4}
\fi
%    \end{macrocode}

% Process as usual until here:
%    \begin{macrocode}
\fi
%    \end{macrocode}

% End of document body:
%    \begin{macrocode}
\end{document}
%    \end{macrocode}
%\iffalse
%</samplemain>
%\fi
%
% %%%%%%%%%%%%%%%%%%%%%%%%%%%%%%%%%%%%%%
% \paragraph{Chapter Include Files.}
%
% The include files are called |cdocsch1.tex| and |cdocsch2.tex|.
%
%\iffalse
%<*samplechap1|samplechap2>
%\fi

% Optional override for |\version| flag:
%    \begin{macrocode}
%%\providecommand{\version}{final}
%    \end{macrocode}

% Include the main document:
%    \begin{macrocode}
\input{childdoc.def}
\childdocof{cdocsamp}
%    \end{macrocode}

%\iffalse
%</samplechap1|samplechap2>
%\fi
%
%\iffalse
%<*samplechap1>
%\fi
% Some text for chapter 1:
%    \begin{macrocode}
\section{one}
some text in chapter one
%    \end{macrocode}

%\iffalse
%</samplechap1>
%\fi
% Some text for chapter 2:
%\iffalse
%<*samplechap2>
%\fi
%    \begin{macrocode}
\section{two}
more text in chapter two
%    \end{macrocode}

%\iffalse
%</samplechap2>
%\fi
%
% %%%%%%%%%%%%%%%%%%%%%%%%%%%%%%%%%%%%%%
% \paragraph{Part Include Files.}
%
% The include files are called |cdocspt3.tex| and |cdocspt4.tex|.
%
%\iffalse
%<*samplepart3|samplepart4>
%\fi

% Optional override for |\version| flag:
%    \begin{macrocode}
%%\providecommand{\version}{final}
%    \end{macrocode}

% Include the main document:
%    \begin{macrocode}
\input{childdoc.def}
\childdocby{cdocsamp}
%    \end{macrocode}

%\iffalse
%</samplepart3|samplepart4>
%\fi
%
%\iffalse
%<*samplepart3>
%\fi
% Some text for part 3:
%    \begin{macrocode}
some text in part three
%    \end{macrocode}

%\iffalse
%</samplepart3>
%\fi
% Some text for part 4:
%\iffalse
%<*samplepart4>
%\fi
%    \begin{macrocode}
more text in part four
%    \end{macrocode}

%\iffalse
%</samplepart4>
%\fi
%
% %%%%%%%%%%%%%%%%%%%%%%%%%%%%%%%%%%%%%%
% \paragraph{Forwarding for a Complete Draft.}
%
% The following forwarding file |cdocsdrf.tex|
% compiles the main document in draft mode:
%\iffalse
%<*sampledraft>
%\fi
%    \begin{macrocode}
\def\version{draft}
\input{childdoc.def}
\childdocforward{cdocsamp}
%    \end{macrocode}

%\iffalse
%</sampledraft>
%\fi
%
% %%%%%%%%%%%%%%%%%%%%%%%%%%%%%%%%%%%%%%
% \paragraph{Forwarding for Final Version of the Chapters.}
%
% The following forwarding files |cdocsfn1.tex| and |cdocsfn2.tex|
% (with identical content)
% compile the final versions of the child documents
% |cdocsch1.tex| and |cdocsch2.tex|, respectively:
%\iffalse
%<*samplefinal>
%\fi
%    \begin{macrocode}
\def\version{final}
\input{childdoc.def}
\childdocforwardprefix[cdocsamp]{cdocsfn}{cdocsch}
%    \end{macrocode}

%\iffalse
%</samplefinal>
%\fi
%
% %%%%%%%%%%%%%%%%%%%%%%%%%%%%%%%%%%%%%%
% \paragraph{Command Line Processing.}
%
% The following three command lines generate the output files
% |cdocscld|, |cdocscl1| and |cdocscl2|
% which should be identical to
% |cdocsdrf|, |cdocsch1| and |cdocsfn2|, respectively:
% \begin{center}
% \begin{tabular}{l}
% |latex -jobname cdocscld \|\\
% |  "\def\version{draft}\input{childdoc.def}\childdocforward{cdocsamp}"|\\
% |latex -jobname cdocscl1 \|\\
% |  "\input{childdoc.def}\childdocforward[cdocsamp]{cdocsch1}"|\\
% |latex -jobname cdocscl2 \|\\
% |  "\def\version{final}\input{childdoc.def}\childdocforward{cdocsch2}"|
% \end{tabular}
% \end{center}
% Note that the trailing backslash on each first line
% merely continues the input to the second line
% (for convenient cut ant paste).
% Furthermore, the command |latex| can be replaced by any
% of its alternative versions such as |pdflatex|.
%
% %%%%%%%%%%%%%%%%%%%%%%%%%%%%%%%%%%%%%%%%%%%%%%%%%%%%%%%%%%%%%%%%%%%%%%%%%%%%%%
% %%%%%%%%%%%%%%%%%%%%%%%%%%%%%%%%%%%%%%%%%%%%%%%%%%%%%%%%%%%%%%%%%%%%%%%%%%%%%%
% \section{Implementation}
%\iffalse
%<*package>
%\fi
%
% This section describes the definitions file |childdoc.def|.

% The definitions cannot be loaded using |\usepackage| or |\RequirePackage|
% which has a mechanism to prevent loading a style file more than once.
% When loading the definitions by means of |\input|
% multiple instances have to be prevented manually:
%\iffalse
%This code needs to be before the `\ProvidesFile' directive
%which is defined at the beginning of this file.
%Therefore it is also placed there and commented out here.
%</package>
%<*discard>
%\fi
%    \begin{macrocode}
\ifdefined\childdocmain\endinput\fi
%    \end{macrocode}
%\iffalse
%</discard>
%<*package>
%\fi
%
% \macro{\ifchilddoc}
% \macro{\ifchilddocmanual}
% The conditional |\ifchilddoc| tells whether a
% child (true) or main (false) document is being compiled.
% The conditional |\ifchilddocmanual| tells whether
% the |\includeonly| mechanism is used (false) or
% the selection of child files must be performed manually (true).
% The definitions initialise to false:
%    \begin{macrocode}
\newif\ifchilddoc
\newif\ifchilddocmanual
%    \end{macrocode}

% \macro{\childdocname}
% \macro{\childdocjob}
% The macro |\childdocname| stores the name of the main document
% to be compiled. The macro |\childdocjob| stores the name of
% the document on which the \LaTeX{} compiler was originally invoked.
% The content of |\jobname| cannot be compared
% to filenames specified in the source due to different catcodes.
% The following code rescans |\jobname|, stores the result
% in |\childdocname| and saves a copy in |\childdocjob|:
%    \begin{macrocode}
\edef\childdocname{\scantokens\expandafter{\jobname\noexpand}}
\let\childdocjob\childdocname
%    \end{macrocode}

% \macro{\childdocdisable}
% The macro |\childdocdisable| prevents the main file
% from being processed more than once.
% At this stage, the main document command |\childdocmain|
% is assumed to be called once again where it should do nothing.
% Any subsequent call to it should prevent
% a secondary processing of the main document
% It overwrites the forwarding commands
% |\childdocof| and |\childdocforward|
% with empty macros to prevent further inclusions of the main document:
%    \begin{macrocode}
\newcommand{\childdocdisable}
{
  \renewcommand{\childdocmain}[1]{\renewcommand{\childdocmain}[1]{\endinput}}
  \renewcommand{\childdocof}[1]{}
  \renewcommand{\childdocby}[2][]{}
  \renewcommand{\childdocforward}[2][]{}
  \renewcommand{\childdocdisable}{}
}
%    \end{macrocode}

% \macro{\childdocmain}
% The macro |\childdocmain| is to be called at the top of the main file
% with nothing or the main filename (without extension) as argument.
% First, it breaks loops.
% If the argument is not empty and does not match |\childdocname|
% (which is set by the first inclusion of |childdoc.def|),
% |\ifchilddoc| is set to true, |\includeonly| is applied to the child file
% and |\jobname| is set to the main file
% (for proper handling of |.aux| files):
%    \begin{macrocode}
\newcommand{\childdocmain}[1]
{
  \childdocdisable\childdocmain{}
  \if?#1?\else
    \begingroup
      \def\childdoctmp{#1}
      \ifx\childdoctmp\childdocname
        \def\childdoctmp{}
      \else
        \def\childdoctmp
        {
          \childdoctrue
          \includeonly{\childdocname}
          \def\childdocjob{#1}
          \def\jobname{#1}
        }
      \fi
      \expandafter
    \endgroup
    \childdoctmp
  \fi
}
%    \end{macrocode}

% \macro{\childdocof}
% The command |\childdocof| redirects
% compilation to the main file |#1|.
%    \begin{macrocode}
\newcommand{\childdocof}[1]
{
  \childdocdisable
  \childdoctrue
  \includeonly{\childdocname}
  \def\jobname{#1}
  \def\childdocjob{#1}
  \input{#1}
}
%    \end{macrocode}

% \macro{\childdocby}
% The command |\childdocby| ....
%    \begin{macrocode}
\newcommand{\childdocby}[2][]
{
  \childdocdisable
  \childdoctrue
  \childdocmanualtrue
  \if?#1?\else
    \def\jobname{#2}
  \fi
  \def\childdocjob{#2}
  \input{#2}
  \endinput
}
%    \end{macrocode}

% \macro{\childdocforward}
% The command |\childdocforward| redirects
% compilation to the main file or
% (if the optional argument is given) a child file.
% Parameters are set as if the main file
% or a child file starting with |\childdocof| was compiled.
% Then compilation is handed over to the main file:
%    \begin{macrocode}
\newcommand{\childdocforward}[2][]
{
  \begingroup
    \if?#1?
      \def\childdoctmp
      {
        \def\childdocname{#2}
        \def\childdocjob{#2}
        \def\jobname{#2}
        \input{#2}
        \endinput
      }
    \else
      \def\childdoctmp
      {
        \childdocdisable
        \def\childdocname{#2}
        \childdoctrue
        \includeonly{#2}
        \def\childdocjob{#1}
        \def\jobname{#1}
        \input{#1}
        \endinput
      }
    \fi
    \expandafter
  \endgroup
  \childdoctmp
}
%    \end{macrocode}

% \macro{\childdocforwardprefix}
% The command |\childdocforwardprefix| redirects
% compilation to the main or a child file by means of a pattern.
% The prefix |#1| in the current filename is replaced by |#2|
% and the suffix of the current filename is kept
% (it is assumed that the filename does not contain the substring `|~~~|'
% which is used as a delimiter).
% Compilation is handed over to the new file by |\childdocforward|:
%    \begin{macrocode}
\newcommand{\childdocforwardprefix}[3][]
{
  \begingroup
    \def\childdocextract #2##1~~~{\def\childdoctmp{\childdocforward[#1]{#3##1}}}
    \expandafter\childdocextract\childdocname~~~
    \expandafter
  \endgroup
  \childdoctmp
}
%    \end{macrocode}

% \macro{\childdoc}
% The deprecated macro |\childdoc| is a legacy version of |\childdocmain|:
%    \begin{macrocode}
\newcommand{\childdoc}{\childdocmain}
%    \end{macrocode}

% \macro{\childdocredirect}
% The deprecated macro |\childdocredirect| is a legacy version
% of |\childdocforward| and |\childdocforwardprefix|:
%    \begin{macrocode}
\newcommand{\childdocredirect}[2][]
{
  \begingroup
    \if?#1?
      \def\childdoctmp{\childdocforward{#2}}
    \else
      \def\childdoctmp{\childdocforwardprefix{#1}{#2}}
    \fi
    \expandafter
  \endgroup
  \childdoctmp
}
%    \end{macrocode}

%\iffalse
%</package>
%\fi
%
\endinput

\childdocmain{}
%    \end{macrocode}

% Optional override for |\version| flag:
%    \begin{macrocode}
%%\ifchilddoc\else\providecommand{\version}{draft}\fi
%    \end{macrocode}

% Define the default values for the |\version| flag
% (|final| for the main file and |draft| for childs):
%    \begin{macrocode}
\ifchilddoc
\providecommand{\version}{draft}
\else
\providecommand{\version}{final}
\fi
%    \end{macrocode}

% Load the standard document class:
%    \begin{macrocode}
\documentclass[12pt]{article}
%    \end{macrocode}

% Start the document body:
%    \begin{macrocode}
\begin{document}
%    \end{macrocode}

% Declare a title page.
% Print title, part of document being processed and version flag:
%    \begin{macrocode}
\addtocounter{page}{-1}
\begin{center}
{\LARGE\bfseries{}childdoc example\par}
\vspace{1cm}
\ifchilddoc
\ifchilddocmanual part\else chapter\fi:
`\childdocname' of `\childdocjob'\par
\else
main document: `\childdocjob'\par
\fi
version: \version\par
\end{center}
\newpage
%    \end{macrocode}

% Manually include selected file,
% otherwise process as usual:
%    \begin{macrocode}
\ifchilddocmanual
\section*{part `\childdocname'}
\input{\childdocname}
\else
%    \end{macrocode}

% Include the two chapters:
%    \begin{macrocode}
\include{cdocsch1}
\include{cdocsch2}
%    \end{macrocode}

% Include the two parts unless only chapters should be displayed:
%    \begin{macrocode}
\ifchilddoc\else
\section{part three}
\input{cdocspt3}
\section{part four}
\input{cdocspt4}
\fi
%    \end{macrocode}

% Process as usual until here:
%    \begin{macrocode}
\fi
%    \end{macrocode}

% End of document body:
%    \begin{macrocode}
\end{document}
%    \end{macrocode}
%\iffalse
%</samplemain>
%\fi
%
% %%%%%%%%%%%%%%%%%%%%%%%%%%%%%%%%%%%%%%
% \paragraph{Chapter Include Files.}
%
% The include files are called |cdocsch1.tex| and |cdocsch2.tex|.
%
%\iffalse
%<*samplechap1|samplechap2>
%\fi

% Optional override for |\version| flag:
%    \begin{macrocode}
%%\providecommand{\version}{final}
%    \end{macrocode}

% Include the main document:
%    \begin{macrocode}
% \iffalse
%
% childdoc.dtx Copyright (C) 2017-2018 Niklas Beisert
%
% This work may be distributed and/or modified under the
% conditions of the LaTeX Project Public License, either version 1.3
% of this license or (at your option) any later version.
% The latest version of this license is in
%   http://www.latex-project.org/lppl.txt
% and version 1.3 or later is part of all distributions of LaTeX
% version 2005/12/01 or later.
%
% This work has the LPPL maintenance status `maintained'.
%
% The Current Maintainer of this work is Niklas Beisert.
%
% This work consists of the files childdoc.dtx and childdoc.ins
% and the derived files childdoc.def and cdocsamp.tex with
% cdocsch1.tex, cdocsch2.tex, cdocsdrf.tex, cdocsfn1.tex, cdocsfn2.tex.
%
%<package>\ifdefined\childdocmain\endinput\fi
%<package>\ProvidesFile{childdoc.def}[2018/12/30 v2.0 child document driver]
%<samplemain>\ProvidesFile{cdocsamp.tex}[2018/12/30 v2.0 sample for childdoc]
%<*driver>
%\ProvidesFile{childdoc.drv}[2018/12/30 v2.0 childdoc reference manual file]
\PassOptionsToClass{10pt,a4paper}{article}
\documentclass{ltxdoc}

\usepackage[margin=35mm]{geometry}
\usepackage{hyperref}
\usepackage{hyperxmp}
\usepackage[usenames]{color}

\hypersetup{colorlinks=true}
\hypersetup{pdfstartview=FitH}
\hypersetup{pdfpagemode=UseNone}
\hypersetup{pdfsource={}}
\hypersetup{pdflang={en-UK}}
\hypersetup{pdfcopyright={Copyright 2017-2018 Niklas Beisert.
  This work may be distributed and/or modified under the
  conditions of the LaTeX Project Public License, either version 1.3
  of this license or (at your option) any later version.}}
\hypersetup{pdflicenseurl={http://www.latex-project.org/lppl.txt}}
\hypersetup{pdfcontactaddress={ETH Zurich, ITP, HIT K,
  Wolfgang-Pauli-Strasse 27}}
\hypersetup{pdfcontactpostcode={8093}}
\hypersetup{pdfcontactcity={Zurich}}
\hypersetup{pdfcontactcountry={Switzerland}}
\hypersetup{pdfcontactemail={nbeisert@itp.phys.ethz.ch}}
\hypersetup{pdfcontacturl={http://people.phys.ethz.ch/\xmptilde nbeisert/}}

\newcommand{\secref}[1]{\hyperref[#1]{section \ref*{#1}}}

\parskip1ex
\parindent0pt
\let\olditemize\itemize
\def\itemize{\olditemize\parskip0pt}

\begin{document}

\title{The \textsf{childdoc} Package}
\hypersetup{pdftitle={The childdoc Package}}
\author{Niklas Beisert\\[2ex]
  Institut f\"ur Theoretische Physik\\
  Eidgen\"ossische Technische Hochschule Z\"urich\\
  Wolfgang-Pauli-Strasse 27, 8093 Z\"urich, Switzerland\\[1ex]
  \href{mailto:nbeisert@itp.phys.ethz.ch}
  {\texttt{nbeisert@itp.phys.ethz.ch}}}
\hypersetup{pdfauthor={Niklas Beisert}}
\hypersetup{pdfsubject={Manual for the LaTeX2e Package childdoc}}
\date{30 December 2018, \textsf{v2.0}}
\maketitle

\begin{abstract}\noindent
\textsf{childdoc} is a \LaTeXe{} package
that enables the direct compilation
of document sections included by |\include|
to individual files.
\end{abstract}

\begingroup
\parskip0ex
\tableofcontents
\endgroup

%%%%%%%%%%%%%%%%%%%%%%%%%%%%%%%%%%%%%%%%%%%%%%%%%%%%%%%%%%%%%%%%%%%%%%%%%%%%%%%%
%%%%%%%%%%%%%%%%%%%%%%%%%%%%%%%%%%%%%%%%%%%%%%%%%%%%%%%%%%%%%%%%%%%%%%%%%%%%%%%%
\section{Introduction}

\LaTeX{} provides a mechanism to structure a large document (such as a book)
into a main file and several child files (containing the chapters)
using the |\include| command.
This mechanism is beneficial for documents
which span hundreds of pages in order to
make the source file(s) more manageable.
Moreover, compilation can be restricted to
selected child files by means of the |\includeonly| command.
The latter feature can be used to reduce the compilation time while editing
(this was significantly more useful in the earlier days of \LaTeX{})
or to generate a smaller document which is easier to navigate.
Another application of |\includeonly| is to generate
documents consisting of selected parts of the complete document.

However, there are a few drawbacks of the plain |\include| mechanism:
\begin{itemize}
\item
The child files cannot be compiled on their own,
they can only be compiled via the main file.
A naive editing environment
(such as a text editor with an option
to have the current file processed by \LaTeX)
may require one to switch to the main file before compiling;
attempting to compile the child file produces errors.
\item
The main file must be modified (each time)
to adjust the |\includeonly| command
to the present needs. This easily leaves the main file in a messy state.
\item
The generated document will always carry the filename
of the main document. This is inconvenient if
several child files are to be compiled and
to be kept for distribution.
\end{itemize}

The present package provides a simple interface
to make child files individually compilable by \LaTeX{}.
Compiling a child file then has the same effect as compiling
the main file with an |\includeonly| command
to select the appropriate child.
Moreover the generated document will carry the name of the child
rather than the main file.
This resolves all three above issues.

This feature is meant to make the editing of books,
thesis documents and lecture notes somewhat more convenient.
However, the package can also be used efficiently for
composing a series of documents (such as exercise sheets)
which are typically distributed individually.
It then assists the author in generating the individual documents
(potentially in different versions)
as well as a document containing the collected series.
Another application is in developing style files
or other kinds of included material
where compilation of the style file could redirect
to a sample or test file.

%%%%%%%%%%%%%%%%%%%%%%%%%%%%%%%%%%%%%%%%%%%%%%%%%%%%%%%%%%%%%%%%%%%%%%%%%%%%%%%%
%%%%%%%%%%%%%%%%%%%%%%%%%%%%%%%%%%%%%%%%%%%%%%%%%%%%%%%%%%%%%%%%%%%%%%%%%%%%%%%%
\section{Usage}

First of all, the package \textsf{childdoc} is \emph{not} a standard
\LaTeXe{} |.sty| style file! Therefore it needs to be invoked in
a non-standard way.

%%%%%%%%%%%%%%%%%%%%%%%%%%%%%%%%%%%%%%%%%%%%%%%%%%%%%%%%%%%%%%%%%%%%%%%%%%%%%%%%
\subsection{Included Files}
\label{sec:include}

%%%%%%%%%%%%%%%%%%%%%%%%%%%%%%%%%%%%%%%%
\DescribeMacro{\childdocmain}
To use the package, add the commands
\begin{center}
\begin{tabular}{l}
|\input{childdoc.def}|\\
|\childdocmain{}|\\
\end{tabular}
\end{center}
at the very top of the main \LaTeX{} file,
in particular \emph{before} the |\documentclass| statement!
The argument of |\childdocmain| should be left empty
(but it must be present).

%%%%%%%%%%%%%%%%%%%%%%%%%%%%%%%%%%%%%%%%
\DescribeMacro{\childdocof}
Furthermore, add the commands
\begin{center}
\begin{tabular}{l}
|\input{childdoc.def}|\\
|\childdocof{|\textit{main}|}|\\
\end{tabular}
\end{center}
at the top of every child file \textit{child}
which is included by |\include{|\textit{child}|}|
from within the main file
(or at least for those files to be compiled individually).
The argument \textit{main} must be the filename of the main file.

There are a couple of
considerations in setting up the main and child documents:

%%%%%%%%%%%%%%%%%%%%%%%%%%%%%%%%%%%%%%%%
\paragraph{Restrictions.}

Please note the following restrictions:
\begin{itemize}
\item
|\childdocmain| must be called with one argument \textit{main}
to ensure compatibility with earlier version of the package.
It must either be empty (|\childdocmain{}|)
or precisely match the filename of the main file in which it is specified.
See \secref{sec:detection} for further information.
\item
The filename \textit{main} must be specified without the |.tex| extension.
\item
The filename \textit{main} is case sensitive
(even in case-insensitive file systems)
due to internal string comparison.
\item
The argument \textit{main} should be fully expanded, it cannot be a macro.
\item
Subdirectories and special characters should be avoided in filenames.
\item
The command |\childdocmain{|\textit{main}|}| must be followed by a whitespace.
It should not be followed immediately by another command
or by a comment mark `|%|'.
This is because the \TeX{} parser reads the token immediately following
the argument of |\childdocmain| and puts it
at the beginning of every child section;
however, a white\-space is ignored.
\end{itemize}

%%%%%%%%%%%%%%%%%%%%%%%%%%%%%%%%%%%%%%%%
\paragraph{Content of Main File.}

It is advisable to place all content in the child files included by |\include|.
Any output contained in the main file will appear in all child documents
unless suppressed manually;
it cannot be suppressed automatically by the |\includeonly| directive
and thus should normally be avoided.
A method to include some content in the main file
by means of conditional processing is described in \secref{sec:conditional}.

%%%%%%%%%%%%%%%%%%%%%%%%%%%%%%%%%%%%%%%%
\paragraph{Page Numbering.}

When only a part of the document is compiled,
the appropriate numbering of pages
(as well as other status parameters)
is determined from the |.aux| files.
The latter contain information from previous passes.
However this information needs to propagate through
all intermediate child documents.
Therefore the page numbering in child documents may well
be inconsistent until the complete document is compiled at least once.

A useful (if unconventional) way to always ensure a consistent
page numbering is to restart the numbering in each child document
and denote the pages by `\textit{child}|.|\textit{page}'
where \textit{child} represents the chapter/section number of the child file.
This can be achieved by the command
|\numberwithin{page}{|\textit{child}|}|
of the \textsf{amsmath} package
where \textit{child} can be |chapter| or |section|
depending on the chosen structuring.
Alternatively, one can modify the macro |\thepage| appropriately
and reset the counter |page| at the start of each child file.

%%%%%%%%%%%%%%%%%%%%%%%%%%%%%%%%%%%%%%%%%%%%%%%%%%%%%%%%%%%%%%%%%%%%%%%%%%%%%%%%
\subsection{Conditional Processing}
\label{sec:conditional}

The package provides a mechanism to compile different versions
of a document. To customise the versions further some conditional processing
can come in handy to distinguish which version is being compiled.
The package provides two macros to describe the compilation context:

%%%%%%%%%%%%%%%%%%%%%%%%%%%%%%%%%%%%%%%%
\DescribeMacro{\ifchilddoc}
The conditional |\ifchilddoc| distinguishes between the compilation of
child documents and the main document:
%
\begin{center}
|\ifchilddoc |\textit{child-code}| |[|\||else |\textit{main-code}]| \||fi|
\end{center}

%%%%%%%%%%%%%%%%%%%%%%%%%%%%%%%%%%%%%%%%
\DescribeMacro{\childdocname}
\DescribeMacro{\childdocjob}
The macro |\childdocname| contains the filename (without extension)
of the main or child file being processed.
Note that |\childdocjob| will always contain the name of the main file.

%%%%%%%%%%%%%%%%%%%%%%%%%%%%%%%%%%%%%%%%
\paragraph{Title Page.}

Conditional processing can be used to include a title or banner page
in the main document when proper precautions are taken.
Importantly, the code in the main file should ensure that the page counter
(as well as other status parameters which are stored in the |.aux| files)
takes the same value after the conditional processing.
Otherwise the page numbers may take divergent values
depending on which part is compiled.

For example, a title page could be declared by:
%
\begin{center}
\begin{tabular}{l}
|\ifchilddoc\||else|\\
|\addtocounter{page}{-1}|\\
\textit{code for title page}\\
|\newpage|\\
|\||fi|
\end{tabular}
\end{center}
%
A banner page for the child documents can be generated by:
%
\begin{center}
\begin{tabular}{l}
|\ifchilddoc|\\
|\addtocounter{page}{-1}|\\
\textit{code for banner page}\\
|\newpage|\\
|\||fi|
\end{tabular}
\end{center}
%
Here one could write a message such as:
\begin{center}
|This is the part \childdocname{} of \childdocjob{}.|
\end{center}

%%%%%%%%%%%%%%%%%%%%%%%%%%%%%%%%%%%%%%%%%%%%%%%%%%%%%%%%%%%%%%%%%%%%%%%%%%%%%%%%
\subsection{Flags}
\label{sec:flags}

The package makes it easy to generate different versions
of the main or child documents.
To this end compilation flags can be defined
and assigned different default values.
They will be particularly useful in conjunction
with the forwarding mechanism described in \secref{sec:forward}.

For example, it may be useful to have a flag |\version|
which can be set to |draft| or |final|.
The document source will contain some conditional code
depending on the value of |\version|.
Suppose further, the flag should default to |final| for the main file
and to |draft| for child files
which is a natural assignment for editing the document.
This is achieved by placing the following code
in the preamble of the main document
(below the |\childdocmain| directive):
%
\begin{center}
\begin{tabular}{l}
|\ifchilddoc|\\
|\providecommand{\version}{draft}|\\
|\||else|\\
|\providecommand{\version}{final}|\\
|\||fi|
\end{tabular}
\end{center}
%
The definition by |\providecommand| makes sure
that previous definitions are not overwritten.
Further statements |\providecommand{\version}{...}|
can thus be added before the above code to override it.

For the main file, one might add a line
(between |\childdocmain| and the above block)
%
\begin{center}
|%\ifchilddoc\||else\providecommand{\version}{draft}\||fi|
\end{center}
%
which can be uncommented to produce a draft version.
Likewise one can add a line to the very top of a child file
(above the |\childdocof{|\textit{main}|}| directive)
%
\begin{center}
|%\providecommand{\version}{final}|
\end{center}
%
which can be uncommented to produce the final version of this child document.

%%%%%%%%%%%%%%%%%%%%%%%%%%%%%%%%%%%%%%%%%%%%%%%%%%%%%%%%%%%%%%%%%%%%%%%%%%%%%%%%
\subsection{Forwarding}
\label{sec:forward}

Different versions of the main or child documents
using compilation flags as described in \secref{sec:flags}
can be (permanently) stored in different files
for convenient compilation, viewing and distribution.
To this end, the package defines a command
to pass on compilation to a different file:

%%%%%%%%%%%%%%%%%%%%%%%%%%%%%%%%%%%%%%%%
\DescribeMacro{\childdocforward}
The command |\childdocforward| redirects processing to
another source file:
%
\begin{center}
\begin{tabular}{l}
|\input{childdoc.def}|\\
|\childdocforward[|\textit{main}|]{|\textit{dest}|}|\\
\end{tabular}
\end{center}
%
The argument \textit{dest} is the destination file
(without extension).
It should be the main file or one of the child files.
Note that further \textsf{childdoc} directives
such as |\childdocof| and |\childdocforward|
in the indicated file will be processed in this form.
The optional argument \textit{main}
passes on directly to the main file \textit{main}
while pretending to compile the child \textit{dest}.
This form behaves as if \textit{dest}
issues |\childdocof{|\textit{main}|}| right away,
and no further \textsf{childdoc} directives will be processed.

%%%%%%%%%%%%%%%%%%%%%%%%%%%%%%%%%%%%%%%%
\DescribeMacro{\...prefix}
In the alternative form |\childdocforwardprefix|,
%
\begin{center}
\begin{tabular}{l}
|\input{childdoc.def}|\\
|\childdocforwardprefix[|\textit{main}|]{|\textit{prefix}|}{|\textit{dest}|}|
\end{tabular}
\end{center}
%
the destination file is determined by a pattern
depending on the current file:
To make this work, the current file must be called
`{\textit{prefix}\hspace{0.2em}\textit{suffix}}'
with \textit{prefix} matching precisely the argument.
Processing is then passed on to the file
`{\textit{dest}\hspace{0.2em}\textit{suffix}}'.
Surely, the same effect is achieved by
directly specifying the
argument `{\textit{dest}\hspace{0.2em}\textit{suffix}}'
in the first form.
However, that requires to set up a different file
for each child. With the alternative form of the command
all these files can have exactly the same content
which simplifies setting them up and maintaining them.

For example, the following file |draft.tex|
with a compilation flag |\version| as described in \secref{sec:flags}
compiles the main document as a draft:
%
\begin{center}
\begin{tabular}{l}
|\def\version{draft}|\\
|\input{childdoc.def}|\\
|\childdocforward{|\textit{main}|}|
\end{tabular}
\end{center}
%
Likewise, the following files |final|\textit{nn}|.tex|
compile the final version of the child document
|child|\textit{nn}|.tex|:
%
\begin{center}
\begin{tabular}{l}
|\def\version{final}|\\
|\input{childdoc.def}|\\
|\childdocforwardprefix{final}{child}|
\end{tabular}
\end{center}
%

Note that when several versions of a main file and/or of each child file
are to be generated, it may be convenient to set up a |Makefile| or
shell script to automatise the process.

%%%%%%%%%%%%%%%%%%%%%%%%%%%%%%%%%%%%%%%%%%%%%%%%%%%%%%%%%%%%%%%%%%%%%%%%%%%%%%%%
\subsection{Command Line Processing}
\label{sec:commandline}

The effect of redirection files can also be achieved by invoking
the \LaTeX{} compiler with a more elaborate command line.
Most conveniently this should be done as part
of a shell script or a |Makefile|.

When using \textsf{childdoc} in the main file, the following
command lines effectively perform a redirection
(note that depending on the shell being used,
backslashes may have to be doubled: `|\|' $\to$ `|\\|'):
%
\begin{center}
|... -jobname "|\textit{target}|" |\\|"|[\textit{flags}]%
|\input{childdoc.def}\childdocforward[|\textit{main}|]{|\textit{dest}|}"|
\end{center}
%
Here \textit{target} is the name of the output file,
\textit{main} is the name of the main file
and \textit{dest} is the name of the main or child file to be processed
(all filenames without extensions).
The optional argument \textit{main} can be omitted
if \textit{main} matches \textit{dest}.
Optionally, compilation \textit{flags} can be defined via |\def| commands.
This command line makes the \TeX{} engine believe
it is compiling the file \textit{target}
whose content is specified as the latter parameter.
The provided code then forwards the processing to
\textit{main} or \textit{dest} as described in \secref{sec:forward}.

%%%%%%%%%%%%%%%%%%%%%%%%%%%%%%%%%%%%%%%%%%%%%%%%%%%%%%%%%%%%%%%%%%%%%%%%%%%%%%%%
\subsection{Include by Input}
\label{sec:input}

Including child documents by |\include| has some restrictions by design.
Most notably, the content of a child document always occupies
its own set of pages; pages cannot be shared between child documents.
Usually, this behaviour makes perfect sense
because each child document contain an essential part of the document.
However, in some situations it may be desirable to compose
a document from a collection of parts
without having mandatory page breaks between then.
For this case, the package
provides a mechanism to include parts
by |\input| which can also be processed individually.
However, by construction this mechanism
requires manual handling of the content to be output.

%%%%%%%%%%%%%%%%%%%%%%%%%%%%%%%%%%%%%%%%
\DescribeMacro{\ifchilddocmanual}
The main file should be prepared as usual, see \secref{sec:include}.
However, the document body must make a distinction
between processing of an individual part and of the main document, e.g.:
%
\begin{center}
\begin{tabular}{l}
|\ifchilddocmanual|\\
|\input{\childdocname}|\\
|\||else|\\
\textit{document body with }|\input{|\textit{part}|}|\\
|\||fi|
\end{tabular}
\end{center}
%
The conditional |\ifchilddocmanual| is true whenever
a part to be included by |\input| is being compiled,
and the name of the part is stored in |\childdocname|.

%%%%%%%%%%%%%%%%%%%%%%%%%%%%%%%%%%%%%%%%
\DescribeMacro{\childdocby}
Each part to be included by |\input| should start with:
%
\begin{center}
\begin{tabular}{l}
|\input{childdoc.def}|\\
|\childdocby{|\textit{main}|}|\\
\end{tabular}
\end{center}
%
The directive |\childdocby| is similar to |\childdocof|
described in \secref{sec:include},
but the subsequent selection of content must be done manually.
To that end, both |\ifchilddoc| and |\ifchilddocmanual|
will be true upon processing of a part,
and the name of the part is stored in |\childdocname|.
Note that |\jobname| will be set to the filename of the current part
so that each part receives an individual |.aux| file
that does not interfere with the |.aux| file(s) of the main document.
This behaviour can be altered by the alternative form
|\childdocby[*]{|\textit{main}|}| (with a non-empty optional argument)
which uses the |.aux| file of the main document
by setting |\jobname| to \textit{main}.

%%%%%%%%%%%%%%%%%%%%%%%%%%%%%%%%%%%%%%%%%%%%%%%%%%%%%%%%%%%%%%%%%%%%%%%%%%%%%%%%
\subsection{Driver Development}
\label{sec:driver}

The \textsf{childdoc} mechanism can also be use for the development
of definition files such as \LaTeX{} styles or classes.
This case differs from the above setup with multiple parts
included by |\include| in that no |\includeonly| should be invoked.
This can be achieved by starting the include file
(before |\ProvidesPackage|) with:
%
\begin{center}
\begin{tabular}{l}
|\input{childdoc.def}|\\
|\childdocforward{|\textit{main}|}|\\
\end{tabular}
\end{center}
%
or alternatively with:
%
\begin{center}
\begin{tabular}{l}
|\input{childdoc.def}|\\
|\childdocby{|\textit{main}|}|\\
\end{tabular}
\end{center}
%
Both forms have slightly different effects as described above.
The main file is prepared as usual, see \secref{sec:include}.

%%%%%%%%%%%%%%%%%%%%%%%%%%%%%%%%%%%%%%%%%%%%%%%%%%%%%%%%%%%%%%%%%%%%%%%%%%%%%%%%
\subsection{Legacy Detection}
\label{sec:detection}

The directive |\childdocmain| in the main file can detect
whether the complete document or merely a child is to be compiled
even without using the directive |\childdocof|.
This method is deprecated because it is less robust
and there is no compelling reason to use it;
it is merely provided for backward compatibility
and it may be removed in future versions.

If the detection mechanism is to be used,
it is mandatory to correctly specify
the filename of the main file as the argument of |\childdocmain|:
%
\begin{center}
\begin{tabular}{l}
|\input{childdoc.def}|\\
|\childdocmain{|\textit{main}|}|\\
\end{tabular}
\end{center}
%
If |\jobname| does not match the argument \textit{main} of |\childdocmain|,
it is assumed that |\jobname| points to the child file to be compiled.
When using |\childdocmain| with the main file specified as argument,
it suffices to start a child file
with just |\input{|\textit{main}|}|
without loading of the package and using |\childdocof|.
If instead all processing is done
with the appropriate \textsf{childdoc} directives,
the argument of \textit{main} of |\childdocmain| can be empty.

An alternative version of the command line processing described
in \secref{sec:commandline} using the detection mechanism reads:
%
\begin{center}
|... -jobname "|\textit{target}|" "|[\textit{flags}]%
[|\def\jobname{|\textit{dest}|}|]|\input{|\textit{main}|}"|
\end{center}

%%%%%%%%%%%%%%%%%%%%%%%%%%%%%%%%%%%%%%%%%%%%%%%%%%%%%%%%%%%%%%%%%%%%%%%%%%%%%%%%
\subsection{Manual Code}
\label{sec:manual}

In case one cannot be certain whether the definitions file |childdoc.def|
is installed on the target \TeX{} distribution
and one prefers not to ship it,
it is conceivable to paste a few relevant commands into the sources.

To that end, drop all statements |\input{childdoc.def}|
and perform the replacements as outlined below.
Instead of |\childdocmain{|\textit{main}|}| add the following code
to the top of the main file:
%
\begin{center}
\begin{tabular}{l}
|\||ifdefined\childdocname\endinput\||fi\newif\ifchilddoc|\\
|\edef\childdocname{\scantokens\expandafter{\jobname\noexpand}}|\\
|\def\childdocmain{|\textit{main}|}\||ifx\childdocmain\childdocname\||else|\\
|\childdoctrue\includeonly{\childdocname}\let\jobname\childdocmain\||fi|\\
\end{tabular}
\end{center}
%
Instead of |\childdocof{|\textit{main}|}| just include the main file
at the top of each child file:
%
\begin{center}
|\input{|\textit{main}|}|
\end{center}
%
A simple redirection |\childdocforward{|\textit{dest}|}| is achieved by:
%
\begin{center}
|\def\jobname{|\textit{dest}|}\input{\jobname}|
\end{center}
%
The redirection with prefix
|\childdocforwardprefix[|\textit{prefix}|]{|\textit{dest}|}|
is accomplished by:
%
\begin{center}
\begin{tabular}{l}
|{\edef\jobname{\scantokens\expandafter{\jobname\noexpand}}|\\
|\def\redirectjob |\textit{prefix}|#1~~~{\gdef\jobname{|\textit{dest}|#1}}|\\
|\expandafter\redirectjob\jobname~~~}\input{\jobname}|
\end{tabular}
\end{center}

In an alternative approach,
child documents can be compiled by a specific command line
without additional code or specific definitions:
%
\begin{center}
|... -jobname "|\textit{target}|" "|[\textit{flags}]%
|\includeonly{|\textit{dest}|}\input{|\textit{main}|}"|
\end{center}
%

%%%%%%%%%%%%%%%%%%%%%%%%%%%%%%%%%%%%%%%%%%%%%%%%%%%%%%%%%%%%%%%%%%%%%%%%%%%%%%%%
%%%%%%%%%%%%%%%%%%%%%%%%%%%%%%%%%%%%%%%%%%%%%%%%%%%%%%%%%%%%%%%%%%%%%%%%%%%%%%%%
\section{Information}

%%%%%%%%%%%%%%%%%%%%%%%%%%%%%%%%%%%%%%%%%%%%%%%%%%%%%%%%%%%%%%%%%%%%%%%%%%%%%%%%
\subsection{Copyright}

Copyright \copyright{} 2017--2018 Niklas Beisert

This work may be distributed and/or modified under the
conditions of the \LaTeX{} Project Public License, either version 1.3
of this license or (at your option) any later version.
The latest version of this license is in
  \url{http://www.latex-project.org/lppl.txt}
and version 1.3 or later is part of all distributions of \LaTeX{}
version 2005/12/01 or later.

This work has the LPPL maintenance status `maintained'.

The Current Maintainer of this work is Niklas Beisert.

This work consists of the files |README.txt|, |childdoc.ins| and |childdoc.dtx|
as well as the derived files |childdoc.def|, |cdocsamp.tex|
with |cdocsch1.tex|, |cdocsch2.tex|, |cdocspt3.tex|, |cdocspt4.tex|,
|cdocsdrf.tex|, |cdocsfn1.tex|, |cdocsfn2.tex|
as well as |childdoc.pdf|.

%%%%%%%%%%%%%%%%%%%%%%%%%%%%%%%%%%%%%%%%%%%%%%%%%%%%%%%%%%%%%%%%%%%%%%%%%%%%%%%%
\subsection{Files and Installation}

The package consists of the files:
%
\begin{center}
\begin{tabular}{ll}
    |README.txt|   & readme file \\
    |childdoc.ins| & installation file \\
    |childdoc.dtx| & source file \\
    |childdoc.def| & definition file \\
    |cdocsamp.tex| & sample main file \\
    |cdocsch1.tex| & sample include file \\
    |cdocsch2.tex| & sample include file \\
    |cdocspt3.tex| & sample part file \\
    |cdocspt4.tex| & sample part file \\
    |cdocsdrf.tex| & sample redirection file \\
    |cdocsfn1.tex| & sample redirection file \\
    |cdocsfn2.tex| & sample redirection file \\
    |childdoc.pdf| & manual
\end{tabular}
\end{center}
%
The distribution consists of the files
|README.txt|, |childdoc.ins| and |childdoc.dtx|.
%
\begin{itemize}
\item
Run (pdf)\LaTeX{} on |childdoc.dtx|
to compile the manual |childdoc.pdf| (this file).
\item
Run \LaTeX{} on |childdoc.ins| to create the definitions file |childdoc.def|
and the sample |cdocsamp.tex| with include files
|cdocsch1.tex|, |cdocsch2.tex|, |cdocspt3.tex|, |cdocspt4.tex|,
|cdocsdrf.tex|, |cdocsfn1.tex|, |cdocsfn2.tex|.
Then copy the file |childdoc.def| to an appropriate directory of your \LaTeX{}
distribution, e.g.\ \textit{texmf-root}|/tex/latex/childdoc|.
\end{itemize}

%%%%%%%%%%%%%%%%%%%%%%%%%%%%%%%%%%%%%%%%%%%%%%%%%%%%%%%%%%%%%%%%%%%%%%%%%%%%%%%%
\subsection{Related CTAN Packages}

There are several other packages which offer a similar functionality:
%
\begin{itemize}
\item
The packages
\href{http://ctan.org/pkg/docmute}{\textsf{docmute}},
\href{http://ctan.org/pkg/includex}{\textsf{includex}} and
\href{http://ctan.org/pkg/standalone}{\textsf{standalone}}
provide commands to include only the document body of
a child file thus allowing both files to be compiled individually.
\item
The packages \href{http://ctan.org/pkg/subdocs}{\textsf{subdocs}}
and \href{http://ctan.org/pkg/subfiles}{\textsf{subfiles}}
provide structures in which the main and child documents can be
encapsulated and allowing them to be compiled individually.
The inclusion mechanism is different from the conventional |\include|.
\item
The package \href{http://ctan.org/pkg/combine}{\textsf{combine}}
is an elaborate solution to combine several documents into one.
\end{itemize}
%
See also the CTAN topic \href{http://ctan.org/topic/subdocs}{\textsf{subdocs}}
for further related packages.
The present package differs from the above solutions in that
a document structure constructed with the conventional |\include| mechanism
just needs two extra commands at the top of every file
such that all constituent files can be compiled individually.

%%%%%%%%%%%%%%%%%%%%%%%%%%%%%%%%%%%%%%%%%%%%%%%%%%%%%%%%%%%%%%%%%%%%%%%%%%%%%%%%
%\subsection{Feature Suggestions}
%
%The following is a list of features which may be useful for future
%versions of this package:
%%
%\begin{itemize}
%\item
%\ldots
%\end{itemize}

%%%%%%%%%%%%%%%%%%%%%%%%%%%%%%%%%%%%%%%%%%%%%%%%%%%%%%%%%%%%%%%%%%%%%%%%%%%%%%%%
\subsection{Revision History}

%%%%%%%%%%%%%%%%%%%%%%%%%%%%%%%%%%%%%%%%
\paragraph{v2.0:} 2018/12/30

\begin{itemize}
\item
immediate forward processing
\item
added |\childdocby| mechanism
\item
manual restructured
\end{itemize}

%%%%%%%%%%%%%%%%%%%%%%%%%%%%%%%%%%%%%%%%
\paragraph{v1.6:} 2018/01/17

\begin{itemize}
\item
application for development of include files
\item
corrections to manual
\end{itemize}

%%%%%%%%%%%%%%%%%%%%%%%%%%%%%%%%%%%%%%%%
\paragraph{v1.5:} 2017/05/21

\begin{itemize}
\item
more complete structuring introduced
\item
|\childdocof| introduced
\item
|\childdoc| renamed to |\childdocmain|
\item
|\childredirect| renamed to |\childdocforward| and |\childdocforwardprefix|
and functionality expanded
\end{itemize}

%%%%%%%%%%%%%%%%%%%%%%%%%%%%%%%%%%%%%%%%
\paragraph{v1.0:} 2017/04/27

\begin{itemize}
\item
manual and install package
\item
first version published on CTAN
\end{itemize}

%%%%%%%%%%%%%%%%%%%%%%%%%%%%%%%%%%%%%%%%
\paragraph{v0.6:} 2017/04/26

\begin{itemize}
\item
redirection mechanism added
\end{itemize}

%%%%%%%%%%%%%%%%%%%%%%%%%%%%%%%%%%%%%%%%
\paragraph{v0.5:} 2017/04/26

\begin{itemize}
\item
functionality in definition file
\end{itemize}


%%%%%%%%%%%%%%%%%%%%%%%%%%%%%%%%%%%%%%%%%%%%%%%%%%%%%%%%%%%%%%%%%%%%%%%%%%%%%%%%
%%%%%%%%%%%%%%%%%%%%%%%%%%%%%%%%%%%%%%%%%%%%%%%%%%%%%%%%%%%%%%%%%%%%%%%%%%%%%%%%
%%%%%%%%%%%%%%%%%%%%%%%%%%%%%%%%%%%%%%%%%%%%%%%%%%%%%%%%%%%%%%%%%%%%%%%%%%%%%%%%
\appendix

\settowidth\MacroIndent{\rmfamily\scriptsize 000\ }

 \DocInput{childdoc.dtx}

\end{document}
%</driver>
% \fi
%
% %%%%%%%%%%%%%%%%%%%%%%%%%%%%%%%%%%%%%%%%%%%%%%%%%%%%%%%%%%%%%%%%%%%%%%%%%%%%%%
% %%%%%%%%%%%%%%%%%%%%%%%%%%%%%%%%%%%%%%%%%%%%%%%%%%%%%%%%%%%%%%%%%%%%%%%%%%%%%%
% \section{Sample}
%\iffalse
%<*samplemain>
%\fi
%
% The following presents a sample document
% with two chapters, two parts, a title page,
% a compile flag as well as three forwarding files to set the flag.
% It consists of eight |.tex| files:
% \begin{center}
% \begin{tabular}{ll}
% |cdocsamp.tex|&main file\\
% |cdocsch1.tex|&include file for chapter 1\\
% |cdocsch2.tex|&include file for chapter 2\\
% |cdocspt3.tex|&include file for part 3\\
% |cdocspt4.tex|&include file for part 4\\
% |cdocsdrf.tex|&forwarding file for main file in draft mode\\
% |cdocsfi1.tex|&forwarding file for final version of chapter 1\\
% |cdocsfi2.tex|&forwarding file for final version of chapter 2\\
% \end{tabular}
% \end{center}
% Each of the eight files can be compiled directly by the \LaTeX{} compiler.
%
% %%%%%%%%%%%%%%%%%%%%%%%%%%%%%%%%%%%%%%
% \paragraph{Main File.}
%
% The main file is called |cdocsamp.tex|.
%
% Load the \textsf{childdoc} definitions and
% declare the filename for the main document:
%    \begin{macrocode}
\input{childdoc.def}
\childdocmain{}
%    \end{macrocode}

% Optional override for |\version| flag:
%    \begin{macrocode}
%%\ifchilddoc\else\providecommand{\version}{draft}\fi
%    \end{macrocode}

% Define the default values for the |\version| flag
% (|final| for the main file and |draft| for childs):
%    \begin{macrocode}
\ifchilddoc
\providecommand{\version}{draft}
\else
\providecommand{\version}{final}
\fi
%    \end{macrocode}

% Load the standard document class:
%    \begin{macrocode}
\documentclass[12pt]{article}
%    \end{macrocode}

% Start the document body:
%    \begin{macrocode}
\begin{document}
%    \end{macrocode}

% Declare a title page.
% Print title, part of document being processed and version flag:
%    \begin{macrocode}
\addtocounter{page}{-1}
\begin{center}
{\LARGE\bfseries{}childdoc example\par}
\vspace{1cm}
\ifchilddoc
\ifchilddocmanual part\else chapter\fi:
`\childdocname' of `\childdocjob'\par
\else
main document: `\childdocjob'\par
\fi
version: \version\par
\end{center}
\newpage
%    \end{macrocode}

% Manually include selected file,
% otherwise process as usual:
%    \begin{macrocode}
\ifchilddocmanual
\section*{part `\childdocname'}
\input{\childdocname}
\else
%    \end{macrocode}

% Include the two chapters:
%    \begin{macrocode}
\include{cdocsch1}
\include{cdocsch2}
%    \end{macrocode}

% Include the two parts unless only chapters should be displayed:
%    \begin{macrocode}
\ifchilddoc\else
\section{part three}
\input{cdocspt3}
\section{part four}
\input{cdocspt4}
\fi
%    \end{macrocode}

% Process as usual until here:
%    \begin{macrocode}
\fi
%    \end{macrocode}

% End of document body:
%    \begin{macrocode}
\end{document}
%    \end{macrocode}
%\iffalse
%</samplemain>
%\fi
%
% %%%%%%%%%%%%%%%%%%%%%%%%%%%%%%%%%%%%%%
% \paragraph{Chapter Include Files.}
%
% The include files are called |cdocsch1.tex| and |cdocsch2.tex|.
%
%\iffalse
%<*samplechap1|samplechap2>
%\fi

% Optional override for |\version| flag:
%    \begin{macrocode}
%%\providecommand{\version}{final}
%    \end{macrocode}

% Include the main document:
%    \begin{macrocode}
\input{childdoc.def}
\childdocof{cdocsamp}
%    \end{macrocode}

%\iffalse
%</samplechap1|samplechap2>
%\fi
%
%\iffalse
%<*samplechap1>
%\fi
% Some text for chapter 1:
%    \begin{macrocode}
\section{one}
some text in chapter one
%    \end{macrocode}

%\iffalse
%</samplechap1>
%\fi
% Some text for chapter 2:
%\iffalse
%<*samplechap2>
%\fi
%    \begin{macrocode}
\section{two}
more text in chapter two
%    \end{macrocode}

%\iffalse
%</samplechap2>
%\fi
%
% %%%%%%%%%%%%%%%%%%%%%%%%%%%%%%%%%%%%%%
% \paragraph{Part Include Files.}
%
% The include files are called |cdocspt3.tex| and |cdocspt4.tex|.
%
%\iffalse
%<*samplepart3|samplepart4>
%\fi

% Optional override for |\version| flag:
%    \begin{macrocode}
%%\providecommand{\version}{final}
%    \end{macrocode}

% Include the main document:
%    \begin{macrocode}
\input{childdoc.def}
\childdocby{cdocsamp}
%    \end{macrocode}

%\iffalse
%</samplepart3|samplepart4>
%\fi
%
%\iffalse
%<*samplepart3>
%\fi
% Some text for part 3:
%    \begin{macrocode}
some text in part three
%    \end{macrocode}

%\iffalse
%</samplepart3>
%\fi
% Some text for part 4:
%\iffalse
%<*samplepart4>
%\fi
%    \begin{macrocode}
more text in part four
%    \end{macrocode}

%\iffalse
%</samplepart4>
%\fi
%
% %%%%%%%%%%%%%%%%%%%%%%%%%%%%%%%%%%%%%%
% \paragraph{Forwarding for a Complete Draft.}
%
% The following forwarding file |cdocsdrf.tex|
% compiles the main document in draft mode:
%\iffalse
%<*sampledraft>
%\fi
%    \begin{macrocode}
\def\version{draft}
\input{childdoc.def}
\childdocforward{cdocsamp}
%    \end{macrocode}

%\iffalse
%</sampledraft>
%\fi
%
% %%%%%%%%%%%%%%%%%%%%%%%%%%%%%%%%%%%%%%
% \paragraph{Forwarding for Final Version of the Chapters.}
%
% The following forwarding files |cdocsfn1.tex| and |cdocsfn2.tex|
% (with identical content)
% compile the final versions of the child documents
% |cdocsch1.tex| and |cdocsch2.tex|, respectively:
%\iffalse
%<*samplefinal>
%\fi
%    \begin{macrocode}
\def\version{final}
\input{childdoc.def}
\childdocforwardprefix[cdocsamp]{cdocsfn}{cdocsch}
%    \end{macrocode}

%\iffalse
%</samplefinal>
%\fi
%
% %%%%%%%%%%%%%%%%%%%%%%%%%%%%%%%%%%%%%%
% \paragraph{Command Line Processing.}
%
% The following three command lines generate the output files
% |cdocscld|, |cdocscl1| and |cdocscl2|
% which should be identical to
% |cdocsdrf|, |cdocsch1| and |cdocsfn2|, respectively:
% \begin{center}
% \begin{tabular}{l}
% |latex -jobname cdocscld \|\\
% |  "\def\version{draft}\input{childdoc.def}\childdocforward{cdocsamp}"|\\
% |latex -jobname cdocscl1 \|\\
% |  "\input{childdoc.def}\childdocforward[cdocsamp]{cdocsch1}"|\\
% |latex -jobname cdocscl2 \|\\
% |  "\def\version{final}\input{childdoc.def}\childdocforward{cdocsch2}"|
% \end{tabular}
% \end{center}
% Note that the trailing backslash on each first line
% merely continues the input to the second line
% (for convenient cut ant paste).
% Furthermore, the command |latex| can be replaced by any
% of its alternative versions such as |pdflatex|.
%
% %%%%%%%%%%%%%%%%%%%%%%%%%%%%%%%%%%%%%%%%%%%%%%%%%%%%%%%%%%%%%%%%%%%%%%%%%%%%%%
% %%%%%%%%%%%%%%%%%%%%%%%%%%%%%%%%%%%%%%%%%%%%%%%%%%%%%%%%%%%%%%%%%%%%%%%%%%%%%%
% \section{Implementation}
%\iffalse
%<*package>
%\fi
%
% This section describes the definitions file |childdoc.def|.

% The definitions cannot be loaded using |\usepackage| or |\RequirePackage|
% which has a mechanism to prevent loading a style file more than once.
% When loading the definitions by means of |\input|
% multiple instances have to be prevented manually:
%\iffalse
%This code needs to be before the `\ProvidesFile' directive
%which is defined at the beginning of this file.
%Therefore it is also placed there and commented out here.
%</package>
%<*discard>
%\fi
%    \begin{macrocode}
\ifdefined\childdocmain\endinput\fi
%    \end{macrocode}
%\iffalse
%</discard>
%<*package>
%\fi
%
% \macro{\ifchilddoc}
% \macro{\ifchilddocmanual}
% The conditional |\ifchilddoc| tells whether a
% child (true) or main (false) document is being compiled.
% The conditional |\ifchilddocmanual| tells whether
% the |\includeonly| mechanism is used (false) or
% the selection of child files must be performed manually (true).
% The definitions initialise to false:
%    \begin{macrocode}
\newif\ifchilddoc
\newif\ifchilddocmanual
%    \end{macrocode}

% \macro{\childdocname}
% \macro{\childdocjob}
% The macro |\childdocname| stores the name of the main document
% to be compiled. The macro |\childdocjob| stores the name of
% the document on which the \LaTeX{} compiler was originally invoked.
% The content of |\jobname| cannot be compared
% to filenames specified in the source due to different catcodes.
% The following code rescans |\jobname|, stores the result
% in |\childdocname| and saves a copy in |\childdocjob|:
%    \begin{macrocode}
\edef\childdocname{\scantokens\expandafter{\jobname\noexpand}}
\let\childdocjob\childdocname
%    \end{macrocode}

% \macro{\childdocdisable}
% The macro |\childdocdisable| prevents the main file
% from being processed more than once.
% At this stage, the main document command |\childdocmain|
% is assumed to be called once again where it should do nothing.
% Any subsequent call to it should prevent
% a secondary processing of the main document
% It overwrites the forwarding commands
% |\childdocof| and |\childdocforward|
% with empty macros to prevent further inclusions of the main document:
%    \begin{macrocode}
\newcommand{\childdocdisable}
{
  \renewcommand{\childdocmain}[1]{\renewcommand{\childdocmain}[1]{\endinput}}
  \renewcommand{\childdocof}[1]{}
  \renewcommand{\childdocby}[2][]{}
  \renewcommand{\childdocforward}[2][]{}
  \renewcommand{\childdocdisable}{}
}
%    \end{macrocode}

% \macro{\childdocmain}
% The macro |\childdocmain| is to be called at the top of the main file
% with nothing or the main filename (without extension) as argument.
% First, it breaks loops.
% If the argument is not empty and does not match |\childdocname|
% (which is set by the first inclusion of |childdoc.def|),
% |\ifchilddoc| is set to true, |\includeonly| is applied to the child file
% and |\jobname| is set to the main file
% (for proper handling of |.aux| files):
%    \begin{macrocode}
\newcommand{\childdocmain}[1]
{
  \childdocdisable\childdocmain{}
  \if?#1?\else
    \begingroup
      \def\childdoctmp{#1}
      \ifx\childdoctmp\childdocname
        \def\childdoctmp{}
      \else
        \def\childdoctmp
        {
          \childdoctrue
          \includeonly{\childdocname}
          \def\childdocjob{#1}
          \def\jobname{#1}
        }
      \fi
      \expandafter
    \endgroup
    \childdoctmp
  \fi
}
%    \end{macrocode}

% \macro{\childdocof}
% The command |\childdocof| redirects
% compilation to the main file |#1|.
%    \begin{macrocode}
\newcommand{\childdocof}[1]
{
  \childdocdisable
  \childdoctrue
  \includeonly{\childdocname}
  \def\jobname{#1}
  \def\childdocjob{#1}
  \input{#1}
}
%    \end{macrocode}

% \macro{\childdocby}
% The command |\childdocby| ....
%    \begin{macrocode}
\newcommand{\childdocby}[2][]
{
  \childdocdisable
  \childdoctrue
  \childdocmanualtrue
  \if?#1?\else
    \def\jobname{#2}
  \fi
  \def\childdocjob{#2}
  \input{#2}
  \endinput
}
%    \end{macrocode}

% \macro{\childdocforward}
% The command |\childdocforward| redirects
% compilation to the main file or
% (if the optional argument is given) a child file.
% Parameters are set as if the main file
% or a child file starting with |\childdocof| was compiled.
% Then compilation is handed over to the main file:
%    \begin{macrocode}
\newcommand{\childdocforward}[2][]
{
  \begingroup
    \if?#1?
      \def\childdoctmp
      {
        \def\childdocname{#2}
        \def\childdocjob{#2}
        \def\jobname{#2}
        \input{#2}
        \endinput
      }
    \else
      \def\childdoctmp
      {
        \childdocdisable
        \def\childdocname{#2}
        \childdoctrue
        \includeonly{#2}
        \def\childdocjob{#1}
        \def\jobname{#1}
        \input{#1}
        \endinput
      }
    \fi
    \expandafter
  \endgroup
  \childdoctmp
}
%    \end{macrocode}

% \macro{\childdocforwardprefix}
% The command |\childdocforwardprefix| redirects
% compilation to the main or a child file by means of a pattern.
% The prefix |#1| in the current filename is replaced by |#2|
% and the suffix of the current filename is kept
% (it is assumed that the filename does not contain the substring `|~~~|'
% which is used as a delimiter).
% Compilation is handed over to the new file by |\childdocforward|:
%    \begin{macrocode}
\newcommand{\childdocforwardprefix}[3][]
{
  \begingroup
    \def\childdocextract #2##1~~~{\def\childdoctmp{\childdocforward[#1]{#3##1}}}
    \expandafter\childdocextract\childdocname~~~
    \expandafter
  \endgroup
  \childdoctmp
}
%    \end{macrocode}

% \macro{\childdoc}
% The deprecated macro |\childdoc| is a legacy version of |\childdocmain|:
%    \begin{macrocode}
\newcommand{\childdoc}{\childdocmain}
%    \end{macrocode}

% \macro{\childdocredirect}
% The deprecated macro |\childdocredirect| is a legacy version
% of |\childdocforward| and |\childdocforwardprefix|:
%    \begin{macrocode}
\newcommand{\childdocredirect}[2][]
{
  \begingroup
    \if?#1?
      \def\childdoctmp{\childdocforward{#2}}
    \else
      \def\childdoctmp{\childdocforwardprefix{#1}{#2}}
    \fi
    \expandafter
  \endgroup
  \childdoctmp
}
%    \end{macrocode}

%\iffalse
%</package>
%\fi
%
\endinput

\childdocof{cdocsamp}
%    \end{macrocode}

%\iffalse
%</samplechap1|samplechap2>
%\fi
%
%\iffalse
%<*samplechap1>
%\fi
% Some text for chapter 1:
%    \begin{macrocode}
\section{one}
some text in chapter one
%    \end{macrocode}

%\iffalse
%</samplechap1>
%\fi
% Some text for chapter 2:
%\iffalse
%<*samplechap2>
%\fi
%    \begin{macrocode}
\section{two}
more text in chapter two
%    \end{macrocode}

%\iffalse
%</samplechap2>
%\fi
%
% %%%%%%%%%%%%%%%%%%%%%%%%%%%%%%%%%%%%%%
% \paragraph{Part Include Files.}
%
% The include files are called |cdocspt3.tex| and |cdocspt4.tex|.
%
%\iffalse
%<*samplepart3|samplepart4>
%\fi

% Optional override for |\version| flag:
%    \begin{macrocode}
%%\providecommand{\version}{final}
%    \end{macrocode}

% Include the main document:
%    \begin{macrocode}
% \iffalse
%
% childdoc.dtx Copyright (C) 2017-2018 Niklas Beisert
%
% This work may be distributed and/or modified under the
% conditions of the LaTeX Project Public License, either version 1.3
% of this license or (at your option) any later version.
% The latest version of this license is in
%   http://www.latex-project.org/lppl.txt
% and version 1.3 or later is part of all distributions of LaTeX
% version 2005/12/01 or later.
%
% This work has the LPPL maintenance status `maintained'.
%
% The Current Maintainer of this work is Niklas Beisert.
%
% This work consists of the files childdoc.dtx and childdoc.ins
% and the derived files childdoc.def and cdocsamp.tex with
% cdocsch1.tex, cdocsch2.tex, cdocsdrf.tex, cdocsfn1.tex, cdocsfn2.tex.
%
%<package>\ifdefined\childdocmain\endinput\fi
%<package>\ProvidesFile{childdoc.def}[2018/12/30 v2.0 child document driver]
%<samplemain>\ProvidesFile{cdocsamp.tex}[2018/12/30 v2.0 sample for childdoc]
%<*driver>
%\ProvidesFile{childdoc.drv}[2018/12/30 v2.0 childdoc reference manual file]
\PassOptionsToClass{10pt,a4paper}{article}
\documentclass{ltxdoc}

\usepackage[margin=35mm]{geometry}
\usepackage{hyperref}
\usepackage{hyperxmp}
\usepackage[usenames]{color}

\hypersetup{colorlinks=true}
\hypersetup{pdfstartview=FitH}
\hypersetup{pdfpagemode=UseNone}
\hypersetup{pdfsource={}}
\hypersetup{pdflang={en-UK}}
\hypersetup{pdfcopyright={Copyright 2017-2018 Niklas Beisert.
  This work may be distributed and/or modified under the
  conditions of the LaTeX Project Public License, either version 1.3
  of this license or (at your option) any later version.}}
\hypersetup{pdflicenseurl={http://www.latex-project.org/lppl.txt}}
\hypersetup{pdfcontactaddress={ETH Zurich, ITP, HIT K,
  Wolfgang-Pauli-Strasse 27}}
\hypersetup{pdfcontactpostcode={8093}}
\hypersetup{pdfcontactcity={Zurich}}
\hypersetup{pdfcontactcountry={Switzerland}}
\hypersetup{pdfcontactemail={nbeisert@itp.phys.ethz.ch}}
\hypersetup{pdfcontacturl={http://people.phys.ethz.ch/\xmptilde nbeisert/}}

\newcommand{\secref}[1]{\hyperref[#1]{section \ref*{#1}}}

\parskip1ex
\parindent0pt
\let\olditemize\itemize
\def\itemize{\olditemize\parskip0pt}

\begin{document}

\title{The \textsf{childdoc} Package}
\hypersetup{pdftitle={The childdoc Package}}
\author{Niklas Beisert\\[2ex]
  Institut f\"ur Theoretische Physik\\
  Eidgen\"ossische Technische Hochschule Z\"urich\\
  Wolfgang-Pauli-Strasse 27, 8093 Z\"urich, Switzerland\\[1ex]
  \href{mailto:nbeisert@itp.phys.ethz.ch}
  {\texttt{nbeisert@itp.phys.ethz.ch}}}
\hypersetup{pdfauthor={Niklas Beisert}}
\hypersetup{pdfsubject={Manual for the LaTeX2e Package childdoc}}
\date{30 December 2018, \textsf{v2.0}}
\maketitle

\begin{abstract}\noindent
\textsf{childdoc} is a \LaTeXe{} package
that enables the direct compilation
of document sections included by |\include|
to individual files.
\end{abstract}

\begingroup
\parskip0ex
\tableofcontents
\endgroup

%%%%%%%%%%%%%%%%%%%%%%%%%%%%%%%%%%%%%%%%%%%%%%%%%%%%%%%%%%%%%%%%%%%%%%%%%%%%%%%%
%%%%%%%%%%%%%%%%%%%%%%%%%%%%%%%%%%%%%%%%%%%%%%%%%%%%%%%%%%%%%%%%%%%%%%%%%%%%%%%%
\section{Introduction}

\LaTeX{} provides a mechanism to structure a large document (such as a book)
into a main file and several child files (containing the chapters)
using the |\include| command.
This mechanism is beneficial for documents
which span hundreds of pages in order to
make the source file(s) more manageable.
Moreover, compilation can be restricted to
selected child files by means of the |\includeonly| command.
The latter feature can be used to reduce the compilation time while editing
(this was significantly more useful in the earlier days of \LaTeX{})
or to generate a smaller document which is easier to navigate.
Another application of |\includeonly| is to generate
documents consisting of selected parts of the complete document.

However, there are a few drawbacks of the plain |\include| mechanism:
\begin{itemize}
\item
The child files cannot be compiled on their own,
they can only be compiled via the main file.
A naive editing environment
(such as a text editor with an option
to have the current file processed by \LaTeX)
may require one to switch to the main file before compiling;
attempting to compile the child file produces errors.
\item
The main file must be modified (each time)
to adjust the |\includeonly| command
to the present needs. This easily leaves the main file in a messy state.
\item
The generated document will always carry the filename
of the main document. This is inconvenient if
several child files are to be compiled and
to be kept for distribution.
\end{itemize}

The present package provides a simple interface
to make child files individually compilable by \LaTeX{}.
Compiling a child file then has the same effect as compiling
the main file with an |\includeonly| command
to select the appropriate child.
Moreover the generated document will carry the name of the child
rather than the main file.
This resolves all three above issues.

This feature is meant to make the editing of books,
thesis documents and lecture notes somewhat more convenient.
However, the package can also be used efficiently for
composing a series of documents (such as exercise sheets)
which are typically distributed individually.
It then assists the author in generating the individual documents
(potentially in different versions)
as well as a document containing the collected series.
Another application is in developing style files
or other kinds of included material
where compilation of the style file could redirect
to a sample or test file.

%%%%%%%%%%%%%%%%%%%%%%%%%%%%%%%%%%%%%%%%%%%%%%%%%%%%%%%%%%%%%%%%%%%%%%%%%%%%%%%%
%%%%%%%%%%%%%%%%%%%%%%%%%%%%%%%%%%%%%%%%%%%%%%%%%%%%%%%%%%%%%%%%%%%%%%%%%%%%%%%%
\section{Usage}

First of all, the package \textsf{childdoc} is \emph{not} a standard
\LaTeXe{} |.sty| style file! Therefore it needs to be invoked in
a non-standard way.

%%%%%%%%%%%%%%%%%%%%%%%%%%%%%%%%%%%%%%%%%%%%%%%%%%%%%%%%%%%%%%%%%%%%%%%%%%%%%%%%
\subsection{Included Files}
\label{sec:include}

%%%%%%%%%%%%%%%%%%%%%%%%%%%%%%%%%%%%%%%%
\DescribeMacro{\childdocmain}
To use the package, add the commands
\begin{center}
\begin{tabular}{l}
|\input{childdoc.def}|\\
|\childdocmain{}|\\
\end{tabular}
\end{center}
at the very top of the main \LaTeX{} file,
in particular \emph{before} the |\documentclass| statement!
The argument of |\childdocmain| should be left empty
(but it must be present).

%%%%%%%%%%%%%%%%%%%%%%%%%%%%%%%%%%%%%%%%
\DescribeMacro{\childdocof}
Furthermore, add the commands
\begin{center}
\begin{tabular}{l}
|\input{childdoc.def}|\\
|\childdocof{|\textit{main}|}|\\
\end{tabular}
\end{center}
at the top of every child file \textit{child}
which is included by |\include{|\textit{child}|}|
from within the main file
(or at least for those files to be compiled individually).
The argument \textit{main} must be the filename of the main file.

There are a couple of
considerations in setting up the main and child documents:

%%%%%%%%%%%%%%%%%%%%%%%%%%%%%%%%%%%%%%%%
\paragraph{Restrictions.}

Please note the following restrictions:
\begin{itemize}
\item
|\childdocmain| must be called with one argument \textit{main}
to ensure compatibility with earlier version of the package.
It must either be empty (|\childdocmain{}|)
or precisely match the filename of the main file in which it is specified.
See \secref{sec:detection} for further information.
\item
The filename \textit{main} must be specified without the |.tex| extension.
\item
The filename \textit{main} is case sensitive
(even in case-insensitive file systems)
due to internal string comparison.
\item
The argument \textit{main} should be fully expanded, it cannot be a macro.
\item
Subdirectories and special characters should be avoided in filenames.
\item
The command |\childdocmain{|\textit{main}|}| must be followed by a whitespace.
It should not be followed immediately by another command
or by a comment mark `|%|'.
This is because the \TeX{} parser reads the token immediately following
the argument of |\childdocmain| and puts it
at the beginning of every child section;
however, a white\-space is ignored.
\end{itemize}

%%%%%%%%%%%%%%%%%%%%%%%%%%%%%%%%%%%%%%%%
\paragraph{Content of Main File.}

It is advisable to place all content in the child files included by |\include|.
Any output contained in the main file will appear in all child documents
unless suppressed manually;
it cannot be suppressed automatically by the |\includeonly| directive
and thus should normally be avoided.
A method to include some content in the main file
by means of conditional processing is described in \secref{sec:conditional}.

%%%%%%%%%%%%%%%%%%%%%%%%%%%%%%%%%%%%%%%%
\paragraph{Page Numbering.}

When only a part of the document is compiled,
the appropriate numbering of pages
(as well as other status parameters)
is determined from the |.aux| files.
The latter contain information from previous passes.
However this information needs to propagate through
all intermediate child documents.
Therefore the page numbering in child documents may well
be inconsistent until the complete document is compiled at least once.

A useful (if unconventional) way to always ensure a consistent
page numbering is to restart the numbering in each child document
and denote the pages by `\textit{child}|.|\textit{page}'
where \textit{child} represents the chapter/section number of the child file.
This can be achieved by the command
|\numberwithin{page}{|\textit{child}|}|
of the \textsf{amsmath} package
where \textit{child} can be |chapter| or |section|
depending on the chosen structuring.
Alternatively, one can modify the macro |\thepage| appropriately
and reset the counter |page| at the start of each child file.

%%%%%%%%%%%%%%%%%%%%%%%%%%%%%%%%%%%%%%%%%%%%%%%%%%%%%%%%%%%%%%%%%%%%%%%%%%%%%%%%
\subsection{Conditional Processing}
\label{sec:conditional}

The package provides a mechanism to compile different versions
of a document. To customise the versions further some conditional processing
can come in handy to distinguish which version is being compiled.
The package provides two macros to describe the compilation context:

%%%%%%%%%%%%%%%%%%%%%%%%%%%%%%%%%%%%%%%%
\DescribeMacro{\ifchilddoc}
The conditional |\ifchilddoc| distinguishes between the compilation of
child documents and the main document:
%
\begin{center}
|\ifchilddoc |\textit{child-code}| |[|\||else |\textit{main-code}]| \||fi|
\end{center}

%%%%%%%%%%%%%%%%%%%%%%%%%%%%%%%%%%%%%%%%
\DescribeMacro{\childdocname}
\DescribeMacro{\childdocjob}
The macro |\childdocname| contains the filename (without extension)
of the main or child file being processed.
Note that |\childdocjob| will always contain the name of the main file.

%%%%%%%%%%%%%%%%%%%%%%%%%%%%%%%%%%%%%%%%
\paragraph{Title Page.}

Conditional processing can be used to include a title or banner page
in the main document when proper precautions are taken.
Importantly, the code in the main file should ensure that the page counter
(as well as other status parameters which are stored in the |.aux| files)
takes the same value after the conditional processing.
Otherwise the page numbers may take divergent values
depending on which part is compiled.

For example, a title page could be declared by:
%
\begin{center}
\begin{tabular}{l}
|\ifchilddoc\||else|\\
|\addtocounter{page}{-1}|\\
\textit{code for title page}\\
|\newpage|\\
|\||fi|
\end{tabular}
\end{center}
%
A banner page for the child documents can be generated by:
%
\begin{center}
\begin{tabular}{l}
|\ifchilddoc|\\
|\addtocounter{page}{-1}|\\
\textit{code for banner page}\\
|\newpage|\\
|\||fi|
\end{tabular}
\end{center}
%
Here one could write a message such as:
\begin{center}
|This is the part \childdocname{} of \childdocjob{}.|
\end{center}

%%%%%%%%%%%%%%%%%%%%%%%%%%%%%%%%%%%%%%%%%%%%%%%%%%%%%%%%%%%%%%%%%%%%%%%%%%%%%%%%
\subsection{Flags}
\label{sec:flags}

The package makes it easy to generate different versions
of the main or child documents.
To this end compilation flags can be defined
and assigned different default values.
They will be particularly useful in conjunction
with the forwarding mechanism described in \secref{sec:forward}.

For example, it may be useful to have a flag |\version|
which can be set to |draft| or |final|.
The document source will contain some conditional code
depending on the value of |\version|.
Suppose further, the flag should default to |final| for the main file
and to |draft| for child files
which is a natural assignment for editing the document.
This is achieved by placing the following code
in the preamble of the main document
(below the |\childdocmain| directive):
%
\begin{center}
\begin{tabular}{l}
|\ifchilddoc|\\
|\providecommand{\version}{draft}|\\
|\||else|\\
|\providecommand{\version}{final}|\\
|\||fi|
\end{tabular}
\end{center}
%
The definition by |\providecommand| makes sure
that previous definitions are not overwritten.
Further statements |\providecommand{\version}{...}|
can thus be added before the above code to override it.

For the main file, one might add a line
(between |\childdocmain| and the above block)
%
\begin{center}
|%\ifchilddoc\||else\providecommand{\version}{draft}\||fi|
\end{center}
%
which can be uncommented to produce a draft version.
Likewise one can add a line to the very top of a child file
(above the |\childdocof{|\textit{main}|}| directive)
%
\begin{center}
|%\providecommand{\version}{final}|
\end{center}
%
which can be uncommented to produce the final version of this child document.

%%%%%%%%%%%%%%%%%%%%%%%%%%%%%%%%%%%%%%%%%%%%%%%%%%%%%%%%%%%%%%%%%%%%%%%%%%%%%%%%
\subsection{Forwarding}
\label{sec:forward}

Different versions of the main or child documents
using compilation flags as described in \secref{sec:flags}
can be (permanently) stored in different files
for convenient compilation, viewing and distribution.
To this end, the package defines a command
to pass on compilation to a different file:

%%%%%%%%%%%%%%%%%%%%%%%%%%%%%%%%%%%%%%%%
\DescribeMacro{\childdocforward}
The command |\childdocforward| redirects processing to
another source file:
%
\begin{center}
\begin{tabular}{l}
|\input{childdoc.def}|\\
|\childdocforward[|\textit{main}|]{|\textit{dest}|}|\\
\end{tabular}
\end{center}
%
The argument \textit{dest} is the destination file
(without extension).
It should be the main file or one of the child files.
Note that further \textsf{childdoc} directives
such as |\childdocof| and |\childdocforward|
in the indicated file will be processed in this form.
The optional argument \textit{main}
passes on directly to the main file \textit{main}
while pretending to compile the child \textit{dest}.
This form behaves as if \textit{dest}
issues |\childdocof{|\textit{main}|}| right away,
and no further \textsf{childdoc} directives will be processed.

%%%%%%%%%%%%%%%%%%%%%%%%%%%%%%%%%%%%%%%%
\DescribeMacro{\...prefix}
In the alternative form |\childdocforwardprefix|,
%
\begin{center}
\begin{tabular}{l}
|\input{childdoc.def}|\\
|\childdocforwardprefix[|\textit{main}|]{|\textit{prefix}|}{|\textit{dest}|}|
\end{tabular}
\end{center}
%
the destination file is determined by a pattern
depending on the current file:
To make this work, the current file must be called
`{\textit{prefix}\hspace{0.2em}\textit{suffix}}'
with \textit{prefix} matching precisely the argument.
Processing is then passed on to the file
`{\textit{dest}\hspace{0.2em}\textit{suffix}}'.
Surely, the same effect is achieved by
directly specifying the
argument `{\textit{dest}\hspace{0.2em}\textit{suffix}}'
in the first form.
However, that requires to set up a different file
for each child. With the alternative form of the command
all these files can have exactly the same content
which simplifies setting them up and maintaining them.

For example, the following file |draft.tex|
with a compilation flag |\version| as described in \secref{sec:flags}
compiles the main document as a draft:
%
\begin{center}
\begin{tabular}{l}
|\def\version{draft}|\\
|\input{childdoc.def}|\\
|\childdocforward{|\textit{main}|}|
\end{tabular}
\end{center}
%
Likewise, the following files |final|\textit{nn}|.tex|
compile the final version of the child document
|child|\textit{nn}|.tex|:
%
\begin{center}
\begin{tabular}{l}
|\def\version{final}|\\
|\input{childdoc.def}|\\
|\childdocforwardprefix{final}{child}|
\end{tabular}
\end{center}
%

Note that when several versions of a main file and/or of each child file
are to be generated, it may be convenient to set up a |Makefile| or
shell script to automatise the process.

%%%%%%%%%%%%%%%%%%%%%%%%%%%%%%%%%%%%%%%%%%%%%%%%%%%%%%%%%%%%%%%%%%%%%%%%%%%%%%%%
\subsection{Command Line Processing}
\label{sec:commandline}

The effect of redirection files can also be achieved by invoking
the \LaTeX{} compiler with a more elaborate command line.
Most conveniently this should be done as part
of a shell script or a |Makefile|.

When using \textsf{childdoc} in the main file, the following
command lines effectively perform a redirection
(note that depending on the shell being used,
backslashes may have to be doubled: `|\|' $\to$ `|\\|'):
%
\begin{center}
|... -jobname "|\textit{target}|" |\\|"|[\textit{flags}]%
|\input{childdoc.def}\childdocforward[|\textit{main}|]{|\textit{dest}|}"|
\end{center}
%
Here \textit{target} is the name of the output file,
\textit{main} is the name of the main file
and \textit{dest} is the name of the main or child file to be processed
(all filenames without extensions).
The optional argument \textit{main} can be omitted
if \textit{main} matches \textit{dest}.
Optionally, compilation \textit{flags} can be defined via |\def| commands.
This command line makes the \TeX{} engine believe
it is compiling the file \textit{target}
whose content is specified as the latter parameter.
The provided code then forwards the processing to
\textit{main} or \textit{dest} as described in \secref{sec:forward}.

%%%%%%%%%%%%%%%%%%%%%%%%%%%%%%%%%%%%%%%%%%%%%%%%%%%%%%%%%%%%%%%%%%%%%%%%%%%%%%%%
\subsection{Include by Input}
\label{sec:input}

Including child documents by |\include| has some restrictions by design.
Most notably, the content of a child document always occupies
its own set of pages; pages cannot be shared between child documents.
Usually, this behaviour makes perfect sense
because each child document contain an essential part of the document.
However, in some situations it may be desirable to compose
a document from a collection of parts
without having mandatory page breaks between then.
For this case, the package
provides a mechanism to include parts
by |\input| which can also be processed individually.
However, by construction this mechanism
requires manual handling of the content to be output.

%%%%%%%%%%%%%%%%%%%%%%%%%%%%%%%%%%%%%%%%
\DescribeMacro{\ifchilddocmanual}
The main file should be prepared as usual, see \secref{sec:include}.
However, the document body must make a distinction
between processing of an individual part and of the main document, e.g.:
%
\begin{center}
\begin{tabular}{l}
|\ifchilddocmanual|\\
|\input{\childdocname}|\\
|\||else|\\
\textit{document body with }|\input{|\textit{part}|}|\\
|\||fi|
\end{tabular}
\end{center}
%
The conditional |\ifchilddocmanual| is true whenever
a part to be included by |\input| is being compiled,
and the name of the part is stored in |\childdocname|.

%%%%%%%%%%%%%%%%%%%%%%%%%%%%%%%%%%%%%%%%
\DescribeMacro{\childdocby}
Each part to be included by |\input| should start with:
%
\begin{center}
\begin{tabular}{l}
|\input{childdoc.def}|\\
|\childdocby{|\textit{main}|}|\\
\end{tabular}
\end{center}
%
The directive |\childdocby| is similar to |\childdocof|
described in \secref{sec:include},
but the subsequent selection of content must be done manually.
To that end, both |\ifchilddoc| and |\ifchilddocmanual|
will be true upon processing of a part,
and the name of the part is stored in |\childdocname|.
Note that |\jobname| will be set to the filename of the current part
so that each part receives an individual |.aux| file
that does not interfere with the |.aux| file(s) of the main document.
This behaviour can be altered by the alternative form
|\childdocby[*]{|\textit{main}|}| (with a non-empty optional argument)
which uses the |.aux| file of the main document
by setting |\jobname| to \textit{main}.

%%%%%%%%%%%%%%%%%%%%%%%%%%%%%%%%%%%%%%%%%%%%%%%%%%%%%%%%%%%%%%%%%%%%%%%%%%%%%%%%
\subsection{Driver Development}
\label{sec:driver}

The \textsf{childdoc} mechanism can also be use for the development
of definition files such as \LaTeX{} styles or classes.
This case differs from the above setup with multiple parts
included by |\include| in that no |\includeonly| should be invoked.
This can be achieved by starting the include file
(before |\ProvidesPackage|) with:
%
\begin{center}
\begin{tabular}{l}
|\input{childdoc.def}|\\
|\childdocforward{|\textit{main}|}|\\
\end{tabular}
\end{center}
%
or alternatively with:
%
\begin{center}
\begin{tabular}{l}
|\input{childdoc.def}|\\
|\childdocby{|\textit{main}|}|\\
\end{tabular}
\end{center}
%
Both forms have slightly different effects as described above.
The main file is prepared as usual, see \secref{sec:include}.

%%%%%%%%%%%%%%%%%%%%%%%%%%%%%%%%%%%%%%%%%%%%%%%%%%%%%%%%%%%%%%%%%%%%%%%%%%%%%%%%
\subsection{Legacy Detection}
\label{sec:detection}

The directive |\childdocmain| in the main file can detect
whether the complete document or merely a child is to be compiled
even without using the directive |\childdocof|.
This method is deprecated because it is less robust
and there is no compelling reason to use it;
it is merely provided for backward compatibility
and it may be removed in future versions.

If the detection mechanism is to be used,
it is mandatory to correctly specify
the filename of the main file as the argument of |\childdocmain|:
%
\begin{center}
\begin{tabular}{l}
|\input{childdoc.def}|\\
|\childdocmain{|\textit{main}|}|\\
\end{tabular}
\end{center}
%
If |\jobname| does not match the argument \textit{main} of |\childdocmain|,
it is assumed that |\jobname| points to the child file to be compiled.
When using |\childdocmain| with the main file specified as argument,
it suffices to start a child file
with just |\input{|\textit{main}|}|
without loading of the package and using |\childdocof|.
If instead all processing is done
with the appropriate \textsf{childdoc} directives,
the argument of \textit{main} of |\childdocmain| can be empty.

An alternative version of the command line processing described
in \secref{sec:commandline} using the detection mechanism reads:
%
\begin{center}
|... -jobname "|\textit{target}|" "|[\textit{flags}]%
[|\def\jobname{|\textit{dest}|}|]|\input{|\textit{main}|}"|
\end{center}

%%%%%%%%%%%%%%%%%%%%%%%%%%%%%%%%%%%%%%%%%%%%%%%%%%%%%%%%%%%%%%%%%%%%%%%%%%%%%%%%
\subsection{Manual Code}
\label{sec:manual}

In case one cannot be certain whether the definitions file |childdoc.def|
is installed on the target \TeX{} distribution
and one prefers not to ship it,
it is conceivable to paste a few relevant commands into the sources.

To that end, drop all statements |\input{childdoc.def}|
and perform the replacements as outlined below.
Instead of |\childdocmain{|\textit{main}|}| add the following code
to the top of the main file:
%
\begin{center}
\begin{tabular}{l}
|\||ifdefined\childdocname\endinput\||fi\newif\ifchilddoc|\\
|\edef\childdocname{\scantokens\expandafter{\jobname\noexpand}}|\\
|\def\childdocmain{|\textit{main}|}\||ifx\childdocmain\childdocname\||else|\\
|\childdoctrue\includeonly{\childdocname}\let\jobname\childdocmain\||fi|\\
\end{tabular}
\end{center}
%
Instead of |\childdocof{|\textit{main}|}| just include the main file
at the top of each child file:
%
\begin{center}
|\input{|\textit{main}|}|
\end{center}
%
A simple redirection |\childdocforward{|\textit{dest}|}| is achieved by:
%
\begin{center}
|\def\jobname{|\textit{dest}|}\input{\jobname}|
\end{center}
%
The redirection with prefix
|\childdocforwardprefix[|\textit{prefix}|]{|\textit{dest}|}|
is accomplished by:
%
\begin{center}
\begin{tabular}{l}
|{\edef\jobname{\scantokens\expandafter{\jobname\noexpand}}|\\
|\def\redirectjob |\textit{prefix}|#1~~~{\gdef\jobname{|\textit{dest}|#1}}|\\
|\expandafter\redirectjob\jobname~~~}\input{\jobname}|
\end{tabular}
\end{center}

In an alternative approach,
child documents can be compiled by a specific command line
without additional code or specific definitions:
%
\begin{center}
|... -jobname "|\textit{target}|" "|[\textit{flags}]%
|\includeonly{|\textit{dest}|}\input{|\textit{main}|}"|
\end{center}
%

%%%%%%%%%%%%%%%%%%%%%%%%%%%%%%%%%%%%%%%%%%%%%%%%%%%%%%%%%%%%%%%%%%%%%%%%%%%%%%%%
%%%%%%%%%%%%%%%%%%%%%%%%%%%%%%%%%%%%%%%%%%%%%%%%%%%%%%%%%%%%%%%%%%%%%%%%%%%%%%%%
\section{Information}

%%%%%%%%%%%%%%%%%%%%%%%%%%%%%%%%%%%%%%%%%%%%%%%%%%%%%%%%%%%%%%%%%%%%%%%%%%%%%%%%
\subsection{Copyright}

Copyright \copyright{} 2017--2018 Niklas Beisert

This work may be distributed and/or modified under the
conditions of the \LaTeX{} Project Public License, either version 1.3
of this license or (at your option) any later version.
The latest version of this license is in
  \url{http://www.latex-project.org/lppl.txt}
and version 1.3 or later is part of all distributions of \LaTeX{}
version 2005/12/01 or later.

This work has the LPPL maintenance status `maintained'.

The Current Maintainer of this work is Niklas Beisert.

This work consists of the files |README.txt|, |childdoc.ins| and |childdoc.dtx|
as well as the derived files |childdoc.def|, |cdocsamp.tex|
with |cdocsch1.tex|, |cdocsch2.tex|, |cdocspt3.tex|, |cdocspt4.tex|,
|cdocsdrf.tex|, |cdocsfn1.tex|, |cdocsfn2.tex|
as well as |childdoc.pdf|.

%%%%%%%%%%%%%%%%%%%%%%%%%%%%%%%%%%%%%%%%%%%%%%%%%%%%%%%%%%%%%%%%%%%%%%%%%%%%%%%%
\subsection{Files and Installation}

The package consists of the files:
%
\begin{center}
\begin{tabular}{ll}
    |README.txt|   & readme file \\
    |childdoc.ins| & installation file \\
    |childdoc.dtx| & source file \\
    |childdoc.def| & definition file \\
    |cdocsamp.tex| & sample main file \\
    |cdocsch1.tex| & sample include file \\
    |cdocsch2.tex| & sample include file \\
    |cdocspt3.tex| & sample part file \\
    |cdocspt4.tex| & sample part file \\
    |cdocsdrf.tex| & sample redirection file \\
    |cdocsfn1.tex| & sample redirection file \\
    |cdocsfn2.tex| & sample redirection file \\
    |childdoc.pdf| & manual
\end{tabular}
\end{center}
%
The distribution consists of the files
|README.txt|, |childdoc.ins| and |childdoc.dtx|.
%
\begin{itemize}
\item
Run (pdf)\LaTeX{} on |childdoc.dtx|
to compile the manual |childdoc.pdf| (this file).
\item
Run \LaTeX{} on |childdoc.ins| to create the definitions file |childdoc.def|
and the sample |cdocsamp.tex| with include files
|cdocsch1.tex|, |cdocsch2.tex|, |cdocspt3.tex|, |cdocspt4.tex|,
|cdocsdrf.tex|, |cdocsfn1.tex|, |cdocsfn2.tex|.
Then copy the file |childdoc.def| to an appropriate directory of your \LaTeX{}
distribution, e.g.\ \textit{texmf-root}|/tex/latex/childdoc|.
\end{itemize}

%%%%%%%%%%%%%%%%%%%%%%%%%%%%%%%%%%%%%%%%%%%%%%%%%%%%%%%%%%%%%%%%%%%%%%%%%%%%%%%%
\subsection{Related CTAN Packages}

There are several other packages which offer a similar functionality:
%
\begin{itemize}
\item
The packages
\href{http://ctan.org/pkg/docmute}{\textsf{docmute}},
\href{http://ctan.org/pkg/includex}{\textsf{includex}} and
\href{http://ctan.org/pkg/standalone}{\textsf{standalone}}
provide commands to include only the document body of
a child file thus allowing both files to be compiled individually.
\item
The packages \href{http://ctan.org/pkg/subdocs}{\textsf{subdocs}}
and \href{http://ctan.org/pkg/subfiles}{\textsf{subfiles}}
provide structures in which the main and child documents can be
encapsulated and allowing them to be compiled individually.
The inclusion mechanism is different from the conventional |\include|.
\item
The package \href{http://ctan.org/pkg/combine}{\textsf{combine}}
is an elaborate solution to combine several documents into one.
\end{itemize}
%
See also the CTAN topic \href{http://ctan.org/topic/subdocs}{\textsf{subdocs}}
for further related packages.
The present package differs from the above solutions in that
a document structure constructed with the conventional |\include| mechanism
just needs two extra commands at the top of every file
such that all constituent files can be compiled individually.

%%%%%%%%%%%%%%%%%%%%%%%%%%%%%%%%%%%%%%%%%%%%%%%%%%%%%%%%%%%%%%%%%%%%%%%%%%%%%%%%
%\subsection{Feature Suggestions}
%
%The following is a list of features which may be useful for future
%versions of this package:
%%
%\begin{itemize}
%\item
%\ldots
%\end{itemize}

%%%%%%%%%%%%%%%%%%%%%%%%%%%%%%%%%%%%%%%%%%%%%%%%%%%%%%%%%%%%%%%%%%%%%%%%%%%%%%%%
\subsection{Revision History}

%%%%%%%%%%%%%%%%%%%%%%%%%%%%%%%%%%%%%%%%
\paragraph{v2.0:} 2018/12/30

\begin{itemize}
\item
immediate forward processing
\item
added |\childdocby| mechanism
\item
manual restructured
\end{itemize}

%%%%%%%%%%%%%%%%%%%%%%%%%%%%%%%%%%%%%%%%
\paragraph{v1.6:} 2018/01/17

\begin{itemize}
\item
application for development of include files
\item
corrections to manual
\end{itemize}

%%%%%%%%%%%%%%%%%%%%%%%%%%%%%%%%%%%%%%%%
\paragraph{v1.5:} 2017/05/21

\begin{itemize}
\item
more complete structuring introduced
\item
|\childdocof| introduced
\item
|\childdoc| renamed to |\childdocmain|
\item
|\childredirect| renamed to |\childdocforward| and |\childdocforwardprefix|
and functionality expanded
\end{itemize}

%%%%%%%%%%%%%%%%%%%%%%%%%%%%%%%%%%%%%%%%
\paragraph{v1.0:} 2017/04/27

\begin{itemize}
\item
manual and install package
\item
first version published on CTAN
\end{itemize}

%%%%%%%%%%%%%%%%%%%%%%%%%%%%%%%%%%%%%%%%
\paragraph{v0.6:} 2017/04/26

\begin{itemize}
\item
redirection mechanism added
\end{itemize}

%%%%%%%%%%%%%%%%%%%%%%%%%%%%%%%%%%%%%%%%
\paragraph{v0.5:} 2017/04/26

\begin{itemize}
\item
functionality in definition file
\end{itemize}


%%%%%%%%%%%%%%%%%%%%%%%%%%%%%%%%%%%%%%%%%%%%%%%%%%%%%%%%%%%%%%%%%%%%%%%%%%%%%%%%
%%%%%%%%%%%%%%%%%%%%%%%%%%%%%%%%%%%%%%%%%%%%%%%%%%%%%%%%%%%%%%%%%%%%%%%%%%%%%%%%
%%%%%%%%%%%%%%%%%%%%%%%%%%%%%%%%%%%%%%%%%%%%%%%%%%%%%%%%%%%%%%%%%%%%%%%%%%%%%%%%
\appendix

\settowidth\MacroIndent{\rmfamily\scriptsize 000\ }

 \DocInput{childdoc.dtx}

\end{document}
%</driver>
% \fi
%
% %%%%%%%%%%%%%%%%%%%%%%%%%%%%%%%%%%%%%%%%%%%%%%%%%%%%%%%%%%%%%%%%%%%%%%%%%%%%%%
% %%%%%%%%%%%%%%%%%%%%%%%%%%%%%%%%%%%%%%%%%%%%%%%%%%%%%%%%%%%%%%%%%%%%%%%%%%%%%%
% \section{Sample}
%\iffalse
%<*samplemain>
%\fi
%
% The following presents a sample document
% with two chapters, two parts, a title page,
% a compile flag as well as three forwarding files to set the flag.
% It consists of eight |.tex| files:
% \begin{center}
% \begin{tabular}{ll}
% |cdocsamp.tex|&main file\\
% |cdocsch1.tex|&include file for chapter 1\\
% |cdocsch2.tex|&include file for chapter 2\\
% |cdocspt3.tex|&include file for part 3\\
% |cdocspt4.tex|&include file for part 4\\
% |cdocsdrf.tex|&forwarding file for main file in draft mode\\
% |cdocsfi1.tex|&forwarding file for final version of chapter 1\\
% |cdocsfi2.tex|&forwarding file for final version of chapter 2\\
% \end{tabular}
% \end{center}
% Each of the eight files can be compiled directly by the \LaTeX{} compiler.
%
% %%%%%%%%%%%%%%%%%%%%%%%%%%%%%%%%%%%%%%
% \paragraph{Main File.}
%
% The main file is called |cdocsamp.tex|.
%
% Load the \textsf{childdoc} definitions and
% declare the filename for the main document:
%    \begin{macrocode}
\input{childdoc.def}
\childdocmain{}
%    \end{macrocode}

% Optional override for |\version| flag:
%    \begin{macrocode}
%%\ifchilddoc\else\providecommand{\version}{draft}\fi
%    \end{macrocode}

% Define the default values for the |\version| flag
% (|final| for the main file and |draft| for childs):
%    \begin{macrocode}
\ifchilddoc
\providecommand{\version}{draft}
\else
\providecommand{\version}{final}
\fi
%    \end{macrocode}

% Load the standard document class:
%    \begin{macrocode}
\documentclass[12pt]{article}
%    \end{macrocode}

% Start the document body:
%    \begin{macrocode}
\begin{document}
%    \end{macrocode}

% Declare a title page.
% Print title, part of document being processed and version flag:
%    \begin{macrocode}
\addtocounter{page}{-1}
\begin{center}
{\LARGE\bfseries{}childdoc example\par}
\vspace{1cm}
\ifchilddoc
\ifchilddocmanual part\else chapter\fi:
`\childdocname' of `\childdocjob'\par
\else
main document: `\childdocjob'\par
\fi
version: \version\par
\end{center}
\newpage
%    \end{macrocode}

% Manually include selected file,
% otherwise process as usual:
%    \begin{macrocode}
\ifchilddocmanual
\section*{part `\childdocname'}
\input{\childdocname}
\else
%    \end{macrocode}

% Include the two chapters:
%    \begin{macrocode}
\include{cdocsch1}
\include{cdocsch2}
%    \end{macrocode}

% Include the two parts unless only chapters should be displayed:
%    \begin{macrocode}
\ifchilddoc\else
\section{part three}
\input{cdocspt3}
\section{part four}
\input{cdocspt4}
\fi
%    \end{macrocode}

% Process as usual until here:
%    \begin{macrocode}
\fi
%    \end{macrocode}

% End of document body:
%    \begin{macrocode}
\end{document}
%    \end{macrocode}
%\iffalse
%</samplemain>
%\fi
%
% %%%%%%%%%%%%%%%%%%%%%%%%%%%%%%%%%%%%%%
% \paragraph{Chapter Include Files.}
%
% The include files are called |cdocsch1.tex| and |cdocsch2.tex|.
%
%\iffalse
%<*samplechap1|samplechap2>
%\fi

% Optional override for |\version| flag:
%    \begin{macrocode}
%%\providecommand{\version}{final}
%    \end{macrocode}

% Include the main document:
%    \begin{macrocode}
\input{childdoc.def}
\childdocof{cdocsamp}
%    \end{macrocode}

%\iffalse
%</samplechap1|samplechap2>
%\fi
%
%\iffalse
%<*samplechap1>
%\fi
% Some text for chapter 1:
%    \begin{macrocode}
\section{one}
some text in chapter one
%    \end{macrocode}

%\iffalse
%</samplechap1>
%\fi
% Some text for chapter 2:
%\iffalse
%<*samplechap2>
%\fi
%    \begin{macrocode}
\section{two}
more text in chapter two
%    \end{macrocode}

%\iffalse
%</samplechap2>
%\fi
%
% %%%%%%%%%%%%%%%%%%%%%%%%%%%%%%%%%%%%%%
% \paragraph{Part Include Files.}
%
% The include files are called |cdocspt3.tex| and |cdocspt4.tex|.
%
%\iffalse
%<*samplepart3|samplepart4>
%\fi

% Optional override for |\version| flag:
%    \begin{macrocode}
%%\providecommand{\version}{final}
%    \end{macrocode}

% Include the main document:
%    \begin{macrocode}
\input{childdoc.def}
\childdocby{cdocsamp}
%    \end{macrocode}

%\iffalse
%</samplepart3|samplepart4>
%\fi
%
%\iffalse
%<*samplepart3>
%\fi
% Some text for part 3:
%    \begin{macrocode}
some text in part three
%    \end{macrocode}

%\iffalse
%</samplepart3>
%\fi
% Some text for part 4:
%\iffalse
%<*samplepart4>
%\fi
%    \begin{macrocode}
more text in part four
%    \end{macrocode}

%\iffalse
%</samplepart4>
%\fi
%
% %%%%%%%%%%%%%%%%%%%%%%%%%%%%%%%%%%%%%%
% \paragraph{Forwarding for a Complete Draft.}
%
% The following forwarding file |cdocsdrf.tex|
% compiles the main document in draft mode:
%\iffalse
%<*sampledraft>
%\fi
%    \begin{macrocode}
\def\version{draft}
\input{childdoc.def}
\childdocforward{cdocsamp}
%    \end{macrocode}

%\iffalse
%</sampledraft>
%\fi
%
% %%%%%%%%%%%%%%%%%%%%%%%%%%%%%%%%%%%%%%
% \paragraph{Forwarding for Final Version of the Chapters.}
%
% The following forwarding files |cdocsfn1.tex| and |cdocsfn2.tex|
% (with identical content)
% compile the final versions of the child documents
% |cdocsch1.tex| and |cdocsch2.tex|, respectively:
%\iffalse
%<*samplefinal>
%\fi
%    \begin{macrocode}
\def\version{final}
\input{childdoc.def}
\childdocforwardprefix[cdocsamp]{cdocsfn}{cdocsch}
%    \end{macrocode}

%\iffalse
%</samplefinal>
%\fi
%
% %%%%%%%%%%%%%%%%%%%%%%%%%%%%%%%%%%%%%%
% \paragraph{Command Line Processing.}
%
% The following three command lines generate the output files
% |cdocscld|, |cdocscl1| and |cdocscl2|
% which should be identical to
% |cdocsdrf|, |cdocsch1| and |cdocsfn2|, respectively:
% \begin{center}
% \begin{tabular}{l}
% |latex -jobname cdocscld \|\\
% |  "\def\version{draft}\input{childdoc.def}\childdocforward{cdocsamp}"|\\
% |latex -jobname cdocscl1 \|\\
% |  "\input{childdoc.def}\childdocforward[cdocsamp]{cdocsch1}"|\\
% |latex -jobname cdocscl2 \|\\
% |  "\def\version{final}\input{childdoc.def}\childdocforward{cdocsch2}"|
% \end{tabular}
% \end{center}
% Note that the trailing backslash on each first line
% merely continues the input to the second line
% (for convenient cut ant paste).
% Furthermore, the command |latex| can be replaced by any
% of its alternative versions such as |pdflatex|.
%
% %%%%%%%%%%%%%%%%%%%%%%%%%%%%%%%%%%%%%%%%%%%%%%%%%%%%%%%%%%%%%%%%%%%%%%%%%%%%%%
% %%%%%%%%%%%%%%%%%%%%%%%%%%%%%%%%%%%%%%%%%%%%%%%%%%%%%%%%%%%%%%%%%%%%%%%%%%%%%%
% \section{Implementation}
%\iffalse
%<*package>
%\fi
%
% This section describes the definitions file |childdoc.def|.

% The definitions cannot be loaded using |\usepackage| or |\RequirePackage|
% which has a mechanism to prevent loading a style file more than once.
% When loading the definitions by means of |\input|
% multiple instances have to be prevented manually:
%\iffalse
%This code needs to be before the `\ProvidesFile' directive
%which is defined at the beginning of this file.
%Therefore it is also placed there and commented out here.
%</package>
%<*discard>
%\fi
%    \begin{macrocode}
\ifdefined\childdocmain\endinput\fi
%    \end{macrocode}
%\iffalse
%</discard>
%<*package>
%\fi
%
% \macro{\ifchilddoc}
% \macro{\ifchilddocmanual}
% The conditional |\ifchilddoc| tells whether a
% child (true) or main (false) document is being compiled.
% The conditional |\ifchilddocmanual| tells whether
% the |\includeonly| mechanism is used (false) or
% the selection of child files must be performed manually (true).
% The definitions initialise to false:
%    \begin{macrocode}
\newif\ifchilddoc
\newif\ifchilddocmanual
%    \end{macrocode}

% \macro{\childdocname}
% \macro{\childdocjob}
% The macro |\childdocname| stores the name of the main document
% to be compiled. The macro |\childdocjob| stores the name of
% the document on which the \LaTeX{} compiler was originally invoked.
% The content of |\jobname| cannot be compared
% to filenames specified in the source due to different catcodes.
% The following code rescans |\jobname|, stores the result
% in |\childdocname| and saves a copy in |\childdocjob|:
%    \begin{macrocode}
\edef\childdocname{\scantokens\expandafter{\jobname\noexpand}}
\let\childdocjob\childdocname
%    \end{macrocode}

% \macro{\childdocdisable}
% The macro |\childdocdisable| prevents the main file
% from being processed more than once.
% At this stage, the main document command |\childdocmain|
% is assumed to be called once again where it should do nothing.
% Any subsequent call to it should prevent
% a secondary processing of the main document
% It overwrites the forwarding commands
% |\childdocof| and |\childdocforward|
% with empty macros to prevent further inclusions of the main document:
%    \begin{macrocode}
\newcommand{\childdocdisable}
{
  \renewcommand{\childdocmain}[1]{\renewcommand{\childdocmain}[1]{\endinput}}
  \renewcommand{\childdocof}[1]{}
  \renewcommand{\childdocby}[2][]{}
  \renewcommand{\childdocforward}[2][]{}
  \renewcommand{\childdocdisable}{}
}
%    \end{macrocode}

% \macro{\childdocmain}
% The macro |\childdocmain| is to be called at the top of the main file
% with nothing or the main filename (without extension) as argument.
% First, it breaks loops.
% If the argument is not empty and does not match |\childdocname|
% (which is set by the first inclusion of |childdoc.def|),
% |\ifchilddoc| is set to true, |\includeonly| is applied to the child file
% and |\jobname| is set to the main file
% (for proper handling of |.aux| files):
%    \begin{macrocode}
\newcommand{\childdocmain}[1]
{
  \childdocdisable\childdocmain{}
  \if?#1?\else
    \begingroup
      \def\childdoctmp{#1}
      \ifx\childdoctmp\childdocname
        \def\childdoctmp{}
      \else
        \def\childdoctmp
        {
          \childdoctrue
          \includeonly{\childdocname}
          \def\childdocjob{#1}
          \def\jobname{#1}
        }
      \fi
      \expandafter
    \endgroup
    \childdoctmp
  \fi
}
%    \end{macrocode}

% \macro{\childdocof}
% The command |\childdocof| redirects
% compilation to the main file |#1|.
%    \begin{macrocode}
\newcommand{\childdocof}[1]
{
  \childdocdisable
  \childdoctrue
  \includeonly{\childdocname}
  \def\jobname{#1}
  \def\childdocjob{#1}
  \input{#1}
}
%    \end{macrocode}

% \macro{\childdocby}
% The command |\childdocby| ....
%    \begin{macrocode}
\newcommand{\childdocby}[2][]
{
  \childdocdisable
  \childdoctrue
  \childdocmanualtrue
  \if?#1?\else
    \def\jobname{#2}
  \fi
  \def\childdocjob{#2}
  \input{#2}
  \endinput
}
%    \end{macrocode}

% \macro{\childdocforward}
% The command |\childdocforward| redirects
% compilation to the main file or
% (if the optional argument is given) a child file.
% Parameters are set as if the main file
% or a child file starting with |\childdocof| was compiled.
% Then compilation is handed over to the main file:
%    \begin{macrocode}
\newcommand{\childdocforward}[2][]
{
  \begingroup
    \if?#1?
      \def\childdoctmp
      {
        \def\childdocname{#2}
        \def\childdocjob{#2}
        \def\jobname{#2}
        \input{#2}
        \endinput
      }
    \else
      \def\childdoctmp
      {
        \childdocdisable
        \def\childdocname{#2}
        \childdoctrue
        \includeonly{#2}
        \def\childdocjob{#1}
        \def\jobname{#1}
        \input{#1}
        \endinput
      }
    \fi
    \expandafter
  \endgroup
  \childdoctmp
}
%    \end{macrocode}

% \macro{\childdocforwardprefix}
% The command |\childdocforwardprefix| redirects
% compilation to the main or a child file by means of a pattern.
% The prefix |#1| in the current filename is replaced by |#2|
% and the suffix of the current filename is kept
% (it is assumed that the filename does not contain the substring `|~~~|'
% which is used as a delimiter).
% Compilation is handed over to the new file by |\childdocforward|:
%    \begin{macrocode}
\newcommand{\childdocforwardprefix}[3][]
{
  \begingroup
    \def\childdocextract #2##1~~~{\def\childdoctmp{\childdocforward[#1]{#3##1}}}
    \expandafter\childdocextract\childdocname~~~
    \expandafter
  \endgroup
  \childdoctmp
}
%    \end{macrocode}

% \macro{\childdoc}
% The deprecated macro |\childdoc| is a legacy version of |\childdocmain|:
%    \begin{macrocode}
\newcommand{\childdoc}{\childdocmain}
%    \end{macrocode}

% \macro{\childdocredirect}
% The deprecated macro |\childdocredirect| is a legacy version
% of |\childdocforward| and |\childdocforwardprefix|:
%    \begin{macrocode}
\newcommand{\childdocredirect}[2][]
{
  \begingroup
    \if?#1?
      \def\childdoctmp{\childdocforward{#2}}
    \else
      \def\childdoctmp{\childdocforwardprefix{#1}{#2}}
    \fi
    \expandafter
  \endgroup
  \childdoctmp
}
%    \end{macrocode}

%\iffalse
%</package>
%\fi
%
\endinput

\childdocby{cdocsamp}
%    \end{macrocode}

%\iffalse
%</samplepart3|samplepart4>
%\fi
%
%\iffalse
%<*samplepart3>
%\fi
% Some text for part 3:
%    \begin{macrocode}
some text in part three
%    \end{macrocode}

%\iffalse
%</samplepart3>
%\fi
% Some text for part 4:
%\iffalse
%<*samplepart4>
%\fi
%    \begin{macrocode}
more text in part four
%    \end{macrocode}

%\iffalse
%</samplepart4>
%\fi
%
% %%%%%%%%%%%%%%%%%%%%%%%%%%%%%%%%%%%%%%
% \paragraph{Forwarding for a Complete Draft.}
%
% The following forwarding file |cdocsdrf.tex|
% compiles the main document in draft mode:
%\iffalse
%<*sampledraft>
%\fi
%    \begin{macrocode}
\def\version{draft}
% \iffalse
%
% childdoc.dtx Copyright (C) 2017-2018 Niklas Beisert
%
% This work may be distributed and/or modified under the
% conditions of the LaTeX Project Public License, either version 1.3
% of this license or (at your option) any later version.
% The latest version of this license is in
%   http://www.latex-project.org/lppl.txt
% and version 1.3 or later is part of all distributions of LaTeX
% version 2005/12/01 or later.
%
% This work has the LPPL maintenance status `maintained'.
%
% The Current Maintainer of this work is Niklas Beisert.
%
% This work consists of the files childdoc.dtx and childdoc.ins
% and the derived files childdoc.def and cdocsamp.tex with
% cdocsch1.tex, cdocsch2.tex, cdocsdrf.tex, cdocsfn1.tex, cdocsfn2.tex.
%
%<package>\ifdefined\childdocmain\endinput\fi
%<package>\ProvidesFile{childdoc.def}[2018/12/30 v2.0 child document driver]
%<samplemain>\ProvidesFile{cdocsamp.tex}[2018/12/30 v2.0 sample for childdoc]
%<*driver>
%\ProvidesFile{childdoc.drv}[2018/12/30 v2.0 childdoc reference manual file]
\PassOptionsToClass{10pt,a4paper}{article}
\documentclass{ltxdoc}

\usepackage[margin=35mm]{geometry}
\usepackage{hyperref}
\usepackage{hyperxmp}
\usepackage[usenames]{color}

\hypersetup{colorlinks=true}
\hypersetup{pdfstartview=FitH}
\hypersetup{pdfpagemode=UseNone}
\hypersetup{pdfsource={}}
\hypersetup{pdflang={en-UK}}
\hypersetup{pdfcopyright={Copyright 2017-2018 Niklas Beisert.
  This work may be distributed and/or modified under the
  conditions of the LaTeX Project Public License, either version 1.3
  of this license or (at your option) any later version.}}
\hypersetup{pdflicenseurl={http://www.latex-project.org/lppl.txt}}
\hypersetup{pdfcontactaddress={ETH Zurich, ITP, HIT K,
  Wolfgang-Pauli-Strasse 27}}
\hypersetup{pdfcontactpostcode={8093}}
\hypersetup{pdfcontactcity={Zurich}}
\hypersetup{pdfcontactcountry={Switzerland}}
\hypersetup{pdfcontactemail={nbeisert@itp.phys.ethz.ch}}
\hypersetup{pdfcontacturl={http://people.phys.ethz.ch/\xmptilde nbeisert/}}

\newcommand{\secref}[1]{\hyperref[#1]{section \ref*{#1}}}

\parskip1ex
\parindent0pt
\let\olditemize\itemize
\def\itemize{\olditemize\parskip0pt}

\begin{document}

\title{The \textsf{childdoc} Package}
\hypersetup{pdftitle={The childdoc Package}}
\author{Niklas Beisert\\[2ex]
  Institut f\"ur Theoretische Physik\\
  Eidgen\"ossische Technische Hochschule Z\"urich\\
  Wolfgang-Pauli-Strasse 27, 8093 Z\"urich, Switzerland\\[1ex]
  \href{mailto:nbeisert@itp.phys.ethz.ch}
  {\texttt{nbeisert@itp.phys.ethz.ch}}}
\hypersetup{pdfauthor={Niklas Beisert}}
\hypersetup{pdfsubject={Manual for the LaTeX2e Package childdoc}}
\date{30 December 2018, \textsf{v2.0}}
\maketitle

\begin{abstract}\noindent
\textsf{childdoc} is a \LaTeXe{} package
that enables the direct compilation
of document sections included by |\include|
to individual files.
\end{abstract}

\begingroup
\parskip0ex
\tableofcontents
\endgroup

%%%%%%%%%%%%%%%%%%%%%%%%%%%%%%%%%%%%%%%%%%%%%%%%%%%%%%%%%%%%%%%%%%%%%%%%%%%%%%%%
%%%%%%%%%%%%%%%%%%%%%%%%%%%%%%%%%%%%%%%%%%%%%%%%%%%%%%%%%%%%%%%%%%%%%%%%%%%%%%%%
\section{Introduction}

\LaTeX{} provides a mechanism to structure a large document (such as a book)
into a main file and several child files (containing the chapters)
using the |\include| command.
This mechanism is beneficial for documents
which span hundreds of pages in order to
make the source file(s) more manageable.
Moreover, compilation can be restricted to
selected child files by means of the |\includeonly| command.
The latter feature can be used to reduce the compilation time while editing
(this was significantly more useful in the earlier days of \LaTeX{})
or to generate a smaller document which is easier to navigate.
Another application of |\includeonly| is to generate
documents consisting of selected parts of the complete document.

However, there are a few drawbacks of the plain |\include| mechanism:
\begin{itemize}
\item
The child files cannot be compiled on their own,
they can only be compiled via the main file.
A naive editing environment
(such as a text editor with an option
to have the current file processed by \LaTeX)
may require one to switch to the main file before compiling;
attempting to compile the child file produces errors.
\item
The main file must be modified (each time)
to adjust the |\includeonly| command
to the present needs. This easily leaves the main file in a messy state.
\item
The generated document will always carry the filename
of the main document. This is inconvenient if
several child files are to be compiled and
to be kept for distribution.
\end{itemize}

The present package provides a simple interface
to make child files individually compilable by \LaTeX{}.
Compiling a child file then has the same effect as compiling
the main file with an |\includeonly| command
to select the appropriate child.
Moreover the generated document will carry the name of the child
rather than the main file.
This resolves all three above issues.

This feature is meant to make the editing of books,
thesis documents and lecture notes somewhat more convenient.
However, the package can also be used efficiently for
composing a series of documents (such as exercise sheets)
which are typically distributed individually.
It then assists the author in generating the individual documents
(potentially in different versions)
as well as a document containing the collected series.
Another application is in developing style files
or other kinds of included material
where compilation of the style file could redirect
to a sample or test file.

%%%%%%%%%%%%%%%%%%%%%%%%%%%%%%%%%%%%%%%%%%%%%%%%%%%%%%%%%%%%%%%%%%%%%%%%%%%%%%%%
%%%%%%%%%%%%%%%%%%%%%%%%%%%%%%%%%%%%%%%%%%%%%%%%%%%%%%%%%%%%%%%%%%%%%%%%%%%%%%%%
\section{Usage}

First of all, the package \textsf{childdoc} is \emph{not} a standard
\LaTeXe{} |.sty| style file! Therefore it needs to be invoked in
a non-standard way.

%%%%%%%%%%%%%%%%%%%%%%%%%%%%%%%%%%%%%%%%%%%%%%%%%%%%%%%%%%%%%%%%%%%%%%%%%%%%%%%%
\subsection{Included Files}
\label{sec:include}

%%%%%%%%%%%%%%%%%%%%%%%%%%%%%%%%%%%%%%%%
\DescribeMacro{\childdocmain}
To use the package, add the commands
\begin{center}
\begin{tabular}{l}
|\input{childdoc.def}|\\
|\childdocmain{}|\\
\end{tabular}
\end{center}
at the very top of the main \LaTeX{} file,
in particular \emph{before} the |\documentclass| statement!
The argument of |\childdocmain| should be left empty
(but it must be present).

%%%%%%%%%%%%%%%%%%%%%%%%%%%%%%%%%%%%%%%%
\DescribeMacro{\childdocof}
Furthermore, add the commands
\begin{center}
\begin{tabular}{l}
|\input{childdoc.def}|\\
|\childdocof{|\textit{main}|}|\\
\end{tabular}
\end{center}
at the top of every child file \textit{child}
which is included by |\include{|\textit{child}|}|
from within the main file
(or at least for those files to be compiled individually).
The argument \textit{main} must be the filename of the main file.

There are a couple of
considerations in setting up the main and child documents:

%%%%%%%%%%%%%%%%%%%%%%%%%%%%%%%%%%%%%%%%
\paragraph{Restrictions.}

Please note the following restrictions:
\begin{itemize}
\item
|\childdocmain| must be called with one argument \textit{main}
to ensure compatibility with earlier version of the package.
It must either be empty (|\childdocmain{}|)
or precisely match the filename of the main file in which it is specified.
See \secref{sec:detection} for further information.
\item
The filename \textit{main} must be specified without the |.tex| extension.
\item
The filename \textit{main} is case sensitive
(even in case-insensitive file systems)
due to internal string comparison.
\item
The argument \textit{main} should be fully expanded, it cannot be a macro.
\item
Subdirectories and special characters should be avoided in filenames.
\item
The command |\childdocmain{|\textit{main}|}| must be followed by a whitespace.
It should not be followed immediately by another command
or by a comment mark `|%|'.
This is because the \TeX{} parser reads the token immediately following
the argument of |\childdocmain| and puts it
at the beginning of every child section;
however, a white\-space is ignored.
\end{itemize}

%%%%%%%%%%%%%%%%%%%%%%%%%%%%%%%%%%%%%%%%
\paragraph{Content of Main File.}

It is advisable to place all content in the child files included by |\include|.
Any output contained in the main file will appear in all child documents
unless suppressed manually;
it cannot be suppressed automatically by the |\includeonly| directive
and thus should normally be avoided.
A method to include some content in the main file
by means of conditional processing is described in \secref{sec:conditional}.

%%%%%%%%%%%%%%%%%%%%%%%%%%%%%%%%%%%%%%%%
\paragraph{Page Numbering.}

When only a part of the document is compiled,
the appropriate numbering of pages
(as well as other status parameters)
is determined from the |.aux| files.
The latter contain information from previous passes.
However this information needs to propagate through
all intermediate child documents.
Therefore the page numbering in child documents may well
be inconsistent until the complete document is compiled at least once.

A useful (if unconventional) way to always ensure a consistent
page numbering is to restart the numbering in each child document
and denote the pages by `\textit{child}|.|\textit{page}'
where \textit{child} represents the chapter/section number of the child file.
This can be achieved by the command
|\numberwithin{page}{|\textit{child}|}|
of the \textsf{amsmath} package
where \textit{child} can be |chapter| or |section|
depending on the chosen structuring.
Alternatively, one can modify the macro |\thepage| appropriately
and reset the counter |page| at the start of each child file.

%%%%%%%%%%%%%%%%%%%%%%%%%%%%%%%%%%%%%%%%%%%%%%%%%%%%%%%%%%%%%%%%%%%%%%%%%%%%%%%%
\subsection{Conditional Processing}
\label{sec:conditional}

The package provides a mechanism to compile different versions
of a document. To customise the versions further some conditional processing
can come in handy to distinguish which version is being compiled.
The package provides two macros to describe the compilation context:

%%%%%%%%%%%%%%%%%%%%%%%%%%%%%%%%%%%%%%%%
\DescribeMacro{\ifchilddoc}
The conditional |\ifchilddoc| distinguishes between the compilation of
child documents and the main document:
%
\begin{center}
|\ifchilddoc |\textit{child-code}| |[|\||else |\textit{main-code}]| \||fi|
\end{center}

%%%%%%%%%%%%%%%%%%%%%%%%%%%%%%%%%%%%%%%%
\DescribeMacro{\childdocname}
\DescribeMacro{\childdocjob}
The macro |\childdocname| contains the filename (without extension)
of the main or child file being processed.
Note that |\childdocjob| will always contain the name of the main file.

%%%%%%%%%%%%%%%%%%%%%%%%%%%%%%%%%%%%%%%%
\paragraph{Title Page.}

Conditional processing can be used to include a title or banner page
in the main document when proper precautions are taken.
Importantly, the code in the main file should ensure that the page counter
(as well as other status parameters which are stored in the |.aux| files)
takes the same value after the conditional processing.
Otherwise the page numbers may take divergent values
depending on which part is compiled.

For example, a title page could be declared by:
%
\begin{center}
\begin{tabular}{l}
|\ifchilddoc\||else|\\
|\addtocounter{page}{-1}|\\
\textit{code for title page}\\
|\newpage|\\
|\||fi|
\end{tabular}
\end{center}
%
A banner page for the child documents can be generated by:
%
\begin{center}
\begin{tabular}{l}
|\ifchilddoc|\\
|\addtocounter{page}{-1}|\\
\textit{code for banner page}\\
|\newpage|\\
|\||fi|
\end{tabular}
\end{center}
%
Here one could write a message such as:
\begin{center}
|This is the part \childdocname{} of \childdocjob{}.|
\end{center}

%%%%%%%%%%%%%%%%%%%%%%%%%%%%%%%%%%%%%%%%%%%%%%%%%%%%%%%%%%%%%%%%%%%%%%%%%%%%%%%%
\subsection{Flags}
\label{sec:flags}

The package makes it easy to generate different versions
of the main or child documents.
To this end compilation flags can be defined
and assigned different default values.
They will be particularly useful in conjunction
with the forwarding mechanism described in \secref{sec:forward}.

For example, it may be useful to have a flag |\version|
which can be set to |draft| or |final|.
The document source will contain some conditional code
depending on the value of |\version|.
Suppose further, the flag should default to |final| for the main file
and to |draft| for child files
which is a natural assignment for editing the document.
This is achieved by placing the following code
in the preamble of the main document
(below the |\childdocmain| directive):
%
\begin{center}
\begin{tabular}{l}
|\ifchilddoc|\\
|\providecommand{\version}{draft}|\\
|\||else|\\
|\providecommand{\version}{final}|\\
|\||fi|
\end{tabular}
\end{center}
%
The definition by |\providecommand| makes sure
that previous definitions are not overwritten.
Further statements |\providecommand{\version}{...}|
can thus be added before the above code to override it.

For the main file, one might add a line
(between |\childdocmain| and the above block)
%
\begin{center}
|%\ifchilddoc\||else\providecommand{\version}{draft}\||fi|
\end{center}
%
which can be uncommented to produce a draft version.
Likewise one can add a line to the very top of a child file
(above the |\childdocof{|\textit{main}|}| directive)
%
\begin{center}
|%\providecommand{\version}{final}|
\end{center}
%
which can be uncommented to produce the final version of this child document.

%%%%%%%%%%%%%%%%%%%%%%%%%%%%%%%%%%%%%%%%%%%%%%%%%%%%%%%%%%%%%%%%%%%%%%%%%%%%%%%%
\subsection{Forwarding}
\label{sec:forward}

Different versions of the main or child documents
using compilation flags as described in \secref{sec:flags}
can be (permanently) stored in different files
for convenient compilation, viewing and distribution.
To this end, the package defines a command
to pass on compilation to a different file:

%%%%%%%%%%%%%%%%%%%%%%%%%%%%%%%%%%%%%%%%
\DescribeMacro{\childdocforward}
The command |\childdocforward| redirects processing to
another source file:
%
\begin{center}
\begin{tabular}{l}
|\input{childdoc.def}|\\
|\childdocforward[|\textit{main}|]{|\textit{dest}|}|\\
\end{tabular}
\end{center}
%
The argument \textit{dest} is the destination file
(without extension).
It should be the main file or one of the child files.
Note that further \textsf{childdoc} directives
such as |\childdocof| and |\childdocforward|
in the indicated file will be processed in this form.
The optional argument \textit{main}
passes on directly to the main file \textit{main}
while pretending to compile the child \textit{dest}.
This form behaves as if \textit{dest}
issues |\childdocof{|\textit{main}|}| right away,
and no further \textsf{childdoc} directives will be processed.

%%%%%%%%%%%%%%%%%%%%%%%%%%%%%%%%%%%%%%%%
\DescribeMacro{\...prefix}
In the alternative form |\childdocforwardprefix|,
%
\begin{center}
\begin{tabular}{l}
|\input{childdoc.def}|\\
|\childdocforwardprefix[|\textit{main}|]{|\textit{prefix}|}{|\textit{dest}|}|
\end{tabular}
\end{center}
%
the destination file is determined by a pattern
depending on the current file:
To make this work, the current file must be called
`{\textit{prefix}\hspace{0.2em}\textit{suffix}}'
with \textit{prefix} matching precisely the argument.
Processing is then passed on to the file
`{\textit{dest}\hspace{0.2em}\textit{suffix}}'.
Surely, the same effect is achieved by
directly specifying the
argument `{\textit{dest}\hspace{0.2em}\textit{suffix}}'
in the first form.
However, that requires to set up a different file
for each child. With the alternative form of the command
all these files can have exactly the same content
which simplifies setting them up and maintaining them.

For example, the following file |draft.tex|
with a compilation flag |\version| as described in \secref{sec:flags}
compiles the main document as a draft:
%
\begin{center}
\begin{tabular}{l}
|\def\version{draft}|\\
|\input{childdoc.def}|\\
|\childdocforward{|\textit{main}|}|
\end{tabular}
\end{center}
%
Likewise, the following files |final|\textit{nn}|.tex|
compile the final version of the child document
|child|\textit{nn}|.tex|:
%
\begin{center}
\begin{tabular}{l}
|\def\version{final}|\\
|\input{childdoc.def}|\\
|\childdocforwardprefix{final}{child}|
\end{tabular}
\end{center}
%

Note that when several versions of a main file and/or of each child file
are to be generated, it may be convenient to set up a |Makefile| or
shell script to automatise the process.

%%%%%%%%%%%%%%%%%%%%%%%%%%%%%%%%%%%%%%%%%%%%%%%%%%%%%%%%%%%%%%%%%%%%%%%%%%%%%%%%
\subsection{Command Line Processing}
\label{sec:commandline}

The effect of redirection files can also be achieved by invoking
the \LaTeX{} compiler with a more elaborate command line.
Most conveniently this should be done as part
of a shell script or a |Makefile|.

When using \textsf{childdoc} in the main file, the following
command lines effectively perform a redirection
(note that depending on the shell being used,
backslashes may have to be doubled: `|\|' $\to$ `|\\|'):
%
\begin{center}
|... -jobname "|\textit{target}|" |\\|"|[\textit{flags}]%
|\input{childdoc.def}\childdocforward[|\textit{main}|]{|\textit{dest}|}"|
\end{center}
%
Here \textit{target} is the name of the output file,
\textit{main} is the name of the main file
and \textit{dest} is the name of the main or child file to be processed
(all filenames without extensions).
The optional argument \textit{main} can be omitted
if \textit{main} matches \textit{dest}.
Optionally, compilation \textit{flags} can be defined via |\def| commands.
This command line makes the \TeX{} engine believe
it is compiling the file \textit{target}
whose content is specified as the latter parameter.
The provided code then forwards the processing to
\textit{main} or \textit{dest} as described in \secref{sec:forward}.

%%%%%%%%%%%%%%%%%%%%%%%%%%%%%%%%%%%%%%%%%%%%%%%%%%%%%%%%%%%%%%%%%%%%%%%%%%%%%%%%
\subsection{Include by Input}
\label{sec:input}

Including child documents by |\include| has some restrictions by design.
Most notably, the content of a child document always occupies
its own set of pages; pages cannot be shared between child documents.
Usually, this behaviour makes perfect sense
because each child document contain an essential part of the document.
However, in some situations it may be desirable to compose
a document from a collection of parts
without having mandatory page breaks between then.
For this case, the package
provides a mechanism to include parts
by |\input| which can also be processed individually.
However, by construction this mechanism
requires manual handling of the content to be output.

%%%%%%%%%%%%%%%%%%%%%%%%%%%%%%%%%%%%%%%%
\DescribeMacro{\ifchilddocmanual}
The main file should be prepared as usual, see \secref{sec:include}.
However, the document body must make a distinction
between processing of an individual part and of the main document, e.g.:
%
\begin{center}
\begin{tabular}{l}
|\ifchilddocmanual|\\
|\input{\childdocname}|\\
|\||else|\\
\textit{document body with }|\input{|\textit{part}|}|\\
|\||fi|
\end{tabular}
\end{center}
%
The conditional |\ifchilddocmanual| is true whenever
a part to be included by |\input| is being compiled,
and the name of the part is stored in |\childdocname|.

%%%%%%%%%%%%%%%%%%%%%%%%%%%%%%%%%%%%%%%%
\DescribeMacro{\childdocby}
Each part to be included by |\input| should start with:
%
\begin{center}
\begin{tabular}{l}
|\input{childdoc.def}|\\
|\childdocby{|\textit{main}|}|\\
\end{tabular}
\end{center}
%
The directive |\childdocby| is similar to |\childdocof|
described in \secref{sec:include},
but the subsequent selection of content must be done manually.
To that end, both |\ifchilddoc| and |\ifchilddocmanual|
will be true upon processing of a part,
and the name of the part is stored in |\childdocname|.
Note that |\jobname| will be set to the filename of the current part
so that each part receives an individual |.aux| file
that does not interfere with the |.aux| file(s) of the main document.
This behaviour can be altered by the alternative form
|\childdocby[*]{|\textit{main}|}| (with a non-empty optional argument)
which uses the |.aux| file of the main document
by setting |\jobname| to \textit{main}.

%%%%%%%%%%%%%%%%%%%%%%%%%%%%%%%%%%%%%%%%%%%%%%%%%%%%%%%%%%%%%%%%%%%%%%%%%%%%%%%%
\subsection{Driver Development}
\label{sec:driver}

The \textsf{childdoc} mechanism can also be use for the development
of definition files such as \LaTeX{} styles or classes.
This case differs from the above setup with multiple parts
included by |\include| in that no |\includeonly| should be invoked.
This can be achieved by starting the include file
(before |\ProvidesPackage|) with:
%
\begin{center}
\begin{tabular}{l}
|\input{childdoc.def}|\\
|\childdocforward{|\textit{main}|}|\\
\end{tabular}
\end{center}
%
or alternatively with:
%
\begin{center}
\begin{tabular}{l}
|\input{childdoc.def}|\\
|\childdocby{|\textit{main}|}|\\
\end{tabular}
\end{center}
%
Both forms have slightly different effects as described above.
The main file is prepared as usual, see \secref{sec:include}.

%%%%%%%%%%%%%%%%%%%%%%%%%%%%%%%%%%%%%%%%%%%%%%%%%%%%%%%%%%%%%%%%%%%%%%%%%%%%%%%%
\subsection{Legacy Detection}
\label{sec:detection}

The directive |\childdocmain| in the main file can detect
whether the complete document or merely a child is to be compiled
even without using the directive |\childdocof|.
This method is deprecated because it is less robust
and there is no compelling reason to use it;
it is merely provided for backward compatibility
and it may be removed in future versions.

If the detection mechanism is to be used,
it is mandatory to correctly specify
the filename of the main file as the argument of |\childdocmain|:
%
\begin{center}
\begin{tabular}{l}
|\input{childdoc.def}|\\
|\childdocmain{|\textit{main}|}|\\
\end{tabular}
\end{center}
%
If |\jobname| does not match the argument \textit{main} of |\childdocmain|,
it is assumed that |\jobname| points to the child file to be compiled.
When using |\childdocmain| with the main file specified as argument,
it suffices to start a child file
with just |\input{|\textit{main}|}|
without loading of the package and using |\childdocof|.
If instead all processing is done
with the appropriate \textsf{childdoc} directives,
the argument of \textit{main} of |\childdocmain| can be empty.

An alternative version of the command line processing described
in \secref{sec:commandline} using the detection mechanism reads:
%
\begin{center}
|... -jobname "|\textit{target}|" "|[\textit{flags}]%
[|\def\jobname{|\textit{dest}|}|]|\input{|\textit{main}|}"|
\end{center}

%%%%%%%%%%%%%%%%%%%%%%%%%%%%%%%%%%%%%%%%%%%%%%%%%%%%%%%%%%%%%%%%%%%%%%%%%%%%%%%%
\subsection{Manual Code}
\label{sec:manual}

In case one cannot be certain whether the definitions file |childdoc.def|
is installed on the target \TeX{} distribution
and one prefers not to ship it,
it is conceivable to paste a few relevant commands into the sources.

To that end, drop all statements |\input{childdoc.def}|
and perform the replacements as outlined below.
Instead of |\childdocmain{|\textit{main}|}| add the following code
to the top of the main file:
%
\begin{center}
\begin{tabular}{l}
|\||ifdefined\childdocname\endinput\||fi\newif\ifchilddoc|\\
|\edef\childdocname{\scantokens\expandafter{\jobname\noexpand}}|\\
|\def\childdocmain{|\textit{main}|}\||ifx\childdocmain\childdocname\||else|\\
|\childdoctrue\includeonly{\childdocname}\let\jobname\childdocmain\||fi|\\
\end{tabular}
\end{center}
%
Instead of |\childdocof{|\textit{main}|}| just include the main file
at the top of each child file:
%
\begin{center}
|\input{|\textit{main}|}|
\end{center}
%
A simple redirection |\childdocforward{|\textit{dest}|}| is achieved by:
%
\begin{center}
|\def\jobname{|\textit{dest}|}\input{\jobname}|
\end{center}
%
The redirection with prefix
|\childdocforwardprefix[|\textit{prefix}|]{|\textit{dest}|}|
is accomplished by:
%
\begin{center}
\begin{tabular}{l}
|{\edef\jobname{\scantokens\expandafter{\jobname\noexpand}}|\\
|\def\redirectjob |\textit{prefix}|#1~~~{\gdef\jobname{|\textit{dest}|#1}}|\\
|\expandafter\redirectjob\jobname~~~}\input{\jobname}|
\end{tabular}
\end{center}

In an alternative approach,
child documents can be compiled by a specific command line
without additional code or specific definitions:
%
\begin{center}
|... -jobname "|\textit{target}|" "|[\textit{flags}]%
|\includeonly{|\textit{dest}|}\input{|\textit{main}|}"|
\end{center}
%

%%%%%%%%%%%%%%%%%%%%%%%%%%%%%%%%%%%%%%%%%%%%%%%%%%%%%%%%%%%%%%%%%%%%%%%%%%%%%%%%
%%%%%%%%%%%%%%%%%%%%%%%%%%%%%%%%%%%%%%%%%%%%%%%%%%%%%%%%%%%%%%%%%%%%%%%%%%%%%%%%
\section{Information}

%%%%%%%%%%%%%%%%%%%%%%%%%%%%%%%%%%%%%%%%%%%%%%%%%%%%%%%%%%%%%%%%%%%%%%%%%%%%%%%%
\subsection{Copyright}

Copyright \copyright{} 2017--2018 Niklas Beisert

This work may be distributed and/or modified under the
conditions of the \LaTeX{} Project Public License, either version 1.3
of this license or (at your option) any later version.
The latest version of this license is in
  \url{http://www.latex-project.org/lppl.txt}
and version 1.3 or later is part of all distributions of \LaTeX{}
version 2005/12/01 or later.

This work has the LPPL maintenance status `maintained'.

The Current Maintainer of this work is Niklas Beisert.

This work consists of the files |README.txt|, |childdoc.ins| and |childdoc.dtx|
as well as the derived files |childdoc.def|, |cdocsamp.tex|
with |cdocsch1.tex|, |cdocsch2.tex|, |cdocspt3.tex|, |cdocspt4.tex|,
|cdocsdrf.tex|, |cdocsfn1.tex|, |cdocsfn2.tex|
as well as |childdoc.pdf|.

%%%%%%%%%%%%%%%%%%%%%%%%%%%%%%%%%%%%%%%%%%%%%%%%%%%%%%%%%%%%%%%%%%%%%%%%%%%%%%%%
\subsection{Files and Installation}

The package consists of the files:
%
\begin{center}
\begin{tabular}{ll}
    |README.txt|   & readme file \\
    |childdoc.ins| & installation file \\
    |childdoc.dtx| & source file \\
    |childdoc.def| & definition file \\
    |cdocsamp.tex| & sample main file \\
    |cdocsch1.tex| & sample include file \\
    |cdocsch2.tex| & sample include file \\
    |cdocspt3.tex| & sample part file \\
    |cdocspt4.tex| & sample part file \\
    |cdocsdrf.tex| & sample redirection file \\
    |cdocsfn1.tex| & sample redirection file \\
    |cdocsfn2.tex| & sample redirection file \\
    |childdoc.pdf| & manual
\end{tabular}
\end{center}
%
The distribution consists of the files
|README.txt|, |childdoc.ins| and |childdoc.dtx|.
%
\begin{itemize}
\item
Run (pdf)\LaTeX{} on |childdoc.dtx|
to compile the manual |childdoc.pdf| (this file).
\item
Run \LaTeX{} on |childdoc.ins| to create the definitions file |childdoc.def|
and the sample |cdocsamp.tex| with include files
|cdocsch1.tex|, |cdocsch2.tex|, |cdocspt3.tex|, |cdocspt4.tex|,
|cdocsdrf.tex|, |cdocsfn1.tex|, |cdocsfn2.tex|.
Then copy the file |childdoc.def| to an appropriate directory of your \LaTeX{}
distribution, e.g.\ \textit{texmf-root}|/tex/latex/childdoc|.
\end{itemize}

%%%%%%%%%%%%%%%%%%%%%%%%%%%%%%%%%%%%%%%%%%%%%%%%%%%%%%%%%%%%%%%%%%%%%%%%%%%%%%%%
\subsection{Related CTAN Packages}

There are several other packages which offer a similar functionality:
%
\begin{itemize}
\item
The packages
\href{http://ctan.org/pkg/docmute}{\textsf{docmute}},
\href{http://ctan.org/pkg/includex}{\textsf{includex}} and
\href{http://ctan.org/pkg/standalone}{\textsf{standalone}}
provide commands to include only the document body of
a child file thus allowing both files to be compiled individually.
\item
The packages \href{http://ctan.org/pkg/subdocs}{\textsf{subdocs}}
and \href{http://ctan.org/pkg/subfiles}{\textsf{subfiles}}
provide structures in which the main and child documents can be
encapsulated and allowing them to be compiled individually.
The inclusion mechanism is different from the conventional |\include|.
\item
The package \href{http://ctan.org/pkg/combine}{\textsf{combine}}
is an elaborate solution to combine several documents into one.
\end{itemize}
%
See also the CTAN topic \href{http://ctan.org/topic/subdocs}{\textsf{subdocs}}
for further related packages.
The present package differs from the above solutions in that
a document structure constructed with the conventional |\include| mechanism
just needs two extra commands at the top of every file
such that all constituent files can be compiled individually.

%%%%%%%%%%%%%%%%%%%%%%%%%%%%%%%%%%%%%%%%%%%%%%%%%%%%%%%%%%%%%%%%%%%%%%%%%%%%%%%%
%\subsection{Feature Suggestions}
%
%The following is a list of features which may be useful for future
%versions of this package:
%%
%\begin{itemize}
%\item
%\ldots
%\end{itemize}

%%%%%%%%%%%%%%%%%%%%%%%%%%%%%%%%%%%%%%%%%%%%%%%%%%%%%%%%%%%%%%%%%%%%%%%%%%%%%%%%
\subsection{Revision History}

%%%%%%%%%%%%%%%%%%%%%%%%%%%%%%%%%%%%%%%%
\paragraph{v2.0:} 2018/12/30

\begin{itemize}
\item
immediate forward processing
\item
added |\childdocby| mechanism
\item
manual restructured
\end{itemize}

%%%%%%%%%%%%%%%%%%%%%%%%%%%%%%%%%%%%%%%%
\paragraph{v1.6:} 2018/01/17

\begin{itemize}
\item
application for development of include files
\item
corrections to manual
\end{itemize}

%%%%%%%%%%%%%%%%%%%%%%%%%%%%%%%%%%%%%%%%
\paragraph{v1.5:} 2017/05/21

\begin{itemize}
\item
more complete structuring introduced
\item
|\childdocof| introduced
\item
|\childdoc| renamed to |\childdocmain|
\item
|\childredirect| renamed to |\childdocforward| and |\childdocforwardprefix|
and functionality expanded
\end{itemize}

%%%%%%%%%%%%%%%%%%%%%%%%%%%%%%%%%%%%%%%%
\paragraph{v1.0:} 2017/04/27

\begin{itemize}
\item
manual and install package
\item
first version published on CTAN
\end{itemize}

%%%%%%%%%%%%%%%%%%%%%%%%%%%%%%%%%%%%%%%%
\paragraph{v0.6:} 2017/04/26

\begin{itemize}
\item
redirection mechanism added
\end{itemize}

%%%%%%%%%%%%%%%%%%%%%%%%%%%%%%%%%%%%%%%%
\paragraph{v0.5:} 2017/04/26

\begin{itemize}
\item
functionality in definition file
\end{itemize}


%%%%%%%%%%%%%%%%%%%%%%%%%%%%%%%%%%%%%%%%%%%%%%%%%%%%%%%%%%%%%%%%%%%%%%%%%%%%%%%%
%%%%%%%%%%%%%%%%%%%%%%%%%%%%%%%%%%%%%%%%%%%%%%%%%%%%%%%%%%%%%%%%%%%%%%%%%%%%%%%%
%%%%%%%%%%%%%%%%%%%%%%%%%%%%%%%%%%%%%%%%%%%%%%%%%%%%%%%%%%%%%%%%%%%%%%%%%%%%%%%%
\appendix

\settowidth\MacroIndent{\rmfamily\scriptsize 000\ }

 \DocInput{childdoc.dtx}

\end{document}
%</driver>
% \fi
%
% %%%%%%%%%%%%%%%%%%%%%%%%%%%%%%%%%%%%%%%%%%%%%%%%%%%%%%%%%%%%%%%%%%%%%%%%%%%%%%
% %%%%%%%%%%%%%%%%%%%%%%%%%%%%%%%%%%%%%%%%%%%%%%%%%%%%%%%%%%%%%%%%%%%%%%%%%%%%%%
% \section{Sample}
%\iffalse
%<*samplemain>
%\fi
%
% The following presents a sample document
% with two chapters, two parts, a title page,
% a compile flag as well as three forwarding files to set the flag.
% It consists of eight |.tex| files:
% \begin{center}
% \begin{tabular}{ll}
% |cdocsamp.tex|&main file\\
% |cdocsch1.tex|&include file for chapter 1\\
% |cdocsch2.tex|&include file for chapter 2\\
% |cdocspt3.tex|&include file for part 3\\
% |cdocspt4.tex|&include file for part 4\\
% |cdocsdrf.tex|&forwarding file for main file in draft mode\\
% |cdocsfi1.tex|&forwarding file for final version of chapter 1\\
% |cdocsfi2.tex|&forwarding file for final version of chapter 2\\
% \end{tabular}
% \end{center}
% Each of the eight files can be compiled directly by the \LaTeX{} compiler.
%
% %%%%%%%%%%%%%%%%%%%%%%%%%%%%%%%%%%%%%%
% \paragraph{Main File.}
%
% The main file is called |cdocsamp.tex|.
%
% Load the \textsf{childdoc} definitions and
% declare the filename for the main document:
%    \begin{macrocode}
\input{childdoc.def}
\childdocmain{}
%    \end{macrocode}

% Optional override for |\version| flag:
%    \begin{macrocode}
%%\ifchilddoc\else\providecommand{\version}{draft}\fi
%    \end{macrocode}

% Define the default values for the |\version| flag
% (|final| for the main file and |draft| for childs):
%    \begin{macrocode}
\ifchilddoc
\providecommand{\version}{draft}
\else
\providecommand{\version}{final}
\fi
%    \end{macrocode}

% Load the standard document class:
%    \begin{macrocode}
\documentclass[12pt]{article}
%    \end{macrocode}

% Start the document body:
%    \begin{macrocode}
\begin{document}
%    \end{macrocode}

% Declare a title page.
% Print title, part of document being processed and version flag:
%    \begin{macrocode}
\addtocounter{page}{-1}
\begin{center}
{\LARGE\bfseries{}childdoc example\par}
\vspace{1cm}
\ifchilddoc
\ifchilddocmanual part\else chapter\fi:
`\childdocname' of `\childdocjob'\par
\else
main document: `\childdocjob'\par
\fi
version: \version\par
\end{center}
\newpage
%    \end{macrocode}

% Manually include selected file,
% otherwise process as usual:
%    \begin{macrocode}
\ifchilddocmanual
\section*{part `\childdocname'}
\input{\childdocname}
\else
%    \end{macrocode}

% Include the two chapters:
%    \begin{macrocode}
\include{cdocsch1}
\include{cdocsch2}
%    \end{macrocode}

% Include the two parts unless only chapters should be displayed:
%    \begin{macrocode}
\ifchilddoc\else
\section{part three}
\input{cdocspt3}
\section{part four}
\input{cdocspt4}
\fi
%    \end{macrocode}

% Process as usual until here:
%    \begin{macrocode}
\fi
%    \end{macrocode}

% End of document body:
%    \begin{macrocode}
\end{document}
%    \end{macrocode}
%\iffalse
%</samplemain>
%\fi
%
% %%%%%%%%%%%%%%%%%%%%%%%%%%%%%%%%%%%%%%
% \paragraph{Chapter Include Files.}
%
% The include files are called |cdocsch1.tex| and |cdocsch2.tex|.
%
%\iffalse
%<*samplechap1|samplechap2>
%\fi

% Optional override for |\version| flag:
%    \begin{macrocode}
%%\providecommand{\version}{final}
%    \end{macrocode}

% Include the main document:
%    \begin{macrocode}
\input{childdoc.def}
\childdocof{cdocsamp}
%    \end{macrocode}

%\iffalse
%</samplechap1|samplechap2>
%\fi
%
%\iffalse
%<*samplechap1>
%\fi
% Some text for chapter 1:
%    \begin{macrocode}
\section{one}
some text in chapter one
%    \end{macrocode}

%\iffalse
%</samplechap1>
%\fi
% Some text for chapter 2:
%\iffalse
%<*samplechap2>
%\fi
%    \begin{macrocode}
\section{two}
more text in chapter two
%    \end{macrocode}

%\iffalse
%</samplechap2>
%\fi
%
% %%%%%%%%%%%%%%%%%%%%%%%%%%%%%%%%%%%%%%
% \paragraph{Part Include Files.}
%
% The include files are called |cdocspt3.tex| and |cdocspt4.tex|.
%
%\iffalse
%<*samplepart3|samplepart4>
%\fi

% Optional override for |\version| flag:
%    \begin{macrocode}
%%\providecommand{\version}{final}
%    \end{macrocode}

% Include the main document:
%    \begin{macrocode}
\input{childdoc.def}
\childdocby{cdocsamp}
%    \end{macrocode}

%\iffalse
%</samplepart3|samplepart4>
%\fi
%
%\iffalse
%<*samplepart3>
%\fi
% Some text for part 3:
%    \begin{macrocode}
some text in part three
%    \end{macrocode}

%\iffalse
%</samplepart3>
%\fi
% Some text for part 4:
%\iffalse
%<*samplepart4>
%\fi
%    \begin{macrocode}
more text in part four
%    \end{macrocode}

%\iffalse
%</samplepart4>
%\fi
%
% %%%%%%%%%%%%%%%%%%%%%%%%%%%%%%%%%%%%%%
% \paragraph{Forwarding for a Complete Draft.}
%
% The following forwarding file |cdocsdrf.tex|
% compiles the main document in draft mode:
%\iffalse
%<*sampledraft>
%\fi
%    \begin{macrocode}
\def\version{draft}
\input{childdoc.def}
\childdocforward{cdocsamp}
%    \end{macrocode}

%\iffalse
%</sampledraft>
%\fi
%
% %%%%%%%%%%%%%%%%%%%%%%%%%%%%%%%%%%%%%%
% \paragraph{Forwarding for Final Version of the Chapters.}
%
% The following forwarding files |cdocsfn1.tex| and |cdocsfn2.tex|
% (with identical content)
% compile the final versions of the child documents
% |cdocsch1.tex| and |cdocsch2.tex|, respectively:
%\iffalse
%<*samplefinal>
%\fi
%    \begin{macrocode}
\def\version{final}
\input{childdoc.def}
\childdocforwardprefix[cdocsamp]{cdocsfn}{cdocsch}
%    \end{macrocode}

%\iffalse
%</samplefinal>
%\fi
%
% %%%%%%%%%%%%%%%%%%%%%%%%%%%%%%%%%%%%%%
% \paragraph{Command Line Processing.}
%
% The following three command lines generate the output files
% |cdocscld|, |cdocscl1| and |cdocscl2|
% which should be identical to
% |cdocsdrf|, |cdocsch1| and |cdocsfn2|, respectively:
% \begin{center}
% \begin{tabular}{l}
% |latex -jobname cdocscld \|\\
% |  "\def\version{draft}\input{childdoc.def}\childdocforward{cdocsamp}"|\\
% |latex -jobname cdocscl1 \|\\
% |  "\input{childdoc.def}\childdocforward[cdocsamp]{cdocsch1}"|\\
% |latex -jobname cdocscl2 \|\\
% |  "\def\version{final}\input{childdoc.def}\childdocforward{cdocsch2}"|
% \end{tabular}
% \end{center}
% Note that the trailing backslash on each first line
% merely continues the input to the second line
% (for convenient cut ant paste).
% Furthermore, the command |latex| can be replaced by any
% of its alternative versions such as |pdflatex|.
%
% %%%%%%%%%%%%%%%%%%%%%%%%%%%%%%%%%%%%%%%%%%%%%%%%%%%%%%%%%%%%%%%%%%%%%%%%%%%%%%
% %%%%%%%%%%%%%%%%%%%%%%%%%%%%%%%%%%%%%%%%%%%%%%%%%%%%%%%%%%%%%%%%%%%%%%%%%%%%%%
% \section{Implementation}
%\iffalse
%<*package>
%\fi
%
% This section describes the definitions file |childdoc.def|.

% The definitions cannot be loaded using |\usepackage| or |\RequirePackage|
% which has a mechanism to prevent loading a style file more than once.
% When loading the definitions by means of |\input|
% multiple instances have to be prevented manually:
%\iffalse
%This code needs to be before the `\ProvidesFile' directive
%which is defined at the beginning of this file.
%Therefore it is also placed there and commented out here.
%</package>
%<*discard>
%\fi
%    \begin{macrocode}
\ifdefined\childdocmain\endinput\fi
%    \end{macrocode}
%\iffalse
%</discard>
%<*package>
%\fi
%
% \macro{\ifchilddoc}
% \macro{\ifchilddocmanual}
% The conditional |\ifchilddoc| tells whether a
% child (true) or main (false) document is being compiled.
% The conditional |\ifchilddocmanual| tells whether
% the |\includeonly| mechanism is used (false) or
% the selection of child files must be performed manually (true).
% The definitions initialise to false:
%    \begin{macrocode}
\newif\ifchilddoc
\newif\ifchilddocmanual
%    \end{macrocode}

% \macro{\childdocname}
% \macro{\childdocjob}
% The macro |\childdocname| stores the name of the main document
% to be compiled. The macro |\childdocjob| stores the name of
% the document on which the \LaTeX{} compiler was originally invoked.
% The content of |\jobname| cannot be compared
% to filenames specified in the source due to different catcodes.
% The following code rescans |\jobname|, stores the result
% in |\childdocname| and saves a copy in |\childdocjob|:
%    \begin{macrocode}
\edef\childdocname{\scantokens\expandafter{\jobname\noexpand}}
\let\childdocjob\childdocname
%    \end{macrocode}

% \macro{\childdocdisable}
% The macro |\childdocdisable| prevents the main file
% from being processed more than once.
% At this stage, the main document command |\childdocmain|
% is assumed to be called once again where it should do nothing.
% Any subsequent call to it should prevent
% a secondary processing of the main document
% It overwrites the forwarding commands
% |\childdocof| and |\childdocforward|
% with empty macros to prevent further inclusions of the main document:
%    \begin{macrocode}
\newcommand{\childdocdisable}
{
  \renewcommand{\childdocmain}[1]{\renewcommand{\childdocmain}[1]{\endinput}}
  \renewcommand{\childdocof}[1]{}
  \renewcommand{\childdocby}[2][]{}
  \renewcommand{\childdocforward}[2][]{}
  \renewcommand{\childdocdisable}{}
}
%    \end{macrocode}

% \macro{\childdocmain}
% The macro |\childdocmain| is to be called at the top of the main file
% with nothing or the main filename (without extension) as argument.
% First, it breaks loops.
% If the argument is not empty and does not match |\childdocname|
% (which is set by the first inclusion of |childdoc.def|),
% |\ifchilddoc| is set to true, |\includeonly| is applied to the child file
% and |\jobname| is set to the main file
% (for proper handling of |.aux| files):
%    \begin{macrocode}
\newcommand{\childdocmain}[1]
{
  \childdocdisable\childdocmain{}
  \if?#1?\else
    \begingroup
      \def\childdoctmp{#1}
      \ifx\childdoctmp\childdocname
        \def\childdoctmp{}
      \else
        \def\childdoctmp
        {
          \childdoctrue
          \includeonly{\childdocname}
          \def\childdocjob{#1}
          \def\jobname{#1}
        }
      \fi
      \expandafter
    \endgroup
    \childdoctmp
  \fi
}
%    \end{macrocode}

% \macro{\childdocof}
% The command |\childdocof| redirects
% compilation to the main file |#1|.
%    \begin{macrocode}
\newcommand{\childdocof}[1]
{
  \childdocdisable
  \childdoctrue
  \includeonly{\childdocname}
  \def\jobname{#1}
  \def\childdocjob{#1}
  \input{#1}
}
%    \end{macrocode}

% \macro{\childdocby}
% The command |\childdocby| ....
%    \begin{macrocode}
\newcommand{\childdocby}[2][]
{
  \childdocdisable
  \childdoctrue
  \childdocmanualtrue
  \if?#1?\else
    \def\jobname{#2}
  \fi
  \def\childdocjob{#2}
  \input{#2}
  \endinput
}
%    \end{macrocode}

% \macro{\childdocforward}
% The command |\childdocforward| redirects
% compilation to the main file or
% (if the optional argument is given) a child file.
% Parameters are set as if the main file
% or a child file starting with |\childdocof| was compiled.
% Then compilation is handed over to the main file:
%    \begin{macrocode}
\newcommand{\childdocforward}[2][]
{
  \begingroup
    \if?#1?
      \def\childdoctmp
      {
        \def\childdocname{#2}
        \def\childdocjob{#2}
        \def\jobname{#2}
        \input{#2}
        \endinput
      }
    \else
      \def\childdoctmp
      {
        \childdocdisable
        \def\childdocname{#2}
        \childdoctrue
        \includeonly{#2}
        \def\childdocjob{#1}
        \def\jobname{#1}
        \input{#1}
        \endinput
      }
    \fi
    \expandafter
  \endgroup
  \childdoctmp
}
%    \end{macrocode}

% \macro{\childdocforwardprefix}
% The command |\childdocforwardprefix| redirects
% compilation to the main or a child file by means of a pattern.
% The prefix |#1| in the current filename is replaced by |#2|
% and the suffix of the current filename is kept
% (it is assumed that the filename does not contain the substring `|~~~|'
% which is used as a delimiter).
% Compilation is handed over to the new file by |\childdocforward|:
%    \begin{macrocode}
\newcommand{\childdocforwardprefix}[3][]
{
  \begingroup
    \def\childdocextract #2##1~~~{\def\childdoctmp{\childdocforward[#1]{#3##1}}}
    \expandafter\childdocextract\childdocname~~~
    \expandafter
  \endgroup
  \childdoctmp
}
%    \end{macrocode}

% \macro{\childdoc}
% The deprecated macro |\childdoc| is a legacy version of |\childdocmain|:
%    \begin{macrocode}
\newcommand{\childdoc}{\childdocmain}
%    \end{macrocode}

% \macro{\childdocredirect}
% The deprecated macro |\childdocredirect| is a legacy version
% of |\childdocforward| and |\childdocforwardprefix|:
%    \begin{macrocode}
\newcommand{\childdocredirect}[2][]
{
  \begingroup
    \if?#1?
      \def\childdoctmp{\childdocforward{#2}}
    \else
      \def\childdoctmp{\childdocforwardprefix{#1}{#2}}
    \fi
    \expandafter
  \endgroup
  \childdoctmp
}
%    \end{macrocode}

%\iffalse
%</package>
%\fi
%
\endinput

\childdocforward{cdocsamp}
%    \end{macrocode}

%\iffalse
%</sampledraft>
%\fi
%
% %%%%%%%%%%%%%%%%%%%%%%%%%%%%%%%%%%%%%%
% \paragraph{Forwarding for Final Version of the Chapters.}
%
% The following forwarding files |cdocsfn1.tex| and |cdocsfn2.tex|
% (with identical content)
% compile the final versions of the child documents
% |cdocsch1.tex| and |cdocsch2.tex|, respectively:
%\iffalse
%<*samplefinal>
%\fi
%    \begin{macrocode}
\def\version{final}
% \iffalse
%
% childdoc.dtx Copyright (C) 2017-2018 Niklas Beisert
%
% This work may be distributed and/or modified under the
% conditions of the LaTeX Project Public License, either version 1.3
% of this license or (at your option) any later version.
% The latest version of this license is in
%   http://www.latex-project.org/lppl.txt
% and version 1.3 or later is part of all distributions of LaTeX
% version 2005/12/01 or later.
%
% This work has the LPPL maintenance status `maintained'.
%
% The Current Maintainer of this work is Niklas Beisert.
%
% This work consists of the files childdoc.dtx and childdoc.ins
% and the derived files childdoc.def and cdocsamp.tex with
% cdocsch1.tex, cdocsch2.tex, cdocsdrf.tex, cdocsfn1.tex, cdocsfn2.tex.
%
%<package>\ifdefined\childdocmain\endinput\fi
%<package>\ProvidesFile{childdoc.def}[2018/12/30 v2.0 child document driver]
%<samplemain>\ProvidesFile{cdocsamp.tex}[2018/12/30 v2.0 sample for childdoc]
%<*driver>
%\ProvidesFile{childdoc.drv}[2018/12/30 v2.0 childdoc reference manual file]
\PassOptionsToClass{10pt,a4paper}{article}
\documentclass{ltxdoc}

\usepackage[margin=35mm]{geometry}
\usepackage{hyperref}
\usepackage{hyperxmp}
\usepackage[usenames]{color}

\hypersetup{colorlinks=true}
\hypersetup{pdfstartview=FitH}
\hypersetup{pdfpagemode=UseNone}
\hypersetup{pdfsource={}}
\hypersetup{pdflang={en-UK}}
\hypersetup{pdfcopyright={Copyright 2017-2018 Niklas Beisert.
  This work may be distributed and/or modified under the
  conditions of the LaTeX Project Public License, either version 1.3
  of this license or (at your option) any later version.}}
\hypersetup{pdflicenseurl={http://www.latex-project.org/lppl.txt}}
\hypersetup{pdfcontactaddress={ETH Zurich, ITP, HIT K,
  Wolfgang-Pauli-Strasse 27}}
\hypersetup{pdfcontactpostcode={8093}}
\hypersetup{pdfcontactcity={Zurich}}
\hypersetup{pdfcontactcountry={Switzerland}}
\hypersetup{pdfcontactemail={nbeisert@itp.phys.ethz.ch}}
\hypersetup{pdfcontacturl={http://people.phys.ethz.ch/\xmptilde nbeisert/}}

\newcommand{\secref}[1]{\hyperref[#1]{section \ref*{#1}}}

\parskip1ex
\parindent0pt
\let\olditemize\itemize
\def\itemize{\olditemize\parskip0pt}

\begin{document}

\title{The \textsf{childdoc} Package}
\hypersetup{pdftitle={The childdoc Package}}
\author{Niklas Beisert\\[2ex]
  Institut f\"ur Theoretische Physik\\
  Eidgen\"ossische Technische Hochschule Z\"urich\\
  Wolfgang-Pauli-Strasse 27, 8093 Z\"urich, Switzerland\\[1ex]
  \href{mailto:nbeisert@itp.phys.ethz.ch}
  {\texttt{nbeisert@itp.phys.ethz.ch}}}
\hypersetup{pdfauthor={Niklas Beisert}}
\hypersetup{pdfsubject={Manual for the LaTeX2e Package childdoc}}
\date{30 December 2018, \textsf{v2.0}}
\maketitle

\begin{abstract}\noindent
\textsf{childdoc} is a \LaTeXe{} package
that enables the direct compilation
of document sections included by |\include|
to individual files.
\end{abstract}

\begingroup
\parskip0ex
\tableofcontents
\endgroup

%%%%%%%%%%%%%%%%%%%%%%%%%%%%%%%%%%%%%%%%%%%%%%%%%%%%%%%%%%%%%%%%%%%%%%%%%%%%%%%%
%%%%%%%%%%%%%%%%%%%%%%%%%%%%%%%%%%%%%%%%%%%%%%%%%%%%%%%%%%%%%%%%%%%%%%%%%%%%%%%%
\section{Introduction}

\LaTeX{} provides a mechanism to structure a large document (such as a book)
into a main file and several child files (containing the chapters)
using the |\include| command.
This mechanism is beneficial for documents
which span hundreds of pages in order to
make the source file(s) more manageable.
Moreover, compilation can be restricted to
selected child files by means of the |\includeonly| command.
The latter feature can be used to reduce the compilation time while editing
(this was significantly more useful in the earlier days of \LaTeX{})
or to generate a smaller document which is easier to navigate.
Another application of |\includeonly| is to generate
documents consisting of selected parts of the complete document.

However, there are a few drawbacks of the plain |\include| mechanism:
\begin{itemize}
\item
The child files cannot be compiled on their own,
they can only be compiled via the main file.
A naive editing environment
(such as a text editor with an option
to have the current file processed by \LaTeX)
may require one to switch to the main file before compiling;
attempting to compile the child file produces errors.
\item
The main file must be modified (each time)
to adjust the |\includeonly| command
to the present needs. This easily leaves the main file in a messy state.
\item
The generated document will always carry the filename
of the main document. This is inconvenient if
several child files are to be compiled and
to be kept for distribution.
\end{itemize}

The present package provides a simple interface
to make child files individually compilable by \LaTeX{}.
Compiling a child file then has the same effect as compiling
the main file with an |\includeonly| command
to select the appropriate child.
Moreover the generated document will carry the name of the child
rather than the main file.
This resolves all three above issues.

This feature is meant to make the editing of books,
thesis documents and lecture notes somewhat more convenient.
However, the package can also be used efficiently for
composing a series of documents (such as exercise sheets)
which are typically distributed individually.
It then assists the author in generating the individual documents
(potentially in different versions)
as well as a document containing the collected series.
Another application is in developing style files
or other kinds of included material
where compilation of the style file could redirect
to a sample or test file.

%%%%%%%%%%%%%%%%%%%%%%%%%%%%%%%%%%%%%%%%%%%%%%%%%%%%%%%%%%%%%%%%%%%%%%%%%%%%%%%%
%%%%%%%%%%%%%%%%%%%%%%%%%%%%%%%%%%%%%%%%%%%%%%%%%%%%%%%%%%%%%%%%%%%%%%%%%%%%%%%%
\section{Usage}

First of all, the package \textsf{childdoc} is \emph{not} a standard
\LaTeXe{} |.sty| style file! Therefore it needs to be invoked in
a non-standard way.

%%%%%%%%%%%%%%%%%%%%%%%%%%%%%%%%%%%%%%%%%%%%%%%%%%%%%%%%%%%%%%%%%%%%%%%%%%%%%%%%
\subsection{Included Files}
\label{sec:include}

%%%%%%%%%%%%%%%%%%%%%%%%%%%%%%%%%%%%%%%%
\DescribeMacro{\childdocmain}
To use the package, add the commands
\begin{center}
\begin{tabular}{l}
|\input{childdoc.def}|\\
|\childdocmain{}|\\
\end{tabular}
\end{center}
at the very top of the main \LaTeX{} file,
in particular \emph{before} the |\documentclass| statement!
The argument of |\childdocmain| should be left empty
(but it must be present).

%%%%%%%%%%%%%%%%%%%%%%%%%%%%%%%%%%%%%%%%
\DescribeMacro{\childdocof}
Furthermore, add the commands
\begin{center}
\begin{tabular}{l}
|\input{childdoc.def}|\\
|\childdocof{|\textit{main}|}|\\
\end{tabular}
\end{center}
at the top of every child file \textit{child}
which is included by |\include{|\textit{child}|}|
from within the main file
(or at least for those files to be compiled individually).
The argument \textit{main} must be the filename of the main file.

There are a couple of
considerations in setting up the main and child documents:

%%%%%%%%%%%%%%%%%%%%%%%%%%%%%%%%%%%%%%%%
\paragraph{Restrictions.}

Please note the following restrictions:
\begin{itemize}
\item
|\childdocmain| must be called with one argument \textit{main}
to ensure compatibility with earlier version of the package.
It must either be empty (|\childdocmain{}|)
or precisely match the filename of the main file in which it is specified.
See \secref{sec:detection} for further information.
\item
The filename \textit{main} must be specified without the |.tex| extension.
\item
The filename \textit{main} is case sensitive
(even in case-insensitive file systems)
due to internal string comparison.
\item
The argument \textit{main} should be fully expanded, it cannot be a macro.
\item
Subdirectories and special characters should be avoided in filenames.
\item
The command |\childdocmain{|\textit{main}|}| must be followed by a whitespace.
It should not be followed immediately by another command
or by a comment mark `|%|'.
This is because the \TeX{} parser reads the token immediately following
the argument of |\childdocmain| and puts it
at the beginning of every child section;
however, a white\-space is ignored.
\end{itemize}

%%%%%%%%%%%%%%%%%%%%%%%%%%%%%%%%%%%%%%%%
\paragraph{Content of Main File.}

It is advisable to place all content in the child files included by |\include|.
Any output contained in the main file will appear in all child documents
unless suppressed manually;
it cannot be suppressed automatically by the |\includeonly| directive
and thus should normally be avoided.
A method to include some content in the main file
by means of conditional processing is described in \secref{sec:conditional}.

%%%%%%%%%%%%%%%%%%%%%%%%%%%%%%%%%%%%%%%%
\paragraph{Page Numbering.}

When only a part of the document is compiled,
the appropriate numbering of pages
(as well as other status parameters)
is determined from the |.aux| files.
The latter contain information from previous passes.
However this information needs to propagate through
all intermediate child documents.
Therefore the page numbering in child documents may well
be inconsistent until the complete document is compiled at least once.

A useful (if unconventional) way to always ensure a consistent
page numbering is to restart the numbering in each child document
and denote the pages by `\textit{child}|.|\textit{page}'
where \textit{child} represents the chapter/section number of the child file.
This can be achieved by the command
|\numberwithin{page}{|\textit{child}|}|
of the \textsf{amsmath} package
where \textit{child} can be |chapter| or |section|
depending on the chosen structuring.
Alternatively, one can modify the macro |\thepage| appropriately
and reset the counter |page| at the start of each child file.

%%%%%%%%%%%%%%%%%%%%%%%%%%%%%%%%%%%%%%%%%%%%%%%%%%%%%%%%%%%%%%%%%%%%%%%%%%%%%%%%
\subsection{Conditional Processing}
\label{sec:conditional}

The package provides a mechanism to compile different versions
of a document. To customise the versions further some conditional processing
can come in handy to distinguish which version is being compiled.
The package provides two macros to describe the compilation context:

%%%%%%%%%%%%%%%%%%%%%%%%%%%%%%%%%%%%%%%%
\DescribeMacro{\ifchilddoc}
The conditional |\ifchilddoc| distinguishes between the compilation of
child documents and the main document:
%
\begin{center}
|\ifchilddoc |\textit{child-code}| |[|\||else |\textit{main-code}]| \||fi|
\end{center}

%%%%%%%%%%%%%%%%%%%%%%%%%%%%%%%%%%%%%%%%
\DescribeMacro{\childdocname}
\DescribeMacro{\childdocjob}
The macro |\childdocname| contains the filename (without extension)
of the main or child file being processed.
Note that |\childdocjob| will always contain the name of the main file.

%%%%%%%%%%%%%%%%%%%%%%%%%%%%%%%%%%%%%%%%
\paragraph{Title Page.}

Conditional processing can be used to include a title or banner page
in the main document when proper precautions are taken.
Importantly, the code in the main file should ensure that the page counter
(as well as other status parameters which are stored in the |.aux| files)
takes the same value after the conditional processing.
Otherwise the page numbers may take divergent values
depending on which part is compiled.

For example, a title page could be declared by:
%
\begin{center}
\begin{tabular}{l}
|\ifchilddoc\||else|\\
|\addtocounter{page}{-1}|\\
\textit{code for title page}\\
|\newpage|\\
|\||fi|
\end{tabular}
\end{center}
%
A banner page for the child documents can be generated by:
%
\begin{center}
\begin{tabular}{l}
|\ifchilddoc|\\
|\addtocounter{page}{-1}|\\
\textit{code for banner page}\\
|\newpage|\\
|\||fi|
\end{tabular}
\end{center}
%
Here one could write a message such as:
\begin{center}
|This is the part \childdocname{} of \childdocjob{}.|
\end{center}

%%%%%%%%%%%%%%%%%%%%%%%%%%%%%%%%%%%%%%%%%%%%%%%%%%%%%%%%%%%%%%%%%%%%%%%%%%%%%%%%
\subsection{Flags}
\label{sec:flags}

The package makes it easy to generate different versions
of the main or child documents.
To this end compilation flags can be defined
and assigned different default values.
They will be particularly useful in conjunction
with the forwarding mechanism described in \secref{sec:forward}.

For example, it may be useful to have a flag |\version|
which can be set to |draft| or |final|.
The document source will contain some conditional code
depending on the value of |\version|.
Suppose further, the flag should default to |final| for the main file
and to |draft| for child files
which is a natural assignment for editing the document.
This is achieved by placing the following code
in the preamble of the main document
(below the |\childdocmain| directive):
%
\begin{center}
\begin{tabular}{l}
|\ifchilddoc|\\
|\providecommand{\version}{draft}|\\
|\||else|\\
|\providecommand{\version}{final}|\\
|\||fi|
\end{tabular}
\end{center}
%
The definition by |\providecommand| makes sure
that previous definitions are not overwritten.
Further statements |\providecommand{\version}{...}|
can thus be added before the above code to override it.

For the main file, one might add a line
(between |\childdocmain| and the above block)
%
\begin{center}
|%\ifchilddoc\||else\providecommand{\version}{draft}\||fi|
\end{center}
%
which can be uncommented to produce a draft version.
Likewise one can add a line to the very top of a child file
(above the |\childdocof{|\textit{main}|}| directive)
%
\begin{center}
|%\providecommand{\version}{final}|
\end{center}
%
which can be uncommented to produce the final version of this child document.

%%%%%%%%%%%%%%%%%%%%%%%%%%%%%%%%%%%%%%%%%%%%%%%%%%%%%%%%%%%%%%%%%%%%%%%%%%%%%%%%
\subsection{Forwarding}
\label{sec:forward}

Different versions of the main or child documents
using compilation flags as described in \secref{sec:flags}
can be (permanently) stored in different files
for convenient compilation, viewing and distribution.
To this end, the package defines a command
to pass on compilation to a different file:

%%%%%%%%%%%%%%%%%%%%%%%%%%%%%%%%%%%%%%%%
\DescribeMacro{\childdocforward}
The command |\childdocforward| redirects processing to
another source file:
%
\begin{center}
\begin{tabular}{l}
|\input{childdoc.def}|\\
|\childdocforward[|\textit{main}|]{|\textit{dest}|}|\\
\end{tabular}
\end{center}
%
The argument \textit{dest} is the destination file
(without extension).
It should be the main file or one of the child files.
Note that further \textsf{childdoc} directives
such as |\childdocof| and |\childdocforward|
in the indicated file will be processed in this form.
The optional argument \textit{main}
passes on directly to the main file \textit{main}
while pretending to compile the child \textit{dest}.
This form behaves as if \textit{dest}
issues |\childdocof{|\textit{main}|}| right away,
and no further \textsf{childdoc} directives will be processed.

%%%%%%%%%%%%%%%%%%%%%%%%%%%%%%%%%%%%%%%%
\DescribeMacro{\...prefix}
In the alternative form |\childdocforwardprefix|,
%
\begin{center}
\begin{tabular}{l}
|\input{childdoc.def}|\\
|\childdocforwardprefix[|\textit{main}|]{|\textit{prefix}|}{|\textit{dest}|}|
\end{tabular}
\end{center}
%
the destination file is determined by a pattern
depending on the current file:
To make this work, the current file must be called
`{\textit{prefix}\hspace{0.2em}\textit{suffix}}'
with \textit{prefix} matching precisely the argument.
Processing is then passed on to the file
`{\textit{dest}\hspace{0.2em}\textit{suffix}}'.
Surely, the same effect is achieved by
directly specifying the
argument `{\textit{dest}\hspace{0.2em}\textit{suffix}}'
in the first form.
However, that requires to set up a different file
for each child. With the alternative form of the command
all these files can have exactly the same content
which simplifies setting them up and maintaining them.

For example, the following file |draft.tex|
with a compilation flag |\version| as described in \secref{sec:flags}
compiles the main document as a draft:
%
\begin{center}
\begin{tabular}{l}
|\def\version{draft}|\\
|\input{childdoc.def}|\\
|\childdocforward{|\textit{main}|}|
\end{tabular}
\end{center}
%
Likewise, the following files |final|\textit{nn}|.tex|
compile the final version of the child document
|child|\textit{nn}|.tex|:
%
\begin{center}
\begin{tabular}{l}
|\def\version{final}|\\
|\input{childdoc.def}|\\
|\childdocforwardprefix{final}{child}|
\end{tabular}
\end{center}
%

Note that when several versions of a main file and/or of each child file
are to be generated, it may be convenient to set up a |Makefile| or
shell script to automatise the process.

%%%%%%%%%%%%%%%%%%%%%%%%%%%%%%%%%%%%%%%%%%%%%%%%%%%%%%%%%%%%%%%%%%%%%%%%%%%%%%%%
\subsection{Command Line Processing}
\label{sec:commandline}

The effect of redirection files can also be achieved by invoking
the \LaTeX{} compiler with a more elaborate command line.
Most conveniently this should be done as part
of a shell script or a |Makefile|.

When using \textsf{childdoc} in the main file, the following
command lines effectively perform a redirection
(note that depending on the shell being used,
backslashes may have to be doubled: `|\|' $\to$ `|\\|'):
%
\begin{center}
|... -jobname "|\textit{target}|" |\\|"|[\textit{flags}]%
|\input{childdoc.def}\childdocforward[|\textit{main}|]{|\textit{dest}|}"|
\end{center}
%
Here \textit{target} is the name of the output file,
\textit{main} is the name of the main file
and \textit{dest} is the name of the main or child file to be processed
(all filenames without extensions).
The optional argument \textit{main} can be omitted
if \textit{main} matches \textit{dest}.
Optionally, compilation \textit{flags} can be defined via |\def| commands.
This command line makes the \TeX{} engine believe
it is compiling the file \textit{target}
whose content is specified as the latter parameter.
The provided code then forwards the processing to
\textit{main} or \textit{dest} as described in \secref{sec:forward}.

%%%%%%%%%%%%%%%%%%%%%%%%%%%%%%%%%%%%%%%%%%%%%%%%%%%%%%%%%%%%%%%%%%%%%%%%%%%%%%%%
\subsection{Include by Input}
\label{sec:input}

Including child documents by |\include| has some restrictions by design.
Most notably, the content of a child document always occupies
its own set of pages; pages cannot be shared between child documents.
Usually, this behaviour makes perfect sense
because each child document contain an essential part of the document.
However, in some situations it may be desirable to compose
a document from a collection of parts
without having mandatory page breaks between then.
For this case, the package
provides a mechanism to include parts
by |\input| which can also be processed individually.
However, by construction this mechanism
requires manual handling of the content to be output.

%%%%%%%%%%%%%%%%%%%%%%%%%%%%%%%%%%%%%%%%
\DescribeMacro{\ifchilddocmanual}
The main file should be prepared as usual, see \secref{sec:include}.
However, the document body must make a distinction
between processing of an individual part and of the main document, e.g.:
%
\begin{center}
\begin{tabular}{l}
|\ifchilddocmanual|\\
|\input{\childdocname}|\\
|\||else|\\
\textit{document body with }|\input{|\textit{part}|}|\\
|\||fi|
\end{tabular}
\end{center}
%
The conditional |\ifchilddocmanual| is true whenever
a part to be included by |\input| is being compiled,
and the name of the part is stored in |\childdocname|.

%%%%%%%%%%%%%%%%%%%%%%%%%%%%%%%%%%%%%%%%
\DescribeMacro{\childdocby}
Each part to be included by |\input| should start with:
%
\begin{center}
\begin{tabular}{l}
|\input{childdoc.def}|\\
|\childdocby{|\textit{main}|}|\\
\end{tabular}
\end{center}
%
The directive |\childdocby| is similar to |\childdocof|
described in \secref{sec:include},
but the subsequent selection of content must be done manually.
To that end, both |\ifchilddoc| and |\ifchilddocmanual|
will be true upon processing of a part,
and the name of the part is stored in |\childdocname|.
Note that |\jobname| will be set to the filename of the current part
so that each part receives an individual |.aux| file
that does not interfere with the |.aux| file(s) of the main document.
This behaviour can be altered by the alternative form
|\childdocby[*]{|\textit{main}|}| (with a non-empty optional argument)
which uses the |.aux| file of the main document
by setting |\jobname| to \textit{main}.

%%%%%%%%%%%%%%%%%%%%%%%%%%%%%%%%%%%%%%%%%%%%%%%%%%%%%%%%%%%%%%%%%%%%%%%%%%%%%%%%
\subsection{Driver Development}
\label{sec:driver}

The \textsf{childdoc} mechanism can also be use for the development
of definition files such as \LaTeX{} styles or classes.
This case differs from the above setup with multiple parts
included by |\include| in that no |\includeonly| should be invoked.
This can be achieved by starting the include file
(before |\ProvidesPackage|) with:
%
\begin{center}
\begin{tabular}{l}
|\input{childdoc.def}|\\
|\childdocforward{|\textit{main}|}|\\
\end{tabular}
\end{center}
%
or alternatively with:
%
\begin{center}
\begin{tabular}{l}
|\input{childdoc.def}|\\
|\childdocby{|\textit{main}|}|\\
\end{tabular}
\end{center}
%
Both forms have slightly different effects as described above.
The main file is prepared as usual, see \secref{sec:include}.

%%%%%%%%%%%%%%%%%%%%%%%%%%%%%%%%%%%%%%%%%%%%%%%%%%%%%%%%%%%%%%%%%%%%%%%%%%%%%%%%
\subsection{Legacy Detection}
\label{sec:detection}

The directive |\childdocmain| in the main file can detect
whether the complete document or merely a child is to be compiled
even without using the directive |\childdocof|.
This method is deprecated because it is less robust
and there is no compelling reason to use it;
it is merely provided for backward compatibility
and it may be removed in future versions.

If the detection mechanism is to be used,
it is mandatory to correctly specify
the filename of the main file as the argument of |\childdocmain|:
%
\begin{center}
\begin{tabular}{l}
|\input{childdoc.def}|\\
|\childdocmain{|\textit{main}|}|\\
\end{tabular}
\end{center}
%
If |\jobname| does not match the argument \textit{main} of |\childdocmain|,
it is assumed that |\jobname| points to the child file to be compiled.
When using |\childdocmain| with the main file specified as argument,
it suffices to start a child file
with just |\input{|\textit{main}|}|
without loading of the package and using |\childdocof|.
If instead all processing is done
with the appropriate \textsf{childdoc} directives,
the argument of \textit{main} of |\childdocmain| can be empty.

An alternative version of the command line processing described
in \secref{sec:commandline} using the detection mechanism reads:
%
\begin{center}
|... -jobname "|\textit{target}|" "|[\textit{flags}]%
[|\def\jobname{|\textit{dest}|}|]|\input{|\textit{main}|}"|
\end{center}

%%%%%%%%%%%%%%%%%%%%%%%%%%%%%%%%%%%%%%%%%%%%%%%%%%%%%%%%%%%%%%%%%%%%%%%%%%%%%%%%
\subsection{Manual Code}
\label{sec:manual}

In case one cannot be certain whether the definitions file |childdoc.def|
is installed on the target \TeX{} distribution
and one prefers not to ship it,
it is conceivable to paste a few relevant commands into the sources.

To that end, drop all statements |\input{childdoc.def}|
and perform the replacements as outlined below.
Instead of |\childdocmain{|\textit{main}|}| add the following code
to the top of the main file:
%
\begin{center}
\begin{tabular}{l}
|\||ifdefined\childdocname\endinput\||fi\newif\ifchilddoc|\\
|\edef\childdocname{\scantokens\expandafter{\jobname\noexpand}}|\\
|\def\childdocmain{|\textit{main}|}\||ifx\childdocmain\childdocname\||else|\\
|\childdoctrue\includeonly{\childdocname}\let\jobname\childdocmain\||fi|\\
\end{tabular}
\end{center}
%
Instead of |\childdocof{|\textit{main}|}| just include the main file
at the top of each child file:
%
\begin{center}
|\input{|\textit{main}|}|
\end{center}
%
A simple redirection |\childdocforward{|\textit{dest}|}| is achieved by:
%
\begin{center}
|\def\jobname{|\textit{dest}|}\input{\jobname}|
\end{center}
%
The redirection with prefix
|\childdocforwardprefix[|\textit{prefix}|]{|\textit{dest}|}|
is accomplished by:
%
\begin{center}
\begin{tabular}{l}
|{\edef\jobname{\scantokens\expandafter{\jobname\noexpand}}|\\
|\def\redirectjob |\textit{prefix}|#1~~~{\gdef\jobname{|\textit{dest}|#1}}|\\
|\expandafter\redirectjob\jobname~~~}\input{\jobname}|
\end{tabular}
\end{center}

In an alternative approach,
child documents can be compiled by a specific command line
without additional code or specific definitions:
%
\begin{center}
|... -jobname "|\textit{target}|" "|[\textit{flags}]%
|\includeonly{|\textit{dest}|}\input{|\textit{main}|}"|
\end{center}
%

%%%%%%%%%%%%%%%%%%%%%%%%%%%%%%%%%%%%%%%%%%%%%%%%%%%%%%%%%%%%%%%%%%%%%%%%%%%%%%%%
%%%%%%%%%%%%%%%%%%%%%%%%%%%%%%%%%%%%%%%%%%%%%%%%%%%%%%%%%%%%%%%%%%%%%%%%%%%%%%%%
\section{Information}

%%%%%%%%%%%%%%%%%%%%%%%%%%%%%%%%%%%%%%%%%%%%%%%%%%%%%%%%%%%%%%%%%%%%%%%%%%%%%%%%
\subsection{Copyright}

Copyright \copyright{} 2017--2018 Niklas Beisert

This work may be distributed and/or modified under the
conditions of the \LaTeX{} Project Public License, either version 1.3
of this license or (at your option) any later version.
The latest version of this license is in
  \url{http://www.latex-project.org/lppl.txt}
and version 1.3 or later is part of all distributions of \LaTeX{}
version 2005/12/01 or later.

This work has the LPPL maintenance status `maintained'.

The Current Maintainer of this work is Niklas Beisert.

This work consists of the files |README.txt|, |childdoc.ins| and |childdoc.dtx|
as well as the derived files |childdoc.def|, |cdocsamp.tex|
with |cdocsch1.tex|, |cdocsch2.tex|, |cdocspt3.tex|, |cdocspt4.tex|,
|cdocsdrf.tex|, |cdocsfn1.tex|, |cdocsfn2.tex|
as well as |childdoc.pdf|.

%%%%%%%%%%%%%%%%%%%%%%%%%%%%%%%%%%%%%%%%%%%%%%%%%%%%%%%%%%%%%%%%%%%%%%%%%%%%%%%%
\subsection{Files and Installation}

The package consists of the files:
%
\begin{center}
\begin{tabular}{ll}
    |README.txt|   & readme file \\
    |childdoc.ins| & installation file \\
    |childdoc.dtx| & source file \\
    |childdoc.def| & definition file \\
    |cdocsamp.tex| & sample main file \\
    |cdocsch1.tex| & sample include file \\
    |cdocsch2.tex| & sample include file \\
    |cdocspt3.tex| & sample part file \\
    |cdocspt4.tex| & sample part file \\
    |cdocsdrf.tex| & sample redirection file \\
    |cdocsfn1.tex| & sample redirection file \\
    |cdocsfn2.tex| & sample redirection file \\
    |childdoc.pdf| & manual
\end{tabular}
\end{center}
%
The distribution consists of the files
|README.txt|, |childdoc.ins| and |childdoc.dtx|.
%
\begin{itemize}
\item
Run (pdf)\LaTeX{} on |childdoc.dtx|
to compile the manual |childdoc.pdf| (this file).
\item
Run \LaTeX{} on |childdoc.ins| to create the definitions file |childdoc.def|
and the sample |cdocsamp.tex| with include files
|cdocsch1.tex|, |cdocsch2.tex|, |cdocspt3.tex|, |cdocspt4.tex|,
|cdocsdrf.tex|, |cdocsfn1.tex|, |cdocsfn2.tex|.
Then copy the file |childdoc.def| to an appropriate directory of your \LaTeX{}
distribution, e.g.\ \textit{texmf-root}|/tex/latex/childdoc|.
\end{itemize}

%%%%%%%%%%%%%%%%%%%%%%%%%%%%%%%%%%%%%%%%%%%%%%%%%%%%%%%%%%%%%%%%%%%%%%%%%%%%%%%%
\subsection{Related CTAN Packages}

There are several other packages which offer a similar functionality:
%
\begin{itemize}
\item
The packages
\href{http://ctan.org/pkg/docmute}{\textsf{docmute}},
\href{http://ctan.org/pkg/includex}{\textsf{includex}} and
\href{http://ctan.org/pkg/standalone}{\textsf{standalone}}
provide commands to include only the document body of
a child file thus allowing both files to be compiled individually.
\item
The packages \href{http://ctan.org/pkg/subdocs}{\textsf{subdocs}}
and \href{http://ctan.org/pkg/subfiles}{\textsf{subfiles}}
provide structures in which the main and child documents can be
encapsulated and allowing them to be compiled individually.
The inclusion mechanism is different from the conventional |\include|.
\item
The package \href{http://ctan.org/pkg/combine}{\textsf{combine}}
is an elaborate solution to combine several documents into one.
\end{itemize}
%
See also the CTAN topic \href{http://ctan.org/topic/subdocs}{\textsf{subdocs}}
for further related packages.
The present package differs from the above solutions in that
a document structure constructed with the conventional |\include| mechanism
just needs two extra commands at the top of every file
such that all constituent files can be compiled individually.

%%%%%%%%%%%%%%%%%%%%%%%%%%%%%%%%%%%%%%%%%%%%%%%%%%%%%%%%%%%%%%%%%%%%%%%%%%%%%%%%
%\subsection{Feature Suggestions}
%
%The following is a list of features which may be useful for future
%versions of this package:
%%
%\begin{itemize}
%\item
%\ldots
%\end{itemize}

%%%%%%%%%%%%%%%%%%%%%%%%%%%%%%%%%%%%%%%%%%%%%%%%%%%%%%%%%%%%%%%%%%%%%%%%%%%%%%%%
\subsection{Revision History}

%%%%%%%%%%%%%%%%%%%%%%%%%%%%%%%%%%%%%%%%
\paragraph{v2.0:} 2018/12/30

\begin{itemize}
\item
immediate forward processing
\item
added |\childdocby| mechanism
\item
manual restructured
\end{itemize}

%%%%%%%%%%%%%%%%%%%%%%%%%%%%%%%%%%%%%%%%
\paragraph{v1.6:} 2018/01/17

\begin{itemize}
\item
application for development of include files
\item
corrections to manual
\end{itemize}

%%%%%%%%%%%%%%%%%%%%%%%%%%%%%%%%%%%%%%%%
\paragraph{v1.5:} 2017/05/21

\begin{itemize}
\item
more complete structuring introduced
\item
|\childdocof| introduced
\item
|\childdoc| renamed to |\childdocmain|
\item
|\childredirect| renamed to |\childdocforward| and |\childdocforwardprefix|
and functionality expanded
\end{itemize}

%%%%%%%%%%%%%%%%%%%%%%%%%%%%%%%%%%%%%%%%
\paragraph{v1.0:} 2017/04/27

\begin{itemize}
\item
manual and install package
\item
first version published on CTAN
\end{itemize}

%%%%%%%%%%%%%%%%%%%%%%%%%%%%%%%%%%%%%%%%
\paragraph{v0.6:} 2017/04/26

\begin{itemize}
\item
redirection mechanism added
\end{itemize}

%%%%%%%%%%%%%%%%%%%%%%%%%%%%%%%%%%%%%%%%
\paragraph{v0.5:} 2017/04/26

\begin{itemize}
\item
functionality in definition file
\end{itemize}


%%%%%%%%%%%%%%%%%%%%%%%%%%%%%%%%%%%%%%%%%%%%%%%%%%%%%%%%%%%%%%%%%%%%%%%%%%%%%%%%
%%%%%%%%%%%%%%%%%%%%%%%%%%%%%%%%%%%%%%%%%%%%%%%%%%%%%%%%%%%%%%%%%%%%%%%%%%%%%%%%
%%%%%%%%%%%%%%%%%%%%%%%%%%%%%%%%%%%%%%%%%%%%%%%%%%%%%%%%%%%%%%%%%%%%%%%%%%%%%%%%
\appendix

\settowidth\MacroIndent{\rmfamily\scriptsize 000\ }

 \DocInput{childdoc.dtx}

\end{document}
%</driver>
% \fi
%
% %%%%%%%%%%%%%%%%%%%%%%%%%%%%%%%%%%%%%%%%%%%%%%%%%%%%%%%%%%%%%%%%%%%%%%%%%%%%%%
% %%%%%%%%%%%%%%%%%%%%%%%%%%%%%%%%%%%%%%%%%%%%%%%%%%%%%%%%%%%%%%%%%%%%%%%%%%%%%%
% \section{Sample}
%\iffalse
%<*samplemain>
%\fi
%
% The following presents a sample document
% with two chapters, two parts, a title page,
% a compile flag as well as three forwarding files to set the flag.
% It consists of eight |.tex| files:
% \begin{center}
% \begin{tabular}{ll}
% |cdocsamp.tex|&main file\\
% |cdocsch1.tex|&include file for chapter 1\\
% |cdocsch2.tex|&include file for chapter 2\\
% |cdocspt3.tex|&include file for part 3\\
% |cdocspt4.tex|&include file for part 4\\
% |cdocsdrf.tex|&forwarding file for main file in draft mode\\
% |cdocsfi1.tex|&forwarding file for final version of chapter 1\\
% |cdocsfi2.tex|&forwarding file for final version of chapter 2\\
% \end{tabular}
% \end{center}
% Each of the eight files can be compiled directly by the \LaTeX{} compiler.
%
% %%%%%%%%%%%%%%%%%%%%%%%%%%%%%%%%%%%%%%
% \paragraph{Main File.}
%
% The main file is called |cdocsamp.tex|.
%
% Load the \textsf{childdoc} definitions and
% declare the filename for the main document:
%    \begin{macrocode}
\input{childdoc.def}
\childdocmain{}
%    \end{macrocode}

% Optional override for |\version| flag:
%    \begin{macrocode}
%%\ifchilddoc\else\providecommand{\version}{draft}\fi
%    \end{macrocode}

% Define the default values for the |\version| flag
% (|final| for the main file and |draft| for childs):
%    \begin{macrocode}
\ifchilddoc
\providecommand{\version}{draft}
\else
\providecommand{\version}{final}
\fi
%    \end{macrocode}

% Load the standard document class:
%    \begin{macrocode}
\documentclass[12pt]{article}
%    \end{macrocode}

% Start the document body:
%    \begin{macrocode}
\begin{document}
%    \end{macrocode}

% Declare a title page.
% Print title, part of document being processed and version flag:
%    \begin{macrocode}
\addtocounter{page}{-1}
\begin{center}
{\LARGE\bfseries{}childdoc example\par}
\vspace{1cm}
\ifchilddoc
\ifchilddocmanual part\else chapter\fi:
`\childdocname' of `\childdocjob'\par
\else
main document: `\childdocjob'\par
\fi
version: \version\par
\end{center}
\newpage
%    \end{macrocode}

% Manually include selected file,
% otherwise process as usual:
%    \begin{macrocode}
\ifchilddocmanual
\section*{part `\childdocname'}
\input{\childdocname}
\else
%    \end{macrocode}

% Include the two chapters:
%    \begin{macrocode}
\include{cdocsch1}
\include{cdocsch2}
%    \end{macrocode}

% Include the two parts unless only chapters should be displayed:
%    \begin{macrocode}
\ifchilddoc\else
\section{part three}
\input{cdocspt3}
\section{part four}
\input{cdocspt4}
\fi
%    \end{macrocode}

% Process as usual until here:
%    \begin{macrocode}
\fi
%    \end{macrocode}

% End of document body:
%    \begin{macrocode}
\end{document}
%    \end{macrocode}
%\iffalse
%</samplemain>
%\fi
%
% %%%%%%%%%%%%%%%%%%%%%%%%%%%%%%%%%%%%%%
% \paragraph{Chapter Include Files.}
%
% The include files are called |cdocsch1.tex| and |cdocsch2.tex|.
%
%\iffalse
%<*samplechap1|samplechap2>
%\fi

% Optional override for |\version| flag:
%    \begin{macrocode}
%%\providecommand{\version}{final}
%    \end{macrocode}

% Include the main document:
%    \begin{macrocode}
\input{childdoc.def}
\childdocof{cdocsamp}
%    \end{macrocode}

%\iffalse
%</samplechap1|samplechap2>
%\fi
%
%\iffalse
%<*samplechap1>
%\fi
% Some text for chapter 1:
%    \begin{macrocode}
\section{one}
some text in chapter one
%    \end{macrocode}

%\iffalse
%</samplechap1>
%\fi
% Some text for chapter 2:
%\iffalse
%<*samplechap2>
%\fi
%    \begin{macrocode}
\section{two}
more text in chapter two
%    \end{macrocode}

%\iffalse
%</samplechap2>
%\fi
%
% %%%%%%%%%%%%%%%%%%%%%%%%%%%%%%%%%%%%%%
% \paragraph{Part Include Files.}
%
% The include files are called |cdocspt3.tex| and |cdocspt4.tex|.
%
%\iffalse
%<*samplepart3|samplepart4>
%\fi

% Optional override for |\version| flag:
%    \begin{macrocode}
%%\providecommand{\version}{final}
%    \end{macrocode}

% Include the main document:
%    \begin{macrocode}
\input{childdoc.def}
\childdocby{cdocsamp}
%    \end{macrocode}

%\iffalse
%</samplepart3|samplepart4>
%\fi
%
%\iffalse
%<*samplepart3>
%\fi
% Some text for part 3:
%    \begin{macrocode}
some text in part three
%    \end{macrocode}

%\iffalse
%</samplepart3>
%\fi
% Some text for part 4:
%\iffalse
%<*samplepart4>
%\fi
%    \begin{macrocode}
more text in part four
%    \end{macrocode}

%\iffalse
%</samplepart4>
%\fi
%
% %%%%%%%%%%%%%%%%%%%%%%%%%%%%%%%%%%%%%%
% \paragraph{Forwarding for a Complete Draft.}
%
% The following forwarding file |cdocsdrf.tex|
% compiles the main document in draft mode:
%\iffalse
%<*sampledraft>
%\fi
%    \begin{macrocode}
\def\version{draft}
\input{childdoc.def}
\childdocforward{cdocsamp}
%    \end{macrocode}

%\iffalse
%</sampledraft>
%\fi
%
% %%%%%%%%%%%%%%%%%%%%%%%%%%%%%%%%%%%%%%
% \paragraph{Forwarding for Final Version of the Chapters.}
%
% The following forwarding files |cdocsfn1.tex| and |cdocsfn2.tex|
% (with identical content)
% compile the final versions of the child documents
% |cdocsch1.tex| and |cdocsch2.tex|, respectively:
%\iffalse
%<*samplefinal>
%\fi
%    \begin{macrocode}
\def\version{final}
\input{childdoc.def}
\childdocforwardprefix[cdocsamp]{cdocsfn}{cdocsch}
%    \end{macrocode}

%\iffalse
%</samplefinal>
%\fi
%
% %%%%%%%%%%%%%%%%%%%%%%%%%%%%%%%%%%%%%%
% \paragraph{Command Line Processing.}
%
% The following three command lines generate the output files
% |cdocscld|, |cdocscl1| and |cdocscl2|
% which should be identical to
% |cdocsdrf|, |cdocsch1| and |cdocsfn2|, respectively:
% \begin{center}
% \begin{tabular}{l}
% |latex -jobname cdocscld \|\\
% |  "\def\version{draft}\input{childdoc.def}\childdocforward{cdocsamp}"|\\
% |latex -jobname cdocscl1 \|\\
% |  "\input{childdoc.def}\childdocforward[cdocsamp]{cdocsch1}"|\\
% |latex -jobname cdocscl2 \|\\
% |  "\def\version{final}\input{childdoc.def}\childdocforward{cdocsch2}"|
% \end{tabular}
% \end{center}
% Note that the trailing backslash on each first line
% merely continues the input to the second line
% (for convenient cut ant paste).
% Furthermore, the command |latex| can be replaced by any
% of its alternative versions such as |pdflatex|.
%
% %%%%%%%%%%%%%%%%%%%%%%%%%%%%%%%%%%%%%%%%%%%%%%%%%%%%%%%%%%%%%%%%%%%%%%%%%%%%%%
% %%%%%%%%%%%%%%%%%%%%%%%%%%%%%%%%%%%%%%%%%%%%%%%%%%%%%%%%%%%%%%%%%%%%%%%%%%%%%%
% \section{Implementation}
%\iffalse
%<*package>
%\fi
%
% This section describes the definitions file |childdoc.def|.

% The definitions cannot be loaded using |\usepackage| or |\RequirePackage|
% which has a mechanism to prevent loading a style file more than once.
% When loading the definitions by means of |\input|
% multiple instances have to be prevented manually:
%\iffalse
%This code needs to be before the `\ProvidesFile' directive
%which is defined at the beginning of this file.
%Therefore it is also placed there and commented out here.
%</package>
%<*discard>
%\fi
%    \begin{macrocode}
\ifdefined\childdocmain\endinput\fi
%    \end{macrocode}
%\iffalse
%</discard>
%<*package>
%\fi
%
% \macro{\ifchilddoc}
% \macro{\ifchilddocmanual}
% The conditional |\ifchilddoc| tells whether a
% child (true) or main (false) document is being compiled.
% The conditional |\ifchilddocmanual| tells whether
% the |\includeonly| mechanism is used (false) or
% the selection of child files must be performed manually (true).
% The definitions initialise to false:
%    \begin{macrocode}
\newif\ifchilddoc
\newif\ifchilddocmanual
%    \end{macrocode}

% \macro{\childdocname}
% \macro{\childdocjob}
% The macro |\childdocname| stores the name of the main document
% to be compiled. The macro |\childdocjob| stores the name of
% the document on which the \LaTeX{} compiler was originally invoked.
% The content of |\jobname| cannot be compared
% to filenames specified in the source due to different catcodes.
% The following code rescans |\jobname|, stores the result
% in |\childdocname| and saves a copy in |\childdocjob|:
%    \begin{macrocode}
\edef\childdocname{\scantokens\expandafter{\jobname\noexpand}}
\let\childdocjob\childdocname
%    \end{macrocode}

% \macro{\childdocdisable}
% The macro |\childdocdisable| prevents the main file
% from being processed more than once.
% At this stage, the main document command |\childdocmain|
% is assumed to be called once again where it should do nothing.
% Any subsequent call to it should prevent
% a secondary processing of the main document
% It overwrites the forwarding commands
% |\childdocof| and |\childdocforward|
% with empty macros to prevent further inclusions of the main document:
%    \begin{macrocode}
\newcommand{\childdocdisable}
{
  \renewcommand{\childdocmain}[1]{\renewcommand{\childdocmain}[1]{\endinput}}
  \renewcommand{\childdocof}[1]{}
  \renewcommand{\childdocby}[2][]{}
  \renewcommand{\childdocforward}[2][]{}
  \renewcommand{\childdocdisable}{}
}
%    \end{macrocode}

% \macro{\childdocmain}
% The macro |\childdocmain| is to be called at the top of the main file
% with nothing or the main filename (without extension) as argument.
% First, it breaks loops.
% If the argument is not empty and does not match |\childdocname|
% (which is set by the first inclusion of |childdoc.def|),
% |\ifchilddoc| is set to true, |\includeonly| is applied to the child file
% and |\jobname| is set to the main file
% (for proper handling of |.aux| files):
%    \begin{macrocode}
\newcommand{\childdocmain}[1]
{
  \childdocdisable\childdocmain{}
  \if?#1?\else
    \begingroup
      \def\childdoctmp{#1}
      \ifx\childdoctmp\childdocname
        \def\childdoctmp{}
      \else
        \def\childdoctmp
        {
          \childdoctrue
          \includeonly{\childdocname}
          \def\childdocjob{#1}
          \def\jobname{#1}
        }
      \fi
      \expandafter
    \endgroup
    \childdoctmp
  \fi
}
%    \end{macrocode}

% \macro{\childdocof}
% The command |\childdocof| redirects
% compilation to the main file |#1|.
%    \begin{macrocode}
\newcommand{\childdocof}[1]
{
  \childdocdisable
  \childdoctrue
  \includeonly{\childdocname}
  \def\jobname{#1}
  \def\childdocjob{#1}
  \input{#1}
}
%    \end{macrocode}

% \macro{\childdocby}
% The command |\childdocby| ....
%    \begin{macrocode}
\newcommand{\childdocby}[2][]
{
  \childdocdisable
  \childdoctrue
  \childdocmanualtrue
  \if?#1?\else
    \def\jobname{#2}
  \fi
  \def\childdocjob{#2}
  \input{#2}
  \endinput
}
%    \end{macrocode}

% \macro{\childdocforward}
% The command |\childdocforward| redirects
% compilation to the main file or
% (if the optional argument is given) a child file.
% Parameters are set as if the main file
% or a child file starting with |\childdocof| was compiled.
% Then compilation is handed over to the main file:
%    \begin{macrocode}
\newcommand{\childdocforward}[2][]
{
  \begingroup
    \if?#1?
      \def\childdoctmp
      {
        \def\childdocname{#2}
        \def\childdocjob{#2}
        \def\jobname{#2}
        \input{#2}
        \endinput
      }
    \else
      \def\childdoctmp
      {
        \childdocdisable
        \def\childdocname{#2}
        \childdoctrue
        \includeonly{#2}
        \def\childdocjob{#1}
        \def\jobname{#1}
        \input{#1}
        \endinput
      }
    \fi
    \expandafter
  \endgroup
  \childdoctmp
}
%    \end{macrocode}

% \macro{\childdocforwardprefix}
% The command |\childdocforwardprefix| redirects
% compilation to the main or a child file by means of a pattern.
% The prefix |#1| in the current filename is replaced by |#2|
% and the suffix of the current filename is kept
% (it is assumed that the filename does not contain the substring `|~~~|'
% which is used as a delimiter).
% Compilation is handed over to the new file by |\childdocforward|:
%    \begin{macrocode}
\newcommand{\childdocforwardprefix}[3][]
{
  \begingroup
    \def\childdocextract #2##1~~~{\def\childdoctmp{\childdocforward[#1]{#3##1}}}
    \expandafter\childdocextract\childdocname~~~
    \expandafter
  \endgroup
  \childdoctmp
}
%    \end{macrocode}

% \macro{\childdoc}
% The deprecated macro |\childdoc| is a legacy version of |\childdocmain|:
%    \begin{macrocode}
\newcommand{\childdoc}{\childdocmain}
%    \end{macrocode}

% \macro{\childdocredirect}
% The deprecated macro |\childdocredirect| is a legacy version
% of |\childdocforward| and |\childdocforwardprefix|:
%    \begin{macrocode}
\newcommand{\childdocredirect}[2][]
{
  \begingroup
    \if?#1?
      \def\childdoctmp{\childdocforward{#2}}
    \else
      \def\childdoctmp{\childdocforwardprefix{#1}{#2}}
    \fi
    \expandafter
  \endgroup
  \childdoctmp
}
%    \end{macrocode}

%\iffalse
%</package>
%\fi
%
\endinput

\childdocforwardprefix[cdocsamp]{cdocsfn}{cdocsch}
%    \end{macrocode}

%\iffalse
%</samplefinal>
%\fi
%
% %%%%%%%%%%%%%%%%%%%%%%%%%%%%%%%%%%%%%%
% \paragraph{Command Line Processing.}
%
% The following three command lines generate the output files
% |cdocscld|, |cdocscl1| and |cdocscl2|
% which should be identical to
% |cdocsdrf|, |cdocsch1| and |cdocsfn2|, respectively:
% \begin{center}
% \begin{tabular}{l}
% |latex -jobname cdocscld \|\\
% |  "\def\version{draft}% \iffalse
%
% childdoc.dtx Copyright (C) 2017-2018 Niklas Beisert
%
% This work may be distributed and/or modified under the
% conditions of the LaTeX Project Public License, either version 1.3
% of this license or (at your option) any later version.
% The latest version of this license is in
%   http://www.latex-project.org/lppl.txt
% and version 1.3 or later is part of all distributions of LaTeX
% version 2005/12/01 or later.
%
% This work has the LPPL maintenance status `maintained'.
%
% The Current Maintainer of this work is Niklas Beisert.
%
% This work consists of the files childdoc.dtx and childdoc.ins
% and the derived files childdoc.def and cdocsamp.tex with
% cdocsch1.tex, cdocsch2.tex, cdocsdrf.tex, cdocsfn1.tex, cdocsfn2.tex.
%
%<package>\ifdefined\childdocmain\endinput\fi
%<package>\ProvidesFile{childdoc.def}[2018/12/30 v2.0 child document driver]
%<samplemain>\ProvidesFile{cdocsamp.tex}[2018/12/30 v2.0 sample for childdoc]
%<*driver>
%\ProvidesFile{childdoc.drv}[2018/12/30 v2.0 childdoc reference manual file]
\PassOptionsToClass{10pt,a4paper}{article}
\documentclass{ltxdoc}

\usepackage[margin=35mm]{geometry}
\usepackage{hyperref}
\usepackage{hyperxmp}
\usepackage[usenames]{color}

\hypersetup{colorlinks=true}
\hypersetup{pdfstartview=FitH}
\hypersetup{pdfpagemode=UseNone}
\hypersetup{pdfsource={}}
\hypersetup{pdflang={en-UK}}
\hypersetup{pdfcopyright={Copyright 2017-2018 Niklas Beisert.
  This work may be distributed and/or modified under the
  conditions of the LaTeX Project Public License, either version 1.3
  of this license or (at your option) any later version.}}
\hypersetup{pdflicenseurl={http://www.latex-project.org/lppl.txt}}
\hypersetup{pdfcontactaddress={ETH Zurich, ITP, HIT K,
  Wolfgang-Pauli-Strasse 27}}
\hypersetup{pdfcontactpostcode={8093}}
\hypersetup{pdfcontactcity={Zurich}}
\hypersetup{pdfcontactcountry={Switzerland}}
\hypersetup{pdfcontactemail={nbeisert@itp.phys.ethz.ch}}
\hypersetup{pdfcontacturl={http://people.phys.ethz.ch/\xmptilde nbeisert/}}

\newcommand{\secref}[1]{\hyperref[#1]{section \ref*{#1}}}

\parskip1ex
\parindent0pt
\let\olditemize\itemize
\def\itemize{\olditemize\parskip0pt}

\begin{document}

\title{The \textsf{childdoc} Package}
\hypersetup{pdftitle={The childdoc Package}}
\author{Niklas Beisert\\[2ex]
  Institut f\"ur Theoretische Physik\\
  Eidgen\"ossische Technische Hochschule Z\"urich\\
  Wolfgang-Pauli-Strasse 27, 8093 Z\"urich, Switzerland\\[1ex]
  \href{mailto:nbeisert@itp.phys.ethz.ch}
  {\texttt{nbeisert@itp.phys.ethz.ch}}}
\hypersetup{pdfauthor={Niklas Beisert}}
\hypersetup{pdfsubject={Manual for the LaTeX2e Package childdoc}}
\date{30 December 2018, \textsf{v2.0}}
\maketitle

\begin{abstract}\noindent
\textsf{childdoc} is a \LaTeXe{} package
that enables the direct compilation
of document sections included by |\include|
to individual files.
\end{abstract}

\begingroup
\parskip0ex
\tableofcontents
\endgroup

%%%%%%%%%%%%%%%%%%%%%%%%%%%%%%%%%%%%%%%%%%%%%%%%%%%%%%%%%%%%%%%%%%%%%%%%%%%%%%%%
%%%%%%%%%%%%%%%%%%%%%%%%%%%%%%%%%%%%%%%%%%%%%%%%%%%%%%%%%%%%%%%%%%%%%%%%%%%%%%%%
\section{Introduction}

\LaTeX{} provides a mechanism to structure a large document (such as a book)
into a main file and several child files (containing the chapters)
using the |\include| command.
This mechanism is beneficial for documents
which span hundreds of pages in order to
make the source file(s) more manageable.
Moreover, compilation can be restricted to
selected child files by means of the |\includeonly| command.
The latter feature can be used to reduce the compilation time while editing
(this was significantly more useful in the earlier days of \LaTeX{})
or to generate a smaller document which is easier to navigate.
Another application of |\includeonly| is to generate
documents consisting of selected parts of the complete document.

However, there are a few drawbacks of the plain |\include| mechanism:
\begin{itemize}
\item
The child files cannot be compiled on their own,
they can only be compiled via the main file.
A naive editing environment
(such as a text editor with an option
to have the current file processed by \LaTeX)
may require one to switch to the main file before compiling;
attempting to compile the child file produces errors.
\item
The main file must be modified (each time)
to adjust the |\includeonly| command
to the present needs. This easily leaves the main file in a messy state.
\item
The generated document will always carry the filename
of the main document. This is inconvenient if
several child files are to be compiled and
to be kept for distribution.
\end{itemize}

The present package provides a simple interface
to make child files individually compilable by \LaTeX{}.
Compiling a child file then has the same effect as compiling
the main file with an |\includeonly| command
to select the appropriate child.
Moreover the generated document will carry the name of the child
rather than the main file.
This resolves all three above issues.

This feature is meant to make the editing of books,
thesis documents and lecture notes somewhat more convenient.
However, the package can also be used efficiently for
composing a series of documents (such as exercise sheets)
which are typically distributed individually.
It then assists the author in generating the individual documents
(potentially in different versions)
as well as a document containing the collected series.
Another application is in developing style files
or other kinds of included material
where compilation of the style file could redirect
to a sample or test file.

%%%%%%%%%%%%%%%%%%%%%%%%%%%%%%%%%%%%%%%%%%%%%%%%%%%%%%%%%%%%%%%%%%%%%%%%%%%%%%%%
%%%%%%%%%%%%%%%%%%%%%%%%%%%%%%%%%%%%%%%%%%%%%%%%%%%%%%%%%%%%%%%%%%%%%%%%%%%%%%%%
\section{Usage}

First of all, the package \textsf{childdoc} is \emph{not} a standard
\LaTeXe{} |.sty| style file! Therefore it needs to be invoked in
a non-standard way.

%%%%%%%%%%%%%%%%%%%%%%%%%%%%%%%%%%%%%%%%%%%%%%%%%%%%%%%%%%%%%%%%%%%%%%%%%%%%%%%%
\subsection{Included Files}
\label{sec:include}

%%%%%%%%%%%%%%%%%%%%%%%%%%%%%%%%%%%%%%%%
\DescribeMacro{\childdocmain}
To use the package, add the commands
\begin{center}
\begin{tabular}{l}
|\input{childdoc.def}|\\
|\childdocmain{}|\\
\end{tabular}
\end{center}
at the very top of the main \LaTeX{} file,
in particular \emph{before} the |\documentclass| statement!
The argument of |\childdocmain| should be left empty
(but it must be present).

%%%%%%%%%%%%%%%%%%%%%%%%%%%%%%%%%%%%%%%%
\DescribeMacro{\childdocof}
Furthermore, add the commands
\begin{center}
\begin{tabular}{l}
|\input{childdoc.def}|\\
|\childdocof{|\textit{main}|}|\\
\end{tabular}
\end{center}
at the top of every child file \textit{child}
which is included by |\include{|\textit{child}|}|
from within the main file
(or at least for those files to be compiled individually).
The argument \textit{main} must be the filename of the main file.

There are a couple of
considerations in setting up the main and child documents:

%%%%%%%%%%%%%%%%%%%%%%%%%%%%%%%%%%%%%%%%
\paragraph{Restrictions.}

Please note the following restrictions:
\begin{itemize}
\item
|\childdocmain| must be called with one argument \textit{main}
to ensure compatibility with earlier version of the package.
It must either be empty (|\childdocmain{}|)
or precisely match the filename of the main file in which it is specified.
See \secref{sec:detection} for further information.
\item
The filename \textit{main} must be specified without the |.tex| extension.
\item
The filename \textit{main} is case sensitive
(even in case-insensitive file systems)
due to internal string comparison.
\item
The argument \textit{main} should be fully expanded, it cannot be a macro.
\item
Subdirectories and special characters should be avoided in filenames.
\item
The command |\childdocmain{|\textit{main}|}| must be followed by a whitespace.
It should not be followed immediately by another command
or by a comment mark `|%|'.
This is because the \TeX{} parser reads the token immediately following
the argument of |\childdocmain| and puts it
at the beginning of every child section;
however, a white\-space is ignored.
\end{itemize}

%%%%%%%%%%%%%%%%%%%%%%%%%%%%%%%%%%%%%%%%
\paragraph{Content of Main File.}

It is advisable to place all content in the child files included by |\include|.
Any output contained in the main file will appear in all child documents
unless suppressed manually;
it cannot be suppressed automatically by the |\includeonly| directive
and thus should normally be avoided.
A method to include some content in the main file
by means of conditional processing is described in \secref{sec:conditional}.

%%%%%%%%%%%%%%%%%%%%%%%%%%%%%%%%%%%%%%%%
\paragraph{Page Numbering.}

When only a part of the document is compiled,
the appropriate numbering of pages
(as well as other status parameters)
is determined from the |.aux| files.
The latter contain information from previous passes.
However this information needs to propagate through
all intermediate child documents.
Therefore the page numbering in child documents may well
be inconsistent until the complete document is compiled at least once.

A useful (if unconventional) way to always ensure a consistent
page numbering is to restart the numbering in each child document
and denote the pages by `\textit{child}|.|\textit{page}'
where \textit{child} represents the chapter/section number of the child file.
This can be achieved by the command
|\numberwithin{page}{|\textit{child}|}|
of the \textsf{amsmath} package
where \textit{child} can be |chapter| or |section|
depending on the chosen structuring.
Alternatively, one can modify the macro |\thepage| appropriately
and reset the counter |page| at the start of each child file.

%%%%%%%%%%%%%%%%%%%%%%%%%%%%%%%%%%%%%%%%%%%%%%%%%%%%%%%%%%%%%%%%%%%%%%%%%%%%%%%%
\subsection{Conditional Processing}
\label{sec:conditional}

The package provides a mechanism to compile different versions
of a document. To customise the versions further some conditional processing
can come in handy to distinguish which version is being compiled.
The package provides two macros to describe the compilation context:

%%%%%%%%%%%%%%%%%%%%%%%%%%%%%%%%%%%%%%%%
\DescribeMacro{\ifchilddoc}
The conditional |\ifchilddoc| distinguishes between the compilation of
child documents and the main document:
%
\begin{center}
|\ifchilddoc |\textit{child-code}| |[|\||else |\textit{main-code}]| \||fi|
\end{center}

%%%%%%%%%%%%%%%%%%%%%%%%%%%%%%%%%%%%%%%%
\DescribeMacro{\childdocname}
\DescribeMacro{\childdocjob}
The macro |\childdocname| contains the filename (without extension)
of the main or child file being processed.
Note that |\childdocjob| will always contain the name of the main file.

%%%%%%%%%%%%%%%%%%%%%%%%%%%%%%%%%%%%%%%%
\paragraph{Title Page.}

Conditional processing can be used to include a title or banner page
in the main document when proper precautions are taken.
Importantly, the code in the main file should ensure that the page counter
(as well as other status parameters which are stored in the |.aux| files)
takes the same value after the conditional processing.
Otherwise the page numbers may take divergent values
depending on which part is compiled.

For example, a title page could be declared by:
%
\begin{center}
\begin{tabular}{l}
|\ifchilddoc\||else|\\
|\addtocounter{page}{-1}|\\
\textit{code for title page}\\
|\newpage|\\
|\||fi|
\end{tabular}
\end{center}
%
A banner page for the child documents can be generated by:
%
\begin{center}
\begin{tabular}{l}
|\ifchilddoc|\\
|\addtocounter{page}{-1}|\\
\textit{code for banner page}\\
|\newpage|\\
|\||fi|
\end{tabular}
\end{center}
%
Here one could write a message such as:
\begin{center}
|This is the part \childdocname{} of \childdocjob{}.|
\end{center}

%%%%%%%%%%%%%%%%%%%%%%%%%%%%%%%%%%%%%%%%%%%%%%%%%%%%%%%%%%%%%%%%%%%%%%%%%%%%%%%%
\subsection{Flags}
\label{sec:flags}

The package makes it easy to generate different versions
of the main or child documents.
To this end compilation flags can be defined
and assigned different default values.
They will be particularly useful in conjunction
with the forwarding mechanism described in \secref{sec:forward}.

For example, it may be useful to have a flag |\version|
which can be set to |draft| or |final|.
The document source will contain some conditional code
depending on the value of |\version|.
Suppose further, the flag should default to |final| for the main file
and to |draft| for child files
which is a natural assignment for editing the document.
This is achieved by placing the following code
in the preamble of the main document
(below the |\childdocmain| directive):
%
\begin{center}
\begin{tabular}{l}
|\ifchilddoc|\\
|\providecommand{\version}{draft}|\\
|\||else|\\
|\providecommand{\version}{final}|\\
|\||fi|
\end{tabular}
\end{center}
%
The definition by |\providecommand| makes sure
that previous definitions are not overwritten.
Further statements |\providecommand{\version}{...}|
can thus be added before the above code to override it.

For the main file, one might add a line
(between |\childdocmain| and the above block)
%
\begin{center}
|%\ifchilddoc\||else\providecommand{\version}{draft}\||fi|
\end{center}
%
which can be uncommented to produce a draft version.
Likewise one can add a line to the very top of a child file
(above the |\childdocof{|\textit{main}|}| directive)
%
\begin{center}
|%\providecommand{\version}{final}|
\end{center}
%
which can be uncommented to produce the final version of this child document.

%%%%%%%%%%%%%%%%%%%%%%%%%%%%%%%%%%%%%%%%%%%%%%%%%%%%%%%%%%%%%%%%%%%%%%%%%%%%%%%%
\subsection{Forwarding}
\label{sec:forward}

Different versions of the main or child documents
using compilation flags as described in \secref{sec:flags}
can be (permanently) stored in different files
for convenient compilation, viewing and distribution.
To this end, the package defines a command
to pass on compilation to a different file:

%%%%%%%%%%%%%%%%%%%%%%%%%%%%%%%%%%%%%%%%
\DescribeMacro{\childdocforward}
The command |\childdocforward| redirects processing to
another source file:
%
\begin{center}
\begin{tabular}{l}
|\input{childdoc.def}|\\
|\childdocforward[|\textit{main}|]{|\textit{dest}|}|\\
\end{tabular}
\end{center}
%
The argument \textit{dest} is the destination file
(without extension).
It should be the main file or one of the child files.
Note that further \textsf{childdoc} directives
such as |\childdocof| and |\childdocforward|
in the indicated file will be processed in this form.
The optional argument \textit{main}
passes on directly to the main file \textit{main}
while pretending to compile the child \textit{dest}.
This form behaves as if \textit{dest}
issues |\childdocof{|\textit{main}|}| right away,
and no further \textsf{childdoc} directives will be processed.

%%%%%%%%%%%%%%%%%%%%%%%%%%%%%%%%%%%%%%%%
\DescribeMacro{\...prefix}
In the alternative form |\childdocforwardprefix|,
%
\begin{center}
\begin{tabular}{l}
|\input{childdoc.def}|\\
|\childdocforwardprefix[|\textit{main}|]{|\textit{prefix}|}{|\textit{dest}|}|
\end{tabular}
\end{center}
%
the destination file is determined by a pattern
depending on the current file:
To make this work, the current file must be called
`{\textit{prefix}\hspace{0.2em}\textit{suffix}}'
with \textit{prefix} matching precisely the argument.
Processing is then passed on to the file
`{\textit{dest}\hspace{0.2em}\textit{suffix}}'.
Surely, the same effect is achieved by
directly specifying the
argument `{\textit{dest}\hspace{0.2em}\textit{suffix}}'
in the first form.
However, that requires to set up a different file
for each child. With the alternative form of the command
all these files can have exactly the same content
which simplifies setting them up and maintaining them.

For example, the following file |draft.tex|
with a compilation flag |\version| as described in \secref{sec:flags}
compiles the main document as a draft:
%
\begin{center}
\begin{tabular}{l}
|\def\version{draft}|\\
|\input{childdoc.def}|\\
|\childdocforward{|\textit{main}|}|
\end{tabular}
\end{center}
%
Likewise, the following files |final|\textit{nn}|.tex|
compile the final version of the child document
|child|\textit{nn}|.tex|:
%
\begin{center}
\begin{tabular}{l}
|\def\version{final}|\\
|\input{childdoc.def}|\\
|\childdocforwardprefix{final}{child}|
\end{tabular}
\end{center}
%

Note that when several versions of a main file and/or of each child file
are to be generated, it may be convenient to set up a |Makefile| or
shell script to automatise the process.

%%%%%%%%%%%%%%%%%%%%%%%%%%%%%%%%%%%%%%%%%%%%%%%%%%%%%%%%%%%%%%%%%%%%%%%%%%%%%%%%
\subsection{Command Line Processing}
\label{sec:commandline}

The effect of redirection files can also be achieved by invoking
the \LaTeX{} compiler with a more elaborate command line.
Most conveniently this should be done as part
of a shell script or a |Makefile|.

When using \textsf{childdoc} in the main file, the following
command lines effectively perform a redirection
(note that depending on the shell being used,
backslashes may have to be doubled: `|\|' $\to$ `|\\|'):
%
\begin{center}
|... -jobname "|\textit{target}|" |\\|"|[\textit{flags}]%
|\input{childdoc.def}\childdocforward[|\textit{main}|]{|\textit{dest}|}"|
\end{center}
%
Here \textit{target} is the name of the output file,
\textit{main} is the name of the main file
and \textit{dest} is the name of the main or child file to be processed
(all filenames without extensions).
The optional argument \textit{main} can be omitted
if \textit{main} matches \textit{dest}.
Optionally, compilation \textit{flags} can be defined via |\def| commands.
This command line makes the \TeX{} engine believe
it is compiling the file \textit{target}
whose content is specified as the latter parameter.
The provided code then forwards the processing to
\textit{main} or \textit{dest} as described in \secref{sec:forward}.

%%%%%%%%%%%%%%%%%%%%%%%%%%%%%%%%%%%%%%%%%%%%%%%%%%%%%%%%%%%%%%%%%%%%%%%%%%%%%%%%
\subsection{Include by Input}
\label{sec:input}

Including child documents by |\include| has some restrictions by design.
Most notably, the content of a child document always occupies
its own set of pages; pages cannot be shared between child documents.
Usually, this behaviour makes perfect sense
because each child document contain an essential part of the document.
However, in some situations it may be desirable to compose
a document from a collection of parts
without having mandatory page breaks between then.
For this case, the package
provides a mechanism to include parts
by |\input| which can also be processed individually.
However, by construction this mechanism
requires manual handling of the content to be output.

%%%%%%%%%%%%%%%%%%%%%%%%%%%%%%%%%%%%%%%%
\DescribeMacro{\ifchilddocmanual}
The main file should be prepared as usual, see \secref{sec:include}.
However, the document body must make a distinction
between processing of an individual part and of the main document, e.g.:
%
\begin{center}
\begin{tabular}{l}
|\ifchilddocmanual|\\
|\input{\childdocname}|\\
|\||else|\\
\textit{document body with }|\input{|\textit{part}|}|\\
|\||fi|
\end{tabular}
\end{center}
%
The conditional |\ifchilddocmanual| is true whenever
a part to be included by |\input| is being compiled,
and the name of the part is stored in |\childdocname|.

%%%%%%%%%%%%%%%%%%%%%%%%%%%%%%%%%%%%%%%%
\DescribeMacro{\childdocby}
Each part to be included by |\input| should start with:
%
\begin{center}
\begin{tabular}{l}
|\input{childdoc.def}|\\
|\childdocby{|\textit{main}|}|\\
\end{tabular}
\end{center}
%
The directive |\childdocby| is similar to |\childdocof|
described in \secref{sec:include},
but the subsequent selection of content must be done manually.
To that end, both |\ifchilddoc| and |\ifchilddocmanual|
will be true upon processing of a part,
and the name of the part is stored in |\childdocname|.
Note that |\jobname| will be set to the filename of the current part
so that each part receives an individual |.aux| file
that does not interfere with the |.aux| file(s) of the main document.
This behaviour can be altered by the alternative form
|\childdocby[*]{|\textit{main}|}| (with a non-empty optional argument)
which uses the |.aux| file of the main document
by setting |\jobname| to \textit{main}.

%%%%%%%%%%%%%%%%%%%%%%%%%%%%%%%%%%%%%%%%%%%%%%%%%%%%%%%%%%%%%%%%%%%%%%%%%%%%%%%%
\subsection{Driver Development}
\label{sec:driver}

The \textsf{childdoc} mechanism can also be use for the development
of definition files such as \LaTeX{} styles or classes.
This case differs from the above setup with multiple parts
included by |\include| in that no |\includeonly| should be invoked.
This can be achieved by starting the include file
(before |\ProvidesPackage|) with:
%
\begin{center}
\begin{tabular}{l}
|\input{childdoc.def}|\\
|\childdocforward{|\textit{main}|}|\\
\end{tabular}
\end{center}
%
or alternatively with:
%
\begin{center}
\begin{tabular}{l}
|\input{childdoc.def}|\\
|\childdocby{|\textit{main}|}|\\
\end{tabular}
\end{center}
%
Both forms have slightly different effects as described above.
The main file is prepared as usual, see \secref{sec:include}.

%%%%%%%%%%%%%%%%%%%%%%%%%%%%%%%%%%%%%%%%%%%%%%%%%%%%%%%%%%%%%%%%%%%%%%%%%%%%%%%%
\subsection{Legacy Detection}
\label{sec:detection}

The directive |\childdocmain| in the main file can detect
whether the complete document or merely a child is to be compiled
even without using the directive |\childdocof|.
This method is deprecated because it is less robust
and there is no compelling reason to use it;
it is merely provided for backward compatibility
and it may be removed in future versions.

If the detection mechanism is to be used,
it is mandatory to correctly specify
the filename of the main file as the argument of |\childdocmain|:
%
\begin{center}
\begin{tabular}{l}
|\input{childdoc.def}|\\
|\childdocmain{|\textit{main}|}|\\
\end{tabular}
\end{center}
%
If |\jobname| does not match the argument \textit{main} of |\childdocmain|,
it is assumed that |\jobname| points to the child file to be compiled.
When using |\childdocmain| with the main file specified as argument,
it suffices to start a child file
with just |\input{|\textit{main}|}|
without loading of the package and using |\childdocof|.
If instead all processing is done
with the appropriate \textsf{childdoc} directives,
the argument of \textit{main} of |\childdocmain| can be empty.

An alternative version of the command line processing described
in \secref{sec:commandline} using the detection mechanism reads:
%
\begin{center}
|... -jobname "|\textit{target}|" "|[\textit{flags}]%
[|\def\jobname{|\textit{dest}|}|]|\input{|\textit{main}|}"|
\end{center}

%%%%%%%%%%%%%%%%%%%%%%%%%%%%%%%%%%%%%%%%%%%%%%%%%%%%%%%%%%%%%%%%%%%%%%%%%%%%%%%%
\subsection{Manual Code}
\label{sec:manual}

In case one cannot be certain whether the definitions file |childdoc.def|
is installed on the target \TeX{} distribution
and one prefers not to ship it,
it is conceivable to paste a few relevant commands into the sources.

To that end, drop all statements |\input{childdoc.def}|
and perform the replacements as outlined below.
Instead of |\childdocmain{|\textit{main}|}| add the following code
to the top of the main file:
%
\begin{center}
\begin{tabular}{l}
|\||ifdefined\childdocname\endinput\||fi\newif\ifchilddoc|\\
|\edef\childdocname{\scantokens\expandafter{\jobname\noexpand}}|\\
|\def\childdocmain{|\textit{main}|}\||ifx\childdocmain\childdocname\||else|\\
|\childdoctrue\includeonly{\childdocname}\let\jobname\childdocmain\||fi|\\
\end{tabular}
\end{center}
%
Instead of |\childdocof{|\textit{main}|}| just include the main file
at the top of each child file:
%
\begin{center}
|\input{|\textit{main}|}|
\end{center}
%
A simple redirection |\childdocforward{|\textit{dest}|}| is achieved by:
%
\begin{center}
|\def\jobname{|\textit{dest}|}\input{\jobname}|
\end{center}
%
The redirection with prefix
|\childdocforwardprefix[|\textit{prefix}|]{|\textit{dest}|}|
is accomplished by:
%
\begin{center}
\begin{tabular}{l}
|{\edef\jobname{\scantokens\expandafter{\jobname\noexpand}}|\\
|\def\redirectjob |\textit{prefix}|#1~~~{\gdef\jobname{|\textit{dest}|#1}}|\\
|\expandafter\redirectjob\jobname~~~}\input{\jobname}|
\end{tabular}
\end{center}

In an alternative approach,
child documents can be compiled by a specific command line
without additional code or specific definitions:
%
\begin{center}
|... -jobname "|\textit{target}|" "|[\textit{flags}]%
|\includeonly{|\textit{dest}|}\input{|\textit{main}|}"|
\end{center}
%

%%%%%%%%%%%%%%%%%%%%%%%%%%%%%%%%%%%%%%%%%%%%%%%%%%%%%%%%%%%%%%%%%%%%%%%%%%%%%%%%
%%%%%%%%%%%%%%%%%%%%%%%%%%%%%%%%%%%%%%%%%%%%%%%%%%%%%%%%%%%%%%%%%%%%%%%%%%%%%%%%
\section{Information}

%%%%%%%%%%%%%%%%%%%%%%%%%%%%%%%%%%%%%%%%%%%%%%%%%%%%%%%%%%%%%%%%%%%%%%%%%%%%%%%%
\subsection{Copyright}

Copyright \copyright{} 2017--2018 Niklas Beisert

This work may be distributed and/or modified under the
conditions of the \LaTeX{} Project Public License, either version 1.3
of this license or (at your option) any later version.
The latest version of this license is in
  \url{http://www.latex-project.org/lppl.txt}
and version 1.3 or later is part of all distributions of \LaTeX{}
version 2005/12/01 or later.

This work has the LPPL maintenance status `maintained'.

The Current Maintainer of this work is Niklas Beisert.

This work consists of the files |README.txt|, |childdoc.ins| and |childdoc.dtx|
as well as the derived files |childdoc.def|, |cdocsamp.tex|
with |cdocsch1.tex|, |cdocsch2.tex|, |cdocspt3.tex|, |cdocspt4.tex|,
|cdocsdrf.tex|, |cdocsfn1.tex|, |cdocsfn2.tex|
as well as |childdoc.pdf|.

%%%%%%%%%%%%%%%%%%%%%%%%%%%%%%%%%%%%%%%%%%%%%%%%%%%%%%%%%%%%%%%%%%%%%%%%%%%%%%%%
\subsection{Files and Installation}

The package consists of the files:
%
\begin{center}
\begin{tabular}{ll}
    |README.txt|   & readme file \\
    |childdoc.ins| & installation file \\
    |childdoc.dtx| & source file \\
    |childdoc.def| & definition file \\
    |cdocsamp.tex| & sample main file \\
    |cdocsch1.tex| & sample include file \\
    |cdocsch2.tex| & sample include file \\
    |cdocspt3.tex| & sample part file \\
    |cdocspt4.tex| & sample part file \\
    |cdocsdrf.tex| & sample redirection file \\
    |cdocsfn1.tex| & sample redirection file \\
    |cdocsfn2.tex| & sample redirection file \\
    |childdoc.pdf| & manual
\end{tabular}
\end{center}
%
The distribution consists of the files
|README.txt|, |childdoc.ins| and |childdoc.dtx|.
%
\begin{itemize}
\item
Run (pdf)\LaTeX{} on |childdoc.dtx|
to compile the manual |childdoc.pdf| (this file).
\item
Run \LaTeX{} on |childdoc.ins| to create the definitions file |childdoc.def|
and the sample |cdocsamp.tex| with include files
|cdocsch1.tex|, |cdocsch2.tex|, |cdocspt3.tex|, |cdocspt4.tex|,
|cdocsdrf.tex|, |cdocsfn1.tex|, |cdocsfn2.tex|.
Then copy the file |childdoc.def| to an appropriate directory of your \LaTeX{}
distribution, e.g.\ \textit{texmf-root}|/tex/latex/childdoc|.
\end{itemize}

%%%%%%%%%%%%%%%%%%%%%%%%%%%%%%%%%%%%%%%%%%%%%%%%%%%%%%%%%%%%%%%%%%%%%%%%%%%%%%%%
\subsection{Related CTAN Packages}

There are several other packages which offer a similar functionality:
%
\begin{itemize}
\item
The packages
\href{http://ctan.org/pkg/docmute}{\textsf{docmute}},
\href{http://ctan.org/pkg/includex}{\textsf{includex}} and
\href{http://ctan.org/pkg/standalone}{\textsf{standalone}}
provide commands to include only the document body of
a child file thus allowing both files to be compiled individually.
\item
The packages \href{http://ctan.org/pkg/subdocs}{\textsf{subdocs}}
and \href{http://ctan.org/pkg/subfiles}{\textsf{subfiles}}
provide structures in which the main and child documents can be
encapsulated and allowing them to be compiled individually.
The inclusion mechanism is different from the conventional |\include|.
\item
The package \href{http://ctan.org/pkg/combine}{\textsf{combine}}
is an elaborate solution to combine several documents into one.
\end{itemize}
%
See also the CTAN topic \href{http://ctan.org/topic/subdocs}{\textsf{subdocs}}
for further related packages.
The present package differs from the above solutions in that
a document structure constructed with the conventional |\include| mechanism
just needs two extra commands at the top of every file
such that all constituent files can be compiled individually.

%%%%%%%%%%%%%%%%%%%%%%%%%%%%%%%%%%%%%%%%%%%%%%%%%%%%%%%%%%%%%%%%%%%%%%%%%%%%%%%%
%\subsection{Feature Suggestions}
%
%The following is a list of features which may be useful for future
%versions of this package:
%%
%\begin{itemize}
%\item
%\ldots
%\end{itemize}

%%%%%%%%%%%%%%%%%%%%%%%%%%%%%%%%%%%%%%%%%%%%%%%%%%%%%%%%%%%%%%%%%%%%%%%%%%%%%%%%
\subsection{Revision History}

%%%%%%%%%%%%%%%%%%%%%%%%%%%%%%%%%%%%%%%%
\paragraph{v2.0:} 2018/12/30

\begin{itemize}
\item
immediate forward processing
\item
added |\childdocby| mechanism
\item
manual restructured
\end{itemize}

%%%%%%%%%%%%%%%%%%%%%%%%%%%%%%%%%%%%%%%%
\paragraph{v1.6:} 2018/01/17

\begin{itemize}
\item
application for development of include files
\item
corrections to manual
\end{itemize}

%%%%%%%%%%%%%%%%%%%%%%%%%%%%%%%%%%%%%%%%
\paragraph{v1.5:} 2017/05/21

\begin{itemize}
\item
more complete structuring introduced
\item
|\childdocof| introduced
\item
|\childdoc| renamed to |\childdocmain|
\item
|\childredirect| renamed to |\childdocforward| and |\childdocforwardprefix|
and functionality expanded
\end{itemize}

%%%%%%%%%%%%%%%%%%%%%%%%%%%%%%%%%%%%%%%%
\paragraph{v1.0:} 2017/04/27

\begin{itemize}
\item
manual and install package
\item
first version published on CTAN
\end{itemize}

%%%%%%%%%%%%%%%%%%%%%%%%%%%%%%%%%%%%%%%%
\paragraph{v0.6:} 2017/04/26

\begin{itemize}
\item
redirection mechanism added
\end{itemize}

%%%%%%%%%%%%%%%%%%%%%%%%%%%%%%%%%%%%%%%%
\paragraph{v0.5:} 2017/04/26

\begin{itemize}
\item
functionality in definition file
\end{itemize}


%%%%%%%%%%%%%%%%%%%%%%%%%%%%%%%%%%%%%%%%%%%%%%%%%%%%%%%%%%%%%%%%%%%%%%%%%%%%%%%%
%%%%%%%%%%%%%%%%%%%%%%%%%%%%%%%%%%%%%%%%%%%%%%%%%%%%%%%%%%%%%%%%%%%%%%%%%%%%%%%%
%%%%%%%%%%%%%%%%%%%%%%%%%%%%%%%%%%%%%%%%%%%%%%%%%%%%%%%%%%%%%%%%%%%%%%%%%%%%%%%%
\appendix

\settowidth\MacroIndent{\rmfamily\scriptsize 000\ }

 \DocInput{childdoc.dtx}

\end{document}
%</driver>
% \fi
%
% %%%%%%%%%%%%%%%%%%%%%%%%%%%%%%%%%%%%%%%%%%%%%%%%%%%%%%%%%%%%%%%%%%%%%%%%%%%%%%
% %%%%%%%%%%%%%%%%%%%%%%%%%%%%%%%%%%%%%%%%%%%%%%%%%%%%%%%%%%%%%%%%%%%%%%%%%%%%%%
% \section{Sample}
%\iffalse
%<*samplemain>
%\fi
%
% The following presents a sample document
% with two chapters, two parts, a title page,
% a compile flag as well as three forwarding files to set the flag.
% It consists of eight |.tex| files:
% \begin{center}
% \begin{tabular}{ll}
% |cdocsamp.tex|&main file\\
% |cdocsch1.tex|&include file for chapter 1\\
% |cdocsch2.tex|&include file for chapter 2\\
% |cdocspt3.tex|&include file for part 3\\
% |cdocspt4.tex|&include file for part 4\\
% |cdocsdrf.tex|&forwarding file for main file in draft mode\\
% |cdocsfi1.tex|&forwarding file for final version of chapter 1\\
% |cdocsfi2.tex|&forwarding file for final version of chapter 2\\
% \end{tabular}
% \end{center}
% Each of the eight files can be compiled directly by the \LaTeX{} compiler.
%
% %%%%%%%%%%%%%%%%%%%%%%%%%%%%%%%%%%%%%%
% \paragraph{Main File.}
%
% The main file is called |cdocsamp.tex|.
%
% Load the \textsf{childdoc} definitions and
% declare the filename for the main document:
%    \begin{macrocode}
\input{childdoc.def}
\childdocmain{}
%    \end{macrocode}

% Optional override for |\version| flag:
%    \begin{macrocode}
%%\ifchilddoc\else\providecommand{\version}{draft}\fi
%    \end{macrocode}

% Define the default values for the |\version| flag
% (|final| for the main file and |draft| for childs):
%    \begin{macrocode}
\ifchilddoc
\providecommand{\version}{draft}
\else
\providecommand{\version}{final}
\fi
%    \end{macrocode}

% Load the standard document class:
%    \begin{macrocode}
\documentclass[12pt]{article}
%    \end{macrocode}

% Start the document body:
%    \begin{macrocode}
\begin{document}
%    \end{macrocode}

% Declare a title page.
% Print title, part of document being processed and version flag:
%    \begin{macrocode}
\addtocounter{page}{-1}
\begin{center}
{\LARGE\bfseries{}childdoc example\par}
\vspace{1cm}
\ifchilddoc
\ifchilddocmanual part\else chapter\fi:
`\childdocname' of `\childdocjob'\par
\else
main document: `\childdocjob'\par
\fi
version: \version\par
\end{center}
\newpage
%    \end{macrocode}

% Manually include selected file,
% otherwise process as usual:
%    \begin{macrocode}
\ifchilddocmanual
\section*{part `\childdocname'}
\input{\childdocname}
\else
%    \end{macrocode}

% Include the two chapters:
%    \begin{macrocode}
\include{cdocsch1}
\include{cdocsch2}
%    \end{macrocode}

% Include the two parts unless only chapters should be displayed:
%    \begin{macrocode}
\ifchilddoc\else
\section{part three}
\input{cdocspt3}
\section{part four}
\input{cdocspt4}
\fi
%    \end{macrocode}

% Process as usual until here:
%    \begin{macrocode}
\fi
%    \end{macrocode}

% End of document body:
%    \begin{macrocode}
\end{document}
%    \end{macrocode}
%\iffalse
%</samplemain>
%\fi
%
% %%%%%%%%%%%%%%%%%%%%%%%%%%%%%%%%%%%%%%
% \paragraph{Chapter Include Files.}
%
% The include files are called |cdocsch1.tex| and |cdocsch2.tex|.
%
%\iffalse
%<*samplechap1|samplechap2>
%\fi

% Optional override for |\version| flag:
%    \begin{macrocode}
%%\providecommand{\version}{final}
%    \end{macrocode}

% Include the main document:
%    \begin{macrocode}
\input{childdoc.def}
\childdocof{cdocsamp}
%    \end{macrocode}

%\iffalse
%</samplechap1|samplechap2>
%\fi
%
%\iffalse
%<*samplechap1>
%\fi
% Some text for chapter 1:
%    \begin{macrocode}
\section{one}
some text in chapter one
%    \end{macrocode}

%\iffalse
%</samplechap1>
%\fi
% Some text for chapter 2:
%\iffalse
%<*samplechap2>
%\fi
%    \begin{macrocode}
\section{two}
more text in chapter two
%    \end{macrocode}

%\iffalse
%</samplechap2>
%\fi
%
% %%%%%%%%%%%%%%%%%%%%%%%%%%%%%%%%%%%%%%
% \paragraph{Part Include Files.}
%
% The include files are called |cdocspt3.tex| and |cdocspt4.tex|.
%
%\iffalse
%<*samplepart3|samplepart4>
%\fi

% Optional override for |\version| flag:
%    \begin{macrocode}
%%\providecommand{\version}{final}
%    \end{macrocode}

% Include the main document:
%    \begin{macrocode}
\input{childdoc.def}
\childdocby{cdocsamp}
%    \end{macrocode}

%\iffalse
%</samplepart3|samplepart4>
%\fi
%
%\iffalse
%<*samplepart3>
%\fi
% Some text for part 3:
%    \begin{macrocode}
some text in part three
%    \end{macrocode}

%\iffalse
%</samplepart3>
%\fi
% Some text for part 4:
%\iffalse
%<*samplepart4>
%\fi
%    \begin{macrocode}
more text in part four
%    \end{macrocode}

%\iffalse
%</samplepart4>
%\fi
%
% %%%%%%%%%%%%%%%%%%%%%%%%%%%%%%%%%%%%%%
% \paragraph{Forwarding for a Complete Draft.}
%
% The following forwarding file |cdocsdrf.tex|
% compiles the main document in draft mode:
%\iffalse
%<*sampledraft>
%\fi
%    \begin{macrocode}
\def\version{draft}
\input{childdoc.def}
\childdocforward{cdocsamp}
%    \end{macrocode}

%\iffalse
%</sampledraft>
%\fi
%
% %%%%%%%%%%%%%%%%%%%%%%%%%%%%%%%%%%%%%%
% \paragraph{Forwarding for Final Version of the Chapters.}
%
% The following forwarding files |cdocsfn1.tex| and |cdocsfn2.tex|
% (with identical content)
% compile the final versions of the child documents
% |cdocsch1.tex| and |cdocsch2.tex|, respectively:
%\iffalse
%<*samplefinal>
%\fi
%    \begin{macrocode}
\def\version{final}
\input{childdoc.def}
\childdocforwardprefix[cdocsamp]{cdocsfn}{cdocsch}
%    \end{macrocode}

%\iffalse
%</samplefinal>
%\fi
%
% %%%%%%%%%%%%%%%%%%%%%%%%%%%%%%%%%%%%%%
% \paragraph{Command Line Processing.}
%
% The following three command lines generate the output files
% |cdocscld|, |cdocscl1| and |cdocscl2|
% which should be identical to
% |cdocsdrf|, |cdocsch1| and |cdocsfn2|, respectively:
% \begin{center}
% \begin{tabular}{l}
% |latex -jobname cdocscld \|\\
% |  "\def\version{draft}\input{childdoc.def}\childdocforward{cdocsamp}"|\\
% |latex -jobname cdocscl1 \|\\
% |  "\input{childdoc.def}\childdocforward[cdocsamp]{cdocsch1}"|\\
% |latex -jobname cdocscl2 \|\\
% |  "\def\version{final}\input{childdoc.def}\childdocforward{cdocsch2}"|
% \end{tabular}
% \end{center}
% Note that the trailing backslash on each first line
% merely continues the input to the second line
% (for convenient cut ant paste).
% Furthermore, the command |latex| can be replaced by any
% of its alternative versions such as |pdflatex|.
%
% %%%%%%%%%%%%%%%%%%%%%%%%%%%%%%%%%%%%%%%%%%%%%%%%%%%%%%%%%%%%%%%%%%%%%%%%%%%%%%
% %%%%%%%%%%%%%%%%%%%%%%%%%%%%%%%%%%%%%%%%%%%%%%%%%%%%%%%%%%%%%%%%%%%%%%%%%%%%%%
% \section{Implementation}
%\iffalse
%<*package>
%\fi
%
% This section describes the definitions file |childdoc.def|.

% The definitions cannot be loaded using |\usepackage| or |\RequirePackage|
% which has a mechanism to prevent loading a style file more than once.
% When loading the definitions by means of |\input|
% multiple instances have to be prevented manually:
%\iffalse
%This code needs to be before the `\ProvidesFile' directive
%which is defined at the beginning of this file.
%Therefore it is also placed there and commented out here.
%</package>
%<*discard>
%\fi
%    \begin{macrocode}
\ifdefined\childdocmain\endinput\fi
%    \end{macrocode}
%\iffalse
%</discard>
%<*package>
%\fi
%
% \macro{\ifchilddoc}
% \macro{\ifchilddocmanual}
% The conditional |\ifchilddoc| tells whether a
% child (true) or main (false) document is being compiled.
% The conditional |\ifchilddocmanual| tells whether
% the |\includeonly| mechanism is used (false) or
% the selection of child files must be performed manually (true).
% The definitions initialise to false:
%    \begin{macrocode}
\newif\ifchilddoc
\newif\ifchilddocmanual
%    \end{macrocode}

% \macro{\childdocname}
% \macro{\childdocjob}
% The macro |\childdocname| stores the name of the main document
% to be compiled. The macro |\childdocjob| stores the name of
% the document on which the \LaTeX{} compiler was originally invoked.
% The content of |\jobname| cannot be compared
% to filenames specified in the source due to different catcodes.
% The following code rescans |\jobname|, stores the result
% in |\childdocname| and saves a copy in |\childdocjob|:
%    \begin{macrocode}
\edef\childdocname{\scantokens\expandafter{\jobname\noexpand}}
\let\childdocjob\childdocname
%    \end{macrocode}

% \macro{\childdocdisable}
% The macro |\childdocdisable| prevents the main file
% from being processed more than once.
% At this stage, the main document command |\childdocmain|
% is assumed to be called once again where it should do nothing.
% Any subsequent call to it should prevent
% a secondary processing of the main document
% It overwrites the forwarding commands
% |\childdocof| and |\childdocforward|
% with empty macros to prevent further inclusions of the main document:
%    \begin{macrocode}
\newcommand{\childdocdisable}
{
  \renewcommand{\childdocmain}[1]{\renewcommand{\childdocmain}[1]{\endinput}}
  \renewcommand{\childdocof}[1]{}
  \renewcommand{\childdocby}[2][]{}
  \renewcommand{\childdocforward}[2][]{}
  \renewcommand{\childdocdisable}{}
}
%    \end{macrocode}

% \macro{\childdocmain}
% The macro |\childdocmain| is to be called at the top of the main file
% with nothing or the main filename (without extension) as argument.
% First, it breaks loops.
% If the argument is not empty and does not match |\childdocname|
% (which is set by the first inclusion of |childdoc.def|),
% |\ifchilddoc| is set to true, |\includeonly| is applied to the child file
% and |\jobname| is set to the main file
% (for proper handling of |.aux| files):
%    \begin{macrocode}
\newcommand{\childdocmain}[1]
{
  \childdocdisable\childdocmain{}
  \if?#1?\else
    \begingroup
      \def\childdoctmp{#1}
      \ifx\childdoctmp\childdocname
        \def\childdoctmp{}
      \else
        \def\childdoctmp
        {
          \childdoctrue
          \includeonly{\childdocname}
          \def\childdocjob{#1}
          \def\jobname{#1}
        }
      \fi
      \expandafter
    \endgroup
    \childdoctmp
  \fi
}
%    \end{macrocode}

% \macro{\childdocof}
% The command |\childdocof| redirects
% compilation to the main file |#1|.
%    \begin{macrocode}
\newcommand{\childdocof}[1]
{
  \childdocdisable
  \childdoctrue
  \includeonly{\childdocname}
  \def\jobname{#1}
  \def\childdocjob{#1}
  \input{#1}
}
%    \end{macrocode}

% \macro{\childdocby}
% The command |\childdocby| ....
%    \begin{macrocode}
\newcommand{\childdocby}[2][]
{
  \childdocdisable
  \childdoctrue
  \childdocmanualtrue
  \if?#1?\else
    \def\jobname{#2}
  \fi
  \def\childdocjob{#2}
  \input{#2}
  \endinput
}
%    \end{macrocode}

% \macro{\childdocforward}
% The command |\childdocforward| redirects
% compilation to the main file or
% (if the optional argument is given) a child file.
% Parameters are set as if the main file
% or a child file starting with |\childdocof| was compiled.
% Then compilation is handed over to the main file:
%    \begin{macrocode}
\newcommand{\childdocforward}[2][]
{
  \begingroup
    \if?#1?
      \def\childdoctmp
      {
        \def\childdocname{#2}
        \def\childdocjob{#2}
        \def\jobname{#2}
        \input{#2}
        \endinput
      }
    \else
      \def\childdoctmp
      {
        \childdocdisable
        \def\childdocname{#2}
        \childdoctrue
        \includeonly{#2}
        \def\childdocjob{#1}
        \def\jobname{#1}
        \input{#1}
        \endinput
      }
    \fi
    \expandafter
  \endgroup
  \childdoctmp
}
%    \end{macrocode}

% \macro{\childdocforwardprefix}
% The command |\childdocforwardprefix| redirects
% compilation to the main or a child file by means of a pattern.
% The prefix |#1| in the current filename is replaced by |#2|
% and the suffix of the current filename is kept
% (it is assumed that the filename does not contain the substring `|~~~|'
% which is used as a delimiter).
% Compilation is handed over to the new file by |\childdocforward|:
%    \begin{macrocode}
\newcommand{\childdocforwardprefix}[3][]
{
  \begingroup
    \def\childdocextract #2##1~~~{\def\childdoctmp{\childdocforward[#1]{#3##1}}}
    \expandafter\childdocextract\childdocname~~~
    \expandafter
  \endgroup
  \childdoctmp
}
%    \end{macrocode}

% \macro{\childdoc}
% The deprecated macro |\childdoc| is a legacy version of |\childdocmain|:
%    \begin{macrocode}
\newcommand{\childdoc}{\childdocmain}
%    \end{macrocode}

% \macro{\childdocredirect}
% The deprecated macro |\childdocredirect| is a legacy version
% of |\childdocforward| and |\childdocforwardprefix|:
%    \begin{macrocode}
\newcommand{\childdocredirect}[2][]
{
  \begingroup
    \if?#1?
      \def\childdoctmp{\childdocforward{#2}}
    \else
      \def\childdoctmp{\childdocforwardprefix{#1}{#2}}
    \fi
    \expandafter
  \endgroup
  \childdoctmp
}
%    \end{macrocode}

%\iffalse
%</package>
%\fi
%
\endinput
\childdocforward{cdocsamp}"|\\
% |latex -jobname cdocscl1 \|\\
% |  "% \iffalse
%
% childdoc.dtx Copyright (C) 2017-2018 Niklas Beisert
%
% This work may be distributed and/or modified under the
% conditions of the LaTeX Project Public License, either version 1.3
% of this license or (at your option) any later version.
% The latest version of this license is in
%   http://www.latex-project.org/lppl.txt
% and version 1.3 or later is part of all distributions of LaTeX
% version 2005/12/01 or later.
%
% This work has the LPPL maintenance status `maintained'.
%
% The Current Maintainer of this work is Niklas Beisert.
%
% This work consists of the files childdoc.dtx and childdoc.ins
% and the derived files childdoc.def and cdocsamp.tex with
% cdocsch1.tex, cdocsch2.tex, cdocsdrf.tex, cdocsfn1.tex, cdocsfn2.tex.
%
%<package>\ifdefined\childdocmain\endinput\fi
%<package>\ProvidesFile{childdoc.def}[2018/12/30 v2.0 child document driver]
%<samplemain>\ProvidesFile{cdocsamp.tex}[2018/12/30 v2.0 sample for childdoc]
%<*driver>
%\ProvidesFile{childdoc.drv}[2018/12/30 v2.0 childdoc reference manual file]
\PassOptionsToClass{10pt,a4paper}{article}
\documentclass{ltxdoc}

\usepackage[margin=35mm]{geometry}
\usepackage{hyperref}
\usepackage{hyperxmp}
\usepackage[usenames]{color}

\hypersetup{colorlinks=true}
\hypersetup{pdfstartview=FitH}
\hypersetup{pdfpagemode=UseNone}
\hypersetup{pdfsource={}}
\hypersetup{pdflang={en-UK}}
\hypersetup{pdfcopyright={Copyright 2017-2018 Niklas Beisert.
  This work may be distributed and/or modified under the
  conditions of the LaTeX Project Public License, either version 1.3
  of this license or (at your option) any later version.}}
\hypersetup{pdflicenseurl={http://www.latex-project.org/lppl.txt}}
\hypersetup{pdfcontactaddress={ETH Zurich, ITP, HIT K,
  Wolfgang-Pauli-Strasse 27}}
\hypersetup{pdfcontactpostcode={8093}}
\hypersetup{pdfcontactcity={Zurich}}
\hypersetup{pdfcontactcountry={Switzerland}}
\hypersetup{pdfcontactemail={nbeisert@itp.phys.ethz.ch}}
\hypersetup{pdfcontacturl={http://people.phys.ethz.ch/\xmptilde nbeisert/}}

\newcommand{\secref}[1]{\hyperref[#1]{section \ref*{#1}}}

\parskip1ex
\parindent0pt
\let\olditemize\itemize
\def\itemize{\olditemize\parskip0pt}

\begin{document}

\title{The \textsf{childdoc} Package}
\hypersetup{pdftitle={The childdoc Package}}
\author{Niklas Beisert\\[2ex]
  Institut f\"ur Theoretische Physik\\
  Eidgen\"ossische Technische Hochschule Z\"urich\\
  Wolfgang-Pauli-Strasse 27, 8093 Z\"urich, Switzerland\\[1ex]
  \href{mailto:nbeisert@itp.phys.ethz.ch}
  {\texttt{nbeisert@itp.phys.ethz.ch}}}
\hypersetup{pdfauthor={Niklas Beisert}}
\hypersetup{pdfsubject={Manual for the LaTeX2e Package childdoc}}
\date{30 December 2018, \textsf{v2.0}}
\maketitle

\begin{abstract}\noindent
\textsf{childdoc} is a \LaTeXe{} package
that enables the direct compilation
of document sections included by |\include|
to individual files.
\end{abstract}

\begingroup
\parskip0ex
\tableofcontents
\endgroup

%%%%%%%%%%%%%%%%%%%%%%%%%%%%%%%%%%%%%%%%%%%%%%%%%%%%%%%%%%%%%%%%%%%%%%%%%%%%%%%%
%%%%%%%%%%%%%%%%%%%%%%%%%%%%%%%%%%%%%%%%%%%%%%%%%%%%%%%%%%%%%%%%%%%%%%%%%%%%%%%%
\section{Introduction}

\LaTeX{} provides a mechanism to structure a large document (such as a book)
into a main file and several child files (containing the chapters)
using the |\include| command.
This mechanism is beneficial for documents
which span hundreds of pages in order to
make the source file(s) more manageable.
Moreover, compilation can be restricted to
selected child files by means of the |\includeonly| command.
The latter feature can be used to reduce the compilation time while editing
(this was significantly more useful in the earlier days of \LaTeX{})
or to generate a smaller document which is easier to navigate.
Another application of |\includeonly| is to generate
documents consisting of selected parts of the complete document.

However, there are a few drawbacks of the plain |\include| mechanism:
\begin{itemize}
\item
The child files cannot be compiled on their own,
they can only be compiled via the main file.
A naive editing environment
(such as a text editor with an option
to have the current file processed by \LaTeX)
may require one to switch to the main file before compiling;
attempting to compile the child file produces errors.
\item
The main file must be modified (each time)
to adjust the |\includeonly| command
to the present needs. This easily leaves the main file in a messy state.
\item
The generated document will always carry the filename
of the main document. This is inconvenient if
several child files are to be compiled and
to be kept for distribution.
\end{itemize}

The present package provides a simple interface
to make child files individually compilable by \LaTeX{}.
Compiling a child file then has the same effect as compiling
the main file with an |\includeonly| command
to select the appropriate child.
Moreover the generated document will carry the name of the child
rather than the main file.
This resolves all three above issues.

This feature is meant to make the editing of books,
thesis documents and lecture notes somewhat more convenient.
However, the package can also be used efficiently for
composing a series of documents (such as exercise sheets)
which are typically distributed individually.
It then assists the author in generating the individual documents
(potentially in different versions)
as well as a document containing the collected series.
Another application is in developing style files
or other kinds of included material
where compilation of the style file could redirect
to a sample or test file.

%%%%%%%%%%%%%%%%%%%%%%%%%%%%%%%%%%%%%%%%%%%%%%%%%%%%%%%%%%%%%%%%%%%%%%%%%%%%%%%%
%%%%%%%%%%%%%%%%%%%%%%%%%%%%%%%%%%%%%%%%%%%%%%%%%%%%%%%%%%%%%%%%%%%%%%%%%%%%%%%%
\section{Usage}

First of all, the package \textsf{childdoc} is \emph{not} a standard
\LaTeXe{} |.sty| style file! Therefore it needs to be invoked in
a non-standard way.

%%%%%%%%%%%%%%%%%%%%%%%%%%%%%%%%%%%%%%%%%%%%%%%%%%%%%%%%%%%%%%%%%%%%%%%%%%%%%%%%
\subsection{Included Files}
\label{sec:include}

%%%%%%%%%%%%%%%%%%%%%%%%%%%%%%%%%%%%%%%%
\DescribeMacro{\childdocmain}
To use the package, add the commands
\begin{center}
\begin{tabular}{l}
|\input{childdoc.def}|\\
|\childdocmain{}|\\
\end{tabular}
\end{center}
at the very top of the main \LaTeX{} file,
in particular \emph{before} the |\documentclass| statement!
The argument of |\childdocmain| should be left empty
(but it must be present).

%%%%%%%%%%%%%%%%%%%%%%%%%%%%%%%%%%%%%%%%
\DescribeMacro{\childdocof}
Furthermore, add the commands
\begin{center}
\begin{tabular}{l}
|\input{childdoc.def}|\\
|\childdocof{|\textit{main}|}|\\
\end{tabular}
\end{center}
at the top of every child file \textit{child}
which is included by |\include{|\textit{child}|}|
from within the main file
(or at least for those files to be compiled individually).
The argument \textit{main} must be the filename of the main file.

There are a couple of
considerations in setting up the main and child documents:

%%%%%%%%%%%%%%%%%%%%%%%%%%%%%%%%%%%%%%%%
\paragraph{Restrictions.}

Please note the following restrictions:
\begin{itemize}
\item
|\childdocmain| must be called with one argument \textit{main}
to ensure compatibility with earlier version of the package.
It must either be empty (|\childdocmain{}|)
or precisely match the filename of the main file in which it is specified.
See \secref{sec:detection} for further information.
\item
The filename \textit{main} must be specified without the |.tex| extension.
\item
The filename \textit{main} is case sensitive
(even in case-insensitive file systems)
due to internal string comparison.
\item
The argument \textit{main} should be fully expanded, it cannot be a macro.
\item
Subdirectories and special characters should be avoided in filenames.
\item
The command |\childdocmain{|\textit{main}|}| must be followed by a whitespace.
It should not be followed immediately by another command
or by a comment mark `|%|'.
This is because the \TeX{} parser reads the token immediately following
the argument of |\childdocmain| and puts it
at the beginning of every child section;
however, a white\-space is ignored.
\end{itemize}

%%%%%%%%%%%%%%%%%%%%%%%%%%%%%%%%%%%%%%%%
\paragraph{Content of Main File.}

It is advisable to place all content in the child files included by |\include|.
Any output contained in the main file will appear in all child documents
unless suppressed manually;
it cannot be suppressed automatically by the |\includeonly| directive
and thus should normally be avoided.
A method to include some content in the main file
by means of conditional processing is described in \secref{sec:conditional}.

%%%%%%%%%%%%%%%%%%%%%%%%%%%%%%%%%%%%%%%%
\paragraph{Page Numbering.}

When only a part of the document is compiled,
the appropriate numbering of pages
(as well as other status parameters)
is determined from the |.aux| files.
The latter contain information from previous passes.
However this information needs to propagate through
all intermediate child documents.
Therefore the page numbering in child documents may well
be inconsistent until the complete document is compiled at least once.

A useful (if unconventional) way to always ensure a consistent
page numbering is to restart the numbering in each child document
and denote the pages by `\textit{child}|.|\textit{page}'
where \textit{child} represents the chapter/section number of the child file.
This can be achieved by the command
|\numberwithin{page}{|\textit{child}|}|
of the \textsf{amsmath} package
where \textit{child} can be |chapter| or |section|
depending on the chosen structuring.
Alternatively, one can modify the macro |\thepage| appropriately
and reset the counter |page| at the start of each child file.

%%%%%%%%%%%%%%%%%%%%%%%%%%%%%%%%%%%%%%%%%%%%%%%%%%%%%%%%%%%%%%%%%%%%%%%%%%%%%%%%
\subsection{Conditional Processing}
\label{sec:conditional}

The package provides a mechanism to compile different versions
of a document. To customise the versions further some conditional processing
can come in handy to distinguish which version is being compiled.
The package provides two macros to describe the compilation context:

%%%%%%%%%%%%%%%%%%%%%%%%%%%%%%%%%%%%%%%%
\DescribeMacro{\ifchilddoc}
The conditional |\ifchilddoc| distinguishes between the compilation of
child documents and the main document:
%
\begin{center}
|\ifchilddoc |\textit{child-code}| |[|\||else |\textit{main-code}]| \||fi|
\end{center}

%%%%%%%%%%%%%%%%%%%%%%%%%%%%%%%%%%%%%%%%
\DescribeMacro{\childdocname}
\DescribeMacro{\childdocjob}
The macro |\childdocname| contains the filename (without extension)
of the main or child file being processed.
Note that |\childdocjob| will always contain the name of the main file.

%%%%%%%%%%%%%%%%%%%%%%%%%%%%%%%%%%%%%%%%
\paragraph{Title Page.}

Conditional processing can be used to include a title or banner page
in the main document when proper precautions are taken.
Importantly, the code in the main file should ensure that the page counter
(as well as other status parameters which are stored in the |.aux| files)
takes the same value after the conditional processing.
Otherwise the page numbers may take divergent values
depending on which part is compiled.

For example, a title page could be declared by:
%
\begin{center}
\begin{tabular}{l}
|\ifchilddoc\||else|\\
|\addtocounter{page}{-1}|\\
\textit{code for title page}\\
|\newpage|\\
|\||fi|
\end{tabular}
\end{center}
%
A banner page for the child documents can be generated by:
%
\begin{center}
\begin{tabular}{l}
|\ifchilddoc|\\
|\addtocounter{page}{-1}|\\
\textit{code for banner page}\\
|\newpage|\\
|\||fi|
\end{tabular}
\end{center}
%
Here one could write a message such as:
\begin{center}
|This is the part \childdocname{} of \childdocjob{}.|
\end{center}

%%%%%%%%%%%%%%%%%%%%%%%%%%%%%%%%%%%%%%%%%%%%%%%%%%%%%%%%%%%%%%%%%%%%%%%%%%%%%%%%
\subsection{Flags}
\label{sec:flags}

The package makes it easy to generate different versions
of the main or child documents.
To this end compilation flags can be defined
and assigned different default values.
They will be particularly useful in conjunction
with the forwarding mechanism described in \secref{sec:forward}.

For example, it may be useful to have a flag |\version|
which can be set to |draft| or |final|.
The document source will contain some conditional code
depending on the value of |\version|.
Suppose further, the flag should default to |final| for the main file
and to |draft| for child files
which is a natural assignment for editing the document.
This is achieved by placing the following code
in the preamble of the main document
(below the |\childdocmain| directive):
%
\begin{center}
\begin{tabular}{l}
|\ifchilddoc|\\
|\providecommand{\version}{draft}|\\
|\||else|\\
|\providecommand{\version}{final}|\\
|\||fi|
\end{tabular}
\end{center}
%
The definition by |\providecommand| makes sure
that previous definitions are not overwritten.
Further statements |\providecommand{\version}{...}|
can thus be added before the above code to override it.

For the main file, one might add a line
(between |\childdocmain| and the above block)
%
\begin{center}
|%\ifchilddoc\||else\providecommand{\version}{draft}\||fi|
\end{center}
%
which can be uncommented to produce a draft version.
Likewise one can add a line to the very top of a child file
(above the |\childdocof{|\textit{main}|}| directive)
%
\begin{center}
|%\providecommand{\version}{final}|
\end{center}
%
which can be uncommented to produce the final version of this child document.

%%%%%%%%%%%%%%%%%%%%%%%%%%%%%%%%%%%%%%%%%%%%%%%%%%%%%%%%%%%%%%%%%%%%%%%%%%%%%%%%
\subsection{Forwarding}
\label{sec:forward}

Different versions of the main or child documents
using compilation flags as described in \secref{sec:flags}
can be (permanently) stored in different files
for convenient compilation, viewing and distribution.
To this end, the package defines a command
to pass on compilation to a different file:

%%%%%%%%%%%%%%%%%%%%%%%%%%%%%%%%%%%%%%%%
\DescribeMacro{\childdocforward}
The command |\childdocforward| redirects processing to
another source file:
%
\begin{center}
\begin{tabular}{l}
|\input{childdoc.def}|\\
|\childdocforward[|\textit{main}|]{|\textit{dest}|}|\\
\end{tabular}
\end{center}
%
The argument \textit{dest} is the destination file
(without extension).
It should be the main file or one of the child files.
Note that further \textsf{childdoc} directives
such as |\childdocof| and |\childdocforward|
in the indicated file will be processed in this form.
The optional argument \textit{main}
passes on directly to the main file \textit{main}
while pretending to compile the child \textit{dest}.
This form behaves as if \textit{dest}
issues |\childdocof{|\textit{main}|}| right away,
and no further \textsf{childdoc} directives will be processed.

%%%%%%%%%%%%%%%%%%%%%%%%%%%%%%%%%%%%%%%%
\DescribeMacro{\...prefix}
In the alternative form |\childdocforwardprefix|,
%
\begin{center}
\begin{tabular}{l}
|\input{childdoc.def}|\\
|\childdocforwardprefix[|\textit{main}|]{|\textit{prefix}|}{|\textit{dest}|}|
\end{tabular}
\end{center}
%
the destination file is determined by a pattern
depending on the current file:
To make this work, the current file must be called
`{\textit{prefix}\hspace{0.2em}\textit{suffix}}'
with \textit{prefix} matching precisely the argument.
Processing is then passed on to the file
`{\textit{dest}\hspace{0.2em}\textit{suffix}}'.
Surely, the same effect is achieved by
directly specifying the
argument `{\textit{dest}\hspace{0.2em}\textit{suffix}}'
in the first form.
However, that requires to set up a different file
for each child. With the alternative form of the command
all these files can have exactly the same content
which simplifies setting them up and maintaining them.

For example, the following file |draft.tex|
with a compilation flag |\version| as described in \secref{sec:flags}
compiles the main document as a draft:
%
\begin{center}
\begin{tabular}{l}
|\def\version{draft}|\\
|\input{childdoc.def}|\\
|\childdocforward{|\textit{main}|}|
\end{tabular}
\end{center}
%
Likewise, the following files |final|\textit{nn}|.tex|
compile the final version of the child document
|child|\textit{nn}|.tex|:
%
\begin{center}
\begin{tabular}{l}
|\def\version{final}|\\
|\input{childdoc.def}|\\
|\childdocforwardprefix{final}{child}|
\end{tabular}
\end{center}
%

Note that when several versions of a main file and/or of each child file
are to be generated, it may be convenient to set up a |Makefile| or
shell script to automatise the process.

%%%%%%%%%%%%%%%%%%%%%%%%%%%%%%%%%%%%%%%%%%%%%%%%%%%%%%%%%%%%%%%%%%%%%%%%%%%%%%%%
\subsection{Command Line Processing}
\label{sec:commandline}

The effect of redirection files can also be achieved by invoking
the \LaTeX{} compiler with a more elaborate command line.
Most conveniently this should be done as part
of a shell script or a |Makefile|.

When using \textsf{childdoc} in the main file, the following
command lines effectively perform a redirection
(note that depending on the shell being used,
backslashes may have to be doubled: `|\|' $\to$ `|\\|'):
%
\begin{center}
|... -jobname "|\textit{target}|" |\\|"|[\textit{flags}]%
|\input{childdoc.def}\childdocforward[|\textit{main}|]{|\textit{dest}|}"|
\end{center}
%
Here \textit{target} is the name of the output file,
\textit{main} is the name of the main file
and \textit{dest} is the name of the main or child file to be processed
(all filenames without extensions).
The optional argument \textit{main} can be omitted
if \textit{main} matches \textit{dest}.
Optionally, compilation \textit{flags} can be defined via |\def| commands.
This command line makes the \TeX{} engine believe
it is compiling the file \textit{target}
whose content is specified as the latter parameter.
The provided code then forwards the processing to
\textit{main} or \textit{dest} as described in \secref{sec:forward}.

%%%%%%%%%%%%%%%%%%%%%%%%%%%%%%%%%%%%%%%%%%%%%%%%%%%%%%%%%%%%%%%%%%%%%%%%%%%%%%%%
\subsection{Include by Input}
\label{sec:input}

Including child documents by |\include| has some restrictions by design.
Most notably, the content of a child document always occupies
its own set of pages; pages cannot be shared between child documents.
Usually, this behaviour makes perfect sense
because each child document contain an essential part of the document.
However, in some situations it may be desirable to compose
a document from a collection of parts
without having mandatory page breaks between then.
For this case, the package
provides a mechanism to include parts
by |\input| which can also be processed individually.
However, by construction this mechanism
requires manual handling of the content to be output.

%%%%%%%%%%%%%%%%%%%%%%%%%%%%%%%%%%%%%%%%
\DescribeMacro{\ifchilddocmanual}
The main file should be prepared as usual, see \secref{sec:include}.
However, the document body must make a distinction
between processing of an individual part and of the main document, e.g.:
%
\begin{center}
\begin{tabular}{l}
|\ifchilddocmanual|\\
|\input{\childdocname}|\\
|\||else|\\
\textit{document body with }|\input{|\textit{part}|}|\\
|\||fi|
\end{tabular}
\end{center}
%
The conditional |\ifchilddocmanual| is true whenever
a part to be included by |\input| is being compiled,
and the name of the part is stored in |\childdocname|.

%%%%%%%%%%%%%%%%%%%%%%%%%%%%%%%%%%%%%%%%
\DescribeMacro{\childdocby}
Each part to be included by |\input| should start with:
%
\begin{center}
\begin{tabular}{l}
|\input{childdoc.def}|\\
|\childdocby{|\textit{main}|}|\\
\end{tabular}
\end{center}
%
The directive |\childdocby| is similar to |\childdocof|
described in \secref{sec:include},
but the subsequent selection of content must be done manually.
To that end, both |\ifchilddoc| and |\ifchilddocmanual|
will be true upon processing of a part,
and the name of the part is stored in |\childdocname|.
Note that |\jobname| will be set to the filename of the current part
so that each part receives an individual |.aux| file
that does not interfere with the |.aux| file(s) of the main document.
This behaviour can be altered by the alternative form
|\childdocby[*]{|\textit{main}|}| (with a non-empty optional argument)
which uses the |.aux| file of the main document
by setting |\jobname| to \textit{main}.

%%%%%%%%%%%%%%%%%%%%%%%%%%%%%%%%%%%%%%%%%%%%%%%%%%%%%%%%%%%%%%%%%%%%%%%%%%%%%%%%
\subsection{Driver Development}
\label{sec:driver}

The \textsf{childdoc} mechanism can also be use for the development
of definition files such as \LaTeX{} styles or classes.
This case differs from the above setup with multiple parts
included by |\include| in that no |\includeonly| should be invoked.
This can be achieved by starting the include file
(before |\ProvidesPackage|) with:
%
\begin{center}
\begin{tabular}{l}
|\input{childdoc.def}|\\
|\childdocforward{|\textit{main}|}|\\
\end{tabular}
\end{center}
%
or alternatively with:
%
\begin{center}
\begin{tabular}{l}
|\input{childdoc.def}|\\
|\childdocby{|\textit{main}|}|\\
\end{tabular}
\end{center}
%
Both forms have slightly different effects as described above.
The main file is prepared as usual, see \secref{sec:include}.

%%%%%%%%%%%%%%%%%%%%%%%%%%%%%%%%%%%%%%%%%%%%%%%%%%%%%%%%%%%%%%%%%%%%%%%%%%%%%%%%
\subsection{Legacy Detection}
\label{sec:detection}

The directive |\childdocmain| in the main file can detect
whether the complete document or merely a child is to be compiled
even without using the directive |\childdocof|.
This method is deprecated because it is less robust
and there is no compelling reason to use it;
it is merely provided for backward compatibility
and it may be removed in future versions.

If the detection mechanism is to be used,
it is mandatory to correctly specify
the filename of the main file as the argument of |\childdocmain|:
%
\begin{center}
\begin{tabular}{l}
|\input{childdoc.def}|\\
|\childdocmain{|\textit{main}|}|\\
\end{tabular}
\end{center}
%
If |\jobname| does not match the argument \textit{main} of |\childdocmain|,
it is assumed that |\jobname| points to the child file to be compiled.
When using |\childdocmain| with the main file specified as argument,
it suffices to start a child file
with just |\input{|\textit{main}|}|
without loading of the package and using |\childdocof|.
If instead all processing is done
with the appropriate \textsf{childdoc} directives,
the argument of \textit{main} of |\childdocmain| can be empty.

An alternative version of the command line processing described
in \secref{sec:commandline} using the detection mechanism reads:
%
\begin{center}
|... -jobname "|\textit{target}|" "|[\textit{flags}]%
[|\def\jobname{|\textit{dest}|}|]|\input{|\textit{main}|}"|
\end{center}

%%%%%%%%%%%%%%%%%%%%%%%%%%%%%%%%%%%%%%%%%%%%%%%%%%%%%%%%%%%%%%%%%%%%%%%%%%%%%%%%
\subsection{Manual Code}
\label{sec:manual}

In case one cannot be certain whether the definitions file |childdoc.def|
is installed on the target \TeX{} distribution
and one prefers not to ship it,
it is conceivable to paste a few relevant commands into the sources.

To that end, drop all statements |\input{childdoc.def}|
and perform the replacements as outlined below.
Instead of |\childdocmain{|\textit{main}|}| add the following code
to the top of the main file:
%
\begin{center}
\begin{tabular}{l}
|\||ifdefined\childdocname\endinput\||fi\newif\ifchilddoc|\\
|\edef\childdocname{\scantokens\expandafter{\jobname\noexpand}}|\\
|\def\childdocmain{|\textit{main}|}\||ifx\childdocmain\childdocname\||else|\\
|\childdoctrue\includeonly{\childdocname}\let\jobname\childdocmain\||fi|\\
\end{tabular}
\end{center}
%
Instead of |\childdocof{|\textit{main}|}| just include the main file
at the top of each child file:
%
\begin{center}
|\input{|\textit{main}|}|
\end{center}
%
A simple redirection |\childdocforward{|\textit{dest}|}| is achieved by:
%
\begin{center}
|\def\jobname{|\textit{dest}|}\input{\jobname}|
\end{center}
%
The redirection with prefix
|\childdocforwardprefix[|\textit{prefix}|]{|\textit{dest}|}|
is accomplished by:
%
\begin{center}
\begin{tabular}{l}
|{\edef\jobname{\scantokens\expandafter{\jobname\noexpand}}|\\
|\def\redirectjob |\textit{prefix}|#1~~~{\gdef\jobname{|\textit{dest}|#1}}|\\
|\expandafter\redirectjob\jobname~~~}\input{\jobname}|
\end{tabular}
\end{center}

In an alternative approach,
child documents can be compiled by a specific command line
without additional code or specific definitions:
%
\begin{center}
|... -jobname "|\textit{target}|" "|[\textit{flags}]%
|\includeonly{|\textit{dest}|}\input{|\textit{main}|}"|
\end{center}
%

%%%%%%%%%%%%%%%%%%%%%%%%%%%%%%%%%%%%%%%%%%%%%%%%%%%%%%%%%%%%%%%%%%%%%%%%%%%%%%%%
%%%%%%%%%%%%%%%%%%%%%%%%%%%%%%%%%%%%%%%%%%%%%%%%%%%%%%%%%%%%%%%%%%%%%%%%%%%%%%%%
\section{Information}

%%%%%%%%%%%%%%%%%%%%%%%%%%%%%%%%%%%%%%%%%%%%%%%%%%%%%%%%%%%%%%%%%%%%%%%%%%%%%%%%
\subsection{Copyright}

Copyright \copyright{} 2017--2018 Niklas Beisert

This work may be distributed and/or modified under the
conditions of the \LaTeX{} Project Public License, either version 1.3
of this license or (at your option) any later version.
The latest version of this license is in
  \url{http://www.latex-project.org/lppl.txt}
and version 1.3 or later is part of all distributions of \LaTeX{}
version 2005/12/01 or later.

This work has the LPPL maintenance status `maintained'.

The Current Maintainer of this work is Niklas Beisert.

This work consists of the files |README.txt|, |childdoc.ins| and |childdoc.dtx|
as well as the derived files |childdoc.def|, |cdocsamp.tex|
with |cdocsch1.tex|, |cdocsch2.tex|, |cdocspt3.tex|, |cdocspt4.tex|,
|cdocsdrf.tex|, |cdocsfn1.tex|, |cdocsfn2.tex|
as well as |childdoc.pdf|.

%%%%%%%%%%%%%%%%%%%%%%%%%%%%%%%%%%%%%%%%%%%%%%%%%%%%%%%%%%%%%%%%%%%%%%%%%%%%%%%%
\subsection{Files and Installation}

The package consists of the files:
%
\begin{center}
\begin{tabular}{ll}
    |README.txt|   & readme file \\
    |childdoc.ins| & installation file \\
    |childdoc.dtx| & source file \\
    |childdoc.def| & definition file \\
    |cdocsamp.tex| & sample main file \\
    |cdocsch1.tex| & sample include file \\
    |cdocsch2.tex| & sample include file \\
    |cdocspt3.tex| & sample part file \\
    |cdocspt4.tex| & sample part file \\
    |cdocsdrf.tex| & sample redirection file \\
    |cdocsfn1.tex| & sample redirection file \\
    |cdocsfn2.tex| & sample redirection file \\
    |childdoc.pdf| & manual
\end{tabular}
\end{center}
%
The distribution consists of the files
|README.txt|, |childdoc.ins| and |childdoc.dtx|.
%
\begin{itemize}
\item
Run (pdf)\LaTeX{} on |childdoc.dtx|
to compile the manual |childdoc.pdf| (this file).
\item
Run \LaTeX{} on |childdoc.ins| to create the definitions file |childdoc.def|
and the sample |cdocsamp.tex| with include files
|cdocsch1.tex|, |cdocsch2.tex|, |cdocspt3.tex|, |cdocspt4.tex|,
|cdocsdrf.tex|, |cdocsfn1.tex|, |cdocsfn2.tex|.
Then copy the file |childdoc.def| to an appropriate directory of your \LaTeX{}
distribution, e.g.\ \textit{texmf-root}|/tex/latex/childdoc|.
\end{itemize}

%%%%%%%%%%%%%%%%%%%%%%%%%%%%%%%%%%%%%%%%%%%%%%%%%%%%%%%%%%%%%%%%%%%%%%%%%%%%%%%%
\subsection{Related CTAN Packages}

There are several other packages which offer a similar functionality:
%
\begin{itemize}
\item
The packages
\href{http://ctan.org/pkg/docmute}{\textsf{docmute}},
\href{http://ctan.org/pkg/includex}{\textsf{includex}} and
\href{http://ctan.org/pkg/standalone}{\textsf{standalone}}
provide commands to include only the document body of
a child file thus allowing both files to be compiled individually.
\item
The packages \href{http://ctan.org/pkg/subdocs}{\textsf{subdocs}}
and \href{http://ctan.org/pkg/subfiles}{\textsf{subfiles}}
provide structures in which the main and child documents can be
encapsulated and allowing them to be compiled individually.
The inclusion mechanism is different from the conventional |\include|.
\item
The package \href{http://ctan.org/pkg/combine}{\textsf{combine}}
is an elaborate solution to combine several documents into one.
\end{itemize}
%
See also the CTAN topic \href{http://ctan.org/topic/subdocs}{\textsf{subdocs}}
for further related packages.
The present package differs from the above solutions in that
a document structure constructed with the conventional |\include| mechanism
just needs two extra commands at the top of every file
such that all constituent files can be compiled individually.

%%%%%%%%%%%%%%%%%%%%%%%%%%%%%%%%%%%%%%%%%%%%%%%%%%%%%%%%%%%%%%%%%%%%%%%%%%%%%%%%
%\subsection{Feature Suggestions}
%
%The following is a list of features which may be useful for future
%versions of this package:
%%
%\begin{itemize}
%\item
%\ldots
%\end{itemize}

%%%%%%%%%%%%%%%%%%%%%%%%%%%%%%%%%%%%%%%%%%%%%%%%%%%%%%%%%%%%%%%%%%%%%%%%%%%%%%%%
\subsection{Revision History}

%%%%%%%%%%%%%%%%%%%%%%%%%%%%%%%%%%%%%%%%
\paragraph{v2.0:} 2018/12/30

\begin{itemize}
\item
immediate forward processing
\item
added |\childdocby| mechanism
\item
manual restructured
\end{itemize}

%%%%%%%%%%%%%%%%%%%%%%%%%%%%%%%%%%%%%%%%
\paragraph{v1.6:} 2018/01/17

\begin{itemize}
\item
application for development of include files
\item
corrections to manual
\end{itemize}

%%%%%%%%%%%%%%%%%%%%%%%%%%%%%%%%%%%%%%%%
\paragraph{v1.5:} 2017/05/21

\begin{itemize}
\item
more complete structuring introduced
\item
|\childdocof| introduced
\item
|\childdoc| renamed to |\childdocmain|
\item
|\childredirect| renamed to |\childdocforward| and |\childdocforwardprefix|
and functionality expanded
\end{itemize}

%%%%%%%%%%%%%%%%%%%%%%%%%%%%%%%%%%%%%%%%
\paragraph{v1.0:} 2017/04/27

\begin{itemize}
\item
manual and install package
\item
first version published on CTAN
\end{itemize}

%%%%%%%%%%%%%%%%%%%%%%%%%%%%%%%%%%%%%%%%
\paragraph{v0.6:} 2017/04/26

\begin{itemize}
\item
redirection mechanism added
\end{itemize}

%%%%%%%%%%%%%%%%%%%%%%%%%%%%%%%%%%%%%%%%
\paragraph{v0.5:} 2017/04/26

\begin{itemize}
\item
functionality in definition file
\end{itemize}


%%%%%%%%%%%%%%%%%%%%%%%%%%%%%%%%%%%%%%%%%%%%%%%%%%%%%%%%%%%%%%%%%%%%%%%%%%%%%%%%
%%%%%%%%%%%%%%%%%%%%%%%%%%%%%%%%%%%%%%%%%%%%%%%%%%%%%%%%%%%%%%%%%%%%%%%%%%%%%%%%
%%%%%%%%%%%%%%%%%%%%%%%%%%%%%%%%%%%%%%%%%%%%%%%%%%%%%%%%%%%%%%%%%%%%%%%%%%%%%%%%
\appendix

\settowidth\MacroIndent{\rmfamily\scriptsize 000\ }

 \DocInput{childdoc.dtx}

\end{document}
%</driver>
% \fi
%
% %%%%%%%%%%%%%%%%%%%%%%%%%%%%%%%%%%%%%%%%%%%%%%%%%%%%%%%%%%%%%%%%%%%%%%%%%%%%%%
% %%%%%%%%%%%%%%%%%%%%%%%%%%%%%%%%%%%%%%%%%%%%%%%%%%%%%%%%%%%%%%%%%%%%%%%%%%%%%%
% \section{Sample}
%\iffalse
%<*samplemain>
%\fi
%
% The following presents a sample document
% with two chapters, two parts, a title page,
% a compile flag as well as three forwarding files to set the flag.
% It consists of eight |.tex| files:
% \begin{center}
% \begin{tabular}{ll}
% |cdocsamp.tex|&main file\\
% |cdocsch1.tex|&include file for chapter 1\\
% |cdocsch2.tex|&include file for chapter 2\\
% |cdocspt3.tex|&include file for part 3\\
% |cdocspt4.tex|&include file for part 4\\
% |cdocsdrf.tex|&forwarding file for main file in draft mode\\
% |cdocsfi1.tex|&forwarding file for final version of chapter 1\\
% |cdocsfi2.tex|&forwarding file for final version of chapter 2\\
% \end{tabular}
% \end{center}
% Each of the eight files can be compiled directly by the \LaTeX{} compiler.
%
% %%%%%%%%%%%%%%%%%%%%%%%%%%%%%%%%%%%%%%
% \paragraph{Main File.}
%
% The main file is called |cdocsamp.tex|.
%
% Load the \textsf{childdoc} definitions and
% declare the filename for the main document:
%    \begin{macrocode}
\input{childdoc.def}
\childdocmain{}
%    \end{macrocode}

% Optional override for |\version| flag:
%    \begin{macrocode}
%%\ifchilddoc\else\providecommand{\version}{draft}\fi
%    \end{macrocode}

% Define the default values for the |\version| flag
% (|final| for the main file and |draft| for childs):
%    \begin{macrocode}
\ifchilddoc
\providecommand{\version}{draft}
\else
\providecommand{\version}{final}
\fi
%    \end{macrocode}

% Load the standard document class:
%    \begin{macrocode}
\documentclass[12pt]{article}
%    \end{macrocode}

% Start the document body:
%    \begin{macrocode}
\begin{document}
%    \end{macrocode}

% Declare a title page.
% Print title, part of document being processed and version flag:
%    \begin{macrocode}
\addtocounter{page}{-1}
\begin{center}
{\LARGE\bfseries{}childdoc example\par}
\vspace{1cm}
\ifchilddoc
\ifchilddocmanual part\else chapter\fi:
`\childdocname' of `\childdocjob'\par
\else
main document: `\childdocjob'\par
\fi
version: \version\par
\end{center}
\newpage
%    \end{macrocode}

% Manually include selected file,
% otherwise process as usual:
%    \begin{macrocode}
\ifchilddocmanual
\section*{part `\childdocname'}
\input{\childdocname}
\else
%    \end{macrocode}

% Include the two chapters:
%    \begin{macrocode}
\include{cdocsch1}
\include{cdocsch2}
%    \end{macrocode}

% Include the two parts unless only chapters should be displayed:
%    \begin{macrocode}
\ifchilddoc\else
\section{part three}
\input{cdocspt3}
\section{part four}
\input{cdocspt4}
\fi
%    \end{macrocode}

% Process as usual until here:
%    \begin{macrocode}
\fi
%    \end{macrocode}

% End of document body:
%    \begin{macrocode}
\end{document}
%    \end{macrocode}
%\iffalse
%</samplemain>
%\fi
%
% %%%%%%%%%%%%%%%%%%%%%%%%%%%%%%%%%%%%%%
% \paragraph{Chapter Include Files.}
%
% The include files are called |cdocsch1.tex| and |cdocsch2.tex|.
%
%\iffalse
%<*samplechap1|samplechap2>
%\fi

% Optional override for |\version| flag:
%    \begin{macrocode}
%%\providecommand{\version}{final}
%    \end{macrocode}

% Include the main document:
%    \begin{macrocode}
\input{childdoc.def}
\childdocof{cdocsamp}
%    \end{macrocode}

%\iffalse
%</samplechap1|samplechap2>
%\fi
%
%\iffalse
%<*samplechap1>
%\fi
% Some text for chapter 1:
%    \begin{macrocode}
\section{one}
some text in chapter one
%    \end{macrocode}

%\iffalse
%</samplechap1>
%\fi
% Some text for chapter 2:
%\iffalse
%<*samplechap2>
%\fi
%    \begin{macrocode}
\section{two}
more text in chapter two
%    \end{macrocode}

%\iffalse
%</samplechap2>
%\fi
%
% %%%%%%%%%%%%%%%%%%%%%%%%%%%%%%%%%%%%%%
% \paragraph{Part Include Files.}
%
% The include files are called |cdocspt3.tex| and |cdocspt4.tex|.
%
%\iffalse
%<*samplepart3|samplepart4>
%\fi

% Optional override for |\version| flag:
%    \begin{macrocode}
%%\providecommand{\version}{final}
%    \end{macrocode}

% Include the main document:
%    \begin{macrocode}
\input{childdoc.def}
\childdocby{cdocsamp}
%    \end{macrocode}

%\iffalse
%</samplepart3|samplepart4>
%\fi
%
%\iffalse
%<*samplepart3>
%\fi
% Some text for part 3:
%    \begin{macrocode}
some text in part three
%    \end{macrocode}

%\iffalse
%</samplepart3>
%\fi
% Some text for part 4:
%\iffalse
%<*samplepart4>
%\fi
%    \begin{macrocode}
more text in part four
%    \end{macrocode}

%\iffalse
%</samplepart4>
%\fi
%
% %%%%%%%%%%%%%%%%%%%%%%%%%%%%%%%%%%%%%%
% \paragraph{Forwarding for a Complete Draft.}
%
% The following forwarding file |cdocsdrf.tex|
% compiles the main document in draft mode:
%\iffalse
%<*sampledraft>
%\fi
%    \begin{macrocode}
\def\version{draft}
\input{childdoc.def}
\childdocforward{cdocsamp}
%    \end{macrocode}

%\iffalse
%</sampledraft>
%\fi
%
% %%%%%%%%%%%%%%%%%%%%%%%%%%%%%%%%%%%%%%
% \paragraph{Forwarding for Final Version of the Chapters.}
%
% The following forwarding files |cdocsfn1.tex| and |cdocsfn2.tex|
% (with identical content)
% compile the final versions of the child documents
% |cdocsch1.tex| and |cdocsch2.tex|, respectively:
%\iffalse
%<*samplefinal>
%\fi
%    \begin{macrocode}
\def\version{final}
\input{childdoc.def}
\childdocforwardprefix[cdocsamp]{cdocsfn}{cdocsch}
%    \end{macrocode}

%\iffalse
%</samplefinal>
%\fi
%
% %%%%%%%%%%%%%%%%%%%%%%%%%%%%%%%%%%%%%%
% \paragraph{Command Line Processing.}
%
% The following three command lines generate the output files
% |cdocscld|, |cdocscl1| and |cdocscl2|
% which should be identical to
% |cdocsdrf|, |cdocsch1| and |cdocsfn2|, respectively:
% \begin{center}
% \begin{tabular}{l}
% |latex -jobname cdocscld \|\\
% |  "\def\version{draft}\input{childdoc.def}\childdocforward{cdocsamp}"|\\
% |latex -jobname cdocscl1 \|\\
% |  "\input{childdoc.def}\childdocforward[cdocsamp]{cdocsch1}"|\\
% |latex -jobname cdocscl2 \|\\
% |  "\def\version{final}\input{childdoc.def}\childdocforward{cdocsch2}"|
% \end{tabular}
% \end{center}
% Note that the trailing backslash on each first line
% merely continues the input to the second line
% (for convenient cut ant paste).
% Furthermore, the command |latex| can be replaced by any
% of its alternative versions such as |pdflatex|.
%
% %%%%%%%%%%%%%%%%%%%%%%%%%%%%%%%%%%%%%%%%%%%%%%%%%%%%%%%%%%%%%%%%%%%%%%%%%%%%%%
% %%%%%%%%%%%%%%%%%%%%%%%%%%%%%%%%%%%%%%%%%%%%%%%%%%%%%%%%%%%%%%%%%%%%%%%%%%%%%%
% \section{Implementation}
%\iffalse
%<*package>
%\fi
%
% This section describes the definitions file |childdoc.def|.

% The definitions cannot be loaded using |\usepackage| or |\RequirePackage|
% which has a mechanism to prevent loading a style file more than once.
% When loading the definitions by means of |\input|
% multiple instances have to be prevented manually:
%\iffalse
%This code needs to be before the `\ProvidesFile' directive
%which is defined at the beginning of this file.
%Therefore it is also placed there and commented out here.
%</package>
%<*discard>
%\fi
%    \begin{macrocode}
\ifdefined\childdocmain\endinput\fi
%    \end{macrocode}
%\iffalse
%</discard>
%<*package>
%\fi
%
% \macro{\ifchilddoc}
% \macro{\ifchilddocmanual}
% The conditional |\ifchilddoc| tells whether a
% child (true) or main (false) document is being compiled.
% The conditional |\ifchilddocmanual| tells whether
% the |\includeonly| mechanism is used (false) or
% the selection of child files must be performed manually (true).
% The definitions initialise to false:
%    \begin{macrocode}
\newif\ifchilddoc
\newif\ifchilddocmanual
%    \end{macrocode}

% \macro{\childdocname}
% \macro{\childdocjob}
% The macro |\childdocname| stores the name of the main document
% to be compiled. The macro |\childdocjob| stores the name of
% the document on which the \LaTeX{} compiler was originally invoked.
% The content of |\jobname| cannot be compared
% to filenames specified in the source due to different catcodes.
% The following code rescans |\jobname|, stores the result
% in |\childdocname| and saves a copy in |\childdocjob|:
%    \begin{macrocode}
\edef\childdocname{\scantokens\expandafter{\jobname\noexpand}}
\let\childdocjob\childdocname
%    \end{macrocode}

% \macro{\childdocdisable}
% The macro |\childdocdisable| prevents the main file
% from being processed more than once.
% At this stage, the main document command |\childdocmain|
% is assumed to be called once again where it should do nothing.
% Any subsequent call to it should prevent
% a secondary processing of the main document
% It overwrites the forwarding commands
% |\childdocof| and |\childdocforward|
% with empty macros to prevent further inclusions of the main document:
%    \begin{macrocode}
\newcommand{\childdocdisable}
{
  \renewcommand{\childdocmain}[1]{\renewcommand{\childdocmain}[1]{\endinput}}
  \renewcommand{\childdocof}[1]{}
  \renewcommand{\childdocby}[2][]{}
  \renewcommand{\childdocforward}[2][]{}
  \renewcommand{\childdocdisable}{}
}
%    \end{macrocode}

% \macro{\childdocmain}
% The macro |\childdocmain| is to be called at the top of the main file
% with nothing or the main filename (without extension) as argument.
% First, it breaks loops.
% If the argument is not empty and does not match |\childdocname|
% (which is set by the first inclusion of |childdoc.def|),
% |\ifchilddoc| is set to true, |\includeonly| is applied to the child file
% and |\jobname| is set to the main file
% (for proper handling of |.aux| files):
%    \begin{macrocode}
\newcommand{\childdocmain}[1]
{
  \childdocdisable\childdocmain{}
  \if?#1?\else
    \begingroup
      \def\childdoctmp{#1}
      \ifx\childdoctmp\childdocname
        \def\childdoctmp{}
      \else
        \def\childdoctmp
        {
          \childdoctrue
          \includeonly{\childdocname}
          \def\childdocjob{#1}
          \def\jobname{#1}
        }
      \fi
      \expandafter
    \endgroup
    \childdoctmp
  \fi
}
%    \end{macrocode}

% \macro{\childdocof}
% The command |\childdocof| redirects
% compilation to the main file |#1|.
%    \begin{macrocode}
\newcommand{\childdocof}[1]
{
  \childdocdisable
  \childdoctrue
  \includeonly{\childdocname}
  \def\jobname{#1}
  \def\childdocjob{#1}
  \input{#1}
}
%    \end{macrocode}

% \macro{\childdocby}
% The command |\childdocby| ....
%    \begin{macrocode}
\newcommand{\childdocby}[2][]
{
  \childdocdisable
  \childdoctrue
  \childdocmanualtrue
  \if?#1?\else
    \def\jobname{#2}
  \fi
  \def\childdocjob{#2}
  \input{#2}
  \endinput
}
%    \end{macrocode}

% \macro{\childdocforward}
% The command |\childdocforward| redirects
% compilation to the main file or
% (if the optional argument is given) a child file.
% Parameters are set as if the main file
% or a child file starting with |\childdocof| was compiled.
% Then compilation is handed over to the main file:
%    \begin{macrocode}
\newcommand{\childdocforward}[2][]
{
  \begingroup
    \if?#1?
      \def\childdoctmp
      {
        \def\childdocname{#2}
        \def\childdocjob{#2}
        \def\jobname{#2}
        \input{#2}
        \endinput
      }
    \else
      \def\childdoctmp
      {
        \childdocdisable
        \def\childdocname{#2}
        \childdoctrue
        \includeonly{#2}
        \def\childdocjob{#1}
        \def\jobname{#1}
        \input{#1}
        \endinput
      }
    \fi
    \expandafter
  \endgroup
  \childdoctmp
}
%    \end{macrocode}

% \macro{\childdocforwardprefix}
% The command |\childdocforwardprefix| redirects
% compilation to the main or a child file by means of a pattern.
% The prefix |#1| in the current filename is replaced by |#2|
% and the suffix of the current filename is kept
% (it is assumed that the filename does not contain the substring `|~~~|'
% which is used as a delimiter).
% Compilation is handed over to the new file by |\childdocforward|:
%    \begin{macrocode}
\newcommand{\childdocforwardprefix}[3][]
{
  \begingroup
    \def\childdocextract #2##1~~~{\def\childdoctmp{\childdocforward[#1]{#3##1}}}
    \expandafter\childdocextract\childdocname~~~
    \expandafter
  \endgroup
  \childdoctmp
}
%    \end{macrocode}

% \macro{\childdoc}
% The deprecated macro |\childdoc| is a legacy version of |\childdocmain|:
%    \begin{macrocode}
\newcommand{\childdoc}{\childdocmain}
%    \end{macrocode}

% \macro{\childdocredirect}
% The deprecated macro |\childdocredirect| is a legacy version
% of |\childdocforward| and |\childdocforwardprefix|:
%    \begin{macrocode}
\newcommand{\childdocredirect}[2][]
{
  \begingroup
    \if?#1?
      \def\childdoctmp{\childdocforward{#2}}
    \else
      \def\childdoctmp{\childdocforwardprefix{#1}{#2}}
    \fi
    \expandafter
  \endgroup
  \childdoctmp
}
%    \end{macrocode}

%\iffalse
%</package>
%\fi
%
\endinput
\childdocforward[cdocsamp]{cdocsch1}"|\\
% |latex -jobname cdocscl2 \|\\
% |  "\def\version{final}% \iffalse
%
% childdoc.dtx Copyright (C) 2017-2018 Niklas Beisert
%
% This work may be distributed and/or modified under the
% conditions of the LaTeX Project Public License, either version 1.3
% of this license or (at your option) any later version.
% The latest version of this license is in
%   http://www.latex-project.org/lppl.txt
% and version 1.3 or later is part of all distributions of LaTeX
% version 2005/12/01 or later.
%
% This work has the LPPL maintenance status `maintained'.
%
% The Current Maintainer of this work is Niklas Beisert.
%
% This work consists of the files childdoc.dtx and childdoc.ins
% and the derived files childdoc.def and cdocsamp.tex with
% cdocsch1.tex, cdocsch2.tex, cdocsdrf.tex, cdocsfn1.tex, cdocsfn2.tex.
%
%<package>\ifdefined\childdocmain\endinput\fi
%<package>\ProvidesFile{childdoc.def}[2018/12/30 v2.0 child document driver]
%<samplemain>\ProvidesFile{cdocsamp.tex}[2018/12/30 v2.0 sample for childdoc]
%<*driver>
%\ProvidesFile{childdoc.drv}[2018/12/30 v2.0 childdoc reference manual file]
\PassOptionsToClass{10pt,a4paper}{article}
\documentclass{ltxdoc}

\usepackage[margin=35mm]{geometry}
\usepackage{hyperref}
\usepackage{hyperxmp}
\usepackage[usenames]{color}

\hypersetup{colorlinks=true}
\hypersetup{pdfstartview=FitH}
\hypersetup{pdfpagemode=UseNone}
\hypersetup{pdfsource={}}
\hypersetup{pdflang={en-UK}}
\hypersetup{pdfcopyright={Copyright 2017-2018 Niklas Beisert.
  This work may be distributed and/or modified under the
  conditions of the LaTeX Project Public License, either version 1.3
  of this license or (at your option) any later version.}}
\hypersetup{pdflicenseurl={http://www.latex-project.org/lppl.txt}}
\hypersetup{pdfcontactaddress={ETH Zurich, ITP, HIT K,
  Wolfgang-Pauli-Strasse 27}}
\hypersetup{pdfcontactpostcode={8093}}
\hypersetup{pdfcontactcity={Zurich}}
\hypersetup{pdfcontactcountry={Switzerland}}
\hypersetup{pdfcontactemail={nbeisert@itp.phys.ethz.ch}}
\hypersetup{pdfcontacturl={http://people.phys.ethz.ch/\xmptilde nbeisert/}}

\newcommand{\secref}[1]{\hyperref[#1]{section \ref*{#1}}}

\parskip1ex
\parindent0pt
\let\olditemize\itemize
\def\itemize{\olditemize\parskip0pt}

\begin{document}

\title{The \textsf{childdoc} Package}
\hypersetup{pdftitle={The childdoc Package}}
\author{Niklas Beisert\\[2ex]
  Institut f\"ur Theoretische Physik\\
  Eidgen\"ossische Technische Hochschule Z\"urich\\
  Wolfgang-Pauli-Strasse 27, 8093 Z\"urich, Switzerland\\[1ex]
  \href{mailto:nbeisert@itp.phys.ethz.ch}
  {\texttt{nbeisert@itp.phys.ethz.ch}}}
\hypersetup{pdfauthor={Niklas Beisert}}
\hypersetup{pdfsubject={Manual for the LaTeX2e Package childdoc}}
\date{30 December 2018, \textsf{v2.0}}
\maketitle

\begin{abstract}\noindent
\textsf{childdoc} is a \LaTeXe{} package
that enables the direct compilation
of document sections included by |\include|
to individual files.
\end{abstract}

\begingroup
\parskip0ex
\tableofcontents
\endgroup

%%%%%%%%%%%%%%%%%%%%%%%%%%%%%%%%%%%%%%%%%%%%%%%%%%%%%%%%%%%%%%%%%%%%%%%%%%%%%%%%
%%%%%%%%%%%%%%%%%%%%%%%%%%%%%%%%%%%%%%%%%%%%%%%%%%%%%%%%%%%%%%%%%%%%%%%%%%%%%%%%
\section{Introduction}

\LaTeX{} provides a mechanism to structure a large document (such as a book)
into a main file and several child files (containing the chapters)
using the |\include| command.
This mechanism is beneficial for documents
which span hundreds of pages in order to
make the source file(s) more manageable.
Moreover, compilation can be restricted to
selected child files by means of the |\includeonly| command.
The latter feature can be used to reduce the compilation time while editing
(this was significantly more useful in the earlier days of \LaTeX{})
or to generate a smaller document which is easier to navigate.
Another application of |\includeonly| is to generate
documents consisting of selected parts of the complete document.

However, there are a few drawbacks of the plain |\include| mechanism:
\begin{itemize}
\item
The child files cannot be compiled on their own,
they can only be compiled via the main file.
A naive editing environment
(such as a text editor with an option
to have the current file processed by \LaTeX)
may require one to switch to the main file before compiling;
attempting to compile the child file produces errors.
\item
The main file must be modified (each time)
to adjust the |\includeonly| command
to the present needs. This easily leaves the main file in a messy state.
\item
The generated document will always carry the filename
of the main document. This is inconvenient if
several child files are to be compiled and
to be kept for distribution.
\end{itemize}

The present package provides a simple interface
to make child files individually compilable by \LaTeX{}.
Compiling a child file then has the same effect as compiling
the main file with an |\includeonly| command
to select the appropriate child.
Moreover the generated document will carry the name of the child
rather than the main file.
This resolves all three above issues.

This feature is meant to make the editing of books,
thesis documents and lecture notes somewhat more convenient.
However, the package can also be used efficiently for
composing a series of documents (such as exercise sheets)
which are typically distributed individually.
It then assists the author in generating the individual documents
(potentially in different versions)
as well as a document containing the collected series.
Another application is in developing style files
or other kinds of included material
where compilation of the style file could redirect
to a sample or test file.

%%%%%%%%%%%%%%%%%%%%%%%%%%%%%%%%%%%%%%%%%%%%%%%%%%%%%%%%%%%%%%%%%%%%%%%%%%%%%%%%
%%%%%%%%%%%%%%%%%%%%%%%%%%%%%%%%%%%%%%%%%%%%%%%%%%%%%%%%%%%%%%%%%%%%%%%%%%%%%%%%
\section{Usage}

First of all, the package \textsf{childdoc} is \emph{not} a standard
\LaTeXe{} |.sty| style file! Therefore it needs to be invoked in
a non-standard way.

%%%%%%%%%%%%%%%%%%%%%%%%%%%%%%%%%%%%%%%%%%%%%%%%%%%%%%%%%%%%%%%%%%%%%%%%%%%%%%%%
\subsection{Included Files}
\label{sec:include}

%%%%%%%%%%%%%%%%%%%%%%%%%%%%%%%%%%%%%%%%
\DescribeMacro{\childdocmain}
To use the package, add the commands
\begin{center}
\begin{tabular}{l}
|\input{childdoc.def}|\\
|\childdocmain{}|\\
\end{tabular}
\end{center}
at the very top of the main \LaTeX{} file,
in particular \emph{before} the |\documentclass| statement!
The argument of |\childdocmain| should be left empty
(but it must be present).

%%%%%%%%%%%%%%%%%%%%%%%%%%%%%%%%%%%%%%%%
\DescribeMacro{\childdocof}
Furthermore, add the commands
\begin{center}
\begin{tabular}{l}
|\input{childdoc.def}|\\
|\childdocof{|\textit{main}|}|\\
\end{tabular}
\end{center}
at the top of every child file \textit{child}
which is included by |\include{|\textit{child}|}|
from within the main file
(or at least for those files to be compiled individually).
The argument \textit{main} must be the filename of the main file.

There are a couple of
considerations in setting up the main and child documents:

%%%%%%%%%%%%%%%%%%%%%%%%%%%%%%%%%%%%%%%%
\paragraph{Restrictions.}

Please note the following restrictions:
\begin{itemize}
\item
|\childdocmain| must be called with one argument \textit{main}
to ensure compatibility with earlier version of the package.
It must either be empty (|\childdocmain{}|)
or precisely match the filename of the main file in which it is specified.
See \secref{sec:detection} for further information.
\item
The filename \textit{main} must be specified without the |.tex| extension.
\item
The filename \textit{main} is case sensitive
(even in case-insensitive file systems)
due to internal string comparison.
\item
The argument \textit{main} should be fully expanded, it cannot be a macro.
\item
Subdirectories and special characters should be avoided in filenames.
\item
The command |\childdocmain{|\textit{main}|}| must be followed by a whitespace.
It should not be followed immediately by another command
or by a comment mark `|%|'.
This is because the \TeX{} parser reads the token immediately following
the argument of |\childdocmain| and puts it
at the beginning of every child section;
however, a white\-space is ignored.
\end{itemize}

%%%%%%%%%%%%%%%%%%%%%%%%%%%%%%%%%%%%%%%%
\paragraph{Content of Main File.}

It is advisable to place all content in the child files included by |\include|.
Any output contained in the main file will appear in all child documents
unless suppressed manually;
it cannot be suppressed automatically by the |\includeonly| directive
and thus should normally be avoided.
A method to include some content in the main file
by means of conditional processing is described in \secref{sec:conditional}.

%%%%%%%%%%%%%%%%%%%%%%%%%%%%%%%%%%%%%%%%
\paragraph{Page Numbering.}

When only a part of the document is compiled,
the appropriate numbering of pages
(as well as other status parameters)
is determined from the |.aux| files.
The latter contain information from previous passes.
However this information needs to propagate through
all intermediate child documents.
Therefore the page numbering in child documents may well
be inconsistent until the complete document is compiled at least once.

A useful (if unconventional) way to always ensure a consistent
page numbering is to restart the numbering in each child document
and denote the pages by `\textit{child}|.|\textit{page}'
where \textit{child} represents the chapter/section number of the child file.
This can be achieved by the command
|\numberwithin{page}{|\textit{child}|}|
of the \textsf{amsmath} package
where \textit{child} can be |chapter| or |section|
depending on the chosen structuring.
Alternatively, one can modify the macro |\thepage| appropriately
and reset the counter |page| at the start of each child file.

%%%%%%%%%%%%%%%%%%%%%%%%%%%%%%%%%%%%%%%%%%%%%%%%%%%%%%%%%%%%%%%%%%%%%%%%%%%%%%%%
\subsection{Conditional Processing}
\label{sec:conditional}

The package provides a mechanism to compile different versions
of a document. To customise the versions further some conditional processing
can come in handy to distinguish which version is being compiled.
The package provides two macros to describe the compilation context:

%%%%%%%%%%%%%%%%%%%%%%%%%%%%%%%%%%%%%%%%
\DescribeMacro{\ifchilddoc}
The conditional |\ifchilddoc| distinguishes between the compilation of
child documents and the main document:
%
\begin{center}
|\ifchilddoc |\textit{child-code}| |[|\||else |\textit{main-code}]| \||fi|
\end{center}

%%%%%%%%%%%%%%%%%%%%%%%%%%%%%%%%%%%%%%%%
\DescribeMacro{\childdocname}
\DescribeMacro{\childdocjob}
The macro |\childdocname| contains the filename (without extension)
of the main or child file being processed.
Note that |\childdocjob| will always contain the name of the main file.

%%%%%%%%%%%%%%%%%%%%%%%%%%%%%%%%%%%%%%%%
\paragraph{Title Page.}

Conditional processing can be used to include a title or banner page
in the main document when proper precautions are taken.
Importantly, the code in the main file should ensure that the page counter
(as well as other status parameters which are stored in the |.aux| files)
takes the same value after the conditional processing.
Otherwise the page numbers may take divergent values
depending on which part is compiled.

For example, a title page could be declared by:
%
\begin{center}
\begin{tabular}{l}
|\ifchilddoc\||else|\\
|\addtocounter{page}{-1}|\\
\textit{code for title page}\\
|\newpage|\\
|\||fi|
\end{tabular}
\end{center}
%
A banner page for the child documents can be generated by:
%
\begin{center}
\begin{tabular}{l}
|\ifchilddoc|\\
|\addtocounter{page}{-1}|\\
\textit{code for banner page}\\
|\newpage|\\
|\||fi|
\end{tabular}
\end{center}
%
Here one could write a message such as:
\begin{center}
|This is the part \childdocname{} of \childdocjob{}.|
\end{center}

%%%%%%%%%%%%%%%%%%%%%%%%%%%%%%%%%%%%%%%%%%%%%%%%%%%%%%%%%%%%%%%%%%%%%%%%%%%%%%%%
\subsection{Flags}
\label{sec:flags}

The package makes it easy to generate different versions
of the main or child documents.
To this end compilation flags can be defined
and assigned different default values.
They will be particularly useful in conjunction
with the forwarding mechanism described in \secref{sec:forward}.

For example, it may be useful to have a flag |\version|
which can be set to |draft| or |final|.
The document source will contain some conditional code
depending on the value of |\version|.
Suppose further, the flag should default to |final| for the main file
and to |draft| for child files
which is a natural assignment for editing the document.
This is achieved by placing the following code
in the preamble of the main document
(below the |\childdocmain| directive):
%
\begin{center}
\begin{tabular}{l}
|\ifchilddoc|\\
|\providecommand{\version}{draft}|\\
|\||else|\\
|\providecommand{\version}{final}|\\
|\||fi|
\end{tabular}
\end{center}
%
The definition by |\providecommand| makes sure
that previous definitions are not overwritten.
Further statements |\providecommand{\version}{...}|
can thus be added before the above code to override it.

For the main file, one might add a line
(between |\childdocmain| and the above block)
%
\begin{center}
|%\ifchilddoc\||else\providecommand{\version}{draft}\||fi|
\end{center}
%
which can be uncommented to produce a draft version.
Likewise one can add a line to the very top of a child file
(above the |\childdocof{|\textit{main}|}| directive)
%
\begin{center}
|%\providecommand{\version}{final}|
\end{center}
%
which can be uncommented to produce the final version of this child document.

%%%%%%%%%%%%%%%%%%%%%%%%%%%%%%%%%%%%%%%%%%%%%%%%%%%%%%%%%%%%%%%%%%%%%%%%%%%%%%%%
\subsection{Forwarding}
\label{sec:forward}

Different versions of the main or child documents
using compilation flags as described in \secref{sec:flags}
can be (permanently) stored in different files
for convenient compilation, viewing and distribution.
To this end, the package defines a command
to pass on compilation to a different file:

%%%%%%%%%%%%%%%%%%%%%%%%%%%%%%%%%%%%%%%%
\DescribeMacro{\childdocforward}
The command |\childdocforward| redirects processing to
another source file:
%
\begin{center}
\begin{tabular}{l}
|\input{childdoc.def}|\\
|\childdocforward[|\textit{main}|]{|\textit{dest}|}|\\
\end{tabular}
\end{center}
%
The argument \textit{dest} is the destination file
(without extension).
It should be the main file or one of the child files.
Note that further \textsf{childdoc} directives
such as |\childdocof| and |\childdocforward|
in the indicated file will be processed in this form.
The optional argument \textit{main}
passes on directly to the main file \textit{main}
while pretending to compile the child \textit{dest}.
This form behaves as if \textit{dest}
issues |\childdocof{|\textit{main}|}| right away,
and no further \textsf{childdoc} directives will be processed.

%%%%%%%%%%%%%%%%%%%%%%%%%%%%%%%%%%%%%%%%
\DescribeMacro{\...prefix}
In the alternative form |\childdocforwardprefix|,
%
\begin{center}
\begin{tabular}{l}
|\input{childdoc.def}|\\
|\childdocforwardprefix[|\textit{main}|]{|\textit{prefix}|}{|\textit{dest}|}|
\end{tabular}
\end{center}
%
the destination file is determined by a pattern
depending on the current file:
To make this work, the current file must be called
`{\textit{prefix}\hspace{0.2em}\textit{suffix}}'
with \textit{prefix} matching precisely the argument.
Processing is then passed on to the file
`{\textit{dest}\hspace{0.2em}\textit{suffix}}'.
Surely, the same effect is achieved by
directly specifying the
argument `{\textit{dest}\hspace{0.2em}\textit{suffix}}'
in the first form.
However, that requires to set up a different file
for each child. With the alternative form of the command
all these files can have exactly the same content
which simplifies setting them up and maintaining them.

For example, the following file |draft.tex|
with a compilation flag |\version| as described in \secref{sec:flags}
compiles the main document as a draft:
%
\begin{center}
\begin{tabular}{l}
|\def\version{draft}|\\
|\input{childdoc.def}|\\
|\childdocforward{|\textit{main}|}|
\end{tabular}
\end{center}
%
Likewise, the following files |final|\textit{nn}|.tex|
compile the final version of the child document
|child|\textit{nn}|.tex|:
%
\begin{center}
\begin{tabular}{l}
|\def\version{final}|\\
|\input{childdoc.def}|\\
|\childdocforwardprefix{final}{child}|
\end{tabular}
\end{center}
%

Note that when several versions of a main file and/or of each child file
are to be generated, it may be convenient to set up a |Makefile| or
shell script to automatise the process.

%%%%%%%%%%%%%%%%%%%%%%%%%%%%%%%%%%%%%%%%%%%%%%%%%%%%%%%%%%%%%%%%%%%%%%%%%%%%%%%%
\subsection{Command Line Processing}
\label{sec:commandline}

The effect of redirection files can also be achieved by invoking
the \LaTeX{} compiler with a more elaborate command line.
Most conveniently this should be done as part
of a shell script or a |Makefile|.

When using \textsf{childdoc} in the main file, the following
command lines effectively perform a redirection
(note that depending on the shell being used,
backslashes may have to be doubled: `|\|' $\to$ `|\\|'):
%
\begin{center}
|... -jobname "|\textit{target}|" |\\|"|[\textit{flags}]%
|\input{childdoc.def}\childdocforward[|\textit{main}|]{|\textit{dest}|}"|
\end{center}
%
Here \textit{target} is the name of the output file,
\textit{main} is the name of the main file
and \textit{dest} is the name of the main or child file to be processed
(all filenames without extensions).
The optional argument \textit{main} can be omitted
if \textit{main} matches \textit{dest}.
Optionally, compilation \textit{flags} can be defined via |\def| commands.
This command line makes the \TeX{} engine believe
it is compiling the file \textit{target}
whose content is specified as the latter parameter.
The provided code then forwards the processing to
\textit{main} or \textit{dest} as described in \secref{sec:forward}.

%%%%%%%%%%%%%%%%%%%%%%%%%%%%%%%%%%%%%%%%%%%%%%%%%%%%%%%%%%%%%%%%%%%%%%%%%%%%%%%%
\subsection{Include by Input}
\label{sec:input}

Including child documents by |\include| has some restrictions by design.
Most notably, the content of a child document always occupies
its own set of pages; pages cannot be shared between child documents.
Usually, this behaviour makes perfect sense
because each child document contain an essential part of the document.
However, in some situations it may be desirable to compose
a document from a collection of parts
without having mandatory page breaks between then.
For this case, the package
provides a mechanism to include parts
by |\input| which can also be processed individually.
However, by construction this mechanism
requires manual handling of the content to be output.

%%%%%%%%%%%%%%%%%%%%%%%%%%%%%%%%%%%%%%%%
\DescribeMacro{\ifchilddocmanual}
The main file should be prepared as usual, see \secref{sec:include}.
However, the document body must make a distinction
between processing of an individual part and of the main document, e.g.:
%
\begin{center}
\begin{tabular}{l}
|\ifchilddocmanual|\\
|\input{\childdocname}|\\
|\||else|\\
\textit{document body with }|\input{|\textit{part}|}|\\
|\||fi|
\end{tabular}
\end{center}
%
The conditional |\ifchilddocmanual| is true whenever
a part to be included by |\input| is being compiled,
and the name of the part is stored in |\childdocname|.

%%%%%%%%%%%%%%%%%%%%%%%%%%%%%%%%%%%%%%%%
\DescribeMacro{\childdocby}
Each part to be included by |\input| should start with:
%
\begin{center}
\begin{tabular}{l}
|\input{childdoc.def}|\\
|\childdocby{|\textit{main}|}|\\
\end{tabular}
\end{center}
%
The directive |\childdocby| is similar to |\childdocof|
described in \secref{sec:include},
but the subsequent selection of content must be done manually.
To that end, both |\ifchilddoc| and |\ifchilddocmanual|
will be true upon processing of a part,
and the name of the part is stored in |\childdocname|.
Note that |\jobname| will be set to the filename of the current part
so that each part receives an individual |.aux| file
that does not interfere with the |.aux| file(s) of the main document.
This behaviour can be altered by the alternative form
|\childdocby[*]{|\textit{main}|}| (with a non-empty optional argument)
which uses the |.aux| file of the main document
by setting |\jobname| to \textit{main}.

%%%%%%%%%%%%%%%%%%%%%%%%%%%%%%%%%%%%%%%%%%%%%%%%%%%%%%%%%%%%%%%%%%%%%%%%%%%%%%%%
\subsection{Driver Development}
\label{sec:driver}

The \textsf{childdoc} mechanism can also be use for the development
of definition files such as \LaTeX{} styles or classes.
This case differs from the above setup with multiple parts
included by |\include| in that no |\includeonly| should be invoked.
This can be achieved by starting the include file
(before |\ProvidesPackage|) with:
%
\begin{center}
\begin{tabular}{l}
|\input{childdoc.def}|\\
|\childdocforward{|\textit{main}|}|\\
\end{tabular}
\end{center}
%
or alternatively with:
%
\begin{center}
\begin{tabular}{l}
|\input{childdoc.def}|\\
|\childdocby{|\textit{main}|}|\\
\end{tabular}
\end{center}
%
Both forms have slightly different effects as described above.
The main file is prepared as usual, see \secref{sec:include}.

%%%%%%%%%%%%%%%%%%%%%%%%%%%%%%%%%%%%%%%%%%%%%%%%%%%%%%%%%%%%%%%%%%%%%%%%%%%%%%%%
\subsection{Legacy Detection}
\label{sec:detection}

The directive |\childdocmain| in the main file can detect
whether the complete document or merely a child is to be compiled
even without using the directive |\childdocof|.
This method is deprecated because it is less robust
and there is no compelling reason to use it;
it is merely provided for backward compatibility
and it may be removed in future versions.

If the detection mechanism is to be used,
it is mandatory to correctly specify
the filename of the main file as the argument of |\childdocmain|:
%
\begin{center}
\begin{tabular}{l}
|\input{childdoc.def}|\\
|\childdocmain{|\textit{main}|}|\\
\end{tabular}
\end{center}
%
If |\jobname| does not match the argument \textit{main} of |\childdocmain|,
it is assumed that |\jobname| points to the child file to be compiled.
When using |\childdocmain| with the main file specified as argument,
it suffices to start a child file
with just |\input{|\textit{main}|}|
without loading of the package and using |\childdocof|.
If instead all processing is done
with the appropriate \textsf{childdoc} directives,
the argument of \textit{main} of |\childdocmain| can be empty.

An alternative version of the command line processing described
in \secref{sec:commandline} using the detection mechanism reads:
%
\begin{center}
|... -jobname "|\textit{target}|" "|[\textit{flags}]%
[|\def\jobname{|\textit{dest}|}|]|\input{|\textit{main}|}"|
\end{center}

%%%%%%%%%%%%%%%%%%%%%%%%%%%%%%%%%%%%%%%%%%%%%%%%%%%%%%%%%%%%%%%%%%%%%%%%%%%%%%%%
\subsection{Manual Code}
\label{sec:manual}

In case one cannot be certain whether the definitions file |childdoc.def|
is installed on the target \TeX{} distribution
and one prefers not to ship it,
it is conceivable to paste a few relevant commands into the sources.

To that end, drop all statements |\input{childdoc.def}|
and perform the replacements as outlined below.
Instead of |\childdocmain{|\textit{main}|}| add the following code
to the top of the main file:
%
\begin{center}
\begin{tabular}{l}
|\||ifdefined\childdocname\endinput\||fi\newif\ifchilddoc|\\
|\edef\childdocname{\scantokens\expandafter{\jobname\noexpand}}|\\
|\def\childdocmain{|\textit{main}|}\||ifx\childdocmain\childdocname\||else|\\
|\childdoctrue\includeonly{\childdocname}\let\jobname\childdocmain\||fi|\\
\end{tabular}
\end{center}
%
Instead of |\childdocof{|\textit{main}|}| just include the main file
at the top of each child file:
%
\begin{center}
|\input{|\textit{main}|}|
\end{center}
%
A simple redirection |\childdocforward{|\textit{dest}|}| is achieved by:
%
\begin{center}
|\def\jobname{|\textit{dest}|}\input{\jobname}|
\end{center}
%
The redirection with prefix
|\childdocforwardprefix[|\textit{prefix}|]{|\textit{dest}|}|
is accomplished by:
%
\begin{center}
\begin{tabular}{l}
|{\edef\jobname{\scantokens\expandafter{\jobname\noexpand}}|\\
|\def\redirectjob |\textit{prefix}|#1~~~{\gdef\jobname{|\textit{dest}|#1}}|\\
|\expandafter\redirectjob\jobname~~~}\input{\jobname}|
\end{tabular}
\end{center}

In an alternative approach,
child documents can be compiled by a specific command line
without additional code or specific definitions:
%
\begin{center}
|... -jobname "|\textit{target}|" "|[\textit{flags}]%
|\includeonly{|\textit{dest}|}\input{|\textit{main}|}"|
\end{center}
%

%%%%%%%%%%%%%%%%%%%%%%%%%%%%%%%%%%%%%%%%%%%%%%%%%%%%%%%%%%%%%%%%%%%%%%%%%%%%%%%%
%%%%%%%%%%%%%%%%%%%%%%%%%%%%%%%%%%%%%%%%%%%%%%%%%%%%%%%%%%%%%%%%%%%%%%%%%%%%%%%%
\section{Information}

%%%%%%%%%%%%%%%%%%%%%%%%%%%%%%%%%%%%%%%%%%%%%%%%%%%%%%%%%%%%%%%%%%%%%%%%%%%%%%%%
\subsection{Copyright}

Copyright \copyright{} 2017--2018 Niklas Beisert

This work may be distributed and/or modified under the
conditions of the \LaTeX{} Project Public License, either version 1.3
of this license or (at your option) any later version.
The latest version of this license is in
  \url{http://www.latex-project.org/lppl.txt}
and version 1.3 or later is part of all distributions of \LaTeX{}
version 2005/12/01 or later.

This work has the LPPL maintenance status `maintained'.

The Current Maintainer of this work is Niklas Beisert.

This work consists of the files |README.txt|, |childdoc.ins| and |childdoc.dtx|
as well as the derived files |childdoc.def|, |cdocsamp.tex|
with |cdocsch1.tex|, |cdocsch2.tex|, |cdocspt3.tex|, |cdocspt4.tex|,
|cdocsdrf.tex|, |cdocsfn1.tex|, |cdocsfn2.tex|
as well as |childdoc.pdf|.

%%%%%%%%%%%%%%%%%%%%%%%%%%%%%%%%%%%%%%%%%%%%%%%%%%%%%%%%%%%%%%%%%%%%%%%%%%%%%%%%
\subsection{Files and Installation}

The package consists of the files:
%
\begin{center}
\begin{tabular}{ll}
    |README.txt|   & readme file \\
    |childdoc.ins| & installation file \\
    |childdoc.dtx| & source file \\
    |childdoc.def| & definition file \\
    |cdocsamp.tex| & sample main file \\
    |cdocsch1.tex| & sample include file \\
    |cdocsch2.tex| & sample include file \\
    |cdocspt3.tex| & sample part file \\
    |cdocspt4.tex| & sample part file \\
    |cdocsdrf.tex| & sample redirection file \\
    |cdocsfn1.tex| & sample redirection file \\
    |cdocsfn2.tex| & sample redirection file \\
    |childdoc.pdf| & manual
\end{tabular}
\end{center}
%
The distribution consists of the files
|README.txt|, |childdoc.ins| and |childdoc.dtx|.
%
\begin{itemize}
\item
Run (pdf)\LaTeX{} on |childdoc.dtx|
to compile the manual |childdoc.pdf| (this file).
\item
Run \LaTeX{} on |childdoc.ins| to create the definitions file |childdoc.def|
and the sample |cdocsamp.tex| with include files
|cdocsch1.tex|, |cdocsch2.tex|, |cdocspt3.tex|, |cdocspt4.tex|,
|cdocsdrf.tex|, |cdocsfn1.tex|, |cdocsfn2.tex|.
Then copy the file |childdoc.def| to an appropriate directory of your \LaTeX{}
distribution, e.g.\ \textit{texmf-root}|/tex/latex/childdoc|.
\end{itemize}

%%%%%%%%%%%%%%%%%%%%%%%%%%%%%%%%%%%%%%%%%%%%%%%%%%%%%%%%%%%%%%%%%%%%%%%%%%%%%%%%
\subsection{Related CTAN Packages}

There are several other packages which offer a similar functionality:
%
\begin{itemize}
\item
The packages
\href{http://ctan.org/pkg/docmute}{\textsf{docmute}},
\href{http://ctan.org/pkg/includex}{\textsf{includex}} and
\href{http://ctan.org/pkg/standalone}{\textsf{standalone}}
provide commands to include only the document body of
a child file thus allowing both files to be compiled individually.
\item
The packages \href{http://ctan.org/pkg/subdocs}{\textsf{subdocs}}
and \href{http://ctan.org/pkg/subfiles}{\textsf{subfiles}}
provide structures in which the main and child documents can be
encapsulated and allowing them to be compiled individually.
The inclusion mechanism is different from the conventional |\include|.
\item
The package \href{http://ctan.org/pkg/combine}{\textsf{combine}}
is an elaborate solution to combine several documents into one.
\end{itemize}
%
See also the CTAN topic \href{http://ctan.org/topic/subdocs}{\textsf{subdocs}}
for further related packages.
The present package differs from the above solutions in that
a document structure constructed with the conventional |\include| mechanism
just needs two extra commands at the top of every file
such that all constituent files can be compiled individually.

%%%%%%%%%%%%%%%%%%%%%%%%%%%%%%%%%%%%%%%%%%%%%%%%%%%%%%%%%%%%%%%%%%%%%%%%%%%%%%%%
%\subsection{Feature Suggestions}
%
%The following is a list of features which may be useful for future
%versions of this package:
%%
%\begin{itemize}
%\item
%\ldots
%\end{itemize}

%%%%%%%%%%%%%%%%%%%%%%%%%%%%%%%%%%%%%%%%%%%%%%%%%%%%%%%%%%%%%%%%%%%%%%%%%%%%%%%%
\subsection{Revision History}

%%%%%%%%%%%%%%%%%%%%%%%%%%%%%%%%%%%%%%%%
\paragraph{v2.0:} 2018/12/30

\begin{itemize}
\item
immediate forward processing
\item
added |\childdocby| mechanism
\item
manual restructured
\end{itemize}

%%%%%%%%%%%%%%%%%%%%%%%%%%%%%%%%%%%%%%%%
\paragraph{v1.6:} 2018/01/17

\begin{itemize}
\item
application for development of include files
\item
corrections to manual
\end{itemize}

%%%%%%%%%%%%%%%%%%%%%%%%%%%%%%%%%%%%%%%%
\paragraph{v1.5:} 2017/05/21

\begin{itemize}
\item
more complete structuring introduced
\item
|\childdocof| introduced
\item
|\childdoc| renamed to |\childdocmain|
\item
|\childredirect| renamed to |\childdocforward| and |\childdocforwardprefix|
and functionality expanded
\end{itemize}

%%%%%%%%%%%%%%%%%%%%%%%%%%%%%%%%%%%%%%%%
\paragraph{v1.0:} 2017/04/27

\begin{itemize}
\item
manual and install package
\item
first version published on CTAN
\end{itemize}

%%%%%%%%%%%%%%%%%%%%%%%%%%%%%%%%%%%%%%%%
\paragraph{v0.6:} 2017/04/26

\begin{itemize}
\item
redirection mechanism added
\end{itemize}

%%%%%%%%%%%%%%%%%%%%%%%%%%%%%%%%%%%%%%%%
\paragraph{v0.5:} 2017/04/26

\begin{itemize}
\item
functionality in definition file
\end{itemize}


%%%%%%%%%%%%%%%%%%%%%%%%%%%%%%%%%%%%%%%%%%%%%%%%%%%%%%%%%%%%%%%%%%%%%%%%%%%%%%%%
%%%%%%%%%%%%%%%%%%%%%%%%%%%%%%%%%%%%%%%%%%%%%%%%%%%%%%%%%%%%%%%%%%%%%%%%%%%%%%%%
%%%%%%%%%%%%%%%%%%%%%%%%%%%%%%%%%%%%%%%%%%%%%%%%%%%%%%%%%%%%%%%%%%%%%%%%%%%%%%%%
\appendix

\settowidth\MacroIndent{\rmfamily\scriptsize 000\ }

 \DocInput{childdoc.dtx}

\end{document}
%</driver>
% \fi
%
% %%%%%%%%%%%%%%%%%%%%%%%%%%%%%%%%%%%%%%%%%%%%%%%%%%%%%%%%%%%%%%%%%%%%%%%%%%%%%%
% %%%%%%%%%%%%%%%%%%%%%%%%%%%%%%%%%%%%%%%%%%%%%%%%%%%%%%%%%%%%%%%%%%%%%%%%%%%%%%
% \section{Sample}
%\iffalse
%<*samplemain>
%\fi
%
% The following presents a sample document
% with two chapters, two parts, a title page,
% a compile flag as well as three forwarding files to set the flag.
% It consists of eight |.tex| files:
% \begin{center}
% \begin{tabular}{ll}
% |cdocsamp.tex|&main file\\
% |cdocsch1.tex|&include file for chapter 1\\
% |cdocsch2.tex|&include file for chapter 2\\
% |cdocspt3.tex|&include file for part 3\\
% |cdocspt4.tex|&include file for part 4\\
% |cdocsdrf.tex|&forwarding file for main file in draft mode\\
% |cdocsfi1.tex|&forwarding file for final version of chapter 1\\
% |cdocsfi2.tex|&forwarding file for final version of chapter 2\\
% \end{tabular}
% \end{center}
% Each of the eight files can be compiled directly by the \LaTeX{} compiler.
%
% %%%%%%%%%%%%%%%%%%%%%%%%%%%%%%%%%%%%%%
% \paragraph{Main File.}
%
% The main file is called |cdocsamp.tex|.
%
% Load the \textsf{childdoc} definitions and
% declare the filename for the main document:
%    \begin{macrocode}
\input{childdoc.def}
\childdocmain{}
%    \end{macrocode}

% Optional override for |\version| flag:
%    \begin{macrocode}
%%\ifchilddoc\else\providecommand{\version}{draft}\fi
%    \end{macrocode}

% Define the default values for the |\version| flag
% (|final| for the main file and |draft| for childs):
%    \begin{macrocode}
\ifchilddoc
\providecommand{\version}{draft}
\else
\providecommand{\version}{final}
\fi
%    \end{macrocode}

% Load the standard document class:
%    \begin{macrocode}
\documentclass[12pt]{article}
%    \end{macrocode}

% Start the document body:
%    \begin{macrocode}
\begin{document}
%    \end{macrocode}

% Declare a title page.
% Print title, part of document being processed and version flag:
%    \begin{macrocode}
\addtocounter{page}{-1}
\begin{center}
{\LARGE\bfseries{}childdoc example\par}
\vspace{1cm}
\ifchilddoc
\ifchilddocmanual part\else chapter\fi:
`\childdocname' of `\childdocjob'\par
\else
main document: `\childdocjob'\par
\fi
version: \version\par
\end{center}
\newpage
%    \end{macrocode}

% Manually include selected file,
% otherwise process as usual:
%    \begin{macrocode}
\ifchilddocmanual
\section*{part `\childdocname'}
\input{\childdocname}
\else
%    \end{macrocode}

% Include the two chapters:
%    \begin{macrocode}
\include{cdocsch1}
\include{cdocsch2}
%    \end{macrocode}

% Include the two parts unless only chapters should be displayed:
%    \begin{macrocode}
\ifchilddoc\else
\section{part three}
\input{cdocspt3}
\section{part four}
\input{cdocspt4}
\fi
%    \end{macrocode}

% Process as usual until here:
%    \begin{macrocode}
\fi
%    \end{macrocode}

% End of document body:
%    \begin{macrocode}
\end{document}
%    \end{macrocode}
%\iffalse
%</samplemain>
%\fi
%
% %%%%%%%%%%%%%%%%%%%%%%%%%%%%%%%%%%%%%%
% \paragraph{Chapter Include Files.}
%
% The include files are called |cdocsch1.tex| and |cdocsch2.tex|.
%
%\iffalse
%<*samplechap1|samplechap2>
%\fi

% Optional override for |\version| flag:
%    \begin{macrocode}
%%\providecommand{\version}{final}
%    \end{macrocode}

% Include the main document:
%    \begin{macrocode}
\input{childdoc.def}
\childdocof{cdocsamp}
%    \end{macrocode}

%\iffalse
%</samplechap1|samplechap2>
%\fi
%
%\iffalse
%<*samplechap1>
%\fi
% Some text for chapter 1:
%    \begin{macrocode}
\section{one}
some text in chapter one
%    \end{macrocode}

%\iffalse
%</samplechap1>
%\fi
% Some text for chapter 2:
%\iffalse
%<*samplechap2>
%\fi
%    \begin{macrocode}
\section{two}
more text in chapter two
%    \end{macrocode}

%\iffalse
%</samplechap2>
%\fi
%
% %%%%%%%%%%%%%%%%%%%%%%%%%%%%%%%%%%%%%%
% \paragraph{Part Include Files.}
%
% The include files are called |cdocspt3.tex| and |cdocspt4.tex|.
%
%\iffalse
%<*samplepart3|samplepart4>
%\fi

% Optional override for |\version| flag:
%    \begin{macrocode}
%%\providecommand{\version}{final}
%    \end{macrocode}

% Include the main document:
%    \begin{macrocode}
\input{childdoc.def}
\childdocby{cdocsamp}
%    \end{macrocode}

%\iffalse
%</samplepart3|samplepart4>
%\fi
%
%\iffalse
%<*samplepart3>
%\fi
% Some text for part 3:
%    \begin{macrocode}
some text in part three
%    \end{macrocode}

%\iffalse
%</samplepart3>
%\fi
% Some text for part 4:
%\iffalse
%<*samplepart4>
%\fi
%    \begin{macrocode}
more text in part four
%    \end{macrocode}

%\iffalse
%</samplepart4>
%\fi
%
% %%%%%%%%%%%%%%%%%%%%%%%%%%%%%%%%%%%%%%
% \paragraph{Forwarding for a Complete Draft.}
%
% The following forwarding file |cdocsdrf.tex|
% compiles the main document in draft mode:
%\iffalse
%<*sampledraft>
%\fi
%    \begin{macrocode}
\def\version{draft}
\input{childdoc.def}
\childdocforward{cdocsamp}
%    \end{macrocode}

%\iffalse
%</sampledraft>
%\fi
%
% %%%%%%%%%%%%%%%%%%%%%%%%%%%%%%%%%%%%%%
% \paragraph{Forwarding for Final Version of the Chapters.}
%
% The following forwarding files |cdocsfn1.tex| and |cdocsfn2.tex|
% (with identical content)
% compile the final versions of the child documents
% |cdocsch1.tex| and |cdocsch2.tex|, respectively:
%\iffalse
%<*samplefinal>
%\fi
%    \begin{macrocode}
\def\version{final}
\input{childdoc.def}
\childdocforwardprefix[cdocsamp]{cdocsfn}{cdocsch}
%    \end{macrocode}

%\iffalse
%</samplefinal>
%\fi
%
% %%%%%%%%%%%%%%%%%%%%%%%%%%%%%%%%%%%%%%
% \paragraph{Command Line Processing.}
%
% The following three command lines generate the output files
% |cdocscld|, |cdocscl1| and |cdocscl2|
% which should be identical to
% |cdocsdrf|, |cdocsch1| and |cdocsfn2|, respectively:
% \begin{center}
% \begin{tabular}{l}
% |latex -jobname cdocscld \|\\
% |  "\def\version{draft}\input{childdoc.def}\childdocforward{cdocsamp}"|\\
% |latex -jobname cdocscl1 \|\\
% |  "\input{childdoc.def}\childdocforward[cdocsamp]{cdocsch1}"|\\
% |latex -jobname cdocscl2 \|\\
% |  "\def\version{final}\input{childdoc.def}\childdocforward{cdocsch2}"|
% \end{tabular}
% \end{center}
% Note that the trailing backslash on each first line
% merely continues the input to the second line
% (for convenient cut ant paste).
% Furthermore, the command |latex| can be replaced by any
% of its alternative versions such as |pdflatex|.
%
% %%%%%%%%%%%%%%%%%%%%%%%%%%%%%%%%%%%%%%%%%%%%%%%%%%%%%%%%%%%%%%%%%%%%%%%%%%%%%%
% %%%%%%%%%%%%%%%%%%%%%%%%%%%%%%%%%%%%%%%%%%%%%%%%%%%%%%%%%%%%%%%%%%%%%%%%%%%%%%
% \section{Implementation}
%\iffalse
%<*package>
%\fi
%
% This section describes the definitions file |childdoc.def|.

% The definitions cannot be loaded using |\usepackage| or |\RequirePackage|
% which has a mechanism to prevent loading a style file more than once.
% When loading the definitions by means of |\input|
% multiple instances have to be prevented manually:
%\iffalse
%This code needs to be before the `\ProvidesFile' directive
%which is defined at the beginning of this file.
%Therefore it is also placed there and commented out here.
%</package>
%<*discard>
%\fi
%    \begin{macrocode}
\ifdefined\childdocmain\endinput\fi
%    \end{macrocode}
%\iffalse
%</discard>
%<*package>
%\fi
%
% \macro{\ifchilddoc}
% \macro{\ifchilddocmanual}
% The conditional |\ifchilddoc| tells whether a
% child (true) or main (false) document is being compiled.
% The conditional |\ifchilddocmanual| tells whether
% the |\includeonly| mechanism is used (false) or
% the selection of child files must be performed manually (true).
% The definitions initialise to false:
%    \begin{macrocode}
\newif\ifchilddoc
\newif\ifchilddocmanual
%    \end{macrocode}

% \macro{\childdocname}
% \macro{\childdocjob}
% The macro |\childdocname| stores the name of the main document
% to be compiled. The macro |\childdocjob| stores the name of
% the document on which the \LaTeX{} compiler was originally invoked.
% The content of |\jobname| cannot be compared
% to filenames specified in the source due to different catcodes.
% The following code rescans |\jobname|, stores the result
% in |\childdocname| and saves a copy in |\childdocjob|:
%    \begin{macrocode}
\edef\childdocname{\scantokens\expandafter{\jobname\noexpand}}
\let\childdocjob\childdocname
%    \end{macrocode}

% \macro{\childdocdisable}
% The macro |\childdocdisable| prevents the main file
% from being processed more than once.
% At this stage, the main document command |\childdocmain|
% is assumed to be called once again where it should do nothing.
% Any subsequent call to it should prevent
% a secondary processing of the main document
% It overwrites the forwarding commands
% |\childdocof| and |\childdocforward|
% with empty macros to prevent further inclusions of the main document:
%    \begin{macrocode}
\newcommand{\childdocdisable}
{
  \renewcommand{\childdocmain}[1]{\renewcommand{\childdocmain}[1]{\endinput}}
  \renewcommand{\childdocof}[1]{}
  \renewcommand{\childdocby}[2][]{}
  \renewcommand{\childdocforward}[2][]{}
  \renewcommand{\childdocdisable}{}
}
%    \end{macrocode}

% \macro{\childdocmain}
% The macro |\childdocmain| is to be called at the top of the main file
% with nothing or the main filename (without extension) as argument.
% First, it breaks loops.
% If the argument is not empty and does not match |\childdocname|
% (which is set by the first inclusion of |childdoc.def|),
% |\ifchilddoc| is set to true, |\includeonly| is applied to the child file
% and |\jobname| is set to the main file
% (for proper handling of |.aux| files):
%    \begin{macrocode}
\newcommand{\childdocmain}[1]
{
  \childdocdisable\childdocmain{}
  \if?#1?\else
    \begingroup
      \def\childdoctmp{#1}
      \ifx\childdoctmp\childdocname
        \def\childdoctmp{}
      \else
        \def\childdoctmp
        {
          \childdoctrue
          \includeonly{\childdocname}
          \def\childdocjob{#1}
          \def\jobname{#1}
        }
      \fi
      \expandafter
    \endgroup
    \childdoctmp
  \fi
}
%    \end{macrocode}

% \macro{\childdocof}
% The command |\childdocof| redirects
% compilation to the main file |#1|.
%    \begin{macrocode}
\newcommand{\childdocof}[1]
{
  \childdocdisable
  \childdoctrue
  \includeonly{\childdocname}
  \def\jobname{#1}
  \def\childdocjob{#1}
  \input{#1}
}
%    \end{macrocode}

% \macro{\childdocby}
% The command |\childdocby| ....
%    \begin{macrocode}
\newcommand{\childdocby}[2][]
{
  \childdocdisable
  \childdoctrue
  \childdocmanualtrue
  \if?#1?\else
    \def\jobname{#2}
  \fi
  \def\childdocjob{#2}
  \input{#2}
  \endinput
}
%    \end{macrocode}

% \macro{\childdocforward}
% The command |\childdocforward| redirects
% compilation to the main file or
% (if the optional argument is given) a child file.
% Parameters are set as if the main file
% or a child file starting with |\childdocof| was compiled.
% Then compilation is handed over to the main file:
%    \begin{macrocode}
\newcommand{\childdocforward}[2][]
{
  \begingroup
    \if?#1?
      \def\childdoctmp
      {
        \def\childdocname{#2}
        \def\childdocjob{#2}
        \def\jobname{#2}
        \input{#2}
        \endinput
      }
    \else
      \def\childdoctmp
      {
        \childdocdisable
        \def\childdocname{#2}
        \childdoctrue
        \includeonly{#2}
        \def\childdocjob{#1}
        \def\jobname{#1}
        \input{#1}
        \endinput
      }
    \fi
    \expandafter
  \endgroup
  \childdoctmp
}
%    \end{macrocode}

% \macro{\childdocforwardprefix}
% The command |\childdocforwardprefix| redirects
% compilation to the main or a child file by means of a pattern.
% The prefix |#1| in the current filename is replaced by |#2|
% and the suffix of the current filename is kept
% (it is assumed that the filename does not contain the substring `|~~~|'
% which is used as a delimiter).
% Compilation is handed over to the new file by |\childdocforward|:
%    \begin{macrocode}
\newcommand{\childdocforwardprefix}[3][]
{
  \begingroup
    \def\childdocextract #2##1~~~{\def\childdoctmp{\childdocforward[#1]{#3##1}}}
    \expandafter\childdocextract\childdocname~~~
    \expandafter
  \endgroup
  \childdoctmp
}
%    \end{macrocode}

% \macro{\childdoc}
% The deprecated macro |\childdoc| is a legacy version of |\childdocmain|:
%    \begin{macrocode}
\newcommand{\childdoc}{\childdocmain}
%    \end{macrocode}

% \macro{\childdocredirect}
% The deprecated macro |\childdocredirect| is a legacy version
% of |\childdocforward| and |\childdocforwardprefix|:
%    \begin{macrocode}
\newcommand{\childdocredirect}[2][]
{
  \begingroup
    \if?#1?
      \def\childdoctmp{\childdocforward{#2}}
    \else
      \def\childdoctmp{\childdocforwardprefix{#1}{#2}}
    \fi
    \expandafter
  \endgroup
  \childdoctmp
}
%    \end{macrocode}

%\iffalse
%</package>
%\fi
%
\endinput
\childdocforward{cdocsch2}"|
% \end{tabular}
% \end{center}
% Note that the trailing backslash on each first line
% merely continues the input to the second line
% (for convenient cut ant paste).
% Furthermore, the command |latex| can be replaced by any
% of its alternative versions such as |pdflatex|.
%
% %%%%%%%%%%%%%%%%%%%%%%%%%%%%%%%%%%%%%%%%%%%%%%%%%%%%%%%%%%%%%%%%%%%%%%%%%%%%%%
% %%%%%%%%%%%%%%%%%%%%%%%%%%%%%%%%%%%%%%%%%%%%%%%%%%%%%%%%%%%%%%%%%%%%%%%%%%%%%%
% \section{Implementation}
%\iffalse
%<*package>
%\fi
%
% This section describes the definitions file |childdoc.def|.

% The definitions cannot be loaded using |\usepackage| or |\RequirePackage|
% which has a mechanism to prevent loading a style file more than once.
% When loading the definitions by means of |\input|
% multiple instances have to be prevented manually:
%\iffalse
%This code needs to be before the `\ProvidesFile' directive
%which is defined at the beginning of this file.
%Therefore it is also placed there and commented out here.
%</package>
%<*discard>
%\fi
%    \begin{macrocode}
\ifdefined\childdocmain\endinput\fi
%    \end{macrocode}
%\iffalse
%</discard>
%<*package>
%\fi
%
% \macro{\ifchilddoc}
% \macro{\ifchilddocmanual}
% The conditional |\ifchilddoc| tells whether a
% child (true) or main (false) document is being compiled.
% The conditional |\ifchilddocmanual| tells whether
% the |\includeonly| mechanism is used (false) or
% the selection of child files must be performed manually (true).
% The definitions initialise to false:
%    \begin{macrocode}
\newif\ifchilddoc
\newif\ifchilddocmanual
%    \end{macrocode}

% \macro{\childdocname}
% \macro{\childdocjob}
% The macro |\childdocname| stores the name of the main document
% to be compiled. The macro |\childdocjob| stores the name of
% the document on which the \LaTeX{} compiler was originally invoked.
% The content of |\jobname| cannot be compared
% to filenames specified in the source due to different catcodes.
% The following code rescans |\jobname|, stores the result
% in |\childdocname| and saves a copy in |\childdocjob|:
%    \begin{macrocode}
\edef\childdocname{\scantokens\expandafter{\jobname\noexpand}}
\let\childdocjob\childdocname
%    \end{macrocode}

% \macro{\childdocdisable}
% The macro |\childdocdisable| prevents the main file
% from being processed more than once.
% At this stage, the main document command |\childdocmain|
% is assumed to be called once again where it should do nothing.
% Any subsequent call to it should prevent
% a secondary processing of the main document
% It overwrites the forwarding commands
% |\childdocof| and |\childdocforward|
% with empty macros to prevent further inclusions of the main document:
%    \begin{macrocode}
\newcommand{\childdocdisable}
{
  \renewcommand{\childdocmain}[1]{\renewcommand{\childdocmain}[1]{\endinput}}
  \renewcommand{\childdocof}[1]{}
  \renewcommand{\childdocby}[2][]{}
  \renewcommand{\childdocforward}[2][]{}
  \renewcommand{\childdocdisable}{}
}
%    \end{macrocode}

% \macro{\childdocmain}
% The macro |\childdocmain| is to be called at the top of the main file
% with nothing or the main filename (without extension) as argument.
% First, it breaks loops.
% If the argument is not empty and does not match |\childdocname|
% (which is set by the first inclusion of |childdoc.def|),
% |\ifchilddoc| is set to true, |\includeonly| is applied to the child file
% and |\jobname| is set to the main file
% (for proper handling of |.aux| files):
%    \begin{macrocode}
\newcommand{\childdocmain}[1]
{
  \childdocdisable\childdocmain{}
  \if?#1?\else
    \begingroup
      \def\childdoctmp{#1}
      \ifx\childdoctmp\childdocname
        \def\childdoctmp{}
      \else
        \def\childdoctmp
        {
          \childdoctrue
          \includeonly{\childdocname}
          \def\childdocjob{#1}
          \def\jobname{#1}
        }
      \fi
      \expandafter
    \endgroup
    \childdoctmp
  \fi
}
%    \end{macrocode}

% \macro{\childdocof}
% The command |\childdocof| redirects
% compilation to the main file |#1|.
%    \begin{macrocode}
\newcommand{\childdocof}[1]
{
  \childdocdisable
  \childdoctrue
  \includeonly{\childdocname}
  \def\jobname{#1}
  \def\childdocjob{#1}
  \input{#1}
}
%    \end{macrocode}

% \macro{\childdocby}
% The command |\childdocby| ....
%    \begin{macrocode}
\newcommand{\childdocby}[2][]
{
  \childdocdisable
  \childdoctrue
  \childdocmanualtrue
  \if?#1?\else
    \def\jobname{#2}
  \fi
  \def\childdocjob{#2}
  \input{#2}
  \endinput
}
%    \end{macrocode}

% \macro{\childdocforward}
% The command |\childdocforward| redirects
% compilation to the main file or
% (if the optional argument is given) a child file.
% Parameters are set as if the main file
% or a child file starting with |\childdocof| was compiled.
% Then compilation is handed over to the main file:
%    \begin{macrocode}
\newcommand{\childdocforward}[2][]
{
  \begingroup
    \if?#1?
      \def\childdoctmp
      {
        \def\childdocname{#2}
        \def\childdocjob{#2}
        \def\jobname{#2}
        \input{#2}
        \endinput
      }
    \else
      \def\childdoctmp
      {
        \childdocdisable
        \def\childdocname{#2}
        \childdoctrue
        \includeonly{#2}
        \def\childdocjob{#1}
        \def\jobname{#1}
        \input{#1}
        \endinput
      }
    \fi
    \expandafter
  \endgroup
  \childdoctmp
}
%    \end{macrocode}

% \macro{\childdocforwardprefix}
% The command |\childdocforwardprefix| redirects
% compilation to the main or a child file by means of a pattern.
% The prefix |#1| in the current filename is replaced by |#2|
% and the suffix of the current filename is kept
% (it is assumed that the filename does not contain the substring `|~~~|'
% which is used as a delimiter).
% Compilation is handed over to the new file by |\childdocforward|:
%    \begin{macrocode}
\newcommand{\childdocforwardprefix}[3][]
{
  \begingroup
    \def\childdocextract #2##1~~~{\def\childdoctmp{\childdocforward[#1]{#3##1}}}
    \expandafter\childdocextract\childdocname~~~
    \expandafter
  \endgroup
  \childdoctmp
}
%    \end{macrocode}

% \macro{\childdoc}
% The deprecated macro |\childdoc| is a legacy version of |\childdocmain|:
%    \begin{macrocode}
\newcommand{\childdoc}{\childdocmain}
%    \end{macrocode}

% \macro{\childdocredirect}
% The deprecated macro |\childdocredirect| is a legacy version
% of |\childdocforward| and |\childdocforwardprefix|:
%    \begin{macrocode}
\newcommand{\childdocredirect}[2][]
{
  \begingroup
    \if?#1?
      \def\childdoctmp{\childdocforward{#2}}
    \else
      \def\childdoctmp{\childdocforwardprefix{#1}{#2}}
    \fi
    \expandafter
  \endgroup
  \childdoctmp
}
%    \end{macrocode}

%\iffalse
%</package>
%\fi
%
\endinput

\childdocby{cdocsamp}
%    \end{macrocode}

%\iffalse
%</samplepart3|samplepart4>
%\fi
%
%\iffalse
%<*samplepart3>
%\fi
% Some text for part 3:
%    \begin{macrocode}
some text in part three
%    \end{macrocode}

%\iffalse
%</samplepart3>
%\fi
% Some text for part 4:
%\iffalse
%<*samplepart4>
%\fi
%    \begin{macrocode}
more text in part four
%    \end{macrocode}

%\iffalse
%</samplepart4>
%\fi
%
% %%%%%%%%%%%%%%%%%%%%%%%%%%%%%%%%%%%%%%
% \paragraph{Forwarding for a Complete Draft.}
%
% The following forwarding file |cdocsdrf.tex|
% compiles the main document in draft mode:
%\iffalse
%<*sampledraft>
%\fi
%    \begin{macrocode}
\def\version{draft}
% \iffalse
%
% childdoc.dtx Copyright (C) 2017-2018 Niklas Beisert
%
% This work may be distributed and/or modified under the
% conditions of the LaTeX Project Public License, either version 1.3
% of this license or (at your option) any later version.
% The latest version of this license is in
%   http://www.latex-project.org/lppl.txt
% and version 1.3 or later is part of all distributions of LaTeX
% version 2005/12/01 or later.
%
% This work has the LPPL maintenance status `maintained'.
%
% The Current Maintainer of this work is Niklas Beisert.
%
% This work consists of the files childdoc.dtx and childdoc.ins
% and the derived files childdoc.def and cdocsamp.tex with
% cdocsch1.tex, cdocsch2.tex, cdocsdrf.tex, cdocsfn1.tex, cdocsfn2.tex.
%
%<package>\ifdefined\childdocmain\endinput\fi
%<package>\ProvidesFile{childdoc.def}[2018/12/30 v2.0 child document driver]
%<samplemain>\ProvidesFile{cdocsamp.tex}[2018/12/30 v2.0 sample for childdoc]
%<*driver>
%\ProvidesFile{childdoc.drv}[2018/12/30 v2.0 childdoc reference manual file]
\PassOptionsToClass{10pt,a4paper}{article}
\documentclass{ltxdoc}

\usepackage[margin=35mm]{geometry}
\usepackage{hyperref}
\usepackage{hyperxmp}
\usepackage[usenames]{color}

\hypersetup{colorlinks=true}
\hypersetup{pdfstartview=FitH}
\hypersetup{pdfpagemode=UseNone}
\hypersetup{pdfsource={}}
\hypersetup{pdflang={en-UK}}
\hypersetup{pdfcopyright={Copyright 2017-2018 Niklas Beisert.
  This work may be distributed and/or modified under the
  conditions of the LaTeX Project Public License, either version 1.3
  of this license or (at your option) any later version.}}
\hypersetup{pdflicenseurl={http://www.latex-project.org/lppl.txt}}
\hypersetup{pdfcontactaddress={ETH Zurich, ITP, HIT K,
  Wolfgang-Pauli-Strasse 27}}
\hypersetup{pdfcontactpostcode={8093}}
\hypersetup{pdfcontactcity={Zurich}}
\hypersetup{pdfcontactcountry={Switzerland}}
\hypersetup{pdfcontactemail={nbeisert@itp.phys.ethz.ch}}
\hypersetup{pdfcontacturl={http://people.phys.ethz.ch/\xmptilde nbeisert/}}

\newcommand{\secref}[1]{\hyperref[#1]{section \ref*{#1}}}

\parskip1ex
\parindent0pt
\let\olditemize\itemize
\def\itemize{\olditemize\parskip0pt}

\begin{document}

\title{The \textsf{childdoc} Package}
\hypersetup{pdftitle={The childdoc Package}}
\author{Niklas Beisert\\[2ex]
  Institut f\"ur Theoretische Physik\\
  Eidgen\"ossische Technische Hochschule Z\"urich\\
  Wolfgang-Pauli-Strasse 27, 8093 Z\"urich, Switzerland\\[1ex]
  \href{mailto:nbeisert@itp.phys.ethz.ch}
  {\texttt{nbeisert@itp.phys.ethz.ch}}}
\hypersetup{pdfauthor={Niklas Beisert}}
\hypersetup{pdfsubject={Manual for the LaTeX2e Package childdoc}}
\date{30 December 2018, \textsf{v2.0}}
\maketitle

\begin{abstract}\noindent
\textsf{childdoc} is a \LaTeXe{} package
that enables the direct compilation
of document sections included by |\include|
to individual files.
\end{abstract}

\begingroup
\parskip0ex
\tableofcontents
\endgroup

%%%%%%%%%%%%%%%%%%%%%%%%%%%%%%%%%%%%%%%%%%%%%%%%%%%%%%%%%%%%%%%%%%%%%%%%%%%%%%%%
%%%%%%%%%%%%%%%%%%%%%%%%%%%%%%%%%%%%%%%%%%%%%%%%%%%%%%%%%%%%%%%%%%%%%%%%%%%%%%%%
\section{Introduction}

\LaTeX{} provides a mechanism to structure a large document (such as a book)
into a main file and several child files (containing the chapters)
using the |\include| command.
This mechanism is beneficial for documents
which span hundreds of pages in order to
make the source file(s) more manageable.
Moreover, compilation can be restricted to
selected child files by means of the |\includeonly| command.
The latter feature can be used to reduce the compilation time while editing
(this was significantly more useful in the earlier days of \LaTeX{})
or to generate a smaller document which is easier to navigate.
Another application of |\includeonly| is to generate
documents consisting of selected parts of the complete document.

However, there are a few drawbacks of the plain |\include| mechanism:
\begin{itemize}
\item
The child files cannot be compiled on their own,
they can only be compiled via the main file.
A naive editing environment
(such as a text editor with an option
to have the current file processed by \LaTeX)
may require one to switch to the main file before compiling;
attempting to compile the child file produces errors.
\item
The main file must be modified (each time)
to adjust the |\includeonly| command
to the present needs. This easily leaves the main file in a messy state.
\item
The generated document will always carry the filename
of the main document. This is inconvenient if
several child files are to be compiled and
to be kept for distribution.
\end{itemize}

The present package provides a simple interface
to make child files individually compilable by \LaTeX{}.
Compiling a child file then has the same effect as compiling
the main file with an |\includeonly| command
to select the appropriate child.
Moreover the generated document will carry the name of the child
rather than the main file.
This resolves all three above issues.

This feature is meant to make the editing of books,
thesis documents and lecture notes somewhat more convenient.
However, the package can also be used efficiently for
composing a series of documents (such as exercise sheets)
which are typically distributed individually.
It then assists the author in generating the individual documents
(potentially in different versions)
as well as a document containing the collected series.
Another application is in developing style files
or other kinds of included material
where compilation of the style file could redirect
to a sample or test file.

%%%%%%%%%%%%%%%%%%%%%%%%%%%%%%%%%%%%%%%%%%%%%%%%%%%%%%%%%%%%%%%%%%%%%%%%%%%%%%%%
%%%%%%%%%%%%%%%%%%%%%%%%%%%%%%%%%%%%%%%%%%%%%%%%%%%%%%%%%%%%%%%%%%%%%%%%%%%%%%%%
\section{Usage}

First of all, the package \textsf{childdoc} is \emph{not} a standard
\LaTeXe{} |.sty| style file! Therefore it needs to be invoked in
a non-standard way.

%%%%%%%%%%%%%%%%%%%%%%%%%%%%%%%%%%%%%%%%%%%%%%%%%%%%%%%%%%%%%%%%%%%%%%%%%%%%%%%%
\subsection{Included Files}
\label{sec:include}

%%%%%%%%%%%%%%%%%%%%%%%%%%%%%%%%%%%%%%%%
\DescribeMacro{\childdocmain}
To use the package, add the commands
\begin{center}
\begin{tabular}{l}
|% \iffalse
%
% childdoc.dtx Copyright (C) 2017-2018 Niklas Beisert
%
% This work may be distributed and/or modified under the
% conditions of the LaTeX Project Public License, either version 1.3
% of this license or (at your option) any later version.
% The latest version of this license is in
%   http://www.latex-project.org/lppl.txt
% and version 1.3 or later is part of all distributions of LaTeX
% version 2005/12/01 or later.
%
% This work has the LPPL maintenance status `maintained'.
%
% The Current Maintainer of this work is Niklas Beisert.
%
% This work consists of the files childdoc.dtx and childdoc.ins
% and the derived files childdoc.def and cdocsamp.tex with
% cdocsch1.tex, cdocsch2.tex, cdocsdrf.tex, cdocsfn1.tex, cdocsfn2.tex.
%
%<package>\ifdefined\childdocmain\endinput\fi
%<package>\ProvidesFile{childdoc.def}[2018/12/30 v2.0 child document driver]
%<samplemain>\ProvidesFile{cdocsamp.tex}[2018/12/30 v2.0 sample for childdoc]
%<*driver>
%\ProvidesFile{childdoc.drv}[2018/12/30 v2.0 childdoc reference manual file]
\PassOptionsToClass{10pt,a4paper}{article}
\documentclass{ltxdoc}

\usepackage[margin=35mm]{geometry}
\usepackage{hyperref}
\usepackage{hyperxmp}
\usepackage[usenames]{color}

\hypersetup{colorlinks=true}
\hypersetup{pdfstartview=FitH}
\hypersetup{pdfpagemode=UseNone}
\hypersetup{pdfsource={}}
\hypersetup{pdflang={en-UK}}
\hypersetup{pdfcopyright={Copyright 2017-2018 Niklas Beisert.
  This work may be distributed and/or modified under the
  conditions of the LaTeX Project Public License, either version 1.3
  of this license or (at your option) any later version.}}
\hypersetup{pdflicenseurl={http://www.latex-project.org/lppl.txt}}
\hypersetup{pdfcontactaddress={ETH Zurich, ITP, HIT K,
  Wolfgang-Pauli-Strasse 27}}
\hypersetup{pdfcontactpostcode={8093}}
\hypersetup{pdfcontactcity={Zurich}}
\hypersetup{pdfcontactcountry={Switzerland}}
\hypersetup{pdfcontactemail={nbeisert@itp.phys.ethz.ch}}
\hypersetup{pdfcontacturl={http://people.phys.ethz.ch/\xmptilde nbeisert/}}

\newcommand{\secref}[1]{\hyperref[#1]{section \ref*{#1}}}

\parskip1ex
\parindent0pt
\let\olditemize\itemize
\def\itemize{\olditemize\parskip0pt}

\begin{document}

\title{The \textsf{childdoc} Package}
\hypersetup{pdftitle={The childdoc Package}}
\author{Niklas Beisert\\[2ex]
  Institut f\"ur Theoretische Physik\\
  Eidgen\"ossische Technische Hochschule Z\"urich\\
  Wolfgang-Pauli-Strasse 27, 8093 Z\"urich, Switzerland\\[1ex]
  \href{mailto:nbeisert@itp.phys.ethz.ch}
  {\texttt{nbeisert@itp.phys.ethz.ch}}}
\hypersetup{pdfauthor={Niklas Beisert}}
\hypersetup{pdfsubject={Manual for the LaTeX2e Package childdoc}}
\date{30 December 2018, \textsf{v2.0}}
\maketitle

\begin{abstract}\noindent
\textsf{childdoc} is a \LaTeXe{} package
that enables the direct compilation
of document sections included by |\include|
to individual files.
\end{abstract}

\begingroup
\parskip0ex
\tableofcontents
\endgroup

%%%%%%%%%%%%%%%%%%%%%%%%%%%%%%%%%%%%%%%%%%%%%%%%%%%%%%%%%%%%%%%%%%%%%%%%%%%%%%%%
%%%%%%%%%%%%%%%%%%%%%%%%%%%%%%%%%%%%%%%%%%%%%%%%%%%%%%%%%%%%%%%%%%%%%%%%%%%%%%%%
\section{Introduction}

\LaTeX{} provides a mechanism to structure a large document (such as a book)
into a main file and several child files (containing the chapters)
using the |\include| command.
This mechanism is beneficial for documents
which span hundreds of pages in order to
make the source file(s) more manageable.
Moreover, compilation can be restricted to
selected child files by means of the |\includeonly| command.
The latter feature can be used to reduce the compilation time while editing
(this was significantly more useful in the earlier days of \LaTeX{})
or to generate a smaller document which is easier to navigate.
Another application of |\includeonly| is to generate
documents consisting of selected parts of the complete document.

However, there are a few drawbacks of the plain |\include| mechanism:
\begin{itemize}
\item
The child files cannot be compiled on their own,
they can only be compiled via the main file.
A naive editing environment
(such as a text editor with an option
to have the current file processed by \LaTeX)
may require one to switch to the main file before compiling;
attempting to compile the child file produces errors.
\item
The main file must be modified (each time)
to adjust the |\includeonly| command
to the present needs. This easily leaves the main file in a messy state.
\item
The generated document will always carry the filename
of the main document. This is inconvenient if
several child files are to be compiled and
to be kept for distribution.
\end{itemize}

The present package provides a simple interface
to make child files individually compilable by \LaTeX{}.
Compiling a child file then has the same effect as compiling
the main file with an |\includeonly| command
to select the appropriate child.
Moreover the generated document will carry the name of the child
rather than the main file.
This resolves all three above issues.

This feature is meant to make the editing of books,
thesis documents and lecture notes somewhat more convenient.
However, the package can also be used efficiently for
composing a series of documents (such as exercise sheets)
which are typically distributed individually.
It then assists the author in generating the individual documents
(potentially in different versions)
as well as a document containing the collected series.
Another application is in developing style files
or other kinds of included material
where compilation of the style file could redirect
to a sample or test file.

%%%%%%%%%%%%%%%%%%%%%%%%%%%%%%%%%%%%%%%%%%%%%%%%%%%%%%%%%%%%%%%%%%%%%%%%%%%%%%%%
%%%%%%%%%%%%%%%%%%%%%%%%%%%%%%%%%%%%%%%%%%%%%%%%%%%%%%%%%%%%%%%%%%%%%%%%%%%%%%%%
\section{Usage}

First of all, the package \textsf{childdoc} is \emph{not} a standard
\LaTeXe{} |.sty| style file! Therefore it needs to be invoked in
a non-standard way.

%%%%%%%%%%%%%%%%%%%%%%%%%%%%%%%%%%%%%%%%%%%%%%%%%%%%%%%%%%%%%%%%%%%%%%%%%%%%%%%%
\subsection{Included Files}
\label{sec:include}

%%%%%%%%%%%%%%%%%%%%%%%%%%%%%%%%%%%%%%%%
\DescribeMacro{\childdocmain}
To use the package, add the commands
\begin{center}
\begin{tabular}{l}
|\input{childdoc.def}|\\
|\childdocmain{}|\\
\end{tabular}
\end{center}
at the very top of the main \LaTeX{} file,
in particular \emph{before} the |\documentclass| statement!
The argument of |\childdocmain| should be left empty
(but it must be present).

%%%%%%%%%%%%%%%%%%%%%%%%%%%%%%%%%%%%%%%%
\DescribeMacro{\childdocof}
Furthermore, add the commands
\begin{center}
\begin{tabular}{l}
|\input{childdoc.def}|\\
|\childdocof{|\textit{main}|}|\\
\end{tabular}
\end{center}
at the top of every child file \textit{child}
which is included by |\include{|\textit{child}|}|
from within the main file
(or at least for those files to be compiled individually).
The argument \textit{main} must be the filename of the main file.

There are a couple of
considerations in setting up the main and child documents:

%%%%%%%%%%%%%%%%%%%%%%%%%%%%%%%%%%%%%%%%
\paragraph{Restrictions.}

Please note the following restrictions:
\begin{itemize}
\item
|\childdocmain| must be called with one argument \textit{main}
to ensure compatibility with earlier version of the package.
It must either be empty (|\childdocmain{}|)
or precisely match the filename of the main file in which it is specified.
See \secref{sec:detection} for further information.
\item
The filename \textit{main} must be specified without the |.tex| extension.
\item
The filename \textit{main} is case sensitive
(even in case-insensitive file systems)
due to internal string comparison.
\item
The argument \textit{main} should be fully expanded, it cannot be a macro.
\item
Subdirectories and special characters should be avoided in filenames.
\item
The command |\childdocmain{|\textit{main}|}| must be followed by a whitespace.
It should not be followed immediately by another command
or by a comment mark `|%|'.
This is because the \TeX{} parser reads the token immediately following
the argument of |\childdocmain| and puts it
at the beginning of every child section;
however, a white\-space is ignored.
\end{itemize}

%%%%%%%%%%%%%%%%%%%%%%%%%%%%%%%%%%%%%%%%
\paragraph{Content of Main File.}

It is advisable to place all content in the child files included by |\include|.
Any output contained in the main file will appear in all child documents
unless suppressed manually;
it cannot be suppressed automatically by the |\includeonly| directive
and thus should normally be avoided.
A method to include some content in the main file
by means of conditional processing is described in \secref{sec:conditional}.

%%%%%%%%%%%%%%%%%%%%%%%%%%%%%%%%%%%%%%%%
\paragraph{Page Numbering.}

When only a part of the document is compiled,
the appropriate numbering of pages
(as well as other status parameters)
is determined from the |.aux| files.
The latter contain information from previous passes.
However this information needs to propagate through
all intermediate child documents.
Therefore the page numbering in child documents may well
be inconsistent until the complete document is compiled at least once.

A useful (if unconventional) way to always ensure a consistent
page numbering is to restart the numbering in each child document
and denote the pages by `\textit{child}|.|\textit{page}'
where \textit{child} represents the chapter/section number of the child file.
This can be achieved by the command
|\numberwithin{page}{|\textit{child}|}|
of the \textsf{amsmath} package
where \textit{child} can be |chapter| or |section|
depending on the chosen structuring.
Alternatively, one can modify the macro |\thepage| appropriately
and reset the counter |page| at the start of each child file.

%%%%%%%%%%%%%%%%%%%%%%%%%%%%%%%%%%%%%%%%%%%%%%%%%%%%%%%%%%%%%%%%%%%%%%%%%%%%%%%%
\subsection{Conditional Processing}
\label{sec:conditional}

The package provides a mechanism to compile different versions
of a document. To customise the versions further some conditional processing
can come in handy to distinguish which version is being compiled.
The package provides two macros to describe the compilation context:

%%%%%%%%%%%%%%%%%%%%%%%%%%%%%%%%%%%%%%%%
\DescribeMacro{\ifchilddoc}
The conditional |\ifchilddoc| distinguishes between the compilation of
child documents and the main document:
%
\begin{center}
|\ifchilddoc |\textit{child-code}| |[|\||else |\textit{main-code}]| \||fi|
\end{center}

%%%%%%%%%%%%%%%%%%%%%%%%%%%%%%%%%%%%%%%%
\DescribeMacro{\childdocname}
\DescribeMacro{\childdocjob}
The macro |\childdocname| contains the filename (without extension)
of the main or child file being processed.
Note that |\childdocjob| will always contain the name of the main file.

%%%%%%%%%%%%%%%%%%%%%%%%%%%%%%%%%%%%%%%%
\paragraph{Title Page.}

Conditional processing can be used to include a title or banner page
in the main document when proper precautions are taken.
Importantly, the code in the main file should ensure that the page counter
(as well as other status parameters which are stored in the |.aux| files)
takes the same value after the conditional processing.
Otherwise the page numbers may take divergent values
depending on which part is compiled.

For example, a title page could be declared by:
%
\begin{center}
\begin{tabular}{l}
|\ifchilddoc\||else|\\
|\addtocounter{page}{-1}|\\
\textit{code for title page}\\
|\newpage|\\
|\||fi|
\end{tabular}
\end{center}
%
A banner page for the child documents can be generated by:
%
\begin{center}
\begin{tabular}{l}
|\ifchilddoc|\\
|\addtocounter{page}{-1}|\\
\textit{code for banner page}\\
|\newpage|\\
|\||fi|
\end{tabular}
\end{center}
%
Here one could write a message such as:
\begin{center}
|This is the part \childdocname{} of \childdocjob{}.|
\end{center}

%%%%%%%%%%%%%%%%%%%%%%%%%%%%%%%%%%%%%%%%%%%%%%%%%%%%%%%%%%%%%%%%%%%%%%%%%%%%%%%%
\subsection{Flags}
\label{sec:flags}

The package makes it easy to generate different versions
of the main or child documents.
To this end compilation flags can be defined
and assigned different default values.
They will be particularly useful in conjunction
with the forwarding mechanism described in \secref{sec:forward}.

For example, it may be useful to have a flag |\version|
which can be set to |draft| or |final|.
The document source will contain some conditional code
depending on the value of |\version|.
Suppose further, the flag should default to |final| for the main file
and to |draft| for child files
which is a natural assignment for editing the document.
This is achieved by placing the following code
in the preamble of the main document
(below the |\childdocmain| directive):
%
\begin{center}
\begin{tabular}{l}
|\ifchilddoc|\\
|\providecommand{\version}{draft}|\\
|\||else|\\
|\providecommand{\version}{final}|\\
|\||fi|
\end{tabular}
\end{center}
%
The definition by |\providecommand| makes sure
that previous definitions are not overwritten.
Further statements |\providecommand{\version}{...}|
can thus be added before the above code to override it.

For the main file, one might add a line
(between |\childdocmain| and the above block)
%
\begin{center}
|%\ifchilddoc\||else\providecommand{\version}{draft}\||fi|
\end{center}
%
which can be uncommented to produce a draft version.
Likewise one can add a line to the very top of a child file
(above the |\childdocof{|\textit{main}|}| directive)
%
\begin{center}
|%\providecommand{\version}{final}|
\end{center}
%
which can be uncommented to produce the final version of this child document.

%%%%%%%%%%%%%%%%%%%%%%%%%%%%%%%%%%%%%%%%%%%%%%%%%%%%%%%%%%%%%%%%%%%%%%%%%%%%%%%%
\subsection{Forwarding}
\label{sec:forward}

Different versions of the main or child documents
using compilation flags as described in \secref{sec:flags}
can be (permanently) stored in different files
for convenient compilation, viewing and distribution.
To this end, the package defines a command
to pass on compilation to a different file:

%%%%%%%%%%%%%%%%%%%%%%%%%%%%%%%%%%%%%%%%
\DescribeMacro{\childdocforward}
The command |\childdocforward| redirects processing to
another source file:
%
\begin{center}
\begin{tabular}{l}
|\input{childdoc.def}|\\
|\childdocforward[|\textit{main}|]{|\textit{dest}|}|\\
\end{tabular}
\end{center}
%
The argument \textit{dest} is the destination file
(without extension).
It should be the main file or one of the child files.
Note that further \textsf{childdoc} directives
such as |\childdocof| and |\childdocforward|
in the indicated file will be processed in this form.
The optional argument \textit{main}
passes on directly to the main file \textit{main}
while pretending to compile the child \textit{dest}.
This form behaves as if \textit{dest}
issues |\childdocof{|\textit{main}|}| right away,
and no further \textsf{childdoc} directives will be processed.

%%%%%%%%%%%%%%%%%%%%%%%%%%%%%%%%%%%%%%%%
\DescribeMacro{\...prefix}
In the alternative form |\childdocforwardprefix|,
%
\begin{center}
\begin{tabular}{l}
|\input{childdoc.def}|\\
|\childdocforwardprefix[|\textit{main}|]{|\textit{prefix}|}{|\textit{dest}|}|
\end{tabular}
\end{center}
%
the destination file is determined by a pattern
depending on the current file:
To make this work, the current file must be called
`{\textit{prefix}\hspace{0.2em}\textit{suffix}}'
with \textit{prefix} matching precisely the argument.
Processing is then passed on to the file
`{\textit{dest}\hspace{0.2em}\textit{suffix}}'.
Surely, the same effect is achieved by
directly specifying the
argument `{\textit{dest}\hspace{0.2em}\textit{suffix}}'
in the first form.
However, that requires to set up a different file
for each child. With the alternative form of the command
all these files can have exactly the same content
which simplifies setting them up and maintaining them.

For example, the following file |draft.tex|
with a compilation flag |\version| as described in \secref{sec:flags}
compiles the main document as a draft:
%
\begin{center}
\begin{tabular}{l}
|\def\version{draft}|\\
|\input{childdoc.def}|\\
|\childdocforward{|\textit{main}|}|
\end{tabular}
\end{center}
%
Likewise, the following files |final|\textit{nn}|.tex|
compile the final version of the child document
|child|\textit{nn}|.tex|:
%
\begin{center}
\begin{tabular}{l}
|\def\version{final}|\\
|\input{childdoc.def}|\\
|\childdocforwardprefix{final}{child}|
\end{tabular}
\end{center}
%

Note that when several versions of a main file and/or of each child file
are to be generated, it may be convenient to set up a |Makefile| or
shell script to automatise the process.

%%%%%%%%%%%%%%%%%%%%%%%%%%%%%%%%%%%%%%%%%%%%%%%%%%%%%%%%%%%%%%%%%%%%%%%%%%%%%%%%
\subsection{Command Line Processing}
\label{sec:commandline}

The effect of redirection files can also be achieved by invoking
the \LaTeX{} compiler with a more elaborate command line.
Most conveniently this should be done as part
of a shell script or a |Makefile|.

When using \textsf{childdoc} in the main file, the following
command lines effectively perform a redirection
(note that depending on the shell being used,
backslashes may have to be doubled: `|\|' $\to$ `|\\|'):
%
\begin{center}
|... -jobname "|\textit{target}|" |\\|"|[\textit{flags}]%
|\input{childdoc.def}\childdocforward[|\textit{main}|]{|\textit{dest}|}"|
\end{center}
%
Here \textit{target} is the name of the output file,
\textit{main} is the name of the main file
and \textit{dest} is the name of the main or child file to be processed
(all filenames without extensions).
The optional argument \textit{main} can be omitted
if \textit{main} matches \textit{dest}.
Optionally, compilation \textit{flags} can be defined via |\def| commands.
This command line makes the \TeX{} engine believe
it is compiling the file \textit{target}
whose content is specified as the latter parameter.
The provided code then forwards the processing to
\textit{main} or \textit{dest} as described in \secref{sec:forward}.

%%%%%%%%%%%%%%%%%%%%%%%%%%%%%%%%%%%%%%%%%%%%%%%%%%%%%%%%%%%%%%%%%%%%%%%%%%%%%%%%
\subsection{Include by Input}
\label{sec:input}

Including child documents by |\include| has some restrictions by design.
Most notably, the content of a child document always occupies
its own set of pages; pages cannot be shared between child documents.
Usually, this behaviour makes perfect sense
because each child document contain an essential part of the document.
However, in some situations it may be desirable to compose
a document from a collection of parts
without having mandatory page breaks between then.
For this case, the package
provides a mechanism to include parts
by |\input| which can also be processed individually.
However, by construction this mechanism
requires manual handling of the content to be output.

%%%%%%%%%%%%%%%%%%%%%%%%%%%%%%%%%%%%%%%%
\DescribeMacro{\ifchilddocmanual}
The main file should be prepared as usual, see \secref{sec:include}.
However, the document body must make a distinction
between processing of an individual part and of the main document, e.g.:
%
\begin{center}
\begin{tabular}{l}
|\ifchilddocmanual|\\
|\input{\childdocname}|\\
|\||else|\\
\textit{document body with }|\input{|\textit{part}|}|\\
|\||fi|
\end{tabular}
\end{center}
%
The conditional |\ifchilddocmanual| is true whenever
a part to be included by |\input| is being compiled,
and the name of the part is stored in |\childdocname|.

%%%%%%%%%%%%%%%%%%%%%%%%%%%%%%%%%%%%%%%%
\DescribeMacro{\childdocby}
Each part to be included by |\input| should start with:
%
\begin{center}
\begin{tabular}{l}
|\input{childdoc.def}|\\
|\childdocby{|\textit{main}|}|\\
\end{tabular}
\end{center}
%
The directive |\childdocby| is similar to |\childdocof|
described in \secref{sec:include},
but the subsequent selection of content must be done manually.
To that end, both |\ifchilddoc| and |\ifchilddocmanual|
will be true upon processing of a part,
and the name of the part is stored in |\childdocname|.
Note that |\jobname| will be set to the filename of the current part
so that each part receives an individual |.aux| file
that does not interfere with the |.aux| file(s) of the main document.
This behaviour can be altered by the alternative form
|\childdocby[*]{|\textit{main}|}| (with a non-empty optional argument)
which uses the |.aux| file of the main document
by setting |\jobname| to \textit{main}.

%%%%%%%%%%%%%%%%%%%%%%%%%%%%%%%%%%%%%%%%%%%%%%%%%%%%%%%%%%%%%%%%%%%%%%%%%%%%%%%%
\subsection{Driver Development}
\label{sec:driver}

The \textsf{childdoc} mechanism can also be use for the development
of definition files such as \LaTeX{} styles or classes.
This case differs from the above setup with multiple parts
included by |\include| in that no |\includeonly| should be invoked.
This can be achieved by starting the include file
(before |\ProvidesPackage|) with:
%
\begin{center}
\begin{tabular}{l}
|\input{childdoc.def}|\\
|\childdocforward{|\textit{main}|}|\\
\end{tabular}
\end{center}
%
or alternatively with:
%
\begin{center}
\begin{tabular}{l}
|\input{childdoc.def}|\\
|\childdocby{|\textit{main}|}|\\
\end{tabular}
\end{center}
%
Both forms have slightly different effects as described above.
The main file is prepared as usual, see \secref{sec:include}.

%%%%%%%%%%%%%%%%%%%%%%%%%%%%%%%%%%%%%%%%%%%%%%%%%%%%%%%%%%%%%%%%%%%%%%%%%%%%%%%%
\subsection{Legacy Detection}
\label{sec:detection}

The directive |\childdocmain| in the main file can detect
whether the complete document or merely a child is to be compiled
even without using the directive |\childdocof|.
This method is deprecated because it is less robust
and there is no compelling reason to use it;
it is merely provided for backward compatibility
and it may be removed in future versions.

If the detection mechanism is to be used,
it is mandatory to correctly specify
the filename of the main file as the argument of |\childdocmain|:
%
\begin{center}
\begin{tabular}{l}
|\input{childdoc.def}|\\
|\childdocmain{|\textit{main}|}|\\
\end{tabular}
\end{center}
%
If |\jobname| does not match the argument \textit{main} of |\childdocmain|,
it is assumed that |\jobname| points to the child file to be compiled.
When using |\childdocmain| with the main file specified as argument,
it suffices to start a child file
with just |\input{|\textit{main}|}|
without loading of the package and using |\childdocof|.
If instead all processing is done
with the appropriate \textsf{childdoc} directives,
the argument of \textit{main} of |\childdocmain| can be empty.

An alternative version of the command line processing described
in \secref{sec:commandline} using the detection mechanism reads:
%
\begin{center}
|... -jobname "|\textit{target}|" "|[\textit{flags}]%
[|\def\jobname{|\textit{dest}|}|]|\input{|\textit{main}|}"|
\end{center}

%%%%%%%%%%%%%%%%%%%%%%%%%%%%%%%%%%%%%%%%%%%%%%%%%%%%%%%%%%%%%%%%%%%%%%%%%%%%%%%%
\subsection{Manual Code}
\label{sec:manual}

In case one cannot be certain whether the definitions file |childdoc.def|
is installed on the target \TeX{} distribution
and one prefers not to ship it,
it is conceivable to paste a few relevant commands into the sources.

To that end, drop all statements |\input{childdoc.def}|
and perform the replacements as outlined below.
Instead of |\childdocmain{|\textit{main}|}| add the following code
to the top of the main file:
%
\begin{center}
\begin{tabular}{l}
|\||ifdefined\childdocname\endinput\||fi\newif\ifchilddoc|\\
|\edef\childdocname{\scantokens\expandafter{\jobname\noexpand}}|\\
|\def\childdocmain{|\textit{main}|}\||ifx\childdocmain\childdocname\||else|\\
|\childdoctrue\includeonly{\childdocname}\let\jobname\childdocmain\||fi|\\
\end{tabular}
\end{center}
%
Instead of |\childdocof{|\textit{main}|}| just include the main file
at the top of each child file:
%
\begin{center}
|\input{|\textit{main}|}|
\end{center}
%
A simple redirection |\childdocforward{|\textit{dest}|}| is achieved by:
%
\begin{center}
|\def\jobname{|\textit{dest}|}\input{\jobname}|
\end{center}
%
The redirection with prefix
|\childdocforwardprefix[|\textit{prefix}|]{|\textit{dest}|}|
is accomplished by:
%
\begin{center}
\begin{tabular}{l}
|{\edef\jobname{\scantokens\expandafter{\jobname\noexpand}}|\\
|\def\redirectjob |\textit{prefix}|#1~~~{\gdef\jobname{|\textit{dest}|#1}}|\\
|\expandafter\redirectjob\jobname~~~}\input{\jobname}|
\end{tabular}
\end{center}

In an alternative approach,
child documents can be compiled by a specific command line
without additional code or specific definitions:
%
\begin{center}
|... -jobname "|\textit{target}|" "|[\textit{flags}]%
|\includeonly{|\textit{dest}|}\input{|\textit{main}|}"|
\end{center}
%

%%%%%%%%%%%%%%%%%%%%%%%%%%%%%%%%%%%%%%%%%%%%%%%%%%%%%%%%%%%%%%%%%%%%%%%%%%%%%%%%
%%%%%%%%%%%%%%%%%%%%%%%%%%%%%%%%%%%%%%%%%%%%%%%%%%%%%%%%%%%%%%%%%%%%%%%%%%%%%%%%
\section{Information}

%%%%%%%%%%%%%%%%%%%%%%%%%%%%%%%%%%%%%%%%%%%%%%%%%%%%%%%%%%%%%%%%%%%%%%%%%%%%%%%%
\subsection{Copyright}

Copyright \copyright{} 2017--2018 Niklas Beisert

This work may be distributed and/or modified under the
conditions of the \LaTeX{} Project Public License, either version 1.3
of this license or (at your option) any later version.
The latest version of this license is in
  \url{http://www.latex-project.org/lppl.txt}
and version 1.3 or later is part of all distributions of \LaTeX{}
version 2005/12/01 or later.

This work has the LPPL maintenance status `maintained'.

The Current Maintainer of this work is Niklas Beisert.

This work consists of the files |README.txt|, |childdoc.ins| and |childdoc.dtx|
as well as the derived files |childdoc.def|, |cdocsamp.tex|
with |cdocsch1.tex|, |cdocsch2.tex|, |cdocspt3.tex|, |cdocspt4.tex|,
|cdocsdrf.tex|, |cdocsfn1.tex|, |cdocsfn2.tex|
as well as |childdoc.pdf|.

%%%%%%%%%%%%%%%%%%%%%%%%%%%%%%%%%%%%%%%%%%%%%%%%%%%%%%%%%%%%%%%%%%%%%%%%%%%%%%%%
\subsection{Files and Installation}

The package consists of the files:
%
\begin{center}
\begin{tabular}{ll}
    |README.txt|   & readme file \\
    |childdoc.ins| & installation file \\
    |childdoc.dtx| & source file \\
    |childdoc.def| & definition file \\
    |cdocsamp.tex| & sample main file \\
    |cdocsch1.tex| & sample include file \\
    |cdocsch2.tex| & sample include file \\
    |cdocspt3.tex| & sample part file \\
    |cdocspt4.tex| & sample part file \\
    |cdocsdrf.tex| & sample redirection file \\
    |cdocsfn1.tex| & sample redirection file \\
    |cdocsfn2.tex| & sample redirection file \\
    |childdoc.pdf| & manual
\end{tabular}
\end{center}
%
The distribution consists of the files
|README.txt|, |childdoc.ins| and |childdoc.dtx|.
%
\begin{itemize}
\item
Run (pdf)\LaTeX{} on |childdoc.dtx|
to compile the manual |childdoc.pdf| (this file).
\item
Run \LaTeX{} on |childdoc.ins| to create the definitions file |childdoc.def|
and the sample |cdocsamp.tex| with include files
|cdocsch1.tex|, |cdocsch2.tex|, |cdocspt3.tex|, |cdocspt4.tex|,
|cdocsdrf.tex|, |cdocsfn1.tex|, |cdocsfn2.tex|.
Then copy the file |childdoc.def| to an appropriate directory of your \LaTeX{}
distribution, e.g.\ \textit{texmf-root}|/tex/latex/childdoc|.
\end{itemize}

%%%%%%%%%%%%%%%%%%%%%%%%%%%%%%%%%%%%%%%%%%%%%%%%%%%%%%%%%%%%%%%%%%%%%%%%%%%%%%%%
\subsection{Related CTAN Packages}

There are several other packages which offer a similar functionality:
%
\begin{itemize}
\item
The packages
\href{http://ctan.org/pkg/docmute}{\textsf{docmute}},
\href{http://ctan.org/pkg/includex}{\textsf{includex}} and
\href{http://ctan.org/pkg/standalone}{\textsf{standalone}}
provide commands to include only the document body of
a child file thus allowing both files to be compiled individually.
\item
The packages \href{http://ctan.org/pkg/subdocs}{\textsf{subdocs}}
and \href{http://ctan.org/pkg/subfiles}{\textsf{subfiles}}
provide structures in which the main and child documents can be
encapsulated and allowing them to be compiled individually.
The inclusion mechanism is different from the conventional |\include|.
\item
The package \href{http://ctan.org/pkg/combine}{\textsf{combine}}
is an elaborate solution to combine several documents into one.
\end{itemize}
%
See also the CTAN topic \href{http://ctan.org/topic/subdocs}{\textsf{subdocs}}
for further related packages.
The present package differs from the above solutions in that
a document structure constructed with the conventional |\include| mechanism
just needs two extra commands at the top of every file
such that all constituent files can be compiled individually.

%%%%%%%%%%%%%%%%%%%%%%%%%%%%%%%%%%%%%%%%%%%%%%%%%%%%%%%%%%%%%%%%%%%%%%%%%%%%%%%%
%\subsection{Feature Suggestions}
%
%The following is a list of features which may be useful for future
%versions of this package:
%%
%\begin{itemize}
%\item
%\ldots
%\end{itemize}

%%%%%%%%%%%%%%%%%%%%%%%%%%%%%%%%%%%%%%%%%%%%%%%%%%%%%%%%%%%%%%%%%%%%%%%%%%%%%%%%
\subsection{Revision History}

%%%%%%%%%%%%%%%%%%%%%%%%%%%%%%%%%%%%%%%%
\paragraph{v2.0:} 2018/12/30

\begin{itemize}
\item
immediate forward processing
\item
added |\childdocby| mechanism
\item
manual restructured
\end{itemize}

%%%%%%%%%%%%%%%%%%%%%%%%%%%%%%%%%%%%%%%%
\paragraph{v1.6:} 2018/01/17

\begin{itemize}
\item
application for development of include files
\item
corrections to manual
\end{itemize}

%%%%%%%%%%%%%%%%%%%%%%%%%%%%%%%%%%%%%%%%
\paragraph{v1.5:} 2017/05/21

\begin{itemize}
\item
more complete structuring introduced
\item
|\childdocof| introduced
\item
|\childdoc| renamed to |\childdocmain|
\item
|\childredirect| renamed to |\childdocforward| and |\childdocforwardprefix|
and functionality expanded
\end{itemize}

%%%%%%%%%%%%%%%%%%%%%%%%%%%%%%%%%%%%%%%%
\paragraph{v1.0:} 2017/04/27

\begin{itemize}
\item
manual and install package
\item
first version published on CTAN
\end{itemize}

%%%%%%%%%%%%%%%%%%%%%%%%%%%%%%%%%%%%%%%%
\paragraph{v0.6:} 2017/04/26

\begin{itemize}
\item
redirection mechanism added
\end{itemize}

%%%%%%%%%%%%%%%%%%%%%%%%%%%%%%%%%%%%%%%%
\paragraph{v0.5:} 2017/04/26

\begin{itemize}
\item
functionality in definition file
\end{itemize}


%%%%%%%%%%%%%%%%%%%%%%%%%%%%%%%%%%%%%%%%%%%%%%%%%%%%%%%%%%%%%%%%%%%%%%%%%%%%%%%%
%%%%%%%%%%%%%%%%%%%%%%%%%%%%%%%%%%%%%%%%%%%%%%%%%%%%%%%%%%%%%%%%%%%%%%%%%%%%%%%%
%%%%%%%%%%%%%%%%%%%%%%%%%%%%%%%%%%%%%%%%%%%%%%%%%%%%%%%%%%%%%%%%%%%%%%%%%%%%%%%%
\appendix

\settowidth\MacroIndent{\rmfamily\scriptsize 000\ }

 \DocInput{childdoc.dtx}

\end{document}
%</driver>
% \fi
%
% %%%%%%%%%%%%%%%%%%%%%%%%%%%%%%%%%%%%%%%%%%%%%%%%%%%%%%%%%%%%%%%%%%%%%%%%%%%%%%
% %%%%%%%%%%%%%%%%%%%%%%%%%%%%%%%%%%%%%%%%%%%%%%%%%%%%%%%%%%%%%%%%%%%%%%%%%%%%%%
% \section{Sample}
%\iffalse
%<*samplemain>
%\fi
%
% The following presents a sample document
% with two chapters, two parts, a title page,
% a compile flag as well as three forwarding files to set the flag.
% It consists of eight |.tex| files:
% \begin{center}
% \begin{tabular}{ll}
% |cdocsamp.tex|&main file\\
% |cdocsch1.tex|&include file for chapter 1\\
% |cdocsch2.tex|&include file for chapter 2\\
% |cdocspt3.tex|&include file for part 3\\
% |cdocspt4.tex|&include file for part 4\\
% |cdocsdrf.tex|&forwarding file for main file in draft mode\\
% |cdocsfi1.tex|&forwarding file for final version of chapter 1\\
% |cdocsfi2.tex|&forwarding file for final version of chapter 2\\
% \end{tabular}
% \end{center}
% Each of the eight files can be compiled directly by the \LaTeX{} compiler.
%
% %%%%%%%%%%%%%%%%%%%%%%%%%%%%%%%%%%%%%%
% \paragraph{Main File.}
%
% The main file is called |cdocsamp.tex|.
%
% Load the \textsf{childdoc} definitions and
% declare the filename for the main document:
%    \begin{macrocode}
\input{childdoc.def}
\childdocmain{}
%    \end{macrocode}

% Optional override for |\version| flag:
%    \begin{macrocode}
%%\ifchilddoc\else\providecommand{\version}{draft}\fi
%    \end{macrocode}

% Define the default values for the |\version| flag
% (|final| for the main file and |draft| for childs):
%    \begin{macrocode}
\ifchilddoc
\providecommand{\version}{draft}
\else
\providecommand{\version}{final}
\fi
%    \end{macrocode}

% Load the standard document class:
%    \begin{macrocode}
\documentclass[12pt]{article}
%    \end{macrocode}

% Start the document body:
%    \begin{macrocode}
\begin{document}
%    \end{macrocode}

% Declare a title page.
% Print title, part of document being processed and version flag:
%    \begin{macrocode}
\addtocounter{page}{-1}
\begin{center}
{\LARGE\bfseries{}childdoc example\par}
\vspace{1cm}
\ifchilddoc
\ifchilddocmanual part\else chapter\fi:
`\childdocname' of `\childdocjob'\par
\else
main document: `\childdocjob'\par
\fi
version: \version\par
\end{center}
\newpage
%    \end{macrocode}

% Manually include selected file,
% otherwise process as usual:
%    \begin{macrocode}
\ifchilddocmanual
\section*{part `\childdocname'}
\input{\childdocname}
\else
%    \end{macrocode}

% Include the two chapters:
%    \begin{macrocode}
\include{cdocsch1}
\include{cdocsch2}
%    \end{macrocode}

% Include the two parts unless only chapters should be displayed:
%    \begin{macrocode}
\ifchilddoc\else
\section{part three}
\input{cdocspt3}
\section{part four}
\input{cdocspt4}
\fi
%    \end{macrocode}

% Process as usual until here:
%    \begin{macrocode}
\fi
%    \end{macrocode}

% End of document body:
%    \begin{macrocode}
\end{document}
%    \end{macrocode}
%\iffalse
%</samplemain>
%\fi
%
% %%%%%%%%%%%%%%%%%%%%%%%%%%%%%%%%%%%%%%
% \paragraph{Chapter Include Files.}
%
% The include files are called |cdocsch1.tex| and |cdocsch2.tex|.
%
%\iffalse
%<*samplechap1|samplechap2>
%\fi

% Optional override for |\version| flag:
%    \begin{macrocode}
%%\providecommand{\version}{final}
%    \end{macrocode}

% Include the main document:
%    \begin{macrocode}
\input{childdoc.def}
\childdocof{cdocsamp}
%    \end{macrocode}

%\iffalse
%</samplechap1|samplechap2>
%\fi
%
%\iffalse
%<*samplechap1>
%\fi
% Some text for chapter 1:
%    \begin{macrocode}
\section{one}
some text in chapter one
%    \end{macrocode}

%\iffalse
%</samplechap1>
%\fi
% Some text for chapter 2:
%\iffalse
%<*samplechap2>
%\fi
%    \begin{macrocode}
\section{two}
more text in chapter two
%    \end{macrocode}

%\iffalse
%</samplechap2>
%\fi
%
% %%%%%%%%%%%%%%%%%%%%%%%%%%%%%%%%%%%%%%
% \paragraph{Part Include Files.}
%
% The include files are called |cdocspt3.tex| and |cdocspt4.tex|.
%
%\iffalse
%<*samplepart3|samplepart4>
%\fi

% Optional override for |\version| flag:
%    \begin{macrocode}
%%\providecommand{\version}{final}
%    \end{macrocode}

% Include the main document:
%    \begin{macrocode}
\input{childdoc.def}
\childdocby{cdocsamp}
%    \end{macrocode}

%\iffalse
%</samplepart3|samplepart4>
%\fi
%
%\iffalse
%<*samplepart3>
%\fi
% Some text for part 3:
%    \begin{macrocode}
some text in part three
%    \end{macrocode}

%\iffalse
%</samplepart3>
%\fi
% Some text for part 4:
%\iffalse
%<*samplepart4>
%\fi
%    \begin{macrocode}
more text in part four
%    \end{macrocode}

%\iffalse
%</samplepart4>
%\fi
%
% %%%%%%%%%%%%%%%%%%%%%%%%%%%%%%%%%%%%%%
% \paragraph{Forwarding for a Complete Draft.}
%
% The following forwarding file |cdocsdrf.tex|
% compiles the main document in draft mode:
%\iffalse
%<*sampledraft>
%\fi
%    \begin{macrocode}
\def\version{draft}
\input{childdoc.def}
\childdocforward{cdocsamp}
%    \end{macrocode}

%\iffalse
%</sampledraft>
%\fi
%
% %%%%%%%%%%%%%%%%%%%%%%%%%%%%%%%%%%%%%%
% \paragraph{Forwarding for Final Version of the Chapters.}
%
% The following forwarding files |cdocsfn1.tex| and |cdocsfn2.tex|
% (with identical content)
% compile the final versions of the child documents
% |cdocsch1.tex| and |cdocsch2.tex|, respectively:
%\iffalse
%<*samplefinal>
%\fi
%    \begin{macrocode}
\def\version{final}
\input{childdoc.def}
\childdocforwardprefix[cdocsamp]{cdocsfn}{cdocsch}
%    \end{macrocode}

%\iffalse
%</samplefinal>
%\fi
%
% %%%%%%%%%%%%%%%%%%%%%%%%%%%%%%%%%%%%%%
% \paragraph{Command Line Processing.}
%
% The following three command lines generate the output files
% |cdocscld|, |cdocscl1| and |cdocscl2|
% which should be identical to
% |cdocsdrf|, |cdocsch1| and |cdocsfn2|, respectively:
% \begin{center}
% \begin{tabular}{l}
% |latex -jobname cdocscld \|\\
% |  "\def\version{draft}\input{childdoc.def}\childdocforward{cdocsamp}"|\\
% |latex -jobname cdocscl1 \|\\
% |  "\input{childdoc.def}\childdocforward[cdocsamp]{cdocsch1}"|\\
% |latex -jobname cdocscl2 \|\\
% |  "\def\version{final}\input{childdoc.def}\childdocforward{cdocsch2}"|
% \end{tabular}
% \end{center}
% Note that the trailing backslash on each first line
% merely continues the input to the second line
% (for convenient cut ant paste).
% Furthermore, the command |latex| can be replaced by any
% of its alternative versions such as |pdflatex|.
%
% %%%%%%%%%%%%%%%%%%%%%%%%%%%%%%%%%%%%%%%%%%%%%%%%%%%%%%%%%%%%%%%%%%%%%%%%%%%%%%
% %%%%%%%%%%%%%%%%%%%%%%%%%%%%%%%%%%%%%%%%%%%%%%%%%%%%%%%%%%%%%%%%%%%%%%%%%%%%%%
% \section{Implementation}
%\iffalse
%<*package>
%\fi
%
% This section describes the definitions file |childdoc.def|.

% The definitions cannot be loaded using |\usepackage| or |\RequirePackage|
% which has a mechanism to prevent loading a style file more than once.
% When loading the definitions by means of |\input|
% multiple instances have to be prevented manually:
%\iffalse
%This code needs to be before the `\ProvidesFile' directive
%which is defined at the beginning of this file.
%Therefore it is also placed there and commented out here.
%</package>
%<*discard>
%\fi
%    \begin{macrocode}
\ifdefined\childdocmain\endinput\fi
%    \end{macrocode}
%\iffalse
%</discard>
%<*package>
%\fi
%
% \macro{\ifchilddoc}
% \macro{\ifchilddocmanual}
% The conditional |\ifchilddoc| tells whether a
% child (true) or main (false) document is being compiled.
% The conditional |\ifchilddocmanual| tells whether
% the |\includeonly| mechanism is used (false) or
% the selection of child files must be performed manually (true).
% The definitions initialise to false:
%    \begin{macrocode}
\newif\ifchilddoc
\newif\ifchilddocmanual
%    \end{macrocode}

% \macro{\childdocname}
% \macro{\childdocjob}
% The macro |\childdocname| stores the name of the main document
% to be compiled. The macro |\childdocjob| stores the name of
% the document on which the \LaTeX{} compiler was originally invoked.
% The content of |\jobname| cannot be compared
% to filenames specified in the source due to different catcodes.
% The following code rescans |\jobname|, stores the result
% in |\childdocname| and saves a copy in |\childdocjob|:
%    \begin{macrocode}
\edef\childdocname{\scantokens\expandafter{\jobname\noexpand}}
\let\childdocjob\childdocname
%    \end{macrocode}

% \macro{\childdocdisable}
% The macro |\childdocdisable| prevents the main file
% from being processed more than once.
% At this stage, the main document command |\childdocmain|
% is assumed to be called once again where it should do nothing.
% Any subsequent call to it should prevent
% a secondary processing of the main document
% It overwrites the forwarding commands
% |\childdocof| and |\childdocforward|
% with empty macros to prevent further inclusions of the main document:
%    \begin{macrocode}
\newcommand{\childdocdisable}
{
  \renewcommand{\childdocmain}[1]{\renewcommand{\childdocmain}[1]{\endinput}}
  \renewcommand{\childdocof}[1]{}
  \renewcommand{\childdocby}[2][]{}
  \renewcommand{\childdocforward}[2][]{}
  \renewcommand{\childdocdisable}{}
}
%    \end{macrocode}

% \macro{\childdocmain}
% The macro |\childdocmain| is to be called at the top of the main file
% with nothing or the main filename (without extension) as argument.
% First, it breaks loops.
% If the argument is not empty and does not match |\childdocname|
% (which is set by the first inclusion of |childdoc.def|),
% |\ifchilddoc| is set to true, |\includeonly| is applied to the child file
% and |\jobname| is set to the main file
% (for proper handling of |.aux| files):
%    \begin{macrocode}
\newcommand{\childdocmain}[1]
{
  \childdocdisable\childdocmain{}
  \if?#1?\else
    \begingroup
      \def\childdoctmp{#1}
      \ifx\childdoctmp\childdocname
        \def\childdoctmp{}
      \else
        \def\childdoctmp
        {
          \childdoctrue
          \includeonly{\childdocname}
          \def\childdocjob{#1}
          \def\jobname{#1}
        }
      \fi
      \expandafter
    \endgroup
    \childdoctmp
  \fi
}
%    \end{macrocode}

% \macro{\childdocof}
% The command |\childdocof| redirects
% compilation to the main file |#1|.
%    \begin{macrocode}
\newcommand{\childdocof}[1]
{
  \childdocdisable
  \childdoctrue
  \includeonly{\childdocname}
  \def\jobname{#1}
  \def\childdocjob{#1}
  \input{#1}
}
%    \end{macrocode}

% \macro{\childdocby}
% The command |\childdocby| ....
%    \begin{macrocode}
\newcommand{\childdocby}[2][]
{
  \childdocdisable
  \childdoctrue
  \childdocmanualtrue
  \if?#1?\else
    \def\jobname{#2}
  \fi
  \def\childdocjob{#2}
  \input{#2}
  \endinput
}
%    \end{macrocode}

% \macro{\childdocforward}
% The command |\childdocforward| redirects
% compilation to the main file or
% (if the optional argument is given) a child file.
% Parameters are set as if the main file
% or a child file starting with |\childdocof| was compiled.
% Then compilation is handed over to the main file:
%    \begin{macrocode}
\newcommand{\childdocforward}[2][]
{
  \begingroup
    \if?#1?
      \def\childdoctmp
      {
        \def\childdocname{#2}
        \def\childdocjob{#2}
        \def\jobname{#2}
        \input{#2}
        \endinput
      }
    \else
      \def\childdoctmp
      {
        \childdocdisable
        \def\childdocname{#2}
        \childdoctrue
        \includeonly{#2}
        \def\childdocjob{#1}
        \def\jobname{#1}
        \input{#1}
        \endinput
      }
    \fi
    \expandafter
  \endgroup
  \childdoctmp
}
%    \end{macrocode}

% \macro{\childdocforwardprefix}
% The command |\childdocforwardprefix| redirects
% compilation to the main or a child file by means of a pattern.
% The prefix |#1| in the current filename is replaced by |#2|
% and the suffix of the current filename is kept
% (it is assumed that the filename does not contain the substring `|~~~|'
% which is used as a delimiter).
% Compilation is handed over to the new file by |\childdocforward|:
%    \begin{macrocode}
\newcommand{\childdocforwardprefix}[3][]
{
  \begingroup
    \def\childdocextract #2##1~~~{\def\childdoctmp{\childdocforward[#1]{#3##1}}}
    \expandafter\childdocextract\childdocname~~~
    \expandafter
  \endgroup
  \childdoctmp
}
%    \end{macrocode}

% \macro{\childdoc}
% The deprecated macro |\childdoc| is a legacy version of |\childdocmain|:
%    \begin{macrocode}
\newcommand{\childdoc}{\childdocmain}
%    \end{macrocode}

% \macro{\childdocredirect}
% The deprecated macro |\childdocredirect| is a legacy version
% of |\childdocforward| and |\childdocforwardprefix|:
%    \begin{macrocode}
\newcommand{\childdocredirect}[2][]
{
  \begingroup
    \if?#1?
      \def\childdoctmp{\childdocforward{#2}}
    \else
      \def\childdoctmp{\childdocforwardprefix{#1}{#2}}
    \fi
    \expandafter
  \endgroup
  \childdoctmp
}
%    \end{macrocode}

%\iffalse
%</package>
%\fi
%
\endinput
|\\
|\childdocmain{}|\\
\end{tabular}
\end{center}
at the very top of the main \LaTeX{} file,
in particular \emph{before} the |\documentclass| statement!
The argument of |\childdocmain| should be left empty
(but it must be present).

%%%%%%%%%%%%%%%%%%%%%%%%%%%%%%%%%%%%%%%%
\DescribeMacro{\childdocof}
Furthermore, add the commands
\begin{center}
\begin{tabular}{l}
|% \iffalse
%
% childdoc.dtx Copyright (C) 2017-2018 Niklas Beisert
%
% This work may be distributed and/or modified under the
% conditions of the LaTeX Project Public License, either version 1.3
% of this license or (at your option) any later version.
% The latest version of this license is in
%   http://www.latex-project.org/lppl.txt
% and version 1.3 or later is part of all distributions of LaTeX
% version 2005/12/01 or later.
%
% This work has the LPPL maintenance status `maintained'.
%
% The Current Maintainer of this work is Niklas Beisert.
%
% This work consists of the files childdoc.dtx and childdoc.ins
% and the derived files childdoc.def and cdocsamp.tex with
% cdocsch1.tex, cdocsch2.tex, cdocsdrf.tex, cdocsfn1.tex, cdocsfn2.tex.
%
%<package>\ifdefined\childdocmain\endinput\fi
%<package>\ProvidesFile{childdoc.def}[2018/12/30 v2.0 child document driver]
%<samplemain>\ProvidesFile{cdocsamp.tex}[2018/12/30 v2.0 sample for childdoc]
%<*driver>
%\ProvidesFile{childdoc.drv}[2018/12/30 v2.0 childdoc reference manual file]
\PassOptionsToClass{10pt,a4paper}{article}
\documentclass{ltxdoc}

\usepackage[margin=35mm]{geometry}
\usepackage{hyperref}
\usepackage{hyperxmp}
\usepackage[usenames]{color}

\hypersetup{colorlinks=true}
\hypersetup{pdfstartview=FitH}
\hypersetup{pdfpagemode=UseNone}
\hypersetup{pdfsource={}}
\hypersetup{pdflang={en-UK}}
\hypersetup{pdfcopyright={Copyright 2017-2018 Niklas Beisert.
  This work may be distributed and/or modified under the
  conditions of the LaTeX Project Public License, either version 1.3
  of this license or (at your option) any later version.}}
\hypersetup{pdflicenseurl={http://www.latex-project.org/lppl.txt}}
\hypersetup{pdfcontactaddress={ETH Zurich, ITP, HIT K,
  Wolfgang-Pauli-Strasse 27}}
\hypersetup{pdfcontactpostcode={8093}}
\hypersetup{pdfcontactcity={Zurich}}
\hypersetup{pdfcontactcountry={Switzerland}}
\hypersetup{pdfcontactemail={nbeisert@itp.phys.ethz.ch}}
\hypersetup{pdfcontacturl={http://people.phys.ethz.ch/\xmptilde nbeisert/}}

\newcommand{\secref}[1]{\hyperref[#1]{section \ref*{#1}}}

\parskip1ex
\parindent0pt
\let\olditemize\itemize
\def\itemize{\olditemize\parskip0pt}

\begin{document}

\title{The \textsf{childdoc} Package}
\hypersetup{pdftitle={The childdoc Package}}
\author{Niklas Beisert\\[2ex]
  Institut f\"ur Theoretische Physik\\
  Eidgen\"ossische Technische Hochschule Z\"urich\\
  Wolfgang-Pauli-Strasse 27, 8093 Z\"urich, Switzerland\\[1ex]
  \href{mailto:nbeisert@itp.phys.ethz.ch}
  {\texttt{nbeisert@itp.phys.ethz.ch}}}
\hypersetup{pdfauthor={Niklas Beisert}}
\hypersetup{pdfsubject={Manual for the LaTeX2e Package childdoc}}
\date{30 December 2018, \textsf{v2.0}}
\maketitle

\begin{abstract}\noindent
\textsf{childdoc} is a \LaTeXe{} package
that enables the direct compilation
of document sections included by |\include|
to individual files.
\end{abstract}

\begingroup
\parskip0ex
\tableofcontents
\endgroup

%%%%%%%%%%%%%%%%%%%%%%%%%%%%%%%%%%%%%%%%%%%%%%%%%%%%%%%%%%%%%%%%%%%%%%%%%%%%%%%%
%%%%%%%%%%%%%%%%%%%%%%%%%%%%%%%%%%%%%%%%%%%%%%%%%%%%%%%%%%%%%%%%%%%%%%%%%%%%%%%%
\section{Introduction}

\LaTeX{} provides a mechanism to structure a large document (such as a book)
into a main file and several child files (containing the chapters)
using the |\include| command.
This mechanism is beneficial for documents
which span hundreds of pages in order to
make the source file(s) more manageable.
Moreover, compilation can be restricted to
selected child files by means of the |\includeonly| command.
The latter feature can be used to reduce the compilation time while editing
(this was significantly more useful in the earlier days of \LaTeX{})
or to generate a smaller document which is easier to navigate.
Another application of |\includeonly| is to generate
documents consisting of selected parts of the complete document.

However, there are a few drawbacks of the plain |\include| mechanism:
\begin{itemize}
\item
The child files cannot be compiled on their own,
they can only be compiled via the main file.
A naive editing environment
(such as a text editor with an option
to have the current file processed by \LaTeX)
may require one to switch to the main file before compiling;
attempting to compile the child file produces errors.
\item
The main file must be modified (each time)
to adjust the |\includeonly| command
to the present needs. This easily leaves the main file in a messy state.
\item
The generated document will always carry the filename
of the main document. This is inconvenient if
several child files are to be compiled and
to be kept for distribution.
\end{itemize}

The present package provides a simple interface
to make child files individually compilable by \LaTeX{}.
Compiling a child file then has the same effect as compiling
the main file with an |\includeonly| command
to select the appropriate child.
Moreover the generated document will carry the name of the child
rather than the main file.
This resolves all three above issues.

This feature is meant to make the editing of books,
thesis documents and lecture notes somewhat more convenient.
However, the package can also be used efficiently for
composing a series of documents (such as exercise sheets)
which are typically distributed individually.
It then assists the author in generating the individual documents
(potentially in different versions)
as well as a document containing the collected series.
Another application is in developing style files
or other kinds of included material
where compilation of the style file could redirect
to a sample or test file.

%%%%%%%%%%%%%%%%%%%%%%%%%%%%%%%%%%%%%%%%%%%%%%%%%%%%%%%%%%%%%%%%%%%%%%%%%%%%%%%%
%%%%%%%%%%%%%%%%%%%%%%%%%%%%%%%%%%%%%%%%%%%%%%%%%%%%%%%%%%%%%%%%%%%%%%%%%%%%%%%%
\section{Usage}

First of all, the package \textsf{childdoc} is \emph{not} a standard
\LaTeXe{} |.sty| style file! Therefore it needs to be invoked in
a non-standard way.

%%%%%%%%%%%%%%%%%%%%%%%%%%%%%%%%%%%%%%%%%%%%%%%%%%%%%%%%%%%%%%%%%%%%%%%%%%%%%%%%
\subsection{Included Files}
\label{sec:include}

%%%%%%%%%%%%%%%%%%%%%%%%%%%%%%%%%%%%%%%%
\DescribeMacro{\childdocmain}
To use the package, add the commands
\begin{center}
\begin{tabular}{l}
|\input{childdoc.def}|\\
|\childdocmain{}|\\
\end{tabular}
\end{center}
at the very top of the main \LaTeX{} file,
in particular \emph{before} the |\documentclass| statement!
The argument of |\childdocmain| should be left empty
(but it must be present).

%%%%%%%%%%%%%%%%%%%%%%%%%%%%%%%%%%%%%%%%
\DescribeMacro{\childdocof}
Furthermore, add the commands
\begin{center}
\begin{tabular}{l}
|\input{childdoc.def}|\\
|\childdocof{|\textit{main}|}|\\
\end{tabular}
\end{center}
at the top of every child file \textit{child}
which is included by |\include{|\textit{child}|}|
from within the main file
(or at least for those files to be compiled individually).
The argument \textit{main} must be the filename of the main file.

There are a couple of
considerations in setting up the main and child documents:

%%%%%%%%%%%%%%%%%%%%%%%%%%%%%%%%%%%%%%%%
\paragraph{Restrictions.}

Please note the following restrictions:
\begin{itemize}
\item
|\childdocmain| must be called with one argument \textit{main}
to ensure compatibility with earlier version of the package.
It must either be empty (|\childdocmain{}|)
or precisely match the filename of the main file in which it is specified.
See \secref{sec:detection} for further information.
\item
The filename \textit{main} must be specified without the |.tex| extension.
\item
The filename \textit{main} is case sensitive
(even in case-insensitive file systems)
due to internal string comparison.
\item
The argument \textit{main} should be fully expanded, it cannot be a macro.
\item
Subdirectories and special characters should be avoided in filenames.
\item
The command |\childdocmain{|\textit{main}|}| must be followed by a whitespace.
It should not be followed immediately by another command
or by a comment mark `|%|'.
This is because the \TeX{} parser reads the token immediately following
the argument of |\childdocmain| and puts it
at the beginning of every child section;
however, a white\-space is ignored.
\end{itemize}

%%%%%%%%%%%%%%%%%%%%%%%%%%%%%%%%%%%%%%%%
\paragraph{Content of Main File.}

It is advisable to place all content in the child files included by |\include|.
Any output contained in the main file will appear in all child documents
unless suppressed manually;
it cannot be suppressed automatically by the |\includeonly| directive
and thus should normally be avoided.
A method to include some content in the main file
by means of conditional processing is described in \secref{sec:conditional}.

%%%%%%%%%%%%%%%%%%%%%%%%%%%%%%%%%%%%%%%%
\paragraph{Page Numbering.}

When only a part of the document is compiled,
the appropriate numbering of pages
(as well as other status parameters)
is determined from the |.aux| files.
The latter contain information from previous passes.
However this information needs to propagate through
all intermediate child documents.
Therefore the page numbering in child documents may well
be inconsistent until the complete document is compiled at least once.

A useful (if unconventional) way to always ensure a consistent
page numbering is to restart the numbering in each child document
and denote the pages by `\textit{child}|.|\textit{page}'
where \textit{child} represents the chapter/section number of the child file.
This can be achieved by the command
|\numberwithin{page}{|\textit{child}|}|
of the \textsf{amsmath} package
where \textit{child} can be |chapter| or |section|
depending on the chosen structuring.
Alternatively, one can modify the macro |\thepage| appropriately
and reset the counter |page| at the start of each child file.

%%%%%%%%%%%%%%%%%%%%%%%%%%%%%%%%%%%%%%%%%%%%%%%%%%%%%%%%%%%%%%%%%%%%%%%%%%%%%%%%
\subsection{Conditional Processing}
\label{sec:conditional}

The package provides a mechanism to compile different versions
of a document. To customise the versions further some conditional processing
can come in handy to distinguish which version is being compiled.
The package provides two macros to describe the compilation context:

%%%%%%%%%%%%%%%%%%%%%%%%%%%%%%%%%%%%%%%%
\DescribeMacro{\ifchilddoc}
The conditional |\ifchilddoc| distinguishes between the compilation of
child documents and the main document:
%
\begin{center}
|\ifchilddoc |\textit{child-code}| |[|\||else |\textit{main-code}]| \||fi|
\end{center}

%%%%%%%%%%%%%%%%%%%%%%%%%%%%%%%%%%%%%%%%
\DescribeMacro{\childdocname}
\DescribeMacro{\childdocjob}
The macro |\childdocname| contains the filename (without extension)
of the main or child file being processed.
Note that |\childdocjob| will always contain the name of the main file.

%%%%%%%%%%%%%%%%%%%%%%%%%%%%%%%%%%%%%%%%
\paragraph{Title Page.}

Conditional processing can be used to include a title or banner page
in the main document when proper precautions are taken.
Importantly, the code in the main file should ensure that the page counter
(as well as other status parameters which are stored in the |.aux| files)
takes the same value after the conditional processing.
Otherwise the page numbers may take divergent values
depending on which part is compiled.

For example, a title page could be declared by:
%
\begin{center}
\begin{tabular}{l}
|\ifchilddoc\||else|\\
|\addtocounter{page}{-1}|\\
\textit{code for title page}\\
|\newpage|\\
|\||fi|
\end{tabular}
\end{center}
%
A banner page for the child documents can be generated by:
%
\begin{center}
\begin{tabular}{l}
|\ifchilddoc|\\
|\addtocounter{page}{-1}|\\
\textit{code for banner page}\\
|\newpage|\\
|\||fi|
\end{tabular}
\end{center}
%
Here one could write a message such as:
\begin{center}
|This is the part \childdocname{} of \childdocjob{}.|
\end{center}

%%%%%%%%%%%%%%%%%%%%%%%%%%%%%%%%%%%%%%%%%%%%%%%%%%%%%%%%%%%%%%%%%%%%%%%%%%%%%%%%
\subsection{Flags}
\label{sec:flags}

The package makes it easy to generate different versions
of the main or child documents.
To this end compilation flags can be defined
and assigned different default values.
They will be particularly useful in conjunction
with the forwarding mechanism described in \secref{sec:forward}.

For example, it may be useful to have a flag |\version|
which can be set to |draft| or |final|.
The document source will contain some conditional code
depending on the value of |\version|.
Suppose further, the flag should default to |final| for the main file
and to |draft| for child files
which is a natural assignment for editing the document.
This is achieved by placing the following code
in the preamble of the main document
(below the |\childdocmain| directive):
%
\begin{center}
\begin{tabular}{l}
|\ifchilddoc|\\
|\providecommand{\version}{draft}|\\
|\||else|\\
|\providecommand{\version}{final}|\\
|\||fi|
\end{tabular}
\end{center}
%
The definition by |\providecommand| makes sure
that previous definitions are not overwritten.
Further statements |\providecommand{\version}{...}|
can thus be added before the above code to override it.

For the main file, one might add a line
(between |\childdocmain| and the above block)
%
\begin{center}
|%\ifchilddoc\||else\providecommand{\version}{draft}\||fi|
\end{center}
%
which can be uncommented to produce a draft version.
Likewise one can add a line to the very top of a child file
(above the |\childdocof{|\textit{main}|}| directive)
%
\begin{center}
|%\providecommand{\version}{final}|
\end{center}
%
which can be uncommented to produce the final version of this child document.

%%%%%%%%%%%%%%%%%%%%%%%%%%%%%%%%%%%%%%%%%%%%%%%%%%%%%%%%%%%%%%%%%%%%%%%%%%%%%%%%
\subsection{Forwarding}
\label{sec:forward}

Different versions of the main or child documents
using compilation flags as described in \secref{sec:flags}
can be (permanently) stored in different files
for convenient compilation, viewing and distribution.
To this end, the package defines a command
to pass on compilation to a different file:

%%%%%%%%%%%%%%%%%%%%%%%%%%%%%%%%%%%%%%%%
\DescribeMacro{\childdocforward}
The command |\childdocforward| redirects processing to
another source file:
%
\begin{center}
\begin{tabular}{l}
|\input{childdoc.def}|\\
|\childdocforward[|\textit{main}|]{|\textit{dest}|}|\\
\end{tabular}
\end{center}
%
The argument \textit{dest} is the destination file
(without extension).
It should be the main file or one of the child files.
Note that further \textsf{childdoc} directives
such as |\childdocof| and |\childdocforward|
in the indicated file will be processed in this form.
The optional argument \textit{main}
passes on directly to the main file \textit{main}
while pretending to compile the child \textit{dest}.
This form behaves as if \textit{dest}
issues |\childdocof{|\textit{main}|}| right away,
and no further \textsf{childdoc} directives will be processed.

%%%%%%%%%%%%%%%%%%%%%%%%%%%%%%%%%%%%%%%%
\DescribeMacro{\...prefix}
In the alternative form |\childdocforwardprefix|,
%
\begin{center}
\begin{tabular}{l}
|\input{childdoc.def}|\\
|\childdocforwardprefix[|\textit{main}|]{|\textit{prefix}|}{|\textit{dest}|}|
\end{tabular}
\end{center}
%
the destination file is determined by a pattern
depending on the current file:
To make this work, the current file must be called
`{\textit{prefix}\hspace{0.2em}\textit{suffix}}'
with \textit{prefix} matching precisely the argument.
Processing is then passed on to the file
`{\textit{dest}\hspace{0.2em}\textit{suffix}}'.
Surely, the same effect is achieved by
directly specifying the
argument `{\textit{dest}\hspace{0.2em}\textit{suffix}}'
in the first form.
However, that requires to set up a different file
for each child. With the alternative form of the command
all these files can have exactly the same content
which simplifies setting them up and maintaining them.

For example, the following file |draft.tex|
with a compilation flag |\version| as described in \secref{sec:flags}
compiles the main document as a draft:
%
\begin{center}
\begin{tabular}{l}
|\def\version{draft}|\\
|\input{childdoc.def}|\\
|\childdocforward{|\textit{main}|}|
\end{tabular}
\end{center}
%
Likewise, the following files |final|\textit{nn}|.tex|
compile the final version of the child document
|child|\textit{nn}|.tex|:
%
\begin{center}
\begin{tabular}{l}
|\def\version{final}|\\
|\input{childdoc.def}|\\
|\childdocforwardprefix{final}{child}|
\end{tabular}
\end{center}
%

Note that when several versions of a main file and/or of each child file
are to be generated, it may be convenient to set up a |Makefile| or
shell script to automatise the process.

%%%%%%%%%%%%%%%%%%%%%%%%%%%%%%%%%%%%%%%%%%%%%%%%%%%%%%%%%%%%%%%%%%%%%%%%%%%%%%%%
\subsection{Command Line Processing}
\label{sec:commandline}

The effect of redirection files can also be achieved by invoking
the \LaTeX{} compiler with a more elaborate command line.
Most conveniently this should be done as part
of a shell script or a |Makefile|.

When using \textsf{childdoc} in the main file, the following
command lines effectively perform a redirection
(note that depending on the shell being used,
backslashes may have to be doubled: `|\|' $\to$ `|\\|'):
%
\begin{center}
|... -jobname "|\textit{target}|" |\\|"|[\textit{flags}]%
|\input{childdoc.def}\childdocforward[|\textit{main}|]{|\textit{dest}|}"|
\end{center}
%
Here \textit{target} is the name of the output file,
\textit{main} is the name of the main file
and \textit{dest} is the name of the main or child file to be processed
(all filenames without extensions).
The optional argument \textit{main} can be omitted
if \textit{main} matches \textit{dest}.
Optionally, compilation \textit{flags} can be defined via |\def| commands.
This command line makes the \TeX{} engine believe
it is compiling the file \textit{target}
whose content is specified as the latter parameter.
The provided code then forwards the processing to
\textit{main} or \textit{dest} as described in \secref{sec:forward}.

%%%%%%%%%%%%%%%%%%%%%%%%%%%%%%%%%%%%%%%%%%%%%%%%%%%%%%%%%%%%%%%%%%%%%%%%%%%%%%%%
\subsection{Include by Input}
\label{sec:input}

Including child documents by |\include| has some restrictions by design.
Most notably, the content of a child document always occupies
its own set of pages; pages cannot be shared between child documents.
Usually, this behaviour makes perfect sense
because each child document contain an essential part of the document.
However, in some situations it may be desirable to compose
a document from a collection of parts
without having mandatory page breaks between then.
For this case, the package
provides a mechanism to include parts
by |\input| which can also be processed individually.
However, by construction this mechanism
requires manual handling of the content to be output.

%%%%%%%%%%%%%%%%%%%%%%%%%%%%%%%%%%%%%%%%
\DescribeMacro{\ifchilddocmanual}
The main file should be prepared as usual, see \secref{sec:include}.
However, the document body must make a distinction
between processing of an individual part and of the main document, e.g.:
%
\begin{center}
\begin{tabular}{l}
|\ifchilddocmanual|\\
|\input{\childdocname}|\\
|\||else|\\
\textit{document body with }|\input{|\textit{part}|}|\\
|\||fi|
\end{tabular}
\end{center}
%
The conditional |\ifchilddocmanual| is true whenever
a part to be included by |\input| is being compiled,
and the name of the part is stored in |\childdocname|.

%%%%%%%%%%%%%%%%%%%%%%%%%%%%%%%%%%%%%%%%
\DescribeMacro{\childdocby}
Each part to be included by |\input| should start with:
%
\begin{center}
\begin{tabular}{l}
|\input{childdoc.def}|\\
|\childdocby{|\textit{main}|}|\\
\end{tabular}
\end{center}
%
The directive |\childdocby| is similar to |\childdocof|
described in \secref{sec:include},
but the subsequent selection of content must be done manually.
To that end, both |\ifchilddoc| and |\ifchilddocmanual|
will be true upon processing of a part,
and the name of the part is stored in |\childdocname|.
Note that |\jobname| will be set to the filename of the current part
so that each part receives an individual |.aux| file
that does not interfere with the |.aux| file(s) of the main document.
This behaviour can be altered by the alternative form
|\childdocby[*]{|\textit{main}|}| (with a non-empty optional argument)
which uses the |.aux| file of the main document
by setting |\jobname| to \textit{main}.

%%%%%%%%%%%%%%%%%%%%%%%%%%%%%%%%%%%%%%%%%%%%%%%%%%%%%%%%%%%%%%%%%%%%%%%%%%%%%%%%
\subsection{Driver Development}
\label{sec:driver}

The \textsf{childdoc} mechanism can also be use for the development
of definition files such as \LaTeX{} styles or classes.
This case differs from the above setup with multiple parts
included by |\include| in that no |\includeonly| should be invoked.
This can be achieved by starting the include file
(before |\ProvidesPackage|) with:
%
\begin{center}
\begin{tabular}{l}
|\input{childdoc.def}|\\
|\childdocforward{|\textit{main}|}|\\
\end{tabular}
\end{center}
%
or alternatively with:
%
\begin{center}
\begin{tabular}{l}
|\input{childdoc.def}|\\
|\childdocby{|\textit{main}|}|\\
\end{tabular}
\end{center}
%
Both forms have slightly different effects as described above.
The main file is prepared as usual, see \secref{sec:include}.

%%%%%%%%%%%%%%%%%%%%%%%%%%%%%%%%%%%%%%%%%%%%%%%%%%%%%%%%%%%%%%%%%%%%%%%%%%%%%%%%
\subsection{Legacy Detection}
\label{sec:detection}

The directive |\childdocmain| in the main file can detect
whether the complete document or merely a child is to be compiled
even without using the directive |\childdocof|.
This method is deprecated because it is less robust
and there is no compelling reason to use it;
it is merely provided for backward compatibility
and it may be removed in future versions.

If the detection mechanism is to be used,
it is mandatory to correctly specify
the filename of the main file as the argument of |\childdocmain|:
%
\begin{center}
\begin{tabular}{l}
|\input{childdoc.def}|\\
|\childdocmain{|\textit{main}|}|\\
\end{tabular}
\end{center}
%
If |\jobname| does not match the argument \textit{main} of |\childdocmain|,
it is assumed that |\jobname| points to the child file to be compiled.
When using |\childdocmain| with the main file specified as argument,
it suffices to start a child file
with just |\input{|\textit{main}|}|
without loading of the package and using |\childdocof|.
If instead all processing is done
with the appropriate \textsf{childdoc} directives,
the argument of \textit{main} of |\childdocmain| can be empty.

An alternative version of the command line processing described
in \secref{sec:commandline} using the detection mechanism reads:
%
\begin{center}
|... -jobname "|\textit{target}|" "|[\textit{flags}]%
[|\def\jobname{|\textit{dest}|}|]|\input{|\textit{main}|}"|
\end{center}

%%%%%%%%%%%%%%%%%%%%%%%%%%%%%%%%%%%%%%%%%%%%%%%%%%%%%%%%%%%%%%%%%%%%%%%%%%%%%%%%
\subsection{Manual Code}
\label{sec:manual}

In case one cannot be certain whether the definitions file |childdoc.def|
is installed on the target \TeX{} distribution
and one prefers not to ship it,
it is conceivable to paste a few relevant commands into the sources.

To that end, drop all statements |\input{childdoc.def}|
and perform the replacements as outlined below.
Instead of |\childdocmain{|\textit{main}|}| add the following code
to the top of the main file:
%
\begin{center}
\begin{tabular}{l}
|\||ifdefined\childdocname\endinput\||fi\newif\ifchilddoc|\\
|\edef\childdocname{\scantokens\expandafter{\jobname\noexpand}}|\\
|\def\childdocmain{|\textit{main}|}\||ifx\childdocmain\childdocname\||else|\\
|\childdoctrue\includeonly{\childdocname}\let\jobname\childdocmain\||fi|\\
\end{tabular}
\end{center}
%
Instead of |\childdocof{|\textit{main}|}| just include the main file
at the top of each child file:
%
\begin{center}
|\input{|\textit{main}|}|
\end{center}
%
A simple redirection |\childdocforward{|\textit{dest}|}| is achieved by:
%
\begin{center}
|\def\jobname{|\textit{dest}|}\input{\jobname}|
\end{center}
%
The redirection with prefix
|\childdocforwardprefix[|\textit{prefix}|]{|\textit{dest}|}|
is accomplished by:
%
\begin{center}
\begin{tabular}{l}
|{\edef\jobname{\scantokens\expandafter{\jobname\noexpand}}|\\
|\def\redirectjob |\textit{prefix}|#1~~~{\gdef\jobname{|\textit{dest}|#1}}|\\
|\expandafter\redirectjob\jobname~~~}\input{\jobname}|
\end{tabular}
\end{center}

In an alternative approach,
child documents can be compiled by a specific command line
without additional code or specific definitions:
%
\begin{center}
|... -jobname "|\textit{target}|" "|[\textit{flags}]%
|\includeonly{|\textit{dest}|}\input{|\textit{main}|}"|
\end{center}
%

%%%%%%%%%%%%%%%%%%%%%%%%%%%%%%%%%%%%%%%%%%%%%%%%%%%%%%%%%%%%%%%%%%%%%%%%%%%%%%%%
%%%%%%%%%%%%%%%%%%%%%%%%%%%%%%%%%%%%%%%%%%%%%%%%%%%%%%%%%%%%%%%%%%%%%%%%%%%%%%%%
\section{Information}

%%%%%%%%%%%%%%%%%%%%%%%%%%%%%%%%%%%%%%%%%%%%%%%%%%%%%%%%%%%%%%%%%%%%%%%%%%%%%%%%
\subsection{Copyright}

Copyright \copyright{} 2017--2018 Niklas Beisert

This work may be distributed and/or modified under the
conditions of the \LaTeX{} Project Public License, either version 1.3
of this license or (at your option) any later version.
The latest version of this license is in
  \url{http://www.latex-project.org/lppl.txt}
and version 1.3 or later is part of all distributions of \LaTeX{}
version 2005/12/01 or later.

This work has the LPPL maintenance status `maintained'.

The Current Maintainer of this work is Niklas Beisert.

This work consists of the files |README.txt|, |childdoc.ins| and |childdoc.dtx|
as well as the derived files |childdoc.def|, |cdocsamp.tex|
with |cdocsch1.tex|, |cdocsch2.tex|, |cdocspt3.tex|, |cdocspt4.tex|,
|cdocsdrf.tex|, |cdocsfn1.tex|, |cdocsfn2.tex|
as well as |childdoc.pdf|.

%%%%%%%%%%%%%%%%%%%%%%%%%%%%%%%%%%%%%%%%%%%%%%%%%%%%%%%%%%%%%%%%%%%%%%%%%%%%%%%%
\subsection{Files and Installation}

The package consists of the files:
%
\begin{center}
\begin{tabular}{ll}
    |README.txt|   & readme file \\
    |childdoc.ins| & installation file \\
    |childdoc.dtx| & source file \\
    |childdoc.def| & definition file \\
    |cdocsamp.tex| & sample main file \\
    |cdocsch1.tex| & sample include file \\
    |cdocsch2.tex| & sample include file \\
    |cdocspt3.tex| & sample part file \\
    |cdocspt4.tex| & sample part file \\
    |cdocsdrf.tex| & sample redirection file \\
    |cdocsfn1.tex| & sample redirection file \\
    |cdocsfn2.tex| & sample redirection file \\
    |childdoc.pdf| & manual
\end{tabular}
\end{center}
%
The distribution consists of the files
|README.txt|, |childdoc.ins| and |childdoc.dtx|.
%
\begin{itemize}
\item
Run (pdf)\LaTeX{} on |childdoc.dtx|
to compile the manual |childdoc.pdf| (this file).
\item
Run \LaTeX{} on |childdoc.ins| to create the definitions file |childdoc.def|
and the sample |cdocsamp.tex| with include files
|cdocsch1.tex|, |cdocsch2.tex|, |cdocspt3.tex|, |cdocspt4.tex|,
|cdocsdrf.tex|, |cdocsfn1.tex|, |cdocsfn2.tex|.
Then copy the file |childdoc.def| to an appropriate directory of your \LaTeX{}
distribution, e.g.\ \textit{texmf-root}|/tex/latex/childdoc|.
\end{itemize}

%%%%%%%%%%%%%%%%%%%%%%%%%%%%%%%%%%%%%%%%%%%%%%%%%%%%%%%%%%%%%%%%%%%%%%%%%%%%%%%%
\subsection{Related CTAN Packages}

There are several other packages which offer a similar functionality:
%
\begin{itemize}
\item
The packages
\href{http://ctan.org/pkg/docmute}{\textsf{docmute}},
\href{http://ctan.org/pkg/includex}{\textsf{includex}} and
\href{http://ctan.org/pkg/standalone}{\textsf{standalone}}
provide commands to include only the document body of
a child file thus allowing both files to be compiled individually.
\item
The packages \href{http://ctan.org/pkg/subdocs}{\textsf{subdocs}}
and \href{http://ctan.org/pkg/subfiles}{\textsf{subfiles}}
provide structures in which the main and child documents can be
encapsulated and allowing them to be compiled individually.
The inclusion mechanism is different from the conventional |\include|.
\item
The package \href{http://ctan.org/pkg/combine}{\textsf{combine}}
is an elaborate solution to combine several documents into one.
\end{itemize}
%
See also the CTAN topic \href{http://ctan.org/topic/subdocs}{\textsf{subdocs}}
for further related packages.
The present package differs from the above solutions in that
a document structure constructed with the conventional |\include| mechanism
just needs two extra commands at the top of every file
such that all constituent files can be compiled individually.

%%%%%%%%%%%%%%%%%%%%%%%%%%%%%%%%%%%%%%%%%%%%%%%%%%%%%%%%%%%%%%%%%%%%%%%%%%%%%%%%
%\subsection{Feature Suggestions}
%
%The following is a list of features which may be useful for future
%versions of this package:
%%
%\begin{itemize}
%\item
%\ldots
%\end{itemize}

%%%%%%%%%%%%%%%%%%%%%%%%%%%%%%%%%%%%%%%%%%%%%%%%%%%%%%%%%%%%%%%%%%%%%%%%%%%%%%%%
\subsection{Revision History}

%%%%%%%%%%%%%%%%%%%%%%%%%%%%%%%%%%%%%%%%
\paragraph{v2.0:} 2018/12/30

\begin{itemize}
\item
immediate forward processing
\item
added |\childdocby| mechanism
\item
manual restructured
\end{itemize}

%%%%%%%%%%%%%%%%%%%%%%%%%%%%%%%%%%%%%%%%
\paragraph{v1.6:} 2018/01/17

\begin{itemize}
\item
application for development of include files
\item
corrections to manual
\end{itemize}

%%%%%%%%%%%%%%%%%%%%%%%%%%%%%%%%%%%%%%%%
\paragraph{v1.5:} 2017/05/21

\begin{itemize}
\item
more complete structuring introduced
\item
|\childdocof| introduced
\item
|\childdoc| renamed to |\childdocmain|
\item
|\childredirect| renamed to |\childdocforward| and |\childdocforwardprefix|
and functionality expanded
\end{itemize}

%%%%%%%%%%%%%%%%%%%%%%%%%%%%%%%%%%%%%%%%
\paragraph{v1.0:} 2017/04/27

\begin{itemize}
\item
manual and install package
\item
first version published on CTAN
\end{itemize}

%%%%%%%%%%%%%%%%%%%%%%%%%%%%%%%%%%%%%%%%
\paragraph{v0.6:} 2017/04/26

\begin{itemize}
\item
redirection mechanism added
\end{itemize}

%%%%%%%%%%%%%%%%%%%%%%%%%%%%%%%%%%%%%%%%
\paragraph{v0.5:} 2017/04/26

\begin{itemize}
\item
functionality in definition file
\end{itemize}


%%%%%%%%%%%%%%%%%%%%%%%%%%%%%%%%%%%%%%%%%%%%%%%%%%%%%%%%%%%%%%%%%%%%%%%%%%%%%%%%
%%%%%%%%%%%%%%%%%%%%%%%%%%%%%%%%%%%%%%%%%%%%%%%%%%%%%%%%%%%%%%%%%%%%%%%%%%%%%%%%
%%%%%%%%%%%%%%%%%%%%%%%%%%%%%%%%%%%%%%%%%%%%%%%%%%%%%%%%%%%%%%%%%%%%%%%%%%%%%%%%
\appendix

\settowidth\MacroIndent{\rmfamily\scriptsize 000\ }

 \DocInput{childdoc.dtx}

\end{document}
%</driver>
% \fi
%
% %%%%%%%%%%%%%%%%%%%%%%%%%%%%%%%%%%%%%%%%%%%%%%%%%%%%%%%%%%%%%%%%%%%%%%%%%%%%%%
% %%%%%%%%%%%%%%%%%%%%%%%%%%%%%%%%%%%%%%%%%%%%%%%%%%%%%%%%%%%%%%%%%%%%%%%%%%%%%%
% \section{Sample}
%\iffalse
%<*samplemain>
%\fi
%
% The following presents a sample document
% with two chapters, two parts, a title page,
% a compile flag as well as three forwarding files to set the flag.
% It consists of eight |.tex| files:
% \begin{center}
% \begin{tabular}{ll}
% |cdocsamp.tex|&main file\\
% |cdocsch1.tex|&include file for chapter 1\\
% |cdocsch2.tex|&include file for chapter 2\\
% |cdocspt3.tex|&include file for part 3\\
% |cdocspt4.tex|&include file for part 4\\
% |cdocsdrf.tex|&forwarding file for main file in draft mode\\
% |cdocsfi1.tex|&forwarding file for final version of chapter 1\\
% |cdocsfi2.tex|&forwarding file for final version of chapter 2\\
% \end{tabular}
% \end{center}
% Each of the eight files can be compiled directly by the \LaTeX{} compiler.
%
% %%%%%%%%%%%%%%%%%%%%%%%%%%%%%%%%%%%%%%
% \paragraph{Main File.}
%
% The main file is called |cdocsamp.tex|.
%
% Load the \textsf{childdoc} definitions and
% declare the filename for the main document:
%    \begin{macrocode}
\input{childdoc.def}
\childdocmain{}
%    \end{macrocode}

% Optional override for |\version| flag:
%    \begin{macrocode}
%%\ifchilddoc\else\providecommand{\version}{draft}\fi
%    \end{macrocode}

% Define the default values for the |\version| flag
% (|final| for the main file and |draft| for childs):
%    \begin{macrocode}
\ifchilddoc
\providecommand{\version}{draft}
\else
\providecommand{\version}{final}
\fi
%    \end{macrocode}

% Load the standard document class:
%    \begin{macrocode}
\documentclass[12pt]{article}
%    \end{macrocode}

% Start the document body:
%    \begin{macrocode}
\begin{document}
%    \end{macrocode}

% Declare a title page.
% Print title, part of document being processed and version flag:
%    \begin{macrocode}
\addtocounter{page}{-1}
\begin{center}
{\LARGE\bfseries{}childdoc example\par}
\vspace{1cm}
\ifchilddoc
\ifchilddocmanual part\else chapter\fi:
`\childdocname' of `\childdocjob'\par
\else
main document: `\childdocjob'\par
\fi
version: \version\par
\end{center}
\newpage
%    \end{macrocode}

% Manually include selected file,
% otherwise process as usual:
%    \begin{macrocode}
\ifchilddocmanual
\section*{part `\childdocname'}
\input{\childdocname}
\else
%    \end{macrocode}

% Include the two chapters:
%    \begin{macrocode}
\include{cdocsch1}
\include{cdocsch2}
%    \end{macrocode}

% Include the two parts unless only chapters should be displayed:
%    \begin{macrocode}
\ifchilddoc\else
\section{part three}
\input{cdocspt3}
\section{part four}
\input{cdocspt4}
\fi
%    \end{macrocode}

% Process as usual until here:
%    \begin{macrocode}
\fi
%    \end{macrocode}

% End of document body:
%    \begin{macrocode}
\end{document}
%    \end{macrocode}
%\iffalse
%</samplemain>
%\fi
%
% %%%%%%%%%%%%%%%%%%%%%%%%%%%%%%%%%%%%%%
% \paragraph{Chapter Include Files.}
%
% The include files are called |cdocsch1.tex| and |cdocsch2.tex|.
%
%\iffalse
%<*samplechap1|samplechap2>
%\fi

% Optional override for |\version| flag:
%    \begin{macrocode}
%%\providecommand{\version}{final}
%    \end{macrocode}

% Include the main document:
%    \begin{macrocode}
\input{childdoc.def}
\childdocof{cdocsamp}
%    \end{macrocode}

%\iffalse
%</samplechap1|samplechap2>
%\fi
%
%\iffalse
%<*samplechap1>
%\fi
% Some text for chapter 1:
%    \begin{macrocode}
\section{one}
some text in chapter one
%    \end{macrocode}

%\iffalse
%</samplechap1>
%\fi
% Some text for chapter 2:
%\iffalse
%<*samplechap2>
%\fi
%    \begin{macrocode}
\section{two}
more text in chapter two
%    \end{macrocode}

%\iffalse
%</samplechap2>
%\fi
%
% %%%%%%%%%%%%%%%%%%%%%%%%%%%%%%%%%%%%%%
% \paragraph{Part Include Files.}
%
% The include files are called |cdocspt3.tex| and |cdocspt4.tex|.
%
%\iffalse
%<*samplepart3|samplepart4>
%\fi

% Optional override for |\version| flag:
%    \begin{macrocode}
%%\providecommand{\version}{final}
%    \end{macrocode}

% Include the main document:
%    \begin{macrocode}
\input{childdoc.def}
\childdocby{cdocsamp}
%    \end{macrocode}

%\iffalse
%</samplepart3|samplepart4>
%\fi
%
%\iffalse
%<*samplepart3>
%\fi
% Some text for part 3:
%    \begin{macrocode}
some text in part three
%    \end{macrocode}

%\iffalse
%</samplepart3>
%\fi
% Some text for part 4:
%\iffalse
%<*samplepart4>
%\fi
%    \begin{macrocode}
more text in part four
%    \end{macrocode}

%\iffalse
%</samplepart4>
%\fi
%
% %%%%%%%%%%%%%%%%%%%%%%%%%%%%%%%%%%%%%%
% \paragraph{Forwarding for a Complete Draft.}
%
% The following forwarding file |cdocsdrf.tex|
% compiles the main document in draft mode:
%\iffalse
%<*sampledraft>
%\fi
%    \begin{macrocode}
\def\version{draft}
\input{childdoc.def}
\childdocforward{cdocsamp}
%    \end{macrocode}

%\iffalse
%</sampledraft>
%\fi
%
% %%%%%%%%%%%%%%%%%%%%%%%%%%%%%%%%%%%%%%
% \paragraph{Forwarding for Final Version of the Chapters.}
%
% The following forwarding files |cdocsfn1.tex| and |cdocsfn2.tex|
% (with identical content)
% compile the final versions of the child documents
% |cdocsch1.tex| and |cdocsch2.tex|, respectively:
%\iffalse
%<*samplefinal>
%\fi
%    \begin{macrocode}
\def\version{final}
\input{childdoc.def}
\childdocforwardprefix[cdocsamp]{cdocsfn}{cdocsch}
%    \end{macrocode}

%\iffalse
%</samplefinal>
%\fi
%
% %%%%%%%%%%%%%%%%%%%%%%%%%%%%%%%%%%%%%%
% \paragraph{Command Line Processing.}
%
% The following three command lines generate the output files
% |cdocscld|, |cdocscl1| and |cdocscl2|
% which should be identical to
% |cdocsdrf|, |cdocsch1| and |cdocsfn2|, respectively:
% \begin{center}
% \begin{tabular}{l}
% |latex -jobname cdocscld \|\\
% |  "\def\version{draft}\input{childdoc.def}\childdocforward{cdocsamp}"|\\
% |latex -jobname cdocscl1 \|\\
% |  "\input{childdoc.def}\childdocforward[cdocsamp]{cdocsch1}"|\\
% |latex -jobname cdocscl2 \|\\
% |  "\def\version{final}\input{childdoc.def}\childdocforward{cdocsch2}"|
% \end{tabular}
% \end{center}
% Note that the trailing backslash on each first line
% merely continues the input to the second line
% (for convenient cut ant paste).
% Furthermore, the command |latex| can be replaced by any
% of its alternative versions such as |pdflatex|.
%
% %%%%%%%%%%%%%%%%%%%%%%%%%%%%%%%%%%%%%%%%%%%%%%%%%%%%%%%%%%%%%%%%%%%%%%%%%%%%%%
% %%%%%%%%%%%%%%%%%%%%%%%%%%%%%%%%%%%%%%%%%%%%%%%%%%%%%%%%%%%%%%%%%%%%%%%%%%%%%%
% \section{Implementation}
%\iffalse
%<*package>
%\fi
%
% This section describes the definitions file |childdoc.def|.

% The definitions cannot be loaded using |\usepackage| or |\RequirePackage|
% which has a mechanism to prevent loading a style file more than once.
% When loading the definitions by means of |\input|
% multiple instances have to be prevented manually:
%\iffalse
%This code needs to be before the `\ProvidesFile' directive
%which is defined at the beginning of this file.
%Therefore it is also placed there and commented out here.
%</package>
%<*discard>
%\fi
%    \begin{macrocode}
\ifdefined\childdocmain\endinput\fi
%    \end{macrocode}
%\iffalse
%</discard>
%<*package>
%\fi
%
% \macro{\ifchilddoc}
% \macro{\ifchilddocmanual}
% The conditional |\ifchilddoc| tells whether a
% child (true) or main (false) document is being compiled.
% The conditional |\ifchilddocmanual| tells whether
% the |\includeonly| mechanism is used (false) or
% the selection of child files must be performed manually (true).
% The definitions initialise to false:
%    \begin{macrocode}
\newif\ifchilddoc
\newif\ifchilddocmanual
%    \end{macrocode}

% \macro{\childdocname}
% \macro{\childdocjob}
% The macro |\childdocname| stores the name of the main document
% to be compiled. The macro |\childdocjob| stores the name of
% the document on which the \LaTeX{} compiler was originally invoked.
% The content of |\jobname| cannot be compared
% to filenames specified in the source due to different catcodes.
% The following code rescans |\jobname|, stores the result
% in |\childdocname| and saves a copy in |\childdocjob|:
%    \begin{macrocode}
\edef\childdocname{\scantokens\expandafter{\jobname\noexpand}}
\let\childdocjob\childdocname
%    \end{macrocode}

% \macro{\childdocdisable}
% The macro |\childdocdisable| prevents the main file
% from being processed more than once.
% At this stage, the main document command |\childdocmain|
% is assumed to be called once again where it should do nothing.
% Any subsequent call to it should prevent
% a secondary processing of the main document
% It overwrites the forwarding commands
% |\childdocof| and |\childdocforward|
% with empty macros to prevent further inclusions of the main document:
%    \begin{macrocode}
\newcommand{\childdocdisable}
{
  \renewcommand{\childdocmain}[1]{\renewcommand{\childdocmain}[1]{\endinput}}
  \renewcommand{\childdocof}[1]{}
  \renewcommand{\childdocby}[2][]{}
  \renewcommand{\childdocforward}[2][]{}
  \renewcommand{\childdocdisable}{}
}
%    \end{macrocode}

% \macro{\childdocmain}
% The macro |\childdocmain| is to be called at the top of the main file
% with nothing or the main filename (without extension) as argument.
% First, it breaks loops.
% If the argument is not empty and does not match |\childdocname|
% (which is set by the first inclusion of |childdoc.def|),
% |\ifchilddoc| is set to true, |\includeonly| is applied to the child file
% and |\jobname| is set to the main file
% (for proper handling of |.aux| files):
%    \begin{macrocode}
\newcommand{\childdocmain}[1]
{
  \childdocdisable\childdocmain{}
  \if?#1?\else
    \begingroup
      \def\childdoctmp{#1}
      \ifx\childdoctmp\childdocname
        \def\childdoctmp{}
      \else
        \def\childdoctmp
        {
          \childdoctrue
          \includeonly{\childdocname}
          \def\childdocjob{#1}
          \def\jobname{#1}
        }
      \fi
      \expandafter
    \endgroup
    \childdoctmp
  \fi
}
%    \end{macrocode}

% \macro{\childdocof}
% The command |\childdocof| redirects
% compilation to the main file |#1|.
%    \begin{macrocode}
\newcommand{\childdocof}[1]
{
  \childdocdisable
  \childdoctrue
  \includeonly{\childdocname}
  \def\jobname{#1}
  \def\childdocjob{#1}
  \input{#1}
}
%    \end{macrocode}

% \macro{\childdocby}
% The command |\childdocby| ....
%    \begin{macrocode}
\newcommand{\childdocby}[2][]
{
  \childdocdisable
  \childdoctrue
  \childdocmanualtrue
  \if?#1?\else
    \def\jobname{#2}
  \fi
  \def\childdocjob{#2}
  \input{#2}
  \endinput
}
%    \end{macrocode}

% \macro{\childdocforward}
% The command |\childdocforward| redirects
% compilation to the main file or
% (if the optional argument is given) a child file.
% Parameters are set as if the main file
% or a child file starting with |\childdocof| was compiled.
% Then compilation is handed over to the main file:
%    \begin{macrocode}
\newcommand{\childdocforward}[2][]
{
  \begingroup
    \if?#1?
      \def\childdoctmp
      {
        \def\childdocname{#2}
        \def\childdocjob{#2}
        \def\jobname{#2}
        \input{#2}
        \endinput
      }
    \else
      \def\childdoctmp
      {
        \childdocdisable
        \def\childdocname{#2}
        \childdoctrue
        \includeonly{#2}
        \def\childdocjob{#1}
        \def\jobname{#1}
        \input{#1}
        \endinput
      }
    \fi
    \expandafter
  \endgroup
  \childdoctmp
}
%    \end{macrocode}

% \macro{\childdocforwardprefix}
% The command |\childdocforwardprefix| redirects
% compilation to the main or a child file by means of a pattern.
% The prefix |#1| in the current filename is replaced by |#2|
% and the suffix of the current filename is kept
% (it is assumed that the filename does not contain the substring `|~~~|'
% which is used as a delimiter).
% Compilation is handed over to the new file by |\childdocforward|:
%    \begin{macrocode}
\newcommand{\childdocforwardprefix}[3][]
{
  \begingroup
    \def\childdocextract #2##1~~~{\def\childdoctmp{\childdocforward[#1]{#3##1}}}
    \expandafter\childdocextract\childdocname~~~
    \expandafter
  \endgroup
  \childdoctmp
}
%    \end{macrocode}

% \macro{\childdoc}
% The deprecated macro |\childdoc| is a legacy version of |\childdocmain|:
%    \begin{macrocode}
\newcommand{\childdoc}{\childdocmain}
%    \end{macrocode}

% \macro{\childdocredirect}
% The deprecated macro |\childdocredirect| is a legacy version
% of |\childdocforward| and |\childdocforwardprefix|:
%    \begin{macrocode}
\newcommand{\childdocredirect}[2][]
{
  \begingroup
    \if?#1?
      \def\childdoctmp{\childdocforward{#2}}
    \else
      \def\childdoctmp{\childdocforwardprefix{#1}{#2}}
    \fi
    \expandafter
  \endgroup
  \childdoctmp
}
%    \end{macrocode}

%\iffalse
%</package>
%\fi
%
\endinput
|\\
|\childdocof{|\textit{main}|}|\\
\end{tabular}
\end{center}
at the top of every child file \textit{child}
which is included by |\include{|\textit{child}|}|
from within the main file
(or at least for those files to be compiled individually).
The argument \textit{main} must be the filename of the main file.

There are a couple of
considerations in setting up the main and child documents:

%%%%%%%%%%%%%%%%%%%%%%%%%%%%%%%%%%%%%%%%
\paragraph{Restrictions.}

Please note the following restrictions:
\begin{itemize}
\item
|\childdocmain| must be called with one argument \textit{main}
to ensure compatibility with earlier version of the package.
It must either be empty (|\childdocmain{}|)
or precisely match the filename of the main file in which it is specified.
See \secref{sec:detection} for further information.
\item
The filename \textit{main} must be specified without the |.tex| extension.
\item
The filename \textit{main} is case sensitive
(even in case-insensitive file systems)
due to internal string comparison.
\item
The argument \textit{main} should be fully expanded, it cannot be a macro.
\item
Subdirectories and special characters should be avoided in filenames.
\item
The command |\childdocmain{|\textit{main}|}| must be followed by a whitespace.
It should not be followed immediately by another command
or by a comment mark `|%|'.
This is because the \TeX{} parser reads the token immediately following
the argument of |\childdocmain| and puts it
at the beginning of every child section;
however, a white\-space is ignored.
\end{itemize}

%%%%%%%%%%%%%%%%%%%%%%%%%%%%%%%%%%%%%%%%
\paragraph{Content of Main File.}

It is advisable to place all content in the child files included by |\include|.
Any output contained in the main file will appear in all child documents
unless suppressed manually;
it cannot be suppressed automatically by the |\includeonly| directive
and thus should normally be avoided.
A method to include some content in the main file
by means of conditional processing is described in \secref{sec:conditional}.

%%%%%%%%%%%%%%%%%%%%%%%%%%%%%%%%%%%%%%%%
\paragraph{Page Numbering.}

When only a part of the document is compiled,
the appropriate numbering of pages
(as well as other status parameters)
is determined from the |.aux| files.
The latter contain information from previous passes.
However this information needs to propagate through
all intermediate child documents.
Therefore the page numbering in child documents may well
be inconsistent until the complete document is compiled at least once.

A useful (if unconventional) way to always ensure a consistent
page numbering is to restart the numbering in each child document
and denote the pages by `\textit{child}|.|\textit{page}'
where \textit{child} represents the chapter/section number of the child file.
This can be achieved by the command
|\numberwithin{page}{|\textit{child}|}|
of the \textsf{amsmath} package
where \textit{child} can be |chapter| or |section|
depending on the chosen structuring.
Alternatively, one can modify the macro |\thepage| appropriately
and reset the counter |page| at the start of each child file.

%%%%%%%%%%%%%%%%%%%%%%%%%%%%%%%%%%%%%%%%%%%%%%%%%%%%%%%%%%%%%%%%%%%%%%%%%%%%%%%%
\subsection{Conditional Processing}
\label{sec:conditional}

The package provides a mechanism to compile different versions
of a document. To customise the versions further some conditional processing
can come in handy to distinguish which version is being compiled.
The package provides two macros to describe the compilation context:

%%%%%%%%%%%%%%%%%%%%%%%%%%%%%%%%%%%%%%%%
\DescribeMacro{\ifchilddoc}
The conditional |\ifchilddoc| distinguishes between the compilation of
child documents and the main document:
%
\begin{center}
|\ifchilddoc |\textit{child-code}| |[|\||else |\textit{main-code}]| \||fi|
\end{center}

%%%%%%%%%%%%%%%%%%%%%%%%%%%%%%%%%%%%%%%%
\DescribeMacro{\childdocname}
\DescribeMacro{\childdocjob}
The macro |\childdocname| contains the filename (without extension)
of the main or child file being processed.
Note that |\childdocjob| will always contain the name of the main file.

%%%%%%%%%%%%%%%%%%%%%%%%%%%%%%%%%%%%%%%%
\paragraph{Title Page.}

Conditional processing can be used to include a title or banner page
in the main document when proper precautions are taken.
Importantly, the code in the main file should ensure that the page counter
(as well as other status parameters which are stored in the |.aux| files)
takes the same value after the conditional processing.
Otherwise the page numbers may take divergent values
depending on which part is compiled.

For example, a title page could be declared by:
%
\begin{center}
\begin{tabular}{l}
|\ifchilddoc\||else|\\
|\addtocounter{page}{-1}|\\
\textit{code for title page}\\
|\newpage|\\
|\||fi|
\end{tabular}
\end{center}
%
A banner page for the child documents can be generated by:
%
\begin{center}
\begin{tabular}{l}
|\ifchilddoc|\\
|\addtocounter{page}{-1}|\\
\textit{code for banner page}\\
|\newpage|\\
|\||fi|
\end{tabular}
\end{center}
%
Here one could write a message such as:
\begin{center}
|This is the part \childdocname{} of \childdocjob{}.|
\end{center}

%%%%%%%%%%%%%%%%%%%%%%%%%%%%%%%%%%%%%%%%%%%%%%%%%%%%%%%%%%%%%%%%%%%%%%%%%%%%%%%%
\subsection{Flags}
\label{sec:flags}

The package makes it easy to generate different versions
of the main or child documents.
To this end compilation flags can be defined
and assigned different default values.
They will be particularly useful in conjunction
with the forwarding mechanism described in \secref{sec:forward}.

For example, it may be useful to have a flag |\version|
which can be set to |draft| or |final|.
The document source will contain some conditional code
depending on the value of |\version|.
Suppose further, the flag should default to |final| for the main file
and to |draft| for child files
which is a natural assignment for editing the document.
This is achieved by placing the following code
in the preamble of the main document
(below the |\childdocmain| directive):
%
\begin{center}
\begin{tabular}{l}
|\ifchilddoc|\\
|\providecommand{\version}{draft}|\\
|\||else|\\
|\providecommand{\version}{final}|\\
|\||fi|
\end{tabular}
\end{center}
%
The definition by |\providecommand| makes sure
that previous definitions are not overwritten.
Further statements |\providecommand{\version}{...}|
can thus be added before the above code to override it.

For the main file, one might add a line
(between |\childdocmain| and the above block)
%
\begin{center}
|%\ifchilddoc\||else\providecommand{\version}{draft}\||fi|
\end{center}
%
which can be uncommented to produce a draft version.
Likewise one can add a line to the very top of a child file
(above the |\childdocof{|\textit{main}|}| directive)
%
\begin{center}
|%\providecommand{\version}{final}|
\end{center}
%
which can be uncommented to produce the final version of this child document.

%%%%%%%%%%%%%%%%%%%%%%%%%%%%%%%%%%%%%%%%%%%%%%%%%%%%%%%%%%%%%%%%%%%%%%%%%%%%%%%%
\subsection{Forwarding}
\label{sec:forward}

Different versions of the main or child documents
using compilation flags as described in \secref{sec:flags}
can be (permanently) stored in different files
for convenient compilation, viewing and distribution.
To this end, the package defines a command
to pass on compilation to a different file:

%%%%%%%%%%%%%%%%%%%%%%%%%%%%%%%%%%%%%%%%
\DescribeMacro{\childdocforward}
The command |\childdocforward| redirects processing to
another source file:
%
\begin{center}
\begin{tabular}{l}
|% \iffalse
%
% childdoc.dtx Copyright (C) 2017-2018 Niklas Beisert
%
% This work may be distributed and/or modified under the
% conditions of the LaTeX Project Public License, either version 1.3
% of this license or (at your option) any later version.
% The latest version of this license is in
%   http://www.latex-project.org/lppl.txt
% and version 1.3 or later is part of all distributions of LaTeX
% version 2005/12/01 or later.
%
% This work has the LPPL maintenance status `maintained'.
%
% The Current Maintainer of this work is Niklas Beisert.
%
% This work consists of the files childdoc.dtx and childdoc.ins
% and the derived files childdoc.def and cdocsamp.tex with
% cdocsch1.tex, cdocsch2.tex, cdocsdrf.tex, cdocsfn1.tex, cdocsfn2.tex.
%
%<package>\ifdefined\childdocmain\endinput\fi
%<package>\ProvidesFile{childdoc.def}[2018/12/30 v2.0 child document driver]
%<samplemain>\ProvidesFile{cdocsamp.tex}[2018/12/30 v2.0 sample for childdoc]
%<*driver>
%\ProvidesFile{childdoc.drv}[2018/12/30 v2.0 childdoc reference manual file]
\PassOptionsToClass{10pt,a4paper}{article}
\documentclass{ltxdoc}

\usepackage[margin=35mm]{geometry}
\usepackage{hyperref}
\usepackage{hyperxmp}
\usepackage[usenames]{color}

\hypersetup{colorlinks=true}
\hypersetup{pdfstartview=FitH}
\hypersetup{pdfpagemode=UseNone}
\hypersetup{pdfsource={}}
\hypersetup{pdflang={en-UK}}
\hypersetup{pdfcopyright={Copyright 2017-2018 Niklas Beisert.
  This work may be distributed and/or modified under the
  conditions of the LaTeX Project Public License, either version 1.3
  of this license or (at your option) any later version.}}
\hypersetup{pdflicenseurl={http://www.latex-project.org/lppl.txt}}
\hypersetup{pdfcontactaddress={ETH Zurich, ITP, HIT K,
  Wolfgang-Pauli-Strasse 27}}
\hypersetup{pdfcontactpostcode={8093}}
\hypersetup{pdfcontactcity={Zurich}}
\hypersetup{pdfcontactcountry={Switzerland}}
\hypersetup{pdfcontactemail={nbeisert@itp.phys.ethz.ch}}
\hypersetup{pdfcontacturl={http://people.phys.ethz.ch/\xmptilde nbeisert/}}

\newcommand{\secref}[1]{\hyperref[#1]{section \ref*{#1}}}

\parskip1ex
\parindent0pt
\let\olditemize\itemize
\def\itemize{\olditemize\parskip0pt}

\begin{document}

\title{The \textsf{childdoc} Package}
\hypersetup{pdftitle={The childdoc Package}}
\author{Niklas Beisert\\[2ex]
  Institut f\"ur Theoretische Physik\\
  Eidgen\"ossische Technische Hochschule Z\"urich\\
  Wolfgang-Pauli-Strasse 27, 8093 Z\"urich, Switzerland\\[1ex]
  \href{mailto:nbeisert@itp.phys.ethz.ch}
  {\texttt{nbeisert@itp.phys.ethz.ch}}}
\hypersetup{pdfauthor={Niklas Beisert}}
\hypersetup{pdfsubject={Manual for the LaTeX2e Package childdoc}}
\date{30 December 2018, \textsf{v2.0}}
\maketitle

\begin{abstract}\noindent
\textsf{childdoc} is a \LaTeXe{} package
that enables the direct compilation
of document sections included by |\include|
to individual files.
\end{abstract}

\begingroup
\parskip0ex
\tableofcontents
\endgroup

%%%%%%%%%%%%%%%%%%%%%%%%%%%%%%%%%%%%%%%%%%%%%%%%%%%%%%%%%%%%%%%%%%%%%%%%%%%%%%%%
%%%%%%%%%%%%%%%%%%%%%%%%%%%%%%%%%%%%%%%%%%%%%%%%%%%%%%%%%%%%%%%%%%%%%%%%%%%%%%%%
\section{Introduction}

\LaTeX{} provides a mechanism to structure a large document (such as a book)
into a main file and several child files (containing the chapters)
using the |\include| command.
This mechanism is beneficial for documents
which span hundreds of pages in order to
make the source file(s) more manageable.
Moreover, compilation can be restricted to
selected child files by means of the |\includeonly| command.
The latter feature can be used to reduce the compilation time while editing
(this was significantly more useful in the earlier days of \LaTeX{})
or to generate a smaller document which is easier to navigate.
Another application of |\includeonly| is to generate
documents consisting of selected parts of the complete document.

However, there are a few drawbacks of the plain |\include| mechanism:
\begin{itemize}
\item
The child files cannot be compiled on their own,
they can only be compiled via the main file.
A naive editing environment
(such as a text editor with an option
to have the current file processed by \LaTeX)
may require one to switch to the main file before compiling;
attempting to compile the child file produces errors.
\item
The main file must be modified (each time)
to adjust the |\includeonly| command
to the present needs. This easily leaves the main file in a messy state.
\item
The generated document will always carry the filename
of the main document. This is inconvenient if
several child files are to be compiled and
to be kept for distribution.
\end{itemize}

The present package provides a simple interface
to make child files individually compilable by \LaTeX{}.
Compiling a child file then has the same effect as compiling
the main file with an |\includeonly| command
to select the appropriate child.
Moreover the generated document will carry the name of the child
rather than the main file.
This resolves all three above issues.

This feature is meant to make the editing of books,
thesis documents and lecture notes somewhat more convenient.
However, the package can also be used efficiently for
composing a series of documents (such as exercise sheets)
which are typically distributed individually.
It then assists the author in generating the individual documents
(potentially in different versions)
as well as a document containing the collected series.
Another application is in developing style files
or other kinds of included material
where compilation of the style file could redirect
to a sample or test file.

%%%%%%%%%%%%%%%%%%%%%%%%%%%%%%%%%%%%%%%%%%%%%%%%%%%%%%%%%%%%%%%%%%%%%%%%%%%%%%%%
%%%%%%%%%%%%%%%%%%%%%%%%%%%%%%%%%%%%%%%%%%%%%%%%%%%%%%%%%%%%%%%%%%%%%%%%%%%%%%%%
\section{Usage}

First of all, the package \textsf{childdoc} is \emph{not} a standard
\LaTeXe{} |.sty| style file! Therefore it needs to be invoked in
a non-standard way.

%%%%%%%%%%%%%%%%%%%%%%%%%%%%%%%%%%%%%%%%%%%%%%%%%%%%%%%%%%%%%%%%%%%%%%%%%%%%%%%%
\subsection{Included Files}
\label{sec:include}

%%%%%%%%%%%%%%%%%%%%%%%%%%%%%%%%%%%%%%%%
\DescribeMacro{\childdocmain}
To use the package, add the commands
\begin{center}
\begin{tabular}{l}
|\input{childdoc.def}|\\
|\childdocmain{}|\\
\end{tabular}
\end{center}
at the very top of the main \LaTeX{} file,
in particular \emph{before} the |\documentclass| statement!
The argument of |\childdocmain| should be left empty
(but it must be present).

%%%%%%%%%%%%%%%%%%%%%%%%%%%%%%%%%%%%%%%%
\DescribeMacro{\childdocof}
Furthermore, add the commands
\begin{center}
\begin{tabular}{l}
|\input{childdoc.def}|\\
|\childdocof{|\textit{main}|}|\\
\end{tabular}
\end{center}
at the top of every child file \textit{child}
which is included by |\include{|\textit{child}|}|
from within the main file
(or at least for those files to be compiled individually).
The argument \textit{main} must be the filename of the main file.

There are a couple of
considerations in setting up the main and child documents:

%%%%%%%%%%%%%%%%%%%%%%%%%%%%%%%%%%%%%%%%
\paragraph{Restrictions.}

Please note the following restrictions:
\begin{itemize}
\item
|\childdocmain| must be called with one argument \textit{main}
to ensure compatibility with earlier version of the package.
It must either be empty (|\childdocmain{}|)
or precisely match the filename of the main file in which it is specified.
See \secref{sec:detection} for further information.
\item
The filename \textit{main} must be specified without the |.tex| extension.
\item
The filename \textit{main} is case sensitive
(even in case-insensitive file systems)
due to internal string comparison.
\item
The argument \textit{main} should be fully expanded, it cannot be a macro.
\item
Subdirectories and special characters should be avoided in filenames.
\item
The command |\childdocmain{|\textit{main}|}| must be followed by a whitespace.
It should not be followed immediately by another command
or by a comment mark `|%|'.
This is because the \TeX{} parser reads the token immediately following
the argument of |\childdocmain| and puts it
at the beginning of every child section;
however, a white\-space is ignored.
\end{itemize}

%%%%%%%%%%%%%%%%%%%%%%%%%%%%%%%%%%%%%%%%
\paragraph{Content of Main File.}

It is advisable to place all content in the child files included by |\include|.
Any output contained in the main file will appear in all child documents
unless suppressed manually;
it cannot be suppressed automatically by the |\includeonly| directive
and thus should normally be avoided.
A method to include some content in the main file
by means of conditional processing is described in \secref{sec:conditional}.

%%%%%%%%%%%%%%%%%%%%%%%%%%%%%%%%%%%%%%%%
\paragraph{Page Numbering.}

When only a part of the document is compiled,
the appropriate numbering of pages
(as well as other status parameters)
is determined from the |.aux| files.
The latter contain information from previous passes.
However this information needs to propagate through
all intermediate child documents.
Therefore the page numbering in child documents may well
be inconsistent until the complete document is compiled at least once.

A useful (if unconventional) way to always ensure a consistent
page numbering is to restart the numbering in each child document
and denote the pages by `\textit{child}|.|\textit{page}'
where \textit{child} represents the chapter/section number of the child file.
This can be achieved by the command
|\numberwithin{page}{|\textit{child}|}|
of the \textsf{amsmath} package
where \textit{child} can be |chapter| or |section|
depending on the chosen structuring.
Alternatively, one can modify the macro |\thepage| appropriately
and reset the counter |page| at the start of each child file.

%%%%%%%%%%%%%%%%%%%%%%%%%%%%%%%%%%%%%%%%%%%%%%%%%%%%%%%%%%%%%%%%%%%%%%%%%%%%%%%%
\subsection{Conditional Processing}
\label{sec:conditional}

The package provides a mechanism to compile different versions
of a document. To customise the versions further some conditional processing
can come in handy to distinguish which version is being compiled.
The package provides two macros to describe the compilation context:

%%%%%%%%%%%%%%%%%%%%%%%%%%%%%%%%%%%%%%%%
\DescribeMacro{\ifchilddoc}
The conditional |\ifchilddoc| distinguishes between the compilation of
child documents and the main document:
%
\begin{center}
|\ifchilddoc |\textit{child-code}| |[|\||else |\textit{main-code}]| \||fi|
\end{center}

%%%%%%%%%%%%%%%%%%%%%%%%%%%%%%%%%%%%%%%%
\DescribeMacro{\childdocname}
\DescribeMacro{\childdocjob}
The macro |\childdocname| contains the filename (without extension)
of the main or child file being processed.
Note that |\childdocjob| will always contain the name of the main file.

%%%%%%%%%%%%%%%%%%%%%%%%%%%%%%%%%%%%%%%%
\paragraph{Title Page.}

Conditional processing can be used to include a title or banner page
in the main document when proper precautions are taken.
Importantly, the code in the main file should ensure that the page counter
(as well as other status parameters which are stored in the |.aux| files)
takes the same value after the conditional processing.
Otherwise the page numbers may take divergent values
depending on which part is compiled.

For example, a title page could be declared by:
%
\begin{center}
\begin{tabular}{l}
|\ifchilddoc\||else|\\
|\addtocounter{page}{-1}|\\
\textit{code for title page}\\
|\newpage|\\
|\||fi|
\end{tabular}
\end{center}
%
A banner page for the child documents can be generated by:
%
\begin{center}
\begin{tabular}{l}
|\ifchilddoc|\\
|\addtocounter{page}{-1}|\\
\textit{code for banner page}\\
|\newpage|\\
|\||fi|
\end{tabular}
\end{center}
%
Here one could write a message such as:
\begin{center}
|This is the part \childdocname{} of \childdocjob{}.|
\end{center}

%%%%%%%%%%%%%%%%%%%%%%%%%%%%%%%%%%%%%%%%%%%%%%%%%%%%%%%%%%%%%%%%%%%%%%%%%%%%%%%%
\subsection{Flags}
\label{sec:flags}

The package makes it easy to generate different versions
of the main or child documents.
To this end compilation flags can be defined
and assigned different default values.
They will be particularly useful in conjunction
with the forwarding mechanism described in \secref{sec:forward}.

For example, it may be useful to have a flag |\version|
which can be set to |draft| or |final|.
The document source will contain some conditional code
depending on the value of |\version|.
Suppose further, the flag should default to |final| for the main file
and to |draft| for child files
which is a natural assignment for editing the document.
This is achieved by placing the following code
in the preamble of the main document
(below the |\childdocmain| directive):
%
\begin{center}
\begin{tabular}{l}
|\ifchilddoc|\\
|\providecommand{\version}{draft}|\\
|\||else|\\
|\providecommand{\version}{final}|\\
|\||fi|
\end{tabular}
\end{center}
%
The definition by |\providecommand| makes sure
that previous definitions are not overwritten.
Further statements |\providecommand{\version}{...}|
can thus be added before the above code to override it.

For the main file, one might add a line
(between |\childdocmain| and the above block)
%
\begin{center}
|%\ifchilddoc\||else\providecommand{\version}{draft}\||fi|
\end{center}
%
which can be uncommented to produce a draft version.
Likewise one can add a line to the very top of a child file
(above the |\childdocof{|\textit{main}|}| directive)
%
\begin{center}
|%\providecommand{\version}{final}|
\end{center}
%
which can be uncommented to produce the final version of this child document.

%%%%%%%%%%%%%%%%%%%%%%%%%%%%%%%%%%%%%%%%%%%%%%%%%%%%%%%%%%%%%%%%%%%%%%%%%%%%%%%%
\subsection{Forwarding}
\label{sec:forward}

Different versions of the main or child documents
using compilation flags as described in \secref{sec:flags}
can be (permanently) stored in different files
for convenient compilation, viewing and distribution.
To this end, the package defines a command
to pass on compilation to a different file:

%%%%%%%%%%%%%%%%%%%%%%%%%%%%%%%%%%%%%%%%
\DescribeMacro{\childdocforward}
The command |\childdocforward| redirects processing to
another source file:
%
\begin{center}
\begin{tabular}{l}
|\input{childdoc.def}|\\
|\childdocforward[|\textit{main}|]{|\textit{dest}|}|\\
\end{tabular}
\end{center}
%
The argument \textit{dest} is the destination file
(without extension).
It should be the main file or one of the child files.
Note that further \textsf{childdoc} directives
such as |\childdocof| and |\childdocforward|
in the indicated file will be processed in this form.
The optional argument \textit{main}
passes on directly to the main file \textit{main}
while pretending to compile the child \textit{dest}.
This form behaves as if \textit{dest}
issues |\childdocof{|\textit{main}|}| right away,
and no further \textsf{childdoc} directives will be processed.

%%%%%%%%%%%%%%%%%%%%%%%%%%%%%%%%%%%%%%%%
\DescribeMacro{\...prefix}
In the alternative form |\childdocforwardprefix|,
%
\begin{center}
\begin{tabular}{l}
|\input{childdoc.def}|\\
|\childdocforwardprefix[|\textit{main}|]{|\textit{prefix}|}{|\textit{dest}|}|
\end{tabular}
\end{center}
%
the destination file is determined by a pattern
depending on the current file:
To make this work, the current file must be called
`{\textit{prefix}\hspace{0.2em}\textit{suffix}}'
with \textit{prefix} matching precisely the argument.
Processing is then passed on to the file
`{\textit{dest}\hspace{0.2em}\textit{suffix}}'.
Surely, the same effect is achieved by
directly specifying the
argument `{\textit{dest}\hspace{0.2em}\textit{suffix}}'
in the first form.
However, that requires to set up a different file
for each child. With the alternative form of the command
all these files can have exactly the same content
which simplifies setting them up and maintaining them.

For example, the following file |draft.tex|
with a compilation flag |\version| as described in \secref{sec:flags}
compiles the main document as a draft:
%
\begin{center}
\begin{tabular}{l}
|\def\version{draft}|\\
|\input{childdoc.def}|\\
|\childdocforward{|\textit{main}|}|
\end{tabular}
\end{center}
%
Likewise, the following files |final|\textit{nn}|.tex|
compile the final version of the child document
|child|\textit{nn}|.tex|:
%
\begin{center}
\begin{tabular}{l}
|\def\version{final}|\\
|\input{childdoc.def}|\\
|\childdocforwardprefix{final}{child}|
\end{tabular}
\end{center}
%

Note that when several versions of a main file and/or of each child file
are to be generated, it may be convenient to set up a |Makefile| or
shell script to automatise the process.

%%%%%%%%%%%%%%%%%%%%%%%%%%%%%%%%%%%%%%%%%%%%%%%%%%%%%%%%%%%%%%%%%%%%%%%%%%%%%%%%
\subsection{Command Line Processing}
\label{sec:commandline}

The effect of redirection files can also be achieved by invoking
the \LaTeX{} compiler with a more elaborate command line.
Most conveniently this should be done as part
of a shell script or a |Makefile|.

When using \textsf{childdoc} in the main file, the following
command lines effectively perform a redirection
(note that depending on the shell being used,
backslashes may have to be doubled: `|\|' $\to$ `|\\|'):
%
\begin{center}
|... -jobname "|\textit{target}|" |\\|"|[\textit{flags}]%
|\input{childdoc.def}\childdocforward[|\textit{main}|]{|\textit{dest}|}"|
\end{center}
%
Here \textit{target} is the name of the output file,
\textit{main} is the name of the main file
and \textit{dest} is the name of the main or child file to be processed
(all filenames without extensions).
The optional argument \textit{main} can be omitted
if \textit{main} matches \textit{dest}.
Optionally, compilation \textit{flags} can be defined via |\def| commands.
This command line makes the \TeX{} engine believe
it is compiling the file \textit{target}
whose content is specified as the latter parameter.
The provided code then forwards the processing to
\textit{main} or \textit{dest} as described in \secref{sec:forward}.

%%%%%%%%%%%%%%%%%%%%%%%%%%%%%%%%%%%%%%%%%%%%%%%%%%%%%%%%%%%%%%%%%%%%%%%%%%%%%%%%
\subsection{Include by Input}
\label{sec:input}

Including child documents by |\include| has some restrictions by design.
Most notably, the content of a child document always occupies
its own set of pages; pages cannot be shared between child documents.
Usually, this behaviour makes perfect sense
because each child document contain an essential part of the document.
However, in some situations it may be desirable to compose
a document from a collection of parts
without having mandatory page breaks between then.
For this case, the package
provides a mechanism to include parts
by |\input| which can also be processed individually.
However, by construction this mechanism
requires manual handling of the content to be output.

%%%%%%%%%%%%%%%%%%%%%%%%%%%%%%%%%%%%%%%%
\DescribeMacro{\ifchilddocmanual}
The main file should be prepared as usual, see \secref{sec:include}.
However, the document body must make a distinction
between processing of an individual part and of the main document, e.g.:
%
\begin{center}
\begin{tabular}{l}
|\ifchilddocmanual|\\
|\input{\childdocname}|\\
|\||else|\\
\textit{document body with }|\input{|\textit{part}|}|\\
|\||fi|
\end{tabular}
\end{center}
%
The conditional |\ifchilddocmanual| is true whenever
a part to be included by |\input| is being compiled,
and the name of the part is stored in |\childdocname|.

%%%%%%%%%%%%%%%%%%%%%%%%%%%%%%%%%%%%%%%%
\DescribeMacro{\childdocby}
Each part to be included by |\input| should start with:
%
\begin{center}
\begin{tabular}{l}
|\input{childdoc.def}|\\
|\childdocby{|\textit{main}|}|\\
\end{tabular}
\end{center}
%
The directive |\childdocby| is similar to |\childdocof|
described in \secref{sec:include},
but the subsequent selection of content must be done manually.
To that end, both |\ifchilddoc| and |\ifchilddocmanual|
will be true upon processing of a part,
and the name of the part is stored in |\childdocname|.
Note that |\jobname| will be set to the filename of the current part
so that each part receives an individual |.aux| file
that does not interfere with the |.aux| file(s) of the main document.
This behaviour can be altered by the alternative form
|\childdocby[*]{|\textit{main}|}| (with a non-empty optional argument)
which uses the |.aux| file of the main document
by setting |\jobname| to \textit{main}.

%%%%%%%%%%%%%%%%%%%%%%%%%%%%%%%%%%%%%%%%%%%%%%%%%%%%%%%%%%%%%%%%%%%%%%%%%%%%%%%%
\subsection{Driver Development}
\label{sec:driver}

The \textsf{childdoc} mechanism can also be use for the development
of definition files such as \LaTeX{} styles or classes.
This case differs from the above setup with multiple parts
included by |\include| in that no |\includeonly| should be invoked.
This can be achieved by starting the include file
(before |\ProvidesPackage|) with:
%
\begin{center}
\begin{tabular}{l}
|\input{childdoc.def}|\\
|\childdocforward{|\textit{main}|}|\\
\end{tabular}
\end{center}
%
or alternatively with:
%
\begin{center}
\begin{tabular}{l}
|\input{childdoc.def}|\\
|\childdocby{|\textit{main}|}|\\
\end{tabular}
\end{center}
%
Both forms have slightly different effects as described above.
The main file is prepared as usual, see \secref{sec:include}.

%%%%%%%%%%%%%%%%%%%%%%%%%%%%%%%%%%%%%%%%%%%%%%%%%%%%%%%%%%%%%%%%%%%%%%%%%%%%%%%%
\subsection{Legacy Detection}
\label{sec:detection}

The directive |\childdocmain| in the main file can detect
whether the complete document or merely a child is to be compiled
even without using the directive |\childdocof|.
This method is deprecated because it is less robust
and there is no compelling reason to use it;
it is merely provided for backward compatibility
and it may be removed in future versions.

If the detection mechanism is to be used,
it is mandatory to correctly specify
the filename of the main file as the argument of |\childdocmain|:
%
\begin{center}
\begin{tabular}{l}
|\input{childdoc.def}|\\
|\childdocmain{|\textit{main}|}|\\
\end{tabular}
\end{center}
%
If |\jobname| does not match the argument \textit{main} of |\childdocmain|,
it is assumed that |\jobname| points to the child file to be compiled.
When using |\childdocmain| with the main file specified as argument,
it suffices to start a child file
with just |\input{|\textit{main}|}|
without loading of the package and using |\childdocof|.
If instead all processing is done
with the appropriate \textsf{childdoc} directives,
the argument of \textit{main} of |\childdocmain| can be empty.

An alternative version of the command line processing described
in \secref{sec:commandline} using the detection mechanism reads:
%
\begin{center}
|... -jobname "|\textit{target}|" "|[\textit{flags}]%
[|\def\jobname{|\textit{dest}|}|]|\input{|\textit{main}|}"|
\end{center}

%%%%%%%%%%%%%%%%%%%%%%%%%%%%%%%%%%%%%%%%%%%%%%%%%%%%%%%%%%%%%%%%%%%%%%%%%%%%%%%%
\subsection{Manual Code}
\label{sec:manual}

In case one cannot be certain whether the definitions file |childdoc.def|
is installed on the target \TeX{} distribution
and one prefers not to ship it,
it is conceivable to paste a few relevant commands into the sources.

To that end, drop all statements |\input{childdoc.def}|
and perform the replacements as outlined below.
Instead of |\childdocmain{|\textit{main}|}| add the following code
to the top of the main file:
%
\begin{center}
\begin{tabular}{l}
|\||ifdefined\childdocname\endinput\||fi\newif\ifchilddoc|\\
|\edef\childdocname{\scantokens\expandafter{\jobname\noexpand}}|\\
|\def\childdocmain{|\textit{main}|}\||ifx\childdocmain\childdocname\||else|\\
|\childdoctrue\includeonly{\childdocname}\let\jobname\childdocmain\||fi|\\
\end{tabular}
\end{center}
%
Instead of |\childdocof{|\textit{main}|}| just include the main file
at the top of each child file:
%
\begin{center}
|\input{|\textit{main}|}|
\end{center}
%
A simple redirection |\childdocforward{|\textit{dest}|}| is achieved by:
%
\begin{center}
|\def\jobname{|\textit{dest}|}\input{\jobname}|
\end{center}
%
The redirection with prefix
|\childdocforwardprefix[|\textit{prefix}|]{|\textit{dest}|}|
is accomplished by:
%
\begin{center}
\begin{tabular}{l}
|{\edef\jobname{\scantokens\expandafter{\jobname\noexpand}}|\\
|\def\redirectjob |\textit{prefix}|#1~~~{\gdef\jobname{|\textit{dest}|#1}}|\\
|\expandafter\redirectjob\jobname~~~}\input{\jobname}|
\end{tabular}
\end{center}

In an alternative approach,
child documents can be compiled by a specific command line
without additional code or specific definitions:
%
\begin{center}
|... -jobname "|\textit{target}|" "|[\textit{flags}]%
|\includeonly{|\textit{dest}|}\input{|\textit{main}|}"|
\end{center}
%

%%%%%%%%%%%%%%%%%%%%%%%%%%%%%%%%%%%%%%%%%%%%%%%%%%%%%%%%%%%%%%%%%%%%%%%%%%%%%%%%
%%%%%%%%%%%%%%%%%%%%%%%%%%%%%%%%%%%%%%%%%%%%%%%%%%%%%%%%%%%%%%%%%%%%%%%%%%%%%%%%
\section{Information}

%%%%%%%%%%%%%%%%%%%%%%%%%%%%%%%%%%%%%%%%%%%%%%%%%%%%%%%%%%%%%%%%%%%%%%%%%%%%%%%%
\subsection{Copyright}

Copyright \copyright{} 2017--2018 Niklas Beisert

This work may be distributed and/or modified under the
conditions of the \LaTeX{} Project Public License, either version 1.3
of this license or (at your option) any later version.
The latest version of this license is in
  \url{http://www.latex-project.org/lppl.txt}
and version 1.3 or later is part of all distributions of \LaTeX{}
version 2005/12/01 or later.

This work has the LPPL maintenance status `maintained'.

The Current Maintainer of this work is Niklas Beisert.

This work consists of the files |README.txt|, |childdoc.ins| and |childdoc.dtx|
as well as the derived files |childdoc.def|, |cdocsamp.tex|
with |cdocsch1.tex|, |cdocsch2.tex|, |cdocspt3.tex|, |cdocspt4.tex|,
|cdocsdrf.tex|, |cdocsfn1.tex|, |cdocsfn2.tex|
as well as |childdoc.pdf|.

%%%%%%%%%%%%%%%%%%%%%%%%%%%%%%%%%%%%%%%%%%%%%%%%%%%%%%%%%%%%%%%%%%%%%%%%%%%%%%%%
\subsection{Files and Installation}

The package consists of the files:
%
\begin{center}
\begin{tabular}{ll}
    |README.txt|   & readme file \\
    |childdoc.ins| & installation file \\
    |childdoc.dtx| & source file \\
    |childdoc.def| & definition file \\
    |cdocsamp.tex| & sample main file \\
    |cdocsch1.tex| & sample include file \\
    |cdocsch2.tex| & sample include file \\
    |cdocspt3.tex| & sample part file \\
    |cdocspt4.tex| & sample part file \\
    |cdocsdrf.tex| & sample redirection file \\
    |cdocsfn1.tex| & sample redirection file \\
    |cdocsfn2.tex| & sample redirection file \\
    |childdoc.pdf| & manual
\end{tabular}
\end{center}
%
The distribution consists of the files
|README.txt|, |childdoc.ins| and |childdoc.dtx|.
%
\begin{itemize}
\item
Run (pdf)\LaTeX{} on |childdoc.dtx|
to compile the manual |childdoc.pdf| (this file).
\item
Run \LaTeX{} on |childdoc.ins| to create the definitions file |childdoc.def|
and the sample |cdocsamp.tex| with include files
|cdocsch1.tex|, |cdocsch2.tex|, |cdocspt3.tex|, |cdocspt4.tex|,
|cdocsdrf.tex|, |cdocsfn1.tex|, |cdocsfn2.tex|.
Then copy the file |childdoc.def| to an appropriate directory of your \LaTeX{}
distribution, e.g.\ \textit{texmf-root}|/tex/latex/childdoc|.
\end{itemize}

%%%%%%%%%%%%%%%%%%%%%%%%%%%%%%%%%%%%%%%%%%%%%%%%%%%%%%%%%%%%%%%%%%%%%%%%%%%%%%%%
\subsection{Related CTAN Packages}

There are several other packages which offer a similar functionality:
%
\begin{itemize}
\item
The packages
\href{http://ctan.org/pkg/docmute}{\textsf{docmute}},
\href{http://ctan.org/pkg/includex}{\textsf{includex}} and
\href{http://ctan.org/pkg/standalone}{\textsf{standalone}}
provide commands to include only the document body of
a child file thus allowing both files to be compiled individually.
\item
The packages \href{http://ctan.org/pkg/subdocs}{\textsf{subdocs}}
and \href{http://ctan.org/pkg/subfiles}{\textsf{subfiles}}
provide structures in which the main and child documents can be
encapsulated and allowing them to be compiled individually.
The inclusion mechanism is different from the conventional |\include|.
\item
The package \href{http://ctan.org/pkg/combine}{\textsf{combine}}
is an elaborate solution to combine several documents into one.
\end{itemize}
%
See also the CTAN topic \href{http://ctan.org/topic/subdocs}{\textsf{subdocs}}
for further related packages.
The present package differs from the above solutions in that
a document structure constructed with the conventional |\include| mechanism
just needs two extra commands at the top of every file
such that all constituent files can be compiled individually.

%%%%%%%%%%%%%%%%%%%%%%%%%%%%%%%%%%%%%%%%%%%%%%%%%%%%%%%%%%%%%%%%%%%%%%%%%%%%%%%%
%\subsection{Feature Suggestions}
%
%The following is a list of features which may be useful for future
%versions of this package:
%%
%\begin{itemize}
%\item
%\ldots
%\end{itemize}

%%%%%%%%%%%%%%%%%%%%%%%%%%%%%%%%%%%%%%%%%%%%%%%%%%%%%%%%%%%%%%%%%%%%%%%%%%%%%%%%
\subsection{Revision History}

%%%%%%%%%%%%%%%%%%%%%%%%%%%%%%%%%%%%%%%%
\paragraph{v2.0:} 2018/12/30

\begin{itemize}
\item
immediate forward processing
\item
added |\childdocby| mechanism
\item
manual restructured
\end{itemize}

%%%%%%%%%%%%%%%%%%%%%%%%%%%%%%%%%%%%%%%%
\paragraph{v1.6:} 2018/01/17

\begin{itemize}
\item
application for development of include files
\item
corrections to manual
\end{itemize}

%%%%%%%%%%%%%%%%%%%%%%%%%%%%%%%%%%%%%%%%
\paragraph{v1.5:} 2017/05/21

\begin{itemize}
\item
more complete structuring introduced
\item
|\childdocof| introduced
\item
|\childdoc| renamed to |\childdocmain|
\item
|\childredirect| renamed to |\childdocforward| and |\childdocforwardprefix|
and functionality expanded
\end{itemize}

%%%%%%%%%%%%%%%%%%%%%%%%%%%%%%%%%%%%%%%%
\paragraph{v1.0:} 2017/04/27

\begin{itemize}
\item
manual and install package
\item
first version published on CTAN
\end{itemize}

%%%%%%%%%%%%%%%%%%%%%%%%%%%%%%%%%%%%%%%%
\paragraph{v0.6:} 2017/04/26

\begin{itemize}
\item
redirection mechanism added
\end{itemize}

%%%%%%%%%%%%%%%%%%%%%%%%%%%%%%%%%%%%%%%%
\paragraph{v0.5:} 2017/04/26

\begin{itemize}
\item
functionality in definition file
\end{itemize}


%%%%%%%%%%%%%%%%%%%%%%%%%%%%%%%%%%%%%%%%%%%%%%%%%%%%%%%%%%%%%%%%%%%%%%%%%%%%%%%%
%%%%%%%%%%%%%%%%%%%%%%%%%%%%%%%%%%%%%%%%%%%%%%%%%%%%%%%%%%%%%%%%%%%%%%%%%%%%%%%%
%%%%%%%%%%%%%%%%%%%%%%%%%%%%%%%%%%%%%%%%%%%%%%%%%%%%%%%%%%%%%%%%%%%%%%%%%%%%%%%%
\appendix

\settowidth\MacroIndent{\rmfamily\scriptsize 000\ }

 \DocInput{childdoc.dtx}

\end{document}
%</driver>
% \fi
%
% %%%%%%%%%%%%%%%%%%%%%%%%%%%%%%%%%%%%%%%%%%%%%%%%%%%%%%%%%%%%%%%%%%%%%%%%%%%%%%
% %%%%%%%%%%%%%%%%%%%%%%%%%%%%%%%%%%%%%%%%%%%%%%%%%%%%%%%%%%%%%%%%%%%%%%%%%%%%%%
% \section{Sample}
%\iffalse
%<*samplemain>
%\fi
%
% The following presents a sample document
% with two chapters, two parts, a title page,
% a compile flag as well as three forwarding files to set the flag.
% It consists of eight |.tex| files:
% \begin{center}
% \begin{tabular}{ll}
% |cdocsamp.tex|&main file\\
% |cdocsch1.tex|&include file for chapter 1\\
% |cdocsch2.tex|&include file for chapter 2\\
% |cdocspt3.tex|&include file for part 3\\
% |cdocspt4.tex|&include file for part 4\\
% |cdocsdrf.tex|&forwarding file for main file in draft mode\\
% |cdocsfi1.tex|&forwarding file for final version of chapter 1\\
% |cdocsfi2.tex|&forwarding file for final version of chapter 2\\
% \end{tabular}
% \end{center}
% Each of the eight files can be compiled directly by the \LaTeX{} compiler.
%
% %%%%%%%%%%%%%%%%%%%%%%%%%%%%%%%%%%%%%%
% \paragraph{Main File.}
%
% The main file is called |cdocsamp.tex|.
%
% Load the \textsf{childdoc} definitions and
% declare the filename for the main document:
%    \begin{macrocode}
\input{childdoc.def}
\childdocmain{}
%    \end{macrocode}

% Optional override for |\version| flag:
%    \begin{macrocode}
%%\ifchilddoc\else\providecommand{\version}{draft}\fi
%    \end{macrocode}

% Define the default values for the |\version| flag
% (|final| for the main file and |draft| for childs):
%    \begin{macrocode}
\ifchilddoc
\providecommand{\version}{draft}
\else
\providecommand{\version}{final}
\fi
%    \end{macrocode}

% Load the standard document class:
%    \begin{macrocode}
\documentclass[12pt]{article}
%    \end{macrocode}

% Start the document body:
%    \begin{macrocode}
\begin{document}
%    \end{macrocode}

% Declare a title page.
% Print title, part of document being processed and version flag:
%    \begin{macrocode}
\addtocounter{page}{-1}
\begin{center}
{\LARGE\bfseries{}childdoc example\par}
\vspace{1cm}
\ifchilddoc
\ifchilddocmanual part\else chapter\fi:
`\childdocname' of `\childdocjob'\par
\else
main document: `\childdocjob'\par
\fi
version: \version\par
\end{center}
\newpage
%    \end{macrocode}

% Manually include selected file,
% otherwise process as usual:
%    \begin{macrocode}
\ifchilddocmanual
\section*{part `\childdocname'}
\input{\childdocname}
\else
%    \end{macrocode}

% Include the two chapters:
%    \begin{macrocode}
\include{cdocsch1}
\include{cdocsch2}
%    \end{macrocode}

% Include the two parts unless only chapters should be displayed:
%    \begin{macrocode}
\ifchilddoc\else
\section{part three}
\input{cdocspt3}
\section{part four}
\input{cdocspt4}
\fi
%    \end{macrocode}

% Process as usual until here:
%    \begin{macrocode}
\fi
%    \end{macrocode}

% End of document body:
%    \begin{macrocode}
\end{document}
%    \end{macrocode}
%\iffalse
%</samplemain>
%\fi
%
% %%%%%%%%%%%%%%%%%%%%%%%%%%%%%%%%%%%%%%
% \paragraph{Chapter Include Files.}
%
% The include files are called |cdocsch1.tex| and |cdocsch2.tex|.
%
%\iffalse
%<*samplechap1|samplechap2>
%\fi

% Optional override for |\version| flag:
%    \begin{macrocode}
%%\providecommand{\version}{final}
%    \end{macrocode}

% Include the main document:
%    \begin{macrocode}
\input{childdoc.def}
\childdocof{cdocsamp}
%    \end{macrocode}

%\iffalse
%</samplechap1|samplechap2>
%\fi
%
%\iffalse
%<*samplechap1>
%\fi
% Some text for chapter 1:
%    \begin{macrocode}
\section{one}
some text in chapter one
%    \end{macrocode}

%\iffalse
%</samplechap1>
%\fi
% Some text for chapter 2:
%\iffalse
%<*samplechap2>
%\fi
%    \begin{macrocode}
\section{two}
more text in chapter two
%    \end{macrocode}

%\iffalse
%</samplechap2>
%\fi
%
% %%%%%%%%%%%%%%%%%%%%%%%%%%%%%%%%%%%%%%
% \paragraph{Part Include Files.}
%
% The include files are called |cdocspt3.tex| and |cdocspt4.tex|.
%
%\iffalse
%<*samplepart3|samplepart4>
%\fi

% Optional override for |\version| flag:
%    \begin{macrocode}
%%\providecommand{\version}{final}
%    \end{macrocode}

% Include the main document:
%    \begin{macrocode}
\input{childdoc.def}
\childdocby{cdocsamp}
%    \end{macrocode}

%\iffalse
%</samplepart3|samplepart4>
%\fi
%
%\iffalse
%<*samplepart3>
%\fi
% Some text for part 3:
%    \begin{macrocode}
some text in part three
%    \end{macrocode}

%\iffalse
%</samplepart3>
%\fi
% Some text for part 4:
%\iffalse
%<*samplepart4>
%\fi
%    \begin{macrocode}
more text in part four
%    \end{macrocode}

%\iffalse
%</samplepart4>
%\fi
%
% %%%%%%%%%%%%%%%%%%%%%%%%%%%%%%%%%%%%%%
% \paragraph{Forwarding for a Complete Draft.}
%
% The following forwarding file |cdocsdrf.tex|
% compiles the main document in draft mode:
%\iffalse
%<*sampledraft>
%\fi
%    \begin{macrocode}
\def\version{draft}
\input{childdoc.def}
\childdocforward{cdocsamp}
%    \end{macrocode}

%\iffalse
%</sampledraft>
%\fi
%
% %%%%%%%%%%%%%%%%%%%%%%%%%%%%%%%%%%%%%%
% \paragraph{Forwarding for Final Version of the Chapters.}
%
% The following forwarding files |cdocsfn1.tex| and |cdocsfn2.tex|
% (with identical content)
% compile the final versions of the child documents
% |cdocsch1.tex| and |cdocsch2.tex|, respectively:
%\iffalse
%<*samplefinal>
%\fi
%    \begin{macrocode}
\def\version{final}
\input{childdoc.def}
\childdocforwardprefix[cdocsamp]{cdocsfn}{cdocsch}
%    \end{macrocode}

%\iffalse
%</samplefinal>
%\fi
%
% %%%%%%%%%%%%%%%%%%%%%%%%%%%%%%%%%%%%%%
% \paragraph{Command Line Processing.}
%
% The following three command lines generate the output files
% |cdocscld|, |cdocscl1| and |cdocscl2|
% which should be identical to
% |cdocsdrf|, |cdocsch1| and |cdocsfn2|, respectively:
% \begin{center}
% \begin{tabular}{l}
% |latex -jobname cdocscld \|\\
% |  "\def\version{draft}\input{childdoc.def}\childdocforward{cdocsamp}"|\\
% |latex -jobname cdocscl1 \|\\
% |  "\input{childdoc.def}\childdocforward[cdocsamp]{cdocsch1}"|\\
% |latex -jobname cdocscl2 \|\\
% |  "\def\version{final}\input{childdoc.def}\childdocforward{cdocsch2}"|
% \end{tabular}
% \end{center}
% Note that the trailing backslash on each first line
% merely continues the input to the second line
% (for convenient cut ant paste).
% Furthermore, the command |latex| can be replaced by any
% of its alternative versions such as |pdflatex|.
%
% %%%%%%%%%%%%%%%%%%%%%%%%%%%%%%%%%%%%%%%%%%%%%%%%%%%%%%%%%%%%%%%%%%%%%%%%%%%%%%
% %%%%%%%%%%%%%%%%%%%%%%%%%%%%%%%%%%%%%%%%%%%%%%%%%%%%%%%%%%%%%%%%%%%%%%%%%%%%%%
% \section{Implementation}
%\iffalse
%<*package>
%\fi
%
% This section describes the definitions file |childdoc.def|.

% The definitions cannot be loaded using |\usepackage| or |\RequirePackage|
% which has a mechanism to prevent loading a style file more than once.
% When loading the definitions by means of |\input|
% multiple instances have to be prevented manually:
%\iffalse
%This code needs to be before the `\ProvidesFile' directive
%which is defined at the beginning of this file.
%Therefore it is also placed there and commented out here.
%</package>
%<*discard>
%\fi
%    \begin{macrocode}
\ifdefined\childdocmain\endinput\fi
%    \end{macrocode}
%\iffalse
%</discard>
%<*package>
%\fi
%
% \macro{\ifchilddoc}
% \macro{\ifchilddocmanual}
% The conditional |\ifchilddoc| tells whether a
% child (true) or main (false) document is being compiled.
% The conditional |\ifchilddocmanual| tells whether
% the |\includeonly| mechanism is used (false) or
% the selection of child files must be performed manually (true).
% The definitions initialise to false:
%    \begin{macrocode}
\newif\ifchilddoc
\newif\ifchilddocmanual
%    \end{macrocode}

% \macro{\childdocname}
% \macro{\childdocjob}
% The macro |\childdocname| stores the name of the main document
% to be compiled. The macro |\childdocjob| stores the name of
% the document on which the \LaTeX{} compiler was originally invoked.
% The content of |\jobname| cannot be compared
% to filenames specified in the source due to different catcodes.
% The following code rescans |\jobname|, stores the result
% in |\childdocname| and saves a copy in |\childdocjob|:
%    \begin{macrocode}
\edef\childdocname{\scantokens\expandafter{\jobname\noexpand}}
\let\childdocjob\childdocname
%    \end{macrocode}

% \macro{\childdocdisable}
% The macro |\childdocdisable| prevents the main file
% from being processed more than once.
% At this stage, the main document command |\childdocmain|
% is assumed to be called once again where it should do nothing.
% Any subsequent call to it should prevent
% a secondary processing of the main document
% It overwrites the forwarding commands
% |\childdocof| and |\childdocforward|
% with empty macros to prevent further inclusions of the main document:
%    \begin{macrocode}
\newcommand{\childdocdisable}
{
  \renewcommand{\childdocmain}[1]{\renewcommand{\childdocmain}[1]{\endinput}}
  \renewcommand{\childdocof}[1]{}
  \renewcommand{\childdocby}[2][]{}
  \renewcommand{\childdocforward}[2][]{}
  \renewcommand{\childdocdisable}{}
}
%    \end{macrocode}

% \macro{\childdocmain}
% The macro |\childdocmain| is to be called at the top of the main file
% with nothing or the main filename (without extension) as argument.
% First, it breaks loops.
% If the argument is not empty and does not match |\childdocname|
% (which is set by the first inclusion of |childdoc.def|),
% |\ifchilddoc| is set to true, |\includeonly| is applied to the child file
% and |\jobname| is set to the main file
% (for proper handling of |.aux| files):
%    \begin{macrocode}
\newcommand{\childdocmain}[1]
{
  \childdocdisable\childdocmain{}
  \if?#1?\else
    \begingroup
      \def\childdoctmp{#1}
      \ifx\childdoctmp\childdocname
        \def\childdoctmp{}
      \else
        \def\childdoctmp
        {
          \childdoctrue
          \includeonly{\childdocname}
          \def\childdocjob{#1}
          \def\jobname{#1}
        }
      \fi
      \expandafter
    \endgroup
    \childdoctmp
  \fi
}
%    \end{macrocode}

% \macro{\childdocof}
% The command |\childdocof| redirects
% compilation to the main file |#1|.
%    \begin{macrocode}
\newcommand{\childdocof}[1]
{
  \childdocdisable
  \childdoctrue
  \includeonly{\childdocname}
  \def\jobname{#1}
  \def\childdocjob{#1}
  \input{#1}
}
%    \end{macrocode}

% \macro{\childdocby}
% The command |\childdocby| ....
%    \begin{macrocode}
\newcommand{\childdocby}[2][]
{
  \childdocdisable
  \childdoctrue
  \childdocmanualtrue
  \if?#1?\else
    \def\jobname{#2}
  \fi
  \def\childdocjob{#2}
  \input{#2}
  \endinput
}
%    \end{macrocode}

% \macro{\childdocforward}
% The command |\childdocforward| redirects
% compilation to the main file or
% (if the optional argument is given) a child file.
% Parameters are set as if the main file
% or a child file starting with |\childdocof| was compiled.
% Then compilation is handed over to the main file:
%    \begin{macrocode}
\newcommand{\childdocforward}[2][]
{
  \begingroup
    \if?#1?
      \def\childdoctmp
      {
        \def\childdocname{#2}
        \def\childdocjob{#2}
        \def\jobname{#2}
        \input{#2}
        \endinput
      }
    \else
      \def\childdoctmp
      {
        \childdocdisable
        \def\childdocname{#2}
        \childdoctrue
        \includeonly{#2}
        \def\childdocjob{#1}
        \def\jobname{#1}
        \input{#1}
        \endinput
      }
    \fi
    \expandafter
  \endgroup
  \childdoctmp
}
%    \end{macrocode}

% \macro{\childdocforwardprefix}
% The command |\childdocforwardprefix| redirects
% compilation to the main or a child file by means of a pattern.
% The prefix |#1| in the current filename is replaced by |#2|
% and the suffix of the current filename is kept
% (it is assumed that the filename does not contain the substring `|~~~|'
% which is used as a delimiter).
% Compilation is handed over to the new file by |\childdocforward|:
%    \begin{macrocode}
\newcommand{\childdocforwardprefix}[3][]
{
  \begingroup
    \def\childdocextract #2##1~~~{\def\childdoctmp{\childdocforward[#1]{#3##1}}}
    \expandafter\childdocextract\childdocname~~~
    \expandafter
  \endgroup
  \childdoctmp
}
%    \end{macrocode}

% \macro{\childdoc}
% The deprecated macro |\childdoc| is a legacy version of |\childdocmain|:
%    \begin{macrocode}
\newcommand{\childdoc}{\childdocmain}
%    \end{macrocode}

% \macro{\childdocredirect}
% The deprecated macro |\childdocredirect| is a legacy version
% of |\childdocforward| and |\childdocforwardprefix|:
%    \begin{macrocode}
\newcommand{\childdocredirect}[2][]
{
  \begingroup
    \if?#1?
      \def\childdoctmp{\childdocforward{#2}}
    \else
      \def\childdoctmp{\childdocforwardprefix{#1}{#2}}
    \fi
    \expandafter
  \endgroup
  \childdoctmp
}
%    \end{macrocode}

%\iffalse
%</package>
%\fi
%
\endinput
|\\
|\childdocforward[|\textit{main}|]{|\textit{dest}|}|\\
\end{tabular}
\end{center}
%
The argument \textit{dest} is the destination file
(without extension).
It should be the main file or one of the child files.
Note that further \textsf{childdoc} directives
such as |\childdocof| and |\childdocforward|
in the indicated file will be processed in this form.
The optional argument \textit{main}
passes on directly to the main file \textit{main}
while pretending to compile the child \textit{dest}.
This form behaves as if \textit{dest}
issues |\childdocof{|\textit{main}|}| right away,
and no further \textsf{childdoc} directives will be processed.

%%%%%%%%%%%%%%%%%%%%%%%%%%%%%%%%%%%%%%%%
\DescribeMacro{\...prefix}
In the alternative form |\childdocforwardprefix|,
%
\begin{center}
\begin{tabular}{l}
|% \iffalse
%
% childdoc.dtx Copyright (C) 2017-2018 Niklas Beisert
%
% This work may be distributed and/or modified under the
% conditions of the LaTeX Project Public License, either version 1.3
% of this license or (at your option) any later version.
% The latest version of this license is in
%   http://www.latex-project.org/lppl.txt
% and version 1.3 or later is part of all distributions of LaTeX
% version 2005/12/01 or later.
%
% This work has the LPPL maintenance status `maintained'.
%
% The Current Maintainer of this work is Niklas Beisert.
%
% This work consists of the files childdoc.dtx and childdoc.ins
% and the derived files childdoc.def and cdocsamp.tex with
% cdocsch1.tex, cdocsch2.tex, cdocsdrf.tex, cdocsfn1.tex, cdocsfn2.tex.
%
%<package>\ifdefined\childdocmain\endinput\fi
%<package>\ProvidesFile{childdoc.def}[2018/12/30 v2.0 child document driver]
%<samplemain>\ProvidesFile{cdocsamp.tex}[2018/12/30 v2.0 sample for childdoc]
%<*driver>
%\ProvidesFile{childdoc.drv}[2018/12/30 v2.0 childdoc reference manual file]
\PassOptionsToClass{10pt,a4paper}{article}
\documentclass{ltxdoc}

\usepackage[margin=35mm]{geometry}
\usepackage{hyperref}
\usepackage{hyperxmp}
\usepackage[usenames]{color}

\hypersetup{colorlinks=true}
\hypersetup{pdfstartview=FitH}
\hypersetup{pdfpagemode=UseNone}
\hypersetup{pdfsource={}}
\hypersetup{pdflang={en-UK}}
\hypersetup{pdfcopyright={Copyright 2017-2018 Niklas Beisert.
  This work may be distributed and/or modified under the
  conditions of the LaTeX Project Public License, either version 1.3
  of this license or (at your option) any later version.}}
\hypersetup{pdflicenseurl={http://www.latex-project.org/lppl.txt}}
\hypersetup{pdfcontactaddress={ETH Zurich, ITP, HIT K,
  Wolfgang-Pauli-Strasse 27}}
\hypersetup{pdfcontactpostcode={8093}}
\hypersetup{pdfcontactcity={Zurich}}
\hypersetup{pdfcontactcountry={Switzerland}}
\hypersetup{pdfcontactemail={nbeisert@itp.phys.ethz.ch}}
\hypersetup{pdfcontacturl={http://people.phys.ethz.ch/\xmptilde nbeisert/}}

\newcommand{\secref}[1]{\hyperref[#1]{section \ref*{#1}}}

\parskip1ex
\parindent0pt
\let\olditemize\itemize
\def\itemize{\olditemize\parskip0pt}

\begin{document}

\title{The \textsf{childdoc} Package}
\hypersetup{pdftitle={The childdoc Package}}
\author{Niklas Beisert\\[2ex]
  Institut f\"ur Theoretische Physik\\
  Eidgen\"ossische Technische Hochschule Z\"urich\\
  Wolfgang-Pauli-Strasse 27, 8093 Z\"urich, Switzerland\\[1ex]
  \href{mailto:nbeisert@itp.phys.ethz.ch}
  {\texttt{nbeisert@itp.phys.ethz.ch}}}
\hypersetup{pdfauthor={Niklas Beisert}}
\hypersetup{pdfsubject={Manual for the LaTeX2e Package childdoc}}
\date{30 December 2018, \textsf{v2.0}}
\maketitle

\begin{abstract}\noindent
\textsf{childdoc} is a \LaTeXe{} package
that enables the direct compilation
of document sections included by |\include|
to individual files.
\end{abstract}

\begingroup
\parskip0ex
\tableofcontents
\endgroup

%%%%%%%%%%%%%%%%%%%%%%%%%%%%%%%%%%%%%%%%%%%%%%%%%%%%%%%%%%%%%%%%%%%%%%%%%%%%%%%%
%%%%%%%%%%%%%%%%%%%%%%%%%%%%%%%%%%%%%%%%%%%%%%%%%%%%%%%%%%%%%%%%%%%%%%%%%%%%%%%%
\section{Introduction}

\LaTeX{} provides a mechanism to structure a large document (such as a book)
into a main file and several child files (containing the chapters)
using the |\include| command.
This mechanism is beneficial for documents
which span hundreds of pages in order to
make the source file(s) more manageable.
Moreover, compilation can be restricted to
selected child files by means of the |\includeonly| command.
The latter feature can be used to reduce the compilation time while editing
(this was significantly more useful in the earlier days of \LaTeX{})
or to generate a smaller document which is easier to navigate.
Another application of |\includeonly| is to generate
documents consisting of selected parts of the complete document.

However, there are a few drawbacks of the plain |\include| mechanism:
\begin{itemize}
\item
The child files cannot be compiled on their own,
they can only be compiled via the main file.
A naive editing environment
(such as a text editor with an option
to have the current file processed by \LaTeX)
may require one to switch to the main file before compiling;
attempting to compile the child file produces errors.
\item
The main file must be modified (each time)
to adjust the |\includeonly| command
to the present needs. This easily leaves the main file in a messy state.
\item
The generated document will always carry the filename
of the main document. This is inconvenient if
several child files are to be compiled and
to be kept for distribution.
\end{itemize}

The present package provides a simple interface
to make child files individually compilable by \LaTeX{}.
Compiling a child file then has the same effect as compiling
the main file with an |\includeonly| command
to select the appropriate child.
Moreover the generated document will carry the name of the child
rather than the main file.
This resolves all three above issues.

This feature is meant to make the editing of books,
thesis documents and lecture notes somewhat more convenient.
However, the package can also be used efficiently for
composing a series of documents (such as exercise sheets)
which are typically distributed individually.
It then assists the author in generating the individual documents
(potentially in different versions)
as well as a document containing the collected series.
Another application is in developing style files
or other kinds of included material
where compilation of the style file could redirect
to a sample or test file.

%%%%%%%%%%%%%%%%%%%%%%%%%%%%%%%%%%%%%%%%%%%%%%%%%%%%%%%%%%%%%%%%%%%%%%%%%%%%%%%%
%%%%%%%%%%%%%%%%%%%%%%%%%%%%%%%%%%%%%%%%%%%%%%%%%%%%%%%%%%%%%%%%%%%%%%%%%%%%%%%%
\section{Usage}

First of all, the package \textsf{childdoc} is \emph{not} a standard
\LaTeXe{} |.sty| style file! Therefore it needs to be invoked in
a non-standard way.

%%%%%%%%%%%%%%%%%%%%%%%%%%%%%%%%%%%%%%%%%%%%%%%%%%%%%%%%%%%%%%%%%%%%%%%%%%%%%%%%
\subsection{Included Files}
\label{sec:include}

%%%%%%%%%%%%%%%%%%%%%%%%%%%%%%%%%%%%%%%%
\DescribeMacro{\childdocmain}
To use the package, add the commands
\begin{center}
\begin{tabular}{l}
|\input{childdoc.def}|\\
|\childdocmain{}|\\
\end{tabular}
\end{center}
at the very top of the main \LaTeX{} file,
in particular \emph{before} the |\documentclass| statement!
The argument of |\childdocmain| should be left empty
(but it must be present).

%%%%%%%%%%%%%%%%%%%%%%%%%%%%%%%%%%%%%%%%
\DescribeMacro{\childdocof}
Furthermore, add the commands
\begin{center}
\begin{tabular}{l}
|\input{childdoc.def}|\\
|\childdocof{|\textit{main}|}|\\
\end{tabular}
\end{center}
at the top of every child file \textit{child}
which is included by |\include{|\textit{child}|}|
from within the main file
(or at least for those files to be compiled individually).
The argument \textit{main} must be the filename of the main file.

There are a couple of
considerations in setting up the main and child documents:

%%%%%%%%%%%%%%%%%%%%%%%%%%%%%%%%%%%%%%%%
\paragraph{Restrictions.}

Please note the following restrictions:
\begin{itemize}
\item
|\childdocmain| must be called with one argument \textit{main}
to ensure compatibility with earlier version of the package.
It must either be empty (|\childdocmain{}|)
or precisely match the filename of the main file in which it is specified.
See \secref{sec:detection} for further information.
\item
The filename \textit{main} must be specified without the |.tex| extension.
\item
The filename \textit{main} is case sensitive
(even in case-insensitive file systems)
due to internal string comparison.
\item
The argument \textit{main} should be fully expanded, it cannot be a macro.
\item
Subdirectories and special characters should be avoided in filenames.
\item
The command |\childdocmain{|\textit{main}|}| must be followed by a whitespace.
It should not be followed immediately by another command
or by a comment mark `|%|'.
This is because the \TeX{} parser reads the token immediately following
the argument of |\childdocmain| and puts it
at the beginning of every child section;
however, a white\-space is ignored.
\end{itemize}

%%%%%%%%%%%%%%%%%%%%%%%%%%%%%%%%%%%%%%%%
\paragraph{Content of Main File.}

It is advisable to place all content in the child files included by |\include|.
Any output contained in the main file will appear in all child documents
unless suppressed manually;
it cannot be suppressed automatically by the |\includeonly| directive
and thus should normally be avoided.
A method to include some content in the main file
by means of conditional processing is described in \secref{sec:conditional}.

%%%%%%%%%%%%%%%%%%%%%%%%%%%%%%%%%%%%%%%%
\paragraph{Page Numbering.}

When only a part of the document is compiled,
the appropriate numbering of pages
(as well as other status parameters)
is determined from the |.aux| files.
The latter contain information from previous passes.
However this information needs to propagate through
all intermediate child documents.
Therefore the page numbering in child documents may well
be inconsistent until the complete document is compiled at least once.

A useful (if unconventional) way to always ensure a consistent
page numbering is to restart the numbering in each child document
and denote the pages by `\textit{child}|.|\textit{page}'
where \textit{child} represents the chapter/section number of the child file.
This can be achieved by the command
|\numberwithin{page}{|\textit{child}|}|
of the \textsf{amsmath} package
where \textit{child} can be |chapter| or |section|
depending on the chosen structuring.
Alternatively, one can modify the macro |\thepage| appropriately
and reset the counter |page| at the start of each child file.

%%%%%%%%%%%%%%%%%%%%%%%%%%%%%%%%%%%%%%%%%%%%%%%%%%%%%%%%%%%%%%%%%%%%%%%%%%%%%%%%
\subsection{Conditional Processing}
\label{sec:conditional}

The package provides a mechanism to compile different versions
of a document. To customise the versions further some conditional processing
can come in handy to distinguish which version is being compiled.
The package provides two macros to describe the compilation context:

%%%%%%%%%%%%%%%%%%%%%%%%%%%%%%%%%%%%%%%%
\DescribeMacro{\ifchilddoc}
The conditional |\ifchilddoc| distinguishes between the compilation of
child documents and the main document:
%
\begin{center}
|\ifchilddoc |\textit{child-code}| |[|\||else |\textit{main-code}]| \||fi|
\end{center}

%%%%%%%%%%%%%%%%%%%%%%%%%%%%%%%%%%%%%%%%
\DescribeMacro{\childdocname}
\DescribeMacro{\childdocjob}
The macro |\childdocname| contains the filename (without extension)
of the main or child file being processed.
Note that |\childdocjob| will always contain the name of the main file.

%%%%%%%%%%%%%%%%%%%%%%%%%%%%%%%%%%%%%%%%
\paragraph{Title Page.}

Conditional processing can be used to include a title or banner page
in the main document when proper precautions are taken.
Importantly, the code in the main file should ensure that the page counter
(as well as other status parameters which are stored in the |.aux| files)
takes the same value after the conditional processing.
Otherwise the page numbers may take divergent values
depending on which part is compiled.

For example, a title page could be declared by:
%
\begin{center}
\begin{tabular}{l}
|\ifchilddoc\||else|\\
|\addtocounter{page}{-1}|\\
\textit{code for title page}\\
|\newpage|\\
|\||fi|
\end{tabular}
\end{center}
%
A banner page for the child documents can be generated by:
%
\begin{center}
\begin{tabular}{l}
|\ifchilddoc|\\
|\addtocounter{page}{-1}|\\
\textit{code for banner page}\\
|\newpage|\\
|\||fi|
\end{tabular}
\end{center}
%
Here one could write a message such as:
\begin{center}
|This is the part \childdocname{} of \childdocjob{}.|
\end{center}

%%%%%%%%%%%%%%%%%%%%%%%%%%%%%%%%%%%%%%%%%%%%%%%%%%%%%%%%%%%%%%%%%%%%%%%%%%%%%%%%
\subsection{Flags}
\label{sec:flags}

The package makes it easy to generate different versions
of the main or child documents.
To this end compilation flags can be defined
and assigned different default values.
They will be particularly useful in conjunction
with the forwarding mechanism described in \secref{sec:forward}.

For example, it may be useful to have a flag |\version|
which can be set to |draft| or |final|.
The document source will contain some conditional code
depending on the value of |\version|.
Suppose further, the flag should default to |final| for the main file
and to |draft| for child files
which is a natural assignment for editing the document.
This is achieved by placing the following code
in the preamble of the main document
(below the |\childdocmain| directive):
%
\begin{center}
\begin{tabular}{l}
|\ifchilddoc|\\
|\providecommand{\version}{draft}|\\
|\||else|\\
|\providecommand{\version}{final}|\\
|\||fi|
\end{tabular}
\end{center}
%
The definition by |\providecommand| makes sure
that previous definitions are not overwritten.
Further statements |\providecommand{\version}{...}|
can thus be added before the above code to override it.

For the main file, one might add a line
(between |\childdocmain| and the above block)
%
\begin{center}
|%\ifchilddoc\||else\providecommand{\version}{draft}\||fi|
\end{center}
%
which can be uncommented to produce a draft version.
Likewise one can add a line to the very top of a child file
(above the |\childdocof{|\textit{main}|}| directive)
%
\begin{center}
|%\providecommand{\version}{final}|
\end{center}
%
which can be uncommented to produce the final version of this child document.

%%%%%%%%%%%%%%%%%%%%%%%%%%%%%%%%%%%%%%%%%%%%%%%%%%%%%%%%%%%%%%%%%%%%%%%%%%%%%%%%
\subsection{Forwarding}
\label{sec:forward}

Different versions of the main or child documents
using compilation flags as described in \secref{sec:flags}
can be (permanently) stored in different files
for convenient compilation, viewing and distribution.
To this end, the package defines a command
to pass on compilation to a different file:

%%%%%%%%%%%%%%%%%%%%%%%%%%%%%%%%%%%%%%%%
\DescribeMacro{\childdocforward}
The command |\childdocforward| redirects processing to
another source file:
%
\begin{center}
\begin{tabular}{l}
|\input{childdoc.def}|\\
|\childdocforward[|\textit{main}|]{|\textit{dest}|}|\\
\end{tabular}
\end{center}
%
The argument \textit{dest} is the destination file
(without extension).
It should be the main file or one of the child files.
Note that further \textsf{childdoc} directives
such as |\childdocof| and |\childdocforward|
in the indicated file will be processed in this form.
The optional argument \textit{main}
passes on directly to the main file \textit{main}
while pretending to compile the child \textit{dest}.
This form behaves as if \textit{dest}
issues |\childdocof{|\textit{main}|}| right away,
and no further \textsf{childdoc} directives will be processed.

%%%%%%%%%%%%%%%%%%%%%%%%%%%%%%%%%%%%%%%%
\DescribeMacro{\...prefix}
In the alternative form |\childdocforwardprefix|,
%
\begin{center}
\begin{tabular}{l}
|\input{childdoc.def}|\\
|\childdocforwardprefix[|\textit{main}|]{|\textit{prefix}|}{|\textit{dest}|}|
\end{tabular}
\end{center}
%
the destination file is determined by a pattern
depending on the current file:
To make this work, the current file must be called
`{\textit{prefix}\hspace{0.2em}\textit{suffix}}'
with \textit{prefix} matching precisely the argument.
Processing is then passed on to the file
`{\textit{dest}\hspace{0.2em}\textit{suffix}}'.
Surely, the same effect is achieved by
directly specifying the
argument `{\textit{dest}\hspace{0.2em}\textit{suffix}}'
in the first form.
However, that requires to set up a different file
for each child. With the alternative form of the command
all these files can have exactly the same content
which simplifies setting them up and maintaining them.

For example, the following file |draft.tex|
with a compilation flag |\version| as described in \secref{sec:flags}
compiles the main document as a draft:
%
\begin{center}
\begin{tabular}{l}
|\def\version{draft}|\\
|\input{childdoc.def}|\\
|\childdocforward{|\textit{main}|}|
\end{tabular}
\end{center}
%
Likewise, the following files |final|\textit{nn}|.tex|
compile the final version of the child document
|child|\textit{nn}|.tex|:
%
\begin{center}
\begin{tabular}{l}
|\def\version{final}|\\
|\input{childdoc.def}|\\
|\childdocforwardprefix{final}{child}|
\end{tabular}
\end{center}
%

Note that when several versions of a main file and/or of each child file
are to be generated, it may be convenient to set up a |Makefile| or
shell script to automatise the process.

%%%%%%%%%%%%%%%%%%%%%%%%%%%%%%%%%%%%%%%%%%%%%%%%%%%%%%%%%%%%%%%%%%%%%%%%%%%%%%%%
\subsection{Command Line Processing}
\label{sec:commandline}

The effect of redirection files can also be achieved by invoking
the \LaTeX{} compiler with a more elaborate command line.
Most conveniently this should be done as part
of a shell script or a |Makefile|.

When using \textsf{childdoc} in the main file, the following
command lines effectively perform a redirection
(note that depending on the shell being used,
backslashes may have to be doubled: `|\|' $\to$ `|\\|'):
%
\begin{center}
|... -jobname "|\textit{target}|" |\\|"|[\textit{flags}]%
|\input{childdoc.def}\childdocforward[|\textit{main}|]{|\textit{dest}|}"|
\end{center}
%
Here \textit{target} is the name of the output file,
\textit{main} is the name of the main file
and \textit{dest} is the name of the main or child file to be processed
(all filenames without extensions).
The optional argument \textit{main} can be omitted
if \textit{main} matches \textit{dest}.
Optionally, compilation \textit{flags} can be defined via |\def| commands.
This command line makes the \TeX{} engine believe
it is compiling the file \textit{target}
whose content is specified as the latter parameter.
The provided code then forwards the processing to
\textit{main} or \textit{dest} as described in \secref{sec:forward}.

%%%%%%%%%%%%%%%%%%%%%%%%%%%%%%%%%%%%%%%%%%%%%%%%%%%%%%%%%%%%%%%%%%%%%%%%%%%%%%%%
\subsection{Include by Input}
\label{sec:input}

Including child documents by |\include| has some restrictions by design.
Most notably, the content of a child document always occupies
its own set of pages; pages cannot be shared between child documents.
Usually, this behaviour makes perfect sense
because each child document contain an essential part of the document.
However, in some situations it may be desirable to compose
a document from a collection of parts
without having mandatory page breaks between then.
For this case, the package
provides a mechanism to include parts
by |\input| which can also be processed individually.
However, by construction this mechanism
requires manual handling of the content to be output.

%%%%%%%%%%%%%%%%%%%%%%%%%%%%%%%%%%%%%%%%
\DescribeMacro{\ifchilddocmanual}
The main file should be prepared as usual, see \secref{sec:include}.
However, the document body must make a distinction
between processing of an individual part and of the main document, e.g.:
%
\begin{center}
\begin{tabular}{l}
|\ifchilddocmanual|\\
|\input{\childdocname}|\\
|\||else|\\
\textit{document body with }|\input{|\textit{part}|}|\\
|\||fi|
\end{tabular}
\end{center}
%
The conditional |\ifchilddocmanual| is true whenever
a part to be included by |\input| is being compiled,
and the name of the part is stored in |\childdocname|.

%%%%%%%%%%%%%%%%%%%%%%%%%%%%%%%%%%%%%%%%
\DescribeMacro{\childdocby}
Each part to be included by |\input| should start with:
%
\begin{center}
\begin{tabular}{l}
|\input{childdoc.def}|\\
|\childdocby{|\textit{main}|}|\\
\end{tabular}
\end{center}
%
The directive |\childdocby| is similar to |\childdocof|
described in \secref{sec:include},
but the subsequent selection of content must be done manually.
To that end, both |\ifchilddoc| and |\ifchilddocmanual|
will be true upon processing of a part,
and the name of the part is stored in |\childdocname|.
Note that |\jobname| will be set to the filename of the current part
so that each part receives an individual |.aux| file
that does not interfere with the |.aux| file(s) of the main document.
This behaviour can be altered by the alternative form
|\childdocby[*]{|\textit{main}|}| (with a non-empty optional argument)
which uses the |.aux| file of the main document
by setting |\jobname| to \textit{main}.

%%%%%%%%%%%%%%%%%%%%%%%%%%%%%%%%%%%%%%%%%%%%%%%%%%%%%%%%%%%%%%%%%%%%%%%%%%%%%%%%
\subsection{Driver Development}
\label{sec:driver}

The \textsf{childdoc} mechanism can also be use for the development
of definition files such as \LaTeX{} styles or classes.
This case differs from the above setup with multiple parts
included by |\include| in that no |\includeonly| should be invoked.
This can be achieved by starting the include file
(before |\ProvidesPackage|) with:
%
\begin{center}
\begin{tabular}{l}
|\input{childdoc.def}|\\
|\childdocforward{|\textit{main}|}|\\
\end{tabular}
\end{center}
%
or alternatively with:
%
\begin{center}
\begin{tabular}{l}
|\input{childdoc.def}|\\
|\childdocby{|\textit{main}|}|\\
\end{tabular}
\end{center}
%
Both forms have slightly different effects as described above.
The main file is prepared as usual, see \secref{sec:include}.

%%%%%%%%%%%%%%%%%%%%%%%%%%%%%%%%%%%%%%%%%%%%%%%%%%%%%%%%%%%%%%%%%%%%%%%%%%%%%%%%
\subsection{Legacy Detection}
\label{sec:detection}

The directive |\childdocmain| in the main file can detect
whether the complete document or merely a child is to be compiled
even without using the directive |\childdocof|.
This method is deprecated because it is less robust
and there is no compelling reason to use it;
it is merely provided for backward compatibility
and it may be removed in future versions.

If the detection mechanism is to be used,
it is mandatory to correctly specify
the filename of the main file as the argument of |\childdocmain|:
%
\begin{center}
\begin{tabular}{l}
|\input{childdoc.def}|\\
|\childdocmain{|\textit{main}|}|\\
\end{tabular}
\end{center}
%
If |\jobname| does not match the argument \textit{main} of |\childdocmain|,
it is assumed that |\jobname| points to the child file to be compiled.
When using |\childdocmain| with the main file specified as argument,
it suffices to start a child file
with just |\input{|\textit{main}|}|
without loading of the package and using |\childdocof|.
If instead all processing is done
with the appropriate \textsf{childdoc} directives,
the argument of \textit{main} of |\childdocmain| can be empty.

An alternative version of the command line processing described
in \secref{sec:commandline} using the detection mechanism reads:
%
\begin{center}
|... -jobname "|\textit{target}|" "|[\textit{flags}]%
[|\def\jobname{|\textit{dest}|}|]|\input{|\textit{main}|}"|
\end{center}

%%%%%%%%%%%%%%%%%%%%%%%%%%%%%%%%%%%%%%%%%%%%%%%%%%%%%%%%%%%%%%%%%%%%%%%%%%%%%%%%
\subsection{Manual Code}
\label{sec:manual}

In case one cannot be certain whether the definitions file |childdoc.def|
is installed on the target \TeX{} distribution
and one prefers not to ship it,
it is conceivable to paste a few relevant commands into the sources.

To that end, drop all statements |\input{childdoc.def}|
and perform the replacements as outlined below.
Instead of |\childdocmain{|\textit{main}|}| add the following code
to the top of the main file:
%
\begin{center}
\begin{tabular}{l}
|\||ifdefined\childdocname\endinput\||fi\newif\ifchilddoc|\\
|\edef\childdocname{\scantokens\expandafter{\jobname\noexpand}}|\\
|\def\childdocmain{|\textit{main}|}\||ifx\childdocmain\childdocname\||else|\\
|\childdoctrue\includeonly{\childdocname}\let\jobname\childdocmain\||fi|\\
\end{tabular}
\end{center}
%
Instead of |\childdocof{|\textit{main}|}| just include the main file
at the top of each child file:
%
\begin{center}
|\input{|\textit{main}|}|
\end{center}
%
A simple redirection |\childdocforward{|\textit{dest}|}| is achieved by:
%
\begin{center}
|\def\jobname{|\textit{dest}|}\input{\jobname}|
\end{center}
%
The redirection with prefix
|\childdocforwardprefix[|\textit{prefix}|]{|\textit{dest}|}|
is accomplished by:
%
\begin{center}
\begin{tabular}{l}
|{\edef\jobname{\scantokens\expandafter{\jobname\noexpand}}|\\
|\def\redirectjob |\textit{prefix}|#1~~~{\gdef\jobname{|\textit{dest}|#1}}|\\
|\expandafter\redirectjob\jobname~~~}\input{\jobname}|
\end{tabular}
\end{center}

In an alternative approach,
child documents can be compiled by a specific command line
without additional code or specific definitions:
%
\begin{center}
|... -jobname "|\textit{target}|" "|[\textit{flags}]%
|\includeonly{|\textit{dest}|}\input{|\textit{main}|}"|
\end{center}
%

%%%%%%%%%%%%%%%%%%%%%%%%%%%%%%%%%%%%%%%%%%%%%%%%%%%%%%%%%%%%%%%%%%%%%%%%%%%%%%%%
%%%%%%%%%%%%%%%%%%%%%%%%%%%%%%%%%%%%%%%%%%%%%%%%%%%%%%%%%%%%%%%%%%%%%%%%%%%%%%%%
\section{Information}

%%%%%%%%%%%%%%%%%%%%%%%%%%%%%%%%%%%%%%%%%%%%%%%%%%%%%%%%%%%%%%%%%%%%%%%%%%%%%%%%
\subsection{Copyright}

Copyright \copyright{} 2017--2018 Niklas Beisert

This work may be distributed and/or modified under the
conditions of the \LaTeX{} Project Public License, either version 1.3
of this license or (at your option) any later version.
The latest version of this license is in
  \url{http://www.latex-project.org/lppl.txt}
and version 1.3 or later is part of all distributions of \LaTeX{}
version 2005/12/01 or later.

This work has the LPPL maintenance status `maintained'.

The Current Maintainer of this work is Niklas Beisert.

This work consists of the files |README.txt|, |childdoc.ins| and |childdoc.dtx|
as well as the derived files |childdoc.def|, |cdocsamp.tex|
with |cdocsch1.tex|, |cdocsch2.tex|, |cdocspt3.tex|, |cdocspt4.tex|,
|cdocsdrf.tex|, |cdocsfn1.tex|, |cdocsfn2.tex|
as well as |childdoc.pdf|.

%%%%%%%%%%%%%%%%%%%%%%%%%%%%%%%%%%%%%%%%%%%%%%%%%%%%%%%%%%%%%%%%%%%%%%%%%%%%%%%%
\subsection{Files and Installation}

The package consists of the files:
%
\begin{center}
\begin{tabular}{ll}
    |README.txt|   & readme file \\
    |childdoc.ins| & installation file \\
    |childdoc.dtx| & source file \\
    |childdoc.def| & definition file \\
    |cdocsamp.tex| & sample main file \\
    |cdocsch1.tex| & sample include file \\
    |cdocsch2.tex| & sample include file \\
    |cdocspt3.tex| & sample part file \\
    |cdocspt4.tex| & sample part file \\
    |cdocsdrf.tex| & sample redirection file \\
    |cdocsfn1.tex| & sample redirection file \\
    |cdocsfn2.tex| & sample redirection file \\
    |childdoc.pdf| & manual
\end{tabular}
\end{center}
%
The distribution consists of the files
|README.txt|, |childdoc.ins| and |childdoc.dtx|.
%
\begin{itemize}
\item
Run (pdf)\LaTeX{} on |childdoc.dtx|
to compile the manual |childdoc.pdf| (this file).
\item
Run \LaTeX{} on |childdoc.ins| to create the definitions file |childdoc.def|
and the sample |cdocsamp.tex| with include files
|cdocsch1.tex|, |cdocsch2.tex|, |cdocspt3.tex|, |cdocspt4.tex|,
|cdocsdrf.tex|, |cdocsfn1.tex|, |cdocsfn2.tex|.
Then copy the file |childdoc.def| to an appropriate directory of your \LaTeX{}
distribution, e.g.\ \textit{texmf-root}|/tex/latex/childdoc|.
\end{itemize}

%%%%%%%%%%%%%%%%%%%%%%%%%%%%%%%%%%%%%%%%%%%%%%%%%%%%%%%%%%%%%%%%%%%%%%%%%%%%%%%%
\subsection{Related CTAN Packages}

There are several other packages which offer a similar functionality:
%
\begin{itemize}
\item
The packages
\href{http://ctan.org/pkg/docmute}{\textsf{docmute}},
\href{http://ctan.org/pkg/includex}{\textsf{includex}} and
\href{http://ctan.org/pkg/standalone}{\textsf{standalone}}
provide commands to include only the document body of
a child file thus allowing both files to be compiled individually.
\item
The packages \href{http://ctan.org/pkg/subdocs}{\textsf{subdocs}}
and \href{http://ctan.org/pkg/subfiles}{\textsf{subfiles}}
provide structures in which the main and child documents can be
encapsulated and allowing them to be compiled individually.
The inclusion mechanism is different from the conventional |\include|.
\item
The package \href{http://ctan.org/pkg/combine}{\textsf{combine}}
is an elaborate solution to combine several documents into one.
\end{itemize}
%
See also the CTAN topic \href{http://ctan.org/topic/subdocs}{\textsf{subdocs}}
for further related packages.
The present package differs from the above solutions in that
a document structure constructed with the conventional |\include| mechanism
just needs two extra commands at the top of every file
such that all constituent files can be compiled individually.

%%%%%%%%%%%%%%%%%%%%%%%%%%%%%%%%%%%%%%%%%%%%%%%%%%%%%%%%%%%%%%%%%%%%%%%%%%%%%%%%
%\subsection{Feature Suggestions}
%
%The following is a list of features which may be useful for future
%versions of this package:
%%
%\begin{itemize}
%\item
%\ldots
%\end{itemize}

%%%%%%%%%%%%%%%%%%%%%%%%%%%%%%%%%%%%%%%%%%%%%%%%%%%%%%%%%%%%%%%%%%%%%%%%%%%%%%%%
\subsection{Revision History}

%%%%%%%%%%%%%%%%%%%%%%%%%%%%%%%%%%%%%%%%
\paragraph{v2.0:} 2018/12/30

\begin{itemize}
\item
immediate forward processing
\item
added |\childdocby| mechanism
\item
manual restructured
\end{itemize}

%%%%%%%%%%%%%%%%%%%%%%%%%%%%%%%%%%%%%%%%
\paragraph{v1.6:} 2018/01/17

\begin{itemize}
\item
application for development of include files
\item
corrections to manual
\end{itemize}

%%%%%%%%%%%%%%%%%%%%%%%%%%%%%%%%%%%%%%%%
\paragraph{v1.5:} 2017/05/21

\begin{itemize}
\item
more complete structuring introduced
\item
|\childdocof| introduced
\item
|\childdoc| renamed to |\childdocmain|
\item
|\childredirect| renamed to |\childdocforward| and |\childdocforwardprefix|
and functionality expanded
\end{itemize}

%%%%%%%%%%%%%%%%%%%%%%%%%%%%%%%%%%%%%%%%
\paragraph{v1.0:} 2017/04/27

\begin{itemize}
\item
manual and install package
\item
first version published on CTAN
\end{itemize}

%%%%%%%%%%%%%%%%%%%%%%%%%%%%%%%%%%%%%%%%
\paragraph{v0.6:} 2017/04/26

\begin{itemize}
\item
redirection mechanism added
\end{itemize}

%%%%%%%%%%%%%%%%%%%%%%%%%%%%%%%%%%%%%%%%
\paragraph{v0.5:} 2017/04/26

\begin{itemize}
\item
functionality in definition file
\end{itemize}


%%%%%%%%%%%%%%%%%%%%%%%%%%%%%%%%%%%%%%%%%%%%%%%%%%%%%%%%%%%%%%%%%%%%%%%%%%%%%%%%
%%%%%%%%%%%%%%%%%%%%%%%%%%%%%%%%%%%%%%%%%%%%%%%%%%%%%%%%%%%%%%%%%%%%%%%%%%%%%%%%
%%%%%%%%%%%%%%%%%%%%%%%%%%%%%%%%%%%%%%%%%%%%%%%%%%%%%%%%%%%%%%%%%%%%%%%%%%%%%%%%
\appendix

\settowidth\MacroIndent{\rmfamily\scriptsize 000\ }

 \DocInput{childdoc.dtx}

\end{document}
%</driver>
% \fi
%
% %%%%%%%%%%%%%%%%%%%%%%%%%%%%%%%%%%%%%%%%%%%%%%%%%%%%%%%%%%%%%%%%%%%%%%%%%%%%%%
% %%%%%%%%%%%%%%%%%%%%%%%%%%%%%%%%%%%%%%%%%%%%%%%%%%%%%%%%%%%%%%%%%%%%%%%%%%%%%%
% \section{Sample}
%\iffalse
%<*samplemain>
%\fi
%
% The following presents a sample document
% with two chapters, two parts, a title page,
% a compile flag as well as three forwarding files to set the flag.
% It consists of eight |.tex| files:
% \begin{center}
% \begin{tabular}{ll}
% |cdocsamp.tex|&main file\\
% |cdocsch1.tex|&include file for chapter 1\\
% |cdocsch2.tex|&include file for chapter 2\\
% |cdocspt3.tex|&include file for part 3\\
% |cdocspt4.tex|&include file for part 4\\
% |cdocsdrf.tex|&forwarding file for main file in draft mode\\
% |cdocsfi1.tex|&forwarding file for final version of chapter 1\\
% |cdocsfi2.tex|&forwarding file for final version of chapter 2\\
% \end{tabular}
% \end{center}
% Each of the eight files can be compiled directly by the \LaTeX{} compiler.
%
% %%%%%%%%%%%%%%%%%%%%%%%%%%%%%%%%%%%%%%
% \paragraph{Main File.}
%
% The main file is called |cdocsamp.tex|.
%
% Load the \textsf{childdoc} definitions and
% declare the filename for the main document:
%    \begin{macrocode}
\input{childdoc.def}
\childdocmain{}
%    \end{macrocode}

% Optional override for |\version| flag:
%    \begin{macrocode}
%%\ifchilddoc\else\providecommand{\version}{draft}\fi
%    \end{macrocode}

% Define the default values for the |\version| flag
% (|final| for the main file and |draft| for childs):
%    \begin{macrocode}
\ifchilddoc
\providecommand{\version}{draft}
\else
\providecommand{\version}{final}
\fi
%    \end{macrocode}

% Load the standard document class:
%    \begin{macrocode}
\documentclass[12pt]{article}
%    \end{macrocode}

% Start the document body:
%    \begin{macrocode}
\begin{document}
%    \end{macrocode}

% Declare a title page.
% Print title, part of document being processed and version flag:
%    \begin{macrocode}
\addtocounter{page}{-1}
\begin{center}
{\LARGE\bfseries{}childdoc example\par}
\vspace{1cm}
\ifchilddoc
\ifchilddocmanual part\else chapter\fi:
`\childdocname' of `\childdocjob'\par
\else
main document: `\childdocjob'\par
\fi
version: \version\par
\end{center}
\newpage
%    \end{macrocode}

% Manually include selected file,
% otherwise process as usual:
%    \begin{macrocode}
\ifchilddocmanual
\section*{part `\childdocname'}
\input{\childdocname}
\else
%    \end{macrocode}

% Include the two chapters:
%    \begin{macrocode}
\include{cdocsch1}
\include{cdocsch2}
%    \end{macrocode}

% Include the two parts unless only chapters should be displayed:
%    \begin{macrocode}
\ifchilddoc\else
\section{part three}
\input{cdocspt3}
\section{part four}
\input{cdocspt4}
\fi
%    \end{macrocode}

% Process as usual until here:
%    \begin{macrocode}
\fi
%    \end{macrocode}

% End of document body:
%    \begin{macrocode}
\end{document}
%    \end{macrocode}
%\iffalse
%</samplemain>
%\fi
%
% %%%%%%%%%%%%%%%%%%%%%%%%%%%%%%%%%%%%%%
% \paragraph{Chapter Include Files.}
%
% The include files are called |cdocsch1.tex| and |cdocsch2.tex|.
%
%\iffalse
%<*samplechap1|samplechap2>
%\fi

% Optional override for |\version| flag:
%    \begin{macrocode}
%%\providecommand{\version}{final}
%    \end{macrocode}

% Include the main document:
%    \begin{macrocode}
\input{childdoc.def}
\childdocof{cdocsamp}
%    \end{macrocode}

%\iffalse
%</samplechap1|samplechap2>
%\fi
%
%\iffalse
%<*samplechap1>
%\fi
% Some text for chapter 1:
%    \begin{macrocode}
\section{one}
some text in chapter one
%    \end{macrocode}

%\iffalse
%</samplechap1>
%\fi
% Some text for chapter 2:
%\iffalse
%<*samplechap2>
%\fi
%    \begin{macrocode}
\section{two}
more text in chapter two
%    \end{macrocode}

%\iffalse
%</samplechap2>
%\fi
%
% %%%%%%%%%%%%%%%%%%%%%%%%%%%%%%%%%%%%%%
% \paragraph{Part Include Files.}
%
% The include files are called |cdocspt3.tex| and |cdocspt4.tex|.
%
%\iffalse
%<*samplepart3|samplepart4>
%\fi

% Optional override for |\version| flag:
%    \begin{macrocode}
%%\providecommand{\version}{final}
%    \end{macrocode}

% Include the main document:
%    \begin{macrocode}
\input{childdoc.def}
\childdocby{cdocsamp}
%    \end{macrocode}

%\iffalse
%</samplepart3|samplepart4>
%\fi
%
%\iffalse
%<*samplepart3>
%\fi
% Some text for part 3:
%    \begin{macrocode}
some text in part three
%    \end{macrocode}

%\iffalse
%</samplepart3>
%\fi
% Some text for part 4:
%\iffalse
%<*samplepart4>
%\fi
%    \begin{macrocode}
more text in part four
%    \end{macrocode}

%\iffalse
%</samplepart4>
%\fi
%
% %%%%%%%%%%%%%%%%%%%%%%%%%%%%%%%%%%%%%%
% \paragraph{Forwarding for a Complete Draft.}
%
% The following forwarding file |cdocsdrf.tex|
% compiles the main document in draft mode:
%\iffalse
%<*sampledraft>
%\fi
%    \begin{macrocode}
\def\version{draft}
\input{childdoc.def}
\childdocforward{cdocsamp}
%    \end{macrocode}

%\iffalse
%</sampledraft>
%\fi
%
% %%%%%%%%%%%%%%%%%%%%%%%%%%%%%%%%%%%%%%
% \paragraph{Forwarding for Final Version of the Chapters.}
%
% The following forwarding files |cdocsfn1.tex| and |cdocsfn2.tex|
% (with identical content)
% compile the final versions of the child documents
% |cdocsch1.tex| and |cdocsch2.tex|, respectively:
%\iffalse
%<*samplefinal>
%\fi
%    \begin{macrocode}
\def\version{final}
\input{childdoc.def}
\childdocforwardprefix[cdocsamp]{cdocsfn}{cdocsch}
%    \end{macrocode}

%\iffalse
%</samplefinal>
%\fi
%
% %%%%%%%%%%%%%%%%%%%%%%%%%%%%%%%%%%%%%%
% \paragraph{Command Line Processing.}
%
% The following three command lines generate the output files
% |cdocscld|, |cdocscl1| and |cdocscl2|
% which should be identical to
% |cdocsdrf|, |cdocsch1| and |cdocsfn2|, respectively:
% \begin{center}
% \begin{tabular}{l}
% |latex -jobname cdocscld \|\\
% |  "\def\version{draft}\input{childdoc.def}\childdocforward{cdocsamp}"|\\
% |latex -jobname cdocscl1 \|\\
% |  "\input{childdoc.def}\childdocforward[cdocsamp]{cdocsch1}"|\\
% |latex -jobname cdocscl2 \|\\
% |  "\def\version{final}\input{childdoc.def}\childdocforward{cdocsch2}"|
% \end{tabular}
% \end{center}
% Note that the trailing backslash on each first line
% merely continues the input to the second line
% (for convenient cut ant paste).
% Furthermore, the command |latex| can be replaced by any
% of its alternative versions such as |pdflatex|.
%
% %%%%%%%%%%%%%%%%%%%%%%%%%%%%%%%%%%%%%%%%%%%%%%%%%%%%%%%%%%%%%%%%%%%%%%%%%%%%%%
% %%%%%%%%%%%%%%%%%%%%%%%%%%%%%%%%%%%%%%%%%%%%%%%%%%%%%%%%%%%%%%%%%%%%%%%%%%%%%%
% \section{Implementation}
%\iffalse
%<*package>
%\fi
%
% This section describes the definitions file |childdoc.def|.

% The definitions cannot be loaded using |\usepackage| or |\RequirePackage|
% which has a mechanism to prevent loading a style file more than once.
% When loading the definitions by means of |\input|
% multiple instances have to be prevented manually:
%\iffalse
%This code needs to be before the `\ProvidesFile' directive
%which is defined at the beginning of this file.
%Therefore it is also placed there and commented out here.
%</package>
%<*discard>
%\fi
%    \begin{macrocode}
\ifdefined\childdocmain\endinput\fi
%    \end{macrocode}
%\iffalse
%</discard>
%<*package>
%\fi
%
% \macro{\ifchilddoc}
% \macro{\ifchilddocmanual}
% The conditional |\ifchilddoc| tells whether a
% child (true) or main (false) document is being compiled.
% The conditional |\ifchilddocmanual| tells whether
% the |\includeonly| mechanism is used (false) or
% the selection of child files must be performed manually (true).
% The definitions initialise to false:
%    \begin{macrocode}
\newif\ifchilddoc
\newif\ifchilddocmanual
%    \end{macrocode}

% \macro{\childdocname}
% \macro{\childdocjob}
% The macro |\childdocname| stores the name of the main document
% to be compiled. The macro |\childdocjob| stores the name of
% the document on which the \LaTeX{} compiler was originally invoked.
% The content of |\jobname| cannot be compared
% to filenames specified in the source due to different catcodes.
% The following code rescans |\jobname|, stores the result
% in |\childdocname| and saves a copy in |\childdocjob|:
%    \begin{macrocode}
\edef\childdocname{\scantokens\expandafter{\jobname\noexpand}}
\let\childdocjob\childdocname
%    \end{macrocode}

% \macro{\childdocdisable}
% The macro |\childdocdisable| prevents the main file
% from being processed more than once.
% At this stage, the main document command |\childdocmain|
% is assumed to be called once again where it should do nothing.
% Any subsequent call to it should prevent
% a secondary processing of the main document
% It overwrites the forwarding commands
% |\childdocof| and |\childdocforward|
% with empty macros to prevent further inclusions of the main document:
%    \begin{macrocode}
\newcommand{\childdocdisable}
{
  \renewcommand{\childdocmain}[1]{\renewcommand{\childdocmain}[1]{\endinput}}
  \renewcommand{\childdocof}[1]{}
  \renewcommand{\childdocby}[2][]{}
  \renewcommand{\childdocforward}[2][]{}
  \renewcommand{\childdocdisable}{}
}
%    \end{macrocode}

% \macro{\childdocmain}
% The macro |\childdocmain| is to be called at the top of the main file
% with nothing or the main filename (without extension) as argument.
% First, it breaks loops.
% If the argument is not empty and does not match |\childdocname|
% (which is set by the first inclusion of |childdoc.def|),
% |\ifchilddoc| is set to true, |\includeonly| is applied to the child file
% and |\jobname| is set to the main file
% (for proper handling of |.aux| files):
%    \begin{macrocode}
\newcommand{\childdocmain}[1]
{
  \childdocdisable\childdocmain{}
  \if?#1?\else
    \begingroup
      \def\childdoctmp{#1}
      \ifx\childdoctmp\childdocname
        \def\childdoctmp{}
      \else
        \def\childdoctmp
        {
          \childdoctrue
          \includeonly{\childdocname}
          \def\childdocjob{#1}
          \def\jobname{#1}
        }
      \fi
      \expandafter
    \endgroup
    \childdoctmp
  \fi
}
%    \end{macrocode}

% \macro{\childdocof}
% The command |\childdocof| redirects
% compilation to the main file |#1|.
%    \begin{macrocode}
\newcommand{\childdocof}[1]
{
  \childdocdisable
  \childdoctrue
  \includeonly{\childdocname}
  \def\jobname{#1}
  \def\childdocjob{#1}
  \input{#1}
}
%    \end{macrocode}

% \macro{\childdocby}
% The command |\childdocby| ....
%    \begin{macrocode}
\newcommand{\childdocby}[2][]
{
  \childdocdisable
  \childdoctrue
  \childdocmanualtrue
  \if?#1?\else
    \def\jobname{#2}
  \fi
  \def\childdocjob{#2}
  \input{#2}
  \endinput
}
%    \end{macrocode}

% \macro{\childdocforward}
% The command |\childdocforward| redirects
% compilation to the main file or
% (if the optional argument is given) a child file.
% Parameters are set as if the main file
% or a child file starting with |\childdocof| was compiled.
% Then compilation is handed over to the main file:
%    \begin{macrocode}
\newcommand{\childdocforward}[2][]
{
  \begingroup
    \if?#1?
      \def\childdoctmp
      {
        \def\childdocname{#2}
        \def\childdocjob{#2}
        \def\jobname{#2}
        \input{#2}
        \endinput
      }
    \else
      \def\childdoctmp
      {
        \childdocdisable
        \def\childdocname{#2}
        \childdoctrue
        \includeonly{#2}
        \def\childdocjob{#1}
        \def\jobname{#1}
        \input{#1}
        \endinput
      }
    \fi
    \expandafter
  \endgroup
  \childdoctmp
}
%    \end{macrocode}

% \macro{\childdocforwardprefix}
% The command |\childdocforwardprefix| redirects
% compilation to the main or a child file by means of a pattern.
% The prefix |#1| in the current filename is replaced by |#2|
% and the suffix of the current filename is kept
% (it is assumed that the filename does not contain the substring `|~~~|'
% which is used as a delimiter).
% Compilation is handed over to the new file by |\childdocforward|:
%    \begin{macrocode}
\newcommand{\childdocforwardprefix}[3][]
{
  \begingroup
    \def\childdocextract #2##1~~~{\def\childdoctmp{\childdocforward[#1]{#3##1}}}
    \expandafter\childdocextract\childdocname~~~
    \expandafter
  \endgroup
  \childdoctmp
}
%    \end{macrocode}

% \macro{\childdoc}
% The deprecated macro |\childdoc| is a legacy version of |\childdocmain|:
%    \begin{macrocode}
\newcommand{\childdoc}{\childdocmain}
%    \end{macrocode}

% \macro{\childdocredirect}
% The deprecated macro |\childdocredirect| is a legacy version
% of |\childdocforward| and |\childdocforwardprefix|:
%    \begin{macrocode}
\newcommand{\childdocredirect}[2][]
{
  \begingroup
    \if?#1?
      \def\childdoctmp{\childdocforward{#2}}
    \else
      \def\childdoctmp{\childdocforwardprefix{#1}{#2}}
    \fi
    \expandafter
  \endgroup
  \childdoctmp
}
%    \end{macrocode}

%\iffalse
%</package>
%\fi
%
\endinput
|\\
|\childdocforwardprefix[|\textit{main}|]{|\textit{prefix}|}{|\textit{dest}|}|
\end{tabular}
\end{center}
%
the destination file is determined by a pattern
depending on the current file:
To make this work, the current file must be called
`{\textit{prefix}\hspace{0.2em}\textit{suffix}}'
with \textit{prefix} matching precisely the argument.
Processing is then passed on to the file
`{\textit{dest}\hspace{0.2em}\textit{suffix}}'.
Surely, the same effect is achieved by
directly specifying the
argument `{\textit{dest}\hspace{0.2em}\textit{suffix}}'
in the first form.
However, that requires to set up a different file
for each child. With the alternative form of the command
all these files can have exactly the same content
which simplifies setting them up and maintaining them.

For example, the following file |draft.tex|
with a compilation flag |\version| as described in \secref{sec:flags}
compiles the main document as a draft:
%
\begin{center}
\begin{tabular}{l}
|\def\version{draft}|\\
|% \iffalse
%
% childdoc.dtx Copyright (C) 2017-2018 Niklas Beisert
%
% This work may be distributed and/or modified under the
% conditions of the LaTeX Project Public License, either version 1.3
% of this license or (at your option) any later version.
% The latest version of this license is in
%   http://www.latex-project.org/lppl.txt
% and version 1.3 or later is part of all distributions of LaTeX
% version 2005/12/01 or later.
%
% This work has the LPPL maintenance status `maintained'.
%
% The Current Maintainer of this work is Niklas Beisert.
%
% This work consists of the files childdoc.dtx and childdoc.ins
% and the derived files childdoc.def and cdocsamp.tex with
% cdocsch1.tex, cdocsch2.tex, cdocsdrf.tex, cdocsfn1.tex, cdocsfn2.tex.
%
%<package>\ifdefined\childdocmain\endinput\fi
%<package>\ProvidesFile{childdoc.def}[2018/12/30 v2.0 child document driver]
%<samplemain>\ProvidesFile{cdocsamp.tex}[2018/12/30 v2.0 sample for childdoc]
%<*driver>
%\ProvidesFile{childdoc.drv}[2018/12/30 v2.0 childdoc reference manual file]
\PassOptionsToClass{10pt,a4paper}{article}
\documentclass{ltxdoc}

\usepackage[margin=35mm]{geometry}
\usepackage{hyperref}
\usepackage{hyperxmp}
\usepackage[usenames]{color}

\hypersetup{colorlinks=true}
\hypersetup{pdfstartview=FitH}
\hypersetup{pdfpagemode=UseNone}
\hypersetup{pdfsource={}}
\hypersetup{pdflang={en-UK}}
\hypersetup{pdfcopyright={Copyright 2017-2018 Niklas Beisert.
  This work may be distributed and/or modified under the
  conditions of the LaTeX Project Public License, either version 1.3
  of this license or (at your option) any later version.}}
\hypersetup{pdflicenseurl={http://www.latex-project.org/lppl.txt}}
\hypersetup{pdfcontactaddress={ETH Zurich, ITP, HIT K,
  Wolfgang-Pauli-Strasse 27}}
\hypersetup{pdfcontactpostcode={8093}}
\hypersetup{pdfcontactcity={Zurich}}
\hypersetup{pdfcontactcountry={Switzerland}}
\hypersetup{pdfcontactemail={nbeisert@itp.phys.ethz.ch}}
\hypersetup{pdfcontacturl={http://people.phys.ethz.ch/\xmptilde nbeisert/}}

\newcommand{\secref}[1]{\hyperref[#1]{section \ref*{#1}}}

\parskip1ex
\parindent0pt
\let\olditemize\itemize
\def\itemize{\olditemize\parskip0pt}

\begin{document}

\title{The \textsf{childdoc} Package}
\hypersetup{pdftitle={The childdoc Package}}
\author{Niklas Beisert\\[2ex]
  Institut f\"ur Theoretische Physik\\
  Eidgen\"ossische Technische Hochschule Z\"urich\\
  Wolfgang-Pauli-Strasse 27, 8093 Z\"urich, Switzerland\\[1ex]
  \href{mailto:nbeisert@itp.phys.ethz.ch}
  {\texttt{nbeisert@itp.phys.ethz.ch}}}
\hypersetup{pdfauthor={Niklas Beisert}}
\hypersetup{pdfsubject={Manual for the LaTeX2e Package childdoc}}
\date{30 December 2018, \textsf{v2.0}}
\maketitle

\begin{abstract}\noindent
\textsf{childdoc} is a \LaTeXe{} package
that enables the direct compilation
of document sections included by |\include|
to individual files.
\end{abstract}

\begingroup
\parskip0ex
\tableofcontents
\endgroup

%%%%%%%%%%%%%%%%%%%%%%%%%%%%%%%%%%%%%%%%%%%%%%%%%%%%%%%%%%%%%%%%%%%%%%%%%%%%%%%%
%%%%%%%%%%%%%%%%%%%%%%%%%%%%%%%%%%%%%%%%%%%%%%%%%%%%%%%%%%%%%%%%%%%%%%%%%%%%%%%%
\section{Introduction}

\LaTeX{} provides a mechanism to structure a large document (such as a book)
into a main file and several child files (containing the chapters)
using the |\include| command.
This mechanism is beneficial for documents
which span hundreds of pages in order to
make the source file(s) more manageable.
Moreover, compilation can be restricted to
selected child files by means of the |\includeonly| command.
The latter feature can be used to reduce the compilation time while editing
(this was significantly more useful in the earlier days of \LaTeX{})
or to generate a smaller document which is easier to navigate.
Another application of |\includeonly| is to generate
documents consisting of selected parts of the complete document.

However, there are a few drawbacks of the plain |\include| mechanism:
\begin{itemize}
\item
The child files cannot be compiled on their own,
they can only be compiled via the main file.
A naive editing environment
(such as a text editor with an option
to have the current file processed by \LaTeX)
may require one to switch to the main file before compiling;
attempting to compile the child file produces errors.
\item
The main file must be modified (each time)
to adjust the |\includeonly| command
to the present needs. This easily leaves the main file in a messy state.
\item
The generated document will always carry the filename
of the main document. This is inconvenient if
several child files are to be compiled and
to be kept for distribution.
\end{itemize}

The present package provides a simple interface
to make child files individually compilable by \LaTeX{}.
Compiling a child file then has the same effect as compiling
the main file with an |\includeonly| command
to select the appropriate child.
Moreover the generated document will carry the name of the child
rather than the main file.
This resolves all three above issues.

This feature is meant to make the editing of books,
thesis documents and lecture notes somewhat more convenient.
However, the package can also be used efficiently for
composing a series of documents (such as exercise sheets)
which are typically distributed individually.
It then assists the author in generating the individual documents
(potentially in different versions)
as well as a document containing the collected series.
Another application is in developing style files
or other kinds of included material
where compilation of the style file could redirect
to a sample or test file.

%%%%%%%%%%%%%%%%%%%%%%%%%%%%%%%%%%%%%%%%%%%%%%%%%%%%%%%%%%%%%%%%%%%%%%%%%%%%%%%%
%%%%%%%%%%%%%%%%%%%%%%%%%%%%%%%%%%%%%%%%%%%%%%%%%%%%%%%%%%%%%%%%%%%%%%%%%%%%%%%%
\section{Usage}

First of all, the package \textsf{childdoc} is \emph{not} a standard
\LaTeXe{} |.sty| style file! Therefore it needs to be invoked in
a non-standard way.

%%%%%%%%%%%%%%%%%%%%%%%%%%%%%%%%%%%%%%%%%%%%%%%%%%%%%%%%%%%%%%%%%%%%%%%%%%%%%%%%
\subsection{Included Files}
\label{sec:include}

%%%%%%%%%%%%%%%%%%%%%%%%%%%%%%%%%%%%%%%%
\DescribeMacro{\childdocmain}
To use the package, add the commands
\begin{center}
\begin{tabular}{l}
|\input{childdoc.def}|\\
|\childdocmain{}|\\
\end{tabular}
\end{center}
at the very top of the main \LaTeX{} file,
in particular \emph{before} the |\documentclass| statement!
The argument of |\childdocmain| should be left empty
(but it must be present).

%%%%%%%%%%%%%%%%%%%%%%%%%%%%%%%%%%%%%%%%
\DescribeMacro{\childdocof}
Furthermore, add the commands
\begin{center}
\begin{tabular}{l}
|\input{childdoc.def}|\\
|\childdocof{|\textit{main}|}|\\
\end{tabular}
\end{center}
at the top of every child file \textit{child}
which is included by |\include{|\textit{child}|}|
from within the main file
(or at least for those files to be compiled individually).
The argument \textit{main} must be the filename of the main file.

There are a couple of
considerations in setting up the main and child documents:

%%%%%%%%%%%%%%%%%%%%%%%%%%%%%%%%%%%%%%%%
\paragraph{Restrictions.}

Please note the following restrictions:
\begin{itemize}
\item
|\childdocmain| must be called with one argument \textit{main}
to ensure compatibility with earlier version of the package.
It must either be empty (|\childdocmain{}|)
or precisely match the filename of the main file in which it is specified.
See \secref{sec:detection} for further information.
\item
The filename \textit{main} must be specified without the |.tex| extension.
\item
The filename \textit{main} is case sensitive
(even in case-insensitive file systems)
due to internal string comparison.
\item
The argument \textit{main} should be fully expanded, it cannot be a macro.
\item
Subdirectories and special characters should be avoided in filenames.
\item
The command |\childdocmain{|\textit{main}|}| must be followed by a whitespace.
It should not be followed immediately by another command
or by a comment mark `|%|'.
This is because the \TeX{} parser reads the token immediately following
the argument of |\childdocmain| and puts it
at the beginning of every child section;
however, a white\-space is ignored.
\end{itemize}

%%%%%%%%%%%%%%%%%%%%%%%%%%%%%%%%%%%%%%%%
\paragraph{Content of Main File.}

It is advisable to place all content in the child files included by |\include|.
Any output contained in the main file will appear in all child documents
unless suppressed manually;
it cannot be suppressed automatically by the |\includeonly| directive
and thus should normally be avoided.
A method to include some content in the main file
by means of conditional processing is described in \secref{sec:conditional}.

%%%%%%%%%%%%%%%%%%%%%%%%%%%%%%%%%%%%%%%%
\paragraph{Page Numbering.}

When only a part of the document is compiled,
the appropriate numbering of pages
(as well as other status parameters)
is determined from the |.aux| files.
The latter contain information from previous passes.
However this information needs to propagate through
all intermediate child documents.
Therefore the page numbering in child documents may well
be inconsistent until the complete document is compiled at least once.

A useful (if unconventional) way to always ensure a consistent
page numbering is to restart the numbering in each child document
and denote the pages by `\textit{child}|.|\textit{page}'
where \textit{child} represents the chapter/section number of the child file.
This can be achieved by the command
|\numberwithin{page}{|\textit{child}|}|
of the \textsf{amsmath} package
where \textit{child} can be |chapter| or |section|
depending on the chosen structuring.
Alternatively, one can modify the macro |\thepage| appropriately
and reset the counter |page| at the start of each child file.

%%%%%%%%%%%%%%%%%%%%%%%%%%%%%%%%%%%%%%%%%%%%%%%%%%%%%%%%%%%%%%%%%%%%%%%%%%%%%%%%
\subsection{Conditional Processing}
\label{sec:conditional}

The package provides a mechanism to compile different versions
of a document. To customise the versions further some conditional processing
can come in handy to distinguish which version is being compiled.
The package provides two macros to describe the compilation context:

%%%%%%%%%%%%%%%%%%%%%%%%%%%%%%%%%%%%%%%%
\DescribeMacro{\ifchilddoc}
The conditional |\ifchilddoc| distinguishes between the compilation of
child documents and the main document:
%
\begin{center}
|\ifchilddoc |\textit{child-code}| |[|\||else |\textit{main-code}]| \||fi|
\end{center}

%%%%%%%%%%%%%%%%%%%%%%%%%%%%%%%%%%%%%%%%
\DescribeMacro{\childdocname}
\DescribeMacro{\childdocjob}
The macro |\childdocname| contains the filename (without extension)
of the main or child file being processed.
Note that |\childdocjob| will always contain the name of the main file.

%%%%%%%%%%%%%%%%%%%%%%%%%%%%%%%%%%%%%%%%
\paragraph{Title Page.}

Conditional processing can be used to include a title or banner page
in the main document when proper precautions are taken.
Importantly, the code in the main file should ensure that the page counter
(as well as other status parameters which are stored in the |.aux| files)
takes the same value after the conditional processing.
Otherwise the page numbers may take divergent values
depending on which part is compiled.

For example, a title page could be declared by:
%
\begin{center}
\begin{tabular}{l}
|\ifchilddoc\||else|\\
|\addtocounter{page}{-1}|\\
\textit{code for title page}\\
|\newpage|\\
|\||fi|
\end{tabular}
\end{center}
%
A banner page for the child documents can be generated by:
%
\begin{center}
\begin{tabular}{l}
|\ifchilddoc|\\
|\addtocounter{page}{-1}|\\
\textit{code for banner page}\\
|\newpage|\\
|\||fi|
\end{tabular}
\end{center}
%
Here one could write a message such as:
\begin{center}
|This is the part \childdocname{} of \childdocjob{}.|
\end{center}

%%%%%%%%%%%%%%%%%%%%%%%%%%%%%%%%%%%%%%%%%%%%%%%%%%%%%%%%%%%%%%%%%%%%%%%%%%%%%%%%
\subsection{Flags}
\label{sec:flags}

The package makes it easy to generate different versions
of the main or child documents.
To this end compilation flags can be defined
and assigned different default values.
They will be particularly useful in conjunction
with the forwarding mechanism described in \secref{sec:forward}.

For example, it may be useful to have a flag |\version|
which can be set to |draft| or |final|.
The document source will contain some conditional code
depending on the value of |\version|.
Suppose further, the flag should default to |final| for the main file
and to |draft| for child files
which is a natural assignment for editing the document.
This is achieved by placing the following code
in the preamble of the main document
(below the |\childdocmain| directive):
%
\begin{center}
\begin{tabular}{l}
|\ifchilddoc|\\
|\providecommand{\version}{draft}|\\
|\||else|\\
|\providecommand{\version}{final}|\\
|\||fi|
\end{tabular}
\end{center}
%
The definition by |\providecommand| makes sure
that previous definitions are not overwritten.
Further statements |\providecommand{\version}{...}|
can thus be added before the above code to override it.

For the main file, one might add a line
(between |\childdocmain| and the above block)
%
\begin{center}
|%\ifchilddoc\||else\providecommand{\version}{draft}\||fi|
\end{center}
%
which can be uncommented to produce a draft version.
Likewise one can add a line to the very top of a child file
(above the |\childdocof{|\textit{main}|}| directive)
%
\begin{center}
|%\providecommand{\version}{final}|
\end{center}
%
which can be uncommented to produce the final version of this child document.

%%%%%%%%%%%%%%%%%%%%%%%%%%%%%%%%%%%%%%%%%%%%%%%%%%%%%%%%%%%%%%%%%%%%%%%%%%%%%%%%
\subsection{Forwarding}
\label{sec:forward}

Different versions of the main or child documents
using compilation flags as described in \secref{sec:flags}
can be (permanently) stored in different files
for convenient compilation, viewing and distribution.
To this end, the package defines a command
to pass on compilation to a different file:

%%%%%%%%%%%%%%%%%%%%%%%%%%%%%%%%%%%%%%%%
\DescribeMacro{\childdocforward}
The command |\childdocforward| redirects processing to
another source file:
%
\begin{center}
\begin{tabular}{l}
|\input{childdoc.def}|\\
|\childdocforward[|\textit{main}|]{|\textit{dest}|}|\\
\end{tabular}
\end{center}
%
The argument \textit{dest} is the destination file
(without extension).
It should be the main file or one of the child files.
Note that further \textsf{childdoc} directives
such as |\childdocof| and |\childdocforward|
in the indicated file will be processed in this form.
The optional argument \textit{main}
passes on directly to the main file \textit{main}
while pretending to compile the child \textit{dest}.
This form behaves as if \textit{dest}
issues |\childdocof{|\textit{main}|}| right away,
and no further \textsf{childdoc} directives will be processed.

%%%%%%%%%%%%%%%%%%%%%%%%%%%%%%%%%%%%%%%%
\DescribeMacro{\...prefix}
In the alternative form |\childdocforwardprefix|,
%
\begin{center}
\begin{tabular}{l}
|\input{childdoc.def}|\\
|\childdocforwardprefix[|\textit{main}|]{|\textit{prefix}|}{|\textit{dest}|}|
\end{tabular}
\end{center}
%
the destination file is determined by a pattern
depending on the current file:
To make this work, the current file must be called
`{\textit{prefix}\hspace{0.2em}\textit{suffix}}'
with \textit{prefix} matching precisely the argument.
Processing is then passed on to the file
`{\textit{dest}\hspace{0.2em}\textit{suffix}}'.
Surely, the same effect is achieved by
directly specifying the
argument `{\textit{dest}\hspace{0.2em}\textit{suffix}}'
in the first form.
However, that requires to set up a different file
for each child. With the alternative form of the command
all these files can have exactly the same content
which simplifies setting them up and maintaining them.

For example, the following file |draft.tex|
with a compilation flag |\version| as described in \secref{sec:flags}
compiles the main document as a draft:
%
\begin{center}
\begin{tabular}{l}
|\def\version{draft}|\\
|\input{childdoc.def}|\\
|\childdocforward{|\textit{main}|}|
\end{tabular}
\end{center}
%
Likewise, the following files |final|\textit{nn}|.tex|
compile the final version of the child document
|child|\textit{nn}|.tex|:
%
\begin{center}
\begin{tabular}{l}
|\def\version{final}|\\
|\input{childdoc.def}|\\
|\childdocforwardprefix{final}{child}|
\end{tabular}
\end{center}
%

Note that when several versions of a main file and/or of each child file
are to be generated, it may be convenient to set up a |Makefile| or
shell script to automatise the process.

%%%%%%%%%%%%%%%%%%%%%%%%%%%%%%%%%%%%%%%%%%%%%%%%%%%%%%%%%%%%%%%%%%%%%%%%%%%%%%%%
\subsection{Command Line Processing}
\label{sec:commandline}

The effect of redirection files can also be achieved by invoking
the \LaTeX{} compiler with a more elaborate command line.
Most conveniently this should be done as part
of a shell script or a |Makefile|.

When using \textsf{childdoc} in the main file, the following
command lines effectively perform a redirection
(note that depending on the shell being used,
backslashes may have to be doubled: `|\|' $\to$ `|\\|'):
%
\begin{center}
|... -jobname "|\textit{target}|" |\\|"|[\textit{flags}]%
|\input{childdoc.def}\childdocforward[|\textit{main}|]{|\textit{dest}|}"|
\end{center}
%
Here \textit{target} is the name of the output file,
\textit{main} is the name of the main file
and \textit{dest} is the name of the main or child file to be processed
(all filenames without extensions).
The optional argument \textit{main} can be omitted
if \textit{main} matches \textit{dest}.
Optionally, compilation \textit{flags} can be defined via |\def| commands.
This command line makes the \TeX{} engine believe
it is compiling the file \textit{target}
whose content is specified as the latter parameter.
The provided code then forwards the processing to
\textit{main} or \textit{dest} as described in \secref{sec:forward}.

%%%%%%%%%%%%%%%%%%%%%%%%%%%%%%%%%%%%%%%%%%%%%%%%%%%%%%%%%%%%%%%%%%%%%%%%%%%%%%%%
\subsection{Include by Input}
\label{sec:input}

Including child documents by |\include| has some restrictions by design.
Most notably, the content of a child document always occupies
its own set of pages; pages cannot be shared between child documents.
Usually, this behaviour makes perfect sense
because each child document contain an essential part of the document.
However, in some situations it may be desirable to compose
a document from a collection of parts
without having mandatory page breaks between then.
For this case, the package
provides a mechanism to include parts
by |\input| which can also be processed individually.
However, by construction this mechanism
requires manual handling of the content to be output.

%%%%%%%%%%%%%%%%%%%%%%%%%%%%%%%%%%%%%%%%
\DescribeMacro{\ifchilddocmanual}
The main file should be prepared as usual, see \secref{sec:include}.
However, the document body must make a distinction
between processing of an individual part and of the main document, e.g.:
%
\begin{center}
\begin{tabular}{l}
|\ifchilddocmanual|\\
|\input{\childdocname}|\\
|\||else|\\
\textit{document body with }|\input{|\textit{part}|}|\\
|\||fi|
\end{tabular}
\end{center}
%
The conditional |\ifchilddocmanual| is true whenever
a part to be included by |\input| is being compiled,
and the name of the part is stored in |\childdocname|.

%%%%%%%%%%%%%%%%%%%%%%%%%%%%%%%%%%%%%%%%
\DescribeMacro{\childdocby}
Each part to be included by |\input| should start with:
%
\begin{center}
\begin{tabular}{l}
|\input{childdoc.def}|\\
|\childdocby{|\textit{main}|}|\\
\end{tabular}
\end{center}
%
The directive |\childdocby| is similar to |\childdocof|
described in \secref{sec:include},
but the subsequent selection of content must be done manually.
To that end, both |\ifchilddoc| and |\ifchilddocmanual|
will be true upon processing of a part,
and the name of the part is stored in |\childdocname|.
Note that |\jobname| will be set to the filename of the current part
so that each part receives an individual |.aux| file
that does not interfere with the |.aux| file(s) of the main document.
This behaviour can be altered by the alternative form
|\childdocby[*]{|\textit{main}|}| (with a non-empty optional argument)
which uses the |.aux| file of the main document
by setting |\jobname| to \textit{main}.

%%%%%%%%%%%%%%%%%%%%%%%%%%%%%%%%%%%%%%%%%%%%%%%%%%%%%%%%%%%%%%%%%%%%%%%%%%%%%%%%
\subsection{Driver Development}
\label{sec:driver}

The \textsf{childdoc} mechanism can also be use for the development
of definition files such as \LaTeX{} styles or classes.
This case differs from the above setup with multiple parts
included by |\include| in that no |\includeonly| should be invoked.
This can be achieved by starting the include file
(before |\ProvidesPackage|) with:
%
\begin{center}
\begin{tabular}{l}
|\input{childdoc.def}|\\
|\childdocforward{|\textit{main}|}|\\
\end{tabular}
\end{center}
%
or alternatively with:
%
\begin{center}
\begin{tabular}{l}
|\input{childdoc.def}|\\
|\childdocby{|\textit{main}|}|\\
\end{tabular}
\end{center}
%
Both forms have slightly different effects as described above.
The main file is prepared as usual, see \secref{sec:include}.

%%%%%%%%%%%%%%%%%%%%%%%%%%%%%%%%%%%%%%%%%%%%%%%%%%%%%%%%%%%%%%%%%%%%%%%%%%%%%%%%
\subsection{Legacy Detection}
\label{sec:detection}

The directive |\childdocmain| in the main file can detect
whether the complete document or merely a child is to be compiled
even without using the directive |\childdocof|.
This method is deprecated because it is less robust
and there is no compelling reason to use it;
it is merely provided for backward compatibility
and it may be removed in future versions.

If the detection mechanism is to be used,
it is mandatory to correctly specify
the filename of the main file as the argument of |\childdocmain|:
%
\begin{center}
\begin{tabular}{l}
|\input{childdoc.def}|\\
|\childdocmain{|\textit{main}|}|\\
\end{tabular}
\end{center}
%
If |\jobname| does not match the argument \textit{main} of |\childdocmain|,
it is assumed that |\jobname| points to the child file to be compiled.
When using |\childdocmain| with the main file specified as argument,
it suffices to start a child file
with just |\input{|\textit{main}|}|
without loading of the package and using |\childdocof|.
If instead all processing is done
with the appropriate \textsf{childdoc} directives,
the argument of \textit{main} of |\childdocmain| can be empty.

An alternative version of the command line processing described
in \secref{sec:commandline} using the detection mechanism reads:
%
\begin{center}
|... -jobname "|\textit{target}|" "|[\textit{flags}]%
[|\def\jobname{|\textit{dest}|}|]|\input{|\textit{main}|}"|
\end{center}

%%%%%%%%%%%%%%%%%%%%%%%%%%%%%%%%%%%%%%%%%%%%%%%%%%%%%%%%%%%%%%%%%%%%%%%%%%%%%%%%
\subsection{Manual Code}
\label{sec:manual}

In case one cannot be certain whether the definitions file |childdoc.def|
is installed on the target \TeX{} distribution
and one prefers not to ship it,
it is conceivable to paste a few relevant commands into the sources.

To that end, drop all statements |\input{childdoc.def}|
and perform the replacements as outlined below.
Instead of |\childdocmain{|\textit{main}|}| add the following code
to the top of the main file:
%
\begin{center}
\begin{tabular}{l}
|\||ifdefined\childdocname\endinput\||fi\newif\ifchilddoc|\\
|\edef\childdocname{\scantokens\expandafter{\jobname\noexpand}}|\\
|\def\childdocmain{|\textit{main}|}\||ifx\childdocmain\childdocname\||else|\\
|\childdoctrue\includeonly{\childdocname}\let\jobname\childdocmain\||fi|\\
\end{tabular}
\end{center}
%
Instead of |\childdocof{|\textit{main}|}| just include the main file
at the top of each child file:
%
\begin{center}
|\input{|\textit{main}|}|
\end{center}
%
A simple redirection |\childdocforward{|\textit{dest}|}| is achieved by:
%
\begin{center}
|\def\jobname{|\textit{dest}|}\input{\jobname}|
\end{center}
%
The redirection with prefix
|\childdocforwardprefix[|\textit{prefix}|]{|\textit{dest}|}|
is accomplished by:
%
\begin{center}
\begin{tabular}{l}
|{\edef\jobname{\scantokens\expandafter{\jobname\noexpand}}|\\
|\def\redirectjob |\textit{prefix}|#1~~~{\gdef\jobname{|\textit{dest}|#1}}|\\
|\expandafter\redirectjob\jobname~~~}\input{\jobname}|
\end{tabular}
\end{center}

In an alternative approach,
child documents can be compiled by a specific command line
without additional code or specific definitions:
%
\begin{center}
|... -jobname "|\textit{target}|" "|[\textit{flags}]%
|\includeonly{|\textit{dest}|}\input{|\textit{main}|}"|
\end{center}
%

%%%%%%%%%%%%%%%%%%%%%%%%%%%%%%%%%%%%%%%%%%%%%%%%%%%%%%%%%%%%%%%%%%%%%%%%%%%%%%%%
%%%%%%%%%%%%%%%%%%%%%%%%%%%%%%%%%%%%%%%%%%%%%%%%%%%%%%%%%%%%%%%%%%%%%%%%%%%%%%%%
\section{Information}

%%%%%%%%%%%%%%%%%%%%%%%%%%%%%%%%%%%%%%%%%%%%%%%%%%%%%%%%%%%%%%%%%%%%%%%%%%%%%%%%
\subsection{Copyright}

Copyright \copyright{} 2017--2018 Niklas Beisert

This work may be distributed and/or modified under the
conditions of the \LaTeX{} Project Public License, either version 1.3
of this license or (at your option) any later version.
The latest version of this license is in
  \url{http://www.latex-project.org/lppl.txt}
and version 1.3 or later is part of all distributions of \LaTeX{}
version 2005/12/01 or later.

This work has the LPPL maintenance status `maintained'.

The Current Maintainer of this work is Niklas Beisert.

This work consists of the files |README.txt|, |childdoc.ins| and |childdoc.dtx|
as well as the derived files |childdoc.def|, |cdocsamp.tex|
with |cdocsch1.tex|, |cdocsch2.tex|, |cdocspt3.tex|, |cdocspt4.tex|,
|cdocsdrf.tex|, |cdocsfn1.tex|, |cdocsfn2.tex|
as well as |childdoc.pdf|.

%%%%%%%%%%%%%%%%%%%%%%%%%%%%%%%%%%%%%%%%%%%%%%%%%%%%%%%%%%%%%%%%%%%%%%%%%%%%%%%%
\subsection{Files and Installation}

The package consists of the files:
%
\begin{center}
\begin{tabular}{ll}
    |README.txt|   & readme file \\
    |childdoc.ins| & installation file \\
    |childdoc.dtx| & source file \\
    |childdoc.def| & definition file \\
    |cdocsamp.tex| & sample main file \\
    |cdocsch1.tex| & sample include file \\
    |cdocsch2.tex| & sample include file \\
    |cdocspt3.tex| & sample part file \\
    |cdocspt4.tex| & sample part file \\
    |cdocsdrf.tex| & sample redirection file \\
    |cdocsfn1.tex| & sample redirection file \\
    |cdocsfn2.tex| & sample redirection file \\
    |childdoc.pdf| & manual
\end{tabular}
\end{center}
%
The distribution consists of the files
|README.txt|, |childdoc.ins| and |childdoc.dtx|.
%
\begin{itemize}
\item
Run (pdf)\LaTeX{} on |childdoc.dtx|
to compile the manual |childdoc.pdf| (this file).
\item
Run \LaTeX{} on |childdoc.ins| to create the definitions file |childdoc.def|
and the sample |cdocsamp.tex| with include files
|cdocsch1.tex|, |cdocsch2.tex|, |cdocspt3.tex|, |cdocspt4.tex|,
|cdocsdrf.tex|, |cdocsfn1.tex|, |cdocsfn2.tex|.
Then copy the file |childdoc.def| to an appropriate directory of your \LaTeX{}
distribution, e.g.\ \textit{texmf-root}|/tex/latex/childdoc|.
\end{itemize}

%%%%%%%%%%%%%%%%%%%%%%%%%%%%%%%%%%%%%%%%%%%%%%%%%%%%%%%%%%%%%%%%%%%%%%%%%%%%%%%%
\subsection{Related CTAN Packages}

There are several other packages which offer a similar functionality:
%
\begin{itemize}
\item
The packages
\href{http://ctan.org/pkg/docmute}{\textsf{docmute}},
\href{http://ctan.org/pkg/includex}{\textsf{includex}} and
\href{http://ctan.org/pkg/standalone}{\textsf{standalone}}
provide commands to include only the document body of
a child file thus allowing both files to be compiled individually.
\item
The packages \href{http://ctan.org/pkg/subdocs}{\textsf{subdocs}}
and \href{http://ctan.org/pkg/subfiles}{\textsf{subfiles}}
provide structures in which the main and child documents can be
encapsulated and allowing them to be compiled individually.
The inclusion mechanism is different from the conventional |\include|.
\item
The package \href{http://ctan.org/pkg/combine}{\textsf{combine}}
is an elaborate solution to combine several documents into one.
\end{itemize}
%
See also the CTAN topic \href{http://ctan.org/topic/subdocs}{\textsf{subdocs}}
for further related packages.
The present package differs from the above solutions in that
a document structure constructed with the conventional |\include| mechanism
just needs two extra commands at the top of every file
such that all constituent files can be compiled individually.

%%%%%%%%%%%%%%%%%%%%%%%%%%%%%%%%%%%%%%%%%%%%%%%%%%%%%%%%%%%%%%%%%%%%%%%%%%%%%%%%
%\subsection{Feature Suggestions}
%
%The following is a list of features which may be useful for future
%versions of this package:
%%
%\begin{itemize}
%\item
%\ldots
%\end{itemize}

%%%%%%%%%%%%%%%%%%%%%%%%%%%%%%%%%%%%%%%%%%%%%%%%%%%%%%%%%%%%%%%%%%%%%%%%%%%%%%%%
\subsection{Revision History}

%%%%%%%%%%%%%%%%%%%%%%%%%%%%%%%%%%%%%%%%
\paragraph{v2.0:} 2018/12/30

\begin{itemize}
\item
immediate forward processing
\item
added |\childdocby| mechanism
\item
manual restructured
\end{itemize}

%%%%%%%%%%%%%%%%%%%%%%%%%%%%%%%%%%%%%%%%
\paragraph{v1.6:} 2018/01/17

\begin{itemize}
\item
application for development of include files
\item
corrections to manual
\end{itemize}

%%%%%%%%%%%%%%%%%%%%%%%%%%%%%%%%%%%%%%%%
\paragraph{v1.5:} 2017/05/21

\begin{itemize}
\item
more complete structuring introduced
\item
|\childdocof| introduced
\item
|\childdoc| renamed to |\childdocmain|
\item
|\childredirect| renamed to |\childdocforward| and |\childdocforwardprefix|
and functionality expanded
\end{itemize}

%%%%%%%%%%%%%%%%%%%%%%%%%%%%%%%%%%%%%%%%
\paragraph{v1.0:} 2017/04/27

\begin{itemize}
\item
manual and install package
\item
first version published on CTAN
\end{itemize}

%%%%%%%%%%%%%%%%%%%%%%%%%%%%%%%%%%%%%%%%
\paragraph{v0.6:} 2017/04/26

\begin{itemize}
\item
redirection mechanism added
\end{itemize}

%%%%%%%%%%%%%%%%%%%%%%%%%%%%%%%%%%%%%%%%
\paragraph{v0.5:} 2017/04/26

\begin{itemize}
\item
functionality in definition file
\end{itemize}


%%%%%%%%%%%%%%%%%%%%%%%%%%%%%%%%%%%%%%%%%%%%%%%%%%%%%%%%%%%%%%%%%%%%%%%%%%%%%%%%
%%%%%%%%%%%%%%%%%%%%%%%%%%%%%%%%%%%%%%%%%%%%%%%%%%%%%%%%%%%%%%%%%%%%%%%%%%%%%%%%
%%%%%%%%%%%%%%%%%%%%%%%%%%%%%%%%%%%%%%%%%%%%%%%%%%%%%%%%%%%%%%%%%%%%%%%%%%%%%%%%
\appendix

\settowidth\MacroIndent{\rmfamily\scriptsize 000\ }

 \DocInput{childdoc.dtx}

\end{document}
%</driver>
% \fi
%
% %%%%%%%%%%%%%%%%%%%%%%%%%%%%%%%%%%%%%%%%%%%%%%%%%%%%%%%%%%%%%%%%%%%%%%%%%%%%%%
% %%%%%%%%%%%%%%%%%%%%%%%%%%%%%%%%%%%%%%%%%%%%%%%%%%%%%%%%%%%%%%%%%%%%%%%%%%%%%%
% \section{Sample}
%\iffalse
%<*samplemain>
%\fi
%
% The following presents a sample document
% with two chapters, two parts, a title page,
% a compile flag as well as three forwarding files to set the flag.
% It consists of eight |.tex| files:
% \begin{center}
% \begin{tabular}{ll}
% |cdocsamp.tex|&main file\\
% |cdocsch1.tex|&include file for chapter 1\\
% |cdocsch2.tex|&include file for chapter 2\\
% |cdocspt3.tex|&include file for part 3\\
% |cdocspt4.tex|&include file for part 4\\
% |cdocsdrf.tex|&forwarding file for main file in draft mode\\
% |cdocsfi1.tex|&forwarding file for final version of chapter 1\\
% |cdocsfi2.tex|&forwarding file for final version of chapter 2\\
% \end{tabular}
% \end{center}
% Each of the eight files can be compiled directly by the \LaTeX{} compiler.
%
% %%%%%%%%%%%%%%%%%%%%%%%%%%%%%%%%%%%%%%
% \paragraph{Main File.}
%
% The main file is called |cdocsamp.tex|.
%
% Load the \textsf{childdoc} definitions and
% declare the filename for the main document:
%    \begin{macrocode}
\input{childdoc.def}
\childdocmain{}
%    \end{macrocode}

% Optional override for |\version| flag:
%    \begin{macrocode}
%%\ifchilddoc\else\providecommand{\version}{draft}\fi
%    \end{macrocode}

% Define the default values for the |\version| flag
% (|final| for the main file and |draft| for childs):
%    \begin{macrocode}
\ifchilddoc
\providecommand{\version}{draft}
\else
\providecommand{\version}{final}
\fi
%    \end{macrocode}

% Load the standard document class:
%    \begin{macrocode}
\documentclass[12pt]{article}
%    \end{macrocode}

% Start the document body:
%    \begin{macrocode}
\begin{document}
%    \end{macrocode}

% Declare a title page.
% Print title, part of document being processed and version flag:
%    \begin{macrocode}
\addtocounter{page}{-1}
\begin{center}
{\LARGE\bfseries{}childdoc example\par}
\vspace{1cm}
\ifchilddoc
\ifchilddocmanual part\else chapter\fi:
`\childdocname' of `\childdocjob'\par
\else
main document: `\childdocjob'\par
\fi
version: \version\par
\end{center}
\newpage
%    \end{macrocode}

% Manually include selected file,
% otherwise process as usual:
%    \begin{macrocode}
\ifchilddocmanual
\section*{part `\childdocname'}
\input{\childdocname}
\else
%    \end{macrocode}

% Include the two chapters:
%    \begin{macrocode}
\include{cdocsch1}
\include{cdocsch2}
%    \end{macrocode}

% Include the two parts unless only chapters should be displayed:
%    \begin{macrocode}
\ifchilddoc\else
\section{part three}
\input{cdocspt3}
\section{part four}
\input{cdocspt4}
\fi
%    \end{macrocode}

% Process as usual until here:
%    \begin{macrocode}
\fi
%    \end{macrocode}

% End of document body:
%    \begin{macrocode}
\end{document}
%    \end{macrocode}
%\iffalse
%</samplemain>
%\fi
%
% %%%%%%%%%%%%%%%%%%%%%%%%%%%%%%%%%%%%%%
% \paragraph{Chapter Include Files.}
%
% The include files are called |cdocsch1.tex| and |cdocsch2.tex|.
%
%\iffalse
%<*samplechap1|samplechap2>
%\fi

% Optional override for |\version| flag:
%    \begin{macrocode}
%%\providecommand{\version}{final}
%    \end{macrocode}

% Include the main document:
%    \begin{macrocode}
\input{childdoc.def}
\childdocof{cdocsamp}
%    \end{macrocode}

%\iffalse
%</samplechap1|samplechap2>
%\fi
%
%\iffalse
%<*samplechap1>
%\fi
% Some text for chapter 1:
%    \begin{macrocode}
\section{one}
some text in chapter one
%    \end{macrocode}

%\iffalse
%</samplechap1>
%\fi
% Some text for chapter 2:
%\iffalse
%<*samplechap2>
%\fi
%    \begin{macrocode}
\section{two}
more text in chapter two
%    \end{macrocode}

%\iffalse
%</samplechap2>
%\fi
%
% %%%%%%%%%%%%%%%%%%%%%%%%%%%%%%%%%%%%%%
% \paragraph{Part Include Files.}
%
% The include files are called |cdocspt3.tex| and |cdocspt4.tex|.
%
%\iffalse
%<*samplepart3|samplepart4>
%\fi

% Optional override for |\version| flag:
%    \begin{macrocode}
%%\providecommand{\version}{final}
%    \end{macrocode}

% Include the main document:
%    \begin{macrocode}
\input{childdoc.def}
\childdocby{cdocsamp}
%    \end{macrocode}

%\iffalse
%</samplepart3|samplepart4>
%\fi
%
%\iffalse
%<*samplepart3>
%\fi
% Some text for part 3:
%    \begin{macrocode}
some text in part three
%    \end{macrocode}

%\iffalse
%</samplepart3>
%\fi
% Some text for part 4:
%\iffalse
%<*samplepart4>
%\fi
%    \begin{macrocode}
more text in part four
%    \end{macrocode}

%\iffalse
%</samplepart4>
%\fi
%
% %%%%%%%%%%%%%%%%%%%%%%%%%%%%%%%%%%%%%%
% \paragraph{Forwarding for a Complete Draft.}
%
% The following forwarding file |cdocsdrf.tex|
% compiles the main document in draft mode:
%\iffalse
%<*sampledraft>
%\fi
%    \begin{macrocode}
\def\version{draft}
\input{childdoc.def}
\childdocforward{cdocsamp}
%    \end{macrocode}

%\iffalse
%</sampledraft>
%\fi
%
% %%%%%%%%%%%%%%%%%%%%%%%%%%%%%%%%%%%%%%
% \paragraph{Forwarding for Final Version of the Chapters.}
%
% The following forwarding files |cdocsfn1.tex| and |cdocsfn2.tex|
% (with identical content)
% compile the final versions of the child documents
% |cdocsch1.tex| and |cdocsch2.tex|, respectively:
%\iffalse
%<*samplefinal>
%\fi
%    \begin{macrocode}
\def\version{final}
\input{childdoc.def}
\childdocforwardprefix[cdocsamp]{cdocsfn}{cdocsch}
%    \end{macrocode}

%\iffalse
%</samplefinal>
%\fi
%
% %%%%%%%%%%%%%%%%%%%%%%%%%%%%%%%%%%%%%%
% \paragraph{Command Line Processing.}
%
% The following three command lines generate the output files
% |cdocscld|, |cdocscl1| and |cdocscl2|
% which should be identical to
% |cdocsdrf|, |cdocsch1| and |cdocsfn2|, respectively:
% \begin{center}
% \begin{tabular}{l}
% |latex -jobname cdocscld \|\\
% |  "\def\version{draft}\input{childdoc.def}\childdocforward{cdocsamp}"|\\
% |latex -jobname cdocscl1 \|\\
% |  "\input{childdoc.def}\childdocforward[cdocsamp]{cdocsch1}"|\\
% |latex -jobname cdocscl2 \|\\
% |  "\def\version{final}\input{childdoc.def}\childdocforward{cdocsch2}"|
% \end{tabular}
% \end{center}
% Note that the trailing backslash on each first line
% merely continues the input to the second line
% (for convenient cut ant paste).
% Furthermore, the command |latex| can be replaced by any
% of its alternative versions such as |pdflatex|.
%
% %%%%%%%%%%%%%%%%%%%%%%%%%%%%%%%%%%%%%%%%%%%%%%%%%%%%%%%%%%%%%%%%%%%%%%%%%%%%%%
% %%%%%%%%%%%%%%%%%%%%%%%%%%%%%%%%%%%%%%%%%%%%%%%%%%%%%%%%%%%%%%%%%%%%%%%%%%%%%%
% \section{Implementation}
%\iffalse
%<*package>
%\fi
%
% This section describes the definitions file |childdoc.def|.

% The definitions cannot be loaded using |\usepackage| or |\RequirePackage|
% which has a mechanism to prevent loading a style file more than once.
% When loading the definitions by means of |\input|
% multiple instances have to be prevented manually:
%\iffalse
%This code needs to be before the `\ProvidesFile' directive
%which is defined at the beginning of this file.
%Therefore it is also placed there and commented out here.
%</package>
%<*discard>
%\fi
%    \begin{macrocode}
\ifdefined\childdocmain\endinput\fi
%    \end{macrocode}
%\iffalse
%</discard>
%<*package>
%\fi
%
% \macro{\ifchilddoc}
% \macro{\ifchilddocmanual}
% The conditional |\ifchilddoc| tells whether a
% child (true) or main (false) document is being compiled.
% The conditional |\ifchilddocmanual| tells whether
% the |\includeonly| mechanism is used (false) or
% the selection of child files must be performed manually (true).
% The definitions initialise to false:
%    \begin{macrocode}
\newif\ifchilddoc
\newif\ifchilddocmanual
%    \end{macrocode}

% \macro{\childdocname}
% \macro{\childdocjob}
% The macro |\childdocname| stores the name of the main document
% to be compiled. The macro |\childdocjob| stores the name of
% the document on which the \LaTeX{} compiler was originally invoked.
% The content of |\jobname| cannot be compared
% to filenames specified in the source due to different catcodes.
% The following code rescans |\jobname|, stores the result
% in |\childdocname| and saves a copy in |\childdocjob|:
%    \begin{macrocode}
\edef\childdocname{\scantokens\expandafter{\jobname\noexpand}}
\let\childdocjob\childdocname
%    \end{macrocode}

% \macro{\childdocdisable}
% The macro |\childdocdisable| prevents the main file
% from being processed more than once.
% At this stage, the main document command |\childdocmain|
% is assumed to be called once again where it should do nothing.
% Any subsequent call to it should prevent
% a secondary processing of the main document
% It overwrites the forwarding commands
% |\childdocof| and |\childdocforward|
% with empty macros to prevent further inclusions of the main document:
%    \begin{macrocode}
\newcommand{\childdocdisable}
{
  \renewcommand{\childdocmain}[1]{\renewcommand{\childdocmain}[1]{\endinput}}
  \renewcommand{\childdocof}[1]{}
  \renewcommand{\childdocby}[2][]{}
  \renewcommand{\childdocforward}[2][]{}
  \renewcommand{\childdocdisable}{}
}
%    \end{macrocode}

% \macro{\childdocmain}
% The macro |\childdocmain| is to be called at the top of the main file
% with nothing or the main filename (without extension) as argument.
% First, it breaks loops.
% If the argument is not empty and does not match |\childdocname|
% (which is set by the first inclusion of |childdoc.def|),
% |\ifchilddoc| is set to true, |\includeonly| is applied to the child file
% and |\jobname| is set to the main file
% (for proper handling of |.aux| files):
%    \begin{macrocode}
\newcommand{\childdocmain}[1]
{
  \childdocdisable\childdocmain{}
  \if?#1?\else
    \begingroup
      \def\childdoctmp{#1}
      \ifx\childdoctmp\childdocname
        \def\childdoctmp{}
      \else
        \def\childdoctmp
        {
          \childdoctrue
          \includeonly{\childdocname}
          \def\childdocjob{#1}
          \def\jobname{#1}
        }
      \fi
      \expandafter
    \endgroup
    \childdoctmp
  \fi
}
%    \end{macrocode}

% \macro{\childdocof}
% The command |\childdocof| redirects
% compilation to the main file |#1|.
%    \begin{macrocode}
\newcommand{\childdocof}[1]
{
  \childdocdisable
  \childdoctrue
  \includeonly{\childdocname}
  \def\jobname{#1}
  \def\childdocjob{#1}
  \input{#1}
}
%    \end{macrocode}

% \macro{\childdocby}
% The command |\childdocby| ....
%    \begin{macrocode}
\newcommand{\childdocby}[2][]
{
  \childdocdisable
  \childdoctrue
  \childdocmanualtrue
  \if?#1?\else
    \def\jobname{#2}
  \fi
  \def\childdocjob{#2}
  \input{#2}
  \endinput
}
%    \end{macrocode}

% \macro{\childdocforward}
% The command |\childdocforward| redirects
% compilation to the main file or
% (if the optional argument is given) a child file.
% Parameters are set as if the main file
% or a child file starting with |\childdocof| was compiled.
% Then compilation is handed over to the main file:
%    \begin{macrocode}
\newcommand{\childdocforward}[2][]
{
  \begingroup
    \if?#1?
      \def\childdoctmp
      {
        \def\childdocname{#2}
        \def\childdocjob{#2}
        \def\jobname{#2}
        \input{#2}
        \endinput
      }
    \else
      \def\childdoctmp
      {
        \childdocdisable
        \def\childdocname{#2}
        \childdoctrue
        \includeonly{#2}
        \def\childdocjob{#1}
        \def\jobname{#1}
        \input{#1}
        \endinput
      }
    \fi
    \expandafter
  \endgroup
  \childdoctmp
}
%    \end{macrocode}

% \macro{\childdocforwardprefix}
% The command |\childdocforwardprefix| redirects
% compilation to the main or a child file by means of a pattern.
% The prefix |#1| in the current filename is replaced by |#2|
% and the suffix of the current filename is kept
% (it is assumed that the filename does not contain the substring `|~~~|'
% which is used as a delimiter).
% Compilation is handed over to the new file by |\childdocforward|:
%    \begin{macrocode}
\newcommand{\childdocforwardprefix}[3][]
{
  \begingroup
    \def\childdocextract #2##1~~~{\def\childdoctmp{\childdocforward[#1]{#3##1}}}
    \expandafter\childdocextract\childdocname~~~
    \expandafter
  \endgroup
  \childdoctmp
}
%    \end{macrocode}

% \macro{\childdoc}
% The deprecated macro |\childdoc| is a legacy version of |\childdocmain|:
%    \begin{macrocode}
\newcommand{\childdoc}{\childdocmain}
%    \end{macrocode}

% \macro{\childdocredirect}
% The deprecated macro |\childdocredirect| is a legacy version
% of |\childdocforward| and |\childdocforwardprefix|:
%    \begin{macrocode}
\newcommand{\childdocredirect}[2][]
{
  \begingroup
    \if?#1?
      \def\childdoctmp{\childdocforward{#2}}
    \else
      \def\childdoctmp{\childdocforwardprefix{#1}{#2}}
    \fi
    \expandafter
  \endgroup
  \childdoctmp
}
%    \end{macrocode}

%\iffalse
%</package>
%\fi
%
\endinput
|\\
|\childdocforward{|\textit{main}|}|
\end{tabular}
\end{center}
%
Likewise, the following files |final|\textit{nn}|.tex|
compile the final version of the child document
|child|\textit{nn}|.tex|:
%
\begin{center}
\begin{tabular}{l}
|\def\version{final}|\\
|% \iffalse
%
% childdoc.dtx Copyright (C) 2017-2018 Niklas Beisert
%
% This work may be distributed and/or modified under the
% conditions of the LaTeX Project Public License, either version 1.3
% of this license or (at your option) any later version.
% The latest version of this license is in
%   http://www.latex-project.org/lppl.txt
% and version 1.3 or later is part of all distributions of LaTeX
% version 2005/12/01 or later.
%
% This work has the LPPL maintenance status `maintained'.
%
% The Current Maintainer of this work is Niklas Beisert.
%
% This work consists of the files childdoc.dtx and childdoc.ins
% and the derived files childdoc.def and cdocsamp.tex with
% cdocsch1.tex, cdocsch2.tex, cdocsdrf.tex, cdocsfn1.tex, cdocsfn2.tex.
%
%<package>\ifdefined\childdocmain\endinput\fi
%<package>\ProvidesFile{childdoc.def}[2018/12/30 v2.0 child document driver]
%<samplemain>\ProvidesFile{cdocsamp.tex}[2018/12/30 v2.0 sample for childdoc]
%<*driver>
%\ProvidesFile{childdoc.drv}[2018/12/30 v2.0 childdoc reference manual file]
\PassOptionsToClass{10pt,a4paper}{article}
\documentclass{ltxdoc}

\usepackage[margin=35mm]{geometry}
\usepackage{hyperref}
\usepackage{hyperxmp}
\usepackage[usenames]{color}

\hypersetup{colorlinks=true}
\hypersetup{pdfstartview=FitH}
\hypersetup{pdfpagemode=UseNone}
\hypersetup{pdfsource={}}
\hypersetup{pdflang={en-UK}}
\hypersetup{pdfcopyright={Copyright 2017-2018 Niklas Beisert.
  This work may be distributed and/or modified under the
  conditions of the LaTeX Project Public License, either version 1.3
  of this license or (at your option) any later version.}}
\hypersetup{pdflicenseurl={http://www.latex-project.org/lppl.txt}}
\hypersetup{pdfcontactaddress={ETH Zurich, ITP, HIT K,
  Wolfgang-Pauli-Strasse 27}}
\hypersetup{pdfcontactpostcode={8093}}
\hypersetup{pdfcontactcity={Zurich}}
\hypersetup{pdfcontactcountry={Switzerland}}
\hypersetup{pdfcontactemail={nbeisert@itp.phys.ethz.ch}}
\hypersetup{pdfcontacturl={http://people.phys.ethz.ch/\xmptilde nbeisert/}}

\newcommand{\secref}[1]{\hyperref[#1]{section \ref*{#1}}}

\parskip1ex
\parindent0pt
\let\olditemize\itemize
\def\itemize{\olditemize\parskip0pt}

\begin{document}

\title{The \textsf{childdoc} Package}
\hypersetup{pdftitle={The childdoc Package}}
\author{Niklas Beisert\\[2ex]
  Institut f\"ur Theoretische Physik\\
  Eidgen\"ossische Technische Hochschule Z\"urich\\
  Wolfgang-Pauli-Strasse 27, 8093 Z\"urich, Switzerland\\[1ex]
  \href{mailto:nbeisert@itp.phys.ethz.ch}
  {\texttt{nbeisert@itp.phys.ethz.ch}}}
\hypersetup{pdfauthor={Niklas Beisert}}
\hypersetup{pdfsubject={Manual for the LaTeX2e Package childdoc}}
\date{30 December 2018, \textsf{v2.0}}
\maketitle

\begin{abstract}\noindent
\textsf{childdoc} is a \LaTeXe{} package
that enables the direct compilation
of document sections included by |\include|
to individual files.
\end{abstract}

\begingroup
\parskip0ex
\tableofcontents
\endgroup

%%%%%%%%%%%%%%%%%%%%%%%%%%%%%%%%%%%%%%%%%%%%%%%%%%%%%%%%%%%%%%%%%%%%%%%%%%%%%%%%
%%%%%%%%%%%%%%%%%%%%%%%%%%%%%%%%%%%%%%%%%%%%%%%%%%%%%%%%%%%%%%%%%%%%%%%%%%%%%%%%
\section{Introduction}

\LaTeX{} provides a mechanism to structure a large document (such as a book)
into a main file and several child files (containing the chapters)
using the |\include| command.
This mechanism is beneficial for documents
which span hundreds of pages in order to
make the source file(s) more manageable.
Moreover, compilation can be restricted to
selected child files by means of the |\includeonly| command.
The latter feature can be used to reduce the compilation time while editing
(this was significantly more useful in the earlier days of \LaTeX{})
or to generate a smaller document which is easier to navigate.
Another application of |\includeonly| is to generate
documents consisting of selected parts of the complete document.

However, there are a few drawbacks of the plain |\include| mechanism:
\begin{itemize}
\item
The child files cannot be compiled on their own,
they can only be compiled via the main file.
A naive editing environment
(such as a text editor with an option
to have the current file processed by \LaTeX)
may require one to switch to the main file before compiling;
attempting to compile the child file produces errors.
\item
The main file must be modified (each time)
to adjust the |\includeonly| command
to the present needs. This easily leaves the main file in a messy state.
\item
The generated document will always carry the filename
of the main document. This is inconvenient if
several child files are to be compiled and
to be kept for distribution.
\end{itemize}

The present package provides a simple interface
to make child files individually compilable by \LaTeX{}.
Compiling a child file then has the same effect as compiling
the main file with an |\includeonly| command
to select the appropriate child.
Moreover the generated document will carry the name of the child
rather than the main file.
This resolves all three above issues.

This feature is meant to make the editing of books,
thesis documents and lecture notes somewhat more convenient.
However, the package can also be used efficiently for
composing a series of documents (such as exercise sheets)
which are typically distributed individually.
It then assists the author in generating the individual documents
(potentially in different versions)
as well as a document containing the collected series.
Another application is in developing style files
or other kinds of included material
where compilation of the style file could redirect
to a sample or test file.

%%%%%%%%%%%%%%%%%%%%%%%%%%%%%%%%%%%%%%%%%%%%%%%%%%%%%%%%%%%%%%%%%%%%%%%%%%%%%%%%
%%%%%%%%%%%%%%%%%%%%%%%%%%%%%%%%%%%%%%%%%%%%%%%%%%%%%%%%%%%%%%%%%%%%%%%%%%%%%%%%
\section{Usage}

First of all, the package \textsf{childdoc} is \emph{not} a standard
\LaTeXe{} |.sty| style file! Therefore it needs to be invoked in
a non-standard way.

%%%%%%%%%%%%%%%%%%%%%%%%%%%%%%%%%%%%%%%%%%%%%%%%%%%%%%%%%%%%%%%%%%%%%%%%%%%%%%%%
\subsection{Included Files}
\label{sec:include}

%%%%%%%%%%%%%%%%%%%%%%%%%%%%%%%%%%%%%%%%
\DescribeMacro{\childdocmain}
To use the package, add the commands
\begin{center}
\begin{tabular}{l}
|\input{childdoc.def}|\\
|\childdocmain{}|\\
\end{tabular}
\end{center}
at the very top of the main \LaTeX{} file,
in particular \emph{before} the |\documentclass| statement!
The argument of |\childdocmain| should be left empty
(but it must be present).

%%%%%%%%%%%%%%%%%%%%%%%%%%%%%%%%%%%%%%%%
\DescribeMacro{\childdocof}
Furthermore, add the commands
\begin{center}
\begin{tabular}{l}
|\input{childdoc.def}|\\
|\childdocof{|\textit{main}|}|\\
\end{tabular}
\end{center}
at the top of every child file \textit{child}
which is included by |\include{|\textit{child}|}|
from within the main file
(or at least for those files to be compiled individually).
The argument \textit{main} must be the filename of the main file.

There are a couple of
considerations in setting up the main and child documents:

%%%%%%%%%%%%%%%%%%%%%%%%%%%%%%%%%%%%%%%%
\paragraph{Restrictions.}

Please note the following restrictions:
\begin{itemize}
\item
|\childdocmain| must be called with one argument \textit{main}
to ensure compatibility with earlier version of the package.
It must either be empty (|\childdocmain{}|)
or precisely match the filename of the main file in which it is specified.
See \secref{sec:detection} for further information.
\item
The filename \textit{main} must be specified without the |.tex| extension.
\item
The filename \textit{main} is case sensitive
(even in case-insensitive file systems)
due to internal string comparison.
\item
The argument \textit{main} should be fully expanded, it cannot be a macro.
\item
Subdirectories and special characters should be avoided in filenames.
\item
The command |\childdocmain{|\textit{main}|}| must be followed by a whitespace.
It should not be followed immediately by another command
or by a comment mark `|%|'.
This is because the \TeX{} parser reads the token immediately following
the argument of |\childdocmain| and puts it
at the beginning of every child section;
however, a white\-space is ignored.
\end{itemize}

%%%%%%%%%%%%%%%%%%%%%%%%%%%%%%%%%%%%%%%%
\paragraph{Content of Main File.}

It is advisable to place all content in the child files included by |\include|.
Any output contained in the main file will appear in all child documents
unless suppressed manually;
it cannot be suppressed automatically by the |\includeonly| directive
and thus should normally be avoided.
A method to include some content in the main file
by means of conditional processing is described in \secref{sec:conditional}.

%%%%%%%%%%%%%%%%%%%%%%%%%%%%%%%%%%%%%%%%
\paragraph{Page Numbering.}

When only a part of the document is compiled,
the appropriate numbering of pages
(as well as other status parameters)
is determined from the |.aux| files.
The latter contain information from previous passes.
However this information needs to propagate through
all intermediate child documents.
Therefore the page numbering in child documents may well
be inconsistent until the complete document is compiled at least once.

A useful (if unconventional) way to always ensure a consistent
page numbering is to restart the numbering in each child document
and denote the pages by `\textit{child}|.|\textit{page}'
where \textit{child} represents the chapter/section number of the child file.
This can be achieved by the command
|\numberwithin{page}{|\textit{child}|}|
of the \textsf{amsmath} package
where \textit{child} can be |chapter| or |section|
depending on the chosen structuring.
Alternatively, one can modify the macro |\thepage| appropriately
and reset the counter |page| at the start of each child file.

%%%%%%%%%%%%%%%%%%%%%%%%%%%%%%%%%%%%%%%%%%%%%%%%%%%%%%%%%%%%%%%%%%%%%%%%%%%%%%%%
\subsection{Conditional Processing}
\label{sec:conditional}

The package provides a mechanism to compile different versions
of a document. To customise the versions further some conditional processing
can come in handy to distinguish which version is being compiled.
The package provides two macros to describe the compilation context:

%%%%%%%%%%%%%%%%%%%%%%%%%%%%%%%%%%%%%%%%
\DescribeMacro{\ifchilddoc}
The conditional |\ifchilddoc| distinguishes between the compilation of
child documents and the main document:
%
\begin{center}
|\ifchilddoc |\textit{child-code}| |[|\||else |\textit{main-code}]| \||fi|
\end{center}

%%%%%%%%%%%%%%%%%%%%%%%%%%%%%%%%%%%%%%%%
\DescribeMacro{\childdocname}
\DescribeMacro{\childdocjob}
The macro |\childdocname| contains the filename (without extension)
of the main or child file being processed.
Note that |\childdocjob| will always contain the name of the main file.

%%%%%%%%%%%%%%%%%%%%%%%%%%%%%%%%%%%%%%%%
\paragraph{Title Page.}

Conditional processing can be used to include a title or banner page
in the main document when proper precautions are taken.
Importantly, the code in the main file should ensure that the page counter
(as well as other status parameters which are stored in the |.aux| files)
takes the same value after the conditional processing.
Otherwise the page numbers may take divergent values
depending on which part is compiled.

For example, a title page could be declared by:
%
\begin{center}
\begin{tabular}{l}
|\ifchilddoc\||else|\\
|\addtocounter{page}{-1}|\\
\textit{code for title page}\\
|\newpage|\\
|\||fi|
\end{tabular}
\end{center}
%
A banner page for the child documents can be generated by:
%
\begin{center}
\begin{tabular}{l}
|\ifchilddoc|\\
|\addtocounter{page}{-1}|\\
\textit{code for banner page}\\
|\newpage|\\
|\||fi|
\end{tabular}
\end{center}
%
Here one could write a message such as:
\begin{center}
|This is the part \childdocname{} of \childdocjob{}.|
\end{center}

%%%%%%%%%%%%%%%%%%%%%%%%%%%%%%%%%%%%%%%%%%%%%%%%%%%%%%%%%%%%%%%%%%%%%%%%%%%%%%%%
\subsection{Flags}
\label{sec:flags}

The package makes it easy to generate different versions
of the main or child documents.
To this end compilation flags can be defined
and assigned different default values.
They will be particularly useful in conjunction
with the forwarding mechanism described in \secref{sec:forward}.

For example, it may be useful to have a flag |\version|
which can be set to |draft| or |final|.
The document source will contain some conditional code
depending on the value of |\version|.
Suppose further, the flag should default to |final| for the main file
and to |draft| for child files
which is a natural assignment for editing the document.
This is achieved by placing the following code
in the preamble of the main document
(below the |\childdocmain| directive):
%
\begin{center}
\begin{tabular}{l}
|\ifchilddoc|\\
|\providecommand{\version}{draft}|\\
|\||else|\\
|\providecommand{\version}{final}|\\
|\||fi|
\end{tabular}
\end{center}
%
The definition by |\providecommand| makes sure
that previous definitions are not overwritten.
Further statements |\providecommand{\version}{...}|
can thus be added before the above code to override it.

For the main file, one might add a line
(between |\childdocmain| and the above block)
%
\begin{center}
|%\ifchilddoc\||else\providecommand{\version}{draft}\||fi|
\end{center}
%
which can be uncommented to produce a draft version.
Likewise one can add a line to the very top of a child file
(above the |\childdocof{|\textit{main}|}| directive)
%
\begin{center}
|%\providecommand{\version}{final}|
\end{center}
%
which can be uncommented to produce the final version of this child document.

%%%%%%%%%%%%%%%%%%%%%%%%%%%%%%%%%%%%%%%%%%%%%%%%%%%%%%%%%%%%%%%%%%%%%%%%%%%%%%%%
\subsection{Forwarding}
\label{sec:forward}

Different versions of the main or child documents
using compilation flags as described in \secref{sec:flags}
can be (permanently) stored in different files
for convenient compilation, viewing and distribution.
To this end, the package defines a command
to pass on compilation to a different file:

%%%%%%%%%%%%%%%%%%%%%%%%%%%%%%%%%%%%%%%%
\DescribeMacro{\childdocforward}
The command |\childdocforward| redirects processing to
another source file:
%
\begin{center}
\begin{tabular}{l}
|\input{childdoc.def}|\\
|\childdocforward[|\textit{main}|]{|\textit{dest}|}|\\
\end{tabular}
\end{center}
%
The argument \textit{dest} is the destination file
(without extension).
It should be the main file or one of the child files.
Note that further \textsf{childdoc} directives
such as |\childdocof| and |\childdocforward|
in the indicated file will be processed in this form.
The optional argument \textit{main}
passes on directly to the main file \textit{main}
while pretending to compile the child \textit{dest}.
This form behaves as if \textit{dest}
issues |\childdocof{|\textit{main}|}| right away,
and no further \textsf{childdoc} directives will be processed.

%%%%%%%%%%%%%%%%%%%%%%%%%%%%%%%%%%%%%%%%
\DescribeMacro{\...prefix}
In the alternative form |\childdocforwardprefix|,
%
\begin{center}
\begin{tabular}{l}
|\input{childdoc.def}|\\
|\childdocforwardprefix[|\textit{main}|]{|\textit{prefix}|}{|\textit{dest}|}|
\end{tabular}
\end{center}
%
the destination file is determined by a pattern
depending on the current file:
To make this work, the current file must be called
`{\textit{prefix}\hspace{0.2em}\textit{suffix}}'
with \textit{prefix} matching precisely the argument.
Processing is then passed on to the file
`{\textit{dest}\hspace{0.2em}\textit{suffix}}'.
Surely, the same effect is achieved by
directly specifying the
argument `{\textit{dest}\hspace{0.2em}\textit{suffix}}'
in the first form.
However, that requires to set up a different file
for each child. With the alternative form of the command
all these files can have exactly the same content
which simplifies setting them up and maintaining them.

For example, the following file |draft.tex|
with a compilation flag |\version| as described in \secref{sec:flags}
compiles the main document as a draft:
%
\begin{center}
\begin{tabular}{l}
|\def\version{draft}|\\
|\input{childdoc.def}|\\
|\childdocforward{|\textit{main}|}|
\end{tabular}
\end{center}
%
Likewise, the following files |final|\textit{nn}|.tex|
compile the final version of the child document
|child|\textit{nn}|.tex|:
%
\begin{center}
\begin{tabular}{l}
|\def\version{final}|\\
|\input{childdoc.def}|\\
|\childdocforwardprefix{final}{child}|
\end{tabular}
\end{center}
%

Note that when several versions of a main file and/or of each child file
are to be generated, it may be convenient to set up a |Makefile| or
shell script to automatise the process.

%%%%%%%%%%%%%%%%%%%%%%%%%%%%%%%%%%%%%%%%%%%%%%%%%%%%%%%%%%%%%%%%%%%%%%%%%%%%%%%%
\subsection{Command Line Processing}
\label{sec:commandline}

The effect of redirection files can also be achieved by invoking
the \LaTeX{} compiler with a more elaborate command line.
Most conveniently this should be done as part
of a shell script or a |Makefile|.

When using \textsf{childdoc} in the main file, the following
command lines effectively perform a redirection
(note that depending on the shell being used,
backslashes may have to be doubled: `|\|' $\to$ `|\\|'):
%
\begin{center}
|... -jobname "|\textit{target}|" |\\|"|[\textit{flags}]%
|\input{childdoc.def}\childdocforward[|\textit{main}|]{|\textit{dest}|}"|
\end{center}
%
Here \textit{target} is the name of the output file,
\textit{main} is the name of the main file
and \textit{dest} is the name of the main or child file to be processed
(all filenames without extensions).
The optional argument \textit{main} can be omitted
if \textit{main} matches \textit{dest}.
Optionally, compilation \textit{flags} can be defined via |\def| commands.
This command line makes the \TeX{} engine believe
it is compiling the file \textit{target}
whose content is specified as the latter parameter.
The provided code then forwards the processing to
\textit{main} or \textit{dest} as described in \secref{sec:forward}.

%%%%%%%%%%%%%%%%%%%%%%%%%%%%%%%%%%%%%%%%%%%%%%%%%%%%%%%%%%%%%%%%%%%%%%%%%%%%%%%%
\subsection{Include by Input}
\label{sec:input}

Including child documents by |\include| has some restrictions by design.
Most notably, the content of a child document always occupies
its own set of pages; pages cannot be shared between child documents.
Usually, this behaviour makes perfect sense
because each child document contain an essential part of the document.
However, in some situations it may be desirable to compose
a document from a collection of parts
without having mandatory page breaks between then.
For this case, the package
provides a mechanism to include parts
by |\input| which can also be processed individually.
However, by construction this mechanism
requires manual handling of the content to be output.

%%%%%%%%%%%%%%%%%%%%%%%%%%%%%%%%%%%%%%%%
\DescribeMacro{\ifchilddocmanual}
The main file should be prepared as usual, see \secref{sec:include}.
However, the document body must make a distinction
between processing of an individual part and of the main document, e.g.:
%
\begin{center}
\begin{tabular}{l}
|\ifchilddocmanual|\\
|\input{\childdocname}|\\
|\||else|\\
\textit{document body with }|\input{|\textit{part}|}|\\
|\||fi|
\end{tabular}
\end{center}
%
The conditional |\ifchilddocmanual| is true whenever
a part to be included by |\input| is being compiled,
and the name of the part is stored in |\childdocname|.

%%%%%%%%%%%%%%%%%%%%%%%%%%%%%%%%%%%%%%%%
\DescribeMacro{\childdocby}
Each part to be included by |\input| should start with:
%
\begin{center}
\begin{tabular}{l}
|\input{childdoc.def}|\\
|\childdocby{|\textit{main}|}|\\
\end{tabular}
\end{center}
%
The directive |\childdocby| is similar to |\childdocof|
described in \secref{sec:include},
but the subsequent selection of content must be done manually.
To that end, both |\ifchilddoc| and |\ifchilddocmanual|
will be true upon processing of a part,
and the name of the part is stored in |\childdocname|.
Note that |\jobname| will be set to the filename of the current part
so that each part receives an individual |.aux| file
that does not interfere with the |.aux| file(s) of the main document.
This behaviour can be altered by the alternative form
|\childdocby[*]{|\textit{main}|}| (with a non-empty optional argument)
which uses the |.aux| file of the main document
by setting |\jobname| to \textit{main}.

%%%%%%%%%%%%%%%%%%%%%%%%%%%%%%%%%%%%%%%%%%%%%%%%%%%%%%%%%%%%%%%%%%%%%%%%%%%%%%%%
\subsection{Driver Development}
\label{sec:driver}

The \textsf{childdoc} mechanism can also be use for the development
of definition files such as \LaTeX{} styles or classes.
This case differs from the above setup with multiple parts
included by |\include| in that no |\includeonly| should be invoked.
This can be achieved by starting the include file
(before |\ProvidesPackage|) with:
%
\begin{center}
\begin{tabular}{l}
|\input{childdoc.def}|\\
|\childdocforward{|\textit{main}|}|\\
\end{tabular}
\end{center}
%
or alternatively with:
%
\begin{center}
\begin{tabular}{l}
|\input{childdoc.def}|\\
|\childdocby{|\textit{main}|}|\\
\end{tabular}
\end{center}
%
Both forms have slightly different effects as described above.
The main file is prepared as usual, see \secref{sec:include}.

%%%%%%%%%%%%%%%%%%%%%%%%%%%%%%%%%%%%%%%%%%%%%%%%%%%%%%%%%%%%%%%%%%%%%%%%%%%%%%%%
\subsection{Legacy Detection}
\label{sec:detection}

The directive |\childdocmain| in the main file can detect
whether the complete document or merely a child is to be compiled
even without using the directive |\childdocof|.
This method is deprecated because it is less robust
and there is no compelling reason to use it;
it is merely provided for backward compatibility
and it may be removed in future versions.

If the detection mechanism is to be used,
it is mandatory to correctly specify
the filename of the main file as the argument of |\childdocmain|:
%
\begin{center}
\begin{tabular}{l}
|\input{childdoc.def}|\\
|\childdocmain{|\textit{main}|}|\\
\end{tabular}
\end{center}
%
If |\jobname| does not match the argument \textit{main} of |\childdocmain|,
it is assumed that |\jobname| points to the child file to be compiled.
When using |\childdocmain| with the main file specified as argument,
it suffices to start a child file
with just |\input{|\textit{main}|}|
without loading of the package and using |\childdocof|.
If instead all processing is done
with the appropriate \textsf{childdoc} directives,
the argument of \textit{main} of |\childdocmain| can be empty.

An alternative version of the command line processing described
in \secref{sec:commandline} using the detection mechanism reads:
%
\begin{center}
|... -jobname "|\textit{target}|" "|[\textit{flags}]%
[|\def\jobname{|\textit{dest}|}|]|\input{|\textit{main}|}"|
\end{center}

%%%%%%%%%%%%%%%%%%%%%%%%%%%%%%%%%%%%%%%%%%%%%%%%%%%%%%%%%%%%%%%%%%%%%%%%%%%%%%%%
\subsection{Manual Code}
\label{sec:manual}

In case one cannot be certain whether the definitions file |childdoc.def|
is installed on the target \TeX{} distribution
and one prefers not to ship it,
it is conceivable to paste a few relevant commands into the sources.

To that end, drop all statements |\input{childdoc.def}|
and perform the replacements as outlined below.
Instead of |\childdocmain{|\textit{main}|}| add the following code
to the top of the main file:
%
\begin{center}
\begin{tabular}{l}
|\||ifdefined\childdocname\endinput\||fi\newif\ifchilddoc|\\
|\edef\childdocname{\scantokens\expandafter{\jobname\noexpand}}|\\
|\def\childdocmain{|\textit{main}|}\||ifx\childdocmain\childdocname\||else|\\
|\childdoctrue\includeonly{\childdocname}\let\jobname\childdocmain\||fi|\\
\end{tabular}
\end{center}
%
Instead of |\childdocof{|\textit{main}|}| just include the main file
at the top of each child file:
%
\begin{center}
|\input{|\textit{main}|}|
\end{center}
%
A simple redirection |\childdocforward{|\textit{dest}|}| is achieved by:
%
\begin{center}
|\def\jobname{|\textit{dest}|}\input{\jobname}|
\end{center}
%
The redirection with prefix
|\childdocforwardprefix[|\textit{prefix}|]{|\textit{dest}|}|
is accomplished by:
%
\begin{center}
\begin{tabular}{l}
|{\edef\jobname{\scantokens\expandafter{\jobname\noexpand}}|\\
|\def\redirectjob |\textit{prefix}|#1~~~{\gdef\jobname{|\textit{dest}|#1}}|\\
|\expandafter\redirectjob\jobname~~~}\input{\jobname}|
\end{tabular}
\end{center}

In an alternative approach,
child documents can be compiled by a specific command line
without additional code or specific definitions:
%
\begin{center}
|... -jobname "|\textit{target}|" "|[\textit{flags}]%
|\includeonly{|\textit{dest}|}\input{|\textit{main}|}"|
\end{center}
%

%%%%%%%%%%%%%%%%%%%%%%%%%%%%%%%%%%%%%%%%%%%%%%%%%%%%%%%%%%%%%%%%%%%%%%%%%%%%%%%%
%%%%%%%%%%%%%%%%%%%%%%%%%%%%%%%%%%%%%%%%%%%%%%%%%%%%%%%%%%%%%%%%%%%%%%%%%%%%%%%%
\section{Information}

%%%%%%%%%%%%%%%%%%%%%%%%%%%%%%%%%%%%%%%%%%%%%%%%%%%%%%%%%%%%%%%%%%%%%%%%%%%%%%%%
\subsection{Copyright}

Copyright \copyright{} 2017--2018 Niklas Beisert

This work may be distributed and/or modified under the
conditions of the \LaTeX{} Project Public License, either version 1.3
of this license or (at your option) any later version.
The latest version of this license is in
  \url{http://www.latex-project.org/lppl.txt}
and version 1.3 or later is part of all distributions of \LaTeX{}
version 2005/12/01 or later.

This work has the LPPL maintenance status `maintained'.

The Current Maintainer of this work is Niklas Beisert.

This work consists of the files |README.txt|, |childdoc.ins| and |childdoc.dtx|
as well as the derived files |childdoc.def|, |cdocsamp.tex|
with |cdocsch1.tex|, |cdocsch2.tex|, |cdocspt3.tex|, |cdocspt4.tex|,
|cdocsdrf.tex|, |cdocsfn1.tex|, |cdocsfn2.tex|
as well as |childdoc.pdf|.

%%%%%%%%%%%%%%%%%%%%%%%%%%%%%%%%%%%%%%%%%%%%%%%%%%%%%%%%%%%%%%%%%%%%%%%%%%%%%%%%
\subsection{Files and Installation}

The package consists of the files:
%
\begin{center}
\begin{tabular}{ll}
    |README.txt|   & readme file \\
    |childdoc.ins| & installation file \\
    |childdoc.dtx| & source file \\
    |childdoc.def| & definition file \\
    |cdocsamp.tex| & sample main file \\
    |cdocsch1.tex| & sample include file \\
    |cdocsch2.tex| & sample include file \\
    |cdocspt3.tex| & sample part file \\
    |cdocspt4.tex| & sample part file \\
    |cdocsdrf.tex| & sample redirection file \\
    |cdocsfn1.tex| & sample redirection file \\
    |cdocsfn2.tex| & sample redirection file \\
    |childdoc.pdf| & manual
\end{tabular}
\end{center}
%
The distribution consists of the files
|README.txt|, |childdoc.ins| and |childdoc.dtx|.
%
\begin{itemize}
\item
Run (pdf)\LaTeX{} on |childdoc.dtx|
to compile the manual |childdoc.pdf| (this file).
\item
Run \LaTeX{} on |childdoc.ins| to create the definitions file |childdoc.def|
and the sample |cdocsamp.tex| with include files
|cdocsch1.tex|, |cdocsch2.tex|, |cdocspt3.tex|, |cdocspt4.tex|,
|cdocsdrf.tex|, |cdocsfn1.tex|, |cdocsfn2.tex|.
Then copy the file |childdoc.def| to an appropriate directory of your \LaTeX{}
distribution, e.g.\ \textit{texmf-root}|/tex/latex/childdoc|.
\end{itemize}

%%%%%%%%%%%%%%%%%%%%%%%%%%%%%%%%%%%%%%%%%%%%%%%%%%%%%%%%%%%%%%%%%%%%%%%%%%%%%%%%
\subsection{Related CTAN Packages}

There are several other packages which offer a similar functionality:
%
\begin{itemize}
\item
The packages
\href{http://ctan.org/pkg/docmute}{\textsf{docmute}},
\href{http://ctan.org/pkg/includex}{\textsf{includex}} and
\href{http://ctan.org/pkg/standalone}{\textsf{standalone}}
provide commands to include only the document body of
a child file thus allowing both files to be compiled individually.
\item
The packages \href{http://ctan.org/pkg/subdocs}{\textsf{subdocs}}
and \href{http://ctan.org/pkg/subfiles}{\textsf{subfiles}}
provide structures in which the main and child documents can be
encapsulated and allowing them to be compiled individually.
The inclusion mechanism is different from the conventional |\include|.
\item
The package \href{http://ctan.org/pkg/combine}{\textsf{combine}}
is an elaborate solution to combine several documents into one.
\end{itemize}
%
See also the CTAN topic \href{http://ctan.org/topic/subdocs}{\textsf{subdocs}}
for further related packages.
The present package differs from the above solutions in that
a document structure constructed with the conventional |\include| mechanism
just needs two extra commands at the top of every file
such that all constituent files can be compiled individually.

%%%%%%%%%%%%%%%%%%%%%%%%%%%%%%%%%%%%%%%%%%%%%%%%%%%%%%%%%%%%%%%%%%%%%%%%%%%%%%%%
%\subsection{Feature Suggestions}
%
%The following is a list of features which may be useful for future
%versions of this package:
%%
%\begin{itemize}
%\item
%\ldots
%\end{itemize}

%%%%%%%%%%%%%%%%%%%%%%%%%%%%%%%%%%%%%%%%%%%%%%%%%%%%%%%%%%%%%%%%%%%%%%%%%%%%%%%%
\subsection{Revision History}

%%%%%%%%%%%%%%%%%%%%%%%%%%%%%%%%%%%%%%%%
\paragraph{v2.0:} 2018/12/30

\begin{itemize}
\item
immediate forward processing
\item
added |\childdocby| mechanism
\item
manual restructured
\end{itemize}

%%%%%%%%%%%%%%%%%%%%%%%%%%%%%%%%%%%%%%%%
\paragraph{v1.6:} 2018/01/17

\begin{itemize}
\item
application for development of include files
\item
corrections to manual
\end{itemize}

%%%%%%%%%%%%%%%%%%%%%%%%%%%%%%%%%%%%%%%%
\paragraph{v1.5:} 2017/05/21

\begin{itemize}
\item
more complete structuring introduced
\item
|\childdocof| introduced
\item
|\childdoc| renamed to |\childdocmain|
\item
|\childredirect| renamed to |\childdocforward| and |\childdocforwardprefix|
and functionality expanded
\end{itemize}

%%%%%%%%%%%%%%%%%%%%%%%%%%%%%%%%%%%%%%%%
\paragraph{v1.0:} 2017/04/27

\begin{itemize}
\item
manual and install package
\item
first version published on CTAN
\end{itemize}

%%%%%%%%%%%%%%%%%%%%%%%%%%%%%%%%%%%%%%%%
\paragraph{v0.6:} 2017/04/26

\begin{itemize}
\item
redirection mechanism added
\end{itemize}

%%%%%%%%%%%%%%%%%%%%%%%%%%%%%%%%%%%%%%%%
\paragraph{v0.5:} 2017/04/26

\begin{itemize}
\item
functionality in definition file
\end{itemize}


%%%%%%%%%%%%%%%%%%%%%%%%%%%%%%%%%%%%%%%%%%%%%%%%%%%%%%%%%%%%%%%%%%%%%%%%%%%%%%%%
%%%%%%%%%%%%%%%%%%%%%%%%%%%%%%%%%%%%%%%%%%%%%%%%%%%%%%%%%%%%%%%%%%%%%%%%%%%%%%%%
%%%%%%%%%%%%%%%%%%%%%%%%%%%%%%%%%%%%%%%%%%%%%%%%%%%%%%%%%%%%%%%%%%%%%%%%%%%%%%%%
\appendix

\settowidth\MacroIndent{\rmfamily\scriptsize 000\ }

 \DocInput{childdoc.dtx}

\end{document}
%</driver>
% \fi
%
% %%%%%%%%%%%%%%%%%%%%%%%%%%%%%%%%%%%%%%%%%%%%%%%%%%%%%%%%%%%%%%%%%%%%%%%%%%%%%%
% %%%%%%%%%%%%%%%%%%%%%%%%%%%%%%%%%%%%%%%%%%%%%%%%%%%%%%%%%%%%%%%%%%%%%%%%%%%%%%
% \section{Sample}
%\iffalse
%<*samplemain>
%\fi
%
% The following presents a sample document
% with two chapters, two parts, a title page,
% a compile flag as well as three forwarding files to set the flag.
% It consists of eight |.tex| files:
% \begin{center}
% \begin{tabular}{ll}
% |cdocsamp.tex|&main file\\
% |cdocsch1.tex|&include file for chapter 1\\
% |cdocsch2.tex|&include file for chapter 2\\
% |cdocspt3.tex|&include file for part 3\\
% |cdocspt4.tex|&include file for part 4\\
% |cdocsdrf.tex|&forwarding file for main file in draft mode\\
% |cdocsfi1.tex|&forwarding file for final version of chapter 1\\
% |cdocsfi2.tex|&forwarding file for final version of chapter 2\\
% \end{tabular}
% \end{center}
% Each of the eight files can be compiled directly by the \LaTeX{} compiler.
%
% %%%%%%%%%%%%%%%%%%%%%%%%%%%%%%%%%%%%%%
% \paragraph{Main File.}
%
% The main file is called |cdocsamp.tex|.
%
% Load the \textsf{childdoc} definitions and
% declare the filename for the main document:
%    \begin{macrocode}
\input{childdoc.def}
\childdocmain{}
%    \end{macrocode}

% Optional override for |\version| flag:
%    \begin{macrocode}
%%\ifchilddoc\else\providecommand{\version}{draft}\fi
%    \end{macrocode}

% Define the default values for the |\version| flag
% (|final| for the main file and |draft| for childs):
%    \begin{macrocode}
\ifchilddoc
\providecommand{\version}{draft}
\else
\providecommand{\version}{final}
\fi
%    \end{macrocode}

% Load the standard document class:
%    \begin{macrocode}
\documentclass[12pt]{article}
%    \end{macrocode}

% Start the document body:
%    \begin{macrocode}
\begin{document}
%    \end{macrocode}

% Declare a title page.
% Print title, part of document being processed and version flag:
%    \begin{macrocode}
\addtocounter{page}{-1}
\begin{center}
{\LARGE\bfseries{}childdoc example\par}
\vspace{1cm}
\ifchilddoc
\ifchilddocmanual part\else chapter\fi:
`\childdocname' of `\childdocjob'\par
\else
main document: `\childdocjob'\par
\fi
version: \version\par
\end{center}
\newpage
%    \end{macrocode}

% Manually include selected file,
% otherwise process as usual:
%    \begin{macrocode}
\ifchilddocmanual
\section*{part `\childdocname'}
\input{\childdocname}
\else
%    \end{macrocode}

% Include the two chapters:
%    \begin{macrocode}
\include{cdocsch1}
\include{cdocsch2}
%    \end{macrocode}

% Include the two parts unless only chapters should be displayed:
%    \begin{macrocode}
\ifchilddoc\else
\section{part three}
\input{cdocspt3}
\section{part four}
\input{cdocspt4}
\fi
%    \end{macrocode}

% Process as usual until here:
%    \begin{macrocode}
\fi
%    \end{macrocode}

% End of document body:
%    \begin{macrocode}
\end{document}
%    \end{macrocode}
%\iffalse
%</samplemain>
%\fi
%
% %%%%%%%%%%%%%%%%%%%%%%%%%%%%%%%%%%%%%%
% \paragraph{Chapter Include Files.}
%
% The include files are called |cdocsch1.tex| and |cdocsch2.tex|.
%
%\iffalse
%<*samplechap1|samplechap2>
%\fi

% Optional override for |\version| flag:
%    \begin{macrocode}
%%\providecommand{\version}{final}
%    \end{macrocode}

% Include the main document:
%    \begin{macrocode}
\input{childdoc.def}
\childdocof{cdocsamp}
%    \end{macrocode}

%\iffalse
%</samplechap1|samplechap2>
%\fi
%
%\iffalse
%<*samplechap1>
%\fi
% Some text for chapter 1:
%    \begin{macrocode}
\section{one}
some text in chapter one
%    \end{macrocode}

%\iffalse
%</samplechap1>
%\fi
% Some text for chapter 2:
%\iffalse
%<*samplechap2>
%\fi
%    \begin{macrocode}
\section{two}
more text in chapter two
%    \end{macrocode}

%\iffalse
%</samplechap2>
%\fi
%
% %%%%%%%%%%%%%%%%%%%%%%%%%%%%%%%%%%%%%%
% \paragraph{Part Include Files.}
%
% The include files are called |cdocspt3.tex| and |cdocspt4.tex|.
%
%\iffalse
%<*samplepart3|samplepart4>
%\fi

% Optional override for |\version| flag:
%    \begin{macrocode}
%%\providecommand{\version}{final}
%    \end{macrocode}

% Include the main document:
%    \begin{macrocode}
\input{childdoc.def}
\childdocby{cdocsamp}
%    \end{macrocode}

%\iffalse
%</samplepart3|samplepart4>
%\fi
%
%\iffalse
%<*samplepart3>
%\fi
% Some text for part 3:
%    \begin{macrocode}
some text in part three
%    \end{macrocode}

%\iffalse
%</samplepart3>
%\fi
% Some text for part 4:
%\iffalse
%<*samplepart4>
%\fi
%    \begin{macrocode}
more text in part four
%    \end{macrocode}

%\iffalse
%</samplepart4>
%\fi
%
% %%%%%%%%%%%%%%%%%%%%%%%%%%%%%%%%%%%%%%
% \paragraph{Forwarding for a Complete Draft.}
%
% The following forwarding file |cdocsdrf.tex|
% compiles the main document in draft mode:
%\iffalse
%<*sampledraft>
%\fi
%    \begin{macrocode}
\def\version{draft}
\input{childdoc.def}
\childdocforward{cdocsamp}
%    \end{macrocode}

%\iffalse
%</sampledraft>
%\fi
%
% %%%%%%%%%%%%%%%%%%%%%%%%%%%%%%%%%%%%%%
% \paragraph{Forwarding for Final Version of the Chapters.}
%
% The following forwarding files |cdocsfn1.tex| and |cdocsfn2.tex|
% (with identical content)
% compile the final versions of the child documents
% |cdocsch1.tex| and |cdocsch2.tex|, respectively:
%\iffalse
%<*samplefinal>
%\fi
%    \begin{macrocode}
\def\version{final}
\input{childdoc.def}
\childdocforwardprefix[cdocsamp]{cdocsfn}{cdocsch}
%    \end{macrocode}

%\iffalse
%</samplefinal>
%\fi
%
% %%%%%%%%%%%%%%%%%%%%%%%%%%%%%%%%%%%%%%
% \paragraph{Command Line Processing.}
%
% The following three command lines generate the output files
% |cdocscld|, |cdocscl1| and |cdocscl2|
% which should be identical to
% |cdocsdrf|, |cdocsch1| and |cdocsfn2|, respectively:
% \begin{center}
% \begin{tabular}{l}
% |latex -jobname cdocscld \|\\
% |  "\def\version{draft}\input{childdoc.def}\childdocforward{cdocsamp}"|\\
% |latex -jobname cdocscl1 \|\\
% |  "\input{childdoc.def}\childdocforward[cdocsamp]{cdocsch1}"|\\
% |latex -jobname cdocscl2 \|\\
% |  "\def\version{final}\input{childdoc.def}\childdocforward{cdocsch2}"|
% \end{tabular}
% \end{center}
% Note that the trailing backslash on each first line
% merely continues the input to the second line
% (for convenient cut ant paste).
% Furthermore, the command |latex| can be replaced by any
% of its alternative versions such as |pdflatex|.
%
% %%%%%%%%%%%%%%%%%%%%%%%%%%%%%%%%%%%%%%%%%%%%%%%%%%%%%%%%%%%%%%%%%%%%%%%%%%%%%%
% %%%%%%%%%%%%%%%%%%%%%%%%%%%%%%%%%%%%%%%%%%%%%%%%%%%%%%%%%%%%%%%%%%%%%%%%%%%%%%
% \section{Implementation}
%\iffalse
%<*package>
%\fi
%
% This section describes the definitions file |childdoc.def|.

% The definitions cannot be loaded using |\usepackage| or |\RequirePackage|
% which has a mechanism to prevent loading a style file more than once.
% When loading the definitions by means of |\input|
% multiple instances have to be prevented manually:
%\iffalse
%This code needs to be before the `\ProvidesFile' directive
%which is defined at the beginning of this file.
%Therefore it is also placed there and commented out here.
%</package>
%<*discard>
%\fi
%    \begin{macrocode}
\ifdefined\childdocmain\endinput\fi
%    \end{macrocode}
%\iffalse
%</discard>
%<*package>
%\fi
%
% \macro{\ifchilddoc}
% \macro{\ifchilddocmanual}
% The conditional |\ifchilddoc| tells whether a
% child (true) or main (false) document is being compiled.
% The conditional |\ifchilddocmanual| tells whether
% the |\includeonly| mechanism is used (false) or
% the selection of child files must be performed manually (true).
% The definitions initialise to false:
%    \begin{macrocode}
\newif\ifchilddoc
\newif\ifchilddocmanual
%    \end{macrocode}

% \macro{\childdocname}
% \macro{\childdocjob}
% The macro |\childdocname| stores the name of the main document
% to be compiled. The macro |\childdocjob| stores the name of
% the document on which the \LaTeX{} compiler was originally invoked.
% The content of |\jobname| cannot be compared
% to filenames specified in the source due to different catcodes.
% The following code rescans |\jobname|, stores the result
% in |\childdocname| and saves a copy in |\childdocjob|:
%    \begin{macrocode}
\edef\childdocname{\scantokens\expandafter{\jobname\noexpand}}
\let\childdocjob\childdocname
%    \end{macrocode}

% \macro{\childdocdisable}
% The macro |\childdocdisable| prevents the main file
% from being processed more than once.
% At this stage, the main document command |\childdocmain|
% is assumed to be called once again where it should do nothing.
% Any subsequent call to it should prevent
% a secondary processing of the main document
% It overwrites the forwarding commands
% |\childdocof| and |\childdocforward|
% with empty macros to prevent further inclusions of the main document:
%    \begin{macrocode}
\newcommand{\childdocdisable}
{
  \renewcommand{\childdocmain}[1]{\renewcommand{\childdocmain}[1]{\endinput}}
  \renewcommand{\childdocof}[1]{}
  \renewcommand{\childdocby}[2][]{}
  \renewcommand{\childdocforward}[2][]{}
  \renewcommand{\childdocdisable}{}
}
%    \end{macrocode}

% \macro{\childdocmain}
% The macro |\childdocmain| is to be called at the top of the main file
% with nothing or the main filename (without extension) as argument.
% First, it breaks loops.
% If the argument is not empty and does not match |\childdocname|
% (which is set by the first inclusion of |childdoc.def|),
% |\ifchilddoc| is set to true, |\includeonly| is applied to the child file
% and |\jobname| is set to the main file
% (for proper handling of |.aux| files):
%    \begin{macrocode}
\newcommand{\childdocmain}[1]
{
  \childdocdisable\childdocmain{}
  \if?#1?\else
    \begingroup
      \def\childdoctmp{#1}
      \ifx\childdoctmp\childdocname
        \def\childdoctmp{}
      \else
        \def\childdoctmp
        {
          \childdoctrue
          \includeonly{\childdocname}
          \def\childdocjob{#1}
          \def\jobname{#1}
        }
      \fi
      \expandafter
    \endgroup
    \childdoctmp
  \fi
}
%    \end{macrocode}

% \macro{\childdocof}
% The command |\childdocof| redirects
% compilation to the main file |#1|.
%    \begin{macrocode}
\newcommand{\childdocof}[1]
{
  \childdocdisable
  \childdoctrue
  \includeonly{\childdocname}
  \def\jobname{#1}
  \def\childdocjob{#1}
  \input{#1}
}
%    \end{macrocode}

% \macro{\childdocby}
% The command |\childdocby| ....
%    \begin{macrocode}
\newcommand{\childdocby}[2][]
{
  \childdocdisable
  \childdoctrue
  \childdocmanualtrue
  \if?#1?\else
    \def\jobname{#2}
  \fi
  \def\childdocjob{#2}
  \input{#2}
  \endinput
}
%    \end{macrocode}

% \macro{\childdocforward}
% The command |\childdocforward| redirects
% compilation to the main file or
% (if the optional argument is given) a child file.
% Parameters are set as if the main file
% or a child file starting with |\childdocof| was compiled.
% Then compilation is handed over to the main file:
%    \begin{macrocode}
\newcommand{\childdocforward}[2][]
{
  \begingroup
    \if?#1?
      \def\childdoctmp
      {
        \def\childdocname{#2}
        \def\childdocjob{#2}
        \def\jobname{#2}
        \input{#2}
        \endinput
      }
    \else
      \def\childdoctmp
      {
        \childdocdisable
        \def\childdocname{#2}
        \childdoctrue
        \includeonly{#2}
        \def\childdocjob{#1}
        \def\jobname{#1}
        \input{#1}
        \endinput
      }
    \fi
    \expandafter
  \endgroup
  \childdoctmp
}
%    \end{macrocode}

% \macro{\childdocforwardprefix}
% The command |\childdocforwardprefix| redirects
% compilation to the main or a child file by means of a pattern.
% The prefix |#1| in the current filename is replaced by |#2|
% and the suffix of the current filename is kept
% (it is assumed that the filename does not contain the substring `|~~~|'
% which is used as a delimiter).
% Compilation is handed over to the new file by |\childdocforward|:
%    \begin{macrocode}
\newcommand{\childdocforwardprefix}[3][]
{
  \begingroup
    \def\childdocextract #2##1~~~{\def\childdoctmp{\childdocforward[#1]{#3##1}}}
    \expandafter\childdocextract\childdocname~~~
    \expandafter
  \endgroup
  \childdoctmp
}
%    \end{macrocode}

% \macro{\childdoc}
% The deprecated macro |\childdoc| is a legacy version of |\childdocmain|:
%    \begin{macrocode}
\newcommand{\childdoc}{\childdocmain}
%    \end{macrocode}

% \macro{\childdocredirect}
% The deprecated macro |\childdocredirect| is a legacy version
% of |\childdocforward| and |\childdocforwardprefix|:
%    \begin{macrocode}
\newcommand{\childdocredirect}[2][]
{
  \begingroup
    \if?#1?
      \def\childdoctmp{\childdocforward{#2}}
    \else
      \def\childdoctmp{\childdocforwardprefix{#1}{#2}}
    \fi
    \expandafter
  \endgroup
  \childdoctmp
}
%    \end{macrocode}

%\iffalse
%</package>
%\fi
%
\endinput
|\\
|\childdocforwardprefix{final}{child}|
\end{tabular}
\end{center}
%

Note that when several versions of a main file and/or of each child file
are to be generated, it may be convenient to set up a |Makefile| or
shell script to automatise the process.

%%%%%%%%%%%%%%%%%%%%%%%%%%%%%%%%%%%%%%%%%%%%%%%%%%%%%%%%%%%%%%%%%%%%%%%%%%%%%%%%
\subsection{Command Line Processing}
\label{sec:commandline}

The effect of redirection files can also be achieved by invoking
the \LaTeX{} compiler with a more elaborate command line.
Most conveniently this should be done as part
of a shell script or a |Makefile|.

When using \textsf{childdoc} in the main file, the following
command lines effectively perform a redirection
(note that depending on the shell being used,
backslashes may have to be doubled: `|\|' $\to$ `|\\|'):
%
\begin{center}
|... -jobname "|\textit{target}|" |\\|"|[\textit{flags}]%
|% \iffalse
%
% childdoc.dtx Copyright (C) 2017-2018 Niklas Beisert
%
% This work may be distributed and/or modified under the
% conditions of the LaTeX Project Public License, either version 1.3
% of this license or (at your option) any later version.
% The latest version of this license is in
%   http://www.latex-project.org/lppl.txt
% and version 1.3 or later is part of all distributions of LaTeX
% version 2005/12/01 or later.
%
% This work has the LPPL maintenance status `maintained'.
%
% The Current Maintainer of this work is Niklas Beisert.
%
% This work consists of the files childdoc.dtx and childdoc.ins
% and the derived files childdoc.def and cdocsamp.tex with
% cdocsch1.tex, cdocsch2.tex, cdocsdrf.tex, cdocsfn1.tex, cdocsfn2.tex.
%
%<package>\ifdefined\childdocmain\endinput\fi
%<package>\ProvidesFile{childdoc.def}[2018/12/30 v2.0 child document driver]
%<samplemain>\ProvidesFile{cdocsamp.tex}[2018/12/30 v2.0 sample for childdoc]
%<*driver>
%\ProvidesFile{childdoc.drv}[2018/12/30 v2.0 childdoc reference manual file]
\PassOptionsToClass{10pt,a4paper}{article}
\documentclass{ltxdoc}

\usepackage[margin=35mm]{geometry}
\usepackage{hyperref}
\usepackage{hyperxmp}
\usepackage[usenames]{color}

\hypersetup{colorlinks=true}
\hypersetup{pdfstartview=FitH}
\hypersetup{pdfpagemode=UseNone}
\hypersetup{pdfsource={}}
\hypersetup{pdflang={en-UK}}
\hypersetup{pdfcopyright={Copyright 2017-2018 Niklas Beisert.
  This work may be distributed and/or modified under the
  conditions of the LaTeX Project Public License, either version 1.3
  of this license or (at your option) any later version.}}
\hypersetup{pdflicenseurl={http://www.latex-project.org/lppl.txt}}
\hypersetup{pdfcontactaddress={ETH Zurich, ITP, HIT K,
  Wolfgang-Pauli-Strasse 27}}
\hypersetup{pdfcontactpostcode={8093}}
\hypersetup{pdfcontactcity={Zurich}}
\hypersetup{pdfcontactcountry={Switzerland}}
\hypersetup{pdfcontactemail={nbeisert@itp.phys.ethz.ch}}
\hypersetup{pdfcontacturl={http://people.phys.ethz.ch/\xmptilde nbeisert/}}

\newcommand{\secref}[1]{\hyperref[#1]{section \ref*{#1}}}

\parskip1ex
\parindent0pt
\let\olditemize\itemize
\def\itemize{\olditemize\parskip0pt}

\begin{document}

\title{The \textsf{childdoc} Package}
\hypersetup{pdftitle={The childdoc Package}}
\author{Niklas Beisert\\[2ex]
  Institut f\"ur Theoretische Physik\\
  Eidgen\"ossische Technische Hochschule Z\"urich\\
  Wolfgang-Pauli-Strasse 27, 8093 Z\"urich, Switzerland\\[1ex]
  \href{mailto:nbeisert@itp.phys.ethz.ch}
  {\texttt{nbeisert@itp.phys.ethz.ch}}}
\hypersetup{pdfauthor={Niklas Beisert}}
\hypersetup{pdfsubject={Manual for the LaTeX2e Package childdoc}}
\date{30 December 2018, \textsf{v2.0}}
\maketitle

\begin{abstract}\noindent
\textsf{childdoc} is a \LaTeXe{} package
that enables the direct compilation
of document sections included by |\include|
to individual files.
\end{abstract}

\begingroup
\parskip0ex
\tableofcontents
\endgroup

%%%%%%%%%%%%%%%%%%%%%%%%%%%%%%%%%%%%%%%%%%%%%%%%%%%%%%%%%%%%%%%%%%%%%%%%%%%%%%%%
%%%%%%%%%%%%%%%%%%%%%%%%%%%%%%%%%%%%%%%%%%%%%%%%%%%%%%%%%%%%%%%%%%%%%%%%%%%%%%%%
\section{Introduction}

\LaTeX{} provides a mechanism to structure a large document (such as a book)
into a main file and several child files (containing the chapters)
using the |\include| command.
This mechanism is beneficial for documents
which span hundreds of pages in order to
make the source file(s) more manageable.
Moreover, compilation can be restricted to
selected child files by means of the |\includeonly| command.
The latter feature can be used to reduce the compilation time while editing
(this was significantly more useful in the earlier days of \LaTeX{})
or to generate a smaller document which is easier to navigate.
Another application of |\includeonly| is to generate
documents consisting of selected parts of the complete document.

However, there are a few drawbacks of the plain |\include| mechanism:
\begin{itemize}
\item
The child files cannot be compiled on their own,
they can only be compiled via the main file.
A naive editing environment
(such as a text editor with an option
to have the current file processed by \LaTeX)
may require one to switch to the main file before compiling;
attempting to compile the child file produces errors.
\item
The main file must be modified (each time)
to adjust the |\includeonly| command
to the present needs. This easily leaves the main file in a messy state.
\item
The generated document will always carry the filename
of the main document. This is inconvenient if
several child files are to be compiled and
to be kept for distribution.
\end{itemize}

The present package provides a simple interface
to make child files individually compilable by \LaTeX{}.
Compiling a child file then has the same effect as compiling
the main file with an |\includeonly| command
to select the appropriate child.
Moreover the generated document will carry the name of the child
rather than the main file.
This resolves all three above issues.

This feature is meant to make the editing of books,
thesis documents and lecture notes somewhat more convenient.
However, the package can also be used efficiently for
composing a series of documents (such as exercise sheets)
which are typically distributed individually.
It then assists the author in generating the individual documents
(potentially in different versions)
as well as a document containing the collected series.
Another application is in developing style files
or other kinds of included material
where compilation of the style file could redirect
to a sample or test file.

%%%%%%%%%%%%%%%%%%%%%%%%%%%%%%%%%%%%%%%%%%%%%%%%%%%%%%%%%%%%%%%%%%%%%%%%%%%%%%%%
%%%%%%%%%%%%%%%%%%%%%%%%%%%%%%%%%%%%%%%%%%%%%%%%%%%%%%%%%%%%%%%%%%%%%%%%%%%%%%%%
\section{Usage}

First of all, the package \textsf{childdoc} is \emph{not} a standard
\LaTeXe{} |.sty| style file! Therefore it needs to be invoked in
a non-standard way.

%%%%%%%%%%%%%%%%%%%%%%%%%%%%%%%%%%%%%%%%%%%%%%%%%%%%%%%%%%%%%%%%%%%%%%%%%%%%%%%%
\subsection{Included Files}
\label{sec:include}

%%%%%%%%%%%%%%%%%%%%%%%%%%%%%%%%%%%%%%%%
\DescribeMacro{\childdocmain}
To use the package, add the commands
\begin{center}
\begin{tabular}{l}
|\input{childdoc.def}|\\
|\childdocmain{}|\\
\end{tabular}
\end{center}
at the very top of the main \LaTeX{} file,
in particular \emph{before} the |\documentclass| statement!
The argument of |\childdocmain| should be left empty
(but it must be present).

%%%%%%%%%%%%%%%%%%%%%%%%%%%%%%%%%%%%%%%%
\DescribeMacro{\childdocof}
Furthermore, add the commands
\begin{center}
\begin{tabular}{l}
|\input{childdoc.def}|\\
|\childdocof{|\textit{main}|}|\\
\end{tabular}
\end{center}
at the top of every child file \textit{child}
which is included by |\include{|\textit{child}|}|
from within the main file
(or at least for those files to be compiled individually).
The argument \textit{main} must be the filename of the main file.

There are a couple of
considerations in setting up the main and child documents:

%%%%%%%%%%%%%%%%%%%%%%%%%%%%%%%%%%%%%%%%
\paragraph{Restrictions.}

Please note the following restrictions:
\begin{itemize}
\item
|\childdocmain| must be called with one argument \textit{main}
to ensure compatibility with earlier version of the package.
It must either be empty (|\childdocmain{}|)
or precisely match the filename of the main file in which it is specified.
See \secref{sec:detection} for further information.
\item
The filename \textit{main} must be specified without the |.tex| extension.
\item
The filename \textit{main} is case sensitive
(even in case-insensitive file systems)
due to internal string comparison.
\item
The argument \textit{main} should be fully expanded, it cannot be a macro.
\item
Subdirectories and special characters should be avoided in filenames.
\item
The command |\childdocmain{|\textit{main}|}| must be followed by a whitespace.
It should not be followed immediately by another command
or by a comment mark `|%|'.
This is because the \TeX{} parser reads the token immediately following
the argument of |\childdocmain| and puts it
at the beginning of every child section;
however, a white\-space is ignored.
\end{itemize}

%%%%%%%%%%%%%%%%%%%%%%%%%%%%%%%%%%%%%%%%
\paragraph{Content of Main File.}

It is advisable to place all content in the child files included by |\include|.
Any output contained in the main file will appear in all child documents
unless suppressed manually;
it cannot be suppressed automatically by the |\includeonly| directive
and thus should normally be avoided.
A method to include some content in the main file
by means of conditional processing is described in \secref{sec:conditional}.

%%%%%%%%%%%%%%%%%%%%%%%%%%%%%%%%%%%%%%%%
\paragraph{Page Numbering.}

When only a part of the document is compiled,
the appropriate numbering of pages
(as well as other status parameters)
is determined from the |.aux| files.
The latter contain information from previous passes.
However this information needs to propagate through
all intermediate child documents.
Therefore the page numbering in child documents may well
be inconsistent until the complete document is compiled at least once.

A useful (if unconventional) way to always ensure a consistent
page numbering is to restart the numbering in each child document
and denote the pages by `\textit{child}|.|\textit{page}'
where \textit{child} represents the chapter/section number of the child file.
This can be achieved by the command
|\numberwithin{page}{|\textit{child}|}|
of the \textsf{amsmath} package
where \textit{child} can be |chapter| or |section|
depending on the chosen structuring.
Alternatively, one can modify the macro |\thepage| appropriately
and reset the counter |page| at the start of each child file.

%%%%%%%%%%%%%%%%%%%%%%%%%%%%%%%%%%%%%%%%%%%%%%%%%%%%%%%%%%%%%%%%%%%%%%%%%%%%%%%%
\subsection{Conditional Processing}
\label{sec:conditional}

The package provides a mechanism to compile different versions
of a document. To customise the versions further some conditional processing
can come in handy to distinguish which version is being compiled.
The package provides two macros to describe the compilation context:

%%%%%%%%%%%%%%%%%%%%%%%%%%%%%%%%%%%%%%%%
\DescribeMacro{\ifchilddoc}
The conditional |\ifchilddoc| distinguishes between the compilation of
child documents and the main document:
%
\begin{center}
|\ifchilddoc |\textit{child-code}| |[|\||else |\textit{main-code}]| \||fi|
\end{center}

%%%%%%%%%%%%%%%%%%%%%%%%%%%%%%%%%%%%%%%%
\DescribeMacro{\childdocname}
\DescribeMacro{\childdocjob}
The macro |\childdocname| contains the filename (without extension)
of the main or child file being processed.
Note that |\childdocjob| will always contain the name of the main file.

%%%%%%%%%%%%%%%%%%%%%%%%%%%%%%%%%%%%%%%%
\paragraph{Title Page.}

Conditional processing can be used to include a title or banner page
in the main document when proper precautions are taken.
Importantly, the code in the main file should ensure that the page counter
(as well as other status parameters which are stored in the |.aux| files)
takes the same value after the conditional processing.
Otherwise the page numbers may take divergent values
depending on which part is compiled.

For example, a title page could be declared by:
%
\begin{center}
\begin{tabular}{l}
|\ifchilddoc\||else|\\
|\addtocounter{page}{-1}|\\
\textit{code for title page}\\
|\newpage|\\
|\||fi|
\end{tabular}
\end{center}
%
A banner page for the child documents can be generated by:
%
\begin{center}
\begin{tabular}{l}
|\ifchilddoc|\\
|\addtocounter{page}{-1}|\\
\textit{code for banner page}\\
|\newpage|\\
|\||fi|
\end{tabular}
\end{center}
%
Here one could write a message such as:
\begin{center}
|This is the part \childdocname{} of \childdocjob{}.|
\end{center}

%%%%%%%%%%%%%%%%%%%%%%%%%%%%%%%%%%%%%%%%%%%%%%%%%%%%%%%%%%%%%%%%%%%%%%%%%%%%%%%%
\subsection{Flags}
\label{sec:flags}

The package makes it easy to generate different versions
of the main or child documents.
To this end compilation flags can be defined
and assigned different default values.
They will be particularly useful in conjunction
with the forwarding mechanism described in \secref{sec:forward}.

For example, it may be useful to have a flag |\version|
which can be set to |draft| or |final|.
The document source will contain some conditional code
depending on the value of |\version|.
Suppose further, the flag should default to |final| for the main file
and to |draft| for child files
which is a natural assignment for editing the document.
This is achieved by placing the following code
in the preamble of the main document
(below the |\childdocmain| directive):
%
\begin{center}
\begin{tabular}{l}
|\ifchilddoc|\\
|\providecommand{\version}{draft}|\\
|\||else|\\
|\providecommand{\version}{final}|\\
|\||fi|
\end{tabular}
\end{center}
%
The definition by |\providecommand| makes sure
that previous definitions are not overwritten.
Further statements |\providecommand{\version}{...}|
can thus be added before the above code to override it.

For the main file, one might add a line
(between |\childdocmain| and the above block)
%
\begin{center}
|%\ifchilddoc\||else\providecommand{\version}{draft}\||fi|
\end{center}
%
which can be uncommented to produce a draft version.
Likewise one can add a line to the very top of a child file
(above the |\childdocof{|\textit{main}|}| directive)
%
\begin{center}
|%\providecommand{\version}{final}|
\end{center}
%
which can be uncommented to produce the final version of this child document.

%%%%%%%%%%%%%%%%%%%%%%%%%%%%%%%%%%%%%%%%%%%%%%%%%%%%%%%%%%%%%%%%%%%%%%%%%%%%%%%%
\subsection{Forwarding}
\label{sec:forward}

Different versions of the main or child documents
using compilation flags as described in \secref{sec:flags}
can be (permanently) stored in different files
for convenient compilation, viewing and distribution.
To this end, the package defines a command
to pass on compilation to a different file:

%%%%%%%%%%%%%%%%%%%%%%%%%%%%%%%%%%%%%%%%
\DescribeMacro{\childdocforward}
The command |\childdocforward| redirects processing to
another source file:
%
\begin{center}
\begin{tabular}{l}
|\input{childdoc.def}|\\
|\childdocforward[|\textit{main}|]{|\textit{dest}|}|\\
\end{tabular}
\end{center}
%
The argument \textit{dest} is the destination file
(without extension).
It should be the main file or one of the child files.
Note that further \textsf{childdoc} directives
such as |\childdocof| and |\childdocforward|
in the indicated file will be processed in this form.
The optional argument \textit{main}
passes on directly to the main file \textit{main}
while pretending to compile the child \textit{dest}.
This form behaves as if \textit{dest}
issues |\childdocof{|\textit{main}|}| right away,
and no further \textsf{childdoc} directives will be processed.

%%%%%%%%%%%%%%%%%%%%%%%%%%%%%%%%%%%%%%%%
\DescribeMacro{\...prefix}
In the alternative form |\childdocforwardprefix|,
%
\begin{center}
\begin{tabular}{l}
|\input{childdoc.def}|\\
|\childdocforwardprefix[|\textit{main}|]{|\textit{prefix}|}{|\textit{dest}|}|
\end{tabular}
\end{center}
%
the destination file is determined by a pattern
depending on the current file:
To make this work, the current file must be called
`{\textit{prefix}\hspace{0.2em}\textit{suffix}}'
with \textit{prefix} matching precisely the argument.
Processing is then passed on to the file
`{\textit{dest}\hspace{0.2em}\textit{suffix}}'.
Surely, the same effect is achieved by
directly specifying the
argument `{\textit{dest}\hspace{0.2em}\textit{suffix}}'
in the first form.
However, that requires to set up a different file
for each child. With the alternative form of the command
all these files can have exactly the same content
which simplifies setting them up and maintaining them.

For example, the following file |draft.tex|
with a compilation flag |\version| as described in \secref{sec:flags}
compiles the main document as a draft:
%
\begin{center}
\begin{tabular}{l}
|\def\version{draft}|\\
|\input{childdoc.def}|\\
|\childdocforward{|\textit{main}|}|
\end{tabular}
\end{center}
%
Likewise, the following files |final|\textit{nn}|.tex|
compile the final version of the child document
|child|\textit{nn}|.tex|:
%
\begin{center}
\begin{tabular}{l}
|\def\version{final}|\\
|\input{childdoc.def}|\\
|\childdocforwardprefix{final}{child}|
\end{tabular}
\end{center}
%

Note that when several versions of a main file and/or of each child file
are to be generated, it may be convenient to set up a |Makefile| or
shell script to automatise the process.

%%%%%%%%%%%%%%%%%%%%%%%%%%%%%%%%%%%%%%%%%%%%%%%%%%%%%%%%%%%%%%%%%%%%%%%%%%%%%%%%
\subsection{Command Line Processing}
\label{sec:commandline}

The effect of redirection files can also be achieved by invoking
the \LaTeX{} compiler with a more elaborate command line.
Most conveniently this should be done as part
of a shell script or a |Makefile|.

When using \textsf{childdoc} in the main file, the following
command lines effectively perform a redirection
(note that depending on the shell being used,
backslashes may have to be doubled: `|\|' $\to$ `|\\|'):
%
\begin{center}
|... -jobname "|\textit{target}|" |\\|"|[\textit{flags}]%
|\input{childdoc.def}\childdocforward[|\textit{main}|]{|\textit{dest}|}"|
\end{center}
%
Here \textit{target} is the name of the output file,
\textit{main} is the name of the main file
and \textit{dest} is the name of the main or child file to be processed
(all filenames without extensions).
The optional argument \textit{main} can be omitted
if \textit{main} matches \textit{dest}.
Optionally, compilation \textit{flags} can be defined via |\def| commands.
This command line makes the \TeX{} engine believe
it is compiling the file \textit{target}
whose content is specified as the latter parameter.
The provided code then forwards the processing to
\textit{main} or \textit{dest} as described in \secref{sec:forward}.

%%%%%%%%%%%%%%%%%%%%%%%%%%%%%%%%%%%%%%%%%%%%%%%%%%%%%%%%%%%%%%%%%%%%%%%%%%%%%%%%
\subsection{Include by Input}
\label{sec:input}

Including child documents by |\include| has some restrictions by design.
Most notably, the content of a child document always occupies
its own set of pages; pages cannot be shared between child documents.
Usually, this behaviour makes perfect sense
because each child document contain an essential part of the document.
However, in some situations it may be desirable to compose
a document from a collection of parts
without having mandatory page breaks between then.
For this case, the package
provides a mechanism to include parts
by |\input| which can also be processed individually.
However, by construction this mechanism
requires manual handling of the content to be output.

%%%%%%%%%%%%%%%%%%%%%%%%%%%%%%%%%%%%%%%%
\DescribeMacro{\ifchilddocmanual}
The main file should be prepared as usual, see \secref{sec:include}.
However, the document body must make a distinction
between processing of an individual part and of the main document, e.g.:
%
\begin{center}
\begin{tabular}{l}
|\ifchilddocmanual|\\
|\input{\childdocname}|\\
|\||else|\\
\textit{document body with }|\input{|\textit{part}|}|\\
|\||fi|
\end{tabular}
\end{center}
%
The conditional |\ifchilddocmanual| is true whenever
a part to be included by |\input| is being compiled,
and the name of the part is stored in |\childdocname|.

%%%%%%%%%%%%%%%%%%%%%%%%%%%%%%%%%%%%%%%%
\DescribeMacro{\childdocby}
Each part to be included by |\input| should start with:
%
\begin{center}
\begin{tabular}{l}
|\input{childdoc.def}|\\
|\childdocby{|\textit{main}|}|\\
\end{tabular}
\end{center}
%
The directive |\childdocby| is similar to |\childdocof|
described in \secref{sec:include},
but the subsequent selection of content must be done manually.
To that end, both |\ifchilddoc| and |\ifchilddocmanual|
will be true upon processing of a part,
and the name of the part is stored in |\childdocname|.
Note that |\jobname| will be set to the filename of the current part
so that each part receives an individual |.aux| file
that does not interfere with the |.aux| file(s) of the main document.
This behaviour can be altered by the alternative form
|\childdocby[*]{|\textit{main}|}| (with a non-empty optional argument)
which uses the |.aux| file of the main document
by setting |\jobname| to \textit{main}.

%%%%%%%%%%%%%%%%%%%%%%%%%%%%%%%%%%%%%%%%%%%%%%%%%%%%%%%%%%%%%%%%%%%%%%%%%%%%%%%%
\subsection{Driver Development}
\label{sec:driver}

The \textsf{childdoc} mechanism can also be use for the development
of definition files such as \LaTeX{} styles or classes.
This case differs from the above setup with multiple parts
included by |\include| in that no |\includeonly| should be invoked.
This can be achieved by starting the include file
(before |\ProvidesPackage|) with:
%
\begin{center}
\begin{tabular}{l}
|\input{childdoc.def}|\\
|\childdocforward{|\textit{main}|}|\\
\end{tabular}
\end{center}
%
or alternatively with:
%
\begin{center}
\begin{tabular}{l}
|\input{childdoc.def}|\\
|\childdocby{|\textit{main}|}|\\
\end{tabular}
\end{center}
%
Both forms have slightly different effects as described above.
The main file is prepared as usual, see \secref{sec:include}.

%%%%%%%%%%%%%%%%%%%%%%%%%%%%%%%%%%%%%%%%%%%%%%%%%%%%%%%%%%%%%%%%%%%%%%%%%%%%%%%%
\subsection{Legacy Detection}
\label{sec:detection}

The directive |\childdocmain| in the main file can detect
whether the complete document or merely a child is to be compiled
even without using the directive |\childdocof|.
This method is deprecated because it is less robust
and there is no compelling reason to use it;
it is merely provided for backward compatibility
and it may be removed in future versions.

If the detection mechanism is to be used,
it is mandatory to correctly specify
the filename of the main file as the argument of |\childdocmain|:
%
\begin{center}
\begin{tabular}{l}
|\input{childdoc.def}|\\
|\childdocmain{|\textit{main}|}|\\
\end{tabular}
\end{center}
%
If |\jobname| does not match the argument \textit{main} of |\childdocmain|,
it is assumed that |\jobname| points to the child file to be compiled.
When using |\childdocmain| with the main file specified as argument,
it suffices to start a child file
with just |\input{|\textit{main}|}|
without loading of the package and using |\childdocof|.
If instead all processing is done
with the appropriate \textsf{childdoc} directives,
the argument of \textit{main} of |\childdocmain| can be empty.

An alternative version of the command line processing described
in \secref{sec:commandline} using the detection mechanism reads:
%
\begin{center}
|... -jobname "|\textit{target}|" "|[\textit{flags}]%
[|\def\jobname{|\textit{dest}|}|]|\input{|\textit{main}|}"|
\end{center}

%%%%%%%%%%%%%%%%%%%%%%%%%%%%%%%%%%%%%%%%%%%%%%%%%%%%%%%%%%%%%%%%%%%%%%%%%%%%%%%%
\subsection{Manual Code}
\label{sec:manual}

In case one cannot be certain whether the definitions file |childdoc.def|
is installed on the target \TeX{} distribution
and one prefers not to ship it,
it is conceivable to paste a few relevant commands into the sources.

To that end, drop all statements |\input{childdoc.def}|
and perform the replacements as outlined below.
Instead of |\childdocmain{|\textit{main}|}| add the following code
to the top of the main file:
%
\begin{center}
\begin{tabular}{l}
|\||ifdefined\childdocname\endinput\||fi\newif\ifchilddoc|\\
|\edef\childdocname{\scantokens\expandafter{\jobname\noexpand}}|\\
|\def\childdocmain{|\textit{main}|}\||ifx\childdocmain\childdocname\||else|\\
|\childdoctrue\includeonly{\childdocname}\let\jobname\childdocmain\||fi|\\
\end{tabular}
\end{center}
%
Instead of |\childdocof{|\textit{main}|}| just include the main file
at the top of each child file:
%
\begin{center}
|\input{|\textit{main}|}|
\end{center}
%
A simple redirection |\childdocforward{|\textit{dest}|}| is achieved by:
%
\begin{center}
|\def\jobname{|\textit{dest}|}\input{\jobname}|
\end{center}
%
The redirection with prefix
|\childdocforwardprefix[|\textit{prefix}|]{|\textit{dest}|}|
is accomplished by:
%
\begin{center}
\begin{tabular}{l}
|{\edef\jobname{\scantokens\expandafter{\jobname\noexpand}}|\\
|\def\redirectjob |\textit{prefix}|#1~~~{\gdef\jobname{|\textit{dest}|#1}}|\\
|\expandafter\redirectjob\jobname~~~}\input{\jobname}|
\end{tabular}
\end{center}

In an alternative approach,
child documents can be compiled by a specific command line
without additional code or specific definitions:
%
\begin{center}
|... -jobname "|\textit{target}|" "|[\textit{flags}]%
|\includeonly{|\textit{dest}|}\input{|\textit{main}|}"|
\end{center}
%

%%%%%%%%%%%%%%%%%%%%%%%%%%%%%%%%%%%%%%%%%%%%%%%%%%%%%%%%%%%%%%%%%%%%%%%%%%%%%%%%
%%%%%%%%%%%%%%%%%%%%%%%%%%%%%%%%%%%%%%%%%%%%%%%%%%%%%%%%%%%%%%%%%%%%%%%%%%%%%%%%
\section{Information}

%%%%%%%%%%%%%%%%%%%%%%%%%%%%%%%%%%%%%%%%%%%%%%%%%%%%%%%%%%%%%%%%%%%%%%%%%%%%%%%%
\subsection{Copyright}

Copyright \copyright{} 2017--2018 Niklas Beisert

This work may be distributed and/or modified under the
conditions of the \LaTeX{} Project Public License, either version 1.3
of this license or (at your option) any later version.
The latest version of this license is in
  \url{http://www.latex-project.org/lppl.txt}
and version 1.3 or later is part of all distributions of \LaTeX{}
version 2005/12/01 or later.

This work has the LPPL maintenance status `maintained'.

The Current Maintainer of this work is Niklas Beisert.

This work consists of the files |README.txt|, |childdoc.ins| and |childdoc.dtx|
as well as the derived files |childdoc.def|, |cdocsamp.tex|
with |cdocsch1.tex|, |cdocsch2.tex|, |cdocspt3.tex|, |cdocspt4.tex|,
|cdocsdrf.tex|, |cdocsfn1.tex|, |cdocsfn2.tex|
as well as |childdoc.pdf|.

%%%%%%%%%%%%%%%%%%%%%%%%%%%%%%%%%%%%%%%%%%%%%%%%%%%%%%%%%%%%%%%%%%%%%%%%%%%%%%%%
\subsection{Files and Installation}

The package consists of the files:
%
\begin{center}
\begin{tabular}{ll}
    |README.txt|   & readme file \\
    |childdoc.ins| & installation file \\
    |childdoc.dtx| & source file \\
    |childdoc.def| & definition file \\
    |cdocsamp.tex| & sample main file \\
    |cdocsch1.tex| & sample include file \\
    |cdocsch2.tex| & sample include file \\
    |cdocspt3.tex| & sample part file \\
    |cdocspt4.tex| & sample part file \\
    |cdocsdrf.tex| & sample redirection file \\
    |cdocsfn1.tex| & sample redirection file \\
    |cdocsfn2.tex| & sample redirection file \\
    |childdoc.pdf| & manual
\end{tabular}
\end{center}
%
The distribution consists of the files
|README.txt|, |childdoc.ins| and |childdoc.dtx|.
%
\begin{itemize}
\item
Run (pdf)\LaTeX{} on |childdoc.dtx|
to compile the manual |childdoc.pdf| (this file).
\item
Run \LaTeX{} on |childdoc.ins| to create the definitions file |childdoc.def|
and the sample |cdocsamp.tex| with include files
|cdocsch1.tex|, |cdocsch2.tex|, |cdocspt3.tex|, |cdocspt4.tex|,
|cdocsdrf.tex|, |cdocsfn1.tex|, |cdocsfn2.tex|.
Then copy the file |childdoc.def| to an appropriate directory of your \LaTeX{}
distribution, e.g.\ \textit{texmf-root}|/tex/latex/childdoc|.
\end{itemize}

%%%%%%%%%%%%%%%%%%%%%%%%%%%%%%%%%%%%%%%%%%%%%%%%%%%%%%%%%%%%%%%%%%%%%%%%%%%%%%%%
\subsection{Related CTAN Packages}

There are several other packages which offer a similar functionality:
%
\begin{itemize}
\item
The packages
\href{http://ctan.org/pkg/docmute}{\textsf{docmute}},
\href{http://ctan.org/pkg/includex}{\textsf{includex}} and
\href{http://ctan.org/pkg/standalone}{\textsf{standalone}}
provide commands to include only the document body of
a child file thus allowing both files to be compiled individually.
\item
The packages \href{http://ctan.org/pkg/subdocs}{\textsf{subdocs}}
and \href{http://ctan.org/pkg/subfiles}{\textsf{subfiles}}
provide structures in which the main and child documents can be
encapsulated and allowing them to be compiled individually.
The inclusion mechanism is different from the conventional |\include|.
\item
The package \href{http://ctan.org/pkg/combine}{\textsf{combine}}
is an elaborate solution to combine several documents into one.
\end{itemize}
%
See also the CTAN topic \href{http://ctan.org/topic/subdocs}{\textsf{subdocs}}
for further related packages.
The present package differs from the above solutions in that
a document structure constructed with the conventional |\include| mechanism
just needs two extra commands at the top of every file
such that all constituent files can be compiled individually.

%%%%%%%%%%%%%%%%%%%%%%%%%%%%%%%%%%%%%%%%%%%%%%%%%%%%%%%%%%%%%%%%%%%%%%%%%%%%%%%%
%\subsection{Feature Suggestions}
%
%The following is a list of features which may be useful for future
%versions of this package:
%%
%\begin{itemize}
%\item
%\ldots
%\end{itemize}

%%%%%%%%%%%%%%%%%%%%%%%%%%%%%%%%%%%%%%%%%%%%%%%%%%%%%%%%%%%%%%%%%%%%%%%%%%%%%%%%
\subsection{Revision History}

%%%%%%%%%%%%%%%%%%%%%%%%%%%%%%%%%%%%%%%%
\paragraph{v2.0:} 2018/12/30

\begin{itemize}
\item
immediate forward processing
\item
added |\childdocby| mechanism
\item
manual restructured
\end{itemize}

%%%%%%%%%%%%%%%%%%%%%%%%%%%%%%%%%%%%%%%%
\paragraph{v1.6:} 2018/01/17

\begin{itemize}
\item
application for development of include files
\item
corrections to manual
\end{itemize}

%%%%%%%%%%%%%%%%%%%%%%%%%%%%%%%%%%%%%%%%
\paragraph{v1.5:} 2017/05/21

\begin{itemize}
\item
more complete structuring introduced
\item
|\childdocof| introduced
\item
|\childdoc| renamed to |\childdocmain|
\item
|\childredirect| renamed to |\childdocforward| and |\childdocforwardprefix|
and functionality expanded
\end{itemize}

%%%%%%%%%%%%%%%%%%%%%%%%%%%%%%%%%%%%%%%%
\paragraph{v1.0:} 2017/04/27

\begin{itemize}
\item
manual and install package
\item
first version published on CTAN
\end{itemize}

%%%%%%%%%%%%%%%%%%%%%%%%%%%%%%%%%%%%%%%%
\paragraph{v0.6:} 2017/04/26

\begin{itemize}
\item
redirection mechanism added
\end{itemize}

%%%%%%%%%%%%%%%%%%%%%%%%%%%%%%%%%%%%%%%%
\paragraph{v0.5:} 2017/04/26

\begin{itemize}
\item
functionality in definition file
\end{itemize}


%%%%%%%%%%%%%%%%%%%%%%%%%%%%%%%%%%%%%%%%%%%%%%%%%%%%%%%%%%%%%%%%%%%%%%%%%%%%%%%%
%%%%%%%%%%%%%%%%%%%%%%%%%%%%%%%%%%%%%%%%%%%%%%%%%%%%%%%%%%%%%%%%%%%%%%%%%%%%%%%%
%%%%%%%%%%%%%%%%%%%%%%%%%%%%%%%%%%%%%%%%%%%%%%%%%%%%%%%%%%%%%%%%%%%%%%%%%%%%%%%%
\appendix

\settowidth\MacroIndent{\rmfamily\scriptsize 000\ }

 \DocInput{childdoc.dtx}

\end{document}
%</driver>
% \fi
%
% %%%%%%%%%%%%%%%%%%%%%%%%%%%%%%%%%%%%%%%%%%%%%%%%%%%%%%%%%%%%%%%%%%%%%%%%%%%%%%
% %%%%%%%%%%%%%%%%%%%%%%%%%%%%%%%%%%%%%%%%%%%%%%%%%%%%%%%%%%%%%%%%%%%%%%%%%%%%%%
% \section{Sample}
%\iffalse
%<*samplemain>
%\fi
%
% The following presents a sample document
% with two chapters, two parts, a title page,
% a compile flag as well as three forwarding files to set the flag.
% It consists of eight |.tex| files:
% \begin{center}
% \begin{tabular}{ll}
% |cdocsamp.tex|&main file\\
% |cdocsch1.tex|&include file for chapter 1\\
% |cdocsch2.tex|&include file for chapter 2\\
% |cdocspt3.tex|&include file for part 3\\
% |cdocspt4.tex|&include file for part 4\\
% |cdocsdrf.tex|&forwarding file for main file in draft mode\\
% |cdocsfi1.tex|&forwarding file for final version of chapter 1\\
% |cdocsfi2.tex|&forwarding file for final version of chapter 2\\
% \end{tabular}
% \end{center}
% Each of the eight files can be compiled directly by the \LaTeX{} compiler.
%
% %%%%%%%%%%%%%%%%%%%%%%%%%%%%%%%%%%%%%%
% \paragraph{Main File.}
%
% The main file is called |cdocsamp.tex|.
%
% Load the \textsf{childdoc} definitions and
% declare the filename for the main document:
%    \begin{macrocode}
\input{childdoc.def}
\childdocmain{}
%    \end{macrocode}

% Optional override for |\version| flag:
%    \begin{macrocode}
%%\ifchilddoc\else\providecommand{\version}{draft}\fi
%    \end{macrocode}

% Define the default values for the |\version| flag
% (|final| for the main file and |draft| for childs):
%    \begin{macrocode}
\ifchilddoc
\providecommand{\version}{draft}
\else
\providecommand{\version}{final}
\fi
%    \end{macrocode}

% Load the standard document class:
%    \begin{macrocode}
\documentclass[12pt]{article}
%    \end{macrocode}

% Start the document body:
%    \begin{macrocode}
\begin{document}
%    \end{macrocode}

% Declare a title page.
% Print title, part of document being processed and version flag:
%    \begin{macrocode}
\addtocounter{page}{-1}
\begin{center}
{\LARGE\bfseries{}childdoc example\par}
\vspace{1cm}
\ifchilddoc
\ifchilddocmanual part\else chapter\fi:
`\childdocname' of `\childdocjob'\par
\else
main document: `\childdocjob'\par
\fi
version: \version\par
\end{center}
\newpage
%    \end{macrocode}

% Manually include selected file,
% otherwise process as usual:
%    \begin{macrocode}
\ifchilddocmanual
\section*{part `\childdocname'}
\input{\childdocname}
\else
%    \end{macrocode}

% Include the two chapters:
%    \begin{macrocode}
\include{cdocsch1}
\include{cdocsch2}
%    \end{macrocode}

% Include the two parts unless only chapters should be displayed:
%    \begin{macrocode}
\ifchilddoc\else
\section{part three}
\input{cdocspt3}
\section{part four}
\input{cdocspt4}
\fi
%    \end{macrocode}

% Process as usual until here:
%    \begin{macrocode}
\fi
%    \end{macrocode}

% End of document body:
%    \begin{macrocode}
\end{document}
%    \end{macrocode}
%\iffalse
%</samplemain>
%\fi
%
% %%%%%%%%%%%%%%%%%%%%%%%%%%%%%%%%%%%%%%
% \paragraph{Chapter Include Files.}
%
% The include files are called |cdocsch1.tex| and |cdocsch2.tex|.
%
%\iffalse
%<*samplechap1|samplechap2>
%\fi

% Optional override for |\version| flag:
%    \begin{macrocode}
%%\providecommand{\version}{final}
%    \end{macrocode}

% Include the main document:
%    \begin{macrocode}
\input{childdoc.def}
\childdocof{cdocsamp}
%    \end{macrocode}

%\iffalse
%</samplechap1|samplechap2>
%\fi
%
%\iffalse
%<*samplechap1>
%\fi
% Some text for chapter 1:
%    \begin{macrocode}
\section{one}
some text in chapter one
%    \end{macrocode}

%\iffalse
%</samplechap1>
%\fi
% Some text for chapter 2:
%\iffalse
%<*samplechap2>
%\fi
%    \begin{macrocode}
\section{two}
more text in chapter two
%    \end{macrocode}

%\iffalse
%</samplechap2>
%\fi
%
% %%%%%%%%%%%%%%%%%%%%%%%%%%%%%%%%%%%%%%
% \paragraph{Part Include Files.}
%
% The include files are called |cdocspt3.tex| and |cdocspt4.tex|.
%
%\iffalse
%<*samplepart3|samplepart4>
%\fi

% Optional override for |\version| flag:
%    \begin{macrocode}
%%\providecommand{\version}{final}
%    \end{macrocode}

% Include the main document:
%    \begin{macrocode}
\input{childdoc.def}
\childdocby{cdocsamp}
%    \end{macrocode}

%\iffalse
%</samplepart3|samplepart4>
%\fi
%
%\iffalse
%<*samplepart3>
%\fi
% Some text for part 3:
%    \begin{macrocode}
some text in part three
%    \end{macrocode}

%\iffalse
%</samplepart3>
%\fi
% Some text for part 4:
%\iffalse
%<*samplepart4>
%\fi
%    \begin{macrocode}
more text in part four
%    \end{macrocode}

%\iffalse
%</samplepart4>
%\fi
%
% %%%%%%%%%%%%%%%%%%%%%%%%%%%%%%%%%%%%%%
% \paragraph{Forwarding for a Complete Draft.}
%
% The following forwarding file |cdocsdrf.tex|
% compiles the main document in draft mode:
%\iffalse
%<*sampledraft>
%\fi
%    \begin{macrocode}
\def\version{draft}
\input{childdoc.def}
\childdocforward{cdocsamp}
%    \end{macrocode}

%\iffalse
%</sampledraft>
%\fi
%
% %%%%%%%%%%%%%%%%%%%%%%%%%%%%%%%%%%%%%%
% \paragraph{Forwarding for Final Version of the Chapters.}
%
% The following forwarding files |cdocsfn1.tex| and |cdocsfn2.tex|
% (with identical content)
% compile the final versions of the child documents
% |cdocsch1.tex| and |cdocsch2.tex|, respectively:
%\iffalse
%<*samplefinal>
%\fi
%    \begin{macrocode}
\def\version{final}
\input{childdoc.def}
\childdocforwardprefix[cdocsamp]{cdocsfn}{cdocsch}
%    \end{macrocode}

%\iffalse
%</samplefinal>
%\fi
%
% %%%%%%%%%%%%%%%%%%%%%%%%%%%%%%%%%%%%%%
% \paragraph{Command Line Processing.}
%
% The following three command lines generate the output files
% |cdocscld|, |cdocscl1| and |cdocscl2|
% which should be identical to
% |cdocsdrf|, |cdocsch1| and |cdocsfn2|, respectively:
% \begin{center}
% \begin{tabular}{l}
% |latex -jobname cdocscld \|\\
% |  "\def\version{draft}\input{childdoc.def}\childdocforward{cdocsamp}"|\\
% |latex -jobname cdocscl1 \|\\
% |  "\input{childdoc.def}\childdocforward[cdocsamp]{cdocsch1}"|\\
% |latex -jobname cdocscl2 \|\\
% |  "\def\version{final}\input{childdoc.def}\childdocforward{cdocsch2}"|
% \end{tabular}
% \end{center}
% Note that the trailing backslash on each first line
% merely continues the input to the second line
% (for convenient cut ant paste).
% Furthermore, the command |latex| can be replaced by any
% of its alternative versions such as |pdflatex|.
%
% %%%%%%%%%%%%%%%%%%%%%%%%%%%%%%%%%%%%%%%%%%%%%%%%%%%%%%%%%%%%%%%%%%%%%%%%%%%%%%
% %%%%%%%%%%%%%%%%%%%%%%%%%%%%%%%%%%%%%%%%%%%%%%%%%%%%%%%%%%%%%%%%%%%%%%%%%%%%%%
% \section{Implementation}
%\iffalse
%<*package>
%\fi
%
% This section describes the definitions file |childdoc.def|.

% The definitions cannot be loaded using |\usepackage| or |\RequirePackage|
% which has a mechanism to prevent loading a style file more than once.
% When loading the definitions by means of |\input|
% multiple instances have to be prevented manually:
%\iffalse
%This code needs to be before the `\ProvidesFile' directive
%which is defined at the beginning of this file.
%Therefore it is also placed there and commented out here.
%</package>
%<*discard>
%\fi
%    \begin{macrocode}
\ifdefined\childdocmain\endinput\fi
%    \end{macrocode}
%\iffalse
%</discard>
%<*package>
%\fi
%
% \macro{\ifchilddoc}
% \macro{\ifchilddocmanual}
% The conditional |\ifchilddoc| tells whether a
% child (true) or main (false) document is being compiled.
% The conditional |\ifchilddocmanual| tells whether
% the |\includeonly| mechanism is used (false) or
% the selection of child files must be performed manually (true).
% The definitions initialise to false:
%    \begin{macrocode}
\newif\ifchilddoc
\newif\ifchilddocmanual
%    \end{macrocode}

% \macro{\childdocname}
% \macro{\childdocjob}
% The macro |\childdocname| stores the name of the main document
% to be compiled. The macro |\childdocjob| stores the name of
% the document on which the \LaTeX{} compiler was originally invoked.
% The content of |\jobname| cannot be compared
% to filenames specified in the source due to different catcodes.
% The following code rescans |\jobname|, stores the result
% in |\childdocname| and saves a copy in |\childdocjob|:
%    \begin{macrocode}
\edef\childdocname{\scantokens\expandafter{\jobname\noexpand}}
\let\childdocjob\childdocname
%    \end{macrocode}

% \macro{\childdocdisable}
% The macro |\childdocdisable| prevents the main file
% from being processed more than once.
% At this stage, the main document command |\childdocmain|
% is assumed to be called once again where it should do nothing.
% Any subsequent call to it should prevent
% a secondary processing of the main document
% It overwrites the forwarding commands
% |\childdocof| and |\childdocforward|
% with empty macros to prevent further inclusions of the main document:
%    \begin{macrocode}
\newcommand{\childdocdisable}
{
  \renewcommand{\childdocmain}[1]{\renewcommand{\childdocmain}[1]{\endinput}}
  \renewcommand{\childdocof}[1]{}
  \renewcommand{\childdocby}[2][]{}
  \renewcommand{\childdocforward}[2][]{}
  \renewcommand{\childdocdisable}{}
}
%    \end{macrocode}

% \macro{\childdocmain}
% The macro |\childdocmain| is to be called at the top of the main file
% with nothing or the main filename (without extension) as argument.
% First, it breaks loops.
% If the argument is not empty and does not match |\childdocname|
% (which is set by the first inclusion of |childdoc.def|),
% |\ifchilddoc| is set to true, |\includeonly| is applied to the child file
% and |\jobname| is set to the main file
% (for proper handling of |.aux| files):
%    \begin{macrocode}
\newcommand{\childdocmain}[1]
{
  \childdocdisable\childdocmain{}
  \if?#1?\else
    \begingroup
      \def\childdoctmp{#1}
      \ifx\childdoctmp\childdocname
        \def\childdoctmp{}
      \else
        \def\childdoctmp
        {
          \childdoctrue
          \includeonly{\childdocname}
          \def\childdocjob{#1}
          \def\jobname{#1}
        }
      \fi
      \expandafter
    \endgroup
    \childdoctmp
  \fi
}
%    \end{macrocode}

% \macro{\childdocof}
% The command |\childdocof| redirects
% compilation to the main file |#1|.
%    \begin{macrocode}
\newcommand{\childdocof}[1]
{
  \childdocdisable
  \childdoctrue
  \includeonly{\childdocname}
  \def\jobname{#1}
  \def\childdocjob{#1}
  \input{#1}
}
%    \end{macrocode}

% \macro{\childdocby}
% The command |\childdocby| ....
%    \begin{macrocode}
\newcommand{\childdocby}[2][]
{
  \childdocdisable
  \childdoctrue
  \childdocmanualtrue
  \if?#1?\else
    \def\jobname{#2}
  \fi
  \def\childdocjob{#2}
  \input{#2}
  \endinput
}
%    \end{macrocode}

% \macro{\childdocforward}
% The command |\childdocforward| redirects
% compilation to the main file or
% (if the optional argument is given) a child file.
% Parameters are set as if the main file
% or a child file starting with |\childdocof| was compiled.
% Then compilation is handed over to the main file:
%    \begin{macrocode}
\newcommand{\childdocforward}[2][]
{
  \begingroup
    \if?#1?
      \def\childdoctmp
      {
        \def\childdocname{#2}
        \def\childdocjob{#2}
        \def\jobname{#2}
        \input{#2}
        \endinput
      }
    \else
      \def\childdoctmp
      {
        \childdocdisable
        \def\childdocname{#2}
        \childdoctrue
        \includeonly{#2}
        \def\childdocjob{#1}
        \def\jobname{#1}
        \input{#1}
        \endinput
      }
    \fi
    \expandafter
  \endgroup
  \childdoctmp
}
%    \end{macrocode}

% \macro{\childdocforwardprefix}
% The command |\childdocforwardprefix| redirects
% compilation to the main or a child file by means of a pattern.
% The prefix |#1| in the current filename is replaced by |#2|
% and the suffix of the current filename is kept
% (it is assumed that the filename does not contain the substring `|~~~|'
% which is used as a delimiter).
% Compilation is handed over to the new file by |\childdocforward|:
%    \begin{macrocode}
\newcommand{\childdocforwardprefix}[3][]
{
  \begingroup
    \def\childdocextract #2##1~~~{\def\childdoctmp{\childdocforward[#1]{#3##1}}}
    \expandafter\childdocextract\childdocname~~~
    \expandafter
  \endgroup
  \childdoctmp
}
%    \end{macrocode}

% \macro{\childdoc}
% The deprecated macro |\childdoc| is a legacy version of |\childdocmain|:
%    \begin{macrocode}
\newcommand{\childdoc}{\childdocmain}
%    \end{macrocode}

% \macro{\childdocredirect}
% The deprecated macro |\childdocredirect| is a legacy version
% of |\childdocforward| and |\childdocforwardprefix|:
%    \begin{macrocode}
\newcommand{\childdocredirect}[2][]
{
  \begingroup
    \if?#1?
      \def\childdoctmp{\childdocforward{#2}}
    \else
      \def\childdoctmp{\childdocforwardprefix{#1}{#2}}
    \fi
    \expandafter
  \endgroup
  \childdoctmp
}
%    \end{macrocode}

%\iffalse
%</package>
%\fi
%
\endinput
\childdocforward[|\textit{main}|]{|\textit{dest}|}"|
\end{center}
%
Here \textit{target} is the name of the output file,
\textit{main} is the name of the main file
and \textit{dest} is the name of the main or child file to be processed
(all filenames without extensions).
The optional argument \textit{main} can be omitted
if \textit{main} matches \textit{dest}.
Optionally, compilation \textit{flags} can be defined via |\def| commands.
This command line makes the \TeX{} engine believe
it is compiling the file \textit{target}
whose content is specified as the latter parameter.
The provided code then forwards the processing to
\textit{main} or \textit{dest} as described in \secref{sec:forward}.

%%%%%%%%%%%%%%%%%%%%%%%%%%%%%%%%%%%%%%%%%%%%%%%%%%%%%%%%%%%%%%%%%%%%%%%%%%%%%%%%
\subsection{Include by Input}
\label{sec:input}

Including child documents by |\include| has some restrictions by design.
Most notably, the content of a child document always occupies
its own set of pages; pages cannot be shared between child documents.
Usually, this behaviour makes perfect sense
because each child document contain an essential part of the document.
However, in some situations it may be desirable to compose
a document from a collection of parts
without having mandatory page breaks between then.
For this case, the package
provides a mechanism to include parts
by |\input| which can also be processed individually.
However, by construction this mechanism
requires manual handling of the content to be output.

%%%%%%%%%%%%%%%%%%%%%%%%%%%%%%%%%%%%%%%%
\DescribeMacro{\ifchilddocmanual}
The main file should be prepared as usual, see \secref{sec:include}.
However, the document body must make a distinction
between processing of an individual part and of the main document, e.g.:
%
\begin{center}
\begin{tabular}{l}
|\ifchilddocmanual|\\
|\input{\childdocname}|\\
|\||else|\\
\textit{document body with }|\input{|\textit{part}|}|\\
|\||fi|
\end{tabular}
\end{center}
%
The conditional |\ifchilddocmanual| is true whenever
a part to be included by |\input| is being compiled,
and the name of the part is stored in |\childdocname|.

%%%%%%%%%%%%%%%%%%%%%%%%%%%%%%%%%%%%%%%%
\DescribeMacro{\childdocby}
Each part to be included by |\input| should start with:
%
\begin{center}
\begin{tabular}{l}
|% \iffalse
%
% childdoc.dtx Copyright (C) 2017-2018 Niklas Beisert
%
% This work may be distributed and/or modified under the
% conditions of the LaTeX Project Public License, either version 1.3
% of this license or (at your option) any later version.
% The latest version of this license is in
%   http://www.latex-project.org/lppl.txt
% and version 1.3 or later is part of all distributions of LaTeX
% version 2005/12/01 or later.
%
% This work has the LPPL maintenance status `maintained'.
%
% The Current Maintainer of this work is Niklas Beisert.
%
% This work consists of the files childdoc.dtx and childdoc.ins
% and the derived files childdoc.def and cdocsamp.tex with
% cdocsch1.tex, cdocsch2.tex, cdocsdrf.tex, cdocsfn1.tex, cdocsfn2.tex.
%
%<package>\ifdefined\childdocmain\endinput\fi
%<package>\ProvidesFile{childdoc.def}[2018/12/30 v2.0 child document driver]
%<samplemain>\ProvidesFile{cdocsamp.tex}[2018/12/30 v2.0 sample for childdoc]
%<*driver>
%\ProvidesFile{childdoc.drv}[2018/12/30 v2.0 childdoc reference manual file]
\PassOptionsToClass{10pt,a4paper}{article}
\documentclass{ltxdoc}

\usepackage[margin=35mm]{geometry}
\usepackage{hyperref}
\usepackage{hyperxmp}
\usepackage[usenames]{color}

\hypersetup{colorlinks=true}
\hypersetup{pdfstartview=FitH}
\hypersetup{pdfpagemode=UseNone}
\hypersetup{pdfsource={}}
\hypersetup{pdflang={en-UK}}
\hypersetup{pdfcopyright={Copyright 2017-2018 Niklas Beisert.
  This work may be distributed and/or modified under the
  conditions of the LaTeX Project Public License, either version 1.3
  of this license or (at your option) any later version.}}
\hypersetup{pdflicenseurl={http://www.latex-project.org/lppl.txt}}
\hypersetup{pdfcontactaddress={ETH Zurich, ITP, HIT K,
  Wolfgang-Pauli-Strasse 27}}
\hypersetup{pdfcontactpostcode={8093}}
\hypersetup{pdfcontactcity={Zurich}}
\hypersetup{pdfcontactcountry={Switzerland}}
\hypersetup{pdfcontactemail={nbeisert@itp.phys.ethz.ch}}
\hypersetup{pdfcontacturl={http://people.phys.ethz.ch/\xmptilde nbeisert/}}

\newcommand{\secref}[1]{\hyperref[#1]{section \ref*{#1}}}

\parskip1ex
\parindent0pt
\let\olditemize\itemize
\def\itemize{\olditemize\parskip0pt}

\begin{document}

\title{The \textsf{childdoc} Package}
\hypersetup{pdftitle={The childdoc Package}}
\author{Niklas Beisert\\[2ex]
  Institut f\"ur Theoretische Physik\\
  Eidgen\"ossische Technische Hochschule Z\"urich\\
  Wolfgang-Pauli-Strasse 27, 8093 Z\"urich, Switzerland\\[1ex]
  \href{mailto:nbeisert@itp.phys.ethz.ch}
  {\texttt{nbeisert@itp.phys.ethz.ch}}}
\hypersetup{pdfauthor={Niklas Beisert}}
\hypersetup{pdfsubject={Manual for the LaTeX2e Package childdoc}}
\date{30 December 2018, \textsf{v2.0}}
\maketitle

\begin{abstract}\noindent
\textsf{childdoc} is a \LaTeXe{} package
that enables the direct compilation
of document sections included by |\include|
to individual files.
\end{abstract}

\begingroup
\parskip0ex
\tableofcontents
\endgroup

%%%%%%%%%%%%%%%%%%%%%%%%%%%%%%%%%%%%%%%%%%%%%%%%%%%%%%%%%%%%%%%%%%%%%%%%%%%%%%%%
%%%%%%%%%%%%%%%%%%%%%%%%%%%%%%%%%%%%%%%%%%%%%%%%%%%%%%%%%%%%%%%%%%%%%%%%%%%%%%%%
\section{Introduction}

\LaTeX{} provides a mechanism to structure a large document (such as a book)
into a main file and several child files (containing the chapters)
using the |\include| command.
This mechanism is beneficial for documents
which span hundreds of pages in order to
make the source file(s) more manageable.
Moreover, compilation can be restricted to
selected child files by means of the |\includeonly| command.
The latter feature can be used to reduce the compilation time while editing
(this was significantly more useful in the earlier days of \LaTeX{})
or to generate a smaller document which is easier to navigate.
Another application of |\includeonly| is to generate
documents consisting of selected parts of the complete document.

However, there are a few drawbacks of the plain |\include| mechanism:
\begin{itemize}
\item
The child files cannot be compiled on their own,
they can only be compiled via the main file.
A naive editing environment
(such as a text editor with an option
to have the current file processed by \LaTeX)
may require one to switch to the main file before compiling;
attempting to compile the child file produces errors.
\item
The main file must be modified (each time)
to adjust the |\includeonly| command
to the present needs. This easily leaves the main file in a messy state.
\item
The generated document will always carry the filename
of the main document. This is inconvenient if
several child files are to be compiled and
to be kept for distribution.
\end{itemize}

The present package provides a simple interface
to make child files individually compilable by \LaTeX{}.
Compiling a child file then has the same effect as compiling
the main file with an |\includeonly| command
to select the appropriate child.
Moreover the generated document will carry the name of the child
rather than the main file.
This resolves all three above issues.

This feature is meant to make the editing of books,
thesis documents and lecture notes somewhat more convenient.
However, the package can also be used efficiently for
composing a series of documents (such as exercise sheets)
which are typically distributed individually.
It then assists the author in generating the individual documents
(potentially in different versions)
as well as a document containing the collected series.
Another application is in developing style files
or other kinds of included material
where compilation of the style file could redirect
to a sample or test file.

%%%%%%%%%%%%%%%%%%%%%%%%%%%%%%%%%%%%%%%%%%%%%%%%%%%%%%%%%%%%%%%%%%%%%%%%%%%%%%%%
%%%%%%%%%%%%%%%%%%%%%%%%%%%%%%%%%%%%%%%%%%%%%%%%%%%%%%%%%%%%%%%%%%%%%%%%%%%%%%%%
\section{Usage}

First of all, the package \textsf{childdoc} is \emph{not} a standard
\LaTeXe{} |.sty| style file! Therefore it needs to be invoked in
a non-standard way.

%%%%%%%%%%%%%%%%%%%%%%%%%%%%%%%%%%%%%%%%%%%%%%%%%%%%%%%%%%%%%%%%%%%%%%%%%%%%%%%%
\subsection{Included Files}
\label{sec:include}

%%%%%%%%%%%%%%%%%%%%%%%%%%%%%%%%%%%%%%%%
\DescribeMacro{\childdocmain}
To use the package, add the commands
\begin{center}
\begin{tabular}{l}
|\input{childdoc.def}|\\
|\childdocmain{}|\\
\end{tabular}
\end{center}
at the very top of the main \LaTeX{} file,
in particular \emph{before} the |\documentclass| statement!
The argument of |\childdocmain| should be left empty
(but it must be present).

%%%%%%%%%%%%%%%%%%%%%%%%%%%%%%%%%%%%%%%%
\DescribeMacro{\childdocof}
Furthermore, add the commands
\begin{center}
\begin{tabular}{l}
|\input{childdoc.def}|\\
|\childdocof{|\textit{main}|}|\\
\end{tabular}
\end{center}
at the top of every child file \textit{child}
which is included by |\include{|\textit{child}|}|
from within the main file
(or at least for those files to be compiled individually).
The argument \textit{main} must be the filename of the main file.

There are a couple of
considerations in setting up the main and child documents:

%%%%%%%%%%%%%%%%%%%%%%%%%%%%%%%%%%%%%%%%
\paragraph{Restrictions.}

Please note the following restrictions:
\begin{itemize}
\item
|\childdocmain| must be called with one argument \textit{main}
to ensure compatibility with earlier version of the package.
It must either be empty (|\childdocmain{}|)
or precisely match the filename of the main file in which it is specified.
See \secref{sec:detection} for further information.
\item
The filename \textit{main} must be specified without the |.tex| extension.
\item
The filename \textit{main} is case sensitive
(even in case-insensitive file systems)
due to internal string comparison.
\item
The argument \textit{main} should be fully expanded, it cannot be a macro.
\item
Subdirectories and special characters should be avoided in filenames.
\item
The command |\childdocmain{|\textit{main}|}| must be followed by a whitespace.
It should not be followed immediately by another command
or by a comment mark `|%|'.
This is because the \TeX{} parser reads the token immediately following
the argument of |\childdocmain| and puts it
at the beginning of every child section;
however, a white\-space is ignored.
\end{itemize}

%%%%%%%%%%%%%%%%%%%%%%%%%%%%%%%%%%%%%%%%
\paragraph{Content of Main File.}

It is advisable to place all content in the child files included by |\include|.
Any output contained in the main file will appear in all child documents
unless suppressed manually;
it cannot be suppressed automatically by the |\includeonly| directive
and thus should normally be avoided.
A method to include some content in the main file
by means of conditional processing is described in \secref{sec:conditional}.

%%%%%%%%%%%%%%%%%%%%%%%%%%%%%%%%%%%%%%%%
\paragraph{Page Numbering.}

When only a part of the document is compiled,
the appropriate numbering of pages
(as well as other status parameters)
is determined from the |.aux| files.
The latter contain information from previous passes.
However this information needs to propagate through
all intermediate child documents.
Therefore the page numbering in child documents may well
be inconsistent until the complete document is compiled at least once.

A useful (if unconventional) way to always ensure a consistent
page numbering is to restart the numbering in each child document
and denote the pages by `\textit{child}|.|\textit{page}'
where \textit{child} represents the chapter/section number of the child file.
This can be achieved by the command
|\numberwithin{page}{|\textit{child}|}|
of the \textsf{amsmath} package
where \textit{child} can be |chapter| or |section|
depending on the chosen structuring.
Alternatively, one can modify the macro |\thepage| appropriately
and reset the counter |page| at the start of each child file.

%%%%%%%%%%%%%%%%%%%%%%%%%%%%%%%%%%%%%%%%%%%%%%%%%%%%%%%%%%%%%%%%%%%%%%%%%%%%%%%%
\subsection{Conditional Processing}
\label{sec:conditional}

The package provides a mechanism to compile different versions
of a document. To customise the versions further some conditional processing
can come in handy to distinguish which version is being compiled.
The package provides two macros to describe the compilation context:

%%%%%%%%%%%%%%%%%%%%%%%%%%%%%%%%%%%%%%%%
\DescribeMacro{\ifchilddoc}
The conditional |\ifchilddoc| distinguishes between the compilation of
child documents and the main document:
%
\begin{center}
|\ifchilddoc |\textit{child-code}| |[|\||else |\textit{main-code}]| \||fi|
\end{center}

%%%%%%%%%%%%%%%%%%%%%%%%%%%%%%%%%%%%%%%%
\DescribeMacro{\childdocname}
\DescribeMacro{\childdocjob}
The macro |\childdocname| contains the filename (without extension)
of the main or child file being processed.
Note that |\childdocjob| will always contain the name of the main file.

%%%%%%%%%%%%%%%%%%%%%%%%%%%%%%%%%%%%%%%%
\paragraph{Title Page.}

Conditional processing can be used to include a title or banner page
in the main document when proper precautions are taken.
Importantly, the code in the main file should ensure that the page counter
(as well as other status parameters which are stored in the |.aux| files)
takes the same value after the conditional processing.
Otherwise the page numbers may take divergent values
depending on which part is compiled.

For example, a title page could be declared by:
%
\begin{center}
\begin{tabular}{l}
|\ifchilddoc\||else|\\
|\addtocounter{page}{-1}|\\
\textit{code for title page}\\
|\newpage|\\
|\||fi|
\end{tabular}
\end{center}
%
A banner page for the child documents can be generated by:
%
\begin{center}
\begin{tabular}{l}
|\ifchilddoc|\\
|\addtocounter{page}{-1}|\\
\textit{code for banner page}\\
|\newpage|\\
|\||fi|
\end{tabular}
\end{center}
%
Here one could write a message such as:
\begin{center}
|This is the part \childdocname{} of \childdocjob{}.|
\end{center}

%%%%%%%%%%%%%%%%%%%%%%%%%%%%%%%%%%%%%%%%%%%%%%%%%%%%%%%%%%%%%%%%%%%%%%%%%%%%%%%%
\subsection{Flags}
\label{sec:flags}

The package makes it easy to generate different versions
of the main or child documents.
To this end compilation flags can be defined
and assigned different default values.
They will be particularly useful in conjunction
with the forwarding mechanism described in \secref{sec:forward}.

For example, it may be useful to have a flag |\version|
which can be set to |draft| or |final|.
The document source will contain some conditional code
depending on the value of |\version|.
Suppose further, the flag should default to |final| for the main file
and to |draft| for child files
which is a natural assignment for editing the document.
This is achieved by placing the following code
in the preamble of the main document
(below the |\childdocmain| directive):
%
\begin{center}
\begin{tabular}{l}
|\ifchilddoc|\\
|\providecommand{\version}{draft}|\\
|\||else|\\
|\providecommand{\version}{final}|\\
|\||fi|
\end{tabular}
\end{center}
%
The definition by |\providecommand| makes sure
that previous definitions are not overwritten.
Further statements |\providecommand{\version}{...}|
can thus be added before the above code to override it.

For the main file, one might add a line
(between |\childdocmain| and the above block)
%
\begin{center}
|%\ifchilddoc\||else\providecommand{\version}{draft}\||fi|
\end{center}
%
which can be uncommented to produce a draft version.
Likewise one can add a line to the very top of a child file
(above the |\childdocof{|\textit{main}|}| directive)
%
\begin{center}
|%\providecommand{\version}{final}|
\end{center}
%
which can be uncommented to produce the final version of this child document.

%%%%%%%%%%%%%%%%%%%%%%%%%%%%%%%%%%%%%%%%%%%%%%%%%%%%%%%%%%%%%%%%%%%%%%%%%%%%%%%%
\subsection{Forwarding}
\label{sec:forward}

Different versions of the main or child documents
using compilation flags as described in \secref{sec:flags}
can be (permanently) stored in different files
for convenient compilation, viewing and distribution.
To this end, the package defines a command
to pass on compilation to a different file:

%%%%%%%%%%%%%%%%%%%%%%%%%%%%%%%%%%%%%%%%
\DescribeMacro{\childdocforward}
The command |\childdocforward| redirects processing to
another source file:
%
\begin{center}
\begin{tabular}{l}
|\input{childdoc.def}|\\
|\childdocforward[|\textit{main}|]{|\textit{dest}|}|\\
\end{tabular}
\end{center}
%
The argument \textit{dest} is the destination file
(without extension).
It should be the main file or one of the child files.
Note that further \textsf{childdoc} directives
such as |\childdocof| and |\childdocforward|
in the indicated file will be processed in this form.
The optional argument \textit{main}
passes on directly to the main file \textit{main}
while pretending to compile the child \textit{dest}.
This form behaves as if \textit{dest}
issues |\childdocof{|\textit{main}|}| right away,
and no further \textsf{childdoc} directives will be processed.

%%%%%%%%%%%%%%%%%%%%%%%%%%%%%%%%%%%%%%%%
\DescribeMacro{\...prefix}
In the alternative form |\childdocforwardprefix|,
%
\begin{center}
\begin{tabular}{l}
|\input{childdoc.def}|\\
|\childdocforwardprefix[|\textit{main}|]{|\textit{prefix}|}{|\textit{dest}|}|
\end{tabular}
\end{center}
%
the destination file is determined by a pattern
depending on the current file:
To make this work, the current file must be called
`{\textit{prefix}\hspace{0.2em}\textit{suffix}}'
with \textit{prefix} matching precisely the argument.
Processing is then passed on to the file
`{\textit{dest}\hspace{0.2em}\textit{suffix}}'.
Surely, the same effect is achieved by
directly specifying the
argument `{\textit{dest}\hspace{0.2em}\textit{suffix}}'
in the first form.
However, that requires to set up a different file
for each child. With the alternative form of the command
all these files can have exactly the same content
which simplifies setting them up and maintaining them.

For example, the following file |draft.tex|
with a compilation flag |\version| as described in \secref{sec:flags}
compiles the main document as a draft:
%
\begin{center}
\begin{tabular}{l}
|\def\version{draft}|\\
|\input{childdoc.def}|\\
|\childdocforward{|\textit{main}|}|
\end{tabular}
\end{center}
%
Likewise, the following files |final|\textit{nn}|.tex|
compile the final version of the child document
|child|\textit{nn}|.tex|:
%
\begin{center}
\begin{tabular}{l}
|\def\version{final}|\\
|\input{childdoc.def}|\\
|\childdocforwardprefix{final}{child}|
\end{tabular}
\end{center}
%

Note that when several versions of a main file and/or of each child file
are to be generated, it may be convenient to set up a |Makefile| or
shell script to automatise the process.

%%%%%%%%%%%%%%%%%%%%%%%%%%%%%%%%%%%%%%%%%%%%%%%%%%%%%%%%%%%%%%%%%%%%%%%%%%%%%%%%
\subsection{Command Line Processing}
\label{sec:commandline}

The effect of redirection files can also be achieved by invoking
the \LaTeX{} compiler with a more elaborate command line.
Most conveniently this should be done as part
of a shell script or a |Makefile|.

When using \textsf{childdoc} in the main file, the following
command lines effectively perform a redirection
(note that depending on the shell being used,
backslashes may have to be doubled: `|\|' $\to$ `|\\|'):
%
\begin{center}
|... -jobname "|\textit{target}|" |\\|"|[\textit{flags}]%
|\input{childdoc.def}\childdocforward[|\textit{main}|]{|\textit{dest}|}"|
\end{center}
%
Here \textit{target} is the name of the output file,
\textit{main} is the name of the main file
and \textit{dest} is the name of the main or child file to be processed
(all filenames without extensions).
The optional argument \textit{main} can be omitted
if \textit{main} matches \textit{dest}.
Optionally, compilation \textit{flags} can be defined via |\def| commands.
This command line makes the \TeX{} engine believe
it is compiling the file \textit{target}
whose content is specified as the latter parameter.
The provided code then forwards the processing to
\textit{main} or \textit{dest} as described in \secref{sec:forward}.

%%%%%%%%%%%%%%%%%%%%%%%%%%%%%%%%%%%%%%%%%%%%%%%%%%%%%%%%%%%%%%%%%%%%%%%%%%%%%%%%
\subsection{Include by Input}
\label{sec:input}

Including child documents by |\include| has some restrictions by design.
Most notably, the content of a child document always occupies
its own set of pages; pages cannot be shared between child documents.
Usually, this behaviour makes perfect sense
because each child document contain an essential part of the document.
However, in some situations it may be desirable to compose
a document from a collection of parts
without having mandatory page breaks between then.
For this case, the package
provides a mechanism to include parts
by |\input| which can also be processed individually.
However, by construction this mechanism
requires manual handling of the content to be output.

%%%%%%%%%%%%%%%%%%%%%%%%%%%%%%%%%%%%%%%%
\DescribeMacro{\ifchilddocmanual}
The main file should be prepared as usual, see \secref{sec:include}.
However, the document body must make a distinction
between processing of an individual part and of the main document, e.g.:
%
\begin{center}
\begin{tabular}{l}
|\ifchilddocmanual|\\
|\input{\childdocname}|\\
|\||else|\\
\textit{document body with }|\input{|\textit{part}|}|\\
|\||fi|
\end{tabular}
\end{center}
%
The conditional |\ifchilddocmanual| is true whenever
a part to be included by |\input| is being compiled,
and the name of the part is stored in |\childdocname|.

%%%%%%%%%%%%%%%%%%%%%%%%%%%%%%%%%%%%%%%%
\DescribeMacro{\childdocby}
Each part to be included by |\input| should start with:
%
\begin{center}
\begin{tabular}{l}
|\input{childdoc.def}|\\
|\childdocby{|\textit{main}|}|\\
\end{tabular}
\end{center}
%
The directive |\childdocby| is similar to |\childdocof|
described in \secref{sec:include},
but the subsequent selection of content must be done manually.
To that end, both |\ifchilddoc| and |\ifchilddocmanual|
will be true upon processing of a part,
and the name of the part is stored in |\childdocname|.
Note that |\jobname| will be set to the filename of the current part
so that each part receives an individual |.aux| file
that does not interfere with the |.aux| file(s) of the main document.
This behaviour can be altered by the alternative form
|\childdocby[*]{|\textit{main}|}| (with a non-empty optional argument)
which uses the |.aux| file of the main document
by setting |\jobname| to \textit{main}.

%%%%%%%%%%%%%%%%%%%%%%%%%%%%%%%%%%%%%%%%%%%%%%%%%%%%%%%%%%%%%%%%%%%%%%%%%%%%%%%%
\subsection{Driver Development}
\label{sec:driver}

The \textsf{childdoc} mechanism can also be use for the development
of definition files such as \LaTeX{} styles or classes.
This case differs from the above setup with multiple parts
included by |\include| in that no |\includeonly| should be invoked.
This can be achieved by starting the include file
(before |\ProvidesPackage|) with:
%
\begin{center}
\begin{tabular}{l}
|\input{childdoc.def}|\\
|\childdocforward{|\textit{main}|}|\\
\end{tabular}
\end{center}
%
or alternatively with:
%
\begin{center}
\begin{tabular}{l}
|\input{childdoc.def}|\\
|\childdocby{|\textit{main}|}|\\
\end{tabular}
\end{center}
%
Both forms have slightly different effects as described above.
The main file is prepared as usual, see \secref{sec:include}.

%%%%%%%%%%%%%%%%%%%%%%%%%%%%%%%%%%%%%%%%%%%%%%%%%%%%%%%%%%%%%%%%%%%%%%%%%%%%%%%%
\subsection{Legacy Detection}
\label{sec:detection}

The directive |\childdocmain| in the main file can detect
whether the complete document or merely a child is to be compiled
even without using the directive |\childdocof|.
This method is deprecated because it is less robust
and there is no compelling reason to use it;
it is merely provided for backward compatibility
and it may be removed in future versions.

If the detection mechanism is to be used,
it is mandatory to correctly specify
the filename of the main file as the argument of |\childdocmain|:
%
\begin{center}
\begin{tabular}{l}
|\input{childdoc.def}|\\
|\childdocmain{|\textit{main}|}|\\
\end{tabular}
\end{center}
%
If |\jobname| does not match the argument \textit{main} of |\childdocmain|,
it is assumed that |\jobname| points to the child file to be compiled.
When using |\childdocmain| with the main file specified as argument,
it suffices to start a child file
with just |\input{|\textit{main}|}|
without loading of the package and using |\childdocof|.
If instead all processing is done
with the appropriate \textsf{childdoc} directives,
the argument of \textit{main} of |\childdocmain| can be empty.

An alternative version of the command line processing described
in \secref{sec:commandline} using the detection mechanism reads:
%
\begin{center}
|... -jobname "|\textit{target}|" "|[\textit{flags}]%
[|\def\jobname{|\textit{dest}|}|]|\input{|\textit{main}|}"|
\end{center}

%%%%%%%%%%%%%%%%%%%%%%%%%%%%%%%%%%%%%%%%%%%%%%%%%%%%%%%%%%%%%%%%%%%%%%%%%%%%%%%%
\subsection{Manual Code}
\label{sec:manual}

In case one cannot be certain whether the definitions file |childdoc.def|
is installed on the target \TeX{} distribution
and one prefers not to ship it,
it is conceivable to paste a few relevant commands into the sources.

To that end, drop all statements |\input{childdoc.def}|
and perform the replacements as outlined below.
Instead of |\childdocmain{|\textit{main}|}| add the following code
to the top of the main file:
%
\begin{center}
\begin{tabular}{l}
|\||ifdefined\childdocname\endinput\||fi\newif\ifchilddoc|\\
|\edef\childdocname{\scantokens\expandafter{\jobname\noexpand}}|\\
|\def\childdocmain{|\textit{main}|}\||ifx\childdocmain\childdocname\||else|\\
|\childdoctrue\includeonly{\childdocname}\let\jobname\childdocmain\||fi|\\
\end{tabular}
\end{center}
%
Instead of |\childdocof{|\textit{main}|}| just include the main file
at the top of each child file:
%
\begin{center}
|\input{|\textit{main}|}|
\end{center}
%
A simple redirection |\childdocforward{|\textit{dest}|}| is achieved by:
%
\begin{center}
|\def\jobname{|\textit{dest}|}\input{\jobname}|
\end{center}
%
The redirection with prefix
|\childdocforwardprefix[|\textit{prefix}|]{|\textit{dest}|}|
is accomplished by:
%
\begin{center}
\begin{tabular}{l}
|{\edef\jobname{\scantokens\expandafter{\jobname\noexpand}}|\\
|\def\redirectjob |\textit{prefix}|#1~~~{\gdef\jobname{|\textit{dest}|#1}}|\\
|\expandafter\redirectjob\jobname~~~}\input{\jobname}|
\end{tabular}
\end{center}

In an alternative approach,
child documents can be compiled by a specific command line
without additional code or specific definitions:
%
\begin{center}
|... -jobname "|\textit{target}|" "|[\textit{flags}]%
|\includeonly{|\textit{dest}|}\input{|\textit{main}|}"|
\end{center}
%

%%%%%%%%%%%%%%%%%%%%%%%%%%%%%%%%%%%%%%%%%%%%%%%%%%%%%%%%%%%%%%%%%%%%%%%%%%%%%%%%
%%%%%%%%%%%%%%%%%%%%%%%%%%%%%%%%%%%%%%%%%%%%%%%%%%%%%%%%%%%%%%%%%%%%%%%%%%%%%%%%
\section{Information}

%%%%%%%%%%%%%%%%%%%%%%%%%%%%%%%%%%%%%%%%%%%%%%%%%%%%%%%%%%%%%%%%%%%%%%%%%%%%%%%%
\subsection{Copyright}

Copyright \copyright{} 2017--2018 Niklas Beisert

This work may be distributed and/or modified under the
conditions of the \LaTeX{} Project Public License, either version 1.3
of this license or (at your option) any later version.
The latest version of this license is in
  \url{http://www.latex-project.org/lppl.txt}
and version 1.3 or later is part of all distributions of \LaTeX{}
version 2005/12/01 or later.

This work has the LPPL maintenance status `maintained'.

The Current Maintainer of this work is Niklas Beisert.

This work consists of the files |README.txt|, |childdoc.ins| and |childdoc.dtx|
as well as the derived files |childdoc.def|, |cdocsamp.tex|
with |cdocsch1.tex|, |cdocsch2.tex|, |cdocspt3.tex|, |cdocspt4.tex|,
|cdocsdrf.tex|, |cdocsfn1.tex|, |cdocsfn2.tex|
as well as |childdoc.pdf|.

%%%%%%%%%%%%%%%%%%%%%%%%%%%%%%%%%%%%%%%%%%%%%%%%%%%%%%%%%%%%%%%%%%%%%%%%%%%%%%%%
\subsection{Files and Installation}

The package consists of the files:
%
\begin{center}
\begin{tabular}{ll}
    |README.txt|   & readme file \\
    |childdoc.ins| & installation file \\
    |childdoc.dtx| & source file \\
    |childdoc.def| & definition file \\
    |cdocsamp.tex| & sample main file \\
    |cdocsch1.tex| & sample include file \\
    |cdocsch2.tex| & sample include file \\
    |cdocspt3.tex| & sample part file \\
    |cdocspt4.tex| & sample part file \\
    |cdocsdrf.tex| & sample redirection file \\
    |cdocsfn1.tex| & sample redirection file \\
    |cdocsfn2.tex| & sample redirection file \\
    |childdoc.pdf| & manual
\end{tabular}
\end{center}
%
The distribution consists of the files
|README.txt|, |childdoc.ins| and |childdoc.dtx|.
%
\begin{itemize}
\item
Run (pdf)\LaTeX{} on |childdoc.dtx|
to compile the manual |childdoc.pdf| (this file).
\item
Run \LaTeX{} on |childdoc.ins| to create the definitions file |childdoc.def|
and the sample |cdocsamp.tex| with include files
|cdocsch1.tex|, |cdocsch2.tex|, |cdocspt3.tex|, |cdocspt4.tex|,
|cdocsdrf.tex|, |cdocsfn1.tex|, |cdocsfn2.tex|.
Then copy the file |childdoc.def| to an appropriate directory of your \LaTeX{}
distribution, e.g.\ \textit{texmf-root}|/tex/latex/childdoc|.
\end{itemize}

%%%%%%%%%%%%%%%%%%%%%%%%%%%%%%%%%%%%%%%%%%%%%%%%%%%%%%%%%%%%%%%%%%%%%%%%%%%%%%%%
\subsection{Related CTAN Packages}

There are several other packages which offer a similar functionality:
%
\begin{itemize}
\item
The packages
\href{http://ctan.org/pkg/docmute}{\textsf{docmute}},
\href{http://ctan.org/pkg/includex}{\textsf{includex}} and
\href{http://ctan.org/pkg/standalone}{\textsf{standalone}}
provide commands to include only the document body of
a child file thus allowing both files to be compiled individually.
\item
The packages \href{http://ctan.org/pkg/subdocs}{\textsf{subdocs}}
and \href{http://ctan.org/pkg/subfiles}{\textsf{subfiles}}
provide structures in which the main and child documents can be
encapsulated and allowing them to be compiled individually.
The inclusion mechanism is different from the conventional |\include|.
\item
The package \href{http://ctan.org/pkg/combine}{\textsf{combine}}
is an elaborate solution to combine several documents into one.
\end{itemize}
%
See also the CTAN topic \href{http://ctan.org/topic/subdocs}{\textsf{subdocs}}
for further related packages.
The present package differs from the above solutions in that
a document structure constructed with the conventional |\include| mechanism
just needs two extra commands at the top of every file
such that all constituent files can be compiled individually.

%%%%%%%%%%%%%%%%%%%%%%%%%%%%%%%%%%%%%%%%%%%%%%%%%%%%%%%%%%%%%%%%%%%%%%%%%%%%%%%%
%\subsection{Feature Suggestions}
%
%The following is a list of features which may be useful for future
%versions of this package:
%%
%\begin{itemize}
%\item
%\ldots
%\end{itemize}

%%%%%%%%%%%%%%%%%%%%%%%%%%%%%%%%%%%%%%%%%%%%%%%%%%%%%%%%%%%%%%%%%%%%%%%%%%%%%%%%
\subsection{Revision History}

%%%%%%%%%%%%%%%%%%%%%%%%%%%%%%%%%%%%%%%%
\paragraph{v2.0:} 2018/12/30

\begin{itemize}
\item
immediate forward processing
\item
added |\childdocby| mechanism
\item
manual restructured
\end{itemize}

%%%%%%%%%%%%%%%%%%%%%%%%%%%%%%%%%%%%%%%%
\paragraph{v1.6:} 2018/01/17

\begin{itemize}
\item
application for development of include files
\item
corrections to manual
\end{itemize}

%%%%%%%%%%%%%%%%%%%%%%%%%%%%%%%%%%%%%%%%
\paragraph{v1.5:} 2017/05/21

\begin{itemize}
\item
more complete structuring introduced
\item
|\childdocof| introduced
\item
|\childdoc| renamed to |\childdocmain|
\item
|\childredirect| renamed to |\childdocforward| and |\childdocforwardprefix|
and functionality expanded
\end{itemize}

%%%%%%%%%%%%%%%%%%%%%%%%%%%%%%%%%%%%%%%%
\paragraph{v1.0:} 2017/04/27

\begin{itemize}
\item
manual and install package
\item
first version published on CTAN
\end{itemize}

%%%%%%%%%%%%%%%%%%%%%%%%%%%%%%%%%%%%%%%%
\paragraph{v0.6:} 2017/04/26

\begin{itemize}
\item
redirection mechanism added
\end{itemize}

%%%%%%%%%%%%%%%%%%%%%%%%%%%%%%%%%%%%%%%%
\paragraph{v0.5:} 2017/04/26

\begin{itemize}
\item
functionality in definition file
\end{itemize}


%%%%%%%%%%%%%%%%%%%%%%%%%%%%%%%%%%%%%%%%%%%%%%%%%%%%%%%%%%%%%%%%%%%%%%%%%%%%%%%%
%%%%%%%%%%%%%%%%%%%%%%%%%%%%%%%%%%%%%%%%%%%%%%%%%%%%%%%%%%%%%%%%%%%%%%%%%%%%%%%%
%%%%%%%%%%%%%%%%%%%%%%%%%%%%%%%%%%%%%%%%%%%%%%%%%%%%%%%%%%%%%%%%%%%%%%%%%%%%%%%%
\appendix

\settowidth\MacroIndent{\rmfamily\scriptsize 000\ }

 \DocInput{childdoc.dtx}

\end{document}
%</driver>
% \fi
%
% %%%%%%%%%%%%%%%%%%%%%%%%%%%%%%%%%%%%%%%%%%%%%%%%%%%%%%%%%%%%%%%%%%%%%%%%%%%%%%
% %%%%%%%%%%%%%%%%%%%%%%%%%%%%%%%%%%%%%%%%%%%%%%%%%%%%%%%%%%%%%%%%%%%%%%%%%%%%%%
% \section{Sample}
%\iffalse
%<*samplemain>
%\fi
%
% The following presents a sample document
% with two chapters, two parts, a title page,
% a compile flag as well as three forwarding files to set the flag.
% It consists of eight |.tex| files:
% \begin{center}
% \begin{tabular}{ll}
% |cdocsamp.tex|&main file\\
% |cdocsch1.tex|&include file for chapter 1\\
% |cdocsch2.tex|&include file for chapter 2\\
% |cdocspt3.tex|&include file for part 3\\
% |cdocspt4.tex|&include file for part 4\\
% |cdocsdrf.tex|&forwarding file for main file in draft mode\\
% |cdocsfi1.tex|&forwarding file for final version of chapter 1\\
% |cdocsfi2.tex|&forwarding file for final version of chapter 2\\
% \end{tabular}
% \end{center}
% Each of the eight files can be compiled directly by the \LaTeX{} compiler.
%
% %%%%%%%%%%%%%%%%%%%%%%%%%%%%%%%%%%%%%%
% \paragraph{Main File.}
%
% The main file is called |cdocsamp.tex|.
%
% Load the \textsf{childdoc} definitions and
% declare the filename for the main document:
%    \begin{macrocode}
\input{childdoc.def}
\childdocmain{}
%    \end{macrocode}

% Optional override for |\version| flag:
%    \begin{macrocode}
%%\ifchilddoc\else\providecommand{\version}{draft}\fi
%    \end{macrocode}

% Define the default values for the |\version| flag
% (|final| for the main file and |draft| for childs):
%    \begin{macrocode}
\ifchilddoc
\providecommand{\version}{draft}
\else
\providecommand{\version}{final}
\fi
%    \end{macrocode}

% Load the standard document class:
%    \begin{macrocode}
\documentclass[12pt]{article}
%    \end{macrocode}

% Start the document body:
%    \begin{macrocode}
\begin{document}
%    \end{macrocode}

% Declare a title page.
% Print title, part of document being processed and version flag:
%    \begin{macrocode}
\addtocounter{page}{-1}
\begin{center}
{\LARGE\bfseries{}childdoc example\par}
\vspace{1cm}
\ifchilddoc
\ifchilddocmanual part\else chapter\fi:
`\childdocname' of `\childdocjob'\par
\else
main document: `\childdocjob'\par
\fi
version: \version\par
\end{center}
\newpage
%    \end{macrocode}

% Manually include selected file,
% otherwise process as usual:
%    \begin{macrocode}
\ifchilddocmanual
\section*{part `\childdocname'}
\input{\childdocname}
\else
%    \end{macrocode}

% Include the two chapters:
%    \begin{macrocode}
\include{cdocsch1}
\include{cdocsch2}
%    \end{macrocode}

% Include the two parts unless only chapters should be displayed:
%    \begin{macrocode}
\ifchilddoc\else
\section{part three}
\input{cdocspt3}
\section{part four}
\input{cdocspt4}
\fi
%    \end{macrocode}

% Process as usual until here:
%    \begin{macrocode}
\fi
%    \end{macrocode}

% End of document body:
%    \begin{macrocode}
\end{document}
%    \end{macrocode}
%\iffalse
%</samplemain>
%\fi
%
% %%%%%%%%%%%%%%%%%%%%%%%%%%%%%%%%%%%%%%
% \paragraph{Chapter Include Files.}
%
% The include files are called |cdocsch1.tex| and |cdocsch2.tex|.
%
%\iffalse
%<*samplechap1|samplechap2>
%\fi

% Optional override for |\version| flag:
%    \begin{macrocode}
%%\providecommand{\version}{final}
%    \end{macrocode}

% Include the main document:
%    \begin{macrocode}
\input{childdoc.def}
\childdocof{cdocsamp}
%    \end{macrocode}

%\iffalse
%</samplechap1|samplechap2>
%\fi
%
%\iffalse
%<*samplechap1>
%\fi
% Some text for chapter 1:
%    \begin{macrocode}
\section{one}
some text in chapter one
%    \end{macrocode}

%\iffalse
%</samplechap1>
%\fi
% Some text for chapter 2:
%\iffalse
%<*samplechap2>
%\fi
%    \begin{macrocode}
\section{two}
more text in chapter two
%    \end{macrocode}

%\iffalse
%</samplechap2>
%\fi
%
% %%%%%%%%%%%%%%%%%%%%%%%%%%%%%%%%%%%%%%
% \paragraph{Part Include Files.}
%
% The include files are called |cdocspt3.tex| and |cdocspt4.tex|.
%
%\iffalse
%<*samplepart3|samplepart4>
%\fi

% Optional override for |\version| flag:
%    \begin{macrocode}
%%\providecommand{\version}{final}
%    \end{macrocode}

% Include the main document:
%    \begin{macrocode}
\input{childdoc.def}
\childdocby{cdocsamp}
%    \end{macrocode}

%\iffalse
%</samplepart3|samplepart4>
%\fi
%
%\iffalse
%<*samplepart3>
%\fi
% Some text for part 3:
%    \begin{macrocode}
some text in part three
%    \end{macrocode}

%\iffalse
%</samplepart3>
%\fi
% Some text for part 4:
%\iffalse
%<*samplepart4>
%\fi
%    \begin{macrocode}
more text in part four
%    \end{macrocode}

%\iffalse
%</samplepart4>
%\fi
%
% %%%%%%%%%%%%%%%%%%%%%%%%%%%%%%%%%%%%%%
% \paragraph{Forwarding for a Complete Draft.}
%
% The following forwarding file |cdocsdrf.tex|
% compiles the main document in draft mode:
%\iffalse
%<*sampledraft>
%\fi
%    \begin{macrocode}
\def\version{draft}
\input{childdoc.def}
\childdocforward{cdocsamp}
%    \end{macrocode}

%\iffalse
%</sampledraft>
%\fi
%
% %%%%%%%%%%%%%%%%%%%%%%%%%%%%%%%%%%%%%%
% \paragraph{Forwarding for Final Version of the Chapters.}
%
% The following forwarding files |cdocsfn1.tex| and |cdocsfn2.tex|
% (with identical content)
% compile the final versions of the child documents
% |cdocsch1.tex| and |cdocsch2.tex|, respectively:
%\iffalse
%<*samplefinal>
%\fi
%    \begin{macrocode}
\def\version{final}
\input{childdoc.def}
\childdocforwardprefix[cdocsamp]{cdocsfn}{cdocsch}
%    \end{macrocode}

%\iffalse
%</samplefinal>
%\fi
%
% %%%%%%%%%%%%%%%%%%%%%%%%%%%%%%%%%%%%%%
% \paragraph{Command Line Processing.}
%
% The following three command lines generate the output files
% |cdocscld|, |cdocscl1| and |cdocscl2|
% which should be identical to
% |cdocsdrf|, |cdocsch1| and |cdocsfn2|, respectively:
% \begin{center}
% \begin{tabular}{l}
% |latex -jobname cdocscld \|\\
% |  "\def\version{draft}\input{childdoc.def}\childdocforward{cdocsamp}"|\\
% |latex -jobname cdocscl1 \|\\
% |  "\input{childdoc.def}\childdocforward[cdocsamp]{cdocsch1}"|\\
% |latex -jobname cdocscl2 \|\\
% |  "\def\version{final}\input{childdoc.def}\childdocforward{cdocsch2}"|
% \end{tabular}
% \end{center}
% Note that the trailing backslash on each first line
% merely continues the input to the second line
% (for convenient cut ant paste).
% Furthermore, the command |latex| can be replaced by any
% of its alternative versions such as |pdflatex|.
%
% %%%%%%%%%%%%%%%%%%%%%%%%%%%%%%%%%%%%%%%%%%%%%%%%%%%%%%%%%%%%%%%%%%%%%%%%%%%%%%
% %%%%%%%%%%%%%%%%%%%%%%%%%%%%%%%%%%%%%%%%%%%%%%%%%%%%%%%%%%%%%%%%%%%%%%%%%%%%%%
% \section{Implementation}
%\iffalse
%<*package>
%\fi
%
% This section describes the definitions file |childdoc.def|.

% The definitions cannot be loaded using |\usepackage| or |\RequirePackage|
% which has a mechanism to prevent loading a style file more than once.
% When loading the definitions by means of |\input|
% multiple instances have to be prevented manually:
%\iffalse
%This code needs to be before the `\ProvidesFile' directive
%which is defined at the beginning of this file.
%Therefore it is also placed there and commented out here.
%</package>
%<*discard>
%\fi
%    \begin{macrocode}
\ifdefined\childdocmain\endinput\fi
%    \end{macrocode}
%\iffalse
%</discard>
%<*package>
%\fi
%
% \macro{\ifchilddoc}
% \macro{\ifchilddocmanual}
% The conditional |\ifchilddoc| tells whether a
% child (true) or main (false) document is being compiled.
% The conditional |\ifchilddocmanual| tells whether
% the |\includeonly| mechanism is used (false) or
% the selection of child files must be performed manually (true).
% The definitions initialise to false:
%    \begin{macrocode}
\newif\ifchilddoc
\newif\ifchilddocmanual
%    \end{macrocode}

% \macro{\childdocname}
% \macro{\childdocjob}
% The macro |\childdocname| stores the name of the main document
% to be compiled. The macro |\childdocjob| stores the name of
% the document on which the \LaTeX{} compiler was originally invoked.
% The content of |\jobname| cannot be compared
% to filenames specified in the source due to different catcodes.
% The following code rescans |\jobname|, stores the result
% in |\childdocname| and saves a copy in |\childdocjob|:
%    \begin{macrocode}
\edef\childdocname{\scantokens\expandafter{\jobname\noexpand}}
\let\childdocjob\childdocname
%    \end{macrocode}

% \macro{\childdocdisable}
% The macro |\childdocdisable| prevents the main file
% from being processed more than once.
% At this stage, the main document command |\childdocmain|
% is assumed to be called once again where it should do nothing.
% Any subsequent call to it should prevent
% a secondary processing of the main document
% It overwrites the forwarding commands
% |\childdocof| and |\childdocforward|
% with empty macros to prevent further inclusions of the main document:
%    \begin{macrocode}
\newcommand{\childdocdisable}
{
  \renewcommand{\childdocmain}[1]{\renewcommand{\childdocmain}[1]{\endinput}}
  \renewcommand{\childdocof}[1]{}
  \renewcommand{\childdocby}[2][]{}
  \renewcommand{\childdocforward}[2][]{}
  \renewcommand{\childdocdisable}{}
}
%    \end{macrocode}

% \macro{\childdocmain}
% The macro |\childdocmain| is to be called at the top of the main file
% with nothing or the main filename (without extension) as argument.
% First, it breaks loops.
% If the argument is not empty and does not match |\childdocname|
% (which is set by the first inclusion of |childdoc.def|),
% |\ifchilddoc| is set to true, |\includeonly| is applied to the child file
% and |\jobname| is set to the main file
% (for proper handling of |.aux| files):
%    \begin{macrocode}
\newcommand{\childdocmain}[1]
{
  \childdocdisable\childdocmain{}
  \if?#1?\else
    \begingroup
      \def\childdoctmp{#1}
      \ifx\childdoctmp\childdocname
        \def\childdoctmp{}
      \else
        \def\childdoctmp
        {
          \childdoctrue
          \includeonly{\childdocname}
          \def\childdocjob{#1}
          \def\jobname{#1}
        }
      \fi
      \expandafter
    \endgroup
    \childdoctmp
  \fi
}
%    \end{macrocode}

% \macro{\childdocof}
% The command |\childdocof| redirects
% compilation to the main file |#1|.
%    \begin{macrocode}
\newcommand{\childdocof}[1]
{
  \childdocdisable
  \childdoctrue
  \includeonly{\childdocname}
  \def\jobname{#1}
  \def\childdocjob{#1}
  \input{#1}
}
%    \end{macrocode}

% \macro{\childdocby}
% The command |\childdocby| ....
%    \begin{macrocode}
\newcommand{\childdocby}[2][]
{
  \childdocdisable
  \childdoctrue
  \childdocmanualtrue
  \if?#1?\else
    \def\jobname{#2}
  \fi
  \def\childdocjob{#2}
  \input{#2}
  \endinput
}
%    \end{macrocode}

% \macro{\childdocforward}
% The command |\childdocforward| redirects
% compilation to the main file or
% (if the optional argument is given) a child file.
% Parameters are set as if the main file
% or a child file starting with |\childdocof| was compiled.
% Then compilation is handed over to the main file:
%    \begin{macrocode}
\newcommand{\childdocforward}[2][]
{
  \begingroup
    \if?#1?
      \def\childdoctmp
      {
        \def\childdocname{#2}
        \def\childdocjob{#2}
        \def\jobname{#2}
        \input{#2}
        \endinput
      }
    \else
      \def\childdoctmp
      {
        \childdocdisable
        \def\childdocname{#2}
        \childdoctrue
        \includeonly{#2}
        \def\childdocjob{#1}
        \def\jobname{#1}
        \input{#1}
        \endinput
      }
    \fi
    \expandafter
  \endgroup
  \childdoctmp
}
%    \end{macrocode}

% \macro{\childdocforwardprefix}
% The command |\childdocforwardprefix| redirects
% compilation to the main or a child file by means of a pattern.
% The prefix |#1| in the current filename is replaced by |#2|
% and the suffix of the current filename is kept
% (it is assumed that the filename does not contain the substring `|~~~|'
% which is used as a delimiter).
% Compilation is handed over to the new file by |\childdocforward|:
%    \begin{macrocode}
\newcommand{\childdocforwardprefix}[3][]
{
  \begingroup
    \def\childdocextract #2##1~~~{\def\childdoctmp{\childdocforward[#1]{#3##1}}}
    \expandafter\childdocextract\childdocname~~~
    \expandafter
  \endgroup
  \childdoctmp
}
%    \end{macrocode}

% \macro{\childdoc}
% The deprecated macro |\childdoc| is a legacy version of |\childdocmain|:
%    \begin{macrocode}
\newcommand{\childdoc}{\childdocmain}
%    \end{macrocode}

% \macro{\childdocredirect}
% The deprecated macro |\childdocredirect| is a legacy version
% of |\childdocforward| and |\childdocforwardprefix|:
%    \begin{macrocode}
\newcommand{\childdocredirect}[2][]
{
  \begingroup
    \if?#1?
      \def\childdoctmp{\childdocforward{#2}}
    \else
      \def\childdoctmp{\childdocforwardprefix{#1}{#2}}
    \fi
    \expandafter
  \endgroup
  \childdoctmp
}
%    \end{macrocode}

%\iffalse
%</package>
%\fi
%
\endinput
|\\
|\childdocby{|\textit{main}|}|\\
\end{tabular}
\end{center}
%
The directive |\childdocby| is similar to |\childdocof|
described in \secref{sec:include},
but the subsequent selection of content must be done manually.
To that end, both |\ifchilddoc| and |\ifchilddocmanual|
will be true upon processing of a part,
and the name of the part is stored in |\childdocname|.
Note that |\jobname| will be set to the filename of the current part
so that each part receives an individual |.aux| file
that does not interfere with the |.aux| file(s) of the main document.
This behaviour can be altered by the alternative form
|\childdocby[*]{|\textit{main}|}| (with a non-empty optional argument)
which uses the |.aux| file of the main document
by setting |\jobname| to \textit{main}.

%%%%%%%%%%%%%%%%%%%%%%%%%%%%%%%%%%%%%%%%%%%%%%%%%%%%%%%%%%%%%%%%%%%%%%%%%%%%%%%%
\subsection{Driver Development}
\label{sec:driver}

The \textsf{childdoc} mechanism can also be use for the development
of definition files such as \LaTeX{} styles or classes.
This case differs from the above setup with multiple parts
included by |\include| in that no |\includeonly| should be invoked.
This can be achieved by starting the include file
(before |\ProvidesPackage|) with:
%
\begin{center}
\begin{tabular}{l}
|% \iffalse
%
% childdoc.dtx Copyright (C) 2017-2018 Niklas Beisert
%
% This work may be distributed and/or modified under the
% conditions of the LaTeX Project Public License, either version 1.3
% of this license or (at your option) any later version.
% The latest version of this license is in
%   http://www.latex-project.org/lppl.txt
% and version 1.3 or later is part of all distributions of LaTeX
% version 2005/12/01 or later.
%
% This work has the LPPL maintenance status `maintained'.
%
% The Current Maintainer of this work is Niklas Beisert.
%
% This work consists of the files childdoc.dtx and childdoc.ins
% and the derived files childdoc.def and cdocsamp.tex with
% cdocsch1.tex, cdocsch2.tex, cdocsdrf.tex, cdocsfn1.tex, cdocsfn2.tex.
%
%<package>\ifdefined\childdocmain\endinput\fi
%<package>\ProvidesFile{childdoc.def}[2018/12/30 v2.0 child document driver]
%<samplemain>\ProvidesFile{cdocsamp.tex}[2018/12/30 v2.0 sample for childdoc]
%<*driver>
%\ProvidesFile{childdoc.drv}[2018/12/30 v2.0 childdoc reference manual file]
\PassOptionsToClass{10pt,a4paper}{article}
\documentclass{ltxdoc}

\usepackage[margin=35mm]{geometry}
\usepackage{hyperref}
\usepackage{hyperxmp}
\usepackage[usenames]{color}

\hypersetup{colorlinks=true}
\hypersetup{pdfstartview=FitH}
\hypersetup{pdfpagemode=UseNone}
\hypersetup{pdfsource={}}
\hypersetup{pdflang={en-UK}}
\hypersetup{pdfcopyright={Copyright 2017-2018 Niklas Beisert.
  This work may be distributed and/or modified under the
  conditions of the LaTeX Project Public License, either version 1.3
  of this license or (at your option) any later version.}}
\hypersetup{pdflicenseurl={http://www.latex-project.org/lppl.txt}}
\hypersetup{pdfcontactaddress={ETH Zurich, ITP, HIT K,
  Wolfgang-Pauli-Strasse 27}}
\hypersetup{pdfcontactpostcode={8093}}
\hypersetup{pdfcontactcity={Zurich}}
\hypersetup{pdfcontactcountry={Switzerland}}
\hypersetup{pdfcontactemail={nbeisert@itp.phys.ethz.ch}}
\hypersetup{pdfcontacturl={http://people.phys.ethz.ch/\xmptilde nbeisert/}}

\newcommand{\secref}[1]{\hyperref[#1]{section \ref*{#1}}}

\parskip1ex
\parindent0pt
\let\olditemize\itemize
\def\itemize{\olditemize\parskip0pt}

\begin{document}

\title{The \textsf{childdoc} Package}
\hypersetup{pdftitle={The childdoc Package}}
\author{Niklas Beisert\\[2ex]
  Institut f\"ur Theoretische Physik\\
  Eidgen\"ossische Technische Hochschule Z\"urich\\
  Wolfgang-Pauli-Strasse 27, 8093 Z\"urich, Switzerland\\[1ex]
  \href{mailto:nbeisert@itp.phys.ethz.ch}
  {\texttt{nbeisert@itp.phys.ethz.ch}}}
\hypersetup{pdfauthor={Niklas Beisert}}
\hypersetup{pdfsubject={Manual for the LaTeX2e Package childdoc}}
\date{30 December 2018, \textsf{v2.0}}
\maketitle

\begin{abstract}\noindent
\textsf{childdoc} is a \LaTeXe{} package
that enables the direct compilation
of document sections included by |\include|
to individual files.
\end{abstract}

\begingroup
\parskip0ex
\tableofcontents
\endgroup

%%%%%%%%%%%%%%%%%%%%%%%%%%%%%%%%%%%%%%%%%%%%%%%%%%%%%%%%%%%%%%%%%%%%%%%%%%%%%%%%
%%%%%%%%%%%%%%%%%%%%%%%%%%%%%%%%%%%%%%%%%%%%%%%%%%%%%%%%%%%%%%%%%%%%%%%%%%%%%%%%
\section{Introduction}

\LaTeX{} provides a mechanism to structure a large document (such as a book)
into a main file and several child files (containing the chapters)
using the |\include| command.
This mechanism is beneficial for documents
which span hundreds of pages in order to
make the source file(s) more manageable.
Moreover, compilation can be restricted to
selected child files by means of the |\includeonly| command.
The latter feature can be used to reduce the compilation time while editing
(this was significantly more useful in the earlier days of \LaTeX{})
or to generate a smaller document which is easier to navigate.
Another application of |\includeonly| is to generate
documents consisting of selected parts of the complete document.

However, there are a few drawbacks of the plain |\include| mechanism:
\begin{itemize}
\item
The child files cannot be compiled on their own,
they can only be compiled via the main file.
A naive editing environment
(such as a text editor with an option
to have the current file processed by \LaTeX)
may require one to switch to the main file before compiling;
attempting to compile the child file produces errors.
\item
The main file must be modified (each time)
to adjust the |\includeonly| command
to the present needs. This easily leaves the main file in a messy state.
\item
The generated document will always carry the filename
of the main document. This is inconvenient if
several child files are to be compiled and
to be kept for distribution.
\end{itemize}

The present package provides a simple interface
to make child files individually compilable by \LaTeX{}.
Compiling a child file then has the same effect as compiling
the main file with an |\includeonly| command
to select the appropriate child.
Moreover the generated document will carry the name of the child
rather than the main file.
This resolves all three above issues.

This feature is meant to make the editing of books,
thesis documents and lecture notes somewhat more convenient.
However, the package can also be used efficiently for
composing a series of documents (such as exercise sheets)
which are typically distributed individually.
It then assists the author in generating the individual documents
(potentially in different versions)
as well as a document containing the collected series.
Another application is in developing style files
or other kinds of included material
where compilation of the style file could redirect
to a sample or test file.

%%%%%%%%%%%%%%%%%%%%%%%%%%%%%%%%%%%%%%%%%%%%%%%%%%%%%%%%%%%%%%%%%%%%%%%%%%%%%%%%
%%%%%%%%%%%%%%%%%%%%%%%%%%%%%%%%%%%%%%%%%%%%%%%%%%%%%%%%%%%%%%%%%%%%%%%%%%%%%%%%
\section{Usage}

First of all, the package \textsf{childdoc} is \emph{not} a standard
\LaTeXe{} |.sty| style file! Therefore it needs to be invoked in
a non-standard way.

%%%%%%%%%%%%%%%%%%%%%%%%%%%%%%%%%%%%%%%%%%%%%%%%%%%%%%%%%%%%%%%%%%%%%%%%%%%%%%%%
\subsection{Included Files}
\label{sec:include}

%%%%%%%%%%%%%%%%%%%%%%%%%%%%%%%%%%%%%%%%
\DescribeMacro{\childdocmain}
To use the package, add the commands
\begin{center}
\begin{tabular}{l}
|\input{childdoc.def}|\\
|\childdocmain{}|\\
\end{tabular}
\end{center}
at the very top of the main \LaTeX{} file,
in particular \emph{before} the |\documentclass| statement!
The argument of |\childdocmain| should be left empty
(but it must be present).

%%%%%%%%%%%%%%%%%%%%%%%%%%%%%%%%%%%%%%%%
\DescribeMacro{\childdocof}
Furthermore, add the commands
\begin{center}
\begin{tabular}{l}
|\input{childdoc.def}|\\
|\childdocof{|\textit{main}|}|\\
\end{tabular}
\end{center}
at the top of every child file \textit{child}
which is included by |\include{|\textit{child}|}|
from within the main file
(or at least for those files to be compiled individually).
The argument \textit{main} must be the filename of the main file.

There are a couple of
considerations in setting up the main and child documents:

%%%%%%%%%%%%%%%%%%%%%%%%%%%%%%%%%%%%%%%%
\paragraph{Restrictions.}

Please note the following restrictions:
\begin{itemize}
\item
|\childdocmain| must be called with one argument \textit{main}
to ensure compatibility with earlier version of the package.
It must either be empty (|\childdocmain{}|)
or precisely match the filename of the main file in which it is specified.
See \secref{sec:detection} for further information.
\item
The filename \textit{main} must be specified without the |.tex| extension.
\item
The filename \textit{main} is case sensitive
(even in case-insensitive file systems)
due to internal string comparison.
\item
The argument \textit{main} should be fully expanded, it cannot be a macro.
\item
Subdirectories and special characters should be avoided in filenames.
\item
The command |\childdocmain{|\textit{main}|}| must be followed by a whitespace.
It should not be followed immediately by another command
or by a comment mark `|%|'.
This is because the \TeX{} parser reads the token immediately following
the argument of |\childdocmain| and puts it
at the beginning of every child section;
however, a white\-space is ignored.
\end{itemize}

%%%%%%%%%%%%%%%%%%%%%%%%%%%%%%%%%%%%%%%%
\paragraph{Content of Main File.}

It is advisable to place all content in the child files included by |\include|.
Any output contained in the main file will appear in all child documents
unless suppressed manually;
it cannot be suppressed automatically by the |\includeonly| directive
and thus should normally be avoided.
A method to include some content in the main file
by means of conditional processing is described in \secref{sec:conditional}.

%%%%%%%%%%%%%%%%%%%%%%%%%%%%%%%%%%%%%%%%
\paragraph{Page Numbering.}

When only a part of the document is compiled,
the appropriate numbering of pages
(as well as other status parameters)
is determined from the |.aux| files.
The latter contain information from previous passes.
However this information needs to propagate through
all intermediate child documents.
Therefore the page numbering in child documents may well
be inconsistent until the complete document is compiled at least once.

A useful (if unconventional) way to always ensure a consistent
page numbering is to restart the numbering in each child document
and denote the pages by `\textit{child}|.|\textit{page}'
where \textit{child} represents the chapter/section number of the child file.
This can be achieved by the command
|\numberwithin{page}{|\textit{child}|}|
of the \textsf{amsmath} package
where \textit{child} can be |chapter| or |section|
depending on the chosen structuring.
Alternatively, one can modify the macro |\thepage| appropriately
and reset the counter |page| at the start of each child file.

%%%%%%%%%%%%%%%%%%%%%%%%%%%%%%%%%%%%%%%%%%%%%%%%%%%%%%%%%%%%%%%%%%%%%%%%%%%%%%%%
\subsection{Conditional Processing}
\label{sec:conditional}

The package provides a mechanism to compile different versions
of a document. To customise the versions further some conditional processing
can come in handy to distinguish which version is being compiled.
The package provides two macros to describe the compilation context:

%%%%%%%%%%%%%%%%%%%%%%%%%%%%%%%%%%%%%%%%
\DescribeMacro{\ifchilddoc}
The conditional |\ifchilddoc| distinguishes between the compilation of
child documents and the main document:
%
\begin{center}
|\ifchilddoc |\textit{child-code}| |[|\||else |\textit{main-code}]| \||fi|
\end{center}

%%%%%%%%%%%%%%%%%%%%%%%%%%%%%%%%%%%%%%%%
\DescribeMacro{\childdocname}
\DescribeMacro{\childdocjob}
The macro |\childdocname| contains the filename (without extension)
of the main or child file being processed.
Note that |\childdocjob| will always contain the name of the main file.

%%%%%%%%%%%%%%%%%%%%%%%%%%%%%%%%%%%%%%%%
\paragraph{Title Page.}

Conditional processing can be used to include a title or banner page
in the main document when proper precautions are taken.
Importantly, the code in the main file should ensure that the page counter
(as well as other status parameters which are stored in the |.aux| files)
takes the same value after the conditional processing.
Otherwise the page numbers may take divergent values
depending on which part is compiled.

For example, a title page could be declared by:
%
\begin{center}
\begin{tabular}{l}
|\ifchilddoc\||else|\\
|\addtocounter{page}{-1}|\\
\textit{code for title page}\\
|\newpage|\\
|\||fi|
\end{tabular}
\end{center}
%
A banner page for the child documents can be generated by:
%
\begin{center}
\begin{tabular}{l}
|\ifchilddoc|\\
|\addtocounter{page}{-1}|\\
\textit{code for banner page}\\
|\newpage|\\
|\||fi|
\end{tabular}
\end{center}
%
Here one could write a message such as:
\begin{center}
|This is the part \childdocname{} of \childdocjob{}.|
\end{center}

%%%%%%%%%%%%%%%%%%%%%%%%%%%%%%%%%%%%%%%%%%%%%%%%%%%%%%%%%%%%%%%%%%%%%%%%%%%%%%%%
\subsection{Flags}
\label{sec:flags}

The package makes it easy to generate different versions
of the main or child documents.
To this end compilation flags can be defined
and assigned different default values.
They will be particularly useful in conjunction
with the forwarding mechanism described in \secref{sec:forward}.

For example, it may be useful to have a flag |\version|
which can be set to |draft| or |final|.
The document source will contain some conditional code
depending on the value of |\version|.
Suppose further, the flag should default to |final| for the main file
and to |draft| for child files
which is a natural assignment for editing the document.
This is achieved by placing the following code
in the preamble of the main document
(below the |\childdocmain| directive):
%
\begin{center}
\begin{tabular}{l}
|\ifchilddoc|\\
|\providecommand{\version}{draft}|\\
|\||else|\\
|\providecommand{\version}{final}|\\
|\||fi|
\end{tabular}
\end{center}
%
The definition by |\providecommand| makes sure
that previous definitions are not overwritten.
Further statements |\providecommand{\version}{...}|
can thus be added before the above code to override it.

For the main file, one might add a line
(between |\childdocmain| and the above block)
%
\begin{center}
|%\ifchilddoc\||else\providecommand{\version}{draft}\||fi|
\end{center}
%
which can be uncommented to produce a draft version.
Likewise one can add a line to the very top of a child file
(above the |\childdocof{|\textit{main}|}| directive)
%
\begin{center}
|%\providecommand{\version}{final}|
\end{center}
%
which can be uncommented to produce the final version of this child document.

%%%%%%%%%%%%%%%%%%%%%%%%%%%%%%%%%%%%%%%%%%%%%%%%%%%%%%%%%%%%%%%%%%%%%%%%%%%%%%%%
\subsection{Forwarding}
\label{sec:forward}

Different versions of the main or child documents
using compilation flags as described in \secref{sec:flags}
can be (permanently) stored in different files
for convenient compilation, viewing and distribution.
To this end, the package defines a command
to pass on compilation to a different file:

%%%%%%%%%%%%%%%%%%%%%%%%%%%%%%%%%%%%%%%%
\DescribeMacro{\childdocforward}
The command |\childdocforward| redirects processing to
another source file:
%
\begin{center}
\begin{tabular}{l}
|\input{childdoc.def}|\\
|\childdocforward[|\textit{main}|]{|\textit{dest}|}|\\
\end{tabular}
\end{center}
%
The argument \textit{dest} is the destination file
(without extension).
It should be the main file or one of the child files.
Note that further \textsf{childdoc} directives
such as |\childdocof| and |\childdocforward|
in the indicated file will be processed in this form.
The optional argument \textit{main}
passes on directly to the main file \textit{main}
while pretending to compile the child \textit{dest}.
This form behaves as if \textit{dest}
issues |\childdocof{|\textit{main}|}| right away,
and no further \textsf{childdoc} directives will be processed.

%%%%%%%%%%%%%%%%%%%%%%%%%%%%%%%%%%%%%%%%
\DescribeMacro{\...prefix}
In the alternative form |\childdocforwardprefix|,
%
\begin{center}
\begin{tabular}{l}
|\input{childdoc.def}|\\
|\childdocforwardprefix[|\textit{main}|]{|\textit{prefix}|}{|\textit{dest}|}|
\end{tabular}
\end{center}
%
the destination file is determined by a pattern
depending on the current file:
To make this work, the current file must be called
`{\textit{prefix}\hspace{0.2em}\textit{suffix}}'
with \textit{prefix} matching precisely the argument.
Processing is then passed on to the file
`{\textit{dest}\hspace{0.2em}\textit{suffix}}'.
Surely, the same effect is achieved by
directly specifying the
argument `{\textit{dest}\hspace{0.2em}\textit{suffix}}'
in the first form.
However, that requires to set up a different file
for each child. With the alternative form of the command
all these files can have exactly the same content
which simplifies setting them up and maintaining them.

For example, the following file |draft.tex|
with a compilation flag |\version| as described in \secref{sec:flags}
compiles the main document as a draft:
%
\begin{center}
\begin{tabular}{l}
|\def\version{draft}|\\
|\input{childdoc.def}|\\
|\childdocforward{|\textit{main}|}|
\end{tabular}
\end{center}
%
Likewise, the following files |final|\textit{nn}|.tex|
compile the final version of the child document
|child|\textit{nn}|.tex|:
%
\begin{center}
\begin{tabular}{l}
|\def\version{final}|\\
|\input{childdoc.def}|\\
|\childdocforwardprefix{final}{child}|
\end{tabular}
\end{center}
%

Note that when several versions of a main file and/or of each child file
are to be generated, it may be convenient to set up a |Makefile| or
shell script to automatise the process.

%%%%%%%%%%%%%%%%%%%%%%%%%%%%%%%%%%%%%%%%%%%%%%%%%%%%%%%%%%%%%%%%%%%%%%%%%%%%%%%%
\subsection{Command Line Processing}
\label{sec:commandline}

The effect of redirection files can also be achieved by invoking
the \LaTeX{} compiler with a more elaborate command line.
Most conveniently this should be done as part
of a shell script or a |Makefile|.

When using \textsf{childdoc} in the main file, the following
command lines effectively perform a redirection
(note that depending on the shell being used,
backslashes may have to be doubled: `|\|' $\to$ `|\\|'):
%
\begin{center}
|... -jobname "|\textit{target}|" |\\|"|[\textit{flags}]%
|\input{childdoc.def}\childdocforward[|\textit{main}|]{|\textit{dest}|}"|
\end{center}
%
Here \textit{target} is the name of the output file,
\textit{main} is the name of the main file
and \textit{dest} is the name of the main or child file to be processed
(all filenames without extensions).
The optional argument \textit{main} can be omitted
if \textit{main} matches \textit{dest}.
Optionally, compilation \textit{flags} can be defined via |\def| commands.
This command line makes the \TeX{} engine believe
it is compiling the file \textit{target}
whose content is specified as the latter parameter.
The provided code then forwards the processing to
\textit{main} or \textit{dest} as described in \secref{sec:forward}.

%%%%%%%%%%%%%%%%%%%%%%%%%%%%%%%%%%%%%%%%%%%%%%%%%%%%%%%%%%%%%%%%%%%%%%%%%%%%%%%%
\subsection{Include by Input}
\label{sec:input}

Including child documents by |\include| has some restrictions by design.
Most notably, the content of a child document always occupies
its own set of pages; pages cannot be shared between child documents.
Usually, this behaviour makes perfect sense
because each child document contain an essential part of the document.
However, in some situations it may be desirable to compose
a document from a collection of parts
without having mandatory page breaks between then.
For this case, the package
provides a mechanism to include parts
by |\input| which can also be processed individually.
However, by construction this mechanism
requires manual handling of the content to be output.

%%%%%%%%%%%%%%%%%%%%%%%%%%%%%%%%%%%%%%%%
\DescribeMacro{\ifchilddocmanual}
The main file should be prepared as usual, see \secref{sec:include}.
However, the document body must make a distinction
between processing of an individual part and of the main document, e.g.:
%
\begin{center}
\begin{tabular}{l}
|\ifchilddocmanual|\\
|\input{\childdocname}|\\
|\||else|\\
\textit{document body with }|\input{|\textit{part}|}|\\
|\||fi|
\end{tabular}
\end{center}
%
The conditional |\ifchilddocmanual| is true whenever
a part to be included by |\input| is being compiled,
and the name of the part is stored in |\childdocname|.

%%%%%%%%%%%%%%%%%%%%%%%%%%%%%%%%%%%%%%%%
\DescribeMacro{\childdocby}
Each part to be included by |\input| should start with:
%
\begin{center}
\begin{tabular}{l}
|\input{childdoc.def}|\\
|\childdocby{|\textit{main}|}|\\
\end{tabular}
\end{center}
%
The directive |\childdocby| is similar to |\childdocof|
described in \secref{sec:include},
but the subsequent selection of content must be done manually.
To that end, both |\ifchilddoc| and |\ifchilddocmanual|
will be true upon processing of a part,
and the name of the part is stored in |\childdocname|.
Note that |\jobname| will be set to the filename of the current part
so that each part receives an individual |.aux| file
that does not interfere with the |.aux| file(s) of the main document.
This behaviour can be altered by the alternative form
|\childdocby[*]{|\textit{main}|}| (with a non-empty optional argument)
which uses the |.aux| file of the main document
by setting |\jobname| to \textit{main}.

%%%%%%%%%%%%%%%%%%%%%%%%%%%%%%%%%%%%%%%%%%%%%%%%%%%%%%%%%%%%%%%%%%%%%%%%%%%%%%%%
\subsection{Driver Development}
\label{sec:driver}

The \textsf{childdoc} mechanism can also be use for the development
of definition files such as \LaTeX{} styles or classes.
This case differs from the above setup with multiple parts
included by |\include| in that no |\includeonly| should be invoked.
This can be achieved by starting the include file
(before |\ProvidesPackage|) with:
%
\begin{center}
\begin{tabular}{l}
|\input{childdoc.def}|\\
|\childdocforward{|\textit{main}|}|\\
\end{tabular}
\end{center}
%
or alternatively with:
%
\begin{center}
\begin{tabular}{l}
|\input{childdoc.def}|\\
|\childdocby{|\textit{main}|}|\\
\end{tabular}
\end{center}
%
Both forms have slightly different effects as described above.
The main file is prepared as usual, see \secref{sec:include}.

%%%%%%%%%%%%%%%%%%%%%%%%%%%%%%%%%%%%%%%%%%%%%%%%%%%%%%%%%%%%%%%%%%%%%%%%%%%%%%%%
\subsection{Legacy Detection}
\label{sec:detection}

The directive |\childdocmain| in the main file can detect
whether the complete document or merely a child is to be compiled
even without using the directive |\childdocof|.
This method is deprecated because it is less robust
and there is no compelling reason to use it;
it is merely provided for backward compatibility
and it may be removed in future versions.

If the detection mechanism is to be used,
it is mandatory to correctly specify
the filename of the main file as the argument of |\childdocmain|:
%
\begin{center}
\begin{tabular}{l}
|\input{childdoc.def}|\\
|\childdocmain{|\textit{main}|}|\\
\end{tabular}
\end{center}
%
If |\jobname| does not match the argument \textit{main} of |\childdocmain|,
it is assumed that |\jobname| points to the child file to be compiled.
When using |\childdocmain| with the main file specified as argument,
it suffices to start a child file
with just |\input{|\textit{main}|}|
without loading of the package and using |\childdocof|.
If instead all processing is done
with the appropriate \textsf{childdoc} directives,
the argument of \textit{main} of |\childdocmain| can be empty.

An alternative version of the command line processing described
in \secref{sec:commandline} using the detection mechanism reads:
%
\begin{center}
|... -jobname "|\textit{target}|" "|[\textit{flags}]%
[|\def\jobname{|\textit{dest}|}|]|\input{|\textit{main}|}"|
\end{center}

%%%%%%%%%%%%%%%%%%%%%%%%%%%%%%%%%%%%%%%%%%%%%%%%%%%%%%%%%%%%%%%%%%%%%%%%%%%%%%%%
\subsection{Manual Code}
\label{sec:manual}

In case one cannot be certain whether the definitions file |childdoc.def|
is installed on the target \TeX{} distribution
and one prefers not to ship it,
it is conceivable to paste a few relevant commands into the sources.

To that end, drop all statements |\input{childdoc.def}|
and perform the replacements as outlined below.
Instead of |\childdocmain{|\textit{main}|}| add the following code
to the top of the main file:
%
\begin{center}
\begin{tabular}{l}
|\||ifdefined\childdocname\endinput\||fi\newif\ifchilddoc|\\
|\edef\childdocname{\scantokens\expandafter{\jobname\noexpand}}|\\
|\def\childdocmain{|\textit{main}|}\||ifx\childdocmain\childdocname\||else|\\
|\childdoctrue\includeonly{\childdocname}\let\jobname\childdocmain\||fi|\\
\end{tabular}
\end{center}
%
Instead of |\childdocof{|\textit{main}|}| just include the main file
at the top of each child file:
%
\begin{center}
|\input{|\textit{main}|}|
\end{center}
%
A simple redirection |\childdocforward{|\textit{dest}|}| is achieved by:
%
\begin{center}
|\def\jobname{|\textit{dest}|}\input{\jobname}|
\end{center}
%
The redirection with prefix
|\childdocforwardprefix[|\textit{prefix}|]{|\textit{dest}|}|
is accomplished by:
%
\begin{center}
\begin{tabular}{l}
|{\edef\jobname{\scantokens\expandafter{\jobname\noexpand}}|\\
|\def\redirectjob |\textit{prefix}|#1~~~{\gdef\jobname{|\textit{dest}|#1}}|\\
|\expandafter\redirectjob\jobname~~~}\input{\jobname}|
\end{tabular}
\end{center}

In an alternative approach,
child documents can be compiled by a specific command line
without additional code or specific definitions:
%
\begin{center}
|... -jobname "|\textit{target}|" "|[\textit{flags}]%
|\includeonly{|\textit{dest}|}\input{|\textit{main}|}"|
\end{center}
%

%%%%%%%%%%%%%%%%%%%%%%%%%%%%%%%%%%%%%%%%%%%%%%%%%%%%%%%%%%%%%%%%%%%%%%%%%%%%%%%%
%%%%%%%%%%%%%%%%%%%%%%%%%%%%%%%%%%%%%%%%%%%%%%%%%%%%%%%%%%%%%%%%%%%%%%%%%%%%%%%%
\section{Information}

%%%%%%%%%%%%%%%%%%%%%%%%%%%%%%%%%%%%%%%%%%%%%%%%%%%%%%%%%%%%%%%%%%%%%%%%%%%%%%%%
\subsection{Copyright}

Copyright \copyright{} 2017--2018 Niklas Beisert

This work may be distributed and/or modified under the
conditions of the \LaTeX{} Project Public License, either version 1.3
of this license or (at your option) any later version.
The latest version of this license is in
  \url{http://www.latex-project.org/lppl.txt}
and version 1.3 or later is part of all distributions of \LaTeX{}
version 2005/12/01 or later.

This work has the LPPL maintenance status `maintained'.

The Current Maintainer of this work is Niklas Beisert.

This work consists of the files |README.txt|, |childdoc.ins| and |childdoc.dtx|
as well as the derived files |childdoc.def|, |cdocsamp.tex|
with |cdocsch1.tex|, |cdocsch2.tex|, |cdocspt3.tex|, |cdocspt4.tex|,
|cdocsdrf.tex|, |cdocsfn1.tex|, |cdocsfn2.tex|
as well as |childdoc.pdf|.

%%%%%%%%%%%%%%%%%%%%%%%%%%%%%%%%%%%%%%%%%%%%%%%%%%%%%%%%%%%%%%%%%%%%%%%%%%%%%%%%
\subsection{Files and Installation}

The package consists of the files:
%
\begin{center}
\begin{tabular}{ll}
    |README.txt|   & readme file \\
    |childdoc.ins| & installation file \\
    |childdoc.dtx| & source file \\
    |childdoc.def| & definition file \\
    |cdocsamp.tex| & sample main file \\
    |cdocsch1.tex| & sample include file \\
    |cdocsch2.tex| & sample include file \\
    |cdocspt3.tex| & sample part file \\
    |cdocspt4.tex| & sample part file \\
    |cdocsdrf.tex| & sample redirection file \\
    |cdocsfn1.tex| & sample redirection file \\
    |cdocsfn2.tex| & sample redirection file \\
    |childdoc.pdf| & manual
\end{tabular}
\end{center}
%
The distribution consists of the files
|README.txt|, |childdoc.ins| and |childdoc.dtx|.
%
\begin{itemize}
\item
Run (pdf)\LaTeX{} on |childdoc.dtx|
to compile the manual |childdoc.pdf| (this file).
\item
Run \LaTeX{} on |childdoc.ins| to create the definitions file |childdoc.def|
and the sample |cdocsamp.tex| with include files
|cdocsch1.tex|, |cdocsch2.tex|, |cdocspt3.tex|, |cdocspt4.tex|,
|cdocsdrf.tex|, |cdocsfn1.tex|, |cdocsfn2.tex|.
Then copy the file |childdoc.def| to an appropriate directory of your \LaTeX{}
distribution, e.g.\ \textit{texmf-root}|/tex/latex/childdoc|.
\end{itemize}

%%%%%%%%%%%%%%%%%%%%%%%%%%%%%%%%%%%%%%%%%%%%%%%%%%%%%%%%%%%%%%%%%%%%%%%%%%%%%%%%
\subsection{Related CTAN Packages}

There are several other packages which offer a similar functionality:
%
\begin{itemize}
\item
The packages
\href{http://ctan.org/pkg/docmute}{\textsf{docmute}},
\href{http://ctan.org/pkg/includex}{\textsf{includex}} and
\href{http://ctan.org/pkg/standalone}{\textsf{standalone}}
provide commands to include only the document body of
a child file thus allowing both files to be compiled individually.
\item
The packages \href{http://ctan.org/pkg/subdocs}{\textsf{subdocs}}
and \href{http://ctan.org/pkg/subfiles}{\textsf{subfiles}}
provide structures in which the main and child documents can be
encapsulated and allowing them to be compiled individually.
The inclusion mechanism is different from the conventional |\include|.
\item
The package \href{http://ctan.org/pkg/combine}{\textsf{combine}}
is an elaborate solution to combine several documents into one.
\end{itemize}
%
See also the CTAN topic \href{http://ctan.org/topic/subdocs}{\textsf{subdocs}}
for further related packages.
The present package differs from the above solutions in that
a document structure constructed with the conventional |\include| mechanism
just needs two extra commands at the top of every file
such that all constituent files can be compiled individually.

%%%%%%%%%%%%%%%%%%%%%%%%%%%%%%%%%%%%%%%%%%%%%%%%%%%%%%%%%%%%%%%%%%%%%%%%%%%%%%%%
%\subsection{Feature Suggestions}
%
%The following is a list of features which may be useful for future
%versions of this package:
%%
%\begin{itemize}
%\item
%\ldots
%\end{itemize}

%%%%%%%%%%%%%%%%%%%%%%%%%%%%%%%%%%%%%%%%%%%%%%%%%%%%%%%%%%%%%%%%%%%%%%%%%%%%%%%%
\subsection{Revision History}

%%%%%%%%%%%%%%%%%%%%%%%%%%%%%%%%%%%%%%%%
\paragraph{v2.0:} 2018/12/30

\begin{itemize}
\item
immediate forward processing
\item
added |\childdocby| mechanism
\item
manual restructured
\end{itemize}

%%%%%%%%%%%%%%%%%%%%%%%%%%%%%%%%%%%%%%%%
\paragraph{v1.6:} 2018/01/17

\begin{itemize}
\item
application for development of include files
\item
corrections to manual
\end{itemize}

%%%%%%%%%%%%%%%%%%%%%%%%%%%%%%%%%%%%%%%%
\paragraph{v1.5:} 2017/05/21

\begin{itemize}
\item
more complete structuring introduced
\item
|\childdocof| introduced
\item
|\childdoc| renamed to |\childdocmain|
\item
|\childredirect| renamed to |\childdocforward| and |\childdocforwardprefix|
and functionality expanded
\end{itemize}

%%%%%%%%%%%%%%%%%%%%%%%%%%%%%%%%%%%%%%%%
\paragraph{v1.0:} 2017/04/27

\begin{itemize}
\item
manual and install package
\item
first version published on CTAN
\end{itemize}

%%%%%%%%%%%%%%%%%%%%%%%%%%%%%%%%%%%%%%%%
\paragraph{v0.6:} 2017/04/26

\begin{itemize}
\item
redirection mechanism added
\end{itemize}

%%%%%%%%%%%%%%%%%%%%%%%%%%%%%%%%%%%%%%%%
\paragraph{v0.5:} 2017/04/26

\begin{itemize}
\item
functionality in definition file
\end{itemize}


%%%%%%%%%%%%%%%%%%%%%%%%%%%%%%%%%%%%%%%%%%%%%%%%%%%%%%%%%%%%%%%%%%%%%%%%%%%%%%%%
%%%%%%%%%%%%%%%%%%%%%%%%%%%%%%%%%%%%%%%%%%%%%%%%%%%%%%%%%%%%%%%%%%%%%%%%%%%%%%%%
%%%%%%%%%%%%%%%%%%%%%%%%%%%%%%%%%%%%%%%%%%%%%%%%%%%%%%%%%%%%%%%%%%%%%%%%%%%%%%%%
\appendix

\settowidth\MacroIndent{\rmfamily\scriptsize 000\ }

 \DocInput{childdoc.dtx}

\end{document}
%</driver>
% \fi
%
% %%%%%%%%%%%%%%%%%%%%%%%%%%%%%%%%%%%%%%%%%%%%%%%%%%%%%%%%%%%%%%%%%%%%%%%%%%%%%%
% %%%%%%%%%%%%%%%%%%%%%%%%%%%%%%%%%%%%%%%%%%%%%%%%%%%%%%%%%%%%%%%%%%%%%%%%%%%%%%
% \section{Sample}
%\iffalse
%<*samplemain>
%\fi
%
% The following presents a sample document
% with two chapters, two parts, a title page,
% a compile flag as well as three forwarding files to set the flag.
% It consists of eight |.tex| files:
% \begin{center}
% \begin{tabular}{ll}
% |cdocsamp.tex|&main file\\
% |cdocsch1.tex|&include file for chapter 1\\
% |cdocsch2.tex|&include file for chapter 2\\
% |cdocspt3.tex|&include file for part 3\\
% |cdocspt4.tex|&include file for part 4\\
% |cdocsdrf.tex|&forwarding file for main file in draft mode\\
% |cdocsfi1.tex|&forwarding file for final version of chapter 1\\
% |cdocsfi2.tex|&forwarding file for final version of chapter 2\\
% \end{tabular}
% \end{center}
% Each of the eight files can be compiled directly by the \LaTeX{} compiler.
%
% %%%%%%%%%%%%%%%%%%%%%%%%%%%%%%%%%%%%%%
% \paragraph{Main File.}
%
% The main file is called |cdocsamp.tex|.
%
% Load the \textsf{childdoc} definitions and
% declare the filename for the main document:
%    \begin{macrocode}
\input{childdoc.def}
\childdocmain{}
%    \end{macrocode}

% Optional override for |\version| flag:
%    \begin{macrocode}
%%\ifchilddoc\else\providecommand{\version}{draft}\fi
%    \end{macrocode}

% Define the default values for the |\version| flag
% (|final| for the main file and |draft| for childs):
%    \begin{macrocode}
\ifchilddoc
\providecommand{\version}{draft}
\else
\providecommand{\version}{final}
\fi
%    \end{macrocode}

% Load the standard document class:
%    \begin{macrocode}
\documentclass[12pt]{article}
%    \end{macrocode}

% Start the document body:
%    \begin{macrocode}
\begin{document}
%    \end{macrocode}

% Declare a title page.
% Print title, part of document being processed and version flag:
%    \begin{macrocode}
\addtocounter{page}{-1}
\begin{center}
{\LARGE\bfseries{}childdoc example\par}
\vspace{1cm}
\ifchilddoc
\ifchilddocmanual part\else chapter\fi:
`\childdocname' of `\childdocjob'\par
\else
main document: `\childdocjob'\par
\fi
version: \version\par
\end{center}
\newpage
%    \end{macrocode}

% Manually include selected file,
% otherwise process as usual:
%    \begin{macrocode}
\ifchilddocmanual
\section*{part `\childdocname'}
\input{\childdocname}
\else
%    \end{macrocode}

% Include the two chapters:
%    \begin{macrocode}
\include{cdocsch1}
\include{cdocsch2}
%    \end{macrocode}

% Include the two parts unless only chapters should be displayed:
%    \begin{macrocode}
\ifchilddoc\else
\section{part three}
\input{cdocspt3}
\section{part four}
\input{cdocspt4}
\fi
%    \end{macrocode}

% Process as usual until here:
%    \begin{macrocode}
\fi
%    \end{macrocode}

% End of document body:
%    \begin{macrocode}
\end{document}
%    \end{macrocode}
%\iffalse
%</samplemain>
%\fi
%
% %%%%%%%%%%%%%%%%%%%%%%%%%%%%%%%%%%%%%%
% \paragraph{Chapter Include Files.}
%
% The include files are called |cdocsch1.tex| and |cdocsch2.tex|.
%
%\iffalse
%<*samplechap1|samplechap2>
%\fi

% Optional override for |\version| flag:
%    \begin{macrocode}
%%\providecommand{\version}{final}
%    \end{macrocode}

% Include the main document:
%    \begin{macrocode}
\input{childdoc.def}
\childdocof{cdocsamp}
%    \end{macrocode}

%\iffalse
%</samplechap1|samplechap2>
%\fi
%
%\iffalse
%<*samplechap1>
%\fi
% Some text for chapter 1:
%    \begin{macrocode}
\section{one}
some text in chapter one
%    \end{macrocode}

%\iffalse
%</samplechap1>
%\fi
% Some text for chapter 2:
%\iffalse
%<*samplechap2>
%\fi
%    \begin{macrocode}
\section{two}
more text in chapter two
%    \end{macrocode}

%\iffalse
%</samplechap2>
%\fi
%
% %%%%%%%%%%%%%%%%%%%%%%%%%%%%%%%%%%%%%%
% \paragraph{Part Include Files.}
%
% The include files are called |cdocspt3.tex| and |cdocspt4.tex|.
%
%\iffalse
%<*samplepart3|samplepart4>
%\fi

% Optional override for |\version| flag:
%    \begin{macrocode}
%%\providecommand{\version}{final}
%    \end{macrocode}

% Include the main document:
%    \begin{macrocode}
\input{childdoc.def}
\childdocby{cdocsamp}
%    \end{macrocode}

%\iffalse
%</samplepart3|samplepart4>
%\fi
%
%\iffalse
%<*samplepart3>
%\fi
% Some text for part 3:
%    \begin{macrocode}
some text in part three
%    \end{macrocode}

%\iffalse
%</samplepart3>
%\fi
% Some text for part 4:
%\iffalse
%<*samplepart4>
%\fi
%    \begin{macrocode}
more text in part four
%    \end{macrocode}

%\iffalse
%</samplepart4>
%\fi
%
% %%%%%%%%%%%%%%%%%%%%%%%%%%%%%%%%%%%%%%
% \paragraph{Forwarding for a Complete Draft.}
%
% The following forwarding file |cdocsdrf.tex|
% compiles the main document in draft mode:
%\iffalse
%<*sampledraft>
%\fi
%    \begin{macrocode}
\def\version{draft}
\input{childdoc.def}
\childdocforward{cdocsamp}
%    \end{macrocode}

%\iffalse
%</sampledraft>
%\fi
%
% %%%%%%%%%%%%%%%%%%%%%%%%%%%%%%%%%%%%%%
% \paragraph{Forwarding for Final Version of the Chapters.}
%
% The following forwarding files |cdocsfn1.tex| and |cdocsfn2.tex|
% (with identical content)
% compile the final versions of the child documents
% |cdocsch1.tex| and |cdocsch2.tex|, respectively:
%\iffalse
%<*samplefinal>
%\fi
%    \begin{macrocode}
\def\version{final}
\input{childdoc.def}
\childdocforwardprefix[cdocsamp]{cdocsfn}{cdocsch}
%    \end{macrocode}

%\iffalse
%</samplefinal>
%\fi
%
% %%%%%%%%%%%%%%%%%%%%%%%%%%%%%%%%%%%%%%
% \paragraph{Command Line Processing.}
%
% The following three command lines generate the output files
% |cdocscld|, |cdocscl1| and |cdocscl2|
% which should be identical to
% |cdocsdrf|, |cdocsch1| and |cdocsfn2|, respectively:
% \begin{center}
% \begin{tabular}{l}
% |latex -jobname cdocscld \|\\
% |  "\def\version{draft}\input{childdoc.def}\childdocforward{cdocsamp}"|\\
% |latex -jobname cdocscl1 \|\\
% |  "\input{childdoc.def}\childdocforward[cdocsamp]{cdocsch1}"|\\
% |latex -jobname cdocscl2 \|\\
% |  "\def\version{final}\input{childdoc.def}\childdocforward{cdocsch2}"|
% \end{tabular}
% \end{center}
% Note that the trailing backslash on each first line
% merely continues the input to the second line
% (for convenient cut ant paste).
% Furthermore, the command |latex| can be replaced by any
% of its alternative versions such as |pdflatex|.
%
% %%%%%%%%%%%%%%%%%%%%%%%%%%%%%%%%%%%%%%%%%%%%%%%%%%%%%%%%%%%%%%%%%%%%%%%%%%%%%%
% %%%%%%%%%%%%%%%%%%%%%%%%%%%%%%%%%%%%%%%%%%%%%%%%%%%%%%%%%%%%%%%%%%%%%%%%%%%%%%
% \section{Implementation}
%\iffalse
%<*package>
%\fi
%
% This section describes the definitions file |childdoc.def|.

% The definitions cannot be loaded using |\usepackage| or |\RequirePackage|
% which has a mechanism to prevent loading a style file more than once.
% When loading the definitions by means of |\input|
% multiple instances have to be prevented manually:
%\iffalse
%This code needs to be before the `\ProvidesFile' directive
%which is defined at the beginning of this file.
%Therefore it is also placed there and commented out here.
%</package>
%<*discard>
%\fi
%    \begin{macrocode}
\ifdefined\childdocmain\endinput\fi
%    \end{macrocode}
%\iffalse
%</discard>
%<*package>
%\fi
%
% \macro{\ifchilddoc}
% \macro{\ifchilddocmanual}
% The conditional |\ifchilddoc| tells whether a
% child (true) or main (false) document is being compiled.
% The conditional |\ifchilddocmanual| tells whether
% the |\includeonly| mechanism is used (false) or
% the selection of child files must be performed manually (true).
% The definitions initialise to false:
%    \begin{macrocode}
\newif\ifchilddoc
\newif\ifchilddocmanual
%    \end{macrocode}

% \macro{\childdocname}
% \macro{\childdocjob}
% The macro |\childdocname| stores the name of the main document
% to be compiled. The macro |\childdocjob| stores the name of
% the document on which the \LaTeX{} compiler was originally invoked.
% The content of |\jobname| cannot be compared
% to filenames specified in the source due to different catcodes.
% The following code rescans |\jobname|, stores the result
% in |\childdocname| and saves a copy in |\childdocjob|:
%    \begin{macrocode}
\edef\childdocname{\scantokens\expandafter{\jobname\noexpand}}
\let\childdocjob\childdocname
%    \end{macrocode}

% \macro{\childdocdisable}
% The macro |\childdocdisable| prevents the main file
% from being processed more than once.
% At this stage, the main document command |\childdocmain|
% is assumed to be called once again where it should do nothing.
% Any subsequent call to it should prevent
% a secondary processing of the main document
% It overwrites the forwarding commands
% |\childdocof| and |\childdocforward|
% with empty macros to prevent further inclusions of the main document:
%    \begin{macrocode}
\newcommand{\childdocdisable}
{
  \renewcommand{\childdocmain}[1]{\renewcommand{\childdocmain}[1]{\endinput}}
  \renewcommand{\childdocof}[1]{}
  \renewcommand{\childdocby}[2][]{}
  \renewcommand{\childdocforward}[2][]{}
  \renewcommand{\childdocdisable}{}
}
%    \end{macrocode}

% \macro{\childdocmain}
% The macro |\childdocmain| is to be called at the top of the main file
% with nothing or the main filename (without extension) as argument.
% First, it breaks loops.
% If the argument is not empty and does not match |\childdocname|
% (which is set by the first inclusion of |childdoc.def|),
% |\ifchilddoc| is set to true, |\includeonly| is applied to the child file
% and |\jobname| is set to the main file
% (for proper handling of |.aux| files):
%    \begin{macrocode}
\newcommand{\childdocmain}[1]
{
  \childdocdisable\childdocmain{}
  \if?#1?\else
    \begingroup
      \def\childdoctmp{#1}
      \ifx\childdoctmp\childdocname
        \def\childdoctmp{}
      \else
        \def\childdoctmp
        {
          \childdoctrue
          \includeonly{\childdocname}
          \def\childdocjob{#1}
          \def\jobname{#1}
        }
      \fi
      \expandafter
    \endgroup
    \childdoctmp
  \fi
}
%    \end{macrocode}

% \macro{\childdocof}
% The command |\childdocof| redirects
% compilation to the main file |#1|.
%    \begin{macrocode}
\newcommand{\childdocof}[1]
{
  \childdocdisable
  \childdoctrue
  \includeonly{\childdocname}
  \def\jobname{#1}
  \def\childdocjob{#1}
  \input{#1}
}
%    \end{macrocode}

% \macro{\childdocby}
% The command |\childdocby| ....
%    \begin{macrocode}
\newcommand{\childdocby}[2][]
{
  \childdocdisable
  \childdoctrue
  \childdocmanualtrue
  \if?#1?\else
    \def\jobname{#2}
  \fi
  \def\childdocjob{#2}
  \input{#2}
  \endinput
}
%    \end{macrocode}

% \macro{\childdocforward}
% The command |\childdocforward| redirects
% compilation to the main file or
% (if the optional argument is given) a child file.
% Parameters are set as if the main file
% or a child file starting with |\childdocof| was compiled.
% Then compilation is handed over to the main file:
%    \begin{macrocode}
\newcommand{\childdocforward}[2][]
{
  \begingroup
    \if?#1?
      \def\childdoctmp
      {
        \def\childdocname{#2}
        \def\childdocjob{#2}
        \def\jobname{#2}
        \input{#2}
        \endinput
      }
    \else
      \def\childdoctmp
      {
        \childdocdisable
        \def\childdocname{#2}
        \childdoctrue
        \includeonly{#2}
        \def\childdocjob{#1}
        \def\jobname{#1}
        \input{#1}
        \endinput
      }
    \fi
    \expandafter
  \endgroup
  \childdoctmp
}
%    \end{macrocode}

% \macro{\childdocforwardprefix}
% The command |\childdocforwardprefix| redirects
% compilation to the main or a child file by means of a pattern.
% The prefix |#1| in the current filename is replaced by |#2|
% and the suffix of the current filename is kept
% (it is assumed that the filename does not contain the substring `|~~~|'
% which is used as a delimiter).
% Compilation is handed over to the new file by |\childdocforward|:
%    \begin{macrocode}
\newcommand{\childdocforwardprefix}[3][]
{
  \begingroup
    \def\childdocextract #2##1~~~{\def\childdoctmp{\childdocforward[#1]{#3##1}}}
    \expandafter\childdocextract\childdocname~~~
    \expandafter
  \endgroup
  \childdoctmp
}
%    \end{macrocode}

% \macro{\childdoc}
% The deprecated macro |\childdoc| is a legacy version of |\childdocmain|:
%    \begin{macrocode}
\newcommand{\childdoc}{\childdocmain}
%    \end{macrocode}

% \macro{\childdocredirect}
% The deprecated macro |\childdocredirect| is a legacy version
% of |\childdocforward| and |\childdocforwardprefix|:
%    \begin{macrocode}
\newcommand{\childdocredirect}[2][]
{
  \begingroup
    \if?#1?
      \def\childdoctmp{\childdocforward{#2}}
    \else
      \def\childdoctmp{\childdocforwardprefix{#1}{#2}}
    \fi
    \expandafter
  \endgroup
  \childdoctmp
}
%    \end{macrocode}

%\iffalse
%</package>
%\fi
%
\endinput
|\\
|\childdocforward{|\textit{main}|}|\\
\end{tabular}
\end{center}
%
or alternatively with:
%
\begin{center}
\begin{tabular}{l}
|% \iffalse
%
% childdoc.dtx Copyright (C) 2017-2018 Niklas Beisert
%
% This work may be distributed and/or modified under the
% conditions of the LaTeX Project Public License, either version 1.3
% of this license or (at your option) any later version.
% The latest version of this license is in
%   http://www.latex-project.org/lppl.txt
% and version 1.3 or later is part of all distributions of LaTeX
% version 2005/12/01 or later.
%
% This work has the LPPL maintenance status `maintained'.
%
% The Current Maintainer of this work is Niklas Beisert.
%
% This work consists of the files childdoc.dtx and childdoc.ins
% and the derived files childdoc.def and cdocsamp.tex with
% cdocsch1.tex, cdocsch2.tex, cdocsdrf.tex, cdocsfn1.tex, cdocsfn2.tex.
%
%<package>\ifdefined\childdocmain\endinput\fi
%<package>\ProvidesFile{childdoc.def}[2018/12/30 v2.0 child document driver]
%<samplemain>\ProvidesFile{cdocsamp.tex}[2018/12/30 v2.0 sample for childdoc]
%<*driver>
%\ProvidesFile{childdoc.drv}[2018/12/30 v2.0 childdoc reference manual file]
\PassOptionsToClass{10pt,a4paper}{article}
\documentclass{ltxdoc}

\usepackage[margin=35mm]{geometry}
\usepackage{hyperref}
\usepackage{hyperxmp}
\usepackage[usenames]{color}

\hypersetup{colorlinks=true}
\hypersetup{pdfstartview=FitH}
\hypersetup{pdfpagemode=UseNone}
\hypersetup{pdfsource={}}
\hypersetup{pdflang={en-UK}}
\hypersetup{pdfcopyright={Copyright 2017-2018 Niklas Beisert.
  This work may be distributed and/or modified under the
  conditions of the LaTeX Project Public License, either version 1.3
  of this license or (at your option) any later version.}}
\hypersetup{pdflicenseurl={http://www.latex-project.org/lppl.txt}}
\hypersetup{pdfcontactaddress={ETH Zurich, ITP, HIT K,
  Wolfgang-Pauli-Strasse 27}}
\hypersetup{pdfcontactpostcode={8093}}
\hypersetup{pdfcontactcity={Zurich}}
\hypersetup{pdfcontactcountry={Switzerland}}
\hypersetup{pdfcontactemail={nbeisert@itp.phys.ethz.ch}}
\hypersetup{pdfcontacturl={http://people.phys.ethz.ch/\xmptilde nbeisert/}}

\newcommand{\secref}[1]{\hyperref[#1]{section \ref*{#1}}}

\parskip1ex
\parindent0pt
\let\olditemize\itemize
\def\itemize{\olditemize\parskip0pt}

\begin{document}

\title{The \textsf{childdoc} Package}
\hypersetup{pdftitle={The childdoc Package}}
\author{Niklas Beisert\\[2ex]
  Institut f\"ur Theoretische Physik\\
  Eidgen\"ossische Technische Hochschule Z\"urich\\
  Wolfgang-Pauli-Strasse 27, 8093 Z\"urich, Switzerland\\[1ex]
  \href{mailto:nbeisert@itp.phys.ethz.ch}
  {\texttt{nbeisert@itp.phys.ethz.ch}}}
\hypersetup{pdfauthor={Niklas Beisert}}
\hypersetup{pdfsubject={Manual for the LaTeX2e Package childdoc}}
\date{30 December 2018, \textsf{v2.0}}
\maketitle

\begin{abstract}\noindent
\textsf{childdoc} is a \LaTeXe{} package
that enables the direct compilation
of document sections included by |\include|
to individual files.
\end{abstract}

\begingroup
\parskip0ex
\tableofcontents
\endgroup

%%%%%%%%%%%%%%%%%%%%%%%%%%%%%%%%%%%%%%%%%%%%%%%%%%%%%%%%%%%%%%%%%%%%%%%%%%%%%%%%
%%%%%%%%%%%%%%%%%%%%%%%%%%%%%%%%%%%%%%%%%%%%%%%%%%%%%%%%%%%%%%%%%%%%%%%%%%%%%%%%
\section{Introduction}

\LaTeX{} provides a mechanism to structure a large document (such as a book)
into a main file and several child files (containing the chapters)
using the |\include| command.
This mechanism is beneficial for documents
which span hundreds of pages in order to
make the source file(s) more manageable.
Moreover, compilation can be restricted to
selected child files by means of the |\includeonly| command.
The latter feature can be used to reduce the compilation time while editing
(this was significantly more useful in the earlier days of \LaTeX{})
or to generate a smaller document which is easier to navigate.
Another application of |\includeonly| is to generate
documents consisting of selected parts of the complete document.

However, there are a few drawbacks of the plain |\include| mechanism:
\begin{itemize}
\item
The child files cannot be compiled on their own,
they can only be compiled via the main file.
A naive editing environment
(such as a text editor with an option
to have the current file processed by \LaTeX)
may require one to switch to the main file before compiling;
attempting to compile the child file produces errors.
\item
The main file must be modified (each time)
to adjust the |\includeonly| command
to the present needs. This easily leaves the main file in a messy state.
\item
The generated document will always carry the filename
of the main document. This is inconvenient if
several child files are to be compiled and
to be kept for distribution.
\end{itemize}

The present package provides a simple interface
to make child files individually compilable by \LaTeX{}.
Compiling a child file then has the same effect as compiling
the main file with an |\includeonly| command
to select the appropriate child.
Moreover the generated document will carry the name of the child
rather than the main file.
This resolves all three above issues.

This feature is meant to make the editing of books,
thesis documents and lecture notes somewhat more convenient.
However, the package can also be used efficiently for
composing a series of documents (such as exercise sheets)
which are typically distributed individually.
It then assists the author in generating the individual documents
(potentially in different versions)
as well as a document containing the collected series.
Another application is in developing style files
or other kinds of included material
where compilation of the style file could redirect
to a sample or test file.

%%%%%%%%%%%%%%%%%%%%%%%%%%%%%%%%%%%%%%%%%%%%%%%%%%%%%%%%%%%%%%%%%%%%%%%%%%%%%%%%
%%%%%%%%%%%%%%%%%%%%%%%%%%%%%%%%%%%%%%%%%%%%%%%%%%%%%%%%%%%%%%%%%%%%%%%%%%%%%%%%
\section{Usage}

First of all, the package \textsf{childdoc} is \emph{not} a standard
\LaTeXe{} |.sty| style file! Therefore it needs to be invoked in
a non-standard way.

%%%%%%%%%%%%%%%%%%%%%%%%%%%%%%%%%%%%%%%%%%%%%%%%%%%%%%%%%%%%%%%%%%%%%%%%%%%%%%%%
\subsection{Included Files}
\label{sec:include}

%%%%%%%%%%%%%%%%%%%%%%%%%%%%%%%%%%%%%%%%
\DescribeMacro{\childdocmain}
To use the package, add the commands
\begin{center}
\begin{tabular}{l}
|\input{childdoc.def}|\\
|\childdocmain{}|\\
\end{tabular}
\end{center}
at the very top of the main \LaTeX{} file,
in particular \emph{before} the |\documentclass| statement!
The argument of |\childdocmain| should be left empty
(but it must be present).

%%%%%%%%%%%%%%%%%%%%%%%%%%%%%%%%%%%%%%%%
\DescribeMacro{\childdocof}
Furthermore, add the commands
\begin{center}
\begin{tabular}{l}
|\input{childdoc.def}|\\
|\childdocof{|\textit{main}|}|\\
\end{tabular}
\end{center}
at the top of every child file \textit{child}
which is included by |\include{|\textit{child}|}|
from within the main file
(or at least for those files to be compiled individually).
The argument \textit{main} must be the filename of the main file.

There are a couple of
considerations in setting up the main and child documents:

%%%%%%%%%%%%%%%%%%%%%%%%%%%%%%%%%%%%%%%%
\paragraph{Restrictions.}

Please note the following restrictions:
\begin{itemize}
\item
|\childdocmain| must be called with one argument \textit{main}
to ensure compatibility with earlier version of the package.
It must either be empty (|\childdocmain{}|)
or precisely match the filename of the main file in which it is specified.
See \secref{sec:detection} for further information.
\item
The filename \textit{main} must be specified without the |.tex| extension.
\item
The filename \textit{main} is case sensitive
(even in case-insensitive file systems)
due to internal string comparison.
\item
The argument \textit{main} should be fully expanded, it cannot be a macro.
\item
Subdirectories and special characters should be avoided in filenames.
\item
The command |\childdocmain{|\textit{main}|}| must be followed by a whitespace.
It should not be followed immediately by another command
or by a comment mark `|%|'.
This is because the \TeX{} parser reads the token immediately following
the argument of |\childdocmain| and puts it
at the beginning of every child section;
however, a white\-space is ignored.
\end{itemize}

%%%%%%%%%%%%%%%%%%%%%%%%%%%%%%%%%%%%%%%%
\paragraph{Content of Main File.}

It is advisable to place all content in the child files included by |\include|.
Any output contained in the main file will appear in all child documents
unless suppressed manually;
it cannot be suppressed automatically by the |\includeonly| directive
and thus should normally be avoided.
A method to include some content in the main file
by means of conditional processing is described in \secref{sec:conditional}.

%%%%%%%%%%%%%%%%%%%%%%%%%%%%%%%%%%%%%%%%
\paragraph{Page Numbering.}

When only a part of the document is compiled,
the appropriate numbering of pages
(as well as other status parameters)
is determined from the |.aux| files.
The latter contain information from previous passes.
However this information needs to propagate through
all intermediate child documents.
Therefore the page numbering in child documents may well
be inconsistent until the complete document is compiled at least once.

A useful (if unconventional) way to always ensure a consistent
page numbering is to restart the numbering in each child document
and denote the pages by `\textit{child}|.|\textit{page}'
where \textit{child} represents the chapter/section number of the child file.
This can be achieved by the command
|\numberwithin{page}{|\textit{child}|}|
of the \textsf{amsmath} package
where \textit{child} can be |chapter| or |section|
depending on the chosen structuring.
Alternatively, one can modify the macro |\thepage| appropriately
and reset the counter |page| at the start of each child file.

%%%%%%%%%%%%%%%%%%%%%%%%%%%%%%%%%%%%%%%%%%%%%%%%%%%%%%%%%%%%%%%%%%%%%%%%%%%%%%%%
\subsection{Conditional Processing}
\label{sec:conditional}

The package provides a mechanism to compile different versions
of a document. To customise the versions further some conditional processing
can come in handy to distinguish which version is being compiled.
The package provides two macros to describe the compilation context:

%%%%%%%%%%%%%%%%%%%%%%%%%%%%%%%%%%%%%%%%
\DescribeMacro{\ifchilddoc}
The conditional |\ifchilddoc| distinguishes between the compilation of
child documents and the main document:
%
\begin{center}
|\ifchilddoc |\textit{child-code}| |[|\||else |\textit{main-code}]| \||fi|
\end{center}

%%%%%%%%%%%%%%%%%%%%%%%%%%%%%%%%%%%%%%%%
\DescribeMacro{\childdocname}
\DescribeMacro{\childdocjob}
The macro |\childdocname| contains the filename (without extension)
of the main or child file being processed.
Note that |\childdocjob| will always contain the name of the main file.

%%%%%%%%%%%%%%%%%%%%%%%%%%%%%%%%%%%%%%%%
\paragraph{Title Page.}

Conditional processing can be used to include a title or banner page
in the main document when proper precautions are taken.
Importantly, the code in the main file should ensure that the page counter
(as well as other status parameters which are stored in the |.aux| files)
takes the same value after the conditional processing.
Otherwise the page numbers may take divergent values
depending on which part is compiled.

For example, a title page could be declared by:
%
\begin{center}
\begin{tabular}{l}
|\ifchilddoc\||else|\\
|\addtocounter{page}{-1}|\\
\textit{code for title page}\\
|\newpage|\\
|\||fi|
\end{tabular}
\end{center}
%
A banner page for the child documents can be generated by:
%
\begin{center}
\begin{tabular}{l}
|\ifchilddoc|\\
|\addtocounter{page}{-1}|\\
\textit{code for banner page}\\
|\newpage|\\
|\||fi|
\end{tabular}
\end{center}
%
Here one could write a message such as:
\begin{center}
|This is the part \childdocname{} of \childdocjob{}.|
\end{center}

%%%%%%%%%%%%%%%%%%%%%%%%%%%%%%%%%%%%%%%%%%%%%%%%%%%%%%%%%%%%%%%%%%%%%%%%%%%%%%%%
\subsection{Flags}
\label{sec:flags}

The package makes it easy to generate different versions
of the main or child documents.
To this end compilation flags can be defined
and assigned different default values.
They will be particularly useful in conjunction
with the forwarding mechanism described in \secref{sec:forward}.

For example, it may be useful to have a flag |\version|
which can be set to |draft| or |final|.
The document source will contain some conditional code
depending on the value of |\version|.
Suppose further, the flag should default to |final| for the main file
and to |draft| for child files
which is a natural assignment for editing the document.
This is achieved by placing the following code
in the preamble of the main document
(below the |\childdocmain| directive):
%
\begin{center}
\begin{tabular}{l}
|\ifchilddoc|\\
|\providecommand{\version}{draft}|\\
|\||else|\\
|\providecommand{\version}{final}|\\
|\||fi|
\end{tabular}
\end{center}
%
The definition by |\providecommand| makes sure
that previous definitions are not overwritten.
Further statements |\providecommand{\version}{...}|
can thus be added before the above code to override it.

For the main file, one might add a line
(between |\childdocmain| and the above block)
%
\begin{center}
|%\ifchilddoc\||else\providecommand{\version}{draft}\||fi|
\end{center}
%
which can be uncommented to produce a draft version.
Likewise one can add a line to the very top of a child file
(above the |\childdocof{|\textit{main}|}| directive)
%
\begin{center}
|%\providecommand{\version}{final}|
\end{center}
%
which can be uncommented to produce the final version of this child document.

%%%%%%%%%%%%%%%%%%%%%%%%%%%%%%%%%%%%%%%%%%%%%%%%%%%%%%%%%%%%%%%%%%%%%%%%%%%%%%%%
\subsection{Forwarding}
\label{sec:forward}

Different versions of the main or child documents
using compilation flags as described in \secref{sec:flags}
can be (permanently) stored in different files
for convenient compilation, viewing and distribution.
To this end, the package defines a command
to pass on compilation to a different file:

%%%%%%%%%%%%%%%%%%%%%%%%%%%%%%%%%%%%%%%%
\DescribeMacro{\childdocforward}
The command |\childdocforward| redirects processing to
another source file:
%
\begin{center}
\begin{tabular}{l}
|\input{childdoc.def}|\\
|\childdocforward[|\textit{main}|]{|\textit{dest}|}|\\
\end{tabular}
\end{center}
%
The argument \textit{dest} is the destination file
(without extension).
It should be the main file or one of the child files.
Note that further \textsf{childdoc} directives
such as |\childdocof| and |\childdocforward|
in the indicated file will be processed in this form.
The optional argument \textit{main}
passes on directly to the main file \textit{main}
while pretending to compile the child \textit{dest}.
This form behaves as if \textit{dest}
issues |\childdocof{|\textit{main}|}| right away,
and no further \textsf{childdoc} directives will be processed.

%%%%%%%%%%%%%%%%%%%%%%%%%%%%%%%%%%%%%%%%
\DescribeMacro{\...prefix}
In the alternative form |\childdocforwardprefix|,
%
\begin{center}
\begin{tabular}{l}
|\input{childdoc.def}|\\
|\childdocforwardprefix[|\textit{main}|]{|\textit{prefix}|}{|\textit{dest}|}|
\end{tabular}
\end{center}
%
the destination file is determined by a pattern
depending on the current file:
To make this work, the current file must be called
`{\textit{prefix}\hspace{0.2em}\textit{suffix}}'
with \textit{prefix} matching precisely the argument.
Processing is then passed on to the file
`{\textit{dest}\hspace{0.2em}\textit{suffix}}'.
Surely, the same effect is achieved by
directly specifying the
argument `{\textit{dest}\hspace{0.2em}\textit{suffix}}'
in the first form.
However, that requires to set up a different file
for each child. With the alternative form of the command
all these files can have exactly the same content
which simplifies setting them up and maintaining them.

For example, the following file |draft.tex|
with a compilation flag |\version| as described in \secref{sec:flags}
compiles the main document as a draft:
%
\begin{center}
\begin{tabular}{l}
|\def\version{draft}|\\
|\input{childdoc.def}|\\
|\childdocforward{|\textit{main}|}|
\end{tabular}
\end{center}
%
Likewise, the following files |final|\textit{nn}|.tex|
compile the final version of the child document
|child|\textit{nn}|.tex|:
%
\begin{center}
\begin{tabular}{l}
|\def\version{final}|\\
|\input{childdoc.def}|\\
|\childdocforwardprefix{final}{child}|
\end{tabular}
\end{center}
%

Note that when several versions of a main file and/or of each child file
are to be generated, it may be convenient to set up a |Makefile| or
shell script to automatise the process.

%%%%%%%%%%%%%%%%%%%%%%%%%%%%%%%%%%%%%%%%%%%%%%%%%%%%%%%%%%%%%%%%%%%%%%%%%%%%%%%%
\subsection{Command Line Processing}
\label{sec:commandline}

The effect of redirection files can also be achieved by invoking
the \LaTeX{} compiler with a more elaborate command line.
Most conveniently this should be done as part
of a shell script or a |Makefile|.

When using \textsf{childdoc} in the main file, the following
command lines effectively perform a redirection
(note that depending on the shell being used,
backslashes may have to be doubled: `|\|' $\to$ `|\\|'):
%
\begin{center}
|... -jobname "|\textit{target}|" |\\|"|[\textit{flags}]%
|\input{childdoc.def}\childdocforward[|\textit{main}|]{|\textit{dest}|}"|
\end{center}
%
Here \textit{target} is the name of the output file,
\textit{main} is the name of the main file
and \textit{dest} is the name of the main or child file to be processed
(all filenames without extensions).
The optional argument \textit{main} can be omitted
if \textit{main} matches \textit{dest}.
Optionally, compilation \textit{flags} can be defined via |\def| commands.
This command line makes the \TeX{} engine believe
it is compiling the file \textit{target}
whose content is specified as the latter parameter.
The provided code then forwards the processing to
\textit{main} or \textit{dest} as described in \secref{sec:forward}.

%%%%%%%%%%%%%%%%%%%%%%%%%%%%%%%%%%%%%%%%%%%%%%%%%%%%%%%%%%%%%%%%%%%%%%%%%%%%%%%%
\subsection{Include by Input}
\label{sec:input}

Including child documents by |\include| has some restrictions by design.
Most notably, the content of a child document always occupies
its own set of pages; pages cannot be shared between child documents.
Usually, this behaviour makes perfect sense
because each child document contain an essential part of the document.
However, in some situations it may be desirable to compose
a document from a collection of parts
without having mandatory page breaks between then.
For this case, the package
provides a mechanism to include parts
by |\input| which can also be processed individually.
However, by construction this mechanism
requires manual handling of the content to be output.

%%%%%%%%%%%%%%%%%%%%%%%%%%%%%%%%%%%%%%%%
\DescribeMacro{\ifchilddocmanual}
The main file should be prepared as usual, see \secref{sec:include}.
However, the document body must make a distinction
between processing of an individual part and of the main document, e.g.:
%
\begin{center}
\begin{tabular}{l}
|\ifchilddocmanual|\\
|\input{\childdocname}|\\
|\||else|\\
\textit{document body with }|\input{|\textit{part}|}|\\
|\||fi|
\end{tabular}
\end{center}
%
The conditional |\ifchilddocmanual| is true whenever
a part to be included by |\input| is being compiled,
and the name of the part is stored in |\childdocname|.

%%%%%%%%%%%%%%%%%%%%%%%%%%%%%%%%%%%%%%%%
\DescribeMacro{\childdocby}
Each part to be included by |\input| should start with:
%
\begin{center}
\begin{tabular}{l}
|\input{childdoc.def}|\\
|\childdocby{|\textit{main}|}|\\
\end{tabular}
\end{center}
%
The directive |\childdocby| is similar to |\childdocof|
described in \secref{sec:include},
but the subsequent selection of content must be done manually.
To that end, both |\ifchilddoc| and |\ifchilddocmanual|
will be true upon processing of a part,
and the name of the part is stored in |\childdocname|.
Note that |\jobname| will be set to the filename of the current part
so that each part receives an individual |.aux| file
that does not interfere with the |.aux| file(s) of the main document.
This behaviour can be altered by the alternative form
|\childdocby[*]{|\textit{main}|}| (with a non-empty optional argument)
which uses the |.aux| file of the main document
by setting |\jobname| to \textit{main}.

%%%%%%%%%%%%%%%%%%%%%%%%%%%%%%%%%%%%%%%%%%%%%%%%%%%%%%%%%%%%%%%%%%%%%%%%%%%%%%%%
\subsection{Driver Development}
\label{sec:driver}

The \textsf{childdoc} mechanism can also be use for the development
of definition files such as \LaTeX{} styles or classes.
This case differs from the above setup with multiple parts
included by |\include| in that no |\includeonly| should be invoked.
This can be achieved by starting the include file
(before |\ProvidesPackage|) with:
%
\begin{center}
\begin{tabular}{l}
|\input{childdoc.def}|\\
|\childdocforward{|\textit{main}|}|\\
\end{tabular}
\end{center}
%
or alternatively with:
%
\begin{center}
\begin{tabular}{l}
|\input{childdoc.def}|\\
|\childdocby{|\textit{main}|}|\\
\end{tabular}
\end{center}
%
Both forms have slightly different effects as described above.
The main file is prepared as usual, see \secref{sec:include}.

%%%%%%%%%%%%%%%%%%%%%%%%%%%%%%%%%%%%%%%%%%%%%%%%%%%%%%%%%%%%%%%%%%%%%%%%%%%%%%%%
\subsection{Legacy Detection}
\label{sec:detection}

The directive |\childdocmain| in the main file can detect
whether the complete document or merely a child is to be compiled
even without using the directive |\childdocof|.
This method is deprecated because it is less robust
and there is no compelling reason to use it;
it is merely provided for backward compatibility
and it may be removed in future versions.

If the detection mechanism is to be used,
it is mandatory to correctly specify
the filename of the main file as the argument of |\childdocmain|:
%
\begin{center}
\begin{tabular}{l}
|\input{childdoc.def}|\\
|\childdocmain{|\textit{main}|}|\\
\end{tabular}
\end{center}
%
If |\jobname| does not match the argument \textit{main} of |\childdocmain|,
it is assumed that |\jobname| points to the child file to be compiled.
When using |\childdocmain| with the main file specified as argument,
it suffices to start a child file
with just |\input{|\textit{main}|}|
without loading of the package and using |\childdocof|.
If instead all processing is done
with the appropriate \textsf{childdoc} directives,
the argument of \textit{main} of |\childdocmain| can be empty.

An alternative version of the command line processing described
in \secref{sec:commandline} using the detection mechanism reads:
%
\begin{center}
|... -jobname "|\textit{target}|" "|[\textit{flags}]%
[|\def\jobname{|\textit{dest}|}|]|\input{|\textit{main}|}"|
\end{center}

%%%%%%%%%%%%%%%%%%%%%%%%%%%%%%%%%%%%%%%%%%%%%%%%%%%%%%%%%%%%%%%%%%%%%%%%%%%%%%%%
\subsection{Manual Code}
\label{sec:manual}

In case one cannot be certain whether the definitions file |childdoc.def|
is installed on the target \TeX{} distribution
and one prefers not to ship it,
it is conceivable to paste a few relevant commands into the sources.

To that end, drop all statements |\input{childdoc.def}|
and perform the replacements as outlined below.
Instead of |\childdocmain{|\textit{main}|}| add the following code
to the top of the main file:
%
\begin{center}
\begin{tabular}{l}
|\||ifdefined\childdocname\endinput\||fi\newif\ifchilddoc|\\
|\edef\childdocname{\scantokens\expandafter{\jobname\noexpand}}|\\
|\def\childdocmain{|\textit{main}|}\||ifx\childdocmain\childdocname\||else|\\
|\childdoctrue\includeonly{\childdocname}\let\jobname\childdocmain\||fi|\\
\end{tabular}
\end{center}
%
Instead of |\childdocof{|\textit{main}|}| just include the main file
at the top of each child file:
%
\begin{center}
|\input{|\textit{main}|}|
\end{center}
%
A simple redirection |\childdocforward{|\textit{dest}|}| is achieved by:
%
\begin{center}
|\def\jobname{|\textit{dest}|}\input{\jobname}|
\end{center}
%
The redirection with prefix
|\childdocforwardprefix[|\textit{prefix}|]{|\textit{dest}|}|
is accomplished by:
%
\begin{center}
\begin{tabular}{l}
|{\edef\jobname{\scantokens\expandafter{\jobname\noexpand}}|\\
|\def\redirectjob |\textit{prefix}|#1~~~{\gdef\jobname{|\textit{dest}|#1}}|\\
|\expandafter\redirectjob\jobname~~~}\input{\jobname}|
\end{tabular}
\end{center}

In an alternative approach,
child documents can be compiled by a specific command line
without additional code or specific definitions:
%
\begin{center}
|... -jobname "|\textit{target}|" "|[\textit{flags}]%
|\includeonly{|\textit{dest}|}\input{|\textit{main}|}"|
\end{center}
%

%%%%%%%%%%%%%%%%%%%%%%%%%%%%%%%%%%%%%%%%%%%%%%%%%%%%%%%%%%%%%%%%%%%%%%%%%%%%%%%%
%%%%%%%%%%%%%%%%%%%%%%%%%%%%%%%%%%%%%%%%%%%%%%%%%%%%%%%%%%%%%%%%%%%%%%%%%%%%%%%%
\section{Information}

%%%%%%%%%%%%%%%%%%%%%%%%%%%%%%%%%%%%%%%%%%%%%%%%%%%%%%%%%%%%%%%%%%%%%%%%%%%%%%%%
\subsection{Copyright}

Copyright \copyright{} 2017--2018 Niklas Beisert

This work may be distributed and/or modified under the
conditions of the \LaTeX{} Project Public License, either version 1.3
of this license or (at your option) any later version.
The latest version of this license is in
  \url{http://www.latex-project.org/lppl.txt}
and version 1.3 or later is part of all distributions of \LaTeX{}
version 2005/12/01 or later.

This work has the LPPL maintenance status `maintained'.

The Current Maintainer of this work is Niklas Beisert.

This work consists of the files |README.txt|, |childdoc.ins| and |childdoc.dtx|
as well as the derived files |childdoc.def|, |cdocsamp.tex|
with |cdocsch1.tex|, |cdocsch2.tex|, |cdocspt3.tex|, |cdocspt4.tex|,
|cdocsdrf.tex|, |cdocsfn1.tex|, |cdocsfn2.tex|
as well as |childdoc.pdf|.

%%%%%%%%%%%%%%%%%%%%%%%%%%%%%%%%%%%%%%%%%%%%%%%%%%%%%%%%%%%%%%%%%%%%%%%%%%%%%%%%
\subsection{Files and Installation}

The package consists of the files:
%
\begin{center}
\begin{tabular}{ll}
    |README.txt|   & readme file \\
    |childdoc.ins| & installation file \\
    |childdoc.dtx| & source file \\
    |childdoc.def| & definition file \\
    |cdocsamp.tex| & sample main file \\
    |cdocsch1.tex| & sample include file \\
    |cdocsch2.tex| & sample include file \\
    |cdocspt3.tex| & sample part file \\
    |cdocspt4.tex| & sample part file \\
    |cdocsdrf.tex| & sample redirection file \\
    |cdocsfn1.tex| & sample redirection file \\
    |cdocsfn2.tex| & sample redirection file \\
    |childdoc.pdf| & manual
\end{tabular}
\end{center}
%
The distribution consists of the files
|README.txt|, |childdoc.ins| and |childdoc.dtx|.
%
\begin{itemize}
\item
Run (pdf)\LaTeX{} on |childdoc.dtx|
to compile the manual |childdoc.pdf| (this file).
\item
Run \LaTeX{} on |childdoc.ins| to create the definitions file |childdoc.def|
and the sample |cdocsamp.tex| with include files
|cdocsch1.tex|, |cdocsch2.tex|, |cdocspt3.tex|, |cdocspt4.tex|,
|cdocsdrf.tex|, |cdocsfn1.tex|, |cdocsfn2.tex|.
Then copy the file |childdoc.def| to an appropriate directory of your \LaTeX{}
distribution, e.g.\ \textit{texmf-root}|/tex/latex/childdoc|.
\end{itemize}

%%%%%%%%%%%%%%%%%%%%%%%%%%%%%%%%%%%%%%%%%%%%%%%%%%%%%%%%%%%%%%%%%%%%%%%%%%%%%%%%
\subsection{Related CTAN Packages}

There are several other packages which offer a similar functionality:
%
\begin{itemize}
\item
The packages
\href{http://ctan.org/pkg/docmute}{\textsf{docmute}},
\href{http://ctan.org/pkg/includex}{\textsf{includex}} and
\href{http://ctan.org/pkg/standalone}{\textsf{standalone}}
provide commands to include only the document body of
a child file thus allowing both files to be compiled individually.
\item
The packages \href{http://ctan.org/pkg/subdocs}{\textsf{subdocs}}
and \href{http://ctan.org/pkg/subfiles}{\textsf{subfiles}}
provide structures in which the main and child documents can be
encapsulated and allowing them to be compiled individually.
The inclusion mechanism is different from the conventional |\include|.
\item
The package \href{http://ctan.org/pkg/combine}{\textsf{combine}}
is an elaborate solution to combine several documents into one.
\end{itemize}
%
See also the CTAN topic \href{http://ctan.org/topic/subdocs}{\textsf{subdocs}}
for further related packages.
The present package differs from the above solutions in that
a document structure constructed with the conventional |\include| mechanism
just needs two extra commands at the top of every file
such that all constituent files can be compiled individually.

%%%%%%%%%%%%%%%%%%%%%%%%%%%%%%%%%%%%%%%%%%%%%%%%%%%%%%%%%%%%%%%%%%%%%%%%%%%%%%%%
%\subsection{Feature Suggestions}
%
%The following is a list of features which may be useful for future
%versions of this package:
%%
%\begin{itemize}
%\item
%\ldots
%\end{itemize}

%%%%%%%%%%%%%%%%%%%%%%%%%%%%%%%%%%%%%%%%%%%%%%%%%%%%%%%%%%%%%%%%%%%%%%%%%%%%%%%%
\subsection{Revision History}

%%%%%%%%%%%%%%%%%%%%%%%%%%%%%%%%%%%%%%%%
\paragraph{v2.0:} 2018/12/30

\begin{itemize}
\item
immediate forward processing
\item
added |\childdocby| mechanism
\item
manual restructured
\end{itemize}

%%%%%%%%%%%%%%%%%%%%%%%%%%%%%%%%%%%%%%%%
\paragraph{v1.6:} 2018/01/17

\begin{itemize}
\item
application for development of include files
\item
corrections to manual
\end{itemize}

%%%%%%%%%%%%%%%%%%%%%%%%%%%%%%%%%%%%%%%%
\paragraph{v1.5:} 2017/05/21

\begin{itemize}
\item
more complete structuring introduced
\item
|\childdocof| introduced
\item
|\childdoc| renamed to |\childdocmain|
\item
|\childredirect| renamed to |\childdocforward| and |\childdocforwardprefix|
and functionality expanded
\end{itemize}

%%%%%%%%%%%%%%%%%%%%%%%%%%%%%%%%%%%%%%%%
\paragraph{v1.0:} 2017/04/27

\begin{itemize}
\item
manual and install package
\item
first version published on CTAN
\end{itemize}

%%%%%%%%%%%%%%%%%%%%%%%%%%%%%%%%%%%%%%%%
\paragraph{v0.6:} 2017/04/26

\begin{itemize}
\item
redirection mechanism added
\end{itemize}

%%%%%%%%%%%%%%%%%%%%%%%%%%%%%%%%%%%%%%%%
\paragraph{v0.5:} 2017/04/26

\begin{itemize}
\item
functionality in definition file
\end{itemize}


%%%%%%%%%%%%%%%%%%%%%%%%%%%%%%%%%%%%%%%%%%%%%%%%%%%%%%%%%%%%%%%%%%%%%%%%%%%%%%%%
%%%%%%%%%%%%%%%%%%%%%%%%%%%%%%%%%%%%%%%%%%%%%%%%%%%%%%%%%%%%%%%%%%%%%%%%%%%%%%%%
%%%%%%%%%%%%%%%%%%%%%%%%%%%%%%%%%%%%%%%%%%%%%%%%%%%%%%%%%%%%%%%%%%%%%%%%%%%%%%%%
\appendix

\settowidth\MacroIndent{\rmfamily\scriptsize 000\ }

 \DocInput{childdoc.dtx}

\end{document}
%</driver>
% \fi
%
% %%%%%%%%%%%%%%%%%%%%%%%%%%%%%%%%%%%%%%%%%%%%%%%%%%%%%%%%%%%%%%%%%%%%%%%%%%%%%%
% %%%%%%%%%%%%%%%%%%%%%%%%%%%%%%%%%%%%%%%%%%%%%%%%%%%%%%%%%%%%%%%%%%%%%%%%%%%%%%
% \section{Sample}
%\iffalse
%<*samplemain>
%\fi
%
% The following presents a sample document
% with two chapters, two parts, a title page,
% a compile flag as well as three forwarding files to set the flag.
% It consists of eight |.tex| files:
% \begin{center}
% \begin{tabular}{ll}
% |cdocsamp.tex|&main file\\
% |cdocsch1.tex|&include file for chapter 1\\
% |cdocsch2.tex|&include file for chapter 2\\
% |cdocspt3.tex|&include file for part 3\\
% |cdocspt4.tex|&include file for part 4\\
% |cdocsdrf.tex|&forwarding file for main file in draft mode\\
% |cdocsfi1.tex|&forwarding file for final version of chapter 1\\
% |cdocsfi2.tex|&forwarding file for final version of chapter 2\\
% \end{tabular}
% \end{center}
% Each of the eight files can be compiled directly by the \LaTeX{} compiler.
%
% %%%%%%%%%%%%%%%%%%%%%%%%%%%%%%%%%%%%%%
% \paragraph{Main File.}
%
% The main file is called |cdocsamp.tex|.
%
% Load the \textsf{childdoc} definitions and
% declare the filename for the main document:
%    \begin{macrocode}
\input{childdoc.def}
\childdocmain{}
%    \end{macrocode}

% Optional override for |\version| flag:
%    \begin{macrocode}
%%\ifchilddoc\else\providecommand{\version}{draft}\fi
%    \end{macrocode}

% Define the default values for the |\version| flag
% (|final| for the main file and |draft| for childs):
%    \begin{macrocode}
\ifchilddoc
\providecommand{\version}{draft}
\else
\providecommand{\version}{final}
\fi
%    \end{macrocode}

% Load the standard document class:
%    \begin{macrocode}
\documentclass[12pt]{article}
%    \end{macrocode}

% Start the document body:
%    \begin{macrocode}
\begin{document}
%    \end{macrocode}

% Declare a title page.
% Print title, part of document being processed and version flag:
%    \begin{macrocode}
\addtocounter{page}{-1}
\begin{center}
{\LARGE\bfseries{}childdoc example\par}
\vspace{1cm}
\ifchilddoc
\ifchilddocmanual part\else chapter\fi:
`\childdocname' of `\childdocjob'\par
\else
main document: `\childdocjob'\par
\fi
version: \version\par
\end{center}
\newpage
%    \end{macrocode}

% Manually include selected file,
% otherwise process as usual:
%    \begin{macrocode}
\ifchilddocmanual
\section*{part `\childdocname'}
\input{\childdocname}
\else
%    \end{macrocode}

% Include the two chapters:
%    \begin{macrocode}
\include{cdocsch1}
\include{cdocsch2}
%    \end{macrocode}

% Include the two parts unless only chapters should be displayed:
%    \begin{macrocode}
\ifchilddoc\else
\section{part three}
\input{cdocspt3}
\section{part four}
\input{cdocspt4}
\fi
%    \end{macrocode}

% Process as usual until here:
%    \begin{macrocode}
\fi
%    \end{macrocode}

% End of document body:
%    \begin{macrocode}
\end{document}
%    \end{macrocode}
%\iffalse
%</samplemain>
%\fi
%
% %%%%%%%%%%%%%%%%%%%%%%%%%%%%%%%%%%%%%%
% \paragraph{Chapter Include Files.}
%
% The include files are called |cdocsch1.tex| and |cdocsch2.tex|.
%
%\iffalse
%<*samplechap1|samplechap2>
%\fi

% Optional override for |\version| flag:
%    \begin{macrocode}
%%\providecommand{\version}{final}
%    \end{macrocode}

% Include the main document:
%    \begin{macrocode}
\input{childdoc.def}
\childdocof{cdocsamp}
%    \end{macrocode}

%\iffalse
%</samplechap1|samplechap2>
%\fi
%
%\iffalse
%<*samplechap1>
%\fi
% Some text for chapter 1:
%    \begin{macrocode}
\section{one}
some text in chapter one
%    \end{macrocode}

%\iffalse
%</samplechap1>
%\fi
% Some text for chapter 2:
%\iffalse
%<*samplechap2>
%\fi
%    \begin{macrocode}
\section{two}
more text in chapter two
%    \end{macrocode}

%\iffalse
%</samplechap2>
%\fi
%
% %%%%%%%%%%%%%%%%%%%%%%%%%%%%%%%%%%%%%%
% \paragraph{Part Include Files.}
%
% The include files are called |cdocspt3.tex| and |cdocspt4.tex|.
%
%\iffalse
%<*samplepart3|samplepart4>
%\fi

% Optional override for |\version| flag:
%    \begin{macrocode}
%%\providecommand{\version}{final}
%    \end{macrocode}

% Include the main document:
%    \begin{macrocode}
\input{childdoc.def}
\childdocby{cdocsamp}
%    \end{macrocode}

%\iffalse
%</samplepart3|samplepart4>
%\fi
%
%\iffalse
%<*samplepart3>
%\fi
% Some text for part 3:
%    \begin{macrocode}
some text in part three
%    \end{macrocode}

%\iffalse
%</samplepart3>
%\fi
% Some text for part 4:
%\iffalse
%<*samplepart4>
%\fi
%    \begin{macrocode}
more text in part four
%    \end{macrocode}

%\iffalse
%</samplepart4>
%\fi
%
% %%%%%%%%%%%%%%%%%%%%%%%%%%%%%%%%%%%%%%
% \paragraph{Forwarding for a Complete Draft.}
%
% The following forwarding file |cdocsdrf.tex|
% compiles the main document in draft mode:
%\iffalse
%<*sampledraft>
%\fi
%    \begin{macrocode}
\def\version{draft}
\input{childdoc.def}
\childdocforward{cdocsamp}
%    \end{macrocode}

%\iffalse
%</sampledraft>
%\fi
%
% %%%%%%%%%%%%%%%%%%%%%%%%%%%%%%%%%%%%%%
% \paragraph{Forwarding for Final Version of the Chapters.}
%
% The following forwarding files |cdocsfn1.tex| and |cdocsfn2.tex|
% (with identical content)
% compile the final versions of the child documents
% |cdocsch1.tex| and |cdocsch2.tex|, respectively:
%\iffalse
%<*samplefinal>
%\fi
%    \begin{macrocode}
\def\version{final}
\input{childdoc.def}
\childdocforwardprefix[cdocsamp]{cdocsfn}{cdocsch}
%    \end{macrocode}

%\iffalse
%</samplefinal>
%\fi
%
% %%%%%%%%%%%%%%%%%%%%%%%%%%%%%%%%%%%%%%
% \paragraph{Command Line Processing.}
%
% The following three command lines generate the output files
% |cdocscld|, |cdocscl1| and |cdocscl2|
% which should be identical to
% |cdocsdrf|, |cdocsch1| and |cdocsfn2|, respectively:
% \begin{center}
% \begin{tabular}{l}
% |latex -jobname cdocscld \|\\
% |  "\def\version{draft}\input{childdoc.def}\childdocforward{cdocsamp}"|\\
% |latex -jobname cdocscl1 \|\\
% |  "\input{childdoc.def}\childdocforward[cdocsamp]{cdocsch1}"|\\
% |latex -jobname cdocscl2 \|\\
% |  "\def\version{final}\input{childdoc.def}\childdocforward{cdocsch2}"|
% \end{tabular}
% \end{center}
% Note that the trailing backslash on each first line
% merely continues the input to the second line
% (for convenient cut ant paste).
% Furthermore, the command |latex| can be replaced by any
% of its alternative versions such as |pdflatex|.
%
% %%%%%%%%%%%%%%%%%%%%%%%%%%%%%%%%%%%%%%%%%%%%%%%%%%%%%%%%%%%%%%%%%%%%%%%%%%%%%%
% %%%%%%%%%%%%%%%%%%%%%%%%%%%%%%%%%%%%%%%%%%%%%%%%%%%%%%%%%%%%%%%%%%%%%%%%%%%%%%
% \section{Implementation}
%\iffalse
%<*package>
%\fi
%
% This section describes the definitions file |childdoc.def|.

% The definitions cannot be loaded using |\usepackage| or |\RequirePackage|
% which has a mechanism to prevent loading a style file more than once.
% When loading the definitions by means of |\input|
% multiple instances have to be prevented manually:
%\iffalse
%This code needs to be before the `\ProvidesFile' directive
%which is defined at the beginning of this file.
%Therefore it is also placed there and commented out here.
%</package>
%<*discard>
%\fi
%    \begin{macrocode}
\ifdefined\childdocmain\endinput\fi
%    \end{macrocode}
%\iffalse
%</discard>
%<*package>
%\fi
%
% \macro{\ifchilddoc}
% \macro{\ifchilddocmanual}
% The conditional |\ifchilddoc| tells whether a
% child (true) or main (false) document is being compiled.
% The conditional |\ifchilddocmanual| tells whether
% the |\includeonly| mechanism is used (false) or
% the selection of child files must be performed manually (true).
% The definitions initialise to false:
%    \begin{macrocode}
\newif\ifchilddoc
\newif\ifchilddocmanual
%    \end{macrocode}

% \macro{\childdocname}
% \macro{\childdocjob}
% The macro |\childdocname| stores the name of the main document
% to be compiled. The macro |\childdocjob| stores the name of
% the document on which the \LaTeX{} compiler was originally invoked.
% The content of |\jobname| cannot be compared
% to filenames specified in the source due to different catcodes.
% The following code rescans |\jobname|, stores the result
% in |\childdocname| and saves a copy in |\childdocjob|:
%    \begin{macrocode}
\edef\childdocname{\scantokens\expandafter{\jobname\noexpand}}
\let\childdocjob\childdocname
%    \end{macrocode}

% \macro{\childdocdisable}
% The macro |\childdocdisable| prevents the main file
% from being processed more than once.
% At this stage, the main document command |\childdocmain|
% is assumed to be called once again where it should do nothing.
% Any subsequent call to it should prevent
% a secondary processing of the main document
% It overwrites the forwarding commands
% |\childdocof| and |\childdocforward|
% with empty macros to prevent further inclusions of the main document:
%    \begin{macrocode}
\newcommand{\childdocdisable}
{
  \renewcommand{\childdocmain}[1]{\renewcommand{\childdocmain}[1]{\endinput}}
  \renewcommand{\childdocof}[1]{}
  \renewcommand{\childdocby}[2][]{}
  \renewcommand{\childdocforward}[2][]{}
  \renewcommand{\childdocdisable}{}
}
%    \end{macrocode}

% \macro{\childdocmain}
% The macro |\childdocmain| is to be called at the top of the main file
% with nothing or the main filename (without extension) as argument.
% First, it breaks loops.
% If the argument is not empty and does not match |\childdocname|
% (which is set by the first inclusion of |childdoc.def|),
% |\ifchilddoc| is set to true, |\includeonly| is applied to the child file
% and |\jobname| is set to the main file
% (for proper handling of |.aux| files):
%    \begin{macrocode}
\newcommand{\childdocmain}[1]
{
  \childdocdisable\childdocmain{}
  \if?#1?\else
    \begingroup
      \def\childdoctmp{#1}
      \ifx\childdoctmp\childdocname
        \def\childdoctmp{}
      \else
        \def\childdoctmp
        {
          \childdoctrue
          \includeonly{\childdocname}
          \def\childdocjob{#1}
          \def\jobname{#1}
        }
      \fi
      \expandafter
    \endgroup
    \childdoctmp
  \fi
}
%    \end{macrocode}

% \macro{\childdocof}
% The command |\childdocof| redirects
% compilation to the main file |#1|.
%    \begin{macrocode}
\newcommand{\childdocof}[1]
{
  \childdocdisable
  \childdoctrue
  \includeonly{\childdocname}
  \def\jobname{#1}
  \def\childdocjob{#1}
  \input{#1}
}
%    \end{macrocode}

% \macro{\childdocby}
% The command |\childdocby| ....
%    \begin{macrocode}
\newcommand{\childdocby}[2][]
{
  \childdocdisable
  \childdoctrue
  \childdocmanualtrue
  \if?#1?\else
    \def\jobname{#2}
  \fi
  \def\childdocjob{#2}
  \input{#2}
  \endinput
}
%    \end{macrocode}

% \macro{\childdocforward}
% The command |\childdocforward| redirects
% compilation to the main file or
% (if the optional argument is given) a child file.
% Parameters are set as if the main file
% or a child file starting with |\childdocof| was compiled.
% Then compilation is handed over to the main file:
%    \begin{macrocode}
\newcommand{\childdocforward}[2][]
{
  \begingroup
    \if?#1?
      \def\childdoctmp
      {
        \def\childdocname{#2}
        \def\childdocjob{#2}
        \def\jobname{#2}
        \input{#2}
        \endinput
      }
    \else
      \def\childdoctmp
      {
        \childdocdisable
        \def\childdocname{#2}
        \childdoctrue
        \includeonly{#2}
        \def\childdocjob{#1}
        \def\jobname{#1}
        \input{#1}
        \endinput
      }
    \fi
    \expandafter
  \endgroup
  \childdoctmp
}
%    \end{macrocode}

% \macro{\childdocforwardprefix}
% The command |\childdocforwardprefix| redirects
% compilation to the main or a child file by means of a pattern.
% The prefix |#1| in the current filename is replaced by |#2|
% and the suffix of the current filename is kept
% (it is assumed that the filename does not contain the substring `|~~~|'
% which is used as a delimiter).
% Compilation is handed over to the new file by |\childdocforward|:
%    \begin{macrocode}
\newcommand{\childdocforwardprefix}[3][]
{
  \begingroup
    \def\childdocextract #2##1~~~{\def\childdoctmp{\childdocforward[#1]{#3##1}}}
    \expandafter\childdocextract\childdocname~~~
    \expandafter
  \endgroup
  \childdoctmp
}
%    \end{macrocode}

% \macro{\childdoc}
% The deprecated macro |\childdoc| is a legacy version of |\childdocmain|:
%    \begin{macrocode}
\newcommand{\childdoc}{\childdocmain}
%    \end{macrocode}

% \macro{\childdocredirect}
% The deprecated macro |\childdocredirect| is a legacy version
% of |\childdocforward| and |\childdocforwardprefix|:
%    \begin{macrocode}
\newcommand{\childdocredirect}[2][]
{
  \begingroup
    \if?#1?
      \def\childdoctmp{\childdocforward{#2}}
    \else
      \def\childdoctmp{\childdocforwardprefix{#1}{#2}}
    \fi
    \expandafter
  \endgroup
  \childdoctmp
}
%    \end{macrocode}

%\iffalse
%</package>
%\fi
%
\endinput
|\\
|\childdocby{|\textit{main}|}|\\
\end{tabular}
\end{center}
%
Both forms have slightly different effects as described above.
The main file is prepared as usual, see \secref{sec:include}.

%%%%%%%%%%%%%%%%%%%%%%%%%%%%%%%%%%%%%%%%%%%%%%%%%%%%%%%%%%%%%%%%%%%%%%%%%%%%%%%%
\subsection{Legacy Detection}
\label{sec:detection}

The directive |\childdocmain| in the main file can detect
whether the complete document or merely a child is to be compiled
even without using the directive |\childdocof|.
This method is deprecated because it is less robust
and there is no compelling reason to use it;
it is merely provided for backward compatibility
and it may be removed in future versions.

If the detection mechanism is to be used,
it is mandatory to correctly specify
the filename of the main file as the argument of |\childdocmain|:
%
\begin{center}
\begin{tabular}{l}
|% \iffalse
%
% childdoc.dtx Copyright (C) 2017-2018 Niklas Beisert
%
% This work may be distributed and/or modified under the
% conditions of the LaTeX Project Public License, either version 1.3
% of this license or (at your option) any later version.
% The latest version of this license is in
%   http://www.latex-project.org/lppl.txt
% and version 1.3 or later is part of all distributions of LaTeX
% version 2005/12/01 or later.
%
% This work has the LPPL maintenance status `maintained'.
%
% The Current Maintainer of this work is Niklas Beisert.
%
% This work consists of the files childdoc.dtx and childdoc.ins
% and the derived files childdoc.def and cdocsamp.tex with
% cdocsch1.tex, cdocsch2.tex, cdocsdrf.tex, cdocsfn1.tex, cdocsfn2.tex.
%
%<package>\ifdefined\childdocmain\endinput\fi
%<package>\ProvidesFile{childdoc.def}[2018/12/30 v2.0 child document driver]
%<samplemain>\ProvidesFile{cdocsamp.tex}[2018/12/30 v2.0 sample for childdoc]
%<*driver>
%\ProvidesFile{childdoc.drv}[2018/12/30 v2.0 childdoc reference manual file]
\PassOptionsToClass{10pt,a4paper}{article}
\documentclass{ltxdoc}

\usepackage[margin=35mm]{geometry}
\usepackage{hyperref}
\usepackage{hyperxmp}
\usepackage[usenames]{color}

\hypersetup{colorlinks=true}
\hypersetup{pdfstartview=FitH}
\hypersetup{pdfpagemode=UseNone}
\hypersetup{pdfsource={}}
\hypersetup{pdflang={en-UK}}
\hypersetup{pdfcopyright={Copyright 2017-2018 Niklas Beisert.
  This work may be distributed and/or modified under the
  conditions of the LaTeX Project Public License, either version 1.3
  of this license or (at your option) any later version.}}
\hypersetup{pdflicenseurl={http://www.latex-project.org/lppl.txt}}
\hypersetup{pdfcontactaddress={ETH Zurich, ITP, HIT K,
  Wolfgang-Pauli-Strasse 27}}
\hypersetup{pdfcontactpostcode={8093}}
\hypersetup{pdfcontactcity={Zurich}}
\hypersetup{pdfcontactcountry={Switzerland}}
\hypersetup{pdfcontactemail={nbeisert@itp.phys.ethz.ch}}
\hypersetup{pdfcontacturl={http://people.phys.ethz.ch/\xmptilde nbeisert/}}

\newcommand{\secref}[1]{\hyperref[#1]{section \ref*{#1}}}

\parskip1ex
\parindent0pt
\let\olditemize\itemize
\def\itemize{\olditemize\parskip0pt}

\begin{document}

\title{The \textsf{childdoc} Package}
\hypersetup{pdftitle={The childdoc Package}}
\author{Niklas Beisert\\[2ex]
  Institut f\"ur Theoretische Physik\\
  Eidgen\"ossische Technische Hochschule Z\"urich\\
  Wolfgang-Pauli-Strasse 27, 8093 Z\"urich, Switzerland\\[1ex]
  \href{mailto:nbeisert@itp.phys.ethz.ch}
  {\texttt{nbeisert@itp.phys.ethz.ch}}}
\hypersetup{pdfauthor={Niklas Beisert}}
\hypersetup{pdfsubject={Manual for the LaTeX2e Package childdoc}}
\date{30 December 2018, \textsf{v2.0}}
\maketitle

\begin{abstract}\noindent
\textsf{childdoc} is a \LaTeXe{} package
that enables the direct compilation
of document sections included by |\include|
to individual files.
\end{abstract}

\begingroup
\parskip0ex
\tableofcontents
\endgroup

%%%%%%%%%%%%%%%%%%%%%%%%%%%%%%%%%%%%%%%%%%%%%%%%%%%%%%%%%%%%%%%%%%%%%%%%%%%%%%%%
%%%%%%%%%%%%%%%%%%%%%%%%%%%%%%%%%%%%%%%%%%%%%%%%%%%%%%%%%%%%%%%%%%%%%%%%%%%%%%%%
\section{Introduction}

\LaTeX{} provides a mechanism to structure a large document (such as a book)
into a main file and several child files (containing the chapters)
using the |\include| command.
This mechanism is beneficial for documents
which span hundreds of pages in order to
make the source file(s) more manageable.
Moreover, compilation can be restricted to
selected child files by means of the |\includeonly| command.
The latter feature can be used to reduce the compilation time while editing
(this was significantly more useful in the earlier days of \LaTeX{})
or to generate a smaller document which is easier to navigate.
Another application of |\includeonly| is to generate
documents consisting of selected parts of the complete document.

However, there are a few drawbacks of the plain |\include| mechanism:
\begin{itemize}
\item
The child files cannot be compiled on their own,
they can only be compiled via the main file.
A naive editing environment
(such as a text editor with an option
to have the current file processed by \LaTeX)
may require one to switch to the main file before compiling;
attempting to compile the child file produces errors.
\item
The main file must be modified (each time)
to adjust the |\includeonly| command
to the present needs. This easily leaves the main file in a messy state.
\item
The generated document will always carry the filename
of the main document. This is inconvenient if
several child files are to be compiled and
to be kept for distribution.
\end{itemize}

The present package provides a simple interface
to make child files individually compilable by \LaTeX{}.
Compiling a child file then has the same effect as compiling
the main file with an |\includeonly| command
to select the appropriate child.
Moreover the generated document will carry the name of the child
rather than the main file.
This resolves all three above issues.

This feature is meant to make the editing of books,
thesis documents and lecture notes somewhat more convenient.
However, the package can also be used efficiently for
composing a series of documents (such as exercise sheets)
which are typically distributed individually.
It then assists the author in generating the individual documents
(potentially in different versions)
as well as a document containing the collected series.
Another application is in developing style files
or other kinds of included material
where compilation of the style file could redirect
to a sample or test file.

%%%%%%%%%%%%%%%%%%%%%%%%%%%%%%%%%%%%%%%%%%%%%%%%%%%%%%%%%%%%%%%%%%%%%%%%%%%%%%%%
%%%%%%%%%%%%%%%%%%%%%%%%%%%%%%%%%%%%%%%%%%%%%%%%%%%%%%%%%%%%%%%%%%%%%%%%%%%%%%%%
\section{Usage}

First of all, the package \textsf{childdoc} is \emph{not} a standard
\LaTeXe{} |.sty| style file! Therefore it needs to be invoked in
a non-standard way.

%%%%%%%%%%%%%%%%%%%%%%%%%%%%%%%%%%%%%%%%%%%%%%%%%%%%%%%%%%%%%%%%%%%%%%%%%%%%%%%%
\subsection{Included Files}
\label{sec:include}

%%%%%%%%%%%%%%%%%%%%%%%%%%%%%%%%%%%%%%%%
\DescribeMacro{\childdocmain}
To use the package, add the commands
\begin{center}
\begin{tabular}{l}
|\input{childdoc.def}|\\
|\childdocmain{}|\\
\end{tabular}
\end{center}
at the very top of the main \LaTeX{} file,
in particular \emph{before} the |\documentclass| statement!
The argument of |\childdocmain| should be left empty
(but it must be present).

%%%%%%%%%%%%%%%%%%%%%%%%%%%%%%%%%%%%%%%%
\DescribeMacro{\childdocof}
Furthermore, add the commands
\begin{center}
\begin{tabular}{l}
|\input{childdoc.def}|\\
|\childdocof{|\textit{main}|}|\\
\end{tabular}
\end{center}
at the top of every child file \textit{child}
which is included by |\include{|\textit{child}|}|
from within the main file
(or at least for those files to be compiled individually).
The argument \textit{main} must be the filename of the main file.

There are a couple of
considerations in setting up the main and child documents:

%%%%%%%%%%%%%%%%%%%%%%%%%%%%%%%%%%%%%%%%
\paragraph{Restrictions.}

Please note the following restrictions:
\begin{itemize}
\item
|\childdocmain| must be called with one argument \textit{main}
to ensure compatibility with earlier version of the package.
It must either be empty (|\childdocmain{}|)
or precisely match the filename of the main file in which it is specified.
See \secref{sec:detection} for further information.
\item
The filename \textit{main} must be specified without the |.tex| extension.
\item
The filename \textit{main} is case sensitive
(even in case-insensitive file systems)
due to internal string comparison.
\item
The argument \textit{main} should be fully expanded, it cannot be a macro.
\item
Subdirectories and special characters should be avoided in filenames.
\item
The command |\childdocmain{|\textit{main}|}| must be followed by a whitespace.
It should not be followed immediately by another command
or by a comment mark `|%|'.
This is because the \TeX{} parser reads the token immediately following
the argument of |\childdocmain| and puts it
at the beginning of every child section;
however, a white\-space is ignored.
\end{itemize}

%%%%%%%%%%%%%%%%%%%%%%%%%%%%%%%%%%%%%%%%
\paragraph{Content of Main File.}

It is advisable to place all content in the child files included by |\include|.
Any output contained in the main file will appear in all child documents
unless suppressed manually;
it cannot be suppressed automatically by the |\includeonly| directive
and thus should normally be avoided.
A method to include some content in the main file
by means of conditional processing is described in \secref{sec:conditional}.

%%%%%%%%%%%%%%%%%%%%%%%%%%%%%%%%%%%%%%%%
\paragraph{Page Numbering.}

When only a part of the document is compiled,
the appropriate numbering of pages
(as well as other status parameters)
is determined from the |.aux| files.
The latter contain information from previous passes.
However this information needs to propagate through
all intermediate child documents.
Therefore the page numbering in child documents may well
be inconsistent until the complete document is compiled at least once.

A useful (if unconventional) way to always ensure a consistent
page numbering is to restart the numbering in each child document
and denote the pages by `\textit{child}|.|\textit{page}'
where \textit{child} represents the chapter/section number of the child file.
This can be achieved by the command
|\numberwithin{page}{|\textit{child}|}|
of the \textsf{amsmath} package
where \textit{child} can be |chapter| or |section|
depending on the chosen structuring.
Alternatively, one can modify the macro |\thepage| appropriately
and reset the counter |page| at the start of each child file.

%%%%%%%%%%%%%%%%%%%%%%%%%%%%%%%%%%%%%%%%%%%%%%%%%%%%%%%%%%%%%%%%%%%%%%%%%%%%%%%%
\subsection{Conditional Processing}
\label{sec:conditional}

The package provides a mechanism to compile different versions
of a document. To customise the versions further some conditional processing
can come in handy to distinguish which version is being compiled.
The package provides two macros to describe the compilation context:

%%%%%%%%%%%%%%%%%%%%%%%%%%%%%%%%%%%%%%%%
\DescribeMacro{\ifchilddoc}
The conditional |\ifchilddoc| distinguishes between the compilation of
child documents and the main document:
%
\begin{center}
|\ifchilddoc |\textit{child-code}| |[|\||else |\textit{main-code}]| \||fi|
\end{center}

%%%%%%%%%%%%%%%%%%%%%%%%%%%%%%%%%%%%%%%%
\DescribeMacro{\childdocname}
\DescribeMacro{\childdocjob}
The macro |\childdocname| contains the filename (without extension)
of the main or child file being processed.
Note that |\childdocjob| will always contain the name of the main file.

%%%%%%%%%%%%%%%%%%%%%%%%%%%%%%%%%%%%%%%%
\paragraph{Title Page.}

Conditional processing can be used to include a title or banner page
in the main document when proper precautions are taken.
Importantly, the code in the main file should ensure that the page counter
(as well as other status parameters which are stored in the |.aux| files)
takes the same value after the conditional processing.
Otherwise the page numbers may take divergent values
depending on which part is compiled.

For example, a title page could be declared by:
%
\begin{center}
\begin{tabular}{l}
|\ifchilddoc\||else|\\
|\addtocounter{page}{-1}|\\
\textit{code for title page}\\
|\newpage|\\
|\||fi|
\end{tabular}
\end{center}
%
A banner page for the child documents can be generated by:
%
\begin{center}
\begin{tabular}{l}
|\ifchilddoc|\\
|\addtocounter{page}{-1}|\\
\textit{code for banner page}\\
|\newpage|\\
|\||fi|
\end{tabular}
\end{center}
%
Here one could write a message such as:
\begin{center}
|This is the part \childdocname{} of \childdocjob{}.|
\end{center}

%%%%%%%%%%%%%%%%%%%%%%%%%%%%%%%%%%%%%%%%%%%%%%%%%%%%%%%%%%%%%%%%%%%%%%%%%%%%%%%%
\subsection{Flags}
\label{sec:flags}

The package makes it easy to generate different versions
of the main or child documents.
To this end compilation flags can be defined
and assigned different default values.
They will be particularly useful in conjunction
with the forwarding mechanism described in \secref{sec:forward}.

For example, it may be useful to have a flag |\version|
which can be set to |draft| or |final|.
The document source will contain some conditional code
depending on the value of |\version|.
Suppose further, the flag should default to |final| for the main file
and to |draft| for child files
which is a natural assignment for editing the document.
This is achieved by placing the following code
in the preamble of the main document
(below the |\childdocmain| directive):
%
\begin{center}
\begin{tabular}{l}
|\ifchilddoc|\\
|\providecommand{\version}{draft}|\\
|\||else|\\
|\providecommand{\version}{final}|\\
|\||fi|
\end{tabular}
\end{center}
%
The definition by |\providecommand| makes sure
that previous definitions are not overwritten.
Further statements |\providecommand{\version}{...}|
can thus be added before the above code to override it.

For the main file, one might add a line
(between |\childdocmain| and the above block)
%
\begin{center}
|%\ifchilddoc\||else\providecommand{\version}{draft}\||fi|
\end{center}
%
which can be uncommented to produce a draft version.
Likewise one can add a line to the very top of a child file
(above the |\childdocof{|\textit{main}|}| directive)
%
\begin{center}
|%\providecommand{\version}{final}|
\end{center}
%
which can be uncommented to produce the final version of this child document.

%%%%%%%%%%%%%%%%%%%%%%%%%%%%%%%%%%%%%%%%%%%%%%%%%%%%%%%%%%%%%%%%%%%%%%%%%%%%%%%%
\subsection{Forwarding}
\label{sec:forward}

Different versions of the main or child documents
using compilation flags as described in \secref{sec:flags}
can be (permanently) stored in different files
for convenient compilation, viewing and distribution.
To this end, the package defines a command
to pass on compilation to a different file:

%%%%%%%%%%%%%%%%%%%%%%%%%%%%%%%%%%%%%%%%
\DescribeMacro{\childdocforward}
The command |\childdocforward| redirects processing to
another source file:
%
\begin{center}
\begin{tabular}{l}
|\input{childdoc.def}|\\
|\childdocforward[|\textit{main}|]{|\textit{dest}|}|\\
\end{tabular}
\end{center}
%
The argument \textit{dest} is the destination file
(without extension).
It should be the main file or one of the child files.
Note that further \textsf{childdoc} directives
such as |\childdocof| and |\childdocforward|
in the indicated file will be processed in this form.
The optional argument \textit{main}
passes on directly to the main file \textit{main}
while pretending to compile the child \textit{dest}.
This form behaves as if \textit{dest}
issues |\childdocof{|\textit{main}|}| right away,
and no further \textsf{childdoc} directives will be processed.

%%%%%%%%%%%%%%%%%%%%%%%%%%%%%%%%%%%%%%%%
\DescribeMacro{\...prefix}
In the alternative form |\childdocforwardprefix|,
%
\begin{center}
\begin{tabular}{l}
|\input{childdoc.def}|\\
|\childdocforwardprefix[|\textit{main}|]{|\textit{prefix}|}{|\textit{dest}|}|
\end{tabular}
\end{center}
%
the destination file is determined by a pattern
depending on the current file:
To make this work, the current file must be called
`{\textit{prefix}\hspace{0.2em}\textit{suffix}}'
with \textit{prefix} matching precisely the argument.
Processing is then passed on to the file
`{\textit{dest}\hspace{0.2em}\textit{suffix}}'.
Surely, the same effect is achieved by
directly specifying the
argument `{\textit{dest}\hspace{0.2em}\textit{suffix}}'
in the first form.
However, that requires to set up a different file
for each child. With the alternative form of the command
all these files can have exactly the same content
which simplifies setting them up and maintaining them.

For example, the following file |draft.tex|
with a compilation flag |\version| as described in \secref{sec:flags}
compiles the main document as a draft:
%
\begin{center}
\begin{tabular}{l}
|\def\version{draft}|\\
|\input{childdoc.def}|\\
|\childdocforward{|\textit{main}|}|
\end{tabular}
\end{center}
%
Likewise, the following files |final|\textit{nn}|.tex|
compile the final version of the child document
|child|\textit{nn}|.tex|:
%
\begin{center}
\begin{tabular}{l}
|\def\version{final}|\\
|\input{childdoc.def}|\\
|\childdocforwardprefix{final}{child}|
\end{tabular}
\end{center}
%

Note that when several versions of a main file and/or of each child file
are to be generated, it may be convenient to set up a |Makefile| or
shell script to automatise the process.

%%%%%%%%%%%%%%%%%%%%%%%%%%%%%%%%%%%%%%%%%%%%%%%%%%%%%%%%%%%%%%%%%%%%%%%%%%%%%%%%
\subsection{Command Line Processing}
\label{sec:commandline}

The effect of redirection files can also be achieved by invoking
the \LaTeX{} compiler with a more elaborate command line.
Most conveniently this should be done as part
of a shell script or a |Makefile|.

When using \textsf{childdoc} in the main file, the following
command lines effectively perform a redirection
(note that depending on the shell being used,
backslashes may have to be doubled: `|\|' $\to$ `|\\|'):
%
\begin{center}
|... -jobname "|\textit{target}|" |\\|"|[\textit{flags}]%
|\input{childdoc.def}\childdocforward[|\textit{main}|]{|\textit{dest}|}"|
\end{center}
%
Here \textit{target} is the name of the output file,
\textit{main} is the name of the main file
and \textit{dest} is the name of the main or child file to be processed
(all filenames without extensions).
The optional argument \textit{main} can be omitted
if \textit{main} matches \textit{dest}.
Optionally, compilation \textit{flags} can be defined via |\def| commands.
This command line makes the \TeX{} engine believe
it is compiling the file \textit{target}
whose content is specified as the latter parameter.
The provided code then forwards the processing to
\textit{main} or \textit{dest} as described in \secref{sec:forward}.

%%%%%%%%%%%%%%%%%%%%%%%%%%%%%%%%%%%%%%%%%%%%%%%%%%%%%%%%%%%%%%%%%%%%%%%%%%%%%%%%
\subsection{Include by Input}
\label{sec:input}

Including child documents by |\include| has some restrictions by design.
Most notably, the content of a child document always occupies
its own set of pages; pages cannot be shared between child documents.
Usually, this behaviour makes perfect sense
because each child document contain an essential part of the document.
However, in some situations it may be desirable to compose
a document from a collection of parts
without having mandatory page breaks between then.
For this case, the package
provides a mechanism to include parts
by |\input| which can also be processed individually.
However, by construction this mechanism
requires manual handling of the content to be output.

%%%%%%%%%%%%%%%%%%%%%%%%%%%%%%%%%%%%%%%%
\DescribeMacro{\ifchilddocmanual}
The main file should be prepared as usual, see \secref{sec:include}.
However, the document body must make a distinction
between processing of an individual part and of the main document, e.g.:
%
\begin{center}
\begin{tabular}{l}
|\ifchilddocmanual|\\
|\input{\childdocname}|\\
|\||else|\\
\textit{document body with }|\input{|\textit{part}|}|\\
|\||fi|
\end{tabular}
\end{center}
%
The conditional |\ifchilddocmanual| is true whenever
a part to be included by |\input| is being compiled,
and the name of the part is stored in |\childdocname|.

%%%%%%%%%%%%%%%%%%%%%%%%%%%%%%%%%%%%%%%%
\DescribeMacro{\childdocby}
Each part to be included by |\input| should start with:
%
\begin{center}
\begin{tabular}{l}
|\input{childdoc.def}|\\
|\childdocby{|\textit{main}|}|\\
\end{tabular}
\end{center}
%
The directive |\childdocby| is similar to |\childdocof|
described in \secref{sec:include},
but the subsequent selection of content must be done manually.
To that end, both |\ifchilddoc| and |\ifchilddocmanual|
will be true upon processing of a part,
and the name of the part is stored in |\childdocname|.
Note that |\jobname| will be set to the filename of the current part
so that each part receives an individual |.aux| file
that does not interfere with the |.aux| file(s) of the main document.
This behaviour can be altered by the alternative form
|\childdocby[*]{|\textit{main}|}| (with a non-empty optional argument)
which uses the |.aux| file of the main document
by setting |\jobname| to \textit{main}.

%%%%%%%%%%%%%%%%%%%%%%%%%%%%%%%%%%%%%%%%%%%%%%%%%%%%%%%%%%%%%%%%%%%%%%%%%%%%%%%%
\subsection{Driver Development}
\label{sec:driver}

The \textsf{childdoc} mechanism can also be use for the development
of definition files such as \LaTeX{} styles or classes.
This case differs from the above setup with multiple parts
included by |\include| in that no |\includeonly| should be invoked.
This can be achieved by starting the include file
(before |\ProvidesPackage|) with:
%
\begin{center}
\begin{tabular}{l}
|\input{childdoc.def}|\\
|\childdocforward{|\textit{main}|}|\\
\end{tabular}
\end{center}
%
or alternatively with:
%
\begin{center}
\begin{tabular}{l}
|\input{childdoc.def}|\\
|\childdocby{|\textit{main}|}|\\
\end{tabular}
\end{center}
%
Both forms have slightly different effects as described above.
The main file is prepared as usual, see \secref{sec:include}.

%%%%%%%%%%%%%%%%%%%%%%%%%%%%%%%%%%%%%%%%%%%%%%%%%%%%%%%%%%%%%%%%%%%%%%%%%%%%%%%%
\subsection{Legacy Detection}
\label{sec:detection}

The directive |\childdocmain| in the main file can detect
whether the complete document or merely a child is to be compiled
even without using the directive |\childdocof|.
This method is deprecated because it is less robust
and there is no compelling reason to use it;
it is merely provided for backward compatibility
and it may be removed in future versions.

If the detection mechanism is to be used,
it is mandatory to correctly specify
the filename of the main file as the argument of |\childdocmain|:
%
\begin{center}
\begin{tabular}{l}
|\input{childdoc.def}|\\
|\childdocmain{|\textit{main}|}|\\
\end{tabular}
\end{center}
%
If |\jobname| does not match the argument \textit{main} of |\childdocmain|,
it is assumed that |\jobname| points to the child file to be compiled.
When using |\childdocmain| with the main file specified as argument,
it suffices to start a child file
with just |\input{|\textit{main}|}|
without loading of the package and using |\childdocof|.
If instead all processing is done
with the appropriate \textsf{childdoc} directives,
the argument of \textit{main} of |\childdocmain| can be empty.

An alternative version of the command line processing described
in \secref{sec:commandline} using the detection mechanism reads:
%
\begin{center}
|... -jobname "|\textit{target}|" "|[\textit{flags}]%
[|\def\jobname{|\textit{dest}|}|]|\input{|\textit{main}|}"|
\end{center}

%%%%%%%%%%%%%%%%%%%%%%%%%%%%%%%%%%%%%%%%%%%%%%%%%%%%%%%%%%%%%%%%%%%%%%%%%%%%%%%%
\subsection{Manual Code}
\label{sec:manual}

In case one cannot be certain whether the definitions file |childdoc.def|
is installed on the target \TeX{} distribution
and one prefers not to ship it,
it is conceivable to paste a few relevant commands into the sources.

To that end, drop all statements |\input{childdoc.def}|
and perform the replacements as outlined below.
Instead of |\childdocmain{|\textit{main}|}| add the following code
to the top of the main file:
%
\begin{center}
\begin{tabular}{l}
|\||ifdefined\childdocname\endinput\||fi\newif\ifchilddoc|\\
|\edef\childdocname{\scantokens\expandafter{\jobname\noexpand}}|\\
|\def\childdocmain{|\textit{main}|}\||ifx\childdocmain\childdocname\||else|\\
|\childdoctrue\includeonly{\childdocname}\let\jobname\childdocmain\||fi|\\
\end{tabular}
\end{center}
%
Instead of |\childdocof{|\textit{main}|}| just include the main file
at the top of each child file:
%
\begin{center}
|\input{|\textit{main}|}|
\end{center}
%
A simple redirection |\childdocforward{|\textit{dest}|}| is achieved by:
%
\begin{center}
|\def\jobname{|\textit{dest}|}\input{\jobname}|
\end{center}
%
The redirection with prefix
|\childdocforwardprefix[|\textit{prefix}|]{|\textit{dest}|}|
is accomplished by:
%
\begin{center}
\begin{tabular}{l}
|{\edef\jobname{\scantokens\expandafter{\jobname\noexpand}}|\\
|\def\redirectjob |\textit{prefix}|#1~~~{\gdef\jobname{|\textit{dest}|#1}}|\\
|\expandafter\redirectjob\jobname~~~}\input{\jobname}|
\end{tabular}
\end{center}

In an alternative approach,
child documents can be compiled by a specific command line
without additional code or specific definitions:
%
\begin{center}
|... -jobname "|\textit{target}|" "|[\textit{flags}]%
|\includeonly{|\textit{dest}|}\input{|\textit{main}|}"|
\end{center}
%

%%%%%%%%%%%%%%%%%%%%%%%%%%%%%%%%%%%%%%%%%%%%%%%%%%%%%%%%%%%%%%%%%%%%%%%%%%%%%%%%
%%%%%%%%%%%%%%%%%%%%%%%%%%%%%%%%%%%%%%%%%%%%%%%%%%%%%%%%%%%%%%%%%%%%%%%%%%%%%%%%
\section{Information}

%%%%%%%%%%%%%%%%%%%%%%%%%%%%%%%%%%%%%%%%%%%%%%%%%%%%%%%%%%%%%%%%%%%%%%%%%%%%%%%%
\subsection{Copyright}

Copyright \copyright{} 2017--2018 Niklas Beisert

This work may be distributed and/or modified under the
conditions of the \LaTeX{} Project Public License, either version 1.3
of this license or (at your option) any later version.
The latest version of this license is in
  \url{http://www.latex-project.org/lppl.txt}
and version 1.3 or later is part of all distributions of \LaTeX{}
version 2005/12/01 or later.

This work has the LPPL maintenance status `maintained'.

The Current Maintainer of this work is Niklas Beisert.

This work consists of the files |README.txt|, |childdoc.ins| and |childdoc.dtx|
as well as the derived files |childdoc.def|, |cdocsamp.tex|
with |cdocsch1.tex|, |cdocsch2.tex|, |cdocspt3.tex|, |cdocspt4.tex|,
|cdocsdrf.tex|, |cdocsfn1.tex|, |cdocsfn2.tex|
as well as |childdoc.pdf|.

%%%%%%%%%%%%%%%%%%%%%%%%%%%%%%%%%%%%%%%%%%%%%%%%%%%%%%%%%%%%%%%%%%%%%%%%%%%%%%%%
\subsection{Files and Installation}

The package consists of the files:
%
\begin{center}
\begin{tabular}{ll}
    |README.txt|   & readme file \\
    |childdoc.ins| & installation file \\
    |childdoc.dtx| & source file \\
    |childdoc.def| & definition file \\
    |cdocsamp.tex| & sample main file \\
    |cdocsch1.tex| & sample include file \\
    |cdocsch2.tex| & sample include file \\
    |cdocspt3.tex| & sample part file \\
    |cdocspt4.tex| & sample part file \\
    |cdocsdrf.tex| & sample redirection file \\
    |cdocsfn1.tex| & sample redirection file \\
    |cdocsfn2.tex| & sample redirection file \\
    |childdoc.pdf| & manual
\end{tabular}
\end{center}
%
The distribution consists of the files
|README.txt|, |childdoc.ins| and |childdoc.dtx|.
%
\begin{itemize}
\item
Run (pdf)\LaTeX{} on |childdoc.dtx|
to compile the manual |childdoc.pdf| (this file).
\item
Run \LaTeX{} on |childdoc.ins| to create the definitions file |childdoc.def|
and the sample |cdocsamp.tex| with include files
|cdocsch1.tex|, |cdocsch2.tex|, |cdocspt3.tex|, |cdocspt4.tex|,
|cdocsdrf.tex|, |cdocsfn1.tex|, |cdocsfn2.tex|.
Then copy the file |childdoc.def| to an appropriate directory of your \LaTeX{}
distribution, e.g.\ \textit{texmf-root}|/tex/latex/childdoc|.
\end{itemize}

%%%%%%%%%%%%%%%%%%%%%%%%%%%%%%%%%%%%%%%%%%%%%%%%%%%%%%%%%%%%%%%%%%%%%%%%%%%%%%%%
\subsection{Related CTAN Packages}

There are several other packages which offer a similar functionality:
%
\begin{itemize}
\item
The packages
\href{http://ctan.org/pkg/docmute}{\textsf{docmute}},
\href{http://ctan.org/pkg/includex}{\textsf{includex}} and
\href{http://ctan.org/pkg/standalone}{\textsf{standalone}}
provide commands to include only the document body of
a child file thus allowing both files to be compiled individually.
\item
The packages \href{http://ctan.org/pkg/subdocs}{\textsf{subdocs}}
and \href{http://ctan.org/pkg/subfiles}{\textsf{subfiles}}
provide structures in which the main and child documents can be
encapsulated and allowing them to be compiled individually.
The inclusion mechanism is different from the conventional |\include|.
\item
The package \href{http://ctan.org/pkg/combine}{\textsf{combine}}
is an elaborate solution to combine several documents into one.
\end{itemize}
%
See also the CTAN topic \href{http://ctan.org/topic/subdocs}{\textsf{subdocs}}
for further related packages.
The present package differs from the above solutions in that
a document structure constructed with the conventional |\include| mechanism
just needs two extra commands at the top of every file
such that all constituent files can be compiled individually.

%%%%%%%%%%%%%%%%%%%%%%%%%%%%%%%%%%%%%%%%%%%%%%%%%%%%%%%%%%%%%%%%%%%%%%%%%%%%%%%%
%\subsection{Feature Suggestions}
%
%The following is a list of features which may be useful for future
%versions of this package:
%%
%\begin{itemize}
%\item
%\ldots
%\end{itemize}

%%%%%%%%%%%%%%%%%%%%%%%%%%%%%%%%%%%%%%%%%%%%%%%%%%%%%%%%%%%%%%%%%%%%%%%%%%%%%%%%
\subsection{Revision History}

%%%%%%%%%%%%%%%%%%%%%%%%%%%%%%%%%%%%%%%%
\paragraph{v2.0:} 2018/12/30

\begin{itemize}
\item
immediate forward processing
\item
added |\childdocby| mechanism
\item
manual restructured
\end{itemize}

%%%%%%%%%%%%%%%%%%%%%%%%%%%%%%%%%%%%%%%%
\paragraph{v1.6:} 2018/01/17

\begin{itemize}
\item
application for development of include files
\item
corrections to manual
\end{itemize}

%%%%%%%%%%%%%%%%%%%%%%%%%%%%%%%%%%%%%%%%
\paragraph{v1.5:} 2017/05/21

\begin{itemize}
\item
more complete structuring introduced
\item
|\childdocof| introduced
\item
|\childdoc| renamed to |\childdocmain|
\item
|\childredirect| renamed to |\childdocforward| and |\childdocforwardprefix|
and functionality expanded
\end{itemize}

%%%%%%%%%%%%%%%%%%%%%%%%%%%%%%%%%%%%%%%%
\paragraph{v1.0:} 2017/04/27

\begin{itemize}
\item
manual and install package
\item
first version published on CTAN
\end{itemize}

%%%%%%%%%%%%%%%%%%%%%%%%%%%%%%%%%%%%%%%%
\paragraph{v0.6:} 2017/04/26

\begin{itemize}
\item
redirection mechanism added
\end{itemize}

%%%%%%%%%%%%%%%%%%%%%%%%%%%%%%%%%%%%%%%%
\paragraph{v0.5:} 2017/04/26

\begin{itemize}
\item
functionality in definition file
\end{itemize}


%%%%%%%%%%%%%%%%%%%%%%%%%%%%%%%%%%%%%%%%%%%%%%%%%%%%%%%%%%%%%%%%%%%%%%%%%%%%%%%%
%%%%%%%%%%%%%%%%%%%%%%%%%%%%%%%%%%%%%%%%%%%%%%%%%%%%%%%%%%%%%%%%%%%%%%%%%%%%%%%%
%%%%%%%%%%%%%%%%%%%%%%%%%%%%%%%%%%%%%%%%%%%%%%%%%%%%%%%%%%%%%%%%%%%%%%%%%%%%%%%%
\appendix

\settowidth\MacroIndent{\rmfamily\scriptsize 000\ }

 \DocInput{childdoc.dtx}

\end{document}
%</driver>
% \fi
%
% %%%%%%%%%%%%%%%%%%%%%%%%%%%%%%%%%%%%%%%%%%%%%%%%%%%%%%%%%%%%%%%%%%%%%%%%%%%%%%
% %%%%%%%%%%%%%%%%%%%%%%%%%%%%%%%%%%%%%%%%%%%%%%%%%%%%%%%%%%%%%%%%%%%%%%%%%%%%%%
% \section{Sample}
%\iffalse
%<*samplemain>
%\fi
%
% The following presents a sample document
% with two chapters, two parts, a title page,
% a compile flag as well as three forwarding files to set the flag.
% It consists of eight |.tex| files:
% \begin{center}
% \begin{tabular}{ll}
% |cdocsamp.tex|&main file\\
% |cdocsch1.tex|&include file for chapter 1\\
% |cdocsch2.tex|&include file for chapter 2\\
% |cdocspt3.tex|&include file for part 3\\
% |cdocspt4.tex|&include file for part 4\\
% |cdocsdrf.tex|&forwarding file for main file in draft mode\\
% |cdocsfi1.tex|&forwarding file for final version of chapter 1\\
% |cdocsfi2.tex|&forwarding file for final version of chapter 2\\
% \end{tabular}
% \end{center}
% Each of the eight files can be compiled directly by the \LaTeX{} compiler.
%
% %%%%%%%%%%%%%%%%%%%%%%%%%%%%%%%%%%%%%%
% \paragraph{Main File.}
%
% The main file is called |cdocsamp.tex|.
%
% Load the \textsf{childdoc} definitions and
% declare the filename for the main document:
%    \begin{macrocode}
\input{childdoc.def}
\childdocmain{}
%    \end{macrocode}

% Optional override for |\version| flag:
%    \begin{macrocode}
%%\ifchilddoc\else\providecommand{\version}{draft}\fi
%    \end{macrocode}

% Define the default values for the |\version| flag
% (|final| for the main file and |draft| for childs):
%    \begin{macrocode}
\ifchilddoc
\providecommand{\version}{draft}
\else
\providecommand{\version}{final}
\fi
%    \end{macrocode}

% Load the standard document class:
%    \begin{macrocode}
\documentclass[12pt]{article}
%    \end{macrocode}

% Start the document body:
%    \begin{macrocode}
\begin{document}
%    \end{macrocode}

% Declare a title page.
% Print title, part of document being processed and version flag:
%    \begin{macrocode}
\addtocounter{page}{-1}
\begin{center}
{\LARGE\bfseries{}childdoc example\par}
\vspace{1cm}
\ifchilddoc
\ifchilddocmanual part\else chapter\fi:
`\childdocname' of `\childdocjob'\par
\else
main document: `\childdocjob'\par
\fi
version: \version\par
\end{center}
\newpage
%    \end{macrocode}

% Manually include selected file,
% otherwise process as usual:
%    \begin{macrocode}
\ifchilddocmanual
\section*{part `\childdocname'}
\input{\childdocname}
\else
%    \end{macrocode}

% Include the two chapters:
%    \begin{macrocode}
\include{cdocsch1}
\include{cdocsch2}
%    \end{macrocode}

% Include the two parts unless only chapters should be displayed:
%    \begin{macrocode}
\ifchilddoc\else
\section{part three}
\input{cdocspt3}
\section{part four}
\input{cdocspt4}
\fi
%    \end{macrocode}

% Process as usual until here:
%    \begin{macrocode}
\fi
%    \end{macrocode}

% End of document body:
%    \begin{macrocode}
\end{document}
%    \end{macrocode}
%\iffalse
%</samplemain>
%\fi
%
% %%%%%%%%%%%%%%%%%%%%%%%%%%%%%%%%%%%%%%
% \paragraph{Chapter Include Files.}
%
% The include files are called |cdocsch1.tex| and |cdocsch2.tex|.
%
%\iffalse
%<*samplechap1|samplechap2>
%\fi

% Optional override for |\version| flag:
%    \begin{macrocode}
%%\providecommand{\version}{final}
%    \end{macrocode}

% Include the main document:
%    \begin{macrocode}
\input{childdoc.def}
\childdocof{cdocsamp}
%    \end{macrocode}

%\iffalse
%</samplechap1|samplechap2>
%\fi
%
%\iffalse
%<*samplechap1>
%\fi
% Some text for chapter 1:
%    \begin{macrocode}
\section{one}
some text in chapter one
%    \end{macrocode}

%\iffalse
%</samplechap1>
%\fi
% Some text for chapter 2:
%\iffalse
%<*samplechap2>
%\fi
%    \begin{macrocode}
\section{two}
more text in chapter two
%    \end{macrocode}

%\iffalse
%</samplechap2>
%\fi
%
% %%%%%%%%%%%%%%%%%%%%%%%%%%%%%%%%%%%%%%
% \paragraph{Part Include Files.}
%
% The include files are called |cdocspt3.tex| and |cdocspt4.tex|.
%
%\iffalse
%<*samplepart3|samplepart4>
%\fi

% Optional override for |\version| flag:
%    \begin{macrocode}
%%\providecommand{\version}{final}
%    \end{macrocode}

% Include the main document:
%    \begin{macrocode}
\input{childdoc.def}
\childdocby{cdocsamp}
%    \end{macrocode}

%\iffalse
%</samplepart3|samplepart4>
%\fi
%
%\iffalse
%<*samplepart3>
%\fi
% Some text for part 3:
%    \begin{macrocode}
some text in part three
%    \end{macrocode}

%\iffalse
%</samplepart3>
%\fi
% Some text for part 4:
%\iffalse
%<*samplepart4>
%\fi
%    \begin{macrocode}
more text in part four
%    \end{macrocode}

%\iffalse
%</samplepart4>
%\fi
%
% %%%%%%%%%%%%%%%%%%%%%%%%%%%%%%%%%%%%%%
% \paragraph{Forwarding for a Complete Draft.}
%
% The following forwarding file |cdocsdrf.tex|
% compiles the main document in draft mode:
%\iffalse
%<*sampledraft>
%\fi
%    \begin{macrocode}
\def\version{draft}
\input{childdoc.def}
\childdocforward{cdocsamp}
%    \end{macrocode}

%\iffalse
%</sampledraft>
%\fi
%
% %%%%%%%%%%%%%%%%%%%%%%%%%%%%%%%%%%%%%%
% \paragraph{Forwarding for Final Version of the Chapters.}
%
% The following forwarding files |cdocsfn1.tex| and |cdocsfn2.tex|
% (with identical content)
% compile the final versions of the child documents
% |cdocsch1.tex| and |cdocsch2.tex|, respectively:
%\iffalse
%<*samplefinal>
%\fi
%    \begin{macrocode}
\def\version{final}
\input{childdoc.def}
\childdocforwardprefix[cdocsamp]{cdocsfn}{cdocsch}
%    \end{macrocode}

%\iffalse
%</samplefinal>
%\fi
%
% %%%%%%%%%%%%%%%%%%%%%%%%%%%%%%%%%%%%%%
% \paragraph{Command Line Processing.}
%
% The following three command lines generate the output files
% |cdocscld|, |cdocscl1| and |cdocscl2|
% which should be identical to
% |cdocsdrf|, |cdocsch1| and |cdocsfn2|, respectively:
% \begin{center}
% \begin{tabular}{l}
% |latex -jobname cdocscld \|\\
% |  "\def\version{draft}\input{childdoc.def}\childdocforward{cdocsamp}"|\\
% |latex -jobname cdocscl1 \|\\
% |  "\input{childdoc.def}\childdocforward[cdocsamp]{cdocsch1}"|\\
% |latex -jobname cdocscl2 \|\\
% |  "\def\version{final}\input{childdoc.def}\childdocforward{cdocsch2}"|
% \end{tabular}
% \end{center}
% Note that the trailing backslash on each first line
% merely continues the input to the second line
% (for convenient cut ant paste).
% Furthermore, the command |latex| can be replaced by any
% of its alternative versions such as |pdflatex|.
%
% %%%%%%%%%%%%%%%%%%%%%%%%%%%%%%%%%%%%%%%%%%%%%%%%%%%%%%%%%%%%%%%%%%%%%%%%%%%%%%
% %%%%%%%%%%%%%%%%%%%%%%%%%%%%%%%%%%%%%%%%%%%%%%%%%%%%%%%%%%%%%%%%%%%%%%%%%%%%%%
% \section{Implementation}
%\iffalse
%<*package>
%\fi
%
% This section describes the definitions file |childdoc.def|.

% The definitions cannot be loaded using |\usepackage| or |\RequirePackage|
% which has a mechanism to prevent loading a style file more than once.
% When loading the definitions by means of |\input|
% multiple instances have to be prevented manually:
%\iffalse
%This code needs to be before the `\ProvidesFile' directive
%which is defined at the beginning of this file.
%Therefore it is also placed there and commented out here.
%</package>
%<*discard>
%\fi
%    \begin{macrocode}
\ifdefined\childdocmain\endinput\fi
%    \end{macrocode}
%\iffalse
%</discard>
%<*package>
%\fi
%
% \macro{\ifchilddoc}
% \macro{\ifchilddocmanual}
% The conditional |\ifchilddoc| tells whether a
% child (true) or main (false) document is being compiled.
% The conditional |\ifchilddocmanual| tells whether
% the |\includeonly| mechanism is used (false) or
% the selection of child files must be performed manually (true).
% The definitions initialise to false:
%    \begin{macrocode}
\newif\ifchilddoc
\newif\ifchilddocmanual
%    \end{macrocode}

% \macro{\childdocname}
% \macro{\childdocjob}
% The macro |\childdocname| stores the name of the main document
% to be compiled. The macro |\childdocjob| stores the name of
% the document on which the \LaTeX{} compiler was originally invoked.
% The content of |\jobname| cannot be compared
% to filenames specified in the source due to different catcodes.
% The following code rescans |\jobname|, stores the result
% in |\childdocname| and saves a copy in |\childdocjob|:
%    \begin{macrocode}
\edef\childdocname{\scantokens\expandafter{\jobname\noexpand}}
\let\childdocjob\childdocname
%    \end{macrocode}

% \macro{\childdocdisable}
% The macro |\childdocdisable| prevents the main file
% from being processed more than once.
% At this stage, the main document command |\childdocmain|
% is assumed to be called once again where it should do nothing.
% Any subsequent call to it should prevent
% a secondary processing of the main document
% It overwrites the forwarding commands
% |\childdocof| and |\childdocforward|
% with empty macros to prevent further inclusions of the main document:
%    \begin{macrocode}
\newcommand{\childdocdisable}
{
  \renewcommand{\childdocmain}[1]{\renewcommand{\childdocmain}[1]{\endinput}}
  \renewcommand{\childdocof}[1]{}
  \renewcommand{\childdocby}[2][]{}
  \renewcommand{\childdocforward}[2][]{}
  \renewcommand{\childdocdisable}{}
}
%    \end{macrocode}

% \macro{\childdocmain}
% The macro |\childdocmain| is to be called at the top of the main file
% with nothing or the main filename (without extension) as argument.
% First, it breaks loops.
% If the argument is not empty and does not match |\childdocname|
% (which is set by the first inclusion of |childdoc.def|),
% |\ifchilddoc| is set to true, |\includeonly| is applied to the child file
% and |\jobname| is set to the main file
% (for proper handling of |.aux| files):
%    \begin{macrocode}
\newcommand{\childdocmain}[1]
{
  \childdocdisable\childdocmain{}
  \if?#1?\else
    \begingroup
      \def\childdoctmp{#1}
      \ifx\childdoctmp\childdocname
        \def\childdoctmp{}
      \else
        \def\childdoctmp
        {
          \childdoctrue
          \includeonly{\childdocname}
          \def\childdocjob{#1}
          \def\jobname{#1}
        }
      \fi
      \expandafter
    \endgroup
    \childdoctmp
  \fi
}
%    \end{macrocode}

% \macro{\childdocof}
% The command |\childdocof| redirects
% compilation to the main file |#1|.
%    \begin{macrocode}
\newcommand{\childdocof}[1]
{
  \childdocdisable
  \childdoctrue
  \includeonly{\childdocname}
  \def\jobname{#1}
  \def\childdocjob{#1}
  \input{#1}
}
%    \end{macrocode}

% \macro{\childdocby}
% The command |\childdocby| ....
%    \begin{macrocode}
\newcommand{\childdocby}[2][]
{
  \childdocdisable
  \childdoctrue
  \childdocmanualtrue
  \if?#1?\else
    \def\jobname{#2}
  \fi
  \def\childdocjob{#2}
  \input{#2}
  \endinput
}
%    \end{macrocode}

% \macro{\childdocforward}
% The command |\childdocforward| redirects
% compilation to the main file or
% (if the optional argument is given) a child file.
% Parameters are set as if the main file
% or a child file starting with |\childdocof| was compiled.
% Then compilation is handed over to the main file:
%    \begin{macrocode}
\newcommand{\childdocforward}[2][]
{
  \begingroup
    \if?#1?
      \def\childdoctmp
      {
        \def\childdocname{#2}
        \def\childdocjob{#2}
        \def\jobname{#2}
        \input{#2}
        \endinput
      }
    \else
      \def\childdoctmp
      {
        \childdocdisable
        \def\childdocname{#2}
        \childdoctrue
        \includeonly{#2}
        \def\childdocjob{#1}
        \def\jobname{#1}
        \input{#1}
        \endinput
      }
    \fi
    \expandafter
  \endgroup
  \childdoctmp
}
%    \end{macrocode}

% \macro{\childdocforwardprefix}
% The command |\childdocforwardprefix| redirects
% compilation to the main or a child file by means of a pattern.
% The prefix |#1| in the current filename is replaced by |#2|
% and the suffix of the current filename is kept
% (it is assumed that the filename does not contain the substring `|~~~|'
% which is used as a delimiter).
% Compilation is handed over to the new file by |\childdocforward|:
%    \begin{macrocode}
\newcommand{\childdocforwardprefix}[3][]
{
  \begingroup
    \def\childdocextract #2##1~~~{\def\childdoctmp{\childdocforward[#1]{#3##1}}}
    \expandafter\childdocextract\childdocname~~~
    \expandafter
  \endgroup
  \childdoctmp
}
%    \end{macrocode}

% \macro{\childdoc}
% The deprecated macro |\childdoc| is a legacy version of |\childdocmain|:
%    \begin{macrocode}
\newcommand{\childdoc}{\childdocmain}
%    \end{macrocode}

% \macro{\childdocredirect}
% The deprecated macro |\childdocredirect| is a legacy version
% of |\childdocforward| and |\childdocforwardprefix|:
%    \begin{macrocode}
\newcommand{\childdocredirect}[2][]
{
  \begingroup
    \if?#1?
      \def\childdoctmp{\childdocforward{#2}}
    \else
      \def\childdoctmp{\childdocforwardprefix{#1}{#2}}
    \fi
    \expandafter
  \endgroup
  \childdoctmp
}
%    \end{macrocode}

%\iffalse
%</package>
%\fi
%
\endinput
|\\
|\childdocmain{|\textit{main}|}|\\
\end{tabular}
\end{center}
%
If |\jobname| does not match the argument \textit{main} of |\childdocmain|,
it is assumed that |\jobname| points to the child file to be compiled.
When using |\childdocmain| with the main file specified as argument,
it suffices to start a child file
with just |\input{|\textit{main}|}|
without loading of the package and using |\childdocof|.
If instead all processing is done
with the appropriate \textsf{childdoc} directives,
the argument of \textit{main} of |\childdocmain| can be empty.

An alternative version of the command line processing described
in \secref{sec:commandline} using the detection mechanism reads:
%
\begin{center}
|... -jobname "|\textit{target}|" "|[\textit{flags}]%
[|\def\jobname{|\textit{dest}|}|]|\input{|\textit{main}|}"|
\end{center}

%%%%%%%%%%%%%%%%%%%%%%%%%%%%%%%%%%%%%%%%%%%%%%%%%%%%%%%%%%%%%%%%%%%%%%%%%%%%%%%%
\subsection{Manual Code}
\label{sec:manual}

In case one cannot be certain whether the definitions file |childdoc.def|
is installed on the target \TeX{} distribution
and one prefers not to ship it,
it is conceivable to paste a few relevant commands into the sources.

To that end, drop all statements |% \iffalse
%
% childdoc.dtx Copyright (C) 2017-2018 Niklas Beisert
%
% This work may be distributed and/or modified under the
% conditions of the LaTeX Project Public License, either version 1.3
% of this license or (at your option) any later version.
% The latest version of this license is in
%   http://www.latex-project.org/lppl.txt
% and version 1.3 or later is part of all distributions of LaTeX
% version 2005/12/01 or later.
%
% This work has the LPPL maintenance status `maintained'.
%
% The Current Maintainer of this work is Niklas Beisert.
%
% This work consists of the files childdoc.dtx and childdoc.ins
% and the derived files childdoc.def and cdocsamp.tex with
% cdocsch1.tex, cdocsch2.tex, cdocsdrf.tex, cdocsfn1.tex, cdocsfn2.tex.
%
%<package>\ifdefined\childdocmain\endinput\fi
%<package>\ProvidesFile{childdoc.def}[2018/12/30 v2.0 child document driver]
%<samplemain>\ProvidesFile{cdocsamp.tex}[2018/12/30 v2.0 sample for childdoc]
%<*driver>
%\ProvidesFile{childdoc.drv}[2018/12/30 v2.0 childdoc reference manual file]
\PassOptionsToClass{10pt,a4paper}{article}
\documentclass{ltxdoc}

\usepackage[margin=35mm]{geometry}
\usepackage{hyperref}
\usepackage{hyperxmp}
\usepackage[usenames]{color}

\hypersetup{colorlinks=true}
\hypersetup{pdfstartview=FitH}
\hypersetup{pdfpagemode=UseNone}
\hypersetup{pdfsource={}}
\hypersetup{pdflang={en-UK}}
\hypersetup{pdfcopyright={Copyright 2017-2018 Niklas Beisert.
  This work may be distributed and/or modified under the
  conditions of the LaTeX Project Public License, either version 1.3
  of this license or (at your option) any later version.}}
\hypersetup{pdflicenseurl={http://www.latex-project.org/lppl.txt}}
\hypersetup{pdfcontactaddress={ETH Zurich, ITP, HIT K,
  Wolfgang-Pauli-Strasse 27}}
\hypersetup{pdfcontactpostcode={8093}}
\hypersetup{pdfcontactcity={Zurich}}
\hypersetup{pdfcontactcountry={Switzerland}}
\hypersetup{pdfcontactemail={nbeisert@itp.phys.ethz.ch}}
\hypersetup{pdfcontacturl={http://people.phys.ethz.ch/\xmptilde nbeisert/}}

\newcommand{\secref}[1]{\hyperref[#1]{section \ref*{#1}}}

\parskip1ex
\parindent0pt
\let\olditemize\itemize
\def\itemize{\olditemize\parskip0pt}

\begin{document}

\title{The \textsf{childdoc} Package}
\hypersetup{pdftitle={The childdoc Package}}
\author{Niklas Beisert\\[2ex]
  Institut f\"ur Theoretische Physik\\
  Eidgen\"ossische Technische Hochschule Z\"urich\\
  Wolfgang-Pauli-Strasse 27, 8093 Z\"urich, Switzerland\\[1ex]
  \href{mailto:nbeisert@itp.phys.ethz.ch}
  {\texttt{nbeisert@itp.phys.ethz.ch}}}
\hypersetup{pdfauthor={Niklas Beisert}}
\hypersetup{pdfsubject={Manual for the LaTeX2e Package childdoc}}
\date{30 December 2018, \textsf{v2.0}}
\maketitle

\begin{abstract}\noindent
\textsf{childdoc} is a \LaTeXe{} package
that enables the direct compilation
of document sections included by |\include|
to individual files.
\end{abstract}

\begingroup
\parskip0ex
\tableofcontents
\endgroup

%%%%%%%%%%%%%%%%%%%%%%%%%%%%%%%%%%%%%%%%%%%%%%%%%%%%%%%%%%%%%%%%%%%%%%%%%%%%%%%%
%%%%%%%%%%%%%%%%%%%%%%%%%%%%%%%%%%%%%%%%%%%%%%%%%%%%%%%%%%%%%%%%%%%%%%%%%%%%%%%%
\section{Introduction}

\LaTeX{} provides a mechanism to structure a large document (such as a book)
into a main file and several child files (containing the chapters)
using the |\include| command.
This mechanism is beneficial for documents
which span hundreds of pages in order to
make the source file(s) more manageable.
Moreover, compilation can be restricted to
selected child files by means of the |\includeonly| command.
The latter feature can be used to reduce the compilation time while editing
(this was significantly more useful in the earlier days of \LaTeX{})
or to generate a smaller document which is easier to navigate.
Another application of |\includeonly| is to generate
documents consisting of selected parts of the complete document.

However, there are a few drawbacks of the plain |\include| mechanism:
\begin{itemize}
\item
The child files cannot be compiled on their own,
they can only be compiled via the main file.
A naive editing environment
(such as a text editor with an option
to have the current file processed by \LaTeX)
may require one to switch to the main file before compiling;
attempting to compile the child file produces errors.
\item
The main file must be modified (each time)
to adjust the |\includeonly| command
to the present needs. This easily leaves the main file in a messy state.
\item
The generated document will always carry the filename
of the main document. This is inconvenient if
several child files are to be compiled and
to be kept for distribution.
\end{itemize}

The present package provides a simple interface
to make child files individually compilable by \LaTeX{}.
Compiling a child file then has the same effect as compiling
the main file with an |\includeonly| command
to select the appropriate child.
Moreover the generated document will carry the name of the child
rather than the main file.
This resolves all three above issues.

This feature is meant to make the editing of books,
thesis documents and lecture notes somewhat more convenient.
However, the package can also be used efficiently for
composing a series of documents (such as exercise sheets)
which are typically distributed individually.
It then assists the author in generating the individual documents
(potentially in different versions)
as well as a document containing the collected series.
Another application is in developing style files
or other kinds of included material
where compilation of the style file could redirect
to a sample or test file.

%%%%%%%%%%%%%%%%%%%%%%%%%%%%%%%%%%%%%%%%%%%%%%%%%%%%%%%%%%%%%%%%%%%%%%%%%%%%%%%%
%%%%%%%%%%%%%%%%%%%%%%%%%%%%%%%%%%%%%%%%%%%%%%%%%%%%%%%%%%%%%%%%%%%%%%%%%%%%%%%%
\section{Usage}

First of all, the package \textsf{childdoc} is \emph{not} a standard
\LaTeXe{} |.sty| style file! Therefore it needs to be invoked in
a non-standard way.

%%%%%%%%%%%%%%%%%%%%%%%%%%%%%%%%%%%%%%%%%%%%%%%%%%%%%%%%%%%%%%%%%%%%%%%%%%%%%%%%
\subsection{Included Files}
\label{sec:include}

%%%%%%%%%%%%%%%%%%%%%%%%%%%%%%%%%%%%%%%%
\DescribeMacro{\childdocmain}
To use the package, add the commands
\begin{center}
\begin{tabular}{l}
|\input{childdoc.def}|\\
|\childdocmain{}|\\
\end{tabular}
\end{center}
at the very top of the main \LaTeX{} file,
in particular \emph{before} the |\documentclass| statement!
The argument of |\childdocmain| should be left empty
(but it must be present).

%%%%%%%%%%%%%%%%%%%%%%%%%%%%%%%%%%%%%%%%
\DescribeMacro{\childdocof}
Furthermore, add the commands
\begin{center}
\begin{tabular}{l}
|\input{childdoc.def}|\\
|\childdocof{|\textit{main}|}|\\
\end{tabular}
\end{center}
at the top of every child file \textit{child}
which is included by |\include{|\textit{child}|}|
from within the main file
(or at least for those files to be compiled individually).
The argument \textit{main} must be the filename of the main file.

There are a couple of
considerations in setting up the main and child documents:

%%%%%%%%%%%%%%%%%%%%%%%%%%%%%%%%%%%%%%%%
\paragraph{Restrictions.}

Please note the following restrictions:
\begin{itemize}
\item
|\childdocmain| must be called with one argument \textit{main}
to ensure compatibility with earlier version of the package.
It must either be empty (|\childdocmain{}|)
or precisely match the filename of the main file in which it is specified.
See \secref{sec:detection} for further information.
\item
The filename \textit{main} must be specified without the |.tex| extension.
\item
The filename \textit{main} is case sensitive
(even in case-insensitive file systems)
due to internal string comparison.
\item
The argument \textit{main} should be fully expanded, it cannot be a macro.
\item
Subdirectories and special characters should be avoided in filenames.
\item
The command |\childdocmain{|\textit{main}|}| must be followed by a whitespace.
It should not be followed immediately by another command
or by a comment mark `|%|'.
This is because the \TeX{} parser reads the token immediately following
the argument of |\childdocmain| and puts it
at the beginning of every child section;
however, a white\-space is ignored.
\end{itemize}

%%%%%%%%%%%%%%%%%%%%%%%%%%%%%%%%%%%%%%%%
\paragraph{Content of Main File.}

It is advisable to place all content in the child files included by |\include|.
Any output contained in the main file will appear in all child documents
unless suppressed manually;
it cannot be suppressed automatically by the |\includeonly| directive
and thus should normally be avoided.
A method to include some content in the main file
by means of conditional processing is described in \secref{sec:conditional}.

%%%%%%%%%%%%%%%%%%%%%%%%%%%%%%%%%%%%%%%%
\paragraph{Page Numbering.}

When only a part of the document is compiled,
the appropriate numbering of pages
(as well as other status parameters)
is determined from the |.aux| files.
The latter contain information from previous passes.
However this information needs to propagate through
all intermediate child documents.
Therefore the page numbering in child documents may well
be inconsistent until the complete document is compiled at least once.

A useful (if unconventional) way to always ensure a consistent
page numbering is to restart the numbering in each child document
and denote the pages by `\textit{child}|.|\textit{page}'
where \textit{child} represents the chapter/section number of the child file.
This can be achieved by the command
|\numberwithin{page}{|\textit{child}|}|
of the \textsf{amsmath} package
where \textit{child} can be |chapter| or |section|
depending on the chosen structuring.
Alternatively, one can modify the macro |\thepage| appropriately
and reset the counter |page| at the start of each child file.

%%%%%%%%%%%%%%%%%%%%%%%%%%%%%%%%%%%%%%%%%%%%%%%%%%%%%%%%%%%%%%%%%%%%%%%%%%%%%%%%
\subsection{Conditional Processing}
\label{sec:conditional}

The package provides a mechanism to compile different versions
of a document. To customise the versions further some conditional processing
can come in handy to distinguish which version is being compiled.
The package provides two macros to describe the compilation context:

%%%%%%%%%%%%%%%%%%%%%%%%%%%%%%%%%%%%%%%%
\DescribeMacro{\ifchilddoc}
The conditional |\ifchilddoc| distinguishes between the compilation of
child documents and the main document:
%
\begin{center}
|\ifchilddoc |\textit{child-code}| |[|\||else |\textit{main-code}]| \||fi|
\end{center}

%%%%%%%%%%%%%%%%%%%%%%%%%%%%%%%%%%%%%%%%
\DescribeMacro{\childdocname}
\DescribeMacro{\childdocjob}
The macro |\childdocname| contains the filename (without extension)
of the main or child file being processed.
Note that |\childdocjob| will always contain the name of the main file.

%%%%%%%%%%%%%%%%%%%%%%%%%%%%%%%%%%%%%%%%
\paragraph{Title Page.}

Conditional processing can be used to include a title or banner page
in the main document when proper precautions are taken.
Importantly, the code in the main file should ensure that the page counter
(as well as other status parameters which are stored in the |.aux| files)
takes the same value after the conditional processing.
Otherwise the page numbers may take divergent values
depending on which part is compiled.

For example, a title page could be declared by:
%
\begin{center}
\begin{tabular}{l}
|\ifchilddoc\||else|\\
|\addtocounter{page}{-1}|\\
\textit{code for title page}\\
|\newpage|\\
|\||fi|
\end{tabular}
\end{center}
%
A banner page for the child documents can be generated by:
%
\begin{center}
\begin{tabular}{l}
|\ifchilddoc|\\
|\addtocounter{page}{-1}|\\
\textit{code for banner page}\\
|\newpage|\\
|\||fi|
\end{tabular}
\end{center}
%
Here one could write a message such as:
\begin{center}
|This is the part \childdocname{} of \childdocjob{}.|
\end{center}

%%%%%%%%%%%%%%%%%%%%%%%%%%%%%%%%%%%%%%%%%%%%%%%%%%%%%%%%%%%%%%%%%%%%%%%%%%%%%%%%
\subsection{Flags}
\label{sec:flags}

The package makes it easy to generate different versions
of the main or child documents.
To this end compilation flags can be defined
and assigned different default values.
They will be particularly useful in conjunction
with the forwarding mechanism described in \secref{sec:forward}.

For example, it may be useful to have a flag |\version|
which can be set to |draft| or |final|.
The document source will contain some conditional code
depending on the value of |\version|.
Suppose further, the flag should default to |final| for the main file
and to |draft| for child files
which is a natural assignment for editing the document.
This is achieved by placing the following code
in the preamble of the main document
(below the |\childdocmain| directive):
%
\begin{center}
\begin{tabular}{l}
|\ifchilddoc|\\
|\providecommand{\version}{draft}|\\
|\||else|\\
|\providecommand{\version}{final}|\\
|\||fi|
\end{tabular}
\end{center}
%
The definition by |\providecommand| makes sure
that previous definitions are not overwritten.
Further statements |\providecommand{\version}{...}|
can thus be added before the above code to override it.

For the main file, one might add a line
(between |\childdocmain| and the above block)
%
\begin{center}
|%\ifchilddoc\||else\providecommand{\version}{draft}\||fi|
\end{center}
%
which can be uncommented to produce a draft version.
Likewise one can add a line to the very top of a child file
(above the |\childdocof{|\textit{main}|}| directive)
%
\begin{center}
|%\providecommand{\version}{final}|
\end{center}
%
which can be uncommented to produce the final version of this child document.

%%%%%%%%%%%%%%%%%%%%%%%%%%%%%%%%%%%%%%%%%%%%%%%%%%%%%%%%%%%%%%%%%%%%%%%%%%%%%%%%
\subsection{Forwarding}
\label{sec:forward}

Different versions of the main or child documents
using compilation flags as described in \secref{sec:flags}
can be (permanently) stored in different files
for convenient compilation, viewing and distribution.
To this end, the package defines a command
to pass on compilation to a different file:

%%%%%%%%%%%%%%%%%%%%%%%%%%%%%%%%%%%%%%%%
\DescribeMacro{\childdocforward}
The command |\childdocforward| redirects processing to
another source file:
%
\begin{center}
\begin{tabular}{l}
|\input{childdoc.def}|\\
|\childdocforward[|\textit{main}|]{|\textit{dest}|}|\\
\end{tabular}
\end{center}
%
The argument \textit{dest} is the destination file
(without extension).
It should be the main file or one of the child files.
Note that further \textsf{childdoc} directives
such as |\childdocof| and |\childdocforward|
in the indicated file will be processed in this form.
The optional argument \textit{main}
passes on directly to the main file \textit{main}
while pretending to compile the child \textit{dest}.
This form behaves as if \textit{dest}
issues |\childdocof{|\textit{main}|}| right away,
and no further \textsf{childdoc} directives will be processed.

%%%%%%%%%%%%%%%%%%%%%%%%%%%%%%%%%%%%%%%%
\DescribeMacro{\...prefix}
In the alternative form |\childdocforwardprefix|,
%
\begin{center}
\begin{tabular}{l}
|\input{childdoc.def}|\\
|\childdocforwardprefix[|\textit{main}|]{|\textit{prefix}|}{|\textit{dest}|}|
\end{tabular}
\end{center}
%
the destination file is determined by a pattern
depending on the current file:
To make this work, the current file must be called
`{\textit{prefix}\hspace{0.2em}\textit{suffix}}'
with \textit{prefix} matching precisely the argument.
Processing is then passed on to the file
`{\textit{dest}\hspace{0.2em}\textit{suffix}}'.
Surely, the same effect is achieved by
directly specifying the
argument `{\textit{dest}\hspace{0.2em}\textit{suffix}}'
in the first form.
However, that requires to set up a different file
for each child. With the alternative form of the command
all these files can have exactly the same content
which simplifies setting them up and maintaining them.

For example, the following file |draft.tex|
with a compilation flag |\version| as described in \secref{sec:flags}
compiles the main document as a draft:
%
\begin{center}
\begin{tabular}{l}
|\def\version{draft}|\\
|\input{childdoc.def}|\\
|\childdocforward{|\textit{main}|}|
\end{tabular}
\end{center}
%
Likewise, the following files |final|\textit{nn}|.tex|
compile the final version of the child document
|child|\textit{nn}|.tex|:
%
\begin{center}
\begin{tabular}{l}
|\def\version{final}|\\
|\input{childdoc.def}|\\
|\childdocforwardprefix{final}{child}|
\end{tabular}
\end{center}
%

Note that when several versions of a main file and/or of each child file
are to be generated, it may be convenient to set up a |Makefile| or
shell script to automatise the process.

%%%%%%%%%%%%%%%%%%%%%%%%%%%%%%%%%%%%%%%%%%%%%%%%%%%%%%%%%%%%%%%%%%%%%%%%%%%%%%%%
\subsection{Command Line Processing}
\label{sec:commandline}

The effect of redirection files can also be achieved by invoking
the \LaTeX{} compiler with a more elaborate command line.
Most conveniently this should be done as part
of a shell script or a |Makefile|.

When using \textsf{childdoc} in the main file, the following
command lines effectively perform a redirection
(note that depending on the shell being used,
backslashes may have to be doubled: `|\|' $\to$ `|\\|'):
%
\begin{center}
|... -jobname "|\textit{target}|" |\\|"|[\textit{flags}]%
|\input{childdoc.def}\childdocforward[|\textit{main}|]{|\textit{dest}|}"|
\end{center}
%
Here \textit{target} is the name of the output file,
\textit{main} is the name of the main file
and \textit{dest} is the name of the main or child file to be processed
(all filenames without extensions).
The optional argument \textit{main} can be omitted
if \textit{main} matches \textit{dest}.
Optionally, compilation \textit{flags} can be defined via |\def| commands.
This command line makes the \TeX{} engine believe
it is compiling the file \textit{target}
whose content is specified as the latter parameter.
The provided code then forwards the processing to
\textit{main} or \textit{dest} as described in \secref{sec:forward}.

%%%%%%%%%%%%%%%%%%%%%%%%%%%%%%%%%%%%%%%%%%%%%%%%%%%%%%%%%%%%%%%%%%%%%%%%%%%%%%%%
\subsection{Include by Input}
\label{sec:input}

Including child documents by |\include| has some restrictions by design.
Most notably, the content of a child document always occupies
its own set of pages; pages cannot be shared between child documents.
Usually, this behaviour makes perfect sense
because each child document contain an essential part of the document.
However, in some situations it may be desirable to compose
a document from a collection of parts
without having mandatory page breaks between then.
For this case, the package
provides a mechanism to include parts
by |\input| which can also be processed individually.
However, by construction this mechanism
requires manual handling of the content to be output.

%%%%%%%%%%%%%%%%%%%%%%%%%%%%%%%%%%%%%%%%
\DescribeMacro{\ifchilddocmanual}
The main file should be prepared as usual, see \secref{sec:include}.
However, the document body must make a distinction
between processing of an individual part and of the main document, e.g.:
%
\begin{center}
\begin{tabular}{l}
|\ifchilddocmanual|\\
|\input{\childdocname}|\\
|\||else|\\
\textit{document body with }|\input{|\textit{part}|}|\\
|\||fi|
\end{tabular}
\end{center}
%
The conditional |\ifchilddocmanual| is true whenever
a part to be included by |\input| is being compiled,
and the name of the part is stored in |\childdocname|.

%%%%%%%%%%%%%%%%%%%%%%%%%%%%%%%%%%%%%%%%
\DescribeMacro{\childdocby}
Each part to be included by |\input| should start with:
%
\begin{center}
\begin{tabular}{l}
|\input{childdoc.def}|\\
|\childdocby{|\textit{main}|}|\\
\end{tabular}
\end{center}
%
The directive |\childdocby| is similar to |\childdocof|
described in \secref{sec:include},
but the subsequent selection of content must be done manually.
To that end, both |\ifchilddoc| and |\ifchilddocmanual|
will be true upon processing of a part,
and the name of the part is stored in |\childdocname|.
Note that |\jobname| will be set to the filename of the current part
so that each part receives an individual |.aux| file
that does not interfere with the |.aux| file(s) of the main document.
This behaviour can be altered by the alternative form
|\childdocby[*]{|\textit{main}|}| (with a non-empty optional argument)
which uses the |.aux| file of the main document
by setting |\jobname| to \textit{main}.

%%%%%%%%%%%%%%%%%%%%%%%%%%%%%%%%%%%%%%%%%%%%%%%%%%%%%%%%%%%%%%%%%%%%%%%%%%%%%%%%
\subsection{Driver Development}
\label{sec:driver}

The \textsf{childdoc} mechanism can also be use for the development
of definition files such as \LaTeX{} styles or classes.
This case differs from the above setup with multiple parts
included by |\include| in that no |\includeonly| should be invoked.
This can be achieved by starting the include file
(before |\ProvidesPackage|) with:
%
\begin{center}
\begin{tabular}{l}
|\input{childdoc.def}|\\
|\childdocforward{|\textit{main}|}|\\
\end{tabular}
\end{center}
%
or alternatively with:
%
\begin{center}
\begin{tabular}{l}
|\input{childdoc.def}|\\
|\childdocby{|\textit{main}|}|\\
\end{tabular}
\end{center}
%
Both forms have slightly different effects as described above.
The main file is prepared as usual, see \secref{sec:include}.

%%%%%%%%%%%%%%%%%%%%%%%%%%%%%%%%%%%%%%%%%%%%%%%%%%%%%%%%%%%%%%%%%%%%%%%%%%%%%%%%
\subsection{Legacy Detection}
\label{sec:detection}

The directive |\childdocmain| in the main file can detect
whether the complete document or merely a child is to be compiled
even without using the directive |\childdocof|.
This method is deprecated because it is less robust
and there is no compelling reason to use it;
it is merely provided for backward compatibility
and it may be removed in future versions.

If the detection mechanism is to be used,
it is mandatory to correctly specify
the filename of the main file as the argument of |\childdocmain|:
%
\begin{center}
\begin{tabular}{l}
|\input{childdoc.def}|\\
|\childdocmain{|\textit{main}|}|\\
\end{tabular}
\end{center}
%
If |\jobname| does not match the argument \textit{main} of |\childdocmain|,
it is assumed that |\jobname| points to the child file to be compiled.
When using |\childdocmain| with the main file specified as argument,
it suffices to start a child file
with just |\input{|\textit{main}|}|
without loading of the package and using |\childdocof|.
If instead all processing is done
with the appropriate \textsf{childdoc} directives,
the argument of \textit{main} of |\childdocmain| can be empty.

An alternative version of the command line processing described
in \secref{sec:commandline} using the detection mechanism reads:
%
\begin{center}
|... -jobname "|\textit{target}|" "|[\textit{flags}]%
[|\def\jobname{|\textit{dest}|}|]|\input{|\textit{main}|}"|
\end{center}

%%%%%%%%%%%%%%%%%%%%%%%%%%%%%%%%%%%%%%%%%%%%%%%%%%%%%%%%%%%%%%%%%%%%%%%%%%%%%%%%
\subsection{Manual Code}
\label{sec:manual}

In case one cannot be certain whether the definitions file |childdoc.def|
is installed on the target \TeX{} distribution
and one prefers not to ship it,
it is conceivable to paste a few relevant commands into the sources.

To that end, drop all statements |\input{childdoc.def}|
and perform the replacements as outlined below.
Instead of |\childdocmain{|\textit{main}|}| add the following code
to the top of the main file:
%
\begin{center}
\begin{tabular}{l}
|\||ifdefined\childdocname\endinput\||fi\newif\ifchilddoc|\\
|\edef\childdocname{\scantokens\expandafter{\jobname\noexpand}}|\\
|\def\childdocmain{|\textit{main}|}\||ifx\childdocmain\childdocname\||else|\\
|\childdoctrue\includeonly{\childdocname}\let\jobname\childdocmain\||fi|\\
\end{tabular}
\end{center}
%
Instead of |\childdocof{|\textit{main}|}| just include the main file
at the top of each child file:
%
\begin{center}
|\input{|\textit{main}|}|
\end{center}
%
A simple redirection |\childdocforward{|\textit{dest}|}| is achieved by:
%
\begin{center}
|\def\jobname{|\textit{dest}|}\input{\jobname}|
\end{center}
%
The redirection with prefix
|\childdocforwardprefix[|\textit{prefix}|]{|\textit{dest}|}|
is accomplished by:
%
\begin{center}
\begin{tabular}{l}
|{\edef\jobname{\scantokens\expandafter{\jobname\noexpand}}|\\
|\def\redirectjob |\textit{prefix}|#1~~~{\gdef\jobname{|\textit{dest}|#1}}|\\
|\expandafter\redirectjob\jobname~~~}\input{\jobname}|
\end{tabular}
\end{center}

In an alternative approach,
child documents can be compiled by a specific command line
without additional code or specific definitions:
%
\begin{center}
|... -jobname "|\textit{target}|" "|[\textit{flags}]%
|\includeonly{|\textit{dest}|}\input{|\textit{main}|}"|
\end{center}
%

%%%%%%%%%%%%%%%%%%%%%%%%%%%%%%%%%%%%%%%%%%%%%%%%%%%%%%%%%%%%%%%%%%%%%%%%%%%%%%%%
%%%%%%%%%%%%%%%%%%%%%%%%%%%%%%%%%%%%%%%%%%%%%%%%%%%%%%%%%%%%%%%%%%%%%%%%%%%%%%%%
\section{Information}

%%%%%%%%%%%%%%%%%%%%%%%%%%%%%%%%%%%%%%%%%%%%%%%%%%%%%%%%%%%%%%%%%%%%%%%%%%%%%%%%
\subsection{Copyright}

Copyright \copyright{} 2017--2018 Niklas Beisert

This work may be distributed and/or modified under the
conditions of the \LaTeX{} Project Public License, either version 1.3
of this license or (at your option) any later version.
The latest version of this license is in
  \url{http://www.latex-project.org/lppl.txt}
and version 1.3 or later is part of all distributions of \LaTeX{}
version 2005/12/01 or later.

This work has the LPPL maintenance status `maintained'.

The Current Maintainer of this work is Niklas Beisert.

This work consists of the files |README.txt|, |childdoc.ins| and |childdoc.dtx|
as well as the derived files |childdoc.def|, |cdocsamp.tex|
with |cdocsch1.tex|, |cdocsch2.tex|, |cdocspt3.tex|, |cdocspt4.tex|,
|cdocsdrf.tex|, |cdocsfn1.tex|, |cdocsfn2.tex|
as well as |childdoc.pdf|.

%%%%%%%%%%%%%%%%%%%%%%%%%%%%%%%%%%%%%%%%%%%%%%%%%%%%%%%%%%%%%%%%%%%%%%%%%%%%%%%%
\subsection{Files and Installation}

The package consists of the files:
%
\begin{center}
\begin{tabular}{ll}
    |README.txt|   & readme file \\
    |childdoc.ins| & installation file \\
    |childdoc.dtx| & source file \\
    |childdoc.def| & definition file \\
    |cdocsamp.tex| & sample main file \\
    |cdocsch1.tex| & sample include file \\
    |cdocsch2.tex| & sample include file \\
    |cdocspt3.tex| & sample part file \\
    |cdocspt4.tex| & sample part file \\
    |cdocsdrf.tex| & sample redirection file \\
    |cdocsfn1.tex| & sample redirection file \\
    |cdocsfn2.tex| & sample redirection file \\
    |childdoc.pdf| & manual
\end{tabular}
\end{center}
%
The distribution consists of the files
|README.txt|, |childdoc.ins| and |childdoc.dtx|.
%
\begin{itemize}
\item
Run (pdf)\LaTeX{} on |childdoc.dtx|
to compile the manual |childdoc.pdf| (this file).
\item
Run \LaTeX{} on |childdoc.ins| to create the definitions file |childdoc.def|
and the sample |cdocsamp.tex| with include files
|cdocsch1.tex|, |cdocsch2.tex|, |cdocspt3.tex|, |cdocspt4.tex|,
|cdocsdrf.tex|, |cdocsfn1.tex|, |cdocsfn2.tex|.
Then copy the file |childdoc.def| to an appropriate directory of your \LaTeX{}
distribution, e.g.\ \textit{texmf-root}|/tex/latex/childdoc|.
\end{itemize}

%%%%%%%%%%%%%%%%%%%%%%%%%%%%%%%%%%%%%%%%%%%%%%%%%%%%%%%%%%%%%%%%%%%%%%%%%%%%%%%%
\subsection{Related CTAN Packages}

There are several other packages which offer a similar functionality:
%
\begin{itemize}
\item
The packages
\href{http://ctan.org/pkg/docmute}{\textsf{docmute}},
\href{http://ctan.org/pkg/includex}{\textsf{includex}} and
\href{http://ctan.org/pkg/standalone}{\textsf{standalone}}
provide commands to include only the document body of
a child file thus allowing both files to be compiled individually.
\item
The packages \href{http://ctan.org/pkg/subdocs}{\textsf{subdocs}}
and \href{http://ctan.org/pkg/subfiles}{\textsf{subfiles}}
provide structures in which the main and child documents can be
encapsulated and allowing them to be compiled individually.
The inclusion mechanism is different from the conventional |\include|.
\item
The package \href{http://ctan.org/pkg/combine}{\textsf{combine}}
is an elaborate solution to combine several documents into one.
\end{itemize}
%
See also the CTAN topic \href{http://ctan.org/topic/subdocs}{\textsf{subdocs}}
for further related packages.
The present package differs from the above solutions in that
a document structure constructed with the conventional |\include| mechanism
just needs two extra commands at the top of every file
such that all constituent files can be compiled individually.

%%%%%%%%%%%%%%%%%%%%%%%%%%%%%%%%%%%%%%%%%%%%%%%%%%%%%%%%%%%%%%%%%%%%%%%%%%%%%%%%
%\subsection{Feature Suggestions}
%
%The following is a list of features which may be useful for future
%versions of this package:
%%
%\begin{itemize}
%\item
%\ldots
%\end{itemize}

%%%%%%%%%%%%%%%%%%%%%%%%%%%%%%%%%%%%%%%%%%%%%%%%%%%%%%%%%%%%%%%%%%%%%%%%%%%%%%%%
\subsection{Revision History}

%%%%%%%%%%%%%%%%%%%%%%%%%%%%%%%%%%%%%%%%
\paragraph{v2.0:} 2018/12/30

\begin{itemize}
\item
immediate forward processing
\item
added |\childdocby| mechanism
\item
manual restructured
\end{itemize}

%%%%%%%%%%%%%%%%%%%%%%%%%%%%%%%%%%%%%%%%
\paragraph{v1.6:} 2018/01/17

\begin{itemize}
\item
application for development of include files
\item
corrections to manual
\end{itemize}

%%%%%%%%%%%%%%%%%%%%%%%%%%%%%%%%%%%%%%%%
\paragraph{v1.5:} 2017/05/21

\begin{itemize}
\item
more complete structuring introduced
\item
|\childdocof| introduced
\item
|\childdoc| renamed to |\childdocmain|
\item
|\childredirect| renamed to |\childdocforward| and |\childdocforwardprefix|
and functionality expanded
\end{itemize}

%%%%%%%%%%%%%%%%%%%%%%%%%%%%%%%%%%%%%%%%
\paragraph{v1.0:} 2017/04/27

\begin{itemize}
\item
manual and install package
\item
first version published on CTAN
\end{itemize}

%%%%%%%%%%%%%%%%%%%%%%%%%%%%%%%%%%%%%%%%
\paragraph{v0.6:} 2017/04/26

\begin{itemize}
\item
redirection mechanism added
\end{itemize}

%%%%%%%%%%%%%%%%%%%%%%%%%%%%%%%%%%%%%%%%
\paragraph{v0.5:} 2017/04/26

\begin{itemize}
\item
functionality in definition file
\end{itemize}


%%%%%%%%%%%%%%%%%%%%%%%%%%%%%%%%%%%%%%%%%%%%%%%%%%%%%%%%%%%%%%%%%%%%%%%%%%%%%%%%
%%%%%%%%%%%%%%%%%%%%%%%%%%%%%%%%%%%%%%%%%%%%%%%%%%%%%%%%%%%%%%%%%%%%%%%%%%%%%%%%
%%%%%%%%%%%%%%%%%%%%%%%%%%%%%%%%%%%%%%%%%%%%%%%%%%%%%%%%%%%%%%%%%%%%%%%%%%%%%%%%
\appendix

\settowidth\MacroIndent{\rmfamily\scriptsize 000\ }

 \DocInput{childdoc.dtx}

\end{document}
%</driver>
% \fi
%
% %%%%%%%%%%%%%%%%%%%%%%%%%%%%%%%%%%%%%%%%%%%%%%%%%%%%%%%%%%%%%%%%%%%%%%%%%%%%%%
% %%%%%%%%%%%%%%%%%%%%%%%%%%%%%%%%%%%%%%%%%%%%%%%%%%%%%%%%%%%%%%%%%%%%%%%%%%%%%%
% \section{Sample}
%\iffalse
%<*samplemain>
%\fi
%
% The following presents a sample document
% with two chapters, two parts, a title page,
% a compile flag as well as three forwarding files to set the flag.
% It consists of eight |.tex| files:
% \begin{center}
% \begin{tabular}{ll}
% |cdocsamp.tex|&main file\\
% |cdocsch1.tex|&include file for chapter 1\\
% |cdocsch2.tex|&include file for chapter 2\\
% |cdocspt3.tex|&include file for part 3\\
% |cdocspt4.tex|&include file for part 4\\
% |cdocsdrf.tex|&forwarding file for main file in draft mode\\
% |cdocsfi1.tex|&forwarding file for final version of chapter 1\\
% |cdocsfi2.tex|&forwarding file for final version of chapter 2\\
% \end{tabular}
% \end{center}
% Each of the eight files can be compiled directly by the \LaTeX{} compiler.
%
% %%%%%%%%%%%%%%%%%%%%%%%%%%%%%%%%%%%%%%
% \paragraph{Main File.}
%
% The main file is called |cdocsamp.tex|.
%
% Load the \textsf{childdoc} definitions and
% declare the filename for the main document:
%    \begin{macrocode}
\input{childdoc.def}
\childdocmain{}
%    \end{macrocode}

% Optional override for |\version| flag:
%    \begin{macrocode}
%%\ifchilddoc\else\providecommand{\version}{draft}\fi
%    \end{macrocode}

% Define the default values for the |\version| flag
% (|final| for the main file and |draft| for childs):
%    \begin{macrocode}
\ifchilddoc
\providecommand{\version}{draft}
\else
\providecommand{\version}{final}
\fi
%    \end{macrocode}

% Load the standard document class:
%    \begin{macrocode}
\documentclass[12pt]{article}
%    \end{macrocode}

% Start the document body:
%    \begin{macrocode}
\begin{document}
%    \end{macrocode}

% Declare a title page.
% Print title, part of document being processed and version flag:
%    \begin{macrocode}
\addtocounter{page}{-1}
\begin{center}
{\LARGE\bfseries{}childdoc example\par}
\vspace{1cm}
\ifchilddoc
\ifchilddocmanual part\else chapter\fi:
`\childdocname' of `\childdocjob'\par
\else
main document: `\childdocjob'\par
\fi
version: \version\par
\end{center}
\newpage
%    \end{macrocode}

% Manually include selected file,
% otherwise process as usual:
%    \begin{macrocode}
\ifchilddocmanual
\section*{part `\childdocname'}
\input{\childdocname}
\else
%    \end{macrocode}

% Include the two chapters:
%    \begin{macrocode}
\include{cdocsch1}
\include{cdocsch2}
%    \end{macrocode}

% Include the two parts unless only chapters should be displayed:
%    \begin{macrocode}
\ifchilddoc\else
\section{part three}
\input{cdocspt3}
\section{part four}
\input{cdocspt4}
\fi
%    \end{macrocode}

% Process as usual until here:
%    \begin{macrocode}
\fi
%    \end{macrocode}

% End of document body:
%    \begin{macrocode}
\end{document}
%    \end{macrocode}
%\iffalse
%</samplemain>
%\fi
%
% %%%%%%%%%%%%%%%%%%%%%%%%%%%%%%%%%%%%%%
% \paragraph{Chapter Include Files.}
%
% The include files are called |cdocsch1.tex| and |cdocsch2.tex|.
%
%\iffalse
%<*samplechap1|samplechap2>
%\fi

% Optional override for |\version| flag:
%    \begin{macrocode}
%%\providecommand{\version}{final}
%    \end{macrocode}

% Include the main document:
%    \begin{macrocode}
\input{childdoc.def}
\childdocof{cdocsamp}
%    \end{macrocode}

%\iffalse
%</samplechap1|samplechap2>
%\fi
%
%\iffalse
%<*samplechap1>
%\fi
% Some text for chapter 1:
%    \begin{macrocode}
\section{one}
some text in chapter one
%    \end{macrocode}

%\iffalse
%</samplechap1>
%\fi
% Some text for chapter 2:
%\iffalse
%<*samplechap2>
%\fi
%    \begin{macrocode}
\section{two}
more text in chapter two
%    \end{macrocode}

%\iffalse
%</samplechap2>
%\fi
%
% %%%%%%%%%%%%%%%%%%%%%%%%%%%%%%%%%%%%%%
% \paragraph{Part Include Files.}
%
% The include files are called |cdocspt3.tex| and |cdocspt4.tex|.
%
%\iffalse
%<*samplepart3|samplepart4>
%\fi

% Optional override for |\version| flag:
%    \begin{macrocode}
%%\providecommand{\version}{final}
%    \end{macrocode}

% Include the main document:
%    \begin{macrocode}
\input{childdoc.def}
\childdocby{cdocsamp}
%    \end{macrocode}

%\iffalse
%</samplepart3|samplepart4>
%\fi
%
%\iffalse
%<*samplepart3>
%\fi
% Some text for part 3:
%    \begin{macrocode}
some text in part three
%    \end{macrocode}

%\iffalse
%</samplepart3>
%\fi
% Some text for part 4:
%\iffalse
%<*samplepart4>
%\fi
%    \begin{macrocode}
more text in part four
%    \end{macrocode}

%\iffalse
%</samplepart4>
%\fi
%
% %%%%%%%%%%%%%%%%%%%%%%%%%%%%%%%%%%%%%%
% \paragraph{Forwarding for a Complete Draft.}
%
% The following forwarding file |cdocsdrf.tex|
% compiles the main document in draft mode:
%\iffalse
%<*sampledraft>
%\fi
%    \begin{macrocode}
\def\version{draft}
\input{childdoc.def}
\childdocforward{cdocsamp}
%    \end{macrocode}

%\iffalse
%</sampledraft>
%\fi
%
% %%%%%%%%%%%%%%%%%%%%%%%%%%%%%%%%%%%%%%
% \paragraph{Forwarding for Final Version of the Chapters.}
%
% The following forwarding files |cdocsfn1.tex| and |cdocsfn2.tex|
% (with identical content)
% compile the final versions of the child documents
% |cdocsch1.tex| and |cdocsch2.tex|, respectively:
%\iffalse
%<*samplefinal>
%\fi
%    \begin{macrocode}
\def\version{final}
\input{childdoc.def}
\childdocforwardprefix[cdocsamp]{cdocsfn}{cdocsch}
%    \end{macrocode}

%\iffalse
%</samplefinal>
%\fi
%
% %%%%%%%%%%%%%%%%%%%%%%%%%%%%%%%%%%%%%%
% \paragraph{Command Line Processing.}
%
% The following three command lines generate the output files
% |cdocscld|, |cdocscl1| and |cdocscl2|
% which should be identical to
% |cdocsdrf|, |cdocsch1| and |cdocsfn2|, respectively:
% \begin{center}
% \begin{tabular}{l}
% |latex -jobname cdocscld \|\\
% |  "\def\version{draft}\input{childdoc.def}\childdocforward{cdocsamp}"|\\
% |latex -jobname cdocscl1 \|\\
% |  "\input{childdoc.def}\childdocforward[cdocsamp]{cdocsch1}"|\\
% |latex -jobname cdocscl2 \|\\
% |  "\def\version{final}\input{childdoc.def}\childdocforward{cdocsch2}"|
% \end{tabular}
% \end{center}
% Note that the trailing backslash on each first line
% merely continues the input to the second line
% (for convenient cut ant paste).
% Furthermore, the command |latex| can be replaced by any
% of its alternative versions such as |pdflatex|.
%
% %%%%%%%%%%%%%%%%%%%%%%%%%%%%%%%%%%%%%%%%%%%%%%%%%%%%%%%%%%%%%%%%%%%%%%%%%%%%%%
% %%%%%%%%%%%%%%%%%%%%%%%%%%%%%%%%%%%%%%%%%%%%%%%%%%%%%%%%%%%%%%%%%%%%%%%%%%%%%%
% \section{Implementation}
%\iffalse
%<*package>
%\fi
%
% This section describes the definitions file |childdoc.def|.

% The definitions cannot be loaded using |\usepackage| or |\RequirePackage|
% which has a mechanism to prevent loading a style file more than once.
% When loading the definitions by means of |\input|
% multiple instances have to be prevented manually:
%\iffalse
%This code needs to be before the `\ProvidesFile' directive
%which is defined at the beginning of this file.
%Therefore it is also placed there and commented out here.
%</package>
%<*discard>
%\fi
%    \begin{macrocode}
\ifdefined\childdocmain\endinput\fi
%    \end{macrocode}
%\iffalse
%</discard>
%<*package>
%\fi
%
% \macro{\ifchilddoc}
% \macro{\ifchilddocmanual}
% The conditional |\ifchilddoc| tells whether a
% child (true) or main (false) document is being compiled.
% The conditional |\ifchilddocmanual| tells whether
% the |\includeonly| mechanism is used (false) or
% the selection of child files must be performed manually (true).
% The definitions initialise to false:
%    \begin{macrocode}
\newif\ifchilddoc
\newif\ifchilddocmanual
%    \end{macrocode}

% \macro{\childdocname}
% \macro{\childdocjob}
% The macro |\childdocname| stores the name of the main document
% to be compiled. The macro |\childdocjob| stores the name of
% the document on which the \LaTeX{} compiler was originally invoked.
% The content of |\jobname| cannot be compared
% to filenames specified in the source due to different catcodes.
% The following code rescans |\jobname|, stores the result
% in |\childdocname| and saves a copy in |\childdocjob|:
%    \begin{macrocode}
\edef\childdocname{\scantokens\expandafter{\jobname\noexpand}}
\let\childdocjob\childdocname
%    \end{macrocode}

% \macro{\childdocdisable}
% The macro |\childdocdisable| prevents the main file
% from being processed more than once.
% At this stage, the main document command |\childdocmain|
% is assumed to be called once again where it should do nothing.
% Any subsequent call to it should prevent
% a secondary processing of the main document
% It overwrites the forwarding commands
% |\childdocof| and |\childdocforward|
% with empty macros to prevent further inclusions of the main document:
%    \begin{macrocode}
\newcommand{\childdocdisable}
{
  \renewcommand{\childdocmain}[1]{\renewcommand{\childdocmain}[1]{\endinput}}
  \renewcommand{\childdocof}[1]{}
  \renewcommand{\childdocby}[2][]{}
  \renewcommand{\childdocforward}[2][]{}
  \renewcommand{\childdocdisable}{}
}
%    \end{macrocode}

% \macro{\childdocmain}
% The macro |\childdocmain| is to be called at the top of the main file
% with nothing or the main filename (without extension) as argument.
% First, it breaks loops.
% If the argument is not empty and does not match |\childdocname|
% (which is set by the first inclusion of |childdoc.def|),
% |\ifchilddoc| is set to true, |\includeonly| is applied to the child file
% and |\jobname| is set to the main file
% (for proper handling of |.aux| files):
%    \begin{macrocode}
\newcommand{\childdocmain}[1]
{
  \childdocdisable\childdocmain{}
  \if?#1?\else
    \begingroup
      \def\childdoctmp{#1}
      \ifx\childdoctmp\childdocname
        \def\childdoctmp{}
      \else
        \def\childdoctmp
        {
          \childdoctrue
          \includeonly{\childdocname}
          \def\childdocjob{#1}
          \def\jobname{#1}
        }
      \fi
      \expandafter
    \endgroup
    \childdoctmp
  \fi
}
%    \end{macrocode}

% \macro{\childdocof}
% The command |\childdocof| redirects
% compilation to the main file |#1|.
%    \begin{macrocode}
\newcommand{\childdocof}[1]
{
  \childdocdisable
  \childdoctrue
  \includeonly{\childdocname}
  \def\jobname{#1}
  \def\childdocjob{#1}
  \input{#1}
}
%    \end{macrocode}

% \macro{\childdocby}
% The command |\childdocby| ....
%    \begin{macrocode}
\newcommand{\childdocby}[2][]
{
  \childdocdisable
  \childdoctrue
  \childdocmanualtrue
  \if?#1?\else
    \def\jobname{#2}
  \fi
  \def\childdocjob{#2}
  \input{#2}
  \endinput
}
%    \end{macrocode}

% \macro{\childdocforward}
% The command |\childdocforward| redirects
% compilation to the main file or
% (if the optional argument is given) a child file.
% Parameters are set as if the main file
% or a child file starting with |\childdocof| was compiled.
% Then compilation is handed over to the main file:
%    \begin{macrocode}
\newcommand{\childdocforward}[2][]
{
  \begingroup
    \if?#1?
      \def\childdoctmp
      {
        \def\childdocname{#2}
        \def\childdocjob{#2}
        \def\jobname{#2}
        \input{#2}
        \endinput
      }
    \else
      \def\childdoctmp
      {
        \childdocdisable
        \def\childdocname{#2}
        \childdoctrue
        \includeonly{#2}
        \def\childdocjob{#1}
        \def\jobname{#1}
        \input{#1}
        \endinput
      }
    \fi
    \expandafter
  \endgroup
  \childdoctmp
}
%    \end{macrocode}

% \macro{\childdocforwardprefix}
% The command |\childdocforwardprefix| redirects
% compilation to the main or a child file by means of a pattern.
% The prefix |#1| in the current filename is replaced by |#2|
% and the suffix of the current filename is kept
% (it is assumed that the filename does not contain the substring `|~~~|'
% which is used as a delimiter).
% Compilation is handed over to the new file by |\childdocforward|:
%    \begin{macrocode}
\newcommand{\childdocforwardprefix}[3][]
{
  \begingroup
    \def\childdocextract #2##1~~~{\def\childdoctmp{\childdocforward[#1]{#3##1}}}
    \expandafter\childdocextract\childdocname~~~
    \expandafter
  \endgroup
  \childdoctmp
}
%    \end{macrocode}

% \macro{\childdoc}
% The deprecated macro |\childdoc| is a legacy version of |\childdocmain|:
%    \begin{macrocode}
\newcommand{\childdoc}{\childdocmain}
%    \end{macrocode}

% \macro{\childdocredirect}
% The deprecated macro |\childdocredirect| is a legacy version
% of |\childdocforward| and |\childdocforwardprefix|:
%    \begin{macrocode}
\newcommand{\childdocredirect}[2][]
{
  \begingroup
    \if?#1?
      \def\childdoctmp{\childdocforward{#2}}
    \else
      \def\childdoctmp{\childdocforwardprefix{#1}{#2}}
    \fi
    \expandafter
  \endgroup
  \childdoctmp
}
%    \end{macrocode}

%\iffalse
%</package>
%\fi
%
\endinput
|
and perform the replacements as outlined below.
Instead of |\childdocmain{|\textit{main}|}| add the following code
to the top of the main file:
%
\begin{center}
\begin{tabular}{l}
|\||ifdefined\childdocname\endinput\||fi\newif\ifchilddoc|\\
|\edef\childdocname{\scantokens\expandafter{\jobname\noexpand}}|\\
|\def\childdocmain{|\textit{main}|}\||ifx\childdocmain\childdocname\||else|\\
|\childdoctrue\includeonly{\childdocname}\let\jobname\childdocmain\||fi|\\
\end{tabular}
\end{center}
%
Instead of |\childdocof{|\textit{main}|}| just include the main file
at the top of each child file:
%
\begin{center}
|\input{|\textit{main}|}|
\end{center}
%
A simple redirection |\childdocforward{|\textit{dest}|}| is achieved by:
%
\begin{center}
|\def\jobname{|\textit{dest}|}\input{\jobname}|
\end{center}
%
The redirection with prefix
|\childdocforwardprefix[|\textit{prefix}|]{|\textit{dest}|}|
is accomplished by:
%
\begin{center}
\begin{tabular}{l}
|{\edef\jobname{\scantokens\expandafter{\jobname\noexpand}}|\\
|\def\redirectjob |\textit{prefix}|#1~~~{\gdef\jobname{|\textit{dest}|#1}}|\\
|\expandafter\redirectjob\jobname~~~}\input{\jobname}|
\end{tabular}
\end{center}

In an alternative approach,
child documents can be compiled by a specific command line
without additional code or specific definitions:
%
\begin{center}
|... -jobname "|\textit{target}|" "|[\textit{flags}]%
|\includeonly{|\textit{dest}|}\input{|\textit{main}|}"|
\end{center}
%

%%%%%%%%%%%%%%%%%%%%%%%%%%%%%%%%%%%%%%%%%%%%%%%%%%%%%%%%%%%%%%%%%%%%%%%%%%%%%%%%
%%%%%%%%%%%%%%%%%%%%%%%%%%%%%%%%%%%%%%%%%%%%%%%%%%%%%%%%%%%%%%%%%%%%%%%%%%%%%%%%
\section{Information}

%%%%%%%%%%%%%%%%%%%%%%%%%%%%%%%%%%%%%%%%%%%%%%%%%%%%%%%%%%%%%%%%%%%%%%%%%%%%%%%%
\subsection{Copyright}

Copyright \copyright{} 2017--2018 Niklas Beisert

This work may be distributed and/or modified under the
conditions of the \LaTeX{} Project Public License, either version 1.3
of this license or (at your option) any later version.
The latest version of this license is in
  \url{http://www.latex-project.org/lppl.txt}
and version 1.3 or later is part of all distributions of \LaTeX{}
version 2005/12/01 or later.

This work has the LPPL maintenance status `maintained'.

The Current Maintainer of this work is Niklas Beisert.

This work consists of the files |README.txt|, |childdoc.ins| and |childdoc.dtx|
as well as the derived files |childdoc.def|, |cdocsamp.tex|
with |cdocsch1.tex|, |cdocsch2.tex|, |cdocspt3.tex|, |cdocspt4.tex|,
|cdocsdrf.tex|, |cdocsfn1.tex|, |cdocsfn2.tex|
as well as |childdoc.pdf|.

%%%%%%%%%%%%%%%%%%%%%%%%%%%%%%%%%%%%%%%%%%%%%%%%%%%%%%%%%%%%%%%%%%%%%%%%%%%%%%%%
\subsection{Files and Installation}

The package consists of the files:
%
\begin{center}
\begin{tabular}{ll}
    |README.txt|   & readme file \\
    |childdoc.ins| & installation file \\
    |childdoc.dtx| & source file \\
    |childdoc.def| & definition file \\
    |cdocsamp.tex| & sample main file \\
    |cdocsch1.tex| & sample include file \\
    |cdocsch2.tex| & sample include file \\
    |cdocspt3.tex| & sample part file \\
    |cdocspt4.tex| & sample part file \\
    |cdocsdrf.tex| & sample redirection file \\
    |cdocsfn1.tex| & sample redirection file \\
    |cdocsfn2.tex| & sample redirection file \\
    |childdoc.pdf| & manual
\end{tabular}
\end{center}
%
The distribution consists of the files
|README.txt|, |childdoc.ins| and |childdoc.dtx|.
%
\begin{itemize}
\item
Run (pdf)\LaTeX{} on |childdoc.dtx|
to compile the manual |childdoc.pdf| (this file).
\item
Run \LaTeX{} on |childdoc.ins| to create the definitions file |childdoc.def|
and the sample |cdocsamp.tex| with include files
|cdocsch1.tex|, |cdocsch2.tex|, |cdocspt3.tex|, |cdocspt4.tex|,
|cdocsdrf.tex|, |cdocsfn1.tex|, |cdocsfn2.tex|.
Then copy the file |childdoc.def| to an appropriate directory of your \LaTeX{}
distribution, e.g.\ \textit{texmf-root}|/tex/latex/childdoc|.
\end{itemize}

%%%%%%%%%%%%%%%%%%%%%%%%%%%%%%%%%%%%%%%%%%%%%%%%%%%%%%%%%%%%%%%%%%%%%%%%%%%%%%%%
\subsection{Related CTAN Packages}

There are several other packages which offer a similar functionality:
%
\begin{itemize}
\item
The packages
\href{http://ctan.org/pkg/docmute}{\textsf{docmute}},
\href{http://ctan.org/pkg/includex}{\textsf{includex}} and
\href{http://ctan.org/pkg/standalone}{\textsf{standalone}}
provide commands to include only the document body of
a child file thus allowing both files to be compiled individually.
\item
The packages \href{http://ctan.org/pkg/subdocs}{\textsf{subdocs}}
and \href{http://ctan.org/pkg/subfiles}{\textsf{subfiles}}
provide structures in which the main and child documents can be
encapsulated and allowing them to be compiled individually.
The inclusion mechanism is different from the conventional |\include|.
\item
The package \href{http://ctan.org/pkg/combine}{\textsf{combine}}
is an elaborate solution to combine several documents into one.
\end{itemize}
%
See also the CTAN topic \href{http://ctan.org/topic/subdocs}{\textsf{subdocs}}
for further related packages.
The present package differs from the above solutions in that
a document structure constructed with the conventional |\include| mechanism
just needs two extra commands at the top of every file
such that all constituent files can be compiled individually.

%%%%%%%%%%%%%%%%%%%%%%%%%%%%%%%%%%%%%%%%%%%%%%%%%%%%%%%%%%%%%%%%%%%%%%%%%%%%%%%%
%\subsection{Feature Suggestions}
%
%The following is a list of features which may be useful for future
%versions of this package:
%%
%\begin{itemize}
%\item
%\ldots
%\end{itemize}

%%%%%%%%%%%%%%%%%%%%%%%%%%%%%%%%%%%%%%%%%%%%%%%%%%%%%%%%%%%%%%%%%%%%%%%%%%%%%%%%
\subsection{Revision History}

%%%%%%%%%%%%%%%%%%%%%%%%%%%%%%%%%%%%%%%%
\paragraph{v2.0:} 2018/12/30

\begin{itemize}
\item
immediate forward processing
\item
added |\childdocby| mechanism
\item
manual restructured
\end{itemize}

%%%%%%%%%%%%%%%%%%%%%%%%%%%%%%%%%%%%%%%%
\paragraph{v1.6:} 2018/01/17

\begin{itemize}
\item
application for development of include files
\item
corrections to manual
\end{itemize}

%%%%%%%%%%%%%%%%%%%%%%%%%%%%%%%%%%%%%%%%
\paragraph{v1.5:} 2017/05/21

\begin{itemize}
\item
more complete structuring introduced
\item
|\childdocof| introduced
\item
|\childdoc| renamed to |\childdocmain|
\item
|\childredirect| renamed to |\childdocforward| and |\childdocforwardprefix|
and functionality expanded
\end{itemize}

%%%%%%%%%%%%%%%%%%%%%%%%%%%%%%%%%%%%%%%%
\paragraph{v1.0:} 2017/04/27

\begin{itemize}
\item
manual and install package
\item
first version published on CTAN
\end{itemize}

%%%%%%%%%%%%%%%%%%%%%%%%%%%%%%%%%%%%%%%%
\paragraph{v0.6:} 2017/04/26

\begin{itemize}
\item
redirection mechanism added
\end{itemize}

%%%%%%%%%%%%%%%%%%%%%%%%%%%%%%%%%%%%%%%%
\paragraph{v0.5:} 2017/04/26

\begin{itemize}
\item
functionality in definition file
\end{itemize}


%%%%%%%%%%%%%%%%%%%%%%%%%%%%%%%%%%%%%%%%%%%%%%%%%%%%%%%%%%%%%%%%%%%%%%%%%%%%%%%%
%%%%%%%%%%%%%%%%%%%%%%%%%%%%%%%%%%%%%%%%%%%%%%%%%%%%%%%%%%%%%%%%%%%%%%%%%%%%%%%%
%%%%%%%%%%%%%%%%%%%%%%%%%%%%%%%%%%%%%%%%%%%%%%%%%%%%%%%%%%%%%%%%%%%%%%%%%%%%%%%%
\appendix

\settowidth\MacroIndent{\rmfamily\scriptsize 000\ }

 \DocInput{childdoc.dtx}

\end{document}
%</driver>
% \fi
%
% %%%%%%%%%%%%%%%%%%%%%%%%%%%%%%%%%%%%%%%%%%%%%%%%%%%%%%%%%%%%%%%%%%%%%%%%%%%%%%
% %%%%%%%%%%%%%%%%%%%%%%%%%%%%%%%%%%%%%%%%%%%%%%%%%%%%%%%%%%%%%%%%%%%%%%%%%%%%%%
% \section{Sample}
%\iffalse
%<*samplemain>
%\fi
%
% The following presents a sample document
% with two chapters, two parts, a title page,
% a compile flag as well as three forwarding files to set the flag.
% It consists of eight |.tex| files:
% \begin{center}
% \begin{tabular}{ll}
% |cdocsamp.tex|&main file\\
% |cdocsch1.tex|&include file for chapter 1\\
% |cdocsch2.tex|&include file for chapter 2\\
% |cdocspt3.tex|&include file for part 3\\
% |cdocspt4.tex|&include file for part 4\\
% |cdocsdrf.tex|&forwarding file for main file in draft mode\\
% |cdocsfi1.tex|&forwarding file for final version of chapter 1\\
% |cdocsfi2.tex|&forwarding file for final version of chapter 2\\
% \end{tabular}
% \end{center}
% Each of the eight files can be compiled directly by the \LaTeX{} compiler.
%
% %%%%%%%%%%%%%%%%%%%%%%%%%%%%%%%%%%%%%%
% \paragraph{Main File.}
%
% The main file is called |cdocsamp.tex|.
%
% Load the \textsf{childdoc} definitions and
% declare the filename for the main document:
%    \begin{macrocode}
% \iffalse
%
% childdoc.dtx Copyright (C) 2017-2018 Niklas Beisert
%
% This work may be distributed and/or modified under the
% conditions of the LaTeX Project Public License, either version 1.3
% of this license or (at your option) any later version.
% The latest version of this license is in
%   http://www.latex-project.org/lppl.txt
% and version 1.3 or later is part of all distributions of LaTeX
% version 2005/12/01 or later.
%
% This work has the LPPL maintenance status `maintained'.
%
% The Current Maintainer of this work is Niklas Beisert.
%
% This work consists of the files childdoc.dtx and childdoc.ins
% and the derived files childdoc.def and cdocsamp.tex with
% cdocsch1.tex, cdocsch2.tex, cdocsdrf.tex, cdocsfn1.tex, cdocsfn2.tex.
%
%<package>\ifdefined\childdocmain\endinput\fi
%<package>\ProvidesFile{childdoc.def}[2018/12/30 v2.0 child document driver]
%<samplemain>\ProvidesFile{cdocsamp.tex}[2018/12/30 v2.0 sample for childdoc]
%<*driver>
%\ProvidesFile{childdoc.drv}[2018/12/30 v2.0 childdoc reference manual file]
\PassOptionsToClass{10pt,a4paper}{article}
\documentclass{ltxdoc}

\usepackage[margin=35mm]{geometry}
\usepackage{hyperref}
\usepackage{hyperxmp}
\usepackage[usenames]{color}

\hypersetup{colorlinks=true}
\hypersetup{pdfstartview=FitH}
\hypersetup{pdfpagemode=UseNone}
\hypersetup{pdfsource={}}
\hypersetup{pdflang={en-UK}}
\hypersetup{pdfcopyright={Copyright 2017-2018 Niklas Beisert.
  This work may be distributed and/or modified under the
  conditions of the LaTeX Project Public License, either version 1.3
  of this license or (at your option) any later version.}}
\hypersetup{pdflicenseurl={http://www.latex-project.org/lppl.txt}}
\hypersetup{pdfcontactaddress={ETH Zurich, ITP, HIT K,
  Wolfgang-Pauli-Strasse 27}}
\hypersetup{pdfcontactpostcode={8093}}
\hypersetup{pdfcontactcity={Zurich}}
\hypersetup{pdfcontactcountry={Switzerland}}
\hypersetup{pdfcontactemail={nbeisert@itp.phys.ethz.ch}}
\hypersetup{pdfcontacturl={http://people.phys.ethz.ch/\xmptilde nbeisert/}}

\newcommand{\secref}[1]{\hyperref[#1]{section \ref*{#1}}}

\parskip1ex
\parindent0pt
\let\olditemize\itemize
\def\itemize{\olditemize\parskip0pt}

\begin{document}

\title{The \textsf{childdoc} Package}
\hypersetup{pdftitle={The childdoc Package}}
\author{Niklas Beisert\\[2ex]
  Institut f\"ur Theoretische Physik\\
  Eidgen\"ossische Technische Hochschule Z\"urich\\
  Wolfgang-Pauli-Strasse 27, 8093 Z\"urich, Switzerland\\[1ex]
  \href{mailto:nbeisert@itp.phys.ethz.ch}
  {\texttt{nbeisert@itp.phys.ethz.ch}}}
\hypersetup{pdfauthor={Niklas Beisert}}
\hypersetup{pdfsubject={Manual for the LaTeX2e Package childdoc}}
\date{30 December 2018, \textsf{v2.0}}
\maketitle

\begin{abstract}\noindent
\textsf{childdoc} is a \LaTeXe{} package
that enables the direct compilation
of document sections included by |\include|
to individual files.
\end{abstract}

\begingroup
\parskip0ex
\tableofcontents
\endgroup

%%%%%%%%%%%%%%%%%%%%%%%%%%%%%%%%%%%%%%%%%%%%%%%%%%%%%%%%%%%%%%%%%%%%%%%%%%%%%%%%
%%%%%%%%%%%%%%%%%%%%%%%%%%%%%%%%%%%%%%%%%%%%%%%%%%%%%%%%%%%%%%%%%%%%%%%%%%%%%%%%
\section{Introduction}

\LaTeX{} provides a mechanism to structure a large document (such as a book)
into a main file and several child files (containing the chapters)
using the |\include| command.
This mechanism is beneficial for documents
which span hundreds of pages in order to
make the source file(s) more manageable.
Moreover, compilation can be restricted to
selected child files by means of the |\includeonly| command.
The latter feature can be used to reduce the compilation time while editing
(this was significantly more useful in the earlier days of \LaTeX{})
or to generate a smaller document which is easier to navigate.
Another application of |\includeonly| is to generate
documents consisting of selected parts of the complete document.

However, there are a few drawbacks of the plain |\include| mechanism:
\begin{itemize}
\item
The child files cannot be compiled on their own,
they can only be compiled via the main file.
A naive editing environment
(such as a text editor with an option
to have the current file processed by \LaTeX)
may require one to switch to the main file before compiling;
attempting to compile the child file produces errors.
\item
The main file must be modified (each time)
to adjust the |\includeonly| command
to the present needs. This easily leaves the main file in a messy state.
\item
The generated document will always carry the filename
of the main document. This is inconvenient if
several child files are to be compiled and
to be kept for distribution.
\end{itemize}

The present package provides a simple interface
to make child files individually compilable by \LaTeX{}.
Compiling a child file then has the same effect as compiling
the main file with an |\includeonly| command
to select the appropriate child.
Moreover the generated document will carry the name of the child
rather than the main file.
This resolves all three above issues.

This feature is meant to make the editing of books,
thesis documents and lecture notes somewhat more convenient.
However, the package can also be used efficiently for
composing a series of documents (such as exercise sheets)
which are typically distributed individually.
It then assists the author in generating the individual documents
(potentially in different versions)
as well as a document containing the collected series.
Another application is in developing style files
or other kinds of included material
where compilation of the style file could redirect
to a sample or test file.

%%%%%%%%%%%%%%%%%%%%%%%%%%%%%%%%%%%%%%%%%%%%%%%%%%%%%%%%%%%%%%%%%%%%%%%%%%%%%%%%
%%%%%%%%%%%%%%%%%%%%%%%%%%%%%%%%%%%%%%%%%%%%%%%%%%%%%%%%%%%%%%%%%%%%%%%%%%%%%%%%
\section{Usage}

First of all, the package \textsf{childdoc} is \emph{not} a standard
\LaTeXe{} |.sty| style file! Therefore it needs to be invoked in
a non-standard way.

%%%%%%%%%%%%%%%%%%%%%%%%%%%%%%%%%%%%%%%%%%%%%%%%%%%%%%%%%%%%%%%%%%%%%%%%%%%%%%%%
\subsection{Included Files}
\label{sec:include}

%%%%%%%%%%%%%%%%%%%%%%%%%%%%%%%%%%%%%%%%
\DescribeMacro{\childdocmain}
To use the package, add the commands
\begin{center}
\begin{tabular}{l}
|\input{childdoc.def}|\\
|\childdocmain{}|\\
\end{tabular}
\end{center}
at the very top of the main \LaTeX{} file,
in particular \emph{before} the |\documentclass| statement!
The argument of |\childdocmain| should be left empty
(but it must be present).

%%%%%%%%%%%%%%%%%%%%%%%%%%%%%%%%%%%%%%%%
\DescribeMacro{\childdocof}
Furthermore, add the commands
\begin{center}
\begin{tabular}{l}
|\input{childdoc.def}|\\
|\childdocof{|\textit{main}|}|\\
\end{tabular}
\end{center}
at the top of every child file \textit{child}
which is included by |\include{|\textit{child}|}|
from within the main file
(or at least for those files to be compiled individually).
The argument \textit{main} must be the filename of the main file.

There are a couple of
considerations in setting up the main and child documents:

%%%%%%%%%%%%%%%%%%%%%%%%%%%%%%%%%%%%%%%%
\paragraph{Restrictions.}

Please note the following restrictions:
\begin{itemize}
\item
|\childdocmain| must be called with one argument \textit{main}
to ensure compatibility with earlier version of the package.
It must either be empty (|\childdocmain{}|)
or precisely match the filename of the main file in which it is specified.
See \secref{sec:detection} for further information.
\item
The filename \textit{main} must be specified without the |.tex| extension.
\item
The filename \textit{main} is case sensitive
(even in case-insensitive file systems)
due to internal string comparison.
\item
The argument \textit{main} should be fully expanded, it cannot be a macro.
\item
Subdirectories and special characters should be avoided in filenames.
\item
The command |\childdocmain{|\textit{main}|}| must be followed by a whitespace.
It should not be followed immediately by another command
or by a comment mark `|%|'.
This is because the \TeX{} parser reads the token immediately following
the argument of |\childdocmain| and puts it
at the beginning of every child section;
however, a white\-space is ignored.
\end{itemize}

%%%%%%%%%%%%%%%%%%%%%%%%%%%%%%%%%%%%%%%%
\paragraph{Content of Main File.}

It is advisable to place all content in the child files included by |\include|.
Any output contained in the main file will appear in all child documents
unless suppressed manually;
it cannot be suppressed automatically by the |\includeonly| directive
and thus should normally be avoided.
A method to include some content in the main file
by means of conditional processing is described in \secref{sec:conditional}.

%%%%%%%%%%%%%%%%%%%%%%%%%%%%%%%%%%%%%%%%
\paragraph{Page Numbering.}

When only a part of the document is compiled,
the appropriate numbering of pages
(as well as other status parameters)
is determined from the |.aux| files.
The latter contain information from previous passes.
However this information needs to propagate through
all intermediate child documents.
Therefore the page numbering in child documents may well
be inconsistent until the complete document is compiled at least once.

A useful (if unconventional) way to always ensure a consistent
page numbering is to restart the numbering in each child document
and denote the pages by `\textit{child}|.|\textit{page}'
where \textit{child} represents the chapter/section number of the child file.
This can be achieved by the command
|\numberwithin{page}{|\textit{child}|}|
of the \textsf{amsmath} package
where \textit{child} can be |chapter| or |section|
depending on the chosen structuring.
Alternatively, one can modify the macro |\thepage| appropriately
and reset the counter |page| at the start of each child file.

%%%%%%%%%%%%%%%%%%%%%%%%%%%%%%%%%%%%%%%%%%%%%%%%%%%%%%%%%%%%%%%%%%%%%%%%%%%%%%%%
\subsection{Conditional Processing}
\label{sec:conditional}

The package provides a mechanism to compile different versions
of a document. To customise the versions further some conditional processing
can come in handy to distinguish which version is being compiled.
The package provides two macros to describe the compilation context:

%%%%%%%%%%%%%%%%%%%%%%%%%%%%%%%%%%%%%%%%
\DescribeMacro{\ifchilddoc}
The conditional |\ifchilddoc| distinguishes between the compilation of
child documents and the main document:
%
\begin{center}
|\ifchilddoc |\textit{child-code}| |[|\||else |\textit{main-code}]| \||fi|
\end{center}

%%%%%%%%%%%%%%%%%%%%%%%%%%%%%%%%%%%%%%%%
\DescribeMacro{\childdocname}
\DescribeMacro{\childdocjob}
The macro |\childdocname| contains the filename (without extension)
of the main or child file being processed.
Note that |\childdocjob| will always contain the name of the main file.

%%%%%%%%%%%%%%%%%%%%%%%%%%%%%%%%%%%%%%%%
\paragraph{Title Page.}

Conditional processing can be used to include a title or banner page
in the main document when proper precautions are taken.
Importantly, the code in the main file should ensure that the page counter
(as well as other status parameters which are stored in the |.aux| files)
takes the same value after the conditional processing.
Otherwise the page numbers may take divergent values
depending on which part is compiled.

For example, a title page could be declared by:
%
\begin{center}
\begin{tabular}{l}
|\ifchilddoc\||else|\\
|\addtocounter{page}{-1}|\\
\textit{code for title page}\\
|\newpage|\\
|\||fi|
\end{tabular}
\end{center}
%
A banner page for the child documents can be generated by:
%
\begin{center}
\begin{tabular}{l}
|\ifchilddoc|\\
|\addtocounter{page}{-1}|\\
\textit{code for banner page}\\
|\newpage|\\
|\||fi|
\end{tabular}
\end{center}
%
Here one could write a message such as:
\begin{center}
|This is the part \childdocname{} of \childdocjob{}.|
\end{center}

%%%%%%%%%%%%%%%%%%%%%%%%%%%%%%%%%%%%%%%%%%%%%%%%%%%%%%%%%%%%%%%%%%%%%%%%%%%%%%%%
\subsection{Flags}
\label{sec:flags}

The package makes it easy to generate different versions
of the main or child documents.
To this end compilation flags can be defined
and assigned different default values.
They will be particularly useful in conjunction
with the forwarding mechanism described in \secref{sec:forward}.

For example, it may be useful to have a flag |\version|
which can be set to |draft| or |final|.
The document source will contain some conditional code
depending on the value of |\version|.
Suppose further, the flag should default to |final| for the main file
and to |draft| for child files
which is a natural assignment for editing the document.
This is achieved by placing the following code
in the preamble of the main document
(below the |\childdocmain| directive):
%
\begin{center}
\begin{tabular}{l}
|\ifchilddoc|\\
|\providecommand{\version}{draft}|\\
|\||else|\\
|\providecommand{\version}{final}|\\
|\||fi|
\end{tabular}
\end{center}
%
The definition by |\providecommand| makes sure
that previous definitions are not overwritten.
Further statements |\providecommand{\version}{...}|
can thus be added before the above code to override it.

For the main file, one might add a line
(between |\childdocmain| and the above block)
%
\begin{center}
|%\ifchilddoc\||else\providecommand{\version}{draft}\||fi|
\end{center}
%
which can be uncommented to produce a draft version.
Likewise one can add a line to the very top of a child file
(above the |\childdocof{|\textit{main}|}| directive)
%
\begin{center}
|%\providecommand{\version}{final}|
\end{center}
%
which can be uncommented to produce the final version of this child document.

%%%%%%%%%%%%%%%%%%%%%%%%%%%%%%%%%%%%%%%%%%%%%%%%%%%%%%%%%%%%%%%%%%%%%%%%%%%%%%%%
\subsection{Forwarding}
\label{sec:forward}

Different versions of the main or child documents
using compilation flags as described in \secref{sec:flags}
can be (permanently) stored in different files
for convenient compilation, viewing and distribution.
To this end, the package defines a command
to pass on compilation to a different file:

%%%%%%%%%%%%%%%%%%%%%%%%%%%%%%%%%%%%%%%%
\DescribeMacro{\childdocforward}
The command |\childdocforward| redirects processing to
another source file:
%
\begin{center}
\begin{tabular}{l}
|\input{childdoc.def}|\\
|\childdocforward[|\textit{main}|]{|\textit{dest}|}|\\
\end{tabular}
\end{center}
%
The argument \textit{dest} is the destination file
(without extension).
It should be the main file or one of the child files.
Note that further \textsf{childdoc} directives
such as |\childdocof| and |\childdocforward|
in the indicated file will be processed in this form.
The optional argument \textit{main}
passes on directly to the main file \textit{main}
while pretending to compile the child \textit{dest}.
This form behaves as if \textit{dest}
issues |\childdocof{|\textit{main}|}| right away,
and no further \textsf{childdoc} directives will be processed.

%%%%%%%%%%%%%%%%%%%%%%%%%%%%%%%%%%%%%%%%
\DescribeMacro{\...prefix}
In the alternative form |\childdocforwardprefix|,
%
\begin{center}
\begin{tabular}{l}
|\input{childdoc.def}|\\
|\childdocforwardprefix[|\textit{main}|]{|\textit{prefix}|}{|\textit{dest}|}|
\end{tabular}
\end{center}
%
the destination file is determined by a pattern
depending on the current file:
To make this work, the current file must be called
`{\textit{prefix}\hspace{0.2em}\textit{suffix}}'
with \textit{prefix} matching precisely the argument.
Processing is then passed on to the file
`{\textit{dest}\hspace{0.2em}\textit{suffix}}'.
Surely, the same effect is achieved by
directly specifying the
argument `{\textit{dest}\hspace{0.2em}\textit{suffix}}'
in the first form.
However, that requires to set up a different file
for each child. With the alternative form of the command
all these files can have exactly the same content
which simplifies setting them up and maintaining them.

For example, the following file |draft.tex|
with a compilation flag |\version| as described in \secref{sec:flags}
compiles the main document as a draft:
%
\begin{center}
\begin{tabular}{l}
|\def\version{draft}|\\
|\input{childdoc.def}|\\
|\childdocforward{|\textit{main}|}|
\end{tabular}
\end{center}
%
Likewise, the following files |final|\textit{nn}|.tex|
compile the final version of the child document
|child|\textit{nn}|.tex|:
%
\begin{center}
\begin{tabular}{l}
|\def\version{final}|\\
|\input{childdoc.def}|\\
|\childdocforwardprefix{final}{child}|
\end{tabular}
\end{center}
%

Note that when several versions of a main file and/or of each child file
are to be generated, it may be convenient to set up a |Makefile| or
shell script to automatise the process.

%%%%%%%%%%%%%%%%%%%%%%%%%%%%%%%%%%%%%%%%%%%%%%%%%%%%%%%%%%%%%%%%%%%%%%%%%%%%%%%%
\subsection{Command Line Processing}
\label{sec:commandline}

The effect of redirection files can also be achieved by invoking
the \LaTeX{} compiler with a more elaborate command line.
Most conveniently this should be done as part
of a shell script or a |Makefile|.

When using \textsf{childdoc} in the main file, the following
command lines effectively perform a redirection
(note that depending on the shell being used,
backslashes may have to be doubled: `|\|' $\to$ `|\\|'):
%
\begin{center}
|... -jobname "|\textit{target}|" |\\|"|[\textit{flags}]%
|\input{childdoc.def}\childdocforward[|\textit{main}|]{|\textit{dest}|}"|
\end{center}
%
Here \textit{target} is the name of the output file,
\textit{main} is the name of the main file
and \textit{dest} is the name of the main or child file to be processed
(all filenames without extensions).
The optional argument \textit{main} can be omitted
if \textit{main} matches \textit{dest}.
Optionally, compilation \textit{flags} can be defined via |\def| commands.
This command line makes the \TeX{} engine believe
it is compiling the file \textit{target}
whose content is specified as the latter parameter.
The provided code then forwards the processing to
\textit{main} or \textit{dest} as described in \secref{sec:forward}.

%%%%%%%%%%%%%%%%%%%%%%%%%%%%%%%%%%%%%%%%%%%%%%%%%%%%%%%%%%%%%%%%%%%%%%%%%%%%%%%%
\subsection{Include by Input}
\label{sec:input}

Including child documents by |\include| has some restrictions by design.
Most notably, the content of a child document always occupies
its own set of pages; pages cannot be shared between child documents.
Usually, this behaviour makes perfect sense
because each child document contain an essential part of the document.
However, in some situations it may be desirable to compose
a document from a collection of parts
without having mandatory page breaks between then.
For this case, the package
provides a mechanism to include parts
by |\input| which can also be processed individually.
However, by construction this mechanism
requires manual handling of the content to be output.

%%%%%%%%%%%%%%%%%%%%%%%%%%%%%%%%%%%%%%%%
\DescribeMacro{\ifchilddocmanual}
The main file should be prepared as usual, see \secref{sec:include}.
However, the document body must make a distinction
between processing of an individual part and of the main document, e.g.:
%
\begin{center}
\begin{tabular}{l}
|\ifchilddocmanual|\\
|\input{\childdocname}|\\
|\||else|\\
\textit{document body with }|\input{|\textit{part}|}|\\
|\||fi|
\end{tabular}
\end{center}
%
The conditional |\ifchilddocmanual| is true whenever
a part to be included by |\input| is being compiled,
and the name of the part is stored in |\childdocname|.

%%%%%%%%%%%%%%%%%%%%%%%%%%%%%%%%%%%%%%%%
\DescribeMacro{\childdocby}
Each part to be included by |\input| should start with:
%
\begin{center}
\begin{tabular}{l}
|\input{childdoc.def}|\\
|\childdocby{|\textit{main}|}|\\
\end{tabular}
\end{center}
%
The directive |\childdocby| is similar to |\childdocof|
described in \secref{sec:include},
but the subsequent selection of content must be done manually.
To that end, both |\ifchilddoc| and |\ifchilddocmanual|
will be true upon processing of a part,
and the name of the part is stored in |\childdocname|.
Note that |\jobname| will be set to the filename of the current part
so that each part receives an individual |.aux| file
that does not interfere with the |.aux| file(s) of the main document.
This behaviour can be altered by the alternative form
|\childdocby[*]{|\textit{main}|}| (with a non-empty optional argument)
which uses the |.aux| file of the main document
by setting |\jobname| to \textit{main}.

%%%%%%%%%%%%%%%%%%%%%%%%%%%%%%%%%%%%%%%%%%%%%%%%%%%%%%%%%%%%%%%%%%%%%%%%%%%%%%%%
\subsection{Driver Development}
\label{sec:driver}

The \textsf{childdoc} mechanism can also be use for the development
of definition files such as \LaTeX{} styles or classes.
This case differs from the above setup with multiple parts
included by |\include| in that no |\includeonly| should be invoked.
This can be achieved by starting the include file
(before |\ProvidesPackage|) with:
%
\begin{center}
\begin{tabular}{l}
|\input{childdoc.def}|\\
|\childdocforward{|\textit{main}|}|\\
\end{tabular}
\end{center}
%
or alternatively with:
%
\begin{center}
\begin{tabular}{l}
|\input{childdoc.def}|\\
|\childdocby{|\textit{main}|}|\\
\end{tabular}
\end{center}
%
Both forms have slightly different effects as described above.
The main file is prepared as usual, see \secref{sec:include}.

%%%%%%%%%%%%%%%%%%%%%%%%%%%%%%%%%%%%%%%%%%%%%%%%%%%%%%%%%%%%%%%%%%%%%%%%%%%%%%%%
\subsection{Legacy Detection}
\label{sec:detection}

The directive |\childdocmain| in the main file can detect
whether the complete document or merely a child is to be compiled
even without using the directive |\childdocof|.
This method is deprecated because it is less robust
and there is no compelling reason to use it;
it is merely provided for backward compatibility
and it may be removed in future versions.

If the detection mechanism is to be used,
it is mandatory to correctly specify
the filename of the main file as the argument of |\childdocmain|:
%
\begin{center}
\begin{tabular}{l}
|\input{childdoc.def}|\\
|\childdocmain{|\textit{main}|}|\\
\end{tabular}
\end{center}
%
If |\jobname| does not match the argument \textit{main} of |\childdocmain|,
it is assumed that |\jobname| points to the child file to be compiled.
When using |\childdocmain| with the main file specified as argument,
it suffices to start a child file
with just |\input{|\textit{main}|}|
without loading of the package and using |\childdocof|.
If instead all processing is done
with the appropriate \textsf{childdoc} directives,
the argument of \textit{main} of |\childdocmain| can be empty.

An alternative version of the command line processing described
in \secref{sec:commandline} using the detection mechanism reads:
%
\begin{center}
|... -jobname "|\textit{target}|" "|[\textit{flags}]%
[|\def\jobname{|\textit{dest}|}|]|\input{|\textit{main}|}"|
\end{center}

%%%%%%%%%%%%%%%%%%%%%%%%%%%%%%%%%%%%%%%%%%%%%%%%%%%%%%%%%%%%%%%%%%%%%%%%%%%%%%%%
\subsection{Manual Code}
\label{sec:manual}

In case one cannot be certain whether the definitions file |childdoc.def|
is installed on the target \TeX{} distribution
and one prefers not to ship it,
it is conceivable to paste a few relevant commands into the sources.

To that end, drop all statements |\input{childdoc.def}|
and perform the replacements as outlined below.
Instead of |\childdocmain{|\textit{main}|}| add the following code
to the top of the main file:
%
\begin{center}
\begin{tabular}{l}
|\||ifdefined\childdocname\endinput\||fi\newif\ifchilddoc|\\
|\edef\childdocname{\scantokens\expandafter{\jobname\noexpand}}|\\
|\def\childdocmain{|\textit{main}|}\||ifx\childdocmain\childdocname\||else|\\
|\childdoctrue\includeonly{\childdocname}\let\jobname\childdocmain\||fi|\\
\end{tabular}
\end{center}
%
Instead of |\childdocof{|\textit{main}|}| just include the main file
at the top of each child file:
%
\begin{center}
|\input{|\textit{main}|}|
\end{center}
%
A simple redirection |\childdocforward{|\textit{dest}|}| is achieved by:
%
\begin{center}
|\def\jobname{|\textit{dest}|}\input{\jobname}|
\end{center}
%
The redirection with prefix
|\childdocforwardprefix[|\textit{prefix}|]{|\textit{dest}|}|
is accomplished by:
%
\begin{center}
\begin{tabular}{l}
|{\edef\jobname{\scantokens\expandafter{\jobname\noexpand}}|\\
|\def\redirectjob |\textit{prefix}|#1~~~{\gdef\jobname{|\textit{dest}|#1}}|\\
|\expandafter\redirectjob\jobname~~~}\input{\jobname}|
\end{tabular}
\end{center}

In an alternative approach,
child documents can be compiled by a specific command line
without additional code or specific definitions:
%
\begin{center}
|... -jobname "|\textit{target}|" "|[\textit{flags}]%
|\includeonly{|\textit{dest}|}\input{|\textit{main}|}"|
\end{center}
%

%%%%%%%%%%%%%%%%%%%%%%%%%%%%%%%%%%%%%%%%%%%%%%%%%%%%%%%%%%%%%%%%%%%%%%%%%%%%%%%%
%%%%%%%%%%%%%%%%%%%%%%%%%%%%%%%%%%%%%%%%%%%%%%%%%%%%%%%%%%%%%%%%%%%%%%%%%%%%%%%%
\section{Information}

%%%%%%%%%%%%%%%%%%%%%%%%%%%%%%%%%%%%%%%%%%%%%%%%%%%%%%%%%%%%%%%%%%%%%%%%%%%%%%%%
\subsection{Copyright}

Copyright \copyright{} 2017--2018 Niklas Beisert

This work may be distributed and/or modified under the
conditions of the \LaTeX{} Project Public License, either version 1.3
of this license or (at your option) any later version.
The latest version of this license is in
  \url{http://www.latex-project.org/lppl.txt}
and version 1.3 or later is part of all distributions of \LaTeX{}
version 2005/12/01 or later.

This work has the LPPL maintenance status `maintained'.

The Current Maintainer of this work is Niklas Beisert.

This work consists of the files |README.txt|, |childdoc.ins| and |childdoc.dtx|
as well as the derived files |childdoc.def|, |cdocsamp.tex|
with |cdocsch1.tex|, |cdocsch2.tex|, |cdocspt3.tex|, |cdocspt4.tex|,
|cdocsdrf.tex|, |cdocsfn1.tex|, |cdocsfn2.tex|
as well as |childdoc.pdf|.

%%%%%%%%%%%%%%%%%%%%%%%%%%%%%%%%%%%%%%%%%%%%%%%%%%%%%%%%%%%%%%%%%%%%%%%%%%%%%%%%
\subsection{Files and Installation}

The package consists of the files:
%
\begin{center}
\begin{tabular}{ll}
    |README.txt|   & readme file \\
    |childdoc.ins| & installation file \\
    |childdoc.dtx| & source file \\
    |childdoc.def| & definition file \\
    |cdocsamp.tex| & sample main file \\
    |cdocsch1.tex| & sample include file \\
    |cdocsch2.tex| & sample include file \\
    |cdocspt3.tex| & sample part file \\
    |cdocspt4.tex| & sample part file \\
    |cdocsdrf.tex| & sample redirection file \\
    |cdocsfn1.tex| & sample redirection file \\
    |cdocsfn2.tex| & sample redirection file \\
    |childdoc.pdf| & manual
\end{tabular}
\end{center}
%
The distribution consists of the files
|README.txt|, |childdoc.ins| and |childdoc.dtx|.
%
\begin{itemize}
\item
Run (pdf)\LaTeX{} on |childdoc.dtx|
to compile the manual |childdoc.pdf| (this file).
\item
Run \LaTeX{} on |childdoc.ins| to create the definitions file |childdoc.def|
and the sample |cdocsamp.tex| with include files
|cdocsch1.tex|, |cdocsch2.tex|, |cdocspt3.tex|, |cdocspt4.tex|,
|cdocsdrf.tex|, |cdocsfn1.tex|, |cdocsfn2.tex|.
Then copy the file |childdoc.def| to an appropriate directory of your \LaTeX{}
distribution, e.g.\ \textit{texmf-root}|/tex/latex/childdoc|.
\end{itemize}

%%%%%%%%%%%%%%%%%%%%%%%%%%%%%%%%%%%%%%%%%%%%%%%%%%%%%%%%%%%%%%%%%%%%%%%%%%%%%%%%
\subsection{Related CTAN Packages}

There are several other packages which offer a similar functionality:
%
\begin{itemize}
\item
The packages
\href{http://ctan.org/pkg/docmute}{\textsf{docmute}},
\href{http://ctan.org/pkg/includex}{\textsf{includex}} and
\href{http://ctan.org/pkg/standalone}{\textsf{standalone}}
provide commands to include only the document body of
a child file thus allowing both files to be compiled individually.
\item
The packages \href{http://ctan.org/pkg/subdocs}{\textsf{subdocs}}
and \href{http://ctan.org/pkg/subfiles}{\textsf{subfiles}}
provide structures in which the main and child documents can be
encapsulated and allowing them to be compiled individually.
The inclusion mechanism is different from the conventional |\include|.
\item
The package \href{http://ctan.org/pkg/combine}{\textsf{combine}}
is an elaborate solution to combine several documents into one.
\end{itemize}
%
See also the CTAN topic \href{http://ctan.org/topic/subdocs}{\textsf{subdocs}}
for further related packages.
The present package differs from the above solutions in that
a document structure constructed with the conventional |\include| mechanism
just needs two extra commands at the top of every file
such that all constituent files can be compiled individually.

%%%%%%%%%%%%%%%%%%%%%%%%%%%%%%%%%%%%%%%%%%%%%%%%%%%%%%%%%%%%%%%%%%%%%%%%%%%%%%%%
%\subsection{Feature Suggestions}
%
%The following is a list of features which may be useful for future
%versions of this package:
%%
%\begin{itemize}
%\item
%\ldots
%\end{itemize}

%%%%%%%%%%%%%%%%%%%%%%%%%%%%%%%%%%%%%%%%%%%%%%%%%%%%%%%%%%%%%%%%%%%%%%%%%%%%%%%%
\subsection{Revision History}

%%%%%%%%%%%%%%%%%%%%%%%%%%%%%%%%%%%%%%%%
\paragraph{v2.0:} 2018/12/30

\begin{itemize}
\item
immediate forward processing
\item
added |\childdocby| mechanism
\item
manual restructured
\end{itemize}

%%%%%%%%%%%%%%%%%%%%%%%%%%%%%%%%%%%%%%%%
\paragraph{v1.6:} 2018/01/17

\begin{itemize}
\item
application for development of include files
\item
corrections to manual
\end{itemize}

%%%%%%%%%%%%%%%%%%%%%%%%%%%%%%%%%%%%%%%%
\paragraph{v1.5:} 2017/05/21

\begin{itemize}
\item
more complete structuring introduced
\item
|\childdocof| introduced
\item
|\childdoc| renamed to |\childdocmain|
\item
|\childredirect| renamed to |\childdocforward| and |\childdocforwardprefix|
and functionality expanded
\end{itemize}

%%%%%%%%%%%%%%%%%%%%%%%%%%%%%%%%%%%%%%%%
\paragraph{v1.0:} 2017/04/27

\begin{itemize}
\item
manual and install package
\item
first version published on CTAN
\end{itemize}

%%%%%%%%%%%%%%%%%%%%%%%%%%%%%%%%%%%%%%%%
\paragraph{v0.6:} 2017/04/26

\begin{itemize}
\item
redirection mechanism added
\end{itemize}

%%%%%%%%%%%%%%%%%%%%%%%%%%%%%%%%%%%%%%%%
\paragraph{v0.5:} 2017/04/26

\begin{itemize}
\item
functionality in definition file
\end{itemize}


%%%%%%%%%%%%%%%%%%%%%%%%%%%%%%%%%%%%%%%%%%%%%%%%%%%%%%%%%%%%%%%%%%%%%%%%%%%%%%%%
%%%%%%%%%%%%%%%%%%%%%%%%%%%%%%%%%%%%%%%%%%%%%%%%%%%%%%%%%%%%%%%%%%%%%%%%%%%%%%%%
%%%%%%%%%%%%%%%%%%%%%%%%%%%%%%%%%%%%%%%%%%%%%%%%%%%%%%%%%%%%%%%%%%%%%%%%%%%%%%%%
\appendix

\settowidth\MacroIndent{\rmfamily\scriptsize 000\ }

 \DocInput{childdoc.dtx}

\end{document}
%</driver>
% \fi
%
% %%%%%%%%%%%%%%%%%%%%%%%%%%%%%%%%%%%%%%%%%%%%%%%%%%%%%%%%%%%%%%%%%%%%%%%%%%%%%%
% %%%%%%%%%%%%%%%%%%%%%%%%%%%%%%%%%%%%%%%%%%%%%%%%%%%%%%%%%%%%%%%%%%%%%%%%%%%%%%
% \section{Sample}
%\iffalse
%<*samplemain>
%\fi
%
% The following presents a sample document
% with two chapters, two parts, a title page,
% a compile flag as well as three forwarding files to set the flag.
% It consists of eight |.tex| files:
% \begin{center}
% \begin{tabular}{ll}
% |cdocsamp.tex|&main file\\
% |cdocsch1.tex|&include file for chapter 1\\
% |cdocsch2.tex|&include file for chapter 2\\
% |cdocspt3.tex|&include file for part 3\\
% |cdocspt4.tex|&include file for part 4\\
% |cdocsdrf.tex|&forwarding file for main file in draft mode\\
% |cdocsfi1.tex|&forwarding file for final version of chapter 1\\
% |cdocsfi2.tex|&forwarding file for final version of chapter 2\\
% \end{tabular}
% \end{center}
% Each of the eight files can be compiled directly by the \LaTeX{} compiler.
%
% %%%%%%%%%%%%%%%%%%%%%%%%%%%%%%%%%%%%%%
% \paragraph{Main File.}
%
% The main file is called |cdocsamp.tex|.
%
% Load the \textsf{childdoc} definitions and
% declare the filename for the main document:
%    \begin{macrocode}
\input{childdoc.def}
\childdocmain{}
%    \end{macrocode}

% Optional override for |\version| flag:
%    \begin{macrocode}
%%\ifchilddoc\else\providecommand{\version}{draft}\fi
%    \end{macrocode}

% Define the default values for the |\version| flag
% (|final| for the main file and |draft| for childs):
%    \begin{macrocode}
\ifchilddoc
\providecommand{\version}{draft}
\else
\providecommand{\version}{final}
\fi
%    \end{macrocode}

% Load the standard document class:
%    \begin{macrocode}
\documentclass[12pt]{article}
%    \end{macrocode}

% Start the document body:
%    \begin{macrocode}
\begin{document}
%    \end{macrocode}

% Declare a title page.
% Print title, part of document being processed and version flag:
%    \begin{macrocode}
\addtocounter{page}{-1}
\begin{center}
{\LARGE\bfseries{}childdoc example\par}
\vspace{1cm}
\ifchilddoc
\ifchilddocmanual part\else chapter\fi:
`\childdocname' of `\childdocjob'\par
\else
main document: `\childdocjob'\par
\fi
version: \version\par
\end{center}
\newpage
%    \end{macrocode}

% Manually include selected file,
% otherwise process as usual:
%    \begin{macrocode}
\ifchilddocmanual
\section*{part `\childdocname'}
\input{\childdocname}
\else
%    \end{macrocode}

% Include the two chapters:
%    \begin{macrocode}
\include{cdocsch1}
\include{cdocsch2}
%    \end{macrocode}

% Include the two parts unless only chapters should be displayed:
%    \begin{macrocode}
\ifchilddoc\else
\section{part three}
\input{cdocspt3}
\section{part four}
\input{cdocspt4}
\fi
%    \end{macrocode}

% Process as usual until here:
%    \begin{macrocode}
\fi
%    \end{macrocode}

% End of document body:
%    \begin{macrocode}
\end{document}
%    \end{macrocode}
%\iffalse
%</samplemain>
%\fi
%
% %%%%%%%%%%%%%%%%%%%%%%%%%%%%%%%%%%%%%%
% \paragraph{Chapter Include Files.}
%
% The include files are called |cdocsch1.tex| and |cdocsch2.tex|.
%
%\iffalse
%<*samplechap1|samplechap2>
%\fi

% Optional override for |\version| flag:
%    \begin{macrocode}
%%\providecommand{\version}{final}
%    \end{macrocode}

% Include the main document:
%    \begin{macrocode}
\input{childdoc.def}
\childdocof{cdocsamp}
%    \end{macrocode}

%\iffalse
%</samplechap1|samplechap2>
%\fi
%
%\iffalse
%<*samplechap1>
%\fi
% Some text for chapter 1:
%    \begin{macrocode}
\section{one}
some text in chapter one
%    \end{macrocode}

%\iffalse
%</samplechap1>
%\fi
% Some text for chapter 2:
%\iffalse
%<*samplechap2>
%\fi
%    \begin{macrocode}
\section{two}
more text in chapter two
%    \end{macrocode}

%\iffalse
%</samplechap2>
%\fi
%
% %%%%%%%%%%%%%%%%%%%%%%%%%%%%%%%%%%%%%%
% \paragraph{Part Include Files.}
%
% The include files are called |cdocspt3.tex| and |cdocspt4.tex|.
%
%\iffalse
%<*samplepart3|samplepart4>
%\fi

% Optional override for |\version| flag:
%    \begin{macrocode}
%%\providecommand{\version}{final}
%    \end{macrocode}

% Include the main document:
%    \begin{macrocode}
\input{childdoc.def}
\childdocby{cdocsamp}
%    \end{macrocode}

%\iffalse
%</samplepart3|samplepart4>
%\fi
%
%\iffalse
%<*samplepart3>
%\fi
% Some text for part 3:
%    \begin{macrocode}
some text in part three
%    \end{macrocode}

%\iffalse
%</samplepart3>
%\fi
% Some text for part 4:
%\iffalse
%<*samplepart4>
%\fi
%    \begin{macrocode}
more text in part four
%    \end{macrocode}

%\iffalse
%</samplepart4>
%\fi
%
% %%%%%%%%%%%%%%%%%%%%%%%%%%%%%%%%%%%%%%
% \paragraph{Forwarding for a Complete Draft.}
%
% The following forwarding file |cdocsdrf.tex|
% compiles the main document in draft mode:
%\iffalse
%<*sampledraft>
%\fi
%    \begin{macrocode}
\def\version{draft}
\input{childdoc.def}
\childdocforward{cdocsamp}
%    \end{macrocode}

%\iffalse
%</sampledraft>
%\fi
%
% %%%%%%%%%%%%%%%%%%%%%%%%%%%%%%%%%%%%%%
% \paragraph{Forwarding for Final Version of the Chapters.}
%
% The following forwarding files |cdocsfn1.tex| and |cdocsfn2.tex|
% (with identical content)
% compile the final versions of the child documents
% |cdocsch1.tex| and |cdocsch2.tex|, respectively:
%\iffalse
%<*samplefinal>
%\fi
%    \begin{macrocode}
\def\version{final}
\input{childdoc.def}
\childdocforwardprefix[cdocsamp]{cdocsfn}{cdocsch}
%    \end{macrocode}

%\iffalse
%</samplefinal>
%\fi
%
% %%%%%%%%%%%%%%%%%%%%%%%%%%%%%%%%%%%%%%
% \paragraph{Command Line Processing.}
%
% The following three command lines generate the output files
% |cdocscld|, |cdocscl1| and |cdocscl2|
% which should be identical to
% |cdocsdrf|, |cdocsch1| and |cdocsfn2|, respectively:
% \begin{center}
% \begin{tabular}{l}
% |latex -jobname cdocscld \|\\
% |  "\def\version{draft}\input{childdoc.def}\childdocforward{cdocsamp}"|\\
% |latex -jobname cdocscl1 \|\\
% |  "\input{childdoc.def}\childdocforward[cdocsamp]{cdocsch1}"|\\
% |latex -jobname cdocscl2 \|\\
% |  "\def\version{final}\input{childdoc.def}\childdocforward{cdocsch2}"|
% \end{tabular}
% \end{center}
% Note that the trailing backslash on each first line
% merely continues the input to the second line
% (for convenient cut ant paste).
% Furthermore, the command |latex| can be replaced by any
% of its alternative versions such as |pdflatex|.
%
% %%%%%%%%%%%%%%%%%%%%%%%%%%%%%%%%%%%%%%%%%%%%%%%%%%%%%%%%%%%%%%%%%%%%%%%%%%%%%%
% %%%%%%%%%%%%%%%%%%%%%%%%%%%%%%%%%%%%%%%%%%%%%%%%%%%%%%%%%%%%%%%%%%%%%%%%%%%%%%
% \section{Implementation}
%\iffalse
%<*package>
%\fi
%
% This section describes the definitions file |childdoc.def|.

% The definitions cannot be loaded using |\usepackage| or |\RequirePackage|
% which has a mechanism to prevent loading a style file more than once.
% When loading the definitions by means of |\input|
% multiple instances have to be prevented manually:
%\iffalse
%This code needs to be before the `\ProvidesFile' directive
%which is defined at the beginning of this file.
%Therefore it is also placed there and commented out here.
%</package>
%<*discard>
%\fi
%    \begin{macrocode}
\ifdefined\childdocmain\endinput\fi
%    \end{macrocode}
%\iffalse
%</discard>
%<*package>
%\fi
%
% \macro{\ifchilddoc}
% \macro{\ifchilddocmanual}
% The conditional |\ifchilddoc| tells whether a
% child (true) or main (false) document is being compiled.
% The conditional |\ifchilddocmanual| tells whether
% the |\includeonly| mechanism is used (false) or
% the selection of child files must be performed manually (true).
% The definitions initialise to false:
%    \begin{macrocode}
\newif\ifchilddoc
\newif\ifchilddocmanual
%    \end{macrocode}

% \macro{\childdocname}
% \macro{\childdocjob}
% The macro |\childdocname| stores the name of the main document
% to be compiled. The macro |\childdocjob| stores the name of
% the document on which the \LaTeX{} compiler was originally invoked.
% The content of |\jobname| cannot be compared
% to filenames specified in the source due to different catcodes.
% The following code rescans |\jobname|, stores the result
% in |\childdocname| and saves a copy in |\childdocjob|:
%    \begin{macrocode}
\edef\childdocname{\scantokens\expandafter{\jobname\noexpand}}
\let\childdocjob\childdocname
%    \end{macrocode}

% \macro{\childdocdisable}
% The macro |\childdocdisable| prevents the main file
% from being processed more than once.
% At this stage, the main document command |\childdocmain|
% is assumed to be called once again where it should do nothing.
% Any subsequent call to it should prevent
% a secondary processing of the main document
% It overwrites the forwarding commands
% |\childdocof| and |\childdocforward|
% with empty macros to prevent further inclusions of the main document:
%    \begin{macrocode}
\newcommand{\childdocdisable}
{
  \renewcommand{\childdocmain}[1]{\renewcommand{\childdocmain}[1]{\endinput}}
  \renewcommand{\childdocof}[1]{}
  \renewcommand{\childdocby}[2][]{}
  \renewcommand{\childdocforward}[2][]{}
  \renewcommand{\childdocdisable}{}
}
%    \end{macrocode}

% \macro{\childdocmain}
% The macro |\childdocmain| is to be called at the top of the main file
% with nothing or the main filename (without extension) as argument.
% First, it breaks loops.
% If the argument is not empty and does not match |\childdocname|
% (which is set by the first inclusion of |childdoc.def|),
% |\ifchilddoc| is set to true, |\includeonly| is applied to the child file
% and |\jobname| is set to the main file
% (for proper handling of |.aux| files):
%    \begin{macrocode}
\newcommand{\childdocmain}[1]
{
  \childdocdisable\childdocmain{}
  \if?#1?\else
    \begingroup
      \def\childdoctmp{#1}
      \ifx\childdoctmp\childdocname
        \def\childdoctmp{}
      \else
        \def\childdoctmp
        {
          \childdoctrue
          \includeonly{\childdocname}
          \def\childdocjob{#1}
          \def\jobname{#1}
        }
      \fi
      \expandafter
    \endgroup
    \childdoctmp
  \fi
}
%    \end{macrocode}

% \macro{\childdocof}
% The command |\childdocof| redirects
% compilation to the main file |#1|.
%    \begin{macrocode}
\newcommand{\childdocof}[1]
{
  \childdocdisable
  \childdoctrue
  \includeonly{\childdocname}
  \def\jobname{#1}
  \def\childdocjob{#1}
  \input{#1}
}
%    \end{macrocode}

% \macro{\childdocby}
% The command |\childdocby| ....
%    \begin{macrocode}
\newcommand{\childdocby}[2][]
{
  \childdocdisable
  \childdoctrue
  \childdocmanualtrue
  \if?#1?\else
    \def\jobname{#2}
  \fi
  \def\childdocjob{#2}
  \input{#2}
  \endinput
}
%    \end{macrocode}

% \macro{\childdocforward}
% The command |\childdocforward| redirects
% compilation to the main file or
% (if the optional argument is given) a child file.
% Parameters are set as if the main file
% or a child file starting with |\childdocof| was compiled.
% Then compilation is handed over to the main file:
%    \begin{macrocode}
\newcommand{\childdocforward}[2][]
{
  \begingroup
    \if?#1?
      \def\childdoctmp
      {
        \def\childdocname{#2}
        \def\childdocjob{#2}
        \def\jobname{#2}
        \input{#2}
        \endinput
      }
    \else
      \def\childdoctmp
      {
        \childdocdisable
        \def\childdocname{#2}
        \childdoctrue
        \includeonly{#2}
        \def\childdocjob{#1}
        \def\jobname{#1}
        \input{#1}
        \endinput
      }
    \fi
    \expandafter
  \endgroup
  \childdoctmp
}
%    \end{macrocode}

% \macro{\childdocforwardprefix}
% The command |\childdocforwardprefix| redirects
% compilation to the main or a child file by means of a pattern.
% The prefix |#1| in the current filename is replaced by |#2|
% and the suffix of the current filename is kept
% (it is assumed that the filename does not contain the substring `|~~~|'
% which is used as a delimiter).
% Compilation is handed over to the new file by |\childdocforward|:
%    \begin{macrocode}
\newcommand{\childdocforwardprefix}[3][]
{
  \begingroup
    \def\childdocextract #2##1~~~{\def\childdoctmp{\childdocforward[#1]{#3##1}}}
    \expandafter\childdocextract\childdocname~~~
    \expandafter
  \endgroup
  \childdoctmp
}
%    \end{macrocode}

% \macro{\childdoc}
% The deprecated macro |\childdoc| is a legacy version of |\childdocmain|:
%    \begin{macrocode}
\newcommand{\childdoc}{\childdocmain}
%    \end{macrocode}

% \macro{\childdocredirect}
% The deprecated macro |\childdocredirect| is a legacy version
% of |\childdocforward| and |\childdocforwardprefix|:
%    \begin{macrocode}
\newcommand{\childdocredirect}[2][]
{
  \begingroup
    \if?#1?
      \def\childdoctmp{\childdocforward{#2}}
    \else
      \def\childdoctmp{\childdocforwardprefix{#1}{#2}}
    \fi
    \expandafter
  \endgroup
  \childdoctmp
}
%    \end{macrocode}

%\iffalse
%</package>
%\fi
%
\endinput

\childdocmain{}
%    \end{macrocode}

% Optional override for |\version| flag:
%    \begin{macrocode}
%%\ifchilddoc\else\providecommand{\version}{draft}\fi
%    \end{macrocode}

% Define the default values for the |\version| flag
% (|final| for the main file and |draft| for childs):
%    \begin{macrocode}
\ifchilddoc
\providecommand{\version}{draft}
\else
\providecommand{\version}{final}
\fi
%    \end{macrocode}

% Load the standard document class:
%    \begin{macrocode}
\documentclass[12pt]{article}
%    \end{macrocode}

% Start the document body:
%    \begin{macrocode}
\begin{document}
%    \end{macrocode}

% Declare a title page.
% Print title, part of document being processed and version flag:
%    \begin{macrocode}
\addtocounter{page}{-1}
\begin{center}
{\LARGE\bfseries{}childdoc example\par}
\vspace{1cm}
\ifchilddoc
\ifchilddocmanual part\else chapter\fi:
`\childdocname' of `\childdocjob'\par
\else
main document: `\childdocjob'\par
\fi
version: \version\par
\end{center}
\newpage
%    \end{macrocode}

% Manually include selected file,
% otherwise process as usual:
%    \begin{macrocode}
\ifchilddocmanual
\section*{part `\childdocname'}
\input{\childdocname}
\else
%    \end{macrocode}

% Include the two chapters:
%    \begin{macrocode}
\include{cdocsch1}
\include{cdocsch2}
%    \end{macrocode}

% Include the two parts unless only chapters should be displayed:
%    \begin{macrocode}
\ifchilddoc\else
\section{part three}
\input{cdocspt3}
\section{part four}
\input{cdocspt4}
\fi
%    \end{macrocode}

% Process as usual until here:
%    \begin{macrocode}
\fi
%    \end{macrocode}

% End of document body:
%    \begin{macrocode}
\end{document}
%    \end{macrocode}
%\iffalse
%</samplemain>
%\fi
%
% %%%%%%%%%%%%%%%%%%%%%%%%%%%%%%%%%%%%%%
% \paragraph{Chapter Include Files.}
%
% The include files are called |cdocsch1.tex| and |cdocsch2.tex|.
%
%\iffalse
%<*samplechap1|samplechap2>
%\fi

% Optional override for |\version| flag:
%    \begin{macrocode}
%%\providecommand{\version}{final}
%    \end{macrocode}

% Include the main document:
%    \begin{macrocode}
% \iffalse
%
% childdoc.dtx Copyright (C) 2017-2018 Niklas Beisert
%
% This work may be distributed and/or modified under the
% conditions of the LaTeX Project Public License, either version 1.3
% of this license or (at your option) any later version.
% The latest version of this license is in
%   http://www.latex-project.org/lppl.txt
% and version 1.3 or later is part of all distributions of LaTeX
% version 2005/12/01 or later.
%
% This work has the LPPL maintenance status `maintained'.
%
% The Current Maintainer of this work is Niklas Beisert.
%
% This work consists of the files childdoc.dtx and childdoc.ins
% and the derived files childdoc.def and cdocsamp.tex with
% cdocsch1.tex, cdocsch2.tex, cdocsdrf.tex, cdocsfn1.tex, cdocsfn2.tex.
%
%<package>\ifdefined\childdocmain\endinput\fi
%<package>\ProvidesFile{childdoc.def}[2018/12/30 v2.0 child document driver]
%<samplemain>\ProvidesFile{cdocsamp.tex}[2018/12/30 v2.0 sample for childdoc]
%<*driver>
%\ProvidesFile{childdoc.drv}[2018/12/30 v2.0 childdoc reference manual file]
\PassOptionsToClass{10pt,a4paper}{article}
\documentclass{ltxdoc}

\usepackage[margin=35mm]{geometry}
\usepackage{hyperref}
\usepackage{hyperxmp}
\usepackage[usenames]{color}

\hypersetup{colorlinks=true}
\hypersetup{pdfstartview=FitH}
\hypersetup{pdfpagemode=UseNone}
\hypersetup{pdfsource={}}
\hypersetup{pdflang={en-UK}}
\hypersetup{pdfcopyright={Copyright 2017-2018 Niklas Beisert.
  This work may be distributed and/or modified under the
  conditions of the LaTeX Project Public License, either version 1.3
  of this license or (at your option) any later version.}}
\hypersetup{pdflicenseurl={http://www.latex-project.org/lppl.txt}}
\hypersetup{pdfcontactaddress={ETH Zurich, ITP, HIT K,
  Wolfgang-Pauli-Strasse 27}}
\hypersetup{pdfcontactpostcode={8093}}
\hypersetup{pdfcontactcity={Zurich}}
\hypersetup{pdfcontactcountry={Switzerland}}
\hypersetup{pdfcontactemail={nbeisert@itp.phys.ethz.ch}}
\hypersetup{pdfcontacturl={http://people.phys.ethz.ch/\xmptilde nbeisert/}}

\newcommand{\secref}[1]{\hyperref[#1]{section \ref*{#1}}}

\parskip1ex
\parindent0pt
\let\olditemize\itemize
\def\itemize{\olditemize\parskip0pt}

\begin{document}

\title{The \textsf{childdoc} Package}
\hypersetup{pdftitle={The childdoc Package}}
\author{Niklas Beisert\\[2ex]
  Institut f\"ur Theoretische Physik\\
  Eidgen\"ossische Technische Hochschule Z\"urich\\
  Wolfgang-Pauli-Strasse 27, 8093 Z\"urich, Switzerland\\[1ex]
  \href{mailto:nbeisert@itp.phys.ethz.ch}
  {\texttt{nbeisert@itp.phys.ethz.ch}}}
\hypersetup{pdfauthor={Niklas Beisert}}
\hypersetup{pdfsubject={Manual for the LaTeX2e Package childdoc}}
\date{30 December 2018, \textsf{v2.0}}
\maketitle

\begin{abstract}\noindent
\textsf{childdoc} is a \LaTeXe{} package
that enables the direct compilation
of document sections included by |\include|
to individual files.
\end{abstract}

\begingroup
\parskip0ex
\tableofcontents
\endgroup

%%%%%%%%%%%%%%%%%%%%%%%%%%%%%%%%%%%%%%%%%%%%%%%%%%%%%%%%%%%%%%%%%%%%%%%%%%%%%%%%
%%%%%%%%%%%%%%%%%%%%%%%%%%%%%%%%%%%%%%%%%%%%%%%%%%%%%%%%%%%%%%%%%%%%%%%%%%%%%%%%
\section{Introduction}

\LaTeX{} provides a mechanism to structure a large document (such as a book)
into a main file and several child files (containing the chapters)
using the |\include| command.
This mechanism is beneficial for documents
which span hundreds of pages in order to
make the source file(s) more manageable.
Moreover, compilation can be restricted to
selected child files by means of the |\includeonly| command.
The latter feature can be used to reduce the compilation time while editing
(this was significantly more useful in the earlier days of \LaTeX{})
or to generate a smaller document which is easier to navigate.
Another application of |\includeonly| is to generate
documents consisting of selected parts of the complete document.

However, there are a few drawbacks of the plain |\include| mechanism:
\begin{itemize}
\item
The child files cannot be compiled on their own,
they can only be compiled via the main file.
A naive editing environment
(such as a text editor with an option
to have the current file processed by \LaTeX)
may require one to switch to the main file before compiling;
attempting to compile the child file produces errors.
\item
The main file must be modified (each time)
to adjust the |\includeonly| command
to the present needs. This easily leaves the main file in a messy state.
\item
The generated document will always carry the filename
of the main document. This is inconvenient if
several child files are to be compiled and
to be kept for distribution.
\end{itemize}

The present package provides a simple interface
to make child files individually compilable by \LaTeX{}.
Compiling a child file then has the same effect as compiling
the main file with an |\includeonly| command
to select the appropriate child.
Moreover the generated document will carry the name of the child
rather than the main file.
This resolves all three above issues.

This feature is meant to make the editing of books,
thesis documents and lecture notes somewhat more convenient.
However, the package can also be used efficiently for
composing a series of documents (such as exercise sheets)
which are typically distributed individually.
It then assists the author in generating the individual documents
(potentially in different versions)
as well as a document containing the collected series.
Another application is in developing style files
or other kinds of included material
where compilation of the style file could redirect
to a sample or test file.

%%%%%%%%%%%%%%%%%%%%%%%%%%%%%%%%%%%%%%%%%%%%%%%%%%%%%%%%%%%%%%%%%%%%%%%%%%%%%%%%
%%%%%%%%%%%%%%%%%%%%%%%%%%%%%%%%%%%%%%%%%%%%%%%%%%%%%%%%%%%%%%%%%%%%%%%%%%%%%%%%
\section{Usage}

First of all, the package \textsf{childdoc} is \emph{not} a standard
\LaTeXe{} |.sty| style file! Therefore it needs to be invoked in
a non-standard way.

%%%%%%%%%%%%%%%%%%%%%%%%%%%%%%%%%%%%%%%%%%%%%%%%%%%%%%%%%%%%%%%%%%%%%%%%%%%%%%%%
\subsection{Included Files}
\label{sec:include}

%%%%%%%%%%%%%%%%%%%%%%%%%%%%%%%%%%%%%%%%
\DescribeMacro{\childdocmain}
To use the package, add the commands
\begin{center}
\begin{tabular}{l}
|\input{childdoc.def}|\\
|\childdocmain{}|\\
\end{tabular}
\end{center}
at the very top of the main \LaTeX{} file,
in particular \emph{before} the |\documentclass| statement!
The argument of |\childdocmain| should be left empty
(but it must be present).

%%%%%%%%%%%%%%%%%%%%%%%%%%%%%%%%%%%%%%%%
\DescribeMacro{\childdocof}
Furthermore, add the commands
\begin{center}
\begin{tabular}{l}
|\input{childdoc.def}|\\
|\childdocof{|\textit{main}|}|\\
\end{tabular}
\end{center}
at the top of every child file \textit{child}
which is included by |\include{|\textit{child}|}|
from within the main file
(or at least for those files to be compiled individually).
The argument \textit{main} must be the filename of the main file.

There are a couple of
considerations in setting up the main and child documents:

%%%%%%%%%%%%%%%%%%%%%%%%%%%%%%%%%%%%%%%%
\paragraph{Restrictions.}

Please note the following restrictions:
\begin{itemize}
\item
|\childdocmain| must be called with one argument \textit{main}
to ensure compatibility with earlier version of the package.
It must either be empty (|\childdocmain{}|)
or precisely match the filename of the main file in which it is specified.
See \secref{sec:detection} for further information.
\item
The filename \textit{main} must be specified without the |.tex| extension.
\item
The filename \textit{main} is case sensitive
(even in case-insensitive file systems)
due to internal string comparison.
\item
The argument \textit{main} should be fully expanded, it cannot be a macro.
\item
Subdirectories and special characters should be avoided in filenames.
\item
The command |\childdocmain{|\textit{main}|}| must be followed by a whitespace.
It should not be followed immediately by another command
or by a comment mark `|%|'.
This is because the \TeX{} parser reads the token immediately following
the argument of |\childdocmain| and puts it
at the beginning of every child section;
however, a white\-space is ignored.
\end{itemize}

%%%%%%%%%%%%%%%%%%%%%%%%%%%%%%%%%%%%%%%%
\paragraph{Content of Main File.}

It is advisable to place all content in the child files included by |\include|.
Any output contained in the main file will appear in all child documents
unless suppressed manually;
it cannot be suppressed automatically by the |\includeonly| directive
and thus should normally be avoided.
A method to include some content in the main file
by means of conditional processing is described in \secref{sec:conditional}.

%%%%%%%%%%%%%%%%%%%%%%%%%%%%%%%%%%%%%%%%
\paragraph{Page Numbering.}

When only a part of the document is compiled,
the appropriate numbering of pages
(as well as other status parameters)
is determined from the |.aux| files.
The latter contain information from previous passes.
However this information needs to propagate through
all intermediate child documents.
Therefore the page numbering in child documents may well
be inconsistent until the complete document is compiled at least once.

A useful (if unconventional) way to always ensure a consistent
page numbering is to restart the numbering in each child document
and denote the pages by `\textit{child}|.|\textit{page}'
where \textit{child} represents the chapter/section number of the child file.
This can be achieved by the command
|\numberwithin{page}{|\textit{child}|}|
of the \textsf{amsmath} package
where \textit{child} can be |chapter| or |section|
depending on the chosen structuring.
Alternatively, one can modify the macro |\thepage| appropriately
and reset the counter |page| at the start of each child file.

%%%%%%%%%%%%%%%%%%%%%%%%%%%%%%%%%%%%%%%%%%%%%%%%%%%%%%%%%%%%%%%%%%%%%%%%%%%%%%%%
\subsection{Conditional Processing}
\label{sec:conditional}

The package provides a mechanism to compile different versions
of a document. To customise the versions further some conditional processing
can come in handy to distinguish which version is being compiled.
The package provides two macros to describe the compilation context:

%%%%%%%%%%%%%%%%%%%%%%%%%%%%%%%%%%%%%%%%
\DescribeMacro{\ifchilddoc}
The conditional |\ifchilddoc| distinguishes between the compilation of
child documents and the main document:
%
\begin{center}
|\ifchilddoc |\textit{child-code}| |[|\||else |\textit{main-code}]| \||fi|
\end{center}

%%%%%%%%%%%%%%%%%%%%%%%%%%%%%%%%%%%%%%%%
\DescribeMacro{\childdocname}
\DescribeMacro{\childdocjob}
The macro |\childdocname| contains the filename (without extension)
of the main or child file being processed.
Note that |\childdocjob| will always contain the name of the main file.

%%%%%%%%%%%%%%%%%%%%%%%%%%%%%%%%%%%%%%%%
\paragraph{Title Page.}

Conditional processing can be used to include a title or banner page
in the main document when proper precautions are taken.
Importantly, the code in the main file should ensure that the page counter
(as well as other status parameters which are stored in the |.aux| files)
takes the same value after the conditional processing.
Otherwise the page numbers may take divergent values
depending on which part is compiled.

For example, a title page could be declared by:
%
\begin{center}
\begin{tabular}{l}
|\ifchilddoc\||else|\\
|\addtocounter{page}{-1}|\\
\textit{code for title page}\\
|\newpage|\\
|\||fi|
\end{tabular}
\end{center}
%
A banner page for the child documents can be generated by:
%
\begin{center}
\begin{tabular}{l}
|\ifchilddoc|\\
|\addtocounter{page}{-1}|\\
\textit{code for banner page}\\
|\newpage|\\
|\||fi|
\end{tabular}
\end{center}
%
Here one could write a message such as:
\begin{center}
|This is the part \childdocname{} of \childdocjob{}.|
\end{center}

%%%%%%%%%%%%%%%%%%%%%%%%%%%%%%%%%%%%%%%%%%%%%%%%%%%%%%%%%%%%%%%%%%%%%%%%%%%%%%%%
\subsection{Flags}
\label{sec:flags}

The package makes it easy to generate different versions
of the main or child documents.
To this end compilation flags can be defined
and assigned different default values.
They will be particularly useful in conjunction
with the forwarding mechanism described in \secref{sec:forward}.

For example, it may be useful to have a flag |\version|
which can be set to |draft| or |final|.
The document source will contain some conditional code
depending on the value of |\version|.
Suppose further, the flag should default to |final| for the main file
and to |draft| for child files
which is a natural assignment for editing the document.
This is achieved by placing the following code
in the preamble of the main document
(below the |\childdocmain| directive):
%
\begin{center}
\begin{tabular}{l}
|\ifchilddoc|\\
|\providecommand{\version}{draft}|\\
|\||else|\\
|\providecommand{\version}{final}|\\
|\||fi|
\end{tabular}
\end{center}
%
The definition by |\providecommand| makes sure
that previous definitions are not overwritten.
Further statements |\providecommand{\version}{...}|
can thus be added before the above code to override it.

For the main file, one might add a line
(between |\childdocmain| and the above block)
%
\begin{center}
|%\ifchilddoc\||else\providecommand{\version}{draft}\||fi|
\end{center}
%
which can be uncommented to produce a draft version.
Likewise one can add a line to the very top of a child file
(above the |\childdocof{|\textit{main}|}| directive)
%
\begin{center}
|%\providecommand{\version}{final}|
\end{center}
%
which can be uncommented to produce the final version of this child document.

%%%%%%%%%%%%%%%%%%%%%%%%%%%%%%%%%%%%%%%%%%%%%%%%%%%%%%%%%%%%%%%%%%%%%%%%%%%%%%%%
\subsection{Forwarding}
\label{sec:forward}

Different versions of the main or child documents
using compilation flags as described in \secref{sec:flags}
can be (permanently) stored in different files
for convenient compilation, viewing and distribution.
To this end, the package defines a command
to pass on compilation to a different file:

%%%%%%%%%%%%%%%%%%%%%%%%%%%%%%%%%%%%%%%%
\DescribeMacro{\childdocforward}
The command |\childdocforward| redirects processing to
another source file:
%
\begin{center}
\begin{tabular}{l}
|\input{childdoc.def}|\\
|\childdocforward[|\textit{main}|]{|\textit{dest}|}|\\
\end{tabular}
\end{center}
%
The argument \textit{dest} is the destination file
(without extension).
It should be the main file or one of the child files.
Note that further \textsf{childdoc} directives
such as |\childdocof| and |\childdocforward|
in the indicated file will be processed in this form.
The optional argument \textit{main}
passes on directly to the main file \textit{main}
while pretending to compile the child \textit{dest}.
This form behaves as if \textit{dest}
issues |\childdocof{|\textit{main}|}| right away,
and no further \textsf{childdoc} directives will be processed.

%%%%%%%%%%%%%%%%%%%%%%%%%%%%%%%%%%%%%%%%
\DescribeMacro{\...prefix}
In the alternative form |\childdocforwardprefix|,
%
\begin{center}
\begin{tabular}{l}
|\input{childdoc.def}|\\
|\childdocforwardprefix[|\textit{main}|]{|\textit{prefix}|}{|\textit{dest}|}|
\end{tabular}
\end{center}
%
the destination file is determined by a pattern
depending on the current file:
To make this work, the current file must be called
`{\textit{prefix}\hspace{0.2em}\textit{suffix}}'
with \textit{prefix} matching precisely the argument.
Processing is then passed on to the file
`{\textit{dest}\hspace{0.2em}\textit{suffix}}'.
Surely, the same effect is achieved by
directly specifying the
argument `{\textit{dest}\hspace{0.2em}\textit{suffix}}'
in the first form.
However, that requires to set up a different file
for each child. With the alternative form of the command
all these files can have exactly the same content
which simplifies setting them up and maintaining them.

For example, the following file |draft.tex|
with a compilation flag |\version| as described in \secref{sec:flags}
compiles the main document as a draft:
%
\begin{center}
\begin{tabular}{l}
|\def\version{draft}|\\
|\input{childdoc.def}|\\
|\childdocforward{|\textit{main}|}|
\end{tabular}
\end{center}
%
Likewise, the following files |final|\textit{nn}|.tex|
compile the final version of the child document
|child|\textit{nn}|.tex|:
%
\begin{center}
\begin{tabular}{l}
|\def\version{final}|\\
|\input{childdoc.def}|\\
|\childdocforwardprefix{final}{child}|
\end{tabular}
\end{center}
%

Note that when several versions of a main file and/or of each child file
are to be generated, it may be convenient to set up a |Makefile| or
shell script to automatise the process.

%%%%%%%%%%%%%%%%%%%%%%%%%%%%%%%%%%%%%%%%%%%%%%%%%%%%%%%%%%%%%%%%%%%%%%%%%%%%%%%%
\subsection{Command Line Processing}
\label{sec:commandline}

The effect of redirection files can also be achieved by invoking
the \LaTeX{} compiler with a more elaborate command line.
Most conveniently this should be done as part
of a shell script or a |Makefile|.

When using \textsf{childdoc} in the main file, the following
command lines effectively perform a redirection
(note that depending on the shell being used,
backslashes may have to be doubled: `|\|' $\to$ `|\\|'):
%
\begin{center}
|... -jobname "|\textit{target}|" |\\|"|[\textit{flags}]%
|\input{childdoc.def}\childdocforward[|\textit{main}|]{|\textit{dest}|}"|
\end{center}
%
Here \textit{target} is the name of the output file,
\textit{main} is the name of the main file
and \textit{dest} is the name of the main or child file to be processed
(all filenames without extensions).
The optional argument \textit{main} can be omitted
if \textit{main} matches \textit{dest}.
Optionally, compilation \textit{flags} can be defined via |\def| commands.
This command line makes the \TeX{} engine believe
it is compiling the file \textit{target}
whose content is specified as the latter parameter.
The provided code then forwards the processing to
\textit{main} or \textit{dest} as described in \secref{sec:forward}.

%%%%%%%%%%%%%%%%%%%%%%%%%%%%%%%%%%%%%%%%%%%%%%%%%%%%%%%%%%%%%%%%%%%%%%%%%%%%%%%%
\subsection{Include by Input}
\label{sec:input}

Including child documents by |\include| has some restrictions by design.
Most notably, the content of a child document always occupies
its own set of pages; pages cannot be shared between child documents.
Usually, this behaviour makes perfect sense
because each child document contain an essential part of the document.
However, in some situations it may be desirable to compose
a document from a collection of parts
without having mandatory page breaks between then.
For this case, the package
provides a mechanism to include parts
by |\input| which can also be processed individually.
However, by construction this mechanism
requires manual handling of the content to be output.

%%%%%%%%%%%%%%%%%%%%%%%%%%%%%%%%%%%%%%%%
\DescribeMacro{\ifchilddocmanual}
The main file should be prepared as usual, see \secref{sec:include}.
However, the document body must make a distinction
between processing of an individual part and of the main document, e.g.:
%
\begin{center}
\begin{tabular}{l}
|\ifchilddocmanual|\\
|\input{\childdocname}|\\
|\||else|\\
\textit{document body with }|\input{|\textit{part}|}|\\
|\||fi|
\end{tabular}
\end{center}
%
The conditional |\ifchilddocmanual| is true whenever
a part to be included by |\input| is being compiled,
and the name of the part is stored in |\childdocname|.

%%%%%%%%%%%%%%%%%%%%%%%%%%%%%%%%%%%%%%%%
\DescribeMacro{\childdocby}
Each part to be included by |\input| should start with:
%
\begin{center}
\begin{tabular}{l}
|\input{childdoc.def}|\\
|\childdocby{|\textit{main}|}|\\
\end{tabular}
\end{center}
%
The directive |\childdocby| is similar to |\childdocof|
described in \secref{sec:include},
but the subsequent selection of content must be done manually.
To that end, both |\ifchilddoc| and |\ifchilddocmanual|
will be true upon processing of a part,
and the name of the part is stored in |\childdocname|.
Note that |\jobname| will be set to the filename of the current part
so that each part receives an individual |.aux| file
that does not interfere with the |.aux| file(s) of the main document.
This behaviour can be altered by the alternative form
|\childdocby[*]{|\textit{main}|}| (with a non-empty optional argument)
which uses the |.aux| file of the main document
by setting |\jobname| to \textit{main}.

%%%%%%%%%%%%%%%%%%%%%%%%%%%%%%%%%%%%%%%%%%%%%%%%%%%%%%%%%%%%%%%%%%%%%%%%%%%%%%%%
\subsection{Driver Development}
\label{sec:driver}

The \textsf{childdoc} mechanism can also be use for the development
of definition files such as \LaTeX{} styles or classes.
This case differs from the above setup with multiple parts
included by |\include| in that no |\includeonly| should be invoked.
This can be achieved by starting the include file
(before |\ProvidesPackage|) with:
%
\begin{center}
\begin{tabular}{l}
|\input{childdoc.def}|\\
|\childdocforward{|\textit{main}|}|\\
\end{tabular}
\end{center}
%
or alternatively with:
%
\begin{center}
\begin{tabular}{l}
|\input{childdoc.def}|\\
|\childdocby{|\textit{main}|}|\\
\end{tabular}
\end{center}
%
Both forms have slightly different effects as described above.
The main file is prepared as usual, see \secref{sec:include}.

%%%%%%%%%%%%%%%%%%%%%%%%%%%%%%%%%%%%%%%%%%%%%%%%%%%%%%%%%%%%%%%%%%%%%%%%%%%%%%%%
\subsection{Legacy Detection}
\label{sec:detection}

The directive |\childdocmain| in the main file can detect
whether the complete document or merely a child is to be compiled
even without using the directive |\childdocof|.
This method is deprecated because it is less robust
and there is no compelling reason to use it;
it is merely provided for backward compatibility
and it may be removed in future versions.

If the detection mechanism is to be used,
it is mandatory to correctly specify
the filename of the main file as the argument of |\childdocmain|:
%
\begin{center}
\begin{tabular}{l}
|\input{childdoc.def}|\\
|\childdocmain{|\textit{main}|}|\\
\end{tabular}
\end{center}
%
If |\jobname| does not match the argument \textit{main} of |\childdocmain|,
it is assumed that |\jobname| points to the child file to be compiled.
When using |\childdocmain| with the main file specified as argument,
it suffices to start a child file
with just |\input{|\textit{main}|}|
without loading of the package and using |\childdocof|.
If instead all processing is done
with the appropriate \textsf{childdoc} directives,
the argument of \textit{main} of |\childdocmain| can be empty.

An alternative version of the command line processing described
in \secref{sec:commandline} using the detection mechanism reads:
%
\begin{center}
|... -jobname "|\textit{target}|" "|[\textit{flags}]%
[|\def\jobname{|\textit{dest}|}|]|\input{|\textit{main}|}"|
\end{center}

%%%%%%%%%%%%%%%%%%%%%%%%%%%%%%%%%%%%%%%%%%%%%%%%%%%%%%%%%%%%%%%%%%%%%%%%%%%%%%%%
\subsection{Manual Code}
\label{sec:manual}

In case one cannot be certain whether the definitions file |childdoc.def|
is installed on the target \TeX{} distribution
and one prefers not to ship it,
it is conceivable to paste a few relevant commands into the sources.

To that end, drop all statements |\input{childdoc.def}|
and perform the replacements as outlined below.
Instead of |\childdocmain{|\textit{main}|}| add the following code
to the top of the main file:
%
\begin{center}
\begin{tabular}{l}
|\||ifdefined\childdocname\endinput\||fi\newif\ifchilddoc|\\
|\edef\childdocname{\scantokens\expandafter{\jobname\noexpand}}|\\
|\def\childdocmain{|\textit{main}|}\||ifx\childdocmain\childdocname\||else|\\
|\childdoctrue\includeonly{\childdocname}\let\jobname\childdocmain\||fi|\\
\end{tabular}
\end{center}
%
Instead of |\childdocof{|\textit{main}|}| just include the main file
at the top of each child file:
%
\begin{center}
|\input{|\textit{main}|}|
\end{center}
%
A simple redirection |\childdocforward{|\textit{dest}|}| is achieved by:
%
\begin{center}
|\def\jobname{|\textit{dest}|}\input{\jobname}|
\end{center}
%
The redirection with prefix
|\childdocforwardprefix[|\textit{prefix}|]{|\textit{dest}|}|
is accomplished by:
%
\begin{center}
\begin{tabular}{l}
|{\edef\jobname{\scantokens\expandafter{\jobname\noexpand}}|\\
|\def\redirectjob |\textit{prefix}|#1~~~{\gdef\jobname{|\textit{dest}|#1}}|\\
|\expandafter\redirectjob\jobname~~~}\input{\jobname}|
\end{tabular}
\end{center}

In an alternative approach,
child documents can be compiled by a specific command line
without additional code or specific definitions:
%
\begin{center}
|... -jobname "|\textit{target}|" "|[\textit{flags}]%
|\includeonly{|\textit{dest}|}\input{|\textit{main}|}"|
\end{center}
%

%%%%%%%%%%%%%%%%%%%%%%%%%%%%%%%%%%%%%%%%%%%%%%%%%%%%%%%%%%%%%%%%%%%%%%%%%%%%%%%%
%%%%%%%%%%%%%%%%%%%%%%%%%%%%%%%%%%%%%%%%%%%%%%%%%%%%%%%%%%%%%%%%%%%%%%%%%%%%%%%%
\section{Information}

%%%%%%%%%%%%%%%%%%%%%%%%%%%%%%%%%%%%%%%%%%%%%%%%%%%%%%%%%%%%%%%%%%%%%%%%%%%%%%%%
\subsection{Copyright}

Copyright \copyright{} 2017--2018 Niklas Beisert

This work may be distributed and/or modified under the
conditions of the \LaTeX{} Project Public License, either version 1.3
of this license or (at your option) any later version.
The latest version of this license is in
  \url{http://www.latex-project.org/lppl.txt}
and version 1.3 or later is part of all distributions of \LaTeX{}
version 2005/12/01 or later.

This work has the LPPL maintenance status `maintained'.

The Current Maintainer of this work is Niklas Beisert.

This work consists of the files |README.txt|, |childdoc.ins| and |childdoc.dtx|
as well as the derived files |childdoc.def|, |cdocsamp.tex|
with |cdocsch1.tex|, |cdocsch2.tex|, |cdocspt3.tex|, |cdocspt4.tex|,
|cdocsdrf.tex|, |cdocsfn1.tex|, |cdocsfn2.tex|
as well as |childdoc.pdf|.

%%%%%%%%%%%%%%%%%%%%%%%%%%%%%%%%%%%%%%%%%%%%%%%%%%%%%%%%%%%%%%%%%%%%%%%%%%%%%%%%
\subsection{Files and Installation}

The package consists of the files:
%
\begin{center}
\begin{tabular}{ll}
    |README.txt|   & readme file \\
    |childdoc.ins| & installation file \\
    |childdoc.dtx| & source file \\
    |childdoc.def| & definition file \\
    |cdocsamp.tex| & sample main file \\
    |cdocsch1.tex| & sample include file \\
    |cdocsch2.tex| & sample include file \\
    |cdocspt3.tex| & sample part file \\
    |cdocspt4.tex| & sample part file \\
    |cdocsdrf.tex| & sample redirection file \\
    |cdocsfn1.tex| & sample redirection file \\
    |cdocsfn2.tex| & sample redirection file \\
    |childdoc.pdf| & manual
\end{tabular}
\end{center}
%
The distribution consists of the files
|README.txt|, |childdoc.ins| and |childdoc.dtx|.
%
\begin{itemize}
\item
Run (pdf)\LaTeX{} on |childdoc.dtx|
to compile the manual |childdoc.pdf| (this file).
\item
Run \LaTeX{} on |childdoc.ins| to create the definitions file |childdoc.def|
and the sample |cdocsamp.tex| with include files
|cdocsch1.tex|, |cdocsch2.tex|, |cdocspt3.tex|, |cdocspt4.tex|,
|cdocsdrf.tex|, |cdocsfn1.tex|, |cdocsfn2.tex|.
Then copy the file |childdoc.def| to an appropriate directory of your \LaTeX{}
distribution, e.g.\ \textit{texmf-root}|/tex/latex/childdoc|.
\end{itemize}

%%%%%%%%%%%%%%%%%%%%%%%%%%%%%%%%%%%%%%%%%%%%%%%%%%%%%%%%%%%%%%%%%%%%%%%%%%%%%%%%
\subsection{Related CTAN Packages}

There are several other packages which offer a similar functionality:
%
\begin{itemize}
\item
The packages
\href{http://ctan.org/pkg/docmute}{\textsf{docmute}},
\href{http://ctan.org/pkg/includex}{\textsf{includex}} and
\href{http://ctan.org/pkg/standalone}{\textsf{standalone}}
provide commands to include only the document body of
a child file thus allowing both files to be compiled individually.
\item
The packages \href{http://ctan.org/pkg/subdocs}{\textsf{subdocs}}
and \href{http://ctan.org/pkg/subfiles}{\textsf{subfiles}}
provide structures in which the main and child documents can be
encapsulated and allowing them to be compiled individually.
The inclusion mechanism is different from the conventional |\include|.
\item
The package \href{http://ctan.org/pkg/combine}{\textsf{combine}}
is an elaborate solution to combine several documents into one.
\end{itemize}
%
See also the CTAN topic \href{http://ctan.org/topic/subdocs}{\textsf{subdocs}}
for further related packages.
The present package differs from the above solutions in that
a document structure constructed with the conventional |\include| mechanism
just needs two extra commands at the top of every file
such that all constituent files can be compiled individually.

%%%%%%%%%%%%%%%%%%%%%%%%%%%%%%%%%%%%%%%%%%%%%%%%%%%%%%%%%%%%%%%%%%%%%%%%%%%%%%%%
%\subsection{Feature Suggestions}
%
%The following is a list of features which may be useful for future
%versions of this package:
%%
%\begin{itemize}
%\item
%\ldots
%\end{itemize}

%%%%%%%%%%%%%%%%%%%%%%%%%%%%%%%%%%%%%%%%%%%%%%%%%%%%%%%%%%%%%%%%%%%%%%%%%%%%%%%%
\subsection{Revision History}

%%%%%%%%%%%%%%%%%%%%%%%%%%%%%%%%%%%%%%%%
\paragraph{v2.0:} 2018/12/30

\begin{itemize}
\item
immediate forward processing
\item
added |\childdocby| mechanism
\item
manual restructured
\end{itemize}

%%%%%%%%%%%%%%%%%%%%%%%%%%%%%%%%%%%%%%%%
\paragraph{v1.6:} 2018/01/17

\begin{itemize}
\item
application for development of include files
\item
corrections to manual
\end{itemize}

%%%%%%%%%%%%%%%%%%%%%%%%%%%%%%%%%%%%%%%%
\paragraph{v1.5:} 2017/05/21

\begin{itemize}
\item
more complete structuring introduced
\item
|\childdocof| introduced
\item
|\childdoc| renamed to |\childdocmain|
\item
|\childredirect| renamed to |\childdocforward| and |\childdocforwardprefix|
and functionality expanded
\end{itemize}

%%%%%%%%%%%%%%%%%%%%%%%%%%%%%%%%%%%%%%%%
\paragraph{v1.0:} 2017/04/27

\begin{itemize}
\item
manual and install package
\item
first version published on CTAN
\end{itemize}

%%%%%%%%%%%%%%%%%%%%%%%%%%%%%%%%%%%%%%%%
\paragraph{v0.6:} 2017/04/26

\begin{itemize}
\item
redirection mechanism added
\end{itemize}

%%%%%%%%%%%%%%%%%%%%%%%%%%%%%%%%%%%%%%%%
\paragraph{v0.5:} 2017/04/26

\begin{itemize}
\item
functionality in definition file
\end{itemize}


%%%%%%%%%%%%%%%%%%%%%%%%%%%%%%%%%%%%%%%%%%%%%%%%%%%%%%%%%%%%%%%%%%%%%%%%%%%%%%%%
%%%%%%%%%%%%%%%%%%%%%%%%%%%%%%%%%%%%%%%%%%%%%%%%%%%%%%%%%%%%%%%%%%%%%%%%%%%%%%%%
%%%%%%%%%%%%%%%%%%%%%%%%%%%%%%%%%%%%%%%%%%%%%%%%%%%%%%%%%%%%%%%%%%%%%%%%%%%%%%%%
\appendix

\settowidth\MacroIndent{\rmfamily\scriptsize 000\ }

 \DocInput{childdoc.dtx}

\end{document}
%</driver>
% \fi
%
% %%%%%%%%%%%%%%%%%%%%%%%%%%%%%%%%%%%%%%%%%%%%%%%%%%%%%%%%%%%%%%%%%%%%%%%%%%%%%%
% %%%%%%%%%%%%%%%%%%%%%%%%%%%%%%%%%%%%%%%%%%%%%%%%%%%%%%%%%%%%%%%%%%%%%%%%%%%%%%
% \section{Sample}
%\iffalse
%<*samplemain>
%\fi
%
% The following presents a sample document
% with two chapters, two parts, a title page,
% a compile flag as well as three forwarding files to set the flag.
% It consists of eight |.tex| files:
% \begin{center}
% \begin{tabular}{ll}
% |cdocsamp.tex|&main file\\
% |cdocsch1.tex|&include file for chapter 1\\
% |cdocsch2.tex|&include file for chapter 2\\
% |cdocspt3.tex|&include file for part 3\\
% |cdocspt4.tex|&include file for part 4\\
% |cdocsdrf.tex|&forwarding file for main file in draft mode\\
% |cdocsfi1.tex|&forwarding file for final version of chapter 1\\
% |cdocsfi2.tex|&forwarding file for final version of chapter 2\\
% \end{tabular}
% \end{center}
% Each of the eight files can be compiled directly by the \LaTeX{} compiler.
%
% %%%%%%%%%%%%%%%%%%%%%%%%%%%%%%%%%%%%%%
% \paragraph{Main File.}
%
% The main file is called |cdocsamp.tex|.
%
% Load the \textsf{childdoc} definitions and
% declare the filename for the main document:
%    \begin{macrocode}
\input{childdoc.def}
\childdocmain{}
%    \end{macrocode}

% Optional override for |\version| flag:
%    \begin{macrocode}
%%\ifchilddoc\else\providecommand{\version}{draft}\fi
%    \end{macrocode}

% Define the default values for the |\version| flag
% (|final| for the main file and |draft| for childs):
%    \begin{macrocode}
\ifchilddoc
\providecommand{\version}{draft}
\else
\providecommand{\version}{final}
\fi
%    \end{macrocode}

% Load the standard document class:
%    \begin{macrocode}
\documentclass[12pt]{article}
%    \end{macrocode}

% Start the document body:
%    \begin{macrocode}
\begin{document}
%    \end{macrocode}

% Declare a title page.
% Print title, part of document being processed and version flag:
%    \begin{macrocode}
\addtocounter{page}{-1}
\begin{center}
{\LARGE\bfseries{}childdoc example\par}
\vspace{1cm}
\ifchilddoc
\ifchilddocmanual part\else chapter\fi:
`\childdocname' of `\childdocjob'\par
\else
main document: `\childdocjob'\par
\fi
version: \version\par
\end{center}
\newpage
%    \end{macrocode}

% Manually include selected file,
% otherwise process as usual:
%    \begin{macrocode}
\ifchilddocmanual
\section*{part `\childdocname'}
\input{\childdocname}
\else
%    \end{macrocode}

% Include the two chapters:
%    \begin{macrocode}
\include{cdocsch1}
\include{cdocsch2}
%    \end{macrocode}

% Include the two parts unless only chapters should be displayed:
%    \begin{macrocode}
\ifchilddoc\else
\section{part three}
\input{cdocspt3}
\section{part four}
\input{cdocspt4}
\fi
%    \end{macrocode}

% Process as usual until here:
%    \begin{macrocode}
\fi
%    \end{macrocode}

% End of document body:
%    \begin{macrocode}
\end{document}
%    \end{macrocode}
%\iffalse
%</samplemain>
%\fi
%
% %%%%%%%%%%%%%%%%%%%%%%%%%%%%%%%%%%%%%%
% \paragraph{Chapter Include Files.}
%
% The include files are called |cdocsch1.tex| and |cdocsch2.tex|.
%
%\iffalse
%<*samplechap1|samplechap2>
%\fi

% Optional override for |\version| flag:
%    \begin{macrocode}
%%\providecommand{\version}{final}
%    \end{macrocode}

% Include the main document:
%    \begin{macrocode}
\input{childdoc.def}
\childdocof{cdocsamp}
%    \end{macrocode}

%\iffalse
%</samplechap1|samplechap2>
%\fi
%
%\iffalse
%<*samplechap1>
%\fi
% Some text for chapter 1:
%    \begin{macrocode}
\section{one}
some text in chapter one
%    \end{macrocode}

%\iffalse
%</samplechap1>
%\fi
% Some text for chapter 2:
%\iffalse
%<*samplechap2>
%\fi
%    \begin{macrocode}
\section{two}
more text in chapter two
%    \end{macrocode}

%\iffalse
%</samplechap2>
%\fi
%
% %%%%%%%%%%%%%%%%%%%%%%%%%%%%%%%%%%%%%%
% \paragraph{Part Include Files.}
%
% The include files are called |cdocspt3.tex| and |cdocspt4.tex|.
%
%\iffalse
%<*samplepart3|samplepart4>
%\fi

% Optional override for |\version| flag:
%    \begin{macrocode}
%%\providecommand{\version}{final}
%    \end{macrocode}

% Include the main document:
%    \begin{macrocode}
\input{childdoc.def}
\childdocby{cdocsamp}
%    \end{macrocode}

%\iffalse
%</samplepart3|samplepart4>
%\fi
%
%\iffalse
%<*samplepart3>
%\fi
% Some text for part 3:
%    \begin{macrocode}
some text in part three
%    \end{macrocode}

%\iffalse
%</samplepart3>
%\fi
% Some text for part 4:
%\iffalse
%<*samplepart4>
%\fi
%    \begin{macrocode}
more text in part four
%    \end{macrocode}

%\iffalse
%</samplepart4>
%\fi
%
% %%%%%%%%%%%%%%%%%%%%%%%%%%%%%%%%%%%%%%
% \paragraph{Forwarding for a Complete Draft.}
%
% The following forwarding file |cdocsdrf.tex|
% compiles the main document in draft mode:
%\iffalse
%<*sampledraft>
%\fi
%    \begin{macrocode}
\def\version{draft}
\input{childdoc.def}
\childdocforward{cdocsamp}
%    \end{macrocode}

%\iffalse
%</sampledraft>
%\fi
%
% %%%%%%%%%%%%%%%%%%%%%%%%%%%%%%%%%%%%%%
% \paragraph{Forwarding for Final Version of the Chapters.}
%
% The following forwarding files |cdocsfn1.tex| and |cdocsfn2.tex|
% (with identical content)
% compile the final versions of the child documents
% |cdocsch1.tex| and |cdocsch2.tex|, respectively:
%\iffalse
%<*samplefinal>
%\fi
%    \begin{macrocode}
\def\version{final}
\input{childdoc.def}
\childdocforwardprefix[cdocsamp]{cdocsfn}{cdocsch}
%    \end{macrocode}

%\iffalse
%</samplefinal>
%\fi
%
% %%%%%%%%%%%%%%%%%%%%%%%%%%%%%%%%%%%%%%
% \paragraph{Command Line Processing.}
%
% The following three command lines generate the output files
% |cdocscld|, |cdocscl1| and |cdocscl2|
% which should be identical to
% |cdocsdrf|, |cdocsch1| and |cdocsfn2|, respectively:
% \begin{center}
% \begin{tabular}{l}
% |latex -jobname cdocscld \|\\
% |  "\def\version{draft}\input{childdoc.def}\childdocforward{cdocsamp}"|\\
% |latex -jobname cdocscl1 \|\\
% |  "\input{childdoc.def}\childdocforward[cdocsamp]{cdocsch1}"|\\
% |latex -jobname cdocscl2 \|\\
% |  "\def\version{final}\input{childdoc.def}\childdocforward{cdocsch2}"|
% \end{tabular}
% \end{center}
% Note that the trailing backslash on each first line
% merely continues the input to the second line
% (for convenient cut ant paste).
% Furthermore, the command |latex| can be replaced by any
% of its alternative versions such as |pdflatex|.
%
% %%%%%%%%%%%%%%%%%%%%%%%%%%%%%%%%%%%%%%%%%%%%%%%%%%%%%%%%%%%%%%%%%%%%%%%%%%%%%%
% %%%%%%%%%%%%%%%%%%%%%%%%%%%%%%%%%%%%%%%%%%%%%%%%%%%%%%%%%%%%%%%%%%%%%%%%%%%%%%
% \section{Implementation}
%\iffalse
%<*package>
%\fi
%
% This section describes the definitions file |childdoc.def|.

% The definitions cannot be loaded using |\usepackage| or |\RequirePackage|
% which has a mechanism to prevent loading a style file more than once.
% When loading the definitions by means of |\input|
% multiple instances have to be prevented manually:
%\iffalse
%This code needs to be before the `\ProvidesFile' directive
%which is defined at the beginning of this file.
%Therefore it is also placed there and commented out here.
%</package>
%<*discard>
%\fi
%    \begin{macrocode}
\ifdefined\childdocmain\endinput\fi
%    \end{macrocode}
%\iffalse
%</discard>
%<*package>
%\fi
%
% \macro{\ifchilddoc}
% \macro{\ifchilddocmanual}
% The conditional |\ifchilddoc| tells whether a
% child (true) or main (false) document is being compiled.
% The conditional |\ifchilddocmanual| tells whether
% the |\includeonly| mechanism is used (false) or
% the selection of child files must be performed manually (true).
% The definitions initialise to false:
%    \begin{macrocode}
\newif\ifchilddoc
\newif\ifchilddocmanual
%    \end{macrocode}

% \macro{\childdocname}
% \macro{\childdocjob}
% The macro |\childdocname| stores the name of the main document
% to be compiled. The macro |\childdocjob| stores the name of
% the document on which the \LaTeX{} compiler was originally invoked.
% The content of |\jobname| cannot be compared
% to filenames specified in the source due to different catcodes.
% The following code rescans |\jobname|, stores the result
% in |\childdocname| and saves a copy in |\childdocjob|:
%    \begin{macrocode}
\edef\childdocname{\scantokens\expandafter{\jobname\noexpand}}
\let\childdocjob\childdocname
%    \end{macrocode}

% \macro{\childdocdisable}
% The macro |\childdocdisable| prevents the main file
% from being processed more than once.
% At this stage, the main document command |\childdocmain|
% is assumed to be called once again where it should do nothing.
% Any subsequent call to it should prevent
% a secondary processing of the main document
% It overwrites the forwarding commands
% |\childdocof| and |\childdocforward|
% with empty macros to prevent further inclusions of the main document:
%    \begin{macrocode}
\newcommand{\childdocdisable}
{
  \renewcommand{\childdocmain}[1]{\renewcommand{\childdocmain}[1]{\endinput}}
  \renewcommand{\childdocof}[1]{}
  \renewcommand{\childdocby}[2][]{}
  \renewcommand{\childdocforward}[2][]{}
  \renewcommand{\childdocdisable}{}
}
%    \end{macrocode}

% \macro{\childdocmain}
% The macro |\childdocmain| is to be called at the top of the main file
% with nothing or the main filename (without extension) as argument.
% First, it breaks loops.
% If the argument is not empty and does not match |\childdocname|
% (which is set by the first inclusion of |childdoc.def|),
% |\ifchilddoc| is set to true, |\includeonly| is applied to the child file
% and |\jobname| is set to the main file
% (for proper handling of |.aux| files):
%    \begin{macrocode}
\newcommand{\childdocmain}[1]
{
  \childdocdisable\childdocmain{}
  \if?#1?\else
    \begingroup
      \def\childdoctmp{#1}
      \ifx\childdoctmp\childdocname
        \def\childdoctmp{}
      \else
        \def\childdoctmp
        {
          \childdoctrue
          \includeonly{\childdocname}
          \def\childdocjob{#1}
          \def\jobname{#1}
        }
      \fi
      \expandafter
    \endgroup
    \childdoctmp
  \fi
}
%    \end{macrocode}

% \macro{\childdocof}
% The command |\childdocof| redirects
% compilation to the main file |#1|.
%    \begin{macrocode}
\newcommand{\childdocof}[1]
{
  \childdocdisable
  \childdoctrue
  \includeonly{\childdocname}
  \def\jobname{#1}
  \def\childdocjob{#1}
  \input{#1}
}
%    \end{macrocode}

% \macro{\childdocby}
% The command |\childdocby| ....
%    \begin{macrocode}
\newcommand{\childdocby}[2][]
{
  \childdocdisable
  \childdoctrue
  \childdocmanualtrue
  \if?#1?\else
    \def\jobname{#2}
  \fi
  \def\childdocjob{#2}
  \input{#2}
  \endinput
}
%    \end{macrocode}

% \macro{\childdocforward}
% The command |\childdocforward| redirects
% compilation to the main file or
% (if the optional argument is given) a child file.
% Parameters are set as if the main file
% or a child file starting with |\childdocof| was compiled.
% Then compilation is handed over to the main file:
%    \begin{macrocode}
\newcommand{\childdocforward}[2][]
{
  \begingroup
    \if?#1?
      \def\childdoctmp
      {
        \def\childdocname{#2}
        \def\childdocjob{#2}
        \def\jobname{#2}
        \input{#2}
        \endinput
      }
    \else
      \def\childdoctmp
      {
        \childdocdisable
        \def\childdocname{#2}
        \childdoctrue
        \includeonly{#2}
        \def\childdocjob{#1}
        \def\jobname{#1}
        \input{#1}
        \endinput
      }
    \fi
    \expandafter
  \endgroup
  \childdoctmp
}
%    \end{macrocode}

% \macro{\childdocforwardprefix}
% The command |\childdocforwardprefix| redirects
% compilation to the main or a child file by means of a pattern.
% The prefix |#1| in the current filename is replaced by |#2|
% and the suffix of the current filename is kept
% (it is assumed that the filename does not contain the substring `|~~~|'
% which is used as a delimiter).
% Compilation is handed over to the new file by |\childdocforward|:
%    \begin{macrocode}
\newcommand{\childdocforwardprefix}[3][]
{
  \begingroup
    \def\childdocextract #2##1~~~{\def\childdoctmp{\childdocforward[#1]{#3##1}}}
    \expandafter\childdocextract\childdocname~~~
    \expandafter
  \endgroup
  \childdoctmp
}
%    \end{macrocode}

% \macro{\childdoc}
% The deprecated macro |\childdoc| is a legacy version of |\childdocmain|:
%    \begin{macrocode}
\newcommand{\childdoc}{\childdocmain}
%    \end{macrocode}

% \macro{\childdocredirect}
% The deprecated macro |\childdocredirect| is a legacy version
% of |\childdocforward| and |\childdocforwardprefix|:
%    \begin{macrocode}
\newcommand{\childdocredirect}[2][]
{
  \begingroup
    \if?#1?
      \def\childdoctmp{\childdocforward{#2}}
    \else
      \def\childdoctmp{\childdocforwardprefix{#1}{#2}}
    \fi
    \expandafter
  \endgroup
  \childdoctmp
}
%    \end{macrocode}

%\iffalse
%</package>
%\fi
%
\endinput

\childdocof{cdocsamp}
%    \end{macrocode}

%\iffalse
%</samplechap1|samplechap2>
%\fi
%
%\iffalse
%<*samplechap1>
%\fi
% Some text for chapter 1:
%    \begin{macrocode}
\section{one}
some text in chapter one
%    \end{macrocode}

%\iffalse
%</samplechap1>
%\fi
% Some text for chapter 2:
%\iffalse
%<*samplechap2>
%\fi
%    \begin{macrocode}
\section{two}
more text in chapter two
%    \end{macrocode}

%\iffalse
%</samplechap2>
%\fi
%
% %%%%%%%%%%%%%%%%%%%%%%%%%%%%%%%%%%%%%%
% \paragraph{Part Include Files.}
%
% The include files are called |cdocspt3.tex| and |cdocspt4.tex|.
%
%\iffalse
%<*samplepart3|samplepart4>
%\fi

% Optional override for |\version| flag:
%    \begin{macrocode}
%%\providecommand{\version}{final}
%    \end{macrocode}

% Include the main document:
%    \begin{macrocode}
% \iffalse
%
% childdoc.dtx Copyright (C) 2017-2018 Niklas Beisert
%
% This work may be distributed and/or modified under the
% conditions of the LaTeX Project Public License, either version 1.3
% of this license or (at your option) any later version.
% The latest version of this license is in
%   http://www.latex-project.org/lppl.txt
% and version 1.3 or later is part of all distributions of LaTeX
% version 2005/12/01 or later.
%
% This work has the LPPL maintenance status `maintained'.
%
% The Current Maintainer of this work is Niklas Beisert.
%
% This work consists of the files childdoc.dtx and childdoc.ins
% and the derived files childdoc.def and cdocsamp.tex with
% cdocsch1.tex, cdocsch2.tex, cdocsdrf.tex, cdocsfn1.tex, cdocsfn2.tex.
%
%<package>\ifdefined\childdocmain\endinput\fi
%<package>\ProvidesFile{childdoc.def}[2018/12/30 v2.0 child document driver]
%<samplemain>\ProvidesFile{cdocsamp.tex}[2018/12/30 v2.0 sample for childdoc]
%<*driver>
%\ProvidesFile{childdoc.drv}[2018/12/30 v2.0 childdoc reference manual file]
\PassOptionsToClass{10pt,a4paper}{article}
\documentclass{ltxdoc}

\usepackage[margin=35mm]{geometry}
\usepackage{hyperref}
\usepackage{hyperxmp}
\usepackage[usenames]{color}

\hypersetup{colorlinks=true}
\hypersetup{pdfstartview=FitH}
\hypersetup{pdfpagemode=UseNone}
\hypersetup{pdfsource={}}
\hypersetup{pdflang={en-UK}}
\hypersetup{pdfcopyright={Copyright 2017-2018 Niklas Beisert.
  This work may be distributed and/or modified under the
  conditions of the LaTeX Project Public License, either version 1.3
  of this license or (at your option) any later version.}}
\hypersetup{pdflicenseurl={http://www.latex-project.org/lppl.txt}}
\hypersetup{pdfcontactaddress={ETH Zurich, ITP, HIT K,
  Wolfgang-Pauli-Strasse 27}}
\hypersetup{pdfcontactpostcode={8093}}
\hypersetup{pdfcontactcity={Zurich}}
\hypersetup{pdfcontactcountry={Switzerland}}
\hypersetup{pdfcontactemail={nbeisert@itp.phys.ethz.ch}}
\hypersetup{pdfcontacturl={http://people.phys.ethz.ch/\xmptilde nbeisert/}}

\newcommand{\secref}[1]{\hyperref[#1]{section \ref*{#1}}}

\parskip1ex
\parindent0pt
\let\olditemize\itemize
\def\itemize{\olditemize\parskip0pt}

\begin{document}

\title{The \textsf{childdoc} Package}
\hypersetup{pdftitle={The childdoc Package}}
\author{Niklas Beisert\\[2ex]
  Institut f\"ur Theoretische Physik\\
  Eidgen\"ossische Technische Hochschule Z\"urich\\
  Wolfgang-Pauli-Strasse 27, 8093 Z\"urich, Switzerland\\[1ex]
  \href{mailto:nbeisert@itp.phys.ethz.ch}
  {\texttt{nbeisert@itp.phys.ethz.ch}}}
\hypersetup{pdfauthor={Niklas Beisert}}
\hypersetup{pdfsubject={Manual for the LaTeX2e Package childdoc}}
\date{30 December 2018, \textsf{v2.0}}
\maketitle

\begin{abstract}\noindent
\textsf{childdoc} is a \LaTeXe{} package
that enables the direct compilation
of document sections included by |\include|
to individual files.
\end{abstract}

\begingroup
\parskip0ex
\tableofcontents
\endgroup

%%%%%%%%%%%%%%%%%%%%%%%%%%%%%%%%%%%%%%%%%%%%%%%%%%%%%%%%%%%%%%%%%%%%%%%%%%%%%%%%
%%%%%%%%%%%%%%%%%%%%%%%%%%%%%%%%%%%%%%%%%%%%%%%%%%%%%%%%%%%%%%%%%%%%%%%%%%%%%%%%
\section{Introduction}

\LaTeX{} provides a mechanism to structure a large document (such as a book)
into a main file and several child files (containing the chapters)
using the |\include| command.
This mechanism is beneficial for documents
which span hundreds of pages in order to
make the source file(s) more manageable.
Moreover, compilation can be restricted to
selected child files by means of the |\includeonly| command.
The latter feature can be used to reduce the compilation time while editing
(this was significantly more useful in the earlier days of \LaTeX{})
or to generate a smaller document which is easier to navigate.
Another application of |\includeonly| is to generate
documents consisting of selected parts of the complete document.

However, there are a few drawbacks of the plain |\include| mechanism:
\begin{itemize}
\item
The child files cannot be compiled on their own,
they can only be compiled via the main file.
A naive editing environment
(such as a text editor with an option
to have the current file processed by \LaTeX)
may require one to switch to the main file before compiling;
attempting to compile the child file produces errors.
\item
The main file must be modified (each time)
to adjust the |\includeonly| command
to the present needs. This easily leaves the main file in a messy state.
\item
The generated document will always carry the filename
of the main document. This is inconvenient if
several child files are to be compiled and
to be kept for distribution.
\end{itemize}

The present package provides a simple interface
to make child files individually compilable by \LaTeX{}.
Compiling a child file then has the same effect as compiling
the main file with an |\includeonly| command
to select the appropriate child.
Moreover the generated document will carry the name of the child
rather than the main file.
This resolves all three above issues.

This feature is meant to make the editing of books,
thesis documents and lecture notes somewhat more convenient.
However, the package can also be used efficiently for
composing a series of documents (such as exercise sheets)
which are typically distributed individually.
It then assists the author in generating the individual documents
(potentially in different versions)
as well as a document containing the collected series.
Another application is in developing style files
or other kinds of included material
where compilation of the style file could redirect
to a sample or test file.

%%%%%%%%%%%%%%%%%%%%%%%%%%%%%%%%%%%%%%%%%%%%%%%%%%%%%%%%%%%%%%%%%%%%%%%%%%%%%%%%
%%%%%%%%%%%%%%%%%%%%%%%%%%%%%%%%%%%%%%%%%%%%%%%%%%%%%%%%%%%%%%%%%%%%%%%%%%%%%%%%
\section{Usage}

First of all, the package \textsf{childdoc} is \emph{not} a standard
\LaTeXe{} |.sty| style file! Therefore it needs to be invoked in
a non-standard way.

%%%%%%%%%%%%%%%%%%%%%%%%%%%%%%%%%%%%%%%%%%%%%%%%%%%%%%%%%%%%%%%%%%%%%%%%%%%%%%%%
\subsection{Included Files}
\label{sec:include}

%%%%%%%%%%%%%%%%%%%%%%%%%%%%%%%%%%%%%%%%
\DescribeMacro{\childdocmain}
To use the package, add the commands
\begin{center}
\begin{tabular}{l}
|\input{childdoc.def}|\\
|\childdocmain{}|\\
\end{tabular}
\end{center}
at the very top of the main \LaTeX{} file,
in particular \emph{before} the |\documentclass| statement!
The argument of |\childdocmain| should be left empty
(but it must be present).

%%%%%%%%%%%%%%%%%%%%%%%%%%%%%%%%%%%%%%%%
\DescribeMacro{\childdocof}
Furthermore, add the commands
\begin{center}
\begin{tabular}{l}
|\input{childdoc.def}|\\
|\childdocof{|\textit{main}|}|\\
\end{tabular}
\end{center}
at the top of every child file \textit{child}
which is included by |\include{|\textit{child}|}|
from within the main file
(or at least for those files to be compiled individually).
The argument \textit{main} must be the filename of the main file.

There are a couple of
considerations in setting up the main and child documents:

%%%%%%%%%%%%%%%%%%%%%%%%%%%%%%%%%%%%%%%%
\paragraph{Restrictions.}

Please note the following restrictions:
\begin{itemize}
\item
|\childdocmain| must be called with one argument \textit{main}
to ensure compatibility with earlier version of the package.
It must either be empty (|\childdocmain{}|)
or precisely match the filename of the main file in which it is specified.
See \secref{sec:detection} for further information.
\item
The filename \textit{main} must be specified without the |.tex| extension.
\item
The filename \textit{main} is case sensitive
(even in case-insensitive file systems)
due to internal string comparison.
\item
The argument \textit{main} should be fully expanded, it cannot be a macro.
\item
Subdirectories and special characters should be avoided in filenames.
\item
The command |\childdocmain{|\textit{main}|}| must be followed by a whitespace.
It should not be followed immediately by another command
or by a comment mark `|%|'.
This is because the \TeX{} parser reads the token immediately following
the argument of |\childdocmain| and puts it
at the beginning of every child section;
however, a white\-space is ignored.
\end{itemize}

%%%%%%%%%%%%%%%%%%%%%%%%%%%%%%%%%%%%%%%%
\paragraph{Content of Main File.}

It is advisable to place all content in the child files included by |\include|.
Any output contained in the main file will appear in all child documents
unless suppressed manually;
it cannot be suppressed automatically by the |\includeonly| directive
and thus should normally be avoided.
A method to include some content in the main file
by means of conditional processing is described in \secref{sec:conditional}.

%%%%%%%%%%%%%%%%%%%%%%%%%%%%%%%%%%%%%%%%
\paragraph{Page Numbering.}

When only a part of the document is compiled,
the appropriate numbering of pages
(as well as other status parameters)
is determined from the |.aux| files.
The latter contain information from previous passes.
However this information needs to propagate through
all intermediate child documents.
Therefore the page numbering in child documents may well
be inconsistent until the complete document is compiled at least once.

A useful (if unconventional) way to always ensure a consistent
page numbering is to restart the numbering in each child document
and denote the pages by `\textit{child}|.|\textit{page}'
where \textit{child} represents the chapter/section number of the child file.
This can be achieved by the command
|\numberwithin{page}{|\textit{child}|}|
of the \textsf{amsmath} package
where \textit{child} can be |chapter| or |section|
depending on the chosen structuring.
Alternatively, one can modify the macro |\thepage| appropriately
and reset the counter |page| at the start of each child file.

%%%%%%%%%%%%%%%%%%%%%%%%%%%%%%%%%%%%%%%%%%%%%%%%%%%%%%%%%%%%%%%%%%%%%%%%%%%%%%%%
\subsection{Conditional Processing}
\label{sec:conditional}

The package provides a mechanism to compile different versions
of a document. To customise the versions further some conditional processing
can come in handy to distinguish which version is being compiled.
The package provides two macros to describe the compilation context:

%%%%%%%%%%%%%%%%%%%%%%%%%%%%%%%%%%%%%%%%
\DescribeMacro{\ifchilddoc}
The conditional |\ifchilddoc| distinguishes between the compilation of
child documents and the main document:
%
\begin{center}
|\ifchilddoc |\textit{child-code}| |[|\||else |\textit{main-code}]| \||fi|
\end{center}

%%%%%%%%%%%%%%%%%%%%%%%%%%%%%%%%%%%%%%%%
\DescribeMacro{\childdocname}
\DescribeMacro{\childdocjob}
The macro |\childdocname| contains the filename (without extension)
of the main or child file being processed.
Note that |\childdocjob| will always contain the name of the main file.

%%%%%%%%%%%%%%%%%%%%%%%%%%%%%%%%%%%%%%%%
\paragraph{Title Page.}

Conditional processing can be used to include a title or banner page
in the main document when proper precautions are taken.
Importantly, the code in the main file should ensure that the page counter
(as well as other status parameters which are stored in the |.aux| files)
takes the same value after the conditional processing.
Otherwise the page numbers may take divergent values
depending on which part is compiled.

For example, a title page could be declared by:
%
\begin{center}
\begin{tabular}{l}
|\ifchilddoc\||else|\\
|\addtocounter{page}{-1}|\\
\textit{code for title page}\\
|\newpage|\\
|\||fi|
\end{tabular}
\end{center}
%
A banner page for the child documents can be generated by:
%
\begin{center}
\begin{tabular}{l}
|\ifchilddoc|\\
|\addtocounter{page}{-1}|\\
\textit{code for banner page}\\
|\newpage|\\
|\||fi|
\end{tabular}
\end{center}
%
Here one could write a message such as:
\begin{center}
|This is the part \childdocname{} of \childdocjob{}.|
\end{center}

%%%%%%%%%%%%%%%%%%%%%%%%%%%%%%%%%%%%%%%%%%%%%%%%%%%%%%%%%%%%%%%%%%%%%%%%%%%%%%%%
\subsection{Flags}
\label{sec:flags}

The package makes it easy to generate different versions
of the main or child documents.
To this end compilation flags can be defined
and assigned different default values.
They will be particularly useful in conjunction
with the forwarding mechanism described in \secref{sec:forward}.

For example, it may be useful to have a flag |\version|
which can be set to |draft| or |final|.
The document source will contain some conditional code
depending on the value of |\version|.
Suppose further, the flag should default to |final| for the main file
and to |draft| for child files
which is a natural assignment for editing the document.
This is achieved by placing the following code
in the preamble of the main document
(below the |\childdocmain| directive):
%
\begin{center}
\begin{tabular}{l}
|\ifchilddoc|\\
|\providecommand{\version}{draft}|\\
|\||else|\\
|\providecommand{\version}{final}|\\
|\||fi|
\end{tabular}
\end{center}
%
The definition by |\providecommand| makes sure
that previous definitions are not overwritten.
Further statements |\providecommand{\version}{...}|
can thus be added before the above code to override it.

For the main file, one might add a line
(between |\childdocmain| and the above block)
%
\begin{center}
|%\ifchilddoc\||else\providecommand{\version}{draft}\||fi|
\end{center}
%
which can be uncommented to produce a draft version.
Likewise one can add a line to the very top of a child file
(above the |\childdocof{|\textit{main}|}| directive)
%
\begin{center}
|%\providecommand{\version}{final}|
\end{center}
%
which can be uncommented to produce the final version of this child document.

%%%%%%%%%%%%%%%%%%%%%%%%%%%%%%%%%%%%%%%%%%%%%%%%%%%%%%%%%%%%%%%%%%%%%%%%%%%%%%%%
\subsection{Forwarding}
\label{sec:forward}

Different versions of the main or child documents
using compilation flags as described in \secref{sec:flags}
can be (permanently) stored in different files
for convenient compilation, viewing and distribution.
To this end, the package defines a command
to pass on compilation to a different file:

%%%%%%%%%%%%%%%%%%%%%%%%%%%%%%%%%%%%%%%%
\DescribeMacro{\childdocforward}
The command |\childdocforward| redirects processing to
another source file:
%
\begin{center}
\begin{tabular}{l}
|\input{childdoc.def}|\\
|\childdocforward[|\textit{main}|]{|\textit{dest}|}|\\
\end{tabular}
\end{center}
%
The argument \textit{dest} is the destination file
(without extension).
It should be the main file or one of the child files.
Note that further \textsf{childdoc} directives
such as |\childdocof| and |\childdocforward|
in the indicated file will be processed in this form.
The optional argument \textit{main}
passes on directly to the main file \textit{main}
while pretending to compile the child \textit{dest}.
This form behaves as if \textit{dest}
issues |\childdocof{|\textit{main}|}| right away,
and no further \textsf{childdoc} directives will be processed.

%%%%%%%%%%%%%%%%%%%%%%%%%%%%%%%%%%%%%%%%
\DescribeMacro{\...prefix}
In the alternative form |\childdocforwardprefix|,
%
\begin{center}
\begin{tabular}{l}
|\input{childdoc.def}|\\
|\childdocforwardprefix[|\textit{main}|]{|\textit{prefix}|}{|\textit{dest}|}|
\end{tabular}
\end{center}
%
the destination file is determined by a pattern
depending on the current file:
To make this work, the current file must be called
`{\textit{prefix}\hspace{0.2em}\textit{suffix}}'
with \textit{prefix} matching precisely the argument.
Processing is then passed on to the file
`{\textit{dest}\hspace{0.2em}\textit{suffix}}'.
Surely, the same effect is achieved by
directly specifying the
argument `{\textit{dest}\hspace{0.2em}\textit{suffix}}'
in the first form.
However, that requires to set up a different file
for each child. With the alternative form of the command
all these files can have exactly the same content
which simplifies setting them up and maintaining them.

For example, the following file |draft.tex|
with a compilation flag |\version| as described in \secref{sec:flags}
compiles the main document as a draft:
%
\begin{center}
\begin{tabular}{l}
|\def\version{draft}|\\
|\input{childdoc.def}|\\
|\childdocforward{|\textit{main}|}|
\end{tabular}
\end{center}
%
Likewise, the following files |final|\textit{nn}|.tex|
compile the final version of the child document
|child|\textit{nn}|.tex|:
%
\begin{center}
\begin{tabular}{l}
|\def\version{final}|\\
|\input{childdoc.def}|\\
|\childdocforwardprefix{final}{child}|
\end{tabular}
\end{center}
%

Note that when several versions of a main file and/or of each child file
are to be generated, it may be convenient to set up a |Makefile| or
shell script to automatise the process.

%%%%%%%%%%%%%%%%%%%%%%%%%%%%%%%%%%%%%%%%%%%%%%%%%%%%%%%%%%%%%%%%%%%%%%%%%%%%%%%%
\subsection{Command Line Processing}
\label{sec:commandline}

The effect of redirection files can also be achieved by invoking
the \LaTeX{} compiler with a more elaborate command line.
Most conveniently this should be done as part
of a shell script or a |Makefile|.

When using \textsf{childdoc} in the main file, the following
command lines effectively perform a redirection
(note that depending on the shell being used,
backslashes may have to be doubled: `|\|' $\to$ `|\\|'):
%
\begin{center}
|... -jobname "|\textit{target}|" |\\|"|[\textit{flags}]%
|\input{childdoc.def}\childdocforward[|\textit{main}|]{|\textit{dest}|}"|
\end{center}
%
Here \textit{target} is the name of the output file,
\textit{main} is the name of the main file
and \textit{dest} is the name of the main or child file to be processed
(all filenames without extensions).
The optional argument \textit{main} can be omitted
if \textit{main} matches \textit{dest}.
Optionally, compilation \textit{flags} can be defined via |\def| commands.
This command line makes the \TeX{} engine believe
it is compiling the file \textit{target}
whose content is specified as the latter parameter.
The provided code then forwards the processing to
\textit{main} or \textit{dest} as described in \secref{sec:forward}.

%%%%%%%%%%%%%%%%%%%%%%%%%%%%%%%%%%%%%%%%%%%%%%%%%%%%%%%%%%%%%%%%%%%%%%%%%%%%%%%%
\subsection{Include by Input}
\label{sec:input}

Including child documents by |\include| has some restrictions by design.
Most notably, the content of a child document always occupies
its own set of pages; pages cannot be shared between child documents.
Usually, this behaviour makes perfect sense
because each child document contain an essential part of the document.
However, in some situations it may be desirable to compose
a document from a collection of parts
without having mandatory page breaks between then.
For this case, the package
provides a mechanism to include parts
by |\input| which can also be processed individually.
However, by construction this mechanism
requires manual handling of the content to be output.

%%%%%%%%%%%%%%%%%%%%%%%%%%%%%%%%%%%%%%%%
\DescribeMacro{\ifchilddocmanual}
The main file should be prepared as usual, see \secref{sec:include}.
However, the document body must make a distinction
between processing of an individual part and of the main document, e.g.:
%
\begin{center}
\begin{tabular}{l}
|\ifchilddocmanual|\\
|\input{\childdocname}|\\
|\||else|\\
\textit{document body with }|\input{|\textit{part}|}|\\
|\||fi|
\end{tabular}
\end{center}
%
The conditional |\ifchilddocmanual| is true whenever
a part to be included by |\input| is being compiled,
and the name of the part is stored in |\childdocname|.

%%%%%%%%%%%%%%%%%%%%%%%%%%%%%%%%%%%%%%%%
\DescribeMacro{\childdocby}
Each part to be included by |\input| should start with:
%
\begin{center}
\begin{tabular}{l}
|\input{childdoc.def}|\\
|\childdocby{|\textit{main}|}|\\
\end{tabular}
\end{center}
%
The directive |\childdocby| is similar to |\childdocof|
described in \secref{sec:include},
but the subsequent selection of content must be done manually.
To that end, both |\ifchilddoc| and |\ifchilddocmanual|
will be true upon processing of a part,
and the name of the part is stored in |\childdocname|.
Note that |\jobname| will be set to the filename of the current part
so that each part receives an individual |.aux| file
that does not interfere with the |.aux| file(s) of the main document.
This behaviour can be altered by the alternative form
|\childdocby[*]{|\textit{main}|}| (with a non-empty optional argument)
which uses the |.aux| file of the main document
by setting |\jobname| to \textit{main}.

%%%%%%%%%%%%%%%%%%%%%%%%%%%%%%%%%%%%%%%%%%%%%%%%%%%%%%%%%%%%%%%%%%%%%%%%%%%%%%%%
\subsection{Driver Development}
\label{sec:driver}

The \textsf{childdoc} mechanism can also be use for the development
of definition files such as \LaTeX{} styles or classes.
This case differs from the above setup with multiple parts
included by |\include| in that no |\includeonly| should be invoked.
This can be achieved by starting the include file
(before |\ProvidesPackage|) with:
%
\begin{center}
\begin{tabular}{l}
|\input{childdoc.def}|\\
|\childdocforward{|\textit{main}|}|\\
\end{tabular}
\end{center}
%
or alternatively with:
%
\begin{center}
\begin{tabular}{l}
|\input{childdoc.def}|\\
|\childdocby{|\textit{main}|}|\\
\end{tabular}
\end{center}
%
Both forms have slightly different effects as described above.
The main file is prepared as usual, see \secref{sec:include}.

%%%%%%%%%%%%%%%%%%%%%%%%%%%%%%%%%%%%%%%%%%%%%%%%%%%%%%%%%%%%%%%%%%%%%%%%%%%%%%%%
\subsection{Legacy Detection}
\label{sec:detection}

The directive |\childdocmain| in the main file can detect
whether the complete document or merely a child is to be compiled
even without using the directive |\childdocof|.
This method is deprecated because it is less robust
and there is no compelling reason to use it;
it is merely provided for backward compatibility
and it may be removed in future versions.

If the detection mechanism is to be used,
it is mandatory to correctly specify
the filename of the main file as the argument of |\childdocmain|:
%
\begin{center}
\begin{tabular}{l}
|\input{childdoc.def}|\\
|\childdocmain{|\textit{main}|}|\\
\end{tabular}
\end{center}
%
If |\jobname| does not match the argument \textit{main} of |\childdocmain|,
it is assumed that |\jobname| points to the child file to be compiled.
When using |\childdocmain| with the main file specified as argument,
it suffices to start a child file
with just |\input{|\textit{main}|}|
without loading of the package and using |\childdocof|.
If instead all processing is done
with the appropriate \textsf{childdoc} directives,
the argument of \textit{main} of |\childdocmain| can be empty.

An alternative version of the command line processing described
in \secref{sec:commandline} using the detection mechanism reads:
%
\begin{center}
|... -jobname "|\textit{target}|" "|[\textit{flags}]%
[|\def\jobname{|\textit{dest}|}|]|\input{|\textit{main}|}"|
\end{center}

%%%%%%%%%%%%%%%%%%%%%%%%%%%%%%%%%%%%%%%%%%%%%%%%%%%%%%%%%%%%%%%%%%%%%%%%%%%%%%%%
\subsection{Manual Code}
\label{sec:manual}

In case one cannot be certain whether the definitions file |childdoc.def|
is installed on the target \TeX{} distribution
and one prefers not to ship it,
it is conceivable to paste a few relevant commands into the sources.

To that end, drop all statements |\input{childdoc.def}|
and perform the replacements as outlined below.
Instead of |\childdocmain{|\textit{main}|}| add the following code
to the top of the main file:
%
\begin{center}
\begin{tabular}{l}
|\||ifdefined\childdocname\endinput\||fi\newif\ifchilddoc|\\
|\edef\childdocname{\scantokens\expandafter{\jobname\noexpand}}|\\
|\def\childdocmain{|\textit{main}|}\||ifx\childdocmain\childdocname\||else|\\
|\childdoctrue\includeonly{\childdocname}\let\jobname\childdocmain\||fi|\\
\end{tabular}
\end{center}
%
Instead of |\childdocof{|\textit{main}|}| just include the main file
at the top of each child file:
%
\begin{center}
|\input{|\textit{main}|}|
\end{center}
%
A simple redirection |\childdocforward{|\textit{dest}|}| is achieved by:
%
\begin{center}
|\def\jobname{|\textit{dest}|}\input{\jobname}|
\end{center}
%
The redirection with prefix
|\childdocforwardprefix[|\textit{prefix}|]{|\textit{dest}|}|
is accomplished by:
%
\begin{center}
\begin{tabular}{l}
|{\edef\jobname{\scantokens\expandafter{\jobname\noexpand}}|\\
|\def\redirectjob |\textit{prefix}|#1~~~{\gdef\jobname{|\textit{dest}|#1}}|\\
|\expandafter\redirectjob\jobname~~~}\input{\jobname}|
\end{tabular}
\end{center}

In an alternative approach,
child documents can be compiled by a specific command line
without additional code or specific definitions:
%
\begin{center}
|... -jobname "|\textit{target}|" "|[\textit{flags}]%
|\includeonly{|\textit{dest}|}\input{|\textit{main}|}"|
\end{center}
%

%%%%%%%%%%%%%%%%%%%%%%%%%%%%%%%%%%%%%%%%%%%%%%%%%%%%%%%%%%%%%%%%%%%%%%%%%%%%%%%%
%%%%%%%%%%%%%%%%%%%%%%%%%%%%%%%%%%%%%%%%%%%%%%%%%%%%%%%%%%%%%%%%%%%%%%%%%%%%%%%%
\section{Information}

%%%%%%%%%%%%%%%%%%%%%%%%%%%%%%%%%%%%%%%%%%%%%%%%%%%%%%%%%%%%%%%%%%%%%%%%%%%%%%%%
\subsection{Copyright}

Copyright \copyright{} 2017--2018 Niklas Beisert

This work may be distributed and/or modified under the
conditions of the \LaTeX{} Project Public License, either version 1.3
of this license or (at your option) any later version.
The latest version of this license is in
  \url{http://www.latex-project.org/lppl.txt}
and version 1.3 or later is part of all distributions of \LaTeX{}
version 2005/12/01 or later.

This work has the LPPL maintenance status `maintained'.

The Current Maintainer of this work is Niklas Beisert.

This work consists of the files |README.txt|, |childdoc.ins| and |childdoc.dtx|
as well as the derived files |childdoc.def|, |cdocsamp.tex|
with |cdocsch1.tex|, |cdocsch2.tex|, |cdocspt3.tex|, |cdocspt4.tex|,
|cdocsdrf.tex|, |cdocsfn1.tex|, |cdocsfn2.tex|
as well as |childdoc.pdf|.

%%%%%%%%%%%%%%%%%%%%%%%%%%%%%%%%%%%%%%%%%%%%%%%%%%%%%%%%%%%%%%%%%%%%%%%%%%%%%%%%
\subsection{Files and Installation}

The package consists of the files:
%
\begin{center}
\begin{tabular}{ll}
    |README.txt|   & readme file \\
    |childdoc.ins| & installation file \\
    |childdoc.dtx| & source file \\
    |childdoc.def| & definition file \\
    |cdocsamp.tex| & sample main file \\
    |cdocsch1.tex| & sample include file \\
    |cdocsch2.tex| & sample include file \\
    |cdocspt3.tex| & sample part file \\
    |cdocspt4.tex| & sample part file \\
    |cdocsdrf.tex| & sample redirection file \\
    |cdocsfn1.tex| & sample redirection file \\
    |cdocsfn2.tex| & sample redirection file \\
    |childdoc.pdf| & manual
\end{tabular}
\end{center}
%
The distribution consists of the files
|README.txt|, |childdoc.ins| and |childdoc.dtx|.
%
\begin{itemize}
\item
Run (pdf)\LaTeX{} on |childdoc.dtx|
to compile the manual |childdoc.pdf| (this file).
\item
Run \LaTeX{} on |childdoc.ins| to create the definitions file |childdoc.def|
and the sample |cdocsamp.tex| with include files
|cdocsch1.tex|, |cdocsch2.tex|, |cdocspt3.tex|, |cdocspt4.tex|,
|cdocsdrf.tex|, |cdocsfn1.tex|, |cdocsfn2.tex|.
Then copy the file |childdoc.def| to an appropriate directory of your \LaTeX{}
distribution, e.g.\ \textit{texmf-root}|/tex/latex/childdoc|.
\end{itemize}

%%%%%%%%%%%%%%%%%%%%%%%%%%%%%%%%%%%%%%%%%%%%%%%%%%%%%%%%%%%%%%%%%%%%%%%%%%%%%%%%
\subsection{Related CTAN Packages}

There are several other packages which offer a similar functionality:
%
\begin{itemize}
\item
The packages
\href{http://ctan.org/pkg/docmute}{\textsf{docmute}},
\href{http://ctan.org/pkg/includex}{\textsf{includex}} and
\href{http://ctan.org/pkg/standalone}{\textsf{standalone}}
provide commands to include only the document body of
a child file thus allowing both files to be compiled individually.
\item
The packages \href{http://ctan.org/pkg/subdocs}{\textsf{subdocs}}
and \href{http://ctan.org/pkg/subfiles}{\textsf{subfiles}}
provide structures in which the main and child documents can be
encapsulated and allowing them to be compiled individually.
The inclusion mechanism is different from the conventional |\include|.
\item
The package \href{http://ctan.org/pkg/combine}{\textsf{combine}}
is an elaborate solution to combine several documents into one.
\end{itemize}
%
See also the CTAN topic \href{http://ctan.org/topic/subdocs}{\textsf{subdocs}}
for further related packages.
The present package differs from the above solutions in that
a document structure constructed with the conventional |\include| mechanism
just needs two extra commands at the top of every file
such that all constituent files can be compiled individually.

%%%%%%%%%%%%%%%%%%%%%%%%%%%%%%%%%%%%%%%%%%%%%%%%%%%%%%%%%%%%%%%%%%%%%%%%%%%%%%%%
%\subsection{Feature Suggestions}
%
%The following is a list of features which may be useful for future
%versions of this package:
%%
%\begin{itemize}
%\item
%\ldots
%\end{itemize}

%%%%%%%%%%%%%%%%%%%%%%%%%%%%%%%%%%%%%%%%%%%%%%%%%%%%%%%%%%%%%%%%%%%%%%%%%%%%%%%%
\subsection{Revision History}

%%%%%%%%%%%%%%%%%%%%%%%%%%%%%%%%%%%%%%%%
\paragraph{v2.0:} 2018/12/30

\begin{itemize}
\item
immediate forward processing
\item
added |\childdocby| mechanism
\item
manual restructured
\end{itemize}

%%%%%%%%%%%%%%%%%%%%%%%%%%%%%%%%%%%%%%%%
\paragraph{v1.6:} 2018/01/17

\begin{itemize}
\item
application for development of include files
\item
corrections to manual
\end{itemize}

%%%%%%%%%%%%%%%%%%%%%%%%%%%%%%%%%%%%%%%%
\paragraph{v1.5:} 2017/05/21

\begin{itemize}
\item
more complete structuring introduced
\item
|\childdocof| introduced
\item
|\childdoc| renamed to |\childdocmain|
\item
|\childredirect| renamed to |\childdocforward| and |\childdocforwardprefix|
and functionality expanded
\end{itemize}

%%%%%%%%%%%%%%%%%%%%%%%%%%%%%%%%%%%%%%%%
\paragraph{v1.0:} 2017/04/27

\begin{itemize}
\item
manual and install package
\item
first version published on CTAN
\end{itemize}

%%%%%%%%%%%%%%%%%%%%%%%%%%%%%%%%%%%%%%%%
\paragraph{v0.6:} 2017/04/26

\begin{itemize}
\item
redirection mechanism added
\end{itemize}

%%%%%%%%%%%%%%%%%%%%%%%%%%%%%%%%%%%%%%%%
\paragraph{v0.5:} 2017/04/26

\begin{itemize}
\item
functionality in definition file
\end{itemize}


%%%%%%%%%%%%%%%%%%%%%%%%%%%%%%%%%%%%%%%%%%%%%%%%%%%%%%%%%%%%%%%%%%%%%%%%%%%%%%%%
%%%%%%%%%%%%%%%%%%%%%%%%%%%%%%%%%%%%%%%%%%%%%%%%%%%%%%%%%%%%%%%%%%%%%%%%%%%%%%%%
%%%%%%%%%%%%%%%%%%%%%%%%%%%%%%%%%%%%%%%%%%%%%%%%%%%%%%%%%%%%%%%%%%%%%%%%%%%%%%%%
\appendix

\settowidth\MacroIndent{\rmfamily\scriptsize 000\ }

 \DocInput{childdoc.dtx}

\end{document}
%</driver>
% \fi
%
% %%%%%%%%%%%%%%%%%%%%%%%%%%%%%%%%%%%%%%%%%%%%%%%%%%%%%%%%%%%%%%%%%%%%%%%%%%%%%%
% %%%%%%%%%%%%%%%%%%%%%%%%%%%%%%%%%%%%%%%%%%%%%%%%%%%%%%%%%%%%%%%%%%%%%%%%%%%%%%
% \section{Sample}
%\iffalse
%<*samplemain>
%\fi
%
% The following presents a sample document
% with two chapters, two parts, a title page,
% a compile flag as well as three forwarding files to set the flag.
% It consists of eight |.tex| files:
% \begin{center}
% \begin{tabular}{ll}
% |cdocsamp.tex|&main file\\
% |cdocsch1.tex|&include file for chapter 1\\
% |cdocsch2.tex|&include file for chapter 2\\
% |cdocspt3.tex|&include file for part 3\\
% |cdocspt4.tex|&include file for part 4\\
% |cdocsdrf.tex|&forwarding file for main file in draft mode\\
% |cdocsfi1.tex|&forwarding file for final version of chapter 1\\
% |cdocsfi2.tex|&forwarding file for final version of chapter 2\\
% \end{tabular}
% \end{center}
% Each of the eight files can be compiled directly by the \LaTeX{} compiler.
%
% %%%%%%%%%%%%%%%%%%%%%%%%%%%%%%%%%%%%%%
% \paragraph{Main File.}
%
% The main file is called |cdocsamp.tex|.
%
% Load the \textsf{childdoc} definitions and
% declare the filename for the main document:
%    \begin{macrocode}
\input{childdoc.def}
\childdocmain{}
%    \end{macrocode}

% Optional override for |\version| flag:
%    \begin{macrocode}
%%\ifchilddoc\else\providecommand{\version}{draft}\fi
%    \end{macrocode}

% Define the default values for the |\version| flag
% (|final| for the main file and |draft| for childs):
%    \begin{macrocode}
\ifchilddoc
\providecommand{\version}{draft}
\else
\providecommand{\version}{final}
\fi
%    \end{macrocode}

% Load the standard document class:
%    \begin{macrocode}
\documentclass[12pt]{article}
%    \end{macrocode}

% Start the document body:
%    \begin{macrocode}
\begin{document}
%    \end{macrocode}

% Declare a title page.
% Print title, part of document being processed and version flag:
%    \begin{macrocode}
\addtocounter{page}{-1}
\begin{center}
{\LARGE\bfseries{}childdoc example\par}
\vspace{1cm}
\ifchilddoc
\ifchilddocmanual part\else chapter\fi:
`\childdocname' of `\childdocjob'\par
\else
main document: `\childdocjob'\par
\fi
version: \version\par
\end{center}
\newpage
%    \end{macrocode}

% Manually include selected file,
% otherwise process as usual:
%    \begin{macrocode}
\ifchilddocmanual
\section*{part `\childdocname'}
\input{\childdocname}
\else
%    \end{macrocode}

% Include the two chapters:
%    \begin{macrocode}
\include{cdocsch1}
\include{cdocsch2}
%    \end{macrocode}

% Include the two parts unless only chapters should be displayed:
%    \begin{macrocode}
\ifchilddoc\else
\section{part three}
\input{cdocspt3}
\section{part four}
\input{cdocspt4}
\fi
%    \end{macrocode}

% Process as usual until here:
%    \begin{macrocode}
\fi
%    \end{macrocode}

% End of document body:
%    \begin{macrocode}
\end{document}
%    \end{macrocode}
%\iffalse
%</samplemain>
%\fi
%
% %%%%%%%%%%%%%%%%%%%%%%%%%%%%%%%%%%%%%%
% \paragraph{Chapter Include Files.}
%
% The include files are called |cdocsch1.tex| and |cdocsch2.tex|.
%
%\iffalse
%<*samplechap1|samplechap2>
%\fi

% Optional override for |\version| flag:
%    \begin{macrocode}
%%\providecommand{\version}{final}
%    \end{macrocode}

% Include the main document:
%    \begin{macrocode}
\input{childdoc.def}
\childdocof{cdocsamp}
%    \end{macrocode}

%\iffalse
%</samplechap1|samplechap2>
%\fi
%
%\iffalse
%<*samplechap1>
%\fi
% Some text for chapter 1:
%    \begin{macrocode}
\section{one}
some text in chapter one
%    \end{macrocode}

%\iffalse
%</samplechap1>
%\fi
% Some text for chapter 2:
%\iffalse
%<*samplechap2>
%\fi
%    \begin{macrocode}
\section{two}
more text in chapter two
%    \end{macrocode}

%\iffalse
%</samplechap2>
%\fi
%
% %%%%%%%%%%%%%%%%%%%%%%%%%%%%%%%%%%%%%%
% \paragraph{Part Include Files.}
%
% The include files are called |cdocspt3.tex| and |cdocspt4.tex|.
%
%\iffalse
%<*samplepart3|samplepart4>
%\fi

% Optional override for |\version| flag:
%    \begin{macrocode}
%%\providecommand{\version}{final}
%    \end{macrocode}

% Include the main document:
%    \begin{macrocode}
\input{childdoc.def}
\childdocby{cdocsamp}
%    \end{macrocode}

%\iffalse
%</samplepart3|samplepart4>
%\fi
%
%\iffalse
%<*samplepart3>
%\fi
% Some text for part 3:
%    \begin{macrocode}
some text in part three
%    \end{macrocode}

%\iffalse
%</samplepart3>
%\fi
% Some text for part 4:
%\iffalse
%<*samplepart4>
%\fi
%    \begin{macrocode}
more text in part four
%    \end{macrocode}

%\iffalse
%</samplepart4>
%\fi
%
% %%%%%%%%%%%%%%%%%%%%%%%%%%%%%%%%%%%%%%
% \paragraph{Forwarding for a Complete Draft.}
%
% The following forwarding file |cdocsdrf.tex|
% compiles the main document in draft mode:
%\iffalse
%<*sampledraft>
%\fi
%    \begin{macrocode}
\def\version{draft}
\input{childdoc.def}
\childdocforward{cdocsamp}
%    \end{macrocode}

%\iffalse
%</sampledraft>
%\fi
%
% %%%%%%%%%%%%%%%%%%%%%%%%%%%%%%%%%%%%%%
% \paragraph{Forwarding for Final Version of the Chapters.}
%
% The following forwarding files |cdocsfn1.tex| and |cdocsfn2.tex|
% (with identical content)
% compile the final versions of the child documents
% |cdocsch1.tex| and |cdocsch2.tex|, respectively:
%\iffalse
%<*samplefinal>
%\fi
%    \begin{macrocode}
\def\version{final}
\input{childdoc.def}
\childdocforwardprefix[cdocsamp]{cdocsfn}{cdocsch}
%    \end{macrocode}

%\iffalse
%</samplefinal>
%\fi
%
% %%%%%%%%%%%%%%%%%%%%%%%%%%%%%%%%%%%%%%
% \paragraph{Command Line Processing.}
%
% The following three command lines generate the output files
% |cdocscld|, |cdocscl1| and |cdocscl2|
% which should be identical to
% |cdocsdrf|, |cdocsch1| and |cdocsfn2|, respectively:
% \begin{center}
% \begin{tabular}{l}
% |latex -jobname cdocscld \|\\
% |  "\def\version{draft}\input{childdoc.def}\childdocforward{cdocsamp}"|\\
% |latex -jobname cdocscl1 \|\\
% |  "\input{childdoc.def}\childdocforward[cdocsamp]{cdocsch1}"|\\
% |latex -jobname cdocscl2 \|\\
% |  "\def\version{final}\input{childdoc.def}\childdocforward{cdocsch2}"|
% \end{tabular}
% \end{center}
% Note that the trailing backslash on each first line
% merely continues the input to the second line
% (for convenient cut ant paste).
% Furthermore, the command |latex| can be replaced by any
% of its alternative versions such as |pdflatex|.
%
% %%%%%%%%%%%%%%%%%%%%%%%%%%%%%%%%%%%%%%%%%%%%%%%%%%%%%%%%%%%%%%%%%%%%%%%%%%%%%%
% %%%%%%%%%%%%%%%%%%%%%%%%%%%%%%%%%%%%%%%%%%%%%%%%%%%%%%%%%%%%%%%%%%%%%%%%%%%%%%
% \section{Implementation}
%\iffalse
%<*package>
%\fi
%
% This section describes the definitions file |childdoc.def|.

% The definitions cannot be loaded using |\usepackage| or |\RequirePackage|
% which has a mechanism to prevent loading a style file more than once.
% When loading the definitions by means of |\input|
% multiple instances have to be prevented manually:
%\iffalse
%This code needs to be before the `\ProvidesFile' directive
%which is defined at the beginning of this file.
%Therefore it is also placed there and commented out here.
%</package>
%<*discard>
%\fi
%    \begin{macrocode}
\ifdefined\childdocmain\endinput\fi
%    \end{macrocode}
%\iffalse
%</discard>
%<*package>
%\fi
%
% \macro{\ifchilddoc}
% \macro{\ifchilddocmanual}
% The conditional |\ifchilddoc| tells whether a
% child (true) or main (false) document is being compiled.
% The conditional |\ifchilddocmanual| tells whether
% the |\includeonly| mechanism is used (false) or
% the selection of child files must be performed manually (true).
% The definitions initialise to false:
%    \begin{macrocode}
\newif\ifchilddoc
\newif\ifchilddocmanual
%    \end{macrocode}

% \macro{\childdocname}
% \macro{\childdocjob}
% The macro |\childdocname| stores the name of the main document
% to be compiled. The macro |\childdocjob| stores the name of
% the document on which the \LaTeX{} compiler was originally invoked.
% The content of |\jobname| cannot be compared
% to filenames specified in the source due to different catcodes.
% The following code rescans |\jobname|, stores the result
% in |\childdocname| and saves a copy in |\childdocjob|:
%    \begin{macrocode}
\edef\childdocname{\scantokens\expandafter{\jobname\noexpand}}
\let\childdocjob\childdocname
%    \end{macrocode}

% \macro{\childdocdisable}
% The macro |\childdocdisable| prevents the main file
% from being processed more than once.
% At this stage, the main document command |\childdocmain|
% is assumed to be called once again where it should do nothing.
% Any subsequent call to it should prevent
% a secondary processing of the main document
% It overwrites the forwarding commands
% |\childdocof| and |\childdocforward|
% with empty macros to prevent further inclusions of the main document:
%    \begin{macrocode}
\newcommand{\childdocdisable}
{
  \renewcommand{\childdocmain}[1]{\renewcommand{\childdocmain}[1]{\endinput}}
  \renewcommand{\childdocof}[1]{}
  \renewcommand{\childdocby}[2][]{}
  \renewcommand{\childdocforward}[2][]{}
  \renewcommand{\childdocdisable}{}
}
%    \end{macrocode}

% \macro{\childdocmain}
% The macro |\childdocmain| is to be called at the top of the main file
% with nothing or the main filename (without extension) as argument.
% First, it breaks loops.
% If the argument is not empty and does not match |\childdocname|
% (which is set by the first inclusion of |childdoc.def|),
% |\ifchilddoc| is set to true, |\includeonly| is applied to the child file
% and |\jobname| is set to the main file
% (for proper handling of |.aux| files):
%    \begin{macrocode}
\newcommand{\childdocmain}[1]
{
  \childdocdisable\childdocmain{}
  \if?#1?\else
    \begingroup
      \def\childdoctmp{#1}
      \ifx\childdoctmp\childdocname
        \def\childdoctmp{}
      \else
        \def\childdoctmp
        {
          \childdoctrue
          \includeonly{\childdocname}
          \def\childdocjob{#1}
          \def\jobname{#1}
        }
      \fi
      \expandafter
    \endgroup
    \childdoctmp
  \fi
}
%    \end{macrocode}

% \macro{\childdocof}
% The command |\childdocof| redirects
% compilation to the main file |#1|.
%    \begin{macrocode}
\newcommand{\childdocof}[1]
{
  \childdocdisable
  \childdoctrue
  \includeonly{\childdocname}
  \def\jobname{#1}
  \def\childdocjob{#1}
  \input{#1}
}
%    \end{macrocode}

% \macro{\childdocby}
% The command |\childdocby| ....
%    \begin{macrocode}
\newcommand{\childdocby}[2][]
{
  \childdocdisable
  \childdoctrue
  \childdocmanualtrue
  \if?#1?\else
    \def\jobname{#2}
  \fi
  \def\childdocjob{#2}
  \input{#2}
  \endinput
}
%    \end{macrocode}

% \macro{\childdocforward}
% The command |\childdocforward| redirects
% compilation to the main file or
% (if the optional argument is given) a child file.
% Parameters are set as if the main file
% or a child file starting with |\childdocof| was compiled.
% Then compilation is handed over to the main file:
%    \begin{macrocode}
\newcommand{\childdocforward}[2][]
{
  \begingroup
    \if?#1?
      \def\childdoctmp
      {
        \def\childdocname{#2}
        \def\childdocjob{#2}
        \def\jobname{#2}
        \input{#2}
        \endinput
      }
    \else
      \def\childdoctmp
      {
        \childdocdisable
        \def\childdocname{#2}
        \childdoctrue
        \includeonly{#2}
        \def\childdocjob{#1}
        \def\jobname{#1}
        \input{#1}
        \endinput
      }
    \fi
    \expandafter
  \endgroup
  \childdoctmp
}
%    \end{macrocode}

% \macro{\childdocforwardprefix}
% The command |\childdocforwardprefix| redirects
% compilation to the main or a child file by means of a pattern.
% The prefix |#1| in the current filename is replaced by |#2|
% and the suffix of the current filename is kept
% (it is assumed that the filename does not contain the substring `|~~~|'
% which is used as a delimiter).
% Compilation is handed over to the new file by |\childdocforward|:
%    \begin{macrocode}
\newcommand{\childdocforwardprefix}[3][]
{
  \begingroup
    \def\childdocextract #2##1~~~{\def\childdoctmp{\childdocforward[#1]{#3##1}}}
    \expandafter\childdocextract\childdocname~~~
    \expandafter
  \endgroup
  \childdoctmp
}
%    \end{macrocode}

% \macro{\childdoc}
% The deprecated macro |\childdoc| is a legacy version of |\childdocmain|:
%    \begin{macrocode}
\newcommand{\childdoc}{\childdocmain}
%    \end{macrocode}

% \macro{\childdocredirect}
% The deprecated macro |\childdocredirect| is a legacy version
% of |\childdocforward| and |\childdocforwardprefix|:
%    \begin{macrocode}
\newcommand{\childdocredirect}[2][]
{
  \begingroup
    \if?#1?
      \def\childdoctmp{\childdocforward{#2}}
    \else
      \def\childdoctmp{\childdocforwardprefix{#1}{#2}}
    \fi
    \expandafter
  \endgroup
  \childdoctmp
}
%    \end{macrocode}

%\iffalse
%</package>
%\fi
%
\endinput

\childdocby{cdocsamp}
%    \end{macrocode}

%\iffalse
%</samplepart3|samplepart4>
%\fi
%
%\iffalse
%<*samplepart3>
%\fi
% Some text for part 3:
%    \begin{macrocode}
some text in part three
%    \end{macrocode}

%\iffalse
%</samplepart3>
%\fi
% Some text for part 4:
%\iffalse
%<*samplepart4>
%\fi
%    \begin{macrocode}
more text in part four
%    \end{macrocode}

%\iffalse
%</samplepart4>
%\fi
%
% %%%%%%%%%%%%%%%%%%%%%%%%%%%%%%%%%%%%%%
% \paragraph{Forwarding for a Complete Draft.}
%
% The following forwarding file |cdocsdrf.tex|
% compiles the main document in draft mode:
%\iffalse
%<*sampledraft>
%\fi
%    \begin{macrocode}
\def\version{draft}
% \iffalse
%
% childdoc.dtx Copyright (C) 2017-2018 Niklas Beisert
%
% This work may be distributed and/or modified under the
% conditions of the LaTeX Project Public License, either version 1.3
% of this license or (at your option) any later version.
% The latest version of this license is in
%   http://www.latex-project.org/lppl.txt
% and version 1.3 or later is part of all distributions of LaTeX
% version 2005/12/01 or later.
%
% This work has the LPPL maintenance status `maintained'.
%
% The Current Maintainer of this work is Niklas Beisert.
%
% This work consists of the files childdoc.dtx and childdoc.ins
% and the derived files childdoc.def and cdocsamp.tex with
% cdocsch1.tex, cdocsch2.tex, cdocsdrf.tex, cdocsfn1.tex, cdocsfn2.tex.
%
%<package>\ifdefined\childdocmain\endinput\fi
%<package>\ProvidesFile{childdoc.def}[2018/12/30 v2.0 child document driver]
%<samplemain>\ProvidesFile{cdocsamp.tex}[2018/12/30 v2.0 sample for childdoc]
%<*driver>
%\ProvidesFile{childdoc.drv}[2018/12/30 v2.0 childdoc reference manual file]
\PassOptionsToClass{10pt,a4paper}{article}
\documentclass{ltxdoc}

\usepackage[margin=35mm]{geometry}
\usepackage{hyperref}
\usepackage{hyperxmp}
\usepackage[usenames]{color}

\hypersetup{colorlinks=true}
\hypersetup{pdfstartview=FitH}
\hypersetup{pdfpagemode=UseNone}
\hypersetup{pdfsource={}}
\hypersetup{pdflang={en-UK}}
\hypersetup{pdfcopyright={Copyright 2017-2018 Niklas Beisert.
  This work may be distributed and/or modified under the
  conditions of the LaTeX Project Public License, either version 1.3
  of this license or (at your option) any later version.}}
\hypersetup{pdflicenseurl={http://www.latex-project.org/lppl.txt}}
\hypersetup{pdfcontactaddress={ETH Zurich, ITP, HIT K,
  Wolfgang-Pauli-Strasse 27}}
\hypersetup{pdfcontactpostcode={8093}}
\hypersetup{pdfcontactcity={Zurich}}
\hypersetup{pdfcontactcountry={Switzerland}}
\hypersetup{pdfcontactemail={nbeisert@itp.phys.ethz.ch}}
\hypersetup{pdfcontacturl={http://people.phys.ethz.ch/\xmptilde nbeisert/}}

\newcommand{\secref}[1]{\hyperref[#1]{section \ref*{#1}}}

\parskip1ex
\parindent0pt
\let\olditemize\itemize
\def\itemize{\olditemize\parskip0pt}

\begin{document}

\title{The \textsf{childdoc} Package}
\hypersetup{pdftitle={The childdoc Package}}
\author{Niklas Beisert\\[2ex]
  Institut f\"ur Theoretische Physik\\
  Eidgen\"ossische Technische Hochschule Z\"urich\\
  Wolfgang-Pauli-Strasse 27, 8093 Z\"urich, Switzerland\\[1ex]
  \href{mailto:nbeisert@itp.phys.ethz.ch}
  {\texttt{nbeisert@itp.phys.ethz.ch}}}
\hypersetup{pdfauthor={Niklas Beisert}}
\hypersetup{pdfsubject={Manual for the LaTeX2e Package childdoc}}
\date{30 December 2018, \textsf{v2.0}}
\maketitle

\begin{abstract}\noindent
\textsf{childdoc} is a \LaTeXe{} package
that enables the direct compilation
of document sections included by |\include|
to individual files.
\end{abstract}

\begingroup
\parskip0ex
\tableofcontents
\endgroup

%%%%%%%%%%%%%%%%%%%%%%%%%%%%%%%%%%%%%%%%%%%%%%%%%%%%%%%%%%%%%%%%%%%%%%%%%%%%%%%%
%%%%%%%%%%%%%%%%%%%%%%%%%%%%%%%%%%%%%%%%%%%%%%%%%%%%%%%%%%%%%%%%%%%%%%%%%%%%%%%%
\section{Introduction}

\LaTeX{} provides a mechanism to structure a large document (such as a book)
into a main file and several child files (containing the chapters)
using the |\include| command.
This mechanism is beneficial for documents
which span hundreds of pages in order to
make the source file(s) more manageable.
Moreover, compilation can be restricted to
selected child files by means of the |\includeonly| command.
The latter feature can be used to reduce the compilation time while editing
(this was significantly more useful in the earlier days of \LaTeX{})
or to generate a smaller document which is easier to navigate.
Another application of |\includeonly| is to generate
documents consisting of selected parts of the complete document.

However, there are a few drawbacks of the plain |\include| mechanism:
\begin{itemize}
\item
The child files cannot be compiled on their own,
they can only be compiled via the main file.
A naive editing environment
(such as a text editor with an option
to have the current file processed by \LaTeX)
may require one to switch to the main file before compiling;
attempting to compile the child file produces errors.
\item
The main file must be modified (each time)
to adjust the |\includeonly| command
to the present needs. This easily leaves the main file in a messy state.
\item
The generated document will always carry the filename
of the main document. This is inconvenient if
several child files are to be compiled and
to be kept for distribution.
\end{itemize}

The present package provides a simple interface
to make child files individually compilable by \LaTeX{}.
Compiling a child file then has the same effect as compiling
the main file with an |\includeonly| command
to select the appropriate child.
Moreover the generated document will carry the name of the child
rather than the main file.
This resolves all three above issues.

This feature is meant to make the editing of books,
thesis documents and lecture notes somewhat more convenient.
However, the package can also be used efficiently for
composing a series of documents (such as exercise sheets)
which are typically distributed individually.
It then assists the author in generating the individual documents
(potentially in different versions)
as well as a document containing the collected series.
Another application is in developing style files
or other kinds of included material
where compilation of the style file could redirect
to a sample or test file.

%%%%%%%%%%%%%%%%%%%%%%%%%%%%%%%%%%%%%%%%%%%%%%%%%%%%%%%%%%%%%%%%%%%%%%%%%%%%%%%%
%%%%%%%%%%%%%%%%%%%%%%%%%%%%%%%%%%%%%%%%%%%%%%%%%%%%%%%%%%%%%%%%%%%%%%%%%%%%%%%%
\section{Usage}

First of all, the package \textsf{childdoc} is \emph{not} a standard
\LaTeXe{} |.sty| style file! Therefore it needs to be invoked in
a non-standard way.

%%%%%%%%%%%%%%%%%%%%%%%%%%%%%%%%%%%%%%%%%%%%%%%%%%%%%%%%%%%%%%%%%%%%%%%%%%%%%%%%
\subsection{Included Files}
\label{sec:include}

%%%%%%%%%%%%%%%%%%%%%%%%%%%%%%%%%%%%%%%%
\DescribeMacro{\childdocmain}
To use the package, add the commands
\begin{center}
\begin{tabular}{l}
|\input{childdoc.def}|\\
|\childdocmain{}|\\
\end{tabular}
\end{center}
at the very top of the main \LaTeX{} file,
in particular \emph{before} the |\documentclass| statement!
The argument of |\childdocmain| should be left empty
(but it must be present).

%%%%%%%%%%%%%%%%%%%%%%%%%%%%%%%%%%%%%%%%
\DescribeMacro{\childdocof}
Furthermore, add the commands
\begin{center}
\begin{tabular}{l}
|\input{childdoc.def}|\\
|\childdocof{|\textit{main}|}|\\
\end{tabular}
\end{center}
at the top of every child file \textit{child}
which is included by |\include{|\textit{child}|}|
from within the main file
(or at least for those files to be compiled individually).
The argument \textit{main} must be the filename of the main file.

There are a couple of
considerations in setting up the main and child documents:

%%%%%%%%%%%%%%%%%%%%%%%%%%%%%%%%%%%%%%%%
\paragraph{Restrictions.}

Please note the following restrictions:
\begin{itemize}
\item
|\childdocmain| must be called with one argument \textit{main}
to ensure compatibility with earlier version of the package.
It must either be empty (|\childdocmain{}|)
or precisely match the filename of the main file in which it is specified.
See \secref{sec:detection} for further information.
\item
The filename \textit{main} must be specified without the |.tex| extension.
\item
The filename \textit{main} is case sensitive
(even in case-insensitive file systems)
due to internal string comparison.
\item
The argument \textit{main} should be fully expanded, it cannot be a macro.
\item
Subdirectories and special characters should be avoided in filenames.
\item
The command |\childdocmain{|\textit{main}|}| must be followed by a whitespace.
It should not be followed immediately by another command
or by a comment mark `|%|'.
This is because the \TeX{} parser reads the token immediately following
the argument of |\childdocmain| and puts it
at the beginning of every child section;
however, a white\-space is ignored.
\end{itemize}

%%%%%%%%%%%%%%%%%%%%%%%%%%%%%%%%%%%%%%%%
\paragraph{Content of Main File.}

It is advisable to place all content in the child files included by |\include|.
Any output contained in the main file will appear in all child documents
unless suppressed manually;
it cannot be suppressed automatically by the |\includeonly| directive
and thus should normally be avoided.
A method to include some content in the main file
by means of conditional processing is described in \secref{sec:conditional}.

%%%%%%%%%%%%%%%%%%%%%%%%%%%%%%%%%%%%%%%%
\paragraph{Page Numbering.}

When only a part of the document is compiled,
the appropriate numbering of pages
(as well as other status parameters)
is determined from the |.aux| files.
The latter contain information from previous passes.
However this information needs to propagate through
all intermediate child documents.
Therefore the page numbering in child documents may well
be inconsistent until the complete document is compiled at least once.

A useful (if unconventional) way to always ensure a consistent
page numbering is to restart the numbering in each child document
and denote the pages by `\textit{child}|.|\textit{page}'
where \textit{child} represents the chapter/section number of the child file.
This can be achieved by the command
|\numberwithin{page}{|\textit{child}|}|
of the \textsf{amsmath} package
where \textit{child} can be |chapter| or |section|
depending on the chosen structuring.
Alternatively, one can modify the macro |\thepage| appropriately
and reset the counter |page| at the start of each child file.

%%%%%%%%%%%%%%%%%%%%%%%%%%%%%%%%%%%%%%%%%%%%%%%%%%%%%%%%%%%%%%%%%%%%%%%%%%%%%%%%
\subsection{Conditional Processing}
\label{sec:conditional}

The package provides a mechanism to compile different versions
of a document. To customise the versions further some conditional processing
can come in handy to distinguish which version is being compiled.
The package provides two macros to describe the compilation context:

%%%%%%%%%%%%%%%%%%%%%%%%%%%%%%%%%%%%%%%%
\DescribeMacro{\ifchilddoc}
The conditional |\ifchilddoc| distinguishes between the compilation of
child documents and the main document:
%
\begin{center}
|\ifchilddoc |\textit{child-code}| |[|\||else |\textit{main-code}]| \||fi|
\end{center}

%%%%%%%%%%%%%%%%%%%%%%%%%%%%%%%%%%%%%%%%
\DescribeMacro{\childdocname}
\DescribeMacro{\childdocjob}
The macro |\childdocname| contains the filename (without extension)
of the main or child file being processed.
Note that |\childdocjob| will always contain the name of the main file.

%%%%%%%%%%%%%%%%%%%%%%%%%%%%%%%%%%%%%%%%
\paragraph{Title Page.}

Conditional processing can be used to include a title or banner page
in the main document when proper precautions are taken.
Importantly, the code in the main file should ensure that the page counter
(as well as other status parameters which are stored in the |.aux| files)
takes the same value after the conditional processing.
Otherwise the page numbers may take divergent values
depending on which part is compiled.

For example, a title page could be declared by:
%
\begin{center}
\begin{tabular}{l}
|\ifchilddoc\||else|\\
|\addtocounter{page}{-1}|\\
\textit{code for title page}\\
|\newpage|\\
|\||fi|
\end{tabular}
\end{center}
%
A banner page for the child documents can be generated by:
%
\begin{center}
\begin{tabular}{l}
|\ifchilddoc|\\
|\addtocounter{page}{-1}|\\
\textit{code for banner page}\\
|\newpage|\\
|\||fi|
\end{tabular}
\end{center}
%
Here one could write a message such as:
\begin{center}
|This is the part \childdocname{} of \childdocjob{}.|
\end{center}

%%%%%%%%%%%%%%%%%%%%%%%%%%%%%%%%%%%%%%%%%%%%%%%%%%%%%%%%%%%%%%%%%%%%%%%%%%%%%%%%
\subsection{Flags}
\label{sec:flags}

The package makes it easy to generate different versions
of the main or child documents.
To this end compilation flags can be defined
and assigned different default values.
They will be particularly useful in conjunction
with the forwarding mechanism described in \secref{sec:forward}.

For example, it may be useful to have a flag |\version|
which can be set to |draft| or |final|.
The document source will contain some conditional code
depending on the value of |\version|.
Suppose further, the flag should default to |final| for the main file
and to |draft| for child files
which is a natural assignment for editing the document.
This is achieved by placing the following code
in the preamble of the main document
(below the |\childdocmain| directive):
%
\begin{center}
\begin{tabular}{l}
|\ifchilddoc|\\
|\providecommand{\version}{draft}|\\
|\||else|\\
|\providecommand{\version}{final}|\\
|\||fi|
\end{tabular}
\end{center}
%
The definition by |\providecommand| makes sure
that previous definitions are not overwritten.
Further statements |\providecommand{\version}{...}|
can thus be added before the above code to override it.

For the main file, one might add a line
(between |\childdocmain| and the above block)
%
\begin{center}
|%\ifchilddoc\||else\providecommand{\version}{draft}\||fi|
\end{center}
%
which can be uncommented to produce a draft version.
Likewise one can add a line to the very top of a child file
(above the |\childdocof{|\textit{main}|}| directive)
%
\begin{center}
|%\providecommand{\version}{final}|
\end{center}
%
which can be uncommented to produce the final version of this child document.

%%%%%%%%%%%%%%%%%%%%%%%%%%%%%%%%%%%%%%%%%%%%%%%%%%%%%%%%%%%%%%%%%%%%%%%%%%%%%%%%
\subsection{Forwarding}
\label{sec:forward}

Different versions of the main or child documents
using compilation flags as described in \secref{sec:flags}
can be (permanently) stored in different files
for convenient compilation, viewing and distribution.
To this end, the package defines a command
to pass on compilation to a different file:

%%%%%%%%%%%%%%%%%%%%%%%%%%%%%%%%%%%%%%%%
\DescribeMacro{\childdocforward}
The command |\childdocforward| redirects processing to
another source file:
%
\begin{center}
\begin{tabular}{l}
|\input{childdoc.def}|\\
|\childdocforward[|\textit{main}|]{|\textit{dest}|}|\\
\end{tabular}
\end{center}
%
The argument \textit{dest} is the destination file
(without extension).
It should be the main file or one of the child files.
Note that further \textsf{childdoc} directives
such as |\childdocof| and |\childdocforward|
in the indicated file will be processed in this form.
The optional argument \textit{main}
passes on directly to the main file \textit{main}
while pretending to compile the child \textit{dest}.
This form behaves as if \textit{dest}
issues |\childdocof{|\textit{main}|}| right away,
and no further \textsf{childdoc} directives will be processed.

%%%%%%%%%%%%%%%%%%%%%%%%%%%%%%%%%%%%%%%%
\DescribeMacro{\...prefix}
In the alternative form |\childdocforwardprefix|,
%
\begin{center}
\begin{tabular}{l}
|\input{childdoc.def}|\\
|\childdocforwardprefix[|\textit{main}|]{|\textit{prefix}|}{|\textit{dest}|}|
\end{tabular}
\end{center}
%
the destination file is determined by a pattern
depending on the current file:
To make this work, the current file must be called
`{\textit{prefix}\hspace{0.2em}\textit{suffix}}'
with \textit{prefix} matching precisely the argument.
Processing is then passed on to the file
`{\textit{dest}\hspace{0.2em}\textit{suffix}}'.
Surely, the same effect is achieved by
directly specifying the
argument `{\textit{dest}\hspace{0.2em}\textit{suffix}}'
in the first form.
However, that requires to set up a different file
for each child. With the alternative form of the command
all these files can have exactly the same content
which simplifies setting them up and maintaining them.

For example, the following file |draft.tex|
with a compilation flag |\version| as described in \secref{sec:flags}
compiles the main document as a draft:
%
\begin{center}
\begin{tabular}{l}
|\def\version{draft}|\\
|\input{childdoc.def}|\\
|\childdocforward{|\textit{main}|}|
\end{tabular}
\end{center}
%
Likewise, the following files |final|\textit{nn}|.tex|
compile the final version of the child document
|child|\textit{nn}|.tex|:
%
\begin{center}
\begin{tabular}{l}
|\def\version{final}|\\
|\input{childdoc.def}|\\
|\childdocforwardprefix{final}{child}|
\end{tabular}
\end{center}
%

Note that when several versions of a main file and/or of each child file
are to be generated, it may be convenient to set up a |Makefile| or
shell script to automatise the process.

%%%%%%%%%%%%%%%%%%%%%%%%%%%%%%%%%%%%%%%%%%%%%%%%%%%%%%%%%%%%%%%%%%%%%%%%%%%%%%%%
\subsection{Command Line Processing}
\label{sec:commandline}

The effect of redirection files can also be achieved by invoking
the \LaTeX{} compiler with a more elaborate command line.
Most conveniently this should be done as part
of a shell script or a |Makefile|.

When using \textsf{childdoc} in the main file, the following
command lines effectively perform a redirection
(note that depending on the shell being used,
backslashes may have to be doubled: `|\|' $\to$ `|\\|'):
%
\begin{center}
|... -jobname "|\textit{target}|" |\\|"|[\textit{flags}]%
|\input{childdoc.def}\childdocforward[|\textit{main}|]{|\textit{dest}|}"|
\end{center}
%
Here \textit{target} is the name of the output file,
\textit{main} is the name of the main file
and \textit{dest} is the name of the main or child file to be processed
(all filenames without extensions).
The optional argument \textit{main} can be omitted
if \textit{main} matches \textit{dest}.
Optionally, compilation \textit{flags} can be defined via |\def| commands.
This command line makes the \TeX{} engine believe
it is compiling the file \textit{target}
whose content is specified as the latter parameter.
The provided code then forwards the processing to
\textit{main} or \textit{dest} as described in \secref{sec:forward}.

%%%%%%%%%%%%%%%%%%%%%%%%%%%%%%%%%%%%%%%%%%%%%%%%%%%%%%%%%%%%%%%%%%%%%%%%%%%%%%%%
\subsection{Include by Input}
\label{sec:input}

Including child documents by |\include| has some restrictions by design.
Most notably, the content of a child document always occupies
its own set of pages; pages cannot be shared between child documents.
Usually, this behaviour makes perfect sense
because each child document contain an essential part of the document.
However, in some situations it may be desirable to compose
a document from a collection of parts
without having mandatory page breaks between then.
For this case, the package
provides a mechanism to include parts
by |\input| which can also be processed individually.
However, by construction this mechanism
requires manual handling of the content to be output.

%%%%%%%%%%%%%%%%%%%%%%%%%%%%%%%%%%%%%%%%
\DescribeMacro{\ifchilddocmanual}
The main file should be prepared as usual, see \secref{sec:include}.
However, the document body must make a distinction
between processing of an individual part and of the main document, e.g.:
%
\begin{center}
\begin{tabular}{l}
|\ifchilddocmanual|\\
|\input{\childdocname}|\\
|\||else|\\
\textit{document body with }|\input{|\textit{part}|}|\\
|\||fi|
\end{tabular}
\end{center}
%
The conditional |\ifchilddocmanual| is true whenever
a part to be included by |\input| is being compiled,
and the name of the part is stored in |\childdocname|.

%%%%%%%%%%%%%%%%%%%%%%%%%%%%%%%%%%%%%%%%
\DescribeMacro{\childdocby}
Each part to be included by |\input| should start with:
%
\begin{center}
\begin{tabular}{l}
|\input{childdoc.def}|\\
|\childdocby{|\textit{main}|}|\\
\end{tabular}
\end{center}
%
The directive |\childdocby| is similar to |\childdocof|
described in \secref{sec:include},
but the subsequent selection of content must be done manually.
To that end, both |\ifchilddoc| and |\ifchilddocmanual|
will be true upon processing of a part,
and the name of the part is stored in |\childdocname|.
Note that |\jobname| will be set to the filename of the current part
so that each part receives an individual |.aux| file
that does not interfere with the |.aux| file(s) of the main document.
This behaviour can be altered by the alternative form
|\childdocby[*]{|\textit{main}|}| (with a non-empty optional argument)
which uses the |.aux| file of the main document
by setting |\jobname| to \textit{main}.

%%%%%%%%%%%%%%%%%%%%%%%%%%%%%%%%%%%%%%%%%%%%%%%%%%%%%%%%%%%%%%%%%%%%%%%%%%%%%%%%
\subsection{Driver Development}
\label{sec:driver}

The \textsf{childdoc} mechanism can also be use for the development
of definition files such as \LaTeX{} styles or classes.
This case differs from the above setup with multiple parts
included by |\include| in that no |\includeonly| should be invoked.
This can be achieved by starting the include file
(before |\ProvidesPackage|) with:
%
\begin{center}
\begin{tabular}{l}
|\input{childdoc.def}|\\
|\childdocforward{|\textit{main}|}|\\
\end{tabular}
\end{center}
%
or alternatively with:
%
\begin{center}
\begin{tabular}{l}
|\input{childdoc.def}|\\
|\childdocby{|\textit{main}|}|\\
\end{tabular}
\end{center}
%
Both forms have slightly different effects as described above.
The main file is prepared as usual, see \secref{sec:include}.

%%%%%%%%%%%%%%%%%%%%%%%%%%%%%%%%%%%%%%%%%%%%%%%%%%%%%%%%%%%%%%%%%%%%%%%%%%%%%%%%
\subsection{Legacy Detection}
\label{sec:detection}

The directive |\childdocmain| in the main file can detect
whether the complete document or merely a child is to be compiled
even without using the directive |\childdocof|.
This method is deprecated because it is less robust
and there is no compelling reason to use it;
it is merely provided for backward compatibility
and it may be removed in future versions.

If the detection mechanism is to be used,
it is mandatory to correctly specify
the filename of the main file as the argument of |\childdocmain|:
%
\begin{center}
\begin{tabular}{l}
|\input{childdoc.def}|\\
|\childdocmain{|\textit{main}|}|\\
\end{tabular}
\end{center}
%
If |\jobname| does not match the argument \textit{main} of |\childdocmain|,
it is assumed that |\jobname| points to the child file to be compiled.
When using |\childdocmain| with the main file specified as argument,
it suffices to start a child file
with just |\input{|\textit{main}|}|
without loading of the package and using |\childdocof|.
If instead all processing is done
with the appropriate \textsf{childdoc} directives,
the argument of \textit{main} of |\childdocmain| can be empty.

An alternative version of the command line processing described
in \secref{sec:commandline} using the detection mechanism reads:
%
\begin{center}
|... -jobname "|\textit{target}|" "|[\textit{flags}]%
[|\def\jobname{|\textit{dest}|}|]|\input{|\textit{main}|}"|
\end{center}

%%%%%%%%%%%%%%%%%%%%%%%%%%%%%%%%%%%%%%%%%%%%%%%%%%%%%%%%%%%%%%%%%%%%%%%%%%%%%%%%
\subsection{Manual Code}
\label{sec:manual}

In case one cannot be certain whether the definitions file |childdoc.def|
is installed on the target \TeX{} distribution
and one prefers not to ship it,
it is conceivable to paste a few relevant commands into the sources.

To that end, drop all statements |\input{childdoc.def}|
and perform the replacements as outlined below.
Instead of |\childdocmain{|\textit{main}|}| add the following code
to the top of the main file:
%
\begin{center}
\begin{tabular}{l}
|\||ifdefined\childdocname\endinput\||fi\newif\ifchilddoc|\\
|\edef\childdocname{\scantokens\expandafter{\jobname\noexpand}}|\\
|\def\childdocmain{|\textit{main}|}\||ifx\childdocmain\childdocname\||else|\\
|\childdoctrue\includeonly{\childdocname}\let\jobname\childdocmain\||fi|\\
\end{tabular}
\end{center}
%
Instead of |\childdocof{|\textit{main}|}| just include the main file
at the top of each child file:
%
\begin{center}
|\input{|\textit{main}|}|
\end{center}
%
A simple redirection |\childdocforward{|\textit{dest}|}| is achieved by:
%
\begin{center}
|\def\jobname{|\textit{dest}|}\input{\jobname}|
\end{center}
%
The redirection with prefix
|\childdocforwardprefix[|\textit{prefix}|]{|\textit{dest}|}|
is accomplished by:
%
\begin{center}
\begin{tabular}{l}
|{\edef\jobname{\scantokens\expandafter{\jobname\noexpand}}|\\
|\def\redirectjob |\textit{prefix}|#1~~~{\gdef\jobname{|\textit{dest}|#1}}|\\
|\expandafter\redirectjob\jobname~~~}\input{\jobname}|
\end{tabular}
\end{center}

In an alternative approach,
child documents can be compiled by a specific command line
without additional code or specific definitions:
%
\begin{center}
|... -jobname "|\textit{target}|" "|[\textit{flags}]%
|\includeonly{|\textit{dest}|}\input{|\textit{main}|}"|
\end{center}
%

%%%%%%%%%%%%%%%%%%%%%%%%%%%%%%%%%%%%%%%%%%%%%%%%%%%%%%%%%%%%%%%%%%%%%%%%%%%%%%%%
%%%%%%%%%%%%%%%%%%%%%%%%%%%%%%%%%%%%%%%%%%%%%%%%%%%%%%%%%%%%%%%%%%%%%%%%%%%%%%%%
\section{Information}

%%%%%%%%%%%%%%%%%%%%%%%%%%%%%%%%%%%%%%%%%%%%%%%%%%%%%%%%%%%%%%%%%%%%%%%%%%%%%%%%
\subsection{Copyright}

Copyright \copyright{} 2017--2018 Niklas Beisert

This work may be distributed and/or modified under the
conditions of the \LaTeX{} Project Public License, either version 1.3
of this license or (at your option) any later version.
The latest version of this license is in
  \url{http://www.latex-project.org/lppl.txt}
and version 1.3 or later is part of all distributions of \LaTeX{}
version 2005/12/01 or later.

This work has the LPPL maintenance status `maintained'.

The Current Maintainer of this work is Niklas Beisert.

This work consists of the files |README.txt|, |childdoc.ins| and |childdoc.dtx|
as well as the derived files |childdoc.def|, |cdocsamp.tex|
with |cdocsch1.tex|, |cdocsch2.tex|, |cdocspt3.tex|, |cdocspt4.tex|,
|cdocsdrf.tex|, |cdocsfn1.tex|, |cdocsfn2.tex|
as well as |childdoc.pdf|.

%%%%%%%%%%%%%%%%%%%%%%%%%%%%%%%%%%%%%%%%%%%%%%%%%%%%%%%%%%%%%%%%%%%%%%%%%%%%%%%%
\subsection{Files and Installation}

The package consists of the files:
%
\begin{center}
\begin{tabular}{ll}
    |README.txt|   & readme file \\
    |childdoc.ins| & installation file \\
    |childdoc.dtx| & source file \\
    |childdoc.def| & definition file \\
    |cdocsamp.tex| & sample main file \\
    |cdocsch1.tex| & sample include file \\
    |cdocsch2.tex| & sample include file \\
    |cdocspt3.tex| & sample part file \\
    |cdocspt4.tex| & sample part file \\
    |cdocsdrf.tex| & sample redirection file \\
    |cdocsfn1.tex| & sample redirection file \\
    |cdocsfn2.tex| & sample redirection file \\
    |childdoc.pdf| & manual
\end{tabular}
\end{center}
%
The distribution consists of the files
|README.txt|, |childdoc.ins| and |childdoc.dtx|.
%
\begin{itemize}
\item
Run (pdf)\LaTeX{} on |childdoc.dtx|
to compile the manual |childdoc.pdf| (this file).
\item
Run \LaTeX{} on |childdoc.ins| to create the definitions file |childdoc.def|
and the sample |cdocsamp.tex| with include files
|cdocsch1.tex|, |cdocsch2.tex|, |cdocspt3.tex|, |cdocspt4.tex|,
|cdocsdrf.tex|, |cdocsfn1.tex|, |cdocsfn2.tex|.
Then copy the file |childdoc.def| to an appropriate directory of your \LaTeX{}
distribution, e.g.\ \textit{texmf-root}|/tex/latex/childdoc|.
\end{itemize}

%%%%%%%%%%%%%%%%%%%%%%%%%%%%%%%%%%%%%%%%%%%%%%%%%%%%%%%%%%%%%%%%%%%%%%%%%%%%%%%%
\subsection{Related CTAN Packages}

There are several other packages which offer a similar functionality:
%
\begin{itemize}
\item
The packages
\href{http://ctan.org/pkg/docmute}{\textsf{docmute}},
\href{http://ctan.org/pkg/includex}{\textsf{includex}} and
\href{http://ctan.org/pkg/standalone}{\textsf{standalone}}
provide commands to include only the document body of
a child file thus allowing both files to be compiled individually.
\item
The packages \href{http://ctan.org/pkg/subdocs}{\textsf{subdocs}}
and \href{http://ctan.org/pkg/subfiles}{\textsf{subfiles}}
provide structures in which the main and child documents can be
encapsulated and allowing them to be compiled individually.
The inclusion mechanism is different from the conventional |\include|.
\item
The package \href{http://ctan.org/pkg/combine}{\textsf{combine}}
is an elaborate solution to combine several documents into one.
\end{itemize}
%
See also the CTAN topic \href{http://ctan.org/topic/subdocs}{\textsf{subdocs}}
for further related packages.
The present package differs from the above solutions in that
a document structure constructed with the conventional |\include| mechanism
just needs two extra commands at the top of every file
such that all constituent files can be compiled individually.

%%%%%%%%%%%%%%%%%%%%%%%%%%%%%%%%%%%%%%%%%%%%%%%%%%%%%%%%%%%%%%%%%%%%%%%%%%%%%%%%
%\subsection{Feature Suggestions}
%
%The following is a list of features which may be useful for future
%versions of this package:
%%
%\begin{itemize}
%\item
%\ldots
%\end{itemize}

%%%%%%%%%%%%%%%%%%%%%%%%%%%%%%%%%%%%%%%%%%%%%%%%%%%%%%%%%%%%%%%%%%%%%%%%%%%%%%%%
\subsection{Revision History}

%%%%%%%%%%%%%%%%%%%%%%%%%%%%%%%%%%%%%%%%
\paragraph{v2.0:} 2018/12/30

\begin{itemize}
\item
immediate forward processing
\item
added |\childdocby| mechanism
\item
manual restructured
\end{itemize}

%%%%%%%%%%%%%%%%%%%%%%%%%%%%%%%%%%%%%%%%
\paragraph{v1.6:} 2018/01/17

\begin{itemize}
\item
application for development of include files
\item
corrections to manual
\end{itemize}

%%%%%%%%%%%%%%%%%%%%%%%%%%%%%%%%%%%%%%%%
\paragraph{v1.5:} 2017/05/21

\begin{itemize}
\item
more complete structuring introduced
\item
|\childdocof| introduced
\item
|\childdoc| renamed to |\childdocmain|
\item
|\childredirect| renamed to |\childdocforward| and |\childdocforwardprefix|
and functionality expanded
\end{itemize}

%%%%%%%%%%%%%%%%%%%%%%%%%%%%%%%%%%%%%%%%
\paragraph{v1.0:} 2017/04/27

\begin{itemize}
\item
manual and install package
\item
first version published on CTAN
\end{itemize}

%%%%%%%%%%%%%%%%%%%%%%%%%%%%%%%%%%%%%%%%
\paragraph{v0.6:} 2017/04/26

\begin{itemize}
\item
redirection mechanism added
\end{itemize}

%%%%%%%%%%%%%%%%%%%%%%%%%%%%%%%%%%%%%%%%
\paragraph{v0.5:} 2017/04/26

\begin{itemize}
\item
functionality in definition file
\end{itemize}


%%%%%%%%%%%%%%%%%%%%%%%%%%%%%%%%%%%%%%%%%%%%%%%%%%%%%%%%%%%%%%%%%%%%%%%%%%%%%%%%
%%%%%%%%%%%%%%%%%%%%%%%%%%%%%%%%%%%%%%%%%%%%%%%%%%%%%%%%%%%%%%%%%%%%%%%%%%%%%%%%
%%%%%%%%%%%%%%%%%%%%%%%%%%%%%%%%%%%%%%%%%%%%%%%%%%%%%%%%%%%%%%%%%%%%%%%%%%%%%%%%
\appendix

\settowidth\MacroIndent{\rmfamily\scriptsize 000\ }

 \DocInput{childdoc.dtx}

\end{document}
%</driver>
% \fi
%
% %%%%%%%%%%%%%%%%%%%%%%%%%%%%%%%%%%%%%%%%%%%%%%%%%%%%%%%%%%%%%%%%%%%%%%%%%%%%%%
% %%%%%%%%%%%%%%%%%%%%%%%%%%%%%%%%%%%%%%%%%%%%%%%%%%%%%%%%%%%%%%%%%%%%%%%%%%%%%%
% \section{Sample}
%\iffalse
%<*samplemain>
%\fi
%
% The following presents a sample document
% with two chapters, two parts, a title page,
% a compile flag as well as three forwarding files to set the flag.
% It consists of eight |.tex| files:
% \begin{center}
% \begin{tabular}{ll}
% |cdocsamp.tex|&main file\\
% |cdocsch1.tex|&include file for chapter 1\\
% |cdocsch2.tex|&include file for chapter 2\\
% |cdocspt3.tex|&include file for part 3\\
% |cdocspt4.tex|&include file for part 4\\
% |cdocsdrf.tex|&forwarding file for main file in draft mode\\
% |cdocsfi1.tex|&forwarding file for final version of chapter 1\\
% |cdocsfi2.tex|&forwarding file for final version of chapter 2\\
% \end{tabular}
% \end{center}
% Each of the eight files can be compiled directly by the \LaTeX{} compiler.
%
% %%%%%%%%%%%%%%%%%%%%%%%%%%%%%%%%%%%%%%
% \paragraph{Main File.}
%
% The main file is called |cdocsamp.tex|.
%
% Load the \textsf{childdoc} definitions and
% declare the filename for the main document:
%    \begin{macrocode}
\input{childdoc.def}
\childdocmain{}
%    \end{macrocode}

% Optional override for |\version| flag:
%    \begin{macrocode}
%%\ifchilddoc\else\providecommand{\version}{draft}\fi
%    \end{macrocode}

% Define the default values for the |\version| flag
% (|final| for the main file and |draft| for childs):
%    \begin{macrocode}
\ifchilddoc
\providecommand{\version}{draft}
\else
\providecommand{\version}{final}
\fi
%    \end{macrocode}

% Load the standard document class:
%    \begin{macrocode}
\documentclass[12pt]{article}
%    \end{macrocode}

% Start the document body:
%    \begin{macrocode}
\begin{document}
%    \end{macrocode}

% Declare a title page.
% Print title, part of document being processed and version flag:
%    \begin{macrocode}
\addtocounter{page}{-1}
\begin{center}
{\LARGE\bfseries{}childdoc example\par}
\vspace{1cm}
\ifchilddoc
\ifchilddocmanual part\else chapter\fi:
`\childdocname' of `\childdocjob'\par
\else
main document: `\childdocjob'\par
\fi
version: \version\par
\end{center}
\newpage
%    \end{macrocode}

% Manually include selected file,
% otherwise process as usual:
%    \begin{macrocode}
\ifchilddocmanual
\section*{part `\childdocname'}
\input{\childdocname}
\else
%    \end{macrocode}

% Include the two chapters:
%    \begin{macrocode}
\include{cdocsch1}
\include{cdocsch2}
%    \end{macrocode}

% Include the two parts unless only chapters should be displayed:
%    \begin{macrocode}
\ifchilddoc\else
\section{part three}
\input{cdocspt3}
\section{part four}
\input{cdocspt4}
\fi
%    \end{macrocode}

% Process as usual until here:
%    \begin{macrocode}
\fi
%    \end{macrocode}

% End of document body:
%    \begin{macrocode}
\end{document}
%    \end{macrocode}
%\iffalse
%</samplemain>
%\fi
%
% %%%%%%%%%%%%%%%%%%%%%%%%%%%%%%%%%%%%%%
% \paragraph{Chapter Include Files.}
%
% The include files are called |cdocsch1.tex| and |cdocsch2.tex|.
%
%\iffalse
%<*samplechap1|samplechap2>
%\fi

% Optional override for |\version| flag:
%    \begin{macrocode}
%%\providecommand{\version}{final}
%    \end{macrocode}

% Include the main document:
%    \begin{macrocode}
\input{childdoc.def}
\childdocof{cdocsamp}
%    \end{macrocode}

%\iffalse
%</samplechap1|samplechap2>
%\fi
%
%\iffalse
%<*samplechap1>
%\fi
% Some text for chapter 1:
%    \begin{macrocode}
\section{one}
some text in chapter one
%    \end{macrocode}

%\iffalse
%</samplechap1>
%\fi
% Some text for chapter 2:
%\iffalse
%<*samplechap2>
%\fi
%    \begin{macrocode}
\section{two}
more text in chapter two
%    \end{macrocode}

%\iffalse
%</samplechap2>
%\fi
%
% %%%%%%%%%%%%%%%%%%%%%%%%%%%%%%%%%%%%%%
% \paragraph{Part Include Files.}
%
% The include files are called |cdocspt3.tex| and |cdocspt4.tex|.
%
%\iffalse
%<*samplepart3|samplepart4>
%\fi

% Optional override for |\version| flag:
%    \begin{macrocode}
%%\providecommand{\version}{final}
%    \end{macrocode}

% Include the main document:
%    \begin{macrocode}
\input{childdoc.def}
\childdocby{cdocsamp}
%    \end{macrocode}

%\iffalse
%</samplepart3|samplepart4>
%\fi
%
%\iffalse
%<*samplepart3>
%\fi
% Some text for part 3:
%    \begin{macrocode}
some text in part three
%    \end{macrocode}

%\iffalse
%</samplepart3>
%\fi
% Some text for part 4:
%\iffalse
%<*samplepart4>
%\fi
%    \begin{macrocode}
more text in part four
%    \end{macrocode}

%\iffalse
%</samplepart4>
%\fi
%
% %%%%%%%%%%%%%%%%%%%%%%%%%%%%%%%%%%%%%%
% \paragraph{Forwarding for a Complete Draft.}
%
% The following forwarding file |cdocsdrf.tex|
% compiles the main document in draft mode:
%\iffalse
%<*sampledraft>
%\fi
%    \begin{macrocode}
\def\version{draft}
\input{childdoc.def}
\childdocforward{cdocsamp}
%    \end{macrocode}

%\iffalse
%</sampledraft>
%\fi
%
% %%%%%%%%%%%%%%%%%%%%%%%%%%%%%%%%%%%%%%
% \paragraph{Forwarding for Final Version of the Chapters.}
%
% The following forwarding files |cdocsfn1.tex| and |cdocsfn2.tex|
% (with identical content)
% compile the final versions of the child documents
% |cdocsch1.tex| and |cdocsch2.tex|, respectively:
%\iffalse
%<*samplefinal>
%\fi
%    \begin{macrocode}
\def\version{final}
\input{childdoc.def}
\childdocforwardprefix[cdocsamp]{cdocsfn}{cdocsch}
%    \end{macrocode}

%\iffalse
%</samplefinal>
%\fi
%
% %%%%%%%%%%%%%%%%%%%%%%%%%%%%%%%%%%%%%%
% \paragraph{Command Line Processing.}
%
% The following three command lines generate the output files
% |cdocscld|, |cdocscl1| and |cdocscl2|
% which should be identical to
% |cdocsdrf|, |cdocsch1| and |cdocsfn2|, respectively:
% \begin{center}
% \begin{tabular}{l}
% |latex -jobname cdocscld \|\\
% |  "\def\version{draft}\input{childdoc.def}\childdocforward{cdocsamp}"|\\
% |latex -jobname cdocscl1 \|\\
% |  "\input{childdoc.def}\childdocforward[cdocsamp]{cdocsch1}"|\\
% |latex -jobname cdocscl2 \|\\
% |  "\def\version{final}\input{childdoc.def}\childdocforward{cdocsch2}"|
% \end{tabular}
% \end{center}
% Note that the trailing backslash on each first line
% merely continues the input to the second line
% (for convenient cut ant paste).
% Furthermore, the command |latex| can be replaced by any
% of its alternative versions such as |pdflatex|.
%
% %%%%%%%%%%%%%%%%%%%%%%%%%%%%%%%%%%%%%%%%%%%%%%%%%%%%%%%%%%%%%%%%%%%%%%%%%%%%%%
% %%%%%%%%%%%%%%%%%%%%%%%%%%%%%%%%%%%%%%%%%%%%%%%%%%%%%%%%%%%%%%%%%%%%%%%%%%%%%%
% \section{Implementation}
%\iffalse
%<*package>
%\fi
%
% This section describes the definitions file |childdoc.def|.

% The definitions cannot be loaded using |\usepackage| or |\RequirePackage|
% which has a mechanism to prevent loading a style file more than once.
% When loading the definitions by means of |\input|
% multiple instances have to be prevented manually:
%\iffalse
%This code needs to be before the `\ProvidesFile' directive
%which is defined at the beginning of this file.
%Therefore it is also placed there and commented out here.
%</package>
%<*discard>
%\fi
%    \begin{macrocode}
\ifdefined\childdocmain\endinput\fi
%    \end{macrocode}
%\iffalse
%</discard>
%<*package>
%\fi
%
% \macro{\ifchilddoc}
% \macro{\ifchilddocmanual}
% The conditional |\ifchilddoc| tells whether a
% child (true) or main (false) document is being compiled.
% The conditional |\ifchilddocmanual| tells whether
% the |\includeonly| mechanism is used (false) or
% the selection of child files must be performed manually (true).
% The definitions initialise to false:
%    \begin{macrocode}
\newif\ifchilddoc
\newif\ifchilddocmanual
%    \end{macrocode}

% \macro{\childdocname}
% \macro{\childdocjob}
% The macro |\childdocname| stores the name of the main document
% to be compiled. The macro |\childdocjob| stores the name of
% the document on which the \LaTeX{} compiler was originally invoked.
% The content of |\jobname| cannot be compared
% to filenames specified in the source due to different catcodes.
% The following code rescans |\jobname|, stores the result
% in |\childdocname| and saves a copy in |\childdocjob|:
%    \begin{macrocode}
\edef\childdocname{\scantokens\expandafter{\jobname\noexpand}}
\let\childdocjob\childdocname
%    \end{macrocode}

% \macro{\childdocdisable}
% The macro |\childdocdisable| prevents the main file
% from being processed more than once.
% At this stage, the main document command |\childdocmain|
% is assumed to be called once again where it should do nothing.
% Any subsequent call to it should prevent
% a secondary processing of the main document
% It overwrites the forwarding commands
% |\childdocof| and |\childdocforward|
% with empty macros to prevent further inclusions of the main document:
%    \begin{macrocode}
\newcommand{\childdocdisable}
{
  \renewcommand{\childdocmain}[1]{\renewcommand{\childdocmain}[1]{\endinput}}
  \renewcommand{\childdocof}[1]{}
  \renewcommand{\childdocby}[2][]{}
  \renewcommand{\childdocforward}[2][]{}
  \renewcommand{\childdocdisable}{}
}
%    \end{macrocode}

% \macro{\childdocmain}
% The macro |\childdocmain| is to be called at the top of the main file
% with nothing or the main filename (without extension) as argument.
% First, it breaks loops.
% If the argument is not empty and does not match |\childdocname|
% (which is set by the first inclusion of |childdoc.def|),
% |\ifchilddoc| is set to true, |\includeonly| is applied to the child file
% and |\jobname| is set to the main file
% (for proper handling of |.aux| files):
%    \begin{macrocode}
\newcommand{\childdocmain}[1]
{
  \childdocdisable\childdocmain{}
  \if?#1?\else
    \begingroup
      \def\childdoctmp{#1}
      \ifx\childdoctmp\childdocname
        \def\childdoctmp{}
      \else
        \def\childdoctmp
        {
          \childdoctrue
          \includeonly{\childdocname}
          \def\childdocjob{#1}
          \def\jobname{#1}
        }
      \fi
      \expandafter
    \endgroup
    \childdoctmp
  \fi
}
%    \end{macrocode}

% \macro{\childdocof}
% The command |\childdocof| redirects
% compilation to the main file |#1|.
%    \begin{macrocode}
\newcommand{\childdocof}[1]
{
  \childdocdisable
  \childdoctrue
  \includeonly{\childdocname}
  \def\jobname{#1}
  \def\childdocjob{#1}
  \input{#1}
}
%    \end{macrocode}

% \macro{\childdocby}
% The command |\childdocby| ....
%    \begin{macrocode}
\newcommand{\childdocby}[2][]
{
  \childdocdisable
  \childdoctrue
  \childdocmanualtrue
  \if?#1?\else
    \def\jobname{#2}
  \fi
  \def\childdocjob{#2}
  \input{#2}
  \endinput
}
%    \end{macrocode}

% \macro{\childdocforward}
% The command |\childdocforward| redirects
% compilation to the main file or
% (if the optional argument is given) a child file.
% Parameters are set as if the main file
% or a child file starting with |\childdocof| was compiled.
% Then compilation is handed over to the main file:
%    \begin{macrocode}
\newcommand{\childdocforward}[2][]
{
  \begingroup
    \if?#1?
      \def\childdoctmp
      {
        \def\childdocname{#2}
        \def\childdocjob{#2}
        \def\jobname{#2}
        \input{#2}
        \endinput
      }
    \else
      \def\childdoctmp
      {
        \childdocdisable
        \def\childdocname{#2}
        \childdoctrue
        \includeonly{#2}
        \def\childdocjob{#1}
        \def\jobname{#1}
        \input{#1}
        \endinput
      }
    \fi
    \expandafter
  \endgroup
  \childdoctmp
}
%    \end{macrocode}

% \macro{\childdocforwardprefix}
% The command |\childdocforwardprefix| redirects
% compilation to the main or a child file by means of a pattern.
% The prefix |#1| in the current filename is replaced by |#2|
% and the suffix of the current filename is kept
% (it is assumed that the filename does not contain the substring `|~~~|'
% which is used as a delimiter).
% Compilation is handed over to the new file by |\childdocforward|:
%    \begin{macrocode}
\newcommand{\childdocforwardprefix}[3][]
{
  \begingroup
    \def\childdocextract #2##1~~~{\def\childdoctmp{\childdocforward[#1]{#3##1}}}
    \expandafter\childdocextract\childdocname~~~
    \expandafter
  \endgroup
  \childdoctmp
}
%    \end{macrocode}

% \macro{\childdoc}
% The deprecated macro |\childdoc| is a legacy version of |\childdocmain|:
%    \begin{macrocode}
\newcommand{\childdoc}{\childdocmain}
%    \end{macrocode}

% \macro{\childdocredirect}
% The deprecated macro |\childdocredirect| is a legacy version
% of |\childdocforward| and |\childdocforwardprefix|:
%    \begin{macrocode}
\newcommand{\childdocredirect}[2][]
{
  \begingroup
    \if?#1?
      \def\childdoctmp{\childdocforward{#2}}
    \else
      \def\childdoctmp{\childdocforwardprefix{#1}{#2}}
    \fi
    \expandafter
  \endgroup
  \childdoctmp
}
%    \end{macrocode}

%\iffalse
%</package>
%\fi
%
\endinput

\childdocforward{cdocsamp}
%    \end{macrocode}

%\iffalse
%</sampledraft>
%\fi
%
% %%%%%%%%%%%%%%%%%%%%%%%%%%%%%%%%%%%%%%
% \paragraph{Forwarding for Final Version of the Chapters.}
%
% The following forwarding files |cdocsfn1.tex| and |cdocsfn2.tex|
% (with identical content)
% compile the final versions of the child documents
% |cdocsch1.tex| and |cdocsch2.tex|, respectively:
%\iffalse
%<*samplefinal>
%\fi
%    \begin{macrocode}
\def\version{final}
% \iffalse
%
% childdoc.dtx Copyright (C) 2017-2018 Niklas Beisert
%
% This work may be distributed and/or modified under the
% conditions of the LaTeX Project Public License, either version 1.3
% of this license or (at your option) any later version.
% The latest version of this license is in
%   http://www.latex-project.org/lppl.txt
% and version 1.3 or later is part of all distributions of LaTeX
% version 2005/12/01 or later.
%
% This work has the LPPL maintenance status `maintained'.
%
% The Current Maintainer of this work is Niklas Beisert.
%
% This work consists of the files childdoc.dtx and childdoc.ins
% and the derived files childdoc.def and cdocsamp.tex with
% cdocsch1.tex, cdocsch2.tex, cdocsdrf.tex, cdocsfn1.tex, cdocsfn2.tex.
%
%<package>\ifdefined\childdocmain\endinput\fi
%<package>\ProvidesFile{childdoc.def}[2018/12/30 v2.0 child document driver]
%<samplemain>\ProvidesFile{cdocsamp.tex}[2018/12/30 v2.0 sample for childdoc]
%<*driver>
%\ProvidesFile{childdoc.drv}[2018/12/30 v2.0 childdoc reference manual file]
\PassOptionsToClass{10pt,a4paper}{article}
\documentclass{ltxdoc}

\usepackage[margin=35mm]{geometry}
\usepackage{hyperref}
\usepackage{hyperxmp}
\usepackage[usenames]{color}

\hypersetup{colorlinks=true}
\hypersetup{pdfstartview=FitH}
\hypersetup{pdfpagemode=UseNone}
\hypersetup{pdfsource={}}
\hypersetup{pdflang={en-UK}}
\hypersetup{pdfcopyright={Copyright 2017-2018 Niklas Beisert.
  This work may be distributed and/or modified under the
  conditions of the LaTeX Project Public License, either version 1.3
  of this license or (at your option) any later version.}}
\hypersetup{pdflicenseurl={http://www.latex-project.org/lppl.txt}}
\hypersetup{pdfcontactaddress={ETH Zurich, ITP, HIT K,
  Wolfgang-Pauli-Strasse 27}}
\hypersetup{pdfcontactpostcode={8093}}
\hypersetup{pdfcontactcity={Zurich}}
\hypersetup{pdfcontactcountry={Switzerland}}
\hypersetup{pdfcontactemail={nbeisert@itp.phys.ethz.ch}}
\hypersetup{pdfcontacturl={http://people.phys.ethz.ch/\xmptilde nbeisert/}}

\newcommand{\secref}[1]{\hyperref[#1]{section \ref*{#1}}}

\parskip1ex
\parindent0pt
\let\olditemize\itemize
\def\itemize{\olditemize\parskip0pt}

\begin{document}

\title{The \textsf{childdoc} Package}
\hypersetup{pdftitle={The childdoc Package}}
\author{Niklas Beisert\\[2ex]
  Institut f\"ur Theoretische Physik\\
  Eidgen\"ossische Technische Hochschule Z\"urich\\
  Wolfgang-Pauli-Strasse 27, 8093 Z\"urich, Switzerland\\[1ex]
  \href{mailto:nbeisert@itp.phys.ethz.ch}
  {\texttt{nbeisert@itp.phys.ethz.ch}}}
\hypersetup{pdfauthor={Niklas Beisert}}
\hypersetup{pdfsubject={Manual for the LaTeX2e Package childdoc}}
\date{30 December 2018, \textsf{v2.0}}
\maketitle

\begin{abstract}\noindent
\textsf{childdoc} is a \LaTeXe{} package
that enables the direct compilation
of document sections included by |\include|
to individual files.
\end{abstract}

\begingroup
\parskip0ex
\tableofcontents
\endgroup

%%%%%%%%%%%%%%%%%%%%%%%%%%%%%%%%%%%%%%%%%%%%%%%%%%%%%%%%%%%%%%%%%%%%%%%%%%%%%%%%
%%%%%%%%%%%%%%%%%%%%%%%%%%%%%%%%%%%%%%%%%%%%%%%%%%%%%%%%%%%%%%%%%%%%%%%%%%%%%%%%
\section{Introduction}

\LaTeX{} provides a mechanism to structure a large document (such as a book)
into a main file and several child files (containing the chapters)
using the |\include| command.
This mechanism is beneficial for documents
which span hundreds of pages in order to
make the source file(s) more manageable.
Moreover, compilation can be restricted to
selected child files by means of the |\includeonly| command.
The latter feature can be used to reduce the compilation time while editing
(this was significantly more useful in the earlier days of \LaTeX{})
or to generate a smaller document which is easier to navigate.
Another application of |\includeonly| is to generate
documents consisting of selected parts of the complete document.

However, there are a few drawbacks of the plain |\include| mechanism:
\begin{itemize}
\item
The child files cannot be compiled on their own,
they can only be compiled via the main file.
A naive editing environment
(such as a text editor with an option
to have the current file processed by \LaTeX)
may require one to switch to the main file before compiling;
attempting to compile the child file produces errors.
\item
The main file must be modified (each time)
to adjust the |\includeonly| command
to the present needs. This easily leaves the main file in a messy state.
\item
The generated document will always carry the filename
of the main document. This is inconvenient if
several child files are to be compiled and
to be kept for distribution.
\end{itemize}

The present package provides a simple interface
to make child files individually compilable by \LaTeX{}.
Compiling a child file then has the same effect as compiling
the main file with an |\includeonly| command
to select the appropriate child.
Moreover the generated document will carry the name of the child
rather than the main file.
This resolves all three above issues.

This feature is meant to make the editing of books,
thesis documents and lecture notes somewhat more convenient.
However, the package can also be used efficiently for
composing a series of documents (such as exercise sheets)
which are typically distributed individually.
It then assists the author in generating the individual documents
(potentially in different versions)
as well as a document containing the collected series.
Another application is in developing style files
or other kinds of included material
where compilation of the style file could redirect
to a sample or test file.

%%%%%%%%%%%%%%%%%%%%%%%%%%%%%%%%%%%%%%%%%%%%%%%%%%%%%%%%%%%%%%%%%%%%%%%%%%%%%%%%
%%%%%%%%%%%%%%%%%%%%%%%%%%%%%%%%%%%%%%%%%%%%%%%%%%%%%%%%%%%%%%%%%%%%%%%%%%%%%%%%
\section{Usage}

First of all, the package \textsf{childdoc} is \emph{not} a standard
\LaTeXe{} |.sty| style file! Therefore it needs to be invoked in
a non-standard way.

%%%%%%%%%%%%%%%%%%%%%%%%%%%%%%%%%%%%%%%%%%%%%%%%%%%%%%%%%%%%%%%%%%%%%%%%%%%%%%%%
\subsection{Included Files}
\label{sec:include}

%%%%%%%%%%%%%%%%%%%%%%%%%%%%%%%%%%%%%%%%
\DescribeMacro{\childdocmain}
To use the package, add the commands
\begin{center}
\begin{tabular}{l}
|\input{childdoc.def}|\\
|\childdocmain{}|\\
\end{tabular}
\end{center}
at the very top of the main \LaTeX{} file,
in particular \emph{before} the |\documentclass| statement!
The argument of |\childdocmain| should be left empty
(but it must be present).

%%%%%%%%%%%%%%%%%%%%%%%%%%%%%%%%%%%%%%%%
\DescribeMacro{\childdocof}
Furthermore, add the commands
\begin{center}
\begin{tabular}{l}
|\input{childdoc.def}|\\
|\childdocof{|\textit{main}|}|\\
\end{tabular}
\end{center}
at the top of every child file \textit{child}
which is included by |\include{|\textit{child}|}|
from within the main file
(or at least for those files to be compiled individually).
The argument \textit{main} must be the filename of the main file.

There are a couple of
considerations in setting up the main and child documents:

%%%%%%%%%%%%%%%%%%%%%%%%%%%%%%%%%%%%%%%%
\paragraph{Restrictions.}

Please note the following restrictions:
\begin{itemize}
\item
|\childdocmain| must be called with one argument \textit{main}
to ensure compatibility with earlier version of the package.
It must either be empty (|\childdocmain{}|)
or precisely match the filename of the main file in which it is specified.
See \secref{sec:detection} for further information.
\item
The filename \textit{main} must be specified without the |.tex| extension.
\item
The filename \textit{main} is case sensitive
(even in case-insensitive file systems)
due to internal string comparison.
\item
The argument \textit{main} should be fully expanded, it cannot be a macro.
\item
Subdirectories and special characters should be avoided in filenames.
\item
The command |\childdocmain{|\textit{main}|}| must be followed by a whitespace.
It should not be followed immediately by another command
or by a comment mark `|%|'.
This is because the \TeX{} parser reads the token immediately following
the argument of |\childdocmain| and puts it
at the beginning of every child section;
however, a white\-space is ignored.
\end{itemize}

%%%%%%%%%%%%%%%%%%%%%%%%%%%%%%%%%%%%%%%%
\paragraph{Content of Main File.}

It is advisable to place all content in the child files included by |\include|.
Any output contained in the main file will appear in all child documents
unless suppressed manually;
it cannot be suppressed automatically by the |\includeonly| directive
and thus should normally be avoided.
A method to include some content in the main file
by means of conditional processing is described in \secref{sec:conditional}.

%%%%%%%%%%%%%%%%%%%%%%%%%%%%%%%%%%%%%%%%
\paragraph{Page Numbering.}

When only a part of the document is compiled,
the appropriate numbering of pages
(as well as other status parameters)
is determined from the |.aux| files.
The latter contain information from previous passes.
However this information needs to propagate through
all intermediate child documents.
Therefore the page numbering in child documents may well
be inconsistent until the complete document is compiled at least once.

A useful (if unconventional) way to always ensure a consistent
page numbering is to restart the numbering in each child document
and denote the pages by `\textit{child}|.|\textit{page}'
where \textit{child} represents the chapter/section number of the child file.
This can be achieved by the command
|\numberwithin{page}{|\textit{child}|}|
of the \textsf{amsmath} package
where \textit{child} can be |chapter| or |section|
depending on the chosen structuring.
Alternatively, one can modify the macro |\thepage| appropriately
and reset the counter |page| at the start of each child file.

%%%%%%%%%%%%%%%%%%%%%%%%%%%%%%%%%%%%%%%%%%%%%%%%%%%%%%%%%%%%%%%%%%%%%%%%%%%%%%%%
\subsection{Conditional Processing}
\label{sec:conditional}

The package provides a mechanism to compile different versions
of a document. To customise the versions further some conditional processing
can come in handy to distinguish which version is being compiled.
The package provides two macros to describe the compilation context:

%%%%%%%%%%%%%%%%%%%%%%%%%%%%%%%%%%%%%%%%
\DescribeMacro{\ifchilddoc}
The conditional |\ifchilddoc| distinguishes between the compilation of
child documents and the main document:
%
\begin{center}
|\ifchilddoc |\textit{child-code}| |[|\||else |\textit{main-code}]| \||fi|
\end{center}

%%%%%%%%%%%%%%%%%%%%%%%%%%%%%%%%%%%%%%%%
\DescribeMacro{\childdocname}
\DescribeMacro{\childdocjob}
The macro |\childdocname| contains the filename (without extension)
of the main or child file being processed.
Note that |\childdocjob| will always contain the name of the main file.

%%%%%%%%%%%%%%%%%%%%%%%%%%%%%%%%%%%%%%%%
\paragraph{Title Page.}

Conditional processing can be used to include a title or banner page
in the main document when proper precautions are taken.
Importantly, the code in the main file should ensure that the page counter
(as well as other status parameters which are stored in the |.aux| files)
takes the same value after the conditional processing.
Otherwise the page numbers may take divergent values
depending on which part is compiled.

For example, a title page could be declared by:
%
\begin{center}
\begin{tabular}{l}
|\ifchilddoc\||else|\\
|\addtocounter{page}{-1}|\\
\textit{code for title page}\\
|\newpage|\\
|\||fi|
\end{tabular}
\end{center}
%
A banner page for the child documents can be generated by:
%
\begin{center}
\begin{tabular}{l}
|\ifchilddoc|\\
|\addtocounter{page}{-1}|\\
\textit{code for banner page}\\
|\newpage|\\
|\||fi|
\end{tabular}
\end{center}
%
Here one could write a message such as:
\begin{center}
|This is the part \childdocname{} of \childdocjob{}.|
\end{center}

%%%%%%%%%%%%%%%%%%%%%%%%%%%%%%%%%%%%%%%%%%%%%%%%%%%%%%%%%%%%%%%%%%%%%%%%%%%%%%%%
\subsection{Flags}
\label{sec:flags}

The package makes it easy to generate different versions
of the main or child documents.
To this end compilation flags can be defined
and assigned different default values.
They will be particularly useful in conjunction
with the forwarding mechanism described in \secref{sec:forward}.

For example, it may be useful to have a flag |\version|
which can be set to |draft| or |final|.
The document source will contain some conditional code
depending on the value of |\version|.
Suppose further, the flag should default to |final| for the main file
and to |draft| for child files
which is a natural assignment for editing the document.
This is achieved by placing the following code
in the preamble of the main document
(below the |\childdocmain| directive):
%
\begin{center}
\begin{tabular}{l}
|\ifchilddoc|\\
|\providecommand{\version}{draft}|\\
|\||else|\\
|\providecommand{\version}{final}|\\
|\||fi|
\end{tabular}
\end{center}
%
The definition by |\providecommand| makes sure
that previous definitions are not overwritten.
Further statements |\providecommand{\version}{...}|
can thus be added before the above code to override it.

For the main file, one might add a line
(between |\childdocmain| and the above block)
%
\begin{center}
|%\ifchilddoc\||else\providecommand{\version}{draft}\||fi|
\end{center}
%
which can be uncommented to produce a draft version.
Likewise one can add a line to the very top of a child file
(above the |\childdocof{|\textit{main}|}| directive)
%
\begin{center}
|%\providecommand{\version}{final}|
\end{center}
%
which can be uncommented to produce the final version of this child document.

%%%%%%%%%%%%%%%%%%%%%%%%%%%%%%%%%%%%%%%%%%%%%%%%%%%%%%%%%%%%%%%%%%%%%%%%%%%%%%%%
\subsection{Forwarding}
\label{sec:forward}

Different versions of the main or child documents
using compilation flags as described in \secref{sec:flags}
can be (permanently) stored in different files
for convenient compilation, viewing and distribution.
To this end, the package defines a command
to pass on compilation to a different file:

%%%%%%%%%%%%%%%%%%%%%%%%%%%%%%%%%%%%%%%%
\DescribeMacro{\childdocforward}
The command |\childdocforward| redirects processing to
another source file:
%
\begin{center}
\begin{tabular}{l}
|\input{childdoc.def}|\\
|\childdocforward[|\textit{main}|]{|\textit{dest}|}|\\
\end{tabular}
\end{center}
%
The argument \textit{dest} is the destination file
(without extension).
It should be the main file or one of the child files.
Note that further \textsf{childdoc} directives
such as |\childdocof| and |\childdocforward|
in the indicated file will be processed in this form.
The optional argument \textit{main}
passes on directly to the main file \textit{main}
while pretending to compile the child \textit{dest}.
This form behaves as if \textit{dest}
issues |\childdocof{|\textit{main}|}| right away,
and no further \textsf{childdoc} directives will be processed.

%%%%%%%%%%%%%%%%%%%%%%%%%%%%%%%%%%%%%%%%
\DescribeMacro{\...prefix}
In the alternative form |\childdocforwardprefix|,
%
\begin{center}
\begin{tabular}{l}
|\input{childdoc.def}|\\
|\childdocforwardprefix[|\textit{main}|]{|\textit{prefix}|}{|\textit{dest}|}|
\end{tabular}
\end{center}
%
the destination file is determined by a pattern
depending on the current file:
To make this work, the current file must be called
`{\textit{prefix}\hspace{0.2em}\textit{suffix}}'
with \textit{prefix} matching precisely the argument.
Processing is then passed on to the file
`{\textit{dest}\hspace{0.2em}\textit{suffix}}'.
Surely, the same effect is achieved by
directly specifying the
argument `{\textit{dest}\hspace{0.2em}\textit{suffix}}'
in the first form.
However, that requires to set up a different file
for each child. With the alternative form of the command
all these files can have exactly the same content
which simplifies setting them up and maintaining them.

For example, the following file |draft.tex|
with a compilation flag |\version| as described in \secref{sec:flags}
compiles the main document as a draft:
%
\begin{center}
\begin{tabular}{l}
|\def\version{draft}|\\
|\input{childdoc.def}|\\
|\childdocforward{|\textit{main}|}|
\end{tabular}
\end{center}
%
Likewise, the following files |final|\textit{nn}|.tex|
compile the final version of the child document
|child|\textit{nn}|.tex|:
%
\begin{center}
\begin{tabular}{l}
|\def\version{final}|\\
|\input{childdoc.def}|\\
|\childdocforwardprefix{final}{child}|
\end{tabular}
\end{center}
%

Note that when several versions of a main file and/or of each child file
are to be generated, it may be convenient to set up a |Makefile| or
shell script to automatise the process.

%%%%%%%%%%%%%%%%%%%%%%%%%%%%%%%%%%%%%%%%%%%%%%%%%%%%%%%%%%%%%%%%%%%%%%%%%%%%%%%%
\subsection{Command Line Processing}
\label{sec:commandline}

The effect of redirection files can also be achieved by invoking
the \LaTeX{} compiler with a more elaborate command line.
Most conveniently this should be done as part
of a shell script or a |Makefile|.

When using \textsf{childdoc} in the main file, the following
command lines effectively perform a redirection
(note that depending on the shell being used,
backslashes may have to be doubled: `|\|' $\to$ `|\\|'):
%
\begin{center}
|... -jobname "|\textit{target}|" |\\|"|[\textit{flags}]%
|\input{childdoc.def}\childdocforward[|\textit{main}|]{|\textit{dest}|}"|
\end{center}
%
Here \textit{target} is the name of the output file,
\textit{main} is the name of the main file
and \textit{dest} is the name of the main or child file to be processed
(all filenames without extensions).
The optional argument \textit{main} can be omitted
if \textit{main} matches \textit{dest}.
Optionally, compilation \textit{flags} can be defined via |\def| commands.
This command line makes the \TeX{} engine believe
it is compiling the file \textit{target}
whose content is specified as the latter parameter.
The provided code then forwards the processing to
\textit{main} or \textit{dest} as described in \secref{sec:forward}.

%%%%%%%%%%%%%%%%%%%%%%%%%%%%%%%%%%%%%%%%%%%%%%%%%%%%%%%%%%%%%%%%%%%%%%%%%%%%%%%%
\subsection{Include by Input}
\label{sec:input}

Including child documents by |\include| has some restrictions by design.
Most notably, the content of a child document always occupies
its own set of pages; pages cannot be shared between child documents.
Usually, this behaviour makes perfect sense
because each child document contain an essential part of the document.
However, in some situations it may be desirable to compose
a document from a collection of parts
without having mandatory page breaks between then.
For this case, the package
provides a mechanism to include parts
by |\input| which can also be processed individually.
However, by construction this mechanism
requires manual handling of the content to be output.

%%%%%%%%%%%%%%%%%%%%%%%%%%%%%%%%%%%%%%%%
\DescribeMacro{\ifchilddocmanual}
The main file should be prepared as usual, see \secref{sec:include}.
However, the document body must make a distinction
between processing of an individual part and of the main document, e.g.:
%
\begin{center}
\begin{tabular}{l}
|\ifchilddocmanual|\\
|\input{\childdocname}|\\
|\||else|\\
\textit{document body with }|\input{|\textit{part}|}|\\
|\||fi|
\end{tabular}
\end{center}
%
The conditional |\ifchilddocmanual| is true whenever
a part to be included by |\input| is being compiled,
and the name of the part is stored in |\childdocname|.

%%%%%%%%%%%%%%%%%%%%%%%%%%%%%%%%%%%%%%%%
\DescribeMacro{\childdocby}
Each part to be included by |\input| should start with:
%
\begin{center}
\begin{tabular}{l}
|\input{childdoc.def}|\\
|\childdocby{|\textit{main}|}|\\
\end{tabular}
\end{center}
%
The directive |\childdocby| is similar to |\childdocof|
described in \secref{sec:include},
but the subsequent selection of content must be done manually.
To that end, both |\ifchilddoc| and |\ifchilddocmanual|
will be true upon processing of a part,
and the name of the part is stored in |\childdocname|.
Note that |\jobname| will be set to the filename of the current part
so that each part receives an individual |.aux| file
that does not interfere with the |.aux| file(s) of the main document.
This behaviour can be altered by the alternative form
|\childdocby[*]{|\textit{main}|}| (with a non-empty optional argument)
which uses the |.aux| file of the main document
by setting |\jobname| to \textit{main}.

%%%%%%%%%%%%%%%%%%%%%%%%%%%%%%%%%%%%%%%%%%%%%%%%%%%%%%%%%%%%%%%%%%%%%%%%%%%%%%%%
\subsection{Driver Development}
\label{sec:driver}

The \textsf{childdoc} mechanism can also be use for the development
of definition files such as \LaTeX{} styles or classes.
This case differs from the above setup with multiple parts
included by |\include| in that no |\includeonly| should be invoked.
This can be achieved by starting the include file
(before |\ProvidesPackage|) with:
%
\begin{center}
\begin{tabular}{l}
|\input{childdoc.def}|\\
|\childdocforward{|\textit{main}|}|\\
\end{tabular}
\end{center}
%
or alternatively with:
%
\begin{center}
\begin{tabular}{l}
|\input{childdoc.def}|\\
|\childdocby{|\textit{main}|}|\\
\end{tabular}
\end{center}
%
Both forms have slightly different effects as described above.
The main file is prepared as usual, see \secref{sec:include}.

%%%%%%%%%%%%%%%%%%%%%%%%%%%%%%%%%%%%%%%%%%%%%%%%%%%%%%%%%%%%%%%%%%%%%%%%%%%%%%%%
\subsection{Legacy Detection}
\label{sec:detection}

The directive |\childdocmain| in the main file can detect
whether the complete document or merely a child is to be compiled
even without using the directive |\childdocof|.
This method is deprecated because it is less robust
and there is no compelling reason to use it;
it is merely provided for backward compatibility
and it may be removed in future versions.

If the detection mechanism is to be used,
it is mandatory to correctly specify
the filename of the main file as the argument of |\childdocmain|:
%
\begin{center}
\begin{tabular}{l}
|\input{childdoc.def}|\\
|\childdocmain{|\textit{main}|}|\\
\end{tabular}
\end{center}
%
If |\jobname| does not match the argument \textit{main} of |\childdocmain|,
it is assumed that |\jobname| points to the child file to be compiled.
When using |\childdocmain| with the main file specified as argument,
it suffices to start a child file
with just |\input{|\textit{main}|}|
without loading of the package and using |\childdocof|.
If instead all processing is done
with the appropriate \textsf{childdoc} directives,
the argument of \textit{main} of |\childdocmain| can be empty.

An alternative version of the command line processing described
in \secref{sec:commandline} using the detection mechanism reads:
%
\begin{center}
|... -jobname "|\textit{target}|" "|[\textit{flags}]%
[|\def\jobname{|\textit{dest}|}|]|\input{|\textit{main}|}"|
\end{center}

%%%%%%%%%%%%%%%%%%%%%%%%%%%%%%%%%%%%%%%%%%%%%%%%%%%%%%%%%%%%%%%%%%%%%%%%%%%%%%%%
\subsection{Manual Code}
\label{sec:manual}

In case one cannot be certain whether the definitions file |childdoc.def|
is installed on the target \TeX{} distribution
and one prefers not to ship it,
it is conceivable to paste a few relevant commands into the sources.

To that end, drop all statements |\input{childdoc.def}|
and perform the replacements as outlined below.
Instead of |\childdocmain{|\textit{main}|}| add the following code
to the top of the main file:
%
\begin{center}
\begin{tabular}{l}
|\||ifdefined\childdocname\endinput\||fi\newif\ifchilddoc|\\
|\edef\childdocname{\scantokens\expandafter{\jobname\noexpand}}|\\
|\def\childdocmain{|\textit{main}|}\||ifx\childdocmain\childdocname\||else|\\
|\childdoctrue\includeonly{\childdocname}\let\jobname\childdocmain\||fi|\\
\end{tabular}
\end{center}
%
Instead of |\childdocof{|\textit{main}|}| just include the main file
at the top of each child file:
%
\begin{center}
|\input{|\textit{main}|}|
\end{center}
%
A simple redirection |\childdocforward{|\textit{dest}|}| is achieved by:
%
\begin{center}
|\def\jobname{|\textit{dest}|}\input{\jobname}|
\end{center}
%
The redirection with prefix
|\childdocforwardprefix[|\textit{prefix}|]{|\textit{dest}|}|
is accomplished by:
%
\begin{center}
\begin{tabular}{l}
|{\edef\jobname{\scantokens\expandafter{\jobname\noexpand}}|\\
|\def\redirectjob |\textit{prefix}|#1~~~{\gdef\jobname{|\textit{dest}|#1}}|\\
|\expandafter\redirectjob\jobname~~~}\input{\jobname}|
\end{tabular}
\end{center}

In an alternative approach,
child documents can be compiled by a specific command line
without additional code or specific definitions:
%
\begin{center}
|... -jobname "|\textit{target}|" "|[\textit{flags}]%
|\includeonly{|\textit{dest}|}\input{|\textit{main}|}"|
\end{center}
%

%%%%%%%%%%%%%%%%%%%%%%%%%%%%%%%%%%%%%%%%%%%%%%%%%%%%%%%%%%%%%%%%%%%%%%%%%%%%%%%%
%%%%%%%%%%%%%%%%%%%%%%%%%%%%%%%%%%%%%%%%%%%%%%%%%%%%%%%%%%%%%%%%%%%%%%%%%%%%%%%%
\section{Information}

%%%%%%%%%%%%%%%%%%%%%%%%%%%%%%%%%%%%%%%%%%%%%%%%%%%%%%%%%%%%%%%%%%%%%%%%%%%%%%%%
\subsection{Copyright}

Copyright \copyright{} 2017--2018 Niklas Beisert

This work may be distributed and/or modified under the
conditions of the \LaTeX{} Project Public License, either version 1.3
of this license or (at your option) any later version.
The latest version of this license is in
  \url{http://www.latex-project.org/lppl.txt}
and version 1.3 or later is part of all distributions of \LaTeX{}
version 2005/12/01 or later.

This work has the LPPL maintenance status `maintained'.

The Current Maintainer of this work is Niklas Beisert.

This work consists of the files |README.txt|, |childdoc.ins| and |childdoc.dtx|
as well as the derived files |childdoc.def|, |cdocsamp.tex|
with |cdocsch1.tex|, |cdocsch2.tex|, |cdocspt3.tex|, |cdocspt4.tex|,
|cdocsdrf.tex|, |cdocsfn1.tex|, |cdocsfn2.tex|
as well as |childdoc.pdf|.

%%%%%%%%%%%%%%%%%%%%%%%%%%%%%%%%%%%%%%%%%%%%%%%%%%%%%%%%%%%%%%%%%%%%%%%%%%%%%%%%
\subsection{Files and Installation}

The package consists of the files:
%
\begin{center}
\begin{tabular}{ll}
    |README.txt|   & readme file \\
    |childdoc.ins| & installation file \\
    |childdoc.dtx| & source file \\
    |childdoc.def| & definition file \\
    |cdocsamp.tex| & sample main file \\
    |cdocsch1.tex| & sample include file \\
    |cdocsch2.tex| & sample include file \\
    |cdocspt3.tex| & sample part file \\
    |cdocspt4.tex| & sample part file \\
    |cdocsdrf.tex| & sample redirection file \\
    |cdocsfn1.tex| & sample redirection file \\
    |cdocsfn2.tex| & sample redirection file \\
    |childdoc.pdf| & manual
\end{tabular}
\end{center}
%
The distribution consists of the files
|README.txt|, |childdoc.ins| and |childdoc.dtx|.
%
\begin{itemize}
\item
Run (pdf)\LaTeX{} on |childdoc.dtx|
to compile the manual |childdoc.pdf| (this file).
\item
Run \LaTeX{} on |childdoc.ins| to create the definitions file |childdoc.def|
and the sample |cdocsamp.tex| with include files
|cdocsch1.tex|, |cdocsch2.tex|, |cdocspt3.tex|, |cdocspt4.tex|,
|cdocsdrf.tex|, |cdocsfn1.tex|, |cdocsfn2.tex|.
Then copy the file |childdoc.def| to an appropriate directory of your \LaTeX{}
distribution, e.g.\ \textit{texmf-root}|/tex/latex/childdoc|.
\end{itemize}

%%%%%%%%%%%%%%%%%%%%%%%%%%%%%%%%%%%%%%%%%%%%%%%%%%%%%%%%%%%%%%%%%%%%%%%%%%%%%%%%
\subsection{Related CTAN Packages}

There are several other packages which offer a similar functionality:
%
\begin{itemize}
\item
The packages
\href{http://ctan.org/pkg/docmute}{\textsf{docmute}},
\href{http://ctan.org/pkg/includex}{\textsf{includex}} and
\href{http://ctan.org/pkg/standalone}{\textsf{standalone}}
provide commands to include only the document body of
a child file thus allowing both files to be compiled individually.
\item
The packages \href{http://ctan.org/pkg/subdocs}{\textsf{subdocs}}
and \href{http://ctan.org/pkg/subfiles}{\textsf{subfiles}}
provide structures in which the main and child documents can be
encapsulated and allowing them to be compiled individually.
The inclusion mechanism is different from the conventional |\include|.
\item
The package \href{http://ctan.org/pkg/combine}{\textsf{combine}}
is an elaborate solution to combine several documents into one.
\end{itemize}
%
See also the CTAN topic \href{http://ctan.org/topic/subdocs}{\textsf{subdocs}}
for further related packages.
The present package differs from the above solutions in that
a document structure constructed with the conventional |\include| mechanism
just needs two extra commands at the top of every file
such that all constituent files can be compiled individually.

%%%%%%%%%%%%%%%%%%%%%%%%%%%%%%%%%%%%%%%%%%%%%%%%%%%%%%%%%%%%%%%%%%%%%%%%%%%%%%%%
%\subsection{Feature Suggestions}
%
%The following is a list of features which may be useful for future
%versions of this package:
%%
%\begin{itemize}
%\item
%\ldots
%\end{itemize}

%%%%%%%%%%%%%%%%%%%%%%%%%%%%%%%%%%%%%%%%%%%%%%%%%%%%%%%%%%%%%%%%%%%%%%%%%%%%%%%%
\subsection{Revision History}

%%%%%%%%%%%%%%%%%%%%%%%%%%%%%%%%%%%%%%%%
\paragraph{v2.0:} 2018/12/30

\begin{itemize}
\item
immediate forward processing
\item
added |\childdocby| mechanism
\item
manual restructured
\end{itemize}

%%%%%%%%%%%%%%%%%%%%%%%%%%%%%%%%%%%%%%%%
\paragraph{v1.6:} 2018/01/17

\begin{itemize}
\item
application for development of include files
\item
corrections to manual
\end{itemize}

%%%%%%%%%%%%%%%%%%%%%%%%%%%%%%%%%%%%%%%%
\paragraph{v1.5:} 2017/05/21

\begin{itemize}
\item
more complete structuring introduced
\item
|\childdocof| introduced
\item
|\childdoc| renamed to |\childdocmain|
\item
|\childredirect| renamed to |\childdocforward| and |\childdocforwardprefix|
and functionality expanded
\end{itemize}

%%%%%%%%%%%%%%%%%%%%%%%%%%%%%%%%%%%%%%%%
\paragraph{v1.0:} 2017/04/27

\begin{itemize}
\item
manual and install package
\item
first version published on CTAN
\end{itemize}

%%%%%%%%%%%%%%%%%%%%%%%%%%%%%%%%%%%%%%%%
\paragraph{v0.6:} 2017/04/26

\begin{itemize}
\item
redirection mechanism added
\end{itemize}

%%%%%%%%%%%%%%%%%%%%%%%%%%%%%%%%%%%%%%%%
\paragraph{v0.5:} 2017/04/26

\begin{itemize}
\item
functionality in definition file
\end{itemize}


%%%%%%%%%%%%%%%%%%%%%%%%%%%%%%%%%%%%%%%%%%%%%%%%%%%%%%%%%%%%%%%%%%%%%%%%%%%%%%%%
%%%%%%%%%%%%%%%%%%%%%%%%%%%%%%%%%%%%%%%%%%%%%%%%%%%%%%%%%%%%%%%%%%%%%%%%%%%%%%%%
%%%%%%%%%%%%%%%%%%%%%%%%%%%%%%%%%%%%%%%%%%%%%%%%%%%%%%%%%%%%%%%%%%%%%%%%%%%%%%%%
\appendix

\settowidth\MacroIndent{\rmfamily\scriptsize 000\ }

 \DocInput{childdoc.dtx}

\end{document}
%</driver>
% \fi
%
% %%%%%%%%%%%%%%%%%%%%%%%%%%%%%%%%%%%%%%%%%%%%%%%%%%%%%%%%%%%%%%%%%%%%%%%%%%%%%%
% %%%%%%%%%%%%%%%%%%%%%%%%%%%%%%%%%%%%%%%%%%%%%%%%%%%%%%%%%%%%%%%%%%%%%%%%%%%%%%
% \section{Sample}
%\iffalse
%<*samplemain>
%\fi
%
% The following presents a sample document
% with two chapters, two parts, a title page,
% a compile flag as well as three forwarding files to set the flag.
% It consists of eight |.tex| files:
% \begin{center}
% \begin{tabular}{ll}
% |cdocsamp.tex|&main file\\
% |cdocsch1.tex|&include file for chapter 1\\
% |cdocsch2.tex|&include file for chapter 2\\
% |cdocspt3.tex|&include file for part 3\\
% |cdocspt4.tex|&include file for part 4\\
% |cdocsdrf.tex|&forwarding file for main file in draft mode\\
% |cdocsfi1.tex|&forwarding file for final version of chapter 1\\
% |cdocsfi2.tex|&forwarding file for final version of chapter 2\\
% \end{tabular}
% \end{center}
% Each of the eight files can be compiled directly by the \LaTeX{} compiler.
%
% %%%%%%%%%%%%%%%%%%%%%%%%%%%%%%%%%%%%%%
% \paragraph{Main File.}
%
% The main file is called |cdocsamp.tex|.
%
% Load the \textsf{childdoc} definitions and
% declare the filename for the main document:
%    \begin{macrocode}
\input{childdoc.def}
\childdocmain{}
%    \end{macrocode}

% Optional override for |\version| flag:
%    \begin{macrocode}
%%\ifchilddoc\else\providecommand{\version}{draft}\fi
%    \end{macrocode}

% Define the default values for the |\version| flag
% (|final| for the main file and |draft| for childs):
%    \begin{macrocode}
\ifchilddoc
\providecommand{\version}{draft}
\else
\providecommand{\version}{final}
\fi
%    \end{macrocode}

% Load the standard document class:
%    \begin{macrocode}
\documentclass[12pt]{article}
%    \end{macrocode}

% Start the document body:
%    \begin{macrocode}
\begin{document}
%    \end{macrocode}

% Declare a title page.
% Print title, part of document being processed and version flag:
%    \begin{macrocode}
\addtocounter{page}{-1}
\begin{center}
{\LARGE\bfseries{}childdoc example\par}
\vspace{1cm}
\ifchilddoc
\ifchilddocmanual part\else chapter\fi:
`\childdocname' of `\childdocjob'\par
\else
main document: `\childdocjob'\par
\fi
version: \version\par
\end{center}
\newpage
%    \end{macrocode}

% Manually include selected file,
% otherwise process as usual:
%    \begin{macrocode}
\ifchilddocmanual
\section*{part `\childdocname'}
\input{\childdocname}
\else
%    \end{macrocode}

% Include the two chapters:
%    \begin{macrocode}
\include{cdocsch1}
\include{cdocsch2}
%    \end{macrocode}

% Include the two parts unless only chapters should be displayed:
%    \begin{macrocode}
\ifchilddoc\else
\section{part three}
\input{cdocspt3}
\section{part four}
\input{cdocspt4}
\fi
%    \end{macrocode}

% Process as usual until here:
%    \begin{macrocode}
\fi
%    \end{macrocode}

% End of document body:
%    \begin{macrocode}
\end{document}
%    \end{macrocode}
%\iffalse
%</samplemain>
%\fi
%
% %%%%%%%%%%%%%%%%%%%%%%%%%%%%%%%%%%%%%%
% \paragraph{Chapter Include Files.}
%
% The include files are called |cdocsch1.tex| and |cdocsch2.tex|.
%
%\iffalse
%<*samplechap1|samplechap2>
%\fi

% Optional override for |\version| flag:
%    \begin{macrocode}
%%\providecommand{\version}{final}
%    \end{macrocode}

% Include the main document:
%    \begin{macrocode}
\input{childdoc.def}
\childdocof{cdocsamp}
%    \end{macrocode}

%\iffalse
%</samplechap1|samplechap2>
%\fi
%
%\iffalse
%<*samplechap1>
%\fi
% Some text for chapter 1:
%    \begin{macrocode}
\section{one}
some text in chapter one
%    \end{macrocode}

%\iffalse
%</samplechap1>
%\fi
% Some text for chapter 2:
%\iffalse
%<*samplechap2>
%\fi
%    \begin{macrocode}
\section{two}
more text in chapter two
%    \end{macrocode}

%\iffalse
%</samplechap2>
%\fi
%
% %%%%%%%%%%%%%%%%%%%%%%%%%%%%%%%%%%%%%%
% \paragraph{Part Include Files.}
%
% The include files are called |cdocspt3.tex| and |cdocspt4.tex|.
%
%\iffalse
%<*samplepart3|samplepart4>
%\fi

% Optional override for |\version| flag:
%    \begin{macrocode}
%%\providecommand{\version}{final}
%    \end{macrocode}

% Include the main document:
%    \begin{macrocode}
\input{childdoc.def}
\childdocby{cdocsamp}
%    \end{macrocode}

%\iffalse
%</samplepart3|samplepart4>
%\fi
%
%\iffalse
%<*samplepart3>
%\fi
% Some text for part 3:
%    \begin{macrocode}
some text in part three
%    \end{macrocode}

%\iffalse
%</samplepart3>
%\fi
% Some text for part 4:
%\iffalse
%<*samplepart4>
%\fi
%    \begin{macrocode}
more text in part four
%    \end{macrocode}

%\iffalse
%</samplepart4>
%\fi
%
% %%%%%%%%%%%%%%%%%%%%%%%%%%%%%%%%%%%%%%
% \paragraph{Forwarding for a Complete Draft.}
%
% The following forwarding file |cdocsdrf.tex|
% compiles the main document in draft mode:
%\iffalse
%<*sampledraft>
%\fi
%    \begin{macrocode}
\def\version{draft}
\input{childdoc.def}
\childdocforward{cdocsamp}
%    \end{macrocode}

%\iffalse
%</sampledraft>
%\fi
%
% %%%%%%%%%%%%%%%%%%%%%%%%%%%%%%%%%%%%%%
% \paragraph{Forwarding for Final Version of the Chapters.}
%
% The following forwarding files |cdocsfn1.tex| and |cdocsfn2.tex|
% (with identical content)
% compile the final versions of the child documents
% |cdocsch1.tex| and |cdocsch2.tex|, respectively:
%\iffalse
%<*samplefinal>
%\fi
%    \begin{macrocode}
\def\version{final}
\input{childdoc.def}
\childdocforwardprefix[cdocsamp]{cdocsfn}{cdocsch}
%    \end{macrocode}

%\iffalse
%</samplefinal>
%\fi
%
% %%%%%%%%%%%%%%%%%%%%%%%%%%%%%%%%%%%%%%
% \paragraph{Command Line Processing.}
%
% The following three command lines generate the output files
% |cdocscld|, |cdocscl1| and |cdocscl2|
% which should be identical to
% |cdocsdrf|, |cdocsch1| and |cdocsfn2|, respectively:
% \begin{center}
% \begin{tabular}{l}
% |latex -jobname cdocscld \|\\
% |  "\def\version{draft}\input{childdoc.def}\childdocforward{cdocsamp}"|\\
% |latex -jobname cdocscl1 \|\\
% |  "\input{childdoc.def}\childdocforward[cdocsamp]{cdocsch1}"|\\
% |latex -jobname cdocscl2 \|\\
% |  "\def\version{final}\input{childdoc.def}\childdocforward{cdocsch2}"|
% \end{tabular}
% \end{center}
% Note that the trailing backslash on each first line
% merely continues the input to the second line
% (for convenient cut ant paste).
% Furthermore, the command |latex| can be replaced by any
% of its alternative versions such as |pdflatex|.
%
% %%%%%%%%%%%%%%%%%%%%%%%%%%%%%%%%%%%%%%%%%%%%%%%%%%%%%%%%%%%%%%%%%%%%%%%%%%%%%%
% %%%%%%%%%%%%%%%%%%%%%%%%%%%%%%%%%%%%%%%%%%%%%%%%%%%%%%%%%%%%%%%%%%%%%%%%%%%%%%
% \section{Implementation}
%\iffalse
%<*package>
%\fi
%
% This section describes the definitions file |childdoc.def|.

% The definitions cannot be loaded using |\usepackage| or |\RequirePackage|
% which has a mechanism to prevent loading a style file more than once.
% When loading the definitions by means of |\input|
% multiple instances have to be prevented manually:
%\iffalse
%This code needs to be before the `\ProvidesFile' directive
%which is defined at the beginning of this file.
%Therefore it is also placed there and commented out here.
%</package>
%<*discard>
%\fi
%    \begin{macrocode}
\ifdefined\childdocmain\endinput\fi
%    \end{macrocode}
%\iffalse
%</discard>
%<*package>
%\fi
%
% \macro{\ifchilddoc}
% \macro{\ifchilddocmanual}
% The conditional |\ifchilddoc| tells whether a
% child (true) or main (false) document is being compiled.
% The conditional |\ifchilddocmanual| tells whether
% the |\includeonly| mechanism is used (false) or
% the selection of child files must be performed manually (true).
% The definitions initialise to false:
%    \begin{macrocode}
\newif\ifchilddoc
\newif\ifchilddocmanual
%    \end{macrocode}

% \macro{\childdocname}
% \macro{\childdocjob}
% The macro |\childdocname| stores the name of the main document
% to be compiled. The macro |\childdocjob| stores the name of
% the document on which the \LaTeX{} compiler was originally invoked.
% The content of |\jobname| cannot be compared
% to filenames specified in the source due to different catcodes.
% The following code rescans |\jobname|, stores the result
% in |\childdocname| and saves a copy in |\childdocjob|:
%    \begin{macrocode}
\edef\childdocname{\scantokens\expandafter{\jobname\noexpand}}
\let\childdocjob\childdocname
%    \end{macrocode}

% \macro{\childdocdisable}
% The macro |\childdocdisable| prevents the main file
% from being processed more than once.
% At this stage, the main document command |\childdocmain|
% is assumed to be called once again where it should do nothing.
% Any subsequent call to it should prevent
% a secondary processing of the main document
% It overwrites the forwarding commands
% |\childdocof| and |\childdocforward|
% with empty macros to prevent further inclusions of the main document:
%    \begin{macrocode}
\newcommand{\childdocdisable}
{
  \renewcommand{\childdocmain}[1]{\renewcommand{\childdocmain}[1]{\endinput}}
  \renewcommand{\childdocof}[1]{}
  \renewcommand{\childdocby}[2][]{}
  \renewcommand{\childdocforward}[2][]{}
  \renewcommand{\childdocdisable}{}
}
%    \end{macrocode}

% \macro{\childdocmain}
% The macro |\childdocmain| is to be called at the top of the main file
% with nothing or the main filename (without extension) as argument.
% First, it breaks loops.
% If the argument is not empty and does not match |\childdocname|
% (which is set by the first inclusion of |childdoc.def|),
% |\ifchilddoc| is set to true, |\includeonly| is applied to the child file
% and |\jobname| is set to the main file
% (for proper handling of |.aux| files):
%    \begin{macrocode}
\newcommand{\childdocmain}[1]
{
  \childdocdisable\childdocmain{}
  \if?#1?\else
    \begingroup
      \def\childdoctmp{#1}
      \ifx\childdoctmp\childdocname
        \def\childdoctmp{}
      \else
        \def\childdoctmp
        {
          \childdoctrue
          \includeonly{\childdocname}
          \def\childdocjob{#1}
          \def\jobname{#1}
        }
      \fi
      \expandafter
    \endgroup
    \childdoctmp
  \fi
}
%    \end{macrocode}

% \macro{\childdocof}
% The command |\childdocof| redirects
% compilation to the main file |#1|.
%    \begin{macrocode}
\newcommand{\childdocof}[1]
{
  \childdocdisable
  \childdoctrue
  \includeonly{\childdocname}
  \def\jobname{#1}
  \def\childdocjob{#1}
  \input{#1}
}
%    \end{macrocode}

% \macro{\childdocby}
% The command |\childdocby| ....
%    \begin{macrocode}
\newcommand{\childdocby}[2][]
{
  \childdocdisable
  \childdoctrue
  \childdocmanualtrue
  \if?#1?\else
    \def\jobname{#2}
  \fi
  \def\childdocjob{#2}
  \input{#2}
  \endinput
}
%    \end{macrocode}

% \macro{\childdocforward}
% The command |\childdocforward| redirects
% compilation to the main file or
% (if the optional argument is given) a child file.
% Parameters are set as if the main file
% or a child file starting with |\childdocof| was compiled.
% Then compilation is handed over to the main file:
%    \begin{macrocode}
\newcommand{\childdocforward}[2][]
{
  \begingroup
    \if?#1?
      \def\childdoctmp
      {
        \def\childdocname{#2}
        \def\childdocjob{#2}
        \def\jobname{#2}
        \input{#2}
        \endinput
      }
    \else
      \def\childdoctmp
      {
        \childdocdisable
        \def\childdocname{#2}
        \childdoctrue
        \includeonly{#2}
        \def\childdocjob{#1}
        \def\jobname{#1}
        \input{#1}
        \endinput
      }
    \fi
    \expandafter
  \endgroup
  \childdoctmp
}
%    \end{macrocode}

% \macro{\childdocforwardprefix}
% The command |\childdocforwardprefix| redirects
% compilation to the main or a child file by means of a pattern.
% The prefix |#1| in the current filename is replaced by |#2|
% and the suffix of the current filename is kept
% (it is assumed that the filename does not contain the substring `|~~~|'
% which is used as a delimiter).
% Compilation is handed over to the new file by |\childdocforward|:
%    \begin{macrocode}
\newcommand{\childdocforwardprefix}[3][]
{
  \begingroup
    \def\childdocextract #2##1~~~{\def\childdoctmp{\childdocforward[#1]{#3##1}}}
    \expandafter\childdocextract\childdocname~~~
    \expandafter
  \endgroup
  \childdoctmp
}
%    \end{macrocode}

% \macro{\childdoc}
% The deprecated macro |\childdoc| is a legacy version of |\childdocmain|:
%    \begin{macrocode}
\newcommand{\childdoc}{\childdocmain}
%    \end{macrocode}

% \macro{\childdocredirect}
% The deprecated macro |\childdocredirect| is a legacy version
% of |\childdocforward| and |\childdocforwardprefix|:
%    \begin{macrocode}
\newcommand{\childdocredirect}[2][]
{
  \begingroup
    \if?#1?
      \def\childdoctmp{\childdocforward{#2}}
    \else
      \def\childdoctmp{\childdocforwardprefix{#1}{#2}}
    \fi
    \expandafter
  \endgroup
  \childdoctmp
}
%    \end{macrocode}

%\iffalse
%</package>
%\fi
%
\endinput

\childdocforwardprefix[cdocsamp]{cdocsfn}{cdocsch}
%    \end{macrocode}

%\iffalse
%</samplefinal>
%\fi
%
% %%%%%%%%%%%%%%%%%%%%%%%%%%%%%%%%%%%%%%
% \paragraph{Command Line Processing.}
%
% The following three command lines generate the output files
% |cdocscld|, |cdocscl1| and |cdocscl2|
% which should be identical to
% |cdocsdrf|, |cdocsch1| and |cdocsfn2|, respectively:
% \begin{center}
% \begin{tabular}{l}
% |latex -jobname cdocscld \|\\
% |  "\def\version{draft}% \iffalse
%
% childdoc.dtx Copyright (C) 2017-2018 Niklas Beisert
%
% This work may be distributed and/or modified under the
% conditions of the LaTeX Project Public License, either version 1.3
% of this license or (at your option) any later version.
% The latest version of this license is in
%   http://www.latex-project.org/lppl.txt
% and version 1.3 or later is part of all distributions of LaTeX
% version 2005/12/01 or later.
%
% This work has the LPPL maintenance status `maintained'.
%
% The Current Maintainer of this work is Niklas Beisert.
%
% This work consists of the files childdoc.dtx and childdoc.ins
% and the derived files childdoc.def and cdocsamp.tex with
% cdocsch1.tex, cdocsch2.tex, cdocsdrf.tex, cdocsfn1.tex, cdocsfn2.tex.
%
%<package>\ifdefined\childdocmain\endinput\fi
%<package>\ProvidesFile{childdoc.def}[2018/12/30 v2.0 child document driver]
%<samplemain>\ProvidesFile{cdocsamp.tex}[2018/12/30 v2.0 sample for childdoc]
%<*driver>
%\ProvidesFile{childdoc.drv}[2018/12/30 v2.0 childdoc reference manual file]
\PassOptionsToClass{10pt,a4paper}{article}
\documentclass{ltxdoc}

\usepackage[margin=35mm]{geometry}
\usepackage{hyperref}
\usepackage{hyperxmp}
\usepackage[usenames]{color}

\hypersetup{colorlinks=true}
\hypersetup{pdfstartview=FitH}
\hypersetup{pdfpagemode=UseNone}
\hypersetup{pdfsource={}}
\hypersetup{pdflang={en-UK}}
\hypersetup{pdfcopyright={Copyright 2017-2018 Niklas Beisert.
  This work may be distributed and/or modified under the
  conditions of the LaTeX Project Public License, either version 1.3
  of this license or (at your option) any later version.}}
\hypersetup{pdflicenseurl={http://www.latex-project.org/lppl.txt}}
\hypersetup{pdfcontactaddress={ETH Zurich, ITP, HIT K,
  Wolfgang-Pauli-Strasse 27}}
\hypersetup{pdfcontactpostcode={8093}}
\hypersetup{pdfcontactcity={Zurich}}
\hypersetup{pdfcontactcountry={Switzerland}}
\hypersetup{pdfcontactemail={nbeisert@itp.phys.ethz.ch}}
\hypersetup{pdfcontacturl={http://people.phys.ethz.ch/\xmptilde nbeisert/}}

\newcommand{\secref}[1]{\hyperref[#1]{section \ref*{#1}}}

\parskip1ex
\parindent0pt
\let\olditemize\itemize
\def\itemize{\olditemize\parskip0pt}

\begin{document}

\title{The \textsf{childdoc} Package}
\hypersetup{pdftitle={The childdoc Package}}
\author{Niklas Beisert\\[2ex]
  Institut f\"ur Theoretische Physik\\
  Eidgen\"ossische Technische Hochschule Z\"urich\\
  Wolfgang-Pauli-Strasse 27, 8093 Z\"urich, Switzerland\\[1ex]
  \href{mailto:nbeisert@itp.phys.ethz.ch}
  {\texttt{nbeisert@itp.phys.ethz.ch}}}
\hypersetup{pdfauthor={Niklas Beisert}}
\hypersetup{pdfsubject={Manual for the LaTeX2e Package childdoc}}
\date{30 December 2018, \textsf{v2.0}}
\maketitle

\begin{abstract}\noindent
\textsf{childdoc} is a \LaTeXe{} package
that enables the direct compilation
of document sections included by |\include|
to individual files.
\end{abstract}

\begingroup
\parskip0ex
\tableofcontents
\endgroup

%%%%%%%%%%%%%%%%%%%%%%%%%%%%%%%%%%%%%%%%%%%%%%%%%%%%%%%%%%%%%%%%%%%%%%%%%%%%%%%%
%%%%%%%%%%%%%%%%%%%%%%%%%%%%%%%%%%%%%%%%%%%%%%%%%%%%%%%%%%%%%%%%%%%%%%%%%%%%%%%%
\section{Introduction}

\LaTeX{} provides a mechanism to structure a large document (such as a book)
into a main file and several child files (containing the chapters)
using the |\include| command.
This mechanism is beneficial for documents
which span hundreds of pages in order to
make the source file(s) more manageable.
Moreover, compilation can be restricted to
selected child files by means of the |\includeonly| command.
The latter feature can be used to reduce the compilation time while editing
(this was significantly more useful in the earlier days of \LaTeX{})
or to generate a smaller document which is easier to navigate.
Another application of |\includeonly| is to generate
documents consisting of selected parts of the complete document.

However, there are a few drawbacks of the plain |\include| mechanism:
\begin{itemize}
\item
The child files cannot be compiled on their own,
they can only be compiled via the main file.
A naive editing environment
(such as a text editor with an option
to have the current file processed by \LaTeX)
may require one to switch to the main file before compiling;
attempting to compile the child file produces errors.
\item
The main file must be modified (each time)
to adjust the |\includeonly| command
to the present needs. This easily leaves the main file in a messy state.
\item
The generated document will always carry the filename
of the main document. This is inconvenient if
several child files are to be compiled and
to be kept for distribution.
\end{itemize}

The present package provides a simple interface
to make child files individually compilable by \LaTeX{}.
Compiling a child file then has the same effect as compiling
the main file with an |\includeonly| command
to select the appropriate child.
Moreover the generated document will carry the name of the child
rather than the main file.
This resolves all three above issues.

This feature is meant to make the editing of books,
thesis documents and lecture notes somewhat more convenient.
However, the package can also be used efficiently for
composing a series of documents (such as exercise sheets)
which are typically distributed individually.
It then assists the author in generating the individual documents
(potentially in different versions)
as well as a document containing the collected series.
Another application is in developing style files
or other kinds of included material
where compilation of the style file could redirect
to a sample or test file.

%%%%%%%%%%%%%%%%%%%%%%%%%%%%%%%%%%%%%%%%%%%%%%%%%%%%%%%%%%%%%%%%%%%%%%%%%%%%%%%%
%%%%%%%%%%%%%%%%%%%%%%%%%%%%%%%%%%%%%%%%%%%%%%%%%%%%%%%%%%%%%%%%%%%%%%%%%%%%%%%%
\section{Usage}

First of all, the package \textsf{childdoc} is \emph{not} a standard
\LaTeXe{} |.sty| style file! Therefore it needs to be invoked in
a non-standard way.

%%%%%%%%%%%%%%%%%%%%%%%%%%%%%%%%%%%%%%%%%%%%%%%%%%%%%%%%%%%%%%%%%%%%%%%%%%%%%%%%
\subsection{Included Files}
\label{sec:include}

%%%%%%%%%%%%%%%%%%%%%%%%%%%%%%%%%%%%%%%%
\DescribeMacro{\childdocmain}
To use the package, add the commands
\begin{center}
\begin{tabular}{l}
|\input{childdoc.def}|\\
|\childdocmain{}|\\
\end{tabular}
\end{center}
at the very top of the main \LaTeX{} file,
in particular \emph{before} the |\documentclass| statement!
The argument of |\childdocmain| should be left empty
(but it must be present).

%%%%%%%%%%%%%%%%%%%%%%%%%%%%%%%%%%%%%%%%
\DescribeMacro{\childdocof}
Furthermore, add the commands
\begin{center}
\begin{tabular}{l}
|\input{childdoc.def}|\\
|\childdocof{|\textit{main}|}|\\
\end{tabular}
\end{center}
at the top of every child file \textit{child}
which is included by |\include{|\textit{child}|}|
from within the main file
(or at least for those files to be compiled individually).
The argument \textit{main} must be the filename of the main file.

There are a couple of
considerations in setting up the main and child documents:

%%%%%%%%%%%%%%%%%%%%%%%%%%%%%%%%%%%%%%%%
\paragraph{Restrictions.}

Please note the following restrictions:
\begin{itemize}
\item
|\childdocmain| must be called with one argument \textit{main}
to ensure compatibility with earlier version of the package.
It must either be empty (|\childdocmain{}|)
or precisely match the filename of the main file in which it is specified.
See \secref{sec:detection} for further information.
\item
The filename \textit{main} must be specified without the |.tex| extension.
\item
The filename \textit{main} is case sensitive
(even in case-insensitive file systems)
due to internal string comparison.
\item
The argument \textit{main} should be fully expanded, it cannot be a macro.
\item
Subdirectories and special characters should be avoided in filenames.
\item
The command |\childdocmain{|\textit{main}|}| must be followed by a whitespace.
It should not be followed immediately by another command
or by a comment mark `|%|'.
This is because the \TeX{} parser reads the token immediately following
the argument of |\childdocmain| and puts it
at the beginning of every child section;
however, a white\-space is ignored.
\end{itemize}

%%%%%%%%%%%%%%%%%%%%%%%%%%%%%%%%%%%%%%%%
\paragraph{Content of Main File.}

It is advisable to place all content in the child files included by |\include|.
Any output contained in the main file will appear in all child documents
unless suppressed manually;
it cannot be suppressed automatically by the |\includeonly| directive
and thus should normally be avoided.
A method to include some content in the main file
by means of conditional processing is described in \secref{sec:conditional}.

%%%%%%%%%%%%%%%%%%%%%%%%%%%%%%%%%%%%%%%%
\paragraph{Page Numbering.}

When only a part of the document is compiled,
the appropriate numbering of pages
(as well as other status parameters)
is determined from the |.aux| files.
The latter contain information from previous passes.
However this information needs to propagate through
all intermediate child documents.
Therefore the page numbering in child documents may well
be inconsistent until the complete document is compiled at least once.

A useful (if unconventional) way to always ensure a consistent
page numbering is to restart the numbering in each child document
and denote the pages by `\textit{child}|.|\textit{page}'
where \textit{child} represents the chapter/section number of the child file.
This can be achieved by the command
|\numberwithin{page}{|\textit{child}|}|
of the \textsf{amsmath} package
where \textit{child} can be |chapter| or |section|
depending on the chosen structuring.
Alternatively, one can modify the macro |\thepage| appropriately
and reset the counter |page| at the start of each child file.

%%%%%%%%%%%%%%%%%%%%%%%%%%%%%%%%%%%%%%%%%%%%%%%%%%%%%%%%%%%%%%%%%%%%%%%%%%%%%%%%
\subsection{Conditional Processing}
\label{sec:conditional}

The package provides a mechanism to compile different versions
of a document. To customise the versions further some conditional processing
can come in handy to distinguish which version is being compiled.
The package provides two macros to describe the compilation context:

%%%%%%%%%%%%%%%%%%%%%%%%%%%%%%%%%%%%%%%%
\DescribeMacro{\ifchilddoc}
The conditional |\ifchilddoc| distinguishes between the compilation of
child documents and the main document:
%
\begin{center}
|\ifchilddoc |\textit{child-code}| |[|\||else |\textit{main-code}]| \||fi|
\end{center}

%%%%%%%%%%%%%%%%%%%%%%%%%%%%%%%%%%%%%%%%
\DescribeMacro{\childdocname}
\DescribeMacro{\childdocjob}
The macro |\childdocname| contains the filename (without extension)
of the main or child file being processed.
Note that |\childdocjob| will always contain the name of the main file.

%%%%%%%%%%%%%%%%%%%%%%%%%%%%%%%%%%%%%%%%
\paragraph{Title Page.}

Conditional processing can be used to include a title or banner page
in the main document when proper precautions are taken.
Importantly, the code in the main file should ensure that the page counter
(as well as other status parameters which are stored in the |.aux| files)
takes the same value after the conditional processing.
Otherwise the page numbers may take divergent values
depending on which part is compiled.

For example, a title page could be declared by:
%
\begin{center}
\begin{tabular}{l}
|\ifchilddoc\||else|\\
|\addtocounter{page}{-1}|\\
\textit{code for title page}\\
|\newpage|\\
|\||fi|
\end{tabular}
\end{center}
%
A banner page for the child documents can be generated by:
%
\begin{center}
\begin{tabular}{l}
|\ifchilddoc|\\
|\addtocounter{page}{-1}|\\
\textit{code for banner page}\\
|\newpage|\\
|\||fi|
\end{tabular}
\end{center}
%
Here one could write a message such as:
\begin{center}
|This is the part \childdocname{} of \childdocjob{}.|
\end{center}

%%%%%%%%%%%%%%%%%%%%%%%%%%%%%%%%%%%%%%%%%%%%%%%%%%%%%%%%%%%%%%%%%%%%%%%%%%%%%%%%
\subsection{Flags}
\label{sec:flags}

The package makes it easy to generate different versions
of the main or child documents.
To this end compilation flags can be defined
and assigned different default values.
They will be particularly useful in conjunction
with the forwarding mechanism described in \secref{sec:forward}.

For example, it may be useful to have a flag |\version|
which can be set to |draft| or |final|.
The document source will contain some conditional code
depending on the value of |\version|.
Suppose further, the flag should default to |final| for the main file
and to |draft| for child files
which is a natural assignment for editing the document.
This is achieved by placing the following code
in the preamble of the main document
(below the |\childdocmain| directive):
%
\begin{center}
\begin{tabular}{l}
|\ifchilddoc|\\
|\providecommand{\version}{draft}|\\
|\||else|\\
|\providecommand{\version}{final}|\\
|\||fi|
\end{tabular}
\end{center}
%
The definition by |\providecommand| makes sure
that previous definitions are not overwritten.
Further statements |\providecommand{\version}{...}|
can thus be added before the above code to override it.

For the main file, one might add a line
(between |\childdocmain| and the above block)
%
\begin{center}
|%\ifchilddoc\||else\providecommand{\version}{draft}\||fi|
\end{center}
%
which can be uncommented to produce a draft version.
Likewise one can add a line to the very top of a child file
(above the |\childdocof{|\textit{main}|}| directive)
%
\begin{center}
|%\providecommand{\version}{final}|
\end{center}
%
which can be uncommented to produce the final version of this child document.

%%%%%%%%%%%%%%%%%%%%%%%%%%%%%%%%%%%%%%%%%%%%%%%%%%%%%%%%%%%%%%%%%%%%%%%%%%%%%%%%
\subsection{Forwarding}
\label{sec:forward}

Different versions of the main or child documents
using compilation flags as described in \secref{sec:flags}
can be (permanently) stored in different files
for convenient compilation, viewing and distribution.
To this end, the package defines a command
to pass on compilation to a different file:

%%%%%%%%%%%%%%%%%%%%%%%%%%%%%%%%%%%%%%%%
\DescribeMacro{\childdocforward}
The command |\childdocforward| redirects processing to
another source file:
%
\begin{center}
\begin{tabular}{l}
|\input{childdoc.def}|\\
|\childdocforward[|\textit{main}|]{|\textit{dest}|}|\\
\end{tabular}
\end{center}
%
The argument \textit{dest} is the destination file
(without extension).
It should be the main file or one of the child files.
Note that further \textsf{childdoc} directives
such as |\childdocof| and |\childdocforward|
in the indicated file will be processed in this form.
The optional argument \textit{main}
passes on directly to the main file \textit{main}
while pretending to compile the child \textit{dest}.
This form behaves as if \textit{dest}
issues |\childdocof{|\textit{main}|}| right away,
and no further \textsf{childdoc} directives will be processed.

%%%%%%%%%%%%%%%%%%%%%%%%%%%%%%%%%%%%%%%%
\DescribeMacro{\...prefix}
In the alternative form |\childdocforwardprefix|,
%
\begin{center}
\begin{tabular}{l}
|\input{childdoc.def}|\\
|\childdocforwardprefix[|\textit{main}|]{|\textit{prefix}|}{|\textit{dest}|}|
\end{tabular}
\end{center}
%
the destination file is determined by a pattern
depending on the current file:
To make this work, the current file must be called
`{\textit{prefix}\hspace{0.2em}\textit{suffix}}'
with \textit{prefix} matching precisely the argument.
Processing is then passed on to the file
`{\textit{dest}\hspace{0.2em}\textit{suffix}}'.
Surely, the same effect is achieved by
directly specifying the
argument `{\textit{dest}\hspace{0.2em}\textit{suffix}}'
in the first form.
However, that requires to set up a different file
for each child. With the alternative form of the command
all these files can have exactly the same content
which simplifies setting them up and maintaining them.

For example, the following file |draft.tex|
with a compilation flag |\version| as described in \secref{sec:flags}
compiles the main document as a draft:
%
\begin{center}
\begin{tabular}{l}
|\def\version{draft}|\\
|\input{childdoc.def}|\\
|\childdocforward{|\textit{main}|}|
\end{tabular}
\end{center}
%
Likewise, the following files |final|\textit{nn}|.tex|
compile the final version of the child document
|child|\textit{nn}|.tex|:
%
\begin{center}
\begin{tabular}{l}
|\def\version{final}|\\
|\input{childdoc.def}|\\
|\childdocforwardprefix{final}{child}|
\end{tabular}
\end{center}
%

Note that when several versions of a main file and/or of each child file
are to be generated, it may be convenient to set up a |Makefile| or
shell script to automatise the process.

%%%%%%%%%%%%%%%%%%%%%%%%%%%%%%%%%%%%%%%%%%%%%%%%%%%%%%%%%%%%%%%%%%%%%%%%%%%%%%%%
\subsection{Command Line Processing}
\label{sec:commandline}

The effect of redirection files can also be achieved by invoking
the \LaTeX{} compiler with a more elaborate command line.
Most conveniently this should be done as part
of a shell script or a |Makefile|.

When using \textsf{childdoc} in the main file, the following
command lines effectively perform a redirection
(note that depending on the shell being used,
backslashes may have to be doubled: `|\|' $\to$ `|\\|'):
%
\begin{center}
|... -jobname "|\textit{target}|" |\\|"|[\textit{flags}]%
|\input{childdoc.def}\childdocforward[|\textit{main}|]{|\textit{dest}|}"|
\end{center}
%
Here \textit{target} is the name of the output file,
\textit{main} is the name of the main file
and \textit{dest} is the name of the main or child file to be processed
(all filenames without extensions).
The optional argument \textit{main} can be omitted
if \textit{main} matches \textit{dest}.
Optionally, compilation \textit{flags} can be defined via |\def| commands.
This command line makes the \TeX{} engine believe
it is compiling the file \textit{target}
whose content is specified as the latter parameter.
The provided code then forwards the processing to
\textit{main} or \textit{dest} as described in \secref{sec:forward}.

%%%%%%%%%%%%%%%%%%%%%%%%%%%%%%%%%%%%%%%%%%%%%%%%%%%%%%%%%%%%%%%%%%%%%%%%%%%%%%%%
\subsection{Include by Input}
\label{sec:input}

Including child documents by |\include| has some restrictions by design.
Most notably, the content of a child document always occupies
its own set of pages; pages cannot be shared between child documents.
Usually, this behaviour makes perfect sense
because each child document contain an essential part of the document.
However, in some situations it may be desirable to compose
a document from a collection of parts
without having mandatory page breaks between then.
For this case, the package
provides a mechanism to include parts
by |\input| which can also be processed individually.
However, by construction this mechanism
requires manual handling of the content to be output.

%%%%%%%%%%%%%%%%%%%%%%%%%%%%%%%%%%%%%%%%
\DescribeMacro{\ifchilddocmanual}
The main file should be prepared as usual, see \secref{sec:include}.
However, the document body must make a distinction
between processing of an individual part and of the main document, e.g.:
%
\begin{center}
\begin{tabular}{l}
|\ifchilddocmanual|\\
|\input{\childdocname}|\\
|\||else|\\
\textit{document body with }|\input{|\textit{part}|}|\\
|\||fi|
\end{tabular}
\end{center}
%
The conditional |\ifchilddocmanual| is true whenever
a part to be included by |\input| is being compiled,
and the name of the part is stored in |\childdocname|.

%%%%%%%%%%%%%%%%%%%%%%%%%%%%%%%%%%%%%%%%
\DescribeMacro{\childdocby}
Each part to be included by |\input| should start with:
%
\begin{center}
\begin{tabular}{l}
|\input{childdoc.def}|\\
|\childdocby{|\textit{main}|}|\\
\end{tabular}
\end{center}
%
The directive |\childdocby| is similar to |\childdocof|
described in \secref{sec:include},
but the subsequent selection of content must be done manually.
To that end, both |\ifchilddoc| and |\ifchilddocmanual|
will be true upon processing of a part,
and the name of the part is stored in |\childdocname|.
Note that |\jobname| will be set to the filename of the current part
so that each part receives an individual |.aux| file
that does not interfere with the |.aux| file(s) of the main document.
This behaviour can be altered by the alternative form
|\childdocby[*]{|\textit{main}|}| (with a non-empty optional argument)
which uses the |.aux| file of the main document
by setting |\jobname| to \textit{main}.

%%%%%%%%%%%%%%%%%%%%%%%%%%%%%%%%%%%%%%%%%%%%%%%%%%%%%%%%%%%%%%%%%%%%%%%%%%%%%%%%
\subsection{Driver Development}
\label{sec:driver}

The \textsf{childdoc} mechanism can also be use for the development
of definition files such as \LaTeX{} styles or classes.
This case differs from the above setup with multiple parts
included by |\include| in that no |\includeonly| should be invoked.
This can be achieved by starting the include file
(before |\ProvidesPackage|) with:
%
\begin{center}
\begin{tabular}{l}
|\input{childdoc.def}|\\
|\childdocforward{|\textit{main}|}|\\
\end{tabular}
\end{center}
%
or alternatively with:
%
\begin{center}
\begin{tabular}{l}
|\input{childdoc.def}|\\
|\childdocby{|\textit{main}|}|\\
\end{tabular}
\end{center}
%
Both forms have slightly different effects as described above.
The main file is prepared as usual, see \secref{sec:include}.

%%%%%%%%%%%%%%%%%%%%%%%%%%%%%%%%%%%%%%%%%%%%%%%%%%%%%%%%%%%%%%%%%%%%%%%%%%%%%%%%
\subsection{Legacy Detection}
\label{sec:detection}

The directive |\childdocmain| in the main file can detect
whether the complete document or merely a child is to be compiled
even without using the directive |\childdocof|.
This method is deprecated because it is less robust
and there is no compelling reason to use it;
it is merely provided for backward compatibility
and it may be removed in future versions.

If the detection mechanism is to be used,
it is mandatory to correctly specify
the filename of the main file as the argument of |\childdocmain|:
%
\begin{center}
\begin{tabular}{l}
|\input{childdoc.def}|\\
|\childdocmain{|\textit{main}|}|\\
\end{tabular}
\end{center}
%
If |\jobname| does not match the argument \textit{main} of |\childdocmain|,
it is assumed that |\jobname| points to the child file to be compiled.
When using |\childdocmain| with the main file specified as argument,
it suffices to start a child file
with just |\input{|\textit{main}|}|
without loading of the package and using |\childdocof|.
If instead all processing is done
with the appropriate \textsf{childdoc} directives,
the argument of \textit{main} of |\childdocmain| can be empty.

An alternative version of the command line processing described
in \secref{sec:commandline} using the detection mechanism reads:
%
\begin{center}
|... -jobname "|\textit{target}|" "|[\textit{flags}]%
[|\def\jobname{|\textit{dest}|}|]|\input{|\textit{main}|}"|
\end{center}

%%%%%%%%%%%%%%%%%%%%%%%%%%%%%%%%%%%%%%%%%%%%%%%%%%%%%%%%%%%%%%%%%%%%%%%%%%%%%%%%
\subsection{Manual Code}
\label{sec:manual}

In case one cannot be certain whether the definitions file |childdoc.def|
is installed on the target \TeX{} distribution
and one prefers not to ship it,
it is conceivable to paste a few relevant commands into the sources.

To that end, drop all statements |\input{childdoc.def}|
and perform the replacements as outlined below.
Instead of |\childdocmain{|\textit{main}|}| add the following code
to the top of the main file:
%
\begin{center}
\begin{tabular}{l}
|\||ifdefined\childdocname\endinput\||fi\newif\ifchilddoc|\\
|\edef\childdocname{\scantokens\expandafter{\jobname\noexpand}}|\\
|\def\childdocmain{|\textit{main}|}\||ifx\childdocmain\childdocname\||else|\\
|\childdoctrue\includeonly{\childdocname}\let\jobname\childdocmain\||fi|\\
\end{tabular}
\end{center}
%
Instead of |\childdocof{|\textit{main}|}| just include the main file
at the top of each child file:
%
\begin{center}
|\input{|\textit{main}|}|
\end{center}
%
A simple redirection |\childdocforward{|\textit{dest}|}| is achieved by:
%
\begin{center}
|\def\jobname{|\textit{dest}|}\input{\jobname}|
\end{center}
%
The redirection with prefix
|\childdocforwardprefix[|\textit{prefix}|]{|\textit{dest}|}|
is accomplished by:
%
\begin{center}
\begin{tabular}{l}
|{\edef\jobname{\scantokens\expandafter{\jobname\noexpand}}|\\
|\def\redirectjob |\textit{prefix}|#1~~~{\gdef\jobname{|\textit{dest}|#1}}|\\
|\expandafter\redirectjob\jobname~~~}\input{\jobname}|
\end{tabular}
\end{center}

In an alternative approach,
child documents can be compiled by a specific command line
without additional code or specific definitions:
%
\begin{center}
|... -jobname "|\textit{target}|" "|[\textit{flags}]%
|\includeonly{|\textit{dest}|}\input{|\textit{main}|}"|
\end{center}
%

%%%%%%%%%%%%%%%%%%%%%%%%%%%%%%%%%%%%%%%%%%%%%%%%%%%%%%%%%%%%%%%%%%%%%%%%%%%%%%%%
%%%%%%%%%%%%%%%%%%%%%%%%%%%%%%%%%%%%%%%%%%%%%%%%%%%%%%%%%%%%%%%%%%%%%%%%%%%%%%%%
\section{Information}

%%%%%%%%%%%%%%%%%%%%%%%%%%%%%%%%%%%%%%%%%%%%%%%%%%%%%%%%%%%%%%%%%%%%%%%%%%%%%%%%
\subsection{Copyright}

Copyright \copyright{} 2017--2018 Niklas Beisert

This work may be distributed and/or modified under the
conditions of the \LaTeX{} Project Public License, either version 1.3
of this license or (at your option) any later version.
The latest version of this license is in
  \url{http://www.latex-project.org/lppl.txt}
and version 1.3 or later is part of all distributions of \LaTeX{}
version 2005/12/01 or later.

This work has the LPPL maintenance status `maintained'.

The Current Maintainer of this work is Niklas Beisert.

This work consists of the files |README.txt|, |childdoc.ins| and |childdoc.dtx|
as well as the derived files |childdoc.def|, |cdocsamp.tex|
with |cdocsch1.tex|, |cdocsch2.tex|, |cdocspt3.tex|, |cdocspt4.tex|,
|cdocsdrf.tex|, |cdocsfn1.tex|, |cdocsfn2.tex|
as well as |childdoc.pdf|.

%%%%%%%%%%%%%%%%%%%%%%%%%%%%%%%%%%%%%%%%%%%%%%%%%%%%%%%%%%%%%%%%%%%%%%%%%%%%%%%%
\subsection{Files and Installation}

The package consists of the files:
%
\begin{center}
\begin{tabular}{ll}
    |README.txt|   & readme file \\
    |childdoc.ins| & installation file \\
    |childdoc.dtx| & source file \\
    |childdoc.def| & definition file \\
    |cdocsamp.tex| & sample main file \\
    |cdocsch1.tex| & sample include file \\
    |cdocsch2.tex| & sample include file \\
    |cdocspt3.tex| & sample part file \\
    |cdocspt4.tex| & sample part file \\
    |cdocsdrf.tex| & sample redirection file \\
    |cdocsfn1.tex| & sample redirection file \\
    |cdocsfn2.tex| & sample redirection file \\
    |childdoc.pdf| & manual
\end{tabular}
\end{center}
%
The distribution consists of the files
|README.txt|, |childdoc.ins| and |childdoc.dtx|.
%
\begin{itemize}
\item
Run (pdf)\LaTeX{} on |childdoc.dtx|
to compile the manual |childdoc.pdf| (this file).
\item
Run \LaTeX{} on |childdoc.ins| to create the definitions file |childdoc.def|
and the sample |cdocsamp.tex| with include files
|cdocsch1.tex|, |cdocsch2.tex|, |cdocspt3.tex|, |cdocspt4.tex|,
|cdocsdrf.tex|, |cdocsfn1.tex|, |cdocsfn2.tex|.
Then copy the file |childdoc.def| to an appropriate directory of your \LaTeX{}
distribution, e.g.\ \textit{texmf-root}|/tex/latex/childdoc|.
\end{itemize}

%%%%%%%%%%%%%%%%%%%%%%%%%%%%%%%%%%%%%%%%%%%%%%%%%%%%%%%%%%%%%%%%%%%%%%%%%%%%%%%%
\subsection{Related CTAN Packages}

There are several other packages which offer a similar functionality:
%
\begin{itemize}
\item
The packages
\href{http://ctan.org/pkg/docmute}{\textsf{docmute}},
\href{http://ctan.org/pkg/includex}{\textsf{includex}} and
\href{http://ctan.org/pkg/standalone}{\textsf{standalone}}
provide commands to include only the document body of
a child file thus allowing both files to be compiled individually.
\item
The packages \href{http://ctan.org/pkg/subdocs}{\textsf{subdocs}}
and \href{http://ctan.org/pkg/subfiles}{\textsf{subfiles}}
provide structures in which the main and child documents can be
encapsulated and allowing them to be compiled individually.
The inclusion mechanism is different from the conventional |\include|.
\item
The package \href{http://ctan.org/pkg/combine}{\textsf{combine}}
is an elaborate solution to combine several documents into one.
\end{itemize}
%
See also the CTAN topic \href{http://ctan.org/topic/subdocs}{\textsf{subdocs}}
for further related packages.
The present package differs from the above solutions in that
a document structure constructed with the conventional |\include| mechanism
just needs two extra commands at the top of every file
such that all constituent files can be compiled individually.

%%%%%%%%%%%%%%%%%%%%%%%%%%%%%%%%%%%%%%%%%%%%%%%%%%%%%%%%%%%%%%%%%%%%%%%%%%%%%%%%
%\subsection{Feature Suggestions}
%
%The following is a list of features which may be useful for future
%versions of this package:
%%
%\begin{itemize}
%\item
%\ldots
%\end{itemize}

%%%%%%%%%%%%%%%%%%%%%%%%%%%%%%%%%%%%%%%%%%%%%%%%%%%%%%%%%%%%%%%%%%%%%%%%%%%%%%%%
\subsection{Revision History}

%%%%%%%%%%%%%%%%%%%%%%%%%%%%%%%%%%%%%%%%
\paragraph{v2.0:} 2018/12/30

\begin{itemize}
\item
immediate forward processing
\item
added |\childdocby| mechanism
\item
manual restructured
\end{itemize}

%%%%%%%%%%%%%%%%%%%%%%%%%%%%%%%%%%%%%%%%
\paragraph{v1.6:} 2018/01/17

\begin{itemize}
\item
application for development of include files
\item
corrections to manual
\end{itemize}

%%%%%%%%%%%%%%%%%%%%%%%%%%%%%%%%%%%%%%%%
\paragraph{v1.5:} 2017/05/21

\begin{itemize}
\item
more complete structuring introduced
\item
|\childdocof| introduced
\item
|\childdoc| renamed to |\childdocmain|
\item
|\childredirect| renamed to |\childdocforward| and |\childdocforwardprefix|
and functionality expanded
\end{itemize}

%%%%%%%%%%%%%%%%%%%%%%%%%%%%%%%%%%%%%%%%
\paragraph{v1.0:} 2017/04/27

\begin{itemize}
\item
manual and install package
\item
first version published on CTAN
\end{itemize}

%%%%%%%%%%%%%%%%%%%%%%%%%%%%%%%%%%%%%%%%
\paragraph{v0.6:} 2017/04/26

\begin{itemize}
\item
redirection mechanism added
\end{itemize}

%%%%%%%%%%%%%%%%%%%%%%%%%%%%%%%%%%%%%%%%
\paragraph{v0.5:} 2017/04/26

\begin{itemize}
\item
functionality in definition file
\end{itemize}


%%%%%%%%%%%%%%%%%%%%%%%%%%%%%%%%%%%%%%%%%%%%%%%%%%%%%%%%%%%%%%%%%%%%%%%%%%%%%%%%
%%%%%%%%%%%%%%%%%%%%%%%%%%%%%%%%%%%%%%%%%%%%%%%%%%%%%%%%%%%%%%%%%%%%%%%%%%%%%%%%
%%%%%%%%%%%%%%%%%%%%%%%%%%%%%%%%%%%%%%%%%%%%%%%%%%%%%%%%%%%%%%%%%%%%%%%%%%%%%%%%
\appendix

\settowidth\MacroIndent{\rmfamily\scriptsize 000\ }

 \DocInput{childdoc.dtx}

\end{document}
%</driver>
% \fi
%
% %%%%%%%%%%%%%%%%%%%%%%%%%%%%%%%%%%%%%%%%%%%%%%%%%%%%%%%%%%%%%%%%%%%%%%%%%%%%%%
% %%%%%%%%%%%%%%%%%%%%%%%%%%%%%%%%%%%%%%%%%%%%%%%%%%%%%%%%%%%%%%%%%%%%%%%%%%%%%%
% \section{Sample}
%\iffalse
%<*samplemain>
%\fi
%
% The following presents a sample document
% with two chapters, two parts, a title page,
% a compile flag as well as three forwarding files to set the flag.
% It consists of eight |.tex| files:
% \begin{center}
% \begin{tabular}{ll}
% |cdocsamp.tex|&main file\\
% |cdocsch1.tex|&include file for chapter 1\\
% |cdocsch2.tex|&include file for chapter 2\\
% |cdocspt3.tex|&include file for part 3\\
% |cdocspt4.tex|&include file for part 4\\
% |cdocsdrf.tex|&forwarding file for main file in draft mode\\
% |cdocsfi1.tex|&forwarding file for final version of chapter 1\\
% |cdocsfi2.tex|&forwarding file for final version of chapter 2\\
% \end{tabular}
% \end{center}
% Each of the eight files can be compiled directly by the \LaTeX{} compiler.
%
% %%%%%%%%%%%%%%%%%%%%%%%%%%%%%%%%%%%%%%
% \paragraph{Main File.}
%
% The main file is called |cdocsamp.tex|.
%
% Load the \textsf{childdoc} definitions and
% declare the filename for the main document:
%    \begin{macrocode}
\input{childdoc.def}
\childdocmain{}
%    \end{macrocode}

% Optional override for |\version| flag:
%    \begin{macrocode}
%%\ifchilddoc\else\providecommand{\version}{draft}\fi
%    \end{macrocode}

% Define the default values for the |\version| flag
% (|final| for the main file and |draft| for childs):
%    \begin{macrocode}
\ifchilddoc
\providecommand{\version}{draft}
\else
\providecommand{\version}{final}
\fi
%    \end{macrocode}

% Load the standard document class:
%    \begin{macrocode}
\documentclass[12pt]{article}
%    \end{macrocode}

% Start the document body:
%    \begin{macrocode}
\begin{document}
%    \end{macrocode}

% Declare a title page.
% Print title, part of document being processed and version flag:
%    \begin{macrocode}
\addtocounter{page}{-1}
\begin{center}
{\LARGE\bfseries{}childdoc example\par}
\vspace{1cm}
\ifchilddoc
\ifchilddocmanual part\else chapter\fi:
`\childdocname' of `\childdocjob'\par
\else
main document: `\childdocjob'\par
\fi
version: \version\par
\end{center}
\newpage
%    \end{macrocode}

% Manually include selected file,
% otherwise process as usual:
%    \begin{macrocode}
\ifchilddocmanual
\section*{part `\childdocname'}
\input{\childdocname}
\else
%    \end{macrocode}

% Include the two chapters:
%    \begin{macrocode}
\include{cdocsch1}
\include{cdocsch2}
%    \end{macrocode}

% Include the two parts unless only chapters should be displayed:
%    \begin{macrocode}
\ifchilddoc\else
\section{part three}
\input{cdocspt3}
\section{part four}
\input{cdocspt4}
\fi
%    \end{macrocode}

% Process as usual until here:
%    \begin{macrocode}
\fi
%    \end{macrocode}

% End of document body:
%    \begin{macrocode}
\end{document}
%    \end{macrocode}
%\iffalse
%</samplemain>
%\fi
%
% %%%%%%%%%%%%%%%%%%%%%%%%%%%%%%%%%%%%%%
% \paragraph{Chapter Include Files.}
%
% The include files are called |cdocsch1.tex| and |cdocsch2.tex|.
%
%\iffalse
%<*samplechap1|samplechap2>
%\fi

% Optional override for |\version| flag:
%    \begin{macrocode}
%%\providecommand{\version}{final}
%    \end{macrocode}

% Include the main document:
%    \begin{macrocode}
\input{childdoc.def}
\childdocof{cdocsamp}
%    \end{macrocode}

%\iffalse
%</samplechap1|samplechap2>
%\fi
%
%\iffalse
%<*samplechap1>
%\fi
% Some text for chapter 1:
%    \begin{macrocode}
\section{one}
some text in chapter one
%    \end{macrocode}

%\iffalse
%</samplechap1>
%\fi
% Some text for chapter 2:
%\iffalse
%<*samplechap2>
%\fi
%    \begin{macrocode}
\section{two}
more text in chapter two
%    \end{macrocode}

%\iffalse
%</samplechap2>
%\fi
%
% %%%%%%%%%%%%%%%%%%%%%%%%%%%%%%%%%%%%%%
% \paragraph{Part Include Files.}
%
% The include files are called |cdocspt3.tex| and |cdocspt4.tex|.
%
%\iffalse
%<*samplepart3|samplepart4>
%\fi

% Optional override for |\version| flag:
%    \begin{macrocode}
%%\providecommand{\version}{final}
%    \end{macrocode}

% Include the main document:
%    \begin{macrocode}
\input{childdoc.def}
\childdocby{cdocsamp}
%    \end{macrocode}

%\iffalse
%</samplepart3|samplepart4>
%\fi
%
%\iffalse
%<*samplepart3>
%\fi
% Some text for part 3:
%    \begin{macrocode}
some text in part three
%    \end{macrocode}

%\iffalse
%</samplepart3>
%\fi
% Some text for part 4:
%\iffalse
%<*samplepart4>
%\fi
%    \begin{macrocode}
more text in part four
%    \end{macrocode}

%\iffalse
%</samplepart4>
%\fi
%
% %%%%%%%%%%%%%%%%%%%%%%%%%%%%%%%%%%%%%%
% \paragraph{Forwarding for a Complete Draft.}
%
% The following forwarding file |cdocsdrf.tex|
% compiles the main document in draft mode:
%\iffalse
%<*sampledraft>
%\fi
%    \begin{macrocode}
\def\version{draft}
\input{childdoc.def}
\childdocforward{cdocsamp}
%    \end{macrocode}

%\iffalse
%</sampledraft>
%\fi
%
% %%%%%%%%%%%%%%%%%%%%%%%%%%%%%%%%%%%%%%
% \paragraph{Forwarding for Final Version of the Chapters.}
%
% The following forwarding files |cdocsfn1.tex| and |cdocsfn2.tex|
% (with identical content)
% compile the final versions of the child documents
% |cdocsch1.tex| and |cdocsch2.tex|, respectively:
%\iffalse
%<*samplefinal>
%\fi
%    \begin{macrocode}
\def\version{final}
\input{childdoc.def}
\childdocforwardprefix[cdocsamp]{cdocsfn}{cdocsch}
%    \end{macrocode}

%\iffalse
%</samplefinal>
%\fi
%
% %%%%%%%%%%%%%%%%%%%%%%%%%%%%%%%%%%%%%%
% \paragraph{Command Line Processing.}
%
% The following three command lines generate the output files
% |cdocscld|, |cdocscl1| and |cdocscl2|
% which should be identical to
% |cdocsdrf|, |cdocsch1| and |cdocsfn2|, respectively:
% \begin{center}
% \begin{tabular}{l}
% |latex -jobname cdocscld \|\\
% |  "\def\version{draft}\input{childdoc.def}\childdocforward{cdocsamp}"|\\
% |latex -jobname cdocscl1 \|\\
% |  "\input{childdoc.def}\childdocforward[cdocsamp]{cdocsch1}"|\\
% |latex -jobname cdocscl2 \|\\
% |  "\def\version{final}\input{childdoc.def}\childdocforward{cdocsch2}"|
% \end{tabular}
% \end{center}
% Note that the trailing backslash on each first line
% merely continues the input to the second line
% (for convenient cut ant paste).
% Furthermore, the command |latex| can be replaced by any
% of its alternative versions such as |pdflatex|.
%
% %%%%%%%%%%%%%%%%%%%%%%%%%%%%%%%%%%%%%%%%%%%%%%%%%%%%%%%%%%%%%%%%%%%%%%%%%%%%%%
% %%%%%%%%%%%%%%%%%%%%%%%%%%%%%%%%%%%%%%%%%%%%%%%%%%%%%%%%%%%%%%%%%%%%%%%%%%%%%%
% \section{Implementation}
%\iffalse
%<*package>
%\fi
%
% This section describes the definitions file |childdoc.def|.

% The definitions cannot be loaded using |\usepackage| or |\RequirePackage|
% which has a mechanism to prevent loading a style file more than once.
% When loading the definitions by means of |\input|
% multiple instances have to be prevented manually:
%\iffalse
%This code needs to be before the `\ProvidesFile' directive
%which is defined at the beginning of this file.
%Therefore it is also placed there and commented out here.
%</package>
%<*discard>
%\fi
%    \begin{macrocode}
\ifdefined\childdocmain\endinput\fi
%    \end{macrocode}
%\iffalse
%</discard>
%<*package>
%\fi
%
% \macro{\ifchilddoc}
% \macro{\ifchilddocmanual}
% The conditional |\ifchilddoc| tells whether a
% child (true) or main (false) document is being compiled.
% The conditional |\ifchilddocmanual| tells whether
% the |\includeonly| mechanism is used (false) or
% the selection of child files must be performed manually (true).
% The definitions initialise to false:
%    \begin{macrocode}
\newif\ifchilddoc
\newif\ifchilddocmanual
%    \end{macrocode}

% \macro{\childdocname}
% \macro{\childdocjob}
% The macro |\childdocname| stores the name of the main document
% to be compiled. The macro |\childdocjob| stores the name of
% the document on which the \LaTeX{} compiler was originally invoked.
% The content of |\jobname| cannot be compared
% to filenames specified in the source due to different catcodes.
% The following code rescans |\jobname|, stores the result
% in |\childdocname| and saves a copy in |\childdocjob|:
%    \begin{macrocode}
\edef\childdocname{\scantokens\expandafter{\jobname\noexpand}}
\let\childdocjob\childdocname
%    \end{macrocode}

% \macro{\childdocdisable}
% The macro |\childdocdisable| prevents the main file
% from being processed more than once.
% At this stage, the main document command |\childdocmain|
% is assumed to be called once again where it should do nothing.
% Any subsequent call to it should prevent
% a secondary processing of the main document
% It overwrites the forwarding commands
% |\childdocof| and |\childdocforward|
% with empty macros to prevent further inclusions of the main document:
%    \begin{macrocode}
\newcommand{\childdocdisable}
{
  \renewcommand{\childdocmain}[1]{\renewcommand{\childdocmain}[1]{\endinput}}
  \renewcommand{\childdocof}[1]{}
  \renewcommand{\childdocby}[2][]{}
  \renewcommand{\childdocforward}[2][]{}
  \renewcommand{\childdocdisable}{}
}
%    \end{macrocode}

% \macro{\childdocmain}
% The macro |\childdocmain| is to be called at the top of the main file
% with nothing or the main filename (without extension) as argument.
% First, it breaks loops.
% If the argument is not empty and does not match |\childdocname|
% (which is set by the first inclusion of |childdoc.def|),
% |\ifchilddoc| is set to true, |\includeonly| is applied to the child file
% and |\jobname| is set to the main file
% (for proper handling of |.aux| files):
%    \begin{macrocode}
\newcommand{\childdocmain}[1]
{
  \childdocdisable\childdocmain{}
  \if?#1?\else
    \begingroup
      \def\childdoctmp{#1}
      \ifx\childdoctmp\childdocname
        \def\childdoctmp{}
      \else
        \def\childdoctmp
        {
          \childdoctrue
          \includeonly{\childdocname}
          \def\childdocjob{#1}
          \def\jobname{#1}
        }
      \fi
      \expandafter
    \endgroup
    \childdoctmp
  \fi
}
%    \end{macrocode}

% \macro{\childdocof}
% The command |\childdocof| redirects
% compilation to the main file |#1|.
%    \begin{macrocode}
\newcommand{\childdocof}[1]
{
  \childdocdisable
  \childdoctrue
  \includeonly{\childdocname}
  \def\jobname{#1}
  \def\childdocjob{#1}
  \input{#1}
}
%    \end{macrocode}

% \macro{\childdocby}
% The command |\childdocby| ....
%    \begin{macrocode}
\newcommand{\childdocby}[2][]
{
  \childdocdisable
  \childdoctrue
  \childdocmanualtrue
  \if?#1?\else
    \def\jobname{#2}
  \fi
  \def\childdocjob{#2}
  \input{#2}
  \endinput
}
%    \end{macrocode}

% \macro{\childdocforward}
% The command |\childdocforward| redirects
% compilation to the main file or
% (if the optional argument is given) a child file.
% Parameters are set as if the main file
% or a child file starting with |\childdocof| was compiled.
% Then compilation is handed over to the main file:
%    \begin{macrocode}
\newcommand{\childdocforward}[2][]
{
  \begingroup
    \if?#1?
      \def\childdoctmp
      {
        \def\childdocname{#2}
        \def\childdocjob{#2}
        \def\jobname{#2}
        \input{#2}
        \endinput
      }
    \else
      \def\childdoctmp
      {
        \childdocdisable
        \def\childdocname{#2}
        \childdoctrue
        \includeonly{#2}
        \def\childdocjob{#1}
        \def\jobname{#1}
        \input{#1}
        \endinput
      }
    \fi
    \expandafter
  \endgroup
  \childdoctmp
}
%    \end{macrocode}

% \macro{\childdocforwardprefix}
% The command |\childdocforwardprefix| redirects
% compilation to the main or a child file by means of a pattern.
% The prefix |#1| in the current filename is replaced by |#2|
% and the suffix of the current filename is kept
% (it is assumed that the filename does not contain the substring `|~~~|'
% which is used as a delimiter).
% Compilation is handed over to the new file by |\childdocforward|:
%    \begin{macrocode}
\newcommand{\childdocforwardprefix}[3][]
{
  \begingroup
    \def\childdocextract #2##1~~~{\def\childdoctmp{\childdocforward[#1]{#3##1}}}
    \expandafter\childdocextract\childdocname~~~
    \expandafter
  \endgroup
  \childdoctmp
}
%    \end{macrocode}

% \macro{\childdoc}
% The deprecated macro |\childdoc| is a legacy version of |\childdocmain|:
%    \begin{macrocode}
\newcommand{\childdoc}{\childdocmain}
%    \end{macrocode}

% \macro{\childdocredirect}
% The deprecated macro |\childdocredirect| is a legacy version
% of |\childdocforward| and |\childdocforwardprefix|:
%    \begin{macrocode}
\newcommand{\childdocredirect}[2][]
{
  \begingroup
    \if?#1?
      \def\childdoctmp{\childdocforward{#2}}
    \else
      \def\childdoctmp{\childdocforwardprefix{#1}{#2}}
    \fi
    \expandafter
  \endgroup
  \childdoctmp
}
%    \end{macrocode}

%\iffalse
%</package>
%\fi
%
\endinput
\childdocforward{cdocsamp}"|\\
% |latex -jobname cdocscl1 \|\\
% |  "% \iffalse
%
% childdoc.dtx Copyright (C) 2017-2018 Niklas Beisert
%
% This work may be distributed and/or modified under the
% conditions of the LaTeX Project Public License, either version 1.3
% of this license or (at your option) any later version.
% The latest version of this license is in
%   http://www.latex-project.org/lppl.txt
% and version 1.3 or later is part of all distributions of LaTeX
% version 2005/12/01 or later.
%
% This work has the LPPL maintenance status `maintained'.
%
% The Current Maintainer of this work is Niklas Beisert.
%
% This work consists of the files childdoc.dtx and childdoc.ins
% and the derived files childdoc.def and cdocsamp.tex with
% cdocsch1.tex, cdocsch2.tex, cdocsdrf.tex, cdocsfn1.tex, cdocsfn2.tex.
%
%<package>\ifdefined\childdocmain\endinput\fi
%<package>\ProvidesFile{childdoc.def}[2018/12/30 v2.0 child document driver]
%<samplemain>\ProvidesFile{cdocsamp.tex}[2018/12/30 v2.0 sample for childdoc]
%<*driver>
%\ProvidesFile{childdoc.drv}[2018/12/30 v2.0 childdoc reference manual file]
\PassOptionsToClass{10pt,a4paper}{article}
\documentclass{ltxdoc}

\usepackage[margin=35mm]{geometry}
\usepackage{hyperref}
\usepackage{hyperxmp}
\usepackage[usenames]{color}

\hypersetup{colorlinks=true}
\hypersetup{pdfstartview=FitH}
\hypersetup{pdfpagemode=UseNone}
\hypersetup{pdfsource={}}
\hypersetup{pdflang={en-UK}}
\hypersetup{pdfcopyright={Copyright 2017-2018 Niklas Beisert.
  This work may be distributed and/or modified under the
  conditions of the LaTeX Project Public License, either version 1.3
  of this license or (at your option) any later version.}}
\hypersetup{pdflicenseurl={http://www.latex-project.org/lppl.txt}}
\hypersetup{pdfcontactaddress={ETH Zurich, ITP, HIT K,
  Wolfgang-Pauli-Strasse 27}}
\hypersetup{pdfcontactpostcode={8093}}
\hypersetup{pdfcontactcity={Zurich}}
\hypersetup{pdfcontactcountry={Switzerland}}
\hypersetup{pdfcontactemail={nbeisert@itp.phys.ethz.ch}}
\hypersetup{pdfcontacturl={http://people.phys.ethz.ch/\xmptilde nbeisert/}}

\newcommand{\secref}[1]{\hyperref[#1]{section \ref*{#1}}}

\parskip1ex
\parindent0pt
\let\olditemize\itemize
\def\itemize{\olditemize\parskip0pt}

\begin{document}

\title{The \textsf{childdoc} Package}
\hypersetup{pdftitle={The childdoc Package}}
\author{Niklas Beisert\\[2ex]
  Institut f\"ur Theoretische Physik\\
  Eidgen\"ossische Technische Hochschule Z\"urich\\
  Wolfgang-Pauli-Strasse 27, 8093 Z\"urich, Switzerland\\[1ex]
  \href{mailto:nbeisert@itp.phys.ethz.ch}
  {\texttt{nbeisert@itp.phys.ethz.ch}}}
\hypersetup{pdfauthor={Niklas Beisert}}
\hypersetup{pdfsubject={Manual for the LaTeX2e Package childdoc}}
\date{30 December 2018, \textsf{v2.0}}
\maketitle

\begin{abstract}\noindent
\textsf{childdoc} is a \LaTeXe{} package
that enables the direct compilation
of document sections included by |\include|
to individual files.
\end{abstract}

\begingroup
\parskip0ex
\tableofcontents
\endgroup

%%%%%%%%%%%%%%%%%%%%%%%%%%%%%%%%%%%%%%%%%%%%%%%%%%%%%%%%%%%%%%%%%%%%%%%%%%%%%%%%
%%%%%%%%%%%%%%%%%%%%%%%%%%%%%%%%%%%%%%%%%%%%%%%%%%%%%%%%%%%%%%%%%%%%%%%%%%%%%%%%
\section{Introduction}

\LaTeX{} provides a mechanism to structure a large document (such as a book)
into a main file and several child files (containing the chapters)
using the |\include| command.
This mechanism is beneficial for documents
which span hundreds of pages in order to
make the source file(s) more manageable.
Moreover, compilation can be restricted to
selected child files by means of the |\includeonly| command.
The latter feature can be used to reduce the compilation time while editing
(this was significantly more useful in the earlier days of \LaTeX{})
or to generate a smaller document which is easier to navigate.
Another application of |\includeonly| is to generate
documents consisting of selected parts of the complete document.

However, there are a few drawbacks of the plain |\include| mechanism:
\begin{itemize}
\item
The child files cannot be compiled on their own,
they can only be compiled via the main file.
A naive editing environment
(such as a text editor with an option
to have the current file processed by \LaTeX)
may require one to switch to the main file before compiling;
attempting to compile the child file produces errors.
\item
The main file must be modified (each time)
to adjust the |\includeonly| command
to the present needs. This easily leaves the main file in a messy state.
\item
The generated document will always carry the filename
of the main document. This is inconvenient if
several child files are to be compiled and
to be kept for distribution.
\end{itemize}

The present package provides a simple interface
to make child files individually compilable by \LaTeX{}.
Compiling a child file then has the same effect as compiling
the main file with an |\includeonly| command
to select the appropriate child.
Moreover the generated document will carry the name of the child
rather than the main file.
This resolves all three above issues.

This feature is meant to make the editing of books,
thesis documents and lecture notes somewhat more convenient.
However, the package can also be used efficiently for
composing a series of documents (such as exercise sheets)
which are typically distributed individually.
It then assists the author in generating the individual documents
(potentially in different versions)
as well as a document containing the collected series.
Another application is in developing style files
or other kinds of included material
where compilation of the style file could redirect
to a sample or test file.

%%%%%%%%%%%%%%%%%%%%%%%%%%%%%%%%%%%%%%%%%%%%%%%%%%%%%%%%%%%%%%%%%%%%%%%%%%%%%%%%
%%%%%%%%%%%%%%%%%%%%%%%%%%%%%%%%%%%%%%%%%%%%%%%%%%%%%%%%%%%%%%%%%%%%%%%%%%%%%%%%
\section{Usage}

First of all, the package \textsf{childdoc} is \emph{not} a standard
\LaTeXe{} |.sty| style file! Therefore it needs to be invoked in
a non-standard way.

%%%%%%%%%%%%%%%%%%%%%%%%%%%%%%%%%%%%%%%%%%%%%%%%%%%%%%%%%%%%%%%%%%%%%%%%%%%%%%%%
\subsection{Included Files}
\label{sec:include}

%%%%%%%%%%%%%%%%%%%%%%%%%%%%%%%%%%%%%%%%
\DescribeMacro{\childdocmain}
To use the package, add the commands
\begin{center}
\begin{tabular}{l}
|\input{childdoc.def}|\\
|\childdocmain{}|\\
\end{tabular}
\end{center}
at the very top of the main \LaTeX{} file,
in particular \emph{before} the |\documentclass| statement!
The argument of |\childdocmain| should be left empty
(but it must be present).

%%%%%%%%%%%%%%%%%%%%%%%%%%%%%%%%%%%%%%%%
\DescribeMacro{\childdocof}
Furthermore, add the commands
\begin{center}
\begin{tabular}{l}
|\input{childdoc.def}|\\
|\childdocof{|\textit{main}|}|\\
\end{tabular}
\end{center}
at the top of every child file \textit{child}
which is included by |\include{|\textit{child}|}|
from within the main file
(or at least for those files to be compiled individually).
The argument \textit{main} must be the filename of the main file.

There are a couple of
considerations in setting up the main and child documents:

%%%%%%%%%%%%%%%%%%%%%%%%%%%%%%%%%%%%%%%%
\paragraph{Restrictions.}

Please note the following restrictions:
\begin{itemize}
\item
|\childdocmain| must be called with one argument \textit{main}
to ensure compatibility with earlier version of the package.
It must either be empty (|\childdocmain{}|)
or precisely match the filename of the main file in which it is specified.
See \secref{sec:detection} for further information.
\item
The filename \textit{main} must be specified without the |.tex| extension.
\item
The filename \textit{main} is case sensitive
(even in case-insensitive file systems)
due to internal string comparison.
\item
The argument \textit{main} should be fully expanded, it cannot be a macro.
\item
Subdirectories and special characters should be avoided in filenames.
\item
The command |\childdocmain{|\textit{main}|}| must be followed by a whitespace.
It should not be followed immediately by another command
or by a comment mark `|%|'.
This is because the \TeX{} parser reads the token immediately following
the argument of |\childdocmain| and puts it
at the beginning of every child section;
however, a white\-space is ignored.
\end{itemize}

%%%%%%%%%%%%%%%%%%%%%%%%%%%%%%%%%%%%%%%%
\paragraph{Content of Main File.}

It is advisable to place all content in the child files included by |\include|.
Any output contained in the main file will appear in all child documents
unless suppressed manually;
it cannot be suppressed automatically by the |\includeonly| directive
and thus should normally be avoided.
A method to include some content in the main file
by means of conditional processing is described in \secref{sec:conditional}.

%%%%%%%%%%%%%%%%%%%%%%%%%%%%%%%%%%%%%%%%
\paragraph{Page Numbering.}

When only a part of the document is compiled,
the appropriate numbering of pages
(as well as other status parameters)
is determined from the |.aux| files.
The latter contain information from previous passes.
However this information needs to propagate through
all intermediate child documents.
Therefore the page numbering in child documents may well
be inconsistent until the complete document is compiled at least once.

A useful (if unconventional) way to always ensure a consistent
page numbering is to restart the numbering in each child document
and denote the pages by `\textit{child}|.|\textit{page}'
where \textit{child} represents the chapter/section number of the child file.
This can be achieved by the command
|\numberwithin{page}{|\textit{child}|}|
of the \textsf{amsmath} package
where \textit{child} can be |chapter| or |section|
depending on the chosen structuring.
Alternatively, one can modify the macro |\thepage| appropriately
and reset the counter |page| at the start of each child file.

%%%%%%%%%%%%%%%%%%%%%%%%%%%%%%%%%%%%%%%%%%%%%%%%%%%%%%%%%%%%%%%%%%%%%%%%%%%%%%%%
\subsection{Conditional Processing}
\label{sec:conditional}

The package provides a mechanism to compile different versions
of a document. To customise the versions further some conditional processing
can come in handy to distinguish which version is being compiled.
The package provides two macros to describe the compilation context:

%%%%%%%%%%%%%%%%%%%%%%%%%%%%%%%%%%%%%%%%
\DescribeMacro{\ifchilddoc}
The conditional |\ifchilddoc| distinguishes between the compilation of
child documents and the main document:
%
\begin{center}
|\ifchilddoc |\textit{child-code}| |[|\||else |\textit{main-code}]| \||fi|
\end{center}

%%%%%%%%%%%%%%%%%%%%%%%%%%%%%%%%%%%%%%%%
\DescribeMacro{\childdocname}
\DescribeMacro{\childdocjob}
The macro |\childdocname| contains the filename (without extension)
of the main or child file being processed.
Note that |\childdocjob| will always contain the name of the main file.

%%%%%%%%%%%%%%%%%%%%%%%%%%%%%%%%%%%%%%%%
\paragraph{Title Page.}

Conditional processing can be used to include a title or banner page
in the main document when proper precautions are taken.
Importantly, the code in the main file should ensure that the page counter
(as well as other status parameters which are stored in the |.aux| files)
takes the same value after the conditional processing.
Otherwise the page numbers may take divergent values
depending on which part is compiled.

For example, a title page could be declared by:
%
\begin{center}
\begin{tabular}{l}
|\ifchilddoc\||else|\\
|\addtocounter{page}{-1}|\\
\textit{code for title page}\\
|\newpage|\\
|\||fi|
\end{tabular}
\end{center}
%
A banner page for the child documents can be generated by:
%
\begin{center}
\begin{tabular}{l}
|\ifchilddoc|\\
|\addtocounter{page}{-1}|\\
\textit{code for banner page}\\
|\newpage|\\
|\||fi|
\end{tabular}
\end{center}
%
Here one could write a message such as:
\begin{center}
|This is the part \childdocname{} of \childdocjob{}.|
\end{center}

%%%%%%%%%%%%%%%%%%%%%%%%%%%%%%%%%%%%%%%%%%%%%%%%%%%%%%%%%%%%%%%%%%%%%%%%%%%%%%%%
\subsection{Flags}
\label{sec:flags}

The package makes it easy to generate different versions
of the main or child documents.
To this end compilation flags can be defined
and assigned different default values.
They will be particularly useful in conjunction
with the forwarding mechanism described in \secref{sec:forward}.

For example, it may be useful to have a flag |\version|
which can be set to |draft| or |final|.
The document source will contain some conditional code
depending on the value of |\version|.
Suppose further, the flag should default to |final| for the main file
and to |draft| for child files
which is a natural assignment for editing the document.
This is achieved by placing the following code
in the preamble of the main document
(below the |\childdocmain| directive):
%
\begin{center}
\begin{tabular}{l}
|\ifchilddoc|\\
|\providecommand{\version}{draft}|\\
|\||else|\\
|\providecommand{\version}{final}|\\
|\||fi|
\end{tabular}
\end{center}
%
The definition by |\providecommand| makes sure
that previous definitions are not overwritten.
Further statements |\providecommand{\version}{...}|
can thus be added before the above code to override it.

For the main file, one might add a line
(between |\childdocmain| and the above block)
%
\begin{center}
|%\ifchilddoc\||else\providecommand{\version}{draft}\||fi|
\end{center}
%
which can be uncommented to produce a draft version.
Likewise one can add a line to the very top of a child file
(above the |\childdocof{|\textit{main}|}| directive)
%
\begin{center}
|%\providecommand{\version}{final}|
\end{center}
%
which can be uncommented to produce the final version of this child document.

%%%%%%%%%%%%%%%%%%%%%%%%%%%%%%%%%%%%%%%%%%%%%%%%%%%%%%%%%%%%%%%%%%%%%%%%%%%%%%%%
\subsection{Forwarding}
\label{sec:forward}

Different versions of the main or child documents
using compilation flags as described in \secref{sec:flags}
can be (permanently) stored in different files
for convenient compilation, viewing and distribution.
To this end, the package defines a command
to pass on compilation to a different file:

%%%%%%%%%%%%%%%%%%%%%%%%%%%%%%%%%%%%%%%%
\DescribeMacro{\childdocforward}
The command |\childdocforward| redirects processing to
another source file:
%
\begin{center}
\begin{tabular}{l}
|\input{childdoc.def}|\\
|\childdocforward[|\textit{main}|]{|\textit{dest}|}|\\
\end{tabular}
\end{center}
%
The argument \textit{dest} is the destination file
(without extension).
It should be the main file or one of the child files.
Note that further \textsf{childdoc} directives
such as |\childdocof| and |\childdocforward|
in the indicated file will be processed in this form.
The optional argument \textit{main}
passes on directly to the main file \textit{main}
while pretending to compile the child \textit{dest}.
This form behaves as if \textit{dest}
issues |\childdocof{|\textit{main}|}| right away,
and no further \textsf{childdoc} directives will be processed.

%%%%%%%%%%%%%%%%%%%%%%%%%%%%%%%%%%%%%%%%
\DescribeMacro{\...prefix}
In the alternative form |\childdocforwardprefix|,
%
\begin{center}
\begin{tabular}{l}
|\input{childdoc.def}|\\
|\childdocforwardprefix[|\textit{main}|]{|\textit{prefix}|}{|\textit{dest}|}|
\end{tabular}
\end{center}
%
the destination file is determined by a pattern
depending on the current file:
To make this work, the current file must be called
`{\textit{prefix}\hspace{0.2em}\textit{suffix}}'
with \textit{prefix} matching precisely the argument.
Processing is then passed on to the file
`{\textit{dest}\hspace{0.2em}\textit{suffix}}'.
Surely, the same effect is achieved by
directly specifying the
argument `{\textit{dest}\hspace{0.2em}\textit{suffix}}'
in the first form.
However, that requires to set up a different file
for each child. With the alternative form of the command
all these files can have exactly the same content
which simplifies setting them up and maintaining them.

For example, the following file |draft.tex|
with a compilation flag |\version| as described in \secref{sec:flags}
compiles the main document as a draft:
%
\begin{center}
\begin{tabular}{l}
|\def\version{draft}|\\
|\input{childdoc.def}|\\
|\childdocforward{|\textit{main}|}|
\end{tabular}
\end{center}
%
Likewise, the following files |final|\textit{nn}|.tex|
compile the final version of the child document
|child|\textit{nn}|.tex|:
%
\begin{center}
\begin{tabular}{l}
|\def\version{final}|\\
|\input{childdoc.def}|\\
|\childdocforwardprefix{final}{child}|
\end{tabular}
\end{center}
%

Note that when several versions of a main file and/or of each child file
are to be generated, it may be convenient to set up a |Makefile| or
shell script to automatise the process.

%%%%%%%%%%%%%%%%%%%%%%%%%%%%%%%%%%%%%%%%%%%%%%%%%%%%%%%%%%%%%%%%%%%%%%%%%%%%%%%%
\subsection{Command Line Processing}
\label{sec:commandline}

The effect of redirection files can also be achieved by invoking
the \LaTeX{} compiler with a more elaborate command line.
Most conveniently this should be done as part
of a shell script or a |Makefile|.

When using \textsf{childdoc} in the main file, the following
command lines effectively perform a redirection
(note that depending on the shell being used,
backslashes may have to be doubled: `|\|' $\to$ `|\\|'):
%
\begin{center}
|... -jobname "|\textit{target}|" |\\|"|[\textit{flags}]%
|\input{childdoc.def}\childdocforward[|\textit{main}|]{|\textit{dest}|}"|
\end{center}
%
Here \textit{target} is the name of the output file,
\textit{main} is the name of the main file
and \textit{dest} is the name of the main or child file to be processed
(all filenames without extensions).
The optional argument \textit{main} can be omitted
if \textit{main} matches \textit{dest}.
Optionally, compilation \textit{flags} can be defined via |\def| commands.
This command line makes the \TeX{} engine believe
it is compiling the file \textit{target}
whose content is specified as the latter parameter.
The provided code then forwards the processing to
\textit{main} or \textit{dest} as described in \secref{sec:forward}.

%%%%%%%%%%%%%%%%%%%%%%%%%%%%%%%%%%%%%%%%%%%%%%%%%%%%%%%%%%%%%%%%%%%%%%%%%%%%%%%%
\subsection{Include by Input}
\label{sec:input}

Including child documents by |\include| has some restrictions by design.
Most notably, the content of a child document always occupies
its own set of pages; pages cannot be shared between child documents.
Usually, this behaviour makes perfect sense
because each child document contain an essential part of the document.
However, in some situations it may be desirable to compose
a document from a collection of parts
without having mandatory page breaks between then.
For this case, the package
provides a mechanism to include parts
by |\input| which can also be processed individually.
However, by construction this mechanism
requires manual handling of the content to be output.

%%%%%%%%%%%%%%%%%%%%%%%%%%%%%%%%%%%%%%%%
\DescribeMacro{\ifchilddocmanual}
The main file should be prepared as usual, see \secref{sec:include}.
However, the document body must make a distinction
between processing of an individual part and of the main document, e.g.:
%
\begin{center}
\begin{tabular}{l}
|\ifchilddocmanual|\\
|\input{\childdocname}|\\
|\||else|\\
\textit{document body with }|\input{|\textit{part}|}|\\
|\||fi|
\end{tabular}
\end{center}
%
The conditional |\ifchilddocmanual| is true whenever
a part to be included by |\input| is being compiled,
and the name of the part is stored in |\childdocname|.

%%%%%%%%%%%%%%%%%%%%%%%%%%%%%%%%%%%%%%%%
\DescribeMacro{\childdocby}
Each part to be included by |\input| should start with:
%
\begin{center}
\begin{tabular}{l}
|\input{childdoc.def}|\\
|\childdocby{|\textit{main}|}|\\
\end{tabular}
\end{center}
%
The directive |\childdocby| is similar to |\childdocof|
described in \secref{sec:include},
but the subsequent selection of content must be done manually.
To that end, both |\ifchilddoc| and |\ifchilddocmanual|
will be true upon processing of a part,
and the name of the part is stored in |\childdocname|.
Note that |\jobname| will be set to the filename of the current part
so that each part receives an individual |.aux| file
that does not interfere with the |.aux| file(s) of the main document.
This behaviour can be altered by the alternative form
|\childdocby[*]{|\textit{main}|}| (with a non-empty optional argument)
which uses the |.aux| file of the main document
by setting |\jobname| to \textit{main}.

%%%%%%%%%%%%%%%%%%%%%%%%%%%%%%%%%%%%%%%%%%%%%%%%%%%%%%%%%%%%%%%%%%%%%%%%%%%%%%%%
\subsection{Driver Development}
\label{sec:driver}

The \textsf{childdoc} mechanism can also be use for the development
of definition files such as \LaTeX{} styles or classes.
This case differs from the above setup with multiple parts
included by |\include| in that no |\includeonly| should be invoked.
This can be achieved by starting the include file
(before |\ProvidesPackage|) with:
%
\begin{center}
\begin{tabular}{l}
|\input{childdoc.def}|\\
|\childdocforward{|\textit{main}|}|\\
\end{tabular}
\end{center}
%
or alternatively with:
%
\begin{center}
\begin{tabular}{l}
|\input{childdoc.def}|\\
|\childdocby{|\textit{main}|}|\\
\end{tabular}
\end{center}
%
Both forms have slightly different effects as described above.
The main file is prepared as usual, see \secref{sec:include}.

%%%%%%%%%%%%%%%%%%%%%%%%%%%%%%%%%%%%%%%%%%%%%%%%%%%%%%%%%%%%%%%%%%%%%%%%%%%%%%%%
\subsection{Legacy Detection}
\label{sec:detection}

The directive |\childdocmain| in the main file can detect
whether the complete document or merely a child is to be compiled
even without using the directive |\childdocof|.
This method is deprecated because it is less robust
and there is no compelling reason to use it;
it is merely provided for backward compatibility
and it may be removed in future versions.

If the detection mechanism is to be used,
it is mandatory to correctly specify
the filename of the main file as the argument of |\childdocmain|:
%
\begin{center}
\begin{tabular}{l}
|\input{childdoc.def}|\\
|\childdocmain{|\textit{main}|}|\\
\end{tabular}
\end{center}
%
If |\jobname| does not match the argument \textit{main} of |\childdocmain|,
it is assumed that |\jobname| points to the child file to be compiled.
When using |\childdocmain| with the main file specified as argument,
it suffices to start a child file
with just |\input{|\textit{main}|}|
without loading of the package and using |\childdocof|.
If instead all processing is done
with the appropriate \textsf{childdoc} directives,
the argument of \textit{main} of |\childdocmain| can be empty.

An alternative version of the command line processing described
in \secref{sec:commandline} using the detection mechanism reads:
%
\begin{center}
|... -jobname "|\textit{target}|" "|[\textit{flags}]%
[|\def\jobname{|\textit{dest}|}|]|\input{|\textit{main}|}"|
\end{center}

%%%%%%%%%%%%%%%%%%%%%%%%%%%%%%%%%%%%%%%%%%%%%%%%%%%%%%%%%%%%%%%%%%%%%%%%%%%%%%%%
\subsection{Manual Code}
\label{sec:manual}

In case one cannot be certain whether the definitions file |childdoc.def|
is installed on the target \TeX{} distribution
and one prefers not to ship it,
it is conceivable to paste a few relevant commands into the sources.

To that end, drop all statements |\input{childdoc.def}|
and perform the replacements as outlined below.
Instead of |\childdocmain{|\textit{main}|}| add the following code
to the top of the main file:
%
\begin{center}
\begin{tabular}{l}
|\||ifdefined\childdocname\endinput\||fi\newif\ifchilddoc|\\
|\edef\childdocname{\scantokens\expandafter{\jobname\noexpand}}|\\
|\def\childdocmain{|\textit{main}|}\||ifx\childdocmain\childdocname\||else|\\
|\childdoctrue\includeonly{\childdocname}\let\jobname\childdocmain\||fi|\\
\end{tabular}
\end{center}
%
Instead of |\childdocof{|\textit{main}|}| just include the main file
at the top of each child file:
%
\begin{center}
|\input{|\textit{main}|}|
\end{center}
%
A simple redirection |\childdocforward{|\textit{dest}|}| is achieved by:
%
\begin{center}
|\def\jobname{|\textit{dest}|}\input{\jobname}|
\end{center}
%
The redirection with prefix
|\childdocforwardprefix[|\textit{prefix}|]{|\textit{dest}|}|
is accomplished by:
%
\begin{center}
\begin{tabular}{l}
|{\edef\jobname{\scantokens\expandafter{\jobname\noexpand}}|\\
|\def\redirectjob |\textit{prefix}|#1~~~{\gdef\jobname{|\textit{dest}|#1}}|\\
|\expandafter\redirectjob\jobname~~~}\input{\jobname}|
\end{tabular}
\end{center}

In an alternative approach,
child documents can be compiled by a specific command line
without additional code or specific definitions:
%
\begin{center}
|... -jobname "|\textit{target}|" "|[\textit{flags}]%
|\includeonly{|\textit{dest}|}\input{|\textit{main}|}"|
\end{center}
%

%%%%%%%%%%%%%%%%%%%%%%%%%%%%%%%%%%%%%%%%%%%%%%%%%%%%%%%%%%%%%%%%%%%%%%%%%%%%%%%%
%%%%%%%%%%%%%%%%%%%%%%%%%%%%%%%%%%%%%%%%%%%%%%%%%%%%%%%%%%%%%%%%%%%%%%%%%%%%%%%%
\section{Information}

%%%%%%%%%%%%%%%%%%%%%%%%%%%%%%%%%%%%%%%%%%%%%%%%%%%%%%%%%%%%%%%%%%%%%%%%%%%%%%%%
\subsection{Copyright}

Copyright \copyright{} 2017--2018 Niklas Beisert

This work may be distributed and/or modified under the
conditions of the \LaTeX{} Project Public License, either version 1.3
of this license or (at your option) any later version.
The latest version of this license is in
  \url{http://www.latex-project.org/lppl.txt}
and version 1.3 or later is part of all distributions of \LaTeX{}
version 2005/12/01 or later.

This work has the LPPL maintenance status `maintained'.

The Current Maintainer of this work is Niklas Beisert.

This work consists of the files |README.txt|, |childdoc.ins| and |childdoc.dtx|
as well as the derived files |childdoc.def|, |cdocsamp.tex|
with |cdocsch1.tex|, |cdocsch2.tex|, |cdocspt3.tex|, |cdocspt4.tex|,
|cdocsdrf.tex|, |cdocsfn1.tex|, |cdocsfn2.tex|
as well as |childdoc.pdf|.

%%%%%%%%%%%%%%%%%%%%%%%%%%%%%%%%%%%%%%%%%%%%%%%%%%%%%%%%%%%%%%%%%%%%%%%%%%%%%%%%
\subsection{Files and Installation}

The package consists of the files:
%
\begin{center}
\begin{tabular}{ll}
    |README.txt|   & readme file \\
    |childdoc.ins| & installation file \\
    |childdoc.dtx| & source file \\
    |childdoc.def| & definition file \\
    |cdocsamp.tex| & sample main file \\
    |cdocsch1.tex| & sample include file \\
    |cdocsch2.tex| & sample include file \\
    |cdocspt3.tex| & sample part file \\
    |cdocspt4.tex| & sample part file \\
    |cdocsdrf.tex| & sample redirection file \\
    |cdocsfn1.tex| & sample redirection file \\
    |cdocsfn2.tex| & sample redirection file \\
    |childdoc.pdf| & manual
\end{tabular}
\end{center}
%
The distribution consists of the files
|README.txt|, |childdoc.ins| and |childdoc.dtx|.
%
\begin{itemize}
\item
Run (pdf)\LaTeX{} on |childdoc.dtx|
to compile the manual |childdoc.pdf| (this file).
\item
Run \LaTeX{} on |childdoc.ins| to create the definitions file |childdoc.def|
and the sample |cdocsamp.tex| with include files
|cdocsch1.tex|, |cdocsch2.tex|, |cdocspt3.tex|, |cdocspt4.tex|,
|cdocsdrf.tex|, |cdocsfn1.tex|, |cdocsfn2.tex|.
Then copy the file |childdoc.def| to an appropriate directory of your \LaTeX{}
distribution, e.g.\ \textit{texmf-root}|/tex/latex/childdoc|.
\end{itemize}

%%%%%%%%%%%%%%%%%%%%%%%%%%%%%%%%%%%%%%%%%%%%%%%%%%%%%%%%%%%%%%%%%%%%%%%%%%%%%%%%
\subsection{Related CTAN Packages}

There are several other packages which offer a similar functionality:
%
\begin{itemize}
\item
The packages
\href{http://ctan.org/pkg/docmute}{\textsf{docmute}},
\href{http://ctan.org/pkg/includex}{\textsf{includex}} and
\href{http://ctan.org/pkg/standalone}{\textsf{standalone}}
provide commands to include only the document body of
a child file thus allowing both files to be compiled individually.
\item
The packages \href{http://ctan.org/pkg/subdocs}{\textsf{subdocs}}
and \href{http://ctan.org/pkg/subfiles}{\textsf{subfiles}}
provide structures in which the main and child documents can be
encapsulated and allowing them to be compiled individually.
The inclusion mechanism is different from the conventional |\include|.
\item
The package \href{http://ctan.org/pkg/combine}{\textsf{combine}}
is an elaborate solution to combine several documents into one.
\end{itemize}
%
See also the CTAN topic \href{http://ctan.org/topic/subdocs}{\textsf{subdocs}}
for further related packages.
The present package differs from the above solutions in that
a document structure constructed with the conventional |\include| mechanism
just needs two extra commands at the top of every file
such that all constituent files can be compiled individually.

%%%%%%%%%%%%%%%%%%%%%%%%%%%%%%%%%%%%%%%%%%%%%%%%%%%%%%%%%%%%%%%%%%%%%%%%%%%%%%%%
%\subsection{Feature Suggestions}
%
%The following is a list of features which may be useful for future
%versions of this package:
%%
%\begin{itemize}
%\item
%\ldots
%\end{itemize}

%%%%%%%%%%%%%%%%%%%%%%%%%%%%%%%%%%%%%%%%%%%%%%%%%%%%%%%%%%%%%%%%%%%%%%%%%%%%%%%%
\subsection{Revision History}

%%%%%%%%%%%%%%%%%%%%%%%%%%%%%%%%%%%%%%%%
\paragraph{v2.0:} 2018/12/30

\begin{itemize}
\item
immediate forward processing
\item
added |\childdocby| mechanism
\item
manual restructured
\end{itemize}

%%%%%%%%%%%%%%%%%%%%%%%%%%%%%%%%%%%%%%%%
\paragraph{v1.6:} 2018/01/17

\begin{itemize}
\item
application for development of include files
\item
corrections to manual
\end{itemize}

%%%%%%%%%%%%%%%%%%%%%%%%%%%%%%%%%%%%%%%%
\paragraph{v1.5:} 2017/05/21

\begin{itemize}
\item
more complete structuring introduced
\item
|\childdocof| introduced
\item
|\childdoc| renamed to |\childdocmain|
\item
|\childredirect| renamed to |\childdocforward| and |\childdocforwardprefix|
and functionality expanded
\end{itemize}

%%%%%%%%%%%%%%%%%%%%%%%%%%%%%%%%%%%%%%%%
\paragraph{v1.0:} 2017/04/27

\begin{itemize}
\item
manual and install package
\item
first version published on CTAN
\end{itemize}

%%%%%%%%%%%%%%%%%%%%%%%%%%%%%%%%%%%%%%%%
\paragraph{v0.6:} 2017/04/26

\begin{itemize}
\item
redirection mechanism added
\end{itemize}

%%%%%%%%%%%%%%%%%%%%%%%%%%%%%%%%%%%%%%%%
\paragraph{v0.5:} 2017/04/26

\begin{itemize}
\item
functionality in definition file
\end{itemize}


%%%%%%%%%%%%%%%%%%%%%%%%%%%%%%%%%%%%%%%%%%%%%%%%%%%%%%%%%%%%%%%%%%%%%%%%%%%%%%%%
%%%%%%%%%%%%%%%%%%%%%%%%%%%%%%%%%%%%%%%%%%%%%%%%%%%%%%%%%%%%%%%%%%%%%%%%%%%%%%%%
%%%%%%%%%%%%%%%%%%%%%%%%%%%%%%%%%%%%%%%%%%%%%%%%%%%%%%%%%%%%%%%%%%%%%%%%%%%%%%%%
\appendix

\settowidth\MacroIndent{\rmfamily\scriptsize 000\ }

 \DocInput{childdoc.dtx}

\end{document}
%</driver>
% \fi
%
% %%%%%%%%%%%%%%%%%%%%%%%%%%%%%%%%%%%%%%%%%%%%%%%%%%%%%%%%%%%%%%%%%%%%%%%%%%%%%%
% %%%%%%%%%%%%%%%%%%%%%%%%%%%%%%%%%%%%%%%%%%%%%%%%%%%%%%%%%%%%%%%%%%%%%%%%%%%%%%
% \section{Sample}
%\iffalse
%<*samplemain>
%\fi
%
% The following presents a sample document
% with two chapters, two parts, a title page,
% a compile flag as well as three forwarding files to set the flag.
% It consists of eight |.tex| files:
% \begin{center}
% \begin{tabular}{ll}
% |cdocsamp.tex|&main file\\
% |cdocsch1.tex|&include file for chapter 1\\
% |cdocsch2.tex|&include file for chapter 2\\
% |cdocspt3.tex|&include file for part 3\\
% |cdocspt4.tex|&include file for part 4\\
% |cdocsdrf.tex|&forwarding file for main file in draft mode\\
% |cdocsfi1.tex|&forwarding file for final version of chapter 1\\
% |cdocsfi2.tex|&forwarding file for final version of chapter 2\\
% \end{tabular}
% \end{center}
% Each of the eight files can be compiled directly by the \LaTeX{} compiler.
%
% %%%%%%%%%%%%%%%%%%%%%%%%%%%%%%%%%%%%%%
% \paragraph{Main File.}
%
% The main file is called |cdocsamp.tex|.
%
% Load the \textsf{childdoc} definitions and
% declare the filename for the main document:
%    \begin{macrocode}
\input{childdoc.def}
\childdocmain{}
%    \end{macrocode}

% Optional override for |\version| flag:
%    \begin{macrocode}
%%\ifchilddoc\else\providecommand{\version}{draft}\fi
%    \end{macrocode}

% Define the default values for the |\version| flag
% (|final| for the main file and |draft| for childs):
%    \begin{macrocode}
\ifchilddoc
\providecommand{\version}{draft}
\else
\providecommand{\version}{final}
\fi
%    \end{macrocode}

% Load the standard document class:
%    \begin{macrocode}
\documentclass[12pt]{article}
%    \end{macrocode}

% Start the document body:
%    \begin{macrocode}
\begin{document}
%    \end{macrocode}

% Declare a title page.
% Print title, part of document being processed and version flag:
%    \begin{macrocode}
\addtocounter{page}{-1}
\begin{center}
{\LARGE\bfseries{}childdoc example\par}
\vspace{1cm}
\ifchilddoc
\ifchilddocmanual part\else chapter\fi:
`\childdocname' of `\childdocjob'\par
\else
main document: `\childdocjob'\par
\fi
version: \version\par
\end{center}
\newpage
%    \end{macrocode}

% Manually include selected file,
% otherwise process as usual:
%    \begin{macrocode}
\ifchilddocmanual
\section*{part `\childdocname'}
\input{\childdocname}
\else
%    \end{macrocode}

% Include the two chapters:
%    \begin{macrocode}
\include{cdocsch1}
\include{cdocsch2}
%    \end{macrocode}

% Include the two parts unless only chapters should be displayed:
%    \begin{macrocode}
\ifchilddoc\else
\section{part three}
\input{cdocspt3}
\section{part four}
\input{cdocspt4}
\fi
%    \end{macrocode}

% Process as usual until here:
%    \begin{macrocode}
\fi
%    \end{macrocode}

% End of document body:
%    \begin{macrocode}
\end{document}
%    \end{macrocode}
%\iffalse
%</samplemain>
%\fi
%
% %%%%%%%%%%%%%%%%%%%%%%%%%%%%%%%%%%%%%%
% \paragraph{Chapter Include Files.}
%
% The include files are called |cdocsch1.tex| and |cdocsch2.tex|.
%
%\iffalse
%<*samplechap1|samplechap2>
%\fi

% Optional override for |\version| flag:
%    \begin{macrocode}
%%\providecommand{\version}{final}
%    \end{macrocode}

% Include the main document:
%    \begin{macrocode}
\input{childdoc.def}
\childdocof{cdocsamp}
%    \end{macrocode}

%\iffalse
%</samplechap1|samplechap2>
%\fi
%
%\iffalse
%<*samplechap1>
%\fi
% Some text for chapter 1:
%    \begin{macrocode}
\section{one}
some text in chapter one
%    \end{macrocode}

%\iffalse
%</samplechap1>
%\fi
% Some text for chapter 2:
%\iffalse
%<*samplechap2>
%\fi
%    \begin{macrocode}
\section{two}
more text in chapter two
%    \end{macrocode}

%\iffalse
%</samplechap2>
%\fi
%
% %%%%%%%%%%%%%%%%%%%%%%%%%%%%%%%%%%%%%%
% \paragraph{Part Include Files.}
%
% The include files are called |cdocspt3.tex| and |cdocspt4.tex|.
%
%\iffalse
%<*samplepart3|samplepart4>
%\fi

% Optional override for |\version| flag:
%    \begin{macrocode}
%%\providecommand{\version}{final}
%    \end{macrocode}

% Include the main document:
%    \begin{macrocode}
\input{childdoc.def}
\childdocby{cdocsamp}
%    \end{macrocode}

%\iffalse
%</samplepart3|samplepart4>
%\fi
%
%\iffalse
%<*samplepart3>
%\fi
% Some text for part 3:
%    \begin{macrocode}
some text in part three
%    \end{macrocode}

%\iffalse
%</samplepart3>
%\fi
% Some text for part 4:
%\iffalse
%<*samplepart4>
%\fi
%    \begin{macrocode}
more text in part four
%    \end{macrocode}

%\iffalse
%</samplepart4>
%\fi
%
% %%%%%%%%%%%%%%%%%%%%%%%%%%%%%%%%%%%%%%
% \paragraph{Forwarding for a Complete Draft.}
%
% The following forwarding file |cdocsdrf.tex|
% compiles the main document in draft mode:
%\iffalse
%<*sampledraft>
%\fi
%    \begin{macrocode}
\def\version{draft}
\input{childdoc.def}
\childdocforward{cdocsamp}
%    \end{macrocode}

%\iffalse
%</sampledraft>
%\fi
%
% %%%%%%%%%%%%%%%%%%%%%%%%%%%%%%%%%%%%%%
% \paragraph{Forwarding for Final Version of the Chapters.}
%
% The following forwarding files |cdocsfn1.tex| and |cdocsfn2.tex|
% (with identical content)
% compile the final versions of the child documents
% |cdocsch1.tex| and |cdocsch2.tex|, respectively:
%\iffalse
%<*samplefinal>
%\fi
%    \begin{macrocode}
\def\version{final}
\input{childdoc.def}
\childdocforwardprefix[cdocsamp]{cdocsfn}{cdocsch}
%    \end{macrocode}

%\iffalse
%</samplefinal>
%\fi
%
% %%%%%%%%%%%%%%%%%%%%%%%%%%%%%%%%%%%%%%
% \paragraph{Command Line Processing.}
%
% The following three command lines generate the output files
% |cdocscld|, |cdocscl1| and |cdocscl2|
% which should be identical to
% |cdocsdrf|, |cdocsch1| and |cdocsfn2|, respectively:
% \begin{center}
% \begin{tabular}{l}
% |latex -jobname cdocscld \|\\
% |  "\def\version{draft}\input{childdoc.def}\childdocforward{cdocsamp}"|\\
% |latex -jobname cdocscl1 \|\\
% |  "\input{childdoc.def}\childdocforward[cdocsamp]{cdocsch1}"|\\
% |latex -jobname cdocscl2 \|\\
% |  "\def\version{final}\input{childdoc.def}\childdocforward{cdocsch2}"|
% \end{tabular}
% \end{center}
% Note that the trailing backslash on each first line
% merely continues the input to the second line
% (for convenient cut ant paste).
% Furthermore, the command |latex| can be replaced by any
% of its alternative versions such as |pdflatex|.
%
% %%%%%%%%%%%%%%%%%%%%%%%%%%%%%%%%%%%%%%%%%%%%%%%%%%%%%%%%%%%%%%%%%%%%%%%%%%%%%%
% %%%%%%%%%%%%%%%%%%%%%%%%%%%%%%%%%%%%%%%%%%%%%%%%%%%%%%%%%%%%%%%%%%%%%%%%%%%%%%
% \section{Implementation}
%\iffalse
%<*package>
%\fi
%
% This section describes the definitions file |childdoc.def|.

% The definitions cannot be loaded using |\usepackage| or |\RequirePackage|
% which has a mechanism to prevent loading a style file more than once.
% When loading the definitions by means of |\input|
% multiple instances have to be prevented manually:
%\iffalse
%This code needs to be before the `\ProvidesFile' directive
%which is defined at the beginning of this file.
%Therefore it is also placed there and commented out here.
%</package>
%<*discard>
%\fi
%    \begin{macrocode}
\ifdefined\childdocmain\endinput\fi
%    \end{macrocode}
%\iffalse
%</discard>
%<*package>
%\fi
%
% \macro{\ifchilddoc}
% \macro{\ifchilddocmanual}
% The conditional |\ifchilddoc| tells whether a
% child (true) or main (false) document is being compiled.
% The conditional |\ifchilddocmanual| tells whether
% the |\includeonly| mechanism is used (false) or
% the selection of child files must be performed manually (true).
% The definitions initialise to false:
%    \begin{macrocode}
\newif\ifchilddoc
\newif\ifchilddocmanual
%    \end{macrocode}

% \macro{\childdocname}
% \macro{\childdocjob}
% The macro |\childdocname| stores the name of the main document
% to be compiled. The macro |\childdocjob| stores the name of
% the document on which the \LaTeX{} compiler was originally invoked.
% The content of |\jobname| cannot be compared
% to filenames specified in the source due to different catcodes.
% The following code rescans |\jobname|, stores the result
% in |\childdocname| and saves a copy in |\childdocjob|:
%    \begin{macrocode}
\edef\childdocname{\scantokens\expandafter{\jobname\noexpand}}
\let\childdocjob\childdocname
%    \end{macrocode}

% \macro{\childdocdisable}
% The macro |\childdocdisable| prevents the main file
% from being processed more than once.
% At this stage, the main document command |\childdocmain|
% is assumed to be called once again where it should do nothing.
% Any subsequent call to it should prevent
% a secondary processing of the main document
% It overwrites the forwarding commands
% |\childdocof| and |\childdocforward|
% with empty macros to prevent further inclusions of the main document:
%    \begin{macrocode}
\newcommand{\childdocdisable}
{
  \renewcommand{\childdocmain}[1]{\renewcommand{\childdocmain}[1]{\endinput}}
  \renewcommand{\childdocof}[1]{}
  \renewcommand{\childdocby}[2][]{}
  \renewcommand{\childdocforward}[2][]{}
  \renewcommand{\childdocdisable}{}
}
%    \end{macrocode}

% \macro{\childdocmain}
% The macro |\childdocmain| is to be called at the top of the main file
% with nothing or the main filename (without extension) as argument.
% First, it breaks loops.
% If the argument is not empty and does not match |\childdocname|
% (which is set by the first inclusion of |childdoc.def|),
% |\ifchilddoc| is set to true, |\includeonly| is applied to the child file
% and |\jobname| is set to the main file
% (for proper handling of |.aux| files):
%    \begin{macrocode}
\newcommand{\childdocmain}[1]
{
  \childdocdisable\childdocmain{}
  \if?#1?\else
    \begingroup
      \def\childdoctmp{#1}
      \ifx\childdoctmp\childdocname
        \def\childdoctmp{}
      \else
        \def\childdoctmp
        {
          \childdoctrue
          \includeonly{\childdocname}
          \def\childdocjob{#1}
          \def\jobname{#1}
        }
      \fi
      \expandafter
    \endgroup
    \childdoctmp
  \fi
}
%    \end{macrocode}

% \macro{\childdocof}
% The command |\childdocof| redirects
% compilation to the main file |#1|.
%    \begin{macrocode}
\newcommand{\childdocof}[1]
{
  \childdocdisable
  \childdoctrue
  \includeonly{\childdocname}
  \def\jobname{#1}
  \def\childdocjob{#1}
  \input{#1}
}
%    \end{macrocode}

% \macro{\childdocby}
% The command |\childdocby| ....
%    \begin{macrocode}
\newcommand{\childdocby}[2][]
{
  \childdocdisable
  \childdoctrue
  \childdocmanualtrue
  \if?#1?\else
    \def\jobname{#2}
  \fi
  \def\childdocjob{#2}
  \input{#2}
  \endinput
}
%    \end{macrocode}

% \macro{\childdocforward}
% The command |\childdocforward| redirects
% compilation to the main file or
% (if the optional argument is given) a child file.
% Parameters are set as if the main file
% or a child file starting with |\childdocof| was compiled.
% Then compilation is handed over to the main file:
%    \begin{macrocode}
\newcommand{\childdocforward}[2][]
{
  \begingroup
    \if?#1?
      \def\childdoctmp
      {
        \def\childdocname{#2}
        \def\childdocjob{#2}
        \def\jobname{#2}
        \input{#2}
        \endinput
      }
    \else
      \def\childdoctmp
      {
        \childdocdisable
        \def\childdocname{#2}
        \childdoctrue
        \includeonly{#2}
        \def\childdocjob{#1}
        \def\jobname{#1}
        \input{#1}
        \endinput
      }
    \fi
    \expandafter
  \endgroup
  \childdoctmp
}
%    \end{macrocode}

% \macro{\childdocforwardprefix}
% The command |\childdocforwardprefix| redirects
% compilation to the main or a child file by means of a pattern.
% The prefix |#1| in the current filename is replaced by |#2|
% and the suffix of the current filename is kept
% (it is assumed that the filename does not contain the substring `|~~~|'
% which is used as a delimiter).
% Compilation is handed over to the new file by |\childdocforward|:
%    \begin{macrocode}
\newcommand{\childdocforwardprefix}[3][]
{
  \begingroup
    \def\childdocextract #2##1~~~{\def\childdoctmp{\childdocforward[#1]{#3##1}}}
    \expandafter\childdocextract\childdocname~~~
    \expandafter
  \endgroup
  \childdoctmp
}
%    \end{macrocode}

% \macro{\childdoc}
% The deprecated macro |\childdoc| is a legacy version of |\childdocmain|:
%    \begin{macrocode}
\newcommand{\childdoc}{\childdocmain}
%    \end{macrocode}

% \macro{\childdocredirect}
% The deprecated macro |\childdocredirect| is a legacy version
% of |\childdocforward| and |\childdocforwardprefix|:
%    \begin{macrocode}
\newcommand{\childdocredirect}[2][]
{
  \begingroup
    \if?#1?
      \def\childdoctmp{\childdocforward{#2}}
    \else
      \def\childdoctmp{\childdocforwardprefix{#1}{#2}}
    \fi
    \expandafter
  \endgroup
  \childdoctmp
}
%    \end{macrocode}

%\iffalse
%</package>
%\fi
%
\endinput
\childdocforward[cdocsamp]{cdocsch1}"|\\
% |latex -jobname cdocscl2 \|\\
% |  "\def\version{final}% \iffalse
%
% childdoc.dtx Copyright (C) 2017-2018 Niklas Beisert
%
% This work may be distributed and/or modified under the
% conditions of the LaTeX Project Public License, either version 1.3
% of this license or (at your option) any later version.
% The latest version of this license is in
%   http://www.latex-project.org/lppl.txt
% and version 1.3 or later is part of all distributions of LaTeX
% version 2005/12/01 or later.
%
% This work has the LPPL maintenance status `maintained'.
%
% The Current Maintainer of this work is Niklas Beisert.
%
% This work consists of the files childdoc.dtx and childdoc.ins
% and the derived files childdoc.def and cdocsamp.tex with
% cdocsch1.tex, cdocsch2.tex, cdocsdrf.tex, cdocsfn1.tex, cdocsfn2.tex.
%
%<package>\ifdefined\childdocmain\endinput\fi
%<package>\ProvidesFile{childdoc.def}[2018/12/30 v2.0 child document driver]
%<samplemain>\ProvidesFile{cdocsamp.tex}[2018/12/30 v2.0 sample for childdoc]
%<*driver>
%\ProvidesFile{childdoc.drv}[2018/12/30 v2.0 childdoc reference manual file]
\PassOptionsToClass{10pt,a4paper}{article}
\documentclass{ltxdoc}

\usepackage[margin=35mm]{geometry}
\usepackage{hyperref}
\usepackage{hyperxmp}
\usepackage[usenames]{color}

\hypersetup{colorlinks=true}
\hypersetup{pdfstartview=FitH}
\hypersetup{pdfpagemode=UseNone}
\hypersetup{pdfsource={}}
\hypersetup{pdflang={en-UK}}
\hypersetup{pdfcopyright={Copyright 2017-2018 Niklas Beisert.
  This work may be distributed and/or modified under the
  conditions of the LaTeX Project Public License, either version 1.3
  of this license or (at your option) any later version.}}
\hypersetup{pdflicenseurl={http://www.latex-project.org/lppl.txt}}
\hypersetup{pdfcontactaddress={ETH Zurich, ITP, HIT K,
  Wolfgang-Pauli-Strasse 27}}
\hypersetup{pdfcontactpostcode={8093}}
\hypersetup{pdfcontactcity={Zurich}}
\hypersetup{pdfcontactcountry={Switzerland}}
\hypersetup{pdfcontactemail={nbeisert@itp.phys.ethz.ch}}
\hypersetup{pdfcontacturl={http://people.phys.ethz.ch/\xmptilde nbeisert/}}

\newcommand{\secref}[1]{\hyperref[#1]{section \ref*{#1}}}

\parskip1ex
\parindent0pt
\let\olditemize\itemize
\def\itemize{\olditemize\parskip0pt}

\begin{document}

\title{The \textsf{childdoc} Package}
\hypersetup{pdftitle={The childdoc Package}}
\author{Niklas Beisert\\[2ex]
  Institut f\"ur Theoretische Physik\\
  Eidgen\"ossische Technische Hochschule Z\"urich\\
  Wolfgang-Pauli-Strasse 27, 8093 Z\"urich, Switzerland\\[1ex]
  \href{mailto:nbeisert@itp.phys.ethz.ch}
  {\texttt{nbeisert@itp.phys.ethz.ch}}}
\hypersetup{pdfauthor={Niklas Beisert}}
\hypersetup{pdfsubject={Manual for the LaTeX2e Package childdoc}}
\date{30 December 2018, \textsf{v2.0}}
\maketitle

\begin{abstract}\noindent
\textsf{childdoc} is a \LaTeXe{} package
that enables the direct compilation
of document sections included by |\include|
to individual files.
\end{abstract}

\begingroup
\parskip0ex
\tableofcontents
\endgroup

%%%%%%%%%%%%%%%%%%%%%%%%%%%%%%%%%%%%%%%%%%%%%%%%%%%%%%%%%%%%%%%%%%%%%%%%%%%%%%%%
%%%%%%%%%%%%%%%%%%%%%%%%%%%%%%%%%%%%%%%%%%%%%%%%%%%%%%%%%%%%%%%%%%%%%%%%%%%%%%%%
\section{Introduction}

\LaTeX{} provides a mechanism to structure a large document (such as a book)
into a main file and several child files (containing the chapters)
using the |\include| command.
This mechanism is beneficial for documents
which span hundreds of pages in order to
make the source file(s) more manageable.
Moreover, compilation can be restricted to
selected child files by means of the |\includeonly| command.
The latter feature can be used to reduce the compilation time while editing
(this was significantly more useful in the earlier days of \LaTeX{})
or to generate a smaller document which is easier to navigate.
Another application of |\includeonly| is to generate
documents consisting of selected parts of the complete document.

However, there are a few drawbacks of the plain |\include| mechanism:
\begin{itemize}
\item
The child files cannot be compiled on their own,
they can only be compiled via the main file.
A naive editing environment
(such as a text editor with an option
to have the current file processed by \LaTeX)
may require one to switch to the main file before compiling;
attempting to compile the child file produces errors.
\item
The main file must be modified (each time)
to adjust the |\includeonly| command
to the present needs. This easily leaves the main file in a messy state.
\item
The generated document will always carry the filename
of the main document. This is inconvenient if
several child files are to be compiled and
to be kept for distribution.
\end{itemize}

The present package provides a simple interface
to make child files individually compilable by \LaTeX{}.
Compiling a child file then has the same effect as compiling
the main file with an |\includeonly| command
to select the appropriate child.
Moreover the generated document will carry the name of the child
rather than the main file.
This resolves all three above issues.

This feature is meant to make the editing of books,
thesis documents and lecture notes somewhat more convenient.
However, the package can also be used efficiently for
composing a series of documents (such as exercise sheets)
which are typically distributed individually.
It then assists the author in generating the individual documents
(potentially in different versions)
as well as a document containing the collected series.
Another application is in developing style files
or other kinds of included material
where compilation of the style file could redirect
to a sample or test file.

%%%%%%%%%%%%%%%%%%%%%%%%%%%%%%%%%%%%%%%%%%%%%%%%%%%%%%%%%%%%%%%%%%%%%%%%%%%%%%%%
%%%%%%%%%%%%%%%%%%%%%%%%%%%%%%%%%%%%%%%%%%%%%%%%%%%%%%%%%%%%%%%%%%%%%%%%%%%%%%%%
\section{Usage}

First of all, the package \textsf{childdoc} is \emph{not} a standard
\LaTeXe{} |.sty| style file! Therefore it needs to be invoked in
a non-standard way.

%%%%%%%%%%%%%%%%%%%%%%%%%%%%%%%%%%%%%%%%%%%%%%%%%%%%%%%%%%%%%%%%%%%%%%%%%%%%%%%%
\subsection{Included Files}
\label{sec:include}

%%%%%%%%%%%%%%%%%%%%%%%%%%%%%%%%%%%%%%%%
\DescribeMacro{\childdocmain}
To use the package, add the commands
\begin{center}
\begin{tabular}{l}
|\input{childdoc.def}|\\
|\childdocmain{}|\\
\end{tabular}
\end{center}
at the very top of the main \LaTeX{} file,
in particular \emph{before} the |\documentclass| statement!
The argument of |\childdocmain| should be left empty
(but it must be present).

%%%%%%%%%%%%%%%%%%%%%%%%%%%%%%%%%%%%%%%%
\DescribeMacro{\childdocof}
Furthermore, add the commands
\begin{center}
\begin{tabular}{l}
|\input{childdoc.def}|\\
|\childdocof{|\textit{main}|}|\\
\end{tabular}
\end{center}
at the top of every child file \textit{child}
which is included by |\include{|\textit{child}|}|
from within the main file
(or at least for those files to be compiled individually).
The argument \textit{main} must be the filename of the main file.

There are a couple of
considerations in setting up the main and child documents:

%%%%%%%%%%%%%%%%%%%%%%%%%%%%%%%%%%%%%%%%
\paragraph{Restrictions.}

Please note the following restrictions:
\begin{itemize}
\item
|\childdocmain| must be called with one argument \textit{main}
to ensure compatibility with earlier version of the package.
It must either be empty (|\childdocmain{}|)
or precisely match the filename of the main file in which it is specified.
See \secref{sec:detection} for further information.
\item
The filename \textit{main} must be specified without the |.tex| extension.
\item
The filename \textit{main} is case sensitive
(even in case-insensitive file systems)
due to internal string comparison.
\item
The argument \textit{main} should be fully expanded, it cannot be a macro.
\item
Subdirectories and special characters should be avoided in filenames.
\item
The command |\childdocmain{|\textit{main}|}| must be followed by a whitespace.
It should not be followed immediately by another command
or by a comment mark `|%|'.
This is because the \TeX{} parser reads the token immediately following
the argument of |\childdocmain| and puts it
at the beginning of every child section;
however, a white\-space is ignored.
\end{itemize}

%%%%%%%%%%%%%%%%%%%%%%%%%%%%%%%%%%%%%%%%
\paragraph{Content of Main File.}

It is advisable to place all content in the child files included by |\include|.
Any output contained in the main file will appear in all child documents
unless suppressed manually;
it cannot be suppressed automatically by the |\includeonly| directive
and thus should normally be avoided.
A method to include some content in the main file
by means of conditional processing is described in \secref{sec:conditional}.

%%%%%%%%%%%%%%%%%%%%%%%%%%%%%%%%%%%%%%%%
\paragraph{Page Numbering.}

When only a part of the document is compiled,
the appropriate numbering of pages
(as well as other status parameters)
is determined from the |.aux| files.
The latter contain information from previous passes.
However this information needs to propagate through
all intermediate child documents.
Therefore the page numbering in child documents may well
be inconsistent until the complete document is compiled at least once.

A useful (if unconventional) way to always ensure a consistent
page numbering is to restart the numbering in each child document
and denote the pages by `\textit{child}|.|\textit{page}'
where \textit{child} represents the chapter/section number of the child file.
This can be achieved by the command
|\numberwithin{page}{|\textit{child}|}|
of the \textsf{amsmath} package
where \textit{child} can be |chapter| or |section|
depending on the chosen structuring.
Alternatively, one can modify the macro |\thepage| appropriately
and reset the counter |page| at the start of each child file.

%%%%%%%%%%%%%%%%%%%%%%%%%%%%%%%%%%%%%%%%%%%%%%%%%%%%%%%%%%%%%%%%%%%%%%%%%%%%%%%%
\subsection{Conditional Processing}
\label{sec:conditional}

The package provides a mechanism to compile different versions
of a document. To customise the versions further some conditional processing
can come in handy to distinguish which version is being compiled.
The package provides two macros to describe the compilation context:

%%%%%%%%%%%%%%%%%%%%%%%%%%%%%%%%%%%%%%%%
\DescribeMacro{\ifchilddoc}
The conditional |\ifchilddoc| distinguishes between the compilation of
child documents and the main document:
%
\begin{center}
|\ifchilddoc |\textit{child-code}| |[|\||else |\textit{main-code}]| \||fi|
\end{center}

%%%%%%%%%%%%%%%%%%%%%%%%%%%%%%%%%%%%%%%%
\DescribeMacro{\childdocname}
\DescribeMacro{\childdocjob}
The macro |\childdocname| contains the filename (without extension)
of the main or child file being processed.
Note that |\childdocjob| will always contain the name of the main file.

%%%%%%%%%%%%%%%%%%%%%%%%%%%%%%%%%%%%%%%%
\paragraph{Title Page.}

Conditional processing can be used to include a title or banner page
in the main document when proper precautions are taken.
Importantly, the code in the main file should ensure that the page counter
(as well as other status parameters which are stored in the |.aux| files)
takes the same value after the conditional processing.
Otherwise the page numbers may take divergent values
depending on which part is compiled.

For example, a title page could be declared by:
%
\begin{center}
\begin{tabular}{l}
|\ifchilddoc\||else|\\
|\addtocounter{page}{-1}|\\
\textit{code for title page}\\
|\newpage|\\
|\||fi|
\end{tabular}
\end{center}
%
A banner page for the child documents can be generated by:
%
\begin{center}
\begin{tabular}{l}
|\ifchilddoc|\\
|\addtocounter{page}{-1}|\\
\textit{code for banner page}\\
|\newpage|\\
|\||fi|
\end{tabular}
\end{center}
%
Here one could write a message such as:
\begin{center}
|This is the part \childdocname{} of \childdocjob{}.|
\end{center}

%%%%%%%%%%%%%%%%%%%%%%%%%%%%%%%%%%%%%%%%%%%%%%%%%%%%%%%%%%%%%%%%%%%%%%%%%%%%%%%%
\subsection{Flags}
\label{sec:flags}

The package makes it easy to generate different versions
of the main or child documents.
To this end compilation flags can be defined
and assigned different default values.
They will be particularly useful in conjunction
with the forwarding mechanism described in \secref{sec:forward}.

For example, it may be useful to have a flag |\version|
which can be set to |draft| or |final|.
The document source will contain some conditional code
depending on the value of |\version|.
Suppose further, the flag should default to |final| for the main file
and to |draft| for child files
which is a natural assignment for editing the document.
This is achieved by placing the following code
in the preamble of the main document
(below the |\childdocmain| directive):
%
\begin{center}
\begin{tabular}{l}
|\ifchilddoc|\\
|\providecommand{\version}{draft}|\\
|\||else|\\
|\providecommand{\version}{final}|\\
|\||fi|
\end{tabular}
\end{center}
%
The definition by |\providecommand| makes sure
that previous definitions are not overwritten.
Further statements |\providecommand{\version}{...}|
can thus be added before the above code to override it.

For the main file, one might add a line
(between |\childdocmain| and the above block)
%
\begin{center}
|%\ifchilddoc\||else\providecommand{\version}{draft}\||fi|
\end{center}
%
which can be uncommented to produce a draft version.
Likewise one can add a line to the very top of a child file
(above the |\childdocof{|\textit{main}|}| directive)
%
\begin{center}
|%\providecommand{\version}{final}|
\end{center}
%
which can be uncommented to produce the final version of this child document.

%%%%%%%%%%%%%%%%%%%%%%%%%%%%%%%%%%%%%%%%%%%%%%%%%%%%%%%%%%%%%%%%%%%%%%%%%%%%%%%%
\subsection{Forwarding}
\label{sec:forward}

Different versions of the main or child documents
using compilation flags as described in \secref{sec:flags}
can be (permanently) stored in different files
for convenient compilation, viewing and distribution.
To this end, the package defines a command
to pass on compilation to a different file:

%%%%%%%%%%%%%%%%%%%%%%%%%%%%%%%%%%%%%%%%
\DescribeMacro{\childdocforward}
The command |\childdocforward| redirects processing to
another source file:
%
\begin{center}
\begin{tabular}{l}
|\input{childdoc.def}|\\
|\childdocforward[|\textit{main}|]{|\textit{dest}|}|\\
\end{tabular}
\end{center}
%
The argument \textit{dest} is the destination file
(without extension).
It should be the main file or one of the child files.
Note that further \textsf{childdoc} directives
such as |\childdocof| and |\childdocforward|
in the indicated file will be processed in this form.
The optional argument \textit{main}
passes on directly to the main file \textit{main}
while pretending to compile the child \textit{dest}.
This form behaves as if \textit{dest}
issues |\childdocof{|\textit{main}|}| right away,
and no further \textsf{childdoc} directives will be processed.

%%%%%%%%%%%%%%%%%%%%%%%%%%%%%%%%%%%%%%%%
\DescribeMacro{\...prefix}
In the alternative form |\childdocforwardprefix|,
%
\begin{center}
\begin{tabular}{l}
|\input{childdoc.def}|\\
|\childdocforwardprefix[|\textit{main}|]{|\textit{prefix}|}{|\textit{dest}|}|
\end{tabular}
\end{center}
%
the destination file is determined by a pattern
depending on the current file:
To make this work, the current file must be called
`{\textit{prefix}\hspace{0.2em}\textit{suffix}}'
with \textit{prefix} matching precisely the argument.
Processing is then passed on to the file
`{\textit{dest}\hspace{0.2em}\textit{suffix}}'.
Surely, the same effect is achieved by
directly specifying the
argument `{\textit{dest}\hspace{0.2em}\textit{suffix}}'
in the first form.
However, that requires to set up a different file
for each child. With the alternative form of the command
all these files can have exactly the same content
which simplifies setting them up and maintaining them.

For example, the following file |draft.tex|
with a compilation flag |\version| as described in \secref{sec:flags}
compiles the main document as a draft:
%
\begin{center}
\begin{tabular}{l}
|\def\version{draft}|\\
|\input{childdoc.def}|\\
|\childdocforward{|\textit{main}|}|
\end{tabular}
\end{center}
%
Likewise, the following files |final|\textit{nn}|.tex|
compile the final version of the child document
|child|\textit{nn}|.tex|:
%
\begin{center}
\begin{tabular}{l}
|\def\version{final}|\\
|\input{childdoc.def}|\\
|\childdocforwardprefix{final}{child}|
\end{tabular}
\end{center}
%

Note that when several versions of a main file and/or of each child file
are to be generated, it may be convenient to set up a |Makefile| or
shell script to automatise the process.

%%%%%%%%%%%%%%%%%%%%%%%%%%%%%%%%%%%%%%%%%%%%%%%%%%%%%%%%%%%%%%%%%%%%%%%%%%%%%%%%
\subsection{Command Line Processing}
\label{sec:commandline}

The effect of redirection files can also be achieved by invoking
the \LaTeX{} compiler with a more elaborate command line.
Most conveniently this should be done as part
of a shell script or a |Makefile|.

When using \textsf{childdoc} in the main file, the following
command lines effectively perform a redirection
(note that depending on the shell being used,
backslashes may have to be doubled: `|\|' $\to$ `|\\|'):
%
\begin{center}
|... -jobname "|\textit{target}|" |\\|"|[\textit{flags}]%
|\input{childdoc.def}\childdocforward[|\textit{main}|]{|\textit{dest}|}"|
\end{center}
%
Here \textit{target} is the name of the output file,
\textit{main} is the name of the main file
and \textit{dest} is the name of the main or child file to be processed
(all filenames without extensions).
The optional argument \textit{main} can be omitted
if \textit{main} matches \textit{dest}.
Optionally, compilation \textit{flags} can be defined via |\def| commands.
This command line makes the \TeX{} engine believe
it is compiling the file \textit{target}
whose content is specified as the latter parameter.
The provided code then forwards the processing to
\textit{main} or \textit{dest} as described in \secref{sec:forward}.

%%%%%%%%%%%%%%%%%%%%%%%%%%%%%%%%%%%%%%%%%%%%%%%%%%%%%%%%%%%%%%%%%%%%%%%%%%%%%%%%
\subsection{Include by Input}
\label{sec:input}

Including child documents by |\include| has some restrictions by design.
Most notably, the content of a child document always occupies
its own set of pages; pages cannot be shared between child documents.
Usually, this behaviour makes perfect sense
because each child document contain an essential part of the document.
However, in some situations it may be desirable to compose
a document from a collection of parts
without having mandatory page breaks between then.
For this case, the package
provides a mechanism to include parts
by |\input| which can also be processed individually.
However, by construction this mechanism
requires manual handling of the content to be output.

%%%%%%%%%%%%%%%%%%%%%%%%%%%%%%%%%%%%%%%%
\DescribeMacro{\ifchilddocmanual}
The main file should be prepared as usual, see \secref{sec:include}.
However, the document body must make a distinction
between processing of an individual part and of the main document, e.g.:
%
\begin{center}
\begin{tabular}{l}
|\ifchilddocmanual|\\
|\input{\childdocname}|\\
|\||else|\\
\textit{document body with }|\input{|\textit{part}|}|\\
|\||fi|
\end{tabular}
\end{center}
%
The conditional |\ifchilddocmanual| is true whenever
a part to be included by |\input| is being compiled,
and the name of the part is stored in |\childdocname|.

%%%%%%%%%%%%%%%%%%%%%%%%%%%%%%%%%%%%%%%%
\DescribeMacro{\childdocby}
Each part to be included by |\input| should start with:
%
\begin{center}
\begin{tabular}{l}
|\input{childdoc.def}|\\
|\childdocby{|\textit{main}|}|\\
\end{tabular}
\end{center}
%
The directive |\childdocby| is similar to |\childdocof|
described in \secref{sec:include},
but the subsequent selection of content must be done manually.
To that end, both |\ifchilddoc| and |\ifchilddocmanual|
will be true upon processing of a part,
and the name of the part is stored in |\childdocname|.
Note that |\jobname| will be set to the filename of the current part
so that each part receives an individual |.aux| file
that does not interfere with the |.aux| file(s) of the main document.
This behaviour can be altered by the alternative form
|\childdocby[*]{|\textit{main}|}| (with a non-empty optional argument)
which uses the |.aux| file of the main document
by setting |\jobname| to \textit{main}.

%%%%%%%%%%%%%%%%%%%%%%%%%%%%%%%%%%%%%%%%%%%%%%%%%%%%%%%%%%%%%%%%%%%%%%%%%%%%%%%%
\subsection{Driver Development}
\label{sec:driver}

The \textsf{childdoc} mechanism can also be use for the development
of definition files such as \LaTeX{} styles or classes.
This case differs from the above setup with multiple parts
included by |\include| in that no |\includeonly| should be invoked.
This can be achieved by starting the include file
(before |\ProvidesPackage|) with:
%
\begin{center}
\begin{tabular}{l}
|\input{childdoc.def}|\\
|\childdocforward{|\textit{main}|}|\\
\end{tabular}
\end{center}
%
or alternatively with:
%
\begin{center}
\begin{tabular}{l}
|\input{childdoc.def}|\\
|\childdocby{|\textit{main}|}|\\
\end{tabular}
\end{center}
%
Both forms have slightly different effects as described above.
The main file is prepared as usual, see \secref{sec:include}.

%%%%%%%%%%%%%%%%%%%%%%%%%%%%%%%%%%%%%%%%%%%%%%%%%%%%%%%%%%%%%%%%%%%%%%%%%%%%%%%%
\subsection{Legacy Detection}
\label{sec:detection}

The directive |\childdocmain| in the main file can detect
whether the complete document or merely a child is to be compiled
even without using the directive |\childdocof|.
This method is deprecated because it is less robust
and there is no compelling reason to use it;
it is merely provided for backward compatibility
and it may be removed in future versions.

If the detection mechanism is to be used,
it is mandatory to correctly specify
the filename of the main file as the argument of |\childdocmain|:
%
\begin{center}
\begin{tabular}{l}
|\input{childdoc.def}|\\
|\childdocmain{|\textit{main}|}|\\
\end{tabular}
\end{center}
%
If |\jobname| does not match the argument \textit{main} of |\childdocmain|,
it is assumed that |\jobname| points to the child file to be compiled.
When using |\childdocmain| with the main file specified as argument,
it suffices to start a child file
with just |\input{|\textit{main}|}|
without loading of the package and using |\childdocof|.
If instead all processing is done
with the appropriate \textsf{childdoc} directives,
the argument of \textit{main} of |\childdocmain| can be empty.

An alternative version of the command line processing described
in \secref{sec:commandline} using the detection mechanism reads:
%
\begin{center}
|... -jobname "|\textit{target}|" "|[\textit{flags}]%
[|\def\jobname{|\textit{dest}|}|]|\input{|\textit{main}|}"|
\end{center}

%%%%%%%%%%%%%%%%%%%%%%%%%%%%%%%%%%%%%%%%%%%%%%%%%%%%%%%%%%%%%%%%%%%%%%%%%%%%%%%%
\subsection{Manual Code}
\label{sec:manual}

In case one cannot be certain whether the definitions file |childdoc.def|
is installed on the target \TeX{} distribution
and one prefers not to ship it,
it is conceivable to paste a few relevant commands into the sources.

To that end, drop all statements |\input{childdoc.def}|
and perform the replacements as outlined below.
Instead of |\childdocmain{|\textit{main}|}| add the following code
to the top of the main file:
%
\begin{center}
\begin{tabular}{l}
|\||ifdefined\childdocname\endinput\||fi\newif\ifchilddoc|\\
|\edef\childdocname{\scantokens\expandafter{\jobname\noexpand}}|\\
|\def\childdocmain{|\textit{main}|}\||ifx\childdocmain\childdocname\||else|\\
|\childdoctrue\includeonly{\childdocname}\let\jobname\childdocmain\||fi|\\
\end{tabular}
\end{center}
%
Instead of |\childdocof{|\textit{main}|}| just include the main file
at the top of each child file:
%
\begin{center}
|\input{|\textit{main}|}|
\end{center}
%
A simple redirection |\childdocforward{|\textit{dest}|}| is achieved by:
%
\begin{center}
|\def\jobname{|\textit{dest}|}\input{\jobname}|
\end{center}
%
The redirection with prefix
|\childdocforwardprefix[|\textit{prefix}|]{|\textit{dest}|}|
is accomplished by:
%
\begin{center}
\begin{tabular}{l}
|{\edef\jobname{\scantokens\expandafter{\jobname\noexpand}}|\\
|\def\redirectjob |\textit{prefix}|#1~~~{\gdef\jobname{|\textit{dest}|#1}}|\\
|\expandafter\redirectjob\jobname~~~}\input{\jobname}|
\end{tabular}
\end{center}

In an alternative approach,
child documents can be compiled by a specific command line
without additional code or specific definitions:
%
\begin{center}
|... -jobname "|\textit{target}|" "|[\textit{flags}]%
|\includeonly{|\textit{dest}|}\input{|\textit{main}|}"|
\end{center}
%

%%%%%%%%%%%%%%%%%%%%%%%%%%%%%%%%%%%%%%%%%%%%%%%%%%%%%%%%%%%%%%%%%%%%%%%%%%%%%%%%
%%%%%%%%%%%%%%%%%%%%%%%%%%%%%%%%%%%%%%%%%%%%%%%%%%%%%%%%%%%%%%%%%%%%%%%%%%%%%%%%
\section{Information}

%%%%%%%%%%%%%%%%%%%%%%%%%%%%%%%%%%%%%%%%%%%%%%%%%%%%%%%%%%%%%%%%%%%%%%%%%%%%%%%%
\subsection{Copyright}

Copyright \copyright{} 2017--2018 Niklas Beisert

This work may be distributed and/or modified under the
conditions of the \LaTeX{} Project Public License, either version 1.3
of this license or (at your option) any later version.
The latest version of this license is in
  \url{http://www.latex-project.org/lppl.txt}
and version 1.3 or later is part of all distributions of \LaTeX{}
version 2005/12/01 or later.

This work has the LPPL maintenance status `maintained'.

The Current Maintainer of this work is Niklas Beisert.

This work consists of the files |README.txt|, |childdoc.ins| and |childdoc.dtx|
as well as the derived files |childdoc.def|, |cdocsamp.tex|
with |cdocsch1.tex|, |cdocsch2.tex|, |cdocspt3.tex|, |cdocspt4.tex|,
|cdocsdrf.tex|, |cdocsfn1.tex|, |cdocsfn2.tex|
as well as |childdoc.pdf|.

%%%%%%%%%%%%%%%%%%%%%%%%%%%%%%%%%%%%%%%%%%%%%%%%%%%%%%%%%%%%%%%%%%%%%%%%%%%%%%%%
\subsection{Files and Installation}

The package consists of the files:
%
\begin{center}
\begin{tabular}{ll}
    |README.txt|   & readme file \\
    |childdoc.ins| & installation file \\
    |childdoc.dtx| & source file \\
    |childdoc.def| & definition file \\
    |cdocsamp.tex| & sample main file \\
    |cdocsch1.tex| & sample include file \\
    |cdocsch2.tex| & sample include file \\
    |cdocspt3.tex| & sample part file \\
    |cdocspt4.tex| & sample part file \\
    |cdocsdrf.tex| & sample redirection file \\
    |cdocsfn1.tex| & sample redirection file \\
    |cdocsfn2.tex| & sample redirection file \\
    |childdoc.pdf| & manual
\end{tabular}
\end{center}
%
The distribution consists of the files
|README.txt|, |childdoc.ins| and |childdoc.dtx|.
%
\begin{itemize}
\item
Run (pdf)\LaTeX{} on |childdoc.dtx|
to compile the manual |childdoc.pdf| (this file).
\item
Run \LaTeX{} on |childdoc.ins| to create the definitions file |childdoc.def|
and the sample |cdocsamp.tex| with include files
|cdocsch1.tex|, |cdocsch2.tex|, |cdocspt3.tex|, |cdocspt4.tex|,
|cdocsdrf.tex|, |cdocsfn1.tex|, |cdocsfn2.tex|.
Then copy the file |childdoc.def| to an appropriate directory of your \LaTeX{}
distribution, e.g.\ \textit{texmf-root}|/tex/latex/childdoc|.
\end{itemize}

%%%%%%%%%%%%%%%%%%%%%%%%%%%%%%%%%%%%%%%%%%%%%%%%%%%%%%%%%%%%%%%%%%%%%%%%%%%%%%%%
\subsection{Related CTAN Packages}

There are several other packages which offer a similar functionality:
%
\begin{itemize}
\item
The packages
\href{http://ctan.org/pkg/docmute}{\textsf{docmute}},
\href{http://ctan.org/pkg/includex}{\textsf{includex}} and
\href{http://ctan.org/pkg/standalone}{\textsf{standalone}}
provide commands to include only the document body of
a child file thus allowing both files to be compiled individually.
\item
The packages \href{http://ctan.org/pkg/subdocs}{\textsf{subdocs}}
and \href{http://ctan.org/pkg/subfiles}{\textsf{subfiles}}
provide structures in which the main and child documents can be
encapsulated and allowing them to be compiled individually.
The inclusion mechanism is different from the conventional |\include|.
\item
The package \href{http://ctan.org/pkg/combine}{\textsf{combine}}
is an elaborate solution to combine several documents into one.
\end{itemize}
%
See also the CTAN topic \href{http://ctan.org/topic/subdocs}{\textsf{subdocs}}
for further related packages.
The present package differs from the above solutions in that
a document structure constructed with the conventional |\include| mechanism
just needs two extra commands at the top of every file
such that all constituent files can be compiled individually.

%%%%%%%%%%%%%%%%%%%%%%%%%%%%%%%%%%%%%%%%%%%%%%%%%%%%%%%%%%%%%%%%%%%%%%%%%%%%%%%%
%\subsection{Feature Suggestions}
%
%The following is a list of features which may be useful for future
%versions of this package:
%%
%\begin{itemize}
%\item
%\ldots
%\end{itemize}

%%%%%%%%%%%%%%%%%%%%%%%%%%%%%%%%%%%%%%%%%%%%%%%%%%%%%%%%%%%%%%%%%%%%%%%%%%%%%%%%
\subsection{Revision History}

%%%%%%%%%%%%%%%%%%%%%%%%%%%%%%%%%%%%%%%%
\paragraph{v2.0:} 2018/12/30

\begin{itemize}
\item
immediate forward processing
\item
added |\childdocby| mechanism
\item
manual restructured
\end{itemize}

%%%%%%%%%%%%%%%%%%%%%%%%%%%%%%%%%%%%%%%%
\paragraph{v1.6:} 2018/01/17

\begin{itemize}
\item
application for development of include files
\item
corrections to manual
\end{itemize}

%%%%%%%%%%%%%%%%%%%%%%%%%%%%%%%%%%%%%%%%
\paragraph{v1.5:} 2017/05/21

\begin{itemize}
\item
more complete structuring introduced
\item
|\childdocof| introduced
\item
|\childdoc| renamed to |\childdocmain|
\item
|\childredirect| renamed to |\childdocforward| and |\childdocforwardprefix|
and functionality expanded
\end{itemize}

%%%%%%%%%%%%%%%%%%%%%%%%%%%%%%%%%%%%%%%%
\paragraph{v1.0:} 2017/04/27

\begin{itemize}
\item
manual and install package
\item
first version published on CTAN
\end{itemize}

%%%%%%%%%%%%%%%%%%%%%%%%%%%%%%%%%%%%%%%%
\paragraph{v0.6:} 2017/04/26

\begin{itemize}
\item
redirection mechanism added
\end{itemize}

%%%%%%%%%%%%%%%%%%%%%%%%%%%%%%%%%%%%%%%%
\paragraph{v0.5:} 2017/04/26

\begin{itemize}
\item
functionality in definition file
\end{itemize}


%%%%%%%%%%%%%%%%%%%%%%%%%%%%%%%%%%%%%%%%%%%%%%%%%%%%%%%%%%%%%%%%%%%%%%%%%%%%%%%%
%%%%%%%%%%%%%%%%%%%%%%%%%%%%%%%%%%%%%%%%%%%%%%%%%%%%%%%%%%%%%%%%%%%%%%%%%%%%%%%%
%%%%%%%%%%%%%%%%%%%%%%%%%%%%%%%%%%%%%%%%%%%%%%%%%%%%%%%%%%%%%%%%%%%%%%%%%%%%%%%%
\appendix

\settowidth\MacroIndent{\rmfamily\scriptsize 000\ }

 \DocInput{childdoc.dtx}

\end{document}
%</driver>
% \fi
%
% %%%%%%%%%%%%%%%%%%%%%%%%%%%%%%%%%%%%%%%%%%%%%%%%%%%%%%%%%%%%%%%%%%%%%%%%%%%%%%
% %%%%%%%%%%%%%%%%%%%%%%%%%%%%%%%%%%%%%%%%%%%%%%%%%%%%%%%%%%%%%%%%%%%%%%%%%%%%%%
% \section{Sample}
%\iffalse
%<*samplemain>
%\fi
%
% The following presents a sample document
% with two chapters, two parts, a title page,
% a compile flag as well as three forwarding files to set the flag.
% It consists of eight |.tex| files:
% \begin{center}
% \begin{tabular}{ll}
% |cdocsamp.tex|&main file\\
% |cdocsch1.tex|&include file for chapter 1\\
% |cdocsch2.tex|&include file for chapter 2\\
% |cdocspt3.tex|&include file for part 3\\
% |cdocspt4.tex|&include file for part 4\\
% |cdocsdrf.tex|&forwarding file for main file in draft mode\\
% |cdocsfi1.tex|&forwarding file for final version of chapter 1\\
% |cdocsfi2.tex|&forwarding file for final version of chapter 2\\
% \end{tabular}
% \end{center}
% Each of the eight files can be compiled directly by the \LaTeX{} compiler.
%
% %%%%%%%%%%%%%%%%%%%%%%%%%%%%%%%%%%%%%%
% \paragraph{Main File.}
%
% The main file is called |cdocsamp.tex|.
%
% Load the \textsf{childdoc} definitions and
% declare the filename for the main document:
%    \begin{macrocode}
\input{childdoc.def}
\childdocmain{}
%    \end{macrocode}

% Optional override for |\version| flag:
%    \begin{macrocode}
%%\ifchilddoc\else\providecommand{\version}{draft}\fi
%    \end{macrocode}

% Define the default values for the |\version| flag
% (|final| for the main file and |draft| for childs):
%    \begin{macrocode}
\ifchilddoc
\providecommand{\version}{draft}
\else
\providecommand{\version}{final}
\fi
%    \end{macrocode}

% Load the standard document class:
%    \begin{macrocode}
\documentclass[12pt]{article}
%    \end{macrocode}

% Start the document body:
%    \begin{macrocode}
\begin{document}
%    \end{macrocode}

% Declare a title page.
% Print title, part of document being processed and version flag:
%    \begin{macrocode}
\addtocounter{page}{-1}
\begin{center}
{\LARGE\bfseries{}childdoc example\par}
\vspace{1cm}
\ifchilddoc
\ifchilddocmanual part\else chapter\fi:
`\childdocname' of `\childdocjob'\par
\else
main document: `\childdocjob'\par
\fi
version: \version\par
\end{center}
\newpage
%    \end{macrocode}

% Manually include selected file,
% otherwise process as usual:
%    \begin{macrocode}
\ifchilddocmanual
\section*{part `\childdocname'}
\input{\childdocname}
\else
%    \end{macrocode}

% Include the two chapters:
%    \begin{macrocode}
\include{cdocsch1}
\include{cdocsch2}
%    \end{macrocode}

% Include the two parts unless only chapters should be displayed:
%    \begin{macrocode}
\ifchilddoc\else
\section{part three}
\input{cdocspt3}
\section{part four}
\input{cdocspt4}
\fi
%    \end{macrocode}

% Process as usual until here:
%    \begin{macrocode}
\fi
%    \end{macrocode}

% End of document body:
%    \begin{macrocode}
\end{document}
%    \end{macrocode}
%\iffalse
%</samplemain>
%\fi
%
% %%%%%%%%%%%%%%%%%%%%%%%%%%%%%%%%%%%%%%
% \paragraph{Chapter Include Files.}
%
% The include files are called |cdocsch1.tex| and |cdocsch2.tex|.
%
%\iffalse
%<*samplechap1|samplechap2>
%\fi

% Optional override for |\version| flag:
%    \begin{macrocode}
%%\providecommand{\version}{final}
%    \end{macrocode}

% Include the main document:
%    \begin{macrocode}
\input{childdoc.def}
\childdocof{cdocsamp}
%    \end{macrocode}

%\iffalse
%</samplechap1|samplechap2>
%\fi
%
%\iffalse
%<*samplechap1>
%\fi
% Some text for chapter 1:
%    \begin{macrocode}
\section{one}
some text in chapter one
%    \end{macrocode}

%\iffalse
%</samplechap1>
%\fi
% Some text for chapter 2:
%\iffalse
%<*samplechap2>
%\fi
%    \begin{macrocode}
\section{two}
more text in chapter two
%    \end{macrocode}

%\iffalse
%</samplechap2>
%\fi
%
% %%%%%%%%%%%%%%%%%%%%%%%%%%%%%%%%%%%%%%
% \paragraph{Part Include Files.}
%
% The include files are called |cdocspt3.tex| and |cdocspt4.tex|.
%
%\iffalse
%<*samplepart3|samplepart4>
%\fi

% Optional override for |\version| flag:
%    \begin{macrocode}
%%\providecommand{\version}{final}
%    \end{macrocode}

% Include the main document:
%    \begin{macrocode}
\input{childdoc.def}
\childdocby{cdocsamp}
%    \end{macrocode}

%\iffalse
%</samplepart3|samplepart4>
%\fi
%
%\iffalse
%<*samplepart3>
%\fi
% Some text for part 3:
%    \begin{macrocode}
some text in part three
%    \end{macrocode}

%\iffalse
%</samplepart3>
%\fi
% Some text for part 4:
%\iffalse
%<*samplepart4>
%\fi
%    \begin{macrocode}
more text in part four
%    \end{macrocode}

%\iffalse
%</samplepart4>
%\fi
%
% %%%%%%%%%%%%%%%%%%%%%%%%%%%%%%%%%%%%%%
% \paragraph{Forwarding for a Complete Draft.}
%
% The following forwarding file |cdocsdrf.tex|
% compiles the main document in draft mode:
%\iffalse
%<*sampledraft>
%\fi
%    \begin{macrocode}
\def\version{draft}
\input{childdoc.def}
\childdocforward{cdocsamp}
%    \end{macrocode}

%\iffalse
%</sampledraft>
%\fi
%
% %%%%%%%%%%%%%%%%%%%%%%%%%%%%%%%%%%%%%%
% \paragraph{Forwarding for Final Version of the Chapters.}
%
% The following forwarding files |cdocsfn1.tex| and |cdocsfn2.tex|
% (with identical content)
% compile the final versions of the child documents
% |cdocsch1.tex| and |cdocsch2.tex|, respectively:
%\iffalse
%<*samplefinal>
%\fi
%    \begin{macrocode}
\def\version{final}
\input{childdoc.def}
\childdocforwardprefix[cdocsamp]{cdocsfn}{cdocsch}
%    \end{macrocode}

%\iffalse
%</samplefinal>
%\fi
%
% %%%%%%%%%%%%%%%%%%%%%%%%%%%%%%%%%%%%%%
% \paragraph{Command Line Processing.}
%
% The following three command lines generate the output files
% |cdocscld|, |cdocscl1| and |cdocscl2|
% which should be identical to
% |cdocsdrf|, |cdocsch1| and |cdocsfn2|, respectively:
% \begin{center}
% \begin{tabular}{l}
% |latex -jobname cdocscld \|\\
% |  "\def\version{draft}\input{childdoc.def}\childdocforward{cdocsamp}"|\\
% |latex -jobname cdocscl1 \|\\
% |  "\input{childdoc.def}\childdocforward[cdocsamp]{cdocsch1}"|\\
% |latex -jobname cdocscl2 \|\\
% |  "\def\version{final}\input{childdoc.def}\childdocforward{cdocsch2}"|
% \end{tabular}
% \end{center}
% Note that the trailing backslash on each first line
% merely continues the input to the second line
% (for convenient cut ant paste).
% Furthermore, the command |latex| can be replaced by any
% of its alternative versions such as |pdflatex|.
%
% %%%%%%%%%%%%%%%%%%%%%%%%%%%%%%%%%%%%%%%%%%%%%%%%%%%%%%%%%%%%%%%%%%%%%%%%%%%%%%
% %%%%%%%%%%%%%%%%%%%%%%%%%%%%%%%%%%%%%%%%%%%%%%%%%%%%%%%%%%%%%%%%%%%%%%%%%%%%%%
% \section{Implementation}
%\iffalse
%<*package>
%\fi
%
% This section describes the definitions file |childdoc.def|.

% The definitions cannot be loaded using |\usepackage| or |\RequirePackage|
% which has a mechanism to prevent loading a style file more than once.
% When loading the definitions by means of |\input|
% multiple instances have to be prevented manually:
%\iffalse
%This code needs to be before the `\ProvidesFile' directive
%which is defined at the beginning of this file.
%Therefore it is also placed there and commented out here.
%</package>
%<*discard>
%\fi
%    \begin{macrocode}
\ifdefined\childdocmain\endinput\fi
%    \end{macrocode}
%\iffalse
%</discard>
%<*package>
%\fi
%
% \macro{\ifchilddoc}
% \macro{\ifchilddocmanual}
% The conditional |\ifchilddoc| tells whether a
% child (true) or main (false) document is being compiled.
% The conditional |\ifchilddocmanual| tells whether
% the |\includeonly| mechanism is used (false) or
% the selection of child files must be performed manually (true).
% The definitions initialise to false:
%    \begin{macrocode}
\newif\ifchilddoc
\newif\ifchilddocmanual
%    \end{macrocode}

% \macro{\childdocname}
% \macro{\childdocjob}
% The macro |\childdocname| stores the name of the main document
% to be compiled. The macro |\childdocjob| stores the name of
% the document on which the \LaTeX{} compiler was originally invoked.
% The content of |\jobname| cannot be compared
% to filenames specified in the source due to different catcodes.
% The following code rescans |\jobname|, stores the result
% in |\childdocname| and saves a copy in |\childdocjob|:
%    \begin{macrocode}
\edef\childdocname{\scantokens\expandafter{\jobname\noexpand}}
\let\childdocjob\childdocname
%    \end{macrocode}

% \macro{\childdocdisable}
% The macro |\childdocdisable| prevents the main file
% from being processed more than once.
% At this stage, the main document command |\childdocmain|
% is assumed to be called once again where it should do nothing.
% Any subsequent call to it should prevent
% a secondary processing of the main document
% It overwrites the forwarding commands
% |\childdocof| and |\childdocforward|
% with empty macros to prevent further inclusions of the main document:
%    \begin{macrocode}
\newcommand{\childdocdisable}
{
  \renewcommand{\childdocmain}[1]{\renewcommand{\childdocmain}[1]{\endinput}}
  \renewcommand{\childdocof}[1]{}
  \renewcommand{\childdocby}[2][]{}
  \renewcommand{\childdocforward}[2][]{}
  \renewcommand{\childdocdisable}{}
}
%    \end{macrocode}

% \macro{\childdocmain}
% The macro |\childdocmain| is to be called at the top of the main file
% with nothing or the main filename (without extension) as argument.
% First, it breaks loops.
% If the argument is not empty and does not match |\childdocname|
% (which is set by the first inclusion of |childdoc.def|),
% |\ifchilddoc| is set to true, |\includeonly| is applied to the child file
% and |\jobname| is set to the main file
% (for proper handling of |.aux| files):
%    \begin{macrocode}
\newcommand{\childdocmain}[1]
{
  \childdocdisable\childdocmain{}
  \if?#1?\else
    \begingroup
      \def\childdoctmp{#1}
      \ifx\childdoctmp\childdocname
        \def\childdoctmp{}
      \else
        \def\childdoctmp
        {
          \childdoctrue
          \includeonly{\childdocname}
          \def\childdocjob{#1}
          \def\jobname{#1}
        }
      \fi
      \expandafter
    \endgroup
    \childdoctmp
  \fi
}
%    \end{macrocode}

% \macro{\childdocof}
% The command |\childdocof| redirects
% compilation to the main file |#1|.
%    \begin{macrocode}
\newcommand{\childdocof}[1]
{
  \childdocdisable
  \childdoctrue
  \includeonly{\childdocname}
  \def\jobname{#1}
  \def\childdocjob{#1}
  \input{#1}
}
%    \end{macrocode}

% \macro{\childdocby}
% The command |\childdocby| ....
%    \begin{macrocode}
\newcommand{\childdocby}[2][]
{
  \childdocdisable
  \childdoctrue
  \childdocmanualtrue
  \if?#1?\else
    \def\jobname{#2}
  \fi
  \def\childdocjob{#2}
  \input{#2}
  \endinput
}
%    \end{macrocode}

% \macro{\childdocforward}
% The command |\childdocforward| redirects
% compilation to the main file or
% (if the optional argument is given) a child file.
% Parameters are set as if the main file
% or a child file starting with |\childdocof| was compiled.
% Then compilation is handed over to the main file:
%    \begin{macrocode}
\newcommand{\childdocforward}[2][]
{
  \begingroup
    \if?#1?
      \def\childdoctmp
      {
        \def\childdocname{#2}
        \def\childdocjob{#2}
        \def\jobname{#2}
        \input{#2}
        \endinput
      }
    \else
      \def\childdoctmp
      {
        \childdocdisable
        \def\childdocname{#2}
        \childdoctrue
        \includeonly{#2}
        \def\childdocjob{#1}
        \def\jobname{#1}
        \input{#1}
        \endinput
      }
    \fi
    \expandafter
  \endgroup
  \childdoctmp
}
%    \end{macrocode}

% \macro{\childdocforwardprefix}
% The command |\childdocforwardprefix| redirects
% compilation to the main or a child file by means of a pattern.
% The prefix |#1| in the current filename is replaced by |#2|
% and the suffix of the current filename is kept
% (it is assumed that the filename does not contain the substring `|~~~|'
% which is used as a delimiter).
% Compilation is handed over to the new file by |\childdocforward|:
%    \begin{macrocode}
\newcommand{\childdocforwardprefix}[3][]
{
  \begingroup
    \def\childdocextract #2##1~~~{\def\childdoctmp{\childdocforward[#1]{#3##1}}}
    \expandafter\childdocextract\childdocname~~~
    \expandafter
  \endgroup
  \childdoctmp
}
%    \end{macrocode}

% \macro{\childdoc}
% The deprecated macro |\childdoc| is a legacy version of |\childdocmain|:
%    \begin{macrocode}
\newcommand{\childdoc}{\childdocmain}
%    \end{macrocode}

% \macro{\childdocredirect}
% The deprecated macro |\childdocredirect| is a legacy version
% of |\childdocforward| and |\childdocforwardprefix|:
%    \begin{macrocode}
\newcommand{\childdocredirect}[2][]
{
  \begingroup
    \if?#1?
      \def\childdoctmp{\childdocforward{#2}}
    \else
      \def\childdoctmp{\childdocforwardprefix{#1}{#2}}
    \fi
    \expandafter
  \endgroup
  \childdoctmp
}
%    \end{macrocode}

%\iffalse
%</package>
%\fi
%
\endinput
\childdocforward{cdocsch2}"|
% \end{tabular}
% \end{center}
% Note that the trailing backslash on each first line
% merely continues the input to the second line
% (for convenient cut ant paste).
% Furthermore, the command |latex| can be replaced by any
% of its alternative versions such as |pdflatex|.
%
% %%%%%%%%%%%%%%%%%%%%%%%%%%%%%%%%%%%%%%%%%%%%%%%%%%%%%%%%%%%%%%%%%%%%%%%%%%%%%%
% %%%%%%%%%%%%%%%%%%%%%%%%%%%%%%%%%%%%%%%%%%%%%%%%%%%%%%%%%%%%%%%%%%%%%%%%%%%%%%
% \section{Implementation}
%\iffalse
%<*package>
%\fi
%
% This section describes the definitions file |childdoc.def|.

% The definitions cannot be loaded using |\usepackage| or |\RequirePackage|
% which has a mechanism to prevent loading a style file more than once.
% When loading the definitions by means of |\input|
% multiple instances have to be prevented manually:
%\iffalse
%This code needs to be before the `\ProvidesFile' directive
%which is defined at the beginning of this file.
%Therefore it is also placed there and commented out here.
%</package>
%<*discard>
%\fi
%    \begin{macrocode}
\ifdefined\childdocmain\endinput\fi
%    \end{macrocode}
%\iffalse
%</discard>
%<*package>
%\fi
%
% \macro{\ifchilddoc}
% \macro{\ifchilddocmanual}
% The conditional |\ifchilddoc| tells whether a
% child (true) or main (false) document is being compiled.
% The conditional |\ifchilddocmanual| tells whether
% the |\includeonly| mechanism is used (false) or
% the selection of child files must be performed manually (true).
% The definitions initialise to false:
%    \begin{macrocode}
\newif\ifchilddoc
\newif\ifchilddocmanual
%    \end{macrocode}

% \macro{\childdocname}
% \macro{\childdocjob}
% The macro |\childdocname| stores the name of the main document
% to be compiled. The macro |\childdocjob| stores the name of
% the document on which the \LaTeX{} compiler was originally invoked.
% The content of |\jobname| cannot be compared
% to filenames specified in the source due to different catcodes.
% The following code rescans |\jobname|, stores the result
% in |\childdocname| and saves a copy in |\childdocjob|:
%    \begin{macrocode}
\edef\childdocname{\scantokens\expandafter{\jobname\noexpand}}
\let\childdocjob\childdocname
%    \end{macrocode}

% \macro{\childdocdisable}
% The macro |\childdocdisable| prevents the main file
% from being processed more than once.
% At this stage, the main document command |\childdocmain|
% is assumed to be called once again where it should do nothing.
% Any subsequent call to it should prevent
% a secondary processing of the main document
% It overwrites the forwarding commands
% |\childdocof| and |\childdocforward|
% with empty macros to prevent further inclusions of the main document:
%    \begin{macrocode}
\newcommand{\childdocdisable}
{
  \renewcommand{\childdocmain}[1]{\renewcommand{\childdocmain}[1]{\endinput}}
  \renewcommand{\childdocof}[1]{}
  \renewcommand{\childdocby}[2][]{}
  \renewcommand{\childdocforward}[2][]{}
  \renewcommand{\childdocdisable}{}
}
%    \end{macrocode}

% \macro{\childdocmain}
% The macro |\childdocmain| is to be called at the top of the main file
% with nothing or the main filename (without extension) as argument.
% First, it breaks loops.
% If the argument is not empty and does not match |\childdocname|
% (which is set by the first inclusion of |childdoc.def|),
% |\ifchilddoc| is set to true, |\includeonly| is applied to the child file
% and |\jobname| is set to the main file
% (for proper handling of |.aux| files):
%    \begin{macrocode}
\newcommand{\childdocmain}[1]
{
  \childdocdisable\childdocmain{}
  \if?#1?\else
    \begingroup
      \def\childdoctmp{#1}
      \ifx\childdoctmp\childdocname
        \def\childdoctmp{}
      \else
        \def\childdoctmp
        {
          \childdoctrue
          \includeonly{\childdocname}
          \def\childdocjob{#1}
          \def\jobname{#1}
        }
      \fi
      \expandafter
    \endgroup
    \childdoctmp
  \fi
}
%    \end{macrocode}

% \macro{\childdocof}
% The command |\childdocof| redirects
% compilation to the main file |#1|.
%    \begin{macrocode}
\newcommand{\childdocof}[1]
{
  \childdocdisable
  \childdoctrue
  \includeonly{\childdocname}
  \def\jobname{#1}
  \def\childdocjob{#1}
  \input{#1}
}
%    \end{macrocode}

% \macro{\childdocby}
% The command |\childdocby| ....
%    \begin{macrocode}
\newcommand{\childdocby}[2][]
{
  \childdocdisable
  \childdoctrue
  \childdocmanualtrue
  \if?#1?\else
    \def\jobname{#2}
  \fi
  \def\childdocjob{#2}
  \input{#2}
  \endinput
}
%    \end{macrocode}

% \macro{\childdocforward}
% The command |\childdocforward| redirects
% compilation to the main file or
% (if the optional argument is given) a child file.
% Parameters are set as if the main file
% or a child file starting with |\childdocof| was compiled.
% Then compilation is handed over to the main file:
%    \begin{macrocode}
\newcommand{\childdocforward}[2][]
{
  \begingroup
    \if?#1?
      \def\childdoctmp
      {
        \def\childdocname{#2}
        \def\childdocjob{#2}
        \def\jobname{#2}
        \input{#2}
        \endinput
      }
    \else
      \def\childdoctmp
      {
        \childdocdisable
        \def\childdocname{#2}
        \childdoctrue
        \includeonly{#2}
        \def\childdocjob{#1}
        \def\jobname{#1}
        \input{#1}
        \endinput
      }
    \fi
    \expandafter
  \endgroup
  \childdoctmp
}
%    \end{macrocode}

% \macro{\childdocforwardprefix}
% The command |\childdocforwardprefix| redirects
% compilation to the main or a child file by means of a pattern.
% The prefix |#1| in the current filename is replaced by |#2|
% and the suffix of the current filename is kept
% (it is assumed that the filename does not contain the substring `|~~~|'
% which is used as a delimiter).
% Compilation is handed over to the new file by |\childdocforward|:
%    \begin{macrocode}
\newcommand{\childdocforwardprefix}[3][]
{
  \begingroup
    \def\childdocextract #2##1~~~{\def\childdoctmp{\childdocforward[#1]{#3##1}}}
    \expandafter\childdocextract\childdocname~~~
    \expandafter
  \endgroup
  \childdoctmp
}
%    \end{macrocode}

% \macro{\childdoc}
% The deprecated macro |\childdoc| is a legacy version of |\childdocmain|:
%    \begin{macrocode}
\newcommand{\childdoc}{\childdocmain}
%    \end{macrocode}

% \macro{\childdocredirect}
% The deprecated macro |\childdocredirect| is a legacy version
% of |\childdocforward| and |\childdocforwardprefix|:
%    \begin{macrocode}
\newcommand{\childdocredirect}[2][]
{
  \begingroup
    \if?#1?
      \def\childdoctmp{\childdocforward{#2}}
    \else
      \def\childdoctmp{\childdocforwardprefix{#1}{#2}}
    \fi
    \expandafter
  \endgroup
  \childdoctmp
}
%    \end{macrocode}

%\iffalse
%</package>
%\fi
%
\endinput

\childdocforward{cdocsamp}
%    \end{macrocode}

%\iffalse
%</sampledraft>
%\fi
%
% %%%%%%%%%%%%%%%%%%%%%%%%%%%%%%%%%%%%%%
% \paragraph{Forwarding for Final Version of the Chapters.}
%
% The following forwarding files |cdocsfn1.tex| and |cdocsfn2.tex|
% (with identical content)
% compile the final versions of the child documents
% |cdocsch1.tex| and |cdocsch2.tex|, respectively:
%\iffalse
%<*samplefinal>
%\fi
%    \begin{macrocode}
\def\version{final}
% \iffalse
%
% childdoc.dtx Copyright (C) 2017-2018 Niklas Beisert
%
% This work may be distributed and/or modified under the
% conditions of the LaTeX Project Public License, either version 1.3
% of this license or (at your option) any later version.
% The latest version of this license is in
%   http://www.latex-project.org/lppl.txt
% and version 1.3 or later is part of all distributions of LaTeX
% version 2005/12/01 or later.
%
% This work has the LPPL maintenance status `maintained'.
%
% The Current Maintainer of this work is Niklas Beisert.
%
% This work consists of the files childdoc.dtx and childdoc.ins
% and the derived files childdoc.def and cdocsamp.tex with
% cdocsch1.tex, cdocsch2.tex, cdocsdrf.tex, cdocsfn1.tex, cdocsfn2.tex.
%
%<package>\ifdefined\childdocmain\endinput\fi
%<package>\ProvidesFile{childdoc.def}[2018/12/30 v2.0 child document driver]
%<samplemain>\ProvidesFile{cdocsamp.tex}[2018/12/30 v2.0 sample for childdoc]
%<*driver>
%\ProvidesFile{childdoc.drv}[2018/12/30 v2.0 childdoc reference manual file]
\PassOptionsToClass{10pt,a4paper}{article}
\documentclass{ltxdoc}

\usepackage[margin=35mm]{geometry}
\usepackage{hyperref}
\usepackage{hyperxmp}
\usepackage[usenames]{color}

\hypersetup{colorlinks=true}
\hypersetup{pdfstartview=FitH}
\hypersetup{pdfpagemode=UseNone}
\hypersetup{pdfsource={}}
\hypersetup{pdflang={en-UK}}
\hypersetup{pdfcopyright={Copyright 2017-2018 Niklas Beisert.
  This work may be distributed and/or modified under the
  conditions of the LaTeX Project Public License, either version 1.3
  of this license or (at your option) any later version.}}
\hypersetup{pdflicenseurl={http://www.latex-project.org/lppl.txt}}
\hypersetup{pdfcontactaddress={ETH Zurich, ITP, HIT K,
  Wolfgang-Pauli-Strasse 27}}
\hypersetup{pdfcontactpostcode={8093}}
\hypersetup{pdfcontactcity={Zurich}}
\hypersetup{pdfcontactcountry={Switzerland}}
\hypersetup{pdfcontactemail={nbeisert@itp.phys.ethz.ch}}
\hypersetup{pdfcontacturl={http://people.phys.ethz.ch/\xmptilde nbeisert/}}

\newcommand{\secref}[1]{\hyperref[#1]{section \ref*{#1}}}

\parskip1ex
\parindent0pt
\let\olditemize\itemize
\def\itemize{\olditemize\parskip0pt}

\begin{document}

\title{The \textsf{childdoc} Package}
\hypersetup{pdftitle={The childdoc Package}}
\author{Niklas Beisert\\[2ex]
  Institut f\"ur Theoretische Physik\\
  Eidgen\"ossische Technische Hochschule Z\"urich\\
  Wolfgang-Pauli-Strasse 27, 8093 Z\"urich, Switzerland\\[1ex]
  \href{mailto:nbeisert@itp.phys.ethz.ch}
  {\texttt{nbeisert@itp.phys.ethz.ch}}}
\hypersetup{pdfauthor={Niklas Beisert}}
\hypersetup{pdfsubject={Manual for the LaTeX2e Package childdoc}}
\date{30 December 2018, \textsf{v2.0}}
\maketitle

\begin{abstract}\noindent
\textsf{childdoc} is a \LaTeXe{} package
that enables the direct compilation
of document sections included by |\include|
to individual files.
\end{abstract}

\begingroup
\parskip0ex
\tableofcontents
\endgroup

%%%%%%%%%%%%%%%%%%%%%%%%%%%%%%%%%%%%%%%%%%%%%%%%%%%%%%%%%%%%%%%%%%%%%%%%%%%%%%%%
%%%%%%%%%%%%%%%%%%%%%%%%%%%%%%%%%%%%%%%%%%%%%%%%%%%%%%%%%%%%%%%%%%%%%%%%%%%%%%%%
\section{Introduction}

\LaTeX{} provides a mechanism to structure a large document (such as a book)
into a main file and several child files (containing the chapters)
using the |\include| command.
This mechanism is beneficial for documents
which span hundreds of pages in order to
make the source file(s) more manageable.
Moreover, compilation can be restricted to
selected child files by means of the |\includeonly| command.
The latter feature can be used to reduce the compilation time while editing
(this was significantly more useful in the earlier days of \LaTeX{})
or to generate a smaller document which is easier to navigate.
Another application of |\includeonly| is to generate
documents consisting of selected parts of the complete document.

However, there are a few drawbacks of the plain |\include| mechanism:
\begin{itemize}
\item
The child files cannot be compiled on their own,
they can only be compiled via the main file.
A naive editing environment
(such as a text editor with an option
to have the current file processed by \LaTeX)
may require one to switch to the main file before compiling;
attempting to compile the child file produces errors.
\item
The main file must be modified (each time)
to adjust the |\includeonly| command
to the present needs. This easily leaves the main file in a messy state.
\item
The generated document will always carry the filename
of the main document. This is inconvenient if
several child files are to be compiled and
to be kept for distribution.
\end{itemize}

The present package provides a simple interface
to make child files individually compilable by \LaTeX{}.
Compiling a child file then has the same effect as compiling
the main file with an |\includeonly| command
to select the appropriate child.
Moreover the generated document will carry the name of the child
rather than the main file.
This resolves all three above issues.

This feature is meant to make the editing of books,
thesis documents and lecture notes somewhat more convenient.
However, the package can also be used efficiently for
composing a series of documents (such as exercise sheets)
which are typically distributed individually.
It then assists the author in generating the individual documents
(potentially in different versions)
as well as a document containing the collected series.
Another application is in developing style files
or other kinds of included material
where compilation of the style file could redirect
to a sample or test file.

%%%%%%%%%%%%%%%%%%%%%%%%%%%%%%%%%%%%%%%%%%%%%%%%%%%%%%%%%%%%%%%%%%%%%%%%%%%%%%%%
%%%%%%%%%%%%%%%%%%%%%%%%%%%%%%%%%%%%%%%%%%%%%%%%%%%%%%%%%%%%%%%%%%%%%%%%%%%%%%%%
\section{Usage}

First of all, the package \textsf{childdoc} is \emph{not} a standard
\LaTeXe{} |.sty| style file! Therefore it needs to be invoked in
a non-standard way.

%%%%%%%%%%%%%%%%%%%%%%%%%%%%%%%%%%%%%%%%%%%%%%%%%%%%%%%%%%%%%%%%%%%%%%%%%%%%%%%%
\subsection{Included Files}
\label{sec:include}

%%%%%%%%%%%%%%%%%%%%%%%%%%%%%%%%%%%%%%%%
\DescribeMacro{\childdocmain}
To use the package, add the commands
\begin{center}
\begin{tabular}{l}
|% \iffalse
%
% childdoc.dtx Copyright (C) 2017-2018 Niklas Beisert
%
% This work may be distributed and/or modified under the
% conditions of the LaTeX Project Public License, either version 1.3
% of this license or (at your option) any later version.
% The latest version of this license is in
%   http://www.latex-project.org/lppl.txt
% and version 1.3 or later is part of all distributions of LaTeX
% version 2005/12/01 or later.
%
% This work has the LPPL maintenance status `maintained'.
%
% The Current Maintainer of this work is Niklas Beisert.
%
% This work consists of the files childdoc.dtx and childdoc.ins
% and the derived files childdoc.def and cdocsamp.tex with
% cdocsch1.tex, cdocsch2.tex, cdocsdrf.tex, cdocsfn1.tex, cdocsfn2.tex.
%
%<package>\ifdefined\childdocmain\endinput\fi
%<package>\ProvidesFile{childdoc.def}[2018/12/30 v2.0 child document driver]
%<samplemain>\ProvidesFile{cdocsamp.tex}[2018/12/30 v2.0 sample for childdoc]
%<*driver>
%\ProvidesFile{childdoc.drv}[2018/12/30 v2.0 childdoc reference manual file]
\PassOptionsToClass{10pt,a4paper}{article}
\documentclass{ltxdoc}

\usepackage[margin=35mm]{geometry}
\usepackage{hyperref}
\usepackage{hyperxmp}
\usepackage[usenames]{color}

\hypersetup{colorlinks=true}
\hypersetup{pdfstartview=FitH}
\hypersetup{pdfpagemode=UseNone}
\hypersetup{pdfsource={}}
\hypersetup{pdflang={en-UK}}
\hypersetup{pdfcopyright={Copyright 2017-2018 Niklas Beisert.
  This work may be distributed and/or modified under the
  conditions of the LaTeX Project Public License, either version 1.3
  of this license or (at your option) any later version.}}
\hypersetup{pdflicenseurl={http://www.latex-project.org/lppl.txt}}
\hypersetup{pdfcontactaddress={ETH Zurich, ITP, HIT K,
  Wolfgang-Pauli-Strasse 27}}
\hypersetup{pdfcontactpostcode={8093}}
\hypersetup{pdfcontactcity={Zurich}}
\hypersetup{pdfcontactcountry={Switzerland}}
\hypersetup{pdfcontactemail={nbeisert@itp.phys.ethz.ch}}
\hypersetup{pdfcontacturl={http://people.phys.ethz.ch/\xmptilde nbeisert/}}

\newcommand{\secref}[1]{\hyperref[#1]{section \ref*{#1}}}

\parskip1ex
\parindent0pt
\let\olditemize\itemize
\def\itemize{\olditemize\parskip0pt}

\begin{document}

\title{The \textsf{childdoc} Package}
\hypersetup{pdftitle={The childdoc Package}}
\author{Niklas Beisert\\[2ex]
  Institut f\"ur Theoretische Physik\\
  Eidgen\"ossische Technische Hochschule Z\"urich\\
  Wolfgang-Pauli-Strasse 27, 8093 Z\"urich, Switzerland\\[1ex]
  \href{mailto:nbeisert@itp.phys.ethz.ch}
  {\texttt{nbeisert@itp.phys.ethz.ch}}}
\hypersetup{pdfauthor={Niklas Beisert}}
\hypersetup{pdfsubject={Manual for the LaTeX2e Package childdoc}}
\date{30 December 2018, \textsf{v2.0}}
\maketitle

\begin{abstract}\noindent
\textsf{childdoc} is a \LaTeXe{} package
that enables the direct compilation
of document sections included by |\include|
to individual files.
\end{abstract}

\begingroup
\parskip0ex
\tableofcontents
\endgroup

%%%%%%%%%%%%%%%%%%%%%%%%%%%%%%%%%%%%%%%%%%%%%%%%%%%%%%%%%%%%%%%%%%%%%%%%%%%%%%%%
%%%%%%%%%%%%%%%%%%%%%%%%%%%%%%%%%%%%%%%%%%%%%%%%%%%%%%%%%%%%%%%%%%%%%%%%%%%%%%%%
\section{Introduction}

\LaTeX{} provides a mechanism to structure a large document (such as a book)
into a main file and several child files (containing the chapters)
using the |\include| command.
This mechanism is beneficial for documents
which span hundreds of pages in order to
make the source file(s) more manageable.
Moreover, compilation can be restricted to
selected child files by means of the |\includeonly| command.
The latter feature can be used to reduce the compilation time while editing
(this was significantly more useful in the earlier days of \LaTeX{})
or to generate a smaller document which is easier to navigate.
Another application of |\includeonly| is to generate
documents consisting of selected parts of the complete document.

However, there are a few drawbacks of the plain |\include| mechanism:
\begin{itemize}
\item
The child files cannot be compiled on their own,
they can only be compiled via the main file.
A naive editing environment
(such as a text editor with an option
to have the current file processed by \LaTeX)
may require one to switch to the main file before compiling;
attempting to compile the child file produces errors.
\item
The main file must be modified (each time)
to adjust the |\includeonly| command
to the present needs. This easily leaves the main file in a messy state.
\item
The generated document will always carry the filename
of the main document. This is inconvenient if
several child files are to be compiled and
to be kept for distribution.
\end{itemize}

The present package provides a simple interface
to make child files individually compilable by \LaTeX{}.
Compiling a child file then has the same effect as compiling
the main file with an |\includeonly| command
to select the appropriate child.
Moreover the generated document will carry the name of the child
rather than the main file.
This resolves all three above issues.

This feature is meant to make the editing of books,
thesis documents and lecture notes somewhat more convenient.
However, the package can also be used efficiently for
composing a series of documents (such as exercise sheets)
which are typically distributed individually.
It then assists the author in generating the individual documents
(potentially in different versions)
as well as a document containing the collected series.
Another application is in developing style files
or other kinds of included material
where compilation of the style file could redirect
to a sample or test file.

%%%%%%%%%%%%%%%%%%%%%%%%%%%%%%%%%%%%%%%%%%%%%%%%%%%%%%%%%%%%%%%%%%%%%%%%%%%%%%%%
%%%%%%%%%%%%%%%%%%%%%%%%%%%%%%%%%%%%%%%%%%%%%%%%%%%%%%%%%%%%%%%%%%%%%%%%%%%%%%%%
\section{Usage}

First of all, the package \textsf{childdoc} is \emph{not} a standard
\LaTeXe{} |.sty| style file! Therefore it needs to be invoked in
a non-standard way.

%%%%%%%%%%%%%%%%%%%%%%%%%%%%%%%%%%%%%%%%%%%%%%%%%%%%%%%%%%%%%%%%%%%%%%%%%%%%%%%%
\subsection{Included Files}
\label{sec:include}

%%%%%%%%%%%%%%%%%%%%%%%%%%%%%%%%%%%%%%%%
\DescribeMacro{\childdocmain}
To use the package, add the commands
\begin{center}
\begin{tabular}{l}
|\input{childdoc.def}|\\
|\childdocmain{}|\\
\end{tabular}
\end{center}
at the very top of the main \LaTeX{} file,
in particular \emph{before} the |\documentclass| statement!
The argument of |\childdocmain| should be left empty
(but it must be present).

%%%%%%%%%%%%%%%%%%%%%%%%%%%%%%%%%%%%%%%%
\DescribeMacro{\childdocof}
Furthermore, add the commands
\begin{center}
\begin{tabular}{l}
|\input{childdoc.def}|\\
|\childdocof{|\textit{main}|}|\\
\end{tabular}
\end{center}
at the top of every child file \textit{child}
which is included by |\include{|\textit{child}|}|
from within the main file
(or at least for those files to be compiled individually).
The argument \textit{main} must be the filename of the main file.

There are a couple of
considerations in setting up the main and child documents:

%%%%%%%%%%%%%%%%%%%%%%%%%%%%%%%%%%%%%%%%
\paragraph{Restrictions.}

Please note the following restrictions:
\begin{itemize}
\item
|\childdocmain| must be called with one argument \textit{main}
to ensure compatibility with earlier version of the package.
It must either be empty (|\childdocmain{}|)
or precisely match the filename of the main file in which it is specified.
See \secref{sec:detection} for further information.
\item
The filename \textit{main} must be specified without the |.tex| extension.
\item
The filename \textit{main} is case sensitive
(even in case-insensitive file systems)
due to internal string comparison.
\item
The argument \textit{main} should be fully expanded, it cannot be a macro.
\item
Subdirectories and special characters should be avoided in filenames.
\item
The command |\childdocmain{|\textit{main}|}| must be followed by a whitespace.
It should not be followed immediately by another command
or by a comment mark `|%|'.
This is because the \TeX{} parser reads the token immediately following
the argument of |\childdocmain| and puts it
at the beginning of every child section;
however, a white\-space is ignored.
\end{itemize}

%%%%%%%%%%%%%%%%%%%%%%%%%%%%%%%%%%%%%%%%
\paragraph{Content of Main File.}

It is advisable to place all content in the child files included by |\include|.
Any output contained in the main file will appear in all child documents
unless suppressed manually;
it cannot be suppressed automatically by the |\includeonly| directive
and thus should normally be avoided.
A method to include some content in the main file
by means of conditional processing is described in \secref{sec:conditional}.

%%%%%%%%%%%%%%%%%%%%%%%%%%%%%%%%%%%%%%%%
\paragraph{Page Numbering.}

When only a part of the document is compiled,
the appropriate numbering of pages
(as well as other status parameters)
is determined from the |.aux| files.
The latter contain information from previous passes.
However this information needs to propagate through
all intermediate child documents.
Therefore the page numbering in child documents may well
be inconsistent until the complete document is compiled at least once.

A useful (if unconventional) way to always ensure a consistent
page numbering is to restart the numbering in each child document
and denote the pages by `\textit{child}|.|\textit{page}'
where \textit{child} represents the chapter/section number of the child file.
This can be achieved by the command
|\numberwithin{page}{|\textit{child}|}|
of the \textsf{amsmath} package
where \textit{child} can be |chapter| or |section|
depending on the chosen structuring.
Alternatively, one can modify the macro |\thepage| appropriately
and reset the counter |page| at the start of each child file.

%%%%%%%%%%%%%%%%%%%%%%%%%%%%%%%%%%%%%%%%%%%%%%%%%%%%%%%%%%%%%%%%%%%%%%%%%%%%%%%%
\subsection{Conditional Processing}
\label{sec:conditional}

The package provides a mechanism to compile different versions
of a document. To customise the versions further some conditional processing
can come in handy to distinguish which version is being compiled.
The package provides two macros to describe the compilation context:

%%%%%%%%%%%%%%%%%%%%%%%%%%%%%%%%%%%%%%%%
\DescribeMacro{\ifchilddoc}
The conditional |\ifchilddoc| distinguishes between the compilation of
child documents and the main document:
%
\begin{center}
|\ifchilddoc |\textit{child-code}| |[|\||else |\textit{main-code}]| \||fi|
\end{center}

%%%%%%%%%%%%%%%%%%%%%%%%%%%%%%%%%%%%%%%%
\DescribeMacro{\childdocname}
\DescribeMacro{\childdocjob}
The macro |\childdocname| contains the filename (without extension)
of the main or child file being processed.
Note that |\childdocjob| will always contain the name of the main file.

%%%%%%%%%%%%%%%%%%%%%%%%%%%%%%%%%%%%%%%%
\paragraph{Title Page.}

Conditional processing can be used to include a title or banner page
in the main document when proper precautions are taken.
Importantly, the code in the main file should ensure that the page counter
(as well as other status parameters which are stored in the |.aux| files)
takes the same value after the conditional processing.
Otherwise the page numbers may take divergent values
depending on which part is compiled.

For example, a title page could be declared by:
%
\begin{center}
\begin{tabular}{l}
|\ifchilddoc\||else|\\
|\addtocounter{page}{-1}|\\
\textit{code for title page}\\
|\newpage|\\
|\||fi|
\end{tabular}
\end{center}
%
A banner page for the child documents can be generated by:
%
\begin{center}
\begin{tabular}{l}
|\ifchilddoc|\\
|\addtocounter{page}{-1}|\\
\textit{code for banner page}\\
|\newpage|\\
|\||fi|
\end{tabular}
\end{center}
%
Here one could write a message such as:
\begin{center}
|This is the part \childdocname{} of \childdocjob{}.|
\end{center}

%%%%%%%%%%%%%%%%%%%%%%%%%%%%%%%%%%%%%%%%%%%%%%%%%%%%%%%%%%%%%%%%%%%%%%%%%%%%%%%%
\subsection{Flags}
\label{sec:flags}

The package makes it easy to generate different versions
of the main or child documents.
To this end compilation flags can be defined
and assigned different default values.
They will be particularly useful in conjunction
with the forwarding mechanism described in \secref{sec:forward}.

For example, it may be useful to have a flag |\version|
which can be set to |draft| or |final|.
The document source will contain some conditional code
depending on the value of |\version|.
Suppose further, the flag should default to |final| for the main file
and to |draft| for child files
which is a natural assignment for editing the document.
This is achieved by placing the following code
in the preamble of the main document
(below the |\childdocmain| directive):
%
\begin{center}
\begin{tabular}{l}
|\ifchilddoc|\\
|\providecommand{\version}{draft}|\\
|\||else|\\
|\providecommand{\version}{final}|\\
|\||fi|
\end{tabular}
\end{center}
%
The definition by |\providecommand| makes sure
that previous definitions are not overwritten.
Further statements |\providecommand{\version}{...}|
can thus be added before the above code to override it.

For the main file, one might add a line
(between |\childdocmain| and the above block)
%
\begin{center}
|%\ifchilddoc\||else\providecommand{\version}{draft}\||fi|
\end{center}
%
which can be uncommented to produce a draft version.
Likewise one can add a line to the very top of a child file
(above the |\childdocof{|\textit{main}|}| directive)
%
\begin{center}
|%\providecommand{\version}{final}|
\end{center}
%
which can be uncommented to produce the final version of this child document.

%%%%%%%%%%%%%%%%%%%%%%%%%%%%%%%%%%%%%%%%%%%%%%%%%%%%%%%%%%%%%%%%%%%%%%%%%%%%%%%%
\subsection{Forwarding}
\label{sec:forward}

Different versions of the main or child documents
using compilation flags as described in \secref{sec:flags}
can be (permanently) stored in different files
for convenient compilation, viewing and distribution.
To this end, the package defines a command
to pass on compilation to a different file:

%%%%%%%%%%%%%%%%%%%%%%%%%%%%%%%%%%%%%%%%
\DescribeMacro{\childdocforward}
The command |\childdocforward| redirects processing to
another source file:
%
\begin{center}
\begin{tabular}{l}
|\input{childdoc.def}|\\
|\childdocforward[|\textit{main}|]{|\textit{dest}|}|\\
\end{tabular}
\end{center}
%
The argument \textit{dest} is the destination file
(without extension).
It should be the main file or one of the child files.
Note that further \textsf{childdoc} directives
such as |\childdocof| and |\childdocforward|
in the indicated file will be processed in this form.
The optional argument \textit{main}
passes on directly to the main file \textit{main}
while pretending to compile the child \textit{dest}.
This form behaves as if \textit{dest}
issues |\childdocof{|\textit{main}|}| right away,
and no further \textsf{childdoc} directives will be processed.

%%%%%%%%%%%%%%%%%%%%%%%%%%%%%%%%%%%%%%%%
\DescribeMacro{\...prefix}
In the alternative form |\childdocforwardprefix|,
%
\begin{center}
\begin{tabular}{l}
|\input{childdoc.def}|\\
|\childdocforwardprefix[|\textit{main}|]{|\textit{prefix}|}{|\textit{dest}|}|
\end{tabular}
\end{center}
%
the destination file is determined by a pattern
depending on the current file:
To make this work, the current file must be called
`{\textit{prefix}\hspace{0.2em}\textit{suffix}}'
with \textit{prefix} matching precisely the argument.
Processing is then passed on to the file
`{\textit{dest}\hspace{0.2em}\textit{suffix}}'.
Surely, the same effect is achieved by
directly specifying the
argument `{\textit{dest}\hspace{0.2em}\textit{suffix}}'
in the first form.
However, that requires to set up a different file
for each child. With the alternative form of the command
all these files can have exactly the same content
which simplifies setting them up and maintaining them.

For example, the following file |draft.tex|
with a compilation flag |\version| as described in \secref{sec:flags}
compiles the main document as a draft:
%
\begin{center}
\begin{tabular}{l}
|\def\version{draft}|\\
|\input{childdoc.def}|\\
|\childdocforward{|\textit{main}|}|
\end{tabular}
\end{center}
%
Likewise, the following files |final|\textit{nn}|.tex|
compile the final version of the child document
|child|\textit{nn}|.tex|:
%
\begin{center}
\begin{tabular}{l}
|\def\version{final}|\\
|\input{childdoc.def}|\\
|\childdocforwardprefix{final}{child}|
\end{tabular}
\end{center}
%

Note that when several versions of a main file and/or of each child file
are to be generated, it may be convenient to set up a |Makefile| or
shell script to automatise the process.

%%%%%%%%%%%%%%%%%%%%%%%%%%%%%%%%%%%%%%%%%%%%%%%%%%%%%%%%%%%%%%%%%%%%%%%%%%%%%%%%
\subsection{Command Line Processing}
\label{sec:commandline}

The effect of redirection files can also be achieved by invoking
the \LaTeX{} compiler with a more elaborate command line.
Most conveniently this should be done as part
of a shell script or a |Makefile|.

When using \textsf{childdoc} in the main file, the following
command lines effectively perform a redirection
(note that depending on the shell being used,
backslashes may have to be doubled: `|\|' $\to$ `|\\|'):
%
\begin{center}
|... -jobname "|\textit{target}|" |\\|"|[\textit{flags}]%
|\input{childdoc.def}\childdocforward[|\textit{main}|]{|\textit{dest}|}"|
\end{center}
%
Here \textit{target} is the name of the output file,
\textit{main} is the name of the main file
and \textit{dest} is the name of the main or child file to be processed
(all filenames without extensions).
The optional argument \textit{main} can be omitted
if \textit{main} matches \textit{dest}.
Optionally, compilation \textit{flags} can be defined via |\def| commands.
This command line makes the \TeX{} engine believe
it is compiling the file \textit{target}
whose content is specified as the latter parameter.
The provided code then forwards the processing to
\textit{main} or \textit{dest} as described in \secref{sec:forward}.

%%%%%%%%%%%%%%%%%%%%%%%%%%%%%%%%%%%%%%%%%%%%%%%%%%%%%%%%%%%%%%%%%%%%%%%%%%%%%%%%
\subsection{Include by Input}
\label{sec:input}

Including child documents by |\include| has some restrictions by design.
Most notably, the content of a child document always occupies
its own set of pages; pages cannot be shared between child documents.
Usually, this behaviour makes perfect sense
because each child document contain an essential part of the document.
However, in some situations it may be desirable to compose
a document from a collection of parts
without having mandatory page breaks between then.
For this case, the package
provides a mechanism to include parts
by |\input| which can also be processed individually.
However, by construction this mechanism
requires manual handling of the content to be output.

%%%%%%%%%%%%%%%%%%%%%%%%%%%%%%%%%%%%%%%%
\DescribeMacro{\ifchilddocmanual}
The main file should be prepared as usual, see \secref{sec:include}.
However, the document body must make a distinction
between processing of an individual part and of the main document, e.g.:
%
\begin{center}
\begin{tabular}{l}
|\ifchilddocmanual|\\
|\input{\childdocname}|\\
|\||else|\\
\textit{document body with }|\input{|\textit{part}|}|\\
|\||fi|
\end{tabular}
\end{center}
%
The conditional |\ifchilddocmanual| is true whenever
a part to be included by |\input| is being compiled,
and the name of the part is stored in |\childdocname|.

%%%%%%%%%%%%%%%%%%%%%%%%%%%%%%%%%%%%%%%%
\DescribeMacro{\childdocby}
Each part to be included by |\input| should start with:
%
\begin{center}
\begin{tabular}{l}
|\input{childdoc.def}|\\
|\childdocby{|\textit{main}|}|\\
\end{tabular}
\end{center}
%
The directive |\childdocby| is similar to |\childdocof|
described in \secref{sec:include},
but the subsequent selection of content must be done manually.
To that end, both |\ifchilddoc| and |\ifchilddocmanual|
will be true upon processing of a part,
and the name of the part is stored in |\childdocname|.
Note that |\jobname| will be set to the filename of the current part
so that each part receives an individual |.aux| file
that does not interfere with the |.aux| file(s) of the main document.
This behaviour can be altered by the alternative form
|\childdocby[*]{|\textit{main}|}| (with a non-empty optional argument)
which uses the |.aux| file of the main document
by setting |\jobname| to \textit{main}.

%%%%%%%%%%%%%%%%%%%%%%%%%%%%%%%%%%%%%%%%%%%%%%%%%%%%%%%%%%%%%%%%%%%%%%%%%%%%%%%%
\subsection{Driver Development}
\label{sec:driver}

The \textsf{childdoc} mechanism can also be use for the development
of definition files such as \LaTeX{} styles or classes.
This case differs from the above setup with multiple parts
included by |\include| in that no |\includeonly| should be invoked.
This can be achieved by starting the include file
(before |\ProvidesPackage|) with:
%
\begin{center}
\begin{tabular}{l}
|\input{childdoc.def}|\\
|\childdocforward{|\textit{main}|}|\\
\end{tabular}
\end{center}
%
or alternatively with:
%
\begin{center}
\begin{tabular}{l}
|\input{childdoc.def}|\\
|\childdocby{|\textit{main}|}|\\
\end{tabular}
\end{center}
%
Both forms have slightly different effects as described above.
The main file is prepared as usual, see \secref{sec:include}.

%%%%%%%%%%%%%%%%%%%%%%%%%%%%%%%%%%%%%%%%%%%%%%%%%%%%%%%%%%%%%%%%%%%%%%%%%%%%%%%%
\subsection{Legacy Detection}
\label{sec:detection}

The directive |\childdocmain| in the main file can detect
whether the complete document or merely a child is to be compiled
even without using the directive |\childdocof|.
This method is deprecated because it is less robust
and there is no compelling reason to use it;
it is merely provided for backward compatibility
and it may be removed in future versions.

If the detection mechanism is to be used,
it is mandatory to correctly specify
the filename of the main file as the argument of |\childdocmain|:
%
\begin{center}
\begin{tabular}{l}
|\input{childdoc.def}|\\
|\childdocmain{|\textit{main}|}|\\
\end{tabular}
\end{center}
%
If |\jobname| does not match the argument \textit{main} of |\childdocmain|,
it is assumed that |\jobname| points to the child file to be compiled.
When using |\childdocmain| with the main file specified as argument,
it suffices to start a child file
with just |\input{|\textit{main}|}|
without loading of the package and using |\childdocof|.
If instead all processing is done
with the appropriate \textsf{childdoc} directives,
the argument of \textit{main} of |\childdocmain| can be empty.

An alternative version of the command line processing described
in \secref{sec:commandline} using the detection mechanism reads:
%
\begin{center}
|... -jobname "|\textit{target}|" "|[\textit{flags}]%
[|\def\jobname{|\textit{dest}|}|]|\input{|\textit{main}|}"|
\end{center}

%%%%%%%%%%%%%%%%%%%%%%%%%%%%%%%%%%%%%%%%%%%%%%%%%%%%%%%%%%%%%%%%%%%%%%%%%%%%%%%%
\subsection{Manual Code}
\label{sec:manual}

In case one cannot be certain whether the definitions file |childdoc.def|
is installed on the target \TeX{} distribution
and one prefers not to ship it,
it is conceivable to paste a few relevant commands into the sources.

To that end, drop all statements |\input{childdoc.def}|
and perform the replacements as outlined below.
Instead of |\childdocmain{|\textit{main}|}| add the following code
to the top of the main file:
%
\begin{center}
\begin{tabular}{l}
|\||ifdefined\childdocname\endinput\||fi\newif\ifchilddoc|\\
|\edef\childdocname{\scantokens\expandafter{\jobname\noexpand}}|\\
|\def\childdocmain{|\textit{main}|}\||ifx\childdocmain\childdocname\||else|\\
|\childdoctrue\includeonly{\childdocname}\let\jobname\childdocmain\||fi|\\
\end{tabular}
\end{center}
%
Instead of |\childdocof{|\textit{main}|}| just include the main file
at the top of each child file:
%
\begin{center}
|\input{|\textit{main}|}|
\end{center}
%
A simple redirection |\childdocforward{|\textit{dest}|}| is achieved by:
%
\begin{center}
|\def\jobname{|\textit{dest}|}\input{\jobname}|
\end{center}
%
The redirection with prefix
|\childdocforwardprefix[|\textit{prefix}|]{|\textit{dest}|}|
is accomplished by:
%
\begin{center}
\begin{tabular}{l}
|{\edef\jobname{\scantokens\expandafter{\jobname\noexpand}}|\\
|\def\redirectjob |\textit{prefix}|#1~~~{\gdef\jobname{|\textit{dest}|#1}}|\\
|\expandafter\redirectjob\jobname~~~}\input{\jobname}|
\end{tabular}
\end{center}

In an alternative approach,
child documents can be compiled by a specific command line
without additional code or specific definitions:
%
\begin{center}
|... -jobname "|\textit{target}|" "|[\textit{flags}]%
|\includeonly{|\textit{dest}|}\input{|\textit{main}|}"|
\end{center}
%

%%%%%%%%%%%%%%%%%%%%%%%%%%%%%%%%%%%%%%%%%%%%%%%%%%%%%%%%%%%%%%%%%%%%%%%%%%%%%%%%
%%%%%%%%%%%%%%%%%%%%%%%%%%%%%%%%%%%%%%%%%%%%%%%%%%%%%%%%%%%%%%%%%%%%%%%%%%%%%%%%
\section{Information}

%%%%%%%%%%%%%%%%%%%%%%%%%%%%%%%%%%%%%%%%%%%%%%%%%%%%%%%%%%%%%%%%%%%%%%%%%%%%%%%%
\subsection{Copyright}

Copyright \copyright{} 2017--2018 Niklas Beisert

This work may be distributed and/or modified under the
conditions of the \LaTeX{} Project Public License, either version 1.3
of this license or (at your option) any later version.
The latest version of this license is in
  \url{http://www.latex-project.org/lppl.txt}
and version 1.3 or later is part of all distributions of \LaTeX{}
version 2005/12/01 or later.

This work has the LPPL maintenance status `maintained'.

The Current Maintainer of this work is Niklas Beisert.

This work consists of the files |README.txt|, |childdoc.ins| and |childdoc.dtx|
as well as the derived files |childdoc.def|, |cdocsamp.tex|
with |cdocsch1.tex|, |cdocsch2.tex|, |cdocspt3.tex|, |cdocspt4.tex|,
|cdocsdrf.tex|, |cdocsfn1.tex|, |cdocsfn2.tex|
as well as |childdoc.pdf|.

%%%%%%%%%%%%%%%%%%%%%%%%%%%%%%%%%%%%%%%%%%%%%%%%%%%%%%%%%%%%%%%%%%%%%%%%%%%%%%%%
\subsection{Files and Installation}

The package consists of the files:
%
\begin{center}
\begin{tabular}{ll}
    |README.txt|   & readme file \\
    |childdoc.ins| & installation file \\
    |childdoc.dtx| & source file \\
    |childdoc.def| & definition file \\
    |cdocsamp.tex| & sample main file \\
    |cdocsch1.tex| & sample include file \\
    |cdocsch2.tex| & sample include file \\
    |cdocspt3.tex| & sample part file \\
    |cdocspt4.tex| & sample part file \\
    |cdocsdrf.tex| & sample redirection file \\
    |cdocsfn1.tex| & sample redirection file \\
    |cdocsfn2.tex| & sample redirection file \\
    |childdoc.pdf| & manual
\end{tabular}
\end{center}
%
The distribution consists of the files
|README.txt|, |childdoc.ins| and |childdoc.dtx|.
%
\begin{itemize}
\item
Run (pdf)\LaTeX{} on |childdoc.dtx|
to compile the manual |childdoc.pdf| (this file).
\item
Run \LaTeX{} on |childdoc.ins| to create the definitions file |childdoc.def|
and the sample |cdocsamp.tex| with include files
|cdocsch1.tex|, |cdocsch2.tex|, |cdocspt3.tex|, |cdocspt4.tex|,
|cdocsdrf.tex|, |cdocsfn1.tex|, |cdocsfn2.tex|.
Then copy the file |childdoc.def| to an appropriate directory of your \LaTeX{}
distribution, e.g.\ \textit{texmf-root}|/tex/latex/childdoc|.
\end{itemize}

%%%%%%%%%%%%%%%%%%%%%%%%%%%%%%%%%%%%%%%%%%%%%%%%%%%%%%%%%%%%%%%%%%%%%%%%%%%%%%%%
\subsection{Related CTAN Packages}

There are several other packages which offer a similar functionality:
%
\begin{itemize}
\item
The packages
\href{http://ctan.org/pkg/docmute}{\textsf{docmute}},
\href{http://ctan.org/pkg/includex}{\textsf{includex}} and
\href{http://ctan.org/pkg/standalone}{\textsf{standalone}}
provide commands to include only the document body of
a child file thus allowing both files to be compiled individually.
\item
The packages \href{http://ctan.org/pkg/subdocs}{\textsf{subdocs}}
and \href{http://ctan.org/pkg/subfiles}{\textsf{subfiles}}
provide structures in which the main and child documents can be
encapsulated and allowing them to be compiled individually.
The inclusion mechanism is different from the conventional |\include|.
\item
The package \href{http://ctan.org/pkg/combine}{\textsf{combine}}
is an elaborate solution to combine several documents into one.
\end{itemize}
%
See also the CTAN topic \href{http://ctan.org/topic/subdocs}{\textsf{subdocs}}
for further related packages.
The present package differs from the above solutions in that
a document structure constructed with the conventional |\include| mechanism
just needs two extra commands at the top of every file
such that all constituent files can be compiled individually.

%%%%%%%%%%%%%%%%%%%%%%%%%%%%%%%%%%%%%%%%%%%%%%%%%%%%%%%%%%%%%%%%%%%%%%%%%%%%%%%%
%\subsection{Feature Suggestions}
%
%The following is a list of features which may be useful for future
%versions of this package:
%%
%\begin{itemize}
%\item
%\ldots
%\end{itemize}

%%%%%%%%%%%%%%%%%%%%%%%%%%%%%%%%%%%%%%%%%%%%%%%%%%%%%%%%%%%%%%%%%%%%%%%%%%%%%%%%
\subsection{Revision History}

%%%%%%%%%%%%%%%%%%%%%%%%%%%%%%%%%%%%%%%%
\paragraph{v2.0:} 2018/12/30

\begin{itemize}
\item
immediate forward processing
\item
added |\childdocby| mechanism
\item
manual restructured
\end{itemize}

%%%%%%%%%%%%%%%%%%%%%%%%%%%%%%%%%%%%%%%%
\paragraph{v1.6:} 2018/01/17

\begin{itemize}
\item
application for development of include files
\item
corrections to manual
\end{itemize}

%%%%%%%%%%%%%%%%%%%%%%%%%%%%%%%%%%%%%%%%
\paragraph{v1.5:} 2017/05/21

\begin{itemize}
\item
more complete structuring introduced
\item
|\childdocof| introduced
\item
|\childdoc| renamed to |\childdocmain|
\item
|\childredirect| renamed to |\childdocforward| and |\childdocforwardprefix|
and functionality expanded
\end{itemize}

%%%%%%%%%%%%%%%%%%%%%%%%%%%%%%%%%%%%%%%%
\paragraph{v1.0:} 2017/04/27

\begin{itemize}
\item
manual and install package
\item
first version published on CTAN
\end{itemize}

%%%%%%%%%%%%%%%%%%%%%%%%%%%%%%%%%%%%%%%%
\paragraph{v0.6:} 2017/04/26

\begin{itemize}
\item
redirection mechanism added
\end{itemize}

%%%%%%%%%%%%%%%%%%%%%%%%%%%%%%%%%%%%%%%%
\paragraph{v0.5:} 2017/04/26

\begin{itemize}
\item
functionality in definition file
\end{itemize}


%%%%%%%%%%%%%%%%%%%%%%%%%%%%%%%%%%%%%%%%%%%%%%%%%%%%%%%%%%%%%%%%%%%%%%%%%%%%%%%%
%%%%%%%%%%%%%%%%%%%%%%%%%%%%%%%%%%%%%%%%%%%%%%%%%%%%%%%%%%%%%%%%%%%%%%%%%%%%%%%%
%%%%%%%%%%%%%%%%%%%%%%%%%%%%%%%%%%%%%%%%%%%%%%%%%%%%%%%%%%%%%%%%%%%%%%%%%%%%%%%%
\appendix

\settowidth\MacroIndent{\rmfamily\scriptsize 000\ }

 \DocInput{childdoc.dtx}

\end{document}
%</driver>
% \fi
%
% %%%%%%%%%%%%%%%%%%%%%%%%%%%%%%%%%%%%%%%%%%%%%%%%%%%%%%%%%%%%%%%%%%%%%%%%%%%%%%
% %%%%%%%%%%%%%%%%%%%%%%%%%%%%%%%%%%%%%%%%%%%%%%%%%%%%%%%%%%%%%%%%%%%%%%%%%%%%%%
% \section{Sample}
%\iffalse
%<*samplemain>
%\fi
%
% The following presents a sample document
% with two chapters, two parts, a title page,
% a compile flag as well as three forwarding files to set the flag.
% It consists of eight |.tex| files:
% \begin{center}
% \begin{tabular}{ll}
% |cdocsamp.tex|&main file\\
% |cdocsch1.tex|&include file for chapter 1\\
% |cdocsch2.tex|&include file for chapter 2\\
% |cdocspt3.tex|&include file for part 3\\
% |cdocspt4.tex|&include file for part 4\\
% |cdocsdrf.tex|&forwarding file for main file in draft mode\\
% |cdocsfi1.tex|&forwarding file for final version of chapter 1\\
% |cdocsfi2.tex|&forwarding file for final version of chapter 2\\
% \end{tabular}
% \end{center}
% Each of the eight files can be compiled directly by the \LaTeX{} compiler.
%
% %%%%%%%%%%%%%%%%%%%%%%%%%%%%%%%%%%%%%%
% \paragraph{Main File.}
%
% The main file is called |cdocsamp.tex|.
%
% Load the \textsf{childdoc} definitions and
% declare the filename for the main document:
%    \begin{macrocode}
\input{childdoc.def}
\childdocmain{}
%    \end{macrocode}

% Optional override for |\version| flag:
%    \begin{macrocode}
%%\ifchilddoc\else\providecommand{\version}{draft}\fi
%    \end{macrocode}

% Define the default values for the |\version| flag
% (|final| for the main file and |draft| for childs):
%    \begin{macrocode}
\ifchilddoc
\providecommand{\version}{draft}
\else
\providecommand{\version}{final}
\fi
%    \end{macrocode}

% Load the standard document class:
%    \begin{macrocode}
\documentclass[12pt]{article}
%    \end{macrocode}

% Start the document body:
%    \begin{macrocode}
\begin{document}
%    \end{macrocode}

% Declare a title page.
% Print title, part of document being processed and version flag:
%    \begin{macrocode}
\addtocounter{page}{-1}
\begin{center}
{\LARGE\bfseries{}childdoc example\par}
\vspace{1cm}
\ifchilddoc
\ifchilddocmanual part\else chapter\fi:
`\childdocname' of `\childdocjob'\par
\else
main document: `\childdocjob'\par
\fi
version: \version\par
\end{center}
\newpage
%    \end{macrocode}

% Manually include selected file,
% otherwise process as usual:
%    \begin{macrocode}
\ifchilddocmanual
\section*{part `\childdocname'}
\input{\childdocname}
\else
%    \end{macrocode}

% Include the two chapters:
%    \begin{macrocode}
\include{cdocsch1}
\include{cdocsch2}
%    \end{macrocode}

% Include the two parts unless only chapters should be displayed:
%    \begin{macrocode}
\ifchilddoc\else
\section{part three}
\input{cdocspt3}
\section{part four}
\input{cdocspt4}
\fi
%    \end{macrocode}

% Process as usual until here:
%    \begin{macrocode}
\fi
%    \end{macrocode}

% End of document body:
%    \begin{macrocode}
\end{document}
%    \end{macrocode}
%\iffalse
%</samplemain>
%\fi
%
% %%%%%%%%%%%%%%%%%%%%%%%%%%%%%%%%%%%%%%
% \paragraph{Chapter Include Files.}
%
% The include files are called |cdocsch1.tex| and |cdocsch2.tex|.
%
%\iffalse
%<*samplechap1|samplechap2>
%\fi

% Optional override for |\version| flag:
%    \begin{macrocode}
%%\providecommand{\version}{final}
%    \end{macrocode}

% Include the main document:
%    \begin{macrocode}
\input{childdoc.def}
\childdocof{cdocsamp}
%    \end{macrocode}

%\iffalse
%</samplechap1|samplechap2>
%\fi
%
%\iffalse
%<*samplechap1>
%\fi
% Some text for chapter 1:
%    \begin{macrocode}
\section{one}
some text in chapter one
%    \end{macrocode}

%\iffalse
%</samplechap1>
%\fi
% Some text for chapter 2:
%\iffalse
%<*samplechap2>
%\fi
%    \begin{macrocode}
\section{two}
more text in chapter two
%    \end{macrocode}

%\iffalse
%</samplechap2>
%\fi
%
% %%%%%%%%%%%%%%%%%%%%%%%%%%%%%%%%%%%%%%
% \paragraph{Part Include Files.}
%
% The include files are called |cdocspt3.tex| and |cdocspt4.tex|.
%
%\iffalse
%<*samplepart3|samplepart4>
%\fi

% Optional override for |\version| flag:
%    \begin{macrocode}
%%\providecommand{\version}{final}
%    \end{macrocode}

% Include the main document:
%    \begin{macrocode}
\input{childdoc.def}
\childdocby{cdocsamp}
%    \end{macrocode}

%\iffalse
%</samplepart3|samplepart4>
%\fi
%
%\iffalse
%<*samplepart3>
%\fi
% Some text for part 3:
%    \begin{macrocode}
some text in part three
%    \end{macrocode}

%\iffalse
%</samplepart3>
%\fi
% Some text for part 4:
%\iffalse
%<*samplepart4>
%\fi
%    \begin{macrocode}
more text in part four
%    \end{macrocode}

%\iffalse
%</samplepart4>
%\fi
%
% %%%%%%%%%%%%%%%%%%%%%%%%%%%%%%%%%%%%%%
% \paragraph{Forwarding for a Complete Draft.}
%
% The following forwarding file |cdocsdrf.tex|
% compiles the main document in draft mode:
%\iffalse
%<*sampledraft>
%\fi
%    \begin{macrocode}
\def\version{draft}
\input{childdoc.def}
\childdocforward{cdocsamp}
%    \end{macrocode}

%\iffalse
%</sampledraft>
%\fi
%
% %%%%%%%%%%%%%%%%%%%%%%%%%%%%%%%%%%%%%%
% \paragraph{Forwarding for Final Version of the Chapters.}
%
% The following forwarding files |cdocsfn1.tex| and |cdocsfn2.tex|
% (with identical content)
% compile the final versions of the child documents
% |cdocsch1.tex| and |cdocsch2.tex|, respectively:
%\iffalse
%<*samplefinal>
%\fi
%    \begin{macrocode}
\def\version{final}
\input{childdoc.def}
\childdocforwardprefix[cdocsamp]{cdocsfn}{cdocsch}
%    \end{macrocode}

%\iffalse
%</samplefinal>
%\fi
%
% %%%%%%%%%%%%%%%%%%%%%%%%%%%%%%%%%%%%%%
% \paragraph{Command Line Processing.}
%
% The following three command lines generate the output files
% |cdocscld|, |cdocscl1| and |cdocscl2|
% which should be identical to
% |cdocsdrf|, |cdocsch1| and |cdocsfn2|, respectively:
% \begin{center}
% \begin{tabular}{l}
% |latex -jobname cdocscld \|\\
% |  "\def\version{draft}\input{childdoc.def}\childdocforward{cdocsamp}"|\\
% |latex -jobname cdocscl1 \|\\
% |  "\input{childdoc.def}\childdocforward[cdocsamp]{cdocsch1}"|\\
% |latex -jobname cdocscl2 \|\\
% |  "\def\version{final}\input{childdoc.def}\childdocforward{cdocsch2}"|
% \end{tabular}
% \end{center}
% Note that the trailing backslash on each first line
% merely continues the input to the second line
% (for convenient cut ant paste).
% Furthermore, the command |latex| can be replaced by any
% of its alternative versions such as |pdflatex|.
%
% %%%%%%%%%%%%%%%%%%%%%%%%%%%%%%%%%%%%%%%%%%%%%%%%%%%%%%%%%%%%%%%%%%%%%%%%%%%%%%
% %%%%%%%%%%%%%%%%%%%%%%%%%%%%%%%%%%%%%%%%%%%%%%%%%%%%%%%%%%%%%%%%%%%%%%%%%%%%%%
% \section{Implementation}
%\iffalse
%<*package>
%\fi
%
% This section describes the definitions file |childdoc.def|.

% The definitions cannot be loaded using |\usepackage| or |\RequirePackage|
% which has a mechanism to prevent loading a style file more than once.
% When loading the definitions by means of |\input|
% multiple instances have to be prevented manually:
%\iffalse
%This code needs to be before the `\ProvidesFile' directive
%which is defined at the beginning of this file.
%Therefore it is also placed there and commented out here.
%</package>
%<*discard>
%\fi
%    \begin{macrocode}
\ifdefined\childdocmain\endinput\fi
%    \end{macrocode}
%\iffalse
%</discard>
%<*package>
%\fi
%
% \macro{\ifchilddoc}
% \macro{\ifchilddocmanual}
% The conditional |\ifchilddoc| tells whether a
% child (true) or main (false) document is being compiled.
% The conditional |\ifchilddocmanual| tells whether
% the |\includeonly| mechanism is used (false) or
% the selection of child files must be performed manually (true).
% The definitions initialise to false:
%    \begin{macrocode}
\newif\ifchilddoc
\newif\ifchilddocmanual
%    \end{macrocode}

% \macro{\childdocname}
% \macro{\childdocjob}
% The macro |\childdocname| stores the name of the main document
% to be compiled. The macro |\childdocjob| stores the name of
% the document on which the \LaTeX{} compiler was originally invoked.
% The content of |\jobname| cannot be compared
% to filenames specified in the source due to different catcodes.
% The following code rescans |\jobname|, stores the result
% in |\childdocname| and saves a copy in |\childdocjob|:
%    \begin{macrocode}
\edef\childdocname{\scantokens\expandafter{\jobname\noexpand}}
\let\childdocjob\childdocname
%    \end{macrocode}

% \macro{\childdocdisable}
% The macro |\childdocdisable| prevents the main file
% from being processed more than once.
% At this stage, the main document command |\childdocmain|
% is assumed to be called once again where it should do nothing.
% Any subsequent call to it should prevent
% a secondary processing of the main document
% It overwrites the forwarding commands
% |\childdocof| and |\childdocforward|
% with empty macros to prevent further inclusions of the main document:
%    \begin{macrocode}
\newcommand{\childdocdisable}
{
  \renewcommand{\childdocmain}[1]{\renewcommand{\childdocmain}[1]{\endinput}}
  \renewcommand{\childdocof}[1]{}
  \renewcommand{\childdocby}[2][]{}
  \renewcommand{\childdocforward}[2][]{}
  \renewcommand{\childdocdisable}{}
}
%    \end{macrocode}

% \macro{\childdocmain}
% The macro |\childdocmain| is to be called at the top of the main file
% with nothing or the main filename (without extension) as argument.
% First, it breaks loops.
% If the argument is not empty and does not match |\childdocname|
% (which is set by the first inclusion of |childdoc.def|),
% |\ifchilddoc| is set to true, |\includeonly| is applied to the child file
% and |\jobname| is set to the main file
% (for proper handling of |.aux| files):
%    \begin{macrocode}
\newcommand{\childdocmain}[1]
{
  \childdocdisable\childdocmain{}
  \if?#1?\else
    \begingroup
      \def\childdoctmp{#1}
      \ifx\childdoctmp\childdocname
        \def\childdoctmp{}
      \else
        \def\childdoctmp
        {
          \childdoctrue
          \includeonly{\childdocname}
          \def\childdocjob{#1}
          \def\jobname{#1}
        }
      \fi
      \expandafter
    \endgroup
    \childdoctmp
  \fi
}
%    \end{macrocode}

% \macro{\childdocof}
% The command |\childdocof| redirects
% compilation to the main file |#1|.
%    \begin{macrocode}
\newcommand{\childdocof}[1]
{
  \childdocdisable
  \childdoctrue
  \includeonly{\childdocname}
  \def\jobname{#1}
  \def\childdocjob{#1}
  \input{#1}
}
%    \end{macrocode}

% \macro{\childdocby}
% The command |\childdocby| ....
%    \begin{macrocode}
\newcommand{\childdocby}[2][]
{
  \childdocdisable
  \childdoctrue
  \childdocmanualtrue
  \if?#1?\else
    \def\jobname{#2}
  \fi
  \def\childdocjob{#2}
  \input{#2}
  \endinput
}
%    \end{macrocode}

% \macro{\childdocforward}
% The command |\childdocforward| redirects
% compilation to the main file or
% (if the optional argument is given) a child file.
% Parameters are set as if the main file
% or a child file starting with |\childdocof| was compiled.
% Then compilation is handed over to the main file:
%    \begin{macrocode}
\newcommand{\childdocforward}[2][]
{
  \begingroup
    \if?#1?
      \def\childdoctmp
      {
        \def\childdocname{#2}
        \def\childdocjob{#2}
        \def\jobname{#2}
        \input{#2}
        \endinput
      }
    \else
      \def\childdoctmp
      {
        \childdocdisable
        \def\childdocname{#2}
        \childdoctrue
        \includeonly{#2}
        \def\childdocjob{#1}
        \def\jobname{#1}
        \input{#1}
        \endinput
      }
    \fi
    \expandafter
  \endgroup
  \childdoctmp
}
%    \end{macrocode}

% \macro{\childdocforwardprefix}
% The command |\childdocforwardprefix| redirects
% compilation to the main or a child file by means of a pattern.
% The prefix |#1| in the current filename is replaced by |#2|
% and the suffix of the current filename is kept
% (it is assumed that the filename does not contain the substring `|~~~|'
% which is used as a delimiter).
% Compilation is handed over to the new file by |\childdocforward|:
%    \begin{macrocode}
\newcommand{\childdocforwardprefix}[3][]
{
  \begingroup
    \def\childdocextract #2##1~~~{\def\childdoctmp{\childdocforward[#1]{#3##1}}}
    \expandafter\childdocextract\childdocname~~~
    \expandafter
  \endgroup
  \childdoctmp
}
%    \end{macrocode}

% \macro{\childdoc}
% The deprecated macro |\childdoc| is a legacy version of |\childdocmain|:
%    \begin{macrocode}
\newcommand{\childdoc}{\childdocmain}
%    \end{macrocode}

% \macro{\childdocredirect}
% The deprecated macro |\childdocredirect| is a legacy version
% of |\childdocforward| and |\childdocforwardprefix|:
%    \begin{macrocode}
\newcommand{\childdocredirect}[2][]
{
  \begingroup
    \if?#1?
      \def\childdoctmp{\childdocforward{#2}}
    \else
      \def\childdoctmp{\childdocforwardprefix{#1}{#2}}
    \fi
    \expandafter
  \endgroup
  \childdoctmp
}
%    \end{macrocode}

%\iffalse
%</package>
%\fi
%
\endinput
|\\
|\childdocmain{}|\\
\end{tabular}
\end{center}
at the very top of the main \LaTeX{} file,
in particular \emph{before} the |\documentclass| statement!
The argument of |\childdocmain| should be left empty
(but it must be present).

%%%%%%%%%%%%%%%%%%%%%%%%%%%%%%%%%%%%%%%%
\DescribeMacro{\childdocof}
Furthermore, add the commands
\begin{center}
\begin{tabular}{l}
|% \iffalse
%
% childdoc.dtx Copyright (C) 2017-2018 Niklas Beisert
%
% This work may be distributed and/or modified under the
% conditions of the LaTeX Project Public License, either version 1.3
% of this license or (at your option) any later version.
% The latest version of this license is in
%   http://www.latex-project.org/lppl.txt
% and version 1.3 or later is part of all distributions of LaTeX
% version 2005/12/01 or later.
%
% This work has the LPPL maintenance status `maintained'.
%
% The Current Maintainer of this work is Niklas Beisert.
%
% This work consists of the files childdoc.dtx and childdoc.ins
% and the derived files childdoc.def and cdocsamp.tex with
% cdocsch1.tex, cdocsch2.tex, cdocsdrf.tex, cdocsfn1.tex, cdocsfn2.tex.
%
%<package>\ifdefined\childdocmain\endinput\fi
%<package>\ProvidesFile{childdoc.def}[2018/12/30 v2.0 child document driver]
%<samplemain>\ProvidesFile{cdocsamp.tex}[2018/12/30 v2.0 sample for childdoc]
%<*driver>
%\ProvidesFile{childdoc.drv}[2018/12/30 v2.0 childdoc reference manual file]
\PassOptionsToClass{10pt,a4paper}{article}
\documentclass{ltxdoc}

\usepackage[margin=35mm]{geometry}
\usepackage{hyperref}
\usepackage{hyperxmp}
\usepackage[usenames]{color}

\hypersetup{colorlinks=true}
\hypersetup{pdfstartview=FitH}
\hypersetup{pdfpagemode=UseNone}
\hypersetup{pdfsource={}}
\hypersetup{pdflang={en-UK}}
\hypersetup{pdfcopyright={Copyright 2017-2018 Niklas Beisert.
  This work may be distributed and/or modified under the
  conditions of the LaTeX Project Public License, either version 1.3
  of this license or (at your option) any later version.}}
\hypersetup{pdflicenseurl={http://www.latex-project.org/lppl.txt}}
\hypersetup{pdfcontactaddress={ETH Zurich, ITP, HIT K,
  Wolfgang-Pauli-Strasse 27}}
\hypersetup{pdfcontactpostcode={8093}}
\hypersetup{pdfcontactcity={Zurich}}
\hypersetup{pdfcontactcountry={Switzerland}}
\hypersetup{pdfcontactemail={nbeisert@itp.phys.ethz.ch}}
\hypersetup{pdfcontacturl={http://people.phys.ethz.ch/\xmptilde nbeisert/}}

\newcommand{\secref}[1]{\hyperref[#1]{section \ref*{#1}}}

\parskip1ex
\parindent0pt
\let\olditemize\itemize
\def\itemize{\olditemize\parskip0pt}

\begin{document}

\title{The \textsf{childdoc} Package}
\hypersetup{pdftitle={The childdoc Package}}
\author{Niklas Beisert\\[2ex]
  Institut f\"ur Theoretische Physik\\
  Eidgen\"ossische Technische Hochschule Z\"urich\\
  Wolfgang-Pauli-Strasse 27, 8093 Z\"urich, Switzerland\\[1ex]
  \href{mailto:nbeisert@itp.phys.ethz.ch}
  {\texttt{nbeisert@itp.phys.ethz.ch}}}
\hypersetup{pdfauthor={Niklas Beisert}}
\hypersetup{pdfsubject={Manual for the LaTeX2e Package childdoc}}
\date{30 December 2018, \textsf{v2.0}}
\maketitle

\begin{abstract}\noindent
\textsf{childdoc} is a \LaTeXe{} package
that enables the direct compilation
of document sections included by |\include|
to individual files.
\end{abstract}

\begingroup
\parskip0ex
\tableofcontents
\endgroup

%%%%%%%%%%%%%%%%%%%%%%%%%%%%%%%%%%%%%%%%%%%%%%%%%%%%%%%%%%%%%%%%%%%%%%%%%%%%%%%%
%%%%%%%%%%%%%%%%%%%%%%%%%%%%%%%%%%%%%%%%%%%%%%%%%%%%%%%%%%%%%%%%%%%%%%%%%%%%%%%%
\section{Introduction}

\LaTeX{} provides a mechanism to structure a large document (such as a book)
into a main file and several child files (containing the chapters)
using the |\include| command.
This mechanism is beneficial for documents
which span hundreds of pages in order to
make the source file(s) more manageable.
Moreover, compilation can be restricted to
selected child files by means of the |\includeonly| command.
The latter feature can be used to reduce the compilation time while editing
(this was significantly more useful in the earlier days of \LaTeX{})
or to generate a smaller document which is easier to navigate.
Another application of |\includeonly| is to generate
documents consisting of selected parts of the complete document.

However, there are a few drawbacks of the plain |\include| mechanism:
\begin{itemize}
\item
The child files cannot be compiled on their own,
they can only be compiled via the main file.
A naive editing environment
(such as a text editor with an option
to have the current file processed by \LaTeX)
may require one to switch to the main file before compiling;
attempting to compile the child file produces errors.
\item
The main file must be modified (each time)
to adjust the |\includeonly| command
to the present needs. This easily leaves the main file in a messy state.
\item
The generated document will always carry the filename
of the main document. This is inconvenient if
several child files are to be compiled and
to be kept for distribution.
\end{itemize}

The present package provides a simple interface
to make child files individually compilable by \LaTeX{}.
Compiling a child file then has the same effect as compiling
the main file with an |\includeonly| command
to select the appropriate child.
Moreover the generated document will carry the name of the child
rather than the main file.
This resolves all three above issues.

This feature is meant to make the editing of books,
thesis documents and lecture notes somewhat more convenient.
However, the package can also be used efficiently for
composing a series of documents (such as exercise sheets)
which are typically distributed individually.
It then assists the author in generating the individual documents
(potentially in different versions)
as well as a document containing the collected series.
Another application is in developing style files
or other kinds of included material
where compilation of the style file could redirect
to a sample or test file.

%%%%%%%%%%%%%%%%%%%%%%%%%%%%%%%%%%%%%%%%%%%%%%%%%%%%%%%%%%%%%%%%%%%%%%%%%%%%%%%%
%%%%%%%%%%%%%%%%%%%%%%%%%%%%%%%%%%%%%%%%%%%%%%%%%%%%%%%%%%%%%%%%%%%%%%%%%%%%%%%%
\section{Usage}

First of all, the package \textsf{childdoc} is \emph{not} a standard
\LaTeXe{} |.sty| style file! Therefore it needs to be invoked in
a non-standard way.

%%%%%%%%%%%%%%%%%%%%%%%%%%%%%%%%%%%%%%%%%%%%%%%%%%%%%%%%%%%%%%%%%%%%%%%%%%%%%%%%
\subsection{Included Files}
\label{sec:include}

%%%%%%%%%%%%%%%%%%%%%%%%%%%%%%%%%%%%%%%%
\DescribeMacro{\childdocmain}
To use the package, add the commands
\begin{center}
\begin{tabular}{l}
|\input{childdoc.def}|\\
|\childdocmain{}|\\
\end{tabular}
\end{center}
at the very top of the main \LaTeX{} file,
in particular \emph{before} the |\documentclass| statement!
The argument of |\childdocmain| should be left empty
(but it must be present).

%%%%%%%%%%%%%%%%%%%%%%%%%%%%%%%%%%%%%%%%
\DescribeMacro{\childdocof}
Furthermore, add the commands
\begin{center}
\begin{tabular}{l}
|\input{childdoc.def}|\\
|\childdocof{|\textit{main}|}|\\
\end{tabular}
\end{center}
at the top of every child file \textit{child}
which is included by |\include{|\textit{child}|}|
from within the main file
(or at least for those files to be compiled individually).
The argument \textit{main} must be the filename of the main file.

There are a couple of
considerations in setting up the main and child documents:

%%%%%%%%%%%%%%%%%%%%%%%%%%%%%%%%%%%%%%%%
\paragraph{Restrictions.}

Please note the following restrictions:
\begin{itemize}
\item
|\childdocmain| must be called with one argument \textit{main}
to ensure compatibility with earlier version of the package.
It must either be empty (|\childdocmain{}|)
or precisely match the filename of the main file in which it is specified.
See \secref{sec:detection} for further information.
\item
The filename \textit{main} must be specified without the |.tex| extension.
\item
The filename \textit{main} is case sensitive
(even in case-insensitive file systems)
due to internal string comparison.
\item
The argument \textit{main} should be fully expanded, it cannot be a macro.
\item
Subdirectories and special characters should be avoided in filenames.
\item
The command |\childdocmain{|\textit{main}|}| must be followed by a whitespace.
It should not be followed immediately by another command
or by a comment mark `|%|'.
This is because the \TeX{} parser reads the token immediately following
the argument of |\childdocmain| and puts it
at the beginning of every child section;
however, a white\-space is ignored.
\end{itemize}

%%%%%%%%%%%%%%%%%%%%%%%%%%%%%%%%%%%%%%%%
\paragraph{Content of Main File.}

It is advisable to place all content in the child files included by |\include|.
Any output contained in the main file will appear in all child documents
unless suppressed manually;
it cannot be suppressed automatically by the |\includeonly| directive
and thus should normally be avoided.
A method to include some content in the main file
by means of conditional processing is described in \secref{sec:conditional}.

%%%%%%%%%%%%%%%%%%%%%%%%%%%%%%%%%%%%%%%%
\paragraph{Page Numbering.}

When only a part of the document is compiled,
the appropriate numbering of pages
(as well as other status parameters)
is determined from the |.aux| files.
The latter contain information from previous passes.
However this information needs to propagate through
all intermediate child documents.
Therefore the page numbering in child documents may well
be inconsistent until the complete document is compiled at least once.

A useful (if unconventional) way to always ensure a consistent
page numbering is to restart the numbering in each child document
and denote the pages by `\textit{child}|.|\textit{page}'
where \textit{child} represents the chapter/section number of the child file.
This can be achieved by the command
|\numberwithin{page}{|\textit{child}|}|
of the \textsf{amsmath} package
where \textit{child} can be |chapter| or |section|
depending on the chosen structuring.
Alternatively, one can modify the macro |\thepage| appropriately
and reset the counter |page| at the start of each child file.

%%%%%%%%%%%%%%%%%%%%%%%%%%%%%%%%%%%%%%%%%%%%%%%%%%%%%%%%%%%%%%%%%%%%%%%%%%%%%%%%
\subsection{Conditional Processing}
\label{sec:conditional}

The package provides a mechanism to compile different versions
of a document. To customise the versions further some conditional processing
can come in handy to distinguish which version is being compiled.
The package provides two macros to describe the compilation context:

%%%%%%%%%%%%%%%%%%%%%%%%%%%%%%%%%%%%%%%%
\DescribeMacro{\ifchilddoc}
The conditional |\ifchilddoc| distinguishes between the compilation of
child documents and the main document:
%
\begin{center}
|\ifchilddoc |\textit{child-code}| |[|\||else |\textit{main-code}]| \||fi|
\end{center}

%%%%%%%%%%%%%%%%%%%%%%%%%%%%%%%%%%%%%%%%
\DescribeMacro{\childdocname}
\DescribeMacro{\childdocjob}
The macro |\childdocname| contains the filename (without extension)
of the main or child file being processed.
Note that |\childdocjob| will always contain the name of the main file.

%%%%%%%%%%%%%%%%%%%%%%%%%%%%%%%%%%%%%%%%
\paragraph{Title Page.}

Conditional processing can be used to include a title or banner page
in the main document when proper precautions are taken.
Importantly, the code in the main file should ensure that the page counter
(as well as other status parameters which are stored in the |.aux| files)
takes the same value after the conditional processing.
Otherwise the page numbers may take divergent values
depending on which part is compiled.

For example, a title page could be declared by:
%
\begin{center}
\begin{tabular}{l}
|\ifchilddoc\||else|\\
|\addtocounter{page}{-1}|\\
\textit{code for title page}\\
|\newpage|\\
|\||fi|
\end{tabular}
\end{center}
%
A banner page for the child documents can be generated by:
%
\begin{center}
\begin{tabular}{l}
|\ifchilddoc|\\
|\addtocounter{page}{-1}|\\
\textit{code for banner page}\\
|\newpage|\\
|\||fi|
\end{tabular}
\end{center}
%
Here one could write a message such as:
\begin{center}
|This is the part \childdocname{} of \childdocjob{}.|
\end{center}

%%%%%%%%%%%%%%%%%%%%%%%%%%%%%%%%%%%%%%%%%%%%%%%%%%%%%%%%%%%%%%%%%%%%%%%%%%%%%%%%
\subsection{Flags}
\label{sec:flags}

The package makes it easy to generate different versions
of the main or child documents.
To this end compilation flags can be defined
and assigned different default values.
They will be particularly useful in conjunction
with the forwarding mechanism described in \secref{sec:forward}.

For example, it may be useful to have a flag |\version|
which can be set to |draft| or |final|.
The document source will contain some conditional code
depending on the value of |\version|.
Suppose further, the flag should default to |final| for the main file
and to |draft| for child files
which is a natural assignment for editing the document.
This is achieved by placing the following code
in the preamble of the main document
(below the |\childdocmain| directive):
%
\begin{center}
\begin{tabular}{l}
|\ifchilddoc|\\
|\providecommand{\version}{draft}|\\
|\||else|\\
|\providecommand{\version}{final}|\\
|\||fi|
\end{tabular}
\end{center}
%
The definition by |\providecommand| makes sure
that previous definitions are not overwritten.
Further statements |\providecommand{\version}{...}|
can thus be added before the above code to override it.

For the main file, one might add a line
(between |\childdocmain| and the above block)
%
\begin{center}
|%\ifchilddoc\||else\providecommand{\version}{draft}\||fi|
\end{center}
%
which can be uncommented to produce a draft version.
Likewise one can add a line to the very top of a child file
(above the |\childdocof{|\textit{main}|}| directive)
%
\begin{center}
|%\providecommand{\version}{final}|
\end{center}
%
which can be uncommented to produce the final version of this child document.

%%%%%%%%%%%%%%%%%%%%%%%%%%%%%%%%%%%%%%%%%%%%%%%%%%%%%%%%%%%%%%%%%%%%%%%%%%%%%%%%
\subsection{Forwarding}
\label{sec:forward}

Different versions of the main or child documents
using compilation flags as described in \secref{sec:flags}
can be (permanently) stored in different files
for convenient compilation, viewing and distribution.
To this end, the package defines a command
to pass on compilation to a different file:

%%%%%%%%%%%%%%%%%%%%%%%%%%%%%%%%%%%%%%%%
\DescribeMacro{\childdocforward}
The command |\childdocforward| redirects processing to
another source file:
%
\begin{center}
\begin{tabular}{l}
|\input{childdoc.def}|\\
|\childdocforward[|\textit{main}|]{|\textit{dest}|}|\\
\end{tabular}
\end{center}
%
The argument \textit{dest} is the destination file
(without extension).
It should be the main file or one of the child files.
Note that further \textsf{childdoc} directives
such as |\childdocof| and |\childdocforward|
in the indicated file will be processed in this form.
The optional argument \textit{main}
passes on directly to the main file \textit{main}
while pretending to compile the child \textit{dest}.
This form behaves as if \textit{dest}
issues |\childdocof{|\textit{main}|}| right away,
and no further \textsf{childdoc} directives will be processed.

%%%%%%%%%%%%%%%%%%%%%%%%%%%%%%%%%%%%%%%%
\DescribeMacro{\...prefix}
In the alternative form |\childdocforwardprefix|,
%
\begin{center}
\begin{tabular}{l}
|\input{childdoc.def}|\\
|\childdocforwardprefix[|\textit{main}|]{|\textit{prefix}|}{|\textit{dest}|}|
\end{tabular}
\end{center}
%
the destination file is determined by a pattern
depending on the current file:
To make this work, the current file must be called
`{\textit{prefix}\hspace{0.2em}\textit{suffix}}'
with \textit{prefix} matching precisely the argument.
Processing is then passed on to the file
`{\textit{dest}\hspace{0.2em}\textit{suffix}}'.
Surely, the same effect is achieved by
directly specifying the
argument `{\textit{dest}\hspace{0.2em}\textit{suffix}}'
in the first form.
However, that requires to set up a different file
for each child. With the alternative form of the command
all these files can have exactly the same content
which simplifies setting them up and maintaining them.

For example, the following file |draft.tex|
with a compilation flag |\version| as described in \secref{sec:flags}
compiles the main document as a draft:
%
\begin{center}
\begin{tabular}{l}
|\def\version{draft}|\\
|\input{childdoc.def}|\\
|\childdocforward{|\textit{main}|}|
\end{tabular}
\end{center}
%
Likewise, the following files |final|\textit{nn}|.tex|
compile the final version of the child document
|child|\textit{nn}|.tex|:
%
\begin{center}
\begin{tabular}{l}
|\def\version{final}|\\
|\input{childdoc.def}|\\
|\childdocforwardprefix{final}{child}|
\end{tabular}
\end{center}
%

Note that when several versions of a main file and/or of each child file
are to be generated, it may be convenient to set up a |Makefile| or
shell script to automatise the process.

%%%%%%%%%%%%%%%%%%%%%%%%%%%%%%%%%%%%%%%%%%%%%%%%%%%%%%%%%%%%%%%%%%%%%%%%%%%%%%%%
\subsection{Command Line Processing}
\label{sec:commandline}

The effect of redirection files can also be achieved by invoking
the \LaTeX{} compiler with a more elaborate command line.
Most conveniently this should be done as part
of a shell script or a |Makefile|.

When using \textsf{childdoc} in the main file, the following
command lines effectively perform a redirection
(note that depending on the shell being used,
backslashes may have to be doubled: `|\|' $\to$ `|\\|'):
%
\begin{center}
|... -jobname "|\textit{target}|" |\\|"|[\textit{flags}]%
|\input{childdoc.def}\childdocforward[|\textit{main}|]{|\textit{dest}|}"|
\end{center}
%
Here \textit{target} is the name of the output file,
\textit{main} is the name of the main file
and \textit{dest} is the name of the main or child file to be processed
(all filenames without extensions).
The optional argument \textit{main} can be omitted
if \textit{main} matches \textit{dest}.
Optionally, compilation \textit{flags} can be defined via |\def| commands.
This command line makes the \TeX{} engine believe
it is compiling the file \textit{target}
whose content is specified as the latter parameter.
The provided code then forwards the processing to
\textit{main} or \textit{dest} as described in \secref{sec:forward}.

%%%%%%%%%%%%%%%%%%%%%%%%%%%%%%%%%%%%%%%%%%%%%%%%%%%%%%%%%%%%%%%%%%%%%%%%%%%%%%%%
\subsection{Include by Input}
\label{sec:input}

Including child documents by |\include| has some restrictions by design.
Most notably, the content of a child document always occupies
its own set of pages; pages cannot be shared between child documents.
Usually, this behaviour makes perfect sense
because each child document contain an essential part of the document.
However, in some situations it may be desirable to compose
a document from a collection of parts
without having mandatory page breaks between then.
For this case, the package
provides a mechanism to include parts
by |\input| which can also be processed individually.
However, by construction this mechanism
requires manual handling of the content to be output.

%%%%%%%%%%%%%%%%%%%%%%%%%%%%%%%%%%%%%%%%
\DescribeMacro{\ifchilddocmanual}
The main file should be prepared as usual, see \secref{sec:include}.
However, the document body must make a distinction
between processing of an individual part and of the main document, e.g.:
%
\begin{center}
\begin{tabular}{l}
|\ifchilddocmanual|\\
|\input{\childdocname}|\\
|\||else|\\
\textit{document body with }|\input{|\textit{part}|}|\\
|\||fi|
\end{tabular}
\end{center}
%
The conditional |\ifchilddocmanual| is true whenever
a part to be included by |\input| is being compiled,
and the name of the part is stored in |\childdocname|.

%%%%%%%%%%%%%%%%%%%%%%%%%%%%%%%%%%%%%%%%
\DescribeMacro{\childdocby}
Each part to be included by |\input| should start with:
%
\begin{center}
\begin{tabular}{l}
|\input{childdoc.def}|\\
|\childdocby{|\textit{main}|}|\\
\end{tabular}
\end{center}
%
The directive |\childdocby| is similar to |\childdocof|
described in \secref{sec:include},
but the subsequent selection of content must be done manually.
To that end, both |\ifchilddoc| and |\ifchilddocmanual|
will be true upon processing of a part,
and the name of the part is stored in |\childdocname|.
Note that |\jobname| will be set to the filename of the current part
so that each part receives an individual |.aux| file
that does not interfere with the |.aux| file(s) of the main document.
This behaviour can be altered by the alternative form
|\childdocby[*]{|\textit{main}|}| (with a non-empty optional argument)
which uses the |.aux| file of the main document
by setting |\jobname| to \textit{main}.

%%%%%%%%%%%%%%%%%%%%%%%%%%%%%%%%%%%%%%%%%%%%%%%%%%%%%%%%%%%%%%%%%%%%%%%%%%%%%%%%
\subsection{Driver Development}
\label{sec:driver}

The \textsf{childdoc} mechanism can also be use for the development
of definition files such as \LaTeX{} styles or classes.
This case differs from the above setup with multiple parts
included by |\include| in that no |\includeonly| should be invoked.
This can be achieved by starting the include file
(before |\ProvidesPackage|) with:
%
\begin{center}
\begin{tabular}{l}
|\input{childdoc.def}|\\
|\childdocforward{|\textit{main}|}|\\
\end{tabular}
\end{center}
%
or alternatively with:
%
\begin{center}
\begin{tabular}{l}
|\input{childdoc.def}|\\
|\childdocby{|\textit{main}|}|\\
\end{tabular}
\end{center}
%
Both forms have slightly different effects as described above.
The main file is prepared as usual, see \secref{sec:include}.

%%%%%%%%%%%%%%%%%%%%%%%%%%%%%%%%%%%%%%%%%%%%%%%%%%%%%%%%%%%%%%%%%%%%%%%%%%%%%%%%
\subsection{Legacy Detection}
\label{sec:detection}

The directive |\childdocmain| in the main file can detect
whether the complete document or merely a child is to be compiled
even without using the directive |\childdocof|.
This method is deprecated because it is less robust
and there is no compelling reason to use it;
it is merely provided for backward compatibility
and it may be removed in future versions.

If the detection mechanism is to be used,
it is mandatory to correctly specify
the filename of the main file as the argument of |\childdocmain|:
%
\begin{center}
\begin{tabular}{l}
|\input{childdoc.def}|\\
|\childdocmain{|\textit{main}|}|\\
\end{tabular}
\end{center}
%
If |\jobname| does not match the argument \textit{main} of |\childdocmain|,
it is assumed that |\jobname| points to the child file to be compiled.
When using |\childdocmain| with the main file specified as argument,
it suffices to start a child file
with just |\input{|\textit{main}|}|
without loading of the package and using |\childdocof|.
If instead all processing is done
with the appropriate \textsf{childdoc} directives,
the argument of \textit{main} of |\childdocmain| can be empty.

An alternative version of the command line processing described
in \secref{sec:commandline} using the detection mechanism reads:
%
\begin{center}
|... -jobname "|\textit{target}|" "|[\textit{flags}]%
[|\def\jobname{|\textit{dest}|}|]|\input{|\textit{main}|}"|
\end{center}

%%%%%%%%%%%%%%%%%%%%%%%%%%%%%%%%%%%%%%%%%%%%%%%%%%%%%%%%%%%%%%%%%%%%%%%%%%%%%%%%
\subsection{Manual Code}
\label{sec:manual}

In case one cannot be certain whether the definitions file |childdoc.def|
is installed on the target \TeX{} distribution
and one prefers not to ship it,
it is conceivable to paste a few relevant commands into the sources.

To that end, drop all statements |\input{childdoc.def}|
and perform the replacements as outlined below.
Instead of |\childdocmain{|\textit{main}|}| add the following code
to the top of the main file:
%
\begin{center}
\begin{tabular}{l}
|\||ifdefined\childdocname\endinput\||fi\newif\ifchilddoc|\\
|\edef\childdocname{\scantokens\expandafter{\jobname\noexpand}}|\\
|\def\childdocmain{|\textit{main}|}\||ifx\childdocmain\childdocname\||else|\\
|\childdoctrue\includeonly{\childdocname}\let\jobname\childdocmain\||fi|\\
\end{tabular}
\end{center}
%
Instead of |\childdocof{|\textit{main}|}| just include the main file
at the top of each child file:
%
\begin{center}
|\input{|\textit{main}|}|
\end{center}
%
A simple redirection |\childdocforward{|\textit{dest}|}| is achieved by:
%
\begin{center}
|\def\jobname{|\textit{dest}|}\input{\jobname}|
\end{center}
%
The redirection with prefix
|\childdocforwardprefix[|\textit{prefix}|]{|\textit{dest}|}|
is accomplished by:
%
\begin{center}
\begin{tabular}{l}
|{\edef\jobname{\scantokens\expandafter{\jobname\noexpand}}|\\
|\def\redirectjob |\textit{prefix}|#1~~~{\gdef\jobname{|\textit{dest}|#1}}|\\
|\expandafter\redirectjob\jobname~~~}\input{\jobname}|
\end{tabular}
\end{center}

In an alternative approach,
child documents can be compiled by a specific command line
without additional code or specific definitions:
%
\begin{center}
|... -jobname "|\textit{target}|" "|[\textit{flags}]%
|\includeonly{|\textit{dest}|}\input{|\textit{main}|}"|
\end{center}
%

%%%%%%%%%%%%%%%%%%%%%%%%%%%%%%%%%%%%%%%%%%%%%%%%%%%%%%%%%%%%%%%%%%%%%%%%%%%%%%%%
%%%%%%%%%%%%%%%%%%%%%%%%%%%%%%%%%%%%%%%%%%%%%%%%%%%%%%%%%%%%%%%%%%%%%%%%%%%%%%%%
\section{Information}

%%%%%%%%%%%%%%%%%%%%%%%%%%%%%%%%%%%%%%%%%%%%%%%%%%%%%%%%%%%%%%%%%%%%%%%%%%%%%%%%
\subsection{Copyright}

Copyright \copyright{} 2017--2018 Niklas Beisert

This work may be distributed and/or modified under the
conditions of the \LaTeX{} Project Public License, either version 1.3
of this license or (at your option) any later version.
The latest version of this license is in
  \url{http://www.latex-project.org/lppl.txt}
and version 1.3 or later is part of all distributions of \LaTeX{}
version 2005/12/01 or later.

This work has the LPPL maintenance status `maintained'.

The Current Maintainer of this work is Niklas Beisert.

This work consists of the files |README.txt|, |childdoc.ins| and |childdoc.dtx|
as well as the derived files |childdoc.def|, |cdocsamp.tex|
with |cdocsch1.tex|, |cdocsch2.tex|, |cdocspt3.tex|, |cdocspt4.tex|,
|cdocsdrf.tex|, |cdocsfn1.tex|, |cdocsfn2.tex|
as well as |childdoc.pdf|.

%%%%%%%%%%%%%%%%%%%%%%%%%%%%%%%%%%%%%%%%%%%%%%%%%%%%%%%%%%%%%%%%%%%%%%%%%%%%%%%%
\subsection{Files and Installation}

The package consists of the files:
%
\begin{center}
\begin{tabular}{ll}
    |README.txt|   & readme file \\
    |childdoc.ins| & installation file \\
    |childdoc.dtx| & source file \\
    |childdoc.def| & definition file \\
    |cdocsamp.tex| & sample main file \\
    |cdocsch1.tex| & sample include file \\
    |cdocsch2.tex| & sample include file \\
    |cdocspt3.tex| & sample part file \\
    |cdocspt4.tex| & sample part file \\
    |cdocsdrf.tex| & sample redirection file \\
    |cdocsfn1.tex| & sample redirection file \\
    |cdocsfn2.tex| & sample redirection file \\
    |childdoc.pdf| & manual
\end{tabular}
\end{center}
%
The distribution consists of the files
|README.txt|, |childdoc.ins| and |childdoc.dtx|.
%
\begin{itemize}
\item
Run (pdf)\LaTeX{} on |childdoc.dtx|
to compile the manual |childdoc.pdf| (this file).
\item
Run \LaTeX{} on |childdoc.ins| to create the definitions file |childdoc.def|
and the sample |cdocsamp.tex| with include files
|cdocsch1.tex|, |cdocsch2.tex|, |cdocspt3.tex|, |cdocspt4.tex|,
|cdocsdrf.tex|, |cdocsfn1.tex|, |cdocsfn2.tex|.
Then copy the file |childdoc.def| to an appropriate directory of your \LaTeX{}
distribution, e.g.\ \textit{texmf-root}|/tex/latex/childdoc|.
\end{itemize}

%%%%%%%%%%%%%%%%%%%%%%%%%%%%%%%%%%%%%%%%%%%%%%%%%%%%%%%%%%%%%%%%%%%%%%%%%%%%%%%%
\subsection{Related CTAN Packages}

There are several other packages which offer a similar functionality:
%
\begin{itemize}
\item
The packages
\href{http://ctan.org/pkg/docmute}{\textsf{docmute}},
\href{http://ctan.org/pkg/includex}{\textsf{includex}} and
\href{http://ctan.org/pkg/standalone}{\textsf{standalone}}
provide commands to include only the document body of
a child file thus allowing both files to be compiled individually.
\item
The packages \href{http://ctan.org/pkg/subdocs}{\textsf{subdocs}}
and \href{http://ctan.org/pkg/subfiles}{\textsf{subfiles}}
provide structures in which the main and child documents can be
encapsulated and allowing them to be compiled individually.
The inclusion mechanism is different from the conventional |\include|.
\item
The package \href{http://ctan.org/pkg/combine}{\textsf{combine}}
is an elaborate solution to combine several documents into one.
\end{itemize}
%
See also the CTAN topic \href{http://ctan.org/topic/subdocs}{\textsf{subdocs}}
for further related packages.
The present package differs from the above solutions in that
a document structure constructed with the conventional |\include| mechanism
just needs two extra commands at the top of every file
such that all constituent files can be compiled individually.

%%%%%%%%%%%%%%%%%%%%%%%%%%%%%%%%%%%%%%%%%%%%%%%%%%%%%%%%%%%%%%%%%%%%%%%%%%%%%%%%
%\subsection{Feature Suggestions}
%
%The following is a list of features which may be useful for future
%versions of this package:
%%
%\begin{itemize}
%\item
%\ldots
%\end{itemize}

%%%%%%%%%%%%%%%%%%%%%%%%%%%%%%%%%%%%%%%%%%%%%%%%%%%%%%%%%%%%%%%%%%%%%%%%%%%%%%%%
\subsection{Revision History}

%%%%%%%%%%%%%%%%%%%%%%%%%%%%%%%%%%%%%%%%
\paragraph{v2.0:} 2018/12/30

\begin{itemize}
\item
immediate forward processing
\item
added |\childdocby| mechanism
\item
manual restructured
\end{itemize}

%%%%%%%%%%%%%%%%%%%%%%%%%%%%%%%%%%%%%%%%
\paragraph{v1.6:} 2018/01/17

\begin{itemize}
\item
application for development of include files
\item
corrections to manual
\end{itemize}

%%%%%%%%%%%%%%%%%%%%%%%%%%%%%%%%%%%%%%%%
\paragraph{v1.5:} 2017/05/21

\begin{itemize}
\item
more complete structuring introduced
\item
|\childdocof| introduced
\item
|\childdoc| renamed to |\childdocmain|
\item
|\childredirect| renamed to |\childdocforward| and |\childdocforwardprefix|
and functionality expanded
\end{itemize}

%%%%%%%%%%%%%%%%%%%%%%%%%%%%%%%%%%%%%%%%
\paragraph{v1.0:} 2017/04/27

\begin{itemize}
\item
manual and install package
\item
first version published on CTAN
\end{itemize}

%%%%%%%%%%%%%%%%%%%%%%%%%%%%%%%%%%%%%%%%
\paragraph{v0.6:} 2017/04/26

\begin{itemize}
\item
redirection mechanism added
\end{itemize}

%%%%%%%%%%%%%%%%%%%%%%%%%%%%%%%%%%%%%%%%
\paragraph{v0.5:} 2017/04/26

\begin{itemize}
\item
functionality in definition file
\end{itemize}


%%%%%%%%%%%%%%%%%%%%%%%%%%%%%%%%%%%%%%%%%%%%%%%%%%%%%%%%%%%%%%%%%%%%%%%%%%%%%%%%
%%%%%%%%%%%%%%%%%%%%%%%%%%%%%%%%%%%%%%%%%%%%%%%%%%%%%%%%%%%%%%%%%%%%%%%%%%%%%%%%
%%%%%%%%%%%%%%%%%%%%%%%%%%%%%%%%%%%%%%%%%%%%%%%%%%%%%%%%%%%%%%%%%%%%%%%%%%%%%%%%
\appendix

\settowidth\MacroIndent{\rmfamily\scriptsize 000\ }

 \DocInput{childdoc.dtx}

\end{document}
%</driver>
% \fi
%
% %%%%%%%%%%%%%%%%%%%%%%%%%%%%%%%%%%%%%%%%%%%%%%%%%%%%%%%%%%%%%%%%%%%%%%%%%%%%%%
% %%%%%%%%%%%%%%%%%%%%%%%%%%%%%%%%%%%%%%%%%%%%%%%%%%%%%%%%%%%%%%%%%%%%%%%%%%%%%%
% \section{Sample}
%\iffalse
%<*samplemain>
%\fi
%
% The following presents a sample document
% with two chapters, two parts, a title page,
% a compile flag as well as three forwarding files to set the flag.
% It consists of eight |.tex| files:
% \begin{center}
% \begin{tabular}{ll}
% |cdocsamp.tex|&main file\\
% |cdocsch1.tex|&include file for chapter 1\\
% |cdocsch2.tex|&include file for chapter 2\\
% |cdocspt3.tex|&include file for part 3\\
% |cdocspt4.tex|&include file for part 4\\
% |cdocsdrf.tex|&forwarding file for main file in draft mode\\
% |cdocsfi1.tex|&forwarding file for final version of chapter 1\\
% |cdocsfi2.tex|&forwarding file for final version of chapter 2\\
% \end{tabular}
% \end{center}
% Each of the eight files can be compiled directly by the \LaTeX{} compiler.
%
% %%%%%%%%%%%%%%%%%%%%%%%%%%%%%%%%%%%%%%
% \paragraph{Main File.}
%
% The main file is called |cdocsamp.tex|.
%
% Load the \textsf{childdoc} definitions and
% declare the filename for the main document:
%    \begin{macrocode}
\input{childdoc.def}
\childdocmain{}
%    \end{macrocode}

% Optional override for |\version| flag:
%    \begin{macrocode}
%%\ifchilddoc\else\providecommand{\version}{draft}\fi
%    \end{macrocode}

% Define the default values for the |\version| flag
% (|final| for the main file and |draft| for childs):
%    \begin{macrocode}
\ifchilddoc
\providecommand{\version}{draft}
\else
\providecommand{\version}{final}
\fi
%    \end{macrocode}

% Load the standard document class:
%    \begin{macrocode}
\documentclass[12pt]{article}
%    \end{macrocode}

% Start the document body:
%    \begin{macrocode}
\begin{document}
%    \end{macrocode}

% Declare a title page.
% Print title, part of document being processed and version flag:
%    \begin{macrocode}
\addtocounter{page}{-1}
\begin{center}
{\LARGE\bfseries{}childdoc example\par}
\vspace{1cm}
\ifchilddoc
\ifchilddocmanual part\else chapter\fi:
`\childdocname' of `\childdocjob'\par
\else
main document: `\childdocjob'\par
\fi
version: \version\par
\end{center}
\newpage
%    \end{macrocode}

% Manually include selected file,
% otherwise process as usual:
%    \begin{macrocode}
\ifchilddocmanual
\section*{part `\childdocname'}
\input{\childdocname}
\else
%    \end{macrocode}

% Include the two chapters:
%    \begin{macrocode}
\include{cdocsch1}
\include{cdocsch2}
%    \end{macrocode}

% Include the two parts unless only chapters should be displayed:
%    \begin{macrocode}
\ifchilddoc\else
\section{part three}
\input{cdocspt3}
\section{part four}
\input{cdocspt4}
\fi
%    \end{macrocode}

% Process as usual until here:
%    \begin{macrocode}
\fi
%    \end{macrocode}

% End of document body:
%    \begin{macrocode}
\end{document}
%    \end{macrocode}
%\iffalse
%</samplemain>
%\fi
%
% %%%%%%%%%%%%%%%%%%%%%%%%%%%%%%%%%%%%%%
% \paragraph{Chapter Include Files.}
%
% The include files are called |cdocsch1.tex| and |cdocsch2.tex|.
%
%\iffalse
%<*samplechap1|samplechap2>
%\fi

% Optional override for |\version| flag:
%    \begin{macrocode}
%%\providecommand{\version}{final}
%    \end{macrocode}

% Include the main document:
%    \begin{macrocode}
\input{childdoc.def}
\childdocof{cdocsamp}
%    \end{macrocode}

%\iffalse
%</samplechap1|samplechap2>
%\fi
%
%\iffalse
%<*samplechap1>
%\fi
% Some text for chapter 1:
%    \begin{macrocode}
\section{one}
some text in chapter one
%    \end{macrocode}

%\iffalse
%</samplechap1>
%\fi
% Some text for chapter 2:
%\iffalse
%<*samplechap2>
%\fi
%    \begin{macrocode}
\section{two}
more text in chapter two
%    \end{macrocode}

%\iffalse
%</samplechap2>
%\fi
%
% %%%%%%%%%%%%%%%%%%%%%%%%%%%%%%%%%%%%%%
% \paragraph{Part Include Files.}
%
% The include files are called |cdocspt3.tex| and |cdocspt4.tex|.
%
%\iffalse
%<*samplepart3|samplepart4>
%\fi

% Optional override for |\version| flag:
%    \begin{macrocode}
%%\providecommand{\version}{final}
%    \end{macrocode}

% Include the main document:
%    \begin{macrocode}
\input{childdoc.def}
\childdocby{cdocsamp}
%    \end{macrocode}

%\iffalse
%</samplepart3|samplepart4>
%\fi
%
%\iffalse
%<*samplepart3>
%\fi
% Some text for part 3:
%    \begin{macrocode}
some text in part three
%    \end{macrocode}

%\iffalse
%</samplepart3>
%\fi
% Some text for part 4:
%\iffalse
%<*samplepart4>
%\fi
%    \begin{macrocode}
more text in part four
%    \end{macrocode}

%\iffalse
%</samplepart4>
%\fi
%
% %%%%%%%%%%%%%%%%%%%%%%%%%%%%%%%%%%%%%%
% \paragraph{Forwarding for a Complete Draft.}
%
% The following forwarding file |cdocsdrf.tex|
% compiles the main document in draft mode:
%\iffalse
%<*sampledraft>
%\fi
%    \begin{macrocode}
\def\version{draft}
\input{childdoc.def}
\childdocforward{cdocsamp}
%    \end{macrocode}

%\iffalse
%</sampledraft>
%\fi
%
% %%%%%%%%%%%%%%%%%%%%%%%%%%%%%%%%%%%%%%
% \paragraph{Forwarding for Final Version of the Chapters.}
%
% The following forwarding files |cdocsfn1.tex| and |cdocsfn2.tex|
% (with identical content)
% compile the final versions of the child documents
% |cdocsch1.tex| and |cdocsch2.tex|, respectively:
%\iffalse
%<*samplefinal>
%\fi
%    \begin{macrocode}
\def\version{final}
\input{childdoc.def}
\childdocforwardprefix[cdocsamp]{cdocsfn}{cdocsch}
%    \end{macrocode}

%\iffalse
%</samplefinal>
%\fi
%
% %%%%%%%%%%%%%%%%%%%%%%%%%%%%%%%%%%%%%%
% \paragraph{Command Line Processing.}
%
% The following three command lines generate the output files
% |cdocscld|, |cdocscl1| and |cdocscl2|
% which should be identical to
% |cdocsdrf|, |cdocsch1| and |cdocsfn2|, respectively:
% \begin{center}
% \begin{tabular}{l}
% |latex -jobname cdocscld \|\\
% |  "\def\version{draft}\input{childdoc.def}\childdocforward{cdocsamp}"|\\
% |latex -jobname cdocscl1 \|\\
% |  "\input{childdoc.def}\childdocforward[cdocsamp]{cdocsch1}"|\\
% |latex -jobname cdocscl2 \|\\
% |  "\def\version{final}\input{childdoc.def}\childdocforward{cdocsch2}"|
% \end{tabular}
% \end{center}
% Note that the trailing backslash on each first line
% merely continues the input to the second line
% (for convenient cut ant paste).
% Furthermore, the command |latex| can be replaced by any
% of its alternative versions such as |pdflatex|.
%
% %%%%%%%%%%%%%%%%%%%%%%%%%%%%%%%%%%%%%%%%%%%%%%%%%%%%%%%%%%%%%%%%%%%%%%%%%%%%%%
% %%%%%%%%%%%%%%%%%%%%%%%%%%%%%%%%%%%%%%%%%%%%%%%%%%%%%%%%%%%%%%%%%%%%%%%%%%%%%%
% \section{Implementation}
%\iffalse
%<*package>
%\fi
%
% This section describes the definitions file |childdoc.def|.

% The definitions cannot be loaded using |\usepackage| or |\RequirePackage|
% which has a mechanism to prevent loading a style file more than once.
% When loading the definitions by means of |\input|
% multiple instances have to be prevented manually:
%\iffalse
%This code needs to be before the `\ProvidesFile' directive
%which is defined at the beginning of this file.
%Therefore it is also placed there and commented out here.
%</package>
%<*discard>
%\fi
%    \begin{macrocode}
\ifdefined\childdocmain\endinput\fi
%    \end{macrocode}
%\iffalse
%</discard>
%<*package>
%\fi
%
% \macro{\ifchilddoc}
% \macro{\ifchilddocmanual}
% The conditional |\ifchilddoc| tells whether a
% child (true) or main (false) document is being compiled.
% The conditional |\ifchilddocmanual| tells whether
% the |\includeonly| mechanism is used (false) or
% the selection of child files must be performed manually (true).
% The definitions initialise to false:
%    \begin{macrocode}
\newif\ifchilddoc
\newif\ifchilddocmanual
%    \end{macrocode}

% \macro{\childdocname}
% \macro{\childdocjob}
% The macro |\childdocname| stores the name of the main document
% to be compiled. The macro |\childdocjob| stores the name of
% the document on which the \LaTeX{} compiler was originally invoked.
% The content of |\jobname| cannot be compared
% to filenames specified in the source due to different catcodes.
% The following code rescans |\jobname|, stores the result
% in |\childdocname| and saves a copy in |\childdocjob|:
%    \begin{macrocode}
\edef\childdocname{\scantokens\expandafter{\jobname\noexpand}}
\let\childdocjob\childdocname
%    \end{macrocode}

% \macro{\childdocdisable}
% The macro |\childdocdisable| prevents the main file
% from being processed more than once.
% At this stage, the main document command |\childdocmain|
% is assumed to be called once again where it should do nothing.
% Any subsequent call to it should prevent
% a secondary processing of the main document
% It overwrites the forwarding commands
% |\childdocof| and |\childdocforward|
% with empty macros to prevent further inclusions of the main document:
%    \begin{macrocode}
\newcommand{\childdocdisable}
{
  \renewcommand{\childdocmain}[1]{\renewcommand{\childdocmain}[1]{\endinput}}
  \renewcommand{\childdocof}[1]{}
  \renewcommand{\childdocby}[2][]{}
  \renewcommand{\childdocforward}[2][]{}
  \renewcommand{\childdocdisable}{}
}
%    \end{macrocode}

% \macro{\childdocmain}
% The macro |\childdocmain| is to be called at the top of the main file
% with nothing or the main filename (without extension) as argument.
% First, it breaks loops.
% If the argument is not empty and does not match |\childdocname|
% (which is set by the first inclusion of |childdoc.def|),
% |\ifchilddoc| is set to true, |\includeonly| is applied to the child file
% and |\jobname| is set to the main file
% (for proper handling of |.aux| files):
%    \begin{macrocode}
\newcommand{\childdocmain}[1]
{
  \childdocdisable\childdocmain{}
  \if?#1?\else
    \begingroup
      \def\childdoctmp{#1}
      \ifx\childdoctmp\childdocname
        \def\childdoctmp{}
      \else
        \def\childdoctmp
        {
          \childdoctrue
          \includeonly{\childdocname}
          \def\childdocjob{#1}
          \def\jobname{#1}
        }
      \fi
      \expandafter
    \endgroup
    \childdoctmp
  \fi
}
%    \end{macrocode}

% \macro{\childdocof}
% The command |\childdocof| redirects
% compilation to the main file |#1|.
%    \begin{macrocode}
\newcommand{\childdocof}[1]
{
  \childdocdisable
  \childdoctrue
  \includeonly{\childdocname}
  \def\jobname{#1}
  \def\childdocjob{#1}
  \input{#1}
}
%    \end{macrocode}

% \macro{\childdocby}
% The command |\childdocby| ....
%    \begin{macrocode}
\newcommand{\childdocby}[2][]
{
  \childdocdisable
  \childdoctrue
  \childdocmanualtrue
  \if?#1?\else
    \def\jobname{#2}
  \fi
  \def\childdocjob{#2}
  \input{#2}
  \endinput
}
%    \end{macrocode}

% \macro{\childdocforward}
% The command |\childdocforward| redirects
% compilation to the main file or
% (if the optional argument is given) a child file.
% Parameters are set as if the main file
% or a child file starting with |\childdocof| was compiled.
% Then compilation is handed over to the main file:
%    \begin{macrocode}
\newcommand{\childdocforward}[2][]
{
  \begingroup
    \if?#1?
      \def\childdoctmp
      {
        \def\childdocname{#2}
        \def\childdocjob{#2}
        \def\jobname{#2}
        \input{#2}
        \endinput
      }
    \else
      \def\childdoctmp
      {
        \childdocdisable
        \def\childdocname{#2}
        \childdoctrue
        \includeonly{#2}
        \def\childdocjob{#1}
        \def\jobname{#1}
        \input{#1}
        \endinput
      }
    \fi
    \expandafter
  \endgroup
  \childdoctmp
}
%    \end{macrocode}

% \macro{\childdocforwardprefix}
% The command |\childdocforwardprefix| redirects
% compilation to the main or a child file by means of a pattern.
% The prefix |#1| in the current filename is replaced by |#2|
% and the suffix of the current filename is kept
% (it is assumed that the filename does not contain the substring `|~~~|'
% which is used as a delimiter).
% Compilation is handed over to the new file by |\childdocforward|:
%    \begin{macrocode}
\newcommand{\childdocforwardprefix}[3][]
{
  \begingroup
    \def\childdocextract #2##1~~~{\def\childdoctmp{\childdocforward[#1]{#3##1}}}
    \expandafter\childdocextract\childdocname~~~
    \expandafter
  \endgroup
  \childdoctmp
}
%    \end{macrocode}

% \macro{\childdoc}
% The deprecated macro |\childdoc| is a legacy version of |\childdocmain|:
%    \begin{macrocode}
\newcommand{\childdoc}{\childdocmain}
%    \end{macrocode}

% \macro{\childdocredirect}
% The deprecated macro |\childdocredirect| is a legacy version
% of |\childdocforward| and |\childdocforwardprefix|:
%    \begin{macrocode}
\newcommand{\childdocredirect}[2][]
{
  \begingroup
    \if?#1?
      \def\childdoctmp{\childdocforward{#2}}
    \else
      \def\childdoctmp{\childdocforwardprefix{#1}{#2}}
    \fi
    \expandafter
  \endgroup
  \childdoctmp
}
%    \end{macrocode}

%\iffalse
%</package>
%\fi
%
\endinput
|\\
|\childdocof{|\textit{main}|}|\\
\end{tabular}
\end{center}
at the top of every child file \textit{child}
which is included by |\include{|\textit{child}|}|
from within the main file
(or at least for those files to be compiled individually).
The argument \textit{main} must be the filename of the main file.

There are a couple of
considerations in setting up the main and child documents:

%%%%%%%%%%%%%%%%%%%%%%%%%%%%%%%%%%%%%%%%
\paragraph{Restrictions.}

Please note the following restrictions:
\begin{itemize}
\item
|\childdocmain| must be called with one argument \textit{main}
to ensure compatibility with earlier version of the package.
It must either be empty (|\childdocmain{}|)
or precisely match the filename of the main file in which it is specified.
See \secref{sec:detection} for further information.
\item
The filename \textit{main} must be specified without the |.tex| extension.
\item
The filename \textit{main} is case sensitive
(even in case-insensitive file systems)
due to internal string comparison.
\item
The argument \textit{main} should be fully expanded, it cannot be a macro.
\item
Subdirectories and special characters should be avoided in filenames.
\item
The command |\childdocmain{|\textit{main}|}| must be followed by a whitespace.
It should not be followed immediately by another command
or by a comment mark `|%|'.
This is because the \TeX{} parser reads the token immediately following
the argument of |\childdocmain| and puts it
at the beginning of every child section;
however, a white\-space is ignored.
\end{itemize}

%%%%%%%%%%%%%%%%%%%%%%%%%%%%%%%%%%%%%%%%
\paragraph{Content of Main File.}

It is advisable to place all content in the child files included by |\include|.
Any output contained in the main file will appear in all child documents
unless suppressed manually;
it cannot be suppressed automatically by the |\includeonly| directive
and thus should normally be avoided.
A method to include some content in the main file
by means of conditional processing is described in \secref{sec:conditional}.

%%%%%%%%%%%%%%%%%%%%%%%%%%%%%%%%%%%%%%%%
\paragraph{Page Numbering.}

When only a part of the document is compiled,
the appropriate numbering of pages
(as well as other status parameters)
is determined from the |.aux| files.
The latter contain information from previous passes.
However this information needs to propagate through
all intermediate child documents.
Therefore the page numbering in child documents may well
be inconsistent until the complete document is compiled at least once.

A useful (if unconventional) way to always ensure a consistent
page numbering is to restart the numbering in each child document
and denote the pages by `\textit{child}|.|\textit{page}'
where \textit{child} represents the chapter/section number of the child file.
This can be achieved by the command
|\numberwithin{page}{|\textit{child}|}|
of the \textsf{amsmath} package
where \textit{child} can be |chapter| or |section|
depending on the chosen structuring.
Alternatively, one can modify the macro |\thepage| appropriately
and reset the counter |page| at the start of each child file.

%%%%%%%%%%%%%%%%%%%%%%%%%%%%%%%%%%%%%%%%%%%%%%%%%%%%%%%%%%%%%%%%%%%%%%%%%%%%%%%%
\subsection{Conditional Processing}
\label{sec:conditional}

The package provides a mechanism to compile different versions
of a document. To customise the versions further some conditional processing
can come in handy to distinguish which version is being compiled.
The package provides two macros to describe the compilation context:

%%%%%%%%%%%%%%%%%%%%%%%%%%%%%%%%%%%%%%%%
\DescribeMacro{\ifchilddoc}
The conditional |\ifchilddoc| distinguishes between the compilation of
child documents and the main document:
%
\begin{center}
|\ifchilddoc |\textit{child-code}| |[|\||else |\textit{main-code}]| \||fi|
\end{center}

%%%%%%%%%%%%%%%%%%%%%%%%%%%%%%%%%%%%%%%%
\DescribeMacro{\childdocname}
\DescribeMacro{\childdocjob}
The macro |\childdocname| contains the filename (without extension)
of the main or child file being processed.
Note that |\childdocjob| will always contain the name of the main file.

%%%%%%%%%%%%%%%%%%%%%%%%%%%%%%%%%%%%%%%%
\paragraph{Title Page.}

Conditional processing can be used to include a title or banner page
in the main document when proper precautions are taken.
Importantly, the code in the main file should ensure that the page counter
(as well as other status parameters which are stored in the |.aux| files)
takes the same value after the conditional processing.
Otherwise the page numbers may take divergent values
depending on which part is compiled.

For example, a title page could be declared by:
%
\begin{center}
\begin{tabular}{l}
|\ifchilddoc\||else|\\
|\addtocounter{page}{-1}|\\
\textit{code for title page}\\
|\newpage|\\
|\||fi|
\end{tabular}
\end{center}
%
A banner page for the child documents can be generated by:
%
\begin{center}
\begin{tabular}{l}
|\ifchilddoc|\\
|\addtocounter{page}{-1}|\\
\textit{code for banner page}\\
|\newpage|\\
|\||fi|
\end{tabular}
\end{center}
%
Here one could write a message such as:
\begin{center}
|This is the part \childdocname{} of \childdocjob{}.|
\end{center}

%%%%%%%%%%%%%%%%%%%%%%%%%%%%%%%%%%%%%%%%%%%%%%%%%%%%%%%%%%%%%%%%%%%%%%%%%%%%%%%%
\subsection{Flags}
\label{sec:flags}

The package makes it easy to generate different versions
of the main or child documents.
To this end compilation flags can be defined
and assigned different default values.
They will be particularly useful in conjunction
with the forwarding mechanism described in \secref{sec:forward}.

For example, it may be useful to have a flag |\version|
which can be set to |draft| or |final|.
The document source will contain some conditional code
depending on the value of |\version|.
Suppose further, the flag should default to |final| for the main file
and to |draft| for child files
which is a natural assignment for editing the document.
This is achieved by placing the following code
in the preamble of the main document
(below the |\childdocmain| directive):
%
\begin{center}
\begin{tabular}{l}
|\ifchilddoc|\\
|\providecommand{\version}{draft}|\\
|\||else|\\
|\providecommand{\version}{final}|\\
|\||fi|
\end{tabular}
\end{center}
%
The definition by |\providecommand| makes sure
that previous definitions are not overwritten.
Further statements |\providecommand{\version}{...}|
can thus be added before the above code to override it.

For the main file, one might add a line
(between |\childdocmain| and the above block)
%
\begin{center}
|%\ifchilddoc\||else\providecommand{\version}{draft}\||fi|
\end{center}
%
which can be uncommented to produce a draft version.
Likewise one can add a line to the very top of a child file
(above the |\childdocof{|\textit{main}|}| directive)
%
\begin{center}
|%\providecommand{\version}{final}|
\end{center}
%
which can be uncommented to produce the final version of this child document.

%%%%%%%%%%%%%%%%%%%%%%%%%%%%%%%%%%%%%%%%%%%%%%%%%%%%%%%%%%%%%%%%%%%%%%%%%%%%%%%%
\subsection{Forwarding}
\label{sec:forward}

Different versions of the main or child documents
using compilation flags as described in \secref{sec:flags}
can be (permanently) stored in different files
for convenient compilation, viewing and distribution.
To this end, the package defines a command
to pass on compilation to a different file:

%%%%%%%%%%%%%%%%%%%%%%%%%%%%%%%%%%%%%%%%
\DescribeMacro{\childdocforward}
The command |\childdocforward| redirects processing to
another source file:
%
\begin{center}
\begin{tabular}{l}
|% \iffalse
%
% childdoc.dtx Copyright (C) 2017-2018 Niklas Beisert
%
% This work may be distributed and/or modified under the
% conditions of the LaTeX Project Public License, either version 1.3
% of this license or (at your option) any later version.
% The latest version of this license is in
%   http://www.latex-project.org/lppl.txt
% and version 1.3 or later is part of all distributions of LaTeX
% version 2005/12/01 or later.
%
% This work has the LPPL maintenance status `maintained'.
%
% The Current Maintainer of this work is Niklas Beisert.
%
% This work consists of the files childdoc.dtx and childdoc.ins
% and the derived files childdoc.def and cdocsamp.tex with
% cdocsch1.tex, cdocsch2.tex, cdocsdrf.tex, cdocsfn1.tex, cdocsfn2.tex.
%
%<package>\ifdefined\childdocmain\endinput\fi
%<package>\ProvidesFile{childdoc.def}[2018/12/30 v2.0 child document driver]
%<samplemain>\ProvidesFile{cdocsamp.tex}[2018/12/30 v2.0 sample for childdoc]
%<*driver>
%\ProvidesFile{childdoc.drv}[2018/12/30 v2.0 childdoc reference manual file]
\PassOptionsToClass{10pt,a4paper}{article}
\documentclass{ltxdoc}

\usepackage[margin=35mm]{geometry}
\usepackage{hyperref}
\usepackage{hyperxmp}
\usepackage[usenames]{color}

\hypersetup{colorlinks=true}
\hypersetup{pdfstartview=FitH}
\hypersetup{pdfpagemode=UseNone}
\hypersetup{pdfsource={}}
\hypersetup{pdflang={en-UK}}
\hypersetup{pdfcopyright={Copyright 2017-2018 Niklas Beisert.
  This work may be distributed and/or modified under the
  conditions of the LaTeX Project Public License, either version 1.3
  of this license or (at your option) any later version.}}
\hypersetup{pdflicenseurl={http://www.latex-project.org/lppl.txt}}
\hypersetup{pdfcontactaddress={ETH Zurich, ITP, HIT K,
  Wolfgang-Pauli-Strasse 27}}
\hypersetup{pdfcontactpostcode={8093}}
\hypersetup{pdfcontactcity={Zurich}}
\hypersetup{pdfcontactcountry={Switzerland}}
\hypersetup{pdfcontactemail={nbeisert@itp.phys.ethz.ch}}
\hypersetup{pdfcontacturl={http://people.phys.ethz.ch/\xmptilde nbeisert/}}

\newcommand{\secref}[1]{\hyperref[#1]{section \ref*{#1}}}

\parskip1ex
\parindent0pt
\let\olditemize\itemize
\def\itemize{\olditemize\parskip0pt}

\begin{document}

\title{The \textsf{childdoc} Package}
\hypersetup{pdftitle={The childdoc Package}}
\author{Niklas Beisert\\[2ex]
  Institut f\"ur Theoretische Physik\\
  Eidgen\"ossische Technische Hochschule Z\"urich\\
  Wolfgang-Pauli-Strasse 27, 8093 Z\"urich, Switzerland\\[1ex]
  \href{mailto:nbeisert@itp.phys.ethz.ch}
  {\texttt{nbeisert@itp.phys.ethz.ch}}}
\hypersetup{pdfauthor={Niklas Beisert}}
\hypersetup{pdfsubject={Manual for the LaTeX2e Package childdoc}}
\date{30 December 2018, \textsf{v2.0}}
\maketitle

\begin{abstract}\noindent
\textsf{childdoc} is a \LaTeXe{} package
that enables the direct compilation
of document sections included by |\include|
to individual files.
\end{abstract}

\begingroup
\parskip0ex
\tableofcontents
\endgroup

%%%%%%%%%%%%%%%%%%%%%%%%%%%%%%%%%%%%%%%%%%%%%%%%%%%%%%%%%%%%%%%%%%%%%%%%%%%%%%%%
%%%%%%%%%%%%%%%%%%%%%%%%%%%%%%%%%%%%%%%%%%%%%%%%%%%%%%%%%%%%%%%%%%%%%%%%%%%%%%%%
\section{Introduction}

\LaTeX{} provides a mechanism to structure a large document (such as a book)
into a main file and several child files (containing the chapters)
using the |\include| command.
This mechanism is beneficial for documents
which span hundreds of pages in order to
make the source file(s) more manageable.
Moreover, compilation can be restricted to
selected child files by means of the |\includeonly| command.
The latter feature can be used to reduce the compilation time while editing
(this was significantly more useful in the earlier days of \LaTeX{})
or to generate a smaller document which is easier to navigate.
Another application of |\includeonly| is to generate
documents consisting of selected parts of the complete document.

However, there are a few drawbacks of the plain |\include| mechanism:
\begin{itemize}
\item
The child files cannot be compiled on their own,
they can only be compiled via the main file.
A naive editing environment
(such as a text editor with an option
to have the current file processed by \LaTeX)
may require one to switch to the main file before compiling;
attempting to compile the child file produces errors.
\item
The main file must be modified (each time)
to adjust the |\includeonly| command
to the present needs. This easily leaves the main file in a messy state.
\item
The generated document will always carry the filename
of the main document. This is inconvenient if
several child files are to be compiled and
to be kept for distribution.
\end{itemize}

The present package provides a simple interface
to make child files individually compilable by \LaTeX{}.
Compiling a child file then has the same effect as compiling
the main file with an |\includeonly| command
to select the appropriate child.
Moreover the generated document will carry the name of the child
rather than the main file.
This resolves all three above issues.

This feature is meant to make the editing of books,
thesis documents and lecture notes somewhat more convenient.
However, the package can also be used efficiently for
composing a series of documents (such as exercise sheets)
which are typically distributed individually.
It then assists the author in generating the individual documents
(potentially in different versions)
as well as a document containing the collected series.
Another application is in developing style files
or other kinds of included material
where compilation of the style file could redirect
to a sample or test file.

%%%%%%%%%%%%%%%%%%%%%%%%%%%%%%%%%%%%%%%%%%%%%%%%%%%%%%%%%%%%%%%%%%%%%%%%%%%%%%%%
%%%%%%%%%%%%%%%%%%%%%%%%%%%%%%%%%%%%%%%%%%%%%%%%%%%%%%%%%%%%%%%%%%%%%%%%%%%%%%%%
\section{Usage}

First of all, the package \textsf{childdoc} is \emph{not} a standard
\LaTeXe{} |.sty| style file! Therefore it needs to be invoked in
a non-standard way.

%%%%%%%%%%%%%%%%%%%%%%%%%%%%%%%%%%%%%%%%%%%%%%%%%%%%%%%%%%%%%%%%%%%%%%%%%%%%%%%%
\subsection{Included Files}
\label{sec:include}

%%%%%%%%%%%%%%%%%%%%%%%%%%%%%%%%%%%%%%%%
\DescribeMacro{\childdocmain}
To use the package, add the commands
\begin{center}
\begin{tabular}{l}
|\input{childdoc.def}|\\
|\childdocmain{}|\\
\end{tabular}
\end{center}
at the very top of the main \LaTeX{} file,
in particular \emph{before} the |\documentclass| statement!
The argument of |\childdocmain| should be left empty
(but it must be present).

%%%%%%%%%%%%%%%%%%%%%%%%%%%%%%%%%%%%%%%%
\DescribeMacro{\childdocof}
Furthermore, add the commands
\begin{center}
\begin{tabular}{l}
|\input{childdoc.def}|\\
|\childdocof{|\textit{main}|}|\\
\end{tabular}
\end{center}
at the top of every child file \textit{child}
which is included by |\include{|\textit{child}|}|
from within the main file
(or at least for those files to be compiled individually).
The argument \textit{main} must be the filename of the main file.

There are a couple of
considerations in setting up the main and child documents:

%%%%%%%%%%%%%%%%%%%%%%%%%%%%%%%%%%%%%%%%
\paragraph{Restrictions.}

Please note the following restrictions:
\begin{itemize}
\item
|\childdocmain| must be called with one argument \textit{main}
to ensure compatibility with earlier version of the package.
It must either be empty (|\childdocmain{}|)
or precisely match the filename of the main file in which it is specified.
See \secref{sec:detection} for further information.
\item
The filename \textit{main} must be specified without the |.tex| extension.
\item
The filename \textit{main} is case sensitive
(even in case-insensitive file systems)
due to internal string comparison.
\item
The argument \textit{main} should be fully expanded, it cannot be a macro.
\item
Subdirectories and special characters should be avoided in filenames.
\item
The command |\childdocmain{|\textit{main}|}| must be followed by a whitespace.
It should not be followed immediately by another command
or by a comment mark `|%|'.
This is because the \TeX{} parser reads the token immediately following
the argument of |\childdocmain| and puts it
at the beginning of every child section;
however, a white\-space is ignored.
\end{itemize}

%%%%%%%%%%%%%%%%%%%%%%%%%%%%%%%%%%%%%%%%
\paragraph{Content of Main File.}

It is advisable to place all content in the child files included by |\include|.
Any output contained in the main file will appear in all child documents
unless suppressed manually;
it cannot be suppressed automatically by the |\includeonly| directive
and thus should normally be avoided.
A method to include some content in the main file
by means of conditional processing is described in \secref{sec:conditional}.

%%%%%%%%%%%%%%%%%%%%%%%%%%%%%%%%%%%%%%%%
\paragraph{Page Numbering.}

When only a part of the document is compiled,
the appropriate numbering of pages
(as well as other status parameters)
is determined from the |.aux| files.
The latter contain information from previous passes.
However this information needs to propagate through
all intermediate child documents.
Therefore the page numbering in child documents may well
be inconsistent until the complete document is compiled at least once.

A useful (if unconventional) way to always ensure a consistent
page numbering is to restart the numbering in each child document
and denote the pages by `\textit{child}|.|\textit{page}'
where \textit{child} represents the chapter/section number of the child file.
This can be achieved by the command
|\numberwithin{page}{|\textit{child}|}|
of the \textsf{amsmath} package
where \textit{child} can be |chapter| or |section|
depending on the chosen structuring.
Alternatively, one can modify the macro |\thepage| appropriately
and reset the counter |page| at the start of each child file.

%%%%%%%%%%%%%%%%%%%%%%%%%%%%%%%%%%%%%%%%%%%%%%%%%%%%%%%%%%%%%%%%%%%%%%%%%%%%%%%%
\subsection{Conditional Processing}
\label{sec:conditional}

The package provides a mechanism to compile different versions
of a document. To customise the versions further some conditional processing
can come in handy to distinguish which version is being compiled.
The package provides two macros to describe the compilation context:

%%%%%%%%%%%%%%%%%%%%%%%%%%%%%%%%%%%%%%%%
\DescribeMacro{\ifchilddoc}
The conditional |\ifchilddoc| distinguishes between the compilation of
child documents and the main document:
%
\begin{center}
|\ifchilddoc |\textit{child-code}| |[|\||else |\textit{main-code}]| \||fi|
\end{center}

%%%%%%%%%%%%%%%%%%%%%%%%%%%%%%%%%%%%%%%%
\DescribeMacro{\childdocname}
\DescribeMacro{\childdocjob}
The macro |\childdocname| contains the filename (without extension)
of the main or child file being processed.
Note that |\childdocjob| will always contain the name of the main file.

%%%%%%%%%%%%%%%%%%%%%%%%%%%%%%%%%%%%%%%%
\paragraph{Title Page.}

Conditional processing can be used to include a title or banner page
in the main document when proper precautions are taken.
Importantly, the code in the main file should ensure that the page counter
(as well as other status parameters which are stored in the |.aux| files)
takes the same value after the conditional processing.
Otherwise the page numbers may take divergent values
depending on which part is compiled.

For example, a title page could be declared by:
%
\begin{center}
\begin{tabular}{l}
|\ifchilddoc\||else|\\
|\addtocounter{page}{-1}|\\
\textit{code for title page}\\
|\newpage|\\
|\||fi|
\end{tabular}
\end{center}
%
A banner page for the child documents can be generated by:
%
\begin{center}
\begin{tabular}{l}
|\ifchilddoc|\\
|\addtocounter{page}{-1}|\\
\textit{code for banner page}\\
|\newpage|\\
|\||fi|
\end{tabular}
\end{center}
%
Here one could write a message such as:
\begin{center}
|This is the part \childdocname{} of \childdocjob{}.|
\end{center}

%%%%%%%%%%%%%%%%%%%%%%%%%%%%%%%%%%%%%%%%%%%%%%%%%%%%%%%%%%%%%%%%%%%%%%%%%%%%%%%%
\subsection{Flags}
\label{sec:flags}

The package makes it easy to generate different versions
of the main or child documents.
To this end compilation flags can be defined
and assigned different default values.
They will be particularly useful in conjunction
with the forwarding mechanism described in \secref{sec:forward}.

For example, it may be useful to have a flag |\version|
which can be set to |draft| or |final|.
The document source will contain some conditional code
depending on the value of |\version|.
Suppose further, the flag should default to |final| for the main file
and to |draft| for child files
which is a natural assignment for editing the document.
This is achieved by placing the following code
in the preamble of the main document
(below the |\childdocmain| directive):
%
\begin{center}
\begin{tabular}{l}
|\ifchilddoc|\\
|\providecommand{\version}{draft}|\\
|\||else|\\
|\providecommand{\version}{final}|\\
|\||fi|
\end{tabular}
\end{center}
%
The definition by |\providecommand| makes sure
that previous definitions are not overwritten.
Further statements |\providecommand{\version}{...}|
can thus be added before the above code to override it.

For the main file, one might add a line
(between |\childdocmain| and the above block)
%
\begin{center}
|%\ifchilddoc\||else\providecommand{\version}{draft}\||fi|
\end{center}
%
which can be uncommented to produce a draft version.
Likewise one can add a line to the very top of a child file
(above the |\childdocof{|\textit{main}|}| directive)
%
\begin{center}
|%\providecommand{\version}{final}|
\end{center}
%
which can be uncommented to produce the final version of this child document.

%%%%%%%%%%%%%%%%%%%%%%%%%%%%%%%%%%%%%%%%%%%%%%%%%%%%%%%%%%%%%%%%%%%%%%%%%%%%%%%%
\subsection{Forwarding}
\label{sec:forward}

Different versions of the main or child documents
using compilation flags as described in \secref{sec:flags}
can be (permanently) stored in different files
for convenient compilation, viewing and distribution.
To this end, the package defines a command
to pass on compilation to a different file:

%%%%%%%%%%%%%%%%%%%%%%%%%%%%%%%%%%%%%%%%
\DescribeMacro{\childdocforward}
The command |\childdocforward| redirects processing to
another source file:
%
\begin{center}
\begin{tabular}{l}
|\input{childdoc.def}|\\
|\childdocforward[|\textit{main}|]{|\textit{dest}|}|\\
\end{tabular}
\end{center}
%
The argument \textit{dest} is the destination file
(without extension).
It should be the main file or one of the child files.
Note that further \textsf{childdoc} directives
such as |\childdocof| and |\childdocforward|
in the indicated file will be processed in this form.
The optional argument \textit{main}
passes on directly to the main file \textit{main}
while pretending to compile the child \textit{dest}.
This form behaves as if \textit{dest}
issues |\childdocof{|\textit{main}|}| right away,
and no further \textsf{childdoc} directives will be processed.

%%%%%%%%%%%%%%%%%%%%%%%%%%%%%%%%%%%%%%%%
\DescribeMacro{\...prefix}
In the alternative form |\childdocforwardprefix|,
%
\begin{center}
\begin{tabular}{l}
|\input{childdoc.def}|\\
|\childdocforwardprefix[|\textit{main}|]{|\textit{prefix}|}{|\textit{dest}|}|
\end{tabular}
\end{center}
%
the destination file is determined by a pattern
depending on the current file:
To make this work, the current file must be called
`{\textit{prefix}\hspace{0.2em}\textit{suffix}}'
with \textit{prefix} matching precisely the argument.
Processing is then passed on to the file
`{\textit{dest}\hspace{0.2em}\textit{suffix}}'.
Surely, the same effect is achieved by
directly specifying the
argument `{\textit{dest}\hspace{0.2em}\textit{suffix}}'
in the first form.
However, that requires to set up a different file
for each child. With the alternative form of the command
all these files can have exactly the same content
which simplifies setting them up and maintaining them.

For example, the following file |draft.tex|
with a compilation flag |\version| as described in \secref{sec:flags}
compiles the main document as a draft:
%
\begin{center}
\begin{tabular}{l}
|\def\version{draft}|\\
|\input{childdoc.def}|\\
|\childdocforward{|\textit{main}|}|
\end{tabular}
\end{center}
%
Likewise, the following files |final|\textit{nn}|.tex|
compile the final version of the child document
|child|\textit{nn}|.tex|:
%
\begin{center}
\begin{tabular}{l}
|\def\version{final}|\\
|\input{childdoc.def}|\\
|\childdocforwardprefix{final}{child}|
\end{tabular}
\end{center}
%

Note that when several versions of a main file and/or of each child file
are to be generated, it may be convenient to set up a |Makefile| or
shell script to automatise the process.

%%%%%%%%%%%%%%%%%%%%%%%%%%%%%%%%%%%%%%%%%%%%%%%%%%%%%%%%%%%%%%%%%%%%%%%%%%%%%%%%
\subsection{Command Line Processing}
\label{sec:commandline}

The effect of redirection files can also be achieved by invoking
the \LaTeX{} compiler with a more elaborate command line.
Most conveniently this should be done as part
of a shell script or a |Makefile|.

When using \textsf{childdoc} in the main file, the following
command lines effectively perform a redirection
(note that depending on the shell being used,
backslashes may have to be doubled: `|\|' $\to$ `|\\|'):
%
\begin{center}
|... -jobname "|\textit{target}|" |\\|"|[\textit{flags}]%
|\input{childdoc.def}\childdocforward[|\textit{main}|]{|\textit{dest}|}"|
\end{center}
%
Here \textit{target} is the name of the output file,
\textit{main} is the name of the main file
and \textit{dest} is the name of the main or child file to be processed
(all filenames without extensions).
The optional argument \textit{main} can be omitted
if \textit{main} matches \textit{dest}.
Optionally, compilation \textit{flags} can be defined via |\def| commands.
This command line makes the \TeX{} engine believe
it is compiling the file \textit{target}
whose content is specified as the latter parameter.
The provided code then forwards the processing to
\textit{main} or \textit{dest} as described in \secref{sec:forward}.

%%%%%%%%%%%%%%%%%%%%%%%%%%%%%%%%%%%%%%%%%%%%%%%%%%%%%%%%%%%%%%%%%%%%%%%%%%%%%%%%
\subsection{Include by Input}
\label{sec:input}

Including child documents by |\include| has some restrictions by design.
Most notably, the content of a child document always occupies
its own set of pages; pages cannot be shared between child documents.
Usually, this behaviour makes perfect sense
because each child document contain an essential part of the document.
However, in some situations it may be desirable to compose
a document from a collection of parts
without having mandatory page breaks between then.
For this case, the package
provides a mechanism to include parts
by |\input| which can also be processed individually.
However, by construction this mechanism
requires manual handling of the content to be output.

%%%%%%%%%%%%%%%%%%%%%%%%%%%%%%%%%%%%%%%%
\DescribeMacro{\ifchilddocmanual}
The main file should be prepared as usual, see \secref{sec:include}.
However, the document body must make a distinction
between processing of an individual part and of the main document, e.g.:
%
\begin{center}
\begin{tabular}{l}
|\ifchilddocmanual|\\
|\input{\childdocname}|\\
|\||else|\\
\textit{document body with }|\input{|\textit{part}|}|\\
|\||fi|
\end{tabular}
\end{center}
%
The conditional |\ifchilddocmanual| is true whenever
a part to be included by |\input| is being compiled,
and the name of the part is stored in |\childdocname|.

%%%%%%%%%%%%%%%%%%%%%%%%%%%%%%%%%%%%%%%%
\DescribeMacro{\childdocby}
Each part to be included by |\input| should start with:
%
\begin{center}
\begin{tabular}{l}
|\input{childdoc.def}|\\
|\childdocby{|\textit{main}|}|\\
\end{tabular}
\end{center}
%
The directive |\childdocby| is similar to |\childdocof|
described in \secref{sec:include},
but the subsequent selection of content must be done manually.
To that end, both |\ifchilddoc| and |\ifchilddocmanual|
will be true upon processing of a part,
and the name of the part is stored in |\childdocname|.
Note that |\jobname| will be set to the filename of the current part
so that each part receives an individual |.aux| file
that does not interfere with the |.aux| file(s) of the main document.
This behaviour can be altered by the alternative form
|\childdocby[*]{|\textit{main}|}| (with a non-empty optional argument)
which uses the |.aux| file of the main document
by setting |\jobname| to \textit{main}.

%%%%%%%%%%%%%%%%%%%%%%%%%%%%%%%%%%%%%%%%%%%%%%%%%%%%%%%%%%%%%%%%%%%%%%%%%%%%%%%%
\subsection{Driver Development}
\label{sec:driver}

The \textsf{childdoc} mechanism can also be use for the development
of definition files such as \LaTeX{} styles or classes.
This case differs from the above setup with multiple parts
included by |\include| in that no |\includeonly| should be invoked.
This can be achieved by starting the include file
(before |\ProvidesPackage|) with:
%
\begin{center}
\begin{tabular}{l}
|\input{childdoc.def}|\\
|\childdocforward{|\textit{main}|}|\\
\end{tabular}
\end{center}
%
or alternatively with:
%
\begin{center}
\begin{tabular}{l}
|\input{childdoc.def}|\\
|\childdocby{|\textit{main}|}|\\
\end{tabular}
\end{center}
%
Both forms have slightly different effects as described above.
The main file is prepared as usual, see \secref{sec:include}.

%%%%%%%%%%%%%%%%%%%%%%%%%%%%%%%%%%%%%%%%%%%%%%%%%%%%%%%%%%%%%%%%%%%%%%%%%%%%%%%%
\subsection{Legacy Detection}
\label{sec:detection}

The directive |\childdocmain| in the main file can detect
whether the complete document or merely a child is to be compiled
even without using the directive |\childdocof|.
This method is deprecated because it is less robust
and there is no compelling reason to use it;
it is merely provided for backward compatibility
and it may be removed in future versions.

If the detection mechanism is to be used,
it is mandatory to correctly specify
the filename of the main file as the argument of |\childdocmain|:
%
\begin{center}
\begin{tabular}{l}
|\input{childdoc.def}|\\
|\childdocmain{|\textit{main}|}|\\
\end{tabular}
\end{center}
%
If |\jobname| does not match the argument \textit{main} of |\childdocmain|,
it is assumed that |\jobname| points to the child file to be compiled.
When using |\childdocmain| with the main file specified as argument,
it suffices to start a child file
with just |\input{|\textit{main}|}|
without loading of the package and using |\childdocof|.
If instead all processing is done
with the appropriate \textsf{childdoc} directives,
the argument of \textit{main} of |\childdocmain| can be empty.

An alternative version of the command line processing described
in \secref{sec:commandline} using the detection mechanism reads:
%
\begin{center}
|... -jobname "|\textit{target}|" "|[\textit{flags}]%
[|\def\jobname{|\textit{dest}|}|]|\input{|\textit{main}|}"|
\end{center}

%%%%%%%%%%%%%%%%%%%%%%%%%%%%%%%%%%%%%%%%%%%%%%%%%%%%%%%%%%%%%%%%%%%%%%%%%%%%%%%%
\subsection{Manual Code}
\label{sec:manual}

In case one cannot be certain whether the definitions file |childdoc.def|
is installed on the target \TeX{} distribution
and one prefers not to ship it,
it is conceivable to paste a few relevant commands into the sources.

To that end, drop all statements |\input{childdoc.def}|
and perform the replacements as outlined below.
Instead of |\childdocmain{|\textit{main}|}| add the following code
to the top of the main file:
%
\begin{center}
\begin{tabular}{l}
|\||ifdefined\childdocname\endinput\||fi\newif\ifchilddoc|\\
|\edef\childdocname{\scantokens\expandafter{\jobname\noexpand}}|\\
|\def\childdocmain{|\textit{main}|}\||ifx\childdocmain\childdocname\||else|\\
|\childdoctrue\includeonly{\childdocname}\let\jobname\childdocmain\||fi|\\
\end{tabular}
\end{center}
%
Instead of |\childdocof{|\textit{main}|}| just include the main file
at the top of each child file:
%
\begin{center}
|\input{|\textit{main}|}|
\end{center}
%
A simple redirection |\childdocforward{|\textit{dest}|}| is achieved by:
%
\begin{center}
|\def\jobname{|\textit{dest}|}\input{\jobname}|
\end{center}
%
The redirection with prefix
|\childdocforwardprefix[|\textit{prefix}|]{|\textit{dest}|}|
is accomplished by:
%
\begin{center}
\begin{tabular}{l}
|{\edef\jobname{\scantokens\expandafter{\jobname\noexpand}}|\\
|\def\redirectjob |\textit{prefix}|#1~~~{\gdef\jobname{|\textit{dest}|#1}}|\\
|\expandafter\redirectjob\jobname~~~}\input{\jobname}|
\end{tabular}
\end{center}

In an alternative approach,
child documents can be compiled by a specific command line
without additional code or specific definitions:
%
\begin{center}
|... -jobname "|\textit{target}|" "|[\textit{flags}]%
|\includeonly{|\textit{dest}|}\input{|\textit{main}|}"|
\end{center}
%

%%%%%%%%%%%%%%%%%%%%%%%%%%%%%%%%%%%%%%%%%%%%%%%%%%%%%%%%%%%%%%%%%%%%%%%%%%%%%%%%
%%%%%%%%%%%%%%%%%%%%%%%%%%%%%%%%%%%%%%%%%%%%%%%%%%%%%%%%%%%%%%%%%%%%%%%%%%%%%%%%
\section{Information}

%%%%%%%%%%%%%%%%%%%%%%%%%%%%%%%%%%%%%%%%%%%%%%%%%%%%%%%%%%%%%%%%%%%%%%%%%%%%%%%%
\subsection{Copyright}

Copyright \copyright{} 2017--2018 Niklas Beisert

This work may be distributed and/or modified under the
conditions of the \LaTeX{} Project Public License, either version 1.3
of this license or (at your option) any later version.
The latest version of this license is in
  \url{http://www.latex-project.org/lppl.txt}
and version 1.3 or later is part of all distributions of \LaTeX{}
version 2005/12/01 or later.

This work has the LPPL maintenance status `maintained'.

The Current Maintainer of this work is Niklas Beisert.

This work consists of the files |README.txt|, |childdoc.ins| and |childdoc.dtx|
as well as the derived files |childdoc.def|, |cdocsamp.tex|
with |cdocsch1.tex|, |cdocsch2.tex|, |cdocspt3.tex|, |cdocspt4.tex|,
|cdocsdrf.tex|, |cdocsfn1.tex|, |cdocsfn2.tex|
as well as |childdoc.pdf|.

%%%%%%%%%%%%%%%%%%%%%%%%%%%%%%%%%%%%%%%%%%%%%%%%%%%%%%%%%%%%%%%%%%%%%%%%%%%%%%%%
\subsection{Files and Installation}

The package consists of the files:
%
\begin{center}
\begin{tabular}{ll}
    |README.txt|   & readme file \\
    |childdoc.ins| & installation file \\
    |childdoc.dtx| & source file \\
    |childdoc.def| & definition file \\
    |cdocsamp.tex| & sample main file \\
    |cdocsch1.tex| & sample include file \\
    |cdocsch2.tex| & sample include file \\
    |cdocspt3.tex| & sample part file \\
    |cdocspt4.tex| & sample part file \\
    |cdocsdrf.tex| & sample redirection file \\
    |cdocsfn1.tex| & sample redirection file \\
    |cdocsfn2.tex| & sample redirection file \\
    |childdoc.pdf| & manual
\end{tabular}
\end{center}
%
The distribution consists of the files
|README.txt|, |childdoc.ins| and |childdoc.dtx|.
%
\begin{itemize}
\item
Run (pdf)\LaTeX{} on |childdoc.dtx|
to compile the manual |childdoc.pdf| (this file).
\item
Run \LaTeX{} on |childdoc.ins| to create the definitions file |childdoc.def|
and the sample |cdocsamp.tex| with include files
|cdocsch1.tex|, |cdocsch2.tex|, |cdocspt3.tex|, |cdocspt4.tex|,
|cdocsdrf.tex|, |cdocsfn1.tex|, |cdocsfn2.tex|.
Then copy the file |childdoc.def| to an appropriate directory of your \LaTeX{}
distribution, e.g.\ \textit{texmf-root}|/tex/latex/childdoc|.
\end{itemize}

%%%%%%%%%%%%%%%%%%%%%%%%%%%%%%%%%%%%%%%%%%%%%%%%%%%%%%%%%%%%%%%%%%%%%%%%%%%%%%%%
\subsection{Related CTAN Packages}

There are several other packages which offer a similar functionality:
%
\begin{itemize}
\item
The packages
\href{http://ctan.org/pkg/docmute}{\textsf{docmute}},
\href{http://ctan.org/pkg/includex}{\textsf{includex}} and
\href{http://ctan.org/pkg/standalone}{\textsf{standalone}}
provide commands to include only the document body of
a child file thus allowing both files to be compiled individually.
\item
The packages \href{http://ctan.org/pkg/subdocs}{\textsf{subdocs}}
and \href{http://ctan.org/pkg/subfiles}{\textsf{subfiles}}
provide structures in which the main and child documents can be
encapsulated and allowing them to be compiled individually.
The inclusion mechanism is different from the conventional |\include|.
\item
The package \href{http://ctan.org/pkg/combine}{\textsf{combine}}
is an elaborate solution to combine several documents into one.
\end{itemize}
%
See also the CTAN topic \href{http://ctan.org/topic/subdocs}{\textsf{subdocs}}
for further related packages.
The present package differs from the above solutions in that
a document structure constructed with the conventional |\include| mechanism
just needs two extra commands at the top of every file
such that all constituent files can be compiled individually.

%%%%%%%%%%%%%%%%%%%%%%%%%%%%%%%%%%%%%%%%%%%%%%%%%%%%%%%%%%%%%%%%%%%%%%%%%%%%%%%%
%\subsection{Feature Suggestions}
%
%The following is a list of features which may be useful for future
%versions of this package:
%%
%\begin{itemize}
%\item
%\ldots
%\end{itemize}

%%%%%%%%%%%%%%%%%%%%%%%%%%%%%%%%%%%%%%%%%%%%%%%%%%%%%%%%%%%%%%%%%%%%%%%%%%%%%%%%
\subsection{Revision History}

%%%%%%%%%%%%%%%%%%%%%%%%%%%%%%%%%%%%%%%%
\paragraph{v2.0:} 2018/12/30

\begin{itemize}
\item
immediate forward processing
\item
added |\childdocby| mechanism
\item
manual restructured
\end{itemize}

%%%%%%%%%%%%%%%%%%%%%%%%%%%%%%%%%%%%%%%%
\paragraph{v1.6:} 2018/01/17

\begin{itemize}
\item
application for development of include files
\item
corrections to manual
\end{itemize}

%%%%%%%%%%%%%%%%%%%%%%%%%%%%%%%%%%%%%%%%
\paragraph{v1.5:} 2017/05/21

\begin{itemize}
\item
more complete structuring introduced
\item
|\childdocof| introduced
\item
|\childdoc| renamed to |\childdocmain|
\item
|\childredirect| renamed to |\childdocforward| and |\childdocforwardprefix|
and functionality expanded
\end{itemize}

%%%%%%%%%%%%%%%%%%%%%%%%%%%%%%%%%%%%%%%%
\paragraph{v1.0:} 2017/04/27

\begin{itemize}
\item
manual and install package
\item
first version published on CTAN
\end{itemize}

%%%%%%%%%%%%%%%%%%%%%%%%%%%%%%%%%%%%%%%%
\paragraph{v0.6:} 2017/04/26

\begin{itemize}
\item
redirection mechanism added
\end{itemize}

%%%%%%%%%%%%%%%%%%%%%%%%%%%%%%%%%%%%%%%%
\paragraph{v0.5:} 2017/04/26

\begin{itemize}
\item
functionality in definition file
\end{itemize}


%%%%%%%%%%%%%%%%%%%%%%%%%%%%%%%%%%%%%%%%%%%%%%%%%%%%%%%%%%%%%%%%%%%%%%%%%%%%%%%%
%%%%%%%%%%%%%%%%%%%%%%%%%%%%%%%%%%%%%%%%%%%%%%%%%%%%%%%%%%%%%%%%%%%%%%%%%%%%%%%%
%%%%%%%%%%%%%%%%%%%%%%%%%%%%%%%%%%%%%%%%%%%%%%%%%%%%%%%%%%%%%%%%%%%%%%%%%%%%%%%%
\appendix

\settowidth\MacroIndent{\rmfamily\scriptsize 000\ }

 \DocInput{childdoc.dtx}

\end{document}
%</driver>
% \fi
%
% %%%%%%%%%%%%%%%%%%%%%%%%%%%%%%%%%%%%%%%%%%%%%%%%%%%%%%%%%%%%%%%%%%%%%%%%%%%%%%
% %%%%%%%%%%%%%%%%%%%%%%%%%%%%%%%%%%%%%%%%%%%%%%%%%%%%%%%%%%%%%%%%%%%%%%%%%%%%%%
% \section{Sample}
%\iffalse
%<*samplemain>
%\fi
%
% The following presents a sample document
% with two chapters, two parts, a title page,
% a compile flag as well as three forwarding files to set the flag.
% It consists of eight |.tex| files:
% \begin{center}
% \begin{tabular}{ll}
% |cdocsamp.tex|&main file\\
% |cdocsch1.tex|&include file for chapter 1\\
% |cdocsch2.tex|&include file for chapter 2\\
% |cdocspt3.tex|&include file for part 3\\
% |cdocspt4.tex|&include file for part 4\\
% |cdocsdrf.tex|&forwarding file for main file in draft mode\\
% |cdocsfi1.tex|&forwarding file for final version of chapter 1\\
% |cdocsfi2.tex|&forwarding file for final version of chapter 2\\
% \end{tabular}
% \end{center}
% Each of the eight files can be compiled directly by the \LaTeX{} compiler.
%
% %%%%%%%%%%%%%%%%%%%%%%%%%%%%%%%%%%%%%%
% \paragraph{Main File.}
%
% The main file is called |cdocsamp.tex|.
%
% Load the \textsf{childdoc} definitions and
% declare the filename for the main document:
%    \begin{macrocode}
\input{childdoc.def}
\childdocmain{}
%    \end{macrocode}

% Optional override for |\version| flag:
%    \begin{macrocode}
%%\ifchilddoc\else\providecommand{\version}{draft}\fi
%    \end{macrocode}

% Define the default values for the |\version| flag
% (|final| for the main file and |draft| for childs):
%    \begin{macrocode}
\ifchilddoc
\providecommand{\version}{draft}
\else
\providecommand{\version}{final}
\fi
%    \end{macrocode}

% Load the standard document class:
%    \begin{macrocode}
\documentclass[12pt]{article}
%    \end{macrocode}

% Start the document body:
%    \begin{macrocode}
\begin{document}
%    \end{macrocode}

% Declare a title page.
% Print title, part of document being processed and version flag:
%    \begin{macrocode}
\addtocounter{page}{-1}
\begin{center}
{\LARGE\bfseries{}childdoc example\par}
\vspace{1cm}
\ifchilddoc
\ifchilddocmanual part\else chapter\fi:
`\childdocname' of `\childdocjob'\par
\else
main document: `\childdocjob'\par
\fi
version: \version\par
\end{center}
\newpage
%    \end{macrocode}

% Manually include selected file,
% otherwise process as usual:
%    \begin{macrocode}
\ifchilddocmanual
\section*{part `\childdocname'}
\input{\childdocname}
\else
%    \end{macrocode}

% Include the two chapters:
%    \begin{macrocode}
\include{cdocsch1}
\include{cdocsch2}
%    \end{macrocode}

% Include the two parts unless only chapters should be displayed:
%    \begin{macrocode}
\ifchilddoc\else
\section{part three}
\input{cdocspt3}
\section{part four}
\input{cdocspt4}
\fi
%    \end{macrocode}

% Process as usual until here:
%    \begin{macrocode}
\fi
%    \end{macrocode}

% End of document body:
%    \begin{macrocode}
\end{document}
%    \end{macrocode}
%\iffalse
%</samplemain>
%\fi
%
% %%%%%%%%%%%%%%%%%%%%%%%%%%%%%%%%%%%%%%
% \paragraph{Chapter Include Files.}
%
% The include files are called |cdocsch1.tex| and |cdocsch2.tex|.
%
%\iffalse
%<*samplechap1|samplechap2>
%\fi

% Optional override for |\version| flag:
%    \begin{macrocode}
%%\providecommand{\version}{final}
%    \end{macrocode}

% Include the main document:
%    \begin{macrocode}
\input{childdoc.def}
\childdocof{cdocsamp}
%    \end{macrocode}

%\iffalse
%</samplechap1|samplechap2>
%\fi
%
%\iffalse
%<*samplechap1>
%\fi
% Some text for chapter 1:
%    \begin{macrocode}
\section{one}
some text in chapter one
%    \end{macrocode}

%\iffalse
%</samplechap1>
%\fi
% Some text for chapter 2:
%\iffalse
%<*samplechap2>
%\fi
%    \begin{macrocode}
\section{two}
more text in chapter two
%    \end{macrocode}

%\iffalse
%</samplechap2>
%\fi
%
% %%%%%%%%%%%%%%%%%%%%%%%%%%%%%%%%%%%%%%
% \paragraph{Part Include Files.}
%
% The include files are called |cdocspt3.tex| and |cdocspt4.tex|.
%
%\iffalse
%<*samplepart3|samplepart4>
%\fi

% Optional override for |\version| flag:
%    \begin{macrocode}
%%\providecommand{\version}{final}
%    \end{macrocode}

% Include the main document:
%    \begin{macrocode}
\input{childdoc.def}
\childdocby{cdocsamp}
%    \end{macrocode}

%\iffalse
%</samplepart3|samplepart4>
%\fi
%
%\iffalse
%<*samplepart3>
%\fi
% Some text for part 3:
%    \begin{macrocode}
some text in part three
%    \end{macrocode}

%\iffalse
%</samplepart3>
%\fi
% Some text for part 4:
%\iffalse
%<*samplepart4>
%\fi
%    \begin{macrocode}
more text in part four
%    \end{macrocode}

%\iffalse
%</samplepart4>
%\fi
%
% %%%%%%%%%%%%%%%%%%%%%%%%%%%%%%%%%%%%%%
% \paragraph{Forwarding for a Complete Draft.}
%
% The following forwarding file |cdocsdrf.tex|
% compiles the main document in draft mode:
%\iffalse
%<*sampledraft>
%\fi
%    \begin{macrocode}
\def\version{draft}
\input{childdoc.def}
\childdocforward{cdocsamp}
%    \end{macrocode}

%\iffalse
%</sampledraft>
%\fi
%
% %%%%%%%%%%%%%%%%%%%%%%%%%%%%%%%%%%%%%%
% \paragraph{Forwarding for Final Version of the Chapters.}
%
% The following forwarding files |cdocsfn1.tex| and |cdocsfn2.tex|
% (with identical content)
% compile the final versions of the child documents
% |cdocsch1.tex| and |cdocsch2.tex|, respectively:
%\iffalse
%<*samplefinal>
%\fi
%    \begin{macrocode}
\def\version{final}
\input{childdoc.def}
\childdocforwardprefix[cdocsamp]{cdocsfn}{cdocsch}
%    \end{macrocode}

%\iffalse
%</samplefinal>
%\fi
%
% %%%%%%%%%%%%%%%%%%%%%%%%%%%%%%%%%%%%%%
% \paragraph{Command Line Processing.}
%
% The following three command lines generate the output files
% |cdocscld|, |cdocscl1| and |cdocscl2|
% which should be identical to
% |cdocsdrf|, |cdocsch1| and |cdocsfn2|, respectively:
% \begin{center}
% \begin{tabular}{l}
% |latex -jobname cdocscld \|\\
% |  "\def\version{draft}\input{childdoc.def}\childdocforward{cdocsamp}"|\\
% |latex -jobname cdocscl1 \|\\
% |  "\input{childdoc.def}\childdocforward[cdocsamp]{cdocsch1}"|\\
% |latex -jobname cdocscl2 \|\\
% |  "\def\version{final}\input{childdoc.def}\childdocforward{cdocsch2}"|
% \end{tabular}
% \end{center}
% Note that the trailing backslash on each first line
% merely continues the input to the second line
% (for convenient cut ant paste).
% Furthermore, the command |latex| can be replaced by any
% of its alternative versions such as |pdflatex|.
%
% %%%%%%%%%%%%%%%%%%%%%%%%%%%%%%%%%%%%%%%%%%%%%%%%%%%%%%%%%%%%%%%%%%%%%%%%%%%%%%
% %%%%%%%%%%%%%%%%%%%%%%%%%%%%%%%%%%%%%%%%%%%%%%%%%%%%%%%%%%%%%%%%%%%%%%%%%%%%%%
% \section{Implementation}
%\iffalse
%<*package>
%\fi
%
% This section describes the definitions file |childdoc.def|.

% The definitions cannot be loaded using |\usepackage| or |\RequirePackage|
% which has a mechanism to prevent loading a style file more than once.
% When loading the definitions by means of |\input|
% multiple instances have to be prevented manually:
%\iffalse
%This code needs to be before the `\ProvidesFile' directive
%which is defined at the beginning of this file.
%Therefore it is also placed there and commented out here.
%</package>
%<*discard>
%\fi
%    \begin{macrocode}
\ifdefined\childdocmain\endinput\fi
%    \end{macrocode}
%\iffalse
%</discard>
%<*package>
%\fi
%
% \macro{\ifchilddoc}
% \macro{\ifchilddocmanual}
% The conditional |\ifchilddoc| tells whether a
% child (true) or main (false) document is being compiled.
% The conditional |\ifchilddocmanual| tells whether
% the |\includeonly| mechanism is used (false) or
% the selection of child files must be performed manually (true).
% The definitions initialise to false:
%    \begin{macrocode}
\newif\ifchilddoc
\newif\ifchilddocmanual
%    \end{macrocode}

% \macro{\childdocname}
% \macro{\childdocjob}
% The macro |\childdocname| stores the name of the main document
% to be compiled. The macro |\childdocjob| stores the name of
% the document on which the \LaTeX{} compiler was originally invoked.
% The content of |\jobname| cannot be compared
% to filenames specified in the source due to different catcodes.
% The following code rescans |\jobname|, stores the result
% in |\childdocname| and saves a copy in |\childdocjob|:
%    \begin{macrocode}
\edef\childdocname{\scantokens\expandafter{\jobname\noexpand}}
\let\childdocjob\childdocname
%    \end{macrocode}

% \macro{\childdocdisable}
% The macro |\childdocdisable| prevents the main file
% from being processed more than once.
% At this stage, the main document command |\childdocmain|
% is assumed to be called once again where it should do nothing.
% Any subsequent call to it should prevent
% a secondary processing of the main document
% It overwrites the forwarding commands
% |\childdocof| and |\childdocforward|
% with empty macros to prevent further inclusions of the main document:
%    \begin{macrocode}
\newcommand{\childdocdisable}
{
  \renewcommand{\childdocmain}[1]{\renewcommand{\childdocmain}[1]{\endinput}}
  \renewcommand{\childdocof}[1]{}
  \renewcommand{\childdocby}[2][]{}
  \renewcommand{\childdocforward}[2][]{}
  \renewcommand{\childdocdisable}{}
}
%    \end{macrocode}

% \macro{\childdocmain}
% The macro |\childdocmain| is to be called at the top of the main file
% with nothing or the main filename (without extension) as argument.
% First, it breaks loops.
% If the argument is not empty and does not match |\childdocname|
% (which is set by the first inclusion of |childdoc.def|),
% |\ifchilddoc| is set to true, |\includeonly| is applied to the child file
% and |\jobname| is set to the main file
% (for proper handling of |.aux| files):
%    \begin{macrocode}
\newcommand{\childdocmain}[1]
{
  \childdocdisable\childdocmain{}
  \if?#1?\else
    \begingroup
      \def\childdoctmp{#1}
      \ifx\childdoctmp\childdocname
        \def\childdoctmp{}
      \else
        \def\childdoctmp
        {
          \childdoctrue
          \includeonly{\childdocname}
          \def\childdocjob{#1}
          \def\jobname{#1}
        }
      \fi
      \expandafter
    \endgroup
    \childdoctmp
  \fi
}
%    \end{macrocode}

% \macro{\childdocof}
% The command |\childdocof| redirects
% compilation to the main file |#1|.
%    \begin{macrocode}
\newcommand{\childdocof}[1]
{
  \childdocdisable
  \childdoctrue
  \includeonly{\childdocname}
  \def\jobname{#1}
  \def\childdocjob{#1}
  \input{#1}
}
%    \end{macrocode}

% \macro{\childdocby}
% The command |\childdocby| ....
%    \begin{macrocode}
\newcommand{\childdocby}[2][]
{
  \childdocdisable
  \childdoctrue
  \childdocmanualtrue
  \if?#1?\else
    \def\jobname{#2}
  \fi
  \def\childdocjob{#2}
  \input{#2}
  \endinput
}
%    \end{macrocode}

% \macro{\childdocforward}
% The command |\childdocforward| redirects
% compilation to the main file or
% (if the optional argument is given) a child file.
% Parameters are set as if the main file
% or a child file starting with |\childdocof| was compiled.
% Then compilation is handed over to the main file:
%    \begin{macrocode}
\newcommand{\childdocforward}[2][]
{
  \begingroup
    \if?#1?
      \def\childdoctmp
      {
        \def\childdocname{#2}
        \def\childdocjob{#2}
        \def\jobname{#2}
        \input{#2}
        \endinput
      }
    \else
      \def\childdoctmp
      {
        \childdocdisable
        \def\childdocname{#2}
        \childdoctrue
        \includeonly{#2}
        \def\childdocjob{#1}
        \def\jobname{#1}
        \input{#1}
        \endinput
      }
    \fi
    \expandafter
  \endgroup
  \childdoctmp
}
%    \end{macrocode}

% \macro{\childdocforwardprefix}
% The command |\childdocforwardprefix| redirects
% compilation to the main or a child file by means of a pattern.
% The prefix |#1| in the current filename is replaced by |#2|
% and the suffix of the current filename is kept
% (it is assumed that the filename does not contain the substring `|~~~|'
% which is used as a delimiter).
% Compilation is handed over to the new file by |\childdocforward|:
%    \begin{macrocode}
\newcommand{\childdocforwardprefix}[3][]
{
  \begingroup
    \def\childdocextract #2##1~~~{\def\childdoctmp{\childdocforward[#1]{#3##1}}}
    \expandafter\childdocextract\childdocname~~~
    \expandafter
  \endgroup
  \childdoctmp
}
%    \end{macrocode}

% \macro{\childdoc}
% The deprecated macro |\childdoc| is a legacy version of |\childdocmain|:
%    \begin{macrocode}
\newcommand{\childdoc}{\childdocmain}
%    \end{macrocode}

% \macro{\childdocredirect}
% The deprecated macro |\childdocredirect| is a legacy version
% of |\childdocforward| and |\childdocforwardprefix|:
%    \begin{macrocode}
\newcommand{\childdocredirect}[2][]
{
  \begingroup
    \if?#1?
      \def\childdoctmp{\childdocforward{#2}}
    \else
      \def\childdoctmp{\childdocforwardprefix{#1}{#2}}
    \fi
    \expandafter
  \endgroup
  \childdoctmp
}
%    \end{macrocode}

%\iffalse
%</package>
%\fi
%
\endinput
|\\
|\childdocforward[|\textit{main}|]{|\textit{dest}|}|\\
\end{tabular}
\end{center}
%
The argument \textit{dest} is the destination file
(without extension).
It should be the main file or one of the child files.
Note that further \textsf{childdoc} directives
such as |\childdocof| and |\childdocforward|
in the indicated file will be processed in this form.
The optional argument \textit{main}
passes on directly to the main file \textit{main}
while pretending to compile the child \textit{dest}.
This form behaves as if \textit{dest}
issues |\childdocof{|\textit{main}|}| right away,
and no further \textsf{childdoc} directives will be processed.

%%%%%%%%%%%%%%%%%%%%%%%%%%%%%%%%%%%%%%%%
\DescribeMacro{\...prefix}
In the alternative form |\childdocforwardprefix|,
%
\begin{center}
\begin{tabular}{l}
|% \iffalse
%
% childdoc.dtx Copyright (C) 2017-2018 Niklas Beisert
%
% This work may be distributed and/or modified under the
% conditions of the LaTeX Project Public License, either version 1.3
% of this license or (at your option) any later version.
% The latest version of this license is in
%   http://www.latex-project.org/lppl.txt
% and version 1.3 or later is part of all distributions of LaTeX
% version 2005/12/01 or later.
%
% This work has the LPPL maintenance status `maintained'.
%
% The Current Maintainer of this work is Niklas Beisert.
%
% This work consists of the files childdoc.dtx and childdoc.ins
% and the derived files childdoc.def and cdocsamp.tex with
% cdocsch1.tex, cdocsch2.tex, cdocsdrf.tex, cdocsfn1.tex, cdocsfn2.tex.
%
%<package>\ifdefined\childdocmain\endinput\fi
%<package>\ProvidesFile{childdoc.def}[2018/12/30 v2.0 child document driver]
%<samplemain>\ProvidesFile{cdocsamp.tex}[2018/12/30 v2.0 sample for childdoc]
%<*driver>
%\ProvidesFile{childdoc.drv}[2018/12/30 v2.0 childdoc reference manual file]
\PassOptionsToClass{10pt,a4paper}{article}
\documentclass{ltxdoc}

\usepackage[margin=35mm]{geometry}
\usepackage{hyperref}
\usepackage{hyperxmp}
\usepackage[usenames]{color}

\hypersetup{colorlinks=true}
\hypersetup{pdfstartview=FitH}
\hypersetup{pdfpagemode=UseNone}
\hypersetup{pdfsource={}}
\hypersetup{pdflang={en-UK}}
\hypersetup{pdfcopyright={Copyright 2017-2018 Niklas Beisert.
  This work may be distributed and/or modified under the
  conditions of the LaTeX Project Public License, either version 1.3
  of this license or (at your option) any later version.}}
\hypersetup{pdflicenseurl={http://www.latex-project.org/lppl.txt}}
\hypersetup{pdfcontactaddress={ETH Zurich, ITP, HIT K,
  Wolfgang-Pauli-Strasse 27}}
\hypersetup{pdfcontactpostcode={8093}}
\hypersetup{pdfcontactcity={Zurich}}
\hypersetup{pdfcontactcountry={Switzerland}}
\hypersetup{pdfcontactemail={nbeisert@itp.phys.ethz.ch}}
\hypersetup{pdfcontacturl={http://people.phys.ethz.ch/\xmptilde nbeisert/}}

\newcommand{\secref}[1]{\hyperref[#1]{section \ref*{#1}}}

\parskip1ex
\parindent0pt
\let\olditemize\itemize
\def\itemize{\olditemize\parskip0pt}

\begin{document}

\title{The \textsf{childdoc} Package}
\hypersetup{pdftitle={The childdoc Package}}
\author{Niklas Beisert\\[2ex]
  Institut f\"ur Theoretische Physik\\
  Eidgen\"ossische Technische Hochschule Z\"urich\\
  Wolfgang-Pauli-Strasse 27, 8093 Z\"urich, Switzerland\\[1ex]
  \href{mailto:nbeisert@itp.phys.ethz.ch}
  {\texttt{nbeisert@itp.phys.ethz.ch}}}
\hypersetup{pdfauthor={Niklas Beisert}}
\hypersetup{pdfsubject={Manual for the LaTeX2e Package childdoc}}
\date{30 December 2018, \textsf{v2.0}}
\maketitle

\begin{abstract}\noindent
\textsf{childdoc} is a \LaTeXe{} package
that enables the direct compilation
of document sections included by |\include|
to individual files.
\end{abstract}

\begingroup
\parskip0ex
\tableofcontents
\endgroup

%%%%%%%%%%%%%%%%%%%%%%%%%%%%%%%%%%%%%%%%%%%%%%%%%%%%%%%%%%%%%%%%%%%%%%%%%%%%%%%%
%%%%%%%%%%%%%%%%%%%%%%%%%%%%%%%%%%%%%%%%%%%%%%%%%%%%%%%%%%%%%%%%%%%%%%%%%%%%%%%%
\section{Introduction}

\LaTeX{} provides a mechanism to structure a large document (such as a book)
into a main file and several child files (containing the chapters)
using the |\include| command.
This mechanism is beneficial for documents
which span hundreds of pages in order to
make the source file(s) more manageable.
Moreover, compilation can be restricted to
selected child files by means of the |\includeonly| command.
The latter feature can be used to reduce the compilation time while editing
(this was significantly more useful in the earlier days of \LaTeX{})
or to generate a smaller document which is easier to navigate.
Another application of |\includeonly| is to generate
documents consisting of selected parts of the complete document.

However, there are a few drawbacks of the plain |\include| mechanism:
\begin{itemize}
\item
The child files cannot be compiled on their own,
they can only be compiled via the main file.
A naive editing environment
(such as a text editor with an option
to have the current file processed by \LaTeX)
may require one to switch to the main file before compiling;
attempting to compile the child file produces errors.
\item
The main file must be modified (each time)
to adjust the |\includeonly| command
to the present needs. This easily leaves the main file in a messy state.
\item
The generated document will always carry the filename
of the main document. This is inconvenient if
several child files are to be compiled and
to be kept for distribution.
\end{itemize}

The present package provides a simple interface
to make child files individually compilable by \LaTeX{}.
Compiling a child file then has the same effect as compiling
the main file with an |\includeonly| command
to select the appropriate child.
Moreover the generated document will carry the name of the child
rather than the main file.
This resolves all three above issues.

This feature is meant to make the editing of books,
thesis documents and lecture notes somewhat more convenient.
However, the package can also be used efficiently for
composing a series of documents (such as exercise sheets)
which are typically distributed individually.
It then assists the author in generating the individual documents
(potentially in different versions)
as well as a document containing the collected series.
Another application is in developing style files
or other kinds of included material
where compilation of the style file could redirect
to a sample or test file.

%%%%%%%%%%%%%%%%%%%%%%%%%%%%%%%%%%%%%%%%%%%%%%%%%%%%%%%%%%%%%%%%%%%%%%%%%%%%%%%%
%%%%%%%%%%%%%%%%%%%%%%%%%%%%%%%%%%%%%%%%%%%%%%%%%%%%%%%%%%%%%%%%%%%%%%%%%%%%%%%%
\section{Usage}

First of all, the package \textsf{childdoc} is \emph{not} a standard
\LaTeXe{} |.sty| style file! Therefore it needs to be invoked in
a non-standard way.

%%%%%%%%%%%%%%%%%%%%%%%%%%%%%%%%%%%%%%%%%%%%%%%%%%%%%%%%%%%%%%%%%%%%%%%%%%%%%%%%
\subsection{Included Files}
\label{sec:include}

%%%%%%%%%%%%%%%%%%%%%%%%%%%%%%%%%%%%%%%%
\DescribeMacro{\childdocmain}
To use the package, add the commands
\begin{center}
\begin{tabular}{l}
|\input{childdoc.def}|\\
|\childdocmain{}|\\
\end{tabular}
\end{center}
at the very top of the main \LaTeX{} file,
in particular \emph{before} the |\documentclass| statement!
The argument of |\childdocmain| should be left empty
(but it must be present).

%%%%%%%%%%%%%%%%%%%%%%%%%%%%%%%%%%%%%%%%
\DescribeMacro{\childdocof}
Furthermore, add the commands
\begin{center}
\begin{tabular}{l}
|\input{childdoc.def}|\\
|\childdocof{|\textit{main}|}|\\
\end{tabular}
\end{center}
at the top of every child file \textit{child}
which is included by |\include{|\textit{child}|}|
from within the main file
(or at least for those files to be compiled individually).
The argument \textit{main} must be the filename of the main file.

There are a couple of
considerations in setting up the main and child documents:

%%%%%%%%%%%%%%%%%%%%%%%%%%%%%%%%%%%%%%%%
\paragraph{Restrictions.}

Please note the following restrictions:
\begin{itemize}
\item
|\childdocmain| must be called with one argument \textit{main}
to ensure compatibility with earlier version of the package.
It must either be empty (|\childdocmain{}|)
or precisely match the filename of the main file in which it is specified.
See \secref{sec:detection} for further information.
\item
The filename \textit{main} must be specified without the |.tex| extension.
\item
The filename \textit{main} is case sensitive
(even in case-insensitive file systems)
due to internal string comparison.
\item
The argument \textit{main} should be fully expanded, it cannot be a macro.
\item
Subdirectories and special characters should be avoided in filenames.
\item
The command |\childdocmain{|\textit{main}|}| must be followed by a whitespace.
It should not be followed immediately by another command
or by a comment mark `|%|'.
This is because the \TeX{} parser reads the token immediately following
the argument of |\childdocmain| and puts it
at the beginning of every child section;
however, a white\-space is ignored.
\end{itemize}

%%%%%%%%%%%%%%%%%%%%%%%%%%%%%%%%%%%%%%%%
\paragraph{Content of Main File.}

It is advisable to place all content in the child files included by |\include|.
Any output contained in the main file will appear in all child documents
unless suppressed manually;
it cannot be suppressed automatically by the |\includeonly| directive
and thus should normally be avoided.
A method to include some content in the main file
by means of conditional processing is described in \secref{sec:conditional}.

%%%%%%%%%%%%%%%%%%%%%%%%%%%%%%%%%%%%%%%%
\paragraph{Page Numbering.}

When only a part of the document is compiled,
the appropriate numbering of pages
(as well as other status parameters)
is determined from the |.aux| files.
The latter contain information from previous passes.
However this information needs to propagate through
all intermediate child documents.
Therefore the page numbering in child documents may well
be inconsistent until the complete document is compiled at least once.

A useful (if unconventional) way to always ensure a consistent
page numbering is to restart the numbering in each child document
and denote the pages by `\textit{child}|.|\textit{page}'
where \textit{child} represents the chapter/section number of the child file.
This can be achieved by the command
|\numberwithin{page}{|\textit{child}|}|
of the \textsf{amsmath} package
where \textit{child} can be |chapter| or |section|
depending on the chosen structuring.
Alternatively, one can modify the macro |\thepage| appropriately
and reset the counter |page| at the start of each child file.

%%%%%%%%%%%%%%%%%%%%%%%%%%%%%%%%%%%%%%%%%%%%%%%%%%%%%%%%%%%%%%%%%%%%%%%%%%%%%%%%
\subsection{Conditional Processing}
\label{sec:conditional}

The package provides a mechanism to compile different versions
of a document. To customise the versions further some conditional processing
can come in handy to distinguish which version is being compiled.
The package provides two macros to describe the compilation context:

%%%%%%%%%%%%%%%%%%%%%%%%%%%%%%%%%%%%%%%%
\DescribeMacro{\ifchilddoc}
The conditional |\ifchilddoc| distinguishes between the compilation of
child documents and the main document:
%
\begin{center}
|\ifchilddoc |\textit{child-code}| |[|\||else |\textit{main-code}]| \||fi|
\end{center}

%%%%%%%%%%%%%%%%%%%%%%%%%%%%%%%%%%%%%%%%
\DescribeMacro{\childdocname}
\DescribeMacro{\childdocjob}
The macro |\childdocname| contains the filename (without extension)
of the main or child file being processed.
Note that |\childdocjob| will always contain the name of the main file.

%%%%%%%%%%%%%%%%%%%%%%%%%%%%%%%%%%%%%%%%
\paragraph{Title Page.}

Conditional processing can be used to include a title or banner page
in the main document when proper precautions are taken.
Importantly, the code in the main file should ensure that the page counter
(as well as other status parameters which are stored in the |.aux| files)
takes the same value after the conditional processing.
Otherwise the page numbers may take divergent values
depending on which part is compiled.

For example, a title page could be declared by:
%
\begin{center}
\begin{tabular}{l}
|\ifchilddoc\||else|\\
|\addtocounter{page}{-1}|\\
\textit{code for title page}\\
|\newpage|\\
|\||fi|
\end{tabular}
\end{center}
%
A banner page for the child documents can be generated by:
%
\begin{center}
\begin{tabular}{l}
|\ifchilddoc|\\
|\addtocounter{page}{-1}|\\
\textit{code for banner page}\\
|\newpage|\\
|\||fi|
\end{tabular}
\end{center}
%
Here one could write a message such as:
\begin{center}
|This is the part \childdocname{} of \childdocjob{}.|
\end{center}

%%%%%%%%%%%%%%%%%%%%%%%%%%%%%%%%%%%%%%%%%%%%%%%%%%%%%%%%%%%%%%%%%%%%%%%%%%%%%%%%
\subsection{Flags}
\label{sec:flags}

The package makes it easy to generate different versions
of the main or child documents.
To this end compilation flags can be defined
and assigned different default values.
They will be particularly useful in conjunction
with the forwarding mechanism described in \secref{sec:forward}.

For example, it may be useful to have a flag |\version|
which can be set to |draft| or |final|.
The document source will contain some conditional code
depending on the value of |\version|.
Suppose further, the flag should default to |final| for the main file
and to |draft| for child files
which is a natural assignment for editing the document.
This is achieved by placing the following code
in the preamble of the main document
(below the |\childdocmain| directive):
%
\begin{center}
\begin{tabular}{l}
|\ifchilddoc|\\
|\providecommand{\version}{draft}|\\
|\||else|\\
|\providecommand{\version}{final}|\\
|\||fi|
\end{tabular}
\end{center}
%
The definition by |\providecommand| makes sure
that previous definitions are not overwritten.
Further statements |\providecommand{\version}{...}|
can thus be added before the above code to override it.

For the main file, one might add a line
(between |\childdocmain| and the above block)
%
\begin{center}
|%\ifchilddoc\||else\providecommand{\version}{draft}\||fi|
\end{center}
%
which can be uncommented to produce a draft version.
Likewise one can add a line to the very top of a child file
(above the |\childdocof{|\textit{main}|}| directive)
%
\begin{center}
|%\providecommand{\version}{final}|
\end{center}
%
which can be uncommented to produce the final version of this child document.

%%%%%%%%%%%%%%%%%%%%%%%%%%%%%%%%%%%%%%%%%%%%%%%%%%%%%%%%%%%%%%%%%%%%%%%%%%%%%%%%
\subsection{Forwarding}
\label{sec:forward}

Different versions of the main or child documents
using compilation flags as described in \secref{sec:flags}
can be (permanently) stored in different files
for convenient compilation, viewing and distribution.
To this end, the package defines a command
to pass on compilation to a different file:

%%%%%%%%%%%%%%%%%%%%%%%%%%%%%%%%%%%%%%%%
\DescribeMacro{\childdocforward}
The command |\childdocforward| redirects processing to
another source file:
%
\begin{center}
\begin{tabular}{l}
|\input{childdoc.def}|\\
|\childdocforward[|\textit{main}|]{|\textit{dest}|}|\\
\end{tabular}
\end{center}
%
The argument \textit{dest} is the destination file
(without extension).
It should be the main file or one of the child files.
Note that further \textsf{childdoc} directives
such as |\childdocof| and |\childdocforward|
in the indicated file will be processed in this form.
The optional argument \textit{main}
passes on directly to the main file \textit{main}
while pretending to compile the child \textit{dest}.
This form behaves as if \textit{dest}
issues |\childdocof{|\textit{main}|}| right away,
and no further \textsf{childdoc} directives will be processed.

%%%%%%%%%%%%%%%%%%%%%%%%%%%%%%%%%%%%%%%%
\DescribeMacro{\...prefix}
In the alternative form |\childdocforwardprefix|,
%
\begin{center}
\begin{tabular}{l}
|\input{childdoc.def}|\\
|\childdocforwardprefix[|\textit{main}|]{|\textit{prefix}|}{|\textit{dest}|}|
\end{tabular}
\end{center}
%
the destination file is determined by a pattern
depending on the current file:
To make this work, the current file must be called
`{\textit{prefix}\hspace{0.2em}\textit{suffix}}'
with \textit{prefix} matching precisely the argument.
Processing is then passed on to the file
`{\textit{dest}\hspace{0.2em}\textit{suffix}}'.
Surely, the same effect is achieved by
directly specifying the
argument `{\textit{dest}\hspace{0.2em}\textit{suffix}}'
in the first form.
However, that requires to set up a different file
for each child. With the alternative form of the command
all these files can have exactly the same content
which simplifies setting them up and maintaining them.

For example, the following file |draft.tex|
with a compilation flag |\version| as described in \secref{sec:flags}
compiles the main document as a draft:
%
\begin{center}
\begin{tabular}{l}
|\def\version{draft}|\\
|\input{childdoc.def}|\\
|\childdocforward{|\textit{main}|}|
\end{tabular}
\end{center}
%
Likewise, the following files |final|\textit{nn}|.tex|
compile the final version of the child document
|child|\textit{nn}|.tex|:
%
\begin{center}
\begin{tabular}{l}
|\def\version{final}|\\
|\input{childdoc.def}|\\
|\childdocforwardprefix{final}{child}|
\end{tabular}
\end{center}
%

Note that when several versions of a main file and/or of each child file
are to be generated, it may be convenient to set up a |Makefile| or
shell script to automatise the process.

%%%%%%%%%%%%%%%%%%%%%%%%%%%%%%%%%%%%%%%%%%%%%%%%%%%%%%%%%%%%%%%%%%%%%%%%%%%%%%%%
\subsection{Command Line Processing}
\label{sec:commandline}

The effect of redirection files can also be achieved by invoking
the \LaTeX{} compiler with a more elaborate command line.
Most conveniently this should be done as part
of a shell script or a |Makefile|.

When using \textsf{childdoc} in the main file, the following
command lines effectively perform a redirection
(note that depending on the shell being used,
backslashes may have to be doubled: `|\|' $\to$ `|\\|'):
%
\begin{center}
|... -jobname "|\textit{target}|" |\\|"|[\textit{flags}]%
|\input{childdoc.def}\childdocforward[|\textit{main}|]{|\textit{dest}|}"|
\end{center}
%
Here \textit{target} is the name of the output file,
\textit{main} is the name of the main file
and \textit{dest} is the name of the main or child file to be processed
(all filenames without extensions).
The optional argument \textit{main} can be omitted
if \textit{main} matches \textit{dest}.
Optionally, compilation \textit{flags} can be defined via |\def| commands.
This command line makes the \TeX{} engine believe
it is compiling the file \textit{target}
whose content is specified as the latter parameter.
The provided code then forwards the processing to
\textit{main} or \textit{dest} as described in \secref{sec:forward}.

%%%%%%%%%%%%%%%%%%%%%%%%%%%%%%%%%%%%%%%%%%%%%%%%%%%%%%%%%%%%%%%%%%%%%%%%%%%%%%%%
\subsection{Include by Input}
\label{sec:input}

Including child documents by |\include| has some restrictions by design.
Most notably, the content of a child document always occupies
its own set of pages; pages cannot be shared between child documents.
Usually, this behaviour makes perfect sense
because each child document contain an essential part of the document.
However, in some situations it may be desirable to compose
a document from a collection of parts
without having mandatory page breaks between then.
For this case, the package
provides a mechanism to include parts
by |\input| which can also be processed individually.
However, by construction this mechanism
requires manual handling of the content to be output.

%%%%%%%%%%%%%%%%%%%%%%%%%%%%%%%%%%%%%%%%
\DescribeMacro{\ifchilddocmanual}
The main file should be prepared as usual, see \secref{sec:include}.
However, the document body must make a distinction
between processing of an individual part and of the main document, e.g.:
%
\begin{center}
\begin{tabular}{l}
|\ifchilddocmanual|\\
|\input{\childdocname}|\\
|\||else|\\
\textit{document body with }|\input{|\textit{part}|}|\\
|\||fi|
\end{tabular}
\end{center}
%
The conditional |\ifchilddocmanual| is true whenever
a part to be included by |\input| is being compiled,
and the name of the part is stored in |\childdocname|.

%%%%%%%%%%%%%%%%%%%%%%%%%%%%%%%%%%%%%%%%
\DescribeMacro{\childdocby}
Each part to be included by |\input| should start with:
%
\begin{center}
\begin{tabular}{l}
|\input{childdoc.def}|\\
|\childdocby{|\textit{main}|}|\\
\end{tabular}
\end{center}
%
The directive |\childdocby| is similar to |\childdocof|
described in \secref{sec:include},
but the subsequent selection of content must be done manually.
To that end, both |\ifchilddoc| and |\ifchilddocmanual|
will be true upon processing of a part,
and the name of the part is stored in |\childdocname|.
Note that |\jobname| will be set to the filename of the current part
so that each part receives an individual |.aux| file
that does not interfere with the |.aux| file(s) of the main document.
This behaviour can be altered by the alternative form
|\childdocby[*]{|\textit{main}|}| (with a non-empty optional argument)
which uses the |.aux| file of the main document
by setting |\jobname| to \textit{main}.

%%%%%%%%%%%%%%%%%%%%%%%%%%%%%%%%%%%%%%%%%%%%%%%%%%%%%%%%%%%%%%%%%%%%%%%%%%%%%%%%
\subsection{Driver Development}
\label{sec:driver}

The \textsf{childdoc} mechanism can also be use for the development
of definition files such as \LaTeX{} styles or classes.
This case differs from the above setup with multiple parts
included by |\include| in that no |\includeonly| should be invoked.
This can be achieved by starting the include file
(before |\ProvidesPackage|) with:
%
\begin{center}
\begin{tabular}{l}
|\input{childdoc.def}|\\
|\childdocforward{|\textit{main}|}|\\
\end{tabular}
\end{center}
%
or alternatively with:
%
\begin{center}
\begin{tabular}{l}
|\input{childdoc.def}|\\
|\childdocby{|\textit{main}|}|\\
\end{tabular}
\end{center}
%
Both forms have slightly different effects as described above.
The main file is prepared as usual, see \secref{sec:include}.

%%%%%%%%%%%%%%%%%%%%%%%%%%%%%%%%%%%%%%%%%%%%%%%%%%%%%%%%%%%%%%%%%%%%%%%%%%%%%%%%
\subsection{Legacy Detection}
\label{sec:detection}

The directive |\childdocmain| in the main file can detect
whether the complete document or merely a child is to be compiled
even without using the directive |\childdocof|.
This method is deprecated because it is less robust
and there is no compelling reason to use it;
it is merely provided for backward compatibility
and it may be removed in future versions.

If the detection mechanism is to be used,
it is mandatory to correctly specify
the filename of the main file as the argument of |\childdocmain|:
%
\begin{center}
\begin{tabular}{l}
|\input{childdoc.def}|\\
|\childdocmain{|\textit{main}|}|\\
\end{tabular}
\end{center}
%
If |\jobname| does not match the argument \textit{main} of |\childdocmain|,
it is assumed that |\jobname| points to the child file to be compiled.
When using |\childdocmain| with the main file specified as argument,
it suffices to start a child file
with just |\input{|\textit{main}|}|
without loading of the package and using |\childdocof|.
If instead all processing is done
with the appropriate \textsf{childdoc} directives,
the argument of \textit{main} of |\childdocmain| can be empty.

An alternative version of the command line processing described
in \secref{sec:commandline} using the detection mechanism reads:
%
\begin{center}
|... -jobname "|\textit{target}|" "|[\textit{flags}]%
[|\def\jobname{|\textit{dest}|}|]|\input{|\textit{main}|}"|
\end{center}

%%%%%%%%%%%%%%%%%%%%%%%%%%%%%%%%%%%%%%%%%%%%%%%%%%%%%%%%%%%%%%%%%%%%%%%%%%%%%%%%
\subsection{Manual Code}
\label{sec:manual}

In case one cannot be certain whether the definitions file |childdoc.def|
is installed on the target \TeX{} distribution
and one prefers not to ship it,
it is conceivable to paste a few relevant commands into the sources.

To that end, drop all statements |\input{childdoc.def}|
and perform the replacements as outlined below.
Instead of |\childdocmain{|\textit{main}|}| add the following code
to the top of the main file:
%
\begin{center}
\begin{tabular}{l}
|\||ifdefined\childdocname\endinput\||fi\newif\ifchilddoc|\\
|\edef\childdocname{\scantokens\expandafter{\jobname\noexpand}}|\\
|\def\childdocmain{|\textit{main}|}\||ifx\childdocmain\childdocname\||else|\\
|\childdoctrue\includeonly{\childdocname}\let\jobname\childdocmain\||fi|\\
\end{tabular}
\end{center}
%
Instead of |\childdocof{|\textit{main}|}| just include the main file
at the top of each child file:
%
\begin{center}
|\input{|\textit{main}|}|
\end{center}
%
A simple redirection |\childdocforward{|\textit{dest}|}| is achieved by:
%
\begin{center}
|\def\jobname{|\textit{dest}|}\input{\jobname}|
\end{center}
%
The redirection with prefix
|\childdocforwardprefix[|\textit{prefix}|]{|\textit{dest}|}|
is accomplished by:
%
\begin{center}
\begin{tabular}{l}
|{\edef\jobname{\scantokens\expandafter{\jobname\noexpand}}|\\
|\def\redirectjob |\textit{prefix}|#1~~~{\gdef\jobname{|\textit{dest}|#1}}|\\
|\expandafter\redirectjob\jobname~~~}\input{\jobname}|
\end{tabular}
\end{center}

In an alternative approach,
child documents can be compiled by a specific command line
without additional code or specific definitions:
%
\begin{center}
|... -jobname "|\textit{target}|" "|[\textit{flags}]%
|\includeonly{|\textit{dest}|}\input{|\textit{main}|}"|
\end{center}
%

%%%%%%%%%%%%%%%%%%%%%%%%%%%%%%%%%%%%%%%%%%%%%%%%%%%%%%%%%%%%%%%%%%%%%%%%%%%%%%%%
%%%%%%%%%%%%%%%%%%%%%%%%%%%%%%%%%%%%%%%%%%%%%%%%%%%%%%%%%%%%%%%%%%%%%%%%%%%%%%%%
\section{Information}

%%%%%%%%%%%%%%%%%%%%%%%%%%%%%%%%%%%%%%%%%%%%%%%%%%%%%%%%%%%%%%%%%%%%%%%%%%%%%%%%
\subsection{Copyright}

Copyright \copyright{} 2017--2018 Niklas Beisert

This work may be distributed and/or modified under the
conditions of the \LaTeX{} Project Public License, either version 1.3
of this license or (at your option) any later version.
The latest version of this license is in
  \url{http://www.latex-project.org/lppl.txt}
and version 1.3 or later is part of all distributions of \LaTeX{}
version 2005/12/01 or later.

This work has the LPPL maintenance status `maintained'.

The Current Maintainer of this work is Niklas Beisert.

This work consists of the files |README.txt|, |childdoc.ins| and |childdoc.dtx|
as well as the derived files |childdoc.def|, |cdocsamp.tex|
with |cdocsch1.tex|, |cdocsch2.tex|, |cdocspt3.tex|, |cdocspt4.tex|,
|cdocsdrf.tex|, |cdocsfn1.tex|, |cdocsfn2.tex|
as well as |childdoc.pdf|.

%%%%%%%%%%%%%%%%%%%%%%%%%%%%%%%%%%%%%%%%%%%%%%%%%%%%%%%%%%%%%%%%%%%%%%%%%%%%%%%%
\subsection{Files and Installation}

The package consists of the files:
%
\begin{center}
\begin{tabular}{ll}
    |README.txt|   & readme file \\
    |childdoc.ins| & installation file \\
    |childdoc.dtx| & source file \\
    |childdoc.def| & definition file \\
    |cdocsamp.tex| & sample main file \\
    |cdocsch1.tex| & sample include file \\
    |cdocsch2.tex| & sample include file \\
    |cdocspt3.tex| & sample part file \\
    |cdocspt4.tex| & sample part file \\
    |cdocsdrf.tex| & sample redirection file \\
    |cdocsfn1.tex| & sample redirection file \\
    |cdocsfn2.tex| & sample redirection file \\
    |childdoc.pdf| & manual
\end{tabular}
\end{center}
%
The distribution consists of the files
|README.txt|, |childdoc.ins| and |childdoc.dtx|.
%
\begin{itemize}
\item
Run (pdf)\LaTeX{} on |childdoc.dtx|
to compile the manual |childdoc.pdf| (this file).
\item
Run \LaTeX{} on |childdoc.ins| to create the definitions file |childdoc.def|
and the sample |cdocsamp.tex| with include files
|cdocsch1.tex|, |cdocsch2.tex|, |cdocspt3.tex|, |cdocspt4.tex|,
|cdocsdrf.tex|, |cdocsfn1.tex|, |cdocsfn2.tex|.
Then copy the file |childdoc.def| to an appropriate directory of your \LaTeX{}
distribution, e.g.\ \textit{texmf-root}|/tex/latex/childdoc|.
\end{itemize}

%%%%%%%%%%%%%%%%%%%%%%%%%%%%%%%%%%%%%%%%%%%%%%%%%%%%%%%%%%%%%%%%%%%%%%%%%%%%%%%%
\subsection{Related CTAN Packages}

There are several other packages which offer a similar functionality:
%
\begin{itemize}
\item
The packages
\href{http://ctan.org/pkg/docmute}{\textsf{docmute}},
\href{http://ctan.org/pkg/includex}{\textsf{includex}} and
\href{http://ctan.org/pkg/standalone}{\textsf{standalone}}
provide commands to include only the document body of
a child file thus allowing both files to be compiled individually.
\item
The packages \href{http://ctan.org/pkg/subdocs}{\textsf{subdocs}}
and \href{http://ctan.org/pkg/subfiles}{\textsf{subfiles}}
provide structures in which the main and child documents can be
encapsulated and allowing them to be compiled individually.
The inclusion mechanism is different from the conventional |\include|.
\item
The package \href{http://ctan.org/pkg/combine}{\textsf{combine}}
is an elaborate solution to combine several documents into one.
\end{itemize}
%
See also the CTAN topic \href{http://ctan.org/topic/subdocs}{\textsf{subdocs}}
for further related packages.
The present package differs from the above solutions in that
a document structure constructed with the conventional |\include| mechanism
just needs two extra commands at the top of every file
such that all constituent files can be compiled individually.

%%%%%%%%%%%%%%%%%%%%%%%%%%%%%%%%%%%%%%%%%%%%%%%%%%%%%%%%%%%%%%%%%%%%%%%%%%%%%%%%
%\subsection{Feature Suggestions}
%
%The following is a list of features which may be useful for future
%versions of this package:
%%
%\begin{itemize}
%\item
%\ldots
%\end{itemize}

%%%%%%%%%%%%%%%%%%%%%%%%%%%%%%%%%%%%%%%%%%%%%%%%%%%%%%%%%%%%%%%%%%%%%%%%%%%%%%%%
\subsection{Revision History}

%%%%%%%%%%%%%%%%%%%%%%%%%%%%%%%%%%%%%%%%
\paragraph{v2.0:} 2018/12/30

\begin{itemize}
\item
immediate forward processing
\item
added |\childdocby| mechanism
\item
manual restructured
\end{itemize}

%%%%%%%%%%%%%%%%%%%%%%%%%%%%%%%%%%%%%%%%
\paragraph{v1.6:} 2018/01/17

\begin{itemize}
\item
application for development of include files
\item
corrections to manual
\end{itemize}

%%%%%%%%%%%%%%%%%%%%%%%%%%%%%%%%%%%%%%%%
\paragraph{v1.5:} 2017/05/21

\begin{itemize}
\item
more complete structuring introduced
\item
|\childdocof| introduced
\item
|\childdoc| renamed to |\childdocmain|
\item
|\childredirect| renamed to |\childdocforward| and |\childdocforwardprefix|
and functionality expanded
\end{itemize}

%%%%%%%%%%%%%%%%%%%%%%%%%%%%%%%%%%%%%%%%
\paragraph{v1.0:} 2017/04/27

\begin{itemize}
\item
manual and install package
\item
first version published on CTAN
\end{itemize}

%%%%%%%%%%%%%%%%%%%%%%%%%%%%%%%%%%%%%%%%
\paragraph{v0.6:} 2017/04/26

\begin{itemize}
\item
redirection mechanism added
\end{itemize}

%%%%%%%%%%%%%%%%%%%%%%%%%%%%%%%%%%%%%%%%
\paragraph{v0.5:} 2017/04/26

\begin{itemize}
\item
functionality in definition file
\end{itemize}


%%%%%%%%%%%%%%%%%%%%%%%%%%%%%%%%%%%%%%%%%%%%%%%%%%%%%%%%%%%%%%%%%%%%%%%%%%%%%%%%
%%%%%%%%%%%%%%%%%%%%%%%%%%%%%%%%%%%%%%%%%%%%%%%%%%%%%%%%%%%%%%%%%%%%%%%%%%%%%%%%
%%%%%%%%%%%%%%%%%%%%%%%%%%%%%%%%%%%%%%%%%%%%%%%%%%%%%%%%%%%%%%%%%%%%%%%%%%%%%%%%
\appendix

\settowidth\MacroIndent{\rmfamily\scriptsize 000\ }

 \DocInput{childdoc.dtx}

\end{document}
%</driver>
% \fi
%
% %%%%%%%%%%%%%%%%%%%%%%%%%%%%%%%%%%%%%%%%%%%%%%%%%%%%%%%%%%%%%%%%%%%%%%%%%%%%%%
% %%%%%%%%%%%%%%%%%%%%%%%%%%%%%%%%%%%%%%%%%%%%%%%%%%%%%%%%%%%%%%%%%%%%%%%%%%%%%%
% \section{Sample}
%\iffalse
%<*samplemain>
%\fi
%
% The following presents a sample document
% with two chapters, two parts, a title page,
% a compile flag as well as three forwarding files to set the flag.
% It consists of eight |.tex| files:
% \begin{center}
% \begin{tabular}{ll}
% |cdocsamp.tex|&main file\\
% |cdocsch1.tex|&include file for chapter 1\\
% |cdocsch2.tex|&include file for chapter 2\\
% |cdocspt3.tex|&include file for part 3\\
% |cdocspt4.tex|&include file for part 4\\
% |cdocsdrf.tex|&forwarding file for main file in draft mode\\
% |cdocsfi1.tex|&forwarding file for final version of chapter 1\\
% |cdocsfi2.tex|&forwarding file for final version of chapter 2\\
% \end{tabular}
% \end{center}
% Each of the eight files can be compiled directly by the \LaTeX{} compiler.
%
% %%%%%%%%%%%%%%%%%%%%%%%%%%%%%%%%%%%%%%
% \paragraph{Main File.}
%
% The main file is called |cdocsamp.tex|.
%
% Load the \textsf{childdoc} definitions and
% declare the filename for the main document:
%    \begin{macrocode}
\input{childdoc.def}
\childdocmain{}
%    \end{macrocode}

% Optional override for |\version| flag:
%    \begin{macrocode}
%%\ifchilddoc\else\providecommand{\version}{draft}\fi
%    \end{macrocode}

% Define the default values for the |\version| flag
% (|final| for the main file and |draft| for childs):
%    \begin{macrocode}
\ifchilddoc
\providecommand{\version}{draft}
\else
\providecommand{\version}{final}
\fi
%    \end{macrocode}

% Load the standard document class:
%    \begin{macrocode}
\documentclass[12pt]{article}
%    \end{macrocode}

% Start the document body:
%    \begin{macrocode}
\begin{document}
%    \end{macrocode}

% Declare a title page.
% Print title, part of document being processed and version flag:
%    \begin{macrocode}
\addtocounter{page}{-1}
\begin{center}
{\LARGE\bfseries{}childdoc example\par}
\vspace{1cm}
\ifchilddoc
\ifchilddocmanual part\else chapter\fi:
`\childdocname' of `\childdocjob'\par
\else
main document: `\childdocjob'\par
\fi
version: \version\par
\end{center}
\newpage
%    \end{macrocode}

% Manually include selected file,
% otherwise process as usual:
%    \begin{macrocode}
\ifchilddocmanual
\section*{part `\childdocname'}
\input{\childdocname}
\else
%    \end{macrocode}

% Include the two chapters:
%    \begin{macrocode}
\include{cdocsch1}
\include{cdocsch2}
%    \end{macrocode}

% Include the two parts unless only chapters should be displayed:
%    \begin{macrocode}
\ifchilddoc\else
\section{part three}
\input{cdocspt3}
\section{part four}
\input{cdocspt4}
\fi
%    \end{macrocode}

% Process as usual until here:
%    \begin{macrocode}
\fi
%    \end{macrocode}

% End of document body:
%    \begin{macrocode}
\end{document}
%    \end{macrocode}
%\iffalse
%</samplemain>
%\fi
%
% %%%%%%%%%%%%%%%%%%%%%%%%%%%%%%%%%%%%%%
% \paragraph{Chapter Include Files.}
%
% The include files are called |cdocsch1.tex| and |cdocsch2.tex|.
%
%\iffalse
%<*samplechap1|samplechap2>
%\fi

% Optional override for |\version| flag:
%    \begin{macrocode}
%%\providecommand{\version}{final}
%    \end{macrocode}

% Include the main document:
%    \begin{macrocode}
\input{childdoc.def}
\childdocof{cdocsamp}
%    \end{macrocode}

%\iffalse
%</samplechap1|samplechap2>
%\fi
%
%\iffalse
%<*samplechap1>
%\fi
% Some text for chapter 1:
%    \begin{macrocode}
\section{one}
some text in chapter one
%    \end{macrocode}

%\iffalse
%</samplechap1>
%\fi
% Some text for chapter 2:
%\iffalse
%<*samplechap2>
%\fi
%    \begin{macrocode}
\section{two}
more text in chapter two
%    \end{macrocode}

%\iffalse
%</samplechap2>
%\fi
%
% %%%%%%%%%%%%%%%%%%%%%%%%%%%%%%%%%%%%%%
% \paragraph{Part Include Files.}
%
% The include files are called |cdocspt3.tex| and |cdocspt4.tex|.
%
%\iffalse
%<*samplepart3|samplepart4>
%\fi

% Optional override for |\version| flag:
%    \begin{macrocode}
%%\providecommand{\version}{final}
%    \end{macrocode}

% Include the main document:
%    \begin{macrocode}
\input{childdoc.def}
\childdocby{cdocsamp}
%    \end{macrocode}

%\iffalse
%</samplepart3|samplepart4>
%\fi
%
%\iffalse
%<*samplepart3>
%\fi
% Some text for part 3:
%    \begin{macrocode}
some text in part three
%    \end{macrocode}

%\iffalse
%</samplepart3>
%\fi
% Some text for part 4:
%\iffalse
%<*samplepart4>
%\fi
%    \begin{macrocode}
more text in part four
%    \end{macrocode}

%\iffalse
%</samplepart4>
%\fi
%
% %%%%%%%%%%%%%%%%%%%%%%%%%%%%%%%%%%%%%%
% \paragraph{Forwarding for a Complete Draft.}
%
% The following forwarding file |cdocsdrf.tex|
% compiles the main document in draft mode:
%\iffalse
%<*sampledraft>
%\fi
%    \begin{macrocode}
\def\version{draft}
\input{childdoc.def}
\childdocforward{cdocsamp}
%    \end{macrocode}

%\iffalse
%</sampledraft>
%\fi
%
% %%%%%%%%%%%%%%%%%%%%%%%%%%%%%%%%%%%%%%
% \paragraph{Forwarding for Final Version of the Chapters.}
%
% The following forwarding files |cdocsfn1.tex| and |cdocsfn2.tex|
% (with identical content)
% compile the final versions of the child documents
% |cdocsch1.tex| and |cdocsch2.tex|, respectively:
%\iffalse
%<*samplefinal>
%\fi
%    \begin{macrocode}
\def\version{final}
\input{childdoc.def}
\childdocforwardprefix[cdocsamp]{cdocsfn}{cdocsch}
%    \end{macrocode}

%\iffalse
%</samplefinal>
%\fi
%
% %%%%%%%%%%%%%%%%%%%%%%%%%%%%%%%%%%%%%%
% \paragraph{Command Line Processing.}
%
% The following three command lines generate the output files
% |cdocscld|, |cdocscl1| and |cdocscl2|
% which should be identical to
% |cdocsdrf|, |cdocsch1| and |cdocsfn2|, respectively:
% \begin{center}
% \begin{tabular}{l}
% |latex -jobname cdocscld \|\\
% |  "\def\version{draft}\input{childdoc.def}\childdocforward{cdocsamp}"|\\
% |latex -jobname cdocscl1 \|\\
% |  "\input{childdoc.def}\childdocforward[cdocsamp]{cdocsch1}"|\\
% |latex -jobname cdocscl2 \|\\
% |  "\def\version{final}\input{childdoc.def}\childdocforward{cdocsch2}"|
% \end{tabular}
% \end{center}
% Note that the trailing backslash on each first line
% merely continues the input to the second line
% (for convenient cut ant paste).
% Furthermore, the command |latex| can be replaced by any
% of its alternative versions such as |pdflatex|.
%
% %%%%%%%%%%%%%%%%%%%%%%%%%%%%%%%%%%%%%%%%%%%%%%%%%%%%%%%%%%%%%%%%%%%%%%%%%%%%%%
% %%%%%%%%%%%%%%%%%%%%%%%%%%%%%%%%%%%%%%%%%%%%%%%%%%%%%%%%%%%%%%%%%%%%%%%%%%%%%%
% \section{Implementation}
%\iffalse
%<*package>
%\fi
%
% This section describes the definitions file |childdoc.def|.

% The definitions cannot be loaded using |\usepackage| or |\RequirePackage|
% which has a mechanism to prevent loading a style file more than once.
% When loading the definitions by means of |\input|
% multiple instances have to be prevented manually:
%\iffalse
%This code needs to be before the `\ProvidesFile' directive
%which is defined at the beginning of this file.
%Therefore it is also placed there and commented out here.
%</package>
%<*discard>
%\fi
%    \begin{macrocode}
\ifdefined\childdocmain\endinput\fi
%    \end{macrocode}
%\iffalse
%</discard>
%<*package>
%\fi
%
% \macro{\ifchilddoc}
% \macro{\ifchilddocmanual}
% The conditional |\ifchilddoc| tells whether a
% child (true) or main (false) document is being compiled.
% The conditional |\ifchilddocmanual| tells whether
% the |\includeonly| mechanism is used (false) or
% the selection of child files must be performed manually (true).
% The definitions initialise to false:
%    \begin{macrocode}
\newif\ifchilddoc
\newif\ifchilddocmanual
%    \end{macrocode}

% \macro{\childdocname}
% \macro{\childdocjob}
% The macro |\childdocname| stores the name of the main document
% to be compiled. The macro |\childdocjob| stores the name of
% the document on which the \LaTeX{} compiler was originally invoked.
% The content of |\jobname| cannot be compared
% to filenames specified in the source due to different catcodes.
% The following code rescans |\jobname|, stores the result
% in |\childdocname| and saves a copy in |\childdocjob|:
%    \begin{macrocode}
\edef\childdocname{\scantokens\expandafter{\jobname\noexpand}}
\let\childdocjob\childdocname
%    \end{macrocode}

% \macro{\childdocdisable}
% The macro |\childdocdisable| prevents the main file
% from being processed more than once.
% At this stage, the main document command |\childdocmain|
% is assumed to be called once again where it should do nothing.
% Any subsequent call to it should prevent
% a secondary processing of the main document
% It overwrites the forwarding commands
% |\childdocof| and |\childdocforward|
% with empty macros to prevent further inclusions of the main document:
%    \begin{macrocode}
\newcommand{\childdocdisable}
{
  \renewcommand{\childdocmain}[1]{\renewcommand{\childdocmain}[1]{\endinput}}
  \renewcommand{\childdocof}[1]{}
  \renewcommand{\childdocby}[2][]{}
  \renewcommand{\childdocforward}[2][]{}
  \renewcommand{\childdocdisable}{}
}
%    \end{macrocode}

% \macro{\childdocmain}
% The macro |\childdocmain| is to be called at the top of the main file
% with nothing or the main filename (without extension) as argument.
% First, it breaks loops.
% If the argument is not empty and does not match |\childdocname|
% (which is set by the first inclusion of |childdoc.def|),
% |\ifchilddoc| is set to true, |\includeonly| is applied to the child file
% and |\jobname| is set to the main file
% (for proper handling of |.aux| files):
%    \begin{macrocode}
\newcommand{\childdocmain}[1]
{
  \childdocdisable\childdocmain{}
  \if?#1?\else
    \begingroup
      \def\childdoctmp{#1}
      \ifx\childdoctmp\childdocname
        \def\childdoctmp{}
      \else
        \def\childdoctmp
        {
          \childdoctrue
          \includeonly{\childdocname}
          \def\childdocjob{#1}
          \def\jobname{#1}
        }
      \fi
      \expandafter
    \endgroup
    \childdoctmp
  \fi
}
%    \end{macrocode}

% \macro{\childdocof}
% The command |\childdocof| redirects
% compilation to the main file |#1|.
%    \begin{macrocode}
\newcommand{\childdocof}[1]
{
  \childdocdisable
  \childdoctrue
  \includeonly{\childdocname}
  \def\jobname{#1}
  \def\childdocjob{#1}
  \input{#1}
}
%    \end{macrocode}

% \macro{\childdocby}
% The command |\childdocby| ....
%    \begin{macrocode}
\newcommand{\childdocby}[2][]
{
  \childdocdisable
  \childdoctrue
  \childdocmanualtrue
  \if?#1?\else
    \def\jobname{#2}
  \fi
  \def\childdocjob{#2}
  \input{#2}
  \endinput
}
%    \end{macrocode}

% \macro{\childdocforward}
% The command |\childdocforward| redirects
% compilation to the main file or
% (if the optional argument is given) a child file.
% Parameters are set as if the main file
% or a child file starting with |\childdocof| was compiled.
% Then compilation is handed over to the main file:
%    \begin{macrocode}
\newcommand{\childdocforward}[2][]
{
  \begingroup
    \if?#1?
      \def\childdoctmp
      {
        \def\childdocname{#2}
        \def\childdocjob{#2}
        \def\jobname{#2}
        \input{#2}
        \endinput
      }
    \else
      \def\childdoctmp
      {
        \childdocdisable
        \def\childdocname{#2}
        \childdoctrue
        \includeonly{#2}
        \def\childdocjob{#1}
        \def\jobname{#1}
        \input{#1}
        \endinput
      }
    \fi
    \expandafter
  \endgroup
  \childdoctmp
}
%    \end{macrocode}

% \macro{\childdocforwardprefix}
% The command |\childdocforwardprefix| redirects
% compilation to the main or a child file by means of a pattern.
% The prefix |#1| in the current filename is replaced by |#2|
% and the suffix of the current filename is kept
% (it is assumed that the filename does not contain the substring `|~~~|'
% which is used as a delimiter).
% Compilation is handed over to the new file by |\childdocforward|:
%    \begin{macrocode}
\newcommand{\childdocforwardprefix}[3][]
{
  \begingroup
    \def\childdocextract #2##1~~~{\def\childdoctmp{\childdocforward[#1]{#3##1}}}
    \expandafter\childdocextract\childdocname~~~
    \expandafter
  \endgroup
  \childdoctmp
}
%    \end{macrocode}

% \macro{\childdoc}
% The deprecated macro |\childdoc| is a legacy version of |\childdocmain|:
%    \begin{macrocode}
\newcommand{\childdoc}{\childdocmain}
%    \end{macrocode}

% \macro{\childdocredirect}
% The deprecated macro |\childdocredirect| is a legacy version
% of |\childdocforward| and |\childdocforwardprefix|:
%    \begin{macrocode}
\newcommand{\childdocredirect}[2][]
{
  \begingroup
    \if?#1?
      \def\childdoctmp{\childdocforward{#2}}
    \else
      \def\childdoctmp{\childdocforwardprefix{#1}{#2}}
    \fi
    \expandafter
  \endgroup
  \childdoctmp
}
%    \end{macrocode}

%\iffalse
%</package>
%\fi
%
\endinput
|\\
|\childdocforwardprefix[|\textit{main}|]{|\textit{prefix}|}{|\textit{dest}|}|
\end{tabular}
\end{center}
%
the destination file is determined by a pattern
depending on the current file:
To make this work, the current file must be called
`{\textit{prefix}\hspace{0.2em}\textit{suffix}}'
with \textit{prefix} matching precisely the argument.
Processing is then passed on to the file
`{\textit{dest}\hspace{0.2em}\textit{suffix}}'.
Surely, the same effect is achieved by
directly specifying the
argument `{\textit{dest}\hspace{0.2em}\textit{suffix}}'
in the first form.
However, that requires to set up a different file
for each child. With the alternative form of the command
all these files can have exactly the same content
which simplifies setting them up and maintaining them.

For example, the following file |draft.tex|
with a compilation flag |\version| as described in \secref{sec:flags}
compiles the main document as a draft:
%
\begin{center}
\begin{tabular}{l}
|\def\version{draft}|\\
|% \iffalse
%
% childdoc.dtx Copyright (C) 2017-2018 Niklas Beisert
%
% This work may be distributed and/or modified under the
% conditions of the LaTeX Project Public License, either version 1.3
% of this license or (at your option) any later version.
% The latest version of this license is in
%   http://www.latex-project.org/lppl.txt
% and version 1.3 or later is part of all distributions of LaTeX
% version 2005/12/01 or later.
%
% This work has the LPPL maintenance status `maintained'.
%
% The Current Maintainer of this work is Niklas Beisert.
%
% This work consists of the files childdoc.dtx and childdoc.ins
% and the derived files childdoc.def and cdocsamp.tex with
% cdocsch1.tex, cdocsch2.tex, cdocsdrf.tex, cdocsfn1.tex, cdocsfn2.tex.
%
%<package>\ifdefined\childdocmain\endinput\fi
%<package>\ProvidesFile{childdoc.def}[2018/12/30 v2.0 child document driver]
%<samplemain>\ProvidesFile{cdocsamp.tex}[2018/12/30 v2.0 sample for childdoc]
%<*driver>
%\ProvidesFile{childdoc.drv}[2018/12/30 v2.0 childdoc reference manual file]
\PassOptionsToClass{10pt,a4paper}{article}
\documentclass{ltxdoc}

\usepackage[margin=35mm]{geometry}
\usepackage{hyperref}
\usepackage{hyperxmp}
\usepackage[usenames]{color}

\hypersetup{colorlinks=true}
\hypersetup{pdfstartview=FitH}
\hypersetup{pdfpagemode=UseNone}
\hypersetup{pdfsource={}}
\hypersetup{pdflang={en-UK}}
\hypersetup{pdfcopyright={Copyright 2017-2018 Niklas Beisert.
  This work may be distributed and/or modified under the
  conditions of the LaTeX Project Public License, either version 1.3
  of this license or (at your option) any later version.}}
\hypersetup{pdflicenseurl={http://www.latex-project.org/lppl.txt}}
\hypersetup{pdfcontactaddress={ETH Zurich, ITP, HIT K,
  Wolfgang-Pauli-Strasse 27}}
\hypersetup{pdfcontactpostcode={8093}}
\hypersetup{pdfcontactcity={Zurich}}
\hypersetup{pdfcontactcountry={Switzerland}}
\hypersetup{pdfcontactemail={nbeisert@itp.phys.ethz.ch}}
\hypersetup{pdfcontacturl={http://people.phys.ethz.ch/\xmptilde nbeisert/}}

\newcommand{\secref}[1]{\hyperref[#1]{section \ref*{#1}}}

\parskip1ex
\parindent0pt
\let\olditemize\itemize
\def\itemize{\olditemize\parskip0pt}

\begin{document}

\title{The \textsf{childdoc} Package}
\hypersetup{pdftitle={The childdoc Package}}
\author{Niklas Beisert\\[2ex]
  Institut f\"ur Theoretische Physik\\
  Eidgen\"ossische Technische Hochschule Z\"urich\\
  Wolfgang-Pauli-Strasse 27, 8093 Z\"urich, Switzerland\\[1ex]
  \href{mailto:nbeisert@itp.phys.ethz.ch}
  {\texttt{nbeisert@itp.phys.ethz.ch}}}
\hypersetup{pdfauthor={Niklas Beisert}}
\hypersetup{pdfsubject={Manual for the LaTeX2e Package childdoc}}
\date{30 December 2018, \textsf{v2.0}}
\maketitle

\begin{abstract}\noindent
\textsf{childdoc} is a \LaTeXe{} package
that enables the direct compilation
of document sections included by |\include|
to individual files.
\end{abstract}

\begingroup
\parskip0ex
\tableofcontents
\endgroup

%%%%%%%%%%%%%%%%%%%%%%%%%%%%%%%%%%%%%%%%%%%%%%%%%%%%%%%%%%%%%%%%%%%%%%%%%%%%%%%%
%%%%%%%%%%%%%%%%%%%%%%%%%%%%%%%%%%%%%%%%%%%%%%%%%%%%%%%%%%%%%%%%%%%%%%%%%%%%%%%%
\section{Introduction}

\LaTeX{} provides a mechanism to structure a large document (such as a book)
into a main file and several child files (containing the chapters)
using the |\include| command.
This mechanism is beneficial for documents
which span hundreds of pages in order to
make the source file(s) more manageable.
Moreover, compilation can be restricted to
selected child files by means of the |\includeonly| command.
The latter feature can be used to reduce the compilation time while editing
(this was significantly more useful in the earlier days of \LaTeX{})
or to generate a smaller document which is easier to navigate.
Another application of |\includeonly| is to generate
documents consisting of selected parts of the complete document.

However, there are a few drawbacks of the plain |\include| mechanism:
\begin{itemize}
\item
The child files cannot be compiled on their own,
they can only be compiled via the main file.
A naive editing environment
(such as a text editor with an option
to have the current file processed by \LaTeX)
may require one to switch to the main file before compiling;
attempting to compile the child file produces errors.
\item
The main file must be modified (each time)
to adjust the |\includeonly| command
to the present needs. This easily leaves the main file in a messy state.
\item
The generated document will always carry the filename
of the main document. This is inconvenient if
several child files are to be compiled and
to be kept for distribution.
\end{itemize}

The present package provides a simple interface
to make child files individually compilable by \LaTeX{}.
Compiling a child file then has the same effect as compiling
the main file with an |\includeonly| command
to select the appropriate child.
Moreover the generated document will carry the name of the child
rather than the main file.
This resolves all three above issues.

This feature is meant to make the editing of books,
thesis documents and lecture notes somewhat more convenient.
However, the package can also be used efficiently for
composing a series of documents (such as exercise sheets)
which are typically distributed individually.
It then assists the author in generating the individual documents
(potentially in different versions)
as well as a document containing the collected series.
Another application is in developing style files
or other kinds of included material
where compilation of the style file could redirect
to a sample or test file.

%%%%%%%%%%%%%%%%%%%%%%%%%%%%%%%%%%%%%%%%%%%%%%%%%%%%%%%%%%%%%%%%%%%%%%%%%%%%%%%%
%%%%%%%%%%%%%%%%%%%%%%%%%%%%%%%%%%%%%%%%%%%%%%%%%%%%%%%%%%%%%%%%%%%%%%%%%%%%%%%%
\section{Usage}

First of all, the package \textsf{childdoc} is \emph{not} a standard
\LaTeXe{} |.sty| style file! Therefore it needs to be invoked in
a non-standard way.

%%%%%%%%%%%%%%%%%%%%%%%%%%%%%%%%%%%%%%%%%%%%%%%%%%%%%%%%%%%%%%%%%%%%%%%%%%%%%%%%
\subsection{Included Files}
\label{sec:include}

%%%%%%%%%%%%%%%%%%%%%%%%%%%%%%%%%%%%%%%%
\DescribeMacro{\childdocmain}
To use the package, add the commands
\begin{center}
\begin{tabular}{l}
|\input{childdoc.def}|\\
|\childdocmain{}|\\
\end{tabular}
\end{center}
at the very top of the main \LaTeX{} file,
in particular \emph{before} the |\documentclass| statement!
The argument of |\childdocmain| should be left empty
(but it must be present).

%%%%%%%%%%%%%%%%%%%%%%%%%%%%%%%%%%%%%%%%
\DescribeMacro{\childdocof}
Furthermore, add the commands
\begin{center}
\begin{tabular}{l}
|\input{childdoc.def}|\\
|\childdocof{|\textit{main}|}|\\
\end{tabular}
\end{center}
at the top of every child file \textit{child}
which is included by |\include{|\textit{child}|}|
from within the main file
(or at least for those files to be compiled individually).
The argument \textit{main} must be the filename of the main file.

There are a couple of
considerations in setting up the main and child documents:

%%%%%%%%%%%%%%%%%%%%%%%%%%%%%%%%%%%%%%%%
\paragraph{Restrictions.}

Please note the following restrictions:
\begin{itemize}
\item
|\childdocmain| must be called with one argument \textit{main}
to ensure compatibility with earlier version of the package.
It must either be empty (|\childdocmain{}|)
or precisely match the filename of the main file in which it is specified.
See \secref{sec:detection} for further information.
\item
The filename \textit{main} must be specified without the |.tex| extension.
\item
The filename \textit{main} is case sensitive
(even in case-insensitive file systems)
due to internal string comparison.
\item
The argument \textit{main} should be fully expanded, it cannot be a macro.
\item
Subdirectories and special characters should be avoided in filenames.
\item
The command |\childdocmain{|\textit{main}|}| must be followed by a whitespace.
It should not be followed immediately by another command
or by a comment mark `|%|'.
This is because the \TeX{} parser reads the token immediately following
the argument of |\childdocmain| and puts it
at the beginning of every child section;
however, a white\-space is ignored.
\end{itemize}

%%%%%%%%%%%%%%%%%%%%%%%%%%%%%%%%%%%%%%%%
\paragraph{Content of Main File.}

It is advisable to place all content in the child files included by |\include|.
Any output contained in the main file will appear in all child documents
unless suppressed manually;
it cannot be suppressed automatically by the |\includeonly| directive
and thus should normally be avoided.
A method to include some content in the main file
by means of conditional processing is described in \secref{sec:conditional}.

%%%%%%%%%%%%%%%%%%%%%%%%%%%%%%%%%%%%%%%%
\paragraph{Page Numbering.}

When only a part of the document is compiled,
the appropriate numbering of pages
(as well as other status parameters)
is determined from the |.aux| files.
The latter contain information from previous passes.
However this information needs to propagate through
all intermediate child documents.
Therefore the page numbering in child documents may well
be inconsistent until the complete document is compiled at least once.

A useful (if unconventional) way to always ensure a consistent
page numbering is to restart the numbering in each child document
and denote the pages by `\textit{child}|.|\textit{page}'
where \textit{child} represents the chapter/section number of the child file.
This can be achieved by the command
|\numberwithin{page}{|\textit{child}|}|
of the \textsf{amsmath} package
where \textit{child} can be |chapter| or |section|
depending on the chosen structuring.
Alternatively, one can modify the macro |\thepage| appropriately
and reset the counter |page| at the start of each child file.

%%%%%%%%%%%%%%%%%%%%%%%%%%%%%%%%%%%%%%%%%%%%%%%%%%%%%%%%%%%%%%%%%%%%%%%%%%%%%%%%
\subsection{Conditional Processing}
\label{sec:conditional}

The package provides a mechanism to compile different versions
of a document. To customise the versions further some conditional processing
can come in handy to distinguish which version is being compiled.
The package provides two macros to describe the compilation context:

%%%%%%%%%%%%%%%%%%%%%%%%%%%%%%%%%%%%%%%%
\DescribeMacro{\ifchilddoc}
The conditional |\ifchilddoc| distinguishes between the compilation of
child documents and the main document:
%
\begin{center}
|\ifchilddoc |\textit{child-code}| |[|\||else |\textit{main-code}]| \||fi|
\end{center}

%%%%%%%%%%%%%%%%%%%%%%%%%%%%%%%%%%%%%%%%
\DescribeMacro{\childdocname}
\DescribeMacro{\childdocjob}
The macro |\childdocname| contains the filename (without extension)
of the main or child file being processed.
Note that |\childdocjob| will always contain the name of the main file.

%%%%%%%%%%%%%%%%%%%%%%%%%%%%%%%%%%%%%%%%
\paragraph{Title Page.}

Conditional processing can be used to include a title or banner page
in the main document when proper precautions are taken.
Importantly, the code in the main file should ensure that the page counter
(as well as other status parameters which are stored in the |.aux| files)
takes the same value after the conditional processing.
Otherwise the page numbers may take divergent values
depending on which part is compiled.

For example, a title page could be declared by:
%
\begin{center}
\begin{tabular}{l}
|\ifchilddoc\||else|\\
|\addtocounter{page}{-1}|\\
\textit{code for title page}\\
|\newpage|\\
|\||fi|
\end{tabular}
\end{center}
%
A banner page for the child documents can be generated by:
%
\begin{center}
\begin{tabular}{l}
|\ifchilddoc|\\
|\addtocounter{page}{-1}|\\
\textit{code for banner page}\\
|\newpage|\\
|\||fi|
\end{tabular}
\end{center}
%
Here one could write a message such as:
\begin{center}
|This is the part \childdocname{} of \childdocjob{}.|
\end{center}

%%%%%%%%%%%%%%%%%%%%%%%%%%%%%%%%%%%%%%%%%%%%%%%%%%%%%%%%%%%%%%%%%%%%%%%%%%%%%%%%
\subsection{Flags}
\label{sec:flags}

The package makes it easy to generate different versions
of the main or child documents.
To this end compilation flags can be defined
and assigned different default values.
They will be particularly useful in conjunction
with the forwarding mechanism described in \secref{sec:forward}.

For example, it may be useful to have a flag |\version|
which can be set to |draft| or |final|.
The document source will contain some conditional code
depending on the value of |\version|.
Suppose further, the flag should default to |final| for the main file
and to |draft| for child files
which is a natural assignment for editing the document.
This is achieved by placing the following code
in the preamble of the main document
(below the |\childdocmain| directive):
%
\begin{center}
\begin{tabular}{l}
|\ifchilddoc|\\
|\providecommand{\version}{draft}|\\
|\||else|\\
|\providecommand{\version}{final}|\\
|\||fi|
\end{tabular}
\end{center}
%
The definition by |\providecommand| makes sure
that previous definitions are not overwritten.
Further statements |\providecommand{\version}{...}|
can thus be added before the above code to override it.

For the main file, one might add a line
(between |\childdocmain| and the above block)
%
\begin{center}
|%\ifchilddoc\||else\providecommand{\version}{draft}\||fi|
\end{center}
%
which can be uncommented to produce a draft version.
Likewise one can add a line to the very top of a child file
(above the |\childdocof{|\textit{main}|}| directive)
%
\begin{center}
|%\providecommand{\version}{final}|
\end{center}
%
which can be uncommented to produce the final version of this child document.

%%%%%%%%%%%%%%%%%%%%%%%%%%%%%%%%%%%%%%%%%%%%%%%%%%%%%%%%%%%%%%%%%%%%%%%%%%%%%%%%
\subsection{Forwarding}
\label{sec:forward}

Different versions of the main or child documents
using compilation flags as described in \secref{sec:flags}
can be (permanently) stored in different files
for convenient compilation, viewing and distribution.
To this end, the package defines a command
to pass on compilation to a different file:

%%%%%%%%%%%%%%%%%%%%%%%%%%%%%%%%%%%%%%%%
\DescribeMacro{\childdocforward}
The command |\childdocforward| redirects processing to
another source file:
%
\begin{center}
\begin{tabular}{l}
|\input{childdoc.def}|\\
|\childdocforward[|\textit{main}|]{|\textit{dest}|}|\\
\end{tabular}
\end{center}
%
The argument \textit{dest} is the destination file
(without extension).
It should be the main file or one of the child files.
Note that further \textsf{childdoc} directives
such as |\childdocof| and |\childdocforward|
in the indicated file will be processed in this form.
The optional argument \textit{main}
passes on directly to the main file \textit{main}
while pretending to compile the child \textit{dest}.
This form behaves as if \textit{dest}
issues |\childdocof{|\textit{main}|}| right away,
and no further \textsf{childdoc} directives will be processed.

%%%%%%%%%%%%%%%%%%%%%%%%%%%%%%%%%%%%%%%%
\DescribeMacro{\...prefix}
In the alternative form |\childdocforwardprefix|,
%
\begin{center}
\begin{tabular}{l}
|\input{childdoc.def}|\\
|\childdocforwardprefix[|\textit{main}|]{|\textit{prefix}|}{|\textit{dest}|}|
\end{tabular}
\end{center}
%
the destination file is determined by a pattern
depending on the current file:
To make this work, the current file must be called
`{\textit{prefix}\hspace{0.2em}\textit{suffix}}'
with \textit{prefix} matching precisely the argument.
Processing is then passed on to the file
`{\textit{dest}\hspace{0.2em}\textit{suffix}}'.
Surely, the same effect is achieved by
directly specifying the
argument `{\textit{dest}\hspace{0.2em}\textit{suffix}}'
in the first form.
However, that requires to set up a different file
for each child. With the alternative form of the command
all these files can have exactly the same content
which simplifies setting them up and maintaining them.

For example, the following file |draft.tex|
with a compilation flag |\version| as described in \secref{sec:flags}
compiles the main document as a draft:
%
\begin{center}
\begin{tabular}{l}
|\def\version{draft}|\\
|\input{childdoc.def}|\\
|\childdocforward{|\textit{main}|}|
\end{tabular}
\end{center}
%
Likewise, the following files |final|\textit{nn}|.tex|
compile the final version of the child document
|child|\textit{nn}|.tex|:
%
\begin{center}
\begin{tabular}{l}
|\def\version{final}|\\
|\input{childdoc.def}|\\
|\childdocforwardprefix{final}{child}|
\end{tabular}
\end{center}
%

Note that when several versions of a main file and/or of each child file
are to be generated, it may be convenient to set up a |Makefile| or
shell script to automatise the process.

%%%%%%%%%%%%%%%%%%%%%%%%%%%%%%%%%%%%%%%%%%%%%%%%%%%%%%%%%%%%%%%%%%%%%%%%%%%%%%%%
\subsection{Command Line Processing}
\label{sec:commandline}

The effect of redirection files can also be achieved by invoking
the \LaTeX{} compiler with a more elaborate command line.
Most conveniently this should be done as part
of a shell script or a |Makefile|.

When using \textsf{childdoc} in the main file, the following
command lines effectively perform a redirection
(note that depending on the shell being used,
backslashes may have to be doubled: `|\|' $\to$ `|\\|'):
%
\begin{center}
|... -jobname "|\textit{target}|" |\\|"|[\textit{flags}]%
|\input{childdoc.def}\childdocforward[|\textit{main}|]{|\textit{dest}|}"|
\end{center}
%
Here \textit{target} is the name of the output file,
\textit{main} is the name of the main file
and \textit{dest} is the name of the main or child file to be processed
(all filenames without extensions).
The optional argument \textit{main} can be omitted
if \textit{main} matches \textit{dest}.
Optionally, compilation \textit{flags} can be defined via |\def| commands.
This command line makes the \TeX{} engine believe
it is compiling the file \textit{target}
whose content is specified as the latter parameter.
The provided code then forwards the processing to
\textit{main} or \textit{dest} as described in \secref{sec:forward}.

%%%%%%%%%%%%%%%%%%%%%%%%%%%%%%%%%%%%%%%%%%%%%%%%%%%%%%%%%%%%%%%%%%%%%%%%%%%%%%%%
\subsection{Include by Input}
\label{sec:input}

Including child documents by |\include| has some restrictions by design.
Most notably, the content of a child document always occupies
its own set of pages; pages cannot be shared between child documents.
Usually, this behaviour makes perfect sense
because each child document contain an essential part of the document.
However, in some situations it may be desirable to compose
a document from a collection of parts
without having mandatory page breaks between then.
For this case, the package
provides a mechanism to include parts
by |\input| which can also be processed individually.
However, by construction this mechanism
requires manual handling of the content to be output.

%%%%%%%%%%%%%%%%%%%%%%%%%%%%%%%%%%%%%%%%
\DescribeMacro{\ifchilddocmanual}
The main file should be prepared as usual, see \secref{sec:include}.
However, the document body must make a distinction
between processing of an individual part and of the main document, e.g.:
%
\begin{center}
\begin{tabular}{l}
|\ifchilddocmanual|\\
|\input{\childdocname}|\\
|\||else|\\
\textit{document body with }|\input{|\textit{part}|}|\\
|\||fi|
\end{tabular}
\end{center}
%
The conditional |\ifchilddocmanual| is true whenever
a part to be included by |\input| is being compiled,
and the name of the part is stored in |\childdocname|.

%%%%%%%%%%%%%%%%%%%%%%%%%%%%%%%%%%%%%%%%
\DescribeMacro{\childdocby}
Each part to be included by |\input| should start with:
%
\begin{center}
\begin{tabular}{l}
|\input{childdoc.def}|\\
|\childdocby{|\textit{main}|}|\\
\end{tabular}
\end{center}
%
The directive |\childdocby| is similar to |\childdocof|
described in \secref{sec:include},
but the subsequent selection of content must be done manually.
To that end, both |\ifchilddoc| and |\ifchilddocmanual|
will be true upon processing of a part,
and the name of the part is stored in |\childdocname|.
Note that |\jobname| will be set to the filename of the current part
so that each part receives an individual |.aux| file
that does not interfere with the |.aux| file(s) of the main document.
This behaviour can be altered by the alternative form
|\childdocby[*]{|\textit{main}|}| (with a non-empty optional argument)
which uses the |.aux| file of the main document
by setting |\jobname| to \textit{main}.

%%%%%%%%%%%%%%%%%%%%%%%%%%%%%%%%%%%%%%%%%%%%%%%%%%%%%%%%%%%%%%%%%%%%%%%%%%%%%%%%
\subsection{Driver Development}
\label{sec:driver}

The \textsf{childdoc} mechanism can also be use for the development
of definition files such as \LaTeX{} styles or classes.
This case differs from the above setup with multiple parts
included by |\include| in that no |\includeonly| should be invoked.
This can be achieved by starting the include file
(before |\ProvidesPackage|) with:
%
\begin{center}
\begin{tabular}{l}
|\input{childdoc.def}|\\
|\childdocforward{|\textit{main}|}|\\
\end{tabular}
\end{center}
%
or alternatively with:
%
\begin{center}
\begin{tabular}{l}
|\input{childdoc.def}|\\
|\childdocby{|\textit{main}|}|\\
\end{tabular}
\end{center}
%
Both forms have slightly different effects as described above.
The main file is prepared as usual, see \secref{sec:include}.

%%%%%%%%%%%%%%%%%%%%%%%%%%%%%%%%%%%%%%%%%%%%%%%%%%%%%%%%%%%%%%%%%%%%%%%%%%%%%%%%
\subsection{Legacy Detection}
\label{sec:detection}

The directive |\childdocmain| in the main file can detect
whether the complete document or merely a child is to be compiled
even without using the directive |\childdocof|.
This method is deprecated because it is less robust
and there is no compelling reason to use it;
it is merely provided for backward compatibility
and it may be removed in future versions.

If the detection mechanism is to be used,
it is mandatory to correctly specify
the filename of the main file as the argument of |\childdocmain|:
%
\begin{center}
\begin{tabular}{l}
|\input{childdoc.def}|\\
|\childdocmain{|\textit{main}|}|\\
\end{tabular}
\end{center}
%
If |\jobname| does not match the argument \textit{main} of |\childdocmain|,
it is assumed that |\jobname| points to the child file to be compiled.
When using |\childdocmain| with the main file specified as argument,
it suffices to start a child file
with just |\input{|\textit{main}|}|
without loading of the package and using |\childdocof|.
If instead all processing is done
with the appropriate \textsf{childdoc} directives,
the argument of \textit{main} of |\childdocmain| can be empty.

An alternative version of the command line processing described
in \secref{sec:commandline} using the detection mechanism reads:
%
\begin{center}
|... -jobname "|\textit{target}|" "|[\textit{flags}]%
[|\def\jobname{|\textit{dest}|}|]|\input{|\textit{main}|}"|
\end{center}

%%%%%%%%%%%%%%%%%%%%%%%%%%%%%%%%%%%%%%%%%%%%%%%%%%%%%%%%%%%%%%%%%%%%%%%%%%%%%%%%
\subsection{Manual Code}
\label{sec:manual}

In case one cannot be certain whether the definitions file |childdoc.def|
is installed on the target \TeX{} distribution
and one prefers not to ship it,
it is conceivable to paste a few relevant commands into the sources.

To that end, drop all statements |\input{childdoc.def}|
and perform the replacements as outlined below.
Instead of |\childdocmain{|\textit{main}|}| add the following code
to the top of the main file:
%
\begin{center}
\begin{tabular}{l}
|\||ifdefined\childdocname\endinput\||fi\newif\ifchilddoc|\\
|\edef\childdocname{\scantokens\expandafter{\jobname\noexpand}}|\\
|\def\childdocmain{|\textit{main}|}\||ifx\childdocmain\childdocname\||else|\\
|\childdoctrue\includeonly{\childdocname}\let\jobname\childdocmain\||fi|\\
\end{tabular}
\end{center}
%
Instead of |\childdocof{|\textit{main}|}| just include the main file
at the top of each child file:
%
\begin{center}
|\input{|\textit{main}|}|
\end{center}
%
A simple redirection |\childdocforward{|\textit{dest}|}| is achieved by:
%
\begin{center}
|\def\jobname{|\textit{dest}|}\input{\jobname}|
\end{center}
%
The redirection with prefix
|\childdocforwardprefix[|\textit{prefix}|]{|\textit{dest}|}|
is accomplished by:
%
\begin{center}
\begin{tabular}{l}
|{\edef\jobname{\scantokens\expandafter{\jobname\noexpand}}|\\
|\def\redirectjob |\textit{prefix}|#1~~~{\gdef\jobname{|\textit{dest}|#1}}|\\
|\expandafter\redirectjob\jobname~~~}\input{\jobname}|
\end{tabular}
\end{center}

In an alternative approach,
child documents can be compiled by a specific command line
without additional code or specific definitions:
%
\begin{center}
|... -jobname "|\textit{target}|" "|[\textit{flags}]%
|\includeonly{|\textit{dest}|}\input{|\textit{main}|}"|
\end{center}
%

%%%%%%%%%%%%%%%%%%%%%%%%%%%%%%%%%%%%%%%%%%%%%%%%%%%%%%%%%%%%%%%%%%%%%%%%%%%%%%%%
%%%%%%%%%%%%%%%%%%%%%%%%%%%%%%%%%%%%%%%%%%%%%%%%%%%%%%%%%%%%%%%%%%%%%%%%%%%%%%%%
\section{Information}

%%%%%%%%%%%%%%%%%%%%%%%%%%%%%%%%%%%%%%%%%%%%%%%%%%%%%%%%%%%%%%%%%%%%%%%%%%%%%%%%
\subsection{Copyright}

Copyright \copyright{} 2017--2018 Niklas Beisert

This work may be distributed and/or modified under the
conditions of the \LaTeX{} Project Public License, either version 1.3
of this license or (at your option) any later version.
The latest version of this license is in
  \url{http://www.latex-project.org/lppl.txt}
and version 1.3 or later is part of all distributions of \LaTeX{}
version 2005/12/01 or later.

This work has the LPPL maintenance status `maintained'.

The Current Maintainer of this work is Niklas Beisert.

This work consists of the files |README.txt|, |childdoc.ins| and |childdoc.dtx|
as well as the derived files |childdoc.def|, |cdocsamp.tex|
with |cdocsch1.tex|, |cdocsch2.tex|, |cdocspt3.tex|, |cdocspt4.tex|,
|cdocsdrf.tex|, |cdocsfn1.tex|, |cdocsfn2.tex|
as well as |childdoc.pdf|.

%%%%%%%%%%%%%%%%%%%%%%%%%%%%%%%%%%%%%%%%%%%%%%%%%%%%%%%%%%%%%%%%%%%%%%%%%%%%%%%%
\subsection{Files and Installation}

The package consists of the files:
%
\begin{center}
\begin{tabular}{ll}
    |README.txt|   & readme file \\
    |childdoc.ins| & installation file \\
    |childdoc.dtx| & source file \\
    |childdoc.def| & definition file \\
    |cdocsamp.tex| & sample main file \\
    |cdocsch1.tex| & sample include file \\
    |cdocsch2.tex| & sample include file \\
    |cdocspt3.tex| & sample part file \\
    |cdocspt4.tex| & sample part file \\
    |cdocsdrf.tex| & sample redirection file \\
    |cdocsfn1.tex| & sample redirection file \\
    |cdocsfn2.tex| & sample redirection file \\
    |childdoc.pdf| & manual
\end{tabular}
\end{center}
%
The distribution consists of the files
|README.txt|, |childdoc.ins| and |childdoc.dtx|.
%
\begin{itemize}
\item
Run (pdf)\LaTeX{} on |childdoc.dtx|
to compile the manual |childdoc.pdf| (this file).
\item
Run \LaTeX{} on |childdoc.ins| to create the definitions file |childdoc.def|
and the sample |cdocsamp.tex| with include files
|cdocsch1.tex|, |cdocsch2.tex|, |cdocspt3.tex|, |cdocspt4.tex|,
|cdocsdrf.tex|, |cdocsfn1.tex|, |cdocsfn2.tex|.
Then copy the file |childdoc.def| to an appropriate directory of your \LaTeX{}
distribution, e.g.\ \textit{texmf-root}|/tex/latex/childdoc|.
\end{itemize}

%%%%%%%%%%%%%%%%%%%%%%%%%%%%%%%%%%%%%%%%%%%%%%%%%%%%%%%%%%%%%%%%%%%%%%%%%%%%%%%%
\subsection{Related CTAN Packages}

There are several other packages which offer a similar functionality:
%
\begin{itemize}
\item
The packages
\href{http://ctan.org/pkg/docmute}{\textsf{docmute}},
\href{http://ctan.org/pkg/includex}{\textsf{includex}} and
\href{http://ctan.org/pkg/standalone}{\textsf{standalone}}
provide commands to include only the document body of
a child file thus allowing both files to be compiled individually.
\item
The packages \href{http://ctan.org/pkg/subdocs}{\textsf{subdocs}}
and \href{http://ctan.org/pkg/subfiles}{\textsf{subfiles}}
provide structures in which the main and child documents can be
encapsulated and allowing them to be compiled individually.
The inclusion mechanism is different from the conventional |\include|.
\item
The package \href{http://ctan.org/pkg/combine}{\textsf{combine}}
is an elaborate solution to combine several documents into one.
\end{itemize}
%
See also the CTAN topic \href{http://ctan.org/topic/subdocs}{\textsf{subdocs}}
for further related packages.
The present package differs from the above solutions in that
a document structure constructed with the conventional |\include| mechanism
just needs two extra commands at the top of every file
such that all constituent files can be compiled individually.

%%%%%%%%%%%%%%%%%%%%%%%%%%%%%%%%%%%%%%%%%%%%%%%%%%%%%%%%%%%%%%%%%%%%%%%%%%%%%%%%
%\subsection{Feature Suggestions}
%
%The following is a list of features which may be useful for future
%versions of this package:
%%
%\begin{itemize}
%\item
%\ldots
%\end{itemize}

%%%%%%%%%%%%%%%%%%%%%%%%%%%%%%%%%%%%%%%%%%%%%%%%%%%%%%%%%%%%%%%%%%%%%%%%%%%%%%%%
\subsection{Revision History}

%%%%%%%%%%%%%%%%%%%%%%%%%%%%%%%%%%%%%%%%
\paragraph{v2.0:} 2018/12/30

\begin{itemize}
\item
immediate forward processing
\item
added |\childdocby| mechanism
\item
manual restructured
\end{itemize}

%%%%%%%%%%%%%%%%%%%%%%%%%%%%%%%%%%%%%%%%
\paragraph{v1.6:} 2018/01/17

\begin{itemize}
\item
application for development of include files
\item
corrections to manual
\end{itemize}

%%%%%%%%%%%%%%%%%%%%%%%%%%%%%%%%%%%%%%%%
\paragraph{v1.5:} 2017/05/21

\begin{itemize}
\item
more complete structuring introduced
\item
|\childdocof| introduced
\item
|\childdoc| renamed to |\childdocmain|
\item
|\childredirect| renamed to |\childdocforward| and |\childdocforwardprefix|
and functionality expanded
\end{itemize}

%%%%%%%%%%%%%%%%%%%%%%%%%%%%%%%%%%%%%%%%
\paragraph{v1.0:} 2017/04/27

\begin{itemize}
\item
manual and install package
\item
first version published on CTAN
\end{itemize}

%%%%%%%%%%%%%%%%%%%%%%%%%%%%%%%%%%%%%%%%
\paragraph{v0.6:} 2017/04/26

\begin{itemize}
\item
redirection mechanism added
\end{itemize}

%%%%%%%%%%%%%%%%%%%%%%%%%%%%%%%%%%%%%%%%
\paragraph{v0.5:} 2017/04/26

\begin{itemize}
\item
functionality in definition file
\end{itemize}


%%%%%%%%%%%%%%%%%%%%%%%%%%%%%%%%%%%%%%%%%%%%%%%%%%%%%%%%%%%%%%%%%%%%%%%%%%%%%%%%
%%%%%%%%%%%%%%%%%%%%%%%%%%%%%%%%%%%%%%%%%%%%%%%%%%%%%%%%%%%%%%%%%%%%%%%%%%%%%%%%
%%%%%%%%%%%%%%%%%%%%%%%%%%%%%%%%%%%%%%%%%%%%%%%%%%%%%%%%%%%%%%%%%%%%%%%%%%%%%%%%
\appendix

\settowidth\MacroIndent{\rmfamily\scriptsize 000\ }

 \DocInput{childdoc.dtx}

\end{document}
%</driver>
% \fi
%
% %%%%%%%%%%%%%%%%%%%%%%%%%%%%%%%%%%%%%%%%%%%%%%%%%%%%%%%%%%%%%%%%%%%%%%%%%%%%%%
% %%%%%%%%%%%%%%%%%%%%%%%%%%%%%%%%%%%%%%%%%%%%%%%%%%%%%%%%%%%%%%%%%%%%%%%%%%%%%%
% \section{Sample}
%\iffalse
%<*samplemain>
%\fi
%
% The following presents a sample document
% with two chapters, two parts, a title page,
% a compile flag as well as three forwarding files to set the flag.
% It consists of eight |.tex| files:
% \begin{center}
% \begin{tabular}{ll}
% |cdocsamp.tex|&main file\\
% |cdocsch1.tex|&include file for chapter 1\\
% |cdocsch2.tex|&include file for chapter 2\\
% |cdocspt3.tex|&include file for part 3\\
% |cdocspt4.tex|&include file for part 4\\
% |cdocsdrf.tex|&forwarding file for main file in draft mode\\
% |cdocsfi1.tex|&forwarding file for final version of chapter 1\\
% |cdocsfi2.tex|&forwarding file for final version of chapter 2\\
% \end{tabular}
% \end{center}
% Each of the eight files can be compiled directly by the \LaTeX{} compiler.
%
% %%%%%%%%%%%%%%%%%%%%%%%%%%%%%%%%%%%%%%
% \paragraph{Main File.}
%
% The main file is called |cdocsamp.tex|.
%
% Load the \textsf{childdoc} definitions and
% declare the filename for the main document:
%    \begin{macrocode}
\input{childdoc.def}
\childdocmain{}
%    \end{macrocode}

% Optional override for |\version| flag:
%    \begin{macrocode}
%%\ifchilddoc\else\providecommand{\version}{draft}\fi
%    \end{macrocode}

% Define the default values for the |\version| flag
% (|final| for the main file and |draft| for childs):
%    \begin{macrocode}
\ifchilddoc
\providecommand{\version}{draft}
\else
\providecommand{\version}{final}
\fi
%    \end{macrocode}

% Load the standard document class:
%    \begin{macrocode}
\documentclass[12pt]{article}
%    \end{macrocode}

% Start the document body:
%    \begin{macrocode}
\begin{document}
%    \end{macrocode}

% Declare a title page.
% Print title, part of document being processed and version flag:
%    \begin{macrocode}
\addtocounter{page}{-1}
\begin{center}
{\LARGE\bfseries{}childdoc example\par}
\vspace{1cm}
\ifchilddoc
\ifchilddocmanual part\else chapter\fi:
`\childdocname' of `\childdocjob'\par
\else
main document: `\childdocjob'\par
\fi
version: \version\par
\end{center}
\newpage
%    \end{macrocode}

% Manually include selected file,
% otherwise process as usual:
%    \begin{macrocode}
\ifchilddocmanual
\section*{part `\childdocname'}
\input{\childdocname}
\else
%    \end{macrocode}

% Include the two chapters:
%    \begin{macrocode}
\include{cdocsch1}
\include{cdocsch2}
%    \end{macrocode}

% Include the two parts unless only chapters should be displayed:
%    \begin{macrocode}
\ifchilddoc\else
\section{part three}
\input{cdocspt3}
\section{part four}
\input{cdocspt4}
\fi
%    \end{macrocode}

% Process as usual until here:
%    \begin{macrocode}
\fi
%    \end{macrocode}

% End of document body:
%    \begin{macrocode}
\end{document}
%    \end{macrocode}
%\iffalse
%</samplemain>
%\fi
%
% %%%%%%%%%%%%%%%%%%%%%%%%%%%%%%%%%%%%%%
% \paragraph{Chapter Include Files.}
%
% The include files are called |cdocsch1.tex| and |cdocsch2.tex|.
%
%\iffalse
%<*samplechap1|samplechap2>
%\fi

% Optional override for |\version| flag:
%    \begin{macrocode}
%%\providecommand{\version}{final}
%    \end{macrocode}

% Include the main document:
%    \begin{macrocode}
\input{childdoc.def}
\childdocof{cdocsamp}
%    \end{macrocode}

%\iffalse
%</samplechap1|samplechap2>
%\fi
%
%\iffalse
%<*samplechap1>
%\fi
% Some text for chapter 1:
%    \begin{macrocode}
\section{one}
some text in chapter one
%    \end{macrocode}

%\iffalse
%</samplechap1>
%\fi
% Some text for chapter 2:
%\iffalse
%<*samplechap2>
%\fi
%    \begin{macrocode}
\section{two}
more text in chapter two
%    \end{macrocode}

%\iffalse
%</samplechap2>
%\fi
%
% %%%%%%%%%%%%%%%%%%%%%%%%%%%%%%%%%%%%%%
% \paragraph{Part Include Files.}
%
% The include files are called |cdocspt3.tex| and |cdocspt4.tex|.
%
%\iffalse
%<*samplepart3|samplepart4>
%\fi

% Optional override for |\version| flag:
%    \begin{macrocode}
%%\providecommand{\version}{final}
%    \end{macrocode}

% Include the main document:
%    \begin{macrocode}
\input{childdoc.def}
\childdocby{cdocsamp}
%    \end{macrocode}

%\iffalse
%</samplepart3|samplepart4>
%\fi
%
%\iffalse
%<*samplepart3>
%\fi
% Some text for part 3:
%    \begin{macrocode}
some text in part three
%    \end{macrocode}

%\iffalse
%</samplepart3>
%\fi
% Some text for part 4:
%\iffalse
%<*samplepart4>
%\fi
%    \begin{macrocode}
more text in part four
%    \end{macrocode}

%\iffalse
%</samplepart4>
%\fi
%
% %%%%%%%%%%%%%%%%%%%%%%%%%%%%%%%%%%%%%%
% \paragraph{Forwarding for a Complete Draft.}
%
% The following forwarding file |cdocsdrf.tex|
% compiles the main document in draft mode:
%\iffalse
%<*sampledraft>
%\fi
%    \begin{macrocode}
\def\version{draft}
\input{childdoc.def}
\childdocforward{cdocsamp}
%    \end{macrocode}

%\iffalse
%</sampledraft>
%\fi
%
% %%%%%%%%%%%%%%%%%%%%%%%%%%%%%%%%%%%%%%
% \paragraph{Forwarding for Final Version of the Chapters.}
%
% The following forwarding files |cdocsfn1.tex| and |cdocsfn2.tex|
% (with identical content)
% compile the final versions of the child documents
% |cdocsch1.tex| and |cdocsch2.tex|, respectively:
%\iffalse
%<*samplefinal>
%\fi
%    \begin{macrocode}
\def\version{final}
\input{childdoc.def}
\childdocforwardprefix[cdocsamp]{cdocsfn}{cdocsch}
%    \end{macrocode}

%\iffalse
%</samplefinal>
%\fi
%
% %%%%%%%%%%%%%%%%%%%%%%%%%%%%%%%%%%%%%%
% \paragraph{Command Line Processing.}
%
% The following three command lines generate the output files
% |cdocscld|, |cdocscl1| and |cdocscl2|
% which should be identical to
% |cdocsdrf|, |cdocsch1| and |cdocsfn2|, respectively:
% \begin{center}
% \begin{tabular}{l}
% |latex -jobname cdocscld \|\\
% |  "\def\version{draft}\input{childdoc.def}\childdocforward{cdocsamp}"|\\
% |latex -jobname cdocscl1 \|\\
% |  "\input{childdoc.def}\childdocforward[cdocsamp]{cdocsch1}"|\\
% |latex -jobname cdocscl2 \|\\
% |  "\def\version{final}\input{childdoc.def}\childdocforward{cdocsch2}"|
% \end{tabular}
% \end{center}
% Note that the trailing backslash on each first line
% merely continues the input to the second line
% (for convenient cut ant paste).
% Furthermore, the command |latex| can be replaced by any
% of its alternative versions such as |pdflatex|.
%
% %%%%%%%%%%%%%%%%%%%%%%%%%%%%%%%%%%%%%%%%%%%%%%%%%%%%%%%%%%%%%%%%%%%%%%%%%%%%%%
% %%%%%%%%%%%%%%%%%%%%%%%%%%%%%%%%%%%%%%%%%%%%%%%%%%%%%%%%%%%%%%%%%%%%%%%%%%%%%%
% \section{Implementation}
%\iffalse
%<*package>
%\fi
%
% This section describes the definitions file |childdoc.def|.

% The definitions cannot be loaded using |\usepackage| or |\RequirePackage|
% which has a mechanism to prevent loading a style file more than once.
% When loading the definitions by means of |\input|
% multiple instances have to be prevented manually:
%\iffalse
%This code needs to be before the `\ProvidesFile' directive
%which is defined at the beginning of this file.
%Therefore it is also placed there and commented out here.
%</package>
%<*discard>
%\fi
%    \begin{macrocode}
\ifdefined\childdocmain\endinput\fi
%    \end{macrocode}
%\iffalse
%</discard>
%<*package>
%\fi
%
% \macro{\ifchilddoc}
% \macro{\ifchilddocmanual}
% The conditional |\ifchilddoc| tells whether a
% child (true) or main (false) document is being compiled.
% The conditional |\ifchilddocmanual| tells whether
% the |\includeonly| mechanism is used (false) or
% the selection of child files must be performed manually (true).
% The definitions initialise to false:
%    \begin{macrocode}
\newif\ifchilddoc
\newif\ifchilddocmanual
%    \end{macrocode}

% \macro{\childdocname}
% \macro{\childdocjob}
% The macro |\childdocname| stores the name of the main document
% to be compiled. The macro |\childdocjob| stores the name of
% the document on which the \LaTeX{} compiler was originally invoked.
% The content of |\jobname| cannot be compared
% to filenames specified in the source due to different catcodes.
% The following code rescans |\jobname|, stores the result
% in |\childdocname| and saves a copy in |\childdocjob|:
%    \begin{macrocode}
\edef\childdocname{\scantokens\expandafter{\jobname\noexpand}}
\let\childdocjob\childdocname
%    \end{macrocode}

% \macro{\childdocdisable}
% The macro |\childdocdisable| prevents the main file
% from being processed more than once.
% At this stage, the main document command |\childdocmain|
% is assumed to be called once again where it should do nothing.
% Any subsequent call to it should prevent
% a secondary processing of the main document
% It overwrites the forwarding commands
% |\childdocof| and |\childdocforward|
% with empty macros to prevent further inclusions of the main document:
%    \begin{macrocode}
\newcommand{\childdocdisable}
{
  \renewcommand{\childdocmain}[1]{\renewcommand{\childdocmain}[1]{\endinput}}
  \renewcommand{\childdocof}[1]{}
  \renewcommand{\childdocby}[2][]{}
  \renewcommand{\childdocforward}[2][]{}
  \renewcommand{\childdocdisable}{}
}
%    \end{macrocode}

% \macro{\childdocmain}
% The macro |\childdocmain| is to be called at the top of the main file
% with nothing or the main filename (without extension) as argument.
% First, it breaks loops.
% If the argument is not empty and does not match |\childdocname|
% (which is set by the first inclusion of |childdoc.def|),
% |\ifchilddoc| is set to true, |\includeonly| is applied to the child file
% and |\jobname| is set to the main file
% (for proper handling of |.aux| files):
%    \begin{macrocode}
\newcommand{\childdocmain}[1]
{
  \childdocdisable\childdocmain{}
  \if?#1?\else
    \begingroup
      \def\childdoctmp{#1}
      \ifx\childdoctmp\childdocname
        \def\childdoctmp{}
      \else
        \def\childdoctmp
        {
          \childdoctrue
          \includeonly{\childdocname}
          \def\childdocjob{#1}
          \def\jobname{#1}
        }
      \fi
      \expandafter
    \endgroup
    \childdoctmp
  \fi
}
%    \end{macrocode}

% \macro{\childdocof}
% The command |\childdocof| redirects
% compilation to the main file |#1|.
%    \begin{macrocode}
\newcommand{\childdocof}[1]
{
  \childdocdisable
  \childdoctrue
  \includeonly{\childdocname}
  \def\jobname{#1}
  \def\childdocjob{#1}
  \input{#1}
}
%    \end{macrocode}

% \macro{\childdocby}
% The command |\childdocby| ....
%    \begin{macrocode}
\newcommand{\childdocby}[2][]
{
  \childdocdisable
  \childdoctrue
  \childdocmanualtrue
  \if?#1?\else
    \def\jobname{#2}
  \fi
  \def\childdocjob{#2}
  \input{#2}
  \endinput
}
%    \end{macrocode}

% \macro{\childdocforward}
% The command |\childdocforward| redirects
% compilation to the main file or
% (if the optional argument is given) a child file.
% Parameters are set as if the main file
% or a child file starting with |\childdocof| was compiled.
% Then compilation is handed over to the main file:
%    \begin{macrocode}
\newcommand{\childdocforward}[2][]
{
  \begingroup
    \if?#1?
      \def\childdoctmp
      {
        \def\childdocname{#2}
        \def\childdocjob{#2}
        \def\jobname{#2}
        \input{#2}
        \endinput
      }
    \else
      \def\childdoctmp
      {
        \childdocdisable
        \def\childdocname{#2}
        \childdoctrue
        \includeonly{#2}
        \def\childdocjob{#1}
        \def\jobname{#1}
        \input{#1}
        \endinput
      }
    \fi
    \expandafter
  \endgroup
  \childdoctmp
}
%    \end{macrocode}

% \macro{\childdocforwardprefix}
% The command |\childdocforwardprefix| redirects
% compilation to the main or a child file by means of a pattern.
% The prefix |#1| in the current filename is replaced by |#2|
% and the suffix of the current filename is kept
% (it is assumed that the filename does not contain the substring `|~~~|'
% which is used as a delimiter).
% Compilation is handed over to the new file by |\childdocforward|:
%    \begin{macrocode}
\newcommand{\childdocforwardprefix}[3][]
{
  \begingroup
    \def\childdocextract #2##1~~~{\def\childdoctmp{\childdocforward[#1]{#3##1}}}
    \expandafter\childdocextract\childdocname~~~
    \expandafter
  \endgroup
  \childdoctmp
}
%    \end{macrocode}

% \macro{\childdoc}
% The deprecated macro |\childdoc| is a legacy version of |\childdocmain|:
%    \begin{macrocode}
\newcommand{\childdoc}{\childdocmain}
%    \end{macrocode}

% \macro{\childdocredirect}
% The deprecated macro |\childdocredirect| is a legacy version
% of |\childdocforward| and |\childdocforwardprefix|:
%    \begin{macrocode}
\newcommand{\childdocredirect}[2][]
{
  \begingroup
    \if?#1?
      \def\childdoctmp{\childdocforward{#2}}
    \else
      \def\childdoctmp{\childdocforwardprefix{#1}{#2}}
    \fi
    \expandafter
  \endgroup
  \childdoctmp
}
%    \end{macrocode}

%\iffalse
%</package>
%\fi
%
\endinput
|\\
|\childdocforward{|\textit{main}|}|
\end{tabular}
\end{center}
%
Likewise, the following files |final|\textit{nn}|.tex|
compile the final version of the child document
|child|\textit{nn}|.tex|:
%
\begin{center}
\begin{tabular}{l}
|\def\version{final}|\\
|% \iffalse
%
% childdoc.dtx Copyright (C) 2017-2018 Niklas Beisert
%
% This work may be distributed and/or modified under the
% conditions of the LaTeX Project Public License, either version 1.3
% of this license or (at your option) any later version.
% The latest version of this license is in
%   http://www.latex-project.org/lppl.txt
% and version 1.3 or later is part of all distributions of LaTeX
% version 2005/12/01 or later.
%
% This work has the LPPL maintenance status `maintained'.
%
% The Current Maintainer of this work is Niklas Beisert.
%
% This work consists of the files childdoc.dtx and childdoc.ins
% and the derived files childdoc.def and cdocsamp.tex with
% cdocsch1.tex, cdocsch2.tex, cdocsdrf.tex, cdocsfn1.tex, cdocsfn2.tex.
%
%<package>\ifdefined\childdocmain\endinput\fi
%<package>\ProvidesFile{childdoc.def}[2018/12/30 v2.0 child document driver]
%<samplemain>\ProvidesFile{cdocsamp.tex}[2018/12/30 v2.0 sample for childdoc]
%<*driver>
%\ProvidesFile{childdoc.drv}[2018/12/30 v2.0 childdoc reference manual file]
\PassOptionsToClass{10pt,a4paper}{article}
\documentclass{ltxdoc}

\usepackage[margin=35mm]{geometry}
\usepackage{hyperref}
\usepackage{hyperxmp}
\usepackage[usenames]{color}

\hypersetup{colorlinks=true}
\hypersetup{pdfstartview=FitH}
\hypersetup{pdfpagemode=UseNone}
\hypersetup{pdfsource={}}
\hypersetup{pdflang={en-UK}}
\hypersetup{pdfcopyright={Copyright 2017-2018 Niklas Beisert.
  This work may be distributed and/or modified under the
  conditions of the LaTeX Project Public License, either version 1.3
  of this license or (at your option) any later version.}}
\hypersetup{pdflicenseurl={http://www.latex-project.org/lppl.txt}}
\hypersetup{pdfcontactaddress={ETH Zurich, ITP, HIT K,
  Wolfgang-Pauli-Strasse 27}}
\hypersetup{pdfcontactpostcode={8093}}
\hypersetup{pdfcontactcity={Zurich}}
\hypersetup{pdfcontactcountry={Switzerland}}
\hypersetup{pdfcontactemail={nbeisert@itp.phys.ethz.ch}}
\hypersetup{pdfcontacturl={http://people.phys.ethz.ch/\xmptilde nbeisert/}}

\newcommand{\secref}[1]{\hyperref[#1]{section \ref*{#1}}}

\parskip1ex
\parindent0pt
\let\olditemize\itemize
\def\itemize{\olditemize\parskip0pt}

\begin{document}

\title{The \textsf{childdoc} Package}
\hypersetup{pdftitle={The childdoc Package}}
\author{Niklas Beisert\\[2ex]
  Institut f\"ur Theoretische Physik\\
  Eidgen\"ossische Technische Hochschule Z\"urich\\
  Wolfgang-Pauli-Strasse 27, 8093 Z\"urich, Switzerland\\[1ex]
  \href{mailto:nbeisert@itp.phys.ethz.ch}
  {\texttt{nbeisert@itp.phys.ethz.ch}}}
\hypersetup{pdfauthor={Niklas Beisert}}
\hypersetup{pdfsubject={Manual for the LaTeX2e Package childdoc}}
\date{30 December 2018, \textsf{v2.0}}
\maketitle

\begin{abstract}\noindent
\textsf{childdoc} is a \LaTeXe{} package
that enables the direct compilation
of document sections included by |\include|
to individual files.
\end{abstract}

\begingroup
\parskip0ex
\tableofcontents
\endgroup

%%%%%%%%%%%%%%%%%%%%%%%%%%%%%%%%%%%%%%%%%%%%%%%%%%%%%%%%%%%%%%%%%%%%%%%%%%%%%%%%
%%%%%%%%%%%%%%%%%%%%%%%%%%%%%%%%%%%%%%%%%%%%%%%%%%%%%%%%%%%%%%%%%%%%%%%%%%%%%%%%
\section{Introduction}

\LaTeX{} provides a mechanism to structure a large document (such as a book)
into a main file and several child files (containing the chapters)
using the |\include| command.
This mechanism is beneficial for documents
which span hundreds of pages in order to
make the source file(s) more manageable.
Moreover, compilation can be restricted to
selected child files by means of the |\includeonly| command.
The latter feature can be used to reduce the compilation time while editing
(this was significantly more useful in the earlier days of \LaTeX{})
or to generate a smaller document which is easier to navigate.
Another application of |\includeonly| is to generate
documents consisting of selected parts of the complete document.

However, there are a few drawbacks of the plain |\include| mechanism:
\begin{itemize}
\item
The child files cannot be compiled on their own,
they can only be compiled via the main file.
A naive editing environment
(such as a text editor with an option
to have the current file processed by \LaTeX)
may require one to switch to the main file before compiling;
attempting to compile the child file produces errors.
\item
The main file must be modified (each time)
to adjust the |\includeonly| command
to the present needs. This easily leaves the main file in a messy state.
\item
The generated document will always carry the filename
of the main document. This is inconvenient if
several child files are to be compiled and
to be kept for distribution.
\end{itemize}

The present package provides a simple interface
to make child files individually compilable by \LaTeX{}.
Compiling a child file then has the same effect as compiling
the main file with an |\includeonly| command
to select the appropriate child.
Moreover the generated document will carry the name of the child
rather than the main file.
This resolves all three above issues.

This feature is meant to make the editing of books,
thesis documents and lecture notes somewhat more convenient.
However, the package can also be used efficiently for
composing a series of documents (such as exercise sheets)
which are typically distributed individually.
It then assists the author in generating the individual documents
(potentially in different versions)
as well as a document containing the collected series.
Another application is in developing style files
or other kinds of included material
where compilation of the style file could redirect
to a sample or test file.

%%%%%%%%%%%%%%%%%%%%%%%%%%%%%%%%%%%%%%%%%%%%%%%%%%%%%%%%%%%%%%%%%%%%%%%%%%%%%%%%
%%%%%%%%%%%%%%%%%%%%%%%%%%%%%%%%%%%%%%%%%%%%%%%%%%%%%%%%%%%%%%%%%%%%%%%%%%%%%%%%
\section{Usage}

First of all, the package \textsf{childdoc} is \emph{not} a standard
\LaTeXe{} |.sty| style file! Therefore it needs to be invoked in
a non-standard way.

%%%%%%%%%%%%%%%%%%%%%%%%%%%%%%%%%%%%%%%%%%%%%%%%%%%%%%%%%%%%%%%%%%%%%%%%%%%%%%%%
\subsection{Included Files}
\label{sec:include}

%%%%%%%%%%%%%%%%%%%%%%%%%%%%%%%%%%%%%%%%
\DescribeMacro{\childdocmain}
To use the package, add the commands
\begin{center}
\begin{tabular}{l}
|\input{childdoc.def}|\\
|\childdocmain{}|\\
\end{tabular}
\end{center}
at the very top of the main \LaTeX{} file,
in particular \emph{before} the |\documentclass| statement!
The argument of |\childdocmain| should be left empty
(but it must be present).

%%%%%%%%%%%%%%%%%%%%%%%%%%%%%%%%%%%%%%%%
\DescribeMacro{\childdocof}
Furthermore, add the commands
\begin{center}
\begin{tabular}{l}
|\input{childdoc.def}|\\
|\childdocof{|\textit{main}|}|\\
\end{tabular}
\end{center}
at the top of every child file \textit{child}
which is included by |\include{|\textit{child}|}|
from within the main file
(or at least for those files to be compiled individually).
The argument \textit{main} must be the filename of the main file.

There are a couple of
considerations in setting up the main and child documents:

%%%%%%%%%%%%%%%%%%%%%%%%%%%%%%%%%%%%%%%%
\paragraph{Restrictions.}

Please note the following restrictions:
\begin{itemize}
\item
|\childdocmain| must be called with one argument \textit{main}
to ensure compatibility with earlier version of the package.
It must either be empty (|\childdocmain{}|)
or precisely match the filename of the main file in which it is specified.
See \secref{sec:detection} for further information.
\item
The filename \textit{main} must be specified without the |.tex| extension.
\item
The filename \textit{main} is case sensitive
(even in case-insensitive file systems)
due to internal string comparison.
\item
The argument \textit{main} should be fully expanded, it cannot be a macro.
\item
Subdirectories and special characters should be avoided in filenames.
\item
The command |\childdocmain{|\textit{main}|}| must be followed by a whitespace.
It should not be followed immediately by another command
or by a comment mark `|%|'.
This is because the \TeX{} parser reads the token immediately following
the argument of |\childdocmain| and puts it
at the beginning of every child section;
however, a white\-space is ignored.
\end{itemize}

%%%%%%%%%%%%%%%%%%%%%%%%%%%%%%%%%%%%%%%%
\paragraph{Content of Main File.}

It is advisable to place all content in the child files included by |\include|.
Any output contained in the main file will appear in all child documents
unless suppressed manually;
it cannot be suppressed automatically by the |\includeonly| directive
and thus should normally be avoided.
A method to include some content in the main file
by means of conditional processing is described in \secref{sec:conditional}.

%%%%%%%%%%%%%%%%%%%%%%%%%%%%%%%%%%%%%%%%
\paragraph{Page Numbering.}

When only a part of the document is compiled,
the appropriate numbering of pages
(as well as other status parameters)
is determined from the |.aux| files.
The latter contain information from previous passes.
However this information needs to propagate through
all intermediate child documents.
Therefore the page numbering in child documents may well
be inconsistent until the complete document is compiled at least once.

A useful (if unconventional) way to always ensure a consistent
page numbering is to restart the numbering in each child document
and denote the pages by `\textit{child}|.|\textit{page}'
where \textit{child} represents the chapter/section number of the child file.
This can be achieved by the command
|\numberwithin{page}{|\textit{child}|}|
of the \textsf{amsmath} package
where \textit{child} can be |chapter| or |section|
depending on the chosen structuring.
Alternatively, one can modify the macro |\thepage| appropriately
and reset the counter |page| at the start of each child file.

%%%%%%%%%%%%%%%%%%%%%%%%%%%%%%%%%%%%%%%%%%%%%%%%%%%%%%%%%%%%%%%%%%%%%%%%%%%%%%%%
\subsection{Conditional Processing}
\label{sec:conditional}

The package provides a mechanism to compile different versions
of a document. To customise the versions further some conditional processing
can come in handy to distinguish which version is being compiled.
The package provides two macros to describe the compilation context:

%%%%%%%%%%%%%%%%%%%%%%%%%%%%%%%%%%%%%%%%
\DescribeMacro{\ifchilddoc}
The conditional |\ifchilddoc| distinguishes between the compilation of
child documents and the main document:
%
\begin{center}
|\ifchilddoc |\textit{child-code}| |[|\||else |\textit{main-code}]| \||fi|
\end{center}

%%%%%%%%%%%%%%%%%%%%%%%%%%%%%%%%%%%%%%%%
\DescribeMacro{\childdocname}
\DescribeMacro{\childdocjob}
The macro |\childdocname| contains the filename (without extension)
of the main or child file being processed.
Note that |\childdocjob| will always contain the name of the main file.

%%%%%%%%%%%%%%%%%%%%%%%%%%%%%%%%%%%%%%%%
\paragraph{Title Page.}

Conditional processing can be used to include a title or banner page
in the main document when proper precautions are taken.
Importantly, the code in the main file should ensure that the page counter
(as well as other status parameters which are stored in the |.aux| files)
takes the same value after the conditional processing.
Otherwise the page numbers may take divergent values
depending on which part is compiled.

For example, a title page could be declared by:
%
\begin{center}
\begin{tabular}{l}
|\ifchilddoc\||else|\\
|\addtocounter{page}{-1}|\\
\textit{code for title page}\\
|\newpage|\\
|\||fi|
\end{tabular}
\end{center}
%
A banner page for the child documents can be generated by:
%
\begin{center}
\begin{tabular}{l}
|\ifchilddoc|\\
|\addtocounter{page}{-1}|\\
\textit{code for banner page}\\
|\newpage|\\
|\||fi|
\end{tabular}
\end{center}
%
Here one could write a message such as:
\begin{center}
|This is the part \childdocname{} of \childdocjob{}.|
\end{center}

%%%%%%%%%%%%%%%%%%%%%%%%%%%%%%%%%%%%%%%%%%%%%%%%%%%%%%%%%%%%%%%%%%%%%%%%%%%%%%%%
\subsection{Flags}
\label{sec:flags}

The package makes it easy to generate different versions
of the main or child documents.
To this end compilation flags can be defined
and assigned different default values.
They will be particularly useful in conjunction
with the forwarding mechanism described in \secref{sec:forward}.

For example, it may be useful to have a flag |\version|
which can be set to |draft| or |final|.
The document source will contain some conditional code
depending on the value of |\version|.
Suppose further, the flag should default to |final| for the main file
and to |draft| for child files
which is a natural assignment for editing the document.
This is achieved by placing the following code
in the preamble of the main document
(below the |\childdocmain| directive):
%
\begin{center}
\begin{tabular}{l}
|\ifchilddoc|\\
|\providecommand{\version}{draft}|\\
|\||else|\\
|\providecommand{\version}{final}|\\
|\||fi|
\end{tabular}
\end{center}
%
The definition by |\providecommand| makes sure
that previous definitions are not overwritten.
Further statements |\providecommand{\version}{...}|
can thus be added before the above code to override it.

For the main file, one might add a line
(between |\childdocmain| and the above block)
%
\begin{center}
|%\ifchilddoc\||else\providecommand{\version}{draft}\||fi|
\end{center}
%
which can be uncommented to produce a draft version.
Likewise one can add a line to the very top of a child file
(above the |\childdocof{|\textit{main}|}| directive)
%
\begin{center}
|%\providecommand{\version}{final}|
\end{center}
%
which can be uncommented to produce the final version of this child document.

%%%%%%%%%%%%%%%%%%%%%%%%%%%%%%%%%%%%%%%%%%%%%%%%%%%%%%%%%%%%%%%%%%%%%%%%%%%%%%%%
\subsection{Forwarding}
\label{sec:forward}

Different versions of the main or child documents
using compilation flags as described in \secref{sec:flags}
can be (permanently) stored in different files
for convenient compilation, viewing and distribution.
To this end, the package defines a command
to pass on compilation to a different file:

%%%%%%%%%%%%%%%%%%%%%%%%%%%%%%%%%%%%%%%%
\DescribeMacro{\childdocforward}
The command |\childdocforward| redirects processing to
another source file:
%
\begin{center}
\begin{tabular}{l}
|\input{childdoc.def}|\\
|\childdocforward[|\textit{main}|]{|\textit{dest}|}|\\
\end{tabular}
\end{center}
%
The argument \textit{dest} is the destination file
(without extension).
It should be the main file or one of the child files.
Note that further \textsf{childdoc} directives
such as |\childdocof| and |\childdocforward|
in the indicated file will be processed in this form.
The optional argument \textit{main}
passes on directly to the main file \textit{main}
while pretending to compile the child \textit{dest}.
This form behaves as if \textit{dest}
issues |\childdocof{|\textit{main}|}| right away,
and no further \textsf{childdoc} directives will be processed.

%%%%%%%%%%%%%%%%%%%%%%%%%%%%%%%%%%%%%%%%
\DescribeMacro{\...prefix}
In the alternative form |\childdocforwardprefix|,
%
\begin{center}
\begin{tabular}{l}
|\input{childdoc.def}|\\
|\childdocforwardprefix[|\textit{main}|]{|\textit{prefix}|}{|\textit{dest}|}|
\end{tabular}
\end{center}
%
the destination file is determined by a pattern
depending on the current file:
To make this work, the current file must be called
`{\textit{prefix}\hspace{0.2em}\textit{suffix}}'
with \textit{prefix} matching precisely the argument.
Processing is then passed on to the file
`{\textit{dest}\hspace{0.2em}\textit{suffix}}'.
Surely, the same effect is achieved by
directly specifying the
argument `{\textit{dest}\hspace{0.2em}\textit{suffix}}'
in the first form.
However, that requires to set up a different file
for each child. With the alternative form of the command
all these files can have exactly the same content
which simplifies setting them up and maintaining them.

For example, the following file |draft.tex|
with a compilation flag |\version| as described in \secref{sec:flags}
compiles the main document as a draft:
%
\begin{center}
\begin{tabular}{l}
|\def\version{draft}|\\
|\input{childdoc.def}|\\
|\childdocforward{|\textit{main}|}|
\end{tabular}
\end{center}
%
Likewise, the following files |final|\textit{nn}|.tex|
compile the final version of the child document
|child|\textit{nn}|.tex|:
%
\begin{center}
\begin{tabular}{l}
|\def\version{final}|\\
|\input{childdoc.def}|\\
|\childdocforwardprefix{final}{child}|
\end{tabular}
\end{center}
%

Note that when several versions of a main file and/or of each child file
are to be generated, it may be convenient to set up a |Makefile| or
shell script to automatise the process.

%%%%%%%%%%%%%%%%%%%%%%%%%%%%%%%%%%%%%%%%%%%%%%%%%%%%%%%%%%%%%%%%%%%%%%%%%%%%%%%%
\subsection{Command Line Processing}
\label{sec:commandline}

The effect of redirection files can also be achieved by invoking
the \LaTeX{} compiler with a more elaborate command line.
Most conveniently this should be done as part
of a shell script or a |Makefile|.

When using \textsf{childdoc} in the main file, the following
command lines effectively perform a redirection
(note that depending on the shell being used,
backslashes may have to be doubled: `|\|' $\to$ `|\\|'):
%
\begin{center}
|... -jobname "|\textit{target}|" |\\|"|[\textit{flags}]%
|\input{childdoc.def}\childdocforward[|\textit{main}|]{|\textit{dest}|}"|
\end{center}
%
Here \textit{target} is the name of the output file,
\textit{main} is the name of the main file
and \textit{dest} is the name of the main or child file to be processed
(all filenames without extensions).
The optional argument \textit{main} can be omitted
if \textit{main} matches \textit{dest}.
Optionally, compilation \textit{flags} can be defined via |\def| commands.
This command line makes the \TeX{} engine believe
it is compiling the file \textit{target}
whose content is specified as the latter parameter.
The provided code then forwards the processing to
\textit{main} or \textit{dest} as described in \secref{sec:forward}.

%%%%%%%%%%%%%%%%%%%%%%%%%%%%%%%%%%%%%%%%%%%%%%%%%%%%%%%%%%%%%%%%%%%%%%%%%%%%%%%%
\subsection{Include by Input}
\label{sec:input}

Including child documents by |\include| has some restrictions by design.
Most notably, the content of a child document always occupies
its own set of pages; pages cannot be shared between child documents.
Usually, this behaviour makes perfect sense
because each child document contain an essential part of the document.
However, in some situations it may be desirable to compose
a document from a collection of parts
without having mandatory page breaks between then.
For this case, the package
provides a mechanism to include parts
by |\input| which can also be processed individually.
However, by construction this mechanism
requires manual handling of the content to be output.

%%%%%%%%%%%%%%%%%%%%%%%%%%%%%%%%%%%%%%%%
\DescribeMacro{\ifchilddocmanual}
The main file should be prepared as usual, see \secref{sec:include}.
However, the document body must make a distinction
between processing of an individual part and of the main document, e.g.:
%
\begin{center}
\begin{tabular}{l}
|\ifchilddocmanual|\\
|\input{\childdocname}|\\
|\||else|\\
\textit{document body with }|\input{|\textit{part}|}|\\
|\||fi|
\end{tabular}
\end{center}
%
The conditional |\ifchilddocmanual| is true whenever
a part to be included by |\input| is being compiled,
and the name of the part is stored in |\childdocname|.

%%%%%%%%%%%%%%%%%%%%%%%%%%%%%%%%%%%%%%%%
\DescribeMacro{\childdocby}
Each part to be included by |\input| should start with:
%
\begin{center}
\begin{tabular}{l}
|\input{childdoc.def}|\\
|\childdocby{|\textit{main}|}|\\
\end{tabular}
\end{center}
%
The directive |\childdocby| is similar to |\childdocof|
described in \secref{sec:include},
but the subsequent selection of content must be done manually.
To that end, both |\ifchilddoc| and |\ifchilddocmanual|
will be true upon processing of a part,
and the name of the part is stored in |\childdocname|.
Note that |\jobname| will be set to the filename of the current part
so that each part receives an individual |.aux| file
that does not interfere with the |.aux| file(s) of the main document.
This behaviour can be altered by the alternative form
|\childdocby[*]{|\textit{main}|}| (with a non-empty optional argument)
which uses the |.aux| file of the main document
by setting |\jobname| to \textit{main}.

%%%%%%%%%%%%%%%%%%%%%%%%%%%%%%%%%%%%%%%%%%%%%%%%%%%%%%%%%%%%%%%%%%%%%%%%%%%%%%%%
\subsection{Driver Development}
\label{sec:driver}

The \textsf{childdoc} mechanism can also be use for the development
of definition files such as \LaTeX{} styles or classes.
This case differs from the above setup with multiple parts
included by |\include| in that no |\includeonly| should be invoked.
This can be achieved by starting the include file
(before |\ProvidesPackage|) with:
%
\begin{center}
\begin{tabular}{l}
|\input{childdoc.def}|\\
|\childdocforward{|\textit{main}|}|\\
\end{tabular}
\end{center}
%
or alternatively with:
%
\begin{center}
\begin{tabular}{l}
|\input{childdoc.def}|\\
|\childdocby{|\textit{main}|}|\\
\end{tabular}
\end{center}
%
Both forms have slightly different effects as described above.
The main file is prepared as usual, see \secref{sec:include}.

%%%%%%%%%%%%%%%%%%%%%%%%%%%%%%%%%%%%%%%%%%%%%%%%%%%%%%%%%%%%%%%%%%%%%%%%%%%%%%%%
\subsection{Legacy Detection}
\label{sec:detection}

The directive |\childdocmain| in the main file can detect
whether the complete document or merely a child is to be compiled
even without using the directive |\childdocof|.
This method is deprecated because it is less robust
and there is no compelling reason to use it;
it is merely provided for backward compatibility
and it may be removed in future versions.

If the detection mechanism is to be used,
it is mandatory to correctly specify
the filename of the main file as the argument of |\childdocmain|:
%
\begin{center}
\begin{tabular}{l}
|\input{childdoc.def}|\\
|\childdocmain{|\textit{main}|}|\\
\end{tabular}
\end{center}
%
If |\jobname| does not match the argument \textit{main} of |\childdocmain|,
it is assumed that |\jobname| points to the child file to be compiled.
When using |\childdocmain| with the main file specified as argument,
it suffices to start a child file
with just |\input{|\textit{main}|}|
without loading of the package and using |\childdocof|.
If instead all processing is done
with the appropriate \textsf{childdoc} directives,
the argument of \textit{main} of |\childdocmain| can be empty.

An alternative version of the command line processing described
in \secref{sec:commandline} using the detection mechanism reads:
%
\begin{center}
|... -jobname "|\textit{target}|" "|[\textit{flags}]%
[|\def\jobname{|\textit{dest}|}|]|\input{|\textit{main}|}"|
\end{center}

%%%%%%%%%%%%%%%%%%%%%%%%%%%%%%%%%%%%%%%%%%%%%%%%%%%%%%%%%%%%%%%%%%%%%%%%%%%%%%%%
\subsection{Manual Code}
\label{sec:manual}

In case one cannot be certain whether the definitions file |childdoc.def|
is installed on the target \TeX{} distribution
and one prefers not to ship it,
it is conceivable to paste a few relevant commands into the sources.

To that end, drop all statements |\input{childdoc.def}|
and perform the replacements as outlined below.
Instead of |\childdocmain{|\textit{main}|}| add the following code
to the top of the main file:
%
\begin{center}
\begin{tabular}{l}
|\||ifdefined\childdocname\endinput\||fi\newif\ifchilddoc|\\
|\edef\childdocname{\scantokens\expandafter{\jobname\noexpand}}|\\
|\def\childdocmain{|\textit{main}|}\||ifx\childdocmain\childdocname\||else|\\
|\childdoctrue\includeonly{\childdocname}\let\jobname\childdocmain\||fi|\\
\end{tabular}
\end{center}
%
Instead of |\childdocof{|\textit{main}|}| just include the main file
at the top of each child file:
%
\begin{center}
|\input{|\textit{main}|}|
\end{center}
%
A simple redirection |\childdocforward{|\textit{dest}|}| is achieved by:
%
\begin{center}
|\def\jobname{|\textit{dest}|}\input{\jobname}|
\end{center}
%
The redirection with prefix
|\childdocforwardprefix[|\textit{prefix}|]{|\textit{dest}|}|
is accomplished by:
%
\begin{center}
\begin{tabular}{l}
|{\edef\jobname{\scantokens\expandafter{\jobname\noexpand}}|\\
|\def\redirectjob |\textit{prefix}|#1~~~{\gdef\jobname{|\textit{dest}|#1}}|\\
|\expandafter\redirectjob\jobname~~~}\input{\jobname}|
\end{tabular}
\end{center}

In an alternative approach,
child documents can be compiled by a specific command line
without additional code or specific definitions:
%
\begin{center}
|... -jobname "|\textit{target}|" "|[\textit{flags}]%
|\includeonly{|\textit{dest}|}\input{|\textit{main}|}"|
\end{center}
%

%%%%%%%%%%%%%%%%%%%%%%%%%%%%%%%%%%%%%%%%%%%%%%%%%%%%%%%%%%%%%%%%%%%%%%%%%%%%%%%%
%%%%%%%%%%%%%%%%%%%%%%%%%%%%%%%%%%%%%%%%%%%%%%%%%%%%%%%%%%%%%%%%%%%%%%%%%%%%%%%%
\section{Information}

%%%%%%%%%%%%%%%%%%%%%%%%%%%%%%%%%%%%%%%%%%%%%%%%%%%%%%%%%%%%%%%%%%%%%%%%%%%%%%%%
\subsection{Copyright}

Copyright \copyright{} 2017--2018 Niklas Beisert

This work may be distributed and/or modified under the
conditions of the \LaTeX{} Project Public License, either version 1.3
of this license or (at your option) any later version.
The latest version of this license is in
  \url{http://www.latex-project.org/lppl.txt}
and version 1.3 or later is part of all distributions of \LaTeX{}
version 2005/12/01 or later.

This work has the LPPL maintenance status `maintained'.

The Current Maintainer of this work is Niklas Beisert.

This work consists of the files |README.txt|, |childdoc.ins| and |childdoc.dtx|
as well as the derived files |childdoc.def|, |cdocsamp.tex|
with |cdocsch1.tex|, |cdocsch2.tex|, |cdocspt3.tex|, |cdocspt4.tex|,
|cdocsdrf.tex|, |cdocsfn1.tex|, |cdocsfn2.tex|
as well as |childdoc.pdf|.

%%%%%%%%%%%%%%%%%%%%%%%%%%%%%%%%%%%%%%%%%%%%%%%%%%%%%%%%%%%%%%%%%%%%%%%%%%%%%%%%
\subsection{Files and Installation}

The package consists of the files:
%
\begin{center}
\begin{tabular}{ll}
    |README.txt|   & readme file \\
    |childdoc.ins| & installation file \\
    |childdoc.dtx| & source file \\
    |childdoc.def| & definition file \\
    |cdocsamp.tex| & sample main file \\
    |cdocsch1.tex| & sample include file \\
    |cdocsch2.tex| & sample include file \\
    |cdocspt3.tex| & sample part file \\
    |cdocspt4.tex| & sample part file \\
    |cdocsdrf.tex| & sample redirection file \\
    |cdocsfn1.tex| & sample redirection file \\
    |cdocsfn2.tex| & sample redirection file \\
    |childdoc.pdf| & manual
\end{tabular}
\end{center}
%
The distribution consists of the files
|README.txt|, |childdoc.ins| and |childdoc.dtx|.
%
\begin{itemize}
\item
Run (pdf)\LaTeX{} on |childdoc.dtx|
to compile the manual |childdoc.pdf| (this file).
\item
Run \LaTeX{} on |childdoc.ins| to create the definitions file |childdoc.def|
and the sample |cdocsamp.tex| with include files
|cdocsch1.tex|, |cdocsch2.tex|, |cdocspt3.tex|, |cdocspt4.tex|,
|cdocsdrf.tex|, |cdocsfn1.tex|, |cdocsfn2.tex|.
Then copy the file |childdoc.def| to an appropriate directory of your \LaTeX{}
distribution, e.g.\ \textit{texmf-root}|/tex/latex/childdoc|.
\end{itemize}

%%%%%%%%%%%%%%%%%%%%%%%%%%%%%%%%%%%%%%%%%%%%%%%%%%%%%%%%%%%%%%%%%%%%%%%%%%%%%%%%
\subsection{Related CTAN Packages}

There are several other packages which offer a similar functionality:
%
\begin{itemize}
\item
The packages
\href{http://ctan.org/pkg/docmute}{\textsf{docmute}},
\href{http://ctan.org/pkg/includex}{\textsf{includex}} and
\href{http://ctan.org/pkg/standalone}{\textsf{standalone}}
provide commands to include only the document body of
a child file thus allowing both files to be compiled individually.
\item
The packages \href{http://ctan.org/pkg/subdocs}{\textsf{subdocs}}
and \href{http://ctan.org/pkg/subfiles}{\textsf{subfiles}}
provide structures in which the main and child documents can be
encapsulated and allowing them to be compiled individually.
The inclusion mechanism is different from the conventional |\include|.
\item
The package \href{http://ctan.org/pkg/combine}{\textsf{combine}}
is an elaborate solution to combine several documents into one.
\end{itemize}
%
See also the CTAN topic \href{http://ctan.org/topic/subdocs}{\textsf{subdocs}}
for further related packages.
The present package differs from the above solutions in that
a document structure constructed with the conventional |\include| mechanism
just needs two extra commands at the top of every file
such that all constituent files can be compiled individually.

%%%%%%%%%%%%%%%%%%%%%%%%%%%%%%%%%%%%%%%%%%%%%%%%%%%%%%%%%%%%%%%%%%%%%%%%%%%%%%%%
%\subsection{Feature Suggestions}
%
%The following is a list of features which may be useful for future
%versions of this package:
%%
%\begin{itemize}
%\item
%\ldots
%\end{itemize}

%%%%%%%%%%%%%%%%%%%%%%%%%%%%%%%%%%%%%%%%%%%%%%%%%%%%%%%%%%%%%%%%%%%%%%%%%%%%%%%%
\subsection{Revision History}

%%%%%%%%%%%%%%%%%%%%%%%%%%%%%%%%%%%%%%%%
\paragraph{v2.0:} 2018/12/30

\begin{itemize}
\item
immediate forward processing
\item
added |\childdocby| mechanism
\item
manual restructured
\end{itemize}

%%%%%%%%%%%%%%%%%%%%%%%%%%%%%%%%%%%%%%%%
\paragraph{v1.6:} 2018/01/17

\begin{itemize}
\item
application for development of include files
\item
corrections to manual
\end{itemize}

%%%%%%%%%%%%%%%%%%%%%%%%%%%%%%%%%%%%%%%%
\paragraph{v1.5:} 2017/05/21

\begin{itemize}
\item
more complete structuring introduced
\item
|\childdocof| introduced
\item
|\childdoc| renamed to |\childdocmain|
\item
|\childredirect| renamed to |\childdocforward| and |\childdocforwardprefix|
and functionality expanded
\end{itemize}

%%%%%%%%%%%%%%%%%%%%%%%%%%%%%%%%%%%%%%%%
\paragraph{v1.0:} 2017/04/27

\begin{itemize}
\item
manual and install package
\item
first version published on CTAN
\end{itemize}

%%%%%%%%%%%%%%%%%%%%%%%%%%%%%%%%%%%%%%%%
\paragraph{v0.6:} 2017/04/26

\begin{itemize}
\item
redirection mechanism added
\end{itemize}

%%%%%%%%%%%%%%%%%%%%%%%%%%%%%%%%%%%%%%%%
\paragraph{v0.5:} 2017/04/26

\begin{itemize}
\item
functionality in definition file
\end{itemize}


%%%%%%%%%%%%%%%%%%%%%%%%%%%%%%%%%%%%%%%%%%%%%%%%%%%%%%%%%%%%%%%%%%%%%%%%%%%%%%%%
%%%%%%%%%%%%%%%%%%%%%%%%%%%%%%%%%%%%%%%%%%%%%%%%%%%%%%%%%%%%%%%%%%%%%%%%%%%%%%%%
%%%%%%%%%%%%%%%%%%%%%%%%%%%%%%%%%%%%%%%%%%%%%%%%%%%%%%%%%%%%%%%%%%%%%%%%%%%%%%%%
\appendix

\settowidth\MacroIndent{\rmfamily\scriptsize 000\ }

 \DocInput{childdoc.dtx}

\end{document}
%</driver>
% \fi
%
% %%%%%%%%%%%%%%%%%%%%%%%%%%%%%%%%%%%%%%%%%%%%%%%%%%%%%%%%%%%%%%%%%%%%%%%%%%%%%%
% %%%%%%%%%%%%%%%%%%%%%%%%%%%%%%%%%%%%%%%%%%%%%%%%%%%%%%%%%%%%%%%%%%%%%%%%%%%%%%
% \section{Sample}
%\iffalse
%<*samplemain>
%\fi
%
% The following presents a sample document
% with two chapters, two parts, a title page,
% a compile flag as well as three forwarding files to set the flag.
% It consists of eight |.tex| files:
% \begin{center}
% \begin{tabular}{ll}
% |cdocsamp.tex|&main file\\
% |cdocsch1.tex|&include file for chapter 1\\
% |cdocsch2.tex|&include file for chapter 2\\
% |cdocspt3.tex|&include file for part 3\\
% |cdocspt4.tex|&include file for part 4\\
% |cdocsdrf.tex|&forwarding file for main file in draft mode\\
% |cdocsfi1.tex|&forwarding file for final version of chapter 1\\
% |cdocsfi2.tex|&forwarding file for final version of chapter 2\\
% \end{tabular}
% \end{center}
% Each of the eight files can be compiled directly by the \LaTeX{} compiler.
%
% %%%%%%%%%%%%%%%%%%%%%%%%%%%%%%%%%%%%%%
% \paragraph{Main File.}
%
% The main file is called |cdocsamp.tex|.
%
% Load the \textsf{childdoc} definitions and
% declare the filename for the main document:
%    \begin{macrocode}
\input{childdoc.def}
\childdocmain{}
%    \end{macrocode}

% Optional override for |\version| flag:
%    \begin{macrocode}
%%\ifchilddoc\else\providecommand{\version}{draft}\fi
%    \end{macrocode}

% Define the default values for the |\version| flag
% (|final| for the main file and |draft| for childs):
%    \begin{macrocode}
\ifchilddoc
\providecommand{\version}{draft}
\else
\providecommand{\version}{final}
\fi
%    \end{macrocode}

% Load the standard document class:
%    \begin{macrocode}
\documentclass[12pt]{article}
%    \end{macrocode}

% Start the document body:
%    \begin{macrocode}
\begin{document}
%    \end{macrocode}

% Declare a title page.
% Print title, part of document being processed and version flag:
%    \begin{macrocode}
\addtocounter{page}{-1}
\begin{center}
{\LARGE\bfseries{}childdoc example\par}
\vspace{1cm}
\ifchilddoc
\ifchilddocmanual part\else chapter\fi:
`\childdocname' of `\childdocjob'\par
\else
main document: `\childdocjob'\par
\fi
version: \version\par
\end{center}
\newpage
%    \end{macrocode}

% Manually include selected file,
% otherwise process as usual:
%    \begin{macrocode}
\ifchilddocmanual
\section*{part `\childdocname'}
\input{\childdocname}
\else
%    \end{macrocode}

% Include the two chapters:
%    \begin{macrocode}
\include{cdocsch1}
\include{cdocsch2}
%    \end{macrocode}

% Include the two parts unless only chapters should be displayed:
%    \begin{macrocode}
\ifchilddoc\else
\section{part three}
\input{cdocspt3}
\section{part four}
\input{cdocspt4}
\fi
%    \end{macrocode}

% Process as usual until here:
%    \begin{macrocode}
\fi
%    \end{macrocode}

% End of document body:
%    \begin{macrocode}
\end{document}
%    \end{macrocode}
%\iffalse
%</samplemain>
%\fi
%
% %%%%%%%%%%%%%%%%%%%%%%%%%%%%%%%%%%%%%%
% \paragraph{Chapter Include Files.}
%
% The include files are called |cdocsch1.tex| and |cdocsch2.tex|.
%
%\iffalse
%<*samplechap1|samplechap2>
%\fi

% Optional override for |\version| flag:
%    \begin{macrocode}
%%\providecommand{\version}{final}
%    \end{macrocode}

% Include the main document:
%    \begin{macrocode}
\input{childdoc.def}
\childdocof{cdocsamp}
%    \end{macrocode}

%\iffalse
%</samplechap1|samplechap2>
%\fi
%
%\iffalse
%<*samplechap1>
%\fi
% Some text for chapter 1:
%    \begin{macrocode}
\section{one}
some text in chapter one
%    \end{macrocode}

%\iffalse
%</samplechap1>
%\fi
% Some text for chapter 2:
%\iffalse
%<*samplechap2>
%\fi
%    \begin{macrocode}
\section{two}
more text in chapter two
%    \end{macrocode}

%\iffalse
%</samplechap2>
%\fi
%
% %%%%%%%%%%%%%%%%%%%%%%%%%%%%%%%%%%%%%%
% \paragraph{Part Include Files.}
%
% The include files are called |cdocspt3.tex| and |cdocspt4.tex|.
%
%\iffalse
%<*samplepart3|samplepart4>
%\fi

% Optional override for |\version| flag:
%    \begin{macrocode}
%%\providecommand{\version}{final}
%    \end{macrocode}

% Include the main document:
%    \begin{macrocode}
\input{childdoc.def}
\childdocby{cdocsamp}
%    \end{macrocode}

%\iffalse
%</samplepart3|samplepart4>
%\fi
%
%\iffalse
%<*samplepart3>
%\fi
% Some text for part 3:
%    \begin{macrocode}
some text in part three
%    \end{macrocode}

%\iffalse
%</samplepart3>
%\fi
% Some text for part 4:
%\iffalse
%<*samplepart4>
%\fi
%    \begin{macrocode}
more text in part four
%    \end{macrocode}

%\iffalse
%</samplepart4>
%\fi
%
% %%%%%%%%%%%%%%%%%%%%%%%%%%%%%%%%%%%%%%
% \paragraph{Forwarding for a Complete Draft.}
%
% The following forwarding file |cdocsdrf.tex|
% compiles the main document in draft mode:
%\iffalse
%<*sampledraft>
%\fi
%    \begin{macrocode}
\def\version{draft}
\input{childdoc.def}
\childdocforward{cdocsamp}
%    \end{macrocode}

%\iffalse
%</sampledraft>
%\fi
%
% %%%%%%%%%%%%%%%%%%%%%%%%%%%%%%%%%%%%%%
% \paragraph{Forwarding for Final Version of the Chapters.}
%
% The following forwarding files |cdocsfn1.tex| and |cdocsfn2.tex|
% (with identical content)
% compile the final versions of the child documents
% |cdocsch1.tex| and |cdocsch2.tex|, respectively:
%\iffalse
%<*samplefinal>
%\fi
%    \begin{macrocode}
\def\version{final}
\input{childdoc.def}
\childdocforwardprefix[cdocsamp]{cdocsfn}{cdocsch}
%    \end{macrocode}

%\iffalse
%</samplefinal>
%\fi
%
% %%%%%%%%%%%%%%%%%%%%%%%%%%%%%%%%%%%%%%
% \paragraph{Command Line Processing.}
%
% The following three command lines generate the output files
% |cdocscld|, |cdocscl1| and |cdocscl2|
% which should be identical to
% |cdocsdrf|, |cdocsch1| and |cdocsfn2|, respectively:
% \begin{center}
% \begin{tabular}{l}
% |latex -jobname cdocscld \|\\
% |  "\def\version{draft}\input{childdoc.def}\childdocforward{cdocsamp}"|\\
% |latex -jobname cdocscl1 \|\\
% |  "\input{childdoc.def}\childdocforward[cdocsamp]{cdocsch1}"|\\
% |latex -jobname cdocscl2 \|\\
% |  "\def\version{final}\input{childdoc.def}\childdocforward{cdocsch2}"|
% \end{tabular}
% \end{center}
% Note that the trailing backslash on each first line
% merely continues the input to the second line
% (for convenient cut ant paste).
% Furthermore, the command |latex| can be replaced by any
% of its alternative versions such as |pdflatex|.
%
% %%%%%%%%%%%%%%%%%%%%%%%%%%%%%%%%%%%%%%%%%%%%%%%%%%%%%%%%%%%%%%%%%%%%%%%%%%%%%%
% %%%%%%%%%%%%%%%%%%%%%%%%%%%%%%%%%%%%%%%%%%%%%%%%%%%%%%%%%%%%%%%%%%%%%%%%%%%%%%
% \section{Implementation}
%\iffalse
%<*package>
%\fi
%
% This section describes the definitions file |childdoc.def|.

% The definitions cannot be loaded using |\usepackage| or |\RequirePackage|
% which has a mechanism to prevent loading a style file more than once.
% When loading the definitions by means of |\input|
% multiple instances have to be prevented manually:
%\iffalse
%This code needs to be before the `\ProvidesFile' directive
%which is defined at the beginning of this file.
%Therefore it is also placed there and commented out here.
%</package>
%<*discard>
%\fi
%    \begin{macrocode}
\ifdefined\childdocmain\endinput\fi
%    \end{macrocode}
%\iffalse
%</discard>
%<*package>
%\fi
%
% \macro{\ifchilddoc}
% \macro{\ifchilddocmanual}
% The conditional |\ifchilddoc| tells whether a
% child (true) or main (false) document is being compiled.
% The conditional |\ifchilddocmanual| tells whether
% the |\includeonly| mechanism is used (false) or
% the selection of child files must be performed manually (true).
% The definitions initialise to false:
%    \begin{macrocode}
\newif\ifchilddoc
\newif\ifchilddocmanual
%    \end{macrocode}

% \macro{\childdocname}
% \macro{\childdocjob}
% The macro |\childdocname| stores the name of the main document
% to be compiled. The macro |\childdocjob| stores the name of
% the document on which the \LaTeX{} compiler was originally invoked.
% The content of |\jobname| cannot be compared
% to filenames specified in the source due to different catcodes.
% The following code rescans |\jobname|, stores the result
% in |\childdocname| and saves a copy in |\childdocjob|:
%    \begin{macrocode}
\edef\childdocname{\scantokens\expandafter{\jobname\noexpand}}
\let\childdocjob\childdocname
%    \end{macrocode}

% \macro{\childdocdisable}
% The macro |\childdocdisable| prevents the main file
% from being processed more than once.
% At this stage, the main document command |\childdocmain|
% is assumed to be called once again where it should do nothing.
% Any subsequent call to it should prevent
% a secondary processing of the main document
% It overwrites the forwarding commands
% |\childdocof| and |\childdocforward|
% with empty macros to prevent further inclusions of the main document:
%    \begin{macrocode}
\newcommand{\childdocdisable}
{
  \renewcommand{\childdocmain}[1]{\renewcommand{\childdocmain}[1]{\endinput}}
  \renewcommand{\childdocof}[1]{}
  \renewcommand{\childdocby}[2][]{}
  \renewcommand{\childdocforward}[2][]{}
  \renewcommand{\childdocdisable}{}
}
%    \end{macrocode}

% \macro{\childdocmain}
% The macro |\childdocmain| is to be called at the top of the main file
% with nothing or the main filename (without extension) as argument.
% First, it breaks loops.
% If the argument is not empty and does not match |\childdocname|
% (which is set by the first inclusion of |childdoc.def|),
% |\ifchilddoc| is set to true, |\includeonly| is applied to the child file
% and |\jobname| is set to the main file
% (for proper handling of |.aux| files):
%    \begin{macrocode}
\newcommand{\childdocmain}[1]
{
  \childdocdisable\childdocmain{}
  \if?#1?\else
    \begingroup
      \def\childdoctmp{#1}
      \ifx\childdoctmp\childdocname
        \def\childdoctmp{}
      \else
        \def\childdoctmp
        {
          \childdoctrue
          \includeonly{\childdocname}
          \def\childdocjob{#1}
          \def\jobname{#1}
        }
      \fi
      \expandafter
    \endgroup
    \childdoctmp
  \fi
}
%    \end{macrocode}

% \macro{\childdocof}
% The command |\childdocof| redirects
% compilation to the main file |#1|.
%    \begin{macrocode}
\newcommand{\childdocof}[1]
{
  \childdocdisable
  \childdoctrue
  \includeonly{\childdocname}
  \def\jobname{#1}
  \def\childdocjob{#1}
  \input{#1}
}
%    \end{macrocode}

% \macro{\childdocby}
% The command |\childdocby| ....
%    \begin{macrocode}
\newcommand{\childdocby}[2][]
{
  \childdocdisable
  \childdoctrue
  \childdocmanualtrue
  \if?#1?\else
    \def\jobname{#2}
  \fi
  \def\childdocjob{#2}
  \input{#2}
  \endinput
}
%    \end{macrocode}

% \macro{\childdocforward}
% The command |\childdocforward| redirects
% compilation to the main file or
% (if the optional argument is given) a child file.
% Parameters are set as if the main file
% or a child file starting with |\childdocof| was compiled.
% Then compilation is handed over to the main file:
%    \begin{macrocode}
\newcommand{\childdocforward}[2][]
{
  \begingroup
    \if?#1?
      \def\childdoctmp
      {
        \def\childdocname{#2}
        \def\childdocjob{#2}
        \def\jobname{#2}
        \input{#2}
        \endinput
      }
    \else
      \def\childdoctmp
      {
        \childdocdisable
        \def\childdocname{#2}
        \childdoctrue
        \includeonly{#2}
        \def\childdocjob{#1}
        \def\jobname{#1}
        \input{#1}
        \endinput
      }
    \fi
    \expandafter
  \endgroup
  \childdoctmp
}
%    \end{macrocode}

% \macro{\childdocforwardprefix}
% The command |\childdocforwardprefix| redirects
% compilation to the main or a child file by means of a pattern.
% The prefix |#1| in the current filename is replaced by |#2|
% and the suffix of the current filename is kept
% (it is assumed that the filename does not contain the substring `|~~~|'
% which is used as a delimiter).
% Compilation is handed over to the new file by |\childdocforward|:
%    \begin{macrocode}
\newcommand{\childdocforwardprefix}[3][]
{
  \begingroup
    \def\childdocextract #2##1~~~{\def\childdoctmp{\childdocforward[#1]{#3##1}}}
    \expandafter\childdocextract\childdocname~~~
    \expandafter
  \endgroup
  \childdoctmp
}
%    \end{macrocode}

% \macro{\childdoc}
% The deprecated macro |\childdoc| is a legacy version of |\childdocmain|:
%    \begin{macrocode}
\newcommand{\childdoc}{\childdocmain}
%    \end{macrocode}

% \macro{\childdocredirect}
% The deprecated macro |\childdocredirect| is a legacy version
% of |\childdocforward| and |\childdocforwardprefix|:
%    \begin{macrocode}
\newcommand{\childdocredirect}[2][]
{
  \begingroup
    \if?#1?
      \def\childdoctmp{\childdocforward{#2}}
    \else
      \def\childdoctmp{\childdocforwardprefix{#1}{#2}}
    \fi
    \expandafter
  \endgroup
  \childdoctmp
}
%    \end{macrocode}

%\iffalse
%</package>
%\fi
%
\endinput
|\\
|\childdocforwardprefix{final}{child}|
\end{tabular}
\end{center}
%

Note that when several versions of a main file and/or of each child file
are to be generated, it may be convenient to set up a |Makefile| or
shell script to automatise the process.

%%%%%%%%%%%%%%%%%%%%%%%%%%%%%%%%%%%%%%%%%%%%%%%%%%%%%%%%%%%%%%%%%%%%%%%%%%%%%%%%
\subsection{Command Line Processing}
\label{sec:commandline}

The effect of redirection files can also be achieved by invoking
the \LaTeX{} compiler with a more elaborate command line.
Most conveniently this should be done as part
of a shell script or a |Makefile|.

When using \textsf{childdoc} in the main file, the following
command lines effectively perform a redirection
(note that depending on the shell being used,
backslashes may have to be doubled: `|\|' $\to$ `|\\|'):
%
\begin{center}
|... -jobname "|\textit{target}|" |\\|"|[\textit{flags}]%
|% \iffalse
%
% childdoc.dtx Copyright (C) 2017-2018 Niklas Beisert
%
% This work may be distributed and/or modified under the
% conditions of the LaTeX Project Public License, either version 1.3
% of this license or (at your option) any later version.
% The latest version of this license is in
%   http://www.latex-project.org/lppl.txt
% and version 1.3 or later is part of all distributions of LaTeX
% version 2005/12/01 or later.
%
% This work has the LPPL maintenance status `maintained'.
%
% The Current Maintainer of this work is Niklas Beisert.
%
% This work consists of the files childdoc.dtx and childdoc.ins
% and the derived files childdoc.def and cdocsamp.tex with
% cdocsch1.tex, cdocsch2.tex, cdocsdrf.tex, cdocsfn1.tex, cdocsfn2.tex.
%
%<package>\ifdefined\childdocmain\endinput\fi
%<package>\ProvidesFile{childdoc.def}[2018/12/30 v2.0 child document driver]
%<samplemain>\ProvidesFile{cdocsamp.tex}[2018/12/30 v2.0 sample for childdoc]
%<*driver>
%\ProvidesFile{childdoc.drv}[2018/12/30 v2.0 childdoc reference manual file]
\PassOptionsToClass{10pt,a4paper}{article}
\documentclass{ltxdoc}

\usepackage[margin=35mm]{geometry}
\usepackage{hyperref}
\usepackage{hyperxmp}
\usepackage[usenames]{color}

\hypersetup{colorlinks=true}
\hypersetup{pdfstartview=FitH}
\hypersetup{pdfpagemode=UseNone}
\hypersetup{pdfsource={}}
\hypersetup{pdflang={en-UK}}
\hypersetup{pdfcopyright={Copyright 2017-2018 Niklas Beisert.
  This work may be distributed and/or modified under the
  conditions of the LaTeX Project Public License, either version 1.3
  of this license or (at your option) any later version.}}
\hypersetup{pdflicenseurl={http://www.latex-project.org/lppl.txt}}
\hypersetup{pdfcontactaddress={ETH Zurich, ITP, HIT K,
  Wolfgang-Pauli-Strasse 27}}
\hypersetup{pdfcontactpostcode={8093}}
\hypersetup{pdfcontactcity={Zurich}}
\hypersetup{pdfcontactcountry={Switzerland}}
\hypersetup{pdfcontactemail={nbeisert@itp.phys.ethz.ch}}
\hypersetup{pdfcontacturl={http://people.phys.ethz.ch/\xmptilde nbeisert/}}

\newcommand{\secref}[1]{\hyperref[#1]{section \ref*{#1}}}

\parskip1ex
\parindent0pt
\let\olditemize\itemize
\def\itemize{\olditemize\parskip0pt}

\begin{document}

\title{The \textsf{childdoc} Package}
\hypersetup{pdftitle={The childdoc Package}}
\author{Niklas Beisert\\[2ex]
  Institut f\"ur Theoretische Physik\\
  Eidgen\"ossische Technische Hochschule Z\"urich\\
  Wolfgang-Pauli-Strasse 27, 8093 Z\"urich, Switzerland\\[1ex]
  \href{mailto:nbeisert@itp.phys.ethz.ch}
  {\texttt{nbeisert@itp.phys.ethz.ch}}}
\hypersetup{pdfauthor={Niklas Beisert}}
\hypersetup{pdfsubject={Manual for the LaTeX2e Package childdoc}}
\date{30 December 2018, \textsf{v2.0}}
\maketitle

\begin{abstract}\noindent
\textsf{childdoc} is a \LaTeXe{} package
that enables the direct compilation
of document sections included by |\include|
to individual files.
\end{abstract}

\begingroup
\parskip0ex
\tableofcontents
\endgroup

%%%%%%%%%%%%%%%%%%%%%%%%%%%%%%%%%%%%%%%%%%%%%%%%%%%%%%%%%%%%%%%%%%%%%%%%%%%%%%%%
%%%%%%%%%%%%%%%%%%%%%%%%%%%%%%%%%%%%%%%%%%%%%%%%%%%%%%%%%%%%%%%%%%%%%%%%%%%%%%%%
\section{Introduction}

\LaTeX{} provides a mechanism to structure a large document (such as a book)
into a main file and several child files (containing the chapters)
using the |\include| command.
This mechanism is beneficial for documents
which span hundreds of pages in order to
make the source file(s) more manageable.
Moreover, compilation can be restricted to
selected child files by means of the |\includeonly| command.
The latter feature can be used to reduce the compilation time while editing
(this was significantly more useful in the earlier days of \LaTeX{})
or to generate a smaller document which is easier to navigate.
Another application of |\includeonly| is to generate
documents consisting of selected parts of the complete document.

However, there are a few drawbacks of the plain |\include| mechanism:
\begin{itemize}
\item
The child files cannot be compiled on their own,
they can only be compiled via the main file.
A naive editing environment
(such as a text editor with an option
to have the current file processed by \LaTeX)
may require one to switch to the main file before compiling;
attempting to compile the child file produces errors.
\item
The main file must be modified (each time)
to adjust the |\includeonly| command
to the present needs. This easily leaves the main file in a messy state.
\item
The generated document will always carry the filename
of the main document. This is inconvenient if
several child files are to be compiled and
to be kept for distribution.
\end{itemize}

The present package provides a simple interface
to make child files individually compilable by \LaTeX{}.
Compiling a child file then has the same effect as compiling
the main file with an |\includeonly| command
to select the appropriate child.
Moreover the generated document will carry the name of the child
rather than the main file.
This resolves all three above issues.

This feature is meant to make the editing of books,
thesis documents and lecture notes somewhat more convenient.
However, the package can also be used efficiently for
composing a series of documents (such as exercise sheets)
which are typically distributed individually.
It then assists the author in generating the individual documents
(potentially in different versions)
as well as a document containing the collected series.
Another application is in developing style files
or other kinds of included material
where compilation of the style file could redirect
to a sample or test file.

%%%%%%%%%%%%%%%%%%%%%%%%%%%%%%%%%%%%%%%%%%%%%%%%%%%%%%%%%%%%%%%%%%%%%%%%%%%%%%%%
%%%%%%%%%%%%%%%%%%%%%%%%%%%%%%%%%%%%%%%%%%%%%%%%%%%%%%%%%%%%%%%%%%%%%%%%%%%%%%%%
\section{Usage}

First of all, the package \textsf{childdoc} is \emph{not} a standard
\LaTeXe{} |.sty| style file! Therefore it needs to be invoked in
a non-standard way.

%%%%%%%%%%%%%%%%%%%%%%%%%%%%%%%%%%%%%%%%%%%%%%%%%%%%%%%%%%%%%%%%%%%%%%%%%%%%%%%%
\subsection{Included Files}
\label{sec:include}

%%%%%%%%%%%%%%%%%%%%%%%%%%%%%%%%%%%%%%%%
\DescribeMacro{\childdocmain}
To use the package, add the commands
\begin{center}
\begin{tabular}{l}
|\input{childdoc.def}|\\
|\childdocmain{}|\\
\end{tabular}
\end{center}
at the very top of the main \LaTeX{} file,
in particular \emph{before} the |\documentclass| statement!
The argument of |\childdocmain| should be left empty
(but it must be present).

%%%%%%%%%%%%%%%%%%%%%%%%%%%%%%%%%%%%%%%%
\DescribeMacro{\childdocof}
Furthermore, add the commands
\begin{center}
\begin{tabular}{l}
|\input{childdoc.def}|\\
|\childdocof{|\textit{main}|}|\\
\end{tabular}
\end{center}
at the top of every child file \textit{child}
which is included by |\include{|\textit{child}|}|
from within the main file
(or at least for those files to be compiled individually).
The argument \textit{main} must be the filename of the main file.

There are a couple of
considerations in setting up the main and child documents:

%%%%%%%%%%%%%%%%%%%%%%%%%%%%%%%%%%%%%%%%
\paragraph{Restrictions.}

Please note the following restrictions:
\begin{itemize}
\item
|\childdocmain| must be called with one argument \textit{main}
to ensure compatibility with earlier version of the package.
It must either be empty (|\childdocmain{}|)
or precisely match the filename of the main file in which it is specified.
See \secref{sec:detection} for further information.
\item
The filename \textit{main} must be specified without the |.tex| extension.
\item
The filename \textit{main} is case sensitive
(even in case-insensitive file systems)
due to internal string comparison.
\item
The argument \textit{main} should be fully expanded, it cannot be a macro.
\item
Subdirectories and special characters should be avoided in filenames.
\item
The command |\childdocmain{|\textit{main}|}| must be followed by a whitespace.
It should not be followed immediately by another command
or by a comment mark `|%|'.
This is because the \TeX{} parser reads the token immediately following
the argument of |\childdocmain| and puts it
at the beginning of every child section;
however, a white\-space is ignored.
\end{itemize}

%%%%%%%%%%%%%%%%%%%%%%%%%%%%%%%%%%%%%%%%
\paragraph{Content of Main File.}

It is advisable to place all content in the child files included by |\include|.
Any output contained in the main file will appear in all child documents
unless suppressed manually;
it cannot be suppressed automatically by the |\includeonly| directive
and thus should normally be avoided.
A method to include some content in the main file
by means of conditional processing is described in \secref{sec:conditional}.

%%%%%%%%%%%%%%%%%%%%%%%%%%%%%%%%%%%%%%%%
\paragraph{Page Numbering.}

When only a part of the document is compiled,
the appropriate numbering of pages
(as well as other status parameters)
is determined from the |.aux| files.
The latter contain information from previous passes.
However this information needs to propagate through
all intermediate child documents.
Therefore the page numbering in child documents may well
be inconsistent until the complete document is compiled at least once.

A useful (if unconventional) way to always ensure a consistent
page numbering is to restart the numbering in each child document
and denote the pages by `\textit{child}|.|\textit{page}'
where \textit{child} represents the chapter/section number of the child file.
This can be achieved by the command
|\numberwithin{page}{|\textit{child}|}|
of the \textsf{amsmath} package
where \textit{child} can be |chapter| or |section|
depending on the chosen structuring.
Alternatively, one can modify the macro |\thepage| appropriately
and reset the counter |page| at the start of each child file.

%%%%%%%%%%%%%%%%%%%%%%%%%%%%%%%%%%%%%%%%%%%%%%%%%%%%%%%%%%%%%%%%%%%%%%%%%%%%%%%%
\subsection{Conditional Processing}
\label{sec:conditional}

The package provides a mechanism to compile different versions
of a document. To customise the versions further some conditional processing
can come in handy to distinguish which version is being compiled.
The package provides two macros to describe the compilation context:

%%%%%%%%%%%%%%%%%%%%%%%%%%%%%%%%%%%%%%%%
\DescribeMacro{\ifchilddoc}
The conditional |\ifchilddoc| distinguishes between the compilation of
child documents and the main document:
%
\begin{center}
|\ifchilddoc |\textit{child-code}| |[|\||else |\textit{main-code}]| \||fi|
\end{center}

%%%%%%%%%%%%%%%%%%%%%%%%%%%%%%%%%%%%%%%%
\DescribeMacro{\childdocname}
\DescribeMacro{\childdocjob}
The macro |\childdocname| contains the filename (without extension)
of the main or child file being processed.
Note that |\childdocjob| will always contain the name of the main file.

%%%%%%%%%%%%%%%%%%%%%%%%%%%%%%%%%%%%%%%%
\paragraph{Title Page.}

Conditional processing can be used to include a title or banner page
in the main document when proper precautions are taken.
Importantly, the code in the main file should ensure that the page counter
(as well as other status parameters which are stored in the |.aux| files)
takes the same value after the conditional processing.
Otherwise the page numbers may take divergent values
depending on which part is compiled.

For example, a title page could be declared by:
%
\begin{center}
\begin{tabular}{l}
|\ifchilddoc\||else|\\
|\addtocounter{page}{-1}|\\
\textit{code for title page}\\
|\newpage|\\
|\||fi|
\end{tabular}
\end{center}
%
A banner page for the child documents can be generated by:
%
\begin{center}
\begin{tabular}{l}
|\ifchilddoc|\\
|\addtocounter{page}{-1}|\\
\textit{code for banner page}\\
|\newpage|\\
|\||fi|
\end{tabular}
\end{center}
%
Here one could write a message such as:
\begin{center}
|This is the part \childdocname{} of \childdocjob{}.|
\end{center}

%%%%%%%%%%%%%%%%%%%%%%%%%%%%%%%%%%%%%%%%%%%%%%%%%%%%%%%%%%%%%%%%%%%%%%%%%%%%%%%%
\subsection{Flags}
\label{sec:flags}

The package makes it easy to generate different versions
of the main or child documents.
To this end compilation flags can be defined
and assigned different default values.
They will be particularly useful in conjunction
with the forwarding mechanism described in \secref{sec:forward}.

For example, it may be useful to have a flag |\version|
which can be set to |draft| or |final|.
The document source will contain some conditional code
depending on the value of |\version|.
Suppose further, the flag should default to |final| for the main file
and to |draft| for child files
which is a natural assignment for editing the document.
This is achieved by placing the following code
in the preamble of the main document
(below the |\childdocmain| directive):
%
\begin{center}
\begin{tabular}{l}
|\ifchilddoc|\\
|\providecommand{\version}{draft}|\\
|\||else|\\
|\providecommand{\version}{final}|\\
|\||fi|
\end{tabular}
\end{center}
%
The definition by |\providecommand| makes sure
that previous definitions are not overwritten.
Further statements |\providecommand{\version}{...}|
can thus be added before the above code to override it.

For the main file, one might add a line
(between |\childdocmain| and the above block)
%
\begin{center}
|%\ifchilddoc\||else\providecommand{\version}{draft}\||fi|
\end{center}
%
which can be uncommented to produce a draft version.
Likewise one can add a line to the very top of a child file
(above the |\childdocof{|\textit{main}|}| directive)
%
\begin{center}
|%\providecommand{\version}{final}|
\end{center}
%
which can be uncommented to produce the final version of this child document.

%%%%%%%%%%%%%%%%%%%%%%%%%%%%%%%%%%%%%%%%%%%%%%%%%%%%%%%%%%%%%%%%%%%%%%%%%%%%%%%%
\subsection{Forwarding}
\label{sec:forward}

Different versions of the main or child documents
using compilation flags as described in \secref{sec:flags}
can be (permanently) stored in different files
for convenient compilation, viewing and distribution.
To this end, the package defines a command
to pass on compilation to a different file:

%%%%%%%%%%%%%%%%%%%%%%%%%%%%%%%%%%%%%%%%
\DescribeMacro{\childdocforward}
The command |\childdocforward| redirects processing to
another source file:
%
\begin{center}
\begin{tabular}{l}
|\input{childdoc.def}|\\
|\childdocforward[|\textit{main}|]{|\textit{dest}|}|\\
\end{tabular}
\end{center}
%
The argument \textit{dest} is the destination file
(without extension).
It should be the main file or one of the child files.
Note that further \textsf{childdoc} directives
such as |\childdocof| and |\childdocforward|
in the indicated file will be processed in this form.
The optional argument \textit{main}
passes on directly to the main file \textit{main}
while pretending to compile the child \textit{dest}.
This form behaves as if \textit{dest}
issues |\childdocof{|\textit{main}|}| right away,
and no further \textsf{childdoc} directives will be processed.

%%%%%%%%%%%%%%%%%%%%%%%%%%%%%%%%%%%%%%%%
\DescribeMacro{\...prefix}
In the alternative form |\childdocforwardprefix|,
%
\begin{center}
\begin{tabular}{l}
|\input{childdoc.def}|\\
|\childdocforwardprefix[|\textit{main}|]{|\textit{prefix}|}{|\textit{dest}|}|
\end{tabular}
\end{center}
%
the destination file is determined by a pattern
depending on the current file:
To make this work, the current file must be called
`{\textit{prefix}\hspace{0.2em}\textit{suffix}}'
with \textit{prefix} matching precisely the argument.
Processing is then passed on to the file
`{\textit{dest}\hspace{0.2em}\textit{suffix}}'.
Surely, the same effect is achieved by
directly specifying the
argument `{\textit{dest}\hspace{0.2em}\textit{suffix}}'
in the first form.
However, that requires to set up a different file
for each child. With the alternative form of the command
all these files can have exactly the same content
which simplifies setting them up and maintaining them.

For example, the following file |draft.tex|
with a compilation flag |\version| as described in \secref{sec:flags}
compiles the main document as a draft:
%
\begin{center}
\begin{tabular}{l}
|\def\version{draft}|\\
|\input{childdoc.def}|\\
|\childdocforward{|\textit{main}|}|
\end{tabular}
\end{center}
%
Likewise, the following files |final|\textit{nn}|.tex|
compile the final version of the child document
|child|\textit{nn}|.tex|:
%
\begin{center}
\begin{tabular}{l}
|\def\version{final}|\\
|\input{childdoc.def}|\\
|\childdocforwardprefix{final}{child}|
\end{tabular}
\end{center}
%

Note that when several versions of a main file and/or of each child file
are to be generated, it may be convenient to set up a |Makefile| or
shell script to automatise the process.

%%%%%%%%%%%%%%%%%%%%%%%%%%%%%%%%%%%%%%%%%%%%%%%%%%%%%%%%%%%%%%%%%%%%%%%%%%%%%%%%
\subsection{Command Line Processing}
\label{sec:commandline}

The effect of redirection files can also be achieved by invoking
the \LaTeX{} compiler with a more elaborate command line.
Most conveniently this should be done as part
of a shell script or a |Makefile|.

When using \textsf{childdoc} in the main file, the following
command lines effectively perform a redirection
(note that depending on the shell being used,
backslashes may have to be doubled: `|\|' $\to$ `|\\|'):
%
\begin{center}
|... -jobname "|\textit{target}|" |\\|"|[\textit{flags}]%
|\input{childdoc.def}\childdocforward[|\textit{main}|]{|\textit{dest}|}"|
\end{center}
%
Here \textit{target} is the name of the output file,
\textit{main} is the name of the main file
and \textit{dest} is the name of the main or child file to be processed
(all filenames without extensions).
The optional argument \textit{main} can be omitted
if \textit{main} matches \textit{dest}.
Optionally, compilation \textit{flags} can be defined via |\def| commands.
This command line makes the \TeX{} engine believe
it is compiling the file \textit{target}
whose content is specified as the latter parameter.
The provided code then forwards the processing to
\textit{main} or \textit{dest} as described in \secref{sec:forward}.

%%%%%%%%%%%%%%%%%%%%%%%%%%%%%%%%%%%%%%%%%%%%%%%%%%%%%%%%%%%%%%%%%%%%%%%%%%%%%%%%
\subsection{Include by Input}
\label{sec:input}

Including child documents by |\include| has some restrictions by design.
Most notably, the content of a child document always occupies
its own set of pages; pages cannot be shared between child documents.
Usually, this behaviour makes perfect sense
because each child document contain an essential part of the document.
However, in some situations it may be desirable to compose
a document from a collection of parts
without having mandatory page breaks between then.
For this case, the package
provides a mechanism to include parts
by |\input| which can also be processed individually.
However, by construction this mechanism
requires manual handling of the content to be output.

%%%%%%%%%%%%%%%%%%%%%%%%%%%%%%%%%%%%%%%%
\DescribeMacro{\ifchilddocmanual}
The main file should be prepared as usual, see \secref{sec:include}.
However, the document body must make a distinction
between processing of an individual part and of the main document, e.g.:
%
\begin{center}
\begin{tabular}{l}
|\ifchilddocmanual|\\
|\input{\childdocname}|\\
|\||else|\\
\textit{document body with }|\input{|\textit{part}|}|\\
|\||fi|
\end{tabular}
\end{center}
%
The conditional |\ifchilddocmanual| is true whenever
a part to be included by |\input| is being compiled,
and the name of the part is stored in |\childdocname|.

%%%%%%%%%%%%%%%%%%%%%%%%%%%%%%%%%%%%%%%%
\DescribeMacro{\childdocby}
Each part to be included by |\input| should start with:
%
\begin{center}
\begin{tabular}{l}
|\input{childdoc.def}|\\
|\childdocby{|\textit{main}|}|\\
\end{tabular}
\end{center}
%
The directive |\childdocby| is similar to |\childdocof|
described in \secref{sec:include},
but the subsequent selection of content must be done manually.
To that end, both |\ifchilddoc| and |\ifchilddocmanual|
will be true upon processing of a part,
and the name of the part is stored in |\childdocname|.
Note that |\jobname| will be set to the filename of the current part
so that each part receives an individual |.aux| file
that does not interfere with the |.aux| file(s) of the main document.
This behaviour can be altered by the alternative form
|\childdocby[*]{|\textit{main}|}| (with a non-empty optional argument)
which uses the |.aux| file of the main document
by setting |\jobname| to \textit{main}.

%%%%%%%%%%%%%%%%%%%%%%%%%%%%%%%%%%%%%%%%%%%%%%%%%%%%%%%%%%%%%%%%%%%%%%%%%%%%%%%%
\subsection{Driver Development}
\label{sec:driver}

The \textsf{childdoc} mechanism can also be use for the development
of definition files such as \LaTeX{} styles or classes.
This case differs from the above setup with multiple parts
included by |\include| in that no |\includeonly| should be invoked.
This can be achieved by starting the include file
(before |\ProvidesPackage|) with:
%
\begin{center}
\begin{tabular}{l}
|\input{childdoc.def}|\\
|\childdocforward{|\textit{main}|}|\\
\end{tabular}
\end{center}
%
or alternatively with:
%
\begin{center}
\begin{tabular}{l}
|\input{childdoc.def}|\\
|\childdocby{|\textit{main}|}|\\
\end{tabular}
\end{center}
%
Both forms have slightly different effects as described above.
The main file is prepared as usual, see \secref{sec:include}.

%%%%%%%%%%%%%%%%%%%%%%%%%%%%%%%%%%%%%%%%%%%%%%%%%%%%%%%%%%%%%%%%%%%%%%%%%%%%%%%%
\subsection{Legacy Detection}
\label{sec:detection}

The directive |\childdocmain| in the main file can detect
whether the complete document or merely a child is to be compiled
even without using the directive |\childdocof|.
This method is deprecated because it is less robust
and there is no compelling reason to use it;
it is merely provided for backward compatibility
and it may be removed in future versions.

If the detection mechanism is to be used,
it is mandatory to correctly specify
the filename of the main file as the argument of |\childdocmain|:
%
\begin{center}
\begin{tabular}{l}
|\input{childdoc.def}|\\
|\childdocmain{|\textit{main}|}|\\
\end{tabular}
\end{center}
%
If |\jobname| does not match the argument \textit{main} of |\childdocmain|,
it is assumed that |\jobname| points to the child file to be compiled.
When using |\childdocmain| with the main file specified as argument,
it suffices to start a child file
with just |\input{|\textit{main}|}|
without loading of the package and using |\childdocof|.
If instead all processing is done
with the appropriate \textsf{childdoc} directives,
the argument of \textit{main} of |\childdocmain| can be empty.

An alternative version of the command line processing described
in \secref{sec:commandline} using the detection mechanism reads:
%
\begin{center}
|... -jobname "|\textit{target}|" "|[\textit{flags}]%
[|\def\jobname{|\textit{dest}|}|]|\input{|\textit{main}|}"|
\end{center}

%%%%%%%%%%%%%%%%%%%%%%%%%%%%%%%%%%%%%%%%%%%%%%%%%%%%%%%%%%%%%%%%%%%%%%%%%%%%%%%%
\subsection{Manual Code}
\label{sec:manual}

In case one cannot be certain whether the definitions file |childdoc.def|
is installed on the target \TeX{} distribution
and one prefers not to ship it,
it is conceivable to paste a few relevant commands into the sources.

To that end, drop all statements |\input{childdoc.def}|
and perform the replacements as outlined below.
Instead of |\childdocmain{|\textit{main}|}| add the following code
to the top of the main file:
%
\begin{center}
\begin{tabular}{l}
|\||ifdefined\childdocname\endinput\||fi\newif\ifchilddoc|\\
|\edef\childdocname{\scantokens\expandafter{\jobname\noexpand}}|\\
|\def\childdocmain{|\textit{main}|}\||ifx\childdocmain\childdocname\||else|\\
|\childdoctrue\includeonly{\childdocname}\let\jobname\childdocmain\||fi|\\
\end{tabular}
\end{center}
%
Instead of |\childdocof{|\textit{main}|}| just include the main file
at the top of each child file:
%
\begin{center}
|\input{|\textit{main}|}|
\end{center}
%
A simple redirection |\childdocforward{|\textit{dest}|}| is achieved by:
%
\begin{center}
|\def\jobname{|\textit{dest}|}\input{\jobname}|
\end{center}
%
The redirection with prefix
|\childdocforwardprefix[|\textit{prefix}|]{|\textit{dest}|}|
is accomplished by:
%
\begin{center}
\begin{tabular}{l}
|{\edef\jobname{\scantokens\expandafter{\jobname\noexpand}}|\\
|\def\redirectjob |\textit{prefix}|#1~~~{\gdef\jobname{|\textit{dest}|#1}}|\\
|\expandafter\redirectjob\jobname~~~}\input{\jobname}|
\end{tabular}
\end{center}

In an alternative approach,
child documents can be compiled by a specific command line
without additional code or specific definitions:
%
\begin{center}
|... -jobname "|\textit{target}|" "|[\textit{flags}]%
|\includeonly{|\textit{dest}|}\input{|\textit{main}|}"|
\end{center}
%

%%%%%%%%%%%%%%%%%%%%%%%%%%%%%%%%%%%%%%%%%%%%%%%%%%%%%%%%%%%%%%%%%%%%%%%%%%%%%%%%
%%%%%%%%%%%%%%%%%%%%%%%%%%%%%%%%%%%%%%%%%%%%%%%%%%%%%%%%%%%%%%%%%%%%%%%%%%%%%%%%
\section{Information}

%%%%%%%%%%%%%%%%%%%%%%%%%%%%%%%%%%%%%%%%%%%%%%%%%%%%%%%%%%%%%%%%%%%%%%%%%%%%%%%%
\subsection{Copyright}

Copyright \copyright{} 2017--2018 Niklas Beisert

This work may be distributed and/or modified under the
conditions of the \LaTeX{} Project Public License, either version 1.3
of this license or (at your option) any later version.
The latest version of this license is in
  \url{http://www.latex-project.org/lppl.txt}
and version 1.3 or later is part of all distributions of \LaTeX{}
version 2005/12/01 or later.

This work has the LPPL maintenance status `maintained'.

The Current Maintainer of this work is Niklas Beisert.

This work consists of the files |README.txt|, |childdoc.ins| and |childdoc.dtx|
as well as the derived files |childdoc.def|, |cdocsamp.tex|
with |cdocsch1.tex|, |cdocsch2.tex|, |cdocspt3.tex|, |cdocspt4.tex|,
|cdocsdrf.tex|, |cdocsfn1.tex|, |cdocsfn2.tex|
as well as |childdoc.pdf|.

%%%%%%%%%%%%%%%%%%%%%%%%%%%%%%%%%%%%%%%%%%%%%%%%%%%%%%%%%%%%%%%%%%%%%%%%%%%%%%%%
\subsection{Files and Installation}

The package consists of the files:
%
\begin{center}
\begin{tabular}{ll}
    |README.txt|   & readme file \\
    |childdoc.ins| & installation file \\
    |childdoc.dtx| & source file \\
    |childdoc.def| & definition file \\
    |cdocsamp.tex| & sample main file \\
    |cdocsch1.tex| & sample include file \\
    |cdocsch2.tex| & sample include file \\
    |cdocspt3.tex| & sample part file \\
    |cdocspt4.tex| & sample part file \\
    |cdocsdrf.tex| & sample redirection file \\
    |cdocsfn1.tex| & sample redirection file \\
    |cdocsfn2.tex| & sample redirection file \\
    |childdoc.pdf| & manual
\end{tabular}
\end{center}
%
The distribution consists of the files
|README.txt|, |childdoc.ins| and |childdoc.dtx|.
%
\begin{itemize}
\item
Run (pdf)\LaTeX{} on |childdoc.dtx|
to compile the manual |childdoc.pdf| (this file).
\item
Run \LaTeX{} on |childdoc.ins| to create the definitions file |childdoc.def|
and the sample |cdocsamp.tex| with include files
|cdocsch1.tex|, |cdocsch2.tex|, |cdocspt3.tex|, |cdocspt4.tex|,
|cdocsdrf.tex|, |cdocsfn1.tex|, |cdocsfn2.tex|.
Then copy the file |childdoc.def| to an appropriate directory of your \LaTeX{}
distribution, e.g.\ \textit{texmf-root}|/tex/latex/childdoc|.
\end{itemize}

%%%%%%%%%%%%%%%%%%%%%%%%%%%%%%%%%%%%%%%%%%%%%%%%%%%%%%%%%%%%%%%%%%%%%%%%%%%%%%%%
\subsection{Related CTAN Packages}

There are several other packages which offer a similar functionality:
%
\begin{itemize}
\item
The packages
\href{http://ctan.org/pkg/docmute}{\textsf{docmute}},
\href{http://ctan.org/pkg/includex}{\textsf{includex}} and
\href{http://ctan.org/pkg/standalone}{\textsf{standalone}}
provide commands to include only the document body of
a child file thus allowing both files to be compiled individually.
\item
The packages \href{http://ctan.org/pkg/subdocs}{\textsf{subdocs}}
and \href{http://ctan.org/pkg/subfiles}{\textsf{subfiles}}
provide structures in which the main and child documents can be
encapsulated and allowing them to be compiled individually.
The inclusion mechanism is different from the conventional |\include|.
\item
The package \href{http://ctan.org/pkg/combine}{\textsf{combine}}
is an elaborate solution to combine several documents into one.
\end{itemize}
%
See also the CTAN topic \href{http://ctan.org/topic/subdocs}{\textsf{subdocs}}
for further related packages.
The present package differs from the above solutions in that
a document structure constructed with the conventional |\include| mechanism
just needs two extra commands at the top of every file
such that all constituent files can be compiled individually.

%%%%%%%%%%%%%%%%%%%%%%%%%%%%%%%%%%%%%%%%%%%%%%%%%%%%%%%%%%%%%%%%%%%%%%%%%%%%%%%%
%\subsection{Feature Suggestions}
%
%The following is a list of features which may be useful for future
%versions of this package:
%%
%\begin{itemize}
%\item
%\ldots
%\end{itemize}

%%%%%%%%%%%%%%%%%%%%%%%%%%%%%%%%%%%%%%%%%%%%%%%%%%%%%%%%%%%%%%%%%%%%%%%%%%%%%%%%
\subsection{Revision History}

%%%%%%%%%%%%%%%%%%%%%%%%%%%%%%%%%%%%%%%%
\paragraph{v2.0:} 2018/12/30

\begin{itemize}
\item
immediate forward processing
\item
added |\childdocby| mechanism
\item
manual restructured
\end{itemize}

%%%%%%%%%%%%%%%%%%%%%%%%%%%%%%%%%%%%%%%%
\paragraph{v1.6:} 2018/01/17

\begin{itemize}
\item
application for development of include files
\item
corrections to manual
\end{itemize}

%%%%%%%%%%%%%%%%%%%%%%%%%%%%%%%%%%%%%%%%
\paragraph{v1.5:} 2017/05/21

\begin{itemize}
\item
more complete structuring introduced
\item
|\childdocof| introduced
\item
|\childdoc| renamed to |\childdocmain|
\item
|\childredirect| renamed to |\childdocforward| and |\childdocforwardprefix|
and functionality expanded
\end{itemize}

%%%%%%%%%%%%%%%%%%%%%%%%%%%%%%%%%%%%%%%%
\paragraph{v1.0:} 2017/04/27

\begin{itemize}
\item
manual and install package
\item
first version published on CTAN
\end{itemize}

%%%%%%%%%%%%%%%%%%%%%%%%%%%%%%%%%%%%%%%%
\paragraph{v0.6:} 2017/04/26

\begin{itemize}
\item
redirection mechanism added
\end{itemize}

%%%%%%%%%%%%%%%%%%%%%%%%%%%%%%%%%%%%%%%%
\paragraph{v0.5:} 2017/04/26

\begin{itemize}
\item
functionality in definition file
\end{itemize}


%%%%%%%%%%%%%%%%%%%%%%%%%%%%%%%%%%%%%%%%%%%%%%%%%%%%%%%%%%%%%%%%%%%%%%%%%%%%%%%%
%%%%%%%%%%%%%%%%%%%%%%%%%%%%%%%%%%%%%%%%%%%%%%%%%%%%%%%%%%%%%%%%%%%%%%%%%%%%%%%%
%%%%%%%%%%%%%%%%%%%%%%%%%%%%%%%%%%%%%%%%%%%%%%%%%%%%%%%%%%%%%%%%%%%%%%%%%%%%%%%%
\appendix

\settowidth\MacroIndent{\rmfamily\scriptsize 000\ }

 \DocInput{childdoc.dtx}

\end{document}
%</driver>
% \fi
%
% %%%%%%%%%%%%%%%%%%%%%%%%%%%%%%%%%%%%%%%%%%%%%%%%%%%%%%%%%%%%%%%%%%%%%%%%%%%%%%
% %%%%%%%%%%%%%%%%%%%%%%%%%%%%%%%%%%%%%%%%%%%%%%%%%%%%%%%%%%%%%%%%%%%%%%%%%%%%%%
% \section{Sample}
%\iffalse
%<*samplemain>
%\fi
%
% The following presents a sample document
% with two chapters, two parts, a title page,
% a compile flag as well as three forwarding files to set the flag.
% It consists of eight |.tex| files:
% \begin{center}
% \begin{tabular}{ll}
% |cdocsamp.tex|&main file\\
% |cdocsch1.tex|&include file for chapter 1\\
% |cdocsch2.tex|&include file for chapter 2\\
% |cdocspt3.tex|&include file for part 3\\
% |cdocspt4.tex|&include file for part 4\\
% |cdocsdrf.tex|&forwarding file for main file in draft mode\\
% |cdocsfi1.tex|&forwarding file for final version of chapter 1\\
% |cdocsfi2.tex|&forwarding file for final version of chapter 2\\
% \end{tabular}
% \end{center}
% Each of the eight files can be compiled directly by the \LaTeX{} compiler.
%
% %%%%%%%%%%%%%%%%%%%%%%%%%%%%%%%%%%%%%%
% \paragraph{Main File.}
%
% The main file is called |cdocsamp.tex|.
%
% Load the \textsf{childdoc} definitions and
% declare the filename for the main document:
%    \begin{macrocode}
\input{childdoc.def}
\childdocmain{}
%    \end{macrocode}

% Optional override for |\version| flag:
%    \begin{macrocode}
%%\ifchilddoc\else\providecommand{\version}{draft}\fi
%    \end{macrocode}

% Define the default values for the |\version| flag
% (|final| for the main file and |draft| for childs):
%    \begin{macrocode}
\ifchilddoc
\providecommand{\version}{draft}
\else
\providecommand{\version}{final}
\fi
%    \end{macrocode}

% Load the standard document class:
%    \begin{macrocode}
\documentclass[12pt]{article}
%    \end{macrocode}

% Start the document body:
%    \begin{macrocode}
\begin{document}
%    \end{macrocode}

% Declare a title page.
% Print title, part of document being processed and version flag:
%    \begin{macrocode}
\addtocounter{page}{-1}
\begin{center}
{\LARGE\bfseries{}childdoc example\par}
\vspace{1cm}
\ifchilddoc
\ifchilddocmanual part\else chapter\fi:
`\childdocname' of `\childdocjob'\par
\else
main document: `\childdocjob'\par
\fi
version: \version\par
\end{center}
\newpage
%    \end{macrocode}

% Manually include selected file,
% otherwise process as usual:
%    \begin{macrocode}
\ifchilddocmanual
\section*{part `\childdocname'}
\input{\childdocname}
\else
%    \end{macrocode}

% Include the two chapters:
%    \begin{macrocode}
\include{cdocsch1}
\include{cdocsch2}
%    \end{macrocode}

% Include the two parts unless only chapters should be displayed:
%    \begin{macrocode}
\ifchilddoc\else
\section{part three}
\input{cdocspt3}
\section{part four}
\input{cdocspt4}
\fi
%    \end{macrocode}

% Process as usual until here:
%    \begin{macrocode}
\fi
%    \end{macrocode}

% End of document body:
%    \begin{macrocode}
\end{document}
%    \end{macrocode}
%\iffalse
%</samplemain>
%\fi
%
% %%%%%%%%%%%%%%%%%%%%%%%%%%%%%%%%%%%%%%
% \paragraph{Chapter Include Files.}
%
% The include files are called |cdocsch1.tex| and |cdocsch2.tex|.
%
%\iffalse
%<*samplechap1|samplechap2>
%\fi

% Optional override for |\version| flag:
%    \begin{macrocode}
%%\providecommand{\version}{final}
%    \end{macrocode}

% Include the main document:
%    \begin{macrocode}
\input{childdoc.def}
\childdocof{cdocsamp}
%    \end{macrocode}

%\iffalse
%</samplechap1|samplechap2>
%\fi
%
%\iffalse
%<*samplechap1>
%\fi
% Some text for chapter 1:
%    \begin{macrocode}
\section{one}
some text in chapter one
%    \end{macrocode}

%\iffalse
%</samplechap1>
%\fi
% Some text for chapter 2:
%\iffalse
%<*samplechap2>
%\fi
%    \begin{macrocode}
\section{two}
more text in chapter two
%    \end{macrocode}

%\iffalse
%</samplechap2>
%\fi
%
% %%%%%%%%%%%%%%%%%%%%%%%%%%%%%%%%%%%%%%
% \paragraph{Part Include Files.}
%
% The include files are called |cdocspt3.tex| and |cdocspt4.tex|.
%
%\iffalse
%<*samplepart3|samplepart4>
%\fi

% Optional override for |\version| flag:
%    \begin{macrocode}
%%\providecommand{\version}{final}
%    \end{macrocode}

% Include the main document:
%    \begin{macrocode}
\input{childdoc.def}
\childdocby{cdocsamp}
%    \end{macrocode}

%\iffalse
%</samplepart3|samplepart4>
%\fi
%
%\iffalse
%<*samplepart3>
%\fi
% Some text for part 3:
%    \begin{macrocode}
some text in part three
%    \end{macrocode}

%\iffalse
%</samplepart3>
%\fi
% Some text for part 4:
%\iffalse
%<*samplepart4>
%\fi
%    \begin{macrocode}
more text in part four
%    \end{macrocode}

%\iffalse
%</samplepart4>
%\fi
%
% %%%%%%%%%%%%%%%%%%%%%%%%%%%%%%%%%%%%%%
% \paragraph{Forwarding for a Complete Draft.}
%
% The following forwarding file |cdocsdrf.tex|
% compiles the main document in draft mode:
%\iffalse
%<*sampledraft>
%\fi
%    \begin{macrocode}
\def\version{draft}
\input{childdoc.def}
\childdocforward{cdocsamp}
%    \end{macrocode}

%\iffalse
%</sampledraft>
%\fi
%
% %%%%%%%%%%%%%%%%%%%%%%%%%%%%%%%%%%%%%%
% \paragraph{Forwarding for Final Version of the Chapters.}
%
% The following forwarding files |cdocsfn1.tex| and |cdocsfn2.tex|
% (with identical content)
% compile the final versions of the child documents
% |cdocsch1.tex| and |cdocsch2.tex|, respectively:
%\iffalse
%<*samplefinal>
%\fi
%    \begin{macrocode}
\def\version{final}
\input{childdoc.def}
\childdocforwardprefix[cdocsamp]{cdocsfn}{cdocsch}
%    \end{macrocode}

%\iffalse
%</samplefinal>
%\fi
%
% %%%%%%%%%%%%%%%%%%%%%%%%%%%%%%%%%%%%%%
% \paragraph{Command Line Processing.}
%
% The following three command lines generate the output files
% |cdocscld|, |cdocscl1| and |cdocscl2|
% which should be identical to
% |cdocsdrf|, |cdocsch1| and |cdocsfn2|, respectively:
% \begin{center}
% \begin{tabular}{l}
% |latex -jobname cdocscld \|\\
% |  "\def\version{draft}\input{childdoc.def}\childdocforward{cdocsamp}"|\\
% |latex -jobname cdocscl1 \|\\
% |  "\input{childdoc.def}\childdocforward[cdocsamp]{cdocsch1}"|\\
% |latex -jobname cdocscl2 \|\\
% |  "\def\version{final}\input{childdoc.def}\childdocforward{cdocsch2}"|
% \end{tabular}
% \end{center}
% Note that the trailing backslash on each first line
% merely continues the input to the second line
% (for convenient cut ant paste).
% Furthermore, the command |latex| can be replaced by any
% of its alternative versions such as |pdflatex|.
%
% %%%%%%%%%%%%%%%%%%%%%%%%%%%%%%%%%%%%%%%%%%%%%%%%%%%%%%%%%%%%%%%%%%%%%%%%%%%%%%
% %%%%%%%%%%%%%%%%%%%%%%%%%%%%%%%%%%%%%%%%%%%%%%%%%%%%%%%%%%%%%%%%%%%%%%%%%%%%%%
% \section{Implementation}
%\iffalse
%<*package>
%\fi
%
% This section describes the definitions file |childdoc.def|.

% The definitions cannot be loaded using |\usepackage| or |\RequirePackage|
% which has a mechanism to prevent loading a style file more than once.
% When loading the definitions by means of |\input|
% multiple instances have to be prevented manually:
%\iffalse
%This code needs to be before the `\ProvidesFile' directive
%which is defined at the beginning of this file.
%Therefore it is also placed there and commented out here.
%</package>
%<*discard>
%\fi
%    \begin{macrocode}
\ifdefined\childdocmain\endinput\fi
%    \end{macrocode}
%\iffalse
%</discard>
%<*package>
%\fi
%
% \macro{\ifchilddoc}
% \macro{\ifchilddocmanual}
% The conditional |\ifchilddoc| tells whether a
% child (true) or main (false) document is being compiled.
% The conditional |\ifchilddocmanual| tells whether
% the |\includeonly| mechanism is used (false) or
% the selection of child files must be performed manually (true).
% The definitions initialise to false:
%    \begin{macrocode}
\newif\ifchilddoc
\newif\ifchilddocmanual
%    \end{macrocode}

% \macro{\childdocname}
% \macro{\childdocjob}
% The macro |\childdocname| stores the name of the main document
% to be compiled. The macro |\childdocjob| stores the name of
% the document on which the \LaTeX{} compiler was originally invoked.
% The content of |\jobname| cannot be compared
% to filenames specified in the source due to different catcodes.
% The following code rescans |\jobname|, stores the result
% in |\childdocname| and saves a copy in |\childdocjob|:
%    \begin{macrocode}
\edef\childdocname{\scantokens\expandafter{\jobname\noexpand}}
\let\childdocjob\childdocname
%    \end{macrocode}

% \macro{\childdocdisable}
% The macro |\childdocdisable| prevents the main file
% from being processed more than once.
% At this stage, the main document command |\childdocmain|
% is assumed to be called once again where it should do nothing.
% Any subsequent call to it should prevent
% a secondary processing of the main document
% It overwrites the forwarding commands
% |\childdocof| and |\childdocforward|
% with empty macros to prevent further inclusions of the main document:
%    \begin{macrocode}
\newcommand{\childdocdisable}
{
  \renewcommand{\childdocmain}[1]{\renewcommand{\childdocmain}[1]{\endinput}}
  \renewcommand{\childdocof}[1]{}
  \renewcommand{\childdocby}[2][]{}
  \renewcommand{\childdocforward}[2][]{}
  \renewcommand{\childdocdisable}{}
}
%    \end{macrocode}

% \macro{\childdocmain}
% The macro |\childdocmain| is to be called at the top of the main file
% with nothing or the main filename (without extension) as argument.
% First, it breaks loops.
% If the argument is not empty and does not match |\childdocname|
% (which is set by the first inclusion of |childdoc.def|),
% |\ifchilddoc| is set to true, |\includeonly| is applied to the child file
% and |\jobname| is set to the main file
% (for proper handling of |.aux| files):
%    \begin{macrocode}
\newcommand{\childdocmain}[1]
{
  \childdocdisable\childdocmain{}
  \if?#1?\else
    \begingroup
      \def\childdoctmp{#1}
      \ifx\childdoctmp\childdocname
        \def\childdoctmp{}
      \else
        \def\childdoctmp
        {
          \childdoctrue
          \includeonly{\childdocname}
          \def\childdocjob{#1}
          \def\jobname{#1}
        }
      \fi
      \expandafter
    \endgroup
    \childdoctmp
  \fi
}
%    \end{macrocode}

% \macro{\childdocof}
% The command |\childdocof| redirects
% compilation to the main file |#1|.
%    \begin{macrocode}
\newcommand{\childdocof}[1]
{
  \childdocdisable
  \childdoctrue
  \includeonly{\childdocname}
  \def\jobname{#1}
  \def\childdocjob{#1}
  \input{#1}
}
%    \end{macrocode}

% \macro{\childdocby}
% The command |\childdocby| ....
%    \begin{macrocode}
\newcommand{\childdocby}[2][]
{
  \childdocdisable
  \childdoctrue
  \childdocmanualtrue
  \if?#1?\else
    \def\jobname{#2}
  \fi
  \def\childdocjob{#2}
  \input{#2}
  \endinput
}
%    \end{macrocode}

% \macro{\childdocforward}
% The command |\childdocforward| redirects
% compilation to the main file or
% (if the optional argument is given) a child file.
% Parameters are set as if the main file
% or a child file starting with |\childdocof| was compiled.
% Then compilation is handed over to the main file:
%    \begin{macrocode}
\newcommand{\childdocforward}[2][]
{
  \begingroup
    \if?#1?
      \def\childdoctmp
      {
        \def\childdocname{#2}
        \def\childdocjob{#2}
        \def\jobname{#2}
        \input{#2}
        \endinput
      }
    \else
      \def\childdoctmp
      {
        \childdocdisable
        \def\childdocname{#2}
        \childdoctrue
        \includeonly{#2}
        \def\childdocjob{#1}
        \def\jobname{#1}
        \input{#1}
        \endinput
      }
    \fi
    \expandafter
  \endgroup
  \childdoctmp
}
%    \end{macrocode}

% \macro{\childdocforwardprefix}
% The command |\childdocforwardprefix| redirects
% compilation to the main or a child file by means of a pattern.
% The prefix |#1| in the current filename is replaced by |#2|
% and the suffix of the current filename is kept
% (it is assumed that the filename does not contain the substring `|~~~|'
% which is used as a delimiter).
% Compilation is handed over to the new file by |\childdocforward|:
%    \begin{macrocode}
\newcommand{\childdocforwardprefix}[3][]
{
  \begingroup
    \def\childdocextract #2##1~~~{\def\childdoctmp{\childdocforward[#1]{#3##1}}}
    \expandafter\childdocextract\childdocname~~~
    \expandafter
  \endgroup
  \childdoctmp
}
%    \end{macrocode}

% \macro{\childdoc}
% The deprecated macro |\childdoc| is a legacy version of |\childdocmain|:
%    \begin{macrocode}
\newcommand{\childdoc}{\childdocmain}
%    \end{macrocode}

% \macro{\childdocredirect}
% The deprecated macro |\childdocredirect| is a legacy version
% of |\childdocforward| and |\childdocforwardprefix|:
%    \begin{macrocode}
\newcommand{\childdocredirect}[2][]
{
  \begingroup
    \if?#1?
      \def\childdoctmp{\childdocforward{#2}}
    \else
      \def\childdoctmp{\childdocforwardprefix{#1}{#2}}
    \fi
    \expandafter
  \endgroup
  \childdoctmp
}
%    \end{macrocode}

%\iffalse
%</package>
%\fi
%
\endinput
\childdocforward[|\textit{main}|]{|\textit{dest}|}"|
\end{center}
%
Here \textit{target} is the name of the output file,
\textit{main} is the name of the main file
and \textit{dest} is the name of the main or child file to be processed
(all filenames without extensions).
The optional argument \textit{main} can be omitted
if \textit{main} matches \textit{dest}.
Optionally, compilation \textit{flags} can be defined via |\def| commands.
This command line makes the \TeX{} engine believe
it is compiling the file \textit{target}
whose content is specified as the latter parameter.
The provided code then forwards the processing to
\textit{main} or \textit{dest} as described in \secref{sec:forward}.

%%%%%%%%%%%%%%%%%%%%%%%%%%%%%%%%%%%%%%%%%%%%%%%%%%%%%%%%%%%%%%%%%%%%%%%%%%%%%%%%
\subsection{Include by Input}
\label{sec:input}

Including child documents by |\include| has some restrictions by design.
Most notably, the content of a child document always occupies
its own set of pages; pages cannot be shared between child documents.
Usually, this behaviour makes perfect sense
because each child document contain an essential part of the document.
However, in some situations it may be desirable to compose
a document from a collection of parts
without having mandatory page breaks between then.
For this case, the package
provides a mechanism to include parts
by |\input| which can also be processed individually.
However, by construction this mechanism
requires manual handling of the content to be output.

%%%%%%%%%%%%%%%%%%%%%%%%%%%%%%%%%%%%%%%%
\DescribeMacro{\ifchilddocmanual}
The main file should be prepared as usual, see \secref{sec:include}.
However, the document body must make a distinction
between processing of an individual part and of the main document, e.g.:
%
\begin{center}
\begin{tabular}{l}
|\ifchilddocmanual|\\
|\input{\childdocname}|\\
|\||else|\\
\textit{document body with }|\input{|\textit{part}|}|\\
|\||fi|
\end{tabular}
\end{center}
%
The conditional |\ifchilddocmanual| is true whenever
a part to be included by |\input| is being compiled,
and the name of the part is stored in |\childdocname|.

%%%%%%%%%%%%%%%%%%%%%%%%%%%%%%%%%%%%%%%%
\DescribeMacro{\childdocby}
Each part to be included by |\input| should start with:
%
\begin{center}
\begin{tabular}{l}
|% \iffalse
%
% childdoc.dtx Copyright (C) 2017-2018 Niklas Beisert
%
% This work may be distributed and/or modified under the
% conditions of the LaTeX Project Public License, either version 1.3
% of this license or (at your option) any later version.
% The latest version of this license is in
%   http://www.latex-project.org/lppl.txt
% and version 1.3 or later is part of all distributions of LaTeX
% version 2005/12/01 or later.
%
% This work has the LPPL maintenance status `maintained'.
%
% The Current Maintainer of this work is Niklas Beisert.
%
% This work consists of the files childdoc.dtx and childdoc.ins
% and the derived files childdoc.def and cdocsamp.tex with
% cdocsch1.tex, cdocsch2.tex, cdocsdrf.tex, cdocsfn1.tex, cdocsfn2.tex.
%
%<package>\ifdefined\childdocmain\endinput\fi
%<package>\ProvidesFile{childdoc.def}[2018/12/30 v2.0 child document driver]
%<samplemain>\ProvidesFile{cdocsamp.tex}[2018/12/30 v2.0 sample for childdoc]
%<*driver>
%\ProvidesFile{childdoc.drv}[2018/12/30 v2.0 childdoc reference manual file]
\PassOptionsToClass{10pt,a4paper}{article}
\documentclass{ltxdoc}

\usepackage[margin=35mm]{geometry}
\usepackage{hyperref}
\usepackage{hyperxmp}
\usepackage[usenames]{color}

\hypersetup{colorlinks=true}
\hypersetup{pdfstartview=FitH}
\hypersetup{pdfpagemode=UseNone}
\hypersetup{pdfsource={}}
\hypersetup{pdflang={en-UK}}
\hypersetup{pdfcopyright={Copyright 2017-2018 Niklas Beisert.
  This work may be distributed and/or modified under the
  conditions of the LaTeX Project Public License, either version 1.3
  of this license or (at your option) any later version.}}
\hypersetup{pdflicenseurl={http://www.latex-project.org/lppl.txt}}
\hypersetup{pdfcontactaddress={ETH Zurich, ITP, HIT K,
  Wolfgang-Pauli-Strasse 27}}
\hypersetup{pdfcontactpostcode={8093}}
\hypersetup{pdfcontactcity={Zurich}}
\hypersetup{pdfcontactcountry={Switzerland}}
\hypersetup{pdfcontactemail={nbeisert@itp.phys.ethz.ch}}
\hypersetup{pdfcontacturl={http://people.phys.ethz.ch/\xmptilde nbeisert/}}

\newcommand{\secref}[1]{\hyperref[#1]{section \ref*{#1}}}

\parskip1ex
\parindent0pt
\let\olditemize\itemize
\def\itemize{\olditemize\parskip0pt}

\begin{document}

\title{The \textsf{childdoc} Package}
\hypersetup{pdftitle={The childdoc Package}}
\author{Niklas Beisert\\[2ex]
  Institut f\"ur Theoretische Physik\\
  Eidgen\"ossische Technische Hochschule Z\"urich\\
  Wolfgang-Pauli-Strasse 27, 8093 Z\"urich, Switzerland\\[1ex]
  \href{mailto:nbeisert@itp.phys.ethz.ch}
  {\texttt{nbeisert@itp.phys.ethz.ch}}}
\hypersetup{pdfauthor={Niklas Beisert}}
\hypersetup{pdfsubject={Manual for the LaTeX2e Package childdoc}}
\date{30 December 2018, \textsf{v2.0}}
\maketitle

\begin{abstract}\noindent
\textsf{childdoc} is a \LaTeXe{} package
that enables the direct compilation
of document sections included by |\include|
to individual files.
\end{abstract}

\begingroup
\parskip0ex
\tableofcontents
\endgroup

%%%%%%%%%%%%%%%%%%%%%%%%%%%%%%%%%%%%%%%%%%%%%%%%%%%%%%%%%%%%%%%%%%%%%%%%%%%%%%%%
%%%%%%%%%%%%%%%%%%%%%%%%%%%%%%%%%%%%%%%%%%%%%%%%%%%%%%%%%%%%%%%%%%%%%%%%%%%%%%%%
\section{Introduction}

\LaTeX{} provides a mechanism to structure a large document (such as a book)
into a main file and several child files (containing the chapters)
using the |\include| command.
This mechanism is beneficial for documents
which span hundreds of pages in order to
make the source file(s) more manageable.
Moreover, compilation can be restricted to
selected child files by means of the |\includeonly| command.
The latter feature can be used to reduce the compilation time while editing
(this was significantly more useful in the earlier days of \LaTeX{})
or to generate a smaller document which is easier to navigate.
Another application of |\includeonly| is to generate
documents consisting of selected parts of the complete document.

However, there are a few drawbacks of the plain |\include| mechanism:
\begin{itemize}
\item
The child files cannot be compiled on their own,
they can only be compiled via the main file.
A naive editing environment
(such as a text editor with an option
to have the current file processed by \LaTeX)
may require one to switch to the main file before compiling;
attempting to compile the child file produces errors.
\item
The main file must be modified (each time)
to adjust the |\includeonly| command
to the present needs. This easily leaves the main file in a messy state.
\item
The generated document will always carry the filename
of the main document. This is inconvenient if
several child files are to be compiled and
to be kept for distribution.
\end{itemize}

The present package provides a simple interface
to make child files individually compilable by \LaTeX{}.
Compiling a child file then has the same effect as compiling
the main file with an |\includeonly| command
to select the appropriate child.
Moreover the generated document will carry the name of the child
rather than the main file.
This resolves all three above issues.

This feature is meant to make the editing of books,
thesis documents and lecture notes somewhat more convenient.
However, the package can also be used efficiently for
composing a series of documents (such as exercise sheets)
which are typically distributed individually.
It then assists the author in generating the individual documents
(potentially in different versions)
as well as a document containing the collected series.
Another application is in developing style files
or other kinds of included material
where compilation of the style file could redirect
to a sample or test file.

%%%%%%%%%%%%%%%%%%%%%%%%%%%%%%%%%%%%%%%%%%%%%%%%%%%%%%%%%%%%%%%%%%%%%%%%%%%%%%%%
%%%%%%%%%%%%%%%%%%%%%%%%%%%%%%%%%%%%%%%%%%%%%%%%%%%%%%%%%%%%%%%%%%%%%%%%%%%%%%%%
\section{Usage}

First of all, the package \textsf{childdoc} is \emph{not} a standard
\LaTeXe{} |.sty| style file! Therefore it needs to be invoked in
a non-standard way.

%%%%%%%%%%%%%%%%%%%%%%%%%%%%%%%%%%%%%%%%%%%%%%%%%%%%%%%%%%%%%%%%%%%%%%%%%%%%%%%%
\subsection{Included Files}
\label{sec:include}

%%%%%%%%%%%%%%%%%%%%%%%%%%%%%%%%%%%%%%%%
\DescribeMacro{\childdocmain}
To use the package, add the commands
\begin{center}
\begin{tabular}{l}
|\input{childdoc.def}|\\
|\childdocmain{}|\\
\end{tabular}
\end{center}
at the very top of the main \LaTeX{} file,
in particular \emph{before} the |\documentclass| statement!
The argument of |\childdocmain| should be left empty
(but it must be present).

%%%%%%%%%%%%%%%%%%%%%%%%%%%%%%%%%%%%%%%%
\DescribeMacro{\childdocof}
Furthermore, add the commands
\begin{center}
\begin{tabular}{l}
|\input{childdoc.def}|\\
|\childdocof{|\textit{main}|}|\\
\end{tabular}
\end{center}
at the top of every child file \textit{child}
which is included by |\include{|\textit{child}|}|
from within the main file
(or at least for those files to be compiled individually).
The argument \textit{main} must be the filename of the main file.

There are a couple of
considerations in setting up the main and child documents:

%%%%%%%%%%%%%%%%%%%%%%%%%%%%%%%%%%%%%%%%
\paragraph{Restrictions.}

Please note the following restrictions:
\begin{itemize}
\item
|\childdocmain| must be called with one argument \textit{main}
to ensure compatibility with earlier version of the package.
It must either be empty (|\childdocmain{}|)
or precisely match the filename of the main file in which it is specified.
See \secref{sec:detection} for further information.
\item
The filename \textit{main} must be specified without the |.tex| extension.
\item
The filename \textit{main} is case sensitive
(even in case-insensitive file systems)
due to internal string comparison.
\item
The argument \textit{main} should be fully expanded, it cannot be a macro.
\item
Subdirectories and special characters should be avoided in filenames.
\item
The command |\childdocmain{|\textit{main}|}| must be followed by a whitespace.
It should not be followed immediately by another command
or by a comment mark `|%|'.
This is because the \TeX{} parser reads the token immediately following
the argument of |\childdocmain| and puts it
at the beginning of every child section;
however, a white\-space is ignored.
\end{itemize}

%%%%%%%%%%%%%%%%%%%%%%%%%%%%%%%%%%%%%%%%
\paragraph{Content of Main File.}

It is advisable to place all content in the child files included by |\include|.
Any output contained in the main file will appear in all child documents
unless suppressed manually;
it cannot be suppressed automatically by the |\includeonly| directive
and thus should normally be avoided.
A method to include some content in the main file
by means of conditional processing is described in \secref{sec:conditional}.

%%%%%%%%%%%%%%%%%%%%%%%%%%%%%%%%%%%%%%%%
\paragraph{Page Numbering.}

When only a part of the document is compiled,
the appropriate numbering of pages
(as well as other status parameters)
is determined from the |.aux| files.
The latter contain information from previous passes.
However this information needs to propagate through
all intermediate child documents.
Therefore the page numbering in child documents may well
be inconsistent until the complete document is compiled at least once.

A useful (if unconventional) way to always ensure a consistent
page numbering is to restart the numbering in each child document
and denote the pages by `\textit{child}|.|\textit{page}'
where \textit{child} represents the chapter/section number of the child file.
This can be achieved by the command
|\numberwithin{page}{|\textit{child}|}|
of the \textsf{amsmath} package
where \textit{child} can be |chapter| or |section|
depending on the chosen structuring.
Alternatively, one can modify the macro |\thepage| appropriately
and reset the counter |page| at the start of each child file.

%%%%%%%%%%%%%%%%%%%%%%%%%%%%%%%%%%%%%%%%%%%%%%%%%%%%%%%%%%%%%%%%%%%%%%%%%%%%%%%%
\subsection{Conditional Processing}
\label{sec:conditional}

The package provides a mechanism to compile different versions
of a document. To customise the versions further some conditional processing
can come in handy to distinguish which version is being compiled.
The package provides two macros to describe the compilation context:

%%%%%%%%%%%%%%%%%%%%%%%%%%%%%%%%%%%%%%%%
\DescribeMacro{\ifchilddoc}
The conditional |\ifchilddoc| distinguishes between the compilation of
child documents and the main document:
%
\begin{center}
|\ifchilddoc |\textit{child-code}| |[|\||else |\textit{main-code}]| \||fi|
\end{center}

%%%%%%%%%%%%%%%%%%%%%%%%%%%%%%%%%%%%%%%%
\DescribeMacro{\childdocname}
\DescribeMacro{\childdocjob}
The macro |\childdocname| contains the filename (without extension)
of the main or child file being processed.
Note that |\childdocjob| will always contain the name of the main file.

%%%%%%%%%%%%%%%%%%%%%%%%%%%%%%%%%%%%%%%%
\paragraph{Title Page.}

Conditional processing can be used to include a title or banner page
in the main document when proper precautions are taken.
Importantly, the code in the main file should ensure that the page counter
(as well as other status parameters which are stored in the |.aux| files)
takes the same value after the conditional processing.
Otherwise the page numbers may take divergent values
depending on which part is compiled.

For example, a title page could be declared by:
%
\begin{center}
\begin{tabular}{l}
|\ifchilddoc\||else|\\
|\addtocounter{page}{-1}|\\
\textit{code for title page}\\
|\newpage|\\
|\||fi|
\end{tabular}
\end{center}
%
A banner page for the child documents can be generated by:
%
\begin{center}
\begin{tabular}{l}
|\ifchilddoc|\\
|\addtocounter{page}{-1}|\\
\textit{code for banner page}\\
|\newpage|\\
|\||fi|
\end{tabular}
\end{center}
%
Here one could write a message such as:
\begin{center}
|This is the part \childdocname{} of \childdocjob{}.|
\end{center}

%%%%%%%%%%%%%%%%%%%%%%%%%%%%%%%%%%%%%%%%%%%%%%%%%%%%%%%%%%%%%%%%%%%%%%%%%%%%%%%%
\subsection{Flags}
\label{sec:flags}

The package makes it easy to generate different versions
of the main or child documents.
To this end compilation flags can be defined
and assigned different default values.
They will be particularly useful in conjunction
with the forwarding mechanism described in \secref{sec:forward}.

For example, it may be useful to have a flag |\version|
which can be set to |draft| or |final|.
The document source will contain some conditional code
depending on the value of |\version|.
Suppose further, the flag should default to |final| for the main file
and to |draft| for child files
which is a natural assignment for editing the document.
This is achieved by placing the following code
in the preamble of the main document
(below the |\childdocmain| directive):
%
\begin{center}
\begin{tabular}{l}
|\ifchilddoc|\\
|\providecommand{\version}{draft}|\\
|\||else|\\
|\providecommand{\version}{final}|\\
|\||fi|
\end{tabular}
\end{center}
%
The definition by |\providecommand| makes sure
that previous definitions are not overwritten.
Further statements |\providecommand{\version}{...}|
can thus be added before the above code to override it.

For the main file, one might add a line
(between |\childdocmain| and the above block)
%
\begin{center}
|%\ifchilddoc\||else\providecommand{\version}{draft}\||fi|
\end{center}
%
which can be uncommented to produce a draft version.
Likewise one can add a line to the very top of a child file
(above the |\childdocof{|\textit{main}|}| directive)
%
\begin{center}
|%\providecommand{\version}{final}|
\end{center}
%
which can be uncommented to produce the final version of this child document.

%%%%%%%%%%%%%%%%%%%%%%%%%%%%%%%%%%%%%%%%%%%%%%%%%%%%%%%%%%%%%%%%%%%%%%%%%%%%%%%%
\subsection{Forwarding}
\label{sec:forward}

Different versions of the main or child documents
using compilation flags as described in \secref{sec:flags}
can be (permanently) stored in different files
for convenient compilation, viewing and distribution.
To this end, the package defines a command
to pass on compilation to a different file:

%%%%%%%%%%%%%%%%%%%%%%%%%%%%%%%%%%%%%%%%
\DescribeMacro{\childdocforward}
The command |\childdocforward| redirects processing to
another source file:
%
\begin{center}
\begin{tabular}{l}
|\input{childdoc.def}|\\
|\childdocforward[|\textit{main}|]{|\textit{dest}|}|\\
\end{tabular}
\end{center}
%
The argument \textit{dest} is the destination file
(without extension).
It should be the main file or one of the child files.
Note that further \textsf{childdoc} directives
such as |\childdocof| and |\childdocforward|
in the indicated file will be processed in this form.
The optional argument \textit{main}
passes on directly to the main file \textit{main}
while pretending to compile the child \textit{dest}.
This form behaves as if \textit{dest}
issues |\childdocof{|\textit{main}|}| right away,
and no further \textsf{childdoc} directives will be processed.

%%%%%%%%%%%%%%%%%%%%%%%%%%%%%%%%%%%%%%%%
\DescribeMacro{\...prefix}
In the alternative form |\childdocforwardprefix|,
%
\begin{center}
\begin{tabular}{l}
|\input{childdoc.def}|\\
|\childdocforwardprefix[|\textit{main}|]{|\textit{prefix}|}{|\textit{dest}|}|
\end{tabular}
\end{center}
%
the destination file is determined by a pattern
depending on the current file:
To make this work, the current file must be called
`{\textit{prefix}\hspace{0.2em}\textit{suffix}}'
with \textit{prefix} matching precisely the argument.
Processing is then passed on to the file
`{\textit{dest}\hspace{0.2em}\textit{suffix}}'.
Surely, the same effect is achieved by
directly specifying the
argument `{\textit{dest}\hspace{0.2em}\textit{suffix}}'
in the first form.
However, that requires to set up a different file
for each child. With the alternative form of the command
all these files can have exactly the same content
which simplifies setting them up and maintaining them.

For example, the following file |draft.tex|
with a compilation flag |\version| as described in \secref{sec:flags}
compiles the main document as a draft:
%
\begin{center}
\begin{tabular}{l}
|\def\version{draft}|\\
|\input{childdoc.def}|\\
|\childdocforward{|\textit{main}|}|
\end{tabular}
\end{center}
%
Likewise, the following files |final|\textit{nn}|.tex|
compile the final version of the child document
|child|\textit{nn}|.tex|:
%
\begin{center}
\begin{tabular}{l}
|\def\version{final}|\\
|\input{childdoc.def}|\\
|\childdocforwardprefix{final}{child}|
\end{tabular}
\end{center}
%

Note that when several versions of a main file and/or of each child file
are to be generated, it may be convenient to set up a |Makefile| or
shell script to automatise the process.

%%%%%%%%%%%%%%%%%%%%%%%%%%%%%%%%%%%%%%%%%%%%%%%%%%%%%%%%%%%%%%%%%%%%%%%%%%%%%%%%
\subsection{Command Line Processing}
\label{sec:commandline}

The effect of redirection files can also be achieved by invoking
the \LaTeX{} compiler with a more elaborate command line.
Most conveniently this should be done as part
of a shell script or a |Makefile|.

When using \textsf{childdoc} in the main file, the following
command lines effectively perform a redirection
(note that depending on the shell being used,
backslashes may have to be doubled: `|\|' $\to$ `|\\|'):
%
\begin{center}
|... -jobname "|\textit{target}|" |\\|"|[\textit{flags}]%
|\input{childdoc.def}\childdocforward[|\textit{main}|]{|\textit{dest}|}"|
\end{center}
%
Here \textit{target} is the name of the output file,
\textit{main} is the name of the main file
and \textit{dest} is the name of the main or child file to be processed
(all filenames without extensions).
The optional argument \textit{main} can be omitted
if \textit{main} matches \textit{dest}.
Optionally, compilation \textit{flags} can be defined via |\def| commands.
This command line makes the \TeX{} engine believe
it is compiling the file \textit{target}
whose content is specified as the latter parameter.
The provided code then forwards the processing to
\textit{main} or \textit{dest} as described in \secref{sec:forward}.

%%%%%%%%%%%%%%%%%%%%%%%%%%%%%%%%%%%%%%%%%%%%%%%%%%%%%%%%%%%%%%%%%%%%%%%%%%%%%%%%
\subsection{Include by Input}
\label{sec:input}

Including child documents by |\include| has some restrictions by design.
Most notably, the content of a child document always occupies
its own set of pages; pages cannot be shared between child documents.
Usually, this behaviour makes perfect sense
because each child document contain an essential part of the document.
However, in some situations it may be desirable to compose
a document from a collection of parts
without having mandatory page breaks between then.
For this case, the package
provides a mechanism to include parts
by |\input| which can also be processed individually.
However, by construction this mechanism
requires manual handling of the content to be output.

%%%%%%%%%%%%%%%%%%%%%%%%%%%%%%%%%%%%%%%%
\DescribeMacro{\ifchilddocmanual}
The main file should be prepared as usual, see \secref{sec:include}.
However, the document body must make a distinction
between processing of an individual part and of the main document, e.g.:
%
\begin{center}
\begin{tabular}{l}
|\ifchilddocmanual|\\
|\input{\childdocname}|\\
|\||else|\\
\textit{document body with }|\input{|\textit{part}|}|\\
|\||fi|
\end{tabular}
\end{center}
%
The conditional |\ifchilddocmanual| is true whenever
a part to be included by |\input| is being compiled,
and the name of the part is stored in |\childdocname|.

%%%%%%%%%%%%%%%%%%%%%%%%%%%%%%%%%%%%%%%%
\DescribeMacro{\childdocby}
Each part to be included by |\input| should start with:
%
\begin{center}
\begin{tabular}{l}
|\input{childdoc.def}|\\
|\childdocby{|\textit{main}|}|\\
\end{tabular}
\end{center}
%
The directive |\childdocby| is similar to |\childdocof|
described in \secref{sec:include},
but the subsequent selection of content must be done manually.
To that end, both |\ifchilddoc| and |\ifchilddocmanual|
will be true upon processing of a part,
and the name of the part is stored in |\childdocname|.
Note that |\jobname| will be set to the filename of the current part
so that each part receives an individual |.aux| file
that does not interfere with the |.aux| file(s) of the main document.
This behaviour can be altered by the alternative form
|\childdocby[*]{|\textit{main}|}| (with a non-empty optional argument)
which uses the |.aux| file of the main document
by setting |\jobname| to \textit{main}.

%%%%%%%%%%%%%%%%%%%%%%%%%%%%%%%%%%%%%%%%%%%%%%%%%%%%%%%%%%%%%%%%%%%%%%%%%%%%%%%%
\subsection{Driver Development}
\label{sec:driver}

The \textsf{childdoc} mechanism can also be use for the development
of definition files such as \LaTeX{} styles or classes.
This case differs from the above setup with multiple parts
included by |\include| in that no |\includeonly| should be invoked.
This can be achieved by starting the include file
(before |\ProvidesPackage|) with:
%
\begin{center}
\begin{tabular}{l}
|\input{childdoc.def}|\\
|\childdocforward{|\textit{main}|}|\\
\end{tabular}
\end{center}
%
or alternatively with:
%
\begin{center}
\begin{tabular}{l}
|\input{childdoc.def}|\\
|\childdocby{|\textit{main}|}|\\
\end{tabular}
\end{center}
%
Both forms have slightly different effects as described above.
The main file is prepared as usual, see \secref{sec:include}.

%%%%%%%%%%%%%%%%%%%%%%%%%%%%%%%%%%%%%%%%%%%%%%%%%%%%%%%%%%%%%%%%%%%%%%%%%%%%%%%%
\subsection{Legacy Detection}
\label{sec:detection}

The directive |\childdocmain| in the main file can detect
whether the complete document or merely a child is to be compiled
even without using the directive |\childdocof|.
This method is deprecated because it is less robust
and there is no compelling reason to use it;
it is merely provided for backward compatibility
and it may be removed in future versions.

If the detection mechanism is to be used,
it is mandatory to correctly specify
the filename of the main file as the argument of |\childdocmain|:
%
\begin{center}
\begin{tabular}{l}
|\input{childdoc.def}|\\
|\childdocmain{|\textit{main}|}|\\
\end{tabular}
\end{center}
%
If |\jobname| does not match the argument \textit{main} of |\childdocmain|,
it is assumed that |\jobname| points to the child file to be compiled.
When using |\childdocmain| with the main file specified as argument,
it suffices to start a child file
with just |\input{|\textit{main}|}|
without loading of the package and using |\childdocof|.
If instead all processing is done
with the appropriate \textsf{childdoc} directives,
the argument of \textit{main} of |\childdocmain| can be empty.

An alternative version of the command line processing described
in \secref{sec:commandline} using the detection mechanism reads:
%
\begin{center}
|... -jobname "|\textit{target}|" "|[\textit{flags}]%
[|\def\jobname{|\textit{dest}|}|]|\input{|\textit{main}|}"|
\end{center}

%%%%%%%%%%%%%%%%%%%%%%%%%%%%%%%%%%%%%%%%%%%%%%%%%%%%%%%%%%%%%%%%%%%%%%%%%%%%%%%%
\subsection{Manual Code}
\label{sec:manual}

In case one cannot be certain whether the definitions file |childdoc.def|
is installed on the target \TeX{} distribution
and one prefers not to ship it,
it is conceivable to paste a few relevant commands into the sources.

To that end, drop all statements |\input{childdoc.def}|
and perform the replacements as outlined below.
Instead of |\childdocmain{|\textit{main}|}| add the following code
to the top of the main file:
%
\begin{center}
\begin{tabular}{l}
|\||ifdefined\childdocname\endinput\||fi\newif\ifchilddoc|\\
|\edef\childdocname{\scantokens\expandafter{\jobname\noexpand}}|\\
|\def\childdocmain{|\textit{main}|}\||ifx\childdocmain\childdocname\||else|\\
|\childdoctrue\includeonly{\childdocname}\let\jobname\childdocmain\||fi|\\
\end{tabular}
\end{center}
%
Instead of |\childdocof{|\textit{main}|}| just include the main file
at the top of each child file:
%
\begin{center}
|\input{|\textit{main}|}|
\end{center}
%
A simple redirection |\childdocforward{|\textit{dest}|}| is achieved by:
%
\begin{center}
|\def\jobname{|\textit{dest}|}\input{\jobname}|
\end{center}
%
The redirection with prefix
|\childdocforwardprefix[|\textit{prefix}|]{|\textit{dest}|}|
is accomplished by:
%
\begin{center}
\begin{tabular}{l}
|{\edef\jobname{\scantokens\expandafter{\jobname\noexpand}}|\\
|\def\redirectjob |\textit{prefix}|#1~~~{\gdef\jobname{|\textit{dest}|#1}}|\\
|\expandafter\redirectjob\jobname~~~}\input{\jobname}|
\end{tabular}
\end{center}

In an alternative approach,
child documents can be compiled by a specific command line
without additional code or specific definitions:
%
\begin{center}
|... -jobname "|\textit{target}|" "|[\textit{flags}]%
|\includeonly{|\textit{dest}|}\input{|\textit{main}|}"|
\end{center}
%

%%%%%%%%%%%%%%%%%%%%%%%%%%%%%%%%%%%%%%%%%%%%%%%%%%%%%%%%%%%%%%%%%%%%%%%%%%%%%%%%
%%%%%%%%%%%%%%%%%%%%%%%%%%%%%%%%%%%%%%%%%%%%%%%%%%%%%%%%%%%%%%%%%%%%%%%%%%%%%%%%
\section{Information}

%%%%%%%%%%%%%%%%%%%%%%%%%%%%%%%%%%%%%%%%%%%%%%%%%%%%%%%%%%%%%%%%%%%%%%%%%%%%%%%%
\subsection{Copyright}

Copyright \copyright{} 2017--2018 Niklas Beisert

This work may be distributed and/or modified under the
conditions of the \LaTeX{} Project Public License, either version 1.3
of this license or (at your option) any later version.
The latest version of this license is in
  \url{http://www.latex-project.org/lppl.txt}
and version 1.3 or later is part of all distributions of \LaTeX{}
version 2005/12/01 or later.

This work has the LPPL maintenance status `maintained'.

The Current Maintainer of this work is Niklas Beisert.

This work consists of the files |README.txt|, |childdoc.ins| and |childdoc.dtx|
as well as the derived files |childdoc.def|, |cdocsamp.tex|
with |cdocsch1.tex|, |cdocsch2.tex|, |cdocspt3.tex|, |cdocspt4.tex|,
|cdocsdrf.tex|, |cdocsfn1.tex|, |cdocsfn2.tex|
as well as |childdoc.pdf|.

%%%%%%%%%%%%%%%%%%%%%%%%%%%%%%%%%%%%%%%%%%%%%%%%%%%%%%%%%%%%%%%%%%%%%%%%%%%%%%%%
\subsection{Files and Installation}

The package consists of the files:
%
\begin{center}
\begin{tabular}{ll}
    |README.txt|   & readme file \\
    |childdoc.ins| & installation file \\
    |childdoc.dtx| & source file \\
    |childdoc.def| & definition file \\
    |cdocsamp.tex| & sample main file \\
    |cdocsch1.tex| & sample include file \\
    |cdocsch2.tex| & sample include file \\
    |cdocspt3.tex| & sample part file \\
    |cdocspt4.tex| & sample part file \\
    |cdocsdrf.tex| & sample redirection file \\
    |cdocsfn1.tex| & sample redirection file \\
    |cdocsfn2.tex| & sample redirection file \\
    |childdoc.pdf| & manual
\end{tabular}
\end{center}
%
The distribution consists of the files
|README.txt|, |childdoc.ins| and |childdoc.dtx|.
%
\begin{itemize}
\item
Run (pdf)\LaTeX{} on |childdoc.dtx|
to compile the manual |childdoc.pdf| (this file).
\item
Run \LaTeX{} on |childdoc.ins| to create the definitions file |childdoc.def|
and the sample |cdocsamp.tex| with include files
|cdocsch1.tex|, |cdocsch2.tex|, |cdocspt3.tex|, |cdocspt4.tex|,
|cdocsdrf.tex|, |cdocsfn1.tex|, |cdocsfn2.tex|.
Then copy the file |childdoc.def| to an appropriate directory of your \LaTeX{}
distribution, e.g.\ \textit{texmf-root}|/tex/latex/childdoc|.
\end{itemize}

%%%%%%%%%%%%%%%%%%%%%%%%%%%%%%%%%%%%%%%%%%%%%%%%%%%%%%%%%%%%%%%%%%%%%%%%%%%%%%%%
\subsection{Related CTAN Packages}

There are several other packages which offer a similar functionality:
%
\begin{itemize}
\item
The packages
\href{http://ctan.org/pkg/docmute}{\textsf{docmute}},
\href{http://ctan.org/pkg/includex}{\textsf{includex}} and
\href{http://ctan.org/pkg/standalone}{\textsf{standalone}}
provide commands to include only the document body of
a child file thus allowing both files to be compiled individually.
\item
The packages \href{http://ctan.org/pkg/subdocs}{\textsf{subdocs}}
and \href{http://ctan.org/pkg/subfiles}{\textsf{subfiles}}
provide structures in which the main and child documents can be
encapsulated and allowing them to be compiled individually.
The inclusion mechanism is different from the conventional |\include|.
\item
The package \href{http://ctan.org/pkg/combine}{\textsf{combine}}
is an elaborate solution to combine several documents into one.
\end{itemize}
%
See also the CTAN topic \href{http://ctan.org/topic/subdocs}{\textsf{subdocs}}
for further related packages.
The present package differs from the above solutions in that
a document structure constructed with the conventional |\include| mechanism
just needs two extra commands at the top of every file
such that all constituent files can be compiled individually.

%%%%%%%%%%%%%%%%%%%%%%%%%%%%%%%%%%%%%%%%%%%%%%%%%%%%%%%%%%%%%%%%%%%%%%%%%%%%%%%%
%\subsection{Feature Suggestions}
%
%The following is a list of features which may be useful for future
%versions of this package:
%%
%\begin{itemize}
%\item
%\ldots
%\end{itemize}

%%%%%%%%%%%%%%%%%%%%%%%%%%%%%%%%%%%%%%%%%%%%%%%%%%%%%%%%%%%%%%%%%%%%%%%%%%%%%%%%
\subsection{Revision History}

%%%%%%%%%%%%%%%%%%%%%%%%%%%%%%%%%%%%%%%%
\paragraph{v2.0:} 2018/12/30

\begin{itemize}
\item
immediate forward processing
\item
added |\childdocby| mechanism
\item
manual restructured
\end{itemize}

%%%%%%%%%%%%%%%%%%%%%%%%%%%%%%%%%%%%%%%%
\paragraph{v1.6:} 2018/01/17

\begin{itemize}
\item
application for development of include files
\item
corrections to manual
\end{itemize}

%%%%%%%%%%%%%%%%%%%%%%%%%%%%%%%%%%%%%%%%
\paragraph{v1.5:} 2017/05/21

\begin{itemize}
\item
more complete structuring introduced
\item
|\childdocof| introduced
\item
|\childdoc| renamed to |\childdocmain|
\item
|\childredirect| renamed to |\childdocforward| and |\childdocforwardprefix|
and functionality expanded
\end{itemize}

%%%%%%%%%%%%%%%%%%%%%%%%%%%%%%%%%%%%%%%%
\paragraph{v1.0:} 2017/04/27

\begin{itemize}
\item
manual and install package
\item
first version published on CTAN
\end{itemize}

%%%%%%%%%%%%%%%%%%%%%%%%%%%%%%%%%%%%%%%%
\paragraph{v0.6:} 2017/04/26

\begin{itemize}
\item
redirection mechanism added
\end{itemize}

%%%%%%%%%%%%%%%%%%%%%%%%%%%%%%%%%%%%%%%%
\paragraph{v0.5:} 2017/04/26

\begin{itemize}
\item
functionality in definition file
\end{itemize}


%%%%%%%%%%%%%%%%%%%%%%%%%%%%%%%%%%%%%%%%%%%%%%%%%%%%%%%%%%%%%%%%%%%%%%%%%%%%%%%%
%%%%%%%%%%%%%%%%%%%%%%%%%%%%%%%%%%%%%%%%%%%%%%%%%%%%%%%%%%%%%%%%%%%%%%%%%%%%%%%%
%%%%%%%%%%%%%%%%%%%%%%%%%%%%%%%%%%%%%%%%%%%%%%%%%%%%%%%%%%%%%%%%%%%%%%%%%%%%%%%%
\appendix

\settowidth\MacroIndent{\rmfamily\scriptsize 000\ }

 \DocInput{childdoc.dtx}

\end{document}
%</driver>
% \fi
%
% %%%%%%%%%%%%%%%%%%%%%%%%%%%%%%%%%%%%%%%%%%%%%%%%%%%%%%%%%%%%%%%%%%%%%%%%%%%%%%
% %%%%%%%%%%%%%%%%%%%%%%%%%%%%%%%%%%%%%%%%%%%%%%%%%%%%%%%%%%%%%%%%%%%%%%%%%%%%%%
% \section{Sample}
%\iffalse
%<*samplemain>
%\fi
%
% The following presents a sample document
% with two chapters, two parts, a title page,
% a compile flag as well as three forwarding files to set the flag.
% It consists of eight |.tex| files:
% \begin{center}
% \begin{tabular}{ll}
% |cdocsamp.tex|&main file\\
% |cdocsch1.tex|&include file for chapter 1\\
% |cdocsch2.tex|&include file for chapter 2\\
% |cdocspt3.tex|&include file for part 3\\
% |cdocspt4.tex|&include file for part 4\\
% |cdocsdrf.tex|&forwarding file for main file in draft mode\\
% |cdocsfi1.tex|&forwarding file for final version of chapter 1\\
% |cdocsfi2.tex|&forwarding file for final version of chapter 2\\
% \end{tabular}
% \end{center}
% Each of the eight files can be compiled directly by the \LaTeX{} compiler.
%
% %%%%%%%%%%%%%%%%%%%%%%%%%%%%%%%%%%%%%%
% \paragraph{Main File.}
%
% The main file is called |cdocsamp.tex|.
%
% Load the \textsf{childdoc} definitions and
% declare the filename for the main document:
%    \begin{macrocode}
\input{childdoc.def}
\childdocmain{}
%    \end{macrocode}

% Optional override for |\version| flag:
%    \begin{macrocode}
%%\ifchilddoc\else\providecommand{\version}{draft}\fi
%    \end{macrocode}

% Define the default values for the |\version| flag
% (|final| for the main file and |draft| for childs):
%    \begin{macrocode}
\ifchilddoc
\providecommand{\version}{draft}
\else
\providecommand{\version}{final}
\fi
%    \end{macrocode}

% Load the standard document class:
%    \begin{macrocode}
\documentclass[12pt]{article}
%    \end{macrocode}

% Start the document body:
%    \begin{macrocode}
\begin{document}
%    \end{macrocode}

% Declare a title page.
% Print title, part of document being processed and version flag:
%    \begin{macrocode}
\addtocounter{page}{-1}
\begin{center}
{\LARGE\bfseries{}childdoc example\par}
\vspace{1cm}
\ifchilddoc
\ifchilddocmanual part\else chapter\fi:
`\childdocname' of `\childdocjob'\par
\else
main document: `\childdocjob'\par
\fi
version: \version\par
\end{center}
\newpage
%    \end{macrocode}

% Manually include selected file,
% otherwise process as usual:
%    \begin{macrocode}
\ifchilddocmanual
\section*{part `\childdocname'}
\input{\childdocname}
\else
%    \end{macrocode}

% Include the two chapters:
%    \begin{macrocode}
\include{cdocsch1}
\include{cdocsch2}
%    \end{macrocode}

% Include the two parts unless only chapters should be displayed:
%    \begin{macrocode}
\ifchilddoc\else
\section{part three}
\input{cdocspt3}
\section{part four}
\input{cdocspt4}
\fi
%    \end{macrocode}

% Process as usual until here:
%    \begin{macrocode}
\fi
%    \end{macrocode}

% End of document body:
%    \begin{macrocode}
\end{document}
%    \end{macrocode}
%\iffalse
%</samplemain>
%\fi
%
% %%%%%%%%%%%%%%%%%%%%%%%%%%%%%%%%%%%%%%
% \paragraph{Chapter Include Files.}
%
% The include files are called |cdocsch1.tex| and |cdocsch2.tex|.
%
%\iffalse
%<*samplechap1|samplechap2>
%\fi

% Optional override for |\version| flag:
%    \begin{macrocode}
%%\providecommand{\version}{final}
%    \end{macrocode}

% Include the main document:
%    \begin{macrocode}
\input{childdoc.def}
\childdocof{cdocsamp}
%    \end{macrocode}

%\iffalse
%</samplechap1|samplechap2>
%\fi
%
%\iffalse
%<*samplechap1>
%\fi
% Some text for chapter 1:
%    \begin{macrocode}
\section{one}
some text in chapter one
%    \end{macrocode}

%\iffalse
%</samplechap1>
%\fi
% Some text for chapter 2:
%\iffalse
%<*samplechap2>
%\fi
%    \begin{macrocode}
\section{two}
more text in chapter two
%    \end{macrocode}

%\iffalse
%</samplechap2>
%\fi
%
% %%%%%%%%%%%%%%%%%%%%%%%%%%%%%%%%%%%%%%
% \paragraph{Part Include Files.}
%
% The include files are called |cdocspt3.tex| and |cdocspt4.tex|.
%
%\iffalse
%<*samplepart3|samplepart4>
%\fi

% Optional override for |\version| flag:
%    \begin{macrocode}
%%\providecommand{\version}{final}
%    \end{macrocode}

% Include the main document:
%    \begin{macrocode}
\input{childdoc.def}
\childdocby{cdocsamp}
%    \end{macrocode}

%\iffalse
%</samplepart3|samplepart4>
%\fi
%
%\iffalse
%<*samplepart3>
%\fi
% Some text for part 3:
%    \begin{macrocode}
some text in part three
%    \end{macrocode}

%\iffalse
%</samplepart3>
%\fi
% Some text for part 4:
%\iffalse
%<*samplepart4>
%\fi
%    \begin{macrocode}
more text in part four
%    \end{macrocode}

%\iffalse
%</samplepart4>
%\fi
%
% %%%%%%%%%%%%%%%%%%%%%%%%%%%%%%%%%%%%%%
% \paragraph{Forwarding for a Complete Draft.}
%
% The following forwarding file |cdocsdrf.tex|
% compiles the main document in draft mode:
%\iffalse
%<*sampledraft>
%\fi
%    \begin{macrocode}
\def\version{draft}
\input{childdoc.def}
\childdocforward{cdocsamp}
%    \end{macrocode}

%\iffalse
%</sampledraft>
%\fi
%
% %%%%%%%%%%%%%%%%%%%%%%%%%%%%%%%%%%%%%%
% \paragraph{Forwarding for Final Version of the Chapters.}
%
% The following forwarding files |cdocsfn1.tex| and |cdocsfn2.tex|
% (with identical content)
% compile the final versions of the child documents
% |cdocsch1.tex| and |cdocsch2.tex|, respectively:
%\iffalse
%<*samplefinal>
%\fi
%    \begin{macrocode}
\def\version{final}
\input{childdoc.def}
\childdocforwardprefix[cdocsamp]{cdocsfn}{cdocsch}
%    \end{macrocode}

%\iffalse
%</samplefinal>
%\fi
%
% %%%%%%%%%%%%%%%%%%%%%%%%%%%%%%%%%%%%%%
% \paragraph{Command Line Processing.}
%
% The following three command lines generate the output files
% |cdocscld|, |cdocscl1| and |cdocscl2|
% which should be identical to
% |cdocsdrf|, |cdocsch1| and |cdocsfn2|, respectively:
% \begin{center}
% \begin{tabular}{l}
% |latex -jobname cdocscld \|\\
% |  "\def\version{draft}\input{childdoc.def}\childdocforward{cdocsamp}"|\\
% |latex -jobname cdocscl1 \|\\
% |  "\input{childdoc.def}\childdocforward[cdocsamp]{cdocsch1}"|\\
% |latex -jobname cdocscl2 \|\\
% |  "\def\version{final}\input{childdoc.def}\childdocforward{cdocsch2}"|
% \end{tabular}
% \end{center}
% Note that the trailing backslash on each first line
% merely continues the input to the second line
% (for convenient cut ant paste).
% Furthermore, the command |latex| can be replaced by any
% of its alternative versions such as |pdflatex|.
%
% %%%%%%%%%%%%%%%%%%%%%%%%%%%%%%%%%%%%%%%%%%%%%%%%%%%%%%%%%%%%%%%%%%%%%%%%%%%%%%
% %%%%%%%%%%%%%%%%%%%%%%%%%%%%%%%%%%%%%%%%%%%%%%%%%%%%%%%%%%%%%%%%%%%%%%%%%%%%%%
% \section{Implementation}
%\iffalse
%<*package>
%\fi
%
% This section describes the definitions file |childdoc.def|.

% The definitions cannot be loaded using |\usepackage| or |\RequirePackage|
% which has a mechanism to prevent loading a style file more than once.
% When loading the definitions by means of |\input|
% multiple instances have to be prevented manually:
%\iffalse
%This code needs to be before the `\ProvidesFile' directive
%which is defined at the beginning of this file.
%Therefore it is also placed there and commented out here.
%</package>
%<*discard>
%\fi
%    \begin{macrocode}
\ifdefined\childdocmain\endinput\fi
%    \end{macrocode}
%\iffalse
%</discard>
%<*package>
%\fi
%
% \macro{\ifchilddoc}
% \macro{\ifchilddocmanual}
% The conditional |\ifchilddoc| tells whether a
% child (true) or main (false) document is being compiled.
% The conditional |\ifchilddocmanual| tells whether
% the |\includeonly| mechanism is used (false) or
% the selection of child files must be performed manually (true).
% The definitions initialise to false:
%    \begin{macrocode}
\newif\ifchilddoc
\newif\ifchilddocmanual
%    \end{macrocode}

% \macro{\childdocname}
% \macro{\childdocjob}
% The macro |\childdocname| stores the name of the main document
% to be compiled. The macro |\childdocjob| stores the name of
% the document on which the \LaTeX{} compiler was originally invoked.
% The content of |\jobname| cannot be compared
% to filenames specified in the source due to different catcodes.
% The following code rescans |\jobname|, stores the result
% in |\childdocname| and saves a copy in |\childdocjob|:
%    \begin{macrocode}
\edef\childdocname{\scantokens\expandafter{\jobname\noexpand}}
\let\childdocjob\childdocname
%    \end{macrocode}

% \macro{\childdocdisable}
% The macro |\childdocdisable| prevents the main file
% from being processed more than once.
% At this stage, the main document command |\childdocmain|
% is assumed to be called once again where it should do nothing.
% Any subsequent call to it should prevent
% a secondary processing of the main document
% It overwrites the forwarding commands
% |\childdocof| and |\childdocforward|
% with empty macros to prevent further inclusions of the main document:
%    \begin{macrocode}
\newcommand{\childdocdisable}
{
  \renewcommand{\childdocmain}[1]{\renewcommand{\childdocmain}[1]{\endinput}}
  \renewcommand{\childdocof}[1]{}
  \renewcommand{\childdocby}[2][]{}
  \renewcommand{\childdocforward}[2][]{}
  \renewcommand{\childdocdisable}{}
}
%    \end{macrocode}

% \macro{\childdocmain}
% The macro |\childdocmain| is to be called at the top of the main file
% with nothing or the main filename (without extension) as argument.
% First, it breaks loops.
% If the argument is not empty and does not match |\childdocname|
% (which is set by the first inclusion of |childdoc.def|),
% |\ifchilddoc| is set to true, |\includeonly| is applied to the child file
% and |\jobname| is set to the main file
% (for proper handling of |.aux| files):
%    \begin{macrocode}
\newcommand{\childdocmain}[1]
{
  \childdocdisable\childdocmain{}
  \if?#1?\else
    \begingroup
      \def\childdoctmp{#1}
      \ifx\childdoctmp\childdocname
        \def\childdoctmp{}
      \else
        \def\childdoctmp
        {
          \childdoctrue
          \includeonly{\childdocname}
          \def\childdocjob{#1}
          \def\jobname{#1}
        }
      \fi
      \expandafter
    \endgroup
    \childdoctmp
  \fi
}
%    \end{macrocode}

% \macro{\childdocof}
% The command |\childdocof| redirects
% compilation to the main file |#1|.
%    \begin{macrocode}
\newcommand{\childdocof}[1]
{
  \childdocdisable
  \childdoctrue
  \includeonly{\childdocname}
  \def\jobname{#1}
  \def\childdocjob{#1}
  \input{#1}
}
%    \end{macrocode}

% \macro{\childdocby}
% The command |\childdocby| ....
%    \begin{macrocode}
\newcommand{\childdocby}[2][]
{
  \childdocdisable
  \childdoctrue
  \childdocmanualtrue
  \if?#1?\else
    \def\jobname{#2}
  \fi
  \def\childdocjob{#2}
  \input{#2}
  \endinput
}
%    \end{macrocode}

% \macro{\childdocforward}
% The command |\childdocforward| redirects
% compilation to the main file or
% (if the optional argument is given) a child file.
% Parameters are set as if the main file
% or a child file starting with |\childdocof| was compiled.
% Then compilation is handed over to the main file:
%    \begin{macrocode}
\newcommand{\childdocforward}[2][]
{
  \begingroup
    \if?#1?
      \def\childdoctmp
      {
        \def\childdocname{#2}
        \def\childdocjob{#2}
        \def\jobname{#2}
        \input{#2}
        \endinput
      }
    \else
      \def\childdoctmp
      {
        \childdocdisable
        \def\childdocname{#2}
        \childdoctrue
        \includeonly{#2}
        \def\childdocjob{#1}
        \def\jobname{#1}
        \input{#1}
        \endinput
      }
    \fi
    \expandafter
  \endgroup
  \childdoctmp
}
%    \end{macrocode}

% \macro{\childdocforwardprefix}
% The command |\childdocforwardprefix| redirects
% compilation to the main or a child file by means of a pattern.
% The prefix |#1| in the current filename is replaced by |#2|
% and the suffix of the current filename is kept
% (it is assumed that the filename does not contain the substring `|~~~|'
% which is used as a delimiter).
% Compilation is handed over to the new file by |\childdocforward|:
%    \begin{macrocode}
\newcommand{\childdocforwardprefix}[3][]
{
  \begingroup
    \def\childdocextract #2##1~~~{\def\childdoctmp{\childdocforward[#1]{#3##1}}}
    \expandafter\childdocextract\childdocname~~~
    \expandafter
  \endgroup
  \childdoctmp
}
%    \end{macrocode}

% \macro{\childdoc}
% The deprecated macro |\childdoc| is a legacy version of |\childdocmain|:
%    \begin{macrocode}
\newcommand{\childdoc}{\childdocmain}
%    \end{macrocode}

% \macro{\childdocredirect}
% The deprecated macro |\childdocredirect| is a legacy version
% of |\childdocforward| and |\childdocforwardprefix|:
%    \begin{macrocode}
\newcommand{\childdocredirect}[2][]
{
  \begingroup
    \if?#1?
      \def\childdoctmp{\childdocforward{#2}}
    \else
      \def\childdoctmp{\childdocforwardprefix{#1}{#2}}
    \fi
    \expandafter
  \endgroup
  \childdoctmp
}
%    \end{macrocode}

%\iffalse
%</package>
%\fi
%
\endinput
|\\
|\childdocby{|\textit{main}|}|\\
\end{tabular}
\end{center}
%
The directive |\childdocby| is similar to |\childdocof|
described in \secref{sec:include},
but the subsequent selection of content must be done manually.
To that end, both |\ifchilddoc| and |\ifchilddocmanual|
will be true upon processing of a part,
and the name of the part is stored in |\childdocname|.
Note that |\jobname| will be set to the filename of the current part
so that each part receives an individual |.aux| file
that does not interfere with the |.aux| file(s) of the main document.
This behaviour can be altered by the alternative form
|\childdocby[*]{|\textit{main}|}| (with a non-empty optional argument)
which uses the |.aux| file of the main document
by setting |\jobname| to \textit{main}.

%%%%%%%%%%%%%%%%%%%%%%%%%%%%%%%%%%%%%%%%%%%%%%%%%%%%%%%%%%%%%%%%%%%%%%%%%%%%%%%%
\subsection{Driver Development}
\label{sec:driver}

The \textsf{childdoc} mechanism can also be use for the development
of definition files such as \LaTeX{} styles or classes.
This case differs from the above setup with multiple parts
included by |\include| in that no |\includeonly| should be invoked.
This can be achieved by starting the include file
(before |\ProvidesPackage|) with:
%
\begin{center}
\begin{tabular}{l}
|% \iffalse
%
% childdoc.dtx Copyright (C) 2017-2018 Niklas Beisert
%
% This work may be distributed and/or modified under the
% conditions of the LaTeX Project Public License, either version 1.3
% of this license or (at your option) any later version.
% The latest version of this license is in
%   http://www.latex-project.org/lppl.txt
% and version 1.3 or later is part of all distributions of LaTeX
% version 2005/12/01 or later.
%
% This work has the LPPL maintenance status `maintained'.
%
% The Current Maintainer of this work is Niklas Beisert.
%
% This work consists of the files childdoc.dtx and childdoc.ins
% and the derived files childdoc.def and cdocsamp.tex with
% cdocsch1.tex, cdocsch2.tex, cdocsdrf.tex, cdocsfn1.tex, cdocsfn2.tex.
%
%<package>\ifdefined\childdocmain\endinput\fi
%<package>\ProvidesFile{childdoc.def}[2018/12/30 v2.0 child document driver]
%<samplemain>\ProvidesFile{cdocsamp.tex}[2018/12/30 v2.0 sample for childdoc]
%<*driver>
%\ProvidesFile{childdoc.drv}[2018/12/30 v2.0 childdoc reference manual file]
\PassOptionsToClass{10pt,a4paper}{article}
\documentclass{ltxdoc}

\usepackage[margin=35mm]{geometry}
\usepackage{hyperref}
\usepackage{hyperxmp}
\usepackage[usenames]{color}

\hypersetup{colorlinks=true}
\hypersetup{pdfstartview=FitH}
\hypersetup{pdfpagemode=UseNone}
\hypersetup{pdfsource={}}
\hypersetup{pdflang={en-UK}}
\hypersetup{pdfcopyright={Copyright 2017-2018 Niklas Beisert.
  This work may be distributed and/or modified under the
  conditions of the LaTeX Project Public License, either version 1.3
  of this license or (at your option) any later version.}}
\hypersetup{pdflicenseurl={http://www.latex-project.org/lppl.txt}}
\hypersetup{pdfcontactaddress={ETH Zurich, ITP, HIT K,
  Wolfgang-Pauli-Strasse 27}}
\hypersetup{pdfcontactpostcode={8093}}
\hypersetup{pdfcontactcity={Zurich}}
\hypersetup{pdfcontactcountry={Switzerland}}
\hypersetup{pdfcontactemail={nbeisert@itp.phys.ethz.ch}}
\hypersetup{pdfcontacturl={http://people.phys.ethz.ch/\xmptilde nbeisert/}}

\newcommand{\secref}[1]{\hyperref[#1]{section \ref*{#1}}}

\parskip1ex
\parindent0pt
\let\olditemize\itemize
\def\itemize{\olditemize\parskip0pt}

\begin{document}

\title{The \textsf{childdoc} Package}
\hypersetup{pdftitle={The childdoc Package}}
\author{Niklas Beisert\\[2ex]
  Institut f\"ur Theoretische Physik\\
  Eidgen\"ossische Technische Hochschule Z\"urich\\
  Wolfgang-Pauli-Strasse 27, 8093 Z\"urich, Switzerland\\[1ex]
  \href{mailto:nbeisert@itp.phys.ethz.ch}
  {\texttt{nbeisert@itp.phys.ethz.ch}}}
\hypersetup{pdfauthor={Niklas Beisert}}
\hypersetup{pdfsubject={Manual for the LaTeX2e Package childdoc}}
\date{30 December 2018, \textsf{v2.0}}
\maketitle

\begin{abstract}\noindent
\textsf{childdoc} is a \LaTeXe{} package
that enables the direct compilation
of document sections included by |\include|
to individual files.
\end{abstract}

\begingroup
\parskip0ex
\tableofcontents
\endgroup

%%%%%%%%%%%%%%%%%%%%%%%%%%%%%%%%%%%%%%%%%%%%%%%%%%%%%%%%%%%%%%%%%%%%%%%%%%%%%%%%
%%%%%%%%%%%%%%%%%%%%%%%%%%%%%%%%%%%%%%%%%%%%%%%%%%%%%%%%%%%%%%%%%%%%%%%%%%%%%%%%
\section{Introduction}

\LaTeX{} provides a mechanism to structure a large document (such as a book)
into a main file and several child files (containing the chapters)
using the |\include| command.
This mechanism is beneficial for documents
which span hundreds of pages in order to
make the source file(s) more manageable.
Moreover, compilation can be restricted to
selected child files by means of the |\includeonly| command.
The latter feature can be used to reduce the compilation time while editing
(this was significantly more useful in the earlier days of \LaTeX{})
or to generate a smaller document which is easier to navigate.
Another application of |\includeonly| is to generate
documents consisting of selected parts of the complete document.

However, there are a few drawbacks of the plain |\include| mechanism:
\begin{itemize}
\item
The child files cannot be compiled on their own,
they can only be compiled via the main file.
A naive editing environment
(such as a text editor with an option
to have the current file processed by \LaTeX)
may require one to switch to the main file before compiling;
attempting to compile the child file produces errors.
\item
The main file must be modified (each time)
to adjust the |\includeonly| command
to the present needs. This easily leaves the main file in a messy state.
\item
The generated document will always carry the filename
of the main document. This is inconvenient if
several child files are to be compiled and
to be kept for distribution.
\end{itemize}

The present package provides a simple interface
to make child files individually compilable by \LaTeX{}.
Compiling a child file then has the same effect as compiling
the main file with an |\includeonly| command
to select the appropriate child.
Moreover the generated document will carry the name of the child
rather than the main file.
This resolves all three above issues.

This feature is meant to make the editing of books,
thesis documents and lecture notes somewhat more convenient.
However, the package can also be used efficiently for
composing a series of documents (such as exercise sheets)
which are typically distributed individually.
It then assists the author in generating the individual documents
(potentially in different versions)
as well as a document containing the collected series.
Another application is in developing style files
or other kinds of included material
where compilation of the style file could redirect
to a sample or test file.

%%%%%%%%%%%%%%%%%%%%%%%%%%%%%%%%%%%%%%%%%%%%%%%%%%%%%%%%%%%%%%%%%%%%%%%%%%%%%%%%
%%%%%%%%%%%%%%%%%%%%%%%%%%%%%%%%%%%%%%%%%%%%%%%%%%%%%%%%%%%%%%%%%%%%%%%%%%%%%%%%
\section{Usage}

First of all, the package \textsf{childdoc} is \emph{not} a standard
\LaTeXe{} |.sty| style file! Therefore it needs to be invoked in
a non-standard way.

%%%%%%%%%%%%%%%%%%%%%%%%%%%%%%%%%%%%%%%%%%%%%%%%%%%%%%%%%%%%%%%%%%%%%%%%%%%%%%%%
\subsection{Included Files}
\label{sec:include}

%%%%%%%%%%%%%%%%%%%%%%%%%%%%%%%%%%%%%%%%
\DescribeMacro{\childdocmain}
To use the package, add the commands
\begin{center}
\begin{tabular}{l}
|\input{childdoc.def}|\\
|\childdocmain{}|\\
\end{tabular}
\end{center}
at the very top of the main \LaTeX{} file,
in particular \emph{before} the |\documentclass| statement!
The argument of |\childdocmain| should be left empty
(but it must be present).

%%%%%%%%%%%%%%%%%%%%%%%%%%%%%%%%%%%%%%%%
\DescribeMacro{\childdocof}
Furthermore, add the commands
\begin{center}
\begin{tabular}{l}
|\input{childdoc.def}|\\
|\childdocof{|\textit{main}|}|\\
\end{tabular}
\end{center}
at the top of every child file \textit{child}
which is included by |\include{|\textit{child}|}|
from within the main file
(or at least for those files to be compiled individually).
The argument \textit{main} must be the filename of the main file.

There are a couple of
considerations in setting up the main and child documents:

%%%%%%%%%%%%%%%%%%%%%%%%%%%%%%%%%%%%%%%%
\paragraph{Restrictions.}

Please note the following restrictions:
\begin{itemize}
\item
|\childdocmain| must be called with one argument \textit{main}
to ensure compatibility with earlier version of the package.
It must either be empty (|\childdocmain{}|)
or precisely match the filename of the main file in which it is specified.
See \secref{sec:detection} for further information.
\item
The filename \textit{main} must be specified without the |.tex| extension.
\item
The filename \textit{main} is case sensitive
(even in case-insensitive file systems)
due to internal string comparison.
\item
The argument \textit{main} should be fully expanded, it cannot be a macro.
\item
Subdirectories and special characters should be avoided in filenames.
\item
The command |\childdocmain{|\textit{main}|}| must be followed by a whitespace.
It should not be followed immediately by another command
or by a comment mark `|%|'.
This is because the \TeX{} parser reads the token immediately following
the argument of |\childdocmain| and puts it
at the beginning of every child section;
however, a white\-space is ignored.
\end{itemize}

%%%%%%%%%%%%%%%%%%%%%%%%%%%%%%%%%%%%%%%%
\paragraph{Content of Main File.}

It is advisable to place all content in the child files included by |\include|.
Any output contained in the main file will appear in all child documents
unless suppressed manually;
it cannot be suppressed automatically by the |\includeonly| directive
and thus should normally be avoided.
A method to include some content in the main file
by means of conditional processing is described in \secref{sec:conditional}.

%%%%%%%%%%%%%%%%%%%%%%%%%%%%%%%%%%%%%%%%
\paragraph{Page Numbering.}

When only a part of the document is compiled,
the appropriate numbering of pages
(as well as other status parameters)
is determined from the |.aux| files.
The latter contain information from previous passes.
However this information needs to propagate through
all intermediate child documents.
Therefore the page numbering in child documents may well
be inconsistent until the complete document is compiled at least once.

A useful (if unconventional) way to always ensure a consistent
page numbering is to restart the numbering in each child document
and denote the pages by `\textit{child}|.|\textit{page}'
where \textit{child} represents the chapter/section number of the child file.
This can be achieved by the command
|\numberwithin{page}{|\textit{child}|}|
of the \textsf{amsmath} package
where \textit{child} can be |chapter| or |section|
depending on the chosen structuring.
Alternatively, one can modify the macro |\thepage| appropriately
and reset the counter |page| at the start of each child file.

%%%%%%%%%%%%%%%%%%%%%%%%%%%%%%%%%%%%%%%%%%%%%%%%%%%%%%%%%%%%%%%%%%%%%%%%%%%%%%%%
\subsection{Conditional Processing}
\label{sec:conditional}

The package provides a mechanism to compile different versions
of a document. To customise the versions further some conditional processing
can come in handy to distinguish which version is being compiled.
The package provides two macros to describe the compilation context:

%%%%%%%%%%%%%%%%%%%%%%%%%%%%%%%%%%%%%%%%
\DescribeMacro{\ifchilddoc}
The conditional |\ifchilddoc| distinguishes between the compilation of
child documents and the main document:
%
\begin{center}
|\ifchilddoc |\textit{child-code}| |[|\||else |\textit{main-code}]| \||fi|
\end{center}

%%%%%%%%%%%%%%%%%%%%%%%%%%%%%%%%%%%%%%%%
\DescribeMacro{\childdocname}
\DescribeMacro{\childdocjob}
The macro |\childdocname| contains the filename (without extension)
of the main or child file being processed.
Note that |\childdocjob| will always contain the name of the main file.

%%%%%%%%%%%%%%%%%%%%%%%%%%%%%%%%%%%%%%%%
\paragraph{Title Page.}

Conditional processing can be used to include a title or banner page
in the main document when proper precautions are taken.
Importantly, the code in the main file should ensure that the page counter
(as well as other status parameters which are stored in the |.aux| files)
takes the same value after the conditional processing.
Otherwise the page numbers may take divergent values
depending on which part is compiled.

For example, a title page could be declared by:
%
\begin{center}
\begin{tabular}{l}
|\ifchilddoc\||else|\\
|\addtocounter{page}{-1}|\\
\textit{code for title page}\\
|\newpage|\\
|\||fi|
\end{tabular}
\end{center}
%
A banner page for the child documents can be generated by:
%
\begin{center}
\begin{tabular}{l}
|\ifchilddoc|\\
|\addtocounter{page}{-1}|\\
\textit{code for banner page}\\
|\newpage|\\
|\||fi|
\end{tabular}
\end{center}
%
Here one could write a message such as:
\begin{center}
|This is the part \childdocname{} of \childdocjob{}.|
\end{center}

%%%%%%%%%%%%%%%%%%%%%%%%%%%%%%%%%%%%%%%%%%%%%%%%%%%%%%%%%%%%%%%%%%%%%%%%%%%%%%%%
\subsection{Flags}
\label{sec:flags}

The package makes it easy to generate different versions
of the main or child documents.
To this end compilation flags can be defined
and assigned different default values.
They will be particularly useful in conjunction
with the forwarding mechanism described in \secref{sec:forward}.

For example, it may be useful to have a flag |\version|
which can be set to |draft| or |final|.
The document source will contain some conditional code
depending on the value of |\version|.
Suppose further, the flag should default to |final| for the main file
and to |draft| for child files
which is a natural assignment for editing the document.
This is achieved by placing the following code
in the preamble of the main document
(below the |\childdocmain| directive):
%
\begin{center}
\begin{tabular}{l}
|\ifchilddoc|\\
|\providecommand{\version}{draft}|\\
|\||else|\\
|\providecommand{\version}{final}|\\
|\||fi|
\end{tabular}
\end{center}
%
The definition by |\providecommand| makes sure
that previous definitions are not overwritten.
Further statements |\providecommand{\version}{...}|
can thus be added before the above code to override it.

For the main file, one might add a line
(between |\childdocmain| and the above block)
%
\begin{center}
|%\ifchilddoc\||else\providecommand{\version}{draft}\||fi|
\end{center}
%
which can be uncommented to produce a draft version.
Likewise one can add a line to the very top of a child file
(above the |\childdocof{|\textit{main}|}| directive)
%
\begin{center}
|%\providecommand{\version}{final}|
\end{center}
%
which can be uncommented to produce the final version of this child document.

%%%%%%%%%%%%%%%%%%%%%%%%%%%%%%%%%%%%%%%%%%%%%%%%%%%%%%%%%%%%%%%%%%%%%%%%%%%%%%%%
\subsection{Forwarding}
\label{sec:forward}

Different versions of the main or child documents
using compilation flags as described in \secref{sec:flags}
can be (permanently) stored in different files
for convenient compilation, viewing and distribution.
To this end, the package defines a command
to pass on compilation to a different file:

%%%%%%%%%%%%%%%%%%%%%%%%%%%%%%%%%%%%%%%%
\DescribeMacro{\childdocforward}
The command |\childdocforward| redirects processing to
another source file:
%
\begin{center}
\begin{tabular}{l}
|\input{childdoc.def}|\\
|\childdocforward[|\textit{main}|]{|\textit{dest}|}|\\
\end{tabular}
\end{center}
%
The argument \textit{dest} is the destination file
(without extension).
It should be the main file or one of the child files.
Note that further \textsf{childdoc} directives
such as |\childdocof| and |\childdocforward|
in the indicated file will be processed in this form.
The optional argument \textit{main}
passes on directly to the main file \textit{main}
while pretending to compile the child \textit{dest}.
This form behaves as if \textit{dest}
issues |\childdocof{|\textit{main}|}| right away,
and no further \textsf{childdoc} directives will be processed.

%%%%%%%%%%%%%%%%%%%%%%%%%%%%%%%%%%%%%%%%
\DescribeMacro{\...prefix}
In the alternative form |\childdocforwardprefix|,
%
\begin{center}
\begin{tabular}{l}
|\input{childdoc.def}|\\
|\childdocforwardprefix[|\textit{main}|]{|\textit{prefix}|}{|\textit{dest}|}|
\end{tabular}
\end{center}
%
the destination file is determined by a pattern
depending on the current file:
To make this work, the current file must be called
`{\textit{prefix}\hspace{0.2em}\textit{suffix}}'
with \textit{prefix} matching precisely the argument.
Processing is then passed on to the file
`{\textit{dest}\hspace{0.2em}\textit{suffix}}'.
Surely, the same effect is achieved by
directly specifying the
argument `{\textit{dest}\hspace{0.2em}\textit{suffix}}'
in the first form.
However, that requires to set up a different file
for each child. With the alternative form of the command
all these files can have exactly the same content
which simplifies setting them up and maintaining them.

For example, the following file |draft.tex|
with a compilation flag |\version| as described in \secref{sec:flags}
compiles the main document as a draft:
%
\begin{center}
\begin{tabular}{l}
|\def\version{draft}|\\
|\input{childdoc.def}|\\
|\childdocforward{|\textit{main}|}|
\end{tabular}
\end{center}
%
Likewise, the following files |final|\textit{nn}|.tex|
compile the final version of the child document
|child|\textit{nn}|.tex|:
%
\begin{center}
\begin{tabular}{l}
|\def\version{final}|\\
|\input{childdoc.def}|\\
|\childdocforwardprefix{final}{child}|
\end{tabular}
\end{center}
%

Note that when several versions of a main file and/or of each child file
are to be generated, it may be convenient to set up a |Makefile| or
shell script to automatise the process.

%%%%%%%%%%%%%%%%%%%%%%%%%%%%%%%%%%%%%%%%%%%%%%%%%%%%%%%%%%%%%%%%%%%%%%%%%%%%%%%%
\subsection{Command Line Processing}
\label{sec:commandline}

The effect of redirection files can also be achieved by invoking
the \LaTeX{} compiler with a more elaborate command line.
Most conveniently this should be done as part
of a shell script or a |Makefile|.

When using \textsf{childdoc} in the main file, the following
command lines effectively perform a redirection
(note that depending on the shell being used,
backslashes may have to be doubled: `|\|' $\to$ `|\\|'):
%
\begin{center}
|... -jobname "|\textit{target}|" |\\|"|[\textit{flags}]%
|\input{childdoc.def}\childdocforward[|\textit{main}|]{|\textit{dest}|}"|
\end{center}
%
Here \textit{target} is the name of the output file,
\textit{main} is the name of the main file
and \textit{dest} is the name of the main or child file to be processed
(all filenames without extensions).
The optional argument \textit{main} can be omitted
if \textit{main} matches \textit{dest}.
Optionally, compilation \textit{flags} can be defined via |\def| commands.
This command line makes the \TeX{} engine believe
it is compiling the file \textit{target}
whose content is specified as the latter parameter.
The provided code then forwards the processing to
\textit{main} or \textit{dest} as described in \secref{sec:forward}.

%%%%%%%%%%%%%%%%%%%%%%%%%%%%%%%%%%%%%%%%%%%%%%%%%%%%%%%%%%%%%%%%%%%%%%%%%%%%%%%%
\subsection{Include by Input}
\label{sec:input}

Including child documents by |\include| has some restrictions by design.
Most notably, the content of a child document always occupies
its own set of pages; pages cannot be shared between child documents.
Usually, this behaviour makes perfect sense
because each child document contain an essential part of the document.
However, in some situations it may be desirable to compose
a document from a collection of parts
without having mandatory page breaks between then.
For this case, the package
provides a mechanism to include parts
by |\input| which can also be processed individually.
However, by construction this mechanism
requires manual handling of the content to be output.

%%%%%%%%%%%%%%%%%%%%%%%%%%%%%%%%%%%%%%%%
\DescribeMacro{\ifchilddocmanual}
The main file should be prepared as usual, see \secref{sec:include}.
However, the document body must make a distinction
between processing of an individual part and of the main document, e.g.:
%
\begin{center}
\begin{tabular}{l}
|\ifchilddocmanual|\\
|\input{\childdocname}|\\
|\||else|\\
\textit{document body with }|\input{|\textit{part}|}|\\
|\||fi|
\end{tabular}
\end{center}
%
The conditional |\ifchilddocmanual| is true whenever
a part to be included by |\input| is being compiled,
and the name of the part is stored in |\childdocname|.

%%%%%%%%%%%%%%%%%%%%%%%%%%%%%%%%%%%%%%%%
\DescribeMacro{\childdocby}
Each part to be included by |\input| should start with:
%
\begin{center}
\begin{tabular}{l}
|\input{childdoc.def}|\\
|\childdocby{|\textit{main}|}|\\
\end{tabular}
\end{center}
%
The directive |\childdocby| is similar to |\childdocof|
described in \secref{sec:include},
but the subsequent selection of content must be done manually.
To that end, both |\ifchilddoc| and |\ifchilddocmanual|
will be true upon processing of a part,
and the name of the part is stored in |\childdocname|.
Note that |\jobname| will be set to the filename of the current part
so that each part receives an individual |.aux| file
that does not interfere with the |.aux| file(s) of the main document.
This behaviour can be altered by the alternative form
|\childdocby[*]{|\textit{main}|}| (with a non-empty optional argument)
which uses the |.aux| file of the main document
by setting |\jobname| to \textit{main}.

%%%%%%%%%%%%%%%%%%%%%%%%%%%%%%%%%%%%%%%%%%%%%%%%%%%%%%%%%%%%%%%%%%%%%%%%%%%%%%%%
\subsection{Driver Development}
\label{sec:driver}

The \textsf{childdoc} mechanism can also be use for the development
of definition files such as \LaTeX{} styles or classes.
This case differs from the above setup with multiple parts
included by |\include| in that no |\includeonly| should be invoked.
This can be achieved by starting the include file
(before |\ProvidesPackage|) with:
%
\begin{center}
\begin{tabular}{l}
|\input{childdoc.def}|\\
|\childdocforward{|\textit{main}|}|\\
\end{tabular}
\end{center}
%
or alternatively with:
%
\begin{center}
\begin{tabular}{l}
|\input{childdoc.def}|\\
|\childdocby{|\textit{main}|}|\\
\end{tabular}
\end{center}
%
Both forms have slightly different effects as described above.
The main file is prepared as usual, see \secref{sec:include}.

%%%%%%%%%%%%%%%%%%%%%%%%%%%%%%%%%%%%%%%%%%%%%%%%%%%%%%%%%%%%%%%%%%%%%%%%%%%%%%%%
\subsection{Legacy Detection}
\label{sec:detection}

The directive |\childdocmain| in the main file can detect
whether the complete document or merely a child is to be compiled
even without using the directive |\childdocof|.
This method is deprecated because it is less robust
and there is no compelling reason to use it;
it is merely provided for backward compatibility
and it may be removed in future versions.

If the detection mechanism is to be used,
it is mandatory to correctly specify
the filename of the main file as the argument of |\childdocmain|:
%
\begin{center}
\begin{tabular}{l}
|\input{childdoc.def}|\\
|\childdocmain{|\textit{main}|}|\\
\end{tabular}
\end{center}
%
If |\jobname| does not match the argument \textit{main} of |\childdocmain|,
it is assumed that |\jobname| points to the child file to be compiled.
When using |\childdocmain| with the main file specified as argument,
it suffices to start a child file
with just |\input{|\textit{main}|}|
without loading of the package and using |\childdocof|.
If instead all processing is done
with the appropriate \textsf{childdoc} directives,
the argument of \textit{main} of |\childdocmain| can be empty.

An alternative version of the command line processing described
in \secref{sec:commandline} using the detection mechanism reads:
%
\begin{center}
|... -jobname "|\textit{target}|" "|[\textit{flags}]%
[|\def\jobname{|\textit{dest}|}|]|\input{|\textit{main}|}"|
\end{center}

%%%%%%%%%%%%%%%%%%%%%%%%%%%%%%%%%%%%%%%%%%%%%%%%%%%%%%%%%%%%%%%%%%%%%%%%%%%%%%%%
\subsection{Manual Code}
\label{sec:manual}

In case one cannot be certain whether the definitions file |childdoc.def|
is installed on the target \TeX{} distribution
and one prefers not to ship it,
it is conceivable to paste a few relevant commands into the sources.

To that end, drop all statements |\input{childdoc.def}|
and perform the replacements as outlined below.
Instead of |\childdocmain{|\textit{main}|}| add the following code
to the top of the main file:
%
\begin{center}
\begin{tabular}{l}
|\||ifdefined\childdocname\endinput\||fi\newif\ifchilddoc|\\
|\edef\childdocname{\scantokens\expandafter{\jobname\noexpand}}|\\
|\def\childdocmain{|\textit{main}|}\||ifx\childdocmain\childdocname\||else|\\
|\childdoctrue\includeonly{\childdocname}\let\jobname\childdocmain\||fi|\\
\end{tabular}
\end{center}
%
Instead of |\childdocof{|\textit{main}|}| just include the main file
at the top of each child file:
%
\begin{center}
|\input{|\textit{main}|}|
\end{center}
%
A simple redirection |\childdocforward{|\textit{dest}|}| is achieved by:
%
\begin{center}
|\def\jobname{|\textit{dest}|}\input{\jobname}|
\end{center}
%
The redirection with prefix
|\childdocforwardprefix[|\textit{prefix}|]{|\textit{dest}|}|
is accomplished by:
%
\begin{center}
\begin{tabular}{l}
|{\edef\jobname{\scantokens\expandafter{\jobname\noexpand}}|\\
|\def\redirectjob |\textit{prefix}|#1~~~{\gdef\jobname{|\textit{dest}|#1}}|\\
|\expandafter\redirectjob\jobname~~~}\input{\jobname}|
\end{tabular}
\end{center}

In an alternative approach,
child documents can be compiled by a specific command line
without additional code or specific definitions:
%
\begin{center}
|... -jobname "|\textit{target}|" "|[\textit{flags}]%
|\includeonly{|\textit{dest}|}\input{|\textit{main}|}"|
\end{center}
%

%%%%%%%%%%%%%%%%%%%%%%%%%%%%%%%%%%%%%%%%%%%%%%%%%%%%%%%%%%%%%%%%%%%%%%%%%%%%%%%%
%%%%%%%%%%%%%%%%%%%%%%%%%%%%%%%%%%%%%%%%%%%%%%%%%%%%%%%%%%%%%%%%%%%%%%%%%%%%%%%%
\section{Information}

%%%%%%%%%%%%%%%%%%%%%%%%%%%%%%%%%%%%%%%%%%%%%%%%%%%%%%%%%%%%%%%%%%%%%%%%%%%%%%%%
\subsection{Copyright}

Copyright \copyright{} 2017--2018 Niklas Beisert

This work may be distributed and/or modified under the
conditions of the \LaTeX{} Project Public License, either version 1.3
of this license or (at your option) any later version.
The latest version of this license is in
  \url{http://www.latex-project.org/lppl.txt}
and version 1.3 or later is part of all distributions of \LaTeX{}
version 2005/12/01 or later.

This work has the LPPL maintenance status `maintained'.

The Current Maintainer of this work is Niklas Beisert.

This work consists of the files |README.txt|, |childdoc.ins| and |childdoc.dtx|
as well as the derived files |childdoc.def|, |cdocsamp.tex|
with |cdocsch1.tex|, |cdocsch2.tex|, |cdocspt3.tex|, |cdocspt4.tex|,
|cdocsdrf.tex|, |cdocsfn1.tex|, |cdocsfn2.tex|
as well as |childdoc.pdf|.

%%%%%%%%%%%%%%%%%%%%%%%%%%%%%%%%%%%%%%%%%%%%%%%%%%%%%%%%%%%%%%%%%%%%%%%%%%%%%%%%
\subsection{Files and Installation}

The package consists of the files:
%
\begin{center}
\begin{tabular}{ll}
    |README.txt|   & readme file \\
    |childdoc.ins| & installation file \\
    |childdoc.dtx| & source file \\
    |childdoc.def| & definition file \\
    |cdocsamp.tex| & sample main file \\
    |cdocsch1.tex| & sample include file \\
    |cdocsch2.tex| & sample include file \\
    |cdocspt3.tex| & sample part file \\
    |cdocspt4.tex| & sample part file \\
    |cdocsdrf.tex| & sample redirection file \\
    |cdocsfn1.tex| & sample redirection file \\
    |cdocsfn2.tex| & sample redirection file \\
    |childdoc.pdf| & manual
\end{tabular}
\end{center}
%
The distribution consists of the files
|README.txt|, |childdoc.ins| and |childdoc.dtx|.
%
\begin{itemize}
\item
Run (pdf)\LaTeX{} on |childdoc.dtx|
to compile the manual |childdoc.pdf| (this file).
\item
Run \LaTeX{} on |childdoc.ins| to create the definitions file |childdoc.def|
and the sample |cdocsamp.tex| with include files
|cdocsch1.tex|, |cdocsch2.tex|, |cdocspt3.tex|, |cdocspt4.tex|,
|cdocsdrf.tex|, |cdocsfn1.tex|, |cdocsfn2.tex|.
Then copy the file |childdoc.def| to an appropriate directory of your \LaTeX{}
distribution, e.g.\ \textit{texmf-root}|/tex/latex/childdoc|.
\end{itemize}

%%%%%%%%%%%%%%%%%%%%%%%%%%%%%%%%%%%%%%%%%%%%%%%%%%%%%%%%%%%%%%%%%%%%%%%%%%%%%%%%
\subsection{Related CTAN Packages}

There are several other packages which offer a similar functionality:
%
\begin{itemize}
\item
The packages
\href{http://ctan.org/pkg/docmute}{\textsf{docmute}},
\href{http://ctan.org/pkg/includex}{\textsf{includex}} and
\href{http://ctan.org/pkg/standalone}{\textsf{standalone}}
provide commands to include only the document body of
a child file thus allowing both files to be compiled individually.
\item
The packages \href{http://ctan.org/pkg/subdocs}{\textsf{subdocs}}
and \href{http://ctan.org/pkg/subfiles}{\textsf{subfiles}}
provide structures in which the main and child documents can be
encapsulated and allowing them to be compiled individually.
The inclusion mechanism is different from the conventional |\include|.
\item
The package \href{http://ctan.org/pkg/combine}{\textsf{combine}}
is an elaborate solution to combine several documents into one.
\end{itemize}
%
See also the CTAN topic \href{http://ctan.org/topic/subdocs}{\textsf{subdocs}}
for further related packages.
The present package differs from the above solutions in that
a document structure constructed with the conventional |\include| mechanism
just needs two extra commands at the top of every file
such that all constituent files can be compiled individually.

%%%%%%%%%%%%%%%%%%%%%%%%%%%%%%%%%%%%%%%%%%%%%%%%%%%%%%%%%%%%%%%%%%%%%%%%%%%%%%%%
%\subsection{Feature Suggestions}
%
%The following is a list of features which may be useful for future
%versions of this package:
%%
%\begin{itemize}
%\item
%\ldots
%\end{itemize}

%%%%%%%%%%%%%%%%%%%%%%%%%%%%%%%%%%%%%%%%%%%%%%%%%%%%%%%%%%%%%%%%%%%%%%%%%%%%%%%%
\subsection{Revision History}

%%%%%%%%%%%%%%%%%%%%%%%%%%%%%%%%%%%%%%%%
\paragraph{v2.0:} 2018/12/30

\begin{itemize}
\item
immediate forward processing
\item
added |\childdocby| mechanism
\item
manual restructured
\end{itemize}

%%%%%%%%%%%%%%%%%%%%%%%%%%%%%%%%%%%%%%%%
\paragraph{v1.6:} 2018/01/17

\begin{itemize}
\item
application for development of include files
\item
corrections to manual
\end{itemize}

%%%%%%%%%%%%%%%%%%%%%%%%%%%%%%%%%%%%%%%%
\paragraph{v1.5:} 2017/05/21

\begin{itemize}
\item
more complete structuring introduced
\item
|\childdocof| introduced
\item
|\childdoc| renamed to |\childdocmain|
\item
|\childredirect| renamed to |\childdocforward| and |\childdocforwardprefix|
and functionality expanded
\end{itemize}

%%%%%%%%%%%%%%%%%%%%%%%%%%%%%%%%%%%%%%%%
\paragraph{v1.0:} 2017/04/27

\begin{itemize}
\item
manual and install package
\item
first version published on CTAN
\end{itemize}

%%%%%%%%%%%%%%%%%%%%%%%%%%%%%%%%%%%%%%%%
\paragraph{v0.6:} 2017/04/26

\begin{itemize}
\item
redirection mechanism added
\end{itemize}

%%%%%%%%%%%%%%%%%%%%%%%%%%%%%%%%%%%%%%%%
\paragraph{v0.5:} 2017/04/26

\begin{itemize}
\item
functionality in definition file
\end{itemize}


%%%%%%%%%%%%%%%%%%%%%%%%%%%%%%%%%%%%%%%%%%%%%%%%%%%%%%%%%%%%%%%%%%%%%%%%%%%%%%%%
%%%%%%%%%%%%%%%%%%%%%%%%%%%%%%%%%%%%%%%%%%%%%%%%%%%%%%%%%%%%%%%%%%%%%%%%%%%%%%%%
%%%%%%%%%%%%%%%%%%%%%%%%%%%%%%%%%%%%%%%%%%%%%%%%%%%%%%%%%%%%%%%%%%%%%%%%%%%%%%%%
\appendix

\settowidth\MacroIndent{\rmfamily\scriptsize 000\ }

 \DocInput{childdoc.dtx}

\end{document}
%</driver>
% \fi
%
% %%%%%%%%%%%%%%%%%%%%%%%%%%%%%%%%%%%%%%%%%%%%%%%%%%%%%%%%%%%%%%%%%%%%%%%%%%%%%%
% %%%%%%%%%%%%%%%%%%%%%%%%%%%%%%%%%%%%%%%%%%%%%%%%%%%%%%%%%%%%%%%%%%%%%%%%%%%%%%
% \section{Sample}
%\iffalse
%<*samplemain>
%\fi
%
% The following presents a sample document
% with two chapters, two parts, a title page,
% a compile flag as well as three forwarding files to set the flag.
% It consists of eight |.tex| files:
% \begin{center}
% \begin{tabular}{ll}
% |cdocsamp.tex|&main file\\
% |cdocsch1.tex|&include file for chapter 1\\
% |cdocsch2.tex|&include file for chapter 2\\
% |cdocspt3.tex|&include file for part 3\\
% |cdocspt4.tex|&include file for part 4\\
% |cdocsdrf.tex|&forwarding file for main file in draft mode\\
% |cdocsfi1.tex|&forwarding file for final version of chapter 1\\
% |cdocsfi2.tex|&forwarding file for final version of chapter 2\\
% \end{tabular}
% \end{center}
% Each of the eight files can be compiled directly by the \LaTeX{} compiler.
%
% %%%%%%%%%%%%%%%%%%%%%%%%%%%%%%%%%%%%%%
% \paragraph{Main File.}
%
% The main file is called |cdocsamp.tex|.
%
% Load the \textsf{childdoc} definitions and
% declare the filename for the main document:
%    \begin{macrocode}
\input{childdoc.def}
\childdocmain{}
%    \end{macrocode}

% Optional override for |\version| flag:
%    \begin{macrocode}
%%\ifchilddoc\else\providecommand{\version}{draft}\fi
%    \end{macrocode}

% Define the default values for the |\version| flag
% (|final| for the main file and |draft| for childs):
%    \begin{macrocode}
\ifchilddoc
\providecommand{\version}{draft}
\else
\providecommand{\version}{final}
\fi
%    \end{macrocode}

% Load the standard document class:
%    \begin{macrocode}
\documentclass[12pt]{article}
%    \end{macrocode}

% Start the document body:
%    \begin{macrocode}
\begin{document}
%    \end{macrocode}

% Declare a title page.
% Print title, part of document being processed and version flag:
%    \begin{macrocode}
\addtocounter{page}{-1}
\begin{center}
{\LARGE\bfseries{}childdoc example\par}
\vspace{1cm}
\ifchilddoc
\ifchilddocmanual part\else chapter\fi:
`\childdocname' of `\childdocjob'\par
\else
main document: `\childdocjob'\par
\fi
version: \version\par
\end{center}
\newpage
%    \end{macrocode}

% Manually include selected file,
% otherwise process as usual:
%    \begin{macrocode}
\ifchilddocmanual
\section*{part `\childdocname'}
\input{\childdocname}
\else
%    \end{macrocode}

% Include the two chapters:
%    \begin{macrocode}
\include{cdocsch1}
\include{cdocsch2}
%    \end{macrocode}

% Include the two parts unless only chapters should be displayed:
%    \begin{macrocode}
\ifchilddoc\else
\section{part three}
\input{cdocspt3}
\section{part four}
\input{cdocspt4}
\fi
%    \end{macrocode}

% Process as usual until here:
%    \begin{macrocode}
\fi
%    \end{macrocode}

% End of document body:
%    \begin{macrocode}
\end{document}
%    \end{macrocode}
%\iffalse
%</samplemain>
%\fi
%
% %%%%%%%%%%%%%%%%%%%%%%%%%%%%%%%%%%%%%%
% \paragraph{Chapter Include Files.}
%
% The include files are called |cdocsch1.tex| and |cdocsch2.tex|.
%
%\iffalse
%<*samplechap1|samplechap2>
%\fi

% Optional override for |\version| flag:
%    \begin{macrocode}
%%\providecommand{\version}{final}
%    \end{macrocode}

% Include the main document:
%    \begin{macrocode}
\input{childdoc.def}
\childdocof{cdocsamp}
%    \end{macrocode}

%\iffalse
%</samplechap1|samplechap2>
%\fi
%
%\iffalse
%<*samplechap1>
%\fi
% Some text for chapter 1:
%    \begin{macrocode}
\section{one}
some text in chapter one
%    \end{macrocode}

%\iffalse
%</samplechap1>
%\fi
% Some text for chapter 2:
%\iffalse
%<*samplechap2>
%\fi
%    \begin{macrocode}
\section{two}
more text in chapter two
%    \end{macrocode}

%\iffalse
%</samplechap2>
%\fi
%
% %%%%%%%%%%%%%%%%%%%%%%%%%%%%%%%%%%%%%%
% \paragraph{Part Include Files.}
%
% The include files are called |cdocspt3.tex| and |cdocspt4.tex|.
%
%\iffalse
%<*samplepart3|samplepart4>
%\fi

% Optional override for |\version| flag:
%    \begin{macrocode}
%%\providecommand{\version}{final}
%    \end{macrocode}

% Include the main document:
%    \begin{macrocode}
\input{childdoc.def}
\childdocby{cdocsamp}
%    \end{macrocode}

%\iffalse
%</samplepart3|samplepart4>
%\fi
%
%\iffalse
%<*samplepart3>
%\fi
% Some text for part 3:
%    \begin{macrocode}
some text in part three
%    \end{macrocode}

%\iffalse
%</samplepart3>
%\fi
% Some text for part 4:
%\iffalse
%<*samplepart4>
%\fi
%    \begin{macrocode}
more text in part four
%    \end{macrocode}

%\iffalse
%</samplepart4>
%\fi
%
% %%%%%%%%%%%%%%%%%%%%%%%%%%%%%%%%%%%%%%
% \paragraph{Forwarding for a Complete Draft.}
%
% The following forwarding file |cdocsdrf.tex|
% compiles the main document in draft mode:
%\iffalse
%<*sampledraft>
%\fi
%    \begin{macrocode}
\def\version{draft}
\input{childdoc.def}
\childdocforward{cdocsamp}
%    \end{macrocode}

%\iffalse
%</sampledraft>
%\fi
%
% %%%%%%%%%%%%%%%%%%%%%%%%%%%%%%%%%%%%%%
% \paragraph{Forwarding for Final Version of the Chapters.}
%
% The following forwarding files |cdocsfn1.tex| and |cdocsfn2.tex|
% (with identical content)
% compile the final versions of the child documents
% |cdocsch1.tex| and |cdocsch2.tex|, respectively:
%\iffalse
%<*samplefinal>
%\fi
%    \begin{macrocode}
\def\version{final}
\input{childdoc.def}
\childdocforwardprefix[cdocsamp]{cdocsfn}{cdocsch}
%    \end{macrocode}

%\iffalse
%</samplefinal>
%\fi
%
% %%%%%%%%%%%%%%%%%%%%%%%%%%%%%%%%%%%%%%
% \paragraph{Command Line Processing.}
%
% The following three command lines generate the output files
% |cdocscld|, |cdocscl1| and |cdocscl2|
% which should be identical to
% |cdocsdrf|, |cdocsch1| and |cdocsfn2|, respectively:
% \begin{center}
% \begin{tabular}{l}
% |latex -jobname cdocscld \|\\
% |  "\def\version{draft}\input{childdoc.def}\childdocforward{cdocsamp}"|\\
% |latex -jobname cdocscl1 \|\\
% |  "\input{childdoc.def}\childdocforward[cdocsamp]{cdocsch1}"|\\
% |latex -jobname cdocscl2 \|\\
% |  "\def\version{final}\input{childdoc.def}\childdocforward{cdocsch2}"|
% \end{tabular}
% \end{center}
% Note that the trailing backslash on each first line
% merely continues the input to the second line
% (for convenient cut ant paste).
% Furthermore, the command |latex| can be replaced by any
% of its alternative versions such as |pdflatex|.
%
% %%%%%%%%%%%%%%%%%%%%%%%%%%%%%%%%%%%%%%%%%%%%%%%%%%%%%%%%%%%%%%%%%%%%%%%%%%%%%%
% %%%%%%%%%%%%%%%%%%%%%%%%%%%%%%%%%%%%%%%%%%%%%%%%%%%%%%%%%%%%%%%%%%%%%%%%%%%%%%
% \section{Implementation}
%\iffalse
%<*package>
%\fi
%
% This section describes the definitions file |childdoc.def|.

% The definitions cannot be loaded using |\usepackage| or |\RequirePackage|
% which has a mechanism to prevent loading a style file more than once.
% When loading the definitions by means of |\input|
% multiple instances have to be prevented manually:
%\iffalse
%This code needs to be before the `\ProvidesFile' directive
%which is defined at the beginning of this file.
%Therefore it is also placed there and commented out here.
%</package>
%<*discard>
%\fi
%    \begin{macrocode}
\ifdefined\childdocmain\endinput\fi
%    \end{macrocode}
%\iffalse
%</discard>
%<*package>
%\fi
%
% \macro{\ifchilddoc}
% \macro{\ifchilddocmanual}
% The conditional |\ifchilddoc| tells whether a
% child (true) or main (false) document is being compiled.
% The conditional |\ifchilddocmanual| tells whether
% the |\includeonly| mechanism is used (false) or
% the selection of child files must be performed manually (true).
% The definitions initialise to false:
%    \begin{macrocode}
\newif\ifchilddoc
\newif\ifchilddocmanual
%    \end{macrocode}

% \macro{\childdocname}
% \macro{\childdocjob}
% The macro |\childdocname| stores the name of the main document
% to be compiled. The macro |\childdocjob| stores the name of
% the document on which the \LaTeX{} compiler was originally invoked.
% The content of |\jobname| cannot be compared
% to filenames specified in the source due to different catcodes.
% The following code rescans |\jobname|, stores the result
% in |\childdocname| and saves a copy in |\childdocjob|:
%    \begin{macrocode}
\edef\childdocname{\scantokens\expandafter{\jobname\noexpand}}
\let\childdocjob\childdocname
%    \end{macrocode}

% \macro{\childdocdisable}
% The macro |\childdocdisable| prevents the main file
% from being processed more than once.
% At this stage, the main document command |\childdocmain|
% is assumed to be called once again where it should do nothing.
% Any subsequent call to it should prevent
% a secondary processing of the main document
% It overwrites the forwarding commands
% |\childdocof| and |\childdocforward|
% with empty macros to prevent further inclusions of the main document:
%    \begin{macrocode}
\newcommand{\childdocdisable}
{
  \renewcommand{\childdocmain}[1]{\renewcommand{\childdocmain}[1]{\endinput}}
  \renewcommand{\childdocof}[1]{}
  \renewcommand{\childdocby}[2][]{}
  \renewcommand{\childdocforward}[2][]{}
  \renewcommand{\childdocdisable}{}
}
%    \end{macrocode}

% \macro{\childdocmain}
% The macro |\childdocmain| is to be called at the top of the main file
% with nothing or the main filename (without extension) as argument.
% First, it breaks loops.
% If the argument is not empty and does not match |\childdocname|
% (which is set by the first inclusion of |childdoc.def|),
% |\ifchilddoc| is set to true, |\includeonly| is applied to the child file
% and |\jobname| is set to the main file
% (for proper handling of |.aux| files):
%    \begin{macrocode}
\newcommand{\childdocmain}[1]
{
  \childdocdisable\childdocmain{}
  \if?#1?\else
    \begingroup
      \def\childdoctmp{#1}
      \ifx\childdoctmp\childdocname
        \def\childdoctmp{}
      \else
        \def\childdoctmp
        {
          \childdoctrue
          \includeonly{\childdocname}
          \def\childdocjob{#1}
          \def\jobname{#1}
        }
      \fi
      \expandafter
    \endgroup
    \childdoctmp
  \fi
}
%    \end{macrocode}

% \macro{\childdocof}
% The command |\childdocof| redirects
% compilation to the main file |#1|.
%    \begin{macrocode}
\newcommand{\childdocof}[1]
{
  \childdocdisable
  \childdoctrue
  \includeonly{\childdocname}
  \def\jobname{#1}
  \def\childdocjob{#1}
  \input{#1}
}
%    \end{macrocode}

% \macro{\childdocby}
% The command |\childdocby| ....
%    \begin{macrocode}
\newcommand{\childdocby}[2][]
{
  \childdocdisable
  \childdoctrue
  \childdocmanualtrue
  \if?#1?\else
    \def\jobname{#2}
  \fi
  \def\childdocjob{#2}
  \input{#2}
  \endinput
}
%    \end{macrocode}

% \macro{\childdocforward}
% The command |\childdocforward| redirects
% compilation to the main file or
% (if the optional argument is given) a child file.
% Parameters are set as if the main file
% or a child file starting with |\childdocof| was compiled.
% Then compilation is handed over to the main file:
%    \begin{macrocode}
\newcommand{\childdocforward}[2][]
{
  \begingroup
    \if?#1?
      \def\childdoctmp
      {
        \def\childdocname{#2}
        \def\childdocjob{#2}
        \def\jobname{#2}
        \input{#2}
        \endinput
      }
    \else
      \def\childdoctmp
      {
        \childdocdisable
        \def\childdocname{#2}
        \childdoctrue
        \includeonly{#2}
        \def\childdocjob{#1}
        \def\jobname{#1}
        \input{#1}
        \endinput
      }
    \fi
    \expandafter
  \endgroup
  \childdoctmp
}
%    \end{macrocode}

% \macro{\childdocforwardprefix}
% The command |\childdocforwardprefix| redirects
% compilation to the main or a child file by means of a pattern.
% The prefix |#1| in the current filename is replaced by |#2|
% and the suffix of the current filename is kept
% (it is assumed that the filename does not contain the substring `|~~~|'
% which is used as a delimiter).
% Compilation is handed over to the new file by |\childdocforward|:
%    \begin{macrocode}
\newcommand{\childdocforwardprefix}[3][]
{
  \begingroup
    \def\childdocextract #2##1~~~{\def\childdoctmp{\childdocforward[#1]{#3##1}}}
    \expandafter\childdocextract\childdocname~~~
    \expandafter
  \endgroup
  \childdoctmp
}
%    \end{macrocode}

% \macro{\childdoc}
% The deprecated macro |\childdoc| is a legacy version of |\childdocmain|:
%    \begin{macrocode}
\newcommand{\childdoc}{\childdocmain}
%    \end{macrocode}

% \macro{\childdocredirect}
% The deprecated macro |\childdocredirect| is a legacy version
% of |\childdocforward| and |\childdocforwardprefix|:
%    \begin{macrocode}
\newcommand{\childdocredirect}[2][]
{
  \begingroup
    \if?#1?
      \def\childdoctmp{\childdocforward{#2}}
    \else
      \def\childdoctmp{\childdocforwardprefix{#1}{#2}}
    \fi
    \expandafter
  \endgroup
  \childdoctmp
}
%    \end{macrocode}

%\iffalse
%</package>
%\fi
%
\endinput
|\\
|\childdocforward{|\textit{main}|}|\\
\end{tabular}
\end{center}
%
or alternatively with:
%
\begin{center}
\begin{tabular}{l}
|% \iffalse
%
% childdoc.dtx Copyright (C) 2017-2018 Niklas Beisert
%
% This work may be distributed and/or modified under the
% conditions of the LaTeX Project Public License, either version 1.3
% of this license or (at your option) any later version.
% The latest version of this license is in
%   http://www.latex-project.org/lppl.txt
% and version 1.3 or later is part of all distributions of LaTeX
% version 2005/12/01 or later.
%
% This work has the LPPL maintenance status `maintained'.
%
% The Current Maintainer of this work is Niklas Beisert.
%
% This work consists of the files childdoc.dtx and childdoc.ins
% and the derived files childdoc.def and cdocsamp.tex with
% cdocsch1.tex, cdocsch2.tex, cdocsdrf.tex, cdocsfn1.tex, cdocsfn2.tex.
%
%<package>\ifdefined\childdocmain\endinput\fi
%<package>\ProvidesFile{childdoc.def}[2018/12/30 v2.0 child document driver]
%<samplemain>\ProvidesFile{cdocsamp.tex}[2018/12/30 v2.0 sample for childdoc]
%<*driver>
%\ProvidesFile{childdoc.drv}[2018/12/30 v2.0 childdoc reference manual file]
\PassOptionsToClass{10pt,a4paper}{article}
\documentclass{ltxdoc}

\usepackage[margin=35mm]{geometry}
\usepackage{hyperref}
\usepackage{hyperxmp}
\usepackage[usenames]{color}

\hypersetup{colorlinks=true}
\hypersetup{pdfstartview=FitH}
\hypersetup{pdfpagemode=UseNone}
\hypersetup{pdfsource={}}
\hypersetup{pdflang={en-UK}}
\hypersetup{pdfcopyright={Copyright 2017-2018 Niklas Beisert.
  This work may be distributed and/or modified under the
  conditions of the LaTeX Project Public License, either version 1.3
  of this license or (at your option) any later version.}}
\hypersetup{pdflicenseurl={http://www.latex-project.org/lppl.txt}}
\hypersetup{pdfcontactaddress={ETH Zurich, ITP, HIT K,
  Wolfgang-Pauli-Strasse 27}}
\hypersetup{pdfcontactpostcode={8093}}
\hypersetup{pdfcontactcity={Zurich}}
\hypersetup{pdfcontactcountry={Switzerland}}
\hypersetup{pdfcontactemail={nbeisert@itp.phys.ethz.ch}}
\hypersetup{pdfcontacturl={http://people.phys.ethz.ch/\xmptilde nbeisert/}}

\newcommand{\secref}[1]{\hyperref[#1]{section \ref*{#1}}}

\parskip1ex
\parindent0pt
\let\olditemize\itemize
\def\itemize{\olditemize\parskip0pt}

\begin{document}

\title{The \textsf{childdoc} Package}
\hypersetup{pdftitle={The childdoc Package}}
\author{Niklas Beisert\\[2ex]
  Institut f\"ur Theoretische Physik\\
  Eidgen\"ossische Technische Hochschule Z\"urich\\
  Wolfgang-Pauli-Strasse 27, 8093 Z\"urich, Switzerland\\[1ex]
  \href{mailto:nbeisert@itp.phys.ethz.ch}
  {\texttt{nbeisert@itp.phys.ethz.ch}}}
\hypersetup{pdfauthor={Niklas Beisert}}
\hypersetup{pdfsubject={Manual for the LaTeX2e Package childdoc}}
\date{30 December 2018, \textsf{v2.0}}
\maketitle

\begin{abstract}\noindent
\textsf{childdoc} is a \LaTeXe{} package
that enables the direct compilation
of document sections included by |\include|
to individual files.
\end{abstract}

\begingroup
\parskip0ex
\tableofcontents
\endgroup

%%%%%%%%%%%%%%%%%%%%%%%%%%%%%%%%%%%%%%%%%%%%%%%%%%%%%%%%%%%%%%%%%%%%%%%%%%%%%%%%
%%%%%%%%%%%%%%%%%%%%%%%%%%%%%%%%%%%%%%%%%%%%%%%%%%%%%%%%%%%%%%%%%%%%%%%%%%%%%%%%
\section{Introduction}

\LaTeX{} provides a mechanism to structure a large document (such as a book)
into a main file and several child files (containing the chapters)
using the |\include| command.
This mechanism is beneficial for documents
which span hundreds of pages in order to
make the source file(s) more manageable.
Moreover, compilation can be restricted to
selected child files by means of the |\includeonly| command.
The latter feature can be used to reduce the compilation time while editing
(this was significantly more useful in the earlier days of \LaTeX{})
or to generate a smaller document which is easier to navigate.
Another application of |\includeonly| is to generate
documents consisting of selected parts of the complete document.

However, there are a few drawbacks of the plain |\include| mechanism:
\begin{itemize}
\item
The child files cannot be compiled on their own,
they can only be compiled via the main file.
A naive editing environment
(such as a text editor with an option
to have the current file processed by \LaTeX)
may require one to switch to the main file before compiling;
attempting to compile the child file produces errors.
\item
The main file must be modified (each time)
to adjust the |\includeonly| command
to the present needs. This easily leaves the main file in a messy state.
\item
The generated document will always carry the filename
of the main document. This is inconvenient if
several child files are to be compiled and
to be kept for distribution.
\end{itemize}

The present package provides a simple interface
to make child files individually compilable by \LaTeX{}.
Compiling a child file then has the same effect as compiling
the main file with an |\includeonly| command
to select the appropriate child.
Moreover the generated document will carry the name of the child
rather than the main file.
This resolves all three above issues.

This feature is meant to make the editing of books,
thesis documents and lecture notes somewhat more convenient.
However, the package can also be used efficiently for
composing a series of documents (such as exercise sheets)
which are typically distributed individually.
It then assists the author in generating the individual documents
(potentially in different versions)
as well as a document containing the collected series.
Another application is in developing style files
or other kinds of included material
where compilation of the style file could redirect
to a sample or test file.

%%%%%%%%%%%%%%%%%%%%%%%%%%%%%%%%%%%%%%%%%%%%%%%%%%%%%%%%%%%%%%%%%%%%%%%%%%%%%%%%
%%%%%%%%%%%%%%%%%%%%%%%%%%%%%%%%%%%%%%%%%%%%%%%%%%%%%%%%%%%%%%%%%%%%%%%%%%%%%%%%
\section{Usage}

First of all, the package \textsf{childdoc} is \emph{not} a standard
\LaTeXe{} |.sty| style file! Therefore it needs to be invoked in
a non-standard way.

%%%%%%%%%%%%%%%%%%%%%%%%%%%%%%%%%%%%%%%%%%%%%%%%%%%%%%%%%%%%%%%%%%%%%%%%%%%%%%%%
\subsection{Included Files}
\label{sec:include}

%%%%%%%%%%%%%%%%%%%%%%%%%%%%%%%%%%%%%%%%
\DescribeMacro{\childdocmain}
To use the package, add the commands
\begin{center}
\begin{tabular}{l}
|\input{childdoc.def}|\\
|\childdocmain{}|\\
\end{tabular}
\end{center}
at the very top of the main \LaTeX{} file,
in particular \emph{before} the |\documentclass| statement!
The argument of |\childdocmain| should be left empty
(but it must be present).

%%%%%%%%%%%%%%%%%%%%%%%%%%%%%%%%%%%%%%%%
\DescribeMacro{\childdocof}
Furthermore, add the commands
\begin{center}
\begin{tabular}{l}
|\input{childdoc.def}|\\
|\childdocof{|\textit{main}|}|\\
\end{tabular}
\end{center}
at the top of every child file \textit{child}
which is included by |\include{|\textit{child}|}|
from within the main file
(or at least for those files to be compiled individually).
The argument \textit{main} must be the filename of the main file.

There are a couple of
considerations in setting up the main and child documents:

%%%%%%%%%%%%%%%%%%%%%%%%%%%%%%%%%%%%%%%%
\paragraph{Restrictions.}

Please note the following restrictions:
\begin{itemize}
\item
|\childdocmain| must be called with one argument \textit{main}
to ensure compatibility with earlier version of the package.
It must either be empty (|\childdocmain{}|)
or precisely match the filename of the main file in which it is specified.
See \secref{sec:detection} for further information.
\item
The filename \textit{main} must be specified without the |.tex| extension.
\item
The filename \textit{main} is case sensitive
(even in case-insensitive file systems)
due to internal string comparison.
\item
The argument \textit{main} should be fully expanded, it cannot be a macro.
\item
Subdirectories and special characters should be avoided in filenames.
\item
The command |\childdocmain{|\textit{main}|}| must be followed by a whitespace.
It should not be followed immediately by another command
or by a comment mark `|%|'.
This is because the \TeX{} parser reads the token immediately following
the argument of |\childdocmain| and puts it
at the beginning of every child section;
however, a white\-space is ignored.
\end{itemize}

%%%%%%%%%%%%%%%%%%%%%%%%%%%%%%%%%%%%%%%%
\paragraph{Content of Main File.}

It is advisable to place all content in the child files included by |\include|.
Any output contained in the main file will appear in all child documents
unless suppressed manually;
it cannot be suppressed automatically by the |\includeonly| directive
and thus should normally be avoided.
A method to include some content in the main file
by means of conditional processing is described in \secref{sec:conditional}.

%%%%%%%%%%%%%%%%%%%%%%%%%%%%%%%%%%%%%%%%
\paragraph{Page Numbering.}

When only a part of the document is compiled,
the appropriate numbering of pages
(as well as other status parameters)
is determined from the |.aux| files.
The latter contain information from previous passes.
However this information needs to propagate through
all intermediate child documents.
Therefore the page numbering in child documents may well
be inconsistent until the complete document is compiled at least once.

A useful (if unconventional) way to always ensure a consistent
page numbering is to restart the numbering in each child document
and denote the pages by `\textit{child}|.|\textit{page}'
where \textit{child} represents the chapter/section number of the child file.
This can be achieved by the command
|\numberwithin{page}{|\textit{child}|}|
of the \textsf{amsmath} package
where \textit{child} can be |chapter| or |section|
depending on the chosen structuring.
Alternatively, one can modify the macro |\thepage| appropriately
and reset the counter |page| at the start of each child file.

%%%%%%%%%%%%%%%%%%%%%%%%%%%%%%%%%%%%%%%%%%%%%%%%%%%%%%%%%%%%%%%%%%%%%%%%%%%%%%%%
\subsection{Conditional Processing}
\label{sec:conditional}

The package provides a mechanism to compile different versions
of a document. To customise the versions further some conditional processing
can come in handy to distinguish which version is being compiled.
The package provides two macros to describe the compilation context:

%%%%%%%%%%%%%%%%%%%%%%%%%%%%%%%%%%%%%%%%
\DescribeMacro{\ifchilddoc}
The conditional |\ifchilddoc| distinguishes between the compilation of
child documents and the main document:
%
\begin{center}
|\ifchilddoc |\textit{child-code}| |[|\||else |\textit{main-code}]| \||fi|
\end{center}

%%%%%%%%%%%%%%%%%%%%%%%%%%%%%%%%%%%%%%%%
\DescribeMacro{\childdocname}
\DescribeMacro{\childdocjob}
The macro |\childdocname| contains the filename (without extension)
of the main or child file being processed.
Note that |\childdocjob| will always contain the name of the main file.

%%%%%%%%%%%%%%%%%%%%%%%%%%%%%%%%%%%%%%%%
\paragraph{Title Page.}

Conditional processing can be used to include a title or banner page
in the main document when proper precautions are taken.
Importantly, the code in the main file should ensure that the page counter
(as well as other status parameters which are stored in the |.aux| files)
takes the same value after the conditional processing.
Otherwise the page numbers may take divergent values
depending on which part is compiled.

For example, a title page could be declared by:
%
\begin{center}
\begin{tabular}{l}
|\ifchilddoc\||else|\\
|\addtocounter{page}{-1}|\\
\textit{code for title page}\\
|\newpage|\\
|\||fi|
\end{tabular}
\end{center}
%
A banner page for the child documents can be generated by:
%
\begin{center}
\begin{tabular}{l}
|\ifchilddoc|\\
|\addtocounter{page}{-1}|\\
\textit{code for banner page}\\
|\newpage|\\
|\||fi|
\end{tabular}
\end{center}
%
Here one could write a message such as:
\begin{center}
|This is the part \childdocname{} of \childdocjob{}.|
\end{center}

%%%%%%%%%%%%%%%%%%%%%%%%%%%%%%%%%%%%%%%%%%%%%%%%%%%%%%%%%%%%%%%%%%%%%%%%%%%%%%%%
\subsection{Flags}
\label{sec:flags}

The package makes it easy to generate different versions
of the main or child documents.
To this end compilation flags can be defined
and assigned different default values.
They will be particularly useful in conjunction
with the forwarding mechanism described in \secref{sec:forward}.

For example, it may be useful to have a flag |\version|
which can be set to |draft| or |final|.
The document source will contain some conditional code
depending on the value of |\version|.
Suppose further, the flag should default to |final| for the main file
and to |draft| for child files
which is a natural assignment for editing the document.
This is achieved by placing the following code
in the preamble of the main document
(below the |\childdocmain| directive):
%
\begin{center}
\begin{tabular}{l}
|\ifchilddoc|\\
|\providecommand{\version}{draft}|\\
|\||else|\\
|\providecommand{\version}{final}|\\
|\||fi|
\end{tabular}
\end{center}
%
The definition by |\providecommand| makes sure
that previous definitions are not overwritten.
Further statements |\providecommand{\version}{...}|
can thus be added before the above code to override it.

For the main file, one might add a line
(between |\childdocmain| and the above block)
%
\begin{center}
|%\ifchilddoc\||else\providecommand{\version}{draft}\||fi|
\end{center}
%
which can be uncommented to produce a draft version.
Likewise one can add a line to the very top of a child file
(above the |\childdocof{|\textit{main}|}| directive)
%
\begin{center}
|%\providecommand{\version}{final}|
\end{center}
%
which can be uncommented to produce the final version of this child document.

%%%%%%%%%%%%%%%%%%%%%%%%%%%%%%%%%%%%%%%%%%%%%%%%%%%%%%%%%%%%%%%%%%%%%%%%%%%%%%%%
\subsection{Forwarding}
\label{sec:forward}

Different versions of the main or child documents
using compilation flags as described in \secref{sec:flags}
can be (permanently) stored in different files
for convenient compilation, viewing and distribution.
To this end, the package defines a command
to pass on compilation to a different file:

%%%%%%%%%%%%%%%%%%%%%%%%%%%%%%%%%%%%%%%%
\DescribeMacro{\childdocforward}
The command |\childdocforward| redirects processing to
another source file:
%
\begin{center}
\begin{tabular}{l}
|\input{childdoc.def}|\\
|\childdocforward[|\textit{main}|]{|\textit{dest}|}|\\
\end{tabular}
\end{center}
%
The argument \textit{dest} is the destination file
(without extension).
It should be the main file or one of the child files.
Note that further \textsf{childdoc} directives
such as |\childdocof| and |\childdocforward|
in the indicated file will be processed in this form.
The optional argument \textit{main}
passes on directly to the main file \textit{main}
while pretending to compile the child \textit{dest}.
This form behaves as if \textit{dest}
issues |\childdocof{|\textit{main}|}| right away,
and no further \textsf{childdoc} directives will be processed.

%%%%%%%%%%%%%%%%%%%%%%%%%%%%%%%%%%%%%%%%
\DescribeMacro{\...prefix}
In the alternative form |\childdocforwardprefix|,
%
\begin{center}
\begin{tabular}{l}
|\input{childdoc.def}|\\
|\childdocforwardprefix[|\textit{main}|]{|\textit{prefix}|}{|\textit{dest}|}|
\end{tabular}
\end{center}
%
the destination file is determined by a pattern
depending on the current file:
To make this work, the current file must be called
`{\textit{prefix}\hspace{0.2em}\textit{suffix}}'
with \textit{prefix} matching precisely the argument.
Processing is then passed on to the file
`{\textit{dest}\hspace{0.2em}\textit{suffix}}'.
Surely, the same effect is achieved by
directly specifying the
argument `{\textit{dest}\hspace{0.2em}\textit{suffix}}'
in the first form.
However, that requires to set up a different file
for each child. With the alternative form of the command
all these files can have exactly the same content
which simplifies setting them up and maintaining them.

For example, the following file |draft.tex|
with a compilation flag |\version| as described in \secref{sec:flags}
compiles the main document as a draft:
%
\begin{center}
\begin{tabular}{l}
|\def\version{draft}|\\
|\input{childdoc.def}|\\
|\childdocforward{|\textit{main}|}|
\end{tabular}
\end{center}
%
Likewise, the following files |final|\textit{nn}|.tex|
compile the final version of the child document
|child|\textit{nn}|.tex|:
%
\begin{center}
\begin{tabular}{l}
|\def\version{final}|\\
|\input{childdoc.def}|\\
|\childdocforwardprefix{final}{child}|
\end{tabular}
\end{center}
%

Note that when several versions of a main file and/or of each child file
are to be generated, it may be convenient to set up a |Makefile| or
shell script to automatise the process.

%%%%%%%%%%%%%%%%%%%%%%%%%%%%%%%%%%%%%%%%%%%%%%%%%%%%%%%%%%%%%%%%%%%%%%%%%%%%%%%%
\subsection{Command Line Processing}
\label{sec:commandline}

The effect of redirection files can also be achieved by invoking
the \LaTeX{} compiler with a more elaborate command line.
Most conveniently this should be done as part
of a shell script or a |Makefile|.

When using \textsf{childdoc} in the main file, the following
command lines effectively perform a redirection
(note that depending on the shell being used,
backslashes may have to be doubled: `|\|' $\to$ `|\\|'):
%
\begin{center}
|... -jobname "|\textit{target}|" |\\|"|[\textit{flags}]%
|\input{childdoc.def}\childdocforward[|\textit{main}|]{|\textit{dest}|}"|
\end{center}
%
Here \textit{target} is the name of the output file,
\textit{main} is the name of the main file
and \textit{dest} is the name of the main or child file to be processed
(all filenames without extensions).
The optional argument \textit{main} can be omitted
if \textit{main} matches \textit{dest}.
Optionally, compilation \textit{flags} can be defined via |\def| commands.
This command line makes the \TeX{} engine believe
it is compiling the file \textit{target}
whose content is specified as the latter parameter.
The provided code then forwards the processing to
\textit{main} or \textit{dest} as described in \secref{sec:forward}.

%%%%%%%%%%%%%%%%%%%%%%%%%%%%%%%%%%%%%%%%%%%%%%%%%%%%%%%%%%%%%%%%%%%%%%%%%%%%%%%%
\subsection{Include by Input}
\label{sec:input}

Including child documents by |\include| has some restrictions by design.
Most notably, the content of a child document always occupies
its own set of pages; pages cannot be shared between child documents.
Usually, this behaviour makes perfect sense
because each child document contain an essential part of the document.
However, in some situations it may be desirable to compose
a document from a collection of parts
without having mandatory page breaks between then.
For this case, the package
provides a mechanism to include parts
by |\input| which can also be processed individually.
However, by construction this mechanism
requires manual handling of the content to be output.

%%%%%%%%%%%%%%%%%%%%%%%%%%%%%%%%%%%%%%%%
\DescribeMacro{\ifchilddocmanual}
The main file should be prepared as usual, see \secref{sec:include}.
However, the document body must make a distinction
between processing of an individual part and of the main document, e.g.:
%
\begin{center}
\begin{tabular}{l}
|\ifchilddocmanual|\\
|\input{\childdocname}|\\
|\||else|\\
\textit{document body with }|\input{|\textit{part}|}|\\
|\||fi|
\end{tabular}
\end{center}
%
The conditional |\ifchilddocmanual| is true whenever
a part to be included by |\input| is being compiled,
and the name of the part is stored in |\childdocname|.

%%%%%%%%%%%%%%%%%%%%%%%%%%%%%%%%%%%%%%%%
\DescribeMacro{\childdocby}
Each part to be included by |\input| should start with:
%
\begin{center}
\begin{tabular}{l}
|\input{childdoc.def}|\\
|\childdocby{|\textit{main}|}|\\
\end{tabular}
\end{center}
%
The directive |\childdocby| is similar to |\childdocof|
described in \secref{sec:include},
but the subsequent selection of content must be done manually.
To that end, both |\ifchilddoc| and |\ifchilddocmanual|
will be true upon processing of a part,
and the name of the part is stored in |\childdocname|.
Note that |\jobname| will be set to the filename of the current part
so that each part receives an individual |.aux| file
that does not interfere with the |.aux| file(s) of the main document.
This behaviour can be altered by the alternative form
|\childdocby[*]{|\textit{main}|}| (with a non-empty optional argument)
which uses the |.aux| file of the main document
by setting |\jobname| to \textit{main}.

%%%%%%%%%%%%%%%%%%%%%%%%%%%%%%%%%%%%%%%%%%%%%%%%%%%%%%%%%%%%%%%%%%%%%%%%%%%%%%%%
\subsection{Driver Development}
\label{sec:driver}

The \textsf{childdoc} mechanism can also be use for the development
of definition files such as \LaTeX{} styles or classes.
This case differs from the above setup with multiple parts
included by |\include| in that no |\includeonly| should be invoked.
This can be achieved by starting the include file
(before |\ProvidesPackage|) with:
%
\begin{center}
\begin{tabular}{l}
|\input{childdoc.def}|\\
|\childdocforward{|\textit{main}|}|\\
\end{tabular}
\end{center}
%
or alternatively with:
%
\begin{center}
\begin{tabular}{l}
|\input{childdoc.def}|\\
|\childdocby{|\textit{main}|}|\\
\end{tabular}
\end{center}
%
Both forms have slightly different effects as described above.
The main file is prepared as usual, see \secref{sec:include}.

%%%%%%%%%%%%%%%%%%%%%%%%%%%%%%%%%%%%%%%%%%%%%%%%%%%%%%%%%%%%%%%%%%%%%%%%%%%%%%%%
\subsection{Legacy Detection}
\label{sec:detection}

The directive |\childdocmain| in the main file can detect
whether the complete document or merely a child is to be compiled
even without using the directive |\childdocof|.
This method is deprecated because it is less robust
and there is no compelling reason to use it;
it is merely provided for backward compatibility
and it may be removed in future versions.

If the detection mechanism is to be used,
it is mandatory to correctly specify
the filename of the main file as the argument of |\childdocmain|:
%
\begin{center}
\begin{tabular}{l}
|\input{childdoc.def}|\\
|\childdocmain{|\textit{main}|}|\\
\end{tabular}
\end{center}
%
If |\jobname| does not match the argument \textit{main} of |\childdocmain|,
it is assumed that |\jobname| points to the child file to be compiled.
When using |\childdocmain| with the main file specified as argument,
it suffices to start a child file
with just |\input{|\textit{main}|}|
without loading of the package and using |\childdocof|.
If instead all processing is done
with the appropriate \textsf{childdoc} directives,
the argument of \textit{main} of |\childdocmain| can be empty.

An alternative version of the command line processing described
in \secref{sec:commandline} using the detection mechanism reads:
%
\begin{center}
|... -jobname "|\textit{target}|" "|[\textit{flags}]%
[|\def\jobname{|\textit{dest}|}|]|\input{|\textit{main}|}"|
\end{center}

%%%%%%%%%%%%%%%%%%%%%%%%%%%%%%%%%%%%%%%%%%%%%%%%%%%%%%%%%%%%%%%%%%%%%%%%%%%%%%%%
\subsection{Manual Code}
\label{sec:manual}

In case one cannot be certain whether the definitions file |childdoc.def|
is installed on the target \TeX{} distribution
and one prefers not to ship it,
it is conceivable to paste a few relevant commands into the sources.

To that end, drop all statements |\input{childdoc.def}|
and perform the replacements as outlined below.
Instead of |\childdocmain{|\textit{main}|}| add the following code
to the top of the main file:
%
\begin{center}
\begin{tabular}{l}
|\||ifdefined\childdocname\endinput\||fi\newif\ifchilddoc|\\
|\edef\childdocname{\scantokens\expandafter{\jobname\noexpand}}|\\
|\def\childdocmain{|\textit{main}|}\||ifx\childdocmain\childdocname\||else|\\
|\childdoctrue\includeonly{\childdocname}\let\jobname\childdocmain\||fi|\\
\end{tabular}
\end{center}
%
Instead of |\childdocof{|\textit{main}|}| just include the main file
at the top of each child file:
%
\begin{center}
|\input{|\textit{main}|}|
\end{center}
%
A simple redirection |\childdocforward{|\textit{dest}|}| is achieved by:
%
\begin{center}
|\def\jobname{|\textit{dest}|}\input{\jobname}|
\end{center}
%
The redirection with prefix
|\childdocforwardprefix[|\textit{prefix}|]{|\textit{dest}|}|
is accomplished by:
%
\begin{center}
\begin{tabular}{l}
|{\edef\jobname{\scantokens\expandafter{\jobname\noexpand}}|\\
|\def\redirectjob |\textit{prefix}|#1~~~{\gdef\jobname{|\textit{dest}|#1}}|\\
|\expandafter\redirectjob\jobname~~~}\input{\jobname}|
\end{tabular}
\end{center}

In an alternative approach,
child documents can be compiled by a specific command line
without additional code or specific definitions:
%
\begin{center}
|... -jobname "|\textit{target}|" "|[\textit{flags}]%
|\includeonly{|\textit{dest}|}\input{|\textit{main}|}"|
\end{center}
%

%%%%%%%%%%%%%%%%%%%%%%%%%%%%%%%%%%%%%%%%%%%%%%%%%%%%%%%%%%%%%%%%%%%%%%%%%%%%%%%%
%%%%%%%%%%%%%%%%%%%%%%%%%%%%%%%%%%%%%%%%%%%%%%%%%%%%%%%%%%%%%%%%%%%%%%%%%%%%%%%%
\section{Information}

%%%%%%%%%%%%%%%%%%%%%%%%%%%%%%%%%%%%%%%%%%%%%%%%%%%%%%%%%%%%%%%%%%%%%%%%%%%%%%%%
\subsection{Copyright}

Copyright \copyright{} 2017--2018 Niklas Beisert

This work may be distributed and/or modified under the
conditions of the \LaTeX{} Project Public License, either version 1.3
of this license or (at your option) any later version.
The latest version of this license is in
  \url{http://www.latex-project.org/lppl.txt}
and version 1.3 or later is part of all distributions of \LaTeX{}
version 2005/12/01 or later.

This work has the LPPL maintenance status `maintained'.

The Current Maintainer of this work is Niklas Beisert.

This work consists of the files |README.txt|, |childdoc.ins| and |childdoc.dtx|
as well as the derived files |childdoc.def|, |cdocsamp.tex|
with |cdocsch1.tex|, |cdocsch2.tex|, |cdocspt3.tex|, |cdocspt4.tex|,
|cdocsdrf.tex|, |cdocsfn1.tex|, |cdocsfn2.tex|
as well as |childdoc.pdf|.

%%%%%%%%%%%%%%%%%%%%%%%%%%%%%%%%%%%%%%%%%%%%%%%%%%%%%%%%%%%%%%%%%%%%%%%%%%%%%%%%
\subsection{Files and Installation}

The package consists of the files:
%
\begin{center}
\begin{tabular}{ll}
    |README.txt|   & readme file \\
    |childdoc.ins| & installation file \\
    |childdoc.dtx| & source file \\
    |childdoc.def| & definition file \\
    |cdocsamp.tex| & sample main file \\
    |cdocsch1.tex| & sample include file \\
    |cdocsch2.tex| & sample include file \\
    |cdocspt3.tex| & sample part file \\
    |cdocspt4.tex| & sample part file \\
    |cdocsdrf.tex| & sample redirection file \\
    |cdocsfn1.tex| & sample redirection file \\
    |cdocsfn2.tex| & sample redirection file \\
    |childdoc.pdf| & manual
\end{tabular}
\end{center}
%
The distribution consists of the files
|README.txt|, |childdoc.ins| and |childdoc.dtx|.
%
\begin{itemize}
\item
Run (pdf)\LaTeX{} on |childdoc.dtx|
to compile the manual |childdoc.pdf| (this file).
\item
Run \LaTeX{} on |childdoc.ins| to create the definitions file |childdoc.def|
and the sample |cdocsamp.tex| with include files
|cdocsch1.tex|, |cdocsch2.tex|, |cdocspt3.tex|, |cdocspt4.tex|,
|cdocsdrf.tex|, |cdocsfn1.tex|, |cdocsfn2.tex|.
Then copy the file |childdoc.def| to an appropriate directory of your \LaTeX{}
distribution, e.g.\ \textit{texmf-root}|/tex/latex/childdoc|.
\end{itemize}

%%%%%%%%%%%%%%%%%%%%%%%%%%%%%%%%%%%%%%%%%%%%%%%%%%%%%%%%%%%%%%%%%%%%%%%%%%%%%%%%
\subsection{Related CTAN Packages}

There are several other packages which offer a similar functionality:
%
\begin{itemize}
\item
The packages
\href{http://ctan.org/pkg/docmute}{\textsf{docmute}},
\href{http://ctan.org/pkg/includex}{\textsf{includex}} and
\href{http://ctan.org/pkg/standalone}{\textsf{standalone}}
provide commands to include only the document body of
a child file thus allowing both files to be compiled individually.
\item
The packages \href{http://ctan.org/pkg/subdocs}{\textsf{subdocs}}
and \href{http://ctan.org/pkg/subfiles}{\textsf{subfiles}}
provide structures in which the main and child documents can be
encapsulated and allowing them to be compiled individually.
The inclusion mechanism is different from the conventional |\include|.
\item
The package \href{http://ctan.org/pkg/combine}{\textsf{combine}}
is an elaborate solution to combine several documents into one.
\end{itemize}
%
See also the CTAN topic \href{http://ctan.org/topic/subdocs}{\textsf{subdocs}}
for further related packages.
The present package differs from the above solutions in that
a document structure constructed with the conventional |\include| mechanism
just needs two extra commands at the top of every file
such that all constituent files can be compiled individually.

%%%%%%%%%%%%%%%%%%%%%%%%%%%%%%%%%%%%%%%%%%%%%%%%%%%%%%%%%%%%%%%%%%%%%%%%%%%%%%%%
%\subsection{Feature Suggestions}
%
%The following is a list of features which may be useful for future
%versions of this package:
%%
%\begin{itemize}
%\item
%\ldots
%\end{itemize}

%%%%%%%%%%%%%%%%%%%%%%%%%%%%%%%%%%%%%%%%%%%%%%%%%%%%%%%%%%%%%%%%%%%%%%%%%%%%%%%%
\subsection{Revision History}

%%%%%%%%%%%%%%%%%%%%%%%%%%%%%%%%%%%%%%%%
\paragraph{v2.0:} 2018/12/30

\begin{itemize}
\item
immediate forward processing
\item
added |\childdocby| mechanism
\item
manual restructured
\end{itemize}

%%%%%%%%%%%%%%%%%%%%%%%%%%%%%%%%%%%%%%%%
\paragraph{v1.6:} 2018/01/17

\begin{itemize}
\item
application for development of include files
\item
corrections to manual
\end{itemize}

%%%%%%%%%%%%%%%%%%%%%%%%%%%%%%%%%%%%%%%%
\paragraph{v1.5:} 2017/05/21

\begin{itemize}
\item
more complete structuring introduced
\item
|\childdocof| introduced
\item
|\childdoc| renamed to |\childdocmain|
\item
|\childredirect| renamed to |\childdocforward| and |\childdocforwardprefix|
and functionality expanded
\end{itemize}

%%%%%%%%%%%%%%%%%%%%%%%%%%%%%%%%%%%%%%%%
\paragraph{v1.0:} 2017/04/27

\begin{itemize}
\item
manual and install package
\item
first version published on CTAN
\end{itemize}

%%%%%%%%%%%%%%%%%%%%%%%%%%%%%%%%%%%%%%%%
\paragraph{v0.6:} 2017/04/26

\begin{itemize}
\item
redirection mechanism added
\end{itemize}

%%%%%%%%%%%%%%%%%%%%%%%%%%%%%%%%%%%%%%%%
\paragraph{v0.5:} 2017/04/26

\begin{itemize}
\item
functionality in definition file
\end{itemize}


%%%%%%%%%%%%%%%%%%%%%%%%%%%%%%%%%%%%%%%%%%%%%%%%%%%%%%%%%%%%%%%%%%%%%%%%%%%%%%%%
%%%%%%%%%%%%%%%%%%%%%%%%%%%%%%%%%%%%%%%%%%%%%%%%%%%%%%%%%%%%%%%%%%%%%%%%%%%%%%%%
%%%%%%%%%%%%%%%%%%%%%%%%%%%%%%%%%%%%%%%%%%%%%%%%%%%%%%%%%%%%%%%%%%%%%%%%%%%%%%%%
\appendix

\settowidth\MacroIndent{\rmfamily\scriptsize 000\ }

 \DocInput{childdoc.dtx}

\end{document}
%</driver>
% \fi
%
% %%%%%%%%%%%%%%%%%%%%%%%%%%%%%%%%%%%%%%%%%%%%%%%%%%%%%%%%%%%%%%%%%%%%%%%%%%%%%%
% %%%%%%%%%%%%%%%%%%%%%%%%%%%%%%%%%%%%%%%%%%%%%%%%%%%%%%%%%%%%%%%%%%%%%%%%%%%%%%
% \section{Sample}
%\iffalse
%<*samplemain>
%\fi
%
% The following presents a sample document
% with two chapters, two parts, a title page,
% a compile flag as well as three forwarding files to set the flag.
% It consists of eight |.tex| files:
% \begin{center}
% \begin{tabular}{ll}
% |cdocsamp.tex|&main file\\
% |cdocsch1.tex|&include file for chapter 1\\
% |cdocsch2.tex|&include file for chapter 2\\
% |cdocspt3.tex|&include file for part 3\\
% |cdocspt4.tex|&include file for part 4\\
% |cdocsdrf.tex|&forwarding file for main file in draft mode\\
% |cdocsfi1.tex|&forwarding file for final version of chapter 1\\
% |cdocsfi2.tex|&forwarding file for final version of chapter 2\\
% \end{tabular}
% \end{center}
% Each of the eight files can be compiled directly by the \LaTeX{} compiler.
%
% %%%%%%%%%%%%%%%%%%%%%%%%%%%%%%%%%%%%%%
% \paragraph{Main File.}
%
% The main file is called |cdocsamp.tex|.
%
% Load the \textsf{childdoc} definitions and
% declare the filename for the main document:
%    \begin{macrocode}
\input{childdoc.def}
\childdocmain{}
%    \end{macrocode}

% Optional override for |\version| flag:
%    \begin{macrocode}
%%\ifchilddoc\else\providecommand{\version}{draft}\fi
%    \end{macrocode}

% Define the default values for the |\version| flag
% (|final| for the main file and |draft| for childs):
%    \begin{macrocode}
\ifchilddoc
\providecommand{\version}{draft}
\else
\providecommand{\version}{final}
\fi
%    \end{macrocode}

% Load the standard document class:
%    \begin{macrocode}
\documentclass[12pt]{article}
%    \end{macrocode}

% Start the document body:
%    \begin{macrocode}
\begin{document}
%    \end{macrocode}

% Declare a title page.
% Print title, part of document being processed and version flag:
%    \begin{macrocode}
\addtocounter{page}{-1}
\begin{center}
{\LARGE\bfseries{}childdoc example\par}
\vspace{1cm}
\ifchilddoc
\ifchilddocmanual part\else chapter\fi:
`\childdocname' of `\childdocjob'\par
\else
main document: `\childdocjob'\par
\fi
version: \version\par
\end{center}
\newpage
%    \end{macrocode}

% Manually include selected file,
% otherwise process as usual:
%    \begin{macrocode}
\ifchilddocmanual
\section*{part `\childdocname'}
\input{\childdocname}
\else
%    \end{macrocode}

% Include the two chapters:
%    \begin{macrocode}
\include{cdocsch1}
\include{cdocsch2}
%    \end{macrocode}

% Include the two parts unless only chapters should be displayed:
%    \begin{macrocode}
\ifchilddoc\else
\section{part three}
\input{cdocspt3}
\section{part four}
\input{cdocspt4}
\fi
%    \end{macrocode}

% Process as usual until here:
%    \begin{macrocode}
\fi
%    \end{macrocode}

% End of document body:
%    \begin{macrocode}
\end{document}
%    \end{macrocode}
%\iffalse
%</samplemain>
%\fi
%
% %%%%%%%%%%%%%%%%%%%%%%%%%%%%%%%%%%%%%%
% \paragraph{Chapter Include Files.}
%
% The include files are called |cdocsch1.tex| and |cdocsch2.tex|.
%
%\iffalse
%<*samplechap1|samplechap2>
%\fi

% Optional override for |\version| flag:
%    \begin{macrocode}
%%\providecommand{\version}{final}
%    \end{macrocode}

% Include the main document:
%    \begin{macrocode}
\input{childdoc.def}
\childdocof{cdocsamp}
%    \end{macrocode}

%\iffalse
%</samplechap1|samplechap2>
%\fi
%
%\iffalse
%<*samplechap1>
%\fi
% Some text for chapter 1:
%    \begin{macrocode}
\section{one}
some text in chapter one
%    \end{macrocode}

%\iffalse
%</samplechap1>
%\fi
% Some text for chapter 2:
%\iffalse
%<*samplechap2>
%\fi
%    \begin{macrocode}
\section{two}
more text in chapter two
%    \end{macrocode}

%\iffalse
%</samplechap2>
%\fi
%
% %%%%%%%%%%%%%%%%%%%%%%%%%%%%%%%%%%%%%%
% \paragraph{Part Include Files.}
%
% The include files are called |cdocspt3.tex| and |cdocspt4.tex|.
%
%\iffalse
%<*samplepart3|samplepart4>
%\fi

% Optional override for |\version| flag:
%    \begin{macrocode}
%%\providecommand{\version}{final}
%    \end{macrocode}

% Include the main document:
%    \begin{macrocode}
\input{childdoc.def}
\childdocby{cdocsamp}
%    \end{macrocode}

%\iffalse
%</samplepart3|samplepart4>
%\fi
%
%\iffalse
%<*samplepart3>
%\fi
% Some text for part 3:
%    \begin{macrocode}
some text in part three
%    \end{macrocode}

%\iffalse
%</samplepart3>
%\fi
% Some text for part 4:
%\iffalse
%<*samplepart4>
%\fi
%    \begin{macrocode}
more text in part four
%    \end{macrocode}

%\iffalse
%</samplepart4>
%\fi
%
% %%%%%%%%%%%%%%%%%%%%%%%%%%%%%%%%%%%%%%
% \paragraph{Forwarding for a Complete Draft.}
%
% The following forwarding file |cdocsdrf.tex|
% compiles the main document in draft mode:
%\iffalse
%<*sampledraft>
%\fi
%    \begin{macrocode}
\def\version{draft}
\input{childdoc.def}
\childdocforward{cdocsamp}
%    \end{macrocode}

%\iffalse
%</sampledraft>
%\fi
%
% %%%%%%%%%%%%%%%%%%%%%%%%%%%%%%%%%%%%%%
% \paragraph{Forwarding for Final Version of the Chapters.}
%
% The following forwarding files |cdocsfn1.tex| and |cdocsfn2.tex|
% (with identical content)
% compile the final versions of the child documents
% |cdocsch1.tex| and |cdocsch2.tex|, respectively:
%\iffalse
%<*samplefinal>
%\fi
%    \begin{macrocode}
\def\version{final}
\input{childdoc.def}
\childdocforwardprefix[cdocsamp]{cdocsfn}{cdocsch}
%    \end{macrocode}

%\iffalse
%</samplefinal>
%\fi
%
% %%%%%%%%%%%%%%%%%%%%%%%%%%%%%%%%%%%%%%
% \paragraph{Command Line Processing.}
%
% The following three command lines generate the output files
% |cdocscld|, |cdocscl1| and |cdocscl2|
% which should be identical to
% |cdocsdrf|, |cdocsch1| and |cdocsfn2|, respectively:
% \begin{center}
% \begin{tabular}{l}
% |latex -jobname cdocscld \|\\
% |  "\def\version{draft}\input{childdoc.def}\childdocforward{cdocsamp}"|\\
% |latex -jobname cdocscl1 \|\\
% |  "\input{childdoc.def}\childdocforward[cdocsamp]{cdocsch1}"|\\
% |latex -jobname cdocscl2 \|\\
% |  "\def\version{final}\input{childdoc.def}\childdocforward{cdocsch2}"|
% \end{tabular}
% \end{center}
% Note that the trailing backslash on each first line
% merely continues the input to the second line
% (for convenient cut ant paste).
% Furthermore, the command |latex| can be replaced by any
% of its alternative versions such as |pdflatex|.
%
% %%%%%%%%%%%%%%%%%%%%%%%%%%%%%%%%%%%%%%%%%%%%%%%%%%%%%%%%%%%%%%%%%%%%%%%%%%%%%%
% %%%%%%%%%%%%%%%%%%%%%%%%%%%%%%%%%%%%%%%%%%%%%%%%%%%%%%%%%%%%%%%%%%%%%%%%%%%%%%
% \section{Implementation}
%\iffalse
%<*package>
%\fi
%
% This section describes the definitions file |childdoc.def|.

% The definitions cannot be loaded using |\usepackage| or |\RequirePackage|
% which has a mechanism to prevent loading a style file more than once.
% When loading the definitions by means of |\input|
% multiple instances have to be prevented manually:
%\iffalse
%This code needs to be before the `\ProvidesFile' directive
%which is defined at the beginning of this file.
%Therefore it is also placed there and commented out here.
%</package>
%<*discard>
%\fi
%    \begin{macrocode}
\ifdefined\childdocmain\endinput\fi
%    \end{macrocode}
%\iffalse
%</discard>
%<*package>
%\fi
%
% \macro{\ifchilddoc}
% \macro{\ifchilddocmanual}
% The conditional |\ifchilddoc| tells whether a
% child (true) or main (false) document is being compiled.
% The conditional |\ifchilddocmanual| tells whether
% the |\includeonly| mechanism is used (false) or
% the selection of child files must be performed manually (true).
% The definitions initialise to false:
%    \begin{macrocode}
\newif\ifchilddoc
\newif\ifchilddocmanual
%    \end{macrocode}

% \macro{\childdocname}
% \macro{\childdocjob}
% The macro |\childdocname| stores the name of the main document
% to be compiled. The macro |\childdocjob| stores the name of
% the document on which the \LaTeX{} compiler was originally invoked.
% The content of |\jobname| cannot be compared
% to filenames specified in the source due to different catcodes.
% The following code rescans |\jobname|, stores the result
% in |\childdocname| and saves a copy in |\childdocjob|:
%    \begin{macrocode}
\edef\childdocname{\scantokens\expandafter{\jobname\noexpand}}
\let\childdocjob\childdocname
%    \end{macrocode}

% \macro{\childdocdisable}
% The macro |\childdocdisable| prevents the main file
% from being processed more than once.
% At this stage, the main document command |\childdocmain|
% is assumed to be called once again where it should do nothing.
% Any subsequent call to it should prevent
% a secondary processing of the main document
% It overwrites the forwarding commands
% |\childdocof| and |\childdocforward|
% with empty macros to prevent further inclusions of the main document:
%    \begin{macrocode}
\newcommand{\childdocdisable}
{
  \renewcommand{\childdocmain}[1]{\renewcommand{\childdocmain}[1]{\endinput}}
  \renewcommand{\childdocof}[1]{}
  \renewcommand{\childdocby}[2][]{}
  \renewcommand{\childdocforward}[2][]{}
  \renewcommand{\childdocdisable}{}
}
%    \end{macrocode}

% \macro{\childdocmain}
% The macro |\childdocmain| is to be called at the top of the main file
% with nothing or the main filename (without extension) as argument.
% First, it breaks loops.
% If the argument is not empty and does not match |\childdocname|
% (which is set by the first inclusion of |childdoc.def|),
% |\ifchilddoc| is set to true, |\includeonly| is applied to the child file
% and |\jobname| is set to the main file
% (for proper handling of |.aux| files):
%    \begin{macrocode}
\newcommand{\childdocmain}[1]
{
  \childdocdisable\childdocmain{}
  \if?#1?\else
    \begingroup
      \def\childdoctmp{#1}
      \ifx\childdoctmp\childdocname
        \def\childdoctmp{}
      \else
        \def\childdoctmp
        {
          \childdoctrue
          \includeonly{\childdocname}
          \def\childdocjob{#1}
          \def\jobname{#1}
        }
      \fi
      \expandafter
    \endgroup
    \childdoctmp
  \fi
}
%    \end{macrocode}

% \macro{\childdocof}
% The command |\childdocof| redirects
% compilation to the main file |#1|.
%    \begin{macrocode}
\newcommand{\childdocof}[1]
{
  \childdocdisable
  \childdoctrue
  \includeonly{\childdocname}
  \def\jobname{#1}
  \def\childdocjob{#1}
  \input{#1}
}
%    \end{macrocode}

% \macro{\childdocby}
% The command |\childdocby| ....
%    \begin{macrocode}
\newcommand{\childdocby}[2][]
{
  \childdocdisable
  \childdoctrue
  \childdocmanualtrue
  \if?#1?\else
    \def\jobname{#2}
  \fi
  \def\childdocjob{#2}
  \input{#2}
  \endinput
}
%    \end{macrocode}

% \macro{\childdocforward}
% The command |\childdocforward| redirects
% compilation to the main file or
% (if the optional argument is given) a child file.
% Parameters are set as if the main file
% or a child file starting with |\childdocof| was compiled.
% Then compilation is handed over to the main file:
%    \begin{macrocode}
\newcommand{\childdocforward}[2][]
{
  \begingroup
    \if?#1?
      \def\childdoctmp
      {
        \def\childdocname{#2}
        \def\childdocjob{#2}
        \def\jobname{#2}
        \input{#2}
        \endinput
      }
    \else
      \def\childdoctmp
      {
        \childdocdisable
        \def\childdocname{#2}
        \childdoctrue
        \includeonly{#2}
        \def\childdocjob{#1}
        \def\jobname{#1}
        \input{#1}
        \endinput
      }
    \fi
    \expandafter
  \endgroup
  \childdoctmp
}
%    \end{macrocode}

% \macro{\childdocforwardprefix}
% The command |\childdocforwardprefix| redirects
% compilation to the main or a child file by means of a pattern.
% The prefix |#1| in the current filename is replaced by |#2|
% and the suffix of the current filename is kept
% (it is assumed that the filename does not contain the substring `|~~~|'
% which is used as a delimiter).
% Compilation is handed over to the new file by |\childdocforward|:
%    \begin{macrocode}
\newcommand{\childdocforwardprefix}[3][]
{
  \begingroup
    \def\childdocextract #2##1~~~{\def\childdoctmp{\childdocforward[#1]{#3##1}}}
    \expandafter\childdocextract\childdocname~~~
    \expandafter
  \endgroup
  \childdoctmp
}
%    \end{macrocode}

% \macro{\childdoc}
% The deprecated macro |\childdoc| is a legacy version of |\childdocmain|:
%    \begin{macrocode}
\newcommand{\childdoc}{\childdocmain}
%    \end{macrocode}

% \macro{\childdocredirect}
% The deprecated macro |\childdocredirect| is a legacy version
% of |\childdocforward| and |\childdocforwardprefix|:
%    \begin{macrocode}
\newcommand{\childdocredirect}[2][]
{
  \begingroup
    \if?#1?
      \def\childdoctmp{\childdocforward{#2}}
    \else
      \def\childdoctmp{\childdocforwardprefix{#1}{#2}}
    \fi
    \expandafter
  \endgroup
  \childdoctmp
}
%    \end{macrocode}

%\iffalse
%</package>
%\fi
%
\endinput
|\\
|\childdocby{|\textit{main}|}|\\
\end{tabular}
\end{center}
%
Both forms have slightly different effects as described above.
The main file is prepared as usual, see \secref{sec:include}.

%%%%%%%%%%%%%%%%%%%%%%%%%%%%%%%%%%%%%%%%%%%%%%%%%%%%%%%%%%%%%%%%%%%%%%%%%%%%%%%%
\subsection{Legacy Detection}
\label{sec:detection}

The directive |\childdocmain| in the main file can detect
whether the complete document or merely a child is to be compiled
even without using the directive |\childdocof|.
This method is deprecated because it is less robust
and there is no compelling reason to use it;
it is merely provided for backward compatibility
and it may be removed in future versions.

If the detection mechanism is to be used,
it is mandatory to correctly specify
the filename of the main file as the argument of |\childdocmain|:
%
\begin{center}
\begin{tabular}{l}
|% \iffalse
%
% childdoc.dtx Copyright (C) 2017-2018 Niklas Beisert
%
% This work may be distributed and/or modified under the
% conditions of the LaTeX Project Public License, either version 1.3
% of this license or (at your option) any later version.
% The latest version of this license is in
%   http://www.latex-project.org/lppl.txt
% and version 1.3 or later is part of all distributions of LaTeX
% version 2005/12/01 or later.
%
% This work has the LPPL maintenance status `maintained'.
%
% The Current Maintainer of this work is Niklas Beisert.
%
% This work consists of the files childdoc.dtx and childdoc.ins
% and the derived files childdoc.def and cdocsamp.tex with
% cdocsch1.tex, cdocsch2.tex, cdocsdrf.tex, cdocsfn1.tex, cdocsfn2.tex.
%
%<package>\ifdefined\childdocmain\endinput\fi
%<package>\ProvidesFile{childdoc.def}[2018/12/30 v2.0 child document driver]
%<samplemain>\ProvidesFile{cdocsamp.tex}[2018/12/30 v2.0 sample for childdoc]
%<*driver>
%\ProvidesFile{childdoc.drv}[2018/12/30 v2.0 childdoc reference manual file]
\PassOptionsToClass{10pt,a4paper}{article}
\documentclass{ltxdoc}

\usepackage[margin=35mm]{geometry}
\usepackage{hyperref}
\usepackage{hyperxmp}
\usepackage[usenames]{color}

\hypersetup{colorlinks=true}
\hypersetup{pdfstartview=FitH}
\hypersetup{pdfpagemode=UseNone}
\hypersetup{pdfsource={}}
\hypersetup{pdflang={en-UK}}
\hypersetup{pdfcopyright={Copyright 2017-2018 Niklas Beisert.
  This work may be distributed and/or modified under the
  conditions of the LaTeX Project Public License, either version 1.3
  of this license or (at your option) any later version.}}
\hypersetup{pdflicenseurl={http://www.latex-project.org/lppl.txt}}
\hypersetup{pdfcontactaddress={ETH Zurich, ITP, HIT K,
  Wolfgang-Pauli-Strasse 27}}
\hypersetup{pdfcontactpostcode={8093}}
\hypersetup{pdfcontactcity={Zurich}}
\hypersetup{pdfcontactcountry={Switzerland}}
\hypersetup{pdfcontactemail={nbeisert@itp.phys.ethz.ch}}
\hypersetup{pdfcontacturl={http://people.phys.ethz.ch/\xmptilde nbeisert/}}

\newcommand{\secref}[1]{\hyperref[#1]{section \ref*{#1}}}

\parskip1ex
\parindent0pt
\let\olditemize\itemize
\def\itemize{\olditemize\parskip0pt}

\begin{document}

\title{The \textsf{childdoc} Package}
\hypersetup{pdftitle={The childdoc Package}}
\author{Niklas Beisert\\[2ex]
  Institut f\"ur Theoretische Physik\\
  Eidgen\"ossische Technische Hochschule Z\"urich\\
  Wolfgang-Pauli-Strasse 27, 8093 Z\"urich, Switzerland\\[1ex]
  \href{mailto:nbeisert@itp.phys.ethz.ch}
  {\texttt{nbeisert@itp.phys.ethz.ch}}}
\hypersetup{pdfauthor={Niklas Beisert}}
\hypersetup{pdfsubject={Manual for the LaTeX2e Package childdoc}}
\date{30 December 2018, \textsf{v2.0}}
\maketitle

\begin{abstract}\noindent
\textsf{childdoc} is a \LaTeXe{} package
that enables the direct compilation
of document sections included by |\include|
to individual files.
\end{abstract}

\begingroup
\parskip0ex
\tableofcontents
\endgroup

%%%%%%%%%%%%%%%%%%%%%%%%%%%%%%%%%%%%%%%%%%%%%%%%%%%%%%%%%%%%%%%%%%%%%%%%%%%%%%%%
%%%%%%%%%%%%%%%%%%%%%%%%%%%%%%%%%%%%%%%%%%%%%%%%%%%%%%%%%%%%%%%%%%%%%%%%%%%%%%%%
\section{Introduction}

\LaTeX{} provides a mechanism to structure a large document (such as a book)
into a main file and several child files (containing the chapters)
using the |\include| command.
This mechanism is beneficial for documents
which span hundreds of pages in order to
make the source file(s) more manageable.
Moreover, compilation can be restricted to
selected child files by means of the |\includeonly| command.
The latter feature can be used to reduce the compilation time while editing
(this was significantly more useful in the earlier days of \LaTeX{})
or to generate a smaller document which is easier to navigate.
Another application of |\includeonly| is to generate
documents consisting of selected parts of the complete document.

However, there are a few drawbacks of the plain |\include| mechanism:
\begin{itemize}
\item
The child files cannot be compiled on their own,
they can only be compiled via the main file.
A naive editing environment
(such as a text editor with an option
to have the current file processed by \LaTeX)
may require one to switch to the main file before compiling;
attempting to compile the child file produces errors.
\item
The main file must be modified (each time)
to adjust the |\includeonly| command
to the present needs. This easily leaves the main file in a messy state.
\item
The generated document will always carry the filename
of the main document. This is inconvenient if
several child files are to be compiled and
to be kept for distribution.
\end{itemize}

The present package provides a simple interface
to make child files individually compilable by \LaTeX{}.
Compiling a child file then has the same effect as compiling
the main file with an |\includeonly| command
to select the appropriate child.
Moreover the generated document will carry the name of the child
rather than the main file.
This resolves all three above issues.

This feature is meant to make the editing of books,
thesis documents and lecture notes somewhat more convenient.
However, the package can also be used efficiently for
composing a series of documents (such as exercise sheets)
which are typically distributed individually.
It then assists the author in generating the individual documents
(potentially in different versions)
as well as a document containing the collected series.
Another application is in developing style files
or other kinds of included material
where compilation of the style file could redirect
to a sample or test file.

%%%%%%%%%%%%%%%%%%%%%%%%%%%%%%%%%%%%%%%%%%%%%%%%%%%%%%%%%%%%%%%%%%%%%%%%%%%%%%%%
%%%%%%%%%%%%%%%%%%%%%%%%%%%%%%%%%%%%%%%%%%%%%%%%%%%%%%%%%%%%%%%%%%%%%%%%%%%%%%%%
\section{Usage}

First of all, the package \textsf{childdoc} is \emph{not} a standard
\LaTeXe{} |.sty| style file! Therefore it needs to be invoked in
a non-standard way.

%%%%%%%%%%%%%%%%%%%%%%%%%%%%%%%%%%%%%%%%%%%%%%%%%%%%%%%%%%%%%%%%%%%%%%%%%%%%%%%%
\subsection{Included Files}
\label{sec:include}

%%%%%%%%%%%%%%%%%%%%%%%%%%%%%%%%%%%%%%%%
\DescribeMacro{\childdocmain}
To use the package, add the commands
\begin{center}
\begin{tabular}{l}
|\input{childdoc.def}|\\
|\childdocmain{}|\\
\end{tabular}
\end{center}
at the very top of the main \LaTeX{} file,
in particular \emph{before} the |\documentclass| statement!
The argument of |\childdocmain| should be left empty
(but it must be present).

%%%%%%%%%%%%%%%%%%%%%%%%%%%%%%%%%%%%%%%%
\DescribeMacro{\childdocof}
Furthermore, add the commands
\begin{center}
\begin{tabular}{l}
|\input{childdoc.def}|\\
|\childdocof{|\textit{main}|}|\\
\end{tabular}
\end{center}
at the top of every child file \textit{child}
which is included by |\include{|\textit{child}|}|
from within the main file
(or at least for those files to be compiled individually).
The argument \textit{main} must be the filename of the main file.

There are a couple of
considerations in setting up the main and child documents:

%%%%%%%%%%%%%%%%%%%%%%%%%%%%%%%%%%%%%%%%
\paragraph{Restrictions.}

Please note the following restrictions:
\begin{itemize}
\item
|\childdocmain| must be called with one argument \textit{main}
to ensure compatibility with earlier version of the package.
It must either be empty (|\childdocmain{}|)
or precisely match the filename of the main file in which it is specified.
See \secref{sec:detection} for further information.
\item
The filename \textit{main} must be specified without the |.tex| extension.
\item
The filename \textit{main} is case sensitive
(even in case-insensitive file systems)
due to internal string comparison.
\item
The argument \textit{main} should be fully expanded, it cannot be a macro.
\item
Subdirectories and special characters should be avoided in filenames.
\item
The command |\childdocmain{|\textit{main}|}| must be followed by a whitespace.
It should not be followed immediately by another command
or by a comment mark `|%|'.
This is because the \TeX{} parser reads the token immediately following
the argument of |\childdocmain| and puts it
at the beginning of every child section;
however, a white\-space is ignored.
\end{itemize}

%%%%%%%%%%%%%%%%%%%%%%%%%%%%%%%%%%%%%%%%
\paragraph{Content of Main File.}

It is advisable to place all content in the child files included by |\include|.
Any output contained in the main file will appear in all child documents
unless suppressed manually;
it cannot be suppressed automatically by the |\includeonly| directive
and thus should normally be avoided.
A method to include some content in the main file
by means of conditional processing is described in \secref{sec:conditional}.

%%%%%%%%%%%%%%%%%%%%%%%%%%%%%%%%%%%%%%%%
\paragraph{Page Numbering.}

When only a part of the document is compiled,
the appropriate numbering of pages
(as well as other status parameters)
is determined from the |.aux| files.
The latter contain information from previous passes.
However this information needs to propagate through
all intermediate child documents.
Therefore the page numbering in child documents may well
be inconsistent until the complete document is compiled at least once.

A useful (if unconventional) way to always ensure a consistent
page numbering is to restart the numbering in each child document
and denote the pages by `\textit{child}|.|\textit{page}'
where \textit{child} represents the chapter/section number of the child file.
This can be achieved by the command
|\numberwithin{page}{|\textit{child}|}|
of the \textsf{amsmath} package
where \textit{child} can be |chapter| or |section|
depending on the chosen structuring.
Alternatively, one can modify the macro |\thepage| appropriately
and reset the counter |page| at the start of each child file.

%%%%%%%%%%%%%%%%%%%%%%%%%%%%%%%%%%%%%%%%%%%%%%%%%%%%%%%%%%%%%%%%%%%%%%%%%%%%%%%%
\subsection{Conditional Processing}
\label{sec:conditional}

The package provides a mechanism to compile different versions
of a document. To customise the versions further some conditional processing
can come in handy to distinguish which version is being compiled.
The package provides two macros to describe the compilation context:

%%%%%%%%%%%%%%%%%%%%%%%%%%%%%%%%%%%%%%%%
\DescribeMacro{\ifchilddoc}
The conditional |\ifchilddoc| distinguishes between the compilation of
child documents and the main document:
%
\begin{center}
|\ifchilddoc |\textit{child-code}| |[|\||else |\textit{main-code}]| \||fi|
\end{center}

%%%%%%%%%%%%%%%%%%%%%%%%%%%%%%%%%%%%%%%%
\DescribeMacro{\childdocname}
\DescribeMacro{\childdocjob}
The macro |\childdocname| contains the filename (without extension)
of the main or child file being processed.
Note that |\childdocjob| will always contain the name of the main file.

%%%%%%%%%%%%%%%%%%%%%%%%%%%%%%%%%%%%%%%%
\paragraph{Title Page.}

Conditional processing can be used to include a title or banner page
in the main document when proper precautions are taken.
Importantly, the code in the main file should ensure that the page counter
(as well as other status parameters which are stored in the |.aux| files)
takes the same value after the conditional processing.
Otherwise the page numbers may take divergent values
depending on which part is compiled.

For example, a title page could be declared by:
%
\begin{center}
\begin{tabular}{l}
|\ifchilddoc\||else|\\
|\addtocounter{page}{-1}|\\
\textit{code for title page}\\
|\newpage|\\
|\||fi|
\end{tabular}
\end{center}
%
A banner page for the child documents can be generated by:
%
\begin{center}
\begin{tabular}{l}
|\ifchilddoc|\\
|\addtocounter{page}{-1}|\\
\textit{code for banner page}\\
|\newpage|\\
|\||fi|
\end{tabular}
\end{center}
%
Here one could write a message such as:
\begin{center}
|This is the part \childdocname{} of \childdocjob{}.|
\end{center}

%%%%%%%%%%%%%%%%%%%%%%%%%%%%%%%%%%%%%%%%%%%%%%%%%%%%%%%%%%%%%%%%%%%%%%%%%%%%%%%%
\subsection{Flags}
\label{sec:flags}

The package makes it easy to generate different versions
of the main or child documents.
To this end compilation flags can be defined
and assigned different default values.
They will be particularly useful in conjunction
with the forwarding mechanism described in \secref{sec:forward}.

For example, it may be useful to have a flag |\version|
which can be set to |draft| or |final|.
The document source will contain some conditional code
depending on the value of |\version|.
Suppose further, the flag should default to |final| for the main file
and to |draft| for child files
which is a natural assignment for editing the document.
This is achieved by placing the following code
in the preamble of the main document
(below the |\childdocmain| directive):
%
\begin{center}
\begin{tabular}{l}
|\ifchilddoc|\\
|\providecommand{\version}{draft}|\\
|\||else|\\
|\providecommand{\version}{final}|\\
|\||fi|
\end{tabular}
\end{center}
%
The definition by |\providecommand| makes sure
that previous definitions are not overwritten.
Further statements |\providecommand{\version}{...}|
can thus be added before the above code to override it.

For the main file, one might add a line
(between |\childdocmain| and the above block)
%
\begin{center}
|%\ifchilddoc\||else\providecommand{\version}{draft}\||fi|
\end{center}
%
which can be uncommented to produce a draft version.
Likewise one can add a line to the very top of a child file
(above the |\childdocof{|\textit{main}|}| directive)
%
\begin{center}
|%\providecommand{\version}{final}|
\end{center}
%
which can be uncommented to produce the final version of this child document.

%%%%%%%%%%%%%%%%%%%%%%%%%%%%%%%%%%%%%%%%%%%%%%%%%%%%%%%%%%%%%%%%%%%%%%%%%%%%%%%%
\subsection{Forwarding}
\label{sec:forward}

Different versions of the main or child documents
using compilation flags as described in \secref{sec:flags}
can be (permanently) stored in different files
for convenient compilation, viewing and distribution.
To this end, the package defines a command
to pass on compilation to a different file:

%%%%%%%%%%%%%%%%%%%%%%%%%%%%%%%%%%%%%%%%
\DescribeMacro{\childdocforward}
The command |\childdocforward| redirects processing to
another source file:
%
\begin{center}
\begin{tabular}{l}
|\input{childdoc.def}|\\
|\childdocforward[|\textit{main}|]{|\textit{dest}|}|\\
\end{tabular}
\end{center}
%
The argument \textit{dest} is the destination file
(without extension).
It should be the main file or one of the child files.
Note that further \textsf{childdoc} directives
such as |\childdocof| and |\childdocforward|
in the indicated file will be processed in this form.
The optional argument \textit{main}
passes on directly to the main file \textit{main}
while pretending to compile the child \textit{dest}.
This form behaves as if \textit{dest}
issues |\childdocof{|\textit{main}|}| right away,
and no further \textsf{childdoc} directives will be processed.

%%%%%%%%%%%%%%%%%%%%%%%%%%%%%%%%%%%%%%%%
\DescribeMacro{\...prefix}
In the alternative form |\childdocforwardprefix|,
%
\begin{center}
\begin{tabular}{l}
|\input{childdoc.def}|\\
|\childdocforwardprefix[|\textit{main}|]{|\textit{prefix}|}{|\textit{dest}|}|
\end{tabular}
\end{center}
%
the destination file is determined by a pattern
depending on the current file:
To make this work, the current file must be called
`{\textit{prefix}\hspace{0.2em}\textit{suffix}}'
with \textit{prefix} matching precisely the argument.
Processing is then passed on to the file
`{\textit{dest}\hspace{0.2em}\textit{suffix}}'.
Surely, the same effect is achieved by
directly specifying the
argument `{\textit{dest}\hspace{0.2em}\textit{suffix}}'
in the first form.
However, that requires to set up a different file
for each child. With the alternative form of the command
all these files can have exactly the same content
which simplifies setting them up and maintaining them.

For example, the following file |draft.tex|
with a compilation flag |\version| as described in \secref{sec:flags}
compiles the main document as a draft:
%
\begin{center}
\begin{tabular}{l}
|\def\version{draft}|\\
|\input{childdoc.def}|\\
|\childdocforward{|\textit{main}|}|
\end{tabular}
\end{center}
%
Likewise, the following files |final|\textit{nn}|.tex|
compile the final version of the child document
|child|\textit{nn}|.tex|:
%
\begin{center}
\begin{tabular}{l}
|\def\version{final}|\\
|\input{childdoc.def}|\\
|\childdocforwardprefix{final}{child}|
\end{tabular}
\end{center}
%

Note that when several versions of a main file and/or of each child file
are to be generated, it may be convenient to set up a |Makefile| or
shell script to automatise the process.

%%%%%%%%%%%%%%%%%%%%%%%%%%%%%%%%%%%%%%%%%%%%%%%%%%%%%%%%%%%%%%%%%%%%%%%%%%%%%%%%
\subsection{Command Line Processing}
\label{sec:commandline}

The effect of redirection files can also be achieved by invoking
the \LaTeX{} compiler with a more elaborate command line.
Most conveniently this should be done as part
of a shell script or a |Makefile|.

When using \textsf{childdoc} in the main file, the following
command lines effectively perform a redirection
(note that depending on the shell being used,
backslashes may have to be doubled: `|\|' $\to$ `|\\|'):
%
\begin{center}
|... -jobname "|\textit{target}|" |\\|"|[\textit{flags}]%
|\input{childdoc.def}\childdocforward[|\textit{main}|]{|\textit{dest}|}"|
\end{center}
%
Here \textit{target} is the name of the output file,
\textit{main} is the name of the main file
and \textit{dest} is the name of the main or child file to be processed
(all filenames without extensions).
The optional argument \textit{main} can be omitted
if \textit{main} matches \textit{dest}.
Optionally, compilation \textit{flags} can be defined via |\def| commands.
This command line makes the \TeX{} engine believe
it is compiling the file \textit{target}
whose content is specified as the latter parameter.
The provided code then forwards the processing to
\textit{main} or \textit{dest} as described in \secref{sec:forward}.

%%%%%%%%%%%%%%%%%%%%%%%%%%%%%%%%%%%%%%%%%%%%%%%%%%%%%%%%%%%%%%%%%%%%%%%%%%%%%%%%
\subsection{Include by Input}
\label{sec:input}

Including child documents by |\include| has some restrictions by design.
Most notably, the content of a child document always occupies
its own set of pages; pages cannot be shared between child documents.
Usually, this behaviour makes perfect sense
because each child document contain an essential part of the document.
However, in some situations it may be desirable to compose
a document from a collection of parts
without having mandatory page breaks between then.
For this case, the package
provides a mechanism to include parts
by |\input| which can also be processed individually.
However, by construction this mechanism
requires manual handling of the content to be output.

%%%%%%%%%%%%%%%%%%%%%%%%%%%%%%%%%%%%%%%%
\DescribeMacro{\ifchilddocmanual}
The main file should be prepared as usual, see \secref{sec:include}.
However, the document body must make a distinction
between processing of an individual part and of the main document, e.g.:
%
\begin{center}
\begin{tabular}{l}
|\ifchilddocmanual|\\
|\input{\childdocname}|\\
|\||else|\\
\textit{document body with }|\input{|\textit{part}|}|\\
|\||fi|
\end{tabular}
\end{center}
%
The conditional |\ifchilddocmanual| is true whenever
a part to be included by |\input| is being compiled,
and the name of the part is stored in |\childdocname|.

%%%%%%%%%%%%%%%%%%%%%%%%%%%%%%%%%%%%%%%%
\DescribeMacro{\childdocby}
Each part to be included by |\input| should start with:
%
\begin{center}
\begin{tabular}{l}
|\input{childdoc.def}|\\
|\childdocby{|\textit{main}|}|\\
\end{tabular}
\end{center}
%
The directive |\childdocby| is similar to |\childdocof|
described in \secref{sec:include},
but the subsequent selection of content must be done manually.
To that end, both |\ifchilddoc| and |\ifchilddocmanual|
will be true upon processing of a part,
and the name of the part is stored in |\childdocname|.
Note that |\jobname| will be set to the filename of the current part
so that each part receives an individual |.aux| file
that does not interfere with the |.aux| file(s) of the main document.
This behaviour can be altered by the alternative form
|\childdocby[*]{|\textit{main}|}| (with a non-empty optional argument)
which uses the |.aux| file of the main document
by setting |\jobname| to \textit{main}.

%%%%%%%%%%%%%%%%%%%%%%%%%%%%%%%%%%%%%%%%%%%%%%%%%%%%%%%%%%%%%%%%%%%%%%%%%%%%%%%%
\subsection{Driver Development}
\label{sec:driver}

The \textsf{childdoc} mechanism can also be use for the development
of definition files such as \LaTeX{} styles or classes.
This case differs from the above setup with multiple parts
included by |\include| in that no |\includeonly| should be invoked.
This can be achieved by starting the include file
(before |\ProvidesPackage|) with:
%
\begin{center}
\begin{tabular}{l}
|\input{childdoc.def}|\\
|\childdocforward{|\textit{main}|}|\\
\end{tabular}
\end{center}
%
or alternatively with:
%
\begin{center}
\begin{tabular}{l}
|\input{childdoc.def}|\\
|\childdocby{|\textit{main}|}|\\
\end{tabular}
\end{center}
%
Both forms have slightly different effects as described above.
The main file is prepared as usual, see \secref{sec:include}.

%%%%%%%%%%%%%%%%%%%%%%%%%%%%%%%%%%%%%%%%%%%%%%%%%%%%%%%%%%%%%%%%%%%%%%%%%%%%%%%%
\subsection{Legacy Detection}
\label{sec:detection}

The directive |\childdocmain| in the main file can detect
whether the complete document or merely a child is to be compiled
even without using the directive |\childdocof|.
This method is deprecated because it is less robust
and there is no compelling reason to use it;
it is merely provided for backward compatibility
and it may be removed in future versions.

If the detection mechanism is to be used,
it is mandatory to correctly specify
the filename of the main file as the argument of |\childdocmain|:
%
\begin{center}
\begin{tabular}{l}
|\input{childdoc.def}|\\
|\childdocmain{|\textit{main}|}|\\
\end{tabular}
\end{center}
%
If |\jobname| does not match the argument \textit{main} of |\childdocmain|,
it is assumed that |\jobname| points to the child file to be compiled.
When using |\childdocmain| with the main file specified as argument,
it suffices to start a child file
with just |\input{|\textit{main}|}|
without loading of the package and using |\childdocof|.
If instead all processing is done
with the appropriate \textsf{childdoc} directives,
the argument of \textit{main} of |\childdocmain| can be empty.

An alternative version of the command line processing described
in \secref{sec:commandline} using the detection mechanism reads:
%
\begin{center}
|... -jobname "|\textit{target}|" "|[\textit{flags}]%
[|\def\jobname{|\textit{dest}|}|]|\input{|\textit{main}|}"|
\end{center}

%%%%%%%%%%%%%%%%%%%%%%%%%%%%%%%%%%%%%%%%%%%%%%%%%%%%%%%%%%%%%%%%%%%%%%%%%%%%%%%%
\subsection{Manual Code}
\label{sec:manual}

In case one cannot be certain whether the definitions file |childdoc.def|
is installed on the target \TeX{} distribution
and one prefers not to ship it,
it is conceivable to paste a few relevant commands into the sources.

To that end, drop all statements |\input{childdoc.def}|
and perform the replacements as outlined below.
Instead of |\childdocmain{|\textit{main}|}| add the following code
to the top of the main file:
%
\begin{center}
\begin{tabular}{l}
|\||ifdefined\childdocname\endinput\||fi\newif\ifchilddoc|\\
|\edef\childdocname{\scantokens\expandafter{\jobname\noexpand}}|\\
|\def\childdocmain{|\textit{main}|}\||ifx\childdocmain\childdocname\||else|\\
|\childdoctrue\includeonly{\childdocname}\let\jobname\childdocmain\||fi|\\
\end{tabular}
\end{center}
%
Instead of |\childdocof{|\textit{main}|}| just include the main file
at the top of each child file:
%
\begin{center}
|\input{|\textit{main}|}|
\end{center}
%
A simple redirection |\childdocforward{|\textit{dest}|}| is achieved by:
%
\begin{center}
|\def\jobname{|\textit{dest}|}\input{\jobname}|
\end{center}
%
The redirection with prefix
|\childdocforwardprefix[|\textit{prefix}|]{|\textit{dest}|}|
is accomplished by:
%
\begin{center}
\begin{tabular}{l}
|{\edef\jobname{\scantokens\expandafter{\jobname\noexpand}}|\\
|\def\redirectjob |\textit{prefix}|#1~~~{\gdef\jobname{|\textit{dest}|#1}}|\\
|\expandafter\redirectjob\jobname~~~}\input{\jobname}|
\end{tabular}
\end{center}

In an alternative approach,
child documents can be compiled by a specific command line
without additional code or specific definitions:
%
\begin{center}
|... -jobname "|\textit{target}|" "|[\textit{flags}]%
|\includeonly{|\textit{dest}|}\input{|\textit{main}|}"|
\end{center}
%

%%%%%%%%%%%%%%%%%%%%%%%%%%%%%%%%%%%%%%%%%%%%%%%%%%%%%%%%%%%%%%%%%%%%%%%%%%%%%%%%
%%%%%%%%%%%%%%%%%%%%%%%%%%%%%%%%%%%%%%%%%%%%%%%%%%%%%%%%%%%%%%%%%%%%%%%%%%%%%%%%
\section{Information}

%%%%%%%%%%%%%%%%%%%%%%%%%%%%%%%%%%%%%%%%%%%%%%%%%%%%%%%%%%%%%%%%%%%%%%%%%%%%%%%%
\subsection{Copyright}

Copyright \copyright{} 2017--2018 Niklas Beisert

This work may be distributed and/or modified under the
conditions of the \LaTeX{} Project Public License, either version 1.3
of this license or (at your option) any later version.
The latest version of this license is in
  \url{http://www.latex-project.org/lppl.txt}
and version 1.3 or later is part of all distributions of \LaTeX{}
version 2005/12/01 or later.

This work has the LPPL maintenance status `maintained'.

The Current Maintainer of this work is Niklas Beisert.

This work consists of the files |README.txt|, |childdoc.ins| and |childdoc.dtx|
as well as the derived files |childdoc.def|, |cdocsamp.tex|
with |cdocsch1.tex|, |cdocsch2.tex|, |cdocspt3.tex|, |cdocspt4.tex|,
|cdocsdrf.tex|, |cdocsfn1.tex|, |cdocsfn2.tex|
as well as |childdoc.pdf|.

%%%%%%%%%%%%%%%%%%%%%%%%%%%%%%%%%%%%%%%%%%%%%%%%%%%%%%%%%%%%%%%%%%%%%%%%%%%%%%%%
\subsection{Files and Installation}

The package consists of the files:
%
\begin{center}
\begin{tabular}{ll}
    |README.txt|   & readme file \\
    |childdoc.ins| & installation file \\
    |childdoc.dtx| & source file \\
    |childdoc.def| & definition file \\
    |cdocsamp.tex| & sample main file \\
    |cdocsch1.tex| & sample include file \\
    |cdocsch2.tex| & sample include file \\
    |cdocspt3.tex| & sample part file \\
    |cdocspt4.tex| & sample part file \\
    |cdocsdrf.tex| & sample redirection file \\
    |cdocsfn1.tex| & sample redirection file \\
    |cdocsfn2.tex| & sample redirection file \\
    |childdoc.pdf| & manual
\end{tabular}
\end{center}
%
The distribution consists of the files
|README.txt|, |childdoc.ins| and |childdoc.dtx|.
%
\begin{itemize}
\item
Run (pdf)\LaTeX{} on |childdoc.dtx|
to compile the manual |childdoc.pdf| (this file).
\item
Run \LaTeX{} on |childdoc.ins| to create the definitions file |childdoc.def|
and the sample |cdocsamp.tex| with include files
|cdocsch1.tex|, |cdocsch2.tex|, |cdocspt3.tex|, |cdocspt4.tex|,
|cdocsdrf.tex|, |cdocsfn1.tex|, |cdocsfn2.tex|.
Then copy the file |childdoc.def| to an appropriate directory of your \LaTeX{}
distribution, e.g.\ \textit{texmf-root}|/tex/latex/childdoc|.
\end{itemize}

%%%%%%%%%%%%%%%%%%%%%%%%%%%%%%%%%%%%%%%%%%%%%%%%%%%%%%%%%%%%%%%%%%%%%%%%%%%%%%%%
\subsection{Related CTAN Packages}

There are several other packages which offer a similar functionality:
%
\begin{itemize}
\item
The packages
\href{http://ctan.org/pkg/docmute}{\textsf{docmute}},
\href{http://ctan.org/pkg/includex}{\textsf{includex}} and
\href{http://ctan.org/pkg/standalone}{\textsf{standalone}}
provide commands to include only the document body of
a child file thus allowing both files to be compiled individually.
\item
The packages \href{http://ctan.org/pkg/subdocs}{\textsf{subdocs}}
and \href{http://ctan.org/pkg/subfiles}{\textsf{subfiles}}
provide structures in which the main and child documents can be
encapsulated and allowing them to be compiled individually.
The inclusion mechanism is different from the conventional |\include|.
\item
The package \href{http://ctan.org/pkg/combine}{\textsf{combine}}
is an elaborate solution to combine several documents into one.
\end{itemize}
%
See also the CTAN topic \href{http://ctan.org/topic/subdocs}{\textsf{subdocs}}
for further related packages.
The present package differs from the above solutions in that
a document structure constructed with the conventional |\include| mechanism
just needs two extra commands at the top of every file
such that all constituent files can be compiled individually.

%%%%%%%%%%%%%%%%%%%%%%%%%%%%%%%%%%%%%%%%%%%%%%%%%%%%%%%%%%%%%%%%%%%%%%%%%%%%%%%%
%\subsection{Feature Suggestions}
%
%The following is a list of features which may be useful for future
%versions of this package:
%%
%\begin{itemize}
%\item
%\ldots
%\end{itemize}

%%%%%%%%%%%%%%%%%%%%%%%%%%%%%%%%%%%%%%%%%%%%%%%%%%%%%%%%%%%%%%%%%%%%%%%%%%%%%%%%
\subsection{Revision History}

%%%%%%%%%%%%%%%%%%%%%%%%%%%%%%%%%%%%%%%%
\paragraph{v2.0:} 2018/12/30

\begin{itemize}
\item
immediate forward processing
\item
added |\childdocby| mechanism
\item
manual restructured
\end{itemize}

%%%%%%%%%%%%%%%%%%%%%%%%%%%%%%%%%%%%%%%%
\paragraph{v1.6:} 2018/01/17

\begin{itemize}
\item
application for development of include files
\item
corrections to manual
\end{itemize}

%%%%%%%%%%%%%%%%%%%%%%%%%%%%%%%%%%%%%%%%
\paragraph{v1.5:} 2017/05/21

\begin{itemize}
\item
more complete structuring introduced
\item
|\childdocof| introduced
\item
|\childdoc| renamed to |\childdocmain|
\item
|\childredirect| renamed to |\childdocforward| and |\childdocforwardprefix|
and functionality expanded
\end{itemize}

%%%%%%%%%%%%%%%%%%%%%%%%%%%%%%%%%%%%%%%%
\paragraph{v1.0:} 2017/04/27

\begin{itemize}
\item
manual and install package
\item
first version published on CTAN
\end{itemize}

%%%%%%%%%%%%%%%%%%%%%%%%%%%%%%%%%%%%%%%%
\paragraph{v0.6:} 2017/04/26

\begin{itemize}
\item
redirection mechanism added
\end{itemize}

%%%%%%%%%%%%%%%%%%%%%%%%%%%%%%%%%%%%%%%%
\paragraph{v0.5:} 2017/04/26

\begin{itemize}
\item
functionality in definition file
\end{itemize}


%%%%%%%%%%%%%%%%%%%%%%%%%%%%%%%%%%%%%%%%%%%%%%%%%%%%%%%%%%%%%%%%%%%%%%%%%%%%%%%%
%%%%%%%%%%%%%%%%%%%%%%%%%%%%%%%%%%%%%%%%%%%%%%%%%%%%%%%%%%%%%%%%%%%%%%%%%%%%%%%%
%%%%%%%%%%%%%%%%%%%%%%%%%%%%%%%%%%%%%%%%%%%%%%%%%%%%%%%%%%%%%%%%%%%%%%%%%%%%%%%%
\appendix

\settowidth\MacroIndent{\rmfamily\scriptsize 000\ }

 \DocInput{childdoc.dtx}

\end{document}
%</driver>
% \fi
%
% %%%%%%%%%%%%%%%%%%%%%%%%%%%%%%%%%%%%%%%%%%%%%%%%%%%%%%%%%%%%%%%%%%%%%%%%%%%%%%
% %%%%%%%%%%%%%%%%%%%%%%%%%%%%%%%%%%%%%%%%%%%%%%%%%%%%%%%%%%%%%%%%%%%%%%%%%%%%%%
% \section{Sample}
%\iffalse
%<*samplemain>
%\fi
%
% The following presents a sample document
% with two chapters, two parts, a title page,
% a compile flag as well as three forwarding files to set the flag.
% It consists of eight |.tex| files:
% \begin{center}
% \begin{tabular}{ll}
% |cdocsamp.tex|&main file\\
% |cdocsch1.tex|&include file for chapter 1\\
% |cdocsch2.tex|&include file for chapter 2\\
% |cdocspt3.tex|&include file for part 3\\
% |cdocspt4.tex|&include file for part 4\\
% |cdocsdrf.tex|&forwarding file for main file in draft mode\\
% |cdocsfi1.tex|&forwarding file for final version of chapter 1\\
% |cdocsfi2.tex|&forwarding file for final version of chapter 2\\
% \end{tabular}
% \end{center}
% Each of the eight files can be compiled directly by the \LaTeX{} compiler.
%
% %%%%%%%%%%%%%%%%%%%%%%%%%%%%%%%%%%%%%%
% \paragraph{Main File.}
%
% The main file is called |cdocsamp.tex|.
%
% Load the \textsf{childdoc} definitions and
% declare the filename for the main document:
%    \begin{macrocode}
\input{childdoc.def}
\childdocmain{}
%    \end{macrocode}

% Optional override for |\version| flag:
%    \begin{macrocode}
%%\ifchilddoc\else\providecommand{\version}{draft}\fi
%    \end{macrocode}

% Define the default values for the |\version| flag
% (|final| for the main file and |draft| for childs):
%    \begin{macrocode}
\ifchilddoc
\providecommand{\version}{draft}
\else
\providecommand{\version}{final}
\fi
%    \end{macrocode}

% Load the standard document class:
%    \begin{macrocode}
\documentclass[12pt]{article}
%    \end{macrocode}

% Start the document body:
%    \begin{macrocode}
\begin{document}
%    \end{macrocode}

% Declare a title page.
% Print title, part of document being processed and version flag:
%    \begin{macrocode}
\addtocounter{page}{-1}
\begin{center}
{\LARGE\bfseries{}childdoc example\par}
\vspace{1cm}
\ifchilddoc
\ifchilddocmanual part\else chapter\fi:
`\childdocname' of `\childdocjob'\par
\else
main document: `\childdocjob'\par
\fi
version: \version\par
\end{center}
\newpage
%    \end{macrocode}

% Manually include selected file,
% otherwise process as usual:
%    \begin{macrocode}
\ifchilddocmanual
\section*{part `\childdocname'}
\input{\childdocname}
\else
%    \end{macrocode}

% Include the two chapters:
%    \begin{macrocode}
\include{cdocsch1}
\include{cdocsch2}
%    \end{macrocode}

% Include the two parts unless only chapters should be displayed:
%    \begin{macrocode}
\ifchilddoc\else
\section{part three}
\input{cdocspt3}
\section{part four}
\input{cdocspt4}
\fi
%    \end{macrocode}

% Process as usual until here:
%    \begin{macrocode}
\fi
%    \end{macrocode}

% End of document body:
%    \begin{macrocode}
\end{document}
%    \end{macrocode}
%\iffalse
%</samplemain>
%\fi
%
% %%%%%%%%%%%%%%%%%%%%%%%%%%%%%%%%%%%%%%
% \paragraph{Chapter Include Files.}
%
% The include files are called |cdocsch1.tex| and |cdocsch2.tex|.
%
%\iffalse
%<*samplechap1|samplechap2>
%\fi

% Optional override for |\version| flag:
%    \begin{macrocode}
%%\providecommand{\version}{final}
%    \end{macrocode}

% Include the main document:
%    \begin{macrocode}
\input{childdoc.def}
\childdocof{cdocsamp}
%    \end{macrocode}

%\iffalse
%</samplechap1|samplechap2>
%\fi
%
%\iffalse
%<*samplechap1>
%\fi
% Some text for chapter 1:
%    \begin{macrocode}
\section{one}
some text in chapter one
%    \end{macrocode}

%\iffalse
%</samplechap1>
%\fi
% Some text for chapter 2:
%\iffalse
%<*samplechap2>
%\fi
%    \begin{macrocode}
\section{two}
more text in chapter two
%    \end{macrocode}

%\iffalse
%</samplechap2>
%\fi
%
% %%%%%%%%%%%%%%%%%%%%%%%%%%%%%%%%%%%%%%
% \paragraph{Part Include Files.}
%
% The include files are called |cdocspt3.tex| and |cdocspt4.tex|.
%
%\iffalse
%<*samplepart3|samplepart4>
%\fi

% Optional override for |\version| flag:
%    \begin{macrocode}
%%\providecommand{\version}{final}
%    \end{macrocode}

% Include the main document:
%    \begin{macrocode}
\input{childdoc.def}
\childdocby{cdocsamp}
%    \end{macrocode}

%\iffalse
%</samplepart3|samplepart4>
%\fi
%
%\iffalse
%<*samplepart3>
%\fi
% Some text for part 3:
%    \begin{macrocode}
some text in part three
%    \end{macrocode}

%\iffalse
%</samplepart3>
%\fi
% Some text for part 4:
%\iffalse
%<*samplepart4>
%\fi
%    \begin{macrocode}
more text in part four
%    \end{macrocode}

%\iffalse
%</samplepart4>
%\fi
%
% %%%%%%%%%%%%%%%%%%%%%%%%%%%%%%%%%%%%%%
% \paragraph{Forwarding for a Complete Draft.}
%
% The following forwarding file |cdocsdrf.tex|
% compiles the main document in draft mode:
%\iffalse
%<*sampledraft>
%\fi
%    \begin{macrocode}
\def\version{draft}
\input{childdoc.def}
\childdocforward{cdocsamp}
%    \end{macrocode}

%\iffalse
%</sampledraft>
%\fi
%
% %%%%%%%%%%%%%%%%%%%%%%%%%%%%%%%%%%%%%%
% \paragraph{Forwarding for Final Version of the Chapters.}
%
% The following forwarding files |cdocsfn1.tex| and |cdocsfn2.tex|
% (with identical content)
% compile the final versions of the child documents
% |cdocsch1.tex| and |cdocsch2.tex|, respectively:
%\iffalse
%<*samplefinal>
%\fi
%    \begin{macrocode}
\def\version{final}
\input{childdoc.def}
\childdocforwardprefix[cdocsamp]{cdocsfn}{cdocsch}
%    \end{macrocode}

%\iffalse
%</samplefinal>
%\fi
%
% %%%%%%%%%%%%%%%%%%%%%%%%%%%%%%%%%%%%%%
% \paragraph{Command Line Processing.}
%
% The following three command lines generate the output files
% |cdocscld|, |cdocscl1| and |cdocscl2|
% which should be identical to
% |cdocsdrf|, |cdocsch1| and |cdocsfn2|, respectively:
% \begin{center}
% \begin{tabular}{l}
% |latex -jobname cdocscld \|\\
% |  "\def\version{draft}\input{childdoc.def}\childdocforward{cdocsamp}"|\\
% |latex -jobname cdocscl1 \|\\
% |  "\input{childdoc.def}\childdocforward[cdocsamp]{cdocsch1}"|\\
% |latex -jobname cdocscl2 \|\\
% |  "\def\version{final}\input{childdoc.def}\childdocforward{cdocsch2}"|
% \end{tabular}
% \end{center}
% Note that the trailing backslash on each first line
% merely continues the input to the second line
% (for convenient cut ant paste).
% Furthermore, the command |latex| can be replaced by any
% of its alternative versions such as |pdflatex|.
%
% %%%%%%%%%%%%%%%%%%%%%%%%%%%%%%%%%%%%%%%%%%%%%%%%%%%%%%%%%%%%%%%%%%%%%%%%%%%%%%
% %%%%%%%%%%%%%%%%%%%%%%%%%%%%%%%%%%%%%%%%%%%%%%%%%%%%%%%%%%%%%%%%%%%%%%%%%%%%%%
% \section{Implementation}
%\iffalse
%<*package>
%\fi
%
% This section describes the definitions file |childdoc.def|.

% The definitions cannot be loaded using |\usepackage| or |\RequirePackage|
% which has a mechanism to prevent loading a style file more than once.
% When loading the definitions by means of |\input|
% multiple instances have to be prevented manually:
%\iffalse
%This code needs to be before the `\ProvidesFile' directive
%which is defined at the beginning of this file.
%Therefore it is also placed there and commented out here.
%</package>
%<*discard>
%\fi
%    \begin{macrocode}
\ifdefined\childdocmain\endinput\fi
%    \end{macrocode}
%\iffalse
%</discard>
%<*package>
%\fi
%
% \macro{\ifchilddoc}
% \macro{\ifchilddocmanual}
% The conditional |\ifchilddoc| tells whether a
% child (true) or main (false) document is being compiled.
% The conditional |\ifchilddocmanual| tells whether
% the |\includeonly| mechanism is used (false) or
% the selection of child files must be performed manually (true).
% The definitions initialise to false:
%    \begin{macrocode}
\newif\ifchilddoc
\newif\ifchilddocmanual
%    \end{macrocode}

% \macro{\childdocname}
% \macro{\childdocjob}
% The macro |\childdocname| stores the name of the main document
% to be compiled. The macro |\childdocjob| stores the name of
% the document on which the \LaTeX{} compiler was originally invoked.
% The content of |\jobname| cannot be compared
% to filenames specified in the source due to different catcodes.
% The following code rescans |\jobname|, stores the result
% in |\childdocname| and saves a copy in |\childdocjob|:
%    \begin{macrocode}
\edef\childdocname{\scantokens\expandafter{\jobname\noexpand}}
\let\childdocjob\childdocname
%    \end{macrocode}

% \macro{\childdocdisable}
% The macro |\childdocdisable| prevents the main file
% from being processed more than once.
% At this stage, the main document command |\childdocmain|
% is assumed to be called once again where it should do nothing.
% Any subsequent call to it should prevent
% a secondary processing of the main document
% It overwrites the forwarding commands
% |\childdocof| and |\childdocforward|
% with empty macros to prevent further inclusions of the main document:
%    \begin{macrocode}
\newcommand{\childdocdisable}
{
  \renewcommand{\childdocmain}[1]{\renewcommand{\childdocmain}[1]{\endinput}}
  \renewcommand{\childdocof}[1]{}
  \renewcommand{\childdocby}[2][]{}
  \renewcommand{\childdocforward}[2][]{}
  \renewcommand{\childdocdisable}{}
}
%    \end{macrocode}

% \macro{\childdocmain}
% The macro |\childdocmain| is to be called at the top of the main file
% with nothing or the main filename (without extension) as argument.
% First, it breaks loops.
% If the argument is not empty and does not match |\childdocname|
% (which is set by the first inclusion of |childdoc.def|),
% |\ifchilddoc| is set to true, |\includeonly| is applied to the child file
% and |\jobname| is set to the main file
% (for proper handling of |.aux| files):
%    \begin{macrocode}
\newcommand{\childdocmain}[1]
{
  \childdocdisable\childdocmain{}
  \if?#1?\else
    \begingroup
      \def\childdoctmp{#1}
      \ifx\childdoctmp\childdocname
        \def\childdoctmp{}
      \else
        \def\childdoctmp
        {
          \childdoctrue
          \includeonly{\childdocname}
          \def\childdocjob{#1}
          \def\jobname{#1}
        }
      \fi
      \expandafter
    \endgroup
    \childdoctmp
  \fi
}
%    \end{macrocode}

% \macro{\childdocof}
% The command |\childdocof| redirects
% compilation to the main file |#1|.
%    \begin{macrocode}
\newcommand{\childdocof}[1]
{
  \childdocdisable
  \childdoctrue
  \includeonly{\childdocname}
  \def\jobname{#1}
  \def\childdocjob{#1}
  \input{#1}
}
%    \end{macrocode}

% \macro{\childdocby}
% The command |\childdocby| ....
%    \begin{macrocode}
\newcommand{\childdocby}[2][]
{
  \childdocdisable
  \childdoctrue
  \childdocmanualtrue
  \if?#1?\else
    \def\jobname{#2}
  \fi
  \def\childdocjob{#2}
  \input{#2}
  \endinput
}
%    \end{macrocode}

% \macro{\childdocforward}
% The command |\childdocforward| redirects
% compilation to the main file or
% (if the optional argument is given) a child file.
% Parameters are set as if the main file
% or a child file starting with |\childdocof| was compiled.
% Then compilation is handed over to the main file:
%    \begin{macrocode}
\newcommand{\childdocforward}[2][]
{
  \begingroup
    \if?#1?
      \def\childdoctmp
      {
        \def\childdocname{#2}
        \def\childdocjob{#2}
        \def\jobname{#2}
        \input{#2}
        \endinput
      }
    \else
      \def\childdoctmp
      {
        \childdocdisable
        \def\childdocname{#2}
        \childdoctrue
        \includeonly{#2}
        \def\childdocjob{#1}
        \def\jobname{#1}
        \input{#1}
        \endinput
      }
    \fi
    \expandafter
  \endgroup
  \childdoctmp
}
%    \end{macrocode}

% \macro{\childdocforwardprefix}
% The command |\childdocforwardprefix| redirects
% compilation to the main or a child file by means of a pattern.
% The prefix |#1| in the current filename is replaced by |#2|
% and the suffix of the current filename is kept
% (it is assumed that the filename does not contain the substring `|~~~|'
% which is used as a delimiter).
% Compilation is handed over to the new file by |\childdocforward|:
%    \begin{macrocode}
\newcommand{\childdocforwardprefix}[3][]
{
  \begingroup
    \def\childdocextract #2##1~~~{\def\childdoctmp{\childdocforward[#1]{#3##1}}}
    \expandafter\childdocextract\childdocname~~~
    \expandafter
  \endgroup
  \childdoctmp
}
%    \end{macrocode}

% \macro{\childdoc}
% The deprecated macro |\childdoc| is a legacy version of |\childdocmain|:
%    \begin{macrocode}
\newcommand{\childdoc}{\childdocmain}
%    \end{macrocode}

% \macro{\childdocredirect}
% The deprecated macro |\childdocredirect| is a legacy version
% of |\childdocforward| and |\childdocforwardprefix|:
%    \begin{macrocode}
\newcommand{\childdocredirect}[2][]
{
  \begingroup
    \if?#1?
      \def\childdoctmp{\childdocforward{#2}}
    \else
      \def\childdoctmp{\childdocforwardprefix{#1}{#2}}
    \fi
    \expandafter
  \endgroup
  \childdoctmp
}
%    \end{macrocode}

%\iffalse
%</package>
%\fi
%
\endinput
|\\
|\childdocmain{|\textit{main}|}|\\
\end{tabular}
\end{center}
%
If |\jobname| does not match the argument \textit{main} of |\childdocmain|,
it is assumed that |\jobname| points to the child file to be compiled.
When using |\childdocmain| with the main file specified as argument,
it suffices to start a child file
with just |\input{|\textit{main}|}|
without loading of the package and using |\childdocof|.
If instead all processing is done
with the appropriate \textsf{childdoc} directives,
the argument of \textit{main} of |\childdocmain| can be empty.

An alternative version of the command line processing described
in \secref{sec:commandline} using the detection mechanism reads:
%
\begin{center}
|... -jobname "|\textit{target}|" "|[\textit{flags}]%
[|\def\jobname{|\textit{dest}|}|]|\input{|\textit{main}|}"|
\end{center}

%%%%%%%%%%%%%%%%%%%%%%%%%%%%%%%%%%%%%%%%%%%%%%%%%%%%%%%%%%%%%%%%%%%%%%%%%%%%%%%%
\subsection{Manual Code}
\label{sec:manual}

In case one cannot be certain whether the definitions file |childdoc.def|
is installed on the target \TeX{} distribution
and one prefers not to ship it,
it is conceivable to paste a few relevant commands into the sources.

To that end, drop all statements |% \iffalse
%
% childdoc.dtx Copyright (C) 2017-2018 Niklas Beisert
%
% This work may be distributed and/or modified under the
% conditions of the LaTeX Project Public License, either version 1.3
% of this license or (at your option) any later version.
% The latest version of this license is in
%   http://www.latex-project.org/lppl.txt
% and version 1.3 or later is part of all distributions of LaTeX
% version 2005/12/01 or later.
%
% This work has the LPPL maintenance status `maintained'.
%
% The Current Maintainer of this work is Niklas Beisert.
%
% This work consists of the files childdoc.dtx and childdoc.ins
% and the derived files childdoc.def and cdocsamp.tex with
% cdocsch1.tex, cdocsch2.tex, cdocsdrf.tex, cdocsfn1.tex, cdocsfn2.tex.
%
%<package>\ifdefined\childdocmain\endinput\fi
%<package>\ProvidesFile{childdoc.def}[2018/12/30 v2.0 child document driver]
%<samplemain>\ProvidesFile{cdocsamp.tex}[2018/12/30 v2.0 sample for childdoc]
%<*driver>
%\ProvidesFile{childdoc.drv}[2018/12/30 v2.0 childdoc reference manual file]
\PassOptionsToClass{10pt,a4paper}{article}
\documentclass{ltxdoc}

\usepackage[margin=35mm]{geometry}
\usepackage{hyperref}
\usepackage{hyperxmp}
\usepackage[usenames]{color}

\hypersetup{colorlinks=true}
\hypersetup{pdfstartview=FitH}
\hypersetup{pdfpagemode=UseNone}
\hypersetup{pdfsource={}}
\hypersetup{pdflang={en-UK}}
\hypersetup{pdfcopyright={Copyright 2017-2018 Niklas Beisert.
  This work may be distributed and/or modified under the
  conditions of the LaTeX Project Public License, either version 1.3
  of this license or (at your option) any later version.}}
\hypersetup{pdflicenseurl={http://www.latex-project.org/lppl.txt}}
\hypersetup{pdfcontactaddress={ETH Zurich, ITP, HIT K,
  Wolfgang-Pauli-Strasse 27}}
\hypersetup{pdfcontactpostcode={8093}}
\hypersetup{pdfcontactcity={Zurich}}
\hypersetup{pdfcontactcountry={Switzerland}}
\hypersetup{pdfcontactemail={nbeisert@itp.phys.ethz.ch}}
\hypersetup{pdfcontacturl={http://people.phys.ethz.ch/\xmptilde nbeisert/}}

\newcommand{\secref}[1]{\hyperref[#1]{section \ref*{#1}}}

\parskip1ex
\parindent0pt
\let\olditemize\itemize
\def\itemize{\olditemize\parskip0pt}

\begin{document}

\title{The \textsf{childdoc} Package}
\hypersetup{pdftitle={The childdoc Package}}
\author{Niklas Beisert\\[2ex]
  Institut f\"ur Theoretische Physik\\
  Eidgen\"ossische Technische Hochschule Z\"urich\\
  Wolfgang-Pauli-Strasse 27, 8093 Z\"urich, Switzerland\\[1ex]
  \href{mailto:nbeisert@itp.phys.ethz.ch}
  {\texttt{nbeisert@itp.phys.ethz.ch}}}
\hypersetup{pdfauthor={Niklas Beisert}}
\hypersetup{pdfsubject={Manual for the LaTeX2e Package childdoc}}
\date{30 December 2018, \textsf{v2.0}}
\maketitle

\begin{abstract}\noindent
\textsf{childdoc} is a \LaTeXe{} package
that enables the direct compilation
of document sections included by |\include|
to individual files.
\end{abstract}

\begingroup
\parskip0ex
\tableofcontents
\endgroup

%%%%%%%%%%%%%%%%%%%%%%%%%%%%%%%%%%%%%%%%%%%%%%%%%%%%%%%%%%%%%%%%%%%%%%%%%%%%%%%%
%%%%%%%%%%%%%%%%%%%%%%%%%%%%%%%%%%%%%%%%%%%%%%%%%%%%%%%%%%%%%%%%%%%%%%%%%%%%%%%%
\section{Introduction}

\LaTeX{} provides a mechanism to structure a large document (such as a book)
into a main file and several child files (containing the chapters)
using the |\include| command.
This mechanism is beneficial for documents
which span hundreds of pages in order to
make the source file(s) more manageable.
Moreover, compilation can be restricted to
selected child files by means of the |\includeonly| command.
The latter feature can be used to reduce the compilation time while editing
(this was significantly more useful in the earlier days of \LaTeX{})
or to generate a smaller document which is easier to navigate.
Another application of |\includeonly| is to generate
documents consisting of selected parts of the complete document.

However, there are a few drawbacks of the plain |\include| mechanism:
\begin{itemize}
\item
The child files cannot be compiled on their own,
they can only be compiled via the main file.
A naive editing environment
(such as a text editor with an option
to have the current file processed by \LaTeX)
may require one to switch to the main file before compiling;
attempting to compile the child file produces errors.
\item
The main file must be modified (each time)
to adjust the |\includeonly| command
to the present needs. This easily leaves the main file in a messy state.
\item
The generated document will always carry the filename
of the main document. This is inconvenient if
several child files are to be compiled and
to be kept for distribution.
\end{itemize}

The present package provides a simple interface
to make child files individually compilable by \LaTeX{}.
Compiling a child file then has the same effect as compiling
the main file with an |\includeonly| command
to select the appropriate child.
Moreover the generated document will carry the name of the child
rather than the main file.
This resolves all three above issues.

This feature is meant to make the editing of books,
thesis documents and lecture notes somewhat more convenient.
However, the package can also be used efficiently for
composing a series of documents (such as exercise sheets)
which are typically distributed individually.
It then assists the author in generating the individual documents
(potentially in different versions)
as well as a document containing the collected series.
Another application is in developing style files
or other kinds of included material
where compilation of the style file could redirect
to a sample or test file.

%%%%%%%%%%%%%%%%%%%%%%%%%%%%%%%%%%%%%%%%%%%%%%%%%%%%%%%%%%%%%%%%%%%%%%%%%%%%%%%%
%%%%%%%%%%%%%%%%%%%%%%%%%%%%%%%%%%%%%%%%%%%%%%%%%%%%%%%%%%%%%%%%%%%%%%%%%%%%%%%%
\section{Usage}

First of all, the package \textsf{childdoc} is \emph{not} a standard
\LaTeXe{} |.sty| style file! Therefore it needs to be invoked in
a non-standard way.

%%%%%%%%%%%%%%%%%%%%%%%%%%%%%%%%%%%%%%%%%%%%%%%%%%%%%%%%%%%%%%%%%%%%%%%%%%%%%%%%
\subsection{Included Files}
\label{sec:include}

%%%%%%%%%%%%%%%%%%%%%%%%%%%%%%%%%%%%%%%%
\DescribeMacro{\childdocmain}
To use the package, add the commands
\begin{center}
\begin{tabular}{l}
|\input{childdoc.def}|\\
|\childdocmain{}|\\
\end{tabular}
\end{center}
at the very top of the main \LaTeX{} file,
in particular \emph{before} the |\documentclass| statement!
The argument of |\childdocmain| should be left empty
(but it must be present).

%%%%%%%%%%%%%%%%%%%%%%%%%%%%%%%%%%%%%%%%
\DescribeMacro{\childdocof}
Furthermore, add the commands
\begin{center}
\begin{tabular}{l}
|\input{childdoc.def}|\\
|\childdocof{|\textit{main}|}|\\
\end{tabular}
\end{center}
at the top of every child file \textit{child}
which is included by |\include{|\textit{child}|}|
from within the main file
(or at least for those files to be compiled individually).
The argument \textit{main} must be the filename of the main file.

There are a couple of
considerations in setting up the main and child documents:

%%%%%%%%%%%%%%%%%%%%%%%%%%%%%%%%%%%%%%%%
\paragraph{Restrictions.}

Please note the following restrictions:
\begin{itemize}
\item
|\childdocmain| must be called with one argument \textit{main}
to ensure compatibility with earlier version of the package.
It must either be empty (|\childdocmain{}|)
or precisely match the filename of the main file in which it is specified.
See \secref{sec:detection} for further information.
\item
The filename \textit{main} must be specified without the |.tex| extension.
\item
The filename \textit{main} is case sensitive
(even in case-insensitive file systems)
due to internal string comparison.
\item
The argument \textit{main} should be fully expanded, it cannot be a macro.
\item
Subdirectories and special characters should be avoided in filenames.
\item
The command |\childdocmain{|\textit{main}|}| must be followed by a whitespace.
It should not be followed immediately by another command
or by a comment mark `|%|'.
This is because the \TeX{} parser reads the token immediately following
the argument of |\childdocmain| and puts it
at the beginning of every child section;
however, a white\-space is ignored.
\end{itemize}

%%%%%%%%%%%%%%%%%%%%%%%%%%%%%%%%%%%%%%%%
\paragraph{Content of Main File.}

It is advisable to place all content in the child files included by |\include|.
Any output contained in the main file will appear in all child documents
unless suppressed manually;
it cannot be suppressed automatically by the |\includeonly| directive
and thus should normally be avoided.
A method to include some content in the main file
by means of conditional processing is described in \secref{sec:conditional}.

%%%%%%%%%%%%%%%%%%%%%%%%%%%%%%%%%%%%%%%%
\paragraph{Page Numbering.}

When only a part of the document is compiled,
the appropriate numbering of pages
(as well as other status parameters)
is determined from the |.aux| files.
The latter contain information from previous passes.
However this information needs to propagate through
all intermediate child documents.
Therefore the page numbering in child documents may well
be inconsistent until the complete document is compiled at least once.

A useful (if unconventional) way to always ensure a consistent
page numbering is to restart the numbering in each child document
and denote the pages by `\textit{child}|.|\textit{page}'
where \textit{child} represents the chapter/section number of the child file.
This can be achieved by the command
|\numberwithin{page}{|\textit{child}|}|
of the \textsf{amsmath} package
where \textit{child} can be |chapter| or |section|
depending on the chosen structuring.
Alternatively, one can modify the macro |\thepage| appropriately
and reset the counter |page| at the start of each child file.

%%%%%%%%%%%%%%%%%%%%%%%%%%%%%%%%%%%%%%%%%%%%%%%%%%%%%%%%%%%%%%%%%%%%%%%%%%%%%%%%
\subsection{Conditional Processing}
\label{sec:conditional}

The package provides a mechanism to compile different versions
of a document. To customise the versions further some conditional processing
can come in handy to distinguish which version is being compiled.
The package provides two macros to describe the compilation context:

%%%%%%%%%%%%%%%%%%%%%%%%%%%%%%%%%%%%%%%%
\DescribeMacro{\ifchilddoc}
The conditional |\ifchilddoc| distinguishes between the compilation of
child documents and the main document:
%
\begin{center}
|\ifchilddoc |\textit{child-code}| |[|\||else |\textit{main-code}]| \||fi|
\end{center}

%%%%%%%%%%%%%%%%%%%%%%%%%%%%%%%%%%%%%%%%
\DescribeMacro{\childdocname}
\DescribeMacro{\childdocjob}
The macro |\childdocname| contains the filename (without extension)
of the main or child file being processed.
Note that |\childdocjob| will always contain the name of the main file.

%%%%%%%%%%%%%%%%%%%%%%%%%%%%%%%%%%%%%%%%
\paragraph{Title Page.}

Conditional processing can be used to include a title or banner page
in the main document when proper precautions are taken.
Importantly, the code in the main file should ensure that the page counter
(as well as other status parameters which are stored in the |.aux| files)
takes the same value after the conditional processing.
Otherwise the page numbers may take divergent values
depending on which part is compiled.

For example, a title page could be declared by:
%
\begin{center}
\begin{tabular}{l}
|\ifchilddoc\||else|\\
|\addtocounter{page}{-1}|\\
\textit{code for title page}\\
|\newpage|\\
|\||fi|
\end{tabular}
\end{center}
%
A banner page for the child documents can be generated by:
%
\begin{center}
\begin{tabular}{l}
|\ifchilddoc|\\
|\addtocounter{page}{-1}|\\
\textit{code for banner page}\\
|\newpage|\\
|\||fi|
\end{tabular}
\end{center}
%
Here one could write a message such as:
\begin{center}
|This is the part \childdocname{} of \childdocjob{}.|
\end{center}

%%%%%%%%%%%%%%%%%%%%%%%%%%%%%%%%%%%%%%%%%%%%%%%%%%%%%%%%%%%%%%%%%%%%%%%%%%%%%%%%
\subsection{Flags}
\label{sec:flags}

The package makes it easy to generate different versions
of the main or child documents.
To this end compilation flags can be defined
and assigned different default values.
They will be particularly useful in conjunction
with the forwarding mechanism described in \secref{sec:forward}.

For example, it may be useful to have a flag |\version|
which can be set to |draft| or |final|.
The document source will contain some conditional code
depending on the value of |\version|.
Suppose further, the flag should default to |final| for the main file
and to |draft| for child files
which is a natural assignment for editing the document.
This is achieved by placing the following code
in the preamble of the main document
(below the |\childdocmain| directive):
%
\begin{center}
\begin{tabular}{l}
|\ifchilddoc|\\
|\providecommand{\version}{draft}|\\
|\||else|\\
|\providecommand{\version}{final}|\\
|\||fi|
\end{tabular}
\end{center}
%
The definition by |\providecommand| makes sure
that previous definitions are not overwritten.
Further statements |\providecommand{\version}{...}|
can thus be added before the above code to override it.

For the main file, one might add a line
(between |\childdocmain| and the above block)
%
\begin{center}
|%\ifchilddoc\||else\providecommand{\version}{draft}\||fi|
\end{center}
%
which can be uncommented to produce a draft version.
Likewise one can add a line to the very top of a child file
(above the |\childdocof{|\textit{main}|}| directive)
%
\begin{center}
|%\providecommand{\version}{final}|
\end{center}
%
which can be uncommented to produce the final version of this child document.

%%%%%%%%%%%%%%%%%%%%%%%%%%%%%%%%%%%%%%%%%%%%%%%%%%%%%%%%%%%%%%%%%%%%%%%%%%%%%%%%
\subsection{Forwarding}
\label{sec:forward}

Different versions of the main or child documents
using compilation flags as described in \secref{sec:flags}
can be (permanently) stored in different files
for convenient compilation, viewing and distribution.
To this end, the package defines a command
to pass on compilation to a different file:

%%%%%%%%%%%%%%%%%%%%%%%%%%%%%%%%%%%%%%%%
\DescribeMacro{\childdocforward}
The command |\childdocforward| redirects processing to
another source file:
%
\begin{center}
\begin{tabular}{l}
|\input{childdoc.def}|\\
|\childdocforward[|\textit{main}|]{|\textit{dest}|}|\\
\end{tabular}
\end{center}
%
The argument \textit{dest} is the destination file
(without extension).
It should be the main file or one of the child files.
Note that further \textsf{childdoc} directives
such as |\childdocof| and |\childdocforward|
in the indicated file will be processed in this form.
The optional argument \textit{main}
passes on directly to the main file \textit{main}
while pretending to compile the child \textit{dest}.
This form behaves as if \textit{dest}
issues |\childdocof{|\textit{main}|}| right away,
and no further \textsf{childdoc} directives will be processed.

%%%%%%%%%%%%%%%%%%%%%%%%%%%%%%%%%%%%%%%%
\DescribeMacro{\...prefix}
In the alternative form |\childdocforwardprefix|,
%
\begin{center}
\begin{tabular}{l}
|\input{childdoc.def}|\\
|\childdocforwardprefix[|\textit{main}|]{|\textit{prefix}|}{|\textit{dest}|}|
\end{tabular}
\end{center}
%
the destination file is determined by a pattern
depending on the current file:
To make this work, the current file must be called
`{\textit{prefix}\hspace{0.2em}\textit{suffix}}'
with \textit{prefix} matching precisely the argument.
Processing is then passed on to the file
`{\textit{dest}\hspace{0.2em}\textit{suffix}}'.
Surely, the same effect is achieved by
directly specifying the
argument `{\textit{dest}\hspace{0.2em}\textit{suffix}}'
in the first form.
However, that requires to set up a different file
for each child. With the alternative form of the command
all these files can have exactly the same content
which simplifies setting them up and maintaining them.

For example, the following file |draft.tex|
with a compilation flag |\version| as described in \secref{sec:flags}
compiles the main document as a draft:
%
\begin{center}
\begin{tabular}{l}
|\def\version{draft}|\\
|\input{childdoc.def}|\\
|\childdocforward{|\textit{main}|}|
\end{tabular}
\end{center}
%
Likewise, the following files |final|\textit{nn}|.tex|
compile the final version of the child document
|child|\textit{nn}|.tex|:
%
\begin{center}
\begin{tabular}{l}
|\def\version{final}|\\
|\input{childdoc.def}|\\
|\childdocforwardprefix{final}{child}|
\end{tabular}
\end{center}
%

Note that when several versions of a main file and/or of each child file
are to be generated, it may be convenient to set up a |Makefile| or
shell script to automatise the process.

%%%%%%%%%%%%%%%%%%%%%%%%%%%%%%%%%%%%%%%%%%%%%%%%%%%%%%%%%%%%%%%%%%%%%%%%%%%%%%%%
\subsection{Command Line Processing}
\label{sec:commandline}

The effect of redirection files can also be achieved by invoking
the \LaTeX{} compiler with a more elaborate command line.
Most conveniently this should be done as part
of a shell script or a |Makefile|.

When using \textsf{childdoc} in the main file, the following
command lines effectively perform a redirection
(note that depending on the shell being used,
backslashes may have to be doubled: `|\|' $\to$ `|\\|'):
%
\begin{center}
|... -jobname "|\textit{target}|" |\\|"|[\textit{flags}]%
|\input{childdoc.def}\childdocforward[|\textit{main}|]{|\textit{dest}|}"|
\end{center}
%
Here \textit{target} is the name of the output file,
\textit{main} is the name of the main file
and \textit{dest} is the name of the main or child file to be processed
(all filenames without extensions).
The optional argument \textit{main} can be omitted
if \textit{main} matches \textit{dest}.
Optionally, compilation \textit{flags} can be defined via |\def| commands.
This command line makes the \TeX{} engine believe
it is compiling the file \textit{target}
whose content is specified as the latter parameter.
The provided code then forwards the processing to
\textit{main} or \textit{dest} as described in \secref{sec:forward}.

%%%%%%%%%%%%%%%%%%%%%%%%%%%%%%%%%%%%%%%%%%%%%%%%%%%%%%%%%%%%%%%%%%%%%%%%%%%%%%%%
\subsection{Include by Input}
\label{sec:input}

Including child documents by |\include| has some restrictions by design.
Most notably, the content of a child document always occupies
its own set of pages; pages cannot be shared between child documents.
Usually, this behaviour makes perfect sense
because each child document contain an essential part of the document.
However, in some situations it may be desirable to compose
a document from a collection of parts
without having mandatory page breaks between then.
For this case, the package
provides a mechanism to include parts
by |\input| which can also be processed individually.
However, by construction this mechanism
requires manual handling of the content to be output.

%%%%%%%%%%%%%%%%%%%%%%%%%%%%%%%%%%%%%%%%
\DescribeMacro{\ifchilddocmanual}
The main file should be prepared as usual, see \secref{sec:include}.
However, the document body must make a distinction
between processing of an individual part and of the main document, e.g.:
%
\begin{center}
\begin{tabular}{l}
|\ifchilddocmanual|\\
|\input{\childdocname}|\\
|\||else|\\
\textit{document body with }|\input{|\textit{part}|}|\\
|\||fi|
\end{tabular}
\end{center}
%
The conditional |\ifchilddocmanual| is true whenever
a part to be included by |\input| is being compiled,
and the name of the part is stored in |\childdocname|.

%%%%%%%%%%%%%%%%%%%%%%%%%%%%%%%%%%%%%%%%
\DescribeMacro{\childdocby}
Each part to be included by |\input| should start with:
%
\begin{center}
\begin{tabular}{l}
|\input{childdoc.def}|\\
|\childdocby{|\textit{main}|}|\\
\end{tabular}
\end{center}
%
The directive |\childdocby| is similar to |\childdocof|
described in \secref{sec:include},
but the subsequent selection of content must be done manually.
To that end, both |\ifchilddoc| and |\ifchilddocmanual|
will be true upon processing of a part,
and the name of the part is stored in |\childdocname|.
Note that |\jobname| will be set to the filename of the current part
so that each part receives an individual |.aux| file
that does not interfere with the |.aux| file(s) of the main document.
This behaviour can be altered by the alternative form
|\childdocby[*]{|\textit{main}|}| (with a non-empty optional argument)
which uses the |.aux| file of the main document
by setting |\jobname| to \textit{main}.

%%%%%%%%%%%%%%%%%%%%%%%%%%%%%%%%%%%%%%%%%%%%%%%%%%%%%%%%%%%%%%%%%%%%%%%%%%%%%%%%
\subsection{Driver Development}
\label{sec:driver}

The \textsf{childdoc} mechanism can also be use for the development
of definition files such as \LaTeX{} styles or classes.
This case differs from the above setup with multiple parts
included by |\include| in that no |\includeonly| should be invoked.
This can be achieved by starting the include file
(before |\ProvidesPackage|) with:
%
\begin{center}
\begin{tabular}{l}
|\input{childdoc.def}|\\
|\childdocforward{|\textit{main}|}|\\
\end{tabular}
\end{center}
%
or alternatively with:
%
\begin{center}
\begin{tabular}{l}
|\input{childdoc.def}|\\
|\childdocby{|\textit{main}|}|\\
\end{tabular}
\end{center}
%
Both forms have slightly different effects as described above.
The main file is prepared as usual, see \secref{sec:include}.

%%%%%%%%%%%%%%%%%%%%%%%%%%%%%%%%%%%%%%%%%%%%%%%%%%%%%%%%%%%%%%%%%%%%%%%%%%%%%%%%
\subsection{Legacy Detection}
\label{sec:detection}

The directive |\childdocmain| in the main file can detect
whether the complete document or merely a child is to be compiled
even without using the directive |\childdocof|.
This method is deprecated because it is less robust
and there is no compelling reason to use it;
it is merely provided for backward compatibility
and it may be removed in future versions.

If the detection mechanism is to be used,
it is mandatory to correctly specify
the filename of the main file as the argument of |\childdocmain|:
%
\begin{center}
\begin{tabular}{l}
|\input{childdoc.def}|\\
|\childdocmain{|\textit{main}|}|\\
\end{tabular}
\end{center}
%
If |\jobname| does not match the argument \textit{main} of |\childdocmain|,
it is assumed that |\jobname| points to the child file to be compiled.
When using |\childdocmain| with the main file specified as argument,
it suffices to start a child file
with just |\input{|\textit{main}|}|
without loading of the package and using |\childdocof|.
If instead all processing is done
with the appropriate \textsf{childdoc} directives,
the argument of \textit{main} of |\childdocmain| can be empty.

An alternative version of the command line processing described
in \secref{sec:commandline} using the detection mechanism reads:
%
\begin{center}
|... -jobname "|\textit{target}|" "|[\textit{flags}]%
[|\def\jobname{|\textit{dest}|}|]|\input{|\textit{main}|}"|
\end{center}

%%%%%%%%%%%%%%%%%%%%%%%%%%%%%%%%%%%%%%%%%%%%%%%%%%%%%%%%%%%%%%%%%%%%%%%%%%%%%%%%
\subsection{Manual Code}
\label{sec:manual}

In case one cannot be certain whether the definitions file |childdoc.def|
is installed on the target \TeX{} distribution
and one prefers not to ship it,
it is conceivable to paste a few relevant commands into the sources.

To that end, drop all statements |\input{childdoc.def}|
and perform the replacements as outlined below.
Instead of |\childdocmain{|\textit{main}|}| add the following code
to the top of the main file:
%
\begin{center}
\begin{tabular}{l}
|\||ifdefined\childdocname\endinput\||fi\newif\ifchilddoc|\\
|\edef\childdocname{\scantokens\expandafter{\jobname\noexpand}}|\\
|\def\childdocmain{|\textit{main}|}\||ifx\childdocmain\childdocname\||else|\\
|\childdoctrue\includeonly{\childdocname}\let\jobname\childdocmain\||fi|\\
\end{tabular}
\end{center}
%
Instead of |\childdocof{|\textit{main}|}| just include the main file
at the top of each child file:
%
\begin{center}
|\input{|\textit{main}|}|
\end{center}
%
A simple redirection |\childdocforward{|\textit{dest}|}| is achieved by:
%
\begin{center}
|\def\jobname{|\textit{dest}|}\input{\jobname}|
\end{center}
%
The redirection with prefix
|\childdocforwardprefix[|\textit{prefix}|]{|\textit{dest}|}|
is accomplished by:
%
\begin{center}
\begin{tabular}{l}
|{\edef\jobname{\scantokens\expandafter{\jobname\noexpand}}|\\
|\def\redirectjob |\textit{prefix}|#1~~~{\gdef\jobname{|\textit{dest}|#1}}|\\
|\expandafter\redirectjob\jobname~~~}\input{\jobname}|
\end{tabular}
\end{center}

In an alternative approach,
child documents can be compiled by a specific command line
without additional code or specific definitions:
%
\begin{center}
|... -jobname "|\textit{target}|" "|[\textit{flags}]%
|\includeonly{|\textit{dest}|}\input{|\textit{main}|}"|
\end{center}
%

%%%%%%%%%%%%%%%%%%%%%%%%%%%%%%%%%%%%%%%%%%%%%%%%%%%%%%%%%%%%%%%%%%%%%%%%%%%%%%%%
%%%%%%%%%%%%%%%%%%%%%%%%%%%%%%%%%%%%%%%%%%%%%%%%%%%%%%%%%%%%%%%%%%%%%%%%%%%%%%%%
\section{Information}

%%%%%%%%%%%%%%%%%%%%%%%%%%%%%%%%%%%%%%%%%%%%%%%%%%%%%%%%%%%%%%%%%%%%%%%%%%%%%%%%
\subsection{Copyright}

Copyright \copyright{} 2017--2018 Niklas Beisert

This work may be distributed and/or modified under the
conditions of the \LaTeX{} Project Public License, either version 1.3
of this license or (at your option) any later version.
The latest version of this license is in
  \url{http://www.latex-project.org/lppl.txt}
and version 1.3 or later is part of all distributions of \LaTeX{}
version 2005/12/01 or later.

This work has the LPPL maintenance status `maintained'.

The Current Maintainer of this work is Niklas Beisert.

This work consists of the files |README.txt|, |childdoc.ins| and |childdoc.dtx|
as well as the derived files |childdoc.def|, |cdocsamp.tex|
with |cdocsch1.tex|, |cdocsch2.tex|, |cdocspt3.tex|, |cdocspt4.tex|,
|cdocsdrf.tex|, |cdocsfn1.tex|, |cdocsfn2.tex|
as well as |childdoc.pdf|.

%%%%%%%%%%%%%%%%%%%%%%%%%%%%%%%%%%%%%%%%%%%%%%%%%%%%%%%%%%%%%%%%%%%%%%%%%%%%%%%%
\subsection{Files and Installation}

The package consists of the files:
%
\begin{center}
\begin{tabular}{ll}
    |README.txt|   & readme file \\
    |childdoc.ins| & installation file \\
    |childdoc.dtx| & source file \\
    |childdoc.def| & definition file \\
    |cdocsamp.tex| & sample main file \\
    |cdocsch1.tex| & sample include file \\
    |cdocsch2.tex| & sample include file \\
    |cdocspt3.tex| & sample part file \\
    |cdocspt4.tex| & sample part file \\
    |cdocsdrf.tex| & sample redirection file \\
    |cdocsfn1.tex| & sample redirection file \\
    |cdocsfn2.tex| & sample redirection file \\
    |childdoc.pdf| & manual
\end{tabular}
\end{center}
%
The distribution consists of the files
|README.txt|, |childdoc.ins| and |childdoc.dtx|.
%
\begin{itemize}
\item
Run (pdf)\LaTeX{} on |childdoc.dtx|
to compile the manual |childdoc.pdf| (this file).
\item
Run \LaTeX{} on |childdoc.ins| to create the definitions file |childdoc.def|
and the sample |cdocsamp.tex| with include files
|cdocsch1.tex|, |cdocsch2.tex|, |cdocspt3.tex|, |cdocspt4.tex|,
|cdocsdrf.tex|, |cdocsfn1.tex|, |cdocsfn2.tex|.
Then copy the file |childdoc.def| to an appropriate directory of your \LaTeX{}
distribution, e.g.\ \textit{texmf-root}|/tex/latex/childdoc|.
\end{itemize}

%%%%%%%%%%%%%%%%%%%%%%%%%%%%%%%%%%%%%%%%%%%%%%%%%%%%%%%%%%%%%%%%%%%%%%%%%%%%%%%%
\subsection{Related CTAN Packages}

There are several other packages which offer a similar functionality:
%
\begin{itemize}
\item
The packages
\href{http://ctan.org/pkg/docmute}{\textsf{docmute}},
\href{http://ctan.org/pkg/includex}{\textsf{includex}} and
\href{http://ctan.org/pkg/standalone}{\textsf{standalone}}
provide commands to include only the document body of
a child file thus allowing both files to be compiled individually.
\item
The packages \href{http://ctan.org/pkg/subdocs}{\textsf{subdocs}}
and \href{http://ctan.org/pkg/subfiles}{\textsf{subfiles}}
provide structures in which the main and child documents can be
encapsulated and allowing them to be compiled individually.
The inclusion mechanism is different from the conventional |\include|.
\item
The package \href{http://ctan.org/pkg/combine}{\textsf{combine}}
is an elaborate solution to combine several documents into one.
\end{itemize}
%
See also the CTAN topic \href{http://ctan.org/topic/subdocs}{\textsf{subdocs}}
for further related packages.
The present package differs from the above solutions in that
a document structure constructed with the conventional |\include| mechanism
just needs two extra commands at the top of every file
such that all constituent files can be compiled individually.

%%%%%%%%%%%%%%%%%%%%%%%%%%%%%%%%%%%%%%%%%%%%%%%%%%%%%%%%%%%%%%%%%%%%%%%%%%%%%%%%
%\subsection{Feature Suggestions}
%
%The following is a list of features which may be useful for future
%versions of this package:
%%
%\begin{itemize}
%\item
%\ldots
%\end{itemize}

%%%%%%%%%%%%%%%%%%%%%%%%%%%%%%%%%%%%%%%%%%%%%%%%%%%%%%%%%%%%%%%%%%%%%%%%%%%%%%%%
\subsection{Revision History}

%%%%%%%%%%%%%%%%%%%%%%%%%%%%%%%%%%%%%%%%
\paragraph{v2.0:} 2018/12/30

\begin{itemize}
\item
immediate forward processing
\item
added |\childdocby| mechanism
\item
manual restructured
\end{itemize}

%%%%%%%%%%%%%%%%%%%%%%%%%%%%%%%%%%%%%%%%
\paragraph{v1.6:} 2018/01/17

\begin{itemize}
\item
application for development of include files
\item
corrections to manual
\end{itemize}

%%%%%%%%%%%%%%%%%%%%%%%%%%%%%%%%%%%%%%%%
\paragraph{v1.5:} 2017/05/21

\begin{itemize}
\item
more complete structuring introduced
\item
|\childdocof| introduced
\item
|\childdoc| renamed to |\childdocmain|
\item
|\childredirect| renamed to |\childdocforward| and |\childdocforwardprefix|
and functionality expanded
\end{itemize}

%%%%%%%%%%%%%%%%%%%%%%%%%%%%%%%%%%%%%%%%
\paragraph{v1.0:} 2017/04/27

\begin{itemize}
\item
manual and install package
\item
first version published on CTAN
\end{itemize}

%%%%%%%%%%%%%%%%%%%%%%%%%%%%%%%%%%%%%%%%
\paragraph{v0.6:} 2017/04/26

\begin{itemize}
\item
redirection mechanism added
\end{itemize}

%%%%%%%%%%%%%%%%%%%%%%%%%%%%%%%%%%%%%%%%
\paragraph{v0.5:} 2017/04/26

\begin{itemize}
\item
functionality in definition file
\end{itemize}


%%%%%%%%%%%%%%%%%%%%%%%%%%%%%%%%%%%%%%%%%%%%%%%%%%%%%%%%%%%%%%%%%%%%%%%%%%%%%%%%
%%%%%%%%%%%%%%%%%%%%%%%%%%%%%%%%%%%%%%%%%%%%%%%%%%%%%%%%%%%%%%%%%%%%%%%%%%%%%%%%
%%%%%%%%%%%%%%%%%%%%%%%%%%%%%%%%%%%%%%%%%%%%%%%%%%%%%%%%%%%%%%%%%%%%%%%%%%%%%%%%
\appendix

\settowidth\MacroIndent{\rmfamily\scriptsize 000\ }

 \DocInput{childdoc.dtx}

\end{document}
%</driver>
% \fi
%
% %%%%%%%%%%%%%%%%%%%%%%%%%%%%%%%%%%%%%%%%%%%%%%%%%%%%%%%%%%%%%%%%%%%%%%%%%%%%%%
% %%%%%%%%%%%%%%%%%%%%%%%%%%%%%%%%%%%%%%%%%%%%%%%%%%%%%%%%%%%%%%%%%%%%%%%%%%%%%%
% \section{Sample}
%\iffalse
%<*samplemain>
%\fi
%
% The following presents a sample document
% with two chapters, two parts, a title page,
% a compile flag as well as three forwarding files to set the flag.
% It consists of eight |.tex| files:
% \begin{center}
% \begin{tabular}{ll}
% |cdocsamp.tex|&main file\\
% |cdocsch1.tex|&include file for chapter 1\\
% |cdocsch2.tex|&include file for chapter 2\\
% |cdocspt3.tex|&include file for part 3\\
% |cdocspt4.tex|&include file for part 4\\
% |cdocsdrf.tex|&forwarding file for main file in draft mode\\
% |cdocsfi1.tex|&forwarding file for final version of chapter 1\\
% |cdocsfi2.tex|&forwarding file for final version of chapter 2\\
% \end{tabular}
% \end{center}
% Each of the eight files can be compiled directly by the \LaTeX{} compiler.
%
% %%%%%%%%%%%%%%%%%%%%%%%%%%%%%%%%%%%%%%
% \paragraph{Main File.}
%
% The main file is called |cdocsamp.tex|.
%
% Load the \textsf{childdoc} definitions and
% declare the filename for the main document:
%    \begin{macrocode}
\input{childdoc.def}
\childdocmain{}
%    \end{macrocode}

% Optional override for |\version| flag:
%    \begin{macrocode}
%%\ifchilddoc\else\providecommand{\version}{draft}\fi
%    \end{macrocode}

% Define the default values for the |\version| flag
% (|final| for the main file and |draft| for childs):
%    \begin{macrocode}
\ifchilddoc
\providecommand{\version}{draft}
\else
\providecommand{\version}{final}
\fi
%    \end{macrocode}

% Load the standard document class:
%    \begin{macrocode}
\documentclass[12pt]{article}
%    \end{macrocode}

% Start the document body:
%    \begin{macrocode}
\begin{document}
%    \end{macrocode}

% Declare a title page.
% Print title, part of document being processed and version flag:
%    \begin{macrocode}
\addtocounter{page}{-1}
\begin{center}
{\LARGE\bfseries{}childdoc example\par}
\vspace{1cm}
\ifchilddoc
\ifchilddocmanual part\else chapter\fi:
`\childdocname' of `\childdocjob'\par
\else
main document: `\childdocjob'\par
\fi
version: \version\par
\end{center}
\newpage
%    \end{macrocode}

% Manually include selected file,
% otherwise process as usual:
%    \begin{macrocode}
\ifchilddocmanual
\section*{part `\childdocname'}
\input{\childdocname}
\else
%    \end{macrocode}

% Include the two chapters:
%    \begin{macrocode}
\include{cdocsch1}
\include{cdocsch2}
%    \end{macrocode}

% Include the two parts unless only chapters should be displayed:
%    \begin{macrocode}
\ifchilddoc\else
\section{part three}
\input{cdocspt3}
\section{part four}
\input{cdocspt4}
\fi
%    \end{macrocode}

% Process as usual until here:
%    \begin{macrocode}
\fi
%    \end{macrocode}

% End of document body:
%    \begin{macrocode}
\end{document}
%    \end{macrocode}
%\iffalse
%</samplemain>
%\fi
%
% %%%%%%%%%%%%%%%%%%%%%%%%%%%%%%%%%%%%%%
% \paragraph{Chapter Include Files.}
%
% The include files are called |cdocsch1.tex| and |cdocsch2.tex|.
%
%\iffalse
%<*samplechap1|samplechap2>
%\fi

% Optional override for |\version| flag:
%    \begin{macrocode}
%%\providecommand{\version}{final}
%    \end{macrocode}

% Include the main document:
%    \begin{macrocode}
\input{childdoc.def}
\childdocof{cdocsamp}
%    \end{macrocode}

%\iffalse
%</samplechap1|samplechap2>
%\fi
%
%\iffalse
%<*samplechap1>
%\fi
% Some text for chapter 1:
%    \begin{macrocode}
\section{one}
some text in chapter one
%    \end{macrocode}

%\iffalse
%</samplechap1>
%\fi
% Some text for chapter 2:
%\iffalse
%<*samplechap2>
%\fi
%    \begin{macrocode}
\section{two}
more text in chapter two
%    \end{macrocode}

%\iffalse
%</samplechap2>
%\fi
%
% %%%%%%%%%%%%%%%%%%%%%%%%%%%%%%%%%%%%%%
% \paragraph{Part Include Files.}
%
% The include files are called |cdocspt3.tex| and |cdocspt4.tex|.
%
%\iffalse
%<*samplepart3|samplepart4>
%\fi

% Optional override for |\version| flag:
%    \begin{macrocode}
%%\providecommand{\version}{final}
%    \end{macrocode}

% Include the main document:
%    \begin{macrocode}
\input{childdoc.def}
\childdocby{cdocsamp}
%    \end{macrocode}

%\iffalse
%</samplepart3|samplepart4>
%\fi
%
%\iffalse
%<*samplepart3>
%\fi
% Some text for part 3:
%    \begin{macrocode}
some text in part three
%    \end{macrocode}

%\iffalse
%</samplepart3>
%\fi
% Some text for part 4:
%\iffalse
%<*samplepart4>
%\fi
%    \begin{macrocode}
more text in part four
%    \end{macrocode}

%\iffalse
%</samplepart4>
%\fi
%
% %%%%%%%%%%%%%%%%%%%%%%%%%%%%%%%%%%%%%%
% \paragraph{Forwarding for a Complete Draft.}
%
% The following forwarding file |cdocsdrf.tex|
% compiles the main document in draft mode:
%\iffalse
%<*sampledraft>
%\fi
%    \begin{macrocode}
\def\version{draft}
\input{childdoc.def}
\childdocforward{cdocsamp}
%    \end{macrocode}

%\iffalse
%</sampledraft>
%\fi
%
% %%%%%%%%%%%%%%%%%%%%%%%%%%%%%%%%%%%%%%
% \paragraph{Forwarding for Final Version of the Chapters.}
%
% The following forwarding files |cdocsfn1.tex| and |cdocsfn2.tex|
% (with identical content)
% compile the final versions of the child documents
% |cdocsch1.tex| and |cdocsch2.tex|, respectively:
%\iffalse
%<*samplefinal>
%\fi
%    \begin{macrocode}
\def\version{final}
\input{childdoc.def}
\childdocforwardprefix[cdocsamp]{cdocsfn}{cdocsch}
%    \end{macrocode}

%\iffalse
%</samplefinal>
%\fi
%
% %%%%%%%%%%%%%%%%%%%%%%%%%%%%%%%%%%%%%%
% \paragraph{Command Line Processing.}
%
% The following three command lines generate the output files
% |cdocscld|, |cdocscl1| and |cdocscl2|
% which should be identical to
% |cdocsdrf|, |cdocsch1| and |cdocsfn2|, respectively:
% \begin{center}
% \begin{tabular}{l}
% |latex -jobname cdocscld \|\\
% |  "\def\version{draft}\input{childdoc.def}\childdocforward{cdocsamp}"|\\
% |latex -jobname cdocscl1 \|\\
% |  "\input{childdoc.def}\childdocforward[cdocsamp]{cdocsch1}"|\\
% |latex -jobname cdocscl2 \|\\
% |  "\def\version{final}\input{childdoc.def}\childdocforward{cdocsch2}"|
% \end{tabular}
% \end{center}
% Note that the trailing backslash on each first line
% merely continues the input to the second line
% (for convenient cut ant paste).
% Furthermore, the command |latex| can be replaced by any
% of its alternative versions such as |pdflatex|.
%
% %%%%%%%%%%%%%%%%%%%%%%%%%%%%%%%%%%%%%%%%%%%%%%%%%%%%%%%%%%%%%%%%%%%%%%%%%%%%%%
% %%%%%%%%%%%%%%%%%%%%%%%%%%%%%%%%%%%%%%%%%%%%%%%%%%%%%%%%%%%%%%%%%%%%%%%%%%%%%%
% \section{Implementation}
%\iffalse
%<*package>
%\fi
%
% This section describes the definitions file |childdoc.def|.

% The definitions cannot be loaded using |\usepackage| or |\RequirePackage|
% which has a mechanism to prevent loading a style file more than once.
% When loading the definitions by means of |\input|
% multiple instances have to be prevented manually:
%\iffalse
%This code needs to be before the `\ProvidesFile' directive
%which is defined at the beginning of this file.
%Therefore it is also placed there and commented out here.
%</package>
%<*discard>
%\fi
%    \begin{macrocode}
\ifdefined\childdocmain\endinput\fi
%    \end{macrocode}
%\iffalse
%</discard>
%<*package>
%\fi
%
% \macro{\ifchilddoc}
% \macro{\ifchilddocmanual}
% The conditional |\ifchilddoc| tells whether a
% child (true) or main (false) document is being compiled.
% The conditional |\ifchilddocmanual| tells whether
% the |\includeonly| mechanism is used (false) or
% the selection of child files must be performed manually (true).
% The definitions initialise to false:
%    \begin{macrocode}
\newif\ifchilddoc
\newif\ifchilddocmanual
%    \end{macrocode}

% \macro{\childdocname}
% \macro{\childdocjob}
% The macro |\childdocname| stores the name of the main document
% to be compiled. The macro |\childdocjob| stores the name of
% the document on which the \LaTeX{} compiler was originally invoked.
% The content of |\jobname| cannot be compared
% to filenames specified in the source due to different catcodes.
% The following code rescans |\jobname|, stores the result
% in |\childdocname| and saves a copy in |\childdocjob|:
%    \begin{macrocode}
\edef\childdocname{\scantokens\expandafter{\jobname\noexpand}}
\let\childdocjob\childdocname
%    \end{macrocode}

% \macro{\childdocdisable}
% The macro |\childdocdisable| prevents the main file
% from being processed more than once.
% At this stage, the main document command |\childdocmain|
% is assumed to be called once again where it should do nothing.
% Any subsequent call to it should prevent
% a secondary processing of the main document
% It overwrites the forwarding commands
% |\childdocof| and |\childdocforward|
% with empty macros to prevent further inclusions of the main document:
%    \begin{macrocode}
\newcommand{\childdocdisable}
{
  \renewcommand{\childdocmain}[1]{\renewcommand{\childdocmain}[1]{\endinput}}
  \renewcommand{\childdocof}[1]{}
  \renewcommand{\childdocby}[2][]{}
  \renewcommand{\childdocforward}[2][]{}
  \renewcommand{\childdocdisable}{}
}
%    \end{macrocode}

% \macro{\childdocmain}
% The macro |\childdocmain| is to be called at the top of the main file
% with nothing or the main filename (without extension) as argument.
% First, it breaks loops.
% If the argument is not empty and does not match |\childdocname|
% (which is set by the first inclusion of |childdoc.def|),
% |\ifchilddoc| is set to true, |\includeonly| is applied to the child file
% and |\jobname| is set to the main file
% (for proper handling of |.aux| files):
%    \begin{macrocode}
\newcommand{\childdocmain}[1]
{
  \childdocdisable\childdocmain{}
  \if?#1?\else
    \begingroup
      \def\childdoctmp{#1}
      \ifx\childdoctmp\childdocname
        \def\childdoctmp{}
      \else
        \def\childdoctmp
        {
          \childdoctrue
          \includeonly{\childdocname}
          \def\childdocjob{#1}
          \def\jobname{#1}
        }
      \fi
      \expandafter
    \endgroup
    \childdoctmp
  \fi
}
%    \end{macrocode}

% \macro{\childdocof}
% The command |\childdocof| redirects
% compilation to the main file |#1|.
%    \begin{macrocode}
\newcommand{\childdocof}[1]
{
  \childdocdisable
  \childdoctrue
  \includeonly{\childdocname}
  \def\jobname{#1}
  \def\childdocjob{#1}
  \input{#1}
}
%    \end{macrocode}

% \macro{\childdocby}
% The command |\childdocby| ....
%    \begin{macrocode}
\newcommand{\childdocby}[2][]
{
  \childdocdisable
  \childdoctrue
  \childdocmanualtrue
  \if?#1?\else
    \def\jobname{#2}
  \fi
  \def\childdocjob{#2}
  \input{#2}
  \endinput
}
%    \end{macrocode}

% \macro{\childdocforward}
% The command |\childdocforward| redirects
% compilation to the main file or
% (if the optional argument is given) a child file.
% Parameters are set as if the main file
% or a child file starting with |\childdocof| was compiled.
% Then compilation is handed over to the main file:
%    \begin{macrocode}
\newcommand{\childdocforward}[2][]
{
  \begingroup
    \if?#1?
      \def\childdoctmp
      {
        \def\childdocname{#2}
        \def\childdocjob{#2}
        \def\jobname{#2}
        \input{#2}
        \endinput
      }
    \else
      \def\childdoctmp
      {
        \childdocdisable
        \def\childdocname{#2}
        \childdoctrue
        \includeonly{#2}
        \def\childdocjob{#1}
        \def\jobname{#1}
        \input{#1}
        \endinput
      }
    \fi
    \expandafter
  \endgroup
  \childdoctmp
}
%    \end{macrocode}

% \macro{\childdocforwardprefix}
% The command |\childdocforwardprefix| redirects
% compilation to the main or a child file by means of a pattern.
% The prefix |#1| in the current filename is replaced by |#2|
% and the suffix of the current filename is kept
% (it is assumed that the filename does not contain the substring `|~~~|'
% which is used as a delimiter).
% Compilation is handed over to the new file by |\childdocforward|:
%    \begin{macrocode}
\newcommand{\childdocforwardprefix}[3][]
{
  \begingroup
    \def\childdocextract #2##1~~~{\def\childdoctmp{\childdocforward[#1]{#3##1}}}
    \expandafter\childdocextract\childdocname~~~
    \expandafter
  \endgroup
  \childdoctmp
}
%    \end{macrocode}

% \macro{\childdoc}
% The deprecated macro |\childdoc| is a legacy version of |\childdocmain|:
%    \begin{macrocode}
\newcommand{\childdoc}{\childdocmain}
%    \end{macrocode}

% \macro{\childdocredirect}
% The deprecated macro |\childdocredirect| is a legacy version
% of |\childdocforward| and |\childdocforwardprefix|:
%    \begin{macrocode}
\newcommand{\childdocredirect}[2][]
{
  \begingroup
    \if?#1?
      \def\childdoctmp{\childdocforward{#2}}
    \else
      \def\childdoctmp{\childdocforwardprefix{#1}{#2}}
    \fi
    \expandafter
  \endgroup
  \childdoctmp
}
%    \end{macrocode}

%\iffalse
%</package>
%\fi
%
\endinput
|
and perform the replacements as outlined below.
Instead of |\childdocmain{|\textit{main}|}| add the following code
to the top of the main file:
%
\begin{center}
\begin{tabular}{l}
|\||ifdefined\childdocname\endinput\||fi\newif\ifchilddoc|\\
|\edef\childdocname{\scantokens\expandafter{\jobname\noexpand}}|\\
|\def\childdocmain{|\textit{main}|}\||ifx\childdocmain\childdocname\||else|\\
|\childdoctrue\includeonly{\childdocname}\let\jobname\childdocmain\||fi|\\
\end{tabular}
\end{center}
%
Instead of |\childdocof{|\textit{main}|}| just include the main file
at the top of each child file:
%
\begin{center}
|\input{|\textit{main}|}|
\end{center}
%
A simple redirection |\childdocforward{|\textit{dest}|}| is achieved by:
%
\begin{center}
|\def\jobname{|\textit{dest}|}\input{\jobname}|
\end{center}
%
The redirection with prefix
|\childdocforwardprefix[|\textit{prefix}|]{|\textit{dest}|}|
is accomplished by:
%
\begin{center}
\begin{tabular}{l}
|{\edef\jobname{\scantokens\expandafter{\jobname\noexpand}}|\\
|\def\redirectjob |\textit{prefix}|#1~~~{\gdef\jobname{|\textit{dest}|#1}}|\\
|\expandafter\redirectjob\jobname~~~}\input{\jobname}|
\end{tabular}
\end{center}

In an alternative approach,
child documents can be compiled by a specific command line
without additional code or specific definitions:
%
\begin{center}
|... -jobname "|\textit{target}|" "|[\textit{flags}]%
|\includeonly{|\textit{dest}|}\input{|\textit{main}|}"|
\end{center}
%

%%%%%%%%%%%%%%%%%%%%%%%%%%%%%%%%%%%%%%%%%%%%%%%%%%%%%%%%%%%%%%%%%%%%%%%%%%%%%%%%
%%%%%%%%%%%%%%%%%%%%%%%%%%%%%%%%%%%%%%%%%%%%%%%%%%%%%%%%%%%%%%%%%%%%%%%%%%%%%%%%
\section{Information}

%%%%%%%%%%%%%%%%%%%%%%%%%%%%%%%%%%%%%%%%%%%%%%%%%%%%%%%%%%%%%%%%%%%%%%%%%%%%%%%%
\subsection{Copyright}

Copyright \copyright{} 2017--2018 Niklas Beisert

This work may be distributed and/or modified under the
conditions of the \LaTeX{} Project Public License, either version 1.3
of this license or (at your option) any later version.
The latest version of this license is in
  \url{http://www.latex-project.org/lppl.txt}
and version 1.3 or later is part of all distributions of \LaTeX{}
version 2005/12/01 or later.

This work has the LPPL maintenance status `maintained'.

The Current Maintainer of this work is Niklas Beisert.

This work consists of the files |README.txt|, |childdoc.ins| and |childdoc.dtx|
as well as the derived files |childdoc.def|, |cdocsamp.tex|
with |cdocsch1.tex|, |cdocsch2.tex|, |cdocspt3.tex|, |cdocspt4.tex|,
|cdocsdrf.tex|, |cdocsfn1.tex|, |cdocsfn2.tex|
as well as |childdoc.pdf|.

%%%%%%%%%%%%%%%%%%%%%%%%%%%%%%%%%%%%%%%%%%%%%%%%%%%%%%%%%%%%%%%%%%%%%%%%%%%%%%%%
\subsection{Files and Installation}

The package consists of the files:
%
\begin{center}
\begin{tabular}{ll}
    |README.txt|   & readme file \\
    |childdoc.ins| & installation file \\
    |childdoc.dtx| & source file \\
    |childdoc.def| & definition file \\
    |cdocsamp.tex| & sample main file \\
    |cdocsch1.tex| & sample include file \\
    |cdocsch2.tex| & sample include file \\
    |cdocspt3.tex| & sample part file \\
    |cdocspt4.tex| & sample part file \\
    |cdocsdrf.tex| & sample redirection file \\
    |cdocsfn1.tex| & sample redirection file \\
    |cdocsfn2.tex| & sample redirection file \\
    |childdoc.pdf| & manual
\end{tabular}
\end{center}
%
The distribution consists of the files
|README.txt|, |childdoc.ins| and |childdoc.dtx|.
%
\begin{itemize}
\item
Run (pdf)\LaTeX{} on |childdoc.dtx|
to compile the manual |childdoc.pdf| (this file).
\item
Run \LaTeX{} on |childdoc.ins| to create the definitions file |childdoc.def|
and the sample |cdocsamp.tex| with include files
|cdocsch1.tex|, |cdocsch2.tex|, |cdocspt3.tex|, |cdocspt4.tex|,
|cdocsdrf.tex|, |cdocsfn1.tex|, |cdocsfn2.tex|.
Then copy the file |childdoc.def| to an appropriate directory of your \LaTeX{}
distribution, e.g.\ \textit{texmf-root}|/tex/latex/childdoc|.
\end{itemize}

%%%%%%%%%%%%%%%%%%%%%%%%%%%%%%%%%%%%%%%%%%%%%%%%%%%%%%%%%%%%%%%%%%%%%%%%%%%%%%%%
\subsection{Related CTAN Packages}

There are several other packages which offer a similar functionality:
%
\begin{itemize}
\item
The packages
\href{http://ctan.org/pkg/docmute}{\textsf{docmute}},
\href{http://ctan.org/pkg/includex}{\textsf{includex}} and
\href{http://ctan.org/pkg/standalone}{\textsf{standalone}}
provide commands to include only the document body of
a child file thus allowing both files to be compiled individually.
\item
The packages \href{http://ctan.org/pkg/subdocs}{\textsf{subdocs}}
and \href{http://ctan.org/pkg/subfiles}{\textsf{subfiles}}
provide structures in which the main and child documents can be
encapsulated and allowing them to be compiled individually.
The inclusion mechanism is different from the conventional |\include|.
\item
The package \href{http://ctan.org/pkg/combine}{\textsf{combine}}
is an elaborate solution to combine several documents into one.
\end{itemize}
%
See also the CTAN topic \href{http://ctan.org/topic/subdocs}{\textsf{subdocs}}
for further related packages.
The present package differs from the above solutions in that
a document structure constructed with the conventional |\include| mechanism
just needs two extra commands at the top of every file
such that all constituent files can be compiled individually.

%%%%%%%%%%%%%%%%%%%%%%%%%%%%%%%%%%%%%%%%%%%%%%%%%%%%%%%%%%%%%%%%%%%%%%%%%%%%%%%%
%\subsection{Feature Suggestions}
%
%The following is a list of features which may be useful for future
%versions of this package:
%%
%\begin{itemize}
%\item
%\ldots
%\end{itemize}

%%%%%%%%%%%%%%%%%%%%%%%%%%%%%%%%%%%%%%%%%%%%%%%%%%%%%%%%%%%%%%%%%%%%%%%%%%%%%%%%
\subsection{Revision History}

%%%%%%%%%%%%%%%%%%%%%%%%%%%%%%%%%%%%%%%%
\paragraph{v2.0:} 2018/12/30

\begin{itemize}
\item
immediate forward processing
\item
added |\childdocby| mechanism
\item
manual restructured
\end{itemize}

%%%%%%%%%%%%%%%%%%%%%%%%%%%%%%%%%%%%%%%%
\paragraph{v1.6:} 2018/01/17

\begin{itemize}
\item
application for development of include files
\item
corrections to manual
\end{itemize}

%%%%%%%%%%%%%%%%%%%%%%%%%%%%%%%%%%%%%%%%
\paragraph{v1.5:} 2017/05/21

\begin{itemize}
\item
more complete structuring introduced
\item
|\childdocof| introduced
\item
|\childdoc| renamed to |\childdocmain|
\item
|\childredirect| renamed to |\childdocforward| and |\childdocforwardprefix|
and functionality expanded
\end{itemize}

%%%%%%%%%%%%%%%%%%%%%%%%%%%%%%%%%%%%%%%%
\paragraph{v1.0:} 2017/04/27

\begin{itemize}
\item
manual and install package
\item
first version published on CTAN
\end{itemize}

%%%%%%%%%%%%%%%%%%%%%%%%%%%%%%%%%%%%%%%%
\paragraph{v0.6:} 2017/04/26

\begin{itemize}
\item
redirection mechanism added
\end{itemize}

%%%%%%%%%%%%%%%%%%%%%%%%%%%%%%%%%%%%%%%%
\paragraph{v0.5:} 2017/04/26

\begin{itemize}
\item
functionality in definition file
\end{itemize}


%%%%%%%%%%%%%%%%%%%%%%%%%%%%%%%%%%%%%%%%%%%%%%%%%%%%%%%%%%%%%%%%%%%%%%%%%%%%%%%%
%%%%%%%%%%%%%%%%%%%%%%%%%%%%%%%%%%%%%%%%%%%%%%%%%%%%%%%%%%%%%%%%%%%%%%%%%%%%%%%%
%%%%%%%%%%%%%%%%%%%%%%%%%%%%%%%%%%%%%%%%%%%%%%%%%%%%%%%%%%%%%%%%%%%%%%%%%%%%%%%%
\appendix

\settowidth\MacroIndent{\rmfamily\scriptsize 000\ }

 \DocInput{childdoc.dtx}

\end{document}
%</driver>
% \fi
%
% %%%%%%%%%%%%%%%%%%%%%%%%%%%%%%%%%%%%%%%%%%%%%%%%%%%%%%%%%%%%%%%%%%%%%%%%%%%%%%
% %%%%%%%%%%%%%%%%%%%%%%%%%%%%%%%%%%%%%%%%%%%%%%%%%%%%%%%%%%%%%%%%%%%%%%%%%%%%%%
% \section{Sample}
%\iffalse
%<*samplemain>
%\fi
%
% The following presents a sample document
% with two chapters, two parts, a title page,
% a compile flag as well as three forwarding files to set the flag.
% It consists of eight |.tex| files:
% \begin{center}
% \begin{tabular}{ll}
% |cdocsamp.tex|&main file\\
% |cdocsch1.tex|&include file for chapter 1\\
% |cdocsch2.tex|&include file for chapter 2\\
% |cdocspt3.tex|&include file for part 3\\
% |cdocspt4.tex|&include file for part 4\\
% |cdocsdrf.tex|&forwarding file for main file in draft mode\\
% |cdocsfi1.tex|&forwarding file for final version of chapter 1\\
% |cdocsfi2.tex|&forwarding file for final version of chapter 2\\
% \end{tabular}
% \end{center}
% Each of the eight files can be compiled directly by the \LaTeX{} compiler.
%
% %%%%%%%%%%%%%%%%%%%%%%%%%%%%%%%%%%%%%%
% \paragraph{Main File.}
%
% The main file is called |cdocsamp.tex|.
%
% Load the \textsf{childdoc} definitions and
% declare the filename for the main document:
%    \begin{macrocode}
% \iffalse
%
% childdoc.dtx Copyright (C) 2017-2018 Niklas Beisert
%
% This work may be distributed and/or modified under the
% conditions of the LaTeX Project Public License, either version 1.3
% of this license or (at your option) any later version.
% The latest version of this license is in
%   http://www.latex-project.org/lppl.txt
% and version 1.3 or later is part of all distributions of LaTeX
% version 2005/12/01 or later.
%
% This work has the LPPL maintenance status `maintained'.
%
% The Current Maintainer of this work is Niklas Beisert.
%
% This work consists of the files childdoc.dtx and childdoc.ins
% and the derived files childdoc.def and cdocsamp.tex with
% cdocsch1.tex, cdocsch2.tex, cdocsdrf.tex, cdocsfn1.tex, cdocsfn2.tex.
%
%<package>\ifdefined\childdocmain\endinput\fi
%<package>\ProvidesFile{childdoc.def}[2018/12/30 v2.0 child document driver]
%<samplemain>\ProvidesFile{cdocsamp.tex}[2018/12/30 v2.0 sample for childdoc]
%<*driver>
%\ProvidesFile{childdoc.drv}[2018/12/30 v2.0 childdoc reference manual file]
\PassOptionsToClass{10pt,a4paper}{article}
\documentclass{ltxdoc}

\usepackage[margin=35mm]{geometry}
\usepackage{hyperref}
\usepackage{hyperxmp}
\usepackage[usenames]{color}

\hypersetup{colorlinks=true}
\hypersetup{pdfstartview=FitH}
\hypersetup{pdfpagemode=UseNone}
\hypersetup{pdfsource={}}
\hypersetup{pdflang={en-UK}}
\hypersetup{pdfcopyright={Copyright 2017-2018 Niklas Beisert.
  This work may be distributed and/or modified under the
  conditions of the LaTeX Project Public License, either version 1.3
  of this license or (at your option) any later version.}}
\hypersetup{pdflicenseurl={http://www.latex-project.org/lppl.txt}}
\hypersetup{pdfcontactaddress={ETH Zurich, ITP, HIT K,
  Wolfgang-Pauli-Strasse 27}}
\hypersetup{pdfcontactpostcode={8093}}
\hypersetup{pdfcontactcity={Zurich}}
\hypersetup{pdfcontactcountry={Switzerland}}
\hypersetup{pdfcontactemail={nbeisert@itp.phys.ethz.ch}}
\hypersetup{pdfcontacturl={http://people.phys.ethz.ch/\xmptilde nbeisert/}}

\newcommand{\secref}[1]{\hyperref[#1]{section \ref*{#1}}}

\parskip1ex
\parindent0pt
\let\olditemize\itemize
\def\itemize{\olditemize\parskip0pt}

\begin{document}

\title{The \textsf{childdoc} Package}
\hypersetup{pdftitle={The childdoc Package}}
\author{Niklas Beisert\\[2ex]
  Institut f\"ur Theoretische Physik\\
  Eidgen\"ossische Technische Hochschule Z\"urich\\
  Wolfgang-Pauli-Strasse 27, 8093 Z\"urich, Switzerland\\[1ex]
  \href{mailto:nbeisert@itp.phys.ethz.ch}
  {\texttt{nbeisert@itp.phys.ethz.ch}}}
\hypersetup{pdfauthor={Niklas Beisert}}
\hypersetup{pdfsubject={Manual for the LaTeX2e Package childdoc}}
\date{30 December 2018, \textsf{v2.0}}
\maketitle

\begin{abstract}\noindent
\textsf{childdoc} is a \LaTeXe{} package
that enables the direct compilation
of document sections included by |\include|
to individual files.
\end{abstract}

\begingroup
\parskip0ex
\tableofcontents
\endgroup

%%%%%%%%%%%%%%%%%%%%%%%%%%%%%%%%%%%%%%%%%%%%%%%%%%%%%%%%%%%%%%%%%%%%%%%%%%%%%%%%
%%%%%%%%%%%%%%%%%%%%%%%%%%%%%%%%%%%%%%%%%%%%%%%%%%%%%%%%%%%%%%%%%%%%%%%%%%%%%%%%
\section{Introduction}

\LaTeX{} provides a mechanism to structure a large document (such as a book)
into a main file and several child files (containing the chapters)
using the |\include| command.
This mechanism is beneficial for documents
which span hundreds of pages in order to
make the source file(s) more manageable.
Moreover, compilation can be restricted to
selected child files by means of the |\includeonly| command.
The latter feature can be used to reduce the compilation time while editing
(this was significantly more useful in the earlier days of \LaTeX{})
or to generate a smaller document which is easier to navigate.
Another application of |\includeonly| is to generate
documents consisting of selected parts of the complete document.

However, there are a few drawbacks of the plain |\include| mechanism:
\begin{itemize}
\item
The child files cannot be compiled on their own,
they can only be compiled via the main file.
A naive editing environment
(such as a text editor with an option
to have the current file processed by \LaTeX)
may require one to switch to the main file before compiling;
attempting to compile the child file produces errors.
\item
The main file must be modified (each time)
to adjust the |\includeonly| command
to the present needs. This easily leaves the main file in a messy state.
\item
The generated document will always carry the filename
of the main document. This is inconvenient if
several child files are to be compiled and
to be kept for distribution.
\end{itemize}

The present package provides a simple interface
to make child files individually compilable by \LaTeX{}.
Compiling a child file then has the same effect as compiling
the main file with an |\includeonly| command
to select the appropriate child.
Moreover the generated document will carry the name of the child
rather than the main file.
This resolves all three above issues.

This feature is meant to make the editing of books,
thesis documents and lecture notes somewhat more convenient.
However, the package can also be used efficiently for
composing a series of documents (such as exercise sheets)
which are typically distributed individually.
It then assists the author in generating the individual documents
(potentially in different versions)
as well as a document containing the collected series.
Another application is in developing style files
or other kinds of included material
where compilation of the style file could redirect
to a sample or test file.

%%%%%%%%%%%%%%%%%%%%%%%%%%%%%%%%%%%%%%%%%%%%%%%%%%%%%%%%%%%%%%%%%%%%%%%%%%%%%%%%
%%%%%%%%%%%%%%%%%%%%%%%%%%%%%%%%%%%%%%%%%%%%%%%%%%%%%%%%%%%%%%%%%%%%%%%%%%%%%%%%
\section{Usage}

First of all, the package \textsf{childdoc} is \emph{not} a standard
\LaTeXe{} |.sty| style file! Therefore it needs to be invoked in
a non-standard way.

%%%%%%%%%%%%%%%%%%%%%%%%%%%%%%%%%%%%%%%%%%%%%%%%%%%%%%%%%%%%%%%%%%%%%%%%%%%%%%%%
\subsection{Included Files}
\label{sec:include}

%%%%%%%%%%%%%%%%%%%%%%%%%%%%%%%%%%%%%%%%
\DescribeMacro{\childdocmain}
To use the package, add the commands
\begin{center}
\begin{tabular}{l}
|\input{childdoc.def}|\\
|\childdocmain{}|\\
\end{tabular}
\end{center}
at the very top of the main \LaTeX{} file,
in particular \emph{before} the |\documentclass| statement!
The argument of |\childdocmain| should be left empty
(but it must be present).

%%%%%%%%%%%%%%%%%%%%%%%%%%%%%%%%%%%%%%%%
\DescribeMacro{\childdocof}
Furthermore, add the commands
\begin{center}
\begin{tabular}{l}
|\input{childdoc.def}|\\
|\childdocof{|\textit{main}|}|\\
\end{tabular}
\end{center}
at the top of every child file \textit{child}
which is included by |\include{|\textit{child}|}|
from within the main file
(or at least for those files to be compiled individually).
The argument \textit{main} must be the filename of the main file.

There are a couple of
considerations in setting up the main and child documents:

%%%%%%%%%%%%%%%%%%%%%%%%%%%%%%%%%%%%%%%%
\paragraph{Restrictions.}

Please note the following restrictions:
\begin{itemize}
\item
|\childdocmain| must be called with one argument \textit{main}
to ensure compatibility with earlier version of the package.
It must either be empty (|\childdocmain{}|)
or precisely match the filename of the main file in which it is specified.
See \secref{sec:detection} for further information.
\item
The filename \textit{main} must be specified without the |.tex| extension.
\item
The filename \textit{main} is case sensitive
(even in case-insensitive file systems)
due to internal string comparison.
\item
The argument \textit{main} should be fully expanded, it cannot be a macro.
\item
Subdirectories and special characters should be avoided in filenames.
\item
The command |\childdocmain{|\textit{main}|}| must be followed by a whitespace.
It should not be followed immediately by another command
or by a comment mark `|%|'.
This is because the \TeX{} parser reads the token immediately following
the argument of |\childdocmain| and puts it
at the beginning of every child section;
however, a white\-space is ignored.
\end{itemize}

%%%%%%%%%%%%%%%%%%%%%%%%%%%%%%%%%%%%%%%%
\paragraph{Content of Main File.}

It is advisable to place all content in the child files included by |\include|.
Any output contained in the main file will appear in all child documents
unless suppressed manually;
it cannot be suppressed automatically by the |\includeonly| directive
and thus should normally be avoided.
A method to include some content in the main file
by means of conditional processing is described in \secref{sec:conditional}.

%%%%%%%%%%%%%%%%%%%%%%%%%%%%%%%%%%%%%%%%
\paragraph{Page Numbering.}

When only a part of the document is compiled,
the appropriate numbering of pages
(as well as other status parameters)
is determined from the |.aux| files.
The latter contain information from previous passes.
However this information needs to propagate through
all intermediate child documents.
Therefore the page numbering in child documents may well
be inconsistent until the complete document is compiled at least once.

A useful (if unconventional) way to always ensure a consistent
page numbering is to restart the numbering in each child document
and denote the pages by `\textit{child}|.|\textit{page}'
where \textit{child} represents the chapter/section number of the child file.
This can be achieved by the command
|\numberwithin{page}{|\textit{child}|}|
of the \textsf{amsmath} package
where \textit{child} can be |chapter| or |section|
depending on the chosen structuring.
Alternatively, one can modify the macro |\thepage| appropriately
and reset the counter |page| at the start of each child file.

%%%%%%%%%%%%%%%%%%%%%%%%%%%%%%%%%%%%%%%%%%%%%%%%%%%%%%%%%%%%%%%%%%%%%%%%%%%%%%%%
\subsection{Conditional Processing}
\label{sec:conditional}

The package provides a mechanism to compile different versions
of a document. To customise the versions further some conditional processing
can come in handy to distinguish which version is being compiled.
The package provides two macros to describe the compilation context:

%%%%%%%%%%%%%%%%%%%%%%%%%%%%%%%%%%%%%%%%
\DescribeMacro{\ifchilddoc}
The conditional |\ifchilddoc| distinguishes between the compilation of
child documents and the main document:
%
\begin{center}
|\ifchilddoc |\textit{child-code}| |[|\||else |\textit{main-code}]| \||fi|
\end{center}

%%%%%%%%%%%%%%%%%%%%%%%%%%%%%%%%%%%%%%%%
\DescribeMacro{\childdocname}
\DescribeMacro{\childdocjob}
The macro |\childdocname| contains the filename (without extension)
of the main or child file being processed.
Note that |\childdocjob| will always contain the name of the main file.

%%%%%%%%%%%%%%%%%%%%%%%%%%%%%%%%%%%%%%%%
\paragraph{Title Page.}

Conditional processing can be used to include a title or banner page
in the main document when proper precautions are taken.
Importantly, the code in the main file should ensure that the page counter
(as well as other status parameters which are stored in the |.aux| files)
takes the same value after the conditional processing.
Otherwise the page numbers may take divergent values
depending on which part is compiled.

For example, a title page could be declared by:
%
\begin{center}
\begin{tabular}{l}
|\ifchilddoc\||else|\\
|\addtocounter{page}{-1}|\\
\textit{code for title page}\\
|\newpage|\\
|\||fi|
\end{tabular}
\end{center}
%
A banner page for the child documents can be generated by:
%
\begin{center}
\begin{tabular}{l}
|\ifchilddoc|\\
|\addtocounter{page}{-1}|\\
\textit{code for banner page}\\
|\newpage|\\
|\||fi|
\end{tabular}
\end{center}
%
Here one could write a message such as:
\begin{center}
|This is the part \childdocname{} of \childdocjob{}.|
\end{center}

%%%%%%%%%%%%%%%%%%%%%%%%%%%%%%%%%%%%%%%%%%%%%%%%%%%%%%%%%%%%%%%%%%%%%%%%%%%%%%%%
\subsection{Flags}
\label{sec:flags}

The package makes it easy to generate different versions
of the main or child documents.
To this end compilation flags can be defined
and assigned different default values.
They will be particularly useful in conjunction
with the forwarding mechanism described in \secref{sec:forward}.

For example, it may be useful to have a flag |\version|
which can be set to |draft| or |final|.
The document source will contain some conditional code
depending on the value of |\version|.
Suppose further, the flag should default to |final| for the main file
and to |draft| for child files
which is a natural assignment for editing the document.
This is achieved by placing the following code
in the preamble of the main document
(below the |\childdocmain| directive):
%
\begin{center}
\begin{tabular}{l}
|\ifchilddoc|\\
|\providecommand{\version}{draft}|\\
|\||else|\\
|\providecommand{\version}{final}|\\
|\||fi|
\end{tabular}
\end{center}
%
The definition by |\providecommand| makes sure
that previous definitions are not overwritten.
Further statements |\providecommand{\version}{...}|
can thus be added before the above code to override it.

For the main file, one might add a line
(between |\childdocmain| and the above block)
%
\begin{center}
|%\ifchilddoc\||else\providecommand{\version}{draft}\||fi|
\end{center}
%
which can be uncommented to produce a draft version.
Likewise one can add a line to the very top of a child file
(above the |\childdocof{|\textit{main}|}| directive)
%
\begin{center}
|%\providecommand{\version}{final}|
\end{center}
%
which can be uncommented to produce the final version of this child document.

%%%%%%%%%%%%%%%%%%%%%%%%%%%%%%%%%%%%%%%%%%%%%%%%%%%%%%%%%%%%%%%%%%%%%%%%%%%%%%%%
\subsection{Forwarding}
\label{sec:forward}

Different versions of the main or child documents
using compilation flags as described in \secref{sec:flags}
can be (permanently) stored in different files
for convenient compilation, viewing and distribution.
To this end, the package defines a command
to pass on compilation to a different file:

%%%%%%%%%%%%%%%%%%%%%%%%%%%%%%%%%%%%%%%%
\DescribeMacro{\childdocforward}
The command |\childdocforward| redirects processing to
another source file:
%
\begin{center}
\begin{tabular}{l}
|\input{childdoc.def}|\\
|\childdocforward[|\textit{main}|]{|\textit{dest}|}|\\
\end{tabular}
\end{center}
%
The argument \textit{dest} is the destination file
(without extension).
It should be the main file or one of the child files.
Note that further \textsf{childdoc} directives
such as |\childdocof| and |\childdocforward|
in the indicated file will be processed in this form.
The optional argument \textit{main}
passes on directly to the main file \textit{main}
while pretending to compile the child \textit{dest}.
This form behaves as if \textit{dest}
issues |\childdocof{|\textit{main}|}| right away,
and no further \textsf{childdoc} directives will be processed.

%%%%%%%%%%%%%%%%%%%%%%%%%%%%%%%%%%%%%%%%
\DescribeMacro{\...prefix}
In the alternative form |\childdocforwardprefix|,
%
\begin{center}
\begin{tabular}{l}
|\input{childdoc.def}|\\
|\childdocforwardprefix[|\textit{main}|]{|\textit{prefix}|}{|\textit{dest}|}|
\end{tabular}
\end{center}
%
the destination file is determined by a pattern
depending on the current file:
To make this work, the current file must be called
`{\textit{prefix}\hspace{0.2em}\textit{suffix}}'
with \textit{prefix} matching precisely the argument.
Processing is then passed on to the file
`{\textit{dest}\hspace{0.2em}\textit{suffix}}'.
Surely, the same effect is achieved by
directly specifying the
argument `{\textit{dest}\hspace{0.2em}\textit{suffix}}'
in the first form.
However, that requires to set up a different file
for each child. With the alternative form of the command
all these files can have exactly the same content
which simplifies setting them up and maintaining them.

For example, the following file |draft.tex|
with a compilation flag |\version| as described in \secref{sec:flags}
compiles the main document as a draft:
%
\begin{center}
\begin{tabular}{l}
|\def\version{draft}|\\
|\input{childdoc.def}|\\
|\childdocforward{|\textit{main}|}|
\end{tabular}
\end{center}
%
Likewise, the following files |final|\textit{nn}|.tex|
compile the final version of the child document
|child|\textit{nn}|.tex|:
%
\begin{center}
\begin{tabular}{l}
|\def\version{final}|\\
|\input{childdoc.def}|\\
|\childdocforwardprefix{final}{child}|
\end{tabular}
\end{center}
%

Note that when several versions of a main file and/or of each child file
are to be generated, it may be convenient to set up a |Makefile| or
shell script to automatise the process.

%%%%%%%%%%%%%%%%%%%%%%%%%%%%%%%%%%%%%%%%%%%%%%%%%%%%%%%%%%%%%%%%%%%%%%%%%%%%%%%%
\subsection{Command Line Processing}
\label{sec:commandline}

The effect of redirection files can also be achieved by invoking
the \LaTeX{} compiler with a more elaborate command line.
Most conveniently this should be done as part
of a shell script or a |Makefile|.

When using \textsf{childdoc} in the main file, the following
command lines effectively perform a redirection
(note that depending on the shell being used,
backslashes may have to be doubled: `|\|' $\to$ `|\\|'):
%
\begin{center}
|... -jobname "|\textit{target}|" |\\|"|[\textit{flags}]%
|\input{childdoc.def}\childdocforward[|\textit{main}|]{|\textit{dest}|}"|
\end{center}
%
Here \textit{target} is the name of the output file,
\textit{main} is the name of the main file
and \textit{dest} is the name of the main or child file to be processed
(all filenames without extensions).
The optional argument \textit{main} can be omitted
if \textit{main} matches \textit{dest}.
Optionally, compilation \textit{flags} can be defined via |\def| commands.
This command line makes the \TeX{} engine believe
it is compiling the file \textit{target}
whose content is specified as the latter parameter.
The provided code then forwards the processing to
\textit{main} or \textit{dest} as described in \secref{sec:forward}.

%%%%%%%%%%%%%%%%%%%%%%%%%%%%%%%%%%%%%%%%%%%%%%%%%%%%%%%%%%%%%%%%%%%%%%%%%%%%%%%%
\subsection{Include by Input}
\label{sec:input}

Including child documents by |\include| has some restrictions by design.
Most notably, the content of a child document always occupies
its own set of pages; pages cannot be shared between child documents.
Usually, this behaviour makes perfect sense
because each child document contain an essential part of the document.
However, in some situations it may be desirable to compose
a document from a collection of parts
without having mandatory page breaks between then.
For this case, the package
provides a mechanism to include parts
by |\input| which can also be processed individually.
However, by construction this mechanism
requires manual handling of the content to be output.

%%%%%%%%%%%%%%%%%%%%%%%%%%%%%%%%%%%%%%%%
\DescribeMacro{\ifchilddocmanual}
The main file should be prepared as usual, see \secref{sec:include}.
However, the document body must make a distinction
between processing of an individual part and of the main document, e.g.:
%
\begin{center}
\begin{tabular}{l}
|\ifchilddocmanual|\\
|\input{\childdocname}|\\
|\||else|\\
\textit{document body with }|\input{|\textit{part}|}|\\
|\||fi|
\end{tabular}
\end{center}
%
The conditional |\ifchilddocmanual| is true whenever
a part to be included by |\input| is being compiled,
and the name of the part is stored in |\childdocname|.

%%%%%%%%%%%%%%%%%%%%%%%%%%%%%%%%%%%%%%%%
\DescribeMacro{\childdocby}
Each part to be included by |\input| should start with:
%
\begin{center}
\begin{tabular}{l}
|\input{childdoc.def}|\\
|\childdocby{|\textit{main}|}|\\
\end{tabular}
\end{center}
%
The directive |\childdocby| is similar to |\childdocof|
described in \secref{sec:include},
but the subsequent selection of content must be done manually.
To that end, both |\ifchilddoc| and |\ifchilddocmanual|
will be true upon processing of a part,
and the name of the part is stored in |\childdocname|.
Note that |\jobname| will be set to the filename of the current part
so that each part receives an individual |.aux| file
that does not interfere with the |.aux| file(s) of the main document.
This behaviour can be altered by the alternative form
|\childdocby[*]{|\textit{main}|}| (with a non-empty optional argument)
which uses the |.aux| file of the main document
by setting |\jobname| to \textit{main}.

%%%%%%%%%%%%%%%%%%%%%%%%%%%%%%%%%%%%%%%%%%%%%%%%%%%%%%%%%%%%%%%%%%%%%%%%%%%%%%%%
\subsection{Driver Development}
\label{sec:driver}

The \textsf{childdoc} mechanism can also be use for the development
of definition files such as \LaTeX{} styles or classes.
This case differs from the above setup with multiple parts
included by |\include| in that no |\includeonly| should be invoked.
This can be achieved by starting the include file
(before |\ProvidesPackage|) with:
%
\begin{center}
\begin{tabular}{l}
|\input{childdoc.def}|\\
|\childdocforward{|\textit{main}|}|\\
\end{tabular}
\end{center}
%
or alternatively with:
%
\begin{center}
\begin{tabular}{l}
|\input{childdoc.def}|\\
|\childdocby{|\textit{main}|}|\\
\end{tabular}
\end{center}
%
Both forms have slightly different effects as described above.
The main file is prepared as usual, see \secref{sec:include}.

%%%%%%%%%%%%%%%%%%%%%%%%%%%%%%%%%%%%%%%%%%%%%%%%%%%%%%%%%%%%%%%%%%%%%%%%%%%%%%%%
\subsection{Legacy Detection}
\label{sec:detection}

The directive |\childdocmain| in the main file can detect
whether the complete document or merely a child is to be compiled
even without using the directive |\childdocof|.
This method is deprecated because it is less robust
and there is no compelling reason to use it;
it is merely provided for backward compatibility
and it may be removed in future versions.

If the detection mechanism is to be used,
it is mandatory to correctly specify
the filename of the main file as the argument of |\childdocmain|:
%
\begin{center}
\begin{tabular}{l}
|\input{childdoc.def}|\\
|\childdocmain{|\textit{main}|}|\\
\end{tabular}
\end{center}
%
If |\jobname| does not match the argument \textit{main} of |\childdocmain|,
it is assumed that |\jobname| points to the child file to be compiled.
When using |\childdocmain| with the main file specified as argument,
it suffices to start a child file
with just |\input{|\textit{main}|}|
without loading of the package and using |\childdocof|.
If instead all processing is done
with the appropriate \textsf{childdoc} directives,
the argument of \textit{main} of |\childdocmain| can be empty.

An alternative version of the command line processing described
in \secref{sec:commandline} using the detection mechanism reads:
%
\begin{center}
|... -jobname "|\textit{target}|" "|[\textit{flags}]%
[|\def\jobname{|\textit{dest}|}|]|\input{|\textit{main}|}"|
\end{center}

%%%%%%%%%%%%%%%%%%%%%%%%%%%%%%%%%%%%%%%%%%%%%%%%%%%%%%%%%%%%%%%%%%%%%%%%%%%%%%%%
\subsection{Manual Code}
\label{sec:manual}

In case one cannot be certain whether the definitions file |childdoc.def|
is installed on the target \TeX{} distribution
and one prefers not to ship it,
it is conceivable to paste a few relevant commands into the sources.

To that end, drop all statements |\input{childdoc.def}|
and perform the replacements as outlined below.
Instead of |\childdocmain{|\textit{main}|}| add the following code
to the top of the main file:
%
\begin{center}
\begin{tabular}{l}
|\||ifdefined\childdocname\endinput\||fi\newif\ifchilddoc|\\
|\edef\childdocname{\scantokens\expandafter{\jobname\noexpand}}|\\
|\def\childdocmain{|\textit{main}|}\||ifx\childdocmain\childdocname\||else|\\
|\childdoctrue\includeonly{\childdocname}\let\jobname\childdocmain\||fi|\\
\end{tabular}
\end{center}
%
Instead of |\childdocof{|\textit{main}|}| just include the main file
at the top of each child file:
%
\begin{center}
|\input{|\textit{main}|}|
\end{center}
%
A simple redirection |\childdocforward{|\textit{dest}|}| is achieved by:
%
\begin{center}
|\def\jobname{|\textit{dest}|}\input{\jobname}|
\end{center}
%
The redirection with prefix
|\childdocforwardprefix[|\textit{prefix}|]{|\textit{dest}|}|
is accomplished by:
%
\begin{center}
\begin{tabular}{l}
|{\edef\jobname{\scantokens\expandafter{\jobname\noexpand}}|\\
|\def\redirectjob |\textit{prefix}|#1~~~{\gdef\jobname{|\textit{dest}|#1}}|\\
|\expandafter\redirectjob\jobname~~~}\input{\jobname}|
\end{tabular}
\end{center}

In an alternative approach,
child documents can be compiled by a specific command line
without additional code or specific definitions:
%
\begin{center}
|... -jobname "|\textit{target}|" "|[\textit{flags}]%
|\includeonly{|\textit{dest}|}\input{|\textit{main}|}"|
\end{center}
%

%%%%%%%%%%%%%%%%%%%%%%%%%%%%%%%%%%%%%%%%%%%%%%%%%%%%%%%%%%%%%%%%%%%%%%%%%%%%%%%%
%%%%%%%%%%%%%%%%%%%%%%%%%%%%%%%%%%%%%%%%%%%%%%%%%%%%%%%%%%%%%%%%%%%%%%%%%%%%%%%%
\section{Information}

%%%%%%%%%%%%%%%%%%%%%%%%%%%%%%%%%%%%%%%%%%%%%%%%%%%%%%%%%%%%%%%%%%%%%%%%%%%%%%%%
\subsection{Copyright}

Copyright \copyright{} 2017--2018 Niklas Beisert

This work may be distributed and/or modified under the
conditions of the \LaTeX{} Project Public License, either version 1.3
of this license or (at your option) any later version.
The latest version of this license is in
  \url{http://www.latex-project.org/lppl.txt}
and version 1.3 or later is part of all distributions of \LaTeX{}
version 2005/12/01 or later.

This work has the LPPL maintenance status `maintained'.

The Current Maintainer of this work is Niklas Beisert.

This work consists of the files |README.txt|, |childdoc.ins| and |childdoc.dtx|
as well as the derived files |childdoc.def|, |cdocsamp.tex|
with |cdocsch1.tex|, |cdocsch2.tex|, |cdocspt3.tex|, |cdocspt4.tex|,
|cdocsdrf.tex|, |cdocsfn1.tex|, |cdocsfn2.tex|
as well as |childdoc.pdf|.

%%%%%%%%%%%%%%%%%%%%%%%%%%%%%%%%%%%%%%%%%%%%%%%%%%%%%%%%%%%%%%%%%%%%%%%%%%%%%%%%
\subsection{Files and Installation}

The package consists of the files:
%
\begin{center}
\begin{tabular}{ll}
    |README.txt|   & readme file \\
    |childdoc.ins| & installation file \\
    |childdoc.dtx| & source file \\
    |childdoc.def| & definition file \\
    |cdocsamp.tex| & sample main file \\
    |cdocsch1.tex| & sample include file \\
    |cdocsch2.tex| & sample include file \\
    |cdocspt3.tex| & sample part file \\
    |cdocspt4.tex| & sample part file \\
    |cdocsdrf.tex| & sample redirection file \\
    |cdocsfn1.tex| & sample redirection file \\
    |cdocsfn2.tex| & sample redirection file \\
    |childdoc.pdf| & manual
\end{tabular}
\end{center}
%
The distribution consists of the files
|README.txt|, |childdoc.ins| and |childdoc.dtx|.
%
\begin{itemize}
\item
Run (pdf)\LaTeX{} on |childdoc.dtx|
to compile the manual |childdoc.pdf| (this file).
\item
Run \LaTeX{} on |childdoc.ins| to create the definitions file |childdoc.def|
and the sample |cdocsamp.tex| with include files
|cdocsch1.tex|, |cdocsch2.tex|, |cdocspt3.tex|, |cdocspt4.tex|,
|cdocsdrf.tex|, |cdocsfn1.tex|, |cdocsfn2.tex|.
Then copy the file |childdoc.def| to an appropriate directory of your \LaTeX{}
distribution, e.g.\ \textit{texmf-root}|/tex/latex/childdoc|.
\end{itemize}

%%%%%%%%%%%%%%%%%%%%%%%%%%%%%%%%%%%%%%%%%%%%%%%%%%%%%%%%%%%%%%%%%%%%%%%%%%%%%%%%
\subsection{Related CTAN Packages}

There are several other packages which offer a similar functionality:
%
\begin{itemize}
\item
The packages
\href{http://ctan.org/pkg/docmute}{\textsf{docmute}},
\href{http://ctan.org/pkg/includex}{\textsf{includex}} and
\href{http://ctan.org/pkg/standalone}{\textsf{standalone}}
provide commands to include only the document body of
a child file thus allowing both files to be compiled individually.
\item
The packages \href{http://ctan.org/pkg/subdocs}{\textsf{subdocs}}
and \href{http://ctan.org/pkg/subfiles}{\textsf{subfiles}}
provide structures in which the main and child documents can be
encapsulated and allowing them to be compiled individually.
The inclusion mechanism is different from the conventional |\include|.
\item
The package \href{http://ctan.org/pkg/combine}{\textsf{combine}}
is an elaborate solution to combine several documents into one.
\end{itemize}
%
See also the CTAN topic \href{http://ctan.org/topic/subdocs}{\textsf{subdocs}}
for further related packages.
The present package differs from the above solutions in that
a document structure constructed with the conventional |\include| mechanism
just needs two extra commands at the top of every file
such that all constituent files can be compiled individually.

%%%%%%%%%%%%%%%%%%%%%%%%%%%%%%%%%%%%%%%%%%%%%%%%%%%%%%%%%%%%%%%%%%%%%%%%%%%%%%%%
%\subsection{Feature Suggestions}
%
%The following is a list of features which may be useful for future
%versions of this package:
%%
%\begin{itemize}
%\item
%\ldots
%\end{itemize}

%%%%%%%%%%%%%%%%%%%%%%%%%%%%%%%%%%%%%%%%%%%%%%%%%%%%%%%%%%%%%%%%%%%%%%%%%%%%%%%%
\subsection{Revision History}

%%%%%%%%%%%%%%%%%%%%%%%%%%%%%%%%%%%%%%%%
\paragraph{v2.0:} 2018/12/30

\begin{itemize}
\item
immediate forward processing
\item
added |\childdocby| mechanism
\item
manual restructured
\end{itemize}

%%%%%%%%%%%%%%%%%%%%%%%%%%%%%%%%%%%%%%%%
\paragraph{v1.6:} 2018/01/17

\begin{itemize}
\item
application for development of include files
\item
corrections to manual
\end{itemize}

%%%%%%%%%%%%%%%%%%%%%%%%%%%%%%%%%%%%%%%%
\paragraph{v1.5:} 2017/05/21

\begin{itemize}
\item
more complete structuring introduced
\item
|\childdocof| introduced
\item
|\childdoc| renamed to |\childdocmain|
\item
|\childredirect| renamed to |\childdocforward| and |\childdocforwardprefix|
and functionality expanded
\end{itemize}

%%%%%%%%%%%%%%%%%%%%%%%%%%%%%%%%%%%%%%%%
\paragraph{v1.0:} 2017/04/27

\begin{itemize}
\item
manual and install package
\item
first version published on CTAN
\end{itemize}

%%%%%%%%%%%%%%%%%%%%%%%%%%%%%%%%%%%%%%%%
\paragraph{v0.6:} 2017/04/26

\begin{itemize}
\item
redirection mechanism added
\end{itemize}

%%%%%%%%%%%%%%%%%%%%%%%%%%%%%%%%%%%%%%%%
\paragraph{v0.5:} 2017/04/26

\begin{itemize}
\item
functionality in definition file
\end{itemize}


%%%%%%%%%%%%%%%%%%%%%%%%%%%%%%%%%%%%%%%%%%%%%%%%%%%%%%%%%%%%%%%%%%%%%%%%%%%%%%%%
%%%%%%%%%%%%%%%%%%%%%%%%%%%%%%%%%%%%%%%%%%%%%%%%%%%%%%%%%%%%%%%%%%%%%%%%%%%%%%%%
%%%%%%%%%%%%%%%%%%%%%%%%%%%%%%%%%%%%%%%%%%%%%%%%%%%%%%%%%%%%%%%%%%%%%%%%%%%%%%%%
\appendix

\settowidth\MacroIndent{\rmfamily\scriptsize 000\ }

 \DocInput{childdoc.dtx}

\end{document}
%</driver>
% \fi
%
% %%%%%%%%%%%%%%%%%%%%%%%%%%%%%%%%%%%%%%%%%%%%%%%%%%%%%%%%%%%%%%%%%%%%%%%%%%%%%%
% %%%%%%%%%%%%%%%%%%%%%%%%%%%%%%%%%%%%%%%%%%%%%%%%%%%%%%%%%%%%%%%%%%%%%%%%%%%%%%
% \section{Sample}
%\iffalse
%<*samplemain>
%\fi
%
% The following presents a sample document
% with two chapters, two parts, a title page,
% a compile flag as well as three forwarding files to set the flag.
% It consists of eight |.tex| files:
% \begin{center}
% \begin{tabular}{ll}
% |cdocsamp.tex|&main file\\
% |cdocsch1.tex|&include file for chapter 1\\
% |cdocsch2.tex|&include file for chapter 2\\
% |cdocspt3.tex|&include file for part 3\\
% |cdocspt4.tex|&include file for part 4\\
% |cdocsdrf.tex|&forwarding file for main file in draft mode\\
% |cdocsfi1.tex|&forwarding file for final version of chapter 1\\
% |cdocsfi2.tex|&forwarding file for final version of chapter 2\\
% \end{tabular}
% \end{center}
% Each of the eight files can be compiled directly by the \LaTeX{} compiler.
%
% %%%%%%%%%%%%%%%%%%%%%%%%%%%%%%%%%%%%%%
% \paragraph{Main File.}
%
% The main file is called |cdocsamp.tex|.
%
% Load the \textsf{childdoc} definitions and
% declare the filename for the main document:
%    \begin{macrocode}
\input{childdoc.def}
\childdocmain{}
%    \end{macrocode}

% Optional override for |\version| flag:
%    \begin{macrocode}
%%\ifchilddoc\else\providecommand{\version}{draft}\fi
%    \end{macrocode}

% Define the default values for the |\version| flag
% (|final| for the main file and |draft| for childs):
%    \begin{macrocode}
\ifchilddoc
\providecommand{\version}{draft}
\else
\providecommand{\version}{final}
\fi
%    \end{macrocode}

% Load the standard document class:
%    \begin{macrocode}
\documentclass[12pt]{article}
%    \end{macrocode}

% Start the document body:
%    \begin{macrocode}
\begin{document}
%    \end{macrocode}

% Declare a title page.
% Print title, part of document being processed and version flag:
%    \begin{macrocode}
\addtocounter{page}{-1}
\begin{center}
{\LARGE\bfseries{}childdoc example\par}
\vspace{1cm}
\ifchilddoc
\ifchilddocmanual part\else chapter\fi:
`\childdocname' of `\childdocjob'\par
\else
main document: `\childdocjob'\par
\fi
version: \version\par
\end{center}
\newpage
%    \end{macrocode}

% Manually include selected file,
% otherwise process as usual:
%    \begin{macrocode}
\ifchilddocmanual
\section*{part `\childdocname'}
\input{\childdocname}
\else
%    \end{macrocode}

% Include the two chapters:
%    \begin{macrocode}
\include{cdocsch1}
\include{cdocsch2}
%    \end{macrocode}

% Include the two parts unless only chapters should be displayed:
%    \begin{macrocode}
\ifchilddoc\else
\section{part three}
\input{cdocspt3}
\section{part four}
\input{cdocspt4}
\fi
%    \end{macrocode}

% Process as usual until here:
%    \begin{macrocode}
\fi
%    \end{macrocode}

% End of document body:
%    \begin{macrocode}
\end{document}
%    \end{macrocode}
%\iffalse
%</samplemain>
%\fi
%
% %%%%%%%%%%%%%%%%%%%%%%%%%%%%%%%%%%%%%%
% \paragraph{Chapter Include Files.}
%
% The include files are called |cdocsch1.tex| and |cdocsch2.tex|.
%
%\iffalse
%<*samplechap1|samplechap2>
%\fi

% Optional override for |\version| flag:
%    \begin{macrocode}
%%\providecommand{\version}{final}
%    \end{macrocode}

% Include the main document:
%    \begin{macrocode}
\input{childdoc.def}
\childdocof{cdocsamp}
%    \end{macrocode}

%\iffalse
%</samplechap1|samplechap2>
%\fi
%
%\iffalse
%<*samplechap1>
%\fi
% Some text for chapter 1:
%    \begin{macrocode}
\section{one}
some text in chapter one
%    \end{macrocode}

%\iffalse
%</samplechap1>
%\fi
% Some text for chapter 2:
%\iffalse
%<*samplechap2>
%\fi
%    \begin{macrocode}
\section{two}
more text in chapter two
%    \end{macrocode}

%\iffalse
%</samplechap2>
%\fi
%
% %%%%%%%%%%%%%%%%%%%%%%%%%%%%%%%%%%%%%%
% \paragraph{Part Include Files.}
%
% The include files are called |cdocspt3.tex| and |cdocspt4.tex|.
%
%\iffalse
%<*samplepart3|samplepart4>
%\fi

% Optional override for |\version| flag:
%    \begin{macrocode}
%%\providecommand{\version}{final}
%    \end{macrocode}

% Include the main document:
%    \begin{macrocode}
\input{childdoc.def}
\childdocby{cdocsamp}
%    \end{macrocode}

%\iffalse
%</samplepart3|samplepart4>
%\fi
%
%\iffalse
%<*samplepart3>
%\fi
% Some text for part 3:
%    \begin{macrocode}
some text in part three
%    \end{macrocode}

%\iffalse
%</samplepart3>
%\fi
% Some text for part 4:
%\iffalse
%<*samplepart4>
%\fi
%    \begin{macrocode}
more text in part four
%    \end{macrocode}

%\iffalse
%</samplepart4>
%\fi
%
% %%%%%%%%%%%%%%%%%%%%%%%%%%%%%%%%%%%%%%
% \paragraph{Forwarding for a Complete Draft.}
%
% The following forwarding file |cdocsdrf.tex|
% compiles the main document in draft mode:
%\iffalse
%<*sampledraft>
%\fi
%    \begin{macrocode}
\def\version{draft}
\input{childdoc.def}
\childdocforward{cdocsamp}
%    \end{macrocode}

%\iffalse
%</sampledraft>
%\fi
%
% %%%%%%%%%%%%%%%%%%%%%%%%%%%%%%%%%%%%%%
% \paragraph{Forwarding for Final Version of the Chapters.}
%
% The following forwarding files |cdocsfn1.tex| and |cdocsfn2.tex|
% (with identical content)
% compile the final versions of the child documents
% |cdocsch1.tex| and |cdocsch2.tex|, respectively:
%\iffalse
%<*samplefinal>
%\fi
%    \begin{macrocode}
\def\version{final}
\input{childdoc.def}
\childdocforwardprefix[cdocsamp]{cdocsfn}{cdocsch}
%    \end{macrocode}

%\iffalse
%</samplefinal>
%\fi
%
% %%%%%%%%%%%%%%%%%%%%%%%%%%%%%%%%%%%%%%
% \paragraph{Command Line Processing.}
%
% The following three command lines generate the output files
% |cdocscld|, |cdocscl1| and |cdocscl2|
% which should be identical to
% |cdocsdrf|, |cdocsch1| and |cdocsfn2|, respectively:
% \begin{center}
% \begin{tabular}{l}
% |latex -jobname cdocscld \|\\
% |  "\def\version{draft}\input{childdoc.def}\childdocforward{cdocsamp}"|\\
% |latex -jobname cdocscl1 \|\\
% |  "\input{childdoc.def}\childdocforward[cdocsamp]{cdocsch1}"|\\
% |latex -jobname cdocscl2 \|\\
% |  "\def\version{final}\input{childdoc.def}\childdocforward{cdocsch2}"|
% \end{tabular}
% \end{center}
% Note that the trailing backslash on each first line
% merely continues the input to the second line
% (for convenient cut ant paste).
% Furthermore, the command |latex| can be replaced by any
% of its alternative versions such as |pdflatex|.
%
% %%%%%%%%%%%%%%%%%%%%%%%%%%%%%%%%%%%%%%%%%%%%%%%%%%%%%%%%%%%%%%%%%%%%%%%%%%%%%%
% %%%%%%%%%%%%%%%%%%%%%%%%%%%%%%%%%%%%%%%%%%%%%%%%%%%%%%%%%%%%%%%%%%%%%%%%%%%%%%
% \section{Implementation}
%\iffalse
%<*package>
%\fi
%
% This section describes the definitions file |childdoc.def|.

% The definitions cannot be loaded using |\usepackage| or |\RequirePackage|
% which has a mechanism to prevent loading a style file more than once.
% When loading the definitions by means of |\input|
% multiple instances have to be prevented manually:
%\iffalse
%This code needs to be before the `\ProvidesFile' directive
%which is defined at the beginning of this file.
%Therefore it is also placed there and commented out here.
%</package>
%<*discard>
%\fi
%    \begin{macrocode}
\ifdefined\childdocmain\endinput\fi
%    \end{macrocode}
%\iffalse
%</discard>
%<*package>
%\fi
%
% \macro{\ifchilddoc}
% \macro{\ifchilddocmanual}
% The conditional |\ifchilddoc| tells whether a
% child (true) or main (false) document is being compiled.
% The conditional |\ifchilddocmanual| tells whether
% the |\includeonly| mechanism is used (false) or
% the selection of child files must be performed manually (true).
% The definitions initialise to false:
%    \begin{macrocode}
\newif\ifchilddoc
\newif\ifchilddocmanual
%    \end{macrocode}

% \macro{\childdocname}
% \macro{\childdocjob}
% The macro |\childdocname| stores the name of the main document
% to be compiled. The macro |\childdocjob| stores the name of
% the document on which the \LaTeX{} compiler was originally invoked.
% The content of |\jobname| cannot be compared
% to filenames specified in the source due to different catcodes.
% The following code rescans |\jobname|, stores the result
% in |\childdocname| and saves a copy in |\childdocjob|:
%    \begin{macrocode}
\edef\childdocname{\scantokens\expandafter{\jobname\noexpand}}
\let\childdocjob\childdocname
%    \end{macrocode}

% \macro{\childdocdisable}
% The macro |\childdocdisable| prevents the main file
% from being processed more than once.
% At this stage, the main document command |\childdocmain|
% is assumed to be called once again where it should do nothing.
% Any subsequent call to it should prevent
% a secondary processing of the main document
% It overwrites the forwarding commands
% |\childdocof| and |\childdocforward|
% with empty macros to prevent further inclusions of the main document:
%    \begin{macrocode}
\newcommand{\childdocdisable}
{
  \renewcommand{\childdocmain}[1]{\renewcommand{\childdocmain}[1]{\endinput}}
  \renewcommand{\childdocof}[1]{}
  \renewcommand{\childdocby}[2][]{}
  \renewcommand{\childdocforward}[2][]{}
  \renewcommand{\childdocdisable}{}
}
%    \end{macrocode}

% \macro{\childdocmain}
% The macro |\childdocmain| is to be called at the top of the main file
% with nothing or the main filename (without extension) as argument.
% First, it breaks loops.
% If the argument is not empty and does not match |\childdocname|
% (which is set by the first inclusion of |childdoc.def|),
% |\ifchilddoc| is set to true, |\includeonly| is applied to the child file
% and |\jobname| is set to the main file
% (for proper handling of |.aux| files):
%    \begin{macrocode}
\newcommand{\childdocmain}[1]
{
  \childdocdisable\childdocmain{}
  \if?#1?\else
    \begingroup
      \def\childdoctmp{#1}
      \ifx\childdoctmp\childdocname
        \def\childdoctmp{}
      \else
        \def\childdoctmp
        {
          \childdoctrue
          \includeonly{\childdocname}
          \def\childdocjob{#1}
          \def\jobname{#1}
        }
      \fi
      \expandafter
    \endgroup
    \childdoctmp
  \fi
}
%    \end{macrocode}

% \macro{\childdocof}
% The command |\childdocof| redirects
% compilation to the main file |#1|.
%    \begin{macrocode}
\newcommand{\childdocof}[1]
{
  \childdocdisable
  \childdoctrue
  \includeonly{\childdocname}
  \def\jobname{#1}
  \def\childdocjob{#1}
  \input{#1}
}
%    \end{macrocode}

% \macro{\childdocby}
% The command |\childdocby| ....
%    \begin{macrocode}
\newcommand{\childdocby}[2][]
{
  \childdocdisable
  \childdoctrue
  \childdocmanualtrue
  \if?#1?\else
    \def\jobname{#2}
  \fi
  \def\childdocjob{#2}
  \input{#2}
  \endinput
}
%    \end{macrocode}

% \macro{\childdocforward}
% The command |\childdocforward| redirects
% compilation to the main file or
% (if the optional argument is given) a child file.
% Parameters are set as if the main file
% or a child file starting with |\childdocof| was compiled.
% Then compilation is handed over to the main file:
%    \begin{macrocode}
\newcommand{\childdocforward}[2][]
{
  \begingroup
    \if?#1?
      \def\childdoctmp
      {
        \def\childdocname{#2}
        \def\childdocjob{#2}
        \def\jobname{#2}
        \input{#2}
        \endinput
      }
    \else
      \def\childdoctmp
      {
        \childdocdisable
        \def\childdocname{#2}
        \childdoctrue
        \includeonly{#2}
        \def\childdocjob{#1}
        \def\jobname{#1}
        \input{#1}
        \endinput
      }
    \fi
    \expandafter
  \endgroup
  \childdoctmp
}
%    \end{macrocode}

% \macro{\childdocforwardprefix}
% The command |\childdocforwardprefix| redirects
% compilation to the main or a child file by means of a pattern.
% The prefix |#1| in the current filename is replaced by |#2|
% and the suffix of the current filename is kept
% (it is assumed that the filename does not contain the substring `|~~~|'
% which is used as a delimiter).
% Compilation is handed over to the new file by |\childdocforward|:
%    \begin{macrocode}
\newcommand{\childdocforwardprefix}[3][]
{
  \begingroup
    \def\childdocextract #2##1~~~{\def\childdoctmp{\childdocforward[#1]{#3##1}}}
    \expandafter\childdocextract\childdocname~~~
    \expandafter
  \endgroup
  \childdoctmp
}
%    \end{macrocode}

% \macro{\childdoc}
% The deprecated macro |\childdoc| is a legacy version of |\childdocmain|:
%    \begin{macrocode}
\newcommand{\childdoc}{\childdocmain}
%    \end{macrocode}

% \macro{\childdocredirect}
% The deprecated macro |\childdocredirect| is a legacy version
% of |\childdocforward| and |\childdocforwardprefix|:
%    \begin{macrocode}
\newcommand{\childdocredirect}[2][]
{
  \begingroup
    \if?#1?
      \def\childdoctmp{\childdocforward{#2}}
    \else
      \def\childdoctmp{\childdocforwardprefix{#1}{#2}}
    \fi
    \expandafter
  \endgroup
  \childdoctmp
}
%    \end{macrocode}

%\iffalse
%</package>
%\fi
%
\endinput

\childdocmain{}
%    \end{macrocode}

% Optional override for |\version| flag:
%    \begin{macrocode}
%%\ifchilddoc\else\providecommand{\version}{draft}\fi
%    \end{macrocode}

% Define the default values for the |\version| flag
% (|final| for the main file and |draft| for childs):
%    \begin{macrocode}
\ifchilddoc
\providecommand{\version}{draft}
\else
\providecommand{\version}{final}
\fi
%    \end{macrocode}

% Load the standard document class:
%    \begin{macrocode}
\documentclass[12pt]{article}
%    \end{macrocode}

% Start the document body:
%    \begin{macrocode}
\begin{document}
%    \end{macrocode}

% Declare a title page.
% Print title, part of document being processed and version flag:
%    \begin{macrocode}
\addtocounter{page}{-1}
\begin{center}
{\LARGE\bfseries{}childdoc example\par}
\vspace{1cm}
\ifchilddoc
\ifchilddocmanual part\else chapter\fi:
`\childdocname' of `\childdocjob'\par
\else
main document: `\childdocjob'\par
\fi
version: \version\par
\end{center}
\newpage
%    \end{macrocode}

% Manually include selected file,
% otherwise process as usual:
%    \begin{macrocode}
\ifchilddocmanual
\section*{part `\childdocname'}
\input{\childdocname}
\else
%    \end{macrocode}

% Include the two chapters:
%    \begin{macrocode}
\include{cdocsch1}
\include{cdocsch2}
%    \end{macrocode}

% Include the two parts unless only chapters should be displayed:
%    \begin{macrocode}
\ifchilddoc\else
\section{part three}
\input{cdocspt3}
\section{part four}
\input{cdocspt4}
\fi
%    \end{macrocode}

% Process as usual until here:
%    \begin{macrocode}
\fi
%    \end{macrocode}

% End of document body:
%    \begin{macrocode}
\end{document}
%    \end{macrocode}
%\iffalse
%</samplemain>
%\fi
%
% %%%%%%%%%%%%%%%%%%%%%%%%%%%%%%%%%%%%%%
% \paragraph{Chapter Include Files.}
%
% The include files are called |cdocsch1.tex| and |cdocsch2.tex|.
%
%\iffalse
%<*samplechap1|samplechap2>
%\fi

% Optional override for |\version| flag:
%    \begin{macrocode}
%%\providecommand{\version}{final}
%    \end{macrocode}

% Include the main document:
%    \begin{macrocode}
% \iffalse
%
% childdoc.dtx Copyright (C) 2017-2018 Niklas Beisert
%
% This work may be distributed and/or modified under the
% conditions of the LaTeX Project Public License, either version 1.3
% of this license or (at your option) any later version.
% The latest version of this license is in
%   http://www.latex-project.org/lppl.txt
% and version 1.3 or later is part of all distributions of LaTeX
% version 2005/12/01 or later.
%
% This work has the LPPL maintenance status `maintained'.
%
% The Current Maintainer of this work is Niklas Beisert.
%
% This work consists of the files childdoc.dtx and childdoc.ins
% and the derived files childdoc.def and cdocsamp.tex with
% cdocsch1.tex, cdocsch2.tex, cdocsdrf.tex, cdocsfn1.tex, cdocsfn2.tex.
%
%<package>\ifdefined\childdocmain\endinput\fi
%<package>\ProvidesFile{childdoc.def}[2018/12/30 v2.0 child document driver]
%<samplemain>\ProvidesFile{cdocsamp.tex}[2018/12/30 v2.0 sample for childdoc]
%<*driver>
%\ProvidesFile{childdoc.drv}[2018/12/30 v2.0 childdoc reference manual file]
\PassOptionsToClass{10pt,a4paper}{article}
\documentclass{ltxdoc}

\usepackage[margin=35mm]{geometry}
\usepackage{hyperref}
\usepackage{hyperxmp}
\usepackage[usenames]{color}

\hypersetup{colorlinks=true}
\hypersetup{pdfstartview=FitH}
\hypersetup{pdfpagemode=UseNone}
\hypersetup{pdfsource={}}
\hypersetup{pdflang={en-UK}}
\hypersetup{pdfcopyright={Copyright 2017-2018 Niklas Beisert.
  This work may be distributed and/or modified under the
  conditions of the LaTeX Project Public License, either version 1.3
  of this license or (at your option) any later version.}}
\hypersetup{pdflicenseurl={http://www.latex-project.org/lppl.txt}}
\hypersetup{pdfcontactaddress={ETH Zurich, ITP, HIT K,
  Wolfgang-Pauli-Strasse 27}}
\hypersetup{pdfcontactpostcode={8093}}
\hypersetup{pdfcontactcity={Zurich}}
\hypersetup{pdfcontactcountry={Switzerland}}
\hypersetup{pdfcontactemail={nbeisert@itp.phys.ethz.ch}}
\hypersetup{pdfcontacturl={http://people.phys.ethz.ch/\xmptilde nbeisert/}}

\newcommand{\secref}[1]{\hyperref[#1]{section \ref*{#1}}}

\parskip1ex
\parindent0pt
\let\olditemize\itemize
\def\itemize{\olditemize\parskip0pt}

\begin{document}

\title{The \textsf{childdoc} Package}
\hypersetup{pdftitle={The childdoc Package}}
\author{Niklas Beisert\\[2ex]
  Institut f\"ur Theoretische Physik\\
  Eidgen\"ossische Technische Hochschule Z\"urich\\
  Wolfgang-Pauli-Strasse 27, 8093 Z\"urich, Switzerland\\[1ex]
  \href{mailto:nbeisert@itp.phys.ethz.ch}
  {\texttt{nbeisert@itp.phys.ethz.ch}}}
\hypersetup{pdfauthor={Niklas Beisert}}
\hypersetup{pdfsubject={Manual for the LaTeX2e Package childdoc}}
\date{30 December 2018, \textsf{v2.0}}
\maketitle

\begin{abstract}\noindent
\textsf{childdoc} is a \LaTeXe{} package
that enables the direct compilation
of document sections included by |\include|
to individual files.
\end{abstract}

\begingroup
\parskip0ex
\tableofcontents
\endgroup

%%%%%%%%%%%%%%%%%%%%%%%%%%%%%%%%%%%%%%%%%%%%%%%%%%%%%%%%%%%%%%%%%%%%%%%%%%%%%%%%
%%%%%%%%%%%%%%%%%%%%%%%%%%%%%%%%%%%%%%%%%%%%%%%%%%%%%%%%%%%%%%%%%%%%%%%%%%%%%%%%
\section{Introduction}

\LaTeX{} provides a mechanism to structure a large document (such as a book)
into a main file and several child files (containing the chapters)
using the |\include| command.
This mechanism is beneficial for documents
which span hundreds of pages in order to
make the source file(s) more manageable.
Moreover, compilation can be restricted to
selected child files by means of the |\includeonly| command.
The latter feature can be used to reduce the compilation time while editing
(this was significantly more useful in the earlier days of \LaTeX{})
or to generate a smaller document which is easier to navigate.
Another application of |\includeonly| is to generate
documents consisting of selected parts of the complete document.

However, there are a few drawbacks of the plain |\include| mechanism:
\begin{itemize}
\item
The child files cannot be compiled on their own,
they can only be compiled via the main file.
A naive editing environment
(such as a text editor with an option
to have the current file processed by \LaTeX)
may require one to switch to the main file before compiling;
attempting to compile the child file produces errors.
\item
The main file must be modified (each time)
to adjust the |\includeonly| command
to the present needs. This easily leaves the main file in a messy state.
\item
The generated document will always carry the filename
of the main document. This is inconvenient if
several child files are to be compiled and
to be kept for distribution.
\end{itemize}

The present package provides a simple interface
to make child files individually compilable by \LaTeX{}.
Compiling a child file then has the same effect as compiling
the main file with an |\includeonly| command
to select the appropriate child.
Moreover the generated document will carry the name of the child
rather than the main file.
This resolves all three above issues.

This feature is meant to make the editing of books,
thesis documents and lecture notes somewhat more convenient.
However, the package can also be used efficiently for
composing a series of documents (such as exercise sheets)
which are typically distributed individually.
It then assists the author in generating the individual documents
(potentially in different versions)
as well as a document containing the collected series.
Another application is in developing style files
or other kinds of included material
where compilation of the style file could redirect
to a sample or test file.

%%%%%%%%%%%%%%%%%%%%%%%%%%%%%%%%%%%%%%%%%%%%%%%%%%%%%%%%%%%%%%%%%%%%%%%%%%%%%%%%
%%%%%%%%%%%%%%%%%%%%%%%%%%%%%%%%%%%%%%%%%%%%%%%%%%%%%%%%%%%%%%%%%%%%%%%%%%%%%%%%
\section{Usage}

First of all, the package \textsf{childdoc} is \emph{not} a standard
\LaTeXe{} |.sty| style file! Therefore it needs to be invoked in
a non-standard way.

%%%%%%%%%%%%%%%%%%%%%%%%%%%%%%%%%%%%%%%%%%%%%%%%%%%%%%%%%%%%%%%%%%%%%%%%%%%%%%%%
\subsection{Included Files}
\label{sec:include}

%%%%%%%%%%%%%%%%%%%%%%%%%%%%%%%%%%%%%%%%
\DescribeMacro{\childdocmain}
To use the package, add the commands
\begin{center}
\begin{tabular}{l}
|\input{childdoc.def}|\\
|\childdocmain{}|\\
\end{tabular}
\end{center}
at the very top of the main \LaTeX{} file,
in particular \emph{before} the |\documentclass| statement!
The argument of |\childdocmain| should be left empty
(but it must be present).

%%%%%%%%%%%%%%%%%%%%%%%%%%%%%%%%%%%%%%%%
\DescribeMacro{\childdocof}
Furthermore, add the commands
\begin{center}
\begin{tabular}{l}
|\input{childdoc.def}|\\
|\childdocof{|\textit{main}|}|\\
\end{tabular}
\end{center}
at the top of every child file \textit{child}
which is included by |\include{|\textit{child}|}|
from within the main file
(or at least for those files to be compiled individually).
The argument \textit{main} must be the filename of the main file.

There are a couple of
considerations in setting up the main and child documents:

%%%%%%%%%%%%%%%%%%%%%%%%%%%%%%%%%%%%%%%%
\paragraph{Restrictions.}

Please note the following restrictions:
\begin{itemize}
\item
|\childdocmain| must be called with one argument \textit{main}
to ensure compatibility with earlier version of the package.
It must either be empty (|\childdocmain{}|)
or precisely match the filename of the main file in which it is specified.
See \secref{sec:detection} for further information.
\item
The filename \textit{main} must be specified without the |.tex| extension.
\item
The filename \textit{main} is case sensitive
(even in case-insensitive file systems)
due to internal string comparison.
\item
The argument \textit{main} should be fully expanded, it cannot be a macro.
\item
Subdirectories and special characters should be avoided in filenames.
\item
The command |\childdocmain{|\textit{main}|}| must be followed by a whitespace.
It should not be followed immediately by another command
or by a comment mark `|%|'.
This is because the \TeX{} parser reads the token immediately following
the argument of |\childdocmain| and puts it
at the beginning of every child section;
however, a white\-space is ignored.
\end{itemize}

%%%%%%%%%%%%%%%%%%%%%%%%%%%%%%%%%%%%%%%%
\paragraph{Content of Main File.}

It is advisable to place all content in the child files included by |\include|.
Any output contained in the main file will appear in all child documents
unless suppressed manually;
it cannot be suppressed automatically by the |\includeonly| directive
and thus should normally be avoided.
A method to include some content in the main file
by means of conditional processing is described in \secref{sec:conditional}.

%%%%%%%%%%%%%%%%%%%%%%%%%%%%%%%%%%%%%%%%
\paragraph{Page Numbering.}

When only a part of the document is compiled,
the appropriate numbering of pages
(as well as other status parameters)
is determined from the |.aux| files.
The latter contain information from previous passes.
However this information needs to propagate through
all intermediate child documents.
Therefore the page numbering in child documents may well
be inconsistent until the complete document is compiled at least once.

A useful (if unconventional) way to always ensure a consistent
page numbering is to restart the numbering in each child document
and denote the pages by `\textit{child}|.|\textit{page}'
where \textit{child} represents the chapter/section number of the child file.
This can be achieved by the command
|\numberwithin{page}{|\textit{child}|}|
of the \textsf{amsmath} package
where \textit{child} can be |chapter| or |section|
depending on the chosen structuring.
Alternatively, one can modify the macro |\thepage| appropriately
and reset the counter |page| at the start of each child file.

%%%%%%%%%%%%%%%%%%%%%%%%%%%%%%%%%%%%%%%%%%%%%%%%%%%%%%%%%%%%%%%%%%%%%%%%%%%%%%%%
\subsection{Conditional Processing}
\label{sec:conditional}

The package provides a mechanism to compile different versions
of a document. To customise the versions further some conditional processing
can come in handy to distinguish which version is being compiled.
The package provides two macros to describe the compilation context:

%%%%%%%%%%%%%%%%%%%%%%%%%%%%%%%%%%%%%%%%
\DescribeMacro{\ifchilddoc}
The conditional |\ifchilddoc| distinguishes between the compilation of
child documents and the main document:
%
\begin{center}
|\ifchilddoc |\textit{child-code}| |[|\||else |\textit{main-code}]| \||fi|
\end{center}

%%%%%%%%%%%%%%%%%%%%%%%%%%%%%%%%%%%%%%%%
\DescribeMacro{\childdocname}
\DescribeMacro{\childdocjob}
The macro |\childdocname| contains the filename (without extension)
of the main or child file being processed.
Note that |\childdocjob| will always contain the name of the main file.

%%%%%%%%%%%%%%%%%%%%%%%%%%%%%%%%%%%%%%%%
\paragraph{Title Page.}

Conditional processing can be used to include a title or banner page
in the main document when proper precautions are taken.
Importantly, the code in the main file should ensure that the page counter
(as well as other status parameters which are stored in the |.aux| files)
takes the same value after the conditional processing.
Otherwise the page numbers may take divergent values
depending on which part is compiled.

For example, a title page could be declared by:
%
\begin{center}
\begin{tabular}{l}
|\ifchilddoc\||else|\\
|\addtocounter{page}{-1}|\\
\textit{code for title page}\\
|\newpage|\\
|\||fi|
\end{tabular}
\end{center}
%
A banner page for the child documents can be generated by:
%
\begin{center}
\begin{tabular}{l}
|\ifchilddoc|\\
|\addtocounter{page}{-1}|\\
\textit{code for banner page}\\
|\newpage|\\
|\||fi|
\end{tabular}
\end{center}
%
Here one could write a message such as:
\begin{center}
|This is the part \childdocname{} of \childdocjob{}.|
\end{center}

%%%%%%%%%%%%%%%%%%%%%%%%%%%%%%%%%%%%%%%%%%%%%%%%%%%%%%%%%%%%%%%%%%%%%%%%%%%%%%%%
\subsection{Flags}
\label{sec:flags}

The package makes it easy to generate different versions
of the main or child documents.
To this end compilation flags can be defined
and assigned different default values.
They will be particularly useful in conjunction
with the forwarding mechanism described in \secref{sec:forward}.

For example, it may be useful to have a flag |\version|
which can be set to |draft| or |final|.
The document source will contain some conditional code
depending on the value of |\version|.
Suppose further, the flag should default to |final| for the main file
and to |draft| for child files
which is a natural assignment for editing the document.
This is achieved by placing the following code
in the preamble of the main document
(below the |\childdocmain| directive):
%
\begin{center}
\begin{tabular}{l}
|\ifchilddoc|\\
|\providecommand{\version}{draft}|\\
|\||else|\\
|\providecommand{\version}{final}|\\
|\||fi|
\end{tabular}
\end{center}
%
The definition by |\providecommand| makes sure
that previous definitions are not overwritten.
Further statements |\providecommand{\version}{...}|
can thus be added before the above code to override it.

For the main file, one might add a line
(between |\childdocmain| and the above block)
%
\begin{center}
|%\ifchilddoc\||else\providecommand{\version}{draft}\||fi|
\end{center}
%
which can be uncommented to produce a draft version.
Likewise one can add a line to the very top of a child file
(above the |\childdocof{|\textit{main}|}| directive)
%
\begin{center}
|%\providecommand{\version}{final}|
\end{center}
%
which can be uncommented to produce the final version of this child document.

%%%%%%%%%%%%%%%%%%%%%%%%%%%%%%%%%%%%%%%%%%%%%%%%%%%%%%%%%%%%%%%%%%%%%%%%%%%%%%%%
\subsection{Forwarding}
\label{sec:forward}

Different versions of the main or child documents
using compilation flags as described in \secref{sec:flags}
can be (permanently) stored in different files
for convenient compilation, viewing and distribution.
To this end, the package defines a command
to pass on compilation to a different file:

%%%%%%%%%%%%%%%%%%%%%%%%%%%%%%%%%%%%%%%%
\DescribeMacro{\childdocforward}
The command |\childdocforward| redirects processing to
another source file:
%
\begin{center}
\begin{tabular}{l}
|\input{childdoc.def}|\\
|\childdocforward[|\textit{main}|]{|\textit{dest}|}|\\
\end{tabular}
\end{center}
%
The argument \textit{dest} is the destination file
(without extension).
It should be the main file or one of the child files.
Note that further \textsf{childdoc} directives
such as |\childdocof| and |\childdocforward|
in the indicated file will be processed in this form.
The optional argument \textit{main}
passes on directly to the main file \textit{main}
while pretending to compile the child \textit{dest}.
This form behaves as if \textit{dest}
issues |\childdocof{|\textit{main}|}| right away,
and no further \textsf{childdoc} directives will be processed.

%%%%%%%%%%%%%%%%%%%%%%%%%%%%%%%%%%%%%%%%
\DescribeMacro{\...prefix}
In the alternative form |\childdocforwardprefix|,
%
\begin{center}
\begin{tabular}{l}
|\input{childdoc.def}|\\
|\childdocforwardprefix[|\textit{main}|]{|\textit{prefix}|}{|\textit{dest}|}|
\end{tabular}
\end{center}
%
the destination file is determined by a pattern
depending on the current file:
To make this work, the current file must be called
`{\textit{prefix}\hspace{0.2em}\textit{suffix}}'
with \textit{prefix} matching precisely the argument.
Processing is then passed on to the file
`{\textit{dest}\hspace{0.2em}\textit{suffix}}'.
Surely, the same effect is achieved by
directly specifying the
argument `{\textit{dest}\hspace{0.2em}\textit{suffix}}'
in the first form.
However, that requires to set up a different file
for each child. With the alternative form of the command
all these files can have exactly the same content
which simplifies setting them up and maintaining them.

For example, the following file |draft.tex|
with a compilation flag |\version| as described in \secref{sec:flags}
compiles the main document as a draft:
%
\begin{center}
\begin{tabular}{l}
|\def\version{draft}|\\
|\input{childdoc.def}|\\
|\childdocforward{|\textit{main}|}|
\end{tabular}
\end{center}
%
Likewise, the following files |final|\textit{nn}|.tex|
compile the final version of the child document
|child|\textit{nn}|.tex|:
%
\begin{center}
\begin{tabular}{l}
|\def\version{final}|\\
|\input{childdoc.def}|\\
|\childdocforwardprefix{final}{child}|
\end{tabular}
\end{center}
%

Note that when several versions of a main file and/or of each child file
are to be generated, it may be convenient to set up a |Makefile| or
shell script to automatise the process.

%%%%%%%%%%%%%%%%%%%%%%%%%%%%%%%%%%%%%%%%%%%%%%%%%%%%%%%%%%%%%%%%%%%%%%%%%%%%%%%%
\subsection{Command Line Processing}
\label{sec:commandline}

The effect of redirection files can also be achieved by invoking
the \LaTeX{} compiler with a more elaborate command line.
Most conveniently this should be done as part
of a shell script or a |Makefile|.

When using \textsf{childdoc} in the main file, the following
command lines effectively perform a redirection
(note that depending on the shell being used,
backslashes may have to be doubled: `|\|' $\to$ `|\\|'):
%
\begin{center}
|... -jobname "|\textit{target}|" |\\|"|[\textit{flags}]%
|\input{childdoc.def}\childdocforward[|\textit{main}|]{|\textit{dest}|}"|
\end{center}
%
Here \textit{target} is the name of the output file,
\textit{main} is the name of the main file
and \textit{dest} is the name of the main or child file to be processed
(all filenames without extensions).
The optional argument \textit{main} can be omitted
if \textit{main} matches \textit{dest}.
Optionally, compilation \textit{flags} can be defined via |\def| commands.
This command line makes the \TeX{} engine believe
it is compiling the file \textit{target}
whose content is specified as the latter parameter.
The provided code then forwards the processing to
\textit{main} or \textit{dest} as described in \secref{sec:forward}.

%%%%%%%%%%%%%%%%%%%%%%%%%%%%%%%%%%%%%%%%%%%%%%%%%%%%%%%%%%%%%%%%%%%%%%%%%%%%%%%%
\subsection{Include by Input}
\label{sec:input}

Including child documents by |\include| has some restrictions by design.
Most notably, the content of a child document always occupies
its own set of pages; pages cannot be shared between child documents.
Usually, this behaviour makes perfect sense
because each child document contain an essential part of the document.
However, in some situations it may be desirable to compose
a document from a collection of parts
without having mandatory page breaks between then.
For this case, the package
provides a mechanism to include parts
by |\input| which can also be processed individually.
However, by construction this mechanism
requires manual handling of the content to be output.

%%%%%%%%%%%%%%%%%%%%%%%%%%%%%%%%%%%%%%%%
\DescribeMacro{\ifchilddocmanual}
The main file should be prepared as usual, see \secref{sec:include}.
However, the document body must make a distinction
between processing of an individual part and of the main document, e.g.:
%
\begin{center}
\begin{tabular}{l}
|\ifchilddocmanual|\\
|\input{\childdocname}|\\
|\||else|\\
\textit{document body with }|\input{|\textit{part}|}|\\
|\||fi|
\end{tabular}
\end{center}
%
The conditional |\ifchilddocmanual| is true whenever
a part to be included by |\input| is being compiled,
and the name of the part is stored in |\childdocname|.

%%%%%%%%%%%%%%%%%%%%%%%%%%%%%%%%%%%%%%%%
\DescribeMacro{\childdocby}
Each part to be included by |\input| should start with:
%
\begin{center}
\begin{tabular}{l}
|\input{childdoc.def}|\\
|\childdocby{|\textit{main}|}|\\
\end{tabular}
\end{center}
%
The directive |\childdocby| is similar to |\childdocof|
described in \secref{sec:include},
but the subsequent selection of content must be done manually.
To that end, both |\ifchilddoc| and |\ifchilddocmanual|
will be true upon processing of a part,
and the name of the part is stored in |\childdocname|.
Note that |\jobname| will be set to the filename of the current part
so that each part receives an individual |.aux| file
that does not interfere with the |.aux| file(s) of the main document.
This behaviour can be altered by the alternative form
|\childdocby[*]{|\textit{main}|}| (with a non-empty optional argument)
which uses the |.aux| file of the main document
by setting |\jobname| to \textit{main}.

%%%%%%%%%%%%%%%%%%%%%%%%%%%%%%%%%%%%%%%%%%%%%%%%%%%%%%%%%%%%%%%%%%%%%%%%%%%%%%%%
\subsection{Driver Development}
\label{sec:driver}

The \textsf{childdoc} mechanism can also be use for the development
of definition files such as \LaTeX{} styles or classes.
This case differs from the above setup with multiple parts
included by |\include| in that no |\includeonly| should be invoked.
This can be achieved by starting the include file
(before |\ProvidesPackage|) with:
%
\begin{center}
\begin{tabular}{l}
|\input{childdoc.def}|\\
|\childdocforward{|\textit{main}|}|\\
\end{tabular}
\end{center}
%
or alternatively with:
%
\begin{center}
\begin{tabular}{l}
|\input{childdoc.def}|\\
|\childdocby{|\textit{main}|}|\\
\end{tabular}
\end{center}
%
Both forms have slightly different effects as described above.
The main file is prepared as usual, see \secref{sec:include}.

%%%%%%%%%%%%%%%%%%%%%%%%%%%%%%%%%%%%%%%%%%%%%%%%%%%%%%%%%%%%%%%%%%%%%%%%%%%%%%%%
\subsection{Legacy Detection}
\label{sec:detection}

The directive |\childdocmain| in the main file can detect
whether the complete document or merely a child is to be compiled
even without using the directive |\childdocof|.
This method is deprecated because it is less robust
and there is no compelling reason to use it;
it is merely provided for backward compatibility
and it may be removed in future versions.

If the detection mechanism is to be used,
it is mandatory to correctly specify
the filename of the main file as the argument of |\childdocmain|:
%
\begin{center}
\begin{tabular}{l}
|\input{childdoc.def}|\\
|\childdocmain{|\textit{main}|}|\\
\end{tabular}
\end{center}
%
If |\jobname| does not match the argument \textit{main} of |\childdocmain|,
it is assumed that |\jobname| points to the child file to be compiled.
When using |\childdocmain| with the main file specified as argument,
it suffices to start a child file
with just |\input{|\textit{main}|}|
without loading of the package and using |\childdocof|.
If instead all processing is done
with the appropriate \textsf{childdoc} directives,
the argument of \textit{main} of |\childdocmain| can be empty.

An alternative version of the command line processing described
in \secref{sec:commandline} using the detection mechanism reads:
%
\begin{center}
|... -jobname "|\textit{target}|" "|[\textit{flags}]%
[|\def\jobname{|\textit{dest}|}|]|\input{|\textit{main}|}"|
\end{center}

%%%%%%%%%%%%%%%%%%%%%%%%%%%%%%%%%%%%%%%%%%%%%%%%%%%%%%%%%%%%%%%%%%%%%%%%%%%%%%%%
\subsection{Manual Code}
\label{sec:manual}

In case one cannot be certain whether the definitions file |childdoc.def|
is installed on the target \TeX{} distribution
and one prefers not to ship it,
it is conceivable to paste a few relevant commands into the sources.

To that end, drop all statements |\input{childdoc.def}|
and perform the replacements as outlined below.
Instead of |\childdocmain{|\textit{main}|}| add the following code
to the top of the main file:
%
\begin{center}
\begin{tabular}{l}
|\||ifdefined\childdocname\endinput\||fi\newif\ifchilddoc|\\
|\edef\childdocname{\scantokens\expandafter{\jobname\noexpand}}|\\
|\def\childdocmain{|\textit{main}|}\||ifx\childdocmain\childdocname\||else|\\
|\childdoctrue\includeonly{\childdocname}\let\jobname\childdocmain\||fi|\\
\end{tabular}
\end{center}
%
Instead of |\childdocof{|\textit{main}|}| just include the main file
at the top of each child file:
%
\begin{center}
|\input{|\textit{main}|}|
\end{center}
%
A simple redirection |\childdocforward{|\textit{dest}|}| is achieved by:
%
\begin{center}
|\def\jobname{|\textit{dest}|}\input{\jobname}|
\end{center}
%
The redirection with prefix
|\childdocforwardprefix[|\textit{prefix}|]{|\textit{dest}|}|
is accomplished by:
%
\begin{center}
\begin{tabular}{l}
|{\edef\jobname{\scantokens\expandafter{\jobname\noexpand}}|\\
|\def\redirectjob |\textit{prefix}|#1~~~{\gdef\jobname{|\textit{dest}|#1}}|\\
|\expandafter\redirectjob\jobname~~~}\input{\jobname}|
\end{tabular}
\end{center}

In an alternative approach,
child documents can be compiled by a specific command line
without additional code or specific definitions:
%
\begin{center}
|... -jobname "|\textit{target}|" "|[\textit{flags}]%
|\includeonly{|\textit{dest}|}\input{|\textit{main}|}"|
\end{center}
%

%%%%%%%%%%%%%%%%%%%%%%%%%%%%%%%%%%%%%%%%%%%%%%%%%%%%%%%%%%%%%%%%%%%%%%%%%%%%%%%%
%%%%%%%%%%%%%%%%%%%%%%%%%%%%%%%%%%%%%%%%%%%%%%%%%%%%%%%%%%%%%%%%%%%%%%%%%%%%%%%%
\section{Information}

%%%%%%%%%%%%%%%%%%%%%%%%%%%%%%%%%%%%%%%%%%%%%%%%%%%%%%%%%%%%%%%%%%%%%%%%%%%%%%%%
\subsection{Copyright}

Copyright \copyright{} 2017--2018 Niklas Beisert

This work may be distributed and/or modified under the
conditions of the \LaTeX{} Project Public License, either version 1.3
of this license or (at your option) any later version.
The latest version of this license is in
  \url{http://www.latex-project.org/lppl.txt}
and version 1.3 or later is part of all distributions of \LaTeX{}
version 2005/12/01 or later.

This work has the LPPL maintenance status `maintained'.

The Current Maintainer of this work is Niklas Beisert.

This work consists of the files |README.txt|, |childdoc.ins| and |childdoc.dtx|
as well as the derived files |childdoc.def|, |cdocsamp.tex|
with |cdocsch1.tex|, |cdocsch2.tex|, |cdocspt3.tex|, |cdocspt4.tex|,
|cdocsdrf.tex|, |cdocsfn1.tex|, |cdocsfn2.tex|
as well as |childdoc.pdf|.

%%%%%%%%%%%%%%%%%%%%%%%%%%%%%%%%%%%%%%%%%%%%%%%%%%%%%%%%%%%%%%%%%%%%%%%%%%%%%%%%
\subsection{Files and Installation}

The package consists of the files:
%
\begin{center}
\begin{tabular}{ll}
    |README.txt|   & readme file \\
    |childdoc.ins| & installation file \\
    |childdoc.dtx| & source file \\
    |childdoc.def| & definition file \\
    |cdocsamp.tex| & sample main file \\
    |cdocsch1.tex| & sample include file \\
    |cdocsch2.tex| & sample include file \\
    |cdocspt3.tex| & sample part file \\
    |cdocspt4.tex| & sample part file \\
    |cdocsdrf.tex| & sample redirection file \\
    |cdocsfn1.tex| & sample redirection file \\
    |cdocsfn2.tex| & sample redirection file \\
    |childdoc.pdf| & manual
\end{tabular}
\end{center}
%
The distribution consists of the files
|README.txt|, |childdoc.ins| and |childdoc.dtx|.
%
\begin{itemize}
\item
Run (pdf)\LaTeX{} on |childdoc.dtx|
to compile the manual |childdoc.pdf| (this file).
\item
Run \LaTeX{} on |childdoc.ins| to create the definitions file |childdoc.def|
and the sample |cdocsamp.tex| with include files
|cdocsch1.tex|, |cdocsch2.tex|, |cdocspt3.tex|, |cdocspt4.tex|,
|cdocsdrf.tex|, |cdocsfn1.tex|, |cdocsfn2.tex|.
Then copy the file |childdoc.def| to an appropriate directory of your \LaTeX{}
distribution, e.g.\ \textit{texmf-root}|/tex/latex/childdoc|.
\end{itemize}

%%%%%%%%%%%%%%%%%%%%%%%%%%%%%%%%%%%%%%%%%%%%%%%%%%%%%%%%%%%%%%%%%%%%%%%%%%%%%%%%
\subsection{Related CTAN Packages}

There are several other packages which offer a similar functionality:
%
\begin{itemize}
\item
The packages
\href{http://ctan.org/pkg/docmute}{\textsf{docmute}},
\href{http://ctan.org/pkg/includex}{\textsf{includex}} and
\href{http://ctan.org/pkg/standalone}{\textsf{standalone}}
provide commands to include only the document body of
a child file thus allowing both files to be compiled individually.
\item
The packages \href{http://ctan.org/pkg/subdocs}{\textsf{subdocs}}
and \href{http://ctan.org/pkg/subfiles}{\textsf{subfiles}}
provide structures in which the main and child documents can be
encapsulated and allowing them to be compiled individually.
The inclusion mechanism is different from the conventional |\include|.
\item
The package \href{http://ctan.org/pkg/combine}{\textsf{combine}}
is an elaborate solution to combine several documents into one.
\end{itemize}
%
See also the CTAN topic \href{http://ctan.org/topic/subdocs}{\textsf{subdocs}}
for further related packages.
The present package differs from the above solutions in that
a document structure constructed with the conventional |\include| mechanism
just needs two extra commands at the top of every file
such that all constituent files can be compiled individually.

%%%%%%%%%%%%%%%%%%%%%%%%%%%%%%%%%%%%%%%%%%%%%%%%%%%%%%%%%%%%%%%%%%%%%%%%%%%%%%%%
%\subsection{Feature Suggestions}
%
%The following is a list of features which may be useful for future
%versions of this package:
%%
%\begin{itemize}
%\item
%\ldots
%\end{itemize}

%%%%%%%%%%%%%%%%%%%%%%%%%%%%%%%%%%%%%%%%%%%%%%%%%%%%%%%%%%%%%%%%%%%%%%%%%%%%%%%%
\subsection{Revision History}

%%%%%%%%%%%%%%%%%%%%%%%%%%%%%%%%%%%%%%%%
\paragraph{v2.0:} 2018/12/30

\begin{itemize}
\item
immediate forward processing
\item
added |\childdocby| mechanism
\item
manual restructured
\end{itemize}

%%%%%%%%%%%%%%%%%%%%%%%%%%%%%%%%%%%%%%%%
\paragraph{v1.6:} 2018/01/17

\begin{itemize}
\item
application for development of include files
\item
corrections to manual
\end{itemize}

%%%%%%%%%%%%%%%%%%%%%%%%%%%%%%%%%%%%%%%%
\paragraph{v1.5:} 2017/05/21

\begin{itemize}
\item
more complete structuring introduced
\item
|\childdocof| introduced
\item
|\childdoc| renamed to |\childdocmain|
\item
|\childredirect| renamed to |\childdocforward| and |\childdocforwardprefix|
and functionality expanded
\end{itemize}

%%%%%%%%%%%%%%%%%%%%%%%%%%%%%%%%%%%%%%%%
\paragraph{v1.0:} 2017/04/27

\begin{itemize}
\item
manual and install package
\item
first version published on CTAN
\end{itemize}

%%%%%%%%%%%%%%%%%%%%%%%%%%%%%%%%%%%%%%%%
\paragraph{v0.6:} 2017/04/26

\begin{itemize}
\item
redirection mechanism added
\end{itemize}

%%%%%%%%%%%%%%%%%%%%%%%%%%%%%%%%%%%%%%%%
\paragraph{v0.5:} 2017/04/26

\begin{itemize}
\item
functionality in definition file
\end{itemize}


%%%%%%%%%%%%%%%%%%%%%%%%%%%%%%%%%%%%%%%%%%%%%%%%%%%%%%%%%%%%%%%%%%%%%%%%%%%%%%%%
%%%%%%%%%%%%%%%%%%%%%%%%%%%%%%%%%%%%%%%%%%%%%%%%%%%%%%%%%%%%%%%%%%%%%%%%%%%%%%%%
%%%%%%%%%%%%%%%%%%%%%%%%%%%%%%%%%%%%%%%%%%%%%%%%%%%%%%%%%%%%%%%%%%%%%%%%%%%%%%%%
\appendix

\settowidth\MacroIndent{\rmfamily\scriptsize 000\ }

 \DocInput{childdoc.dtx}

\end{document}
%</driver>
% \fi
%
% %%%%%%%%%%%%%%%%%%%%%%%%%%%%%%%%%%%%%%%%%%%%%%%%%%%%%%%%%%%%%%%%%%%%%%%%%%%%%%
% %%%%%%%%%%%%%%%%%%%%%%%%%%%%%%%%%%%%%%%%%%%%%%%%%%%%%%%%%%%%%%%%%%%%%%%%%%%%%%
% \section{Sample}
%\iffalse
%<*samplemain>
%\fi
%
% The following presents a sample document
% with two chapters, two parts, a title page,
% a compile flag as well as three forwarding files to set the flag.
% It consists of eight |.tex| files:
% \begin{center}
% \begin{tabular}{ll}
% |cdocsamp.tex|&main file\\
% |cdocsch1.tex|&include file for chapter 1\\
% |cdocsch2.tex|&include file for chapter 2\\
% |cdocspt3.tex|&include file for part 3\\
% |cdocspt4.tex|&include file for part 4\\
% |cdocsdrf.tex|&forwarding file for main file in draft mode\\
% |cdocsfi1.tex|&forwarding file for final version of chapter 1\\
% |cdocsfi2.tex|&forwarding file for final version of chapter 2\\
% \end{tabular}
% \end{center}
% Each of the eight files can be compiled directly by the \LaTeX{} compiler.
%
% %%%%%%%%%%%%%%%%%%%%%%%%%%%%%%%%%%%%%%
% \paragraph{Main File.}
%
% The main file is called |cdocsamp.tex|.
%
% Load the \textsf{childdoc} definitions and
% declare the filename for the main document:
%    \begin{macrocode}
\input{childdoc.def}
\childdocmain{}
%    \end{macrocode}

% Optional override for |\version| flag:
%    \begin{macrocode}
%%\ifchilddoc\else\providecommand{\version}{draft}\fi
%    \end{macrocode}

% Define the default values for the |\version| flag
% (|final| for the main file and |draft| for childs):
%    \begin{macrocode}
\ifchilddoc
\providecommand{\version}{draft}
\else
\providecommand{\version}{final}
\fi
%    \end{macrocode}

% Load the standard document class:
%    \begin{macrocode}
\documentclass[12pt]{article}
%    \end{macrocode}

% Start the document body:
%    \begin{macrocode}
\begin{document}
%    \end{macrocode}

% Declare a title page.
% Print title, part of document being processed and version flag:
%    \begin{macrocode}
\addtocounter{page}{-1}
\begin{center}
{\LARGE\bfseries{}childdoc example\par}
\vspace{1cm}
\ifchilddoc
\ifchilddocmanual part\else chapter\fi:
`\childdocname' of `\childdocjob'\par
\else
main document: `\childdocjob'\par
\fi
version: \version\par
\end{center}
\newpage
%    \end{macrocode}

% Manually include selected file,
% otherwise process as usual:
%    \begin{macrocode}
\ifchilddocmanual
\section*{part `\childdocname'}
\input{\childdocname}
\else
%    \end{macrocode}

% Include the two chapters:
%    \begin{macrocode}
\include{cdocsch1}
\include{cdocsch2}
%    \end{macrocode}

% Include the two parts unless only chapters should be displayed:
%    \begin{macrocode}
\ifchilddoc\else
\section{part three}
\input{cdocspt3}
\section{part four}
\input{cdocspt4}
\fi
%    \end{macrocode}

% Process as usual until here:
%    \begin{macrocode}
\fi
%    \end{macrocode}

% End of document body:
%    \begin{macrocode}
\end{document}
%    \end{macrocode}
%\iffalse
%</samplemain>
%\fi
%
% %%%%%%%%%%%%%%%%%%%%%%%%%%%%%%%%%%%%%%
% \paragraph{Chapter Include Files.}
%
% The include files are called |cdocsch1.tex| and |cdocsch2.tex|.
%
%\iffalse
%<*samplechap1|samplechap2>
%\fi

% Optional override for |\version| flag:
%    \begin{macrocode}
%%\providecommand{\version}{final}
%    \end{macrocode}

% Include the main document:
%    \begin{macrocode}
\input{childdoc.def}
\childdocof{cdocsamp}
%    \end{macrocode}

%\iffalse
%</samplechap1|samplechap2>
%\fi
%
%\iffalse
%<*samplechap1>
%\fi
% Some text for chapter 1:
%    \begin{macrocode}
\section{one}
some text in chapter one
%    \end{macrocode}

%\iffalse
%</samplechap1>
%\fi
% Some text for chapter 2:
%\iffalse
%<*samplechap2>
%\fi
%    \begin{macrocode}
\section{two}
more text in chapter two
%    \end{macrocode}

%\iffalse
%</samplechap2>
%\fi
%
% %%%%%%%%%%%%%%%%%%%%%%%%%%%%%%%%%%%%%%
% \paragraph{Part Include Files.}
%
% The include files are called |cdocspt3.tex| and |cdocspt4.tex|.
%
%\iffalse
%<*samplepart3|samplepart4>
%\fi

% Optional override for |\version| flag:
%    \begin{macrocode}
%%\providecommand{\version}{final}
%    \end{macrocode}

% Include the main document:
%    \begin{macrocode}
\input{childdoc.def}
\childdocby{cdocsamp}
%    \end{macrocode}

%\iffalse
%</samplepart3|samplepart4>
%\fi
%
%\iffalse
%<*samplepart3>
%\fi
% Some text for part 3:
%    \begin{macrocode}
some text in part three
%    \end{macrocode}

%\iffalse
%</samplepart3>
%\fi
% Some text for part 4:
%\iffalse
%<*samplepart4>
%\fi
%    \begin{macrocode}
more text in part four
%    \end{macrocode}

%\iffalse
%</samplepart4>
%\fi
%
% %%%%%%%%%%%%%%%%%%%%%%%%%%%%%%%%%%%%%%
% \paragraph{Forwarding for a Complete Draft.}
%
% The following forwarding file |cdocsdrf.tex|
% compiles the main document in draft mode:
%\iffalse
%<*sampledraft>
%\fi
%    \begin{macrocode}
\def\version{draft}
\input{childdoc.def}
\childdocforward{cdocsamp}
%    \end{macrocode}

%\iffalse
%</sampledraft>
%\fi
%
% %%%%%%%%%%%%%%%%%%%%%%%%%%%%%%%%%%%%%%
% \paragraph{Forwarding for Final Version of the Chapters.}
%
% The following forwarding files |cdocsfn1.tex| and |cdocsfn2.tex|
% (with identical content)
% compile the final versions of the child documents
% |cdocsch1.tex| and |cdocsch2.tex|, respectively:
%\iffalse
%<*samplefinal>
%\fi
%    \begin{macrocode}
\def\version{final}
\input{childdoc.def}
\childdocforwardprefix[cdocsamp]{cdocsfn}{cdocsch}
%    \end{macrocode}

%\iffalse
%</samplefinal>
%\fi
%
% %%%%%%%%%%%%%%%%%%%%%%%%%%%%%%%%%%%%%%
% \paragraph{Command Line Processing.}
%
% The following three command lines generate the output files
% |cdocscld|, |cdocscl1| and |cdocscl2|
% which should be identical to
% |cdocsdrf|, |cdocsch1| and |cdocsfn2|, respectively:
% \begin{center}
% \begin{tabular}{l}
% |latex -jobname cdocscld \|\\
% |  "\def\version{draft}\input{childdoc.def}\childdocforward{cdocsamp}"|\\
% |latex -jobname cdocscl1 \|\\
% |  "\input{childdoc.def}\childdocforward[cdocsamp]{cdocsch1}"|\\
% |latex -jobname cdocscl2 \|\\
% |  "\def\version{final}\input{childdoc.def}\childdocforward{cdocsch2}"|
% \end{tabular}
% \end{center}
% Note that the trailing backslash on each first line
% merely continues the input to the second line
% (for convenient cut ant paste).
% Furthermore, the command |latex| can be replaced by any
% of its alternative versions such as |pdflatex|.
%
% %%%%%%%%%%%%%%%%%%%%%%%%%%%%%%%%%%%%%%%%%%%%%%%%%%%%%%%%%%%%%%%%%%%%%%%%%%%%%%
% %%%%%%%%%%%%%%%%%%%%%%%%%%%%%%%%%%%%%%%%%%%%%%%%%%%%%%%%%%%%%%%%%%%%%%%%%%%%%%
% \section{Implementation}
%\iffalse
%<*package>
%\fi
%
% This section describes the definitions file |childdoc.def|.

% The definitions cannot be loaded using |\usepackage| or |\RequirePackage|
% which has a mechanism to prevent loading a style file more than once.
% When loading the definitions by means of |\input|
% multiple instances have to be prevented manually:
%\iffalse
%This code needs to be before the `\ProvidesFile' directive
%which is defined at the beginning of this file.
%Therefore it is also placed there and commented out here.
%</package>
%<*discard>
%\fi
%    \begin{macrocode}
\ifdefined\childdocmain\endinput\fi
%    \end{macrocode}
%\iffalse
%</discard>
%<*package>
%\fi
%
% \macro{\ifchilddoc}
% \macro{\ifchilddocmanual}
% The conditional |\ifchilddoc| tells whether a
% child (true) or main (false) document is being compiled.
% The conditional |\ifchilddocmanual| tells whether
% the |\includeonly| mechanism is used (false) or
% the selection of child files must be performed manually (true).
% The definitions initialise to false:
%    \begin{macrocode}
\newif\ifchilddoc
\newif\ifchilddocmanual
%    \end{macrocode}

% \macro{\childdocname}
% \macro{\childdocjob}
% The macro |\childdocname| stores the name of the main document
% to be compiled. The macro |\childdocjob| stores the name of
% the document on which the \LaTeX{} compiler was originally invoked.
% The content of |\jobname| cannot be compared
% to filenames specified in the source due to different catcodes.
% The following code rescans |\jobname|, stores the result
% in |\childdocname| and saves a copy in |\childdocjob|:
%    \begin{macrocode}
\edef\childdocname{\scantokens\expandafter{\jobname\noexpand}}
\let\childdocjob\childdocname
%    \end{macrocode}

% \macro{\childdocdisable}
% The macro |\childdocdisable| prevents the main file
% from being processed more than once.
% At this stage, the main document command |\childdocmain|
% is assumed to be called once again where it should do nothing.
% Any subsequent call to it should prevent
% a secondary processing of the main document
% It overwrites the forwarding commands
% |\childdocof| and |\childdocforward|
% with empty macros to prevent further inclusions of the main document:
%    \begin{macrocode}
\newcommand{\childdocdisable}
{
  \renewcommand{\childdocmain}[1]{\renewcommand{\childdocmain}[1]{\endinput}}
  \renewcommand{\childdocof}[1]{}
  \renewcommand{\childdocby}[2][]{}
  \renewcommand{\childdocforward}[2][]{}
  \renewcommand{\childdocdisable}{}
}
%    \end{macrocode}

% \macro{\childdocmain}
% The macro |\childdocmain| is to be called at the top of the main file
% with nothing or the main filename (without extension) as argument.
% First, it breaks loops.
% If the argument is not empty and does not match |\childdocname|
% (which is set by the first inclusion of |childdoc.def|),
% |\ifchilddoc| is set to true, |\includeonly| is applied to the child file
% and |\jobname| is set to the main file
% (for proper handling of |.aux| files):
%    \begin{macrocode}
\newcommand{\childdocmain}[1]
{
  \childdocdisable\childdocmain{}
  \if?#1?\else
    \begingroup
      \def\childdoctmp{#1}
      \ifx\childdoctmp\childdocname
        \def\childdoctmp{}
      \else
        \def\childdoctmp
        {
          \childdoctrue
          \includeonly{\childdocname}
          \def\childdocjob{#1}
          \def\jobname{#1}
        }
      \fi
      \expandafter
    \endgroup
    \childdoctmp
  \fi
}
%    \end{macrocode}

% \macro{\childdocof}
% The command |\childdocof| redirects
% compilation to the main file |#1|.
%    \begin{macrocode}
\newcommand{\childdocof}[1]
{
  \childdocdisable
  \childdoctrue
  \includeonly{\childdocname}
  \def\jobname{#1}
  \def\childdocjob{#1}
  \input{#1}
}
%    \end{macrocode}

% \macro{\childdocby}
% The command |\childdocby| ....
%    \begin{macrocode}
\newcommand{\childdocby}[2][]
{
  \childdocdisable
  \childdoctrue
  \childdocmanualtrue
  \if?#1?\else
    \def\jobname{#2}
  \fi
  \def\childdocjob{#2}
  \input{#2}
  \endinput
}
%    \end{macrocode}

% \macro{\childdocforward}
% The command |\childdocforward| redirects
% compilation to the main file or
% (if the optional argument is given) a child file.
% Parameters are set as if the main file
% or a child file starting with |\childdocof| was compiled.
% Then compilation is handed over to the main file:
%    \begin{macrocode}
\newcommand{\childdocforward}[2][]
{
  \begingroup
    \if?#1?
      \def\childdoctmp
      {
        \def\childdocname{#2}
        \def\childdocjob{#2}
        \def\jobname{#2}
        \input{#2}
        \endinput
      }
    \else
      \def\childdoctmp
      {
        \childdocdisable
        \def\childdocname{#2}
        \childdoctrue
        \includeonly{#2}
        \def\childdocjob{#1}
        \def\jobname{#1}
        \input{#1}
        \endinput
      }
    \fi
    \expandafter
  \endgroup
  \childdoctmp
}
%    \end{macrocode}

% \macro{\childdocforwardprefix}
% The command |\childdocforwardprefix| redirects
% compilation to the main or a child file by means of a pattern.
% The prefix |#1| in the current filename is replaced by |#2|
% and the suffix of the current filename is kept
% (it is assumed that the filename does not contain the substring `|~~~|'
% which is used as a delimiter).
% Compilation is handed over to the new file by |\childdocforward|:
%    \begin{macrocode}
\newcommand{\childdocforwardprefix}[3][]
{
  \begingroup
    \def\childdocextract #2##1~~~{\def\childdoctmp{\childdocforward[#1]{#3##1}}}
    \expandafter\childdocextract\childdocname~~~
    \expandafter
  \endgroup
  \childdoctmp
}
%    \end{macrocode}

% \macro{\childdoc}
% The deprecated macro |\childdoc| is a legacy version of |\childdocmain|:
%    \begin{macrocode}
\newcommand{\childdoc}{\childdocmain}
%    \end{macrocode}

% \macro{\childdocredirect}
% The deprecated macro |\childdocredirect| is a legacy version
% of |\childdocforward| and |\childdocforwardprefix|:
%    \begin{macrocode}
\newcommand{\childdocredirect}[2][]
{
  \begingroup
    \if?#1?
      \def\childdoctmp{\childdocforward{#2}}
    \else
      \def\childdoctmp{\childdocforwardprefix{#1}{#2}}
    \fi
    \expandafter
  \endgroup
  \childdoctmp
}
%    \end{macrocode}

%\iffalse
%</package>
%\fi
%
\endinput

\childdocof{cdocsamp}
%    \end{macrocode}

%\iffalse
%</samplechap1|samplechap2>
%\fi
%
%\iffalse
%<*samplechap1>
%\fi
% Some text for chapter 1:
%    \begin{macrocode}
\section{one}
some text in chapter one
%    \end{macrocode}

%\iffalse
%</samplechap1>
%\fi
% Some text for chapter 2:
%\iffalse
%<*samplechap2>
%\fi
%    \begin{macrocode}
\section{two}
more text in chapter two
%    \end{macrocode}

%\iffalse
%</samplechap2>
%\fi
%
% %%%%%%%%%%%%%%%%%%%%%%%%%%%%%%%%%%%%%%
% \paragraph{Part Include Files.}
%
% The include files are called |cdocspt3.tex| and |cdocspt4.tex|.
%
%\iffalse
%<*samplepart3|samplepart4>
%\fi

% Optional override for |\version| flag:
%    \begin{macrocode}
%%\providecommand{\version}{final}
%    \end{macrocode}

% Include the main document:
%    \begin{macrocode}
% \iffalse
%
% childdoc.dtx Copyright (C) 2017-2018 Niklas Beisert
%
% This work may be distributed and/or modified under the
% conditions of the LaTeX Project Public License, either version 1.3
% of this license or (at your option) any later version.
% The latest version of this license is in
%   http://www.latex-project.org/lppl.txt
% and version 1.3 or later is part of all distributions of LaTeX
% version 2005/12/01 or later.
%
% This work has the LPPL maintenance status `maintained'.
%
% The Current Maintainer of this work is Niklas Beisert.
%
% This work consists of the files childdoc.dtx and childdoc.ins
% and the derived files childdoc.def and cdocsamp.tex with
% cdocsch1.tex, cdocsch2.tex, cdocsdrf.tex, cdocsfn1.tex, cdocsfn2.tex.
%
%<package>\ifdefined\childdocmain\endinput\fi
%<package>\ProvidesFile{childdoc.def}[2018/12/30 v2.0 child document driver]
%<samplemain>\ProvidesFile{cdocsamp.tex}[2018/12/30 v2.0 sample for childdoc]
%<*driver>
%\ProvidesFile{childdoc.drv}[2018/12/30 v2.0 childdoc reference manual file]
\PassOptionsToClass{10pt,a4paper}{article}
\documentclass{ltxdoc}

\usepackage[margin=35mm]{geometry}
\usepackage{hyperref}
\usepackage{hyperxmp}
\usepackage[usenames]{color}

\hypersetup{colorlinks=true}
\hypersetup{pdfstartview=FitH}
\hypersetup{pdfpagemode=UseNone}
\hypersetup{pdfsource={}}
\hypersetup{pdflang={en-UK}}
\hypersetup{pdfcopyright={Copyright 2017-2018 Niklas Beisert.
  This work may be distributed and/or modified under the
  conditions of the LaTeX Project Public License, either version 1.3
  of this license or (at your option) any later version.}}
\hypersetup{pdflicenseurl={http://www.latex-project.org/lppl.txt}}
\hypersetup{pdfcontactaddress={ETH Zurich, ITP, HIT K,
  Wolfgang-Pauli-Strasse 27}}
\hypersetup{pdfcontactpostcode={8093}}
\hypersetup{pdfcontactcity={Zurich}}
\hypersetup{pdfcontactcountry={Switzerland}}
\hypersetup{pdfcontactemail={nbeisert@itp.phys.ethz.ch}}
\hypersetup{pdfcontacturl={http://people.phys.ethz.ch/\xmptilde nbeisert/}}

\newcommand{\secref}[1]{\hyperref[#1]{section \ref*{#1}}}

\parskip1ex
\parindent0pt
\let\olditemize\itemize
\def\itemize{\olditemize\parskip0pt}

\begin{document}

\title{The \textsf{childdoc} Package}
\hypersetup{pdftitle={The childdoc Package}}
\author{Niklas Beisert\\[2ex]
  Institut f\"ur Theoretische Physik\\
  Eidgen\"ossische Technische Hochschule Z\"urich\\
  Wolfgang-Pauli-Strasse 27, 8093 Z\"urich, Switzerland\\[1ex]
  \href{mailto:nbeisert@itp.phys.ethz.ch}
  {\texttt{nbeisert@itp.phys.ethz.ch}}}
\hypersetup{pdfauthor={Niklas Beisert}}
\hypersetup{pdfsubject={Manual for the LaTeX2e Package childdoc}}
\date{30 December 2018, \textsf{v2.0}}
\maketitle

\begin{abstract}\noindent
\textsf{childdoc} is a \LaTeXe{} package
that enables the direct compilation
of document sections included by |\include|
to individual files.
\end{abstract}

\begingroup
\parskip0ex
\tableofcontents
\endgroup

%%%%%%%%%%%%%%%%%%%%%%%%%%%%%%%%%%%%%%%%%%%%%%%%%%%%%%%%%%%%%%%%%%%%%%%%%%%%%%%%
%%%%%%%%%%%%%%%%%%%%%%%%%%%%%%%%%%%%%%%%%%%%%%%%%%%%%%%%%%%%%%%%%%%%%%%%%%%%%%%%
\section{Introduction}

\LaTeX{} provides a mechanism to structure a large document (such as a book)
into a main file and several child files (containing the chapters)
using the |\include| command.
This mechanism is beneficial for documents
which span hundreds of pages in order to
make the source file(s) more manageable.
Moreover, compilation can be restricted to
selected child files by means of the |\includeonly| command.
The latter feature can be used to reduce the compilation time while editing
(this was significantly more useful in the earlier days of \LaTeX{})
or to generate a smaller document which is easier to navigate.
Another application of |\includeonly| is to generate
documents consisting of selected parts of the complete document.

However, there are a few drawbacks of the plain |\include| mechanism:
\begin{itemize}
\item
The child files cannot be compiled on their own,
they can only be compiled via the main file.
A naive editing environment
(such as a text editor with an option
to have the current file processed by \LaTeX)
may require one to switch to the main file before compiling;
attempting to compile the child file produces errors.
\item
The main file must be modified (each time)
to adjust the |\includeonly| command
to the present needs. This easily leaves the main file in a messy state.
\item
The generated document will always carry the filename
of the main document. This is inconvenient if
several child files are to be compiled and
to be kept for distribution.
\end{itemize}

The present package provides a simple interface
to make child files individually compilable by \LaTeX{}.
Compiling a child file then has the same effect as compiling
the main file with an |\includeonly| command
to select the appropriate child.
Moreover the generated document will carry the name of the child
rather than the main file.
This resolves all three above issues.

This feature is meant to make the editing of books,
thesis documents and lecture notes somewhat more convenient.
However, the package can also be used efficiently for
composing a series of documents (such as exercise sheets)
which are typically distributed individually.
It then assists the author in generating the individual documents
(potentially in different versions)
as well as a document containing the collected series.
Another application is in developing style files
or other kinds of included material
where compilation of the style file could redirect
to a sample or test file.

%%%%%%%%%%%%%%%%%%%%%%%%%%%%%%%%%%%%%%%%%%%%%%%%%%%%%%%%%%%%%%%%%%%%%%%%%%%%%%%%
%%%%%%%%%%%%%%%%%%%%%%%%%%%%%%%%%%%%%%%%%%%%%%%%%%%%%%%%%%%%%%%%%%%%%%%%%%%%%%%%
\section{Usage}

First of all, the package \textsf{childdoc} is \emph{not} a standard
\LaTeXe{} |.sty| style file! Therefore it needs to be invoked in
a non-standard way.

%%%%%%%%%%%%%%%%%%%%%%%%%%%%%%%%%%%%%%%%%%%%%%%%%%%%%%%%%%%%%%%%%%%%%%%%%%%%%%%%
\subsection{Included Files}
\label{sec:include}

%%%%%%%%%%%%%%%%%%%%%%%%%%%%%%%%%%%%%%%%
\DescribeMacro{\childdocmain}
To use the package, add the commands
\begin{center}
\begin{tabular}{l}
|\input{childdoc.def}|\\
|\childdocmain{}|\\
\end{tabular}
\end{center}
at the very top of the main \LaTeX{} file,
in particular \emph{before} the |\documentclass| statement!
The argument of |\childdocmain| should be left empty
(but it must be present).

%%%%%%%%%%%%%%%%%%%%%%%%%%%%%%%%%%%%%%%%
\DescribeMacro{\childdocof}
Furthermore, add the commands
\begin{center}
\begin{tabular}{l}
|\input{childdoc.def}|\\
|\childdocof{|\textit{main}|}|\\
\end{tabular}
\end{center}
at the top of every child file \textit{child}
which is included by |\include{|\textit{child}|}|
from within the main file
(or at least for those files to be compiled individually).
The argument \textit{main} must be the filename of the main file.

There are a couple of
considerations in setting up the main and child documents:

%%%%%%%%%%%%%%%%%%%%%%%%%%%%%%%%%%%%%%%%
\paragraph{Restrictions.}

Please note the following restrictions:
\begin{itemize}
\item
|\childdocmain| must be called with one argument \textit{main}
to ensure compatibility with earlier version of the package.
It must either be empty (|\childdocmain{}|)
or precisely match the filename of the main file in which it is specified.
See \secref{sec:detection} for further information.
\item
The filename \textit{main} must be specified without the |.tex| extension.
\item
The filename \textit{main} is case sensitive
(even in case-insensitive file systems)
due to internal string comparison.
\item
The argument \textit{main} should be fully expanded, it cannot be a macro.
\item
Subdirectories and special characters should be avoided in filenames.
\item
The command |\childdocmain{|\textit{main}|}| must be followed by a whitespace.
It should not be followed immediately by another command
or by a comment mark `|%|'.
This is because the \TeX{} parser reads the token immediately following
the argument of |\childdocmain| and puts it
at the beginning of every child section;
however, a white\-space is ignored.
\end{itemize}

%%%%%%%%%%%%%%%%%%%%%%%%%%%%%%%%%%%%%%%%
\paragraph{Content of Main File.}

It is advisable to place all content in the child files included by |\include|.
Any output contained in the main file will appear in all child documents
unless suppressed manually;
it cannot be suppressed automatically by the |\includeonly| directive
and thus should normally be avoided.
A method to include some content in the main file
by means of conditional processing is described in \secref{sec:conditional}.

%%%%%%%%%%%%%%%%%%%%%%%%%%%%%%%%%%%%%%%%
\paragraph{Page Numbering.}

When only a part of the document is compiled,
the appropriate numbering of pages
(as well as other status parameters)
is determined from the |.aux| files.
The latter contain information from previous passes.
However this information needs to propagate through
all intermediate child documents.
Therefore the page numbering in child documents may well
be inconsistent until the complete document is compiled at least once.

A useful (if unconventional) way to always ensure a consistent
page numbering is to restart the numbering in each child document
and denote the pages by `\textit{child}|.|\textit{page}'
where \textit{child} represents the chapter/section number of the child file.
This can be achieved by the command
|\numberwithin{page}{|\textit{child}|}|
of the \textsf{amsmath} package
where \textit{child} can be |chapter| or |section|
depending on the chosen structuring.
Alternatively, one can modify the macro |\thepage| appropriately
and reset the counter |page| at the start of each child file.

%%%%%%%%%%%%%%%%%%%%%%%%%%%%%%%%%%%%%%%%%%%%%%%%%%%%%%%%%%%%%%%%%%%%%%%%%%%%%%%%
\subsection{Conditional Processing}
\label{sec:conditional}

The package provides a mechanism to compile different versions
of a document. To customise the versions further some conditional processing
can come in handy to distinguish which version is being compiled.
The package provides two macros to describe the compilation context:

%%%%%%%%%%%%%%%%%%%%%%%%%%%%%%%%%%%%%%%%
\DescribeMacro{\ifchilddoc}
The conditional |\ifchilddoc| distinguishes between the compilation of
child documents and the main document:
%
\begin{center}
|\ifchilddoc |\textit{child-code}| |[|\||else |\textit{main-code}]| \||fi|
\end{center}

%%%%%%%%%%%%%%%%%%%%%%%%%%%%%%%%%%%%%%%%
\DescribeMacro{\childdocname}
\DescribeMacro{\childdocjob}
The macro |\childdocname| contains the filename (without extension)
of the main or child file being processed.
Note that |\childdocjob| will always contain the name of the main file.

%%%%%%%%%%%%%%%%%%%%%%%%%%%%%%%%%%%%%%%%
\paragraph{Title Page.}

Conditional processing can be used to include a title or banner page
in the main document when proper precautions are taken.
Importantly, the code in the main file should ensure that the page counter
(as well as other status parameters which are stored in the |.aux| files)
takes the same value after the conditional processing.
Otherwise the page numbers may take divergent values
depending on which part is compiled.

For example, a title page could be declared by:
%
\begin{center}
\begin{tabular}{l}
|\ifchilddoc\||else|\\
|\addtocounter{page}{-1}|\\
\textit{code for title page}\\
|\newpage|\\
|\||fi|
\end{tabular}
\end{center}
%
A banner page for the child documents can be generated by:
%
\begin{center}
\begin{tabular}{l}
|\ifchilddoc|\\
|\addtocounter{page}{-1}|\\
\textit{code for banner page}\\
|\newpage|\\
|\||fi|
\end{tabular}
\end{center}
%
Here one could write a message such as:
\begin{center}
|This is the part \childdocname{} of \childdocjob{}.|
\end{center}

%%%%%%%%%%%%%%%%%%%%%%%%%%%%%%%%%%%%%%%%%%%%%%%%%%%%%%%%%%%%%%%%%%%%%%%%%%%%%%%%
\subsection{Flags}
\label{sec:flags}

The package makes it easy to generate different versions
of the main or child documents.
To this end compilation flags can be defined
and assigned different default values.
They will be particularly useful in conjunction
with the forwarding mechanism described in \secref{sec:forward}.

For example, it may be useful to have a flag |\version|
which can be set to |draft| or |final|.
The document source will contain some conditional code
depending on the value of |\version|.
Suppose further, the flag should default to |final| for the main file
and to |draft| for child files
which is a natural assignment for editing the document.
This is achieved by placing the following code
in the preamble of the main document
(below the |\childdocmain| directive):
%
\begin{center}
\begin{tabular}{l}
|\ifchilddoc|\\
|\providecommand{\version}{draft}|\\
|\||else|\\
|\providecommand{\version}{final}|\\
|\||fi|
\end{tabular}
\end{center}
%
The definition by |\providecommand| makes sure
that previous definitions are not overwritten.
Further statements |\providecommand{\version}{...}|
can thus be added before the above code to override it.

For the main file, one might add a line
(between |\childdocmain| and the above block)
%
\begin{center}
|%\ifchilddoc\||else\providecommand{\version}{draft}\||fi|
\end{center}
%
which can be uncommented to produce a draft version.
Likewise one can add a line to the very top of a child file
(above the |\childdocof{|\textit{main}|}| directive)
%
\begin{center}
|%\providecommand{\version}{final}|
\end{center}
%
which can be uncommented to produce the final version of this child document.

%%%%%%%%%%%%%%%%%%%%%%%%%%%%%%%%%%%%%%%%%%%%%%%%%%%%%%%%%%%%%%%%%%%%%%%%%%%%%%%%
\subsection{Forwarding}
\label{sec:forward}

Different versions of the main or child documents
using compilation flags as described in \secref{sec:flags}
can be (permanently) stored in different files
for convenient compilation, viewing and distribution.
To this end, the package defines a command
to pass on compilation to a different file:

%%%%%%%%%%%%%%%%%%%%%%%%%%%%%%%%%%%%%%%%
\DescribeMacro{\childdocforward}
The command |\childdocforward| redirects processing to
another source file:
%
\begin{center}
\begin{tabular}{l}
|\input{childdoc.def}|\\
|\childdocforward[|\textit{main}|]{|\textit{dest}|}|\\
\end{tabular}
\end{center}
%
The argument \textit{dest} is the destination file
(without extension).
It should be the main file or one of the child files.
Note that further \textsf{childdoc} directives
such as |\childdocof| and |\childdocforward|
in the indicated file will be processed in this form.
The optional argument \textit{main}
passes on directly to the main file \textit{main}
while pretending to compile the child \textit{dest}.
This form behaves as if \textit{dest}
issues |\childdocof{|\textit{main}|}| right away,
and no further \textsf{childdoc} directives will be processed.

%%%%%%%%%%%%%%%%%%%%%%%%%%%%%%%%%%%%%%%%
\DescribeMacro{\...prefix}
In the alternative form |\childdocforwardprefix|,
%
\begin{center}
\begin{tabular}{l}
|\input{childdoc.def}|\\
|\childdocforwardprefix[|\textit{main}|]{|\textit{prefix}|}{|\textit{dest}|}|
\end{tabular}
\end{center}
%
the destination file is determined by a pattern
depending on the current file:
To make this work, the current file must be called
`{\textit{prefix}\hspace{0.2em}\textit{suffix}}'
with \textit{prefix} matching precisely the argument.
Processing is then passed on to the file
`{\textit{dest}\hspace{0.2em}\textit{suffix}}'.
Surely, the same effect is achieved by
directly specifying the
argument `{\textit{dest}\hspace{0.2em}\textit{suffix}}'
in the first form.
However, that requires to set up a different file
for each child. With the alternative form of the command
all these files can have exactly the same content
which simplifies setting them up and maintaining them.

For example, the following file |draft.tex|
with a compilation flag |\version| as described in \secref{sec:flags}
compiles the main document as a draft:
%
\begin{center}
\begin{tabular}{l}
|\def\version{draft}|\\
|\input{childdoc.def}|\\
|\childdocforward{|\textit{main}|}|
\end{tabular}
\end{center}
%
Likewise, the following files |final|\textit{nn}|.tex|
compile the final version of the child document
|child|\textit{nn}|.tex|:
%
\begin{center}
\begin{tabular}{l}
|\def\version{final}|\\
|\input{childdoc.def}|\\
|\childdocforwardprefix{final}{child}|
\end{tabular}
\end{center}
%

Note that when several versions of a main file and/or of each child file
are to be generated, it may be convenient to set up a |Makefile| or
shell script to automatise the process.

%%%%%%%%%%%%%%%%%%%%%%%%%%%%%%%%%%%%%%%%%%%%%%%%%%%%%%%%%%%%%%%%%%%%%%%%%%%%%%%%
\subsection{Command Line Processing}
\label{sec:commandline}

The effect of redirection files can also be achieved by invoking
the \LaTeX{} compiler with a more elaborate command line.
Most conveniently this should be done as part
of a shell script or a |Makefile|.

When using \textsf{childdoc} in the main file, the following
command lines effectively perform a redirection
(note that depending on the shell being used,
backslashes may have to be doubled: `|\|' $\to$ `|\\|'):
%
\begin{center}
|... -jobname "|\textit{target}|" |\\|"|[\textit{flags}]%
|\input{childdoc.def}\childdocforward[|\textit{main}|]{|\textit{dest}|}"|
\end{center}
%
Here \textit{target} is the name of the output file,
\textit{main} is the name of the main file
and \textit{dest} is the name of the main or child file to be processed
(all filenames without extensions).
The optional argument \textit{main} can be omitted
if \textit{main} matches \textit{dest}.
Optionally, compilation \textit{flags} can be defined via |\def| commands.
This command line makes the \TeX{} engine believe
it is compiling the file \textit{target}
whose content is specified as the latter parameter.
The provided code then forwards the processing to
\textit{main} or \textit{dest} as described in \secref{sec:forward}.

%%%%%%%%%%%%%%%%%%%%%%%%%%%%%%%%%%%%%%%%%%%%%%%%%%%%%%%%%%%%%%%%%%%%%%%%%%%%%%%%
\subsection{Include by Input}
\label{sec:input}

Including child documents by |\include| has some restrictions by design.
Most notably, the content of a child document always occupies
its own set of pages; pages cannot be shared between child documents.
Usually, this behaviour makes perfect sense
because each child document contain an essential part of the document.
However, in some situations it may be desirable to compose
a document from a collection of parts
without having mandatory page breaks between then.
For this case, the package
provides a mechanism to include parts
by |\input| which can also be processed individually.
However, by construction this mechanism
requires manual handling of the content to be output.

%%%%%%%%%%%%%%%%%%%%%%%%%%%%%%%%%%%%%%%%
\DescribeMacro{\ifchilddocmanual}
The main file should be prepared as usual, see \secref{sec:include}.
However, the document body must make a distinction
between processing of an individual part and of the main document, e.g.:
%
\begin{center}
\begin{tabular}{l}
|\ifchilddocmanual|\\
|\input{\childdocname}|\\
|\||else|\\
\textit{document body with }|\input{|\textit{part}|}|\\
|\||fi|
\end{tabular}
\end{center}
%
The conditional |\ifchilddocmanual| is true whenever
a part to be included by |\input| is being compiled,
and the name of the part is stored in |\childdocname|.

%%%%%%%%%%%%%%%%%%%%%%%%%%%%%%%%%%%%%%%%
\DescribeMacro{\childdocby}
Each part to be included by |\input| should start with:
%
\begin{center}
\begin{tabular}{l}
|\input{childdoc.def}|\\
|\childdocby{|\textit{main}|}|\\
\end{tabular}
\end{center}
%
The directive |\childdocby| is similar to |\childdocof|
described in \secref{sec:include},
but the subsequent selection of content must be done manually.
To that end, both |\ifchilddoc| and |\ifchilddocmanual|
will be true upon processing of a part,
and the name of the part is stored in |\childdocname|.
Note that |\jobname| will be set to the filename of the current part
so that each part receives an individual |.aux| file
that does not interfere with the |.aux| file(s) of the main document.
This behaviour can be altered by the alternative form
|\childdocby[*]{|\textit{main}|}| (with a non-empty optional argument)
which uses the |.aux| file of the main document
by setting |\jobname| to \textit{main}.

%%%%%%%%%%%%%%%%%%%%%%%%%%%%%%%%%%%%%%%%%%%%%%%%%%%%%%%%%%%%%%%%%%%%%%%%%%%%%%%%
\subsection{Driver Development}
\label{sec:driver}

The \textsf{childdoc} mechanism can also be use for the development
of definition files such as \LaTeX{} styles or classes.
This case differs from the above setup with multiple parts
included by |\include| in that no |\includeonly| should be invoked.
This can be achieved by starting the include file
(before |\ProvidesPackage|) with:
%
\begin{center}
\begin{tabular}{l}
|\input{childdoc.def}|\\
|\childdocforward{|\textit{main}|}|\\
\end{tabular}
\end{center}
%
or alternatively with:
%
\begin{center}
\begin{tabular}{l}
|\input{childdoc.def}|\\
|\childdocby{|\textit{main}|}|\\
\end{tabular}
\end{center}
%
Both forms have slightly different effects as described above.
The main file is prepared as usual, see \secref{sec:include}.

%%%%%%%%%%%%%%%%%%%%%%%%%%%%%%%%%%%%%%%%%%%%%%%%%%%%%%%%%%%%%%%%%%%%%%%%%%%%%%%%
\subsection{Legacy Detection}
\label{sec:detection}

The directive |\childdocmain| in the main file can detect
whether the complete document or merely a child is to be compiled
even without using the directive |\childdocof|.
This method is deprecated because it is less robust
and there is no compelling reason to use it;
it is merely provided for backward compatibility
and it may be removed in future versions.

If the detection mechanism is to be used,
it is mandatory to correctly specify
the filename of the main file as the argument of |\childdocmain|:
%
\begin{center}
\begin{tabular}{l}
|\input{childdoc.def}|\\
|\childdocmain{|\textit{main}|}|\\
\end{tabular}
\end{center}
%
If |\jobname| does not match the argument \textit{main} of |\childdocmain|,
it is assumed that |\jobname| points to the child file to be compiled.
When using |\childdocmain| with the main file specified as argument,
it suffices to start a child file
with just |\input{|\textit{main}|}|
without loading of the package and using |\childdocof|.
If instead all processing is done
with the appropriate \textsf{childdoc} directives,
the argument of \textit{main} of |\childdocmain| can be empty.

An alternative version of the command line processing described
in \secref{sec:commandline} using the detection mechanism reads:
%
\begin{center}
|... -jobname "|\textit{target}|" "|[\textit{flags}]%
[|\def\jobname{|\textit{dest}|}|]|\input{|\textit{main}|}"|
\end{center}

%%%%%%%%%%%%%%%%%%%%%%%%%%%%%%%%%%%%%%%%%%%%%%%%%%%%%%%%%%%%%%%%%%%%%%%%%%%%%%%%
\subsection{Manual Code}
\label{sec:manual}

In case one cannot be certain whether the definitions file |childdoc.def|
is installed on the target \TeX{} distribution
and one prefers not to ship it,
it is conceivable to paste a few relevant commands into the sources.

To that end, drop all statements |\input{childdoc.def}|
and perform the replacements as outlined below.
Instead of |\childdocmain{|\textit{main}|}| add the following code
to the top of the main file:
%
\begin{center}
\begin{tabular}{l}
|\||ifdefined\childdocname\endinput\||fi\newif\ifchilddoc|\\
|\edef\childdocname{\scantokens\expandafter{\jobname\noexpand}}|\\
|\def\childdocmain{|\textit{main}|}\||ifx\childdocmain\childdocname\||else|\\
|\childdoctrue\includeonly{\childdocname}\let\jobname\childdocmain\||fi|\\
\end{tabular}
\end{center}
%
Instead of |\childdocof{|\textit{main}|}| just include the main file
at the top of each child file:
%
\begin{center}
|\input{|\textit{main}|}|
\end{center}
%
A simple redirection |\childdocforward{|\textit{dest}|}| is achieved by:
%
\begin{center}
|\def\jobname{|\textit{dest}|}\input{\jobname}|
\end{center}
%
The redirection with prefix
|\childdocforwardprefix[|\textit{prefix}|]{|\textit{dest}|}|
is accomplished by:
%
\begin{center}
\begin{tabular}{l}
|{\edef\jobname{\scantokens\expandafter{\jobname\noexpand}}|\\
|\def\redirectjob |\textit{prefix}|#1~~~{\gdef\jobname{|\textit{dest}|#1}}|\\
|\expandafter\redirectjob\jobname~~~}\input{\jobname}|
\end{tabular}
\end{center}

In an alternative approach,
child documents can be compiled by a specific command line
without additional code or specific definitions:
%
\begin{center}
|... -jobname "|\textit{target}|" "|[\textit{flags}]%
|\includeonly{|\textit{dest}|}\input{|\textit{main}|}"|
\end{center}
%

%%%%%%%%%%%%%%%%%%%%%%%%%%%%%%%%%%%%%%%%%%%%%%%%%%%%%%%%%%%%%%%%%%%%%%%%%%%%%%%%
%%%%%%%%%%%%%%%%%%%%%%%%%%%%%%%%%%%%%%%%%%%%%%%%%%%%%%%%%%%%%%%%%%%%%%%%%%%%%%%%
\section{Information}

%%%%%%%%%%%%%%%%%%%%%%%%%%%%%%%%%%%%%%%%%%%%%%%%%%%%%%%%%%%%%%%%%%%%%%%%%%%%%%%%
\subsection{Copyright}

Copyright \copyright{} 2017--2018 Niklas Beisert

This work may be distributed and/or modified under the
conditions of the \LaTeX{} Project Public License, either version 1.3
of this license or (at your option) any later version.
The latest version of this license is in
  \url{http://www.latex-project.org/lppl.txt}
and version 1.3 or later is part of all distributions of \LaTeX{}
version 2005/12/01 or later.

This work has the LPPL maintenance status `maintained'.

The Current Maintainer of this work is Niklas Beisert.

This work consists of the files |README.txt|, |childdoc.ins| and |childdoc.dtx|
as well as the derived files |childdoc.def|, |cdocsamp.tex|
with |cdocsch1.tex|, |cdocsch2.tex|, |cdocspt3.tex|, |cdocspt4.tex|,
|cdocsdrf.tex|, |cdocsfn1.tex|, |cdocsfn2.tex|
as well as |childdoc.pdf|.

%%%%%%%%%%%%%%%%%%%%%%%%%%%%%%%%%%%%%%%%%%%%%%%%%%%%%%%%%%%%%%%%%%%%%%%%%%%%%%%%
\subsection{Files and Installation}

The package consists of the files:
%
\begin{center}
\begin{tabular}{ll}
    |README.txt|   & readme file \\
    |childdoc.ins| & installation file \\
    |childdoc.dtx| & source file \\
    |childdoc.def| & definition file \\
    |cdocsamp.tex| & sample main file \\
    |cdocsch1.tex| & sample include file \\
    |cdocsch2.tex| & sample include file \\
    |cdocspt3.tex| & sample part file \\
    |cdocspt4.tex| & sample part file \\
    |cdocsdrf.tex| & sample redirection file \\
    |cdocsfn1.tex| & sample redirection file \\
    |cdocsfn2.tex| & sample redirection file \\
    |childdoc.pdf| & manual
\end{tabular}
\end{center}
%
The distribution consists of the files
|README.txt|, |childdoc.ins| and |childdoc.dtx|.
%
\begin{itemize}
\item
Run (pdf)\LaTeX{} on |childdoc.dtx|
to compile the manual |childdoc.pdf| (this file).
\item
Run \LaTeX{} on |childdoc.ins| to create the definitions file |childdoc.def|
and the sample |cdocsamp.tex| with include files
|cdocsch1.tex|, |cdocsch2.tex|, |cdocspt3.tex|, |cdocspt4.tex|,
|cdocsdrf.tex|, |cdocsfn1.tex|, |cdocsfn2.tex|.
Then copy the file |childdoc.def| to an appropriate directory of your \LaTeX{}
distribution, e.g.\ \textit{texmf-root}|/tex/latex/childdoc|.
\end{itemize}

%%%%%%%%%%%%%%%%%%%%%%%%%%%%%%%%%%%%%%%%%%%%%%%%%%%%%%%%%%%%%%%%%%%%%%%%%%%%%%%%
\subsection{Related CTAN Packages}

There are several other packages which offer a similar functionality:
%
\begin{itemize}
\item
The packages
\href{http://ctan.org/pkg/docmute}{\textsf{docmute}},
\href{http://ctan.org/pkg/includex}{\textsf{includex}} and
\href{http://ctan.org/pkg/standalone}{\textsf{standalone}}
provide commands to include only the document body of
a child file thus allowing both files to be compiled individually.
\item
The packages \href{http://ctan.org/pkg/subdocs}{\textsf{subdocs}}
and \href{http://ctan.org/pkg/subfiles}{\textsf{subfiles}}
provide structures in which the main and child documents can be
encapsulated and allowing them to be compiled individually.
The inclusion mechanism is different from the conventional |\include|.
\item
The package \href{http://ctan.org/pkg/combine}{\textsf{combine}}
is an elaborate solution to combine several documents into one.
\end{itemize}
%
See also the CTAN topic \href{http://ctan.org/topic/subdocs}{\textsf{subdocs}}
for further related packages.
The present package differs from the above solutions in that
a document structure constructed with the conventional |\include| mechanism
just needs two extra commands at the top of every file
such that all constituent files can be compiled individually.

%%%%%%%%%%%%%%%%%%%%%%%%%%%%%%%%%%%%%%%%%%%%%%%%%%%%%%%%%%%%%%%%%%%%%%%%%%%%%%%%
%\subsection{Feature Suggestions}
%
%The following is a list of features which may be useful for future
%versions of this package:
%%
%\begin{itemize}
%\item
%\ldots
%\end{itemize}

%%%%%%%%%%%%%%%%%%%%%%%%%%%%%%%%%%%%%%%%%%%%%%%%%%%%%%%%%%%%%%%%%%%%%%%%%%%%%%%%
\subsection{Revision History}

%%%%%%%%%%%%%%%%%%%%%%%%%%%%%%%%%%%%%%%%
\paragraph{v2.0:} 2018/12/30

\begin{itemize}
\item
immediate forward processing
\item
added |\childdocby| mechanism
\item
manual restructured
\end{itemize}

%%%%%%%%%%%%%%%%%%%%%%%%%%%%%%%%%%%%%%%%
\paragraph{v1.6:} 2018/01/17

\begin{itemize}
\item
application for development of include files
\item
corrections to manual
\end{itemize}

%%%%%%%%%%%%%%%%%%%%%%%%%%%%%%%%%%%%%%%%
\paragraph{v1.5:} 2017/05/21

\begin{itemize}
\item
more complete structuring introduced
\item
|\childdocof| introduced
\item
|\childdoc| renamed to |\childdocmain|
\item
|\childredirect| renamed to |\childdocforward| and |\childdocforwardprefix|
and functionality expanded
\end{itemize}

%%%%%%%%%%%%%%%%%%%%%%%%%%%%%%%%%%%%%%%%
\paragraph{v1.0:} 2017/04/27

\begin{itemize}
\item
manual and install package
\item
first version published on CTAN
\end{itemize}

%%%%%%%%%%%%%%%%%%%%%%%%%%%%%%%%%%%%%%%%
\paragraph{v0.6:} 2017/04/26

\begin{itemize}
\item
redirection mechanism added
\end{itemize}

%%%%%%%%%%%%%%%%%%%%%%%%%%%%%%%%%%%%%%%%
\paragraph{v0.5:} 2017/04/26

\begin{itemize}
\item
functionality in definition file
\end{itemize}


%%%%%%%%%%%%%%%%%%%%%%%%%%%%%%%%%%%%%%%%%%%%%%%%%%%%%%%%%%%%%%%%%%%%%%%%%%%%%%%%
%%%%%%%%%%%%%%%%%%%%%%%%%%%%%%%%%%%%%%%%%%%%%%%%%%%%%%%%%%%%%%%%%%%%%%%%%%%%%%%%
%%%%%%%%%%%%%%%%%%%%%%%%%%%%%%%%%%%%%%%%%%%%%%%%%%%%%%%%%%%%%%%%%%%%%%%%%%%%%%%%
\appendix

\settowidth\MacroIndent{\rmfamily\scriptsize 000\ }

 \DocInput{childdoc.dtx}

\end{document}
%</driver>
% \fi
%
% %%%%%%%%%%%%%%%%%%%%%%%%%%%%%%%%%%%%%%%%%%%%%%%%%%%%%%%%%%%%%%%%%%%%%%%%%%%%%%
% %%%%%%%%%%%%%%%%%%%%%%%%%%%%%%%%%%%%%%%%%%%%%%%%%%%%%%%%%%%%%%%%%%%%%%%%%%%%%%
% \section{Sample}
%\iffalse
%<*samplemain>
%\fi
%
% The following presents a sample document
% with two chapters, two parts, a title page,
% a compile flag as well as three forwarding files to set the flag.
% It consists of eight |.tex| files:
% \begin{center}
% \begin{tabular}{ll}
% |cdocsamp.tex|&main file\\
% |cdocsch1.tex|&include file for chapter 1\\
% |cdocsch2.tex|&include file for chapter 2\\
% |cdocspt3.tex|&include file for part 3\\
% |cdocspt4.tex|&include file for part 4\\
% |cdocsdrf.tex|&forwarding file for main file in draft mode\\
% |cdocsfi1.tex|&forwarding file for final version of chapter 1\\
% |cdocsfi2.tex|&forwarding file for final version of chapter 2\\
% \end{tabular}
% \end{center}
% Each of the eight files can be compiled directly by the \LaTeX{} compiler.
%
% %%%%%%%%%%%%%%%%%%%%%%%%%%%%%%%%%%%%%%
% \paragraph{Main File.}
%
% The main file is called |cdocsamp.tex|.
%
% Load the \textsf{childdoc} definitions and
% declare the filename for the main document:
%    \begin{macrocode}
\input{childdoc.def}
\childdocmain{}
%    \end{macrocode}

% Optional override for |\version| flag:
%    \begin{macrocode}
%%\ifchilddoc\else\providecommand{\version}{draft}\fi
%    \end{macrocode}

% Define the default values for the |\version| flag
% (|final| for the main file and |draft| for childs):
%    \begin{macrocode}
\ifchilddoc
\providecommand{\version}{draft}
\else
\providecommand{\version}{final}
\fi
%    \end{macrocode}

% Load the standard document class:
%    \begin{macrocode}
\documentclass[12pt]{article}
%    \end{macrocode}

% Start the document body:
%    \begin{macrocode}
\begin{document}
%    \end{macrocode}

% Declare a title page.
% Print title, part of document being processed and version flag:
%    \begin{macrocode}
\addtocounter{page}{-1}
\begin{center}
{\LARGE\bfseries{}childdoc example\par}
\vspace{1cm}
\ifchilddoc
\ifchilddocmanual part\else chapter\fi:
`\childdocname' of `\childdocjob'\par
\else
main document: `\childdocjob'\par
\fi
version: \version\par
\end{center}
\newpage
%    \end{macrocode}

% Manually include selected file,
% otherwise process as usual:
%    \begin{macrocode}
\ifchilddocmanual
\section*{part `\childdocname'}
\input{\childdocname}
\else
%    \end{macrocode}

% Include the two chapters:
%    \begin{macrocode}
\include{cdocsch1}
\include{cdocsch2}
%    \end{macrocode}

% Include the two parts unless only chapters should be displayed:
%    \begin{macrocode}
\ifchilddoc\else
\section{part three}
\input{cdocspt3}
\section{part four}
\input{cdocspt4}
\fi
%    \end{macrocode}

% Process as usual until here:
%    \begin{macrocode}
\fi
%    \end{macrocode}

% End of document body:
%    \begin{macrocode}
\end{document}
%    \end{macrocode}
%\iffalse
%</samplemain>
%\fi
%
% %%%%%%%%%%%%%%%%%%%%%%%%%%%%%%%%%%%%%%
% \paragraph{Chapter Include Files.}
%
% The include files are called |cdocsch1.tex| and |cdocsch2.tex|.
%
%\iffalse
%<*samplechap1|samplechap2>
%\fi

% Optional override for |\version| flag:
%    \begin{macrocode}
%%\providecommand{\version}{final}
%    \end{macrocode}

% Include the main document:
%    \begin{macrocode}
\input{childdoc.def}
\childdocof{cdocsamp}
%    \end{macrocode}

%\iffalse
%</samplechap1|samplechap2>
%\fi
%
%\iffalse
%<*samplechap1>
%\fi
% Some text for chapter 1:
%    \begin{macrocode}
\section{one}
some text in chapter one
%    \end{macrocode}

%\iffalse
%</samplechap1>
%\fi
% Some text for chapter 2:
%\iffalse
%<*samplechap2>
%\fi
%    \begin{macrocode}
\section{two}
more text in chapter two
%    \end{macrocode}

%\iffalse
%</samplechap2>
%\fi
%
% %%%%%%%%%%%%%%%%%%%%%%%%%%%%%%%%%%%%%%
% \paragraph{Part Include Files.}
%
% The include files are called |cdocspt3.tex| and |cdocspt4.tex|.
%
%\iffalse
%<*samplepart3|samplepart4>
%\fi

% Optional override for |\version| flag:
%    \begin{macrocode}
%%\providecommand{\version}{final}
%    \end{macrocode}

% Include the main document:
%    \begin{macrocode}
\input{childdoc.def}
\childdocby{cdocsamp}
%    \end{macrocode}

%\iffalse
%</samplepart3|samplepart4>
%\fi
%
%\iffalse
%<*samplepart3>
%\fi
% Some text for part 3:
%    \begin{macrocode}
some text in part three
%    \end{macrocode}

%\iffalse
%</samplepart3>
%\fi
% Some text for part 4:
%\iffalse
%<*samplepart4>
%\fi
%    \begin{macrocode}
more text in part four
%    \end{macrocode}

%\iffalse
%</samplepart4>
%\fi
%
% %%%%%%%%%%%%%%%%%%%%%%%%%%%%%%%%%%%%%%
% \paragraph{Forwarding for a Complete Draft.}
%
% The following forwarding file |cdocsdrf.tex|
% compiles the main document in draft mode:
%\iffalse
%<*sampledraft>
%\fi
%    \begin{macrocode}
\def\version{draft}
\input{childdoc.def}
\childdocforward{cdocsamp}
%    \end{macrocode}

%\iffalse
%</sampledraft>
%\fi
%
% %%%%%%%%%%%%%%%%%%%%%%%%%%%%%%%%%%%%%%
% \paragraph{Forwarding for Final Version of the Chapters.}
%
% The following forwarding files |cdocsfn1.tex| and |cdocsfn2.tex|
% (with identical content)
% compile the final versions of the child documents
% |cdocsch1.tex| and |cdocsch2.tex|, respectively:
%\iffalse
%<*samplefinal>
%\fi
%    \begin{macrocode}
\def\version{final}
\input{childdoc.def}
\childdocforwardprefix[cdocsamp]{cdocsfn}{cdocsch}
%    \end{macrocode}

%\iffalse
%</samplefinal>
%\fi
%
% %%%%%%%%%%%%%%%%%%%%%%%%%%%%%%%%%%%%%%
% \paragraph{Command Line Processing.}
%
% The following three command lines generate the output files
% |cdocscld|, |cdocscl1| and |cdocscl2|
% which should be identical to
% |cdocsdrf|, |cdocsch1| and |cdocsfn2|, respectively:
% \begin{center}
% \begin{tabular}{l}
% |latex -jobname cdocscld \|\\
% |  "\def\version{draft}\input{childdoc.def}\childdocforward{cdocsamp}"|\\
% |latex -jobname cdocscl1 \|\\
% |  "\input{childdoc.def}\childdocforward[cdocsamp]{cdocsch1}"|\\
% |latex -jobname cdocscl2 \|\\
% |  "\def\version{final}\input{childdoc.def}\childdocforward{cdocsch2}"|
% \end{tabular}
% \end{center}
% Note that the trailing backslash on each first line
% merely continues the input to the second line
% (for convenient cut ant paste).
% Furthermore, the command |latex| can be replaced by any
% of its alternative versions such as |pdflatex|.
%
% %%%%%%%%%%%%%%%%%%%%%%%%%%%%%%%%%%%%%%%%%%%%%%%%%%%%%%%%%%%%%%%%%%%%%%%%%%%%%%
% %%%%%%%%%%%%%%%%%%%%%%%%%%%%%%%%%%%%%%%%%%%%%%%%%%%%%%%%%%%%%%%%%%%%%%%%%%%%%%
% \section{Implementation}
%\iffalse
%<*package>
%\fi
%
% This section describes the definitions file |childdoc.def|.

% The definitions cannot be loaded using |\usepackage| or |\RequirePackage|
% which has a mechanism to prevent loading a style file more than once.
% When loading the definitions by means of |\input|
% multiple instances have to be prevented manually:
%\iffalse
%This code needs to be before the `\ProvidesFile' directive
%which is defined at the beginning of this file.
%Therefore it is also placed there and commented out here.
%</package>
%<*discard>
%\fi
%    \begin{macrocode}
\ifdefined\childdocmain\endinput\fi
%    \end{macrocode}
%\iffalse
%</discard>
%<*package>
%\fi
%
% \macro{\ifchilddoc}
% \macro{\ifchilddocmanual}
% The conditional |\ifchilddoc| tells whether a
% child (true) or main (false) document is being compiled.
% The conditional |\ifchilddocmanual| tells whether
% the |\includeonly| mechanism is used (false) or
% the selection of child files must be performed manually (true).
% The definitions initialise to false:
%    \begin{macrocode}
\newif\ifchilddoc
\newif\ifchilddocmanual
%    \end{macrocode}

% \macro{\childdocname}
% \macro{\childdocjob}
% The macro |\childdocname| stores the name of the main document
% to be compiled. The macro |\childdocjob| stores the name of
% the document on which the \LaTeX{} compiler was originally invoked.
% The content of |\jobname| cannot be compared
% to filenames specified in the source due to different catcodes.
% The following code rescans |\jobname|, stores the result
% in |\childdocname| and saves a copy in |\childdocjob|:
%    \begin{macrocode}
\edef\childdocname{\scantokens\expandafter{\jobname\noexpand}}
\let\childdocjob\childdocname
%    \end{macrocode}

% \macro{\childdocdisable}
% The macro |\childdocdisable| prevents the main file
% from being processed more than once.
% At this stage, the main document command |\childdocmain|
% is assumed to be called once again where it should do nothing.
% Any subsequent call to it should prevent
% a secondary processing of the main document
% It overwrites the forwarding commands
% |\childdocof| and |\childdocforward|
% with empty macros to prevent further inclusions of the main document:
%    \begin{macrocode}
\newcommand{\childdocdisable}
{
  \renewcommand{\childdocmain}[1]{\renewcommand{\childdocmain}[1]{\endinput}}
  \renewcommand{\childdocof}[1]{}
  \renewcommand{\childdocby}[2][]{}
  \renewcommand{\childdocforward}[2][]{}
  \renewcommand{\childdocdisable}{}
}
%    \end{macrocode}

% \macro{\childdocmain}
% The macro |\childdocmain| is to be called at the top of the main file
% with nothing or the main filename (without extension) as argument.
% First, it breaks loops.
% If the argument is not empty and does not match |\childdocname|
% (which is set by the first inclusion of |childdoc.def|),
% |\ifchilddoc| is set to true, |\includeonly| is applied to the child file
% and |\jobname| is set to the main file
% (for proper handling of |.aux| files):
%    \begin{macrocode}
\newcommand{\childdocmain}[1]
{
  \childdocdisable\childdocmain{}
  \if?#1?\else
    \begingroup
      \def\childdoctmp{#1}
      \ifx\childdoctmp\childdocname
        \def\childdoctmp{}
      \else
        \def\childdoctmp
        {
          \childdoctrue
          \includeonly{\childdocname}
          \def\childdocjob{#1}
          \def\jobname{#1}
        }
      \fi
      \expandafter
    \endgroup
    \childdoctmp
  \fi
}
%    \end{macrocode}

% \macro{\childdocof}
% The command |\childdocof| redirects
% compilation to the main file |#1|.
%    \begin{macrocode}
\newcommand{\childdocof}[1]
{
  \childdocdisable
  \childdoctrue
  \includeonly{\childdocname}
  \def\jobname{#1}
  \def\childdocjob{#1}
  \input{#1}
}
%    \end{macrocode}

% \macro{\childdocby}
% The command |\childdocby| ....
%    \begin{macrocode}
\newcommand{\childdocby}[2][]
{
  \childdocdisable
  \childdoctrue
  \childdocmanualtrue
  \if?#1?\else
    \def\jobname{#2}
  \fi
  \def\childdocjob{#2}
  \input{#2}
  \endinput
}
%    \end{macrocode}

% \macro{\childdocforward}
% The command |\childdocforward| redirects
% compilation to the main file or
% (if the optional argument is given) a child file.
% Parameters are set as if the main file
% or a child file starting with |\childdocof| was compiled.
% Then compilation is handed over to the main file:
%    \begin{macrocode}
\newcommand{\childdocforward}[2][]
{
  \begingroup
    \if?#1?
      \def\childdoctmp
      {
        \def\childdocname{#2}
        \def\childdocjob{#2}
        \def\jobname{#2}
        \input{#2}
        \endinput
      }
    \else
      \def\childdoctmp
      {
        \childdocdisable
        \def\childdocname{#2}
        \childdoctrue
        \includeonly{#2}
        \def\childdocjob{#1}
        \def\jobname{#1}
        \input{#1}
        \endinput
      }
    \fi
    \expandafter
  \endgroup
  \childdoctmp
}
%    \end{macrocode}

% \macro{\childdocforwardprefix}
% The command |\childdocforwardprefix| redirects
% compilation to the main or a child file by means of a pattern.
% The prefix |#1| in the current filename is replaced by |#2|
% and the suffix of the current filename is kept
% (it is assumed that the filename does not contain the substring `|~~~|'
% which is used as a delimiter).
% Compilation is handed over to the new file by |\childdocforward|:
%    \begin{macrocode}
\newcommand{\childdocforwardprefix}[3][]
{
  \begingroup
    \def\childdocextract #2##1~~~{\def\childdoctmp{\childdocforward[#1]{#3##1}}}
    \expandafter\childdocextract\childdocname~~~
    \expandafter
  \endgroup
  \childdoctmp
}
%    \end{macrocode}

% \macro{\childdoc}
% The deprecated macro |\childdoc| is a legacy version of |\childdocmain|:
%    \begin{macrocode}
\newcommand{\childdoc}{\childdocmain}
%    \end{macrocode}

% \macro{\childdocredirect}
% The deprecated macro |\childdocredirect| is a legacy version
% of |\childdocforward| and |\childdocforwardprefix|:
%    \begin{macrocode}
\newcommand{\childdocredirect}[2][]
{
  \begingroup
    \if?#1?
      \def\childdoctmp{\childdocforward{#2}}
    \else
      \def\childdoctmp{\childdocforwardprefix{#1}{#2}}
    \fi
    \expandafter
  \endgroup
  \childdoctmp
}
%    \end{macrocode}

%\iffalse
%</package>
%\fi
%
\endinput

\childdocby{cdocsamp}
%    \end{macrocode}

%\iffalse
%</samplepart3|samplepart4>
%\fi
%
%\iffalse
%<*samplepart3>
%\fi
% Some text for part 3:
%    \begin{macrocode}
some text in part three
%    \end{macrocode}

%\iffalse
%</samplepart3>
%\fi
% Some text for part 4:
%\iffalse
%<*samplepart4>
%\fi
%    \begin{macrocode}
more text in part four
%    \end{macrocode}

%\iffalse
%</samplepart4>
%\fi
%
% %%%%%%%%%%%%%%%%%%%%%%%%%%%%%%%%%%%%%%
% \paragraph{Forwarding for a Complete Draft.}
%
% The following forwarding file |cdocsdrf.tex|
% compiles the main document in draft mode:
%\iffalse
%<*sampledraft>
%\fi
%    \begin{macrocode}
\def\version{draft}
% \iffalse
%
% childdoc.dtx Copyright (C) 2017-2018 Niklas Beisert
%
% This work may be distributed and/or modified under the
% conditions of the LaTeX Project Public License, either version 1.3
% of this license or (at your option) any later version.
% The latest version of this license is in
%   http://www.latex-project.org/lppl.txt
% and version 1.3 or later is part of all distributions of LaTeX
% version 2005/12/01 or later.
%
% This work has the LPPL maintenance status `maintained'.
%
% The Current Maintainer of this work is Niklas Beisert.
%
% This work consists of the files childdoc.dtx and childdoc.ins
% and the derived files childdoc.def and cdocsamp.tex with
% cdocsch1.tex, cdocsch2.tex, cdocsdrf.tex, cdocsfn1.tex, cdocsfn2.tex.
%
%<package>\ifdefined\childdocmain\endinput\fi
%<package>\ProvidesFile{childdoc.def}[2018/12/30 v2.0 child document driver]
%<samplemain>\ProvidesFile{cdocsamp.tex}[2018/12/30 v2.0 sample for childdoc]
%<*driver>
%\ProvidesFile{childdoc.drv}[2018/12/30 v2.0 childdoc reference manual file]
\PassOptionsToClass{10pt,a4paper}{article}
\documentclass{ltxdoc}

\usepackage[margin=35mm]{geometry}
\usepackage{hyperref}
\usepackage{hyperxmp}
\usepackage[usenames]{color}

\hypersetup{colorlinks=true}
\hypersetup{pdfstartview=FitH}
\hypersetup{pdfpagemode=UseNone}
\hypersetup{pdfsource={}}
\hypersetup{pdflang={en-UK}}
\hypersetup{pdfcopyright={Copyright 2017-2018 Niklas Beisert.
  This work may be distributed and/or modified under the
  conditions of the LaTeX Project Public License, either version 1.3
  of this license or (at your option) any later version.}}
\hypersetup{pdflicenseurl={http://www.latex-project.org/lppl.txt}}
\hypersetup{pdfcontactaddress={ETH Zurich, ITP, HIT K,
  Wolfgang-Pauli-Strasse 27}}
\hypersetup{pdfcontactpostcode={8093}}
\hypersetup{pdfcontactcity={Zurich}}
\hypersetup{pdfcontactcountry={Switzerland}}
\hypersetup{pdfcontactemail={nbeisert@itp.phys.ethz.ch}}
\hypersetup{pdfcontacturl={http://people.phys.ethz.ch/\xmptilde nbeisert/}}

\newcommand{\secref}[1]{\hyperref[#1]{section \ref*{#1}}}

\parskip1ex
\parindent0pt
\let\olditemize\itemize
\def\itemize{\olditemize\parskip0pt}

\begin{document}

\title{The \textsf{childdoc} Package}
\hypersetup{pdftitle={The childdoc Package}}
\author{Niklas Beisert\\[2ex]
  Institut f\"ur Theoretische Physik\\
  Eidgen\"ossische Technische Hochschule Z\"urich\\
  Wolfgang-Pauli-Strasse 27, 8093 Z\"urich, Switzerland\\[1ex]
  \href{mailto:nbeisert@itp.phys.ethz.ch}
  {\texttt{nbeisert@itp.phys.ethz.ch}}}
\hypersetup{pdfauthor={Niklas Beisert}}
\hypersetup{pdfsubject={Manual for the LaTeX2e Package childdoc}}
\date{30 December 2018, \textsf{v2.0}}
\maketitle

\begin{abstract}\noindent
\textsf{childdoc} is a \LaTeXe{} package
that enables the direct compilation
of document sections included by |\include|
to individual files.
\end{abstract}

\begingroup
\parskip0ex
\tableofcontents
\endgroup

%%%%%%%%%%%%%%%%%%%%%%%%%%%%%%%%%%%%%%%%%%%%%%%%%%%%%%%%%%%%%%%%%%%%%%%%%%%%%%%%
%%%%%%%%%%%%%%%%%%%%%%%%%%%%%%%%%%%%%%%%%%%%%%%%%%%%%%%%%%%%%%%%%%%%%%%%%%%%%%%%
\section{Introduction}

\LaTeX{} provides a mechanism to structure a large document (such as a book)
into a main file and several child files (containing the chapters)
using the |\include| command.
This mechanism is beneficial for documents
which span hundreds of pages in order to
make the source file(s) more manageable.
Moreover, compilation can be restricted to
selected child files by means of the |\includeonly| command.
The latter feature can be used to reduce the compilation time while editing
(this was significantly more useful in the earlier days of \LaTeX{})
or to generate a smaller document which is easier to navigate.
Another application of |\includeonly| is to generate
documents consisting of selected parts of the complete document.

However, there are a few drawbacks of the plain |\include| mechanism:
\begin{itemize}
\item
The child files cannot be compiled on their own,
they can only be compiled via the main file.
A naive editing environment
(such as a text editor with an option
to have the current file processed by \LaTeX)
may require one to switch to the main file before compiling;
attempting to compile the child file produces errors.
\item
The main file must be modified (each time)
to adjust the |\includeonly| command
to the present needs. This easily leaves the main file in a messy state.
\item
The generated document will always carry the filename
of the main document. This is inconvenient if
several child files are to be compiled and
to be kept for distribution.
\end{itemize}

The present package provides a simple interface
to make child files individually compilable by \LaTeX{}.
Compiling a child file then has the same effect as compiling
the main file with an |\includeonly| command
to select the appropriate child.
Moreover the generated document will carry the name of the child
rather than the main file.
This resolves all three above issues.

This feature is meant to make the editing of books,
thesis documents and lecture notes somewhat more convenient.
However, the package can also be used efficiently for
composing a series of documents (such as exercise sheets)
which are typically distributed individually.
It then assists the author in generating the individual documents
(potentially in different versions)
as well as a document containing the collected series.
Another application is in developing style files
or other kinds of included material
where compilation of the style file could redirect
to a sample or test file.

%%%%%%%%%%%%%%%%%%%%%%%%%%%%%%%%%%%%%%%%%%%%%%%%%%%%%%%%%%%%%%%%%%%%%%%%%%%%%%%%
%%%%%%%%%%%%%%%%%%%%%%%%%%%%%%%%%%%%%%%%%%%%%%%%%%%%%%%%%%%%%%%%%%%%%%%%%%%%%%%%
\section{Usage}

First of all, the package \textsf{childdoc} is \emph{not} a standard
\LaTeXe{} |.sty| style file! Therefore it needs to be invoked in
a non-standard way.

%%%%%%%%%%%%%%%%%%%%%%%%%%%%%%%%%%%%%%%%%%%%%%%%%%%%%%%%%%%%%%%%%%%%%%%%%%%%%%%%
\subsection{Included Files}
\label{sec:include}

%%%%%%%%%%%%%%%%%%%%%%%%%%%%%%%%%%%%%%%%
\DescribeMacro{\childdocmain}
To use the package, add the commands
\begin{center}
\begin{tabular}{l}
|\input{childdoc.def}|\\
|\childdocmain{}|\\
\end{tabular}
\end{center}
at the very top of the main \LaTeX{} file,
in particular \emph{before} the |\documentclass| statement!
The argument of |\childdocmain| should be left empty
(but it must be present).

%%%%%%%%%%%%%%%%%%%%%%%%%%%%%%%%%%%%%%%%
\DescribeMacro{\childdocof}
Furthermore, add the commands
\begin{center}
\begin{tabular}{l}
|\input{childdoc.def}|\\
|\childdocof{|\textit{main}|}|\\
\end{tabular}
\end{center}
at the top of every child file \textit{child}
which is included by |\include{|\textit{child}|}|
from within the main file
(or at least for those files to be compiled individually).
The argument \textit{main} must be the filename of the main file.

There are a couple of
considerations in setting up the main and child documents:

%%%%%%%%%%%%%%%%%%%%%%%%%%%%%%%%%%%%%%%%
\paragraph{Restrictions.}

Please note the following restrictions:
\begin{itemize}
\item
|\childdocmain| must be called with one argument \textit{main}
to ensure compatibility with earlier version of the package.
It must either be empty (|\childdocmain{}|)
or precisely match the filename of the main file in which it is specified.
See \secref{sec:detection} for further information.
\item
The filename \textit{main} must be specified without the |.tex| extension.
\item
The filename \textit{main} is case sensitive
(even in case-insensitive file systems)
due to internal string comparison.
\item
The argument \textit{main} should be fully expanded, it cannot be a macro.
\item
Subdirectories and special characters should be avoided in filenames.
\item
The command |\childdocmain{|\textit{main}|}| must be followed by a whitespace.
It should not be followed immediately by another command
or by a comment mark `|%|'.
This is because the \TeX{} parser reads the token immediately following
the argument of |\childdocmain| and puts it
at the beginning of every child section;
however, a white\-space is ignored.
\end{itemize}

%%%%%%%%%%%%%%%%%%%%%%%%%%%%%%%%%%%%%%%%
\paragraph{Content of Main File.}

It is advisable to place all content in the child files included by |\include|.
Any output contained in the main file will appear in all child documents
unless suppressed manually;
it cannot be suppressed automatically by the |\includeonly| directive
and thus should normally be avoided.
A method to include some content in the main file
by means of conditional processing is described in \secref{sec:conditional}.

%%%%%%%%%%%%%%%%%%%%%%%%%%%%%%%%%%%%%%%%
\paragraph{Page Numbering.}

When only a part of the document is compiled,
the appropriate numbering of pages
(as well as other status parameters)
is determined from the |.aux| files.
The latter contain information from previous passes.
However this information needs to propagate through
all intermediate child documents.
Therefore the page numbering in child documents may well
be inconsistent until the complete document is compiled at least once.

A useful (if unconventional) way to always ensure a consistent
page numbering is to restart the numbering in each child document
and denote the pages by `\textit{child}|.|\textit{page}'
where \textit{child} represents the chapter/section number of the child file.
This can be achieved by the command
|\numberwithin{page}{|\textit{child}|}|
of the \textsf{amsmath} package
where \textit{child} can be |chapter| or |section|
depending on the chosen structuring.
Alternatively, one can modify the macro |\thepage| appropriately
and reset the counter |page| at the start of each child file.

%%%%%%%%%%%%%%%%%%%%%%%%%%%%%%%%%%%%%%%%%%%%%%%%%%%%%%%%%%%%%%%%%%%%%%%%%%%%%%%%
\subsection{Conditional Processing}
\label{sec:conditional}

The package provides a mechanism to compile different versions
of a document. To customise the versions further some conditional processing
can come in handy to distinguish which version is being compiled.
The package provides two macros to describe the compilation context:

%%%%%%%%%%%%%%%%%%%%%%%%%%%%%%%%%%%%%%%%
\DescribeMacro{\ifchilddoc}
The conditional |\ifchilddoc| distinguishes between the compilation of
child documents and the main document:
%
\begin{center}
|\ifchilddoc |\textit{child-code}| |[|\||else |\textit{main-code}]| \||fi|
\end{center}

%%%%%%%%%%%%%%%%%%%%%%%%%%%%%%%%%%%%%%%%
\DescribeMacro{\childdocname}
\DescribeMacro{\childdocjob}
The macro |\childdocname| contains the filename (without extension)
of the main or child file being processed.
Note that |\childdocjob| will always contain the name of the main file.

%%%%%%%%%%%%%%%%%%%%%%%%%%%%%%%%%%%%%%%%
\paragraph{Title Page.}

Conditional processing can be used to include a title or banner page
in the main document when proper precautions are taken.
Importantly, the code in the main file should ensure that the page counter
(as well as other status parameters which are stored in the |.aux| files)
takes the same value after the conditional processing.
Otherwise the page numbers may take divergent values
depending on which part is compiled.

For example, a title page could be declared by:
%
\begin{center}
\begin{tabular}{l}
|\ifchilddoc\||else|\\
|\addtocounter{page}{-1}|\\
\textit{code for title page}\\
|\newpage|\\
|\||fi|
\end{tabular}
\end{center}
%
A banner page for the child documents can be generated by:
%
\begin{center}
\begin{tabular}{l}
|\ifchilddoc|\\
|\addtocounter{page}{-1}|\\
\textit{code for banner page}\\
|\newpage|\\
|\||fi|
\end{tabular}
\end{center}
%
Here one could write a message such as:
\begin{center}
|This is the part \childdocname{} of \childdocjob{}.|
\end{center}

%%%%%%%%%%%%%%%%%%%%%%%%%%%%%%%%%%%%%%%%%%%%%%%%%%%%%%%%%%%%%%%%%%%%%%%%%%%%%%%%
\subsection{Flags}
\label{sec:flags}

The package makes it easy to generate different versions
of the main or child documents.
To this end compilation flags can be defined
and assigned different default values.
They will be particularly useful in conjunction
with the forwarding mechanism described in \secref{sec:forward}.

For example, it may be useful to have a flag |\version|
which can be set to |draft| or |final|.
The document source will contain some conditional code
depending on the value of |\version|.
Suppose further, the flag should default to |final| for the main file
and to |draft| for child files
which is a natural assignment for editing the document.
This is achieved by placing the following code
in the preamble of the main document
(below the |\childdocmain| directive):
%
\begin{center}
\begin{tabular}{l}
|\ifchilddoc|\\
|\providecommand{\version}{draft}|\\
|\||else|\\
|\providecommand{\version}{final}|\\
|\||fi|
\end{tabular}
\end{center}
%
The definition by |\providecommand| makes sure
that previous definitions are not overwritten.
Further statements |\providecommand{\version}{...}|
can thus be added before the above code to override it.

For the main file, one might add a line
(between |\childdocmain| and the above block)
%
\begin{center}
|%\ifchilddoc\||else\providecommand{\version}{draft}\||fi|
\end{center}
%
which can be uncommented to produce a draft version.
Likewise one can add a line to the very top of a child file
(above the |\childdocof{|\textit{main}|}| directive)
%
\begin{center}
|%\providecommand{\version}{final}|
\end{center}
%
which can be uncommented to produce the final version of this child document.

%%%%%%%%%%%%%%%%%%%%%%%%%%%%%%%%%%%%%%%%%%%%%%%%%%%%%%%%%%%%%%%%%%%%%%%%%%%%%%%%
\subsection{Forwarding}
\label{sec:forward}

Different versions of the main or child documents
using compilation flags as described in \secref{sec:flags}
can be (permanently) stored in different files
for convenient compilation, viewing and distribution.
To this end, the package defines a command
to pass on compilation to a different file:

%%%%%%%%%%%%%%%%%%%%%%%%%%%%%%%%%%%%%%%%
\DescribeMacro{\childdocforward}
The command |\childdocforward| redirects processing to
another source file:
%
\begin{center}
\begin{tabular}{l}
|\input{childdoc.def}|\\
|\childdocforward[|\textit{main}|]{|\textit{dest}|}|\\
\end{tabular}
\end{center}
%
The argument \textit{dest} is the destination file
(without extension).
It should be the main file or one of the child files.
Note that further \textsf{childdoc} directives
such as |\childdocof| and |\childdocforward|
in the indicated file will be processed in this form.
The optional argument \textit{main}
passes on directly to the main file \textit{main}
while pretending to compile the child \textit{dest}.
This form behaves as if \textit{dest}
issues |\childdocof{|\textit{main}|}| right away,
and no further \textsf{childdoc} directives will be processed.

%%%%%%%%%%%%%%%%%%%%%%%%%%%%%%%%%%%%%%%%
\DescribeMacro{\...prefix}
In the alternative form |\childdocforwardprefix|,
%
\begin{center}
\begin{tabular}{l}
|\input{childdoc.def}|\\
|\childdocforwardprefix[|\textit{main}|]{|\textit{prefix}|}{|\textit{dest}|}|
\end{tabular}
\end{center}
%
the destination file is determined by a pattern
depending on the current file:
To make this work, the current file must be called
`{\textit{prefix}\hspace{0.2em}\textit{suffix}}'
with \textit{prefix} matching precisely the argument.
Processing is then passed on to the file
`{\textit{dest}\hspace{0.2em}\textit{suffix}}'.
Surely, the same effect is achieved by
directly specifying the
argument `{\textit{dest}\hspace{0.2em}\textit{suffix}}'
in the first form.
However, that requires to set up a different file
for each child. With the alternative form of the command
all these files can have exactly the same content
which simplifies setting them up and maintaining them.

For example, the following file |draft.tex|
with a compilation flag |\version| as described in \secref{sec:flags}
compiles the main document as a draft:
%
\begin{center}
\begin{tabular}{l}
|\def\version{draft}|\\
|\input{childdoc.def}|\\
|\childdocforward{|\textit{main}|}|
\end{tabular}
\end{center}
%
Likewise, the following files |final|\textit{nn}|.tex|
compile the final version of the child document
|child|\textit{nn}|.tex|:
%
\begin{center}
\begin{tabular}{l}
|\def\version{final}|\\
|\input{childdoc.def}|\\
|\childdocforwardprefix{final}{child}|
\end{tabular}
\end{center}
%

Note that when several versions of a main file and/or of each child file
are to be generated, it may be convenient to set up a |Makefile| or
shell script to automatise the process.

%%%%%%%%%%%%%%%%%%%%%%%%%%%%%%%%%%%%%%%%%%%%%%%%%%%%%%%%%%%%%%%%%%%%%%%%%%%%%%%%
\subsection{Command Line Processing}
\label{sec:commandline}

The effect of redirection files can also be achieved by invoking
the \LaTeX{} compiler with a more elaborate command line.
Most conveniently this should be done as part
of a shell script or a |Makefile|.

When using \textsf{childdoc} in the main file, the following
command lines effectively perform a redirection
(note that depending on the shell being used,
backslashes may have to be doubled: `|\|' $\to$ `|\\|'):
%
\begin{center}
|... -jobname "|\textit{target}|" |\\|"|[\textit{flags}]%
|\input{childdoc.def}\childdocforward[|\textit{main}|]{|\textit{dest}|}"|
\end{center}
%
Here \textit{target} is the name of the output file,
\textit{main} is the name of the main file
and \textit{dest} is the name of the main or child file to be processed
(all filenames without extensions).
The optional argument \textit{main} can be omitted
if \textit{main} matches \textit{dest}.
Optionally, compilation \textit{flags} can be defined via |\def| commands.
This command line makes the \TeX{} engine believe
it is compiling the file \textit{target}
whose content is specified as the latter parameter.
The provided code then forwards the processing to
\textit{main} or \textit{dest} as described in \secref{sec:forward}.

%%%%%%%%%%%%%%%%%%%%%%%%%%%%%%%%%%%%%%%%%%%%%%%%%%%%%%%%%%%%%%%%%%%%%%%%%%%%%%%%
\subsection{Include by Input}
\label{sec:input}

Including child documents by |\include| has some restrictions by design.
Most notably, the content of a child document always occupies
its own set of pages; pages cannot be shared between child documents.
Usually, this behaviour makes perfect sense
because each child document contain an essential part of the document.
However, in some situations it may be desirable to compose
a document from a collection of parts
without having mandatory page breaks between then.
For this case, the package
provides a mechanism to include parts
by |\input| which can also be processed individually.
However, by construction this mechanism
requires manual handling of the content to be output.

%%%%%%%%%%%%%%%%%%%%%%%%%%%%%%%%%%%%%%%%
\DescribeMacro{\ifchilddocmanual}
The main file should be prepared as usual, see \secref{sec:include}.
However, the document body must make a distinction
between processing of an individual part and of the main document, e.g.:
%
\begin{center}
\begin{tabular}{l}
|\ifchilddocmanual|\\
|\input{\childdocname}|\\
|\||else|\\
\textit{document body with }|\input{|\textit{part}|}|\\
|\||fi|
\end{tabular}
\end{center}
%
The conditional |\ifchilddocmanual| is true whenever
a part to be included by |\input| is being compiled,
and the name of the part is stored in |\childdocname|.

%%%%%%%%%%%%%%%%%%%%%%%%%%%%%%%%%%%%%%%%
\DescribeMacro{\childdocby}
Each part to be included by |\input| should start with:
%
\begin{center}
\begin{tabular}{l}
|\input{childdoc.def}|\\
|\childdocby{|\textit{main}|}|\\
\end{tabular}
\end{center}
%
The directive |\childdocby| is similar to |\childdocof|
described in \secref{sec:include},
but the subsequent selection of content must be done manually.
To that end, both |\ifchilddoc| and |\ifchilddocmanual|
will be true upon processing of a part,
and the name of the part is stored in |\childdocname|.
Note that |\jobname| will be set to the filename of the current part
so that each part receives an individual |.aux| file
that does not interfere with the |.aux| file(s) of the main document.
This behaviour can be altered by the alternative form
|\childdocby[*]{|\textit{main}|}| (with a non-empty optional argument)
which uses the |.aux| file of the main document
by setting |\jobname| to \textit{main}.

%%%%%%%%%%%%%%%%%%%%%%%%%%%%%%%%%%%%%%%%%%%%%%%%%%%%%%%%%%%%%%%%%%%%%%%%%%%%%%%%
\subsection{Driver Development}
\label{sec:driver}

The \textsf{childdoc} mechanism can also be use for the development
of definition files such as \LaTeX{} styles or classes.
This case differs from the above setup with multiple parts
included by |\include| in that no |\includeonly| should be invoked.
This can be achieved by starting the include file
(before |\ProvidesPackage|) with:
%
\begin{center}
\begin{tabular}{l}
|\input{childdoc.def}|\\
|\childdocforward{|\textit{main}|}|\\
\end{tabular}
\end{center}
%
or alternatively with:
%
\begin{center}
\begin{tabular}{l}
|\input{childdoc.def}|\\
|\childdocby{|\textit{main}|}|\\
\end{tabular}
\end{center}
%
Both forms have slightly different effects as described above.
The main file is prepared as usual, see \secref{sec:include}.

%%%%%%%%%%%%%%%%%%%%%%%%%%%%%%%%%%%%%%%%%%%%%%%%%%%%%%%%%%%%%%%%%%%%%%%%%%%%%%%%
\subsection{Legacy Detection}
\label{sec:detection}

The directive |\childdocmain| in the main file can detect
whether the complete document or merely a child is to be compiled
even without using the directive |\childdocof|.
This method is deprecated because it is less robust
and there is no compelling reason to use it;
it is merely provided for backward compatibility
and it may be removed in future versions.

If the detection mechanism is to be used,
it is mandatory to correctly specify
the filename of the main file as the argument of |\childdocmain|:
%
\begin{center}
\begin{tabular}{l}
|\input{childdoc.def}|\\
|\childdocmain{|\textit{main}|}|\\
\end{tabular}
\end{center}
%
If |\jobname| does not match the argument \textit{main} of |\childdocmain|,
it is assumed that |\jobname| points to the child file to be compiled.
When using |\childdocmain| with the main file specified as argument,
it suffices to start a child file
with just |\input{|\textit{main}|}|
without loading of the package and using |\childdocof|.
If instead all processing is done
with the appropriate \textsf{childdoc} directives,
the argument of \textit{main} of |\childdocmain| can be empty.

An alternative version of the command line processing described
in \secref{sec:commandline} using the detection mechanism reads:
%
\begin{center}
|... -jobname "|\textit{target}|" "|[\textit{flags}]%
[|\def\jobname{|\textit{dest}|}|]|\input{|\textit{main}|}"|
\end{center}

%%%%%%%%%%%%%%%%%%%%%%%%%%%%%%%%%%%%%%%%%%%%%%%%%%%%%%%%%%%%%%%%%%%%%%%%%%%%%%%%
\subsection{Manual Code}
\label{sec:manual}

In case one cannot be certain whether the definitions file |childdoc.def|
is installed on the target \TeX{} distribution
and one prefers not to ship it,
it is conceivable to paste a few relevant commands into the sources.

To that end, drop all statements |\input{childdoc.def}|
and perform the replacements as outlined below.
Instead of |\childdocmain{|\textit{main}|}| add the following code
to the top of the main file:
%
\begin{center}
\begin{tabular}{l}
|\||ifdefined\childdocname\endinput\||fi\newif\ifchilddoc|\\
|\edef\childdocname{\scantokens\expandafter{\jobname\noexpand}}|\\
|\def\childdocmain{|\textit{main}|}\||ifx\childdocmain\childdocname\||else|\\
|\childdoctrue\includeonly{\childdocname}\let\jobname\childdocmain\||fi|\\
\end{tabular}
\end{center}
%
Instead of |\childdocof{|\textit{main}|}| just include the main file
at the top of each child file:
%
\begin{center}
|\input{|\textit{main}|}|
\end{center}
%
A simple redirection |\childdocforward{|\textit{dest}|}| is achieved by:
%
\begin{center}
|\def\jobname{|\textit{dest}|}\input{\jobname}|
\end{center}
%
The redirection with prefix
|\childdocforwardprefix[|\textit{prefix}|]{|\textit{dest}|}|
is accomplished by:
%
\begin{center}
\begin{tabular}{l}
|{\edef\jobname{\scantokens\expandafter{\jobname\noexpand}}|\\
|\def\redirectjob |\textit{prefix}|#1~~~{\gdef\jobname{|\textit{dest}|#1}}|\\
|\expandafter\redirectjob\jobname~~~}\input{\jobname}|
\end{tabular}
\end{center}

In an alternative approach,
child documents can be compiled by a specific command line
without additional code or specific definitions:
%
\begin{center}
|... -jobname "|\textit{target}|" "|[\textit{flags}]%
|\includeonly{|\textit{dest}|}\input{|\textit{main}|}"|
\end{center}
%

%%%%%%%%%%%%%%%%%%%%%%%%%%%%%%%%%%%%%%%%%%%%%%%%%%%%%%%%%%%%%%%%%%%%%%%%%%%%%%%%
%%%%%%%%%%%%%%%%%%%%%%%%%%%%%%%%%%%%%%%%%%%%%%%%%%%%%%%%%%%%%%%%%%%%%%%%%%%%%%%%
\section{Information}

%%%%%%%%%%%%%%%%%%%%%%%%%%%%%%%%%%%%%%%%%%%%%%%%%%%%%%%%%%%%%%%%%%%%%%%%%%%%%%%%
\subsection{Copyright}

Copyright \copyright{} 2017--2018 Niklas Beisert

This work may be distributed and/or modified under the
conditions of the \LaTeX{} Project Public License, either version 1.3
of this license or (at your option) any later version.
The latest version of this license is in
  \url{http://www.latex-project.org/lppl.txt}
and version 1.3 or later is part of all distributions of \LaTeX{}
version 2005/12/01 or later.

This work has the LPPL maintenance status `maintained'.

The Current Maintainer of this work is Niklas Beisert.

This work consists of the files |README.txt|, |childdoc.ins| and |childdoc.dtx|
as well as the derived files |childdoc.def|, |cdocsamp.tex|
with |cdocsch1.tex|, |cdocsch2.tex|, |cdocspt3.tex|, |cdocspt4.tex|,
|cdocsdrf.tex|, |cdocsfn1.tex|, |cdocsfn2.tex|
as well as |childdoc.pdf|.

%%%%%%%%%%%%%%%%%%%%%%%%%%%%%%%%%%%%%%%%%%%%%%%%%%%%%%%%%%%%%%%%%%%%%%%%%%%%%%%%
\subsection{Files and Installation}

The package consists of the files:
%
\begin{center}
\begin{tabular}{ll}
    |README.txt|   & readme file \\
    |childdoc.ins| & installation file \\
    |childdoc.dtx| & source file \\
    |childdoc.def| & definition file \\
    |cdocsamp.tex| & sample main file \\
    |cdocsch1.tex| & sample include file \\
    |cdocsch2.tex| & sample include file \\
    |cdocspt3.tex| & sample part file \\
    |cdocspt4.tex| & sample part file \\
    |cdocsdrf.tex| & sample redirection file \\
    |cdocsfn1.tex| & sample redirection file \\
    |cdocsfn2.tex| & sample redirection file \\
    |childdoc.pdf| & manual
\end{tabular}
\end{center}
%
The distribution consists of the files
|README.txt|, |childdoc.ins| and |childdoc.dtx|.
%
\begin{itemize}
\item
Run (pdf)\LaTeX{} on |childdoc.dtx|
to compile the manual |childdoc.pdf| (this file).
\item
Run \LaTeX{} on |childdoc.ins| to create the definitions file |childdoc.def|
and the sample |cdocsamp.tex| with include files
|cdocsch1.tex|, |cdocsch2.tex|, |cdocspt3.tex|, |cdocspt4.tex|,
|cdocsdrf.tex|, |cdocsfn1.tex|, |cdocsfn2.tex|.
Then copy the file |childdoc.def| to an appropriate directory of your \LaTeX{}
distribution, e.g.\ \textit{texmf-root}|/tex/latex/childdoc|.
\end{itemize}

%%%%%%%%%%%%%%%%%%%%%%%%%%%%%%%%%%%%%%%%%%%%%%%%%%%%%%%%%%%%%%%%%%%%%%%%%%%%%%%%
\subsection{Related CTAN Packages}

There are several other packages which offer a similar functionality:
%
\begin{itemize}
\item
The packages
\href{http://ctan.org/pkg/docmute}{\textsf{docmute}},
\href{http://ctan.org/pkg/includex}{\textsf{includex}} and
\href{http://ctan.org/pkg/standalone}{\textsf{standalone}}
provide commands to include only the document body of
a child file thus allowing both files to be compiled individually.
\item
The packages \href{http://ctan.org/pkg/subdocs}{\textsf{subdocs}}
and \href{http://ctan.org/pkg/subfiles}{\textsf{subfiles}}
provide structures in which the main and child documents can be
encapsulated and allowing them to be compiled individually.
The inclusion mechanism is different from the conventional |\include|.
\item
The package \href{http://ctan.org/pkg/combine}{\textsf{combine}}
is an elaborate solution to combine several documents into one.
\end{itemize}
%
See also the CTAN topic \href{http://ctan.org/topic/subdocs}{\textsf{subdocs}}
for further related packages.
The present package differs from the above solutions in that
a document structure constructed with the conventional |\include| mechanism
just needs two extra commands at the top of every file
such that all constituent files can be compiled individually.

%%%%%%%%%%%%%%%%%%%%%%%%%%%%%%%%%%%%%%%%%%%%%%%%%%%%%%%%%%%%%%%%%%%%%%%%%%%%%%%%
%\subsection{Feature Suggestions}
%
%The following is a list of features which may be useful for future
%versions of this package:
%%
%\begin{itemize}
%\item
%\ldots
%\end{itemize}

%%%%%%%%%%%%%%%%%%%%%%%%%%%%%%%%%%%%%%%%%%%%%%%%%%%%%%%%%%%%%%%%%%%%%%%%%%%%%%%%
\subsection{Revision History}

%%%%%%%%%%%%%%%%%%%%%%%%%%%%%%%%%%%%%%%%
\paragraph{v2.0:} 2018/12/30

\begin{itemize}
\item
immediate forward processing
\item
added |\childdocby| mechanism
\item
manual restructured
\end{itemize}

%%%%%%%%%%%%%%%%%%%%%%%%%%%%%%%%%%%%%%%%
\paragraph{v1.6:} 2018/01/17

\begin{itemize}
\item
application for development of include files
\item
corrections to manual
\end{itemize}

%%%%%%%%%%%%%%%%%%%%%%%%%%%%%%%%%%%%%%%%
\paragraph{v1.5:} 2017/05/21

\begin{itemize}
\item
more complete structuring introduced
\item
|\childdocof| introduced
\item
|\childdoc| renamed to |\childdocmain|
\item
|\childredirect| renamed to |\childdocforward| and |\childdocforwardprefix|
and functionality expanded
\end{itemize}

%%%%%%%%%%%%%%%%%%%%%%%%%%%%%%%%%%%%%%%%
\paragraph{v1.0:} 2017/04/27

\begin{itemize}
\item
manual and install package
\item
first version published on CTAN
\end{itemize}

%%%%%%%%%%%%%%%%%%%%%%%%%%%%%%%%%%%%%%%%
\paragraph{v0.6:} 2017/04/26

\begin{itemize}
\item
redirection mechanism added
\end{itemize}

%%%%%%%%%%%%%%%%%%%%%%%%%%%%%%%%%%%%%%%%
\paragraph{v0.5:} 2017/04/26

\begin{itemize}
\item
functionality in definition file
\end{itemize}


%%%%%%%%%%%%%%%%%%%%%%%%%%%%%%%%%%%%%%%%%%%%%%%%%%%%%%%%%%%%%%%%%%%%%%%%%%%%%%%%
%%%%%%%%%%%%%%%%%%%%%%%%%%%%%%%%%%%%%%%%%%%%%%%%%%%%%%%%%%%%%%%%%%%%%%%%%%%%%%%%
%%%%%%%%%%%%%%%%%%%%%%%%%%%%%%%%%%%%%%%%%%%%%%%%%%%%%%%%%%%%%%%%%%%%%%%%%%%%%%%%
\appendix

\settowidth\MacroIndent{\rmfamily\scriptsize 000\ }

 \DocInput{childdoc.dtx}

\end{document}
%</driver>
% \fi
%
% %%%%%%%%%%%%%%%%%%%%%%%%%%%%%%%%%%%%%%%%%%%%%%%%%%%%%%%%%%%%%%%%%%%%%%%%%%%%%%
% %%%%%%%%%%%%%%%%%%%%%%%%%%%%%%%%%%%%%%%%%%%%%%%%%%%%%%%%%%%%%%%%%%%%%%%%%%%%%%
% \section{Sample}
%\iffalse
%<*samplemain>
%\fi
%
% The following presents a sample document
% with two chapters, two parts, a title page,
% a compile flag as well as three forwarding files to set the flag.
% It consists of eight |.tex| files:
% \begin{center}
% \begin{tabular}{ll}
% |cdocsamp.tex|&main file\\
% |cdocsch1.tex|&include file for chapter 1\\
% |cdocsch2.tex|&include file for chapter 2\\
% |cdocspt3.tex|&include file for part 3\\
% |cdocspt4.tex|&include file for part 4\\
% |cdocsdrf.tex|&forwarding file for main file in draft mode\\
% |cdocsfi1.tex|&forwarding file for final version of chapter 1\\
% |cdocsfi2.tex|&forwarding file for final version of chapter 2\\
% \end{tabular}
% \end{center}
% Each of the eight files can be compiled directly by the \LaTeX{} compiler.
%
% %%%%%%%%%%%%%%%%%%%%%%%%%%%%%%%%%%%%%%
% \paragraph{Main File.}
%
% The main file is called |cdocsamp.tex|.
%
% Load the \textsf{childdoc} definitions and
% declare the filename for the main document:
%    \begin{macrocode}
\input{childdoc.def}
\childdocmain{}
%    \end{macrocode}

% Optional override for |\version| flag:
%    \begin{macrocode}
%%\ifchilddoc\else\providecommand{\version}{draft}\fi
%    \end{macrocode}

% Define the default values for the |\version| flag
% (|final| for the main file and |draft| for childs):
%    \begin{macrocode}
\ifchilddoc
\providecommand{\version}{draft}
\else
\providecommand{\version}{final}
\fi
%    \end{macrocode}

% Load the standard document class:
%    \begin{macrocode}
\documentclass[12pt]{article}
%    \end{macrocode}

% Start the document body:
%    \begin{macrocode}
\begin{document}
%    \end{macrocode}

% Declare a title page.
% Print title, part of document being processed and version flag:
%    \begin{macrocode}
\addtocounter{page}{-1}
\begin{center}
{\LARGE\bfseries{}childdoc example\par}
\vspace{1cm}
\ifchilddoc
\ifchilddocmanual part\else chapter\fi:
`\childdocname' of `\childdocjob'\par
\else
main document: `\childdocjob'\par
\fi
version: \version\par
\end{center}
\newpage
%    \end{macrocode}

% Manually include selected file,
% otherwise process as usual:
%    \begin{macrocode}
\ifchilddocmanual
\section*{part `\childdocname'}
\input{\childdocname}
\else
%    \end{macrocode}

% Include the two chapters:
%    \begin{macrocode}
\include{cdocsch1}
\include{cdocsch2}
%    \end{macrocode}

% Include the two parts unless only chapters should be displayed:
%    \begin{macrocode}
\ifchilddoc\else
\section{part three}
\input{cdocspt3}
\section{part four}
\input{cdocspt4}
\fi
%    \end{macrocode}

% Process as usual until here:
%    \begin{macrocode}
\fi
%    \end{macrocode}

% End of document body:
%    \begin{macrocode}
\end{document}
%    \end{macrocode}
%\iffalse
%</samplemain>
%\fi
%
% %%%%%%%%%%%%%%%%%%%%%%%%%%%%%%%%%%%%%%
% \paragraph{Chapter Include Files.}
%
% The include files are called |cdocsch1.tex| and |cdocsch2.tex|.
%
%\iffalse
%<*samplechap1|samplechap2>
%\fi

% Optional override for |\version| flag:
%    \begin{macrocode}
%%\providecommand{\version}{final}
%    \end{macrocode}

% Include the main document:
%    \begin{macrocode}
\input{childdoc.def}
\childdocof{cdocsamp}
%    \end{macrocode}

%\iffalse
%</samplechap1|samplechap2>
%\fi
%
%\iffalse
%<*samplechap1>
%\fi
% Some text for chapter 1:
%    \begin{macrocode}
\section{one}
some text in chapter one
%    \end{macrocode}

%\iffalse
%</samplechap1>
%\fi
% Some text for chapter 2:
%\iffalse
%<*samplechap2>
%\fi
%    \begin{macrocode}
\section{two}
more text in chapter two
%    \end{macrocode}

%\iffalse
%</samplechap2>
%\fi
%
% %%%%%%%%%%%%%%%%%%%%%%%%%%%%%%%%%%%%%%
% \paragraph{Part Include Files.}
%
% The include files are called |cdocspt3.tex| and |cdocspt4.tex|.
%
%\iffalse
%<*samplepart3|samplepart4>
%\fi

% Optional override for |\version| flag:
%    \begin{macrocode}
%%\providecommand{\version}{final}
%    \end{macrocode}

% Include the main document:
%    \begin{macrocode}
\input{childdoc.def}
\childdocby{cdocsamp}
%    \end{macrocode}

%\iffalse
%</samplepart3|samplepart4>
%\fi
%
%\iffalse
%<*samplepart3>
%\fi
% Some text for part 3:
%    \begin{macrocode}
some text in part three
%    \end{macrocode}

%\iffalse
%</samplepart3>
%\fi
% Some text for part 4:
%\iffalse
%<*samplepart4>
%\fi
%    \begin{macrocode}
more text in part four
%    \end{macrocode}

%\iffalse
%</samplepart4>
%\fi
%
% %%%%%%%%%%%%%%%%%%%%%%%%%%%%%%%%%%%%%%
% \paragraph{Forwarding for a Complete Draft.}
%
% The following forwarding file |cdocsdrf.tex|
% compiles the main document in draft mode:
%\iffalse
%<*sampledraft>
%\fi
%    \begin{macrocode}
\def\version{draft}
\input{childdoc.def}
\childdocforward{cdocsamp}
%    \end{macrocode}

%\iffalse
%</sampledraft>
%\fi
%
% %%%%%%%%%%%%%%%%%%%%%%%%%%%%%%%%%%%%%%
% \paragraph{Forwarding for Final Version of the Chapters.}
%
% The following forwarding files |cdocsfn1.tex| and |cdocsfn2.tex|
% (with identical content)
% compile the final versions of the child documents
% |cdocsch1.tex| and |cdocsch2.tex|, respectively:
%\iffalse
%<*samplefinal>
%\fi
%    \begin{macrocode}
\def\version{final}
\input{childdoc.def}
\childdocforwardprefix[cdocsamp]{cdocsfn}{cdocsch}
%    \end{macrocode}

%\iffalse
%</samplefinal>
%\fi
%
% %%%%%%%%%%%%%%%%%%%%%%%%%%%%%%%%%%%%%%
% \paragraph{Command Line Processing.}
%
% The following three command lines generate the output files
% |cdocscld|, |cdocscl1| and |cdocscl2|
% which should be identical to
% |cdocsdrf|, |cdocsch1| and |cdocsfn2|, respectively:
% \begin{center}
% \begin{tabular}{l}
% |latex -jobname cdocscld \|\\
% |  "\def\version{draft}\input{childdoc.def}\childdocforward{cdocsamp}"|\\
% |latex -jobname cdocscl1 \|\\
% |  "\input{childdoc.def}\childdocforward[cdocsamp]{cdocsch1}"|\\
% |latex -jobname cdocscl2 \|\\
% |  "\def\version{final}\input{childdoc.def}\childdocforward{cdocsch2}"|
% \end{tabular}
% \end{center}
% Note that the trailing backslash on each first line
% merely continues the input to the second line
% (for convenient cut ant paste).
% Furthermore, the command |latex| can be replaced by any
% of its alternative versions such as |pdflatex|.
%
% %%%%%%%%%%%%%%%%%%%%%%%%%%%%%%%%%%%%%%%%%%%%%%%%%%%%%%%%%%%%%%%%%%%%%%%%%%%%%%
% %%%%%%%%%%%%%%%%%%%%%%%%%%%%%%%%%%%%%%%%%%%%%%%%%%%%%%%%%%%%%%%%%%%%%%%%%%%%%%
% \section{Implementation}
%\iffalse
%<*package>
%\fi
%
% This section describes the definitions file |childdoc.def|.

% The definitions cannot be loaded using |\usepackage| or |\RequirePackage|
% which has a mechanism to prevent loading a style file more than once.
% When loading the definitions by means of |\input|
% multiple instances have to be prevented manually:
%\iffalse
%This code needs to be before the `\ProvidesFile' directive
%which is defined at the beginning of this file.
%Therefore it is also placed there and commented out here.
%</package>
%<*discard>
%\fi
%    \begin{macrocode}
\ifdefined\childdocmain\endinput\fi
%    \end{macrocode}
%\iffalse
%</discard>
%<*package>
%\fi
%
% \macro{\ifchilddoc}
% \macro{\ifchilddocmanual}
% The conditional |\ifchilddoc| tells whether a
% child (true) or main (false) document is being compiled.
% The conditional |\ifchilddocmanual| tells whether
% the |\includeonly| mechanism is used (false) or
% the selection of child files must be performed manually (true).
% The definitions initialise to false:
%    \begin{macrocode}
\newif\ifchilddoc
\newif\ifchilddocmanual
%    \end{macrocode}

% \macro{\childdocname}
% \macro{\childdocjob}
% The macro |\childdocname| stores the name of the main document
% to be compiled. The macro |\childdocjob| stores the name of
% the document on which the \LaTeX{} compiler was originally invoked.
% The content of |\jobname| cannot be compared
% to filenames specified in the source due to different catcodes.
% The following code rescans |\jobname|, stores the result
% in |\childdocname| and saves a copy in |\childdocjob|:
%    \begin{macrocode}
\edef\childdocname{\scantokens\expandafter{\jobname\noexpand}}
\let\childdocjob\childdocname
%    \end{macrocode}

% \macro{\childdocdisable}
% The macro |\childdocdisable| prevents the main file
% from being processed more than once.
% At this stage, the main document command |\childdocmain|
% is assumed to be called once again where it should do nothing.
% Any subsequent call to it should prevent
% a secondary processing of the main document
% It overwrites the forwarding commands
% |\childdocof| and |\childdocforward|
% with empty macros to prevent further inclusions of the main document:
%    \begin{macrocode}
\newcommand{\childdocdisable}
{
  \renewcommand{\childdocmain}[1]{\renewcommand{\childdocmain}[1]{\endinput}}
  \renewcommand{\childdocof}[1]{}
  \renewcommand{\childdocby}[2][]{}
  \renewcommand{\childdocforward}[2][]{}
  \renewcommand{\childdocdisable}{}
}
%    \end{macrocode}

% \macro{\childdocmain}
% The macro |\childdocmain| is to be called at the top of the main file
% with nothing or the main filename (without extension) as argument.
% First, it breaks loops.
% If the argument is not empty and does not match |\childdocname|
% (which is set by the first inclusion of |childdoc.def|),
% |\ifchilddoc| is set to true, |\includeonly| is applied to the child file
% and |\jobname| is set to the main file
% (for proper handling of |.aux| files):
%    \begin{macrocode}
\newcommand{\childdocmain}[1]
{
  \childdocdisable\childdocmain{}
  \if?#1?\else
    \begingroup
      \def\childdoctmp{#1}
      \ifx\childdoctmp\childdocname
        \def\childdoctmp{}
      \else
        \def\childdoctmp
        {
          \childdoctrue
          \includeonly{\childdocname}
          \def\childdocjob{#1}
          \def\jobname{#1}
        }
      \fi
      \expandafter
    \endgroup
    \childdoctmp
  \fi
}
%    \end{macrocode}

% \macro{\childdocof}
% The command |\childdocof| redirects
% compilation to the main file |#1|.
%    \begin{macrocode}
\newcommand{\childdocof}[1]
{
  \childdocdisable
  \childdoctrue
  \includeonly{\childdocname}
  \def\jobname{#1}
  \def\childdocjob{#1}
  \input{#1}
}
%    \end{macrocode}

% \macro{\childdocby}
% The command |\childdocby| ....
%    \begin{macrocode}
\newcommand{\childdocby}[2][]
{
  \childdocdisable
  \childdoctrue
  \childdocmanualtrue
  \if?#1?\else
    \def\jobname{#2}
  \fi
  \def\childdocjob{#2}
  \input{#2}
  \endinput
}
%    \end{macrocode}

% \macro{\childdocforward}
% The command |\childdocforward| redirects
% compilation to the main file or
% (if the optional argument is given) a child file.
% Parameters are set as if the main file
% or a child file starting with |\childdocof| was compiled.
% Then compilation is handed over to the main file:
%    \begin{macrocode}
\newcommand{\childdocforward}[2][]
{
  \begingroup
    \if?#1?
      \def\childdoctmp
      {
        \def\childdocname{#2}
        \def\childdocjob{#2}
        \def\jobname{#2}
        \input{#2}
        \endinput
      }
    \else
      \def\childdoctmp
      {
        \childdocdisable
        \def\childdocname{#2}
        \childdoctrue
        \includeonly{#2}
        \def\childdocjob{#1}
        \def\jobname{#1}
        \input{#1}
        \endinput
      }
    \fi
    \expandafter
  \endgroup
  \childdoctmp
}
%    \end{macrocode}

% \macro{\childdocforwardprefix}
% The command |\childdocforwardprefix| redirects
% compilation to the main or a child file by means of a pattern.
% The prefix |#1| in the current filename is replaced by |#2|
% and the suffix of the current filename is kept
% (it is assumed that the filename does not contain the substring `|~~~|'
% which is used as a delimiter).
% Compilation is handed over to the new file by |\childdocforward|:
%    \begin{macrocode}
\newcommand{\childdocforwardprefix}[3][]
{
  \begingroup
    \def\childdocextract #2##1~~~{\def\childdoctmp{\childdocforward[#1]{#3##1}}}
    \expandafter\childdocextract\childdocname~~~
    \expandafter
  \endgroup
  \childdoctmp
}
%    \end{macrocode}

% \macro{\childdoc}
% The deprecated macro |\childdoc| is a legacy version of |\childdocmain|:
%    \begin{macrocode}
\newcommand{\childdoc}{\childdocmain}
%    \end{macrocode}

% \macro{\childdocredirect}
% The deprecated macro |\childdocredirect| is a legacy version
% of |\childdocforward| and |\childdocforwardprefix|:
%    \begin{macrocode}
\newcommand{\childdocredirect}[2][]
{
  \begingroup
    \if?#1?
      \def\childdoctmp{\childdocforward{#2}}
    \else
      \def\childdoctmp{\childdocforwardprefix{#1}{#2}}
    \fi
    \expandafter
  \endgroup
  \childdoctmp
}
%    \end{macrocode}

%\iffalse
%</package>
%\fi
%
\endinput

\childdocforward{cdocsamp}
%    \end{macrocode}

%\iffalse
%</sampledraft>
%\fi
%
% %%%%%%%%%%%%%%%%%%%%%%%%%%%%%%%%%%%%%%
% \paragraph{Forwarding for Final Version of the Chapters.}
%
% The following forwarding files |cdocsfn1.tex| and |cdocsfn2.tex|
% (with identical content)
% compile the final versions of the child documents
% |cdocsch1.tex| and |cdocsch2.tex|, respectively:
%\iffalse
%<*samplefinal>
%\fi
%    \begin{macrocode}
\def\version{final}
% \iffalse
%
% childdoc.dtx Copyright (C) 2017-2018 Niklas Beisert
%
% This work may be distributed and/or modified under the
% conditions of the LaTeX Project Public License, either version 1.3
% of this license or (at your option) any later version.
% The latest version of this license is in
%   http://www.latex-project.org/lppl.txt
% and version 1.3 or later is part of all distributions of LaTeX
% version 2005/12/01 or later.
%
% This work has the LPPL maintenance status `maintained'.
%
% The Current Maintainer of this work is Niklas Beisert.
%
% This work consists of the files childdoc.dtx and childdoc.ins
% and the derived files childdoc.def and cdocsamp.tex with
% cdocsch1.tex, cdocsch2.tex, cdocsdrf.tex, cdocsfn1.tex, cdocsfn2.tex.
%
%<package>\ifdefined\childdocmain\endinput\fi
%<package>\ProvidesFile{childdoc.def}[2018/12/30 v2.0 child document driver]
%<samplemain>\ProvidesFile{cdocsamp.tex}[2018/12/30 v2.0 sample for childdoc]
%<*driver>
%\ProvidesFile{childdoc.drv}[2018/12/30 v2.0 childdoc reference manual file]
\PassOptionsToClass{10pt,a4paper}{article}
\documentclass{ltxdoc}

\usepackage[margin=35mm]{geometry}
\usepackage{hyperref}
\usepackage{hyperxmp}
\usepackage[usenames]{color}

\hypersetup{colorlinks=true}
\hypersetup{pdfstartview=FitH}
\hypersetup{pdfpagemode=UseNone}
\hypersetup{pdfsource={}}
\hypersetup{pdflang={en-UK}}
\hypersetup{pdfcopyright={Copyright 2017-2018 Niklas Beisert.
  This work may be distributed and/or modified under the
  conditions of the LaTeX Project Public License, either version 1.3
  of this license or (at your option) any later version.}}
\hypersetup{pdflicenseurl={http://www.latex-project.org/lppl.txt}}
\hypersetup{pdfcontactaddress={ETH Zurich, ITP, HIT K,
  Wolfgang-Pauli-Strasse 27}}
\hypersetup{pdfcontactpostcode={8093}}
\hypersetup{pdfcontactcity={Zurich}}
\hypersetup{pdfcontactcountry={Switzerland}}
\hypersetup{pdfcontactemail={nbeisert@itp.phys.ethz.ch}}
\hypersetup{pdfcontacturl={http://people.phys.ethz.ch/\xmptilde nbeisert/}}

\newcommand{\secref}[1]{\hyperref[#1]{section \ref*{#1}}}

\parskip1ex
\parindent0pt
\let\olditemize\itemize
\def\itemize{\olditemize\parskip0pt}

\begin{document}

\title{The \textsf{childdoc} Package}
\hypersetup{pdftitle={The childdoc Package}}
\author{Niklas Beisert\\[2ex]
  Institut f\"ur Theoretische Physik\\
  Eidgen\"ossische Technische Hochschule Z\"urich\\
  Wolfgang-Pauli-Strasse 27, 8093 Z\"urich, Switzerland\\[1ex]
  \href{mailto:nbeisert@itp.phys.ethz.ch}
  {\texttt{nbeisert@itp.phys.ethz.ch}}}
\hypersetup{pdfauthor={Niklas Beisert}}
\hypersetup{pdfsubject={Manual for the LaTeX2e Package childdoc}}
\date{30 December 2018, \textsf{v2.0}}
\maketitle

\begin{abstract}\noindent
\textsf{childdoc} is a \LaTeXe{} package
that enables the direct compilation
of document sections included by |\include|
to individual files.
\end{abstract}

\begingroup
\parskip0ex
\tableofcontents
\endgroup

%%%%%%%%%%%%%%%%%%%%%%%%%%%%%%%%%%%%%%%%%%%%%%%%%%%%%%%%%%%%%%%%%%%%%%%%%%%%%%%%
%%%%%%%%%%%%%%%%%%%%%%%%%%%%%%%%%%%%%%%%%%%%%%%%%%%%%%%%%%%%%%%%%%%%%%%%%%%%%%%%
\section{Introduction}

\LaTeX{} provides a mechanism to structure a large document (such as a book)
into a main file and several child files (containing the chapters)
using the |\include| command.
This mechanism is beneficial for documents
which span hundreds of pages in order to
make the source file(s) more manageable.
Moreover, compilation can be restricted to
selected child files by means of the |\includeonly| command.
The latter feature can be used to reduce the compilation time while editing
(this was significantly more useful in the earlier days of \LaTeX{})
or to generate a smaller document which is easier to navigate.
Another application of |\includeonly| is to generate
documents consisting of selected parts of the complete document.

However, there are a few drawbacks of the plain |\include| mechanism:
\begin{itemize}
\item
The child files cannot be compiled on their own,
they can only be compiled via the main file.
A naive editing environment
(such as a text editor with an option
to have the current file processed by \LaTeX)
may require one to switch to the main file before compiling;
attempting to compile the child file produces errors.
\item
The main file must be modified (each time)
to adjust the |\includeonly| command
to the present needs. This easily leaves the main file in a messy state.
\item
The generated document will always carry the filename
of the main document. This is inconvenient if
several child files are to be compiled and
to be kept for distribution.
\end{itemize}

The present package provides a simple interface
to make child files individually compilable by \LaTeX{}.
Compiling a child file then has the same effect as compiling
the main file with an |\includeonly| command
to select the appropriate child.
Moreover the generated document will carry the name of the child
rather than the main file.
This resolves all three above issues.

This feature is meant to make the editing of books,
thesis documents and lecture notes somewhat more convenient.
However, the package can also be used efficiently for
composing a series of documents (such as exercise sheets)
which are typically distributed individually.
It then assists the author in generating the individual documents
(potentially in different versions)
as well as a document containing the collected series.
Another application is in developing style files
or other kinds of included material
where compilation of the style file could redirect
to a sample or test file.

%%%%%%%%%%%%%%%%%%%%%%%%%%%%%%%%%%%%%%%%%%%%%%%%%%%%%%%%%%%%%%%%%%%%%%%%%%%%%%%%
%%%%%%%%%%%%%%%%%%%%%%%%%%%%%%%%%%%%%%%%%%%%%%%%%%%%%%%%%%%%%%%%%%%%%%%%%%%%%%%%
\section{Usage}

First of all, the package \textsf{childdoc} is \emph{not} a standard
\LaTeXe{} |.sty| style file! Therefore it needs to be invoked in
a non-standard way.

%%%%%%%%%%%%%%%%%%%%%%%%%%%%%%%%%%%%%%%%%%%%%%%%%%%%%%%%%%%%%%%%%%%%%%%%%%%%%%%%
\subsection{Included Files}
\label{sec:include}

%%%%%%%%%%%%%%%%%%%%%%%%%%%%%%%%%%%%%%%%
\DescribeMacro{\childdocmain}
To use the package, add the commands
\begin{center}
\begin{tabular}{l}
|\input{childdoc.def}|\\
|\childdocmain{}|\\
\end{tabular}
\end{center}
at the very top of the main \LaTeX{} file,
in particular \emph{before} the |\documentclass| statement!
The argument of |\childdocmain| should be left empty
(but it must be present).

%%%%%%%%%%%%%%%%%%%%%%%%%%%%%%%%%%%%%%%%
\DescribeMacro{\childdocof}
Furthermore, add the commands
\begin{center}
\begin{tabular}{l}
|\input{childdoc.def}|\\
|\childdocof{|\textit{main}|}|\\
\end{tabular}
\end{center}
at the top of every child file \textit{child}
which is included by |\include{|\textit{child}|}|
from within the main file
(or at least for those files to be compiled individually).
The argument \textit{main} must be the filename of the main file.

There are a couple of
considerations in setting up the main and child documents:

%%%%%%%%%%%%%%%%%%%%%%%%%%%%%%%%%%%%%%%%
\paragraph{Restrictions.}

Please note the following restrictions:
\begin{itemize}
\item
|\childdocmain| must be called with one argument \textit{main}
to ensure compatibility with earlier version of the package.
It must either be empty (|\childdocmain{}|)
or precisely match the filename of the main file in which it is specified.
See \secref{sec:detection} for further information.
\item
The filename \textit{main} must be specified without the |.tex| extension.
\item
The filename \textit{main} is case sensitive
(even in case-insensitive file systems)
due to internal string comparison.
\item
The argument \textit{main} should be fully expanded, it cannot be a macro.
\item
Subdirectories and special characters should be avoided in filenames.
\item
The command |\childdocmain{|\textit{main}|}| must be followed by a whitespace.
It should not be followed immediately by another command
or by a comment mark `|%|'.
This is because the \TeX{} parser reads the token immediately following
the argument of |\childdocmain| and puts it
at the beginning of every child section;
however, a white\-space is ignored.
\end{itemize}

%%%%%%%%%%%%%%%%%%%%%%%%%%%%%%%%%%%%%%%%
\paragraph{Content of Main File.}

It is advisable to place all content in the child files included by |\include|.
Any output contained in the main file will appear in all child documents
unless suppressed manually;
it cannot be suppressed automatically by the |\includeonly| directive
and thus should normally be avoided.
A method to include some content in the main file
by means of conditional processing is described in \secref{sec:conditional}.

%%%%%%%%%%%%%%%%%%%%%%%%%%%%%%%%%%%%%%%%
\paragraph{Page Numbering.}

When only a part of the document is compiled,
the appropriate numbering of pages
(as well as other status parameters)
is determined from the |.aux| files.
The latter contain information from previous passes.
However this information needs to propagate through
all intermediate child documents.
Therefore the page numbering in child documents may well
be inconsistent until the complete document is compiled at least once.

A useful (if unconventional) way to always ensure a consistent
page numbering is to restart the numbering in each child document
and denote the pages by `\textit{child}|.|\textit{page}'
where \textit{child} represents the chapter/section number of the child file.
This can be achieved by the command
|\numberwithin{page}{|\textit{child}|}|
of the \textsf{amsmath} package
where \textit{child} can be |chapter| or |section|
depending on the chosen structuring.
Alternatively, one can modify the macro |\thepage| appropriately
and reset the counter |page| at the start of each child file.

%%%%%%%%%%%%%%%%%%%%%%%%%%%%%%%%%%%%%%%%%%%%%%%%%%%%%%%%%%%%%%%%%%%%%%%%%%%%%%%%
\subsection{Conditional Processing}
\label{sec:conditional}

The package provides a mechanism to compile different versions
of a document. To customise the versions further some conditional processing
can come in handy to distinguish which version is being compiled.
The package provides two macros to describe the compilation context:

%%%%%%%%%%%%%%%%%%%%%%%%%%%%%%%%%%%%%%%%
\DescribeMacro{\ifchilddoc}
The conditional |\ifchilddoc| distinguishes between the compilation of
child documents and the main document:
%
\begin{center}
|\ifchilddoc |\textit{child-code}| |[|\||else |\textit{main-code}]| \||fi|
\end{center}

%%%%%%%%%%%%%%%%%%%%%%%%%%%%%%%%%%%%%%%%
\DescribeMacro{\childdocname}
\DescribeMacro{\childdocjob}
The macro |\childdocname| contains the filename (without extension)
of the main or child file being processed.
Note that |\childdocjob| will always contain the name of the main file.

%%%%%%%%%%%%%%%%%%%%%%%%%%%%%%%%%%%%%%%%
\paragraph{Title Page.}

Conditional processing can be used to include a title or banner page
in the main document when proper precautions are taken.
Importantly, the code in the main file should ensure that the page counter
(as well as other status parameters which are stored in the |.aux| files)
takes the same value after the conditional processing.
Otherwise the page numbers may take divergent values
depending on which part is compiled.

For example, a title page could be declared by:
%
\begin{center}
\begin{tabular}{l}
|\ifchilddoc\||else|\\
|\addtocounter{page}{-1}|\\
\textit{code for title page}\\
|\newpage|\\
|\||fi|
\end{tabular}
\end{center}
%
A banner page for the child documents can be generated by:
%
\begin{center}
\begin{tabular}{l}
|\ifchilddoc|\\
|\addtocounter{page}{-1}|\\
\textit{code for banner page}\\
|\newpage|\\
|\||fi|
\end{tabular}
\end{center}
%
Here one could write a message such as:
\begin{center}
|This is the part \childdocname{} of \childdocjob{}.|
\end{center}

%%%%%%%%%%%%%%%%%%%%%%%%%%%%%%%%%%%%%%%%%%%%%%%%%%%%%%%%%%%%%%%%%%%%%%%%%%%%%%%%
\subsection{Flags}
\label{sec:flags}

The package makes it easy to generate different versions
of the main or child documents.
To this end compilation flags can be defined
and assigned different default values.
They will be particularly useful in conjunction
with the forwarding mechanism described in \secref{sec:forward}.

For example, it may be useful to have a flag |\version|
which can be set to |draft| or |final|.
The document source will contain some conditional code
depending on the value of |\version|.
Suppose further, the flag should default to |final| for the main file
and to |draft| for child files
which is a natural assignment for editing the document.
This is achieved by placing the following code
in the preamble of the main document
(below the |\childdocmain| directive):
%
\begin{center}
\begin{tabular}{l}
|\ifchilddoc|\\
|\providecommand{\version}{draft}|\\
|\||else|\\
|\providecommand{\version}{final}|\\
|\||fi|
\end{tabular}
\end{center}
%
The definition by |\providecommand| makes sure
that previous definitions are not overwritten.
Further statements |\providecommand{\version}{...}|
can thus be added before the above code to override it.

For the main file, one might add a line
(between |\childdocmain| and the above block)
%
\begin{center}
|%\ifchilddoc\||else\providecommand{\version}{draft}\||fi|
\end{center}
%
which can be uncommented to produce a draft version.
Likewise one can add a line to the very top of a child file
(above the |\childdocof{|\textit{main}|}| directive)
%
\begin{center}
|%\providecommand{\version}{final}|
\end{center}
%
which can be uncommented to produce the final version of this child document.

%%%%%%%%%%%%%%%%%%%%%%%%%%%%%%%%%%%%%%%%%%%%%%%%%%%%%%%%%%%%%%%%%%%%%%%%%%%%%%%%
\subsection{Forwarding}
\label{sec:forward}

Different versions of the main or child documents
using compilation flags as described in \secref{sec:flags}
can be (permanently) stored in different files
for convenient compilation, viewing and distribution.
To this end, the package defines a command
to pass on compilation to a different file:

%%%%%%%%%%%%%%%%%%%%%%%%%%%%%%%%%%%%%%%%
\DescribeMacro{\childdocforward}
The command |\childdocforward| redirects processing to
another source file:
%
\begin{center}
\begin{tabular}{l}
|\input{childdoc.def}|\\
|\childdocforward[|\textit{main}|]{|\textit{dest}|}|\\
\end{tabular}
\end{center}
%
The argument \textit{dest} is the destination file
(without extension).
It should be the main file or one of the child files.
Note that further \textsf{childdoc} directives
such as |\childdocof| and |\childdocforward|
in the indicated file will be processed in this form.
The optional argument \textit{main}
passes on directly to the main file \textit{main}
while pretending to compile the child \textit{dest}.
This form behaves as if \textit{dest}
issues |\childdocof{|\textit{main}|}| right away,
and no further \textsf{childdoc} directives will be processed.

%%%%%%%%%%%%%%%%%%%%%%%%%%%%%%%%%%%%%%%%
\DescribeMacro{\...prefix}
In the alternative form |\childdocforwardprefix|,
%
\begin{center}
\begin{tabular}{l}
|\input{childdoc.def}|\\
|\childdocforwardprefix[|\textit{main}|]{|\textit{prefix}|}{|\textit{dest}|}|
\end{tabular}
\end{center}
%
the destination file is determined by a pattern
depending on the current file:
To make this work, the current file must be called
`{\textit{prefix}\hspace{0.2em}\textit{suffix}}'
with \textit{prefix} matching precisely the argument.
Processing is then passed on to the file
`{\textit{dest}\hspace{0.2em}\textit{suffix}}'.
Surely, the same effect is achieved by
directly specifying the
argument `{\textit{dest}\hspace{0.2em}\textit{suffix}}'
in the first form.
However, that requires to set up a different file
for each child. With the alternative form of the command
all these files can have exactly the same content
which simplifies setting them up and maintaining them.

For example, the following file |draft.tex|
with a compilation flag |\version| as described in \secref{sec:flags}
compiles the main document as a draft:
%
\begin{center}
\begin{tabular}{l}
|\def\version{draft}|\\
|\input{childdoc.def}|\\
|\childdocforward{|\textit{main}|}|
\end{tabular}
\end{center}
%
Likewise, the following files |final|\textit{nn}|.tex|
compile the final version of the child document
|child|\textit{nn}|.tex|:
%
\begin{center}
\begin{tabular}{l}
|\def\version{final}|\\
|\input{childdoc.def}|\\
|\childdocforwardprefix{final}{child}|
\end{tabular}
\end{center}
%

Note that when several versions of a main file and/or of each child file
are to be generated, it may be convenient to set up a |Makefile| or
shell script to automatise the process.

%%%%%%%%%%%%%%%%%%%%%%%%%%%%%%%%%%%%%%%%%%%%%%%%%%%%%%%%%%%%%%%%%%%%%%%%%%%%%%%%
\subsection{Command Line Processing}
\label{sec:commandline}

The effect of redirection files can also be achieved by invoking
the \LaTeX{} compiler with a more elaborate command line.
Most conveniently this should be done as part
of a shell script or a |Makefile|.

When using \textsf{childdoc} in the main file, the following
command lines effectively perform a redirection
(note that depending on the shell being used,
backslashes may have to be doubled: `|\|' $\to$ `|\\|'):
%
\begin{center}
|... -jobname "|\textit{target}|" |\\|"|[\textit{flags}]%
|\input{childdoc.def}\childdocforward[|\textit{main}|]{|\textit{dest}|}"|
\end{center}
%
Here \textit{target} is the name of the output file,
\textit{main} is the name of the main file
and \textit{dest} is the name of the main or child file to be processed
(all filenames without extensions).
The optional argument \textit{main} can be omitted
if \textit{main} matches \textit{dest}.
Optionally, compilation \textit{flags} can be defined via |\def| commands.
This command line makes the \TeX{} engine believe
it is compiling the file \textit{target}
whose content is specified as the latter parameter.
The provided code then forwards the processing to
\textit{main} or \textit{dest} as described in \secref{sec:forward}.

%%%%%%%%%%%%%%%%%%%%%%%%%%%%%%%%%%%%%%%%%%%%%%%%%%%%%%%%%%%%%%%%%%%%%%%%%%%%%%%%
\subsection{Include by Input}
\label{sec:input}

Including child documents by |\include| has some restrictions by design.
Most notably, the content of a child document always occupies
its own set of pages; pages cannot be shared between child documents.
Usually, this behaviour makes perfect sense
because each child document contain an essential part of the document.
However, in some situations it may be desirable to compose
a document from a collection of parts
without having mandatory page breaks between then.
For this case, the package
provides a mechanism to include parts
by |\input| which can also be processed individually.
However, by construction this mechanism
requires manual handling of the content to be output.

%%%%%%%%%%%%%%%%%%%%%%%%%%%%%%%%%%%%%%%%
\DescribeMacro{\ifchilddocmanual}
The main file should be prepared as usual, see \secref{sec:include}.
However, the document body must make a distinction
between processing of an individual part and of the main document, e.g.:
%
\begin{center}
\begin{tabular}{l}
|\ifchilddocmanual|\\
|\input{\childdocname}|\\
|\||else|\\
\textit{document body with }|\input{|\textit{part}|}|\\
|\||fi|
\end{tabular}
\end{center}
%
The conditional |\ifchilddocmanual| is true whenever
a part to be included by |\input| is being compiled,
and the name of the part is stored in |\childdocname|.

%%%%%%%%%%%%%%%%%%%%%%%%%%%%%%%%%%%%%%%%
\DescribeMacro{\childdocby}
Each part to be included by |\input| should start with:
%
\begin{center}
\begin{tabular}{l}
|\input{childdoc.def}|\\
|\childdocby{|\textit{main}|}|\\
\end{tabular}
\end{center}
%
The directive |\childdocby| is similar to |\childdocof|
described in \secref{sec:include},
but the subsequent selection of content must be done manually.
To that end, both |\ifchilddoc| and |\ifchilddocmanual|
will be true upon processing of a part,
and the name of the part is stored in |\childdocname|.
Note that |\jobname| will be set to the filename of the current part
so that each part receives an individual |.aux| file
that does not interfere with the |.aux| file(s) of the main document.
This behaviour can be altered by the alternative form
|\childdocby[*]{|\textit{main}|}| (with a non-empty optional argument)
which uses the |.aux| file of the main document
by setting |\jobname| to \textit{main}.

%%%%%%%%%%%%%%%%%%%%%%%%%%%%%%%%%%%%%%%%%%%%%%%%%%%%%%%%%%%%%%%%%%%%%%%%%%%%%%%%
\subsection{Driver Development}
\label{sec:driver}

The \textsf{childdoc} mechanism can also be use for the development
of definition files such as \LaTeX{} styles or classes.
This case differs from the above setup with multiple parts
included by |\include| in that no |\includeonly| should be invoked.
This can be achieved by starting the include file
(before |\ProvidesPackage|) with:
%
\begin{center}
\begin{tabular}{l}
|\input{childdoc.def}|\\
|\childdocforward{|\textit{main}|}|\\
\end{tabular}
\end{center}
%
or alternatively with:
%
\begin{center}
\begin{tabular}{l}
|\input{childdoc.def}|\\
|\childdocby{|\textit{main}|}|\\
\end{tabular}
\end{center}
%
Both forms have slightly different effects as described above.
The main file is prepared as usual, see \secref{sec:include}.

%%%%%%%%%%%%%%%%%%%%%%%%%%%%%%%%%%%%%%%%%%%%%%%%%%%%%%%%%%%%%%%%%%%%%%%%%%%%%%%%
\subsection{Legacy Detection}
\label{sec:detection}

The directive |\childdocmain| in the main file can detect
whether the complete document or merely a child is to be compiled
even without using the directive |\childdocof|.
This method is deprecated because it is less robust
and there is no compelling reason to use it;
it is merely provided for backward compatibility
and it may be removed in future versions.

If the detection mechanism is to be used,
it is mandatory to correctly specify
the filename of the main file as the argument of |\childdocmain|:
%
\begin{center}
\begin{tabular}{l}
|\input{childdoc.def}|\\
|\childdocmain{|\textit{main}|}|\\
\end{tabular}
\end{center}
%
If |\jobname| does not match the argument \textit{main} of |\childdocmain|,
it is assumed that |\jobname| points to the child file to be compiled.
When using |\childdocmain| with the main file specified as argument,
it suffices to start a child file
with just |\input{|\textit{main}|}|
without loading of the package and using |\childdocof|.
If instead all processing is done
with the appropriate \textsf{childdoc} directives,
the argument of \textit{main} of |\childdocmain| can be empty.

An alternative version of the command line processing described
in \secref{sec:commandline} using the detection mechanism reads:
%
\begin{center}
|... -jobname "|\textit{target}|" "|[\textit{flags}]%
[|\def\jobname{|\textit{dest}|}|]|\input{|\textit{main}|}"|
\end{center}

%%%%%%%%%%%%%%%%%%%%%%%%%%%%%%%%%%%%%%%%%%%%%%%%%%%%%%%%%%%%%%%%%%%%%%%%%%%%%%%%
\subsection{Manual Code}
\label{sec:manual}

In case one cannot be certain whether the definitions file |childdoc.def|
is installed on the target \TeX{} distribution
and one prefers not to ship it,
it is conceivable to paste a few relevant commands into the sources.

To that end, drop all statements |\input{childdoc.def}|
and perform the replacements as outlined below.
Instead of |\childdocmain{|\textit{main}|}| add the following code
to the top of the main file:
%
\begin{center}
\begin{tabular}{l}
|\||ifdefined\childdocname\endinput\||fi\newif\ifchilddoc|\\
|\edef\childdocname{\scantokens\expandafter{\jobname\noexpand}}|\\
|\def\childdocmain{|\textit{main}|}\||ifx\childdocmain\childdocname\||else|\\
|\childdoctrue\includeonly{\childdocname}\let\jobname\childdocmain\||fi|\\
\end{tabular}
\end{center}
%
Instead of |\childdocof{|\textit{main}|}| just include the main file
at the top of each child file:
%
\begin{center}
|\input{|\textit{main}|}|
\end{center}
%
A simple redirection |\childdocforward{|\textit{dest}|}| is achieved by:
%
\begin{center}
|\def\jobname{|\textit{dest}|}\input{\jobname}|
\end{center}
%
The redirection with prefix
|\childdocforwardprefix[|\textit{prefix}|]{|\textit{dest}|}|
is accomplished by:
%
\begin{center}
\begin{tabular}{l}
|{\edef\jobname{\scantokens\expandafter{\jobname\noexpand}}|\\
|\def\redirectjob |\textit{prefix}|#1~~~{\gdef\jobname{|\textit{dest}|#1}}|\\
|\expandafter\redirectjob\jobname~~~}\input{\jobname}|
\end{tabular}
\end{center}

In an alternative approach,
child documents can be compiled by a specific command line
without additional code or specific definitions:
%
\begin{center}
|... -jobname "|\textit{target}|" "|[\textit{flags}]%
|\includeonly{|\textit{dest}|}\input{|\textit{main}|}"|
\end{center}
%

%%%%%%%%%%%%%%%%%%%%%%%%%%%%%%%%%%%%%%%%%%%%%%%%%%%%%%%%%%%%%%%%%%%%%%%%%%%%%%%%
%%%%%%%%%%%%%%%%%%%%%%%%%%%%%%%%%%%%%%%%%%%%%%%%%%%%%%%%%%%%%%%%%%%%%%%%%%%%%%%%
\section{Information}

%%%%%%%%%%%%%%%%%%%%%%%%%%%%%%%%%%%%%%%%%%%%%%%%%%%%%%%%%%%%%%%%%%%%%%%%%%%%%%%%
\subsection{Copyright}

Copyright \copyright{} 2017--2018 Niklas Beisert

This work may be distributed and/or modified under the
conditions of the \LaTeX{} Project Public License, either version 1.3
of this license or (at your option) any later version.
The latest version of this license is in
  \url{http://www.latex-project.org/lppl.txt}
and version 1.3 or later is part of all distributions of \LaTeX{}
version 2005/12/01 or later.

This work has the LPPL maintenance status `maintained'.

The Current Maintainer of this work is Niklas Beisert.

This work consists of the files |README.txt|, |childdoc.ins| and |childdoc.dtx|
as well as the derived files |childdoc.def|, |cdocsamp.tex|
with |cdocsch1.tex|, |cdocsch2.tex|, |cdocspt3.tex|, |cdocspt4.tex|,
|cdocsdrf.tex|, |cdocsfn1.tex|, |cdocsfn2.tex|
as well as |childdoc.pdf|.

%%%%%%%%%%%%%%%%%%%%%%%%%%%%%%%%%%%%%%%%%%%%%%%%%%%%%%%%%%%%%%%%%%%%%%%%%%%%%%%%
\subsection{Files and Installation}

The package consists of the files:
%
\begin{center}
\begin{tabular}{ll}
    |README.txt|   & readme file \\
    |childdoc.ins| & installation file \\
    |childdoc.dtx| & source file \\
    |childdoc.def| & definition file \\
    |cdocsamp.tex| & sample main file \\
    |cdocsch1.tex| & sample include file \\
    |cdocsch2.tex| & sample include file \\
    |cdocspt3.tex| & sample part file \\
    |cdocspt4.tex| & sample part file \\
    |cdocsdrf.tex| & sample redirection file \\
    |cdocsfn1.tex| & sample redirection file \\
    |cdocsfn2.tex| & sample redirection file \\
    |childdoc.pdf| & manual
\end{tabular}
\end{center}
%
The distribution consists of the files
|README.txt|, |childdoc.ins| and |childdoc.dtx|.
%
\begin{itemize}
\item
Run (pdf)\LaTeX{} on |childdoc.dtx|
to compile the manual |childdoc.pdf| (this file).
\item
Run \LaTeX{} on |childdoc.ins| to create the definitions file |childdoc.def|
and the sample |cdocsamp.tex| with include files
|cdocsch1.tex|, |cdocsch2.tex|, |cdocspt3.tex|, |cdocspt4.tex|,
|cdocsdrf.tex|, |cdocsfn1.tex|, |cdocsfn2.tex|.
Then copy the file |childdoc.def| to an appropriate directory of your \LaTeX{}
distribution, e.g.\ \textit{texmf-root}|/tex/latex/childdoc|.
\end{itemize}

%%%%%%%%%%%%%%%%%%%%%%%%%%%%%%%%%%%%%%%%%%%%%%%%%%%%%%%%%%%%%%%%%%%%%%%%%%%%%%%%
\subsection{Related CTAN Packages}

There are several other packages which offer a similar functionality:
%
\begin{itemize}
\item
The packages
\href{http://ctan.org/pkg/docmute}{\textsf{docmute}},
\href{http://ctan.org/pkg/includex}{\textsf{includex}} and
\href{http://ctan.org/pkg/standalone}{\textsf{standalone}}
provide commands to include only the document body of
a child file thus allowing both files to be compiled individually.
\item
The packages \href{http://ctan.org/pkg/subdocs}{\textsf{subdocs}}
and \href{http://ctan.org/pkg/subfiles}{\textsf{subfiles}}
provide structures in which the main and child documents can be
encapsulated and allowing them to be compiled individually.
The inclusion mechanism is different from the conventional |\include|.
\item
The package \href{http://ctan.org/pkg/combine}{\textsf{combine}}
is an elaborate solution to combine several documents into one.
\end{itemize}
%
See also the CTAN topic \href{http://ctan.org/topic/subdocs}{\textsf{subdocs}}
for further related packages.
The present package differs from the above solutions in that
a document structure constructed with the conventional |\include| mechanism
just needs two extra commands at the top of every file
such that all constituent files can be compiled individually.

%%%%%%%%%%%%%%%%%%%%%%%%%%%%%%%%%%%%%%%%%%%%%%%%%%%%%%%%%%%%%%%%%%%%%%%%%%%%%%%%
%\subsection{Feature Suggestions}
%
%The following is a list of features which may be useful for future
%versions of this package:
%%
%\begin{itemize}
%\item
%\ldots
%\end{itemize}

%%%%%%%%%%%%%%%%%%%%%%%%%%%%%%%%%%%%%%%%%%%%%%%%%%%%%%%%%%%%%%%%%%%%%%%%%%%%%%%%
\subsection{Revision History}

%%%%%%%%%%%%%%%%%%%%%%%%%%%%%%%%%%%%%%%%
\paragraph{v2.0:} 2018/12/30

\begin{itemize}
\item
immediate forward processing
\item
added |\childdocby| mechanism
\item
manual restructured
\end{itemize}

%%%%%%%%%%%%%%%%%%%%%%%%%%%%%%%%%%%%%%%%
\paragraph{v1.6:} 2018/01/17

\begin{itemize}
\item
application for development of include files
\item
corrections to manual
\end{itemize}

%%%%%%%%%%%%%%%%%%%%%%%%%%%%%%%%%%%%%%%%
\paragraph{v1.5:} 2017/05/21

\begin{itemize}
\item
more complete structuring introduced
\item
|\childdocof| introduced
\item
|\childdoc| renamed to |\childdocmain|
\item
|\childredirect| renamed to |\childdocforward| and |\childdocforwardprefix|
and functionality expanded
\end{itemize}

%%%%%%%%%%%%%%%%%%%%%%%%%%%%%%%%%%%%%%%%
\paragraph{v1.0:} 2017/04/27

\begin{itemize}
\item
manual and install package
\item
first version published on CTAN
\end{itemize}

%%%%%%%%%%%%%%%%%%%%%%%%%%%%%%%%%%%%%%%%
\paragraph{v0.6:} 2017/04/26

\begin{itemize}
\item
redirection mechanism added
\end{itemize}

%%%%%%%%%%%%%%%%%%%%%%%%%%%%%%%%%%%%%%%%
\paragraph{v0.5:} 2017/04/26

\begin{itemize}
\item
functionality in definition file
\end{itemize}


%%%%%%%%%%%%%%%%%%%%%%%%%%%%%%%%%%%%%%%%%%%%%%%%%%%%%%%%%%%%%%%%%%%%%%%%%%%%%%%%
%%%%%%%%%%%%%%%%%%%%%%%%%%%%%%%%%%%%%%%%%%%%%%%%%%%%%%%%%%%%%%%%%%%%%%%%%%%%%%%%
%%%%%%%%%%%%%%%%%%%%%%%%%%%%%%%%%%%%%%%%%%%%%%%%%%%%%%%%%%%%%%%%%%%%%%%%%%%%%%%%
\appendix

\settowidth\MacroIndent{\rmfamily\scriptsize 000\ }

 \DocInput{childdoc.dtx}

\end{document}
%</driver>
% \fi
%
% %%%%%%%%%%%%%%%%%%%%%%%%%%%%%%%%%%%%%%%%%%%%%%%%%%%%%%%%%%%%%%%%%%%%%%%%%%%%%%
% %%%%%%%%%%%%%%%%%%%%%%%%%%%%%%%%%%%%%%%%%%%%%%%%%%%%%%%%%%%%%%%%%%%%%%%%%%%%%%
% \section{Sample}
%\iffalse
%<*samplemain>
%\fi
%
% The following presents a sample document
% with two chapters, two parts, a title page,
% a compile flag as well as three forwarding files to set the flag.
% It consists of eight |.tex| files:
% \begin{center}
% \begin{tabular}{ll}
% |cdocsamp.tex|&main file\\
% |cdocsch1.tex|&include file for chapter 1\\
% |cdocsch2.tex|&include file for chapter 2\\
% |cdocspt3.tex|&include file for part 3\\
% |cdocspt4.tex|&include file for part 4\\
% |cdocsdrf.tex|&forwarding file for main file in draft mode\\
% |cdocsfi1.tex|&forwarding file for final version of chapter 1\\
% |cdocsfi2.tex|&forwarding file for final version of chapter 2\\
% \end{tabular}
% \end{center}
% Each of the eight files can be compiled directly by the \LaTeX{} compiler.
%
% %%%%%%%%%%%%%%%%%%%%%%%%%%%%%%%%%%%%%%
% \paragraph{Main File.}
%
% The main file is called |cdocsamp.tex|.
%
% Load the \textsf{childdoc} definitions and
% declare the filename for the main document:
%    \begin{macrocode}
\input{childdoc.def}
\childdocmain{}
%    \end{macrocode}

% Optional override for |\version| flag:
%    \begin{macrocode}
%%\ifchilddoc\else\providecommand{\version}{draft}\fi
%    \end{macrocode}

% Define the default values for the |\version| flag
% (|final| for the main file and |draft| for childs):
%    \begin{macrocode}
\ifchilddoc
\providecommand{\version}{draft}
\else
\providecommand{\version}{final}
\fi
%    \end{macrocode}

% Load the standard document class:
%    \begin{macrocode}
\documentclass[12pt]{article}
%    \end{macrocode}

% Start the document body:
%    \begin{macrocode}
\begin{document}
%    \end{macrocode}

% Declare a title page.
% Print title, part of document being processed and version flag:
%    \begin{macrocode}
\addtocounter{page}{-1}
\begin{center}
{\LARGE\bfseries{}childdoc example\par}
\vspace{1cm}
\ifchilddoc
\ifchilddocmanual part\else chapter\fi:
`\childdocname' of `\childdocjob'\par
\else
main document: `\childdocjob'\par
\fi
version: \version\par
\end{center}
\newpage
%    \end{macrocode}

% Manually include selected file,
% otherwise process as usual:
%    \begin{macrocode}
\ifchilddocmanual
\section*{part `\childdocname'}
\input{\childdocname}
\else
%    \end{macrocode}

% Include the two chapters:
%    \begin{macrocode}
\include{cdocsch1}
\include{cdocsch2}
%    \end{macrocode}

% Include the two parts unless only chapters should be displayed:
%    \begin{macrocode}
\ifchilddoc\else
\section{part three}
\input{cdocspt3}
\section{part four}
\input{cdocspt4}
\fi
%    \end{macrocode}

% Process as usual until here:
%    \begin{macrocode}
\fi
%    \end{macrocode}

% End of document body:
%    \begin{macrocode}
\end{document}
%    \end{macrocode}
%\iffalse
%</samplemain>
%\fi
%
% %%%%%%%%%%%%%%%%%%%%%%%%%%%%%%%%%%%%%%
% \paragraph{Chapter Include Files.}
%
% The include files are called |cdocsch1.tex| and |cdocsch2.tex|.
%
%\iffalse
%<*samplechap1|samplechap2>
%\fi

% Optional override for |\version| flag:
%    \begin{macrocode}
%%\providecommand{\version}{final}
%    \end{macrocode}

% Include the main document:
%    \begin{macrocode}
\input{childdoc.def}
\childdocof{cdocsamp}
%    \end{macrocode}

%\iffalse
%</samplechap1|samplechap2>
%\fi
%
%\iffalse
%<*samplechap1>
%\fi
% Some text for chapter 1:
%    \begin{macrocode}
\section{one}
some text in chapter one
%    \end{macrocode}

%\iffalse
%</samplechap1>
%\fi
% Some text for chapter 2:
%\iffalse
%<*samplechap2>
%\fi
%    \begin{macrocode}
\section{two}
more text in chapter two
%    \end{macrocode}

%\iffalse
%</samplechap2>
%\fi
%
% %%%%%%%%%%%%%%%%%%%%%%%%%%%%%%%%%%%%%%
% \paragraph{Part Include Files.}
%
% The include files are called |cdocspt3.tex| and |cdocspt4.tex|.
%
%\iffalse
%<*samplepart3|samplepart4>
%\fi

% Optional override for |\version| flag:
%    \begin{macrocode}
%%\providecommand{\version}{final}
%    \end{macrocode}

% Include the main document:
%    \begin{macrocode}
\input{childdoc.def}
\childdocby{cdocsamp}
%    \end{macrocode}

%\iffalse
%</samplepart3|samplepart4>
%\fi
%
%\iffalse
%<*samplepart3>
%\fi
% Some text for part 3:
%    \begin{macrocode}
some text in part three
%    \end{macrocode}

%\iffalse
%</samplepart3>
%\fi
% Some text for part 4:
%\iffalse
%<*samplepart4>
%\fi
%    \begin{macrocode}
more text in part four
%    \end{macrocode}

%\iffalse
%</samplepart4>
%\fi
%
% %%%%%%%%%%%%%%%%%%%%%%%%%%%%%%%%%%%%%%
% \paragraph{Forwarding for a Complete Draft.}
%
% The following forwarding file |cdocsdrf.tex|
% compiles the main document in draft mode:
%\iffalse
%<*sampledraft>
%\fi
%    \begin{macrocode}
\def\version{draft}
\input{childdoc.def}
\childdocforward{cdocsamp}
%    \end{macrocode}

%\iffalse
%</sampledraft>
%\fi
%
% %%%%%%%%%%%%%%%%%%%%%%%%%%%%%%%%%%%%%%
% \paragraph{Forwarding for Final Version of the Chapters.}
%
% The following forwarding files |cdocsfn1.tex| and |cdocsfn2.tex|
% (with identical content)
% compile the final versions of the child documents
% |cdocsch1.tex| and |cdocsch2.tex|, respectively:
%\iffalse
%<*samplefinal>
%\fi
%    \begin{macrocode}
\def\version{final}
\input{childdoc.def}
\childdocforwardprefix[cdocsamp]{cdocsfn}{cdocsch}
%    \end{macrocode}

%\iffalse
%</samplefinal>
%\fi
%
% %%%%%%%%%%%%%%%%%%%%%%%%%%%%%%%%%%%%%%
% \paragraph{Command Line Processing.}
%
% The following three command lines generate the output files
% |cdocscld|, |cdocscl1| and |cdocscl2|
% which should be identical to
% |cdocsdrf|, |cdocsch1| and |cdocsfn2|, respectively:
% \begin{center}
% \begin{tabular}{l}
% |latex -jobname cdocscld \|\\
% |  "\def\version{draft}\input{childdoc.def}\childdocforward{cdocsamp}"|\\
% |latex -jobname cdocscl1 \|\\
% |  "\input{childdoc.def}\childdocforward[cdocsamp]{cdocsch1}"|\\
% |latex -jobname cdocscl2 \|\\
% |  "\def\version{final}\input{childdoc.def}\childdocforward{cdocsch2}"|
% \end{tabular}
% \end{center}
% Note that the trailing backslash on each first line
% merely continues the input to the second line
% (for convenient cut ant paste).
% Furthermore, the command |latex| can be replaced by any
% of its alternative versions such as |pdflatex|.
%
% %%%%%%%%%%%%%%%%%%%%%%%%%%%%%%%%%%%%%%%%%%%%%%%%%%%%%%%%%%%%%%%%%%%%%%%%%%%%%%
% %%%%%%%%%%%%%%%%%%%%%%%%%%%%%%%%%%%%%%%%%%%%%%%%%%%%%%%%%%%%%%%%%%%%%%%%%%%%%%
% \section{Implementation}
%\iffalse
%<*package>
%\fi
%
% This section describes the definitions file |childdoc.def|.

% The definitions cannot be loaded using |\usepackage| or |\RequirePackage|
% which has a mechanism to prevent loading a style file more than once.
% When loading the definitions by means of |\input|
% multiple instances have to be prevented manually:
%\iffalse
%This code needs to be before the `\ProvidesFile' directive
%which is defined at the beginning of this file.
%Therefore it is also placed there and commented out here.
%</package>
%<*discard>
%\fi
%    \begin{macrocode}
\ifdefined\childdocmain\endinput\fi
%    \end{macrocode}
%\iffalse
%</discard>
%<*package>
%\fi
%
% \macro{\ifchilddoc}
% \macro{\ifchilddocmanual}
% The conditional |\ifchilddoc| tells whether a
% child (true) or main (false) document is being compiled.
% The conditional |\ifchilddocmanual| tells whether
% the |\includeonly| mechanism is used (false) or
% the selection of child files must be performed manually (true).
% The definitions initialise to false:
%    \begin{macrocode}
\newif\ifchilddoc
\newif\ifchilddocmanual
%    \end{macrocode}

% \macro{\childdocname}
% \macro{\childdocjob}
% The macro |\childdocname| stores the name of the main document
% to be compiled. The macro |\childdocjob| stores the name of
% the document on which the \LaTeX{} compiler was originally invoked.
% The content of |\jobname| cannot be compared
% to filenames specified in the source due to different catcodes.
% The following code rescans |\jobname|, stores the result
% in |\childdocname| and saves a copy in |\childdocjob|:
%    \begin{macrocode}
\edef\childdocname{\scantokens\expandafter{\jobname\noexpand}}
\let\childdocjob\childdocname
%    \end{macrocode}

% \macro{\childdocdisable}
% The macro |\childdocdisable| prevents the main file
% from being processed more than once.
% At this stage, the main document command |\childdocmain|
% is assumed to be called once again where it should do nothing.
% Any subsequent call to it should prevent
% a secondary processing of the main document
% It overwrites the forwarding commands
% |\childdocof| and |\childdocforward|
% with empty macros to prevent further inclusions of the main document:
%    \begin{macrocode}
\newcommand{\childdocdisable}
{
  \renewcommand{\childdocmain}[1]{\renewcommand{\childdocmain}[1]{\endinput}}
  \renewcommand{\childdocof}[1]{}
  \renewcommand{\childdocby}[2][]{}
  \renewcommand{\childdocforward}[2][]{}
  \renewcommand{\childdocdisable}{}
}
%    \end{macrocode}

% \macro{\childdocmain}
% The macro |\childdocmain| is to be called at the top of the main file
% with nothing or the main filename (without extension) as argument.
% First, it breaks loops.
% If the argument is not empty and does not match |\childdocname|
% (which is set by the first inclusion of |childdoc.def|),
% |\ifchilddoc| is set to true, |\includeonly| is applied to the child file
% and |\jobname| is set to the main file
% (for proper handling of |.aux| files):
%    \begin{macrocode}
\newcommand{\childdocmain}[1]
{
  \childdocdisable\childdocmain{}
  \if?#1?\else
    \begingroup
      \def\childdoctmp{#1}
      \ifx\childdoctmp\childdocname
        \def\childdoctmp{}
      \else
        \def\childdoctmp
        {
          \childdoctrue
          \includeonly{\childdocname}
          \def\childdocjob{#1}
          \def\jobname{#1}
        }
      \fi
      \expandafter
    \endgroup
    \childdoctmp
  \fi
}
%    \end{macrocode}

% \macro{\childdocof}
% The command |\childdocof| redirects
% compilation to the main file |#1|.
%    \begin{macrocode}
\newcommand{\childdocof}[1]
{
  \childdocdisable
  \childdoctrue
  \includeonly{\childdocname}
  \def\jobname{#1}
  \def\childdocjob{#1}
  \input{#1}
}
%    \end{macrocode}

% \macro{\childdocby}
% The command |\childdocby| ....
%    \begin{macrocode}
\newcommand{\childdocby}[2][]
{
  \childdocdisable
  \childdoctrue
  \childdocmanualtrue
  \if?#1?\else
    \def\jobname{#2}
  \fi
  \def\childdocjob{#2}
  \input{#2}
  \endinput
}
%    \end{macrocode}

% \macro{\childdocforward}
% The command |\childdocforward| redirects
% compilation to the main file or
% (if the optional argument is given) a child file.
% Parameters are set as if the main file
% or a child file starting with |\childdocof| was compiled.
% Then compilation is handed over to the main file:
%    \begin{macrocode}
\newcommand{\childdocforward}[2][]
{
  \begingroup
    \if?#1?
      \def\childdoctmp
      {
        \def\childdocname{#2}
        \def\childdocjob{#2}
        \def\jobname{#2}
        \input{#2}
        \endinput
      }
    \else
      \def\childdoctmp
      {
        \childdocdisable
        \def\childdocname{#2}
        \childdoctrue
        \includeonly{#2}
        \def\childdocjob{#1}
        \def\jobname{#1}
        \input{#1}
        \endinput
      }
    \fi
    \expandafter
  \endgroup
  \childdoctmp
}
%    \end{macrocode}

% \macro{\childdocforwardprefix}
% The command |\childdocforwardprefix| redirects
% compilation to the main or a child file by means of a pattern.
% The prefix |#1| in the current filename is replaced by |#2|
% and the suffix of the current filename is kept
% (it is assumed that the filename does not contain the substring `|~~~|'
% which is used as a delimiter).
% Compilation is handed over to the new file by |\childdocforward|:
%    \begin{macrocode}
\newcommand{\childdocforwardprefix}[3][]
{
  \begingroup
    \def\childdocextract #2##1~~~{\def\childdoctmp{\childdocforward[#1]{#3##1}}}
    \expandafter\childdocextract\childdocname~~~
    \expandafter
  \endgroup
  \childdoctmp
}
%    \end{macrocode}

% \macro{\childdoc}
% The deprecated macro |\childdoc| is a legacy version of |\childdocmain|:
%    \begin{macrocode}
\newcommand{\childdoc}{\childdocmain}
%    \end{macrocode}

% \macro{\childdocredirect}
% The deprecated macro |\childdocredirect| is a legacy version
% of |\childdocforward| and |\childdocforwardprefix|:
%    \begin{macrocode}
\newcommand{\childdocredirect}[2][]
{
  \begingroup
    \if?#1?
      \def\childdoctmp{\childdocforward{#2}}
    \else
      \def\childdoctmp{\childdocforwardprefix{#1}{#2}}
    \fi
    \expandafter
  \endgroup
  \childdoctmp
}
%    \end{macrocode}

%\iffalse
%</package>
%\fi
%
\endinput

\childdocforwardprefix[cdocsamp]{cdocsfn}{cdocsch}
%    \end{macrocode}

%\iffalse
%</samplefinal>
%\fi
%
% %%%%%%%%%%%%%%%%%%%%%%%%%%%%%%%%%%%%%%
% \paragraph{Command Line Processing.}
%
% The following three command lines generate the output files
% |cdocscld|, |cdocscl1| and |cdocscl2|
% which should be identical to
% |cdocsdrf|, |cdocsch1| and |cdocsfn2|, respectively:
% \begin{center}
% \begin{tabular}{l}
% |latex -jobname cdocscld \|\\
% |  "\def\version{draft}% \iffalse
%
% childdoc.dtx Copyright (C) 2017-2018 Niklas Beisert
%
% This work may be distributed and/or modified under the
% conditions of the LaTeX Project Public License, either version 1.3
% of this license or (at your option) any later version.
% The latest version of this license is in
%   http://www.latex-project.org/lppl.txt
% and version 1.3 or later is part of all distributions of LaTeX
% version 2005/12/01 or later.
%
% This work has the LPPL maintenance status `maintained'.
%
% The Current Maintainer of this work is Niklas Beisert.
%
% This work consists of the files childdoc.dtx and childdoc.ins
% and the derived files childdoc.def and cdocsamp.tex with
% cdocsch1.tex, cdocsch2.tex, cdocsdrf.tex, cdocsfn1.tex, cdocsfn2.tex.
%
%<package>\ifdefined\childdocmain\endinput\fi
%<package>\ProvidesFile{childdoc.def}[2018/12/30 v2.0 child document driver]
%<samplemain>\ProvidesFile{cdocsamp.tex}[2018/12/30 v2.0 sample for childdoc]
%<*driver>
%\ProvidesFile{childdoc.drv}[2018/12/30 v2.0 childdoc reference manual file]
\PassOptionsToClass{10pt,a4paper}{article}
\documentclass{ltxdoc}

\usepackage[margin=35mm]{geometry}
\usepackage{hyperref}
\usepackage{hyperxmp}
\usepackage[usenames]{color}

\hypersetup{colorlinks=true}
\hypersetup{pdfstartview=FitH}
\hypersetup{pdfpagemode=UseNone}
\hypersetup{pdfsource={}}
\hypersetup{pdflang={en-UK}}
\hypersetup{pdfcopyright={Copyright 2017-2018 Niklas Beisert.
  This work may be distributed and/or modified under the
  conditions of the LaTeX Project Public License, either version 1.3
  of this license or (at your option) any later version.}}
\hypersetup{pdflicenseurl={http://www.latex-project.org/lppl.txt}}
\hypersetup{pdfcontactaddress={ETH Zurich, ITP, HIT K,
  Wolfgang-Pauli-Strasse 27}}
\hypersetup{pdfcontactpostcode={8093}}
\hypersetup{pdfcontactcity={Zurich}}
\hypersetup{pdfcontactcountry={Switzerland}}
\hypersetup{pdfcontactemail={nbeisert@itp.phys.ethz.ch}}
\hypersetup{pdfcontacturl={http://people.phys.ethz.ch/\xmptilde nbeisert/}}

\newcommand{\secref}[1]{\hyperref[#1]{section \ref*{#1}}}

\parskip1ex
\parindent0pt
\let\olditemize\itemize
\def\itemize{\olditemize\parskip0pt}

\begin{document}

\title{The \textsf{childdoc} Package}
\hypersetup{pdftitle={The childdoc Package}}
\author{Niklas Beisert\\[2ex]
  Institut f\"ur Theoretische Physik\\
  Eidgen\"ossische Technische Hochschule Z\"urich\\
  Wolfgang-Pauli-Strasse 27, 8093 Z\"urich, Switzerland\\[1ex]
  \href{mailto:nbeisert@itp.phys.ethz.ch}
  {\texttt{nbeisert@itp.phys.ethz.ch}}}
\hypersetup{pdfauthor={Niklas Beisert}}
\hypersetup{pdfsubject={Manual for the LaTeX2e Package childdoc}}
\date{30 December 2018, \textsf{v2.0}}
\maketitle

\begin{abstract}\noindent
\textsf{childdoc} is a \LaTeXe{} package
that enables the direct compilation
of document sections included by |\include|
to individual files.
\end{abstract}

\begingroup
\parskip0ex
\tableofcontents
\endgroup

%%%%%%%%%%%%%%%%%%%%%%%%%%%%%%%%%%%%%%%%%%%%%%%%%%%%%%%%%%%%%%%%%%%%%%%%%%%%%%%%
%%%%%%%%%%%%%%%%%%%%%%%%%%%%%%%%%%%%%%%%%%%%%%%%%%%%%%%%%%%%%%%%%%%%%%%%%%%%%%%%
\section{Introduction}

\LaTeX{} provides a mechanism to structure a large document (such as a book)
into a main file and several child files (containing the chapters)
using the |\include| command.
This mechanism is beneficial for documents
which span hundreds of pages in order to
make the source file(s) more manageable.
Moreover, compilation can be restricted to
selected child files by means of the |\includeonly| command.
The latter feature can be used to reduce the compilation time while editing
(this was significantly more useful in the earlier days of \LaTeX{})
or to generate a smaller document which is easier to navigate.
Another application of |\includeonly| is to generate
documents consisting of selected parts of the complete document.

However, there are a few drawbacks of the plain |\include| mechanism:
\begin{itemize}
\item
The child files cannot be compiled on their own,
they can only be compiled via the main file.
A naive editing environment
(such as a text editor with an option
to have the current file processed by \LaTeX)
may require one to switch to the main file before compiling;
attempting to compile the child file produces errors.
\item
The main file must be modified (each time)
to adjust the |\includeonly| command
to the present needs. This easily leaves the main file in a messy state.
\item
The generated document will always carry the filename
of the main document. This is inconvenient if
several child files are to be compiled and
to be kept for distribution.
\end{itemize}

The present package provides a simple interface
to make child files individually compilable by \LaTeX{}.
Compiling a child file then has the same effect as compiling
the main file with an |\includeonly| command
to select the appropriate child.
Moreover the generated document will carry the name of the child
rather than the main file.
This resolves all three above issues.

This feature is meant to make the editing of books,
thesis documents and lecture notes somewhat more convenient.
However, the package can also be used efficiently for
composing a series of documents (such as exercise sheets)
which are typically distributed individually.
It then assists the author in generating the individual documents
(potentially in different versions)
as well as a document containing the collected series.
Another application is in developing style files
or other kinds of included material
where compilation of the style file could redirect
to a sample or test file.

%%%%%%%%%%%%%%%%%%%%%%%%%%%%%%%%%%%%%%%%%%%%%%%%%%%%%%%%%%%%%%%%%%%%%%%%%%%%%%%%
%%%%%%%%%%%%%%%%%%%%%%%%%%%%%%%%%%%%%%%%%%%%%%%%%%%%%%%%%%%%%%%%%%%%%%%%%%%%%%%%
\section{Usage}

First of all, the package \textsf{childdoc} is \emph{not} a standard
\LaTeXe{} |.sty| style file! Therefore it needs to be invoked in
a non-standard way.

%%%%%%%%%%%%%%%%%%%%%%%%%%%%%%%%%%%%%%%%%%%%%%%%%%%%%%%%%%%%%%%%%%%%%%%%%%%%%%%%
\subsection{Included Files}
\label{sec:include}

%%%%%%%%%%%%%%%%%%%%%%%%%%%%%%%%%%%%%%%%
\DescribeMacro{\childdocmain}
To use the package, add the commands
\begin{center}
\begin{tabular}{l}
|\input{childdoc.def}|\\
|\childdocmain{}|\\
\end{tabular}
\end{center}
at the very top of the main \LaTeX{} file,
in particular \emph{before} the |\documentclass| statement!
The argument of |\childdocmain| should be left empty
(but it must be present).

%%%%%%%%%%%%%%%%%%%%%%%%%%%%%%%%%%%%%%%%
\DescribeMacro{\childdocof}
Furthermore, add the commands
\begin{center}
\begin{tabular}{l}
|\input{childdoc.def}|\\
|\childdocof{|\textit{main}|}|\\
\end{tabular}
\end{center}
at the top of every child file \textit{child}
which is included by |\include{|\textit{child}|}|
from within the main file
(or at least for those files to be compiled individually).
The argument \textit{main} must be the filename of the main file.

There are a couple of
considerations in setting up the main and child documents:

%%%%%%%%%%%%%%%%%%%%%%%%%%%%%%%%%%%%%%%%
\paragraph{Restrictions.}

Please note the following restrictions:
\begin{itemize}
\item
|\childdocmain| must be called with one argument \textit{main}
to ensure compatibility with earlier version of the package.
It must either be empty (|\childdocmain{}|)
or precisely match the filename of the main file in which it is specified.
See \secref{sec:detection} for further information.
\item
The filename \textit{main} must be specified without the |.tex| extension.
\item
The filename \textit{main} is case sensitive
(even in case-insensitive file systems)
due to internal string comparison.
\item
The argument \textit{main} should be fully expanded, it cannot be a macro.
\item
Subdirectories and special characters should be avoided in filenames.
\item
The command |\childdocmain{|\textit{main}|}| must be followed by a whitespace.
It should not be followed immediately by another command
or by a comment mark `|%|'.
This is because the \TeX{} parser reads the token immediately following
the argument of |\childdocmain| and puts it
at the beginning of every child section;
however, a white\-space is ignored.
\end{itemize}

%%%%%%%%%%%%%%%%%%%%%%%%%%%%%%%%%%%%%%%%
\paragraph{Content of Main File.}

It is advisable to place all content in the child files included by |\include|.
Any output contained in the main file will appear in all child documents
unless suppressed manually;
it cannot be suppressed automatically by the |\includeonly| directive
and thus should normally be avoided.
A method to include some content in the main file
by means of conditional processing is described in \secref{sec:conditional}.

%%%%%%%%%%%%%%%%%%%%%%%%%%%%%%%%%%%%%%%%
\paragraph{Page Numbering.}

When only a part of the document is compiled,
the appropriate numbering of pages
(as well as other status parameters)
is determined from the |.aux| files.
The latter contain information from previous passes.
However this information needs to propagate through
all intermediate child documents.
Therefore the page numbering in child documents may well
be inconsistent until the complete document is compiled at least once.

A useful (if unconventional) way to always ensure a consistent
page numbering is to restart the numbering in each child document
and denote the pages by `\textit{child}|.|\textit{page}'
where \textit{child} represents the chapter/section number of the child file.
This can be achieved by the command
|\numberwithin{page}{|\textit{child}|}|
of the \textsf{amsmath} package
where \textit{child} can be |chapter| or |section|
depending on the chosen structuring.
Alternatively, one can modify the macro |\thepage| appropriately
and reset the counter |page| at the start of each child file.

%%%%%%%%%%%%%%%%%%%%%%%%%%%%%%%%%%%%%%%%%%%%%%%%%%%%%%%%%%%%%%%%%%%%%%%%%%%%%%%%
\subsection{Conditional Processing}
\label{sec:conditional}

The package provides a mechanism to compile different versions
of a document. To customise the versions further some conditional processing
can come in handy to distinguish which version is being compiled.
The package provides two macros to describe the compilation context:

%%%%%%%%%%%%%%%%%%%%%%%%%%%%%%%%%%%%%%%%
\DescribeMacro{\ifchilddoc}
The conditional |\ifchilddoc| distinguishes between the compilation of
child documents and the main document:
%
\begin{center}
|\ifchilddoc |\textit{child-code}| |[|\||else |\textit{main-code}]| \||fi|
\end{center}

%%%%%%%%%%%%%%%%%%%%%%%%%%%%%%%%%%%%%%%%
\DescribeMacro{\childdocname}
\DescribeMacro{\childdocjob}
The macro |\childdocname| contains the filename (without extension)
of the main or child file being processed.
Note that |\childdocjob| will always contain the name of the main file.

%%%%%%%%%%%%%%%%%%%%%%%%%%%%%%%%%%%%%%%%
\paragraph{Title Page.}

Conditional processing can be used to include a title or banner page
in the main document when proper precautions are taken.
Importantly, the code in the main file should ensure that the page counter
(as well as other status parameters which are stored in the |.aux| files)
takes the same value after the conditional processing.
Otherwise the page numbers may take divergent values
depending on which part is compiled.

For example, a title page could be declared by:
%
\begin{center}
\begin{tabular}{l}
|\ifchilddoc\||else|\\
|\addtocounter{page}{-1}|\\
\textit{code for title page}\\
|\newpage|\\
|\||fi|
\end{tabular}
\end{center}
%
A banner page for the child documents can be generated by:
%
\begin{center}
\begin{tabular}{l}
|\ifchilddoc|\\
|\addtocounter{page}{-1}|\\
\textit{code for banner page}\\
|\newpage|\\
|\||fi|
\end{tabular}
\end{center}
%
Here one could write a message such as:
\begin{center}
|This is the part \childdocname{} of \childdocjob{}.|
\end{center}

%%%%%%%%%%%%%%%%%%%%%%%%%%%%%%%%%%%%%%%%%%%%%%%%%%%%%%%%%%%%%%%%%%%%%%%%%%%%%%%%
\subsection{Flags}
\label{sec:flags}

The package makes it easy to generate different versions
of the main or child documents.
To this end compilation flags can be defined
and assigned different default values.
They will be particularly useful in conjunction
with the forwarding mechanism described in \secref{sec:forward}.

For example, it may be useful to have a flag |\version|
which can be set to |draft| or |final|.
The document source will contain some conditional code
depending on the value of |\version|.
Suppose further, the flag should default to |final| for the main file
and to |draft| for child files
which is a natural assignment for editing the document.
This is achieved by placing the following code
in the preamble of the main document
(below the |\childdocmain| directive):
%
\begin{center}
\begin{tabular}{l}
|\ifchilddoc|\\
|\providecommand{\version}{draft}|\\
|\||else|\\
|\providecommand{\version}{final}|\\
|\||fi|
\end{tabular}
\end{center}
%
The definition by |\providecommand| makes sure
that previous definitions are not overwritten.
Further statements |\providecommand{\version}{...}|
can thus be added before the above code to override it.

For the main file, one might add a line
(between |\childdocmain| and the above block)
%
\begin{center}
|%\ifchilddoc\||else\providecommand{\version}{draft}\||fi|
\end{center}
%
which can be uncommented to produce a draft version.
Likewise one can add a line to the very top of a child file
(above the |\childdocof{|\textit{main}|}| directive)
%
\begin{center}
|%\providecommand{\version}{final}|
\end{center}
%
which can be uncommented to produce the final version of this child document.

%%%%%%%%%%%%%%%%%%%%%%%%%%%%%%%%%%%%%%%%%%%%%%%%%%%%%%%%%%%%%%%%%%%%%%%%%%%%%%%%
\subsection{Forwarding}
\label{sec:forward}

Different versions of the main or child documents
using compilation flags as described in \secref{sec:flags}
can be (permanently) stored in different files
for convenient compilation, viewing and distribution.
To this end, the package defines a command
to pass on compilation to a different file:

%%%%%%%%%%%%%%%%%%%%%%%%%%%%%%%%%%%%%%%%
\DescribeMacro{\childdocforward}
The command |\childdocforward| redirects processing to
another source file:
%
\begin{center}
\begin{tabular}{l}
|\input{childdoc.def}|\\
|\childdocforward[|\textit{main}|]{|\textit{dest}|}|\\
\end{tabular}
\end{center}
%
The argument \textit{dest} is the destination file
(without extension).
It should be the main file or one of the child files.
Note that further \textsf{childdoc} directives
such as |\childdocof| and |\childdocforward|
in the indicated file will be processed in this form.
The optional argument \textit{main}
passes on directly to the main file \textit{main}
while pretending to compile the child \textit{dest}.
This form behaves as if \textit{dest}
issues |\childdocof{|\textit{main}|}| right away,
and no further \textsf{childdoc} directives will be processed.

%%%%%%%%%%%%%%%%%%%%%%%%%%%%%%%%%%%%%%%%
\DescribeMacro{\...prefix}
In the alternative form |\childdocforwardprefix|,
%
\begin{center}
\begin{tabular}{l}
|\input{childdoc.def}|\\
|\childdocforwardprefix[|\textit{main}|]{|\textit{prefix}|}{|\textit{dest}|}|
\end{tabular}
\end{center}
%
the destination file is determined by a pattern
depending on the current file:
To make this work, the current file must be called
`{\textit{prefix}\hspace{0.2em}\textit{suffix}}'
with \textit{prefix} matching precisely the argument.
Processing is then passed on to the file
`{\textit{dest}\hspace{0.2em}\textit{suffix}}'.
Surely, the same effect is achieved by
directly specifying the
argument `{\textit{dest}\hspace{0.2em}\textit{suffix}}'
in the first form.
However, that requires to set up a different file
for each child. With the alternative form of the command
all these files can have exactly the same content
which simplifies setting them up and maintaining them.

For example, the following file |draft.tex|
with a compilation flag |\version| as described in \secref{sec:flags}
compiles the main document as a draft:
%
\begin{center}
\begin{tabular}{l}
|\def\version{draft}|\\
|\input{childdoc.def}|\\
|\childdocforward{|\textit{main}|}|
\end{tabular}
\end{center}
%
Likewise, the following files |final|\textit{nn}|.tex|
compile the final version of the child document
|child|\textit{nn}|.tex|:
%
\begin{center}
\begin{tabular}{l}
|\def\version{final}|\\
|\input{childdoc.def}|\\
|\childdocforwardprefix{final}{child}|
\end{tabular}
\end{center}
%

Note that when several versions of a main file and/or of each child file
are to be generated, it may be convenient to set up a |Makefile| or
shell script to automatise the process.

%%%%%%%%%%%%%%%%%%%%%%%%%%%%%%%%%%%%%%%%%%%%%%%%%%%%%%%%%%%%%%%%%%%%%%%%%%%%%%%%
\subsection{Command Line Processing}
\label{sec:commandline}

The effect of redirection files can also be achieved by invoking
the \LaTeX{} compiler with a more elaborate command line.
Most conveniently this should be done as part
of a shell script or a |Makefile|.

When using \textsf{childdoc} in the main file, the following
command lines effectively perform a redirection
(note that depending on the shell being used,
backslashes may have to be doubled: `|\|' $\to$ `|\\|'):
%
\begin{center}
|... -jobname "|\textit{target}|" |\\|"|[\textit{flags}]%
|\input{childdoc.def}\childdocforward[|\textit{main}|]{|\textit{dest}|}"|
\end{center}
%
Here \textit{target} is the name of the output file,
\textit{main} is the name of the main file
and \textit{dest} is the name of the main or child file to be processed
(all filenames without extensions).
The optional argument \textit{main} can be omitted
if \textit{main} matches \textit{dest}.
Optionally, compilation \textit{flags} can be defined via |\def| commands.
This command line makes the \TeX{} engine believe
it is compiling the file \textit{target}
whose content is specified as the latter parameter.
The provided code then forwards the processing to
\textit{main} or \textit{dest} as described in \secref{sec:forward}.

%%%%%%%%%%%%%%%%%%%%%%%%%%%%%%%%%%%%%%%%%%%%%%%%%%%%%%%%%%%%%%%%%%%%%%%%%%%%%%%%
\subsection{Include by Input}
\label{sec:input}

Including child documents by |\include| has some restrictions by design.
Most notably, the content of a child document always occupies
its own set of pages; pages cannot be shared between child documents.
Usually, this behaviour makes perfect sense
because each child document contain an essential part of the document.
However, in some situations it may be desirable to compose
a document from a collection of parts
without having mandatory page breaks between then.
For this case, the package
provides a mechanism to include parts
by |\input| which can also be processed individually.
However, by construction this mechanism
requires manual handling of the content to be output.

%%%%%%%%%%%%%%%%%%%%%%%%%%%%%%%%%%%%%%%%
\DescribeMacro{\ifchilddocmanual}
The main file should be prepared as usual, see \secref{sec:include}.
However, the document body must make a distinction
between processing of an individual part and of the main document, e.g.:
%
\begin{center}
\begin{tabular}{l}
|\ifchilddocmanual|\\
|\input{\childdocname}|\\
|\||else|\\
\textit{document body with }|\input{|\textit{part}|}|\\
|\||fi|
\end{tabular}
\end{center}
%
The conditional |\ifchilddocmanual| is true whenever
a part to be included by |\input| is being compiled,
and the name of the part is stored in |\childdocname|.

%%%%%%%%%%%%%%%%%%%%%%%%%%%%%%%%%%%%%%%%
\DescribeMacro{\childdocby}
Each part to be included by |\input| should start with:
%
\begin{center}
\begin{tabular}{l}
|\input{childdoc.def}|\\
|\childdocby{|\textit{main}|}|\\
\end{tabular}
\end{center}
%
The directive |\childdocby| is similar to |\childdocof|
described in \secref{sec:include},
but the subsequent selection of content must be done manually.
To that end, both |\ifchilddoc| and |\ifchilddocmanual|
will be true upon processing of a part,
and the name of the part is stored in |\childdocname|.
Note that |\jobname| will be set to the filename of the current part
so that each part receives an individual |.aux| file
that does not interfere with the |.aux| file(s) of the main document.
This behaviour can be altered by the alternative form
|\childdocby[*]{|\textit{main}|}| (with a non-empty optional argument)
which uses the |.aux| file of the main document
by setting |\jobname| to \textit{main}.

%%%%%%%%%%%%%%%%%%%%%%%%%%%%%%%%%%%%%%%%%%%%%%%%%%%%%%%%%%%%%%%%%%%%%%%%%%%%%%%%
\subsection{Driver Development}
\label{sec:driver}

The \textsf{childdoc} mechanism can also be use for the development
of definition files such as \LaTeX{} styles or classes.
This case differs from the above setup with multiple parts
included by |\include| in that no |\includeonly| should be invoked.
This can be achieved by starting the include file
(before |\ProvidesPackage|) with:
%
\begin{center}
\begin{tabular}{l}
|\input{childdoc.def}|\\
|\childdocforward{|\textit{main}|}|\\
\end{tabular}
\end{center}
%
or alternatively with:
%
\begin{center}
\begin{tabular}{l}
|\input{childdoc.def}|\\
|\childdocby{|\textit{main}|}|\\
\end{tabular}
\end{center}
%
Both forms have slightly different effects as described above.
The main file is prepared as usual, see \secref{sec:include}.

%%%%%%%%%%%%%%%%%%%%%%%%%%%%%%%%%%%%%%%%%%%%%%%%%%%%%%%%%%%%%%%%%%%%%%%%%%%%%%%%
\subsection{Legacy Detection}
\label{sec:detection}

The directive |\childdocmain| in the main file can detect
whether the complete document or merely a child is to be compiled
even without using the directive |\childdocof|.
This method is deprecated because it is less robust
and there is no compelling reason to use it;
it is merely provided for backward compatibility
and it may be removed in future versions.

If the detection mechanism is to be used,
it is mandatory to correctly specify
the filename of the main file as the argument of |\childdocmain|:
%
\begin{center}
\begin{tabular}{l}
|\input{childdoc.def}|\\
|\childdocmain{|\textit{main}|}|\\
\end{tabular}
\end{center}
%
If |\jobname| does not match the argument \textit{main} of |\childdocmain|,
it is assumed that |\jobname| points to the child file to be compiled.
When using |\childdocmain| with the main file specified as argument,
it suffices to start a child file
with just |\input{|\textit{main}|}|
without loading of the package and using |\childdocof|.
If instead all processing is done
with the appropriate \textsf{childdoc} directives,
the argument of \textit{main} of |\childdocmain| can be empty.

An alternative version of the command line processing described
in \secref{sec:commandline} using the detection mechanism reads:
%
\begin{center}
|... -jobname "|\textit{target}|" "|[\textit{flags}]%
[|\def\jobname{|\textit{dest}|}|]|\input{|\textit{main}|}"|
\end{center}

%%%%%%%%%%%%%%%%%%%%%%%%%%%%%%%%%%%%%%%%%%%%%%%%%%%%%%%%%%%%%%%%%%%%%%%%%%%%%%%%
\subsection{Manual Code}
\label{sec:manual}

In case one cannot be certain whether the definitions file |childdoc.def|
is installed on the target \TeX{} distribution
and one prefers not to ship it,
it is conceivable to paste a few relevant commands into the sources.

To that end, drop all statements |\input{childdoc.def}|
and perform the replacements as outlined below.
Instead of |\childdocmain{|\textit{main}|}| add the following code
to the top of the main file:
%
\begin{center}
\begin{tabular}{l}
|\||ifdefined\childdocname\endinput\||fi\newif\ifchilddoc|\\
|\edef\childdocname{\scantokens\expandafter{\jobname\noexpand}}|\\
|\def\childdocmain{|\textit{main}|}\||ifx\childdocmain\childdocname\||else|\\
|\childdoctrue\includeonly{\childdocname}\let\jobname\childdocmain\||fi|\\
\end{tabular}
\end{center}
%
Instead of |\childdocof{|\textit{main}|}| just include the main file
at the top of each child file:
%
\begin{center}
|\input{|\textit{main}|}|
\end{center}
%
A simple redirection |\childdocforward{|\textit{dest}|}| is achieved by:
%
\begin{center}
|\def\jobname{|\textit{dest}|}\input{\jobname}|
\end{center}
%
The redirection with prefix
|\childdocforwardprefix[|\textit{prefix}|]{|\textit{dest}|}|
is accomplished by:
%
\begin{center}
\begin{tabular}{l}
|{\edef\jobname{\scantokens\expandafter{\jobname\noexpand}}|\\
|\def\redirectjob |\textit{prefix}|#1~~~{\gdef\jobname{|\textit{dest}|#1}}|\\
|\expandafter\redirectjob\jobname~~~}\input{\jobname}|
\end{tabular}
\end{center}

In an alternative approach,
child documents can be compiled by a specific command line
without additional code or specific definitions:
%
\begin{center}
|... -jobname "|\textit{target}|" "|[\textit{flags}]%
|\includeonly{|\textit{dest}|}\input{|\textit{main}|}"|
\end{center}
%

%%%%%%%%%%%%%%%%%%%%%%%%%%%%%%%%%%%%%%%%%%%%%%%%%%%%%%%%%%%%%%%%%%%%%%%%%%%%%%%%
%%%%%%%%%%%%%%%%%%%%%%%%%%%%%%%%%%%%%%%%%%%%%%%%%%%%%%%%%%%%%%%%%%%%%%%%%%%%%%%%
\section{Information}

%%%%%%%%%%%%%%%%%%%%%%%%%%%%%%%%%%%%%%%%%%%%%%%%%%%%%%%%%%%%%%%%%%%%%%%%%%%%%%%%
\subsection{Copyright}

Copyright \copyright{} 2017--2018 Niklas Beisert

This work may be distributed and/or modified under the
conditions of the \LaTeX{} Project Public License, either version 1.3
of this license or (at your option) any later version.
The latest version of this license is in
  \url{http://www.latex-project.org/lppl.txt}
and version 1.3 or later is part of all distributions of \LaTeX{}
version 2005/12/01 or later.

This work has the LPPL maintenance status `maintained'.

The Current Maintainer of this work is Niklas Beisert.

This work consists of the files |README.txt|, |childdoc.ins| and |childdoc.dtx|
as well as the derived files |childdoc.def|, |cdocsamp.tex|
with |cdocsch1.tex|, |cdocsch2.tex|, |cdocspt3.tex|, |cdocspt4.tex|,
|cdocsdrf.tex|, |cdocsfn1.tex|, |cdocsfn2.tex|
as well as |childdoc.pdf|.

%%%%%%%%%%%%%%%%%%%%%%%%%%%%%%%%%%%%%%%%%%%%%%%%%%%%%%%%%%%%%%%%%%%%%%%%%%%%%%%%
\subsection{Files and Installation}

The package consists of the files:
%
\begin{center}
\begin{tabular}{ll}
    |README.txt|   & readme file \\
    |childdoc.ins| & installation file \\
    |childdoc.dtx| & source file \\
    |childdoc.def| & definition file \\
    |cdocsamp.tex| & sample main file \\
    |cdocsch1.tex| & sample include file \\
    |cdocsch2.tex| & sample include file \\
    |cdocspt3.tex| & sample part file \\
    |cdocspt4.tex| & sample part file \\
    |cdocsdrf.tex| & sample redirection file \\
    |cdocsfn1.tex| & sample redirection file \\
    |cdocsfn2.tex| & sample redirection file \\
    |childdoc.pdf| & manual
\end{tabular}
\end{center}
%
The distribution consists of the files
|README.txt|, |childdoc.ins| and |childdoc.dtx|.
%
\begin{itemize}
\item
Run (pdf)\LaTeX{} on |childdoc.dtx|
to compile the manual |childdoc.pdf| (this file).
\item
Run \LaTeX{} on |childdoc.ins| to create the definitions file |childdoc.def|
and the sample |cdocsamp.tex| with include files
|cdocsch1.tex|, |cdocsch2.tex|, |cdocspt3.tex|, |cdocspt4.tex|,
|cdocsdrf.tex|, |cdocsfn1.tex|, |cdocsfn2.tex|.
Then copy the file |childdoc.def| to an appropriate directory of your \LaTeX{}
distribution, e.g.\ \textit{texmf-root}|/tex/latex/childdoc|.
\end{itemize}

%%%%%%%%%%%%%%%%%%%%%%%%%%%%%%%%%%%%%%%%%%%%%%%%%%%%%%%%%%%%%%%%%%%%%%%%%%%%%%%%
\subsection{Related CTAN Packages}

There are several other packages which offer a similar functionality:
%
\begin{itemize}
\item
The packages
\href{http://ctan.org/pkg/docmute}{\textsf{docmute}},
\href{http://ctan.org/pkg/includex}{\textsf{includex}} and
\href{http://ctan.org/pkg/standalone}{\textsf{standalone}}
provide commands to include only the document body of
a child file thus allowing both files to be compiled individually.
\item
The packages \href{http://ctan.org/pkg/subdocs}{\textsf{subdocs}}
and \href{http://ctan.org/pkg/subfiles}{\textsf{subfiles}}
provide structures in which the main and child documents can be
encapsulated and allowing them to be compiled individually.
The inclusion mechanism is different from the conventional |\include|.
\item
The package \href{http://ctan.org/pkg/combine}{\textsf{combine}}
is an elaborate solution to combine several documents into one.
\end{itemize}
%
See also the CTAN topic \href{http://ctan.org/topic/subdocs}{\textsf{subdocs}}
for further related packages.
The present package differs from the above solutions in that
a document structure constructed with the conventional |\include| mechanism
just needs two extra commands at the top of every file
such that all constituent files can be compiled individually.

%%%%%%%%%%%%%%%%%%%%%%%%%%%%%%%%%%%%%%%%%%%%%%%%%%%%%%%%%%%%%%%%%%%%%%%%%%%%%%%%
%\subsection{Feature Suggestions}
%
%The following is a list of features which may be useful for future
%versions of this package:
%%
%\begin{itemize}
%\item
%\ldots
%\end{itemize}

%%%%%%%%%%%%%%%%%%%%%%%%%%%%%%%%%%%%%%%%%%%%%%%%%%%%%%%%%%%%%%%%%%%%%%%%%%%%%%%%
\subsection{Revision History}

%%%%%%%%%%%%%%%%%%%%%%%%%%%%%%%%%%%%%%%%
\paragraph{v2.0:} 2018/12/30

\begin{itemize}
\item
immediate forward processing
\item
added |\childdocby| mechanism
\item
manual restructured
\end{itemize}

%%%%%%%%%%%%%%%%%%%%%%%%%%%%%%%%%%%%%%%%
\paragraph{v1.6:} 2018/01/17

\begin{itemize}
\item
application for development of include files
\item
corrections to manual
\end{itemize}

%%%%%%%%%%%%%%%%%%%%%%%%%%%%%%%%%%%%%%%%
\paragraph{v1.5:} 2017/05/21

\begin{itemize}
\item
more complete structuring introduced
\item
|\childdocof| introduced
\item
|\childdoc| renamed to |\childdocmain|
\item
|\childredirect| renamed to |\childdocforward| and |\childdocforwardprefix|
and functionality expanded
\end{itemize}

%%%%%%%%%%%%%%%%%%%%%%%%%%%%%%%%%%%%%%%%
\paragraph{v1.0:} 2017/04/27

\begin{itemize}
\item
manual and install package
\item
first version published on CTAN
\end{itemize}

%%%%%%%%%%%%%%%%%%%%%%%%%%%%%%%%%%%%%%%%
\paragraph{v0.6:} 2017/04/26

\begin{itemize}
\item
redirection mechanism added
\end{itemize}

%%%%%%%%%%%%%%%%%%%%%%%%%%%%%%%%%%%%%%%%
\paragraph{v0.5:} 2017/04/26

\begin{itemize}
\item
functionality in definition file
\end{itemize}


%%%%%%%%%%%%%%%%%%%%%%%%%%%%%%%%%%%%%%%%%%%%%%%%%%%%%%%%%%%%%%%%%%%%%%%%%%%%%%%%
%%%%%%%%%%%%%%%%%%%%%%%%%%%%%%%%%%%%%%%%%%%%%%%%%%%%%%%%%%%%%%%%%%%%%%%%%%%%%%%%
%%%%%%%%%%%%%%%%%%%%%%%%%%%%%%%%%%%%%%%%%%%%%%%%%%%%%%%%%%%%%%%%%%%%%%%%%%%%%%%%
\appendix

\settowidth\MacroIndent{\rmfamily\scriptsize 000\ }

 \DocInput{childdoc.dtx}

\end{document}
%</driver>
% \fi
%
% %%%%%%%%%%%%%%%%%%%%%%%%%%%%%%%%%%%%%%%%%%%%%%%%%%%%%%%%%%%%%%%%%%%%%%%%%%%%%%
% %%%%%%%%%%%%%%%%%%%%%%%%%%%%%%%%%%%%%%%%%%%%%%%%%%%%%%%%%%%%%%%%%%%%%%%%%%%%%%
% \section{Sample}
%\iffalse
%<*samplemain>
%\fi
%
% The following presents a sample document
% with two chapters, two parts, a title page,
% a compile flag as well as three forwarding files to set the flag.
% It consists of eight |.tex| files:
% \begin{center}
% \begin{tabular}{ll}
% |cdocsamp.tex|&main file\\
% |cdocsch1.tex|&include file for chapter 1\\
% |cdocsch2.tex|&include file for chapter 2\\
% |cdocspt3.tex|&include file for part 3\\
% |cdocspt4.tex|&include file for part 4\\
% |cdocsdrf.tex|&forwarding file for main file in draft mode\\
% |cdocsfi1.tex|&forwarding file for final version of chapter 1\\
% |cdocsfi2.tex|&forwarding file for final version of chapter 2\\
% \end{tabular}
% \end{center}
% Each of the eight files can be compiled directly by the \LaTeX{} compiler.
%
% %%%%%%%%%%%%%%%%%%%%%%%%%%%%%%%%%%%%%%
% \paragraph{Main File.}
%
% The main file is called |cdocsamp.tex|.
%
% Load the \textsf{childdoc} definitions and
% declare the filename for the main document:
%    \begin{macrocode}
\input{childdoc.def}
\childdocmain{}
%    \end{macrocode}

% Optional override for |\version| flag:
%    \begin{macrocode}
%%\ifchilddoc\else\providecommand{\version}{draft}\fi
%    \end{macrocode}

% Define the default values for the |\version| flag
% (|final| for the main file and |draft| for childs):
%    \begin{macrocode}
\ifchilddoc
\providecommand{\version}{draft}
\else
\providecommand{\version}{final}
\fi
%    \end{macrocode}

% Load the standard document class:
%    \begin{macrocode}
\documentclass[12pt]{article}
%    \end{macrocode}

% Start the document body:
%    \begin{macrocode}
\begin{document}
%    \end{macrocode}

% Declare a title page.
% Print title, part of document being processed and version flag:
%    \begin{macrocode}
\addtocounter{page}{-1}
\begin{center}
{\LARGE\bfseries{}childdoc example\par}
\vspace{1cm}
\ifchilddoc
\ifchilddocmanual part\else chapter\fi:
`\childdocname' of `\childdocjob'\par
\else
main document: `\childdocjob'\par
\fi
version: \version\par
\end{center}
\newpage
%    \end{macrocode}

% Manually include selected file,
% otherwise process as usual:
%    \begin{macrocode}
\ifchilddocmanual
\section*{part `\childdocname'}
\input{\childdocname}
\else
%    \end{macrocode}

% Include the two chapters:
%    \begin{macrocode}
\include{cdocsch1}
\include{cdocsch2}
%    \end{macrocode}

% Include the two parts unless only chapters should be displayed:
%    \begin{macrocode}
\ifchilddoc\else
\section{part three}
\input{cdocspt3}
\section{part four}
\input{cdocspt4}
\fi
%    \end{macrocode}

% Process as usual until here:
%    \begin{macrocode}
\fi
%    \end{macrocode}

% End of document body:
%    \begin{macrocode}
\end{document}
%    \end{macrocode}
%\iffalse
%</samplemain>
%\fi
%
% %%%%%%%%%%%%%%%%%%%%%%%%%%%%%%%%%%%%%%
% \paragraph{Chapter Include Files.}
%
% The include files are called |cdocsch1.tex| and |cdocsch2.tex|.
%
%\iffalse
%<*samplechap1|samplechap2>
%\fi

% Optional override for |\version| flag:
%    \begin{macrocode}
%%\providecommand{\version}{final}
%    \end{macrocode}

% Include the main document:
%    \begin{macrocode}
\input{childdoc.def}
\childdocof{cdocsamp}
%    \end{macrocode}

%\iffalse
%</samplechap1|samplechap2>
%\fi
%
%\iffalse
%<*samplechap1>
%\fi
% Some text for chapter 1:
%    \begin{macrocode}
\section{one}
some text in chapter one
%    \end{macrocode}

%\iffalse
%</samplechap1>
%\fi
% Some text for chapter 2:
%\iffalse
%<*samplechap2>
%\fi
%    \begin{macrocode}
\section{two}
more text in chapter two
%    \end{macrocode}

%\iffalse
%</samplechap2>
%\fi
%
% %%%%%%%%%%%%%%%%%%%%%%%%%%%%%%%%%%%%%%
% \paragraph{Part Include Files.}
%
% The include files are called |cdocspt3.tex| and |cdocspt4.tex|.
%
%\iffalse
%<*samplepart3|samplepart4>
%\fi

% Optional override for |\version| flag:
%    \begin{macrocode}
%%\providecommand{\version}{final}
%    \end{macrocode}

% Include the main document:
%    \begin{macrocode}
\input{childdoc.def}
\childdocby{cdocsamp}
%    \end{macrocode}

%\iffalse
%</samplepart3|samplepart4>
%\fi
%
%\iffalse
%<*samplepart3>
%\fi
% Some text for part 3:
%    \begin{macrocode}
some text in part three
%    \end{macrocode}

%\iffalse
%</samplepart3>
%\fi
% Some text for part 4:
%\iffalse
%<*samplepart4>
%\fi
%    \begin{macrocode}
more text in part four
%    \end{macrocode}

%\iffalse
%</samplepart4>
%\fi
%
% %%%%%%%%%%%%%%%%%%%%%%%%%%%%%%%%%%%%%%
% \paragraph{Forwarding for a Complete Draft.}
%
% The following forwarding file |cdocsdrf.tex|
% compiles the main document in draft mode:
%\iffalse
%<*sampledraft>
%\fi
%    \begin{macrocode}
\def\version{draft}
\input{childdoc.def}
\childdocforward{cdocsamp}
%    \end{macrocode}

%\iffalse
%</sampledraft>
%\fi
%
% %%%%%%%%%%%%%%%%%%%%%%%%%%%%%%%%%%%%%%
% \paragraph{Forwarding for Final Version of the Chapters.}
%
% The following forwarding files |cdocsfn1.tex| and |cdocsfn2.tex|
% (with identical content)
% compile the final versions of the child documents
% |cdocsch1.tex| and |cdocsch2.tex|, respectively:
%\iffalse
%<*samplefinal>
%\fi
%    \begin{macrocode}
\def\version{final}
\input{childdoc.def}
\childdocforwardprefix[cdocsamp]{cdocsfn}{cdocsch}
%    \end{macrocode}

%\iffalse
%</samplefinal>
%\fi
%
% %%%%%%%%%%%%%%%%%%%%%%%%%%%%%%%%%%%%%%
% \paragraph{Command Line Processing.}
%
% The following three command lines generate the output files
% |cdocscld|, |cdocscl1| and |cdocscl2|
% which should be identical to
% |cdocsdrf|, |cdocsch1| and |cdocsfn2|, respectively:
% \begin{center}
% \begin{tabular}{l}
% |latex -jobname cdocscld \|\\
% |  "\def\version{draft}\input{childdoc.def}\childdocforward{cdocsamp}"|\\
% |latex -jobname cdocscl1 \|\\
% |  "\input{childdoc.def}\childdocforward[cdocsamp]{cdocsch1}"|\\
% |latex -jobname cdocscl2 \|\\
% |  "\def\version{final}\input{childdoc.def}\childdocforward{cdocsch2}"|
% \end{tabular}
% \end{center}
% Note that the trailing backslash on each first line
% merely continues the input to the second line
% (for convenient cut ant paste).
% Furthermore, the command |latex| can be replaced by any
% of its alternative versions such as |pdflatex|.
%
% %%%%%%%%%%%%%%%%%%%%%%%%%%%%%%%%%%%%%%%%%%%%%%%%%%%%%%%%%%%%%%%%%%%%%%%%%%%%%%
% %%%%%%%%%%%%%%%%%%%%%%%%%%%%%%%%%%%%%%%%%%%%%%%%%%%%%%%%%%%%%%%%%%%%%%%%%%%%%%
% \section{Implementation}
%\iffalse
%<*package>
%\fi
%
% This section describes the definitions file |childdoc.def|.

% The definitions cannot be loaded using |\usepackage| or |\RequirePackage|
% which has a mechanism to prevent loading a style file more than once.
% When loading the definitions by means of |\input|
% multiple instances have to be prevented manually:
%\iffalse
%This code needs to be before the `\ProvidesFile' directive
%which is defined at the beginning of this file.
%Therefore it is also placed there and commented out here.
%</package>
%<*discard>
%\fi
%    \begin{macrocode}
\ifdefined\childdocmain\endinput\fi
%    \end{macrocode}
%\iffalse
%</discard>
%<*package>
%\fi
%
% \macro{\ifchilddoc}
% \macro{\ifchilddocmanual}
% The conditional |\ifchilddoc| tells whether a
% child (true) or main (false) document is being compiled.
% The conditional |\ifchilddocmanual| tells whether
% the |\includeonly| mechanism is used (false) or
% the selection of child files must be performed manually (true).
% The definitions initialise to false:
%    \begin{macrocode}
\newif\ifchilddoc
\newif\ifchilddocmanual
%    \end{macrocode}

% \macro{\childdocname}
% \macro{\childdocjob}
% The macro |\childdocname| stores the name of the main document
% to be compiled. The macro |\childdocjob| stores the name of
% the document on which the \LaTeX{} compiler was originally invoked.
% The content of |\jobname| cannot be compared
% to filenames specified in the source due to different catcodes.
% The following code rescans |\jobname|, stores the result
% in |\childdocname| and saves a copy in |\childdocjob|:
%    \begin{macrocode}
\edef\childdocname{\scantokens\expandafter{\jobname\noexpand}}
\let\childdocjob\childdocname
%    \end{macrocode}

% \macro{\childdocdisable}
% The macro |\childdocdisable| prevents the main file
% from being processed more than once.
% At this stage, the main document command |\childdocmain|
% is assumed to be called once again where it should do nothing.
% Any subsequent call to it should prevent
% a secondary processing of the main document
% It overwrites the forwarding commands
% |\childdocof| and |\childdocforward|
% with empty macros to prevent further inclusions of the main document:
%    \begin{macrocode}
\newcommand{\childdocdisable}
{
  \renewcommand{\childdocmain}[1]{\renewcommand{\childdocmain}[1]{\endinput}}
  \renewcommand{\childdocof}[1]{}
  \renewcommand{\childdocby}[2][]{}
  \renewcommand{\childdocforward}[2][]{}
  \renewcommand{\childdocdisable}{}
}
%    \end{macrocode}

% \macro{\childdocmain}
% The macro |\childdocmain| is to be called at the top of the main file
% with nothing or the main filename (without extension) as argument.
% First, it breaks loops.
% If the argument is not empty and does not match |\childdocname|
% (which is set by the first inclusion of |childdoc.def|),
% |\ifchilddoc| is set to true, |\includeonly| is applied to the child file
% and |\jobname| is set to the main file
% (for proper handling of |.aux| files):
%    \begin{macrocode}
\newcommand{\childdocmain}[1]
{
  \childdocdisable\childdocmain{}
  \if?#1?\else
    \begingroup
      \def\childdoctmp{#1}
      \ifx\childdoctmp\childdocname
        \def\childdoctmp{}
      \else
        \def\childdoctmp
        {
          \childdoctrue
          \includeonly{\childdocname}
          \def\childdocjob{#1}
          \def\jobname{#1}
        }
      \fi
      \expandafter
    \endgroup
    \childdoctmp
  \fi
}
%    \end{macrocode}

% \macro{\childdocof}
% The command |\childdocof| redirects
% compilation to the main file |#1|.
%    \begin{macrocode}
\newcommand{\childdocof}[1]
{
  \childdocdisable
  \childdoctrue
  \includeonly{\childdocname}
  \def\jobname{#1}
  \def\childdocjob{#1}
  \input{#1}
}
%    \end{macrocode}

% \macro{\childdocby}
% The command |\childdocby| ....
%    \begin{macrocode}
\newcommand{\childdocby}[2][]
{
  \childdocdisable
  \childdoctrue
  \childdocmanualtrue
  \if?#1?\else
    \def\jobname{#2}
  \fi
  \def\childdocjob{#2}
  \input{#2}
  \endinput
}
%    \end{macrocode}

% \macro{\childdocforward}
% The command |\childdocforward| redirects
% compilation to the main file or
% (if the optional argument is given) a child file.
% Parameters are set as if the main file
% or a child file starting with |\childdocof| was compiled.
% Then compilation is handed over to the main file:
%    \begin{macrocode}
\newcommand{\childdocforward}[2][]
{
  \begingroup
    \if?#1?
      \def\childdoctmp
      {
        \def\childdocname{#2}
        \def\childdocjob{#2}
        \def\jobname{#2}
        \input{#2}
        \endinput
      }
    \else
      \def\childdoctmp
      {
        \childdocdisable
        \def\childdocname{#2}
        \childdoctrue
        \includeonly{#2}
        \def\childdocjob{#1}
        \def\jobname{#1}
        \input{#1}
        \endinput
      }
    \fi
    \expandafter
  \endgroup
  \childdoctmp
}
%    \end{macrocode}

% \macro{\childdocforwardprefix}
% The command |\childdocforwardprefix| redirects
% compilation to the main or a child file by means of a pattern.
% The prefix |#1| in the current filename is replaced by |#2|
% and the suffix of the current filename is kept
% (it is assumed that the filename does not contain the substring `|~~~|'
% which is used as a delimiter).
% Compilation is handed over to the new file by |\childdocforward|:
%    \begin{macrocode}
\newcommand{\childdocforwardprefix}[3][]
{
  \begingroup
    \def\childdocextract #2##1~~~{\def\childdoctmp{\childdocforward[#1]{#3##1}}}
    \expandafter\childdocextract\childdocname~~~
    \expandafter
  \endgroup
  \childdoctmp
}
%    \end{macrocode}

% \macro{\childdoc}
% The deprecated macro |\childdoc| is a legacy version of |\childdocmain|:
%    \begin{macrocode}
\newcommand{\childdoc}{\childdocmain}
%    \end{macrocode}

% \macro{\childdocredirect}
% The deprecated macro |\childdocredirect| is a legacy version
% of |\childdocforward| and |\childdocforwardprefix|:
%    \begin{macrocode}
\newcommand{\childdocredirect}[2][]
{
  \begingroup
    \if?#1?
      \def\childdoctmp{\childdocforward{#2}}
    \else
      \def\childdoctmp{\childdocforwardprefix{#1}{#2}}
    \fi
    \expandafter
  \endgroup
  \childdoctmp
}
%    \end{macrocode}

%\iffalse
%</package>
%\fi
%
\endinput
\childdocforward{cdocsamp}"|\\
% |latex -jobname cdocscl1 \|\\
% |  "% \iffalse
%
% childdoc.dtx Copyright (C) 2017-2018 Niklas Beisert
%
% This work may be distributed and/or modified under the
% conditions of the LaTeX Project Public License, either version 1.3
% of this license or (at your option) any later version.
% The latest version of this license is in
%   http://www.latex-project.org/lppl.txt
% and version 1.3 or later is part of all distributions of LaTeX
% version 2005/12/01 or later.
%
% This work has the LPPL maintenance status `maintained'.
%
% The Current Maintainer of this work is Niklas Beisert.
%
% This work consists of the files childdoc.dtx and childdoc.ins
% and the derived files childdoc.def and cdocsamp.tex with
% cdocsch1.tex, cdocsch2.tex, cdocsdrf.tex, cdocsfn1.tex, cdocsfn2.tex.
%
%<package>\ifdefined\childdocmain\endinput\fi
%<package>\ProvidesFile{childdoc.def}[2018/12/30 v2.0 child document driver]
%<samplemain>\ProvidesFile{cdocsamp.tex}[2018/12/30 v2.0 sample for childdoc]
%<*driver>
%\ProvidesFile{childdoc.drv}[2018/12/30 v2.0 childdoc reference manual file]
\PassOptionsToClass{10pt,a4paper}{article}
\documentclass{ltxdoc}

\usepackage[margin=35mm]{geometry}
\usepackage{hyperref}
\usepackage{hyperxmp}
\usepackage[usenames]{color}

\hypersetup{colorlinks=true}
\hypersetup{pdfstartview=FitH}
\hypersetup{pdfpagemode=UseNone}
\hypersetup{pdfsource={}}
\hypersetup{pdflang={en-UK}}
\hypersetup{pdfcopyright={Copyright 2017-2018 Niklas Beisert.
  This work may be distributed and/or modified under the
  conditions of the LaTeX Project Public License, either version 1.3
  of this license or (at your option) any later version.}}
\hypersetup{pdflicenseurl={http://www.latex-project.org/lppl.txt}}
\hypersetup{pdfcontactaddress={ETH Zurich, ITP, HIT K,
  Wolfgang-Pauli-Strasse 27}}
\hypersetup{pdfcontactpostcode={8093}}
\hypersetup{pdfcontactcity={Zurich}}
\hypersetup{pdfcontactcountry={Switzerland}}
\hypersetup{pdfcontactemail={nbeisert@itp.phys.ethz.ch}}
\hypersetup{pdfcontacturl={http://people.phys.ethz.ch/\xmptilde nbeisert/}}

\newcommand{\secref}[1]{\hyperref[#1]{section \ref*{#1}}}

\parskip1ex
\parindent0pt
\let\olditemize\itemize
\def\itemize{\olditemize\parskip0pt}

\begin{document}

\title{The \textsf{childdoc} Package}
\hypersetup{pdftitle={The childdoc Package}}
\author{Niklas Beisert\\[2ex]
  Institut f\"ur Theoretische Physik\\
  Eidgen\"ossische Technische Hochschule Z\"urich\\
  Wolfgang-Pauli-Strasse 27, 8093 Z\"urich, Switzerland\\[1ex]
  \href{mailto:nbeisert@itp.phys.ethz.ch}
  {\texttt{nbeisert@itp.phys.ethz.ch}}}
\hypersetup{pdfauthor={Niklas Beisert}}
\hypersetup{pdfsubject={Manual for the LaTeX2e Package childdoc}}
\date{30 December 2018, \textsf{v2.0}}
\maketitle

\begin{abstract}\noindent
\textsf{childdoc} is a \LaTeXe{} package
that enables the direct compilation
of document sections included by |\include|
to individual files.
\end{abstract}

\begingroup
\parskip0ex
\tableofcontents
\endgroup

%%%%%%%%%%%%%%%%%%%%%%%%%%%%%%%%%%%%%%%%%%%%%%%%%%%%%%%%%%%%%%%%%%%%%%%%%%%%%%%%
%%%%%%%%%%%%%%%%%%%%%%%%%%%%%%%%%%%%%%%%%%%%%%%%%%%%%%%%%%%%%%%%%%%%%%%%%%%%%%%%
\section{Introduction}

\LaTeX{} provides a mechanism to structure a large document (such as a book)
into a main file and several child files (containing the chapters)
using the |\include| command.
This mechanism is beneficial for documents
which span hundreds of pages in order to
make the source file(s) more manageable.
Moreover, compilation can be restricted to
selected child files by means of the |\includeonly| command.
The latter feature can be used to reduce the compilation time while editing
(this was significantly more useful in the earlier days of \LaTeX{})
or to generate a smaller document which is easier to navigate.
Another application of |\includeonly| is to generate
documents consisting of selected parts of the complete document.

However, there are a few drawbacks of the plain |\include| mechanism:
\begin{itemize}
\item
The child files cannot be compiled on their own,
they can only be compiled via the main file.
A naive editing environment
(such as a text editor with an option
to have the current file processed by \LaTeX)
may require one to switch to the main file before compiling;
attempting to compile the child file produces errors.
\item
The main file must be modified (each time)
to adjust the |\includeonly| command
to the present needs. This easily leaves the main file in a messy state.
\item
The generated document will always carry the filename
of the main document. This is inconvenient if
several child files are to be compiled and
to be kept for distribution.
\end{itemize}

The present package provides a simple interface
to make child files individually compilable by \LaTeX{}.
Compiling a child file then has the same effect as compiling
the main file with an |\includeonly| command
to select the appropriate child.
Moreover the generated document will carry the name of the child
rather than the main file.
This resolves all three above issues.

This feature is meant to make the editing of books,
thesis documents and lecture notes somewhat more convenient.
However, the package can also be used efficiently for
composing a series of documents (such as exercise sheets)
which are typically distributed individually.
It then assists the author in generating the individual documents
(potentially in different versions)
as well as a document containing the collected series.
Another application is in developing style files
or other kinds of included material
where compilation of the style file could redirect
to a sample or test file.

%%%%%%%%%%%%%%%%%%%%%%%%%%%%%%%%%%%%%%%%%%%%%%%%%%%%%%%%%%%%%%%%%%%%%%%%%%%%%%%%
%%%%%%%%%%%%%%%%%%%%%%%%%%%%%%%%%%%%%%%%%%%%%%%%%%%%%%%%%%%%%%%%%%%%%%%%%%%%%%%%
\section{Usage}

First of all, the package \textsf{childdoc} is \emph{not} a standard
\LaTeXe{} |.sty| style file! Therefore it needs to be invoked in
a non-standard way.

%%%%%%%%%%%%%%%%%%%%%%%%%%%%%%%%%%%%%%%%%%%%%%%%%%%%%%%%%%%%%%%%%%%%%%%%%%%%%%%%
\subsection{Included Files}
\label{sec:include}

%%%%%%%%%%%%%%%%%%%%%%%%%%%%%%%%%%%%%%%%
\DescribeMacro{\childdocmain}
To use the package, add the commands
\begin{center}
\begin{tabular}{l}
|\input{childdoc.def}|\\
|\childdocmain{}|\\
\end{tabular}
\end{center}
at the very top of the main \LaTeX{} file,
in particular \emph{before} the |\documentclass| statement!
The argument of |\childdocmain| should be left empty
(but it must be present).

%%%%%%%%%%%%%%%%%%%%%%%%%%%%%%%%%%%%%%%%
\DescribeMacro{\childdocof}
Furthermore, add the commands
\begin{center}
\begin{tabular}{l}
|\input{childdoc.def}|\\
|\childdocof{|\textit{main}|}|\\
\end{tabular}
\end{center}
at the top of every child file \textit{child}
which is included by |\include{|\textit{child}|}|
from within the main file
(or at least for those files to be compiled individually).
The argument \textit{main} must be the filename of the main file.

There are a couple of
considerations in setting up the main and child documents:

%%%%%%%%%%%%%%%%%%%%%%%%%%%%%%%%%%%%%%%%
\paragraph{Restrictions.}

Please note the following restrictions:
\begin{itemize}
\item
|\childdocmain| must be called with one argument \textit{main}
to ensure compatibility with earlier version of the package.
It must either be empty (|\childdocmain{}|)
or precisely match the filename of the main file in which it is specified.
See \secref{sec:detection} for further information.
\item
The filename \textit{main} must be specified without the |.tex| extension.
\item
The filename \textit{main} is case sensitive
(even in case-insensitive file systems)
due to internal string comparison.
\item
The argument \textit{main} should be fully expanded, it cannot be a macro.
\item
Subdirectories and special characters should be avoided in filenames.
\item
The command |\childdocmain{|\textit{main}|}| must be followed by a whitespace.
It should not be followed immediately by another command
or by a comment mark `|%|'.
This is because the \TeX{} parser reads the token immediately following
the argument of |\childdocmain| and puts it
at the beginning of every child section;
however, a white\-space is ignored.
\end{itemize}

%%%%%%%%%%%%%%%%%%%%%%%%%%%%%%%%%%%%%%%%
\paragraph{Content of Main File.}

It is advisable to place all content in the child files included by |\include|.
Any output contained in the main file will appear in all child documents
unless suppressed manually;
it cannot be suppressed automatically by the |\includeonly| directive
and thus should normally be avoided.
A method to include some content in the main file
by means of conditional processing is described in \secref{sec:conditional}.

%%%%%%%%%%%%%%%%%%%%%%%%%%%%%%%%%%%%%%%%
\paragraph{Page Numbering.}

When only a part of the document is compiled,
the appropriate numbering of pages
(as well as other status parameters)
is determined from the |.aux| files.
The latter contain information from previous passes.
However this information needs to propagate through
all intermediate child documents.
Therefore the page numbering in child documents may well
be inconsistent until the complete document is compiled at least once.

A useful (if unconventional) way to always ensure a consistent
page numbering is to restart the numbering in each child document
and denote the pages by `\textit{child}|.|\textit{page}'
where \textit{child} represents the chapter/section number of the child file.
This can be achieved by the command
|\numberwithin{page}{|\textit{child}|}|
of the \textsf{amsmath} package
where \textit{child} can be |chapter| or |section|
depending on the chosen structuring.
Alternatively, one can modify the macro |\thepage| appropriately
and reset the counter |page| at the start of each child file.

%%%%%%%%%%%%%%%%%%%%%%%%%%%%%%%%%%%%%%%%%%%%%%%%%%%%%%%%%%%%%%%%%%%%%%%%%%%%%%%%
\subsection{Conditional Processing}
\label{sec:conditional}

The package provides a mechanism to compile different versions
of a document. To customise the versions further some conditional processing
can come in handy to distinguish which version is being compiled.
The package provides two macros to describe the compilation context:

%%%%%%%%%%%%%%%%%%%%%%%%%%%%%%%%%%%%%%%%
\DescribeMacro{\ifchilddoc}
The conditional |\ifchilddoc| distinguishes between the compilation of
child documents and the main document:
%
\begin{center}
|\ifchilddoc |\textit{child-code}| |[|\||else |\textit{main-code}]| \||fi|
\end{center}

%%%%%%%%%%%%%%%%%%%%%%%%%%%%%%%%%%%%%%%%
\DescribeMacro{\childdocname}
\DescribeMacro{\childdocjob}
The macro |\childdocname| contains the filename (without extension)
of the main or child file being processed.
Note that |\childdocjob| will always contain the name of the main file.

%%%%%%%%%%%%%%%%%%%%%%%%%%%%%%%%%%%%%%%%
\paragraph{Title Page.}

Conditional processing can be used to include a title or banner page
in the main document when proper precautions are taken.
Importantly, the code in the main file should ensure that the page counter
(as well as other status parameters which are stored in the |.aux| files)
takes the same value after the conditional processing.
Otherwise the page numbers may take divergent values
depending on which part is compiled.

For example, a title page could be declared by:
%
\begin{center}
\begin{tabular}{l}
|\ifchilddoc\||else|\\
|\addtocounter{page}{-1}|\\
\textit{code for title page}\\
|\newpage|\\
|\||fi|
\end{tabular}
\end{center}
%
A banner page for the child documents can be generated by:
%
\begin{center}
\begin{tabular}{l}
|\ifchilddoc|\\
|\addtocounter{page}{-1}|\\
\textit{code for banner page}\\
|\newpage|\\
|\||fi|
\end{tabular}
\end{center}
%
Here one could write a message such as:
\begin{center}
|This is the part \childdocname{} of \childdocjob{}.|
\end{center}

%%%%%%%%%%%%%%%%%%%%%%%%%%%%%%%%%%%%%%%%%%%%%%%%%%%%%%%%%%%%%%%%%%%%%%%%%%%%%%%%
\subsection{Flags}
\label{sec:flags}

The package makes it easy to generate different versions
of the main or child documents.
To this end compilation flags can be defined
and assigned different default values.
They will be particularly useful in conjunction
with the forwarding mechanism described in \secref{sec:forward}.

For example, it may be useful to have a flag |\version|
which can be set to |draft| or |final|.
The document source will contain some conditional code
depending on the value of |\version|.
Suppose further, the flag should default to |final| for the main file
and to |draft| for child files
which is a natural assignment for editing the document.
This is achieved by placing the following code
in the preamble of the main document
(below the |\childdocmain| directive):
%
\begin{center}
\begin{tabular}{l}
|\ifchilddoc|\\
|\providecommand{\version}{draft}|\\
|\||else|\\
|\providecommand{\version}{final}|\\
|\||fi|
\end{tabular}
\end{center}
%
The definition by |\providecommand| makes sure
that previous definitions are not overwritten.
Further statements |\providecommand{\version}{...}|
can thus be added before the above code to override it.

For the main file, one might add a line
(between |\childdocmain| and the above block)
%
\begin{center}
|%\ifchilddoc\||else\providecommand{\version}{draft}\||fi|
\end{center}
%
which can be uncommented to produce a draft version.
Likewise one can add a line to the very top of a child file
(above the |\childdocof{|\textit{main}|}| directive)
%
\begin{center}
|%\providecommand{\version}{final}|
\end{center}
%
which can be uncommented to produce the final version of this child document.

%%%%%%%%%%%%%%%%%%%%%%%%%%%%%%%%%%%%%%%%%%%%%%%%%%%%%%%%%%%%%%%%%%%%%%%%%%%%%%%%
\subsection{Forwarding}
\label{sec:forward}

Different versions of the main or child documents
using compilation flags as described in \secref{sec:flags}
can be (permanently) stored in different files
for convenient compilation, viewing and distribution.
To this end, the package defines a command
to pass on compilation to a different file:

%%%%%%%%%%%%%%%%%%%%%%%%%%%%%%%%%%%%%%%%
\DescribeMacro{\childdocforward}
The command |\childdocforward| redirects processing to
another source file:
%
\begin{center}
\begin{tabular}{l}
|\input{childdoc.def}|\\
|\childdocforward[|\textit{main}|]{|\textit{dest}|}|\\
\end{tabular}
\end{center}
%
The argument \textit{dest} is the destination file
(without extension).
It should be the main file or one of the child files.
Note that further \textsf{childdoc} directives
such as |\childdocof| and |\childdocforward|
in the indicated file will be processed in this form.
The optional argument \textit{main}
passes on directly to the main file \textit{main}
while pretending to compile the child \textit{dest}.
This form behaves as if \textit{dest}
issues |\childdocof{|\textit{main}|}| right away,
and no further \textsf{childdoc} directives will be processed.

%%%%%%%%%%%%%%%%%%%%%%%%%%%%%%%%%%%%%%%%
\DescribeMacro{\...prefix}
In the alternative form |\childdocforwardprefix|,
%
\begin{center}
\begin{tabular}{l}
|\input{childdoc.def}|\\
|\childdocforwardprefix[|\textit{main}|]{|\textit{prefix}|}{|\textit{dest}|}|
\end{tabular}
\end{center}
%
the destination file is determined by a pattern
depending on the current file:
To make this work, the current file must be called
`{\textit{prefix}\hspace{0.2em}\textit{suffix}}'
with \textit{prefix} matching precisely the argument.
Processing is then passed on to the file
`{\textit{dest}\hspace{0.2em}\textit{suffix}}'.
Surely, the same effect is achieved by
directly specifying the
argument `{\textit{dest}\hspace{0.2em}\textit{suffix}}'
in the first form.
However, that requires to set up a different file
for each child. With the alternative form of the command
all these files can have exactly the same content
which simplifies setting them up and maintaining them.

For example, the following file |draft.tex|
with a compilation flag |\version| as described in \secref{sec:flags}
compiles the main document as a draft:
%
\begin{center}
\begin{tabular}{l}
|\def\version{draft}|\\
|\input{childdoc.def}|\\
|\childdocforward{|\textit{main}|}|
\end{tabular}
\end{center}
%
Likewise, the following files |final|\textit{nn}|.tex|
compile the final version of the child document
|child|\textit{nn}|.tex|:
%
\begin{center}
\begin{tabular}{l}
|\def\version{final}|\\
|\input{childdoc.def}|\\
|\childdocforwardprefix{final}{child}|
\end{tabular}
\end{center}
%

Note that when several versions of a main file and/or of each child file
are to be generated, it may be convenient to set up a |Makefile| or
shell script to automatise the process.

%%%%%%%%%%%%%%%%%%%%%%%%%%%%%%%%%%%%%%%%%%%%%%%%%%%%%%%%%%%%%%%%%%%%%%%%%%%%%%%%
\subsection{Command Line Processing}
\label{sec:commandline}

The effect of redirection files can also be achieved by invoking
the \LaTeX{} compiler with a more elaborate command line.
Most conveniently this should be done as part
of a shell script or a |Makefile|.

When using \textsf{childdoc} in the main file, the following
command lines effectively perform a redirection
(note that depending on the shell being used,
backslashes may have to be doubled: `|\|' $\to$ `|\\|'):
%
\begin{center}
|... -jobname "|\textit{target}|" |\\|"|[\textit{flags}]%
|\input{childdoc.def}\childdocforward[|\textit{main}|]{|\textit{dest}|}"|
\end{center}
%
Here \textit{target} is the name of the output file,
\textit{main} is the name of the main file
and \textit{dest} is the name of the main or child file to be processed
(all filenames without extensions).
The optional argument \textit{main} can be omitted
if \textit{main} matches \textit{dest}.
Optionally, compilation \textit{flags} can be defined via |\def| commands.
This command line makes the \TeX{} engine believe
it is compiling the file \textit{target}
whose content is specified as the latter parameter.
The provided code then forwards the processing to
\textit{main} or \textit{dest} as described in \secref{sec:forward}.

%%%%%%%%%%%%%%%%%%%%%%%%%%%%%%%%%%%%%%%%%%%%%%%%%%%%%%%%%%%%%%%%%%%%%%%%%%%%%%%%
\subsection{Include by Input}
\label{sec:input}

Including child documents by |\include| has some restrictions by design.
Most notably, the content of a child document always occupies
its own set of pages; pages cannot be shared between child documents.
Usually, this behaviour makes perfect sense
because each child document contain an essential part of the document.
However, in some situations it may be desirable to compose
a document from a collection of parts
without having mandatory page breaks between then.
For this case, the package
provides a mechanism to include parts
by |\input| which can also be processed individually.
However, by construction this mechanism
requires manual handling of the content to be output.

%%%%%%%%%%%%%%%%%%%%%%%%%%%%%%%%%%%%%%%%
\DescribeMacro{\ifchilddocmanual}
The main file should be prepared as usual, see \secref{sec:include}.
However, the document body must make a distinction
between processing of an individual part and of the main document, e.g.:
%
\begin{center}
\begin{tabular}{l}
|\ifchilddocmanual|\\
|\input{\childdocname}|\\
|\||else|\\
\textit{document body with }|\input{|\textit{part}|}|\\
|\||fi|
\end{tabular}
\end{center}
%
The conditional |\ifchilddocmanual| is true whenever
a part to be included by |\input| is being compiled,
and the name of the part is stored in |\childdocname|.

%%%%%%%%%%%%%%%%%%%%%%%%%%%%%%%%%%%%%%%%
\DescribeMacro{\childdocby}
Each part to be included by |\input| should start with:
%
\begin{center}
\begin{tabular}{l}
|\input{childdoc.def}|\\
|\childdocby{|\textit{main}|}|\\
\end{tabular}
\end{center}
%
The directive |\childdocby| is similar to |\childdocof|
described in \secref{sec:include},
but the subsequent selection of content must be done manually.
To that end, both |\ifchilddoc| and |\ifchilddocmanual|
will be true upon processing of a part,
and the name of the part is stored in |\childdocname|.
Note that |\jobname| will be set to the filename of the current part
so that each part receives an individual |.aux| file
that does not interfere with the |.aux| file(s) of the main document.
This behaviour can be altered by the alternative form
|\childdocby[*]{|\textit{main}|}| (with a non-empty optional argument)
which uses the |.aux| file of the main document
by setting |\jobname| to \textit{main}.

%%%%%%%%%%%%%%%%%%%%%%%%%%%%%%%%%%%%%%%%%%%%%%%%%%%%%%%%%%%%%%%%%%%%%%%%%%%%%%%%
\subsection{Driver Development}
\label{sec:driver}

The \textsf{childdoc} mechanism can also be use for the development
of definition files such as \LaTeX{} styles or classes.
This case differs from the above setup with multiple parts
included by |\include| in that no |\includeonly| should be invoked.
This can be achieved by starting the include file
(before |\ProvidesPackage|) with:
%
\begin{center}
\begin{tabular}{l}
|\input{childdoc.def}|\\
|\childdocforward{|\textit{main}|}|\\
\end{tabular}
\end{center}
%
or alternatively with:
%
\begin{center}
\begin{tabular}{l}
|\input{childdoc.def}|\\
|\childdocby{|\textit{main}|}|\\
\end{tabular}
\end{center}
%
Both forms have slightly different effects as described above.
The main file is prepared as usual, see \secref{sec:include}.

%%%%%%%%%%%%%%%%%%%%%%%%%%%%%%%%%%%%%%%%%%%%%%%%%%%%%%%%%%%%%%%%%%%%%%%%%%%%%%%%
\subsection{Legacy Detection}
\label{sec:detection}

The directive |\childdocmain| in the main file can detect
whether the complete document or merely a child is to be compiled
even without using the directive |\childdocof|.
This method is deprecated because it is less robust
and there is no compelling reason to use it;
it is merely provided for backward compatibility
and it may be removed in future versions.

If the detection mechanism is to be used,
it is mandatory to correctly specify
the filename of the main file as the argument of |\childdocmain|:
%
\begin{center}
\begin{tabular}{l}
|\input{childdoc.def}|\\
|\childdocmain{|\textit{main}|}|\\
\end{tabular}
\end{center}
%
If |\jobname| does not match the argument \textit{main} of |\childdocmain|,
it is assumed that |\jobname| points to the child file to be compiled.
When using |\childdocmain| with the main file specified as argument,
it suffices to start a child file
with just |\input{|\textit{main}|}|
without loading of the package and using |\childdocof|.
If instead all processing is done
with the appropriate \textsf{childdoc} directives,
the argument of \textit{main} of |\childdocmain| can be empty.

An alternative version of the command line processing described
in \secref{sec:commandline} using the detection mechanism reads:
%
\begin{center}
|... -jobname "|\textit{target}|" "|[\textit{flags}]%
[|\def\jobname{|\textit{dest}|}|]|\input{|\textit{main}|}"|
\end{center}

%%%%%%%%%%%%%%%%%%%%%%%%%%%%%%%%%%%%%%%%%%%%%%%%%%%%%%%%%%%%%%%%%%%%%%%%%%%%%%%%
\subsection{Manual Code}
\label{sec:manual}

In case one cannot be certain whether the definitions file |childdoc.def|
is installed on the target \TeX{} distribution
and one prefers not to ship it,
it is conceivable to paste a few relevant commands into the sources.

To that end, drop all statements |\input{childdoc.def}|
and perform the replacements as outlined below.
Instead of |\childdocmain{|\textit{main}|}| add the following code
to the top of the main file:
%
\begin{center}
\begin{tabular}{l}
|\||ifdefined\childdocname\endinput\||fi\newif\ifchilddoc|\\
|\edef\childdocname{\scantokens\expandafter{\jobname\noexpand}}|\\
|\def\childdocmain{|\textit{main}|}\||ifx\childdocmain\childdocname\||else|\\
|\childdoctrue\includeonly{\childdocname}\let\jobname\childdocmain\||fi|\\
\end{tabular}
\end{center}
%
Instead of |\childdocof{|\textit{main}|}| just include the main file
at the top of each child file:
%
\begin{center}
|\input{|\textit{main}|}|
\end{center}
%
A simple redirection |\childdocforward{|\textit{dest}|}| is achieved by:
%
\begin{center}
|\def\jobname{|\textit{dest}|}\input{\jobname}|
\end{center}
%
The redirection with prefix
|\childdocforwardprefix[|\textit{prefix}|]{|\textit{dest}|}|
is accomplished by:
%
\begin{center}
\begin{tabular}{l}
|{\edef\jobname{\scantokens\expandafter{\jobname\noexpand}}|\\
|\def\redirectjob |\textit{prefix}|#1~~~{\gdef\jobname{|\textit{dest}|#1}}|\\
|\expandafter\redirectjob\jobname~~~}\input{\jobname}|
\end{tabular}
\end{center}

In an alternative approach,
child documents can be compiled by a specific command line
without additional code or specific definitions:
%
\begin{center}
|... -jobname "|\textit{target}|" "|[\textit{flags}]%
|\includeonly{|\textit{dest}|}\input{|\textit{main}|}"|
\end{center}
%

%%%%%%%%%%%%%%%%%%%%%%%%%%%%%%%%%%%%%%%%%%%%%%%%%%%%%%%%%%%%%%%%%%%%%%%%%%%%%%%%
%%%%%%%%%%%%%%%%%%%%%%%%%%%%%%%%%%%%%%%%%%%%%%%%%%%%%%%%%%%%%%%%%%%%%%%%%%%%%%%%
\section{Information}

%%%%%%%%%%%%%%%%%%%%%%%%%%%%%%%%%%%%%%%%%%%%%%%%%%%%%%%%%%%%%%%%%%%%%%%%%%%%%%%%
\subsection{Copyright}

Copyright \copyright{} 2017--2018 Niklas Beisert

This work may be distributed and/or modified under the
conditions of the \LaTeX{} Project Public License, either version 1.3
of this license or (at your option) any later version.
The latest version of this license is in
  \url{http://www.latex-project.org/lppl.txt}
and version 1.3 or later is part of all distributions of \LaTeX{}
version 2005/12/01 or later.

This work has the LPPL maintenance status `maintained'.

The Current Maintainer of this work is Niklas Beisert.

This work consists of the files |README.txt|, |childdoc.ins| and |childdoc.dtx|
as well as the derived files |childdoc.def|, |cdocsamp.tex|
with |cdocsch1.tex|, |cdocsch2.tex|, |cdocspt3.tex|, |cdocspt4.tex|,
|cdocsdrf.tex|, |cdocsfn1.tex|, |cdocsfn2.tex|
as well as |childdoc.pdf|.

%%%%%%%%%%%%%%%%%%%%%%%%%%%%%%%%%%%%%%%%%%%%%%%%%%%%%%%%%%%%%%%%%%%%%%%%%%%%%%%%
\subsection{Files and Installation}

The package consists of the files:
%
\begin{center}
\begin{tabular}{ll}
    |README.txt|   & readme file \\
    |childdoc.ins| & installation file \\
    |childdoc.dtx| & source file \\
    |childdoc.def| & definition file \\
    |cdocsamp.tex| & sample main file \\
    |cdocsch1.tex| & sample include file \\
    |cdocsch2.tex| & sample include file \\
    |cdocspt3.tex| & sample part file \\
    |cdocspt4.tex| & sample part file \\
    |cdocsdrf.tex| & sample redirection file \\
    |cdocsfn1.tex| & sample redirection file \\
    |cdocsfn2.tex| & sample redirection file \\
    |childdoc.pdf| & manual
\end{tabular}
\end{center}
%
The distribution consists of the files
|README.txt|, |childdoc.ins| and |childdoc.dtx|.
%
\begin{itemize}
\item
Run (pdf)\LaTeX{} on |childdoc.dtx|
to compile the manual |childdoc.pdf| (this file).
\item
Run \LaTeX{} on |childdoc.ins| to create the definitions file |childdoc.def|
and the sample |cdocsamp.tex| with include files
|cdocsch1.tex|, |cdocsch2.tex|, |cdocspt3.tex|, |cdocspt4.tex|,
|cdocsdrf.tex|, |cdocsfn1.tex|, |cdocsfn2.tex|.
Then copy the file |childdoc.def| to an appropriate directory of your \LaTeX{}
distribution, e.g.\ \textit{texmf-root}|/tex/latex/childdoc|.
\end{itemize}

%%%%%%%%%%%%%%%%%%%%%%%%%%%%%%%%%%%%%%%%%%%%%%%%%%%%%%%%%%%%%%%%%%%%%%%%%%%%%%%%
\subsection{Related CTAN Packages}

There are several other packages which offer a similar functionality:
%
\begin{itemize}
\item
The packages
\href{http://ctan.org/pkg/docmute}{\textsf{docmute}},
\href{http://ctan.org/pkg/includex}{\textsf{includex}} and
\href{http://ctan.org/pkg/standalone}{\textsf{standalone}}
provide commands to include only the document body of
a child file thus allowing both files to be compiled individually.
\item
The packages \href{http://ctan.org/pkg/subdocs}{\textsf{subdocs}}
and \href{http://ctan.org/pkg/subfiles}{\textsf{subfiles}}
provide structures in which the main and child documents can be
encapsulated and allowing them to be compiled individually.
The inclusion mechanism is different from the conventional |\include|.
\item
The package \href{http://ctan.org/pkg/combine}{\textsf{combine}}
is an elaborate solution to combine several documents into one.
\end{itemize}
%
See also the CTAN topic \href{http://ctan.org/topic/subdocs}{\textsf{subdocs}}
for further related packages.
The present package differs from the above solutions in that
a document structure constructed with the conventional |\include| mechanism
just needs two extra commands at the top of every file
such that all constituent files can be compiled individually.

%%%%%%%%%%%%%%%%%%%%%%%%%%%%%%%%%%%%%%%%%%%%%%%%%%%%%%%%%%%%%%%%%%%%%%%%%%%%%%%%
%\subsection{Feature Suggestions}
%
%The following is a list of features which may be useful for future
%versions of this package:
%%
%\begin{itemize}
%\item
%\ldots
%\end{itemize}

%%%%%%%%%%%%%%%%%%%%%%%%%%%%%%%%%%%%%%%%%%%%%%%%%%%%%%%%%%%%%%%%%%%%%%%%%%%%%%%%
\subsection{Revision History}

%%%%%%%%%%%%%%%%%%%%%%%%%%%%%%%%%%%%%%%%
\paragraph{v2.0:} 2018/12/30

\begin{itemize}
\item
immediate forward processing
\item
added |\childdocby| mechanism
\item
manual restructured
\end{itemize}

%%%%%%%%%%%%%%%%%%%%%%%%%%%%%%%%%%%%%%%%
\paragraph{v1.6:} 2018/01/17

\begin{itemize}
\item
application for development of include files
\item
corrections to manual
\end{itemize}

%%%%%%%%%%%%%%%%%%%%%%%%%%%%%%%%%%%%%%%%
\paragraph{v1.5:} 2017/05/21

\begin{itemize}
\item
more complete structuring introduced
\item
|\childdocof| introduced
\item
|\childdoc| renamed to |\childdocmain|
\item
|\childredirect| renamed to |\childdocforward| and |\childdocforwardprefix|
and functionality expanded
\end{itemize}

%%%%%%%%%%%%%%%%%%%%%%%%%%%%%%%%%%%%%%%%
\paragraph{v1.0:} 2017/04/27

\begin{itemize}
\item
manual and install package
\item
first version published on CTAN
\end{itemize}

%%%%%%%%%%%%%%%%%%%%%%%%%%%%%%%%%%%%%%%%
\paragraph{v0.6:} 2017/04/26

\begin{itemize}
\item
redirection mechanism added
\end{itemize}

%%%%%%%%%%%%%%%%%%%%%%%%%%%%%%%%%%%%%%%%
\paragraph{v0.5:} 2017/04/26

\begin{itemize}
\item
functionality in definition file
\end{itemize}


%%%%%%%%%%%%%%%%%%%%%%%%%%%%%%%%%%%%%%%%%%%%%%%%%%%%%%%%%%%%%%%%%%%%%%%%%%%%%%%%
%%%%%%%%%%%%%%%%%%%%%%%%%%%%%%%%%%%%%%%%%%%%%%%%%%%%%%%%%%%%%%%%%%%%%%%%%%%%%%%%
%%%%%%%%%%%%%%%%%%%%%%%%%%%%%%%%%%%%%%%%%%%%%%%%%%%%%%%%%%%%%%%%%%%%%%%%%%%%%%%%
\appendix

\settowidth\MacroIndent{\rmfamily\scriptsize 000\ }

 \DocInput{childdoc.dtx}

\end{document}
%</driver>
% \fi
%
% %%%%%%%%%%%%%%%%%%%%%%%%%%%%%%%%%%%%%%%%%%%%%%%%%%%%%%%%%%%%%%%%%%%%%%%%%%%%%%
% %%%%%%%%%%%%%%%%%%%%%%%%%%%%%%%%%%%%%%%%%%%%%%%%%%%%%%%%%%%%%%%%%%%%%%%%%%%%%%
% \section{Sample}
%\iffalse
%<*samplemain>
%\fi
%
% The following presents a sample document
% with two chapters, two parts, a title page,
% a compile flag as well as three forwarding files to set the flag.
% It consists of eight |.tex| files:
% \begin{center}
% \begin{tabular}{ll}
% |cdocsamp.tex|&main file\\
% |cdocsch1.tex|&include file for chapter 1\\
% |cdocsch2.tex|&include file for chapter 2\\
% |cdocspt3.tex|&include file for part 3\\
% |cdocspt4.tex|&include file for part 4\\
% |cdocsdrf.tex|&forwarding file for main file in draft mode\\
% |cdocsfi1.tex|&forwarding file for final version of chapter 1\\
% |cdocsfi2.tex|&forwarding file for final version of chapter 2\\
% \end{tabular}
% \end{center}
% Each of the eight files can be compiled directly by the \LaTeX{} compiler.
%
% %%%%%%%%%%%%%%%%%%%%%%%%%%%%%%%%%%%%%%
% \paragraph{Main File.}
%
% The main file is called |cdocsamp.tex|.
%
% Load the \textsf{childdoc} definitions and
% declare the filename for the main document:
%    \begin{macrocode}
\input{childdoc.def}
\childdocmain{}
%    \end{macrocode}

% Optional override for |\version| flag:
%    \begin{macrocode}
%%\ifchilddoc\else\providecommand{\version}{draft}\fi
%    \end{macrocode}

% Define the default values for the |\version| flag
% (|final| for the main file and |draft| for childs):
%    \begin{macrocode}
\ifchilddoc
\providecommand{\version}{draft}
\else
\providecommand{\version}{final}
\fi
%    \end{macrocode}

% Load the standard document class:
%    \begin{macrocode}
\documentclass[12pt]{article}
%    \end{macrocode}

% Start the document body:
%    \begin{macrocode}
\begin{document}
%    \end{macrocode}

% Declare a title page.
% Print title, part of document being processed and version flag:
%    \begin{macrocode}
\addtocounter{page}{-1}
\begin{center}
{\LARGE\bfseries{}childdoc example\par}
\vspace{1cm}
\ifchilddoc
\ifchilddocmanual part\else chapter\fi:
`\childdocname' of `\childdocjob'\par
\else
main document: `\childdocjob'\par
\fi
version: \version\par
\end{center}
\newpage
%    \end{macrocode}

% Manually include selected file,
% otherwise process as usual:
%    \begin{macrocode}
\ifchilddocmanual
\section*{part `\childdocname'}
\input{\childdocname}
\else
%    \end{macrocode}

% Include the two chapters:
%    \begin{macrocode}
\include{cdocsch1}
\include{cdocsch2}
%    \end{macrocode}

% Include the two parts unless only chapters should be displayed:
%    \begin{macrocode}
\ifchilddoc\else
\section{part three}
\input{cdocspt3}
\section{part four}
\input{cdocspt4}
\fi
%    \end{macrocode}

% Process as usual until here:
%    \begin{macrocode}
\fi
%    \end{macrocode}

% End of document body:
%    \begin{macrocode}
\end{document}
%    \end{macrocode}
%\iffalse
%</samplemain>
%\fi
%
% %%%%%%%%%%%%%%%%%%%%%%%%%%%%%%%%%%%%%%
% \paragraph{Chapter Include Files.}
%
% The include files are called |cdocsch1.tex| and |cdocsch2.tex|.
%
%\iffalse
%<*samplechap1|samplechap2>
%\fi

% Optional override for |\version| flag:
%    \begin{macrocode}
%%\providecommand{\version}{final}
%    \end{macrocode}

% Include the main document:
%    \begin{macrocode}
\input{childdoc.def}
\childdocof{cdocsamp}
%    \end{macrocode}

%\iffalse
%</samplechap1|samplechap2>
%\fi
%
%\iffalse
%<*samplechap1>
%\fi
% Some text for chapter 1:
%    \begin{macrocode}
\section{one}
some text in chapter one
%    \end{macrocode}

%\iffalse
%</samplechap1>
%\fi
% Some text for chapter 2:
%\iffalse
%<*samplechap2>
%\fi
%    \begin{macrocode}
\section{two}
more text in chapter two
%    \end{macrocode}

%\iffalse
%</samplechap2>
%\fi
%
% %%%%%%%%%%%%%%%%%%%%%%%%%%%%%%%%%%%%%%
% \paragraph{Part Include Files.}
%
% The include files are called |cdocspt3.tex| and |cdocspt4.tex|.
%
%\iffalse
%<*samplepart3|samplepart4>
%\fi

% Optional override for |\version| flag:
%    \begin{macrocode}
%%\providecommand{\version}{final}
%    \end{macrocode}

% Include the main document:
%    \begin{macrocode}
\input{childdoc.def}
\childdocby{cdocsamp}
%    \end{macrocode}

%\iffalse
%</samplepart3|samplepart4>
%\fi
%
%\iffalse
%<*samplepart3>
%\fi
% Some text for part 3:
%    \begin{macrocode}
some text in part three
%    \end{macrocode}

%\iffalse
%</samplepart3>
%\fi
% Some text for part 4:
%\iffalse
%<*samplepart4>
%\fi
%    \begin{macrocode}
more text in part four
%    \end{macrocode}

%\iffalse
%</samplepart4>
%\fi
%
% %%%%%%%%%%%%%%%%%%%%%%%%%%%%%%%%%%%%%%
% \paragraph{Forwarding for a Complete Draft.}
%
% The following forwarding file |cdocsdrf.tex|
% compiles the main document in draft mode:
%\iffalse
%<*sampledraft>
%\fi
%    \begin{macrocode}
\def\version{draft}
\input{childdoc.def}
\childdocforward{cdocsamp}
%    \end{macrocode}

%\iffalse
%</sampledraft>
%\fi
%
% %%%%%%%%%%%%%%%%%%%%%%%%%%%%%%%%%%%%%%
% \paragraph{Forwarding for Final Version of the Chapters.}
%
% The following forwarding files |cdocsfn1.tex| and |cdocsfn2.tex|
% (with identical content)
% compile the final versions of the child documents
% |cdocsch1.tex| and |cdocsch2.tex|, respectively:
%\iffalse
%<*samplefinal>
%\fi
%    \begin{macrocode}
\def\version{final}
\input{childdoc.def}
\childdocforwardprefix[cdocsamp]{cdocsfn}{cdocsch}
%    \end{macrocode}

%\iffalse
%</samplefinal>
%\fi
%
% %%%%%%%%%%%%%%%%%%%%%%%%%%%%%%%%%%%%%%
% \paragraph{Command Line Processing.}
%
% The following three command lines generate the output files
% |cdocscld|, |cdocscl1| and |cdocscl2|
% which should be identical to
% |cdocsdrf|, |cdocsch1| and |cdocsfn2|, respectively:
% \begin{center}
% \begin{tabular}{l}
% |latex -jobname cdocscld \|\\
% |  "\def\version{draft}\input{childdoc.def}\childdocforward{cdocsamp}"|\\
% |latex -jobname cdocscl1 \|\\
% |  "\input{childdoc.def}\childdocforward[cdocsamp]{cdocsch1}"|\\
% |latex -jobname cdocscl2 \|\\
% |  "\def\version{final}\input{childdoc.def}\childdocforward{cdocsch2}"|
% \end{tabular}
% \end{center}
% Note that the trailing backslash on each first line
% merely continues the input to the second line
% (for convenient cut ant paste).
% Furthermore, the command |latex| can be replaced by any
% of its alternative versions such as |pdflatex|.
%
% %%%%%%%%%%%%%%%%%%%%%%%%%%%%%%%%%%%%%%%%%%%%%%%%%%%%%%%%%%%%%%%%%%%%%%%%%%%%%%
% %%%%%%%%%%%%%%%%%%%%%%%%%%%%%%%%%%%%%%%%%%%%%%%%%%%%%%%%%%%%%%%%%%%%%%%%%%%%%%
% \section{Implementation}
%\iffalse
%<*package>
%\fi
%
% This section describes the definitions file |childdoc.def|.

% The definitions cannot be loaded using |\usepackage| or |\RequirePackage|
% which has a mechanism to prevent loading a style file more than once.
% When loading the definitions by means of |\input|
% multiple instances have to be prevented manually:
%\iffalse
%This code needs to be before the `\ProvidesFile' directive
%which is defined at the beginning of this file.
%Therefore it is also placed there and commented out here.
%</package>
%<*discard>
%\fi
%    \begin{macrocode}
\ifdefined\childdocmain\endinput\fi
%    \end{macrocode}
%\iffalse
%</discard>
%<*package>
%\fi
%
% \macro{\ifchilddoc}
% \macro{\ifchilddocmanual}
% The conditional |\ifchilddoc| tells whether a
% child (true) or main (false) document is being compiled.
% The conditional |\ifchilddocmanual| tells whether
% the |\includeonly| mechanism is used (false) or
% the selection of child files must be performed manually (true).
% The definitions initialise to false:
%    \begin{macrocode}
\newif\ifchilddoc
\newif\ifchilddocmanual
%    \end{macrocode}

% \macro{\childdocname}
% \macro{\childdocjob}
% The macro |\childdocname| stores the name of the main document
% to be compiled. The macro |\childdocjob| stores the name of
% the document on which the \LaTeX{} compiler was originally invoked.
% The content of |\jobname| cannot be compared
% to filenames specified in the source due to different catcodes.
% The following code rescans |\jobname|, stores the result
% in |\childdocname| and saves a copy in |\childdocjob|:
%    \begin{macrocode}
\edef\childdocname{\scantokens\expandafter{\jobname\noexpand}}
\let\childdocjob\childdocname
%    \end{macrocode}

% \macro{\childdocdisable}
% The macro |\childdocdisable| prevents the main file
% from being processed more than once.
% At this stage, the main document command |\childdocmain|
% is assumed to be called once again where it should do nothing.
% Any subsequent call to it should prevent
% a secondary processing of the main document
% It overwrites the forwarding commands
% |\childdocof| and |\childdocforward|
% with empty macros to prevent further inclusions of the main document:
%    \begin{macrocode}
\newcommand{\childdocdisable}
{
  \renewcommand{\childdocmain}[1]{\renewcommand{\childdocmain}[1]{\endinput}}
  \renewcommand{\childdocof}[1]{}
  \renewcommand{\childdocby}[2][]{}
  \renewcommand{\childdocforward}[2][]{}
  \renewcommand{\childdocdisable}{}
}
%    \end{macrocode}

% \macro{\childdocmain}
% The macro |\childdocmain| is to be called at the top of the main file
% with nothing or the main filename (without extension) as argument.
% First, it breaks loops.
% If the argument is not empty and does not match |\childdocname|
% (which is set by the first inclusion of |childdoc.def|),
% |\ifchilddoc| is set to true, |\includeonly| is applied to the child file
% and |\jobname| is set to the main file
% (for proper handling of |.aux| files):
%    \begin{macrocode}
\newcommand{\childdocmain}[1]
{
  \childdocdisable\childdocmain{}
  \if?#1?\else
    \begingroup
      \def\childdoctmp{#1}
      \ifx\childdoctmp\childdocname
        \def\childdoctmp{}
      \else
        \def\childdoctmp
        {
          \childdoctrue
          \includeonly{\childdocname}
          \def\childdocjob{#1}
          \def\jobname{#1}
        }
      \fi
      \expandafter
    \endgroup
    \childdoctmp
  \fi
}
%    \end{macrocode}

% \macro{\childdocof}
% The command |\childdocof| redirects
% compilation to the main file |#1|.
%    \begin{macrocode}
\newcommand{\childdocof}[1]
{
  \childdocdisable
  \childdoctrue
  \includeonly{\childdocname}
  \def\jobname{#1}
  \def\childdocjob{#1}
  \input{#1}
}
%    \end{macrocode}

% \macro{\childdocby}
% The command |\childdocby| ....
%    \begin{macrocode}
\newcommand{\childdocby}[2][]
{
  \childdocdisable
  \childdoctrue
  \childdocmanualtrue
  \if?#1?\else
    \def\jobname{#2}
  \fi
  \def\childdocjob{#2}
  \input{#2}
  \endinput
}
%    \end{macrocode}

% \macro{\childdocforward}
% The command |\childdocforward| redirects
% compilation to the main file or
% (if the optional argument is given) a child file.
% Parameters are set as if the main file
% or a child file starting with |\childdocof| was compiled.
% Then compilation is handed over to the main file:
%    \begin{macrocode}
\newcommand{\childdocforward}[2][]
{
  \begingroup
    \if?#1?
      \def\childdoctmp
      {
        \def\childdocname{#2}
        \def\childdocjob{#2}
        \def\jobname{#2}
        \input{#2}
        \endinput
      }
    \else
      \def\childdoctmp
      {
        \childdocdisable
        \def\childdocname{#2}
        \childdoctrue
        \includeonly{#2}
        \def\childdocjob{#1}
        \def\jobname{#1}
        \input{#1}
        \endinput
      }
    \fi
    \expandafter
  \endgroup
  \childdoctmp
}
%    \end{macrocode}

% \macro{\childdocforwardprefix}
% The command |\childdocforwardprefix| redirects
% compilation to the main or a child file by means of a pattern.
% The prefix |#1| in the current filename is replaced by |#2|
% and the suffix of the current filename is kept
% (it is assumed that the filename does not contain the substring `|~~~|'
% which is used as a delimiter).
% Compilation is handed over to the new file by |\childdocforward|:
%    \begin{macrocode}
\newcommand{\childdocforwardprefix}[3][]
{
  \begingroup
    \def\childdocextract #2##1~~~{\def\childdoctmp{\childdocforward[#1]{#3##1}}}
    \expandafter\childdocextract\childdocname~~~
    \expandafter
  \endgroup
  \childdoctmp
}
%    \end{macrocode}

% \macro{\childdoc}
% The deprecated macro |\childdoc| is a legacy version of |\childdocmain|:
%    \begin{macrocode}
\newcommand{\childdoc}{\childdocmain}
%    \end{macrocode}

% \macro{\childdocredirect}
% The deprecated macro |\childdocredirect| is a legacy version
% of |\childdocforward| and |\childdocforwardprefix|:
%    \begin{macrocode}
\newcommand{\childdocredirect}[2][]
{
  \begingroup
    \if?#1?
      \def\childdoctmp{\childdocforward{#2}}
    \else
      \def\childdoctmp{\childdocforwardprefix{#1}{#2}}
    \fi
    \expandafter
  \endgroup
  \childdoctmp
}
%    \end{macrocode}

%\iffalse
%</package>
%\fi
%
\endinput
\childdocforward[cdocsamp]{cdocsch1}"|\\
% |latex -jobname cdocscl2 \|\\
% |  "\def\version{final}% \iffalse
%
% childdoc.dtx Copyright (C) 2017-2018 Niklas Beisert
%
% This work may be distributed and/or modified under the
% conditions of the LaTeX Project Public License, either version 1.3
% of this license or (at your option) any later version.
% The latest version of this license is in
%   http://www.latex-project.org/lppl.txt
% and version 1.3 or later is part of all distributions of LaTeX
% version 2005/12/01 or later.
%
% This work has the LPPL maintenance status `maintained'.
%
% The Current Maintainer of this work is Niklas Beisert.
%
% This work consists of the files childdoc.dtx and childdoc.ins
% and the derived files childdoc.def and cdocsamp.tex with
% cdocsch1.tex, cdocsch2.tex, cdocsdrf.tex, cdocsfn1.tex, cdocsfn2.tex.
%
%<package>\ifdefined\childdocmain\endinput\fi
%<package>\ProvidesFile{childdoc.def}[2018/12/30 v2.0 child document driver]
%<samplemain>\ProvidesFile{cdocsamp.tex}[2018/12/30 v2.0 sample for childdoc]
%<*driver>
%\ProvidesFile{childdoc.drv}[2018/12/30 v2.0 childdoc reference manual file]
\PassOptionsToClass{10pt,a4paper}{article}
\documentclass{ltxdoc}

\usepackage[margin=35mm]{geometry}
\usepackage{hyperref}
\usepackage{hyperxmp}
\usepackage[usenames]{color}

\hypersetup{colorlinks=true}
\hypersetup{pdfstartview=FitH}
\hypersetup{pdfpagemode=UseNone}
\hypersetup{pdfsource={}}
\hypersetup{pdflang={en-UK}}
\hypersetup{pdfcopyright={Copyright 2017-2018 Niklas Beisert.
  This work may be distributed and/or modified under the
  conditions of the LaTeX Project Public License, either version 1.3
  of this license or (at your option) any later version.}}
\hypersetup{pdflicenseurl={http://www.latex-project.org/lppl.txt}}
\hypersetup{pdfcontactaddress={ETH Zurich, ITP, HIT K,
  Wolfgang-Pauli-Strasse 27}}
\hypersetup{pdfcontactpostcode={8093}}
\hypersetup{pdfcontactcity={Zurich}}
\hypersetup{pdfcontactcountry={Switzerland}}
\hypersetup{pdfcontactemail={nbeisert@itp.phys.ethz.ch}}
\hypersetup{pdfcontacturl={http://people.phys.ethz.ch/\xmptilde nbeisert/}}

\newcommand{\secref}[1]{\hyperref[#1]{section \ref*{#1}}}

\parskip1ex
\parindent0pt
\let\olditemize\itemize
\def\itemize{\olditemize\parskip0pt}

\begin{document}

\title{The \textsf{childdoc} Package}
\hypersetup{pdftitle={The childdoc Package}}
\author{Niklas Beisert\\[2ex]
  Institut f\"ur Theoretische Physik\\
  Eidgen\"ossische Technische Hochschule Z\"urich\\
  Wolfgang-Pauli-Strasse 27, 8093 Z\"urich, Switzerland\\[1ex]
  \href{mailto:nbeisert@itp.phys.ethz.ch}
  {\texttt{nbeisert@itp.phys.ethz.ch}}}
\hypersetup{pdfauthor={Niklas Beisert}}
\hypersetup{pdfsubject={Manual for the LaTeX2e Package childdoc}}
\date{30 December 2018, \textsf{v2.0}}
\maketitle

\begin{abstract}\noindent
\textsf{childdoc} is a \LaTeXe{} package
that enables the direct compilation
of document sections included by |\include|
to individual files.
\end{abstract}

\begingroup
\parskip0ex
\tableofcontents
\endgroup

%%%%%%%%%%%%%%%%%%%%%%%%%%%%%%%%%%%%%%%%%%%%%%%%%%%%%%%%%%%%%%%%%%%%%%%%%%%%%%%%
%%%%%%%%%%%%%%%%%%%%%%%%%%%%%%%%%%%%%%%%%%%%%%%%%%%%%%%%%%%%%%%%%%%%%%%%%%%%%%%%
\section{Introduction}

\LaTeX{} provides a mechanism to structure a large document (such as a book)
into a main file and several child files (containing the chapters)
using the |\include| command.
This mechanism is beneficial for documents
which span hundreds of pages in order to
make the source file(s) more manageable.
Moreover, compilation can be restricted to
selected child files by means of the |\includeonly| command.
The latter feature can be used to reduce the compilation time while editing
(this was significantly more useful in the earlier days of \LaTeX{})
or to generate a smaller document which is easier to navigate.
Another application of |\includeonly| is to generate
documents consisting of selected parts of the complete document.

However, there are a few drawbacks of the plain |\include| mechanism:
\begin{itemize}
\item
The child files cannot be compiled on their own,
they can only be compiled via the main file.
A naive editing environment
(such as a text editor with an option
to have the current file processed by \LaTeX)
may require one to switch to the main file before compiling;
attempting to compile the child file produces errors.
\item
The main file must be modified (each time)
to adjust the |\includeonly| command
to the present needs. This easily leaves the main file in a messy state.
\item
The generated document will always carry the filename
of the main document. This is inconvenient if
several child files are to be compiled and
to be kept for distribution.
\end{itemize}

The present package provides a simple interface
to make child files individually compilable by \LaTeX{}.
Compiling a child file then has the same effect as compiling
the main file with an |\includeonly| command
to select the appropriate child.
Moreover the generated document will carry the name of the child
rather than the main file.
This resolves all three above issues.

This feature is meant to make the editing of books,
thesis documents and lecture notes somewhat more convenient.
However, the package can also be used efficiently for
composing a series of documents (such as exercise sheets)
which are typically distributed individually.
It then assists the author in generating the individual documents
(potentially in different versions)
as well as a document containing the collected series.
Another application is in developing style files
or other kinds of included material
where compilation of the style file could redirect
to a sample or test file.

%%%%%%%%%%%%%%%%%%%%%%%%%%%%%%%%%%%%%%%%%%%%%%%%%%%%%%%%%%%%%%%%%%%%%%%%%%%%%%%%
%%%%%%%%%%%%%%%%%%%%%%%%%%%%%%%%%%%%%%%%%%%%%%%%%%%%%%%%%%%%%%%%%%%%%%%%%%%%%%%%
\section{Usage}

First of all, the package \textsf{childdoc} is \emph{not} a standard
\LaTeXe{} |.sty| style file! Therefore it needs to be invoked in
a non-standard way.

%%%%%%%%%%%%%%%%%%%%%%%%%%%%%%%%%%%%%%%%%%%%%%%%%%%%%%%%%%%%%%%%%%%%%%%%%%%%%%%%
\subsection{Included Files}
\label{sec:include}

%%%%%%%%%%%%%%%%%%%%%%%%%%%%%%%%%%%%%%%%
\DescribeMacro{\childdocmain}
To use the package, add the commands
\begin{center}
\begin{tabular}{l}
|\input{childdoc.def}|\\
|\childdocmain{}|\\
\end{tabular}
\end{center}
at the very top of the main \LaTeX{} file,
in particular \emph{before} the |\documentclass| statement!
The argument of |\childdocmain| should be left empty
(but it must be present).

%%%%%%%%%%%%%%%%%%%%%%%%%%%%%%%%%%%%%%%%
\DescribeMacro{\childdocof}
Furthermore, add the commands
\begin{center}
\begin{tabular}{l}
|\input{childdoc.def}|\\
|\childdocof{|\textit{main}|}|\\
\end{tabular}
\end{center}
at the top of every child file \textit{child}
which is included by |\include{|\textit{child}|}|
from within the main file
(or at least for those files to be compiled individually).
The argument \textit{main} must be the filename of the main file.

There are a couple of
considerations in setting up the main and child documents:

%%%%%%%%%%%%%%%%%%%%%%%%%%%%%%%%%%%%%%%%
\paragraph{Restrictions.}

Please note the following restrictions:
\begin{itemize}
\item
|\childdocmain| must be called with one argument \textit{main}
to ensure compatibility with earlier version of the package.
It must either be empty (|\childdocmain{}|)
or precisely match the filename of the main file in which it is specified.
See \secref{sec:detection} for further information.
\item
The filename \textit{main} must be specified without the |.tex| extension.
\item
The filename \textit{main} is case sensitive
(even in case-insensitive file systems)
due to internal string comparison.
\item
The argument \textit{main} should be fully expanded, it cannot be a macro.
\item
Subdirectories and special characters should be avoided in filenames.
\item
The command |\childdocmain{|\textit{main}|}| must be followed by a whitespace.
It should not be followed immediately by another command
or by a comment mark `|%|'.
This is because the \TeX{} parser reads the token immediately following
the argument of |\childdocmain| and puts it
at the beginning of every child section;
however, a white\-space is ignored.
\end{itemize}

%%%%%%%%%%%%%%%%%%%%%%%%%%%%%%%%%%%%%%%%
\paragraph{Content of Main File.}

It is advisable to place all content in the child files included by |\include|.
Any output contained in the main file will appear in all child documents
unless suppressed manually;
it cannot be suppressed automatically by the |\includeonly| directive
and thus should normally be avoided.
A method to include some content in the main file
by means of conditional processing is described in \secref{sec:conditional}.

%%%%%%%%%%%%%%%%%%%%%%%%%%%%%%%%%%%%%%%%
\paragraph{Page Numbering.}

When only a part of the document is compiled,
the appropriate numbering of pages
(as well as other status parameters)
is determined from the |.aux| files.
The latter contain information from previous passes.
However this information needs to propagate through
all intermediate child documents.
Therefore the page numbering in child documents may well
be inconsistent until the complete document is compiled at least once.

A useful (if unconventional) way to always ensure a consistent
page numbering is to restart the numbering in each child document
and denote the pages by `\textit{child}|.|\textit{page}'
where \textit{child} represents the chapter/section number of the child file.
This can be achieved by the command
|\numberwithin{page}{|\textit{child}|}|
of the \textsf{amsmath} package
where \textit{child} can be |chapter| or |section|
depending on the chosen structuring.
Alternatively, one can modify the macro |\thepage| appropriately
and reset the counter |page| at the start of each child file.

%%%%%%%%%%%%%%%%%%%%%%%%%%%%%%%%%%%%%%%%%%%%%%%%%%%%%%%%%%%%%%%%%%%%%%%%%%%%%%%%
\subsection{Conditional Processing}
\label{sec:conditional}

The package provides a mechanism to compile different versions
of a document. To customise the versions further some conditional processing
can come in handy to distinguish which version is being compiled.
The package provides two macros to describe the compilation context:

%%%%%%%%%%%%%%%%%%%%%%%%%%%%%%%%%%%%%%%%
\DescribeMacro{\ifchilddoc}
The conditional |\ifchilddoc| distinguishes between the compilation of
child documents and the main document:
%
\begin{center}
|\ifchilddoc |\textit{child-code}| |[|\||else |\textit{main-code}]| \||fi|
\end{center}

%%%%%%%%%%%%%%%%%%%%%%%%%%%%%%%%%%%%%%%%
\DescribeMacro{\childdocname}
\DescribeMacro{\childdocjob}
The macro |\childdocname| contains the filename (without extension)
of the main or child file being processed.
Note that |\childdocjob| will always contain the name of the main file.

%%%%%%%%%%%%%%%%%%%%%%%%%%%%%%%%%%%%%%%%
\paragraph{Title Page.}

Conditional processing can be used to include a title or banner page
in the main document when proper precautions are taken.
Importantly, the code in the main file should ensure that the page counter
(as well as other status parameters which are stored in the |.aux| files)
takes the same value after the conditional processing.
Otherwise the page numbers may take divergent values
depending on which part is compiled.

For example, a title page could be declared by:
%
\begin{center}
\begin{tabular}{l}
|\ifchilddoc\||else|\\
|\addtocounter{page}{-1}|\\
\textit{code for title page}\\
|\newpage|\\
|\||fi|
\end{tabular}
\end{center}
%
A banner page for the child documents can be generated by:
%
\begin{center}
\begin{tabular}{l}
|\ifchilddoc|\\
|\addtocounter{page}{-1}|\\
\textit{code for banner page}\\
|\newpage|\\
|\||fi|
\end{tabular}
\end{center}
%
Here one could write a message such as:
\begin{center}
|This is the part \childdocname{} of \childdocjob{}.|
\end{center}

%%%%%%%%%%%%%%%%%%%%%%%%%%%%%%%%%%%%%%%%%%%%%%%%%%%%%%%%%%%%%%%%%%%%%%%%%%%%%%%%
\subsection{Flags}
\label{sec:flags}

The package makes it easy to generate different versions
of the main or child documents.
To this end compilation flags can be defined
and assigned different default values.
They will be particularly useful in conjunction
with the forwarding mechanism described in \secref{sec:forward}.

For example, it may be useful to have a flag |\version|
which can be set to |draft| or |final|.
The document source will contain some conditional code
depending on the value of |\version|.
Suppose further, the flag should default to |final| for the main file
and to |draft| for child files
which is a natural assignment for editing the document.
This is achieved by placing the following code
in the preamble of the main document
(below the |\childdocmain| directive):
%
\begin{center}
\begin{tabular}{l}
|\ifchilddoc|\\
|\providecommand{\version}{draft}|\\
|\||else|\\
|\providecommand{\version}{final}|\\
|\||fi|
\end{tabular}
\end{center}
%
The definition by |\providecommand| makes sure
that previous definitions are not overwritten.
Further statements |\providecommand{\version}{...}|
can thus be added before the above code to override it.

For the main file, one might add a line
(between |\childdocmain| and the above block)
%
\begin{center}
|%\ifchilddoc\||else\providecommand{\version}{draft}\||fi|
\end{center}
%
which can be uncommented to produce a draft version.
Likewise one can add a line to the very top of a child file
(above the |\childdocof{|\textit{main}|}| directive)
%
\begin{center}
|%\providecommand{\version}{final}|
\end{center}
%
which can be uncommented to produce the final version of this child document.

%%%%%%%%%%%%%%%%%%%%%%%%%%%%%%%%%%%%%%%%%%%%%%%%%%%%%%%%%%%%%%%%%%%%%%%%%%%%%%%%
\subsection{Forwarding}
\label{sec:forward}

Different versions of the main or child documents
using compilation flags as described in \secref{sec:flags}
can be (permanently) stored in different files
for convenient compilation, viewing and distribution.
To this end, the package defines a command
to pass on compilation to a different file:

%%%%%%%%%%%%%%%%%%%%%%%%%%%%%%%%%%%%%%%%
\DescribeMacro{\childdocforward}
The command |\childdocforward| redirects processing to
another source file:
%
\begin{center}
\begin{tabular}{l}
|\input{childdoc.def}|\\
|\childdocforward[|\textit{main}|]{|\textit{dest}|}|\\
\end{tabular}
\end{center}
%
The argument \textit{dest} is the destination file
(without extension).
It should be the main file or one of the child files.
Note that further \textsf{childdoc} directives
such as |\childdocof| and |\childdocforward|
in the indicated file will be processed in this form.
The optional argument \textit{main}
passes on directly to the main file \textit{main}
while pretending to compile the child \textit{dest}.
This form behaves as if \textit{dest}
issues |\childdocof{|\textit{main}|}| right away,
and no further \textsf{childdoc} directives will be processed.

%%%%%%%%%%%%%%%%%%%%%%%%%%%%%%%%%%%%%%%%
\DescribeMacro{\...prefix}
In the alternative form |\childdocforwardprefix|,
%
\begin{center}
\begin{tabular}{l}
|\input{childdoc.def}|\\
|\childdocforwardprefix[|\textit{main}|]{|\textit{prefix}|}{|\textit{dest}|}|
\end{tabular}
\end{center}
%
the destination file is determined by a pattern
depending on the current file:
To make this work, the current file must be called
`{\textit{prefix}\hspace{0.2em}\textit{suffix}}'
with \textit{prefix} matching precisely the argument.
Processing is then passed on to the file
`{\textit{dest}\hspace{0.2em}\textit{suffix}}'.
Surely, the same effect is achieved by
directly specifying the
argument `{\textit{dest}\hspace{0.2em}\textit{suffix}}'
in the first form.
However, that requires to set up a different file
for each child. With the alternative form of the command
all these files can have exactly the same content
which simplifies setting them up and maintaining them.

For example, the following file |draft.tex|
with a compilation flag |\version| as described in \secref{sec:flags}
compiles the main document as a draft:
%
\begin{center}
\begin{tabular}{l}
|\def\version{draft}|\\
|\input{childdoc.def}|\\
|\childdocforward{|\textit{main}|}|
\end{tabular}
\end{center}
%
Likewise, the following files |final|\textit{nn}|.tex|
compile the final version of the child document
|child|\textit{nn}|.tex|:
%
\begin{center}
\begin{tabular}{l}
|\def\version{final}|\\
|\input{childdoc.def}|\\
|\childdocforwardprefix{final}{child}|
\end{tabular}
\end{center}
%

Note that when several versions of a main file and/or of each child file
are to be generated, it may be convenient to set up a |Makefile| or
shell script to automatise the process.

%%%%%%%%%%%%%%%%%%%%%%%%%%%%%%%%%%%%%%%%%%%%%%%%%%%%%%%%%%%%%%%%%%%%%%%%%%%%%%%%
\subsection{Command Line Processing}
\label{sec:commandline}

The effect of redirection files can also be achieved by invoking
the \LaTeX{} compiler with a more elaborate command line.
Most conveniently this should be done as part
of a shell script or a |Makefile|.

When using \textsf{childdoc} in the main file, the following
command lines effectively perform a redirection
(note that depending on the shell being used,
backslashes may have to be doubled: `|\|' $\to$ `|\\|'):
%
\begin{center}
|... -jobname "|\textit{target}|" |\\|"|[\textit{flags}]%
|\input{childdoc.def}\childdocforward[|\textit{main}|]{|\textit{dest}|}"|
\end{center}
%
Here \textit{target} is the name of the output file,
\textit{main} is the name of the main file
and \textit{dest} is the name of the main or child file to be processed
(all filenames without extensions).
The optional argument \textit{main} can be omitted
if \textit{main} matches \textit{dest}.
Optionally, compilation \textit{flags} can be defined via |\def| commands.
This command line makes the \TeX{} engine believe
it is compiling the file \textit{target}
whose content is specified as the latter parameter.
The provided code then forwards the processing to
\textit{main} or \textit{dest} as described in \secref{sec:forward}.

%%%%%%%%%%%%%%%%%%%%%%%%%%%%%%%%%%%%%%%%%%%%%%%%%%%%%%%%%%%%%%%%%%%%%%%%%%%%%%%%
\subsection{Include by Input}
\label{sec:input}

Including child documents by |\include| has some restrictions by design.
Most notably, the content of a child document always occupies
its own set of pages; pages cannot be shared between child documents.
Usually, this behaviour makes perfect sense
because each child document contain an essential part of the document.
However, in some situations it may be desirable to compose
a document from a collection of parts
without having mandatory page breaks between then.
For this case, the package
provides a mechanism to include parts
by |\input| which can also be processed individually.
However, by construction this mechanism
requires manual handling of the content to be output.

%%%%%%%%%%%%%%%%%%%%%%%%%%%%%%%%%%%%%%%%
\DescribeMacro{\ifchilddocmanual}
The main file should be prepared as usual, see \secref{sec:include}.
However, the document body must make a distinction
between processing of an individual part and of the main document, e.g.:
%
\begin{center}
\begin{tabular}{l}
|\ifchilddocmanual|\\
|\input{\childdocname}|\\
|\||else|\\
\textit{document body with }|\input{|\textit{part}|}|\\
|\||fi|
\end{tabular}
\end{center}
%
The conditional |\ifchilddocmanual| is true whenever
a part to be included by |\input| is being compiled,
and the name of the part is stored in |\childdocname|.

%%%%%%%%%%%%%%%%%%%%%%%%%%%%%%%%%%%%%%%%
\DescribeMacro{\childdocby}
Each part to be included by |\input| should start with:
%
\begin{center}
\begin{tabular}{l}
|\input{childdoc.def}|\\
|\childdocby{|\textit{main}|}|\\
\end{tabular}
\end{center}
%
The directive |\childdocby| is similar to |\childdocof|
described in \secref{sec:include},
but the subsequent selection of content must be done manually.
To that end, both |\ifchilddoc| and |\ifchilddocmanual|
will be true upon processing of a part,
and the name of the part is stored in |\childdocname|.
Note that |\jobname| will be set to the filename of the current part
so that each part receives an individual |.aux| file
that does not interfere with the |.aux| file(s) of the main document.
This behaviour can be altered by the alternative form
|\childdocby[*]{|\textit{main}|}| (with a non-empty optional argument)
which uses the |.aux| file of the main document
by setting |\jobname| to \textit{main}.

%%%%%%%%%%%%%%%%%%%%%%%%%%%%%%%%%%%%%%%%%%%%%%%%%%%%%%%%%%%%%%%%%%%%%%%%%%%%%%%%
\subsection{Driver Development}
\label{sec:driver}

The \textsf{childdoc} mechanism can also be use for the development
of definition files such as \LaTeX{} styles or classes.
This case differs from the above setup with multiple parts
included by |\include| in that no |\includeonly| should be invoked.
This can be achieved by starting the include file
(before |\ProvidesPackage|) with:
%
\begin{center}
\begin{tabular}{l}
|\input{childdoc.def}|\\
|\childdocforward{|\textit{main}|}|\\
\end{tabular}
\end{center}
%
or alternatively with:
%
\begin{center}
\begin{tabular}{l}
|\input{childdoc.def}|\\
|\childdocby{|\textit{main}|}|\\
\end{tabular}
\end{center}
%
Both forms have slightly different effects as described above.
The main file is prepared as usual, see \secref{sec:include}.

%%%%%%%%%%%%%%%%%%%%%%%%%%%%%%%%%%%%%%%%%%%%%%%%%%%%%%%%%%%%%%%%%%%%%%%%%%%%%%%%
\subsection{Legacy Detection}
\label{sec:detection}

The directive |\childdocmain| in the main file can detect
whether the complete document or merely a child is to be compiled
even without using the directive |\childdocof|.
This method is deprecated because it is less robust
and there is no compelling reason to use it;
it is merely provided for backward compatibility
and it may be removed in future versions.

If the detection mechanism is to be used,
it is mandatory to correctly specify
the filename of the main file as the argument of |\childdocmain|:
%
\begin{center}
\begin{tabular}{l}
|\input{childdoc.def}|\\
|\childdocmain{|\textit{main}|}|\\
\end{tabular}
\end{center}
%
If |\jobname| does not match the argument \textit{main} of |\childdocmain|,
it is assumed that |\jobname| points to the child file to be compiled.
When using |\childdocmain| with the main file specified as argument,
it suffices to start a child file
with just |\input{|\textit{main}|}|
without loading of the package and using |\childdocof|.
If instead all processing is done
with the appropriate \textsf{childdoc} directives,
the argument of \textit{main} of |\childdocmain| can be empty.

An alternative version of the command line processing described
in \secref{sec:commandline} using the detection mechanism reads:
%
\begin{center}
|... -jobname "|\textit{target}|" "|[\textit{flags}]%
[|\def\jobname{|\textit{dest}|}|]|\input{|\textit{main}|}"|
\end{center}

%%%%%%%%%%%%%%%%%%%%%%%%%%%%%%%%%%%%%%%%%%%%%%%%%%%%%%%%%%%%%%%%%%%%%%%%%%%%%%%%
\subsection{Manual Code}
\label{sec:manual}

In case one cannot be certain whether the definitions file |childdoc.def|
is installed on the target \TeX{} distribution
and one prefers not to ship it,
it is conceivable to paste a few relevant commands into the sources.

To that end, drop all statements |\input{childdoc.def}|
and perform the replacements as outlined below.
Instead of |\childdocmain{|\textit{main}|}| add the following code
to the top of the main file:
%
\begin{center}
\begin{tabular}{l}
|\||ifdefined\childdocname\endinput\||fi\newif\ifchilddoc|\\
|\edef\childdocname{\scantokens\expandafter{\jobname\noexpand}}|\\
|\def\childdocmain{|\textit{main}|}\||ifx\childdocmain\childdocname\||else|\\
|\childdoctrue\includeonly{\childdocname}\let\jobname\childdocmain\||fi|\\
\end{tabular}
\end{center}
%
Instead of |\childdocof{|\textit{main}|}| just include the main file
at the top of each child file:
%
\begin{center}
|\input{|\textit{main}|}|
\end{center}
%
A simple redirection |\childdocforward{|\textit{dest}|}| is achieved by:
%
\begin{center}
|\def\jobname{|\textit{dest}|}\input{\jobname}|
\end{center}
%
The redirection with prefix
|\childdocforwardprefix[|\textit{prefix}|]{|\textit{dest}|}|
is accomplished by:
%
\begin{center}
\begin{tabular}{l}
|{\edef\jobname{\scantokens\expandafter{\jobname\noexpand}}|\\
|\def\redirectjob |\textit{prefix}|#1~~~{\gdef\jobname{|\textit{dest}|#1}}|\\
|\expandafter\redirectjob\jobname~~~}\input{\jobname}|
\end{tabular}
\end{center}

In an alternative approach,
child documents can be compiled by a specific command line
without additional code or specific definitions:
%
\begin{center}
|... -jobname "|\textit{target}|" "|[\textit{flags}]%
|\includeonly{|\textit{dest}|}\input{|\textit{main}|}"|
\end{center}
%

%%%%%%%%%%%%%%%%%%%%%%%%%%%%%%%%%%%%%%%%%%%%%%%%%%%%%%%%%%%%%%%%%%%%%%%%%%%%%%%%
%%%%%%%%%%%%%%%%%%%%%%%%%%%%%%%%%%%%%%%%%%%%%%%%%%%%%%%%%%%%%%%%%%%%%%%%%%%%%%%%
\section{Information}

%%%%%%%%%%%%%%%%%%%%%%%%%%%%%%%%%%%%%%%%%%%%%%%%%%%%%%%%%%%%%%%%%%%%%%%%%%%%%%%%
\subsection{Copyright}

Copyright \copyright{} 2017--2018 Niklas Beisert

This work may be distributed and/or modified under the
conditions of the \LaTeX{} Project Public License, either version 1.3
of this license or (at your option) any later version.
The latest version of this license is in
  \url{http://www.latex-project.org/lppl.txt}
and version 1.3 or later is part of all distributions of \LaTeX{}
version 2005/12/01 or later.

This work has the LPPL maintenance status `maintained'.

The Current Maintainer of this work is Niklas Beisert.

This work consists of the files |README.txt|, |childdoc.ins| and |childdoc.dtx|
as well as the derived files |childdoc.def|, |cdocsamp.tex|
with |cdocsch1.tex|, |cdocsch2.tex|, |cdocspt3.tex|, |cdocspt4.tex|,
|cdocsdrf.tex|, |cdocsfn1.tex|, |cdocsfn2.tex|
as well as |childdoc.pdf|.

%%%%%%%%%%%%%%%%%%%%%%%%%%%%%%%%%%%%%%%%%%%%%%%%%%%%%%%%%%%%%%%%%%%%%%%%%%%%%%%%
\subsection{Files and Installation}

The package consists of the files:
%
\begin{center}
\begin{tabular}{ll}
    |README.txt|   & readme file \\
    |childdoc.ins| & installation file \\
    |childdoc.dtx| & source file \\
    |childdoc.def| & definition file \\
    |cdocsamp.tex| & sample main file \\
    |cdocsch1.tex| & sample include file \\
    |cdocsch2.tex| & sample include file \\
    |cdocspt3.tex| & sample part file \\
    |cdocspt4.tex| & sample part file \\
    |cdocsdrf.tex| & sample redirection file \\
    |cdocsfn1.tex| & sample redirection file \\
    |cdocsfn2.tex| & sample redirection file \\
    |childdoc.pdf| & manual
\end{tabular}
\end{center}
%
The distribution consists of the files
|README.txt|, |childdoc.ins| and |childdoc.dtx|.
%
\begin{itemize}
\item
Run (pdf)\LaTeX{} on |childdoc.dtx|
to compile the manual |childdoc.pdf| (this file).
\item
Run \LaTeX{} on |childdoc.ins| to create the definitions file |childdoc.def|
and the sample |cdocsamp.tex| with include files
|cdocsch1.tex|, |cdocsch2.tex|, |cdocspt3.tex|, |cdocspt4.tex|,
|cdocsdrf.tex|, |cdocsfn1.tex|, |cdocsfn2.tex|.
Then copy the file |childdoc.def| to an appropriate directory of your \LaTeX{}
distribution, e.g.\ \textit{texmf-root}|/tex/latex/childdoc|.
\end{itemize}

%%%%%%%%%%%%%%%%%%%%%%%%%%%%%%%%%%%%%%%%%%%%%%%%%%%%%%%%%%%%%%%%%%%%%%%%%%%%%%%%
\subsection{Related CTAN Packages}

There are several other packages which offer a similar functionality:
%
\begin{itemize}
\item
The packages
\href{http://ctan.org/pkg/docmute}{\textsf{docmute}},
\href{http://ctan.org/pkg/includex}{\textsf{includex}} and
\href{http://ctan.org/pkg/standalone}{\textsf{standalone}}
provide commands to include only the document body of
a child file thus allowing both files to be compiled individually.
\item
The packages \href{http://ctan.org/pkg/subdocs}{\textsf{subdocs}}
and \href{http://ctan.org/pkg/subfiles}{\textsf{subfiles}}
provide structures in which the main and child documents can be
encapsulated and allowing them to be compiled individually.
The inclusion mechanism is different from the conventional |\include|.
\item
The package \href{http://ctan.org/pkg/combine}{\textsf{combine}}
is an elaborate solution to combine several documents into one.
\end{itemize}
%
See also the CTAN topic \href{http://ctan.org/topic/subdocs}{\textsf{subdocs}}
for further related packages.
The present package differs from the above solutions in that
a document structure constructed with the conventional |\include| mechanism
just needs two extra commands at the top of every file
such that all constituent files can be compiled individually.

%%%%%%%%%%%%%%%%%%%%%%%%%%%%%%%%%%%%%%%%%%%%%%%%%%%%%%%%%%%%%%%%%%%%%%%%%%%%%%%%
%\subsection{Feature Suggestions}
%
%The following is a list of features which may be useful for future
%versions of this package:
%%
%\begin{itemize}
%\item
%\ldots
%\end{itemize}

%%%%%%%%%%%%%%%%%%%%%%%%%%%%%%%%%%%%%%%%%%%%%%%%%%%%%%%%%%%%%%%%%%%%%%%%%%%%%%%%
\subsection{Revision History}

%%%%%%%%%%%%%%%%%%%%%%%%%%%%%%%%%%%%%%%%
\paragraph{v2.0:} 2018/12/30

\begin{itemize}
\item
immediate forward processing
\item
added |\childdocby| mechanism
\item
manual restructured
\end{itemize}

%%%%%%%%%%%%%%%%%%%%%%%%%%%%%%%%%%%%%%%%
\paragraph{v1.6:} 2018/01/17

\begin{itemize}
\item
application for development of include files
\item
corrections to manual
\end{itemize}

%%%%%%%%%%%%%%%%%%%%%%%%%%%%%%%%%%%%%%%%
\paragraph{v1.5:} 2017/05/21

\begin{itemize}
\item
more complete structuring introduced
\item
|\childdocof| introduced
\item
|\childdoc| renamed to |\childdocmain|
\item
|\childredirect| renamed to |\childdocforward| and |\childdocforwardprefix|
and functionality expanded
\end{itemize}

%%%%%%%%%%%%%%%%%%%%%%%%%%%%%%%%%%%%%%%%
\paragraph{v1.0:} 2017/04/27

\begin{itemize}
\item
manual and install package
\item
first version published on CTAN
\end{itemize}

%%%%%%%%%%%%%%%%%%%%%%%%%%%%%%%%%%%%%%%%
\paragraph{v0.6:} 2017/04/26

\begin{itemize}
\item
redirection mechanism added
\end{itemize}

%%%%%%%%%%%%%%%%%%%%%%%%%%%%%%%%%%%%%%%%
\paragraph{v0.5:} 2017/04/26

\begin{itemize}
\item
functionality in definition file
\end{itemize}


%%%%%%%%%%%%%%%%%%%%%%%%%%%%%%%%%%%%%%%%%%%%%%%%%%%%%%%%%%%%%%%%%%%%%%%%%%%%%%%%
%%%%%%%%%%%%%%%%%%%%%%%%%%%%%%%%%%%%%%%%%%%%%%%%%%%%%%%%%%%%%%%%%%%%%%%%%%%%%%%%
%%%%%%%%%%%%%%%%%%%%%%%%%%%%%%%%%%%%%%%%%%%%%%%%%%%%%%%%%%%%%%%%%%%%%%%%%%%%%%%%
\appendix

\settowidth\MacroIndent{\rmfamily\scriptsize 000\ }

 \DocInput{childdoc.dtx}

\end{document}
%</driver>
% \fi
%
% %%%%%%%%%%%%%%%%%%%%%%%%%%%%%%%%%%%%%%%%%%%%%%%%%%%%%%%%%%%%%%%%%%%%%%%%%%%%%%
% %%%%%%%%%%%%%%%%%%%%%%%%%%%%%%%%%%%%%%%%%%%%%%%%%%%%%%%%%%%%%%%%%%%%%%%%%%%%%%
% \section{Sample}
%\iffalse
%<*samplemain>
%\fi
%
% The following presents a sample document
% with two chapters, two parts, a title page,
% a compile flag as well as three forwarding files to set the flag.
% It consists of eight |.tex| files:
% \begin{center}
% \begin{tabular}{ll}
% |cdocsamp.tex|&main file\\
% |cdocsch1.tex|&include file for chapter 1\\
% |cdocsch2.tex|&include file for chapter 2\\
% |cdocspt3.tex|&include file for part 3\\
% |cdocspt4.tex|&include file for part 4\\
% |cdocsdrf.tex|&forwarding file for main file in draft mode\\
% |cdocsfi1.tex|&forwarding file for final version of chapter 1\\
% |cdocsfi2.tex|&forwarding file for final version of chapter 2\\
% \end{tabular}
% \end{center}
% Each of the eight files can be compiled directly by the \LaTeX{} compiler.
%
% %%%%%%%%%%%%%%%%%%%%%%%%%%%%%%%%%%%%%%
% \paragraph{Main File.}
%
% The main file is called |cdocsamp.tex|.
%
% Load the \textsf{childdoc} definitions and
% declare the filename for the main document:
%    \begin{macrocode}
\input{childdoc.def}
\childdocmain{}
%    \end{macrocode}

% Optional override for |\version| flag:
%    \begin{macrocode}
%%\ifchilddoc\else\providecommand{\version}{draft}\fi
%    \end{macrocode}

% Define the default values for the |\version| flag
% (|final| for the main file and |draft| for childs):
%    \begin{macrocode}
\ifchilddoc
\providecommand{\version}{draft}
\else
\providecommand{\version}{final}
\fi
%    \end{macrocode}

% Load the standard document class:
%    \begin{macrocode}
\documentclass[12pt]{article}
%    \end{macrocode}

% Start the document body:
%    \begin{macrocode}
\begin{document}
%    \end{macrocode}

% Declare a title page.
% Print title, part of document being processed and version flag:
%    \begin{macrocode}
\addtocounter{page}{-1}
\begin{center}
{\LARGE\bfseries{}childdoc example\par}
\vspace{1cm}
\ifchilddoc
\ifchilddocmanual part\else chapter\fi:
`\childdocname' of `\childdocjob'\par
\else
main document: `\childdocjob'\par
\fi
version: \version\par
\end{center}
\newpage
%    \end{macrocode}

% Manually include selected file,
% otherwise process as usual:
%    \begin{macrocode}
\ifchilddocmanual
\section*{part `\childdocname'}
\input{\childdocname}
\else
%    \end{macrocode}

% Include the two chapters:
%    \begin{macrocode}
\include{cdocsch1}
\include{cdocsch2}
%    \end{macrocode}

% Include the two parts unless only chapters should be displayed:
%    \begin{macrocode}
\ifchilddoc\else
\section{part three}
\input{cdocspt3}
\section{part four}
\input{cdocspt4}
\fi
%    \end{macrocode}

% Process as usual until here:
%    \begin{macrocode}
\fi
%    \end{macrocode}

% End of document body:
%    \begin{macrocode}
\end{document}
%    \end{macrocode}
%\iffalse
%</samplemain>
%\fi
%
% %%%%%%%%%%%%%%%%%%%%%%%%%%%%%%%%%%%%%%
% \paragraph{Chapter Include Files.}
%
% The include files are called |cdocsch1.tex| and |cdocsch2.tex|.
%
%\iffalse
%<*samplechap1|samplechap2>
%\fi

% Optional override for |\version| flag:
%    \begin{macrocode}
%%\providecommand{\version}{final}
%    \end{macrocode}

% Include the main document:
%    \begin{macrocode}
\input{childdoc.def}
\childdocof{cdocsamp}
%    \end{macrocode}

%\iffalse
%</samplechap1|samplechap2>
%\fi
%
%\iffalse
%<*samplechap1>
%\fi
% Some text for chapter 1:
%    \begin{macrocode}
\section{one}
some text in chapter one
%    \end{macrocode}

%\iffalse
%</samplechap1>
%\fi
% Some text for chapter 2:
%\iffalse
%<*samplechap2>
%\fi
%    \begin{macrocode}
\section{two}
more text in chapter two
%    \end{macrocode}

%\iffalse
%</samplechap2>
%\fi
%
% %%%%%%%%%%%%%%%%%%%%%%%%%%%%%%%%%%%%%%
% \paragraph{Part Include Files.}
%
% The include files are called |cdocspt3.tex| and |cdocspt4.tex|.
%
%\iffalse
%<*samplepart3|samplepart4>
%\fi

% Optional override for |\version| flag:
%    \begin{macrocode}
%%\providecommand{\version}{final}
%    \end{macrocode}

% Include the main document:
%    \begin{macrocode}
\input{childdoc.def}
\childdocby{cdocsamp}
%    \end{macrocode}

%\iffalse
%</samplepart3|samplepart4>
%\fi
%
%\iffalse
%<*samplepart3>
%\fi
% Some text for part 3:
%    \begin{macrocode}
some text in part three
%    \end{macrocode}

%\iffalse
%</samplepart3>
%\fi
% Some text for part 4:
%\iffalse
%<*samplepart4>
%\fi
%    \begin{macrocode}
more text in part four
%    \end{macrocode}

%\iffalse
%</samplepart4>
%\fi
%
% %%%%%%%%%%%%%%%%%%%%%%%%%%%%%%%%%%%%%%
% \paragraph{Forwarding for a Complete Draft.}
%
% The following forwarding file |cdocsdrf.tex|
% compiles the main document in draft mode:
%\iffalse
%<*sampledraft>
%\fi
%    \begin{macrocode}
\def\version{draft}
\input{childdoc.def}
\childdocforward{cdocsamp}
%    \end{macrocode}

%\iffalse
%</sampledraft>
%\fi
%
% %%%%%%%%%%%%%%%%%%%%%%%%%%%%%%%%%%%%%%
% \paragraph{Forwarding for Final Version of the Chapters.}
%
% The following forwarding files |cdocsfn1.tex| and |cdocsfn2.tex|
% (with identical content)
% compile the final versions of the child documents
% |cdocsch1.tex| and |cdocsch2.tex|, respectively:
%\iffalse
%<*samplefinal>
%\fi
%    \begin{macrocode}
\def\version{final}
\input{childdoc.def}
\childdocforwardprefix[cdocsamp]{cdocsfn}{cdocsch}
%    \end{macrocode}

%\iffalse
%</samplefinal>
%\fi
%
% %%%%%%%%%%%%%%%%%%%%%%%%%%%%%%%%%%%%%%
% \paragraph{Command Line Processing.}
%
% The following three command lines generate the output files
% |cdocscld|, |cdocscl1| and |cdocscl2|
% which should be identical to
% |cdocsdrf|, |cdocsch1| and |cdocsfn2|, respectively:
% \begin{center}
% \begin{tabular}{l}
% |latex -jobname cdocscld \|\\
% |  "\def\version{draft}\input{childdoc.def}\childdocforward{cdocsamp}"|\\
% |latex -jobname cdocscl1 \|\\
% |  "\input{childdoc.def}\childdocforward[cdocsamp]{cdocsch1}"|\\
% |latex -jobname cdocscl2 \|\\
% |  "\def\version{final}\input{childdoc.def}\childdocforward{cdocsch2}"|
% \end{tabular}
% \end{center}
% Note that the trailing backslash on each first line
% merely continues the input to the second line
% (for convenient cut ant paste).
% Furthermore, the command |latex| can be replaced by any
% of its alternative versions such as |pdflatex|.
%
% %%%%%%%%%%%%%%%%%%%%%%%%%%%%%%%%%%%%%%%%%%%%%%%%%%%%%%%%%%%%%%%%%%%%%%%%%%%%%%
% %%%%%%%%%%%%%%%%%%%%%%%%%%%%%%%%%%%%%%%%%%%%%%%%%%%%%%%%%%%%%%%%%%%%%%%%%%%%%%
% \section{Implementation}
%\iffalse
%<*package>
%\fi
%
% This section describes the definitions file |childdoc.def|.

% The definitions cannot be loaded using |\usepackage| or |\RequirePackage|
% which has a mechanism to prevent loading a style file more than once.
% When loading the definitions by means of |\input|
% multiple instances have to be prevented manually:
%\iffalse
%This code needs to be before the `\ProvidesFile' directive
%which is defined at the beginning of this file.
%Therefore it is also placed there and commented out here.
%</package>
%<*discard>
%\fi
%    \begin{macrocode}
\ifdefined\childdocmain\endinput\fi
%    \end{macrocode}
%\iffalse
%</discard>
%<*package>
%\fi
%
% \macro{\ifchilddoc}
% \macro{\ifchilddocmanual}
% The conditional |\ifchilddoc| tells whether a
% child (true) or main (false) document is being compiled.
% The conditional |\ifchilddocmanual| tells whether
% the |\includeonly| mechanism is used (false) or
% the selection of child files must be performed manually (true).
% The definitions initialise to false:
%    \begin{macrocode}
\newif\ifchilddoc
\newif\ifchilddocmanual
%    \end{macrocode}

% \macro{\childdocname}
% \macro{\childdocjob}
% The macro |\childdocname| stores the name of the main document
% to be compiled. The macro |\childdocjob| stores the name of
% the document on which the \LaTeX{} compiler was originally invoked.
% The content of |\jobname| cannot be compared
% to filenames specified in the source due to different catcodes.
% The following code rescans |\jobname|, stores the result
% in |\childdocname| and saves a copy in |\childdocjob|:
%    \begin{macrocode}
\edef\childdocname{\scantokens\expandafter{\jobname\noexpand}}
\let\childdocjob\childdocname
%    \end{macrocode}

% \macro{\childdocdisable}
% The macro |\childdocdisable| prevents the main file
% from being processed more than once.
% At this stage, the main document command |\childdocmain|
% is assumed to be called once again where it should do nothing.
% Any subsequent call to it should prevent
% a secondary processing of the main document
% It overwrites the forwarding commands
% |\childdocof| and |\childdocforward|
% with empty macros to prevent further inclusions of the main document:
%    \begin{macrocode}
\newcommand{\childdocdisable}
{
  \renewcommand{\childdocmain}[1]{\renewcommand{\childdocmain}[1]{\endinput}}
  \renewcommand{\childdocof}[1]{}
  \renewcommand{\childdocby}[2][]{}
  \renewcommand{\childdocforward}[2][]{}
  \renewcommand{\childdocdisable}{}
}
%    \end{macrocode}

% \macro{\childdocmain}
% The macro |\childdocmain| is to be called at the top of the main file
% with nothing or the main filename (without extension) as argument.
% First, it breaks loops.
% If the argument is not empty and does not match |\childdocname|
% (which is set by the first inclusion of |childdoc.def|),
% |\ifchilddoc| is set to true, |\includeonly| is applied to the child file
% and |\jobname| is set to the main file
% (for proper handling of |.aux| files):
%    \begin{macrocode}
\newcommand{\childdocmain}[1]
{
  \childdocdisable\childdocmain{}
  \if?#1?\else
    \begingroup
      \def\childdoctmp{#1}
      \ifx\childdoctmp\childdocname
        \def\childdoctmp{}
      \else
        \def\childdoctmp
        {
          \childdoctrue
          \includeonly{\childdocname}
          \def\childdocjob{#1}
          \def\jobname{#1}
        }
      \fi
      \expandafter
    \endgroup
    \childdoctmp
  \fi
}
%    \end{macrocode}

% \macro{\childdocof}
% The command |\childdocof| redirects
% compilation to the main file |#1|.
%    \begin{macrocode}
\newcommand{\childdocof}[1]
{
  \childdocdisable
  \childdoctrue
  \includeonly{\childdocname}
  \def\jobname{#1}
  \def\childdocjob{#1}
  \input{#1}
}
%    \end{macrocode}

% \macro{\childdocby}
% The command |\childdocby| ....
%    \begin{macrocode}
\newcommand{\childdocby}[2][]
{
  \childdocdisable
  \childdoctrue
  \childdocmanualtrue
  \if?#1?\else
    \def\jobname{#2}
  \fi
  \def\childdocjob{#2}
  \input{#2}
  \endinput
}
%    \end{macrocode}

% \macro{\childdocforward}
% The command |\childdocforward| redirects
% compilation to the main file or
% (if the optional argument is given) a child file.
% Parameters are set as if the main file
% or a child file starting with |\childdocof| was compiled.
% Then compilation is handed over to the main file:
%    \begin{macrocode}
\newcommand{\childdocforward}[2][]
{
  \begingroup
    \if?#1?
      \def\childdoctmp
      {
        \def\childdocname{#2}
        \def\childdocjob{#2}
        \def\jobname{#2}
        \input{#2}
        \endinput
      }
    \else
      \def\childdoctmp
      {
        \childdocdisable
        \def\childdocname{#2}
        \childdoctrue
        \includeonly{#2}
        \def\childdocjob{#1}
        \def\jobname{#1}
        \input{#1}
        \endinput
      }
    \fi
    \expandafter
  \endgroup
  \childdoctmp
}
%    \end{macrocode}

% \macro{\childdocforwardprefix}
% The command |\childdocforwardprefix| redirects
% compilation to the main or a child file by means of a pattern.
% The prefix |#1| in the current filename is replaced by |#2|
% and the suffix of the current filename is kept
% (it is assumed that the filename does not contain the substring `|~~~|'
% which is used as a delimiter).
% Compilation is handed over to the new file by |\childdocforward|:
%    \begin{macrocode}
\newcommand{\childdocforwardprefix}[3][]
{
  \begingroup
    \def\childdocextract #2##1~~~{\def\childdoctmp{\childdocforward[#1]{#3##1}}}
    \expandafter\childdocextract\childdocname~~~
    \expandafter
  \endgroup
  \childdoctmp
}
%    \end{macrocode}

% \macro{\childdoc}
% The deprecated macro |\childdoc| is a legacy version of |\childdocmain|:
%    \begin{macrocode}
\newcommand{\childdoc}{\childdocmain}
%    \end{macrocode}

% \macro{\childdocredirect}
% The deprecated macro |\childdocredirect| is a legacy version
% of |\childdocforward| and |\childdocforwardprefix|:
%    \begin{macrocode}
\newcommand{\childdocredirect}[2][]
{
  \begingroup
    \if?#1?
      \def\childdoctmp{\childdocforward{#2}}
    \else
      \def\childdoctmp{\childdocforwardprefix{#1}{#2}}
    \fi
    \expandafter
  \endgroup
  \childdoctmp
}
%    \end{macrocode}

%\iffalse
%</package>
%\fi
%
\endinput
\childdocforward{cdocsch2}"|
% \end{tabular}
% \end{center}
% Note that the trailing backslash on each first line
% merely continues the input to the second line
% (for convenient cut ant paste).
% Furthermore, the command |latex| can be replaced by any
% of its alternative versions such as |pdflatex|.
%
% %%%%%%%%%%%%%%%%%%%%%%%%%%%%%%%%%%%%%%%%%%%%%%%%%%%%%%%%%%%%%%%%%%%%%%%%%%%%%%
% %%%%%%%%%%%%%%%%%%%%%%%%%%%%%%%%%%%%%%%%%%%%%%%%%%%%%%%%%%%%%%%%%%%%%%%%%%%%%%
% \section{Implementation}
%\iffalse
%<*package>
%\fi
%
% This section describes the definitions file |childdoc.def|.

% The definitions cannot be loaded using |\usepackage| or |\RequirePackage|
% which has a mechanism to prevent loading a style file more than once.
% When loading the definitions by means of |\input|
% multiple instances have to be prevented manually:
%\iffalse
%This code needs to be before the `\ProvidesFile' directive
%which is defined at the beginning of this file.
%Therefore it is also placed there and commented out here.
%</package>
%<*discard>
%\fi
%    \begin{macrocode}
\ifdefined\childdocmain\endinput\fi
%    \end{macrocode}
%\iffalse
%</discard>
%<*package>
%\fi
%
% \macro{\ifchilddoc}
% \macro{\ifchilddocmanual}
% The conditional |\ifchilddoc| tells whether a
% child (true) or main (false) document is being compiled.
% The conditional |\ifchilddocmanual| tells whether
% the |\includeonly| mechanism is used (false) or
% the selection of child files must be performed manually (true).
% The definitions initialise to false:
%    \begin{macrocode}
\newif\ifchilddoc
\newif\ifchilddocmanual
%    \end{macrocode}

% \macro{\childdocname}
% \macro{\childdocjob}
% The macro |\childdocname| stores the name of the main document
% to be compiled. The macro |\childdocjob| stores the name of
% the document on which the \LaTeX{} compiler was originally invoked.
% The content of |\jobname| cannot be compared
% to filenames specified in the source due to different catcodes.
% The following code rescans |\jobname|, stores the result
% in |\childdocname| and saves a copy in |\childdocjob|:
%    \begin{macrocode}
\edef\childdocname{\scantokens\expandafter{\jobname\noexpand}}
\let\childdocjob\childdocname
%    \end{macrocode}

% \macro{\childdocdisable}
% The macro |\childdocdisable| prevents the main file
% from being processed more than once.
% At this stage, the main document command |\childdocmain|
% is assumed to be called once again where it should do nothing.
% Any subsequent call to it should prevent
% a secondary processing of the main document
% It overwrites the forwarding commands
% |\childdocof| and |\childdocforward|
% with empty macros to prevent further inclusions of the main document:
%    \begin{macrocode}
\newcommand{\childdocdisable}
{
  \renewcommand{\childdocmain}[1]{\renewcommand{\childdocmain}[1]{\endinput}}
  \renewcommand{\childdocof}[1]{}
  \renewcommand{\childdocby}[2][]{}
  \renewcommand{\childdocforward}[2][]{}
  \renewcommand{\childdocdisable}{}
}
%    \end{macrocode}

% \macro{\childdocmain}
% The macro |\childdocmain| is to be called at the top of the main file
% with nothing or the main filename (without extension) as argument.
% First, it breaks loops.
% If the argument is not empty and does not match |\childdocname|
% (which is set by the first inclusion of |childdoc.def|),
% |\ifchilddoc| is set to true, |\includeonly| is applied to the child file
% and |\jobname| is set to the main file
% (for proper handling of |.aux| files):
%    \begin{macrocode}
\newcommand{\childdocmain}[1]
{
  \childdocdisable\childdocmain{}
  \if?#1?\else
    \begingroup
      \def\childdoctmp{#1}
      \ifx\childdoctmp\childdocname
        \def\childdoctmp{}
      \else
        \def\childdoctmp
        {
          \childdoctrue
          \includeonly{\childdocname}
          \def\childdocjob{#1}
          \def\jobname{#1}
        }
      \fi
      \expandafter
    \endgroup
    \childdoctmp
  \fi
}
%    \end{macrocode}

% \macro{\childdocof}
% The command |\childdocof| redirects
% compilation to the main file |#1|.
%    \begin{macrocode}
\newcommand{\childdocof}[1]
{
  \childdocdisable
  \childdoctrue
  \includeonly{\childdocname}
  \def\jobname{#1}
  \def\childdocjob{#1}
  \input{#1}
}
%    \end{macrocode}

% \macro{\childdocby}
% The command |\childdocby| ....
%    \begin{macrocode}
\newcommand{\childdocby}[2][]
{
  \childdocdisable
  \childdoctrue
  \childdocmanualtrue
  \if?#1?\else
    \def\jobname{#2}
  \fi
  \def\childdocjob{#2}
  \input{#2}
  \endinput
}
%    \end{macrocode}

% \macro{\childdocforward}
% The command |\childdocforward| redirects
% compilation to the main file or
% (if the optional argument is given) a child file.
% Parameters are set as if the main file
% or a child file starting with |\childdocof| was compiled.
% Then compilation is handed over to the main file:
%    \begin{macrocode}
\newcommand{\childdocforward}[2][]
{
  \begingroup
    \if?#1?
      \def\childdoctmp
      {
        \def\childdocname{#2}
        \def\childdocjob{#2}
        \def\jobname{#2}
        \input{#2}
        \endinput
      }
    \else
      \def\childdoctmp
      {
        \childdocdisable
        \def\childdocname{#2}
        \childdoctrue
        \includeonly{#2}
        \def\childdocjob{#1}
        \def\jobname{#1}
        \input{#1}
        \endinput
      }
    \fi
    \expandafter
  \endgroup
  \childdoctmp
}
%    \end{macrocode}

% \macro{\childdocforwardprefix}
% The command |\childdocforwardprefix| redirects
% compilation to the main or a child file by means of a pattern.
% The prefix |#1| in the current filename is replaced by |#2|
% and the suffix of the current filename is kept
% (it is assumed that the filename does not contain the substring `|~~~|'
% which is used as a delimiter).
% Compilation is handed over to the new file by |\childdocforward|:
%    \begin{macrocode}
\newcommand{\childdocforwardprefix}[3][]
{
  \begingroup
    \def\childdocextract #2##1~~~{\def\childdoctmp{\childdocforward[#1]{#3##1}}}
    \expandafter\childdocextract\childdocname~~~
    \expandafter
  \endgroup
  \childdoctmp
}
%    \end{macrocode}

% \macro{\childdoc}
% The deprecated macro |\childdoc| is a legacy version of |\childdocmain|:
%    \begin{macrocode}
\newcommand{\childdoc}{\childdocmain}
%    \end{macrocode}

% \macro{\childdocredirect}
% The deprecated macro |\childdocredirect| is a legacy version
% of |\childdocforward| and |\childdocforwardprefix|:
%    \begin{macrocode}
\newcommand{\childdocredirect}[2][]
{
  \begingroup
    \if?#1?
      \def\childdoctmp{\childdocforward{#2}}
    \else
      \def\childdoctmp{\childdocforwardprefix{#1}{#2}}
    \fi
    \expandafter
  \endgroup
  \childdoctmp
}
%    \end{macrocode}

%\iffalse
%</package>
%\fi
%
\endinput

\childdocforwardprefix[cdocsamp]{cdocsfn}{cdocsch}
%    \end{macrocode}

%\iffalse
%</samplefinal>
%\fi
%
% %%%%%%%%%%%%%%%%%%%%%%%%%%%%%%%%%%%%%%
% \paragraph{Command Line Processing.}
%
% The following three command lines generate the output files
% |cdocscld|, |cdocscl1| and |cdocscl2|
% which should be identical to
% |cdocsdrf|, |cdocsch1| and |cdocsfn2|, respectively:
% \begin{center}
% \begin{tabular}{l}
% |latex -jobname cdocscld \|\\
% |  "\def\version{draft}% \iffalse
%
% childdoc.dtx Copyright (C) 2017-2018 Niklas Beisert
%
% This work may be distributed and/or modified under the
% conditions of the LaTeX Project Public License, either version 1.3
% of this license or (at your option) any later version.
% The latest version of this license is in
%   http://www.latex-project.org/lppl.txt
% and version 1.3 or later is part of all distributions of LaTeX
% version 2005/12/01 or later.
%
% This work has the LPPL maintenance status `maintained'.
%
% The Current Maintainer of this work is Niklas Beisert.
%
% This work consists of the files childdoc.dtx and childdoc.ins
% and the derived files childdoc.def and cdocsamp.tex with
% cdocsch1.tex, cdocsch2.tex, cdocsdrf.tex, cdocsfn1.tex, cdocsfn2.tex.
%
%<package>\ifdefined\childdocmain\endinput\fi
%<package>\ProvidesFile{childdoc.def}[2018/12/30 v2.0 child document driver]
%<samplemain>\ProvidesFile{cdocsamp.tex}[2018/12/30 v2.0 sample for childdoc]
%<*driver>
%\ProvidesFile{childdoc.drv}[2018/12/30 v2.0 childdoc reference manual file]
\PassOptionsToClass{10pt,a4paper}{article}
\documentclass{ltxdoc}

\usepackage[margin=35mm]{geometry}
\usepackage{hyperref}
\usepackage{hyperxmp}
\usepackage[usenames]{color}

\hypersetup{colorlinks=true}
\hypersetup{pdfstartview=FitH}
\hypersetup{pdfpagemode=UseNone}
\hypersetup{pdfsource={}}
\hypersetup{pdflang={en-UK}}
\hypersetup{pdfcopyright={Copyright 2017-2018 Niklas Beisert.
  This work may be distributed and/or modified under the
  conditions of the LaTeX Project Public License, either version 1.3
  of this license or (at your option) any later version.}}
\hypersetup{pdflicenseurl={http://www.latex-project.org/lppl.txt}}
\hypersetup{pdfcontactaddress={ETH Zurich, ITP, HIT K,
  Wolfgang-Pauli-Strasse 27}}
\hypersetup{pdfcontactpostcode={8093}}
\hypersetup{pdfcontactcity={Zurich}}
\hypersetup{pdfcontactcountry={Switzerland}}
\hypersetup{pdfcontactemail={nbeisert@itp.phys.ethz.ch}}
\hypersetup{pdfcontacturl={http://people.phys.ethz.ch/\xmptilde nbeisert/}}

\newcommand{\secref}[1]{\hyperref[#1]{section \ref*{#1}}}

\parskip1ex
\parindent0pt
\let\olditemize\itemize
\def\itemize{\olditemize\parskip0pt}

\begin{document}

\title{The \textsf{childdoc} Package}
\hypersetup{pdftitle={The childdoc Package}}
\author{Niklas Beisert\\[2ex]
  Institut f\"ur Theoretische Physik\\
  Eidgen\"ossische Technische Hochschule Z\"urich\\
  Wolfgang-Pauli-Strasse 27, 8093 Z\"urich, Switzerland\\[1ex]
  \href{mailto:nbeisert@itp.phys.ethz.ch}
  {\texttt{nbeisert@itp.phys.ethz.ch}}}
\hypersetup{pdfauthor={Niklas Beisert}}
\hypersetup{pdfsubject={Manual for the LaTeX2e Package childdoc}}
\date{30 December 2018, \textsf{v2.0}}
\maketitle

\begin{abstract}\noindent
\textsf{childdoc} is a \LaTeXe{} package
that enables the direct compilation
of document sections included by |\include|
to individual files.
\end{abstract}

\begingroup
\parskip0ex
\tableofcontents
\endgroup

%%%%%%%%%%%%%%%%%%%%%%%%%%%%%%%%%%%%%%%%%%%%%%%%%%%%%%%%%%%%%%%%%%%%%%%%%%%%%%%%
%%%%%%%%%%%%%%%%%%%%%%%%%%%%%%%%%%%%%%%%%%%%%%%%%%%%%%%%%%%%%%%%%%%%%%%%%%%%%%%%
\section{Introduction}

\LaTeX{} provides a mechanism to structure a large document (such as a book)
into a main file and several child files (containing the chapters)
using the |\include| command.
This mechanism is beneficial for documents
which span hundreds of pages in order to
make the source file(s) more manageable.
Moreover, compilation can be restricted to
selected child files by means of the |\includeonly| command.
The latter feature can be used to reduce the compilation time while editing
(this was significantly more useful in the earlier days of \LaTeX{})
or to generate a smaller document which is easier to navigate.
Another application of |\includeonly| is to generate
documents consisting of selected parts of the complete document.

However, there are a few drawbacks of the plain |\include| mechanism:
\begin{itemize}
\item
The child files cannot be compiled on their own,
they can only be compiled via the main file.
A naive editing environment
(such as a text editor with an option
to have the current file processed by \LaTeX)
may require one to switch to the main file before compiling;
attempting to compile the child file produces errors.
\item
The main file must be modified (each time)
to adjust the |\includeonly| command
to the present needs. This easily leaves the main file in a messy state.
\item
The generated document will always carry the filename
of the main document. This is inconvenient if
several child files are to be compiled and
to be kept for distribution.
\end{itemize}

The present package provides a simple interface
to make child files individually compilable by \LaTeX{}.
Compiling a child file then has the same effect as compiling
the main file with an |\includeonly| command
to select the appropriate child.
Moreover the generated document will carry the name of the child
rather than the main file.
This resolves all three above issues.

This feature is meant to make the editing of books,
thesis documents and lecture notes somewhat more convenient.
However, the package can also be used efficiently for
composing a series of documents (such as exercise sheets)
which are typically distributed individually.
It then assists the author in generating the individual documents
(potentially in different versions)
as well as a document containing the collected series.
Another application is in developing style files
or other kinds of included material
where compilation of the style file could redirect
to a sample or test file.

%%%%%%%%%%%%%%%%%%%%%%%%%%%%%%%%%%%%%%%%%%%%%%%%%%%%%%%%%%%%%%%%%%%%%%%%%%%%%%%%
%%%%%%%%%%%%%%%%%%%%%%%%%%%%%%%%%%%%%%%%%%%%%%%%%%%%%%%%%%%%%%%%%%%%%%%%%%%%%%%%
\section{Usage}

First of all, the package \textsf{childdoc} is \emph{not} a standard
\LaTeXe{} |.sty| style file! Therefore it needs to be invoked in
a non-standard way.

%%%%%%%%%%%%%%%%%%%%%%%%%%%%%%%%%%%%%%%%%%%%%%%%%%%%%%%%%%%%%%%%%%%%%%%%%%%%%%%%
\subsection{Included Files}
\label{sec:include}

%%%%%%%%%%%%%%%%%%%%%%%%%%%%%%%%%%%%%%%%
\DescribeMacro{\childdocmain}
To use the package, add the commands
\begin{center}
\begin{tabular}{l}
|% \iffalse
%
% childdoc.dtx Copyright (C) 2017-2018 Niklas Beisert
%
% This work may be distributed and/or modified under the
% conditions of the LaTeX Project Public License, either version 1.3
% of this license or (at your option) any later version.
% The latest version of this license is in
%   http://www.latex-project.org/lppl.txt
% and version 1.3 or later is part of all distributions of LaTeX
% version 2005/12/01 or later.
%
% This work has the LPPL maintenance status `maintained'.
%
% The Current Maintainer of this work is Niklas Beisert.
%
% This work consists of the files childdoc.dtx and childdoc.ins
% and the derived files childdoc.def and cdocsamp.tex with
% cdocsch1.tex, cdocsch2.tex, cdocsdrf.tex, cdocsfn1.tex, cdocsfn2.tex.
%
%<package>\ifdefined\childdocmain\endinput\fi
%<package>\ProvidesFile{childdoc.def}[2018/12/30 v2.0 child document driver]
%<samplemain>\ProvidesFile{cdocsamp.tex}[2018/12/30 v2.0 sample for childdoc]
%<*driver>
%\ProvidesFile{childdoc.drv}[2018/12/30 v2.0 childdoc reference manual file]
\PassOptionsToClass{10pt,a4paper}{article}
\documentclass{ltxdoc}

\usepackage[margin=35mm]{geometry}
\usepackage{hyperref}
\usepackage{hyperxmp}
\usepackage[usenames]{color}

\hypersetup{colorlinks=true}
\hypersetup{pdfstartview=FitH}
\hypersetup{pdfpagemode=UseNone}
\hypersetup{pdfsource={}}
\hypersetup{pdflang={en-UK}}
\hypersetup{pdfcopyright={Copyright 2017-2018 Niklas Beisert.
  This work may be distributed and/or modified under the
  conditions of the LaTeX Project Public License, either version 1.3
  of this license or (at your option) any later version.}}
\hypersetup{pdflicenseurl={http://www.latex-project.org/lppl.txt}}
\hypersetup{pdfcontactaddress={ETH Zurich, ITP, HIT K,
  Wolfgang-Pauli-Strasse 27}}
\hypersetup{pdfcontactpostcode={8093}}
\hypersetup{pdfcontactcity={Zurich}}
\hypersetup{pdfcontactcountry={Switzerland}}
\hypersetup{pdfcontactemail={nbeisert@itp.phys.ethz.ch}}
\hypersetup{pdfcontacturl={http://people.phys.ethz.ch/\xmptilde nbeisert/}}

\newcommand{\secref}[1]{\hyperref[#1]{section \ref*{#1}}}

\parskip1ex
\parindent0pt
\let\olditemize\itemize
\def\itemize{\olditemize\parskip0pt}

\begin{document}

\title{The \textsf{childdoc} Package}
\hypersetup{pdftitle={The childdoc Package}}
\author{Niklas Beisert\\[2ex]
  Institut f\"ur Theoretische Physik\\
  Eidgen\"ossische Technische Hochschule Z\"urich\\
  Wolfgang-Pauli-Strasse 27, 8093 Z\"urich, Switzerland\\[1ex]
  \href{mailto:nbeisert@itp.phys.ethz.ch}
  {\texttt{nbeisert@itp.phys.ethz.ch}}}
\hypersetup{pdfauthor={Niklas Beisert}}
\hypersetup{pdfsubject={Manual for the LaTeX2e Package childdoc}}
\date{30 December 2018, \textsf{v2.0}}
\maketitle

\begin{abstract}\noindent
\textsf{childdoc} is a \LaTeXe{} package
that enables the direct compilation
of document sections included by |\include|
to individual files.
\end{abstract}

\begingroup
\parskip0ex
\tableofcontents
\endgroup

%%%%%%%%%%%%%%%%%%%%%%%%%%%%%%%%%%%%%%%%%%%%%%%%%%%%%%%%%%%%%%%%%%%%%%%%%%%%%%%%
%%%%%%%%%%%%%%%%%%%%%%%%%%%%%%%%%%%%%%%%%%%%%%%%%%%%%%%%%%%%%%%%%%%%%%%%%%%%%%%%
\section{Introduction}

\LaTeX{} provides a mechanism to structure a large document (such as a book)
into a main file and several child files (containing the chapters)
using the |\include| command.
This mechanism is beneficial for documents
which span hundreds of pages in order to
make the source file(s) more manageable.
Moreover, compilation can be restricted to
selected child files by means of the |\includeonly| command.
The latter feature can be used to reduce the compilation time while editing
(this was significantly more useful in the earlier days of \LaTeX{})
or to generate a smaller document which is easier to navigate.
Another application of |\includeonly| is to generate
documents consisting of selected parts of the complete document.

However, there are a few drawbacks of the plain |\include| mechanism:
\begin{itemize}
\item
The child files cannot be compiled on their own,
they can only be compiled via the main file.
A naive editing environment
(such as a text editor with an option
to have the current file processed by \LaTeX)
may require one to switch to the main file before compiling;
attempting to compile the child file produces errors.
\item
The main file must be modified (each time)
to adjust the |\includeonly| command
to the present needs. This easily leaves the main file in a messy state.
\item
The generated document will always carry the filename
of the main document. This is inconvenient if
several child files are to be compiled and
to be kept for distribution.
\end{itemize}

The present package provides a simple interface
to make child files individually compilable by \LaTeX{}.
Compiling a child file then has the same effect as compiling
the main file with an |\includeonly| command
to select the appropriate child.
Moreover the generated document will carry the name of the child
rather than the main file.
This resolves all three above issues.

This feature is meant to make the editing of books,
thesis documents and lecture notes somewhat more convenient.
However, the package can also be used efficiently for
composing a series of documents (such as exercise sheets)
which are typically distributed individually.
It then assists the author in generating the individual documents
(potentially in different versions)
as well as a document containing the collected series.
Another application is in developing style files
or other kinds of included material
where compilation of the style file could redirect
to a sample or test file.

%%%%%%%%%%%%%%%%%%%%%%%%%%%%%%%%%%%%%%%%%%%%%%%%%%%%%%%%%%%%%%%%%%%%%%%%%%%%%%%%
%%%%%%%%%%%%%%%%%%%%%%%%%%%%%%%%%%%%%%%%%%%%%%%%%%%%%%%%%%%%%%%%%%%%%%%%%%%%%%%%
\section{Usage}

First of all, the package \textsf{childdoc} is \emph{not} a standard
\LaTeXe{} |.sty| style file! Therefore it needs to be invoked in
a non-standard way.

%%%%%%%%%%%%%%%%%%%%%%%%%%%%%%%%%%%%%%%%%%%%%%%%%%%%%%%%%%%%%%%%%%%%%%%%%%%%%%%%
\subsection{Included Files}
\label{sec:include}

%%%%%%%%%%%%%%%%%%%%%%%%%%%%%%%%%%%%%%%%
\DescribeMacro{\childdocmain}
To use the package, add the commands
\begin{center}
\begin{tabular}{l}
|\input{childdoc.def}|\\
|\childdocmain{}|\\
\end{tabular}
\end{center}
at the very top of the main \LaTeX{} file,
in particular \emph{before} the |\documentclass| statement!
The argument of |\childdocmain| should be left empty
(but it must be present).

%%%%%%%%%%%%%%%%%%%%%%%%%%%%%%%%%%%%%%%%
\DescribeMacro{\childdocof}
Furthermore, add the commands
\begin{center}
\begin{tabular}{l}
|\input{childdoc.def}|\\
|\childdocof{|\textit{main}|}|\\
\end{tabular}
\end{center}
at the top of every child file \textit{child}
which is included by |\include{|\textit{child}|}|
from within the main file
(or at least for those files to be compiled individually).
The argument \textit{main} must be the filename of the main file.

There are a couple of
considerations in setting up the main and child documents:

%%%%%%%%%%%%%%%%%%%%%%%%%%%%%%%%%%%%%%%%
\paragraph{Restrictions.}

Please note the following restrictions:
\begin{itemize}
\item
|\childdocmain| must be called with one argument \textit{main}
to ensure compatibility with earlier version of the package.
It must either be empty (|\childdocmain{}|)
or precisely match the filename of the main file in which it is specified.
See \secref{sec:detection} for further information.
\item
The filename \textit{main} must be specified without the |.tex| extension.
\item
The filename \textit{main} is case sensitive
(even in case-insensitive file systems)
due to internal string comparison.
\item
The argument \textit{main} should be fully expanded, it cannot be a macro.
\item
Subdirectories and special characters should be avoided in filenames.
\item
The command |\childdocmain{|\textit{main}|}| must be followed by a whitespace.
It should not be followed immediately by another command
or by a comment mark `|%|'.
This is because the \TeX{} parser reads the token immediately following
the argument of |\childdocmain| and puts it
at the beginning of every child section;
however, a white\-space is ignored.
\end{itemize}

%%%%%%%%%%%%%%%%%%%%%%%%%%%%%%%%%%%%%%%%
\paragraph{Content of Main File.}

It is advisable to place all content in the child files included by |\include|.
Any output contained in the main file will appear in all child documents
unless suppressed manually;
it cannot be suppressed automatically by the |\includeonly| directive
and thus should normally be avoided.
A method to include some content in the main file
by means of conditional processing is described in \secref{sec:conditional}.

%%%%%%%%%%%%%%%%%%%%%%%%%%%%%%%%%%%%%%%%
\paragraph{Page Numbering.}

When only a part of the document is compiled,
the appropriate numbering of pages
(as well as other status parameters)
is determined from the |.aux| files.
The latter contain information from previous passes.
However this information needs to propagate through
all intermediate child documents.
Therefore the page numbering in child documents may well
be inconsistent until the complete document is compiled at least once.

A useful (if unconventional) way to always ensure a consistent
page numbering is to restart the numbering in each child document
and denote the pages by `\textit{child}|.|\textit{page}'
where \textit{child} represents the chapter/section number of the child file.
This can be achieved by the command
|\numberwithin{page}{|\textit{child}|}|
of the \textsf{amsmath} package
where \textit{child} can be |chapter| or |section|
depending on the chosen structuring.
Alternatively, one can modify the macro |\thepage| appropriately
and reset the counter |page| at the start of each child file.

%%%%%%%%%%%%%%%%%%%%%%%%%%%%%%%%%%%%%%%%%%%%%%%%%%%%%%%%%%%%%%%%%%%%%%%%%%%%%%%%
\subsection{Conditional Processing}
\label{sec:conditional}

The package provides a mechanism to compile different versions
of a document. To customise the versions further some conditional processing
can come in handy to distinguish which version is being compiled.
The package provides two macros to describe the compilation context:

%%%%%%%%%%%%%%%%%%%%%%%%%%%%%%%%%%%%%%%%
\DescribeMacro{\ifchilddoc}
The conditional |\ifchilddoc| distinguishes between the compilation of
child documents and the main document:
%
\begin{center}
|\ifchilddoc |\textit{child-code}| |[|\||else |\textit{main-code}]| \||fi|
\end{center}

%%%%%%%%%%%%%%%%%%%%%%%%%%%%%%%%%%%%%%%%
\DescribeMacro{\childdocname}
\DescribeMacro{\childdocjob}
The macro |\childdocname| contains the filename (without extension)
of the main or child file being processed.
Note that |\childdocjob| will always contain the name of the main file.

%%%%%%%%%%%%%%%%%%%%%%%%%%%%%%%%%%%%%%%%
\paragraph{Title Page.}

Conditional processing can be used to include a title or banner page
in the main document when proper precautions are taken.
Importantly, the code in the main file should ensure that the page counter
(as well as other status parameters which are stored in the |.aux| files)
takes the same value after the conditional processing.
Otherwise the page numbers may take divergent values
depending on which part is compiled.

For example, a title page could be declared by:
%
\begin{center}
\begin{tabular}{l}
|\ifchilddoc\||else|\\
|\addtocounter{page}{-1}|\\
\textit{code for title page}\\
|\newpage|\\
|\||fi|
\end{tabular}
\end{center}
%
A banner page for the child documents can be generated by:
%
\begin{center}
\begin{tabular}{l}
|\ifchilddoc|\\
|\addtocounter{page}{-1}|\\
\textit{code for banner page}\\
|\newpage|\\
|\||fi|
\end{tabular}
\end{center}
%
Here one could write a message such as:
\begin{center}
|This is the part \childdocname{} of \childdocjob{}.|
\end{center}

%%%%%%%%%%%%%%%%%%%%%%%%%%%%%%%%%%%%%%%%%%%%%%%%%%%%%%%%%%%%%%%%%%%%%%%%%%%%%%%%
\subsection{Flags}
\label{sec:flags}

The package makes it easy to generate different versions
of the main or child documents.
To this end compilation flags can be defined
and assigned different default values.
They will be particularly useful in conjunction
with the forwarding mechanism described in \secref{sec:forward}.

For example, it may be useful to have a flag |\version|
which can be set to |draft| or |final|.
The document source will contain some conditional code
depending on the value of |\version|.
Suppose further, the flag should default to |final| for the main file
and to |draft| for child files
which is a natural assignment for editing the document.
This is achieved by placing the following code
in the preamble of the main document
(below the |\childdocmain| directive):
%
\begin{center}
\begin{tabular}{l}
|\ifchilddoc|\\
|\providecommand{\version}{draft}|\\
|\||else|\\
|\providecommand{\version}{final}|\\
|\||fi|
\end{tabular}
\end{center}
%
The definition by |\providecommand| makes sure
that previous definitions are not overwritten.
Further statements |\providecommand{\version}{...}|
can thus be added before the above code to override it.

For the main file, one might add a line
(between |\childdocmain| and the above block)
%
\begin{center}
|%\ifchilddoc\||else\providecommand{\version}{draft}\||fi|
\end{center}
%
which can be uncommented to produce a draft version.
Likewise one can add a line to the very top of a child file
(above the |\childdocof{|\textit{main}|}| directive)
%
\begin{center}
|%\providecommand{\version}{final}|
\end{center}
%
which can be uncommented to produce the final version of this child document.

%%%%%%%%%%%%%%%%%%%%%%%%%%%%%%%%%%%%%%%%%%%%%%%%%%%%%%%%%%%%%%%%%%%%%%%%%%%%%%%%
\subsection{Forwarding}
\label{sec:forward}

Different versions of the main or child documents
using compilation flags as described in \secref{sec:flags}
can be (permanently) stored in different files
for convenient compilation, viewing and distribution.
To this end, the package defines a command
to pass on compilation to a different file:

%%%%%%%%%%%%%%%%%%%%%%%%%%%%%%%%%%%%%%%%
\DescribeMacro{\childdocforward}
The command |\childdocforward| redirects processing to
another source file:
%
\begin{center}
\begin{tabular}{l}
|\input{childdoc.def}|\\
|\childdocforward[|\textit{main}|]{|\textit{dest}|}|\\
\end{tabular}
\end{center}
%
The argument \textit{dest} is the destination file
(without extension).
It should be the main file or one of the child files.
Note that further \textsf{childdoc} directives
such as |\childdocof| and |\childdocforward|
in the indicated file will be processed in this form.
The optional argument \textit{main}
passes on directly to the main file \textit{main}
while pretending to compile the child \textit{dest}.
This form behaves as if \textit{dest}
issues |\childdocof{|\textit{main}|}| right away,
and no further \textsf{childdoc} directives will be processed.

%%%%%%%%%%%%%%%%%%%%%%%%%%%%%%%%%%%%%%%%
\DescribeMacro{\...prefix}
In the alternative form |\childdocforwardprefix|,
%
\begin{center}
\begin{tabular}{l}
|\input{childdoc.def}|\\
|\childdocforwardprefix[|\textit{main}|]{|\textit{prefix}|}{|\textit{dest}|}|
\end{tabular}
\end{center}
%
the destination file is determined by a pattern
depending on the current file:
To make this work, the current file must be called
`{\textit{prefix}\hspace{0.2em}\textit{suffix}}'
with \textit{prefix} matching precisely the argument.
Processing is then passed on to the file
`{\textit{dest}\hspace{0.2em}\textit{suffix}}'.
Surely, the same effect is achieved by
directly specifying the
argument `{\textit{dest}\hspace{0.2em}\textit{suffix}}'
in the first form.
However, that requires to set up a different file
for each child. With the alternative form of the command
all these files can have exactly the same content
which simplifies setting them up and maintaining them.

For example, the following file |draft.tex|
with a compilation flag |\version| as described in \secref{sec:flags}
compiles the main document as a draft:
%
\begin{center}
\begin{tabular}{l}
|\def\version{draft}|\\
|\input{childdoc.def}|\\
|\childdocforward{|\textit{main}|}|
\end{tabular}
\end{center}
%
Likewise, the following files |final|\textit{nn}|.tex|
compile the final version of the child document
|child|\textit{nn}|.tex|:
%
\begin{center}
\begin{tabular}{l}
|\def\version{final}|\\
|\input{childdoc.def}|\\
|\childdocforwardprefix{final}{child}|
\end{tabular}
\end{center}
%

Note that when several versions of a main file and/or of each child file
are to be generated, it may be convenient to set up a |Makefile| or
shell script to automatise the process.

%%%%%%%%%%%%%%%%%%%%%%%%%%%%%%%%%%%%%%%%%%%%%%%%%%%%%%%%%%%%%%%%%%%%%%%%%%%%%%%%
\subsection{Command Line Processing}
\label{sec:commandline}

The effect of redirection files can also be achieved by invoking
the \LaTeX{} compiler with a more elaborate command line.
Most conveniently this should be done as part
of a shell script or a |Makefile|.

When using \textsf{childdoc} in the main file, the following
command lines effectively perform a redirection
(note that depending on the shell being used,
backslashes may have to be doubled: `|\|' $\to$ `|\\|'):
%
\begin{center}
|... -jobname "|\textit{target}|" |\\|"|[\textit{flags}]%
|\input{childdoc.def}\childdocforward[|\textit{main}|]{|\textit{dest}|}"|
\end{center}
%
Here \textit{target} is the name of the output file,
\textit{main} is the name of the main file
and \textit{dest} is the name of the main or child file to be processed
(all filenames without extensions).
The optional argument \textit{main} can be omitted
if \textit{main} matches \textit{dest}.
Optionally, compilation \textit{flags} can be defined via |\def| commands.
This command line makes the \TeX{} engine believe
it is compiling the file \textit{target}
whose content is specified as the latter parameter.
The provided code then forwards the processing to
\textit{main} or \textit{dest} as described in \secref{sec:forward}.

%%%%%%%%%%%%%%%%%%%%%%%%%%%%%%%%%%%%%%%%%%%%%%%%%%%%%%%%%%%%%%%%%%%%%%%%%%%%%%%%
\subsection{Include by Input}
\label{sec:input}

Including child documents by |\include| has some restrictions by design.
Most notably, the content of a child document always occupies
its own set of pages; pages cannot be shared between child documents.
Usually, this behaviour makes perfect sense
because each child document contain an essential part of the document.
However, in some situations it may be desirable to compose
a document from a collection of parts
without having mandatory page breaks between then.
For this case, the package
provides a mechanism to include parts
by |\input| which can also be processed individually.
However, by construction this mechanism
requires manual handling of the content to be output.

%%%%%%%%%%%%%%%%%%%%%%%%%%%%%%%%%%%%%%%%
\DescribeMacro{\ifchilddocmanual}
The main file should be prepared as usual, see \secref{sec:include}.
However, the document body must make a distinction
between processing of an individual part and of the main document, e.g.:
%
\begin{center}
\begin{tabular}{l}
|\ifchilddocmanual|\\
|\input{\childdocname}|\\
|\||else|\\
\textit{document body with }|\input{|\textit{part}|}|\\
|\||fi|
\end{tabular}
\end{center}
%
The conditional |\ifchilddocmanual| is true whenever
a part to be included by |\input| is being compiled,
and the name of the part is stored in |\childdocname|.

%%%%%%%%%%%%%%%%%%%%%%%%%%%%%%%%%%%%%%%%
\DescribeMacro{\childdocby}
Each part to be included by |\input| should start with:
%
\begin{center}
\begin{tabular}{l}
|\input{childdoc.def}|\\
|\childdocby{|\textit{main}|}|\\
\end{tabular}
\end{center}
%
The directive |\childdocby| is similar to |\childdocof|
described in \secref{sec:include},
but the subsequent selection of content must be done manually.
To that end, both |\ifchilddoc| and |\ifchilddocmanual|
will be true upon processing of a part,
and the name of the part is stored in |\childdocname|.
Note that |\jobname| will be set to the filename of the current part
so that each part receives an individual |.aux| file
that does not interfere with the |.aux| file(s) of the main document.
This behaviour can be altered by the alternative form
|\childdocby[*]{|\textit{main}|}| (with a non-empty optional argument)
which uses the |.aux| file of the main document
by setting |\jobname| to \textit{main}.

%%%%%%%%%%%%%%%%%%%%%%%%%%%%%%%%%%%%%%%%%%%%%%%%%%%%%%%%%%%%%%%%%%%%%%%%%%%%%%%%
\subsection{Driver Development}
\label{sec:driver}

The \textsf{childdoc} mechanism can also be use for the development
of definition files such as \LaTeX{} styles or classes.
This case differs from the above setup with multiple parts
included by |\include| in that no |\includeonly| should be invoked.
This can be achieved by starting the include file
(before |\ProvidesPackage|) with:
%
\begin{center}
\begin{tabular}{l}
|\input{childdoc.def}|\\
|\childdocforward{|\textit{main}|}|\\
\end{tabular}
\end{center}
%
or alternatively with:
%
\begin{center}
\begin{tabular}{l}
|\input{childdoc.def}|\\
|\childdocby{|\textit{main}|}|\\
\end{tabular}
\end{center}
%
Both forms have slightly different effects as described above.
The main file is prepared as usual, see \secref{sec:include}.

%%%%%%%%%%%%%%%%%%%%%%%%%%%%%%%%%%%%%%%%%%%%%%%%%%%%%%%%%%%%%%%%%%%%%%%%%%%%%%%%
\subsection{Legacy Detection}
\label{sec:detection}

The directive |\childdocmain| in the main file can detect
whether the complete document or merely a child is to be compiled
even without using the directive |\childdocof|.
This method is deprecated because it is less robust
and there is no compelling reason to use it;
it is merely provided for backward compatibility
and it may be removed in future versions.

If the detection mechanism is to be used,
it is mandatory to correctly specify
the filename of the main file as the argument of |\childdocmain|:
%
\begin{center}
\begin{tabular}{l}
|\input{childdoc.def}|\\
|\childdocmain{|\textit{main}|}|\\
\end{tabular}
\end{center}
%
If |\jobname| does not match the argument \textit{main} of |\childdocmain|,
it is assumed that |\jobname| points to the child file to be compiled.
When using |\childdocmain| with the main file specified as argument,
it suffices to start a child file
with just |\input{|\textit{main}|}|
without loading of the package and using |\childdocof|.
If instead all processing is done
with the appropriate \textsf{childdoc} directives,
the argument of \textit{main} of |\childdocmain| can be empty.

An alternative version of the command line processing described
in \secref{sec:commandline} using the detection mechanism reads:
%
\begin{center}
|... -jobname "|\textit{target}|" "|[\textit{flags}]%
[|\def\jobname{|\textit{dest}|}|]|\input{|\textit{main}|}"|
\end{center}

%%%%%%%%%%%%%%%%%%%%%%%%%%%%%%%%%%%%%%%%%%%%%%%%%%%%%%%%%%%%%%%%%%%%%%%%%%%%%%%%
\subsection{Manual Code}
\label{sec:manual}

In case one cannot be certain whether the definitions file |childdoc.def|
is installed on the target \TeX{} distribution
and one prefers not to ship it,
it is conceivable to paste a few relevant commands into the sources.

To that end, drop all statements |\input{childdoc.def}|
and perform the replacements as outlined below.
Instead of |\childdocmain{|\textit{main}|}| add the following code
to the top of the main file:
%
\begin{center}
\begin{tabular}{l}
|\||ifdefined\childdocname\endinput\||fi\newif\ifchilddoc|\\
|\edef\childdocname{\scantokens\expandafter{\jobname\noexpand}}|\\
|\def\childdocmain{|\textit{main}|}\||ifx\childdocmain\childdocname\||else|\\
|\childdoctrue\includeonly{\childdocname}\let\jobname\childdocmain\||fi|\\
\end{tabular}
\end{center}
%
Instead of |\childdocof{|\textit{main}|}| just include the main file
at the top of each child file:
%
\begin{center}
|\input{|\textit{main}|}|
\end{center}
%
A simple redirection |\childdocforward{|\textit{dest}|}| is achieved by:
%
\begin{center}
|\def\jobname{|\textit{dest}|}\input{\jobname}|
\end{center}
%
The redirection with prefix
|\childdocforwardprefix[|\textit{prefix}|]{|\textit{dest}|}|
is accomplished by:
%
\begin{center}
\begin{tabular}{l}
|{\edef\jobname{\scantokens\expandafter{\jobname\noexpand}}|\\
|\def\redirectjob |\textit{prefix}|#1~~~{\gdef\jobname{|\textit{dest}|#1}}|\\
|\expandafter\redirectjob\jobname~~~}\input{\jobname}|
\end{tabular}
\end{center}

In an alternative approach,
child documents can be compiled by a specific command line
without additional code or specific definitions:
%
\begin{center}
|... -jobname "|\textit{target}|" "|[\textit{flags}]%
|\includeonly{|\textit{dest}|}\input{|\textit{main}|}"|
\end{center}
%

%%%%%%%%%%%%%%%%%%%%%%%%%%%%%%%%%%%%%%%%%%%%%%%%%%%%%%%%%%%%%%%%%%%%%%%%%%%%%%%%
%%%%%%%%%%%%%%%%%%%%%%%%%%%%%%%%%%%%%%%%%%%%%%%%%%%%%%%%%%%%%%%%%%%%%%%%%%%%%%%%
\section{Information}

%%%%%%%%%%%%%%%%%%%%%%%%%%%%%%%%%%%%%%%%%%%%%%%%%%%%%%%%%%%%%%%%%%%%%%%%%%%%%%%%
\subsection{Copyright}

Copyright \copyright{} 2017--2018 Niklas Beisert

This work may be distributed and/or modified under the
conditions of the \LaTeX{} Project Public License, either version 1.3
of this license or (at your option) any later version.
The latest version of this license is in
  \url{http://www.latex-project.org/lppl.txt}
and version 1.3 or later is part of all distributions of \LaTeX{}
version 2005/12/01 or later.

This work has the LPPL maintenance status `maintained'.

The Current Maintainer of this work is Niklas Beisert.

This work consists of the files |README.txt|, |childdoc.ins| and |childdoc.dtx|
as well as the derived files |childdoc.def|, |cdocsamp.tex|
with |cdocsch1.tex|, |cdocsch2.tex|, |cdocspt3.tex|, |cdocspt4.tex|,
|cdocsdrf.tex|, |cdocsfn1.tex|, |cdocsfn2.tex|
as well as |childdoc.pdf|.

%%%%%%%%%%%%%%%%%%%%%%%%%%%%%%%%%%%%%%%%%%%%%%%%%%%%%%%%%%%%%%%%%%%%%%%%%%%%%%%%
\subsection{Files and Installation}

The package consists of the files:
%
\begin{center}
\begin{tabular}{ll}
    |README.txt|   & readme file \\
    |childdoc.ins| & installation file \\
    |childdoc.dtx| & source file \\
    |childdoc.def| & definition file \\
    |cdocsamp.tex| & sample main file \\
    |cdocsch1.tex| & sample include file \\
    |cdocsch2.tex| & sample include file \\
    |cdocspt3.tex| & sample part file \\
    |cdocspt4.tex| & sample part file \\
    |cdocsdrf.tex| & sample redirection file \\
    |cdocsfn1.tex| & sample redirection file \\
    |cdocsfn2.tex| & sample redirection file \\
    |childdoc.pdf| & manual
\end{tabular}
\end{center}
%
The distribution consists of the files
|README.txt|, |childdoc.ins| and |childdoc.dtx|.
%
\begin{itemize}
\item
Run (pdf)\LaTeX{} on |childdoc.dtx|
to compile the manual |childdoc.pdf| (this file).
\item
Run \LaTeX{} on |childdoc.ins| to create the definitions file |childdoc.def|
and the sample |cdocsamp.tex| with include files
|cdocsch1.tex|, |cdocsch2.tex|, |cdocspt3.tex|, |cdocspt4.tex|,
|cdocsdrf.tex|, |cdocsfn1.tex|, |cdocsfn2.tex|.
Then copy the file |childdoc.def| to an appropriate directory of your \LaTeX{}
distribution, e.g.\ \textit{texmf-root}|/tex/latex/childdoc|.
\end{itemize}

%%%%%%%%%%%%%%%%%%%%%%%%%%%%%%%%%%%%%%%%%%%%%%%%%%%%%%%%%%%%%%%%%%%%%%%%%%%%%%%%
\subsection{Related CTAN Packages}

There are several other packages which offer a similar functionality:
%
\begin{itemize}
\item
The packages
\href{http://ctan.org/pkg/docmute}{\textsf{docmute}},
\href{http://ctan.org/pkg/includex}{\textsf{includex}} and
\href{http://ctan.org/pkg/standalone}{\textsf{standalone}}
provide commands to include only the document body of
a child file thus allowing both files to be compiled individually.
\item
The packages \href{http://ctan.org/pkg/subdocs}{\textsf{subdocs}}
and \href{http://ctan.org/pkg/subfiles}{\textsf{subfiles}}
provide structures in which the main and child documents can be
encapsulated and allowing them to be compiled individually.
The inclusion mechanism is different from the conventional |\include|.
\item
The package \href{http://ctan.org/pkg/combine}{\textsf{combine}}
is an elaborate solution to combine several documents into one.
\end{itemize}
%
See also the CTAN topic \href{http://ctan.org/topic/subdocs}{\textsf{subdocs}}
for further related packages.
The present package differs from the above solutions in that
a document structure constructed with the conventional |\include| mechanism
just needs two extra commands at the top of every file
such that all constituent files can be compiled individually.

%%%%%%%%%%%%%%%%%%%%%%%%%%%%%%%%%%%%%%%%%%%%%%%%%%%%%%%%%%%%%%%%%%%%%%%%%%%%%%%%
%\subsection{Feature Suggestions}
%
%The following is a list of features which may be useful for future
%versions of this package:
%%
%\begin{itemize}
%\item
%\ldots
%\end{itemize}

%%%%%%%%%%%%%%%%%%%%%%%%%%%%%%%%%%%%%%%%%%%%%%%%%%%%%%%%%%%%%%%%%%%%%%%%%%%%%%%%
\subsection{Revision History}

%%%%%%%%%%%%%%%%%%%%%%%%%%%%%%%%%%%%%%%%
\paragraph{v2.0:} 2018/12/30

\begin{itemize}
\item
immediate forward processing
\item
added |\childdocby| mechanism
\item
manual restructured
\end{itemize}

%%%%%%%%%%%%%%%%%%%%%%%%%%%%%%%%%%%%%%%%
\paragraph{v1.6:} 2018/01/17

\begin{itemize}
\item
application for development of include files
\item
corrections to manual
\end{itemize}

%%%%%%%%%%%%%%%%%%%%%%%%%%%%%%%%%%%%%%%%
\paragraph{v1.5:} 2017/05/21

\begin{itemize}
\item
more complete structuring introduced
\item
|\childdocof| introduced
\item
|\childdoc| renamed to |\childdocmain|
\item
|\childredirect| renamed to |\childdocforward| and |\childdocforwardprefix|
and functionality expanded
\end{itemize}

%%%%%%%%%%%%%%%%%%%%%%%%%%%%%%%%%%%%%%%%
\paragraph{v1.0:} 2017/04/27

\begin{itemize}
\item
manual and install package
\item
first version published on CTAN
\end{itemize}

%%%%%%%%%%%%%%%%%%%%%%%%%%%%%%%%%%%%%%%%
\paragraph{v0.6:} 2017/04/26

\begin{itemize}
\item
redirection mechanism added
\end{itemize}

%%%%%%%%%%%%%%%%%%%%%%%%%%%%%%%%%%%%%%%%
\paragraph{v0.5:} 2017/04/26

\begin{itemize}
\item
functionality in definition file
\end{itemize}


%%%%%%%%%%%%%%%%%%%%%%%%%%%%%%%%%%%%%%%%%%%%%%%%%%%%%%%%%%%%%%%%%%%%%%%%%%%%%%%%
%%%%%%%%%%%%%%%%%%%%%%%%%%%%%%%%%%%%%%%%%%%%%%%%%%%%%%%%%%%%%%%%%%%%%%%%%%%%%%%%
%%%%%%%%%%%%%%%%%%%%%%%%%%%%%%%%%%%%%%%%%%%%%%%%%%%%%%%%%%%%%%%%%%%%%%%%%%%%%%%%
\appendix

\settowidth\MacroIndent{\rmfamily\scriptsize 000\ }

 \DocInput{childdoc.dtx}

\end{document}
%</driver>
% \fi
%
% %%%%%%%%%%%%%%%%%%%%%%%%%%%%%%%%%%%%%%%%%%%%%%%%%%%%%%%%%%%%%%%%%%%%%%%%%%%%%%
% %%%%%%%%%%%%%%%%%%%%%%%%%%%%%%%%%%%%%%%%%%%%%%%%%%%%%%%%%%%%%%%%%%%%%%%%%%%%%%
% \section{Sample}
%\iffalse
%<*samplemain>
%\fi
%
% The following presents a sample document
% with two chapters, two parts, a title page,
% a compile flag as well as three forwarding files to set the flag.
% It consists of eight |.tex| files:
% \begin{center}
% \begin{tabular}{ll}
% |cdocsamp.tex|&main file\\
% |cdocsch1.tex|&include file for chapter 1\\
% |cdocsch2.tex|&include file for chapter 2\\
% |cdocspt3.tex|&include file for part 3\\
% |cdocspt4.tex|&include file for part 4\\
% |cdocsdrf.tex|&forwarding file for main file in draft mode\\
% |cdocsfi1.tex|&forwarding file for final version of chapter 1\\
% |cdocsfi2.tex|&forwarding file for final version of chapter 2\\
% \end{tabular}
% \end{center}
% Each of the eight files can be compiled directly by the \LaTeX{} compiler.
%
% %%%%%%%%%%%%%%%%%%%%%%%%%%%%%%%%%%%%%%
% \paragraph{Main File.}
%
% The main file is called |cdocsamp.tex|.
%
% Load the \textsf{childdoc} definitions and
% declare the filename for the main document:
%    \begin{macrocode}
\input{childdoc.def}
\childdocmain{}
%    \end{macrocode}

% Optional override for |\version| flag:
%    \begin{macrocode}
%%\ifchilddoc\else\providecommand{\version}{draft}\fi
%    \end{macrocode}

% Define the default values for the |\version| flag
% (|final| for the main file and |draft| for childs):
%    \begin{macrocode}
\ifchilddoc
\providecommand{\version}{draft}
\else
\providecommand{\version}{final}
\fi
%    \end{macrocode}

% Load the standard document class:
%    \begin{macrocode}
\documentclass[12pt]{article}
%    \end{macrocode}

% Start the document body:
%    \begin{macrocode}
\begin{document}
%    \end{macrocode}

% Declare a title page.
% Print title, part of document being processed and version flag:
%    \begin{macrocode}
\addtocounter{page}{-1}
\begin{center}
{\LARGE\bfseries{}childdoc example\par}
\vspace{1cm}
\ifchilddoc
\ifchilddocmanual part\else chapter\fi:
`\childdocname' of `\childdocjob'\par
\else
main document: `\childdocjob'\par
\fi
version: \version\par
\end{center}
\newpage
%    \end{macrocode}

% Manually include selected file,
% otherwise process as usual:
%    \begin{macrocode}
\ifchilddocmanual
\section*{part `\childdocname'}
\input{\childdocname}
\else
%    \end{macrocode}

% Include the two chapters:
%    \begin{macrocode}
\include{cdocsch1}
\include{cdocsch2}
%    \end{macrocode}

% Include the two parts unless only chapters should be displayed:
%    \begin{macrocode}
\ifchilddoc\else
\section{part three}
\input{cdocspt3}
\section{part four}
\input{cdocspt4}
\fi
%    \end{macrocode}

% Process as usual until here:
%    \begin{macrocode}
\fi
%    \end{macrocode}

% End of document body:
%    \begin{macrocode}
\end{document}
%    \end{macrocode}
%\iffalse
%</samplemain>
%\fi
%
% %%%%%%%%%%%%%%%%%%%%%%%%%%%%%%%%%%%%%%
% \paragraph{Chapter Include Files.}
%
% The include files are called |cdocsch1.tex| and |cdocsch2.tex|.
%
%\iffalse
%<*samplechap1|samplechap2>
%\fi

% Optional override for |\version| flag:
%    \begin{macrocode}
%%\providecommand{\version}{final}
%    \end{macrocode}

% Include the main document:
%    \begin{macrocode}
\input{childdoc.def}
\childdocof{cdocsamp}
%    \end{macrocode}

%\iffalse
%</samplechap1|samplechap2>
%\fi
%
%\iffalse
%<*samplechap1>
%\fi
% Some text for chapter 1:
%    \begin{macrocode}
\section{one}
some text in chapter one
%    \end{macrocode}

%\iffalse
%</samplechap1>
%\fi
% Some text for chapter 2:
%\iffalse
%<*samplechap2>
%\fi
%    \begin{macrocode}
\section{two}
more text in chapter two
%    \end{macrocode}

%\iffalse
%</samplechap2>
%\fi
%
% %%%%%%%%%%%%%%%%%%%%%%%%%%%%%%%%%%%%%%
% \paragraph{Part Include Files.}
%
% The include files are called |cdocspt3.tex| and |cdocspt4.tex|.
%
%\iffalse
%<*samplepart3|samplepart4>
%\fi

% Optional override for |\version| flag:
%    \begin{macrocode}
%%\providecommand{\version}{final}
%    \end{macrocode}

% Include the main document:
%    \begin{macrocode}
\input{childdoc.def}
\childdocby{cdocsamp}
%    \end{macrocode}

%\iffalse
%</samplepart3|samplepart4>
%\fi
%
%\iffalse
%<*samplepart3>
%\fi
% Some text for part 3:
%    \begin{macrocode}
some text in part three
%    \end{macrocode}

%\iffalse
%</samplepart3>
%\fi
% Some text for part 4:
%\iffalse
%<*samplepart4>
%\fi
%    \begin{macrocode}
more text in part four
%    \end{macrocode}

%\iffalse
%</samplepart4>
%\fi
%
% %%%%%%%%%%%%%%%%%%%%%%%%%%%%%%%%%%%%%%
% \paragraph{Forwarding for a Complete Draft.}
%
% The following forwarding file |cdocsdrf.tex|
% compiles the main document in draft mode:
%\iffalse
%<*sampledraft>
%\fi
%    \begin{macrocode}
\def\version{draft}
\input{childdoc.def}
\childdocforward{cdocsamp}
%    \end{macrocode}

%\iffalse
%</sampledraft>
%\fi
%
% %%%%%%%%%%%%%%%%%%%%%%%%%%%%%%%%%%%%%%
% \paragraph{Forwarding for Final Version of the Chapters.}
%
% The following forwarding files |cdocsfn1.tex| and |cdocsfn2.tex|
% (with identical content)
% compile the final versions of the child documents
% |cdocsch1.tex| and |cdocsch2.tex|, respectively:
%\iffalse
%<*samplefinal>
%\fi
%    \begin{macrocode}
\def\version{final}
\input{childdoc.def}
\childdocforwardprefix[cdocsamp]{cdocsfn}{cdocsch}
%    \end{macrocode}

%\iffalse
%</samplefinal>
%\fi
%
% %%%%%%%%%%%%%%%%%%%%%%%%%%%%%%%%%%%%%%
% \paragraph{Command Line Processing.}
%
% The following three command lines generate the output files
% |cdocscld|, |cdocscl1| and |cdocscl2|
% which should be identical to
% |cdocsdrf|, |cdocsch1| and |cdocsfn2|, respectively:
% \begin{center}
% \begin{tabular}{l}
% |latex -jobname cdocscld \|\\
% |  "\def\version{draft}\input{childdoc.def}\childdocforward{cdocsamp}"|\\
% |latex -jobname cdocscl1 \|\\
% |  "\input{childdoc.def}\childdocforward[cdocsamp]{cdocsch1}"|\\
% |latex -jobname cdocscl2 \|\\
% |  "\def\version{final}\input{childdoc.def}\childdocforward{cdocsch2}"|
% \end{tabular}
% \end{center}
% Note that the trailing backslash on each first line
% merely continues the input to the second line
% (for convenient cut ant paste).
% Furthermore, the command |latex| can be replaced by any
% of its alternative versions such as |pdflatex|.
%
% %%%%%%%%%%%%%%%%%%%%%%%%%%%%%%%%%%%%%%%%%%%%%%%%%%%%%%%%%%%%%%%%%%%%%%%%%%%%%%
% %%%%%%%%%%%%%%%%%%%%%%%%%%%%%%%%%%%%%%%%%%%%%%%%%%%%%%%%%%%%%%%%%%%%%%%%%%%%%%
% \section{Implementation}
%\iffalse
%<*package>
%\fi
%
% This section describes the definitions file |childdoc.def|.

% The definitions cannot be loaded using |\usepackage| or |\RequirePackage|
% which has a mechanism to prevent loading a style file more than once.
% When loading the definitions by means of |\input|
% multiple instances have to be prevented manually:
%\iffalse
%This code needs to be before the `\ProvidesFile' directive
%which is defined at the beginning of this file.
%Therefore it is also placed there and commented out here.
%</package>
%<*discard>
%\fi
%    \begin{macrocode}
\ifdefined\childdocmain\endinput\fi
%    \end{macrocode}
%\iffalse
%</discard>
%<*package>
%\fi
%
% \macro{\ifchilddoc}
% \macro{\ifchilddocmanual}
% The conditional |\ifchilddoc| tells whether a
% child (true) or main (false) document is being compiled.
% The conditional |\ifchilddocmanual| tells whether
% the |\includeonly| mechanism is used (false) or
% the selection of child files must be performed manually (true).
% The definitions initialise to false:
%    \begin{macrocode}
\newif\ifchilddoc
\newif\ifchilddocmanual
%    \end{macrocode}

% \macro{\childdocname}
% \macro{\childdocjob}
% The macro |\childdocname| stores the name of the main document
% to be compiled. The macro |\childdocjob| stores the name of
% the document on which the \LaTeX{} compiler was originally invoked.
% The content of |\jobname| cannot be compared
% to filenames specified in the source due to different catcodes.
% The following code rescans |\jobname|, stores the result
% in |\childdocname| and saves a copy in |\childdocjob|:
%    \begin{macrocode}
\edef\childdocname{\scantokens\expandafter{\jobname\noexpand}}
\let\childdocjob\childdocname
%    \end{macrocode}

% \macro{\childdocdisable}
% The macro |\childdocdisable| prevents the main file
% from being processed more than once.
% At this stage, the main document command |\childdocmain|
% is assumed to be called once again where it should do nothing.
% Any subsequent call to it should prevent
% a secondary processing of the main document
% It overwrites the forwarding commands
% |\childdocof| and |\childdocforward|
% with empty macros to prevent further inclusions of the main document:
%    \begin{macrocode}
\newcommand{\childdocdisable}
{
  \renewcommand{\childdocmain}[1]{\renewcommand{\childdocmain}[1]{\endinput}}
  \renewcommand{\childdocof}[1]{}
  \renewcommand{\childdocby}[2][]{}
  \renewcommand{\childdocforward}[2][]{}
  \renewcommand{\childdocdisable}{}
}
%    \end{macrocode}

% \macro{\childdocmain}
% The macro |\childdocmain| is to be called at the top of the main file
% with nothing or the main filename (without extension) as argument.
% First, it breaks loops.
% If the argument is not empty and does not match |\childdocname|
% (which is set by the first inclusion of |childdoc.def|),
% |\ifchilddoc| is set to true, |\includeonly| is applied to the child file
% and |\jobname| is set to the main file
% (for proper handling of |.aux| files):
%    \begin{macrocode}
\newcommand{\childdocmain}[1]
{
  \childdocdisable\childdocmain{}
  \if?#1?\else
    \begingroup
      \def\childdoctmp{#1}
      \ifx\childdoctmp\childdocname
        \def\childdoctmp{}
      \else
        \def\childdoctmp
        {
          \childdoctrue
          \includeonly{\childdocname}
          \def\childdocjob{#1}
          \def\jobname{#1}
        }
      \fi
      \expandafter
    \endgroup
    \childdoctmp
  \fi
}
%    \end{macrocode}

% \macro{\childdocof}
% The command |\childdocof| redirects
% compilation to the main file |#1|.
%    \begin{macrocode}
\newcommand{\childdocof}[1]
{
  \childdocdisable
  \childdoctrue
  \includeonly{\childdocname}
  \def\jobname{#1}
  \def\childdocjob{#1}
  \input{#1}
}
%    \end{macrocode}

% \macro{\childdocby}
% The command |\childdocby| ....
%    \begin{macrocode}
\newcommand{\childdocby}[2][]
{
  \childdocdisable
  \childdoctrue
  \childdocmanualtrue
  \if?#1?\else
    \def\jobname{#2}
  \fi
  \def\childdocjob{#2}
  \input{#2}
  \endinput
}
%    \end{macrocode}

% \macro{\childdocforward}
% The command |\childdocforward| redirects
% compilation to the main file or
% (if the optional argument is given) a child file.
% Parameters are set as if the main file
% or a child file starting with |\childdocof| was compiled.
% Then compilation is handed over to the main file:
%    \begin{macrocode}
\newcommand{\childdocforward}[2][]
{
  \begingroup
    \if?#1?
      \def\childdoctmp
      {
        \def\childdocname{#2}
        \def\childdocjob{#2}
        \def\jobname{#2}
        \input{#2}
        \endinput
      }
    \else
      \def\childdoctmp
      {
        \childdocdisable
        \def\childdocname{#2}
        \childdoctrue
        \includeonly{#2}
        \def\childdocjob{#1}
        \def\jobname{#1}
        \input{#1}
        \endinput
      }
    \fi
    \expandafter
  \endgroup
  \childdoctmp
}
%    \end{macrocode}

% \macro{\childdocforwardprefix}
% The command |\childdocforwardprefix| redirects
% compilation to the main or a child file by means of a pattern.
% The prefix |#1| in the current filename is replaced by |#2|
% and the suffix of the current filename is kept
% (it is assumed that the filename does not contain the substring `|~~~|'
% which is used as a delimiter).
% Compilation is handed over to the new file by |\childdocforward|:
%    \begin{macrocode}
\newcommand{\childdocforwardprefix}[3][]
{
  \begingroup
    \def\childdocextract #2##1~~~{\def\childdoctmp{\childdocforward[#1]{#3##1}}}
    \expandafter\childdocextract\childdocname~~~
    \expandafter
  \endgroup
  \childdoctmp
}
%    \end{macrocode}

% \macro{\childdoc}
% The deprecated macro |\childdoc| is a legacy version of |\childdocmain|:
%    \begin{macrocode}
\newcommand{\childdoc}{\childdocmain}
%    \end{macrocode}

% \macro{\childdocredirect}
% The deprecated macro |\childdocredirect| is a legacy version
% of |\childdocforward| and |\childdocforwardprefix|:
%    \begin{macrocode}
\newcommand{\childdocredirect}[2][]
{
  \begingroup
    \if?#1?
      \def\childdoctmp{\childdocforward{#2}}
    \else
      \def\childdoctmp{\childdocforwardprefix{#1}{#2}}
    \fi
    \expandafter
  \endgroup
  \childdoctmp
}
%    \end{macrocode}

%\iffalse
%</package>
%\fi
%
\endinput
|\\
|\childdocmain{}|\\
\end{tabular}
\end{center}
at the very top of the main \LaTeX{} file,
in particular \emph{before} the |\documentclass| statement!
The argument of |\childdocmain| should be left empty
(but it must be present).

%%%%%%%%%%%%%%%%%%%%%%%%%%%%%%%%%%%%%%%%
\DescribeMacro{\childdocof}
Furthermore, add the commands
\begin{center}
\begin{tabular}{l}
|% \iffalse
%
% childdoc.dtx Copyright (C) 2017-2018 Niklas Beisert
%
% This work may be distributed and/or modified under the
% conditions of the LaTeX Project Public License, either version 1.3
% of this license or (at your option) any later version.
% The latest version of this license is in
%   http://www.latex-project.org/lppl.txt
% and version 1.3 or later is part of all distributions of LaTeX
% version 2005/12/01 or later.
%
% This work has the LPPL maintenance status `maintained'.
%
% The Current Maintainer of this work is Niklas Beisert.
%
% This work consists of the files childdoc.dtx and childdoc.ins
% and the derived files childdoc.def and cdocsamp.tex with
% cdocsch1.tex, cdocsch2.tex, cdocsdrf.tex, cdocsfn1.tex, cdocsfn2.tex.
%
%<package>\ifdefined\childdocmain\endinput\fi
%<package>\ProvidesFile{childdoc.def}[2018/12/30 v2.0 child document driver]
%<samplemain>\ProvidesFile{cdocsamp.tex}[2018/12/30 v2.0 sample for childdoc]
%<*driver>
%\ProvidesFile{childdoc.drv}[2018/12/30 v2.0 childdoc reference manual file]
\PassOptionsToClass{10pt,a4paper}{article}
\documentclass{ltxdoc}

\usepackage[margin=35mm]{geometry}
\usepackage{hyperref}
\usepackage{hyperxmp}
\usepackage[usenames]{color}

\hypersetup{colorlinks=true}
\hypersetup{pdfstartview=FitH}
\hypersetup{pdfpagemode=UseNone}
\hypersetup{pdfsource={}}
\hypersetup{pdflang={en-UK}}
\hypersetup{pdfcopyright={Copyright 2017-2018 Niklas Beisert.
  This work may be distributed and/or modified under the
  conditions of the LaTeX Project Public License, either version 1.3
  of this license or (at your option) any later version.}}
\hypersetup{pdflicenseurl={http://www.latex-project.org/lppl.txt}}
\hypersetup{pdfcontactaddress={ETH Zurich, ITP, HIT K,
  Wolfgang-Pauli-Strasse 27}}
\hypersetup{pdfcontactpostcode={8093}}
\hypersetup{pdfcontactcity={Zurich}}
\hypersetup{pdfcontactcountry={Switzerland}}
\hypersetup{pdfcontactemail={nbeisert@itp.phys.ethz.ch}}
\hypersetup{pdfcontacturl={http://people.phys.ethz.ch/\xmptilde nbeisert/}}

\newcommand{\secref}[1]{\hyperref[#1]{section \ref*{#1}}}

\parskip1ex
\parindent0pt
\let\olditemize\itemize
\def\itemize{\olditemize\parskip0pt}

\begin{document}

\title{The \textsf{childdoc} Package}
\hypersetup{pdftitle={The childdoc Package}}
\author{Niklas Beisert\\[2ex]
  Institut f\"ur Theoretische Physik\\
  Eidgen\"ossische Technische Hochschule Z\"urich\\
  Wolfgang-Pauli-Strasse 27, 8093 Z\"urich, Switzerland\\[1ex]
  \href{mailto:nbeisert@itp.phys.ethz.ch}
  {\texttt{nbeisert@itp.phys.ethz.ch}}}
\hypersetup{pdfauthor={Niklas Beisert}}
\hypersetup{pdfsubject={Manual for the LaTeX2e Package childdoc}}
\date{30 December 2018, \textsf{v2.0}}
\maketitle

\begin{abstract}\noindent
\textsf{childdoc} is a \LaTeXe{} package
that enables the direct compilation
of document sections included by |\include|
to individual files.
\end{abstract}

\begingroup
\parskip0ex
\tableofcontents
\endgroup

%%%%%%%%%%%%%%%%%%%%%%%%%%%%%%%%%%%%%%%%%%%%%%%%%%%%%%%%%%%%%%%%%%%%%%%%%%%%%%%%
%%%%%%%%%%%%%%%%%%%%%%%%%%%%%%%%%%%%%%%%%%%%%%%%%%%%%%%%%%%%%%%%%%%%%%%%%%%%%%%%
\section{Introduction}

\LaTeX{} provides a mechanism to structure a large document (such as a book)
into a main file and several child files (containing the chapters)
using the |\include| command.
This mechanism is beneficial for documents
which span hundreds of pages in order to
make the source file(s) more manageable.
Moreover, compilation can be restricted to
selected child files by means of the |\includeonly| command.
The latter feature can be used to reduce the compilation time while editing
(this was significantly more useful in the earlier days of \LaTeX{})
or to generate a smaller document which is easier to navigate.
Another application of |\includeonly| is to generate
documents consisting of selected parts of the complete document.

However, there are a few drawbacks of the plain |\include| mechanism:
\begin{itemize}
\item
The child files cannot be compiled on their own,
they can only be compiled via the main file.
A naive editing environment
(such as a text editor with an option
to have the current file processed by \LaTeX)
may require one to switch to the main file before compiling;
attempting to compile the child file produces errors.
\item
The main file must be modified (each time)
to adjust the |\includeonly| command
to the present needs. This easily leaves the main file in a messy state.
\item
The generated document will always carry the filename
of the main document. This is inconvenient if
several child files are to be compiled and
to be kept for distribution.
\end{itemize}

The present package provides a simple interface
to make child files individually compilable by \LaTeX{}.
Compiling a child file then has the same effect as compiling
the main file with an |\includeonly| command
to select the appropriate child.
Moreover the generated document will carry the name of the child
rather than the main file.
This resolves all three above issues.

This feature is meant to make the editing of books,
thesis documents and lecture notes somewhat more convenient.
However, the package can also be used efficiently for
composing a series of documents (such as exercise sheets)
which are typically distributed individually.
It then assists the author in generating the individual documents
(potentially in different versions)
as well as a document containing the collected series.
Another application is in developing style files
or other kinds of included material
where compilation of the style file could redirect
to a sample or test file.

%%%%%%%%%%%%%%%%%%%%%%%%%%%%%%%%%%%%%%%%%%%%%%%%%%%%%%%%%%%%%%%%%%%%%%%%%%%%%%%%
%%%%%%%%%%%%%%%%%%%%%%%%%%%%%%%%%%%%%%%%%%%%%%%%%%%%%%%%%%%%%%%%%%%%%%%%%%%%%%%%
\section{Usage}

First of all, the package \textsf{childdoc} is \emph{not} a standard
\LaTeXe{} |.sty| style file! Therefore it needs to be invoked in
a non-standard way.

%%%%%%%%%%%%%%%%%%%%%%%%%%%%%%%%%%%%%%%%%%%%%%%%%%%%%%%%%%%%%%%%%%%%%%%%%%%%%%%%
\subsection{Included Files}
\label{sec:include}

%%%%%%%%%%%%%%%%%%%%%%%%%%%%%%%%%%%%%%%%
\DescribeMacro{\childdocmain}
To use the package, add the commands
\begin{center}
\begin{tabular}{l}
|\input{childdoc.def}|\\
|\childdocmain{}|\\
\end{tabular}
\end{center}
at the very top of the main \LaTeX{} file,
in particular \emph{before} the |\documentclass| statement!
The argument of |\childdocmain| should be left empty
(but it must be present).

%%%%%%%%%%%%%%%%%%%%%%%%%%%%%%%%%%%%%%%%
\DescribeMacro{\childdocof}
Furthermore, add the commands
\begin{center}
\begin{tabular}{l}
|\input{childdoc.def}|\\
|\childdocof{|\textit{main}|}|\\
\end{tabular}
\end{center}
at the top of every child file \textit{child}
which is included by |\include{|\textit{child}|}|
from within the main file
(or at least for those files to be compiled individually).
The argument \textit{main} must be the filename of the main file.

There are a couple of
considerations in setting up the main and child documents:

%%%%%%%%%%%%%%%%%%%%%%%%%%%%%%%%%%%%%%%%
\paragraph{Restrictions.}

Please note the following restrictions:
\begin{itemize}
\item
|\childdocmain| must be called with one argument \textit{main}
to ensure compatibility with earlier version of the package.
It must either be empty (|\childdocmain{}|)
or precisely match the filename of the main file in which it is specified.
See \secref{sec:detection} for further information.
\item
The filename \textit{main} must be specified without the |.tex| extension.
\item
The filename \textit{main} is case sensitive
(even in case-insensitive file systems)
due to internal string comparison.
\item
The argument \textit{main} should be fully expanded, it cannot be a macro.
\item
Subdirectories and special characters should be avoided in filenames.
\item
The command |\childdocmain{|\textit{main}|}| must be followed by a whitespace.
It should not be followed immediately by another command
or by a comment mark `|%|'.
This is because the \TeX{} parser reads the token immediately following
the argument of |\childdocmain| and puts it
at the beginning of every child section;
however, a white\-space is ignored.
\end{itemize}

%%%%%%%%%%%%%%%%%%%%%%%%%%%%%%%%%%%%%%%%
\paragraph{Content of Main File.}

It is advisable to place all content in the child files included by |\include|.
Any output contained in the main file will appear in all child documents
unless suppressed manually;
it cannot be suppressed automatically by the |\includeonly| directive
and thus should normally be avoided.
A method to include some content in the main file
by means of conditional processing is described in \secref{sec:conditional}.

%%%%%%%%%%%%%%%%%%%%%%%%%%%%%%%%%%%%%%%%
\paragraph{Page Numbering.}

When only a part of the document is compiled,
the appropriate numbering of pages
(as well as other status parameters)
is determined from the |.aux| files.
The latter contain information from previous passes.
However this information needs to propagate through
all intermediate child documents.
Therefore the page numbering in child documents may well
be inconsistent until the complete document is compiled at least once.

A useful (if unconventional) way to always ensure a consistent
page numbering is to restart the numbering in each child document
and denote the pages by `\textit{child}|.|\textit{page}'
where \textit{child} represents the chapter/section number of the child file.
This can be achieved by the command
|\numberwithin{page}{|\textit{child}|}|
of the \textsf{amsmath} package
where \textit{child} can be |chapter| or |section|
depending on the chosen structuring.
Alternatively, one can modify the macro |\thepage| appropriately
and reset the counter |page| at the start of each child file.

%%%%%%%%%%%%%%%%%%%%%%%%%%%%%%%%%%%%%%%%%%%%%%%%%%%%%%%%%%%%%%%%%%%%%%%%%%%%%%%%
\subsection{Conditional Processing}
\label{sec:conditional}

The package provides a mechanism to compile different versions
of a document. To customise the versions further some conditional processing
can come in handy to distinguish which version is being compiled.
The package provides two macros to describe the compilation context:

%%%%%%%%%%%%%%%%%%%%%%%%%%%%%%%%%%%%%%%%
\DescribeMacro{\ifchilddoc}
The conditional |\ifchilddoc| distinguishes between the compilation of
child documents and the main document:
%
\begin{center}
|\ifchilddoc |\textit{child-code}| |[|\||else |\textit{main-code}]| \||fi|
\end{center}

%%%%%%%%%%%%%%%%%%%%%%%%%%%%%%%%%%%%%%%%
\DescribeMacro{\childdocname}
\DescribeMacro{\childdocjob}
The macro |\childdocname| contains the filename (without extension)
of the main or child file being processed.
Note that |\childdocjob| will always contain the name of the main file.

%%%%%%%%%%%%%%%%%%%%%%%%%%%%%%%%%%%%%%%%
\paragraph{Title Page.}

Conditional processing can be used to include a title or banner page
in the main document when proper precautions are taken.
Importantly, the code in the main file should ensure that the page counter
(as well as other status parameters which are stored in the |.aux| files)
takes the same value after the conditional processing.
Otherwise the page numbers may take divergent values
depending on which part is compiled.

For example, a title page could be declared by:
%
\begin{center}
\begin{tabular}{l}
|\ifchilddoc\||else|\\
|\addtocounter{page}{-1}|\\
\textit{code for title page}\\
|\newpage|\\
|\||fi|
\end{tabular}
\end{center}
%
A banner page for the child documents can be generated by:
%
\begin{center}
\begin{tabular}{l}
|\ifchilddoc|\\
|\addtocounter{page}{-1}|\\
\textit{code for banner page}\\
|\newpage|\\
|\||fi|
\end{tabular}
\end{center}
%
Here one could write a message such as:
\begin{center}
|This is the part \childdocname{} of \childdocjob{}.|
\end{center}

%%%%%%%%%%%%%%%%%%%%%%%%%%%%%%%%%%%%%%%%%%%%%%%%%%%%%%%%%%%%%%%%%%%%%%%%%%%%%%%%
\subsection{Flags}
\label{sec:flags}

The package makes it easy to generate different versions
of the main or child documents.
To this end compilation flags can be defined
and assigned different default values.
They will be particularly useful in conjunction
with the forwarding mechanism described in \secref{sec:forward}.

For example, it may be useful to have a flag |\version|
which can be set to |draft| or |final|.
The document source will contain some conditional code
depending on the value of |\version|.
Suppose further, the flag should default to |final| for the main file
and to |draft| for child files
which is a natural assignment for editing the document.
This is achieved by placing the following code
in the preamble of the main document
(below the |\childdocmain| directive):
%
\begin{center}
\begin{tabular}{l}
|\ifchilddoc|\\
|\providecommand{\version}{draft}|\\
|\||else|\\
|\providecommand{\version}{final}|\\
|\||fi|
\end{tabular}
\end{center}
%
The definition by |\providecommand| makes sure
that previous definitions are not overwritten.
Further statements |\providecommand{\version}{...}|
can thus be added before the above code to override it.

For the main file, one might add a line
(between |\childdocmain| and the above block)
%
\begin{center}
|%\ifchilddoc\||else\providecommand{\version}{draft}\||fi|
\end{center}
%
which can be uncommented to produce a draft version.
Likewise one can add a line to the very top of a child file
(above the |\childdocof{|\textit{main}|}| directive)
%
\begin{center}
|%\providecommand{\version}{final}|
\end{center}
%
which can be uncommented to produce the final version of this child document.

%%%%%%%%%%%%%%%%%%%%%%%%%%%%%%%%%%%%%%%%%%%%%%%%%%%%%%%%%%%%%%%%%%%%%%%%%%%%%%%%
\subsection{Forwarding}
\label{sec:forward}

Different versions of the main or child documents
using compilation flags as described in \secref{sec:flags}
can be (permanently) stored in different files
for convenient compilation, viewing and distribution.
To this end, the package defines a command
to pass on compilation to a different file:

%%%%%%%%%%%%%%%%%%%%%%%%%%%%%%%%%%%%%%%%
\DescribeMacro{\childdocforward}
The command |\childdocforward| redirects processing to
another source file:
%
\begin{center}
\begin{tabular}{l}
|\input{childdoc.def}|\\
|\childdocforward[|\textit{main}|]{|\textit{dest}|}|\\
\end{tabular}
\end{center}
%
The argument \textit{dest} is the destination file
(without extension).
It should be the main file or one of the child files.
Note that further \textsf{childdoc} directives
such as |\childdocof| and |\childdocforward|
in the indicated file will be processed in this form.
The optional argument \textit{main}
passes on directly to the main file \textit{main}
while pretending to compile the child \textit{dest}.
This form behaves as if \textit{dest}
issues |\childdocof{|\textit{main}|}| right away,
and no further \textsf{childdoc} directives will be processed.

%%%%%%%%%%%%%%%%%%%%%%%%%%%%%%%%%%%%%%%%
\DescribeMacro{\...prefix}
In the alternative form |\childdocforwardprefix|,
%
\begin{center}
\begin{tabular}{l}
|\input{childdoc.def}|\\
|\childdocforwardprefix[|\textit{main}|]{|\textit{prefix}|}{|\textit{dest}|}|
\end{tabular}
\end{center}
%
the destination file is determined by a pattern
depending on the current file:
To make this work, the current file must be called
`{\textit{prefix}\hspace{0.2em}\textit{suffix}}'
with \textit{prefix} matching precisely the argument.
Processing is then passed on to the file
`{\textit{dest}\hspace{0.2em}\textit{suffix}}'.
Surely, the same effect is achieved by
directly specifying the
argument `{\textit{dest}\hspace{0.2em}\textit{suffix}}'
in the first form.
However, that requires to set up a different file
for each child. With the alternative form of the command
all these files can have exactly the same content
which simplifies setting them up and maintaining them.

For example, the following file |draft.tex|
with a compilation flag |\version| as described in \secref{sec:flags}
compiles the main document as a draft:
%
\begin{center}
\begin{tabular}{l}
|\def\version{draft}|\\
|\input{childdoc.def}|\\
|\childdocforward{|\textit{main}|}|
\end{tabular}
\end{center}
%
Likewise, the following files |final|\textit{nn}|.tex|
compile the final version of the child document
|child|\textit{nn}|.tex|:
%
\begin{center}
\begin{tabular}{l}
|\def\version{final}|\\
|\input{childdoc.def}|\\
|\childdocforwardprefix{final}{child}|
\end{tabular}
\end{center}
%

Note that when several versions of a main file and/or of each child file
are to be generated, it may be convenient to set up a |Makefile| or
shell script to automatise the process.

%%%%%%%%%%%%%%%%%%%%%%%%%%%%%%%%%%%%%%%%%%%%%%%%%%%%%%%%%%%%%%%%%%%%%%%%%%%%%%%%
\subsection{Command Line Processing}
\label{sec:commandline}

The effect of redirection files can also be achieved by invoking
the \LaTeX{} compiler with a more elaborate command line.
Most conveniently this should be done as part
of a shell script or a |Makefile|.

When using \textsf{childdoc} in the main file, the following
command lines effectively perform a redirection
(note that depending on the shell being used,
backslashes may have to be doubled: `|\|' $\to$ `|\\|'):
%
\begin{center}
|... -jobname "|\textit{target}|" |\\|"|[\textit{flags}]%
|\input{childdoc.def}\childdocforward[|\textit{main}|]{|\textit{dest}|}"|
\end{center}
%
Here \textit{target} is the name of the output file,
\textit{main} is the name of the main file
and \textit{dest} is the name of the main or child file to be processed
(all filenames without extensions).
The optional argument \textit{main} can be omitted
if \textit{main} matches \textit{dest}.
Optionally, compilation \textit{flags} can be defined via |\def| commands.
This command line makes the \TeX{} engine believe
it is compiling the file \textit{target}
whose content is specified as the latter parameter.
The provided code then forwards the processing to
\textit{main} or \textit{dest} as described in \secref{sec:forward}.

%%%%%%%%%%%%%%%%%%%%%%%%%%%%%%%%%%%%%%%%%%%%%%%%%%%%%%%%%%%%%%%%%%%%%%%%%%%%%%%%
\subsection{Include by Input}
\label{sec:input}

Including child documents by |\include| has some restrictions by design.
Most notably, the content of a child document always occupies
its own set of pages; pages cannot be shared between child documents.
Usually, this behaviour makes perfect sense
because each child document contain an essential part of the document.
However, in some situations it may be desirable to compose
a document from a collection of parts
without having mandatory page breaks between then.
For this case, the package
provides a mechanism to include parts
by |\input| which can also be processed individually.
However, by construction this mechanism
requires manual handling of the content to be output.

%%%%%%%%%%%%%%%%%%%%%%%%%%%%%%%%%%%%%%%%
\DescribeMacro{\ifchilddocmanual}
The main file should be prepared as usual, see \secref{sec:include}.
However, the document body must make a distinction
between processing of an individual part and of the main document, e.g.:
%
\begin{center}
\begin{tabular}{l}
|\ifchilddocmanual|\\
|\input{\childdocname}|\\
|\||else|\\
\textit{document body with }|\input{|\textit{part}|}|\\
|\||fi|
\end{tabular}
\end{center}
%
The conditional |\ifchilddocmanual| is true whenever
a part to be included by |\input| is being compiled,
and the name of the part is stored in |\childdocname|.

%%%%%%%%%%%%%%%%%%%%%%%%%%%%%%%%%%%%%%%%
\DescribeMacro{\childdocby}
Each part to be included by |\input| should start with:
%
\begin{center}
\begin{tabular}{l}
|\input{childdoc.def}|\\
|\childdocby{|\textit{main}|}|\\
\end{tabular}
\end{center}
%
The directive |\childdocby| is similar to |\childdocof|
described in \secref{sec:include},
but the subsequent selection of content must be done manually.
To that end, both |\ifchilddoc| and |\ifchilddocmanual|
will be true upon processing of a part,
and the name of the part is stored in |\childdocname|.
Note that |\jobname| will be set to the filename of the current part
so that each part receives an individual |.aux| file
that does not interfere with the |.aux| file(s) of the main document.
This behaviour can be altered by the alternative form
|\childdocby[*]{|\textit{main}|}| (with a non-empty optional argument)
which uses the |.aux| file of the main document
by setting |\jobname| to \textit{main}.

%%%%%%%%%%%%%%%%%%%%%%%%%%%%%%%%%%%%%%%%%%%%%%%%%%%%%%%%%%%%%%%%%%%%%%%%%%%%%%%%
\subsection{Driver Development}
\label{sec:driver}

The \textsf{childdoc} mechanism can also be use for the development
of definition files such as \LaTeX{} styles or classes.
This case differs from the above setup with multiple parts
included by |\include| in that no |\includeonly| should be invoked.
This can be achieved by starting the include file
(before |\ProvidesPackage|) with:
%
\begin{center}
\begin{tabular}{l}
|\input{childdoc.def}|\\
|\childdocforward{|\textit{main}|}|\\
\end{tabular}
\end{center}
%
or alternatively with:
%
\begin{center}
\begin{tabular}{l}
|\input{childdoc.def}|\\
|\childdocby{|\textit{main}|}|\\
\end{tabular}
\end{center}
%
Both forms have slightly different effects as described above.
The main file is prepared as usual, see \secref{sec:include}.

%%%%%%%%%%%%%%%%%%%%%%%%%%%%%%%%%%%%%%%%%%%%%%%%%%%%%%%%%%%%%%%%%%%%%%%%%%%%%%%%
\subsection{Legacy Detection}
\label{sec:detection}

The directive |\childdocmain| in the main file can detect
whether the complete document or merely a child is to be compiled
even without using the directive |\childdocof|.
This method is deprecated because it is less robust
and there is no compelling reason to use it;
it is merely provided for backward compatibility
and it may be removed in future versions.

If the detection mechanism is to be used,
it is mandatory to correctly specify
the filename of the main file as the argument of |\childdocmain|:
%
\begin{center}
\begin{tabular}{l}
|\input{childdoc.def}|\\
|\childdocmain{|\textit{main}|}|\\
\end{tabular}
\end{center}
%
If |\jobname| does not match the argument \textit{main} of |\childdocmain|,
it is assumed that |\jobname| points to the child file to be compiled.
When using |\childdocmain| with the main file specified as argument,
it suffices to start a child file
with just |\input{|\textit{main}|}|
without loading of the package and using |\childdocof|.
If instead all processing is done
with the appropriate \textsf{childdoc} directives,
the argument of \textit{main} of |\childdocmain| can be empty.

An alternative version of the command line processing described
in \secref{sec:commandline} using the detection mechanism reads:
%
\begin{center}
|... -jobname "|\textit{target}|" "|[\textit{flags}]%
[|\def\jobname{|\textit{dest}|}|]|\input{|\textit{main}|}"|
\end{center}

%%%%%%%%%%%%%%%%%%%%%%%%%%%%%%%%%%%%%%%%%%%%%%%%%%%%%%%%%%%%%%%%%%%%%%%%%%%%%%%%
\subsection{Manual Code}
\label{sec:manual}

In case one cannot be certain whether the definitions file |childdoc.def|
is installed on the target \TeX{} distribution
and one prefers not to ship it,
it is conceivable to paste a few relevant commands into the sources.

To that end, drop all statements |\input{childdoc.def}|
and perform the replacements as outlined below.
Instead of |\childdocmain{|\textit{main}|}| add the following code
to the top of the main file:
%
\begin{center}
\begin{tabular}{l}
|\||ifdefined\childdocname\endinput\||fi\newif\ifchilddoc|\\
|\edef\childdocname{\scantokens\expandafter{\jobname\noexpand}}|\\
|\def\childdocmain{|\textit{main}|}\||ifx\childdocmain\childdocname\||else|\\
|\childdoctrue\includeonly{\childdocname}\let\jobname\childdocmain\||fi|\\
\end{tabular}
\end{center}
%
Instead of |\childdocof{|\textit{main}|}| just include the main file
at the top of each child file:
%
\begin{center}
|\input{|\textit{main}|}|
\end{center}
%
A simple redirection |\childdocforward{|\textit{dest}|}| is achieved by:
%
\begin{center}
|\def\jobname{|\textit{dest}|}\input{\jobname}|
\end{center}
%
The redirection with prefix
|\childdocforwardprefix[|\textit{prefix}|]{|\textit{dest}|}|
is accomplished by:
%
\begin{center}
\begin{tabular}{l}
|{\edef\jobname{\scantokens\expandafter{\jobname\noexpand}}|\\
|\def\redirectjob |\textit{prefix}|#1~~~{\gdef\jobname{|\textit{dest}|#1}}|\\
|\expandafter\redirectjob\jobname~~~}\input{\jobname}|
\end{tabular}
\end{center}

In an alternative approach,
child documents can be compiled by a specific command line
without additional code or specific definitions:
%
\begin{center}
|... -jobname "|\textit{target}|" "|[\textit{flags}]%
|\includeonly{|\textit{dest}|}\input{|\textit{main}|}"|
\end{center}
%

%%%%%%%%%%%%%%%%%%%%%%%%%%%%%%%%%%%%%%%%%%%%%%%%%%%%%%%%%%%%%%%%%%%%%%%%%%%%%%%%
%%%%%%%%%%%%%%%%%%%%%%%%%%%%%%%%%%%%%%%%%%%%%%%%%%%%%%%%%%%%%%%%%%%%%%%%%%%%%%%%
\section{Information}

%%%%%%%%%%%%%%%%%%%%%%%%%%%%%%%%%%%%%%%%%%%%%%%%%%%%%%%%%%%%%%%%%%%%%%%%%%%%%%%%
\subsection{Copyright}

Copyright \copyright{} 2017--2018 Niklas Beisert

This work may be distributed and/or modified under the
conditions of the \LaTeX{} Project Public License, either version 1.3
of this license or (at your option) any later version.
The latest version of this license is in
  \url{http://www.latex-project.org/lppl.txt}
and version 1.3 or later is part of all distributions of \LaTeX{}
version 2005/12/01 or later.

This work has the LPPL maintenance status `maintained'.

The Current Maintainer of this work is Niklas Beisert.

This work consists of the files |README.txt|, |childdoc.ins| and |childdoc.dtx|
as well as the derived files |childdoc.def|, |cdocsamp.tex|
with |cdocsch1.tex|, |cdocsch2.tex|, |cdocspt3.tex|, |cdocspt4.tex|,
|cdocsdrf.tex|, |cdocsfn1.tex|, |cdocsfn2.tex|
as well as |childdoc.pdf|.

%%%%%%%%%%%%%%%%%%%%%%%%%%%%%%%%%%%%%%%%%%%%%%%%%%%%%%%%%%%%%%%%%%%%%%%%%%%%%%%%
\subsection{Files and Installation}

The package consists of the files:
%
\begin{center}
\begin{tabular}{ll}
    |README.txt|   & readme file \\
    |childdoc.ins| & installation file \\
    |childdoc.dtx| & source file \\
    |childdoc.def| & definition file \\
    |cdocsamp.tex| & sample main file \\
    |cdocsch1.tex| & sample include file \\
    |cdocsch2.tex| & sample include file \\
    |cdocspt3.tex| & sample part file \\
    |cdocspt4.tex| & sample part file \\
    |cdocsdrf.tex| & sample redirection file \\
    |cdocsfn1.tex| & sample redirection file \\
    |cdocsfn2.tex| & sample redirection file \\
    |childdoc.pdf| & manual
\end{tabular}
\end{center}
%
The distribution consists of the files
|README.txt|, |childdoc.ins| and |childdoc.dtx|.
%
\begin{itemize}
\item
Run (pdf)\LaTeX{} on |childdoc.dtx|
to compile the manual |childdoc.pdf| (this file).
\item
Run \LaTeX{} on |childdoc.ins| to create the definitions file |childdoc.def|
and the sample |cdocsamp.tex| with include files
|cdocsch1.tex|, |cdocsch2.tex|, |cdocspt3.tex|, |cdocspt4.tex|,
|cdocsdrf.tex|, |cdocsfn1.tex|, |cdocsfn2.tex|.
Then copy the file |childdoc.def| to an appropriate directory of your \LaTeX{}
distribution, e.g.\ \textit{texmf-root}|/tex/latex/childdoc|.
\end{itemize}

%%%%%%%%%%%%%%%%%%%%%%%%%%%%%%%%%%%%%%%%%%%%%%%%%%%%%%%%%%%%%%%%%%%%%%%%%%%%%%%%
\subsection{Related CTAN Packages}

There are several other packages which offer a similar functionality:
%
\begin{itemize}
\item
The packages
\href{http://ctan.org/pkg/docmute}{\textsf{docmute}},
\href{http://ctan.org/pkg/includex}{\textsf{includex}} and
\href{http://ctan.org/pkg/standalone}{\textsf{standalone}}
provide commands to include only the document body of
a child file thus allowing both files to be compiled individually.
\item
The packages \href{http://ctan.org/pkg/subdocs}{\textsf{subdocs}}
and \href{http://ctan.org/pkg/subfiles}{\textsf{subfiles}}
provide structures in which the main and child documents can be
encapsulated and allowing them to be compiled individually.
The inclusion mechanism is different from the conventional |\include|.
\item
The package \href{http://ctan.org/pkg/combine}{\textsf{combine}}
is an elaborate solution to combine several documents into one.
\end{itemize}
%
See also the CTAN topic \href{http://ctan.org/topic/subdocs}{\textsf{subdocs}}
for further related packages.
The present package differs from the above solutions in that
a document structure constructed with the conventional |\include| mechanism
just needs two extra commands at the top of every file
such that all constituent files can be compiled individually.

%%%%%%%%%%%%%%%%%%%%%%%%%%%%%%%%%%%%%%%%%%%%%%%%%%%%%%%%%%%%%%%%%%%%%%%%%%%%%%%%
%\subsection{Feature Suggestions}
%
%The following is a list of features which may be useful for future
%versions of this package:
%%
%\begin{itemize}
%\item
%\ldots
%\end{itemize}

%%%%%%%%%%%%%%%%%%%%%%%%%%%%%%%%%%%%%%%%%%%%%%%%%%%%%%%%%%%%%%%%%%%%%%%%%%%%%%%%
\subsection{Revision History}

%%%%%%%%%%%%%%%%%%%%%%%%%%%%%%%%%%%%%%%%
\paragraph{v2.0:} 2018/12/30

\begin{itemize}
\item
immediate forward processing
\item
added |\childdocby| mechanism
\item
manual restructured
\end{itemize}

%%%%%%%%%%%%%%%%%%%%%%%%%%%%%%%%%%%%%%%%
\paragraph{v1.6:} 2018/01/17

\begin{itemize}
\item
application for development of include files
\item
corrections to manual
\end{itemize}

%%%%%%%%%%%%%%%%%%%%%%%%%%%%%%%%%%%%%%%%
\paragraph{v1.5:} 2017/05/21

\begin{itemize}
\item
more complete structuring introduced
\item
|\childdocof| introduced
\item
|\childdoc| renamed to |\childdocmain|
\item
|\childredirect| renamed to |\childdocforward| and |\childdocforwardprefix|
and functionality expanded
\end{itemize}

%%%%%%%%%%%%%%%%%%%%%%%%%%%%%%%%%%%%%%%%
\paragraph{v1.0:} 2017/04/27

\begin{itemize}
\item
manual and install package
\item
first version published on CTAN
\end{itemize}

%%%%%%%%%%%%%%%%%%%%%%%%%%%%%%%%%%%%%%%%
\paragraph{v0.6:} 2017/04/26

\begin{itemize}
\item
redirection mechanism added
\end{itemize}

%%%%%%%%%%%%%%%%%%%%%%%%%%%%%%%%%%%%%%%%
\paragraph{v0.5:} 2017/04/26

\begin{itemize}
\item
functionality in definition file
\end{itemize}


%%%%%%%%%%%%%%%%%%%%%%%%%%%%%%%%%%%%%%%%%%%%%%%%%%%%%%%%%%%%%%%%%%%%%%%%%%%%%%%%
%%%%%%%%%%%%%%%%%%%%%%%%%%%%%%%%%%%%%%%%%%%%%%%%%%%%%%%%%%%%%%%%%%%%%%%%%%%%%%%%
%%%%%%%%%%%%%%%%%%%%%%%%%%%%%%%%%%%%%%%%%%%%%%%%%%%%%%%%%%%%%%%%%%%%%%%%%%%%%%%%
\appendix

\settowidth\MacroIndent{\rmfamily\scriptsize 000\ }

 \DocInput{childdoc.dtx}

\end{document}
%</driver>
% \fi
%
% %%%%%%%%%%%%%%%%%%%%%%%%%%%%%%%%%%%%%%%%%%%%%%%%%%%%%%%%%%%%%%%%%%%%%%%%%%%%%%
% %%%%%%%%%%%%%%%%%%%%%%%%%%%%%%%%%%%%%%%%%%%%%%%%%%%%%%%%%%%%%%%%%%%%%%%%%%%%%%
% \section{Sample}
%\iffalse
%<*samplemain>
%\fi
%
% The following presents a sample document
% with two chapters, two parts, a title page,
% a compile flag as well as three forwarding files to set the flag.
% It consists of eight |.tex| files:
% \begin{center}
% \begin{tabular}{ll}
% |cdocsamp.tex|&main file\\
% |cdocsch1.tex|&include file for chapter 1\\
% |cdocsch2.tex|&include file for chapter 2\\
% |cdocspt3.tex|&include file for part 3\\
% |cdocspt4.tex|&include file for part 4\\
% |cdocsdrf.tex|&forwarding file for main file in draft mode\\
% |cdocsfi1.tex|&forwarding file for final version of chapter 1\\
% |cdocsfi2.tex|&forwarding file for final version of chapter 2\\
% \end{tabular}
% \end{center}
% Each of the eight files can be compiled directly by the \LaTeX{} compiler.
%
% %%%%%%%%%%%%%%%%%%%%%%%%%%%%%%%%%%%%%%
% \paragraph{Main File.}
%
% The main file is called |cdocsamp.tex|.
%
% Load the \textsf{childdoc} definitions and
% declare the filename for the main document:
%    \begin{macrocode}
\input{childdoc.def}
\childdocmain{}
%    \end{macrocode}

% Optional override for |\version| flag:
%    \begin{macrocode}
%%\ifchilddoc\else\providecommand{\version}{draft}\fi
%    \end{macrocode}

% Define the default values for the |\version| flag
% (|final| for the main file and |draft| for childs):
%    \begin{macrocode}
\ifchilddoc
\providecommand{\version}{draft}
\else
\providecommand{\version}{final}
\fi
%    \end{macrocode}

% Load the standard document class:
%    \begin{macrocode}
\documentclass[12pt]{article}
%    \end{macrocode}

% Start the document body:
%    \begin{macrocode}
\begin{document}
%    \end{macrocode}

% Declare a title page.
% Print title, part of document being processed and version flag:
%    \begin{macrocode}
\addtocounter{page}{-1}
\begin{center}
{\LARGE\bfseries{}childdoc example\par}
\vspace{1cm}
\ifchilddoc
\ifchilddocmanual part\else chapter\fi:
`\childdocname' of `\childdocjob'\par
\else
main document: `\childdocjob'\par
\fi
version: \version\par
\end{center}
\newpage
%    \end{macrocode}

% Manually include selected file,
% otherwise process as usual:
%    \begin{macrocode}
\ifchilddocmanual
\section*{part `\childdocname'}
\input{\childdocname}
\else
%    \end{macrocode}

% Include the two chapters:
%    \begin{macrocode}
\include{cdocsch1}
\include{cdocsch2}
%    \end{macrocode}

% Include the two parts unless only chapters should be displayed:
%    \begin{macrocode}
\ifchilddoc\else
\section{part three}
\input{cdocspt3}
\section{part four}
\input{cdocspt4}
\fi
%    \end{macrocode}

% Process as usual until here:
%    \begin{macrocode}
\fi
%    \end{macrocode}

% End of document body:
%    \begin{macrocode}
\end{document}
%    \end{macrocode}
%\iffalse
%</samplemain>
%\fi
%
% %%%%%%%%%%%%%%%%%%%%%%%%%%%%%%%%%%%%%%
% \paragraph{Chapter Include Files.}
%
% The include files are called |cdocsch1.tex| and |cdocsch2.tex|.
%
%\iffalse
%<*samplechap1|samplechap2>
%\fi

% Optional override for |\version| flag:
%    \begin{macrocode}
%%\providecommand{\version}{final}
%    \end{macrocode}

% Include the main document:
%    \begin{macrocode}
\input{childdoc.def}
\childdocof{cdocsamp}
%    \end{macrocode}

%\iffalse
%</samplechap1|samplechap2>
%\fi
%
%\iffalse
%<*samplechap1>
%\fi
% Some text for chapter 1:
%    \begin{macrocode}
\section{one}
some text in chapter one
%    \end{macrocode}

%\iffalse
%</samplechap1>
%\fi
% Some text for chapter 2:
%\iffalse
%<*samplechap2>
%\fi
%    \begin{macrocode}
\section{two}
more text in chapter two
%    \end{macrocode}

%\iffalse
%</samplechap2>
%\fi
%
% %%%%%%%%%%%%%%%%%%%%%%%%%%%%%%%%%%%%%%
% \paragraph{Part Include Files.}
%
% The include files are called |cdocspt3.tex| and |cdocspt4.tex|.
%
%\iffalse
%<*samplepart3|samplepart4>
%\fi

% Optional override for |\version| flag:
%    \begin{macrocode}
%%\providecommand{\version}{final}
%    \end{macrocode}

% Include the main document:
%    \begin{macrocode}
\input{childdoc.def}
\childdocby{cdocsamp}
%    \end{macrocode}

%\iffalse
%</samplepart3|samplepart4>
%\fi
%
%\iffalse
%<*samplepart3>
%\fi
% Some text for part 3:
%    \begin{macrocode}
some text in part three
%    \end{macrocode}

%\iffalse
%</samplepart3>
%\fi
% Some text for part 4:
%\iffalse
%<*samplepart4>
%\fi
%    \begin{macrocode}
more text in part four
%    \end{macrocode}

%\iffalse
%</samplepart4>
%\fi
%
% %%%%%%%%%%%%%%%%%%%%%%%%%%%%%%%%%%%%%%
% \paragraph{Forwarding for a Complete Draft.}
%
% The following forwarding file |cdocsdrf.tex|
% compiles the main document in draft mode:
%\iffalse
%<*sampledraft>
%\fi
%    \begin{macrocode}
\def\version{draft}
\input{childdoc.def}
\childdocforward{cdocsamp}
%    \end{macrocode}

%\iffalse
%</sampledraft>
%\fi
%
% %%%%%%%%%%%%%%%%%%%%%%%%%%%%%%%%%%%%%%
% \paragraph{Forwarding for Final Version of the Chapters.}
%
% The following forwarding files |cdocsfn1.tex| and |cdocsfn2.tex|
% (with identical content)
% compile the final versions of the child documents
% |cdocsch1.tex| and |cdocsch2.tex|, respectively:
%\iffalse
%<*samplefinal>
%\fi
%    \begin{macrocode}
\def\version{final}
\input{childdoc.def}
\childdocforwardprefix[cdocsamp]{cdocsfn}{cdocsch}
%    \end{macrocode}

%\iffalse
%</samplefinal>
%\fi
%
% %%%%%%%%%%%%%%%%%%%%%%%%%%%%%%%%%%%%%%
% \paragraph{Command Line Processing.}
%
% The following three command lines generate the output files
% |cdocscld|, |cdocscl1| and |cdocscl2|
% which should be identical to
% |cdocsdrf|, |cdocsch1| and |cdocsfn2|, respectively:
% \begin{center}
% \begin{tabular}{l}
% |latex -jobname cdocscld \|\\
% |  "\def\version{draft}\input{childdoc.def}\childdocforward{cdocsamp}"|\\
% |latex -jobname cdocscl1 \|\\
% |  "\input{childdoc.def}\childdocforward[cdocsamp]{cdocsch1}"|\\
% |latex -jobname cdocscl2 \|\\
% |  "\def\version{final}\input{childdoc.def}\childdocforward{cdocsch2}"|
% \end{tabular}
% \end{center}
% Note that the trailing backslash on each first line
% merely continues the input to the second line
% (for convenient cut ant paste).
% Furthermore, the command |latex| can be replaced by any
% of its alternative versions such as |pdflatex|.
%
% %%%%%%%%%%%%%%%%%%%%%%%%%%%%%%%%%%%%%%%%%%%%%%%%%%%%%%%%%%%%%%%%%%%%%%%%%%%%%%
% %%%%%%%%%%%%%%%%%%%%%%%%%%%%%%%%%%%%%%%%%%%%%%%%%%%%%%%%%%%%%%%%%%%%%%%%%%%%%%
% \section{Implementation}
%\iffalse
%<*package>
%\fi
%
% This section describes the definitions file |childdoc.def|.

% The definitions cannot be loaded using |\usepackage| or |\RequirePackage|
% which has a mechanism to prevent loading a style file more than once.
% When loading the definitions by means of |\input|
% multiple instances have to be prevented manually:
%\iffalse
%This code needs to be before the `\ProvidesFile' directive
%which is defined at the beginning of this file.
%Therefore it is also placed there and commented out here.
%</package>
%<*discard>
%\fi
%    \begin{macrocode}
\ifdefined\childdocmain\endinput\fi
%    \end{macrocode}
%\iffalse
%</discard>
%<*package>
%\fi
%
% \macro{\ifchilddoc}
% \macro{\ifchilddocmanual}
% The conditional |\ifchilddoc| tells whether a
% child (true) or main (false) document is being compiled.
% The conditional |\ifchilddocmanual| tells whether
% the |\includeonly| mechanism is used (false) or
% the selection of child files must be performed manually (true).
% The definitions initialise to false:
%    \begin{macrocode}
\newif\ifchilddoc
\newif\ifchilddocmanual
%    \end{macrocode}

% \macro{\childdocname}
% \macro{\childdocjob}
% The macro |\childdocname| stores the name of the main document
% to be compiled. The macro |\childdocjob| stores the name of
% the document on which the \LaTeX{} compiler was originally invoked.
% The content of |\jobname| cannot be compared
% to filenames specified in the source due to different catcodes.
% The following code rescans |\jobname|, stores the result
% in |\childdocname| and saves a copy in |\childdocjob|:
%    \begin{macrocode}
\edef\childdocname{\scantokens\expandafter{\jobname\noexpand}}
\let\childdocjob\childdocname
%    \end{macrocode}

% \macro{\childdocdisable}
% The macro |\childdocdisable| prevents the main file
% from being processed more than once.
% At this stage, the main document command |\childdocmain|
% is assumed to be called once again where it should do nothing.
% Any subsequent call to it should prevent
% a secondary processing of the main document
% It overwrites the forwarding commands
% |\childdocof| and |\childdocforward|
% with empty macros to prevent further inclusions of the main document:
%    \begin{macrocode}
\newcommand{\childdocdisable}
{
  \renewcommand{\childdocmain}[1]{\renewcommand{\childdocmain}[1]{\endinput}}
  \renewcommand{\childdocof}[1]{}
  \renewcommand{\childdocby}[2][]{}
  \renewcommand{\childdocforward}[2][]{}
  \renewcommand{\childdocdisable}{}
}
%    \end{macrocode}

% \macro{\childdocmain}
% The macro |\childdocmain| is to be called at the top of the main file
% with nothing or the main filename (without extension) as argument.
% First, it breaks loops.
% If the argument is not empty and does not match |\childdocname|
% (which is set by the first inclusion of |childdoc.def|),
% |\ifchilddoc| is set to true, |\includeonly| is applied to the child file
% and |\jobname| is set to the main file
% (for proper handling of |.aux| files):
%    \begin{macrocode}
\newcommand{\childdocmain}[1]
{
  \childdocdisable\childdocmain{}
  \if?#1?\else
    \begingroup
      \def\childdoctmp{#1}
      \ifx\childdoctmp\childdocname
        \def\childdoctmp{}
      \else
        \def\childdoctmp
        {
          \childdoctrue
          \includeonly{\childdocname}
          \def\childdocjob{#1}
          \def\jobname{#1}
        }
      \fi
      \expandafter
    \endgroup
    \childdoctmp
  \fi
}
%    \end{macrocode}

% \macro{\childdocof}
% The command |\childdocof| redirects
% compilation to the main file |#1|.
%    \begin{macrocode}
\newcommand{\childdocof}[1]
{
  \childdocdisable
  \childdoctrue
  \includeonly{\childdocname}
  \def\jobname{#1}
  \def\childdocjob{#1}
  \input{#1}
}
%    \end{macrocode}

% \macro{\childdocby}
% The command |\childdocby| ....
%    \begin{macrocode}
\newcommand{\childdocby}[2][]
{
  \childdocdisable
  \childdoctrue
  \childdocmanualtrue
  \if?#1?\else
    \def\jobname{#2}
  \fi
  \def\childdocjob{#2}
  \input{#2}
  \endinput
}
%    \end{macrocode}

% \macro{\childdocforward}
% The command |\childdocforward| redirects
% compilation to the main file or
% (if the optional argument is given) a child file.
% Parameters are set as if the main file
% or a child file starting with |\childdocof| was compiled.
% Then compilation is handed over to the main file:
%    \begin{macrocode}
\newcommand{\childdocforward}[2][]
{
  \begingroup
    \if?#1?
      \def\childdoctmp
      {
        \def\childdocname{#2}
        \def\childdocjob{#2}
        \def\jobname{#2}
        \input{#2}
        \endinput
      }
    \else
      \def\childdoctmp
      {
        \childdocdisable
        \def\childdocname{#2}
        \childdoctrue
        \includeonly{#2}
        \def\childdocjob{#1}
        \def\jobname{#1}
        \input{#1}
        \endinput
      }
    \fi
    \expandafter
  \endgroup
  \childdoctmp
}
%    \end{macrocode}

% \macro{\childdocforwardprefix}
% The command |\childdocforwardprefix| redirects
% compilation to the main or a child file by means of a pattern.
% The prefix |#1| in the current filename is replaced by |#2|
% and the suffix of the current filename is kept
% (it is assumed that the filename does not contain the substring `|~~~|'
% which is used as a delimiter).
% Compilation is handed over to the new file by |\childdocforward|:
%    \begin{macrocode}
\newcommand{\childdocforwardprefix}[3][]
{
  \begingroup
    \def\childdocextract #2##1~~~{\def\childdoctmp{\childdocforward[#1]{#3##1}}}
    \expandafter\childdocextract\childdocname~~~
    \expandafter
  \endgroup
  \childdoctmp
}
%    \end{macrocode}

% \macro{\childdoc}
% The deprecated macro |\childdoc| is a legacy version of |\childdocmain|:
%    \begin{macrocode}
\newcommand{\childdoc}{\childdocmain}
%    \end{macrocode}

% \macro{\childdocredirect}
% The deprecated macro |\childdocredirect| is a legacy version
% of |\childdocforward| and |\childdocforwardprefix|:
%    \begin{macrocode}
\newcommand{\childdocredirect}[2][]
{
  \begingroup
    \if?#1?
      \def\childdoctmp{\childdocforward{#2}}
    \else
      \def\childdoctmp{\childdocforwardprefix{#1}{#2}}
    \fi
    \expandafter
  \endgroup
  \childdoctmp
}
%    \end{macrocode}

%\iffalse
%</package>
%\fi
%
\endinput
|\\
|\childdocof{|\textit{main}|}|\\
\end{tabular}
\end{center}
at the top of every child file \textit{child}
which is included by |\include{|\textit{child}|}|
from within the main file
(or at least for those files to be compiled individually).
The argument \textit{main} must be the filename of the main file.

There are a couple of
considerations in setting up the main and child documents:

%%%%%%%%%%%%%%%%%%%%%%%%%%%%%%%%%%%%%%%%
\paragraph{Restrictions.}

Please note the following restrictions:
\begin{itemize}
\item
|\childdocmain| must be called with one argument \textit{main}
to ensure compatibility with earlier version of the package.
It must either be empty (|\childdocmain{}|)
or precisely match the filename of the main file in which it is specified.
See \secref{sec:detection} for further information.
\item
The filename \textit{main} must be specified without the |.tex| extension.
\item
The filename \textit{main} is case sensitive
(even in case-insensitive file systems)
due to internal string comparison.
\item
The argument \textit{main} should be fully expanded, it cannot be a macro.
\item
Subdirectories and special characters should be avoided in filenames.
\item
The command |\childdocmain{|\textit{main}|}| must be followed by a whitespace.
It should not be followed immediately by another command
or by a comment mark `|%|'.
This is because the \TeX{} parser reads the token immediately following
the argument of |\childdocmain| and puts it
at the beginning of every child section;
however, a white\-space is ignored.
\end{itemize}

%%%%%%%%%%%%%%%%%%%%%%%%%%%%%%%%%%%%%%%%
\paragraph{Content of Main File.}

It is advisable to place all content in the child files included by |\include|.
Any output contained in the main file will appear in all child documents
unless suppressed manually;
it cannot be suppressed automatically by the |\includeonly| directive
and thus should normally be avoided.
A method to include some content in the main file
by means of conditional processing is described in \secref{sec:conditional}.

%%%%%%%%%%%%%%%%%%%%%%%%%%%%%%%%%%%%%%%%
\paragraph{Page Numbering.}

When only a part of the document is compiled,
the appropriate numbering of pages
(as well as other status parameters)
is determined from the |.aux| files.
The latter contain information from previous passes.
However this information needs to propagate through
all intermediate child documents.
Therefore the page numbering in child documents may well
be inconsistent until the complete document is compiled at least once.

A useful (if unconventional) way to always ensure a consistent
page numbering is to restart the numbering in each child document
and denote the pages by `\textit{child}|.|\textit{page}'
where \textit{child} represents the chapter/section number of the child file.
This can be achieved by the command
|\numberwithin{page}{|\textit{child}|}|
of the \textsf{amsmath} package
where \textit{child} can be |chapter| or |section|
depending on the chosen structuring.
Alternatively, one can modify the macro |\thepage| appropriately
and reset the counter |page| at the start of each child file.

%%%%%%%%%%%%%%%%%%%%%%%%%%%%%%%%%%%%%%%%%%%%%%%%%%%%%%%%%%%%%%%%%%%%%%%%%%%%%%%%
\subsection{Conditional Processing}
\label{sec:conditional}

The package provides a mechanism to compile different versions
of a document. To customise the versions further some conditional processing
can come in handy to distinguish which version is being compiled.
The package provides two macros to describe the compilation context:

%%%%%%%%%%%%%%%%%%%%%%%%%%%%%%%%%%%%%%%%
\DescribeMacro{\ifchilddoc}
The conditional |\ifchilddoc| distinguishes between the compilation of
child documents and the main document:
%
\begin{center}
|\ifchilddoc |\textit{child-code}| |[|\||else |\textit{main-code}]| \||fi|
\end{center}

%%%%%%%%%%%%%%%%%%%%%%%%%%%%%%%%%%%%%%%%
\DescribeMacro{\childdocname}
\DescribeMacro{\childdocjob}
The macro |\childdocname| contains the filename (without extension)
of the main or child file being processed.
Note that |\childdocjob| will always contain the name of the main file.

%%%%%%%%%%%%%%%%%%%%%%%%%%%%%%%%%%%%%%%%
\paragraph{Title Page.}

Conditional processing can be used to include a title or banner page
in the main document when proper precautions are taken.
Importantly, the code in the main file should ensure that the page counter
(as well as other status parameters which are stored in the |.aux| files)
takes the same value after the conditional processing.
Otherwise the page numbers may take divergent values
depending on which part is compiled.

For example, a title page could be declared by:
%
\begin{center}
\begin{tabular}{l}
|\ifchilddoc\||else|\\
|\addtocounter{page}{-1}|\\
\textit{code for title page}\\
|\newpage|\\
|\||fi|
\end{tabular}
\end{center}
%
A banner page for the child documents can be generated by:
%
\begin{center}
\begin{tabular}{l}
|\ifchilddoc|\\
|\addtocounter{page}{-1}|\\
\textit{code for banner page}\\
|\newpage|\\
|\||fi|
\end{tabular}
\end{center}
%
Here one could write a message such as:
\begin{center}
|This is the part \childdocname{} of \childdocjob{}.|
\end{center}

%%%%%%%%%%%%%%%%%%%%%%%%%%%%%%%%%%%%%%%%%%%%%%%%%%%%%%%%%%%%%%%%%%%%%%%%%%%%%%%%
\subsection{Flags}
\label{sec:flags}

The package makes it easy to generate different versions
of the main or child documents.
To this end compilation flags can be defined
and assigned different default values.
They will be particularly useful in conjunction
with the forwarding mechanism described in \secref{sec:forward}.

For example, it may be useful to have a flag |\version|
which can be set to |draft| or |final|.
The document source will contain some conditional code
depending on the value of |\version|.
Suppose further, the flag should default to |final| for the main file
and to |draft| for child files
which is a natural assignment for editing the document.
This is achieved by placing the following code
in the preamble of the main document
(below the |\childdocmain| directive):
%
\begin{center}
\begin{tabular}{l}
|\ifchilddoc|\\
|\providecommand{\version}{draft}|\\
|\||else|\\
|\providecommand{\version}{final}|\\
|\||fi|
\end{tabular}
\end{center}
%
The definition by |\providecommand| makes sure
that previous definitions are not overwritten.
Further statements |\providecommand{\version}{...}|
can thus be added before the above code to override it.

For the main file, one might add a line
(between |\childdocmain| and the above block)
%
\begin{center}
|%\ifchilddoc\||else\providecommand{\version}{draft}\||fi|
\end{center}
%
which can be uncommented to produce a draft version.
Likewise one can add a line to the very top of a child file
(above the |\childdocof{|\textit{main}|}| directive)
%
\begin{center}
|%\providecommand{\version}{final}|
\end{center}
%
which can be uncommented to produce the final version of this child document.

%%%%%%%%%%%%%%%%%%%%%%%%%%%%%%%%%%%%%%%%%%%%%%%%%%%%%%%%%%%%%%%%%%%%%%%%%%%%%%%%
\subsection{Forwarding}
\label{sec:forward}

Different versions of the main or child documents
using compilation flags as described in \secref{sec:flags}
can be (permanently) stored in different files
for convenient compilation, viewing and distribution.
To this end, the package defines a command
to pass on compilation to a different file:

%%%%%%%%%%%%%%%%%%%%%%%%%%%%%%%%%%%%%%%%
\DescribeMacro{\childdocforward}
The command |\childdocforward| redirects processing to
another source file:
%
\begin{center}
\begin{tabular}{l}
|% \iffalse
%
% childdoc.dtx Copyright (C) 2017-2018 Niklas Beisert
%
% This work may be distributed and/or modified under the
% conditions of the LaTeX Project Public License, either version 1.3
% of this license or (at your option) any later version.
% The latest version of this license is in
%   http://www.latex-project.org/lppl.txt
% and version 1.3 or later is part of all distributions of LaTeX
% version 2005/12/01 or later.
%
% This work has the LPPL maintenance status `maintained'.
%
% The Current Maintainer of this work is Niklas Beisert.
%
% This work consists of the files childdoc.dtx and childdoc.ins
% and the derived files childdoc.def and cdocsamp.tex with
% cdocsch1.tex, cdocsch2.tex, cdocsdrf.tex, cdocsfn1.tex, cdocsfn2.tex.
%
%<package>\ifdefined\childdocmain\endinput\fi
%<package>\ProvidesFile{childdoc.def}[2018/12/30 v2.0 child document driver]
%<samplemain>\ProvidesFile{cdocsamp.tex}[2018/12/30 v2.0 sample for childdoc]
%<*driver>
%\ProvidesFile{childdoc.drv}[2018/12/30 v2.0 childdoc reference manual file]
\PassOptionsToClass{10pt,a4paper}{article}
\documentclass{ltxdoc}

\usepackage[margin=35mm]{geometry}
\usepackage{hyperref}
\usepackage{hyperxmp}
\usepackage[usenames]{color}

\hypersetup{colorlinks=true}
\hypersetup{pdfstartview=FitH}
\hypersetup{pdfpagemode=UseNone}
\hypersetup{pdfsource={}}
\hypersetup{pdflang={en-UK}}
\hypersetup{pdfcopyright={Copyright 2017-2018 Niklas Beisert.
  This work may be distributed and/or modified under the
  conditions of the LaTeX Project Public License, either version 1.3
  of this license or (at your option) any later version.}}
\hypersetup{pdflicenseurl={http://www.latex-project.org/lppl.txt}}
\hypersetup{pdfcontactaddress={ETH Zurich, ITP, HIT K,
  Wolfgang-Pauli-Strasse 27}}
\hypersetup{pdfcontactpostcode={8093}}
\hypersetup{pdfcontactcity={Zurich}}
\hypersetup{pdfcontactcountry={Switzerland}}
\hypersetup{pdfcontactemail={nbeisert@itp.phys.ethz.ch}}
\hypersetup{pdfcontacturl={http://people.phys.ethz.ch/\xmptilde nbeisert/}}

\newcommand{\secref}[1]{\hyperref[#1]{section \ref*{#1}}}

\parskip1ex
\parindent0pt
\let\olditemize\itemize
\def\itemize{\olditemize\parskip0pt}

\begin{document}

\title{The \textsf{childdoc} Package}
\hypersetup{pdftitle={The childdoc Package}}
\author{Niklas Beisert\\[2ex]
  Institut f\"ur Theoretische Physik\\
  Eidgen\"ossische Technische Hochschule Z\"urich\\
  Wolfgang-Pauli-Strasse 27, 8093 Z\"urich, Switzerland\\[1ex]
  \href{mailto:nbeisert@itp.phys.ethz.ch}
  {\texttt{nbeisert@itp.phys.ethz.ch}}}
\hypersetup{pdfauthor={Niklas Beisert}}
\hypersetup{pdfsubject={Manual for the LaTeX2e Package childdoc}}
\date{30 December 2018, \textsf{v2.0}}
\maketitle

\begin{abstract}\noindent
\textsf{childdoc} is a \LaTeXe{} package
that enables the direct compilation
of document sections included by |\include|
to individual files.
\end{abstract}

\begingroup
\parskip0ex
\tableofcontents
\endgroup

%%%%%%%%%%%%%%%%%%%%%%%%%%%%%%%%%%%%%%%%%%%%%%%%%%%%%%%%%%%%%%%%%%%%%%%%%%%%%%%%
%%%%%%%%%%%%%%%%%%%%%%%%%%%%%%%%%%%%%%%%%%%%%%%%%%%%%%%%%%%%%%%%%%%%%%%%%%%%%%%%
\section{Introduction}

\LaTeX{} provides a mechanism to structure a large document (such as a book)
into a main file and several child files (containing the chapters)
using the |\include| command.
This mechanism is beneficial for documents
which span hundreds of pages in order to
make the source file(s) more manageable.
Moreover, compilation can be restricted to
selected child files by means of the |\includeonly| command.
The latter feature can be used to reduce the compilation time while editing
(this was significantly more useful in the earlier days of \LaTeX{})
or to generate a smaller document which is easier to navigate.
Another application of |\includeonly| is to generate
documents consisting of selected parts of the complete document.

However, there are a few drawbacks of the plain |\include| mechanism:
\begin{itemize}
\item
The child files cannot be compiled on their own,
they can only be compiled via the main file.
A naive editing environment
(such as a text editor with an option
to have the current file processed by \LaTeX)
may require one to switch to the main file before compiling;
attempting to compile the child file produces errors.
\item
The main file must be modified (each time)
to adjust the |\includeonly| command
to the present needs. This easily leaves the main file in a messy state.
\item
The generated document will always carry the filename
of the main document. This is inconvenient if
several child files are to be compiled and
to be kept for distribution.
\end{itemize}

The present package provides a simple interface
to make child files individually compilable by \LaTeX{}.
Compiling a child file then has the same effect as compiling
the main file with an |\includeonly| command
to select the appropriate child.
Moreover the generated document will carry the name of the child
rather than the main file.
This resolves all three above issues.

This feature is meant to make the editing of books,
thesis documents and lecture notes somewhat more convenient.
However, the package can also be used efficiently for
composing a series of documents (such as exercise sheets)
which are typically distributed individually.
It then assists the author in generating the individual documents
(potentially in different versions)
as well as a document containing the collected series.
Another application is in developing style files
or other kinds of included material
where compilation of the style file could redirect
to a sample or test file.

%%%%%%%%%%%%%%%%%%%%%%%%%%%%%%%%%%%%%%%%%%%%%%%%%%%%%%%%%%%%%%%%%%%%%%%%%%%%%%%%
%%%%%%%%%%%%%%%%%%%%%%%%%%%%%%%%%%%%%%%%%%%%%%%%%%%%%%%%%%%%%%%%%%%%%%%%%%%%%%%%
\section{Usage}

First of all, the package \textsf{childdoc} is \emph{not} a standard
\LaTeXe{} |.sty| style file! Therefore it needs to be invoked in
a non-standard way.

%%%%%%%%%%%%%%%%%%%%%%%%%%%%%%%%%%%%%%%%%%%%%%%%%%%%%%%%%%%%%%%%%%%%%%%%%%%%%%%%
\subsection{Included Files}
\label{sec:include}

%%%%%%%%%%%%%%%%%%%%%%%%%%%%%%%%%%%%%%%%
\DescribeMacro{\childdocmain}
To use the package, add the commands
\begin{center}
\begin{tabular}{l}
|\input{childdoc.def}|\\
|\childdocmain{}|\\
\end{tabular}
\end{center}
at the very top of the main \LaTeX{} file,
in particular \emph{before} the |\documentclass| statement!
The argument of |\childdocmain| should be left empty
(but it must be present).

%%%%%%%%%%%%%%%%%%%%%%%%%%%%%%%%%%%%%%%%
\DescribeMacro{\childdocof}
Furthermore, add the commands
\begin{center}
\begin{tabular}{l}
|\input{childdoc.def}|\\
|\childdocof{|\textit{main}|}|\\
\end{tabular}
\end{center}
at the top of every child file \textit{child}
which is included by |\include{|\textit{child}|}|
from within the main file
(or at least for those files to be compiled individually).
The argument \textit{main} must be the filename of the main file.

There are a couple of
considerations in setting up the main and child documents:

%%%%%%%%%%%%%%%%%%%%%%%%%%%%%%%%%%%%%%%%
\paragraph{Restrictions.}

Please note the following restrictions:
\begin{itemize}
\item
|\childdocmain| must be called with one argument \textit{main}
to ensure compatibility with earlier version of the package.
It must either be empty (|\childdocmain{}|)
or precisely match the filename of the main file in which it is specified.
See \secref{sec:detection} for further information.
\item
The filename \textit{main} must be specified without the |.tex| extension.
\item
The filename \textit{main} is case sensitive
(even in case-insensitive file systems)
due to internal string comparison.
\item
The argument \textit{main} should be fully expanded, it cannot be a macro.
\item
Subdirectories and special characters should be avoided in filenames.
\item
The command |\childdocmain{|\textit{main}|}| must be followed by a whitespace.
It should not be followed immediately by another command
or by a comment mark `|%|'.
This is because the \TeX{} parser reads the token immediately following
the argument of |\childdocmain| and puts it
at the beginning of every child section;
however, a white\-space is ignored.
\end{itemize}

%%%%%%%%%%%%%%%%%%%%%%%%%%%%%%%%%%%%%%%%
\paragraph{Content of Main File.}

It is advisable to place all content in the child files included by |\include|.
Any output contained in the main file will appear in all child documents
unless suppressed manually;
it cannot be suppressed automatically by the |\includeonly| directive
and thus should normally be avoided.
A method to include some content in the main file
by means of conditional processing is described in \secref{sec:conditional}.

%%%%%%%%%%%%%%%%%%%%%%%%%%%%%%%%%%%%%%%%
\paragraph{Page Numbering.}

When only a part of the document is compiled,
the appropriate numbering of pages
(as well as other status parameters)
is determined from the |.aux| files.
The latter contain information from previous passes.
However this information needs to propagate through
all intermediate child documents.
Therefore the page numbering in child documents may well
be inconsistent until the complete document is compiled at least once.

A useful (if unconventional) way to always ensure a consistent
page numbering is to restart the numbering in each child document
and denote the pages by `\textit{child}|.|\textit{page}'
where \textit{child} represents the chapter/section number of the child file.
This can be achieved by the command
|\numberwithin{page}{|\textit{child}|}|
of the \textsf{amsmath} package
where \textit{child} can be |chapter| or |section|
depending on the chosen structuring.
Alternatively, one can modify the macro |\thepage| appropriately
and reset the counter |page| at the start of each child file.

%%%%%%%%%%%%%%%%%%%%%%%%%%%%%%%%%%%%%%%%%%%%%%%%%%%%%%%%%%%%%%%%%%%%%%%%%%%%%%%%
\subsection{Conditional Processing}
\label{sec:conditional}

The package provides a mechanism to compile different versions
of a document. To customise the versions further some conditional processing
can come in handy to distinguish which version is being compiled.
The package provides two macros to describe the compilation context:

%%%%%%%%%%%%%%%%%%%%%%%%%%%%%%%%%%%%%%%%
\DescribeMacro{\ifchilddoc}
The conditional |\ifchilddoc| distinguishes between the compilation of
child documents and the main document:
%
\begin{center}
|\ifchilddoc |\textit{child-code}| |[|\||else |\textit{main-code}]| \||fi|
\end{center}

%%%%%%%%%%%%%%%%%%%%%%%%%%%%%%%%%%%%%%%%
\DescribeMacro{\childdocname}
\DescribeMacro{\childdocjob}
The macro |\childdocname| contains the filename (without extension)
of the main or child file being processed.
Note that |\childdocjob| will always contain the name of the main file.

%%%%%%%%%%%%%%%%%%%%%%%%%%%%%%%%%%%%%%%%
\paragraph{Title Page.}

Conditional processing can be used to include a title or banner page
in the main document when proper precautions are taken.
Importantly, the code in the main file should ensure that the page counter
(as well as other status parameters which are stored in the |.aux| files)
takes the same value after the conditional processing.
Otherwise the page numbers may take divergent values
depending on which part is compiled.

For example, a title page could be declared by:
%
\begin{center}
\begin{tabular}{l}
|\ifchilddoc\||else|\\
|\addtocounter{page}{-1}|\\
\textit{code for title page}\\
|\newpage|\\
|\||fi|
\end{tabular}
\end{center}
%
A banner page for the child documents can be generated by:
%
\begin{center}
\begin{tabular}{l}
|\ifchilddoc|\\
|\addtocounter{page}{-1}|\\
\textit{code for banner page}\\
|\newpage|\\
|\||fi|
\end{tabular}
\end{center}
%
Here one could write a message such as:
\begin{center}
|This is the part \childdocname{} of \childdocjob{}.|
\end{center}

%%%%%%%%%%%%%%%%%%%%%%%%%%%%%%%%%%%%%%%%%%%%%%%%%%%%%%%%%%%%%%%%%%%%%%%%%%%%%%%%
\subsection{Flags}
\label{sec:flags}

The package makes it easy to generate different versions
of the main or child documents.
To this end compilation flags can be defined
and assigned different default values.
They will be particularly useful in conjunction
with the forwarding mechanism described in \secref{sec:forward}.

For example, it may be useful to have a flag |\version|
which can be set to |draft| or |final|.
The document source will contain some conditional code
depending on the value of |\version|.
Suppose further, the flag should default to |final| for the main file
and to |draft| for child files
which is a natural assignment for editing the document.
This is achieved by placing the following code
in the preamble of the main document
(below the |\childdocmain| directive):
%
\begin{center}
\begin{tabular}{l}
|\ifchilddoc|\\
|\providecommand{\version}{draft}|\\
|\||else|\\
|\providecommand{\version}{final}|\\
|\||fi|
\end{tabular}
\end{center}
%
The definition by |\providecommand| makes sure
that previous definitions are not overwritten.
Further statements |\providecommand{\version}{...}|
can thus be added before the above code to override it.

For the main file, one might add a line
(between |\childdocmain| and the above block)
%
\begin{center}
|%\ifchilddoc\||else\providecommand{\version}{draft}\||fi|
\end{center}
%
which can be uncommented to produce a draft version.
Likewise one can add a line to the very top of a child file
(above the |\childdocof{|\textit{main}|}| directive)
%
\begin{center}
|%\providecommand{\version}{final}|
\end{center}
%
which can be uncommented to produce the final version of this child document.

%%%%%%%%%%%%%%%%%%%%%%%%%%%%%%%%%%%%%%%%%%%%%%%%%%%%%%%%%%%%%%%%%%%%%%%%%%%%%%%%
\subsection{Forwarding}
\label{sec:forward}

Different versions of the main or child documents
using compilation flags as described in \secref{sec:flags}
can be (permanently) stored in different files
for convenient compilation, viewing and distribution.
To this end, the package defines a command
to pass on compilation to a different file:

%%%%%%%%%%%%%%%%%%%%%%%%%%%%%%%%%%%%%%%%
\DescribeMacro{\childdocforward}
The command |\childdocforward| redirects processing to
another source file:
%
\begin{center}
\begin{tabular}{l}
|\input{childdoc.def}|\\
|\childdocforward[|\textit{main}|]{|\textit{dest}|}|\\
\end{tabular}
\end{center}
%
The argument \textit{dest} is the destination file
(without extension).
It should be the main file or one of the child files.
Note that further \textsf{childdoc} directives
such as |\childdocof| and |\childdocforward|
in the indicated file will be processed in this form.
The optional argument \textit{main}
passes on directly to the main file \textit{main}
while pretending to compile the child \textit{dest}.
This form behaves as if \textit{dest}
issues |\childdocof{|\textit{main}|}| right away,
and no further \textsf{childdoc} directives will be processed.

%%%%%%%%%%%%%%%%%%%%%%%%%%%%%%%%%%%%%%%%
\DescribeMacro{\...prefix}
In the alternative form |\childdocforwardprefix|,
%
\begin{center}
\begin{tabular}{l}
|\input{childdoc.def}|\\
|\childdocforwardprefix[|\textit{main}|]{|\textit{prefix}|}{|\textit{dest}|}|
\end{tabular}
\end{center}
%
the destination file is determined by a pattern
depending on the current file:
To make this work, the current file must be called
`{\textit{prefix}\hspace{0.2em}\textit{suffix}}'
with \textit{prefix} matching precisely the argument.
Processing is then passed on to the file
`{\textit{dest}\hspace{0.2em}\textit{suffix}}'.
Surely, the same effect is achieved by
directly specifying the
argument `{\textit{dest}\hspace{0.2em}\textit{suffix}}'
in the first form.
However, that requires to set up a different file
for each child. With the alternative form of the command
all these files can have exactly the same content
which simplifies setting them up and maintaining them.

For example, the following file |draft.tex|
with a compilation flag |\version| as described in \secref{sec:flags}
compiles the main document as a draft:
%
\begin{center}
\begin{tabular}{l}
|\def\version{draft}|\\
|\input{childdoc.def}|\\
|\childdocforward{|\textit{main}|}|
\end{tabular}
\end{center}
%
Likewise, the following files |final|\textit{nn}|.tex|
compile the final version of the child document
|child|\textit{nn}|.tex|:
%
\begin{center}
\begin{tabular}{l}
|\def\version{final}|\\
|\input{childdoc.def}|\\
|\childdocforwardprefix{final}{child}|
\end{tabular}
\end{center}
%

Note that when several versions of a main file and/or of each child file
are to be generated, it may be convenient to set up a |Makefile| or
shell script to automatise the process.

%%%%%%%%%%%%%%%%%%%%%%%%%%%%%%%%%%%%%%%%%%%%%%%%%%%%%%%%%%%%%%%%%%%%%%%%%%%%%%%%
\subsection{Command Line Processing}
\label{sec:commandline}

The effect of redirection files can also be achieved by invoking
the \LaTeX{} compiler with a more elaborate command line.
Most conveniently this should be done as part
of a shell script or a |Makefile|.

When using \textsf{childdoc} in the main file, the following
command lines effectively perform a redirection
(note that depending on the shell being used,
backslashes may have to be doubled: `|\|' $\to$ `|\\|'):
%
\begin{center}
|... -jobname "|\textit{target}|" |\\|"|[\textit{flags}]%
|\input{childdoc.def}\childdocforward[|\textit{main}|]{|\textit{dest}|}"|
\end{center}
%
Here \textit{target} is the name of the output file,
\textit{main} is the name of the main file
and \textit{dest} is the name of the main or child file to be processed
(all filenames without extensions).
The optional argument \textit{main} can be omitted
if \textit{main} matches \textit{dest}.
Optionally, compilation \textit{flags} can be defined via |\def| commands.
This command line makes the \TeX{} engine believe
it is compiling the file \textit{target}
whose content is specified as the latter parameter.
The provided code then forwards the processing to
\textit{main} or \textit{dest} as described in \secref{sec:forward}.

%%%%%%%%%%%%%%%%%%%%%%%%%%%%%%%%%%%%%%%%%%%%%%%%%%%%%%%%%%%%%%%%%%%%%%%%%%%%%%%%
\subsection{Include by Input}
\label{sec:input}

Including child documents by |\include| has some restrictions by design.
Most notably, the content of a child document always occupies
its own set of pages; pages cannot be shared between child documents.
Usually, this behaviour makes perfect sense
because each child document contain an essential part of the document.
However, in some situations it may be desirable to compose
a document from a collection of parts
without having mandatory page breaks between then.
For this case, the package
provides a mechanism to include parts
by |\input| which can also be processed individually.
However, by construction this mechanism
requires manual handling of the content to be output.

%%%%%%%%%%%%%%%%%%%%%%%%%%%%%%%%%%%%%%%%
\DescribeMacro{\ifchilddocmanual}
The main file should be prepared as usual, see \secref{sec:include}.
However, the document body must make a distinction
between processing of an individual part and of the main document, e.g.:
%
\begin{center}
\begin{tabular}{l}
|\ifchilddocmanual|\\
|\input{\childdocname}|\\
|\||else|\\
\textit{document body with }|\input{|\textit{part}|}|\\
|\||fi|
\end{tabular}
\end{center}
%
The conditional |\ifchilddocmanual| is true whenever
a part to be included by |\input| is being compiled,
and the name of the part is stored in |\childdocname|.

%%%%%%%%%%%%%%%%%%%%%%%%%%%%%%%%%%%%%%%%
\DescribeMacro{\childdocby}
Each part to be included by |\input| should start with:
%
\begin{center}
\begin{tabular}{l}
|\input{childdoc.def}|\\
|\childdocby{|\textit{main}|}|\\
\end{tabular}
\end{center}
%
The directive |\childdocby| is similar to |\childdocof|
described in \secref{sec:include},
but the subsequent selection of content must be done manually.
To that end, both |\ifchilddoc| and |\ifchilddocmanual|
will be true upon processing of a part,
and the name of the part is stored in |\childdocname|.
Note that |\jobname| will be set to the filename of the current part
so that each part receives an individual |.aux| file
that does not interfere with the |.aux| file(s) of the main document.
This behaviour can be altered by the alternative form
|\childdocby[*]{|\textit{main}|}| (with a non-empty optional argument)
which uses the |.aux| file of the main document
by setting |\jobname| to \textit{main}.

%%%%%%%%%%%%%%%%%%%%%%%%%%%%%%%%%%%%%%%%%%%%%%%%%%%%%%%%%%%%%%%%%%%%%%%%%%%%%%%%
\subsection{Driver Development}
\label{sec:driver}

The \textsf{childdoc} mechanism can also be use for the development
of definition files such as \LaTeX{} styles or classes.
This case differs from the above setup with multiple parts
included by |\include| in that no |\includeonly| should be invoked.
This can be achieved by starting the include file
(before |\ProvidesPackage|) with:
%
\begin{center}
\begin{tabular}{l}
|\input{childdoc.def}|\\
|\childdocforward{|\textit{main}|}|\\
\end{tabular}
\end{center}
%
or alternatively with:
%
\begin{center}
\begin{tabular}{l}
|\input{childdoc.def}|\\
|\childdocby{|\textit{main}|}|\\
\end{tabular}
\end{center}
%
Both forms have slightly different effects as described above.
The main file is prepared as usual, see \secref{sec:include}.

%%%%%%%%%%%%%%%%%%%%%%%%%%%%%%%%%%%%%%%%%%%%%%%%%%%%%%%%%%%%%%%%%%%%%%%%%%%%%%%%
\subsection{Legacy Detection}
\label{sec:detection}

The directive |\childdocmain| in the main file can detect
whether the complete document or merely a child is to be compiled
even without using the directive |\childdocof|.
This method is deprecated because it is less robust
and there is no compelling reason to use it;
it is merely provided for backward compatibility
and it may be removed in future versions.

If the detection mechanism is to be used,
it is mandatory to correctly specify
the filename of the main file as the argument of |\childdocmain|:
%
\begin{center}
\begin{tabular}{l}
|\input{childdoc.def}|\\
|\childdocmain{|\textit{main}|}|\\
\end{tabular}
\end{center}
%
If |\jobname| does not match the argument \textit{main} of |\childdocmain|,
it is assumed that |\jobname| points to the child file to be compiled.
When using |\childdocmain| with the main file specified as argument,
it suffices to start a child file
with just |\input{|\textit{main}|}|
without loading of the package and using |\childdocof|.
If instead all processing is done
with the appropriate \textsf{childdoc} directives,
the argument of \textit{main} of |\childdocmain| can be empty.

An alternative version of the command line processing described
in \secref{sec:commandline} using the detection mechanism reads:
%
\begin{center}
|... -jobname "|\textit{target}|" "|[\textit{flags}]%
[|\def\jobname{|\textit{dest}|}|]|\input{|\textit{main}|}"|
\end{center}

%%%%%%%%%%%%%%%%%%%%%%%%%%%%%%%%%%%%%%%%%%%%%%%%%%%%%%%%%%%%%%%%%%%%%%%%%%%%%%%%
\subsection{Manual Code}
\label{sec:manual}

In case one cannot be certain whether the definitions file |childdoc.def|
is installed on the target \TeX{} distribution
and one prefers not to ship it,
it is conceivable to paste a few relevant commands into the sources.

To that end, drop all statements |\input{childdoc.def}|
and perform the replacements as outlined below.
Instead of |\childdocmain{|\textit{main}|}| add the following code
to the top of the main file:
%
\begin{center}
\begin{tabular}{l}
|\||ifdefined\childdocname\endinput\||fi\newif\ifchilddoc|\\
|\edef\childdocname{\scantokens\expandafter{\jobname\noexpand}}|\\
|\def\childdocmain{|\textit{main}|}\||ifx\childdocmain\childdocname\||else|\\
|\childdoctrue\includeonly{\childdocname}\let\jobname\childdocmain\||fi|\\
\end{tabular}
\end{center}
%
Instead of |\childdocof{|\textit{main}|}| just include the main file
at the top of each child file:
%
\begin{center}
|\input{|\textit{main}|}|
\end{center}
%
A simple redirection |\childdocforward{|\textit{dest}|}| is achieved by:
%
\begin{center}
|\def\jobname{|\textit{dest}|}\input{\jobname}|
\end{center}
%
The redirection with prefix
|\childdocforwardprefix[|\textit{prefix}|]{|\textit{dest}|}|
is accomplished by:
%
\begin{center}
\begin{tabular}{l}
|{\edef\jobname{\scantokens\expandafter{\jobname\noexpand}}|\\
|\def\redirectjob |\textit{prefix}|#1~~~{\gdef\jobname{|\textit{dest}|#1}}|\\
|\expandafter\redirectjob\jobname~~~}\input{\jobname}|
\end{tabular}
\end{center}

In an alternative approach,
child documents can be compiled by a specific command line
without additional code or specific definitions:
%
\begin{center}
|... -jobname "|\textit{target}|" "|[\textit{flags}]%
|\includeonly{|\textit{dest}|}\input{|\textit{main}|}"|
\end{center}
%

%%%%%%%%%%%%%%%%%%%%%%%%%%%%%%%%%%%%%%%%%%%%%%%%%%%%%%%%%%%%%%%%%%%%%%%%%%%%%%%%
%%%%%%%%%%%%%%%%%%%%%%%%%%%%%%%%%%%%%%%%%%%%%%%%%%%%%%%%%%%%%%%%%%%%%%%%%%%%%%%%
\section{Information}

%%%%%%%%%%%%%%%%%%%%%%%%%%%%%%%%%%%%%%%%%%%%%%%%%%%%%%%%%%%%%%%%%%%%%%%%%%%%%%%%
\subsection{Copyright}

Copyright \copyright{} 2017--2018 Niklas Beisert

This work may be distributed and/or modified under the
conditions of the \LaTeX{} Project Public License, either version 1.3
of this license or (at your option) any later version.
The latest version of this license is in
  \url{http://www.latex-project.org/lppl.txt}
and version 1.3 or later is part of all distributions of \LaTeX{}
version 2005/12/01 or later.

This work has the LPPL maintenance status `maintained'.

The Current Maintainer of this work is Niklas Beisert.

This work consists of the files |README.txt|, |childdoc.ins| and |childdoc.dtx|
as well as the derived files |childdoc.def|, |cdocsamp.tex|
with |cdocsch1.tex|, |cdocsch2.tex|, |cdocspt3.tex|, |cdocspt4.tex|,
|cdocsdrf.tex|, |cdocsfn1.tex|, |cdocsfn2.tex|
as well as |childdoc.pdf|.

%%%%%%%%%%%%%%%%%%%%%%%%%%%%%%%%%%%%%%%%%%%%%%%%%%%%%%%%%%%%%%%%%%%%%%%%%%%%%%%%
\subsection{Files and Installation}

The package consists of the files:
%
\begin{center}
\begin{tabular}{ll}
    |README.txt|   & readme file \\
    |childdoc.ins| & installation file \\
    |childdoc.dtx| & source file \\
    |childdoc.def| & definition file \\
    |cdocsamp.tex| & sample main file \\
    |cdocsch1.tex| & sample include file \\
    |cdocsch2.tex| & sample include file \\
    |cdocspt3.tex| & sample part file \\
    |cdocspt4.tex| & sample part file \\
    |cdocsdrf.tex| & sample redirection file \\
    |cdocsfn1.tex| & sample redirection file \\
    |cdocsfn2.tex| & sample redirection file \\
    |childdoc.pdf| & manual
\end{tabular}
\end{center}
%
The distribution consists of the files
|README.txt|, |childdoc.ins| and |childdoc.dtx|.
%
\begin{itemize}
\item
Run (pdf)\LaTeX{} on |childdoc.dtx|
to compile the manual |childdoc.pdf| (this file).
\item
Run \LaTeX{} on |childdoc.ins| to create the definitions file |childdoc.def|
and the sample |cdocsamp.tex| with include files
|cdocsch1.tex|, |cdocsch2.tex|, |cdocspt3.tex|, |cdocspt4.tex|,
|cdocsdrf.tex|, |cdocsfn1.tex|, |cdocsfn2.tex|.
Then copy the file |childdoc.def| to an appropriate directory of your \LaTeX{}
distribution, e.g.\ \textit{texmf-root}|/tex/latex/childdoc|.
\end{itemize}

%%%%%%%%%%%%%%%%%%%%%%%%%%%%%%%%%%%%%%%%%%%%%%%%%%%%%%%%%%%%%%%%%%%%%%%%%%%%%%%%
\subsection{Related CTAN Packages}

There are several other packages which offer a similar functionality:
%
\begin{itemize}
\item
The packages
\href{http://ctan.org/pkg/docmute}{\textsf{docmute}},
\href{http://ctan.org/pkg/includex}{\textsf{includex}} and
\href{http://ctan.org/pkg/standalone}{\textsf{standalone}}
provide commands to include only the document body of
a child file thus allowing both files to be compiled individually.
\item
The packages \href{http://ctan.org/pkg/subdocs}{\textsf{subdocs}}
and \href{http://ctan.org/pkg/subfiles}{\textsf{subfiles}}
provide structures in which the main and child documents can be
encapsulated and allowing them to be compiled individually.
The inclusion mechanism is different from the conventional |\include|.
\item
The package \href{http://ctan.org/pkg/combine}{\textsf{combine}}
is an elaborate solution to combine several documents into one.
\end{itemize}
%
See also the CTAN topic \href{http://ctan.org/topic/subdocs}{\textsf{subdocs}}
for further related packages.
The present package differs from the above solutions in that
a document structure constructed with the conventional |\include| mechanism
just needs two extra commands at the top of every file
such that all constituent files can be compiled individually.

%%%%%%%%%%%%%%%%%%%%%%%%%%%%%%%%%%%%%%%%%%%%%%%%%%%%%%%%%%%%%%%%%%%%%%%%%%%%%%%%
%\subsection{Feature Suggestions}
%
%The following is a list of features which may be useful for future
%versions of this package:
%%
%\begin{itemize}
%\item
%\ldots
%\end{itemize}

%%%%%%%%%%%%%%%%%%%%%%%%%%%%%%%%%%%%%%%%%%%%%%%%%%%%%%%%%%%%%%%%%%%%%%%%%%%%%%%%
\subsection{Revision History}

%%%%%%%%%%%%%%%%%%%%%%%%%%%%%%%%%%%%%%%%
\paragraph{v2.0:} 2018/12/30

\begin{itemize}
\item
immediate forward processing
\item
added |\childdocby| mechanism
\item
manual restructured
\end{itemize}

%%%%%%%%%%%%%%%%%%%%%%%%%%%%%%%%%%%%%%%%
\paragraph{v1.6:} 2018/01/17

\begin{itemize}
\item
application for development of include files
\item
corrections to manual
\end{itemize}

%%%%%%%%%%%%%%%%%%%%%%%%%%%%%%%%%%%%%%%%
\paragraph{v1.5:} 2017/05/21

\begin{itemize}
\item
more complete structuring introduced
\item
|\childdocof| introduced
\item
|\childdoc| renamed to |\childdocmain|
\item
|\childredirect| renamed to |\childdocforward| and |\childdocforwardprefix|
and functionality expanded
\end{itemize}

%%%%%%%%%%%%%%%%%%%%%%%%%%%%%%%%%%%%%%%%
\paragraph{v1.0:} 2017/04/27

\begin{itemize}
\item
manual and install package
\item
first version published on CTAN
\end{itemize}

%%%%%%%%%%%%%%%%%%%%%%%%%%%%%%%%%%%%%%%%
\paragraph{v0.6:} 2017/04/26

\begin{itemize}
\item
redirection mechanism added
\end{itemize}

%%%%%%%%%%%%%%%%%%%%%%%%%%%%%%%%%%%%%%%%
\paragraph{v0.5:} 2017/04/26

\begin{itemize}
\item
functionality in definition file
\end{itemize}


%%%%%%%%%%%%%%%%%%%%%%%%%%%%%%%%%%%%%%%%%%%%%%%%%%%%%%%%%%%%%%%%%%%%%%%%%%%%%%%%
%%%%%%%%%%%%%%%%%%%%%%%%%%%%%%%%%%%%%%%%%%%%%%%%%%%%%%%%%%%%%%%%%%%%%%%%%%%%%%%%
%%%%%%%%%%%%%%%%%%%%%%%%%%%%%%%%%%%%%%%%%%%%%%%%%%%%%%%%%%%%%%%%%%%%%%%%%%%%%%%%
\appendix

\settowidth\MacroIndent{\rmfamily\scriptsize 000\ }

 \DocInput{childdoc.dtx}

\end{document}
%</driver>
% \fi
%
% %%%%%%%%%%%%%%%%%%%%%%%%%%%%%%%%%%%%%%%%%%%%%%%%%%%%%%%%%%%%%%%%%%%%%%%%%%%%%%
% %%%%%%%%%%%%%%%%%%%%%%%%%%%%%%%%%%%%%%%%%%%%%%%%%%%%%%%%%%%%%%%%%%%%%%%%%%%%%%
% \section{Sample}
%\iffalse
%<*samplemain>
%\fi
%
% The following presents a sample document
% with two chapters, two parts, a title page,
% a compile flag as well as three forwarding files to set the flag.
% It consists of eight |.tex| files:
% \begin{center}
% \begin{tabular}{ll}
% |cdocsamp.tex|&main file\\
% |cdocsch1.tex|&include file for chapter 1\\
% |cdocsch2.tex|&include file for chapter 2\\
% |cdocspt3.tex|&include file for part 3\\
% |cdocspt4.tex|&include file for part 4\\
% |cdocsdrf.tex|&forwarding file for main file in draft mode\\
% |cdocsfi1.tex|&forwarding file for final version of chapter 1\\
% |cdocsfi2.tex|&forwarding file for final version of chapter 2\\
% \end{tabular}
% \end{center}
% Each of the eight files can be compiled directly by the \LaTeX{} compiler.
%
% %%%%%%%%%%%%%%%%%%%%%%%%%%%%%%%%%%%%%%
% \paragraph{Main File.}
%
% The main file is called |cdocsamp.tex|.
%
% Load the \textsf{childdoc} definitions and
% declare the filename for the main document:
%    \begin{macrocode}
\input{childdoc.def}
\childdocmain{}
%    \end{macrocode}

% Optional override for |\version| flag:
%    \begin{macrocode}
%%\ifchilddoc\else\providecommand{\version}{draft}\fi
%    \end{macrocode}

% Define the default values for the |\version| flag
% (|final| for the main file and |draft| for childs):
%    \begin{macrocode}
\ifchilddoc
\providecommand{\version}{draft}
\else
\providecommand{\version}{final}
\fi
%    \end{macrocode}

% Load the standard document class:
%    \begin{macrocode}
\documentclass[12pt]{article}
%    \end{macrocode}

% Start the document body:
%    \begin{macrocode}
\begin{document}
%    \end{macrocode}

% Declare a title page.
% Print title, part of document being processed and version flag:
%    \begin{macrocode}
\addtocounter{page}{-1}
\begin{center}
{\LARGE\bfseries{}childdoc example\par}
\vspace{1cm}
\ifchilddoc
\ifchilddocmanual part\else chapter\fi:
`\childdocname' of `\childdocjob'\par
\else
main document: `\childdocjob'\par
\fi
version: \version\par
\end{center}
\newpage
%    \end{macrocode}

% Manually include selected file,
% otherwise process as usual:
%    \begin{macrocode}
\ifchilddocmanual
\section*{part `\childdocname'}
\input{\childdocname}
\else
%    \end{macrocode}

% Include the two chapters:
%    \begin{macrocode}
\include{cdocsch1}
\include{cdocsch2}
%    \end{macrocode}

% Include the two parts unless only chapters should be displayed:
%    \begin{macrocode}
\ifchilddoc\else
\section{part three}
\input{cdocspt3}
\section{part four}
\input{cdocspt4}
\fi
%    \end{macrocode}

% Process as usual until here:
%    \begin{macrocode}
\fi
%    \end{macrocode}

% End of document body:
%    \begin{macrocode}
\end{document}
%    \end{macrocode}
%\iffalse
%</samplemain>
%\fi
%
% %%%%%%%%%%%%%%%%%%%%%%%%%%%%%%%%%%%%%%
% \paragraph{Chapter Include Files.}
%
% The include files are called |cdocsch1.tex| and |cdocsch2.tex|.
%
%\iffalse
%<*samplechap1|samplechap2>
%\fi

% Optional override for |\version| flag:
%    \begin{macrocode}
%%\providecommand{\version}{final}
%    \end{macrocode}

% Include the main document:
%    \begin{macrocode}
\input{childdoc.def}
\childdocof{cdocsamp}
%    \end{macrocode}

%\iffalse
%</samplechap1|samplechap2>
%\fi
%
%\iffalse
%<*samplechap1>
%\fi
% Some text for chapter 1:
%    \begin{macrocode}
\section{one}
some text in chapter one
%    \end{macrocode}

%\iffalse
%</samplechap1>
%\fi
% Some text for chapter 2:
%\iffalse
%<*samplechap2>
%\fi
%    \begin{macrocode}
\section{two}
more text in chapter two
%    \end{macrocode}

%\iffalse
%</samplechap2>
%\fi
%
% %%%%%%%%%%%%%%%%%%%%%%%%%%%%%%%%%%%%%%
% \paragraph{Part Include Files.}
%
% The include files are called |cdocspt3.tex| and |cdocspt4.tex|.
%
%\iffalse
%<*samplepart3|samplepart4>
%\fi

% Optional override for |\version| flag:
%    \begin{macrocode}
%%\providecommand{\version}{final}
%    \end{macrocode}

% Include the main document:
%    \begin{macrocode}
\input{childdoc.def}
\childdocby{cdocsamp}
%    \end{macrocode}

%\iffalse
%</samplepart3|samplepart4>
%\fi
%
%\iffalse
%<*samplepart3>
%\fi
% Some text for part 3:
%    \begin{macrocode}
some text in part three
%    \end{macrocode}

%\iffalse
%</samplepart3>
%\fi
% Some text for part 4:
%\iffalse
%<*samplepart4>
%\fi
%    \begin{macrocode}
more text in part four
%    \end{macrocode}

%\iffalse
%</samplepart4>
%\fi
%
% %%%%%%%%%%%%%%%%%%%%%%%%%%%%%%%%%%%%%%
% \paragraph{Forwarding for a Complete Draft.}
%
% The following forwarding file |cdocsdrf.tex|
% compiles the main document in draft mode:
%\iffalse
%<*sampledraft>
%\fi
%    \begin{macrocode}
\def\version{draft}
\input{childdoc.def}
\childdocforward{cdocsamp}
%    \end{macrocode}

%\iffalse
%</sampledraft>
%\fi
%
% %%%%%%%%%%%%%%%%%%%%%%%%%%%%%%%%%%%%%%
% \paragraph{Forwarding for Final Version of the Chapters.}
%
% The following forwarding files |cdocsfn1.tex| and |cdocsfn2.tex|
% (with identical content)
% compile the final versions of the child documents
% |cdocsch1.tex| and |cdocsch2.tex|, respectively:
%\iffalse
%<*samplefinal>
%\fi
%    \begin{macrocode}
\def\version{final}
\input{childdoc.def}
\childdocforwardprefix[cdocsamp]{cdocsfn}{cdocsch}
%    \end{macrocode}

%\iffalse
%</samplefinal>
%\fi
%
% %%%%%%%%%%%%%%%%%%%%%%%%%%%%%%%%%%%%%%
% \paragraph{Command Line Processing.}
%
% The following three command lines generate the output files
% |cdocscld|, |cdocscl1| and |cdocscl2|
% which should be identical to
% |cdocsdrf|, |cdocsch1| and |cdocsfn2|, respectively:
% \begin{center}
% \begin{tabular}{l}
% |latex -jobname cdocscld \|\\
% |  "\def\version{draft}\input{childdoc.def}\childdocforward{cdocsamp}"|\\
% |latex -jobname cdocscl1 \|\\
% |  "\input{childdoc.def}\childdocforward[cdocsamp]{cdocsch1}"|\\
% |latex -jobname cdocscl2 \|\\
% |  "\def\version{final}\input{childdoc.def}\childdocforward{cdocsch2}"|
% \end{tabular}
% \end{center}
% Note that the trailing backslash on each first line
% merely continues the input to the second line
% (for convenient cut ant paste).
% Furthermore, the command |latex| can be replaced by any
% of its alternative versions such as |pdflatex|.
%
% %%%%%%%%%%%%%%%%%%%%%%%%%%%%%%%%%%%%%%%%%%%%%%%%%%%%%%%%%%%%%%%%%%%%%%%%%%%%%%
% %%%%%%%%%%%%%%%%%%%%%%%%%%%%%%%%%%%%%%%%%%%%%%%%%%%%%%%%%%%%%%%%%%%%%%%%%%%%%%
% \section{Implementation}
%\iffalse
%<*package>
%\fi
%
% This section describes the definitions file |childdoc.def|.

% The definitions cannot be loaded using |\usepackage| or |\RequirePackage|
% which has a mechanism to prevent loading a style file more than once.
% When loading the definitions by means of |\input|
% multiple instances have to be prevented manually:
%\iffalse
%This code needs to be before the `\ProvidesFile' directive
%which is defined at the beginning of this file.
%Therefore it is also placed there and commented out here.
%</package>
%<*discard>
%\fi
%    \begin{macrocode}
\ifdefined\childdocmain\endinput\fi
%    \end{macrocode}
%\iffalse
%</discard>
%<*package>
%\fi
%
% \macro{\ifchilddoc}
% \macro{\ifchilddocmanual}
% The conditional |\ifchilddoc| tells whether a
% child (true) or main (false) document is being compiled.
% The conditional |\ifchilddocmanual| tells whether
% the |\includeonly| mechanism is used (false) or
% the selection of child files must be performed manually (true).
% The definitions initialise to false:
%    \begin{macrocode}
\newif\ifchilddoc
\newif\ifchilddocmanual
%    \end{macrocode}

% \macro{\childdocname}
% \macro{\childdocjob}
% The macro |\childdocname| stores the name of the main document
% to be compiled. The macro |\childdocjob| stores the name of
% the document on which the \LaTeX{} compiler was originally invoked.
% The content of |\jobname| cannot be compared
% to filenames specified in the source due to different catcodes.
% The following code rescans |\jobname|, stores the result
% in |\childdocname| and saves a copy in |\childdocjob|:
%    \begin{macrocode}
\edef\childdocname{\scantokens\expandafter{\jobname\noexpand}}
\let\childdocjob\childdocname
%    \end{macrocode}

% \macro{\childdocdisable}
% The macro |\childdocdisable| prevents the main file
% from being processed more than once.
% At this stage, the main document command |\childdocmain|
% is assumed to be called once again where it should do nothing.
% Any subsequent call to it should prevent
% a secondary processing of the main document
% It overwrites the forwarding commands
% |\childdocof| and |\childdocforward|
% with empty macros to prevent further inclusions of the main document:
%    \begin{macrocode}
\newcommand{\childdocdisable}
{
  \renewcommand{\childdocmain}[1]{\renewcommand{\childdocmain}[1]{\endinput}}
  \renewcommand{\childdocof}[1]{}
  \renewcommand{\childdocby}[2][]{}
  \renewcommand{\childdocforward}[2][]{}
  \renewcommand{\childdocdisable}{}
}
%    \end{macrocode}

% \macro{\childdocmain}
% The macro |\childdocmain| is to be called at the top of the main file
% with nothing or the main filename (without extension) as argument.
% First, it breaks loops.
% If the argument is not empty and does not match |\childdocname|
% (which is set by the first inclusion of |childdoc.def|),
% |\ifchilddoc| is set to true, |\includeonly| is applied to the child file
% and |\jobname| is set to the main file
% (for proper handling of |.aux| files):
%    \begin{macrocode}
\newcommand{\childdocmain}[1]
{
  \childdocdisable\childdocmain{}
  \if?#1?\else
    \begingroup
      \def\childdoctmp{#1}
      \ifx\childdoctmp\childdocname
        \def\childdoctmp{}
      \else
        \def\childdoctmp
        {
          \childdoctrue
          \includeonly{\childdocname}
          \def\childdocjob{#1}
          \def\jobname{#1}
        }
      \fi
      \expandafter
    \endgroup
    \childdoctmp
  \fi
}
%    \end{macrocode}

% \macro{\childdocof}
% The command |\childdocof| redirects
% compilation to the main file |#1|.
%    \begin{macrocode}
\newcommand{\childdocof}[1]
{
  \childdocdisable
  \childdoctrue
  \includeonly{\childdocname}
  \def\jobname{#1}
  \def\childdocjob{#1}
  \input{#1}
}
%    \end{macrocode}

% \macro{\childdocby}
% The command |\childdocby| ....
%    \begin{macrocode}
\newcommand{\childdocby}[2][]
{
  \childdocdisable
  \childdoctrue
  \childdocmanualtrue
  \if?#1?\else
    \def\jobname{#2}
  \fi
  \def\childdocjob{#2}
  \input{#2}
  \endinput
}
%    \end{macrocode}

% \macro{\childdocforward}
% The command |\childdocforward| redirects
% compilation to the main file or
% (if the optional argument is given) a child file.
% Parameters are set as if the main file
% or a child file starting with |\childdocof| was compiled.
% Then compilation is handed over to the main file:
%    \begin{macrocode}
\newcommand{\childdocforward}[2][]
{
  \begingroup
    \if?#1?
      \def\childdoctmp
      {
        \def\childdocname{#2}
        \def\childdocjob{#2}
        \def\jobname{#2}
        \input{#2}
        \endinput
      }
    \else
      \def\childdoctmp
      {
        \childdocdisable
        \def\childdocname{#2}
        \childdoctrue
        \includeonly{#2}
        \def\childdocjob{#1}
        \def\jobname{#1}
        \input{#1}
        \endinput
      }
    \fi
    \expandafter
  \endgroup
  \childdoctmp
}
%    \end{macrocode}

% \macro{\childdocforwardprefix}
% The command |\childdocforwardprefix| redirects
% compilation to the main or a child file by means of a pattern.
% The prefix |#1| in the current filename is replaced by |#2|
% and the suffix of the current filename is kept
% (it is assumed that the filename does not contain the substring `|~~~|'
% which is used as a delimiter).
% Compilation is handed over to the new file by |\childdocforward|:
%    \begin{macrocode}
\newcommand{\childdocforwardprefix}[3][]
{
  \begingroup
    \def\childdocextract #2##1~~~{\def\childdoctmp{\childdocforward[#1]{#3##1}}}
    \expandafter\childdocextract\childdocname~~~
    \expandafter
  \endgroup
  \childdoctmp
}
%    \end{macrocode}

% \macro{\childdoc}
% The deprecated macro |\childdoc| is a legacy version of |\childdocmain|:
%    \begin{macrocode}
\newcommand{\childdoc}{\childdocmain}
%    \end{macrocode}

% \macro{\childdocredirect}
% The deprecated macro |\childdocredirect| is a legacy version
% of |\childdocforward| and |\childdocforwardprefix|:
%    \begin{macrocode}
\newcommand{\childdocredirect}[2][]
{
  \begingroup
    \if?#1?
      \def\childdoctmp{\childdocforward{#2}}
    \else
      \def\childdoctmp{\childdocforwardprefix{#1}{#2}}
    \fi
    \expandafter
  \endgroup
  \childdoctmp
}
%    \end{macrocode}

%\iffalse
%</package>
%\fi
%
\endinput
|\\
|\childdocforward[|\textit{main}|]{|\textit{dest}|}|\\
\end{tabular}
\end{center}
%
The argument \textit{dest} is the destination file
(without extension).
It should be the main file or one of the child files.
Note that further \textsf{childdoc} directives
such as |\childdocof| and |\childdocforward|
in the indicated file will be processed in this form.
The optional argument \textit{main}
passes on directly to the main file \textit{main}
while pretending to compile the child \textit{dest}.
This form behaves as if \textit{dest}
issues |\childdocof{|\textit{main}|}| right away,
and no further \textsf{childdoc} directives will be processed.

%%%%%%%%%%%%%%%%%%%%%%%%%%%%%%%%%%%%%%%%
\DescribeMacro{\...prefix}
In the alternative form |\childdocforwardprefix|,
%
\begin{center}
\begin{tabular}{l}
|% \iffalse
%
% childdoc.dtx Copyright (C) 2017-2018 Niklas Beisert
%
% This work may be distributed and/or modified under the
% conditions of the LaTeX Project Public License, either version 1.3
% of this license or (at your option) any later version.
% The latest version of this license is in
%   http://www.latex-project.org/lppl.txt
% and version 1.3 or later is part of all distributions of LaTeX
% version 2005/12/01 or later.
%
% This work has the LPPL maintenance status `maintained'.
%
% The Current Maintainer of this work is Niklas Beisert.
%
% This work consists of the files childdoc.dtx and childdoc.ins
% and the derived files childdoc.def and cdocsamp.tex with
% cdocsch1.tex, cdocsch2.tex, cdocsdrf.tex, cdocsfn1.tex, cdocsfn2.tex.
%
%<package>\ifdefined\childdocmain\endinput\fi
%<package>\ProvidesFile{childdoc.def}[2018/12/30 v2.0 child document driver]
%<samplemain>\ProvidesFile{cdocsamp.tex}[2018/12/30 v2.0 sample for childdoc]
%<*driver>
%\ProvidesFile{childdoc.drv}[2018/12/30 v2.0 childdoc reference manual file]
\PassOptionsToClass{10pt,a4paper}{article}
\documentclass{ltxdoc}

\usepackage[margin=35mm]{geometry}
\usepackage{hyperref}
\usepackage{hyperxmp}
\usepackage[usenames]{color}

\hypersetup{colorlinks=true}
\hypersetup{pdfstartview=FitH}
\hypersetup{pdfpagemode=UseNone}
\hypersetup{pdfsource={}}
\hypersetup{pdflang={en-UK}}
\hypersetup{pdfcopyright={Copyright 2017-2018 Niklas Beisert.
  This work may be distributed and/or modified under the
  conditions of the LaTeX Project Public License, either version 1.3
  of this license or (at your option) any later version.}}
\hypersetup{pdflicenseurl={http://www.latex-project.org/lppl.txt}}
\hypersetup{pdfcontactaddress={ETH Zurich, ITP, HIT K,
  Wolfgang-Pauli-Strasse 27}}
\hypersetup{pdfcontactpostcode={8093}}
\hypersetup{pdfcontactcity={Zurich}}
\hypersetup{pdfcontactcountry={Switzerland}}
\hypersetup{pdfcontactemail={nbeisert@itp.phys.ethz.ch}}
\hypersetup{pdfcontacturl={http://people.phys.ethz.ch/\xmptilde nbeisert/}}

\newcommand{\secref}[1]{\hyperref[#1]{section \ref*{#1}}}

\parskip1ex
\parindent0pt
\let\olditemize\itemize
\def\itemize{\olditemize\parskip0pt}

\begin{document}

\title{The \textsf{childdoc} Package}
\hypersetup{pdftitle={The childdoc Package}}
\author{Niklas Beisert\\[2ex]
  Institut f\"ur Theoretische Physik\\
  Eidgen\"ossische Technische Hochschule Z\"urich\\
  Wolfgang-Pauli-Strasse 27, 8093 Z\"urich, Switzerland\\[1ex]
  \href{mailto:nbeisert@itp.phys.ethz.ch}
  {\texttt{nbeisert@itp.phys.ethz.ch}}}
\hypersetup{pdfauthor={Niklas Beisert}}
\hypersetup{pdfsubject={Manual for the LaTeX2e Package childdoc}}
\date{30 December 2018, \textsf{v2.0}}
\maketitle

\begin{abstract}\noindent
\textsf{childdoc} is a \LaTeXe{} package
that enables the direct compilation
of document sections included by |\include|
to individual files.
\end{abstract}

\begingroup
\parskip0ex
\tableofcontents
\endgroup

%%%%%%%%%%%%%%%%%%%%%%%%%%%%%%%%%%%%%%%%%%%%%%%%%%%%%%%%%%%%%%%%%%%%%%%%%%%%%%%%
%%%%%%%%%%%%%%%%%%%%%%%%%%%%%%%%%%%%%%%%%%%%%%%%%%%%%%%%%%%%%%%%%%%%%%%%%%%%%%%%
\section{Introduction}

\LaTeX{} provides a mechanism to structure a large document (such as a book)
into a main file and several child files (containing the chapters)
using the |\include| command.
This mechanism is beneficial for documents
which span hundreds of pages in order to
make the source file(s) more manageable.
Moreover, compilation can be restricted to
selected child files by means of the |\includeonly| command.
The latter feature can be used to reduce the compilation time while editing
(this was significantly more useful in the earlier days of \LaTeX{})
or to generate a smaller document which is easier to navigate.
Another application of |\includeonly| is to generate
documents consisting of selected parts of the complete document.

However, there are a few drawbacks of the plain |\include| mechanism:
\begin{itemize}
\item
The child files cannot be compiled on their own,
they can only be compiled via the main file.
A naive editing environment
(such as a text editor with an option
to have the current file processed by \LaTeX)
may require one to switch to the main file before compiling;
attempting to compile the child file produces errors.
\item
The main file must be modified (each time)
to adjust the |\includeonly| command
to the present needs. This easily leaves the main file in a messy state.
\item
The generated document will always carry the filename
of the main document. This is inconvenient if
several child files are to be compiled and
to be kept for distribution.
\end{itemize}

The present package provides a simple interface
to make child files individually compilable by \LaTeX{}.
Compiling a child file then has the same effect as compiling
the main file with an |\includeonly| command
to select the appropriate child.
Moreover the generated document will carry the name of the child
rather than the main file.
This resolves all three above issues.

This feature is meant to make the editing of books,
thesis documents and lecture notes somewhat more convenient.
However, the package can also be used efficiently for
composing a series of documents (such as exercise sheets)
which are typically distributed individually.
It then assists the author in generating the individual documents
(potentially in different versions)
as well as a document containing the collected series.
Another application is in developing style files
or other kinds of included material
where compilation of the style file could redirect
to a sample or test file.

%%%%%%%%%%%%%%%%%%%%%%%%%%%%%%%%%%%%%%%%%%%%%%%%%%%%%%%%%%%%%%%%%%%%%%%%%%%%%%%%
%%%%%%%%%%%%%%%%%%%%%%%%%%%%%%%%%%%%%%%%%%%%%%%%%%%%%%%%%%%%%%%%%%%%%%%%%%%%%%%%
\section{Usage}

First of all, the package \textsf{childdoc} is \emph{not} a standard
\LaTeXe{} |.sty| style file! Therefore it needs to be invoked in
a non-standard way.

%%%%%%%%%%%%%%%%%%%%%%%%%%%%%%%%%%%%%%%%%%%%%%%%%%%%%%%%%%%%%%%%%%%%%%%%%%%%%%%%
\subsection{Included Files}
\label{sec:include}

%%%%%%%%%%%%%%%%%%%%%%%%%%%%%%%%%%%%%%%%
\DescribeMacro{\childdocmain}
To use the package, add the commands
\begin{center}
\begin{tabular}{l}
|\input{childdoc.def}|\\
|\childdocmain{}|\\
\end{tabular}
\end{center}
at the very top of the main \LaTeX{} file,
in particular \emph{before} the |\documentclass| statement!
The argument of |\childdocmain| should be left empty
(but it must be present).

%%%%%%%%%%%%%%%%%%%%%%%%%%%%%%%%%%%%%%%%
\DescribeMacro{\childdocof}
Furthermore, add the commands
\begin{center}
\begin{tabular}{l}
|\input{childdoc.def}|\\
|\childdocof{|\textit{main}|}|\\
\end{tabular}
\end{center}
at the top of every child file \textit{child}
which is included by |\include{|\textit{child}|}|
from within the main file
(or at least for those files to be compiled individually).
The argument \textit{main} must be the filename of the main file.

There are a couple of
considerations in setting up the main and child documents:

%%%%%%%%%%%%%%%%%%%%%%%%%%%%%%%%%%%%%%%%
\paragraph{Restrictions.}

Please note the following restrictions:
\begin{itemize}
\item
|\childdocmain| must be called with one argument \textit{main}
to ensure compatibility with earlier version of the package.
It must either be empty (|\childdocmain{}|)
or precisely match the filename of the main file in which it is specified.
See \secref{sec:detection} for further information.
\item
The filename \textit{main} must be specified without the |.tex| extension.
\item
The filename \textit{main} is case sensitive
(even in case-insensitive file systems)
due to internal string comparison.
\item
The argument \textit{main} should be fully expanded, it cannot be a macro.
\item
Subdirectories and special characters should be avoided in filenames.
\item
The command |\childdocmain{|\textit{main}|}| must be followed by a whitespace.
It should not be followed immediately by another command
or by a comment mark `|%|'.
This is because the \TeX{} parser reads the token immediately following
the argument of |\childdocmain| and puts it
at the beginning of every child section;
however, a white\-space is ignored.
\end{itemize}

%%%%%%%%%%%%%%%%%%%%%%%%%%%%%%%%%%%%%%%%
\paragraph{Content of Main File.}

It is advisable to place all content in the child files included by |\include|.
Any output contained in the main file will appear in all child documents
unless suppressed manually;
it cannot be suppressed automatically by the |\includeonly| directive
and thus should normally be avoided.
A method to include some content in the main file
by means of conditional processing is described in \secref{sec:conditional}.

%%%%%%%%%%%%%%%%%%%%%%%%%%%%%%%%%%%%%%%%
\paragraph{Page Numbering.}

When only a part of the document is compiled,
the appropriate numbering of pages
(as well as other status parameters)
is determined from the |.aux| files.
The latter contain information from previous passes.
However this information needs to propagate through
all intermediate child documents.
Therefore the page numbering in child documents may well
be inconsistent until the complete document is compiled at least once.

A useful (if unconventional) way to always ensure a consistent
page numbering is to restart the numbering in each child document
and denote the pages by `\textit{child}|.|\textit{page}'
where \textit{child} represents the chapter/section number of the child file.
This can be achieved by the command
|\numberwithin{page}{|\textit{child}|}|
of the \textsf{amsmath} package
where \textit{child} can be |chapter| or |section|
depending on the chosen structuring.
Alternatively, one can modify the macro |\thepage| appropriately
and reset the counter |page| at the start of each child file.

%%%%%%%%%%%%%%%%%%%%%%%%%%%%%%%%%%%%%%%%%%%%%%%%%%%%%%%%%%%%%%%%%%%%%%%%%%%%%%%%
\subsection{Conditional Processing}
\label{sec:conditional}

The package provides a mechanism to compile different versions
of a document. To customise the versions further some conditional processing
can come in handy to distinguish which version is being compiled.
The package provides two macros to describe the compilation context:

%%%%%%%%%%%%%%%%%%%%%%%%%%%%%%%%%%%%%%%%
\DescribeMacro{\ifchilddoc}
The conditional |\ifchilddoc| distinguishes between the compilation of
child documents and the main document:
%
\begin{center}
|\ifchilddoc |\textit{child-code}| |[|\||else |\textit{main-code}]| \||fi|
\end{center}

%%%%%%%%%%%%%%%%%%%%%%%%%%%%%%%%%%%%%%%%
\DescribeMacro{\childdocname}
\DescribeMacro{\childdocjob}
The macro |\childdocname| contains the filename (without extension)
of the main or child file being processed.
Note that |\childdocjob| will always contain the name of the main file.

%%%%%%%%%%%%%%%%%%%%%%%%%%%%%%%%%%%%%%%%
\paragraph{Title Page.}

Conditional processing can be used to include a title or banner page
in the main document when proper precautions are taken.
Importantly, the code in the main file should ensure that the page counter
(as well as other status parameters which are stored in the |.aux| files)
takes the same value after the conditional processing.
Otherwise the page numbers may take divergent values
depending on which part is compiled.

For example, a title page could be declared by:
%
\begin{center}
\begin{tabular}{l}
|\ifchilddoc\||else|\\
|\addtocounter{page}{-1}|\\
\textit{code for title page}\\
|\newpage|\\
|\||fi|
\end{tabular}
\end{center}
%
A banner page for the child documents can be generated by:
%
\begin{center}
\begin{tabular}{l}
|\ifchilddoc|\\
|\addtocounter{page}{-1}|\\
\textit{code for banner page}\\
|\newpage|\\
|\||fi|
\end{tabular}
\end{center}
%
Here one could write a message such as:
\begin{center}
|This is the part \childdocname{} of \childdocjob{}.|
\end{center}

%%%%%%%%%%%%%%%%%%%%%%%%%%%%%%%%%%%%%%%%%%%%%%%%%%%%%%%%%%%%%%%%%%%%%%%%%%%%%%%%
\subsection{Flags}
\label{sec:flags}

The package makes it easy to generate different versions
of the main or child documents.
To this end compilation flags can be defined
and assigned different default values.
They will be particularly useful in conjunction
with the forwarding mechanism described in \secref{sec:forward}.

For example, it may be useful to have a flag |\version|
which can be set to |draft| or |final|.
The document source will contain some conditional code
depending on the value of |\version|.
Suppose further, the flag should default to |final| for the main file
and to |draft| for child files
which is a natural assignment for editing the document.
This is achieved by placing the following code
in the preamble of the main document
(below the |\childdocmain| directive):
%
\begin{center}
\begin{tabular}{l}
|\ifchilddoc|\\
|\providecommand{\version}{draft}|\\
|\||else|\\
|\providecommand{\version}{final}|\\
|\||fi|
\end{tabular}
\end{center}
%
The definition by |\providecommand| makes sure
that previous definitions are not overwritten.
Further statements |\providecommand{\version}{...}|
can thus be added before the above code to override it.

For the main file, one might add a line
(between |\childdocmain| and the above block)
%
\begin{center}
|%\ifchilddoc\||else\providecommand{\version}{draft}\||fi|
\end{center}
%
which can be uncommented to produce a draft version.
Likewise one can add a line to the very top of a child file
(above the |\childdocof{|\textit{main}|}| directive)
%
\begin{center}
|%\providecommand{\version}{final}|
\end{center}
%
which can be uncommented to produce the final version of this child document.

%%%%%%%%%%%%%%%%%%%%%%%%%%%%%%%%%%%%%%%%%%%%%%%%%%%%%%%%%%%%%%%%%%%%%%%%%%%%%%%%
\subsection{Forwarding}
\label{sec:forward}

Different versions of the main or child documents
using compilation flags as described in \secref{sec:flags}
can be (permanently) stored in different files
for convenient compilation, viewing and distribution.
To this end, the package defines a command
to pass on compilation to a different file:

%%%%%%%%%%%%%%%%%%%%%%%%%%%%%%%%%%%%%%%%
\DescribeMacro{\childdocforward}
The command |\childdocforward| redirects processing to
another source file:
%
\begin{center}
\begin{tabular}{l}
|\input{childdoc.def}|\\
|\childdocforward[|\textit{main}|]{|\textit{dest}|}|\\
\end{tabular}
\end{center}
%
The argument \textit{dest} is the destination file
(without extension).
It should be the main file or one of the child files.
Note that further \textsf{childdoc} directives
such as |\childdocof| and |\childdocforward|
in the indicated file will be processed in this form.
The optional argument \textit{main}
passes on directly to the main file \textit{main}
while pretending to compile the child \textit{dest}.
This form behaves as if \textit{dest}
issues |\childdocof{|\textit{main}|}| right away,
and no further \textsf{childdoc} directives will be processed.

%%%%%%%%%%%%%%%%%%%%%%%%%%%%%%%%%%%%%%%%
\DescribeMacro{\...prefix}
In the alternative form |\childdocforwardprefix|,
%
\begin{center}
\begin{tabular}{l}
|\input{childdoc.def}|\\
|\childdocforwardprefix[|\textit{main}|]{|\textit{prefix}|}{|\textit{dest}|}|
\end{tabular}
\end{center}
%
the destination file is determined by a pattern
depending on the current file:
To make this work, the current file must be called
`{\textit{prefix}\hspace{0.2em}\textit{suffix}}'
with \textit{prefix} matching precisely the argument.
Processing is then passed on to the file
`{\textit{dest}\hspace{0.2em}\textit{suffix}}'.
Surely, the same effect is achieved by
directly specifying the
argument `{\textit{dest}\hspace{0.2em}\textit{suffix}}'
in the first form.
However, that requires to set up a different file
for each child. With the alternative form of the command
all these files can have exactly the same content
which simplifies setting them up and maintaining them.

For example, the following file |draft.tex|
with a compilation flag |\version| as described in \secref{sec:flags}
compiles the main document as a draft:
%
\begin{center}
\begin{tabular}{l}
|\def\version{draft}|\\
|\input{childdoc.def}|\\
|\childdocforward{|\textit{main}|}|
\end{tabular}
\end{center}
%
Likewise, the following files |final|\textit{nn}|.tex|
compile the final version of the child document
|child|\textit{nn}|.tex|:
%
\begin{center}
\begin{tabular}{l}
|\def\version{final}|\\
|\input{childdoc.def}|\\
|\childdocforwardprefix{final}{child}|
\end{tabular}
\end{center}
%

Note that when several versions of a main file and/or of each child file
are to be generated, it may be convenient to set up a |Makefile| or
shell script to automatise the process.

%%%%%%%%%%%%%%%%%%%%%%%%%%%%%%%%%%%%%%%%%%%%%%%%%%%%%%%%%%%%%%%%%%%%%%%%%%%%%%%%
\subsection{Command Line Processing}
\label{sec:commandline}

The effect of redirection files can also be achieved by invoking
the \LaTeX{} compiler with a more elaborate command line.
Most conveniently this should be done as part
of a shell script or a |Makefile|.

When using \textsf{childdoc} in the main file, the following
command lines effectively perform a redirection
(note that depending on the shell being used,
backslashes may have to be doubled: `|\|' $\to$ `|\\|'):
%
\begin{center}
|... -jobname "|\textit{target}|" |\\|"|[\textit{flags}]%
|\input{childdoc.def}\childdocforward[|\textit{main}|]{|\textit{dest}|}"|
\end{center}
%
Here \textit{target} is the name of the output file,
\textit{main} is the name of the main file
and \textit{dest} is the name of the main or child file to be processed
(all filenames without extensions).
The optional argument \textit{main} can be omitted
if \textit{main} matches \textit{dest}.
Optionally, compilation \textit{flags} can be defined via |\def| commands.
This command line makes the \TeX{} engine believe
it is compiling the file \textit{target}
whose content is specified as the latter parameter.
The provided code then forwards the processing to
\textit{main} or \textit{dest} as described in \secref{sec:forward}.

%%%%%%%%%%%%%%%%%%%%%%%%%%%%%%%%%%%%%%%%%%%%%%%%%%%%%%%%%%%%%%%%%%%%%%%%%%%%%%%%
\subsection{Include by Input}
\label{sec:input}

Including child documents by |\include| has some restrictions by design.
Most notably, the content of a child document always occupies
its own set of pages; pages cannot be shared between child documents.
Usually, this behaviour makes perfect sense
because each child document contain an essential part of the document.
However, in some situations it may be desirable to compose
a document from a collection of parts
without having mandatory page breaks between then.
For this case, the package
provides a mechanism to include parts
by |\input| which can also be processed individually.
However, by construction this mechanism
requires manual handling of the content to be output.

%%%%%%%%%%%%%%%%%%%%%%%%%%%%%%%%%%%%%%%%
\DescribeMacro{\ifchilddocmanual}
The main file should be prepared as usual, see \secref{sec:include}.
However, the document body must make a distinction
between processing of an individual part and of the main document, e.g.:
%
\begin{center}
\begin{tabular}{l}
|\ifchilddocmanual|\\
|\input{\childdocname}|\\
|\||else|\\
\textit{document body with }|\input{|\textit{part}|}|\\
|\||fi|
\end{tabular}
\end{center}
%
The conditional |\ifchilddocmanual| is true whenever
a part to be included by |\input| is being compiled,
and the name of the part is stored in |\childdocname|.

%%%%%%%%%%%%%%%%%%%%%%%%%%%%%%%%%%%%%%%%
\DescribeMacro{\childdocby}
Each part to be included by |\input| should start with:
%
\begin{center}
\begin{tabular}{l}
|\input{childdoc.def}|\\
|\childdocby{|\textit{main}|}|\\
\end{tabular}
\end{center}
%
The directive |\childdocby| is similar to |\childdocof|
described in \secref{sec:include},
but the subsequent selection of content must be done manually.
To that end, both |\ifchilddoc| and |\ifchilddocmanual|
will be true upon processing of a part,
and the name of the part is stored in |\childdocname|.
Note that |\jobname| will be set to the filename of the current part
so that each part receives an individual |.aux| file
that does not interfere with the |.aux| file(s) of the main document.
This behaviour can be altered by the alternative form
|\childdocby[*]{|\textit{main}|}| (with a non-empty optional argument)
which uses the |.aux| file of the main document
by setting |\jobname| to \textit{main}.

%%%%%%%%%%%%%%%%%%%%%%%%%%%%%%%%%%%%%%%%%%%%%%%%%%%%%%%%%%%%%%%%%%%%%%%%%%%%%%%%
\subsection{Driver Development}
\label{sec:driver}

The \textsf{childdoc} mechanism can also be use for the development
of definition files such as \LaTeX{} styles or classes.
This case differs from the above setup with multiple parts
included by |\include| in that no |\includeonly| should be invoked.
This can be achieved by starting the include file
(before |\ProvidesPackage|) with:
%
\begin{center}
\begin{tabular}{l}
|\input{childdoc.def}|\\
|\childdocforward{|\textit{main}|}|\\
\end{tabular}
\end{center}
%
or alternatively with:
%
\begin{center}
\begin{tabular}{l}
|\input{childdoc.def}|\\
|\childdocby{|\textit{main}|}|\\
\end{tabular}
\end{center}
%
Both forms have slightly different effects as described above.
The main file is prepared as usual, see \secref{sec:include}.

%%%%%%%%%%%%%%%%%%%%%%%%%%%%%%%%%%%%%%%%%%%%%%%%%%%%%%%%%%%%%%%%%%%%%%%%%%%%%%%%
\subsection{Legacy Detection}
\label{sec:detection}

The directive |\childdocmain| in the main file can detect
whether the complete document or merely a child is to be compiled
even without using the directive |\childdocof|.
This method is deprecated because it is less robust
and there is no compelling reason to use it;
it is merely provided for backward compatibility
and it may be removed in future versions.

If the detection mechanism is to be used,
it is mandatory to correctly specify
the filename of the main file as the argument of |\childdocmain|:
%
\begin{center}
\begin{tabular}{l}
|\input{childdoc.def}|\\
|\childdocmain{|\textit{main}|}|\\
\end{tabular}
\end{center}
%
If |\jobname| does not match the argument \textit{main} of |\childdocmain|,
it is assumed that |\jobname| points to the child file to be compiled.
When using |\childdocmain| with the main file specified as argument,
it suffices to start a child file
with just |\input{|\textit{main}|}|
without loading of the package and using |\childdocof|.
If instead all processing is done
with the appropriate \textsf{childdoc} directives,
the argument of \textit{main} of |\childdocmain| can be empty.

An alternative version of the command line processing described
in \secref{sec:commandline} using the detection mechanism reads:
%
\begin{center}
|... -jobname "|\textit{target}|" "|[\textit{flags}]%
[|\def\jobname{|\textit{dest}|}|]|\input{|\textit{main}|}"|
\end{center}

%%%%%%%%%%%%%%%%%%%%%%%%%%%%%%%%%%%%%%%%%%%%%%%%%%%%%%%%%%%%%%%%%%%%%%%%%%%%%%%%
\subsection{Manual Code}
\label{sec:manual}

In case one cannot be certain whether the definitions file |childdoc.def|
is installed on the target \TeX{} distribution
and one prefers not to ship it,
it is conceivable to paste a few relevant commands into the sources.

To that end, drop all statements |\input{childdoc.def}|
and perform the replacements as outlined below.
Instead of |\childdocmain{|\textit{main}|}| add the following code
to the top of the main file:
%
\begin{center}
\begin{tabular}{l}
|\||ifdefined\childdocname\endinput\||fi\newif\ifchilddoc|\\
|\edef\childdocname{\scantokens\expandafter{\jobname\noexpand}}|\\
|\def\childdocmain{|\textit{main}|}\||ifx\childdocmain\childdocname\||else|\\
|\childdoctrue\includeonly{\childdocname}\let\jobname\childdocmain\||fi|\\
\end{tabular}
\end{center}
%
Instead of |\childdocof{|\textit{main}|}| just include the main file
at the top of each child file:
%
\begin{center}
|\input{|\textit{main}|}|
\end{center}
%
A simple redirection |\childdocforward{|\textit{dest}|}| is achieved by:
%
\begin{center}
|\def\jobname{|\textit{dest}|}\input{\jobname}|
\end{center}
%
The redirection with prefix
|\childdocforwardprefix[|\textit{prefix}|]{|\textit{dest}|}|
is accomplished by:
%
\begin{center}
\begin{tabular}{l}
|{\edef\jobname{\scantokens\expandafter{\jobname\noexpand}}|\\
|\def\redirectjob |\textit{prefix}|#1~~~{\gdef\jobname{|\textit{dest}|#1}}|\\
|\expandafter\redirectjob\jobname~~~}\input{\jobname}|
\end{tabular}
\end{center}

In an alternative approach,
child documents can be compiled by a specific command line
without additional code or specific definitions:
%
\begin{center}
|... -jobname "|\textit{target}|" "|[\textit{flags}]%
|\includeonly{|\textit{dest}|}\input{|\textit{main}|}"|
\end{center}
%

%%%%%%%%%%%%%%%%%%%%%%%%%%%%%%%%%%%%%%%%%%%%%%%%%%%%%%%%%%%%%%%%%%%%%%%%%%%%%%%%
%%%%%%%%%%%%%%%%%%%%%%%%%%%%%%%%%%%%%%%%%%%%%%%%%%%%%%%%%%%%%%%%%%%%%%%%%%%%%%%%
\section{Information}

%%%%%%%%%%%%%%%%%%%%%%%%%%%%%%%%%%%%%%%%%%%%%%%%%%%%%%%%%%%%%%%%%%%%%%%%%%%%%%%%
\subsection{Copyright}

Copyright \copyright{} 2017--2018 Niklas Beisert

This work may be distributed and/or modified under the
conditions of the \LaTeX{} Project Public License, either version 1.3
of this license or (at your option) any later version.
The latest version of this license is in
  \url{http://www.latex-project.org/lppl.txt}
and version 1.3 or later is part of all distributions of \LaTeX{}
version 2005/12/01 or later.

This work has the LPPL maintenance status `maintained'.

The Current Maintainer of this work is Niklas Beisert.

This work consists of the files |README.txt|, |childdoc.ins| and |childdoc.dtx|
as well as the derived files |childdoc.def|, |cdocsamp.tex|
with |cdocsch1.tex|, |cdocsch2.tex|, |cdocspt3.tex|, |cdocspt4.tex|,
|cdocsdrf.tex|, |cdocsfn1.tex|, |cdocsfn2.tex|
as well as |childdoc.pdf|.

%%%%%%%%%%%%%%%%%%%%%%%%%%%%%%%%%%%%%%%%%%%%%%%%%%%%%%%%%%%%%%%%%%%%%%%%%%%%%%%%
\subsection{Files and Installation}

The package consists of the files:
%
\begin{center}
\begin{tabular}{ll}
    |README.txt|   & readme file \\
    |childdoc.ins| & installation file \\
    |childdoc.dtx| & source file \\
    |childdoc.def| & definition file \\
    |cdocsamp.tex| & sample main file \\
    |cdocsch1.tex| & sample include file \\
    |cdocsch2.tex| & sample include file \\
    |cdocspt3.tex| & sample part file \\
    |cdocspt4.tex| & sample part file \\
    |cdocsdrf.tex| & sample redirection file \\
    |cdocsfn1.tex| & sample redirection file \\
    |cdocsfn2.tex| & sample redirection file \\
    |childdoc.pdf| & manual
\end{tabular}
\end{center}
%
The distribution consists of the files
|README.txt|, |childdoc.ins| and |childdoc.dtx|.
%
\begin{itemize}
\item
Run (pdf)\LaTeX{} on |childdoc.dtx|
to compile the manual |childdoc.pdf| (this file).
\item
Run \LaTeX{} on |childdoc.ins| to create the definitions file |childdoc.def|
and the sample |cdocsamp.tex| with include files
|cdocsch1.tex|, |cdocsch2.tex|, |cdocspt3.tex|, |cdocspt4.tex|,
|cdocsdrf.tex|, |cdocsfn1.tex|, |cdocsfn2.tex|.
Then copy the file |childdoc.def| to an appropriate directory of your \LaTeX{}
distribution, e.g.\ \textit{texmf-root}|/tex/latex/childdoc|.
\end{itemize}

%%%%%%%%%%%%%%%%%%%%%%%%%%%%%%%%%%%%%%%%%%%%%%%%%%%%%%%%%%%%%%%%%%%%%%%%%%%%%%%%
\subsection{Related CTAN Packages}

There are several other packages which offer a similar functionality:
%
\begin{itemize}
\item
The packages
\href{http://ctan.org/pkg/docmute}{\textsf{docmute}},
\href{http://ctan.org/pkg/includex}{\textsf{includex}} and
\href{http://ctan.org/pkg/standalone}{\textsf{standalone}}
provide commands to include only the document body of
a child file thus allowing both files to be compiled individually.
\item
The packages \href{http://ctan.org/pkg/subdocs}{\textsf{subdocs}}
and \href{http://ctan.org/pkg/subfiles}{\textsf{subfiles}}
provide structures in which the main and child documents can be
encapsulated and allowing them to be compiled individually.
The inclusion mechanism is different from the conventional |\include|.
\item
The package \href{http://ctan.org/pkg/combine}{\textsf{combine}}
is an elaborate solution to combine several documents into one.
\end{itemize}
%
See also the CTAN topic \href{http://ctan.org/topic/subdocs}{\textsf{subdocs}}
for further related packages.
The present package differs from the above solutions in that
a document structure constructed with the conventional |\include| mechanism
just needs two extra commands at the top of every file
such that all constituent files can be compiled individually.

%%%%%%%%%%%%%%%%%%%%%%%%%%%%%%%%%%%%%%%%%%%%%%%%%%%%%%%%%%%%%%%%%%%%%%%%%%%%%%%%
%\subsection{Feature Suggestions}
%
%The following is a list of features which may be useful for future
%versions of this package:
%%
%\begin{itemize}
%\item
%\ldots
%\end{itemize}

%%%%%%%%%%%%%%%%%%%%%%%%%%%%%%%%%%%%%%%%%%%%%%%%%%%%%%%%%%%%%%%%%%%%%%%%%%%%%%%%
\subsection{Revision History}

%%%%%%%%%%%%%%%%%%%%%%%%%%%%%%%%%%%%%%%%
\paragraph{v2.0:} 2018/12/30

\begin{itemize}
\item
immediate forward processing
\item
added |\childdocby| mechanism
\item
manual restructured
\end{itemize}

%%%%%%%%%%%%%%%%%%%%%%%%%%%%%%%%%%%%%%%%
\paragraph{v1.6:} 2018/01/17

\begin{itemize}
\item
application for development of include files
\item
corrections to manual
\end{itemize}

%%%%%%%%%%%%%%%%%%%%%%%%%%%%%%%%%%%%%%%%
\paragraph{v1.5:} 2017/05/21

\begin{itemize}
\item
more complete structuring introduced
\item
|\childdocof| introduced
\item
|\childdoc| renamed to |\childdocmain|
\item
|\childredirect| renamed to |\childdocforward| and |\childdocforwardprefix|
and functionality expanded
\end{itemize}

%%%%%%%%%%%%%%%%%%%%%%%%%%%%%%%%%%%%%%%%
\paragraph{v1.0:} 2017/04/27

\begin{itemize}
\item
manual and install package
\item
first version published on CTAN
\end{itemize}

%%%%%%%%%%%%%%%%%%%%%%%%%%%%%%%%%%%%%%%%
\paragraph{v0.6:} 2017/04/26

\begin{itemize}
\item
redirection mechanism added
\end{itemize}

%%%%%%%%%%%%%%%%%%%%%%%%%%%%%%%%%%%%%%%%
\paragraph{v0.5:} 2017/04/26

\begin{itemize}
\item
functionality in definition file
\end{itemize}


%%%%%%%%%%%%%%%%%%%%%%%%%%%%%%%%%%%%%%%%%%%%%%%%%%%%%%%%%%%%%%%%%%%%%%%%%%%%%%%%
%%%%%%%%%%%%%%%%%%%%%%%%%%%%%%%%%%%%%%%%%%%%%%%%%%%%%%%%%%%%%%%%%%%%%%%%%%%%%%%%
%%%%%%%%%%%%%%%%%%%%%%%%%%%%%%%%%%%%%%%%%%%%%%%%%%%%%%%%%%%%%%%%%%%%%%%%%%%%%%%%
\appendix

\settowidth\MacroIndent{\rmfamily\scriptsize 000\ }

 \DocInput{childdoc.dtx}

\end{document}
%</driver>
% \fi
%
% %%%%%%%%%%%%%%%%%%%%%%%%%%%%%%%%%%%%%%%%%%%%%%%%%%%%%%%%%%%%%%%%%%%%%%%%%%%%%%
% %%%%%%%%%%%%%%%%%%%%%%%%%%%%%%%%%%%%%%%%%%%%%%%%%%%%%%%%%%%%%%%%%%%%%%%%%%%%%%
% \section{Sample}
%\iffalse
%<*samplemain>
%\fi
%
% The following presents a sample document
% with two chapters, two parts, a title page,
% a compile flag as well as three forwarding files to set the flag.
% It consists of eight |.tex| files:
% \begin{center}
% \begin{tabular}{ll}
% |cdocsamp.tex|&main file\\
% |cdocsch1.tex|&include file for chapter 1\\
% |cdocsch2.tex|&include file for chapter 2\\
% |cdocspt3.tex|&include file for part 3\\
% |cdocspt4.tex|&include file for part 4\\
% |cdocsdrf.tex|&forwarding file for main file in draft mode\\
% |cdocsfi1.tex|&forwarding file for final version of chapter 1\\
% |cdocsfi2.tex|&forwarding file for final version of chapter 2\\
% \end{tabular}
% \end{center}
% Each of the eight files can be compiled directly by the \LaTeX{} compiler.
%
% %%%%%%%%%%%%%%%%%%%%%%%%%%%%%%%%%%%%%%
% \paragraph{Main File.}
%
% The main file is called |cdocsamp.tex|.
%
% Load the \textsf{childdoc} definitions and
% declare the filename for the main document:
%    \begin{macrocode}
\input{childdoc.def}
\childdocmain{}
%    \end{macrocode}

% Optional override for |\version| flag:
%    \begin{macrocode}
%%\ifchilddoc\else\providecommand{\version}{draft}\fi
%    \end{macrocode}

% Define the default values for the |\version| flag
% (|final| for the main file and |draft| for childs):
%    \begin{macrocode}
\ifchilddoc
\providecommand{\version}{draft}
\else
\providecommand{\version}{final}
\fi
%    \end{macrocode}

% Load the standard document class:
%    \begin{macrocode}
\documentclass[12pt]{article}
%    \end{macrocode}

% Start the document body:
%    \begin{macrocode}
\begin{document}
%    \end{macrocode}

% Declare a title page.
% Print title, part of document being processed and version flag:
%    \begin{macrocode}
\addtocounter{page}{-1}
\begin{center}
{\LARGE\bfseries{}childdoc example\par}
\vspace{1cm}
\ifchilddoc
\ifchilddocmanual part\else chapter\fi:
`\childdocname' of `\childdocjob'\par
\else
main document: `\childdocjob'\par
\fi
version: \version\par
\end{center}
\newpage
%    \end{macrocode}

% Manually include selected file,
% otherwise process as usual:
%    \begin{macrocode}
\ifchilddocmanual
\section*{part `\childdocname'}
\input{\childdocname}
\else
%    \end{macrocode}

% Include the two chapters:
%    \begin{macrocode}
\include{cdocsch1}
\include{cdocsch2}
%    \end{macrocode}

% Include the two parts unless only chapters should be displayed:
%    \begin{macrocode}
\ifchilddoc\else
\section{part three}
\input{cdocspt3}
\section{part four}
\input{cdocspt4}
\fi
%    \end{macrocode}

% Process as usual until here:
%    \begin{macrocode}
\fi
%    \end{macrocode}

% End of document body:
%    \begin{macrocode}
\end{document}
%    \end{macrocode}
%\iffalse
%</samplemain>
%\fi
%
% %%%%%%%%%%%%%%%%%%%%%%%%%%%%%%%%%%%%%%
% \paragraph{Chapter Include Files.}
%
% The include files are called |cdocsch1.tex| and |cdocsch2.tex|.
%
%\iffalse
%<*samplechap1|samplechap2>
%\fi

% Optional override for |\version| flag:
%    \begin{macrocode}
%%\providecommand{\version}{final}
%    \end{macrocode}

% Include the main document:
%    \begin{macrocode}
\input{childdoc.def}
\childdocof{cdocsamp}
%    \end{macrocode}

%\iffalse
%</samplechap1|samplechap2>
%\fi
%
%\iffalse
%<*samplechap1>
%\fi
% Some text for chapter 1:
%    \begin{macrocode}
\section{one}
some text in chapter one
%    \end{macrocode}

%\iffalse
%</samplechap1>
%\fi
% Some text for chapter 2:
%\iffalse
%<*samplechap2>
%\fi
%    \begin{macrocode}
\section{two}
more text in chapter two
%    \end{macrocode}

%\iffalse
%</samplechap2>
%\fi
%
% %%%%%%%%%%%%%%%%%%%%%%%%%%%%%%%%%%%%%%
% \paragraph{Part Include Files.}
%
% The include files are called |cdocspt3.tex| and |cdocspt4.tex|.
%
%\iffalse
%<*samplepart3|samplepart4>
%\fi

% Optional override for |\version| flag:
%    \begin{macrocode}
%%\providecommand{\version}{final}
%    \end{macrocode}

% Include the main document:
%    \begin{macrocode}
\input{childdoc.def}
\childdocby{cdocsamp}
%    \end{macrocode}

%\iffalse
%</samplepart3|samplepart4>
%\fi
%
%\iffalse
%<*samplepart3>
%\fi
% Some text for part 3:
%    \begin{macrocode}
some text in part three
%    \end{macrocode}

%\iffalse
%</samplepart3>
%\fi
% Some text for part 4:
%\iffalse
%<*samplepart4>
%\fi
%    \begin{macrocode}
more text in part four
%    \end{macrocode}

%\iffalse
%</samplepart4>
%\fi
%
% %%%%%%%%%%%%%%%%%%%%%%%%%%%%%%%%%%%%%%
% \paragraph{Forwarding for a Complete Draft.}
%
% The following forwarding file |cdocsdrf.tex|
% compiles the main document in draft mode:
%\iffalse
%<*sampledraft>
%\fi
%    \begin{macrocode}
\def\version{draft}
\input{childdoc.def}
\childdocforward{cdocsamp}
%    \end{macrocode}

%\iffalse
%</sampledraft>
%\fi
%
% %%%%%%%%%%%%%%%%%%%%%%%%%%%%%%%%%%%%%%
% \paragraph{Forwarding for Final Version of the Chapters.}
%
% The following forwarding files |cdocsfn1.tex| and |cdocsfn2.tex|
% (with identical content)
% compile the final versions of the child documents
% |cdocsch1.tex| and |cdocsch2.tex|, respectively:
%\iffalse
%<*samplefinal>
%\fi
%    \begin{macrocode}
\def\version{final}
\input{childdoc.def}
\childdocforwardprefix[cdocsamp]{cdocsfn}{cdocsch}
%    \end{macrocode}

%\iffalse
%</samplefinal>
%\fi
%
% %%%%%%%%%%%%%%%%%%%%%%%%%%%%%%%%%%%%%%
% \paragraph{Command Line Processing.}
%
% The following three command lines generate the output files
% |cdocscld|, |cdocscl1| and |cdocscl2|
% which should be identical to
% |cdocsdrf|, |cdocsch1| and |cdocsfn2|, respectively:
% \begin{center}
% \begin{tabular}{l}
% |latex -jobname cdocscld \|\\
% |  "\def\version{draft}\input{childdoc.def}\childdocforward{cdocsamp}"|\\
% |latex -jobname cdocscl1 \|\\
% |  "\input{childdoc.def}\childdocforward[cdocsamp]{cdocsch1}"|\\
% |latex -jobname cdocscl2 \|\\
% |  "\def\version{final}\input{childdoc.def}\childdocforward{cdocsch2}"|
% \end{tabular}
% \end{center}
% Note that the trailing backslash on each first line
% merely continues the input to the second line
% (for convenient cut ant paste).
% Furthermore, the command |latex| can be replaced by any
% of its alternative versions such as |pdflatex|.
%
% %%%%%%%%%%%%%%%%%%%%%%%%%%%%%%%%%%%%%%%%%%%%%%%%%%%%%%%%%%%%%%%%%%%%%%%%%%%%%%
% %%%%%%%%%%%%%%%%%%%%%%%%%%%%%%%%%%%%%%%%%%%%%%%%%%%%%%%%%%%%%%%%%%%%%%%%%%%%%%
% \section{Implementation}
%\iffalse
%<*package>
%\fi
%
% This section describes the definitions file |childdoc.def|.

% The definitions cannot be loaded using |\usepackage| or |\RequirePackage|
% which has a mechanism to prevent loading a style file more than once.
% When loading the definitions by means of |\input|
% multiple instances have to be prevented manually:
%\iffalse
%This code needs to be before the `\ProvidesFile' directive
%which is defined at the beginning of this file.
%Therefore it is also placed there and commented out here.
%</package>
%<*discard>
%\fi
%    \begin{macrocode}
\ifdefined\childdocmain\endinput\fi
%    \end{macrocode}
%\iffalse
%</discard>
%<*package>
%\fi
%
% \macro{\ifchilddoc}
% \macro{\ifchilddocmanual}
% The conditional |\ifchilddoc| tells whether a
% child (true) or main (false) document is being compiled.
% The conditional |\ifchilddocmanual| tells whether
% the |\includeonly| mechanism is used (false) or
% the selection of child files must be performed manually (true).
% The definitions initialise to false:
%    \begin{macrocode}
\newif\ifchilddoc
\newif\ifchilddocmanual
%    \end{macrocode}

% \macro{\childdocname}
% \macro{\childdocjob}
% The macro |\childdocname| stores the name of the main document
% to be compiled. The macro |\childdocjob| stores the name of
% the document on which the \LaTeX{} compiler was originally invoked.
% The content of |\jobname| cannot be compared
% to filenames specified in the source due to different catcodes.
% The following code rescans |\jobname|, stores the result
% in |\childdocname| and saves a copy in |\childdocjob|:
%    \begin{macrocode}
\edef\childdocname{\scantokens\expandafter{\jobname\noexpand}}
\let\childdocjob\childdocname
%    \end{macrocode}

% \macro{\childdocdisable}
% The macro |\childdocdisable| prevents the main file
% from being processed more than once.
% At this stage, the main document command |\childdocmain|
% is assumed to be called once again where it should do nothing.
% Any subsequent call to it should prevent
% a secondary processing of the main document
% It overwrites the forwarding commands
% |\childdocof| and |\childdocforward|
% with empty macros to prevent further inclusions of the main document:
%    \begin{macrocode}
\newcommand{\childdocdisable}
{
  \renewcommand{\childdocmain}[1]{\renewcommand{\childdocmain}[1]{\endinput}}
  \renewcommand{\childdocof}[1]{}
  \renewcommand{\childdocby}[2][]{}
  \renewcommand{\childdocforward}[2][]{}
  \renewcommand{\childdocdisable}{}
}
%    \end{macrocode}

% \macro{\childdocmain}
% The macro |\childdocmain| is to be called at the top of the main file
% with nothing or the main filename (without extension) as argument.
% First, it breaks loops.
% If the argument is not empty and does not match |\childdocname|
% (which is set by the first inclusion of |childdoc.def|),
% |\ifchilddoc| is set to true, |\includeonly| is applied to the child file
% and |\jobname| is set to the main file
% (for proper handling of |.aux| files):
%    \begin{macrocode}
\newcommand{\childdocmain}[1]
{
  \childdocdisable\childdocmain{}
  \if?#1?\else
    \begingroup
      \def\childdoctmp{#1}
      \ifx\childdoctmp\childdocname
        \def\childdoctmp{}
      \else
        \def\childdoctmp
        {
          \childdoctrue
          \includeonly{\childdocname}
          \def\childdocjob{#1}
          \def\jobname{#1}
        }
      \fi
      \expandafter
    \endgroup
    \childdoctmp
  \fi
}
%    \end{macrocode}

% \macro{\childdocof}
% The command |\childdocof| redirects
% compilation to the main file |#1|.
%    \begin{macrocode}
\newcommand{\childdocof}[1]
{
  \childdocdisable
  \childdoctrue
  \includeonly{\childdocname}
  \def\jobname{#1}
  \def\childdocjob{#1}
  \input{#1}
}
%    \end{macrocode}

% \macro{\childdocby}
% The command |\childdocby| ....
%    \begin{macrocode}
\newcommand{\childdocby}[2][]
{
  \childdocdisable
  \childdoctrue
  \childdocmanualtrue
  \if?#1?\else
    \def\jobname{#2}
  \fi
  \def\childdocjob{#2}
  \input{#2}
  \endinput
}
%    \end{macrocode}

% \macro{\childdocforward}
% The command |\childdocforward| redirects
% compilation to the main file or
% (if the optional argument is given) a child file.
% Parameters are set as if the main file
% or a child file starting with |\childdocof| was compiled.
% Then compilation is handed over to the main file:
%    \begin{macrocode}
\newcommand{\childdocforward}[2][]
{
  \begingroup
    \if?#1?
      \def\childdoctmp
      {
        \def\childdocname{#2}
        \def\childdocjob{#2}
        \def\jobname{#2}
        \input{#2}
        \endinput
      }
    \else
      \def\childdoctmp
      {
        \childdocdisable
        \def\childdocname{#2}
        \childdoctrue
        \includeonly{#2}
        \def\childdocjob{#1}
        \def\jobname{#1}
        \input{#1}
        \endinput
      }
    \fi
    \expandafter
  \endgroup
  \childdoctmp
}
%    \end{macrocode}

% \macro{\childdocforwardprefix}
% The command |\childdocforwardprefix| redirects
% compilation to the main or a child file by means of a pattern.
% The prefix |#1| in the current filename is replaced by |#2|
% and the suffix of the current filename is kept
% (it is assumed that the filename does not contain the substring `|~~~|'
% which is used as a delimiter).
% Compilation is handed over to the new file by |\childdocforward|:
%    \begin{macrocode}
\newcommand{\childdocforwardprefix}[3][]
{
  \begingroup
    \def\childdocextract #2##1~~~{\def\childdoctmp{\childdocforward[#1]{#3##1}}}
    \expandafter\childdocextract\childdocname~~~
    \expandafter
  \endgroup
  \childdoctmp
}
%    \end{macrocode}

% \macro{\childdoc}
% The deprecated macro |\childdoc| is a legacy version of |\childdocmain|:
%    \begin{macrocode}
\newcommand{\childdoc}{\childdocmain}
%    \end{macrocode}

% \macro{\childdocredirect}
% The deprecated macro |\childdocredirect| is a legacy version
% of |\childdocforward| and |\childdocforwardprefix|:
%    \begin{macrocode}
\newcommand{\childdocredirect}[2][]
{
  \begingroup
    \if?#1?
      \def\childdoctmp{\childdocforward{#2}}
    \else
      \def\childdoctmp{\childdocforwardprefix{#1}{#2}}
    \fi
    \expandafter
  \endgroup
  \childdoctmp
}
%    \end{macrocode}

%\iffalse
%</package>
%\fi
%
\endinput
|\\
|\childdocforwardprefix[|\textit{main}|]{|\textit{prefix}|}{|\textit{dest}|}|
\end{tabular}
\end{center}
%
the destination file is determined by a pattern
depending on the current file:
To make this work, the current file must be called
`{\textit{prefix}\hspace{0.2em}\textit{suffix}}'
with \textit{prefix} matching precisely the argument.
Processing is then passed on to the file
`{\textit{dest}\hspace{0.2em}\textit{suffix}}'.
Surely, the same effect is achieved by
directly specifying the
argument `{\textit{dest}\hspace{0.2em}\textit{suffix}}'
in the first form.
However, that requires to set up a different file
for each child. With the alternative form of the command
all these files can have exactly the same content
which simplifies setting them up and maintaining them.

For example, the following file |draft.tex|
with a compilation flag |\version| as described in \secref{sec:flags}
compiles the main document as a draft:
%
\begin{center}
\begin{tabular}{l}
|\def\version{draft}|\\
|% \iffalse
%
% childdoc.dtx Copyright (C) 2017-2018 Niklas Beisert
%
% This work may be distributed and/or modified under the
% conditions of the LaTeX Project Public License, either version 1.3
% of this license or (at your option) any later version.
% The latest version of this license is in
%   http://www.latex-project.org/lppl.txt
% and version 1.3 or later is part of all distributions of LaTeX
% version 2005/12/01 or later.
%
% This work has the LPPL maintenance status `maintained'.
%
% The Current Maintainer of this work is Niklas Beisert.
%
% This work consists of the files childdoc.dtx and childdoc.ins
% and the derived files childdoc.def and cdocsamp.tex with
% cdocsch1.tex, cdocsch2.tex, cdocsdrf.tex, cdocsfn1.tex, cdocsfn2.tex.
%
%<package>\ifdefined\childdocmain\endinput\fi
%<package>\ProvidesFile{childdoc.def}[2018/12/30 v2.0 child document driver]
%<samplemain>\ProvidesFile{cdocsamp.tex}[2018/12/30 v2.0 sample for childdoc]
%<*driver>
%\ProvidesFile{childdoc.drv}[2018/12/30 v2.0 childdoc reference manual file]
\PassOptionsToClass{10pt,a4paper}{article}
\documentclass{ltxdoc}

\usepackage[margin=35mm]{geometry}
\usepackage{hyperref}
\usepackage{hyperxmp}
\usepackage[usenames]{color}

\hypersetup{colorlinks=true}
\hypersetup{pdfstartview=FitH}
\hypersetup{pdfpagemode=UseNone}
\hypersetup{pdfsource={}}
\hypersetup{pdflang={en-UK}}
\hypersetup{pdfcopyright={Copyright 2017-2018 Niklas Beisert.
  This work may be distributed and/or modified under the
  conditions of the LaTeX Project Public License, either version 1.3
  of this license or (at your option) any later version.}}
\hypersetup{pdflicenseurl={http://www.latex-project.org/lppl.txt}}
\hypersetup{pdfcontactaddress={ETH Zurich, ITP, HIT K,
  Wolfgang-Pauli-Strasse 27}}
\hypersetup{pdfcontactpostcode={8093}}
\hypersetup{pdfcontactcity={Zurich}}
\hypersetup{pdfcontactcountry={Switzerland}}
\hypersetup{pdfcontactemail={nbeisert@itp.phys.ethz.ch}}
\hypersetup{pdfcontacturl={http://people.phys.ethz.ch/\xmptilde nbeisert/}}

\newcommand{\secref}[1]{\hyperref[#1]{section \ref*{#1}}}

\parskip1ex
\parindent0pt
\let\olditemize\itemize
\def\itemize{\olditemize\parskip0pt}

\begin{document}

\title{The \textsf{childdoc} Package}
\hypersetup{pdftitle={The childdoc Package}}
\author{Niklas Beisert\\[2ex]
  Institut f\"ur Theoretische Physik\\
  Eidgen\"ossische Technische Hochschule Z\"urich\\
  Wolfgang-Pauli-Strasse 27, 8093 Z\"urich, Switzerland\\[1ex]
  \href{mailto:nbeisert@itp.phys.ethz.ch}
  {\texttt{nbeisert@itp.phys.ethz.ch}}}
\hypersetup{pdfauthor={Niklas Beisert}}
\hypersetup{pdfsubject={Manual for the LaTeX2e Package childdoc}}
\date{30 December 2018, \textsf{v2.0}}
\maketitle

\begin{abstract}\noindent
\textsf{childdoc} is a \LaTeXe{} package
that enables the direct compilation
of document sections included by |\include|
to individual files.
\end{abstract}

\begingroup
\parskip0ex
\tableofcontents
\endgroup

%%%%%%%%%%%%%%%%%%%%%%%%%%%%%%%%%%%%%%%%%%%%%%%%%%%%%%%%%%%%%%%%%%%%%%%%%%%%%%%%
%%%%%%%%%%%%%%%%%%%%%%%%%%%%%%%%%%%%%%%%%%%%%%%%%%%%%%%%%%%%%%%%%%%%%%%%%%%%%%%%
\section{Introduction}

\LaTeX{} provides a mechanism to structure a large document (such as a book)
into a main file and several child files (containing the chapters)
using the |\include| command.
This mechanism is beneficial for documents
which span hundreds of pages in order to
make the source file(s) more manageable.
Moreover, compilation can be restricted to
selected child files by means of the |\includeonly| command.
The latter feature can be used to reduce the compilation time while editing
(this was significantly more useful in the earlier days of \LaTeX{})
or to generate a smaller document which is easier to navigate.
Another application of |\includeonly| is to generate
documents consisting of selected parts of the complete document.

However, there are a few drawbacks of the plain |\include| mechanism:
\begin{itemize}
\item
The child files cannot be compiled on their own,
they can only be compiled via the main file.
A naive editing environment
(such as a text editor with an option
to have the current file processed by \LaTeX)
may require one to switch to the main file before compiling;
attempting to compile the child file produces errors.
\item
The main file must be modified (each time)
to adjust the |\includeonly| command
to the present needs. This easily leaves the main file in a messy state.
\item
The generated document will always carry the filename
of the main document. This is inconvenient if
several child files are to be compiled and
to be kept for distribution.
\end{itemize}

The present package provides a simple interface
to make child files individually compilable by \LaTeX{}.
Compiling a child file then has the same effect as compiling
the main file with an |\includeonly| command
to select the appropriate child.
Moreover the generated document will carry the name of the child
rather than the main file.
This resolves all three above issues.

This feature is meant to make the editing of books,
thesis documents and lecture notes somewhat more convenient.
However, the package can also be used efficiently for
composing a series of documents (such as exercise sheets)
which are typically distributed individually.
It then assists the author in generating the individual documents
(potentially in different versions)
as well as a document containing the collected series.
Another application is in developing style files
or other kinds of included material
where compilation of the style file could redirect
to a sample or test file.

%%%%%%%%%%%%%%%%%%%%%%%%%%%%%%%%%%%%%%%%%%%%%%%%%%%%%%%%%%%%%%%%%%%%%%%%%%%%%%%%
%%%%%%%%%%%%%%%%%%%%%%%%%%%%%%%%%%%%%%%%%%%%%%%%%%%%%%%%%%%%%%%%%%%%%%%%%%%%%%%%
\section{Usage}

First of all, the package \textsf{childdoc} is \emph{not} a standard
\LaTeXe{} |.sty| style file! Therefore it needs to be invoked in
a non-standard way.

%%%%%%%%%%%%%%%%%%%%%%%%%%%%%%%%%%%%%%%%%%%%%%%%%%%%%%%%%%%%%%%%%%%%%%%%%%%%%%%%
\subsection{Included Files}
\label{sec:include}

%%%%%%%%%%%%%%%%%%%%%%%%%%%%%%%%%%%%%%%%
\DescribeMacro{\childdocmain}
To use the package, add the commands
\begin{center}
\begin{tabular}{l}
|\input{childdoc.def}|\\
|\childdocmain{}|\\
\end{tabular}
\end{center}
at the very top of the main \LaTeX{} file,
in particular \emph{before} the |\documentclass| statement!
The argument of |\childdocmain| should be left empty
(but it must be present).

%%%%%%%%%%%%%%%%%%%%%%%%%%%%%%%%%%%%%%%%
\DescribeMacro{\childdocof}
Furthermore, add the commands
\begin{center}
\begin{tabular}{l}
|\input{childdoc.def}|\\
|\childdocof{|\textit{main}|}|\\
\end{tabular}
\end{center}
at the top of every child file \textit{child}
which is included by |\include{|\textit{child}|}|
from within the main file
(or at least for those files to be compiled individually).
The argument \textit{main} must be the filename of the main file.

There are a couple of
considerations in setting up the main and child documents:

%%%%%%%%%%%%%%%%%%%%%%%%%%%%%%%%%%%%%%%%
\paragraph{Restrictions.}

Please note the following restrictions:
\begin{itemize}
\item
|\childdocmain| must be called with one argument \textit{main}
to ensure compatibility with earlier version of the package.
It must either be empty (|\childdocmain{}|)
or precisely match the filename of the main file in which it is specified.
See \secref{sec:detection} for further information.
\item
The filename \textit{main} must be specified without the |.tex| extension.
\item
The filename \textit{main} is case sensitive
(even in case-insensitive file systems)
due to internal string comparison.
\item
The argument \textit{main} should be fully expanded, it cannot be a macro.
\item
Subdirectories and special characters should be avoided in filenames.
\item
The command |\childdocmain{|\textit{main}|}| must be followed by a whitespace.
It should not be followed immediately by another command
or by a comment mark `|%|'.
This is because the \TeX{} parser reads the token immediately following
the argument of |\childdocmain| and puts it
at the beginning of every child section;
however, a white\-space is ignored.
\end{itemize}

%%%%%%%%%%%%%%%%%%%%%%%%%%%%%%%%%%%%%%%%
\paragraph{Content of Main File.}

It is advisable to place all content in the child files included by |\include|.
Any output contained in the main file will appear in all child documents
unless suppressed manually;
it cannot be suppressed automatically by the |\includeonly| directive
and thus should normally be avoided.
A method to include some content in the main file
by means of conditional processing is described in \secref{sec:conditional}.

%%%%%%%%%%%%%%%%%%%%%%%%%%%%%%%%%%%%%%%%
\paragraph{Page Numbering.}

When only a part of the document is compiled,
the appropriate numbering of pages
(as well as other status parameters)
is determined from the |.aux| files.
The latter contain information from previous passes.
However this information needs to propagate through
all intermediate child documents.
Therefore the page numbering in child documents may well
be inconsistent until the complete document is compiled at least once.

A useful (if unconventional) way to always ensure a consistent
page numbering is to restart the numbering in each child document
and denote the pages by `\textit{child}|.|\textit{page}'
where \textit{child} represents the chapter/section number of the child file.
This can be achieved by the command
|\numberwithin{page}{|\textit{child}|}|
of the \textsf{amsmath} package
where \textit{child} can be |chapter| or |section|
depending on the chosen structuring.
Alternatively, one can modify the macro |\thepage| appropriately
and reset the counter |page| at the start of each child file.

%%%%%%%%%%%%%%%%%%%%%%%%%%%%%%%%%%%%%%%%%%%%%%%%%%%%%%%%%%%%%%%%%%%%%%%%%%%%%%%%
\subsection{Conditional Processing}
\label{sec:conditional}

The package provides a mechanism to compile different versions
of a document. To customise the versions further some conditional processing
can come in handy to distinguish which version is being compiled.
The package provides two macros to describe the compilation context:

%%%%%%%%%%%%%%%%%%%%%%%%%%%%%%%%%%%%%%%%
\DescribeMacro{\ifchilddoc}
The conditional |\ifchilddoc| distinguishes between the compilation of
child documents and the main document:
%
\begin{center}
|\ifchilddoc |\textit{child-code}| |[|\||else |\textit{main-code}]| \||fi|
\end{center}

%%%%%%%%%%%%%%%%%%%%%%%%%%%%%%%%%%%%%%%%
\DescribeMacro{\childdocname}
\DescribeMacro{\childdocjob}
The macro |\childdocname| contains the filename (without extension)
of the main or child file being processed.
Note that |\childdocjob| will always contain the name of the main file.

%%%%%%%%%%%%%%%%%%%%%%%%%%%%%%%%%%%%%%%%
\paragraph{Title Page.}

Conditional processing can be used to include a title or banner page
in the main document when proper precautions are taken.
Importantly, the code in the main file should ensure that the page counter
(as well as other status parameters which are stored in the |.aux| files)
takes the same value after the conditional processing.
Otherwise the page numbers may take divergent values
depending on which part is compiled.

For example, a title page could be declared by:
%
\begin{center}
\begin{tabular}{l}
|\ifchilddoc\||else|\\
|\addtocounter{page}{-1}|\\
\textit{code for title page}\\
|\newpage|\\
|\||fi|
\end{tabular}
\end{center}
%
A banner page for the child documents can be generated by:
%
\begin{center}
\begin{tabular}{l}
|\ifchilddoc|\\
|\addtocounter{page}{-1}|\\
\textit{code for banner page}\\
|\newpage|\\
|\||fi|
\end{tabular}
\end{center}
%
Here one could write a message such as:
\begin{center}
|This is the part \childdocname{} of \childdocjob{}.|
\end{center}

%%%%%%%%%%%%%%%%%%%%%%%%%%%%%%%%%%%%%%%%%%%%%%%%%%%%%%%%%%%%%%%%%%%%%%%%%%%%%%%%
\subsection{Flags}
\label{sec:flags}

The package makes it easy to generate different versions
of the main or child documents.
To this end compilation flags can be defined
and assigned different default values.
They will be particularly useful in conjunction
with the forwarding mechanism described in \secref{sec:forward}.

For example, it may be useful to have a flag |\version|
which can be set to |draft| or |final|.
The document source will contain some conditional code
depending on the value of |\version|.
Suppose further, the flag should default to |final| for the main file
and to |draft| for child files
which is a natural assignment for editing the document.
This is achieved by placing the following code
in the preamble of the main document
(below the |\childdocmain| directive):
%
\begin{center}
\begin{tabular}{l}
|\ifchilddoc|\\
|\providecommand{\version}{draft}|\\
|\||else|\\
|\providecommand{\version}{final}|\\
|\||fi|
\end{tabular}
\end{center}
%
The definition by |\providecommand| makes sure
that previous definitions are not overwritten.
Further statements |\providecommand{\version}{...}|
can thus be added before the above code to override it.

For the main file, one might add a line
(between |\childdocmain| and the above block)
%
\begin{center}
|%\ifchilddoc\||else\providecommand{\version}{draft}\||fi|
\end{center}
%
which can be uncommented to produce a draft version.
Likewise one can add a line to the very top of a child file
(above the |\childdocof{|\textit{main}|}| directive)
%
\begin{center}
|%\providecommand{\version}{final}|
\end{center}
%
which can be uncommented to produce the final version of this child document.

%%%%%%%%%%%%%%%%%%%%%%%%%%%%%%%%%%%%%%%%%%%%%%%%%%%%%%%%%%%%%%%%%%%%%%%%%%%%%%%%
\subsection{Forwarding}
\label{sec:forward}

Different versions of the main or child documents
using compilation flags as described in \secref{sec:flags}
can be (permanently) stored in different files
for convenient compilation, viewing and distribution.
To this end, the package defines a command
to pass on compilation to a different file:

%%%%%%%%%%%%%%%%%%%%%%%%%%%%%%%%%%%%%%%%
\DescribeMacro{\childdocforward}
The command |\childdocforward| redirects processing to
another source file:
%
\begin{center}
\begin{tabular}{l}
|\input{childdoc.def}|\\
|\childdocforward[|\textit{main}|]{|\textit{dest}|}|\\
\end{tabular}
\end{center}
%
The argument \textit{dest} is the destination file
(without extension).
It should be the main file or one of the child files.
Note that further \textsf{childdoc} directives
such as |\childdocof| and |\childdocforward|
in the indicated file will be processed in this form.
The optional argument \textit{main}
passes on directly to the main file \textit{main}
while pretending to compile the child \textit{dest}.
This form behaves as if \textit{dest}
issues |\childdocof{|\textit{main}|}| right away,
and no further \textsf{childdoc} directives will be processed.

%%%%%%%%%%%%%%%%%%%%%%%%%%%%%%%%%%%%%%%%
\DescribeMacro{\...prefix}
In the alternative form |\childdocforwardprefix|,
%
\begin{center}
\begin{tabular}{l}
|\input{childdoc.def}|\\
|\childdocforwardprefix[|\textit{main}|]{|\textit{prefix}|}{|\textit{dest}|}|
\end{tabular}
\end{center}
%
the destination file is determined by a pattern
depending on the current file:
To make this work, the current file must be called
`{\textit{prefix}\hspace{0.2em}\textit{suffix}}'
with \textit{prefix} matching precisely the argument.
Processing is then passed on to the file
`{\textit{dest}\hspace{0.2em}\textit{suffix}}'.
Surely, the same effect is achieved by
directly specifying the
argument `{\textit{dest}\hspace{0.2em}\textit{suffix}}'
in the first form.
However, that requires to set up a different file
for each child. With the alternative form of the command
all these files can have exactly the same content
which simplifies setting them up and maintaining them.

For example, the following file |draft.tex|
with a compilation flag |\version| as described in \secref{sec:flags}
compiles the main document as a draft:
%
\begin{center}
\begin{tabular}{l}
|\def\version{draft}|\\
|\input{childdoc.def}|\\
|\childdocforward{|\textit{main}|}|
\end{tabular}
\end{center}
%
Likewise, the following files |final|\textit{nn}|.tex|
compile the final version of the child document
|child|\textit{nn}|.tex|:
%
\begin{center}
\begin{tabular}{l}
|\def\version{final}|\\
|\input{childdoc.def}|\\
|\childdocforwardprefix{final}{child}|
\end{tabular}
\end{center}
%

Note that when several versions of a main file and/or of each child file
are to be generated, it may be convenient to set up a |Makefile| or
shell script to automatise the process.

%%%%%%%%%%%%%%%%%%%%%%%%%%%%%%%%%%%%%%%%%%%%%%%%%%%%%%%%%%%%%%%%%%%%%%%%%%%%%%%%
\subsection{Command Line Processing}
\label{sec:commandline}

The effect of redirection files can also be achieved by invoking
the \LaTeX{} compiler with a more elaborate command line.
Most conveniently this should be done as part
of a shell script or a |Makefile|.

When using \textsf{childdoc} in the main file, the following
command lines effectively perform a redirection
(note that depending on the shell being used,
backslashes may have to be doubled: `|\|' $\to$ `|\\|'):
%
\begin{center}
|... -jobname "|\textit{target}|" |\\|"|[\textit{flags}]%
|\input{childdoc.def}\childdocforward[|\textit{main}|]{|\textit{dest}|}"|
\end{center}
%
Here \textit{target} is the name of the output file,
\textit{main} is the name of the main file
and \textit{dest} is the name of the main or child file to be processed
(all filenames without extensions).
The optional argument \textit{main} can be omitted
if \textit{main} matches \textit{dest}.
Optionally, compilation \textit{flags} can be defined via |\def| commands.
This command line makes the \TeX{} engine believe
it is compiling the file \textit{target}
whose content is specified as the latter parameter.
The provided code then forwards the processing to
\textit{main} or \textit{dest} as described in \secref{sec:forward}.

%%%%%%%%%%%%%%%%%%%%%%%%%%%%%%%%%%%%%%%%%%%%%%%%%%%%%%%%%%%%%%%%%%%%%%%%%%%%%%%%
\subsection{Include by Input}
\label{sec:input}

Including child documents by |\include| has some restrictions by design.
Most notably, the content of a child document always occupies
its own set of pages; pages cannot be shared between child documents.
Usually, this behaviour makes perfect sense
because each child document contain an essential part of the document.
However, in some situations it may be desirable to compose
a document from a collection of parts
without having mandatory page breaks between then.
For this case, the package
provides a mechanism to include parts
by |\input| which can also be processed individually.
However, by construction this mechanism
requires manual handling of the content to be output.

%%%%%%%%%%%%%%%%%%%%%%%%%%%%%%%%%%%%%%%%
\DescribeMacro{\ifchilddocmanual}
The main file should be prepared as usual, see \secref{sec:include}.
However, the document body must make a distinction
between processing of an individual part and of the main document, e.g.:
%
\begin{center}
\begin{tabular}{l}
|\ifchilddocmanual|\\
|\input{\childdocname}|\\
|\||else|\\
\textit{document body with }|\input{|\textit{part}|}|\\
|\||fi|
\end{tabular}
\end{center}
%
The conditional |\ifchilddocmanual| is true whenever
a part to be included by |\input| is being compiled,
and the name of the part is stored in |\childdocname|.

%%%%%%%%%%%%%%%%%%%%%%%%%%%%%%%%%%%%%%%%
\DescribeMacro{\childdocby}
Each part to be included by |\input| should start with:
%
\begin{center}
\begin{tabular}{l}
|\input{childdoc.def}|\\
|\childdocby{|\textit{main}|}|\\
\end{tabular}
\end{center}
%
The directive |\childdocby| is similar to |\childdocof|
described in \secref{sec:include},
but the subsequent selection of content must be done manually.
To that end, both |\ifchilddoc| and |\ifchilddocmanual|
will be true upon processing of a part,
and the name of the part is stored in |\childdocname|.
Note that |\jobname| will be set to the filename of the current part
so that each part receives an individual |.aux| file
that does not interfere with the |.aux| file(s) of the main document.
This behaviour can be altered by the alternative form
|\childdocby[*]{|\textit{main}|}| (with a non-empty optional argument)
which uses the |.aux| file of the main document
by setting |\jobname| to \textit{main}.

%%%%%%%%%%%%%%%%%%%%%%%%%%%%%%%%%%%%%%%%%%%%%%%%%%%%%%%%%%%%%%%%%%%%%%%%%%%%%%%%
\subsection{Driver Development}
\label{sec:driver}

The \textsf{childdoc} mechanism can also be use for the development
of definition files such as \LaTeX{} styles or classes.
This case differs from the above setup with multiple parts
included by |\include| in that no |\includeonly| should be invoked.
This can be achieved by starting the include file
(before |\ProvidesPackage|) with:
%
\begin{center}
\begin{tabular}{l}
|\input{childdoc.def}|\\
|\childdocforward{|\textit{main}|}|\\
\end{tabular}
\end{center}
%
or alternatively with:
%
\begin{center}
\begin{tabular}{l}
|\input{childdoc.def}|\\
|\childdocby{|\textit{main}|}|\\
\end{tabular}
\end{center}
%
Both forms have slightly different effects as described above.
The main file is prepared as usual, see \secref{sec:include}.

%%%%%%%%%%%%%%%%%%%%%%%%%%%%%%%%%%%%%%%%%%%%%%%%%%%%%%%%%%%%%%%%%%%%%%%%%%%%%%%%
\subsection{Legacy Detection}
\label{sec:detection}

The directive |\childdocmain| in the main file can detect
whether the complete document or merely a child is to be compiled
even without using the directive |\childdocof|.
This method is deprecated because it is less robust
and there is no compelling reason to use it;
it is merely provided for backward compatibility
and it may be removed in future versions.

If the detection mechanism is to be used,
it is mandatory to correctly specify
the filename of the main file as the argument of |\childdocmain|:
%
\begin{center}
\begin{tabular}{l}
|\input{childdoc.def}|\\
|\childdocmain{|\textit{main}|}|\\
\end{tabular}
\end{center}
%
If |\jobname| does not match the argument \textit{main} of |\childdocmain|,
it is assumed that |\jobname| points to the child file to be compiled.
When using |\childdocmain| with the main file specified as argument,
it suffices to start a child file
with just |\input{|\textit{main}|}|
without loading of the package and using |\childdocof|.
If instead all processing is done
with the appropriate \textsf{childdoc} directives,
the argument of \textit{main} of |\childdocmain| can be empty.

An alternative version of the command line processing described
in \secref{sec:commandline} using the detection mechanism reads:
%
\begin{center}
|... -jobname "|\textit{target}|" "|[\textit{flags}]%
[|\def\jobname{|\textit{dest}|}|]|\input{|\textit{main}|}"|
\end{center}

%%%%%%%%%%%%%%%%%%%%%%%%%%%%%%%%%%%%%%%%%%%%%%%%%%%%%%%%%%%%%%%%%%%%%%%%%%%%%%%%
\subsection{Manual Code}
\label{sec:manual}

In case one cannot be certain whether the definitions file |childdoc.def|
is installed on the target \TeX{} distribution
and one prefers not to ship it,
it is conceivable to paste a few relevant commands into the sources.

To that end, drop all statements |\input{childdoc.def}|
and perform the replacements as outlined below.
Instead of |\childdocmain{|\textit{main}|}| add the following code
to the top of the main file:
%
\begin{center}
\begin{tabular}{l}
|\||ifdefined\childdocname\endinput\||fi\newif\ifchilddoc|\\
|\edef\childdocname{\scantokens\expandafter{\jobname\noexpand}}|\\
|\def\childdocmain{|\textit{main}|}\||ifx\childdocmain\childdocname\||else|\\
|\childdoctrue\includeonly{\childdocname}\let\jobname\childdocmain\||fi|\\
\end{tabular}
\end{center}
%
Instead of |\childdocof{|\textit{main}|}| just include the main file
at the top of each child file:
%
\begin{center}
|\input{|\textit{main}|}|
\end{center}
%
A simple redirection |\childdocforward{|\textit{dest}|}| is achieved by:
%
\begin{center}
|\def\jobname{|\textit{dest}|}\input{\jobname}|
\end{center}
%
The redirection with prefix
|\childdocforwardprefix[|\textit{prefix}|]{|\textit{dest}|}|
is accomplished by:
%
\begin{center}
\begin{tabular}{l}
|{\edef\jobname{\scantokens\expandafter{\jobname\noexpand}}|\\
|\def\redirectjob |\textit{prefix}|#1~~~{\gdef\jobname{|\textit{dest}|#1}}|\\
|\expandafter\redirectjob\jobname~~~}\input{\jobname}|
\end{tabular}
\end{center}

In an alternative approach,
child documents can be compiled by a specific command line
without additional code or specific definitions:
%
\begin{center}
|... -jobname "|\textit{target}|" "|[\textit{flags}]%
|\includeonly{|\textit{dest}|}\input{|\textit{main}|}"|
\end{center}
%

%%%%%%%%%%%%%%%%%%%%%%%%%%%%%%%%%%%%%%%%%%%%%%%%%%%%%%%%%%%%%%%%%%%%%%%%%%%%%%%%
%%%%%%%%%%%%%%%%%%%%%%%%%%%%%%%%%%%%%%%%%%%%%%%%%%%%%%%%%%%%%%%%%%%%%%%%%%%%%%%%
\section{Information}

%%%%%%%%%%%%%%%%%%%%%%%%%%%%%%%%%%%%%%%%%%%%%%%%%%%%%%%%%%%%%%%%%%%%%%%%%%%%%%%%
\subsection{Copyright}

Copyright \copyright{} 2017--2018 Niklas Beisert

This work may be distributed and/or modified under the
conditions of the \LaTeX{} Project Public License, either version 1.3
of this license or (at your option) any later version.
The latest version of this license is in
  \url{http://www.latex-project.org/lppl.txt}
and version 1.3 or later is part of all distributions of \LaTeX{}
version 2005/12/01 or later.

This work has the LPPL maintenance status `maintained'.

The Current Maintainer of this work is Niklas Beisert.

This work consists of the files |README.txt|, |childdoc.ins| and |childdoc.dtx|
as well as the derived files |childdoc.def|, |cdocsamp.tex|
with |cdocsch1.tex|, |cdocsch2.tex|, |cdocspt3.tex|, |cdocspt4.tex|,
|cdocsdrf.tex|, |cdocsfn1.tex|, |cdocsfn2.tex|
as well as |childdoc.pdf|.

%%%%%%%%%%%%%%%%%%%%%%%%%%%%%%%%%%%%%%%%%%%%%%%%%%%%%%%%%%%%%%%%%%%%%%%%%%%%%%%%
\subsection{Files and Installation}

The package consists of the files:
%
\begin{center}
\begin{tabular}{ll}
    |README.txt|   & readme file \\
    |childdoc.ins| & installation file \\
    |childdoc.dtx| & source file \\
    |childdoc.def| & definition file \\
    |cdocsamp.tex| & sample main file \\
    |cdocsch1.tex| & sample include file \\
    |cdocsch2.tex| & sample include file \\
    |cdocspt3.tex| & sample part file \\
    |cdocspt4.tex| & sample part file \\
    |cdocsdrf.tex| & sample redirection file \\
    |cdocsfn1.tex| & sample redirection file \\
    |cdocsfn2.tex| & sample redirection file \\
    |childdoc.pdf| & manual
\end{tabular}
\end{center}
%
The distribution consists of the files
|README.txt|, |childdoc.ins| and |childdoc.dtx|.
%
\begin{itemize}
\item
Run (pdf)\LaTeX{} on |childdoc.dtx|
to compile the manual |childdoc.pdf| (this file).
\item
Run \LaTeX{} on |childdoc.ins| to create the definitions file |childdoc.def|
and the sample |cdocsamp.tex| with include files
|cdocsch1.tex|, |cdocsch2.tex|, |cdocspt3.tex|, |cdocspt4.tex|,
|cdocsdrf.tex|, |cdocsfn1.tex|, |cdocsfn2.tex|.
Then copy the file |childdoc.def| to an appropriate directory of your \LaTeX{}
distribution, e.g.\ \textit{texmf-root}|/tex/latex/childdoc|.
\end{itemize}

%%%%%%%%%%%%%%%%%%%%%%%%%%%%%%%%%%%%%%%%%%%%%%%%%%%%%%%%%%%%%%%%%%%%%%%%%%%%%%%%
\subsection{Related CTAN Packages}

There are several other packages which offer a similar functionality:
%
\begin{itemize}
\item
The packages
\href{http://ctan.org/pkg/docmute}{\textsf{docmute}},
\href{http://ctan.org/pkg/includex}{\textsf{includex}} and
\href{http://ctan.org/pkg/standalone}{\textsf{standalone}}
provide commands to include only the document body of
a child file thus allowing both files to be compiled individually.
\item
The packages \href{http://ctan.org/pkg/subdocs}{\textsf{subdocs}}
and \href{http://ctan.org/pkg/subfiles}{\textsf{subfiles}}
provide structures in which the main and child documents can be
encapsulated and allowing them to be compiled individually.
The inclusion mechanism is different from the conventional |\include|.
\item
The package \href{http://ctan.org/pkg/combine}{\textsf{combine}}
is an elaborate solution to combine several documents into one.
\end{itemize}
%
See also the CTAN topic \href{http://ctan.org/topic/subdocs}{\textsf{subdocs}}
for further related packages.
The present package differs from the above solutions in that
a document structure constructed with the conventional |\include| mechanism
just needs two extra commands at the top of every file
such that all constituent files can be compiled individually.

%%%%%%%%%%%%%%%%%%%%%%%%%%%%%%%%%%%%%%%%%%%%%%%%%%%%%%%%%%%%%%%%%%%%%%%%%%%%%%%%
%\subsection{Feature Suggestions}
%
%The following is a list of features which may be useful for future
%versions of this package:
%%
%\begin{itemize}
%\item
%\ldots
%\end{itemize}

%%%%%%%%%%%%%%%%%%%%%%%%%%%%%%%%%%%%%%%%%%%%%%%%%%%%%%%%%%%%%%%%%%%%%%%%%%%%%%%%
\subsection{Revision History}

%%%%%%%%%%%%%%%%%%%%%%%%%%%%%%%%%%%%%%%%
\paragraph{v2.0:} 2018/12/30

\begin{itemize}
\item
immediate forward processing
\item
added |\childdocby| mechanism
\item
manual restructured
\end{itemize}

%%%%%%%%%%%%%%%%%%%%%%%%%%%%%%%%%%%%%%%%
\paragraph{v1.6:} 2018/01/17

\begin{itemize}
\item
application for development of include files
\item
corrections to manual
\end{itemize}

%%%%%%%%%%%%%%%%%%%%%%%%%%%%%%%%%%%%%%%%
\paragraph{v1.5:} 2017/05/21

\begin{itemize}
\item
more complete structuring introduced
\item
|\childdocof| introduced
\item
|\childdoc| renamed to |\childdocmain|
\item
|\childredirect| renamed to |\childdocforward| and |\childdocforwardprefix|
and functionality expanded
\end{itemize}

%%%%%%%%%%%%%%%%%%%%%%%%%%%%%%%%%%%%%%%%
\paragraph{v1.0:} 2017/04/27

\begin{itemize}
\item
manual and install package
\item
first version published on CTAN
\end{itemize}

%%%%%%%%%%%%%%%%%%%%%%%%%%%%%%%%%%%%%%%%
\paragraph{v0.6:} 2017/04/26

\begin{itemize}
\item
redirection mechanism added
\end{itemize}

%%%%%%%%%%%%%%%%%%%%%%%%%%%%%%%%%%%%%%%%
\paragraph{v0.5:} 2017/04/26

\begin{itemize}
\item
functionality in definition file
\end{itemize}


%%%%%%%%%%%%%%%%%%%%%%%%%%%%%%%%%%%%%%%%%%%%%%%%%%%%%%%%%%%%%%%%%%%%%%%%%%%%%%%%
%%%%%%%%%%%%%%%%%%%%%%%%%%%%%%%%%%%%%%%%%%%%%%%%%%%%%%%%%%%%%%%%%%%%%%%%%%%%%%%%
%%%%%%%%%%%%%%%%%%%%%%%%%%%%%%%%%%%%%%%%%%%%%%%%%%%%%%%%%%%%%%%%%%%%%%%%%%%%%%%%
\appendix

\settowidth\MacroIndent{\rmfamily\scriptsize 000\ }

 \DocInput{childdoc.dtx}

\end{document}
%</driver>
% \fi
%
% %%%%%%%%%%%%%%%%%%%%%%%%%%%%%%%%%%%%%%%%%%%%%%%%%%%%%%%%%%%%%%%%%%%%%%%%%%%%%%
% %%%%%%%%%%%%%%%%%%%%%%%%%%%%%%%%%%%%%%%%%%%%%%%%%%%%%%%%%%%%%%%%%%%%%%%%%%%%%%
% \section{Sample}
%\iffalse
%<*samplemain>
%\fi
%
% The following presents a sample document
% with two chapters, two parts, a title page,
% a compile flag as well as three forwarding files to set the flag.
% It consists of eight |.tex| files:
% \begin{center}
% \begin{tabular}{ll}
% |cdocsamp.tex|&main file\\
% |cdocsch1.tex|&include file for chapter 1\\
% |cdocsch2.tex|&include file for chapter 2\\
% |cdocspt3.tex|&include file for part 3\\
% |cdocspt4.tex|&include file for part 4\\
% |cdocsdrf.tex|&forwarding file for main file in draft mode\\
% |cdocsfi1.tex|&forwarding file for final version of chapter 1\\
% |cdocsfi2.tex|&forwarding file for final version of chapter 2\\
% \end{tabular}
% \end{center}
% Each of the eight files can be compiled directly by the \LaTeX{} compiler.
%
% %%%%%%%%%%%%%%%%%%%%%%%%%%%%%%%%%%%%%%
% \paragraph{Main File.}
%
% The main file is called |cdocsamp.tex|.
%
% Load the \textsf{childdoc} definitions and
% declare the filename for the main document:
%    \begin{macrocode}
\input{childdoc.def}
\childdocmain{}
%    \end{macrocode}

% Optional override for |\version| flag:
%    \begin{macrocode}
%%\ifchilddoc\else\providecommand{\version}{draft}\fi
%    \end{macrocode}

% Define the default values for the |\version| flag
% (|final| for the main file and |draft| for childs):
%    \begin{macrocode}
\ifchilddoc
\providecommand{\version}{draft}
\else
\providecommand{\version}{final}
\fi
%    \end{macrocode}

% Load the standard document class:
%    \begin{macrocode}
\documentclass[12pt]{article}
%    \end{macrocode}

% Start the document body:
%    \begin{macrocode}
\begin{document}
%    \end{macrocode}

% Declare a title page.
% Print title, part of document being processed and version flag:
%    \begin{macrocode}
\addtocounter{page}{-1}
\begin{center}
{\LARGE\bfseries{}childdoc example\par}
\vspace{1cm}
\ifchilddoc
\ifchilddocmanual part\else chapter\fi:
`\childdocname' of `\childdocjob'\par
\else
main document: `\childdocjob'\par
\fi
version: \version\par
\end{center}
\newpage
%    \end{macrocode}

% Manually include selected file,
% otherwise process as usual:
%    \begin{macrocode}
\ifchilddocmanual
\section*{part `\childdocname'}
\input{\childdocname}
\else
%    \end{macrocode}

% Include the two chapters:
%    \begin{macrocode}
\include{cdocsch1}
\include{cdocsch2}
%    \end{macrocode}

% Include the two parts unless only chapters should be displayed:
%    \begin{macrocode}
\ifchilddoc\else
\section{part three}
\input{cdocspt3}
\section{part four}
\input{cdocspt4}
\fi
%    \end{macrocode}

% Process as usual until here:
%    \begin{macrocode}
\fi
%    \end{macrocode}

% End of document body:
%    \begin{macrocode}
\end{document}
%    \end{macrocode}
%\iffalse
%</samplemain>
%\fi
%
% %%%%%%%%%%%%%%%%%%%%%%%%%%%%%%%%%%%%%%
% \paragraph{Chapter Include Files.}
%
% The include files are called |cdocsch1.tex| and |cdocsch2.tex|.
%
%\iffalse
%<*samplechap1|samplechap2>
%\fi

% Optional override for |\version| flag:
%    \begin{macrocode}
%%\providecommand{\version}{final}
%    \end{macrocode}

% Include the main document:
%    \begin{macrocode}
\input{childdoc.def}
\childdocof{cdocsamp}
%    \end{macrocode}

%\iffalse
%</samplechap1|samplechap2>
%\fi
%
%\iffalse
%<*samplechap1>
%\fi
% Some text for chapter 1:
%    \begin{macrocode}
\section{one}
some text in chapter one
%    \end{macrocode}

%\iffalse
%</samplechap1>
%\fi
% Some text for chapter 2:
%\iffalse
%<*samplechap2>
%\fi
%    \begin{macrocode}
\section{two}
more text in chapter two
%    \end{macrocode}

%\iffalse
%</samplechap2>
%\fi
%
% %%%%%%%%%%%%%%%%%%%%%%%%%%%%%%%%%%%%%%
% \paragraph{Part Include Files.}
%
% The include files are called |cdocspt3.tex| and |cdocspt4.tex|.
%
%\iffalse
%<*samplepart3|samplepart4>
%\fi

% Optional override for |\version| flag:
%    \begin{macrocode}
%%\providecommand{\version}{final}
%    \end{macrocode}

% Include the main document:
%    \begin{macrocode}
\input{childdoc.def}
\childdocby{cdocsamp}
%    \end{macrocode}

%\iffalse
%</samplepart3|samplepart4>
%\fi
%
%\iffalse
%<*samplepart3>
%\fi
% Some text for part 3:
%    \begin{macrocode}
some text in part three
%    \end{macrocode}

%\iffalse
%</samplepart3>
%\fi
% Some text for part 4:
%\iffalse
%<*samplepart4>
%\fi
%    \begin{macrocode}
more text in part four
%    \end{macrocode}

%\iffalse
%</samplepart4>
%\fi
%
% %%%%%%%%%%%%%%%%%%%%%%%%%%%%%%%%%%%%%%
% \paragraph{Forwarding for a Complete Draft.}
%
% The following forwarding file |cdocsdrf.tex|
% compiles the main document in draft mode:
%\iffalse
%<*sampledraft>
%\fi
%    \begin{macrocode}
\def\version{draft}
\input{childdoc.def}
\childdocforward{cdocsamp}
%    \end{macrocode}

%\iffalse
%</sampledraft>
%\fi
%
% %%%%%%%%%%%%%%%%%%%%%%%%%%%%%%%%%%%%%%
% \paragraph{Forwarding for Final Version of the Chapters.}
%
% The following forwarding files |cdocsfn1.tex| and |cdocsfn2.tex|
% (with identical content)
% compile the final versions of the child documents
% |cdocsch1.tex| and |cdocsch2.tex|, respectively:
%\iffalse
%<*samplefinal>
%\fi
%    \begin{macrocode}
\def\version{final}
\input{childdoc.def}
\childdocforwardprefix[cdocsamp]{cdocsfn}{cdocsch}
%    \end{macrocode}

%\iffalse
%</samplefinal>
%\fi
%
% %%%%%%%%%%%%%%%%%%%%%%%%%%%%%%%%%%%%%%
% \paragraph{Command Line Processing.}
%
% The following three command lines generate the output files
% |cdocscld|, |cdocscl1| and |cdocscl2|
% which should be identical to
% |cdocsdrf|, |cdocsch1| and |cdocsfn2|, respectively:
% \begin{center}
% \begin{tabular}{l}
% |latex -jobname cdocscld \|\\
% |  "\def\version{draft}\input{childdoc.def}\childdocforward{cdocsamp}"|\\
% |latex -jobname cdocscl1 \|\\
% |  "\input{childdoc.def}\childdocforward[cdocsamp]{cdocsch1}"|\\
% |latex -jobname cdocscl2 \|\\
% |  "\def\version{final}\input{childdoc.def}\childdocforward{cdocsch2}"|
% \end{tabular}
% \end{center}
% Note that the trailing backslash on each first line
% merely continues the input to the second line
% (for convenient cut ant paste).
% Furthermore, the command |latex| can be replaced by any
% of its alternative versions such as |pdflatex|.
%
% %%%%%%%%%%%%%%%%%%%%%%%%%%%%%%%%%%%%%%%%%%%%%%%%%%%%%%%%%%%%%%%%%%%%%%%%%%%%%%
% %%%%%%%%%%%%%%%%%%%%%%%%%%%%%%%%%%%%%%%%%%%%%%%%%%%%%%%%%%%%%%%%%%%%%%%%%%%%%%
% \section{Implementation}
%\iffalse
%<*package>
%\fi
%
% This section describes the definitions file |childdoc.def|.

% The definitions cannot be loaded using |\usepackage| or |\RequirePackage|
% which has a mechanism to prevent loading a style file more than once.
% When loading the definitions by means of |\input|
% multiple instances have to be prevented manually:
%\iffalse
%This code needs to be before the `\ProvidesFile' directive
%which is defined at the beginning of this file.
%Therefore it is also placed there and commented out here.
%</package>
%<*discard>
%\fi
%    \begin{macrocode}
\ifdefined\childdocmain\endinput\fi
%    \end{macrocode}
%\iffalse
%</discard>
%<*package>
%\fi
%
% \macro{\ifchilddoc}
% \macro{\ifchilddocmanual}
% The conditional |\ifchilddoc| tells whether a
% child (true) or main (false) document is being compiled.
% The conditional |\ifchilddocmanual| tells whether
% the |\includeonly| mechanism is used (false) or
% the selection of child files must be performed manually (true).
% The definitions initialise to false:
%    \begin{macrocode}
\newif\ifchilddoc
\newif\ifchilddocmanual
%    \end{macrocode}

% \macro{\childdocname}
% \macro{\childdocjob}
% The macro |\childdocname| stores the name of the main document
% to be compiled. The macro |\childdocjob| stores the name of
% the document on which the \LaTeX{} compiler was originally invoked.
% The content of |\jobname| cannot be compared
% to filenames specified in the source due to different catcodes.
% The following code rescans |\jobname|, stores the result
% in |\childdocname| and saves a copy in |\childdocjob|:
%    \begin{macrocode}
\edef\childdocname{\scantokens\expandafter{\jobname\noexpand}}
\let\childdocjob\childdocname
%    \end{macrocode}

% \macro{\childdocdisable}
% The macro |\childdocdisable| prevents the main file
% from being processed more than once.
% At this stage, the main document command |\childdocmain|
% is assumed to be called once again where it should do nothing.
% Any subsequent call to it should prevent
% a secondary processing of the main document
% It overwrites the forwarding commands
% |\childdocof| and |\childdocforward|
% with empty macros to prevent further inclusions of the main document:
%    \begin{macrocode}
\newcommand{\childdocdisable}
{
  \renewcommand{\childdocmain}[1]{\renewcommand{\childdocmain}[1]{\endinput}}
  \renewcommand{\childdocof}[1]{}
  \renewcommand{\childdocby}[2][]{}
  \renewcommand{\childdocforward}[2][]{}
  \renewcommand{\childdocdisable}{}
}
%    \end{macrocode}

% \macro{\childdocmain}
% The macro |\childdocmain| is to be called at the top of the main file
% with nothing or the main filename (without extension) as argument.
% First, it breaks loops.
% If the argument is not empty and does not match |\childdocname|
% (which is set by the first inclusion of |childdoc.def|),
% |\ifchilddoc| is set to true, |\includeonly| is applied to the child file
% and |\jobname| is set to the main file
% (for proper handling of |.aux| files):
%    \begin{macrocode}
\newcommand{\childdocmain}[1]
{
  \childdocdisable\childdocmain{}
  \if?#1?\else
    \begingroup
      \def\childdoctmp{#1}
      \ifx\childdoctmp\childdocname
        \def\childdoctmp{}
      \else
        \def\childdoctmp
        {
          \childdoctrue
          \includeonly{\childdocname}
          \def\childdocjob{#1}
          \def\jobname{#1}
        }
      \fi
      \expandafter
    \endgroup
    \childdoctmp
  \fi
}
%    \end{macrocode}

% \macro{\childdocof}
% The command |\childdocof| redirects
% compilation to the main file |#1|.
%    \begin{macrocode}
\newcommand{\childdocof}[1]
{
  \childdocdisable
  \childdoctrue
  \includeonly{\childdocname}
  \def\jobname{#1}
  \def\childdocjob{#1}
  \input{#1}
}
%    \end{macrocode}

% \macro{\childdocby}
% The command |\childdocby| ....
%    \begin{macrocode}
\newcommand{\childdocby}[2][]
{
  \childdocdisable
  \childdoctrue
  \childdocmanualtrue
  \if?#1?\else
    \def\jobname{#2}
  \fi
  \def\childdocjob{#2}
  \input{#2}
  \endinput
}
%    \end{macrocode}

% \macro{\childdocforward}
% The command |\childdocforward| redirects
% compilation to the main file or
% (if the optional argument is given) a child file.
% Parameters are set as if the main file
% or a child file starting with |\childdocof| was compiled.
% Then compilation is handed over to the main file:
%    \begin{macrocode}
\newcommand{\childdocforward}[2][]
{
  \begingroup
    \if?#1?
      \def\childdoctmp
      {
        \def\childdocname{#2}
        \def\childdocjob{#2}
        \def\jobname{#2}
        \input{#2}
        \endinput
      }
    \else
      \def\childdoctmp
      {
        \childdocdisable
        \def\childdocname{#2}
        \childdoctrue
        \includeonly{#2}
        \def\childdocjob{#1}
        \def\jobname{#1}
        \input{#1}
        \endinput
      }
    \fi
    \expandafter
  \endgroup
  \childdoctmp
}
%    \end{macrocode}

% \macro{\childdocforwardprefix}
% The command |\childdocforwardprefix| redirects
% compilation to the main or a child file by means of a pattern.
% The prefix |#1| in the current filename is replaced by |#2|
% and the suffix of the current filename is kept
% (it is assumed that the filename does not contain the substring `|~~~|'
% which is used as a delimiter).
% Compilation is handed over to the new file by |\childdocforward|:
%    \begin{macrocode}
\newcommand{\childdocforwardprefix}[3][]
{
  \begingroup
    \def\childdocextract #2##1~~~{\def\childdoctmp{\childdocforward[#1]{#3##1}}}
    \expandafter\childdocextract\childdocname~~~
    \expandafter
  \endgroup
  \childdoctmp
}
%    \end{macrocode}

% \macro{\childdoc}
% The deprecated macro |\childdoc| is a legacy version of |\childdocmain|:
%    \begin{macrocode}
\newcommand{\childdoc}{\childdocmain}
%    \end{macrocode}

% \macro{\childdocredirect}
% The deprecated macro |\childdocredirect| is a legacy version
% of |\childdocforward| and |\childdocforwardprefix|:
%    \begin{macrocode}
\newcommand{\childdocredirect}[2][]
{
  \begingroup
    \if?#1?
      \def\childdoctmp{\childdocforward{#2}}
    \else
      \def\childdoctmp{\childdocforwardprefix{#1}{#2}}
    \fi
    \expandafter
  \endgroup
  \childdoctmp
}
%    \end{macrocode}

%\iffalse
%</package>
%\fi
%
\endinput
|\\
|\childdocforward{|\textit{main}|}|
\end{tabular}
\end{center}
%
Likewise, the following files |final|\textit{nn}|.tex|
compile the final version of the child document
|child|\textit{nn}|.tex|:
%
\begin{center}
\begin{tabular}{l}
|\def\version{final}|\\
|% \iffalse
%
% childdoc.dtx Copyright (C) 2017-2018 Niklas Beisert
%
% This work may be distributed and/or modified under the
% conditions of the LaTeX Project Public License, either version 1.3
% of this license or (at your option) any later version.
% The latest version of this license is in
%   http://www.latex-project.org/lppl.txt
% and version 1.3 or later is part of all distributions of LaTeX
% version 2005/12/01 or later.
%
% This work has the LPPL maintenance status `maintained'.
%
% The Current Maintainer of this work is Niklas Beisert.
%
% This work consists of the files childdoc.dtx and childdoc.ins
% and the derived files childdoc.def and cdocsamp.tex with
% cdocsch1.tex, cdocsch2.tex, cdocsdrf.tex, cdocsfn1.tex, cdocsfn2.tex.
%
%<package>\ifdefined\childdocmain\endinput\fi
%<package>\ProvidesFile{childdoc.def}[2018/12/30 v2.0 child document driver]
%<samplemain>\ProvidesFile{cdocsamp.tex}[2018/12/30 v2.0 sample for childdoc]
%<*driver>
%\ProvidesFile{childdoc.drv}[2018/12/30 v2.0 childdoc reference manual file]
\PassOptionsToClass{10pt,a4paper}{article}
\documentclass{ltxdoc}

\usepackage[margin=35mm]{geometry}
\usepackage{hyperref}
\usepackage{hyperxmp}
\usepackage[usenames]{color}

\hypersetup{colorlinks=true}
\hypersetup{pdfstartview=FitH}
\hypersetup{pdfpagemode=UseNone}
\hypersetup{pdfsource={}}
\hypersetup{pdflang={en-UK}}
\hypersetup{pdfcopyright={Copyright 2017-2018 Niklas Beisert.
  This work may be distributed and/or modified under the
  conditions of the LaTeX Project Public License, either version 1.3
  of this license or (at your option) any later version.}}
\hypersetup{pdflicenseurl={http://www.latex-project.org/lppl.txt}}
\hypersetup{pdfcontactaddress={ETH Zurich, ITP, HIT K,
  Wolfgang-Pauli-Strasse 27}}
\hypersetup{pdfcontactpostcode={8093}}
\hypersetup{pdfcontactcity={Zurich}}
\hypersetup{pdfcontactcountry={Switzerland}}
\hypersetup{pdfcontactemail={nbeisert@itp.phys.ethz.ch}}
\hypersetup{pdfcontacturl={http://people.phys.ethz.ch/\xmptilde nbeisert/}}

\newcommand{\secref}[1]{\hyperref[#1]{section \ref*{#1}}}

\parskip1ex
\parindent0pt
\let\olditemize\itemize
\def\itemize{\olditemize\parskip0pt}

\begin{document}

\title{The \textsf{childdoc} Package}
\hypersetup{pdftitle={The childdoc Package}}
\author{Niklas Beisert\\[2ex]
  Institut f\"ur Theoretische Physik\\
  Eidgen\"ossische Technische Hochschule Z\"urich\\
  Wolfgang-Pauli-Strasse 27, 8093 Z\"urich, Switzerland\\[1ex]
  \href{mailto:nbeisert@itp.phys.ethz.ch}
  {\texttt{nbeisert@itp.phys.ethz.ch}}}
\hypersetup{pdfauthor={Niklas Beisert}}
\hypersetup{pdfsubject={Manual for the LaTeX2e Package childdoc}}
\date{30 December 2018, \textsf{v2.0}}
\maketitle

\begin{abstract}\noindent
\textsf{childdoc} is a \LaTeXe{} package
that enables the direct compilation
of document sections included by |\include|
to individual files.
\end{abstract}

\begingroup
\parskip0ex
\tableofcontents
\endgroup

%%%%%%%%%%%%%%%%%%%%%%%%%%%%%%%%%%%%%%%%%%%%%%%%%%%%%%%%%%%%%%%%%%%%%%%%%%%%%%%%
%%%%%%%%%%%%%%%%%%%%%%%%%%%%%%%%%%%%%%%%%%%%%%%%%%%%%%%%%%%%%%%%%%%%%%%%%%%%%%%%
\section{Introduction}

\LaTeX{} provides a mechanism to structure a large document (such as a book)
into a main file and several child files (containing the chapters)
using the |\include| command.
This mechanism is beneficial for documents
which span hundreds of pages in order to
make the source file(s) more manageable.
Moreover, compilation can be restricted to
selected child files by means of the |\includeonly| command.
The latter feature can be used to reduce the compilation time while editing
(this was significantly more useful in the earlier days of \LaTeX{})
or to generate a smaller document which is easier to navigate.
Another application of |\includeonly| is to generate
documents consisting of selected parts of the complete document.

However, there are a few drawbacks of the plain |\include| mechanism:
\begin{itemize}
\item
The child files cannot be compiled on their own,
they can only be compiled via the main file.
A naive editing environment
(such as a text editor with an option
to have the current file processed by \LaTeX)
may require one to switch to the main file before compiling;
attempting to compile the child file produces errors.
\item
The main file must be modified (each time)
to adjust the |\includeonly| command
to the present needs. This easily leaves the main file in a messy state.
\item
The generated document will always carry the filename
of the main document. This is inconvenient if
several child files are to be compiled and
to be kept for distribution.
\end{itemize}

The present package provides a simple interface
to make child files individually compilable by \LaTeX{}.
Compiling a child file then has the same effect as compiling
the main file with an |\includeonly| command
to select the appropriate child.
Moreover the generated document will carry the name of the child
rather than the main file.
This resolves all three above issues.

This feature is meant to make the editing of books,
thesis documents and lecture notes somewhat more convenient.
However, the package can also be used efficiently for
composing a series of documents (such as exercise sheets)
which are typically distributed individually.
It then assists the author in generating the individual documents
(potentially in different versions)
as well as a document containing the collected series.
Another application is in developing style files
or other kinds of included material
where compilation of the style file could redirect
to a sample or test file.

%%%%%%%%%%%%%%%%%%%%%%%%%%%%%%%%%%%%%%%%%%%%%%%%%%%%%%%%%%%%%%%%%%%%%%%%%%%%%%%%
%%%%%%%%%%%%%%%%%%%%%%%%%%%%%%%%%%%%%%%%%%%%%%%%%%%%%%%%%%%%%%%%%%%%%%%%%%%%%%%%
\section{Usage}

First of all, the package \textsf{childdoc} is \emph{not} a standard
\LaTeXe{} |.sty| style file! Therefore it needs to be invoked in
a non-standard way.

%%%%%%%%%%%%%%%%%%%%%%%%%%%%%%%%%%%%%%%%%%%%%%%%%%%%%%%%%%%%%%%%%%%%%%%%%%%%%%%%
\subsection{Included Files}
\label{sec:include}

%%%%%%%%%%%%%%%%%%%%%%%%%%%%%%%%%%%%%%%%
\DescribeMacro{\childdocmain}
To use the package, add the commands
\begin{center}
\begin{tabular}{l}
|\input{childdoc.def}|\\
|\childdocmain{}|\\
\end{tabular}
\end{center}
at the very top of the main \LaTeX{} file,
in particular \emph{before} the |\documentclass| statement!
The argument of |\childdocmain| should be left empty
(but it must be present).

%%%%%%%%%%%%%%%%%%%%%%%%%%%%%%%%%%%%%%%%
\DescribeMacro{\childdocof}
Furthermore, add the commands
\begin{center}
\begin{tabular}{l}
|\input{childdoc.def}|\\
|\childdocof{|\textit{main}|}|\\
\end{tabular}
\end{center}
at the top of every child file \textit{child}
which is included by |\include{|\textit{child}|}|
from within the main file
(or at least for those files to be compiled individually).
The argument \textit{main} must be the filename of the main file.

There are a couple of
considerations in setting up the main and child documents:

%%%%%%%%%%%%%%%%%%%%%%%%%%%%%%%%%%%%%%%%
\paragraph{Restrictions.}

Please note the following restrictions:
\begin{itemize}
\item
|\childdocmain| must be called with one argument \textit{main}
to ensure compatibility with earlier version of the package.
It must either be empty (|\childdocmain{}|)
or precisely match the filename of the main file in which it is specified.
See \secref{sec:detection} for further information.
\item
The filename \textit{main} must be specified without the |.tex| extension.
\item
The filename \textit{main} is case sensitive
(even in case-insensitive file systems)
due to internal string comparison.
\item
The argument \textit{main} should be fully expanded, it cannot be a macro.
\item
Subdirectories and special characters should be avoided in filenames.
\item
The command |\childdocmain{|\textit{main}|}| must be followed by a whitespace.
It should not be followed immediately by another command
or by a comment mark `|%|'.
This is because the \TeX{} parser reads the token immediately following
the argument of |\childdocmain| and puts it
at the beginning of every child section;
however, a white\-space is ignored.
\end{itemize}

%%%%%%%%%%%%%%%%%%%%%%%%%%%%%%%%%%%%%%%%
\paragraph{Content of Main File.}

It is advisable to place all content in the child files included by |\include|.
Any output contained in the main file will appear in all child documents
unless suppressed manually;
it cannot be suppressed automatically by the |\includeonly| directive
and thus should normally be avoided.
A method to include some content in the main file
by means of conditional processing is described in \secref{sec:conditional}.

%%%%%%%%%%%%%%%%%%%%%%%%%%%%%%%%%%%%%%%%
\paragraph{Page Numbering.}

When only a part of the document is compiled,
the appropriate numbering of pages
(as well as other status parameters)
is determined from the |.aux| files.
The latter contain information from previous passes.
However this information needs to propagate through
all intermediate child documents.
Therefore the page numbering in child documents may well
be inconsistent until the complete document is compiled at least once.

A useful (if unconventional) way to always ensure a consistent
page numbering is to restart the numbering in each child document
and denote the pages by `\textit{child}|.|\textit{page}'
where \textit{child} represents the chapter/section number of the child file.
This can be achieved by the command
|\numberwithin{page}{|\textit{child}|}|
of the \textsf{amsmath} package
where \textit{child} can be |chapter| or |section|
depending on the chosen structuring.
Alternatively, one can modify the macro |\thepage| appropriately
and reset the counter |page| at the start of each child file.

%%%%%%%%%%%%%%%%%%%%%%%%%%%%%%%%%%%%%%%%%%%%%%%%%%%%%%%%%%%%%%%%%%%%%%%%%%%%%%%%
\subsection{Conditional Processing}
\label{sec:conditional}

The package provides a mechanism to compile different versions
of a document. To customise the versions further some conditional processing
can come in handy to distinguish which version is being compiled.
The package provides two macros to describe the compilation context:

%%%%%%%%%%%%%%%%%%%%%%%%%%%%%%%%%%%%%%%%
\DescribeMacro{\ifchilddoc}
The conditional |\ifchilddoc| distinguishes between the compilation of
child documents and the main document:
%
\begin{center}
|\ifchilddoc |\textit{child-code}| |[|\||else |\textit{main-code}]| \||fi|
\end{center}

%%%%%%%%%%%%%%%%%%%%%%%%%%%%%%%%%%%%%%%%
\DescribeMacro{\childdocname}
\DescribeMacro{\childdocjob}
The macro |\childdocname| contains the filename (without extension)
of the main or child file being processed.
Note that |\childdocjob| will always contain the name of the main file.

%%%%%%%%%%%%%%%%%%%%%%%%%%%%%%%%%%%%%%%%
\paragraph{Title Page.}

Conditional processing can be used to include a title or banner page
in the main document when proper precautions are taken.
Importantly, the code in the main file should ensure that the page counter
(as well as other status parameters which are stored in the |.aux| files)
takes the same value after the conditional processing.
Otherwise the page numbers may take divergent values
depending on which part is compiled.

For example, a title page could be declared by:
%
\begin{center}
\begin{tabular}{l}
|\ifchilddoc\||else|\\
|\addtocounter{page}{-1}|\\
\textit{code for title page}\\
|\newpage|\\
|\||fi|
\end{tabular}
\end{center}
%
A banner page for the child documents can be generated by:
%
\begin{center}
\begin{tabular}{l}
|\ifchilddoc|\\
|\addtocounter{page}{-1}|\\
\textit{code for banner page}\\
|\newpage|\\
|\||fi|
\end{tabular}
\end{center}
%
Here one could write a message such as:
\begin{center}
|This is the part \childdocname{} of \childdocjob{}.|
\end{center}

%%%%%%%%%%%%%%%%%%%%%%%%%%%%%%%%%%%%%%%%%%%%%%%%%%%%%%%%%%%%%%%%%%%%%%%%%%%%%%%%
\subsection{Flags}
\label{sec:flags}

The package makes it easy to generate different versions
of the main or child documents.
To this end compilation flags can be defined
and assigned different default values.
They will be particularly useful in conjunction
with the forwarding mechanism described in \secref{sec:forward}.

For example, it may be useful to have a flag |\version|
which can be set to |draft| or |final|.
The document source will contain some conditional code
depending on the value of |\version|.
Suppose further, the flag should default to |final| for the main file
and to |draft| for child files
which is a natural assignment for editing the document.
This is achieved by placing the following code
in the preamble of the main document
(below the |\childdocmain| directive):
%
\begin{center}
\begin{tabular}{l}
|\ifchilddoc|\\
|\providecommand{\version}{draft}|\\
|\||else|\\
|\providecommand{\version}{final}|\\
|\||fi|
\end{tabular}
\end{center}
%
The definition by |\providecommand| makes sure
that previous definitions are not overwritten.
Further statements |\providecommand{\version}{...}|
can thus be added before the above code to override it.

For the main file, one might add a line
(between |\childdocmain| and the above block)
%
\begin{center}
|%\ifchilddoc\||else\providecommand{\version}{draft}\||fi|
\end{center}
%
which can be uncommented to produce a draft version.
Likewise one can add a line to the very top of a child file
(above the |\childdocof{|\textit{main}|}| directive)
%
\begin{center}
|%\providecommand{\version}{final}|
\end{center}
%
which can be uncommented to produce the final version of this child document.

%%%%%%%%%%%%%%%%%%%%%%%%%%%%%%%%%%%%%%%%%%%%%%%%%%%%%%%%%%%%%%%%%%%%%%%%%%%%%%%%
\subsection{Forwarding}
\label{sec:forward}

Different versions of the main or child documents
using compilation flags as described in \secref{sec:flags}
can be (permanently) stored in different files
for convenient compilation, viewing and distribution.
To this end, the package defines a command
to pass on compilation to a different file:

%%%%%%%%%%%%%%%%%%%%%%%%%%%%%%%%%%%%%%%%
\DescribeMacro{\childdocforward}
The command |\childdocforward| redirects processing to
another source file:
%
\begin{center}
\begin{tabular}{l}
|\input{childdoc.def}|\\
|\childdocforward[|\textit{main}|]{|\textit{dest}|}|\\
\end{tabular}
\end{center}
%
The argument \textit{dest} is the destination file
(without extension).
It should be the main file or one of the child files.
Note that further \textsf{childdoc} directives
such as |\childdocof| and |\childdocforward|
in the indicated file will be processed in this form.
The optional argument \textit{main}
passes on directly to the main file \textit{main}
while pretending to compile the child \textit{dest}.
This form behaves as if \textit{dest}
issues |\childdocof{|\textit{main}|}| right away,
and no further \textsf{childdoc} directives will be processed.

%%%%%%%%%%%%%%%%%%%%%%%%%%%%%%%%%%%%%%%%
\DescribeMacro{\...prefix}
In the alternative form |\childdocforwardprefix|,
%
\begin{center}
\begin{tabular}{l}
|\input{childdoc.def}|\\
|\childdocforwardprefix[|\textit{main}|]{|\textit{prefix}|}{|\textit{dest}|}|
\end{tabular}
\end{center}
%
the destination file is determined by a pattern
depending on the current file:
To make this work, the current file must be called
`{\textit{prefix}\hspace{0.2em}\textit{suffix}}'
with \textit{prefix} matching precisely the argument.
Processing is then passed on to the file
`{\textit{dest}\hspace{0.2em}\textit{suffix}}'.
Surely, the same effect is achieved by
directly specifying the
argument `{\textit{dest}\hspace{0.2em}\textit{suffix}}'
in the first form.
However, that requires to set up a different file
for each child. With the alternative form of the command
all these files can have exactly the same content
which simplifies setting them up and maintaining them.

For example, the following file |draft.tex|
with a compilation flag |\version| as described in \secref{sec:flags}
compiles the main document as a draft:
%
\begin{center}
\begin{tabular}{l}
|\def\version{draft}|\\
|\input{childdoc.def}|\\
|\childdocforward{|\textit{main}|}|
\end{tabular}
\end{center}
%
Likewise, the following files |final|\textit{nn}|.tex|
compile the final version of the child document
|child|\textit{nn}|.tex|:
%
\begin{center}
\begin{tabular}{l}
|\def\version{final}|\\
|\input{childdoc.def}|\\
|\childdocforwardprefix{final}{child}|
\end{tabular}
\end{center}
%

Note that when several versions of a main file and/or of each child file
are to be generated, it may be convenient to set up a |Makefile| or
shell script to automatise the process.

%%%%%%%%%%%%%%%%%%%%%%%%%%%%%%%%%%%%%%%%%%%%%%%%%%%%%%%%%%%%%%%%%%%%%%%%%%%%%%%%
\subsection{Command Line Processing}
\label{sec:commandline}

The effect of redirection files can also be achieved by invoking
the \LaTeX{} compiler with a more elaborate command line.
Most conveniently this should be done as part
of a shell script or a |Makefile|.

When using \textsf{childdoc} in the main file, the following
command lines effectively perform a redirection
(note that depending on the shell being used,
backslashes may have to be doubled: `|\|' $\to$ `|\\|'):
%
\begin{center}
|... -jobname "|\textit{target}|" |\\|"|[\textit{flags}]%
|\input{childdoc.def}\childdocforward[|\textit{main}|]{|\textit{dest}|}"|
\end{center}
%
Here \textit{target} is the name of the output file,
\textit{main} is the name of the main file
and \textit{dest} is the name of the main or child file to be processed
(all filenames without extensions).
The optional argument \textit{main} can be omitted
if \textit{main} matches \textit{dest}.
Optionally, compilation \textit{flags} can be defined via |\def| commands.
This command line makes the \TeX{} engine believe
it is compiling the file \textit{target}
whose content is specified as the latter parameter.
The provided code then forwards the processing to
\textit{main} or \textit{dest} as described in \secref{sec:forward}.

%%%%%%%%%%%%%%%%%%%%%%%%%%%%%%%%%%%%%%%%%%%%%%%%%%%%%%%%%%%%%%%%%%%%%%%%%%%%%%%%
\subsection{Include by Input}
\label{sec:input}

Including child documents by |\include| has some restrictions by design.
Most notably, the content of a child document always occupies
its own set of pages; pages cannot be shared between child documents.
Usually, this behaviour makes perfect sense
because each child document contain an essential part of the document.
However, in some situations it may be desirable to compose
a document from a collection of parts
without having mandatory page breaks between then.
For this case, the package
provides a mechanism to include parts
by |\input| which can also be processed individually.
However, by construction this mechanism
requires manual handling of the content to be output.

%%%%%%%%%%%%%%%%%%%%%%%%%%%%%%%%%%%%%%%%
\DescribeMacro{\ifchilddocmanual}
The main file should be prepared as usual, see \secref{sec:include}.
However, the document body must make a distinction
between processing of an individual part and of the main document, e.g.:
%
\begin{center}
\begin{tabular}{l}
|\ifchilddocmanual|\\
|\input{\childdocname}|\\
|\||else|\\
\textit{document body with }|\input{|\textit{part}|}|\\
|\||fi|
\end{tabular}
\end{center}
%
The conditional |\ifchilddocmanual| is true whenever
a part to be included by |\input| is being compiled,
and the name of the part is stored in |\childdocname|.

%%%%%%%%%%%%%%%%%%%%%%%%%%%%%%%%%%%%%%%%
\DescribeMacro{\childdocby}
Each part to be included by |\input| should start with:
%
\begin{center}
\begin{tabular}{l}
|\input{childdoc.def}|\\
|\childdocby{|\textit{main}|}|\\
\end{tabular}
\end{center}
%
The directive |\childdocby| is similar to |\childdocof|
described in \secref{sec:include},
but the subsequent selection of content must be done manually.
To that end, both |\ifchilddoc| and |\ifchilddocmanual|
will be true upon processing of a part,
and the name of the part is stored in |\childdocname|.
Note that |\jobname| will be set to the filename of the current part
so that each part receives an individual |.aux| file
that does not interfere with the |.aux| file(s) of the main document.
This behaviour can be altered by the alternative form
|\childdocby[*]{|\textit{main}|}| (with a non-empty optional argument)
which uses the |.aux| file of the main document
by setting |\jobname| to \textit{main}.

%%%%%%%%%%%%%%%%%%%%%%%%%%%%%%%%%%%%%%%%%%%%%%%%%%%%%%%%%%%%%%%%%%%%%%%%%%%%%%%%
\subsection{Driver Development}
\label{sec:driver}

The \textsf{childdoc} mechanism can also be use for the development
of definition files such as \LaTeX{} styles or classes.
This case differs from the above setup with multiple parts
included by |\include| in that no |\includeonly| should be invoked.
This can be achieved by starting the include file
(before |\ProvidesPackage|) with:
%
\begin{center}
\begin{tabular}{l}
|\input{childdoc.def}|\\
|\childdocforward{|\textit{main}|}|\\
\end{tabular}
\end{center}
%
or alternatively with:
%
\begin{center}
\begin{tabular}{l}
|\input{childdoc.def}|\\
|\childdocby{|\textit{main}|}|\\
\end{tabular}
\end{center}
%
Both forms have slightly different effects as described above.
The main file is prepared as usual, see \secref{sec:include}.

%%%%%%%%%%%%%%%%%%%%%%%%%%%%%%%%%%%%%%%%%%%%%%%%%%%%%%%%%%%%%%%%%%%%%%%%%%%%%%%%
\subsection{Legacy Detection}
\label{sec:detection}

The directive |\childdocmain| in the main file can detect
whether the complete document or merely a child is to be compiled
even without using the directive |\childdocof|.
This method is deprecated because it is less robust
and there is no compelling reason to use it;
it is merely provided for backward compatibility
and it may be removed in future versions.

If the detection mechanism is to be used,
it is mandatory to correctly specify
the filename of the main file as the argument of |\childdocmain|:
%
\begin{center}
\begin{tabular}{l}
|\input{childdoc.def}|\\
|\childdocmain{|\textit{main}|}|\\
\end{tabular}
\end{center}
%
If |\jobname| does not match the argument \textit{main} of |\childdocmain|,
it is assumed that |\jobname| points to the child file to be compiled.
When using |\childdocmain| with the main file specified as argument,
it suffices to start a child file
with just |\input{|\textit{main}|}|
without loading of the package and using |\childdocof|.
If instead all processing is done
with the appropriate \textsf{childdoc} directives,
the argument of \textit{main} of |\childdocmain| can be empty.

An alternative version of the command line processing described
in \secref{sec:commandline} using the detection mechanism reads:
%
\begin{center}
|... -jobname "|\textit{target}|" "|[\textit{flags}]%
[|\def\jobname{|\textit{dest}|}|]|\input{|\textit{main}|}"|
\end{center}

%%%%%%%%%%%%%%%%%%%%%%%%%%%%%%%%%%%%%%%%%%%%%%%%%%%%%%%%%%%%%%%%%%%%%%%%%%%%%%%%
\subsection{Manual Code}
\label{sec:manual}

In case one cannot be certain whether the definitions file |childdoc.def|
is installed on the target \TeX{} distribution
and one prefers not to ship it,
it is conceivable to paste a few relevant commands into the sources.

To that end, drop all statements |\input{childdoc.def}|
and perform the replacements as outlined below.
Instead of |\childdocmain{|\textit{main}|}| add the following code
to the top of the main file:
%
\begin{center}
\begin{tabular}{l}
|\||ifdefined\childdocname\endinput\||fi\newif\ifchilddoc|\\
|\edef\childdocname{\scantokens\expandafter{\jobname\noexpand}}|\\
|\def\childdocmain{|\textit{main}|}\||ifx\childdocmain\childdocname\||else|\\
|\childdoctrue\includeonly{\childdocname}\let\jobname\childdocmain\||fi|\\
\end{tabular}
\end{center}
%
Instead of |\childdocof{|\textit{main}|}| just include the main file
at the top of each child file:
%
\begin{center}
|\input{|\textit{main}|}|
\end{center}
%
A simple redirection |\childdocforward{|\textit{dest}|}| is achieved by:
%
\begin{center}
|\def\jobname{|\textit{dest}|}\input{\jobname}|
\end{center}
%
The redirection with prefix
|\childdocforwardprefix[|\textit{prefix}|]{|\textit{dest}|}|
is accomplished by:
%
\begin{center}
\begin{tabular}{l}
|{\edef\jobname{\scantokens\expandafter{\jobname\noexpand}}|\\
|\def\redirectjob |\textit{prefix}|#1~~~{\gdef\jobname{|\textit{dest}|#1}}|\\
|\expandafter\redirectjob\jobname~~~}\input{\jobname}|
\end{tabular}
\end{center}

In an alternative approach,
child documents can be compiled by a specific command line
without additional code or specific definitions:
%
\begin{center}
|... -jobname "|\textit{target}|" "|[\textit{flags}]%
|\includeonly{|\textit{dest}|}\input{|\textit{main}|}"|
\end{center}
%

%%%%%%%%%%%%%%%%%%%%%%%%%%%%%%%%%%%%%%%%%%%%%%%%%%%%%%%%%%%%%%%%%%%%%%%%%%%%%%%%
%%%%%%%%%%%%%%%%%%%%%%%%%%%%%%%%%%%%%%%%%%%%%%%%%%%%%%%%%%%%%%%%%%%%%%%%%%%%%%%%
\section{Information}

%%%%%%%%%%%%%%%%%%%%%%%%%%%%%%%%%%%%%%%%%%%%%%%%%%%%%%%%%%%%%%%%%%%%%%%%%%%%%%%%
\subsection{Copyright}

Copyright \copyright{} 2017--2018 Niklas Beisert

This work may be distributed and/or modified under the
conditions of the \LaTeX{} Project Public License, either version 1.3
of this license or (at your option) any later version.
The latest version of this license is in
  \url{http://www.latex-project.org/lppl.txt}
and version 1.3 or later is part of all distributions of \LaTeX{}
version 2005/12/01 or later.

This work has the LPPL maintenance status `maintained'.

The Current Maintainer of this work is Niklas Beisert.

This work consists of the files |README.txt|, |childdoc.ins| and |childdoc.dtx|
as well as the derived files |childdoc.def|, |cdocsamp.tex|
with |cdocsch1.tex|, |cdocsch2.tex|, |cdocspt3.tex|, |cdocspt4.tex|,
|cdocsdrf.tex|, |cdocsfn1.tex|, |cdocsfn2.tex|
as well as |childdoc.pdf|.

%%%%%%%%%%%%%%%%%%%%%%%%%%%%%%%%%%%%%%%%%%%%%%%%%%%%%%%%%%%%%%%%%%%%%%%%%%%%%%%%
\subsection{Files and Installation}

The package consists of the files:
%
\begin{center}
\begin{tabular}{ll}
    |README.txt|   & readme file \\
    |childdoc.ins| & installation file \\
    |childdoc.dtx| & source file \\
    |childdoc.def| & definition file \\
    |cdocsamp.tex| & sample main file \\
    |cdocsch1.tex| & sample include file \\
    |cdocsch2.tex| & sample include file \\
    |cdocspt3.tex| & sample part file \\
    |cdocspt4.tex| & sample part file \\
    |cdocsdrf.tex| & sample redirection file \\
    |cdocsfn1.tex| & sample redirection file \\
    |cdocsfn2.tex| & sample redirection file \\
    |childdoc.pdf| & manual
\end{tabular}
\end{center}
%
The distribution consists of the files
|README.txt|, |childdoc.ins| and |childdoc.dtx|.
%
\begin{itemize}
\item
Run (pdf)\LaTeX{} on |childdoc.dtx|
to compile the manual |childdoc.pdf| (this file).
\item
Run \LaTeX{} on |childdoc.ins| to create the definitions file |childdoc.def|
and the sample |cdocsamp.tex| with include files
|cdocsch1.tex|, |cdocsch2.tex|, |cdocspt3.tex|, |cdocspt4.tex|,
|cdocsdrf.tex|, |cdocsfn1.tex|, |cdocsfn2.tex|.
Then copy the file |childdoc.def| to an appropriate directory of your \LaTeX{}
distribution, e.g.\ \textit{texmf-root}|/tex/latex/childdoc|.
\end{itemize}

%%%%%%%%%%%%%%%%%%%%%%%%%%%%%%%%%%%%%%%%%%%%%%%%%%%%%%%%%%%%%%%%%%%%%%%%%%%%%%%%
\subsection{Related CTAN Packages}

There are several other packages which offer a similar functionality:
%
\begin{itemize}
\item
The packages
\href{http://ctan.org/pkg/docmute}{\textsf{docmute}},
\href{http://ctan.org/pkg/includex}{\textsf{includex}} and
\href{http://ctan.org/pkg/standalone}{\textsf{standalone}}
provide commands to include only the document body of
a child file thus allowing both files to be compiled individually.
\item
The packages \href{http://ctan.org/pkg/subdocs}{\textsf{subdocs}}
and \href{http://ctan.org/pkg/subfiles}{\textsf{subfiles}}
provide structures in which the main and child documents can be
encapsulated and allowing them to be compiled individually.
The inclusion mechanism is different from the conventional |\include|.
\item
The package \href{http://ctan.org/pkg/combine}{\textsf{combine}}
is an elaborate solution to combine several documents into one.
\end{itemize}
%
See also the CTAN topic \href{http://ctan.org/topic/subdocs}{\textsf{subdocs}}
for further related packages.
The present package differs from the above solutions in that
a document structure constructed with the conventional |\include| mechanism
just needs two extra commands at the top of every file
such that all constituent files can be compiled individually.

%%%%%%%%%%%%%%%%%%%%%%%%%%%%%%%%%%%%%%%%%%%%%%%%%%%%%%%%%%%%%%%%%%%%%%%%%%%%%%%%
%\subsection{Feature Suggestions}
%
%The following is a list of features which may be useful for future
%versions of this package:
%%
%\begin{itemize}
%\item
%\ldots
%\end{itemize}

%%%%%%%%%%%%%%%%%%%%%%%%%%%%%%%%%%%%%%%%%%%%%%%%%%%%%%%%%%%%%%%%%%%%%%%%%%%%%%%%
\subsection{Revision History}

%%%%%%%%%%%%%%%%%%%%%%%%%%%%%%%%%%%%%%%%
\paragraph{v2.0:} 2018/12/30

\begin{itemize}
\item
immediate forward processing
\item
added |\childdocby| mechanism
\item
manual restructured
\end{itemize}

%%%%%%%%%%%%%%%%%%%%%%%%%%%%%%%%%%%%%%%%
\paragraph{v1.6:} 2018/01/17

\begin{itemize}
\item
application for development of include files
\item
corrections to manual
\end{itemize}

%%%%%%%%%%%%%%%%%%%%%%%%%%%%%%%%%%%%%%%%
\paragraph{v1.5:} 2017/05/21

\begin{itemize}
\item
more complete structuring introduced
\item
|\childdocof| introduced
\item
|\childdoc| renamed to |\childdocmain|
\item
|\childredirect| renamed to |\childdocforward| and |\childdocforwardprefix|
and functionality expanded
\end{itemize}

%%%%%%%%%%%%%%%%%%%%%%%%%%%%%%%%%%%%%%%%
\paragraph{v1.0:} 2017/04/27

\begin{itemize}
\item
manual and install package
\item
first version published on CTAN
\end{itemize}

%%%%%%%%%%%%%%%%%%%%%%%%%%%%%%%%%%%%%%%%
\paragraph{v0.6:} 2017/04/26

\begin{itemize}
\item
redirection mechanism added
\end{itemize}

%%%%%%%%%%%%%%%%%%%%%%%%%%%%%%%%%%%%%%%%
\paragraph{v0.5:} 2017/04/26

\begin{itemize}
\item
functionality in definition file
\end{itemize}


%%%%%%%%%%%%%%%%%%%%%%%%%%%%%%%%%%%%%%%%%%%%%%%%%%%%%%%%%%%%%%%%%%%%%%%%%%%%%%%%
%%%%%%%%%%%%%%%%%%%%%%%%%%%%%%%%%%%%%%%%%%%%%%%%%%%%%%%%%%%%%%%%%%%%%%%%%%%%%%%%
%%%%%%%%%%%%%%%%%%%%%%%%%%%%%%%%%%%%%%%%%%%%%%%%%%%%%%%%%%%%%%%%%%%%%%%%%%%%%%%%
\appendix

\settowidth\MacroIndent{\rmfamily\scriptsize 000\ }

 \DocInput{childdoc.dtx}

\end{document}
%</driver>
% \fi
%
% %%%%%%%%%%%%%%%%%%%%%%%%%%%%%%%%%%%%%%%%%%%%%%%%%%%%%%%%%%%%%%%%%%%%%%%%%%%%%%
% %%%%%%%%%%%%%%%%%%%%%%%%%%%%%%%%%%%%%%%%%%%%%%%%%%%%%%%%%%%%%%%%%%%%%%%%%%%%%%
% \section{Sample}
%\iffalse
%<*samplemain>
%\fi
%
% The following presents a sample document
% with two chapters, two parts, a title page,
% a compile flag as well as three forwarding files to set the flag.
% It consists of eight |.tex| files:
% \begin{center}
% \begin{tabular}{ll}
% |cdocsamp.tex|&main file\\
% |cdocsch1.tex|&include file for chapter 1\\
% |cdocsch2.tex|&include file for chapter 2\\
% |cdocspt3.tex|&include file for part 3\\
% |cdocspt4.tex|&include file for part 4\\
% |cdocsdrf.tex|&forwarding file for main file in draft mode\\
% |cdocsfi1.tex|&forwarding file for final version of chapter 1\\
% |cdocsfi2.tex|&forwarding file for final version of chapter 2\\
% \end{tabular}
% \end{center}
% Each of the eight files can be compiled directly by the \LaTeX{} compiler.
%
% %%%%%%%%%%%%%%%%%%%%%%%%%%%%%%%%%%%%%%
% \paragraph{Main File.}
%
% The main file is called |cdocsamp.tex|.
%
% Load the \textsf{childdoc} definitions and
% declare the filename for the main document:
%    \begin{macrocode}
\input{childdoc.def}
\childdocmain{}
%    \end{macrocode}

% Optional override for |\version| flag:
%    \begin{macrocode}
%%\ifchilddoc\else\providecommand{\version}{draft}\fi
%    \end{macrocode}

% Define the default values for the |\version| flag
% (|final| for the main file and |draft| for childs):
%    \begin{macrocode}
\ifchilddoc
\providecommand{\version}{draft}
\else
\providecommand{\version}{final}
\fi
%    \end{macrocode}

% Load the standard document class:
%    \begin{macrocode}
\documentclass[12pt]{article}
%    \end{macrocode}

% Start the document body:
%    \begin{macrocode}
\begin{document}
%    \end{macrocode}

% Declare a title page.
% Print title, part of document being processed and version flag:
%    \begin{macrocode}
\addtocounter{page}{-1}
\begin{center}
{\LARGE\bfseries{}childdoc example\par}
\vspace{1cm}
\ifchilddoc
\ifchilddocmanual part\else chapter\fi:
`\childdocname' of `\childdocjob'\par
\else
main document: `\childdocjob'\par
\fi
version: \version\par
\end{center}
\newpage
%    \end{macrocode}

% Manually include selected file,
% otherwise process as usual:
%    \begin{macrocode}
\ifchilddocmanual
\section*{part `\childdocname'}
\input{\childdocname}
\else
%    \end{macrocode}

% Include the two chapters:
%    \begin{macrocode}
\include{cdocsch1}
\include{cdocsch2}
%    \end{macrocode}

% Include the two parts unless only chapters should be displayed:
%    \begin{macrocode}
\ifchilddoc\else
\section{part three}
\input{cdocspt3}
\section{part four}
\input{cdocspt4}
\fi
%    \end{macrocode}

% Process as usual until here:
%    \begin{macrocode}
\fi
%    \end{macrocode}

% End of document body:
%    \begin{macrocode}
\end{document}
%    \end{macrocode}
%\iffalse
%</samplemain>
%\fi
%
% %%%%%%%%%%%%%%%%%%%%%%%%%%%%%%%%%%%%%%
% \paragraph{Chapter Include Files.}
%
% The include files are called |cdocsch1.tex| and |cdocsch2.tex|.
%
%\iffalse
%<*samplechap1|samplechap2>
%\fi

% Optional override for |\version| flag:
%    \begin{macrocode}
%%\providecommand{\version}{final}
%    \end{macrocode}

% Include the main document:
%    \begin{macrocode}
\input{childdoc.def}
\childdocof{cdocsamp}
%    \end{macrocode}

%\iffalse
%</samplechap1|samplechap2>
%\fi
%
%\iffalse
%<*samplechap1>
%\fi
% Some text for chapter 1:
%    \begin{macrocode}
\section{one}
some text in chapter one
%    \end{macrocode}

%\iffalse
%</samplechap1>
%\fi
% Some text for chapter 2:
%\iffalse
%<*samplechap2>
%\fi
%    \begin{macrocode}
\section{two}
more text in chapter two
%    \end{macrocode}

%\iffalse
%</samplechap2>
%\fi
%
% %%%%%%%%%%%%%%%%%%%%%%%%%%%%%%%%%%%%%%
% \paragraph{Part Include Files.}
%
% The include files are called |cdocspt3.tex| and |cdocspt4.tex|.
%
%\iffalse
%<*samplepart3|samplepart4>
%\fi

% Optional override for |\version| flag:
%    \begin{macrocode}
%%\providecommand{\version}{final}
%    \end{macrocode}

% Include the main document:
%    \begin{macrocode}
\input{childdoc.def}
\childdocby{cdocsamp}
%    \end{macrocode}

%\iffalse
%</samplepart3|samplepart4>
%\fi
%
%\iffalse
%<*samplepart3>
%\fi
% Some text for part 3:
%    \begin{macrocode}
some text in part three
%    \end{macrocode}

%\iffalse
%</samplepart3>
%\fi
% Some text for part 4:
%\iffalse
%<*samplepart4>
%\fi
%    \begin{macrocode}
more text in part four
%    \end{macrocode}

%\iffalse
%</samplepart4>
%\fi
%
% %%%%%%%%%%%%%%%%%%%%%%%%%%%%%%%%%%%%%%
% \paragraph{Forwarding for a Complete Draft.}
%
% The following forwarding file |cdocsdrf.tex|
% compiles the main document in draft mode:
%\iffalse
%<*sampledraft>
%\fi
%    \begin{macrocode}
\def\version{draft}
\input{childdoc.def}
\childdocforward{cdocsamp}
%    \end{macrocode}

%\iffalse
%</sampledraft>
%\fi
%
% %%%%%%%%%%%%%%%%%%%%%%%%%%%%%%%%%%%%%%
% \paragraph{Forwarding for Final Version of the Chapters.}
%
% The following forwarding files |cdocsfn1.tex| and |cdocsfn2.tex|
% (with identical content)
% compile the final versions of the child documents
% |cdocsch1.tex| and |cdocsch2.tex|, respectively:
%\iffalse
%<*samplefinal>
%\fi
%    \begin{macrocode}
\def\version{final}
\input{childdoc.def}
\childdocforwardprefix[cdocsamp]{cdocsfn}{cdocsch}
%    \end{macrocode}

%\iffalse
%</samplefinal>
%\fi
%
% %%%%%%%%%%%%%%%%%%%%%%%%%%%%%%%%%%%%%%
% \paragraph{Command Line Processing.}
%
% The following three command lines generate the output files
% |cdocscld|, |cdocscl1| and |cdocscl2|
% which should be identical to
% |cdocsdrf|, |cdocsch1| and |cdocsfn2|, respectively:
% \begin{center}
% \begin{tabular}{l}
% |latex -jobname cdocscld \|\\
% |  "\def\version{draft}\input{childdoc.def}\childdocforward{cdocsamp}"|\\
% |latex -jobname cdocscl1 \|\\
% |  "\input{childdoc.def}\childdocforward[cdocsamp]{cdocsch1}"|\\
% |latex -jobname cdocscl2 \|\\
% |  "\def\version{final}\input{childdoc.def}\childdocforward{cdocsch2}"|
% \end{tabular}
% \end{center}
% Note that the trailing backslash on each first line
% merely continues the input to the second line
% (for convenient cut ant paste).
% Furthermore, the command |latex| can be replaced by any
% of its alternative versions such as |pdflatex|.
%
% %%%%%%%%%%%%%%%%%%%%%%%%%%%%%%%%%%%%%%%%%%%%%%%%%%%%%%%%%%%%%%%%%%%%%%%%%%%%%%
% %%%%%%%%%%%%%%%%%%%%%%%%%%%%%%%%%%%%%%%%%%%%%%%%%%%%%%%%%%%%%%%%%%%%%%%%%%%%%%
% \section{Implementation}
%\iffalse
%<*package>
%\fi
%
% This section describes the definitions file |childdoc.def|.

% The definitions cannot be loaded using |\usepackage| or |\RequirePackage|
% which has a mechanism to prevent loading a style file more than once.
% When loading the definitions by means of |\input|
% multiple instances have to be prevented manually:
%\iffalse
%This code needs to be before the `\ProvidesFile' directive
%which is defined at the beginning of this file.
%Therefore it is also placed there and commented out here.
%</package>
%<*discard>
%\fi
%    \begin{macrocode}
\ifdefined\childdocmain\endinput\fi
%    \end{macrocode}
%\iffalse
%</discard>
%<*package>
%\fi
%
% \macro{\ifchilddoc}
% \macro{\ifchilddocmanual}
% The conditional |\ifchilddoc| tells whether a
% child (true) or main (false) document is being compiled.
% The conditional |\ifchilddocmanual| tells whether
% the |\includeonly| mechanism is used (false) or
% the selection of child files must be performed manually (true).
% The definitions initialise to false:
%    \begin{macrocode}
\newif\ifchilddoc
\newif\ifchilddocmanual
%    \end{macrocode}

% \macro{\childdocname}
% \macro{\childdocjob}
% The macro |\childdocname| stores the name of the main document
% to be compiled. The macro |\childdocjob| stores the name of
% the document on which the \LaTeX{} compiler was originally invoked.
% The content of |\jobname| cannot be compared
% to filenames specified in the source due to different catcodes.
% The following code rescans |\jobname|, stores the result
% in |\childdocname| and saves a copy in |\childdocjob|:
%    \begin{macrocode}
\edef\childdocname{\scantokens\expandafter{\jobname\noexpand}}
\let\childdocjob\childdocname
%    \end{macrocode}

% \macro{\childdocdisable}
% The macro |\childdocdisable| prevents the main file
% from being processed more than once.
% At this stage, the main document command |\childdocmain|
% is assumed to be called once again where it should do nothing.
% Any subsequent call to it should prevent
% a secondary processing of the main document
% It overwrites the forwarding commands
% |\childdocof| and |\childdocforward|
% with empty macros to prevent further inclusions of the main document:
%    \begin{macrocode}
\newcommand{\childdocdisable}
{
  \renewcommand{\childdocmain}[1]{\renewcommand{\childdocmain}[1]{\endinput}}
  \renewcommand{\childdocof}[1]{}
  \renewcommand{\childdocby}[2][]{}
  \renewcommand{\childdocforward}[2][]{}
  \renewcommand{\childdocdisable}{}
}
%    \end{macrocode}

% \macro{\childdocmain}
% The macro |\childdocmain| is to be called at the top of the main file
% with nothing or the main filename (without extension) as argument.
% First, it breaks loops.
% If the argument is not empty and does not match |\childdocname|
% (which is set by the first inclusion of |childdoc.def|),
% |\ifchilddoc| is set to true, |\includeonly| is applied to the child file
% and |\jobname| is set to the main file
% (for proper handling of |.aux| files):
%    \begin{macrocode}
\newcommand{\childdocmain}[1]
{
  \childdocdisable\childdocmain{}
  \if?#1?\else
    \begingroup
      \def\childdoctmp{#1}
      \ifx\childdoctmp\childdocname
        \def\childdoctmp{}
      \else
        \def\childdoctmp
        {
          \childdoctrue
          \includeonly{\childdocname}
          \def\childdocjob{#1}
          \def\jobname{#1}
        }
      \fi
      \expandafter
    \endgroup
    \childdoctmp
  \fi
}
%    \end{macrocode}

% \macro{\childdocof}
% The command |\childdocof| redirects
% compilation to the main file |#1|.
%    \begin{macrocode}
\newcommand{\childdocof}[1]
{
  \childdocdisable
  \childdoctrue
  \includeonly{\childdocname}
  \def\jobname{#1}
  \def\childdocjob{#1}
  \input{#1}
}
%    \end{macrocode}

% \macro{\childdocby}
% The command |\childdocby| ....
%    \begin{macrocode}
\newcommand{\childdocby}[2][]
{
  \childdocdisable
  \childdoctrue
  \childdocmanualtrue
  \if?#1?\else
    \def\jobname{#2}
  \fi
  \def\childdocjob{#2}
  \input{#2}
  \endinput
}
%    \end{macrocode}

% \macro{\childdocforward}
% The command |\childdocforward| redirects
% compilation to the main file or
% (if the optional argument is given) a child file.
% Parameters are set as if the main file
% or a child file starting with |\childdocof| was compiled.
% Then compilation is handed over to the main file:
%    \begin{macrocode}
\newcommand{\childdocforward}[2][]
{
  \begingroup
    \if?#1?
      \def\childdoctmp
      {
        \def\childdocname{#2}
        \def\childdocjob{#2}
        \def\jobname{#2}
        \input{#2}
        \endinput
      }
    \else
      \def\childdoctmp
      {
        \childdocdisable
        \def\childdocname{#2}
        \childdoctrue
        \includeonly{#2}
        \def\childdocjob{#1}
        \def\jobname{#1}
        \input{#1}
        \endinput
      }
    \fi
    \expandafter
  \endgroup
  \childdoctmp
}
%    \end{macrocode}

% \macro{\childdocforwardprefix}
% The command |\childdocforwardprefix| redirects
% compilation to the main or a child file by means of a pattern.
% The prefix |#1| in the current filename is replaced by |#2|
% and the suffix of the current filename is kept
% (it is assumed that the filename does not contain the substring `|~~~|'
% which is used as a delimiter).
% Compilation is handed over to the new file by |\childdocforward|:
%    \begin{macrocode}
\newcommand{\childdocforwardprefix}[3][]
{
  \begingroup
    \def\childdocextract #2##1~~~{\def\childdoctmp{\childdocforward[#1]{#3##1}}}
    \expandafter\childdocextract\childdocname~~~
    \expandafter
  \endgroup
  \childdoctmp
}
%    \end{macrocode}

% \macro{\childdoc}
% The deprecated macro |\childdoc| is a legacy version of |\childdocmain|:
%    \begin{macrocode}
\newcommand{\childdoc}{\childdocmain}
%    \end{macrocode}

% \macro{\childdocredirect}
% The deprecated macro |\childdocredirect| is a legacy version
% of |\childdocforward| and |\childdocforwardprefix|:
%    \begin{macrocode}
\newcommand{\childdocredirect}[2][]
{
  \begingroup
    \if?#1?
      \def\childdoctmp{\childdocforward{#2}}
    \else
      \def\childdoctmp{\childdocforwardprefix{#1}{#2}}
    \fi
    \expandafter
  \endgroup
  \childdoctmp
}
%    \end{macrocode}

%\iffalse
%</package>
%\fi
%
\endinput
|\\
|\childdocforwardprefix{final}{child}|
\end{tabular}
\end{center}
%

Note that when several versions of a main file and/or of each child file
are to be generated, it may be convenient to set up a |Makefile| or
shell script to automatise the process.

%%%%%%%%%%%%%%%%%%%%%%%%%%%%%%%%%%%%%%%%%%%%%%%%%%%%%%%%%%%%%%%%%%%%%%%%%%%%%%%%
\subsection{Command Line Processing}
\label{sec:commandline}

The effect of redirection files can also be achieved by invoking
the \LaTeX{} compiler with a more elaborate command line.
Most conveniently this should be done as part
of a shell script or a |Makefile|.

When using \textsf{childdoc} in the main file, the following
command lines effectively perform a redirection
(note that depending on the shell being used,
backslashes may have to be doubled: `|\|' $\to$ `|\\|'):
%
\begin{center}
|... -jobname "|\textit{target}|" |\\|"|[\textit{flags}]%
|% \iffalse
%
% childdoc.dtx Copyright (C) 2017-2018 Niklas Beisert
%
% This work may be distributed and/or modified under the
% conditions of the LaTeX Project Public License, either version 1.3
% of this license or (at your option) any later version.
% The latest version of this license is in
%   http://www.latex-project.org/lppl.txt
% and version 1.3 or later is part of all distributions of LaTeX
% version 2005/12/01 or later.
%
% This work has the LPPL maintenance status `maintained'.
%
% The Current Maintainer of this work is Niklas Beisert.
%
% This work consists of the files childdoc.dtx and childdoc.ins
% and the derived files childdoc.def and cdocsamp.tex with
% cdocsch1.tex, cdocsch2.tex, cdocsdrf.tex, cdocsfn1.tex, cdocsfn2.tex.
%
%<package>\ifdefined\childdocmain\endinput\fi
%<package>\ProvidesFile{childdoc.def}[2018/12/30 v2.0 child document driver]
%<samplemain>\ProvidesFile{cdocsamp.tex}[2018/12/30 v2.0 sample for childdoc]
%<*driver>
%\ProvidesFile{childdoc.drv}[2018/12/30 v2.0 childdoc reference manual file]
\PassOptionsToClass{10pt,a4paper}{article}
\documentclass{ltxdoc}

\usepackage[margin=35mm]{geometry}
\usepackage{hyperref}
\usepackage{hyperxmp}
\usepackage[usenames]{color}

\hypersetup{colorlinks=true}
\hypersetup{pdfstartview=FitH}
\hypersetup{pdfpagemode=UseNone}
\hypersetup{pdfsource={}}
\hypersetup{pdflang={en-UK}}
\hypersetup{pdfcopyright={Copyright 2017-2018 Niklas Beisert.
  This work may be distributed and/or modified under the
  conditions of the LaTeX Project Public License, either version 1.3
  of this license or (at your option) any later version.}}
\hypersetup{pdflicenseurl={http://www.latex-project.org/lppl.txt}}
\hypersetup{pdfcontactaddress={ETH Zurich, ITP, HIT K,
  Wolfgang-Pauli-Strasse 27}}
\hypersetup{pdfcontactpostcode={8093}}
\hypersetup{pdfcontactcity={Zurich}}
\hypersetup{pdfcontactcountry={Switzerland}}
\hypersetup{pdfcontactemail={nbeisert@itp.phys.ethz.ch}}
\hypersetup{pdfcontacturl={http://people.phys.ethz.ch/\xmptilde nbeisert/}}

\newcommand{\secref}[1]{\hyperref[#1]{section \ref*{#1}}}

\parskip1ex
\parindent0pt
\let\olditemize\itemize
\def\itemize{\olditemize\parskip0pt}

\begin{document}

\title{The \textsf{childdoc} Package}
\hypersetup{pdftitle={The childdoc Package}}
\author{Niklas Beisert\\[2ex]
  Institut f\"ur Theoretische Physik\\
  Eidgen\"ossische Technische Hochschule Z\"urich\\
  Wolfgang-Pauli-Strasse 27, 8093 Z\"urich, Switzerland\\[1ex]
  \href{mailto:nbeisert@itp.phys.ethz.ch}
  {\texttt{nbeisert@itp.phys.ethz.ch}}}
\hypersetup{pdfauthor={Niklas Beisert}}
\hypersetup{pdfsubject={Manual for the LaTeX2e Package childdoc}}
\date{30 December 2018, \textsf{v2.0}}
\maketitle

\begin{abstract}\noindent
\textsf{childdoc} is a \LaTeXe{} package
that enables the direct compilation
of document sections included by |\include|
to individual files.
\end{abstract}

\begingroup
\parskip0ex
\tableofcontents
\endgroup

%%%%%%%%%%%%%%%%%%%%%%%%%%%%%%%%%%%%%%%%%%%%%%%%%%%%%%%%%%%%%%%%%%%%%%%%%%%%%%%%
%%%%%%%%%%%%%%%%%%%%%%%%%%%%%%%%%%%%%%%%%%%%%%%%%%%%%%%%%%%%%%%%%%%%%%%%%%%%%%%%
\section{Introduction}

\LaTeX{} provides a mechanism to structure a large document (such as a book)
into a main file and several child files (containing the chapters)
using the |\include| command.
This mechanism is beneficial for documents
which span hundreds of pages in order to
make the source file(s) more manageable.
Moreover, compilation can be restricted to
selected child files by means of the |\includeonly| command.
The latter feature can be used to reduce the compilation time while editing
(this was significantly more useful in the earlier days of \LaTeX{})
or to generate a smaller document which is easier to navigate.
Another application of |\includeonly| is to generate
documents consisting of selected parts of the complete document.

However, there are a few drawbacks of the plain |\include| mechanism:
\begin{itemize}
\item
The child files cannot be compiled on their own,
they can only be compiled via the main file.
A naive editing environment
(such as a text editor with an option
to have the current file processed by \LaTeX)
may require one to switch to the main file before compiling;
attempting to compile the child file produces errors.
\item
The main file must be modified (each time)
to adjust the |\includeonly| command
to the present needs. This easily leaves the main file in a messy state.
\item
The generated document will always carry the filename
of the main document. This is inconvenient if
several child files are to be compiled and
to be kept for distribution.
\end{itemize}

The present package provides a simple interface
to make child files individually compilable by \LaTeX{}.
Compiling a child file then has the same effect as compiling
the main file with an |\includeonly| command
to select the appropriate child.
Moreover the generated document will carry the name of the child
rather than the main file.
This resolves all three above issues.

This feature is meant to make the editing of books,
thesis documents and lecture notes somewhat more convenient.
However, the package can also be used efficiently for
composing a series of documents (such as exercise sheets)
which are typically distributed individually.
It then assists the author in generating the individual documents
(potentially in different versions)
as well as a document containing the collected series.
Another application is in developing style files
or other kinds of included material
where compilation of the style file could redirect
to a sample or test file.

%%%%%%%%%%%%%%%%%%%%%%%%%%%%%%%%%%%%%%%%%%%%%%%%%%%%%%%%%%%%%%%%%%%%%%%%%%%%%%%%
%%%%%%%%%%%%%%%%%%%%%%%%%%%%%%%%%%%%%%%%%%%%%%%%%%%%%%%%%%%%%%%%%%%%%%%%%%%%%%%%
\section{Usage}

First of all, the package \textsf{childdoc} is \emph{not} a standard
\LaTeXe{} |.sty| style file! Therefore it needs to be invoked in
a non-standard way.

%%%%%%%%%%%%%%%%%%%%%%%%%%%%%%%%%%%%%%%%%%%%%%%%%%%%%%%%%%%%%%%%%%%%%%%%%%%%%%%%
\subsection{Included Files}
\label{sec:include}

%%%%%%%%%%%%%%%%%%%%%%%%%%%%%%%%%%%%%%%%
\DescribeMacro{\childdocmain}
To use the package, add the commands
\begin{center}
\begin{tabular}{l}
|\input{childdoc.def}|\\
|\childdocmain{}|\\
\end{tabular}
\end{center}
at the very top of the main \LaTeX{} file,
in particular \emph{before} the |\documentclass| statement!
The argument of |\childdocmain| should be left empty
(but it must be present).

%%%%%%%%%%%%%%%%%%%%%%%%%%%%%%%%%%%%%%%%
\DescribeMacro{\childdocof}
Furthermore, add the commands
\begin{center}
\begin{tabular}{l}
|\input{childdoc.def}|\\
|\childdocof{|\textit{main}|}|\\
\end{tabular}
\end{center}
at the top of every child file \textit{child}
which is included by |\include{|\textit{child}|}|
from within the main file
(or at least for those files to be compiled individually).
The argument \textit{main} must be the filename of the main file.

There are a couple of
considerations in setting up the main and child documents:

%%%%%%%%%%%%%%%%%%%%%%%%%%%%%%%%%%%%%%%%
\paragraph{Restrictions.}

Please note the following restrictions:
\begin{itemize}
\item
|\childdocmain| must be called with one argument \textit{main}
to ensure compatibility with earlier version of the package.
It must either be empty (|\childdocmain{}|)
or precisely match the filename of the main file in which it is specified.
See \secref{sec:detection} for further information.
\item
The filename \textit{main} must be specified without the |.tex| extension.
\item
The filename \textit{main} is case sensitive
(even in case-insensitive file systems)
due to internal string comparison.
\item
The argument \textit{main} should be fully expanded, it cannot be a macro.
\item
Subdirectories and special characters should be avoided in filenames.
\item
The command |\childdocmain{|\textit{main}|}| must be followed by a whitespace.
It should not be followed immediately by another command
or by a comment mark `|%|'.
This is because the \TeX{} parser reads the token immediately following
the argument of |\childdocmain| and puts it
at the beginning of every child section;
however, a white\-space is ignored.
\end{itemize}

%%%%%%%%%%%%%%%%%%%%%%%%%%%%%%%%%%%%%%%%
\paragraph{Content of Main File.}

It is advisable to place all content in the child files included by |\include|.
Any output contained in the main file will appear in all child documents
unless suppressed manually;
it cannot be suppressed automatically by the |\includeonly| directive
and thus should normally be avoided.
A method to include some content in the main file
by means of conditional processing is described in \secref{sec:conditional}.

%%%%%%%%%%%%%%%%%%%%%%%%%%%%%%%%%%%%%%%%
\paragraph{Page Numbering.}

When only a part of the document is compiled,
the appropriate numbering of pages
(as well as other status parameters)
is determined from the |.aux| files.
The latter contain information from previous passes.
However this information needs to propagate through
all intermediate child documents.
Therefore the page numbering in child documents may well
be inconsistent until the complete document is compiled at least once.

A useful (if unconventional) way to always ensure a consistent
page numbering is to restart the numbering in each child document
and denote the pages by `\textit{child}|.|\textit{page}'
where \textit{child} represents the chapter/section number of the child file.
This can be achieved by the command
|\numberwithin{page}{|\textit{child}|}|
of the \textsf{amsmath} package
where \textit{child} can be |chapter| or |section|
depending on the chosen structuring.
Alternatively, one can modify the macro |\thepage| appropriately
and reset the counter |page| at the start of each child file.

%%%%%%%%%%%%%%%%%%%%%%%%%%%%%%%%%%%%%%%%%%%%%%%%%%%%%%%%%%%%%%%%%%%%%%%%%%%%%%%%
\subsection{Conditional Processing}
\label{sec:conditional}

The package provides a mechanism to compile different versions
of a document. To customise the versions further some conditional processing
can come in handy to distinguish which version is being compiled.
The package provides two macros to describe the compilation context:

%%%%%%%%%%%%%%%%%%%%%%%%%%%%%%%%%%%%%%%%
\DescribeMacro{\ifchilddoc}
The conditional |\ifchilddoc| distinguishes between the compilation of
child documents and the main document:
%
\begin{center}
|\ifchilddoc |\textit{child-code}| |[|\||else |\textit{main-code}]| \||fi|
\end{center}

%%%%%%%%%%%%%%%%%%%%%%%%%%%%%%%%%%%%%%%%
\DescribeMacro{\childdocname}
\DescribeMacro{\childdocjob}
The macro |\childdocname| contains the filename (without extension)
of the main or child file being processed.
Note that |\childdocjob| will always contain the name of the main file.

%%%%%%%%%%%%%%%%%%%%%%%%%%%%%%%%%%%%%%%%
\paragraph{Title Page.}

Conditional processing can be used to include a title or banner page
in the main document when proper precautions are taken.
Importantly, the code in the main file should ensure that the page counter
(as well as other status parameters which are stored in the |.aux| files)
takes the same value after the conditional processing.
Otherwise the page numbers may take divergent values
depending on which part is compiled.

For example, a title page could be declared by:
%
\begin{center}
\begin{tabular}{l}
|\ifchilddoc\||else|\\
|\addtocounter{page}{-1}|\\
\textit{code for title page}\\
|\newpage|\\
|\||fi|
\end{tabular}
\end{center}
%
A banner page for the child documents can be generated by:
%
\begin{center}
\begin{tabular}{l}
|\ifchilddoc|\\
|\addtocounter{page}{-1}|\\
\textit{code for banner page}\\
|\newpage|\\
|\||fi|
\end{tabular}
\end{center}
%
Here one could write a message such as:
\begin{center}
|This is the part \childdocname{} of \childdocjob{}.|
\end{center}

%%%%%%%%%%%%%%%%%%%%%%%%%%%%%%%%%%%%%%%%%%%%%%%%%%%%%%%%%%%%%%%%%%%%%%%%%%%%%%%%
\subsection{Flags}
\label{sec:flags}

The package makes it easy to generate different versions
of the main or child documents.
To this end compilation flags can be defined
and assigned different default values.
They will be particularly useful in conjunction
with the forwarding mechanism described in \secref{sec:forward}.

For example, it may be useful to have a flag |\version|
which can be set to |draft| or |final|.
The document source will contain some conditional code
depending on the value of |\version|.
Suppose further, the flag should default to |final| for the main file
and to |draft| for child files
which is a natural assignment for editing the document.
This is achieved by placing the following code
in the preamble of the main document
(below the |\childdocmain| directive):
%
\begin{center}
\begin{tabular}{l}
|\ifchilddoc|\\
|\providecommand{\version}{draft}|\\
|\||else|\\
|\providecommand{\version}{final}|\\
|\||fi|
\end{tabular}
\end{center}
%
The definition by |\providecommand| makes sure
that previous definitions are not overwritten.
Further statements |\providecommand{\version}{...}|
can thus be added before the above code to override it.

For the main file, one might add a line
(between |\childdocmain| and the above block)
%
\begin{center}
|%\ifchilddoc\||else\providecommand{\version}{draft}\||fi|
\end{center}
%
which can be uncommented to produce a draft version.
Likewise one can add a line to the very top of a child file
(above the |\childdocof{|\textit{main}|}| directive)
%
\begin{center}
|%\providecommand{\version}{final}|
\end{center}
%
which can be uncommented to produce the final version of this child document.

%%%%%%%%%%%%%%%%%%%%%%%%%%%%%%%%%%%%%%%%%%%%%%%%%%%%%%%%%%%%%%%%%%%%%%%%%%%%%%%%
\subsection{Forwarding}
\label{sec:forward}

Different versions of the main or child documents
using compilation flags as described in \secref{sec:flags}
can be (permanently) stored in different files
for convenient compilation, viewing and distribution.
To this end, the package defines a command
to pass on compilation to a different file:

%%%%%%%%%%%%%%%%%%%%%%%%%%%%%%%%%%%%%%%%
\DescribeMacro{\childdocforward}
The command |\childdocforward| redirects processing to
another source file:
%
\begin{center}
\begin{tabular}{l}
|\input{childdoc.def}|\\
|\childdocforward[|\textit{main}|]{|\textit{dest}|}|\\
\end{tabular}
\end{center}
%
The argument \textit{dest} is the destination file
(without extension).
It should be the main file or one of the child files.
Note that further \textsf{childdoc} directives
such as |\childdocof| and |\childdocforward|
in the indicated file will be processed in this form.
The optional argument \textit{main}
passes on directly to the main file \textit{main}
while pretending to compile the child \textit{dest}.
This form behaves as if \textit{dest}
issues |\childdocof{|\textit{main}|}| right away,
and no further \textsf{childdoc} directives will be processed.

%%%%%%%%%%%%%%%%%%%%%%%%%%%%%%%%%%%%%%%%
\DescribeMacro{\...prefix}
In the alternative form |\childdocforwardprefix|,
%
\begin{center}
\begin{tabular}{l}
|\input{childdoc.def}|\\
|\childdocforwardprefix[|\textit{main}|]{|\textit{prefix}|}{|\textit{dest}|}|
\end{tabular}
\end{center}
%
the destination file is determined by a pattern
depending on the current file:
To make this work, the current file must be called
`{\textit{prefix}\hspace{0.2em}\textit{suffix}}'
with \textit{prefix} matching precisely the argument.
Processing is then passed on to the file
`{\textit{dest}\hspace{0.2em}\textit{suffix}}'.
Surely, the same effect is achieved by
directly specifying the
argument `{\textit{dest}\hspace{0.2em}\textit{suffix}}'
in the first form.
However, that requires to set up a different file
for each child. With the alternative form of the command
all these files can have exactly the same content
which simplifies setting them up and maintaining them.

For example, the following file |draft.tex|
with a compilation flag |\version| as described in \secref{sec:flags}
compiles the main document as a draft:
%
\begin{center}
\begin{tabular}{l}
|\def\version{draft}|\\
|\input{childdoc.def}|\\
|\childdocforward{|\textit{main}|}|
\end{tabular}
\end{center}
%
Likewise, the following files |final|\textit{nn}|.tex|
compile the final version of the child document
|child|\textit{nn}|.tex|:
%
\begin{center}
\begin{tabular}{l}
|\def\version{final}|\\
|\input{childdoc.def}|\\
|\childdocforwardprefix{final}{child}|
\end{tabular}
\end{center}
%

Note that when several versions of a main file and/or of each child file
are to be generated, it may be convenient to set up a |Makefile| or
shell script to automatise the process.

%%%%%%%%%%%%%%%%%%%%%%%%%%%%%%%%%%%%%%%%%%%%%%%%%%%%%%%%%%%%%%%%%%%%%%%%%%%%%%%%
\subsection{Command Line Processing}
\label{sec:commandline}

The effect of redirection files can also be achieved by invoking
the \LaTeX{} compiler with a more elaborate command line.
Most conveniently this should be done as part
of a shell script or a |Makefile|.

When using \textsf{childdoc} in the main file, the following
command lines effectively perform a redirection
(note that depending on the shell being used,
backslashes may have to be doubled: `|\|' $\to$ `|\\|'):
%
\begin{center}
|... -jobname "|\textit{target}|" |\\|"|[\textit{flags}]%
|\input{childdoc.def}\childdocforward[|\textit{main}|]{|\textit{dest}|}"|
\end{center}
%
Here \textit{target} is the name of the output file,
\textit{main} is the name of the main file
and \textit{dest} is the name of the main or child file to be processed
(all filenames without extensions).
The optional argument \textit{main} can be omitted
if \textit{main} matches \textit{dest}.
Optionally, compilation \textit{flags} can be defined via |\def| commands.
This command line makes the \TeX{} engine believe
it is compiling the file \textit{target}
whose content is specified as the latter parameter.
The provided code then forwards the processing to
\textit{main} or \textit{dest} as described in \secref{sec:forward}.

%%%%%%%%%%%%%%%%%%%%%%%%%%%%%%%%%%%%%%%%%%%%%%%%%%%%%%%%%%%%%%%%%%%%%%%%%%%%%%%%
\subsection{Include by Input}
\label{sec:input}

Including child documents by |\include| has some restrictions by design.
Most notably, the content of a child document always occupies
its own set of pages; pages cannot be shared between child documents.
Usually, this behaviour makes perfect sense
because each child document contain an essential part of the document.
However, in some situations it may be desirable to compose
a document from a collection of parts
without having mandatory page breaks between then.
For this case, the package
provides a mechanism to include parts
by |\input| which can also be processed individually.
However, by construction this mechanism
requires manual handling of the content to be output.

%%%%%%%%%%%%%%%%%%%%%%%%%%%%%%%%%%%%%%%%
\DescribeMacro{\ifchilddocmanual}
The main file should be prepared as usual, see \secref{sec:include}.
However, the document body must make a distinction
between processing of an individual part and of the main document, e.g.:
%
\begin{center}
\begin{tabular}{l}
|\ifchilddocmanual|\\
|\input{\childdocname}|\\
|\||else|\\
\textit{document body with }|\input{|\textit{part}|}|\\
|\||fi|
\end{tabular}
\end{center}
%
The conditional |\ifchilddocmanual| is true whenever
a part to be included by |\input| is being compiled,
and the name of the part is stored in |\childdocname|.

%%%%%%%%%%%%%%%%%%%%%%%%%%%%%%%%%%%%%%%%
\DescribeMacro{\childdocby}
Each part to be included by |\input| should start with:
%
\begin{center}
\begin{tabular}{l}
|\input{childdoc.def}|\\
|\childdocby{|\textit{main}|}|\\
\end{tabular}
\end{center}
%
The directive |\childdocby| is similar to |\childdocof|
described in \secref{sec:include},
but the subsequent selection of content must be done manually.
To that end, both |\ifchilddoc| and |\ifchilddocmanual|
will be true upon processing of a part,
and the name of the part is stored in |\childdocname|.
Note that |\jobname| will be set to the filename of the current part
so that each part receives an individual |.aux| file
that does not interfere with the |.aux| file(s) of the main document.
This behaviour can be altered by the alternative form
|\childdocby[*]{|\textit{main}|}| (with a non-empty optional argument)
which uses the |.aux| file of the main document
by setting |\jobname| to \textit{main}.

%%%%%%%%%%%%%%%%%%%%%%%%%%%%%%%%%%%%%%%%%%%%%%%%%%%%%%%%%%%%%%%%%%%%%%%%%%%%%%%%
\subsection{Driver Development}
\label{sec:driver}

The \textsf{childdoc} mechanism can also be use for the development
of definition files such as \LaTeX{} styles or classes.
This case differs from the above setup with multiple parts
included by |\include| in that no |\includeonly| should be invoked.
This can be achieved by starting the include file
(before |\ProvidesPackage|) with:
%
\begin{center}
\begin{tabular}{l}
|\input{childdoc.def}|\\
|\childdocforward{|\textit{main}|}|\\
\end{tabular}
\end{center}
%
or alternatively with:
%
\begin{center}
\begin{tabular}{l}
|\input{childdoc.def}|\\
|\childdocby{|\textit{main}|}|\\
\end{tabular}
\end{center}
%
Both forms have slightly different effects as described above.
The main file is prepared as usual, see \secref{sec:include}.

%%%%%%%%%%%%%%%%%%%%%%%%%%%%%%%%%%%%%%%%%%%%%%%%%%%%%%%%%%%%%%%%%%%%%%%%%%%%%%%%
\subsection{Legacy Detection}
\label{sec:detection}

The directive |\childdocmain| in the main file can detect
whether the complete document or merely a child is to be compiled
even without using the directive |\childdocof|.
This method is deprecated because it is less robust
and there is no compelling reason to use it;
it is merely provided for backward compatibility
and it may be removed in future versions.

If the detection mechanism is to be used,
it is mandatory to correctly specify
the filename of the main file as the argument of |\childdocmain|:
%
\begin{center}
\begin{tabular}{l}
|\input{childdoc.def}|\\
|\childdocmain{|\textit{main}|}|\\
\end{tabular}
\end{center}
%
If |\jobname| does not match the argument \textit{main} of |\childdocmain|,
it is assumed that |\jobname| points to the child file to be compiled.
When using |\childdocmain| with the main file specified as argument,
it suffices to start a child file
with just |\input{|\textit{main}|}|
without loading of the package and using |\childdocof|.
If instead all processing is done
with the appropriate \textsf{childdoc} directives,
the argument of \textit{main} of |\childdocmain| can be empty.

An alternative version of the command line processing described
in \secref{sec:commandline} using the detection mechanism reads:
%
\begin{center}
|... -jobname "|\textit{target}|" "|[\textit{flags}]%
[|\def\jobname{|\textit{dest}|}|]|\input{|\textit{main}|}"|
\end{center}

%%%%%%%%%%%%%%%%%%%%%%%%%%%%%%%%%%%%%%%%%%%%%%%%%%%%%%%%%%%%%%%%%%%%%%%%%%%%%%%%
\subsection{Manual Code}
\label{sec:manual}

In case one cannot be certain whether the definitions file |childdoc.def|
is installed on the target \TeX{} distribution
and one prefers not to ship it,
it is conceivable to paste a few relevant commands into the sources.

To that end, drop all statements |\input{childdoc.def}|
and perform the replacements as outlined below.
Instead of |\childdocmain{|\textit{main}|}| add the following code
to the top of the main file:
%
\begin{center}
\begin{tabular}{l}
|\||ifdefined\childdocname\endinput\||fi\newif\ifchilddoc|\\
|\edef\childdocname{\scantokens\expandafter{\jobname\noexpand}}|\\
|\def\childdocmain{|\textit{main}|}\||ifx\childdocmain\childdocname\||else|\\
|\childdoctrue\includeonly{\childdocname}\let\jobname\childdocmain\||fi|\\
\end{tabular}
\end{center}
%
Instead of |\childdocof{|\textit{main}|}| just include the main file
at the top of each child file:
%
\begin{center}
|\input{|\textit{main}|}|
\end{center}
%
A simple redirection |\childdocforward{|\textit{dest}|}| is achieved by:
%
\begin{center}
|\def\jobname{|\textit{dest}|}\input{\jobname}|
\end{center}
%
The redirection with prefix
|\childdocforwardprefix[|\textit{prefix}|]{|\textit{dest}|}|
is accomplished by:
%
\begin{center}
\begin{tabular}{l}
|{\edef\jobname{\scantokens\expandafter{\jobname\noexpand}}|\\
|\def\redirectjob |\textit{prefix}|#1~~~{\gdef\jobname{|\textit{dest}|#1}}|\\
|\expandafter\redirectjob\jobname~~~}\input{\jobname}|
\end{tabular}
\end{center}

In an alternative approach,
child documents can be compiled by a specific command line
without additional code or specific definitions:
%
\begin{center}
|... -jobname "|\textit{target}|" "|[\textit{flags}]%
|\includeonly{|\textit{dest}|}\input{|\textit{main}|}"|
\end{center}
%

%%%%%%%%%%%%%%%%%%%%%%%%%%%%%%%%%%%%%%%%%%%%%%%%%%%%%%%%%%%%%%%%%%%%%%%%%%%%%%%%
%%%%%%%%%%%%%%%%%%%%%%%%%%%%%%%%%%%%%%%%%%%%%%%%%%%%%%%%%%%%%%%%%%%%%%%%%%%%%%%%
\section{Information}

%%%%%%%%%%%%%%%%%%%%%%%%%%%%%%%%%%%%%%%%%%%%%%%%%%%%%%%%%%%%%%%%%%%%%%%%%%%%%%%%
\subsection{Copyright}

Copyright \copyright{} 2017--2018 Niklas Beisert

This work may be distributed and/or modified under the
conditions of the \LaTeX{} Project Public License, either version 1.3
of this license or (at your option) any later version.
The latest version of this license is in
  \url{http://www.latex-project.org/lppl.txt}
and version 1.3 or later is part of all distributions of \LaTeX{}
version 2005/12/01 or later.

This work has the LPPL maintenance status `maintained'.

The Current Maintainer of this work is Niklas Beisert.

This work consists of the files |README.txt|, |childdoc.ins| and |childdoc.dtx|
as well as the derived files |childdoc.def|, |cdocsamp.tex|
with |cdocsch1.tex|, |cdocsch2.tex|, |cdocspt3.tex|, |cdocspt4.tex|,
|cdocsdrf.tex|, |cdocsfn1.tex|, |cdocsfn2.tex|
as well as |childdoc.pdf|.

%%%%%%%%%%%%%%%%%%%%%%%%%%%%%%%%%%%%%%%%%%%%%%%%%%%%%%%%%%%%%%%%%%%%%%%%%%%%%%%%
\subsection{Files and Installation}

The package consists of the files:
%
\begin{center}
\begin{tabular}{ll}
    |README.txt|   & readme file \\
    |childdoc.ins| & installation file \\
    |childdoc.dtx| & source file \\
    |childdoc.def| & definition file \\
    |cdocsamp.tex| & sample main file \\
    |cdocsch1.tex| & sample include file \\
    |cdocsch2.tex| & sample include file \\
    |cdocspt3.tex| & sample part file \\
    |cdocspt4.tex| & sample part file \\
    |cdocsdrf.tex| & sample redirection file \\
    |cdocsfn1.tex| & sample redirection file \\
    |cdocsfn2.tex| & sample redirection file \\
    |childdoc.pdf| & manual
\end{tabular}
\end{center}
%
The distribution consists of the files
|README.txt|, |childdoc.ins| and |childdoc.dtx|.
%
\begin{itemize}
\item
Run (pdf)\LaTeX{} on |childdoc.dtx|
to compile the manual |childdoc.pdf| (this file).
\item
Run \LaTeX{} on |childdoc.ins| to create the definitions file |childdoc.def|
and the sample |cdocsamp.tex| with include files
|cdocsch1.tex|, |cdocsch2.tex|, |cdocspt3.tex|, |cdocspt4.tex|,
|cdocsdrf.tex|, |cdocsfn1.tex|, |cdocsfn2.tex|.
Then copy the file |childdoc.def| to an appropriate directory of your \LaTeX{}
distribution, e.g.\ \textit{texmf-root}|/tex/latex/childdoc|.
\end{itemize}

%%%%%%%%%%%%%%%%%%%%%%%%%%%%%%%%%%%%%%%%%%%%%%%%%%%%%%%%%%%%%%%%%%%%%%%%%%%%%%%%
\subsection{Related CTAN Packages}

There are several other packages which offer a similar functionality:
%
\begin{itemize}
\item
The packages
\href{http://ctan.org/pkg/docmute}{\textsf{docmute}},
\href{http://ctan.org/pkg/includex}{\textsf{includex}} and
\href{http://ctan.org/pkg/standalone}{\textsf{standalone}}
provide commands to include only the document body of
a child file thus allowing both files to be compiled individually.
\item
The packages \href{http://ctan.org/pkg/subdocs}{\textsf{subdocs}}
and \href{http://ctan.org/pkg/subfiles}{\textsf{subfiles}}
provide structures in which the main and child documents can be
encapsulated and allowing them to be compiled individually.
The inclusion mechanism is different from the conventional |\include|.
\item
The package \href{http://ctan.org/pkg/combine}{\textsf{combine}}
is an elaborate solution to combine several documents into one.
\end{itemize}
%
See also the CTAN topic \href{http://ctan.org/topic/subdocs}{\textsf{subdocs}}
for further related packages.
The present package differs from the above solutions in that
a document structure constructed with the conventional |\include| mechanism
just needs two extra commands at the top of every file
such that all constituent files can be compiled individually.

%%%%%%%%%%%%%%%%%%%%%%%%%%%%%%%%%%%%%%%%%%%%%%%%%%%%%%%%%%%%%%%%%%%%%%%%%%%%%%%%
%\subsection{Feature Suggestions}
%
%The following is a list of features which may be useful for future
%versions of this package:
%%
%\begin{itemize}
%\item
%\ldots
%\end{itemize}

%%%%%%%%%%%%%%%%%%%%%%%%%%%%%%%%%%%%%%%%%%%%%%%%%%%%%%%%%%%%%%%%%%%%%%%%%%%%%%%%
\subsection{Revision History}

%%%%%%%%%%%%%%%%%%%%%%%%%%%%%%%%%%%%%%%%
\paragraph{v2.0:} 2018/12/30

\begin{itemize}
\item
immediate forward processing
\item
added |\childdocby| mechanism
\item
manual restructured
\end{itemize}

%%%%%%%%%%%%%%%%%%%%%%%%%%%%%%%%%%%%%%%%
\paragraph{v1.6:} 2018/01/17

\begin{itemize}
\item
application for development of include files
\item
corrections to manual
\end{itemize}

%%%%%%%%%%%%%%%%%%%%%%%%%%%%%%%%%%%%%%%%
\paragraph{v1.5:} 2017/05/21

\begin{itemize}
\item
more complete structuring introduced
\item
|\childdocof| introduced
\item
|\childdoc| renamed to |\childdocmain|
\item
|\childredirect| renamed to |\childdocforward| and |\childdocforwardprefix|
and functionality expanded
\end{itemize}

%%%%%%%%%%%%%%%%%%%%%%%%%%%%%%%%%%%%%%%%
\paragraph{v1.0:} 2017/04/27

\begin{itemize}
\item
manual and install package
\item
first version published on CTAN
\end{itemize}

%%%%%%%%%%%%%%%%%%%%%%%%%%%%%%%%%%%%%%%%
\paragraph{v0.6:} 2017/04/26

\begin{itemize}
\item
redirection mechanism added
\end{itemize}

%%%%%%%%%%%%%%%%%%%%%%%%%%%%%%%%%%%%%%%%
\paragraph{v0.5:} 2017/04/26

\begin{itemize}
\item
functionality in definition file
\end{itemize}


%%%%%%%%%%%%%%%%%%%%%%%%%%%%%%%%%%%%%%%%%%%%%%%%%%%%%%%%%%%%%%%%%%%%%%%%%%%%%%%%
%%%%%%%%%%%%%%%%%%%%%%%%%%%%%%%%%%%%%%%%%%%%%%%%%%%%%%%%%%%%%%%%%%%%%%%%%%%%%%%%
%%%%%%%%%%%%%%%%%%%%%%%%%%%%%%%%%%%%%%%%%%%%%%%%%%%%%%%%%%%%%%%%%%%%%%%%%%%%%%%%
\appendix

\settowidth\MacroIndent{\rmfamily\scriptsize 000\ }

 \DocInput{childdoc.dtx}

\end{document}
%</driver>
% \fi
%
% %%%%%%%%%%%%%%%%%%%%%%%%%%%%%%%%%%%%%%%%%%%%%%%%%%%%%%%%%%%%%%%%%%%%%%%%%%%%%%
% %%%%%%%%%%%%%%%%%%%%%%%%%%%%%%%%%%%%%%%%%%%%%%%%%%%%%%%%%%%%%%%%%%%%%%%%%%%%%%
% \section{Sample}
%\iffalse
%<*samplemain>
%\fi
%
% The following presents a sample document
% with two chapters, two parts, a title page,
% a compile flag as well as three forwarding files to set the flag.
% It consists of eight |.tex| files:
% \begin{center}
% \begin{tabular}{ll}
% |cdocsamp.tex|&main file\\
% |cdocsch1.tex|&include file for chapter 1\\
% |cdocsch2.tex|&include file for chapter 2\\
% |cdocspt3.tex|&include file for part 3\\
% |cdocspt4.tex|&include file for part 4\\
% |cdocsdrf.tex|&forwarding file for main file in draft mode\\
% |cdocsfi1.tex|&forwarding file for final version of chapter 1\\
% |cdocsfi2.tex|&forwarding file for final version of chapter 2\\
% \end{tabular}
% \end{center}
% Each of the eight files can be compiled directly by the \LaTeX{} compiler.
%
% %%%%%%%%%%%%%%%%%%%%%%%%%%%%%%%%%%%%%%
% \paragraph{Main File.}
%
% The main file is called |cdocsamp.tex|.
%
% Load the \textsf{childdoc} definitions and
% declare the filename for the main document:
%    \begin{macrocode}
\input{childdoc.def}
\childdocmain{}
%    \end{macrocode}

% Optional override for |\version| flag:
%    \begin{macrocode}
%%\ifchilddoc\else\providecommand{\version}{draft}\fi
%    \end{macrocode}

% Define the default values for the |\version| flag
% (|final| for the main file and |draft| for childs):
%    \begin{macrocode}
\ifchilddoc
\providecommand{\version}{draft}
\else
\providecommand{\version}{final}
\fi
%    \end{macrocode}

% Load the standard document class:
%    \begin{macrocode}
\documentclass[12pt]{article}
%    \end{macrocode}

% Start the document body:
%    \begin{macrocode}
\begin{document}
%    \end{macrocode}

% Declare a title page.
% Print title, part of document being processed and version flag:
%    \begin{macrocode}
\addtocounter{page}{-1}
\begin{center}
{\LARGE\bfseries{}childdoc example\par}
\vspace{1cm}
\ifchilddoc
\ifchilddocmanual part\else chapter\fi:
`\childdocname' of `\childdocjob'\par
\else
main document: `\childdocjob'\par
\fi
version: \version\par
\end{center}
\newpage
%    \end{macrocode}

% Manually include selected file,
% otherwise process as usual:
%    \begin{macrocode}
\ifchilddocmanual
\section*{part `\childdocname'}
\input{\childdocname}
\else
%    \end{macrocode}

% Include the two chapters:
%    \begin{macrocode}
\include{cdocsch1}
\include{cdocsch2}
%    \end{macrocode}

% Include the two parts unless only chapters should be displayed:
%    \begin{macrocode}
\ifchilddoc\else
\section{part three}
\input{cdocspt3}
\section{part four}
\input{cdocspt4}
\fi
%    \end{macrocode}

% Process as usual until here:
%    \begin{macrocode}
\fi
%    \end{macrocode}

% End of document body:
%    \begin{macrocode}
\end{document}
%    \end{macrocode}
%\iffalse
%</samplemain>
%\fi
%
% %%%%%%%%%%%%%%%%%%%%%%%%%%%%%%%%%%%%%%
% \paragraph{Chapter Include Files.}
%
% The include files are called |cdocsch1.tex| and |cdocsch2.tex|.
%
%\iffalse
%<*samplechap1|samplechap2>
%\fi

% Optional override for |\version| flag:
%    \begin{macrocode}
%%\providecommand{\version}{final}
%    \end{macrocode}

% Include the main document:
%    \begin{macrocode}
\input{childdoc.def}
\childdocof{cdocsamp}
%    \end{macrocode}

%\iffalse
%</samplechap1|samplechap2>
%\fi
%
%\iffalse
%<*samplechap1>
%\fi
% Some text for chapter 1:
%    \begin{macrocode}
\section{one}
some text in chapter one
%    \end{macrocode}

%\iffalse
%</samplechap1>
%\fi
% Some text for chapter 2:
%\iffalse
%<*samplechap2>
%\fi
%    \begin{macrocode}
\section{two}
more text in chapter two
%    \end{macrocode}

%\iffalse
%</samplechap2>
%\fi
%
% %%%%%%%%%%%%%%%%%%%%%%%%%%%%%%%%%%%%%%
% \paragraph{Part Include Files.}
%
% The include files are called |cdocspt3.tex| and |cdocspt4.tex|.
%
%\iffalse
%<*samplepart3|samplepart4>
%\fi

% Optional override for |\version| flag:
%    \begin{macrocode}
%%\providecommand{\version}{final}
%    \end{macrocode}

% Include the main document:
%    \begin{macrocode}
\input{childdoc.def}
\childdocby{cdocsamp}
%    \end{macrocode}

%\iffalse
%</samplepart3|samplepart4>
%\fi
%
%\iffalse
%<*samplepart3>
%\fi
% Some text for part 3:
%    \begin{macrocode}
some text in part three
%    \end{macrocode}

%\iffalse
%</samplepart3>
%\fi
% Some text for part 4:
%\iffalse
%<*samplepart4>
%\fi
%    \begin{macrocode}
more text in part four
%    \end{macrocode}

%\iffalse
%</samplepart4>
%\fi
%
% %%%%%%%%%%%%%%%%%%%%%%%%%%%%%%%%%%%%%%
% \paragraph{Forwarding for a Complete Draft.}
%
% The following forwarding file |cdocsdrf.tex|
% compiles the main document in draft mode:
%\iffalse
%<*sampledraft>
%\fi
%    \begin{macrocode}
\def\version{draft}
\input{childdoc.def}
\childdocforward{cdocsamp}
%    \end{macrocode}

%\iffalse
%</sampledraft>
%\fi
%
% %%%%%%%%%%%%%%%%%%%%%%%%%%%%%%%%%%%%%%
% \paragraph{Forwarding for Final Version of the Chapters.}
%
% The following forwarding files |cdocsfn1.tex| and |cdocsfn2.tex|
% (with identical content)
% compile the final versions of the child documents
% |cdocsch1.tex| and |cdocsch2.tex|, respectively:
%\iffalse
%<*samplefinal>
%\fi
%    \begin{macrocode}
\def\version{final}
\input{childdoc.def}
\childdocforwardprefix[cdocsamp]{cdocsfn}{cdocsch}
%    \end{macrocode}

%\iffalse
%</samplefinal>
%\fi
%
% %%%%%%%%%%%%%%%%%%%%%%%%%%%%%%%%%%%%%%
% \paragraph{Command Line Processing.}
%
% The following three command lines generate the output files
% |cdocscld|, |cdocscl1| and |cdocscl2|
% which should be identical to
% |cdocsdrf|, |cdocsch1| and |cdocsfn2|, respectively:
% \begin{center}
% \begin{tabular}{l}
% |latex -jobname cdocscld \|\\
% |  "\def\version{draft}\input{childdoc.def}\childdocforward{cdocsamp}"|\\
% |latex -jobname cdocscl1 \|\\
% |  "\input{childdoc.def}\childdocforward[cdocsamp]{cdocsch1}"|\\
% |latex -jobname cdocscl2 \|\\
% |  "\def\version{final}\input{childdoc.def}\childdocforward{cdocsch2}"|
% \end{tabular}
% \end{center}
% Note that the trailing backslash on each first line
% merely continues the input to the second line
% (for convenient cut ant paste).
% Furthermore, the command |latex| can be replaced by any
% of its alternative versions such as |pdflatex|.
%
% %%%%%%%%%%%%%%%%%%%%%%%%%%%%%%%%%%%%%%%%%%%%%%%%%%%%%%%%%%%%%%%%%%%%%%%%%%%%%%
% %%%%%%%%%%%%%%%%%%%%%%%%%%%%%%%%%%%%%%%%%%%%%%%%%%%%%%%%%%%%%%%%%%%%%%%%%%%%%%
% \section{Implementation}
%\iffalse
%<*package>
%\fi
%
% This section describes the definitions file |childdoc.def|.

% The definitions cannot be loaded using |\usepackage| or |\RequirePackage|
% which has a mechanism to prevent loading a style file more than once.
% When loading the definitions by means of |\input|
% multiple instances have to be prevented manually:
%\iffalse
%This code needs to be before the `\ProvidesFile' directive
%which is defined at the beginning of this file.
%Therefore it is also placed there and commented out here.
%</package>
%<*discard>
%\fi
%    \begin{macrocode}
\ifdefined\childdocmain\endinput\fi
%    \end{macrocode}
%\iffalse
%</discard>
%<*package>
%\fi
%
% \macro{\ifchilddoc}
% \macro{\ifchilddocmanual}
% The conditional |\ifchilddoc| tells whether a
% child (true) or main (false) document is being compiled.
% The conditional |\ifchilddocmanual| tells whether
% the |\includeonly| mechanism is used (false) or
% the selection of child files must be performed manually (true).
% The definitions initialise to false:
%    \begin{macrocode}
\newif\ifchilddoc
\newif\ifchilddocmanual
%    \end{macrocode}

% \macro{\childdocname}
% \macro{\childdocjob}
% The macro |\childdocname| stores the name of the main document
% to be compiled. The macro |\childdocjob| stores the name of
% the document on which the \LaTeX{} compiler was originally invoked.
% The content of |\jobname| cannot be compared
% to filenames specified in the source due to different catcodes.
% The following code rescans |\jobname|, stores the result
% in |\childdocname| and saves a copy in |\childdocjob|:
%    \begin{macrocode}
\edef\childdocname{\scantokens\expandafter{\jobname\noexpand}}
\let\childdocjob\childdocname
%    \end{macrocode}

% \macro{\childdocdisable}
% The macro |\childdocdisable| prevents the main file
% from being processed more than once.
% At this stage, the main document command |\childdocmain|
% is assumed to be called once again where it should do nothing.
% Any subsequent call to it should prevent
% a secondary processing of the main document
% It overwrites the forwarding commands
% |\childdocof| and |\childdocforward|
% with empty macros to prevent further inclusions of the main document:
%    \begin{macrocode}
\newcommand{\childdocdisable}
{
  \renewcommand{\childdocmain}[1]{\renewcommand{\childdocmain}[1]{\endinput}}
  \renewcommand{\childdocof}[1]{}
  \renewcommand{\childdocby}[2][]{}
  \renewcommand{\childdocforward}[2][]{}
  \renewcommand{\childdocdisable}{}
}
%    \end{macrocode}

% \macro{\childdocmain}
% The macro |\childdocmain| is to be called at the top of the main file
% with nothing or the main filename (without extension) as argument.
% First, it breaks loops.
% If the argument is not empty and does not match |\childdocname|
% (which is set by the first inclusion of |childdoc.def|),
% |\ifchilddoc| is set to true, |\includeonly| is applied to the child file
% and |\jobname| is set to the main file
% (for proper handling of |.aux| files):
%    \begin{macrocode}
\newcommand{\childdocmain}[1]
{
  \childdocdisable\childdocmain{}
  \if?#1?\else
    \begingroup
      \def\childdoctmp{#1}
      \ifx\childdoctmp\childdocname
        \def\childdoctmp{}
      \else
        \def\childdoctmp
        {
          \childdoctrue
          \includeonly{\childdocname}
          \def\childdocjob{#1}
          \def\jobname{#1}
        }
      \fi
      \expandafter
    \endgroup
    \childdoctmp
  \fi
}
%    \end{macrocode}

% \macro{\childdocof}
% The command |\childdocof| redirects
% compilation to the main file |#1|.
%    \begin{macrocode}
\newcommand{\childdocof}[1]
{
  \childdocdisable
  \childdoctrue
  \includeonly{\childdocname}
  \def\jobname{#1}
  \def\childdocjob{#1}
  \input{#1}
}
%    \end{macrocode}

% \macro{\childdocby}
% The command |\childdocby| ....
%    \begin{macrocode}
\newcommand{\childdocby}[2][]
{
  \childdocdisable
  \childdoctrue
  \childdocmanualtrue
  \if?#1?\else
    \def\jobname{#2}
  \fi
  \def\childdocjob{#2}
  \input{#2}
  \endinput
}
%    \end{macrocode}

% \macro{\childdocforward}
% The command |\childdocforward| redirects
% compilation to the main file or
% (if the optional argument is given) a child file.
% Parameters are set as if the main file
% or a child file starting with |\childdocof| was compiled.
% Then compilation is handed over to the main file:
%    \begin{macrocode}
\newcommand{\childdocforward}[2][]
{
  \begingroup
    \if?#1?
      \def\childdoctmp
      {
        \def\childdocname{#2}
        \def\childdocjob{#2}
        \def\jobname{#2}
        \input{#2}
        \endinput
      }
    \else
      \def\childdoctmp
      {
        \childdocdisable
        \def\childdocname{#2}
        \childdoctrue
        \includeonly{#2}
        \def\childdocjob{#1}
        \def\jobname{#1}
        \input{#1}
        \endinput
      }
    \fi
    \expandafter
  \endgroup
  \childdoctmp
}
%    \end{macrocode}

% \macro{\childdocforwardprefix}
% The command |\childdocforwardprefix| redirects
% compilation to the main or a child file by means of a pattern.
% The prefix |#1| in the current filename is replaced by |#2|
% and the suffix of the current filename is kept
% (it is assumed that the filename does not contain the substring `|~~~|'
% which is used as a delimiter).
% Compilation is handed over to the new file by |\childdocforward|:
%    \begin{macrocode}
\newcommand{\childdocforwardprefix}[3][]
{
  \begingroup
    \def\childdocextract #2##1~~~{\def\childdoctmp{\childdocforward[#1]{#3##1}}}
    \expandafter\childdocextract\childdocname~~~
    \expandafter
  \endgroup
  \childdoctmp
}
%    \end{macrocode}

% \macro{\childdoc}
% The deprecated macro |\childdoc| is a legacy version of |\childdocmain|:
%    \begin{macrocode}
\newcommand{\childdoc}{\childdocmain}
%    \end{macrocode}

% \macro{\childdocredirect}
% The deprecated macro |\childdocredirect| is a legacy version
% of |\childdocforward| and |\childdocforwardprefix|:
%    \begin{macrocode}
\newcommand{\childdocredirect}[2][]
{
  \begingroup
    \if?#1?
      \def\childdoctmp{\childdocforward{#2}}
    \else
      \def\childdoctmp{\childdocforwardprefix{#1}{#2}}
    \fi
    \expandafter
  \endgroup
  \childdoctmp
}
%    \end{macrocode}

%\iffalse
%</package>
%\fi
%
\endinput
\childdocforward[|\textit{main}|]{|\textit{dest}|}"|
\end{center}
%
Here \textit{target} is the name of the output file,
\textit{main} is the name of the main file
and \textit{dest} is the name of the main or child file to be processed
(all filenames without extensions).
The optional argument \textit{main} can be omitted
if \textit{main} matches \textit{dest}.
Optionally, compilation \textit{flags} can be defined via |\def| commands.
This command line makes the \TeX{} engine believe
it is compiling the file \textit{target}
whose content is specified as the latter parameter.
The provided code then forwards the processing to
\textit{main} or \textit{dest} as described in \secref{sec:forward}.

%%%%%%%%%%%%%%%%%%%%%%%%%%%%%%%%%%%%%%%%%%%%%%%%%%%%%%%%%%%%%%%%%%%%%%%%%%%%%%%%
\subsection{Include by Input}
\label{sec:input}

Including child documents by |\include| has some restrictions by design.
Most notably, the content of a child document always occupies
its own set of pages; pages cannot be shared between child documents.
Usually, this behaviour makes perfect sense
because each child document contain an essential part of the document.
However, in some situations it may be desirable to compose
a document from a collection of parts
without having mandatory page breaks between then.
For this case, the package
provides a mechanism to include parts
by |\input| which can also be processed individually.
However, by construction this mechanism
requires manual handling of the content to be output.

%%%%%%%%%%%%%%%%%%%%%%%%%%%%%%%%%%%%%%%%
\DescribeMacro{\ifchilddocmanual}
The main file should be prepared as usual, see \secref{sec:include}.
However, the document body must make a distinction
between processing of an individual part and of the main document, e.g.:
%
\begin{center}
\begin{tabular}{l}
|\ifchilddocmanual|\\
|\input{\childdocname}|\\
|\||else|\\
\textit{document body with }|\input{|\textit{part}|}|\\
|\||fi|
\end{tabular}
\end{center}
%
The conditional |\ifchilddocmanual| is true whenever
a part to be included by |\input| is being compiled,
and the name of the part is stored in |\childdocname|.

%%%%%%%%%%%%%%%%%%%%%%%%%%%%%%%%%%%%%%%%
\DescribeMacro{\childdocby}
Each part to be included by |\input| should start with:
%
\begin{center}
\begin{tabular}{l}
|% \iffalse
%
% childdoc.dtx Copyright (C) 2017-2018 Niklas Beisert
%
% This work may be distributed and/or modified under the
% conditions of the LaTeX Project Public License, either version 1.3
% of this license or (at your option) any later version.
% The latest version of this license is in
%   http://www.latex-project.org/lppl.txt
% and version 1.3 or later is part of all distributions of LaTeX
% version 2005/12/01 or later.
%
% This work has the LPPL maintenance status `maintained'.
%
% The Current Maintainer of this work is Niklas Beisert.
%
% This work consists of the files childdoc.dtx and childdoc.ins
% and the derived files childdoc.def and cdocsamp.tex with
% cdocsch1.tex, cdocsch2.tex, cdocsdrf.tex, cdocsfn1.tex, cdocsfn2.tex.
%
%<package>\ifdefined\childdocmain\endinput\fi
%<package>\ProvidesFile{childdoc.def}[2018/12/30 v2.0 child document driver]
%<samplemain>\ProvidesFile{cdocsamp.tex}[2018/12/30 v2.0 sample for childdoc]
%<*driver>
%\ProvidesFile{childdoc.drv}[2018/12/30 v2.0 childdoc reference manual file]
\PassOptionsToClass{10pt,a4paper}{article}
\documentclass{ltxdoc}

\usepackage[margin=35mm]{geometry}
\usepackage{hyperref}
\usepackage{hyperxmp}
\usepackage[usenames]{color}

\hypersetup{colorlinks=true}
\hypersetup{pdfstartview=FitH}
\hypersetup{pdfpagemode=UseNone}
\hypersetup{pdfsource={}}
\hypersetup{pdflang={en-UK}}
\hypersetup{pdfcopyright={Copyright 2017-2018 Niklas Beisert.
  This work may be distributed and/or modified under the
  conditions of the LaTeX Project Public License, either version 1.3
  of this license or (at your option) any later version.}}
\hypersetup{pdflicenseurl={http://www.latex-project.org/lppl.txt}}
\hypersetup{pdfcontactaddress={ETH Zurich, ITP, HIT K,
  Wolfgang-Pauli-Strasse 27}}
\hypersetup{pdfcontactpostcode={8093}}
\hypersetup{pdfcontactcity={Zurich}}
\hypersetup{pdfcontactcountry={Switzerland}}
\hypersetup{pdfcontactemail={nbeisert@itp.phys.ethz.ch}}
\hypersetup{pdfcontacturl={http://people.phys.ethz.ch/\xmptilde nbeisert/}}

\newcommand{\secref}[1]{\hyperref[#1]{section \ref*{#1}}}

\parskip1ex
\parindent0pt
\let\olditemize\itemize
\def\itemize{\olditemize\parskip0pt}

\begin{document}

\title{The \textsf{childdoc} Package}
\hypersetup{pdftitle={The childdoc Package}}
\author{Niklas Beisert\\[2ex]
  Institut f\"ur Theoretische Physik\\
  Eidgen\"ossische Technische Hochschule Z\"urich\\
  Wolfgang-Pauli-Strasse 27, 8093 Z\"urich, Switzerland\\[1ex]
  \href{mailto:nbeisert@itp.phys.ethz.ch}
  {\texttt{nbeisert@itp.phys.ethz.ch}}}
\hypersetup{pdfauthor={Niklas Beisert}}
\hypersetup{pdfsubject={Manual for the LaTeX2e Package childdoc}}
\date{30 December 2018, \textsf{v2.0}}
\maketitle

\begin{abstract}\noindent
\textsf{childdoc} is a \LaTeXe{} package
that enables the direct compilation
of document sections included by |\include|
to individual files.
\end{abstract}

\begingroup
\parskip0ex
\tableofcontents
\endgroup

%%%%%%%%%%%%%%%%%%%%%%%%%%%%%%%%%%%%%%%%%%%%%%%%%%%%%%%%%%%%%%%%%%%%%%%%%%%%%%%%
%%%%%%%%%%%%%%%%%%%%%%%%%%%%%%%%%%%%%%%%%%%%%%%%%%%%%%%%%%%%%%%%%%%%%%%%%%%%%%%%
\section{Introduction}

\LaTeX{} provides a mechanism to structure a large document (such as a book)
into a main file and several child files (containing the chapters)
using the |\include| command.
This mechanism is beneficial for documents
which span hundreds of pages in order to
make the source file(s) more manageable.
Moreover, compilation can be restricted to
selected child files by means of the |\includeonly| command.
The latter feature can be used to reduce the compilation time while editing
(this was significantly more useful in the earlier days of \LaTeX{})
or to generate a smaller document which is easier to navigate.
Another application of |\includeonly| is to generate
documents consisting of selected parts of the complete document.

However, there are a few drawbacks of the plain |\include| mechanism:
\begin{itemize}
\item
The child files cannot be compiled on their own,
they can only be compiled via the main file.
A naive editing environment
(such as a text editor with an option
to have the current file processed by \LaTeX)
may require one to switch to the main file before compiling;
attempting to compile the child file produces errors.
\item
The main file must be modified (each time)
to adjust the |\includeonly| command
to the present needs. This easily leaves the main file in a messy state.
\item
The generated document will always carry the filename
of the main document. This is inconvenient if
several child files are to be compiled and
to be kept for distribution.
\end{itemize}

The present package provides a simple interface
to make child files individually compilable by \LaTeX{}.
Compiling a child file then has the same effect as compiling
the main file with an |\includeonly| command
to select the appropriate child.
Moreover the generated document will carry the name of the child
rather than the main file.
This resolves all three above issues.

This feature is meant to make the editing of books,
thesis documents and lecture notes somewhat more convenient.
However, the package can also be used efficiently for
composing a series of documents (such as exercise sheets)
which are typically distributed individually.
It then assists the author in generating the individual documents
(potentially in different versions)
as well as a document containing the collected series.
Another application is in developing style files
or other kinds of included material
where compilation of the style file could redirect
to a sample or test file.

%%%%%%%%%%%%%%%%%%%%%%%%%%%%%%%%%%%%%%%%%%%%%%%%%%%%%%%%%%%%%%%%%%%%%%%%%%%%%%%%
%%%%%%%%%%%%%%%%%%%%%%%%%%%%%%%%%%%%%%%%%%%%%%%%%%%%%%%%%%%%%%%%%%%%%%%%%%%%%%%%
\section{Usage}

First of all, the package \textsf{childdoc} is \emph{not} a standard
\LaTeXe{} |.sty| style file! Therefore it needs to be invoked in
a non-standard way.

%%%%%%%%%%%%%%%%%%%%%%%%%%%%%%%%%%%%%%%%%%%%%%%%%%%%%%%%%%%%%%%%%%%%%%%%%%%%%%%%
\subsection{Included Files}
\label{sec:include}

%%%%%%%%%%%%%%%%%%%%%%%%%%%%%%%%%%%%%%%%
\DescribeMacro{\childdocmain}
To use the package, add the commands
\begin{center}
\begin{tabular}{l}
|\input{childdoc.def}|\\
|\childdocmain{}|\\
\end{tabular}
\end{center}
at the very top of the main \LaTeX{} file,
in particular \emph{before} the |\documentclass| statement!
The argument of |\childdocmain| should be left empty
(but it must be present).

%%%%%%%%%%%%%%%%%%%%%%%%%%%%%%%%%%%%%%%%
\DescribeMacro{\childdocof}
Furthermore, add the commands
\begin{center}
\begin{tabular}{l}
|\input{childdoc.def}|\\
|\childdocof{|\textit{main}|}|\\
\end{tabular}
\end{center}
at the top of every child file \textit{child}
which is included by |\include{|\textit{child}|}|
from within the main file
(or at least for those files to be compiled individually).
The argument \textit{main} must be the filename of the main file.

There are a couple of
considerations in setting up the main and child documents:

%%%%%%%%%%%%%%%%%%%%%%%%%%%%%%%%%%%%%%%%
\paragraph{Restrictions.}

Please note the following restrictions:
\begin{itemize}
\item
|\childdocmain| must be called with one argument \textit{main}
to ensure compatibility with earlier version of the package.
It must either be empty (|\childdocmain{}|)
or precisely match the filename of the main file in which it is specified.
See \secref{sec:detection} for further information.
\item
The filename \textit{main} must be specified without the |.tex| extension.
\item
The filename \textit{main} is case sensitive
(even in case-insensitive file systems)
due to internal string comparison.
\item
The argument \textit{main} should be fully expanded, it cannot be a macro.
\item
Subdirectories and special characters should be avoided in filenames.
\item
The command |\childdocmain{|\textit{main}|}| must be followed by a whitespace.
It should not be followed immediately by another command
or by a comment mark `|%|'.
This is because the \TeX{} parser reads the token immediately following
the argument of |\childdocmain| and puts it
at the beginning of every child section;
however, a white\-space is ignored.
\end{itemize}

%%%%%%%%%%%%%%%%%%%%%%%%%%%%%%%%%%%%%%%%
\paragraph{Content of Main File.}

It is advisable to place all content in the child files included by |\include|.
Any output contained in the main file will appear in all child documents
unless suppressed manually;
it cannot be suppressed automatically by the |\includeonly| directive
and thus should normally be avoided.
A method to include some content in the main file
by means of conditional processing is described in \secref{sec:conditional}.

%%%%%%%%%%%%%%%%%%%%%%%%%%%%%%%%%%%%%%%%
\paragraph{Page Numbering.}

When only a part of the document is compiled,
the appropriate numbering of pages
(as well as other status parameters)
is determined from the |.aux| files.
The latter contain information from previous passes.
However this information needs to propagate through
all intermediate child documents.
Therefore the page numbering in child documents may well
be inconsistent until the complete document is compiled at least once.

A useful (if unconventional) way to always ensure a consistent
page numbering is to restart the numbering in each child document
and denote the pages by `\textit{child}|.|\textit{page}'
where \textit{child} represents the chapter/section number of the child file.
This can be achieved by the command
|\numberwithin{page}{|\textit{child}|}|
of the \textsf{amsmath} package
where \textit{child} can be |chapter| or |section|
depending on the chosen structuring.
Alternatively, one can modify the macro |\thepage| appropriately
and reset the counter |page| at the start of each child file.

%%%%%%%%%%%%%%%%%%%%%%%%%%%%%%%%%%%%%%%%%%%%%%%%%%%%%%%%%%%%%%%%%%%%%%%%%%%%%%%%
\subsection{Conditional Processing}
\label{sec:conditional}

The package provides a mechanism to compile different versions
of a document. To customise the versions further some conditional processing
can come in handy to distinguish which version is being compiled.
The package provides two macros to describe the compilation context:

%%%%%%%%%%%%%%%%%%%%%%%%%%%%%%%%%%%%%%%%
\DescribeMacro{\ifchilddoc}
The conditional |\ifchilddoc| distinguishes between the compilation of
child documents and the main document:
%
\begin{center}
|\ifchilddoc |\textit{child-code}| |[|\||else |\textit{main-code}]| \||fi|
\end{center}

%%%%%%%%%%%%%%%%%%%%%%%%%%%%%%%%%%%%%%%%
\DescribeMacro{\childdocname}
\DescribeMacro{\childdocjob}
The macro |\childdocname| contains the filename (without extension)
of the main or child file being processed.
Note that |\childdocjob| will always contain the name of the main file.

%%%%%%%%%%%%%%%%%%%%%%%%%%%%%%%%%%%%%%%%
\paragraph{Title Page.}

Conditional processing can be used to include a title or banner page
in the main document when proper precautions are taken.
Importantly, the code in the main file should ensure that the page counter
(as well as other status parameters which are stored in the |.aux| files)
takes the same value after the conditional processing.
Otherwise the page numbers may take divergent values
depending on which part is compiled.

For example, a title page could be declared by:
%
\begin{center}
\begin{tabular}{l}
|\ifchilddoc\||else|\\
|\addtocounter{page}{-1}|\\
\textit{code for title page}\\
|\newpage|\\
|\||fi|
\end{tabular}
\end{center}
%
A banner page for the child documents can be generated by:
%
\begin{center}
\begin{tabular}{l}
|\ifchilddoc|\\
|\addtocounter{page}{-1}|\\
\textit{code for banner page}\\
|\newpage|\\
|\||fi|
\end{tabular}
\end{center}
%
Here one could write a message such as:
\begin{center}
|This is the part \childdocname{} of \childdocjob{}.|
\end{center}

%%%%%%%%%%%%%%%%%%%%%%%%%%%%%%%%%%%%%%%%%%%%%%%%%%%%%%%%%%%%%%%%%%%%%%%%%%%%%%%%
\subsection{Flags}
\label{sec:flags}

The package makes it easy to generate different versions
of the main or child documents.
To this end compilation flags can be defined
and assigned different default values.
They will be particularly useful in conjunction
with the forwarding mechanism described in \secref{sec:forward}.

For example, it may be useful to have a flag |\version|
which can be set to |draft| or |final|.
The document source will contain some conditional code
depending on the value of |\version|.
Suppose further, the flag should default to |final| for the main file
and to |draft| for child files
which is a natural assignment for editing the document.
This is achieved by placing the following code
in the preamble of the main document
(below the |\childdocmain| directive):
%
\begin{center}
\begin{tabular}{l}
|\ifchilddoc|\\
|\providecommand{\version}{draft}|\\
|\||else|\\
|\providecommand{\version}{final}|\\
|\||fi|
\end{tabular}
\end{center}
%
The definition by |\providecommand| makes sure
that previous definitions are not overwritten.
Further statements |\providecommand{\version}{...}|
can thus be added before the above code to override it.

For the main file, one might add a line
(between |\childdocmain| and the above block)
%
\begin{center}
|%\ifchilddoc\||else\providecommand{\version}{draft}\||fi|
\end{center}
%
which can be uncommented to produce a draft version.
Likewise one can add a line to the very top of a child file
(above the |\childdocof{|\textit{main}|}| directive)
%
\begin{center}
|%\providecommand{\version}{final}|
\end{center}
%
which can be uncommented to produce the final version of this child document.

%%%%%%%%%%%%%%%%%%%%%%%%%%%%%%%%%%%%%%%%%%%%%%%%%%%%%%%%%%%%%%%%%%%%%%%%%%%%%%%%
\subsection{Forwarding}
\label{sec:forward}

Different versions of the main or child documents
using compilation flags as described in \secref{sec:flags}
can be (permanently) stored in different files
for convenient compilation, viewing and distribution.
To this end, the package defines a command
to pass on compilation to a different file:

%%%%%%%%%%%%%%%%%%%%%%%%%%%%%%%%%%%%%%%%
\DescribeMacro{\childdocforward}
The command |\childdocforward| redirects processing to
another source file:
%
\begin{center}
\begin{tabular}{l}
|\input{childdoc.def}|\\
|\childdocforward[|\textit{main}|]{|\textit{dest}|}|\\
\end{tabular}
\end{center}
%
The argument \textit{dest} is the destination file
(without extension).
It should be the main file or one of the child files.
Note that further \textsf{childdoc} directives
such as |\childdocof| and |\childdocforward|
in the indicated file will be processed in this form.
The optional argument \textit{main}
passes on directly to the main file \textit{main}
while pretending to compile the child \textit{dest}.
This form behaves as if \textit{dest}
issues |\childdocof{|\textit{main}|}| right away,
and no further \textsf{childdoc} directives will be processed.

%%%%%%%%%%%%%%%%%%%%%%%%%%%%%%%%%%%%%%%%
\DescribeMacro{\...prefix}
In the alternative form |\childdocforwardprefix|,
%
\begin{center}
\begin{tabular}{l}
|\input{childdoc.def}|\\
|\childdocforwardprefix[|\textit{main}|]{|\textit{prefix}|}{|\textit{dest}|}|
\end{tabular}
\end{center}
%
the destination file is determined by a pattern
depending on the current file:
To make this work, the current file must be called
`{\textit{prefix}\hspace{0.2em}\textit{suffix}}'
with \textit{prefix} matching precisely the argument.
Processing is then passed on to the file
`{\textit{dest}\hspace{0.2em}\textit{suffix}}'.
Surely, the same effect is achieved by
directly specifying the
argument `{\textit{dest}\hspace{0.2em}\textit{suffix}}'
in the first form.
However, that requires to set up a different file
for each child. With the alternative form of the command
all these files can have exactly the same content
which simplifies setting them up and maintaining them.

For example, the following file |draft.tex|
with a compilation flag |\version| as described in \secref{sec:flags}
compiles the main document as a draft:
%
\begin{center}
\begin{tabular}{l}
|\def\version{draft}|\\
|\input{childdoc.def}|\\
|\childdocforward{|\textit{main}|}|
\end{tabular}
\end{center}
%
Likewise, the following files |final|\textit{nn}|.tex|
compile the final version of the child document
|child|\textit{nn}|.tex|:
%
\begin{center}
\begin{tabular}{l}
|\def\version{final}|\\
|\input{childdoc.def}|\\
|\childdocforwardprefix{final}{child}|
\end{tabular}
\end{center}
%

Note that when several versions of a main file and/or of each child file
are to be generated, it may be convenient to set up a |Makefile| or
shell script to automatise the process.

%%%%%%%%%%%%%%%%%%%%%%%%%%%%%%%%%%%%%%%%%%%%%%%%%%%%%%%%%%%%%%%%%%%%%%%%%%%%%%%%
\subsection{Command Line Processing}
\label{sec:commandline}

The effect of redirection files can also be achieved by invoking
the \LaTeX{} compiler with a more elaborate command line.
Most conveniently this should be done as part
of a shell script or a |Makefile|.

When using \textsf{childdoc} in the main file, the following
command lines effectively perform a redirection
(note that depending on the shell being used,
backslashes may have to be doubled: `|\|' $\to$ `|\\|'):
%
\begin{center}
|... -jobname "|\textit{target}|" |\\|"|[\textit{flags}]%
|\input{childdoc.def}\childdocforward[|\textit{main}|]{|\textit{dest}|}"|
\end{center}
%
Here \textit{target} is the name of the output file,
\textit{main} is the name of the main file
and \textit{dest} is the name of the main or child file to be processed
(all filenames without extensions).
The optional argument \textit{main} can be omitted
if \textit{main} matches \textit{dest}.
Optionally, compilation \textit{flags} can be defined via |\def| commands.
This command line makes the \TeX{} engine believe
it is compiling the file \textit{target}
whose content is specified as the latter parameter.
The provided code then forwards the processing to
\textit{main} or \textit{dest} as described in \secref{sec:forward}.

%%%%%%%%%%%%%%%%%%%%%%%%%%%%%%%%%%%%%%%%%%%%%%%%%%%%%%%%%%%%%%%%%%%%%%%%%%%%%%%%
\subsection{Include by Input}
\label{sec:input}

Including child documents by |\include| has some restrictions by design.
Most notably, the content of a child document always occupies
its own set of pages; pages cannot be shared between child documents.
Usually, this behaviour makes perfect sense
because each child document contain an essential part of the document.
However, in some situations it may be desirable to compose
a document from a collection of parts
without having mandatory page breaks between then.
For this case, the package
provides a mechanism to include parts
by |\input| which can also be processed individually.
However, by construction this mechanism
requires manual handling of the content to be output.

%%%%%%%%%%%%%%%%%%%%%%%%%%%%%%%%%%%%%%%%
\DescribeMacro{\ifchilddocmanual}
The main file should be prepared as usual, see \secref{sec:include}.
However, the document body must make a distinction
between processing of an individual part and of the main document, e.g.:
%
\begin{center}
\begin{tabular}{l}
|\ifchilddocmanual|\\
|\input{\childdocname}|\\
|\||else|\\
\textit{document body with }|\input{|\textit{part}|}|\\
|\||fi|
\end{tabular}
\end{center}
%
The conditional |\ifchilddocmanual| is true whenever
a part to be included by |\input| is being compiled,
and the name of the part is stored in |\childdocname|.

%%%%%%%%%%%%%%%%%%%%%%%%%%%%%%%%%%%%%%%%
\DescribeMacro{\childdocby}
Each part to be included by |\input| should start with:
%
\begin{center}
\begin{tabular}{l}
|\input{childdoc.def}|\\
|\childdocby{|\textit{main}|}|\\
\end{tabular}
\end{center}
%
The directive |\childdocby| is similar to |\childdocof|
described in \secref{sec:include},
but the subsequent selection of content must be done manually.
To that end, both |\ifchilddoc| and |\ifchilddocmanual|
will be true upon processing of a part,
and the name of the part is stored in |\childdocname|.
Note that |\jobname| will be set to the filename of the current part
so that each part receives an individual |.aux| file
that does not interfere with the |.aux| file(s) of the main document.
This behaviour can be altered by the alternative form
|\childdocby[*]{|\textit{main}|}| (with a non-empty optional argument)
which uses the |.aux| file of the main document
by setting |\jobname| to \textit{main}.

%%%%%%%%%%%%%%%%%%%%%%%%%%%%%%%%%%%%%%%%%%%%%%%%%%%%%%%%%%%%%%%%%%%%%%%%%%%%%%%%
\subsection{Driver Development}
\label{sec:driver}

The \textsf{childdoc} mechanism can also be use for the development
of definition files such as \LaTeX{} styles or classes.
This case differs from the above setup with multiple parts
included by |\include| in that no |\includeonly| should be invoked.
This can be achieved by starting the include file
(before |\ProvidesPackage|) with:
%
\begin{center}
\begin{tabular}{l}
|\input{childdoc.def}|\\
|\childdocforward{|\textit{main}|}|\\
\end{tabular}
\end{center}
%
or alternatively with:
%
\begin{center}
\begin{tabular}{l}
|\input{childdoc.def}|\\
|\childdocby{|\textit{main}|}|\\
\end{tabular}
\end{center}
%
Both forms have slightly different effects as described above.
The main file is prepared as usual, see \secref{sec:include}.

%%%%%%%%%%%%%%%%%%%%%%%%%%%%%%%%%%%%%%%%%%%%%%%%%%%%%%%%%%%%%%%%%%%%%%%%%%%%%%%%
\subsection{Legacy Detection}
\label{sec:detection}

The directive |\childdocmain| in the main file can detect
whether the complete document or merely a child is to be compiled
even without using the directive |\childdocof|.
This method is deprecated because it is less robust
and there is no compelling reason to use it;
it is merely provided for backward compatibility
and it may be removed in future versions.

If the detection mechanism is to be used,
it is mandatory to correctly specify
the filename of the main file as the argument of |\childdocmain|:
%
\begin{center}
\begin{tabular}{l}
|\input{childdoc.def}|\\
|\childdocmain{|\textit{main}|}|\\
\end{tabular}
\end{center}
%
If |\jobname| does not match the argument \textit{main} of |\childdocmain|,
it is assumed that |\jobname| points to the child file to be compiled.
When using |\childdocmain| with the main file specified as argument,
it suffices to start a child file
with just |\input{|\textit{main}|}|
without loading of the package and using |\childdocof|.
If instead all processing is done
with the appropriate \textsf{childdoc} directives,
the argument of \textit{main} of |\childdocmain| can be empty.

An alternative version of the command line processing described
in \secref{sec:commandline} using the detection mechanism reads:
%
\begin{center}
|... -jobname "|\textit{target}|" "|[\textit{flags}]%
[|\def\jobname{|\textit{dest}|}|]|\input{|\textit{main}|}"|
\end{center}

%%%%%%%%%%%%%%%%%%%%%%%%%%%%%%%%%%%%%%%%%%%%%%%%%%%%%%%%%%%%%%%%%%%%%%%%%%%%%%%%
\subsection{Manual Code}
\label{sec:manual}

In case one cannot be certain whether the definitions file |childdoc.def|
is installed on the target \TeX{} distribution
and one prefers not to ship it,
it is conceivable to paste a few relevant commands into the sources.

To that end, drop all statements |\input{childdoc.def}|
and perform the replacements as outlined below.
Instead of |\childdocmain{|\textit{main}|}| add the following code
to the top of the main file:
%
\begin{center}
\begin{tabular}{l}
|\||ifdefined\childdocname\endinput\||fi\newif\ifchilddoc|\\
|\edef\childdocname{\scantokens\expandafter{\jobname\noexpand}}|\\
|\def\childdocmain{|\textit{main}|}\||ifx\childdocmain\childdocname\||else|\\
|\childdoctrue\includeonly{\childdocname}\let\jobname\childdocmain\||fi|\\
\end{tabular}
\end{center}
%
Instead of |\childdocof{|\textit{main}|}| just include the main file
at the top of each child file:
%
\begin{center}
|\input{|\textit{main}|}|
\end{center}
%
A simple redirection |\childdocforward{|\textit{dest}|}| is achieved by:
%
\begin{center}
|\def\jobname{|\textit{dest}|}\input{\jobname}|
\end{center}
%
The redirection with prefix
|\childdocforwardprefix[|\textit{prefix}|]{|\textit{dest}|}|
is accomplished by:
%
\begin{center}
\begin{tabular}{l}
|{\edef\jobname{\scantokens\expandafter{\jobname\noexpand}}|\\
|\def\redirectjob |\textit{prefix}|#1~~~{\gdef\jobname{|\textit{dest}|#1}}|\\
|\expandafter\redirectjob\jobname~~~}\input{\jobname}|
\end{tabular}
\end{center}

In an alternative approach,
child documents can be compiled by a specific command line
without additional code or specific definitions:
%
\begin{center}
|... -jobname "|\textit{target}|" "|[\textit{flags}]%
|\includeonly{|\textit{dest}|}\input{|\textit{main}|}"|
\end{center}
%

%%%%%%%%%%%%%%%%%%%%%%%%%%%%%%%%%%%%%%%%%%%%%%%%%%%%%%%%%%%%%%%%%%%%%%%%%%%%%%%%
%%%%%%%%%%%%%%%%%%%%%%%%%%%%%%%%%%%%%%%%%%%%%%%%%%%%%%%%%%%%%%%%%%%%%%%%%%%%%%%%
\section{Information}

%%%%%%%%%%%%%%%%%%%%%%%%%%%%%%%%%%%%%%%%%%%%%%%%%%%%%%%%%%%%%%%%%%%%%%%%%%%%%%%%
\subsection{Copyright}

Copyright \copyright{} 2017--2018 Niklas Beisert

This work may be distributed and/or modified under the
conditions of the \LaTeX{} Project Public License, either version 1.3
of this license or (at your option) any later version.
The latest version of this license is in
  \url{http://www.latex-project.org/lppl.txt}
and version 1.3 or later is part of all distributions of \LaTeX{}
version 2005/12/01 or later.

This work has the LPPL maintenance status `maintained'.

The Current Maintainer of this work is Niklas Beisert.

This work consists of the files |README.txt|, |childdoc.ins| and |childdoc.dtx|
as well as the derived files |childdoc.def|, |cdocsamp.tex|
with |cdocsch1.tex|, |cdocsch2.tex|, |cdocspt3.tex|, |cdocspt4.tex|,
|cdocsdrf.tex|, |cdocsfn1.tex|, |cdocsfn2.tex|
as well as |childdoc.pdf|.

%%%%%%%%%%%%%%%%%%%%%%%%%%%%%%%%%%%%%%%%%%%%%%%%%%%%%%%%%%%%%%%%%%%%%%%%%%%%%%%%
\subsection{Files and Installation}

The package consists of the files:
%
\begin{center}
\begin{tabular}{ll}
    |README.txt|   & readme file \\
    |childdoc.ins| & installation file \\
    |childdoc.dtx| & source file \\
    |childdoc.def| & definition file \\
    |cdocsamp.tex| & sample main file \\
    |cdocsch1.tex| & sample include file \\
    |cdocsch2.tex| & sample include file \\
    |cdocspt3.tex| & sample part file \\
    |cdocspt4.tex| & sample part file \\
    |cdocsdrf.tex| & sample redirection file \\
    |cdocsfn1.tex| & sample redirection file \\
    |cdocsfn2.tex| & sample redirection file \\
    |childdoc.pdf| & manual
\end{tabular}
\end{center}
%
The distribution consists of the files
|README.txt|, |childdoc.ins| and |childdoc.dtx|.
%
\begin{itemize}
\item
Run (pdf)\LaTeX{} on |childdoc.dtx|
to compile the manual |childdoc.pdf| (this file).
\item
Run \LaTeX{} on |childdoc.ins| to create the definitions file |childdoc.def|
and the sample |cdocsamp.tex| with include files
|cdocsch1.tex|, |cdocsch2.tex|, |cdocspt3.tex|, |cdocspt4.tex|,
|cdocsdrf.tex|, |cdocsfn1.tex|, |cdocsfn2.tex|.
Then copy the file |childdoc.def| to an appropriate directory of your \LaTeX{}
distribution, e.g.\ \textit{texmf-root}|/tex/latex/childdoc|.
\end{itemize}

%%%%%%%%%%%%%%%%%%%%%%%%%%%%%%%%%%%%%%%%%%%%%%%%%%%%%%%%%%%%%%%%%%%%%%%%%%%%%%%%
\subsection{Related CTAN Packages}

There are several other packages which offer a similar functionality:
%
\begin{itemize}
\item
The packages
\href{http://ctan.org/pkg/docmute}{\textsf{docmute}},
\href{http://ctan.org/pkg/includex}{\textsf{includex}} and
\href{http://ctan.org/pkg/standalone}{\textsf{standalone}}
provide commands to include only the document body of
a child file thus allowing both files to be compiled individually.
\item
The packages \href{http://ctan.org/pkg/subdocs}{\textsf{subdocs}}
and \href{http://ctan.org/pkg/subfiles}{\textsf{subfiles}}
provide structures in which the main and child documents can be
encapsulated and allowing them to be compiled individually.
The inclusion mechanism is different from the conventional |\include|.
\item
The package \href{http://ctan.org/pkg/combine}{\textsf{combine}}
is an elaborate solution to combine several documents into one.
\end{itemize}
%
See also the CTAN topic \href{http://ctan.org/topic/subdocs}{\textsf{subdocs}}
for further related packages.
The present package differs from the above solutions in that
a document structure constructed with the conventional |\include| mechanism
just needs two extra commands at the top of every file
such that all constituent files can be compiled individually.

%%%%%%%%%%%%%%%%%%%%%%%%%%%%%%%%%%%%%%%%%%%%%%%%%%%%%%%%%%%%%%%%%%%%%%%%%%%%%%%%
%\subsection{Feature Suggestions}
%
%The following is a list of features which may be useful for future
%versions of this package:
%%
%\begin{itemize}
%\item
%\ldots
%\end{itemize}

%%%%%%%%%%%%%%%%%%%%%%%%%%%%%%%%%%%%%%%%%%%%%%%%%%%%%%%%%%%%%%%%%%%%%%%%%%%%%%%%
\subsection{Revision History}

%%%%%%%%%%%%%%%%%%%%%%%%%%%%%%%%%%%%%%%%
\paragraph{v2.0:} 2018/12/30

\begin{itemize}
\item
immediate forward processing
\item
added |\childdocby| mechanism
\item
manual restructured
\end{itemize}

%%%%%%%%%%%%%%%%%%%%%%%%%%%%%%%%%%%%%%%%
\paragraph{v1.6:} 2018/01/17

\begin{itemize}
\item
application for development of include files
\item
corrections to manual
\end{itemize}

%%%%%%%%%%%%%%%%%%%%%%%%%%%%%%%%%%%%%%%%
\paragraph{v1.5:} 2017/05/21

\begin{itemize}
\item
more complete structuring introduced
\item
|\childdocof| introduced
\item
|\childdoc| renamed to |\childdocmain|
\item
|\childredirect| renamed to |\childdocforward| and |\childdocforwardprefix|
and functionality expanded
\end{itemize}

%%%%%%%%%%%%%%%%%%%%%%%%%%%%%%%%%%%%%%%%
\paragraph{v1.0:} 2017/04/27

\begin{itemize}
\item
manual and install package
\item
first version published on CTAN
\end{itemize}

%%%%%%%%%%%%%%%%%%%%%%%%%%%%%%%%%%%%%%%%
\paragraph{v0.6:} 2017/04/26

\begin{itemize}
\item
redirection mechanism added
\end{itemize}

%%%%%%%%%%%%%%%%%%%%%%%%%%%%%%%%%%%%%%%%
\paragraph{v0.5:} 2017/04/26

\begin{itemize}
\item
functionality in definition file
\end{itemize}


%%%%%%%%%%%%%%%%%%%%%%%%%%%%%%%%%%%%%%%%%%%%%%%%%%%%%%%%%%%%%%%%%%%%%%%%%%%%%%%%
%%%%%%%%%%%%%%%%%%%%%%%%%%%%%%%%%%%%%%%%%%%%%%%%%%%%%%%%%%%%%%%%%%%%%%%%%%%%%%%%
%%%%%%%%%%%%%%%%%%%%%%%%%%%%%%%%%%%%%%%%%%%%%%%%%%%%%%%%%%%%%%%%%%%%%%%%%%%%%%%%
\appendix

\settowidth\MacroIndent{\rmfamily\scriptsize 000\ }

 \DocInput{childdoc.dtx}

\end{document}
%</driver>
% \fi
%
% %%%%%%%%%%%%%%%%%%%%%%%%%%%%%%%%%%%%%%%%%%%%%%%%%%%%%%%%%%%%%%%%%%%%%%%%%%%%%%
% %%%%%%%%%%%%%%%%%%%%%%%%%%%%%%%%%%%%%%%%%%%%%%%%%%%%%%%%%%%%%%%%%%%%%%%%%%%%%%
% \section{Sample}
%\iffalse
%<*samplemain>
%\fi
%
% The following presents a sample document
% with two chapters, two parts, a title page,
% a compile flag as well as three forwarding files to set the flag.
% It consists of eight |.tex| files:
% \begin{center}
% \begin{tabular}{ll}
% |cdocsamp.tex|&main file\\
% |cdocsch1.tex|&include file for chapter 1\\
% |cdocsch2.tex|&include file for chapter 2\\
% |cdocspt3.tex|&include file for part 3\\
% |cdocspt4.tex|&include file for part 4\\
% |cdocsdrf.tex|&forwarding file for main file in draft mode\\
% |cdocsfi1.tex|&forwarding file for final version of chapter 1\\
% |cdocsfi2.tex|&forwarding file for final version of chapter 2\\
% \end{tabular}
% \end{center}
% Each of the eight files can be compiled directly by the \LaTeX{} compiler.
%
% %%%%%%%%%%%%%%%%%%%%%%%%%%%%%%%%%%%%%%
% \paragraph{Main File.}
%
% The main file is called |cdocsamp.tex|.
%
% Load the \textsf{childdoc} definitions and
% declare the filename for the main document:
%    \begin{macrocode}
\input{childdoc.def}
\childdocmain{}
%    \end{macrocode}

% Optional override for |\version| flag:
%    \begin{macrocode}
%%\ifchilddoc\else\providecommand{\version}{draft}\fi
%    \end{macrocode}

% Define the default values for the |\version| flag
% (|final| for the main file and |draft| for childs):
%    \begin{macrocode}
\ifchilddoc
\providecommand{\version}{draft}
\else
\providecommand{\version}{final}
\fi
%    \end{macrocode}

% Load the standard document class:
%    \begin{macrocode}
\documentclass[12pt]{article}
%    \end{macrocode}

% Start the document body:
%    \begin{macrocode}
\begin{document}
%    \end{macrocode}

% Declare a title page.
% Print title, part of document being processed and version flag:
%    \begin{macrocode}
\addtocounter{page}{-1}
\begin{center}
{\LARGE\bfseries{}childdoc example\par}
\vspace{1cm}
\ifchilddoc
\ifchilddocmanual part\else chapter\fi:
`\childdocname' of `\childdocjob'\par
\else
main document: `\childdocjob'\par
\fi
version: \version\par
\end{center}
\newpage
%    \end{macrocode}

% Manually include selected file,
% otherwise process as usual:
%    \begin{macrocode}
\ifchilddocmanual
\section*{part `\childdocname'}
\input{\childdocname}
\else
%    \end{macrocode}

% Include the two chapters:
%    \begin{macrocode}
\include{cdocsch1}
\include{cdocsch2}
%    \end{macrocode}

% Include the two parts unless only chapters should be displayed:
%    \begin{macrocode}
\ifchilddoc\else
\section{part three}
\input{cdocspt3}
\section{part four}
\input{cdocspt4}
\fi
%    \end{macrocode}

% Process as usual until here:
%    \begin{macrocode}
\fi
%    \end{macrocode}

% End of document body:
%    \begin{macrocode}
\end{document}
%    \end{macrocode}
%\iffalse
%</samplemain>
%\fi
%
% %%%%%%%%%%%%%%%%%%%%%%%%%%%%%%%%%%%%%%
% \paragraph{Chapter Include Files.}
%
% The include files are called |cdocsch1.tex| and |cdocsch2.tex|.
%
%\iffalse
%<*samplechap1|samplechap2>
%\fi

% Optional override for |\version| flag:
%    \begin{macrocode}
%%\providecommand{\version}{final}
%    \end{macrocode}

% Include the main document:
%    \begin{macrocode}
\input{childdoc.def}
\childdocof{cdocsamp}
%    \end{macrocode}

%\iffalse
%</samplechap1|samplechap2>
%\fi
%
%\iffalse
%<*samplechap1>
%\fi
% Some text for chapter 1:
%    \begin{macrocode}
\section{one}
some text in chapter one
%    \end{macrocode}

%\iffalse
%</samplechap1>
%\fi
% Some text for chapter 2:
%\iffalse
%<*samplechap2>
%\fi
%    \begin{macrocode}
\section{two}
more text in chapter two
%    \end{macrocode}

%\iffalse
%</samplechap2>
%\fi
%
% %%%%%%%%%%%%%%%%%%%%%%%%%%%%%%%%%%%%%%
% \paragraph{Part Include Files.}
%
% The include files are called |cdocspt3.tex| and |cdocspt4.tex|.
%
%\iffalse
%<*samplepart3|samplepart4>
%\fi

% Optional override for |\version| flag:
%    \begin{macrocode}
%%\providecommand{\version}{final}
%    \end{macrocode}

% Include the main document:
%    \begin{macrocode}
\input{childdoc.def}
\childdocby{cdocsamp}
%    \end{macrocode}

%\iffalse
%</samplepart3|samplepart4>
%\fi
%
%\iffalse
%<*samplepart3>
%\fi
% Some text for part 3:
%    \begin{macrocode}
some text in part three
%    \end{macrocode}

%\iffalse
%</samplepart3>
%\fi
% Some text for part 4:
%\iffalse
%<*samplepart4>
%\fi
%    \begin{macrocode}
more text in part four
%    \end{macrocode}

%\iffalse
%</samplepart4>
%\fi
%
% %%%%%%%%%%%%%%%%%%%%%%%%%%%%%%%%%%%%%%
% \paragraph{Forwarding for a Complete Draft.}
%
% The following forwarding file |cdocsdrf.tex|
% compiles the main document in draft mode:
%\iffalse
%<*sampledraft>
%\fi
%    \begin{macrocode}
\def\version{draft}
\input{childdoc.def}
\childdocforward{cdocsamp}
%    \end{macrocode}

%\iffalse
%</sampledraft>
%\fi
%
% %%%%%%%%%%%%%%%%%%%%%%%%%%%%%%%%%%%%%%
% \paragraph{Forwarding for Final Version of the Chapters.}
%
% The following forwarding files |cdocsfn1.tex| and |cdocsfn2.tex|
% (with identical content)
% compile the final versions of the child documents
% |cdocsch1.tex| and |cdocsch2.tex|, respectively:
%\iffalse
%<*samplefinal>
%\fi
%    \begin{macrocode}
\def\version{final}
\input{childdoc.def}
\childdocforwardprefix[cdocsamp]{cdocsfn}{cdocsch}
%    \end{macrocode}

%\iffalse
%</samplefinal>
%\fi
%
% %%%%%%%%%%%%%%%%%%%%%%%%%%%%%%%%%%%%%%
% \paragraph{Command Line Processing.}
%
% The following three command lines generate the output files
% |cdocscld|, |cdocscl1| and |cdocscl2|
% which should be identical to
% |cdocsdrf|, |cdocsch1| and |cdocsfn2|, respectively:
% \begin{center}
% \begin{tabular}{l}
% |latex -jobname cdocscld \|\\
% |  "\def\version{draft}\input{childdoc.def}\childdocforward{cdocsamp}"|\\
% |latex -jobname cdocscl1 \|\\
% |  "\input{childdoc.def}\childdocforward[cdocsamp]{cdocsch1}"|\\
% |latex -jobname cdocscl2 \|\\
% |  "\def\version{final}\input{childdoc.def}\childdocforward{cdocsch2}"|
% \end{tabular}
% \end{center}
% Note that the trailing backslash on each first line
% merely continues the input to the second line
% (for convenient cut ant paste).
% Furthermore, the command |latex| can be replaced by any
% of its alternative versions such as |pdflatex|.
%
% %%%%%%%%%%%%%%%%%%%%%%%%%%%%%%%%%%%%%%%%%%%%%%%%%%%%%%%%%%%%%%%%%%%%%%%%%%%%%%
% %%%%%%%%%%%%%%%%%%%%%%%%%%%%%%%%%%%%%%%%%%%%%%%%%%%%%%%%%%%%%%%%%%%%%%%%%%%%%%
% \section{Implementation}
%\iffalse
%<*package>
%\fi
%
% This section describes the definitions file |childdoc.def|.

% The definitions cannot be loaded using |\usepackage| or |\RequirePackage|
% which has a mechanism to prevent loading a style file more than once.
% When loading the definitions by means of |\input|
% multiple instances have to be prevented manually:
%\iffalse
%This code needs to be before the `\ProvidesFile' directive
%which is defined at the beginning of this file.
%Therefore it is also placed there and commented out here.
%</package>
%<*discard>
%\fi
%    \begin{macrocode}
\ifdefined\childdocmain\endinput\fi
%    \end{macrocode}
%\iffalse
%</discard>
%<*package>
%\fi
%
% \macro{\ifchilddoc}
% \macro{\ifchilddocmanual}
% The conditional |\ifchilddoc| tells whether a
% child (true) or main (false) document is being compiled.
% The conditional |\ifchilddocmanual| tells whether
% the |\includeonly| mechanism is used (false) or
% the selection of child files must be performed manually (true).
% The definitions initialise to false:
%    \begin{macrocode}
\newif\ifchilddoc
\newif\ifchilddocmanual
%    \end{macrocode}

% \macro{\childdocname}
% \macro{\childdocjob}
% The macro |\childdocname| stores the name of the main document
% to be compiled. The macro |\childdocjob| stores the name of
% the document on which the \LaTeX{} compiler was originally invoked.
% The content of |\jobname| cannot be compared
% to filenames specified in the source due to different catcodes.
% The following code rescans |\jobname|, stores the result
% in |\childdocname| and saves a copy in |\childdocjob|:
%    \begin{macrocode}
\edef\childdocname{\scantokens\expandafter{\jobname\noexpand}}
\let\childdocjob\childdocname
%    \end{macrocode}

% \macro{\childdocdisable}
% The macro |\childdocdisable| prevents the main file
% from being processed more than once.
% At this stage, the main document command |\childdocmain|
% is assumed to be called once again where it should do nothing.
% Any subsequent call to it should prevent
% a secondary processing of the main document
% It overwrites the forwarding commands
% |\childdocof| and |\childdocforward|
% with empty macros to prevent further inclusions of the main document:
%    \begin{macrocode}
\newcommand{\childdocdisable}
{
  \renewcommand{\childdocmain}[1]{\renewcommand{\childdocmain}[1]{\endinput}}
  \renewcommand{\childdocof}[1]{}
  \renewcommand{\childdocby}[2][]{}
  \renewcommand{\childdocforward}[2][]{}
  \renewcommand{\childdocdisable}{}
}
%    \end{macrocode}

% \macro{\childdocmain}
% The macro |\childdocmain| is to be called at the top of the main file
% with nothing or the main filename (without extension) as argument.
% First, it breaks loops.
% If the argument is not empty and does not match |\childdocname|
% (which is set by the first inclusion of |childdoc.def|),
% |\ifchilddoc| is set to true, |\includeonly| is applied to the child file
% and |\jobname| is set to the main file
% (for proper handling of |.aux| files):
%    \begin{macrocode}
\newcommand{\childdocmain}[1]
{
  \childdocdisable\childdocmain{}
  \if?#1?\else
    \begingroup
      \def\childdoctmp{#1}
      \ifx\childdoctmp\childdocname
        \def\childdoctmp{}
      \else
        \def\childdoctmp
        {
          \childdoctrue
          \includeonly{\childdocname}
          \def\childdocjob{#1}
          \def\jobname{#1}
        }
      \fi
      \expandafter
    \endgroup
    \childdoctmp
  \fi
}
%    \end{macrocode}

% \macro{\childdocof}
% The command |\childdocof| redirects
% compilation to the main file |#1|.
%    \begin{macrocode}
\newcommand{\childdocof}[1]
{
  \childdocdisable
  \childdoctrue
  \includeonly{\childdocname}
  \def\jobname{#1}
  \def\childdocjob{#1}
  \input{#1}
}
%    \end{macrocode}

% \macro{\childdocby}
% The command |\childdocby| ....
%    \begin{macrocode}
\newcommand{\childdocby}[2][]
{
  \childdocdisable
  \childdoctrue
  \childdocmanualtrue
  \if?#1?\else
    \def\jobname{#2}
  \fi
  \def\childdocjob{#2}
  \input{#2}
  \endinput
}
%    \end{macrocode}

% \macro{\childdocforward}
% The command |\childdocforward| redirects
% compilation to the main file or
% (if the optional argument is given) a child file.
% Parameters are set as if the main file
% or a child file starting with |\childdocof| was compiled.
% Then compilation is handed over to the main file:
%    \begin{macrocode}
\newcommand{\childdocforward}[2][]
{
  \begingroup
    \if?#1?
      \def\childdoctmp
      {
        \def\childdocname{#2}
        \def\childdocjob{#2}
        \def\jobname{#2}
        \input{#2}
        \endinput
      }
    \else
      \def\childdoctmp
      {
        \childdocdisable
        \def\childdocname{#2}
        \childdoctrue
        \includeonly{#2}
        \def\childdocjob{#1}
        \def\jobname{#1}
        \input{#1}
        \endinput
      }
    \fi
    \expandafter
  \endgroup
  \childdoctmp
}
%    \end{macrocode}

% \macro{\childdocforwardprefix}
% The command |\childdocforwardprefix| redirects
% compilation to the main or a child file by means of a pattern.
% The prefix |#1| in the current filename is replaced by |#2|
% and the suffix of the current filename is kept
% (it is assumed that the filename does not contain the substring `|~~~|'
% which is used as a delimiter).
% Compilation is handed over to the new file by |\childdocforward|:
%    \begin{macrocode}
\newcommand{\childdocforwardprefix}[3][]
{
  \begingroup
    \def\childdocextract #2##1~~~{\def\childdoctmp{\childdocforward[#1]{#3##1}}}
    \expandafter\childdocextract\childdocname~~~
    \expandafter
  \endgroup
  \childdoctmp
}
%    \end{macrocode}

% \macro{\childdoc}
% The deprecated macro |\childdoc| is a legacy version of |\childdocmain|:
%    \begin{macrocode}
\newcommand{\childdoc}{\childdocmain}
%    \end{macrocode}

% \macro{\childdocredirect}
% The deprecated macro |\childdocredirect| is a legacy version
% of |\childdocforward| and |\childdocforwardprefix|:
%    \begin{macrocode}
\newcommand{\childdocredirect}[2][]
{
  \begingroup
    \if?#1?
      \def\childdoctmp{\childdocforward{#2}}
    \else
      \def\childdoctmp{\childdocforwardprefix{#1}{#2}}
    \fi
    \expandafter
  \endgroup
  \childdoctmp
}
%    \end{macrocode}

%\iffalse
%</package>
%\fi
%
\endinput
|\\
|\childdocby{|\textit{main}|}|\\
\end{tabular}
\end{center}
%
The directive |\childdocby| is similar to |\childdocof|
described in \secref{sec:include},
but the subsequent selection of content must be done manually.
To that end, both |\ifchilddoc| and |\ifchilddocmanual|
will be true upon processing of a part,
and the name of the part is stored in |\childdocname|.
Note that |\jobname| will be set to the filename of the current part
so that each part receives an individual |.aux| file
that does not interfere with the |.aux| file(s) of the main document.
This behaviour can be altered by the alternative form
|\childdocby[*]{|\textit{main}|}| (with a non-empty optional argument)
which uses the |.aux| file of the main document
by setting |\jobname| to \textit{main}.

%%%%%%%%%%%%%%%%%%%%%%%%%%%%%%%%%%%%%%%%%%%%%%%%%%%%%%%%%%%%%%%%%%%%%%%%%%%%%%%%
\subsection{Driver Development}
\label{sec:driver}

The \textsf{childdoc} mechanism can also be use for the development
of definition files such as \LaTeX{} styles or classes.
This case differs from the above setup with multiple parts
included by |\include| in that no |\includeonly| should be invoked.
This can be achieved by starting the include file
(before |\ProvidesPackage|) with:
%
\begin{center}
\begin{tabular}{l}
|% \iffalse
%
% childdoc.dtx Copyright (C) 2017-2018 Niklas Beisert
%
% This work may be distributed and/or modified under the
% conditions of the LaTeX Project Public License, either version 1.3
% of this license or (at your option) any later version.
% The latest version of this license is in
%   http://www.latex-project.org/lppl.txt
% and version 1.3 or later is part of all distributions of LaTeX
% version 2005/12/01 or later.
%
% This work has the LPPL maintenance status `maintained'.
%
% The Current Maintainer of this work is Niklas Beisert.
%
% This work consists of the files childdoc.dtx and childdoc.ins
% and the derived files childdoc.def and cdocsamp.tex with
% cdocsch1.tex, cdocsch2.tex, cdocsdrf.tex, cdocsfn1.tex, cdocsfn2.tex.
%
%<package>\ifdefined\childdocmain\endinput\fi
%<package>\ProvidesFile{childdoc.def}[2018/12/30 v2.0 child document driver]
%<samplemain>\ProvidesFile{cdocsamp.tex}[2018/12/30 v2.0 sample for childdoc]
%<*driver>
%\ProvidesFile{childdoc.drv}[2018/12/30 v2.0 childdoc reference manual file]
\PassOptionsToClass{10pt,a4paper}{article}
\documentclass{ltxdoc}

\usepackage[margin=35mm]{geometry}
\usepackage{hyperref}
\usepackage{hyperxmp}
\usepackage[usenames]{color}

\hypersetup{colorlinks=true}
\hypersetup{pdfstartview=FitH}
\hypersetup{pdfpagemode=UseNone}
\hypersetup{pdfsource={}}
\hypersetup{pdflang={en-UK}}
\hypersetup{pdfcopyright={Copyright 2017-2018 Niklas Beisert.
  This work may be distributed and/or modified under the
  conditions of the LaTeX Project Public License, either version 1.3
  of this license or (at your option) any later version.}}
\hypersetup{pdflicenseurl={http://www.latex-project.org/lppl.txt}}
\hypersetup{pdfcontactaddress={ETH Zurich, ITP, HIT K,
  Wolfgang-Pauli-Strasse 27}}
\hypersetup{pdfcontactpostcode={8093}}
\hypersetup{pdfcontactcity={Zurich}}
\hypersetup{pdfcontactcountry={Switzerland}}
\hypersetup{pdfcontactemail={nbeisert@itp.phys.ethz.ch}}
\hypersetup{pdfcontacturl={http://people.phys.ethz.ch/\xmptilde nbeisert/}}

\newcommand{\secref}[1]{\hyperref[#1]{section \ref*{#1}}}

\parskip1ex
\parindent0pt
\let\olditemize\itemize
\def\itemize{\olditemize\parskip0pt}

\begin{document}

\title{The \textsf{childdoc} Package}
\hypersetup{pdftitle={The childdoc Package}}
\author{Niklas Beisert\\[2ex]
  Institut f\"ur Theoretische Physik\\
  Eidgen\"ossische Technische Hochschule Z\"urich\\
  Wolfgang-Pauli-Strasse 27, 8093 Z\"urich, Switzerland\\[1ex]
  \href{mailto:nbeisert@itp.phys.ethz.ch}
  {\texttt{nbeisert@itp.phys.ethz.ch}}}
\hypersetup{pdfauthor={Niklas Beisert}}
\hypersetup{pdfsubject={Manual for the LaTeX2e Package childdoc}}
\date{30 December 2018, \textsf{v2.0}}
\maketitle

\begin{abstract}\noindent
\textsf{childdoc} is a \LaTeXe{} package
that enables the direct compilation
of document sections included by |\include|
to individual files.
\end{abstract}

\begingroup
\parskip0ex
\tableofcontents
\endgroup

%%%%%%%%%%%%%%%%%%%%%%%%%%%%%%%%%%%%%%%%%%%%%%%%%%%%%%%%%%%%%%%%%%%%%%%%%%%%%%%%
%%%%%%%%%%%%%%%%%%%%%%%%%%%%%%%%%%%%%%%%%%%%%%%%%%%%%%%%%%%%%%%%%%%%%%%%%%%%%%%%
\section{Introduction}

\LaTeX{} provides a mechanism to structure a large document (such as a book)
into a main file and several child files (containing the chapters)
using the |\include| command.
This mechanism is beneficial for documents
which span hundreds of pages in order to
make the source file(s) more manageable.
Moreover, compilation can be restricted to
selected child files by means of the |\includeonly| command.
The latter feature can be used to reduce the compilation time while editing
(this was significantly more useful in the earlier days of \LaTeX{})
or to generate a smaller document which is easier to navigate.
Another application of |\includeonly| is to generate
documents consisting of selected parts of the complete document.

However, there are a few drawbacks of the plain |\include| mechanism:
\begin{itemize}
\item
The child files cannot be compiled on their own,
they can only be compiled via the main file.
A naive editing environment
(such as a text editor with an option
to have the current file processed by \LaTeX)
may require one to switch to the main file before compiling;
attempting to compile the child file produces errors.
\item
The main file must be modified (each time)
to adjust the |\includeonly| command
to the present needs. This easily leaves the main file in a messy state.
\item
The generated document will always carry the filename
of the main document. This is inconvenient if
several child files are to be compiled and
to be kept for distribution.
\end{itemize}

The present package provides a simple interface
to make child files individually compilable by \LaTeX{}.
Compiling a child file then has the same effect as compiling
the main file with an |\includeonly| command
to select the appropriate child.
Moreover the generated document will carry the name of the child
rather than the main file.
This resolves all three above issues.

This feature is meant to make the editing of books,
thesis documents and lecture notes somewhat more convenient.
However, the package can also be used efficiently for
composing a series of documents (such as exercise sheets)
which are typically distributed individually.
It then assists the author in generating the individual documents
(potentially in different versions)
as well as a document containing the collected series.
Another application is in developing style files
or other kinds of included material
where compilation of the style file could redirect
to a sample or test file.

%%%%%%%%%%%%%%%%%%%%%%%%%%%%%%%%%%%%%%%%%%%%%%%%%%%%%%%%%%%%%%%%%%%%%%%%%%%%%%%%
%%%%%%%%%%%%%%%%%%%%%%%%%%%%%%%%%%%%%%%%%%%%%%%%%%%%%%%%%%%%%%%%%%%%%%%%%%%%%%%%
\section{Usage}

First of all, the package \textsf{childdoc} is \emph{not} a standard
\LaTeXe{} |.sty| style file! Therefore it needs to be invoked in
a non-standard way.

%%%%%%%%%%%%%%%%%%%%%%%%%%%%%%%%%%%%%%%%%%%%%%%%%%%%%%%%%%%%%%%%%%%%%%%%%%%%%%%%
\subsection{Included Files}
\label{sec:include}

%%%%%%%%%%%%%%%%%%%%%%%%%%%%%%%%%%%%%%%%
\DescribeMacro{\childdocmain}
To use the package, add the commands
\begin{center}
\begin{tabular}{l}
|\input{childdoc.def}|\\
|\childdocmain{}|\\
\end{tabular}
\end{center}
at the very top of the main \LaTeX{} file,
in particular \emph{before} the |\documentclass| statement!
The argument of |\childdocmain| should be left empty
(but it must be present).

%%%%%%%%%%%%%%%%%%%%%%%%%%%%%%%%%%%%%%%%
\DescribeMacro{\childdocof}
Furthermore, add the commands
\begin{center}
\begin{tabular}{l}
|\input{childdoc.def}|\\
|\childdocof{|\textit{main}|}|\\
\end{tabular}
\end{center}
at the top of every child file \textit{child}
which is included by |\include{|\textit{child}|}|
from within the main file
(or at least for those files to be compiled individually).
The argument \textit{main} must be the filename of the main file.

There are a couple of
considerations in setting up the main and child documents:

%%%%%%%%%%%%%%%%%%%%%%%%%%%%%%%%%%%%%%%%
\paragraph{Restrictions.}

Please note the following restrictions:
\begin{itemize}
\item
|\childdocmain| must be called with one argument \textit{main}
to ensure compatibility with earlier version of the package.
It must either be empty (|\childdocmain{}|)
or precisely match the filename of the main file in which it is specified.
See \secref{sec:detection} for further information.
\item
The filename \textit{main} must be specified without the |.tex| extension.
\item
The filename \textit{main} is case sensitive
(even in case-insensitive file systems)
due to internal string comparison.
\item
The argument \textit{main} should be fully expanded, it cannot be a macro.
\item
Subdirectories and special characters should be avoided in filenames.
\item
The command |\childdocmain{|\textit{main}|}| must be followed by a whitespace.
It should not be followed immediately by another command
or by a comment mark `|%|'.
This is because the \TeX{} parser reads the token immediately following
the argument of |\childdocmain| and puts it
at the beginning of every child section;
however, a white\-space is ignored.
\end{itemize}

%%%%%%%%%%%%%%%%%%%%%%%%%%%%%%%%%%%%%%%%
\paragraph{Content of Main File.}

It is advisable to place all content in the child files included by |\include|.
Any output contained in the main file will appear in all child documents
unless suppressed manually;
it cannot be suppressed automatically by the |\includeonly| directive
and thus should normally be avoided.
A method to include some content in the main file
by means of conditional processing is described in \secref{sec:conditional}.

%%%%%%%%%%%%%%%%%%%%%%%%%%%%%%%%%%%%%%%%
\paragraph{Page Numbering.}

When only a part of the document is compiled,
the appropriate numbering of pages
(as well as other status parameters)
is determined from the |.aux| files.
The latter contain information from previous passes.
However this information needs to propagate through
all intermediate child documents.
Therefore the page numbering in child documents may well
be inconsistent until the complete document is compiled at least once.

A useful (if unconventional) way to always ensure a consistent
page numbering is to restart the numbering in each child document
and denote the pages by `\textit{child}|.|\textit{page}'
where \textit{child} represents the chapter/section number of the child file.
This can be achieved by the command
|\numberwithin{page}{|\textit{child}|}|
of the \textsf{amsmath} package
where \textit{child} can be |chapter| or |section|
depending on the chosen structuring.
Alternatively, one can modify the macro |\thepage| appropriately
and reset the counter |page| at the start of each child file.

%%%%%%%%%%%%%%%%%%%%%%%%%%%%%%%%%%%%%%%%%%%%%%%%%%%%%%%%%%%%%%%%%%%%%%%%%%%%%%%%
\subsection{Conditional Processing}
\label{sec:conditional}

The package provides a mechanism to compile different versions
of a document. To customise the versions further some conditional processing
can come in handy to distinguish which version is being compiled.
The package provides two macros to describe the compilation context:

%%%%%%%%%%%%%%%%%%%%%%%%%%%%%%%%%%%%%%%%
\DescribeMacro{\ifchilddoc}
The conditional |\ifchilddoc| distinguishes between the compilation of
child documents and the main document:
%
\begin{center}
|\ifchilddoc |\textit{child-code}| |[|\||else |\textit{main-code}]| \||fi|
\end{center}

%%%%%%%%%%%%%%%%%%%%%%%%%%%%%%%%%%%%%%%%
\DescribeMacro{\childdocname}
\DescribeMacro{\childdocjob}
The macro |\childdocname| contains the filename (without extension)
of the main or child file being processed.
Note that |\childdocjob| will always contain the name of the main file.

%%%%%%%%%%%%%%%%%%%%%%%%%%%%%%%%%%%%%%%%
\paragraph{Title Page.}

Conditional processing can be used to include a title or banner page
in the main document when proper precautions are taken.
Importantly, the code in the main file should ensure that the page counter
(as well as other status parameters which are stored in the |.aux| files)
takes the same value after the conditional processing.
Otherwise the page numbers may take divergent values
depending on which part is compiled.

For example, a title page could be declared by:
%
\begin{center}
\begin{tabular}{l}
|\ifchilddoc\||else|\\
|\addtocounter{page}{-1}|\\
\textit{code for title page}\\
|\newpage|\\
|\||fi|
\end{tabular}
\end{center}
%
A banner page for the child documents can be generated by:
%
\begin{center}
\begin{tabular}{l}
|\ifchilddoc|\\
|\addtocounter{page}{-1}|\\
\textit{code for banner page}\\
|\newpage|\\
|\||fi|
\end{tabular}
\end{center}
%
Here one could write a message such as:
\begin{center}
|This is the part \childdocname{} of \childdocjob{}.|
\end{center}

%%%%%%%%%%%%%%%%%%%%%%%%%%%%%%%%%%%%%%%%%%%%%%%%%%%%%%%%%%%%%%%%%%%%%%%%%%%%%%%%
\subsection{Flags}
\label{sec:flags}

The package makes it easy to generate different versions
of the main or child documents.
To this end compilation flags can be defined
and assigned different default values.
They will be particularly useful in conjunction
with the forwarding mechanism described in \secref{sec:forward}.

For example, it may be useful to have a flag |\version|
which can be set to |draft| or |final|.
The document source will contain some conditional code
depending on the value of |\version|.
Suppose further, the flag should default to |final| for the main file
and to |draft| for child files
which is a natural assignment for editing the document.
This is achieved by placing the following code
in the preamble of the main document
(below the |\childdocmain| directive):
%
\begin{center}
\begin{tabular}{l}
|\ifchilddoc|\\
|\providecommand{\version}{draft}|\\
|\||else|\\
|\providecommand{\version}{final}|\\
|\||fi|
\end{tabular}
\end{center}
%
The definition by |\providecommand| makes sure
that previous definitions are not overwritten.
Further statements |\providecommand{\version}{...}|
can thus be added before the above code to override it.

For the main file, one might add a line
(between |\childdocmain| and the above block)
%
\begin{center}
|%\ifchilddoc\||else\providecommand{\version}{draft}\||fi|
\end{center}
%
which can be uncommented to produce a draft version.
Likewise one can add a line to the very top of a child file
(above the |\childdocof{|\textit{main}|}| directive)
%
\begin{center}
|%\providecommand{\version}{final}|
\end{center}
%
which can be uncommented to produce the final version of this child document.

%%%%%%%%%%%%%%%%%%%%%%%%%%%%%%%%%%%%%%%%%%%%%%%%%%%%%%%%%%%%%%%%%%%%%%%%%%%%%%%%
\subsection{Forwarding}
\label{sec:forward}

Different versions of the main or child documents
using compilation flags as described in \secref{sec:flags}
can be (permanently) stored in different files
for convenient compilation, viewing and distribution.
To this end, the package defines a command
to pass on compilation to a different file:

%%%%%%%%%%%%%%%%%%%%%%%%%%%%%%%%%%%%%%%%
\DescribeMacro{\childdocforward}
The command |\childdocforward| redirects processing to
another source file:
%
\begin{center}
\begin{tabular}{l}
|\input{childdoc.def}|\\
|\childdocforward[|\textit{main}|]{|\textit{dest}|}|\\
\end{tabular}
\end{center}
%
The argument \textit{dest} is the destination file
(without extension).
It should be the main file or one of the child files.
Note that further \textsf{childdoc} directives
such as |\childdocof| and |\childdocforward|
in the indicated file will be processed in this form.
The optional argument \textit{main}
passes on directly to the main file \textit{main}
while pretending to compile the child \textit{dest}.
This form behaves as if \textit{dest}
issues |\childdocof{|\textit{main}|}| right away,
and no further \textsf{childdoc} directives will be processed.

%%%%%%%%%%%%%%%%%%%%%%%%%%%%%%%%%%%%%%%%
\DescribeMacro{\...prefix}
In the alternative form |\childdocforwardprefix|,
%
\begin{center}
\begin{tabular}{l}
|\input{childdoc.def}|\\
|\childdocforwardprefix[|\textit{main}|]{|\textit{prefix}|}{|\textit{dest}|}|
\end{tabular}
\end{center}
%
the destination file is determined by a pattern
depending on the current file:
To make this work, the current file must be called
`{\textit{prefix}\hspace{0.2em}\textit{suffix}}'
with \textit{prefix} matching precisely the argument.
Processing is then passed on to the file
`{\textit{dest}\hspace{0.2em}\textit{suffix}}'.
Surely, the same effect is achieved by
directly specifying the
argument `{\textit{dest}\hspace{0.2em}\textit{suffix}}'
in the first form.
However, that requires to set up a different file
for each child. With the alternative form of the command
all these files can have exactly the same content
which simplifies setting them up and maintaining them.

For example, the following file |draft.tex|
with a compilation flag |\version| as described in \secref{sec:flags}
compiles the main document as a draft:
%
\begin{center}
\begin{tabular}{l}
|\def\version{draft}|\\
|\input{childdoc.def}|\\
|\childdocforward{|\textit{main}|}|
\end{tabular}
\end{center}
%
Likewise, the following files |final|\textit{nn}|.tex|
compile the final version of the child document
|child|\textit{nn}|.tex|:
%
\begin{center}
\begin{tabular}{l}
|\def\version{final}|\\
|\input{childdoc.def}|\\
|\childdocforwardprefix{final}{child}|
\end{tabular}
\end{center}
%

Note that when several versions of a main file and/or of each child file
are to be generated, it may be convenient to set up a |Makefile| or
shell script to automatise the process.

%%%%%%%%%%%%%%%%%%%%%%%%%%%%%%%%%%%%%%%%%%%%%%%%%%%%%%%%%%%%%%%%%%%%%%%%%%%%%%%%
\subsection{Command Line Processing}
\label{sec:commandline}

The effect of redirection files can also be achieved by invoking
the \LaTeX{} compiler with a more elaborate command line.
Most conveniently this should be done as part
of a shell script or a |Makefile|.

When using \textsf{childdoc} in the main file, the following
command lines effectively perform a redirection
(note that depending on the shell being used,
backslashes may have to be doubled: `|\|' $\to$ `|\\|'):
%
\begin{center}
|... -jobname "|\textit{target}|" |\\|"|[\textit{flags}]%
|\input{childdoc.def}\childdocforward[|\textit{main}|]{|\textit{dest}|}"|
\end{center}
%
Here \textit{target} is the name of the output file,
\textit{main} is the name of the main file
and \textit{dest} is the name of the main or child file to be processed
(all filenames without extensions).
The optional argument \textit{main} can be omitted
if \textit{main} matches \textit{dest}.
Optionally, compilation \textit{flags} can be defined via |\def| commands.
This command line makes the \TeX{} engine believe
it is compiling the file \textit{target}
whose content is specified as the latter parameter.
The provided code then forwards the processing to
\textit{main} or \textit{dest} as described in \secref{sec:forward}.

%%%%%%%%%%%%%%%%%%%%%%%%%%%%%%%%%%%%%%%%%%%%%%%%%%%%%%%%%%%%%%%%%%%%%%%%%%%%%%%%
\subsection{Include by Input}
\label{sec:input}

Including child documents by |\include| has some restrictions by design.
Most notably, the content of a child document always occupies
its own set of pages; pages cannot be shared between child documents.
Usually, this behaviour makes perfect sense
because each child document contain an essential part of the document.
However, in some situations it may be desirable to compose
a document from a collection of parts
without having mandatory page breaks between then.
For this case, the package
provides a mechanism to include parts
by |\input| which can also be processed individually.
However, by construction this mechanism
requires manual handling of the content to be output.

%%%%%%%%%%%%%%%%%%%%%%%%%%%%%%%%%%%%%%%%
\DescribeMacro{\ifchilddocmanual}
The main file should be prepared as usual, see \secref{sec:include}.
However, the document body must make a distinction
between processing of an individual part and of the main document, e.g.:
%
\begin{center}
\begin{tabular}{l}
|\ifchilddocmanual|\\
|\input{\childdocname}|\\
|\||else|\\
\textit{document body with }|\input{|\textit{part}|}|\\
|\||fi|
\end{tabular}
\end{center}
%
The conditional |\ifchilddocmanual| is true whenever
a part to be included by |\input| is being compiled,
and the name of the part is stored in |\childdocname|.

%%%%%%%%%%%%%%%%%%%%%%%%%%%%%%%%%%%%%%%%
\DescribeMacro{\childdocby}
Each part to be included by |\input| should start with:
%
\begin{center}
\begin{tabular}{l}
|\input{childdoc.def}|\\
|\childdocby{|\textit{main}|}|\\
\end{tabular}
\end{center}
%
The directive |\childdocby| is similar to |\childdocof|
described in \secref{sec:include},
but the subsequent selection of content must be done manually.
To that end, both |\ifchilddoc| and |\ifchilddocmanual|
will be true upon processing of a part,
and the name of the part is stored in |\childdocname|.
Note that |\jobname| will be set to the filename of the current part
so that each part receives an individual |.aux| file
that does not interfere with the |.aux| file(s) of the main document.
This behaviour can be altered by the alternative form
|\childdocby[*]{|\textit{main}|}| (with a non-empty optional argument)
which uses the |.aux| file of the main document
by setting |\jobname| to \textit{main}.

%%%%%%%%%%%%%%%%%%%%%%%%%%%%%%%%%%%%%%%%%%%%%%%%%%%%%%%%%%%%%%%%%%%%%%%%%%%%%%%%
\subsection{Driver Development}
\label{sec:driver}

The \textsf{childdoc} mechanism can also be use for the development
of definition files such as \LaTeX{} styles or classes.
This case differs from the above setup with multiple parts
included by |\include| in that no |\includeonly| should be invoked.
This can be achieved by starting the include file
(before |\ProvidesPackage|) with:
%
\begin{center}
\begin{tabular}{l}
|\input{childdoc.def}|\\
|\childdocforward{|\textit{main}|}|\\
\end{tabular}
\end{center}
%
or alternatively with:
%
\begin{center}
\begin{tabular}{l}
|\input{childdoc.def}|\\
|\childdocby{|\textit{main}|}|\\
\end{tabular}
\end{center}
%
Both forms have slightly different effects as described above.
The main file is prepared as usual, see \secref{sec:include}.

%%%%%%%%%%%%%%%%%%%%%%%%%%%%%%%%%%%%%%%%%%%%%%%%%%%%%%%%%%%%%%%%%%%%%%%%%%%%%%%%
\subsection{Legacy Detection}
\label{sec:detection}

The directive |\childdocmain| in the main file can detect
whether the complete document or merely a child is to be compiled
even without using the directive |\childdocof|.
This method is deprecated because it is less robust
and there is no compelling reason to use it;
it is merely provided for backward compatibility
and it may be removed in future versions.

If the detection mechanism is to be used,
it is mandatory to correctly specify
the filename of the main file as the argument of |\childdocmain|:
%
\begin{center}
\begin{tabular}{l}
|\input{childdoc.def}|\\
|\childdocmain{|\textit{main}|}|\\
\end{tabular}
\end{center}
%
If |\jobname| does not match the argument \textit{main} of |\childdocmain|,
it is assumed that |\jobname| points to the child file to be compiled.
When using |\childdocmain| with the main file specified as argument,
it suffices to start a child file
with just |\input{|\textit{main}|}|
without loading of the package and using |\childdocof|.
If instead all processing is done
with the appropriate \textsf{childdoc} directives,
the argument of \textit{main} of |\childdocmain| can be empty.

An alternative version of the command line processing described
in \secref{sec:commandline} using the detection mechanism reads:
%
\begin{center}
|... -jobname "|\textit{target}|" "|[\textit{flags}]%
[|\def\jobname{|\textit{dest}|}|]|\input{|\textit{main}|}"|
\end{center}

%%%%%%%%%%%%%%%%%%%%%%%%%%%%%%%%%%%%%%%%%%%%%%%%%%%%%%%%%%%%%%%%%%%%%%%%%%%%%%%%
\subsection{Manual Code}
\label{sec:manual}

In case one cannot be certain whether the definitions file |childdoc.def|
is installed on the target \TeX{} distribution
and one prefers not to ship it,
it is conceivable to paste a few relevant commands into the sources.

To that end, drop all statements |\input{childdoc.def}|
and perform the replacements as outlined below.
Instead of |\childdocmain{|\textit{main}|}| add the following code
to the top of the main file:
%
\begin{center}
\begin{tabular}{l}
|\||ifdefined\childdocname\endinput\||fi\newif\ifchilddoc|\\
|\edef\childdocname{\scantokens\expandafter{\jobname\noexpand}}|\\
|\def\childdocmain{|\textit{main}|}\||ifx\childdocmain\childdocname\||else|\\
|\childdoctrue\includeonly{\childdocname}\let\jobname\childdocmain\||fi|\\
\end{tabular}
\end{center}
%
Instead of |\childdocof{|\textit{main}|}| just include the main file
at the top of each child file:
%
\begin{center}
|\input{|\textit{main}|}|
\end{center}
%
A simple redirection |\childdocforward{|\textit{dest}|}| is achieved by:
%
\begin{center}
|\def\jobname{|\textit{dest}|}\input{\jobname}|
\end{center}
%
The redirection with prefix
|\childdocforwardprefix[|\textit{prefix}|]{|\textit{dest}|}|
is accomplished by:
%
\begin{center}
\begin{tabular}{l}
|{\edef\jobname{\scantokens\expandafter{\jobname\noexpand}}|\\
|\def\redirectjob |\textit{prefix}|#1~~~{\gdef\jobname{|\textit{dest}|#1}}|\\
|\expandafter\redirectjob\jobname~~~}\input{\jobname}|
\end{tabular}
\end{center}

In an alternative approach,
child documents can be compiled by a specific command line
without additional code or specific definitions:
%
\begin{center}
|... -jobname "|\textit{target}|" "|[\textit{flags}]%
|\includeonly{|\textit{dest}|}\input{|\textit{main}|}"|
\end{center}
%

%%%%%%%%%%%%%%%%%%%%%%%%%%%%%%%%%%%%%%%%%%%%%%%%%%%%%%%%%%%%%%%%%%%%%%%%%%%%%%%%
%%%%%%%%%%%%%%%%%%%%%%%%%%%%%%%%%%%%%%%%%%%%%%%%%%%%%%%%%%%%%%%%%%%%%%%%%%%%%%%%
\section{Information}

%%%%%%%%%%%%%%%%%%%%%%%%%%%%%%%%%%%%%%%%%%%%%%%%%%%%%%%%%%%%%%%%%%%%%%%%%%%%%%%%
\subsection{Copyright}

Copyright \copyright{} 2017--2018 Niklas Beisert

This work may be distributed and/or modified under the
conditions of the \LaTeX{} Project Public License, either version 1.3
of this license or (at your option) any later version.
The latest version of this license is in
  \url{http://www.latex-project.org/lppl.txt}
and version 1.3 or later is part of all distributions of \LaTeX{}
version 2005/12/01 or later.

This work has the LPPL maintenance status `maintained'.

The Current Maintainer of this work is Niklas Beisert.

This work consists of the files |README.txt|, |childdoc.ins| and |childdoc.dtx|
as well as the derived files |childdoc.def|, |cdocsamp.tex|
with |cdocsch1.tex|, |cdocsch2.tex|, |cdocspt3.tex|, |cdocspt4.tex|,
|cdocsdrf.tex|, |cdocsfn1.tex|, |cdocsfn2.tex|
as well as |childdoc.pdf|.

%%%%%%%%%%%%%%%%%%%%%%%%%%%%%%%%%%%%%%%%%%%%%%%%%%%%%%%%%%%%%%%%%%%%%%%%%%%%%%%%
\subsection{Files and Installation}

The package consists of the files:
%
\begin{center}
\begin{tabular}{ll}
    |README.txt|   & readme file \\
    |childdoc.ins| & installation file \\
    |childdoc.dtx| & source file \\
    |childdoc.def| & definition file \\
    |cdocsamp.tex| & sample main file \\
    |cdocsch1.tex| & sample include file \\
    |cdocsch2.tex| & sample include file \\
    |cdocspt3.tex| & sample part file \\
    |cdocspt4.tex| & sample part file \\
    |cdocsdrf.tex| & sample redirection file \\
    |cdocsfn1.tex| & sample redirection file \\
    |cdocsfn2.tex| & sample redirection file \\
    |childdoc.pdf| & manual
\end{tabular}
\end{center}
%
The distribution consists of the files
|README.txt|, |childdoc.ins| and |childdoc.dtx|.
%
\begin{itemize}
\item
Run (pdf)\LaTeX{} on |childdoc.dtx|
to compile the manual |childdoc.pdf| (this file).
\item
Run \LaTeX{} on |childdoc.ins| to create the definitions file |childdoc.def|
and the sample |cdocsamp.tex| with include files
|cdocsch1.tex|, |cdocsch2.tex|, |cdocspt3.tex|, |cdocspt4.tex|,
|cdocsdrf.tex|, |cdocsfn1.tex|, |cdocsfn2.tex|.
Then copy the file |childdoc.def| to an appropriate directory of your \LaTeX{}
distribution, e.g.\ \textit{texmf-root}|/tex/latex/childdoc|.
\end{itemize}

%%%%%%%%%%%%%%%%%%%%%%%%%%%%%%%%%%%%%%%%%%%%%%%%%%%%%%%%%%%%%%%%%%%%%%%%%%%%%%%%
\subsection{Related CTAN Packages}

There are several other packages which offer a similar functionality:
%
\begin{itemize}
\item
The packages
\href{http://ctan.org/pkg/docmute}{\textsf{docmute}},
\href{http://ctan.org/pkg/includex}{\textsf{includex}} and
\href{http://ctan.org/pkg/standalone}{\textsf{standalone}}
provide commands to include only the document body of
a child file thus allowing both files to be compiled individually.
\item
The packages \href{http://ctan.org/pkg/subdocs}{\textsf{subdocs}}
and \href{http://ctan.org/pkg/subfiles}{\textsf{subfiles}}
provide structures in which the main and child documents can be
encapsulated and allowing them to be compiled individually.
The inclusion mechanism is different from the conventional |\include|.
\item
The package \href{http://ctan.org/pkg/combine}{\textsf{combine}}
is an elaborate solution to combine several documents into one.
\end{itemize}
%
See also the CTAN topic \href{http://ctan.org/topic/subdocs}{\textsf{subdocs}}
for further related packages.
The present package differs from the above solutions in that
a document structure constructed with the conventional |\include| mechanism
just needs two extra commands at the top of every file
such that all constituent files can be compiled individually.

%%%%%%%%%%%%%%%%%%%%%%%%%%%%%%%%%%%%%%%%%%%%%%%%%%%%%%%%%%%%%%%%%%%%%%%%%%%%%%%%
%\subsection{Feature Suggestions}
%
%The following is a list of features which may be useful for future
%versions of this package:
%%
%\begin{itemize}
%\item
%\ldots
%\end{itemize}

%%%%%%%%%%%%%%%%%%%%%%%%%%%%%%%%%%%%%%%%%%%%%%%%%%%%%%%%%%%%%%%%%%%%%%%%%%%%%%%%
\subsection{Revision History}

%%%%%%%%%%%%%%%%%%%%%%%%%%%%%%%%%%%%%%%%
\paragraph{v2.0:} 2018/12/30

\begin{itemize}
\item
immediate forward processing
\item
added |\childdocby| mechanism
\item
manual restructured
\end{itemize}

%%%%%%%%%%%%%%%%%%%%%%%%%%%%%%%%%%%%%%%%
\paragraph{v1.6:} 2018/01/17

\begin{itemize}
\item
application for development of include files
\item
corrections to manual
\end{itemize}

%%%%%%%%%%%%%%%%%%%%%%%%%%%%%%%%%%%%%%%%
\paragraph{v1.5:} 2017/05/21

\begin{itemize}
\item
more complete structuring introduced
\item
|\childdocof| introduced
\item
|\childdoc| renamed to |\childdocmain|
\item
|\childredirect| renamed to |\childdocforward| and |\childdocforwardprefix|
and functionality expanded
\end{itemize}

%%%%%%%%%%%%%%%%%%%%%%%%%%%%%%%%%%%%%%%%
\paragraph{v1.0:} 2017/04/27

\begin{itemize}
\item
manual and install package
\item
first version published on CTAN
\end{itemize}

%%%%%%%%%%%%%%%%%%%%%%%%%%%%%%%%%%%%%%%%
\paragraph{v0.6:} 2017/04/26

\begin{itemize}
\item
redirection mechanism added
\end{itemize}

%%%%%%%%%%%%%%%%%%%%%%%%%%%%%%%%%%%%%%%%
\paragraph{v0.5:} 2017/04/26

\begin{itemize}
\item
functionality in definition file
\end{itemize}


%%%%%%%%%%%%%%%%%%%%%%%%%%%%%%%%%%%%%%%%%%%%%%%%%%%%%%%%%%%%%%%%%%%%%%%%%%%%%%%%
%%%%%%%%%%%%%%%%%%%%%%%%%%%%%%%%%%%%%%%%%%%%%%%%%%%%%%%%%%%%%%%%%%%%%%%%%%%%%%%%
%%%%%%%%%%%%%%%%%%%%%%%%%%%%%%%%%%%%%%%%%%%%%%%%%%%%%%%%%%%%%%%%%%%%%%%%%%%%%%%%
\appendix

\settowidth\MacroIndent{\rmfamily\scriptsize 000\ }

 \DocInput{childdoc.dtx}

\end{document}
%</driver>
% \fi
%
% %%%%%%%%%%%%%%%%%%%%%%%%%%%%%%%%%%%%%%%%%%%%%%%%%%%%%%%%%%%%%%%%%%%%%%%%%%%%%%
% %%%%%%%%%%%%%%%%%%%%%%%%%%%%%%%%%%%%%%%%%%%%%%%%%%%%%%%%%%%%%%%%%%%%%%%%%%%%%%
% \section{Sample}
%\iffalse
%<*samplemain>
%\fi
%
% The following presents a sample document
% with two chapters, two parts, a title page,
% a compile flag as well as three forwarding files to set the flag.
% It consists of eight |.tex| files:
% \begin{center}
% \begin{tabular}{ll}
% |cdocsamp.tex|&main file\\
% |cdocsch1.tex|&include file for chapter 1\\
% |cdocsch2.tex|&include file for chapter 2\\
% |cdocspt3.tex|&include file for part 3\\
% |cdocspt4.tex|&include file for part 4\\
% |cdocsdrf.tex|&forwarding file for main file in draft mode\\
% |cdocsfi1.tex|&forwarding file for final version of chapter 1\\
% |cdocsfi2.tex|&forwarding file for final version of chapter 2\\
% \end{tabular}
% \end{center}
% Each of the eight files can be compiled directly by the \LaTeX{} compiler.
%
% %%%%%%%%%%%%%%%%%%%%%%%%%%%%%%%%%%%%%%
% \paragraph{Main File.}
%
% The main file is called |cdocsamp.tex|.
%
% Load the \textsf{childdoc} definitions and
% declare the filename for the main document:
%    \begin{macrocode}
\input{childdoc.def}
\childdocmain{}
%    \end{macrocode}

% Optional override for |\version| flag:
%    \begin{macrocode}
%%\ifchilddoc\else\providecommand{\version}{draft}\fi
%    \end{macrocode}

% Define the default values for the |\version| flag
% (|final| for the main file and |draft| for childs):
%    \begin{macrocode}
\ifchilddoc
\providecommand{\version}{draft}
\else
\providecommand{\version}{final}
\fi
%    \end{macrocode}

% Load the standard document class:
%    \begin{macrocode}
\documentclass[12pt]{article}
%    \end{macrocode}

% Start the document body:
%    \begin{macrocode}
\begin{document}
%    \end{macrocode}

% Declare a title page.
% Print title, part of document being processed and version flag:
%    \begin{macrocode}
\addtocounter{page}{-1}
\begin{center}
{\LARGE\bfseries{}childdoc example\par}
\vspace{1cm}
\ifchilddoc
\ifchilddocmanual part\else chapter\fi:
`\childdocname' of `\childdocjob'\par
\else
main document: `\childdocjob'\par
\fi
version: \version\par
\end{center}
\newpage
%    \end{macrocode}

% Manually include selected file,
% otherwise process as usual:
%    \begin{macrocode}
\ifchilddocmanual
\section*{part `\childdocname'}
\input{\childdocname}
\else
%    \end{macrocode}

% Include the two chapters:
%    \begin{macrocode}
\include{cdocsch1}
\include{cdocsch2}
%    \end{macrocode}

% Include the two parts unless only chapters should be displayed:
%    \begin{macrocode}
\ifchilddoc\else
\section{part three}
\input{cdocspt3}
\section{part four}
\input{cdocspt4}
\fi
%    \end{macrocode}

% Process as usual until here:
%    \begin{macrocode}
\fi
%    \end{macrocode}

% End of document body:
%    \begin{macrocode}
\end{document}
%    \end{macrocode}
%\iffalse
%</samplemain>
%\fi
%
% %%%%%%%%%%%%%%%%%%%%%%%%%%%%%%%%%%%%%%
% \paragraph{Chapter Include Files.}
%
% The include files are called |cdocsch1.tex| and |cdocsch2.tex|.
%
%\iffalse
%<*samplechap1|samplechap2>
%\fi

% Optional override for |\version| flag:
%    \begin{macrocode}
%%\providecommand{\version}{final}
%    \end{macrocode}

% Include the main document:
%    \begin{macrocode}
\input{childdoc.def}
\childdocof{cdocsamp}
%    \end{macrocode}

%\iffalse
%</samplechap1|samplechap2>
%\fi
%
%\iffalse
%<*samplechap1>
%\fi
% Some text for chapter 1:
%    \begin{macrocode}
\section{one}
some text in chapter one
%    \end{macrocode}

%\iffalse
%</samplechap1>
%\fi
% Some text for chapter 2:
%\iffalse
%<*samplechap2>
%\fi
%    \begin{macrocode}
\section{two}
more text in chapter two
%    \end{macrocode}

%\iffalse
%</samplechap2>
%\fi
%
% %%%%%%%%%%%%%%%%%%%%%%%%%%%%%%%%%%%%%%
% \paragraph{Part Include Files.}
%
% The include files are called |cdocspt3.tex| and |cdocspt4.tex|.
%
%\iffalse
%<*samplepart3|samplepart4>
%\fi

% Optional override for |\version| flag:
%    \begin{macrocode}
%%\providecommand{\version}{final}
%    \end{macrocode}

% Include the main document:
%    \begin{macrocode}
\input{childdoc.def}
\childdocby{cdocsamp}
%    \end{macrocode}

%\iffalse
%</samplepart3|samplepart4>
%\fi
%
%\iffalse
%<*samplepart3>
%\fi
% Some text for part 3:
%    \begin{macrocode}
some text in part three
%    \end{macrocode}

%\iffalse
%</samplepart3>
%\fi
% Some text for part 4:
%\iffalse
%<*samplepart4>
%\fi
%    \begin{macrocode}
more text in part four
%    \end{macrocode}

%\iffalse
%</samplepart4>
%\fi
%
% %%%%%%%%%%%%%%%%%%%%%%%%%%%%%%%%%%%%%%
% \paragraph{Forwarding for a Complete Draft.}
%
% The following forwarding file |cdocsdrf.tex|
% compiles the main document in draft mode:
%\iffalse
%<*sampledraft>
%\fi
%    \begin{macrocode}
\def\version{draft}
\input{childdoc.def}
\childdocforward{cdocsamp}
%    \end{macrocode}

%\iffalse
%</sampledraft>
%\fi
%
% %%%%%%%%%%%%%%%%%%%%%%%%%%%%%%%%%%%%%%
% \paragraph{Forwarding for Final Version of the Chapters.}
%
% The following forwarding files |cdocsfn1.tex| and |cdocsfn2.tex|
% (with identical content)
% compile the final versions of the child documents
% |cdocsch1.tex| and |cdocsch2.tex|, respectively:
%\iffalse
%<*samplefinal>
%\fi
%    \begin{macrocode}
\def\version{final}
\input{childdoc.def}
\childdocforwardprefix[cdocsamp]{cdocsfn}{cdocsch}
%    \end{macrocode}

%\iffalse
%</samplefinal>
%\fi
%
% %%%%%%%%%%%%%%%%%%%%%%%%%%%%%%%%%%%%%%
% \paragraph{Command Line Processing.}
%
% The following three command lines generate the output files
% |cdocscld|, |cdocscl1| and |cdocscl2|
% which should be identical to
% |cdocsdrf|, |cdocsch1| and |cdocsfn2|, respectively:
% \begin{center}
% \begin{tabular}{l}
% |latex -jobname cdocscld \|\\
% |  "\def\version{draft}\input{childdoc.def}\childdocforward{cdocsamp}"|\\
% |latex -jobname cdocscl1 \|\\
% |  "\input{childdoc.def}\childdocforward[cdocsamp]{cdocsch1}"|\\
% |latex -jobname cdocscl2 \|\\
% |  "\def\version{final}\input{childdoc.def}\childdocforward{cdocsch2}"|
% \end{tabular}
% \end{center}
% Note that the trailing backslash on each first line
% merely continues the input to the second line
% (for convenient cut ant paste).
% Furthermore, the command |latex| can be replaced by any
% of its alternative versions such as |pdflatex|.
%
% %%%%%%%%%%%%%%%%%%%%%%%%%%%%%%%%%%%%%%%%%%%%%%%%%%%%%%%%%%%%%%%%%%%%%%%%%%%%%%
% %%%%%%%%%%%%%%%%%%%%%%%%%%%%%%%%%%%%%%%%%%%%%%%%%%%%%%%%%%%%%%%%%%%%%%%%%%%%%%
% \section{Implementation}
%\iffalse
%<*package>
%\fi
%
% This section describes the definitions file |childdoc.def|.

% The definitions cannot be loaded using |\usepackage| or |\RequirePackage|
% which has a mechanism to prevent loading a style file more than once.
% When loading the definitions by means of |\input|
% multiple instances have to be prevented manually:
%\iffalse
%This code needs to be before the `\ProvidesFile' directive
%which is defined at the beginning of this file.
%Therefore it is also placed there and commented out here.
%</package>
%<*discard>
%\fi
%    \begin{macrocode}
\ifdefined\childdocmain\endinput\fi
%    \end{macrocode}
%\iffalse
%</discard>
%<*package>
%\fi
%
% \macro{\ifchilddoc}
% \macro{\ifchilddocmanual}
% The conditional |\ifchilddoc| tells whether a
% child (true) or main (false) document is being compiled.
% The conditional |\ifchilddocmanual| tells whether
% the |\includeonly| mechanism is used (false) or
% the selection of child files must be performed manually (true).
% The definitions initialise to false:
%    \begin{macrocode}
\newif\ifchilddoc
\newif\ifchilddocmanual
%    \end{macrocode}

% \macro{\childdocname}
% \macro{\childdocjob}
% The macro |\childdocname| stores the name of the main document
% to be compiled. The macro |\childdocjob| stores the name of
% the document on which the \LaTeX{} compiler was originally invoked.
% The content of |\jobname| cannot be compared
% to filenames specified in the source due to different catcodes.
% The following code rescans |\jobname|, stores the result
% in |\childdocname| and saves a copy in |\childdocjob|:
%    \begin{macrocode}
\edef\childdocname{\scantokens\expandafter{\jobname\noexpand}}
\let\childdocjob\childdocname
%    \end{macrocode}

% \macro{\childdocdisable}
% The macro |\childdocdisable| prevents the main file
% from being processed more than once.
% At this stage, the main document command |\childdocmain|
% is assumed to be called once again where it should do nothing.
% Any subsequent call to it should prevent
% a secondary processing of the main document
% It overwrites the forwarding commands
% |\childdocof| and |\childdocforward|
% with empty macros to prevent further inclusions of the main document:
%    \begin{macrocode}
\newcommand{\childdocdisable}
{
  \renewcommand{\childdocmain}[1]{\renewcommand{\childdocmain}[1]{\endinput}}
  \renewcommand{\childdocof}[1]{}
  \renewcommand{\childdocby}[2][]{}
  \renewcommand{\childdocforward}[2][]{}
  \renewcommand{\childdocdisable}{}
}
%    \end{macrocode}

% \macro{\childdocmain}
% The macro |\childdocmain| is to be called at the top of the main file
% with nothing or the main filename (without extension) as argument.
% First, it breaks loops.
% If the argument is not empty and does not match |\childdocname|
% (which is set by the first inclusion of |childdoc.def|),
% |\ifchilddoc| is set to true, |\includeonly| is applied to the child file
% and |\jobname| is set to the main file
% (for proper handling of |.aux| files):
%    \begin{macrocode}
\newcommand{\childdocmain}[1]
{
  \childdocdisable\childdocmain{}
  \if?#1?\else
    \begingroup
      \def\childdoctmp{#1}
      \ifx\childdoctmp\childdocname
        \def\childdoctmp{}
      \else
        \def\childdoctmp
        {
          \childdoctrue
          \includeonly{\childdocname}
          \def\childdocjob{#1}
          \def\jobname{#1}
        }
      \fi
      \expandafter
    \endgroup
    \childdoctmp
  \fi
}
%    \end{macrocode}

% \macro{\childdocof}
% The command |\childdocof| redirects
% compilation to the main file |#1|.
%    \begin{macrocode}
\newcommand{\childdocof}[1]
{
  \childdocdisable
  \childdoctrue
  \includeonly{\childdocname}
  \def\jobname{#1}
  \def\childdocjob{#1}
  \input{#1}
}
%    \end{macrocode}

% \macro{\childdocby}
% The command |\childdocby| ....
%    \begin{macrocode}
\newcommand{\childdocby}[2][]
{
  \childdocdisable
  \childdoctrue
  \childdocmanualtrue
  \if?#1?\else
    \def\jobname{#2}
  \fi
  \def\childdocjob{#2}
  \input{#2}
  \endinput
}
%    \end{macrocode}

% \macro{\childdocforward}
% The command |\childdocforward| redirects
% compilation to the main file or
% (if the optional argument is given) a child file.
% Parameters are set as if the main file
% or a child file starting with |\childdocof| was compiled.
% Then compilation is handed over to the main file:
%    \begin{macrocode}
\newcommand{\childdocforward}[2][]
{
  \begingroup
    \if?#1?
      \def\childdoctmp
      {
        \def\childdocname{#2}
        \def\childdocjob{#2}
        \def\jobname{#2}
        \input{#2}
        \endinput
      }
    \else
      \def\childdoctmp
      {
        \childdocdisable
        \def\childdocname{#2}
        \childdoctrue
        \includeonly{#2}
        \def\childdocjob{#1}
        \def\jobname{#1}
        \input{#1}
        \endinput
      }
    \fi
    \expandafter
  \endgroup
  \childdoctmp
}
%    \end{macrocode}

% \macro{\childdocforwardprefix}
% The command |\childdocforwardprefix| redirects
% compilation to the main or a child file by means of a pattern.
% The prefix |#1| in the current filename is replaced by |#2|
% and the suffix of the current filename is kept
% (it is assumed that the filename does not contain the substring `|~~~|'
% which is used as a delimiter).
% Compilation is handed over to the new file by |\childdocforward|:
%    \begin{macrocode}
\newcommand{\childdocforwardprefix}[3][]
{
  \begingroup
    \def\childdocextract #2##1~~~{\def\childdoctmp{\childdocforward[#1]{#3##1}}}
    \expandafter\childdocextract\childdocname~~~
    \expandafter
  \endgroup
  \childdoctmp
}
%    \end{macrocode}

% \macro{\childdoc}
% The deprecated macro |\childdoc| is a legacy version of |\childdocmain|:
%    \begin{macrocode}
\newcommand{\childdoc}{\childdocmain}
%    \end{macrocode}

% \macro{\childdocredirect}
% The deprecated macro |\childdocredirect| is a legacy version
% of |\childdocforward| and |\childdocforwardprefix|:
%    \begin{macrocode}
\newcommand{\childdocredirect}[2][]
{
  \begingroup
    \if?#1?
      \def\childdoctmp{\childdocforward{#2}}
    \else
      \def\childdoctmp{\childdocforwardprefix{#1}{#2}}
    \fi
    \expandafter
  \endgroup
  \childdoctmp
}
%    \end{macrocode}

%\iffalse
%</package>
%\fi
%
\endinput
|\\
|\childdocforward{|\textit{main}|}|\\
\end{tabular}
\end{center}
%
or alternatively with:
%
\begin{center}
\begin{tabular}{l}
|% \iffalse
%
% childdoc.dtx Copyright (C) 2017-2018 Niklas Beisert
%
% This work may be distributed and/or modified under the
% conditions of the LaTeX Project Public License, either version 1.3
% of this license or (at your option) any later version.
% The latest version of this license is in
%   http://www.latex-project.org/lppl.txt
% and version 1.3 or later is part of all distributions of LaTeX
% version 2005/12/01 or later.
%
% This work has the LPPL maintenance status `maintained'.
%
% The Current Maintainer of this work is Niklas Beisert.
%
% This work consists of the files childdoc.dtx and childdoc.ins
% and the derived files childdoc.def and cdocsamp.tex with
% cdocsch1.tex, cdocsch2.tex, cdocsdrf.tex, cdocsfn1.tex, cdocsfn2.tex.
%
%<package>\ifdefined\childdocmain\endinput\fi
%<package>\ProvidesFile{childdoc.def}[2018/12/30 v2.0 child document driver]
%<samplemain>\ProvidesFile{cdocsamp.tex}[2018/12/30 v2.0 sample for childdoc]
%<*driver>
%\ProvidesFile{childdoc.drv}[2018/12/30 v2.0 childdoc reference manual file]
\PassOptionsToClass{10pt,a4paper}{article}
\documentclass{ltxdoc}

\usepackage[margin=35mm]{geometry}
\usepackage{hyperref}
\usepackage{hyperxmp}
\usepackage[usenames]{color}

\hypersetup{colorlinks=true}
\hypersetup{pdfstartview=FitH}
\hypersetup{pdfpagemode=UseNone}
\hypersetup{pdfsource={}}
\hypersetup{pdflang={en-UK}}
\hypersetup{pdfcopyright={Copyright 2017-2018 Niklas Beisert.
  This work may be distributed and/or modified under the
  conditions of the LaTeX Project Public License, either version 1.3
  of this license or (at your option) any later version.}}
\hypersetup{pdflicenseurl={http://www.latex-project.org/lppl.txt}}
\hypersetup{pdfcontactaddress={ETH Zurich, ITP, HIT K,
  Wolfgang-Pauli-Strasse 27}}
\hypersetup{pdfcontactpostcode={8093}}
\hypersetup{pdfcontactcity={Zurich}}
\hypersetup{pdfcontactcountry={Switzerland}}
\hypersetup{pdfcontactemail={nbeisert@itp.phys.ethz.ch}}
\hypersetup{pdfcontacturl={http://people.phys.ethz.ch/\xmptilde nbeisert/}}

\newcommand{\secref}[1]{\hyperref[#1]{section \ref*{#1}}}

\parskip1ex
\parindent0pt
\let\olditemize\itemize
\def\itemize{\olditemize\parskip0pt}

\begin{document}

\title{The \textsf{childdoc} Package}
\hypersetup{pdftitle={The childdoc Package}}
\author{Niklas Beisert\\[2ex]
  Institut f\"ur Theoretische Physik\\
  Eidgen\"ossische Technische Hochschule Z\"urich\\
  Wolfgang-Pauli-Strasse 27, 8093 Z\"urich, Switzerland\\[1ex]
  \href{mailto:nbeisert@itp.phys.ethz.ch}
  {\texttt{nbeisert@itp.phys.ethz.ch}}}
\hypersetup{pdfauthor={Niklas Beisert}}
\hypersetup{pdfsubject={Manual for the LaTeX2e Package childdoc}}
\date{30 December 2018, \textsf{v2.0}}
\maketitle

\begin{abstract}\noindent
\textsf{childdoc} is a \LaTeXe{} package
that enables the direct compilation
of document sections included by |\include|
to individual files.
\end{abstract}

\begingroup
\parskip0ex
\tableofcontents
\endgroup

%%%%%%%%%%%%%%%%%%%%%%%%%%%%%%%%%%%%%%%%%%%%%%%%%%%%%%%%%%%%%%%%%%%%%%%%%%%%%%%%
%%%%%%%%%%%%%%%%%%%%%%%%%%%%%%%%%%%%%%%%%%%%%%%%%%%%%%%%%%%%%%%%%%%%%%%%%%%%%%%%
\section{Introduction}

\LaTeX{} provides a mechanism to structure a large document (such as a book)
into a main file and several child files (containing the chapters)
using the |\include| command.
This mechanism is beneficial for documents
which span hundreds of pages in order to
make the source file(s) more manageable.
Moreover, compilation can be restricted to
selected child files by means of the |\includeonly| command.
The latter feature can be used to reduce the compilation time while editing
(this was significantly more useful in the earlier days of \LaTeX{})
or to generate a smaller document which is easier to navigate.
Another application of |\includeonly| is to generate
documents consisting of selected parts of the complete document.

However, there are a few drawbacks of the plain |\include| mechanism:
\begin{itemize}
\item
The child files cannot be compiled on their own,
they can only be compiled via the main file.
A naive editing environment
(such as a text editor with an option
to have the current file processed by \LaTeX)
may require one to switch to the main file before compiling;
attempting to compile the child file produces errors.
\item
The main file must be modified (each time)
to adjust the |\includeonly| command
to the present needs. This easily leaves the main file in a messy state.
\item
The generated document will always carry the filename
of the main document. This is inconvenient if
several child files are to be compiled and
to be kept for distribution.
\end{itemize}

The present package provides a simple interface
to make child files individually compilable by \LaTeX{}.
Compiling a child file then has the same effect as compiling
the main file with an |\includeonly| command
to select the appropriate child.
Moreover the generated document will carry the name of the child
rather than the main file.
This resolves all three above issues.

This feature is meant to make the editing of books,
thesis documents and lecture notes somewhat more convenient.
However, the package can also be used efficiently for
composing a series of documents (such as exercise sheets)
which are typically distributed individually.
It then assists the author in generating the individual documents
(potentially in different versions)
as well as a document containing the collected series.
Another application is in developing style files
or other kinds of included material
where compilation of the style file could redirect
to a sample or test file.

%%%%%%%%%%%%%%%%%%%%%%%%%%%%%%%%%%%%%%%%%%%%%%%%%%%%%%%%%%%%%%%%%%%%%%%%%%%%%%%%
%%%%%%%%%%%%%%%%%%%%%%%%%%%%%%%%%%%%%%%%%%%%%%%%%%%%%%%%%%%%%%%%%%%%%%%%%%%%%%%%
\section{Usage}

First of all, the package \textsf{childdoc} is \emph{not} a standard
\LaTeXe{} |.sty| style file! Therefore it needs to be invoked in
a non-standard way.

%%%%%%%%%%%%%%%%%%%%%%%%%%%%%%%%%%%%%%%%%%%%%%%%%%%%%%%%%%%%%%%%%%%%%%%%%%%%%%%%
\subsection{Included Files}
\label{sec:include}

%%%%%%%%%%%%%%%%%%%%%%%%%%%%%%%%%%%%%%%%
\DescribeMacro{\childdocmain}
To use the package, add the commands
\begin{center}
\begin{tabular}{l}
|\input{childdoc.def}|\\
|\childdocmain{}|\\
\end{tabular}
\end{center}
at the very top of the main \LaTeX{} file,
in particular \emph{before} the |\documentclass| statement!
The argument of |\childdocmain| should be left empty
(but it must be present).

%%%%%%%%%%%%%%%%%%%%%%%%%%%%%%%%%%%%%%%%
\DescribeMacro{\childdocof}
Furthermore, add the commands
\begin{center}
\begin{tabular}{l}
|\input{childdoc.def}|\\
|\childdocof{|\textit{main}|}|\\
\end{tabular}
\end{center}
at the top of every child file \textit{child}
which is included by |\include{|\textit{child}|}|
from within the main file
(or at least for those files to be compiled individually).
The argument \textit{main} must be the filename of the main file.

There are a couple of
considerations in setting up the main and child documents:

%%%%%%%%%%%%%%%%%%%%%%%%%%%%%%%%%%%%%%%%
\paragraph{Restrictions.}

Please note the following restrictions:
\begin{itemize}
\item
|\childdocmain| must be called with one argument \textit{main}
to ensure compatibility with earlier version of the package.
It must either be empty (|\childdocmain{}|)
or precisely match the filename of the main file in which it is specified.
See \secref{sec:detection} for further information.
\item
The filename \textit{main} must be specified without the |.tex| extension.
\item
The filename \textit{main} is case sensitive
(even in case-insensitive file systems)
due to internal string comparison.
\item
The argument \textit{main} should be fully expanded, it cannot be a macro.
\item
Subdirectories and special characters should be avoided in filenames.
\item
The command |\childdocmain{|\textit{main}|}| must be followed by a whitespace.
It should not be followed immediately by another command
or by a comment mark `|%|'.
This is because the \TeX{} parser reads the token immediately following
the argument of |\childdocmain| and puts it
at the beginning of every child section;
however, a white\-space is ignored.
\end{itemize}

%%%%%%%%%%%%%%%%%%%%%%%%%%%%%%%%%%%%%%%%
\paragraph{Content of Main File.}

It is advisable to place all content in the child files included by |\include|.
Any output contained in the main file will appear in all child documents
unless suppressed manually;
it cannot be suppressed automatically by the |\includeonly| directive
and thus should normally be avoided.
A method to include some content in the main file
by means of conditional processing is described in \secref{sec:conditional}.

%%%%%%%%%%%%%%%%%%%%%%%%%%%%%%%%%%%%%%%%
\paragraph{Page Numbering.}

When only a part of the document is compiled,
the appropriate numbering of pages
(as well as other status parameters)
is determined from the |.aux| files.
The latter contain information from previous passes.
However this information needs to propagate through
all intermediate child documents.
Therefore the page numbering in child documents may well
be inconsistent until the complete document is compiled at least once.

A useful (if unconventional) way to always ensure a consistent
page numbering is to restart the numbering in each child document
and denote the pages by `\textit{child}|.|\textit{page}'
where \textit{child} represents the chapter/section number of the child file.
This can be achieved by the command
|\numberwithin{page}{|\textit{child}|}|
of the \textsf{amsmath} package
where \textit{child} can be |chapter| or |section|
depending on the chosen structuring.
Alternatively, one can modify the macro |\thepage| appropriately
and reset the counter |page| at the start of each child file.

%%%%%%%%%%%%%%%%%%%%%%%%%%%%%%%%%%%%%%%%%%%%%%%%%%%%%%%%%%%%%%%%%%%%%%%%%%%%%%%%
\subsection{Conditional Processing}
\label{sec:conditional}

The package provides a mechanism to compile different versions
of a document. To customise the versions further some conditional processing
can come in handy to distinguish which version is being compiled.
The package provides two macros to describe the compilation context:

%%%%%%%%%%%%%%%%%%%%%%%%%%%%%%%%%%%%%%%%
\DescribeMacro{\ifchilddoc}
The conditional |\ifchilddoc| distinguishes between the compilation of
child documents and the main document:
%
\begin{center}
|\ifchilddoc |\textit{child-code}| |[|\||else |\textit{main-code}]| \||fi|
\end{center}

%%%%%%%%%%%%%%%%%%%%%%%%%%%%%%%%%%%%%%%%
\DescribeMacro{\childdocname}
\DescribeMacro{\childdocjob}
The macro |\childdocname| contains the filename (without extension)
of the main or child file being processed.
Note that |\childdocjob| will always contain the name of the main file.

%%%%%%%%%%%%%%%%%%%%%%%%%%%%%%%%%%%%%%%%
\paragraph{Title Page.}

Conditional processing can be used to include a title or banner page
in the main document when proper precautions are taken.
Importantly, the code in the main file should ensure that the page counter
(as well as other status parameters which are stored in the |.aux| files)
takes the same value after the conditional processing.
Otherwise the page numbers may take divergent values
depending on which part is compiled.

For example, a title page could be declared by:
%
\begin{center}
\begin{tabular}{l}
|\ifchilddoc\||else|\\
|\addtocounter{page}{-1}|\\
\textit{code for title page}\\
|\newpage|\\
|\||fi|
\end{tabular}
\end{center}
%
A banner page for the child documents can be generated by:
%
\begin{center}
\begin{tabular}{l}
|\ifchilddoc|\\
|\addtocounter{page}{-1}|\\
\textit{code for banner page}\\
|\newpage|\\
|\||fi|
\end{tabular}
\end{center}
%
Here one could write a message such as:
\begin{center}
|This is the part \childdocname{} of \childdocjob{}.|
\end{center}

%%%%%%%%%%%%%%%%%%%%%%%%%%%%%%%%%%%%%%%%%%%%%%%%%%%%%%%%%%%%%%%%%%%%%%%%%%%%%%%%
\subsection{Flags}
\label{sec:flags}

The package makes it easy to generate different versions
of the main or child documents.
To this end compilation flags can be defined
and assigned different default values.
They will be particularly useful in conjunction
with the forwarding mechanism described in \secref{sec:forward}.

For example, it may be useful to have a flag |\version|
which can be set to |draft| or |final|.
The document source will contain some conditional code
depending on the value of |\version|.
Suppose further, the flag should default to |final| for the main file
and to |draft| for child files
which is a natural assignment for editing the document.
This is achieved by placing the following code
in the preamble of the main document
(below the |\childdocmain| directive):
%
\begin{center}
\begin{tabular}{l}
|\ifchilddoc|\\
|\providecommand{\version}{draft}|\\
|\||else|\\
|\providecommand{\version}{final}|\\
|\||fi|
\end{tabular}
\end{center}
%
The definition by |\providecommand| makes sure
that previous definitions are not overwritten.
Further statements |\providecommand{\version}{...}|
can thus be added before the above code to override it.

For the main file, one might add a line
(between |\childdocmain| and the above block)
%
\begin{center}
|%\ifchilddoc\||else\providecommand{\version}{draft}\||fi|
\end{center}
%
which can be uncommented to produce a draft version.
Likewise one can add a line to the very top of a child file
(above the |\childdocof{|\textit{main}|}| directive)
%
\begin{center}
|%\providecommand{\version}{final}|
\end{center}
%
which can be uncommented to produce the final version of this child document.

%%%%%%%%%%%%%%%%%%%%%%%%%%%%%%%%%%%%%%%%%%%%%%%%%%%%%%%%%%%%%%%%%%%%%%%%%%%%%%%%
\subsection{Forwarding}
\label{sec:forward}

Different versions of the main or child documents
using compilation flags as described in \secref{sec:flags}
can be (permanently) stored in different files
for convenient compilation, viewing and distribution.
To this end, the package defines a command
to pass on compilation to a different file:

%%%%%%%%%%%%%%%%%%%%%%%%%%%%%%%%%%%%%%%%
\DescribeMacro{\childdocforward}
The command |\childdocforward| redirects processing to
another source file:
%
\begin{center}
\begin{tabular}{l}
|\input{childdoc.def}|\\
|\childdocforward[|\textit{main}|]{|\textit{dest}|}|\\
\end{tabular}
\end{center}
%
The argument \textit{dest} is the destination file
(without extension).
It should be the main file or one of the child files.
Note that further \textsf{childdoc} directives
such as |\childdocof| and |\childdocforward|
in the indicated file will be processed in this form.
The optional argument \textit{main}
passes on directly to the main file \textit{main}
while pretending to compile the child \textit{dest}.
This form behaves as if \textit{dest}
issues |\childdocof{|\textit{main}|}| right away,
and no further \textsf{childdoc} directives will be processed.

%%%%%%%%%%%%%%%%%%%%%%%%%%%%%%%%%%%%%%%%
\DescribeMacro{\...prefix}
In the alternative form |\childdocforwardprefix|,
%
\begin{center}
\begin{tabular}{l}
|\input{childdoc.def}|\\
|\childdocforwardprefix[|\textit{main}|]{|\textit{prefix}|}{|\textit{dest}|}|
\end{tabular}
\end{center}
%
the destination file is determined by a pattern
depending on the current file:
To make this work, the current file must be called
`{\textit{prefix}\hspace{0.2em}\textit{suffix}}'
with \textit{prefix} matching precisely the argument.
Processing is then passed on to the file
`{\textit{dest}\hspace{0.2em}\textit{suffix}}'.
Surely, the same effect is achieved by
directly specifying the
argument `{\textit{dest}\hspace{0.2em}\textit{suffix}}'
in the first form.
However, that requires to set up a different file
for each child. With the alternative form of the command
all these files can have exactly the same content
which simplifies setting them up and maintaining them.

For example, the following file |draft.tex|
with a compilation flag |\version| as described in \secref{sec:flags}
compiles the main document as a draft:
%
\begin{center}
\begin{tabular}{l}
|\def\version{draft}|\\
|\input{childdoc.def}|\\
|\childdocforward{|\textit{main}|}|
\end{tabular}
\end{center}
%
Likewise, the following files |final|\textit{nn}|.tex|
compile the final version of the child document
|child|\textit{nn}|.tex|:
%
\begin{center}
\begin{tabular}{l}
|\def\version{final}|\\
|\input{childdoc.def}|\\
|\childdocforwardprefix{final}{child}|
\end{tabular}
\end{center}
%

Note that when several versions of a main file and/or of each child file
are to be generated, it may be convenient to set up a |Makefile| or
shell script to automatise the process.

%%%%%%%%%%%%%%%%%%%%%%%%%%%%%%%%%%%%%%%%%%%%%%%%%%%%%%%%%%%%%%%%%%%%%%%%%%%%%%%%
\subsection{Command Line Processing}
\label{sec:commandline}

The effect of redirection files can also be achieved by invoking
the \LaTeX{} compiler with a more elaborate command line.
Most conveniently this should be done as part
of a shell script or a |Makefile|.

When using \textsf{childdoc} in the main file, the following
command lines effectively perform a redirection
(note that depending on the shell being used,
backslashes may have to be doubled: `|\|' $\to$ `|\\|'):
%
\begin{center}
|... -jobname "|\textit{target}|" |\\|"|[\textit{flags}]%
|\input{childdoc.def}\childdocforward[|\textit{main}|]{|\textit{dest}|}"|
\end{center}
%
Here \textit{target} is the name of the output file,
\textit{main} is the name of the main file
and \textit{dest} is the name of the main or child file to be processed
(all filenames without extensions).
The optional argument \textit{main} can be omitted
if \textit{main} matches \textit{dest}.
Optionally, compilation \textit{flags} can be defined via |\def| commands.
This command line makes the \TeX{} engine believe
it is compiling the file \textit{target}
whose content is specified as the latter parameter.
The provided code then forwards the processing to
\textit{main} or \textit{dest} as described in \secref{sec:forward}.

%%%%%%%%%%%%%%%%%%%%%%%%%%%%%%%%%%%%%%%%%%%%%%%%%%%%%%%%%%%%%%%%%%%%%%%%%%%%%%%%
\subsection{Include by Input}
\label{sec:input}

Including child documents by |\include| has some restrictions by design.
Most notably, the content of a child document always occupies
its own set of pages; pages cannot be shared between child documents.
Usually, this behaviour makes perfect sense
because each child document contain an essential part of the document.
However, in some situations it may be desirable to compose
a document from a collection of parts
without having mandatory page breaks between then.
For this case, the package
provides a mechanism to include parts
by |\input| which can also be processed individually.
However, by construction this mechanism
requires manual handling of the content to be output.

%%%%%%%%%%%%%%%%%%%%%%%%%%%%%%%%%%%%%%%%
\DescribeMacro{\ifchilddocmanual}
The main file should be prepared as usual, see \secref{sec:include}.
However, the document body must make a distinction
between processing of an individual part and of the main document, e.g.:
%
\begin{center}
\begin{tabular}{l}
|\ifchilddocmanual|\\
|\input{\childdocname}|\\
|\||else|\\
\textit{document body with }|\input{|\textit{part}|}|\\
|\||fi|
\end{tabular}
\end{center}
%
The conditional |\ifchilddocmanual| is true whenever
a part to be included by |\input| is being compiled,
and the name of the part is stored in |\childdocname|.

%%%%%%%%%%%%%%%%%%%%%%%%%%%%%%%%%%%%%%%%
\DescribeMacro{\childdocby}
Each part to be included by |\input| should start with:
%
\begin{center}
\begin{tabular}{l}
|\input{childdoc.def}|\\
|\childdocby{|\textit{main}|}|\\
\end{tabular}
\end{center}
%
The directive |\childdocby| is similar to |\childdocof|
described in \secref{sec:include},
but the subsequent selection of content must be done manually.
To that end, both |\ifchilddoc| and |\ifchilddocmanual|
will be true upon processing of a part,
and the name of the part is stored in |\childdocname|.
Note that |\jobname| will be set to the filename of the current part
so that each part receives an individual |.aux| file
that does not interfere with the |.aux| file(s) of the main document.
This behaviour can be altered by the alternative form
|\childdocby[*]{|\textit{main}|}| (with a non-empty optional argument)
which uses the |.aux| file of the main document
by setting |\jobname| to \textit{main}.

%%%%%%%%%%%%%%%%%%%%%%%%%%%%%%%%%%%%%%%%%%%%%%%%%%%%%%%%%%%%%%%%%%%%%%%%%%%%%%%%
\subsection{Driver Development}
\label{sec:driver}

The \textsf{childdoc} mechanism can also be use for the development
of definition files such as \LaTeX{} styles or classes.
This case differs from the above setup with multiple parts
included by |\include| in that no |\includeonly| should be invoked.
This can be achieved by starting the include file
(before |\ProvidesPackage|) with:
%
\begin{center}
\begin{tabular}{l}
|\input{childdoc.def}|\\
|\childdocforward{|\textit{main}|}|\\
\end{tabular}
\end{center}
%
or alternatively with:
%
\begin{center}
\begin{tabular}{l}
|\input{childdoc.def}|\\
|\childdocby{|\textit{main}|}|\\
\end{tabular}
\end{center}
%
Both forms have slightly different effects as described above.
The main file is prepared as usual, see \secref{sec:include}.

%%%%%%%%%%%%%%%%%%%%%%%%%%%%%%%%%%%%%%%%%%%%%%%%%%%%%%%%%%%%%%%%%%%%%%%%%%%%%%%%
\subsection{Legacy Detection}
\label{sec:detection}

The directive |\childdocmain| in the main file can detect
whether the complete document or merely a child is to be compiled
even without using the directive |\childdocof|.
This method is deprecated because it is less robust
and there is no compelling reason to use it;
it is merely provided for backward compatibility
and it may be removed in future versions.

If the detection mechanism is to be used,
it is mandatory to correctly specify
the filename of the main file as the argument of |\childdocmain|:
%
\begin{center}
\begin{tabular}{l}
|\input{childdoc.def}|\\
|\childdocmain{|\textit{main}|}|\\
\end{tabular}
\end{center}
%
If |\jobname| does not match the argument \textit{main} of |\childdocmain|,
it is assumed that |\jobname| points to the child file to be compiled.
When using |\childdocmain| with the main file specified as argument,
it suffices to start a child file
with just |\input{|\textit{main}|}|
without loading of the package and using |\childdocof|.
If instead all processing is done
with the appropriate \textsf{childdoc} directives,
the argument of \textit{main} of |\childdocmain| can be empty.

An alternative version of the command line processing described
in \secref{sec:commandline} using the detection mechanism reads:
%
\begin{center}
|... -jobname "|\textit{target}|" "|[\textit{flags}]%
[|\def\jobname{|\textit{dest}|}|]|\input{|\textit{main}|}"|
\end{center}

%%%%%%%%%%%%%%%%%%%%%%%%%%%%%%%%%%%%%%%%%%%%%%%%%%%%%%%%%%%%%%%%%%%%%%%%%%%%%%%%
\subsection{Manual Code}
\label{sec:manual}

In case one cannot be certain whether the definitions file |childdoc.def|
is installed on the target \TeX{} distribution
and one prefers not to ship it,
it is conceivable to paste a few relevant commands into the sources.

To that end, drop all statements |\input{childdoc.def}|
and perform the replacements as outlined below.
Instead of |\childdocmain{|\textit{main}|}| add the following code
to the top of the main file:
%
\begin{center}
\begin{tabular}{l}
|\||ifdefined\childdocname\endinput\||fi\newif\ifchilddoc|\\
|\edef\childdocname{\scantokens\expandafter{\jobname\noexpand}}|\\
|\def\childdocmain{|\textit{main}|}\||ifx\childdocmain\childdocname\||else|\\
|\childdoctrue\includeonly{\childdocname}\let\jobname\childdocmain\||fi|\\
\end{tabular}
\end{center}
%
Instead of |\childdocof{|\textit{main}|}| just include the main file
at the top of each child file:
%
\begin{center}
|\input{|\textit{main}|}|
\end{center}
%
A simple redirection |\childdocforward{|\textit{dest}|}| is achieved by:
%
\begin{center}
|\def\jobname{|\textit{dest}|}\input{\jobname}|
\end{center}
%
The redirection with prefix
|\childdocforwardprefix[|\textit{prefix}|]{|\textit{dest}|}|
is accomplished by:
%
\begin{center}
\begin{tabular}{l}
|{\edef\jobname{\scantokens\expandafter{\jobname\noexpand}}|\\
|\def\redirectjob |\textit{prefix}|#1~~~{\gdef\jobname{|\textit{dest}|#1}}|\\
|\expandafter\redirectjob\jobname~~~}\input{\jobname}|
\end{tabular}
\end{center}

In an alternative approach,
child documents can be compiled by a specific command line
without additional code or specific definitions:
%
\begin{center}
|... -jobname "|\textit{target}|" "|[\textit{flags}]%
|\includeonly{|\textit{dest}|}\input{|\textit{main}|}"|
\end{center}
%

%%%%%%%%%%%%%%%%%%%%%%%%%%%%%%%%%%%%%%%%%%%%%%%%%%%%%%%%%%%%%%%%%%%%%%%%%%%%%%%%
%%%%%%%%%%%%%%%%%%%%%%%%%%%%%%%%%%%%%%%%%%%%%%%%%%%%%%%%%%%%%%%%%%%%%%%%%%%%%%%%
\section{Information}

%%%%%%%%%%%%%%%%%%%%%%%%%%%%%%%%%%%%%%%%%%%%%%%%%%%%%%%%%%%%%%%%%%%%%%%%%%%%%%%%
\subsection{Copyright}

Copyright \copyright{} 2017--2018 Niklas Beisert

This work may be distributed and/or modified under the
conditions of the \LaTeX{} Project Public License, either version 1.3
of this license or (at your option) any later version.
The latest version of this license is in
  \url{http://www.latex-project.org/lppl.txt}
and version 1.3 or later is part of all distributions of \LaTeX{}
version 2005/12/01 or later.

This work has the LPPL maintenance status `maintained'.

The Current Maintainer of this work is Niklas Beisert.

This work consists of the files |README.txt|, |childdoc.ins| and |childdoc.dtx|
as well as the derived files |childdoc.def|, |cdocsamp.tex|
with |cdocsch1.tex|, |cdocsch2.tex|, |cdocspt3.tex|, |cdocspt4.tex|,
|cdocsdrf.tex|, |cdocsfn1.tex|, |cdocsfn2.tex|
as well as |childdoc.pdf|.

%%%%%%%%%%%%%%%%%%%%%%%%%%%%%%%%%%%%%%%%%%%%%%%%%%%%%%%%%%%%%%%%%%%%%%%%%%%%%%%%
\subsection{Files and Installation}

The package consists of the files:
%
\begin{center}
\begin{tabular}{ll}
    |README.txt|   & readme file \\
    |childdoc.ins| & installation file \\
    |childdoc.dtx| & source file \\
    |childdoc.def| & definition file \\
    |cdocsamp.tex| & sample main file \\
    |cdocsch1.tex| & sample include file \\
    |cdocsch2.tex| & sample include file \\
    |cdocspt3.tex| & sample part file \\
    |cdocspt4.tex| & sample part file \\
    |cdocsdrf.tex| & sample redirection file \\
    |cdocsfn1.tex| & sample redirection file \\
    |cdocsfn2.tex| & sample redirection file \\
    |childdoc.pdf| & manual
\end{tabular}
\end{center}
%
The distribution consists of the files
|README.txt|, |childdoc.ins| and |childdoc.dtx|.
%
\begin{itemize}
\item
Run (pdf)\LaTeX{} on |childdoc.dtx|
to compile the manual |childdoc.pdf| (this file).
\item
Run \LaTeX{} on |childdoc.ins| to create the definitions file |childdoc.def|
and the sample |cdocsamp.tex| with include files
|cdocsch1.tex|, |cdocsch2.tex|, |cdocspt3.tex|, |cdocspt4.tex|,
|cdocsdrf.tex|, |cdocsfn1.tex|, |cdocsfn2.tex|.
Then copy the file |childdoc.def| to an appropriate directory of your \LaTeX{}
distribution, e.g.\ \textit{texmf-root}|/tex/latex/childdoc|.
\end{itemize}

%%%%%%%%%%%%%%%%%%%%%%%%%%%%%%%%%%%%%%%%%%%%%%%%%%%%%%%%%%%%%%%%%%%%%%%%%%%%%%%%
\subsection{Related CTAN Packages}

There are several other packages which offer a similar functionality:
%
\begin{itemize}
\item
The packages
\href{http://ctan.org/pkg/docmute}{\textsf{docmute}},
\href{http://ctan.org/pkg/includex}{\textsf{includex}} and
\href{http://ctan.org/pkg/standalone}{\textsf{standalone}}
provide commands to include only the document body of
a child file thus allowing both files to be compiled individually.
\item
The packages \href{http://ctan.org/pkg/subdocs}{\textsf{subdocs}}
and \href{http://ctan.org/pkg/subfiles}{\textsf{subfiles}}
provide structures in which the main and child documents can be
encapsulated and allowing them to be compiled individually.
The inclusion mechanism is different from the conventional |\include|.
\item
The package \href{http://ctan.org/pkg/combine}{\textsf{combine}}
is an elaborate solution to combine several documents into one.
\end{itemize}
%
See also the CTAN topic \href{http://ctan.org/topic/subdocs}{\textsf{subdocs}}
for further related packages.
The present package differs from the above solutions in that
a document structure constructed with the conventional |\include| mechanism
just needs two extra commands at the top of every file
such that all constituent files can be compiled individually.

%%%%%%%%%%%%%%%%%%%%%%%%%%%%%%%%%%%%%%%%%%%%%%%%%%%%%%%%%%%%%%%%%%%%%%%%%%%%%%%%
%\subsection{Feature Suggestions}
%
%The following is a list of features which may be useful for future
%versions of this package:
%%
%\begin{itemize}
%\item
%\ldots
%\end{itemize}

%%%%%%%%%%%%%%%%%%%%%%%%%%%%%%%%%%%%%%%%%%%%%%%%%%%%%%%%%%%%%%%%%%%%%%%%%%%%%%%%
\subsection{Revision History}

%%%%%%%%%%%%%%%%%%%%%%%%%%%%%%%%%%%%%%%%
\paragraph{v2.0:} 2018/12/30

\begin{itemize}
\item
immediate forward processing
\item
added |\childdocby| mechanism
\item
manual restructured
\end{itemize}

%%%%%%%%%%%%%%%%%%%%%%%%%%%%%%%%%%%%%%%%
\paragraph{v1.6:} 2018/01/17

\begin{itemize}
\item
application for development of include files
\item
corrections to manual
\end{itemize}

%%%%%%%%%%%%%%%%%%%%%%%%%%%%%%%%%%%%%%%%
\paragraph{v1.5:} 2017/05/21

\begin{itemize}
\item
more complete structuring introduced
\item
|\childdocof| introduced
\item
|\childdoc| renamed to |\childdocmain|
\item
|\childredirect| renamed to |\childdocforward| and |\childdocforwardprefix|
and functionality expanded
\end{itemize}

%%%%%%%%%%%%%%%%%%%%%%%%%%%%%%%%%%%%%%%%
\paragraph{v1.0:} 2017/04/27

\begin{itemize}
\item
manual and install package
\item
first version published on CTAN
\end{itemize}

%%%%%%%%%%%%%%%%%%%%%%%%%%%%%%%%%%%%%%%%
\paragraph{v0.6:} 2017/04/26

\begin{itemize}
\item
redirection mechanism added
\end{itemize}

%%%%%%%%%%%%%%%%%%%%%%%%%%%%%%%%%%%%%%%%
\paragraph{v0.5:} 2017/04/26

\begin{itemize}
\item
functionality in definition file
\end{itemize}


%%%%%%%%%%%%%%%%%%%%%%%%%%%%%%%%%%%%%%%%%%%%%%%%%%%%%%%%%%%%%%%%%%%%%%%%%%%%%%%%
%%%%%%%%%%%%%%%%%%%%%%%%%%%%%%%%%%%%%%%%%%%%%%%%%%%%%%%%%%%%%%%%%%%%%%%%%%%%%%%%
%%%%%%%%%%%%%%%%%%%%%%%%%%%%%%%%%%%%%%%%%%%%%%%%%%%%%%%%%%%%%%%%%%%%%%%%%%%%%%%%
\appendix

\settowidth\MacroIndent{\rmfamily\scriptsize 000\ }

 \DocInput{childdoc.dtx}

\end{document}
%</driver>
% \fi
%
% %%%%%%%%%%%%%%%%%%%%%%%%%%%%%%%%%%%%%%%%%%%%%%%%%%%%%%%%%%%%%%%%%%%%%%%%%%%%%%
% %%%%%%%%%%%%%%%%%%%%%%%%%%%%%%%%%%%%%%%%%%%%%%%%%%%%%%%%%%%%%%%%%%%%%%%%%%%%%%
% \section{Sample}
%\iffalse
%<*samplemain>
%\fi
%
% The following presents a sample document
% with two chapters, two parts, a title page,
% a compile flag as well as three forwarding files to set the flag.
% It consists of eight |.tex| files:
% \begin{center}
% \begin{tabular}{ll}
% |cdocsamp.tex|&main file\\
% |cdocsch1.tex|&include file for chapter 1\\
% |cdocsch2.tex|&include file for chapter 2\\
% |cdocspt3.tex|&include file for part 3\\
% |cdocspt4.tex|&include file for part 4\\
% |cdocsdrf.tex|&forwarding file for main file in draft mode\\
% |cdocsfi1.tex|&forwarding file for final version of chapter 1\\
% |cdocsfi2.tex|&forwarding file for final version of chapter 2\\
% \end{tabular}
% \end{center}
% Each of the eight files can be compiled directly by the \LaTeX{} compiler.
%
% %%%%%%%%%%%%%%%%%%%%%%%%%%%%%%%%%%%%%%
% \paragraph{Main File.}
%
% The main file is called |cdocsamp.tex|.
%
% Load the \textsf{childdoc} definitions and
% declare the filename for the main document:
%    \begin{macrocode}
\input{childdoc.def}
\childdocmain{}
%    \end{macrocode}

% Optional override for |\version| flag:
%    \begin{macrocode}
%%\ifchilddoc\else\providecommand{\version}{draft}\fi
%    \end{macrocode}

% Define the default values for the |\version| flag
% (|final| for the main file and |draft| for childs):
%    \begin{macrocode}
\ifchilddoc
\providecommand{\version}{draft}
\else
\providecommand{\version}{final}
\fi
%    \end{macrocode}

% Load the standard document class:
%    \begin{macrocode}
\documentclass[12pt]{article}
%    \end{macrocode}

% Start the document body:
%    \begin{macrocode}
\begin{document}
%    \end{macrocode}

% Declare a title page.
% Print title, part of document being processed and version flag:
%    \begin{macrocode}
\addtocounter{page}{-1}
\begin{center}
{\LARGE\bfseries{}childdoc example\par}
\vspace{1cm}
\ifchilddoc
\ifchilddocmanual part\else chapter\fi:
`\childdocname' of `\childdocjob'\par
\else
main document: `\childdocjob'\par
\fi
version: \version\par
\end{center}
\newpage
%    \end{macrocode}

% Manually include selected file,
% otherwise process as usual:
%    \begin{macrocode}
\ifchilddocmanual
\section*{part `\childdocname'}
\input{\childdocname}
\else
%    \end{macrocode}

% Include the two chapters:
%    \begin{macrocode}
\include{cdocsch1}
\include{cdocsch2}
%    \end{macrocode}

% Include the two parts unless only chapters should be displayed:
%    \begin{macrocode}
\ifchilddoc\else
\section{part three}
\input{cdocspt3}
\section{part four}
\input{cdocspt4}
\fi
%    \end{macrocode}

% Process as usual until here:
%    \begin{macrocode}
\fi
%    \end{macrocode}

% End of document body:
%    \begin{macrocode}
\end{document}
%    \end{macrocode}
%\iffalse
%</samplemain>
%\fi
%
% %%%%%%%%%%%%%%%%%%%%%%%%%%%%%%%%%%%%%%
% \paragraph{Chapter Include Files.}
%
% The include files are called |cdocsch1.tex| and |cdocsch2.tex|.
%
%\iffalse
%<*samplechap1|samplechap2>
%\fi

% Optional override for |\version| flag:
%    \begin{macrocode}
%%\providecommand{\version}{final}
%    \end{macrocode}

% Include the main document:
%    \begin{macrocode}
\input{childdoc.def}
\childdocof{cdocsamp}
%    \end{macrocode}

%\iffalse
%</samplechap1|samplechap2>
%\fi
%
%\iffalse
%<*samplechap1>
%\fi
% Some text for chapter 1:
%    \begin{macrocode}
\section{one}
some text in chapter one
%    \end{macrocode}

%\iffalse
%</samplechap1>
%\fi
% Some text for chapter 2:
%\iffalse
%<*samplechap2>
%\fi
%    \begin{macrocode}
\section{two}
more text in chapter two
%    \end{macrocode}

%\iffalse
%</samplechap2>
%\fi
%
% %%%%%%%%%%%%%%%%%%%%%%%%%%%%%%%%%%%%%%
% \paragraph{Part Include Files.}
%
% The include files are called |cdocspt3.tex| and |cdocspt4.tex|.
%
%\iffalse
%<*samplepart3|samplepart4>
%\fi

% Optional override for |\version| flag:
%    \begin{macrocode}
%%\providecommand{\version}{final}
%    \end{macrocode}

% Include the main document:
%    \begin{macrocode}
\input{childdoc.def}
\childdocby{cdocsamp}
%    \end{macrocode}

%\iffalse
%</samplepart3|samplepart4>
%\fi
%
%\iffalse
%<*samplepart3>
%\fi
% Some text for part 3:
%    \begin{macrocode}
some text in part three
%    \end{macrocode}

%\iffalse
%</samplepart3>
%\fi
% Some text for part 4:
%\iffalse
%<*samplepart4>
%\fi
%    \begin{macrocode}
more text in part four
%    \end{macrocode}

%\iffalse
%</samplepart4>
%\fi
%
% %%%%%%%%%%%%%%%%%%%%%%%%%%%%%%%%%%%%%%
% \paragraph{Forwarding for a Complete Draft.}
%
% The following forwarding file |cdocsdrf.tex|
% compiles the main document in draft mode:
%\iffalse
%<*sampledraft>
%\fi
%    \begin{macrocode}
\def\version{draft}
\input{childdoc.def}
\childdocforward{cdocsamp}
%    \end{macrocode}

%\iffalse
%</sampledraft>
%\fi
%
% %%%%%%%%%%%%%%%%%%%%%%%%%%%%%%%%%%%%%%
% \paragraph{Forwarding for Final Version of the Chapters.}
%
% The following forwarding files |cdocsfn1.tex| and |cdocsfn2.tex|
% (with identical content)
% compile the final versions of the child documents
% |cdocsch1.tex| and |cdocsch2.tex|, respectively:
%\iffalse
%<*samplefinal>
%\fi
%    \begin{macrocode}
\def\version{final}
\input{childdoc.def}
\childdocforwardprefix[cdocsamp]{cdocsfn}{cdocsch}
%    \end{macrocode}

%\iffalse
%</samplefinal>
%\fi
%
% %%%%%%%%%%%%%%%%%%%%%%%%%%%%%%%%%%%%%%
% \paragraph{Command Line Processing.}
%
% The following three command lines generate the output files
% |cdocscld|, |cdocscl1| and |cdocscl2|
% which should be identical to
% |cdocsdrf|, |cdocsch1| and |cdocsfn2|, respectively:
% \begin{center}
% \begin{tabular}{l}
% |latex -jobname cdocscld \|\\
% |  "\def\version{draft}\input{childdoc.def}\childdocforward{cdocsamp}"|\\
% |latex -jobname cdocscl1 \|\\
% |  "\input{childdoc.def}\childdocforward[cdocsamp]{cdocsch1}"|\\
% |latex -jobname cdocscl2 \|\\
% |  "\def\version{final}\input{childdoc.def}\childdocforward{cdocsch2}"|
% \end{tabular}
% \end{center}
% Note that the trailing backslash on each first line
% merely continues the input to the second line
% (for convenient cut ant paste).
% Furthermore, the command |latex| can be replaced by any
% of its alternative versions such as |pdflatex|.
%
% %%%%%%%%%%%%%%%%%%%%%%%%%%%%%%%%%%%%%%%%%%%%%%%%%%%%%%%%%%%%%%%%%%%%%%%%%%%%%%
% %%%%%%%%%%%%%%%%%%%%%%%%%%%%%%%%%%%%%%%%%%%%%%%%%%%%%%%%%%%%%%%%%%%%%%%%%%%%%%
% \section{Implementation}
%\iffalse
%<*package>
%\fi
%
% This section describes the definitions file |childdoc.def|.

% The definitions cannot be loaded using |\usepackage| or |\RequirePackage|
% which has a mechanism to prevent loading a style file more than once.
% When loading the definitions by means of |\input|
% multiple instances have to be prevented manually:
%\iffalse
%This code needs to be before the `\ProvidesFile' directive
%which is defined at the beginning of this file.
%Therefore it is also placed there and commented out here.
%</package>
%<*discard>
%\fi
%    \begin{macrocode}
\ifdefined\childdocmain\endinput\fi
%    \end{macrocode}
%\iffalse
%</discard>
%<*package>
%\fi
%
% \macro{\ifchilddoc}
% \macro{\ifchilddocmanual}
% The conditional |\ifchilddoc| tells whether a
% child (true) or main (false) document is being compiled.
% The conditional |\ifchilddocmanual| tells whether
% the |\includeonly| mechanism is used (false) or
% the selection of child files must be performed manually (true).
% The definitions initialise to false:
%    \begin{macrocode}
\newif\ifchilddoc
\newif\ifchilddocmanual
%    \end{macrocode}

% \macro{\childdocname}
% \macro{\childdocjob}
% The macro |\childdocname| stores the name of the main document
% to be compiled. The macro |\childdocjob| stores the name of
% the document on which the \LaTeX{} compiler was originally invoked.
% The content of |\jobname| cannot be compared
% to filenames specified in the source due to different catcodes.
% The following code rescans |\jobname|, stores the result
% in |\childdocname| and saves a copy in |\childdocjob|:
%    \begin{macrocode}
\edef\childdocname{\scantokens\expandafter{\jobname\noexpand}}
\let\childdocjob\childdocname
%    \end{macrocode}

% \macro{\childdocdisable}
% The macro |\childdocdisable| prevents the main file
% from being processed more than once.
% At this stage, the main document command |\childdocmain|
% is assumed to be called once again where it should do nothing.
% Any subsequent call to it should prevent
% a secondary processing of the main document
% It overwrites the forwarding commands
% |\childdocof| and |\childdocforward|
% with empty macros to prevent further inclusions of the main document:
%    \begin{macrocode}
\newcommand{\childdocdisable}
{
  \renewcommand{\childdocmain}[1]{\renewcommand{\childdocmain}[1]{\endinput}}
  \renewcommand{\childdocof}[1]{}
  \renewcommand{\childdocby}[2][]{}
  \renewcommand{\childdocforward}[2][]{}
  \renewcommand{\childdocdisable}{}
}
%    \end{macrocode}

% \macro{\childdocmain}
% The macro |\childdocmain| is to be called at the top of the main file
% with nothing or the main filename (without extension) as argument.
% First, it breaks loops.
% If the argument is not empty and does not match |\childdocname|
% (which is set by the first inclusion of |childdoc.def|),
% |\ifchilddoc| is set to true, |\includeonly| is applied to the child file
% and |\jobname| is set to the main file
% (for proper handling of |.aux| files):
%    \begin{macrocode}
\newcommand{\childdocmain}[1]
{
  \childdocdisable\childdocmain{}
  \if?#1?\else
    \begingroup
      \def\childdoctmp{#1}
      \ifx\childdoctmp\childdocname
        \def\childdoctmp{}
      \else
        \def\childdoctmp
        {
          \childdoctrue
          \includeonly{\childdocname}
          \def\childdocjob{#1}
          \def\jobname{#1}
        }
      \fi
      \expandafter
    \endgroup
    \childdoctmp
  \fi
}
%    \end{macrocode}

% \macro{\childdocof}
% The command |\childdocof| redirects
% compilation to the main file |#1|.
%    \begin{macrocode}
\newcommand{\childdocof}[1]
{
  \childdocdisable
  \childdoctrue
  \includeonly{\childdocname}
  \def\jobname{#1}
  \def\childdocjob{#1}
  \input{#1}
}
%    \end{macrocode}

% \macro{\childdocby}
% The command |\childdocby| ....
%    \begin{macrocode}
\newcommand{\childdocby}[2][]
{
  \childdocdisable
  \childdoctrue
  \childdocmanualtrue
  \if?#1?\else
    \def\jobname{#2}
  \fi
  \def\childdocjob{#2}
  \input{#2}
  \endinput
}
%    \end{macrocode}

% \macro{\childdocforward}
% The command |\childdocforward| redirects
% compilation to the main file or
% (if the optional argument is given) a child file.
% Parameters are set as if the main file
% or a child file starting with |\childdocof| was compiled.
% Then compilation is handed over to the main file:
%    \begin{macrocode}
\newcommand{\childdocforward}[2][]
{
  \begingroup
    \if?#1?
      \def\childdoctmp
      {
        \def\childdocname{#2}
        \def\childdocjob{#2}
        \def\jobname{#2}
        \input{#2}
        \endinput
      }
    \else
      \def\childdoctmp
      {
        \childdocdisable
        \def\childdocname{#2}
        \childdoctrue
        \includeonly{#2}
        \def\childdocjob{#1}
        \def\jobname{#1}
        \input{#1}
        \endinput
      }
    \fi
    \expandafter
  \endgroup
  \childdoctmp
}
%    \end{macrocode}

% \macro{\childdocforwardprefix}
% The command |\childdocforwardprefix| redirects
% compilation to the main or a child file by means of a pattern.
% The prefix |#1| in the current filename is replaced by |#2|
% and the suffix of the current filename is kept
% (it is assumed that the filename does not contain the substring `|~~~|'
% which is used as a delimiter).
% Compilation is handed over to the new file by |\childdocforward|:
%    \begin{macrocode}
\newcommand{\childdocforwardprefix}[3][]
{
  \begingroup
    \def\childdocextract #2##1~~~{\def\childdoctmp{\childdocforward[#1]{#3##1}}}
    \expandafter\childdocextract\childdocname~~~
    \expandafter
  \endgroup
  \childdoctmp
}
%    \end{macrocode}

% \macro{\childdoc}
% The deprecated macro |\childdoc| is a legacy version of |\childdocmain|:
%    \begin{macrocode}
\newcommand{\childdoc}{\childdocmain}
%    \end{macrocode}

% \macro{\childdocredirect}
% The deprecated macro |\childdocredirect| is a legacy version
% of |\childdocforward| and |\childdocforwardprefix|:
%    \begin{macrocode}
\newcommand{\childdocredirect}[2][]
{
  \begingroup
    \if?#1?
      \def\childdoctmp{\childdocforward{#2}}
    \else
      \def\childdoctmp{\childdocforwardprefix{#1}{#2}}
    \fi
    \expandafter
  \endgroup
  \childdoctmp
}
%    \end{macrocode}

%\iffalse
%</package>
%\fi
%
\endinput
|\\
|\childdocby{|\textit{main}|}|\\
\end{tabular}
\end{center}
%
Both forms have slightly different effects as described above.
The main file is prepared as usual, see \secref{sec:include}.

%%%%%%%%%%%%%%%%%%%%%%%%%%%%%%%%%%%%%%%%%%%%%%%%%%%%%%%%%%%%%%%%%%%%%%%%%%%%%%%%
\subsection{Legacy Detection}
\label{sec:detection}

The directive |\childdocmain| in the main file can detect
whether the complete document or merely a child is to be compiled
even without using the directive |\childdocof|.
This method is deprecated because it is less robust
and there is no compelling reason to use it;
it is merely provided for backward compatibility
and it may be removed in future versions.

If the detection mechanism is to be used,
it is mandatory to correctly specify
the filename of the main file as the argument of |\childdocmain|:
%
\begin{center}
\begin{tabular}{l}
|% \iffalse
%
% childdoc.dtx Copyright (C) 2017-2018 Niklas Beisert
%
% This work may be distributed and/or modified under the
% conditions of the LaTeX Project Public License, either version 1.3
% of this license or (at your option) any later version.
% The latest version of this license is in
%   http://www.latex-project.org/lppl.txt
% and version 1.3 or later is part of all distributions of LaTeX
% version 2005/12/01 or later.
%
% This work has the LPPL maintenance status `maintained'.
%
% The Current Maintainer of this work is Niklas Beisert.
%
% This work consists of the files childdoc.dtx and childdoc.ins
% and the derived files childdoc.def and cdocsamp.tex with
% cdocsch1.tex, cdocsch2.tex, cdocsdrf.tex, cdocsfn1.tex, cdocsfn2.tex.
%
%<package>\ifdefined\childdocmain\endinput\fi
%<package>\ProvidesFile{childdoc.def}[2018/12/30 v2.0 child document driver]
%<samplemain>\ProvidesFile{cdocsamp.tex}[2018/12/30 v2.0 sample for childdoc]
%<*driver>
%\ProvidesFile{childdoc.drv}[2018/12/30 v2.0 childdoc reference manual file]
\PassOptionsToClass{10pt,a4paper}{article}
\documentclass{ltxdoc}

\usepackage[margin=35mm]{geometry}
\usepackage{hyperref}
\usepackage{hyperxmp}
\usepackage[usenames]{color}

\hypersetup{colorlinks=true}
\hypersetup{pdfstartview=FitH}
\hypersetup{pdfpagemode=UseNone}
\hypersetup{pdfsource={}}
\hypersetup{pdflang={en-UK}}
\hypersetup{pdfcopyright={Copyright 2017-2018 Niklas Beisert.
  This work may be distributed and/or modified under the
  conditions of the LaTeX Project Public License, either version 1.3
  of this license or (at your option) any later version.}}
\hypersetup{pdflicenseurl={http://www.latex-project.org/lppl.txt}}
\hypersetup{pdfcontactaddress={ETH Zurich, ITP, HIT K,
  Wolfgang-Pauli-Strasse 27}}
\hypersetup{pdfcontactpostcode={8093}}
\hypersetup{pdfcontactcity={Zurich}}
\hypersetup{pdfcontactcountry={Switzerland}}
\hypersetup{pdfcontactemail={nbeisert@itp.phys.ethz.ch}}
\hypersetup{pdfcontacturl={http://people.phys.ethz.ch/\xmptilde nbeisert/}}

\newcommand{\secref}[1]{\hyperref[#1]{section \ref*{#1}}}

\parskip1ex
\parindent0pt
\let\olditemize\itemize
\def\itemize{\olditemize\parskip0pt}

\begin{document}

\title{The \textsf{childdoc} Package}
\hypersetup{pdftitle={The childdoc Package}}
\author{Niklas Beisert\\[2ex]
  Institut f\"ur Theoretische Physik\\
  Eidgen\"ossische Technische Hochschule Z\"urich\\
  Wolfgang-Pauli-Strasse 27, 8093 Z\"urich, Switzerland\\[1ex]
  \href{mailto:nbeisert@itp.phys.ethz.ch}
  {\texttt{nbeisert@itp.phys.ethz.ch}}}
\hypersetup{pdfauthor={Niklas Beisert}}
\hypersetup{pdfsubject={Manual for the LaTeX2e Package childdoc}}
\date{30 December 2018, \textsf{v2.0}}
\maketitle

\begin{abstract}\noindent
\textsf{childdoc} is a \LaTeXe{} package
that enables the direct compilation
of document sections included by |\include|
to individual files.
\end{abstract}

\begingroup
\parskip0ex
\tableofcontents
\endgroup

%%%%%%%%%%%%%%%%%%%%%%%%%%%%%%%%%%%%%%%%%%%%%%%%%%%%%%%%%%%%%%%%%%%%%%%%%%%%%%%%
%%%%%%%%%%%%%%%%%%%%%%%%%%%%%%%%%%%%%%%%%%%%%%%%%%%%%%%%%%%%%%%%%%%%%%%%%%%%%%%%
\section{Introduction}

\LaTeX{} provides a mechanism to structure a large document (such as a book)
into a main file and several child files (containing the chapters)
using the |\include| command.
This mechanism is beneficial for documents
which span hundreds of pages in order to
make the source file(s) more manageable.
Moreover, compilation can be restricted to
selected child files by means of the |\includeonly| command.
The latter feature can be used to reduce the compilation time while editing
(this was significantly more useful in the earlier days of \LaTeX{})
or to generate a smaller document which is easier to navigate.
Another application of |\includeonly| is to generate
documents consisting of selected parts of the complete document.

However, there are a few drawbacks of the plain |\include| mechanism:
\begin{itemize}
\item
The child files cannot be compiled on their own,
they can only be compiled via the main file.
A naive editing environment
(such as a text editor with an option
to have the current file processed by \LaTeX)
may require one to switch to the main file before compiling;
attempting to compile the child file produces errors.
\item
The main file must be modified (each time)
to adjust the |\includeonly| command
to the present needs. This easily leaves the main file in a messy state.
\item
The generated document will always carry the filename
of the main document. This is inconvenient if
several child files are to be compiled and
to be kept for distribution.
\end{itemize}

The present package provides a simple interface
to make child files individually compilable by \LaTeX{}.
Compiling a child file then has the same effect as compiling
the main file with an |\includeonly| command
to select the appropriate child.
Moreover the generated document will carry the name of the child
rather than the main file.
This resolves all three above issues.

This feature is meant to make the editing of books,
thesis documents and lecture notes somewhat more convenient.
However, the package can also be used efficiently for
composing a series of documents (such as exercise sheets)
which are typically distributed individually.
It then assists the author in generating the individual documents
(potentially in different versions)
as well as a document containing the collected series.
Another application is in developing style files
or other kinds of included material
where compilation of the style file could redirect
to a sample or test file.

%%%%%%%%%%%%%%%%%%%%%%%%%%%%%%%%%%%%%%%%%%%%%%%%%%%%%%%%%%%%%%%%%%%%%%%%%%%%%%%%
%%%%%%%%%%%%%%%%%%%%%%%%%%%%%%%%%%%%%%%%%%%%%%%%%%%%%%%%%%%%%%%%%%%%%%%%%%%%%%%%
\section{Usage}

First of all, the package \textsf{childdoc} is \emph{not} a standard
\LaTeXe{} |.sty| style file! Therefore it needs to be invoked in
a non-standard way.

%%%%%%%%%%%%%%%%%%%%%%%%%%%%%%%%%%%%%%%%%%%%%%%%%%%%%%%%%%%%%%%%%%%%%%%%%%%%%%%%
\subsection{Included Files}
\label{sec:include}

%%%%%%%%%%%%%%%%%%%%%%%%%%%%%%%%%%%%%%%%
\DescribeMacro{\childdocmain}
To use the package, add the commands
\begin{center}
\begin{tabular}{l}
|\input{childdoc.def}|\\
|\childdocmain{}|\\
\end{tabular}
\end{center}
at the very top of the main \LaTeX{} file,
in particular \emph{before} the |\documentclass| statement!
The argument of |\childdocmain| should be left empty
(but it must be present).

%%%%%%%%%%%%%%%%%%%%%%%%%%%%%%%%%%%%%%%%
\DescribeMacro{\childdocof}
Furthermore, add the commands
\begin{center}
\begin{tabular}{l}
|\input{childdoc.def}|\\
|\childdocof{|\textit{main}|}|\\
\end{tabular}
\end{center}
at the top of every child file \textit{child}
which is included by |\include{|\textit{child}|}|
from within the main file
(or at least for those files to be compiled individually).
The argument \textit{main} must be the filename of the main file.

There are a couple of
considerations in setting up the main and child documents:

%%%%%%%%%%%%%%%%%%%%%%%%%%%%%%%%%%%%%%%%
\paragraph{Restrictions.}

Please note the following restrictions:
\begin{itemize}
\item
|\childdocmain| must be called with one argument \textit{main}
to ensure compatibility with earlier version of the package.
It must either be empty (|\childdocmain{}|)
or precisely match the filename of the main file in which it is specified.
See \secref{sec:detection} for further information.
\item
The filename \textit{main} must be specified without the |.tex| extension.
\item
The filename \textit{main} is case sensitive
(even in case-insensitive file systems)
due to internal string comparison.
\item
The argument \textit{main} should be fully expanded, it cannot be a macro.
\item
Subdirectories and special characters should be avoided in filenames.
\item
The command |\childdocmain{|\textit{main}|}| must be followed by a whitespace.
It should not be followed immediately by another command
or by a comment mark `|%|'.
This is because the \TeX{} parser reads the token immediately following
the argument of |\childdocmain| and puts it
at the beginning of every child section;
however, a white\-space is ignored.
\end{itemize}

%%%%%%%%%%%%%%%%%%%%%%%%%%%%%%%%%%%%%%%%
\paragraph{Content of Main File.}

It is advisable to place all content in the child files included by |\include|.
Any output contained in the main file will appear in all child documents
unless suppressed manually;
it cannot be suppressed automatically by the |\includeonly| directive
and thus should normally be avoided.
A method to include some content in the main file
by means of conditional processing is described in \secref{sec:conditional}.

%%%%%%%%%%%%%%%%%%%%%%%%%%%%%%%%%%%%%%%%
\paragraph{Page Numbering.}

When only a part of the document is compiled,
the appropriate numbering of pages
(as well as other status parameters)
is determined from the |.aux| files.
The latter contain information from previous passes.
However this information needs to propagate through
all intermediate child documents.
Therefore the page numbering in child documents may well
be inconsistent until the complete document is compiled at least once.

A useful (if unconventional) way to always ensure a consistent
page numbering is to restart the numbering in each child document
and denote the pages by `\textit{child}|.|\textit{page}'
where \textit{child} represents the chapter/section number of the child file.
This can be achieved by the command
|\numberwithin{page}{|\textit{child}|}|
of the \textsf{amsmath} package
where \textit{child} can be |chapter| or |section|
depending on the chosen structuring.
Alternatively, one can modify the macro |\thepage| appropriately
and reset the counter |page| at the start of each child file.

%%%%%%%%%%%%%%%%%%%%%%%%%%%%%%%%%%%%%%%%%%%%%%%%%%%%%%%%%%%%%%%%%%%%%%%%%%%%%%%%
\subsection{Conditional Processing}
\label{sec:conditional}

The package provides a mechanism to compile different versions
of a document. To customise the versions further some conditional processing
can come in handy to distinguish which version is being compiled.
The package provides two macros to describe the compilation context:

%%%%%%%%%%%%%%%%%%%%%%%%%%%%%%%%%%%%%%%%
\DescribeMacro{\ifchilddoc}
The conditional |\ifchilddoc| distinguishes between the compilation of
child documents and the main document:
%
\begin{center}
|\ifchilddoc |\textit{child-code}| |[|\||else |\textit{main-code}]| \||fi|
\end{center}

%%%%%%%%%%%%%%%%%%%%%%%%%%%%%%%%%%%%%%%%
\DescribeMacro{\childdocname}
\DescribeMacro{\childdocjob}
The macro |\childdocname| contains the filename (without extension)
of the main or child file being processed.
Note that |\childdocjob| will always contain the name of the main file.

%%%%%%%%%%%%%%%%%%%%%%%%%%%%%%%%%%%%%%%%
\paragraph{Title Page.}

Conditional processing can be used to include a title or banner page
in the main document when proper precautions are taken.
Importantly, the code in the main file should ensure that the page counter
(as well as other status parameters which are stored in the |.aux| files)
takes the same value after the conditional processing.
Otherwise the page numbers may take divergent values
depending on which part is compiled.

For example, a title page could be declared by:
%
\begin{center}
\begin{tabular}{l}
|\ifchilddoc\||else|\\
|\addtocounter{page}{-1}|\\
\textit{code for title page}\\
|\newpage|\\
|\||fi|
\end{tabular}
\end{center}
%
A banner page for the child documents can be generated by:
%
\begin{center}
\begin{tabular}{l}
|\ifchilddoc|\\
|\addtocounter{page}{-1}|\\
\textit{code for banner page}\\
|\newpage|\\
|\||fi|
\end{tabular}
\end{center}
%
Here one could write a message such as:
\begin{center}
|This is the part \childdocname{} of \childdocjob{}.|
\end{center}

%%%%%%%%%%%%%%%%%%%%%%%%%%%%%%%%%%%%%%%%%%%%%%%%%%%%%%%%%%%%%%%%%%%%%%%%%%%%%%%%
\subsection{Flags}
\label{sec:flags}

The package makes it easy to generate different versions
of the main or child documents.
To this end compilation flags can be defined
and assigned different default values.
They will be particularly useful in conjunction
with the forwarding mechanism described in \secref{sec:forward}.

For example, it may be useful to have a flag |\version|
which can be set to |draft| or |final|.
The document source will contain some conditional code
depending on the value of |\version|.
Suppose further, the flag should default to |final| for the main file
and to |draft| for child files
which is a natural assignment for editing the document.
This is achieved by placing the following code
in the preamble of the main document
(below the |\childdocmain| directive):
%
\begin{center}
\begin{tabular}{l}
|\ifchilddoc|\\
|\providecommand{\version}{draft}|\\
|\||else|\\
|\providecommand{\version}{final}|\\
|\||fi|
\end{tabular}
\end{center}
%
The definition by |\providecommand| makes sure
that previous definitions are not overwritten.
Further statements |\providecommand{\version}{...}|
can thus be added before the above code to override it.

For the main file, one might add a line
(between |\childdocmain| and the above block)
%
\begin{center}
|%\ifchilddoc\||else\providecommand{\version}{draft}\||fi|
\end{center}
%
which can be uncommented to produce a draft version.
Likewise one can add a line to the very top of a child file
(above the |\childdocof{|\textit{main}|}| directive)
%
\begin{center}
|%\providecommand{\version}{final}|
\end{center}
%
which can be uncommented to produce the final version of this child document.

%%%%%%%%%%%%%%%%%%%%%%%%%%%%%%%%%%%%%%%%%%%%%%%%%%%%%%%%%%%%%%%%%%%%%%%%%%%%%%%%
\subsection{Forwarding}
\label{sec:forward}

Different versions of the main or child documents
using compilation flags as described in \secref{sec:flags}
can be (permanently) stored in different files
for convenient compilation, viewing and distribution.
To this end, the package defines a command
to pass on compilation to a different file:

%%%%%%%%%%%%%%%%%%%%%%%%%%%%%%%%%%%%%%%%
\DescribeMacro{\childdocforward}
The command |\childdocforward| redirects processing to
another source file:
%
\begin{center}
\begin{tabular}{l}
|\input{childdoc.def}|\\
|\childdocforward[|\textit{main}|]{|\textit{dest}|}|\\
\end{tabular}
\end{center}
%
The argument \textit{dest} is the destination file
(without extension).
It should be the main file or one of the child files.
Note that further \textsf{childdoc} directives
such as |\childdocof| and |\childdocforward|
in the indicated file will be processed in this form.
The optional argument \textit{main}
passes on directly to the main file \textit{main}
while pretending to compile the child \textit{dest}.
This form behaves as if \textit{dest}
issues |\childdocof{|\textit{main}|}| right away,
and no further \textsf{childdoc} directives will be processed.

%%%%%%%%%%%%%%%%%%%%%%%%%%%%%%%%%%%%%%%%
\DescribeMacro{\...prefix}
In the alternative form |\childdocforwardprefix|,
%
\begin{center}
\begin{tabular}{l}
|\input{childdoc.def}|\\
|\childdocforwardprefix[|\textit{main}|]{|\textit{prefix}|}{|\textit{dest}|}|
\end{tabular}
\end{center}
%
the destination file is determined by a pattern
depending on the current file:
To make this work, the current file must be called
`{\textit{prefix}\hspace{0.2em}\textit{suffix}}'
with \textit{prefix} matching precisely the argument.
Processing is then passed on to the file
`{\textit{dest}\hspace{0.2em}\textit{suffix}}'.
Surely, the same effect is achieved by
directly specifying the
argument `{\textit{dest}\hspace{0.2em}\textit{suffix}}'
in the first form.
However, that requires to set up a different file
for each child. With the alternative form of the command
all these files can have exactly the same content
which simplifies setting them up and maintaining them.

For example, the following file |draft.tex|
with a compilation flag |\version| as described in \secref{sec:flags}
compiles the main document as a draft:
%
\begin{center}
\begin{tabular}{l}
|\def\version{draft}|\\
|\input{childdoc.def}|\\
|\childdocforward{|\textit{main}|}|
\end{tabular}
\end{center}
%
Likewise, the following files |final|\textit{nn}|.tex|
compile the final version of the child document
|child|\textit{nn}|.tex|:
%
\begin{center}
\begin{tabular}{l}
|\def\version{final}|\\
|\input{childdoc.def}|\\
|\childdocforwardprefix{final}{child}|
\end{tabular}
\end{center}
%

Note that when several versions of a main file and/or of each child file
are to be generated, it may be convenient to set up a |Makefile| or
shell script to automatise the process.

%%%%%%%%%%%%%%%%%%%%%%%%%%%%%%%%%%%%%%%%%%%%%%%%%%%%%%%%%%%%%%%%%%%%%%%%%%%%%%%%
\subsection{Command Line Processing}
\label{sec:commandline}

The effect of redirection files can also be achieved by invoking
the \LaTeX{} compiler with a more elaborate command line.
Most conveniently this should be done as part
of a shell script or a |Makefile|.

When using \textsf{childdoc} in the main file, the following
command lines effectively perform a redirection
(note that depending on the shell being used,
backslashes may have to be doubled: `|\|' $\to$ `|\\|'):
%
\begin{center}
|... -jobname "|\textit{target}|" |\\|"|[\textit{flags}]%
|\input{childdoc.def}\childdocforward[|\textit{main}|]{|\textit{dest}|}"|
\end{center}
%
Here \textit{target} is the name of the output file,
\textit{main} is the name of the main file
and \textit{dest} is the name of the main or child file to be processed
(all filenames without extensions).
The optional argument \textit{main} can be omitted
if \textit{main} matches \textit{dest}.
Optionally, compilation \textit{flags} can be defined via |\def| commands.
This command line makes the \TeX{} engine believe
it is compiling the file \textit{target}
whose content is specified as the latter parameter.
The provided code then forwards the processing to
\textit{main} or \textit{dest} as described in \secref{sec:forward}.

%%%%%%%%%%%%%%%%%%%%%%%%%%%%%%%%%%%%%%%%%%%%%%%%%%%%%%%%%%%%%%%%%%%%%%%%%%%%%%%%
\subsection{Include by Input}
\label{sec:input}

Including child documents by |\include| has some restrictions by design.
Most notably, the content of a child document always occupies
its own set of pages; pages cannot be shared between child documents.
Usually, this behaviour makes perfect sense
because each child document contain an essential part of the document.
However, in some situations it may be desirable to compose
a document from a collection of parts
without having mandatory page breaks between then.
For this case, the package
provides a mechanism to include parts
by |\input| which can also be processed individually.
However, by construction this mechanism
requires manual handling of the content to be output.

%%%%%%%%%%%%%%%%%%%%%%%%%%%%%%%%%%%%%%%%
\DescribeMacro{\ifchilddocmanual}
The main file should be prepared as usual, see \secref{sec:include}.
However, the document body must make a distinction
between processing of an individual part and of the main document, e.g.:
%
\begin{center}
\begin{tabular}{l}
|\ifchilddocmanual|\\
|\input{\childdocname}|\\
|\||else|\\
\textit{document body with }|\input{|\textit{part}|}|\\
|\||fi|
\end{tabular}
\end{center}
%
The conditional |\ifchilddocmanual| is true whenever
a part to be included by |\input| is being compiled,
and the name of the part is stored in |\childdocname|.

%%%%%%%%%%%%%%%%%%%%%%%%%%%%%%%%%%%%%%%%
\DescribeMacro{\childdocby}
Each part to be included by |\input| should start with:
%
\begin{center}
\begin{tabular}{l}
|\input{childdoc.def}|\\
|\childdocby{|\textit{main}|}|\\
\end{tabular}
\end{center}
%
The directive |\childdocby| is similar to |\childdocof|
described in \secref{sec:include},
but the subsequent selection of content must be done manually.
To that end, both |\ifchilddoc| and |\ifchilddocmanual|
will be true upon processing of a part,
and the name of the part is stored in |\childdocname|.
Note that |\jobname| will be set to the filename of the current part
so that each part receives an individual |.aux| file
that does not interfere with the |.aux| file(s) of the main document.
This behaviour can be altered by the alternative form
|\childdocby[*]{|\textit{main}|}| (with a non-empty optional argument)
which uses the |.aux| file of the main document
by setting |\jobname| to \textit{main}.

%%%%%%%%%%%%%%%%%%%%%%%%%%%%%%%%%%%%%%%%%%%%%%%%%%%%%%%%%%%%%%%%%%%%%%%%%%%%%%%%
\subsection{Driver Development}
\label{sec:driver}

The \textsf{childdoc} mechanism can also be use for the development
of definition files such as \LaTeX{} styles or classes.
This case differs from the above setup with multiple parts
included by |\include| in that no |\includeonly| should be invoked.
This can be achieved by starting the include file
(before |\ProvidesPackage|) with:
%
\begin{center}
\begin{tabular}{l}
|\input{childdoc.def}|\\
|\childdocforward{|\textit{main}|}|\\
\end{tabular}
\end{center}
%
or alternatively with:
%
\begin{center}
\begin{tabular}{l}
|\input{childdoc.def}|\\
|\childdocby{|\textit{main}|}|\\
\end{tabular}
\end{center}
%
Both forms have slightly different effects as described above.
The main file is prepared as usual, see \secref{sec:include}.

%%%%%%%%%%%%%%%%%%%%%%%%%%%%%%%%%%%%%%%%%%%%%%%%%%%%%%%%%%%%%%%%%%%%%%%%%%%%%%%%
\subsection{Legacy Detection}
\label{sec:detection}

The directive |\childdocmain| in the main file can detect
whether the complete document or merely a child is to be compiled
even without using the directive |\childdocof|.
This method is deprecated because it is less robust
and there is no compelling reason to use it;
it is merely provided for backward compatibility
and it may be removed in future versions.

If the detection mechanism is to be used,
it is mandatory to correctly specify
the filename of the main file as the argument of |\childdocmain|:
%
\begin{center}
\begin{tabular}{l}
|\input{childdoc.def}|\\
|\childdocmain{|\textit{main}|}|\\
\end{tabular}
\end{center}
%
If |\jobname| does not match the argument \textit{main} of |\childdocmain|,
it is assumed that |\jobname| points to the child file to be compiled.
When using |\childdocmain| with the main file specified as argument,
it suffices to start a child file
with just |\input{|\textit{main}|}|
without loading of the package and using |\childdocof|.
If instead all processing is done
with the appropriate \textsf{childdoc} directives,
the argument of \textit{main} of |\childdocmain| can be empty.

An alternative version of the command line processing described
in \secref{sec:commandline} using the detection mechanism reads:
%
\begin{center}
|... -jobname "|\textit{target}|" "|[\textit{flags}]%
[|\def\jobname{|\textit{dest}|}|]|\input{|\textit{main}|}"|
\end{center}

%%%%%%%%%%%%%%%%%%%%%%%%%%%%%%%%%%%%%%%%%%%%%%%%%%%%%%%%%%%%%%%%%%%%%%%%%%%%%%%%
\subsection{Manual Code}
\label{sec:manual}

In case one cannot be certain whether the definitions file |childdoc.def|
is installed on the target \TeX{} distribution
and one prefers not to ship it,
it is conceivable to paste a few relevant commands into the sources.

To that end, drop all statements |\input{childdoc.def}|
and perform the replacements as outlined below.
Instead of |\childdocmain{|\textit{main}|}| add the following code
to the top of the main file:
%
\begin{center}
\begin{tabular}{l}
|\||ifdefined\childdocname\endinput\||fi\newif\ifchilddoc|\\
|\edef\childdocname{\scantokens\expandafter{\jobname\noexpand}}|\\
|\def\childdocmain{|\textit{main}|}\||ifx\childdocmain\childdocname\||else|\\
|\childdoctrue\includeonly{\childdocname}\let\jobname\childdocmain\||fi|\\
\end{tabular}
\end{center}
%
Instead of |\childdocof{|\textit{main}|}| just include the main file
at the top of each child file:
%
\begin{center}
|\input{|\textit{main}|}|
\end{center}
%
A simple redirection |\childdocforward{|\textit{dest}|}| is achieved by:
%
\begin{center}
|\def\jobname{|\textit{dest}|}\input{\jobname}|
\end{center}
%
The redirection with prefix
|\childdocforwardprefix[|\textit{prefix}|]{|\textit{dest}|}|
is accomplished by:
%
\begin{center}
\begin{tabular}{l}
|{\edef\jobname{\scantokens\expandafter{\jobname\noexpand}}|\\
|\def\redirectjob |\textit{prefix}|#1~~~{\gdef\jobname{|\textit{dest}|#1}}|\\
|\expandafter\redirectjob\jobname~~~}\input{\jobname}|
\end{tabular}
\end{center}

In an alternative approach,
child documents can be compiled by a specific command line
without additional code or specific definitions:
%
\begin{center}
|... -jobname "|\textit{target}|" "|[\textit{flags}]%
|\includeonly{|\textit{dest}|}\input{|\textit{main}|}"|
\end{center}
%

%%%%%%%%%%%%%%%%%%%%%%%%%%%%%%%%%%%%%%%%%%%%%%%%%%%%%%%%%%%%%%%%%%%%%%%%%%%%%%%%
%%%%%%%%%%%%%%%%%%%%%%%%%%%%%%%%%%%%%%%%%%%%%%%%%%%%%%%%%%%%%%%%%%%%%%%%%%%%%%%%
\section{Information}

%%%%%%%%%%%%%%%%%%%%%%%%%%%%%%%%%%%%%%%%%%%%%%%%%%%%%%%%%%%%%%%%%%%%%%%%%%%%%%%%
\subsection{Copyright}

Copyright \copyright{} 2017--2018 Niklas Beisert

This work may be distributed and/or modified under the
conditions of the \LaTeX{} Project Public License, either version 1.3
of this license or (at your option) any later version.
The latest version of this license is in
  \url{http://www.latex-project.org/lppl.txt}
and version 1.3 or later is part of all distributions of \LaTeX{}
version 2005/12/01 or later.

This work has the LPPL maintenance status `maintained'.

The Current Maintainer of this work is Niklas Beisert.

This work consists of the files |README.txt|, |childdoc.ins| and |childdoc.dtx|
as well as the derived files |childdoc.def|, |cdocsamp.tex|
with |cdocsch1.tex|, |cdocsch2.tex|, |cdocspt3.tex|, |cdocspt4.tex|,
|cdocsdrf.tex|, |cdocsfn1.tex|, |cdocsfn2.tex|
as well as |childdoc.pdf|.

%%%%%%%%%%%%%%%%%%%%%%%%%%%%%%%%%%%%%%%%%%%%%%%%%%%%%%%%%%%%%%%%%%%%%%%%%%%%%%%%
\subsection{Files and Installation}

The package consists of the files:
%
\begin{center}
\begin{tabular}{ll}
    |README.txt|   & readme file \\
    |childdoc.ins| & installation file \\
    |childdoc.dtx| & source file \\
    |childdoc.def| & definition file \\
    |cdocsamp.tex| & sample main file \\
    |cdocsch1.tex| & sample include file \\
    |cdocsch2.tex| & sample include file \\
    |cdocspt3.tex| & sample part file \\
    |cdocspt4.tex| & sample part file \\
    |cdocsdrf.tex| & sample redirection file \\
    |cdocsfn1.tex| & sample redirection file \\
    |cdocsfn2.tex| & sample redirection file \\
    |childdoc.pdf| & manual
\end{tabular}
\end{center}
%
The distribution consists of the files
|README.txt|, |childdoc.ins| and |childdoc.dtx|.
%
\begin{itemize}
\item
Run (pdf)\LaTeX{} on |childdoc.dtx|
to compile the manual |childdoc.pdf| (this file).
\item
Run \LaTeX{} on |childdoc.ins| to create the definitions file |childdoc.def|
and the sample |cdocsamp.tex| with include files
|cdocsch1.tex|, |cdocsch2.tex|, |cdocspt3.tex|, |cdocspt4.tex|,
|cdocsdrf.tex|, |cdocsfn1.tex|, |cdocsfn2.tex|.
Then copy the file |childdoc.def| to an appropriate directory of your \LaTeX{}
distribution, e.g.\ \textit{texmf-root}|/tex/latex/childdoc|.
\end{itemize}

%%%%%%%%%%%%%%%%%%%%%%%%%%%%%%%%%%%%%%%%%%%%%%%%%%%%%%%%%%%%%%%%%%%%%%%%%%%%%%%%
\subsection{Related CTAN Packages}

There are several other packages which offer a similar functionality:
%
\begin{itemize}
\item
The packages
\href{http://ctan.org/pkg/docmute}{\textsf{docmute}},
\href{http://ctan.org/pkg/includex}{\textsf{includex}} and
\href{http://ctan.org/pkg/standalone}{\textsf{standalone}}
provide commands to include only the document body of
a child file thus allowing both files to be compiled individually.
\item
The packages \href{http://ctan.org/pkg/subdocs}{\textsf{subdocs}}
and \href{http://ctan.org/pkg/subfiles}{\textsf{subfiles}}
provide structures in which the main and child documents can be
encapsulated and allowing them to be compiled individually.
The inclusion mechanism is different from the conventional |\include|.
\item
The package \href{http://ctan.org/pkg/combine}{\textsf{combine}}
is an elaborate solution to combine several documents into one.
\end{itemize}
%
See also the CTAN topic \href{http://ctan.org/topic/subdocs}{\textsf{subdocs}}
for further related packages.
The present package differs from the above solutions in that
a document structure constructed with the conventional |\include| mechanism
just needs two extra commands at the top of every file
such that all constituent files can be compiled individually.

%%%%%%%%%%%%%%%%%%%%%%%%%%%%%%%%%%%%%%%%%%%%%%%%%%%%%%%%%%%%%%%%%%%%%%%%%%%%%%%%
%\subsection{Feature Suggestions}
%
%The following is a list of features which may be useful for future
%versions of this package:
%%
%\begin{itemize}
%\item
%\ldots
%\end{itemize}

%%%%%%%%%%%%%%%%%%%%%%%%%%%%%%%%%%%%%%%%%%%%%%%%%%%%%%%%%%%%%%%%%%%%%%%%%%%%%%%%
\subsection{Revision History}

%%%%%%%%%%%%%%%%%%%%%%%%%%%%%%%%%%%%%%%%
\paragraph{v2.0:} 2018/12/30

\begin{itemize}
\item
immediate forward processing
\item
added |\childdocby| mechanism
\item
manual restructured
\end{itemize}

%%%%%%%%%%%%%%%%%%%%%%%%%%%%%%%%%%%%%%%%
\paragraph{v1.6:} 2018/01/17

\begin{itemize}
\item
application for development of include files
\item
corrections to manual
\end{itemize}

%%%%%%%%%%%%%%%%%%%%%%%%%%%%%%%%%%%%%%%%
\paragraph{v1.5:} 2017/05/21

\begin{itemize}
\item
more complete structuring introduced
\item
|\childdocof| introduced
\item
|\childdoc| renamed to |\childdocmain|
\item
|\childredirect| renamed to |\childdocforward| and |\childdocforwardprefix|
and functionality expanded
\end{itemize}

%%%%%%%%%%%%%%%%%%%%%%%%%%%%%%%%%%%%%%%%
\paragraph{v1.0:} 2017/04/27

\begin{itemize}
\item
manual and install package
\item
first version published on CTAN
\end{itemize}

%%%%%%%%%%%%%%%%%%%%%%%%%%%%%%%%%%%%%%%%
\paragraph{v0.6:} 2017/04/26

\begin{itemize}
\item
redirection mechanism added
\end{itemize}

%%%%%%%%%%%%%%%%%%%%%%%%%%%%%%%%%%%%%%%%
\paragraph{v0.5:} 2017/04/26

\begin{itemize}
\item
functionality in definition file
\end{itemize}


%%%%%%%%%%%%%%%%%%%%%%%%%%%%%%%%%%%%%%%%%%%%%%%%%%%%%%%%%%%%%%%%%%%%%%%%%%%%%%%%
%%%%%%%%%%%%%%%%%%%%%%%%%%%%%%%%%%%%%%%%%%%%%%%%%%%%%%%%%%%%%%%%%%%%%%%%%%%%%%%%
%%%%%%%%%%%%%%%%%%%%%%%%%%%%%%%%%%%%%%%%%%%%%%%%%%%%%%%%%%%%%%%%%%%%%%%%%%%%%%%%
\appendix

\settowidth\MacroIndent{\rmfamily\scriptsize 000\ }

 \DocInput{childdoc.dtx}

\end{document}
%</driver>
% \fi
%
% %%%%%%%%%%%%%%%%%%%%%%%%%%%%%%%%%%%%%%%%%%%%%%%%%%%%%%%%%%%%%%%%%%%%%%%%%%%%%%
% %%%%%%%%%%%%%%%%%%%%%%%%%%%%%%%%%%%%%%%%%%%%%%%%%%%%%%%%%%%%%%%%%%%%%%%%%%%%%%
% \section{Sample}
%\iffalse
%<*samplemain>
%\fi
%
% The following presents a sample document
% with two chapters, two parts, a title page,
% a compile flag as well as three forwarding files to set the flag.
% It consists of eight |.tex| files:
% \begin{center}
% \begin{tabular}{ll}
% |cdocsamp.tex|&main file\\
% |cdocsch1.tex|&include file for chapter 1\\
% |cdocsch2.tex|&include file for chapter 2\\
% |cdocspt3.tex|&include file for part 3\\
% |cdocspt4.tex|&include file for part 4\\
% |cdocsdrf.tex|&forwarding file for main file in draft mode\\
% |cdocsfi1.tex|&forwarding file for final version of chapter 1\\
% |cdocsfi2.tex|&forwarding file for final version of chapter 2\\
% \end{tabular}
% \end{center}
% Each of the eight files can be compiled directly by the \LaTeX{} compiler.
%
% %%%%%%%%%%%%%%%%%%%%%%%%%%%%%%%%%%%%%%
% \paragraph{Main File.}
%
% The main file is called |cdocsamp.tex|.
%
% Load the \textsf{childdoc} definitions and
% declare the filename for the main document:
%    \begin{macrocode}
\input{childdoc.def}
\childdocmain{}
%    \end{macrocode}

% Optional override for |\version| flag:
%    \begin{macrocode}
%%\ifchilddoc\else\providecommand{\version}{draft}\fi
%    \end{macrocode}

% Define the default values for the |\version| flag
% (|final| for the main file and |draft| for childs):
%    \begin{macrocode}
\ifchilddoc
\providecommand{\version}{draft}
\else
\providecommand{\version}{final}
\fi
%    \end{macrocode}

% Load the standard document class:
%    \begin{macrocode}
\documentclass[12pt]{article}
%    \end{macrocode}

% Start the document body:
%    \begin{macrocode}
\begin{document}
%    \end{macrocode}

% Declare a title page.
% Print title, part of document being processed and version flag:
%    \begin{macrocode}
\addtocounter{page}{-1}
\begin{center}
{\LARGE\bfseries{}childdoc example\par}
\vspace{1cm}
\ifchilddoc
\ifchilddocmanual part\else chapter\fi:
`\childdocname' of `\childdocjob'\par
\else
main document: `\childdocjob'\par
\fi
version: \version\par
\end{center}
\newpage
%    \end{macrocode}

% Manually include selected file,
% otherwise process as usual:
%    \begin{macrocode}
\ifchilddocmanual
\section*{part `\childdocname'}
\input{\childdocname}
\else
%    \end{macrocode}

% Include the two chapters:
%    \begin{macrocode}
\include{cdocsch1}
\include{cdocsch2}
%    \end{macrocode}

% Include the two parts unless only chapters should be displayed:
%    \begin{macrocode}
\ifchilddoc\else
\section{part three}
\input{cdocspt3}
\section{part four}
\input{cdocspt4}
\fi
%    \end{macrocode}

% Process as usual until here:
%    \begin{macrocode}
\fi
%    \end{macrocode}

% End of document body:
%    \begin{macrocode}
\end{document}
%    \end{macrocode}
%\iffalse
%</samplemain>
%\fi
%
% %%%%%%%%%%%%%%%%%%%%%%%%%%%%%%%%%%%%%%
% \paragraph{Chapter Include Files.}
%
% The include files are called |cdocsch1.tex| and |cdocsch2.tex|.
%
%\iffalse
%<*samplechap1|samplechap2>
%\fi

% Optional override for |\version| flag:
%    \begin{macrocode}
%%\providecommand{\version}{final}
%    \end{macrocode}

% Include the main document:
%    \begin{macrocode}
\input{childdoc.def}
\childdocof{cdocsamp}
%    \end{macrocode}

%\iffalse
%</samplechap1|samplechap2>
%\fi
%
%\iffalse
%<*samplechap1>
%\fi
% Some text for chapter 1:
%    \begin{macrocode}
\section{one}
some text in chapter one
%    \end{macrocode}

%\iffalse
%</samplechap1>
%\fi
% Some text for chapter 2:
%\iffalse
%<*samplechap2>
%\fi
%    \begin{macrocode}
\section{two}
more text in chapter two
%    \end{macrocode}

%\iffalse
%</samplechap2>
%\fi
%
% %%%%%%%%%%%%%%%%%%%%%%%%%%%%%%%%%%%%%%
% \paragraph{Part Include Files.}
%
% The include files are called |cdocspt3.tex| and |cdocspt4.tex|.
%
%\iffalse
%<*samplepart3|samplepart4>
%\fi

% Optional override for |\version| flag:
%    \begin{macrocode}
%%\providecommand{\version}{final}
%    \end{macrocode}

% Include the main document:
%    \begin{macrocode}
\input{childdoc.def}
\childdocby{cdocsamp}
%    \end{macrocode}

%\iffalse
%</samplepart3|samplepart4>
%\fi
%
%\iffalse
%<*samplepart3>
%\fi
% Some text for part 3:
%    \begin{macrocode}
some text in part three
%    \end{macrocode}

%\iffalse
%</samplepart3>
%\fi
% Some text for part 4:
%\iffalse
%<*samplepart4>
%\fi
%    \begin{macrocode}
more text in part four
%    \end{macrocode}

%\iffalse
%</samplepart4>
%\fi
%
% %%%%%%%%%%%%%%%%%%%%%%%%%%%%%%%%%%%%%%
% \paragraph{Forwarding for a Complete Draft.}
%
% The following forwarding file |cdocsdrf.tex|
% compiles the main document in draft mode:
%\iffalse
%<*sampledraft>
%\fi
%    \begin{macrocode}
\def\version{draft}
\input{childdoc.def}
\childdocforward{cdocsamp}
%    \end{macrocode}

%\iffalse
%</sampledraft>
%\fi
%
% %%%%%%%%%%%%%%%%%%%%%%%%%%%%%%%%%%%%%%
% \paragraph{Forwarding for Final Version of the Chapters.}
%
% The following forwarding files |cdocsfn1.tex| and |cdocsfn2.tex|
% (with identical content)
% compile the final versions of the child documents
% |cdocsch1.tex| and |cdocsch2.tex|, respectively:
%\iffalse
%<*samplefinal>
%\fi
%    \begin{macrocode}
\def\version{final}
\input{childdoc.def}
\childdocforwardprefix[cdocsamp]{cdocsfn}{cdocsch}
%    \end{macrocode}

%\iffalse
%</samplefinal>
%\fi
%
% %%%%%%%%%%%%%%%%%%%%%%%%%%%%%%%%%%%%%%
% \paragraph{Command Line Processing.}
%
% The following three command lines generate the output files
% |cdocscld|, |cdocscl1| and |cdocscl2|
% which should be identical to
% |cdocsdrf|, |cdocsch1| and |cdocsfn2|, respectively:
% \begin{center}
% \begin{tabular}{l}
% |latex -jobname cdocscld \|\\
% |  "\def\version{draft}\input{childdoc.def}\childdocforward{cdocsamp}"|\\
% |latex -jobname cdocscl1 \|\\
% |  "\input{childdoc.def}\childdocforward[cdocsamp]{cdocsch1}"|\\
% |latex -jobname cdocscl2 \|\\
% |  "\def\version{final}\input{childdoc.def}\childdocforward{cdocsch2}"|
% \end{tabular}
% \end{center}
% Note that the trailing backslash on each first line
% merely continues the input to the second line
% (for convenient cut ant paste).
% Furthermore, the command |latex| can be replaced by any
% of its alternative versions such as |pdflatex|.
%
% %%%%%%%%%%%%%%%%%%%%%%%%%%%%%%%%%%%%%%%%%%%%%%%%%%%%%%%%%%%%%%%%%%%%%%%%%%%%%%
% %%%%%%%%%%%%%%%%%%%%%%%%%%%%%%%%%%%%%%%%%%%%%%%%%%%%%%%%%%%%%%%%%%%%%%%%%%%%%%
% \section{Implementation}
%\iffalse
%<*package>
%\fi
%
% This section describes the definitions file |childdoc.def|.

% The definitions cannot be loaded using |\usepackage| or |\RequirePackage|
% which has a mechanism to prevent loading a style file more than once.
% When loading the definitions by means of |\input|
% multiple instances have to be prevented manually:
%\iffalse
%This code needs to be before the `\ProvidesFile' directive
%which is defined at the beginning of this file.
%Therefore it is also placed there and commented out here.
%</package>
%<*discard>
%\fi
%    \begin{macrocode}
\ifdefined\childdocmain\endinput\fi
%    \end{macrocode}
%\iffalse
%</discard>
%<*package>
%\fi
%
% \macro{\ifchilddoc}
% \macro{\ifchilddocmanual}
% The conditional |\ifchilddoc| tells whether a
% child (true) or main (false) document is being compiled.
% The conditional |\ifchilddocmanual| tells whether
% the |\includeonly| mechanism is used (false) or
% the selection of child files must be performed manually (true).
% The definitions initialise to false:
%    \begin{macrocode}
\newif\ifchilddoc
\newif\ifchilddocmanual
%    \end{macrocode}

% \macro{\childdocname}
% \macro{\childdocjob}
% The macro |\childdocname| stores the name of the main document
% to be compiled. The macro |\childdocjob| stores the name of
% the document on which the \LaTeX{} compiler was originally invoked.
% The content of |\jobname| cannot be compared
% to filenames specified in the source due to different catcodes.
% The following code rescans |\jobname|, stores the result
% in |\childdocname| and saves a copy in |\childdocjob|:
%    \begin{macrocode}
\edef\childdocname{\scantokens\expandafter{\jobname\noexpand}}
\let\childdocjob\childdocname
%    \end{macrocode}

% \macro{\childdocdisable}
% The macro |\childdocdisable| prevents the main file
% from being processed more than once.
% At this stage, the main document command |\childdocmain|
% is assumed to be called once again where it should do nothing.
% Any subsequent call to it should prevent
% a secondary processing of the main document
% It overwrites the forwarding commands
% |\childdocof| and |\childdocforward|
% with empty macros to prevent further inclusions of the main document:
%    \begin{macrocode}
\newcommand{\childdocdisable}
{
  \renewcommand{\childdocmain}[1]{\renewcommand{\childdocmain}[1]{\endinput}}
  \renewcommand{\childdocof}[1]{}
  \renewcommand{\childdocby}[2][]{}
  \renewcommand{\childdocforward}[2][]{}
  \renewcommand{\childdocdisable}{}
}
%    \end{macrocode}

% \macro{\childdocmain}
% The macro |\childdocmain| is to be called at the top of the main file
% with nothing or the main filename (without extension) as argument.
% First, it breaks loops.
% If the argument is not empty and does not match |\childdocname|
% (which is set by the first inclusion of |childdoc.def|),
% |\ifchilddoc| is set to true, |\includeonly| is applied to the child file
% and |\jobname| is set to the main file
% (for proper handling of |.aux| files):
%    \begin{macrocode}
\newcommand{\childdocmain}[1]
{
  \childdocdisable\childdocmain{}
  \if?#1?\else
    \begingroup
      \def\childdoctmp{#1}
      \ifx\childdoctmp\childdocname
        \def\childdoctmp{}
      \else
        \def\childdoctmp
        {
          \childdoctrue
          \includeonly{\childdocname}
          \def\childdocjob{#1}
          \def\jobname{#1}
        }
      \fi
      \expandafter
    \endgroup
    \childdoctmp
  \fi
}
%    \end{macrocode}

% \macro{\childdocof}
% The command |\childdocof| redirects
% compilation to the main file |#1|.
%    \begin{macrocode}
\newcommand{\childdocof}[1]
{
  \childdocdisable
  \childdoctrue
  \includeonly{\childdocname}
  \def\jobname{#1}
  \def\childdocjob{#1}
  \input{#1}
}
%    \end{macrocode}

% \macro{\childdocby}
% The command |\childdocby| ....
%    \begin{macrocode}
\newcommand{\childdocby}[2][]
{
  \childdocdisable
  \childdoctrue
  \childdocmanualtrue
  \if?#1?\else
    \def\jobname{#2}
  \fi
  \def\childdocjob{#2}
  \input{#2}
  \endinput
}
%    \end{macrocode}

% \macro{\childdocforward}
% The command |\childdocforward| redirects
% compilation to the main file or
% (if the optional argument is given) a child file.
% Parameters are set as if the main file
% or a child file starting with |\childdocof| was compiled.
% Then compilation is handed over to the main file:
%    \begin{macrocode}
\newcommand{\childdocforward}[2][]
{
  \begingroup
    \if?#1?
      \def\childdoctmp
      {
        \def\childdocname{#2}
        \def\childdocjob{#2}
        \def\jobname{#2}
        \input{#2}
        \endinput
      }
    \else
      \def\childdoctmp
      {
        \childdocdisable
        \def\childdocname{#2}
        \childdoctrue
        \includeonly{#2}
        \def\childdocjob{#1}
        \def\jobname{#1}
        \input{#1}
        \endinput
      }
    \fi
    \expandafter
  \endgroup
  \childdoctmp
}
%    \end{macrocode}

% \macro{\childdocforwardprefix}
% The command |\childdocforwardprefix| redirects
% compilation to the main or a child file by means of a pattern.
% The prefix |#1| in the current filename is replaced by |#2|
% and the suffix of the current filename is kept
% (it is assumed that the filename does not contain the substring `|~~~|'
% which is used as a delimiter).
% Compilation is handed over to the new file by |\childdocforward|:
%    \begin{macrocode}
\newcommand{\childdocforwardprefix}[3][]
{
  \begingroup
    \def\childdocextract #2##1~~~{\def\childdoctmp{\childdocforward[#1]{#3##1}}}
    \expandafter\childdocextract\childdocname~~~
    \expandafter
  \endgroup
  \childdoctmp
}
%    \end{macrocode}

% \macro{\childdoc}
% The deprecated macro |\childdoc| is a legacy version of |\childdocmain|:
%    \begin{macrocode}
\newcommand{\childdoc}{\childdocmain}
%    \end{macrocode}

% \macro{\childdocredirect}
% The deprecated macro |\childdocredirect| is a legacy version
% of |\childdocforward| and |\childdocforwardprefix|:
%    \begin{macrocode}
\newcommand{\childdocredirect}[2][]
{
  \begingroup
    \if?#1?
      \def\childdoctmp{\childdocforward{#2}}
    \else
      \def\childdoctmp{\childdocforwardprefix{#1}{#2}}
    \fi
    \expandafter
  \endgroup
  \childdoctmp
}
%    \end{macrocode}

%\iffalse
%</package>
%\fi
%
\endinput
|\\
|\childdocmain{|\textit{main}|}|\\
\end{tabular}
\end{center}
%
If |\jobname| does not match the argument \textit{main} of |\childdocmain|,
it is assumed that |\jobname| points to the child file to be compiled.
When using |\childdocmain| with the main file specified as argument,
it suffices to start a child file
with just |\input{|\textit{main}|}|
without loading of the package and using |\childdocof|.
If instead all processing is done
with the appropriate \textsf{childdoc} directives,
the argument of \textit{main} of |\childdocmain| can be empty.

An alternative version of the command line processing described
in \secref{sec:commandline} using the detection mechanism reads:
%
\begin{center}
|... -jobname "|\textit{target}|" "|[\textit{flags}]%
[|\def\jobname{|\textit{dest}|}|]|\input{|\textit{main}|}"|
\end{center}

%%%%%%%%%%%%%%%%%%%%%%%%%%%%%%%%%%%%%%%%%%%%%%%%%%%%%%%%%%%%%%%%%%%%%%%%%%%%%%%%
\subsection{Manual Code}
\label{sec:manual}

In case one cannot be certain whether the definitions file |childdoc.def|
is installed on the target \TeX{} distribution
and one prefers not to ship it,
it is conceivable to paste a few relevant commands into the sources.

To that end, drop all statements |% \iffalse
%
% childdoc.dtx Copyright (C) 2017-2018 Niklas Beisert
%
% This work may be distributed and/or modified under the
% conditions of the LaTeX Project Public License, either version 1.3
% of this license or (at your option) any later version.
% The latest version of this license is in
%   http://www.latex-project.org/lppl.txt
% and version 1.3 or later is part of all distributions of LaTeX
% version 2005/12/01 or later.
%
% This work has the LPPL maintenance status `maintained'.
%
% The Current Maintainer of this work is Niklas Beisert.
%
% This work consists of the files childdoc.dtx and childdoc.ins
% and the derived files childdoc.def and cdocsamp.tex with
% cdocsch1.tex, cdocsch2.tex, cdocsdrf.tex, cdocsfn1.tex, cdocsfn2.tex.
%
%<package>\ifdefined\childdocmain\endinput\fi
%<package>\ProvidesFile{childdoc.def}[2018/12/30 v2.0 child document driver]
%<samplemain>\ProvidesFile{cdocsamp.tex}[2018/12/30 v2.0 sample for childdoc]
%<*driver>
%\ProvidesFile{childdoc.drv}[2018/12/30 v2.0 childdoc reference manual file]
\PassOptionsToClass{10pt,a4paper}{article}
\documentclass{ltxdoc}

\usepackage[margin=35mm]{geometry}
\usepackage{hyperref}
\usepackage{hyperxmp}
\usepackage[usenames]{color}

\hypersetup{colorlinks=true}
\hypersetup{pdfstartview=FitH}
\hypersetup{pdfpagemode=UseNone}
\hypersetup{pdfsource={}}
\hypersetup{pdflang={en-UK}}
\hypersetup{pdfcopyright={Copyright 2017-2018 Niklas Beisert.
  This work may be distributed and/or modified under the
  conditions of the LaTeX Project Public License, either version 1.3
  of this license or (at your option) any later version.}}
\hypersetup{pdflicenseurl={http://www.latex-project.org/lppl.txt}}
\hypersetup{pdfcontactaddress={ETH Zurich, ITP, HIT K,
  Wolfgang-Pauli-Strasse 27}}
\hypersetup{pdfcontactpostcode={8093}}
\hypersetup{pdfcontactcity={Zurich}}
\hypersetup{pdfcontactcountry={Switzerland}}
\hypersetup{pdfcontactemail={nbeisert@itp.phys.ethz.ch}}
\hypersetup{pdfcontacturl={http://people.phys.ethz.ch/\xmptilde nbeisert/}}

\newcommand{\secref}[1]{\hyperref[#1]{section \ref*{#1}}}

\parskip1ex
\parindent0pt
\let\olditemize\itemize
\def\itemize{\olditemize\parskip0pt}

\begin{document}

\title{The \textsf{childdoc} Package}
\hypersetup{pdftitle={The childdoc Package}}
\author{Niklas Beisert\\[2ex]
  Institut f\"ur Theoretische Physik\\
  Eidgen\"ossische Technische Hochschule Z\"urich\\
  Wolfgang-Pauli-Strasse 27, 8093 Z\"urich, Switzerland\\[1ex]
  \href{mailto:nbeisert@itp.phys.ethz.ch}
  {\texttt{nbeisert@itp.phys.ethz.ch}}}
\hypersetup{pdfauthor={Niklas Beisert}}
\hypersetup{pdfsubject={Manual for the LaTeX2e Package childdoc}}
\date{30 December 2018, \textsf{v2.0}}
\maketitle

\begin{abstract}\noindent
\textsf{childdoc} is a \LaTeXe{} package
that enables the direct compilation
of document sections included by |\include|
to individual files.
\end{abstract}

\begingroup
\parskip0ex
\tableofcontents
\endgroup

%%%%%%%%%%%%%%%%%%%%%%%%%%%%%%%%%%%%%%%%%%%%%%%%%%%%%%%%%%%%%%%%%%%%%%%%%%%%%%%%
%%%%%%%%%%%%%%%%%%%%%%%%%%%%%%%%%%%%%%%%%%%%%%%%%%%%%%%%%%%%%%%%%%%%%%%%%%%%%%%%
\section{Introduction}

\LaTeX{} provides a mechanism to structure a large document (such as a book)
into a main file and several child files (containing the chapters)
using the |\include| command.
This mechanism is beneficial for documents
which span hundreds of pages in order to
make the source file(s) more manageable.
Moreover, compilation can be restricted to
selected child files by means of the |\includeonly| command.
The latter feature can be used to reduce the compilation time while editing
(this was significantly more useful in the earlier days of \LaTeX{})
or to generate a smaller document which is easier to navigate.
Another application of |\includeonly| is to generate
documents consisting of selected parts of the complete document.

However, there are a few drawbacks of the plain |\include| mechanism:
\begin{itemize}
\item
The child files cannot be compiled on their own,
they can only be compiled via the main file.
A naive editing environment
(such as a text editor with an option
to have the current file processed by \LaTeX)
may require one to switch to the main file before compiling;
attempting to compile the child file produces errors.
\item
The main file must be modified (each time)
to adjust the |\includeonly| command
to the present needs. This easily leaves the main file in a messy state.
\item
The generated document will always carry the filename
of the main document. This is inconvenient if
several child files are to be compiled and
to be kept for distribution.
\end{itemize}

The present package provides a simple interface
to make child files individually compilable by \LaTeX{}.
Compiling a child file then has the same effect as compiling
the main file with an |\includeonly| command
to select the appropriate child.
Moreover the generated document will carry the name of the child
rather than the main file.
This resolves all three above issues.

This feature is meant to make the editing of books,
thesis documents and lecture notes somewhat more convenient.
However, the package can also be used efficiently for
composing a series of documents (such as exercise sheets)
which are typically distributed individually.
It then assists the author in generating the individual documents
(potentially in different versions)
as well as a document containing the collected series.
Another application is in developing style files
or other kinds of included material
where compilation of the style file could redirect
to a sample or test file.

%%%%%%%%%%%%%%%%%%%%%%%%%%%%%%%%%%%%%%%%%%%%%%%%%%%%%%%%%%%%%%%%%%%%%%%%%%%%%%%%
%%%%%%%%%%%%%%%%%%%%%%%%%%%%%%%%%%%%%%%%%%%%%%%%%%%%%%%%%%%%%%%%%%%%%%%%%%%%%%%%
\section{Usage}

First of all, the package \textsf{childdoc} is \emph{not} a standard
\LaTeXe{} |.sty| style file! Therefore it needs to be invoked in
a non-standard way.

%%%%%%%%%%%%%%%%%%%%%%%%%%%%%%%%%%%%%%%%%%%%%%%%%%%%%%%%%%%%%%%%%%%%%%%%%%%%%%%%
\subsection{Included Files}
\label{sec:include}

%%%%%%%%%%%%%%%%%%%%%%%%%%%%%%%%%%%%%%%%
\DescribeMacro{\childdocmain}
To use the package, add the commands
\begin{center}
\begin{tabular}{l}
|\input{childdoc.def}|\\
|\childdocmain{}|\\
\end{tabular}
\end{center}
at the very top of the main \LaTeX{} file,
in particular \emph{before} the |\documentclass| statement!
The argument of |\childdocmain| should be left empty
(but it must be present).

%%%%%%%%%%%%%%%%%%%%%%%%%%%%%%%%%%%%%%%%
\DescribeMacro{\childdocof}
Furthermore, add the commands
\begin{center}
\begin{tabular}{l}
|\input{childdoc.def}|\\
|\childdocof{|\textit{main}|}|\\
\end{tabular}
\end{center}
at the top of every child file \textit{child}
which is included by |\include{|\textit{child}|}|
from within the main file
(or at least for those files to be compiled individually).
The argument \textit{main} must be the filename of the main file.

There are a couple of
considerations in setting up the main and child documents:

%%%%%%%%%%%%%%%%%%%%%%%%%%%%%%%%%%%%%%%%
\paragraph{Restrictions.}

Please note the following restrictions:
\begin{itemize}
\item
|\childdocmain| must be called with one argument \textit{main}
to ensure compatibility with earlier version of the package.
It must either be empty (|\childdocmain{}|)
or precisely match the filename of the main file in which it is specified.
See \secref{sec:detection} for further information.
\item
The filename \textit{main} must be specified without the |.tex| extension.
\item
The filename \textit{main} is case sensitive
(even in case-insensitive file systems)
due to internal string comparison.
\item
The argument \textit{main} should be fully expanded, it cannot be a macro.
\item
Subdirectories and special characters should be avoided in filenames.
\item
The command |\childdocmain{|\textit{main}|}| must be followed by a whitespace.
It should not be followed immediately by another command
or by a comment mark `|%|'.
This is because the \TeX{} parser reads the token immediately following
the argument of |\childdocmain| and puts it
at the beginning of every child section;
however, a white\-space is ignored.
\end{itemize}

%%%%%%%%%%%%%%%%%%%%%%%%%%%%%%%%%%%%%%%%
\paragraph{Content of Main File.}

It is advisable to place all content in the child files included by |\include|.
Any output contained in the main file will appear in all child documents
unless suppressed manually;
it cannot be suppressed automatically by the |\includeonly| directive
and thus should normally be avoided.
A method to include some content in the main file
by means of conditional processing is described in \secref{sec:conditional}.

%%%%%%%%%%%%%%%%%%%%%%%%%%%%%%%%%%%%%%%%
\paragraph{Page Numbering.}

When only a part of the document is compiled,
the appropriate numbering of pages
(as well as other status parameters)
is determined from the |.aux| files.
The latter contain information from previous passes.
However this information needs to propagate through
all intermediate child documents.
Therefore the page numbering in child documents may well
be inconsistent until the complete document is compiled at least once.

A useful (if unconventional) way to always ensure a consistent
page numbering is to restart the numbering in each child document
and denote the pages by `\textit{child}|.|\textit{page}'
where \textit{child} represents the chapter/section number of the child file.
This can be achieved by the command
|\numberwithin{page}{|\textit{child}|}|
of the \textsf{amsmath} package
where \textit{child} can be |chapter| or |section|
depending on the chosen structuring.
Alternatively, one can modify the macro |\thepage| appropriately
and reset the counter |page| at the start of each child file.

%%%%%%%%%%%%%%%%%%%%%%%%%%%%%%%%%%%%%%%%%%%%%%%%%%%%%%%%%%%%%%%%%%%%%%%%%%%%%%%%
\subsection{Conditional Processing}
\label{sec:conditional}

The package provides a mechanism to compile different versions
of a document. To customise the versions further some conditional processing
can come in handy to distinguish which version is being compiled.
The package provides two macros to describe the compilation context:

%%%%%%%%%%%%%%%%%%%%%%%%%%%%%%%%%%%%%%%%
\DescribeMacro{\ifchilddoc}
The conditional |\ifchilddoc| distinguishes between the compilation of
child documents and the main document:
%
\begin{center}
|\ifchilddoc |\textit{child-code}| |[|\||else |\textit{main-code}]| \||fi|
\end{center}

%%%%%%%%%%%%%%%%%%%%%%%%%%%%%%%%%%%%%%%%
\DescribeMacro{\childdocname}
\DescribeMacro{\childdocjob}
The macro |\childdocname| contains the filename (without extension)
of the main or child file being processed.
Note that |\childdocjob| will always contain the name of the main file.

%%%%%%%%%%%%%%%%%%%%%%%%%%%%%%%%%%%%%%%%
\paragraph{Title Page.}

Conditional processing can be used to include a title or banner page
in the main document when proper precautions are taken.
Importantly, the code in the main file should ensure that the page counter
(as well as other status parameters which are stored in the |.aux| files)
takes the same value after the conditional processing.
Otherwise the page numbers may take divergent values
depending on which part is compiled.

For example, a title page could be declared by:
%
\begin{center}
\begin{tabular}{l}
|\ifchilddoc\||else|\\
|\addtocounter{page}{-1}|\\
\textit{code for title page}\\
|\newpage|\\
|\||fi|
\end{tabular}
\end{center}
%
A banner page for the child documents can be generated by:
%
\begin{center}
\begin{tabular}{l}
|\ifchilddoc|\\
|\addtocounter{page}{-1}|\\
\textit{code for banner page}\\
|\newpage|\\
|\||fi|
\end{tabular}
\end{center}
%
Here one could write a message such as:
\begin{center}
|This is the part \childdocname{} of \childdocjob{}.|
\end{center}

%%%%%%%%%%%%%%%%%%%%%%%%%%%%%%%%%%%%%%%%%%%%%%%%%%%%%%%%%%%%%%%%%%%%%%%%%%%%%%%%
\subsection{Flags}
\label{sec:flags}

The package makes it easy to generate different versions
of the main or child documents.
To this end compilation flags can be defined
and assigned different default values.
They will be particularly useful in conjunction
with the forwarding mechanism described in \secref{sec:forward}.

For example, it may be useful to have a flag |\version|
which can be set to |draft| or |final|.
The document source will contain some conditional code
depending on the value of |\version|.
Suppose further, the flag should default to |final| for the main file
and to |draft| for child files
which is a natural assignment for editing the document.
This is achieved by placing the following code
in the preamble of the main document
(below the |\childdocmain| directive):
%
\begin{center}
\begin{tabular}{l}
|\ifchilddoc|\\
|\providecommand{\version}{draft}|\\
|\||else|\\
|\providecommand{\version}{final}|\\
|\||fi|
\end{tabular}
\end{center}
%
The definition by |\providecommand| makes sure
that previous definitions are not overwritten.
Further statements |\providecommand{\version}{...}|
can thus be added before the above code to override it.

For the main file, one might add a line
(between |\childdocmain| and the above block)
%
\begin{center}
|%\ifchilddoc\||else\providecommand{\version}{draft}\||fi|
\end{center}
%
which can be uncommented to produce a draft version.
Likewise one can add a line to the very top of a child file
(above the |\childdocof{|\textit{main}|}| directive)
%
\begin{center}
|%\providecommand{\version}{final}|
\end{center}
%
which can be uncommented to produce the final version of this child document.

%%%%%%%%%%%%%%%%%%%%%%%%%%%%%%%%%%%%%%%%%%%%%%%%%%%%%%%%%%%%%%%%%%%%%%%%%%%%%%%%
\subsection{Forwarding}
\label{sec:forward}

Different versions of the main or child documents
using compilation flags as described in \secref{sec:flags}
can be (permanently) stored in different files
for convenient compilation, viewing and distribution.
To this end, the package defines a command
to pass on compilation to a different file:

%%%%%%%%%%%%%%%%%%%%%%%%%%%%%%%%%%%%%%%%
\DescribeMacro{\childdocforward}
The command |\childdocforward| redirects processing to
another source file:
%
\begin{center}
\begin{tabular}{l}
|\input{childdoc.def}|\\
|\childdocforward[|\textit{main}|]{|\textit{dest}|}|\\
\end{tabular}
\end{center}
%
The argument \textit{dest} is the destination file
(without extension).
It should be the main file or one of the child files.
Note that further \textsf{childdoc} directives
such as |\childdocof| and |\childdocforward|
in the indicated file will be processed in this form.
The optional argument \textit{main}
passes on directly to the main file \textit{main}
while pretending to compile the child \textit{dest}.
This form behaves as if \textit{dest}
issues |\childdocof{|\textit{main}|}| right away,
and no further \textsf{childdoc} directives will be processed.

%%%%%%%%%%%%%%%%%%%%%%%%%%%%%%%%%%%%%%%%
\DescribeMacro{\...prefix}
In the alternative form |\childdocforwardprefix|,
%
\begin{center}
\begin{tabular}{l}
|\input{childdoc.def}|\\
|\childdocforwardprefix[|\textit{main}|]{|\textit{prefix}|}{|\textit{dest}|}|
\end{tabular}
\end{center}
%
the destination file is determined by a pattern
depending on the current file:
To make this work, the current file must be called
`{\textit{prefix}\hspace{0.2em}\textit{suffix}}'
with \textit{prefix} matching precisely the argument.
Processing is then passed on to the file
`{\textit{dest}\hspace{0.2em}\textit{suffix}}'.
Surely, the same effect is achieved by
directly specifying the
argument `{\textit{dest}\hspace{0.2em}\textit{suffix}}'
in the first form.
However, that requires to set up a different file
for each child. With the alternative form of the command
all these files can have exactly the same content
which simplifies setting them up and maintaining them.

For example, the following file |draft.tex|
with a compilation flag |\version| as described in \secref{sec:flags}
compiles the main document as a draft:
%
\begin{center}
\begin{tabular}{l}
|\def\version{draft}|\\
|\input{childdoc.def}|\\
|\childdocforward{|\textit{main}|}|
\end{tabular}
\end{center}
%
Likewise, the following files |final|\textit{nn}|.tex|
compile the final version of the child document
|child|\textit{nn}|.tex|:
%
\begin{center}
\begin{tabular}{l}
|\def\version{final}|\\
|\input{childdoc.def}|\\
|\childdocforwardprefix{final}{child}|
\end{tabular}
\end{center}
%

Note that when several versions of a main file and/or of each child file
are to be generated, it may be convenient to set up a |Makefile| or
shell script to automatise the process.

%%%%%%%%%%%%%%%%%%%%%%%%%%%%%%%%%%%%%%%%%%%%%%%%%%%%%%%%%%%%%%%%%%%%%%%%%%%%%%%%
\subsection{Command Line Processing}
\label{sec:commandline}

The effect of redirection files can also be achieved by invoking
the \LaTeX{} compiler with a more elaborate command line.
Most conveniently this should be done as part
of a shell script or a |Makefile|.

When using \textsf{childdoc} in the main file, the following
command lines effectively perform a redirection
(note that depending on the shell being used,
backslashes may have to be doubled: `|\|' $\to$ `|\\|'):
%
\begin{center}
|... -jobname "|\textit{target}|" |\\|"|[\textit{flags}]%
|\input{childdoc.def}\childdocforward[|\textit{main}|]{|\textit{dest}|}"|
\end{center}
%
Here \textit{target} is the name of the output file,
\textit{main} is the name of the main file
and \textit{dest} is the name of the main or child file to be processed
(all filenames without extensions).
The optional argument \textit{main} can be omitted
if \textit{main} matches \textit{dest}.
Optionally, compilation \textit{flags} can be defined via |\def| commands.
This command line makes the \TeX{} engine believe
it is compiling the file \textit{target}
whose content is specified as the latter parameter.
The provided code then forwards the processing to
\textit{main} or \textit{dest} as described in \secref{sec:forward}.

%%%%%%%%%%%%%%%%%%%%%%%%%%%%%%%%%%%%%%%%%%%%%%%%%%%%%%%%%%%%%%%%%%%%%%%%%%%%%%%%
\subsection{Include by Input}
\label{sec:input}

Including child documents by |\include| has some restrictions by design.
Most notably, the content of a child document always occupies
its own set of pages; pages cannot be shared between child documents.
Usually, this behaviour makes perfect sense
because each child document contain an essential part of the document.
However, in some situations it may be desirable to compose
a document from a collection of parts
without having mandatory page breaks between then.
For this case, the package
provides a mechanism to include parts
by |\input| which can also be processed individually.
However, by construction this mechanism
requires manual handling of the content to be output.

%%%%%%%%%%%%%%%%%%%%%%%%%%%%%%%%%%%%%%%%
\DescribeMacro{\ifchilddocmanual}
The main file should be prepared as usual, see \secref{sec:include}.
However, the document body must make a distinction
between processing of an individual part and of the main document, e.g.:
%
\begin{center}
\begin{tabular}{l}
|\ifchilddocmanual|\\
|\input{\childdocname}|\\
|\||else|\\
\textit{document body with }|\input{|\textit{part}|}|\\
|\||fi|
\end{tabular}
\end{center}
%
The conditional |\ifchilddocmanual| is true whenever
a part to be included by |\input| is being compiled,
and the name of the part is stored in |\childdocname|.

%%%%%%%%%%%%%%%%%%%%%%%%%%%%%%%%%%%%%%%%
\DescribeMacro{\childdocby}
Each part to be included by |\input| should start with:
%
\begin{center}
\begin{tabular}{l}
|\input{childdoc.def}|\\
|\childdocby{|\textit{main}|}|\\
\end{tabular}
\end{center}
%
The directive |\childdocby| is similar to |\childdocof|
described in \secref{sec:include},
but the subsequent selection of content must be done manually.
To that end, both |\ifchilddoc| and |\ifchilddocmanual|
will be true upon processing of a part,
and the name of the part is stored in |\childdocname|.
Note that |\jobname| will be set to the filename of the current part
so that each part receives an individual |.aux| file
that does not interfere with the |.aux| file(s) of the main document.
This behaviour can be altered by the alternative form
|\childdocby[*]{|\textit{main}|}| (with a non-empty optional argument)
which uses the |.aux| file of the main document
by setting |\jobname| to \textit{main}.

%%%%%%%%%%%%%%%%%%%%%%%%%%%%%%%%%%%%%%%%%%%%%%%%%%%%%%%%%%%%%%%%%%%%%%%%%%%%%%%%
\subsection{Driver Development}
\label{sec:driver}

The \textsf{childdoc} mechanism can also be use for the development
of definition files such as \LaTeX{} styles or classes.
This case differs from the above setup with multiple parts
included by |\include| in that no |\includeonly| should be invoked.
This can be achieved by starting the include file
(before |\ProvidesPackage|) with:
%
\begin{center}
\begin{tabular}{l}
|\input{childdoc.def}|\\
|\childdocforward{|\textit{main}|}|\\
\end{tabular}
\end{center}
%
or alternatively with:
%
\begin{center}
\begin{tabular}{l}
|\input{childdoc.def}|\\
|\childdocby{|\textit{main}|}|\\
\end{tabular}
\end{center}
%
Both forms have slightly different effects as described above.
The main file is prepared as usual, see \secref{sec:include}.

%%%%%%%%%%%%%%%%%%%%%%%%%%%%%%%%%%%%%%%%%%%%%%%%%%%%%%%%%%%%%%%%%%%%%%%%%%%%%%%%
\subsection{Legacy Detection}
\label{sec:detection}

The directive |\childdocmain| in the main file can detect
whether the complete document or merely a child is to be compiled
even without using the directive |\childdocof|.
This method is deprecated because it is less robust
and there is no compelling reason to use it;
it is merely provided for backward compatibility
and it may be removed in future versions.

If the detection mechanism is to be used,
it is mandatory to correctly specify
the filename of the main file as the argument of |\childdocmain|:
%
\begin{center}
\begin{tabular}{l}
|\input{childdoc.def}|\\
|\childdocmain{|\textit{main}|}|\\
\end{tabular}
\end{center}
%
If |\jobname| does not match the argument \textit{main} of |\childdocmain|,
it is assumed that |\jobname| points to the child file to be compiled.
When using |\childdocmain| with the main file specified as argument,
it suffices to start a child file
with just |\input{|\textit{main}|}|
without loading of the package and using |\childdocof|.
If instead all processing is done
with the appropriate \textsf{childdoc} directives,
the argument of \textit{main} of |\childdocmain| can be empty.

An alternative version of the command line processing described
in \secref{sec:commandline} using the detection mechanism reads:
%
\begin{center}
|... -jobname "|\textit{target}|" "|[\textit{flags}]%
[|\def\jobname{|\textit{dest}|}|]|\input{|\textit{main}|}"|
\end{center}

%%%%%%%%%%%%%%%%%%%%%%%%%%%%%%%%%%%%%%%%%%%%%%%%%%%%%%%%%%%%%%%%%%%%%%%%%%%%%%%%
\subsection{Manual Code}
\label{sec:manual}

In case one cannot be certain whether the definitions file |childdoc.def|
is installed on the target \TeX{} distribution
and one prefers not to ship it,
it is conceivable to paste a few relevant commands into the sources.

To that end, drop all statements |\input{childdoc.def}|
and perform the replacements as outlined below.
Instead of |\childdocmain{|\textit{main}|}| add the following code
to the top of the main file:
%
\begin{center}
\begin{tabular}{l}
|\||ifdefined\childdocname\endinput\||fi\newif\ifchilddoc|\\
|\edef\childdocname{\scantokens\expandafter{\jobname\noexpand}}|\\
|\def\childdocmain{|\textit{main}|}\||ifx\childdocmain\childdocname\||else|\\
|\childdoctrue\includeonly{\childdocname}\let\jobname\childdocmain\||fi|\\
\end{tabular}
\end{center}
%
Instead of |\childdocof{|\textit{main}|}| just include the main file
at the top of each child file:
%
\begin{center}
|\input{|\textit{main}|}|
\end{center}
%
A simple redirection |\childdocforward{|\textit{dest}|}| is achieved by:
%
\begin{center}
|\def\jobname{|\textit{dest}|}\input{\jobname}|
\end{center}
%
The redirection with prefix
|\childdocforwardprefix[|\textit{prefix}|]{|\textit{dest}|}|
is accomplished by:
%
\begin{center}
\begin{tabular}{l}
|{\edef\jobname{\scantokens\expandafter{\jobname\noexpand}}|\\
|\def\redirectjob |\textit{prefix}|#1~~~{\gdef\jobname{|\textit{dest}|#1}}|\\
|\expandafter\redirectjob\jobname~~~}\input{\jobname}|
\end{tabular}
\end{center}

In an alternative approach,
child documents can be compiled by a specific command line
without additional code or specific definitions:
%
\begin{center}
|... -jobname "|\textit{target}|" "|[\textit{flags}]%
|\includeonly{|\textit{dest}|}\input{|\textit{main}|}"|
\end{center}
%

%%%%%%%%%%%%%%%%%%%%%%%%%%%%%%%%%%%%%%%%%%%%%%%%%%%%%%%%%%%%%%%%%%%%%%%%%%%%%%%%
%%%%%%%%%%%%%%%%%%%%%%%%%%%%%%%%%%%%%%%%%%%%%%%%%%%%%%%%%%%%%%%%%%%%%%%%%%%%%%%%
\section{Information}

%%%%%%%%%%%%%%%%%%%%%%%%%%%%%%%%%%%%%%%%%%%%%%%%%%%%%%%%%%%%%%%%%%%%%%%%%%%%%%%%
\subsection{Copyright}

Copyright \copyright{} 2017--2018 Niklas Beisert

This work may be distributed and/or modified under the
conditions of the \LaTeX{} Project Public License, either version 1.3
of this license or (at your option) any later version.
The latest version of this license is in
  \url{http://www.latex-project.org/lppl.txt}
and version 1.3 or later is part of all distributions of \LaTeX{}
version 2005/12/01 or later.

This work has the LPPL maintenance status `maintained'.

The Current Maintainer of this work is Niklas Beisert.

This work consists of the files |README.txt|, |childdoc.ins| and |childdoc.dtx|
as well as the derived files |childdoc.def|, |cdocsamp.tex|
with |cdocsch1.tex|, |cdocsch2.tex|, |cdocspt3.tex|, |cdocspt4.tex|,
|cdocsdrf.tex|, |cdocsfn1.tex|, |cdocsfn2.tex|
as well as |childdoc.pdf|.

%%%%%%%%%%%%%%%%%%%%%%%%%%%%%%%%%%%%%%%%%%%%%%%%%%%%%%%%%%%%%%%%%%%%%%%%%%%%%%%%
\subsection{Files and Installation}

The package consists of the files:
%
\begin{center}
\begin{tabular}{ll}
    |README.txt|   & readme file \\
    |childdoc.ins| & installation file \\
    |childdoc.dtx| & source file \\
    |childdoc.def| & definition file \\
    |cdocsamp.tex| & sample main file \\
    |cdocsch1.tex| & sample include file \\
    |cdocsch2.tex| & sample include file \\
    |cdocspt3.tex| & sample part file \\
    |cdocspt4.tex| & sample part file \\
    |cdocsdrf.tex| & sample redirection file \\
    |cdocsfn1.tex| & sample redirection file \\
    |cdocsfn2.tex| & sample redirection file \\
    |childdoc.pdf| & manual
\end{tabular}
\end{center}
%
The distribution consists of the files
|README.txt|, |childdoc.ins| and |childdoc.dtx|.
%
\begin{itemize}
\item
Run (pdf)\LaTeX{} on |childdoc.dtx|
to compile the manual |childdoc.pdf| (this file).
\item
Run \LaTeX{} on |childdoc.ins| to create the definitions file |childdoc.def|
and the sample |cdocsamp.tex| with include files
|cdocsch1.tex|, |cdocsch2.tex|, |cdocspt3.tex|, |cdocspt4.tex|,
|cdocsdrf.tex|, |cdocsfn1.tex|, |cdocsfn2.tex|.
Then copy the file |childdoc.def| to an appropriate directory of your \LaTeX{}
distribution, e.g.\ \textit{texmf-root}|/tex/latex/childdoc|.
\end{itemize}

%%%%%%%%%%%%%%%%%%%%%%%%%%%%%%%%%%%%%%%%%%%%%%%%%%%%%%%%%%%%%%%%%%%%%%%%%%%%%%%%
\subsection{Related CTAN Packages}

There are several other packages which offer a similar functionality:
%
\begin{itemize}
\item
The packages
\href{http://ctan.org/pkg/docmute}{\textsf{docmute}},
\href{http://ctan.org/pkg/includex}{\textsf{includex}} and
\href{http://ctan.org/pkg/standalone}{\textsf{standalone}}
provide commands to include only the document body of
a child file thus allowing both files to be compiled individually.
\item
The packages \href{http://ctan.org/pkg/subdocs}{\textsf{subdocs}}
and \href{http://ctan.org/pkg/subfiles}{\textsf{subfiles}}
provide structures in which the main and child documents can be
encapsulated and allowing them to be compiled individually.
The inclusion mechanism is different from the conventional |\include|.
\item
The package \href{http://ctan.org/pkg/combine}{\textsf{combine}}
is an elaborate solution to combine several documents into one.
\end{itemize}
%
See also the CTAN topic \href{http://ctan.org/topic/subdocs}{\textsf{subdocs}}
for further related packages.
The present package differs from the above solutions in that
a document structure constructed with the conventional |\include| mechanism
just needs two extra commands at the top of every file
such that all constituent files can be compiled individually.

%%%%%%%%%%%%%%%%%%%%%%%%%%%%%%%%%%%%%%%%%%%%%%%%%%%%%%%%%%%%%%%%%%%%%%%%%%%%%%%%
%\subsection{Feature Suggestions}
%
%The following is a list of features which may be useful for future
%versions of this package:
%%
%\begin{itemize}
%\item
%\ldots
%\end{itemize}

%%%%%%%%%%%%%%%%%%%%%%%%%%%%%%%%%%%%%%%%%%%%%%%%%%%%%%%%%%%%%%%%%%%%%%%%%%%%%%%%
\subsection{Revision History}

%%%%%%%%%%%%%%%%%%%%%%%%%%%%%%%%%%%%%%%%
\paragraph{v2.0:} 2018/12/30

\begin{itemize}
\item
immediate forward processing
\item
added |\childdocby| mechanism
\item
manual restructured
\end{itemize}

%%%%%%%%%%%%%%%%%%%%%%%%%%%%%%%%%%%%%%%%
\paragraph{v1.6:} 2018/01/17

\begin{itemize}
\item
application for development of include files
\item
corrections to manual
\end{itemize}

%%%%%%%%%%%%%%%%%%%%%%%%%%%%%%%%%%%%%%%%
\paragraph{v1.5:} 2017/05/21

\begin{itemize}
\item
more complete structuring introduced
\item
|\childdocof| introduced
\item
|\childdoc| renamed to |\childdocmain|
\item
|\childredirect| renamed to |\childdocforward| and |\childdocforwardprefix|
and functionality expanded
\end{itemize}

%%%%%%%%%%%%%%%%%%%%%%%%%%%%%%%%%%%%%%%%
\paragraph{v1.0:} 2017/04/27

\begin{itemize}
\item
manual and install package
\item
first version published on CTAN
\end{itemize}

%%%%%%%%%%%%%%%%%%%%%%%%%%%%%%%%%%%%%%%%
\paragraph{v0.6:} 2017/04/26

\begin{itemize}
\item
redirection mechanism added
\end{itemize}

%%%%%%%%%%%%%%%%%%%%%%%%%%%%%%%%%%%%%%%%
\paragraph{v0.5:} 2017/04/26

\begin{itemize}
\item
functionality in definition file
\end{itemize}


%%%%%%%%%%%%%%%%%%%%%%%%%%%%%%%%%%%%%%%%%%%%%%%%%%%%%%%%%%%%%%%%%%%%%%%%%%%%%%%%
%%%%%%%%%%%%%%%%%%%%%%%%%%%%%%%%%%%%%%%%%%%%%%%%%%%%%%%%%%%%%%%%%%%%%%%%%%%%%%%%
%%%%%%%%%%%%%%%%%%%%%%%%%%%%%%%%%%%%%%%%%%%%%%%%%%%%%%%%%%%%%%%%%%%%%%%%%%%%%%%%
\appendix

\settowidth\MacroIndent{\rmfamily\scriptsize 000\ }

 \DocInput{childdoc.dtx}

\end{document}
%</driver>
% \fi
%
% %%%%%%%%%%%%%%%%%%%%%%%%%%%%%%%%%%%%%%%%%%%%%%%%%%%%%%%%%%%%%%%%%%%%%%%%%%%%%%
% %%%%%%%%%%%%%%%%%%%%%%%%%%%%%%%%%%%%%%%%%%%%%%%%%%%%%%%%%%%%%%%%%%%%%%%%%%%%%%
% \section{Sample}
%\iffalse
%<*samplemain>
%\fi
%
% The following presents a sample document
% with two chapters, two parts, a title page,
% a compile flag as well as three forwarding files to set the flag.
% It consists of eight |.tex| files:
% \begin{center}
% \begin{tabular}{ll}
% |cdocsamp.tex|&main file\\
% |cdocsch1.tex|&include file for chapter 1\\
% |cdocsch2.tex|&include file for chapter 2\\
% |cdocspt3.tex|&include file for part 3\\
% |cdocspt4.tex|&include file for part 4\\
% |cdocsdrf.tex|&forwarding file for main file in draft mode\\
% |cdocsfi1.tex|&forwarding file for final version of chapter 1\\
% |cdocsfi2.tex|&forwarding file for final version of chapter 2\\
% \end{tabular}
% \end{center}
% Each of the eight files can be compiled directly by the \LaTeX{} compiler.
%
% %%%%%%%%%%%%%%%%%%%%%%%%%%%%%%%%%%%%%%
% \paragraph{Main File.}
%
% The main file is called |cdocsamp.tex|.
%
% Load the \textsf{childdoc} definitions and
% declare the filename for the main document:
%    \begin{macrocode}
\input{childdoc.def}
\childdocmain{}
%    \end{macrocode}

% Optional override for |\version| flag:
%    \begin{macrocode}
%%\ifchilddoc\else\providecommand{\version}{draft}\fi
%    \end{macrocode}

% Define the default values for the |\version| flag
% (|final| for the main file and |draft| for childs):
%    \begin{macrocode}
\ifchilddoc
\providecommand{\version}{draft}
\else
\providecommand{\version}{final}
\fi
%    \end{macrocode}

% Load the standard document class:
%    \begin{macrocode}
\documentclass[12pt]{article}
%    \end{macrocode}

% Start the document body:
%    \begin{macrocode}
\begin{document}
%    \end{macrocode}

% Declare a title page.
% Print title, part of document being processed and version flag:
%    \begin{macrocode}
\addtocounter{page}{-1}
\begin{center}
{\LARGE\bfseries{}childdoc example\par}
\vspace{1cm}
\ifchilddoc
\ifchilddocmanual part\else chapter\fi:
`\childdocname' of `\childdocjob'\par
\else
main document: `\childdocjob'\par
\fi
version: \version\par
\end{center}
\newpage
%    \end{macrocode}

% Manually include selected file,
% otherwise process as usual:
%    \begin{macrocode}
\ifchilddocmanual
\section*{part `\childdocname'}
\input{\childdocname}
\else
%    \end{macrocode}

% Include the two chapters:
%    \begin{macrocode}
\include{cdocsch1}
\include{cdocsch2}
%    \end{macrocode}

% Include the two parts unless only chapters should be displayed:
%    \begin{macrocode}
\ifchilddoc\else
\section{part three}
\input{cdocspt3}
\section{part four}
\input{cdocspt4}
\fi
%    \end{macrocode}

% Process as usual until here:
%    \begin{macrocode}
\fi
%    \end{macrocode}

% End of document body:
%    \begin{macrocode}
\end{document}
%    \end{macrocode}
%\iffalse
%</samplemain>
%\fi
%
% %%%%%%%%%%%%%%%%%%%%%%%%%%%%%%%%%%%%%%
% \paragraph{Chapter Include Files.}
%
% The include files are called |cdocsch1.tex| and |cdocsch2.tex|.
%
%\iffalse
%<*samplechap1|samplechap2>
%\fi

% Optional override for |\version| flag:
%    \begin{macrocode}
%%\providecommand{\version}{final}
%    \end{macrocode}

% Include the main document:
%    \begin{macrocode}
\input{childdoc.def}
\childdocof{cdocsamp}
%    \end{macrocode}

%\iffalse
%</samplechap1|samplechap2>
%\fi
%
%\iffalse
%<*samplechap1>
%\fi
% Some text for chapter 1:
%    \begin{macrocode}
\section{one}
some text in chapter one
%    \end{macrocode}

%\iffalse
%</samplechap1>
%\fi
% Some text for chapter 2:
%\iffalse
%<*samplechap2>
%\fi
%    \begin{macrocode}
\section{two}
more text in chapter two
%    \end{macrocode}

%\iffalse
%</samplechap2>
%\fi
%
% %%%%%%%%%%%%%%%%%%%%%%%%%%%%%%%%%%%%%%
% \paragraph{Part Include Files.}
%
% The include files are called |cdocspt3.tex| and |cdocspt4.tex|.
%
%\iffalse
%<*samplepart3|samplepart4>
%\fi

% Optional override for |\version| flag:
%    \begin{macrocode}
%%\providecommand{\version}{final}
%    \end{macrocode}

% Include the main document:
%    \begin{macrocode}
\input{childdoc.def}
\childdocby{cdocsamp}
%    \end{macrocode}

%\iffalse
%</samplepart3|samplepart4>
%\fi
%
%\iffalse
%<*samplepart3>
%\fi
% Some text for part 3:
%    \begin{macrocode}
some text in part three
%    \end{macrocode}

%\iffalse
%</samplepart3>
%\fi
% Some text for part 4:
%\iffalse
%<*samplepart4>
%\fi
%    \begin{macrocode}
more text in part four
%    \end{macrocode}

%\iffalse
%</samplepart4>
%\fi
%
% %%%%%%%%%%%%%%%%%%%%%%%%%%%%%%%%%%%%%%
% \paragraph{Forwarding for a Complete Draft.}
%
% The following forwarding file |cdocsdrf.tex|
% compiles the main document in draft mode:
%\iffalse
%<*sampledraft>
%\fi
%    \begin{macrocode}
\def\version{draft}
\input{childdoc.def}
\childdocforward{cdocsamp}
%    \end{macrocode}

%\iffalse
%</sampledraft>
%\fi
%
% %%%%%%%%%%%%%%%%%%%%%%%%%%%%%%%%%%%%%%
% \paragraph{Forwarding for Final Version of the Chapters.}
%
% The following forwarding files |cdocsfn1.tex| and |cdocsfn2.tex|
% (with identical content)
% compile the final versions of the child documents
% |cdocsch1.tex| and |cdocsch2.tex|, respectively:
%\iffalse
%<*samplefinal>
%\fi
%    \begin{macrocode}
\def\version{final}
\input{childdoc.def}
\childdocforwardprefix[cdocsamp]{cdocsfn}{cdocsch}
%    \end{macrocode}

%\iffalse
%</samplefinal>
%\fi
%
% %%%%%%%%%%%%%%%%%%%%%%%%%%%%%%%%%%%%%%
% \paragraph{Command Line Processing.}
%
% The following three command lines generate the output files
% |cdocscld|, |cdocscl1| and |cdocscl2|
% which should be identical to
% |cdocsdrf|, |cdocsch1| and |cdocsfn2|, respectively:
% \begin{center}
% \begin{tabular}{l}
% |latex -jobname cdocscld \|\\
% |  "\def\version{draft}\input{childdoc.def}\childdocforward{cdocsamp}"|\\
% |latex -jobname cdocscl1 \|\\
% |  "\input{childdoc.def}\childdocforward[cdocsamp]{cdocsch1}"|\\
% |latex -jobname cdocscl2 \|\\
% |  "\def\version{final}\input{childdoc.def}\childdocforward{cdocsch2}"|
% \end{tabular}
% \end{center}
% Note that the trailing backslash on each first line
% merely continues the input to the second line
% (for convenient cut ant paste).
% Furthermore, the command |latex| can be replaced by any
% of its alternative versions such as |pdflatex|.
%
% %%%%%%%%%%%%%%%%%%%%%%%%%%%%%%%%%%%%%%%%%%%%%%%%%%%%%%%%%%%%%%%%%%%%%%%%%%%%%%
% %%%%%%%%%%%%%%%%%%%%%%%%%%%%%%%%%%%%%%%%%%%%%%%%%%%%%%%%%%%%%%%%%%%%%%%%%%%%%%
% \section{Implementation}
%\iffalse
%<*package>
%\fi
%
% This section describes the definitions file |childdoc.def|.

% The definitions cannot be loaded using |\usepackage| or |\RequirePackage|
% which has a mechanism to prevent loading a style file more than once.
% When loading the definitions by means of |\input|
% multiple instances have to be prevented manually:
%\iffalse
%This code needs to be before the `\ProvidesFile' directive
%which is defined at the beginning of this file.
%Therefore it is also placed there and commented out here.
%</package>
%<*discard>
%\fi
%    \begin{macrocode}
\ifdefined\childdocmain\endinput\fi
%    \end{macrocode}
%\iffalse
%</discard>
%<*package>
%\fi
%
% \macro{\ifchilddoc}
% \macro{\ifchilddocmanual}
% The conditional |\ifchilddoc| tells whether a
% child (true) or main (false) document is being compiled.
% The conditional |\ifchilddocmanual| tells whether
% the |\includeonly| mechanism is used (false) or
% the selection of child files must be performed manually (true).
% The definitions initialise to false:
%    \begin{macrocode}
\newif\ifchilddoc
\newif\ifchilddocmanual
%    \end{macrocode}

% \macro{\childdocname}
% \macro{\childdocjob}
% The macro |\childdocname| stores the name of the main document
% to be compiled. The macro |\childdocjob| stores the name of
% the document on which the \LaTeX{} compiler was originally invoked.
% The content of |\jobname| cannot be compared
% to filenames specified in the source due to different catcodes.
% The following code rescans |\jobname|, stores the result
% in |\childdocname| and saves a copy in |\childdocjob|:
%    \begin{macrocode}
\edef\childdocname{\scantokens\expandafter{\jobname\noexpand}}
\let\childdocjob\childdocname
%    \end{macrocode}

% \macro{\childdocdisable}
% The macro |\childdocdisable| prevents the main file
% from being processed more than once.
% At this stage, the main document command |\childdocmain|
% is assumed to be called once again where it should do nothing.
% Any subsequent call to it should prevent
% a secondary processing of the main document
% It overwrites the forwarding commands
% |\childdocof| and |\childdocforward|
% with empty macros to prevent further inclusions of the main document:
%    \begin{macrocode}
\newcommand{\childdocdisable}
{
  \renewcommand{\childdocmain}[1]{\renewcommand{\childdocmain}[1]{\endinput}}
  \renewcommand{\childdocof}[1]{}
  \renewcommand{\childdocby}[2][]{}
  \renewcommand{\childdocforward}[2][]{}
  \renewcommand{\childdocdisable}{}
}
%    \end{macrocode}

% \macro{\childdocmain}
% The macro |\childdocmain| is to be called at the top of the main file
% with nothing or the main filename (without extension) as argument.
% First, it breaks loops.
% If the argument is not empty and does not match |\childdocname|
% (which is set by the first inclusion of |childdoc.def|),
% |\ifchilddoc| is set to true, |\includeonly| is applied to the child file
% and |\jobname| is set to the main file
% (for proper handling of |.aux| files):
%    \begin{macrocode}
\newcommand{\childdocmain}[1]
{
  \childdocdisable\childdocmain{}
  \if?#1?\else
    \begingroup
      \def\childdoctmp{#1}
      \ifx\childdoctmp\childdocname
        \def\childdoctmp{}
      \else
        \def\childdoctmp
        {
          \childdoctrue
          \includeonly{\childdocname}
          \def\childdocjob{#1}
          \def\jobname{#1}
        }
      \fi
      \expandafter
    \endgroup
    \childdoctmp
  \fi
}
%    \end{macrocode}

% \macro{\childdocof}
% The command |\childdocof| redirects
% compilation to the main file |#1|.
%    \begin{macrocode}
\newcommand{\childdocof}[1]
{
  \childdocdisable
  \childdoctrue
  \includeonly{\childdocname}
  \def\jobname{#1}
  \def\childdocjob{#1}
  \input{#1}
}
%    \end{macrocode}

% \macro{\childdocby}
% The command |\childdocby| ....
%    \begin{macrocode}
\newcommand{\childdocby}[2][]
{
  \childdocdisable
  \childdoctrue
  \childdocmanualtrue
  \if?#1?\else
    \def\jobname{#2}
  \fi
  \def\childdocjob{#2}
  \input{#2}
  \endinput
}
%    \end{macrocode}

% \macro{\childdocforward}
% The command |\childdocforward| redirects
% compilation to the main file or
% (if the optional argument is given) a child file.
% Parameters are set as if the main file
% or a child file starting with |\childdocof| was compiled.
% Then compilation is handed over to the main file:
%    \begin{macrocode}
\newcommand{\childdocforward}[2][]
{
  \begingroup
    \if?#1?
      \def\childdoctmp
      {
        \def\childdocname{#2}
        \def\childdocjob{#2}
        \def\jobname{#2}
        \input{#2}
        \endinput
      }
    \else
      \def\childdoctmp
      {
        \childdocdisable
        \def\childdocname{#2}
        \childdoctrue
        \includeonly{#2}
        \def\childdocjob{#1}
        \def\jobname{#1}
        \input{#1}
        \endinput
      }
    \fi
    \expandafter
  \endgroup
  \childdoctmp
}
%    \end{macrocode}

% \macro{\childdocforwardprefix}
% The command |\childdocforwardprefix| redirects
% compilation to the main or a child file by means of a pattern.
% The prefix |#1| in the current filename is replaced by |#2|
% and the suffix of the current filename is kept
% (it is assumed that the filename does not contain the substring `|~~~|'
% which is used as a delimiter).
% Compilation is handed over to the new file by |\childdocforward|:
%    \begin{macrocode}
\newcommand{\childdocforwardprefix}[3][]
{
  \begingroup
    \def\childdocextract #2##1~~~{\def\childdoctmp{\childdocforward[#1]{#3##1}}}
    \expandafter\childdocextract\childdocname~~~
    \expandafter
  \endgroup
  \childdoctmp
}
%    \end{macrocode}

% \macro{\childdoc}
% The deprecated macro |\childdoc| is a legacy version of |\childdocmain|:
%    \begin{macrocode}
\newcommand{\childdoc}{\childdocmain}
%    \end{macrocode}

% \macro{\childdocredirect}
% The deprecated macro |\childdocredirect| is a legacy version
% of |\childdocforward| and |\childdocforwardprefix|:
%    \begin{macrocode}
\newcommand{\childdocredirect}[2][]
{
  \begingroup
    \if?#1?
      \def\childdoctmp{\childdocforward{#2}}
    \else
      \def\childdoctmp{\childdocforwardprefix{#1}{#2}}
    \fi
    \expandafter
  \endgroup
  \childdoctmp
}
%    \end{macrocode}

%\iffalse
%</package>
%\fi
%
\endinput
|
and perform the replacements as outlined below.
Instead of |\childdocmain{|\textit{main}|}| add the following code
to the top of the main file:
%
\begin{center}
\begin{tabular}{l}
|\||ifdefined\childdocname\endinput\||fi\newif\ifchilddoc|\\
|\edef\childdocname{\scantokens\expandafter{\jobname\noexpand}}|\\
|\def\childdocmain{|\textit{main}|}\||ifx\childdocmain\childdocname\||else|\\
|\childdoctrue\includeonly{\childdocname}\let\jobname\childdocmain\||fi|\\
\end{tabular}
\end{center}
%
Instead of |\childdocof{|\textit{main}|}| just include the main file
at the top of each child file:
%
\begin{center}
|\input{|\textit{main}|}|
\end{center}
%
A simple redirection |\childdocforward{|\textit{dest}|}| is achieved by:
%
\begin{center}
|\def\jobname{|\textit{dest}|}\input{\jobname}|
\end{center}
%
The redirection with prefix
|\childdocforwardprefix[|\textit{prefix}|]{|\textit{dest}|}|
is accomplished by:
%
\begin{center}
\begin{tabular}{l}
|{\edef\jobname{\scantokens\expandafter{\jobname\noexpand}}|\\
|\def\redirectjob |\textit{prefix}|#1~~~{\gdef\jobname{|\textit{dest}|#1}}|\\
|\expandafter\redirectjob\jobname~~~}\input{\jobname}|
\end{tabular}
\end{center}

In an alternative approach,
child documents can be compiled by a specific command line
without additional code or specific definitions:
%
\begin{center}
|... -jobname "|\textit{target}|" "|[\textit{flags}]%
|\includeonly{|\textit{dest}|}\input{|\textit{main}|}"|
\end{center}
%

%%%%%%%%%%%%%%%%%%%%%%%%%%%%%%%%%%%%%%%%%%%%%%%%%%%%%%%%%%%%%%%%%%%%%%%%%%%%%%%%
%%%%%%%%%%%%%%%%%%%%%%%%%%%%%%%%%%%%%%%%%%%%%%%%%%%%%%%%%%%%%%%%%%%%%%%%%%%%%%%%
\section{Information}

%%%%%%%%%%%%%%%%%%%%%%%%%%%%%%%%%%%%%%%%%%%%%%%%%%%%%%%%%%%%%%%%%%%%%%%%%%%%%%%%
\subsection{Copyright}

Copyright \copyright{} 2017--2018 Niklas Beisert

This work may be distributed and/or modified under the
conditions of the \LaTeX{} Project Public License, either version 1.3
of this license or (at your option) any later version.
The latest version of this license is in
  \url{http://www.latex-project.org/lppl.txt}
and version 1.3 or later is part of all distributions of \LaTeX{}
version 2005/12/01 or later.

This work has the LPPL maintenance status `maintained'.

The Current Maintainer of this work is Niklas Beisert.

This work consists of the files |README.txt|, |childdoc.ins| and |childdoc.dtx|
as well as the derived files |childdoc.def|, |cdocsamp.tex|
with |cdocsch1.tex|, |cdocsch2.tex|, |cdocspt3.tex|, |cdocspt4.tex|,
|cdocsdrf.tex|, |cdocsfn1.tex|, |cdocsfn2.tex|
as well as |childdoc.pdf|.

%%%%%%%%%%%%%%%%%%%%%%%%%%%%%%%%%%%%%%%%%%%%%%%%%%%%%%%%%%%%%%%%%%%%%%%%%%%%%%%%
\subsection{Files and Installation}

The package consists of the files:
%
\begin{center}
\begin{tabular}{ll}
    |README.txt|   & readme file \\
    |childdoc.ins| & installation file \\
    |childdoc.dtx| & source file \\
    |childdoc.def| & definition file \\
    |cdocsamp.tex| & sample main file \\
    |cdocsch1.tex| & sample include file \\
    |cdocsch2.tex| & sample include file \\
    |cdocspt3.tex| & sample part file \\
    |cdocspt4.tex| & sample part file \\
    |cdocsdrf.tex| & sample redirection file \\
    |cdocsfn1.tex| & sample redirection file \\
    |cdocsfn2.tex| & sample redirection file \\
    |childdoc.pdf| & manual
\end{tabular}
\end{center}
%
The distribution consists of the files
|README.txt|, |childdoc.ins| and |childdoc.dtx|.
%
\begin{itemize}
\item
Run (pdf)\LaTeX{} on |childdoc.dtx|
to compile the manual |childdoc.pdf| (this file).
\item
Run \LaTeX{} on |childdoc.ins| to create the definitions file |childdoc.def|
and the sample |cdocsamp.tex| with include files
|cdocsch1.tex|, |cdocsch2.tex|, |cdocspt3.tex|, |cdocspt4.tex|,
|cdocsdrf.tex|, |cdocsfn1.tex|, |cdocsfn2.tex|.
Then copy the file |childdoc.def| to an appropriate directory of your \LaTeX{}
distribution, e.g.\ \textit{texmf-root}|/tex/latex/childdoc|.
\end{itemize}

%%%%%%%%%%%%%%%%%%%%%%%%%%%%%%%%%%%%%%%%%%%%%%%%%%%%%%%%%%%%%%%%%%%%%%%%%%%%%%%%
\subsection{Related CTAN Packages}

There are several other packages which offer a similar functionality:
%
\begin{itemize}
\item
The packages
\href{http://ctan.org/pkg/docmute}{\textsf{docmute}},
\href{http://ctan.org/pkg/includex}{\textsf{includex}} and
\href{http://ctan.org/pkg/standalone}{\textsf{standalone}}
provide commands to include only the document body of
a child file thus allowing both files to be compiled individually.
\item
The packages \href{http://ctan.org/pkg/subdocs}{\textsf{subdocs}}
and \href{http://ctan.org/pkg/subfiles}{\textsf{subfiles}}
provide structures in which the main and child documents can be
encapsulated and allowing them to be compiled individually.
The inclusion mechanism is different from the conventional |\include|.
\item
The package \href{http://ctan.org/pkg/combine}{\textsf{combine}}
is an elaborate solution to combine several documents into one.
\end{itemize}
%
See also the CTAN topic \href{http://ctan.org/topic/subdocs}{\textsf{subdocs}}
for further related packages.
The present package differs from the above solutions in that
a document structure constructed with the conventional |\include| mechanism
just needs two extra commands at the top of every file
such that all constituent files can be compiled individually.

%%%%%%%%%%%%%%%%%%%%%%%%%%%%%%%%%%%%%%%%%%%%%%%%%%%%%%%%%%%%%%%%%%%%%%%%%%%%%%%%
%\subsection{Feature Suggestions}
%
%The following is a list of features which may be useful for future
%versions of this package:
%%
%\begin{itemize}
%\item
%\ldots
%\end{itemize}

%%%%%%%%%%%%%%%%%%%%%%%%%%%%%%%%%%%%%%%%%%%%%%%%%%%%%%%%%%%%%%%%%%%%%%%%%%%%%%%%
\subsection{Revision History}

%%%%%%%%%%%%%%%%%%%%%%%%%%%%%%%%%%%%%%%%
\paragraph{v2.0:} 2018/12/30

\begin{itemize}
\item
immediate forward processing
\item
added |\childdocby| mechanism
\item
manual restructured
\end{itemize}

%%%%%%%%%%%%%%%%%%%%%%%%%%%%%%%%%%%%%%%%
\paragraph{v1.6:} 2018/01/17

\begin{itemize}
\item
application for development of include files
\item
corrections to manual
\end{itemize}

%%%%%%%%%%%%%%%%%%%%%%%%%%%%%%%%%%%%%%%%
\paragraph{v1.5:} 2017/05/21

\begin{itemize}
\item
more complete structuring introduced
\item
|\childdocof| introduced
\item
|\childdoc| renamed to |\childdocmain|
\item
|\childredirect| renamed to |\childdocforward| and |\childdocforwardprefix|
and functionality expanded
\end{itemize}

%%%%%%%%%%%%%%%%%%%%%%%%%%%%%%%%%%%%%%%%
\paragraph{v1.0:} 2017/04/27

\begin{itemize}
\item
manual and install package
\item
first version published on CTAN
\end{itemize}

%%%%%%%%%%%%%%%%%%%%%%%%%%%%%%%%%%%%%%%%
\paragraph{v0.6:} 2017/04/26

\begin{itemize}
\item
redirection mechanism added
\end{itemize}

%%%%%%%%%%%%%%%%%%%%%%%%%%%%%%%%%%%%%%%%
\paragraph{v0.5:} 2017/04/26

\begin{itemize}
\item
functionality in definition file
\end{itemize}


%%%%%%%%%%%%%%%%%%%%%%%%%%%%%%%%%%%%%%%%%%%%%%%%%%%%%%%%%%%%%%%%%%%%%%%%%%%%%%%%
%%%%%%%%%%%%%%%%%%%%%%%%%%%%%%%%%%%%%%%%%%%%%%%%%%%%%%%%%%%%%%%%%%%%%%%%%%%%%%%%
%%%%%%%%%%%%%%%%%%%%%%%%%%%%%%%%%%%%%%%%%%%%%%%%%%%%%%%%%%%%%%%%%%%%%%%%%%%%%%%%
\appendix

\settowidth\MacroIndent{\rmfamily\scriptsize 000\ }

 \DocInput{childdoc.dtx}

\end{document}
%</driver>
% \fi
%
% %%%%%%%%%%%%%%%%%%%%%%%%%%%%%%%%%%%%%%%%%%%%%%%%%%%%%%%%%%%%%%%%%%%%%%%%%%%%%%
% %%%%%%%%%%%%%%%%%%%%%%%%%%%%%%%%%%%%%%%%%%%%%%%%%%%%%%%%%%%%%%%%%%%%%%%%%%%%%%
% \section{Sample}
%\iffalse
%<*samplemain>
%\fi
%
% The following presents a sample document
% with two chapters, two parts, a title page,
% a compile flag as well as three forwarding files to set the flag.
% It consists of eight |.tex| files:
% \begin{center}
% \begin{tabular}{ll}
% |cdocsamp.tex|&main file\\
% |cdocsch1.tex|&include file for chapter 1\\
% |cdocsch2.tex|&include file for chapter 2\\
% |cdocspt3.tex|&include file for part 3\\
% |cdocspt4.tex|&include file for part 4\\
% |cdocsdrf.tex|&forwarding file for main file in draft mode\\
% |cdocsfi1.tex|&forwarding file for final version of chapter 1\\
% |cdocsfi2.tex|&forwarding file for final version of chapter 2\\
% \end{tabular}
% \end{center}
% Each of the eight files can be compiled directly by the \LaTeX{} compiler.
%
% %%%%%%%%%%%%%%%%%%%%%%%%%%%%%%%%%%%%%%
% \paragraph{Main File.}
%
% The main file is called |cdocsamp.tex|.
%
% Load the \textsf{childdoc} definitions and
% declare the filename for the main document:
%    \begin{macrocode}
% \iffalse
%
% childdoc.dtx Copyright (C) 2017-2018 Niklas Beisert
%
% This work may be distributed and/or modified under the
% conditions of the LaTeX Project Public License, either version 1.3
% of this license or (at your option) any later version.
% The latest version of this license is in
%   http://www.latex-project.org/lppl.txt
% and version 1.3 or later is part of all distributions of LaTeX
% version 2005/12/01 or later.
%
% This work has the LPPL maintenance status `maintained'.
%
% The Current Maintainer of this work is Niklas Beisert.
%
% This work consists of the files childdoc.dtx and childdoc.ins
% and the derived files childdoc.def and cdocsamp.tex with
% cdocsch1.tex, cdocsch2.tex, cdocsdrf.tex, cdocsfn1.tex, cdocsfn2.tex.
%
%<package>\ifdefined\childdocmain\endinput\fi
%<package>\ProvidesFile{childdoc.def}[2018/12/30 v2.0 child document driver]
%<samplemain>\ProvidesFile{cdocsamp.tex}[2018/12/30 v2.0 sample for childdoc]
%<*driver>
%\ProvidesFile{childdoc.drv}[2018/12/30 v2.0 childdoc reference manual file]
\PassOptionsToClass{10pt,a4paper}{article}
\documentclass{ltxdoc}

\usepackage[margin=35mm]{geometry}
\usepackage{hyperref}
\usepackage{hyperxmp}
\usepackage[usenames]{color}

\hypersetup{colorlinks=true}
\hypersetup{pdfstartview=FitH}
\hypersetup{pdfpagemode=UseNone}
\hypersetup{pdfsource={}}
\hypersetup{pdflang={en-UK}}
\hypersetup{pdfcopyright={Copyright 2017-2018 Niklas Beisert.
  This work may be distributed and/or modified under the
  conditions of the LaTeX Project Public License, either version 1.3
  of this license or (at your option) any later version.}}
\hypersetup{pdflicenseurl={http://www.latex-project.org/lppl.txt}}
\hypersetup{pdfcontactaddress={ETH Zurich, ITP, HIT K,
  Wolfgang-Pauli-Strasse 27}}
\hypersetup{pdfcontactpostcode={8093}}
\hypersetup{pdfcontactcity={Zurich}}
\hypersetup{pdfcontactcountry={Switzerland}}
\hypersetup{pdfcontactemail={nbeisert@itp.phys.ethz.ch}}
\hypersetup{pdfcontacturl={http://people.phys.ethz.ch/\xmptilde nbeisert/}}

\newcommand{\secref}[1]{\hyperref[#1]{section \ref*{#1}}}

\parskip1ex
\parindent0pt
\let\olditemize\itemize
\def\itemize{\olditemize\parskip0pt}

\begin{document}

\title{The \textsf{childdoc} Package}
\hypersetup{pdftitle={The childdoc Package}}
\author{Niklas Beisert\\[2ex]
  Institut f\"ur Theoretische Physik\\
  Eidgen\"ossische Technische Hochschule Z\"urich\\
  Wolfgang-Pauli-Strasse 27, 8093 Z\"urich, Switzerland\\[1ex]
  \href{mailto:nbeisert@itp.phys.ethz.ch}
  {\texttt{nbeisert@itp.phys.ethz.ch}}}
\hypersetup{pdfauthor={Niklas Beisert}}
\hypersetup{pdfsubject={Manual for the LaTeX2e Package childdoc}}
\date{30 December 2018, \textsf{v2.0}}
\maketitle

\begin{abstract}\noindent
\textsf{childdoc} is a \LaTeXe{} package
that enables the direct compilation
of document sections included by |\include|
to individual files.
\end{abstract}

\begingroup
\parskip0ex
\tableofcontents
\endgroup

%%%%%%%%%%%%%%%%%%%%%%%%%%%%%%%%%%%%%%%%%%%%%%%%%%%%%%%%%%%%%%%%%%%%%%%%%%%%%%%%
%%%%%%%%%%%%%%%%%%%%%%%%%%%%%%%%%%%%%%%%%%%%%%%%%%%%%%%%%%%%%%%%%%%%%%%%%%%%%%%%
\section{Introduction}

\LaTeX{} provides a mechanism to structure a large document (such as a book)
into a main file and several child files (containing the chapters)
using the |\include| command.
This mechanism is beneficial for documents
which span hundreds of pages in order to
make the source file(s) more manageable.
Moreover, compilation can be restricted to
selected child files by means of the |\includeonly| command.
The latter feature can be used to reduce the compilation time while editing
(this was significantly more useful in the earlier days of \LaTeX{})
or to generate a smaller document which is easier to navigate.
Another application of |\includeonly| is to generate
documents consisting of selected parts of the complete document.

However, there are a few drawbacks of the plain |\include| mechanism:
\begin{itemize}
\item
The child files cannot be compiled on their own,
they can only be compiled via the main file.
A naive editing environment
(such as a text editor with an option
to have the current file processed by \LaTeX)
may require one to switch to the main file before compiling;
attempting to compile the child file produces errors.
\item
The main file must be modified (each time)
to adjust the |\includeonly| command
to the present needs. This easily leaves the main file in a messy state.
\item
The generated document will always carry the filename
of the main document. This is inconvenient if
several child files are to be compiled and
to be kept for distribution.
\end{itemize}

The present package provides a simple interface
to make child files individually compilable by \LaTeX{}.
Compiling a child file then has the same effect as compiling
the main file with an |\includeonly| command
to select the appropriate child.
Moreover the generated document will carry the name of the child
rather than the main file.
This resolves all three above issues.

This feature is meant to make the editing of books,
thesis documents and lecture notes somewhat more convenient.
However, the package can also be used efficiently for
composing a series of documents (such as exercise sheets)
which are typically distributed individually.
It then assists the author in generating the individual documents
(potentially in different versions)
as well as a document containing the collected series.
Another application is in developing style files
or other kinds of included material
where compilation of the style file could redirect
to a sample or test file.

%%%%%%%%%%%%%%%%%%%%%%%%%%%%%%%%%%%%%%%%%%%%%%%%%%%%%%%%%%%%%%%%%%%%%%%%%%%%%%%%
%%%%%%%%%%%%%%%%%%%%%%%%%%%%%%%%%%%%%%%%%%%%%%%%%%%%%%%%%%%%%%%%%%%%%%%%%%%%%%%%
\section{Usage}

First of all, the package \textsf{childdoc} is \emph{not} a standard
\LaTeXe{} |.sty| style file! Therefore it needs to be invoked in
a non-standard way.

%%%%%%%%%%%%%%%%%%%%%%%%%%%%%%%%%%%%%%%%%%%%%%%%%%%%%%%%%%%%%%%%%%%%%%%%%%%%%%%%
\subsection{Included Files}
\label{sec:include}

%%%%%%%%%%%%%%%%%%%%%%%%%%%%%%%%%%%%%%%%
\DescribeMacro{\childdocmain}
To use the package, add the commands
\begin{center}
\begin{tabular}{l}
|\input{childdoc.def}|\\
|\childdocmain{}|\\
\end{tabular}
\end{center}
at the very top of the main \LaTeX{} file,
in particular \emph{before} the |\documentclass| statement!
The argument of |\childdocmain| should be left empty
(but it must be present).

%%%%%%%%%%%%%%%%%%%%%%%%%%%%%%%%%%%%%%%%
\DescribeMacro{\childdocof}
Furthermore, add the commands
\begin{center}
\begin{tabular}{l}
|\input{childdoc.def}|\\
|\childdocof{|\textit{main}|}|\\
\end{tabular}
\end{center}
at the top of every child file \textit{child}
which is included by |\include{|\textit{child}|}|
from within the main file
(or at least for those files to be compiled individually).
The argument \textit{main} must be the filename of the main file.

There are a couple of
considerations in setting up the main and child documents:

%%%%%%%%%%%%%%%%%%%%%%%%%%%%%%%%%%%%%%%%
\paragraph{Restrictions.}

Please note the following restrictions:
\begin{itemize}
\item
|\childdocmain| must be called with one argument \textit{main}
to ensure compatibility with earlier version of the package.
It must either be empty (|\childdocmain{}|)
or precisely match the filename of the main file in which it is specified.
See \secref{sec:detection} for further information.
\item
The filename \textit{main} must be specified without the |.tex| extension.
\item
The filename \textit{main} is case sensitive
(even in case-insensitive file systems)
due to internal string comparison.
\item
The argument \textit{main} should be fully expanded, it cannot be a macro.
\item
Subdirectories and special characters should be avoided in filenames.
\item
The command |\childdocmain{|\textit{main}|}| must be followed by a whitespace.
It should not be followed immediately by another command
or by a comment mark `|%|'.
This is because the \TeX{} parser reads the token immediately following
the argument of |\childdocmain| and puts it
at the beginning of every child section;
however, a white\-space is ignored.
\end{itemize}

%%%%%%%%%%%%%%%%%%%%%%%%%%%%%%%%%%%%%%%%
\paragraph{Content of Main File.}

It is advisable to place all content in the child files included by |\include|.
Any output contained in the main file will appear in all child documents
unless suppressed manually;
it cannot be suppressed automatically by the |\includeonly| directive
and thus should normally be avoided.
A method to include some content in the main file
by means of conditional processing is described in \secref{sec:conditional}.

%%%%%%%%%%%%%%%%%%%%%%%%%%%%%%%%%%%%%%%%
\paragraph{Page Numbering.}

When only a part of the document is compiled,
the appropriate numbering of pages
(as well as other status parameters)
is determined from the |.aux| files.
The latter contain information from previous passes.
However this information needs to propagate through
all intermediate child documents.
Therefore the page numbering in child documents may well
be inconsistent until the complete document is compiled at least once.

A useful (if unconventional) way to always ensure a consistent
page numbering is to restart the numbering in each child document
and denote the pages by `\textit{child}|.|\textit{page}'
where \textit{child} represents the chapter/section number of the child file.
This can be achieved by the command
|\numberwithin{page}{|\textit{child}|}|
of the \textsf{amsmath} package
where \textit{child} can be |chapter| or |section|
depending on the chosen structuring.
Alternatively, one can modify the macro |\thepage| appropriately
and reset the counter |page| at the start of each child file.

%%%%%%%%%%%%%%%%%%%%%%%%%%%%%%%%%%%%%%%%%%%%%%%%%%%%%%%%%%%%%%%%%%%%%%%%%%%%%%%%
\subsection{Conditional Processing}
\label{sec:conditional}

The package provides a mechanism to compile different versions
of a document. To customise the versions further some conditional processing
can come in handy to distinguish which version is being compiled.
The package provides two macros to describe the compilation context:

%%%%%%%%%%%%%%%%%%%%%%%%%%%%%%%%%%%%%%%%
\DescribeMacro{\ifchilddoc}
The conditional |\ifchilddoc| distinguishes between the compilation of
child documents and the main document:
%
\begin{center}
|\ifchilddoc |\textit{child-code}| |[|\||else |\textit{main-code}]| \||fi|
\end{center}

%%%%%%%%%%%%%%%%%%%%%%%%%%%%%%%%%%%%%%%%
\DescribeMacro{\childdocname}
\DescribeMacro{\childdocjob}
The macro |\childdocname| contains the filename (without extension)
of the main or child file being processed.
Note that |\childdocjob| will always contain the name of the main file.

%%%%%%%%%%%%%%%%%%%%%%%%%%%%%%%%%%%%%%%%
\paragraph{Title Page.}

Conditional processing can be used to include a title or banner page
in the main document when proper precautions are taken.
Importantly, the code in the main file should ensure that the page counter
(as well as other status parameters which are stored in the |.aux| files)
takes the same value after the conditional processing.
Otherwise the page numbers may take divergent values
depending on which part is compiled.

For example, a title page could be declared by:
%
\begin{center}
\begin{tabular}{l}
|\ifchilddoc\||else|\\
|\addtocounter{page}{-1}|\\
\textit{code for title page}\\
|\newpage|\\
|\||fi|
\end{tabular}
\end{center}
%
A banner page for the child documents can be generated by:
%
\begin{center}
\begin{tabular}{l}
|\ifchilddoc|\\
|\addtocounter{page}{-1}|\\
\textit{code for banner page}\\
|\newpage|\\
|\||fi|
\end{tabular}
\end{center}
%
Here one could write a message such as:
\begin{center}
|This is the part \childdocname{} of \childdocjob{}.|
\end{center}

%%%%%%%%%%%%%%%%%%%%%%%%%%%%%%%%%%%%%%%%%%%%%%%%%%%%%%%%%%%%%%%%%%%%%%%%%%%%%%%%
\subsection{Flags}
\label{sec:flags}

The package makes it easy to generate different versions
of the main or child documents.
To this end compilation flags can be defined
and assigned different default values.
They will be particularly useful in conjunction
with the forwarding mechanism described in \secref{sec:forward}.

For example, it may be useful to have a flag |\version|
which can be set to |draft| or |final|.
The document source will contain some conditional code
depending on the value of |\version|.
Suppose further, the flag should default to |final| for the main file
and to |draft| for child files
which is a natural assignment for editing the document.
This is achieved by placing the following code
in the preamble of the main document
(below the |\childdocmain| directive):
%
\begin{center}
\begin{tabular}{l}
|\ifchilddoc|\\
|\providecommand{\version}{draft}|\\
|\||else|\\
|\providecommand{\version}{final}|\\
|\||fi|
\end{tabular}
\end{center}
%
The definition by |\providecommand| makes sure
that previous definitions are not overwritten.
Further statements |\providecommand{\version}{...}|
can thus be added before the above code to override it.

For the main file, one might add a line
(between |\childdocmain| and the above block)
%
\begin{center}
|%\ifchilddoc\||else\providecommand{\version}{draft}\||fi|
\end{center}
%
which can be uncommented to produce a draft version.
Likewise one can add a line to the very top of a child file
(above the |\childdocof{|\textit{main}|}| directive)
%
\begin{center}
|%\providecommand{\version}{final}|
\end{center}
%
which can be uncommented to produce the final version of this child document.

%%%%%%%%%%%%%%%%%%%%%%%%%%%%%%%%%%%%%%%%%%%%%%%%%%%%%%%%%%%%%%%%%%%%%%%%%%%%%%%%
\subsection{Forwarding}
\label{sec:forward}

Different versions of the main or child documents
using compilation flags as described in \secref{sec:flags}
can be (permanently) stored in different files
for convenient compilation, viewing and distribution.
To this end, the package defines a command
to pass on compilation to a different file:

%%%%%%%%%%%%%%%%%%%%%%%%%%%%%%%%%%%%%%%%
\DescribeMacro{\childdocforward}
The command |\childdocforward| redirects processing to
another source file:
%
\begin{center}
\begin{tabular}{l}
|\input{childdoc.def}|\\
|\childdocforward[|\textit{main}|]{|\textit{dest}|}|\\
\end{tabular}
\end{center}
%
The argument \textit{dest} is the destination file
(without extension).
It should be the main file or one of the child files.
Note that further \textsf{childdoc} directives
such as |\childdocof| and |\childdocforward|
in the indicated file will be processed in this form.
The optional argument \textit{main}
passes on directly to the main file \textit{main}
while pretending to compile the child \textit{dest}.
This form behaves as if \textit{dest}
issues |\childdocof{|\textit{main}|}| right away,
and no further \textsf{childdoc} directives will be processed.

%%%%%%%%%%%%%%%%%%%%%%%%%%%%%%%%%%%%%%%%
\DescribeMacro{\...prefix}
In the alternative form |\childdocforwardprefix|,
%
\begin{center}
\begin{tabular}{l}
|\input{childdoc.def}|\\
|\childdocforwardprefix[|\textit{main}|]{|\textit{prefix}|}{|\textit{dest}|}|
\end{tabular}
\end{center}
%
the destination file is determined by a pattern
depending on the current file:
To make this work, the current file must be called
`{\textit{prefix}\hspace{0.2em}\textit{suffix}}'
with \textit{prefix} matching precisely the argument.
Processing is then passed on to the file
`{\textit{dest}\hspace{0.2em}\textit{suffix}}'.
Surely, the same effect is achieved by
directly specifying the
argument `{\textit{dest}\hspace{0.2em}\textit{suffix}}'
in the first form.
However, that requires to set up a different file
for each child. With the alternative form of the command
all these files can have exactly the same content
which simplifies setting them up and maintaining them.

For example, the following file |draft.tex|
with a compilation flag |\version| as described in \secref{sec:flags}
compiles the main document as a draft:
%
\begin{center}
\begin{tabular}{l}
|\def\version{draft}|\\
|\input{childdoc.def}|\\
|\childdocforward{|\textit{main}|}|
\end{tabular}
\end{center}
%
Likewise, the following files |final|\textit{nn}|.tex|
compile the final version of the child document
|child|\textit{nn}|.tex|:
%
\begin{center}
\begin{tabular}{l}
|\def\version{final}|\\
|\input{childdoc.def}|\\
|\childdocforwardprefix{final}{child}|
\end{tabular}
\end{center}
%

Note that when several versions of a main file and/or of each child file
are to be generated, it may be convenient to set up a |Makefile| or
shell script to automatise the process.

%%%%%%%%%%%%%%%%%%%%%%%%%%%%%%%%%%%%%%%%%%%%%%%%%%%%%%%%%%%%%%%%%%%%%%%%%%%%%%%%
\subsection{Command Line Processing}
\label{sec:commandline}

The effect of redirection files can also be achieved by invoking
the \LaTeX{} compiler with a more elaborate command line.
Most conveniently this should be done as part
of a shell script or a |Makefile|.

When using \textsf{childdoc} in the main file, the following
command lines effectively perform a redirection
(note that depending on the shell being used,
backslashes may have to be doubled: `|\|' $\to$ `|\\|'):
%
\begin{center}
|... -jobname "|\textit{target}|" |\\|"|[\textit{flags}]%
|\input{childdoc.def}\childdocforward[|\textit{main}|]{|\textit{dest}|}"|
\end{center}
%
Here \textit{target} is the name of the output file,
\textit{main} is the name of the main file
and \textit{dest} is the name of the main or child file to be processed
(all filenames without extensions).
The optional argument \textit{main} can be omitted
if \textit{main} matches \textit{dest}.
Optionally, compilation \textit{flags} can be defined via |\def| commands.
This command line makes the \TeX{} engine believe
it is compiling the file \textit{target}
whose content is specified as the latter parameter.
The provided code then forwards the processing to
\textit{main} or \textit{dest} as described in \secref{sec:forward}.

%%%%%%%%%%%%%%%%%%%%%%%%%%%%%%%%%%%%%%%%%%%%%%%%%%%%%%%%%%%%%%%%%%%%%%%%%%%%%%%%
\subsection{Include by Input}
\label{sec:input}

Including child documents by |\include| has some restrictions by design.
Most notably, the content of a child document always occupies
its own set of pages; pages cannot be shared between child documents.
Usually, this behaviour makes perfect sense
because each child document contain an essential part of the document.
However, in some situations it may be desirable to compose
a document from a collection of parts
without having mandatory page breaks between then.
For this case, the package
provides a mechanism to include parts
by |\input| which can also be processed individually.
However, by construction this mechanism
requires manual handling of the content to be output.

%%%%%%%%%%%%%%%%%%%%%%%%%%%%%%%%%%%%%%%%
\DescribeMacro{\ifchilddocmanual}
The main file should be prepared as usual, see \secref{sec:include}.
However, the document body must make a distinction
between processing of an individual part and of the main document, e.g.:
%
\begin{center}
\begin{tabular}{l}
|\ifchilddocmanual|\\
|\input{\childdocname}|\\
|\||else|\\
\textit{document body with }|\input{|\textit{part}|}|\\
|\||fi|
\end{tabular}
\end{center}
%
The conditional |\ifchilddocmanual| is true whenever
a part to be included by |\input| is being compiled,
and the name of the part is stored in |\childdocname|.

%%%%%%%%%%%%%%%%%%%%%%%%%%%%%%%%%%%%%%%%
\DescribeMacro{\childdocby}
Each part to be included by |\input| should start with:
%
\begin{center}
\begin{tabular}{l}
|\input{childdoc.def}|\\
|\childdocby{|\textit{main}|}|\\
\end{tabular}
\end{center}
%
The directive |\childdocby| is similar to |\childdocof|
described in \secref{sec:include},
but the subsequent selection of content must be done manually.
To that end, both |\ifchilddoc| and |\ifchilddocmanual|
will be true upon processing of a part,
and the name of the part is stored in |\childdocname|.
Note that |\jobname| will be set to the filename of the current part
so that each part receives an individual |.aux| file
that does not interfere with the |.aux| file(s) of the main document.
This behaviour can be altered by the alternative form
|\childdocby[*]{|\textit{main}|}| (with a non-empty optional argument)
which uses the |.aux| file of the main document
by setting |\jobname| to \textit{main}.

%%%%%%%%%%%%%%%%%%%%%%%%%%%%%%%%%%%%%%%%%%%%%%%%%%%%%%%%%%%%%%%%%%%%%%%%%%%%%%%%
\subsection{Driver Development}
\label{sec:driver}

The \textsf{childdoc} mechanism can also be use for the development
of definition files such as \LaTeX{} styles or classes.
This case differs from the above setup with multiple parts
included by |\include| in that no |\includeonly| should be invoked.
This can be achieved by starting the include file
(before |\ProvidesPackage|) with:
%
\begin{center}
\begin{tabular}{l}
|\input{childdoc.def}|\\
|\childdocforward{|\textit{main}|}|\\
\end{tabular}
\end{center}
%
or alternatively with:
%
\begin{center}
\begin{tabular}{l}
|\input{childdoc.def}|\\
|\childdocby{|\textit{main}|}|\\
\end{tabular}
\end{center}
%
Both forms have slightly different effects as described above.
The main file is prepared as usual, see \secref{sec:include}.

%%%%%%%%%%%%%%%%%%%%%%%%%%%%%%%%%%%%%%%%%%%%%%%%%%%%%%%%%%%%%%%%%%%%%%%%%%%%%%%%
\subsection{Legacy Detection}
\label{sec:detection}

The directive |\childdocmain| in the main file can detect
whether the complete document or merely a child is to be compiled
even without using the directive |\childdocof|.
This method is deprecated because it is less robust
and there is no compelling reason to use it;
it is merely provided for backward compatibility
and it may be removed in future versions.

If the detection mechanism is to be used,
it is mandatory to correctly specify
the filename of the main file as the argument of |\childdocmain|:
%
\begin{center}
\begin{tabular}{l}
|\input{childdoc.def}|\\
|\childdocmain{|\textit{main}|}|\\
\end{tabular}
\end{center}
%
If |\jobname| does not match the argument \textit{main} of |\childdocmain|,
it is assumed that |\jobname| points to the child file to be compiled.
When using |\childdocmain| with the main file specified as argument,
it suffices to start a child file
with just |\input{|\textit{main}|}|
without loading of the package and using |\childdocof|.
If instead all processing is done
with the appropriate \textsf{childdoc} directives,
the argument of \textit{main} of |\childdocmain| can be empty.

An alternative version of the command line processing described
in \secref{sec:commandline} using the detection mechanism reads:
%
\begin{center}
|... -jobname "|\textit{target}|" "|[\textit{flags}]%
[|\def\jobname{|\textit{dest}|}|]|\input{|\textit{main}|}"|
\end{center}

%%%%%%%%%%%%%%%%%%%%%%%%%%%%%%%%%%%%%%%%%%%%%%%%%%%%%%%%%%%%%%%%%%%%%%%%%%%%%%%%
\subsection{Manual Code}
\label{sec:manual}

In case one cannot be certain whether the definitions file |childdoc.def|
is installed on the target \TeX{} distribution
and one prefers not to ship it,
it is conceivable to paste a few relevant commands into the sources.

To that end, drop all statements |\input{childdoc.def}|
and perform the replacements as outlined below.
Instead of |\childdocmain{|\textit{main}|}| add the following code
to the top of the main file:
%
\begin{center}
\begin{tabular}{l}
|\||ifdefined\childdocname\endinput\||fi\newif\ifchilddoc|\\
|\edef\childdocname{\scantokens\expandafter{\jobname\noexpand}}|\\
|\def\childdocmain{|\textit{main}|}\||ifx\childdocmain\childdocname\||else|\\
|\childdoctrue\includeonly{\childdocname}\let\jobname\childdocmain\||fi|\\
\end{tabular}
\end{center}
%
Instead of |\childdocof{|\textit{main}|}| just include the main file
at the top of each child file:
%
\begin{center}
|\input{|\textit{main}|}|
\end{center}
%
A simple redirection |\childdocforward{|\textit{dest}|}| is achieved by:
%
\begin{center}
|\def\jobname{|\textit{dest}|}\input{\jobname}|
\end{center}
%
The redirection with prefix
|\childdocforwardprefix[|\textit{prefix}|]{|\textit{dest}|}|
is accomplished by:
%
\begin{center}
\begin{tabular}{l}
|{\edef\jobname{\scantokens\expandafter{\jobname\noexpand}}|\\
|\def\redirectjob |\textit{prefix}|#1~~~{\gdef\jobname{|\textit{dest}|#1}}|\\
|\expandafter\redirectjob\jobname~~~}\input{\jobname}|
\end{tabular}
\end{center}

In an alternative approach,
child documents can be compiled by a specific command line
without additional code or specific definitions:
%
\begin{center}
|... -jobname "|\textit{target}|" "|[\textit{flags}]%
|\includeonly{|\textit{dest}|}\input{|\textit{main}|}"|
\end{center}
%

%%%%%%%%%%%%%%%%%%%%%%%%%%%%%%%%%%%%%%%%%%%%%%%%%%%%%%%%%%%%%%%%%%%%%%%%%%%%%%%%
%%%%%%%%%%%%%%%%%%%%%%%%%%%%%%%%%%%%%%%%%%%%%%%%%%%%%%%%%%%%%%%%%%%%%%%%%%%%%%%%
\section{Information}

%%%%%%%%%%%%%%%%%%%%%%%%%%%%%%%%%%%%%%%%%%%%%%%%%%%%%%%%%%%%%%%%%%%%%%%%%%%%%%%%
\subsection{Copyright}

Copyright \copyright{} 2017--2018 Niklas Beisert

This work may be distributed and/or modified under the
conditions of the \LaTeX{} Project Public License, either version 1.3
of this license or (at your option) any later version.
The latest version of this license is in
  \url{http://www.latex-project.org/lppl.txt}
and version 1.3 or later is part of all distributions of \LaTeX{}
version 2005/12/01 or later.

This work has the LPPL maintenance status `maintained'.

The Current Maintainer of this work is Niklas Beisert.

This work consists of the files |README.txt|, |childdoc.ins| and |childdoc.dtx|
as well as the derived files |childdoc.def|, |cdocsamp.tex|
with |cdocsch1.tex|, |cdocsch2.tex|, |cdocspt3.tex|, |cdocspt4.tex|,
|cdocsdrf.tex|, |cdocsfn1.tex|, |cdocsfn2.tex|
as well as |childdoc.pdf|.

%%%%%%%%%%%%%%%%%%%%%%%%%%%%%%%%%%%%%%%%%%%%%%%%%%%%%%%%%%%%%%%%%%%%%%%%%%%%%%%%
\subsection{Files and Installation}

The package consists of the files:
%
\begin{center}
\begin{tabular}{ll}
    |README.txt|   & readme file \\
    |childdoc.ins| & installation file \\
    |childdoc.dtx| & source file \\
    |childdoc.def| & definition file \\
    |cdocsamp.tex| & sample main file \\
    |cdocsch1.tex| & sample include file \\
    |cdocsch2.tex| & sample include file \\
    |cdocspt3.tex| & sample part file \\
    |cdocspt4.tex| & sample part file \\
    |cdocsdrf.tex| & sample redirection file \\
    |cdocsfn1.tex| & sample redirection file \\
    |cdocsfn2.tex| & sample redirection file \\
    |childdoc.pdf| & manual
\end{tabular}
\end{center}
%
The distribution consists of the files
|README.txt|, |childdoc.ins| and |childdoc.dtx|.
%
\begin{itemize}
\item
Run (pdf)\LaTeX{} on |childdoc.dtx|
to compile the manual |childdoc.pdf| (this file).
\item
Run \LaTeX{} on |childdoc.ins| to create the definitions file |childdoc.def|
and the sample |cdocsamp.tex| with include files
|cdocsch1.tex|, |cdocsch2.tex|, |cdocspt3.tex|, |cdocspt4.tex|,
|cdocsdrf.tex|, |cdocsfn1.tex|, |cdocsfn2.tex|.
Then copy the file |childdoc.def| to an appropriate directory of your \LaTeX{}
distribution, e.g.\ \textit{texmf-root}|/tex/latex/childdoc|.
\end{itemize}

%%%%%%%%%%%%%%%%%%%%%%%%%%%%%%%%%%%%%%%%%%%%%%%%%%%%%%%%%%%%%%%%%%%%%%%%%%%%%%%%
\subsection{Related CTAN Packages}

There are several other packages which offer a similar functionality:
%
\begin{itemize}
\item
The packages
\href{http://ctan.org/pkg/docmute}{\textsf{docmute}},
\href{http://ctan.org/pkg/includex}{\textsf{includex}} and
\href{http://ctan.org/pkg/standalone}{\textsf{standalone}}
provide commands to include only the document body of
a child file thus allowing both files to be compiled individually.
\item
The packages \href{http://ctan.org/pkg/subdocs}{\textsf{subdocs}}
and \href{http://ctan.org/pkg/subfiles}{\textsf{subfiles}}
provide structures in which the main and child documents can be
encapsulated and allowing them to be compiled individually.
The inclusion mechanism is different from the conventional |\include|.
\item
The package \href{http://ctan.org/pkg/combine}{\textsf{combine}}
is an elaborate solution to combine several documents into one.
\end{itemize}
%
See also the CTAN topic \href{http://ctan.org/topic/subdocs}{\textsf{subdocs}}
for further related packages.
The present package differs from the above solutions in that
a document structure constructed with the conventional |\include| mechanism
just needs two extra commands at the top of every file
such that all constituent files can be compiled individually.

%%%%%%%%%%%%%%%%%%%%%%%%%%%%%%%%%%%%%%%%%%%%%%%%%%%%%%%%%%%%%%%%%%%%%%%%%%%%%%%%
%\subsection{Feature Suggestions}
%
%The following is a list of features which may be useful for future
%versions of this package:
%%
%\begin{itemize}
%\item
%\ldots
%\end{itemize}

%%%%%%%%%%%%%%%%%%%%%%%%%%%%%%%%%%%%%%%%%%%%%%%%%%%%%%%%%%%%%%%%%%%%%%%%%%%%%%%%
\subsection{Revision History}

%%%%%%%%%%%%%%%%%%%%%%%%%%%%%%%%%%%%%%%%
\paragraph{v2.0:} 2018/12/30

\begin{itemize}
\item
immediate forward processing
\item
added |\childdocby| mechanism
\item
manual restructured
\end{itemize}

%%%%%%%%%%%%%%%%%%%%%%%%%%%%%%%%%%%%%%%%
\paragraph{v1.6:} 2018/01/17

\begin{itemize}
\item
application for development of include files
\item
corrections to manual
\end{itemize}

%%%%%%%%%%%%%%%%%%%%%%%%%%%%%%%%%%%%%%%%
\paragraph{v1.5:} 2017/05/21

\begin{itemize}
\item
more complete structuring introduced
\item
|\childdocof| introduced
\item
|\childdoc| renamed to |\childdocmain|
\item
|\childredirect| renamed to |\childdocforward| and |\childdocforwardprefix|
and functionality expanded
\end{itemize}

%%%%%%%%%%%%%%%%%%%%%%%%%%%%%%%%%%%%%%%%
\paragraph{v1.0:} 2017/04/27

\begin{itemize}
\item
manual and install package
\item
first version published on CTAN
\end{itemize}

%%%%%%%%%%%%%%%%%%%%%%%%%%%%%%%%%%%%%%%%
\paragraph{v0.6:} 2017/04/26

\begin{itemize}
\item
redirection mechanism added
\end{itemize}

%%%%%%%%%%%%%%%%%%%%%%%%%%%%%%%%%%%%%%%%
\paragraph{v0.5:} 2017/04/26

\begin{itemize}
\item
functionality in definition file
\end{itemize}


%%%%%%%%%%%%%%%%%%%%%%%%%%%%%%%%%%%%%%%%%%%%%%%%%%%%%%%%%%%%%%%%%%%%%%%%%%%%%%%%
%%%%%%%%%%%%%%%%%%%%%%%%%%%%%%%%%%%%%%%%%%%%%%%%%%%%%%%%%%%%%%%%%%%%%%%%%%%%%%%%
%%%%%%%%%%%%%%%%%%%%%%%%%%%%%%%%%%%%%%%%%%%%%%%%%%%%%%%%%%%%%%%%%%%%%%%%%%%%%%%%
\appendix

\settowidth\MacroIndent{\rmfamily\scriptsize 000\ }

 \DocInput{childdoc.dtx}

\end{document}
%</driver>
% \fi
%
% %%%%%%%%%%%%%%%%%%%%%%%%%%%%%%%%%%%%%%%%%%%%%%%%%%%%%%%%%%%%%%%%%%%%%%%%%%%%%%
% %%%%%%%%%%%%%%%%%%%%%%%%%%%%%%%%%%%%%%%%%%%%%%%%%%%%%%%%%%%%%%%%%%%%%%%%%%%%%%
% \section{Sample}
%\iffalse
%<*samplemain>
%\fi
%
% The following presents a sample document
% with two chapters, two parts, a title page,
% a compile flag as well as three forwarding files to set the flag.
% It consists of eight |.tex| files:
% \begin{center}
% \begin{tabular}{ll}
% |cdocsamp.tex|&main file\\
% |cdocsch1.tex|&include file for chapter 1\\
% |cdocsch2.tex|&include file for chapter 2\\
% |cdocspt3.tex|&include file for part 3\\
% |cdocspt4.tex|&include file for part 4\\
% |cdocsdrf.tex|&forwarding file for main file in draft mode\\
% |cdocsfi1.tex|&forwarding file for final version of chapter 1\\
% |cdocsfi2.tex|&forwarding file for final version of chapter 2\\
% \end{tabular}
% \end{center}
% Each of the eight files can be compiled directly by the \LaTeX{} compiler.
%
% %%%%%%%%%%%%%%%%%%%%%%%%%%%%%%%%%%%%%%
% \paragraph{Main File.}
%
% The main file is called |cdocsamp.tex|.
%
% Load the \textsf{childdoc} definitions and
% declare the filename for the main document:
%    \begin{macrocode}
\input{childdoc.def}
\childdocmain{}
%    \end{macrocode}

% Optional override for |\version| flag:
%    \begin{macrocode}
%%\ifchilddoc\else\providecommand{\version}{draft}\fi
%    \end{macrocode}

% Define the default values for the |\version| flag
% (|final| for the main file and |draft| for childs):
%    \begin{macrocode}
\ifchilddoc
\providecommand{\version}{draft}
\else
\providecommand{\version}{final}
\fi
%    \end{macrocode}

% Load the standard document class:
%    \begin{macrocode}
\documentclass[12pt]{article}
%    \end{macrocode}

% Start the document body:
%    \begin{macrocode}
\begin{document}
%    \end{macrocode}

% Declare a title page.
% Print title, part of document being processed and version flag:
%    \begin{macrocode}
\addtocounter{page}{-1}
\begin{center}
{\LARGE\bfseries{}childdoc example\par}
\vspace{1cm}
\ifchilddoc
\ifchilddocmanual part\else chapter\fi:
`\childdocname' of `\childdocjob'\par
\else
main document: `\childdocjob'\par
\fi
version: \version\par
\end{center}
\newpage
%    \end{macrocode}

% Manually include selected file,
% otherwise process as usual:
%    \begin{macrocode}
\ifchilddocmanual
\section*{part `\childdocname'}
\input{\childdocname}
\else
%    \end{macrocode}

% Include the two chapters:
%    \begin{macrocode}
\include{cdocsch1}
\include{cdocsch2}
%    \end{macrocode}

% Include the two parts unless only chapters should be displayed:
%    \begin{macrocode}
\ifchilddoc\else
\section{part three}
\input{cdocspt3}
\section{part four}
\input{cdocspt4}
\fi
%    \end{macrocode}

% Process as usual until here:
%    \begin{macrocode}
\fi
%    \end{macrocode}

% End of document body:
%    \begin{macrocode}
\end{document}
%    \end{macrocode}
%\iffalse
%</samplemain>
%\fi
%
% %%%%%%%%%%%%%%%%%%%%%%%%%%%%%%%%%%%%%%
% \paragraph{Chapter Include Files.}
%
% The include files are called |cdocsch1.tex| and |cdocsch2.tex|.
%
%\iffalse
%<*samplechap1|samplechap2>
%\fi

% Optional override for |\version| flag:
%    \begin{macrocode}
%%\providecommand{\version}{final}
%    \end{macrocode}

% Include the main document:
%    \begin{macrocode}
\input{childdoc.def}
\childdocof{cdocsamp}
%    \end{macrocode}

%\iffalse
%</samplechap1|samplechap2>
%\fi
%
%\iffalse
%<*samplechap1>
%\fi
% Some text for chapter 1:
%    \begin{macrocode}
\section{one}
some text in chapter one
%    \end{macrocode}

%\iffalse
%</samplechap1>
%\fi
% Some text for chapter 2:
%\iffalse
%<*samplechap2>
%\fi
%    \begin{macrocode}
\section{two}
more text in chapter two
%    \end{macrocode}

%\iffalse
%</samplechap2>
%\fi
%
% %%%%%%%%%%%%%%%%%%%%%%%%%%%%%%%%%%%%%%
% \paragraph{Part Include Files.}
%
% The include files are called |cdocspt3.tex| and |cdocspt4.tex|.
%
%\iffalse
%<*samplepart3|samplepart4>
%\fi

% Optional override for |\version| flag:
%    \begin{macrocode}
%%\providecommand{\version}{final}
%    \end{macrocode}

% Include the main document:
%    \begin{macrocode}
\input{childdoc.def}
\childdocby{cdocsamp}
%    \end{macrocode}

%\iffalse
%</samplepart3|samplepart4>
%\fi
%
%\iffalse
%<*samplepart3>
%\fi
% Some text for part 3:
%    \begin{macrocode}
some text in part three
%    \end{macrocode}

%\iffalse
%</samplepart3>
%\fi
% Some text for part 4:
%\iffalse
%<*samplepart4>
%\fi
%    \begin{macrocode}
more text in part four
%    \end{macrocode}

%\iffalse
%</samplepart4>
%\fi
%
% %%%%%%%%%%%%%%%%%%%%%%%%%%%%%%%%%%%%%%
% \paragraph{Forwarding for a Complete Draft.}
%
% The following forwarding file |cdocsdrf.tex|
% compiles the main document in draft mode:
%\iffalse
%<*sampledraft>
%\fi
%    \begin{macrocode}
\def\version{draft}
\input{childdoc.def}
\childdocforward{cdocsamp}
%    \end{macrocode}

%\iffalse
%</sampledraft>
%\fi
%
% %%%%%%%%%%%%%%%%%%%%%%%%%%%%%%%%%%%%%%
% \paragraph{Forwarding for Final Version of the Chapters.}
%
% The following forwarding files |cdocsfn1.tex| and |cdocsfn2.tex|
% (with identical content)
% compile the final versions of the child documents
% |cdocsch1.tex| and |cdocsch2.tex|, respectively:
%\iffalse
%<*samplefinal>
%\fi
%    \begin{macrocode}
\def\version{final}
\input{childdoc.def}
\childdocforwardprefix[cdocsamp]{cdocsfn}{cdocsch}
%    \end{macrocode}

%\iffalse
%</samplefinal>
%\fi
%
% %%%%%%%%%%%%%%%%%%%%%%%%%%%%%%%%%%%%%%
% \paragraph{Command Line Processing.}
%
% The following three command lines generate the output files
% |cdocscld|, |cdocscl1| and |cdocscl2|
% which should be identical to
% |cdocsdrf|, |cdocsch1| and |cdocsfn2|, respectively:
% \begin{center}
% \begin{tabular}{l}
% |latex -jobname cdocscld \|\\
% |  "\def\version{draft}\input{childdoc.def}\childdocforward{cdocsamp}"|\\
% |latex -jobname cdocscl1 \|\\
% |  "\input{childdoc.def}\childdocforward[cdocsamp]{cdocsch1}"|\\
% |latex -jobname cdocscl2 \|\\
% |  "\def\version{final}\input{childdoc.def}\childdocforward{cdocsch2}"|
% \end{tabular}
% \end{center}
% Note that the trailing backslash on each first line
% merely continues the input to the second line
% (for convenient cut ant paste).
% Furthermore, the command |latex| can be replaced by any
% of its alternative versions such as |pdflatex|.
%
% %%%%%%%%%%%%%%%%%%%%%%%%%%%%%%%%%%%%%%%%%%%%%%%%%%%%%%%%%%%%%%%%%%%%%%%%%%%%%%
% %%%%%%%%%%%%%%%%%%%%%%%%%%%%%%%%%%%%%%%%%%%%%%%%%%%%%%%%%%%%%%%%%%%%%%%%%%%%%%
% \section{Implementation}
%\iffalse
%<*package>
%\fi
%
% This section describes the definitions file |childdoc.def|.

% The definitions cannot be loaded using |\usepackage| or |\RequirePackage|
% which has a mechanism to prevent loading a style file more than once.
% When loading the definitions by means of |\input|
% multiple instances have to be prevented manually:
%\iffalse
%This code needs to be before the `\ProvidesFile' directive
%which is defined at the beginning of this file.
%Therefore it is also placed there and commented out here.
%</package>
%<*discard>
%\fi
%    \begin{macrocode}
\ifdefined\childdocmain\endinput\fi
%    \end{macrocode}
%\iffalse
%</discard>
%<*package>
%\fi
%
% \macro{\ifchilddoc}
% \macro{\ifchilddocmanual}
% The conditional |\ifchilddoc| tells whether a
% child (true) or main (false) document is being compiled.
% The conditional |\ifchilddocmanual| tells whether
% the |\includeonly| mechanism is used (false) or
% the selection of child files must be performed manually (true).
% The definitions initialise to false:
%    \begin{macrocode}
\newif\ifchilddoc
\newif\ifchilddocmanual
%    \end{macrocode}

% \macro{\childdocname}
% \macro{\childdocjob}
% The macro |\childdocname| stores the name of the main document
% to be compiled. The macro |\childdocjob| stores the name of
% the document on which the \LaTeX{} compiler was originally invoked.
% The content of |\jobname| cannot be compared
% to filenames specified in the source due to different catcodes.
% The following code rescans |\jobname|, stores the result
% in |\childdocname| and saves a copy in |\childdocjob|:
%    \begin{macrocode}
\edef\childdocname{\scantokens\expandafter{\jobname\noexpand}}
\let\childdocjob\childdocname
%    \end{macrocode}

% \macro{\childdocdisable}
% The macro |\childdocdisable| prevents the main file
% from being processed more than once.
% At this stage, the main document command |\childdocmain|
% is assumed to be called once again where it should do nothing.
% Any subsequent call to it should prevent
% a secondary processing of the main document
% It overwrites the forwarding commands
% |\childdocof| and |\childdocforward|
% with empty macros to prevent further inclusions of the main document:
%    \begin{macrocode}
\newcommand{\childdocdisable}
{
  \renewcommand{\childdocmain}[1]{\renewcommand{\childdocmain}[1]{\endinput}}
  \renewcommand{\childdocof}[1]{}
  \renewcommand{\childdocby}[2][]{}
  \renewcommand{\childdocforward}[2][]{}
  \renewcommand{\childdocdisable}{}
}
%    \end{macrocode}

% \macro{\childdocmain}
% The macro |\childdocmain| is to be called at the top of the main file
% with nothing or the main filename (without extension) as argument.
% First, it breaks loops.
% If the argument is not empty and does not match |\childdocname|
% (which is set by the first inclusion of |childdoc.def|),
% |\ifchilddoc| is set to true, |\includeonly| is applied to the child file
% and |\jobname| is set to the main file
% (for proper handling of |.aux| files):
%    \begin{macrocode}
\newcommand{\childdocmain}[1]
{
  \childdocdisable\childdocmain{}
  \if?#1?\else
    \begingroup
      \def\childdoctmp{#1}
      \ifx\childdoctmp\childdocname
        \def\childdoctmp{}
      \else
        \def\childdoctmp
        {
          \childdoctrue
          \includeonly{\childdocname}
          \def\childdocjob{#1}
          \def\jobname{#1}
        }
      \fi
      \expandafter
    \endgroup
    \childdoctmp
  \fi
}
%    \end{macrocode}

% \macro{\childdocof}
% The command |\childdocof| redirects
% compilation to the main file |#1|.
%    \begin{macrocode}
\newcommand{\childdocof}[1]
{
  \childdocdisable
  \childdoctrue
  \includeonly{\childdocname}
  \def\jobname{#1}
  \def\childdocjob{#1}
  \input{#1}
}
%    \end{macrocode}

% \macro{\childdocby}
% The command |\childdocby| ....
%    \begin{macrocode}
\newcommand{\childdocby}[2][]
{
  \childdocdisable
  \childdoctrue
  \childdocmanualtrue
  \if?#1?\else
    \def\jobname{#2}
  \fi
  \def\childdocjob{#2}
  \input{#2}
  \endinput
}
%    \end{macrocode}

% \macro{\childdocforward}
% The command |\childdocforward| redirects
% compilation to the main file or
% (if the optional argument is given) a child file.
% Parameters are set as if the main file
% or a child file starting with |\childdocof| was compiled.
% Then compilation is handed over to the main file:
%    \begin{macrocode}
\newcommand{\childdocforward}[2][]
{
  \begingroup
    \if?#1?
      \def\childdoctmp
      {
        \def\childdocname{#2}
        \def\childdocjob{#2}
        \def\jobname{#2}
        \input{#2}
        \endinput
      }
    \else
      \def\childdoctmp
      {
        \childdocdisable
        \def\childdocname{#2}
        \childdoctrue
        \includeonly{#2}
        \def\childdocjob{#1}
        \def\jobname{#1}
        \input{#1}
        \endinput
      }
    \fi
    \expandafter
  \endgroup
  \childdoctmp
}
%    \end{macrocode}

% \macro{\childdocforwardprefix}
% The command |\childdocforwardprefix| redirects
% compilation to the main or a child file by means of a pattern.
% The prefix |#1| in the current filename is replaced by |#2|
% and the suffix of the current filename is kept
% (it is assumed that the filename does not contain the substring `|~~~|'
% which is used as a delimiter).
% Compilation is handed over to the new file by |\childdocforward|:
%    \begin{macrocode}
\newcommand{\childdocforwardprefix}[3][]
{
  \begingroup
    \def\childdocextract #2##1~~~{\def\childdoctmp{\childdocforward[#1]{#3##1}}}
    \expandafter\childdocextract\childdocname~~~
    \expandafter
  \endgroup
  \childdoctmp
}
%    \end{macrocode}

% \macro{\childdoc}
% The deprecated macro |\childdoc| is a legacy version of |\childdocmain|:
%    \begin{macrocode}
\newcommand{\childdoc}{\childdocmain}
%    \end{macrocode}

% \macro{\childdocredirect}
% The deprecated macro |\childdocredirect| is a legacy version
% of |\childdocforward| and |\childdocforwardprefix|:
%    \begin{macrocode}
\newcommand{\childdocredirect}[2][]
{
  \begingroup
    \if?#1?
      \def\childdoctmp{\childdocforward{#2}}
    \else
      \def\childdoctmp{\childdocforwardprefix{#1}{#2}}
    \fi
    \expandafter
  \endgroup
  \childdoctmp
}
%    \end{macrocode}

%\iffalse
%</package>
%\fi
%
\endinput

\childdocmain{}
%    \end{macrocode}

% Optional override for |\version| flag:
%    \begin{macrocode}
%%\ifchilddoc\else\providecommand{\version}{draft}\fi
%    \end{macrocode}

% Define the default values for the |\version| flag
% (|final| for the main file and |draft| for childs):
%    \begin{macrocode}
\ifchilddoc
\providecommand{\version}{draft}
\else
\providecommand{\version}{final}
\fi
%    \end{macrocode}

% Load the standard document class:
%    \begin{macrocode}
\documentclass[12pt]{article}
%    \end{macrocode}

% Start the document body:
%    \begin{macrocode}
\begin{document}
%    \end{macrocode}

% Declare a title page.
% Print title, part of document being processed and version flag:
%    \begin{macrocode}
\addtocounter{page}{-1}
\begin{center}
{\LARGE\bfseries{}childdoc example\par}
\vspace{1cm}
\ifchilddoc
\ifchilddocmanual part\else chapter\fi:
`\childdocname' of `\childdocjob'\par
\else
main document: `\childdocjob'\par
\fi
version: \version\par
\end{center}
\newpage
%    \end{macrocode}

% Manually include selected file,
% otherwise process as usual:
%    \begin{macrocode}
\ifchilddocmanual
\section*{part `\childdocname'}
\input{\childdocname}
\else
%    \end{macrocode}

% Include the two chapters:
%    \begin{macrocode}
\include{cdocsch1}
\include{cdocsch2}
%    \end{macrocode}

% Include the two parts unless only chapters should be displayed:
%    \begin{macrocode}
\ifchilddoc\else
\section{part three}
\input{cdocspt3}
\section{part four}
\input{cdocspt4}
\fi
%    \end{macrocode}

% Process as usual until here:
%    \begin{macrocode}
\fi
%    \end{macrocode}

% End of document body:
%    \begin{macrocode}
\end{document}
%    \end{macrocode}
%\iffalse
%</samplemain>
%\fi
%
% %%%%%%%%%%%%%%%%%%%%%%%%%%%%%%%%%%%%%%
% \paragraph{Chapter Include Files.}
%
% The include files are called |cdocsch1.tex| and |cdocsch2.tex|.
%
%\iffalse
%<*samplechap1|samplechap2>
%\fi

% Optional override for |\version| flag:
%    \begin{macrocode}
%%\providecommand{\version}{final}
%    \end{macrocode}

% Include the main document:
%    \begin{macrocode}
% \iffalse
%
% childdoc.dtx Copyright (C) 2017-2018 Niklas Beisert
%
% This work may be distributed and/or modified under the
% conditions of the LaTeX Project Public License, either version 1.3
% of this license or (at your option) any later version.
% The latest version of this license is in
%   http://www.latex-project.org/lppl.txt
% and version 1.3 or later is part of all distributions of LaTeX
% version 2005/12/01 or later.
%
% This work has the LPPL maintenance status `maintained'.
%
% The Current Maintainer of this work is Niklas Beisert.
%
% This work consists of the files childdoc.dtx and childdoc.ins
% and the derived files childdoc.def and cdocsamp.tex with
% cdocsch1.tex, cdocsch2.tex, cdocsdrf.tex, cdocsfn1.tex, cdocsfn2.tex.
%
%<package>\ifdefined\childdocmain\endinput\fi
%<package>\ProvidesFile{childdoc.def}[2018/12/30 v2.0 child document driver]
%<samplemain>\ProvidesFile{cdocsamp.tex}[2018/12/30 v2.0 sample for childdoc]
%<*driver>
%\ProvidesFile{childdoc.drv}[2018/12/30 v2.0 childdoc reference manual file]
\PassOptionsToClass{10pt,a4paper}{article}
\documentclass{ltxdoc}

\usepackage[margin=35mm]{geometry}
\usepackage{hyperref}
\usepackage{hyperxmp}
\usepackage[usenames]{color}

\hypersetup{colorlinks=true}
\hypersetup{pdfstartview=FitH}
\hypersetup{pdfpagemode=UseNone}
\hypersetup{pdfsource={}}
\hypersetup{pdflang={en-UK}}
\hypersetup{pdfcopyright={Copyright 2017-2018 Niklas Beisert.
  This work may be distributed and/or modified under the
  conditions of the LaTeX Project Public License, either version 1.3
  of this license or (at your option) any later version.}}
\hypersetup{pdflicenseurl={http://www.latex-project.org/lppl.txt}}
\hypersetup{pdfcontactaddress={ETH Zurich, ITP, HIT K,
  Wolfgang-Pauli-Strasse 27}}
\hypersetup{pdfcontactpostcode={8093}}
\hypersetup{pdfcontactcity={Zurich}}
\hypersetup{pdfcontactcountry={Switzerland}}
\hypersetup{pdfcontactemail={nbeisert@itp.phys.ethz.ch}}
\hypersetup{pdfcontacturl={http://people.phys.ethz.ch/\xmptilde nbeisert/}}

\newcommand{\secref}[1]{\hyperref[#1]{section \ref*{#1}}}

\parskip1ex
\parindent0pt
\let\olditemize\itemize
\def\itemize{\olditemize\parskip0pt}

\begin{document}

\title{The \textsf{childdoc} Package}
\hypersetup{pdftitle={The childdoc Package}}
\author{Niklas Beisert\\[2ex]
  Institut f\"ur Theoretische Physik\\
  Eidgen\"ossische Technische Hochschule Z\"urich\\
  Wolfgang-Pauli-Strasse 27, 8093 Z\"urich, Switzerland\\[1ex]
  \href{mailto:nbeisert@itp.phys.ethz.ch}
  {\texttt{nbeisert@itp.phys.ethz.ch}}}
\hypersetup{pdfauthor={Niklas Beisert}}
\hypersetup{pdfsubject={Manual for the LaTeX2e Package childdoc}}
\date{30 December 2018, \textsf{v2.0}}
\maketitle

\begin{abstract}\noindent
\textsf{childdoc} is a \LaTeXe{} package
that enables the direct compilation
of document sections included by |\include|
to individual files.
\end{abstract}

\begingroup
\parskip0ex
\tableofcontents
\endgroup

%%%%%%%%%%%%%%%%%%%%%%%%%%%%%%%%%%%%%%%%%%%%%%%%%%%%%%%%%%%%%%%%%%%%%%%%%%%%%%%%
%%%%%%%%%%%%%%%%%%%%%%%%%%%%%%%%%%%%%%%%%%%%%%%%%%%%%%%%%%%%%%%%%%%%%%%%%%%%%%%%
\section{Introduction}

\LaTeX{} provides a mechanism to structure a large document (such as a book)
into a main file and several child files (containing the chapters)
using the |\include| command.
This mechanism is beneficial for documents
which span hundreds of pages in order to
make the source file(s) more manageable.
Moreover, compilation can be restricted to
selected child files by means of the |\includeonly| command.
The latter feature can be used to reduce the compilation time while editing
(this was significantly more useful in the earlier days of \LaTeX{})
or to generate a smaller document which is easier to navigate.
Another application of |\includeonly| is to generate
documents consisting of selected parts of the complete document.

However, there are a few drawbacks of the plain |\include| mechanism:
\begin{itemize}
\item
The child files cannot be compiled on their own,
they can only be compiled via the main file.
A naive editing environment
(such as a text editor with an option
to have the current file processed by \LaTeX)
may require one to switch to the main file before compiling;
attempting to compile the child file produces errors.
\item
The main file must be modified (each time)
to adjust the |\includeonly| command
to the present needs. This easily leaves the main file in a messy state.
\item
The generated document will always carry the filename
of the main document. This is inconvenient if
several child files are to be compiled and
to be kept for distribution.
\end{itemize}

The present package provides a simple interface
to make child files individually compilable by \LaTeX{}.
Compiling a child file then has the same effect as compiling
the main file with an |\includeonly| command
to select the appropriate child.
Moreover the generated document will carry the name of the child
rather than the main file.
This resolves all three above issues.

This feature is meant to make the editing of books,
thesis documents and lecture notes somewhat more convenient.
However, the package can also be used efficiently for
composing a series of documents (such as exercise sheets)
which are typically distributed individually.
It then assists the author in generating the individual documents
(potentially in different versions)
as well as a document containing the collected series.
Another application is in developing style files
or other kinds of included material
where compilation of the style file could redirect
to a sample or test file.

%%%%%%%%%%%%%%%%%%%%%%%%%%%%%%%%%%%%%%%%%%%%%%%%%%%%%%%%%%%%%%%%%%%%%%%%%%%%%%%%
%%%%%%%%%%%%%%%%%%%%%%%%%%%%%%%%%%%%%%%%%%%%%%%%%%%%%%%%%%%%%%%%%%%%%%%%%%%%%%%%
\section{Usage}

First of all, the package \textsf{childdoc} is \emph{not} a standard
\LaTeXe{} |.sty| style file! Therefore it needs to be invoked in
a non-standard way.

%%%%%%%%%%%%%%%%%%%%%%%%%%%%%%%%%%%%%%%%%%%%%%%%%%%%%%%%%%%%%%%%%%%%%%%%%%%%%%%%
\subsection{Included Files}
\label{sec:include}

%%%%%%%%%%%%%%%%%%%%%%%%%%%%%%%%%%%%%%%%
\DescribeMacro{\childdocmain}
To use the package, add the commands
\begin{center}
\begin{tabular}{l}
|\input{childdoc.def}|\\
|\childdocmain{}|\\
\end{tabular}
\end{center}
at the very top of the main \LaTeX{} file,
in particular \emph{before} the |\documentclass| statement!
The argument of |\childdocmain| should be left empty
(but it must be present).

%%%%%%%%%%%%%%%%%%%%%%%%%%%%%%%%%%%%%%%%
\DescribeMacro{\childdocof}
Furthermore, add the commands
\begin{center}
\begin{tabular}{l}
|\input{childdoc.def}|\\
|\childdocof{|\textit{main}|}|\\
\end{tabular}
\end{center}
at the top of every child file \textit{child}
which is included by |\include{|\textit{child}|}|
from within the main file
(or at least for those files to be compiled individually).
The argument \textit{main} must be the filename of the main file.

There are a couple of
considerations in setting up the main and child documents:

%%%%%%%%%%%%%%%%%%%%%%%%%%%%%%%%%%%%%%%%
\paragraph{Restrictions.}

Please note the following restrictions:
\begin{itemize}
\item
|\childdocmain| must be called with one argument \textit{main}
to ensure compatibility with earlier version of the package.
It must either be empty (|\childdocmain{}|)
or precisely match the filename of the main file in which it is specified.
See \secref{sec:detection} for further information.
\item
The filename \textit{main} must be specified without the |.tex| extension.
\item
The filename \textit{main} is case sensitive
(even in case-insensitive file systems)
due to internal string comparison.
\item
The argument \textit{main} should be fully expanded, it cannot be a macro.
\item
Subdirectories and special characters should be avoided in filenames.
\item
The command |\childdocmain{|\textit{main}|}| must be followed by a whitespace.
It should not be followed immediately by another command
or by a comment mark `|%|'.
This is because the \TeX{} parser reads the token immediately following
the argument of |\childdocmain| and puts it
at the beginning of every child section;
however, a white\-space is ignored.
\end{itemize}

%%%%%%%%%%%%%%%%%%%%%%%%%%%%%%%%%%%%%%%%
\paragraph{Content of Main File.}

It is advisable to place all content in the child files included by |\include|.
Any output contained in the main file will appear in all child documents
unless suppressed manually;
it cannot be suppressed automatically by the |\includeonly| directive
and thus should normally be avoided.
A method to include some content in the main file
by means of conditional processing is described in \secref{sec:conditional}.

%%%%%%%%%%%%%%%%%%%%%%%%%%%%%%%%%%%%%%%%
\paragraph{Page Numbering.}

When only a part of the document is compiled,
the appropriate numbering of pages
(as well as other status parameters)
is determined from the |.aux| files.
The latter contain information from previous passes.
However this information needs to propagate through
all intermediate child documents.
Therefore the page numbering in child documents may well
be inconsistent until the complete document is compiled at least once.

A useful (if unconventional) way to always ensure a consistent
page numbering is to restart the numbering in each child document
and denote the pages by `\textit{child}|.|\textit{page}'
where \textit{child} represents the chapter/section number of the child file.
This can be achieved by the command
|\numberwithin{page}{|\textit{child}|}|
of the \textsf{amsmath} package
where \textit{child} can be |chapter| or |section|
depending on the chosen structuring.
Alternatively, one can modify the macro |\thepage| appropriately
and reset the counter |page| at the start of each child file.

%%%%%%%%%%%%%%%%%%%%%%%%%%%%%%%%%%%%%%%%%%%%%%%%%%%%%%%%%%%%%%%%%%%%%%%%%%%%%%%%
\subsection{Conditional Processing}
\label{sec:conditional}

The package provides a mechanism to compile different versions
of a document. To customise the versions further some conditional processing
can come in handy to distinguish which version is being compiled.
The package provides two macros to describe the compilation context:

%%%%%%%%%%%%%%%%%%%%%%%%%%%%%%%%%%%%%%%%
\DescribeMacro{\ifchilddoc}
The conditional |\ifchilddoc| distinguishes between the compilation of
child documents and the main document:
%
\begin{center}
|\ifchilddoc |\textit{child-code}| |[|\||else |\textit{main-code}]| \||fi|
\end{center}

%%%%%%%%%%%%%%%%%%%%%%%%%%%%%%%%%%%%%%%%
\DescribeMacro{\childdocname}
\DescribeMacro{\childdocjob}
The macro |\childdocname| contains the filename (without extension)
of the main or child file being processed.
Note that |\childdocjob| will always contain the name of the main file.

%%%%%%%%%%%%%%%%%%%%%%%%%%%%%%%%%%%%%%%%
\paragraph{Title Page.}

Conditional processing can be used to include a title or banner page
in the main document when proper precautions are taken.
Importantly, the code in the main file should ensure that the page counter
(as well as other status parameters which are stored in the |.aux| files)
takes the same value after the conditional processing.
Otherwise the page numbers may take divergent values
depending on which part is compiled.

For example, a title page could be declared by:
%
\begin{center}
\begin{tabular}{l}
|\ifchilddoc\||else|\\
|\addtocounter{page}{-1}|\\
\textit{code for title page}\\
|\newpage|\\
|\||fi|
\end{tabular}
\end{center}
%
A banner page for the child documents can be generated by:
%
\begin{center}
\begin{tabular}{l}
|\ifchilddoc|\\
|\addtocounter{page}{-1}|\\
\textit{code for banner page}\\
|\newpage|\\
|\||fi|
\end{tabular}
\end{center}
%
Here one could write a message such as:
\begin{center}
|This is the part \childdocname{} of \childdocjob{}.|
\end{center}

%%%%%%%%%%%%%%%%%%%%%%%%%%%%%%%%%%%%%%%%%%%%%%%%%%%%%%%%%%%%%%%%%%%%%%%%%%%%%%%%
\subsection{Flags}
\label{sec:flags}

The package makes it easy to generate different versions
of the main or child documents.
To this end compilation flags can be defined
and assigned different default values.
They will be particularly useful in conjunction
with the forwarding mechanism described in \secref{sec:forward}.

For example, it may be useful to have a flag |\version|
which can be set to |draft| or |final|.
The document source will contain some conditional code
depending on the value of |\version|.
Suppose further, the flag should default to |final| for the main file
and to |draft| for child files
which is a natural assignment for editing the document.
This is achieved by placing the following code
in the preamble of the main document
(below the |\childdocmain| directive):
%
\begin{center}
\begin{tabular}{l}
|\ifchilddoc|\\
|\providecommand{\version}{draft}|\\
|\||else|\\
|\providecommand{\version}{final}|\\
|\||fi|
\end{tabular}
\end{center}
%
The definition by |\providecommand| makes sure
that previous definitions are not overwritten.
Further statements |\providecommand{\version}{...}|
can thus be added before the above code to override it.

For the main file, one might add a line
(between |\childdocmain| and the above block)
%
\begin{center}
|%\ifchilddoc\||else\providecommand{\version}{draft}\||fi|
\end{center}
%
which can be uncommented to produce a draft version.
Likewise one can add a line to the very top of a child file
(above the |\childdocof{|\textit{main}|}| directive)
%
\begin{center}
|%\providecommand{\version}{final}|
\end{center}
%
which can be uncommented to produce the final version of this child document.

%%%%%%%%%%%%%%%%%%%%%%%%%%%%%%%%%%%%%%%%%%%%%%%%%%%%%%%%%%%%%%%%%%%%%%%%%%%%%%%%
\subsection{Forwarding}
\label{sec:forward}

Different versions of the main or child documents
using compilation flags as described in \secref{sec:flags}
can be (permanently) stored in different files
for convenient compilation, viewing and distribution.
To this end, the package defines a command
to pass on compilation to a different file:

%%%%%%%%%%%%%%%%%%%%%%%%%%%%%%%%%%%%%%%%
\DescribeMacro{\childdocforward}
The command |\childdocforward| redirects processing to
another source file:
%
\begin{center}
\begin{tabular}{l}
|\input{childdoc.def}|\\
|\childdocforward[|\textit{main}|]{|\textit{dest}|}|\\
\end{tabular}
\end{center}
%
The argument \textit{dest} is the destination file
(without extension).
It should be the main file or one of the child files.
Note that further \textsf{childdoc} directives
such as |\childdocof| and |\childdocforward|
in the indicated file will be processed in this form.
The optional argument \textit{main}
passes on directly to the main file \textit{main}
while pretending to compile the child \textit{dest}.
This form behaves as if \textit{dest}
issues |\childdocof{|\textit{main}|}| right away,
and no further \textsf{childdoc} directives will be processed.

%%%%%%%%%%%%%%%%%%%%%%%%%%%%%%%%%%%%%%%%
\DescribeMacro{\...prefix}
In the alternative form |\childdocforwardprefix|,
%
\begin{center}
\begin{tabular}{l}
|\input{childdoc.def}|\\
|\childdocforwardprefix[|\textit{main}|]{|\textit{prefix}|}{|\textit{dest}|}|
\end{tabular}
\end{center}
%
the destination file is determined by a pattern
depending on the current file:
To make this work, the current file must be called
`{\textit{prefix}\hspace{0.2em}\textit{suffix}}'
with \textit{prefix} matching precisely the argument.
Processing is then passed on to the file
`{\textit{dest}\hspace{0.2em}\textit{suffix}}'.
Surely, the same effect is achieved by
directly specifying the
argument `{\textit{dest}\hspace{0.2em}\textit{suffix}}'
in the first form.
However, that requires to set up a different file
for each child. With the alternative form of the command
all these files can have exactly the same content
which simplifies setting them up and maintaining them.

For example, the following file |draft.tex|
with a compilation flag |\version| as described in \secref{sec:flags}
compiles the main document as a draft:
%
\begin{center}
\begin{tabular}{l}
|\def\version{draft}|\\
|\input{childdoc.def}|\\
|\childdocforward{|\textit{main}|}|
\end{tabular}
\end{center}
%
Likewise, the following files |final|\textit{nn}|.tex|
compile the final version of the child document
|child|\textit{nn}|.tex|:
%
\begin{center}
\begin{tabular}{l}
|\def\version{final}|\\
|\input{childdoc.def}|\\
|\childdocforwardprefix{final}{child}|
\end{tabular}
\end{center}
%

Note that when several versions of a main file and/or of each child file
are to be generated, it may be convenient to set up a |Makefile| or
shell script to automatise the process.

%%%%%%%%%%%%%%%%%%%%%%%%%%%%%%%%%%%%%%%%%%%%%%%%%%%%%%%%%%%%%%%%%%%%%%%%%%%%%%%%
\subsection{Command Line Processing}
\label{sec:commandline}

The effect of redirection files can also be achieved by invoking
the \LaTeX{} compiler with a more elaborate command line.
Most conveniently this should be done as part
of a shell script or a |Makefile|.

When using \textsf{childdoc} in the main file, the following
command lines effectively perform a redirection
(note that depending on the shell being used,
backslashes may have to be doubled: `|\|' $\to$ `|\\|'):
%
\begin{center}
|... -jobname "|\textit{target}|" |\\|"|[\textit{flags}]%
|\input{childdoc.def}\childdocforward[|\textit{main}|]{|\textit{dest}|}"|
\end{center}
%
Here \textit{target} is the name of the output file,
\textit{main} is the name of the main file
and \textit{dest} is the name of the main or child file to be processed
(all filenames without extensions).
The optional argument \textit{main} can be omitted
if \textit{main} matches \textit{dest}.
Optionally, compilation \textit{flags} can be defined via |\def| commands.
This command line makes the \TeX{} engine believe
it is compiling the file \textit{target}
whose content is specified as the latter parameter.
The provided code then forwards the processing to
\textit{main} or \textit{dest} as described in \secref{sec:forward}.

%%%%%%%%%%%%%%%%%%%%%%%%%%%%%%%%%%%%%%%%%%%%%%%%%%%%%%%%%%%%%%%%%%%%%%%%%%%%%%%%
\subsection{Include by Input}
\label{sec:input}

Including child documents by |\include| has some restrictions by design.
Most notably, the content of a child document always occupies
its own set of pages; pages cannot be shared between child documents.
Usually, this behaviour makes perfect sense
because each child document contain an essential part of the document.
However, in some situations it may be desirable to compose
a document from a collection of parts
without having mandatory page breaks between then.
For this case, the package
provides a mechanism to include parts
by |\input| which can also be processed individually.
However, by construction this mechanism
requires manual handling of the content to be output.

%%%%%%%%%%%%%%%%%%%%%%%%%%%%%%%%%%%%%%%%
\DescribeMacro{\ifchilddocmanual}
The main file should be prepared as usual, see \secref{sec:include}.
However, the document body must make a distinction
between processing of an individual part and of the main document, e.g.:
%
\begin{center}
\begin{tabular}{l}
|\ifchilddocmanual|\\
|\input{\childdocname}|\\
|\||else|\\
\textit{document body with }|\input{|\textit{part}|}|\\
|\||fi|
\end{tabular}
\end{center}
%
The conditional |\ifchilddocmanual| is true whenever
a part to be included by |\input| is being compiled,
and the name of the part is stored in |\childdocname|.

%%%%%%%%%%%%%%%%%%%%%%%%%%%%%%%%%%%%%%%%
\DescribeMacro{\childdocby}
Each part to be included by |\input| should start with:
%
\begin{center}
\begin{tabular}{l}
|\input{childdoc.def}|\\
|\childdocby{|\textit{main}|}|\\
\end{tabular}
\end{center}
%
The directive |\childdocby| is similar to |\childdocof|
described in \secref{sec:include},
but the subsequent selection of content must be done manually.
To that end, both |\ifchilddoc| and |\ifchilddocmanual|
will be true upon processing of a part,
and the name of the part is stored in |\childdocname|.
Note that |\jobname| will be set to the filename of the current part
so that each part receives an individual |.aux| file
that does not interfere with the |.aux| file(s) of the main document.
This behaviour can be altered by the alternative form
|\childdocby[*]{|\textit{main}|}| (with a non-empty optional argument)
which uses the |.aux| file of the main document
by setting |\jobname| to \textit{main}.

%%%%%%%%%%%%%%%%%%%%%%%%%%%%%%%%%%%%%%%%%%%%%%%%%%%%%%%%%%%%%%%%%%%%%%%%%%%%%%%%
\subsection{Driver Development}
\label{sec:driver}

The \textsf{childdoc} mechanism can also be use for the development
of definition files such as \LaTeX{} styles or classes.
This case differs from the above setup with multiple parts
included by |\include| in that no |\includeonly| should be invoked.
This can be achieved by starting the include file
(before |\ProvidesPackage|) with:
%
\begin{center}
\begin{tabular}{l}
|\input{childdoc.def}|\\
|\childdocforward{|\textit{main}|}|\\
\end{tabular}
\end{center}
%
or alternatively with:
%
\begin{center}
\begin{tabular}{l}
|\input{childdoc.def}|\\
|\childdocby{|\textit{main}|}|\\
\end{tabular}
\end{center}
%
Both forms have slightly different effects as described above.
The main file is prepared as usual, see \secref{sec:include}.

%%%%%%%%%%%%%%%%%%%%%%%%%%%%%%%%%%%%%%%%%%%%%%%%%%%%%%%%%%%%%%%%%%%%%%%%%%%%%%%%
\subsection{Legacy Detection}
\label{sec:detection}

The directive |\childdocmain| in the main file can detect
whether the complete document or merely a child is to be compiled
even without using the directive |\childdocof|.
This method is deprecated because it is less robust
and there is no compelling reason to use it;
it is merely provided for backward compatibility
and it may be removed in future versions.

If the detection mechanism is to be used,
it is mandatory to correctly specify
the filename of the main file as the argument of |\childdocmain|:
%
\begin{center}
\begin{tabular}{l}
|\input{childdoc.def}|\\
|\childdocmain{|\textit{main}|}|\\
\end{tabular}
\end{center}
%
If |\jobname| does not match the argument \textit{main} of |\childdocmain|,
it is assumed that |\jobname| points to the child file to be compiled.
When using |\childdocmain| with the main file specified as argument,
it suffices to start a child file
with just |\input{|\textit{main}|}|
without loading of the package and using |\childdocof|.
If instead all processing is done
with the appropriate \textsf{childdoc} directives,
the argument of \textit{main} of |\childdocmain| can be empty.

An alternative version of the command line processing described
in \secref{sec:commandline} using the detection mechanism reads:
%
\begin{center}
|... -jobname "|\textit{target}|" "|[\textit{flags}]%
[|\def\jobname{|\textit{dest}|}|]|\input{|\textit{main}|}"|
\end{center}

%%%%%%%%%%%%%%%%%%%%%%%%%%%%%%%%%%%%%%%%%%%%%%%%%%%%%%%%%%%%%%%%%%%%%%%%%%%%%%%%
\subsection{Manual Code}
\label{sec:manual}

In case one cannot be certain whether the definitions file |childdoc.def|
is installed on the target \TeX{} distribution
and one prefers not to ship it,
it is conceivable to paste a few relevant commands into the sources.

To that end, drop all statements |\input{childdoc.def}|
and perform the replacements as outlined below.
Instead of |\childdocmain{|\textit{main}|}| add the following code
to the top of the main file:
%
\begin{center}
\begin{tabular}{l}
|\||ifdefined\childdocname\endinput\||fi\newif\ifchilddoc|\\
|\edef\childdocname{\scantokens\expandafter{\jobname\noexpand}}|\\
|\def\childdocmain{|\textit{main}|}\||ifx\childdocmain\childdocname\||else|\\
|\childdoctrue\includeonly{\childdocname}\let\jobname\childdocmain\||fi|\\
\end{tabular}
\end{center}
%
Instead of |\childdocof{|\textit{main}|}| just include the main file
at the top of each child file:
%
\begin{center}
|\input{|\textit{main}|}|
\end{center}
%
A simple redirection |\childdocforward{|\textit{dest}|}| is achieved by:
%
\begin{center}
|\def\jobname{|\textit{dest}|}\input{\jobname}|
\end{center}
%
The redirection with prefix
|\childdocforwardprefix[|\textit{prefix}|]{|\textit{dest}|}|
is accomplished by:
%
\begin{center}
\begin{tabular}{l}
|{\edef\jobname{\scantokens\expandafter{\jobname\noexpand}}|\\
|\def\redirectjob |\textit{prefix}|#1~~~{\gdef\jobname{|\textit{dest}|#1}}|\\
|\expandafter\redirectjob\jobname~~~}\input{\jobname}|
\end{tabular}
\end{center}

In an alternative approach,
child documents can be compiled by a specific command line
without additional code or specific definitions:
%
\begin{center}
|... -jobname "|\textit{target}|" "|[\textit{flags}]%
|\includeonly{|\textit{dest}|}\input{|\textit{main}|}"|
\end{center}
%

%%%%%%%%%%%%%%%%%%%%%%%%%%%%%%%%%%%%%%%%%%%%%%%%%%%%%%%%%%%%%%%%%%%%%%%%%%%%%%%%
%%%%%%%%%%%%%%%%%%%%%%%%%%%%%%%%%%%%%%%%%%%%%%%%%%%%%%%%%%%%%%%%%%%%%%%%%%%%%%%%
\section{Information}

%%%%%%%%%%%%%%%%%%%%%%%%%%%%%%%%%%%%%%%%%%%%%%%%%%%%%%%%%%%%%%%%%%%%%%%%%%%%%%%%
\subsection{Copyright}

Copyright \copyright{} 2017--2018 Niklas Beisert

This work may be distributed and/or modified under the
conditions of the \LaTeX{} Project Public License, either version 1.3
of this license or (at your option) any later version.
The latest version of this license is in
  \url{http://www.latex-project.org/lppl.txt}
and version 1.3 or later is part of all distributions of \LaTeX{}
version 2005/12/01 or later.

This work has the LPPL maintenance status `maintained'.

The Current Maintainer of this work is Niklas Beisert.

This work consists of the files |README.txt|, |childdoc.ins| and |childdoc.dtx|
as well as the derived files |childdoc.def|, |cdocsamp.tex|
with |cdocsch1.tex|, |cdocsch2.tex|, |cdocspt3.tex|, |cdocspt4.tex|,
|cdocsdrf.tex|, |cdocsfn1.tex|, |cdocsfn2.tex|
as well as |childdoc.pdf|.

%%%%%%%%%%%%%%%%%%%%%%%%%%%%%%%%%%%%%%%%%%%%%%%%%%%%%%%%%%%%%%%%%%%%%%%%%%%%%%%%
\subsection{Files and Installation}

The package consists of the files:
%
\begin{center}
\begin{tabular}{ll}
    |README.txt|   & readme file \\
    |childdoc.ins| & installation file \\
    |childdoc.dtx| & source file \\
    |childdoc.def| & definition file \\
    |cdocsamp.tex| & sample main file \\
    |cdocsch1.tex| & sample include file \\
    |cdocsch2.tex| & sample include file \\
    |cdocspt3.tex| & sample part file \\
    |cdocspt4.tex| & sample part file \\
    |cdocsdrf.tex| & sample redirection file \\
    |cdocsfn1.tex| & sample redirection file \\
    |cdocsfn2.tex| & sample redirection file \\
    |childdoc.pdf| & manual
\end{tabular}
\end{center}
%
The distribution consists of the files
|README.txt|, |childdoc.ins| and |childdoc.dtx|.
%
\begin{itemize}
\item
Run (pdf)\LaTeX{} on |childdoc.dtx|
to compile the manual |childdoc.pdf| (this file).
\item
Run \LaTeX{} on |childdoc.ins| to create the definitions file |childdoc.def|
and the sample |cdocsamp.tex| with include files
|cdocsch1.tex|, |cdocsch2.tex|, |cdocspt3.tex|, |cdocspt4.tex|,
|cdocsdrf.tex|, |cdocsfn1.tex|, |cdocsfn2.tex|.
Then copy the file |childdoc.def| to an appropriate directory of your \LaTeX{}
distribution, e.g.\ \textit{texmf-root}|/tex/latex/childdoc|.
\end{itemize}

%%%%%%%%%%%%%%%%%%%%%%%%%%%%%%%%%%%%%%%%%%%%%%%%%%%%%%%%%%%%%%%%%%%%%%%%%%%%%%%%
\subsection{Related CTAN Packages}

There are several other packages which offer a similar functionality:
%
\begin{itemize}
\item
The packages
\href{http://ctan.org/pkg/docmute}{\textsf{docmute}},
\href{http://ctan.org/pkg/includex}{\textsf{includex}} and
\href{http://ctan.org/pkg/standalone}{\textsf{standalone}}
provide commands to include only the document body of
a child file thus allowing both files to be compiled individually.
\item
The packages \href{http://ctan.org/pkg/subdocs}{\textsf{subdocs}}
and \href{http://ctan.org/pkg/subfiles}{\textsf{subfiles}}
provide structures in which the main and child documents can be
encapsulated and allowing them to be compiled individually.
The inclusion mechanism is different from the conventional |\include|.
\item
The package \href{http://ctan.org/pkg/combine}{\textsf{combine}}
is an elaborate solution to combine several documents into one.
\end{itemize}
%
See also the CTAN topic \href{http://ctan.org/topic/subdocs}{\textsf{subdocs}}
for further related packages.
The present package differs from the above solutions in that
a document structure constructed with the conventional |\include| mechanism
just needs two extra commands at the top of every file
such that all constituent files can be compiled individually.

%%%%%%%%%%%%%%%%%%%%%%%%%%%%%%%%%%%%%%%%%%%%%%%%%%%%%%%%%%%%%%%%%%%%%%%%%%%%%%%%
%\subsection{Feature Suggestions}
%
%The following is a list of features which may be useful for future
%versions of this package:
%%
%\begin{itemize}
%\item
%\ldots
%\end{itemize}

%%%%%%%%%%%%%%%%%%%%%%%%%%%%%%%%%%%%%%%%%%%%%%%%%%%%%%%%%%%%%%%%%%%%%%%%%%%%%%%%
\subsection{Revision History}

%%%%%%%%%%%%%%%%%%%%%%%%%%%%%%%%%%%%%%%%
\paragraph{v2.0:} 2018/12/30

\begin{itemize}
\item
immediate forward processing
\item
added |\childdocby| mechanism
\item
manual restructured
\end{itemize}

%%%%%%%%%%%%%%%%%%%%%%%%%%%%%%%%%%%%%%%%
\paragraph{v1.6:} 2018/01/17

\begin{itemize}
\item
application for development of include files
\item
corrections to manual
\end{itemize}

%%%%%%%%%%%%%%%%%%%%%%%%%%%%%%%%%%%%%%%%
\paragraph{v1.5:} 2017/05/21

\begin{itemize}
\item
more complete structuring introduced
\item
|\childdocof| introduced
\item
|\childdoc| renamed to |\childdocmain|
\item
|\childredirect| renamed to |\childdocforward| and |\childdocforwardprefix|
and functionality expanded
\end{itemize}

%%%%%%%%%%%%%%%%%%%%%%%%%%%%%%%%%%%%%%%%
\paragraph{v1.0:} 2017/04/27

\begin{itemize}
\item
manual and install package
\item
first version published on CTAN
\end{itemize}

%%%%%%%%%%%%%%%%%%%%%%%%%%%%%%%%%%%%%%%%
\paragraph{v0.6:} 2017/04/26

\begin{itemize}
\item
redirection mechanism added
\end{itemize}

%%%%%%%%%%%%%%%%%%%%%%%%%%%%%%%%%%%%%%%%
\paragraph{v0.5:} 2017/04/26

\begin{itemize}
\item
functionality in definition file
\end{itemize}


%%%%%%%%%%%%%%%%%%%%%%%%%%%%%%%%%%%%%%%%%%%%%%%%%%%%%%%%%%%%%%%%%%%%%%%%%%%%%%%%
%%%%%%%%%%%%%%%%%%%%%%%%%%%%%%%%%%%%%%%%%%%%%%%%%%%%%%%%%%%%%%%%%%%%%%%%%%%%%%%%
%%%%%%%%%%%%%%%%%%%%%%%%%%%%%%%%%%%%%%%%%%%%%%%%%%%%%%%%%%%%%%%%%%%%%%%%%%%%%%%%
\appendix

\settowidth\MacroIndent{\rmfamily\scriptsize 000\ }

 \DocInput{childdoc.dtx}

\end{document}
%</driver>
% \fi
%
% %%%%%%%%%%%%%%%%%%%%%%%%%%%%%%%%%%%%%%%%%%%%%%%%%%%%%%%%%%%%%%%%%%%%%%%%%%%%%%
% %%%%%%%%%%%%%%%%%%%%%%%%%%%%%%%%%%%%%%%%%%%%%%%%%%%%%%%%%%%%%%%%%%%%%%%%%%%%%%
% \section{Sample}
%\iffalse
%<*samplemain>
%\fi
%
% The following presents a sample document
% with two chapters, two parts, a title page,
% a compile flag as well as three forwarding files to set the flag.
% It consists of eight |.tex| files:
% \begin{center}
% \begin{tabular}{ll}
% |cdocsamp.tex|&main file\\
% |cdocsch1.tex|&include file for chapter 1\\
% |cdocsch2.tex|&include file for chapter 2\\
% |cdocspt3.tex|&include file for part 3\\
% |cdocspt4.tex|&include file for part 4\\
% |cdocsdrf.tex|&forwarding file for main file in draft mode\\
% |cdocsfi1.tex|&forwarding file for final version of chapter 1\\
% |cdocsfi2.tex|&forwarding file for final version of chapter 2\\
% \end{tabular}
% \end{center}
% Each of the eight files can be compiled directly by the \LaTeX{} compiler.
%
% %%%%%%%%%%%%%%%%%%%%%%%%%%%%%%%%%%%%%%
% \paragraph{Main File.}
%
% The main file is called |cdocsamp.tex|.
%
% Load the \textsf{childdoc} definitions and
% declare the filename for the main document:
%    \begin{macrocode}
\input{childdoc.def}
\childdocmain{}
%    \end{macrocode}

% Optional override for |\version| flag:
%    \begin{macrocode}
%%\ifchilddoc\else\providecommand{\version}{draft}\fi
%    \end{macrocode}

% Define the default values for the |\version| flag
% (|final| for the main file and |draft| for childs):
%    \begin{macrocode}
\ifchilddoc
\providecommand{\version}{draft}
\else
\providecommand{\version}{final}
\fi
%    \end{macrocode}

% Load the standard document class:
%    \begin{macrocode}
\documentclass[12pt]{article}
%    \end{macrocode}

% Start the document body:
%    \begin{macrocode}
\begin{document}
%    \end{macrocode}

% Declare a title page.
% Print title, part of document being processed and version flag:
%    \begin{macrocode}
\addtocounter{page}{-1}
\begin{center}
{\LARGE\bfseries{}childdoc example\par}
\vspace{1cm}
\ifchilddoc
\ifchilddocmanual part\else chapter\fi:
`\childdocname' of `\childdocjob'\par
\else
main document: `\childdocjob'\par
\fi
version: \version\par
\end{center}
\newpage
%    \end{macrocode}

% Manually include selected file,
% otherwise process as usual:
%    \begin{macrocode}
\ifchilddocmanual
\section*{part `\childdocname'}
\input{\childdocname}
\else
%    \end{macrocode}

% Include the two chapters:
%    \begin{macrocode}
\include{cdocsch1}
\include{cdocsch2}
%    \end{macrocode}

% Include the two parts unless only chapters should be displayed:
%    \begin{macrocode}
\ifchilddoc\else
\section{part three}
\input{cdocspt3}
\section{part four}
\input{cdocspt4}
\fi
%    \end{macrocode}

% Process as usual until here:
%    \begin{macrocode}
\fi
%    \end{macrocode}

% End of document body:
%    \begin{macrocode}
\end{document}
%    \end{macrocode}
%\iffalse
%</samplemain>
%\fi
%
% %%%%%%%%%%%%%%%%%%%%%%%%%%%%%%%%%%%%%%
% \paragraph{Chapter Include Files.}
%
% The include files are called |cdocsch1.tex| and |cdocsch2.tex|.
%
%\iffalse
%<*samplechap1|samplechap2>
%\fi

% Optional override for |\version| flag:
%    \begin{macrocode}
%%\providecommand{\version}{final}
%    \end{macrocode}

% Include the main document:
%    \begin{macrocode}
\input{childdoc.def}
\childdocof{cdocsamp}
%    \end{macrocode}

%\iffalse
%</samplechap1|samplechap2>
%\fi
%
%\iffalse
%<*samplechap1>
%\fi
% Some text for chapter 1:
%    \begin{macrocode}
\section{one}
some text in chapter one
%    \end{macrocode}

%\iffalse
%</samplechap1>
%\fi
% Some text for chapter 2:
%\iffalse
%<*samplechap2>
%\fi
%    \begin{macrocode}
\section{two}
more text in chapter two
%    \end{macrocode}

%\iffalse
%</samplechap2>
%\fi
%
% %%%%%%%%%%%%%%%%%%%%%%%%%%%%%%%%%%%%%%
% \paragraph{Part Include Files.}
%
% The include files are called |cdocspt3.tex| and |cdocspt4.tex|.
%
%\iffalse
%<*samplepart3|samplepart4>
%\fi

% Optional override for |\version| flag:
%    \begin{macrocode}
%%\providecommand{\version}{final}
%    \end{macrocode}

% Include the main document:
%    \begin{macrocode}
\input{childdoc.def}
\childdocby{cdocsamp}
%    \end{macrocode}

%\iffalse
%</samplepart3|samplepart4>
%\fi
%
%\iffalse
%<*samplepart3>
%\fi
% Some text for part 3:
%    \begin{macrocode}
some text in part three
%    \end{macrocode}

%\iffalse
%</samplepart3>
%\fi
% Some text for part 4:
%\iffalse
%<*samplepart4>
%\fi
%    \begin{macrocode}
more text in part four
%    \end{macrocode}

%\iffalse
%</samplepart4>
%\fi
%
% %%%%%%%%%%%%%%%%%%%%%%%%%%%%%%%%%%%%%%
% \paragraph{Forwarding for a Complete Draft.}
%
% The following forwarding file |cdocsdrf.tex|
% compiles the main document in draft mode:
%\iffalse
%<*sampledraft>
%\fi
%    \begin{macrocode}
\def\version{draft}
\input{childdoc.def}
\childdocforward{cdocsamp}
%    \end{macrocode}

%\iffalse
%</sampledraft>
%\fi
%
% %%%%%%%%%%%%%%%%%%%%%%%%%%%%%%%%%%%%%%
% \paragraph{Forwarding for Final Version of the Chapters.}
%
% The following forwarding files |cdocsfn1.tex| and |cdocsfn2.tex|
% (with identical content)
% compile the final versions of the child documents
% |cdocsch1.tex| and |cdocsch2.tex|, respectively:
%\iffalse
%<*samplefinal>
%\fi
%    \begin{macrocode}
\def\version{final}
\input{childdoc.def}
\childdocforwardprefix[cdocsamp]{cdocsfn}{cdocsch}
%    \end{macrocode}

%\iffalse
%</samplefinal>
%\fi
%
% %%%%%%%%%%%%%%%%%%%%%%%%%%%%%%%%%%%%%%
% \paragraph{Command Line Processing.}
%
% The following three command lines generate the output files
% |cdocscld|, |cdocscl1| and |cdocscl2|
% which should be identical to
% |cdocsdrf|, |cdocsch1| and |cdocsfn2|, respectively:
% \begin{center}
% \begin{tabular}{l}
% |latex -jobname cdocscld \|\\
% |  "\def\version{draft}\input{childdoc.def}\childdocforward{cdocsamp}"|\\
% |latex -jobname cdocscl1 \|\\
% |  "\input{childdoc.def}\childdocforward[cdocsamp]{cdocsch1}"|\\
% |latex -jobname cdocscl2 \|\\
% |  "\def\version{final}\input{childdoc.def}\childdocforward{cdocsch2}"|
% \end{tabular}
% \end{center}
% Note that the trailing backslash on each first line
% merely continues the input to the second line
% (for convenient cut ant paste).
% Furthermore, the command |latex| can be replaced by any
% of its alternative versions such as |pdflatex|.
%
% %%%%%%%%%%%%%%%%%%%%%%%%%%%%%%%%%%%%%%%%%%%%%%%%%%%%%%%%%%%%%%%%%%%%%%%%%%%%%%
% %%%%%%%%%%%%%%%%%%%%%%%%%%%%%%%%%%%%%%%%%%%%%%%%%%%%%%%%%%%%%%%%%%%%%%%%%%%%%%
% \section{Implementation}
%\iffalse
%<*package>
%\fi
%
% This section describes the definitions file |childdoc.def|.

% The definitions cannot be loaded using |\usepackage| or |\RequirePackage|
% which has a mechanism to prevent loading a style file more than once.
% When loading the definitions by means of |\input|
% multiple instances have to be prevented manually:
%\iffalse
%This code needs to be before the `\ProvidesFile' directive
%which is defined at the beginning of this file.
%Therefore it is also placed there and commented out here.
%</package>
%<*discard>
%\fi
%    \begin{macrocode}
\ifdefined\childdocmain\endinput\fi
%    \end{macrocode}
%\iffalse
%</discard>
%<*package>
%\fi
%
% \macro{\ifchilddoc}
% \macro{\ifchilddocmanual}
% The conditional |\ifchilddoc| tells whether a
% child (true) or main (false) document is being compiled.
% The conditional |\ifchilddocmanual| tells whether
% the |\includeonly| mechanism is used (false) or
% the selection of child files must be performed manually (true).
% The definitions initialise to false:
%    \begin{macrocode}
\newif\ifchilddoc
\newif\ifchilddocmanual
%    \end{macrocode}

% \macro{\childdocname}
% \macro{\childdocjob}
% The macro |\childdocname| stores the name of the main document
% to be compiled. The macro |\childdocjob| stores the name of
% the document on which the \LaTeX{} compiler was originally invoked.
% The content of |\jobname| cannot be compared
% to filenames specified in the source due to different catcodes.
% The following code rescans |\jobname|, stores the result
% in |\childdocname| and saves a copy in |\childdocjob|:
%    \begin{macrocode}
\edef\childdocname{\scantokens\expandafter{\jobname\noexpand}}
\let\childdocjob\childdocname
%    \end{macrocode}

% \macro{\childdocdisable}
% The macro |\childdocdisable| prevents the main file
% from being processed more than once.
% At this stage, the main document command |\childdocmain|
% is assumed to be called once again where it should do nothing.
% Any subsequent call to it should prevent
% a secondary processing of the main document
% It overwrites the forwarding commands
% |\childdocof| and |\childdocforward|
% with empty macros to prevent further inclusions of the main document:
%    \begin{macrocode}
\newcommand{\childdocdisable}
{
  \renewcommand{\childdocmain}[1]{\renewcommand{\childdocmain}[1]{\endinput}}
  \renewcommand{\childdocof}[1]{}
  \renewcommand{\childdocby}[2][]{}
  \renewcommand{\childdocforward}[2][]{}
  \renewcommand{\childdocdisable}{}
}
%    \end{macrocode}

% \macro{\childdocmain}
% The macro |\childdocmain| is to be called at the top of the main file
% with nothing or the main filename (without extension) as argument.
% First, it breaks loops.
% If the argument is not empty and does not match |\childdocname|
% (which is set by the first inclusion of |childdoc.def|),
% |\ifchilddoc| is set to true, |\includeonly| is applied to the child file
% and |\jobname| is set to the main file
% (for proper handling of |.aux| files):
%    \begin{macrocode}
\newcommand{\childdocmain}[1]
{
  \childdocdisable\childdocmain{}
  \if?#1?\else
    \begingroup
      \def\childdoctmp{#1}
      \ifx\childdoctmp\childdocname
        \def\childdoctmp{}
      \else
        \def\childdoctmp
        {
          \childdoctrue
          \includeonly{\childdocname}
          \def\childdocjob{#1}
          \def\jobname{#1}
        }
      \fi
      \expandafter
    \endgroup
    \childdoctmp
  \fi
}
%    \end{macrocode}

% \macro{\childdocof}
% The command |\childdocof| redirects
% compilation to the main file |#1|.
%    \begin{macrocode}
\newcommand{\childdocof}[1]
{
  \childdocdisable
  \childdoctrue
  \includeonly{\childdocname}
  \def\jobname{#1}
  \def\childdocjob{#1}
  \input{#1}
}
%    \end{macrocode}

% \macro{\childdocby}
% The command |\childdocby| ....
%    \begin{macrocode}
\newcommand{\childdocby}[2][]
{
  \childdocdisable
  \childdoctrue
  \childdocmanualtrue
  \if?#1?\else
    \def\jobname{#2}
  \fi
  \def\childdocjob{#2}
  \input{#2}
  \endinput
}
%    \end{macrocode}

% \macro{\childdocforward}
% The command |\childdocforward| redirects
% compilation to the main file or
% (if the optional argument is given) a child file.
% Parameters are set as if the main file
% or a child file starting with |\childdocof| was compiled.
% Then compilation is handed over to the main file:
%    \begin{macrocode}
\newcommand{\childdocforward}[2][]
{
  \begingroup
    \if?#1?
      \def\childdoctmp
      {
        \def\childdocname{#2}
        \def\childdocjob{#2}
        \def\jobname{#2}
        \input{#2}
        \endinput
      }
    \else
      \def\childdoctmp
      {
        \childdocdisable
        \def\childdocname{#2}
        \childdoctrue
        \includeonly{#2}
        \def\childdocjob{#1}
        \def\jobname{#1}
        \input{#1}
        \endinput
      }
    \fi
    \expandafter
  \endgroup
  \childdoctmp
}
%    \end{macrocode}

% \macro{\childdocforwardprefix}
% The command |\childdocforwardprefix| redirects
% compilation to the main or a child file by means of a pattern.
% The prefix |#1| in the current filename is replaced by |#2|
% and the suffix of the current filename is kept
% (it is assumed that the filename does not contain the substring `|~~~|'
% which is used as a delimiter).
% Compilation is handed over to the new file by |\childdocforward|:
%    \begin{macrocode}
\newcommand{\childdocforwardprefix}[3][]
{
  \begingroup
    \def\childdocextract #2##1~~~{\def\childdoctmp{\childdocforward[#1]{#3##1}}}
    \expandafter\childdocextract\childdocname~~~
    \expandafter
  \endgroup
  \childdoctmp
}
%    \end{macrocode}

% \macro{\childdoc}
% The deprecated macro |\childdoc| is a legacy version of |\childdocmain|:
%    \begin{macrocode}
\newcommand{\childdoc}{\childdocmain}
%    \end{macrocode}

% \macro{\childdocredirect}
% The deprecated macro |\childdocredirect| is a legacy version
% of |\childdocforward| and |\childdocforwardprefix|:
%    \begin{macrocode}
\newcommand{\childdocredirect}[2][]
{
  \begingroup
    \if?#1?
      \def\childdoctmp{\childdocforward{#2}}
    \else
      \def\childdoctmp{\childdocforwardprefix{#1}{#2}}
    \fi
    \expandafter
  \endgroup
  \childdoctmp
}
%    \end{macrocode}

%\iffalse
%</package>
%\fi
%
\endinput

\childdocof{cdocsamp}
%    \end{macrocode}

%\iffalse
%</samplechap1|samplechap2>
%\fi
%
%\iffalse
%<*samplechap1>
%\fi
% Some text for chapter 1:
%    \begin{macrocode}
\section{one}
some text in chapter one
%    \end{macrocode}

%\iffalse
%</samplechap1>
%\fi
% Some text for chapter 2:
%\iffalse
%<*samplechap2>
%\fi
%    \begin{macrocode}
\section{two}
more text in chapter two
%    \end{macrocode}

%\iffalse
%</samplechap2>
%\fi
%
% %%%%%%%%%%%%%%%%%%%%%%%%%%%%%%%%%%%%%%
% \paragraph{Part Include Files.}
%
% The include files are called |cdocspt3.tex| and |cdocspt4.tex|.
%
%\iffalse
%<*samplepart3|samplepart4>
%\fi

% Optional override for |\version| flag:
%    \begin{macrocode}
%%\providecommand{\version}{final}
%    \end{macrocode}

% Include the main document:
%    \begin{macrocode}
% \iffalse
%
% childdoc.dtx Copyright (C) 2017-2018 Niklas Beisert
%
% This work may be distributed and/or modified under the
% conditions of the LaTeX Project Public License, either version 1.3
% of this license or (at your option) any later version.
% The latest version of this license is in
%   http://www.latex-project.org/lppl.txt
% and version 1.3 or later is part of all distributions of LaTeX
% version 2005/12/01 or later.
%
% This work has the LPPL maintenance status `maintained'.
%
% The Current Maintainer of this work is Niklas Beisert.
%
% This work consists of the files childdoc.dtx and childdoc.ins
% and the derived files childdoc.def and cdocsamp.tex with
% cdocsch1.tex, cdocsch2.tex, cdocsdrf.tex, cdocsfn1.tex, cdocsfn2.tex.
%
%<package>\ifdefined\childdocmain\endinput\fi
%<package>\ProvidesFile{childdoc.def}[2018/12/30 v2.0 child document driver]
%<samplemain>\ProvidesFile{cdocsamp.tex}[2018/12/30 v2.0 sample for childdoc]
%<*driver>
%\ProvidesFile{childdoc.drv}[2018/12/30 v2.0 childdoc reference manual file]
\PassOptionsToClass{10pt,a4paper}{article}
\documentclass{ltxdoc}

\usepackage[margin=35mm]{geometry}
\usepackage{hyperref}
\usepackage{hyperxmp}
\usepackage[usenames]{color}

\hypersetup{colorlinks=true}
\hypersetup{pdfstartview=FitH}
\hypersetup{pdfpagemode=UseNone}
\hypersetup{pdfsource={}}
\hypersetup{pdflang={en-UK}}
\hypersetup{pdfcopyright={Copyright 2017-2018 Niklas Beisert.
  This work may be distributed and/or modified under the
  conditions of the LaTeX Project Public License, either version 1.3
  of this license or (at your option) any later version.}}
\hypersetup{pdflicenseurl={http://www.latex-project.org/lppl.txt}}
\hypersetup{pdfcontactaddress={ETH Zurich, ITP, HIT K,
  Wolfgang-Pauli-Strasse 27}}
\hypersetup{pdfcontactpostcode={8093}}
\hypersetup{pdfcontactcity={Zurich}}
\hypersetup{pdfcontactcountry={Switzerland}}
\hypersetup{pdfcontactemail={nbeisert@itp.phys.ethz.ch}}
\hypersetup{pdfcontacturl={http://people.phys.ethz.ch/\xmptilde nbeisert/}}

\newcommand{\secref}[1]{\hyperref[#1]{section \ref*{#1}}}

\parskip1ex
\parindent0pt
\let\olditemize\itemize
\def\itemize{\olditemize\parskip0pt}

\begin{document}

\title{The \textsf{childdoc} Package}
\hypersetup{pdftitle={The childdoc Package}}
\author{Niklas Beisert\\[2ex]
  Institut f\"ur Theoretische Physik\\
  Eidgen\"ossische Technische Hochschule Z\"urich\\
  Wolfgang-Pauli-Strasse 27, 8093 Z\"urich, Switzerland\\[1ex]
  \href{mailto:nbeisert@itp.phys.ethz.ch}
  {\texttt{nbeisert@itp.phys.ethz.ch}}}
\hypersetup{pdfauthor={Niklas Beisert}}
\hypersetup{pdfsubject={Manual for the LaTeX2e Package childdoc}}
\date{30 December 2018, \textsf{v2.0}}
\maketitle

\begin{abstract}\noindent
\textsf{childdoc} is a \LaTeXe{} package
that enables the direct compilation
of document sections included by |\include|
to individual files.
\end{abstract}

\begingroup
\parskip0ex
\tableofcontents
\endgroup

%%%%%%%%%%%%%%%%%%%%%%%%%%%%%%%%%%%%%%%%%%%%%%%%%%%%%%%%%%%%%%%%%%%%%%%%%%%%%%%%
%%%%%%%%%%%%%%%%%%%%%%%%%%%%%%%%%%%%%%%%%%%%%%%%%%%%%%%%%%%%%%%%%%%%%%%%%%%%%%%%
\section{Introduction}

\LaTeX{} provides a mechanism to structure a large document (such as a book)
into a main file and several child files (containing the chapters)
using the |\include| command.
This mechanism is beneficial for documents
which span hundreds of pages in order to
make the source file(s) more manageable.
Moreover, compilation can be restricted to
selected child files by means of the |\includeonly| command.
The latter feature can be used to reduce the compilation time while editing
(this was significantly more useful in the earlier days of \LaTeX{})
or to generate a smaller document which is easier to navigate.
Another application of |\includeonly| is to generate
documents consisting of selected parts of the complete document.

However, there are a few drawbacks of the plain |\include| mechanism:
\begin{itemize}
\item
The child files cannot be compiled on their own,
they can only be compiled via the main file.
A naive editing environment
(such as a text editor with an option
to have the current file processed by \LaTeX)
may require one to switch to the main file before compiling;
attempting to compile the child file produces errors.
\item
The main file must be modified (each time)
to adjust the |\includeonly| command
to the present needs. This easily leaves the main file in a messy state.
\item
The generated document will always carry the filename
of the main document. This is inconvenient if
several child files are to be compiled and
to be kept for distribution.
\end{itemize}

The present package provides a simple interface
to make child files individually compilable by \LaTeX{}.
Compiling a child file then has the same effect as compiling
the main file with an |\includeonly| command
to select the appropriate child.
Moreover the generated document will carry the name of the child
rather than the main file.
This resolves all three above issues.

This feature is meant to make the editing of books,
thesis documents and lecture notes somewhat more convenient.
However, the package can also be used efficiently for
composing a series of documents (such as exercise sheets)
which are typically distributed individually.
It then assists the author in generating the individual documents
(potentially in different versions)
as well as a document containing the collected series.
Another application is in developing style files
or other kinds of included material
where compilation of the style file could redirect
to a sample or test file.

%%%%%%%%%%%%%%%%%%%%%%%%%%%%%%%%%%%%%%%%%%%%%%%%%%%%%%%%%%%%%%%%%%%%%%%%%%%%%%%%
%%%%%%%%%%%%%%%%%%%%%%%%%%%%%%%%%%%%%%%%%%%%%%%%%%%%%%%%%%%%%%%%%%%%%%%%%%%%%%%%
\section{Usage}

First of all, the package \textsf{childdoc} is \emph{not} a standard
\LaTeXe{} |.sty| style file! Therefore it needs to be invoked in
a non-standard way.

%%%%%%%%%%%%%%%%%%%%%%%%%%%%%%%%%%%%%%%%%%%%%%%%%%%%%%%%%%%%%%%%%%%%%%%%%%%%%%%%
\subsection{Included Files}
\label{sec:include}

%%%%%%%%%%%%%%%%%%%%%%%%%%%%%%%%%%%%%%%%
\DescribeMacro{\childdocmain}
To use the package, add the commands
\begin{center}
\begin{tabular}{l}
|\input{childdoc.def}|\\
|\childdocmain{}|\\
\end{tabular}
\end{center}
at the very top of the main \LaTeX{} file,
in particular \emph{before} the |\documentclass| statement!
The argument of |\childdocmain| should be left empty
(but it must be present).

%%%%%%%%%%%%%%%%%%%%%%%%%%%%%%%%%%%%%%%%
\DescribeMacro{\childdocof}
Furthermore, add the commands
\begin{center}
\begin{tabular}{l}
|\input{childdoc.def}|\\
|\childdocof{|\textit{main}|}|\\
\end{tabular}
\end{center}
at the top of every child file \textit{child}
which is included by |\include{|\textit{child}|}|
from within the main file
(or at least for those files to be compiled individually).
The argument \textit{main} must be the filename of the main file.

There are a couple of
considerations in setting up the main and child documents:

%%%%%%%%%%%%%%%%%%%%%%%%%%%%%%%%%%%%%%%%
\paragraph{Restrictions.}

Please note the following restrictions:
\begin{itemize}
\item
|\childdocmain| must be called with one argument \textit{main}
to ensure compatibility with earlier version of the package.
It must either be empty (|\childdocmain{}|)
or precisely match the filename of the main file in which it is specified.
See \secref{sec:detection} for further information.
\item
The filename \textit{main} must be specified without the |.tex| extension.
\item
The filename \textit{main} is case sensitive
(even in case-insensitive file systems)
due to internal string comparison.
\item
The argument \textit{main} should be fully expanded, it cannot be a macro.
\item
Subdirectories and special characters should be avoided in filenames.
\item
The command |\childdocmain{|\textit{main}|}| must be followed by a whitespace.
It should not be followed immediately by another command
or by a comment mark `|%|'.
This is because the \TeX{} parser reads the token immediately following
the argument of |\childdocmain| and puts it
at the beginning of every child section;
however, a white\-space is ignored.
\end{itemize}

%%%%%%%%%%%%%%%%%%%%%%%%%%%%%%%%%%%%%%%%
\paragraph{Content of Main File.}

It is advisable to place all content in the child files included by |\include|.
Any output contained in the main file will appear in all child documents
unless suppressed manually;
it cannot be suppressed automatically by the |\includeonly| directive
and thus should normally be avoided.
A method to include some content in the main file
by means of conditional processing is described in \secref{sec:conditional}.

%%%%%%%%%%%%%%%%%%%%%%%%%%%%%%%%%%%%%%%%
\paragraph{Page Numbering.}

When only a part of the document is compiled,
the appropriate numbering of pages
(as well as other status parameters)
is determined from the |.aux| files.
The latter contain information from previous passes.
However this information needs to propagate through
all intermediate child documents.
Therefore the page numbering in child documents may well
be inconsistent until the complete document is compiled at least once.

A useful (if unconventional) way to always ensure a consistent
page numbering is to restart the numbering in each child document
and denote the pages by `\textit{child}|.|\textit{page}'
where \textit{child} represents the chapter/section number of the child file.
This can be achieved by the command
|\numberwithin{page}{|\textit{child}|}|
of the \textsf{amsmath} package
where \textit{child} can be |chapter| or |section|
depending on the chosen structuring.
Alternatively, one can modify the macro |\thepage| appropriately
and reset the counter |page| at the start of each child file.

%%%%%%%%%%%%%%%%%%%%%%%%%%%%%%%%%%%%%%%%%%%%%%%%%%%%%%%%%%%%%%%%%%%%%%%%%%%%%%%%
\subsection{Conditional Processing}
\label{sec:conditional}

The package provides a mechanism to compile different versions
of a document. To customise the versions further some conditional processing
can come in handy to distinguish which version is being compiled.
The package provides two macros to describe the compilation context:

%%%%%%%%%%%%%%%%%%%%%%%%%%%%%%%%%%%%%%%%
\DescribeMacro{\ifchilddoc}
The conditional |\ifchilddoc| distinguishes between the compilation of
child documents and the main document:
%
\begin{center}
|\ifchilddoc |\textit{child-code}| |[|\||else |\textit{main-code}]| \||fi|
\end{center}

%%%%%%%%%%%%%%%%%%%%%%%%%%%%%%%%%%%%%%%%
\DescribeMacro{\childdocname}
\DescribeMacro{\childdocjob}
The macro |\childdocname| contains the filename (without extension)
of the main or child file being processed.
Note that |\childdocjob| will always contain the name of the main file.

%%%%%%%%%%%%%%%%%%%%%%%%%%%%%%%%%%%%%%%%
\paragraph{Title Page.}

Conditional processing can be used to include a title or banner page
in the main document when proper precautions are taken.
Importantly, the code in the main file should ensure that the page counter
(as well as other status parameters which are stored in the |.aux| files)
takes the same value after the conditional processing.
Otherwise the page numbers may take divergent values
depending on which part is compiled.

For example, a title page could be declared by:
%
\begin{center}
\begin{tabular}{l}
|\ifchilddoc\||else|\\
|\addtocounter{page}{-1}|\\
\textit{code for title page}\\
|\newpage|\\
|\||fi|
\end{tabular}
\end{center}
%
A banner page for the child documents can be generated by:
%
\begin{center}
\begin{tabular}{l}
|\ifchilddoc|\\
|\addtocounter{page}{-1}|\\
\textit{code for banner page}\\
|\newpage|\\
|\||fi|
\end{tabular}
\end{center}
%
Here one could write a message such as:
\begin{center}
|This is the part \childdocname{} of \childdocjob{}.|
\end{center}

%%%%%%%%%%%%%%%%%%%%%%%%%%%%%%%%%%%%%%%%%%%%%%%%%%%%%%%%%%%%%%%%%%%%%%%%%%%%%%%%
\subsection{Flags}
\label{sec:flags}

The package makes it easy to generate different versions
of the main or child documents.
To this end compilation flags can be defined
and assigned different default values.
They will be particularly useful in conjunction
with the forwarding mechanism described in \secref{sec:forward}.

For example, it may be useful to have a flag |\version|
which can be set to |draft| or |final|.
The document source will contain some conditional code
depending on the value of |\version|.
Suppose further, the flag should default to |final| for the main file
and to |draft| for child files
which is a natural assignment for editing the document.
This is achieved by placing the following code
in the preamble of the main document
(below the |\childdocmain| directive):
%
\begin{center}
\begin{tabular}{l}
|\ifchilddoc|\\
|\providecommand{\version}{draft}|\\
|\||else|\\
|\providecommand{\version}{final}|\\
|\||fi|
\end{tabular}
\end{center}
%
The definition by |\providecommand| makes sure
that previous definitions are not overwritten.
Further statements |\providecommand{\version}{...}|
can thus be added before the above code to override it.

For the main file, one might add a line
(between |\childdocmain| and the above block)
%
\begin{center}
|%\ifchilddoc\||else\providecommand{\version}{draft}\||fi|
\end{center}
%
which can be uncommented to produce a draft version.
Likewise one can add a line to the very top of a child file
(above the |\childdocof{|\textit{main}|}| directive)
%
\begin{center}
|%\providecommand{\version}{final}|
\end{center}
%
which can be uncommented to produce the final version of this child document.

%%%%%%%%%%%%%%%%%%%%%%%%%%%%%%%%%%%%%%%%%%%%%%%%%%%%%%%%%%%%%%%%%%%%%%%%%%%%%%%%
\subsection{Forwarding}
\label{sec:forward}

Different versions of the main or child documents
using compilation flags as described in \secref{sec:flags}
can be (permanently) stored in different files
for convenient compilation, viewing and distribution.
To this end, the package defines a command
to pass on compilation to a different file:

%%%%%%%%%%%%%%%%%%%%%%%%%%%%%%%%%%%%%%%%
\DescribeMacro{\childdocforward}
The command |\childdocforward| redirects processing to
another source file:
%
\begin{center}
\begin{tabular}{l}
|\input{childdoc.def}|\\
|\childdocforward[|\textit{main}|]{|\textit{dest}|}|\\
\end{tabular}
\end{center}
%
The argument \textit{dest} is the destination file
(without extension).
It should be the main file or one of the child files.
Note that further \textsf{childdoc} directives
such as |\childdocof| and |\childdocforward|
in the indicated file will be processed in this form.
The optional argument \textit{main}
passes on directly to the main file \textit{main}
while pretending to compile the child \textit{dest}.
This form behaves as if \textit{dest}
issues |\childdocof{|\textit{main}|}| right away,
and no further \textsf{childdoc} directives will be processed.

%%%%%%%%%%%%%%%%%%%%%%%%%%%%%%%%%%%%%%%%
\DescribeMacro{\...prefix}
In the alternative form |\childdocforwardprefix|,
%
\begin{center}
\begin{tabular}{l}
|\input{childdoc.def}|\\
|\childdocforwardprefix[|\textit{main}|]{|\textit{prefix}|}{|\textit{dest}|}|
\end{tabular}
\end{center}
%
the destination file is determined by a pattern
depending on the current file:
To make this work, the current file must be called
`{\textit{prefix}\hspace{0.2em}\textit{suffix}}'
with \textit{prefix} matching precisely the argument.
Processing is then passed on to the file
`{\textit{dest}\hspace{0.2em}\textit{suffix}}'.
Surely, the same effect is achieved by
directly specifying the
argument `{\textit{dest}\hspace{0.2em}\textit{suffix}}'
in the first form.
However, that requires to set up a different file
for each child. With the alternative form of the command
all these files can have exactly the same content
which simplifies setting them up and maintaining them.

For example, the following file |draft.tex|
with a compilation flag |\version| as described in \secref{sec:flags}
compiles the main document as a draft:
%
\begin{center}
\begin{tabular}{l}
|\def\version{draft}|\\
|\input{childdoc.def}|\\
|\childdocforward{|\textit{main}|}|
\end{tabular}
\end{center}
%
Likewise, the following files |final|\textit{nn}|.tex|
compile the final version of the child document
|child|\textit{nn}|.tex|:
%
\begin{center}
\begin{tabular}{l}
|\def\version{final}|\\
|\input{childdoc.def}|\\
|\childdocforwardprefix{final}{child}|
\end{tabular}
\end{center}
%

Note that when several versions of a main file and/or of each child file
are to be generated, it may be convenient to set up a |Makefile| or
shell script to automatise the process.

%%%%%%%%%%%%%%%%%%%%%%%%%%%%%%%%%%%%%%%%%%%%%%%%%%%%%%%%%%%%%%%%%%%%%%%%%%%%%%%%
\subsection{Command Line Processing}
\label{sec:commandline}

The effect of redirection files can also be achieved by invoking
the \LaTeX{} compiler with a more elaborate command line.
Most conveniently this should be done as part
of a shell script or a |Makefile|.

When using \textsf{childdoc} in the main file, the following
command lines effectively perform a redirection
(note that depending on the shell being used,
backslashes may have to be doubled: `|\|' $\to$ `|\\|'):
%
\begin{center}
|... -jobname "|\textit{target}|" |\\|"|[\textit{flags}]%
|\input{childdoc.def}\childdocforward[|\textit{main}|]{|\textit{dest}|}"|
\end{center}
%
Here \textit{target} is the name of the output file,
\textit{main} is the name of the main file
and \textit{dest} is the name of the main or child file to be processed
(all filenames without extensions).
The optional argument \textit{main} can be omitted
if \textit{main} matches \textit{dest}.
Optionally, compilation \textit{flags} can be defined via |\def| commands.
This command line makes the \TeX{} engine believe
it is compiling the file \textit{target}
whose content is specified as the latter parameter.
The provided code then forwards the processing to
\textit{main} or \textit{dest} as described in \secref{sec:forward}.

%%%%%%%%%%%%%%%%%%%%%%%%%%%%%%%%%%%%%%%%%%%%%%%%%%%%%%%%%%%%%%%%%%%%%%%%%%%%%%%%
\subsection{Include by Input}
\label{sec:input}

Including child documents by |\include| has some restrictions by design.
Most notably, the content of a child document always occupies
its own set of pages; pages cannot be shared between child documents.
Usually, this behaviour makes perfect sense
because each child document contain an essential part of the document.
However, in some situations it may be desirable to compose
a document from a collection of parts
without having mandatory page breaks between then.
For this case, the package
provides a mechanism to include parts
by |\input| which can also be processed individually.
However, by construction this mechanism
requires manual handling of the content to be output.

%%%%%%%%%%%%%%%%%%%%%%%%%%%%%%%%%%%%%%%%
\DescribeMacro{\ifchilddocmanual}
The main file should be prepared as usual, see \secref{sec:include}.
However, the document body must make a distinction
between processing of an individual part and of the main document, e.g.:
%
\begin{center}
\begin{tabular}{l}
|\ifchilddocmanual|\\
|\input{\childdocname}|\\
|\||else|\\
\textit{document body with }|\input{|\textit{part}|}|\\
|\||fi|
\end{tabular}
\end{center}
%
The conditional |\ifchilddocmanual| is true whenever
a part to be included by |\input| is being compiled,
and the name of the part is stored in |\childdocname|.

%%%%%%%%%%%%%%%%%%%%%%%%%%%%%%%%%%%%%%%%
\DescribeMacro{\childdocby}
Each part to be included by |\input| should start with:
%
\begin{center}
\begin{tabular}{l}
|\input{childdoc.def}|\\
|\childdocby{|\textit{main}|}|\\
\end{tabular}
\end{center}
%
The directive |\childdocby| is similar to |\childdocof|
described in \secref{sec:include},
but the subsequent selection of content must be done manually.
To that end, both |\ifchilddoc| and |\ifchilddocmanual|
will be true upon processing of a part,
and the name of the part is stored in |\childdocname|.
Note that |\jobname| will be set to the filename of the current part
so that each part receives an individual |.aux| file
that does not interfere with the |.aux| file(s) of the main document.
This behaviour can be altered by the alternative form
|\childdocby[*]{|\textit{main}|}| (with a non-empty optional argument)
which uses the |.aux| file of the main document
by setting |\jobname| to \textit{main}.

%%%%%%%%%%%%%%%%%%%%%%%%%%%%%%%%%%%%%%%%%%%%%%%%%%%%%%%%%%%%%%%%%%%%%%%%%%%%%%%%
\subsection{Driver Development}
\label{sec:driver}

The \textsf{childdoc} mechanism can also be use for the development
of definition files such as \LaTeX{} styles or classes.
This case differs from the above setup with multiple parts
included by |\include| in that no |\includeonly| should be invoked.
This can be achieved by starting the include file
(before |\ProvidesPackage|) with:
%
\begin{center}
\begin{tabular}{l}
|\input{childdoc.def}|\\
|\childdocforward{|\textit{main}|}|\\
\end{tabular}
\end{center}
%
or alternatively with:
%
\begin{center}
\begin{tabular}{l}
|\input{childdoc.def}|\\
|\childdocby{|\textit{main}|}|\\
\end{tabular}
\end{center}
%
Both forms have slightly different effects as described above.
The main file is prepared as usual, see \secref{sec:include}.

%%%%%%%%%%%%%%%%%%%%%%%%%%%%%%%%%%%%%%%%%%%%%%%%%%%%%%%%%%%%%%%%%%%%%%%%%%%%%%%%
\subsection{Legacy Detection}
\label{sec:detection}

The directive |\childdocmain| in the main file can detect
whether the complete document or merely a child is to be compiled
even without using the directive |\childdocof|.
This method is deprecated because it is less robust
and there is no compelling reason to use it;
it is merely provided for backward compatibility
and it may be removed in future versions.

If the detection mechanism is to be used,
it is mandatory to correctly specify
the filename of the main file as the argument of |\childdocmain|:
%
\begin{center}
\begin{tabular}{l}
|\input{childdoc.def}|\\
|\childdocmain{|\textit{main}|}|\\
\end{tabular}
\end{center}
%
If |\jobname| does not match the argument \textit{main} of |\childdocmain|,
it is assumed that |\jobname| points to the child file to be compiled.
When using |\childdocmain| with the main file specified as argument,
it suffices to start a child file
with just |\input{|\textit{main}|}|
without loading of the package and using |\childdocof|.
If instead all processing is done
with the appropriate \textsf{childdoc} directives,
the argument of \textit{main} of |\childdocmain| can be empty.

An alternative version of the command line processing described
in \secref{sec:commandline} using the detection mechanism reads:
%
\begin{center}
|... -jobname "|\textit{target}|" "|[\textit{flags}]%
[|\def\jobname{|\textit{dest}|}|]|\input{|\textit{main}|}"|
\end{center}

%%%%%%%%%%%%%%%%%%%%%%%%%%%%%%%%%%%%%%%%%%%%%%%%%%%%%%%%%%%%%%%%%%%%%%%%%%%%%%%%
\subsection{Manual Code}
\label{sec:manual}

In case one cannot be certain whether the definitions file |childdoc.def|
is installed on the target \TeX{} distribution
and one prefers not to ship it,
it is conceivable to paste a few relevant commands into the sources.

To that end, drop all statements |\input{childdoc.def}|
and perform the replacements as outlined below.
Instead of |\childdocmain{|\textit{main}|}| add the following code
to the top of the main file:
%
\begin{center}
\begin{tabular}{l}
|\||ifdefined\childdocname\endinput\||fi\newif\ifchilddoc|\\
|\edef\childdocname{\scantokens\expandafter{\jobname\noexpand}}|\\
|\def\childdocmain{|\textit{main}|}\||ifx\childdocmain\childdocname\||else|\\
|\childdoctrue\includeonly{\childdocname}\let\jobname\childdocmain\||fi|\\
\end{tabular}
\end{center}
%
Instead of |\childdocof{|\textit{main}|}| just include the main file
at the top of each child file:
%
\begin{center}
|\input{|\textit{main}|}|
\end{center}
%
A simple redirection |\childdocforward{|\textit{dest}|}| is achieved by:
%
\begin{center}
|\def\jobname{|\textit{dest}|}\input{\jobname}|
\end{center}
%
The redirection with prefix
|\childdocforwardprefix[|\textit{prefix}|]{|\textit{dest}|}|
is accomplished by:
%
\begin{center}
\begin{tabular}{l}
|{\edef\jobname{\scantokens\expandafter{\jobname\noexpand}}|\\
|\def\redirectjob |\textit{prefix}|#1~~~{\gdef\jobname{|\textit{dest}|#1}}|\\
|\expandafter\redirectjob\jobname~~~}\input{\jobname}|
\end{tabular}
\end{center}

In an alternative approach,
child documents can be compiled by a specific command line
without additional code or specific definitions:
%
\begin{center}
|... -jobname "|\textit{target}|" "|[\textit{flags}]%
|\includeonly{|\textit{dest}|}\input{|\textit{main}|}"|
\end{center}
%

%%%%%%%%%%%%%%%%%%%%%%%%%%%%%%%%%%%%%%%%%%%%%%%%%%%%%%%%%%%%%%%%%%%%%%%%%%%%%%%%
%%%%%%%%%%%%%%%%%%%%%%%%%%%%%%%%%%%%%%%%%%%%%%%%%%%%%%%%%%%%%%%%%%%%%%%%%%%%%%%%
\section{Information}

%%%%%%%%%%%%%%%%%%%%%%%%%%%%%%%%%%%%%%%%%%%%%%%%%%%%%%%%%%%%%%%%%%%%%%%%%%%%%%%%
\subsection{Copyright}

Copyright \copyright{} 2017--2018 Niklas Beisert

This work may be distributed and/or modified under the
conditions of the \LaTeX{} Project Public License, either version 1.3
of this license or (at your option) any later version.
The latest version of this license is in
  \url{http://www.latex-project.org/lppl.txt}
and version 1.3 or later is part of all distributions of \LaTeX{}
version 2005/12/01 or later.

This work has the LPPL maintenance status `maintained'.

The Current Maintainer of this work is Niklas Beisert.

This work consists of the files |README.txt|, |childdoc.ins| and |childdoc.dtx|
as well as the derived files |childdoc.def|, |cdocsamp.tex|
with |cdocsch1.tex|, |cdocsch2.tex|, |cdocspt3.tex|, |cdocspt4.tex|,
|cdocsdrf.tex|, |cdocsfn1.tex|, |cdocsfn2.tex|
as well as |childdoc.pdf|.

%%%%%%%%%%%%%%%%%%%%%%%%%%%%%%%%%%%%%%%%%%%%%%%%%%%%%%%%%%%%%%%%%%%%%%%%%%%%%%%%
\subsection{Files and Installation}

The package consists of the files:
%
\begin{center}
\begin{tabular}{ll}
    |README.txt|   & readme file \\
    |childdoc.ins| & installation file \\
    |childdoc.dtx| & source file \\
    |childdoc.def| & definition file \\
    |cdocsamp.tex| & sample main file \\
    |cdocsch1.tex| & sample include file \\
    |cdocsch2.tex| & sample include file \\
    |cdocspt3.tex| & sample part file \\
    |cdocspt4.tex| & sample part file \\
    |cdocsdrf.tex| & sample redirection file \\
    |cdocsfn1.tex| & sample redirection file \\
    |cdocsfn2.tex| & sample redirection file \\
    |childdoc.pdf| & manual
\end{tabular}
\end{center}
%
The distribution consists of the files
|README.txt|, |childdoc.ins| and |childdoc.dtx|.
%
\begin{itemize}
\item
Run (pdf)\LaTeX{} on |childdoc.dtx|
to compile the manual |childdoc.pdf| (this file).
\item
Run \LaTeX{} on |childdoc.ins| to create the definitions file |childdoc.def|
and the sample |cdocsamp.tex| with include files
|cdocsch1.tex|, |cdocsch2.tex|, |cdocspt3.tex|, |cdocspt4.tex|,
|cdocsdrf.tex|, |cdocsfn1.tex|, |cdocsfn2.tex|.
Then copy the file |childdoc.def| to an appropriate directory of your \LaTeX{}
distribution, e.g.\ \textit{texmf-root}|/tex/latex/childdoc|.
\end{itemize}

%%%%%%%%%%%%%%%%%%%%%%%%%%%%%%%%%%%%%%%%%%%%%%%%%%%%%%%%%%%%%%%%%%%%%%%%%%%%%%%%
\subsection{Related CTAN Packages}

There are several other packages which offer a similar functionality:
%
\begin{itemize}
\item
The packages
\href{http://ctan.org/pkg/docmute}{\textsf{docmute}},
\href{http://ctan.org/pkg/includex}{\textsf{includex}} and
\href{http://ctan.org/pkg/standalone}{\textsf{standalone}}
provide commands to include only the document body of
a child file thus allowing both files to be compiled individually.
\item
The packages \href{http://ctan.org/pkg/subdocs}{\textsf{subdocs}}
and \href{http://ctan.org/pkg/subfiles}{\textsf{subfiles}}
provide structures in which the main and child documents can be
encapsulated and allowing them to be compiled individually.
The inclusion mechanism is different from the conventional |\include|.
\item
The package \href{http://ctan.org/pkg/combine}{\textsf{combine}}
is an elaborate solution to combine several documents into one.
\end{itemize}
%
See also the CTAN topic \href{http://ctan.org/topic/subdocs}{\textsf{subdocs}}
for further related packages.
The present package differs from the above solutions in that
a document structure constructed with the conventional |\include| mechanism
just needs two extra commands at the top of every file
such that all constituent files can be compiled individually.

%%%%%%%%%%%%%%%%%%%%%%%%%%%%%%%%%%%%%%%%%%%%%%%%%%%%%%%%%%%%%%%%%%%%%%%%%%%%%%%%
%\subsection{Feature Suggestions}
%
%The following is a list of features which may be useful for future
%versions of this package:
%%
%\begin{itemize}
%\item
%\ldots
%\end{itemize}

%%%%%%%%%%%%%%%%%%%%%%%%%%%%%%%%%%%%%%%%%%%%%%%%%%%%%%%%%%%%%%%%%%%%%%%%%%%%%%%%
\subsection{Revision History}

%%%%%%%%%%%%%%%%%%%%%%%%%%%%%%%%%%%%%%%%
\paragraph{v2.0:} 2018/12/30

\begin{itemize}
\item
immediate forward processing
\item
added |\childdocby| mechanism
\item
manual restructured
\end{itemize}

%%%%%%%%%%%%%%%%%%%%%%%%%%%%%%%%%%%%%%%%
\paragraph{v1.6:} 2018/01/17

\begin{itemize}
\item
application for development of include files
\item
corrections to manual
\end{itemize}

%%%%%%%%%%%%%%%%%%%%%%%%%%%%%%%%%%%%%%%%
\paragraph{v1.5:} 2017/05/21

\begin{itemize}
\item
more complete structuring introduced
\item
|\childdocof| introduced
\item
|\childdoc| renamed to |\childdocmain|
\item
|\childredirect| renamed to |\childdocforward| and |\childdocforwardprefix|
and functionality expanded
\end{itemize}

%%%%%%%%%%%%%%%%%%%%%%%%%%%%%%%%%%%%%%%%
\paragraph{v1.0:} 2017/04/27

\begin{itemize}
\item
manual and install package
\item
first version published on CTAN
\end{itemize}

%%%%%%%%%%%%%%%%%%%%%%%%%%%%%%%%%%%%%%%%
\paragraph{v0.6:} 2017/04/26

\begin{itemize}
\item
redirection mechanism added
\end{itemize}

%%%%%%%%%%%%%%%%%%%%%%%%%%%%%%%%%%%%%%%%
\paragraph{v0.5:} 2017/04/26

\begin{itemize}
\item
functionality in definition file
\end{itemize}


%%%%%%%%%%%%%%%%%%%%%%%%%%%%%%%%%%%%%%%%%%%%%%%%%%%%%%%%%%%%%%%%%%%%%%%%%%%%%%%%
%%%%%%%%%%%%%%%%%%%%%%%%%%%%%%%%%%%%%%%%%%%%%%%%%%%%%%%%%%%%%%%%%%%%%%%%%%%%%%%%
%%%%%%%%%%%%%%%%%%%%%%%%%%%%%%%%%%%%%%%%%%%%%%%%%%%%%%%%%%%%%%%%%%%%%%%%%%%%%%%%
\appendix

\settowidth\MacroIndent{\rmfamily\scriptsize 000\ }

 \DocInput{childdoc.dtx}

\end{document}
%</driver>
% \fi
%
% %%%%%%%%%%%%%%%%%%%%%%%%%%%%%%%%%%%%%%%%%%%%%%%%%%%%%%%%%%%%%%%%%%%%%%%%%%%%%%
% %%%%%%%%%%%%%%%%%%%%%%%%%%%%%%%%%%%%%%%%%%%%%%%%%%%%%%%%%%%%%%%%%%%%%%%%%%%%%%
% \section{Sample}
%\iffalse
%<*samplemain>
%\fi
%
% The following presents a sample document
% with two chapters, two parts, a title page,
% a compile flag as well as three forwarding files to set the flag.
% It consists of eight |.tex| files:
% \begin{center}
% \begin{tabular}{ll}
% |cdocsamp.tex|&main file\\
% |cdocsch1.tex|&include file for chapter 1\\
% |cdocsch2.tex|&include file for chapter 2\\
% |cdocspt3.tex|&include file for part 3\\
% |cdocspt4.tex|&include file for part 4\\
% |cdocsdrf.tex|&forwarding file for main file in draft mode\\
% |cdocsfi1.tex|&forwarding file for final version of chapter 1\\
% |cdocsfi2.tex|&forwarding file for final version of chapter 2\\
% \end{tabular}
% \end{center}
% Each of the eight files can be compiled directly by the \LaTeX{} compiler.
%
% %%%%%%%%%%%%%%%%%%%%%%%%%%%%%%%%%%%%%%
% \paragraph{Main File.}
%
% The main file is called |cdocsamp.tex|.
%
% Load the \textsf{childdoc} definitions and
% declare the filename for the main document:
%    \begin{macrocode}
\input{childdoc.def}
\childdocmain{}
%    \end{macrocode}

% Optional override for |\version| flag:
%    \begin{macrocode}
%%\ifchilddoc\else\providecommand{\version}{draft}\fi
%    \end{macrocode}

% Define the default values for the |\version| flag
% (|final| for the main file and |draft| for childs):
%    \begin{macrocode}
\ifchilddoc
\providecommand{\version}{draft}
\else
\providecommand{\version}{final}
\fi
%    \end{macrocode}

% Load the standard document class:
%    \begin{macrocode}
\documentclass[12pt]{article}
%    \end{macrocode}

% Start the document body:
%    \begin{macrocode}
\begin{document}
%    \end{macrocode}

% Declare a title page.
% Print title, part of document being processed and version flag:
%    \begin{macrocode}
\addtocounter{page}{-1}
\begin{center}
{\LARGE\bfseries{}childdoc example\par}
\vspace{1cm}
\ifchilddoc
\ifchilddocmanual part\else chapter\fi:
`\childdocname' of `\childdocjob'\par
\else
main document: `\childdocjob'\par
\fi
version: \version\par
\end{center}
\newpage
%    \end{macrocode}

% Manually include selected file,
% otherwise process as usual:
%    \begin{macrocode}
\ifchilddocmanual
\section*{part `\childdocname'}
\input{\childdocname}
\else
%    \end{macrocode}

% Include the two chapters:
%    \begin{macrocode}
\include{cdocsch1}
\include{cdocsch2}
%    \end{macrocode}

% Include the two parts unless only chapters should be displayed:
%    \begin{macrocode}
\ifchilddoc\else
\section{part three}
\input{cdocspt3}
\section{part four}
\input{cdocspt4}
\fi
%    \end{macrocode}

% Process as usual until here:
%    \begin{macrocode}
\fi
%    \end{macrocode}

% End of document body:
%    \begin{macrocode}
\end{document}
%    \end{macrocode}
%\iffalse
%</samplemain>
%\fi
%
% %%%%%%%%%%%%%%%%%%%%%%%%%%%%%%%%%%%%%%
% \paragraph{Chapter Include Files.}
%
% The include files are called |cdocsch1.tex| and |cdocsch2.tex|.
%
%\iffalse
%<*samplechap1|samplechap2>
%\fi

% Optional override for |\version| flag:
%    \begin{macrocode}
%%\providecommand{\version}{final}
%    \end{macrocode}

% Include the main document:
%    \begin{macrocode}
\input{childdoc.def}
\childdocof{cdocsamp}
%    \end{macrocode}

%\iffalse
%</samplechap1|samplechap2>
%\fi
%
%\iffalse
%<*samplechap1>
%\fi
% Some text for chapter 1:
%    \begin{macrocode}
\section{one}
some text in chapter one
%    \end{macrocode}

%\iffalse
%</samplechap1>
%\fi
% Some text for chapter 2:
%\iffalse
%<*samplechap2>
%\fi
%    \begin{macrocode}
\section{two}
more text in chapter two
%    \end{macrocode}

%\iffalse
%</samplechap2>
%\fi
%
% %%%%%%%%%%%%%%%%%%%%%%%%%%%%%%%%%%%%%%
% \paragraph{Part Include Files.}
%
% The include files are called |cdocspt3.tex| and |cdocspt4.tex|.
%
%\iffalse
%<*samplepart3|samplepart4>
%\fi

% Optional override for |\version| flag:
%    \begin{macrocode}
%%\providecommand{\version}{final}
%    \end{macrocode}

% Include the main document:
%    \begin{macrocode}
\input{childdoc.def}
\childdocby{cdocsamp}
%    \end{macrocode}

%\iffalse
%</samplepart3|samplepart4>
%\fi
%
%\iffalse
%<*samplepart3>
%\fi
% Some text for part 3:
%    \begin{macrocode}
some text in part three
%    \end{macrocode}

%\iffalse
%</samplepart3>
%\fi
% Some text for part 4:
%\iffalse
%<*samplepart4>
%\fi
%    \begin{macrocode}
more text in part four
%    \end{macrocode}

%\iffalse
%</samplepart4>
%\fi
%
% %%%%%%%%%%%%%%%%%%%%%%%%%%%%%%%%%%%%%%
% \paragraph{Forwarding for a Complete Draft.}
%
% The following forwarding file |cdocsdrf.tex|
% compiles the main document in draft mode:
%\iffalse
%<*sampledraft>
%\fi
%    \begin{macrocode}
\def\version{draft}
\input{childdoc.def}
\childdocforward{cdocsamp}
%    \end{macrocode}

%\iffalse
%</sampledraft>
%\fi
%
% %%%%%%%%%%%%%%%%%%%%%%%%%%%%%%%%%%%%%%
% \paragraph{Forwarding for Final Version of the Chapters.}
%
% The following forwarding files |cdocsfn1.tex| and |cdocsfn2.tex|
% (with identical content)
% compile the final versions of the child documents
% |cdocsch1.tex| and |cdocsch2.tex|, respectively:
%\iffalse
%<*samplefinal>
%\fi
%    \begin{macrocode}
\def\version{final}
\input{childdoc.def}
\childdocforwardprefix[cdocsamp]{cdocsfn}{cdocsch}
%    \end{macrocode}

%\iffalse
%</samplefinal>
%\fi
%
% %%%%%%%%%%%%%%%%%%%%%%%%%%%%%%%%%%%%%%
% \paragraph{Command Line Processing.}
%
% The following three command lines generate the output files
% |cdocscld|, |cdocscl1| and |cdocscl2|
% which should be identical to
% |cdocsdrf|, |cdocsch1| and |cdocsfn2|, respectively:
% \begin{center}
% \begin{tabular}{l}
% |latex -jobname cdocscld \|\\
% |  "\def\version{draft}\input{childdoc.def}\childdocforward{cdocsamp}"|\\
% |latex -jobname cdocscl1 \|\\
% |  "\input{childdoc.def}\childdocforward[cdocsamp]{cdocsch1}"|\\
% |latex -jobname cdocscl2 \|\\
% |  "\def\version{final}\input{childdoc.def}\childdocforward{cdocsch2}"|
% \end{tabular}
% \end{center}
% Note that the trailing backslash on each first line
% merely continues the input to the second line
% (for convenient cut ant paste).
% Furthermore, the command |latex| can be replaced by any
% of its alternative versions such as |pdflatex|.
%
% %%%%%%%%%%%%%%%%%%%%%%%%%%%%%%%%%%%%%%%%%%%%%%%%%%%%%%%%%%%%%%%%%%%%%%%%%%%%%%
% %%%%%%%%%%%%%%%%%%%%%%%%%%%%%%%%%%%%%%%%%%%%%%%%%%%%%%%%%%%%%%%%%%%%%%%%%%%%%%
% \section{Implementation}
%\iffalse
%<*package>
%\fi
%
% This section describes the definitions file |childdoc.def|.

% The definitions cannot be loaded using |\usepackage| or |\RequirePackage|
% which has a mechanism to prevent loading a style file more than once.
% When loading the definitions by means of |\input|
% multiple instances have to be prevented manually:
%\iffalse
%This code needs to be before the `\ProvidesFile' directive
%which is defined at the beginning of this file.
%Therefore it is also placed there and commented out here.
%</package>
%<*discard>
%\fi
%    \begin{macrocode}
\ifdefined\childdocmain\endinput\fi
%    \end{macrocode}
%\iffalse
%</discard>
%<*package>
%\fi
%
% \macro{\ifchilddoc}
% \macro{\ifchilddocmanual}
% The conditional |\ifchilddoc| tells whether a
% child (true) or main (false) document is being compiled.
% The conditional |\ifchilddocmanual| tells whether
% the |\includeonly| mechanism is used (false) or
% the selection of child files must be performed manually (true).
% The definitions initialise to false:
%    \begin{macrocode}
\newif\ifchilddoc
\newif\ifchilddocmanual
%    \end{macrocode}

% \macro{\childdocname}
% \macro{\childdocjob}
% The macro |\childdocname| stores the name of the main document
% to be compiled. The macro |\childdocjob| stores the name of
% the document on which the \LaTeX{} compiler was originally invoked.
% The content of |\jobname| cannot be compared
% to filenames specified in the source due to different catcodes.
% The following code rescans |\jobname|, stores the result
% in |\childdocname| and saves a copy in |\childdocjob|:
%    \begin{macrocode}
\edef\childdocname{\scantokens\expandafter{\jobname\noexpand}}
\let\childdocjob\childdocname
%    \end{macrocode}

% \macro{\childdocdisable}
% The macro |\childdocdisable| prevents the main file
% from being processed more than once.
% At this stage, the main document command |\childdocmain|
% is assumed to be called once again where it should do nothing.
% Any subsequent call to it should prevent
% a secondary processing of the main document
% It overwrites the forwarding commands
% |\childdocof| and |\childdocforward|
% with empty macros to prevent further inclusions of the main document:
%    \begin{macrocode}
\newcommand{\childdocdisable}
{
  \renewcommand{\childdocmain}[1]{\renewcommand{\childdocmain}[1]{\endinput}}
  \renewcommand{\childdocof}[1]{}
  \renewcommand{\childdocby}[2][]{}
  \renewcommand{\childdocforward}[2][]{}
  \renewcommand{\childdocdisable}{}
}
%    \end{macrocode}

% \macro{\childdocmain}
% The macro |\childdocmain| is to be called at the top of the main file
% with nothing or the main filename (without extension) as argument.
% First, it breaks loops.
% If the argument is not empty and does not match |\childdocname|
% (which is set by the first inclusion of |childdoc.def|),
% |\ifchilddoc| is set to true, |\includeonly| is applied to the child file
% and |\jobname| is set to the main file
% (for proper handling of |.aux| files):
%    \begin{macrocode}
\newcommand{\childdocmain}[1]
{
  \childdocdisable\childdocmain{}
  \if?#1?\else
    \begingroup
      \def\childdoctmp{#1}
      \ifx\childdoctmp\childdocname
        \def\childdoctmp{}
      \else
        \def\childdoctmp
        {
          \childdoctrue
          \includeonly{\childdocname}
          \def\childdocjob{#1}
          \def\jobname{#1}
        }
      \fi
      \expandafter
    \endgroup
    \childdoctmp
  \fi
}
%    \end{macrocode}

% \macro{\childdocof}
% The command |\childdocof| redirects
% compilation to the main file |#1|.
%    \begin{macrocode}
\newcommand{\childdocof}[1]
{
  \childdocdisable
  \childdoctrue
  \includeonly{\childdocname}
  \def\jobname{#1}
  \def\childdocjob{#1}
  \input{#1}
}
%    \end{macrocode}

% \macro{\childdocby}
% The command |\childdocby| ....
%    \begin{macrocode}
\newcommand{\childdocby}[2][]
{
  \childdocdisable
  \childdoctrue
  \childdocmanualtrue
  \if?#1?\else
    \def\jobname{#2}
  \fi
  \def\childdocjob{#2}
  \input{#2}
  \endinput
}
%    \end{macrocode}

% \macro{\childdocforward}
% The command |\childdocforward| redirects
% compilation to the main file or
% (if the optional argument is given) a child file.
% Parameters are set as if the main file
% or a child file starting with |\childdocof| was compiled.
% Then compilation is handed over to the main file:
%    \begin{macrocode}
\newcommand{\childdocforward}[2][]
{
  \begingroup
    \if?#1?
      \def\childdoctmp
      {
        \def\childdocname{#2}
        \def\childdocjob{#2}
        \def\jobname{#2}
        \input{#2}
        \endinput
      }
    \else
      \def\childdoctmp
      {
        \childdocdisable
        \def\childdocname{#2}
        \childdoctrue
        \includeonly{#2}
        \def\childdocjob{#1}
        \def\jobname{#1}
        \input{#1}
        \endinput
      }
    \fi
    \expandafter
  \endgroup
  \childdoctmp
}
%    \end{macrocode}

% \macro{\childdocforwardprefix}
% The command |\childdocforwardprefix| redirects
% compilation to the main or a child file by means of a pattern.
% The prefix |#1| in the current filename is replaced by |#2|
% and the suffix of the current filename is kept
% (it is assumed that the filename does not contain the substring `|~~~|'
% which is used as a delimiter).
% Compilation is handed over to the new file by |\childdocforward|:
%    \begin{macrocode}
\newcommand{\childdocforwardprefix}[3][]
{
  \begingroup
    \def\childdocextract #2##1~~~{\def\childdoctmp{\childdocforward[#1]{#3##1}}}
    \expandafter\childdocextract\childdocname~~~
    \expandafter
  \endgroup
  \childdoctmp
}
%    \end{macrocode}

% \macro{\childdoc}
% The deprecated macro |\childdoc| is a legacy version of |\childdocmain|:
%    \begin{macrocode}
\newcommand{\childdoc}{\childdocmain}
%    \end{macrocode}

% \macro{\childdocredirect}
% The deprecated macro |\childdocredirect| is a legacy version
% of |\childdocforward| and |\childdocforwardprefix|:
%    \begin{macrocode}
\newcommand{\childdocredirect}[2][]
{
  \begingroup
    \if?#1?
      \def\childdoctmp{\childdocforward{#2}}
    \else
      \def\childdoctmp{\childdocforwardprefix{#1}{#2}}
    \fi
    \expandafter
  \endgroup
  \childdoctmp
}
%    \end{macrocode}

%\iffalse
%</package>
%\fi
%
\endinput

\childdocby{cdocsamp}
%    \end{macrocode}

%\iffalse
%</samplepart3|samplepart4>
%\fi
%
%\iffalse
%<*samplepart3>
%\fi
% Some text for part 3:
%    \begin{macrocode}
some text in part three
%    \end{macrocode}

%\iffalse
%</samplepart3>
%\fi
% Some text for part 4:
%\iffalse
%<*samplepart4>
%\fi
%    \begin{macrocode}
more text in part four
%    \end{macrocode}

%\iffalse
%</samplepart4>
%\fi
%
% %%%%%%%%%%%%%%%%%%%%%%%%%%%%%%%%%%%%%%
% \paragraph{Forwarding for a Complete Draft.}
%
% The following forwarding file |cdocsdrf.tex|
% compiles the main document in draft mode:
%\iffalse
%<*sampledraft>
%\fi
%    \begin{macrocode}
\def\version{draft}
% \iffalse
%
% childdoc.dtx Copyright (C) 2017-2018 Niklas Beisert
%
% This work may be distributed and/or modified under the
% conditions of the LaTeX Project Public License, either version 1.3
% of this license or (at your option) any later version.
% The latest version of this license is in
%   http://www.latex-project.org/lppl.txt
% and version 1.3 or later is part of all distributions of LaTeX
% version 2005/12/01 or later.
%
% This work has the LPPL maintenance status `maintained'.
%
% The Current Maintainer of this work is Niklas Beisert.
%
% This work consists of the files childdoc.dtx and childdoc.ins
% and the derived files childdoc.def and cdocsamp.tex with
% cdocsch1.tex, cdocsch2.tex, cdocsdrf.tex, cdocsfn1.tex, cdocsfn2.tex.
%
%<package>\ifdefined\childdocmain\endinput\fi
%<package>\ProvidesFile{childdoc.def}[2018/12/30 v2.0 child document driver]
%<samplemain>\ProvidesFile{cdocsamp.tex}[2018/12/30 v2.0 sample for childdoc]
%<*driver>
%\ProvidesFile{childdoc.drv}[2018/12/30 v2.0 childdoc reference manual file]
\PassOptionsToClass{10pt,a4paper}{article}
\documentclass{ltxdoc}

\usepackage[margin=35mm]{geometry}
\usepackage{hyperref}
\usepackage{hyperxmp}
\usepackage[usenames]{color}

\hypersetup{colorlinks=true}
\hypersetup{pdfstartview=FitH}
\hypersetup{pdfpagemode=UseNone}
\hypersetup{pdfsource={}}
\hypersetup{pdflang={en-UK}}
\hypersetup{pdfcopyright={Copyright 2017-2018 Niklas Beisert.
  This work may be distributed and/or modified under the
  conditions of the LaTeX Project Public License, either version 1.3
  of this license or (at your option) any later version.}}
\hypersetup{pdflicenseurl={http://www.latex-project.org/lppl.txt}}
\hypersetup{pdfcontactaddress={ETH Zurich, ITP, HIT K,
  Wolfgang-Pauli-Strasse 27}}
\hypersetup{pdfcontactpostcode={8093}}
\hypersetup{pdfcontactcity={Zurich}}
\hypersetup{pdfcontactcountry={Switzerland}}
\hypersetup{pdfcontactemail={nbeisert@itp.phys.ethz.ch}}
\hypersetup{pdfcontacturl={http://people.phys.ethz.ch/\xmptilde nbeisert/}}

\newcommand{\secref}[1]{\hyperref[#1]{section \ref*{#1}}}

\parskip1ex
\parindent0pt
\let\olditemize\itemize
\def\itemize{\olditemize\parskip0pt}

\begin{document}

\title{The \textsf{childdoc} Package}
\hypersetup{pdftitle={The childdoc Package}}
\author{Niklas Beisert\\[2ex]
  Institut f\"ur Theoretische Physik\\
  Eidgen\"ossische Technische Hochschule Z\"urich\\
  Wolfgang-Pauli-Strasse 27, 8093 Z\"urich, Switzerland\\[1ex]
  \href{mailto:nbeisert@itp.phys.ethz.ch}
  {\texttt{nbeisert@itp.phys.ethz.ch}}}
\hypersetup{pdfauthor={Niklas Beisert}}
\hypersetup{pdfsubject={Manual for the LaTeX2e Package childdoc}}
\date{30 December 2018, \textsf{v2.0}}
\maketitle

\begin{abstract}\noindent
\textsf{childdoc} is a \LaTeXe{} package
that enables the direct compilation
of document sections included by |\include|
to individual files.
\end{abstract}

\begingroup
\parskip0ex
\tableofcontents
\endgroup

%%%%%%%%%%%%%%%%%%%%%%%%%%%%%%%%%%%%%%%%%%%%%%%%%%%%%%%%%%%%%%%%%%%%%%%%%%%%%%%%
%%%%%%%%%%%%%%%%%%%%%%%%%%%%%%%%%%%%%%%%%%%%%%%%%%%%%%%%%%%%%%%%%%%%%%%%%%%%%%%%
\section{Introduction}

\LaTeX{} provides a mechanism to structure a large document (such as a book)
into a main file and several child files (containing the chapters)
using the |\include| command.
This mechanism is beneficial for documents
which span hundreds of pages in order to
make the source file(s) more manageable.
Moreover, compilation can be restricted to
selected child files by means of the |\includeonly| command.
The latter feature can be used to reduce the compilation time while editing
(this was significantly more useful in the earlier days of \LaTeX{})
or to generate a smaller document which is easier to navigate.
Another application of |\includeonly| is to generate
documents consisting of selected parts of the complete document.

However, there are a few drawbacks of the plain |\include| mechanism:
\begin{itemize}
\item
The child files cannot be compiled on their own,
they can only be compiled via the main file.
A naive editing environment
(such as a text editor with an option
to have the current file processed by \LaTeX)
may require one to switch to the main file before compiling;
attempting to compile the child file produces errors.
\item
The main file must be modified (each time)
to adjust the |\includeonly| command
to the present needs. This easily leaves the main file in a messy state.
\item
The generated document will always carry the filename
of the main document. This is inconvenient if
several child files are to be compiled and
to be kept for distribution.
\end{itemize}

The present package provides a simple interface
to make child files individually compilable by \LaTeX{}.
Compiling a child file then has the same effect as compiling
the main file with an |\includeonly| command
to select the appropriate child.
Moreover the generated document will carry the name of the child
rather than the main file.
This resolves all three above issues.

This feature is meant to make the editing of books,
thesis documents and lecture notes somewhat more convenient.
However, the package can also be used efficiently for
composing a series of documents (such as exercise sheets)
which are typically distributed individually.
It then assists the author in generating the individual documents
(potentially in different versions)
as well as a document containing the collected series.
Another application is in developing style files
or other kinds of included material
where compilation of the style file could redirect
to a sample or test file.

%%%%%%%%%%%%%%%%%%%%%%%%%%%%%%%%%%%%%%%%%%%%%%%%%%%%%%%%%%%%%%%%%%%%%%%%%%%%%%%%
%%%%%%%%%%%%%%%%%%%%%%%%%%%%%%%%%%%%%%%%%%%%%%%%%%%%%%%%%%%%%%%%%%%%%%%%%%%%%%%%
\section{Usage}

First of all, the package \textsf{childdoc} is \emph{not} a standard
\LaTeXe{} |.sty| style file! Therefore it needs to be invoked in
a non-standard way.

%%%%%%%%%%%%%%%%%%%%%%%%%%%%%%%%%%%%%%%%%%%%%%%%%%%%%%%%%%%%%%%%%%%%%%%%%%%%%%%%
\subsection{Included Files}
\label{sec:include}

%%%%%%%%%%%%%%%%%%%%%%%%%%%%%%%%%%%%%%%%
\DescribeMacro{\childdocmain}
To use the package, add the commands
\begin{center}
\begin{tabular}{l}
|\input{childdoc.def}|\\
|\childdocmain{}|\\
\end{tabular}
\end{center}
at the very top of the main \LaTeX{} file,
in particular \emph{before} the |\documentclass| statement!
The argument of |\childdocmain| should be left empty
(but it must be present).

%%%%%%%%%%%%%%%%%%%%%%%%%%%%%%%%%%%%%%%%
\DescribeMacro{\childdocof}
Furthermore, add the commands
\begin{center}
\begin{tabular}{l}
|\input{childdoc.def}|\\
|\childdocof{|\textit{main}|}|\\
\end{tabular}
\end{center}
at the top of every child file \textit{child}
which is included by |\include{|\textit{child}|}|
from within the main file
(or at least for those files to be compiled individually).
The argument \textit{main} must be the filename of the main file.

There are a couple of
considerations in setting up the main and child documents:

%%%%%%%%%%%%%%%%%%%%%%%%%%%%%%%%%%%%%%%%
\paragraph{Restrictions.}

Please note the following restrictions:
\begin{itemize}
\item
|\childdocmain| must be called with one argument \textit{main}
to ensure compatibility with earlier version of the package.
It must either be empty (|\childdocmain{}|)
or precisely match the filename of the main file in which it is specified.
See \secref{sec:detection} for further information.
\item
The filename \textit{main} must be specified without the |.tex| extension.
\item
The filename \textit{main} is case sensitive
(even in case-insensitive file systems)
due to internal string comparison.
\item
The argument \textit{main} should be fully expanded, it cannot be a macro.
\item
Subdirectories and special characters should be avoided in filenames.
\item
The command |\childdocmain{|\textit{main}|}| must be followed by a whitespace.
It should not be followed immediately by another command
or by a comment mark `|%|'.
This is because the \TeX{} parser reads the token immediately following
the argument of |\childdocmain| and puts it
at the beginning of every child section;
however, a white\-space is ignored.
\end{itemize}

%%%%%%%%%%%%%%%%%%%%%%%%%%%%%%%%%%%%%%%%
\paragraph{Content of Main File.}

It is advisable to place all content in the child files included by |\include|.
Any output contained in the main file will appear in all child documents
unless suppressed manually;
it cannot be suppressed automatically by the |\includeonly| directive
and thus should normally be avoided.
A method to include some content in the main file
by means of conditional processing is described in \secref{sec:conditional}.

%%%%%%%%%%%%%%%%%%%%%%%%%%%%%%%%%%%%%%%%
\paragraph{Page Numbering.}

When only a part of the document is compiled,
the appropriate numbering of pages
(as well as other status parameters)
is determined from the |.aux| files.
The latter contain information from previous passes.
However this information needs to propagate through
all intermediate child documents.
Therefore the page numbering in child documents may well
be inconsistent until the complete document is compiled at least once.

A useful (if unconventional) way to always ensure a consistent
page numbering is to restart the numbering in each child document
and denote the pages by `\textit{child}|.|\textit{page}'
where \textit{child} represents the chapter/section number of the child file.
This can be achieved by the command
|\numberwithin{page}{|\textit{child}|}|
of the \textsf{amsmath} package
where \textit{child} can be |chapter| or |section|
depending on the chosen structuring.
Alternatively, one can modify the macro |\thepage| appropriately
and reset the counter |page| at the start of each child file.

%%%%%%%%%%%%%%%%%%%%%%%%%%%%%%%%%%%%%%%%%%%%%%%%%%%%%%%%%%%%%%%%%%%%%%%%%%%%%%%%
\subsection{Conditional Processing}
\label{sec:conditional}

The package provides a mechanism to compile different versions
of a document. To customise the versions further some conditional processing
can come in handy to distinguish which version is being compiled.
The package provides two macros to describe the compilation context:

%%%%%%%%%%%%%%%%%%%%%%%%%%%%%%%%%%%%%%%%
\DescribeMacro{\ifchilddoc}
The conditional |\ifchilddoc| distinguishes between the compilation of
child documents and the main document:
%
\begin{center}
|\ifchilddoc |\textit{child-code}| |[|\||else |\textit{main-code}]| \||fi|
\end{center}

%%%%%%%%%%%%%%%%%%%%%%%%%%%%%%%%%%%%%%%%
\DescribeMacro{\childdocname}
\DescribeMacro{\childdocjob}
The macro |\childdocname| contains the filename (without extension)
of the main or child file being processed.
Note that |\childdocjob| will always contain the name of the main file.

%%%%%%%%%%%%%%%%%%%%%%%%%%%%%%%%%%%%%%%%
\paragraph{Title Page.}

Conditional processing can be used to include a title or banner page
in the main document when proper precautions are taken.
Importantly, the code in the main file should ensure that the page counter
(as well as other status parameters which are stored in the |.aux| files)
takes the same value after the conditional processing.
Otherwise the page numbers may take divergent values
depending on which part is compiled.

For example, a title page could be declared by:
%
\begin{center}
\begin{tabular}{l}
|\ifchilddoc\||else|\\
|\addtocounter{page}{-1}|\\
\textit{code for title page}\\
|\newpage|\\
|\||fi|
\end{tabular}
\end{center}
%
A banner page for the child documents can be generated by:
%
\begin{center}
\begin{tabular}{l}
|\ifchilddoc|\\
|\addtocounter{page}{-1}|\\
\textit{code for banner page}\\
|\newpage|\\
|\||fi|
\end{tabular}
\end{center}
%
Here one could write a message such as:
\begin{center}
|This is the part \childdocname{} of \childdocjob{}.|
\end{center}

%%%%%%%%%%%%%%%%%%%%%%%%%%%%%%%%%%%%%%%%%%%%%%%%%%%%%%%%%%%%%%%%%%%%%%%%%%%%%%%%
\subsection{Flags}
\label{sec:flags}

The package makes it easy to generate different versions
of the main or child documents.
To this end compilation flags can be defined
and assigned different default values.
They will be particularly useful in conjunction
with the forwarding mechanism described in \secref{sec:forward}.

For example, it may be useful to have a flag |\version|
which can be set to |draft| or |final|.
The document source will contain some conditional code
depending on the value of |\version|.
Suppose further, the flag should default to |final| for the main file
and to |draft| for child files
which is a natural assignment for editing the document.
This is achieved by placing the following code
in the preamble of the main document
(below the |\childdocmain| directive):
%
\begin{center}
\begin{tabular}{l}
|\ifchilddoc|\\
|\providecommand{\version}{draft}|\\
|\||else|\\
|\providecommand{\version}{final}|\\
|\||fi|
\end{tabular}
\end{center}
%
The definition by |\providecommand| makes sure
that previous definitions are not overwritten.
Further statements |\providecommand{\version}{...}|
can thus be added before the above code to override it.

For the main file, one might add a line
(between |\childdocmain| and the above block)
%
\begin{center}
|%\ifchilddoc\||else\providecommand{\version}{draft}\||fi|
\end{center}
%
which can be uncommented to produce a draft version.
Likewise one can add a line to the very top of a child file
(above the |\childdocof{|\textit{main}|}| directive)
%
\begin{center}
|%\providecommand{\version}{final}|
\end{center}
%
which can be uncommented to produce the final version of this child document.

%%%%%%%%%%%%%%%%%%%%%%%%%%%%%%%%%%%%%%%%%%%%%%%%%%%%%%%%%%%%%%%%%%%%%%%%%%%%%%%%
\subsection{Forwarding}
\label{sec:forward}

Different versions of the main or child documents
using compilation flags as described in \secref{sec:flags}
can be (permanently) stored in different files
for convenient compilation, viewing and distribution.
To this end, the package defines a command
to pass on compilation to a different file:

%%%%%%%%%%%%%%%%%%%%%%%%%%%%%%%%%%%%%%%%
\DescribeMacro{\childdocforward}
The command |\childdocforward| redirects processing to
another source file:
%
\begin{center}
\begin{tabular}{l}
|\input{childdoc.def}|\\
|\childdocforward[|\textit{main}|]{|\textit{dest}|}|\\
\end{tabular}
\end{center}
%
The argument \textit{dest} is the destination file
(without extension).
It should be the main file or one of the child files.
Note that further \textsf{childdoc} directives
such as |\childdocof| and |\childdocforward|
in the indicated file will be processed in this form.
The optional argument \textit{main}
passes on directly to the main file \textit{main}
while pretending to compile the child \textit{dest}.
This form behaves as if \textit{dest}
issues |\childdocof{|\textit{main}|}| right away,
and no further \textsf{childdoc} directives will be processed.

%%%%%%%%%%%%%%%%%%%%%%%%%%%%%%%%%%%%%%%%
\DescribeMacro{\...prefix}
In the alternative form |\childdocforwardprefix|,
%
\begin{center}
\begin{tabular}{l}
|\input{childdoc.def}|\\
|\childdocforwardprefix[|\textit{main}|]{|\textit{prefix}|}{|\textit{dest}|}|
\end{tabular}
\end{center}
%
the destination file is determined by a pattern
depending on the current file:
To make this work, the current file must be called
`{\textit{prefix}\hspace{0.2em}\textit{suffix}}'
with \textit{prefix} matching precisely the argument.
Processing is then passed on to the file
`{\textit{dest}\hspace{0.2em}\textit{suffix}}'.
Surely, the same effect is achieved by
directly specifying the
argument `{\textit{dest}\hspace{0.2em}\textit{suffix}}'
in the first form.
However, that requires to set up a different file
for each child. With the alternative form of the command
all these files can have exactly the same content
which simplifies setting them up and maintaining them.

For example, the following file |draft.tex|
with a compilation flag |\version| as described in \secref{sec:flags}
compiles the main document as a draft:
%
\begin{center}
\begin{tabular}{l}
|\def\version{draft}|\\
|\input{childdoc.def}|\\
|\childdocforward{|\textit{main}|}|
\end{tabular}
\end{center}
%
Likewise, the following files |final|\textit{nn}|.tex|
compile the final version of the child document
|child|\textit{nn}|.tex|:
%
\begin{center}
\begin{tabular}{l}
|\def\version{final}|\\
|\input{childdoc.def}|\\
|\childdocforwardprefix{final}{child}|
\end{tabular}
\end{center}
%

Note that when several versions of a main file and/or of each child file
are to be generated, it may be convenient to set up a |Makefile| or
shell script to automatise the process.

%%%%%%%%%%%%%%%%%%%%%%%%%%%%%%%%%%%%%%%%%%%%%%%%%%%%%%%%%%%%%%%%%%%%%%%%%%%%%%%%
\subsection{Command Line Processing}
\label{sec:commandline}

The effect of redirection files can also be achieved by invoking
the \LaTeX{} compiler with a more elaborate command line.
Most conveniently this should be done as part
of a shell script or a |Makefile|.

When using \textsf{childdoc} in the main file, the following
command lines effectively perform a redirection
(note that depending on the shell being used,
backslashes may have to be doubled: `|\|' $\to$ `|\\|'):
%
\begin{center}
|... -jobname "|\textit{target}|" |\\|"|[\textit{flags}]%
|\input{childdoc.def}\childdocforward[|\textit{main}|]{|\textit{dest}|}"|
\end{center}
%
Here \textit{target} is the name of the output file,
\textit{main} is the name of the main file
and \textit{dest} is the name of the main or child file to be processed
(all filenames without extensions).
The optional argument \textit{main} can be omitted
if \textit{main} matches \textit{dest}.
Optionally, compilation \textit{flags} can be defined via |\def| commands.
This command line makes the \TeX{} engine believe
it is compiling the file \textit{target}
whose content is specified as the latter parameter.
The provided code then forwards the processing to
\textit{main} or \textit{dest} as described in \secref{sec:forward}.

%%%%%%%%%%%%%%%%%%%%%%%%%%%%%%%%%%%%%%%%%%%%%%%%%%%%%%%%%%%%%%%%%%%%%%%%%%%%%%%%
\subsection{Include by Input}
\label{sec:input}

Including child documents by |\include| has some restrictions by design.
Most notably, the content of a child document always occupies
its own set of pages; pages cannot be shared between child documents.
Usually, this behaviour makes perfect sense
because each child document contain an essential part of the document.
However, in some situations it may be desirable to compose
a document from a collection of parts
without having mandatory page breaks between then.
For this case, the package
provides a mechanism to include parts
by |\input| which can also be processed individually.
However, by construction this mechanism
requires manual handling of the content to be output.

%%%%%%%%%%%%%%%%%%%%%%%%%%%%%%%%%%%%%%%%
\DescribeMacro{\ifchilddocmanual}
The main file should be prepared as usual, see \secref{sec:include}.
However, the document body must make a distinction
between processing of an individual part and of the main document, e.g.:
%
\begin{center}
\begin{tabular}{l}
|\ifchilddocmanual|\\
|\input{\childdocname}|\\
|\||else|\\
\textit{document body with }|\input{|\textit{part}|}|\\
|\||fi|
\end{tabular}
\end{center}
%
The conditional |\ifchilddocmanual| is true whenever
a part to be included by |\input| is being compiled,
and the name of the part is stored in |\childdocname|.

%%%%%%%%%%%%%%%%%%%%%%%%%%%%%%%%%%%%%%%%
\DescribeMacro{\childdocby}
Each part to be included by |\input| should start with:
%
\begin{center}
\begin{tabular}{l}
|\input{childdoc.def}|\\
|\childdocby{|\textit{main}|}|\\
\end{tabular}
\end{center}
%
The directive |\childdocby| is similar to |\childdocof|
described in \secref{sec:include},
but the subsequent selection of content must be done manually.
To that end, both |\ifchilddoc| and |\ifchilddocmanual|
will be true upon processing of a part,
and the name of the part is stored in |\childdocname|.
Note that |\jobname| will be set to the filename of the current part
so that each part receives an individual |.aux| file
that does not interfere with the |.aux| file(s) of the main document.
This behaviour can be altered by the alternative form
|\childdocby[*]{|\textit{main}|}| (with a non-empty optional argument)
which uses the |.aux| file of the main document
by setting |\jobname| to \textit{main}.

%%%%%%%%%%%%%%%%%%%%%%%%%%%%%%%%%%%%%%%%%%%%%%%%%%%%%%%%%%%%%%%%%%%%%%%%%%%%%%%%
\subsection{Driver Development}
\label{sec:driver}

The \textsf{childdoc} mechanism can also be use for the development
of definition files such as \LaTeX{} styles or classes.
This case differs from the above setup with multiple parts
included by |\include| in that no |\includeonly| should be invoked.
This can be achieved by starting the include file
(before |\ProvidesPackage|) with:
%
\begin{center}
\begin{tabular}{l}
|\input{childdoc.def}|\\
|\childdocforward{|\textit{main}|}|\\
\end{tabular}
\end{center}
%
or alternatively with:
%
\begin{center}
\begin{tabular}{l}
|\input{childdoc.def}|\\
|\childdocby{|\textit{main}|}|\\
\end{tabular}
\end{center}
%
Both forms have slightly different effects as described above.
The main file is prepared as usual, see \secref{sec:include}.

%%%%%%%%%%%%%%%%%%%%%%%%%%%%%%%%%%%%%%%%%%%%%%%%%%%%%%%%%%%%%%%%%%%%%%%%%%%%%%%%
\subsection{Legacy Detection}
\label{sec:detection}

The directive |\childdocmain| in the main file can detect
whether the complete document or merely a child is to be compiled
even without using the directive |\childdocof|.
This method is deprecated because it is less robust
and there is no compelling reason to use it;
it is merely provided for backward compatibility
and it may be removed in future versions.

If the detection mechanism is to be used,
it is mandatory to correctly specify
the filename of the main file as the argument of |\childdocmain|:
%
\begin{center}
\begin{tabular}{l}
|\input{childdoc.def}|\\
|\childdocmain{|\textit{main}|}|\\
\end{tabular}
\end{center}
%
If |\jobname| does not match the argument \textit{main} of |\childdocmain|,
it is assumed that |\jobname| points to the child file to be compiled.
When using |\childdocmain| with the main file specified as argument,
it suffices to start a child file
with just |\input{|\textit{main}|}|
without loading of the package and using |\childdocof|.
If instead all processing is done
with the appropriate \textsf{childdoc} directives,
the argument of \textit{main} of |\childdocmain| can be empty.

An alternative version of the command line processing described
in \secref{sec:commandline} using the detection mechanism reads:
%
\begin{center}
|... -jobname "|\textit{target}|" "|[\textit{flags}]%
[|\def\jobname{|\textit{dest}|}|]|\input{|\textit{main}|}"|
\end{center}

%%%%%%%%%%%%%%%%%%%%%%%%%%%%%%%%%%%%%%%%%%%%%%%%%%%%%%%%%%%%%%%%%%%%%%%%%%%%%%%%
\subsection{Manual Code}
\label{sec:manual}

In case one cannot be certain whether the definitions file |childdoc.def|
is installed on the target \TeX{} distribution
and one prefers not to ship it,
it is conceivable to paste a few relevant commands into the sources.

To that end, drop all statements |\input{childdoc.def}|
and perform the replacements as outlined below.
Instead of |\childdocmain{|\textit{main}|}| add the following code
to the top of the main file:
%
\begin{center}
\begin{tabular}{l}
|\||ifdefined\childdocname\endinput\||fi\newif\ifchilddoc|\\
|\edef\childdocname{\scantokens\expandafter{\jobname\noexpand}}|\\
|\def\childdocmain{|\textit{main}|}\||ifx\childdocmain\childdocname\||else|\\
|\childdoctrue\includeonly{\childdocname}\let\jobname\childdocmain\||fi|\\
\end{tabular}
\end{center}
%
Instead of |\childdocof{|\textit{main}|}| just include the main file
at the top of each child file:
%
\begin{center}
|\input{|\textit{main}|}|
\end{center}
%
A simple redirection |\childdocforward{|\textit{dest}|}| is achieved by:
%
\begin{center}
|\def\jobname{|\textit{dest}|}\input{\jobname}|
\end{center}
%
The redirection with prefix
|\childdocforwardprefix[|\textit{prefix}|]{|\textit{dest}|}|
is accomplished by:
%
\begin{center}
\begin{tabular}{l}
|{\edef\jobname{\scantokens\expandafter{\jobname\noexpand}}|\\
|\def\redirectjob |\textit{prefix}|#1~~~{\gdef\jobname{|\textit{dest}|#1}}|\\
|\expandafter\redirectjob\jobname~~~}\input{\jobname}|
\end{tabular}
\end{center}

In an alternative approach,
child documents can be compiled by a specific command line
without additional code or specific definitions:
%
\begin{center}
|... -jobname "|\textit{target}|" "|[\textit{flags}]%
|\includeonly{|\textit{dest}|}\input{|\textit{main}|}"|
\end{center}
%

%%%%%%%%%%%%%%%%%%%%%%%%%%%%%%%%%%%%%%%%%%%%%%%%%%%%%%%%%%%%%%%%%%%%%%%%%%%%%%%%
%%%%%%%%%%%%%%%%%%%%%%%%%%%%%%%%%%%%%%%%%%%%%%%%%%%%%%%%%%%%%%%%%%%%%%%%%%%%%%%%
\section{Information}

%%%%%%%%%%%%%%%%%%%%%%%%%%%%%%%%%%%%%%%%%%%%%%%%%%%%%%%%%%%%%%%%%%%%%%%%%%%%%%%%
\subsection{Copyright}

Copyright \copyright{} 2017--2018 Niklas Beisert

This work may be distributed and/or modified under the
conditions of the \LaTeX{} Project Public License, either version 1.3
of this license or (at your option) any later version.
The latest version of this license is in
  \url{http://www.latex-project.org/lppl.txt}
and version 1.3 or later is part of all distributions of \LaTeX{}
version 2005/12/01 or later.

This work has the LPPL maintenance status `maintained'.

The Current Maintainer of this work is Niklas Beisert.

This work consists of the files |README.txt|, |childdoc.ins| and |childdoc.dtx|
as well as the derived files |childdoc.def|, |cdocsamp.tex|
with |cdocsch1.tex|, |cdocsch2.tex|, |cdocspt3.tex|, |cdocspt4.tex|,
|cdocsdrf.tex|, |cdocsfn1.tex|, |cdocsfn2.tex|
as well as |childdoc.pdf|.

%%%%%%%%%%%%%%%%%%%%%%%%%%%%%%%%%%%%%%%%%%%%%%%%%%%%%%%%%%%%%%%%%%%%%%%%%%%%%%%%
\subsection{Files and Installation}

The package consists of the files:
%
\begin{center}
\begin{tabular}{ll}
    |README.txt|   & readme file \\
    |childdoc.ins| & installation file \\
    |childdoc.dtx| & source file \\
    |childdoc.def| & definition file \\
    |cdocsamp.tex| & sample main file \\
    |cdocsch1.tex| & sample include file \\
    |cdocsch2.tex| & sample include file \\
    |cdocspt3.tex| & sample part file \\
    |cdocspt4.tex| & sample part file \\
    |cdocsdrf.tex| & sample redirection file \\
    |cdocsfn1.tex| & sample redirection file \\
    |cdocsfn2.tex| & sample redirection file \\
    |childdoc.pdf| & manual
\end{tabular}
\end{center}
%
The distribution consists of the files
|README.txt|, |childdoc.ins| and |childdoc.dtx|.
%
\begin{itemize}
\item
Run (pdf)\LaTeX{} on |childdoc.dtx|
to compile the manual |childdoc.pdf| (this file).
\item
Run \LaTeX{} on |childdoc.ins| to create the definitions file |childdoc.def|
and the sample |cdocsamp.tex| with include files
|cdocsch1.tex|, |cdocsch2.tex|, |cdocspt3.tex|, |cdocspt4.tex|,
|cdocsdrf.tex|, |cdocsfn1.tex|, |cdocsfn2.tex|.
Then copy the file |childdoc.def| to an appropriate directory of your \LaTeX{}
distribution, e.g.\ \textit{texmf-root}|/tex/latex/childdoc|.
\end{itemize}

%%%%%%%%%%%%%%%%%%%%%%%%%%%%%%%%%%%%%%%%%%%%%%%%%%%%%%%%%%%%%%%%%%%%%%%%%%%%%%%%
\subsection{Related CTAN Packages}

There are several other packages which offer a similar functionality:
%
\begin{itemize}
\item
The packages
\href{http://ctan.org/pkg/docmute}{\textsf{docmute}},
\href{http://ctan.org/pkg/includex}{\textsf{includex}} and
\href{http://ctan.org/pkg/standalone}{\textsf{standalone}}
provide commands to include only the document body of
a child file thus allowing both files to be compiled individually.
\item
The packages \href{http://ctan.org/pkg/subdocs}{\textsf{subdocs}}
and \href{http://ctan.org/pkg/subfiles}{\textsf{subfiles}}
provide structures in which the main and child documents can be
encapsulated and allowing them to be compiled individually.
The inclusion mechanism is different from the conventional |\include|.
\item
The package \href{http://ctan.org/pkg/combine}{\textsf{combine}}
is an elaborate solution to combine several documents into one.
\end{itemize}
%
See also the CTAN topic \href{http://ctan.org/topic/subdocs}{\textsf{subdocs}}
for further related packages.
The present package differs from the above solutions in that
a document structure constructed with the conventional |\include| mechanism
just needs two extra commands at the top of every file
such that all constituent files can be compiled individually.

%%%%%%%%%%%%%%%%%%%%%%%%%%%%%%%%%%%%%%%%%%%%%%%%%%%%%%%%%%%%%%%%%%%%%%%%%%%%%%%%
%\subsection{Feature Suggestions}
%
%The following is a list of features which may be useful for future
%versions of this package:
%%
%\begin{itemize}
%\item
%\ldots
%\end{itemize}

%%%%%%%%%%%%%%%%%%%%%%%%%%%%%%%%%%%%%%%%%%%%%%%%%%%%%%%%%%%%%%%%%%%%%%%%%%%%%%%%
\subsection{Revision History}

%%%%%%%%%%%%%%%%%%%%%%%%%%%%%%%%%%%%%%%%
\paragraph{v2.0:} 2018/12/30

\begin{itemize}
\item
immediate forward processing
\item
added |\childdocby| mechanism
\item
manual restructured
\end{itemize}

%%%%%%%%%%%%%%%%%%%%%%%%%%%%%%%%%%%%%%%%
\paragraph{v1.6:} 2018/01/17

\begin{itemize}
\item
application for development of include files
\item
corrections to manual
\end{itemize}

%%%%%%%%%%%%%%%%%%%%%%%%%%%%%%%%%%%%%%%%
\paragraph{v1.5:} 2017/05/21

\begin{itemize}
\item
more complete structuring introduced
\item
|\childdocof| introduced
\item
|\childdoc| renamed to |\childdocmain|
\item
|\childredirect| renamed to |\childdocforward| and |\childdocforwardprefix|
and functionality expanded
\end{itemize}

%%%%%%%%%%%%%%%%%%%%%%%%%%%%%%%%%%%%%%%%
\paragraph{v1.0:} 2017/04/27

\begin{itemize}
\item
manual and install package
\item
first version published on CTAN
\end{itemize}

%%%%%%%%%%%%%%%%%%%%%%%%%%%%%%%%%%%%%%%%
\paragraph{v0.6:} 2017/04/26

\begin{itemize}
\item
redirection mechanism added
\end{itemize}

%%%%%%%%%%%%%%%%%%%%%%%%%%%%%%%%%%%%%%%%
\paragraph{v0.5:} 2017/04/26

\begin{itemize}
\item
functionality in definition file
\end{itemize}


%%%%%%%%%%%%%%%%%%%%%%%%%%%%%%%%%%%%%%%%%%%%%%%%%%%%%%%%%%%%%%%%%%%%%%%%%%%%%%%%
%%%%%%%%%%%%%%%%%%%%%%%%%%%%%%%%%%%%%%%%%%%%%%%%%%%%%%%%%%%%%%%%%%%%%%%%%%%%%%%%
%%%%%%%%%%%%%%%%%%%%%%%%%%%%%%%%%%%%%%%%%%%%%%%%%%%%%%%%%%%%%%%%%%%%%%%%%%%%%%%%
\appendix

\settowidth\MacroIndent{\rmfamily\scriptsize 000\ }

 \DocInput{childdoc.dtx}

\end{document}
%</driver>
% \fi
%
% %%%%%%%%%%%%%%%%%%%%%%%%%%%%%%%%%%%%%%%%%%%%%%%%%%%%%%%%%%%%%%%%%%%%%%%%%%%%%%
% %%%%%%%%%%%%%%%%%%%%%%%%%%%%%%%%%%%%%%%%%%%%%%%%%%%%%%%%%%%%%%%%%%%%%%%%%%%%%%
% \section{Sample}
%\iffalse
%<*samplemain>
%\fi
%
% The following presents a sample document
% with two chapters, two parts, a title page,
% a compile flag as well as three forwarding files to set the flag.
% It consists of eight |.tex| files:
% \begin{center}
% \begin{tabular}{ll}
% |cdocsamp.tex|&main file\\
% |cdocsch1.tex|&include file for chapter 1\\
% |cdocsch2.tex|&include file for chapter 2\\
% |cdocspt3.tex|&include file for part 3\\
% |cdocspt4.tex|&include file for part 4\\
% |cdocsdrf.tex|&forwarding file for main file in draft mode\\
% |cdocsfi1.tex|&forwarding file for final version of chapter 1\\
% |cdocsfi2.tex|&forwarding file for final version of chapter 2\\
% \end{tabular}
% \end{center}
% Each of the eight files can be compiled directly by the \LaTeX{} compiler.
%
% %%%%%%%%%%%%%%%%%%%%%%%%%%%%%%%%%%%%%%
% \paragraph{Main File.}
%
% The main file is called |cdocsamp.tex|.
%
% Load the \textsf{childdoc} definitions and
% declare the filename for the main document:
%    \begin{macrocode}
\input{childdoc.def}
\childdocmain{}
%    \end{macrocode}

% Optional override for |\version| flag:
%    \begin{macrocode}
%%\ifchilddoc\else\providecommand{\version}{draft}\fi
%    \end{macrocode}

% Define the default values for the |\version| flag
% (|final| for the main file and |draft| for childs):
%    \begin{macrocode}
\ifchilddoc
\providecommand{\version}{draft}
\else
\providecommand{\version}{final}
\fi
%    \end{macrocode}

% Load the standard document class:
%    \begin{macrocode}
\documentclass[12pt]{article}
%    \end{macrocode}

% Start the document body:
%    \begin{macrocode}
\begin{document}
%    \end{macrocode}

% Declare a title page.
% Print title, part of document being processed and version flag:
%    \begin{macrocode}
\addtocounter{page}{-1}
\begin{center}
{\LARGE\bfseries{}childdoc example\par}
\vspace{1cm}
\ifchilddoc
\ifchilddocmanual part\else chapter\fi:
`\childdocname' of `\childdocjob'\par
\else
main document: `\childdocjob'\par
\fi
version: \version\par
\end{center}
\newpage
%    \end{macrocode}

% Manually include selected file,
% otherwise process as usual:
%    \begin{macrocode}
\ifchilddocmanual
\section*{part `\childdocname'}
\input{\childdocname}
\else
%    \end{macrocode}

% Include the two chapters:
%    \begin{macrocode}
\include{cdocsch1}
\include{cdocsch2}
%    \end{macrocode}

% Include the two parts unless only chapters should be displayed:
%    \begin{macrocode}
\ifchilddoc\else
\section{part three}
\input{cdocspt3}
\section{part four}
\input{cdocspt4}
\fi
%    \end{macrocode}

% Process as usual until here:
%    \begin{macrocode}
\fi
%    \end{macrocode}

% End of document body:
%    \begin{macrocode}
\end{document}
%    \end{macrocode}
%\iffalse
%</samplemain>
%\fi
%
% %%%%%%%%%%%%%%%%%%%%%%%%%%%%%%%%%%%%%%
% \paragraph{Chapter Include Files.}
%
% The include files are called |cdocsch1.tex| and |cdocsch2.tex|.
%
%\iffalse
%<*samplechap1|samplechap2>
%\fi

% Optional override for |\version| flag:
%    \begin{macrocode}
%%\providecommand{\version}{final}
%    \end{macrocode}

% Include the main document:
%    \begin{macrocode}
\input{childdoc.def}
\childdocof{cdocsamp}
%    \end{macrocode}

%\iffalse
%</samplechap1|samplechap2>
%\fi
%
%\iffalse
%<*samplechap1>
%\fi
% Some text for chapter 1:
%    \begin{macrocode}
\section{one}
some text in chapter one
%    \end{macrocode}

%\iffalse
%</samplechap1>
%\fi
% Some text for chapter 2:
%\iffalse
%<*samplechap2>
%\fi
%    \begin{macrocode}
\section{two}
more text in chapter two
%    \end{macrocode}

%\iffalse
%</samplechap2>
%\fi
%
% %%%%%%%%%%%%%%%%%%%%%%%%%%%%%%%%%%%%%%
% \paragraph{Part Include Files.}
%
% The include files are called |cdocspt3.tex| and |cdocspt4.tex|.
%
%\iffalse
%<*samplepart3|samplepart4>
%\fi

% Optional override for |\version| flag:
%    \begin{macrocode}
%%\providecommand{\version}{final}
%    \end{macrocode}

% Include the main document:
%    \begin{macrocode}
\input{childdoc.def}
\childdocby{cdocsamp}
%    \end{macrocode}

%\iffalse
%</samplepart3|samplepart4>
%\fi
%
%\iffalse
%<*samplepart3>
%\fi
% Some text for part 3:
%    \begin{macrocode}
some text in part three
%    \end{macrocode}

%\iffalse
%</samplepart3>
%\fi
% Some text for part 4:
%\iffalse
%<*samplepart4>
%\fi
%    \begin{macrocode}
more text in part four
%    \end{macrocode}

%\iffalse
%</samplepart4>
%\fi
%
% %%%%%%%%%%%%%%%%%%%%%%%%%%%%%%%%%%%%%%
% \paragraph{Forwarding for a Complete Draft.}
%
% The following forwarding file |cdocsdrf.tex|
% compiles the main document in draft mode:
%\iffalse
%<*sampledraft>
%\fi
%    \begin{macrocode}
\def\version{draft}
\input{childdoc.def}
\childdocforward{cdocsamp}
%    \end{macrocode}

%\iffalse
%</sampledraft>
%\fi
%
% %%%%%%%%%%%%%%%%%%%%%%%%%%%%%%%%%%%%%%
% \paragraph{Forwarding for Final Version of the Chapters.}
%
% The following forwarding files |cdocsfn1.tex| and |cdocsfn2.tex|
% (with identical content)
% compile the final versions of the child documents
% |cdocsch1.tex| and |cdocsch2.tex|, respectively:
%\iffalse
%<*samplefinal>
%\fi
%    \begin{macrocode}
\def\version{final}
\input{childdoc.def}
\childdocforwardprefix[cdocsamp]{cdocsfn}{cdocsch}
%    \end{macrocode}

%\iffalse
%</samplefinal>
%\fi
%
% %%%%%%%%%%%%%%%%%%%%%%%%%%%%%%%%%%%%%%
% \paragraph{Command Line Processing.}
%
% The following three command lines generate the output files
% |cdocscld|, |cdocscl1| and |cdocscl2|
% which should be identical to
% |cdocsdrf|, |cdocsch1| and |cdocsfn2|, respectively:
% \begin{center}
% \begin{tabular}{l}
% |latex -jobname cdocscld \|\\
% |  "\def\version{draft}\input{childdoc.def}\childdocforward{cdocsamp}"|\\
% |latex -jobname cdocscl1 \|\\
% |  "\input{childdoc.def}\childdocforward[cdocsamp]{cdocsch1}"|\\
% |latex -jobname cdocscl2 \|\\
% |  "\def\version{final}\input{childdoc.def}\childdocforward{cdocsch2}"|
% \end{tabular}
% \end{center}
% Note that the trailing backslash on each first line
% merely continues the input to the second line
% (for convenient cut ant paste).
% Furthermore, the command |latex| can be replaced by any
% of its alternative versions such as |pdflatex|.
%
% %%%%%%%%%%%%%%%%%%%%%%%%%%%%%%%%%%%%%%%%%%%%%%%%%%%%%%%%%%%%%%%%%%%%%%%%%%%%%%
% %%%%%%%%%%%%%%%%%%%%%%%%%%%%%%%%%%%%%%%%%%%%%%%%%%%%%%%%%%%%%%%%%%%%%%%%%%%%%%
% \section{Implementation}
%\iffalse
%<*package>
%\fi
%
% This section describes the definitions file |childdoc.def|.

% The definitions cannot be loaded using |\usepackage| or |\RequirePackage|
% which has a mechanism to prevent loading a style file more than once.
% When loading the definitions by means of |\input|
% multiple instances have to be prevented manually:
%\iffalse
%This code needs to be before the `\ProvidesFile' directive
%which is defined at the beginning of this file.
%Therefore it is also placed there and commented out here.
%</package>
%<*discard>
%\fi
%    \begin{macrocode}
\ifdefined\childdocmain\endinput\fi
%    \end{macrocode}
%\iffalse
%</discard>
%<*package>
%\fi
%
% \macro{\ifchilddoc}
% \macro{\ifchilddocmanual}
% The conditional |\ifchilddoc| tells whether a
% child (true) or main (false) document is being compiled.
% The conditional |\ifchilddocmanual| tells whether
% the |\includeonly| mechanism is used (false) or
% the selection of child files must be performed manually (true).
% The definitions initialise to false:
%    \begin{macrocode}
\newif\ifchilddoc
\newif\ifchilddocmanual
%    \end{macrocode}

% \macro{\childdocname}
% \macro{\childdocjob}
% The macro |\childdocname| stores the name of the main document
% to be compiled. The macro |\childdocjob| stores the name of
% the document on which the \LaTeX{} compiler was originally invoked.
% The content of |\jobname| cannot be compared
% to filenames specified in the source due to different catcodes.
% The following code rescans |\jobname|, stores the result
% in |\childdocname| and saves a copy in |\childdocjob|:
%    \begin{macrocode}
\edef\childdocname{\scantokens\expandafter{\jobname\noexpand}}
\let\childdocjob\childdocname
%    \end{macrocode}

% \macro{\childdocdisable}
% The macro |\childdocdisable| prevents the main file
% from being processed more than once.
% At this stage, the main document command |\childdocmain|
% is assumed to be called once again where it should do nothing.
% Any subsequent call to it should prevent
% a secondary processing of the main document
% It overwrites the forwarding commands
% |\childdocof| and |\childdocforward|
% with empty macros to prevent further inclusions of the main document:
%    \begin{macrocode}
\newcommand{\childdocdisable}
{
  \renewcommand{\childdocmain}[1]{\renewcommand{\childdocmain}[1]{\endinput}}
  \renewcommand{\childdocof}[1]{}
  \renewcommand{\childdocby}[2][]{}
  \renewcommand{\childdocforward}[2][]{}
  \renewcommand{\childdocdisable}{}
}
%    \end{macrocode}

% \macro{\childdocmain}
% The macro |\childdocmain| is to be called at the top of the main file
% with nothing or the main filename (without extension) as argument.
% First, it breaks loops.
% If the argument is not empty and does not match |\childdocname|
% (which is set by the first inclusion of |childdoc.def|),
% |\ifchilddoc| is set to true, |\includeonly| is applied to the child file
% and |\jobname| is set to the main file
% (for proper handling of |.aux| files):
%    \begin{macrocode}
\newcommand{\childdocmain}[1]
{
  \childdocdisable\childdocmain{}
  \if?#1?\else
    \begingroup
      \def\childdoctmp{#1}
      \ifx\childdoctmp\childdocname
        \def\childdoctmp{}
      \else
        \def\childdoctmp
        {
          \childdoctrue
          \includeonly{\childdocname}
          \def\childdocjob{#1}
          \def\jobname{#1}
        }
      \fi
      \expandafter
    \endgroup
    \childdoctmp
  \fi
}
%    \end{macrocode}

% \macro{\childdocof}
% The command |\childdocof| redirects
% compilation to the main file |#1|.
%    \begin{macrocode}
\newcommand{\childdocof}[1]
{
  \childdocdisable
  \childdoctrue
  \includeonly{\childdocname}
  \def\jobname{#1}
  \def\childdocjob{#1}
  \input{#1}
}
%    \end{macrocode}

% \macro{\childdocby}
% The command |\childdocby| ....
%    \begin{macrocode}
\newcommand{\childdocby}[2][]
{
  \childdocdisable
  \childdoctrue
  \childdocmanualtrue
  \if?#1?\else
    \def\jobname{#2}
  \fi
  \def\childdocjob{#2}
  \input{#2}
  \endinput
}
%    \end{macrocode}

% \macro{\childdocforward}
% The command |\childdocforward| redirects
% compilation to the main file or
% (if the optional argument is given) a child file.
% Parameters are set as if the main file
% or a child file starting with |\childdocof| was compiled.
% Then compilation is handed over to the main file:
%    \begin{macrocode}
\newcommand{\childdocforward}[2][]
{
  \begingroup
    \if?#1?
      \def\childdoctmp
      {
        \def\childdocname{#2}
        \def\childdocjob{#2}
        \def\jobname{#2}
        \input{#2}
        \endinput
      }
    \else
      \def\childdoctmp
      {
        \childdocdisable
        \def\childdocname{#2}
        \childdoctrue
        \includeonly{#2}
        \def\childdocjob{#1}
        \def\jobname{#1}
        \input{#1}
        \endinput
      }
    \fi
    \expandafter
  \endgroup
  \childdoctmp
}
%    \end{macrocode}

% \macro{\childdocforwardprefix}
% The command |\childdocforwardprefix| redirects
% compilation to the main or a child file by means of a pattern.
% The prefix |#1| in the current filename is replaced by |#2|
% and the suffix of the current filename is kept
% (it is assumed that the filename does not contain the substring `|~~~|'
% which is used as a delimiter).
% Compilation is handed over to the new file by |\childdocforward|:
%    \begin{macrocode}
\newcommand{\childdocforwardprefix}[3][]
{
  \begingroup
    \def\childdocextract #2##1~~~{\def\childdoctmp{\childdocforward[#1]{#3##1}}}
    \expandafter\childdocextract\childdocname~~~
    \expandafter
  \endgroup
  \childdoctmp
}
%    \end{macrocode}

% \macro{\childdoc}
% The deprecated macro |\childdoc| is a legacy version of |\childdocmain|:
%    \begin{macrocode}
\newcommand{\childdoc}{\childdocmain}
%    \end{macrocode}

% \macro{\childdocredirect}
% The deprecated macro |\childdocredirect| is a legacy version
% of |\childdocforward| and |\childdocforwardprefix|:
%    \begin{macrocode}
\newcommand{\childdocredirect}[2][]
{
  \begingroup
    \if?#1?
      \def\childdoctmp{\childdocforward{#2}}
    \else
      \def\childdoctmp{\childdocforwardprefix{#1}{#2}}
    \fi
    \expandafter
  \endgroup
  \childdoctmp
}
%    \end{macrocode}

%\iffalse
%</package>
%\fi
%
\endinput

\childdocforward{cdocsamp}
%    \end{macrocode}

%\iffalse
%</sampledraft>
%\fi
%
% %%%%%%%%%%%%%%%%%%%%%%%%%%%%%%%%%%%%%%
% \paragraph{Forwarding for Final Version of the Chapters.}
%
% The following forwarding files |cdocsfn1.tex| and |cdocsfn2.tex|
% (with identical content)
% compile the final versions of the child documents
% |cdocsch1.tex| and |cdocsch2.tex|, respectively:
%\iffalse
%<*samplefinal>
%\fi
%    \begin{macrocode}
\def\version{final}
% \iffalse
%
% childdoc.dtx Copyright (C) 2017-2018 Niklas Beisert
%
% This work may be distributed and/or modified under the
% conditions of the LaTeX Project Public License, either version 1.3
% of this license or (at your option) any later version.
% The latest version of this license is in
%   http://www.latex-project.org/lppl.txt
% and version 1.3 or later is part of all distributions of LaTeX
% version 2005/12/01 or later.
%
% This work has the LPPL maintenance status `maintained'.
%
% The Current Maintainer of this work is Niklas Beisert.
%
% This work consists of the files childdoc.dtx and childdoc.ins
% and the derived files childdoc.def and cdocsamp.tex with
% cdocsch1.tex, cdocsch2.tex, cdocsdrf.tex, cdocsfn1.tex, cdocsfn2.tex.
%
%<package>\ifdefined\childdocmain\endinput\fi
%<package>\ProvidesFile{childdoc.def}[2018/12/30 v2.0 child document driver]
%<samplemain>\ProvidesFile{cdocsamp.tex}[2018/12/30 v2.0 sample for childdoc]
%<*driver>
%\ProvidesFile{childdoc.drv}[2018/12/30 v2.0 childdoc reference manual file]
\PassOptionsToClass{10pt,a4paper}{article}
\documentclass{ltxdoc}

\usepackage[margin=35mm]{geometry}
\usepackage{hyperref}
\usepackage{hyperxmp}
\usepackage[usenames]{color}

\hypersetup{colorlinks=true}
\hypersetup{pdfstartview=FitH}
\hypersetup{pdfpagemode=UseNone}
\hypersetup{pdfsource={}}
\hypersetup{pdflang={en-UK}}
\hypersetup{pdfcopyright={Copyright 2017-2018 Niklas Beisert.
  This work may be distributed and/or modified under the
  conditions of the LaTeX Project Public License, either version 1.3
  of this license or (at your option) any later version.}}
\hypersetup{pdflicenseurl={http://www.latex-project.org/lppl.txt}}
\hypersetup{pdfcontactaddress={ETH Zurich, ITP, HIT K,
  Wolfgang-Pauli-Strasse 27}}
\hypersetup{pdfcontactpostcode={8093}}
\hypersetup{pdfcontactcity={Zurich}}
\hypersetup{pdfcontactcountry={Switzerland}}
\hypersetup{pdfcontactemail={nbeisert@itp.phys.ethz.ch}}
\hypersetup{pdfcontacturl={http://people.phys.ethz.ch/\xmptilde nbeisert/}}

\newcommand{\secref}[1]{\hyperref[#1]{section \ref*{#1}}}

\parskip1ex
\parindent0pt
\let\olditemize\itemize
\def\itemize{\olditemize\parskip0pt}

\begin{document}

\title{The \textsf{childdoc} Package}
\hypersetup{pdftitle={The childdoc Package}}
\author{Niklas Beisert\\[2ex]
  Institut f\"ur Theoretische Physik\\
  Eidgen\"ossische Technische Hochschule Z\"urich\\
  Wolfgang-Pauli-Strasse 27, 8093 Z\"urich, Switzerland\\[1ex]
  \href{mailto:nbeisert@itp.phys.ethz.ch}
  {\texttt{nbeisert@itp.phys.ethz.ch}}}
\hypersetup{pdfauthor={Niklas Beisert}}
\hypersetup{pdfsubject={Manual for the LaTeX2e Package childdoc}}
\date{30 December 2018, \textsf{v2.0}}
\maketitle

\begin{abstract}\noindent
\textsf{childdoc} is a \LaTeXe{} package
that enables the direct compilation
of document sections included by |\include|
to individual files.
\end{abstract}

\begingroup
\parskip0ex
\tableofcontents
\endgroup

%%%%%%%%%%%%%%%%%%%%%%%%%%%%%%%%%%%%%%%%%%%%%%%%%%%%%%%%%%%%%%%%%%%%%%%%%%%%%%%%
%%%%%%%%%%%%%%%%%%%%%%%%%%%%%%%%%%%%%%%%%%%%%%%%%%%%%%%%%%%%%%%%%%%%%%%%%%%%%%%%
\section{Introduction}

\LaTeX{} provides a mechanism to structure a large document (such as a book)
into a main file and several child files (containing the chapters)
using the |\include| command.
This mechanism is beneficial for documents
which span hundreds of pages in order to
make the source file(s) more manageable.
Moreover, compilation can be restricted to
selected child files by means of the |\includeonly| command.
The latter feature can be used to reduce the compilation time while editing
(this was significantly more useful in the earlier days of \LaTeX{})
or to generate a smaller document which is easier to navigate.
Another application of |\includeonly| is to generate
documents consisting of selected parts of the complete document.

However, there are a few drawbacks of the plain |\include| mechanism:
\begin{itemize}
\item
The child files cannot be compiled on their own,
they can only be compiled via the main file.
A naive editing environment
(such as a text editor with an option
to have the current file processed by \LaTeX)
may require one to switch to the main file before compiling;
attempting to compile the child file produces errors.
\item
The main file must be modified (each time)
to adjust the |\includeonly| command
to the present needs. This easily leaves the main file in a messy state.
\item
The generated document will always carry the filename
of the main document. This is inconvenient if
several child files are to be compiled and
to be kept for distribution.
\end{itemize}

The present package provides a simple interface
to make child files individually compilable by \LaTeX{}.
Compiling a child file then has the same effect as compiling
the main file with an |\includeonly| command
to select the appropriate child.
Moreover the generated document will carry the name of the child
rather than the main file.
This resolves all three above issues.

This feature is meant to make the editing of books,
thesis documents and lecture notes somewhat more convenient.
However, the package can also be used efficiently for
composing a series of documents (such as exercise sheets)
which are typically distributed individually.
It then assists the author in generating the individual documents
(potentially in different versions)
as well as a document containing the collected series.
Another application is in developing style files
or other kinds of included material
where compilation of the style file could redirect
to a sample or test file.

%%%%%%%%%%%%%%%%%%%%%%%%%%%%%%%%%%%%%%%%%%%%%%%%%%%%%%%%%%%%%%%%%%%%%%%%%%%%%%%%
%%%%%%%%%%%%%%%%%%%%%%%%%%%%%%%%%%%%%%%%%%%%%%%%%%%%%%%%%%%%%%%%%%%%%%%%%%%%%%%%
\section{Usage}

First of all, the package \textsf{childdoc} is \emph{not} a standard
\LaTeXe{} |.sty| style file! Therefore it needs to be invoked in
a non-standard way.

%%%%%%%%%%%%%%%%%%%%%%%%%%%%%%%%%%%%%%%%%%%%%%%%%%%%%%%%%%%%%%%%%%%%%%%%%%%%%%%%
\subsection{Included Files}
\label{sec:include}

%%%%%%%%%%%%%%%%%%%%%%%%%%%%%%%%%%%%%%%%
\DescribeMacro{\childdocmain}
To use the package, add the commands
\begin{center}
\begin{tabular}{l}
|\input{childdoc.def}|\\
|\childdocmain{}|\\
\end{tabular}
\end{center}
at the very top of the main \LaTeX{} file,
in particular \emph{before} the |\documentclass| statement!
The argument of |\childdocmain| should be left empty
(but it must be present).

%%%%%%%%%%%%%%%%%%%%%%%%%%%%%%%%%%%%%%%%
\DescribeMacro{\childdocof}
Furthermore, add the commands
\begin{center}
\begin{tabular}{l}
|\input{childdoc.def}|\\
|\childdocof{|\textit{main}|}|\\
\end{tabular}
\end{center}
at the top of every child file \textit{child}
which is included by |\include{|\textit{child}|}|
from within the main file
(or at least for those files to be compiled individually).
The argument \textit{main} must be the filename of the main file.

There are a couple of
considerations in setting up the main and child documents:

%%%%%%%%%%%%%%%%%%%%%%%%%%%%%%%%%%%%%%%%
\paragraph{Restrictions.}

Please note the following restrictions:
\begin{itemize}
\item
|\childdocmain| must be called with one argument \textit{main}
to ensure compatibility with earlier version of the package.
It must either be empty (|\childdocmain{}|)
or precisely match the filename of the main file in which it is specified.
See \secref{sec:detection} for further information.
\item
The filename \textit{main} must be specified without the |.tex| extension.
\item
The filename \textit{main} is case sensitive
(even in case-insensitive file systems)
due to internal string comparison.
\item
The argument \textit{main} should be fully expanded, it cannot be a macro.
\item
Subdirectories and special characters should be avoided in filenames.
\item
The command |\childdocmain{|\textit{main}|}| must be followed by a whitespace.
It should not be followed immediately by another command
or by a comment mark `|%|'.
This is because the \TeX{} parser reads the token immediately following
the argument of |\childdocmain| and puts it
at the beginning of every child section;
however, a white\-space is ignored.
\end{itemize}

%%%%%%%%%%%%%%%%%%%%%%%%%%%%%%%%%%%%%%%%
\paragraph{Content of Main File.}

It is advisable to place all content in the child files included by |\include|.
Any output contained in the main file will appear in all child documents
unless suppressed manually;
it cannot be suppressed automatically by the |\includeonly| directive
and thus should normally be avoided.
A method to include some content in the main file
by means of conditional processing is described in \secref{sec:conditional}.

%%%%%%%%%%%%%%%%%%%%%%%%%%%%%%%%%%%%%%%%
\paragraph{Page Numbering.}

When only a part of the document is compiled,
the appropriate numbering of pages
(as well as other status parameters)
is determined from the |.aux| files.
The latter contain information from previous passes.
However this information needs to propagate through
all intermediate child documents.
Therefore the page numbering in child documents may well
be inconsistent until the complete document is compiled at least once.

A useful (if unconventional) way to always ensure a consistent
page numbering is to restart the numbering in each child document
and denote the pages by `\textit{child}|.|\textit{page}'
where \textit{child} represents the chapter/section number of the child file.
This can be achieved by the command
|\numberwithin{page}{|\textit{child}|}|
of the \textsf{amsmath} package
where \textit{child} can be |chapter| or |section|
depending on the chosen structuring.
Alternatively, one can modify the macro |\thepage| appropriately
and reset the counter |page| at the start of each child file.

%%%%%%%%%%%%%%%%%%%%%%%%%%%%%%%%%%%%%%%%%%%%%%%%%%%%%%%%%%%%%%%%%%%%%%%%%%%%%%%%
\subsection{Conditional Processing}
\label{sec:conditional}

The package provides a mechanism to compile different versions
of a document. To customise the versions further some conditional processing
can come in handy to distinguish which version is being compiled.
The package provides two macros to describe the compilation context:

%%%%%%%%%%%%%%%%%%%%%%%%%%%%%%%%%%%%%%%%
\DescribeMacro{\ifchilddoc}
The conditional |\ifchilddoc| distinguishes between the compilation of
child documents and the main document:
%
\begin{center}
|\ifchilddoc |\textit{child-code}| |[|\||else |\textit{main-code}]| \||fi|
\end{center}

%%%%%%%%%%%%%%%%%%%%%%%%%%%%%%%%%%%%%%%%
\DescribeMacro{\childdocname}
\DescribeMacro{\childdocjob}
The macro |\childdocname| contains the filename (without extension)
of the main or child file being processed.
Note that |\childdocjob| will always contain the name of the main file.

%%%%%%%%%%%%%%%%%%%%%%%%%%%%%%%%%%%%%%%%
\paragraph{Title Page.}

Conditional processing can be used to include a title or banner page
in the main document when proper precautions are taken.
Importantly, the code in the main file should ensure that the page counter
(as well as other status parameters which are stored in the |.aux| files)
takes the same value after the conditional processing.
Otherwise the page numbers may take divergent values
depending on which part is compiled.

For example, a title page could be declared by:
%
\begin{center}
\begin{tabular}{l}
|\ifchilddoc\||else|\\
|\addtocounter{page}{-1}|\\
\textit{code for title page}\\
|\newpage|\\
|\||fi|
\end{tabular}
\end{center}
%
A banner page for the child documents can be generated by:
%
\begin{center}
\begin{tabular}{l}
|\ifchilddoc|\\
|\addtocounter{page}{-1}|\\
\textit{code for banner page}\\
|\newpage|\\
|\||fi|
\end{tabular}
\end{center}
%
Here one could write a message such as:
\begin{center}
|This is the part \childdocname{} of \childdocjob{}.|
\end{center}

%%%%%%%%%%%%%%%%%%%%%%%%%%%%%%%%%%%%%%%%%%%%%%%%%%%%%%%%%%%%%%%%%%%%%%%%%%%%%%%%
\subsection{Flags}
\label{sec:flags}

The package makes it easy to generate different versions
of the main or child documents.
To this end compilation flags can be defined
and assigned different default values.
They will be particularly useful in conjunction
with the forwarding mechanism described in \secref{sec:forward}.

For example, it may be useful to have a flag |\version|
which can be set to |draft| or |final|.
The document source will contain some conditional code
depending on the value of |\version|.
Suppose further, the flag should default to |final| for the main file
and to |draft| for child files
which is a natural assignment for editing the document.
This is achieved by placing the following code
in the preamble of the main document
(below the |\childdocmain| directive):
%
\begin{center}
\begin{tabular}{l}
|\ifchilddoc|\\
|\providecommand{\version}{draft}|\\
|\||else|\\
|\providecommand{\version}{final}|\\
|\||fi|
\end{tabular}
\end{center}
%
The definition by |\providecommand| makes sure
that previous definitions are not overwritten.
Further statements |\providecommand{\version}{...}|
can thus be added before the above code to override it.

For the main file, one might add a line
(between |\childdocmain| and the above block)
%
\begin{center}
|%\ifchilddoc\||else\providecommand{\version}{draft}\||fi|
\end{center}
%
which can be uncommented to produce a draft version.
Likewise one can add a line to the very top of a child file
(above the |\childdocof{|\textit{main}|}| directive)
%
\begin{center}
|%\providecommand{\version}{final}|
\end{center}
%
which can be uncommented to produce the final version of this child document.

%%%%%%%%%%%%%%%%%%%%%%%%%%%%%%%%%%%%%%%%%%%%%%%%%%%%%%%%%%%%%%%%%%%%%%%%%%%%%%%%
\subsection{Forwarding}
\label{sec:forward}

Different versions of the main or child documents
using compilation flags as described in \secref{sec:flags}
can be (permanently) stored in different files
for convenient compilation, viewing and distribution.
To this end, the package defines a command
to pass on compilation to a different file:

%%%%%%%%%%%%%%%%%%%%%%%%%%%%%%%%%%%%%%%%
\DescribeMacro{\childdocforward}
The command |\childdocforward| redirects processing to
another source file:
%
\begin{center}
\begin{tabular}{l}
|\input{childdoc.def}|\\
|\childdocforward[|\textit{main}|]{|\textit{dest}|}|\\
\end{tabular}
\end{center}
%
The argument \textit{dest} is the destination file
(without extension).
It should be the main file or one of the child files.
Note that further \textsf{childdoc} directives
such as |\childdocof| and |\childdocforward|
in the indicated file will be processed in this form.
The optional argument \textit{main}
passes on directly to the main file \textit{main}
while pretending to compile the child \textit{dest}.
This form behaves as if \textit{dest}
issues |\childdocof{|\textit{main}|}| right away,
and no further \textsf{childdoc} directives will be processed.

%%%%%%%%%%%%%%%%%%%%%%%%%%%%%%%%%%%%%%%%
\DescribeMacro{\...prefix}
In the alternative form |\childdocforwardprefix|,
%
\begin{center}
\begin{tabular}{l}
|\input{childdoc.def}|\\
|\childdocforwardprefix[|\textit{main}|]{|\textit{prefix}|}{|\textit{dest}|}|
\end{tabular}
\end{center}
%
the destination file is determined by a pattern
depending on the current file:
To make this work, the current file must be called
`{\textit{prefix}\hspace{0.2em}\textit{suffix}}'
with \textit{prefix} matching precisely the argument.
Processing is then passed on to the file
`{\textit{dest}\hspace{0.2em}\textit{suffix}}'.
Surely, the same effect is achieved by
directly specifying the
argument `{\textit{dest}\hspace{0.2em}\textit{suffix}}'
in the first form.
However, that requires to set up a different file
for each child. With the alternative form of the command
all these files can have exactly the same content
which simplifies setting them up and maintaining them.

For example, the following file |draft.tex|
with a compilation flag |\version| as described in \secref{sec:flags}
compiles the main document as a draft:
%
\begin{center}
\begin{tabular}{l}
|\def\version{draft}|\\
|\input{childdoc.def}|\\
|\childdocforward{|\textit{main}|}|
\end{tabular}
\end{center}
%
Likewise, the following files |final|\textit{nn}|.tex|
compile the final version of the child document
|child|\textit{nn}|.tex|:
%
\begin{center}
\begin{tabular}{l}
|\def\version{final}|\\
|\input{childdoc.def}|\\
|\childdocforwardprefix{final}{child}|
\end{tabular}
\end{center}
%

Note that when several versions of a main file and/or of each child file
are to be generated, it may be convenient to set up a |Makefile| or
shell script to automatise the process.

%%%%%%%%%%%%%%%%%%%%%%%%%%%%%%%%%%%%%%%%%%%%%%%%%%%%%%%%%%%%%%%%%%%%%%%%%%%%%%%%
\subsection{Command Line Processing}
\label{sec:commandline}

The effect of redirection files can also be achieved by invoking
the \LaTeX{} compiler with a more elaborate command line.
Most conveniently this should be done as part
of a shell script or a |Makefile|.

When using \textsf{childdoc} in the main file, the following
command lines effectively perform a redirection
(note that depending on the shell being used,
backslashes may have to be doubled: `|\|' $\to$ `|\\|'):
%
\begin{center}
|... -jobname "|\textit{target}|" |\\|"|[\textit{flags}]%
|\input{childdoc.def}\childdocforward[|\textit{main}|]{|\textit{dest}|}"|
\end{center}
%
Here \textit{target} is the name of the output file,
\textit{main} is the name of the main file
and \textit{dest} is the name of the main or child file to be processed
(all filenames without extensions).
The optional argument \textit{main} can be omitted
if \textit{main} matches \textit{dest}.
Optionally, compilation \textit{flags} can be defined via |\def| commands.
This command line makes the \TeX{} engine believe
it is compiling the file \textit{target}
whose content is specified as the latter parameter.
The provided code then forwards the processing to
\textit{main} or \textit{dest} as described in \secref{sec:forward}.

%%%%%%%%%%%%%%%%%%%%%%%%%%%%%%%%%%%%%%%%%%%%%%%%%%%%%%%%%%%%%%%%%%%%%%%%%%%%%%%%
\subsection{Include by Input}
\label{sec:input}

Including child documents by |\include| has some restrictions by design.
Most notably, the content of a child document always occupies
its own set of pages; pages cannot be shared between child documents.
Usually, this behaviour makes perfect sense
because each child document contain an essential part of the document.
However, in some situations it may be desirable to compose
a document from a collection of parts
without having mandatory page breaks between then.
For this case, the package
provides a mechanism to include parts
by |\input| which can also be processed individually.
However, by construction this mechanism
requires manual handling of the content to be output.

%%%%%%%%%%%%%%%%%%%%%%%%%%%%%%%%%%%%%%%%
\DescribeMacro{\ifchilddocmanual}
The main file should be prepared as usual, see \secref{sec:include}.
However, the document body must make a distinction
between processing of an individual part and of the main document, e.g.:
%
\begin{center}
\begin{tabular}{l}
|\ifchilddocmanual|\\
|\input{\childdocname}|\\
|\||else|\\
\textit{document body with }|\input{|\textit{part}|}|\\
|\||fi|
\end{tabular}
\end{center}
%
The conditional |\ifchilddocmanual| is true whenever
a part to be included by |\input| is being compiled,
and the name of the part is stored in |\childdocname|.

%%%%%%%%%%%%%%%%%%%%%%%%%%%%%%%%%%%%%%%%
\DescribeMacro{\childdocby}
Each part to be included by |\input| should start with:
%
\begin{center}
\begin{tabular}{l}
|\input{childdoc.def}|\\
|\childdocby{|\textit{main}|}|\\
\end{tabular}
\end{center}
%
The directive |\childdocby| is similar to |\childdocof|
described in \secref{sec:include},
but the subsequent selection of content must be done manually.
To that end, both |\ifchilddoc| and |\ifchilddocmanual|
will be true upon processing of a part,
and the name of the part is stored in |\childdocname|.
Note that |\jobname| will be set to the filename of the current part
so that each part receives an individual |.aux| file
that does not interfere with the |.aux| file(s) of the main document.
This behaviour can be altered by the alternative form
|\childdocby[*]{|\textit{main}|}| (with a non-empty optional argument)
which uses the |.aux| file of the main document
by setting |\jobname| to \textit{main}.

%%%%%%%%%%%%%%%%%%%%%%%%%%%%%%%%%%%%%%%%%%%%%%%%%%%%%%%%%%%%%%%%%%%%%%%%%%%%%%%%
\subsection{Driver Development}
\label{sec:driver}

The \textsf{childdoc} mechanism can also be use for the development
of definition files such as \LaTeX{} styles or classes.
This case differs from the above setup with multiple parts
included by |\include| in that no |\includeonly| should be invoked.
This can be achieved by starting the include file
(before |\ProvidesPackage|) with:
%
\begin{center}
\begin{tabular}{l}
|\input{childdoc.def}|\\
|\childdocforward{|\textit{main}|}|\\
\end{tabular}
\end{center}
%
or alternatively with:
%
\begin{center}
\begin{tabular}{l}
|\input{childdoc.def}|\\
|\childdocby{|\textit{main}|}|\\
\end{tabular}
\end{center}
%
Both forms have slightly different effects as described above.
The main file is prepared as usual, see \secref{sec:include}.

%%%%%%%%%%%%%%%%%%%%%%%%%%%%%%%%%%%%%%%%%%%%%%%%%%%%%%%%%%%%%%%%%%%%%%%%%%%%%%%%
\subsection{Legacy Detection}
\label{sec:detection}

The directive |\childdocmain| in the main file can detect
whether the complete document or merely a child is to be compiled
even without using the directive |\childdocof|.
This method is deprecated because it is less robust
and there is no compelling reason to use it;
it is merely provided for backward compatibility
and it may be removed in future versions.

If the detection mechanism is to be used,
it is mandatory to correctly specify
the filename of the main file as the argument of |\childdocmain|:
%
\begin{center}
\begin{tabular}{l}
|\input{childdoc.def}|\\
|\childdocmain{|\textit{main}|}|\\
\end{tabular}
\end{center}
%
If |\jobname| does not match the argument \textit{main} of |\childdocmain|,
it is assumed that |\jobname| points to the child file to be compiled.
When using |\childdocmain| with the main file specified as argument,
it suffices to start a child file
with just |\input{|\textit{main}|}|
without loading of the package and using |\childdocof|.
If instead all processing is done
with the appropriate \textsf{childdoc} directives,
the argument of \textit{main} of |\childdocmain| can be empty.

An alternative version of the command line processing described
in \secref{sec:commandline} using the detection mechanism reads:
%
\begin{center}
|... -jobname "|\textit{target}|" "|[\textit{flags}]%
[|\def\jobname{|\textit{dest}|}|]|\input{|\textit{main}|}"|
\end{center}

%%%%%%%%%%%%%%%%%%%%%%%%%%%%%%%%%%%%%%%%%%%%%%%%%%%%%%%%%%%%%%%%%%%%%%%%%%%%%%%%
\subsection{Manual Code}
\label{sec:manual}

In case one cannot be certain whether the definitions file |childdoc.def|
is installed on the target \TeX{} distribution
and one prefers not to ship it,
it is conceivable to paste a few relevant commands into the sources.

To that end, drop all statements |\input{childdoc.def}|
and perform the replacements as outlined below.
Instead of |\childdocmain{|\textit{main}|}| add the following code
to the top of the main file:
%
\begin{center}
\begin{tabular}{l}
|\||ifdefined\childdocname\endinput\||fi\newif\ifchilddoc|\\
|\edef\childdocname{\scantokens\expandafter{\jobname\noexpand}}|\\
|\def\childdocmain{|\textit{main}|}\||ifx\childdocmain\childdocname\||else|\\
|\childdoctrue\includeonly{\childdocname}\let\jobname\childdocmain\||fi|\\
\end{tabular}
\end{center}
%
Instead of |\childdocof{|\textit{main}|}| just include the main file
at the top of each child file:
%
\begin{center}
|\input{|\textit{main}|}|
\end{center}
%
A simple redirection |\childdocforward{|\textit{dest}|}| is achieved by:
%
\begin{center}
|\def\jobname{|\textit{dest}|}\input{\jobname}|
\end{center}
%
The redirection with prefix
|\childdocforwardprefix[|\textit{prefix}|]{|\textit{dest}|}|
is accomplished by:
%
\begin{center}
\begin{tabular}{l}
|{\edef\jobname{\scantokens\expandafter{\jobname\noexpand}}|\\
|\def\redirectjob |\textit{prefix}|#1~~~{\gdef\jobname{|\textit{dest}|#1}}|\\
|\expandafter\redirectjob\jobname~~~}\input{\jobname}|
\end{tabular}
\end{center}

In an alternative approach,
child documents can be compiled by a specific command line
without additional code or specific definitions:
%
\begin{center}
|... -jobname "|\textit{target}|" "|[\textit{flags}]%
|\includeonly{|\textit{dest}|}\input{|\textit{main}|}"|
\end{center}
%

%%%%%%%%%%%%%%%%%%%%%%%%%%%%%%%%%%%%%%%%%%%%%%%%%%%%%%%%%%%%%%%%%%%%%%%%%%%%%%%%
%%%%%%%%%%%%%%%%%%%%%%%%%%%%%%%%%%%%%%%%%%%%%%%%%%%%%%%%%%%%%%%%%%%%%%%%%%%%%%%%
\section{Information}

%%%%%%%%%%%%%%%%%%%%%%%%%%%%%%%%%%%%%%%%%%%%%%%%%%%%%%%%%%%%%%%%%%%%%%%%%%%%%%%%
\subsection{Copyright}

Copyright \copyright{} 2017--2018 Niklas Beisert

This work may be distributed and/or modified under the
conditions of the \LaTeX{} Project Public License, either version 1.3
of this license or (at your option) any later version.
The latest version of this license is in
  \url{http://www.latex-project.org/lppl.txt}
and version 1.3 or later is part of all distributions of \LaTeX{}
version 2005/12/01 or later.

This work has the LPPL maintenance status `maintained'.

The Current Maintainer of this work is Niklas Beisert.

This work consists of the files |README.txt|, |childdoc.ins| and |childdoc.dtx|
as well as the derived files |childdoc.def|, |cdocsamp.tex|
with |cdocsch1.tex|, |cdocsch2.tex|, |cdocspt3.tex|, |cdocspt4.tex|,
|cdocsdrf.tex|, |cdocsfn1.tex|, |cdocsfn2.tex|
as well as |childdoc.pdf|.

%%%%%%%%%%%%%%%%%%%%%%%%%%%%%%%%%%%%%%%%%%%%%%%%%%%%%%%%%%%%%%%%%%%%%%%%%%%%%%%%
\subsection{Files and Installation}

The package consists of the files:
%
\begin{center}
\begin{tabular}{ll}
    |README.txt|   & readme file \\
    |childdoc.ins| & installation file \\
    |childdoc.dtx| & source file \\
    |childdoc.def| & definition file \\
    |cdocsamp.tex| & sample main file \\
    |cdocsch1.tex| & sample include file \\
    |cdocsch2.tex| & sample include file \\
    |cdocspt3.tex| & sample part file \\
    |cdocspt4.tex| & sample part file \\
    |cdocsdrf.tex| & sample redirection file \\
    |cdocsfn1.tex| & sample redirection file \\
    |cdocsfn2.tex| & sample redirection file \\
    |childdoc.pdf| & manual
\end{tabular}
\end{center}
%
The distribution consists of the files
|README.txt|, |childdoc.ins| and |childdoc.dtx|.
%
\begin{itemize}
\item
Run (pdf)\LaTeX{} on |childdoc.dtx|
to compile the manual |childdoc.pdf| (this file).
\item
Run \LaTeX{} on |childdoc.ins| to create the definitions file |childdoc.def|
and the sample |cdocsamp.tex| with include files
|cdocsch1.tex|, |cdocsch2.tex|, |cdocspt3.tex|, |cdocspt4.tex|,
|cdocsdrf.tex|, |cdocsfn1.tex|, |cdocsfn2.tex|.
Then copy the file |childdoc.def| to an appropriate directory of your \LaTeX{}
distribution, e.g.\ \textit{texmf-root}|/tex/latex/childdoc|.
\end{itemize}

%%%%%%%%%%%%%%%%%%%%%%%%%%%%%%%%%%%%%%%%%%%%%%%%%%%%%%%%%%%%%%%%%%%%%%%%%%%%%%%%
\subsection{Related CTAN Packages}

There are several other packages which offer a similar functionality:
%
\begin{itemize}
\item
The packages
\href{http://ctan.org/pkg/docmute}{\textsf{docmute}},
\href{http://ctan.org/pkg/includex}{\textsf{includex}} and
\href{http://ctan.org/pkg/standalone}{\textsf{standalone}}
provide commands to include only the document body of
a child file thus allowing both files to be compiled individually.
\item
The packages \href{http://ctan.org/pkg/subdocs}{\textsf{subdocs}}
and \href{http://ctan.org/pkg/subfiles}{\textsf{subfiles}}
provide structures in which the main and child documents can be
encapsulated and allowing them to be compiled individually.
The inclusion mechanism is different from the conventional |\include|.
\item
The package \href{http://ctan.org/pkg/combine}{\textsf{combine}}
is an elaborate solution to combine several documents into one.
\end{itemize}
%
See also the CTAN topic \href{http://ctan.org/topic/subdocs}{\textsf{subdocs}}
for further related packages.
The present package differs from the above solutions in that
a document structure constructed with the conventional |\include| mechanism
just needs two extra commands at the top of every file
such that all constituent files can be compiled individually.

%%%%%%%%%%%%%%%%%%%%%%%%%%%%%%%%%%%%%%%%%%%%%%%%%%%%%%%%%%%%%%%%%%%%%%%%%%%%%%%%
%\subsection{Feature Suggestions}
%
%The following is a list of features which may be useful for future
%versions of this package:
%%
%\begin{itemize}
%\item
%\ldots
%\end{itemize}

%%%%%%%%%%%%%%%%%%%%%%%%%%%%%%%%%%%%%%%%%%%%%%%%%%%%%%%%%%%%%%%%%%%%%%%%%%%%%%%%
\subsection{Revision History}

%%%%%%%%%%%%%%%%%%%%%%%%%%%%%%%%%%%%%%%%
\paragraph{v2.0:} 2018/12/30

\begin{itemize}
\item
immediate forward processing
\item
added |\childdocby| mechanism
\item
manual restructured
\end{itemize}

%%%%%%%%%%%%%%%%%%%%%%%%%%%%%%%%%%%%%%%%
\paragraph{v1.6:} 2018/01/17

\begin{itemize}
\item
application for development of include files
\item
corrections to manual
\end{itemize}

%%%%%%%%%%%%%%%%%%%%%%%%%%%%%%%%%%%%%%%%
\paragraph{v1.5:} 2017/05/21

\begin{itemize}
\item
more complete structuring introduced
\item
|\childdocof| introduced
\item
|\childdoc| renamed to |\childdocmain|
\item
|\childredirect| renamed to |\childdocforward| and |\childdocforwardprefix|
and functionality expanded
\end{itemize}

%%%%%%%%%%%%%%%%%%%%%%%%%%%%%%%%%%%%%%%%
\paragraph{v1.0:} 2017/04/27

\begin{itemize}
\item
manual and install package
\item
first version published on CTAN
\end{itemize}

%%%%%%%%%%%%%%%%%%%%%%%%%%%%%%%%%%%%%%%%
\paragraph{v0.6:} 2017/04/26

\begin{itemize}
\item
redirection mechanism added
\end{itemize}

%%%%%%%%%%%%%%%%%%%%%%%%%%%%%%%%%%%%%%%%
\paragraph{v0.5:} 2017/04/26

\begin{itemize}
\item
functionality in definition file
\end{itemize}


%%%%%%%%%%%%%%%%%%%%%%%%%%%%%%%%%%%%%%%%%%%%%%%%%%%%%%%%%%%%%%%%%%%%%%%%%%%%%%%%
%%%%%%%%%%%%%%%%%%%%%%%%%%%%%%%%%%%%%%%%%%%%%%%%%%%%%%%%%%%%%%%%%%%%%%%%%%%%%%%%
%%%%%%%%%%%%%%%%%%%%%%%%%%%%%%%%%%%%%%%%%%%%%%%%%%%%%%%%%%%%%%%%%%%%%%%%%%%%%%%%
\appendix

\settowidth\MacroIndent{\rmfamily\scriptsize 000\ }

 \DocInput{childdoc.dtx}

\end{document}
%</driver>
% \fi
%
% %%%%%%%%%%%%%%%%%%%%%%%%%%%%%%%%%%%%%%%%%%%%%%%%%%%%%%%%%%%%%%%%%%%%%%%%%%%%%%
% %%%%%%%%%%%%%%%%%%%%%%%%%%%%%%%%%%%%%%%%%%%%%%%%%%%%%%%%%%%%%%%%%%%%%%%%%%%%%%
% \section{Sample}
%\iffalse
%<*samplemain>
%\fi
%
% The following presents a sample document
% with two chapters, two parts, a title page,
% a compile flag as well as three forwarding files to set the flag.
% It consists of eight |.tex| files:
% \begin{center}
% \begin{tabular}{ll}
% |cdocsamp.tex|&main file\\
% |cdocsch1.tex|&include file for chapter 1\\
% |cdocsch2.tex|&include file for chapter 2\\
% |cdocspt3.tex|&include file for part 3\\
% |cdocspt4.tex|&include file for part 4\\
% |cdocsdrf.tex|&forwarding file for main file in draft mode\\
% |cdocsfi1.tex|&forwarding file for final version of chapter 1\\
% |cdocsfi2.tex|&forwarding file for final version of chapter 2\\
% \end{tabular}
% \end{center}
% Each of the eight files can be compiled directly by the \LaTeX{} compiler.
%
% %%%%%%%%%%%%%%%%%%%%%%%%%%%%%%%%%%%%%%
% \paragraph{Main File.}
%
% The main file is called |cdocsamp.tex|.
%
% Load the \textsf{childdoc} definitions and
% declare the filename for the main document:
%    \begin{macrocode}
\input{childdoc.def}
\childdocmain{}
%    \end{macrocode}

% Optional override for |\version| flag:
%    \begin{macrocode}
%%\ifchilddoc\else\providecommand{\version}{draft}\fi
%    \end{macrocode}

% Define the default values for the |\version| flag
% (|final| for the main file and |draft| for childs):
%    \begin{macrocode}
\ifchilddoc
\providecommand{\version}{draft}
\else
\providecommand{\version}{final}
\fi
%    \end{macrocode}

% Load the standard document class:
%    \begin{macrocode}
\documentclass[12pt]{article}
%    \end{macrocode}

% Start the document body:
%    \begin{macrocode}
\begin{document}
%    \end{macrocode}

% Declare a title page.
% Print title, part of document being processed and version flag:
%    \begin{macrocode}
\addtocounter{page}{-1}
\begin{center}
{\LARGE\bfseries{}childdoc example\par}
\vspace{1cm}
\ifchilddoc
\ifchilddocmanual part\else chapter\fi:
`\childdocname' of `\childdocjob'\par
\else
main document: `\childdocjob'\par
\fi
version: \version\par
\end{center}
\newpage
%    \end{macrocode}

% Manually include selected file,
% otherwise process as usual:
%    \begin{macrocode}
\ifchilddocmanual
\section*{part `\childdocname'}
\input{\childdocname}
\else
%    \end{macrocode}

% Include the two chapters:
%    \begin{macrocode}
\include{cdocsch1}
\include{cdocsch2}
%    \end{macrocode}

% Include the two parts unless only chapters should be displayed:
%    \begin{macrocode}
\ifchilddoc\else
\section{part three}
\input{cdocspt3}
\section{part four}
\input{cdocspt4}
\fi
%    \end{macrocode}

% Process as usual until here:
%    \begin{macrocode}
\fi
%    \end{macrocode}

% End of document body:
%    \begin{macrocode}
\end{document}
%    \end{macrocode}
%\iffalse
%</samplemain>
%\fi
%
% %%%%%%%%%%%%%%%%%%%%%%%%%%%%%%%%%%%%%%
% \paragraph{Chapter Include Files.}
%
% The include files are called |cdocsch1.tex| and |cdocsch2.tex|.
%
%\iffalse
%<*samplechap1|samplechap2>
%\fi

% Optional override for |\version| flag:
%    \begin{macrocode}
%%\providecommand{\version}{final}
%    \end{macrocode}

% Include the main document:
%    \begin{macrocode}
\input{childdoc.def}
\childdocof{cdocsamp}
%    \end{macrocode}

%\iffalse
%</samplechap1|samplechap2>
%\fi
%
%\iffalse
%<*samplechap1>
%\fi
% Some text for chapter 1:
%    \begin{macrocode}
\section{one}
some text in chapter one
%    \end{macrocode}

%\iffalse
%</samplechap1>
%\fi
% Some text for chapter 2:
%\iffalse
%<*samplechap2>
%\fi
%    \begin{macrocode}
\section{two}
more text in chapter two
%    \end{macrocode}

%\iffalse
%</samplechap2>
%\fi
%
% %%%%%%%%%%%%%%%%%%%%%%%%%%%%%%%%%%%%%%
% \paragraph{Part Include Files.}
%
% The include files are called |cdocspt3.tex| and |cdocspt4.tex|.
%
%\iffalse
%<*samplepart3|samplepart4>
%\fi

% Optional override for |\version| flag:
%    \begin{macrocode}
%%\providecommand{\version}{final}
%    \end{macrocode}

% Include the main document:
%    \begin{macrocode}
\input{childdoc.def}
\childdocby{cdocsamp}
%    \end{macrocode}

%\iffalse
%</samplepart3|samplepart4>
%\fi
%
%\iffalse
%<*samplepart3>
%\fi
% Some text for part 3:
%    \begin{macrocode}
some text in part three
%    \end{macrocode}

%\iffalse
%</samplepart3>
%\fi
% Some text for part 4:
%\iffalse
%<*samplepart4>
%\fi
%    \begin{macrocode}
more text in part four
%    \end{macrocode}

%\iffalse
%</samplepart4>
%\fi
%
% %%%%%%%%%%%%%%%%%%%%%%%%%%%%%%%%%%%%%%
% \paragraph{Forwarding for a Complete Draft.}
%
% The following forwarding file |cdocsdrf.tex|
% compiles the main document in draft mode:
%\iffalse
%<*sampledraft>
%\fi
%    \begin{macrocode}
\def\version{draft}
\input{childdoc.def}
\childdocforward{cdocsamp}
%    \end{macrocode}

%\iffalse
%</sampledraft>
%\fi
%
% %%%%%%%%%%%%%%%%%%%%%%%%%%%%%%%%%%%%%%
% \paragraph{Forwarding for Final Version of the Chapters.}
%
% The following forwarding files |cdocsfn1.tex| and |cdocsfn2.tex|
% (with identical content)
% compile the final versions of the child documents
% |cdocsch1.tex| and |cdocsch2.tex|, respectively:
%\iffalse
%<*samplefinal>
%\fi
%    \begin{macrocode}
\def\version{final}
\input{childdoc.def}
\childdocforwardprefix[cdocsamp]{cdocsfn}{cdocsch}
%    \end{macrocode}

%\iffalse
%</samplefinal>
%\fi
%
% %%%%%%%%%%%%%%%%%%%%%%%%%%%%%%%%%%%%%%
% \paragraph{Command Line Processing.}
%
% The following three command lines generate the output files
% |cdocscld|, |cdocscl1| and |cdocscl2|
% which should be identical to
% |cdocsdrf|, |cdocsch1| and |cdocsfn2|, respectively:
% \begin{center}
% \begin{tabular}{l}
% |latex -jobname cdocscld \|\\
% |  "\def\version{draft}\input{childdoc.def}\childdocforward{cdocsamp}"|\\
% |latex -jobname cdocscl1 \|\\
% |  "\input{childdoc.def}\childdocforward[cdocsamp]{cdocsch1}"|\\
% |latex -jobname cdocscl2 \|\\
% |  "\def\version{final}\input{childdoc.def}\childdocforward{cdocsch2}"|
% \end{tabular}
% \end{center}
% Note that the trailing backslash on each first line
% merely continues the input to the second line
% (for convenient cut ant paste).
% Furthermore, the command |latex| can be replaced by any
% of its alternative versions such as |pdflatex|.
%
% %%%%%%%%%%%%%%%%%%%%%%%%%%%%%%%%%%%%%%%%%%%%%%%%%%%%%%%%%%%%%%%%%%%%%%%%%%%%%%
% %%%%%%%%%%%%%%%%%%%%%%%%%%%%%%%%%%%%%%%%%%%%%%%%%%%%%%%%%%%%%%%%%%%%%%%%%%%%%%
% \section{Implementation}
%\iffalse
%<*package>
%\fi
%
% This section describes the definitions file |childdoc.def|.

% The definitions cannot be loaded using |\usepackage| or |\RequirePackage|
% which has a mechanism to prevent loading a style file more than once.
% When loading the definitions by means of |\input|
% multiple instances have to be prevented manually:
%\iffalse
%This code needs to be before the `\ProvidesFile' directive
%which is defined at the beginning of this file.
%Therefore it is also placed there and commented out here.
%</package>
%<*discard>
%\fi
%    \begin{macrocode}
\ifdefined\childdocmain\endinput\fi
%    \end{macrocode}
%\iffalse
%</discard>
%<*package>
%\fi
%
% \macro{\ifchilddoc}
% \macro{\ifchilddocmanual}
% The conditional |\ifchilddoc| tells whether a
% child (true) or main (false) document is being compiled.
% The conditional |\ifchilddocmanual| tells whether
% the |\includeonly| mechanism is used (false) or
% the selection of child files must be performed manually (true).
% The definitions initialise to false:
%    \begin{macrocode}
\newif\ifchilddoc
\newif\ifchilddocmanual
%    \end{macrocode}

% \macro{\childdocname}
% \macro{\childdocjob}
% The macro |\childdocname| stores the name of the main document
% to be compiled. The macro |\childdocjob| stores the name of
% the document on which the \LaTeX{} compiler was originally invoked.
% The content of |\jobname| cannot be compared
% to filenames specified in the source due to different catcodes.
% The following code rescans |\jobname|, stores the result
% in |\childdocname| and saves a copy in |\childdocjob|:
%    \begin{macrocode}
\edef\childdocname{\scantokens\expandafter{\jobname\noexpand}}
\let\childdocjob\childdocname
%    \end{macrocode}

% \macro{\childdocdisable}
% The macro |\childdocdisable| prevents the main file
% from being processed more than once.
% At this stage, the main document command |\childdocmain|
% is assumed to be called once again where it should do nothing.
% Any subsequent call to it should prevent
% a secondary processing of the main document
% It overwrites the forwarding commands
% |\childdocof| and |\childdocforward|
% with empty macros to prevent further inclusions of the main document:
%    \begin{macrocode}
\newcommand{\childdocdisable}
{
  \renewcommand{\childdocmain}[1]{\renewcommand{\childdocmain}[1]{\endinput}}
  \renewcommand{\childdocof}[1]{}
  \renewcommand{\childdocby}[2][]{}
  \renewcommand{\childdocforward}[2][]{}
  \renewcommand{\childdocdisable}{}
}
%    \end{macrocode}

% \macro{\childdocmain}
% The macro |\childdocmain| is to be called at the top of the main file
% with nothing or the main filename (without extension) as argument.
% First, it breaks loops.
% If the argument is not empty and does not match |\childdocname|
% (which is set by the first inclusion of |childdoc.def|),
% |\ifchilddoc| is set to true, |\includeonly| is applied to the child file
% and |\jobname| is set to the main file
% (for proper handling of |.aux| files):
%    \begin{macrocode}
\newcommand{\childdocmain}[1]
{
  \childdocdisable\childdocmain{}
  \if?#1?\else
    \begingroup
      \def\childdoctmp{#1}
      \ifx\childdoctmp\childdocname
        \def\childdoctmp{}
      \else
        \def\childdoctmp
        {
          \childdoctrue
          \includeonly{\childdocname}
          \def\childdocjob{#1}
          \def\jobname{#1}
        }
      \fi
      \expandafter
    \endgroup
    \childdoctmp
  \fi
}
%    \end{macrocode}

% \macro{\childdocof}
% The command |\childdocof| redirects
% compilation to the main file |#1|.
%    \begin{macrocode}
\newcommand{\childdocof}[1]
{
  \childdocdisable
  \childdoctrue
  \includeonly{\childdocname}
  \def\jobname{#1}
  \def\childdocjob{#1}
  \input{#1}
}
%    \end{macrocode}

% \macro{\childdocby}
% The command |\childdocby| ....
%    \begin{macrocode}
\newcommand{\childdocby}[2][]
{
  \childdocdisable
  \childdoctrue
  \childdocmanualtrue
  \if?#1?\else
    \def\jobname{#2}
  \fi
  \def\childdocjob{#2}
  \input{#2}
  \endinput
}
%    \end{macrocode}

% \macro{\childdocforward}
% The command |\childdocforward| redirects
% compilation to the main file or
% (if the optional argument is given) a child file.
% Parameters are set as if the main file
% or a child file starting with |\childdocof| was compiled.
% Then compilation is handed over to the main file:
%    \begin{macrocode}
\newcommand{\childdocforward}[2][]
{
  \begingroup
    \if?#1?
      \def\childdoctmp
      {
        \def\childdocname{#2}
        \def\childdocjob{#2}
        \def\jobname{#2}
        \input{#2}
        \endinput
      }
    \else
      \def\childdoctmp
      {
        \childdocdisable
        \def\childdocname{#2}
        \childdoctrue
        \includeonly{#2}
        \def\childdocjob{#1}
        \def\jobname{#1}
        \input{#1}
        \endinput
      }
    \fi
    \expandafter
  \endgroup
  \childdoctmp
}
%    \end{macrocode}

% \macro{\childdocforwardprefix}
% The command |\childdocforwardprefix| redirects
% compilation to the main or a child file by means of a pattern.
% The prefix |#1| in the current filename is replaced by |#2|
% and the suffix of the current filename is kept
% (it is assumed that the filename does not contain the substring `|~~~|'
% which is used as a delimiter).
% Compilation is handed over to the new file by |\childdocforward|:
%    \begin{macrocode}
\newcommand{\childdocforwardprefix}[3][]
{
  \begingroup
    \def\childdocextract #2##1~~~{\def\childdoctmp{\childdocforward[#1]{#3##1}}}
    \expandafter\childdocextract\childdocname~~~
    \expandafter
  \endgroup
  \childdoctmp
}
%    \end{macrocode}

% \macro{\childdoc}
% The deprecated macro |\childdoc| is a legacy version of |\childdocmain|:
%    \begin{macrocode}
\newcommand{\childdoc}{\childdocmain}
%    \end{macrocode}

% \macro{\childdocredirect}
% The deprecated macro |\childdocredirect| is a legacy version
% of |\childdocforward| and |\childdocforwardprefix|:
%    \begin{macrocode}
\newcommand{\childdocredirect}[2][]
{
  \begingroup
    \if?#1?
      \def\childdoctmp{\childdocforward{#2}}
    \else
      \def\childdoctmp{\childdocforwardprefix{#1}{#2}}
    \fi
    \expandafter
  \endgroup
  \childdoctmp
}
%    \end{macrocode}

%\iffalse
%</package>
%\fi
%
\endinput

\childdocforwardprefix[cdocsamp]{cdocsfn}{cdocsch}
%    \end{macrocode}

%\iffalse
%</samplefinal>
%\fi
%
% %%%%%%%%%%%%%%%%%%%%%%%%%%%%%%%%%%%%%%
% \paragraph{Command Line Processing.}
%
% The following three command lines generate the output files
% |cdocscld|, |cdocscl1| and |cdocscl2|
% which should be identical to
% |cdocsdrf|, |cdocsch1| and |cdocsfn2|, respectively:
% \begin{center}
% \begin{tabular}{l}
% |latex -jobname cdocscld \|\\
% |  "\def\version{draft}% \iffalse
%
% childdoc.dtx Copyright (C) 2017-2018 Niklas Beisert
%
% This work may be distributed and/or modified under the
% conditions of the LaTeX Project Public License, either version 1.3
% of this license or (at your option) any later version.
% The latest version of this license is in
%   http://www.latex-project.org/lppl.txt
% and version 1.3 or later is part of all distributions of LaTeX
% version 2005/12/01 or later.
%
% This work has the LPPL maintenance status `maintained'.
%
% The Current Maintainer of this work is Niklas Beisert.
%
% This work consists of the files childdoc.dtx and childdoc.ins
% and the derived files childdoc.def and cdocsamp.tex with
% cdocsch1.tex, cdocsch2.tex, cdocsdrf.tex, cdocsfn1.tex, cdocsfn2.tex.
%
%<package>\ifdefined\childdocmain\endinput\fi
%<package>\ProvidesFile{childdoc.def}[2018/12/30 v2.0 child document driver]
%<samplemain>\ProvidesFile{cdocsamp.tex}[2018/12/30 v2.0 sample for childdoc]
%<*driver>
%\ProvidesFile{childdoc.drv}[2018/12/30 v2.0 childdoc reference manual file]
\PassOptionsToClass{10pt,a4paper}{article}
\documentclass{ltxdoc}

\usepackage[margin=35mm]{geometry}
\usepackage{hyperref}
\usepackage{hyperxmp}
\usepackage[usenames]{color}

\hypersetup{colorlinks=true}
\hypersetup{pdfstartview=FitH}
\hypersetup{pdfpagemode=UseNone}
\hypersetup{pdfsource={}}
\hypersetup{pdflang={en-UK}}
\hypersetup{pdfcopyright={Copyright 2017-2018 Niklas Beisert.
  This work may be distributed and/or modified under the
  conditions of the LaTeX Project Public License, either version 1.3
  of this license or (at your option) any later version.}}
\hypersetup{pdflicenseurl={http://www.latex-project.org/lppl.txt}}
\hypersetup{pdfcontactaddress={ETH Zurich, ITP, HIT K,
  Wolfgang-Pauli-Strasse 27}}
\hypersetup{pdfcontactpostcode={8093}}
\hypersetup{pdfcontactcity={Zurich}}
\hypersetup{pdfcontactcountry={Switzerland}}
\hypersetup{pdfcontactemail={nbeisert@itp.phys.ethz.ch}}
\hypersetup{pdfcontacturl={http://people.phys.ethz.ch/\xmptilde nbeisert/}}

\newcommand{\secref}[1]{\hyperref[#1]{section \ref*{#1}}}

\parskip1ex
\parindent0pt
\let\olditemize\itemize
\def\itemize{\olditemize\parskip0pt}

\begin{document}

\title{The \textsf{childdoc} Package}
\hypersetup{pdftitle={The childdoc Package}}
\author{Niklas Beisert\\[2ex]
  Institut f\"ur Theoretische Physik\\
  Eidgen\"ossische Technische Hochschule Z\"urich\\
  Wolfgang-Pauli-Strasse 27, 8093 Z\"urich, Switzerland\\[1ex]
  \href{mailto:nbeisert@itp.phys.ethz.ch}
  {\texttt{nbeisert@itp.phys.ethz.ch}}}
\hypersetup{pdfauthor={Niklas Beisert}}
\hypersetup{pdfsubject={Manual for the LaTeX2e Package childdoc}}
\date{30 December 2018, \textsf{v2.0}}
\maketitle

\begin{abstract}\noindent
\textsf{childdoc} is a \LaTeXe{} package
that enables the direct compilation
of document sections included by |\include|
to individual files.
\end{abstract}

\begingroup
\parskip0ex
\tableofcontents
\endgroup

%%%%%%%%%%%%%%%%%%%%%%%%%%%%%%%%%%%%%%%%%%%%%%%%%%%%%%%%%%%%%%%%%%%%%%%%%%%%%%%%
%%%%%%%%%%%%%%%%%%%%%%%%%%%%%%%%%%%%%%%%%%%%%%%%%%%%%%%%%%%%%%%%%%%%%%%%%%%%%%%%
\section{Introduction}

\LaTeX{} provides a mechanism to structure a large document (such as a book)
into a main file and several child files (containing the chapters)
using the |\include| command.
This mechanism is beneficial for documents
which span hundreds of pages in order to
make the source file(s) more manageable.
Moreover, compilation can be restricted to
selected child files by means of the |\includeonly| command.
The latter feature can be used to reduce the compilation time while editing
(this was significantly more useful in the earlier days of \LaTeX{})
or to generate a smaller document which is easier to navigate.
Another application of |\includeonly| is to generate
documents consisting of selected parts of the complete document.

However, there are a few drawbacks of the plain |\include| mechanism:
\begin{itemize}
\item
The child files cannot be compiled on their own,
they can only be compiled via the main file.
A naive editing environment
(such as a text editor with an option
to have the current file processed by \LaTeX)
may require one to switch to the main file before compiling;
attempting to compile the child file produces errors.
\item
The main file must be modified (each time)
to adjust the |\includeonly| command
to the present needs. This easily leaves the main file in a messy state.
\item
The generated document will always carry the filename
of the main document. This is inconvenient if
several child files are to be compiled and
to be kept for distribution.
\end{itemize}

The present package provides a simple interface
to make child files individually compilable by \LaTeX{}.
Compiling a child file then has the same effect as compiling
the main file with an |\includeonly| command
to select the appropriate child.
Moreover the generated document will carry the name of the child
rather than the main file.
This resolves all three above issues.

This feature is meant to make the editing of books,
thesis documents and lecture notes somewhat more convenient.
However, the package can also be used efficiently for
composing a series of documents (such as exercise sheets)
which are typically distributed individually.
It then assists the author in generating the individual documents
(potentially in different versions)
as well as a document containing the collected series.
Another application is in developing style files
or other kinds of included material
where compilation of the style file could redirect
to a sample or test file.

%%%%%%%%%%%%%%%%%%%%%%%%%%%%%%%%%%%%%%%%%%%%%%%%%%%%%%%%%%%%%%%%%%%%%%%%%%%%%%%%
%%%%%%%%%%%%%%%%%%%%%%%%%%%%%%%%%%%%%%%%%%%%%%%%%%%%%%%%%%%%%%%%%%%%%%%%%%%%%%%%
\section{Usage}

First of all, the package \textsf{childdoc} is \emph{not} a standard
\LaTeXe{} |.sty| style file! Therefore it needs to be invoked in
a non-standard way.

%%%%%%%%%%%%%%%%%%%%%%%%%%%%%%%%%%%%%%%%%%%%%%%%%%%%%%%%%%%%%%%%%%%%%%%%%%%%%%%%
\subsection{Included Files}
\label{sec:include}

%%%%%%%%%%%%%%%%%%%%%%%%%%%%%%%%%%%%%%%%
\DescribeMacro{\childdocmain}
To use the package, add the commands
\begin{center}
\begin{tabular}{l}
|\input{childdoc.def}|\\
|\childdocmain{}|\\
\end{tabular}
\end{center}
at the very top of the main \LaTeX{} file,
in particular \emph{before} the |\documentclass| statement!
The argument of |\childdocmain| should be left empty
(but it must be present).

%%%%%%%%%%%%%%%%%%%%%%%%%%%%%%%%%%%%%%%%
\DescribeMacro{\childdocof}
Furthermore, add the commands
\begin{center}
\begin{tabular}{l}
|\input{childdoc.def}|\\
|\childdocof{|\textit{main}|}|\\
\end{tabular}
\end{center}
at the top of every child file \textit{child}
which is included by |\include{|\textit{child}|}|
from within the main file
(or at least for those files to be compiled individually).
The argument \textit{main} must be the filename of the main file.

There are a couple of
considerations in setting up the main and child documents:

%%%%%%%%%%%%%%%%%%%%%%%%%%%%%%%%%%%%%%%%
\paragraph{Restrictions.}

Please note the following restrictions:
\begin{itemize}
\item
|\childdocmain| must be called with one argument \textit{main}
to ensure compatibility with earlier version of the package.
It must either be empty (|\childdocmain{}|)
or precisely match the filename of the main file in which it is specified.
See \secref{sec:detection} for further information.
\item
The filename \textit{main} must be specified without the |.tex| extension.
\item
The filename \textit{main} is case sensitive
(even in case-insensitive file systems)
due to internal string comparison.
\item
The argument \textit{main} should be fully expanded, it cannot be a macro.
\item
Subdirectories and special characters should be avoided in filenames.
\item
The command |\childdocmain{|\textit{main}|}| must be followed by a whitespace.
It should not be followed immediately by another command
or by a comment mark `|%|'.
This is because the \TeX{} parser reads the token immediately following
the argument of |\childdocmain| and puts it
at the beginning of every child section;
however, a white\-space is ignored.
\end{itemize}

%%%%%%%%%%%%%%%%%%%%%%%%%%%%%%%%%%%%%%%%
\paragraph{Content of Main File.}

It is advisable to place all content in the child files included by |\include|.
Any output contained in the main file will appear in all child documents
unless suppressed manually;
it cannot be suppressed automatically by the |\includeonly| directive
and thus should normally be avoided.
A method to include some content in the main file
by means of conditional processing is described in \secref{sec:conditional}.

%%%%%%%%%%%%%%%%%%%%%%%%%%%%%%%%%%%%%%%%
\paragraph{Page Numbering.}

When only a part of the document is compiled,
the appropriate numbering of pages
(as well as other status parameters)
is determined from the |.aux| files.
The latter contain information from previous passes.
However this information needs to propagate through
all intermediate child documents.
Therefore the page numbering in child documents may well
be inconsistent until the complete document is compiled at least once.

A useful (if unconventional) way to always ensure a consistent
page numbering is to restart the numbering in each child document
and denote the pages by `\textit{child}|.|\textit{page}'
where \textit{child} represents the chapter/section number of the child file.
This can be achieved by the command
|\numberwithin{page}{|\textit{child}|}|
of the \textsf{amsmath} package
where \textit{child} can be |chapter| or |section|
depending on the chosen structuring.
Alternatively, one can modify the macro |\thepage| appropriately
and reset the counter |page| at the start of each child file.

%%%%%%%%%%%%%%%%%%%%%%%%%%%%%%%%%%%%%%%%%%%%%%%%%%%%%%%%%%%%%%%%%%%%%%%%%%%%%%%%
\subsection{Conditional Processing}
\label{sec:conditional}

The package provides a mechanism to compile different versions
of a document. To customise the versions further some conditional processing
can come in handy to distinguish which version is being compiled.
The package provides two macros to describe the compilation context:

%%%%%%%%%%%%%%%%%%%%%%%%%%%%%%%%%%%%%%%%
\DescribeMacro{\ifchilddoc}
The conditional |\ifchilddoc| distinguishes between the compilation of
child documents and the main document:
%
\begin{center}
|\ifchilddoc |\textit{child-code}| |[|\||else |\textit{main-code}]| \||fi|
\end{center}

%%%%%%%%%%%%%%%%%%%%%%%%%%%%%%%%%%%%%%%%
\DescribeMacro{\childdocname}
\DescribeMacro{\childdocjob}
The macro |\childdocname| contains the filename (without extension)
of the main or child file being processed.
Note that |\childdocjob| will always contain the name of the main file.

%%%%%%%%%%%%%%%%%%%%%%%%%%%%%%%%%%%%%%%%
\paragraph{Title Page.}

Conditional processing can be used to include a title or banner page
in the main document when proper precautions are taken.
Importantly, the code in the main file should ensure that the page counter
(as well as other status parameters which are stored in the |.aux| files)
takes the same value after the conditional processing.
Otherwise the page numbers may take divergent values
depending on which part is compiled.

For example, a title page could be declared by:
%
\begin{center}
\begin{tabular}{l}
|\ifchilddoc\||else|\\
|\addtocounter{page}{-1}|\\
\textit{code for title page}\\
|\newpage|\\
|\||fi|
\end{tabular}
\end{center}
%
A banner page for the child documents can be generated by:
%
\begin{center}
\begin{tabular}{l}
|\ifchilddoc|\\
|\addtocounter{page}{-1}|\\
\textit{code for banner page}\\
|\newpage|\\
|\||fi|
\end{tabular}
\end{center}
%
Here one could write a message such as:
\begin{center}
|This is the part \childdocname{} of \childdocjob{}.|
\end{center}

%%%%%%%%%%%%%%%%%%%%%%%%%%%%%%%%%%%%%%%%%%%%%%%%%%%%%%%%%%%%%%%%%%%%%%%%%%%%%%%%
\subsection{Flags}
\label{sec:flags}

The package makes it easy to generate different versions
of the main or child documents.
To this end compilation flags can be defined
and assigned different default values.
They will be particularly useful in conjunction
with the forwarding mechanism described in \secref{sec:forward}.

For example, it may be useful to have a flag |\version|
which can be set to |draft| or |final|.
The document source will contain some conditional code
depending on the value of |\version|.
Suppose further, the flag should default to |final| for the main file
and to |draft| for child files
which is a natural assignment for editing the document.
This is achieved by placing the following code
in the preamble of the main document
(below the |\childdocmain| directive):
%
\begin{center}
\begin{tabular}{l}
|\ifchilddoc|\\
|\providecommand{\version}{draft}|\\
|\||else|\\
|\providecommand{\version}{final}|\\
|\||fi|
\end{tabular}
\end{center}
%
The definition by |\providecommand| makes sure
that previous definitions are not overwritten.
Further statements |\providecommand{\version}{...}|
can thus be added before the above code to override it.

For the main file, one might add a line
(between |\childdocmain| and the above block)
%
\begin{center}
|%\ifchilddoc\||else\providecommand{\version}{draft}\||fi|
\end{center}
%
which can be uncommented to produce a draft version.
Likewise one can add a line to the very top of a child file
(above the |\childdocof{|\textit{main}|}| directive)
%
\begin{center}
|%\providecommand{\version}{final}|
\end{center}
%
which can be uncommented to produce the final version of this child document.

%%%%%%%%%%%%%%%%%%%%%%%%%%%%%%%%%%%%%%%%%%%%%%%%%%%%%%%%%%%%%%%%%%%%%%%%%%%%%%%%
\subsection{Forwarding}
\label{sec:forward}

Different versions of the main or child documents
using compilation flags as described in \secref{sec:flags}
can be (permanently) stored in different files
for convenient compilation, viewing and distribution.
To this end, the package defines a command
to pass on compilation to a different file:

%%%%%%%%%%%%%%%%%%%%%%%%%%%%%%%%%%%%%%%%
\DescribeMacro{\childdocforward}
The command |\childdocforward| redirects processing to
another source file:
%
\begin{center}
\begin{tabular}{l}
|\input{childdoc.def}|\\
|\childdocforward[|\textit{main}|]{|\textit{dest}|}|\\
\end{tabular}
\end{center}
%
The argument \textit{dest} is the destination file
(without extension).
It should be the main file or one of the child files.
Note that further \textsf{childdoc} directives
such as |\childdocof| and |\childdocforward|
in the indicated file will be processed in this form.
The optional argument \textit{main}
passes on directly to the main file \textit{main}
while pretending to compile the child \textit{dest}.
This form behaves as if \textit{dest}
issues |\childdocof{|\textit{main}|}| right away,
and no further \textsf{childdoc} directives will be processed.

%%%%%%%%%%%%%%%%%%%%%%%%%%%%%%%%%%%%%%%%
\DescribeMacro{\...prefix}
In the alternative form |\childdocforwardprefix|,
%
\begin{center}
\begin{tabular}{l}
|\input{childdoc.def}|\\
|\childdocforwardprefix[|\textit{main}|]{|\textit{prefix}|}{|\textit{dest}|}|
\end{tabular}
\end{center}
%
the destination file is determined by a pattern
depending on the current file:
To make this work, the current file must be called
`{\textit{prefix}\hspace{0.2em}\textit{suffix}}'
with \textit{prefix} matching precisely the argument.
Processing is then passed on to the file
`{\textit{dest}\hspace{0.2em}\textit{suffix}}'.
Surely, the same effect is achieved by
directly specifying the
argument `{\textit{dest}\hspace{0.2em}\textit{suffix}}'
in the first form.
However, that requires to set up a different file
for each child. With the alternative form of the command
all these files can have exactly the same content
which simplifies setting them up and maintaining them.

For example, the following file |draft.tex|
with a compilation flag |\version| as described in \secref{sec:flags}
compiles the main document as a draft:
%
\begin{center}
\begin{tabular}{l}
|\def\version{draft}|\\
|\input{childdoc.def}|\\
|\childdocforward{|\textit{main}|}|
\end{tabular}
\end{center}
%
Likewise, the following files |final|\textit{nn}|.tex|
compile the final version of the child document
|child|\textit{nn}|.tex|:
%
\begin{center}
\begin{tabular}{l}
|\def\version{final}|\\
|\input{childdoc.def}|\\
|\childdocforwardprefix{final}{child}|
\end{tabular}
\end{center}
%

Note that when several versions of a main file and/or of each child file
are to be generated, it may be convenient to set up a |Makefile| or
shell script to automatise the process.

%%%%%%%%%%%%%%%%%%%%%%%%%%%%%%%%%%%%%%%%%%%%%%%%%%%%%%%%%%%%%%%%%%%%%%%%%%%%%%%%
\subsection{Command Line Processing}
\label{sec:commandline}

The effect of redirection files can also be achieved by invoking
the \LaTeX{} compiler with a more elaborate command line.
Most conveniently this should be done as part
of a shell script or a |Makefile|.

When using \textsf{childdoc} in the main file, the following
command lines effectively perform a redirection
(note that depending on the shell being used,
backslashes may have to be doubled: `|\|' $\to$ `|\\|'):
%
\begin{center}
|... -jobname "|\textit{target}|" |\\|"|[\textit{flags}]%
|\input{childdoc.def}\childdocforward[|\textit{main}|]{|\textit{dest}|}"|
\end{center}
%
Here \textit{target} is the name of the output file,
\textit{main} is the name of the main file
and \textit{dest} is the name of the main or child file to be processed
(all filenames without extensions).
The optional argument \textit{main} can be omitted
if \textit{main} matches \textit{dest}.
Optionally, compilation \textit{flags} can be defined via |\def| commands.
This command line makes the \TeX{} engine believe
it is compiling the file \textit{target}
whose content is specified as the latter parameter.
The provided code then forwards the processing to
\textit{main} or \textit{dest} as described in \secref{sec:forward}.

%%%%%%%%%%%%%%%%%%%%%%%%%%%%%%%%%%%%%%%%%%%%%%%%%%%%%%%%%%%%%%%%%%%%%%%%%%%%%%%%
\subsection{Include by Input}
\label{sec:input}

Including child documents by |\include| has some restrictions by design.
Most notably, the content of a child document always occupies
its own set of pages; pages cannot be shared between child documents.
Usually, this behaviour makes perfect sense
because each child document contain an essential part of the document.
However, in some situations it may be desirable to compose
a document from a collection of parts
without having mandatory page breaks between then.
For this case, the package
provides a mechanism to include parts
by |\input| which can also be processed individually.
However, by construction this mechanism
requires manual handling of the content to be output.

%%%%%%%%%%%%%%%%%%%%%%%%%%%%%%%%%%%%%%%%
\DescribeMacro{\ifchilddocmanual}
The main file should be prepared as usual, see \secref{sec:include}.
However, the document body must make a distinction
between processing of an individual part and of the main document, e.g.:
%
\begin{center}
\begin{tabular}{l}
|\ifchilddocmanual|\\
|\input{\childdocname}|\\
|\||else|\\
\textit{document body with }|\input{|\textit{part}|}|\\
|\||fi|
\end{tabular}
\end{center}
%
The conditional |\ifchilddocmanual| is true whenever
a part to be included by |\input| is being compiled,
and the name of the part is stored in |\childdocname|.

%%%%%%%%%%%%%%%%%%%%%%%%%%%%%%%%%%%%%%%%
\DescribeMacro{\childdocby}
Each part to be included by |\input| should start with:
%
\begin{center}
\begin{tabular}{l}
|\input{childdoc.def}|\\
|\childdocby{|\textit{main}|}|\\
\end{tabular}
\end{center}
%
The directive |\childdocby| is similar to |\childdocof|
described in \secref{sec:include},
but the subsequent selection of content must be done manually.
To that end, both |\ifchilddoc| and |\ifchilddocmanual|
will be true upon processing of a part,
and the name of the part is stored in |\childdocname|.
Note that |\jobname| will be set to the filename of the current part
so that each part receives an individual |.aux| file
that does not interfere with the |.aux| file(s) of the main document.
This behaviour can be altered by the alternative form
|\childdocby[*]{|\textit{main}|}| (with a non-empty optional argument)
which uses the |.aux| file of the main document
by setting |\jobname| to \textit{main}.

%%%%%%%%%%%%%%%%%%%%%%%%%%%%%%%%%%%%%%%%%%%%%%%%%%%%%%%%%%%%%%%%%%%%%%%%%%%%%%%%
\subsection{Driver Development}
\label{sec:driver}

The \textsf{childdoc} mechanism can also be use for the development
of definition files such as \LaTeX{} styles or classes.
This case differs from the above setup with multiple parts
included by |\include| in that no |\includeonly| should be invoked.
This can be achieved by starting the include file
(before |\ProvidesPackage|) with:
%
\begin{center}
\begin{tabular}{l}
|\input{childdoc.def}|\\
|\childdocforward{|\textit{main}|}|\\
\end{tabular}
\end{center}
%
or alternatively with:
%
\begin{center}
\begin{tabular}{l}
|\input{childdoc.def}|\\
|\childdocby{|\textit{main}|}|\\
\end{tabular}
\end{center}
%
Both forms have slightly different effects as described above.
The main file is prepared as usual, see \secref{sec:include}.

%%%%%%%%%%%%%%%%%%%%%%%%%%%%%%%%%%%%%%%%%%%%%%%%%%%%%%%%%%%%%%%%%%%%%%%%%%%%%%%%
\subsection{Legacy Detection}
\label{sec:detection}

The directive |\childdocmain| in the main file can detect
whether the complete document or merely a child is to be compiled
even without using the directive |\childdocof|.
This method is deprecated because it is less robust
and there is no compelling reason to use it;
it is merely provided for backward compatibility
and it may be removed in future versions.

If the detection mechanism is to be used,
it is mandatory to correctly specify
the filename of the main file as the argument of |\childdocmain|:
%
\begin{center}
\begin{tabular}{l}
|\input{childdoc.def}|\\
|\childdocmain{|\textit{main}|}|\\
\end{tabular}
\end{center}
%
If |\jobname| does not match the argument \textit{main} of |\childdocmain|,
it is assumed that |\jobname| points to the child file to be compiled.
When using |\childdocmain| with the main file specified as argument,
it suffices to start a child file
with just |\input{|\textit{main}|}|
without loading of the package and using |\childdocof|.
If instead all processing is done
with the appropriate \textsf{childdoc} directives,
the argument of \textit{main} of |\childdocmain| can be empty.

An alternative version of the command line processing described
in \secref{sec:commandline} using the detection mechanism reads:
%
\begin{center}
|... -jobname "|\textit{target}|" "|[\textit{flags}]%
[|\def\jobname{|\textit{dest}|}|]|\input{|\textit{main}|}"|
\end{center}

%%%%%%%%%%%%%%%%%%%%%%%%%%%%%%%%%%%%%%%%%%%%%%%%%%%%%%%%%%%%%%%%%%%%%%%%%%%%%%%%
\subsection{Manual Code}
\label{sec:manual}

In case one cannot be certain whether the definitions file |childdoc.def|
is installed on the target \TeX{} distribution
and one prefers not to ship it,
it is conceivable to paste a few relevant commands into the sources.

To that end, drop all statements |\input{childdoc.def}|
and perform the replacements as outlined below.
Instead of |\childdocmain{|\textit{main}|}| add the following code
to the top of the main file:
%
\begin{center}
\begin{tabular}{l}
|\||ifdefined\childdocname\endinput\||fi\newif\ifchilddoc|\\
|\edef\childdocname{\scantokens\expandafter{\jobname\noexpand}}|\\
|\def\childdocmain{|\textit{main}|}\||ifx\childdocmain\childdocname\||else|\\
|\childdoctrue\includeonly{\childdocname}\let\jobname\childdocmain\||fi|\\
\end{tabular}
\end{center}
%
Instead of |\childdocof{|\textit{main}|}| just include the main file
at the top of each child file:
%
\begin{center}
|\input{|\textit{main}|}|
\end{center}
%
A simple redirection |\childdocforward{|\textit{dest}|}| is achieved by:
%
\begin{center}
|\def\jobname{|\textit{dest}|}\input{\jobname}|
\end{center}
%
The redirection with prefix
|\childdocforwardprefix[|\textit{prefix}|]{|\textit{dest}|}|
is accomplished by:
%
\begin{center}
\begin{tabular}{l}
|{\edef\jobname{\scantokens\expandafter{\jobname\noexpand}}|\\
|\def\redirectjob |\textit{prefix}|#1~~~{\gdef\jobname{|\textit{dest}|#1}}|\\
|\expandafter\redirectjob\jobname~~~}\input{\jobname}|
\end{tabular}
\end{center}

In an alternative approach,
child documents can be compiled by a specific command line
without additional code or specific definitions:
%
\begin{center}
|... -jobname "|\textit{target}|" "|[\textit{flags}]%
|\includeonly{|\textit{dest}|}\input{|\textit{main}|}"|
\end{center}
%

%%%%%%%%%%%%%%%%%%%%%%%%%%%%%%%%%%%%%%%%%%%%%%%%%%%%%%%%%%%%%%%%%%%%%%%%%%%%%%%%
%%%%%%%%%%%%%%%%%%%%%%%%%%%%%%%%%%%%%%%%%%%%%%%%%%%%%%%%%%%%%%%%%%%%%%%%%%%%%%%%
\section{Information}

%%%%%%%%%%%%%%%%%%%%%%%%%%%%%%%%%%%%%%%%%%%%%%%%%%%%%%%%%%%%%%%%%%%%%%%%%%%%%%%%
\subsection{Copyright}

Copyright \copyright{} 2017--2018 Niklas Beisert

This work may be distributed and/or modified under the
conditions of the \LaTeX{} Project Public License, either version 1.3
of this license or (at your option) any later version.
The latest version of this license is in
  \url{http://www.latex-project.org/lppl.txt}
and version 1.3 or later is part of all distributions of \LaTeX{}
version 2005/12/01 or later.

This work has the LPPL maintenance status `maintained'.

The Current Maintainer of this work is Niklas Beisert.

This work consists of the files |README.txt|, |childdoc.ins| and |childdoc.dtx|
as well as the derived files |childdoc.def|, |cdocsamp.tex|
with |cdocsch1.tex|, |cdocsch2.tex|, |cdocspt3.tex|, |cdocspt4.tex|,
|cdocsdrf.tex|, |cdocsfn1.tex|, |cdocsfn2.tex|
as well as |childdoc.pdf|.

%%%%%%%%%%%%%%%%%%%%%%%%%%%%%%%%%%%%%%%%%%%%%%%%%%%%%%%%%%%%%%%%%%%%%%%%%%%%%%%%
\subsection{Files and Installation}

The package consists of the files:
%
\begin{center}
\begin{tabular}{ll}
    |README.txt|   & readme file \\
    |childdoc.ins| & installation file \\
    |childdoc.dtx| & source file \\
    |childdoc.def| & definition file \\
    |cdocsamp.tex| & sample main file \\
    |cdocsch1.tex| & sample include file \\
    |cdocsch2.tex| & sample include file \\
    |cdocspt3.tex| & sample part file \\
    |cdocspt4.tex| & sample part file \\
    |cdocsdrf.tex| & sample redirection file \\
    |cdocsfn1.tex| & sample redirection file \\
    |cdocsfn2.tex| & sample redirection file \\
    |childdoc.pdf| & manual
\end{tabular}
\end{center}
%
The distribution consists of the files
|README.txt|, |childdoc.ins| and |childdoc.dtx|.
%
\begin{itemize}
\item
Run (pdf)\LaTeX{} on |childdoc.dtx|
to compile the manual |childdoc.pdf| (this file).
\item
Run \LaTeX{} on |childdoc.ins| to create the definitions file |childdoc.def|
and the sample |cdocsamp.tex| with include files
|cdocsch1.tex|, |cdocsch2.tex|, |cdocspt3.tex|, |cdocspt4.tex|,
|cdocsdrf.tex|, |cdocsfn1.tex|, |cdocsfn2.tex|.
Then copy the file |childdoc.def| to an appropriate directory of your \LaTeX{}
distribution, e.g.\ \textit{texmf-root}|/tex/latex/childdoc|.
\end{itemize}

%%%%%%%%%%%%%%%%%%%%%%%%%%%%%%%%%%%%%%%%%%%%%%%%%%%%%%%%%%%%%%%%%%%%%%%%%%%%%%%%
\subsection{Related CTAN Packages}

There are several other packages which offer a similar functionality:
%
\begin{itemize}
\item
The packages
\href{http://ctan.org/pkg/docmute}{\textsf{docmute}},
\href{http://ctan.org/pkg/includex}{\textsf{includex}} and
\href{http://ctan.org/pkg/standalone}{\textsf{standalone}}
provide commands to include only the document body of
a child file thus allowing both files to be compiled individually.
\item
The packages \href{http://ctan.org/pkg/subdocs}{\textsf{subdocs}}
and \href{http://ctan.org/pkg/subfiles}{\textsf{subfiles}}
provide structures in which the main and child documents can be
encapsulated and allowing them to be compiled individually.
The inclusion mechanism is different from the conventional |\include|.
\item
The package \href{http://ctan.org/pkg/combine}{\textsf{combine}}
is an elaborate solution to combine several documents into one.
\end{itemize}
%
See also the CTAN topic \href{http://ctan.org/topic/subdocs}{\textsf{subdocs}}
for further related packages.
The present package differs from the above solutions in that
a document structure constructed with the conventional |\include| mechanism
just needs two extra commands at the top of every file
such that all constituent files can be compiled individually.

%%%%%%%%%%%%%%%%%%%%%%%%%%%%%%%%%%%%%%%%%%%%%%%%%%%%%%%%%%%%%%%%%%%%%%%%%%%%%%%%
%\subsection{Feature Suggestions}
%
%The following is a list of features which may be useful for future
%versions of this package:
%%
%\begin{itemize}
%\item
%\ldots
%\end{itemize}

%%%%%%%%%%%%%%%%%%%%%%%%%%%%%%%%%%%%%%%%%%%%%%%%%%%%%%%%%%%%%%%%%%%%%%%%%%%%%%%%
\subsection{Revision History}

%%%%%%%%%%%%%%%%%%%%%%%%%%%%%%%%%%%%%%%%
\paragraph{v2.0:} 2018/12/30

\begin{itemize}
\item
immediate forward processing
\item
added |\childdocby| mechanism
\item
manual restructured
\end{itemize}

%%%%%%%%%%%%%%%%%%%%%%%%%%%%%%%%%%%%%%%%
\paragraph{v1.6:} 2018/01/17

\begin{itemize}
\item
application for development of include files
\item
corrections to manual
\end{itemize}

%%%%%%%%%%%%%%%%%%%%%%%%%%%%%%%%%%%%%%%%
\paragraph{v1.5:} 2017/05/21

\begin{itemize}
\item
more complete structuring introduced
\item
|\childdocof| introduced
\item
|\childdoc| renamed to |\childdocmain|
\item
|\childredirect| renamed to |\childdocforward| and |\childdocforwardprefix|
and functionality expanded
\end{itemize}

%%%%%%%%%%%%%%%%%%%%%%%%%%%%%%%%%%%%%%%%
\paragraph{v1.0:} 2017/04/27

\begin{itemize}
\item
manual and install package
\item
first version published on CTAN
\end{itemize}

%%%%%%%%%%%%%%%%%%%%%%%%%%%%%%%%%%%%%%%%
\paragraph{v0.6:} 2017/04/26

\begin{itemize}
\item
redirection mechanism added
\end{itemize}

%%%%%%%%%%%%%%%%%%%%%%%%%%%%%%%%%%%%%%%%
\paragraph{v0.5:} 2017/04/26

\begin{itemize}
\item
functionality in definition file
\end{itemize}


%%%%%%%%%%%%%%%%%%%%%%%%%%%%%%%%%%%%%%%%%%%%%%%%%%%%%%%%%%%%%%%%%%%%%%%%%%%%%%%%
%%%%%%%%%%%%%%%%%%%%%%%%%%%%%%%%%%%%%%%%%%%%%%%%%%%%%%%%%%%%%%%%%%%%%%%%%%%%%%%%
%%%%%%%%%%%%%%%%%%%%%%%%%%%%%%%%%%%%%%%%%%%%%%%%%%%%%%%%%%%%%%%%%%%%%%%%%%%%%%%%
\appendix

\settowidth\MacroIndent{\rmfamily\scriptsize 000\ }

 \DocInput{childdoc.dtx}

\end{document}
%</driver>
% \fi
%
% %%%%%%%%%%%%%%%%%%%%%%%%%%%%%%%%%%%%%%%%%%%%%%%%%%%%%%%%%%%%%%%%%%%%%%%%%%%%%%
% %%%%%%%%%%%%%%%%%%%%%%%%%%%%%%%%%%%%%%%%%%%%%%%%%%%%%%%%%%%%%%%%%%%%%%%%%%%%%%
% \section{Sample}
%\iffalse
%<*samplemain>
%\fi
%
% The following presents a sample document
% with two chapters, two parts, a title page,
% a compile flag as well as three forwarding files to set the flag.
% It consists of eight |.tex| files:
% \begin{center}
% \begin{tabular}{ll}
% |cdocsamp.tex|&main file\\
% |cdocsch1.tex|&include file for chapter 1\\
% |cdocsch2.tex|&include file for chapter 2\\
% |cdocspt3.tex|&include file for part 3\\
% |cdocspt4.tex|&include file for part 4\\
% |cdocsdrf.tex|&forwarding file for main file in draft mode\\
% |cdocsfi1.tex|&forwarding file for final version of chapter 1\\
% |cdocsfi2.tex|&forwarding file for final version of chapter 2\\
% \end{tabular}
% \end{center}
% Each of the eight files can be compiled directly by the \LaTeX{} compiler.
%
% %%%%%%%%%%%%%%%%%%%%%%%%%%%%%%%%%%%%%%
% \paragraph{Main File.}
%
% The main file is called |cdocsamp.tex|.
%
% Load the \textsf{childdoc} definitions and
% declare the filename for the main document:
%    \begin{macrocode}
\input{childdoc.def}
\childdocmain{}
%    \end{macrocode}

% Optional override for |\version| flag:
%    \begin{macrocode}
%%\ifchilddoc\else\providecommand{\version}{draft}\fi
%    \end{macrocode}

% Define the default values for the |\version| flag
% (|final| for the main file and |draft| for childs):
%    \begin{macrocode}
\ifchilddoc
\providecommand{\version}{draft}
\else
\providecommand{\version}{final}
\fi
%    \end{macrocode}

% Load the standard document class:
%    \begin{macrocode}
\documentclass[12pt]{article}
%    \end{macrocode}

% Start the document body:
%    \begin{macrocode}
\begin{document}
%    \end{macrocode}

% Declare a title page.
% Print title, part of document being processed and version flag:
%    \begin{macrocode}
\addtocounter{page}{-1}
\begin{center}
{\LARGE\bfseries{}childdoc example\par}
\vspace{1cm}
\ifchilddoc
\ifchilddocmanual part\else chapter\fi:
`\childdocname' of `\childdocjob'\par
\else
main document: `\childdocjob'\par
\fi
version: \version\par
\end{center}
\newpage
%    \end{macrocode}

% Manually include selected file,
% otherwise process as usual:
%    \begin{macrocode}
\ifchilddocmanual
\section*{part `\childdocname'}
\input{\childdocname}
\else
%    \end{macrocode}

% Include the two chapters:
%    \begin{macrocode}
\include{cdocsch1}
\include{cdocsch2}
%    \end{macrocode}

% Include the two parts unless only chapters should be displayed:
%    \begin{macrocode}
\ifchilddoc\else
\section{part three}
\input{cdocspt3}
\section{part four}
\input{cdocspt4}
\fi
%    \end{macrocode}

% Process as usual until here:
%    \begin{macrocode}
\fi
%    \end{macrocode}

% End of document body:
%    \begin{macrocode}
\end{document}
%    \end{macrocode}
%\iffalse
%</samplemain>
%\fi
%
% %%%%%%%%%%%%%%%%%%%%%%%%%%%%%%%%%%%%%%
% \paragraph{Chapter Include Files.}
%
% The include files are called |cdocsch1.tex| and |cdocsch2.tex|.
%
%\iffalse
%<*samplechap1|samplechap2>
%\fi

% Optional override for |\version| flag:
%    \begin{macrocode}
%%\providecommand{\version}{final}
%    \end{macrocode}

% Include the main document:
%    \begin{macrocode}
\input{childdoc.def}
\childdocof{cdocsamp}
%    \end{macrocode}

%\iffalse
%</samplechap1|samplechap2>
%\fi
%
%\iffalse
%<*samplechap1>
%\fi
% Some text for chapter 1:
%    \begin{macrocode}
\section{one}
some text in chapter one
%    \end{macrocode}

%\iffalse
%</samplechap1>
%\fi
% Some text for chapter 2:
%\iffalse
%<*samplechap2>
%\fi
%    \begin{macrocode}
\section{two}
more text in chapter two
%    \end{macrocode}

%\iffalse
%</samplechap2>
%\fi
%
% %%%%%%%%%%%%%%%%%%%%%%%%%%%%%%%%%%%%%%
% \paragraph{Part Include Files.}
%
% The include files are called |cdocspt3.tex| and |cdocspt4.tex|.
%
%\iffalse
%<*samplepart3|samplepart4>
%\fi

% Optional override for |\version| flag:
%    \begin{macrocode}
%%\providecommand{\version}{final}
%    \end{macrocode}

% Include the main document:
%    \begin{macrocode}
\input{childdoc.def}
\childdocby{cdocsamp}
%    \end{macrocode}

%\iffalse
%</samplepart3|samplepart4>
%\fi
%
%\iffalse
%<*samplepart3>
%\fi
% Some text for part 3:
%    \begin{macrocode}
some text in part three
%    \end{macrocode}

%\iffalse
%</samplepart3>
%\fi
% Some text for part 4:
%\iffalse
%<*samplepart4>
%\fi
%    \begin{macrocode}
more text in part four
%    \end{macrocode}

%\iffalse
%</samplepart4>
%\fi
%
% %%%%%%%%%%%%%%%%%%%%%%%%%%%%%%%%%%%%%%
% \paragraph{Forwarding for a Complete Draft.}
%
% The following forwarding file |cdocsdrf.tex|
% compiles the main document in draft mode:
%\iffalse
%<*sampledraft>
%\fi
%    \begin{macrocode}
\def\version{draft}
\input{childdoc.def}
\childdocforward{cdocsamp}
%    \end{macrocode}

%\iffalse
%</sampledraft>
%\fi
%
% %%%%%%%%%%%%%%%%%%%%%%%%%%%%%%%%%%%%%%
% \paragraph{Forwarding for Final Version of the Chapters.}
%
% The following forwarding files |cdocsfn1.tex| and |cdocsfn2.tex|
% (with identical content)
% compile the final versions of the child documents
% |cdocsch1.tex| and |cdocsch2.tex|, respectively:
%\iffalse
%<*samplefinal>
%\fi
%    \begin{macrocode}
\def\version{final}
\input{childdoc.def}
\childdocforwardprefix[cdocsamp]{cdocsfn}{cdocsch}
%    \end{macrocode}

%\iffalse
%</samplefinal>
%\fi
%
% %%%%%%%%%%%%%%%%%%%%%%%%%%%%%%%%%%%%%%
% \paragraph{Command Line Processing.}
%
% The following three command lines generate the output files
% |cdocscld|, |cdocscl1| and |cdocscl2|
% which should be identical to
% |cdocsdrf|, |cdocsch1| and |cdocsfn2|, respectively:
% \begin{center}
% \begin{tabular}{l}
% |latex -jobname cdocscld \|\\
% |  "\def\version{draft}\input{childdoc.def}\childdocforward{cdocsamp}"|\\
% |latex -jobname cdocscl1 \|\\
% |  "\input{childdoc.def}\childdocforward[cdocsamp]{cdocsch1}"|\\
% |latex -jobname cdocscl2 \|\\
% |  "\def\version{final}\input{childdoc.def}\childdocforward{cdocsch2}"|
% \end{tabular}
% \end{center}
% Note that the trailing backslash on each first line
% merely continues the input to the second line
% (for convenient cut ant paste).
% Furthermore, the command |latex| can be replaced by any
% of its alternative versions such as |pdflatex|.
%
% %%%%%%%%%%%%%%%%%%%%%%%%%%%%%%%%%%%%%%%%%%%%%%%%%%%%%%%%%%%%%%%%%%%%%%%%%%%%%%
% %%%%%%%%%%%%%%%%%%%%%%%%%%%%%%%%%%%%%%%%%%%%%%%%%%%%%%%%%%%%%%%%%%%%%%%%%%%%%%
% \section{Implementation}
%\iffalse
%<*package>
%\fi
%
% This section describes the definitions file |childdoc.def|.

% The definitions cannot be loaded using |\usepackage| or |\RequirePackage|
% which has a mechanism to prevent loading a style file more than once.
% When loading the definitions by means of |\input|
% multiple instances have to be prevented manually:
%\iffalse
%This code needs to be before the `\ProvidesFile' directive
%which is defined at the beginning of this file.
%Therefore it is also placed there and commented out here.
%</package>
%<*discard>
%\fi
%    \begin{macrocode}
\ifdefined\childdocmain\endinput\fi
%    \end{macrocode}
%\iffalse
%</discard>
%<*package>
%\fi
%
% \macro{\ifchilddoc}
% \macro{\ifchilddocmanual}
% The conditional |\ifchilddoc| tells whether a
% child (true) or main (false) document is being compiled.
% The conditional |\ifchilddocmanual| tells whether
% the |\includeonly| mechanism is used (false) or
% the selection of child files must be performed manually (true).
% The definitions initialise to false:
%    \begin{macrocode}
\newif\ifchilddoc
\newif\ifchilddocmanual
%    \end{macrocode}

% \macro{\childdocname}
% \macro{\childdocjob}
% The macro |\childdocname| stores the name of the main document
% to be compiled. The macro |\childdocjob| stores the name of
% the document on which the \LaTeX{} compiler was originally invoked.
% The content of |\jobname| cannot be compared
% to filenames specified in the source due to different catcodes.
% The following code rescans |\jobname|, stores the result
% in |\childdocname| and saves a copy in |\childdocjob|:
%    \begin{macrocode}
\edef\childdocname{\scantokens\expandafter{\jobname\noexpand}}
\let\childdocjob\childdocname
%    \end{macrocode}

% \macro{\childdocdisable}
% The macro |\childdocdisable| prevents the main file
% from being processed more than once.
% At this stage, the main document command |\childdocmain|
% is assumed to be called once again where it should do nothing.
% Any subsequent call to it should prevent
% a secondary processing of the main document
% It overwrites the forwarding commands
% |\childdocof| and |\childdocforward|
% with empty macros to prevent further inclusions of the main document:
%    \begin{macrocode}
\newcommand{\childdocdisable}
{
  \renewcommand{\childdocmain}[1]{\renewcommand{\childdocmain}[1]{\endinput}}
  \renewcommand{\childdocof}[1]{}
  \renewcommand{\childdocby}[2][]{}
  \renewcommand{\childdocforward}[2][]{}
  \renewcommand{\childdocdisable}{}
}
%    \end{macrocode}

% \macro{\childdocmain}
% The macro |\childdocmain| is to be called at the top of the main file
% with nothing or the main filename (without extension) as argument.
% First, it breaks loops.
% If the argument is not empty and does not match |\childdocname|
% (which is set by the first inclusion of |childdoc.def|),
% |\ifchilddoc| is set to true, |\includeonly| is applied to the child file
% and |\jobname| is set to the main file
% (for proper handling of |.aux| files):
%    \begin{macrocode}
\newcommand{\childdocmain}[1]
{
  \childdocdisable\childdocmain{}
  \if?#1?\else
    \begingroup
      \def\childdoctmp{#1}
      \ifx\childdoctmp\childdocname
        \def\childdoctmp{}
      \else
        \def\childdoctmp
        {
          \childdoctrue
          \includeonly{\childdocname}
          \def\childdocjob{#1}
          \def\jobname{#1}
        }
      \fi
      \expandafter
    \endgroup
    \childdoctmp
  \fi
}
%    \end{macrocode}

% \macro{\childdocof}
% The command |\childdocof| redirects
% compilation to the main file |#1|.
%    \begin{macrocode}
\newcommand{\childdocof}[1]
{
  \childdocdisable
  \childdoctrue
  \includeonly{\childdocname}
  \def\jobname{#1}
  \def\childdocjob{#1}
  \input{#1}
}
%    \end{macrocode}

% \macro{\childdocby}
% The command |\childdocby| ....
%    \begin{macrocode}
\newcommand{\childdocby}[2][]
{
  \childdocdisable
  \childdoctrue
  \childdocmanualtrue
  \if?#1?\else
    \def\jobname{#2}
  \fi
  \def\childdocjob{#2}
  \input{#2}
  \endinput
}
%    \end{macrocode}

% \macro{\childdocforward}
% The command |\childdocforward| redirects
% compilation to the main file or
% (if the optional argument is given) a child file.
% Parameters are set as if the main file
% or a child file starting with |\childdocof| was compiled.
% Then compilation is handed over to the main file:
%    \begin{macrocode}
\newcommand{\childdocforward}[2][]
{
  \begingroup
    \if?#1?
      \def\childdoctmp
      {
        \def\childdocname{#2}
        \def\childdocjob{#2}
        \def\jobname{#2}
        \input{#2}
        \endinput
      }
    \else
      \def\childdoctmp
      {
        \childdocdisable
        \def\childdocname{#2}
        \childdoctrue
        \includeonly{#2}
        \def\childdocjob{#1}
        \def\jobname{#1}
        \input{#1}
        \endinput
      }
    \fi
    \expandafter
  \endgroup
  \childdoctmp
}
%    \end{macrocode}

% \macro{\childdocforwardprefix}
% The command |\childdocforwardprefix| redirects
% compilation to the main or a child file by means of a pattern.
% The prefix |#1| in the current filename is replaced by |#2|
% and the suffix of the current filename is kept
% (it is assumed that the filename does not contain the substring `|~~~|'
% which is used as a delimiter).
% Compilation is handed over to the new file by |\childdocforward|:
%    \begin{macrocode}
\newcommand{\childdocforwardprefix}[3][]
{
  \begingroup
    \def\childdocextract #2##1~~~{\def\childdoctmp{\childdocforward[#1]{#3##1}}}
    \expandafter\childdocextract\childdocname~~~
    \expandafter
  \endgroup
  \childdoctmp
}
%    \end{macrocode}

% \macro{\childdoc}
% The deprecated macro |\childdoc| is a legacy version of |\childdocmain|:
%    \begin{macrocode}
\newcommand{\childdoc}{\childdocmain}
%    \end{macrocode}

% \macro{\childdocredirect}
% The deprecated macro |\childdocredirect| is a legacy version
% of |\childdocforward| and |\childdocforwardprefix|:
%    \begin{macrocode}
\newcommand{\childdocredirect}[2][]
{
  \begingroup
    \if?#1?
      \def\childdoctmp{\childdocforward{#2}}
    \else
      \def\childdoctmp{\childdocforwardprefix{#1}{#2}}
    \fi
    \expandafter
  \endgroup
  \childdoctmp
}
%    \end{macrocode}

%\iffalse
%</package>
%\fi
%
\endinput
\childdocforward{cdocsamp}"|\\
% |latex -jobname cdocscl1 \|\\
% |  "% \iffalse
%
% childdoc.dtx Copyright (C) 2017-2018 Niklas Beisert
%
% This work may be distributed and/or modified under the
% conditions of the LaTeX Project Public License, either version 1.3
% of this license or (at your option) any later version.
% The latest version of this license is in
%   http://www.latex-project.org/lppl.txt
% and version 1.3 or later is part of all distributions of LaTeX
% version 2005/12/01 or later.
%
% This work has the LPPL maintenance status `maintained'.
%
% The Current Maintainer of this work is Niklas Beisert.
%
% This work consists of the files childdoc.dtx and childdoc.ins
% and the derived files childdoc.def and cdocsamp.tex with
% cdocsch1.tex, cdocsch2.tex, cdocsdrf.tex, cdocsfn1.tex, cdocsfn2.tex.
%
%<package>\ifdefined\childdocmain\endinput\fi
%<package>\ProvidesFile{childdoc.def}[2018/12/30 v2.0 child document driver]
%<samplemain>\ProvidesFile{cdocsamp.tex}[2018/12/30 v2.0 sample for childdoc]
%<*driver>
%\ProvidesFile{childdoc.drv}[2018/12/30 v2.0 childdoc reference manual file]
\PassOptionsToClass{10pt,a4paper}{article}
\documentclass{ltxdoc}

\usepackage[margin=35mm]{geometry}
\usepackage{hyperref}
\usepackage{hyperxmp}
\usepackage[usenames]{color}

\hypersetup{colorlinks=true}
\hypersetup{pdfstartview=FitH}
\hypersetup{pdfpagemode=UseNone}
\hypersetup{pdfsource={}}
\hypersetup{pdflang={en-UK}}
\hypersetup{pdfcopyright={Copyright 2017-2018 Niklas Beisert.
  This work may be distributed and/or modified under the
  conditions of the LaTeX Project Public License, either version 1.3
  of this license or (at your option) any later version.}}
\hypersetup{pdflicenseurl={http://www.latex-project.org/lppl.txt}}
\hypersetup{pdfcontactaddress={ETH Zurich, ITP, HIT K,
  Wolfgang-Pauli-Strasse 27}}
\hypersetup{pdfcontactpostcode={8093}}
\hypersetup{pdfcontactcity={Zurich}}
\hypersetup{pdfcontactcountry={Switzerland}}
\hypersetup{pdfcontactemail={nbeisert@itp.phys.ethz.ch}}
\hypersetup{pdfcontacturl={http://people.phys.ethz.ch/\xmptilde nbeisert/}}

\newcommand{\secref}[1]{\hyperref[#1]{section \ref*{#1}}}

\parskip1ex
\parindent0pt
\let\olditemize\itemize
\def\itemize{\olditemize\parskip0pt}

\begin{document}

\title{The \textsf{childdoc} Package}
\hypersetup{pdftitle={The childdoc Package}}
\author{Niklas Beisert\\[2ex]
  Institut f\"ur Theoretische Physik\\
  Eidgen\"ossische Technische Hochschule Z\"urich\\
  Wolfgang-Pauli-Strasse 27, 8093 Z\"urich, Switzerland\\[1ex]
  \href{mailto:nbeisert@itp.phys.ethz.ch}
  {\texttt{nbeisert@itp.phys.ethz.ch}}}
\hypersetup{pdfauthor={Niklas Beisert}}
\hypersetup{pdfsubject={Manual for the LaTeX2e Package childdoc}}
\date{30 December 2018, \textsf{v2.0}}
\maketitle

\begin{abstract}\noindent
\textsf{childdoc} is a \LaTeXe{} package
that enables the direct compilation
of document sections included by |\include|
to individual files.
\end{abstract}

\begingroup
\parskip0ex
\tableofcontents
\endgroup

%%%%%%%%%%%%%%%%%%%%%%%%%%%%%%%%%%%%%%%%%%%%%%%%%%%%%%%%%%%%%%%%%%%%%%%%%%%%%%%%
%%%%%%%%%%%%%%%%%%%%%%%%%%%%%%%%%%%%%%%%%%%%%%%%%%%%%%%%%%%%%%%%%%%%%%%%%%%%%%%%
\section{Introduction}

\LaTeX{} provides a mechanism to structure a large document (such as a book)
into a main file and several child files (containing the chapters)
using the |\include| command.
This mechanism is beneficial for documents
which span hundreds of pages in order to
make the source file(s) more manageable.
Moreover, compilation can be restricted to
selected child files by means of the |\includeonly| command.
The latter feature can be used to reduce the compilation time while editing
(this was significantly more useful in the earlier days of \LaTeX{})
or to generate a smaller document which is easier to navigate.
Another application of |\includeonly| is to generate
documents consisting of selected parts of the complete document.

However, there are a few drawbacks of the plain |\include| mechanism:
\begin{itemize}
\item
The child files cannot be compiled on their own,
they can only be compiled via the main file.
A naive editing environment
(such as a text editor with an option
to have the current file processed by \LaTeX)
may require one to switch to the main file before compiling;
attempting to compile the child file produces errors.
\item
The main file must be modified (each time)
to adjust the |\includeonly| command
to the present needs. This easily leaves the main file in a messy state.
\item
The generated document will always carry the filename
of the main document. This is inconvenient if
several child files are to be compiled and
to be kept for distribution.
\end{itemize}

The present package provides a simple interface
to make child files individually compilable by \LaTeX{}.
Compiling a child file then has the same effect as compiling
the main file with an |\includeonly| command
to select the appropriate child.
Moreover the generated document will carry the name of the child
rather than the main file.
This resolves all three above issues.

This feature is meant to make the editing of books,
thesis documents and lecture notes somewhat more convenient.
However, the package can also be used efficiently for
composing a series of documents (such as exercise sheets)
which are typically distributed individually.
It then assists the author in generating the individual documents
(potentially in different versions)
as well as a document containing the collected series.
Another application is in developing style files
or other kinds of included material
where compilation of the style file could redirect
to a sample or test file.

%%%%%%%%%%%%%%%%%%%%%%%%%%%%%%%%%%%%%%%%%%%%%%%%%%%%%%%%%%%%%%%%%%%%%%%%%%%%%%%%
%%%%%%%%%%%%%%%%%%%%%%%%%%%%%%%%%%%%%%%%%%%%%%%%%%%%%%%%%%%%%%%%%%%%%%%%%%%%%%%%
\section{Usage}

First of all, the package \textsf{childdoc} is \emph{not} a standard
\LaTeXe{} |.sty| style file! Therefore it needs to be invoked in
a non-standard way.

%%%%%%%%%%%%%%%%%%%%%%%%%%%%%%%%%%%%%%%%%%%%%%%%%%%%%%%%%%%%%%%%%%%%%%%%%%%%%%%%
\subsection{Included Files}
\label{sec:include}

%%%%%%%%%%%%%%%%%%%%%%%%%%%%%%%%%%%%%%%%
\DescribeMacro{\childdocmain}
To use the package, add the commands
\begin{center}
\begin{tabular}{l}
|\input{childdoc.def}|\\
|\childdocmain{}|\\
\end{tabular}
\end{center}
at the very top of the main \LaTeX{} file,
in particular \emph{before} the |\documentclass| statement!
The argument of |\childdocmain| should be left empty
(but it must be present).

%%%%%%%%%%%%%%%%%%%%%%%%%%%%%%%%%%%%%%%%
\DescribeMacro{\childdocof}
Furthermore, add the commands
\begin{center}
\begin{tabular}{l}
|\input{childdoc.def}|\\
|\childdocof{|\textit{main}|}|\\
\end{tabular}
\end{center}
at the top of every child file \textit{child}
which is included by |\include{|\textit{child}|}|
from within the main file
(or at least for those files to be compiled individually).
The argument \textit{main} must be the filename of the main file.

There are a couple of
considerations in setting up the main and child documents:

%%%%%%%%%%%%%%%%%%%%%%%%%%%%%%%%%%%%%%%%
\paragraph{Restrictions.}

Please note the following restrictions:
\begin{itemize}
\item
|\childdocmain| must be called with one argument \textit{main}
to ensure compatibility with earlier version of the package.
It must either be empty (|\childdocmain{}|)
or precisely match the filename of the main file in which it is specified.
See \secref{sec:detection} for further information.
\item
The filename \textit{main} must be specified without the |.tex| extension.
\item
The filename \textit{main} is case sensitive
(even in case-insensitive file systems)
due to internal string comparison.
\item
The argument \textit{main} should be fully expanded, it cannot be a macro.
\item
Subdirectories and special characters should be avoided in filenames.
\item
The command |\childdocmain{|\textit{main}|}| must be followed by a whitespace.
It should not be followed immediately by another command
or by a comment mark `|%|'.
This is because the \TeX{} parser reads the token immediately following
the argument of |\childdocmain| and puts it
at the beginning of every child section;
however, a white\-space is ignored.
\end{itemize}

%%%%%%%%%%%%%%%%%%%%%%%%%%%%%%%%%%%%%%%%
\paragraph{Content of Main File.}

It is advisable to place all content in the child files included by |\include|.
Any output contained in the main file will appear in all child documents
unless suppressed manually;
it cannot be suppressed automatically by the |\includeonly| directive
and thus should normally be avoided.
A method to include some content in the main file
by means of conditional processing is described in \secref{sec:conditional}.

%%%%%%%%%%%%%%%%%%%%%%%%%%%%%%%%%%%%%%%%
\paragraph{Page Numbering.}

When only a part of the document is compiled,
the appropriate numbering of pages
(as well as other status parameters)
is determined from the |.aux| files.
The latter contain information from previous passes.
However this information needs to propagate through
all intermediate child documents.
Therefore the page numbering in child documents may well
be inconsistent until the complete document is compiled at least once.

A useful (if unconventional) way to always ensure a consistent
page numbering is to restart the numbering in each child document
and denote the pages by `\textit{child}|.|\textit{page}'
where \textit{child} represents the chapter/section number of the child file.
This can be achieved by the command
|\numberwithin{page}{|\textit{child}|}|
of the \textsf{amsmath} package
where \textit{child} can be |chapter| or |section|
depending on the chosen structuring.
Alternatively, one can modify the macro |\thepage| appropriately
and reset the counter |page| at the start of each child file.

%%%%%%%%%%%%%%%%%%%%%%%%%%%%%%%%%%%%%%%%%%%%%%%%%%%%%%%%%%%%%%%%%%%%%%%%%%%%%%%%
\subsection{Conditional Processing}
\label{sec:conditional}

The package provides a mechanism to compile different versions
of a document. To customise the versions further some conditional processing
can come in handy to distinguish which version is being compiled.
The package provides two macros to describe the compilation context:

%%%%%%%%%%%%%%%%%%%%%%%%%%%%%%%%%%%%%%%%
\DescribeMacro{\ifchilddoc}
The conditional |\ifchilddoc| distinguishes between the compilation of
child documents and the main document:
%
\begin{center}
|\ifchilddoc |\textit{child-code}| |[|\||else |\textit{main-code}]| \||fi|
\end{center}

%%%%%%%%%%%%%%%%%%%%%%%%%%%%%%%%%%%%%%%%
\DescribeMacro{\childdocname}
\DescribeMacro{\childdocjob}
The macro |\childdocname| contains the filename (without extension)
of the main or child file being processed.
Note that |\childdocjob| will always contain the name of the main file.

%%%%%%%%%%%%%%%%%%%%%%%%%%%%%%%%%%%%%%%%
\paragraph{Title Page.}

Conditional processing can be used to include a title or banner page
in the main document when proper precautions are taken.
Importantly, the code in the main file should ensure that the page counter
(as well as other status parameters which are stored in the |.aux| files)
takes the same value after the conditional processing.
Otherwise the page numbers may take divergent values
depending on which part is compiled.

For example, a title page could be declared by:
%
\begin{center}
\begin{tabular}{l}
|\ifchilddoc\||else|\\
|\addtocounter{page}{-1}|\\
\textit{code for title page}\\
|\newpage|\\
|\||fi|
\end{tabular}
\end{center}
%
A banner page for the child documents can be generated by:
%
\begin{center}
\begin{tabular}{l}
|\ifchilddoc|\\
|\addtocounter{page}{-1}|\\
\textit{code for banner page}\\
|\newpage|\\
|\||fi|
\end{tabular}
\end{center}
%
Here one could write a message such as:
\begin{center}
|This is the part \childdocname{} of \childdocjob{}.|
\end{center}

%%%%%%%%%%%%%%%%%%%%%%%%%%%%%%%%%%%%%%%%%%%%%%%%%%%%%%%%%%%%%%%%%%%%%%%%%%%%%%%%
\subsection{Flags}
\label{sec:flags}

The package makes it easy to generate different versions
of the main or child documents.
To this end compilation flags can be defined
and assigned different default values.
They will be particularly useful in conjunction
with the forwarding mechanism described in \secref{sec:forward}.

For example, it may be useful to have a flag |\version|
which can be set to |draft| or |final|.
The document source will contain some conditional code
depending on the value of |\version|.
Suppose further, the flag should default to |final| for the main file
and to |draft| for child files
which is a natural assignment for editing the document.
This is achieved by placing the following code
in the preamble of the main document
(below the |\childdocmain| directive):
%
\begin{center}
\begin{tabular}{l}
|\ifchilddoc|\\
|\providecommand{\version}{draft}|\\
|\||else|\\
|\providecommand{\version}{final}|\\
|\||fi|
\end{tabular}
\end{center}
%
The definition by |\providecommand| makes sure
that previous definitions are not overwritten.
Further statements |\providecommand{\version}{...}|
can thus be added before the above code to override it.

For the main file, one might add a line
(between |\childdocmain| and the above block)
%
\begin{center}
|%\ifchilddoc\||else\providecommand{\version}{draft}\||fi|
\end{center}
%
which can be uncommented to produce a draft version.
Likewise one can add a line to the very top of a child file
(above the |\childdocof{|\textit{main}|}| directive)
%
\begin{center}
|%\providecommand{\version}{final}|
\end{center}
%
which can be uncommented to produce the final version of this child document.

%%%%%%%%%%%%%%%%%%%%%%%%%%%%%%%%%%%%%%%%%%%%%%%%%%%%%%%%%%%%%%%%%%%%%%%%%%%%%%%%
\subsection{Forwarding}
\label{sec:forward}

Different versions of the main or child documents
using compilation flags as described in \secref{sec:flags}
can be (permanently) stored in different files
for convenient compilation, viewing and distribution.
To this end, the package defines a command
to pass on compilation to a different file:

%%%%%%%%%%%%%%%%%%%%%%%%%%%%%%%%%%%%%%%%
\DescribeMacro{\childdocforward}
The command |\childdocforward| redirects processing to
another source file:
%
\begin{center}
\begin{tabular}{l}
|\input{childdoc.def}|\\
|\childdocforward[|\textit{main}|]{|\textit{dest}|}|\\
\end{tabular}
\end{center}
%
The argument \textit{dest} is the destination file
(without extension).
It should be the main file or one of the child files.
Note that further \textsf{childdoc} directives
such as |\childdocof| and |\childdocforward|
in the indicated file will be processed in this form.
The optional argument \textit{main}
passes on directly to the main file \textit{main}
while pretending to compile the child \textit{dest}.
This form behaves as if \textit{dest}
issues |\childdocof{|\textit{main}|}| right away,
and no further \textsf{childdoc} directives will be processed.

%%%%%%%%%%%%%%%%%%%%%%%%%%%%%%%%%%%%%%%%
\DescribeMacro{\...prefix}
In the alternative form |\childdocforwardprefix|,
%
\begin{center}
\begin{tabular}{l}
|\input{childdoc.def}|\\
|\childdocforwardprefix[|\textit{main}|]{|\textit{prefix}|}{|\textit{dest}|}|
\end{tabular}
\end{center}
%
the destination file is determined by a pattern
depending on the current file:
To make this work, the current file must be called
`{\textit{prefix}\hspace{0.2em}\textit{suffix}}'
with \textit{prefix} matching precisely the argument.
Processing is then passed on to the file
`{\textit{dest}\hspace{0.2em}\textit{suffix}}'.
Surely, the same effect is achieved by
directly specifying the
argument `{\textit{dest}\hspace{0.2em}\textit{suffix}}'
in the first form.
However, that requires to set up a different file
for each child. With the alternative form of the command
all these files can have exactly the same content
which simplifies setting them up and maintaining them.

For example, the following file |draft.tex|
with a compilation flag |\version| as described in \secref{sec:flags}
compiles the main document as a draft:
%
\begin{center}
\begin{tabular}{l}
|\def\version{draft}|\\
|\input{childdoc.def}|\\
|\childdocforward{|\textit{main}|}|
\end{tabular}
\end{center}
%
Likewise, the following files |final|\textit{nn}|.tex|
compile the final version of the child document
|child|\textit{nn}|.tex|:
%
\begin{center}
\begin{tabular}{l}
|\def\version{final}|\\
|\input{childdoc.def}|\\
|\childdocforwardprefix{final}{child}|
\end{tabular}
\end{center}
%

Note that when several versions of a main file and/or of each child file
are to be generated, it may be convenient to set up a |Makefile| or
shell script to automatise the process.

%%%%%%%%%%%%%%%%%%%%%%%%%%%%%%%%%%%%%%%%%%%%%%%%%%%%%%%%%%%%%%%%%%%%%%%%%%%%%%%%
\subsection{Command Line Processing}
\label{sec:commandline}

The effect of redirection files can also be achieved by invoking
the \LaTeX{} compiler with a more elaborate command line.
Most conveniently this should be done as part
of a shell script or a |Makefile|.

When using \textsf{childdoc} in the main file, the following
command lines effectively perform a redirection
(note that depending on the shell being used,
backslashes may have to be doubled: `|\|' $\to$ `|\\|'):
%
\begin{center}
|... -jobname "|\textit{target}|" |\\|"|[\textit{flags}]%
|\input{childdoc.def}\childdocforward[|\textit{main}|]{|\textit{dest}|}"|
\end{center}
%
Here \textit{target} is the name of the output file,
\textit{main} is the name of the main file
and \textit{dest} is the name of the main or child file to be processed
(all filenames without extensions).
The optional argument \textit{main} can be omitted
if \textit{main} matches \textit{dest}.
Optionally, compilation \textit{flags} can be defined via |\def| commands.
This command line makes the \TeX{} engine believe
it is compiling the file \textit{target}
whose content is specified as the latter parameter.
The provided code then forwards the processing to
\textit{main} or \textit{dest} as described in \secref{sec:forward}.

%%%%%%%%%%%%%%%%%%%%%%%%%%%%%%%%%%%%%%%%%%%%%%%%%%%%%%%%%%%%%%%%%%%%%%%%%%%%%%%%
\subsection{Include by Input}
\label{sec:input}

Including child documents by |\include| has some restrictions by design.
Most notably, the content of a child document always occupies
its own set of pages; pages cannot be shared between child documents.
Usually, this behaviour makes perfect sense
because each child document contain an essential part of the document.
However, in some situations it may be desirable to compose
a document from a collection of parts
without having mandatory page breaks between then.
For this case, the package
provides a mechanism to include parts
by |\input| which can also be processed individually.
However, by construction this mechanism
requires manual handling of the content to be output.

%%%%%%%%%%%%%%%%%%%%%%%%%%%%%%%%%%%%%%%%
\DescribeMacro{\ifchilddocmanual}
The main file should be prepared as usual, see \secref{sec:include}.
However, the document body must make a distinction
between processing of an individual part and of the main document, e.g.:
%
\begin{center}
\begin{tabular}{l}
|\ifchilddocmanual|\\
|\input{\childdocname}|\\
|\||else|\\
\textit{document body with }|\input{|\textit{part}|}|\\
|\||fi|
\end{tabular}
\end{center}
%
The conditional |\ifchilddocmanual| is true whenever
a part to be included by |\input| is being compiled,
and the name of the part is stored in |\childdocname|.

%%%%%%%%%%%%%%%%%%%%%%%%%%%%%%%%%%%%%%%%
\DescribeMacro{\childdocby}
Each part to be included by |\input| should start with:
%
\begin{center}
\begin{tabular}{l}
|\input{childdoc.def}|\\
|\childdocby{|\textit{main}|}|\\
\end{tabular}
\end{center}
%
The directive |\childdocby| is similar to |\childdocof|
described in \secref{sec:include},
but the subsequent selection of content must be done manually.
To that end, both |\ifchilddoc| and |\ifchilddocmanual|
will be true upon processing of a part,
and the name of the part is stored in |\childdocname|.
Note that |\jobname| will be set to the filename of the current part
so that each part receives an individual |.aux| file
that does not interfere with the |.aux| file(s) of the main document.
This behaviour can be altered by the alternative form
|\childdocby[*]{|\textit{main}|}| (with a non-empty optional argument)
which uses the |.aux| file of the main document
by setting |\jobname| to \textit{main}.

%%%%%%%%%%%%%%%%%%%%%%%%%%%%%%%%%%%%%%%%%%%%%%%%%%%%%%%%%%%%%%%%%%%%%%%%%%%%%%%%
\subsection{Driver Development}
\label{sec:driver}

The \textsf{childdoc} mechanism can also be use for the development
of definition files such as \LaTeX{} styles or classes.
This case differs from the above setup with multiple parts
included by |\include| in that no |\includeonly| should be invoked.
This can be achieved by starting the include file
(before |\ProvidesPackage|) with:
%
\begin{center}
\begin{tabular}{l}
|\input{childdoc.def}|\\
|\childdocforward{|\textit{main}|}|\\
\end{tabular}
\end{center}
%
or alternatively with:
%
\begin{center}
\begin{tabular}{l}
|\input{childdoc.def}|\\
|\childdocby{|\textit{main}|}|\\
\end{tabular}
\end{center}
%
Both forms have slightly different effects as described above.
The main file is prepared as usual, see \secref{sec:include}.

%%%%%%%%%%%%%%%%%%%%%%%%%%%%%%%%%%%%%%%%%%%%%%%%%%%%%%%%%%%%%%%%%%%%%%%%%%%%%%%%
\subsection{Legacy Detection}
\label{sec:detection}

The directive |\childdocmain| in the main file can detect
whether the complete document or merely a child is to be compiled
even without using the directive |\childdocof|.
This method is deprecated because it is less robust
and there is no compelling reason to use it;
it is merely provided for backward compatibility
and it may be removed in future versions.

If the detection mechanism is to be used,
it is mandatory to correctly specify
the filename of the main file as the argument of |\childdocmain|:
%
\begin{center}
\begin{tabular}{l}
|\input{childdoc.def}|\\
|\childdocmain{|\textit{main}|}|\\
\end{tabular}
\end{center}
%
If |\jobname| does not match the argument \textit{main} of |\childdocmain|,
it is assumed that |\jobname| points to the child file to be compiled.
When using |\childdocmain| with the main file specified as argument,
it suffices to start a child file
with just |\input{|\textit{main}|}|
without loading of the package and using |\childdocof|.
If instead all processing is done
with the appropriate \textsf{childdoc} directives,
the argument of \textit{main} of |\childdocmain| can be empty.

An alternative version of the command line processing described
in \secref{sec:commandline} using the detection mechanism reads:
%
\begin{center}
|... -jobname "|\textit{target}|" "|[\textit{flags}]%
[|\def\jobname{|\textit{dest}|}|]|\input{|\textit{main}|}"|
\end{center}

%%%%%%%%%%%%%%%%%%%%%%%%%%%%%%%%%%%%%%%%%%%%%%%%%%%%%%%%%%%%%%%%%%%%%%%%%%%%%%%%
\subsection{Manual Code}
\label{sec:manual}

In case one cannot be certain whether the definitions file |childdoc.def|
is installed on the target \TeX{} distribution
and one prefers not to ship it,
it is conceivable to paste a few relevant commands into the sources.

To that end, drop all statements |\input{childdoc.def}|
and perform the replacements as outlined below.
Instead of |\childdocmain{|\textit{main}|}| add the following code
to the top of the main file:
%
\begin{center}
\begin{tabular}{l}
|\||ifdefined\childdocname\endinput\||fi\newif\ifchilddoc|\\
|\edef\childdocname{\scantokens\expandafter{\jobname\noexpand}}|\\
|\def\childdocmain{|\textit{main}|}\||ifx\childdocmain\childdocname\||else|\\
|\childdoctrue\includeonly{\childdocname}\let\jobname\childdocmain\||fi|\\
\end{tabular}
\end{center}
%
Instead of |\childdocof{|\textit{main}|}| just include the main file
at the top of each child file:
%
\begin{center}
|\input{|\textit{main}|}|
\end{center}
%
A simple redirection |\childdocforward{|\textit{dest}|}| is achieved by:
%
\begin{center}
|\def\jobname{|\textit{dest}|}\input{\jobname}|
\end{center}
%
The redirection with prefix
|\childdocforwardprefix[|\textit{prefix}|]{|\textit{dest}|}|
is accomplished by:
%
\begin{center}
\begin{tabular}{l}
|{\edef\jobname{\scantokens\expandafter{\jobname\noexpand}}|\\
|\def\redirectjob |\textit{prefix}|#1~~~{\gdef\jobname{|\textit{dest}|#1}}|\\
|\expandafter\redirectjob\jobname~~~}\input{\jobname}|
\end{tabular}
\end{center}

In an alternative approach,
child documents can be compiled by a specific command line
without additional code or specific definitions:
%
\begin{center}
|... -jobname "|\textit{target}|" "|[\textit{flags}]%
|\includeonly{|\textit{dest}|}\input{|\textit{main}|}"|
\end{center}
%

%%%%%%%%%%%%%%%%%%%%%%%%%%%%%%%%%%%%%%%%%%%%%%%%%%%%%%%%%%%%%%%%%%%%%%%%%%%%%%%%
%%%%%%%%%%%%%%%%%%%%%%%%%%%%%%%%%%%%%%%%%%%%%%%%%%%%%%%%%%%%%%%%%%%%%%%%%%%%%%%%
\section{Information}

%%%%%%%%%%%%%%%%%%%%%%%%%%%%%%%%%%%%%%%%%%%%%%%%%%%%%%%%%%%%%%%%%%%%%%%%%%%%%%%%
\subsection{Copyright}

Copyright \copyright{} 2017--2018 Niklas Beisert

This work may be distributed and/or modified under the
conditions of the \LaTeX{} Project Public License, either version 1.3
of this license or (at your option) any later version.
The latest version of this license is in
  \url{http://www.latex-project.org/lppl.txt}
and version 1.3 or later is part of all distributions of \LaTeX{}
version 2005/12/01 or later.

This work has the LPPL maintenance status `maintained'.

The Current Maintainer of this work is Niklas Beisert.

This work consists of the files |README.txt|, |childdoc.ins| and |childdoc.dtx|
as well as the derived files |childdoc.def|, |cdocsamp.tex|
with |cdocsch1.tex|, |cdocsch2.tex|, |cdocspt3.tex|, |cdocspt4.tex|,
|cdocsdrf.tex|, |cdocsfn1.tex|, |cdocsfn2.tex|
as well as |childdoc.pdf|.

%%%%%%%%%%%%%%%%%%%%%%%%%%%%%%%%%%%%%%%%%%%%%%%%%%%%%%%%%%%%%%%%%%%%%%%%%%%%%%%%
\subsection{Files and Installation}

The package consists of the files:
%
\begin{center}
\begin{tabular}{ll}
    |README.txt|   & readme file \\
    |childdoc.ins| & installation file \\
    |childdoc.dtx| & source file \\
    |childdoc.def| & definition file \\
    |cdocsamp.tex| & sample main file \\
    |cdocsch1.tex| & sample include file \\
    |cdocsch2.tex| & sample include file \\
    |cdocspt3.tex| & sample part file \\
    |cdocspt4.tex| & sample part file \\
    |cdocsdrf.tex| & sample redirection file \\
    |cdocsfn1.tex| & sample redirection file \\
    |cdocsfn2.tex| & sample redirection file \\
    |childdoc.pdf| & manual
\end{tabular}
\end{center}
%
The distribution consists of the files
|README.txt|, |childdoc.ins| and |childdoc.dtx|.
%
\begin{itemize}
\item
Run (pdf)\LaTeX{} on |childdoc.dtx|
to compile the manual |childdoc.pdf| (this file).
\item
Run \LaTeX{} on |childdoc.ins| to create the definitions file |childdoc.def|
and the sample |cdocsamp.tex| with include files
|cdocsch1.tex|, |cdocsch2.tex|, |cdocspt3.tex|, |cdocspt4.tex|,
|cdocsdrf.tex|, |cdocsfn1.tex|, |cdocsfn2.tex|.
Then copy the file |childdoc.def| to an appropriate directory of your \LaTeX{}
distribution, e.g.\ \textit{texmf-root}|/tex/latex/childdoc|.
\end{itemize}

%%%%%%%%%%%%%%%%%%%%%%%%%%%%%%%%%%%%%%%%%%%%%%%%%%%%%%%%%%%%%%%%%%%%%%%%%%%%%%%%
\subsection{Related CTAN Packages}

There are several other packages which offer a similar functionality:
%
\begin{itemize}
\item
The packages
\href{http://ctan.org/pkg/docmute}{\textsf{docmute}},
\href{http://ctan.org/pkg/includex}{\textsf{includex}} and
\href{http://ctan.org/pkg/standalone}{\textsf{standalone}}
provide commands to include only the document body of
a child file thus allowing both files to be compiled individually.
\item
The packages \href{http://ctan.org/pkg/subdocs}{\textsf{subdocs}}
and \href{http://ctan.org/pkg/subfiles}{\textsf{subfiles}}
provide structures in which the main and child documents can be
encapsulated and allowing them to be compiled individually.
The inclusion mechanism is different from the conventional |\include|.
\item
The package \href{http://ctan.org/pkg/combine}{\textsf{combine}}
is an elaborate solution to combine several documents into one.
\end{itemize}
%
See also the CTAN topic \href{http://ctan.org/topic/subdocs}{\textsf{subdocs}}
for further related packages.
The present package differs from the above solutions in that
a document structure constructed with the conventional |\include| mechanism
just needs two extra commands at the top of every file
such that all constituent files can be compiled individually.

%%%%%%%%%%%%%%%%%%%%%%%%%%%%%%%%%%%%%%%%%%%%%%%%%%%%%%%%%%%%%%%%%%%%%%%%%%%%%%%%
%\subsection{Feature Suggestions}
%
%The following is a list of features which may be useful for future
%versions of this package:
%%
%\begin{itemize}
%\item
%\ldots
%\end{itemize}

%%%%%%%%%%%%%%%%%%%%%%%%%%%%%%%%%%%%%%%%%%%%%%%%%%%%%%%%%%%%%%%%%%%%%%%%%%%%%%%%
\subsection{Revision History}

%%%%%%%%%%%%%%%%%%%%%%%%%%%%%%%%%%%%%%%%
\paragraph{v2.0:} 2018/12/30

\begin{itemize}
\item
immediate forward processing
\item
added |\childdocby| mechanism
\item
manual restructured
\end{itemize}

%%%%%%%%%%%%%%%%%%%%%%%%%%%%%%%%%%%%%%%%
\paragraph{v1.6:} 2018/01/17

\begin{itemize}
\item
application for development of include files
\item
corrections to manual
\end{itemize}

%%%%%%%%%%%%%%%%%%%%%%%%%%%%%%%%%%%%%%%%
\paragraph{v1.5:} 2017/05/21

\begin{itemize}
\item
more complete structuring introduced
\item
|\childdocof| introduced
\item
|\childdoc| renamed to |\childdocmain|
\item
|\childredirect| renamed to |\childdocforward| and |\childdocforwardprefix|
and functionality expanded
\end{itemize}

%%%%%%%%%%%%%%%%%%%%%%%%%%%%%%%%%%%%%%%%
\paragraph{v1.0:} 2017/04/27

\begin{itemize}
\item
manual and install package
\item
first version published on CTAN
\end{itemize}

%%%%%%%%%%%%%%%%%%%%%%%%%%%%%%%%%%%%%%%%
\paragraph{v0.6:} 2017/04/26

\begin{itemize}
\item
redirection mechanism added
\end{itemize}

%%%%%%%%%%%%%%%%%%%%%%%%%%%%%%%%%%%%%%%%
\paragraph{v0.5:} 2017/04/26

\begin{itemize}
\item
functionality in definition file
\end{itemize}


%%%%%%%%%%%%%%%%%%%%%%%%%%%%%%%%%%%%%%%%%%%%%%%%%%%%%%%%%%%%%%%%%%%%%%%%%%%%%%%%
%%%%%%%%%%%%%%%%%%%%%%%%%%%%%%%%%%%%%%%%%%%%%%%%%%%%%%%%%%%%%%%%%%%%%%%%%%%%%%%%
%%%%%%%%%%%%%%%%%%%%%%%%%%%%%%%%%%%%%%%%%%%%%%%%%%%%%%%%%%%%%%%%%%%%%%%%%%%%%%%%
\appendix

\settowidth\MacroIndent{\rmfamily\scriptsize 000\ }

 \DocInput{childdoc.dtx}

\end{document}
%</driver>
% \fi
%
% %%%%%%%%%%%%%%%%%%%%%%%%%%%%%%%%%%%%%%%%%%%%%%%%%%%%%%%%%%%%%%%%%%%%%%%%%%%%%%
% %%%%%%%%%%%%%%%%%%%%%%%%%%%%%%%%%%%%%%%%%%%%%%%%%%%%%%%%%%%%%%%%%%%%%%%%%%%%%%
% \section{Sample}
%\iffalse
%<*samplemain>
%\fi
%
% The following presents a sample document
% with two chapters, two parts, a title page,
% a compile flag as well as three forwarding files to set the flag.
% It consists of eight |.tex| files:
% \begin{center}
% \begin{tabular}{ll}
% |cdocsamp.tex|&main file\\
% |cdocsch1.tex|&include file for chapter 1\\
% |cdocsch2.tex|&include file for chapter 2\\
% |cdocspt3.tex|&include file for part 3\\
% |cdocspt4.tex|&include file for part 4\\
% |cdocsdrf.tex|&forwarding file for main file in draft mode\\
% |cdocsfi1.tex|&forwarding file for final version of chapter 1\\
% |cdocsfi2.tex|&forwarding file for final version of chapter 2\\
% \end{tabular}
% \end{center}
% Each of the eight files can be compiled directly by the \LaTeX{} compiler.
%
% %%%%%%%%%%%%%%%%%%%%%%%%%%%%%%%%%%%%%%
% \paragraph{Main File.}
%
% The main file is called |cdocsamp.tex|.
%
% Load the \textsf{childdoc} definitions and
% declare the filename for the main document:
%    \begin{macrocode}
\input{childdoc.def}
\childdocmain{}
%    \end{macrocode}

% Optional override for |\version| flag:
%    \begin{macrocode}
%%\ifchilddoc\else\providecommand{\version}{draft}\fi
%    \end{macrocode}

% Define the default values for the |\version| flag
% (|final| for the main file and |draft| for childs):
%    \begin{macrocode}
\ifchilddoc
\providecommand{\version}{draft}
\else
\providecommand{\version}{final}
\fi
%    \end{macrocode}

% Load the standard document class:
%    \begin{macrocode}
\documentclass[12pt]{article}
%    \end{macrocode}

% Start the document body:
%    \begin{macrocode}
\begin{document}
%    \end{macrocode}

% Declare a title page.
% Print title, part of document being processed and version flag:
%    \begin{macrocode}
\addtocounter{page}{-1}
\begin{center}
{\LARGE\bfseries{}childdoc example\par}
\vspace{1cm}
\ifchilddoc
\ifchilddocmanual part\else chapter\fi:
`\childdocname' of `\childdocjob'\par
\else
main document: `\childdocjob'\par
\fi
version: \version\par
\end{center}
\newpage
%    \end{macrocode}

% Manually include selected file,
% otherwise process as usual:
%    \begin{macrocode}
\ifchilddocmanual
\section*{part `\childdocname'}
\input{\childdocname}
\else
%    \end{macrocode}

% Include the two chapters:
%    \begin{macrocode}
\include{cdocsch1}
\include{cdocsch2}
%    \end{macrocode}

% Include the two parts unless only chapters should be displayed:
%    \begin{macrocode}
\ifchilddoc\else
\section{part three}
\input{cdocspt3}
\section{part four}
\input{cdocspt4}
\fi
%    \end{macrocode}

% Process as usual until here:
%    \begin{macrocode}
\fi
%    \end{macrocode}

% End of document body:
%    \begin{macrocode}
\end{document}
%    \end{macrocode}
%\iffalse
%</samplemain>
%\fi
%
% %%%%%%%%%%%%%%%%%%%%%%%%%%%%%%%%%%%%%%
% \paragraph{Chapter Include Files.}
%
% The include files are called |cdocsch1.tex| and |cdocsch2.tex|.
%
%\iffalse
%<*samplechap1|samplechap2>
%\fi

% Optional override for |\version| flag:
%    \begin{macrocode}
%%\providecommand{\version}{final}
%    \end{macrocode}

% Include the main document:
%    \begin{macrocode}
\input{childdoc.def}
\childdocof{cdocsamp}
%    \end{macrocode}

%\iffalse
%</samplechap1|samplechap2>
%\fi
%
%\iffalse
%<*samplechap1>
%\fi
% Some text for chapter 1:
%    \begin{macrocode}
\section{one}
some text in chapter one
%    \end{macrocode}

%\iffalse
%</samplechap1>
%\fi
% Some text for chapter 2:
%\iffalse
%<*samplechap2>
%\fi
%    \begin{macrocode}
\section{two}
more text in chapter two
%    \end{macrocode}

%\iffalse
%</samplechap2>
%\fi
%
% %%%%%%%%%%%%%%%%%%%%%%%%%%%%%%%%%%%%%%
% \paragraph{Part Include Files.}
%
% The include files are called |cdocspt3.tex| and |cdocspt4.tex|.
%
%\iffalse
%<*samplepart3|samplepart4>
%\fi

% Optional override for |\version| flag:
%    \begin{macrocode}
%%\providecommand{\version}{final}
%    \end{macrocode}

% Include the main document:
%    \begin{macrocode}
\input{childdoc.def}
\childdocby{cdocsamp}
%    \end{macrocode}

%\iffalse
%</samplepart3|samplepart4>
%\fi
%
%\iffalse
%<*samplepart3>
%\fi
% Some text for part 3:
%    \begin{macrocode}
some text in part three
%    \end{macrocode}

%\iffalse
%</samplepart3>
%\fi
% Some text for part 4:
%\iffalse
%<*samplepart4>
%\fi
%    \begin{macrocode}
more text in part four
%    \end{macrocode}

%\iffalse
%</samplepart4>
%\fi
%
% %%%%%%%%%%%%%%%%%%%%%%%%%%%%%%%%%%%%%%
% \paragraph{Forwarding for a Complete Draft.}
%
% The following forwarding file |cdocsdrf.tex|
% compiles the main document in draft mode:
%\iffalse
%<*sampledraft>
%\fi
%    \begin{macrocode}
\def\version{draft}
\input{childdoc.def}
\childdocforward{cdocsamp}
%    \end{macrocode}

%\iffalse
%</sampledraft>
%\fi
%
% %%%%%%%%%%%%%%%%%%%%%%%%%%%%%%%%%%%%%%
% \paragraph{Forwarding for Final Version of the Chapters.}
%
% The following forwarding files |cdocsfn1.tex| and |cdocsfn2.tex|
% (with identical content)
% compile the final versions of the child documents
% |cdocsch1.tex| and |cdocsch2.tex|, respectively:
%\iffalse
%<*samplefinal>
%\fi
%    \begin{macrocode}
\def\version{final}
\input{childdoc.def}
\childdocforwardprefix[cdocsamp]{cdocsfn}{cdocsch}
%    \end{macrocode}

%\iffalse
%</samplefinal>
%\fi
%
% %%%%%%%%%%%%%%%%%%%%%%%%%%%%%%%%%%%%%%
% \paragraph{Command Line Processing.}
%
% The following three command lines generate the output files
% |cdocscld|, |cdocscl1| and |cdocscl2|
% which should be identical to
% |cdocsdrf|, |cdocsch1| and |cdocsfn2|, respectively:
% \begin{center}
% \begin{tabular}{l}
% |latex -jobname cdocscld \|\\
% |  "\def\version{draft}\input{childdoc.def}\childdocforward{cdocsamp}"|\\
% |latex -jobname cdocscl1 \|\\
% |  "\input{childdoc.def}\childdocforward[cdocsamp]{cdocsch1}"|\\
% |latex -jobname cdocscl2 \|\\
% |  "\def\version{final}\input{childdoc.def}\childdocforward{cdocsch2}"|
% \end{tabular}
% \end{center}
% Note that the trailing backslash on each first line
% merely continues the input to the second line
% (for convenient cut ant paste).
% Furthermore, the command |latex| can be replaced by any
% of its alternative versions such as |pdflatex|.
%
% %%%%%%%%%%%%%%%%%%%%%%%%%%%%%%%%%%%%%%%%%%%%%%%%%%%%%%%%%%%%%%%%%%%%%%%%%%%%%%
% %%%%%%%%%%%%%%%%%%%%%%%%%%%%%%%%%%%%%%%%%%%%%%%%%%%%%%%%%%%%%%%%%%%%%%%%%%%%%%
% \section{Implementation}
%\iffalse
%<*package>
%\fi
%
% This section describes the definitions file |childdoc.def|.

% The definitions cannot be loaded using |\usepackage| or |\RequirePackage|
% which has a mechanism to prevent loading a style file more than once.
% When loading the definitions by means of |\input|
% multiple instances have to be prevented manually:
%\iffalse
%This code needs to be before the `\ProvidesFile' directive
%which is defined at the beginning of this file.
%Therefore it is also placed there and commented out here.
%</package>
%<*discard>
%\fi
%    \begin{macrocode}
\ifdefined\childdocmain\endinput\fi
%    \end{macrocode}
%\iffalse
%</discard>
%<*package>
%\fi
%
% \macro{\ifchilddoc}
% \macro{\ifchilddocmanual}
% The conditional |\ifchilddoc| tells whether a
% child (true) or main (false) document is being compiled.
% The conditional |\ifchilddocmanual| tells whether
% the |\includeonly| mechanism is used (false) or
% the selection of child files must be performed manually (true).
% The definitions initialise to false:
%    \begin{macrocode}
\newif\ifchilddoc
\newif\ifchilddocmanual
%    \end{macrocode}

% \macro{\childdocname}
% \macro{\childdocjob}
% The macro |\childdocname| stores the name of the main document
% to be compiled. The macro |\childdocjob| stores the name of
% the document on which the \LaTeX{} compiler was originally invoked.
% The content of |\jobname| cannot be compared
% to filenames specified in the source due to different catcodes.
% The following code rescans |\jobname|, stores the result
% in |\childdocname| and saves a copy in |\childdocjob|:
%    \begin{macrocode}
\edef\childdocname{\scantokens\expandafter{\jobname\noexpand}}
\let\childdocjob\childdocname
%    \end{macrocode}

% \macro{\childdocdisable}
% The macro |\childdocdisable| prevents the main file
% from being processed more than once.
% At this stage, the main document command |\childdocmain|
% is assumed to be called once again where it should do nothing.
% Any subsequent call to it should prevent
% a secondary processing of the main document
% It overwrites the forwarding commands
% |\childdocof| and |\childdocforward|
% with empty macros to prevent further inclusions of the main document:
%    \begin{macrocode}
\newcommand{\childdocdisable}
{
  \renewcommand{\childdocmain}[1]{\renewcommand{\childdocmain}[1]{\endinput}}
  \renewcommand{\childdocof}[1]{}
  \renewcommand{\childdocby}[2][]{}
  \renewcommand{\childdocforward}[2][]{}
  \renewcommand{\childdocdisable}{}
}
%    \end{macrocode}

% \macro{\childdocmain}
% The macro |\childdocmain| is to be called at the top of the main file
% with nothing or the main filename (without extension) as argument.
% First, it breaks loops.
% If the argument is not empty and does not match |\childdocname|
% (which is set by the first inclusion of |childdoc.def|),
% |\ifchilddoc| is set to true, |\includeonly| is applied to the child file
% and |\jobname| is set to the main file
% (for proper handling of |.aux| files):
%    \begin{macrocode}
\newcommand{\childdocmain}[1]
{
  \childdocdisable\childdocmain{}
  \if?#1?\else
    \begingroup
      \def\childdoctmp{#1}
      \ifx\childdoctmp\childdocname
        \def\childdoctmp{}
      \else
        \def\childdoctmp
        {
          \childdoctrue
          \includeonly{\childdocname}
          \def\childdocjob{#1}
          \def\jobname{#1}
        }
      \fi
      \expandafter
    \endgroup
    \childdoctmp
  \fi
}
%    \end{macrocode}

% \macro{\childdocof}
% The command |\childdocof| redirects
% compilation to the main file |#1|.
%    \begin{macrocode}
\newcommand{\childdocof}[1]
{
  \childdocdisable
  \childdoctrue
  \includeonly{\childdocname}
  \def\jobname{#1}
  \def\childdocjob{#1}
  \input{#1}
}
%    \end{macrocode}

% \macro{\childdocby}
% The command |\childdocby| ....
%    \begin{macrocode}
\newcommand{\childdocby}[2][]
{
  \childdocdisable
  \childdoctrue
  \childdocmanualtrue
  \if?#1?\else
    \def\jobname{#2}
  \fi
  \def\childdocjob{#2}
  \input{#2}
  \endinput
}
%    \end{macrocode}

% \macro{\childdocforward}
% The command |\childdocforward| redirects
% compilation to the main file or
% (if the optional argument is given) a child file.
% Parameters are set as if the main file
% or a child file starting with |\childdocof| was compiled.
% Then compilation is handed over to the main file:
%    \begin{macrocode}
\newcommand{\childdocforward}[2][]
{
  \begingroup
    \if?#1?
      \def\childdoctmp
      {
        \def\childdocname{#2}
        \def\childdocjob{#2}
        \def\jobname{#2}
        \input{#2}
        \endinput
      }
    \else
      \def\childdoctmp
      {
        \childdocdisable
        \def\childdocname{#2}
        \childdoctrue
        \includeonly{#2}
        \def\childdocjob{#1}
        \def\jobname{#1}
        \input{#1}
        \endinput
      }
    \fi
    \expandafter
  \endgroup
  \childdoctmp
}
%    \end{macrocode}

% \macro{\childdocforwardprefix}
% The command |\childdocforwardprefix| redirects
% compilation to the main or a child file by means of a pattern.
% The prefix |#1| in the current filename is replaced by |#2|
% and the suffix of the current filename is kept
% (it is assumed that the filename does not contain the substring `|~~~|'
% which is used as a delimiter).
% Compilation is handed over to the new file by |\childdocforward|:
%    \begin{macrocode}
\newcommand{\childdocforwardprefix}[3][]
{
  \begingroup
    \def\childdocextract #2##1~~~{\def\childdoctmp{\childdocforward[#1]{#3##1}}}
    \expandafter\childdocextract\childdocname~~~
    \expandafter
  \endgroup
  \childdoctmp
}
%    \end{macrocode}

% \macro{\childdoc}
% The deprecated macro |\childdoc| is a legacy version of |\childdocmain|:
%    \begin{macrocode}
\newcommand{\childdoc}{\childdocmain}
%    \end{macrocode}

% \macro{\childdocredirect}
% The deprecated macro |\childdocredirect| is a legacy version
% of |\childdocforward| and |\childdocforwardprefix|:
%    \begin{macrocode}
\newcommand{\childdocredirect}[2][]
{
  \begingroup
    \if?#1?
      \def\childdoctmp{\childdocforward{#2}}
    \else
      \def\childdoctmp{\childdocforwardprefix{#1}{#2}}
    \fi
    \expandafter
  \endgroup
  \childdoctmp
}
%    \end{macrocode}

%\iffalse
%</package>
%\fi
%
\endinput
\childdocforward[cdocsamp]{cdocsch1}"|\\
% |latex -jobname cdocscl2 \|\\
% |  "\def\version{final}% \iffalse
%
% childdoc.dtx Copyright (C) 2017-2018 Niklas Beisert
%
% This work may be distributed and/or modified under the
% conditions of the LaTeX Project Public License, either version 1.3
% of this license or (at your option) any later version.
% The latest version of this license is in
%   http://www.latex-project.org/lppl.txt
% and version 1.3 or later is part of all distributions of LaTeX
% version 2005/12/01 or later.
%
% This work has the LPPL maintenance status `maintained'.
%
% The Current Maintainer of this work is Niklas Beisert.
%
% This work consists of the files childdoc.dtx and childdoc.ins
% and the derived files childdoc.def and cdocsamp.tex with
% cdocsch1.tex, cdocsch2.tex, cdocsdrf.tex, cdocsfn1.tex, cdocsfn2.tex.
%
%<package>\ifdefined\childdocmain\endinput\fi
%<package>\ProvidesFile{childdoc.def}[2018/12/30 v2.0 child document driver]
%<samplemain>\ProvidesFile{cdocsamp.tex}[2018/12/30 v2.0 sample for childdoc]
%<*driver>
%\ProvidesFile{childdoc.drv}[2018/12/30 v2.0 childdoc reference manual file]
\PassOptionsToClass{10pt,a4paper}{article}
\documentclass{ltxdoc}

\usepackage[margin=35mm]{geometry}
\usepackage{hyperref}
\usepackage{hyperxmp}
\usepackage[usenames]{color}

\hypersetup{colorlinks=true}
\hypersetup{pdfstartview=FitH}
\hypersetup{pdfpagemode=UseNone}
\hypersetup{pdfsource={}}
\hypersetup{pdflang={en-UK}}
\hypersetup{pdfcopyright={Copyright 2017-2018 Niklas Beisert.
  This work may be distributed and/or modified under the
  conditions of the LaTeX Project Public License, either version 1.3
  of this license or (at your option) any later version.}}
\hypersetup{pdflicenseurl={http://www.latex-project.org/lppl.txt}}
\hypersetup{pdfcontactaddress={ETH Zurich, ITP, HIT K,
  Wolfgang-Pauli-Strasse 27}}
\hypersetup{pdfcontactpostcode={8093}}
\hypersetup{pdfcontactcity={Zurich}}
\hypersetup{pdfcontactcountry={Switzerland}}
\hypersetup{pdfcontactemail={nbeisert@itp.phys.ethz.ch}}
\hypersetup{pdfcontacturl={http://people.phys.ethz.ch/\xmptilde nbeisert/}}

\newcommand{\secref}[1]{\hyperref[#1]{section \ref*{#1}}}

\parskip1ex
\parindent0pt
\let\olditemize\itemize
\def\itemize{\olditemize\parskip0pt}

\begin{document}

\title{The \textsf{childdoc} Package}
\hypersetup{pdftitle={The childdoc Package}}
\author{Niklas Beisert\\[2ex]
  Institut f\"ur Theoretische Physik\\
  Eidgen\"ossische Technische Hochschule Z\"urich\\
  Wolfgang-Pauli-Strasse 27, 8093 Z\"urich, Switzerland\\[1ex]
  \href{mailto:nbeisert@itp.phys.ethz.ch}
  {\texttt{nbeisert@itp.phys.ethz.ch}}}
\hypersetup{pdfauthor={Niklas Beisert}}
\hypersetup{pdfsubject={Manual for the LaTeX2e Package childdoc}}
\date{30 December 2018, \textsf{v2.0}}
\maketitle

\begin{abstract}\noindent
\textsf{childdoc} is a \LaTeXe{} package
that enables the direct compilation
of document sections included by |\include|
to individual files.
\end{abstract}

\begingroup
\parskip0ex
\tableofcontents
\endgroup

%%%%%%%%%%%%%%%%%%%%%%%%%%%%%%%%%%%%%%%%%%%%%%%%%%%%%%%%%%%%%%%%%%%%%%%%%%%%%%%%
%%%%%%%%%%%%%%%%%%%%%%%%%%%%%%%%%%%%%%%%%%%%%%%%%%%%%%%%%%%%%%%%%%%%%%%%%%%%%%%%
\section{Introduction}

\LaTeX{} provides a mechanism to structure a large document (such as a book)
into a main file and several child files (containing the chapters)
using the |\include| command.
This mechanism is beneficial for documents
which span hundreds of pages in order to
make the source file(s) more manageable.
Moreover, compilation can be restricted to
selected child files by means of the |\includeonly| command.
The latter feature can be used to reduce the compilation time while editing
(this was significantly more useful in the earlier days of \LaTeX{})
or to generate a smaller document which is easier to navigate.
Another application of |\includeonly| is to generate
documents consisting of selected parts of the complete document.

However, there are a few drawbacks of the plain |\include| mechanism:
\begin{itemize}
\item
The child files cannot be compiled on their own,
they can only be compiled via the main file.
A naive editing environment
(such as a text editor with an option
to have the current file processed by \LaTeX)
may require one to switch to the main file before compiling;
attempting to compile the child file produces errors.
\item
The main file must be modified (each time)
to adjust the |\includeonly| command
to the present needs. This easily leaves the main file in a messy state.
\item
The generated document will always carry the filename
of the main document. This is inconvenient if
several child files are to be compiled and
to be kept for distribution.
\end{itemize}

The present package provides a simple interface
to make child files individually compilable by \LaTeX{}.
Compiling a child file then has the same effect as compiling
the main file with an |\includeonly| command
to select the appropriate child.
Moreover the generated document will carry the name of the child
rather than the main file.
This resolves all three above issues.

This feature is meant to make the editing of books,
thesis documents and lecture notes somewhat more convenient.
However, the package can also be used efficiently for
composing a series of documents (such as exercise sheets)
which are typically distributed individually.
It then assists the author in generating the individual documents
(potentially in different versions)
as well as a document containing the collected series.
Another application is in developing style files
or other kinds of included material
where compilation of the style file could redirect
to a sample or test file.

%%%%%%%%%%%%%%%%%%%%%%%%%%%%%%%%%%%%%%%%%%%%%%%%%%%%%%%%%%%%%%%%%%%%%%%%%%%%%%%%
%%%%%%%%%%%%%%%%%%%%%%%%%%%%%%%%%%%%%%%%%%%%%%%%%%%%%%%%%%%%%%%%%%%%%%%%%%%%%%%%
\section{Usage}

First of all, the package \textsf{childdoc} is \emph{not} a standard
\LaTeXe{} |.sty| style file! Therefore it needs to be invoked in
a non-standard way.

%%%%%%%%%%%%%%%%%%%%%%%%%%%%%%%%%%%%%%%%%%%%%%%%%%%%%%%%%%%%%%%%%%%%%%%%%%%%%%%%
\subsection{Included Files}
\label{sec:include}

%%%%%%%%%%%%%%%%%%%%%%%%%%%%%%%%%%%%%%%%
\DescribeMacro{\childdocmain}
To use the package, add the commands
\begin{center}
\begin{tabular}{l}
|\input{childdoc.def}|\\
|\childdocmain{}|\\
\end{tabular}
\end{center}
at the very top of the main \LaTeX{} file,
in particular \emph{before} the |\documentclass| statement!
The argument of |\childdocmain| should be left empty
(but it must be present).

%%%%%%%%%%%%%%%%%%%%%%%%%%%%%%%%%%%%%%%%
\DescribeMacro{\childdocof}
Furthermore, add the commands
\begin{center}
\begin{tabular}{l}
|\input{childdoc.def}|\\
|\childdocof{|\textit{main}|}|\\
\end{tabular}
\end{center}
at the top of every child file \textit{child}
which is included by |\include{|\textit{child}|}|
from within the main file
(or at least for those files to be compiled individually).
The argument \textit{main} must be the filename of the main file.

There are a couple of
considerations in setting up the main and child documents:

%%%%%%%%%%%%%%%%%%%%%%%%%%%%%%%%%%%%%%%%
\paragraph{Restrictions.}

Please note the following restrictions:
\begin{itemize}
\item
|\childdocmain| must be called with one argument \textit{main}
to ensure compatibility with earlier version of the package.
It must either be empty (|\childdocmain{}|)
or precisely match the filename of the main file in which it is specified.
See \secref{sec:detection} for further information.
\item
The filename \textit{main} must be specified without the |.tex| extension.
\item
The filename \textit{main} is case sensitive
(even in case-insensitive file systems)
due to internal string comparison.
\item
The argument \textit{main} should be fully expanded, it cannot be a macro.
\item
Subdirectories and special characters should be avoided in filenames.
\item
The command |\childdocmain{|\textit{main}|}| must be followed by a whitespace.
It should not be followed immediately by another command
or by a comment mark `|%|'.
This is because the \TeX{} parser reads the token immediately following
the argument of |\childdocmain| and puts it
at the beginning of every child section;
however, a white\-space is ignored.
\end{itemize}

%%%%%%%%%%%%%%%%%%%%%%%%%%%%%%%%%%%%%%%%
\paragraph{Content of Main File.}

It is advisable to place all content in the child files included by |\include|.
Any output contained in the main file will appear in all child documents
unless suppressed manually;
it cannot be suppressed automatically by the |\includeonly| directive
and thus should normally be avoided.
A method to include some content in the main file
by means of conditional processing is described in \secref{sec:conditional}.

%%%%%%%%%%%%%%%%%%%%%%%%%%%%%%%%%%%%%%%%
\paragraph{Page Numbering.}

When only a part of the document is compiled,
the appropriate numbering of pages
(as well as other status parameters)
is determined from the |.aux| files.
The latter contain information from previous passes.
However this information needs to propagate through
all intermediate child documents.
Therefore the page numbering in child documents may well
be inconsistent until the complete document is compiled at least once.

A useful (if unconventional) way to always ensure a consistent
page numbering is to restart the numbering in each child document
and denote the pages by `\textit{child}|.|\textit{page}'
where \textit{child} represents the chapter/section number of the child file.
This can be achieved by the command
|\numberwithin{page}{|\textit{child}|}|
of the \textsf{amsmath} package
where \textit{child} can be |chapter| or |section|
depending on the chosen structuring.
Alternatively, one can modify the macro |\thepage| appropriately
and reset the counter |page| at the start of each child file.

%%%%%%%%%%%%%%%%%%%%%%%%%%%%%%%%%%%%%%%%%%%%%%%%%%%%%%%%%%%%%%%%%%%%%%%%%%%%%%%%
\subsection{Conditional Processing}
\label{sec:conditional}

The package provides a mechanism to compile different versions
of a document. To customise the versions further some conditional processing
can come in handy to distinguish which version is being compiled.
The package provides two macros to describe the compilation context:

%%%%%%%%%%%%%%%%%%%%%%%%%%%%%%%%%%%%%%%%
\DescribeMacro{\ifchilddoc}
The conditional |\ifchilddoc| distinguishes between the compilation of
child documents and the main document:
%
\begin{center}
|\ifchilddoc |\textit{child-code}| |[|\||else |\textit{main-code}]| \||fi|
\end{center}

%%%%%%%%%%%%%%%%%%%%%%%%%%%%%%%%%%%%%%%%
\DescribeMacro{\childdocname}
\DescribeMacro{\childdocjob}
The macro |\childdocname| contains the filename (without extension)
of the main or child file being processed.
Note that |\childdocjob| will always contain the name of the main file.

%%%%%%%%%%%%%%%%%%%%%%%%%%%%%%%%%%%%%%%%
\paragraph{Title Page.}

Conditional processing can be used to include a title or banner page
in the main document when proper precautions are taken.
Importantly, the code in the main file should ensure that the page counter
(as well as other status parameters which are stored in the |.aux| files)
takes the same value after the conditional processing.
Otherwise the page numbers may take divergent values
depending on which part is compiled.

For example, a title page could be declared by:
%
\begin{center}
\begin{tabular}{l}
|\ifchilddoc\||else|\\
|\addtocounter{page}{-1}|\\
\textit{code for title page}\\
|\newpage|\\
|\||fi|
\end{tabular}
\end{center}
%
A banner page for the child documents can be generated by:
%
\begin{center}
\begin{tabular}{l}
|\ifchilddoc|\\
|\addtocounter{page}{-1}|\\
\textit{code for banner page}\\
|\newpage|\\
|\||fi|
\end{tabular}
\end{center}
%
Here one could write a message such as:
\begin{center}
|This is the part \childdocname{} of \childdocjob{}.|
\end{center}

%%%%%%%%%%%%%%%%%%%%%%%%%%%%%%%%%%%%%%%%%%%%%%%%%%%%%%%%%%%%%%%%%%%%%%%%%%%%%%%%
\subsection{Flags}
\label{sec:flags}

The package makes it easy to generate different versions
of the main or child documents.
To this end compilation flags can be defined
and assigned different default values.
They will be particularly useful in conjunction
with the forwarding mechanism described in \secref{sec:forward}.

For example, it may be useful to have a flag |\version|
which can be set to |draft| or |final|.
The document source will contain some conditional code
depending on the value of |\version|.
Suppose further, the flag should default to |final| for the main file
and to |draft| for child files
which is a natural assignment for editing the document.
This is achieved by placing the following code
in the preamble of the main document
(below the |\childdocmain| directive):
%
\begin{center}
\begin{tabular}{l}
|\ifchilddoc|\\
|\providecommand{\version}{draft}|\\
|\||else|\\
|\providecommand{\version}{final}|\\
|\||fi|
\end{tabular}
\end{center}
%
The definition by |\providecommand| makes sure
that previous definitions are not overwritten.
Further statements |\providecommand{\version}{...}|
can thus be added before the above code to override it.

For the main file, one might add a line
(between |\childdocmain| and the above block)
%
\begin{center}
|%\ifchilddoc\||else\providecommand{\version}{draft}\||fi|
\end{center}
%
which can be uncommented to produce a draft version.
Likewise one can add a line to the very top of a child file
(above the |\childdocof{|\textit{main}|}| directive)
%
\begin{center}
|%\providecommand{\version}{final}|
\end{center}
%
which can be uncommented to produce the final version of this child document.

%%%%%%%%%%%%%%%%%%%%%%%%%%%%%%%%%%%%%%%%%%%%%%%%%%%%%%%%%%%%%%%%%%%%%%%%%%%%%%%%
\subsection{Forwarding}
\label{sec:forward}

Different versions of the main or child documents
using compilation flags as described in \secref{sec:flags}
can be (permanently) stored in different files
for convenient compilation, viewing and distribution.
To this end, the package defines a command
to pass on compilation to a different file:

%%%%%%%%%%%%%%%%%%%%%%%%%%%%%%%%%%%%%%%%
\DescribeMacro{\childdocforward}
The command |\childdocforward| redirects processing to
another source file:
%
\begin{center}
\begin{tabular}{l}
|\input{childdoc.def}|\\
|\childdocforward[|\textit{main}|]{|\textit{dest}|}|\\
\end{tabular}
\end{center}
%
The argument \textit{dest} is the destination file
(without extension).
It should be the main file or one of the child files.
Note that further \textsf{childdoc} directives
such as |\childdocof| and |\childdocforward|
in the indicated file will be processed in this form.
The optional argument \textit{main}
passes on directly to the main file \textit{main}
while pretending to compile the child \textit{dest}.
This form behaves as if \textit{dest}
issues |\childdocof{|\textit{main}|}| right away,
and no further \textsf{childdoc} directives will be processed.

%%%%%%%%%%%%%%%%%%%%%%%%%%%%%%%%%%%%%%%%
\DescribeMacro{\...prefix}
In the alternative form |\childdocforwardprefix|,
%
\begin{center}
\begin{tabular}{l}
|\input{childdoc.def}|\\
|\childdocforwardprefix[|\textit{main}|]{|\textit{prefix}|}{|\textit{dest}|}|
\end{tabular}
\end{center}
%
the destination file is determined by a pattern
depending on the current file:
To make this work, the current file must be called
`{\textit{prefix}\hspace{0.2em}\textit{suffix}}'
with \textit{prefix} matching precisely the argument.
Processing is then passed on to the file
`{\textit{dest}\hspace{0.2em}\textit{suffix}}'.
Surely, the same effect is achieved by
directly specifying the
argument `{\textit{dest}\hspace{0.2em}\textit{suffix}}'
in the first form.
However, that requires to set up a different file
for each child. With the alternative form of the command
all these files can have exactly the same content
which simplifies setting them up and maintaining them.

For example, the following file |draft.tex|
with a compilation flag |\version| as described in \secref{sec:flags}
compiles the main document as a draft:
%
\begin{center}
\begin{tabular}{l}
|\def\version{draft}|\\
|\input{childdoc.def}|\\
|\childdocforward{|\textit{main}|}|
\end{tabular}
\end{center}
%
Likewise, the following files |final|\textit{nn}|.tex|
compile the final version of the child document
|child|\textit{nn}|.tex|:
%
\begin{center}
\begin{tabular}{l}
|\def\version{final}|\\
|\input{childdoc.def}|\\
|\childdocforwardprefix{final}{child}|
\end{tabular}
\end{center}
%

Note that when several versions of a main file and/or of each child file
are to be generated, it may be convenient to set up a |Makefile| or
shell script to automatise the process.

%%%%%%%%%%%%%%%%%%%%%%%%%%%%%%%%%%%%%%%%%%%%%%%%%%%%%%%%%%%%%%%%%%%%%%%%%%%%%%%%
\subsection{Command Line Processing}
\label{sec:commandline}

The effect of redirection files can also be achieved by invoking
the \LaTeX{} compiler with a more elaborate command line.
Most conveniently this should be done as part
of a shell script or a |Makefile|.

When using \textsf{childdoc} in the main file, the following
command lines effectively perform a redirection
(note that depending on the shell being used,
backslashes may have to be doubled: `|\|' $\to$ `|\\|'):
%
\begin{center}
|... -jobname "|\textit{target}|" |\\|"|[\textit{flags}]%
|\input{childdoc.def}\childdocforward[|\textit{main}|]{|\textit{dest}|}"|
\end{center}
%
Here \textit{target} is the name of the output file,
\textit{main} is the name of the main file
and \textit{dest} is the name of the main or child file to be processed
(all filenames without extensions).
The optional argument \textit{main} can be omitted
if \textit{main} matches \textit{dest}.
Optionally, compilation \textit{flags} can be defined via |\def| commands.
This command line makes the \TeX{} engine believe
it is compiling the file \textit{target}
whose content is specified as the latter parameter.
The provided code then forwards the processing to
\textit{main} or \textit{dest} as described in \secref{sec:forward}.

%%%%%%%%%%%%%%%%%%%%%%%%%%%%%%%%%%%%%%%%%%%%%%%%%%%%%%%%%%%%%%%%%%%%%%%%%%%%%%%%
\subsection{Include by Input}
\label{sec:input}

Including child documents by |\include| has some restrictions by design.
Most notably, the content of a child document always occupies
its own set of pages; pages cannot be shared between child documents.
Usually, this behaviour makes perfect sense
because each child document contain an essential part of the document.
However, in some situations it may be desirable to compose
a document from a collection of parts
without having mandatory page breaks between then.
For this case, the package
provides a mechanism to include parts
by |\input| which can also be processed individually.
However, by construction this mechanism
requires manual handling of the content to be output.

%%%%%%%%%%%%%%%%%%%%%%%%%%%%%%%%%%%%%%%%
\DescribeMacro{\ifchilddocmanual}
The main file should be prepared as usual, see \secref{sec:include}.
However, the document body must make a distinction
between processing of an individual part and of the main document, e.g.:
%
\begin{center}
\begin{tabular}{l}
|\ifchilddocmanual|\\
|\input{\childdocname}|\\
|\||else|\\
\textit{document body with }|\input{|\textit{part}|}|\\
|\||fi|
\end{tabular}
\end{center}
%
The conditional |\ifchilddocmanual| is true whenever
a part to be included by |\input| is being compiled,
and the name of the part is stored in |\childdocname|.

%%%%%%%%%%%%%%%%%%%%%%%%%%%%%%%%%%%%%%%%
\DescribeMacro{\childdocby}
Each part to be included by |\input| should start with:
%
\begin{center}
\begin{tabular}{l}
|\input{childdoc.def}|\\
|\childdocby{|\textit{main}|}|\\
\end{tabular}
\end{center}
%
The directive |\childdocby| is similar to |\childdocof|
described in \secref{sec:include},
but the subsequent selection of content must be done manually.
To that end, both |\ifchilddoc| and |\ifchilddocmanual|
will be true upon processing of a part,
and the name of the part is stored in |\childdocname|.
Note that |\jobname| will be set to the filename of the current part
so that each part receives an individual |.aux| file
that does not interfere with the |.aux| file(s) of the main document.
This behaviour can be altered by the alternative form
|\childdocby[*]{|\textit{main}|}| (with a non-empty optional argument)
which uses the |.aux| file of the main document
by setting |\jobname| to \textit{main}.

%%%%%%%%%%%%%%%%%%%%%%%%%%%%%%%%%%%%%%%%%%%%%%%%%%%%%%%%%%%%%%%%%%%%%%%%%%%%%%%%
\subsection{Driver Development}
\label{sec:driver}

The \textsf{childdoc} mechanism can also be use for the development
of definition files such as \LaTeX{} styles or classes.
This case differs from the above setup with multiple parts
included by |\include| in that no |\includeonly| should be invoked.
This can be achieved by starting the include file
(before |\ProvidesPackage|) with:
%
\begin{center}
\begin{tabular}{l}
|\input{childdoc.def}|\\
|\childdocforward{|\textit{main}|}|\\
\end{tabular}
\end{center}
%
or alternatively with:
%
\begin{center}
\begin{tabular}{l}
|\input{childdoc.def}|\\
|\childdocby{|\textit{main}|}|\\
\end{tabular}
\end{center}
%
Both forms have slightly different effects as described above.
The main file is prepared as usual, see \secref{sec:include}.

%%%%%%%%%%%%%%%%%%%%%%%%%%%%%%%%%%%%%%%%%%%%%%%%%%%%%%%%%%%%%%%%%%%%%%%%%%%%%%%%
\subsection{Legacy Detection}
\label{sec:detection}

The directive |\childdocmain| in the main file can detect
whether the complete document or merely a child is to be compiled
even without using the directive |\childdocof|.
This method is deprecated because it is less robust
and there is no compelling reason to use it;
it is merely provided for backward compatibility
and it may be removed in future versions.

If the detection mechanism is to be used,
it is mandatory to correctly specify
the filename of the main file as the argument of |\childdocmain|:
%
\begin{center}
\begin{tabular}{l}
|\input{childdoc.def}|\\
|\childdocmain{|\textit{main}|}|\\
\end{tabular}
\end{center}
%
If |\jobname| does not match the argument \textit{main} of |\childdocmain|,
it is assumed that |\jobname| points to the child file to be compiled.
When using |\childdocmain| with the main file specified as argument,
it suffices to start a child file
with just |\input{|\textit{main}|}|
without loading of the package and using |\childdocof|.
If instead all processing is done
with the appropriate \textsf{childdoc} directives,
the argument of \textit{main} of |\childdocmain| can be empty.

An alternative version of the command line processing described
in \secref{sec:commandline} using the detection mechanism reads:
%
\begin{center}
|... -jobname "|\textit{target}|" "|[\textit{flags}]%
[|\def\jobname{|\textit{dest}|}|]|\input{|\textit{main}|}"|
\end{center}

%%%%%%%%%%%%%%%%%%%%%%%%%%%%%%%%%%%%%%%%%%%%%%%%%%%%%%%%%%%%%%%%%%%%%%%%%%%%%%%%
\subsection{Manual Code}
\label{sec:manual}

In case one cannot be certain whether the definitions file |childdoc.def|
is installed on the target \TeX{} distribution
and one prefers not to ship it,
it is conceivable to paste a few relevant commands into the sources.

To that end, drop all statements |\input{childdoc.def}|
and perform the replacements as outlined below.
Instead of |\childdocmain{|\textit{main}|}| add the following code
to the top of the main file:
%
\begin{center}
\begin{tabular}{l}
|\||ifdefined\childdocname\endinput\||fi\newif\ifchilddoc|\\
|\edef\childdocname{\scantokens\expandafter{\jobname\noexpand}}|\\
|\def\childdocmain{|\textit{main}|}\||ifx\childdocmain\childdocname\||else|\\
|\childdoctrue\includeonly{\childdocname}\let\jobname\childdocmain\||fi|\\
\end{tabular}
\end{center}
%
Instead of |\childdocof{|\textit{main}|}| just include the main file
at the top of each child file:
%
\begin{center}
|\input{|\textit{main}|}|
\end{center}
%
A simple redirection |\childdocforward{|\textit{dest}|}| is achieved by:
%
\begin{center}
|\def\jobname{|\textit{dest}|}\input{\jobname}|
\end{center}
%
The redirection with prefix
|\childdocforwardprefix[|\textit{prefix}|]{|\textit{dest}|}|
is accomplished by:
%
\begin{center}
\begin{tabular}{l}
|{\edef\jobname{\scantokens\expandafter{\jobname\noexpand}}|\\
|\def\redirectjob |\textit{prefix}|#1~~~{\gdef\jobname{|\textit{dest}|#1}}|\\
|\expandafter\redirectjob\jobname~~~}\input{\jobname}|
\end{tabular}
\end{center}

In an alternative approach,
child documents can be compiled by a specific command line
without additional code or specific definitions:
%
\begin{center}
|... -jobname "|\textit{target}|" "|[\textit{flags}]%
|\includeonly{|\textit{dest}|}\input{|\textit{main}|}"|
\end{center}
%

%%%%%%%%%%%%%%%%%%%%%%%%%%%%%%%%%%%%%%%%%%%%%%%%%%%%%%%%%%%%%%%%%%%%%%%%%%%%%%%%
%%%%%%%%%%%%%%%%%%%%%%%%%%%%%%%%%%%%%%%%%%%%%%%%%%%%%%%%%%%%%%%%%%%%%%%%%%%%%%%%
\section{Information}

%%%%%%%%%%%%%%%%%%%%%%%%%%%%%%%%%%%%%%%%%%%%%%%%%%%%%%%%%%%%%%%%%%%%%%%%%%%%%%%%
\subsection{Copyright}

Copyright \copyright{} 2017--2018 Niklas Beisert

This work may be distributed and/or modified under the
conditions of the \LaTeX{} Project Public License, either version 1.3
of this license or (at your option) any later version.
The latest version of this license is in
  \url{http://www.latex-project.org/lppl.txt}
and version 1.3 or later is part of all distributions of \LaTeX{}
version 2005/12/01 or later.

This work has the LPPL maintenance status `maintained'.

The Current Maintainer of this work is Niklas Beisert.

This work consists of the files |README.txt|, |childdoc.ins| and |childdoc.dtx|
as well as the derived files |childdoc.def|, |cdocsamp.tex|
with |cdocsch1.tex|, |cdocsch2.tex|, |cdocspt3.tex|, |cdocspt4.tex|,
|cdocsdrf.tex|, |cdocsfn1.tex|, |cdocsfn2.tex|
as well as |childdoc.pdf|.

%%%%%%%%%%%%%%%%%%%%%%%%%%%%%%%%%%%%%%%%%%%%%%%%%%%%%%%%%%%%%%%%%%%%%%%%%%%%%%%%
\subsection{Files and Installation}

The package consists of the files:
%
\begin{center}
\begin{tabular}{ll}
    |README.txt|   & readme file \\
    |childdoc.ins| & installation file \\
    |childdoc.dtx| & source file \\
    |childdoc.def| & definition file \\
    |cdocsamp.tex| & sample main file \\
    |cdocsch1.tex| & sample include file \\
    |cdocsch2.tex| & sample include file \\
    |cdocspt3.tex| & sample part file \\
    |cdocspt4.tex| & sample part file \\
    |cdocsdrf.tex| & sample redirection file \\
    |cdocsfn1.tex| & sample redirection file \\
    |cdocsfn2.tex| & sample redirection file \\
    |childdoc.pdf| & manual
\end{tabular}
\end{center}
%
The distribution consists of the files
|README.txt|, |childdoc.ins| and |childdoc.dtx|.
%
\begin{itemize}
\item
Run (pdf)\LaTeX{} on |childdoc.dtx|
to compile the manual |childdoc.pdf| (this file).
\item
Run \LaTeX{} on |childdoc.ins| to create the definitions file |childdoc.def|
and the sample |cdocsamp.tex| with include files
|cdocsch1.tex|, |cdocsch2.tex|, |cdocspt3.tex|, |cdocspt4.tex|,
|cdocsdrf.tex|, |cdocsfn1.tex|, |cdocsfn2.tex|.
Then copy the file |childdoc.def| to an appropriate directory of your \LaTeX{}
distribution, e.g.\ \textit{texmf-root}|/tex/latex/childdoc|.
\end{itemize}

%%%%%%%%%%%%%%%%%%%%%%%%%%%%%%%%%%%%%%%%%%%%%%%%%%%%%%%%%%%%%%%%%%%%%%%%%%%%%%%%
\subsection{Related CTAN Packages}

There are several other packages which offer a similar functionality:
%
\begin{itemize}
\item
The packages
\href{http://ctan.org/pkg/docmute}{\textsf{docmute}},
\href{http://ctan.org/pkg/includex}{\textsf{includex}} and
\href{http://ctan.org/pkg/standalone}{\textsf{standalone}}
provide commands to include only the document body of
a child file thus allowing both files to be compiled individually.
\item
The packages \href{http://ctan.org/pkg/subdocs}{\textsf{subdocs}}
and \href{http://ctan.org/pkg/subfiles}{\textsf{subfiles}}
provide structures in which the main and child documents can be
encapsulated and allowing them to be compiled individually.
The inclusion mechanism is different from the conventional |\include|.
\item
The package \href{http://ctan.org/pkg/combine}{\textsf{combine}}
is an elaborate solution to combine several documents into one.
\end{itemize}
%
See also the CTAN topic \href{http://ctan.org/topic/subdocs}{\textsf{subdocs}}
for further related packages.
The present package differs from the above solutions in that
a document structure constructed with the conventional |\include| mechanism
just needs two extra commands at the top of every file
such that all constituent files can be compiled individually.

%%%%%%%%%%%%%%%%%%%%%%%%%%%%%%%%%%%%%%%%%%%%%%%%%%%%%%%%%%%%%%%%%%%%%%%%%%%%%%%%
%\subsection{Feature Suggestions}
%
%The following is a list of features which may be useful for future
%versions of this package:
%%
%\begin{itemize}
%\item
%\ldots
%\end{itemize}

%%%%%%%%%%%%%%%%%%%%%%%%%%%%%%%%%%%%%%%%%%%%%%%%%%%%%%%%%%%%%%%%%%%%%%%%%%%%%%%%
\subsection{Revision History}

%%%%%%%%%%%%%%%%%%%%%%%%%%%%%%%%%%%%%%%%
\paragraph{v2.0:} 2018/12/30

\begin{itemize}
\item
immediate forward processing
\item
added |\childdocby| mechanism
\item
manual restructured
\end{itemize}

%%%%%%%%%%%%%%%%%%%%%%%%%%%%%%%%%%%%%%%%
\paragraph{v1.6:} 2018/01/17

\begin{itemize}
\item
application for development of include files
\item
corrections to manual
\end{itemize}

%%%%%%%%%%%%%%%%%%%%%%%%%%%%%%%%%%%%%%%%
\paragraph{v1.5:} 2017/05/21

\begin{itemize}
\item
more complete structuring introduced
\item
|\childdocof| introduced
\item
|\childdoc| renamed to |\childdocmain|
\item
|\childredirect| renamed to |\childdocforward| and |\childdocforwardprefix|
and functionality expanded
\end{itemize}

%%%%%%%%%%%%%%%%%%%%%%%%%%%%%%%%%%%%%%%%
\paragraph{v1.0:} 2017/04/27

\begin{itemize}
\item
manual and install package
\item
first version published on CTAN
\end{itemize}

%%%%%%%%%%%%%%%%%%%%%%%%%%%%%%%%%%%%%%%%
\paragraph{v0.6:} 2017/04/26

\begin{itemize}
\item
redirection mechanism added
\end{itemize}

%%%%%%%%%%%%%%%%%%%%%%%%%%%%%%%%%%%%%%%%
\paragraph{v0.5:} 2017/04/26

\begin{itemize}
\item
functionality in definition file
\end{itemize}


%%%%%%%%%%%%%%%%%%%%%%%%%%%%%%%%%%%%%%%%%%%%%%%%%%%%%%%%%%%%%%%%%%%%%%%%%%%%%%%%
%%%%%%%%%%%%%%%%%%%%%%%%%%%%%%%%%%%%%%%%%%%%%%%%%%%%%%%%%%%%%%%%%%%%%%%%%%%%%%%%
%%%%%%%%%%%%%%%%%%%%%%%%%%%%%%%%%%%%%%%%%%%%%%%%%%%%%%%%%%%%%%%%%%%%%%%%%%%%%%%%
\appendix

\settowidth\MacroIndent{\rmfamily\scriptsize 000\ }

 \DocInput{childdoc.dtx}

\end{document}
%</driver>
% \fi
%
% %%%%%%%%%%%%%%%%%%%%%%%%%%%%%%%%%%%%%%%%%%%%%%%%%%%%%%%%%%%%%%%%%%%%%%%%%%%%%%
% %%%%%%%%%%%%%%%%%%%%%%%%%%%%%%%%%%%%%%%%%%%%%%%%%%%%%%%%%%%%%%%%%%%%%%%%%%%%%%
% \section{Sample}
%\iffalse
%<*samplemain>
%\fi
%
% The following presents a sample document
% with two chapters, two parts, a title page,
% a compile flag as well as three forwarding files to set the flag.
% It consists of eight |.tex| files:
% \begin{center}
% \begin{tabular}{ll}
% |cdocsamp.tex|&main file\\
% |cdocsch1.tex|&include file for chapter 1\\
% |cdocsch2.tex|&include file for chapter 2\\
% |cdocspt3.tex|&include file for part 3\\
% |cdocspt4.tex|&include file for part 4\\
% |cdocsdrf.tex|&forwarding file for main file in draft mode\\
% |cdocsfi1.tex|&forwarding file for final version of chapter 1\\
% |cdocsfi2.tex|&forwarding file for final version of chapter 2\\
% \end{tabular}
% \end{center}
% Each of the eight files can be compiled directly by the \LaTeX{} compiler.
%
% %%%%%%%%%%%%%%%%%%%%%%%%%%%%%%%%%%%%%%
% \paragraph{Main File.}
%
% The main file is called |cdocsamp.tex|.
%
% Load the \textsf{childdoc} definitions and
% declare the filename for the main document:
%    \begin{macrocode}
\input{childdoc.def}
\childdocmain{}
%    \end{macrocode}

% Optional override for |\version| flag:
%    \begin{macrocode}
%%\ifchilddoc\else\providecommand{\version}{draft}\fi
%    \end{macrocode}

% Define the default values for the |\version| flag
% (|final| for the main file and |draft| for childs):
%    \begin{macrocode}
\ifchilddoc
\providecommand{\version}{draft}
\else
\providecommand{\version}{final}
\fi
%    \end{macrocode}

% Load the standard document class:
%    \begin{macrocode}
\documentclass[12pt]{article}
%    \end{macrocode}

% Start the document body:
%    \begin{macrocode}
\begin{document}
%    \end{macrocode}

% Declare a title page.
% Print title, part of document being processed and version flag:
%    \begin{macrocode}
\addtocounter{page}{-1}
\begin{center}
{\LARGE\bfseries{}childdoc example\par}
\vspace{1cm}
\ifchilddoc
\ifchilddocmanual part\else chapter\fi:
`\childdocname' of `\childdocjob'\par
\else
main document: `\childdocjob'\par
\fi
version: \version\par
\end{center}
\newpage
%    \end{macrocode}

% Manually include selected file,
% otherwise process as usual:
%    \begin{macrocode}
\ifchilddocmanual
\section*{part `\childdocname'}
\input{\childdocname}
\else
%    \end{macrocode}

% Include the two chapters:
%    \begin{macrocode}
\include{cdocsch1}
\include{cdocsch2}
%    \end{macrocode}

% Include the two parts unless only chapters should be displayed:
%    \begin{macrocode}
\ifchilddoc\else
\section{part three}
\input{cdocspt3}
\section{part four}
\input{cdocspt4}
\fi
%    \end{macrocode}

% Process as usual until here:
%    \begin{macrocode}
\fi
%    \end{macrocode}

% End of document body:
%    \begin{macrocode}
\end{document}
%    \end{macrocode}
%\iffalse
%</samplemain>
%\fi
%
% %%%%%%%%%%%%%%%%%%%%%%%%%%%%%%%%%%%%%%
% \paragraph{Chapter Include Files.}
%
% The include files are called |cdocsch1.tex| and |cdocsch2.tex|.
%
%\iffalse
%<*samplechap1|samplechap2>
%\fi

% Optional override for |\version| flag:
%    \begin{macrocode}
%%\providecommand{\version}{final}
%    \end{macrocode}

% Include the main document:
%    \begin{macrocode}
\input{childdoc.def}
\childdocof{cdocsamp}
%    \end{macrocode}

%\iffalse
%</samplechap1|samplechap2>
%\fi
%
%\iffalse
%<*samplechap1>
%\fi
% Some text for chapter 1:
%    \begin{macrocode}
\section{one}
some text in chapter one
%    \end{macrocode}

%\iffalse
%</samplechap1>
%\fi
% Some text for chapter 2:
%\iffalse
%<*samplechap2>
%\fi
%    \begin{macrocode}
\section{two}
more text in chapter two
%    \end{macrocode}

%\iffalse
%</samplechap2>
%\fi
%
% %%%%%%%%%%%%%%%%%%%%%%%%%%%%%%%%%%%%%%
% \paragraph{Part Include Files.}
%
% The include files are called |cdocspt3.tex| and |cdocspt4.tex|.
%
%\iffalse
%<*samplepart3|samplepart4>
%\fi

% Optional override for |\version| flag:
%    \begin{macrocode}
%%\providecommand{\version}{final}
%    \end{macrocode}

% Include the main document:
%    \begin{macrocode}
\input{childdoc.def}
\childdocby{cdocsamp}
%    \end{macrocode}

%\iffalse
%</samplepart3|samplepart4>
%\fi
%
%\iffalse
%<*samplepart3>
%\fi
% Some text for part 3:
%    \begin{macrocode}
some text in part three
%    \end{macrocode}

%\iffalse
%</samplepart3>
%\fi
% Some text for part 4:
%\iffalse
%<*samplepart4>
%\fi
%    \begin{macrocode}
more text in part four
%    \end{macrocode}

%\iffalse
%</samplepart4>
%\fi
%
% %%%%%%%%%%%%%%%%%%%%%%%%%%%%%%%%%%%%%%
% \paragraph{Forwarding for a Complete Draft.}
%
% The following forwarding file |cdocsdrf.tex|
% compiles the main document in draft mode:
%\iffalse
%<*sampledraft>
%\fi
%    \begin{macrocode}
\def\version{draft}
\input{childdoc.def}
\childdocforward{cdocsamp}
%    \end{macrocode}

%\iffalse
%</sampledraft>
%\fi
%
% %%%%%%%%%%%%%%%%%%%%%%%%%%%%%%%%%%%%%%
% \paragraph{Forwarding for Final Version of the Chapters.}
%
% The following forwarding files |cdocsfn1.tex| and |cdocsfn2.tex|
% (with identical content)
% compile the final versions of the child documents
% |cdocsch1.tex| and |cdocsch2.tex|, respectively:
%\iffalse
%<*samplefinal>
%\fi
%    \begin{macrocode}
\def\version{final}
\input{childdoc.def}
\childdocforwardprefix[cdocsamp]{cdocsfn}{cdocsch}
%    \end{macrocode}

%\iffalse
%</samplefinal>
%\fi
%
% %%%%%%%%%%%%%%%%%%%%%%%%%%%%%%%%%%%%%%
% \paragraph{Command Line Processing.}
%
% The following three command lines generate the output files
% |cdocscld|, |cdocscl1| and |cdocscl2|
% which should be identical to
% |cdocsdrf|, |cdocsch1| and |cdocsfn2|, respectively:
% \begin{center}
% \begin{tabular}{l}
% |latex -jobname cdocscld \|\\
% |  "\def\version{draft}\input{childdoc.def}\childdocforward{cdocsamp}"|\\
% |latex -jobname cdocscl1 \|\\
% |  "\input{childdoc.def}\childdocforward[cdocsamp]{cdocsch1}"|\\
% |latex -jobname cdocscl2 \|\\
% |  "\def\version{final}\input{childdoc.def}\childdocforward{cdocsch2}"|
% \end{tabular}
% \end{center}
% Note that the trailing backslash on each first line
% merely continues the input to the second line
% (for convenient cut ant paste).
% Furthermore, the command |latex| can be replaced by any
% of its alternative versions such as |pdflatex|.
%
% %%%%%%%%%%%%%%%%%%%%%%%%%%%%%%%%%%%%%%%%%%%%%%%%%%%%%%%%%%%%%%%%%%%%%%%%%%%%%%
% %%%%%%%%%%%%%%%%%%%%%%%%%%%%%%%%%%%%%%%%%%%%%%%%%%%%%%%%%%%%%%%%%%%%%%%%%%%%%%
% \section{Implementation}
%\iffalse
%<*package>
%\fi
%
% This section describes the definitions file |childdoc.def|.

% The definitions cannot be loaded using |\usepackage| or |\RequirePackage|
% which has a mechanism to prevent loading a style file more than once.
% When loading the definitions by means of |\input|
% multiple instances have to be prevented manually:
%\iffalse
%This code needs to be before the `\ProvidesFile' directive
%which is defined at the beginning of this file.
%Therefore it is also placed there and commented out here.
%</package>
%<*discard>
%\fi
%    \begin{macrocode}
\ifdefined\childdocmain\endinput\fi
%    \end{macrocode}
%\iffalse
%</discard>
%<*package>
%\fi
%
% \macro{\ifchilddoc}
% \macro{\ifchilddocmanual}
% The conditional |\ifchilddoc| tells whether a
% child (true) or main (false) document is being compiled.
% The conditional |\ifchilddocmanual| tells whether
% the |\includeonly| mechanism is used (false) or
% the selection of child files must be performed manually (true).
% The definitions initialise to false:
%    \begin{macrocode}
\newif\ifchilddoc
\newif\ifchilddocmanual
%    \end{macrocode}

% \macro{\childdocname}
% \macro{\childdocjob}
% The macro |\childdocname| stores the name of the main document
% to be compiled. The macro |\childdocjob| stores the name of
% the document on which the \LaTeX{} compiler was originally invoked.
% The content of |\jobname| cannot be compared
% to filenames specified in the source due to different catcodes.
% The following code rescans |\jobname|, stores the result
% in |\childdocname| and saves a copy in |\childdocjob|:
%    \begin{macrocode}
\edef\childdocname{\scantokens\expandafter{\jobname\noexpand}}
\let\childdocjob\childdocname
%    \end{macrocode}

% \macro{\childdocdisable}
% The macro |\childdocdisable| prevents the main file
% from being processed more than once.
% At this stage, the main document command |\childdocmain|
% is assumed to be called once again where it should do nothing.
% Any subsequent call to it should prevent
% a secondary processing of the main document
% It overwrites the forwarding commands
% |\childdocof| and |\childdocforward|
% with empty macros to prevent further inclusions of the main document:
%    \begin{macrocode}
\newcommand{\childdocdisable}
{
  \renewcommand{\childdocmain}[1]{\renewcommand{\childdocmain}[1]{\endinput}}
  \renewcommand{\childdocof}[1]{}
  \renewcommand{\childdocby}[2][]{}
  \renewcommand{\childdocforward}[2][]{}
  \renewcommand{\childdocdisable}{}
}
%    \end{macrocode}

% \macro{\childdocmain}
% The macro |\childdocmain| is to be called at the top of the main file
% with nothing or the main filename (without extension) as argument.
% First, it breaks loops.
% If the argument is not empty and does not match |\childdocname|
% (which is set by the first inclusion of |childdoc.def|),
% |\ifchilddoc| is set to true, |\includeonly| is applied to the child file
% and |\jobname| is set to the main file
% (for proper handling of |.aux| files):
%    \begin{macrocode}
\newcommand{\childdocmain}[1]
{
  \childdocdisable\childdocmain{}
  \if?#1?\else
    \begingroup
      \def\childdoctmp{#1}
      \ifx\childdoctmp\childdocname
        \def\childdoctmp{}
      \else
        \def\childdoctmp
        {
          \childdoctrue
          \includeonly{\childdocname}
          \def\childdocjob{#1}
          \def\jobname{#1}
        }
      \fi
      \expandafter
    \endgroup
    \childdoctmp
  \fi
}
%    \end{macrocode}

% \macro{\childdocof}
% The command |\childdocof| redirects
% compilation to the main file |#1|.
%    \begin{macrocode}
\newcommand{\childdocof}[1]
{
  \childdocdisable
  \childdoctrue
  \includeonly{\childdocname}
  \def\jobname{#1}
  \def\childdocjob{#1}
  \input{#1}
}
%    \end{macrocode}

% \macro{\childdocby}
% The command |\childdocby| ....
%    \begin{macrocode}
\newcommand{\childdocby}[2][]
{
  \childdocdisable
  \childdoctrue
  \childdocmanualtrue
  \if?#1?\else
    \def\jobname{#2}
  \fi
  \def\childdocjob{#2}
  \input{#2}
  \endinput
}
%    \end{macrocode}

% \macro{\childdocforward}
% The command |\childdocforward| redirects
% compilation to the main file or
% (if the optional argument is given) a child file.
% Parameters are set as if the main file
% or a child file starting with |\childdocof| was compiled.
% Then compilation is handed over to the main file:
%    \begin{macrocode}
\newcommand{\childdocforward}[2][]
{
  \begingroup
    \if?#1?
      \def\childdoctmp
      {
        \def\childdocname{#2}
        \def\childdocjob{#2}
        \def\jobname{#2}
        \input{#2}
        \endinput
      }
    \else
      \def\childdoctmp
      {
        \childdocdisable
        \def\childdocname{#2}
        \childdoctrue
        \includeonly{#2}
        \def\childdocjob{#1}
        \def\jobname{#1}
        \input{#1}
        \endinput
      }
    \fi
    \expandafter
  \endgroup
  \childdoctmp
}
%    \end{macrocode}

% \macro{\childdocforwardprefix}
% The command |\childdocforwardprefix| redirects
% compilation to the main or a child file by means of a pattern.
% The prefix |#1| in the current filename is replaced by |#2|
% and the suffix of the current filename is kept
% (it is assumed that the filename does not contain the substring `|~~~|'
% which is used as a delimiter).
% Compilation is handed over to the new file by |\childdocforward|:
%    \begin{macrocode}
\newcommand{\childdocforwardprefix}[3][]
{
  \begingroup
    \def\childdocextract #2##1~~~{\def\childdoctmp{\childdocforward[#1]{#3##1}}}
    \expandafter\childdocextract\childdocname~~~
    \expandafter
  \endgroup
  \childdoctmp
}
%    \end{macrocode}

% \macro{\childdoc}
% The deprecated macro |\childdoc| is a legacy version of |\childdocmain|:
%    \begin{macrocode}
\newcommand{\childdoc}{\childdocmain}
%    \end{macrocode}

% \macro{\childdocredirect}
% The deprecated macro |\childdocredirect| is a legacy version
% of |\childdocforward| and |\childdocforwardprefix|:
%    \begin{macrocode}
\newcommand{\childdocredirect}[2][]
{
  \begingroup
    \if?#1?
      \def\childdoctmp{\childdocforward{#2}}
    \else
      \def\childdoctmp{\childdocforwardprefix{#1}{#2}}
    \fi
    \expandafter
  \endgroup
  \childdoctmp
}
%    \end{macrocode}

%\iffalse
%</package>
%\fi
%
\endinput
\childdocforward{cdocsch2}"|
% \end{tabular}
% \end{center}
% Note that the trailing backslash on each first line
% merely continues the input to the second line
% (for convenient cut ant paste).
% Furthermore, the command |latex| can be replaced by any
% of its alternative versions such as |pdflatex|.
%
% %%%%%%%%%%%%%%%%%%%%%%%%%%%%%%%%%%%%%%%%%%%%%%%%%%%%%%%%%%%%%%%%%%%%%%%%%%%%%%
% %%%%%%%%%%%%%%%%%%%%%%%%%%%%%%%%%%%%%%%%%%%%%%%%%%%%%%%%%%%%%%%%%%%%%%%%%%%%%%
% \section{Implementation}
%\iffalse
%<*package>
%\fi
%
% This section describes the definitions file |childdoc.def|.

% The definitions cannot be loaded using |\usepackage| or |\RequirePackage|
% which has a mechanism to prevent loading a style file more than once.
% When loading the definitions by means of |\input|
% multiple instances have to be prevented manually:
%\iffalse
%This code needs to be before the `\ProvidesFile' directive
%which is defined at the beginning of this file.
%Therefore it is also placed there and commented out here.
%</package>
%<*discard>
%\fi
%    \begin{macrocode}
\ifdefined\childdocmain\endinput\fi
%    \end{macrocode}
%\iffalse
%</discard>
%<*package>
%\fi
%
% \macro{\ifchilddoc}
% \macro{\ifchilddocmanual}
% The conditional |\ifchilddoc| tells whether a
% child (true) or main (false) document is being compiled.
% The conditional |\ifchilddocmanual| tells whether
% the |\includeonly| mechanism is used (false) or
% the selection of child files must be performed manually (true).
% The definitions initialise to false:
%    \begin{macrocode}
\newif\ifchilddoc
\newif\ifchilddocmanual
%    \end{macrocode}

% \macro{\childdocname}
% \macro{\childdocjob}
% The macro |\childdocname| stores the name of the main document
% to be compiled. The macro |\childdocjob| stores the name of
% the document on which the \LaTeX{} compiler was originally invoked.
% The content of |\jobname| cannot be compared
% to filenames specified in the source due to different catcodes.
% The following code rescans |\jobname|, stores the result
% in |\childdocname| and saves a copy in |\childdocjob|:
%    \begin{macrocode}
\edef\childdocname{\scantokens\expandafter{\jobname\noexpand}}
\let\childdocjob\childdocname
%    \end{macrocode}

% \macro{\childdocdisable}
% The macro |\childdocdisable| prevents the main file
% from being processed more than once.
% At this stage, the main document command |\childdocmain|
% is assumed to be called once again where it should do nothing.
% Any subsequent call to it should prevent
% a secondary processing of the main document
% It overwrites the forwarding commands
% |\childdocof| and |\childdocforward|
% with empty macros to prevent further inclusions of the main document:
%    \begin{macrocode}
\newcommand{\childdocdisable}
{
  \renewcommand{\childdocmain}[1]{\renewcommand{\childdocmain}[1]{\endinput}}
  \renewcommand{\childdocof}[1]{}
  \renewcommand{\childdocby}[2][]{}
  \renewcommand{\childdocforward}[2][]{}
  \renewcommand{\childdocdisable}{}
}
%    \end{macrocode}

% \macro{\childdocmain}
% The macro |\childdocmain| is to be called at the top of the main file
% with nothing or the main filename (without extension) as argument.
% First, it breaks loops.
% If the argument is not empty and does not match |\childdocname|
% (which is set by the first inclusion of |childdoc.def|),
% |\ifchilddoc| is set to true, |\includeonly| is applied to the child file
% and |\jobname| is set to the main file
% (for proper handling of |.aux| files):
%    \begin{macrocode}
\newcommand{\childdocmain}[1]
{
  \childdocdisable\childdocmain{}
  \if?#1?\else
    \begingroup
      \def\childdoctmp{#1}
      \ifx\childdoctmp\childdocname
        \def\childdoctmp{}
      \else
        \def\childdoctmp
        {
          \childdoctrue
          \includeonly{\childdocname}
          \def\childdocjob{#1}
          \def\jobname{#1}
        }
      \fi
      \expandafter
    \endgroup
    \childdoctmp
  \fi
}
%    \end{macrocode}

% \macro{\childdocof}
% The command |\childdocof| redirects
% compilation to the main file |#1|.
%    \begin{macrocode}
\newcommand{\childdocof}[1]
{
  \childdocdisable
  \childdoctrue
  \includeonly{\childdocname}
  \def\jobname{#1}
  \def\childdocjob{#1}
  \input{#1}
}
%    \end{macrocode}

% \macro{\childdocby}
% The command |\childdocby| ....
%    \begin{macrocode}
\newcommand{\childdocby}[2][]
{
  \childdocdisable
  \childdoctrue
  \childdocmanualtrue
  \if?#1?\else
    \def\jobname{#2}
  \fi
  \def\childdocjob{#2}
  \input{#2}
  \endinput
}
%    \end{macrocode}

% \macro{\childdocforward}
% The command |\childdocforward| redirects
% compilation to the main file or
% (if the optional argument is given) a child file.
% Parameters are set as if the main file
% or a child file starting with |\childdocof| was compiled.
% Then compilation is handed over to the main file:
%    \begin{macrocode}
\newcommand{\childdocforward}[2][]
{
  \begingroup
    \if?#1?
      \def\childdoctmp
      {
        \def\childdocname{#2}
        \def\childdocjob{#2}
        \def\jobname{#2}
        \input{#2}
        \endinput
      }
    \else
      \def\childdoctmp
      {
        \childdocdisable
        \def\childdocname{#2}
        \childdoctrue
        \includeonly{#2}
        \def\childdocjob{#1}
        \def\jobname{#1}
        \input{#1}
        \endinput
      }
    \fi
    \expandafter
  \endgroup
  \childdoctmp
}
%    \end{macrocode}

% \macro{\childdocforwardprefix}
% The command |\childdocforwardprefix| redirects
% compilation to the main or a child file by means of a pattern.
% The prefix |#1| in the current filename is replaced by |#2|
% and the suffix of the current filename is kept
% (it is assumed that the filename does not contain the substring `|~~~|'
% which is used as a delimiter).
% Compilation is handed over to the new file by |\childdocforward|:
%    \begin{macrocode}
\newcommand{\childdocforwardprefix}[3][]
{
  \begingroup
    \def\childdocextract #2##1~~~{\def\childdoctmp{\childdocforward[#1]{#3##1}}}
    \expandafter\childdocextract\childdocname~~~
    \expandafter
  \endgroup
  \childdoctmp
}
%    \end{macrocode}

% \macro{\childdoc}
% The deprecated macro |\childdoc| is a legacy version of |\childdocmain|:
%    \begin{macrocode}
\newcommand{\childdoc}{\childdocmain}
%    \end{macrocode}

% \macro{\childdocredirect}
% The deprecated macro |\childdocredirect| is a legacy version
% of |\childdocforward| and |\childdocforwardprefix|:
%    \begin{macrocode}
\newcommand{\childdocredirect}[2][]
{
  \begingroup
    \if?#1?
      \def\childdoctmp{\childdocforward{#2}}
    \else
      \def\childdoctmp{\childdocforwardprefix{#1}{#2}}
    \fi
    \expandafter
  \endgroup
  \childdoctmp
}
%    \end{macrocode}

%\iffalse
%</package>
%\fi
%
\endinput
\childdocforward{cdocsamp}"|\\
% |latex -jobname cdocscl1 \|\\
% |  "% \iffalse
%
% childdoc.dtx Copyright (C) 2017-2018 Niklas Beisert
%
% This work may be distributed and/or modified under the
% conditions of the LaTeX Project Public License, either version 1.3
% of this license or (at your option) any later version.
% The latest version of this license is in
%   http://www.latex-project.org/lppl.txt
% and version 1.3 or later is part of all distributions of LaTeX
% version 2005/12/01 or later.
%
% This work has the LPPL maintenance status `maintained'.
%
% The Current Maintainer of this work is Niklas Beisert.
%
% This work consists of the files childdoc.dtx and childdoc.ins
% and the derived files childdoc.def and cdocsamp.tex with
% cdocsch1.tex, cdocsch2.tex, cdocsdrf.tex, cdocsfn1.tex, cdocsfn2.tex.
%
%<package>\ifdefined\childdocmain\endinput\fi
%<package>\ProvidesFile{childdoc.def}[2018/12/30 v2.0 child document driver]
%<samplemain>\ProvidesFile{cdocsamp.tex}[2018/12/30 v2.0 sample for childdoc]
%<*driver>
%\ProvidesFile{childdoc.drv}[2018/12/30 v2.0 childdoc reference manual file]
\PassOptionsToClass{10pt,a4paper}{article}
\documentclass{ltxdoc}

\usepackage[margin=35mm]{geometry}
\usepackage{hyperref}
\usepackage{hyperxmp}
\usepackage[usenames]{color}

\hypersetup{colorlinks=true}
\hypersetup{pdfstartview=FitH}
\hypersetup{pdfpagemode=UseNone}
\hypersetup{pdfsource={}}
\hypersetup{pdflang={en-UK}}
\hypersetup{pdfcopyright={Copyright 2017-2018 Niklas Beisert.
  This work may be distributed and/or modified under the
  conditions of the LaTeX Project Public License, either version 1.3
  of this license or (at your option) any later version.}}
\hypersetup{pdflicenseurl={http://www.latex-project.org/lppl.txt}}
\hypersetup{pdfcontactaddress={ETH Zurich, ITP, HIT K,
  Wolfgang-Pauli-Strasse 27}}
\hypersetup{pdfcontactpostcode={8093}}
\hypersetup{pdfcontactcity={Zurich}}
\hypersetup{pdfcontactcountry={Switzerland}}
\hypersetup{pdfcontactemail={nbeisert@itp.phys.ethz.ch}}
\hypersetup{pdfcontacturl={http://people.phys.ethz.ch/\xmptilde nbeisert/}}

\newcommand{\secref}[1]{\hyperref[#1]{section \ref*{#1}}}

\parskip1ex
\parindent0pt
\let\olditemize\itemize
\def\itemize{\olditemize\parskip0pt}

\begin{document}

\title{The \textsf{childdoc} Package}
\hypersetup{pdftitle={The childdoc Package}}
\author{Niklas Beisert\\[2ex]
  Institut f\"ur Theoretische Physik\\
  Eidgen\"ossische Technische Hochschule Z\"urich\\
  Wolfgang-Pauli-Strasse 27, 8093 Z\"urich, Switzerland\\[1ex]
  \href{mailto:nbeisert@itp.phys.ethz.ch}
  {\texttt{nbeisert@itp.phys.ethz.ch}}}
\hypersetup{pdfauthor={Niklas Beisert}}
\hypersetup{pdfsubject={Manual for the LaTeX2e Package childdoc}}
\date{30 December 2018, \textsf{v2.0}}
\maketitle

\begin{abstract}\noindent
\textsf{childdoc} is a \LaTeXe{} package
that enables the direct compilation
of document sections included by |\include|
to individual files.
\end{abstract}

\begingroup
\parskip0ex
\tableofcontents
\endgroup

%%%%%%%%%%%%%%%%%%%%%%%%%%%%%%%%%%%%%%%%%%%%%%%%%%%%%%%%%%%%%%%%%%%%%%%%%%%%%%%%
%%%%%%%%%%%%%%%%%%%%%%%%%%%%%%%%%%%%%%%%%%%%%%%%%%%%%%%%%%%%%%%%%%%%%%%%%%%%%%%%
\section{Introduction}

\LaTeX{} provides a mechanism to structure a large document (such as a book)
into a main file and several child files (containing the chapters)
using the |\include| command.
This mechanism is beneficial for documents
which span hundreds of pages in order to
make the source file(s) more manageable.
Moreover, compilation can be restricted to
selected child files by means of the |\includeonly| command.
The latter feature can be used to reduce the compilation time while editing
(this was significantly more useful in the earlier days of \LaTeX{})
or to generate a smaller document which is easier to navigate.
Another application of |\includeonly| is to generate
documents consisting of selected parts of the complete document.

However, there are a few drawbacks of the plain |\include| mechanism:
\begin{itemize}
\item
The child files cannot be compiled on their own,
they can only be compiled via the main file.
A naive editing environment
(such as a text editor with an option
to have the current file processed by \LaTeX)
may require one to switch to the main file before compiling;
attempting to compile the child file produces errors.
\item
The main file must be modified (each time)
to adjust the |\includeonly| command
to the present needs. This easily leaves the main file in a messy state.
\item
The generated document will always carry the filename
of the main document. This is inconvenient if
several child files are to be compiled and
to be kept for distribution.
\end{itemize}

The present package provides a simple interface
to make child files individually compilable by \LaTeX{}.
Compiling a child file then has the same effect as compiling
the main file with an |\includeonly| command
to select the appropriate child.
Moreover the generated document will carry the name of the child
rather than the main file.
This resolves all three above issues.

This feature is meant to make the editing of books,
thesis documents and lecture notes somewhat more convenient.
However, the package can also be used efficiently for
composing a series of documents (such as exercise sheets)
which are typically distributed individually.
It then assists the author in generating the individual documents
(potentially in different versions)
as well as a document containing the collected series.
Another application is in developing style files
or other kinds of included material
where compilation of the style file could redirect
to a sample or test file.

%%%%%%%%%%%%%%%%%%%%%%%%%%%%%%%%%%%%%%%%%%%%%%%%%%%%%%%%%%%%%%%%%%%%%%%%%%%%%%%%
%%%%%%%%%%%%%%%%%%%%%%%%%%%%%%%%%%%%%%%%%%%%%%%%%%%%%%%%%%%%%%%%%%%%%%%%%%%%%%%%
\section{Usage}

First of all, the package \textsf{childdoc} is \emph{not} a standard
\LaTeXe{} |.sty| style file! Therefore it needs to be invoked in
a non-standard way.

%%%%%%%%%%%%%%%%%%%%%%%%%%%%%%%%%%%%%%%%%%%%%%%%%%%%%%%%%%%%%%%%%%%%%%%%%%%%%%%%
\subsection{Included Files}
\label{sec:include}

%%%%%%%%%%%%%%%%%%%%%%%%%%%%%%%%%%%%%%%%
\DescribeMacro{\childdocmain}
To use the package, add the commands
\begin{center}
\begin{tabular}{l}
|% \iffalse
%
% childdoc.dtx Copyright (C) 2017-2018 Niklas Beisert
%
% This work may be distributed and/or modified under the
% conditions of the LaTeX Project Public License, either version 1.3
% of this license or (at your option) any later version.
% The latest version of this license is in
%   http://www.latex-project.org/lppl.txt
% and version 1.3 or later is part of all distributions of LaTeX
% version 2005/12/01 or later.
%
% This work has the LPPL maintenance status `maintained'.
%
% The Current Maintainer of this work is Niklas Beisert.
%
% This work consists of the files childdoc.dtx and childdoc.ins
% and the derived files childdoc.def and cdocsamp.tex with
% cdocsch1.tex, cdocsch2.tex, cdocsdrf.tex, cdocsfn1.tex, cdocsfn2.tex.
%
%<package>\ifdefined\childdocmain\endinput\fi
%<package>\ProvidesFile{childdoc.def}[2018/12/30 v2.0 child document driver]
%<samplemain>\ProvidesFile{cdocsamp.tex}[2018/12/30 v2.0 sample for childdoc]
%<*driver>
%\ProvidesFile{childdoc.drv}[2018/12/30 v2.0 childdoc reference manual file]
\PassOptionsToClass{10pt,a4paper}{article}
\documentclass{ltxdoc}

\usepackage[margin=35mm]{geometry}
\usepackage{hyperref}
\usepackage{hyperxmp}
\usepackage[usenames]{color}

\hypersetup{colorlinks=true}
\hypersetup{pdfstartview=FitH}
\hypersetup{pdfpagemode=UseNone}
\hypersetup{pdfsource={}}
\hypersetup{pdflang={en-UK}}
\hypersetup{pdfcopyright={Copyright 2017-2018 Niklas Beisert.
  This work may be distributed and/or modified under the
  conditions of the LaTeX Project Public License, either version 1.3
  of this license or (at your option) any later version.}}
\hypersetup{pdflicenseurl={http://www.latex-project.org/lppl.txt}}
\hypersetup{pdfcontactaddress={ETH Zurich, ITP, HIT K,
  Wolfgang-Pauli-Strasse 27}}
\hypersetup{pdfcontactpostcode={8093}}
\hypersetup{pdfcontactcity={Zurich}}
\hypersetup{pdfcontactcountry={Switzerland}}
\hypersetup{pdfcontactemail={nbeisert@itp.phys.ethz.ch}}
\hypersetup{pdfcontacturl={http://people.phys.ethz.ch/\xmptilde nbeisert/}}

\newcommand{\secref}[1]{\hyperref[#1]{section \ref*{#1}}}

\parskip1ex
\parindent0pt
\let\olditemize\itemize
\def\itemize{\olditemize\parskip0pt}

\begin{document}

\title{The \textsf{childdoc} Package}
\hypersetup{pdftitle={The childdoc Package}}
\author{Niklas Beisert\\[2ex]
  Institut f\"ur Theoretische Physik\\
  Eidgen\"ossische Technische Hochschule Z\"urich\\
  Wolfgang-Pauli-Strasse 27, 8093 Z\"urich, Switzerland\\[1ex]
  \href{mailto:nbeisert@itp.phys.ethz.ch}
  {\texttt{nbeisert@itp.phys.ethz.ch}}}
\hypersetup{pdfauthor={Niklas Beisert}}
\hypersetup{pdfsubject={Manual for the LaTeX2e Package childdoc}}
\date{30 December 2018, \textsf{v2.0}}
\maketitle

\begin{abstract}\noindent
\textsf{childdoc} is a \LaTeXe{} package
that enables the direct compilation
of document sections included by |\include|
to individual files.
\end{abstract}

\begingroup
\parskip0ex
\tableofcontents
\endgroup

%%%%%%%%%%%%%%%%%%%%%%%%%%%%%%%%%%%%%%%%%%%%%%%%%%%%%%%%%%%%%%%%%%%%%%%%%%%%%%%%
%%%%%%%%%%%%%%%%%%%%%%%%%%%%%%%%%%%%%%%%%%%%%%%%%%%%%%%%%%%%%%%%%%%%%%%%%%%%%%%%
\section{Introduction}

\LaTeX{} provides a mechanism to structure a large document (such as a book)
into a main file and several child files (containing the chapters)
using the |\include| command.
This mechanism is beneficial for documents
which span hundreds of pages in order to
make the source file(s) more manageable.
Moreover, compilation can be restricted to
selected child files by means of the |\includeonly| command.
The latter feature can be used to reduce the compilation time while editing
(this was significantly more useful in the earlier days of \LaTeX{})
or to generate a smaller document which is easier to navigate.
Another application of |\includeonly| is to generate
documents consisting of selected parts of the complete document.

However, there are a few drawbacks of the plain |\include| mechanism:
\begin{itemize}
\item
The child files cannot be compiled on their own,
they can only be compiled via the main file.
A naive editing environment
(such as a text editor with an option
to have the current file processed by \LaTeX)
may require one to switch to the main file before compiling;
attempting to compile the child file produces errors.
\item
The main file must be modified (each time)
to adjust the |\includeonly| command
to the present needs. This easily leaves the main file in a messy state.
\item
The generated document will always carry the filename
of the main document. This is inconvenient if
several child files are to be compiled and
to be kept for distribution.
\end{itemize}

The present package provides a simple interface
to make child files individually compilable by \LaTeX{}.
Compiling a child file then has the same effect as compiling
the main file with an |\includeonly| command
to select the appropriate child.
Moreover the generated document will carry the name of the child
rather than the main file.
This resolves all three above issues.

This feature is meant to make the editing of books,
thesis documents and lecture notes somewhat more convenient.
However, the package can also be used efficiently for
composing a series of documents (such as exercise sheets)
which are typically distributed individually.
It then assists the author in generating the individual documents
(potentially in different versions)
as well as a document containing the collected series.
Another application is in developing style files
or other kinds of included material
where compilation of the style file could redirect
to a sample or test file.

%%%%%%%%%%%%%%%%%%%%%%%%%%%%%%%%%%%%%%%%%%%%%%%%%%%%%%%%%%%%%%%%%%%%%%%%%%%%%%%%
%%%%%%%%%%%%%%%%%%%%%%%%%%%%%%%%%%%%%%%%%%%%%%%%%%%%%%%%%%%%%%%%%%%%%%%%%%%%%%%%
\section{Usage}

First of all, the package \textsf{childdoc} is \emph{not} a standard
\LaTeXe{} |.sty| style file! Therefore it needs to be invoked in
a non-standard way.

%%%%%%%%%%%%%%%%%%%%%%%%%%%%%%%%%%%%%%%%%%%%%%%%%%%%%%%%%%%%%%%%%%%%%%%%%%%%%%%%
\subsection{Included Files}
\label{sec:include}

%%%%%%%%%%%%%%%%%%%%%%%%%%%%%%%%%%%%%%%%
\DescribeMacro{\childdocmain}
To use the package, add the commands
\begin{center}
\begin{tabular}{l}
|\input{childdoc.def}|\\
|\childdocmain{}|\\
\end{tabular}
\end{center}
at the very top of the main \LaTeX{} file,
in particular \emph{before} the |\documentclass| statement!
The argument of |\childdocmain| should be left empty
(but it must be present).

%%%%%%%%%%%%%%%%%%%%%%%%%%%%%%%%%%%%%%%%
\DescribeMacro{\childdocof}
Furthermore, add the commands
\begin{center}
\begin{tabular}{l}
|\input{childdoc.def}|\\
|\childdocof{|\textit{main}|}|\\
\end{tabular}
\end{center}
at the top of every child file \textit{child}
which is included by |\include{|\textit{child}|}|
from within the main file
(or at least for those files to be compiled individually).
The argument \textit{main} must be the filename of the main file.

There are a couple of
considerations in setting up the main and child documents:

%%%%%%%%%%%%%%%%%%%%%%%%%%%%%%%%%%%%%%%%
\paragraph{Restrictions.}

Please note the following restrictions:
\begin{itemize}
\item
|\childdocmain| must be called with one argument \textit{main}
to ensure compatibility with earlier version of the package.
It must either be empty (|\childdocmain{}|)
or precisely match the filename of the main file in which it is specified.
See \secref{sec:detection} for further information.
\item
The filename \textit{main} must be specified without the |.tex| extension.
\item
The filename \textit{main} is case sensitive
(even in case-insensitive file systems)
due to internal string comparison.
\item
The argument \textit{main} should be fully expanded, it cannot be a macro.
\item
Subdirectories and special characters should be avoided in filenames.
\item
The command |\childdocmain{|\textit{main}|}| must be followed by a whitespace.
It should not be followed immediately by another command
or by a comment mark `|%|'.
This is because the \TeX{} parser reads the token immediately following
the argument of |\childdocmain| and puts it
at the beginning of every child section;
however, a white\-space is ignored.
\end{itemize}

%%%%%%%%%%%%%%%%%%%%%%%%%%%%%%%%%%%%%%%%
\paragraph{Content of Main File.}

It is advisable to place all content in the child files included by |\include|.
Any output contained in the main file will appear in all child documents
unless suppressed manually;
it cannot be suppressed automatically by the |\includeonly| directive
and thus should normally be avoided.
A method to include some content in the main file
by means of conditional processing is described in \secref{sec:conditional}.

%%%%%%%%%%%%%%%%%%%%%%%%%%%%%%%%%%%%%%%%
\paragraph{Page Numbering.}

When only a part of the document is compiled,
the appropriate numbering of pages
(as well as other status parameters)
is determined from the |.aux| files.
The latter contain information from previous passes.
However this information needs to propagate through
all intermediate child documents.
Therefore the page numbering in child documents may well
be inconsistent until the complete document is compiled at least once.

A useful (if unconventional) way to always ensure a consistent
page numbering is to restart the numbering in each child document
and denote the pages by `\textit{child}|.|\textit{page}'
where \textit{child} represents the chapter/section number of the child file.
This can be achieved by the command
|\numberwithin{page}{|\textit{child}|}|
of the \textsf{amsmath} package
where \textit{child} can be |chapter| or |section|
depending on the chosen structuring.
Alternatively, one can modify the macro |\thepage| appropriately
and reset the counter |page| at the start of each child file.

%%%%%%%%%%%%%%%%%%%%%%%%%%%%%%%%%%%%%%%%%%%%%%%%%%%%%%%%%%%%%%%%%%%%%%%%%%%%%%%%
\subsection{Conditional Processing}
\label{sec:conditional}

The package provides a mechanism to compile different versions
of a document. To customise the versions further some conditional processing
can come in handy to distinguish which version is being compiled.
The package provides two macros to describe the compilation context:

%%%%%%%%%%%%%%%%%%%%%%%%%%%%%%%%%%%%%%%%
\DescribeMacro{\ifchilddoc}
The conditional |\ifchilddoc| distinguishes between the compilation of
child documents and the main document:
%
\begin{center}
|\ifchilddoc |\textit{child-code}| |[|\||else |\textit{main-code}]| \||fi|
\end{center}

%%%%%%%%%%%%%%%%%%%%%%%%%%%%%%%%%%%%%%%%
\DescribeMacro{\childdocname}
\DescribeMacro{\childdocjob}
The macro |\childdocname| contains the filename (without extension)
of the main or child file being processed.
Note that |\childdocjob| will always contain the name of the main file.

%%%%%%%%%%%%%%%%%%%%%%%%%%%%%%%%%%%%%%%%
\paragraph{Title Page.}

Conditional processing can be used to include a title or banner page
in the main document when proper precautions are taken.
Importantly, the code in the main file should ensure that the page counter
(as well as other status parameters which are stored in the |.aux| files)
takes the same value after the conditional processing.
Otherwise the page numbers may take divergent values
depending on which part is compiled.

For example, a title page could be declared by:
%
\begin{center}
\begin{tabular}{l}
|\ifchilddoc\||else|\\
|\addtocounter{page}{-1}|\\
\textit{code for title page}\\
|\newpage|\\
|\||fi|
\end{tabular}
\end{center}
%
A banner page for the child documents can be generated by:
%
\begin{center}
\begin{tabular}{l}
|\ifchilddoc|\\
|\addtocounter{page}{-1}|\\
\textit{code for banner page}\\
|\newpage|\\
|\||fi|
\end{tabular}
\end{center}
%
Here one could write a message such as:
\begin{center}
|This is the part \childdocname{} of \childdocjob{}.|
\end{center}

%%%%%%%%%%%%%%%%%%%%%%%%%%%%%%%%%%%%%%%%%%%%%%%%%%%%%%%%%%%%%%%%%%%%%%%%%%%%%%%%
\subsection{Flags}
\label{sec:flags}

The package makes it easy to generate different versions
of the main or child documents.
To this end compilation flags can be defined
and assigned different default values.
They will be particularly useful in conjunction
with the forwarding mechanism described in \secref{sec:forward}.

For example, it may be useful to have a flag |\version|
which can be set to |draft| or |final|.
The document source will contain some conditional code
depending on the value of |\version|.
Suppose further, the flag should default to |final| for the main file
and to |draft| for child files
which is a natural assignment for editing the document.
This is achieved by placing the following code
in the preamble of the main document
(below the |\childdocmain| directive):
%
\begin{center}
\begin{tabular}{l}
|\ifchilddoc|\\
|\providecommand{\version}{draft}|\\
|\||else|\\
|\providecommand{\version}{final}|\\
|\||fi|
\end{tabular}
\end{center}
%
The definition by |\providecommand| makes sure
that previous definitions are not overwritten.
Further statements |\providecommand{\version}{...}|
can thus be added before the above code to override it.

For the main file, one might add a line
(between |\childdocmain| and the above block)
%
\begin{center}
|%\ifchilddoc\||else\providecommand{\version}{draft}\||fi|
\end{center}
%
which can be uncommented to produce a draft version.
Likewise one can add a line to the very top of a child file
(above the |\childdocof{|\textit{main}|}| directive)
%
\begin{center}
|%\providecommand{\version}{final}|
\end{center}
%
which can be uncommented to produce the final version of this child document.

%%%%%%%%%%%%%%%%%%%%%%%%%%%%%%%%%%%%%%%%%%%%%%%%%%%%%%%%%%%%%%%%%%%%%%%%%%%%%%%%
\subsection{Forwarding}
\label{sec:forward}

Different versions of the main or child documents
using compilation flags as described in \secref{sec:flags}
can be (permanently) stored in different files
for convenient compilation, viewing and distribution.
To this end, the package defines a command
to pass on compilation to a different file:

%%%%%%%%%%%%%%%%%%%%%%%%%%%%%%%%%%%%%%%%
\DescribeMacro{\childdocforward}
The command |\childdocforward| redirects processing to
another source file:
%
\begin{center}
\begin{tabular}{l}
|\input{childdoc.def}|\\
|\childdocforward[|\textit{main}|]{|\textit{dest}|}|\\
\end{tabular}
\end{center}
%
The argument \textit{dest} is the destination file
(without extension).
It should be the main file or one of the child files.
Note that further \textsf{childdoc} directives
such as |\childdocof| and |\childdocforward|
in the indicated file will be processed in this form.
The optional argument \textit{main}
passes on directly to the main file \textit{main}
while pretending to compile the child \textit{dest}.
This form behaves as if \textit{dest}
issues |\childdocof{|\textit{main}|}| right away,
and no further \textsf{childdoc} directives will be processed.

%%%%%%%%%%%%%%%%%%%%%%%%%%%%%%%%%%%%%%%%
\DescribeMacro{\...prefix}
In the alternative form |\childdocforwardprefix|,
%
\begin{center}
\begin{tabular}{l}
|\input{childdoc.def}|\\
|\childdocforwardprefix[|\textit{main}|]{|\textit{prefix}|}{|\textit{dest}|}|
\end{tabular}
\end{center}
%
the destination file is determined by a pattern
depending on the current file:
To make this work, the current file must be called
`{\textit{prefix}\hspace{0.2em}\textit{suffix}}'
with \textit{prefix} matching precisely the argument.
Processing is then passed on to the file
`{\textit{dest}\hspace{0.2em}\textit{suffix}}'.
Surely, the same effect is achieved by
directly specifying the
argument `{\textit{dest}\hspace{0.2em}\textit{suffix}}'
in the first form.
However, that requires to set up a different file
for each child. With the alternative form of the command
all these files can have exactly the same content
which simplifies setting them up and maintaining them.

For example, the following file |draft.tex|
with a compilation flag |\version| as described in \secref{sec:flags}
compiles the main document as a draft:
%
\begin{center}
\begin{tabular}{l}
|\def\version{draft}|\\
|\input{childdoc.def}|\\
|\childdocforward{|\textit{main}|}|
\end{tabular}
\end{center}
%
Likewise, the following files |final|\textit{nn}|.tex|
compile the final version of the child document
|child|\textit{nn}|.tex|:
%
\begin{center}
\begin{tabular}{l}
|\def\version{final}|\\
|\input{childdoc.def}|\\
|\childdocforwardprefix{final}{child}|
\end{tabular}
\end{center}
%

Note that when several versions of a main file and/or of each child file
are to be generated, it may be convenient to set up a |Makefile| or
shell script to automatise the process.

%%%%%%%%%%%%%%%%%%%%%%%%%%%%%%%%%%%%%%%%%%%%%%%%%%%%%%%%%%%%%%%%%%%%%%%%%%%%%%%%
\subsection{Command Line Processing}
\label{sec:commandline}

The effect of redirection files can also be achieved by invoking
the \LaTeX{} compiler with a more elaborate command line.
Most conveniently this should be done as part
of a shell script or a |Makefile|.

When using \textsf{childdoc} in the main file, the following
command lines effectively perform a redirection
(note that depending on the shell being used,
backslashes may have to be doubled: `|\|' $\to$ `|\\|'):
%
\begin{center}
|... -jobname "|\textit{target}|" |\\|"|[\textit{flags}]%
|\input{childdoc.def}\childdocforward[|\textit{main}|]{|\textit{dest}|}"|
\end{center}
%
Here \textit{target} is the name of the output file,
\textit{main} is the name of the main file
and \textit{dest} is the name of the main or child file to be processed
(all filenames without extensions).
The optional argument \textit{main} can be omitted
if \textit{main} matches \textit{dest}.
Optionally, compilation \textit{flags} can be defined via |\def| commands.
This command line makes the \TeX{} engine believe
it is compiling the file \textit{target}
whose content is specified as the latter parameter.
The provided code then forwards the processing to
\textit{main} or \textit{dest} as described in \secref{sec:forward}.

%%%%%%%%%%%%%%%%%%%%%%%%%%%%%%%%%%%%%%%%%%%%%%%%%%%%%%%%%%%%%%%%%%%%%%%%%%%%%%%%
\subsection{Include by Input}
\label{sec:input}

Including child documents by |\include| has some restrictions by design.
Most notably, the content of a child document always occupies
its own set of pages; pages cannot be shared between child documents.
Usually, this behaviour makes perfect sense
because each child document contain an essential part of the document.
However, in some situations it may be desirable to compose
a document from a collection of parts
without having mandatory page breaks between then.
For this case, the package
provides a mechanism to include parts
by |\input| which can also be processed individually.
However, by construction this mechanism
requires manual handling of the content to be output.

%%%%%%%%%%%%%%%%%%%%%%%%%%%%%%%%%%%%%%%%
\DescribeMacro{\ifchilddocmanual}
The main file should be prepared as usual, see \secref{sec:include}.
However, the document body must make a distinction
between processing of an individual part and of the main document, e.g.:
%
\begin{center}
\begin{tabular}{l}
|\ifchilddocmanual|\\
|\input{\childdocname}|\\
|\||else|\\
\textit{document body with }|\input{|\textit{part}|}|\\
|\||fi|
\end{tabular}
\end{center}
%
The conditional |\ifchilddocmanual| is true whenever
a part to be included by |\input| is being compiled,
and the name of the part is stored in |\childdocname|.

%%%%%%%%%%%%%%%%%%%%%%%%%%%%%%%%%%%%%%%%
\DescribeMacro{\childdocby}
Each part to be included by |\input| should start with:
%
\begin{center}
\begin{tabular}{l}
|\input{childdoc.def}|\\
|\childdocby{|\textit{main}|}|\\
\end{tabular}
\end{center}
%
The directive |\childdocby| is similar to |\childdocof|
described in \secref{sec:include},
but the subsequent selection of content must be done manually.
To that end, both |\ifchilddoc| and |\ifchilddocmanual|
will be true upon processing of a part,
and the name of the part is stored in |\childdocname|.
Note that |\jobname| will be set to the filename of the current part
so that each part receives an individual |.aux| file
that does not interfere with the |.aux| file(s) of the main document.
This behaviour can be altered by the alternative form
|\childdocby[*]{|\textit{main}|}| (with a non-empty optional argument)
which uses the |.aux| file of the main document
by setting |\jobname| to \textit{main}.

%%%%%%%%%%%%%%%%%%%%%%%%%%%%%%%%%%%%%%%%%%%%%%%%%%%%%%%%%%%%%%%%%%%%%%%%%%%%%%%%
\subsection{Driver Development}
\label{sec:driver}

The \textsf{childdoc} mechanism can also be use for the development
of definition files such as \LaTeX{} styles or classes.
This case differs from the above setup with multiple parts
included by |\include| in that no |\includeonly| should be invoked.
This can be achieved by starting the include file
(before |\ProvidesPackage|) with:
%
\begin{center}
\begin{tabular}{l}
|\input{childdoc.def}|\\
|\childdocforward{|\textit{main}|}|\\
\end{tabular}
\end{center}
%
or alternatively with:
%
\begin{center}
\begin{tabular}{l}
|\input{childdoc.def}|\\
|\childdocby{|\textit{main}|}|\\
\end{tabular}
\end{center}
%
Both forms have slightly different effects as described above.
The main file is prepared as usual, see \secref{sec:include}.

%%%%%%%%%%%%%%%%%%%%%%%%%%%%%%%%%%%%%%%%%%%%%%%%%%%%%%%%%%%%%%%%%%%%%%%%%%%%%%%%
\subsection{Legacy Detection}
\label{sec:detection}

The directive |\childdocmain| in the main file can detect
whether the complete document or merely a child is to be compiled
even without using the directive |\childdocof|.
This method is deprecated because it is less robust
and there is no compelling reason to use it;
it is merely provided for backward compatibility
and it may be removed in future versions.

If the detection mechanism is to be used,
it is mandatory to correctly specify
the filename of the main file as the argument of |\childdocmain|:
%
\begin{center}
\begin{tabular}{l}
|\input{childdoc.def}|\\
|\childdocmain{|\textit{main}|}|\\
\end{tabular}
\end{center}
%
If |\jobname| does not match the argument \textit{main} of |\childdocmain|,
it is assumed that |\jobname| points to the child file to be compiled.
When using |\childdocmain| with the main file specified as argument,
it suffices to start a child file
with just |\input{|\textit{main}|}|
without loading of the package and using |\childdocof|.
If instead all processing is done
with the appropriate \textsf{childdoc} directives,
the argument of \textit{main} of |\childdocmain| can be empty.

An alternative version of the command line processing described
in \secref{sec:commandline} using the detection mechanism reads:
%
\begin{center}
|... -jobname "|\textit{target}|" "|[\textit{flags}]%
[|\def\jobname{|\textit{dest}|}|]|\input{|\textit{main}|}"|
\end{center}

%%%%%%%%%%%%%%%%%%%%%%%%%%%%%%%%%%%%%%%%%%%%%%%%%%%%%%%%%%%%%%%%%%%%%%%%%%%%%%%%
\subsection{Manual Code}
\label{sec:manual}

In case one cannot be certain whether the definitions file |childdoc.def|
is installed on the target \TeX{} distribution
and one prefers not to ship it,
it is conceivable to paste a few relevant commands into the sources.

To that end, drop all statements |\input{childdoc.def}|
and perform the replacements as outlined below.
Instead of |\childdocmain{|\textit{main}|}| add the following code
to the top of the main file:
%
\begin{center}
\begin{tabular}{l}
|\||ifdefined\childdocname\endinput\||fi\newif\ifchilddoc|\\
|\edef\childdocname{\scantokens\expandafter{\jobname\noexpand}}|\\
|\def\childdocmain{|\textit{main}|}\||ifx\childdocmain\childdocname\||else|\\
|\childdoctrue\includeonly{\childdocname}\let\jobname\childdocmain\||fi|\\
\end{tabular}
\end{center}
%
Instead of |\childdocof{|\textit{main}|}| just include the main file
at the top of each child file:
%
\begin{center}
|\input{|\textit{main}|}|
\end{center}
%
A simple redirection |\childdocforward{|\textit{dest}|}| is achieved by:
%
\begin{center}
|\def\jobname{|\textit{dest}|}\input{\jobname}|
\end{center}
%
The redirection with prefix
|\childdocforwardprefix[|\textit{prefix}|]{|\textit{dest}|}|
is accomplished by:
%
\begin{center}
\begin{tabular}{l}
|{\edef\jobname{\scantokens\expandafter{\jobname\noexpand}}|\\
|\def\redirectjob |\textit{prefix}|#1~~~{\gdef\jobname{|\textit{dest}|#1}}|\\
|\expandafter\redirectjob\jobname~~~}\input{\jobname}|
\end{tabular}
\end{center}

In an alternative approach,
child documents can be compiled by a specific command line
without additional code or specific definitions:
%
\begin{center}
|... -jobname "|\textit{target}|" "|[\textit{flags}]%
|\includeonly{|\textit{dest}|}\input{|\textit{main}|}"|
\end{center}
%

%%%%%%%%%%%%%%%%%%%%%%%%%%%%%%%%%%%%%%%%%%%%%%%%%%%%%%%%%%%%%%%%%%%%%%%%%%%%%%%%
%%%%%%%%%%%%%%%%%%%%%%%%%%%%%%%%%%%%%%%%%%%%%%%%%%%%%%%%%%%%%%%%%%%%%%%%%%%%%%%%
\section{Information}

%%%%%%%%%%%%%%%%%%%%%%%%%%%%%%%%%%%%%%%%%%%%%%%%%%%%%%%%%%%%%%%%%%%%%%%%%%%%%%%%
\subsection{Copyright}

Copyright \copyright{} 2017--2018 Niklas Beisert

This work may be distributed and/or modified under the
conditions of the \LaTeX{} Project Public License, either version 1.3
of this license or (at your option) any later version.
The latest version of this license is in
  \url{http://www.latex-project.org/lppl.txt}
and version 1.3 or later is part of all distributions of \LaTeX{}
version 2005/12/01 or later.

This work has the LPPL maintenance status `maintained'.

The Current Maintainer of this work is Niklas Beisert.

This work consists of the files |README.txt|, |childdoc.ins| and |childdoc.dtx|
as well as the derived files |childdoc.def|, |cdocsamp.tex|
with |cdocsch1.tex|, |cdocsch2.tex|, |cdocspt3.tex|, |cdocspt4.tex|,
|cdocsdrf.tex|, |cdocsfn1.tex|, |cdocsfn2.tex|
as well as |childdoc.pdf|.

%%%%%%%%%%%%%%%%%%%%%%%%%%%%%%%%%%%%%%%%%%%%%%%%%%%%%%%%%%%%%%%%%%%%%%%%%%%%%%%%
\subsection{Files and Installation}

The package consists of the files:
%
\begin{center}
\begin{tabular}{ll}
    |README.txt|   & readme file \\
    |childdoc.ins| & installation file \\
    |childdoc.dtx| & source file \\
    |childdoc.def| & definition file \\
    |cdocsamp.tex| & sample main file \\
    |cdocsch1.tex| & sample include file \\
    |cdocsch2.tex| & sample include file \\
    |cdocspt3.tex| & sample part file \\
    |cdocspt4.tex| & sample part file \\
    |cdocsdrf.tex| & sample redirection file \\
    |cdocsfn1.tex| & sample redirection file \\
    |cdocsfn2.tex| & sample redirection file \\
    |childdoc.pdf| & manual
\end{tabular}
\end{center}
%
The distribution consists of the files
|README.txt|, |childdoc.ins| and |childdoc.dtx|.
%
\begin{itemize}
\item
Run (pdf)\LaTeX{} on |childdoc.dtx|
to compile the manual |childdoc.pdf| (this file).
\item
Run \LaTeX{} on |childdoc.ins| to create the definitions file |childdoc.def|
and the sample |cdocsamp.tex| with include files
|cdocsch1.tex|, |cdocsch2.tex|, |cdocspt3.tex|, |cdocspt4.tex|,
|cdocsdrf.tex|, |cdocsfn1.tex|, |cdocsfn2.tex|.
Then copy the file |childdoc.def| to an appropriate directory of your \LaTeX{}
distribution, e.g.\ \textit{texmf-root}|/tex/latex/childdoc|.
\end{itemize}

%%%%%%%%%%%%%%%%%%%%%%%%%%%%%%%%%%%%%%%%%%%%%%%%%%%%%%%%%%%%%%%%%%%%%%%%%%%%%%%%
\subsection{Related CTAN Packages}

There are several other packages which offer a similar functionality:
%
\begin{itemize}
\item
The packages
\href{http://ctan.org/pkg/docmute}{\textsf{docmute}},
\href{http://ctan.org/pkg/includex}{\textsf{includex}} and
\href{http://ctan.org/pkg/standalone}{\textsf{standalone}}
provide commands to include only the document body of
a child file thus allowing both files to be compiled individually.
\item
The packages \href{http://ctan.org/pkg/subdocs}{\textsf{subdocs}}
and \href{http://ctan.org/pkg/subfiles}{\textsf{subfiles}}
provide structures in which the main and child documents can be
encapsulated and allowing them to be compiled individually.
The inclusion mechanism is different from the conventional |\include|.
\item
The package \href{http://ctan.org/pkg/combine}{\textsf{combine}}
is an elaborate solution to combine several documents into one.
\end{itemize}
%
See also the CTAN topic \href{http://ctan.org/topic/subdocs}{\textsf{subdocs}}
for further related packages.
The present package differs from the above solutions in that
a document structure constructed with the conventional |\include| mechanism
just needs two extra commands at the top of every file
such that all constituent files can be compiled individually.

%%%%%%%%%%%%%%%%%%%%%%%%%%%%%%%%%%%%%%%%%%%%%%%%%%%%%%%%%%%%%%%%%%%%%%%%%%%%%%%%
%\subsection{Feature Suggestions}
%
%The following is a list of features which may be useful for future
%versions of this package:
%%
%\begin{itemize}
%\item
%\ldots
%\end{itemize}

%%%%%%%%%%%%%%%%%%%%%%%%%%%%%%%%%%%%%%%%%%%%%%%%%%%%%%%%%%%%%%%%%%%%%%%%%%%%%%%%
\subsection{Revision History}

%%%%%%%%%%%%%%%%%%%%%%%%%%%%%%%%%%%%%%%%
\paragraph{v2.0:} 2018/12/30

\begin{itemize}
\item
immediate forward processing
\item
added |\childdocby| mechanism
\item
manual restructured
\end{itemize}

%%%%%%%%%%%%%%%%%%%%%%%%%%%%%%%%%%%%%%%%
\paragraph{v1.6:} 2018/01/17

\begin{itemize}
\item
application for development of include files
\item
corrections to manual
\end{itemize}

%%%%%%%%%%%%%%%%%%%%%%%%%%%%%%%%%%%%%%%%
\paragraph{v1.5:} 2017/05/21

\begin{itemize}
\item
more complete structuring introduced
\item
|\childdocof| introduced
\item
|\childdoc| renamed to |\childdocmain|
\item
|\childredirect| renamed to |\childdocforward| and |\childdocforwardprefix|
and functionality expanded
\end{itemize}

%%%%%%%%%%%%%%%%%%%%%%%%%%%%%%%%%%%%%%%%
\paragraph{v1.0:} 2017/04/27

\begin{itemize}
\item
manual and install package
\item
first version published on CTAN
\end{itemize}

%%%%%%%%%%%%%%%%%%%%%%%%%%%%%%%%%%%%%%%%
\paragraph{v0.6:} 2017/04/26

\begin{itemize}
\item
redirection mechanism added
\end{itemize}

%%%%%%%%%%%%%%%%%%%%%%%%%%%%%%%%%%%%%%%%
\paragraph{v0.5:} 2017/04/26

\begin{itemize}
\item
functionality in definition file
\end{itemize}


%%%%%%%%%%%%%%%%%%%%%%%%%%%%%%%%%%%%%%%%%%%%%%%%%%%%%%%%%%%%%%%%%%%%%%%%%%%%%%%%
%%%%%%%%%%%%%%%%%%%%%%%%%%%%%%%%%%%%%%%%%%%%%%%%%%%%%%%%%%%%%%%%%%%%%%%%%%%%%%%%
%%%%%%%%%%%%%%%%%%%%%%%%%%%%%%%%%%%%%%%%%%%%%%%%%%%%%%%%%%%%%%%%%%%%%%%%%%%%%%%%
\appendix

\settowidth\MacroIndent{\rmfamily\scriptsize 000\ }

 \DocInput{childdoc.dtx}

\end{document}
%</driver>
% \fi
%
% %%%%%%%%%%%%%%%%%%%%%%%%%%%%%%%%%%%%%%%%%%%%%%%%%%%%%%%%%%%%%%%%%%%%%%%%%%%%%%
% %%%%%%%%%%%%%%%%%%%%%%%%%%%%%%%%%%%%%%%%%%%%%%%%%%%%%%%%%%%%%%%%%%%%%%%%%%%%%%
% \section{Sample}
%\iffalse
%<*samplemain>
%\fi
%
% The following presents a sample document
% with two chapters, two parts, a title page,
% a compile flag as well as three forwarding files to set the flag.
% It consists of eight |.tex| files:
% \begin{center}
% \begin{tabular}{ll}
% |cdocsamp.tex|&main file\\
% |cdocsch1.tex|&include file for chapter 1\\
% |cdocsch2.tex|&include file for chapter 2\\
% |cdocspt3.tex|&include file for part 3\\
% |cdocspt4.tex|&include file for part 4\\
% |cdocsdrf.tex|&forwarding file for main file in draft mode\\
% |cdocsfi1.tex|&forwarding file for final version of chapter 1\\
% |cdocsfi2.tex|&forwarding file for final version of chapter 2\\
% \end{tabular}
% \end{center}
% Each of the eight files can be compiled directly by the \LaTeX{} compiler.
%
% %%%%%%%%%%%%%%%%%%%%%%%%%%%%%%%%%%%%%%
% \paragraph{Main File.}
%
% The main file is called |cdocsamp.tex|.
%
% Load the \textsf{childdoc} definitions and
% declare the filename for the main document:
%    \begin{macrocode}
\input{childdoc.def}
\childdocmain{}
%    \end{macrocode}

% Optional override for |\version| flag:
%    \begin{macrocode}
%%\ifchilddoc\else\providecommand{\version}{draft}\fi
%    \end{macrocode}

% Define the default values for the |\version| flag
% (|final| for the main file and |draft| for childs):
%    \begin{macrocode}
\ifchilddoc
\providecommand{\version}{draft}
\else
\providecommand{\version}{final}
\fi
%    \end{macrocode}

% Load the standard document class:
%    \begin{macrocode}
\documentclass[12pt]{article}
%    \end{macrocode}

% Start the document body:
%    \begin{macrocode}
\begin{document}
%    \end{macrocode}

% Declare a title page.
% Print title, part of document being processed and version flag:
%    \begin{macrocode}
\addtocounter{page}{-1}
\begin{center}
{\LARGE\bfseries{}childdoc example\par}
\vspace{1cm}
\ifchilddoc
\ifchilddocmanual part\else chapter\fi:
`\childdocname' of `\childdocjob'\par
\else
main document: `\childdocjob'\par
\fi
version: \version\par
\end{center}
\newpage
%    \end{macrocode}

% Manually include selected file,
% otherwise process as usual:
%    \begin{macrocode}
\ifchilddocmanual
\section*{part `\childdocname'}
\input{\childdocname}
\else
%    \end{macrocode}

% Include the two chapters:
%    \begin{macrocode}
\include{cdocsch1}
\include{cdocsch2}
%    \end{macrocode}

% Include the two parts unless only chapters should be displayed:
%    \begin{macrocode}
\ifchilddoc\else
\section{part three}
\input{cdocspt3}
\section{part four}
\input{cdocspt4}
\fi
%    \end{macrocode}

% Process as usual until here:
%    \begin{macrocode}
\fi
%    \end{macrocode}

% End of document body:
%    \begin{macrocode}
\end{document}
%    \end{macrocode}
%\iffalse
%</samplemain>
%\fi
%
% %%%%%%%%%%%%%%%%%%%%%%%%%%%%%%%%%%%%%%
% \paragraph{Chapter Include Files.}
%
% The include files are called |cdocsch1.tex| and |cdocsch2.tex|.
%
%\iffalse
%<*samplechap1|samplechap2>
%\fi

% Optional override for |\version| flag:
%    \begin{macrocode}
%%\providecommand{\version}{final}
%    \end{macrocode}

% Include the main document:
%    \begin{macrocode}
\input{childdoc.def}
\childdocof{cdocsamp}
%    \end{macrocode}

%\iffalse
%</samplechap1|samplechap2>
%\fi
%
%\iffalse
%<*samplechap1>
%\fi
% Some text for chapter 1:
%    \begin{macrocode}
\section{one}
some text in chapter one
%    \end{macrocode}

%\iffalse
%</samplechap1>
%\fi
% Some text for chapter 2:
%\iffalse
%<*samplechap2>
%\fi
%    \begin{macrocode}
\section{two}
more text in chapter two
%    \end{macrocode}

%\iffalse
%</samplechap2>
%\fi
%
% %%%%%%%%%%%%%%%%%%%%%%%%%%%%%%%%%%%%%%
% \paragraph{Part Include Files.}
%
% The include files are called |cdocspt3.tex| and |cdocspt4.tex|.
%
%\iffalse
%<*samplepart3|samplepart4>
%\fi

% Optional override for |\version| flag:
%    \begin{macrocode}
%%\providecommand{\version}{final}
%    \end{macrocode}

% Include the main document:
%    \begin{macrocode}
\input{childdoc.def}
\childdocby{cdocsamp}
%    \end{macrocode}

%\iffalse
%</samplepart3|samplepart4>
%\fi
%
%\iffalse
%<*samplepart3>
%\fi
% Some text for part 3:
%    \begin{macrocode}
some text in part three
%    \end{macrocode}

%\iffalse
%</samplepart3>
%\fi
% Some text for part 4:
%\iffalse
%<*samplepart4>
%\fi
%    \begin{macrocode}
more text in part four
%    \end{macrocode}

%\iffalse
%</samplepart4>
%\fi
%
% %%%%%%%%%%%%%%%%%%%%%%%%%%%%%%%%%%%%%%
% \paragraph{Forwarding for a Complete Draft.}
%
% The following forwarding file |cdocsdrf.tex|
% compiles the main document in draft mode:
%\iffalse
%<*sampledraft>
%\fi
%    \begin{macrocode}
\def\version{draft}
\input{childdoc.def}
\childdocforward{cdocsamp}
%    \end{macrocode}

%\iffalse
%</sampledraft>
%\fi
%
% %%%%%%%%%%%%%%%%%%%%%%%%%%%%%%%%%%%%%%
% \paragraph{Forwarding for Final Version of the Chapters.}
%
% The following forwarding files |cdocsfn1.tex| and |cdocsfn2.tex|
% (with identical content)
% compile the final versions of the child documents
% |cdocsch1.tex| and |cdocsch2.tex|, respectively:
%\iffalse
%<*samplefinal>
%\fi
%    \begin{macrocode}
\def\version{final}
\input{childdoc.def}
\childdocforwardprefix[cdocsamp]{cdocsfn}{cdocsch}
%    \end{macrocode}

%\iffalse
%</samplefinal>
%\fi
%
% %%%%%%%%%%%%%%%%%%%%%%%%%%%%%%%%%%%%%%
% \paragraph{Command Line Processing.}
%
% The following three command lines generate the output files
% |cdocscld|, |cdocscl1| and |cdocscl2|
% which should be identical to
% |cdocsdrf|, |cdocsch1| and |cdocsfn2|, respectively:
% \begin{center}
% \begin{tabular}{l}
% |latex -jobname cdocscld \|\\
% |  "\def\version{draft}\input{childdoc.def}\childdocforward{cdocsamp}"|\\
% |latex -jobname cdocscl1 \|\\
% |  "\input{childdoc.def}\childdocforward[cdocsamp]{cdocsch1}"|\\
% |latex -jobname cdocscl2 \|\\
% |  "\def\version{final}\input{childdoc.def}\childdocforward{cdocsch2}"|
% \end{tabular}
% \end{center}
% Note that the trailing backslash on each first line
% merely continues the input to the second line
% (for convenient cut ant paste).
% Furthermore, the command |latex| can be replaced by any
% of its alternative versions such as |pdflatex|.
%
% %%%%%%%%%%%%%%%%%%%%%%%%%%%%%%%%%%%%%%%%%%%%%%%%%%%%%%%%%%%%%%%%%%%%%%%%%%%%%%
% %%%%%%%%%%%%%%%%%%%%%%%%%%%%%%%%%%%%%%%%%%%%%%%%%%%%%%%%%%%%%%%%%%%%%%%%%%%%%%
% \section{Implementation}
%\iffalse
%<*package>
%\fi
%
% This section describes the definitions file |childdoc.def|.

% The definitions cannot be loaded using |\usepackage| or |\RequirePackage|
% which has a mechanism to prevent loading a style file more than once.
% When loading the definitions by means of |\input|
% multiple instances have to be prevented manually:
%\iffalse
%This code needs to be before the `\ProvidesFile' directive
%which is defined at the beginning of this file.
%Therefore it is also placed there and commented out here.
%</package>
%<*discard>
%\fi
%    \begin{macrocode}
\ifdefined\childdocmain\endinput\fi
%    \end{macrocode}
%\iffalse
%</discard>
%<*package>
%\fi
%
% \macro{\ifchilddoc}
% \macro{\ifchilddocmanual}
% The conditional |\ifchilddoc| tells whether a
% child (true) or main (false) document is being compiled.
% The conditional |\ifchilddocmanual| tells whether
% the |\includeonly| mechanism is used (false) or
% the selection of child files must be performed manually (true).
% The definitions initialise to false:
%    \begin{macrocode}
\newif\ifchilddoc
\newif\ifchilddocmanual
%    \end{macrocode}

% \macro{\childdocname}
% \macro{\childdocjob}
% The macro |\childdocname| stores the name of the main document
% to be compiled. The macro |\childdocjob| stores the name of
% the document on which the \LaTeX{} compiler was originally invoked.
% The content of |\jobname| cannot be compared
% to filenames specified in the source due to different catcodes.
% The following code rescans |\jobname|, stores the result
% in |\childdocname| and saves a copy in |\childdocjob|:
%    \begin{macrocode}
\edef\childdocname{\scantokens\expandafter{\jobname\noexpand}}
\let\childdocjob\childdocname
%    \end{macrocode}

% \macro{\childdocdisable}
% The macro |\childdocdisable| prevents the main file
% from being processed more than once.
% At this stage, the main document command |\childdocmain|
% is assumed to be called once again where it should do nothing.
% Any subsequent call to it should prevent
% a secondary processing of the main document
% It overwrites the forwarding commands
% |\childdocof| and |\childdocforward|
% with empty macros to prevent further inclusions of the main document:
%    \begin{macrocode}
\newcommand{\childdocdisable}
{
  \renewcommand{\childdocmain}[1]{\renewcommand{\childdocmain}[1]{\endinput}}
  \renewcommand{\childdocof}[1]{}
  \renewcommand{\childdocby}[2][]{}
  \renewcommand{\childdocforward}[2][]{}
  \renewcommand{\childdocdisable}{}
}
%    \end{macrocode}

% \macro{\childdocmain}
% The macro |\childdocmain| is to be called at the top of the main file
% with nothing or the main filename (without extension) as argument.
% First, it breaks loops.
% If the argument is not empty and does not match |\childdocname|
% (which is set by the first inclusion of |childdoc.def|),
% |\ifchilddoc| is set to true, |\includeonly| is applied to the child file
% and |\jobname| is set to the main file
% (for proper handling of |.aux| files):
%    \begin{macrocode}
\newcommand{\childdocmain}[1]
{
  \childdocdisable\childdocmain{}
  \if?#1?\else
    \begingroup
      \def\childdoctmp{#1}
      \ifx\childdoctmp\childdocname
        \def\childdoctmp{}
      \else
        \def\childdoctmp
        {
          \childdoctrue
          \includeonly{\childdocname}
          \def\childdocjob{#1}
          \def\jobname{#1}
        }
      \fi
      \expandafter
    \endgroup
    \childdoctmp
  \fi
}
%    \end{macrocode}

% \macro{\childdocof}
% The command |\childdocof| redirects
% compilation to the main file |#1|.
%    \begin{macrocode}
\newcommand{\childdocof}[1]
{
  \childdocdisable
  \childdoctrue
  \includeonly{\childdocname}
  \def\jobname{#1}
  \def\childdocjob{#1}
  \input{#1}
}
%    \end{macrocode}

% \macro{\childdocby}
% The command |\childdocby| ....
%    \begin{macrocode}
\newcommand{\childdocby}[2][]
{
  \childdocdisable
  \childdoctrue
  \childdocmanualtrue
  \if?#1?\else
    \def\jobname{#2}
  \fi
  \def\childdocjob{#2}
  \input{#2}
  \endinput
}
%    \end{macrocode}

% \macro{\childdocforward}
% The command |\childdocforward| redirects
% compilation to the main file or
% (if the optional argument is given) a child file.
% Parameters are set as if the main file
% or a child file starting with |\childdocof| was compiled.
% Then compilation is handed over to the main file:
%    \begin{macrocode}
\newcommand{\childdocforward}[2][]
{
  \begingroup
    \if?#1?
      \def\childdoctmp
      {
        \def\childdocname{#2}
        \def\childdocjob{#2}
        \def\jobname{#2}
        \input{#2}
        \endinput
      }
    \else
      \def\childdoctmp
      {
        \childdocdisable
        \def\childdocname{#2}
        \childdoctrue
        \includeonly{#2}
        \def\childdocjob{#1}
        \def\jobname{#1}
        \input{#1}
        \endinput
      }
    \fi
    \expandafter
  \endgroup
  \childdoctmp
}
%    \end{macrocode}

% \macro{\childdocforwardprefix}
% The command |\childdocforwardprefix| redirects
% compilation to the main or a child file by means of a pattern.
% The prefix |#1| in the current filename is replaced by |#2|
% and the suffix of the current filename is kept
% (it is assumed that the filename does not contain the substring `|~~~|'
% which is used as a delimiter).
% Compilation is handed over to the new file by |\childdocforward|:
%    \begin{macrocode}
\newcommand{\childdocforwardprefix}[3][]
{
  \begingroup
    \def\childdocextract #2##1~~~{\def\childdoctmp{\childdocforward[#1]{#3##1}}}
    \expandafter\childdocextract\childdocname~~~
    \expandafter
  \endgroup
  \childdoctmp
}
%    \end{macrocode}

% \macro{\childdoc}
% The deprecated macro |\childdoc| is a legacy version of |\childdocmain|:
%    \begin{macrocode}
\newcommand{\childdoc}{\childdocmain}
%    \end{macrocode}

% \macro{\childdocredirect}
% The deprecated macro |\childdocredirect| is a legacy version
% of |\childdocforward| and |\childdocforwardprefix|:
%    \begin{macrocode}
\newcommand{\childdocredirect}[2][]
{
  \begingroup
    \if?#1?
      \def\childdoctmp{\childdocforward{#2}}
    \else
      \def\childdoctmp{\childdocforwardprefix{#1}{#2}}
    \fi
    \expandafter
  \endgroup
  \childdoctmp
}
%    \end{macrocode}

%\iffalse
%</package>
%\fi
%
\endinput
|\\
|\childdocmain{}|\\
\end{tabular}
\end{center}
at the very top of the main \LaTeX{} file,
in particular \emph{before} the |\documentclass| statement!
The argument of |\childdocmain| should be left empty
(but it must be present).

%%%%%%%%%%%%%%%%%%%%%%%%%%%%%%%%%%%%%%%%
\DescribeMacro{\childdocof}
Furthermore, add the commands
\begin{center}
\begin{tabular}{l}
|% \iffalse
%
% childdoc.dtx Copyright (C) 2017-2018 Niklas Beisert
%
% This work may be distributed and/or modified under the
% conditions of the LaTeX Project Public License, either version 1.3
% of this license or (at your option) any later version.
% The latest version of this license is in
%   http://www.latex-project.org/lppl.txt
% and version 1.3 or later is part of all distributions of LaTeX
% version 2005/12/01 or later.
%
% This work has the LPPL maintenance status `maintained'.
%
% The Current Maintainer of this work is Niklas Beisert.
%
% This work consists of the files childdoc.dtx and childdoc.ins
% and the derived files childdoc.def and cdocsamp.tex with
% cdocsch1.tex, cdocsch2.tex, cdocsdrf.tex, cdocsfn1.tex, cdocsfn2.tex.
%
%<package>\ifdefined\childdocmain\endinput\fi
%<package>\ProvidesFile{childdoc.def}[2018/12/30 v2.0 child document driver]
%<samplemain>\ProvidesFile{cdocsamp.tex}[2018/12/30 v2.0 sample for childdoc]
%<*driver>
%\ProvidesFile{childdoc.drv}[2018/12/30 v2.0 childdoc reference manual file]
\PassOptionsToClass{10pt,a4paper}{article}
\documentclass{ltxdoc}

\usepackage[margin=35mm]{geometry}
\usepackage{hyperref}
\usepackage{hyperxmp}
\usepackage[usenames]{color}

\hypersetup{colorlinks=true}
\hypersetup{pdfstartview=FitH}
\hypersetup{pdfpagemode=UseNone}
\hypersetup{pdfsource={}}
\hypersetup{pdflang={en-UK}}
\hypersetup{pdfcopyright={Copyright 2017-2018 Niklas Beisert.
  This work may be distributed and/or modified under the
  conditions of the LaTeX Project Public License, either version 1.3
  of this license or (at your option) any later version.}}
\hypersetup{pdflicenseurl={http://www.latex-project.org/lppl.txt}}
\hypersetup{pdfcontactaddress={ETH Zurich, ITP, HIT K,
  Wolfgang-Pauli-Strasse 27}}
\hypersetup{pdfcontactpostcode={8093}}
\hypersetup{pdfcontactcity={Zurich}}
\hypersetup{pdfcontactcountry={Switzerland}}
\hypersetup{pdfcontactemail={nbeisert@itp.phys.ethz.ch}}
\hypersetup{pdfcontacturl={http://people.phys.ethz.ch/\xmptilde nbeisert/}}

\newcommand{\secref}[1]{\hyperref[#1]{section \ref*{#1}}}

\parskip1ex
\parindent0pt
\let\olditemize\itemize
\def\itemize{\olditemize\parskip0pt}

\begin{document}

\title{The \textsf{childdoc} Package}
\hypersetup{pdftitle={The childdoc Package}}
\author{Niklas Beisert\\[2ex]
  Institut f\"ur Theoretische Physik\\
  Eidgen\"ossische Technische Hochschule Z\"urich\\
  Wolfgang-Pauli-Strasse 27, 8093 Z\"urich, Switzerland\\[1ex]
  \href{mailto:nbeisert@itp.phys.ethz.ch}
  {\texttt{nbeisert@itp.phys.ethz.ch}}}
\hypersetup{pdfauthor={Niklas Beisert}}
\hypersetup{pdfsubject={Manual for the LaTeX2e Package childdoc}}
\date{30 December 2018, \textsf{v2.0}}
\maketitle

\begin{abstract}\noindent
\textsf{childdoc} is a \LaTeXe{} package
that enables the direct compilation
of document sections included by |\include|
to individual files.
\end{abstract}

\begingroup
\parskip0ex
\tableofcontents
\endgroup

%%%%%%%%%%%%%%%%%%%%%%%%%%%%%%%%%%%%%%%%%%%%%%%%%%%%%%%%%%%%%%%%%%%%%%%%%%%%%%%%
%%%%%%%%%%%%%%%%%%%%%%%%%%%%%%%%%%%%%%%%%%%%%%%%%%%%%%%%%%%%%%%%%%%%%%%%%%%%%%%%
\section{Introduction}

\LaTeX{} provides a mechanism to structure a large document (such as a book)
into a main file and several child files (containing the chapters)
using the |\include| command.
This mechanism is beneficial for documents
which span hundreds of pages in order to
make the source file(s) more manageable.
Moreover, compilation can be restricted to
selected child files by means of the |\includeonly| command.
The latter feature can be used to reduce the compilation time while editing
(this was significantly more useful in the earlier days of \LaTeX{})
or to generate a smaller document which is easier to navigate.
Another application of |\includeonly| is to generate
documents consisting of selected parts of the complete document.

However, there are a few drawbacks of the plain |\include| mechanism:
\begin{itemize}
\item
The child files cannot be compiled on their own,
they can only be compiled via the main file.
A naive editing environment
(such as a text editor with an option
to have the current file processed by \LaTeX)
may require one to switch to the main file before compiling;
attempting to compile the child file produces errors.
\item
The main file must be modified (each time)
to adjust the |\includeonly| command
to the present needs. This easily leaves the main file in a messy state.
\item
The generated document will always carry the filename
of the main document. This is inconvenient if
several child files are to be compiled and
to be kept for distribution.
\end{itemize}

The present package provides a simple interface
to make child files individually compilable by \LaTeX{}.
Compiling a child file then has the same effect as compiling
the main file with an |\includeonly| command
to select the appropriate child.
Moreover the generated document will carry the name of the child
rather than the main file.
This resolves all three above issues.

This feature is meant to make the editing of books,
thesis documents and lecture notes somewhat more convenient.
However, the package can also be used efficiently for
composing a series of documents (such as exercise sheets)
which are typically distributed individually.
It then assists the author in generating the individual documents
(potentially in different versions)
as well as a document containing the collected series.
Another application is in developing style files
or other kinds of included material
where compilation of the style file could redirect
to a sample or test file.

%%%%%%%%%%%%%%%%%%%%%%%%%%%%%%%%%%%%%%%%%%%%%%%%%%%%%%%%%%%%%%%%%%%%%%%%%%%%%%%%
%%%%%%%%%%%%%%%%%%%%%%%%%%%%%%%%%%%%%%%%%%%%%%%%%%%%%%%%%%%%%%%%%%%%%%%%%%%%%%%%
\section{Usage}

First of all, the package \textsf{childdoc} is \emph{not} a standard
\LaTeXe{} |.sty| style file! Therefore it needs to be invoked in
a non-standard way.

%%%%%%%%%%%%%%%%%%%%%%%%%%%%%%%%%%%%%%%%%%%%%%%%%%%%%%%%%%%%%%%%%%%%%%%%%%%%%%%%
\subsection{Included Files}
\label{sec:include}

%%%%%%%%%%%%%%%%%%%%%%%%%%%%%%%%%%%%%%%%
\DescribeMacro{\childdocmain}
To use the package, add the commands
\begin{center}
\begin{tabular}{l}
|\input{childdoc.def}|\\
|\childdocmain{}|\\
\end{tabular}
\end{center}
at the very top of the main \LaTeX{} file,
in particular \emph{before} the |\documentclass| statement!
The argument of |\childdocmain| should be left empty
(but it must be present).

%%%%%%%%%%%%%%%%%%%%%%%%%%%%%%%%%%%%%%%%
\DescribeMacro{\childdocof}
Furthermore, add the commands
\begin{center}
\begin{tabular}{l}
|\input{childdoc.def}|\\
|\childdocof{|\textit{main}|}|\\
\end{tabular}
\end{center}
at the top of every child file \textit{child}
which is included by |\include{|\textit{child}|}|
from within the main file
(or at least for those files to be compiled individually).
The argument \textit{main} must be the filename of the main file.

There are a couple of
considerations in setting up the main and child documents:

%%%%%%%%%%%%%%%%%%%%%%%%%%%%%%%%%%%%%%%%
\paragraph{Restrictions.}

Please note the following restrictions:
\begin{itemize}
\item
|\childdocmain| must be called with one argument \textit{main}
to ensure compatibility with earlier version of the package.
It must either be empty (|\childdocmain{}|)
or precisely match the filename of the main file in which it is specified.
See \secref{sec:detection} for further information.
\item
The filename \textit{main} must be specified without the |.tex| extension.
\item
The filename \textit{main} is case sensitive
(even in case-insensitive file systems)
due to internal string comparison.
\item
The argument \textit{main} should be fully expanded, it cannot be a macro.
\item
Subdirectories and special characters should be avoided in filenames.
\item
The command |\childdocmain{|\textit{main}|}| must be followed by a whitespace.
It should not be followed immediately by another command
or by a comment mark `|%|'.
This is because the \TeX{} parser reads the token immediately following
the argument of |\childdocmain| and puts it
at the beginning of every child section;
however, a white\-space is ignored.
\end{itemize}

%%%%%%%%%%%%%%%%%%%%%%%%%%%%%%%%%%%%%%%%
\paragraph{Content of Main File.}

It is advisable to place all content in the child files included by |\include|.
Any output contained in the main file will appear in all child documents
unless suppressed manually;
it cannot be suppressed automatically by the |\includeonly| directive
and thus should normally be avoided.
A method to include some content in the main file
by means of conditional processing is described in \secref{sec:conditional}.

%%%%%%%%%%%%%%%%%%%%%%%%%%%%%%%%%%%%%%%%
\paragraph{Page Numbering.}

When only a part of the document is compiled,
the appropriate numbering of pages
(as well as other status parameters)
is determined from the |.aux| files.
The latter contain information from previous passes.
However this information needs to propagate through
all intermediate child documents.
Therefore the page numbering in child documents may well
be inconsistent until the complete document is compiled at least once.

A useful (if unconventional) way to always ensure a consistent
page numbering is to restart the numbering in each child document
and denote the pages by `\textit{child}|.|\textit{page}'
where \textit{child} represents the chapter/section number of the child file.
This can be achieved by the command
|\numberwithin{page}{|\textit{child}|}|
of the \textsf{amsmath} package
where \textit{child} can be |chapter| or |section|
depending on the chosen structuring.
Alternatively, one can modify the macro |\thepage| appropriately
and reset the counter |page| at the start of each child file.

%%%%%%%%%%%%%%%%%%%%%%%%%%%%%%%%%%%%%%%%%%%%%%%%%%%%%%%%%%%%%%%%%%%%%%%%%%%%%%%%
\subsection{Conditional Processing}
\label{sec:conditional}

The package provides a mechanism to compile different versions
of a document. To customise the versions further some conditional processing
can come in handy to distinguish which version is being compiled.
The package provides two macros to describe the compilation context:

%%%%%%%%%%%%%%%%%%%%%%%%%%%%%%%%%%%%%%%%
\DescribeMacro{\ifchilddoc}
The conditional |\ifchilddoc| distinguishes between the compilation of
child documents and the main document:
%
\begin{center}
|\ifchilddoc |\textit{child-code}| |[|\||else |\textit{main-code}]| \||fi|
\end{center}

%%%%%%%%%%%%%%%%%%%%%%%%%%%%%%%%%%%%%%%%
\DescribeMacro{\childdocname}
\DescribeMacro{\childdocjob}
The macro |\childdocname| contains the filename (without extension)
of the main or child file being processed.
Note that |\childdocjob| will always contain the name of the main file.

%%%%%%%%%%%%%%%%%%%%%%%%%%%%%%%%%%%%%%%%
\paragraph{Title Page.}

Conditional processing can be used to include a title or banner page
in the main document when proper precautions are taken.
Importantly, the code in the main file should ensure that the page counter
(as well as other status parameters which are stored in the |.aux| files)
takes the same value after the conditional processing.
Otherwise the page numbers may take divergent values
depending on which part is compiled.

For example, a title page could be declared by:
%
\begin{center}
\begin{tabular}{l}
|\ifchilddoc\||else|\\
|\addtocounter{page}{-1}|\\
\textit{code for title page}\\
|\newpage|\\
|\||fi|
\end{tabular}
\end{center}
%
A banner page for the child documents can be generated by:
%
\begin{center}
\begin{tabular}{l}
|\ifchilddoc|\\
|\addtocounter{page}{-1}|\\
\textit{code for banner page}\\
|\newpage|\\
|\||fi|
\end{tabular}
\end{center}
%
Here one could write a message such as:
\begin{center}
|This is the part \childdocname{} of \childdocjob{}.|
\end{center}

%%%%%%%%%%%%%%%%%%%%%%%%%%%%%%%%%%%%%%%%%%%%%%%%%%%%%%%%%%%%%%%%%%%%%%%%%%%%%%%%
\subsection{Flags}
\label{sec:flags}

The package makes it easy to generate different versions
of the main or child documents.
To this end compilation flags can be defined
and assigned different default values.
They will be particularly useful in conjunction
with the forwarding mechanism described in \secref{sec:forward}.

For example, it may be useful to have a flag |\version|
which can be set to |draft| or |final|.
The document source will contain some conditional code
depending on the value of |\version|.
Suppose further, the flag should default to |final| for the main file
and to |draft| for child files
which is a natural assignment for editing the document.
This is achieved by placing the following code
in the preamble of the main document
(below the |\childdocmain| directive):
%
\begin{center}
\begin{tabular}{l}
|\ifchilddoc|\\
|\providecommand{\version}{draft}|\\
|\||else|\\
|\providecommand{\version}{final}|\\
|\||fi|
\end{tabular}
\end{center}
%
The definition by |\providecommand| makes sure
that previous definitions are not overwritten.
Further statements |\providecommand{\version}{...}|
can thus be added before the above code to override it.

For the main file, one might add a line
(between |\childdocmain| and the above block)
%
\begin{center}
|%\ifchilddoc\||else\providecommand{\version}{draft}\||fi|
\end{center}
%
which can be uncommented to produce a draft version.
Likewise one can add a line to the very top of a child file
(above the |\childdocof{|\textit{main}|}| directive)
%
\begin{center}
|%\providecommand{\version}{final}|
\end{center}
%
which can be uncommented to produce the final version of this child document.

%%%%%%%%%%%%%%%%%%%%%%%%%%%%%%%%%%%%%%%%%%%%%%%%%%%%%%%%%%%%%%%%%%%%%%%%%%%%%%%%
\subsection{Forwarding}
\label{sec:forward}

Different versions of the main or child documents
using compilation flags as described in \secref{sec:flags}
can be (permanently) stored in different files
for convenient compilation, viewing and distribution.
To this end, the package defines a command
to pass on compilation to a different file:

%%%%%%%%%%%%%%%%%%%%%%%%%%%%%%%%%%%%%%%%
\DescribeMacro{\childdocforward}
The command |\childdocforward| redirects processing to
another source file:
%
\begin{center}
\begin{tabular}{l}
|\input{childdoc.def}|\\
|\childdocforward[|\textit{main}|]{|\textit{dest}|}|\\
\end{tabular}
\end{center}
%
The argument \textit{dest} is the destination file
(without extension).
It should be the main file or one of the child files.
Note that further \textsf{childdoc} directives
such as |\childdocof| and |\childdocforward|
in the indicated file will be processed in this form.
The optional argument \textit{main}
passes on directly to the main file \textit{main}
while pretending to compile the child \textit{dest}.
This form behaves as if \textit{dest}
issues |\childdocof{|\textit{main}|}| right away,
and no further \textsf{childdoc} directives will be processed.

%%%%%%%%%%%%%%%%%%%%%%%%%%%%%%%%%%%%%%%%
\DescribeMacro{\...prefix}
In the alternative form |\childdocforwardprefix|,
%
\begin{center}
\begin{tabular}{l}
|\input{childdoc.def}|\\
|\childdocforwardprefix[|\textit{main}|]{|\textit{prefix}|}{|\textit{dest}|}|
\end{tabular}
\end{center}
%
the destination file is determined by a pattern
depending on the current file:
To make this work, the current file must be called
`{\textit{prefix}\hspace{0.2em}\textit{suffix}}'
with \textit{prefix} matching precisely the argument.
Processing is then passed on to the file
`{\textit{dest}\hspace{0.2em}\textit{suffix}}'.
Surely, the same effect is achieved by
directly specifying the
argument `{\textit{dest}\hspace{0.2em}\textit{suffix}}'
in the first form.
However, that requires to set up a different file
for each child. With the alternative form of the command
all these files can have exactly the same content
which simplifies setting them up and maintaining them.

For example, the following file |draft.tex|
with a compilation flag |\version| as described in \secref{sec:flags}
compiles the main document as a draft:
%
\begin{center}
\begin{tabular}{l}
|\def\version{draft}|\\
|\input{childdoc.def}|\\
|\childdocforward{|\textit{main}|}|
\end{tabular}
\end{center}
%
Likewise, the following files |final|\textit{nn}|.tex|
compile the final version of the child document
|child|\textit{nn}|.tex|:
%
\begin{center}
\begin{tabular}{l}
|\def\version{final}|\\
|\input{childdoc.def}|\\
|\childdocforwardprefix{final}{child}|
\end{tabular}
\end{center}
%

Note that when several versions of a main file and/or of each child file
are to be generated, it may be convenient to set up a |Makefile| or
shell script to automatise the process.

%%%%%%%%%%%%%%%%%%%%%%%%%%%%%%%%%%%%%%%%%%%%%%%%%%%%%%%%%%%%%%%%%%%%%%%%%%%%%%%%
\subsection{Command Line Processing}
\label{sec:commandline}

The effect of redirection files can also be achieved by invoking
the \LaTeX{} compiler with a more elaborate command line.
Most conveniently this should be done as part
of a shell script or a |Makefile|.

When using \textsf{childdoc} in the main file, the following
command lines effectively perform a redirection
(note that depending on the shell being used,
backslashes may have to be doubled: `|\|' $\to$ `|\\|'):
%
\begin{center}
|... -jobname "|\textit{target}|" |\\|"|[\textit{flags}]%
|\input{childdoc.def}\childdocforward[|\textit{main}|]{|\textit{dest}|}"|
\end{center}
%
Here \textit{target} is the name of the output file,
\textit{main} is the name of the main file
and \textit{dest} is the name of the main or child file to be processed
(all filenames without extensions).
The optional argument \textit{main} can be omitted
if \textit{main} matches \textit{dest}.
Optionally, compilation \textit{flags} can be defined via |\def| commands.
This command line makes the \TeX{} engine believe
it is compiling the file \textit{target}
whose content is specified as the latter parameter.
The provided code then forwards the processing to
\textit{main} or \textit{dest} as described in \secref{sec:forward}.

%%%%%%%%%%%%%%%%%%%%%%%%%%%%%%%%%%%%%%%%%%%%%%%%%%%%%%%%%%%%%%%%%%%%%%%%%%%%%%%%
\subsection{Include by Input}
\label{sec:input}

Including child documents by |\include| has some restrictions by design.
Most notably, the content of a child document always occupies
its own set of pages; pages cannot be shared between child documents.
Usually, this behaviour makes perfect sense
because each child document contain an essential part of the document.
However, in some situations it may be desirable to compose
a document from a collection of parts
without having mandatory page breaks between then.
For this case, the package
provides a mechanism to include parts
by |\input| which can also be processed individually.
However, by construction this mechanism
requires manual handling of the content to be output.

%%%%%%%%%%%%%%%%%%%%%%%%%%%%%%%%%%%%%%%%
\DescribeMacro{\ifchilddocmanual}
The main file should be prepared as usual, see \secref{sec:include}.
However, the document body must make a distinction
between processing of an individual part and of the main document, e.g.:
%
\begin{center}
\begin{tabular}{l}
|\ifchilddocmanual|\\
|\input{\childdocname}|\\
|\||else|\\
\textit{document body with }|\input{|\textit{part}|}|\\
|\||fi|
\end{tabular}
\end{center}
%
The conditional |\ifchilddocmanual| is true whenever
a part to be included by |\input| is being compiled,
and the name of the part is stored in |\childdocname|.

%%%%%%%%%%%%%%%%%%%%%%%%%%%%%%%%%%%%%%%%
\DescribeMacro{\childdocby}
Each part to be included by |\input| should start with:
%
\begin{center}
\begin{tabular}{l}
|\input{childdoc.def}|\\
|\childdocby{|\textit{main}|}|\\
\end{tabular}
\end{center}
%
The directive |\childdocby| is similar to |\childdocof|
described in \secref{sec:include},
but the subsequent selection of content must be done manually.
To that end, both |\ifchilddoc| and |\ifchilddocmanual|
will be true upon processing of a part,
and the name of the part is stored in |\childdocname|.
Note that |\jobname| will be set to the filename of the current part
so that each part receives an individual |.aux| file
that does not interfere with the |.aux| file(s) of the main document.
This behaviour can be altered by the alternative form
|\childdocby[*]{|\textit{main}|}| (with a non-empty optional argument)
which uses the |.aux| file of the main document
by setting |\jobname| to \textit{main}.

%%%%%%%%%%%%%%%%%%%%%%%%%%%%%%%%%%%%%%%%%%%%%%%%%%%%%%%%%%%%%%%%%%%%%%%%%%%%%%%%
\subsection{Driver Development}
\label{sec:driver}

The \textsf{childdoc} mechanism can also be use for the development
of definition files such as \LaTeX{} styles or classes.
This case differs from the above setup with multiple parts
included by |\include| in that no |\includeonly| should be invoked.
This can be achieved by starting the include file
(before |\ProvidesPackage|) with:
%
\begin{center}
\begin{tabular}{l}
|\input{childdoc.def}|\\
|\childdocforward{|\textit{main}|}|\\
\end{tabular}
\end{center}
%
or alternatively with:
%
\begin{center}
\begin{tabular}{l}
|\input{childdoc.def}|\\
|\childdocby{|\textit{main}|}|\\
\end{tabular}
\end{center}
%
Both forms have slightly different effects as described above.
The main file is prepared as usual, see \secref{sec:include}.

%%%%%%%%%%%%%%%%%%%%%%%%%%%%%%%%%%%%%%%%%%%%%%%%%%%%%%%%%%%%%%%%%%%%%%%%%%%%%%%%
\subsection{Legacy Detection}
\label{sec:detection}

The directive |\childdocmain| in the main file can detect
whether the complete document or merely a child is to be compiled
even without using the directive |\childdocof|.
This method is deprecated because it is less robust
and there is no compelling reason to use it;
it is merely provided for backward compatibility
and it may be removed in future versions.

If the detection mechanism is to be used,
it is mandatory to correctly specify
the filename of the main file as the argument of |\childdocmain|:
%
\begin{center}
\begin{tabular}{l}
|\input{childdoc.def}|\\
|\childdocmain{|\textit{main}|}|\\
\end{tabular}
\end{center}
%
If |\jobname| does not match the argument \textit{main} of |\childdocmain|,
it is assumed that |\jobname| points to the child file to be compiled.
When using |\childdocmain| with the main file specified as argument,
it suffices to start a child file
with just |\input{|\textit{main}|}|
without loading of the package and using |\childdocof|.
If instead all processing is done
with the appropriate \textsf{childdoc} directives,
the argument of \textit{main} of |\childdocmain| can be empty.

An alternative version of the command line processing described
in \secref{sec:commandline} using the detection mechanism reads:
%
\begin{center}
|... -jobname "|\textit{target}|" "|[\textit{flags}]%
[|\def\jobname{|\textit{dest}|}|]|\input{|\textit{main}|}"|
\end{center}

%%%%%%%%%%%%%%%%%%%%%%%%%%%%%%%%%%%%%%%%%%%%%%%%%%%%%%%%%%%%%%%%%%%%%%%%%%%%%%%%
\subsection{Manual Code}
\label{sec:manual}

In case one cannot be certain whether the definitions file |childdoc.def|
is installed on the target \TeX{} distribution
and one prefers not to ship it,
it is conceivable to paste a few relevant commands into the sources.

To that end, drop all statements |\input{childdoc.def}|
and perform the replacements as outlined below.
Instead of |\childdocmain{|\textit{main}|}| add the following code
to the top of the main file:
%
\begin{center}
\begin{tabular}{l}
|\||ifdefined\childdocname\endinput\||fi\newif\ifchilddoc|\\
|\edef\childdocname{\scantokens\expandafter{\jobname\noexpand}}|\\
|\def\childdocmain{|\textit{main}|}\||ifx\childdocmain\childdocname\||else|\\
|\childdoctrue\includeonly{\childdocname}\let\jobname\childdocmain\||fi|\\
\end{tabular}
\end{center}
%
Instead of |\childdocof{|\textit{main}|}| just include the main file
at the top of each child file:
%
\begin{center}
|\input{|\textit{main}|}|
\end{center}
%
A simple redirection |\childdocforward{|\textit{dest}|}| is achieved by:
%
\begin{center}
|\def\jobname{|\textit{dest}|}\input{\jobname}|
\end{center}
%
The redirection with prefix
|\childdocforwardprefix[|\textit{prefix}|]{|\textit{dest}|}|
is accomplished by:
%
\begin{center}
\begin{tabular}{l}
|{\edef\jobname{\scantokens\expandafter{\jobname\noexpand}}|\\
|\def\redirectjob |\textit{prefix}|#1~~~{\gdef\jobname{|\textit{dest}|#1}}|\\
|\expandafter\redirectjob\jobname~~~}\input{\jobname}|
\end{tabular}
\end{center}

In an alternative approach,
child documents can be compiled by a specific command line
without additional code or specific definitions:
%
\begin{center}
|... -jobname "|\textit{target}|" "|[\textit{flags}]%
|\includeonly{|\textit{dest}|}\input{|\textit{main}|}"|
\end{center}
%

%%%%%%%%%%%%%%%%%%%%%%%%%%%%%%%%%%%%%%%%%%%%%%%%%%%%%%%%%%%%%%%%%%%%%%%%%%%%%%%%
%%%%%%%%%%%%%%%%%%%%%%%%%%%%%%%%%%%%%%%%%%%%%%%%%%%%%%%%%%%%%%%%%%%%%%%%%%%%%%%%
\section{Information}

%%%%%%%%%%%%%%%%%%%%%%%%%%%%%%%%%%%%%%%%%%%%%%%%%%%%%%%%%%%%%%%%%%%%%%%%%%%%%%%%
\subsection{Copyright}

Copyright \copyright{} 2017--2018 Niklas Beisert

This work may be distributed and/or modified under the
conditions of the \LaTeX{} Project Public License, either version 1.3
of this license or (at your option) any later version.
The latest version of this license is in
  \url{http://www.latex-project.org/lppl.txt}
and version 1.3 or later is part of all distributions of \LaTeX{}
version 2005/12/01 or later.

This work has the LPPL maintenance status `maintained'.

The Current Maintainer of this work is Niklas Beisert.

This work consists of the files |README.txt|, |childdoc.ins| and |childdoc.dtx|
as well as the derived files |childdoc.def|, |cdocsamp.tex|
with |cdocsch1.tex|, |cdocsch2.tex|, |cdocspt3.tex|, |cdocspt4.tex|,
|cdocsdrf.tex|, |cdocsfn1.tex|, |cdocsfn2.tex|
as well as |childdoc.pdf|.

%%%%%%%%%%%%%%%%%%%%%%%%%%%%%%%%%%%%%%%%%%%%%%%%%%%%%%%%%%%%%%%%%%%%%%%%%%%%%%%%
\subsection{Files and Installation}

The package consists of the files:
%
\begin{center}
\begin{tabular}{ll}
    |README.txt|   & readme file \\
    |childdoc.ins| & installation file \\
    |childdoc.dtx| & source file \\
    |childdoc.def| & definition file \\
    |cdocsamp.tex| & sample main file \\
    |cdocsch1.tex| & sample include file \\
    |cdocsch2.tex| & sample include file \\
    |cdocspt3.tex| & sample part file \\
    |cdocspt4.tex| & sample part file \\
    |cdocsdrf.tex| & sample redirection file \\
    |cdocsfn1.tex| & sample redirection file \\
    |cdocsfn2.tex| & sample redirection file \\
    |childdoc.pdf| & manual
\end{tabular}
\end{center}
%
The distribution consists of the files
|README.txt|, |childdoc.ins| and |childdoc.dtx|.
%
\begin{itemize}
\item
Run (pdf)\LaTeX{} on |childdoc.dtx|
to compile the manual |childdoc.pdf| (this file).
\item
Run \LaTeX{} on |childdoc.ins| to create the definitions file |childdoc.def|
and the sample |cdocsamp.tex| with include files
|cdocsch1.tex|, |cdocsch2.tex|, |cdocspt3.tex|, |cdocspt4.tex|,
|cdocsdrf.tex|, |cdocsfn1.tex|, |cdocsfn2.tex|.
Then copy the file |childdoc.def| to an appropriate directory of your \LaTeX{}
distribution, e.g.\ \textit{texmf-root}|/tex/latex/childdoc|.
\end{itemize}

%%%%%%%%%%%%%%%%%%%%%%%%%%%%%%%%%%%%%%%%%%%%%%%%%%%%%%%%%%%%%%%%%%%%%%%%%%%%%%%%
\subsection{Related CTAN Packages}

There are several other packages which offer a similar functionality:
%
\begin{itemize}
\item
The packages
\href{http://ctan.org/pkg/docmute}{\textsf{docmute}},
\href{http://ctan.org/pkg/includex}{\textsf{includex}} and
\href{http://ctan.org/pkg/standalone}{\textsf{standalone}}
provide commands to include only the document body of
a child file thus allowing both files to be compiled individually.
\item
The packages \href{http://ctan.org/pkg/subdocs}{\textsf{subdocs}}
and \href{http://ctan.org/pkg/subfiles}{\textsf{subfiles}}
provide structures in which the main and child documents can be
encapsulated and allowing them to be compiled individually.
The inclusion mechanism is different from the conventional |\include|.
\item
The package \href{http://ctan.org/pkg/combine}{\textsf{combine}}
is an elaborate solution to combine several documents into one.
\end{itemize}
%
See also the CTAN topic \href{http://ctan.org/topic/subdocs}{\textsf{subdocs}}
for further related packages.
The present package differs from the above solutions in that
a document structure constructed with the conventional |\include| mechanism
just needs two extra commands at the top of every file
such that all constituent files can be compiled individually.

%%%%%%%%%%%%%%%%%%%%%%%%%%%%%%%%%%%%%%%%%%%%%%%%%%%%%%%%%%%%%%%%%%%%%%%%%%%%%%%%
%\subsection{Feature Suggestions}
%
%The following is a list of features which may be useful for future
%versions of this package:
%%
%\begin{itemize}
%\item
%\ldots
%\end{itemize}

%%%%%%%%%%%%%%%%%%%%%%%%%%%%%%%%%%%%%%%%%%%%%%%%%%%%%%%%%%%%%%%%%%%%%%%%%%%%%%%%
\subsection{Revision History}

%%%%%%%%%%%%%%%%%%%%%%%%%%%%%%%%%%%%%%%%
\paragraph{v2.0:} 2018/12/30

\begin{itemize}
\item
immediate forward processing
\item
added |\childdocby| mechanism
\item
manual restructured
\end{itemize}

%%%%%%%%%%%%%%%%%%%%%%%%%%%%%%%%%%%%%%%%
\paragraph{v1.6:} 2018/01/17

\begin{itemize}
\item
application for development of include files
\item
corrections to manual
\end{itemize}

%%%%%%%%%%%%%%%%%%%%%%%%%%%%%%%%%%%%%%%%
\paragraph{v1.5:} 2017/05/21

\begin{itemize}
\item
more complete structuring introduced
\item
|\childdocof| introduced
\item
|\childdoc| renamed to |\childdocmain|
\item
|\childredirect| renamed to |\childdocforward| and |\childdocforwardprefix|
and functionality expanded
\end{itemize}

%%%%%%%%%%%%%%%%%%%%%%%%%%%%%%%%%%%%%%%%
\paragraph{v1.0:} 2017/04/27

\begin{itemize}
\item
manual and install package
\item
first version published on CTAN
\end{itemize}

%%%%%%%%%%%%%%%%%%%%%%%%%%%%%%%%%%%%%%%%
\paragraph{v0.6:} 2017/04/26

\begin{itemize}
\item
redirection mechanism added
\end{itemize}

%%%%%%%%%%%%%%%%%%%%%%%%%%%%%%%%%%%%%%%%
\paragraph{v0.5:} 2017/04/26

\begin{itemize}
\item
functionality in definition file
\end{itemize}


%%%%%%%%%%%%%%%%%%%%%%%%%%%%%%%%%%%%%%%%%%%%%%%%%%%%%%%%%%%%%%%%%%%%%%%%%%%%%%%%
%%%%%%%%%%%%%%%%%%%%%%%%%%%%%%%%%%%%%%%%%%%%%%%%%%%%%%%%%%%%%%%%%%%%%%%%%%%%%%%%
%%%%%%%%%%%%%%%%%%%%%%%%%%%%%%%%%%%%%%%%%%%%%%%%%%%%%%%%%%%%%%%%%%%%%%%%%%%%%%%%
\appendix

\settowidth\MacroIndent{\rmfamily\scriptsize 000\ }

 \DocInput{childdoc.dtx}

\end{document}
%</driver>
% \fi
%
% %%%%%%%%%%%%%%%%%%%%%%%%%%%%%%%%%%%%%%%%%%%%%%%%%%%%%%%%%%%%%%%%%%%%%%%%%%%%%%
% %%%%%%%%%%%%%%%%%%%%%%%%%%%%%%%%%%%%%%%%%%%%%%%%%%%%%%%%%%%%%%%%%%%%%%%%%%%%%%
% \section{Sample}
%\iffalse
%<*samplemain>
%\fi
%
% The following presents a sample document
% with two chapters, two parts, a title page,
% a compile flag as well as three forwarding files to set the flag.
% It consists of eight |.tex| files:
% \begin{center}
% \begin{tabular}{ll}
% |cdocsamp.tex|&main file\\
% |cdocsch1.tex|&include file for chapter 1\\
% |cdocsch2.tex|&include file for chapter 2\\
% |cdocspt3.tex|&include file for part 3\\
% |cdocspt4.tex|&include file for part 4\\
% |cdocsdrf.tex|&forwarding file for main file in draft mode\\
% |cdocsfi1.tex|&forwarding file for final version of chapter 1\\
% |cdocsfi2.tex|&forwarding file for final version of chapter 2\\
% \end{tabular}
% \end{center}
% Each of the eight files can be compiled directly by the \LaTeX{} compiler.
%
% %%%%%%%%%%%%%%%%%%%%%%%%%%%%%%%%%%%%%%
% \paragraph{Main File.}
%
% The main file is called |cdocsamp.tex|.
%
% Load the \textsf{childdoc} definitions and
% declare the filename for the main document:
%    \begin{macrocode}
\input{childdoc.def}
\childdocmain{}
%    \end{macrocode}

% Optional override for |\version| flag:
%    \begin{macrocode}
%%\ifchilddoc\else\providecommand{\version}{draft}\fi
%    \end{macrocode}

% Define the default values for the |\version| flag
% (|final| for the main file and |draft| for childs):
%    \begin{macrocode}
\ifchilddoc
\providecommand{\version}{draft}
\else
\providecommand{\version}{final}
\fi
%    \end{macrocode}

% Load the standard document class:
%    \begin{macrocode}
\documentclass[12pt]{article}
%    \end{macrocode}

% Start the document body:
%    \begin{macrocode}
\begin{document}
%    \end{macrocode}

% Declare a title page.
% Print title, part of document being processed and version flag:
%    \begin{macrocode}
\addtocounter{page}{-1}
\begin{center}
{\LARGE\bfseries{}childdoc example\par}
\vspace{1cm}
\ifchilddoc
\ifchilddocmanual part\else chapter\fi:
`\childdocname' of `\childdocjob'\par
\else
main document: `\childdocjob'\par
\fi
version: \version\par
\end{center}
\newpage
%    \end{macrocode}

% Manually include selected file,
% otherwise process as usual:
%    \begin{macrocode}
\ifchilddocmanual
\section*{part `\childdocname'}
\input{\childdocname}
\else
%    \end{macrocode}

% Include the two chapters:
%    \begin{macrocode}
\include{cdocsch1}
\include{cdocsch2}
%    \end{macrocode}

% Include the two parts unless only chapters should be displayed:
%    \begin{macrocode}
\ifchilddoc\else
\section{part three}
\input{cdocspt3}
\section{part four}
\input{cdocspt4}
\fi
%    \end{macrocode}

% Process as usual until here:
%    \begin{macrocode}
\fi
%    \end{macrocode}

% End of document body:
%    \begin{macrocode}
\end{document}
%    \end{macrocode}
%\iffalse
%</samplemain>
%\fi
%
% %%%%%%%%%%%%%%%%%%%%%%%%%%%%%%%%%%%%%%
% \paragraph{Chapter Include Files.}
%
% The include files are called |cdocsch1.tex| and |cdocsch2.tex|.
%
%\iffalse
%<*samplechap1|samplechap2>
%\fi

% Optional override for |\version| flag:
%    \begin{macrocode}
%%\providecommand{\version}{final}
%    \end{macrocode}

% Include the main document:
%    \begin{macrocode}
\input{childdoc.def}
\childdocof{cdocsamp}
%    \end{macrocode}

%\iffalse
%</samplechap1|samplechap2>
%\fi
%
%\iffalse
%<*samplechap1>
%\fi
% Some text for chapter 1:
%    \begin{macrocode}
\section{one}
some text in chapter one
%    \end{macrocode}

%\iffalse
%</samplechap1>
%\fi
% Some text for chapter 2:
%\iffalse
%<*samplechap2>
%\fi
%    \begin{macrocode}
\section{two}
more text in chapter two
%    \end{macrocode}

%\iffalse
%</samplechap2>
%\fi
%
% %%%%%%%%%%%%%%%%%%%%%%%%%%%%%%%%%%%%%%
% \paragraph{Part Include Files.}
%
% The include files are called |cdocspt3.tex| and |cdocspt4.tex|.
%
%\iffalse
%<*samplepart3|samplepart4>
%\fi

% Optional override for |\version| flag:
%    \begin{macrocode}
%%\providecommand{\version}{final}
%    \end{macrocode}

% Include the main document:
%    \begin{macrocode}
\input{childdoc.def}
\childdocby{cdocsamp}
%    \end{macrocode}

%\iffalse
%</samplepart3|samplepart4>
%\fi
%
%\iffalse
%<*samplepart3>
%\fi
% Some text for part 3:
%    \begin{macrocode}
some text in part three
%    \end{macrocode}

%\iffalse
%</samplepart3>
%\fi
% Some text for part 4:
%\iffalse
%<*samplepart4>
%\fi
%    \begin{macrocode}
more text in part four
%    \end{macrocode}

%\iffalse
%</samplepart4>
%\fi
%
% %%%%%%%%%%%%%%%%%%%%%%%%%%%%%%%%%%%%%%
% \paragraph{Forwarding for a Complete Draft.}
%
% The following forwarding file |cdocsdrf.tex|
% compiles the main document in draft mode:
%\iffalse
%<*sampledraft>
%\fi
%    \begin{macrocode}
\def\version{draft}
\input{childdoc.def}
\childdocforward{cdocsamp}
%    \end{macrocode}

%\iffalse
%</sampledraft>
%\fi
%
% %%%%%%%%%%%%%%%%%%%%%%%%%%%%%%%%%%%%%%
% \paragraph{Forwarding for Final Version of the Chapters.}
%
% The following forwarding files |cdocsfn1.tex| and |cdocsfn2.tex|
% (with identical content)
% compile the final versions of the child documents
% |cdocsch1.tex| and |cdocsch2.tex|, respectively:
%\iffalse
%<*samplefinal>
%\fi
%    \begin{macrocode}
\def\version{final}
\input{childdoc.def}
\childdocforwardprefix[cdocsamp]{cdocsfn}{cdocsch}
%    \end{macrocode}

%\iffalse
%</samplefinal>
%\fi
%
% %%%%%%%%%%%%%%%%%%%%%%%%%%%%%%%%%%%%%%
% \paragraph{Command Line Processing.}
%
% The following three command lines generate the output files
% |cdocscld|, |cdocscl1| and |cdocscl2|
% which should be identical to
% |cdocsdrf|, |cdocsch1| and |cdocsfn2|, respectively:
% \begin{center}
% \begin{tabular}{l}
% |latex -jobname cdocscld \|\\
% |  "\def\version{draft}\input{childdoc.def}\childdocforward{cdocsamp}"|\\
% |latex -jobname cdocscl1 \|\\
% |  "\input{childdoc.def}\childdocforward[cdocsamp]{cdocsch1}"|\\
% |latex -jobname cdocscl2 \|\\
% |  "\def\version{final}\input{childdoc.def}\childdocforward{cdocsch2}"|
% \end{tabular}
% \end{center}
% Note that the trailing backslash on each first line
% merely continues the input to the second line
% (for convenient cut ant paste).
% Furthermore, the command |latex| can be replaced by any
% of its alternative versions such as |pdflatex|.
%
% %%%%%%%%%%%%%%%%%%%%%%%%%%%%%%%%%%%%%%%%%%%%%%%%%%%%%%%%%%%%%%%%%%%%%%%%%%%%%%
% %%%%%%%%%%%%%%%%%%%%%%%%%%%%%%%%%%%%%%%%%%%%%%%%%%%%%%%%%%%%%%%%%%%%%%%%%%%%%%
% \section{Implementation}
%\iffalse
%<*package>
%\fi
%
% This section describes the definitions file |childdoc.def|.

% The definitions cannot be loaded using |\usepackage| or |\RequirePackage|
% which has a mechanism to prevent loading a style file more than once.
% When loading the definitions by means of |\input|
% multiple instances have to be prevented manually:
%\iffalse
%This code needs to be before the `\ProvidesFile' directive
%which is defined at the beginning of this file.
%Therefore it is also placed there and commented out here.
%</package>
%<*discard>
%\fi
%    \begin{macrocode}
\ifdefined\childdocmain\endinput\fi
%    \end{macrocode}
%\iffalse
%</discard>
%<*package>
%\fi
%
% \macro{\ifchilddoc}
% \macro{\ifchilddocmanual}
% The conditional |\ifchilddoc| tells whether a
% child (true) or main (false) document is being compiled.
% The conditional |\ifchilddocmanual| tells whether
% the |\includeonly| mechanism is used (false) or
% the selection of child files must be performed manually (true).
% The definitions initialise to false:
%    \begin{macrocode}
\newif\ifchilddoc
\newif\ifchilddocmanual
%    \end{macrocode}

% \macro{\childdocname}
% \macro{\childdocjob}
% The macro |\childdocname| stores the name of the main document
% to be compiled. The macro |\childdocjob| stores the name of
% the document on which the \LaTeX{} compiler was originally invoked.
% The content of |\jobname| cannot be compared
% to filenames specified in the source due to different catcodes.
% The following code rescans |\jobname|, stores the result
% in |\childdocname| and saves a copy in |\childdocjob|:
%    \begin{macrocode}
\edef\childdocname{\scantokens\expandafter{\jobname\noexpand}}
\let\childdocjob\childdocname
%    \end{macrocode}

% \macro{\childdocdisable}
% The macro |\childdocdisable| prevents the main file
% from being processed more than once.
% At this stage, the main document command |\childdocmain|
% is assumed to be called once again where it should do nothing.
% Any subsequent call to it should prevent
% a secondary processing of the main document
% It overwrites the forwarding commands
% |\childdocof| and |\childdocforward|
% with empty macros to prevent further inclusions of the main document:
%    \begin{macrocode}
\newcommand{\childdocdisable}
{
  \renewcommand{\childdocmain}[1]{\renewcommand{\childdocmain}[1]{\endinput}}
  \renewcommand{\childdocof}[1]{}
  \renewcommand{\childdocby}[2][]{}
  \renewcommand{\childdocforward}[2][]{}
  \renewcommand{\childdocdisable}{}
}
%    \end{macrocode}

% \macro{\childdocmain}
% The macro |\childdocmain| is to be called at the top of the main file
% with nothing or the main filename (without extension) as argument.
% First, it breaks loops.
% If the argument is not empty and does not match |\childdocname|
% (which is set by the first inclusion of |childdoc.def|),
% |\ifchilddoc| is set to true, |\includeonly| is applied to the child file
% and |\jobname| is set to the main file
% (for proper handling of |.aux| files):
%    \begin{macrocode}
\newcommand{\childdocmain}[1]
{
  \childdocdisable\childdocmain{}
  \if?#1?\else
    \begingroup
      \def\childdoctmp{#1}
      \ifx\childdoctmp\childdocname
        \def\childdoctmp{}
      \else
        \def\childdoctmp
        {
          \childdoctrue
          \includeonly{\childdocname}
          \def\childdocjob{#1}
          \def\jobname{#1}
        }
      \fi
      \expandafter
    \endgroup
    \childdoctmp
  \fi
}
%    \end{macrocode}

% \macro{\childdocof}
% The command |\childdocof| redirects
% compilation to the main file |#1|.
%    \begin{macrocode}
\newcommand{\childdocof}[1]
{
  \childdocdisable
  \childdoctrue
  \includeonly{\childdocname}
  \def\jobname{#1}
  \def\childdocjob{#1}
  \input{#1}
}
%    \end{macrocode}

% \macro{\childdocby}
% The command |\childdocby| ....
%    \begin{macrocode}
\newcommand{\childdocby}[2][]
{
  \childdocdisable
  \childdoctrue
  \childdocmanualtrue
  \if?#1?\else
    \def\jobname{#2}
  \fi
  \def\childdocjob{#2}
  \input{#2}
  \endinput
}
%    \end{macrocode}

% \macro{\childdocforward}
% The command |\childdocforward| redirects
% compilation to the main file or
% (if the optional argument is given) a child file.
% Parameters are set as if the main file
% or a child file starting with |\childdocof| was compiled.
% Then compilation is handed over to the main file:
%    \begin{macrocode}
\newcommand{\childdocforward}[2][]
{
  \begingroup
    \if?#1?
      \def\childdoctmp
      {
        \def\childdocname{#2}
        \def\childdocjob{#2}
        \def\jobname{#2}
        \input{#2}
        \endinput
      }
    \else
      \def\childdoctmp
      {
        \childdocdisable
        \def\childdocname{#2}
        \childdoctrue
        \includeonly{#2}
        \def\childdocjob{#1}
        \def\jobname{#1}
        \input{#1}
        \endinput
      }
    \fi
    \expandafter
  \endgroup
  \childdoctmp
}
%    \end{macrocode}

% \macro{\childdocforwardprefix}
% The command |\childdocforwardprefix| redirects
% compilation to the main or a child file by means of a pattern.
% The prefix |#1| in the current filename is replaced by |#2|
% and the suffix of the current filename is kept
% (it is assumed that the filename does not contain the substring `|~~~|'
% which is used as a delimiter).
% Compilation is handed over to the new file by |\childdocforward|:
%    \begin{macrocode}
\newcommand{\childdocforwardprefix}[3][]
{
  \begingroup
    \def\childdocextract #2##1~~~{\def\childdoctmp{\childdocforward[#1]{#3##1}}}
    \expandafter\childdocextract\childdocname~~~
    \expandafter
  \endgroup
  \childdoctmp
}
%    \end{macrocode}

% \macro{\childdoc}
% The deprecated macro |\childdoc| is a legacy version of |\childdocmain|:
%    \begin{macrocode}
\newcommand{\childdoc}{\childdocmain}
%    \end{macrocode}

% \macro{\childdocredirect}
% The deprecated macro |\childdocredirect| is a legacy version
% of |\childdocforward| and |\childdocforwardprefix|:
%    \begin{macrocode}
\newcommand{\childdocredirect}[2][]
{
  \begingroup
    \if?#1?
      \def\childdoctmp{\childdocforward{#2}}
    \else
      \def\childdoctmp{\childdocforwardprefix{#1}{#2}}
    \fi
    \expandafter
  \endgroup
  \childdoctmp
}
%    \end{macrocode}

%\iffalse
%</package>
%\fi
%
\endinput
|\\
|\childdocof{|\textit{main}|}|\\
\end{tabular}
\end{center}
at the top of every child file \textit{child}
which is included by |\include{|\textit{child}|}|
from within the main file
(or at least for those files to be compiled individually).
The argument \textit{main} must be the filename of the main file.

There are a couple of
considerations in setting up the main and child documents:

%%%%%%%%%%%%%%%%%%%%%%%%%%%%%%%%%%%%%%%%
\paragraph{Restrictions.}

Please note the following restrictions:
\begin{itemize}
\item
|\childdocmain| must be called with one argument \textit{main}
to ensure compatibility with earlier version of the package.
It must either be empty (|\childdocmain{}|)
or precisely match the filename of the main file in which it is specified.
See \secref{sec:detection} for further information.
\item
The filename \textit{main} must be specified without the |.tex| extension.
\item
The filename \textit{main} is case sensitive
(even in case-insensitive file systems)
due to internal string comparison.
\item
The argument \textit{main} should be fully expanded, it cannot be a macro.
\item
Subdirectories and special characters should be avoided in filenames.
\item
The command |\childdocmain{|\textit{main}|}| must be followed by a whitespace.
It should not be followed immediately by another command
or by a comment mark `|%|'.
This is because the \TeX{} parser reads the token immediately following
the argument of |\childdocmain| and puts it
at the beginning of every child section;
however, a white\-space is ignored.
\end{itemize}

%%%%%%%%%%%%%%%%%%%%%%%%%%%%%%%%%%%%%%%%
\paragraph{Content of Main File.}

It is advisable to place all content in the child files included by |\include|.
Any output contained in the main file will appear in all child documents
unless suppressed manually;
it cannot be suppressed automatically by the |\includeonly| directive
and thus should normally be avoided.
A method to include some content in the main file
by means of conditional processing is described in \secref{sec:conditional}.

%%%%%%%%%%%%%%%%%%%%%%%%%%%%%%%%%%%%%%%%
\paragraph{Page Numbering.}

When only a part of the document is compiled,
the appropriate numbering of pages
(as well as other status parameters)
is determined from the |.aux| files.
The latter contain information from previous passes.
However this information needs to propagate through
all intermediate child documents.
Therefore the page numbering in child documents may well
be inconsistent until the complete document is compiled at least once.

A useful (if unconventional) way to always ensure a consistent
page numbering is to restart the numbering in each child document
and denote the pages by `\textit{child}|.|\textit{page}'
where \textit{child} represents the chapter/section number of the child file.
This can be achieved by the command
|\numberwithin{page}{|\textit{child}|}|
of the \textsf{amsmath} package
where \textit{child} can be |chapter| or |section|
depending on the chosen structuring.
Alternatively, one can modify the macro |\thepage| appropriately
and reset the counter |page| at the start of each child file.

%%%%%%%%%%%%%%%%%%%%%%%%%%%%%%%%%%%%%%%%%%%%%%%%%%%%%%%%%%%%%%%%%%%%%%%%%%%%%%%%
\subsection{Conditional Processing}
\label{sec:conditional}

The package provides a mechanism to compile different versions
of a document. To customise the versions further some conditional processing
can come in handy to distinguish which version is being compiled.
The package provides two macros to describe the compilation context:

%%%%%%%%%%%%%%%%%%%%%%%%%%%%%%%%%%%%%%%%
\DescribeMacro{\ifchilddoc}
The conditional |\ifchilddoc| distinguishes between the compilation of
child documents and the main document:
%
\begin{center}
|\ifchilddoc |\textit{child-code}| |[|\||else |\textit{main-code}]| \||fi|
\end{center}

%%%%%%%%%%%%%%%%%%%%%%%%%%%%%%%%%%%%%%%%
\DescribeMacro{\childdocname}
\DescribeMacro{\childdocjob}
The macro |\childdocname| contains the filename (without extension)
of the main or child file being processed.
Note that |\childdocjob| will always contain the name of the main file.

%%%%%%%%%%%%%%%%%%%%%%%%%%%%%%%%%%%%%%%%
\paragraph{Title Page.}

Conditional processing can be used to include a title or banner page
in the main document when proper precautions are taken.
Importantly, the code in the main file should ensure that the page counter
(as well as other status parameters which are stored in the |.aux| files)
takes the same value after the conditional processing.
Otherwise the page numbers may take divergent values
depending on which part is compiled.

For example, a title page could be declared by:
%
\begin{center}
\begin{tabular}{l}
|\ifchilddoc\||else|\\
|\addtocounter{page}{-1}|\\
\textit{code for title page}\\
|\newpage|\\
|\||fi|
\end{tabular}
\end{center}
%
A banner page for the child documents can be generated by:
%
\begin{center}
\begin{tabular}{l}
|\ifchilddoc|\\
|\addtocounter{page}{-1}|\\
\textit{code for banner page}\\
|\newpage|\\
|\||fi|
\end{tabular}
\end{center}
%
Here one could write a message such as:
\begin{center}
|This is the part \childdocname{} of \childdocjob{}.|
\end{center}

%%%%%%%%%%%%%%%%%%%%%%%%%%%%%%%%%%%%%%%%%%%%%%%%%%%%%%%%%%%%%%%%%%%%%%%%%%%%%%%%
\subsection{Flags}
\label{sec:flags}

The package makes it easy to generate different versions
of the main or child documents.
To this end compilation flags can be defined
and assigned different default values.
They will be particularly useful in conjunction
with the forwarding mechanism described in \secref{sec:forward}.

For example, it may be useful to have a flag |\version|
which can be set to |draft| or |final|.
The document source will contain some conditional code
depending on the value of |\version|.
Suppose further, the flag should default to |final| for the main file
and to |draft| for child files
which is a natural assignment for editing the document.
This is achieved by placing the following code
in the preamble of the main document
(below the |\childdocmain| directive):
%
\begin{center}
\begin{tabular}{l}
|\ifchilddoc|\\
|\providecommand{\version}{draft}|\\
|\||else|\\
|\providecommand{\version}{final}|\\
|\||fi|
\end{tabular}
\end{center}
%
The definition by |\providecommand| makes sure
that previous definitions are not overwritten.
Further statements |\providecommand{\version}{...}|
can thus be added before the above code to override it.

For the main file, one might add a line
(between |\childdocmain| and the above block)
%
\begin{center}
|%\ifchilddoc\||else\providecommand{\version}{draft}\||fi|
\end{center}
%
which can be uncommented to produce a draft version.
Likewise one can add a line to the very top of a child file
(above the |\childdocof{|\textit{main}|}| directive)
%
\begin{center}
|%\providecommand{\version}{final}|
\end{center}
%
which can be uncommented to produce the final version of this child document.

%%%%%%%%%%%%%%%%%%%%%%%%%%%%%%%%%%%%%%%%%%%%%%%%%%%%%%%%%%%%%%%%%%%%%%%%%%%%%%%%
\subsection{Forwarding}
\label{sec:forward}

Different versions of the main or child documents
using compilation flags as described in \secref{sec:flags}
can be (permanently) stored in different files
for convenient compilation, viewing and distribution.
To this end, the package defines a command
to pass on compilation to a different file:

%%%%%%%%%%%%%%%%%%%%%%%%%%%%%%%%%%%%%%%%
\DescribeMacro{\childdocforward}
The command |\childdocforward| redirects processing to
another source file:
%
\begin{center}
\begin{tabular}{l}
|% \iffalse
%
% childdoc.dtx Copyright (C) 2017-2018 Niklas Beisert
%
% This work may be distributed and/or modified under the
% conditions of the LaTeX Project Public License, either version 1.3
% of this license or (at your option) any later version.
% The latest version of this license is in
%   http://www.latex-project.org/lppl.txt
% and version 1.3 or later is part of all distributions of LaTeX
% version 2005/12/01 or later.
%
% This work has the LPPL maintenance status `maintained'.
%
% The Current Maintainer of this work is Niklas Beisert.
%
% This work consists of the files childdoc.dtx and childdoc.ins
% and the derived files childdoc.def and cdocsamp.tex with
% cdocsch1.tex, cdocsch2.tex, cdocsdrf.tex, cdocsfn1.tex, cdocsfn2.tex.
%
%<package>\ifdefined\childdocmain\endinput\fi
%<package>\ProvidesFile{childdoc.def}[2018/12/30 v2.0 child document driver]
%<samplemain>\ProvidesFile{cdocsamp.tex}[2018/12/30 v2.0 sample for childdoc]
%<*driver>
%\ProvidesFile{childdoc.drv}[2018/12/30 v2.0 childdoc reference manual file]
\PassOptionsToClass{10pt,a4paper}{article}
\documentclass{ltxdoc}

\usepackage[margin=35mm]{geometry}
\usepackage{hyperref}
\usepackage{hyperxmp}
\usepackage[usenames]{color}

\hypersetup{colorlinks=true}
\hypersetup{pdfstartview=FitH}
\hypersetup{pdfpagemode=UseNone}
\hypersetup{pdfsource={}}
\hypersetup{pdflang={en-UK}}
\hypersetup{pdfcopyright={Copyright 2017-2018 Niklas Beisert.
  This work may be distributed and/or modified under the
  conditions of the LaTeX Project Public License, either version 1.3
  of this license or (at your option) any later version.}}
\hypersetup{pdflicenseurl={http://www.latex-project.org/lppl.txt}}
\hypersetup{pdfcontactaddress={ETH Zurich, ITP, HIT K,
  Wolfgang-Pauli-Strasse 27}}
\hypersetup{pdfcontactpostcode={8093}}
\hypersetup{pdfcontactcity={Zurich}}
\hypersetup{pdfcontactcountry={Switzerland}}
\hypersetup{pdfcontactemail={nbeisert@itp.phys.ethz.ch}}
\hypersetup{pdfcontacturl={http://people.phys.ethz.ch/\xmptilde nbeisert/}}

\newcommand{\secref}[1]{\hyperref[#1]{section \ref*{#1}}}

\parskip1ex
\parindent0pt
\let\olditemize\itemize
\def\itemize{\olditemize\parskip0pt}

\begin{document}

\title{The \textsf{childdoc} Package}
\hypersetup{pdftitle={The childdoc Package}}
\author{Niklas Beisert\\[2ex]
  Institut f\"ur Theoretische Physik\\
  Eidgen\"ossische Technische Hochschule Z\"urich\\
  Wolfgang-Pauli-Strasse 27, 8093 Z\"urich, Switzerland\\[1ex]
  \href{mailto:nbeisert@itp.phys.ethz.ch}
  {\texttt{nbeisert@itp.phys.ethz.ch}}}
\hypersetup{pdfauthor={Niklas Beisert}}
\hypersetup{pdfsubject={Manual for the LaTeX2e Package childdoc}}
\date{30 December 2018, \textsf{v2.0}}
\maketitle

\begin{abstract}\noindent
\textsf{childdoc} is a \LaTeXe{} package
that enables the direct compilation
of document sections included by |\include|
to individual files.
\end{abstract}

\begingroup
\parskip0ex
\tableofcontents
\endgroup

%%%%%%%%%%%%%%%%%%%%%%%%%%%%%%%%%%%%%%%%%%%%%%%%%%%%%%%%%%%%%%%%%%%%%%%%%%%%%%%%
%%%%%%%%%%%%%%%%%%%%%%%%%%%%%%%%%%%%%%%%%%%%%%%%%%%%%%%%%%%%%%%%%%%%%%%%%%%%%%%%
\section{Introduction}

\LaTeX{} provides a mechanism to structure a large document (such as a book)
into a main file and several child files (containing the chapters)
using the |\include| command.
This mechanism is beneficial for documents
which span hundreds of pages in order to
make the source file(s) more manageable.
Moreover, compilation can be restricted to
selected child files by means of the |\includeonly| command.
The latter feature can be used to reduce the compilation time while editing
(this was significantly more useful in the earlier days of \LaTeX{})
or to generate a smaller document which is easier to navigate.
Another application of |\includeonly| is to generate
documents consisting of selected parts of the complete document.

However, there are a few drawbacks of the plain |\include| mechanism:
\begin{itemize}
\item
The child files cannot be compiled on their own,
they can only be compiled via the main file.
A naive editing environment
(such as a text editor with an option
to have the current file processed by \LaTeX)
may require one to switch to the main file before compiling;
attempting to compile the child file produces errors.
\item
The main file must be modified (each time)
to adjust the |\includeonly| command
to the present needs. This easily leaves the main file in a messy state.
\item
The generated document will always carry the filename
of the main document. This is inconvenient if
several child files are to be compiled and
to be kept for distribution.
\end{itemize}

The present package provides a simple interface
to make child files individually compilable by \LaTeX{}.
Compiling a child file then has the same effect as compiling
the main file with an |\includeonly| command
to select the appropriate child.
Moreover the generated document will carry the name of the child
rather than the main file.
This resolves all three above issues.

This feature is meant to make the editing of books,
thesis documents and lecture notes somewhat more convenient.
However, the package can also be used efficiently for
composing a series of documents (such as exercise sheets)
which are typically distributed individually.
It then assists the author in generating the individual documents
(potentially in different versions)
as well as a document containing the collected series.
Another application is in developing style files
or other kinds of included material
where compilation of the style file could redirect
to a sample or test file.

%%%%%%%%%%%%%%%%%%%%%%%%%%%%%%%%%%%%%%%%%%%%%%%%%%%%%%%%%%%%%%%%%%%%%%%%%%%%%%%%
%%%%%%%%%%%%%%%%%%%%%%%%%%%%%%%%%%%%%%%%%%%%%%%%%%%%%%%%%%%%%%%%%%%%%%%%%%%%%%%%
\section{Usage}

First of all, the package \textsf{childdoc} is \emph{not} a standard
\LaTeXe{} |.sty| style file! Therefore it needs to be invoked in
a non-standard way.

%%%%%%%%%%%%%%%%%%%%%%%%%%%%%%%%%%%%%%%%%%%%%%%%%%%%%%%%%%%%%%%%%%%%%%%%%%%%%%%%
\subsection{Included Files}
\label{sec:include}

%%%%%%%%%%%%%%%%%%%%%%%%%%%%%%%%%%%%%%%%
\DescribeMacro{\childdocmain}
To use the package, add the commands
\begin{center}
\begin{tabular}{l}
|\input{childdoc.def}|\\
|\childdocmain{}|\\
\end{tabular}
\end{center}
at the very top of the main \LaTeX{} file,
in particular \emph{before} the |\documentclass| statement!
The argument of |\childdocmain| should be left empty
(but it must be present).

%%%%%%%%%%%%%%%%%%%%%%%%%%%%%%%%%%%%%%%%
\DescribeMacro{\childdocof}
Furthermore, add the commands
\begin{center}
\begin{tabular}{l}
|\input{childdoc.def}|\\
|\childdocof{|\textit{main}|}|\\
\end{tabular}
\end{center}
at the top of every child file \textit{child}
which is included by |\include{|\textit{child}|}|
from within the main file
(or at least for those files to be compiled individually).
The argument \textit{main} must be the filename of the main file.

There are a couple of
considerations in setting up the main and child documents:

%%%%%%%%%%%%%%%%%%%%%%%%%%%%%%%%%%%%%%%%
\paragraph{Restrictions.}

Please note the following restrictions:
\begin{itemize}
\item
|\childdocmain| must be called with one argument \textit{main}
to ensure compatibility with earlier version of the package.
It must either be empty (|\childdocmain{}|)
or precisely match the filename of the main file in which it is specified.
See \secref{sec:detection} for further information.
\item
The filename \textit{main} must be specified without the |.tex| extension.
\item
The filename \textit{main} is case sensitive
(even in case-insensitive file systems)
due to internal string comparison.
\item
The argument \textit{main} should be fully expanded, it cannot be a macro.
\item
Subdirectories and special characters should be avoided in filenames.
\item
The command |\childdocmain{|\textit{main}|}| must be followed by a whitespace.
It should not be followed immediately by another command
or by a comment mark `|%|'.
This is because the \TeX{} parser reads the token immediately following
the argument of |\childdocmain| and puts it
at the beginning of every child section;
however, a white\-space is ignored.
\end{itemize}

%%%%%%%%%%%%%%%%%%%%%%%%%%%%%%%%%%%%%%%%
\paragraph{Content of Main File.}

It is advisable to place all content in the child files included by |\include|.
Any output contained in the main file will appear in all child documents
unless suppressed manually;
it cannot be suppressed automatically by the |\includeonly| directive
and thus should normally be avoided.
A method to include some content in the main file
by means of conditional processing is described in \secref{sec:conditional}.

%%%%%%%%%%%%%%%%%%%%%%%%%%%%%%%%%%%%%%%%
\paragraph{Page Numbering.}

When only a part of the document is compiled,
the appropriate numbering of pages
(as well as other status parameters)
is determined from the |.aux| files.
The latter contain information from previous passes.
However this information needs to propagate through
all intermediate child documents.
Therefore the page numbering in child documents may well
be inconsistent until the complete document is compiled at least once.

A useful (if unconventional) way to always ensure a consistent
page numbering is to restart the numbering in each child document
and denote the pages by `\textit{child}|.|\textit{page}'
where \textit{child} represents the chapter/section number of the child file.
This can be achieved by the command
|\numberwithin{page}{|\textit{child}|}|
of the \textsf{amsmath} package
where \textit{child} can be |chapter| or |section|
depending on the chosen structuring.
Alternatively, one can modify the macro |\thepage| appropriately
and reset the counter |page| at the start of each child file.

%%%%%%%%%%%%%%%%%%%%%%%%%%%%%%%%%%%%%%%%%%%%%%%%%%%%%%%%%%%%%%%%%%%%%%%%%%%%%%%%
\subsection{Conditional Processing}
\label{sec:conditional}

The package provides a mechanism to compile different versions
of a document. To customise the versions further some conditional processing
can come in handy to distinguish which version is being compiled.
The package provides two macros to describe the compilation context:

%%%%%%%%%%%%%%%%%%%%%%%%%%%%%%%%%%%%%%%%
\DescribeMacro{\ifchilddoc}
The conditional |\ifchilddoc| distinguishes between the compilation of
child documents and the main document:
%
\begin{center}
|\ifchilddoc |\textit{child-code}| |[|\||else |\textit{main-code}]| \||fi|
\end{center}

%%%%%%%%%%%%%%%%%%%%%%%%%%%%%%%%%%%%%%%%
\DescribeMacro{\childdocname}
\DescribeMacro{\childdocjob}
The macro |\childdocname| contains the filename (without extension)
of the main or child file being processed.
Note that |\childdocjob| will always contain the name of the main file.

%%%%%%%%%%%%%%%%%%%%%%%%%%%%%%%%%%%%%%%%
\paragraph{Title Page.}

Conditional processing can be used to include a title or banner page
in the main document when proper precautions are taken.
Importantly, the code in the main file should ensure that the page counter
(as well as other status parameters which are stored in the |.aux| files)
takes the same value after the conditional processing.
Otherwise the page numbers may take divergent values
depending on which part is compiled.

For example, a title page could be declared by:
%
\begin{center}
\begin{tabular}{l}
|\ifchilddoc\||else|\\
|\addtocounter{page}{-1}|\\
\textit{code for title page}\\
|\newpage|\\
|\||fi|
\end{tabular}
\end{center}
%
A banner page for the child documents can be generated by:
%
\begin{center}
\begin{tabular}{l}
|\ifchilddoc|\\
|\addtocounter{page}{-1}|\\
\textit{code for banner page}\\
|\newpage|\\
|\||fi|
\end{tabular}
\end{center}
%
Here one could write a message such as:
\begin{center}
|This is the part \childdocname{} of \childdocjob{}.|
\end{center}

%%%%%%%%%%%%%%%%%%%%%%%%%%%%%%%%%%%%%%%%%%%%%%%%%%%%%%%%%%%%%%%%%%%%%%%%%%%%%%%%
\subsection{Flags}
\label{sec:flags}

The package makes it easy to generate different versions
of the main or child documents.
To this end compilation flags can be defined
and assigned different default values.
They will be particularly useful in conjunction
with the forwarding mechanism described in \secref{sec:forward}.

For example, it may be useful to have a flag |\version|
which can be set to |draft| or |final|.
The document source will contain some conditional code
depending on the value of |\version|.
Suppose further, the flag should default to |final| for the main file
and to |draft| for child files
which is a natural assignment for editing the document.
This is achieved by placing the following code
in the preamble of the main document
(below the |\childdocmain| directive):
%
\begin{center}
\begin{tabular}{l}
|\ifchilddoc|\\
|\providecommand{\version}{draft}|\\
|\||else|\\
|\providecommand{\version}{final}|\\
|\||fi|
\end{tabular}
\end{center}
%
The definition by |\providecommand| makes sure
that previous definitions are not overwritten.
Further statements |\providecommand{\version}{...}|
can thus be added before the above code to override it.

For the main file, one might add a line
(between |\childdocmain| and the above block)
%
\begin{center}
|%\ifchilddoc\||else\providecommand{\version}{draft}\||fi|
\end{center}
%
which can be uncommented to produce a draft version.
Likewise one can add a line to the very top of a child file
(above the |\childdocof{|\textit{main}|}| directive)
%
\begin{center}
|%\providecommand{\version}{final}|
\end{center}
%
which can be uncommented to produce the final version of this child document.

%%%%%%%%%%%%%%%%%%%%%%%%%%%%%%%%%%%%%%%%%%%%%%%%%%%%%%%%%%%%%%%%%%%%%%%%%%%%%%%%
\subsection{Forwarding}
\label{sec:forward}

Different versions of the main or child documents
using compilation flags as described in \secref{sec:flags}
can be (permanently) stored in different files
for convenient compilation, viewing and distribution.
To this end, the package defines a command
to pass on compilation to a different file:

%%%%%%%%%%%%%%%%%%%%%%%%%%%%%%%%%%%%%%%%
\DescribeMacro{\childdocforward}
The command |\childdocforward| redirects processing to
another source file:
%
\begin{center}
\begin{tabular}{l}
|\input{childdoc.def}|\\
|\childdocforward[|\textit{main}|]{|\textit{dest}|}|\\
\end{tabular}
\end{center}
%
The argument \textit{dest} is the destination file
(without extension).
It should be the main file or one of the child files.
Note that further \textsf{childdoc} directives
such as |\childdocof| and |\childdocforward|
in the indicated file will be processed in this form.
The optional argument \textit{main}
passes on directly to the main file \textit{main}
while pretending to compile the child \textit{dest}.
This form behaves as if \textit{dest}
issues |\childdocof{|\textit{main}|}| right away,
and no further \textsf{childdoc} directives will be processed.

%%%%%%%%%%%%%%%%%%%%%%%%%%%%%%%%%%%%%%%%
\DescribeMacro{\...prefix}
In the alternative form |\childdocforwardprefix|,
%
\begin{center}
\begin{tabular}{l}
|\input{childdoc.def}|\\
|\childdocforwardprefix[|\textit{main}|]{|\textit{prefix}|}{|\textit{dest}|}|
\end{tabular}
\end{center}
%
the destination file is determined by a pattern
depending on the current file:
To make this work, the current file must be called
`{\textit{prefix}\hspace{0.2em}\textit{suffix}}'
with \textit{prefix} matching precisely the argument.
Processing is then passed on to the file
`{\textit{dest}\hspace{0.2em}\textit{suffix}}'.
Surely, the same effect is achieved by
directly specifying the
argument `{\textit{dest}\hspace{0.2em}\textit{suffix}}'
in the first form.
However, that requires to set up a different file
for each child. With the alternative form of the command
all these files can have exactly the same content
which simplifies setting them up and maintaining them.

For example, the following file |draft.tex|
with a compilation flag |\version| as described in \secref{sec:flags}
compiles the main document as a draft:
%
\begin{center}
\begin{tabular}{l}
|\def\version{draft}|\\
|\input{childdoc.def}|\\
|\childdocforward{|\textit{main}|}|
\end{tabular}
\end{center}
%
Likewise, the following files |final|\textit{nn}|.tex|
compile the final version of the child document
|child|\textit{nn}|.tex|:
%
\begin{center}
\begin{tabular}{l}
|\def\version{final}|\\
|\input{childdoc.def}|\\
|\childdocforwardprefix{final}{child}|
\end{tabular}
\end{center}
%

Note that when several versions of a main file and/or of each child file
are to be generated, it may be convenient to set up a |Makefile| or
shell script to automatise the process.

%%%%%%%%%%%%%%%%%%%%%%%%%%%%%%%%%%%%%%%%%%%%%%%%%%%%%%%%%%%%%%%%%%%%%%%%%%%%%%%%
\subsection{Command Line Processing}
\label{sec:commandline}

The effect of redirection files can also be achieved by invoking
the \LaTeX{} compiler with a more elaborate command line.
Most conveniently this should be done as part
of a shell script or a |Makefile|.

When using \textsf{childdoc} in the main file, the following
command lines effectively perform a redirection
(note that depending on the shell being used,
backslashes may have to be doubled: `|\|' $\to$ `|\\|'):
%
\begin{center}
|... -jobname "|\textit{target}|" |\\|"|[\textit{flags}]%
|\input{childdoc.def}\childdocforward[|\textit{main}|]{|\textit{dest}|}"|
\end{center}
%
Here \textit{target} is the name of the output file,
\textit{main} is the name of the main file
and \textit{dest} is the name of the main or child file to be processed
(all filenames without extensions).
The optional argument \textit{main} can be omitted
if \textit{main} matches \textit{dest}.
Optionally, compilation \textit{flags} can be defined via |\def| commands.
This command line makes the \TeX{} engine believe
it is compiling the file \textit{target}
whose content is specified as the latter parameter.
The provided code then forwards the processing to
\textit{main} or \textit{dest} as described in \secref{sec:forward}.

%%%%%%%%%%%%%%%%%%%%%%%%%%%%%%%%%%%%%%%%%%%%%%%%%%%%%%%%%%%%%%%%%%%%%%%%%%%%%%%%
\subsection{Include by Input}
\label{sec:input}

Including child documents by |\include| has some restrictions by design.
Most notably, the content of a child document always occupies
its own set of pages; pages cannot be shared between child documents.
Usually, this behaviour makes perfect sense
because each child document contain an essential part of the document.
However, in some situations it may be desirable to compose
a document from a collection of parts
without having mandatory page breaks between then.
For this case, the package
provides a mechanism to include parts
by |\input| which can also be processed individually.
However, by construction this mechanism
requires manual handling of the content to be output.

%%%%%%%%%%%%%%%%%%%%%%%%%%%%%%%%%%%%%%%%
\DescribeMacro{\ifchilddocmanual}
The main file should be prepared as usual, see \secref{sec:include}.
However, the document body must make a distinction
between processing of an individual part and of the main document, e.g.:
%
\begin{center}
\begin{tabular}{l}
|\ifchilddocmanual|\\
|\input{\childdocname}|\\
|\||else|\\
\textit{document body with }|\input{|\textit{part}|}|\\
|\||fi|
\end{tabular}
\end{center}
%
The conditional |\ifchilddocmanual| is true whenever
a part to be included by |\input| is being compiled,
and the name of the part is stored in |\childdocname|.

%%%%%%%%%%%%%%%%%%%%%%%%%%%%%%%%%%%%%%%%
\DescribeMacro{\childdocby}
Each part to be included by |\input| should start with:
%
\begin{center}
\begin{tabular}{l}
|\input{childdoc.def}|\\
|\childdocby{|\textit{main}|}|\\
\end{tabular}
\end{center}
%
The directive |\childdocby| is similar to |\childdocof|
described in \secref{sec:include},
but the subsequent selection of content must be done manually.
To that end, both |\ifchilddoc| and |\ifchilddocmanual|
will be true upon processing of a part,
and the name of the part is stored in |\childdocname|.
Note that |\jobname| will be set to the filename of the current part
so that each part receives an individual |.aux| file
that does not interfere with the |.aux| file(s) of the main document.
This behaviour can be altered by the alternative form
|\childdocby[*]{|\textit{main}|}| (with a non-empty optional argument)
which uses the |.aux| file of the main document
by setting |\jobname| to \textit{main}.

%%%%%%%%%%%%%%%%%%%%%%%%%%%%%%%%%%%%%%%%%%%%%%%%%%%%%%%%%%%%%%%%%%%%%%%%%%%%%%%%
\subsection{Driver Development}
\label{sec:driver}

The \textsf{childdoc} mechanism can also be use for the development
of definition files such as \LaTeX{} styles or classes.
This case differs from the above setup with multiple parts
included by |\include| in that no |\includeonly| should be invoked.
This can be achieved by starting the include file
(before |\ProvidesPackage|) with:
%
\begin{center}
\begin{tabular}{l}
|\input{childdoc.def}|\\
|\childdocforward{|\textit{main}|}|\\
\end{tabular}
\end{center}
%
or alternatively with:
%
\begin{center}
\begin{tabular}{l}
|\input{childdoc.def}|\\
|\childdocby{|\textit{main}|}|\\
\end{tabular}
\end{center}
%
Both forms have slightly different effects as described above.
The main file is prepared as usual, see \secref{sec:include}.

%%%%%%%%%%%%%%%%%%%%%%%%%%%%%%%%%%%%%%%%%%%%%%%%%%%%%%%%%%%%%%%%%%%%%%%%%%%%%%%%
\subsection{Legacy Detection}
\label{sec:detection}

The directive |\childdocmain| in the main file can detect
whether the complete document or merely a child is to be compiled
even without using the directive |\childdocof|.
This method is deprecated because it is less robust
and there is no compelling reason to use it;
it is merely provided for backward compatibility
and it may be removed in future versions.

If the detection mechanism is to be used,
it is mandatory to correctly specify
the filename of the main file as the argument of |\childdocmain|:
%
\begin{center}
\begin{tabular}{l}
|\input{childdoc.def}|\\
|\childdocmain{|\textit{main}|}|\\
\end{tabular}
\end{center}
%
If |\jobname| does not match the argument \textit{main} of |\childdocmain|,
it is assumed that |\jobname| points to the child file to be compiled.
When using |\childdocmain| with the main file specified as argument,
it suffices to start a child file
with just |\input{|\textit{main}|}|
without loading of the package and using |\childdocof|.
If instead all processing is done
with the appropriate \textsf{childdoc} directives,
the argument of \textit{main} of |\childdocmain| can be empty.

An alternative version of the command line processing described
in \secref{sec:commandline} using the detection mechanism reads:
%
\begin{center}
|... -jobname "|\textit{target}|" "|[\textit{flags}]%
[|\def\jobname{|\textit{dest}|}|]|\input{|\textit{main}|}"|
\end{center}

%%%%%%%%%%%%%%%%%%%%%%%%%%%%%%%%%%%%%%%%%%%%%%%%%%%%%%%%%%%%%%%%%%%%%%%%%%%%%%%%
\subsection{Manual Code}
\label{sec:manual}

In case one cannot be certain whether the definitions file |childdoc.def|
is installed on the target \TeX{} distribution
and one prefers not to ship it,
it is conceivable to paste a few relevant commands into the sources.

To that end, drop all statements |\input{childdoc.def}|
and perform the replacements as outlined below.
Instead of |\childdocmain{|\textit{main}|}| add the following code
to the top of the main file:
%
\begin{center}
\begin{tabular}{l}
|\||ifdefined\childdocname\endinput\||fi\newif\ifchilddoc|\\
|\edef\childdocname{\scantokens\expandafter{\jobname\noexpand}}|\\
|\def\childdocmain{|\textit{main}|}\||ifx\childdocmain\childdocname\||else|\\
|\childdoctrue\includeonly{\childdocname}\let\jobname\childdocmain\||fi|\\
\end{tabular}
\end{center}
%
Instead of |\childdocof{|\textit{main}|}| just include the main file
at the top of each child file:
%
\begin{center}
|\input{|\textit{main}|}|
\end{center}
%
A simple redirection |\childdocforward{|\textit{dest}|}| is achieved by:
%
\begin{center}
|\def\jobname{|\textit{dest}|}\input{\jobname}|
\end{center}
%
The redirection with prefix
|\childdocforwardprefix[|\textit{prefix}|]{|\textit{dest}|}|
is accomplished by:
%
\begin{center}
\begin{tabular}{l}
|{\edef\jobname{\scantokens\expandafter{\jobname\noexpand}}|\\
|\def\redirectjob |\textit{prefix}|#1~~~{\gdef\jobname{|\textit{dest}|#1}}|\\
|\expandafter\redirectjob\jobname~~~}\input{\jobname}|
\end{tabular}
\end{center}

In an alternative approach,
child documents can be compiled by a specific command line
without additional code or specific definitions:
%
\begin{center}
|... -jobname "|\textit{target}|" "|[\textit{flags}]%
|\includeonly{|\textit{dest}|}\input{|\textit{main}|}"|
\end{center}
%

%%%%%%%%%%%%%%%%%%%%%%%%%%%%%%%%%%%%%%%%%%%%%%%%%%%%%%%%%%%%%%%%%%%%%%%%%%%%%%%%
%%%%%%%%%%%%%%%%%%%%%%%%%%%%%%%%%%%%%%%%%%%%%%%%%%%%%%%%%%%%%%%%%%%%%%%%%%%%%%%%
\section{Information}

%%%%%%%%%%%%%%%%%%%%%%%%%%%%%%%%%%%%%%%%%%%%%%%%%%%%%%%%%%%%%%%%%%%%%%%%%%%%%%%%
\subsection{Copyright}

Copyright \copyright{} 2017--2018 Niklas Beisert

This work may be distributed and/or modified under the
conditions of the \LaTeX{} Project Public License, either version 1.3
of this license or (at your option) any later version.
The latest version of this license is in
  \url{http://www.latex-project.org/lppl.txt}
and version 1.3 or later is part of all distributions of \LaTeX{}
version 2005/12/01 or later.

This work has the LPPL maintenance status `maintained'.

The Current Maintainer of this work is Niklas Beisert.

This work consists of the files |README.txt|, |childdoc.ins| and |childdoc.dtx|
as well as the derived files |childdoc.def|, |cdocsamp.tex|
with |cdocsch1.tex|, |cdocsch2.tex|, |cdocspt3.tex|, |cdocspt4.tex|,
|cdocsdrf.tex|, |cdocsfn1.tex|, |cdocsfn2.tex|
as well as |childdoc.pdf|.

%%%%%%%%%%%%%%%%%%%%%%%%%%%%%%%%%%%%%%%%%%%%%%%%%%%%%%%%%%%%%%%%%%%%%%%%%%%%%%%%
\subsection{Files and Installation}

The package consists of the files:
%
\begin{center}
\begin{tabular}{ll}
    |README.txt|   & readme file \\
    |childdoc.ins| & installation file \\
    |childdoc.dtx| & source file \\
    |childdoc.def| & definition file \\
    |cdocsamp.tex| & sample main file \\
    |cdocsch1.tex| & sample include file \\
    |cdocsch2.tex| & sample include file \\
    |cdocspt3.tex| & sample part file \\
    |cdocspt4.tex| & sample part file \\
    |cdocsdrf.tex| & sample redirection file \\
    |cdocsfn1.tex| & sample redirection file \\
    |cdocsfn2.tex| & sample redirection file \\
    |childdoc.pdf| & manual
\end{tabular}
\end{center}
%
The distribution consists of the files
|README.txt|, |childdoc.ins| and |childdoc.dtx|.
%
\begin{itemize}
\item
Run (pdf)\LaTeX{} on |childdoc.dtx|
to compile the manual |childdoc.pdf| (this file).
\item
Run \LaTeX{} on |childdoc.ins| to create the definitions file |childdoc.def|
and the sample |cdocsamp.tex| with include files
|cdocsch1.tex|, |cdocsch2.tex|, |cdocspt3.tex|, |cdocspt4.tex|,
|cdocsdrf.tex|, |cdocsfn1.tex|, |cdocsfn2.tex|.
Then copy the file |childdoc.def| to an appropriate directory of your \LaTeX{}
distribution, e.g.\ \textit{texmf-root}|/tex/latex/childdoc|.
\end{itemize}

%%%%%%%%%%%%%%%%%%%%%%%%%%%%%%%%%%%%%%%%%%%%%%%%%%%%%%%%%%%%%%%%%%%%%%%%%%%%%%%%
\subsection{Related CTAN Packages}

There are several other packages which offer a similar functionality:
%
\begin{itemize}
\item
The packages
\href{http://ctan.org/pkg/docmute}{\textsf{docmute}},
\href{http://ctan.org/pkg/includex}{\textsf{includex}} and
\href{http://ctan.org/pkg/standalone}{\textsf{standalone}}
provide commands to include only the document body of
a child file thus allowing both files to be compiled individually.
\item
The packages \href{http://ctan.org/pkg/subdocs}{\textsf{subdocs}}
and \href{http://ctan.org/pkg/subfiles}{\textsf{subfiles}}
provide structures in which the main and child documents can be
encapsulated and allowing them to be compiled individually.
The inclusion mechanism is different from the conventional |\include|.
\item
The package \href{http://ctan.org/pkg/combine}{\textsf{combine}}
is an elaborate solution to combine several documents into one.
\end{itemize}
%
See also the CTAN topic \href{http://ctan.org/topic/subdocs}{\textsf{subdocs}}
for further related packages.
The present package differs from the above solutions in that
a document structure constructed with the conventional |\include| mechanism
just needs two extra commands at the top of every file
such that all constituent files can be compiled individually.

%%%%%%%%%%%%%%%%%%%%%%%%%%%%%%%%%%%%%%%%%%%%%%%%%%%%%%%%%%%%%%%%%%%%%%%%%%%%%%%%
%\subsection{Feature Suggestions}
%
%The following is a list of features which may be useful for future
%versions of this package:
%%
%\begin{itemize}
%\item
%\ldots
%\end{itemize}

%%%%%%%%%%%%%%%%%%%%%%%%%%%%%%%%%%%%%%%%%%%%%%%%%%%%%%%%%%%%%%%%%%%%%%%%%%%%%%%%
\subsection{Revision History}

%%%%%%%%%%%%%%%%%%%%%%%%%%%%%%%%%%%%%%%%
\paragraph{v2.0:} 2018/12/30

\begin{itemize}
\item
immediate forward processing
\item
added |\childdocby| mechanism
\item
manual restructured
\end{itemize}

%%%%%%%%%%%%%%%%%%%%%%%%%%%%%%%%%%%%%%%%
\paragraph{v1.6:} 2018/01/17

\begin{itemize}
\item
application for development of include files
\item
corrections to manual
\end{itemize}

%%%%%%%%%%%%%%%%%%%%%%%%%%%%%%%%%%%%%%%%
\paragraph{v1.5:} 2017/05/21

\begin{itemize}
\item
more complete structuring introduced
\item
|\childdocof| introduced
\item
|\childdoc| renamed to |\childdocmain|
\item
|\childredirect| renamed to |\childdocforward| and |\childdocforwardprefix|
and functionality expanded
\end{itemize}

%%%%%%%%%%%%%%%%%%%%%%%%%%%%%%%%%%%%%%%%
\paragraph{v1.0:} 2017/04/27

\begin{itemize}
\item
manual and install package
\item
first version published on CTAN
\end{itemize}

%%%%%%%%%%%%%%%%%%%%%%%%%%%%%%%%%%%%%%%%
\paragraph{v0.6:} 2017/04/26

\begin{itemize}
\item
redirection mechanism added
\end{itemize}

%%%%%%%%%%%%%%%%%%%%%%%%%%%%%%%%%%%%%%%%
\paragraph{v0.5:} 2017/04/26

\begin{itemize}
\item
functionality in definition file
\end{itemize}


%%%%%%%%%%%%%%%%%%%%%%%%%%%%%%%%%%%%%%%%%%%%%%%%%%%%%%%%%%%%%%%%%%%%%%%%%%%%%%%%
%%%%%%%%%%%%%%%%%%%%%%%%%%%%%%%%%%%%%%%%%%%%%%%%%%%%%%%%%%%%%%%%%%%%%%%%%%%%%%%%
%%%%%%%%%%%%%%%%%%%%%%%%%%%%%%%%%%%%%%%%%%%%%%%%%%%%%%%%%%%%%%%%%%%%%%%%%%%%%%%%
\appendix

\settowidth\MacroIndent{\rmfamily\scriptsize 000\ }

 \DocInput{childdoc.dtx}

\end{document}
%</driver>
% \fi
%
% %%%%%%%%%%%%%%%%%%%%%%%%%%%%%%%%%%%%%%%%%%%%%%%%%%%%%%%%%%%%%%%%%%%%%%%%%%%%%%
% %%%%%%%%%%%%%%%%%%%%%%%%%%%%%%%%%%%%%%%%%%%%%%%%%%%%%%%%%%%%%%%%%%%%%%%%%%%%%%
% \section{Sample}
%\iffalse
%<*samplemain>
%\fi
%
% The following presents a sample document
% with two chapters, two parts, a title page,
% a compile flag as well as three forwarding files to set the flag.
% It consists of eight |.tex| files:
% \begin{center}
% \begin{tabular}{ll}
% |cdocsamp.tex|&main file\\
% |cdocsch1.tex|&include file for chapter 1\\
% |cdocsch2.tex|&include file for chapter 2\\
% |cdocspt3.tex|&include file for part 3\\
% |cdocspt4.tex|&include file for part 4\\
% |cdocsdrf.tex|&forwarding file for main file in draft mode\\
% |cdocsfi1.tex|&forwarding file for final version of chapter 1\\
% |cdocsfi2.tex|&forwarding file for final version of chapter 2\\
% \end{tabular}
% \end{center}
% Each of the eight files can be compiled directly by the \LaTeX{} compiler.
%
% %%%%%%%%%%%%%%%%%%%%%%%%%%%%%%%%%%%%%%
% \paragraph{Main File.}
%
% The main file is called |cdocsamp.tex|.
%
% Load the \textsf{childdoc} definitions and
% declare the filename for the main document:
%    \begin{macrocode}
\input{childdoc.def}
\childdocmain{}
%    \end{macrocode}

% Optional override for |\version| flag:
%    \begin{macrocode}
%%\ifchilddoc\else\providecommand{\version}{draft}\fi
%    \end{macrocode}

% Define the default values for the |\version| flag
% (|final| for the main file and |draft| for childs):
%    \begin{macrocode}
\ifchilddoc
\providecommand{\version}{draft}
\else
\providecommand{\version}{final}
\fi
%    \end{macrocode}

% Load the standard document class:
%    \begin{macrocode}
\documentclass[12pt]{article}
%    \end{macrocode}

% Start the document body:
%    \begin{macrocode}
\begin{document}
%    \end{macrocode}

% Declare a title page.
% Print title, part of document being processed and version flag:
%    \begin{macrocode}
\addtocounter{page}{-1}
\begin{center}
{\LARGE\bfseries{}childdoc example\par}
\vspace{1cm}
\ifchilddoc
\ifchilddocmanual part\else chapter\fi:
`\childdocname' of `\childdocjob'\par
\else
main document: `\childdocjob'\par
\fi
version: \version\par
\end{center}
\newpage
%    \end{macrocode}

% Manually include selected file,
% otherwise process as usual:
%    \begin{macrocode}
\ifchilddocmanual
\section*{part `\childdocname'}
\input{\childdocname}
\else
%    \end{macrocode}

% Include the two chapters:
%    \begin{macrocode}
\include{cdocsch1}
\include{cdocsch2}
%    \end{macrocode}

% Include the two parts unless only chapters should be displayed:
%    \begin{macrocode}
\ifchilddoc\else
\section{part three}
\input{cdocspt3}
\section{part four}
\input{cdocspt4}
\fi
%    \end{macrocode}

% Process as usual until here:
%    \begin{macrocode}
\fi
%    \end{macrocode}

% End of document body:
%    \begin{macrocode}
\end{document}
%    \end{macrocode}
%\iffalse
%</samplemain>
%\fi
%
% %%%%%%%%%%%%%%%%%%%%%%%%%%%%%%%%%%%%%%
% \paragraph{Chapter Include Files.}
%
% The include files are called |cdocsch1.tex| and |cdocsch2.tex|.
%
%\iffalse
%<*samplechap1|samplechap2>
%\fi

% Optional override for |\version| flag:
%    \begin{macrocode}
%%\providecommand{\version}{final}
%    \end{macrocode}

% Include the main document:
%    \begin{macrocode}
\input{childdoc.def}
\childdocof{cdocsamp}
%    \end{macrocode}

%\iffalse
%</samplechap1|samplechap2>
%\fi
%
%\iffalse
%<*samplechap1>
%\fi
% Some text for chapter 1:
%    \begin{macrocode}
\section{one}
some text in chapter one
%    \end{macrocode}

%\iffalse
%</samplechap1>
%\fi
% Some text for chapter 2:
%\iffalse
%<*samplechap2>
%\fi
%    \begin{macrocode}
\section{two}
more text in chapter two
%    \end{macrocode}

%\iffalse
%</samplechap2>
%\fi
%
% %%%%%%%%%%%%%%%%%%%%%%%%%%%%%%%%%%%%%%
% \paragraph{Part Include Files.}
%
% The include files are called |cdocspt3.tex| and |cdocspt4.tex|.
%
%\iffalse
%<*samplepart3|samplepart4>
%\fi

% Optional override for |\version| flag:
%    \begin{macrocode}
%%\providecommand{\version}{final}
%    \end{macrocode}

% Include the main document:
%    \begin{macrocode}
\input{childdoc.def}
\childdocby{cdocsamp}
%    \end{macrocode}

%\iffalse
%</samplepart3|samplepart4>
%\fi
%
%\iffalse
%<*samplepart3>
%\fi
% Some text for part 3:
%    \begin{macrocode}
some text in part three
%    \end{macrocode}

%\iffalse
%</samplepart3>
%\fi
% Some text for part 4:
%\iffalse
%<*samplepart4>
%\fi
%    \begin{macrocode}
more text in part four
%    \end{macrocode}

%\iffalse
%</samplepart4>
%\fi
%
% %%%%%%%%%%%%%%%%%%%%%%%%%%%%%%%%%%%%%%
% \paragraph{Forwarding for a Complete Draft.}
%
% The following forwarding file |cdocsdrf.tex|
% compiles the main document in draft mode:
%\iffalse
%<*sampledraft>
%\fi
%    \begin{macrocode}
\def\version{draft}
\input{childdoc.def}
\childdocforward{cdocsamp}
%    \end{macrocode}

%\iffalse
%</sampledraft>
%\fi
%
% %%%%%%%%%%%%%%%%%%%%%%%%%%%%%%%%%%%%%%
% \paragraph{Forwarding for Final Version of the Chapters.}
%
% The following forwarding files |cdocsfn1.tex| and |cdocsfn2.tex|
% (with identical content)
% compile the final versions of the child documents
% |cdocsch1.tex| and |cdocsch2.tex|, respectively:
%\iffalse
%<*samplefinal>
%\fi
%    \begin{macrocode}
\def\version{final}
\input{childdoc.def}
\childdocforwardprefix[cdocsamp]{cdocsfn}{cdocsch}
%    \end{macrocode}

%\iffalse
%</samplefinal>
%\fi
%
% %%%%%%%%%%%%%%%%%%%%%%%%%%%%%%%%%%%%%%
% \paragraph{Command Line Processing.}
%
% The following three command lines generate the output files
% |cdocscld|, |cdocscl1| and |cdocscl2|
% which should be identical to
% |cdocsdrf|, |cdocsch1| and |cdocsfn2|, respectively:
% \begin{center}
% \begin{tabular}{l}
% |latex -jobname cdocscld \|\\
% |  "\def\version{draft}\input{childdoc.def}\childdocforward{cdocsamp}"|\\
% |latex -jobname cdocscl1 \|\\
% |  "\input{childdoc.def}\childdocforward[cdocsamp]{cdocsch1}"|\\
% |latex -jobname cdocscl2 \|\\
% |  "\def\version{final}\input{childdoc.def}\childdocforward{cdocsch2}"|
% \end{tabular}
% \end{center}
% Note that the trailing backslash on each first line
% merely continues the input to the second line
% (for convenient cut ant paste).
% Furthermore, the command |latex| can be replaced by any
% of its alternative versions such as |pdflatex|.
%
% %%%%%%%%%%%%%%%%%%%%%%%%%%%%%%%%%%%%%%%%%%%%%%%%%%%%%%%%%%%%%%%%%%%%%%%%%%%%%%
% %%%%%%%%%%%%%%%%%%%%%%%%%%%%%%%%%%%%%%%%%%%%%%%%%%%%%%%%%%%%%%%%%%%%%%%%%%%%%%
% \section{Implementation}
%\iffalse
%<*package>
%\fi
%
% This section describes the definitions file |childdoc.def|.

% The definitions cannot be loaded using |\usepackage| or |\RequirePackage|
% which has a mechanism to prevent loading a style file more than once.
% When loading the definitions by means of |\input|
% multiple instances have to be prevented manually:
%\iffalse
%This code needs to be before the `\ProvidesFile' directive
%which is defined at the beginning of this file.
%Therefore it is also placed there and commented out here.
%</package>
%<*discard>
%\fi
%    \begin{macrocode}
\ifdefined\childdocmain\endinput\fi
%    \end{macrocode}
%\iffalse
%</discard>
%<*package>
%\fi
%
% \macro{\ifchilddoc}
% \macro{\ifchilddocmanual}
% The conditional |\ifchilddoc| tells whether a
% child (true) or main (false) document is being compiled.
% The conditional |\ifchilddocmanual| tells whether
% the |\includeonly| mechanism is used (false) or
% the selection of child files must be performed manually (true).
% The definitions initialise to false:
%    \begin{macrocode}
\newif\ifchilddoc
\newif\ifchilddocmanual
%    \end{macrocode}

% \macro{\childdocname}
% \macro{\childdocjob}
% The macro |\childdocname| stores the name of the main document
% to be compiled. The macro |\childdocjob| stores the name of
% the document on which the \LaTeX{} compiler was originally invoked.
% The content of |\jobname| cannot be compared
% to filenames specified in the source due to different catcodes.
% The following code rescans |\jobname|, stores the result
% in |\childdocname| and saves a copy in |\childdocjob|:
%    \begin{macrocode}
\edef\childdocname{\scantokens\expandafter{\jobname\noexpand}}
\let\childdocjob\childdocname
%    \end{macrocode}

% \macro{\childdocdisable}
% The macro |\childdocdisable| prevents the main file
% from being processed more than once.
% At this stage, the main document command |\childdocmain|
% is assumed to be called once again where it should do nothing.
% Any subsequent call to it should prevent
% a secondary processing of the main document
% It overwrites the forwarding commands
% |\childdocof| and |\childdocforward|
% with empty macros to prevent further inclusions of the main document:
%    \begin{macrocode}
\newcommand{\childdocdisable}
{
  \renewcommand{\childdocmain}[1]{\renewcommand{\childdocmain}[1]{\endinput}}
  \renewcommand{\childdocof}[1]{}
  \renewcommand{\childdocby}[2][]{}
  \renewcommand{\childdocforward}[2][]{}
  \renewcommand{\childdocdisable}{}
}
%    \end{macrocode}

% \macro{\childdocmain}
% The macro |\childdocmain| is to be called at the top of the main file
% with nothing or the main filename (without extension) as argument.
% First, it breaks loops.
% If the argument is not empty and does not match |\childdocname|
% (which is set by the first inclusion of |childdoc.def|),
% |\ifchilddoc| is set to true, |\includeonly| is applied to the child file
% and |\jobname| is set to the main file
% (for proper handling of |.aux| files):
%    \begin{macrocode}
\newcommand{\childdocmain}[1]
{
  \childdocdisable\childdocmain{}
  \if?#1?\else
    \begingroup
      \def\childdoctmp{#1}
      \ifx\childdoctmp\childdocname
        \def\childdoctmp{}
      \else
        \def\childdoctmp
        {
          \childdoctrue
          \includeonly{\childdocname}
          \def\childdocjob{#1}
          \def\jobname{#1}
        }
      \fi
      \expandafter
    \endgroup
    \childdoctmp
  \fi
}
%    \end{macrocode}

% \macro{\childdocof}
% The command |\childdocof| redirects
% compilation to the main file |#1|.
%    \begin{macrocode}
\newcommand{\childdocof}[1]
{
  \childdocdisable
  \childdoctrue
  \includeonly{\childdocname}
  \def\jobname{#1}
  \def\childdocjob{#1}
  \input{#1}
}
%    \end{macrocode}

% \macro{\childdocby}
% The command |\childdocby| ....
%    \begin{macrocode}
\newcommand{\childdocby}[2][]
{
  \childdocdisable
  \childdoctrue
  \childdocmanualtrue
  \if?#1?\else
    \def\jobname{#2}
  \fi
  \def\childdocjob{#2}
  \input{#2}
  \endinput
}
%    \end{macrocode}

% \macro{\childdocforward}
% The command |\childdocforward| redirects
% compilation to the main file or
% (if the optional argument is given) a child file.
% Parameters are set as if the main file
% or a child file starting with |\childdocof| was compiled.
% Then compilation is handed over to the main file:
%    \begin{macrocode}
\newcommand{\childdocforward}[2][]
{
  \begingroup
    \if?#1?
      \def\childdoctmp
      {
        \def\childdocname{#2}
        \def\childdocjob{#2}
        \def\jobname{#2}
        \input{#2}
        \endinput
      }
    \else
      \def\childdoctmp
      {
        \childdocdisable
        \def\childdocname{#2}
        \childdoctrue
        \includeonly{#2}
        \def\childdocjob{#1}
        \def\jobname{#1}
        \input{#1}
        \endinput
      }
    \fi
    \expandafter
  \endgroup
  \childdoctmp
}
%    \end{macrocode}

% \macro{\childdocforwardprefix}
% The command |\childdocforwardprefix| redirects
% compilation to the main or a child file by means of a pattern.
% The prefix |#1| in the current filename is replaced by |#2|
% and the suffix of the current filename is kept
% (it is assumed that the filename does not contain the substring `|~~~|'
% which is used as a delimiter).
% Compilation is handed over to the new file by |\childdocforward|:
%    \begin{macrocode}
\newcommand{\childdocforwardprefix}[3][]
{
  \begingroup
    \def\childdocextract #2##1~~~{\def\childdoctmp{\childdocforward[#1]{#3##1}}}
    \expandafter\childdocextract\childdocname~~~
    \expandafter
  \endgroup
  \childdoctmp
}
%    \end{macrocode}

% \macro{\childdoc}
% The deprecated macro |\childdoc| is a legacy version of |\childdocmain|:
%    \begin{macrocode}
\newcommand{\childdoc}{\childdocmain}
%    \end{macrocode}

% \macro{\childdocredirect}
% The deprecated macro |\childdocredirect| is a legacy version
% of |\childdocforward| and |\childdocforwardprefix|:
%    \begin{macrocode}
\newcommand{\childdocredirect}[2][]
{
  \begingroup
    \if?#1?
      \def\childdoctmp{\childdocforward{#2}}
    \else
      \def\childdoctmp{\childdocforwardprefix{#1}{#2}}
    \fi
    \expandafter
  \endgroup
  \childdoctmp
}
%    \end{macrocode}

%\iffalse
%</package>
%\fi
%
\endinput
|\\
|\childdocforward[|\textit{main}|]{|\textit{dest}|}|\\
\end{tabular}
\end{center}
%
The argument \textit{dest} is the destination file
(without extension).
It should be the main file or one of the child files.
Note that further \textsf{childdoc} directives
such as |\childdocof| and |\childdocforward|
in the indicated file will be processed in this form.
The optional argument \textit{main}
passes on directly to the main file \textit{main}
while pretending to compile the child \textit{dest}.
This form behaves as if \textit{dest}
issues |\childdocof{|\textit{main}|}| right away,
and no further \textsf{childdoc} directives will be processed.

%%%%%%%%%%%%%%%%%%%%%%%%%%%%%%%%%%%%%%%%
\DescribeMacro{\...prefix}
In the alternative form |\childdocforwardprefix|,
%
\begin{center}
\begin{tabular}{l}
|% \iffalse
%
% childdoc.dtx Copyright (C) 2017-2018 Niklas Beisert
%
% This work may be distributed and/or modified under the
% conditions of the LaTeX Project Public License, either version 1.3
% of this license or (at your option) any later version.
% The latest version of this license is in
%   http://www.latex-project.org/lppl.txt
% and version 1.3 or later is part of all distributions of LaTeX
% version 2005/12/01 or later.
%
% This work has the LPPL maintenance status `maintained'.
%
% The Current Maintainer of this work is Niklas Beisert.
%
% This work consists of the files childdoc.dtx and childdoc.ins
% and the derived files childdoc.def and cdocsamp.tex with
% cdocsch1.tex, cdocsch2.tex, cdocsdrf.tex, cdocsfn1.tex, cdocsfn2.tex.
%
%<package>\ifdefined\childdocmain\endinput\fi
%<package>\ProvidesFile{childdoc.def}[2018/12/30 v2.0 child document driver]
%<samplemain>\ProvidesFile{cdocsamp.tex}[2018/12/30 v2.0 sample for childdoc]
%<*driver>
%\ProvidesFile{childdoc.drv}[2018/12/30 v2.0 childdoc reference manual file]
\PassOptionsToClass{10pt,a4paper}{article}
\documentclass{ltxdoc}

\usepackage[margin=35mm]{geometry}
\usepackage{hyperref}
\usepackage{hyperxmp}
\usepackage[usenames]{color}

\hypersetup{colorlinks=true}
\hypersetup{pdfstartview=FitH}
\hypersetup{pdfpagemode=UseNone}
\hypersetup{pdfsource={}}
\hypersetup{pdflang={en-UK}}
\hypersetup{pdfcopyright={Copyright 2017-2018 Niklas Beisert.
  This work may be distributed and/or modified under the
  conditions of the LaTeX Project Public License, either version 1.3
  of this license or (at your option) any later version.}}
\hypersetup{pdflicenseurl={http://www.latex-project.org/lppl.txt}}
\hypersetup{pdfcontactaddress={ETH Zurich, ITP, HIT K,
  Wolfgang-Pauli-Strasse 27}}
\hypersetup{pdfcontactpostcode={8093}}
\hypersetup{pdfcontactcity={Zurich}}
\hypersetup{pdfcontactcountry={Switzerland}}
\hypersetup{pdfcontactemail={nbeisert@itp.phys.ethz.ch}}
\hypersetup{pdfcontacturl={http://people.phys.ethz.ch/\xmptilde nbeisert/}}

\newcommand{\secref}[1]{\hyperref[#1]{section \ref*{#1}}}

\parskip1ex
\parindent0pt
\let\olditemize\itemize
\def\itemize{\olditemize\parskip0pt}

\begin{document}

\title{The \textsf{childdoc} Package}
\hypersetup{pdftitle={The childdoc Package}}
\author{Niklas Beisert\\[2ex]
  Institut f\"ur Theoretische Physik\\
  Eidgen\"ossische Technische Hochschule Z\"urich\\
  Wolfgang-Pauli-Strasse 27, 8093 Z\"urich, Switzerland\\[1ex]
  \href{mailto:nbeisert@itp.phys.ethz.ch}
  {\texttt{nbeisert@itp.phys.ethz.ch}}}
\hypersetup{pdfauthor={Niklas Beisert}}
\hypersetup{pdfsubject={Manual for the LaTeX2e Package childdoc}}
\date{30 December 2018, \textsf{v2.0}}
\maketitle

\begin{abstract}\noindent
\textsf{childdoc} is a \LaTeXe{} package
that enables the direct compilation
of document sections included by |\include|
to individual files.
\end{abstract}

\begingroup
\parskip0ex
\tableofcontents
\endgroup

%%%%%%%%%%%%%%%%%%%%%%%%%%%%%%%%%%%%%%%%%%%%%%%%%%%%%%%%%%%%%%%%%%%%%%%%%%%%%%%%
%%%%%%%%%%%%%%%%%%%%%%%%%%%%%%%%%%%%%%%%%%%%%%%%%%%%%%%%%%%%%%%%%%%%%%%%%%%%%%%%
\section{Introduction}

\LaTeX{} provides a mechanism to structure a large document (such as a book)
into a main file and several child files (containing the chapters)
using the |\include| command.
This mechanism is beneficial for documents
which span hundreds of pages in order to
make the source file(s) more manageable.
Moreover, compilation can be restricted to
selected child files by means of the |\includeonly| command.
The latter feature can be used to reduce the compilation time while editing
(this was significantly more useful in the earlier days of \LaTeX{})
or to generate a smaller document which is easier to navigate.
Another application of |\includeonly| is to generate
documents consisting of selected parts of the complete document.

However, there are a few drawbacks of the plain |\include| mechanism:
\begin{itemize}
\item
The child files cannot be compiled on their own,
they can only be compiled via the main file.
A naive editing environment
(such as a text editor with an option
to have the current file processed by \LaTeX)
may require one to switch to the main file before compiling;
attempting to compile the child file produces errors.
\item
The main file must be modified (each time)
to adjust the |\includeonly| command
to the present needs. This easily leaves the main file in a messy state.
\item
The generated document will always carry the filename
of the main document. This is inconvenient if
several child files are to be compiled and
to be kept for distribution.
\end{itemize}

The present package provides a simple interface
to make child files individually compilable by \LaTeX{}.
Compiling a child file then has the same effect as compiling
the main file with an |\includeonly| command
to select the appropriate child.
Moreover the generated document will carry the name of the child
rather than the main file.
This resolves all three above issues.

This feature is meant to make the editing of books,
thesis documents and lecture notes somewhat more convenient.
However, the package can also be used efficiently for
composing a series of documents (such as exercise sheets)
which are typically distributed individually.
It then assists the author in generating the individual documents
(potentially in different versions)
as well as a document containing the collected series.
Another application is in developing style files
or other kinds of included material
where compilation of the style file could redirect
to a sample or test file.

%%%%%%%%%%%%%%%%%%%%%%%%%%%%%%%%%%%%%%%%%%%%%%%%%%%%%%%%%%%%%%%%%%%%%%%%%%%%%%%%
%%%%%%%%%%%%%%%%%%%%%%%%%%%%%%%%%%%%%%%%%%%%%%%%%%%%%%%%%%%%%%%%%%%%%%%%%%%%%%%%
\section{Usage}

First of all, the package \textsf{childdoc} is \emph{not} a standard
\LaTeXe{} |.sty| style file! Therefore it needs to be invoked in
a non-standard way.

%%%%%%%%%%%%%%%%%%%%%%%%%%%%%%%%%%%%%%%%%%%%%%%%%%%%%%%%%%%%%%%%%%%%%%%%%%%%%%%%
\subsection{Included Files}
\label{sec:include}

%%%%%%%%%%%%%%%%%%%%%%%%%%%%%%%%%%%%%%%%
\DescribeMacro{\childdocmain}
To use the package, add the commands
\begin{center}
\begin{tabular}{l}
|\input{childdoc.def}|\\
|\childdocmain{}|\\
\end{tabular}
\end{center}
at the very top of the main \LaTeX{} file,
in particular \emph{before} the |\documentclass| statement!
The argument of |\childdocmain| should be left empty
(but it must be present).

%%%%%%%%%%%%%%%%%%%%%%%%%%%%%%%%%%%%%%%%
\DescribeMacro{\childdocof}
Furthermore, add the commands
\begin{center}
\begin{tabular}{l}
|\input{childdoc.def}|\\
|\childdocof{|\textit{main}|}|\\
\end{tabular}
\end{center}
at the top of every child file \textit{child}
which is included by |\include{|\textit{child}|}|
from within the main file
(or at least for those files to be compiled individually).
The argument \textit{main} must be the filename of the main file.

There are a couple of
considerations in setting up the main and child documents:

%%%%%%%%%%%%%%%%%%%%%%%%%%%%%%%%%%%%%%%%
\paragraph{Restrictions.}

Please note the following restrictions:
\begin{itemize}
\item
|\childdocmain| must be called with one argument \textit{main}
to ensure compatibility with earlier version of the package.
It must either be empty (|\childdocmain{}|)
or precisely match the filename of the main file in which it is specified.
See \secref{sec:detection} for further information.
\item
The filename \textit{main} must be specified without the |.tex| extension.
\item
The filename \textit{main} is case sensitive
(even in case-insensitive file systems)
due to internal string comparison.
\item
The argument \textit{main} should be fully expanded, it cannot be a macro.
\item
Subdirectories and special characters should be avoided in filenames.
\item
The command |\childdocmain{|\textit{main}|}| must be followed by a whitespace.
It should not be followed immediately by another command
or by a comment mark `|%|'.
This is because the \TeX{} parser reads the token immediately following
the argument of |\childdocmain| and puts it
at the beginning of every child section;
however, a white\-space is ignored.
\end{itemize}

%%%%%%%%%%%%%%%%%%%%%%%%%%%%%%%%%%%%%%%%
\paragraph{Content of Main File.}

It is advisable to place all content in the child files included by |\include|.
Any output contained in the main file will appear in all child documents
unless suppressed manually;
it cannot be suppressed automatically by the |\includeonly| directive
and thus should normally be avoided.
A method to include some content in the main file
by means of conditional processing is described in \secref{sec:conditional}.

%%%%%%%%%%%%%%%%%%%%%%%%%%%%%%%%%%%%%%%%
\paragraph{Page Numbering.}

When only a part of the document is compiled,
the appropriate numbering of pages
(as well as other status parameters)
is determined from the |.aux| files.
The latter contain information from previous passes.
However this information needs to propagate through
all intermediate child documents.
Therefore the page numbering in child documents may well
be inconsistent until the complete document is compiled at least once.

A useful (if unconventional) way to always ensure a consistent
page numbering is to restart the numbering in each child document
and denote the pages by `\textit{child}|.|\textit{page}'
where \textit{child} represents the chapter/section number of the child file.
This can be achieved by the command
|\numberwithin{page}{|\textit{child}|}|
of the \textsf{amsmath} package
where \textit{child} can be |chapter| or |section|
depending on the chosen structuring.
Alternatively, one can modify the macro |\thepage| appropriately
and reset the counter |page| at the start of each child file.

%%%%%%%%%%%%%%%%%%%%%%%%%%%%%%%%%%%%%%%%%%%%%%%%%%%%%%%%%%%%%%%%%%%%%%%%%%%%%%%%
\subsection{Conditional Processing}
\label{sec:conditional}

The package provides a mechanism to compile different versions
of a document. To customise the versions further some conditional processing
can come in handy to distinguish which version is being compiled.
The package provides two macros to describe the compilation context:

%%%%%%%%%%%%%%%%%%%%%%%%%%%%%%%%%%%%%%%%
\DescribeMacro{\ifchilddoc}
The conditional |\ifchilddoc| distinguishes between the compilation of
child documents and the main document:
%
\begin{center}
|\ifchilddoc |\textit{child-code}| |[|\||else |\textit{main-code}]| \||fi|
\end{center}

%%%%%%%%%%%%%%%%%%%%%%%%%%%%%%%%%%%%%%%%
\DescribeMacro{\childdocname}
\DescribeMacro{\childdocjob}
The macro |\childdocname| contains the filename (without extension)
of the main or child file being processed.
Note that |\childdocjob| will always contain the name of the main file.

%%%%%%%%%%%%%%%%%%%%%%%%%%%%%%%%%%%%%%%%
\paragraph{Title Page.}

Conditional processing can be used to include a title or banner page
in the main document when proper precautions are taken.
Importantly, the code in the main file should ensure that the page counter
(as well as other status parameters which are stored in the |.aux| files)
takes the same value after the conditional processing.
Otherwise the page numbers may take divergent values
depending on which part is compiled.

For example, a title page could be declared by:
%
\begin{center}
\begin{tabular}{l}
|\ifchilddoc\||else|\\
|\addtocounter{page}{-1}|\\
\textit{code for title page}\\
|\newpage|\\
|\||fi|
\end{tabular}
\end{center}
%
A banner page for the child documents can be generated by:
%
\begin{center}
\begin{tabular}{l}
|\ifchilddoc|\\
|\addtocounter{page}{-1}|\\
\textit{code for banner page}\\
|\newpage|\\
|\||fi|
\end{tabular}
\end{center}
%
Here one could write a message such as:
\begin{center}
|This is the part \childdocname{} of \childdocjob{}.|
\end{center}

%%%%%%%%%%%%%%%%%%%%%%%%%%%%%%%%%%%%%%%%%%%%%%%%%%%%%%%%%%%%%%%%%%%%%%%%%%%%%%%%
\subsection{Flags}
\label{sec:flags}

The package makes it easy to generate different versions
of the main or child documents.
To this end compilation flags can be defined
and assigned different default values.
They will be particularly useful in conjunction
with the forwarding mechanism described in \secref{sec:forward}.

For example, it may be useful to have a flag |\version|
which can be set to |draft| or |final|.
The document source will contain some conditional code
depending on the value of |\version|.
Suppose further, the flag should default to |final| for the main file
and to |draft| for child files
which is a natural assignment for editing the document.
This is achieved by placing the following code
in the preamble of the main document
(below the |\childdocmain| directive):
%
\begin{center}
\begin{tabular}{l}
|\ifchilddoc|\\
|\providecommand{\version}{draft}|\\
|\||else|\\
|\providecommand{\version}{final}|\\
|\||fi|
\end{tabular}
\end{center}
%
The definition by |\providecommand| makes sure
that previous definitions are not overwritten.
Further statements |\providecommand{\version}{...}|
can thus be added before the above code to override it.

For the main file, one might add a line
(between |\childdocmain| and the above block)
%
\begin{center}
|%\ifchilddoc\||else\providecommand{\version}{draft}\||fi|
\end{center}
%
which can be uncommented to produce a draft version.
Likewise one can add a line to the very top of a child file
(above the |\childdocof{|\textit{main}|}| directive)
%
\begin{center}
|%\providecommand{\version}{final}|
\end{center}
%
which can be uncommented to produce the final version of this child document.

%%%%%%%%%%%%%%%%%%%%%%%%%%%%%%%%%%%%%%%%%%%%%%%%%%%%%%%%%%%%%%%%%%%%%%%%%%%%%%%%
\subsection{Forwarding}
\label{sec:forward}

Different versions of the main or child documents
using compilation flags as described in \secref{sec:flags}
can be (permanently) stored in different files
for convenient compilation, viewing and distribution.
To this end, the package defines a command
to pass on compilation to a different file:

%%%%%%%%%%%%%%%%%%%%%%%%%%%%%%%%%%%%%%%%
\DescribeMacro{\childdocforward}
The command |\childdocforward| redirects processing to
another source file:
%
\begin{center}
\begin{tabular}{l}
|\input{childdoc.def}|\\
|\childdocforward[|\textit{main}|]{|\textit{dest}|}|\\
\end{tabular}
\end{center}
%
The argument \textit{dest} is the destination file
(without extension).
It should be the main file or one of the child files.
Note that further \textsf{childdoc} directives
such as |\childdocof| and |\childdocforward|
in the indicated file will be processed in this form.
The optional argument \textit{main}
passes on directly to the main file \textit{main}
while pretending to compile the child \textit{dest}.
This form behaves as if \textit{dest}
issues |\childdocof{|\textit{main}|}| right away,
and no further \textsf{childdoc} directives will be processed.

%%%%%%%%%%%%%%%%%%%%%%%%%%%%%%%%%%%%%%%%
\DescribeMacro{\...prefix}
In the alternative form |\childdocforwardprefix|,
%
\begin{center}
\begin{tabular}{l}
|\input{childdoc.def}|\\
|\childdocforwardprefix[|\textit{main}|]{|\textit{prefix}|}{|\textit{dest}|}|
\end{tabular}
\end{center}
%
the destination file is determined by a pattern
depending on the current file:
To make this work, the current file must be called
`{\textit{prefix}\hspace{0.2em}\textit{suffix}}'
with \textit{prefix} matching precisely the argument.
Processing is then passed on to the file
`{\textit{dest}\hspace{0.2em}\textit{suffix}}'.
Surely, the same effect is achieved by
directly specifying the
argument `{\textit{dest}\hspace{0.2em}\textit{suffix}}'
in the first form.
However, that requires to set up a different file
for each child. With the alternative form of the command
all these files can have exactly the same content
which simplifies setting them up and maintaining them.

For example, the following file |draft.tex|
with a compilation flag |\version| as described in \secref{sec:flags}
compiles the main document as a draft:
%
\begin{center}
\begin{tabular}{l}
|\def\version{draft}|\\
|\input{childdoc.def}|\\
|\childdocforward{|\textit{main}|}|
\end{tabular}
\end{center}
%
Likewise, the following files |final|\textit{nn}|.tex|
compile the final version of the child document
|child|\textit{nn}|.tex|:
%
\begin{center}
\begin{tabular}{l}
|\def\version{final}|\\
|\input{childdoc.def}|\\
|\childdocforwardprefix{final}{child}|
\end{tabular}
\end{center}
%

Note that when several versions of a main file and/or of each child file
are to be generated, it may be convenient to set up a |Makefile| or
shell script to automatise the process.

%%%%%%%%%%%%%%%%%%%%%%%%%%%%%%%%%%%%%%%%%%%%%%%%%%%%%%%%%%%%%%%%%%%%%%%%%%%%%%%%
\subsection{Command Line Processing}
\label{sec:commandline}

The effect of redirection files can also be achieved by invoking
the \LaTeX{} compiler with a more elaborate command line.
Most conveniently this should be done as part
of a shell script or a |Makefile|.

When using \textsf{childdoc} in the main file, the following
command lines effectively perform a redirection
(note that depending on the shell being used,
backslashes may have to be doubled: `|\|' $\to$ `|\\|'):
%
\begin{center}
|... -jobname "|\textit{target}|" |\\|"|[\textit{flags}]%
|\input{childdoc.def}\childdocforward[|\textit{main}|]{|\textit{dest}|}"|
\end{center}
%
Here \textit{target} is the name of the output file,
\textit{main} is the name of the main file
and \textit{dest} is the name of the main or child file to be processed
(all filenames without extensions).
The optional argument \textit{main} can be omitted
if \textit{main} matches \textit{dest}.
Optionally, compilation \textit{flags} can be defined via |\def| commands.
This command line makes the \TeX{} engine believe
it is compiling the file \textit{target}
whose content is specified as the latter parameter.
The provided code then forwards the processing to
\textit{main} or \textit{dest} as described in \secref{sec:forward}.

%%%%%%%%%%%%%%%%%%%%%%%%%%%%%%%%%%%%%%%%%%%%%%%%%%%%%%%%%%%%%%%%%%%%%%%%%%%%%%%%
\subsection{Include by Input}
\label{sec:input}

Including child documents by |\include| has some restrictions by design.
Most notably, the content of a child document always occupies
its own set of pages; pages cannot be shared between child documents.
Usually, this behaviour makes perfect sense
because each child document contain an essential part of the document.
However, in some situations it may be desirable to compose
a document from a collection of parts
without having mandatory page breaks between then.
For this case, the package
provides a mechanism to include parts
by |\input| which can also be processed individually.
However, by construction this mechanism
requires manual handling of the content to be output.

%%%%%%%%%%%%%%%%%%%%%%%%%%%%%%%%%%%%%%%%
\DescribeMacro{\ifchilddocmanual}
The main file should be prepared as usual, see \secref{sec:include}.
However, the document body must make a distinction
between processing of an individual part and of the main document, e.g.:
%
\begin{center}
\begin{tabular}{l}
|\ifchilddocmanual|\\
|\input{\childdocname}|\\
|\||else|\\
\textit{document body with }|\input{|\textit{part}|}|\\
|\||fi|
\end{tabular}
\end{center}
%
The conditional |\ifchilddocmanual| is true whenever
a part to be included by |\input| is being compiled,
and the name of the part is stored in |\childdocname|.

%%%%%%%%%%%%%%%%%%%%%%%%%%%%%%%%%%%%%%%%
\DescribeMacro{\childdocby}
Each part to be included by |\input| should start with:
%
\begin{center}
\begin{tabular}{l}
|\input{childdoc.def}|\\
|\childdocby{|\textit{main}|}|\\
\end{tabular}
\end{center}
%
The directive |\childdocby| is similar to |\childdocof|
described in \secref{sec:include},
but the subsequent selection of content must be done manually.
To that end, both |\ifchilddoc| and |\ifchilddocmanual|
will be true upon processing of a part,
and the name of the part is stored in |\childdocname|.
Note that |\jobname| will be set to the filename of the current part
so that each part receives an individual |.aux| file
that does not interfere with the |.aux| file(s) of the main document.
This behaviour can be altered by the alternative form
|\childdocby[*]{|\textit{main}|}| (with a non-empty optional argument)
which uses the |.aux| file of the main document
by setting |\jobname| to \textit{main}.

%%%%%%%%%%%%%%%%%%%%%%%%%%%%%%%%%%%%%%%%%%%%%%%%%%%%%%%%%%%%%%%%%%%%%%%%%%%%%%%%
\subsection{Driver Development}
\label{sec:driver}

The \textsf{childdoc} mechanism can also be use for the development
of definition files such as \LaTeX{} styles or classes.
This case differs from the above setup with multiple parts
included by |\include| in that no |\includeonly| should be invoked.
This can be achieved by starting the include file
(before |\ProvidesPackage|) with:
%
\begin{center}
\begin{tabular}{l}
|\input{childdoc.def}|\\
|\childdocforward{|\textit{main}|}|\\
\end{tabular}
\end{center}
%
or alternatively with:
%
\begin{center}
\begin{tabular}{l}
|\input{childdoc.def}|\\
|\childdocby{|\textit{main}|}|\\
\end{tabular}
\end{center}
%
Both forms have slightly different effects as described above.
The main file is prepared as usual, see \secref{sec:include}.

%%%%%%%%%%%%%%%%%%%%%%%%%%%%%%%%%%%%%%%%%%%%%%%%%%%%%%%%%%%%%%%%%%%%%%%%%%%%%%%%
\subsection{Legacy Detection}
\label{sec:detection}

The directive |\childdocmain| in the main file can detect
whether the complete document or merely a child is to be compiled
even without using the directive |\childdocof|.
This method is deprecated because it is less robust
and there is no compelling reason to use it;
it is merely provided for backward compatibility
and it may be removed in future versions.

If the detection mechanism is to be used,
it is mandatory to correctly specify
the filename of the main file as the argument of |\childdocmain|:
%
\begin{center}
\begin{tabular}{l}
|\input{childdoc.def}|\\
|\childdocmain{|\textit{main}|}|\\
\end{tabular}
\end{center}
%
If |\jobname| does not match the argument \textit{main} of |\childdocmain|,
it is assumed that |\jobname| points to the child file to be compiled.
When using |\childdocmain| with the main file specified as argument,
it suffices to start a child file
with just |\input{|\textit{main}|}|
without loading of the package and using |\childdocof|.
If instead all processing is done
with the appropriate \textsf{childdoc} directives,
the argument of \textit{main} of |\childdocmain| can be empty.

An alternative version of the command line processing described
in \secref{sec:commandline} using the detection mechanism reads:
%
\begin{center}
|... -jobname "|\textit{target}|" "|[\textit{flags}]%
[|\def\jobname{|\textit{dest}|}|]|\input{|\textit{main}|}"|
\end{center}

%%%%%%%%%%%%%%%%%%%%%%%%%%%%%%%%%%%%%%%%%%%%%%%%%%%%%%%%%%%%%%%%%%%%%%%%%%%%%%%%
\subsection{Manual Code}
\label{sec:manual}

In case one cannot be certain whether the definitions file |childdoc.def|
is installed on the target \TeX{} distribution
and one prefers not to ship it,
it is conceivable to paste a few relevant commands into the sources.

To that end, drop all statements |\input{childdoc.def}|
and perform the replacements as outlined below.
Instead of |\childdocmain{|\textit{main}|}| add the following code
to the top of the main file:
%
\begin{center}
\begin{tabular}{l}
|\||ifdefined\childdocname\endinput\||fi\newif\ifchilddoc|\\
|\edef\childdocname{\scantokens\expandafter{\jobname\noexpand}}|\\
|\def\childdocmain{|\textit{main}|}\||ifx\childdocmain\childdocname\||else|\\
|\childdoctrue\includeonly{\childdocname}\let\jobname\childdocmain\||fi|\\
\end{tabular}
\end{center}
%
Instead of |\childdocof{|\textit{main}|}| just include the main file
at the top of each child file:
%
\begin{center}
|\input{|\textit{main}|}|
\end{center}
%
A simple redirection |\childdocforward{|\textit{dest}|}| is achieved by:
%
\begin{center}
|\def\jobname{|\textit{dest}|}\input{\jobname}|
\end{center}
%
The redirection with prefix
|\childdocforwardprefix[|\textit{prefix}|]{|\textit{dest}|}|
is accomplished by:
%
\begin{center}
\begin{tabular}{l}
|{\edef\jobname{\scantokens\expandafter{\jobname\noexpand}}|\\
|\def\redirectjob |\textit{prefix}|#1~~~{\gdef\jobname{|\textit{dest}|#1}}|\\
|\expandafter\redirectjob\jobname~~~}\input{\jobname}|
\end{tabular}
\end{center}

In an alternative approach,
child documents can be compiled by a specific command line
without additional code or specific definitions:
%
\begin{center}
|... -jobname "|\textit{target}|" "|[\textit{flags}]%
|\includeonly{|\textit{dest}|}\input{|\textit{main}|}"|
\end{center}
%

%%%%%%%%%%%%%%%%%%%%%%%%%%%%%%%%%%%%%%%%%%%%%%%%%%%%%%%%%%%%%%%%%%%%%%%%%%%%%%%%
%%%%%%%%%%%%%%%%%%%%%%%%%%%%%%%%%%%%%%%%%%%%%%%%%%%%%%%%%%%%%%%%%%%%%%%%%%%%%%%%
\section{Information}

%%%%%%%%%%%%%%%%%%%%%%%%%%%%%%%%%%%%%%%%%%%%%%%%%%%%%%%%%%%%%%%%%%%%%%%%%%%%%%%%
\subsection{Copyright}

Copyright \copyright{} 2017--2018 Niklas Beisert

This work may be distributed and/or modified under the
conditions of the \LaTeX{} Project Public License, either version 1.3
of this license or (at your option) any later version.
The latest version of this license is in
  \url{http://www.latex-project.org/lppl.txt}
and version 1.3 or later is part of all distributions of \LaTeX{}
version 2005/12/01 or later.

This work has the LPPL maintenance status `maintained'.

The Current Maintainer of this work is Niklas Beisert.

This work consists of the files |README.txt|, |childdoc.ins| and |childdoc.dtx|
as well as the derived files |childdoc.def|, |cdocsamp.tex|
with |cdocsch1.tex|, |cdocsch2.tex|, |cdocspt3.tex|, |cdocspt4.tex|,
|cdocsdrf.tex|, |cdocsfn1.tex|, |cdocsfn2.tex|
as well as |childdoc.pdf|.

%%%%%%%%%%%%%%%%%%%%%%%%%%%%%%%%%%%%%%%%%%%%%%%%%%%%%%%%%%%%%%%%%%%%%%%%%%%%%%%%
\subsection{Files and Installation}

The package consists of the files:
%
\begin{center}
\begin{tabular}{ll}
    |README.txt|   & readme file \\
    |childdoc.ins| & installation file \\
    |childdoc.dtx| & source file \\
    |childdoc.def| & definition file \\
    |cdocsamp.tex| & sample main file \\
    |cdocsch1.tex| & sample include file \\
    |cdocsch2.tex| & sample include file \\
    |cdocspt3.tex| & sample part file \\
    |cdocspt4.tex| & sample part file \\
    |cdocsdrf.tex| & sample redirection file \\
    |cdocsfn1.tex| & sample redirection file \\
    |cdocsfn2.tex| & sample redirection file \\
    |childdoc.pdf| & manual
\end{tabular}
\end{center}
%
The distribution consists of the files
|README.txt|, |childdoc.ins| and |childdoc.dtx|.
%
\begin{itemize}
\item
Run (pdf)\LaTeX{} on |childdoc.dtx|
to compile the manual |childdoc.pdf| (this file).
\item
Run \LaTeX{} on |childdoc.ins| to create the definitions file |childdoc.def|
and the sample |cdocsamp.tex| with include files
|cdocsch1.tex|, |cdocsch2.tex|, |cdocspt3.tex|, |cdocspt4.tex|,
|cdocsdrf.tex|, |cdocsfn1.tex|, |cdocsfn2.tex|.
Then copy the file |childdoc.def| to an appropriate directory of your \LaTeX{}
distribution, e.g.\ \textit{texmf-root}|/tex/latex/childdoc|.
\end{itemize}

%%%%%%%%%%%%%%%%%%%%%%%%%%%%%%%%%%%%%%%%%%%%%%%%%%%%%%%%%%%%%%%%%%%%%%%%%%%%%%%%
\subsection{Related CTAN Packages}

There are several other packages which offer a similar functionality:
%
\begin{itemize}
\item
The packages
\href{http://ctan.org/pkg/docmute}{\textsf{docmute}},
\href{http://ctan.org/pkg/includex}{\textsf{includex}} and
\href{http://ctan.org/pkg/standalone}{\textsf{standalone}}
provide commands to include only the document body of
a child file thus allowing both files to be compiled individually.
\item
The packages \href{http://ctan.org/pkg/subdocs}{\textsf{subdocs}}
and \href{http://ctan.org/pkg/subfiles}{\textsf{subfiles}}
provide structures in which the main and child documents can be
encapsulated and allowing them to be compiled individually.
The inclusion mechanism is different from the conventional |\include|.
\item
The package \href{http://ctan.org/pkg/combine}{\textsf{combine}}
is an elaborate solution to combine several documents into one.
\end{itemize}
%
See also the CTAN topic \href{http://ctan.org/topic/subdocs}{\textsf{subdocs}}
for further related packages.
The present package differs from the above solutions in that
a document structure constructed with the conventional |\include| mechanism
just needs two extra commands at the top of every file
such that all constituent files can be compiled individually.

%%%%%%%%%%%%%%%%%%%%%%%%%%%%%%%%%%%%%%%%%%%%%%%%%%%%%%%%%%%%%%%%%%%%%%%%%%%%%%%%
%\subsection{Feature Suggestions}
%
%The following is a list of features which may be useful for future
%versions of this package:
%%
%\begin{itemize}
%\item
%\ldots
%\end{itemize}

%%%%%%%%%%%%%%%%%%%%%%%%%%%%%%%%%%%%%%%%%%%%%%%%%%%%%%%%%%%%%%%%%%%%%%%%%%%%%%%%
\subsection{Revision History}

%%%%%%%%%%%%%%%%%%%%%%%%%%%%%%%%%%%%%%%%
\paragraph{v2.0:} 2018/12/30

\begin{itemize}
\item
immediate forward processing
\item
added |\childdocby| mechanism
\item
manual restructured
\end{itemize}

%%%%%%%%%%%%%%%%%%%%%%%%%%%%%%%%%%%%%%%%
\paragraph{v1.6:} 2018/01/17

\begin{itemize}
\item
application for development of include files
\item
corrections to manual
\end{itemize}

%%%%%%%%%%%%%%%%%%%%%%%%%%%%%%%%%%%%%%%%
\paragraph{v1.5:} 2017/05/21

\begin{itemize}
\item
more complete structuring introduced
\item
|\childdocof| introduced
\item
|\childdoc| renamed to |\childdocmain|
\item
|\childredirect| renamed to |\childdocforward| and |\childdocforwardprefix|
and functionality expanded
\end{itemize}

%%%%%%%%%%%%%%%%%%%%%%%%%%%%%%%%%%%%%%%%
\paragraph{v1.0:} 2017/04/27

\begin{itemize}
\item
manual and install package
\item
first version published on CTAN
\end{itemize}

%%%%%%%%%%%%%%%%%%%%%%%%%%%%%%%%%%%%%%%%
\paragraph{v0.6:} 2017/04/26

\begin{itemize}
\item
redirection mechanism added
\end{itemize}

%%%%%%%%%%%%%%%%%%%%%%%%%%%%%%%%%%%%%%%%
\paragraph{v0.5:} 2017/04/26

\begin{itemize}
\item
functionality in definition file
\end{itemize}


%%%%%%%%%%%%%%%%%%%%%%%%%%%%%%%%%%%%%%%%%%%%%%%%%%%%%%%%%%%%%%%%%%%%%%%%%%%%%%%%
%%%%%%%%%%%%%%%%%%%%%%%%%%%%%%%%%%%%%%%%%%%%%%%%%%%%%%%%%%%%%%%%%%%%%%%%%%%%%%%%
%%%%%%%%%%%%%%%%%%%%%%%%%%%%%%%%%%%%%%%%%%%%%%%%%%%%%%%%%%%%%%%%%%%%%%%%%%%%%%%%
\appendix

\settowidth\MacroIndent{\rmfamily\scriptsize 000\ }

 \DocInput{childdoc.dtx}

\end{document}
%</driver>
% \fi
%
% %%%%%%%%%%%%%%%%%%%%%%%%%%%%%%%%%%%%%%%%%%%%%%%%%%%%%%%%%%%%%%%%%%%%%%%%%%%%%%
% %%%%%%%%%%%%%%%%%%%%%%%%%%%%%%%%%%%%%%%%%%%%%%%%%%%%%%%%%%%%%%%%%%%%%%%%%%%%%%
% \section{Sample}
%\iffalse
%<*samplemain>
%\fi
%
% The following presents a sample document
% with two chapters, two parts, a title page,
% a compile flag as well as three forwarding files to set the flag.
% It consists of eight |.tex| files:
% \begin{center}
% \begin{tabular}{ll}
% |cdocsamp.tex|&main file\\
% |cdocsch1.tex|&include file for chapter 1\\
% |cdocsch2.tex|&include file for chapter 2\\
% |cdocspt3.tex|&include file for part 3\\
% |cdocspt4.tex|&include file for part 4\\
% |cdocsdrf.tex|&forwarding file for main file in draft mode\\
% |cdocsfi1.tex|&forwarding file for final version of chapter 1\\
% |cdocsfi2.tex|&forwarding file for final version of chapter 2\\
% \end{tabular}
% \end{center}
% Each of the eight files can be compiled directly by the \LaTeX{} compiler.
%
% %%%%%%%%%%%%%%%%%%%%%%%%%%%%%%%%%%%%%%
% \paragraph{Main File.}
%
% The main file is called |cdocsamp.tex|.
%
% Load the \textsf{childdoc} definitions and
% declare the filename for the main document:
%    \begin{macrocode}
\input{childdoc.def}
\childdocmain{}
%    \end{macrocode}

% Optional override for |\version| flag:
%    \begin{macrocode}
%%\ifchilddoc\else\providecommand{\version}{draft}\fi
%    \end{macrocode}

% Define the default values for the |\version| flag
% (|final| for the main file and |draft| for childs):
%    \begin{macrocode}
\ifchilddoc
\providecommand{\version}{draft}
\else
\providecommand{\version}{final}
\fi
%    \end{macrocode}

% Load the standard document class:
%    \begin{macrocode}
\documentclass[12pt]{article}
%    \end{macrocode}

% Start the document body:
%    \begin{macrocode}
\begin{document}
%    \end{macrocode}

% Declare a title page.
% Print title, part of document being processed and version flag:
%    \begin{macrocode}
\addtocounter{page}{-1}
\begin{center}
{\LARGE\bfseries{}childdoc example\par}
\vspace{1cm}
\ifchilddoc
\ifchilddocmanual part\else chapter\fi:
`\childdocname' of `\childdocjob'\par
\else
main document: `\childdocjob'\par
\fi
version: \version\par
\end{center}
\newpage
%    \end{macrocode}

% Manually include selected file,
% otherwise process as usual:
%    \begin{macrocode}
\ifchilddocmanual
\section*{part `\childdocname'}
\input{\childdocname}
\else
%    \end{macrocode}

% Include the two chapters:
%    \begin{macrocode}
\include{cdocsch1}
\include{cdocsch2}
%    \end{macrocode}

% Include the two parts unless only chapters should be displayed:
%    \begin{macrocode}
\ifchilddoc\else
\section{part three}
\input{cdocspt3}
\section{part four}
\input{cdocspt4}
\fi
%    \end{macrocode}

% Process as usual until here:
%    \begin{macrocode}
\fi
%    \end{macrocode}

% End of document body:
%    \begin{macrocode}
\end{document}
%    \end{macrocode}
%\iffalse
%</samplemain>
%\fi
%
% %%%%%%%%%%%%%%%%%%%%%%%%%%%%%%%%%%%%%%
% \paragraph{Chapter Include Files.}
%
% The include files are called |cdocsch1.tex| and |cdocsch2.tex|.
%
%\iffalse
%<*samplechap1|samplechap2>
%\fi

% Optional override for |\version| flag:
%    \begin{macrocode}
%%\providecommand{\version}{final}
%    \end{macrocode}

% Include the main document:
%    \begin{macrocode}
\input{childdoc.def}
\childdocof{cdocsamp}
%    \end{macrocode}

%\iffalse
%</samplechap1|samplechap2>
%\fi
%
%\iffalse
%<*samplechap1>
%\fi
% Some text for chapter 1:
%    \begin{macrocode}
\section{one}
some text in chapter one
%    \end{macrocode}

%\iffalse
%</samplechap1>
%\fi
% Some text for chapter 2:
%\iffalse
%<*samplechap2>
%\fi
%    \begin{macrocode}
\section{two}
more text in chapter two
%    \end{macrocode}

%\iffalse
%</samplechap2>
%\fi
%
% %%%%%%%%%%%%%%%%%%%%%%%%%%%%%%%%%%%%%%
% \paragraph{Part Include Files.}
%
% The include files are called |cdocspt3.tex| and |cdocspt4.tex|.
%
%\iffalse
%<*samplepart3|samplepart4>
%\fi

% Optional override for |\version| flag:
%    \begin{macrocode}
%%\providecommand{\version}{final}
%    \end{macrocode}

% Include the main document:
%    \begin{macrocode}
\input{childdoc.def}
\childdocby{cdocsamp}
%    \end{macrocode}

%\iffalse
%</samplepart3|samplepart4>
%\fi
%
%\iffalse
%<*samplepart3>
%\fi
% Some text for part 3:
%    \begin{macrocode}
some text in part three
%    \end{macrocode}

%\iffalse
%</samplepart3>
%\fi
% Some text for part 4:
%\iffalse
%<*samplepart4>
%\fi
%    \begin{macrocode}
more text in part four
%    \end{macrocode}

%\iffalse
%</samplepart4>
%\fi
%
% %%%%%%%%%%%%%%%%%%%%%%%%%%%%%%%%%%%%%%
% \paragraph{Forwarding for a Complete Draft.}
%
% The following forwarding file |cdocsdrf.tex|
% compiles the main document in draft mode:
%\iffalse
%<*sampledraft>
%\fi
%    \begin{macrocode}
\def\version{draft}
\input{childdoc.def}
\childdocforward{cdocsamp}
%    \end{macrocode}

%\iffalse
%</sampledraft>
%\fi
%
% %%%%%%%%%%%%%%%%%%%%%%%%%%%%%%%%%%%%%%
% \paragraph{Forwarding for Final Version of the Chapters.}
%
% The following forwarding files |cdocsfn1.tex| and |cdocsfn2.tex|
% (with identical content)
% compile the final versions of the child documents
% |cdocsch1.tex| and |cdocsch2.tex|, respectively:
%\iffalse
%<*samplefinal>
%\fi
%    \begin{macrocode}
\def\version{final}
\input{childdoc.def}
\childdocforwardprefix[cdocsamp]{cdocsfn}{cdocsch}
%    \end{macrocode}

%\iffalse
%</samplefinal>
%\fi
%
% %%%%%%%%%%%%%%%%%%%%%%%%%%%%%%%%%%%%%%
% \paragraph{Command Line Processing.}
%
% The following three command lines generate the output files
% |cdocscld|, |cdocscl1| and |cdocscl2|
% which should be identical to
% |cdocsdrf|, |cdocsch1| and |cdocsfn2|, respectively:
% \begin{center}
% \begin{tabular}{l}
% |latex -jobname cdocscld \|\\
% |  "\def\version{draft}\input{childdoc.def}\childdocforward{cdocsamp}"|\\
% |latex -jobname cdocscl1 \|\\
% |  "\input{childdoc.def}\childdocforward[cdocsamp]{cdocsch1}"|\\
% |latex -jobname cdocscl2 \|\\
% |  "\def\version{final}\input{childdoc.def}\childdocforward{cdocsch2}"|
% \end{tabular}
% \end{center}
% Note that the trailing backslash on each first line
% merely continues the input to the second line
% (for convenient cut ant paste).
% Furthermore, the command |latex| can be replaced by any
% of its alternative versions such as |pdflatex|.
%
% %%%%%%%%%%%%%%%%%%%%%%%%%%%%%%%%%%%%%%%%%%%%%%%%%%%%%%%%%%%%%%%%%%%%%%%%%%%%%%
% %%%%%%%%%%%%%%%%%%%%%%%%%%%%%%%%%%%%%%%%%%%%%%%%%%%%%%%%%%%%%%%%%%%%%%%%%%%%%%
% \section{Implementation}
%\iffalse
%<*package>
%\fi
%
% This section describes the definitions file |childdoc.def|.

% The definitions cannot be loaded using |\usepackage| or |\RequirePackage|
% which has a mechanism to prevent loading a style file more than once.
% When loading the definitions by means of |\input|
% multiple instances have to be prevented manually:
%\iffalse
%This code needs to be before the `\ProvidesFile' directive
%which is defined at the beginning of this file.
%Therefore it is also placed there and commented out here.
%</package>
%<*discard>
%\fi
%    \begin{macrocode}
\ifdefined\childdocmain\endinput\fi
%    \end{macrocode}
%\iffalse
%</discard>
%<*package>
%\fi
%
% \macro{\ifchilddoc}
% \macro{\ifchilddocmanual}
% The conditional |\ifchilddoc| tells whether a
% child (true) or main (false) document is being compiled.
% The conditional |\ifchilddocmanual| tells whether
% the |\includeonly| mechanism is used (false) or
% the selection of child files must be performed manually (true).
% The definitions initialise to false:
%    \begin{macrocode}
\newif\ifchilddoc
\newif\ifchilddocmanual
%    \end{macrocode}

% \macro{\childdocname}
% \macro{\childdocjob}
% The macro |\childdocname| stores the name of the main document
% to be compiled. The macro |\childdocjob| stores the name of
% the document on which the \LaTeX{} compiler was originally invoked.
% The content of |\jobname| cannot be compared
% to filenames specified in the source due to different catcodes.
% The following code rescans |\jobname|, stores the result
% in |\childdocname| and saves a copy in |\childdocjob|:
%    \begin{macrocode}
\edef\childdocname{\scantokens\expandafter{\jobname\noexpand}}
\let\childdocjob\childdocname
%    \end{macrocode}

% \macro{\childdocdisable}
% The macro |\childdocdisable| prevents the main file
% from being processed more than once.
% At this stage, the main document command |\childdocmain|
% is assumed to be called once again where it should do nothing.
% Any subsequent call to it should prevent
% a secondary processing of the main document
% It overwrites the forwarding commands
% |\childdocof| and |\childdocforward|
% with empty macros to prevent further inclusions of the main document:
%    \begin{macrocode}
\newcommand{\childdocdisable}
{
  \renewcommand{\childdocmain}[1]{\renewcommand{\childdocmain}[1]{\endinput}}
  \renewcommand{\childdocof}[1]{}
  \renewcommand{\childdocby}[2][]{}
  \renewcommand{\childdocforward}[2][]{}
  \renewcommand{\childdocdisable}{}
}
%    \end{macrocode}

% \macro{\childdocmain}
% The macro |\childdocmain| is to be called at the top of the main file
% with nothing or the main filename (without extension) as argument.
% First, it breaks loops.
% If the argument is not empty and does not match |\childdocname|
% (which is set by the first inclusion of |childdoc.def|),
% |\ifchilddoc| is set to true, |\includeonly| is applied to the child file
% and |\jobname| is set to the main file
% (for proper handling of |.aux| files):
%    \begin{macrocode}
\newcommand{\childdocmain}[1]
{
  \childdocdisable\childdocmain{}
  \if?#1?\else
    \begingroup
      \def\childdoctmp{#1}
      \ifx\childdoctmp\childdocname
        \def\childdoctmp{}
      \else
        \def\childdoctmp
        {
          \childdoctrue
          \includeonly{\childdocname}
          \def\childdocjob{#1}
          \def\jobname{#1}
        }
      \fi
      \expandafter
    \endgroup
    \childdoctmp
  \fi
}
%    \end{macrocode}

% \macro{\childdocof}
% The command |\childdocof| redirects
% compilation to the main file |#1|.
%    \begin{macrocode}
\newcommand{\childdocof}[1]
{
  \childdocdisable
  \childdoctrue
  \includeonly{\childdocname}
  \def\jobname{#1}
  \def\childdocjob{#1}
  \input{#1}
}
%    \end{macrocode}

% \macro{\childdocby}
% The command |\childdocby| ....
%    \begin{macrocode}
\newcommand{\childdocby}[2][]
{
  \childdocdisable
  \childdoctrue
  \childdocmanualtrue
  \if?#1?\else
    \def\jobname{#2}
  \fi
  \def\childdocjob{#2}
  \input{#2}
  \endinput
}
%    \end{macrocode}

% \macro{\childdocforward}
% The command |\childdocforward| redirects
% compilation to the main file or
% (if the optional argument is given) a child file.
% Parameters are set as if the main file
% or a child file starting with |\childdocof| was compiled.
% Then compilation is handed over to the main file:
%    \begin{macrocode}
\newcommand{\childdocforward}[2][]
{
  \begingroup
    \if?#1?
      \def\childdoctmp
      {
        \def\childdocname{#2}
        \def\childdocjob{#2}
        \def\jobname{#2}
        \input{#2}
        \endinput
      }
    \else
      \def\childdoctmp
      {
        \childdocdisable
        \def\childdocname{#2}
        \childdoctrue
        \includeonly{#2}
        \def\childdocjob{#1}
        \def\jobname{#1}
        \input{#1}
        \endinput
      }
    \fi
    \expandafter
  \endgroup
  \childdoctmp
}
%    \end{macrocode}

% \macro{\childdocforwardprefix}
% The command |\childdocforwardprefix| redirects
% compilation to the main or a child file by means of a pattern.
% The prefix |#1| in the current filename is replaced by |#2|
% and the suffix of the current filename is kept
% (it is assumed that the filename does not contain the substring `|~~~|'
% which is used as a delimiter).
% Compilation is handed over to the new file by |\childdocforward|:
%    \begin{macrocode}
\newcommand{\childdocforwardprefix}[3][]
{
  \begingroup
    \def\childdocextract #2##1~~~{\def\childdoctmp{\childdocforward[#1]{#3##1}}}
    \expandafter\childdocextract\childdocname~~~
    \expandafter
  \endgroup
  \childdoctmp
}
%    \end{macrocode}

% \macro{\childdoc}
% The deprecated macro |\childdoc| is a legacy version of |\childdocmain|:
%    \begin{macrocode}
\newcommand{\childdoc}{\childdocmain}
%    \end{macrocode}

% \macro{\childdocredirect}
% The deprecated macro |\childdocredirect| is a legacy version
% of |\childdocforward| and |\childdocforwardprefix|:
%    \begin{macrocode}
\newcommand{\childdocredirect}[2][]
{
  \begingroup
    \if?#1?
      \def\childdoctmp{\childdocforward{#2}}
    \else
      \def\childdoctmp{\childdocforwardprefix{#1}{#2}}
    \fi
    \expandafter
  \endgroup
  \childdoctmp
}
%    \end{macrocode}

%\iffalse
%</package>
%\fi
%
\endinput
|\\
|\childdocforwardprefix[|\textit{main}|]{|\textit{prefix}|}{|\textit{dest}|}|
\end{tabular}
\end{center}
%
the destination file is determined by a pattern
depending on the current file:
To make this work, the current file must be called
`{\textit{prefix}\hspace{0.2em}\textit{suffix}}'
with \textit{prefix} matching precisely the argument.
Processing is then passed on to the file
`{\textit{dest}\hspace{0.2em}\textit{suffix}}'.
Surely, the same effect is achieved by
directly specifying the
argument `{\textit{dest}\hspace{0.2em}\textit{suffix}}'
in the first form.
However, that requires to set up a different file
for each child. With the alternative form of the command
all these files can have exactly the same content
which simplifies setting them up and maintaining them.

For example, the following file |draft.tex|
with a compilation flag |\version| as described in \secref{sec:flags}
compiles the main document as a draft:
%
\begin{center}
\begin{tabular}{l}
|\def\version{draft}|\\
|% \iffalse
%
% childdoc.dtx Copyright (C) 2017-2018 Niklas Beisert
%
% This work may be distributed and/or modified under the
% conditions of the LaTeX Project Public License, either version 1.3
% of this license or (at your option) any later version.
% The latest version of this license is in
%   http://www.latex-project.org/lppl.txt
% and version 1.3 or later is part of all distributions of LaTeX
% version 2005/12/01 or later.
%
% This work has the LPPL maintenance status `maintained'.
%
% The Current Maintainer of this work is Niklas Beisert.
%
% This work consists of the files childdoc.dtx and childdoc.ins
% and the derived files childdoc.def and cdocsamp.tex with
% cdocsch1.tex, cdocsch2.tex, cdocsdrf.tex, cdocsfn1.tex, cdocsfn2.tex.
%
%<package>\ifdefined\childdocmain\endinput\fi
%<package>\ProvidesFile{childdoc.def}[2018/12/30 v2.0 child document driver]
%<samplemain>\ProvidesFile{cdocsamp.tex}[2018/12/30 v2.0 sample for childdoc]
%<*driver>
%\ProvidesFile{childdoc.drv}[2018/12/30 v2.0 childdoc reference manual file]
\PassOptionsToClass{10pt,a4paper}{article}
\documentclass{ltxdoc}

\usepackage[margin=35mm]{geometry}
\usepackage{hyperref}
\usepackage{hyperxmp}
\usepackage[usenames]{color}

\hypersetup{colorlinks=true}
\hypersetup{pdfstartview=FitH}
\hypersetup{pdfpagemode=UseNone}
\hypersetup{pdfsource={}}
\hypersetup{pdflang={en-UK}}
\hypersetup{pdfcopyright={Copyright 2017-2018 Niklas Beisert.
  This work may be distributed and/or modified under the
  conditions of the LaTeX Project Public License, either version 1.3
  of this license or (at your option) any later version.}}
\hypersetup{pdflicenseurl={http://www.latex-project.org/lppl.txt}}
\hypersetup{pdfcontactaddress={ETH Zurich, ITP, HIT K,
  Wolfgang-Pauli-Strasse 27}}
\hypersetup{pdfcontactpostcode={8093}}
\hypersetup{pdfcontactcity={Zurich}}
\hypersetup{pdfcontactcountry={Switzerland}}
\hypersetup{pdfcontactemail={nbeisert@itp.phys.ethz.ch}}
\hypersetup{pdfcontacturl={http://people.phys.ethz.ch/\xmptilde nbeisert/}}

\newcommand{\secref}[1]{\hyperref[#1]{section \ref*{#1}}}

\parskip1ex
\parindent0pt
\let\olditemize\itemize
\def\itemize{\olditemize\parskip0pt}

\begin{document}

\title{The \textsf{childdoc} Package}
\hypersetup{pdftitle={The childdoc Package}}
\author{Niklas Beisert\\[2ex]
  Institut f\"ur Theoretische Physik\\
  Eidgen\"ossische Technische Hochschule Z\"urich\\
  Wolfgang-Pauli-Strasse 27, 8093 Z\"urich, Switzerland\\[1ex]
  \href{mailto:nbeisert@itp.phys.ethz.ch}
  {\texttt{nbeisert@itp.phys.ethz.ch}}}
\hypersetup{pdfauthor={Niklas Beisert}}
\hypersetup{pdfsubject={Manual for the LaTeX2e Package childdoc}}
\date{30 December 2018, \textsf{v2.0}}
\maketitle

\begin{abstract}\noindent
\textsf{childdoc} is a \LaTeXe{} package
that enables the direct compilation
of document sections included by |\include|
to individual files.
\end{abstract}

\begingroup
\parskip0ex
\tableofcontents
\endgroup

%%%%%%%%%%%%%%%%%%%%%%%%%%%%%%%%%%%%%%%%%%%%%%%%%%%%%%%%%%%%%%%%%%%%%%%%%%%%%%%%
%%%%%%%%%%%%%%%%%%%%%%%%%%%%%%%%%%%%%%%%%%%%%%%%%%%%%%%%%%%%%%%%%%%%%%%%%%%%%%%%
\section{Introduction}

\LaTeX{} provides a mechanism to structure a large document (such as a book)
into a main file and several child files (containing the chapters)
using the |\include| command.
This mechanism is beneficial for documents
which span hundreds of pages in order to
make the source file(s) more manageable.
Moreover, compilation can be restricted to
selected child files by means of the |\includeonly| command.
The latter feature can be used to reduce the compilation time while editing
(this was significantly more useful in the earlier days of \LaTeX{})
or to generate a smaller document which is easier to navigate.
Another application of |\includeonly| is to generate
documents consisting of selected parts of the complete document.

However, there are a few drawbacks of the plain |\include| mechanism:
\begin{itemize}
\item
The child files cannot be compiled on their own,
they can only be compiled via the main file.
A naive editing environment
(such as a text editor with an option
to have the current file processed by \LaTeX)
may require one to switch to the main file before compiling;
attempting to compile the child file produces errors.
\item
The main file must be modified (each time)
to adjust the |\includeonly| command
to the present needs. This easily leaves the main file in a messy state.
\item
The generated document will always carry the filename
of the main document. This is inconvenient if
several child files are to be compiled and
to be kept for distribution.
\end{itemize}

The present package provides a simple interface
to make child files individually compilable by \LaTeX{}.
Compiling a child file then has the same effect as compiling
the main file with an |\includeonly| command
to select the appropriate child.
Moreover the generated document will carry the name of the child
rather than the main file.
This resolves all three above issues.

This feature is meant to make the editing of books,
thesis documents and lecture notes somewhat more convenient.
However, the package can also be used efficiently for
composing a series of documents (such as exercise sheets)
which are typically distributed individually.
It then assists the author in generating the individual documents
(potentially in different versions)
as well as a document containing the collected series.
Another application is in developing style files
or other kinds of included material
where compilation of the style file could redirect
to a sample or test file.

%%%%%%%%%%%%%%%%%%%%%%%%%%%%%%%%%%%%%%%%%%%%%%%%%%%%%%%%%%%%%%%%%%%%%%%%%%%%%%%%
%%%%%%%%%%%%%%%%%%%%%%%%%%%%%%%%%%%%%%%%%%%%%%%%%%%%%%%%%%%%%%%%%%%%%%%%%%%%%%%%
\section{Usage}

First of all, the package \textsf{childdoc} is \emph{not} a standard
\LaTeXe{} |.sty| style file! Therefore it needs to be invoked in
a non-standard way.

%%%%%%%%%%%%%%%%%%%%%%%%%%%%%%%%%%%%%%%%%%%%%%%%%%%%%%%%%%%%%%%%%%%%%%%%%%%%%%%%
\subsection{Included Files}
\label{sec:include}

%%%%%%%%%%%%%%%%%%%%%%%%%%%%%%%%%%%%%%%%
\DescribeMacro{\childdocmain}
To use the package, add the commands
\begin{center}
\begin{tabular}{l}
|\input{childdoc.def}|\\
|\childdocmain{}|\\
\end{tabular}
\end{center}
at the very top of the main \LaTeX{} file,
in particular \emph{before} the |\documentclass| statement!
The argument of |\childdocmain| should be left empty
(but it must be present).

%%%%%%%%%%%%%%%%%%%%%%%%%%%%%%%%%%%%%%%%
\DescribeMacro{\childdocof}
Furthermore, add the commands
\begin{center}
\begin{tabular}{l}
|\input{childdoc.def}|\\
|\childdocof{|\textit{main}|}|\\
\end{tabular}
\end{center}
at the top of every child file \textit{child}
which is included by |\include{|\textit{child}|}|
from within the main file
(or at least for those files to be compiled individually).
The argument \textit{main} must be the filename of the main file.

There are a couple of
considerations in setting up the main and child documents:

%%%%%%%%%%%%%%%%%%%%%%%%%%%%%%%%%%%%%%%%
\paragraph{Restrictions.}

Please note the following restrictions:
\begin{itemize}
\item
|\childdocmain| must be called with one argument \textit{main}
to ensure compatibility with earlier version of the package.
It must either be empty (|\childdocmain{}|)
or precisely match the filename of the main file in which it is specified.
See \secref{sec:detection} for further information.
\item
The filename \textit{main} must be specified without the |.tex| extension.
\item
The filename \textit{main} is case sensitive
(even in case-insensitive file systems)
due to internal string comparison.
\item
The argument \textit{main} should be fully expanded, it cannot be a macro.
\item
Subdirectories and special characters should be avoided in filenames.
\item
The command |\childdocmain{|\textit{main}|}| must be followed by a whitespace.
It should not be followed immediately by another command
or by a comment mark `|%|'.
This is because the \TeX{} parser reads the token immediately following
the argument of |\childdocmain| and puts it
at the beginning of every child section;
however, a white\-space is ignored.
\end{itemize}

%%%%%%%%%%%%%%%%%%%%%%%%%%%%%%%%%%%%%%%%
\paragraph{Content of Main File.}

It is advisable to place all content in the child files included by |\include|.
Any output contained in the main file will appear in all child documents
unless suppressed manually;
it cannot be suppressed automatically by the |\includeonly| directive
and thus should normally be avoided.
A method to include some content in the main file
by means of conditional processing is described in \secref{sec:conditional}.

%%%%%%%%%%%%%%%%%%%%%%%%%%%%%%%%%%%%%%%%
\paragraph{Page Numbering.}

When only a part of the document is compiled,
the appropriate numbering of pages
(as well as other status parameters)
is determined from the |.aux| files.
The latter contain information from previous passes.
However this information needs to propagate through
all intermediate child documents.
Therefore the page numbering in child documents may well
be inconsistent until the complete document is compiled at least once.

A useful (if unconventional) way to always ensure a consistent
page numbering is to restart the numbering in each child document
and denote the pages by `\textit{child}|.|\textit{page}'
where \textit{child} represents the chapter/section number of the child file.
This can be achieved by the command
|\numberwithin{page}{|\textit{child}|}|
of the \textsf{amsmath} package
where \textit{child} can be |chapter| or |section|
depending on the chosen structuring.
Alternatively, one can modify the macro |\thepage| appropriately
and reset the counter |page| at the start of each child file.

%%%%%%%%%%%%%%%%%%%%%%%%%%%%%%%%%%%%%%%%%%%%%%%%%%%%%%%%%%%%%%%%%%%%%%%%%%%%%%%%
\subsection{Conditional Processing}
\label{sec:conditional}

The package provides a mechanism to compile different versions
of a document. To customise the versions further some conditional processing
can come in handy to distinguish which version is being compiled.
The package provides two macros to describe the compilation context:

%%%%%%%%%%%%%%%%%%%%%%%%%%%%%%%%%%%%%%%%
\DescribeMacro{\ifchilddoc}
The conditional |\ifchilddoc| distinguishes between the compilation of
child documents and the main document:
%
\begin{center}
|\ifchilddoc |\textit{child-code}| |[|\||else |\textit{main-code}]| \||fi|
\end{center}

%%%%%%%%%%%%%%%%%%%%%%%%%%%%%%%%%%%%%%%%
\DescribeMacro{\childdocname}
\DescribeMacro{\childdocjob}
The macro |\childdocname| contains the filename (without extension)
of the main or child file being processed.
Note that |\childdocjob| will always contain the name of the main file.

%%%%%%%%%%%%%%%%%%%%%%%%%%%%%%%%%%%%%%%%
\paragraph{Title Page.}

Conditional processing can be used to include a title or banner page
in the main document when proper precautions are taken.
Importantly, the code in the main file should ensure that the page counter
(as well as other status parameters which are stored in the |.aux| files)
takes the same value after the conditional processing.
Otherwise the page numbers may take divergent values
depending on which part is compiled.

For example, a title page could be declared by:
%
\begin{center}
\begin{tabular}{l}
|\ifchilddoc\||else|\\
|\addtocounter{page}{-1}|\\
\textit{code for title page}\\
|\newpage|\\
|\||fi|
\end{tabular}
\end{center}
%
A banner page for the child documents can be generated by:
%
\begin{center}
\begin{tabular}{l}
|\ifchilddoc|\\
|\addtocounter{page}{-1}|\\
\textit{code for banner page}\\
|\newpage|\\
|\||fi|
\end{tabular}
\end{center}
%
Here one could write a message such as:
\begin{center}
|This is the part \childdocname{} of \childdocjob{}.|
\end{center}

%%%%%%%%%%%%%%%%%%%%%%%%%%%%%%%%%%%%%%%%%%%%%%%%%%%%%%%%%%%%%%%%%%%%%%%%%%%%%%%%
\subsection{Flags}
\label{sec:flags}

The package makes it easy to generate different versions
of the main or child documents.
To this end compilation flags can be defined
and assigned different default values.
They will be particularly useful in conjunction
with the forwarding mechanism described in \secref{sec:forward}.

For example, it may be useful to have a flag |\version|
which can be set to |draft| or |final|.
The document source will contain some conditional code
depending on the value of |\version|.
Suppose further, the flag should default to |final| for the main file
and to |draft| for child files
which is a natural assignment for editing the document.
This is achieved by placing the following code
in the preamble of the main document
(below the |\childdocmain| directive):
%
\begin{center}
\begin{tabular}{l}
|\ifchilddoc|\\
|\providecommand{\version}{draft}|\\
|\||else|\\
|\providecommand{\version}{final}|\\
|\||fi|
\end{tabular}
\end{center}
%
The definition by |\providecommand| makes sure
that previous definitions are not overwritten.
Further statements |\providecommand{\version}{...}|
can thus be added before the above code to override it.

For the main file, one might add a line
(between |\childdocmain| and the above block)
%
\begin{center}
|%\ifchilddoc\||else\providecommand{\version}{draft}\||fi|
\end{center}
%
which can be uncommented to produce a draft version.
Likewise one can add a line to the very top of a child file
(above the |\childdocof{|\textit{main}|}| directive)
%
\begin{center}
|%\providecommand{\version}{final}|
\end{center}
%
which can be uncommented to produce the final version of this child document.

%%%%%%%%%%%%%%%%%%%%%%%%%%%%%%%%%%%%%%%%%%%%%%%%%%%%%%%%%%%%%%%%%%%%%%%%%%%%%%%%
\subsection{Forwarding}
\label{sec:forward}

Different versions of the main or child documents
using compilation flags as described in \secref{sec:flags}
can be (permanently) stored in different files
for convenient compilation, viewing and distribution.
To this end, the package defines a command
to pass on compilation to a different file:

%%%%%%%%%%%%%%%%%%%%%%%%%%%%%%%%%%%%%%%%
\DescribeMacro{\childdocforward}
The command |\childdocforward| redirects processing to
another source file:
%
\begin{center}
\begin{tabular}{l}
|\input{childdoc.def}|\\
|\childdocforward[|\textit{main}|]{|\textit{dest}|}|\\
\end{tabular}
\end{center}
%
The argument \textit{dest} is the destination file
(without extension).
It should be the main file or one of the child files.
Note that further \textsf{childdoc} directives
such as |\childdocof| and |\childdocforward|
in the indicated file will be processed in this form.
The optional argument \textit{main}
passes on directly to the main file \textit{main}
while pretending to compile the child \textit{dest}.
This form behaves as if \textit{dest}
issues |\childdocof{|\textit{main}|}| right away,
and no further \textsf{childdoc} directives will be processed.

%%%%%%%%%%%%%%%%%%%%%%%%%%%%%%%%%%%%%%%%
\DescribeMacro{\...prefix}
In the alternative form |\childdocforwardprefix|,
%
\begin{center}
\begin{tabular}{l}
|\input{childdoc.def}|\\
|\childdocforwardprefix[|\textit{main}|]{|\textit{prefix}|}{|\textit{dest}|}|
\end{tabular}
\end{center}
%
the destination file is determined by a pattern
depending on the current file:
To make this work, the current file must be called
`{\textit{prefix}\hspace{0.2em}\textit{suffix}}'
with \textit{prefix} matching precisely the argument.
Processing is then passed on to the file
`{\textit{dest}\hspace{0.2em}\textit{suffix}}'.
Surely, the same effect is achieved by
directly specifying the
argument `{\textit{dest}\hspace{0.2em}\textit{suffix}}'
in the first form.
However, that requires to set up a different file
for each child. With the alternative form of the command
all these files can have exactly the same content
which simplifies setting them up and maintaining them.

For example, the following file |draft.tex|
with a compilation flag |\version| as described in \secref{sec:flags}
compiles the main document as a draft:
%
\begin{center}
\begin{tabular}{l}
|\def\version{draft}|\\
|\input{childdoc.def}|\\
|\childdocforward{|\textit{main}|}|
\end{tabular}
\end{center}
%
Likewise, the following files |final|\textit{nn}|.tex|
compile the final version of the child document
|child|\textit{nn}|.tex|:
%
\begin{center}
\begin{tabular}{l}
|\def\version{final}|\\
|\input{childdoc.def}|\\
|\childdocforwardprefix{final}{child}|
\end{tabular}
\end{center}
%

Note that when several versions of a main file and/or of each child file
are to be generated, it may be convenient to set up a |Makefile| or
shell script to automatise the process.

%%%%%%%%%%%%%%%%%%%%%%%%%%%%%%%%%%%%%%%%%%%%%%%%%%%%%%%%%%%%%%%%%%%%%%%%%%%%%%%%
\subsection{Command Line Processing}
\label{sec:commandline}

The effect of redirection files can also be achieved by invoking
the \LaTeX{} compiler with a more elaborate command line.
Most conveniently this should be done as part
of a shell script or a |Makefile|.

When using \textsf{childdoc} in the main file, the following
command lines effectively perform a redirection
(note that depending on the shell being used,
backslashes may have to be doubled: `|\|' $\to$ `|\\|'):
%
\begin{center}
|... -jobname "|\textit{target}|" |\\|"|[\textit{flags}]%
|\input{childdoc.def}\childdocforward[|\textit{main}|]{|\textit{dest}|}"|
\end{center}
%
Here \textit{target} is the name of the output file,
\textit{main} is the name of the main file
and \textit{dest} is the name of the main or child file to be processed
(all filenames without extensions).
The optional argument \textit{main} can be omitted
if \textit{main} matches \textit{dest}.
Optionally, compilation \textit{flags} can be defined via |\def| commands.
This command line makes the \TeX{} engine believe
it is compiling the file \textit{target}
whose content is specified as the latter parameter.
The provided code then forwards the processing to
\textit{main} or \textit{dest} as described in \secref{sec:forward}.

%%%%%%%%%%%%%%%%%%%%%%%%%%%%%%%%%%%%%%%%%%%%%%%%%%%%%%%%%%%%%%%%%%%%%%%%%%%%%%%%
\subsection{Include by Input}
\label{sec:input}

Including child documents by |\include| has some restrictions by design.
Most notably, the content of a child document always occupies
its own set of pages; pages cannot be shared between child documents.
Usually, this behaviour makes perfect sense
because each child document contain an essential part of the document.
However, in some situations it may be desirable to compose
a document from a collection of parts
without having mandatory page breaks between then.
For this case, the package
provides a mechanism to include parts
by |\input| which can also be processed individually.
However, by construction this mechanism
requires manual handling of the content to be output.

%%%%%%%%%%%%%%%%%%%%%%%%%%%%%%%%%%%%%%%%
\DescribeMacro{\ifchilddocmanual}
The main file should be prepared as usual, see \secref{sec:include}.
However, the document body must make a distinction
between processing of an individual part and of the main document, e.g.:
%
\begin{center}
\begin{tabular}{l}
|\ifchilddocmanual|\\
|\input{\childdocname}|\\
|\||else|\\
\textit{document body with }|\input{|\textit{part}|}|\\
|\||fi|
\end{tabular}
\end{center}
%
The conditional |\ifchilddocmanual| is true whenever
a part to be included by |\input| is being compiled,
and the name of the part is stored in |\childdocname|.

%%%%%%%%%%%%%%%%%%%%%%%%%%%%%%%%%%%%%%%%
\DescribeMacro{\childdocby}
Each part to be included by |\input| should start with:
%
\begin{center}
\begin{tabular}{l}
|\input{childdoc.def}|\\
|\childdocby{|\textit{main}|}|\\
\end{tabular}
\end{center}
%
The directive |\childdocby| is similar to |\childdocof|
described in \secref{sec:include},
but the subsequent selection of content must be done manually.
To that end, both |\ifchilddoc| and |\ifchilddocmanual|
will be true upon processing of a part,
and the name of the part is stored in |\childdocname|.
Note that |\jobname| will be set to the filename of the current part
so that each part receives an individual |.aux| file
that does not interfere with the |.aux| file(s) of the main document.
This behaviour can be altered by the alternative form
|\childdocby[*]{|\textit{main}|}| (with a non-empty optional argument)
which uses the |.aux| file of the main document
by setting |\jobname| to \textit{main}.

%%%%%%%%%%%%%%%%%%%%%%%%%%%%%%%%%%%%%%%%%%%%%%%%%%%%%%%%%%%%%%%%%%%%%%%%%%%%%%%%
\subsection{Driver Development}
\label{sec:driver}

The \textsf{childdoc} mechanism can also be use for the development
of definition files such as \LaTeX{} styles or classes.
This case differs from the above setup with multiple parts
included by |\include| in that no |\includeonly| should be invoked.
This can be achieved by starting the include file
(before |\ProvidesPackage|) with:
%
\begin{center}
\begin{tabular}{l}
|\input{childdoc.def}|\\
|\childdocforward{|\textit{main}|}|\\
\end{tabular}
\end{center}
%
or alternatively with:
%
\begin{center}
\begin{tabular}{l}
|\input{childdoc.def}|\\
|\childdocby{|\textit{main}|}|\\
\end{tabular}
\end{center}
%
Both forms have slightly different effects as described above.
The main file is prepared as usual, see \secref{sec:include}.

%%%%%%%%%%%%%%%%%%%%%%%%%%%%%%%%%%%%%%%%%%%%%%%%%%%%%%%%%%%%%%%%%%%%%%%%%%%%%%%%
\subsection{Legacy Detection}
\label{sec:detection}

The directive |\childdocmain| in the main file can detect
whether the complete document or merely a child is to be compiled
even without using the directive |\childdocof|.
This method is deprecated because it is less robust
and there is no compelling reason to use it;
it is merely provided for backward compatibility
and it may be removed in future versions.

If the detection mechanism is to be used,
it is mandatory to correctly specify
the filename of the main file as the argument of |\childdocmain|:
%
\begin{center}
\begin{tabular}{l}
|\input{childdoc.def}|\\
|\childdocmain{|\textit{main}|}|\\
\end{tabular}
\end{center}
%
If |\jobname| does not match the argument \textit{main} of |\childdocmain|,
it is assumed that |\jobname| points to the child file to be compiled.
When using |\childdocmain| with the main file specified as argument,
it suffices to start a child file
with just |\input{|\textit{main}|}|
without loading of the package and using |\childdocof|.
If instead all processing is done
with the appropriate \textsf{childdoc} directives,
the argument of \textit{main} of |\childdocmain| can be empty.

An alternative version of the command line processing described
in \secref{sec:commandline} using the detection mechanism reads:
%
\begin{center}
|... -jobname "|\textit{target}|" "|[\textit{flags}]%
[|\def\jobname{|\textit{dest}|}|]|\input{|\textit{main}|}"|
\end{center}

%%%%%%%%%%%%%%%%%%%%%%%%%%%%%%%%%%%%%%%%%%%%%%%%%%%%%%%%%%%%%%%%%%%%%%%%%%%%%%%%
\subsection{Manual Code}
\label{sec:manual}

In case one cannot be certain whether the definitions file |childdoc.def|
is installed on the target \TeX{} distribution
and one prefers not to ship it,
it is conceivable to paste a few relevant commands into the sources.

To that end, drop all statements |\input{childdoc.def}|
and perform the replacements as outlined below.
Instead of |\childdocmain{|\textit{main}|}| add the following code
to the top of the main file:
%
\begin{center}
\begin{tabular}{l}
|\||ifdefined\childdocname\endinput\||fi\newif\ifchilddoc|\\
|\edef\childdocname{\scantokens\expandafter{\jobname\noexpand}}|\\
|\def\childdocmain{|\textit{main}|}\||ifx\childdocmain\childdocname\||else|\\
|\childdoctrue\includeonly{\childdocname}\let\jobname\childdocmain\||fi|\\
\end{tabular}
\end{center}
%
Instead of |\childdocof{|\textit{main}|}| just include the main file
at the top of each child file:
%
\begin{center}
|\input{|\textit{main}|}|
\end{center}
%
A simple redirection |\childdocforward{|\textit{dest}|}| is achieved by:
%
\begin{center}
|\def\jobname{|\textit{dest}|}\input{\jobname}|
\end{center}
%
The redirection with prefix
|\childdocforwardprefix[|\textit{prefix}|]{|\textit{dest}|}|
is accomplished by:
%
\begin{center}
\begin{tabular}{l}
|{\edef\jobname{\scantokens\expandafter{\jobname\noexpand}}|\\
|\def\redirectjob |\textit{prefix}|#1~~~{\gdef\jobname{|\textit{dest}|#1}}|\\
|\expandafter\redirectjob\jobname~~~}\input{\jobname}|
\end{tabular}
\end{center}

In an alternative approach,
child documents can be compiled by a specific command line
without additional code or specific definitions:
%
\begin{center}
|... -jobname "|\textit{target}|" "|[\textit{flags}]%
|\includeonly{|\textit{dest}|}\input{|\textit{main}|}"|
\end{center}
%

%%%%%%%%%%%%%%%%%%%%%%%%%%%%%%%%%%%%%%%%%%%%%%%%%%%%%%%%%%%%%%%%%%%%%%%%%%%%%%%%
%%%%%%%%%%%%%%%%%%%%%%%%%%%%%%%%%%%%%%%%%%%%%%%%%%%%%%%%%%%%%%%%%%%%%%%%%%%%%%%%
\section{Information}

%%%%%%%%%%%%%%%%%%%%%%%%%%%%%%%%%%%%%%%%%%%%%%%%%%%%%%%%%%%%%%%%%%%%%%%%%%%%%%%%
\subsection{Copyright}

Copyright \copyright{} 2017--2018 Niklas Beisert

This work may be distributed and/or modified under the
conditions of the \LaTeX{} Project Public License, either version 1.3
of this license or (at your option) any later version.
The latest version of this license is in
  \url{http://www.latex-project.org/lppl.txt}
and version 1.3 or later is part of all distributions of \LaTeX{}
version 2005/12/01 or later.

This work has the LPPL maintenance status `maintained'.

The Current Maintainer of this work is Niklas Beisert.

This work consists of the files |README.txt|, |childdoc.ins| and |childdoc.dtx|
as well as the derived files |childdoc.def|, |cdocsamp.tex|
with |cdocsch1.tex|, |cdocsch2.tex|, |cdocspt3.tex|, |cdocspt4.tex|,
|cdocsdrf.tex|, |cdocsfn1.tex|, |cdocsfn2.tex|
as well as |childdoc.pdf|.

%%%%%%%%%%%%%%%%%%%%%%%%%%%%%%%%%%%%%%%%%%%%%%%%%%%%%%%%%%%%%%%%%%%%%%%%%%%%%%%%
\subsection{Files and Installation}

The package consists of the files:
%
\begin{center}
\begin{tabular}{ll}
    |README.txt|   & readme file \\
    |childdoc.ins| & installation file \\
    |childdoc.dtx| & source file \\
    |childdoc.def| & definition file \\
    |cdocsamp.tex| & sample main file \\
    |cdocsch1.tex| & sample include file \\
    |cdocsch2.tex| & sample include file \\
    |cdocspt3.tex| & sample part file \\
    |cdocspt4.tex| & sample part file \\
    |cdocsdrf.tex| & sample redirection file \\
    |cdocsfn1.tex| & sample redirection file \\
    |cdocsfn2.tex| & sample redirection file \\
    |childdoc.pdf| & manual
\end{tabular}
\end{center}
%
The distribution consists of the files
|README.txt|, |childdoc.ins| and |childdoc.dtx|.
%
\begin{itemize}
\item
Run (pdf)\LaTeX{} on |childdoc.dtx|
to compile the manual |childdoc.pdf| (this file).
\item
Run \LaTeX{} on |childdoc.ins| to create the definitions file |childdoc.def|
and the sample |cdocsamp.tex| with include files
|cdocsch1.tex|, |cdocsch2.tex|, |cdocspt3.tex|, |cdocspt4.tex|,
|cdocsdrf.tex|, |cdocsfn1.tex|, |cdocsfn2.tex|.
Then copy the file |childdoc.def| to an appropriate directory of your \LaTeX{}
distribution, e.g.\ \textit{texmf-root}|/tex/latex/childdoc|.
\end{itemize}

%%%%%%%%%%%%%%%%%%%%%%%%%%%%%%%%%%%%%%%%%%%%%%%%%%%%%%%%%%%%%%%%%%%%%%%%%%%%%%%%
\subsection{Related CTAN Packages}

There are several other packages which offer a similar functionality:
%
\begin{itemize}
\item
The packages
\href{http://ctan.org/pkg/docmute}{\textsf{docmute}},
\href{http://ctan.org/pkg/includex}{\textsf{includex}} and
\href{http://ctan.org/pkg/standalone}{\textsf{standalone}}
provide commands to include only the document body of
a child file thus allowing both files to be compiled individually.
\item
The packages \href{http://ctan.org/pkg/subdocs}{\textsf{subdocs}}
and \href{http://ctan.org/pkg/subfiles}{\textsf{subfiles}}
provide structures in which the main and child documents can be
encapsulated and allowing them to be compiled individually.
The inclusion mechanism is different from the conventional |\include|.
\item
The package \href{http://ctan.org/pkg/combine}{\textsf{combine}}
is an elaborate solution to combine several documents into one.
\end{itemize}
%
See also the CTAN topic \href{http://ctan.org/topic/subdocs}{\textsf{subdocs}}
for further related packages.
The present package differs from the above solutions in that
a document structure constructed with the conventional |\include| mechanism
just needs two extra commands at the top of every file
such that all constituent files can be compiled individually.

%%%%%%%%%%%%%%%%%%%%%%%%%%%%%%%%%%%%%%%%%%%%%%%%%%%%%%%%%%%%%%%%%%%%%%%%%%%%%%%%
%\subsection{Feature Suggestions}
%
%The following is a list of features which may be useful for future
%versions of this package:
%%
%\begin{itemize}
%\item
%\ldots
%\end{itemize}

%%%%%%%%%%%%%%%%%%%%%%%%%%%%%%%%%%%%%%%%%%%%%%%%%%%%%%%%%%%%%%%%%%%%%%%%%%%%%%%%
\subsection{Revision History}

%%%%%%%%%%%%%%%%%%%%%%%%%%%%%%%%%%%%%%%%
\paragraph{v2.0:} 2018/12/30

\begin{itemize}
\item
immediate forward processing
\item
added |\childdocby| mechanism
\item
manual restructured
\end{itemize}

%%%%%%%%%%%%%%%%%%%%%%%%%%%%%%%%%%%%%%%%
\paragraph{v1.6:} 2018/01/17

\begin{itemize}
\item
application for development of include files
\item
corrections to manual
\end{itemize}

%%%%%%%%%%%%%%%%%%%%%%%%%%%%%%%%%%%%%%%%
\paragraph{v1.5:} 2017/05/21

\begin{itemize}
\item
more complete structuring introduced
\item
|\childdocof| introduced
\item
|\childdoc| renamed to |\childdocmain|
\item
|\childredirect| renamed to |\childdocforward| and |\childdocforwardprefix|
and functionality expanded
\end{itemize}

%%%%%%%%%%%%%%%%%%%%%%%%%%%%%%%%%%%%%%%%
\paragraph{v1.0:} 2017/04/27

\begin{itemize}
\item
manual and install package
\item
first version published on CTAN
\end{itemize}

%%%%%%%%%%%%%%%%%%%%%%%%%%%%%%%%%%%%%%%%
\paragraph{v0.6:} 2017/04/26

\begin{itemize}
\item
redirection mechanism added
\end{itemize}

%%%%%%%%%%%%%%%%%%%%%%%%%%%%%%%%%%%%%%%%
\paragraph{v0.5:} 2017/04/26

\begin{itemize}
\item
functionality in definition file
\end{itemize}


%%%%%%%%%%%%%%%%%%%%%%%%%%%%%%%%%%%%%%%%%%%%%%%%%%%%%%%%%%%%%%%%%%%%%%%%%%%%%%%%
%%%%%%%%%%%%%%%%%%%%%%%%%%%%%%%%%%%%%%%%%%%%%%%%%%%%%%%%%%%%%%%%%%%%%%%%%%%%%%%%
%%%%%%%%%%%%%%%%%%%%%%%%%%%%%%%%%%%%%%%%%%%%%%%%%%%%%%%%%%%%%%%%%%%%%%%%%%%%%%%%
\appendix

\settowidth\MacroIndent{\rmfamily\scriptsize 000\ }

 \DocInput{childdoc.dtx}

\end{document}
%</driver>
% \fi
%
% %%%%%%%%%%%%%%%%%%%%%%%%%%%%%%%%%%%%%%%%%%%%%%%%%%%%%%%%%%%%%%%%%%%%%%%%%%%%%%
% %%%%%%%%%%%%%%%%%%%%%%%%%%%%%%%%%%%%%%%%%%%%%%%%%%%%%%%%%%%%%%%%%%%%%%%%%%%%%%
% \section{Sample}
%\iffalse
%<*samplemain>
%\fi
%
% The following presents a sample document
% with two chapters, two parts, a title page,
% a compile flag as well as three forwarding files to set the flag.
% It consists of eight |.tex| files:
% \begin{center}
% \begin{tabular}{ll}
% |cdocsamp.tex|&main file\\
% |cdocsch1.tex|&include file for chapter 1\\
% |cdocsch2.tex|&include file for chapter 2\\
% |cdocspt3.tex|&include file for part 3\\
% |cdocspt4.tex|&include file for part 4\\
% |cdocsdrf.tex|&forwarding file for main file in draft mode\\
% |cdocsfi1.tex|&forwarding file for final version of chapter 1\\
% |cdocsfi2.tex|&forwarding file for final version of chapter 2\\
% \end{tabular}
% \end{center}
% Each of the eight files can be compiled directly by the \LaTeX{} compiler.
%
% %%%%%%%%%%%%%%%%%%%%%%%%%%%%%%%%%%%%%%
% \paragraph{Main File.}
%
% The main file is called |cdocsamp.tex|.
%
% Load the \textsf{childdoc} definitions and
% declare the filename for the main document:
%    \begin{macrocode}
\input{childdoc.def}
\childdocmain{}
%    \end{macrocode}

% Optional override for |\version| flag:
%    \begin{macrocode}
%%\ifchilddoc\else\providecommand{\version}{draft}\fi
%    \end{macrocode}

% Define the default values for the |\version| flag
% (|final| for the main file and |draft| for childs):
%    \begin{macrocode}
\ifchilddoc
\providecommand{\version}{draft}
\else
\providecommand{\version}{final}
\fi
%    \end{macrocode}

% Load the standard document class:
%    \begin{macrocode}
\documentclass[12pt]{article}
%    \end{macrocode}

% Start the document body:
%    \begin{macrocode}
\begin{document}
%    \end{macrocode}

% Declare a title page.
% Print title, part of document being processed and version flag:
%    \begin{macrocode}
\addtocounter{page}{-1}
\begin{center}
{\LARGE\bfseries{}childdoc example\par}
\vspace{1cm}
\ifchilddoc
\ifchilddocmanual part\else chapter\fi:
`\childdocname' of `\childdocjob'\par
\else
main document: `\childdocjob'\par
\fi
version: \version\par
\end{center}
\newpage
%    \end{macrocode}

% Manually include selected file,
% otherwise process as usual:
%    \begin{macrocode}
\ifchilddocmanual
\section*{part `\childdocname'}
\input{\childdocname}
\else
%    \end{macrocode}

% Include the two chapters:
%    \begin{macrocode}
\include{cdocsch1}
\include{cdocsch2}
%    \end{macrocode}

% Include the two parts unless only chapters should be displayed:
%    \begin{macrocode}
\ifchilddoc\else
\section{part three}
\input{cdocspt3}
\section{part four}
\input{cdocspt4}
\fi
%    \end{macrocode}

% Process as usual until here:
%    \begin{macrocode}
\fi
%    \end{macrocode}

% End of document body:
%    \begin{macrocode}
\end{document}
%    \end{macrocode}
%\iffalse
%</samplemain>
%\fi
%
% %%%%%%%%%%%%%%%%%%%%%%%%%%%%%%%%%%%%%%
% \paragraph{Chapter Include Files.}
%
% The include files are called |cdocsch1.tex| and |cdocsch2.tex|.
%
%\iffalse
%<*samplechap1|samplechap2>
%\fi

% Optional override for |\version| flag:
%    \begin{macrocode}
%%\providecommand{\version}{final}
%    \end{macrocode}

% Include the main document:
%    \begin{macrocode}
\input{childdoc.def}
\childdocof{cdocsamp}
%    \end{macrocode}

%\iffalse
%</samplechap1|samplechap2>
%\fi
%
%\iffalse
%<*samplechap1>
%\fi
% Some text for chapter 1:
%    \begin{macrocode}
\section{one}
some text in chapter one
%    \end{macrocode}

%\iffalse
%</samplechap1>
%\fi
% Some text for chapter 2:
%\iffalse
%<*samplechap2>
%\fi
%    \begin{macrocode}
\section{two}
more text in chapter two
%    \end{macrocode}

%\iffalse
%</samplechap2>
%\fi
%
% %%%%%%%%%%%%%%%%%%%%%%%%%%%%%%%%%%%%%%
% \paragraph{Part Include Files.}
%
% The include files are called |cdocspt3.tex| and |cdocspt4.tex|.
%
%\iffalse
%<*samplepart3|samplepart4>
%\fi

% Optional override for |\version| flag:
%    \begin{macrocode}
%%\providecommand{\version}{final}
%    \end{macrocode}

% Include the main document:
%    \begin{macrocode}
\input{childdoc.def}
\childdocby{cdocsamp}
%    \end{macrocode}

%\iffalse
%</samplepart3|samplepart4>
%\fi
%
%\iffalse
%<*samplepart3>
%\fi
% Some text for part 3:
%    \begin{macrocode}
some text in part three
%    \end{macrocode}

%\iffalse
%</samplepart3>
%\fi
% Some text for part 4:
%\iffalse
%<*samplepart4>
%\fi
%    \begin{macrocode}
more text in part four
%    \end{macrocode}

%\iffalse
%</samplepart4>
%\fi
%
% %%%%%%%%%%%%%%%%%%%%%%%%%%%%%%%%%%%%%%
% \paragraph{Forwarding for a Complete Draft.}
%
% The following forwarding file |cdocsdrf.tex|
% compiles the main document in draft mode:
%\iffalse
%<*sampledraft>
%\fi
%    \begin{macrocode}
\def\version{draft}
\input{childdoc.def}
\childdocforward{cdocsamp}
%    \end{macrocode}

%\iffalse
%</sampledraft>
%\fi
%
% %%%%%%%%%%%%%%%%%%%%%%%%%%%%%%%%%%%%%%
% \paragraph{Forwarding for Final Version of the Chapters.}
%
% The following forwarding files |cdocsfn1.tex| and |cdocsfn2.tex|
% (with identical content)
% compile the final versions of the child documents
% |cdocsch1.tex| and |cdocsch2.tex|, respectively:
%\iffalse
%<*samplefinal>
%\fi
%    \begin{macrocode}
\def\version{final}
\input{childdoc.def}
\childdocforwardprefix[cdocsamp]{cdocsfn}{cdocsch}
%    \end{macrocode}

%\iffalse
%</samplefinal>
%\fi
%
% %%%%%%%%%%%%%%%%%%%%%%%%%%%%%%%%%%%%%%
% \paragraph{Command Line Processing.}
%
% The following three command lines generate the output files
% |cdocscld|, |cdocscl1| and |cdocscl2|
% which should be identical to
% |cdocsdrf|, |cdocsch1| and |cdocsfn2|, respectively:
% \begin{center}
% \begin{tabular}{l}
% |latex -jobname cdocscld \|\\
% |  "\def\version{draft}\input{childdoc.def}\childdocforward{cdocsamp}"|\\
% |latex -jobname cdocscl1 \|\\
% |  "\input{childdoc.def}\childdocforward[cdocsamp]{cdocsch1}"|\\
% |latex -jobname cdocscl2 \|\\
% |  "\def\version{final}\input{childdoc.def}\childdocforward{cdocsch2}"|
% \end{tabular}
% \end{center}
% Note that the trailing backslash on each first line
% merely continues the input to the second line
% (for convenient cut ant paste).
% Furthermore, the command |latex| can be replaced by any
% of its alternative versions such as |pdflatex|.
%
% %%%%%%%%%%%%%%%%%%%%%%%%%%%%%%%%%%%%%%%%%%%%%%%%%%%%%%%%%%%%%%%%%%%%%%%%%%%%%%
% %%%%%%%%%%%%%%%%%%%%%%%%%%%%%%%%%%%%%%%%%%%%%%%%%%%%%%%%%%%%%%%%%%%%%%%%%%%%%%
% \section{Implementation}
%\iffalse
%<*package>
%\fi
%
% This section describes the definitions file |childdoc.def|.

% The definitions cannot be loaded using |\usepackage| or |\RequirePackage|
% which has a mechanism to prevent loading a style file more than once.
% When loading the definitions by means of |\input|
% multiple instances have to be prevented manually:
%\iffalse
%This code needs to be before the `\ProvidesFile' directive
%which is defined at the beginning of this file.
%Therefore it is also placed there and commented out here.
%</package>
%<*discard>
%\fi
%    \begin{macrocode}
\ifdefined\childdocmain\endinput\fi
%    \end{macrocode}
%\iffalse
%</discard>
%<*package>
%\fi
%
% \macro{\ifchilddoc}
% \macro{\ifchilddocmanual}
% The conditional |\ifchilddoc| tells whether a
% child (true) or main (false) document is being compiled.
% The conditional |\ifchilddocmanual| tells whether
% the |\includeonly| mechanism is used (false) or
% the selection of child files must be performed manually (true).
% The definitions initialise to false:
%    \begin{macrocode}
\newif\ifchilddoc
\newif\ifchilddocmanual
%    \end{macrocode}

% \macro{\childdocname}
% \macro{\childdocjob}
% The macro |\childdocname| stores the name of the main document
% to be compiled. The macro |\childdocjob| stores the name of
% the document on which the \LaTeX{} compiler was originally invoked.
% The content of |\jobname| cannot be compared
% to filenames specified in the source due to different catcodes.
% The following code rescans |\jobname|, stores the result
% in |\childdocname| and saves a copy in |\childdocjob|:
%    \begin{macrocode}
\edef\childdocname{\scantokens\expandafter{\jobname\noexpand}}
\let\childdocjob\childdocname
%    \end{macrocode}

% \macro{\childdocdisable}
% The macro |\childdocdisable| prevents the main file
% from being processed more than once.
% At this stage, the main document command |\childdocmain|
% is assumed to be called once again where it should do nothing.
% Any subsequent call to it should prevent
% a secondary processing of the main document
% It overwrites the forwarding commands
% |\childdocof| and |\childdocforward|
% with empty macros to prevent further inclusions of the main document:
%    \begin{macrocode}
\newcommand{\childdocdisable}
{
  \renewcommand{\childdocmain}[1]{\renewcommand{\childdocmain}[1]{\endinput}}
  \renewcommand{\childdocof}[1]{}
  \renewcommand{\childdocby}[2][]{}
  \renewcommand{\childdocforward}[2][]{}
  \renewcommand{\childdocdisable}{}
}
%    \end{macrocode}

% \macro{\childdocmain}
% The macro |\childdocmain| is to be called at the top of the main file
% with nothing or the main filename (without extension) as argument.
% First, it breaks loops.
% If the argument is not empty and does not match |\childdocname|
% (which is set by the first inclusion of |childdoc.def|),
% |\ifchilddoc| is set to true, |\includeonly| is applied to the child file
% and |\jobname| is set to the main file
% (for proper handling of |.aux| files):
%    \begin{macrocode}
\newcommand{\childdocmain}[1]
{
  \childdocdisable\childdocmain{}
  \if?#1?\else
    \begingroup
      \def\childdoctmp{#1}
      \ifx\childdoctmp\childdocname
        \def\childdoctmp{}
      \else
        \def\childdoctmp
        {
          \childdoctrue
          \includeonly{\childdocname}
          \def\childdocjob{#1}
          \def\jobname{#1}
        }
      \fi
      \expandafter
    \endgroup
    \childdoctmp
  \fi
}
%    \end{macrocode}

% \macro{\childdocof}
% The command |\childdocof| redirects
% compilation to the main file |#1|.
%    \begin{macrocode}
\newcommand{\childdocof}[1]
{
  \childdocdisable
  \childdoctrue
  \includeonly{\childdocname}
  \def\jobname{#1}
  \def\childdocjob{#1}
  \input{#1}
}
%    \end{macrocode}

% \macro{\childdocby}
% The command |\childdocby| ....
%    \begin{macrocode}
\newcommand{\childdocby}[2][]
{
  \childdocdisable
  \childdoctrue
  \childdocmanualtrue
  \if?#1?\else
    \def\jobname{#2}
  \fi
  \def\childdocjob{#2}
  \input{#2}
  \endinput
}
%    \end{macrocode}

% \macro{\childdocforward}
% The command |\childdocforward| redirects
% compilation to the main file or
% (if the optional argument is given) a child file.
% Parameters are set as if the main file
% or a child file starting with |\childdocof| was compiled.
% Then compilation is handed over to the main file:
%    \begin{macrocode}
\newcommand{\childdocforward}[2][]
{
  \begingroup
    \if?#1?
      \def\childdoctmp
      {
        \def\childdocname{#2}
        \def\childdocjob{#2}
        \def\jobname{#2}
        \input{#2}
        \endinput
      }
    \else
      \def\childdoctmp
      {
        \childdocdisable
        \def\childdocname{#2}
        \childdoctrue
        \includeonly{#2}
        \def\childdocjob{#1}
        \def\jobname{#1}
        \input{#1}
        \endinput
      }
    \fi
    \expandafter
  \endgroup
  \childdoctmp
}
%    \end{macrocode}

% \macro{\childdocforwardprefix}
% The command |\childdocforwardprefix| redirects
% compilation to the main or a child file by means of a pattern.
% The prefix |#1| in the current filename is replaced by |#2|
% and the suffix of the current filename is kept
% (it is assumed that the filename does not contain the substring `|~~~|'
% which is used as a delimiter).
% Compilation is handed over to the new file by |\childdocforward|:
%    \begin{macrocode}
\newcommand{\childdocforwardprefix}[3][]
{
  \begingroup
    \def\childdocextract #2##1~~~{\def\childdoctmp{\childdocforward[#1]{#3##1}}}
    \expandafter\childdocextract\childdocname~~~
    \expandafter
  \endgroup
  \childdoctmp
}
%    \end{macrocode}

% \macro{\childdoc}
% The deprecated macro |\childdoc| is a legacy version of |\childdocmain|:
%    \begin{macrocode}
\newcommand{\childdoc}{\childdocmain}
%    \end{macrocode}

% \macro{\childdocredirect}
% The deprecated macro |\childdocredirect| is a legacy version
% of |\childdocforward| and |\childdocforwardprefix|:
%    \begin{macrocode}
\newcommand{\childdocredirect}[2][]
{
  \begingroup
    \if?#1?
      \def\childdoctmp{\childdocforward{#2}}
    \else
      \def\childdoctmp{\childdocforwardprefix{#1}{#2}}
    \fi
    \expandafter
  \endgroup
  \childdoctmp
}
%    \end{macrocode}

%\iffalse
%</package>
%\fi
%
\endinput
|\\
|\childdocforward{|\textit{main}|}|
\end{tabular}
\end{center}
%
Likewise, the following files |final|\textit{nn}|.tex|
compile the final version of the child document
|child|\textit{nn}|.tex|:
%
\begin{center}
\begin{tabular}{l}
|\def\version{final}|\\
|% \iffalse
%
% childdoc.dtx Copyright (C) 2017-2018 Niklas Beisert
%
% This work may be distributed and/or modified under the
% conditions of the LaTeX Project Public License, either version 1.3
% of this license or (at your option) any later version.
% The latest version of this license is in
%   http://www.latex-project.org/lppl.txt
% and version 1.3 or later is part of all distributions of LaTeX
% version 2005/12/01 or later.
%
% This work has the LPPL maintenance status `maintained'.
%
% The Current Maintainer of this work is Niklas Beisert.
%
% This work consists of the files childdoc.dtx and childdoc.ins
% and the derived files childdoc.def and cdocsamp.tex with
% cdocsch1.tex, cdocsch2.tex, cdocsdrf.tex, cdocsfn1.tex, cdocsfn2.tex.
%
%<package>\ifdefined\childdocmain\endinput\fi
%<package>\ProvidesFile{childdoc.def}[2018/12/30 v2.0 child document driver]
%<samplemain>\ProvidesFile{cdocsamp.tex}[2018/12/30 v2.0 sample for childdoc]
%<*driver>
%\ProvidesFile{childdoc.drv}[2018/12/30 v2.0 childdoc reference manual file]
\PassOptionsToClass{10pt,a4paper}{article}
\documentclass{ltxdoc}

\usepackage[margin=35mm]{geometry}
\usepackage{hyperref}
\usepackage{hyperxmp}
\usepackage[usenames]{color}

\hypersetup{colorlinks=true}
\hypersetup{pdfstartview=FitH}
\hypersetup{pdfpagemode=UseNone}
\hypersetup{pdfsource={}}
\hypersetup{pdflang={en-UK}}
\hypersetup{pdfcopyright={Copyright 2017-2018 Niklas Beisert.
  This work may be distributed and/or modified under the
  conditions of the LaTeX Project Public License, either version 1.3
  of this license or (at your option) any later version.}}
\hypersetup{pdflicenseurl={http://www.latex-project.org/lppl.txt}}
\hypersetup{pdfcontactaddress={ETH Zurich, ITP, HIT K,
  Wolfgang-Pauli-Strasse 27}}
\hypersetup{pdfcontactpostcode={8093}}
\hypersetup{pdfcontactcity={Zurich}}
\hypersetup{pdfcontactcountry={Switzerland}}
\hypersetup{pdfcontactemail={nbeisert@itp.phys.ethz.ch}}
\hypersetup{pdfcontacturl={http://people.phys.ethz.ch/\xmptilde nbeisert/}}

\newcommand{\secref}[1]{\hyperref[#1]{section \ref*{#1}}}

\parskip1ex
\parindent0pt
\let\olditemize\itemize
\def\itemize{\olditemize\parskip0pt}

\begin{document}

\title{The \textsf{childdoc} Package}
\hypersetup{pdftitle={The childdoc Package}}
\author{Niklas Beisert\\[2ex]
  Institut f\"ur Theoretische Physik\\
  Eidgen\"ossische Technische Hochschule Z\"urich\\
  Wolfgang-Pauli-Strasse 27, 8093 Z\"urich, Switzerland\\[1ex]
  \href{mailto:nbeisert@itp.phys.ethz.ch}
  {\texttt{nbeisert@itp.phys.ethz.ch}}}
\hypersetup{pdfauthor={Niklas Beisert}}
\hypersetup{pdfsubject={Manual for the LaTeX2e Package childdoc}}
\date{30 December 2018, \textsf{v2.0}}
\maketitle

\begin{abstract}\noindent
\textsf{childdoc} is a \LaTeXe{} package
that enables the direct compilation
of document sections included by |\include|
to individual files.
\end{abstract}

\begingroup
\parskip0ex
\tableofcontents
\endgroup

%%%%%%%%%%%%%%%%%%%%%%%%%%%%%%%%%%%%%%%%%%%%%%%%%%%%%%%%%%%%%%%%%%%%%%%%%%%%%%%%
%%%%%%%%%%%%%%%%%%%%%%%%%%%%%%%%%%%%%%%%%%%%%%%%%%%%%%%%%%%%%%%%%%%%%%%%%%%%%%%%
\section{Introduction}

\LaTeX{} provides a mechanism to structure a large document (such as a book)
into a main file and several child files (containing the chapters)
using the |\include| command.
This mechanism is beneficial for documents
which span hundreds of pages in order to
make the source file(s) more manageable.
Moreover, compilation can be restricted to
selected child files by means of the |\includeonly| command.
The latter feature can be used to reduce the compilation time while editing
(this was significantly more useful in the earlier days of \LaTeX{})
or to generate a smaller document which is easier to navigate.
Another application of |\includeonly| is to generate
documents consisting of selected parts of the complete document.

However, there are a few drawbacks of the plain |\include| mechanism:
\begin{itemize}
\item
The child files cannot be compiled on their own,
they can only be compiled via the main file.
A naive editing environment
(such as a text editor with an option
to have the current file processed by \LaTeX)
may require one to switch to the main file before compiling;
attempting to compile the child file produces errors.
\item
The main file must be modified (each time)
to adjust the |\includeonly| command
to the present needs. This easily leaves the main file in a messy state.
\item
The generated document will always carry the filename
of the main document. This is inconvenient if
several child files are to be compiled and
to be kept for distribution.
\end{itemize}

The present package provides a simple interface
to make child files individually compilable by \LaTeX{}.
Compiling a child file then has the same effect as compiling
the main file with an |\includeonly| command
to select the appropriate child.
Moreover the generated document will carry the name of the child
rather than the main file.
This resolves all three above issues.

This feature is meant to make the editing of books,
thesis documents and lecture notes somewhat more convenient.
However, the package can also be used efficiently for
composing a series of documents (such as exercise sheets)
which are typically distributed individually.
It then assists the author in generating the individual documents
(potentially in different versions)
as well as a document containing the collected series.
Another application is in developing style files
or other kinds of included material
where compilation of the style file could redirect
to a sample or test file.

%%%%%%%%%%%%%%%%%%%%%%%%%%%%%%%%%%%%%%%%%%%%%%%%%%%%%%%%%%%%%%%%%%%%%%%%%%%%%%%%
%%%%%%%%%%%%%%%%%%%%%%%%%%%%%%%%%%%%%%%%%%%%%%%%%%%%%%%%%%%%%%%%%%%%%%%%%%%%%%%%
\section{Usage}

First of all, the package \textsf{childdoc} is \emph{not} a standard
\LaTeXe{} |.sty| style file! Therefore it needs to be invoked in
a non-standard way.

%%%%%%%%%%%%%%%%%%%%%%%%%%%%%%%%%%%%%%%%%%%%%%%%%%%%%%%%%%%%%%%%%%%%%%%%%%%%%%%%
\subsection{Included Files}
\label{sec:include}

%%%%%%%%%%%%%%%%%%%%%%%%%%%%%%%%%%%%%%%%
\DescribeMacro{\childdocmain}
To use the package, add the commands
\begin{center}
\begin{tabular}{l}
|\input{childdoc.def}|\\
|\childdocmain{}|\\
\end{tabular}
\end{center}
at the very top of the main \LaTeX{} file,
in particular \emph{before} the |\documentclass| statement!
The argument of |\childdocmain| should be left empty
(but it must be present).

%%%%%%%%%%%%%%%%%%%%%%%%%%%%%%%%%%%%%%%%
\DescribeMacro{\childdocof}
Furthermore, add the commands
\begin{center}
\begin{tabular}{l}
|\input{childdoc.def}|\\
|\childdocof{|\textit{main}|}|\\
\end{tabular}
\end{center}
at the top of every child file \textit{child}
which is included by |\include{|\textit{child}|}|
from within the main file
(or at least for those files to be compiled individually).
The argument \textit{main} must be the filename of the main file.

There are a couple of
considerations in setting up the main and child documents:

%%%%%%%%%%%%%%%%%%%%%%%%%%%%%%%%%%%%%%%%
\paragraph{Restrictions.}

Please note the following restrictions:
\begin{itemize}
\item
|\childdocmain| must be called with one argument \textit{main}
to ensure compatibility with earlier version of the package.
It must either be empty (|\childdocmain{}|)
or precisely match the filename of the main file in which it is specified.
See \secref{sec:detection} for further information.
\item
The filename \textit{main} must be specified without the |.tex| extension.
\item
The filename \textit{main} is case sensitive
(even in case-insensitive file systems)
due to internal string comparison.
\item
The argument \textit{main} should be fully expanded, it cannot be a macro.
\item
Subdirectories and special characters should be avoided in filenames.
\item
The command |\childdocmain{|\textit{main}|}| must be followed by a whitespace.
It should not be followed immediately by another command
or by a comment mark `|%|'.
This is because the \TeX{} parser reads the token immediately following
the argument of |\childdocmain| and puts it
at the beginning of every child section;
however, a white\-space is ignored.
\end{itemize}

%%%%%%%%%%%%%%%%%%%%%%%%%%%%%%%%%%%%%%%%
\paragraph{Content of Main File.}

It is advisable to place all content in the child files included by |\include|.
Any output contained in the main file will appear in all child documents
unless suppressed manually;
it cannot be suppressed automatically by the |\includeonly| directive
and thus should normally be avoided.
A method to include some content in the main file
by means of conditional processing is described in \secref{sec:conditional}.

%%%%%%%%%%%%%%%%%%%%%%%%%%%%%%%%%%%%%%%%
\paragraph{Page Numbering.}

When only a part of the document is compiled,
the appropriate numbering of pages
(as well as other status parameters)
is determined from the |.aux| files.
The latter contain information from previous passes.
However this information needs to propagate through
all intermediate child documents.
Therefore the page numbering in child documents may well
be inconsistent until the complete document is compiled at least once.

A useful (if unconventional) way to always ensure a consistent
page numbering is to restart the numbering in each child document
and denote the pages by `\textit{child}|.|\textit{page}'
where \textit{child} represents the chapter/section number of the child file.
This can be achieved by the command
|\numberwithin{page}{|\textit{child}|}|
of the \textsf{amsmath} package
where \textit{child} can be |chapter| or |section|
depending on the chosen structuring.
Alternatively, one can modify the macro |\thepage| appropriately
and reset the counter |page| at the start of each child file.

%%%%%%%%%%%%%%%%%%%%%%%%%%%%%%%%%%%%%%%%%%%%%%%%%%%%%%%%%%%%%%%%%%%%%%%%%%%%%%%%
\subsection{Conditional Processing}
\label{sec:conditional}

The package provides a mechanism to compile different versions
of a document. To customise the versions further some conditional processing
can come in handy to distinguish which version is being compiled.
The package provides two macros to describe the compilation context:

%%%%%%%%%%%%%%%%%%%%%%%%%%%%%%%%%%%%%%%%
\DescribeMacro{\ifchilddoc}
The conditional |\ifchilddoc| distinguishes between the compilation of
child documents and the main document:
%
\begin{center}
|\ifchilddoc |\textit{child-code}| |[|\||else |\textit{main-code}]| \||fi|
\end{center}

%%%%%%%%%%%%%%%%%%%%%%%%%%%%%%%%%%%%%%%%
\DescribeMacro{\childdocname}
\DescribeMacro{\childdocjob}
The macro |\childdocname| contains the filename (without extension)
of the main or child file being processed.
Note that |\childdocjob| will always contain the name of the main file.

%%%%%%%%%%%%%%%%%%%%%%%%%%%%%%%%%%%%%%%%
\paragraph{Title Page.}

Conditional processing can be used to include a title or banner page
in the main document when proper precautions are taken.
Importantly, the code in the main file should ensure that the page counter
(as well as other status parameters which are stored in the |.aux| files)
takes the same value after the conditional processing.
Otherwise the page numbers may take divergent values
depending on which part is compiled.

For example, a title page could be declared by:
%
\begin{center}
\begin{tabular}{l}
|\ifchilddoc\||else|\\
|\addtocounter{page}{-1}|\\
\textit{code for title page}\\
|\newpage|\\
|\||fi|
\end{tabular}
\end{center}
%
A banner page for the child documents can be generated by:
%
\begin{center}
\begin{tabular}{l}
|\ifchilddoc|\\
|\addtocounter{page}{-1}|\\
\textit{code for banner page}\\
|\newpage|\\
|\||fi|
\end{tabular}
\end{center}
%
Here one could write a message such as:
\begin{center}
|This is the part \childdocname{} of \childdocjob{}.|
\end{center}

%%%%%%%%%%%%%%%%%%%%%%%%%%%%%%%%%%%%%%%%%%%%%%%%%%%%%%%%%%%%%%%%%%%%%%%%%%%%%%%%
\subsection{Flags}
\label{sec:flags}

The package makes it easy to generate different versions
of the main or child documents.
To this end compilation flags can be defined
and assigned different default values.
They will be particularly useful in conjunction
with the forwarding mechanism described in \secref{sec:forward}.

For example, it may be useful to have a flag |\version|
which can be set to |draft| or |final|.
The document source will contain some conditional code
depending on the value of |\version|.
Suppose further, the flag should default to |final| for the main file
and to |draft| for child files
which is a natural assignment for editing the document.
This is achieved by placing the following code
in the preamble of the main document
(below the |\childdocmain| directive):
%
\begin{center}
\begin{tabular}{l}
|\ifchilddoc|\\
|\providecommand{\version}{draft}|\\
|\||else|\\
|\providecommand{\version}{final}|\\
|\||fi|
\end{tabular}
\end{center}
%
The definition by |\providecommand| makes sure
that previous definitions are not overwritten.
Further statements |\providecommand{\version}{...}|
can thus be added before the above code to override it.

For the main file, one might add a line
(between |\childdocmain| and the above block)
%
\begin{center}
|%\ifchilddoc\||else\providecommand{\version}{draft}\||fi|
\end{center}
%
which can be uncommented to produce a draft version.
Likewise one can add a line to the very top of a child file
(above the |\childdocof{|\textit{main}|}| directive)
%
\begin{center}
|%\providecommand{\version}{final}|
\end{center}
%
which can be uncommented to produce the final version of this child document.

%%%%%%%%%%%%%%%%%%%%%%%%%%%%%%%%%%%%%%%%%%%%%%%%%%%%%%%%%%%%%%%%%%%%%%%%%%%%%%%%
\subsection{Forwarding}
\label{sec:forward}

Different versions of the main or child documents
using compilation flags as described in \secref{sec:flags}
can be (permanently) stored in different files
for convenient compilation, viewing and distribution.
To this end, the package defines a command
to pass on compilation to a different file:

%%%%%%%%%%%%%%%%%%%%%%%%%%%%%%%%%%%%%%%%
\DescribeMacro{\childdocforward}
The command |\childdocforward| redirects processing to
another source file:
%
\begin{center}
\begin{tabular}{l}
|\input{childdoc.def}|\\
|\childdocforward[|\textit{main}|]{|\textit{dest}|}|\\
\end{tabular}
\end{center}
%
The argument \textit{dest} is the destination file
(without extension).
It should be the main file or one of the child files.
Note that further \textsf{childdoc} directives
such as |\childdocof| and |\childdocforward|
in the indicated file will be processed in this form.
The optional argument \textit{main}
passes on directly to the main file \textit{main}
while pretending to compile the child \textit{dest}.
This form behaves as if \textit{dest}
issues |\childdocof{|\textit{main}|}| right away,
and no further \textsf{childdoc} directives will be processed.

%%%%%%%%%%%%%%%%%%%%%%%%%%%%%%%%%%%%%%%%
\DescribeMacro{\...prefix}
In the alternative form |\childdocforwardprefix|,
%
\begin{center}
\begin{tabular}{l}
|\input{childdoc.def}|\\
|\childdocforwardprefix[|\textit{main}|]{|\textit{prefix}|}{|\textit{dest}|}|
\end{tabular}
\end{center}
%
the destination file is determined by a pattern
depending on the current file:
To make this work, the current file must be called
`{\textit{prefix}\hspace{0.2em}\textit{suffix}}'
with \textit{prefix} matching precisely the argument.
Processing is then passed on to the file
`{\textit{dest}\hspace{0.2em}\textit{suffix}}'.
Surely, the same effect is achieved by
directly specifying the
argument `{\textit{dest}\hspace{0.2em}\textit{suffix}}'
in the first form.
However, that requires to set up a different file
for each child. With the alternative form of the command
all these files can have exactly the same content
which simplifies setting them up and maintaining them.

For example, the following file |draft.tex|
with a compilation flag |\version| as described in \secref{sec:flags}
compiles the main document as a draft:
%
\begin{center}
\begin{tabular}{l}
|\def\version{draft}|\\
|\input{childdoc.def}|\\
|\childdocforward{|\textit{main}|}|
\end{tabular}
\end{center}
%
Likewise, the following files |final|\textit{nn}|.tex|
compile the final version of the child document
|child|\textit{nn}|.tex|:
%
\begin{center}
\begin{tabular}{l}
|\def\version{final}|\\
|\input{childdoc.def}|\\
|\childdocforwardprefix{final}{child}|
\end{tabular}
\end{center}
%

Note that when several versions of a main file and/or of each child file
are to be generated, it may be convenient to set up a |Makefile| or
shell script to automatise the process.

%%%%%%%%%%%%%%%%%%%%%%%%%%%%%%%%%%%%%%%%%%%%%%%%%%%%%%%%%%%%%%%%%%%%%%%%%%%%%%%%
\subsection{Command Line Processing}
\label{sec:commandline}

The effect of redirection files can also be achieved by invoking
the \LaTeX{} compiler with a more elaborate command line.
Most conveniently this should be done as part
of a shell script or a |Makefile|.

When using \textsf{childdoc} in the main file, the following
command lines effectively perform a redirection
(note that depending on the shell being used,
backslashes may have to be doubled: `|\|' $\to$ `|\\|'):
%
\begin{center}
|... -jobname "|\textit{target}|" |\\|"|[\textit{flags}]%
|\input{childdoc.def}\childdocforward[|\textit{main}|]{|\textit{dest}|}"|
\end{center}
%
Here \textit{target} is the name of the output file,
\textit{main} is the name of the main file
and \textit{dest} is the name of the main or child file to be processed
(all filenames without extensions).
The optional argument \textit{main} can be omitted
if \textit{main} matches \textit{dest}.
Optionally, compilation \textit{flags} can be defined via |\def| commands.
This command line makes the \TeX{} engine believe
it is compiling the file \textit{target}
whose content is specified as the latter parameter.
The provided code then forwards the processing to
\textit{main} or \textit{dest} as described in \secref{sec:forward}.

%%%%%%%%%%%%%%%%%%%%%%%%%%%%%%%%%%%%%%%%%%%%%%%%%%%%%%%%%%%%%%%%%%%%%%%%%%%%%%%%
\subsection{Include by Input}
\label{sec:input}

Including child documents by |\include| has some restrictions by design.
Most notably, the content of a child document always occupies
its own set of pages; pages cannot be shared between child documents.
Usually, this behaviour makes perfect sense
because each child document contain an essential part of the document.
However, in some situations it may be desirable to compose
a document from a collection of parts
without having mandatory page breaks between then.
For this case, the package
provides a mechanism to include parts
by |\input| which can also be processed individually.
However, by construction this mechanism
requires manual handling of the content to be output.

%%%%%%%%%%%%%%%%%%%%%%%%%%%%%%%%%%%%%%%%
\DescribeMacro{\ifchilddocmanual}
The main file should be prepared as usual, see \secref{sec:include}.
However, the document body must make a distinction
between processing of an individual part and of the main document, e.g.:
%
\begin{center}
\begin{tabular}{l}
|\ifchilddocmanual|\\
|\input{\childdocname}|\\
|\||else|\\
\textit{document body with }|\input{|\textit{part}|}|\\
|\||fi|
\end{tabular}
\end{center}
%
The conditional |\ifchilddocmanual| is true whenever
a part to be included by |\input| is being compiled,
and the name of the part is stored in |\childdocname|.

%%%%%%%%%%%%%%%%%%%%%%%%%%%%%%%%%%%%%%%%
\DescribeMacro{\childdocby}
Each part to be included by |\input| should start with:
%
\begin{center}
\begin{tabular}{l}
|\input{childdoc.def}|\\
|\childdocby{|\textit{main}|}|\\
\end{tabular}
\end{center}
%
The directive |\childdocby| is similar to |\childdocof|
described in \secref{sec:include},
but the subsequent selection of content must be done manually.
To that end, both |\ifchilddoc| and |\ifchilddocmanual|
will be true upon processing of a part,
and the name of the part is stored in |\childdocname|.
Note that |\jobname| will be set to the filename of the current part
so that each part receives an individual |.aux| file
that does not interfere with the |.aux| file(s) of the main document.
This behaviour can be altered by the alternative form
|\childdocby[*]{|\textit{main}|}| (with a non-empty optional argument)
which uses the |.aux| file of the main document
by setting |\jobname| to \textit{main}.

%%%%%%%%%%%%%%%%%%%%%%%%%%%%%%%%%%%%%%%%%%%%%%%%%%%%%%%%%%%%%%%%%%%%%%%%%%%%%%%%
\subsection{Driver Development}
\label{sec:driver}

The \textsf{childdoc} mechanism can also be use for the development
of definition files such as \LaTeX{} styles or classes.
This case differs from the above setup with multiple parts
included by |\include| in that no |\includeonly| should be invoked.
This can be achieved by starting the include file
(before |\ProvidesPackage|) with:
%
\begin{center}
\begin{tabular}{l}
|\input{childdoc.def}|\\
|\childdocforward{|\textit{main}|}|\\
\end{tabular}
\end{center}
%
or alternatively with:
%
\begin{center}
\begin{tabular}{l}
|\input{childdoc.def}|\\
|\childdocby{|\textit{main}|}|\\
\end{tabular}
\end{center}
%
Both forms have slightly different effects as described above.
The main file is prepared as usual, see \secref{sec:include}.

%%%%%%%%%%%%%%%%%%%%%%%%%%%%%%%%%%%%%%%%%%%%%%%%%%%%%%%%%%%%%%%%%%%%%%%%%%%%%%%%
\subsection{Legacy Detection}
\label{sec:detection}

The directive |\childdocmain| in the main file can detect
whether the complete document or merely a child is to be compiled
even without using the directive |\childdocof|.
This method is deprecated because it is less robust
and there is no compelling reason to use it;
it is merely provided for backward compatibility
and it may be removed in future versions.

If the detection mechanism is to be used,
it is mandatory to correctly specify
the filename of the main file as the argument of |\childdocmain|:
%
\begin{center}
\begin{tabular}{l}
|\input{childdoc.def}|\\
|\childdocmain{|\textit{main}|}|\\
\end{tabular}
\end{center}
%
If |\jobname| does not match the argument \textit{main} of |\childdocmain|,
it is assumed that |\jobname| points to the child file to be compiled.
When using |\childdocmain| with the main file specified as argument,
it suffices to start a child file
with just |\input{|\textit{main}|}|
without loading of the package and using |\childdocof|.
If instead all processing is done
with the appropriate \textsf{childdoc} directives,
the argument of \textit{main} of |\childdocmain| can be empty.

An alternative version of the command line processing described
in \secref{sec:commandline} using the detection mechanism reads:
%
\begin{center}
|... -jobname "|\textit{target}|" "|[\textit{flags}]%
[|\def\jobname{|\textit{dest}|}|]|\input{|\textit{main}|}"|
\end{center}

%%%%%%%%%%%%%%%%%%%%%%%%%%%%%%%%%%%%%%%%%%%%%%%%%%%%%%%%%%%%%%%%%%%%%%%%%%%%%%%%
\subsection{Manual Code}
\label{sec:manual}

In case one cannot be certain whether the definitions file |childdoc.def|
is installed on the target \TeX{} distribution
and one prefers not to ship it,
it is conceivable to paste a few relevant commands into the sources.

To that end, drop all statements |\input{childdoc.def}|
and perform the replacements as outlined below.
Instead of |\childdocmain{|\textit{main}|}| add the following code
to the top of the main file:
%
\begin{center}
\begin{tabular}{l}
|\||ifdefined\childdocname\endinput\||fi\newif\ifchilddoc|\\
|\edef\childdocname{\scantokens\expandafter{\jobname\noexpand}}|\\
|\def\childdocmain{|\textit{main}|}\||ifx\childdocmain\childdocname\||else|\\
|\childdoctrue\includeonly{\childdocname}\let\jobname\childdocmain\||fi|\\
\end{tabular}
\end{center}
%
Instead of |\childdocof{|\textit{main}|}| just include the main file
at the top of each child file:
%
\begin{center}
|\input{|\textit{main}|}|
\end{center}
%
A simple redirection |\childdocforward{|\textit{dest}|}| is achieved by:
%
\begin{center}
|\def\jobname{|\textit{dest}|}\input{\jobname}|
\end{center}
%
The redirection with prefix
|\childdocforwardprefix[|\textit{prefix}|]{|\textit{dest}|}|
is accomplished by:
%
\begin{center}
\begin{tabular}{l}
|{\edef\jobname{\scantokens\expandafter{\jobname\noexpand}}|\\
|\def\redirectjob |\textit{prefix}|#1~~~{\gdef\jobname{|\textit{dest}|#1}}|\\
|\expandafter\redirectjob\jobname~~~}\input{\jobname}|
\end{tabular}
\end{center}

In an alternative approach,
child documents can be compiled by a specific command line
without additional code or specific definitions:
%
\begin{center}
|... -jobname "|\textit{target}|" "|[\textit{flags}]%
|\includeonly{|\textit{dest}|}\input{|\textit{main}|}"|
\end{center}
%

%%%%%%%%%%%%%%%%%%%%%%%%%%%%%%%%%%%%%%%%%%%%%%%%%%%%%%%%%%%%%%%%%%%%%%%%%%%%%%%%
%%%%%%%%%%%%%%%%%%%%%%%%%%%%%%%%%%%%%%%%%%%%%%%%%%%%%%%%%%%%%%%%%%%%%%%%%%%%%%%%
\section{Information}

%%%%%%%%%%%%%%%%%%%%%%%%%%%%%%%%%%%%%%%%%%%%%%%%%%%%%%%%%%%%%%%%%%%%%%%%%%%%%%%%
\subsection{Copyright}

Copyright \copyright{} 2017--2018 Niklas Beisert

This work may be distributed and/or modified under the
conditions of the \LaTeX{} Project Public License, either version 1.3
of this license or (at your option) any later version.
The latest version of this license is in
  \url{http://www.latex-project.org/lppl.txt}
and version 1.3 or later is part of all distributions of \LaTeX{}
version 2005/12/01 or later.

This work has the LPPL maintenance status `maintained'.

The Current Maintainer of this work is Niklas Beisert.

This work consists of the files |README.txt|, |childdoc.ins| and |childdoc.dtx|
as well as the derived files |childdoc.def|, |cdocsamp.tex|
with |cdocsch1.tex|, |cdocsch2.tex|, |cdocspt3.tex|, |cdocspt4.tex|,
|cdocsdrf.tex|, |cdocsfn1.tex|, |cdocsfn2.tex|
as well as |childdoc.pdf|.

%%%%%%%%%%%%%%%%%%%%%%%%%%%%%%%%%%%%%%%%%%%%%%%%%%%%%%%%%%%%%%%%%%%%%%%%%%%%%%%%
\subsection{Files and Installation}

The package consists of the files:
%
\begin{center}
\begin{tabular}{ll}
    |README.txt|   & readme file \\
    |childdoc.ins| & installation file \\
    |childdoc.dtx| & source file \\
    |childdoc.def| & definition file \\
    |cdocsamp.tex| & sample main file \\
    |cdocsch1.tex| & sample include file \\
    |cdocsch2.tex| & sample include file \\
    |cdocspt3.tex| & sample part file \\
    |cdocspt4.tex| & sample part file \\
    |cdocsdrf.tex| & sample redirection file \\
    |cdocsfn1.tex| & sample redirection file \\
    |cdocsfn2.tex| & sample redirection file \\
    |childdoc.pdf| & manual
\end{tabular}
\end{center}
%
The distribution consists of the files
|README.txt|, |childdoc.ins| and |childdoc.dtx|.
%
\begin{itemize}
\item
Run (pdf)\LaTeX{} on |childdoc.dtx|
to compile the manual |childdoc.pdf| (this file).
\item
Run \LaTeX{} on |childdoc.ins| to create the definitions file |childdoc.def|
and the sample |cdocsamp.tex| with include files
|cdocsch1.tex|, |cdocsch2.tex|, |cdocspt3.tex|, |cdocspt4.tex|,
|cdocsdrf.tex|, |cdocsfn1.tex|, |cdocsfn2.tex|.
Then copy the file |childdoc.def| to an appropriate directory of your \LaTeX{}
distribution, e.g.\ \textit{texmf-root}|/tex/latex/childdoc|.
\end{itemize}

%%%%%%%%%%%%%%%%%%%%%%%%%%%%%%%%%%%%%%%%%%%%%%%%%%%%%%%%%%%%%%%%%%%%%%%%%%%%%%%%
\subsection{Related CTAN Packages}

There are several other packages which offer a similar functionality:
%
\begin{itemize}
\item
The packages
\href{http://ctan.org/pkg/docmute}{\textsf{docmute}},
\href{http://ctan.org/pkg/includex}{\textsf{includex}} and
\href{http://ctan.org/pkg/standalone}{\textsf{standalone}}
provide commands to include only the document body of
a child file thus allowing both files to be compiled individually.
\item
The packages \href{http://ctan.org/pkg/subdocs}{\textsf{subdocs}}
and \href{http://ctan.org/pkg/subfiles}{\textsf{subfiles}}
provide structures in which the main and child documents can be
encapsulated and allowing them to be compiled individually.
The inclusion mechanism is different from the conventional |\include|.
\item
The package \href{http://ctan.org/pkg/combine}{\textsf{combine}}
is an elaborate solution to combine several documents into one.
\end{itemize}
%
See also the CTAN topic \href{http://ctan.org/topic/subdocs}{\textsf{subdocs}}
for further related packages.
The present package differs from the above solutions in that
a document structure constructed with the conventional |\include| mechanism
just needs two extra commands at the top of every file
such that all constituent files can be compiled individually.

%%%%%%%%%%%%%%%%%%%%%%%%%%%%%%%%%%%%%%%%%%%%%%%%%%%%%%%%%%%%%%%%%%%%%%%%%%%%%%%%
%\subsection{Feature Suggestions}
%
%The following is a list of features which may be useful for future
%versions of this package:
%%
%\begin{itemize}
%\item
%\ldots
%\end{itemize}

%%%%%%%%%%%%%%%%%%%%%%%%%%%%%%%%%%%%%%%%%%%%%%%%%%%%%%%%%%%%%%%%%%%%%%%%%%%%%%%%
\subsection{Revision History}

%%%%%%%%%%%%%%%%%%%%%%%%%%%%%%%%%%%%%%%%
\paragraph{v2.0:} 2018/12/30

\begin{itemize}
\item
immediate forward processing
\item
added |\childdocby| mechanism
\item
manual restructured
\end{itemize}

%%%%%%%%%%%%%%%%%%%%%%%%%%%%%%%%%%%%%%%%
\paragraph{v1.6:} 2018/01/17

\begin{itemize}
\item
application for development of include files
\item
corrections to manual
\end{itemize}

%%%%%%%%%%%%%%%%%%%%%%%%%%%%%%%%%%%%%%%%
\paragraph{v1.5:} 2017/05/21

\begin{itemize}
\item
more complete structuring introduced
\item
|\childdocof| introduced
\item
|\childdoc| renamed to |\childdocmain|
\item
|\childredirect| renamed to |\childdocforward| and |\childdocforwardprefix|
and functionality expanded
\end{itemize}

%%%%%%%%%%%%%%%%%%%%%%%%%%%%%%%%%%%%%%%%
\paragraph{v1.0:} 2017/04/27

\begin{itemize}
\item
manual and install package
\item
first version published on CTAN
\end{itemize}

%%%%%%%%%%%%%%%%%%%%%%%%%%%%%%%%%%%%%%%%
\paragraph{v0.6:} 2017/04/26

\begin{itemize}
\item
redirection mechanism added
\end{itemize}

%%%%%%%%%%%%%%%%%%%%%%%%%%%%%%%%%%%%%%%%
\paragraph{v0.5:} 2017/04/26

\begin{itemize}
\item
functionality in definition file
\end{itemize}


%%%%%%%%%%%%%%%%%%%%%%%%%%%%%%%%%%%%%%%%%%%%%%%%%%%%%%%%%%%%%%%%%%%%%%%%%%%%%%%%
%%%%%%%%%%%%%%%%%%%%%%%%%%%%%%%%%%%%%%%%%%%%%%%%%%%%%%%%%%%%%%%%%%%%%%%%%%%%%%%%
%%%%%%%%%%%%%%%%%%%%%%%%%%%%%%%%%%%%%%%%%%%%%%%%%%%%%%%%%%%%%%%%%%%%%%%%%%%%%%%%
\appendix

\settowidth\MacroIndent{\rmfamily\scriptsize 000\ }

 \DocInput{childdoc.dtx}

\end{document}
%</driver>
% \fi
%
% %%%%%%%%%%%%%%%%%%%%%%%%%%%%%%%%%%%%%%%%%%%%%%%%%%%%%%%%%%%%%%%%%%%%%%%%%%%%%%
% %%%%%%%%%%%%%%%%%%%%%%%%%%%%%%%%%%%%%%%%%%%%%%%%%%%%%%%%%%%%%%%%%%%%%%%%%%%%%%
% \section{Sample}
%\iffalse
%<*samplemain>
%\fi
%
% The following presents a sample document
% with two chapters, two parts, a title page,
% a compile flag as well as three forwarding files to set the flag.
% It consists of eight |.tex| files:
% \begin{center}
% \begin{tabular}{ll}
% |cdocsamp.tex|&main file\\
% |cdocsch1.tex|&include file for chapter 1\\
% |cdocsch2.tex|&include file for chapter 2\\
% |cdocspt3.tex|&include file for part 3\\
% |cdocspt4.tex|&include file for part 4\\
% |cdocsdrf.tex|&forwarding file for main file in draft mode\\
% |cdocsfi1.tex|&forwarding file for final version of chapter 1\\
% |cdocsfi2.tex|&forwarding file for final version of chapter 2\\
% \end{tabular}
% \end{center}
% Each of the eight files can be compiled directly by the \LaTeX{} compiler.
%
% %%%%%%%%%%%%%%%%%%%%%%%%%%%%%%%%%%%%%%
% \paragraph{Main File.}
%
% The main file is called |cdocsamp.tex|.
%
% Load the \textsf{childdoc} definitions and
% declare the filename for the main document:
%    \begin{macrocode}
\input{childdoc.def}
\childdocmain{}
%    \end{macrocode}

% Optional override for |\version| flag:
%    \begin{macrocode}
%%\ifchilddoc\else\providecommand{\version}{draft}\fi
%    \end{macrocode}

% Define the default values for the |\version| flag
% (|final| for the main file and |draft| for childs):
%    \begin{macrocode}
\ifchilddoc
\providecommand{\version}{draft}
\else
\providecommand{\version}{final}
\fi
%    \end{macrocode}

% Load the standard document class:
%    \begin{macrocode}
\documentclass[12pt]{article}
%    \end{macrocode}

% Start the document body:
%    \begin{macrocode}
\begin{document}
%    \end{macrocode}

% Declare a title page.
% Print title, part of document being processed and version flag:
%    \begin{macrocode}
\addtocounter{page}{-1}
\begin{center}
{\LARGE\bfseries{}childdoc example\par}
\vspace{1cm}
\ifchilddoc
\ifchilddocmanual part\else chapter\fi:
`\childdocname' of `\childdocjob'\par
\else
main document: `\childdocjob'\par
\fi
version: \version\par
\end{center}
\newpage
%    \end{macrocode}

% Manually include selected file,
% otherwise process as usual:
%    \begin{macrocode}
\ifchilddocmanual
\section*{part `\childdocname'}
\input{\childdocname}
\else
%    \end{macrocode}

% Include the two chapters:
%    \begin{macrocode}
\include{cdocsch1}
\include{cdocsch2}
%    \end{macrocode}

% Include the two parts unless only chapters should be displayed:
%    \begin{macrocode}
\ifchilddoc\else
\section{part three}
\input{cdocspt3}
\section{part four}
\input{cdocspt4}
\fi
%    \end{macrocode}

% Process as usual until here:
%    \begin{macrocode}
\fi
%    \end{macrocode}

% End of document body:
%    \begin{macrocode}
\end{document}
%    \end{macrocode}
%\iffalse
%</samplemain>
%\fi
%
% %%%%%%%%%%%%%%%%%%%%%%%%%%%%%%%%%%%%%%
% \paragraph{Chapter Include Files.}
%
% The include files are called |cdocsch1.tex| and |cdocsch2.tex|.
%
%\iffalse
%<*samplechap1|samplechap2>
%\fi

% Optional override for |\version| flag:
%    \begin{macrocode}
%%\providecommand{\version}{final}
%    \end{macrocode}

% Include the main document:
%    \begin{macrocode}
\input{childdoc.def}
\childdocof{cdocsamp}
%    \end{macrocode}

%\iffalse
%</samplechap1|samplechap2>
%\fi
%
%\iffalse
%<*samplechap1>
%\fi
% Some text for chapter 1:
%    \begin{macrocode}
\section{one}
some text in chapter one
%    \end{macrocode}

%\iffalse
%</samplechap1>
%\fi
% Some text for chapter 2:
%\iffalse
%<*samplechap2>
%\fi
%    \begin{macrocode}
\section{two}
more text in chapter two
%    \end{macrocode}

%\iffalse
%</samplechap2>
%\fi
%
% %%%%%%%%%%%%%%%%%%%%%%%%%%%%%%%%%%%%%%
% \paragraph{Part Include Files.}
%
% The include files are called |cdocspt3.tex| and |cdocspt4.tex|.
%
%\iffalse
%<*samplepart3|samplepart4>
%\fi

% Optional override for |\version| flag:
%    \begin{macrocode}
%%\providecommand{\version}{final}
%    \end{macrocode}

% Include the main document:
%    \begin{macrocode}
\input{childdoc.def}
\childdocby{cdocsamp}
%    \end{macrocode}

%\iffalse
%</samplepart3|samplepart4>
%\fi
%
%\iffalse
%<*samplepart3>
%\fi
% Some text for part 3:
%    \begin{macrocode}
some text in part three
%    \end{macrocode}

%\iffalse
%</samplepart3>
%\fi
% Some text for part 4:
%\iffalse
%<*samplepart4>
%\fi
%    \begin{macrocode}
more text in part four
%    \end{macrocode}

%\iffalse
%</samplepart4>
%\fi
%
% %%%%%%%%%%%%%%%%%%%%%%%%%%%%%%%%%%%%%%
% \paragraph{Forwarding for a Complete Draft.}
%
% The following forwarding file |cdocsdrf.tex|
% compiles the main document in draft mode:
%\iffalse
%<*sampledraft>
%\fi
%    \begin{macrocode}
\def\version{draft}
\input{childdoc.def}
\childdocforward{cdocsamp}
%    \end{macrocode}

%\iffalse
%</sampledraft>
%\fi
%
% %%%%%%%%%%%%%%%%%%%%%%%%%%%%%%%%%%%%%%
% \paragraph{Forwarding for Final Version of the Chapters.}
%
% The following forwarding files |cdocsfn1.tex| and |cdocsfn2.tex|
% (with identical content)
% compile the final versions of the child documents
% |cdocsch1.tex| and |cdocsch2.tex|, respectively:
%\iffalse
%<*samplefinal>
%\fi
%    \begin{macrocode}
\def\version{final}
\input{childdoc.def}
\childdocforwardprefix[cdocsamp]{cdocsfn}{cdocsch}
%    \end{macrocode}

%\iffalse
%</samplefinal>
%\fi
%
% %%%%%%%%%%%%%%%%%%%%%%%%%%%%%%%%%%%%%%
% \paragraph{Command Line Processing.}
%
% The following three command lines generate the output files
% |cdocscld|, |cdocscl1| and |cdocscl2|
% which should be identical to
% |cdocsdrf|, |cdocsch1| and |cdocsfn2|, respectively:
% \begin{center}
% \begin{tabular}{l}
% |latex -jobname cdocscld \|\\
% |  "\def\version{draft}\input{childdoc.def}\childdocforward{cdocsamp}"|\\
% |latex -jobname cdocscl1 \|\\
% |  "\input{childdoc.def}\childdocforward[cdocsamp]{cdocsch1}"|\\
% |latex -jobname cdocscl2 \|\\
% |  "\def\version{final}\input{childdoc.def}\childdocforward{cdocsch2}"|
% \end{tabular}
% \end{center}
% Note that the trailing backslash on each first line
% merely continues the input to the second line
% (for convenient cut ant paste).
% Furthermore, the command |latex| can be replaced by any
% of its alternative versions such as |pdflatex|.
%
% %%%%%%%%%%%%%%%%%%%%%%%%%%%%%%%%%%%%%%%%%%%%%%%%%%%%%%%%%%%%%%%%%%%%%%%%%%%%%%
% %%%%%%%%%%%%%%%%%%%%%%%%%%%%%%%%%%%%%%%%%%%%%%%%%%%%%%%%%%%%%%%%%%%%%%%%%%%%%%
% \section{Implementation}
%\iffalse
%<*package>
%\fi
%
% This section describes the definitions file |childdoc.def|.

% The definitions cannot be loaded using |\usepackage| or |\RequirePackage|
% which has a mechanism to prevent loading a style file more than once.
% When loading the definitions by means of |\input|
% multiple instances have to be prevented manually:
%\iffalse
%This code needs to be before the `\ProvidesFile' directive
%which is defined at the beginning of this file.
%Therefore it is also placed there and commented out here.
%</package>
%<*discard>
%\fi
%    \begin{macrocode}
\ifdefined\childdocmain\endinput\fi
%    \end{macrocode}
%\iffalse
%</discard>
%<*package>
%\fi
%
% \macro{\ifchilddoc}
% \macro{\ifchilddocmanual}
% The conditional |\ifchilddoc| tells whether a
% child (true) or main (false) document is being compiled.
% The conditional |\ifchilddocmanual| tells whether
% the |\includeonly| mechanism is used (false) or
% the selection of child files must be performed manually (true).
% The definitions initialise to false:
%    \begin{macrocode}
\newif\ifchilddoc
\newif\ifchilddocmanual
%    \end{macrocode}

% \macro{\childdocname}
% \macro{\childdocjob}
% The macro |\childdocname| stores the name of the main document
% to be compiled. The macro |\childdocjob| stores the name of
% the document on which the \LaTeX{} compiler was originally invoked.
% The content of |\jobname| cannot be compared
% to filenames specified in the source due to different catcodes.
% The following code rescans |\jobname|, stores the result
% in |\childdocname| and saves a copy in |\childdocjob|:
%    \begin{macrocode}
\edef\childdocname{\scantokens\expandafter{\jobname\noexpand}}
\let\childdocjob\childdocname
%    \end{macrocode}

% \macro{\childdocdisable}
% The macro |\childdocdisable| prevents the main file
% from being processed more than once.
% At this stage, the main document command |\childdocmain|
% is assumed to be called once again where it should do nothing.
% Any subsequent call to it should prevent
% a secondary processing of the main document
% It overwrites the forwarding commands
% |\childdocof| and |\childdocforward|
% with empty macros to prevent further inclusions of the main document:
%    \begin{macrocode}
\newcommand{\childdocdisable}
{
  \renewcommand{\childdocmain}[1]{\renewcommand{\childdocmain}[1]{\endinput}}
  \renewcommand{\childdocof}[1]{}
  \renewcommand{\childdocby}[2][]{}
  \renewcommand{\childdocforward}[2][]{}
  \renewcommand{\childdocdisable}{}
}
%    \end{macrocode}

% \macro{\childdocmain}
% The macro |\childdocmain| is to be called at the top of the main file
% with nothing or the main filename (without extension) as argument.
% First, it breaks loops.
% If the argument is not empty and does not match |\childdocname|
% (which is set by the first inclusion of |childdoc.def|),
% |\ifchilddoc| is set to true, |\includeonly| is applied to the child file
% and |\jobname| is set to the main file
% (for proper handling of |.aux| files):
%    \begin{macrocode}
\newcommand{\childdocmain}[1]
{
  \childdocdisable\childdocmain{}
  \if?#1?\else
    \begingroup
      \def\childdoctmp{#1}
      \ifx\childdoctmp\childdocname
        \def\childdoctmp{}
      \else
        \def\childdoctmp
        {
          \childdoctrue
          \includeonly{\childdocname}
          \def\childdocjob{#1}
          \def\jobname{#1}
        }
      \fi
      \expandafter
    \endgroup
    \childdoctmp
  \fi
}
%    \end{macrocode}

% \macro{\childdocof}
% The command |\childdocof| redirects
% compilation to the main file |#1|.
%    \begin{macrocode}
\newcommand{\childdocof}[1]
{
  \childdocdisable
  \childdoctrue
  \includeonly{\childdocname}
  \def\jobname{#1}
  \def\childdocjob{#1}
  \input{#1}
}
%    \end{macrocode}

% \macro{\childdocby}
% The command |\childdocby| ....
%    \begin{macrocode}
\newcommand{\childdocby}[2][]
{
  \childdocdisable
  \childdoctrue
  \childdocmanualtrue
  \if?#1?\else
    \def\jobname{#2}
  \fi
  \def\childdocjob{#2}
  \input{#2}
  \endinput
}
%    \end{macrocode}

% \macro{\childdocforward}
% The command |\childdocforward| redirects
% compilation to the main file or
% (if the optional argument is given) a child file.
% Parameters are set as if the main file
% or a child file starting with |\childdocof| was compiled.
% Then compilation is handed over to the main file:
%    \begin{macrocode}
\newcommand{\childdocforward}[2][]
{
  \begingroup
    \if?#1?
      \def\childdoctmp
      {
        \def\childdocname{#2}
        \def\childdocjob{#2}
        \def\jobname{#2}
        \input{#2}
        \endinput
      }
    \else
      \def\childdoctmp
      {
        \childdocdisable
        \def\childdocname{#2}
        \childdoctrue
        \includeonly{#2}
        \def\childdocjob{#1}
        \def\jobname{#1}
        \input{#1}
        \endinput
      }
    \fi
    \expandafter
  \endgroup
  \childdoctmp
}
%    \end{macrocode}

% \macro{\childdocforwardprefix}
% The command |\childdocforwardprefix| redirects
% compilation to the main or a child file by means of a pattern.
% The prefix |#1| in the current filename is replaced by |#2|
% and the suffix of the current filename is kept
% (it is assumed that the filename does not contain the substring `|~~~|'
% which is used as a delimiter).
% Compilation is handed over to the new file by |\childdocforward|:
%    \begin{macrocode}
\newcommand{\childdocforwardprefix}[3][]
{
  \begingroup
    \def\childdocextract #2##1~~~{\def\childdoctmp{\childdocforward[#1]{#3##1}}}
    \expandafter\childdocextract\childdocname~~~
    \expandafter
  \endgroup
  \childdoctmp
}
%    \end{macrocode}

% \macro{\childdoc}
% The deprecated macro |\childdoc| is a legacy version of |\childdocmain|:
%    \begin{macrocode}
\newcommand{\childdoc}{\childdocmain}
%    \end{macrocode}

% \macro{\childdocredirect}
% The deprecated macro |\childdocredirect| is a legacy version
% of |\childdocforward| and |\childdocforwardprefix|:
%    \begin{macrocode}
\newcommand{\childdocredirect}[2][]
{
  \begingroup
    \if?#1?
      \def\childdoctmp{\childdocforward{#2}}
    \else
      \def\childdoctmp{\childdocforwardprefix{#1}{#2}}
    \fi
    \expandafter
  \endgroup
  \childdoctmp
}
%    \end{macrocode}

%\iffalse
%</package>
%\fi
%
\endinput
|\\
|\childdocforwardprefix{final}{child}|
\end{tabular}
\end{center}
%

Note that when several versions of a main file and/or of each child file
are to be generated, it may be convenient to set up a |Makefile| or
shell script to automatise the process.

%%%%%%%%%%%%%%%%%%%%%%%%%%%%%%%%%%%%%%%%%%%%%%%%%%%%%%%%%%%%%%%%%%%%%%%%%%%%%%%%
\subsection{Command Line Processing}
\label{sec:commandline}

The effect of redirection files can also be achieved by invoking
the \LaTeX{} compiler with a more elaborate command line.
Most conveniently this should be done as part
of a shell script or a |Makefile|.

When using \textsf{childdoc} in the main file, the following
command lines effectively perform a redirection
(note that depending on the shell being used,
backslashes may have to be doubled: `|\|' $\to$ `|\\|'):
%
\begin{center}
|... -jobname "|\textit{target}|" |\\|"|[\textit{flags}]%
|% \iffalse
%
% childdoc.dtx Copyright (C) 2017-2018 Niklas Beisert
%
% This work may be distributed and/or modified under the
% conditions of the LaTeX Project Public License, either version 1.3
% of this license or (at your option) any later version.
% The latest version of this license is in
%   http://www.latex-project.org/lppl.txt
% and version 1.3 or later is part of all distributions of LaTeX
% version 2005/12/01 or later.
%
% This work has the LPPL maintenance status `maintained'.
%
% The Current Maintainer of this work is Niklas Beisert.
%
% This work consists of the files childdoc.dtx and childdoc.ins
% and the derived files childdoc.def and cdocsamp.tex with
% cdocsch1.tex, cdocsch2.tex, cdocsdrf.tex, cdocsfn1.tex, cdocsfn2.tex.
%
%<package>\ifdefined\childdocmain\endinput\fi
%<package>\ProvidesFile{childdoc.def}[2018/12/30 v2.0 child document driver]
%<samplemain>\ProvidesFile{cdocsamp.tex}[2018/12/30 v2.0 sample for childdoc]
%<*driver>
%\ProvidesFile{childdoc.drv}[2018/12/30 v2.0 childdoc reference manual file]
\PassOptionsToClass{10pt,a4paper}{article}
\documentclass{ltxdoc}

\usepackage[margin=35mm]{geometry}
\usepackage{hyperref}
\usepackage{hyperxmp}
\usepackage[usenames]{color}

\hypersetup{colorlinks=true}
\hypersetup{pdfstartview=FitH}
\hypersetup{pdfpagemode=UseNone}
\hypersetup{pdfsource={}}
\hypersetup{pdflang={en-UK}}
\hypersetup{pdfcopyright={Copyright 2017-2018 Niklas Beisert.
  This work may be distributed and/or modified under the
  conditions of the LaTeX Project Public License, either version 1.3
  of this license or (at your option) any later version.}}
\hypersetup{pdflicenseurl={http://www.latex-project.org/lppl.txt}}
\hypersetup{pdfcontactaddress={ETH Zurich, ITP, HIT K,
  Wolfgang-Pauli-Strasse 27}}
\hypersetup{pdfcontactpostcode={8093}}
\hypersetup{pdfcontactcity={Zurich}}
\hypersetup{pdfcontactcountry={Switzerland}}
\hypersetup{pdfcontactemail={nbeisert@itp.phys.ethz.ch}}
\hypersetup{pdfcontacturl={http://people.phys.ethz.ch/\xmptilde nbeisert/}}

\newcommand{\secref}[1]{\hyperref[#1]{section \ref*{#1}}}

\parskip1ex
\parindent0pt
\let\olditemize\itemize
\def\itemize{\olditemize\parskip0pt}

\begin{document}

\title{The \textsf{childdoc} Package}
\hypersetup{pdftitle={The childdoc Package}}
\author{Niklas Beisert\\[2ex]
  Institut f\"ur Theoretische Physik\\
  Eidgen\"ossische Technische Hochschule Z\"urich\\
  Wolfgang-Pauli-Strasse 27, 8093 Z\"urich, Switzerland\\[1ex]
  \href{mailto:nbeisert@itp.phys.ethz.ch}
  {\texttt{nbeisert@itp.phys.ethz.ch}}}
\hypersetup{pdfauthor={Niklas Beisert}}
\hypersetup{pdfsubject={Manual for the LaTeX2e Package childdoc}}
\date{30 December 2018, \textsf{v2.0}}
\maketitle

\begin{abstract}\noindent
\textsf{childdoc} is a \LaTeXe{} package
that enables the direct compilation
of document sections included by |\include|
to individual files.
\end{abstract}

\begingroup
\parskip0ex
\tableofcontents
\endgroup

%%%%%%%%%%%%%%%%%%%%%%%%%%%%%%%%%%%%%%%%%%%%%%%%%%%%%%%%%%%%%%%%%%%%%%%%%%%%%%%%
%%%%%%%%%%%%%%%%%%%%%%%%%%%%%%%%%%%%%%%%%%%%%%%%%%%%%%%%%%%%%%%%%%%%%%%%%%%%%%%%
\section{Introduction}

\LaTeX{} provides a mechanism to structure a large document (such as a book)
into a main file and several child files (containing the chapters)
using the |\include| command.
This mechanism is beneficial for documents
which span hundreds of pages in order to
make the source file(s) more manageable.
Moreover, compilation can be restricted to
selected child files by means of the |\includeonly| command.
The latter feature can be used to reduce the compilation time while editing
(this was significantly more useful in the earlier days of \LaTeX{})
or to generate a smaller document which is easier to navigate.
Another application of |\includeonly| is to generate
documents consisting of selected parts of the complete document.

However, there are a few drawbacks of the plain |\include| mechanism:
\begin{itemize}
\item
The child files cannot be compiled on their own,
they can only be compiled via the main file.
A naive editing environment
(such as a text editor with an option
to have the current file processed by \LaTeX)
may require one to switch to the main file before compiling;
attempting to compile the child file produces errors.
\item
The main file must be modified (each time)
to adjust the |\includeonly| command
to the present needs. This easily leaves the main file in a messy state.
\item
The generated document will always carry the filename
of the main document. This is inconvenient if
several child files are to be compiled and
to be kept for distribution.
\end{itemize}

The present package provides a simple interface
to make child files individually compilable by \LaTeX{}.
Compiling a child file then has the same effect as compiling
the main file with an |\includeonly| command
to select the appropriate child.
Moreover the generated document will carry the name of the child
rather than the main file.
This resolves all three above issues.

This feature is meant to make the editing of books,
thesis documents and lecture notes somewhat more convenient.
However, the package can also be used efficiently for
composing a series of documents (such as exercise sheets)
which are typically distributed individually.
It then assists the author in generating the individual documents
(potentially in different versions)
as well as a document containing the collected series.
Another application is in developing style files
or other kinds of included material
where compilation of the style file could redirect
to a sample or test file.

%%%%%%%%%%%%%%%%%%%%%%%%%%%%%%%%%%%%%%%%%%%%%%%%%%%%%%%%%%%%%%%%%%%%%%%%%%%%%%%%
%%%%%%%%%%%%%%%%%%%%%%%%%%%%%%%%%%%%%%%%%%%%%%%%%%%%%%%%%%%%%%%%%%%%%%%%%%%%%%%%
\section{Usage}

First of all, the package \textsf{childdoc} is \emph{not} a standard
\LaTeXe{} |.sty| style file! Therefore it needs to be invoked in
a non-standard way.

%%%%%%%%%%%%%%%%%%%%%%%%%%%%%%%%%%%%%%%%%%%%%%%%%%%%%%%%%%%%%%%%%%%%%%%%%%%%%%%%
\subsection{Included Files}
\label{sec:include}

%%%%%%%%%%%%%%%%%%%%%%%%%%%%%%%%%%%%%%%%
\DescribeMacro{\childdocmain}
To use the package, add the commands
\begin{center}
\begin{tabular}{l}
|\input{childdoc.def}|\\
|\childdocmain{}|\\
\end{tabular}
\end{center}
at the very top of the main \LaTeX{} file,
in particular \emph{before} the |\documentclass| statement!
The argument of |\childdocmain| should be left empty
(but it must be present).

%%%%%%%%%%%%%%%%%%%%%%%%%%%%%%%%%%%%%%%%
\DescribeMacro{\childdocof}
Furthermore, add the commands
\begin{center}
\begin{tabular}{l}
|\input{childdoc.def}|\\
|\childdocof{|\textit{main}|}|\\
\end{tabular}
\end{center}
at the top of every child file \textit{child}
which is included by |\include{|\textit{child}|}|
from within the main file
(or at least for those files to be compiled individually).
The argument \textit{main} must be the filename of the main file.

There are a couple of
considerations in setting up the main and child documents:

%%%%%%%%%%%%%%%%%%%%%%%%%%%%%%%%%%%%%%%%
\paragraph{Restrictions.}

Please note the following restrictions:
\begin{itemize}
\item
|\childdocmain| must be called with one argument \textit{main}
to ensure compatibility with earlier version of the package.
It must either be empty (|\childdocmain{}|)
or precisely match the filename of the main file in which it is specified.
See \secref{sec:detection} for further information.
\item
The filename \textit{main} must be specified without the |.tex| extension.
\item
The filename \textit{main} is case sensitive
(even in case-insensitive file systems)
due to internal string comparison.
\item
The argument \textit{main} should be fully expanded, it cannot be a macro.
\item
Subdirectories and special characters should be avoided in filenames.
\item
The command |\childdocmain{|\textit{main}|}| must be followed by a whitespace.
It should not be followed immediately by another command
or by a comment mark `|%|'.
This is because the \TeX{} parser reads the token immediately following
the argument of |\childdocmain| and puts it
at the beginning of every child section;
however, a white\-space is ignored.
\end{itemize}

%%%%%%%%%%%%%%%%%%%%%%%%%%%%%%%%%%%%%%%%
\paragraph{Content of Main File.}

It is advisable to place all content in the child files included by |\include|.
Any output contained in the main file will appear in all child documents
unless suppressed manually;
it cannot be suppressed automatically by the |\includeonly| directive
and thus should normally be avoided.
A method to include some content in the main file
by means of conditional processing is described in \secref{sec:conditional}.

%%%%%%%%%%%%%%%%%%%%%%%%%%%%%%%%%%%%%%%%
\paragraph{Page Numbering.}

When only a part of the document is compiled,
the appropriate numbering of pages
(as well as other status parameters)
is determined from the |.aux| files.
The latter contain information from previous passes.
However this information needs to propagate through
all intermediate child documents.
Therefore the page numbering in child documents may well
be inconsistent until the complete document is compiled at least once.

A useful (if unconventional) way to always ensure a consistent
page numbering is to restart the numbering in each child document
and denote the pages by `\textit{child}|.|\textit{page}'
where \textit{child} represents the chapter/section number of the child file.
This can be achieved by the command
|\numberwithin{page}{|\textit{child}|}|
of the \textsf{amsmath} package
where \textit{child} can be |chapter| or |section|
depending on the chosen structuring.
Alternatively, one can modify the macro |\thepage| appropriately
and reset the counter |page| at the start of each child file.

%%%%%%%%%%%%%%%%%%%%%%%%%%%%%%%%%%%%%%%%%%%%%%%%%%%%%%%%%%%%%%%%%%%%%%%%%%%%%%%%
\subsection{Conditional Processing}
\label{sec:conditional}

The package provides a mechanism to compile different versions
of a document. To customise the versions further some conditional processing
can come in handy to distinguish which version is being compiled.
The package provides two macros to describe the compilation context:

%%%%%%%%%%%%%%%%%%%%%%%%%%%%%%%%%%%%%%%%
\DescribeMacro{\ifchilddoc}
The conditional |\ifchilddoc| distinguishes between the compilation of
child documents and the main document:
%
\begin{center}
|\ifchilddoc |\textit{child-code}| |[|\||else |\textit{main-code}]| \||fi|
\end{center}

%%%%%%%%%%%%%%%%%%%%%%%%%%%%%%%%%%%%%%%%
\DescribeMacro{\childdocname}
\DescribeMacro{\childdocjob}
The macro |\childdocname| contains the filename (without extension)
of the main or child file being processed.
Note that |\childdocjob| will always contain the name of the main file.

%%%%%%%%%%%%%%%%%%%%%%%%%%%%%%%%%%%%%%%%
\paragraph{Title Page.}

Conditional processing can be used to include a title or banner page
in the main document when proper precautions are taken.
Importantly, the code in the main file should ensure that the page counter
(as well as other status parameters which are stored in the |.aux| files)
takes the same value after the conditional processing.
Otherwise the page numbers may take divergent values
depending on which part is compiled.

For example, a title page could be declared by:
%
\begin{center}
\begin{tabular}{l}
|\ifchilddoc\||else|\\
|\addtocounter{page}{-1}|\\
\textit{code for title page}\\
|\newpage|\\
|\||fi|
\end{tabular}
\end{center}
%
A banner page for the child documents can be generated by:
%
\begin{center}
\begin{tabular}{l}
|\ifchilddoc|\\
|\addtocounter{page}{-1}|\\
\textit{code for banner page}\\
|\newpage|\\
|\||fi|
\end{tabular}
\end{center}
%
Here one could write a message such as:
\begin{center}
|This is the part \childdocname{} of \childdocjob{}.|
\end{center}

%%%%%%%%%%%%%%%%%%%%%%%%%%%%%%%%%%%%%%%%%%%%%%%%%%%%%%%%%%%%%%%%%%%%%%%%%%%%%%%%
\subsection{Flags}
\label{sec:flags}

The package makes it easy to generate different versions
of the main or child documents.
To this end compilation flags can be defined
and assigned different default values.
They will be particularly useful in conjunction
with the forwarding mechanism described in \secref{sec:forward}.

For example, it may be useful to have a flag |\version|
which can be set to |draft| or |final|.
The document source will contain some conditional code
depending on the value of |\version|.
Suppose further, the flag should default to |final| for the main file
and to |draft| for child files
which is a natural assignment for editing the document.
This is achieved by placing the following code
in the preamble of the main document
(below the |\childdocmain| directive):
%
\begin{center}
\begin{tabular}{l}
|\ifchilddoc|\\
|\providecommand{\version}{draft}|\\
|\||else|\\
|\providecommand{\version}{final}|\\
|\||fi|
\end{tabular}
\end{center}
%
The definition by |\providecommand| makes sure
that previous definitions are not overwritten.
Further statements |\providecommand{\version}{...}|
can thus be added before the above code to override it.

For the main file, one might add a line
(between |\childdocmain| and the above block)
%
\begin{center}
|%\ifchilddoc\||else\providecommand{\version}{draft}\||fi|
\end{center}
%
which can be uncommented to produce a draft version.
Likewise one can add a line to the very top of a child file
(above the |\childdocof{|\textit{main}|}| directive)
%
\begin{center}
|%\providecommand{\version}{final}|
\end{center}
%
which can be uncommented to produce the final version of this child document.

%%%%%%%%%%%%%%%%%%%%%%%%%%%%%%%%%%%%%%%%%%%%%%%%%%%%%%%%%%%%%%%%%%%%%%%%%%%%%%%%
\subsection{Forwarding}
\label{sec:forward}

Different versions of the main or child documents
using compilation flags as described in \secref{sec:flags}
can be (permanently) stored in different files
for convenient compilation, viewing and distribution.
To this end, the package defines a command
to pass on compilation to a different file:

%%%%%%%%%%%%%%%%%%%%%%%%%%%%%%%%%%%%%%%%
\DescribeMacro{\childdocforward}
The command |\childdocforward| redirects processing to
another source file:
%
\begin{center}
\begin{tabular}{l}
|\input{childdoc.def}|\\
|\childdocforward[|\textit{main}|]{|\textit{dest}|}|\\
\end{tabular}
\end{center}
%
The argument \textit{dest} is the destination file
(without extension).
It should be the main file or one of the child files.
Note that further \textsf{childdoc} directives
such as |\childdocof| and |\childdocforward|
in the indicated file will be processed in this form.
The optional argument \textit{main}
passes on directly to the main file \textit{main}
while pretending to compile the child \textit{dest}.
This form behaves as if \textit{dest}
issues |\childdocof{|\textit{main}|}| right away,
and no further \textsf{childdoc} directives will be processed.

%%%%%%%%%%%%%%%%%%%%%%%%%%%%%%%%%%%%%%%%
\DescribeMacro{\...prefix}
In the alternative form |\childdocforwardprefix|,
%
\begin{center}
\begin{tabular}{l}
|\input{childdoc.def}|\\
|\childdocforwardprefix[|\textit{main}|]{|\textit{prefix}|}{|\textit{dest}|}|
\end{tabular}
\end{center}
%
the destination file is determined by a pattern
depending on the current file:
To make this work, the current file must be called
`{\textit{prefix}\hspace{0.2em}\textit{suffix}}'
with \textit{prefix} matching precisely the argument.
Processing is then passed on to the file
`{\textit{dest}\hspace{0.2em}\textit{suffix}}'.
Surely, the same effect is achieved by
directly specifying the
argument `{\textit{dest}\hspace{0.2em}\textit{suffix}}'
in the first form.
However, that requires to set up a different file
for each child. With the alternative form of the command
all these files can have exactly the same content
which simplifies setting them up and maintaining them.

For example, the following file |draft.tex|
with a compilation flag |\version| as described in \secref{sec:flags}
compiles the main document as a draft:
%
\begin{center}
\begin{tabular}{l}
|\def\version{draft}|\\
|\input{childdoc.def}|\\
|\childdocforward{|\textit{main}|}|
\end{tabular}
\end{center}
%
Likewise, the following files |final|\textit{nn}|.tex|
compile the final version of the child document
|child|\textit{nn}|.tex|:
%
\begin{center}
\begin{tabular}{l}
|\def\version{final}|\\
|\input{childdoc.def}|\\
|\childdocforwardprefix{final}{child}|
\end{tabular}
\end{center}
%

Note that when several versions of a main file and/or of each child file
are to be generated, it may be convenient to set up a |Makefile| or
shell script to automatise the process.

%%%%%%%%%%%%%%%%%%%%%%%%%%%%%%%%%%%%%%%%%%%%%%%%%%%%%%%%%%%%%%%%%%%%%%%%%%%%%%%%
\subsection{Command Line Processing}
\label{sec:commandline}

The effect of redirection files can also be achieved by invoking
the \LaTeX{} compiler with a more elaborate command line.
Most conveniently this should be done as part
of a shell script or a |Makefile|.

When using \textsf{childdoc} in the main file, the following
command lines effectively perform a redirection
(note that depending on the shell being used,
backslashes may have to be doubled: `|\|' $\to$ `|\\|'):
%
\begin{center}
|... -jobname "|\textit{target}|" |\\|"|[\textit{flags}]%
|\input{childdoc.def}\childdocforward[|\textit{main}|]{|\textit{dest}|}"|
\end{center}
%
Here \textit{target} is the name of the output file,
\textit{main} is the name of the main file
and \textit{dest} is the name of the main or child file to be processed
(all filenames without extensions).
The optional argument \textit{main} can be omitted
if \textit{main} matches \textit{dest}.
Optionally, compilation \textit{flags} can be defined via |\def| commands.
This command line makes the \TeX{} engine believe
it is compiling the file \textit{target}
whose content is specified as the latter parameter.
The provided code then forwards the processing to
\textit{main} or \textit{dest} as described in \secref{sec:forward}.

%%%%%%%%%%%%%%%%%%%%%%%%%%%%%%%%%%%%%%%%%%%%%%%%%%%%%%%%%%%%%%%%%%%%%%%%%%%%%%%%
\subsection{Include by Input}
\label{sec:input}

Including child documents by |\include| has some restrictions by design.
Most notably, the content of a child document always occupies
its own set of pages; pages cannot be shared between child documents.
Usually, this behaviour makes perfect sense
because each child document contain an essential part of the document.
However, in some situations it may be desirable to compose
a document from a collection of parts
without having mandatory page breaks between then.
For this case, the package
provides a mechanism to include parts
by |\input| which can also be processed individually.
However, by construction this mechanism
requires manual handling of the content to be output.

%%%%%%%%%%%%%%%%%%%%%%%%%%%%%%%%%%%%%%%%
\DescribeMacro{\ifchilddocmanual}
The main file should be prepared as usual, see \secref{sec:include}.
However, the document body must make a distinction
between processing of an individual part and of the main document, e.g.:
%
\begin{center}
\begin{tabular}{l}
|\ifchilddocmanual|\\
|\input{\childdocname}|\\
|\||else|\\
\textit{document body with }|\input{|\textit{part}|}|\\
|\||fi|
\end{tabular}
\end{center}
%
The conditional |\ifchilddocmanual| is true whenever
a part to be included by |\input| is being compiled,
and the name of the part is stored in |\childdocname|.

%%%%%%%%%%%%%%%%%%%%%%%%%%%%%%%%%%%%%%%%
\DescribeMacro{\childdocby}
Each part to be included by |\input| should start with:
%
\begin{center}
\begin{tabular}{l}
|\input{childdoc.def}|\\
|\childdocby{|\textit{main}|}|\\
\end{tabular}
\end{center}
%
The directive |\childdocby| is similar to |\childdocof|
described in \secref{sec:include},
but the subsequent selection of content must be done manually.
To that end, both |\ifchilddoc| and |\ifchilddocmanual|
will be true upon processing of a part,
and the name of the part is stored in |\childdocname|.
Note that |\jobname| will be set to the filename of the current part
so that each part receives an individual |.aux| file
that does not interfere with the |.aux| file(s) of the main document.
This behaviour can be altered by the alternative form
|\childdocby[*]{|\textit{main}|}| (with a non-empty optional argument)
which uses the |.aux| file of the main document
by setting |\jobname| to \textit{main}.

%%%%%%%%%%%%%%%%%%%%%%%%%%%%%%%%%%%%%%%%%%%%%%%%%%%%%%%%%%%%%%%%%%%%%%%%%%%%%%%%
\subsection{Driver Development}
\label{sec:driver}

The \textsf{childdoc} mechanism can also be use for the development
of definition files such as \LaTeX{} styles or classes.
This case differs from the above setup with multiple parts
included by |\include| in that no |\includeonly| should be invoked.
This can be achieved by starting the include file
(before |\ProvidesPackage|) with:
%
\begin{center}
\begin{tabular}{l}
|\input{childdoc.def}|\\
|\childdocforward{|\textit{main}|}|\\
\end{tabular}
\end{center}
%
or alternatively with:
%
\begin{center}
\begin{tabular}{l}
|\input{childdoc.def}|\\
|\childdocby{|\textit{main}|}|\\
\end{tabular}
\end{center}
%
Both forms have slightly different effects as described above.
The main file is prepared as usual, see \secref{sec:include}.

%%%%%%%%%%%%%%%%%%%%%%%%%%%%%%%%%%%%%%%%%%%%%%%%%%%%%%%%%%%%%%%%%%%%%%%%%%%%%%%%
\subsection{Legacy Detection}
\label{sec:detection}

The directive |\childdocmain| in the main file can detect
whether the complete document or merely a child is to be compiled
even without using the directive |\childdocof|.
This method is deprecated because it is less robust
and there is no compelling reason to use it;
it is merely provided for backward compatibility
and it may be removed in future versions.

If the detection mechanism is to be used,
it is mandatory to correctly specify
the filename of the main file as the argument of |\childdocmain|:
%
\begin{center}
\begin{tabular}{l}
|\input{childdoc.def}|\\
|\childdocmain{|\textit{main}|}|\\
\end{tabular}
\end{center}
%
If |\jobname| does not match the argument \textit{main} of |\childdocmain|,
it is assumed that |\jobname| points to the child file to be compiled.
When using |\childdocmain| with the main file specified as argument,
it suffices to start a child file
with just |\input{|\textit{main}|}|
without loading of the package and using |\childdocof|.
If instead all processing is done
with the appropriate \textsf{childdoc} directives,
the argument of \textit{main} of |\childdocmain| can be empty.

An alternative version of the command line processing described
in \secref{sec:commandline} using the detection mechanism reads:
%
\begin{center}
|... -jobname "|\textit{target}|" "|[\textit{flags}]%
[|\def\jobname{|\textit{dest}|}|]|\input{|\textit{main}|}"|
\end{center}

%%%%%%%%%%%%%%%%%%%%%%%%%%%%%%%%%%%%%%%%%%%%%%%%%%%%%%%%%%%%%%%%%%%%%%%%%%%%%%%%
\subsection{Manual Code}
\label{sec:manual}

In case one cannot be certain whether the definitions file |childdoc.def|
is installed on the target \TeX{} distribution
and one prefers not to ship it,
it is conceivable to paste a few relevant commands into the sources.

To that end, drop all statements |\input{childdoc.def}|
and perform the replacements as outlined below.
Instead of |\childdocmain{|\textit{main}|}| add the following code
to the top of the main file:
%
\begin{center}
\begin{tabular}{l}
|\||ifdefined\childdocname\endinput\||fi\newif\ifchilddoc|\\
|\edef\childdocname{\scantokens\expandafter{\jobname\noexpand}}|\\
|\def\childdocmain{|\textit{main}|}\||ifx\childdocmain\childdocname\||else|\\
|\childdoctrue\includeonly{\childdocname}\let\jobname\childdocmain\||fi|\\
\end{tabular}
\end{center}
%
Instead of |\childdocof{|\textit{main}|}| just include the main file
at the top of each child file:
%
\begin{center}
|\input{|\textit{main}|}|
\end{center}
%
A simple redirection |\childdocforward{|\textit{dest}|}| is achieved by:
%
\begin{center}
|\def\jobname{|\textit{dest}|}\input{\jobname}|
\end{center}
%
The redirection with prefix
|\childdocforwardprefix[|\textit{prefix}|]{|\textit{dest}|}|
is accomplished by:
%
\begin{center}
\begin{tabular}{l}
|{\edef\jobname{\scantokens\expandafter{\jobname\noexpand}}|\\
|\def\redirectjob |\textit{prefix}|#1~~~{\gdef\jobname{|\textit{dest}|#1}}|\\
|\expandafter\redirectjob\jobname~~~}\input{\jobname}|
\end{tabular}
\end{center}

In an alternative approach,
child documents can be compiled by a specific command line
without additional code or specific definitions:
%
\begin{center}
|... -jobname "|\textit{target}|" "|[\textit{flags}]%
|\includeonly{|\textit{dest}|}\input{|\textit{main}|}"|
\end{center}
%

%%%%%%%%%%%%%%%%%%%%%%%%%%%%%%%%%%%%%%%%%%%%%%%%%%%%%%%%%%%%%%%%%%%%%%%%%%%%%%%%
%%%%%%%%%%%%%%%%%%%%%%%%%%%%%%%%%%%%%%%%%%%%%%%%%%%%%%%%%%%%%%%%%%%%%%%%%%%%%%%%
\section{Information}

%%%%%%%%%%%%%%%%%%%%%%%%%%%%%%%%%%%%%%%%%%%%%%%%%%%%%%%%%%%%%%%%%%%%%%%%%%%%%%%%
\subsection{Copyright}

Copyright \copyright{} 2017--2018 Niklas Beisert

This work may be distributed and/or modified under the
conditions of the \LaTeX{} Project Public License, either version 1.3
of this license or (at your option) any later version.
The latest version of this license is in
  \url{http://www.latex-project.org/lppl.txt}
and version 1.3 or later is part of all distributions of \LaTeX{}
version 2005/12/01 or later.

This work has the LPPL maintenance status `maintained'.

The Current Maintainer of this work is Niklas Beisert.

This work consists of the files |README.txt|, |childdoc.ins| and |childdoc.dtx|
as well as the derived files |childdoc.def|, |cdocsamp.tex|
with |cdocsch1.tex|, |cdocsch2.tex|, |cdocspt3.tex|, |cdocspt4.tex|,
|cdocsdrf.tex|, |cdocsfn1.tex|, |cdocsfn2.tex|
as well as |childdoc.pdf|.

%%%%%%%%%%%%%%%%%%%%%%%%%%%%%%%%%%%%%%%%%%%%%%%%%%%%%%%%%%%%%%%%%%%%%%%%%%%%%%%%
\subsection{Files and Installation}

The package consists of the files:
%
\begin{center}
\begin{tabular}{ll}
    |README.txt|   & readme file \\
    |childdoc.ins| & installation file \\
    |childdoc.dtx| & source file \\
    |childdoc.def| & definition file \\
    |cdocsamp.tex| & sample main file \\
    |cdocsch1.tex| & sample include file \\
    |cdocsch2.tex| & sample include file \\
    |cdocspt3.tex| & sample part file \\
    |cdocspt4.tex| & sample part file \\
    |cdocsdrf.tex| & sample redirection file \\
    |cdocsfn1.tex| & sample redirection file \\
    |cdocsfn2.tex| & sample redirection file \\
    |childdoc.pdf| & manual
\end{tabular}
\end{center}
%
The distribution consists of the files
|README.txt|, |childdoc.ins| and |childdoc.dtx|.
%
\begin{itemize}
\item
Run (pdf)\LaTeX{} on |childdoc.dtx|
to compile the manual |childdoc.pdf| (this file).
\item
Run \LaTeX{} on |childdoc.ins| to create the definitions file |childdoc.def|
and the sample |cdocsamp.tex| with include files
|cdocsch1.tex|, |cdocsch2.tex|, |cdocspt3.tex|, |cdocspt4.tex|,
|cdocsdrf.tex|, |cdocsfn1.tex|, |cdocsfn2.tex|.
Then copy the file |childdoc.def| to an appropriate directory of your \LaTeX{}
distribution, e.g.\ \textit{texmf-root}|/tex/latex/childdoc|.
\end{itemize}

%%%%%%%%%%%%%%%%%%%%%%%%%%%%%%%%%%%%%%%%%%%%%%%%%%%%%%%%%%%%%%%%%%%%%%%%%%%%%%%%
\subsection{Related CTAN Packages}

There are several other packages which offer a similar functionality:
%
\begin{itemize}
\item
The packages
\href{http://ctan.org/pkg/docmute}{\textsf{docmute}},
\href{http://ctan.org/pkg/includex}{\textsf{includex}} and
\href{http://ctan.org/pkg/standalone}{\textsf{standalone}}
provide commands to include only the document body of
a child file thus allowing both files to be compiled individually.
\item
The packages \href{http://ctan.org/pkg/subdocs}{\textsf{subdocs}}
and \href{http://ctan.org/pkg/subfiles}{\textsf{subfiles}}
provide structures in which the main and child documents can be
encapsulated and allowing them to be compiled individually.
The inclusion mechanism is different from the conventional |\include|.
\item
The package \href{http://ctan.org/pkg/combine}{\textsf{combine}}
is an elaborate solution to combine several documents into one.
\end{itemize}
%
See also the CTAN topic \href{http://ctan.org/topic/subdocs}{\textsf{subdocs}}
for further related packages.
The present package differs from the above solutions in that
a document structure constructed with the conventional |\include| mechanism
just needs two extra commands at the top of every file
such that all constituent files can be compiled individually.

%%%%%%%%%%%%%%%%%%%%%%%%%%%%%%%%%%%%%%%%%%%%%%%%%%%%%%%%%%%%%%%%%%%%%%%%%%%%%%%%
%\subsection{Feature Suggestions}
%
%The following is a list of features which may be useful for future
%versions of this package:
%%
%\begin{itemize}
%\item
%\ldots
%\end{itemize}

%%%%%%%%%%%%%%%%%%%%%%%%%%%%%%%%%%%%%%%%%%%%%%%%%%%%%%%%%%%%%%%%%%%%%%%%%%%%%%%%
\subsection{Revision History}

%%%%%%%%%%%%%%%%%%%%%%%%%%%%%%%%%%%%%%%%
\paragraph{v2.0:} 2018/12/30

\begin{itemize}
\item
immediate forward processing
\item
added |\childdocby| mechanism
\item
manual restructured
\end{itemize}

%%%%%%%%%%%%%%%%%%%%%%%%%%%%%%%%%%%%%%%%
\paragraph{v1.6:} 2018/01/17

\begin{itemize}
\item
application for development of include files
\item
corrections to manual
\end{itemize}

%%%%%%%%%%%%%%%%%%%%%%%%%%%%%%%%%%%%%%%%
\paragraph{v1.5:} 2017/05/21

\begin{itemize}
\item
more complete structuring introduced
\item
|\childdocof| introduced
\item
|\childdoc| renamed to |\childdocmain|
\item
|\childredirect| renamed to |\childdocforward| and |\childdocforwardprefix|
and functionality expanded
\end{itemize}

%%%%%%%%%%%%%%%%%%%%%%%%%%%%%%%%%%%%%%%%
\paragraph{v1.0:} 2017/04/27

\begin{itemize}
\item
manual and install package
\item
first version published on CTAN
\end{itemize}

%%%%%%%%%%%%%%%%%%%%%%%%%%%%%%%%%%%%%%%%
\paragraph{v0.6:} 2017/04/26

\begin{itemize}
\item
redirection mechanism added
\end{itemize}

%%%%%%%%%%%%%%%%%%%%%%%%%%%%%%%%%%%%%%%%
\paragraph{v0.5:} 2017/04/26

\begin{itemize}
\item
functionality in definition file
\end{itemize}


%%%%%%%%%%%%%%%%%%%%%%%%%%%%%%%%%%%%%%%%%%%%%%%%%%%%%%%%%%%%%%%%%%%%%%%%%%%%%%%%
%%%%%%%%%%%%%%%%%%%%%%%%%%%%%%%%%%%%%%%%%%%%%%%%%%%%%%%%%%%%%%%%%%%%%%%%%%%%%%%%
%%%%%%%%%%%%%%%%%%%%%%%%%%%%%%%%%%%%%%%%%%%%%%%%%%%%%%%%%%%%%%%%%%%%%%%%%%%%%%%%
\appendix

\settowidth\MacroIndent{\rmfamily\scriptsize 000\ }

 \DocInput{childdoc.dtx}

\end{document}
%</driver>
% \fi
%
% %%%%%%%%%%%%%%%%%%%%%%%%%%%%%%%%%%%%%%%%%%%%%%%%%%%%%%%%%%%%%%%%%%%%%%%%%%%%%%
% %%%%%%%%%%%%%%%%%%%%%%%%%%%%%%%%%%%%%%%%%%%%%%%%%%%%%%%%%%%%%%%%%%%%%%%%%%%%%%
% \section{Sample}
%\iffalse
%<*samplemain>
%\fi
%
% The following presents a sample document
% with two chapters, two parts, a title page,
% a compile flag as well as three forwarding files to set the flag.
% It consists of eight |.tex| files:
% \begin{center}
% \begin{tabular}{ll}
% |cdocsamp.tex|&main file\\
% |cdocsch1.tex|&include file for chapter 1\\
% |cdocsch2.tex|&include file for chapter 2\\
% |cdocspt3.tex|&include file for part 3\\
% |cdocspt4.tex|&include file for part 4\\
% |cdocsdrf.tex|&forwarding file for main file in draft mode\\
% |cdocsfi1.tex|&forwarding file for final version of chapter 1\\
% |cdocsfi2.tex|&forwarding file for final version of chapter 2\\
% \end{tabular}
% \end{center}
% Each of the eight files can be compiled directly by the \LaTeX{} compiler.
%
% %%%%%%%%%%%%%%%%%%%%%%%%%%%%%%%%%%%%%%
% \paragraph{Main File.}
%
% The main file is called |cdocsamp.tex|.
%
% Load the \textsf{childdoc} definitions and
% declare the filename for the main document:
%    \begin{macrocode}
\input{childdoc.def}
\childdocmain{}
%    \end{macrocode}

% Optional override for |\version| flag:
%    \begin{macrocode}
%%\ifchilddoc\else\providecommand{\version}{draft}\fi
%    \end{macrocode}

% Define the default values for the |\version| flag
% (|final| for the main file and |draft| for childs):
%    \begin{macrocode}
\ifchilddoc
\providecommand{\version}{draft}
\else
\providecommand{\version}{final}
\fi
%    \end{macrocode}

% Load the standard document class:
%    \begin{macrocode}
\documentclass[12pt]{article}
%    \end{macrocode}

% Start the document body:
%    \begin{macrocode}
\begin{document}
%    \end{macrocode}

% Declare a title page.
% Print title, part of document being processed and version flag:
%    \begin{macrocode}
\addtocounter{page}{-1}
\begin{center}
{\LARGE\bfseries{}childdoc example\par}
\vspace{1cm}
\ifchilddoc
\ifchilddocmanual part\else chapter\fi:
`\childdocname' of `\childdocjob'\par
\else
main document: `\childdocjob'\par
\fi
version: \version\par
\end{center}
\newpage
%    \end{macrocode}

% Manually include selected file,
% otherwise process as usual:
%    \begin{macrocode}
\ifchilddocmanual
\section*{part `\childdocname'}
\input{\childdocname}
\else
%    \end{macrocode}

% Include the two chapters:
%    \begin{macrocode}
\include{cdocsch1}
\include{cdocsch2}
%    \end{macrocode}

% Include the two parts unless only chapters should be displayed:
%    \begin{macrocode}
\ifchilddoc\else
\section{part three}
\input{cdocspt3}
\section{part four}
\input{cdocspt4}
\fi
%    \end{macrocode}

% Process as usual until here:
%    \begin{macrocode}
\fi
%    \end{macrocode}

% End of document body:
%    \begin{macrocode}
\end{document}
%    \end{macrocode}
%\iffalse
%</samplemain>
%\fi
%
% %%%%%%%%%%%%%%%%%%%%%%%%%%%%%%%%%%%%%%
% \paragraph{Chapter Include Files.}
%
% The include files are called |cdocsch1.tex| and |cdocsch2.tex|.
%
%\iffalse
%<*samplechap1|samplechap2>
%\fi

% Optional override for |\version| flag:
%    \begin{macrocode}
%%\providecommand{\version}{final}
%    \end{macrocode}

% Include the main document:
%    \begin{macrocode}
\input{childdoc.def}
\childdocof{cdocsamp}
%    \end{macrocode}

%\iffalse
%</samplechap1|samplechap2>
%\fi
%
%\iffalse
%<*samplechap1>
%\fi
% Some text for chapter 1:
%    \begin{macrocode}
\section{one}
some text in chapter one
%    \end{macrocode}

%\iffalse
%</samplechap1>
%\fi
% Some text for chapter 2:
%\iffalse
%<*samplechap2>
%\fi
%    \begin{macrocode}
\section{two}
more text in chapter two
%    \end{macrocode}

%\iffalse
%</samplechap2>
%\fi
%
% %%%%%%%%%%%%%%%%%%%%%%%%%%%%%%%%%%%%%%
% \paragraph{Part Include Files.}
%
% The include files are called |cdocspt3.tex| and |cdocspt4.tex|.
%
%\iffalse
%<*samplepart3|samplepart4>
%\fi

% Optional override for |\version| flag:
%    \begin{macrocode}
%%\providecommand{\version}{final}
%    \end{macrocode}

% Include the main document:
%    \begin{macrocode}
\input{childdoc.def}
\childdocby{cdocsamp}
%    \end{macrocode}

%\iffalse
%</samplepart3|samplepart4>
%\fi
%
%\iffalse
%<*samplepart3>
%\fi
% Some text for part 3:
%    \begin{macrocode}
some text in part three
%    \end{macrocode}

%\iffalse
%</samplepart3>
%\fi
% Some text for part 4:
%\iffalse
%<*samplepart4>
%\fi
%    \begin{macrocode}
more text in part four
%    \end{macrocode}

%\iffalse
%</samplepart4>
%\fi
%
% %%%%%%%%%%%%%%%%%%%%%%%%%%%%%%%%%%%%%%
% \paragraph{Forwarding for a Complete Draft.}
%
% The following forwarding file |cdocsdrf.tex|
% compiles the main document in draft mode:
%\iffalse
%<*sampledraft>
%\fi
%    \begin{macrocode}
\def\version{draft}
\input{childdoc.def}
\childdocforward{cdocsamp}
%    \end{macrocode}

%\iffalse
%</sampledraft>
%\fi
%
% %%%%%%%%%%%%%%%%%%%%%%%%%%%%%%%%%%%%%%
% \paragraph{Forwarding for Final Version of the Chapters.}
%
% The following forwarding files |cdocsfn1.tex| and |cdocsfn2.tex|
% (with identical content)
% compile the final versions of the child documents
% |cdocsch1.tex| and |cdocsch2.tex|, respectively:
%\iffalse
%<*samplefinal>
%\fi
%    \begin{macrocode}
\def\version{final}
\input{childdoc.def}
\childdocforwardprefix[cdocsamp]{cdocsfn}{cdocsch}
%    \end{macrocode}

%\iffalse
%</samplefinal>
%\fi
%
% %%%%%%%%%%%%%%%%%%%%%%%%%%%%%%%%%%%%%%
% \paragraph{Command Line Processing.}
%
% The following three command lines generate the output files
% |cdocscld|, |cdocscl1| and |cdocscl2|
% which should be identical to
% |cdocsdrf|, |cdocsch1| and |cdocsfn2|, respectively:
% \begin{center}
% \begin{tabular}{l}
% |latex -jobname cdocscld \|\\
% |  "\def\version{draft}\input{childdoc.def}\childdocforward{cdocsamp}"|\\
% |latex -jobname cdocscl1 \|\\
% |  "\input{childdoc.def}\childdocforward[cdocsamp]{cdocsch1}"|\\
% |latex -jobname cdocscl2 \|\\
% |  "\def\version{final}\input{childdoc.def}\childdocforward{cdocsch2}"|
% \end{tabular}
% \end{center}
% Note that the trailing backslash on each first line
% merely continues the input to the second line
% (for convenient cut ant paste).
% Furthermore, the command |latex| can be replaced by any
% of its alternative versions such as |pdflatex|.
%
% %%%%%%%%%%%%%%%%%%%%%%%%%%%%%%%%%%%%%%%%%%%%%%%%%%%%%%%%%%%%%%%%%%%%%%%%%%%%%%
% %%%%%%%%%%%%%%%%%%%%%%%%%%%%%%%%%%%%%%%%%%%%%%%%%%%%%%%%%%%%%%%%%%%%%%%%%%%%%%
% \section{Implementation}
%\iffalse
%<*package>
%\fi
%
% This section describes the definitions file |childdoc.def|.

% The definitions cannot be loaded using |\usepackage| or |\RequirePackage|
% which has a mechanism to prevent loading a style file more than once.
% When loading the definitions by means of |\input|
% multiple instances have to be prevented manually:
%\iffalse
%This code needs to be before the `\ProvidesFile' directive
%which is defined at the beginning of this file.
%Therefore it is also placed there and commented out here.
%</package>
%<*discard>
%\fi
%    \begin{macrocode}
\ifdefined\childdocmain\endinput\fi
%    \end{macrocode}
%\iffalse
%</discard>
%<*package>
%\fi
%
% \macro{\ifchilddoc}
% \macro{\ifchilddocmanual}
% The conditional |\ifchilddoc| tells whether a
% child (true) or main (false) document is being compiled.
% The conditional |\ifchilddocmanual| tells whether
% the |\includeonly| mechanism is used (false) or
% the selection of child files must be performed manually (true).
% The definitions initialise to false:
%    \begin{macrocode}
\newif\ifchilddoc
\newif\ifchilddocmanual
%    \end{macrocode}

% \macro{\childdocname}
% \macro{\childdocjob}
% The macro |\childdocname| stores the name of the main document
% to be compiled. The macro |\childdocjob| stores the name of
% the document on which the \LaTeX{} compiler was originally invoked.
% The content of |\jobname| cannot be compared
% to filenames specified in the source due to different catcodes.
% The following code rescans |\jobname|, stores the result
% in |\childdocname| and saves a copy in |\childdocjob|:
%    \begin{macrocode}
\edef\childdocname{\scantokens\expandafter{\jobname\noexpand}}
\let\childdocjob\childdocname
%    \end{macrocode}

% \macro{\childdocdisable}
% The macro |\childdocdisable| prevents the main file
% from being processed more than once.
% At this stage, the main document command |\childdocmain|
% is assumed to be called once again where it should do nothing.
% Any subsequent call to it should prevent
% a secondary processing of the main document
% It overwrites the forwarding commands
% |\childdocof| and |\childdocforward|
% with empty macros to prevent further inclusions of the main document:
%    \begin{macrocode}
\newcommand{\childdocdisable}
{
  \renewcommand{\childdocmain}[1]{\renewcommand{\childdocmain}[1]{\endinput}}
  \renewcommand{\childdocof}[1]{}
  \renewcommand{\childdocby}[2][]{}
  \renewcommand{\childdocforward}[2][]{}
  \renewcommand{\childdocdisable}{}
}
%    \end{macrocode}

% \macro{\childdocmain}
% The macro |\childdocmain| is to be called at the top of the main file
% with nothing or the main filename (without extension) as argument.
% First, it breaks loops.
% If the argument is not empty and does not match |\childdocname|
% (which is set by the first inclusion of |childdoc.def|),
% |\ifchilddoc| is set to true, |\includeonly| is applied to the child file
% and |\jobname| is set to the main file
% (for proper handling of |.aux| files):
%    \begin{macrocode}
\newcommand{\childdocmain}[1]
{
  \childdocdisable\childdocmain{}
  \if?#1?\else
    \begingroup
      \def\childdoctmp{#1}
      \ifx\childdoctmp\childdocname
        \def\childdoctmp{}
      \else
        \def\childdoctmp
        {
          \childdoctrue
          \includeonly{\childdocname}
          \def\childdocjob{#1}
          \def\jobname{#1}
        }
      \fi
      \expandafter
    \endgroup
    \childdoctmp
  \fi
}
%    \end{macrocode}

% \macro{\childdocof}
% The command |\childdocof| redirects
% compilation to the main file |#1|.
%    \begin{macrocode}
\newcommand{\childdocof}[1]
{
  \childdocdisable
  \childdoctrue
  \includeonly{\childdocname}
  \def\jobname{#1}
  \def\childdocjob{#1}
  \input{#1}
}
%    \end{macrocode}

% \macro{\childdocby}
% The command |\childdocby| ....
%    \begin{macrocode}
\newcommand{\childdocby}[2][]
{
  \childdocdisable
  \childdoctrue
  \childdocmanualtrue
  \if?#1?\else
    \def\jobname{#2}
  \fi
  \def\childdocjob{#2}
  \input{#2}
  \endinput
}
%    \end{macrocode}

% \macro{\childdocforward}
% The command |\childdocforward| redirects
% compilation to the main file or
% (if the optional argument is given) a child file.
% Parameters are set as if the main file
% or a child file starting with |\childdocof| was compiled.
% Then compilation is handed over to the main file:
%    \begin{macrocode}
\newcommand{\childdocforward}[2][]
{
  \begingroup
    \if?#1?
      \def\childdoctmp
      {
        \def\childdocname{#2}
        \def\childdocjob{#2}
        \def\jobname{#2}
        \input{#2}
        \endinput
      }
    \else
      \def\childdoctmp
      {
        \childdocdisable
        \def\childdocname{#2}
        \childdoctrue
        \includeonly{#2}
        \def\childdocjob{#1}
        \def\jobname{#1}
        \input{#1}
        \endinput
      }
    \fi
    \expandafter
  \endgroup
  \childdoctmp
}
%    \end{macrocode}

% \macro{\childdocforwardprefix}
% The command |\childdocforwardprefix| redirects
% compilation to the main or a child file by means of a pattern.
% The prefix |#1| in the current filename is replaced by |#2|
% and the suffix of the current filename is kept
% (it is assumed that the filename does not contain the substring `|~~~|'
% which is used as a delimiter).
% Compilation is handed over to the new file by |\childdocforward|:
%    \begin{macrocode}
\newcommand{\childdocforwardprefix}[3][]
{
  \begingroup
    \def\childdocextract #2##1~~~{\def\childdoctmp{\childdocforward[#1]{#3##1}}}
    \expandafter\childdocextract\childdocname~~~
    \expandafter
  \endgroup
  \childdoctmp
}
%    \end{macrocode}

% \macro{\childdoc}
% The deprecated macro |\childdoc| is a legacy version of |\childdocmain|:
%    \begin{macrocode}
\newcommand{\childdoc}{\childdocmain}
%    \end{macrocode}

% \macro{\childdocredirect}
% The deprecated macro |\childdocredirect| is a legacy version
% of |\childdocforward| and |\childdocforwardprefix|:
%    \begin{macrocode}
\newcommand{\childdocredirect}[2][]
{
  \begingroup
    \if?#1?
      \def\childdoctmp{\childdocforward{#2}}
    \else
      \def\childdoctmp{\childdocforwardprefix{#1}{#2}}
    \fi
    \expandafter
  \endgroup
  \childdoctmp
}
%    \end{macrocode}

%\iffalse
%</package>
%\fi
%
\endinput
\childdocforward[|\textit{main}|]{|\textit{dest}|}"|
\end{center}
%
Here \textit{target} is the name of the output file,
\textit{main} is the name of the main file
and \textit{dest} is the name of the main or child file to be processed
(all filenames without extensions).
The optional argument \textit{main} can be omitted
if \textit{main} matches \textit{dest}.
Optionally, compilation \textit{flags} can be defined via |\def| commands.
This command line makes the \TeX{} engine believe
it is compiling the file \textit{target}
whose content is specified as the latter parameter.
The provided code then forwards the processing to
\textit{main} or \textit{dest} as described in \secref{sec:forward}.

%%%%%%%%%%%%%%%%%%%%%%%%%%%%%%%%%%%%%%%%%%%%%%%%%%%%%%%%%%%%%%%%%%%%%%%%%%%%%%%%
\subsection{Include by Input}
\label{sec:input}

Including child documents by |\include| has some restrictions by design.
Most notably, the content of a child document always occupies
its own set of pages; pages cannot be shared between child documents.
Usually, this behaviour makes perfect sense
because each child document contain an essential part of the document.
However, in some situations it may be desirable to compose
a document from a collection of parts
without having mandatory page breaks between then.
For this case, the package
provides a mechanism to include parts
by |\input| which can also be processed individually.
However, by construction this mechanism
requires manual handling of the content to be output.

%%%%%%%%%%%%%%%%%%%%%%%%%%%%%%%%%%%%%%%%
\DescribeMacro{\ifchilddocmanual}
The main file should be prepared as usual, see \secref{sec:include}.
However, the document body must make a distinction
between processing of an individual part and of the main document, e.g.:
%
\begin{center}
\begin{tabular}{l}
|\ifchilddocmanual|\\
|\input{\childdocname}|\\
|\||else|\\
\textit{document body with }|\input{|\textit{part}|}|\\
|\||fi|
\end{tabular}
\end{center}
%
The conditional |\ifchilddocmanual| is true whenever
a part to be included by |\input| is being compiled,
and the name of the part is stored in |\childdocname|.

%%%%%%%%%%%%%%%%%%%%%%%%%%%%%%%%%%%%%%%%
\DescribeMacro{\childdocby}
Each part to be included by |\input| should start with:
%
\begin{center}
\begin{tabular}{l}
|% \iffalse
%
% childdoc.dtx Copyright (C) 2017-2018 Niklas Beisert
%
% This work may be distributed and/or modified under the
% conditions of the LaTeX Project Public License, either version 1.3
% of this license or (at your option) any later version.
% The latest version of this license is in
%   http://www.latex-project.org/lppl.txt
% and version 1.3 or later is part of all distributions of LaTeX
% version 2005/12/01 or later.
%
% This work has the LPPL maintenance status `maintained'.
%
% The Current Maintainer of this work is Niklas Beisert.
%
% This work consists of the files childdoc.dtx and childdoc.ins
% and the derived files childdoc.def and cdocsamp.tex with
% cdocsch1.tex, cdocsch2.tex, cdocsdrf.tex, cdocsfn1.tex, cdocsfn2.tex.
%
%<package>\ifdefined\childdocmain\endinput\fi
%<package>\ProvidesFile{childdoc.def}[2018/12/30 v2.0 child document driver]
%<samplemain>\ProvidesFile{cdocsamp.tex}[2018/12/30 v2.0 sample for childdoc]
%<*driver>
%\ProvidesFile{childdoc.drv}[2018/12/30 v2.0 childdoc reference manual file]
\PassOptionsToClass{10pt,a4paper}{article}
\documentclass{ltxdoc}

\usepackage[margin=35mm]{geometry}
\usepackage{hyperref}
\usepackage{hyperxmp}
\usepackage[usenames]{color}

\hypersetup{colorlinks=true}
\hypersetup{pdfstartview=FitH}
\hypersetup{pdfpagemode=UseNone}
\hypersetup{pdfsource={}}
\hypersetup{pdflang={en-UK}}
\hypersetup{pdfcopyright={Copyright 2017-2018 Niklas Beisert.
  This work may be distributed and/or modified under the
  conditions of the LaTeX Project Public License, either version 1.3
  of this license or (at your option) any later version.}}
\hypersetup{pdflicenseurl={http://www.latex-project.org/lppl.txt}}
\hypersetup{pdfcontactaddress={ETH Zurich, ITP, HIT K,
  Wolfgang-Pauli-Strasse 27}}
\hypersetup{pdfcontactpostcode={8093}}
\hypersetup{pdfcontactcity={Zurich}}
\hypersetup{pdfcontactcountry={Switzerland}}
\hypersetup{pdfcontactemail={nbeisert@itp.phys.ethz.ch}}
\hypersetup{pdfcontacturl={http://people.phys.ethz.ch/\xmptilde nbeisert/}}

\newcommand{\secref}[1]{\hyperref[#1]{section \ref*{#1}}}

\parskip1ex
\parindent0pt
\let\olditemize\itemize
\def\itemize{\olditemize\parskip0pt}

\begin{document}

\title{The \textsf{childdoc} Package}
\hypersetup{pdftitle={The childdoc Package}}
\author{Niklas Beisert\\[2ex]
  Institut f\"ur Theoretische Physik\\
  Eidgen\"ossische Technische Hochschule Z\"urich\\
  Wolfgang-Pauli-Strasse 27, 8093 Z\"urich, Switzerland\\[1ex]
  \href{mailto:nbeisert@itp.phys.ethz.ch}
  {\texttt{nbeisert@itp.phys.ethz.ch}}}
\hypersetup{pdfauthor={Niklas Beisert}}
\hypersetup{pdfsubject={Manual for the LaTeX2e Package childdoc}}
\date{30 December 2018, \textsf{v2.0}}
\maketitle

\begin{abstract}\noindent
\textsf{childdoc} is a \LaTeXe{} package
that enables the direct compilation
of document sections included by |\include|
to individual files.
\end{abstract}

\begingroup
\parskip0ex
\tableofcontents
\endgroup

%%%%%%%%%%%%%%%%%%%%%%%%%%%%%%%%%%%%%%%%%%%%%%%%%%%%%%%%%%%%%%%%%%%%%%%%%%%%%%%%
%%%%%%%%%%%%%%%%%%%%%%%%%%%%%%%%%%%%%%%%%%%%%%%%%%%%%%%%%%%%%%%%%%%%%%%%%%%%%%%%
\section{Introduction}

\LaTeX{} provides a mechanism to structure a large document (such as a book)
into a main file and several child files (containing the chapters)
using the |\include| command.
This mechanism is beneficial for documents
which span hundreds of pages in order to
make the source file(s) more manageable.
Moreover, compilation can be restricted to
selected child files by means of the |\includeonly| command.
The latter feature can be used to reduce the compilation time while editing
(this was significantly more useful in the earlier days of \LaTeX{})
or to generate a smaller document which is easier to navigate.
Another application of |\includeonly| is to generate
documents consisting of selected parts of the complete document.

However, there are a few drawbacks of the plain |\include| mechanism:
\begin{itemize}
\item
The child files cannot be compiled on their own,
they can only be compiled via the main file.
A naive editing environment
(such as a text editor with an option
to have the current file processed by \LaTeX)
may require one to switch to the main file before compiling;
attempting to compile the child file produces errors.
\item
The main file must be modified (each time)
to adjust the |\includeonly| command
to the present needs. This easily leaves the main file in a messy state.
\item
The generated document will always carry the filename
of the main document. This is inconvenient if
several child files are to be compiled and
to be kept for distribution.
\end{itemize}

The present package provides a simple interface
to make child files individually compilable by \LaTeX{}.
Compiling a child file then has the same effect as compiling
the main file with an |\includeonly| command
to select the appropriate child.
Moreover the generated document will carry the name of the child
rather than the main file.
This resolves all three above issues.

This feature is meant to make the editing of books,
thesis documents and lecture notes somewhat more convenient.
However, the package can also be used efficiently for
composing a series of documents (such as exercise sheets)
which are typically distributed individually.
It then assists the author in generating the individual documents
(potentially in different versions)
as well as a document containing the collected series.
Another application is in developing style files
or other kinds of included material
where compilation of the style file could redirect
to a sample or test file.

%%%%%%%%%%%%%%%%%%%%%%%%%%%%%%%%%%%%%%%%%%%%%%%%%%%%%%%%%%%%%%%%%%%%%%%%%%%%%%%%
%%%%%%%%%%%%%%%%%%%%%%%%%%%%%%%%%%%%%%%%%%%%%%%%%%%%%%%%%%%%%%%%%%%%%%%%%%%%%%%%
\section{Usage}

First of all, the package \textsf{childdoc} is \emph{not} a standard
\LaTeXe{} |.sty| style file! Therefore it needs to be invoked in
a non-standard way.

%%%%%%%%%%%%%%%%%%%%%%%%%%%%%%%%%%%%%%%%%%%%%%%%%%%%%%%%%%%%%%%%%%%%%%%%%%%%%%%%
\subsection{Included Files}
\label{sec:include}

%%%%%%%%%%%%%%%%%%%%%%%%%%%%%%%%%%%%%%%%
\DescribeMacro{\childdocmain}
To use the package, add the commands
\begin{center}
\begin{tabular}{l}
|\input{childdoc.def}|\\
|\childdocmain{}|\\
\end{tabular}
\end{center}
at the very top of the main \LaTeX{} file,
in particular \emph{before} the |\documentclass| statement!
The argument of |\childdocmain| should be left empty
(but it must be present).

%%%%%%%%%%%%%%%%%%%%%%%%%%%%%%%%%%%%%%%%
\DescribeMacro{\childdocof}
Furthermore, add the commands
\begin{center}
\begin{tabular}{l}
|\input{childdoc.def}|\\
|\childdocof{|\textit{main}|}|\\
\end{tabular}
\end{center}
at the top of every child file \textit{child}
which is included by |\include{|\textit{child}|}|
from within the main file
(or at least for those files to be compiled individually).
The argument \textit{main} must be the filename of the main file.

There are a couple of
considerations in setting up the main and child documents:

%%%%%%%%%%%%%%%%%%%%%%%%%%%%%%%%%%%%%%%%
\paragraph{Restrictions.}

Please note the following restrictions:
\begin{itemize}
\item
|\childdocmain| must be called with one argument \textit{main}
to ensure compatibility with earlier version of the package.
It must either be empty (|\childdocmain{}|)
or precisely match the filename of the main file in which it is specified.
See \secref{sec:detection} for further information.
\item
The filename \textit{main} must be specified without the |.tex| extension.
\item
The filename \textit{main} is case sensitive
(even in case-insensitive file systems)
due to internal string comparison.
\item
The argument \textit{main} should be fully expanded, it cannot be a macro.
\item
Subdirectories and special characters should be avoided in filenames.
\item
The command |\childdocmain{|\textit{main}|}| must be followed by a whitespace.
It should not be followed immediately by another command
or by a comment mark `|%|'.
This is because the \TeX{} parser reads the token immediately following
the argument of |\childdocmain| and puts it
at the beginning of every child section;
however, a white\-space is ignored.
\end{itemize}

%%%%%%%%%%%%%%%%%%%%%%%%%%%%%%%%%%%%%%%%
\paragraph{Content of Main File.}

It is advisable to place all content in the child files included by |\include|.
Any output contained in the main file will appear in all child documents
unless suppressed manually;
it cannot be suppressed automatically by the |\includeonly| directive
and thus should normally be avoided.
A method to include some content in the main file
by means of conditional processing is described in \secref{sec:conditional}.

%%%%%%%%%%%%%%%%%%%%%%%%%%%%%%%%%%%%%%%%
\paragraph{Page Numbering.}

When only a part of the document is compiled,
the appropriate numbering of pages
(as well as other status parameters)
is determined from the |.aux| files.
The latter contain information from previous passes.
However this information needs to propagate through
all intermediate child documents.
Therefore the page numbering in child documents may well
be inconsistent until the complete document is compiled at least once.

A useful (if unconventional) way to always ensure a consistent
page numbering is to restart the numbering in each child document
and denote the pages by `\textit{child}|.|\textit{page}'
where \textit{child} represents the chapter/section number of the child file.
This can be achieved by the command
|\numberwithin{page}{|\textit{child}|}|
of the \textsf{amsmath} package
where \textit{child} can be |chapter| or |section|
depending on the chosen structuring.
Alternatively, one can modify the macro |\thepage| appropriately
and reset the counter |page| at the start of each child file.

%%%%%%%%%%%%%%%%%%%%%%%%%%%%%%%%%%%%%%%%%%%%%%%%%%%%%%%%%%%%%%%%%%%%%%%%%%%%%%%%
\subsection{Conditional Processing}
\label{sec:conditional}

The package provides a mechanism to compile different versions
of a document. To customise the versions further some conditional processing
can come in handy to distinguish which version is being compiled.
The package provides two macros to describe the compilation context:

%%%%%%%%%%%%%%%%%%%%%%%%%%%%%%%%%%%%%%%%
\DescribeMacro{\ifchilddoc}
The conditional |\ifchilddoc| distinguishes between the compilation of
child documents and the main document:
%
\begin{center}
|\ifchilddoc |\textit{child-code}| |[|\||else |\textit{main-code}]| \||fi|
\end{center}

%%%%%%%%%%%%%%%%%%%%%%%%%%%%%%%%%%%%%%%%
\DescribeMacro{\childdocname}
\DescribeMacro{\childdocjob}
The macro |\childdocname| contains the filename (without extension)
of the main or child file being processed.
Note that |\childdocjob| will always contain the name of the main file.

%%%%%%%%%%%%%%%%%%%%%%%%%%%%%%%%%%%%%%%%
\paragraph{Title Page.}

Conditional processing can be used to include a title or banner page
in the main document when proper precautions are taken.
Importantly, the code in the main file should ensure that the page counter
(as well as other status parameters which are stored in the |.aux| files)
takes the same value after the conditional processing.
Otherwise the page numbers may take divergent values
depending on which part is compiled.

For example, a title page could be declared by:
%
\begin{center}
\begin{tabular}{l}
|\ifchilddoc\||else|\\
|\addtocounter{page}{-1}|\\
\textit{code for title page}\\
|\newpage|\\
|\||fi|
\end{tabular}
\end{center}
%
A banner page for the child documents can be generated by:
%
\begin{center}
\begin{tabular}{l}
|\ifchilddoc|\\
|\addtocounter{page}{-1}|\\
\textit{code for banner page}\\
|\newpage|\\
|\||fi|
\end{tabular}
\end{center}
%
Here one could write a message such as:
\begin{center}
|This is the part \childdocname{} of \childdocjob{}.|
\end{center}

%%%%%%%%%%%%%%%%%%%%%%%%%%%%%%%%%%%%%%%%%%%%%%%%%%%%%%%%%%%%%%%%%%%%%%%%%%%%%%%%
\subsection{Flags}
\label{sec:flags}

The package makes it easy to generate different versions
of the main or child documents.
To this end compilation flags can be defined
and assigned different default values.
They will be particularly useful in conjunction
with the forwarding mechanism described in \secref{sec:forward}.

For example, it may be useful to have a flag |\version|
which can be set to |draft| or |final|.
The document source will contain some conditional code
depending on the value of |\version|.
Suppose further, the flag should default to |final| for the main file
and to |draft| for child files
which is a natural assignment for editing the document.
This is achieved by placing the following code
in the preamble of the main document
(below the |\childdocmain| directive):
%
\begin{center}
\begin{tabular}{l}
|\ifchilddoc|\\
|\providecommand{\version}{draft}|\\
|\||else|\\
|\providecommand{\version}{final}|\\
|\||fi|
\end{tabular}
\end{center}
%
The definition by |\providecommand| makes sure
that previous definitions are not overwritten.
Further statements |\providecommand{\version}{...}|
can thus be added before the above code to override it.

For the main file, one might add a line
(between |\childdocmain| and the above block)
%
\begin{center}
|%\ifchilddoc\||else\providecommand{\version}{draft}\||fi|
\end{center}
%
which can be uncommented to produce a draft version.
Likewise one can add a line to the very top of a child file
(above the |\childdocof{|\textit{main}|}| directive)
%
\begin{center}
|%\providecommand{\version}{final}|
\end{center}
%
which can be uncommented to produce the final version of this child document.

%%%%%%%%%%%%%%%%%%%%%%%%%%%%%%%%%%%%%%%%%%%%%%%%%%%%%%%%%%%%%%%%%%%%%%%%%%%%%%%%
\subsection{Forwarding}
\label{sec:forward}

Different versions of the main or child documents
using compilation flags as described in \secref{sec:flags}
can be (permanently) stored in different files
for convenient compilation, viewing and distribution.
To this end, the package defines a command
to pass on compilation to a different file:

%%%%%%%%%%%%%%%%%%%%%%%%%%%%%%%%%%%%%%%%
\DescribeMacro{\childdocforward}
The command |\childdocforward| redirects processing to
another source file:
%
\begin{center}
\begin{tabular}{l}
|\input{childdoc.def}|\\
|\childdocforward[|\textit{main}|]{|\textit{dest}|}|\\
\end{tabular}
\end{center}
%
The argument \textit{dest} is the destination file
(without extension).
It should be the main file or one of the child files.
Note that further \textsf{childdoc} directives
such as |\childdocof| and |\childdocforward|
in the indicated file will be processed in this form.
The optional argument \textit{main}
passes on directly to the main file \textit{main}
while pretending to compile the child \textit{dest}.
This form behaves as if \textit{dest}
issues |\childdocof{|\textit{main}|}| right away,
and no further \textsf{childdoc} directives will be processed.

%%%%%%%%%%%%%%%%%%%%%%%%%%%%%%%%%%%%%%%%
\DescribeMacro{\...prefix}
In the alternative form |\childdocforwardprefix|,
%
\begin{center}
\begin{tabular}{l}
|\input{childdoc.def}|\\
|\childdocforwardprefix[|\textit{main}|]{|\textit{prefix}|}{|\textit{dest}|}|
\end{tabular}
\end{center}
%
the destination file is determined by a pattern
depending on the current file:
To make this work, the current file must be called
`{\textit{prefix}\hspace{0.2em}\textit{suffix}}'
with \textit{prefix} matching precisely the argument.
Processing is then passed on to the file
`{\textit{dest}\hspace{0.2em}\textit{suffix}}'.
Surely, the same effect is achieved by
directly specifying the
argument `{\textit{dest}\hspace{0.2em}\textit{suffix}}'
in the first form.
However, that requires to set up a different file
for each child. With the alternative form of the command
all these files can have exactly the same content
which simplifies setting them up and maintaining them.

For example, the following file |draft.tex|
with a compilation flag |\version| as described in \secref{sec:flags}
compiles the main document as a draft:
%
\begin{center}
\begin{tabular}{l}
|\def\version{draft}|\\
|\input{childdoc.def}|\\
|\childdocforward{|\textit{main}|}|
\end{tabular}
\end{center}
%
Likewise, the following files |final|\textit{nn}|.tex|
compile the final version of the child document
|child|\textit{nn}|.tex|:
%
\begin{center}
\begin{tabular}{l}
|\def\version{final}|\\
|\input{childdoc.def}|\\
|\childdocforwardprefix{final}{child}|
\end{tabular}
\end{center}
%

Note that when several versions of a main file and/or of each child file
are to be generated, it may be convenient to set up a |Makefile| or
shell script to automatise the process.

%%%%%%%%%%%%%%%%%%%%%%%%%%%%%%%%%%%%%%%%%%%%%%%%%%%%%%%%%%%%%%%%%%%%%%%%%%%%%%%%
\subsection{Command Line Processing}
\label{sec:commandline}

The effect of redirection files can also be achieved by invoking
the \LaTeX{} compiler with a more elaborate command line.
Most conveniently this should be done as part
of a shell script or a |Makefile|.

When using \textsf{childdoc} in the main file, the following
command lines effectively perform a redirection
(note that depending on the shell being used,
backslashes may have to be doubled: `|\|' $\to$ `|\\|'):
%
\begin{center}
|... -jobname "|\textit{target}|" |\\|"|[\textit{flags}]%
|\input{childdoc.def}\childdocforward[|\textit{main}|]{|\textit{dest}|}"|
\end{center}
%
Here \textit{target} is the name of the output file,
\textit{main} is the name of the main file
and \textit{dest} is the name of the main or child file to be processed
(all filenames without extensions).
The optional argument \textit{main} can be omitted
if \textit{main} matches \textit{dest}.
Optionally, compilation \textit{flags} can be defined via |\def| commands.
This command line makes the \TeX{} engine believe
it is compiling the file \textit{target}
whose content is specified as the latter parameter.
The provided code then forwards the processing to
\textit{main} or \textit{dest} as described in \secref{sec:forward}.

%%%%%%%%%%%%%%%%%%%%%%%%%%%%%%%%%%%%%%%%%%%%%%%%%%%%%%%%%%%%%%%%%%%%%%%%%%%%%%%%
\subsection{Include by Input}
\label{sec:input}

Including child documents by |\include| has some restrictions by design.
Most notably, the content of a child document always occupies
its own set of pages; pages cannot be shared between child documents.
Usually, this behaviour makes perfect sense
because each child document contain an essential part of the document.
However, in some situations it may be desirable to compose
a document from a collection of parts
without having mandatory page breaks between then.
For this case, the package
provides a mechanism to include parts
by |\input| which can also be processed individually.
However, by construction this mechanism
requires manual handling of the content to be output.

%%%%%%%%%%%%%%%%%%%%%%%%%%%%%%%%%%%%%%%%
\DescribeMacro{\ifchilddocmanual}
The main file should be prepared as usual, see \secref{sec:include}.
However, the document body must make a distinction
between processing of an individual part and of the main document, e.g.:
%
\begin{center}
\begin{tabular}{l}
|\ifchilddocmanual|\\
|\input{\childdocname}|\\
|\||else|\\
\textit{document body with }|\input{|\textit{part}|}|\\
|\||fi|
\end{tabular}
\end{center}
%
The conditional |\ifchilddocmanual| is true whenever
a part to be included by |\input| is being compiled,
and the name of the part is stored in |\childdocname|.

%%%%%%%%%%%%%%%%%%%%%%%%%%%%%%%%%%%%%%%%
\DescribeMacro{\childdocby}
Each part to be included by |\input| should start with:
%
\begin{center}
\begin{tabular}{l}
|\input{childdoc.def}|\\
|\childdocby{|\textit{main}|}|\\
\end{tabular}
\end{center}
%
The directive |\childdocby| is similar to |\childdocof|
described in \secref{sec:include},
but the subsequent selection of content must be done manually.
To that end, both |\ifchilddoc| and |\ifchilddocmanual|
will be true upon processing of a part,
and the name of the part is stored in |\childdocname|.
Note that |\jobname| will be set to the filename of the current part
so that each part receives an individual |.aux| file
that does not interfere with the |.aux| file(s) of the main document.
This behaviour can be altered by the alternative form
|\childdocby[*]{|\textit{main}|}| (with a non-empty optional argument)
which uses the |.aux| file of the main document
by setting |\jobname| to \textit{main}.

%%%%%%%%%%%%%%%%%%%%%%%%%%%%%%%%%%%%%%%%%%%%%%%%%%%%%%%%%%%%%%%%%%%%%%%%%%%%%%%%
\subsection{Driver Development}
\label{sec:driver}

The \textsf{childdoc} mechanism can also be use for the development
of definition files such as \LaTeX{} styles or classes.
This case differs from the above setup with multiple parts
included by |\include| in that no |\includeonly| should be invoked.
This can be achieved by starting the include file
(before |\ProvidesPackage|) with:
%
\begin{center}
\begin{tabular}{l}
|\input{childdoc.def}|\\
|\childdocforward{|\textit{main}|}|\\
\end{tabular}
\end{center}
%
or alternatively with:
%
\begin{center}
\begin{tabular}{l}
|\input{childdoc.def}|\\
|\childdocby{|\textit{main}|}|\\
\end{tabular}
\end{center}
%
Both forms have slightly different effects as described above.
The main file is prepared as usual, see \secref{sec:include}.

%%%%%%%%%%%%%%%%%%%%%%%%%%%%%%%%%%%%%%%%%%%%%%%%%%%%%%%%%%%%%%%%%%%%%%%%%%%%%%%%
\subsection{Legacy Detection}
\label{sec:detection}

The directive |\childdocmain| in the main file can detect
whether the complete document or merely a child is to be compiled
even without using the directive |\childdocof|.
This method is deprecated because it is less robust
and there is no compelling reason to use it;
it is merely provided for backward compatibility
and it may be removed in future versions.

If the detection mechanism is to be used,
it is mandatory to correctly specify
the filename of the main file as the argument of |\childdocmain|:
%
\begin{center}
\begin{tabular}{l}
|\input{childdoc.def}|\\
|\childdocmain{|\textit{main}|}|\\
\end{tabular}
\end{center}
%
If |\jobname| does not match the argument \textit{main} of |\childdocmain|,
it is assumed that |\jobname| points to the child file to be compiled.
When using |\childdocmain| with the main file specified as argument,
it suffices to start a child file
with just |\input{|\textit{main}|}|
without loading of the package and using |\childdocof|.
If instead all processing is done
with the appropriate \textsf{childdoc} directives,
the argument of \textit{main} of |\childdocmain| can be empty.

An alternative version of the command line processing described
in \secref{sec:commandline} using the detection mechanism reads:
%
\begin{center}
|... -jobname "|\textit{target}|" "|[\textit{flags}]%
[|\def\jobname{|\textit{dest}|}|]|\input{|\textit{main}|}"|
\end{center}

%%%%%%%%%%%%%%%%%%%%%%%%%%%%%%%%%%%%%%%%%%%%%%%%%%%%%%%%%%%%%%%%%%%%%%%%%%%%%%%%
\subsection{Manual Code}
\label{sec:manual}

In case one cannot be certain whether the definitions file |childdoc.def|
is installed on the target \TeX{} distribution
and one prefers not to ship it,
it is conceivable to paste a few relevant commands into the sources.

To that end, drop all statements |\input{childdoc.def}|
and perform the replacements as outlined below.
Instead of |\childdocmain{|\textit{main}|}| add the following code
to the top of the main file:
%
\begin{center}
\begin{tabular}{l}
|\||ifdefined\childdocname\endinput\||fi\newif\ifchilddoc|\\
|\edef\childdocname{\scantokens\expandafter{\jobname\noexpand}}|\\
|\def\childdocmain{|\textit{main}|}\||ifx\childdocmain\childdocname\||else|\\
|\childdoctrue\includeonly{\childdocname}\let\jobname\childdocmain\||fi|\\
\end{tabular}
\end{center}
%
Instead of |\childdocof{|\textit{main}|}| just include the main file
at the top of each child file:
%
\begin{center}
|\input{|\textit{main}|}|
\end{center}
%
A simple redirection |\childdocforward{|\textit{dest}|}| is achieved by:
%
\begin{center}
|\def\jobname{|\textit{dest}|}\input{\jobname}|
\end{center}
%
The redirection with prefix
|\childdocforwardprefix[|\textit{prefix}|]{|\textit{dest}|}|
is accomplished by:
%
\begin{center}
\begin{tabular}{l}
|{\edef\jobname{\scantokens\expandafter{\jobname\noexpand}}|\\
|\def\redirectjob |\textit{prefix}|#1~~~{\gdef\jobname{|\textit{dest}|#1}}|\\
|\expandafter\redirectjob\jobname~~~}\input{\jobname}|
\end{tabular}
\end{center}

In an alternative approach,
child documents can be compiled by a specific command line
without additional code or specific definitions:
%
\begin{center}
|... -jobname "|\textit{target}|" "|[\textit{flags}]%
|\includeonly{|\textit{dest}|}\input{|\textit{main}|}"|
\end{center}
%

%%%%%%%%%%%%%%%%%%%%%%%%%%%%%%%%%%%%%%%%%%%%%%%%%%%%%%%%%%%%%%%%%%%%%%%%%%%%%%%%
%%%%%%%%%%%%%%%%%%%%%%%%%%%%%%%%%%%%%%%%%%%%%%%%%%%%%%%%%%%%%%%%%%%%%%%%%%%%%%%%
\section{Information}

%%%%%%%%%%%%%%%%%%%%%%%%%%%%%%%%%%%%%%%%%%%%%%%%%%%%%%%%%%%%%%%%%%%%%%%%%%%%%%%%
\subsection{Copyright}

Copyright \copyright{} 2017--2018 Niklas Beisert

This work may be distributed and/or modified under the
conditions of the \LaTeX{} Project Public License, either version 1.3
of this license or (at your option) any later version.
The latest version of this license is in
  \url{http://www.latex-project.org/lppl.txt}
and version 1.3 or later is part of all distributions of \LaTeX{}
version 2005/12/01 or later.

This work has the LPPL maintenance status `maintained'.

The Current Maintainer of this work is Niklas Beisert.

This work consists of the files |README.txt|, |childdoc.ins| and |childdoc.dtx|
as well as the derived files |childdoc.def|, |cdocsamp.tex|
with |cdocsch1.tex|, |cdocsch2.tex|, |cdocspt3.tex|, |cdocspt4.tex|,
|cdocsdrf.tex|, |cdocsfn1.tex|, |cdocsfn2.tex|
as well as |childdoc.pdf|.

%%%%%%%%%%%%%%%%%%%%%%%%%%%%%%%%%%%%%%%%%%%%%%%%%%%%%%%%%%%%%%%%%%%%%%%%%%%%%%%%
\subsection{Files and Installation}

The package consists of the files:
%
\begin{center}
\begin{tabular}{ll}
    |README.txt|   & readme file \\
    |childdoc.ins| & installation file \\
    |childdoc.dtx| & source file \\
    |childdoc.def| & definition file \\
    |cdocsamp.tex| & sample main file \\
    |cdocsch1.tex| & sample include file \\
    |cdocsch2.tex| & sample include file \\
    |cdocspt3.tex| & sample part file \\
    |cdocspt4.tex| & sample part file \\
    |cdocsdrf.tex| & sample redirection file \\
    |cdocsfn1.tex| & sample redirection file \\
    |cdocsfn2.tex| & sample redirection file \\
    |childdoc.pdf| & manual
\end{tabular}
\end{center}
%
The distribution consists of the files
|README.txt|, |childdoc.ins| and |childdoc.dtx|.
%
\begin{itemize}
\item
Run (pdf)\LaTeX{} on |childdoc.dtx|
to compile the manual |childdoc.pdf| (this file).
\item
Run \LaTeX{} on |childdoc.ins| to create the definitions file |childdoc.def|
and the sample |cdocsamp.tex| with include files
|cdocsch1.tex|, |cdocsch2.tex|, |cdocspt3.tex|, |cdocspt4.tex|,
|cdocsdrf.tex|, |cdocsfn1.tex|, |cdocsfn2.tex|.
Then copy the file |childdoc.def| to an appropriate directory of your \LaTeX{}
distribution, e.g.\ \textit{texmf-root}|/tex/latex/childdoc|.
\end{itemize}

%%%%%%%%%%%%%%%%%%%%%%%%%%%%%%%%%%%%%%%%%%%%%%%%%%%%%%%%%%%%%%%%%%%%%%%%%%%%%%%%
\subsection{Related CTAN Packages}

There are several other packages which offer a similar functionality:
%
\begin{itemize}
\item
The packages
\href{http://ctan.org/pkg/docmute}{\textsf{docmute}},
\href{http://ctan.org/pkg/includex}{\textsf{includex}} and
\href{http://ctan.org/pkg/standalone}{\textsf{standalone}}
provide commands to include only the document body of
a child file thus allowing both files to be compiled individually.
\item
The packages \href{http://ctan.org/pkg/subdocs}{\textsf{subdocs}}
and \href{http://ctan.org/pkg/subfiles}{\textsf{subfiles}}
provide structures in which the main and child documents can be
encapsulated and allowing them to be compiled individually.
The inclusion mechanism is different from the conventional |\include|.
\item
The package \href{http://ctan.org/pkg/combine}{\textsf{combine}}
is an elaborate solution to combine several documents into one.
\end{itemize}
%
See also the CTAN topic \href{http://ctan.org/topic/subdocs}{\textsf{subdocs}}
for further related packages.
The present package differs from the above solutions in that
a document structure constructed with the conventional |\include| mechanism
just needs two extra commands at the top of every file
such that all constituent files can be compiled individually.

%%%%%%%%%%%%%%%%%%%%%%%%%%%%%%%%%%%%%%%%%%%%%%%%%%%%%%%%%%%%%%%%%%%%%%%%%%%%%%%%
%\subsection{Feature Suggestions}
%
%The following is a list of features which may be useful for future
%versions of this package:
%%
%\begin{itemize}
%\item
%\ldots
%\end{itemize}

%%%%%%%%%%%%%%%%%%%%%%%%%%%%%%%%%%%%%%%%%%%%%%%%%%%%%%%%%%%%%%%%%%%%%%%%%%%%%%%%
\subsection{Revision History}

%%%%%%%%%%%%%%%%%%%%%%%%%%%%%%%%%%%%%%%%
\paragraph{v2.0:} 2018/12/30

\begin{itemize}
\item
immediate forward processing
\item
added |\childdocby| mechanism
\item
manual restructured
\end{itemize}

%%%%%%%%%%%%%%%%%%%%%%%%%%%%%%%%%%%%%%%%
\paragraph{v1.6:} 2018/01/17

\begin{itemize}
\item
application for development of include files
\item
corrections to manual
\end{itemize}

%%%%%%%%%%%%%%%%%%%%%%%%%%%%%%%%%%%%%%%%
\paragraph{v1.5:} 2017/05/21

\begin{itemize}
\item
more complete structuring introduced
\item
|\childdocof| introduced
\item
|\childdoc| renamed to |\childdocmain|
\item
|\childredirect| renamed to |\childdocforward| and |\childdocforwardprefix|
and functionality expanded
\end{itemize}

%%%%%%%%%%%%%%%%%%%%%%%%%%%%%%%%%%%%%%%%
\paragraph{v1.0:} 2017/04/27

\begin{itemize}
\item
manual and install package
\item
first version published on CTAN
\end{itemize}

%%%%%%%%%%%%%%%%%%%%%%%%%%%%%%%%%%%%%%%%
\paragraph{v0.6:} 2017/04/26

\begin{itemize}
\item
redirection mechanism added
\end{itemize}

%%%%%%%%%%%%%%%%%%%%%%%%%%%%%%%%%%%%%%%%
\paragraph{v0.5:} 2017/04/26

\begin{itemize}
\item
functionality in definition file
\end{itemize}


%%%%%%%%%%%%%%%%%%%%%%%%%%%%%%%%%%%%%%%%%%%%%%%%%%%%%%%%%%%%%%%%%%%%%%%%%%%%%%%%
%%%%%%%%%%%%%%%%%%%%%%%%%%%%%%%%%%%%%%%%%%%%%%%%%%%%%%%%%%%%%%%%%%%%%%%%%%%%%%%%
%%%%%%%%%%%%%%%%%%%%%%%%%%%%%%%%%%%%%%%%%%%%%%%%%%%%%%%%%%%%%%%%%%%%%%%%%%%%%%%%
\appendix

\settowidth\MacroIndent{\rmfamily\scriptsize 000\ }

 \DocInput{childdoc.dtx}

\end{document}
%</driver>
% \fi
%
% %%%%%%%%%%%%%%%%%%%%%%%%%%%%%%%%%%%%%%%%%%%%%%%%%%%%%%%%%%%%%%%%%%%%%%%%%%%%%%
% %%%%%%%%%%%%%%%%%%%%%%%%%%%%%%%%%%%%%%%%%%%%%%%%%%%%%%%%%%%%%%%%%%%%%%%%%%%%%%
% \section{Sample}
%\iffalse
%<*samplemain>
%\fi
%
% The following presents a sample document
% with two chapters, two parts, a title page,
% a compile flag as well as three forwarding files to set the flag.
% It consists of eight |.tex| files:
% \begin{center}
% \begin{tabular}{ll}
% |cdocsamp.tex|&main file\\
% |cdocsch1.tex|&include file for chapter 1\\
% |cdocsch2.tex|&include file for chapter 2\\
% |cdocspt3.tex|&include file for part 3\\
% |cdocspt4.tex|&include file for part 4\\
% |cdocsdrf.tex|&forwarding file for main file in draft mode\\
% |cdocsfi1.tex|&forwarding file for final version of chapter 1\\
% |cdocsfi2.tex|&forwarding file for final version of chapter 2\\
% \end{tabular}
% \end{center}
% Each of the eight files can be compiled directly by the \LaTeX{} compiler.
%
% %%%%%%%%%%%%%%%%%%%%%%%%%%%%%%%%%%%%%%
% \paragraph{Main File.}
%
% The main file is called |cdocsamp.tex|.
%
% Load the \textsf{childdoc} definitions and
% declare the filename for the main document:
%    \begin{macrocode}
\input{childdoc.def}
\childdocmain{}
%    \end{macrocode}

% Optional override for |\version| flag:
%    \begin{macrocode}
%%\ifchilddoc\else\providecommand{\version}{draft}\fi
%    \end{macrocode}

% Define the default values for the |\version| flag
% (|final| for the main file and |draft| for childs):
%    \begin{macrocode}
\ifchilddoc
\providecommand{\version}{draft}
\else
\providecommand{\version}{final}
\fi
%    \end{macrocode}

% Load the standard document class:
%    \begin{macrocode}
\documentclass[12pt]{article}
%    \end{macrocode}

% Start the document body:
%    \begin{macrocode}
\begin{document}
%    \end{macrocode}

% Declare a title page.
% Print title, part of document being processed and version flag:
%    \begin{macrocode}
\addtocounter{page}{-1}
\begin{center}
{\LARGE\bfseries{}childdoc example\par}
\vspace{1cm}
\ifchilddoc
\ifchilddocmanual part\else chapter\fi:
`\childdocname' of `\childdocjob'\par
\else
main document: `\childdocjob'\par
\fi
version: \version\par
\end{center}
\newpage
%    \end{macrocode}

% Manually include selected file,
% otherwise process as usual:
%    \begin{macrocode}
\ifchilddocmanual
\section*{part `\childdocname'}
\input{\childdocname}
\else
%    \end{macrocode}

% Include the two chapters:
%    \begin{macrocode}
\include{cdocsch1}
\include{cdocsch2}
%    \end{macrocode}

% Include the two parts unless only chapters should be displayed:
%    \begin{macrocode}
\ifchilddoc\else
\section{part three}
\input{cdocspt3}
\section{part four}
\input{cdocspt4}
\fi
%    \end{macrocode}

% Process as usual until here:
%    \begin{macrocode}
\fi
%    \end{macrocode}

% End of document body:
%    \begin{macrocode}
\end{document}
%    \end{macrocode}
%\iffalse
%</samplemain>
%\fi
%
% %%%%%%%%%%%%%%%%%%%%%%%%%%%%%%%%%%%%%%
% \paragraph{Chapter Include Files.}
%
% The include files are called |cdocsch1.tex| and |cdocsch2.tex|.
%
%\iffalse
%<*samplechap1|samplechap2>
%\fi

% Optional override for |\version| flag:
%    \begin{macrocode}
%%\providecommand{\version}{final}
%    \end{macrocode}

% Include the main document:
%    \begin{macrocode}
\input{childdoc.def}
\childdocof{cdocsamp}
%    \end{macrocode}

%\iffalse
%</samplechap1|samplechap2>
%\fi
%
%\iffalse
%<*samplechap1>
%\fi
% Some text for chapter 1:
%    \begin{macrocode}
\section{one}
some text in chapter one
%    \end{macrocode}

%\iffalse
%</samplechap1>
%\fi
% Some text for chapter 2:
%\iffalse
%<*samplechap2>
%\fi
%    \begin{macrocode}
\section{two}
more text in chapter two
%    \end{macrocode}

%\iffalse
%</samplechap2>
%\fi
%
% %%%%%%%%%%%%%%%%%%%%%%%%%%%%%%%%%%%%%%
% \paragraph{Part Include Files.}
%
% The include files are called |cdocspt3.tex| and |cdocspt4.tex|.
%
%\iffalse
%<*samplepart3|samplepart4>
%\fi

% Optional override for |\version| flag:
%    \begin{macrocode}
%%\providecommand{\version}{final}
%    \end{macrocode}

% Include the main document:
%    \begin{macrocode}
\input{childdoc.def}
\childdocby{cdocsamp}
%    \end{macrocode}

%\iffalse
%</samplepart3|samplepart4>
%\fi
%
%\iffalse
%<*samplepart3>
%\fi
% Some text for part 3:
%    \begin{macrocode}
some text in part three
%    \end{macrocode}

%\iffalse
%</samplepart3>
%\fi
% Some text for part 4:
%\iffalse
%<*samplepart4>
%\fi
%    \begin{macrocode}
more text in part four
%    \end{macrocode}

%\iffalse
%</samplepart4>
%\fi
%
% %%%%%%%%%%%%%%%%%%%%%%%%%%%%%%%%%%%%%%
% \paragraph{Forwarding for a Complete Draft.}
%
% The following forwarding file |cdocsdrf.tex|
% compiles the main document in draft mode:
%\iffalse
%<*sampledraft>
%\fi
%    \begin{macrocode}
\def\version{draft}
\input{childdoc.def}
\childdocforward{cdocsamp}
%    \end{macrocode}

%\iffalse
%</sampledraft>
%\fi
%
% %%%%%%%%%%%%%%%%%%%%%%%%%%%%%%%%%%%%%%
% \paragraph{Forwarding for Final Version of the Chapters.}
%
% The following forwarding files |cdocsfn1.tex| and |cdocsfn2.tex|
% (with identical content)
% compile the final versions of the child documents
% |cdocsch1.tex| and |cdocsch2.tex|, respectively:
%\iffalse
%<*samplefinal>
%\fi
%    \begin{macrocode}
\def\version{final}
\input{childdoc.def}
\childdocforwardprefix[cdocsamp]{cdocsfn}{cdocsch}
%    \end{macrocode}

%\iffalse
%</samplefinal>
%\fi
%
% %%%%%%%%%%%%%%%%%%%%%%%%%%%%%%%%%%%%%%
% \paragraph{Command Line Processing.}
%
% The following three command lines generate the output files
% |cdocscld|, |cdocscl1| and |cdocscl2|
% which should be identical to
% |cdocsdrf|, |cdocsch1| and |cdocsfn2|, respectively:
% \begin{center}
% \begin{tabular}{l}
% |latex -jobname cdocscld \|\\
% |  "\def\version{draft}\input{childdoc.def}\childdocforward{cdocsamp}"|\\
% |latex -jobname cdocscl1 \|\\
% |  "\input{childdoc.def}\childdocforward[cdocsamp]{cdocsch1}"|\\
% |latex -jobname cdocscl2 \|\\
% |  "\def\version{final}\input{childdoc.def}\childdocforward{cdocsch2}"|
% \end{tabular}
% \end{center}
% Note that the trailing backslash on each first line
% merely continues the input to the second line
% (for convenient cut ant paste).
% Furthermore, the command |latex| can be replaced by any
% of its alternative versions such as |pdflatex|.
%
% %%%%%%%%%%%%%%%%%%%%%%%%%%%%%%%%%%%%%%%%%%%%%%%%%%%%%%%%%%%%%%%%%%%%%%%%%%%%%%
% %%%%%%%%%%%%%%%%%%%%%%%%%%%%%%%%%%%%%%%%%%%%%%%%%%%%%%%%%%%%%%%%%%%%%%%%%%%%%%
% \section{Implementation}
%\iffalse
%<*package>
%\fi
%
% This section describes the definitions file |childdoc.def|.

% The definitions cannot be loaded using |\usepackage| or |\RequirePackage|
% which has a mechanism to prevent loading a style file more than once.
% When loading the definitions by means of |\input|
% multiple instances have to be prevented manually:
%\iffalse
%This code needs to be before the `\ProvidesFile' directive
%which is defined at the beginning of this file.
%Therefore it is also placed there and commented out here.
%</package>
%<*discard>
%\fi
%    \begin{macrocode}
\ifdefined\childdocmain\endinput\fi
%    \end{macrocode}
%\iffalse
%</discard>
%<*package>
%\fi
%
% \macro{\ifchilddoc}
% \macro{\ifchilddocmanual}
% The conditional |\ifchilddoc| tells whether a
% child (true) or main (false) document is being compiled.
% The conditional |\ifchilddocmanual| tells whether
% the |\includeonly| mechanism is used (false) or
% the selection of child files must be performed manually (true).
% The definitions initialise to false:
%    \begin{macrocode}
\newif\ifchilddoc
\newif\ifchilddocmanual
%    \end{macrocode}

% \macro{\childdocname}
% \macro{\childdocjob}
% The macro |\childdocname| stores the name of the main document
% to be compiled. The macro |\childdocjob| stores the name of
% the document on which the \LaTeX{} compiler was originally invoked.
% The content of |\jobname| cannot be compared
% to filenames specified in the source due to different catcodes.
% The following code rescans |\jobname|, stores the result
% in |\childdocname| and saves a copy in |\childdocjob|:
%    \begin{macrocode}
\edef\childdocname{\scantokens\expandafter{\jobname\noexpand}}
\let\childdocjob\childdocname
%    \end{macrocode}

% \macro{\childdocdisable}
% The macro |\childdocdisable| prevents the main file
% from being processed more than once.
% At this stage, the main document command |\childdocmain|
% is assumed to be called once again where it should do nothing.
% Any subsequent call to it should prevent
% a secondary processing of the main document
% It overwrites the forwarding commands
% |\childdocof| and |\childdocforward|
% with empty macros to prevent further inclusions of the main document:
%    \begin{macrocode}
\newcommand{\childdocdisable}
{
  \renewcommand{\childdocmain}[1]{\renewcommand{\childdocmain}[1]{\endinput}}
  \renewcommand{\childdocof}[1]{}
  \renewcommand{\childdocby}[2][]{}
  \renewcommand{\childdocforward}[2][]{}
  \renewcommand{\childdocdisable}{}
}
%    \end{macrocode}

% \macro{\childdocmain}
% The macro |\childdocmain| is to be called at the top of the main file
% with nothing or the main filename (without extension) as argument.
% First, it breaks loops.
% If the argument is not empty and does not match |\childdocname|
% (which is set by the first inclusion of |childdoc.def|),
% |\ifchilddoc| is set to true, |\includeonly| is applied to the child file
% and |\jobname| is set to the main file
% (for proper handling of |.aux| files):
%    \begin{macrocode}
\newcommand{\childdocmain}[1]
{
  \childdocdisable\childdocmain{}
  \if?#1?\else
    \begingroup
      \def\childdoctmp{#1}
      \ifx\childdoctmp\childdocname
        \def\childdoctmp{}
      \else
        \def\childdoctmp
        {
          \childdoctrue
          \includeonly{\childdocname}
          \def\childdocjob{#1}
          \def\jobname{#1}
        }
      \fi
      \expandafter
    \endgroup
    \childdoctmp
  \fi
}
%    \end{macrocode}

% \macro{\childdocof}
% The command |\childdocof| redirects
% compilation to the main file |#1|.
%    \begin{macrocode}
\newcommand{\childdocof}[1]
{
  \childdocdisable
  \childdoctrue
  \includeonly{\childdocname}
  \def\jobname{#1}
  \def\childdocjob{#1}
  \input{#1}
}
%    \end{macrocode}

% \macro{\childdocby}
% The command |\childdocby| ....
%    \begin{macrocode}
\newcommand{\childdocby}[2][]
{
  \childdocdisable
  \childdoctrue
  \childdocmanualtrue
  \if?#1?\else
    \def\jobname{#2}
  \fi
  \def\childdocjob{#2}
  \input{#2}
  \endinput
}
%    \end{macrocode}

% \macro{\childdocforward}
% The command |\childdocforward| redirects
% compilation to the main file or
% (if the optional argument is given) a child file.
% Parameters are set as if the main file
% or a child file starting with |\childdocof| was compiled.
% Then compilation is handed over to the main file:
%    \begin{macrocode}
\newcommand{\childdocforward}[2][]
{
  \begingroup
    \if?#1?
      \def\childdoctmp
      {
        \def\childdocname{#2}
        \def\childdocjob{#2}
        \def\jobname{#2}
        \input{#2}
        \endinput
      }
    \else
      \def\childdoctmp
      {
        \childdocdisable
        \def\childdocname{#2}
        \childdoctrue
        \includeonly{#2}
        \def\childdocjob{#1}
        \def\jobname{#1}
        \input{#1}
        \endinput
      }
    \fi
    \expandafter
  \endgroup
  \childdoctmp
}
%    \end{macrocode}

% \macro{\childdocforwardprefix}
% The command |\childdocforwardprefix| redirects
% compilation to the main or a child file by means of a pattern.
% The prefix |#1| in the current filename is replaced by |#2|
% and the suffix of the current filename is kept
% (it is assumed that the filename does not contain the substring `|~~~|'
% which is used as a delimiter).
% Compilation is handed over to the new file by |\childdocforward|:
%    \begin{macrocode}
\newcommand{\childdocforwardprefix}[3][]
{
  \begingroup
    \def\childdocextract #2##1~~~{\def\childdoctmp{\childdocforward[#1]{#3##1}}}
    \expandafter\childdocextract\childdocname~~~
    \expandafter
  \endgroup
  \childdoctmp
}
%    \end{macrocode}

% \macro{\childdoc}
% The deprecated macro |\childdoc| is a legacy version of |\childdocmain|:
%    \begin{macrocode}
\newcommand{\childdoc}{\childdocmain}
%    \end{macrocode}

% \macro{\childdocredirect}
% The deprecated macro |\childdocredirect| is a legacy version
% of |\childdocforward| and |\childdocforwardprefix|:
%    \begin{macrocode}
\newcommand{\childdocredirect}[2][]
{
  \begingroup
    \if?#1?
      \def\childdoctmp{\childdocforward{#2}}
    \else
      \def\childdoctmp{\childdocforwardprefix{#1}{#2}}
    \fi
    \expandafter
  \endgroup
  \childdoctmp
}
%    \end{macrocode}

%\iffalse
%</package>
%\fi
%
\endinput
|\\
|\childdocby{|\textit{main}|}|\\
\end{tabular}
\end{center}
%
The directive |\childdocby| is similar to |\childdocof|
described in \secref{sec:include},
but the subsequent selection of content must be done manually.
To that end, both |\ifchilddoc| and |\ifchilddocmanual|
will be true upon processing of a part,
and the name of the part is stored in |\childdocname|.
Note that |\jobname| will be set to the filename of the current part
so that each part receives an individual |.aux| file
that does not interfere with the |.aux| file(s) of the main document.
This behaviour can be altered by the alternative form
|\childdocby[*]{|\textit{main}|}| (with a non-empty optional argument)
which uses the |.aux| file of the main document
by setting |\jobname| to \textit{main}.

%%%%%%%%%%%%%%%%%%%%%%%%%%%%%%%%%%%%%%%%%%%%%%%%%%%%%%%%%%%%%%%%%%%%%%%%%%%%%%%%
\subsection{Driver Development}
\label{sec:driver}

The \textsf{childdoc} mechanism can also be use for the development
of definition files such as \LaTeX{} styles or classes.
This case differs from the above setup with multiple parts
included by |\include| in that no |\includeonly| should be invoked.
This can be achieved by starting the include file
(before |\ProvidesPackage|) with:
%
\begin{center}
\begin{tabular}{l}
|% \iffalse
%
% childdoc.dtx Copyright (C) 2017-2018 Niklas Beisert
%
% This work may be distributed and/or modified under the
% conditions of the LaTeX Project Public License, either version 1.3
% of this license or (at your option) any later version.
% The latest version of this license is in
%   http://www.latex-project.org/lppl.txt
% and version 1.3 or later is part of all distributions of LaTeX
% version 2005/12/01 or later.
%
% This work has the LPPL maintenance status `maintained'.
%
% The Current Maintainer of this work is Niklas Beisert.
%
% This work consists of the files childdoc.dtx and childdoc.ins
% and the derived files childdoc.def and cdocsamp.tex with
% cdocsch1.tex, cdocsch2.tex, cdocsdrf.tex, cdocsfn1.tex, cdocsfn2.tex.
%
%<package>\ifdefined\childdocmain\endinput\fi
%<package>\ProvidesFile{childdoc.def}[2018/12/30 v2.0 child document driver]
%<samplemain>\ProvidesFile{cdocsamp.tex}[2018/12/30 v2.0 sample for childdoc]
%<*driver>
%\ProvidesFile{childdoc.drv}[2018/12/30 v2.0 childdoc reference manual file]
\PassOptionsToClass{10pt,a4paper}{article}
\documentclass{ltxdoc}

\usepackage[margin=35mm]{geometry}
\usepackage{hyperref}
\usepackage{hyperxmp}
\usepackage[usenames]{color}

\hypersetup{colorlinks=true}
\hypersetup{pdfstartview=FitH}
\hypersetup{pdfpagemode=UseNone}
\hypersetup{pdfsource={}}
\hypersetup{pdflang={en-UK}}
\hypersetup{pdfcopyright={Copyright 2017-2018 Niklas Beisert.
  This work may be distributed and/or modified under the
  conditions of the LaTeX Project Public License, either version 1.3
  of this license or (at your option) any later version.}}
\hypersetup{pdflicenseurl={http://www.latex-project.org/lppl.txt}}
\hypersetup{pdfcontactaddress={ETH Zurich, ITP, HIT K,
  Wolfgang-Pauli-Strasse 27}}
\hypersetup{pdfcontactpostcode={8093}}
\hypersetup{pdfcontactcity={Zurich}}
\hypersetup{pdfcontactcountry={Switzerland}}
\hypersetup{pdfcontactemail={nbeisert@itp.phys.ethz.ch}}
\hypersetup{pdfcontacturl={http://people.phys.ethz.ch/\xmptilde nbeisert/}}

\newcommand{\secref}[1]{\hyperref[#1]{section \ref*{#1}}}

\parskip1ex
\parindent0pt
\let\olditemize\itemize
\def\itemize{\olditemize\parskip0pt}

\begin{document}

\title{The \textsf{childdoc} Package}
\hypersetup{pdftitle={The childdoc Package}}
\author{Niklas Beisert\\[2ex]
  Institut f\"ur Theoretische Physik\\
  Eidgen\"ossische Technische Hochschule Z\"urich\\
  Wolfgang-Pauli-Strasse 27, 8093 Z\"urich, Switzerland\\[1ex]
  \href{mailto:nbeisert@itp.phys.ethz.ch}
  {\texttt{nbeisert@itp.phys.ethz.ch}}}
\hypersetup{pdfauthor={Niklas Beisert}}
\hypersetup{pdfsubject={Manual for the LaTeX2e Package childdoc}}
\date{30 December 2018, \textsf{v2.0}}
\maketitle

\begin{abstract}\noindent
\textsf{childdoc} is a \LaTeXe{} package
that enables the direct compilation
of document sections included by |\include|
to individual files.
\end{abstract}

\begingroup
\parskip0ex
\tableofcontents
\endgroup

%%%%%%%%%%%%%%%%%%%%%%%%%%%%%%%%%%%%%%%%%%%%%%%%%%%%%%%%%%%%%%%%%%%%%%%%%%%%%%%%
%%%%%%%%%%%%%%%%%%%%%%%%%%%%%%%%%%%%%%%%%%%%%%%%%%%%%%%%%%%%%%%%%%%%%%%%%%%%%%%%
\section{Introduction}

\LaTeX{} provides a mechanism to structure a large document (such as a book)
into a main file and several child files (containing the chapters)
using the |\include| command.
This mechanism is beneficial for documents
which span hundreds of pages in order to
make the source file(s) more manageable.
Moreover, compilation can be restricted to
selected child files by means of the |\includeonly| command.
The latter feature can be used to reduce the compilation time while editing
(this was significantly more useful in the earlier days of \LaTeX{})
or to generate a smaller document which is easier to navigate.
Another application of |\includeonly| is to generate
documents consisting of selected parts of the complete document.

However, there are a few drawbacks of the plain |\include| mechanism:
\begin{itemize}
\item
The child files cannot be compiled on their own,
they can only be compiled via the main file.
A naive editing environment
(such as a text editor with an option
to have the current file processed by \LaTeX)
may require one to switch to the main file before compiling;
attempting to compile the child file produces errors.
\item
The main file must be modified (each time)
to adjust the |\includeonly| command
to the present needs. This easily leaves the main file in a messy state.
\item
The generated document will always carry the filename
of the main document. This is inconvenient if
several child files are to be compiled and
to be kept for distribution.
\end{itemize}

The present package provides a simple interface
to make child files individually compilable by \LaTeX{}.
Compiling a child file then has the same effect as compiling
the main file with an |\includeonly| command
to select the appropriate child.
Moreover the generated document will carry the name of the child
rather than the main file.
This resolves all three above issues.

This feature is meant to make the editing of books,
thesis documents and lecture notes somewhat more convenient.
However, the package can also be used efficiently for
composing a series of documents (such as exercise sheets)
which are typically distributed individually.
It then assists the author in generating the individual documents
(potentially in different versions)
as well as a document containing the collected series.
Another application is in developing style files
or other kinds of included material
where compilation of the style file could redirect
to a sample or test file.

%%%%%%%%%%%%%%%%%%%%%%%%%%%%%%%%%%%%%%%%%%%%%%%%%%%%%%%%%%%%%%%%%%%%%%%%%%%%%%%%
%%%%%%%%%%%%%%%%%%%%%%%%%%%%%%%%%%%%%%%%%%%%%%%%%%%%%%%%%%%%%%%%%%%%%%%%%%%%%%%%
\section{Usage}

First of all, the package \textsf{childdoc} is \emph{not} a standard
\LaTeXe{} |.sty| style file! Therefore it needs to be invoked in
a non-standard way.

%%%%%%%%%%%%%%%%%%%%%%%%%%%%%%%%%%%%%%%%%%%%%%%%%%%%%%%%%%%%%%%%%%%%%%%%%%%%%%%%
\subsection{Included Files}
\label{sec:include}

%%%%%%%%%%%%%%%%%%%%%%%%%%%%%%%%%%%%%%%%
\DescribeMacro{\childdocmain}
To use the package, add the commands
\begin{center}
\begin{tabular}{l}
|\input{childdoc.def}|\\
|\childdocmain{}|\\
\end{tabular}
\end{center}
at the very top of the main \LaTeX{} file,
in particular \emph{before} the |\documentclass| statement!
The argument of |\childdocmain| should be left empty
(but it must be present).

%%%%%%%%%%%%%%%%%%%%%%%%%%%%%%%%%%%%%%%%
\DescribeMacro{\childdocof}
Furthermore, add the commands
\begin{center}
\begin{tabular}{l}
|\input{childdoc.def}|\\
|\childdocof{|\textit{main}|}|\\
\end{tabular}
\end{center}
at the top of every child file \textit{child}
which is included by |\include{|\textit{child}|}|
from within the main file
(or at least for those files to be compiled individually).
The argument \textit{main} must be the filename of the main file.

There are a couple of
considerations in setting up the main and child documents:

%%%%%%%%%%%%%%%%%%%%%%%%%%%%%%%%%%%%%%%%
\paragraph{Restrictions.}

Please note the following restrictions:
\begin{itemize}
\item
|\childdocmain| must be called with one argument \textit{main}
to ensure compatibility with earlier version of the package.
It must either be empty (|\childdocmain{}|)
or precisely match the filename of the main file in which it is specified.
See \secref{sec:detection} for further information.
\item
The filename \textit{main} must be specified without the |.tex| extension.
\item
The filename \textit{main} is case sensitive
(even in case-insensitive file systems)
due to internal string comparison.
\item
The argument \textit{main} should be fully expanded, it cannot be a macro.
\item
Subdirectories and special characters should be avoided in filenames.
\item
The command |\childdocmain{|\textit{main}|}| must be followed by a whitespace.
It should not be followed immediately by another command
or by a comment mark `|%|'.
This is because the \TeX{} parser reads the token immediately following
the argument of |\childdocmain| and puts it
at the beginning of every child section;
however, a white\-space is ignored.
\end{itemize}

%%%%%%%%%%%%%%%%%%%%%%%%%%%%%%%%%%%%%%%%
\paragraph{Content of Main File.}

It is advisable to place all content in the child files included by |\include|.
Any output contained in the main file will appear in all child documents
unless suppressed manually;
it cannot be suppressed automatically by the |\includeonly| directive
and thus should normally be avoided.
A method to include some content in the main file
by means of conditional processing is described in \secref{sec:conditional}.

%%%%%%%%%%%%%%%%%%%%%%%%%%%%%%%%%%%%%%%%
\paragraph{Page Numbering.}

When only a part of the document is compiled,
the appropriate numbering of pages
(as well as other status parameters)
is determined from the |.aux| files.
The latter contain information from previous passes.
However this information needs to propagate through
all intermediate child documents.
Therefore the page numbering in child documents may well
be inconsistent until the complete document is compiled at least once.

A useful (if unconventional) way to always ensure a consistent
page numbering is to restart the numbering in each child document
and denote the pages by `\textit{child}|.|\textit{page}'
where \textit{child} represents the chapter/section number of the child file.
This can be achieved by the command
|\numberwithin{page}{|\textit{child}|}|
of the \textsf{amsmath} package
where \textit{child} can be |chapter| or |section|
depending on the chosen structuring.
Alternatively, one can modify the macro |\thepage| appropriately
and reset the counter |page| at the start of each child file.

%%%%%%%%%%%%%%%%%%%%%%%%%%%%%%%%%%%%%%%%%%%%%%%%%%%%%%%%%%%%%%%%%%%%%%%%%%%%%%%%
\subsection{Conditional Processing}
\label{sec:conditional}

The package provides a mechanism to compile different versions
of a document. To customise the versions further some conditional processing
can come in handy to distinguish which version is being compiled.
The package provides two macros to describe the compilation context:

%%%%%%%%%%%%%%%%%%%%%%%%%%%%%%%%%%%%%%%%
\DescribeMacro{\ifchilddoc}
The conditional |\ifchilddoc| distinguishes between the compilation of
child documents and the main document:
%
\begin{center}
|\ifchilddoc |\textit{child-code}| |[|\||else |\textit{main-code}]| \||fi|
\end{center}

%%%%%%%%%%%%%%%%%%%%%%%%%%%%%%%%%%%%%%%%
\DescribeMacro{\childdocname}
\DescribeMacro{\childdocjob}
The macro |\childdocname| contains the filename (without extension)
of the main or child file being processed.
Note that |\childdocjob| will always contain the name of the main file.

%%%%%%%%%%%%%%%%%%%%%%%%%%%%%%%%%%%%%%%%
\paragraph{Title Page.}

Conditional processing can be used to include a title or banner page
in the main document when proper precautions are taken.
Importantly, the code in the main file should ensure that the page counter
(as well as other status parameters which are stored in the |.aux| files)
takes the same value after the conditional processing.
Otherwise the page numbers may take divergent values
depending on which part is compiled.

For example, a title page could be declared by:
%
\begin{center}
\begin{tabular}{l}
|\ifchilddoc\||else|\\
|\addtocounter{page}{-1}|\\
\textit{code for title page}\\
|\newpage|\\
|\||fi|
\end{tabular}
\end{center}
%
A banner page for the child documents can be generated by:
%
\begin{center}
\begin{tabular}{l}
|\ifchilddoc|\\
|\addtocounter{page}{-1}|\\
\textit{code for banner page}\\
|\newpage|\\
|\||fi|
\end{tabular}
\end{center}
%
Here one could write a message such as:
\begin{center}
|This is the part \childdocname{} of \childdocjob{}.|
\end{center}

%%%%%%%%%%%%%%%%%%%%%%%%%%%%%%%%%%%%%%%%%%%%%%%%%%%%%%%%%%%%%%%%%%%%%%%%%%%%%%%%
\subsection{Flags}
\label{sec:flags}

The package makes it easy to generate different versions
of the main or child documents.
To this end compilation flags can be defined
and assigned different default values.
They will be particularly useful in conjunction
with the forwarding mechanism described in \secref{sec:forward}.

For example, it may be useful to have a flag |\version|
which can be set to |draft| or |final|.
The document source will contain some conditional code
depending on the value of |\version|.
Suppose further, the flag should default to |final| for the main file
and to |draft| for child files
which is a natural assignment for editing the document.
This is achieved by placing the following code
in the preamble of the main document
(below the |\childdocmain| directive):
%
\begin{center}
\begin{tabular}{l}
|\ifchilddoc|\\
|\providecommand{\version}{draft}|\\
|\||else|\\
|\providecommand{\version}{final}|\\
|\||fi|
\end{tabular}
\end{center}
%
The definition by |\providecommand| makes sure
that previous definitions are not overwritten.
Further statements |\providecommand{\version}{...}|
can thus be added before the above code to override it.

For the main file, one might add a line
(between |\childdocmain| and the above block)
%
\begin{center}
|%\ifchilddoc\||else\providecommand{\version}{draft}\||fi|
\end{center}
%
which can be uncommented to produce a draft version.
Likewise one can add a line to the very top of a child file
(above the |\childdocof{|\textit{main}|}| directive)
%
\begin{center}
|%\providecommand{\version}{final}|
\end{center}
%
which can be uncommented to produce the final version of this child document.

%%%%%%%%%%%%%%%%%%%%%%%%%%%%%%%%%%%%%%%%%%%%%%%%%%%%%%%%%%%%%%%%%%%%%%%%%%%%%%%%
\subsection{Forwarding}
\label{sec:forward}

Different versions of the main or child documents
using compilation flags as described in \secref{sec:flags}
can be (permanently) stored in different files
for convenient compilation, viewing and distribution.
To this end, the package defines a command
to pass on compilation to a different file:

%%%%%%%%%%%%%%%%%%%%%%%%%%%%%%%%%%%%%%%%
\DescribeMacro{\childdocforward}
The command |\childdocforward| redirects processing to
another source file:
%
\begin{center}
\begin{tabular}{l}
|\input{childdoc.def}|\\
|\childdocforward[|\textit{main}|]{|\textit{dest}|}|\\
\end{tabular}
\end{center}
%
The argument \textit{dest} is the destination file
(without extension).
It should be the main file or one of the child files.
Note that further \textsf{childdoc} directives
such as |\childdocof| and |\childdocforward|
in the indicated file will be processed in this form.
The optional argument \textit{main}
passes on directly to the main file \textit{main}
while pretending to compile the child \textit{dest}.
This form behaves as if \textit{dest}
issues |\childdocof{|\textit{main}|}| right away,
and no further \textsf{childdoc} directives will be processed.

%%%%%%%%%%%%%%%%%%%%%%%%%%%%%%%%%%%%%%%%
\DescribeMacro{\...prefix}
In the alternative form |\childdocforwardprefix|,
%
\begin{center}
\begin{tabular}{l}
|\input{childdoc.def}|\\
|\childdocforwardprefix[|\textit{main}|]{|\textit{prefix}|}{|\textit{dest}|}|
\end{tabular}
\end{center}
%
the destination file is determined by a pattern
depending on the current file:
To make this work, the current file must be called
`{\textit{prefix}\hspace{0.2em}\textit{suffix}}'
with \textit{prefix} matching precisely the argument.
Processing is then passed on to the file
`{\textit{dest}\hspace{0.2em}\textit{suffix}}'.
Surely, the same effect is achieved by
directly specifying the
argument `{\textit{dest}\hspace{0.2em}\textit{suffix}}'
in the first form.
However, that requires to set up a different file
for each child. With the alternative form of the command
all these files can have exactly the same content
which simplifies setting them up and maintaining them.

For example, the following file |draft.tex|
with a compilation flag |\version| as described in \secref{sec:flags}
compiles the main document as a draft:
%
\begin{center}
\begin{tabular}{l}
|\def\version{draft}|\\
|\input{childdoc.def}|\\
|\childdocforward{|\textit{main}|}|
\end{tabular}
\end{center}
%
Likewise, the following files |final|\textit{nn}|.tex|
compile the final version of the child document
|child|\textit{nn}|.tex|:
%
\begin{center}
\begin{tabular}{l}
|\def\version{final}|\\
|\input{childdoc.def}|\\
|\childdocforwardprefix{final}{child}|
\end{tabular}
\end{center}
%

Note that when several versions of a main file and/or of each child file
are to be generated, it may be convenient to set up a |Makefile| or
shell script to automatise the process.

%%%%%%%%%%%%%%%%%%%%%%%%%%%%%%%%%%%%%%%%%%%%%%%%%%%%%%%%%%%%%%%%%%%%%%%%%%%%%%%%
\subsection{Command Line Processing}
\label{sec:commandline}

The effect of redirection files can also be achieved by invoking
the \LaTeX{} compiler with a more elaborate command line.
Most conveniently this should be done as part
of a shell script or a |Makefile|.

When using \textsf{childdoc} in the main file, the following
command lines effectively perform a redirection
(note that depending on the shell being used,
backslashes may have to be doubled: `|\|' $\to$ `|\\|'):
%
\begin{center}
|... -jobname "|\textit{target}|" |\\|"|[\textit{flags}]%
|\input{childdoc.def}\childdocforward[|\textit{main}|]{|\textit{dest}|}"|
\end{center}
%
Here \textit{target} is the name of the output file,
\textit{main} is the name of the main file
and \textit{dest} is the name of the main or child file to be processed
(all filenames without extensions).
The optional argument \textit{main} can be omitted
if \textit{main} matches \textit{dest}.
Optionally, compilation \textit{flags} can be defined via |\def| commands.
This command line makes the \TeX{} engine believe
it is compiling the file \textit{target}
whose content is specified as the latter parameter.
The provided code then forwards the processing to
\textit{main} or \textit{dest} as described in \secref{sec:forward}.

%%%%%%%%%%%%%%%%%%%%%%%%%%%%%%%%%%%%%%%%%%%%%%%%%%%%%%%%%%%%%%%%%%%%%%%%%%%%%%%%
\subsection{Include by Input}
\label{sec:input}

Including child documents by |\include| has some restrictions by design.
Most notably, the content of a child document always occupies
its own set of pages; pages cannot be shared between child documents.
Usually, this behaviour makes perfect sense
because each child document contain an essential part of the document.
However, in some situations it may be desirable to compose
a document from a collection of parts
without having mandatory page breaks between then.
For this case, the package
provides a mechanism to include parts
by |\input| which can also be processed individually.
However, by construction this mechanism
requires manual handling of the content to be output.

%%%%%%%%%%%%%%%%%%%%%%%%%%%%%%%%%%%%%%%%
\DescribeMacro{\ifchilddocmanual}
The main file should be prepared as usual, see \secref{sec:include}.
However, the document body must make a distinction
between processing of an individual part and of the main document, e.g.:
%
\begin{center}
\begin{tabular}{l}
|\ifchilddocmanual|\\
|\input{\childdocname}|\\
|\||else|\\
\textit{document body with }|\input{|\textit{part}|}|\\
|\||fi|
\end{tabular}
\end{center}
%
The conditional |\ifchilddocmanual| is true whenever
a part to be included by |\input| is being compiled,
and the name of the part is stored in |\childdocname|.

%%%%%%%%%%%%%%%%%%%%%%%%%%%%%%%%%%%%%%%%
\DescribeMacro{\childdocby}
Each part to be included by |\input| should start with:
%
\begin{center}
\begin{tabular}{l}
|\input{childdoc.def}|\\
|\childdocby{|\textit{main}|}|\\
\end{tabular}
\end{center}
%
The directive |\childdocby| is similar to |\childdocof|
described in \secref{sec:include},
but the subsequent selection of content must be done manually.
To that end, both |\ifchilddoc| and |\ifchilddocmanual|
will be true upon processing of a part,
and the name of the part is stored in |\childdocname|.
Note that |\jobname| will be set to the filename of the current part
so that each part receives an individual |.aux| file
that does not interfere with the |.aux| file(s) of the main document.
This behaviour can be altered by the alternative form
|\childdocby[*]{|\textit{main}|}| (with a non-empty optional argument)
which uses the |.aux| file of the main document
by setting |\jobname| to \textit{main}.

%%%%%%%%%%%%%%%%%%%%%%%%%%%%%%%%%%%%%%%%%%%%%%%%%%%%%%%%%%%%%%%%%%%%%%%%%%%%%%%%
\subsection{Driver Development}
\label{sec:driver}

The \textsf{childdoc} mechanism can also be use for the development
of definition files such as \LaTeX{} styles or classes.
This case differs from the above setup with multiple parts
included by |\include| in that no |\includeonly| should be invoked.
This can be achieved by starting the include file
(before |\ProvidesPackage|) with:
%
\begin{center}
\begin{tabular}{l}
|\input{childdoc.def}|\\
|\childdocforward{|\textit{main}|}|\\
\end{tabular}
\end{center}
%
or alternatively with:
%
\begin{center}
\begin{tabular}{l}
|\input{childdoc.def}|\\
|\childdocby{|\textit{main}|}|\\
\end{tabular}
\end{center}
%
Both forms have slightly different effects as described above.
The main file is prepared as usual, see \secref{sec:include}.

%%%%%%%%%%%%%%%%%%%%%%%%%%%%%%%%%%%%%%%%%%%%%%%%%%%%%%%%%%%%%%%%%%%%%%%%%%%%%%%%
\subsection{Legacy Detection}
\label{sec:detection}

The directive |\childdocmain| in the main file can detect
whether the complete document or merely a child is to be compiled
even without using the directive |\childdocof|.
This method is deprecated because it is less robust
and there is no compelling reason to use it;
it is merely provided for backward compatibility
and it may be removed in future versions.

If the detection mechanism is to be used,
it is mandatory to correctly specify
the filename of the main file as the argument of |\childdocmain|:
%
\begin{center}
\begin{tabular}{l}
|\input{childdoc.def}|\\
|\childdocmain{|\textit{main}|}|\\
\end{tabular}
\end{center}
%
If |\jobname| does not match the argument \textit{main} of |\childdocmain|,
it is assumed that |\jobname| points to the child file to be compiled.
When using |\childdocmain| with the main file specified as argument,
it suffices to start a child file
with just |\input{|\textit{main}|}|
without loading of the package and using |\childdocof|.
If instead all processing is done
with the appropriate \textsf{childdoc} directives,
the argument of \textit{main} of |\childdocmain| can be empty.

An alternative version of the command line processing described
in \secref{sec:commandline} using the detection mechanism reads:
%
\begin{center}
|... -jobname "|\textit{target}|" "|[\textit{flags}]%
[|\def\jobname{|\textit{dest}|}|]|\input{|\textit{main}|}"|
\end{center}

%%%%%%%%%%%%%%%%%%%%%%%%%%%%%%%%%%%%%%%%%%%%%%%%%%%%%%%%%%%%%%%%%%%%%%%%%%%%%%%%
\subsection{Manual Code}
\label{sec:manual}

In case one cannot be certain whether the definitions file |childdoc.def|
is installed on the target \TeX{} distribution
and one prefers not to ship it,
it is conceivable to paste a few relevant commands into the sources.

To that end, drop all statements |\input{childdoc.def}|
and perform the replacements as outlined below.
Instead of |\childdocmain{|\textit{main}|}| add the following code
to the top of the main file:
%
\begin{center}
\begin{tabular}{l}
|\||ifdefined\childdocname\endinput\||fi\newif\ifchilddoc|\\
|\edef\childdocname{\scantokens\expandafter{\jobname\noexpand}}|\\
|\def\childdocmain{|\textit{main}|}\||ifx\childdocmain\childdocname\||else|\\
|\childdoctrue\includeonly{\childdocname}\let\jobname\childdocmain\||fi|\\
\end{tabular}
\end{center}
%
Instead of |\childdocof{|\textit{main}|}| just include the main file
at the top of each child file:
%
\begin{center}
|\input{|\textit{main}|}|
\end{center}
%
A simple redirection |\childdocforward{|\textit{dest}|}| is achieved by:
%
\begin{center}
|\def\jobname{|\textit{dest}|}\input{\jobname}|
\end{center}
%
The redirection with prefix
|\childdocforwardprefix[|\textit{prefix}|]{|\textit{dest}|}|
is accomplished by:
%
\begin{center}
\begin{tabular}{l}
|{\edef\jobname{\scantokens\expandafter{\jobname\noexpand}}|\\
|\def\redirectjob |\textit{prefix}|#1~~~{\gdef\jobname{|\textit{dest}|#1}}|\\
|\expandafter\redirectjob\jobname~~~}\input{\jobname}|
\end{tabular}
\end{center}

In an alternative approach,
child documents can be compiled by a specific command line
without additional code or specific definitions:
%
\begin{center}
|... -jobname "|\textit{target}|" "|[\textit{flags}]%
|\includeonly{|\textit{dest}|}\input{|\textit{main}|}"|
\end{center}
%

%%%%%%%%%%%%%%%%%%%%%%%%%%%%%%%%%%%%%%%%%%%%%%%%%%%%%%%%%%%%%%%%%%%%%%%%%%%%%%%%
%%%%%%%%%%%%%%%%%%%%%%%%%%%%%%%%%%%%%%%%%%%%%%%%%%%%%%%%%%%%%%%%%%%%%%%%%%%%%%%%
\section{Information}

%%%%%%%%%%%%%%%%%%%%%%%%%%%%%%%%%%%%%%%%%%%%%%%%%%%%%%%%%%%%%%%%%%%%%%%%%%%%%%%%
\subsection{Copyright}

Copyright \copyright{} 2017--2018 Niklas Beisert

This work may be distributed and/or modified under the
conditions of the \LaTeX{} Project Public License, either version 1.3
of this license or (at your option) any later version.
The latest version of this license is in
  \url{http://www.latex-project.org/lppl.txt}
and version 1.3 or later is part of all distributions of \LaTeX{}
version 2005/12/01 or later.

This work has the LPPL maintenance status `maintained'.

The Current Maintainer of this work is Niklas Beisert.

This work consists of the files |README.txt|, |childdoc.ins| and |childdoc.dtx|
as well as the derived files |childdoc.def|, |cdocsamp.tex|
with |cdocsch1.tex|, |cdocsch2.tex|, |cdocspt3.tex|, |cdocspt4.tex|,
|cdocsdrf.tex|, |cdocsfn1.tex|, |cdocsfn2.tex|
as well as |childdoc.pdf|.

%%%%%%%%%%%%%%%%%%%%%%%%%%%%%%%%%%%%%%%%%%%%%%%%%%%%%%%%%%%%%%%%%%%%%%%%%%%%%%%%
\subsection{Files and Installation}

The package consists of the files:
%
\begin{center}
\begin{tabular}{ll}
    |README.txt|   & readme file \\
    |childdoc.ins| & installation file \\
    |childdoc.dtx| & source file \\
    |childdoc.def| & definition file \\
    |cdocsamp.tex| & sample main file \\
    |cdocsch1.tex| & sample include file \\
    |cdocsch2.tex| & sample include file \\
    |cdocspt3.tex| & sample part file \\
    |cdocspt4.tex| & sample part file \\
    |cdocsdrf.tex| & sample redirection file \\
    |cdocsfn1.tex| & sample redirection file \\
    |cdocsfn2.tex| & sample redirection file \\
    |childdoc.pdf| & manual
\end{tabular}
\end{center}
%
The distribution consists of the files
|README.txt|, |childdoc.ins| and |childdoc.dtx|.
%
\begin{itemize}
\item
Run (pdf)\LaTeX{} on |childdoc.dtx|
to compile the manual |childdoc.pdf| (this file).
\item
Run \LaTeX{} on |childdoc.ins| to create the definitions file |childdoc.def|
and the sample |cdocsamp.tex| with include files
|cdocsch1.tex|, |cdocsch2.tex|, |cdocspt3.tex|, |cdocspt4.tex|,
|cdocsdrf.tex|, |cdocsfn1.tex|, |cdocsfn2.tex|.
Then copy the file |childdoc.def| to an appropriate directory of your \LaTeX{}
distribution, e.g.\ \textit{texmf-root}|/tex/latex/childdoc|.
\end{itemize}

%%%%%%%%%%%%%%%%%%%%%%%%%%%%%%%%%%%%%%%%%%%%%%%%%%%%%%%%%%%%%%%%%%%%%%%%%%%%%%%%
\subsection{Related CTAN Packages}

There are several other packages which offer a similar functionality:
%
\begin{itemize}
\item
The packages
\href{http://ctan.org/pkg/docmute}{\textsf{docmute}},
\href{http://ctan.org/pkg/includex}{\textsf{includex}} and
\href{http://ctan.org/pkg/standalone}{\textsf{standalone}}
provide commands to include only the document body of
a child file thus allowing both files to be compiled individually.
\item
The packages \href{http://ctan.org/pkg/subdocs}{\textsf{subdocs}}
and \href{http://ctan.org/pkg/subfiles}{\textsf{subfiles}}
provide structures in which the main and child documents can be
encapsulated and allowing them to be compiled individually.
The inclusion mechanism is different from the conventional |\include|.
\item
The package \href{http://ctan.org/pkg/combine}{\textsf{combine}}
is an elaborate solution to combine several documents into one.
\end{itemize}
%
See also the CTAN topic \href{http://ctan.org/topic/subdocs}{\textsf{subdocs}}
for further related packages.
The present package differs from the above solutions in that
a document structure constructed with the conventional |\include| mechanism
just needs two extra commands at the top of every file
such that all constituent files can be compiled individually.

%%%%%%%%%%%%%%%%%%%%%%%%%%%%%%%%%%%%%%%%%%%%%%%%%%%%%%%%%%%%%%%%%%%%%%%%%%%%%%%%
%\subsection{Feature Suggestions}
%
%The following is a list of features which may be useful for future
%versions of this package:
%%
%\begin{itemize}
%\item
%\ldots
%\end{itemize}

%%%%%%%%%%%%%%%%%%%%%%%%%%%%%%%%%%%%%%%%%%%%%%%%%%%%%%%%%%%%%%%%%%%%%%%%%%%%%%%%
\subsection{Revision History}

%%%%%%%%%%%%%%%%%%%%%%%%%%%%%%%%%%%%%%%%
\paragraph{v2.0:} 2018/12/30

\begin{itemize}
\item
immediate forward processing
\item
added |\childdocby| mechanism
\item
manual restructured
\end{itemize}

%%%%%%%%%%%%%%%%%%%%%%%%%%%%%%%%%%%%%%%%
\paragraph{v1.6:} 2018/01/17

\begin{itemize}
\item
application for development of include files
\item
corrections to manual
\end{itemize}

%%%%%%%%%%%%%%%%%%%%%%%%%%%%%%%%%%%%%%%%
\paragraph{v1.5:} 2017/05/21

\begin{itemize}
\item
more complete structuring introduced
\item
|\childdocof| introduced
\item
|\childdoc| renamed to |\childdocmain|
\item
|\childredirect| renamed to |\childdocforward| and |\childdocforwardprefix|
and functionality expanded
\end{itemize}

%%%%%%%%%%%%%%%%%%%%%%%%%%%%%%%%%%%%%%%%
\paragraph{v1.0:} 2017/04/27

\begin{itemize}
\item
manual and install package
\item
first version published on CTAN
\end{itemize}

%%%%%%%%%%%%%%%%%%%%%%%%%%%%%%%%%%%%%%%%
\paragraph{v0.6:} 2017/04/26

\begin{itemize}
\item
redirection mechanism added
\end{itemize}

%%%%%%%%%%%%%%%%%%%%%%%%%%%%%%%%%%%%%%%%
\paragraph{v0.5:} 2017/04/26

\begin{itemize}
\item
functionality in definition file
\end{itemize}


%%%%%%%%%%%%%%%%%%%%%%%%%%%%%%%%%%%%%%%%%%%%%%%%%%%%%%%%%%%%%%%%%%%%%%%%%%%%%%%%
%%%%%%%%%%%%%%%%%%%%%%%%%%%%%%%%%%%%%%%%%%%%%%%%%%%%%%%%%%%%%%%%%%%%%%%%%%%%%%%%
%%%%%%%%%%%%%%%%%%%%%%%%%%%%%%%%%%%%%%%%%%%%%%%%%%%%%%%%%%%%%%%%%%%%%%%%%%%%%%%%
\appendix

\settowidth\MacroIndent{\rmfamily\scriptsize 000\ }

 \DocInput{childdoc.dtx}

\end{document}
%</driver>
% \fi
%
% %%%%%%%%%%%%%%%%%%%%%%%%%%%%%%%%%%%%%%%%%%%%%%%%%%%%%%%%%%%%%%%%%%%%%%%%%%%%%%
% %%%%%%%%%%%%%%%%%%%%%%%%%%%%%%%%%%%%%%%%%%%%%%%%%%%%%%%%%%%%%%%%%%%%%%%%%%%%%%
% \section{Sample}
%\iffalse
%<*samplemain>
%\fi
%
% The following presents a sample document
% with two chapters, two parts, a title page,
% a compile flag as well as three forwarding files to set the flag.
% It consists of eight |.tex| files:
% \begin{center}
% \begin{tabular}{ll}
% |cdocsamp.tex|&main file\\
% |cdocsch1.tex|&include file for chapter 1\\
% |cdocsch2.tex|&include file for chapter 2\\
% |cdocspt3.tex|&include file for part 3\\
% |cdocspt4.tex|&include file for part 4\\
% |cdocsdrf.tex|&forwarding file for main file in draft mode\\
% |cdocsfi1.tex|&forwarding file for final version of chapter 1\\
% |cdocsfi2.tex|&forwarding file for final version of chapter 2\\
% \end{tabular}
% \end{center}
% Each of the eight files can be compiled directly by the \LaTeX{} compiler.
%
% %%%%%%%%%%%%%%%%%%%%%%%%%%%%%%%%%%%%%%
% \paragraph{Main File.}
%
% The main file is called |cdocsamp.tex|.
%
% Load the \textsf{childdoc} definitions and
% declare the filename for the main document:
%    \begin{macrocode}
\input{childdoc.def}
\childdocmain{}
%    \end{macrocode}

% Optional override for |\version| flag:
%    \begin{macrocode}
%%\ifchilddoc\else\providecommand{\version}{draft}\fi
%    \end{macrocode}

% Define the default values for the |\version| flag
% (|final| for the main file and |draft| for childs):
%    \begin{macrocode}
\ifchilddoc
\providecommand{\version}{draft}
\else
\providecommand{\version}{final}
\fi
%    \end{macrocode}

% Load the standard document class:
%    \begin{macrocode}
\documentclass[12pt]{article}
%    \end{macrocode}

% Start the document body:
%    \begin{macrocode}
\begin{document}
%    \end{macrocode}

% Declare a title page.
% Print title, part of document being processed and version flag:
%    \begin{macrocode}
\addtocounter{page}{-1}
\begin{center}
{\LARGE\bfseries{}childdoc example\par}
\vspace{1cm}
\ifchilddoc
\ifchilddocmanual part\else chapter\fi:
`\childdocname' of `\childdocjob'\par
\else
main document: `\childdocjob'\par
\fi
version: \version\par
\end{center}
\newpage
%    \end{macrocode}

% Manually include selected file,
% otherwise process as usual:
%    \begin{macrocode}
\ifchilddocmanual
\section*{part `\childdocname'}
\input{\childdocname}
\else
%    \end{macrocode}

% Include the two chapters:
%    \begin{macrocode}
\include{cdocsch1}
\include{cdocsch2}
%    \end{macrocode}

% Include the two parts unless only chapters should be displayed:
%    \begin{macrocode}
\ifchilddoc\else
\section{part three}
\input{cdocspt3}
\section{part four}
\input{cdocspt4}
\fi
%    \end{macrocode}

% Process as usual until here:
%    \begin{macrocode}
\fi
%    \end{macrocode}

% End of document body:
%    \begin{macrocode}
\end{document}
%    \end{macrocode}
%\iffalse
%</samplemain>
%\fi
%
% %%%%%%%%%%%%%%%%%%%%%%%%%%%%%%%%%%%%%%
% \paragraph{Chapter Include Files.}
%
% The include files are called |cdocsch1.tex| and |cdocsch2.tex|.
%
%\iffalse
%<*samplechap1|samplechap2>
%\fi

% Optional override for |\version| flag:
%    \begin{macrocode}
%%\providecommand{\version}{final}
%    \end{macrocode}

% Include the main document:
%    \begin{macrocode}
\input{childdoc.def}
\childdocof{cdocsamp}
%    \end{macrocode}

%\iffalse
%</samplechap1|samplechap2>
%\fi
%
%\iffalse
%<*samplechap1>
%\fi
% Some text for chapter 1:
%    \begin{macrocode}
\section{one}
some text in chapter one
%    \end{macrocode}

%\iffalse
%</samplechap1>
%\fi
% Some text for chapter 2:
%\iffalse
%<*samplechap2>
%\fi
%    \begin{macrocode}
\section{two}
more text in chapter two
%    \end{macrocode}

%\iffalse
%</samplechap2>
%\fi
%
% %%%%%%%%%%%%%%%%%%%%%%%%%%%%%%%%%%%%%%
% \paragraph{Part Include Files.}
%
% The include files are called |cdocspt3.tex| and |cdocspt4.tex|.
%
%\iffalse
%<*samplepart3|samplepart4>
%\fi

% Optional override for |\version| flag:
%    \begin{macrocode}
%%\providecommand{\version}{final}
%    \end{macrocode}

% Include the main document:
%    \begin{macrocode}
\input{childdoc.def}
\childdocby{cdocsamp}
%    \end{macrocode}

%\iffalse
%</samplepart3|samplepart4>
%\fi
%
%\iffalse
%<*samplepart3>
%\fi
% Some text for part 3:
%    \begin{macrocode}
some text in part three
%    \end{macrocode}

%\iffalse
%</samplepart3>
%\fi
% Some text for part 4:
%\iffalse
%<*samplepart4>
%\fi
%    \begin{macrocode}
more text in part four
%    \end{macrocode}

%\iffalse
%</samplepart4>
%\fi
%
% %%%%%%%%%%%%%%%%%%%%%%%%%%%%%%%%%%%%%%
% \paragraph{Forwarding for a Complete Draft.}
%
% The following forwarding file |cdocsdrf.tex|
% compiles the main document in draft mode:
%\iffalse
%<*sampledraft>
%\fi
%    \begin{macrocode}
\def\version{draft}
\input{childdoc.def}
\childdocforward{cdocsamp}
%    \end{macrocode}

%\iffalse
%</sampledraft>
%\fi
%
% %%%%%%%%%%%%%%%%%%%%%%%%%%%%%%%%%%%%%%
% \paragraph{Forwarding for Final Version of the Chapters.}
%
% The following forwarding files |cdocsfn1.tex| and |cdocsfn2.tex|
% (with identical content)
% compile the final versions of the child documents
% |cdocsch1.tex| and |cdocsch2.tex|, respectively:
%\iffalse
%<*samplefinal>
%\fi
%    \begin{macrocode}
\def\version{final}
\input{childdoc.def}
\childdocforwardprefix[cdocsamp]{cdocsfn}{cdocsch}
%    \end{macrocode}

%\iffalse
%</samplefinal>
%\fi
%
% %%%%%%%%%%%%%%%%%%%%%%%%%%%%%%%%%%%%%%
% \paragraph{Command Line Processing.}
%
% The following three command lines generate the output files
% |cdocscld|, |cdocscl1| and |cdocscl2|
% which should be identical to
% |cdocsdrf|, |cdocsch1| and |cdocsfn2|, respectively:
% \begin{center}
% \begin{tabular}{l}
% |latex -jobname cdocscld \|\\
% |  "\def\version{draft}\input{childdoc.def}\childdocforward{cdocsamp}"|\\
% |latex -jobname cdocscl1 \|\\
% |  "\input{childdoc.def}\childdocforward[cdocsamp]{cdocsch1}"|\\
% |latex -jobname cdocscl2 \|\\
% |  "\def\version{final}\input{childdoc.def}\childdocforward{cdocsch2}"|
% \end{tabular}
% \end{center}
% Note that the trailing backslash on each first line
% merely continues the input to the second line
% (for convenient cut ant paste).
% Furthermore, the command |latex| can be replaced by any
% of its alternative versions such as |pdflatex|.
%
% %%%%%%%%%%%%%%%%%%%%%%%%%%%%%%%%%%%%%%%%%%%%%%%%%%%%%%%%%%%%%%%%%%%%%%%%%%%%%%
% %%%%%%%%%%%%%%%%%%%%%%%%%%%%%%%%%%%%%%%%%%%%%%%%%%%%%%%%%%%%%%%%%%%%%%%%%%%%%%
% \section{Implementation}
%\iffalse
%<*package>
%\fi
%
% This section describes the definitions file |childdoc.def|.

% The definitions cannot be loaded using |\usepackage| or |\RequirePackage|
% which has a mechanism to prevent loading a style file more than once.
% When loading the definitions by means of |\input|
% multiple instances have to be prevented manually:
%\iffalse
%This code needs to be before the `\ProvidesFile' directive
%which is defined at the beginning of this file.
%Therefore it is also placed there and commented out here.
%</package>
%<*discard>
%\fi
%    \begin{macrocode}
\ifdefined\childdocmain\endinput\fi
%    \end{macrocode}
%\iffalse
%</discard>
%<*package>
%\fi
%
% \macro{\ifchilddoc}
% \macro{\ifchilddocmanual}
% The conditional |\ifchilddoc| tells whether a
% child (true) or main (false) document is being compiled.
% The conditional |\ifchilddocmanual| tells whether
% the |\includeonly| mechanism is used (false) or
% the selection of child files must be performed manually (true).
% The definitions initialise to false:
%    \begin{macrocode}
\newif\ifchilddoc
\newif\ifchilddocmanual
%    \end{macrocode}

% \macro{\childdocname}
% \macro{\childdocjob}
% The macro |\childdocname| stores the name of the main document
% to be compiled. The macro |\childdocjob| stores the name of
% the document on which the \LaTeX{} compiler was originally invoked.
% The content of |\jobname| cannot be compared
% to filenames specified in the source due to different catcodes.
% The following code rescans |\jobname|, stores the result
% in |\childdocname| and saves a copy in |\childdocjob|:
%    \begin{macrocode}
\edef\childdocname{\scantokens\expandafter{\jobname\noexpand}}
\let\childdocjob\childdocname
%    \end{macrocode}

% \macro{\childdocdisable}
% The macro |\childdocdisable| prevents the main file
% from being processed more than once.
% At this stage, the main document command |\childdocmain|
% is assumed to be called once again where it should do nothing.
% Any subsequent call to it should prevent
% a secondary processing of the main document
% It overwrites the forwarding commands
% |\childdocof| and |\childdocforward|
% with empty macros to prevent further inclusions of the main document:
%    \begin{macrocode}
\newcommand{\childdocdisable}
{
  \renewcommand{\childdocmain}[1]{\renewcommand{\childdocmain}[1]{\endinput}}
  \renewcommand{\childdocof}[1]{}
  \renewcommand{\childdocby}[2][]{}
  \renewcommand{\childdocforward}[2][]{}
  \renewcommand{\childdocdisable}{}
}
%    \end{macrocode}

% \macro{\childdocmain}
% The macro |\childdocmain| is to be called at the top of the main file
% with nothing or the main filename (without extension) as argument.
% First, it breaks loops.
% If the argument is not empty and does not match |\childdocname|
% (which is set by the first inclusion of |childdoc.def|),
% |\ifchilddoc| is set to true, |\includeonly| is applied to the child file
% and |\jobname| is set to the main file
% (for proper handling of |.aux| files):
%    \begin{macrocode}
\newcommand{\childdocmain}[1]
{
  \childdocdisable\childdocmain{}
  \if?#1?\else
    \begingroup
      \def\childdoctmp{#1}
      \ifx\childdoctmp\childdocname
        \def\childdoctmp{}
      \else
        \def\childdoctmp
        {
          \childdoctrue
          \includeonly{\childdocname}
          \def\childdocjob{#1}
          \def\jobname{#1}
        }
      \fi
      \expandafter
    \endgroup
    \childdoctmp
  \fi
}
%    \end{macrocode}

% \macro{\childdocof}
% The command |\childdocof| redirects
% compilation to the main file |#1|.
%    \begin{macrocode}
\newcommand{\childdocof}[1]
{
  \childdocdisable
  \childdoctrue
  \includeonly{\childdocname}
  \def\jobname{#1}
  \def\childdocjob{#1}
  \input{#1}
}
%    \end{macrocode}

% \macro{\childdocby}
% The command |\childdocby| ....
%    \begin{macrocode}
\newcommand{\childdocby}[2][]
{
  \childdocdisable
  \childdoctrue
  \childdocmanualtrue
  \if?#1?\else
    \def\jobname{#2}
  \fi
  \def\childdocjob{#2}
  \input{#2}
  \endinput
}
%    \end{macrocode}

% \macro{\childdocforward}
% The command |\childdocforward| redirects
% compilation to the main file or
% (if the optional argument is given) a child file.
% Parameters are set as if the main file
% or a child file starting with |\childdocof| was compiled.
% Then compilation is handed over to the main file:
%    \begin{macrocode}
\newcommand{\childdocforward}[2][]
{
  \begingroup
    \if?#1?
      \def\childdoctmp
      {
        \def\childdocname{#2}
        \def\childdocjob{#2}
        \def\jobname{#2}
        \input{#2}
        \endinput
      }
    \else
      \def\childdoctmp
      {
        \childdocdisable
        \def\childdocname{#2}
        \childdoctrue
        \includeonly{#2}
        \def\childdocjob{#1}
        \def\jobname{#1}
        \input{#1}
        \endinput
      }
    \fi
    \expandafter
  \endgroup
  \childdoctmp
}
%    \end{macrocode}

% \macro{\childdocforwardprefix}
% The command |\childdocforwardprefix| redirects
% compilation to the main or a child file by means of a pattern.
% The prefix |#1| in the current filename is replaced by |#2|
% and the suffix of the current filename is kept
% (it is assumed that the filename does not contain the substring `|~~~|'
% which is used as a delimiter).
% Compilation is handed over to the new file by |\childdocforward|:
%    \begin{macrocode}
\newcommand{\childdocforwardprefix}[3][]
{
  \begingroup
    \def\childdocextract #2##1~~~{\def\childdoctmp{\childdocforward[#1]{#3##1}}}
    \expandafter\childdocextract\childdocname~~~
    \expandafter
  \endgroup
  \childdoctmp
}
%    \end{macrocode}

% \macro{\childdoc}
% The deprecated macro |\childdoc| is a legacy version of |\childdocmain|:
%    \begin{macrocode}
\newcommand{\childdoc}{\childdocmain}
%    \end{macrocode}

% \macro{\childdocredirect}
% The deprecated macro |\childdocredirect| is a legacy version
% of |\childdocforward| and |\childdocforwardprefix|:
%    \begin{macrocode}
\newcommand{\childdocredirect}[2][]
{
  \begingroup
    \if?#1?
      \def\childdoctmp{\childdocforward{#2}}
    \else
      \def\childdoctmp{\childdocforwardprefix{#1}{#2}}
    \fi
    \expandafter
  \endgroup
  \childdoctmp
}
%    \end{macrocode}

%\iffalse
%</package>
%\fi
%
\endinput
|\\
|\childdocforward{|\textit{main}|}|\\
\end{tabular}
\end{center}
%
or alternatively with:
%
\begin{center}
\begin{tabular}{l}
|% \iffalse
%
% childdoc.dtx Copyright (C) 2017-2018 Niklas Beisert
%
% This work may be distributed and/or modified under the
% conditions of the LaTeX Project Public License, either version 1.3
% of this license or (at your option) any later version.
% The latest version of this license is in
%   http://www.latex-project.org/lppl.txt
% and version 1.3 or later is part of all distributions of LaTeX
% version 2005/12/01 or later.
%
% This work has the LPPL maintenance status `maintained'.
%
% The Current Maintainer of this work is Niklas Beisert.
%
% This work consists of the files childdoc.dtx and childdoc.ins
% and the derived files childdoc.def and cdocsamp.tex with
% cdocsch1.tex, cdocsch2.tex, cdocsdrf.tex, cdocsfn1.tex, cdocsfn2.tex.
%
%<package>\ifdefined\childdocmain\endinput\fi
%<package>\ProvidesFile{childdoc.def}[2018/12/30 v2.0 child document driver]
%<samplemain>\ProvidesFile{cdocsamp.tex}[2018/12/30 v2.0 sample for childdoc]
%<*driver>
%\ProvidesFile{childdoc.drv}[2018/12/30 v2.0 childdoc reference manual file]
\PassOptionsToClass{10pt,a4paper}{article}
\documentclass{ltxdoc}

\usepackage[margin=35mm]{geometry}
\usepackage{hyperref}
\usepackage{hyperxmp}
\usepackage[usenames]{color}

\hypersetup{colorlinks=true}
\hypersetup{pdfstartview=FitH}
\hypersetup{pdfpagemode=UseNone}
\hypersetup{pdfsource={}}
\hypersetup{pdflang={en-UK}}
\hypersetup{pdfcopyright={Copyright 2017-2018 Niklas Beisert.
  This work may be distributed and/or modified under the
  conditions of the LaTeX Project Public License, either version 1.3
  of this license or (at your option) any later version.}}
\hypersetup{pdflicenseurl={http://www.latex-project.org/lppl.txt}}
\hypersetup{pdfcontactaddress={ETH Zurich, ITP, HIT K,
  Wolfgang-Pauli-Strasse 27}}
\hypersetup{pdfcontactpostcode={8093}}
\hypersetup{pdfcontactcity={Zurich}}
\hypersetup{pdfcontactcountry={Switzerland}}
\hypersetup{pdfcontactemail={nbeisert@itp.phys.ethz.ch}}
\hypersetup{pdfcontacturl={http://people.phys.ethz.ch/\xmptilde nbeisert/}}

\newcommand{\secref}[1]{\hyperref[#1]{section \ref*{#1}}}

\parskip1ex
\parindent0pt
\let\olditemize\itemize
\def\itemize{\olditemize\parskip0pt}

\begin{document}

\title{The \textsf{childdoc} Package}
\hypersetup{pdftitle={The childdoc Package}}
\author{Niklas Beisert\\[2ex]
  Institut f\"ur Theoretische Physik\\
  Eidgen\"ossische Technische Hochschule Z\"urich\\
  Wolfgang-Pauli-Strasse 27, 8093 Z\"urich, Switzerland\\[1ex]
  \href{mailto:nbeisert@itp.phys.ethz.ch}
  {\texttt{nbeisert@itp.phys.ethz.ch}}}
\hypersetup{pdfauthor={Niklas Beisert}}
\hypersetup{pdfsubject={Manual for the LaTeX2e Package childdoc}}
\date{30 December 2018, \textsf{v2.0}}
\maketitle

\begin{abstract}\noindent
\textsf{childdoc} is a \LaTeXe{} package
that enables the direct compilation
of document sections included by |\include|
to individual files.
\end{abstract}

\begingroup
\parskip0ex
\tableofcontents
\endgroup

%%%%%%%%%%%%%%%%%%%%%%%%%%%%%%%%%%%%%%%%%%%%%%%%%%%%%%%%%%%%%%%%%%%%%%%%%%%%%%%%
%%%%%%%%%%%%%%%%%%%%%%%%%%%%%%%%%%%%%%%%%%%%%%%%%%%%%%%%%%%%%%%%%%%%%%%%%%%%%%%%
\section{Introduction}

\LaTeX{} provides a mechanism to structure a large document (such as a book)
into a main file and several child files (containing the chapters)
using the |\include| command.
This mechanism is beneficial for documents
which span hundreds of pages in order to
make the source file(s) more manageable.
Moreover, compilation can be restricted to
selected child files by means of the |\includeonly| command.
The latter feature can be used to reduce the compilation time while editing
(this was significantly more useful in the earlier days of \LaTeX{})
or to generate a smaller document which is easier to navigate.
Another application of |\includeonly| is to generate
documents consisting of selected parts of the complete document.

However, there are a few drawbacks of the plain |\include| mechanism:
\begin{itemize}
\item
The child files cannot be compiled on their own,
they can only be compiled via the main file.
A naive editing environment
(such as a text editor with an option
to have the current file processed by \LaTeX)
may require one to switch to the main file before compiling;
attempting to compile the child file produces errors.
\item
The main file must be modified (each time)
to adjust the |\includeonly| command
to the present needs. This easily leaves the main file in a messy state.
\item
The generated document will always carry the filename
of the main document. This is inconvenient if
several child files are to be compiled and
to be kept for distribution.
\end{itemize}

The present package provides a simple interface
to make child files individually compilable by \LaTeX{}.
Compiling a child file then has the same effect as compiling
the main file with an |\includeonly| command
to select the appropriate child.
Moreover the generated document will carry the name of the child
rather than the main file.
This resolves all three above issues.

This feature is meant to make the editing of books,
thesis documents and lecture notes somewhat more convenient.
However, the package can also be used efficiently for
composing a series of documents (such as exercise sheets)
which are typically distributed individually.
It then assists the author in generating the individual documents
(potentially in different versions)
as well as a document containing the collected series.
Another application is in developing style files
or other kinds of included material
where compilation of the style file could redirect
to a sample or test file.

%%%%%%%%%%%%%%%%%%%%%%%%%%%%%%%%%%%%%%%%%%%%%%%%%%%%%%%%%%%%%%%%%%%%%%%%%%%%%%%%
%%%%%%%%%%%%%%%%%%%%%%%%%%%%%%%%%%%%%%%%%%%%%%%%%%%%%%%%%%%%%%%%%%%%%%%%%%%%%%%%
\section{Usage}

First of all, the package \textsf{childdoc} is \emph{not} a standard
\LaTeXe{} |.sty| style file! Therefore it needs to be invoked in
a non-standard way.

%%%%%%%%%%%%%%%%%%%%%%%%%%%%%%%%%%%%%%%%%%%%%%%%%%%%%%%%%%%%%%%%%%%%%%%%%%%%%%%%
\subsection{Included Files}
\label{sec:include}

%%%%%%%%%%%%%%%%%%%%%%%%%%%%%%%%%%%%%%%%
\DescribeMacro{\childdocmain}
To use the package, add the commands
\begin{center}
\begin{tabular}{l}
|\input{childdoc.def}|\\
|\childdocmain{}|\\
\end{tabular}
\end{center}
at the very top of the main \LaTeX{} file,
in particular \emph{before} the |\documentclass| statement!
The argument of |\childdocmain| should be left empty
(but it must be present).

%%%%%%%%%%%%%%%%%%%%%%%%%%%%%%%%%%%%%%%%
\DescribeMacro{\childdocof}
Furthermore, add the commands
\begin{center}
\begin{tabular}{l}
|\input{childdoc.def}|\\
|\childdocof{|\textit{main}|}|\\
\end{tabular}
\end{center}
at the top of every child file \textit{child}
which is included by |\include{|\textit{child}|}|
from within the main file
(or at least for those files to be compiled individually).
The argument \textit{main} must be the filename of the main file.

There are a couple of
considerations in setting up the main and child documents:

%%%%%%%%%%%%%%%%%%%%%%%%%%%%%%%%%%%%%%%%
\paragraph{Restrictions.}

Please note the following restrictions:
\begin{itemize}
\item
|\childdocmain| must be called with one argument \textit{main}
to ensure compatibility with earlier version of the package.
It must either be empty (|\childdocmain{}|)
or precisely match the filename of the main file in which it is specified.
See \secref{sec:detection} for further information.
\item
The filename \textit{main} must be specified without the |.tex| extension.
\item
The filename \textit{main} is case sensitive
(even in case-insensitive file systems)
due to internal string comparison.
\item
The argument \textit{main} should be fully expanded, it cannot be a macro.
\item
Subdirectories and special characters should be avoided in filenames.
\item
The command |\childdocmain{|\textit{main}|}| must be followed by a whitespace.
It should not be followed immediately by another command
or by a comment mark `|%|'.
This is because the \TeX{} parser reads the token immediately following
the argument of |\childdocmain| and puts it
at the beginning of every child section;
however, a white\-space is ignored.
\end{itemize}

%%%%%%%%%%%%%%%%%%%%%%%%%%%%%%%%%%%%%%%%
\paragraph{Content of Main File.}

It is advisable to place all content in the child files included by |\include|.
Any output contained in the main file will appear in all child documents
unless suppressed manually;
it cannot be suppressed automatically by the |\includeonly| directive
and thus should normally be avoided.
A method to include some content in the main file
by means of conditional processing is described in \secref{sec:conditional}.

%%%%%%%%%%%%%%%%%%%%%%%%%%%%%%%%%%%%%%%%
\paragraph{Page Numbering.}

When only a part of the document is compiled,
the appropriate numbering of pages
(as well as other status parameters)
is determined from the |.aux| files.
The latter contain information from previous passes.
However this information needs to propagate through
all intermediate child documents.
Therefore the page numbering in child documents may well
be inconsistent until the complete document is compiled at least once.

A useful (if unconventional) way to always ensure a consistent
page numbering is to restart the numbering in each child document
and denote the pages by `\textit{child}|.|\textit{page}'
where \textit{child} represents the chapter/section number of the child file.
This can be achieved by the command
|\numberwithin{page}{|\textit{child}|}|
of the \textsf{amsmath} package
where \textit{child} can be |chapter| or |section|
depending on the chosen structuring.
Alternatively, one can modify the macro |\thepage| appropriately
and reset the counter |page| at the start of each child file.

%%%%%%%%%%%%%%%%%%%%%%%%%%%%%%%%%%%%%%%%%%%%%%%%%%%%%%%%%%%%%%%%%%%%%%%%%%%%%%%%
\subsection{Conditional Processing}
\label{sec:conditional}

The package provides a mechanism to compile different versions
of a document. To customise the versions further some conditional processing
can come in handy to distinguish which version is being compiled.
The package provides two macros to describe the compilation context:

%%%%%%%%%%%%%%%%%%%%%%%%%%%%%%%%%%%%%%%%
\DescribeMacro{\ifchilddoc}
The conditional |\ifchilddoc| distinguishes between the compilation of
child documents and the main document:
%
\begin{center}
|\ifchilddoc |\textit{child-code}| |[|\||else |\textit{main-code}]| \||fi|
\end{center}

%%%%%%%%%%%%%%%%%%%%%%%%%%%%%%%%%%%%%%%%
\DescribeMacro{\childdocname}
\DescribeMacro{\childdocjob}
The macro |\childdocname| contains the filename (without extension)
of the main or child file being processed.
Note that |\childdocjob| will always contain the name of the main file.

%%%%%%%%%%%%%%%%%%%%%%%%%%%%%%%%%%%%%%%%
\paragraph{Title Page.}

Conditional processing can be used to include a title or banner page
in the main document when proper precautions are taken.
Importantly, the code in the main file should ensure that the page counter
(as well as other status parameters which are stored in the |.aux| files)
takes the same value after the conditional processing.
Otherwise the page numbers may take divergent values
depending on which part is compiled.

For example, a title page could be declared by:
%
\begin{center}
\begin{tabular}{l}
|\ifchilddoc\||else|\\
|\addtocounter{page}{-1}|\\
\textit{code for title page}\\
|\newpage|\\
|\||fi|
\end{tabular}
\end{center}
%
A banner page for the child documents can be generated by:
%
\begin{center}
\begin{tabular}{l}
|\ifchilddoc|\\
|\addtocounter{page}{-1}|\\
\textit{code for banner page}\\
|\newpage|\\
|\||fi|
\end{tabular}
\end{center}
%
Here one could write a message such as:
\begin{center}
|This is the part \childdocname{} of \childdocjob{}.|
\end{center}

%%%%%%%%%%%%%%%%%%%%%%%%%%%%%%%%%%%%%%%%%%%%%%%%%%%%%%%%%%%%%%%%%%%%%%%%%%%%%%%%
\subsection{Flags}
\label{sec:flags}

The package makes it easy to generate different versions
of the main or child documents.
To this end compilation flags can be defined
and assigned different default values.
They will be particularly useful in conjunction
with the forwarding mechanism described in \secref{sec:forward}.

For example, it may be useful to have a flag |\version|
which can be set to |draft| or |final|.
The document source will contain some conditional code
depending on the value of |\version|.
Suppose further, the flag should default to |final| for the main file
and to |draft| for child files
which is a natural assignment for editing the document.
This is achieved by placing the following code
in the preamble of the main document
(below the |\childdocmain| directive):
%
\begin{center}
\begin{tabular}{l}
|\ifchilddoc|\\
|\providecommand{\version}{draft}|\\
|\||else|\\
|\providecommand{\version}{final}|\\
|\||fi|
\end{tabular}
\end{center}
%
The definition by |\providecommand| makes sure
that previous definitions are not overwritten.
Further statements |\providecommand{\version}{...}|
can thus be added before the above code to override it.

For the main file, one might add a line
(between |\childdocmain| and the above block)
%
\begin{center}
|%\ifchilddoc\||else\providecommand{\version}{draft}\||fi|
\end{center}
%
which can be uncommented to produce a draft version.
Likewise one can add a line to the very top of a child file
(above the |\childdocof{|\textit{main}|}| directive)
%
\begin{center}
|%\providecommand{\version}{final}|
\end{center}
%
which can be uncommented to produce the final version of this child document.

%%%%%%%%%%%%%%%%%%%%%%%%%%%%%%%%%%%%%%%%%%%%%%%%%%%%%%%%%%%%%%%%%%%%%%%%%%%%%%%%
\subsection{Forwarding}
\label{sec:forward}

Different versions of the main or child documents
using compilation flags as described in \secref{sec:flags}
can be (permanently) stored in different files
for convenient compilation, viewing and distribution.
To this end, the package defines a command
to pass on compilation to a different file:

%%%%%%%%%%%%%%%%%%%%%%%%%%%%%%%%%%%%%%%%
\DescribeMacro{\childdocforward}
The command |\childdocforward| redirects processing to
another source file:
%
\begin{center}
\begin{tabular}{l}
|\input{childdoc.def}|\\
|\childdocforward[|\textit{main}|]{|\textit{dest}|}|\\
\end{tabular}
\end{center}
%
The argument \textit{dest} is the destination file
(without extension).
It should be the main file or one of the child files.
Note that further \textsf{childdoc} directives
such as |\childdocof| and |\childdocforward|
in the indicated file will be processed in this form.
The optional argument \textit{main}
passes on directly to the main file \textit{main}
while pretending to compile the child \textit{dest}.
This form behaves as if \textit{dest}
issues |\childdocof{|\textit{main}|}| right away,
and no further \textsf{childdoc} directives will be processed.

%%%%%%%%%%%%%%%%%%%%%%%%%%%%%%%%%%%%%%%%
\DescribeMacro{\...prefix}
In the alternative form |\childdocforwardprefix|,
%
\begin{center}
\begin{tabular}{l}
|\input{childdoc.def}|\\
|\childdocforwardprefix[|\textit{main}|]{|\textit{prefix}|}{|\textit{dest}|}|
\end{tabular}
\end{center}
%
the destination file is determined by a pattern
depending on the current file:
To make this work, the current file must be called
`{\textit{prefix}\hspace{0.2em}\textit{suffix}}'
with \textit{prefix} matching precisely the argument.
Processing is then passed on to the file
`{\textit{dest}\hspace{0.2em}\textit{suffix}}'.
Surely, the same effect is achieved by
directly specifying the
argument `{\textit{dest}\hspace{0.2em}\textit{suffix}}'
in the first form.
However, that requires to set up a different file
for each child. With the alternative form of the command
all these files can have exactly the same content
which simplifies setting them up and maintaining them.

For example, the following file |draft.tex|
with a compilation flag |\version| as described in \secref{sec:flags}
compiles the main document as a draft:
%
\begin{center}
\begin{tabular}{l}
|\def\version{draft}|\\
|\input{childdoc.def}|\\
|\childdocforward{|\textit{main}|}|
\end{tabular}
\end{center}
%
Likewise, the following files |final|\textit{nn}|.tex|
compile the final version of the child document
|child|\textit{nn}|.tex|:
%
\begin{center}
\begin{tabular}{l}
|\def\version{final}|\\
|\input{childdoc.def}|\\
|\childdocforwardprefix{final}{child}|
\end{tabular}
\end{center}
%

Note that when several versions of a main file and/or of each child file
are to be generated, it may be convenient to set up a |Makefile| or
shell script to automatise the process.

%%%%%%%%%%%%%%%%%%%%%%%%%%%%%%%%%%%%%%%%%%%%%%%%%%%%%%%%%%%%%%%%%%%%%%%%%%%%%%%%
\subsection{Command Line Processing}
\label{sec:commandline}

The effect of redirection files can also be achieved by invoking
the \LaTeX{} compiler with a more elaborate command line.
Most conveniently this should be done as part
of a shell script or a |Makefile|.

When using \textsf{childdoc} in the main file, the following
command lines effectively perform a redirection
(note that depending on the shell being used,
backslashes may have to be doubled: `|\|' $\to$ `|\\|'):
%
\begin{center}
|... -jobname "|\textit{target}|" |\\|"|[\textit{flags}]%
|\input{childdoc.def}\childdocforward[|\textit{main}|]{|\textit{dest}|}"|
\end{center}
%
Here \textit{target} is the name of the output file,
\textit{main} is the name of the main file
and \textit{dest} is the name of the main or child file to be processed
(all filenames without extensions).
The optional argument \textit{main} can be omitted
if \textit{main} matches \textit{dest}.
Optionally, compilation \textit{flags} can be defined via |\def| commands.
This command line makes the \TeX{} engine believe
it is compiling the file \textit{target}
whose content is specified as the latter parameter.
The provided code then forwards the processing to
\textit{main} or \textit{dest} as described in \secref{sec:forward}.

%%%%%%%%%%%%%%%%%%%%%%%%%%%%%%%%%%%%%%%%%%%%%%%%%%%%%%%%%%%%%%%%%%%%%%%%%%%%%%%%
\subsection{Include by Input}
\label{sec:input}

Including child documents by |\include| has some restrictions by design.
Most notably, the content of a child document always occupies
its own set of pages; pages cannot be shared between child documents.
Usually, this behaviour makes perfect sense
because each child document contain an essential part of the document.
However, in some situations it may be desirable to compose
a document from a collection of parts
without having mandatory page breaks between then.
For this case, the package
provides a mechanism to include parts
by |\input| which can also be processed individually.
However, by construction this mechanism
requires manual handling of the content to be output.

%%%%%%%%%%%%%%%%%%%%%%%%%%%%%%%%%%%%%%%%
\DescribeMacro{\ifchilddocmanual}
The main file should be prepared as usual, see \secref{sec:include}.
However, the document body must make a distinction
between processing of an individual part and of the main document, e.g.:
%
\begin{center}
\begin{tabular}{l}
|\ifchilddocmanual|\\
|\input{\childdocname}|\\
|\||else|\\
\textit{document body with }|\input{|\textit{part}|}|\\
|\||fi|
\end{tabular}
\end{center}
%
The conditional |\ifchilddocmanual| is true whenever
a part to be included by |\input| is being compiled,
and the name of the part is stored in |\childdocname|.

%%%%%%%%%%%%%%%%%%%%%%%%%%%%%%%%%%%%%%%%
\DescribeMacro{\childdocby}
Each part to be included by |\input| should start with:
%
\begin{center}
\begin{tabular}{l}
|\input{childdoc.def}|\\
|\childdocby{|\textit{main}|}|\\
\end{tabular}
\end{center}
%
The directive |\childdocby| is similar to |\childdocof|
described in \secref{sec:include},
but the subsequent selection of content must be done manually.
To that end, both |\ifchilddoc| and |\ifchilddocmanual|
will be true upon processing of a part,
and the name of the part is stored in |\childdocname|.
Note that |\jobname| will be set to the filename of the current part
so that each part receives an individual |.aux| file
that does not interfere with the |.aux| file(s) of the main document.
This behaviour can be altered by the alternative form
|\childdocby[*]{|\textit{main}|}| (with a non-empty optional argument)
which uses the |.aux| file of the main document
by setting |\jobname| to \textit{main}.

%%%%%%%%%%%%%%%%%%%%%%%%%%%%%%%%%%%%%%%%%%%%%%%%%%%%%%%%%%%%%%%%%%%%%%%%%%%%%%%%
\subsection{Driver Development}
\label{sec:driver}

The \textsf{childdoc} mechanism can also be use for the development
of definition files such as \LaTeX{} styles or classes.
This case differs from the above setup with multiple parts
included by |\include| in that no |\includeonly| should be invoked.
This can be achieved by starting the include file
(before |\ProvidesPackage|) with:
%
\begin{center}
\begin{tabular}{l}
|\input{childdoc.def}|\\
|\childdocforward{|\textit{main}|}|\\
\end{tabular}
\end{center}
%
or alternatively with:
%
\begin{center}
\begin{tabular}{l}
|\input{childdoc.def}|\\
|\childdocby{|\textit{main}|}|\\
\end{tabular}
\end{center}
%
Both forms have slightly different effects as described above.
The main file is prepared as usual, see \secref{sec:include}.

%%%%%%%%%%%%%%%%%%%%%%%%%%%%%%%%%%%%%%%%%%%%%%%%%%%%%%%%%%%%%%%%%%%%%%%%%%%%%%%%
\subsection{Legacy Detection}
\label{sec:detection}

The directive |\childdocmain| in the main file can detect
whether the complete document or merely a child is to be compiled
even without using the directive |\childdocof|.
This method is deprecated because it is less robust
and there is no compelling reason to use it;
it is merely provided for backward compatibility
and it may be removed in future versions.

If the detection mechanism is to be used,
it is mandatory to correctly specify
the filename of the main file as the argument of |\childdocmain|:
%
\begin{center}
\begin{tabular}{l}
|\input{childdoc.def}|\\
|\childdocmain{|\textit{main}|}|\\
\end{tabular}
\end{center}
%
If |\jobname| does not match the argument \textit{main} of |\childdocmain|,
it is assumed that |\jobname| points to the child file to be compiled.
When using |\childdocmain| with the main file specified as argument,
it suffices to start a child file
with just |\input{|\textit{main}|}|
without loading of the package and using |\childdocof|.
If instead all processing is done
with the appropriate \textsf{childdoc} directives,
the argument of \textit{main} of |\childdocmain| can be empty.

An alternative version of the command line processing described
in \secref{sec:commandline} using the detection mechanism reads:
%
\begin{center}
|... -jobname "|\textit{target}|" "|[\textit{flags}]%
[|\def\jobname{|\textit{dest}|}|]|\input{|\textit{main}|}"|
\end{center}

%%%%%%%%%%%%%%%%%%%%%%%%%%%%%%%%%%%%%%%%%%%%%%%%%%%%%%%%%%%%%%%%%%%%%%%%%%%%%%%%
\subsection{Manual Code}
\label{sec:manual}

In case one cannot be certain whether the definitions file |childdoc.def|
is installed on the target \TeX{} distribution
and one prefers not to ship it,
it is conceivable to paste a few relevant commands into the sources.

To that end, drop all statements |\input{childdoc.def}|
and perform the replacements as outlined below.
Instead of |\childdocmain{|\textit{main}|}| add the following code
to the top of the main file:
%
\begin{center}
\begin{tabular}{l}
|\||ifdefined\childdocname\endinput\||fi\newif\ifchilddoc|\\
|\edef\childdocname{\scantokens\expandafter{\jobname\noexpand}}|\\
|\def\childdocmain{|\textit{main}|}\||ifx\childdocmain\childdocname\||else|\\
|\childdoctrue\includeonly{\childdocname}\let\jobname\childdocmain\||fi|\\
\end{tabular}
\end{center}
%
Instead of |\childdocof{|\textit{main}|}| just include the main file
at the top of each child file:
%
\begin{center}
|\input{|\textit{main}|}|
\end{center}
%
A simple redirection |\childdocforward{|\textit{dest}|}| is achieved by:
%
\begin{center}
|\def\jobname{|\textit{dest}|}\input{\jobname}|
\end{center}
%
The redirection with prefix
|\childdocforwardprefix[|\textit{prefix}|]{|\textit{dest}|}|
is accomplished by:
%
\begin{center}
\begin{tabular}{l}
|{\edef\jobname{\scantokens\expandafter{\jobname\noexpand}}|\\
|\def\redirectjob |\textit{prefix}|#1~~~{\gdef\jobname{|\textit{dest}|#1}}|\\
|\expandafter\redirectjob\jobname~~~}\input{\jobname}|
\end{tabular}
\end{center}

In an alternative approach,
child documents can be compiled by a specific command line
without additional code or specific definitions:
%
\begin{center}
|... -jobname "|\textit{target}|" "|[\textit{flags}]%
|\includeonly{|\textit{dest}|}\input{|\textit{main}|}"|
\end{center}
%

%%%%%%%%%%%%%%%%%%%%%%%%%%%%%%%%%%%%%%%%%%%%%%%%%%%%%%%%%%%%%%%%%%%%%%%%%%%%%%%%
%%%%%%%%%%%%%%%%%%%%%%%%%%%%%%%%%%%%%%%%%%%%%%%%%%%%%%%%%%%%%%%%%%%%%%%%%%%%%%%%
\section{Information}

%%%%%%%%%%%%%%%%%%%%%%%%%%%%%%%%%%%%%%%%%%%%%%%%%%%%%%%%%%%%%%%%%%%%%%%%%%%%%%%%
\subsection{Copyright}

Copyright \copyright{} 2017--2018 Niklas Beisert

This work may be distributed and/or modified under the
conditions of the \LaTeX{} Project Public License, either version 1.3
of this license or (at your option) any later version.
The latest version of this license is in
  \url{http://www.latex-project.org/lppl.txt}
and version 1.3 or later is part of all distributions of \LaTeX{}
version 2005/12/01 or later.

This work has the LPPL maintenance status `maintained'.

The Current Maintainer of this work is Niklas Beisert.

This work consists of the files |README.txt|, |childdoc.ins| and |childdoc.dtx|
as well as the derived files |childdoc.def|, |cdocsamp.tex|
with |cdocsch1.tex|, |cdocsch2.tex|, |cdocspt3.tex|, |cdocspt4.tex|,
|cdocsdrf.tex|, |cdocsfn1.tex|, |cdocsfn2.tex|
as well as |childdoc.pdf|.

%%%%%%%%%%%%%%%%%%%%%%%%%%%%%%%%%%%%%%%%%%%%%%%%%%%%%%%%%%%%%%%%%%%%%%%%%%%%%%%%
\subsection{Files and Installation}

The package consists of the files:
%
\begin{center}
\begin{tabular}{ll}
    |README.txt|   & readme file \\
    |childdoc.ins| & installation file \\
    |childdoc.dtx| & source file \\
    |childdoc.def| & definition file \\
    |cdocsamp.tex| & sample main file \\
    |cdocsch1.tex| & sample include file \\
    |cdocsch2.tex| & sample include file \\
    |cdocspt3.tex| & sample part file \\
    |cdocspt4.tex| & sample part file \\
    |cdocsdrf.tex| & sample redirection file \\
    |cdocsfn1.tex| & sample redirection file \\
    |cdocsfn2.tex| & sample redirection file \\
    |childdoc.pdf| & manual
\end{tabular}
\end{center}
%
The distribution consists of the files
|README.txt|, |childdoc.ins| and |childdoc.dtx|.
%
\begin{itemize}
\item
Run (pdf)\LaTeX{} on |childdoc.dtx|
to compile the manual |childdoc.pdf| (this file).
\item
Run \LaTeX{} on |childdoc.ins| to create the definitions file |childdoc.def|
and the sample |cdocsamp.tex| with include files
|cdocsch1.tex|, |cdocsch2.tex|, |cdocspt3.tex|, |cdocspt4.tex|,
|cdocsdrf.tex|, |cdocsfn1.tex|, |cdocsfn2.tex|.
Then copy the file |childdoc.def| to an appropriate directory of your \LaTeX{}
distribution, e.g.\ \textit{texmf-root}|/tex/latex/childdoc|.
\end{itemize}

%%%%%%%%%%%%%%%%%%%%%%%%%%%%%%%%%%%%%%%%%%%%%%%%%%%%%%%%%%%%%%%%%%%%%%%%%%%%%%%%
\subsection{Related CTAN Packages}

There are several other packages which offer a similar functionality:
%
\begin{itemize}
\item
The packages
\href{http://ctan.org/pkg/docmute}{\textsf{docmute}},
\href{http://ctan.org/pkg/includex}{\textsf{includex}} and
\href{http://ctan.org/pkg/standalone}{\textsf{standalone}}
provide commands to include only the document body of
a child file thus allowing both files to be compiled individually.
\item
The packages \href{http://ctan.org/pkg/subdocs}{\textsf{subdocs}}
and \href{http://ctan.org/pkg/subfiles}{\textsf{subfiles}}
provide structures in which the main and child documents can be
encapsulated and allowing them to be compiled individually.
The inclusion mechanism is different from the conventional |\include|.
\item
The package \href{http://ctan.org/pkg/combine}{\textsf{combine}}
is an elaborate solution to combine several documents into one.
\end{itemize}
%
See also the CTAN topic \href{http://ctan.org/topic/subdocs}{\textsf{subdocs}}
for further related packages.
The present package differs from the above solutions in that
a document structure constructed with the conventional |\include| mechanism
just needs two extra commands at the top of every file
such that all constituent files can be compiled individually.

%%%%%%%%%%%%%%%%%%%%%%%%%%%%%%%%%%%%%%%%%%%%%%%%%%%%%%%%%%%%%%%%%%%%%%%%%%%%%%%%
%\subsection{Feature Suggestions}
%
%The following is a list of features which may be useful for future
%versions of this package:
%%
%\begin{itemize}
%\item
%\ldots
%\end{itemize}

%%%%%%%%%%%%%%%%%%%%%%%%%%%%%%%%%%%%%%%%%%%%%%%%%%%%%%%%%%%%%%%%%%%%%%%%%%%%%%%%
\subsection{Revision History}

%%%%%%%%%%%%%%%%%%%%%%%%%%%%%%%%%%%%%%%%
\paragraph{v2.0:} 2018/12/30

\begin{itemize}
\item
immediate forward processing
\item
added |\childdocby| mechanism
\item
manual restructured
\end{itemize}

%%%%%%%%%%%%%%%%%%%%%%%%%%%%%%%%%%%%%%%%
\paragraph{v1.6:} 2018/01/17

\begin{itemize}
\item
application for development of include files
\item
corrections to manual
\end{itemize}

%%%%%%%%%%%%%%%%%%%%%%%%%%%%%%%%%%%%%%%%
\paragraph{v1.5:} 2017/05/21

\begin{itemize}
\item
more complete structuring introduced
\item
|\childdocof| introduced
\item
|\childdoc| renamed to |\childdocmain|
\item
|\childredirect| renamed to |\childdocforward| and |\childdocforwardprefix|
and functionality expanded
\end{itemize}

%%%%%%%%%%%%%%%%%%%%%%%%%%%%%%%%%%%%%%%%
\paragraph{v1.0:} 2017/04/27

\begin{itemize}
\item
manual and install package
\item
first version published on CTAN
\end{itemize}

%%%%%%%%%%%%%%%%%%%%%%%%%%%%%%%%%%%%%%%%
\paragraph{v0.6:} 2017/04/26

\begin{itemize}
\item
redirection mechanism added
\end{itemize}

%%%%%%%%%%%%%%%%%%%%%%%%%%%%%%%%%%%%%%%%
\paragraph{v0.5:} 2017/04/26

\begin{itemize}
\item
functionality in definition file
\end{itemize}


%%%%%%%%%%%%%%%%%%%%%%%%%%%%%%%%%%%%%%%%%%%%%%%%%%%%%%%%%%%%%%%%%%%%%%%%%%%%%%%%
%%%%%%%%%%%%%%%%%%%%%%%%%%%%%%%%%%%%%%%%%%%%%%%%%%%%%%%%%%%%%%%%%%%%%%%%%%%%%%%%
%%%%%%%%%%%%%%%%%%%%%%%%%%%%%%%%%%%%%%%%%%%%%%%%%%%%%%%%%%%%%%%%%%%%%%%%%%%%%%%%
\appendix

\settowidth\MacroIndent{\rmfamily\scriptsize 000\ }

 \DocInput{childdoc.dtx}

\end{document}
%</driver>
% \fi
%
% %%%%%%%%%%%%%%%%%%%%%%%%%%%%%%%%%%%%%%%%%%%%%%%%%%%%%%%%%%%%%%%%%%%%%%%%%%%%%%
% %%%%%%%%%%%%%%%%%%%%%%%%%%%%%%%%%%%%%%%%%%%%%%%%%%%%%%%%%%%%%%%%%%%%%%%%%%%%%%
% \section{Sample}
%\iffalse
%<*samplemain>
%\fi
%
% The following presents a sample document
% with two chapters, two parts, a title page,
% a compile flag as well as three forwarding files to set the flag.
% It consists of eight |.tex| files:
% \begin{center}
% \begin{tabular}{ll}
% |cdocsamp.tex|&main file\\
% |cdocsch1.tex|&include file for chapter 1\\
% |cdocsch2.tex|&include file for chapter 2\\
% |cdocspt3.tex|&include file for part 3\\
% |cdocspt4.tex|&include file for part 4\\
% |cdocsdrf.tex|&forwarding file for main file in draft mode\\
% |cdocsfi1.tex|&forwarding file for final version of chapter 1\\
% |cdocsfi2.tex|&forwarding file for final version of chapter 2\\
% \end{tabular}
% \end{center}
% Each of the eight files can be compiled directly by the \LaTeX{} compiler.
%
% %%%%%%%%%%%%%%%%%%%%%%%%%%%%%%%%%%%%%%
% \paragraph{Main File.}
%
% The main file is called |cdocsamp.tex|.
%
% Load the \textsf{childdoc} definitions and
% declare the filename for the main document:
%    \begin{macrocode}
\input{childdoc.def}
\childdocmain{}
%    \end{macrocode}

% Optional override for |\version| flag:
%    \begin{macrocode}
%%\ifchilddoc\else\providecommand{\version}{draft}\fi
%    \end{macrocode}

% Define the default values for the |\version| flag
% (|final| for the main file and |draft| for childs):
%    \begin{macrocode}
\ifchilddoc
\providecommand{\version}{draft}
\else
\providecommand{\version}{final}
\fi
%    \end{macrocode}

% Load the standard document class:
%    \begin{macrocode}
\documentclass[12pt]{article}
%    \end{macrocode}

% Start the document body:
%    \begin{macrocode}
\begin{document}
%    \end{macrocode}

% Declare a title page.
% Print title, part of document being processed and version flag:
%    \begin{macrocode}
\addtocounter{page}{-1}
\begin{center}
{\LARGE\bfseries{}childdoc example\par}
\vspace{1cm}
\ifchilddoc
\ifchilddocmanual part\else chapter\fi:
`\childdocname' of `\childdocjob'\par
\else
main document: `\childdocjob'\par
\fi
version: \version\par
\end{center}
\newpage
%    \end{macrocode}

% Manually include selected file,
% otherwise process as usual:
%    \begin{macrocode}
\ifchilddocmanual
\section*{part `\childdocname'}
\input{\childdocname}
\else
%    \end{macrocode}

% Include the two chapters:
%    \begin{macrocode}
\include{cdocsch1}
\include{cdocsch2}
%    \end{macrocode}

% Include the two parts unless only chapters should be displayed:
%    \begin{macrocode}
\ifchilddoc\else
\section{part three}
\input{cdocspt3}
\section{part four}
\input{cdocspt4}
\fi
%    \end{macrocode}

% Process as usual until here:
%    \begin{macrocode}
\fi
%    \end{macrocode}

% End of document body:
%    \begin{macrocode}
\end{document}
%    \end{macrocode}
%\iffalse
%</samplemain>
%\fi
%
% %%%%%%%%%%%%%%%%%%%%%%%%%%%%%%%%%%%%%%
% \paragraph{Chapter Include Files.}
%
% The include files are called |cdocsch1.tex| and |cdocsch2.tex|.
%
%\iffalse
%<*samplechap1|samplechap2>
%\fi

% Optional override for |\version| flag:
%    \begin{macrocode}
%%\providecommand{\version}{final}
%    \end{macrocode}

% Include the main document:
%    \begin{macrocode}
\input{childdoc.def}
\childdocof{cdocsamp}
%    \end{macrocode}

%\iffalse
%</samplechap1|samplechap2>
%\fi
%
%\iffalse
%<*samplechap1>
%\fi
% Some text for chapter 1:
%    \begin{macrocode}
\section{one}
some text in chapter one
%    \end{macrocode}

%\iffalse
%</samplechap1>
%\fi
% Some text for chapter 2:
%\iffalse
%<*samplechap2>
%\fi
%    \begin{macrocode}
\section{two}
more text in chapter two
%    \end{macrocode}

%\iffalse
%</samplechap2>
%\fi
%
% %%%%%%%%%%%%%%%%%%%%%%%%%%%%%%%%%%%%%%
% \paragraph{Part Include Files.}
%
% The include files are called |cdocspt3.tex| and |cdocspt4.tex|.
%
%\iffalse
%<*samplepart3|samplepart4>
%\fi

% Optional override for |\version| flag:
%    \begin{macrocode}
%%\providecommand{\version}{final}
%    \end{macrocode}

% Include the main document:
%    \begin{macrocode}
\input{childdoc.def}
\childdocby{cdocsamp}
%    \end{macrocode}

%\iffalse
%</samplepart3|samplepart4>
%\fi
%
%\iffalse
%<*samplepart3>
%\fi
% Some text for part 3:
%    \begin{macrocode}
some text in part three
%    \end{macrocode}

%\iffalse
%</samplepart3>
%\fi
% Some text for part 4:
%\iffalse
%<*samplepart4>
%\fi
%    \begin{macrocode}
more text in part four
%    \end{macrocode}

%\iffalse
%</samplepart4>
%\fi
%
% %%%%%%%%%%%%%%%%%%%%%%%%%%%%%%%%%%%%%%
% \paragraph{Forwarding for a Complete Draft.}
%
% The following forwarding file |cdocsdrf.tex|
% compiles the main document in draft mode:
%\iffalse
%<*sampledraft>
%\fi
%    \begin{macrocode}
\def\version{draft}
\input{childdoc.def}
\childdocforward{cdocsamp}
%    \end{macrocode}

%\iffalse
%</sampledraft>
%\fi
%
% %%%%%%%%%%%%%%%%%%%%%%%%%%%%%%%%%%%%%%
% \paragraph{Forwarding for Final Version of the Chapters.}
%
% The following forwarding files |cdocsfn1.tex| and |cdocsfn2.tex|
% (with identical content)
% compile the final versions of the child documents
% |cdocsch1.tex| and |cdocsch2.tex|, respectively:
%\iffalse
%<*samplefinal>
%\fi
%    \begin{macrocode}
\def\version{final}
\input{childdoc.def}
\childdocforwardprefix[cdocsamp]{cdocsfn}{cdocsch}
%    \end{macrocode}

%\iffalse
%</samplefinal>
%\fi
%
% %%%%%%%%%%%%%%%%%%%%%%%%%%%%%%%%%%%%%%
% \paragraph{Command Line Processing.}
%
% The following three command lines generate the output files
% |cdocscld|, |cdocscl1| and |cdocscl2|
% which should be identical to
% |cdocsdrf|, |cdocsch1| and |cdocsfn2|, respectively:
% \begin{center}
% \begin{tabular}{l}
% |latex -jobname cdocscld \|\\
% |  "\def\version{draft}\input{childdoc.def}\childdocforward{cdocsamp}"|\\
% |latex -jobname cdocscl1 \|\\
% |  "\input{childdoc.def}\childdocforward[cdocsamp]{cdocsch1}"|\\
% |latex -jobname cdocscl2 \|\\
% |  "\def\version{final}\input{childdoc.def}\childdocforward{cdocsch2}"|
% \end{tabular}
% \end{center}
% Note that the trailing backslash on each first line
% merely continues the input to the second line
% (for convenient cut ant paste).
% Furthermore, the command |latex| can be replaced by any
% of its alternative versions such as |pdflatex|.
%
% %%%%%%%%%%%%%%%%%%%%%%%%%%%%%%%%%%%%%%%%%%%%%%%%%%%%%%%%%%%%%%%%%%%%%%%%%%%%%%
% %%%%%%%%%%%%%%%%%%%%%%%%%%%%%%%%%%%%%%%%%%%%%%%%%%%%%%%%%%%%%%%%%%%%%%%%%%%%%%
% \section{Implementation}
%\iffalse
%<*package>
%\fi
%
% This section describes the definitions file |childdoc.def|.

% The definitions cannot be loaded using |\usepackage| or |\RequirePackage|
% which has a mechanism to prevent loading a style file more than once.
% When loading the definitions by means of |\input|
% multiple instances have to be prevented manually:
%\iffalse
%This code needs to be before the `\ProvidesFile' directive
%which is defined at the beginning of this file.
%Therefore it is also placed there and commented out here.
%</package>
%<*discard>
%\fi
%    \begin{macrocode}
\ifdefined\childdocmain\endinput\fi
%    \end{macrocode}
%\iffalse
%</discard>
%<*package>
%\fi
%
% \macro{\ifchilddoc}
% \macro{\ifchilddocmanual}
% The conditional |\ifchilddoc| tells whether a
% child (true) or main (false) document is being compiled.
% The conditional |\ifchilddocmanual| tells whether
% the |\includeonly| mechanism is used (false) or
% the selection of child files must be performed manually (true).
% The definitions initialise to false:
%    \begin{macrocode}
\newif\ifchilddoc
\newif\ifchilddocmanual
%    \end{macrocode}

% \macro{\childdocname}
% \macro{\childdocjob}
% The macro |\childdocname| stores the name of the main document
% to be compiled. The macro |\childdocjob| stores the name of
% the document on which the \LaTeX{} compiler was originally invoked.
% The content of |\jobname| cannot be compared
% to filenames specified in the source due to different catcodes.
% The following code rescans |\jobname|, stores the result
% in |\childdocname| and saves a copy in |\childdocjob|:
%    \begin{macrocode}
\edef\childdocname{\scantokens\expandafter{\jobname\noexpand}}
\let\childdocjob\childdocname
%    \end{macrocode}

% \macro{\childdocdisable}
% The macro |\childdocdisable| prevents the main file
% from being processed more than once.
% At this stage, the main document command |\childdocmain|
% is assumed to be called once again where it should do nothing.
% Any subsequent call to it should prevent
% a secondary processing of the main document
% It overwrites the forwarding commands
% |\childdocof| and |\childdocforward|
% with empty macros to prevent further inclusions of the main document:
%    \begin{macrocode}
\newcommand{\childdocdisable}
{
  \renewcommand{\childdocmain}[1]{\renewcommand{\childdocmain}[1]{\endinput}}
  \renewcommand{\childdocof}[1]{}
  \renewcommand{\childdocby}[2][]{}
  \renewcommand{\childdocforward}[2][]{}
  \renewcommand{\childdocdisable}{}
}
%    \end{macrocode}

% \macro{\childdocmain}
% The macro |\childdocmain| is to be called at the top of the main file
% with nothing or the main filename (without extension) as argument.
% First, it breaks loops.
% If the argument is not empty and does not match |\childdocname|
% (which is set by the first inclusion of |childdoc.def|),
% |\ifchilddoc| is set to true, |\includeonly| is applied to the child file
% and |\jobname| is set to the main file
% (for proper handling of |.aux| files):
%    \begin{macrocode}
\newcommand{\childdocmain}[1]
{
  \childdocdisable\childdocmain{}
  \if?#1?\else
    \begingroup
      \def\childdoctmp{#1}
      \ifx\childdoctmp\childdocname
        \def\childdoctmp{}
      \else
        \def\childdoctmp
        {
          \childdoctrue
          \includeonly{\childdocname}
          \def\childdocjob{#1}
          \def\jobname{#1}
        }
      \fi
      \expandafter
    \endgroup
    \childdoctmp
  \fi
}
%    \end{macrocode}

% \macro{\childdocof}
% The command |\childdocof| redirects
% compilation to the main file |#1|.
%    \begin{macrocode}
\newcommand{\childdocof}[1]
{
  \childdocdisable
  \childdoctrue
  \includeonly{\childdocname}
  \def\jobname{#1}
  \def\childdocjob{#1}
  \input{#1}
}
%    \end{macrocode}

% \macro{\childdocby}
% The command |\childdocby| ....
%    \begin{macrocode}
\newcommand{\childdocby}[2][]
{
  \childdocdisable
  \childdoctrue
  \childdocmanualtrue
  \if?#1?\else
    \def\jobname{#2}
  \fi
  \def\childdocjob{#2}
  \input{#2}
  \endinput
}
%    \end{macrocode}

% \macro{\childdocforward}
% The command |\childdocforward| redirects
% compilation to the main file or
% (if the optional argument is given) a child file.
% Parameters are set as if the main file
% or a child file starting with |\childdocof| was compiled.
% Then compilation is handed over to the main file:
%    \begin{macrocode}
\newcommand{\childdocforward}[2][]
{
  \begingroup
    \if?#1?
      \def\childdoctmp
      {
        \def\childdocname{#2}
        \def\childdocjob{#2}
        \def\jobname{#2}
        \input{#2}
        \endinput
      }
    \else
      \def\childdoctmp
      {
        \childdocdisable
        \def\childdocname{#2}
        \childdoctrue
        \includeonly{#2}
        \def\childdocjob{#1}
        \def\jobname{#1}
        \input{#1}
        \endinput
      }
    \fi
    \expandafter
  \endgroup
  \childdoctmp
}
%    \end{macrocode}

% \macro{\childdocforwardprefix}
% The command |\childdocforwardprefix| redirects
% compilation to the main or a child file by means of a pattern.
% The prefix |#1| in the current filename is replaced by |#2|
% and the suffix of the current filename is kept
% (it is assumed that the filename does not contain the substring `|~~~|'
% which is used as a delimiter).
% Compilation is handed over to the new file by |\childdocforward|:
%    \begin{macrocode}
\newcommand{\childdocforwardprefix}[3][]
{
  \begingroup
    \def\childdocextract #2##1~~~{\def\childdoctmp{\childdocforward[#1]{#3##1}}}
    \expandafter\childdocextract\childdocname~~~
    \expandafter
  \endgroup
  \childdoctmp
}
%    \end{macrocode}

% \macro{\childdoc}
% The deprecated macro |\childdoc| is a legacy version of |\childdocmain|:
%    \begin{macrocode}
\newcommand{\childdoc}{\childdocmain}
%    \end{macrocode}

% \macro{\childdocredirect}
% The deprecated macro |\childdocredirect| is a legacy version
% of |\childdocforward| and |\childdocforwardprefix|:
%    \begin{macrocode}
\newcommand{\childdocredirect}[2][]
{
  \begingroup
    \if?#1?
      \def\childdoctmp{\childdocforward{#2}}
    \else
      \def\childdoctmp{\childdocforwardprefix{#1}{#2}}
    \fi
    \expandafter
  \endgroup
  \childdoctmp
}
%    \end{macrocode}

%\iffalse
%</package>
%\fi
%
\endinput
|\\
|\childdocby{|\textit{main}|}|\\
\end{tabular}
\end{center}
%
Both forms have slightly different effects as described above.
The main file is prepared as usual, see \secref{sec:include}.

%%%%%%%%%%%%%%%%%%%%%%%%%%%%%%%%%%%%%%%%%%%%%%%%%%%%%%%%%%%%%%%%%%%%%%%%%%%%%%%%
\subsection{Legacy Detection}
\label{sec:detection}

The directive |\childdocmain| in the main file can detect
whether the complete document or merely a child is to be compiled
even without using the directive |\childdocof|.
This method is deprecated because it is less robust
and there is no compelling reason to use it;
it is merely provided for backward compatibility
and it may be removed in future versions.

If the detection mechanism is to be used,
it is mandatory to correctly specify
the filename of the main file as the argument of |\childdocmain|:
%
\begin{center}
\begin{tabular}{l}
|% \iffalse
%
% childdoc.dtx Copyright (C) 2017-2018 Niklas Beisert
%
% This work may be distributed and/or modified under the
% conditions of the LaTeX Project Public License, either version 1.3
% of this license or (at your option) any later version.
% The latest version of this license is in
%   http://www.latex-project.org/lppl.txt
% and version 1.3 or later is part of all distributions of LaTeX
% version 2005/12/01 or later.
%
% This work has the LPPL maintenance status `maintained'.
%
% The Current Maintainer of this work is Niklas Beisert.
%
% This work consists of the files childdoc.dtx and childdoc.ins
% and the derived files childdoc.def and cdocsamp.tex with
% cdocsch1.tex, cdocsch2.tex, cdocsdrf.tex, cdocsfn1.tex, cdocsfn2.tex.
%
%<package>\ifdefined\childdocmain\endinput\fi
%<package>\ProvidesFile{childdoc.def}[2018/12/30 v2.0 child document driver]
%<samplemain>\ProvidesFile{cdocsamp.tex}[2018/12/30 v2.0 sample for childdoc]
%<*driver>
%\ProvidesFile{childdoc.drv}[2018/12/30 v2.0 childdoc reference manual file]
\PassOptionsToClass{10pt,a4paper}{article}
\documentclass{ltxdoc}

\usepackage[margin=35mm]{geometry}
\usepackage{hyperref}
\usepackage{hyperxmp}
\usepackage[usenames]{color}

\hypersetup{colorlinks=true}
\hypersetup{pdfstartview=FitH}
\hypersetup{pdfpagemode=UseNone}
\hypersetup{pdfsource={}}
\hypersetup{pdflang={en-UK}}
\hypersetup{pdfcopyright={Copyright 2017-2018 Niklas Beisert.
  This work may be distributed and/or modified under the
  conditions of the LaTeX Project Public License, either version 1.3
  of this license or (at your option) any later version.}}
\hypersetup{pdflicenseurl={http://www.latex-project.org/lppl.txt}}
\hypersetup{pdfcontactaddress={ETH Zurich, ITP, HIT K,
  Wolfgang-Pauli-Strasse 27}}
\hypersetup{pdfcontactpostcode={8093}}
\hypersetup{pdfcontactcity={Zurich}}
\hypersetup{pdfcontactcountry={Switzerland}}
\hypersetup{pdfcontactemail={nbeisert@itp.phys.ethz.ch}}
\hypersetup{pdfcontacturl={http://people.phys.ethz.ch/\xmptilde nbeisert/}}

\newcommand{\secref}[1]{\hyperref[#1]{section \ref*{#1}}}

\parskip1ex
\parindent0pt
\let\olditemize\itemize
\def\itemize{\olditemize\parskip0pt}

\begin{document}

\title{The \textsf{childdoc} Package}
\hypersetup{pdftitle={The childdoc Package}}
\author{Niklas Beisert\\[2ex]
  Institut f\"ur Theoretische Physik\\
  Eidgen\"ossische Technische Hochschule Z\"urich\\
  Wolfgang-Pauli-Strasse 27, 8093 Z\"urich, Switzerland\\[1ex]
  \href{mailto:nbeisert@itp.phys.ethz.ch}
  {\texttt{nbeisert@itp.phys.ethz.ch}}}
\hypersetup{pdfauthor={Niklas Beisert}}
\hypersetup{pdfsubject={Manual for the LaTeX2e Package childdoc}}
\date{30 December 2018, \textsf{v2.0}}
\maketitle

\begin{abstract}\noindent
\textsf{childdoc} is a \LaTeXe{} package
that enables the direct compilation
of document sections included by |\include|
to individual files.
\end{abstract}

\begingroup
\parskip0ex
\tableofcontents
\endgroup

%%%%%%%%%%%%%%%%%%%%%%%%%%%%%%%%%%%%%%%%%%%%%%%%%%%%%%%%%%%%%%%%%%%%%%%%%%%%%%%%
%%%%%%%%%%%%%%%%%%%%%%%%%%%%%%%%%%%%%%%%%%%%%%%%%%%%%%%%%%%%%%%%%%%%%%%%%%%%%%%%
\section{Introduction}

\LaTeX{} provides a mechanism to structure a large document (such as a book)
into a main file and several child files (containing the chapters)
using the |\include| command.
This mechanism is beneficial for documents
which span hundreds of pages in order to
make the source file(s) more manageable.
Moreover, compilation can be restricted to
selected child files by means of the |\includeonly| command.
The latter feature can be used to reduce the compilation time while editing
(this was significantly more useful in the earlier days of \LaTeX{})
or to generate a smaller document which is easier to navigate.
Another application of |\includeonly| is to generate
documents consisting of selected parts of the complete document.

However, there are a few drawbacks of the plain |\include| mechanism:
\begin{itemize}
\item
The child files cannot be compiled on their own,
they can only be compiled via the main file.
A naive editing environment
(such as a text editor with an option
to have the current file processed by \LaTeX)
may require one to switch to the main file before compiling;
attempting to compile the child file produces errors.
\item
The main file must be modified (each time)
to adjust the |\includeonly| command
to the present needs. This easily leaves the main file in a messy state.
\item
The generated document will always carry the filename
of the main document. This is inconvenient if
several child files are to be compiled and
to be kept for distribution.
\end{itemize}

The present package provides a simple interface
to make child files individually compilable by \LaTeX{}.
Compiling a child file then has the same effect as compiling
the main file with an |\includeonly| command
to select the appropriate child.
Moreover the generated document will carry the name of the child
rather than the main file.
This resolves all three above issues.

This feature is meant to make the editing of books,
thesis documents and lecture notes somewhat more convenient.
However, the package can also be used efficiently for
composing a series of documents (such as exercise sheets)
which are typically distributed individually.
It then assists the author in generating the individual documents
(potentially in different versions)
as well as a document containing the collected series.
Another application is in developing style files
or other kinds of included material
where compilation of the style file could redirect
to a sample or test file.

%%%%%%%%%%%%%%%%%%%%%%%%%%%%%%%%%%%%%%%%%%%%%%%%%%%%%%%%%%%%%%%%%%%%%%%%%%%%%%%%
%%%%%%%%%%%%%%%%%%%%%%%%%%%%%%%%%%%%%%%%%%%%%%%%%%%%%%%%%%%%%%%%%%%%%%%%%%%%%%%%
\section{Usage}

First of all, the package \textsf{childdoc} is \emph{not} a standard
\LaTeXe{} |.sty| style file! Therefore it needs to be invoked in
a non-standard way.

%%%%%%%%%%%%%%%%%%%%%%%%%%%%%%%%%%%%%%%%%%%%%%%%%%%%%%%%%%%%%%%%%%%%%%%%%%%%%%%%
\subsection{Included Files}
\label{sec:include}

%%%%%%%%%%%%%%%%%%%%%%%%%%%%%%%%%%%%%%%%
\DescribeMacro{\childdocmain}
To use the package, add the commands
\begin{center}
\begin{tabular}{l}
|\input{childdoc.def}|\\
|\childdocmain{}|\\
\end{tabular}
\end{center}
at the very top of the main \LaTeX{} file,
in particular \emph{before} the |\documentclass| statement!
The argument of |\childdocmain| should be left empty
(but it must be present).

%%%%%%%%%%%%%%%%%%%%%%%%%%%%%%%%%%%%%%%%
\DescribeMacro{\childdocof}
Furthermore, add the commands
\begin{center}
\begin{tabular}{l}
|\input{childdoc.def}|\\
|\childdocof{|\textit{main}|}|\\
\end{tabular}
\end{center}
at the top of every child file \textit{child}
which is included by |\include{|\textit{child}|}|
from within the main file
(or at least for those files to be compiled individually).
The argument \textit{main} must be the filename of the main file.

There are a couple of
considerations in setting up the main and child documents:

%%%%%%%%%%%%%%%%%%%%%%%%%%%%%%%%%%%%%%%%
\paragraph{Restrictions.}

Please note the following restrictions:
\begin{itemize}
\item
|\childdocmain| must be called with one argument \textit{main}
to ensure compatibility with earlier version of the package.
It must either be empty (|\childdocmain{}|)
or precisely match the filename of the main file in which it is specified.
See \secref{sec:detection} for further information.
\item
The filename \textit{main} must be specified without the |.tex| extension.
\item
The filename \textit{main} is case sensitive
(even in case-insensitive file systems)
due to internal string comparison.
\item
The argument \textit{main} should be fully expanded, it cannot be a macro.
\item
Subdirectories and special characters should be avoided in filenames.
\item
The command |\childdocmain{|\textit{main}|}| must be followed by a whitespace.
It should not be followed immediately by another command
or by a comment mark `|%|'.
This is because the \TeX{} parser reads the token immediately following
the argument of |\childdocmain| and puts it
at the beginning of every child section;
however, a white\-space is ignored.
\end{itemize}

%%%%%%%%%%%%%%%%%%%%%%%%%%%%%%%%%%%%%%%%
\paragraph{Content of Main File.}

It is advisable to place all content in the child files included by |\include|.
Any output contained in the main file will appear in all child documents
unless suppressed manually;
it cannot be suppressed automatically by the |\includeonly| directive
and thus should normally be avoided.
A method to include some content in the main file
by means of conditional processing is described in \secref{sec:conditional}.

%%%%%%%%%%%%%%%%%%%%%%%%%%%%%%%%%%%%%%%%
\paragraph{Page Numbering.}

When only a part of the document is compiled,
the appropriate numbering of pages
(as well as other status parameters)
is determined from the |.aux| files.
The latter contain information from previous passes.
However this information needs to propagate through
all intermediate child documents.
Therefore the page numbering in child documents may well
be inconsistent until the complete document is compiled at least once.

A useful (if unconventional) way to always ensure a consistent
page numbering is to restart the numbering in each child document
and denote the pages by `\textit{child}|.|\textit{page}'
where \textit{child} represents the chapter/section number of the child file.
This can be achieved by the command
|\numberwithin{page}{|\textit{child}|}|
of the \textsf{amsmath} package
where \textit{child} can be |chapter| or |section|
depending on the chosen structuring.
Alternatively, one can modify the macro |\thepage| appropriately
and reset the counter |page| at the start of each child file.

%%%%%%%%%%%%%%%%%%%%%%%%%%%%%%%%%%%%%%%%%%%%%%%%%%%%%%%%%%%%%%%%%%%%%%%%%%%%%%%%
\subsection{Conditional Processing}
\label{sec:conditional}

The package provides a mechanism to compile different versions
of a document. To customise the versions further some conditional processing
can come in handy to distinguish which version is being compiled.
The package provides two macros to describe the compilation context:

%%%%%%%%%%%%%%%%%%%%%%%%%%%%%%%%%%%%%%%%
\DescribeMacro{\ifchilddoc}
The conditional |\ifchilddoc| distinguishes between the compilation of
child documents and the main document:
%
\begin{center}
|\ifchilddoc |\textit{child-code}| |[|\||else |\textit{main-code}]| \||fi|
\end{center}

%%%%%%%%%%%%%%%%%%%%%%%%%%%%%%%%%%%%%%%%
\DescribeMacro{\childdocname}
\DescribeMacro{\childdocjob}
The macro |\childdocname| contains the filename (without extension)
of the main or child file being processed.
Note that |\childdocjob| will always contain the name of the main file.

%%%%%%%%%%%%%%%%%%%%%%%%%%%%%%%%%%%%%%%%
\paragraph{Title Page.}

Conditional processing can be used to include a title or banner page
in the main document when proper precautions are taken.
Importantly, the code in the main file should ensure that the page counter
(as well as other status parameters which are stored in the |.aux| files)
takes the same value after the conditional processing.
Otherwise the page numbers may take divergent values
depending on which part is compiled.

For example, a title page could be declared by:
%
\begin{center}
\begin{tabular}{l}
|\ifchilddoc\||else|\\
|\addtocounter{page}{-1}|\\
\textit{code for title page}\\
|\newpage|\\
|\||fi|
\end{tabular}
\end{center}
%
A banner page for the child documents can be generated by:
%
\begin{center}
\begin{tabular}{l}
|\ifchilddoc|\\
|\addtocounter{page}{-1}|\\
\textit{code for banner page}\\
|\newpage|\\
|\||fi|
\end{tabular}
\end{center}
%
Here one could write a message such as:
\begin{center}
|This is the part \childdocname{} of \childdocjob{}.|
\end{center}

%%%%%%%%%%%%%%%%%%%%%%%%%%%%%%%%%%%%%%%%%%%%%%%%%%%%%%%%%%%%%%%%%%%%%%%%%%%%%%%%
\subsection{Flags}
\label{sec:flags}

The package makes it easy to generate different versions
of the main or child documents.
To this end compilation flags can be defined
and assigned different default values.
They will be particularly useful in conjunction
with the forwarding mechanism described in \secref{sec:forward}.

For example, it may be useful to have a flag |\version|
which can be set to |draft| or |final|.
The document source will contain some conditional code
depending on the value of |\version|.
Suppose further, the flag should default to |final| for the main file
and to |draft| for child files
which is a natural assignment for editing the document.
This is achieved by placing the following code
in the preamble of the main document
(below the |\childdocmain| directive):
%
\begin{center}
\begin{tabular}{l}
|\ifchilddoc|\\
|\providecommand{\version}{draft}|\\
|\||else|\\
|\providecommand{\version}{final}|\\
|\||fi|
\end{tabular}
\end{center}
%
The definition by |\providecommand| makes sure
that previous definitions are not overwritten.
Further statements |\providecommand{\version}{...}|
can thus be added before the above code to override it.

For the main file, one might add a line
(between |\childdocmain| and the above block)
%
\begin{center}
|%\ifchilddoc\||else\providecommand{\version}{draft}\||fi|
\end{center}
%
which can be uncommented to produce a draft version.
Likewise one can add a line to the very top of a child file
(above the |\childdocof{|\textit{main}|}| directive)
%
\begin{center}
|%\providecommand{\version}{final}|
\end{center}
%
which can be uncommented to produce the final version of this child document.

%%%%%%%%%%%%%%%%%%%%%%%%%%%%%%%%%%%%%%%%%%%%%%%%%%%%%%%%%%%%%%%%%%%%%%%%%%%%%%%%
\subsection{Forwarding}
\label{sec:forward}

Different versions of the main or child documents
using compilation flags as described in \secref{sec:flags}
can be (permanently) stored in different files
for convenient compilation, viewing and distribution.
To this end, the package defines a command
to pass on compilation to a different file:

%%%%%%%%%%%%%%%%%%%%%%%%%%%%%%%%%%%%%%%%
\DescribeMacro{\childdocforward}
The command |\childdocforward| redirects processing to
another source file:
%
\begin{center}
\begin{tabular}{l}
|\input{childdoc.def}|\\
|\childdocforward[|\textit{main}|]{|\textit{dest}|}|\\
\end{tabular}
\end{center}
%
The argument \textit{dest} is the destination file
(without extension).
It should be the main file or one of the child files.
Note that further \textsf{childdoc} directives
such as |\childdocof| and |\childdocforward|
in the indicated file will be processed in this form.
The optional argument \textit{main}
passes on directly to the main file \textit{main}
while pretending to compile the child \textit{dest}.
This form behaves as if \textit{dest}
issues |\childdocof{|\textit{main}|}| right away,
and no further \textsf{childdoc} directives will be processed.

%%%%%%%%%%%%%%%%%%%%%%%%%%%%%%%%%%%%%%%%
\DescribeMacro{\...prefix}
In the alternative form |\childdocforwardprefix|,
%
\begin{center}
\begin{tabular}{l}
|\input{childdoc.def}|\\
|\childdocforwardprefix[|\textit{main}|]{|\textit{prefix}|}{|\textit{dest}|}|
\end{tabular}
\end{center}
%
the destination file is determined by a pattern
depending on the current file:
To make this work, the current file must be called
`{\textit{prefix}\hspace{0.2em}\textit{suffix}}'
with \textit{prefix} matching precisely the argument.
Processing is then passed on to the file
`{\textit{dest}\hspace{0.2em}\textit{suffix}}'.
Surely, the same effect is achieved by
directly specifying the
argument `{\textit{dest}\hspace{0.2em}\textit{suffix}}'
in the first form.
However, that requires to set up a different file
for each child. With the alternative form of the command
all these files can have exactly the same content
which simplifies setting them up and maintaining them.

For example, the following file |draft.tex|
with a compilation flag |\version| as described in \secref{sec:flags}
compiles the main document as a draft:
%
\begin{center}
\begin{tabular}{l}
|\def\version{draft}|\\
|\input{childdoc.def}|\\
|\childdocforward{|\textit{main}|}|
\end{tabular}
\end{center}
%
Likewise, the following files |final|\textit{nn}|.tex|
compile the final version of the child document
|child|\textit{nn}|.tex|:
%
\begin{center}
\begin{tabular}{l}
|\def\version{final}|\\
|\input{childdoc.def}|\\
|\childdocforwardprefix{final}{child}|
\end{tabular}
\end{center}
%

Note that when several versions of a main file and/or of each child file
are to be generated, it may be convenient to set up a |Makefile| or
shell script to automatise the process.

%%%%%%%%%%%%%%%%%%%%%%%%%%%%%%%%%%%%%%%%%%%%%%%%%%%%%%%%%%%%%%%%%%%%%%%%%%%%%%%%
\subsection{Command Line Processing}
\label{sec:commandline}

The effect of redirection files can also be achieved by invoking
the \LaTeX{} compiler with a more elaborate command line.
Most conveniently this should be done as part
of a shell script or a |Makefile|.

When using \textsf{childdoc} in the main file, the following
command lines effectively perform a redirection
(note that depending on the shell being used,
backslashes may have to be doubled: `|\|' $\to$ `|\\|'):
%
\begin{center}
|... -jobname "|\textit{target}|" |\\|"|[\textit{flags}]%
|\input{childdoc.def}\childdocforward[|\textit{main}|]{|\textit{dest}|}"|
\end{center}
%
Here \textit{target} is the name of the output file,
\textit{main} is the name of the main file
and \textit{dest} is the name of the main or child file to be processed
(all filenames without extensions).
The optional argument \textit{main} can be omitted
if \textit{main} matches \textit{dest}.
Optionally, compilation \textit{flags} can be defined via |\def| commands.
This command line makes the \TeX{} engine believe
it is compiling the file \textit{target}
whose content is specified as the latter parameter.
The provided code then forwards the processing to
\textit{main} or \textit{dest} as described in \secref{sec:forward}.

%%%%%%%%%%%%%%%%%%%%%%%%%%%%%%%%%%%%%%%%%%%%%%%%%%%%%%%%%%%%%%%%%%%%%%%%%%%%%%%%
\subsection{Include by Input}
\label{sec:input}

Including child documents by |\include| has some restrictions by design.
Most notably, the content of a child document always occupies
its own set of pages; pages cannot be shared between child documents.
Usually, this behaviour makes perfect sense
because each child document contain an essential part of the document.
However, in some situations it may be desirable to compose
a document from a collection of parts
without having mandatory page breaks between then.
For this case, the package
provides a mechanism to include parts
by |\input| which can also be processed individually.
However, by construction this mechanism
requires manual handling of the content to be output.

%%%%%%%%%%%%%%%%%%%%%%%%%%%%%%%%%%%%%%%%
\DescribeMacro{\ifchilddocmanual}
The main file should be prepared as usual, see \secref{sec:include}.
However, the document body must make a distinction
between processing of an individual part and of the main document, e.g.:
%
\begin{center}
\begin{tabular}{l}
|\ifchilddocmanual|\\
|\input{\childdocname}|\\
|\||else|\\
\textit{document body with }|\input{|\textit{part}|}|\\
|\||fi|
\end{tabular}
\end{center}
%
The conditional |\ifchilddocmanual| is true whenever
a part to be included by |\input| is being compiled,
and the name of the part is stored in |\childdocname|.

%%%%%%%%%%%%%%%%%%%%%%%%%%%%%%%%%%%%%%%%
\DescribeMacro{\childdocby}
Each part to be included by |\input| should start with:
%
\begin{center}
\begin{tabular}{l}
|\input{childdoc.def}|\\
|\childdocby{|\textit{main}|}|\\
\end{tabular}
\end{center}
%
The directive |\childdocby| is similar to |\childdocof|
described in \secref{sec:include},
but the subsequent selection of content must be done manually.
To that end, both |\ifchilddoc| and |\ifchilddocmanual|
will be true upon processing of a part,
and the name of the part is stored in |\childdocname|.
Note that |\jobname| will be set to the filename of the current part
so that each part receives an individual |.aux| file
that does not interfere with the |.aux| file(s) of the main document.
This behaviour can be altered by the alternative form
|\childdocby[*]{|\textit{main}|}| (with a non-empty optional argument)
which uses the |.aux| file of the main document
by setting |\jobname| to \textit{main}.

%%%%%%%%%%%%%%%%%%%%%%%%%%%%%%%%%%%%%%%%%%%%%%%%%%%%%%%%%%%%%%%%%%%%%%%%%%%%%%%%
\subsection{Driver Development}
\label{sec:driver}

The \textsf{childdoc} mechanism can also be use for the development
of definition files such as \LaTeX{} styles or classes.
This case differs from the above setup with multiple parts
included by |\include| in that no |\includeonly| should be invoked.
This can be achieved by starting the include file
(before |\ProvidesPackage|) with:
%
\begin{center}
\begin{tabular}{l}
|\input{childdoc.def}|\\
|\childdocforward{|\textit{main}|}|\\
\end{tabular}
\end{center}
%
or alternatively with:
%
\begin{center}
\begin{tabular}{l}
|\input{childdoc.def}|\\
|\childdocby{|\textit{main}|}|\\
\end{tabular}
\end{center}
%
Both forms have slightly different effects as described above.
The main file is prepared as usual, see \secref{sec:include}.

%%%%%%%%%%%%%%%%%%%%%%%%%%%%%%%%%%%%%%%%%%%%%%%%%%%%%%%%%%%%%%%%%%%%%%%%%%%%%%%%
\subsection{Legacy Detection}
\label{sec:detection}

The directive |\childdocmain| in the main file can detect
whether the complete document or merely a child is to be compiled
even without using the directive |\childdocof|.
This method is deprecated because it is less robust
and there is no compelling reason to use it;
it is merely provided for backward compatibility
and it may be removed in future versions.

If the detection mechanism is to be used,
it is mandatory to correctly specify
the filename of the main file as the argument of |\childdocmain|:
%
\begin{center}
\begin{tabular}{l}
|\input{childdoc.def}|\\
|\childdocmain{|\textit{main}|}|\\
\end{tabular}
\end{center}
%
If |\jobname| does not match the argument \textit{main} of |\childdocmain|,
it is assumed that |\jobname| points to the child file to be compiled.
When using |\childdocmain| with the main file specified as argument,
it suffices to start a child file
with just |\input{|\textit{main}|}|
without loading of the package and using |\childdocof|.
If instead all processing is done
with the appropriate \textsf{childdoc} directives,
the argument of \textit{main} of |\childdocmain| can be empty.

An alternative version of the command line processing described
in \secref{sec:commandline} using the detection mechanism reads:
%
\begin{center}
|... -jobname "|\textit{target}|" "|[\textit{flags}]%
[|\def\jobname{|\textit{dest}|}|]|\input{|\textit{main}|}"|
\end{center}

%%%%%%%%%%%%%%%%%%%%%%%%%%%%%%%%%%%%%%%%%%%%%%%%%%%%%%%%%%%%%%%%%%%%%%%%%%%%%%%%
\subsection{Manual Code}
\label{sec:manual}

In case one cannot be certain whether the definitions file |childdoc.def|
is installed on the target \TeX{} distribution
and one prefers not to ship it,
it is conceivable to paste a few relevant commands into the sources.

To that end, drop all statements |\input{childdoc.def}|
and perform the replacements as outlined below.
Instead of |\childdocmain{|\textit{main}|}| add the following code
to the top of the main file:
%
\begin{center}
\begin{tabular}{l}
|\||ifdefined\childdocname\endinput\||fi\newif\ifchilddoc|\\
|\edef\childdocname{\scantokens\expandafter{\jobname\noexpand}}|\\
|\def\childdocmain{|\textit{main}|}\||ifx\childdocmain\childdocname\||else|\\
|\childdoctrue\includeonly{\childdocname}\let\jobname\childdocmain\||fi|\\
\end{tabular}
\end{center}
%
Instead of |\childdocof{|\textit{main}|}| just include the main file
at the top of each child file:
%
\begin{center}
|\input{|\textit{main}|}|
\end{center}
%
A simple redirection |\childdocforward{|\textit{dest}|}| is achieved by:
%
\begin{center}
|\def\jobname{|\textit{dest}|}\input{\jobname}|
\end{center}
%
The redirection with prefix
|\childdocforwardprefix[|\textit{prefix}|]{|\textit{dest}|}|
is accomplished by:
%
\begin{center}
\begin{tabular}{l}
|{\edef\jobname{\scantokens\expandafter{\jobname\noexpand}}|\\
|\def\redirectjob |\textit{prefix}|#1~~~{\gdef\jobname{|\textit{dest}|#1}}|\\
|\expandafter\redirectjob\jobname~~~}\input{\jobname}|
\end{tabular}
\end{center}

In an alternative approach,
child documents can be compiled by a specific command line
without additional code or specific definitions:
%
\begin{center}
|... -jobname "|\textit{target}|" "|[\textit{flags}]%
|\includeonly{|\textit{dest}|}\input{|\textit{main}|}"|
\end{center}
%

%%%%%%%%%%%%%%%%%%%%%%%%%%%%%%%%%%%%%%%%%%%%%%%%%%%%%%%%%%%%%%%%%%%%%%%%%%%%%%%%
%%%%%%%%%%%%%%%%%%%%%%%%%%%%%%%%%%%%%%%%%%%%%%%%%%%%%%%%%%%%%%%%%%%%%%%%%%%%%%%%
\section{Information}

%%%%%%%%%%%%%%%%%%%%%%%%%%%%%%%%%%%%%%%%%%%%%%%%%%%%%%%%%%%%%%%%%%%%%%%%%%%%%%%%
\subsection{Copyright}

Copyright \copyright{} 2017--2018 Niklas Beisert

This work may be distributed and/or modified under the
conditions of the \LaTeX{} Project Public License, either version 1.3
of this license or (at your option) any later version.
The latest version of this license is in
  \url{http://www.latex-project.org/lppl.txt}
and version 1.3 or later is part of all distributions of \LaTeX{}
version 2005/12/01 or later.

This work has the LPPL maintenance status `maintained'.

The Current Maintainer of this work is Niklas Beisert.

This work consists of the files |README.txt|, |childdoc.ins| and |childdoc.dtx|
as well as the derived files |childdoc.def|, |cdocsamp.tex|
with |cdocsch1.tex|, |cdocsch2.tex|, |cdocspt3.tex|, |cdocspt4.tex|,
|cdocsdrf.tex|, |cdocsfn1.tex|, |cdocsfn2.tex|
as well as |childdoc.pdf|.

%%%%%%%%%%%%%%%%%%%%%%%%%%%%%%%%%%%%%%%%%%%%%%%%%%%%%%%%%%%%%%%%%%%%%%%%%%%%%%%%
\subsection{Files and Installation}

The package consists of the files:
%
\begin{center}
\begin{tabular}{ll}
    |README.txt|   & readme file \\
    |childdoc.ins| & installation file \\
    |childdoc.dtx| & source file \\
    |childdoc.def| & definition file \\
    |cdocsamp.tex| & sample main file \\
    |cdocsch1.tex| & sample include file \\
    |cdocsch2.tex| & sample include file \\
    |cdocspt3.tex| & sample part file \\
    |cdocspt4.tex| & sample part file \\
    |cdocsdrf.tex| & sample redirection file \\
    |cdocsfn1.tex| & sample redirection file \\
    |cdocsfn2.tex| & sample redirection file \\
    |childdoc.pdf| & manual
\end{tabular}
\end{center}
%
The distribution consists of the files
|README.txt|, |childdoc.ins| and |childdoc.dtx|.
%
\begin{itemize}
\item
Run (pdf)\LaTeX{} on |childdoc.dtx|
to compile the manual |childdoc.pdf| (this file).
\item
Run \LaTeX{} on |childdoc.ins| to create the definitions file |childdoc.def|
and the sample |cdocsamp.tex| with include files
|cdocsch1.tex|, |cdocsch2.tex|, |cdocspt3.tex|, |cdocspt4.tex|,
|cdocsdrf.tex|, |cdocsfn1.tex|, |cdocsfn2.tex|.
Then copy the file |childdoc.def| to an appropriate directory of your \LaTeX{}
distribution, e.g.\ \textit{texmf-root}|/tex/latex/childdoc|.
\end{itemize}

%%%%%%%%%%%%%%%%%%%%%%%%%%%%%%%%%%%%%%%%%%%%%%%%%%%%%%%%%%%%%%%%%%%%%%%%%%%%%%%%
\subsection{Related CTAN Packages}

There are several other packages which offer a similar functionality:
%
\begin{itemize}
\item
The packages
\href{http://ctan.org/pkg/docmute}{\textsf{docmute}},
\href{http://ctan.org/pkg/includex}{\textsf{includex}} and
\href{http://ctan.org/pkg/standalone}{\textsf{standalone}}
provide commands to include only the document body of
a child file thus allowing both files to be compiled individually.
\item
The packages \href{http://ctan.org/pkg/subdocs}{\textsf{subdocs}}
and \href{http://ctan.org/pkg/subfiles}{\textsf{subfiles}}
provide structures in which the main and child documents can be
encapsulated and allowing them to be compiled individually.
The inclusion mechanism is different from the conventional |\include|.
\item
The package \href{http://ctan.org/pkg/combine}{\textsf{combine}}
is an elaborate solution to combine several documents into one.
\end{itemize}
%
See also the CTAN topic \href{http://ctan.org/topic/subdocs}{\textsf{subdocs}}
for further related packages.
The present package differs from the above solutions in that
a document structure constructed with the conventional |\include| mechanism
just needs two extra commands at the top of every file
such that all constituent files can be compiled individually.

%%%%%%%%%%%%%%%%%%%%%%%%%%%%%%%%%%%%%%%%%%%%%%%%%%%%%%%%%%%%%%%%%%%%%%%%%%%%%%%%
%\subsection{Feature Suggestions}
%
%The following is a list of features which may be useful for future
%versions of this package:
%%
%\begin{itemize}
%\item
%\ldots
%\end{itemize}

%%%%%%%%%%%%%%%%%%%%%%%%%%%%%%%%%%%%%%%%%%%%%%%%%%%%%%%%%%%%%%%%%%%%%%%%%%%%%%%%
\subsection{Revision History}

%%%%%%%%%%%%%%%%%%%%%%%%%%%%%%%%%%%%%%%%
\paragraph{v2.0:} 2018/12/30

\begin{itemize}
\item
immediate forward processing
\item
added |\childdocby| mechanism
\item
manual restructured
\end{itemize}

%%%%%%%%%%%%%%%%%%%%%%%%%%%%%%%%%%%%%%%%
\paragraph{v1.6:} 2018/01/17

\begin{itemize}
\item
application for development of include files
\item
corrections to manual
\end{itemize}

%%%%%%%%%%%%%%%%%%%%%%%%%%%%%%%%%%%%%%%%
\paragraph{v1.5:} 2017/05/21

\begin{itemize}
\item
more complete structuring introduced
\item
|\childdocof| introduced
\item
|\childdoc| renamed to |\childdocmain|
\item
|\childredirect| renamed to |\childdocforward| and |\childdocforwardprefix|
and functionality expanded
\end{itemize}

%%%%%%%%%%%%%%%%%%%%%%%%%%%%%%%%%%%%%%%%
\paragraph{v1.0:} 2017/04/27

\begin{itemize}
\item
manual and install package
\item
first version published on CTAN
\end{itemize}

%%%%%%%%%%%%%%%%%%%%%%%%%%%%%%%%%%%%%%%%
\paragraph{v0.6:} 2017/04/26

\begin{itemize}
\item
redirection mechanism added
\end{itemize}

%%%%%%%%%%%%%%%%%%%%%%%%%%%%%%%%%%%%%%%%
\paragraph{v0.5:} 2017/04/26

\begin{itemize}
\item
functionality in definition file
\end{itemize}


%%%%%%%%%%%%%%%%%%%%%%%%%%%%%%%%%%%%%%%%%%%%%%%%%%%%%%%%%%%%%%%%%%%%%%%%%%%%%%%%
%%%%%%%%%%%%%%%%%%%%%%%%%%%%%%%%%%%%%%%%%%%%%%%%%%%%%%%%%%%%%%%%%%%%%%%%%%%%%%%%
%%%%%%%%%%%%%%%%%%%%%%%%%%%%%%%%%%%%%%%%%%%%%%%%%%%%%%%%%%%%%%%%%%%%%%%%%%%%%%%%
\appendix

\settowidth\MacroIndent{\rmfamily\scriptsize 000\ }

 \DocInput{childdoc.dtx}

\end{document}
%</driver>
% \fi
%
% %%%%%%%%%%%%%%%%%%%%%%%%%%%%%%%%%%%%%%%%%%%%%%%%%%%%%%%%%%%%%%%%%%%%%%%%%%%%%%
% %%%%%%%%%%%%%%%%%%%%%%%%%%%%%%%%%%%%%%%%%%%%%%%%%%%%%%%%%%%%%%%%%%%%%%%%%%%%%%
% \section{Sample}
%\iffalse
%<*samplemain>
%\fi
%
% The following presents a sample document
% with two chapters, two parts, a title page,
% a compile flag as well as three forwarding files to set the flag.
% It consists of eight |.tex| files:
% \begin{center}
% \begin{tabular}{ll}
% |cdocsamp.tex|&main file\\
% |cdocsch1.tex|&include file for chapter 1\\
% |cdocsch2.tex|&include file for chapter 2\\
% |cdocspt3.tex|&include file for part 3\\
% |cdocspt4.tex|&include file for part 4\\
% |cdocsdrf.tex|&forwarding file for main file in draft mode\\
% |cdocsfi1.tex|&forwarding file for final version of chapter 1\\
% |cdocsfi2.tex|&forwarding file for final version of chapter 2\\
% \end{tabular}
% \end{center}
% Each of the eight files can be compiled directly by the \LaTeX{} compiler.
%
% %%%%%%%%%%%%%%%%%%%%%%%%%%%%%%%%%%%%%%
% \paragraph{Main File.}
%
% The main file is called |cdocsamp.tex|.
%
% Load the \textsf{childdoc} definitions and
% declare the filename for the main document:
%    \begin{macrocode}
\input{childdoc.def}
\childdocmain{}
%    \end{macrocode}

% Optional override for |\version| flag:
%    \begin{macrocode}
%%\ifchilddoc\else\providecommand{\version}{draft}\fi
%    \end{macrocode}

% Define the default values for the |\version| flag
% (|final| for the main file and |draft| for childs):
%    \begin{macrocode}
\ifchilddoc
\providecommand{\version}{draft}
\else
\providecommand{\version}{final}
\fi
%    \end{macrocode}

% Load the standard document class:
%    \begin{macrocode}
\documentclass[12pt]{article}
%    \end{macrocode}

% Start the document body:
%    \begin{macrocode}
\begin{document}
%    \end{macrocode}

% Declare a title page.
% Print title, part of document being processed and version flag:
%    \begin{macrocode}
\addtocounter{page}{-1}
\begin{center}
{\LARGE\bfseries{}childdoc example\par}
\vspace{1cm}
\ifchilddoc
\ifchilddocmanual part\else chapter\fi:
`\childdocname' of `\childdocjob'\par
\else
main document: `\childdocjob'\par
\fi
version: \version\par
\end{center}
\newpage
%    \end{macrocode}

% Manually include selected file,
% otherwise process as usual:
%    \begin{macrocode}
\ifchilddocmanual
\section*{part `\childdocname'}
\input{\childdocname}
\else
%    \end{macrocode}

% Include the two chapters:
%    \begin{macrocode}
\include{cdocsch1}
\include{cdocsch2}
%    \end{macrocode}

% Include the two parts unless only chapters should be displayed:
%    \begin{macrocode}
\ifchilddoc\else
\section{part three}
\input{cdocspt3}
\section{part four}
\input{cdocspt4}
\fi
%    \end{macrocode}

% Process as usual until here:
%    \begin{macrocode}
\fi
%    \end{macrocode}

% End of document body:
%    \begin{macrocode}
\end{document}
%    \end{macrocode}
%\iffalse
%</samplemain>
%\fi
%
% %%%%%%%%%%%%%%%%%%%%%%%%%%%%%%%%%%%%%%
% \paragraph{Chapter Include Files.}
%
% The include files are called |cdocsch1.tex| and |cdocsch2.tex|.
%
%\iffalse
%<*samplechap1|samplechap2>
%\fi

% Optional override for |\version| flag:
%    \begin{macrocode}
%%\providecommand{\version}{final}
%    \end{macrocode}

% Include the main document:
%    \begin{macrocode}
\input{childdoc.def}
\childdocof{cdocsamp}
%    \end{macrocode}

%\iffalse
%</samplechap1|samplechap2>
%\fi
%
%\iffalse
%<*samplechap1>
%\fi
% Some text for chapter 1:
%    \begin{macrocode}
\section{one}
some text in chapter one
%    \end{macrocode}

%\iffalse
%</samplechap1>
%\fi
% Some text for chapter 2:
%\iffalse
%<*samplechap2>
%\fi
%    \begin{macrocode}
\section{two}
more text in chapter two
%    \end{macrocode}

%\iffalse
%</samplechap2>
%\fi
%
% %%%%%%%%%%%%%%%%%%%%%%%%%%%%%%%%%%%%%%
% \paragraph{Part Include Files.}
%
% The include files are called |cdocspt3.tex| and |cdocspt4.tex|.
%
%\iffalse
%<*samplepart3|samplepart4>
%\fi

% Optional override for |\version| flag:
%    \begin{macrocode}
%%\providecommand{\version}{final}
%    \end{macrocode}

% Include the main document:
%    \begin{macrocode}
\input{childdoc.def}
\childdocby{cdocsamp}
%    \end{macrocode}

%\iffalse
%</samplepart3|samplepart4>
%\fi
%
%\iffalse
%<*samplepart3>
%\fi
% Some text for part 3:
%    \begin{macrocode}
some text in part three
%    \end{macrocode}

%\iffalse
%</samplepart3>
%\fi
% Some text for part 4:
%\iffalse
%<*samplepart4>
%\fi
%    \begin{macrocode}
more text in part four
%    \end{macrocode}

%\iffalse
%</samplepart4>
%\fi
%
% %%%%%%%%%%%%%%%%%%%%%%%%%%%%%%%%%%%%%%
% \paragraph{Forwarding for a Complete Draft.}
%
% The following forwarding file |cdocsdrf.tex|
% compiles the main document in draft mode:
%\iffalse
%<*sampledraft>
%\fi
%    \begin{macrocode}
\def\version{draft}
\input{childdoc.def}
\childdocforward{cdocsamp}
%    \end{macrocode}

%\iffalse
%</sampledraft>
%\fi
%
% %%%%%%%%%%%%%%%%%%%%%%%%%%%%%%%%%%%%%%
% \paragraph{Forwarding for Final Version of the Chapters.}
%
% The following forwarding files |cdocsfn1.tex| and |cdocsfn2.tex|
% (with identical content)
% compile the final versions of the child documents
% |cdocsch1.tex| and |cdocsch2.tex|, respectively:
%\iffalse
%<*samplefinal>
%\fi
%    \begin{macrocode}
\def\version{final}
\input{childdoc.def}
\childdocforwardprefix[cdocsamp]{cdocsfn}{cdocsch}
%    \end{macrocode}

%\iffalse
%</samplefinal>
%\fi
%
% %%%%%%%%%%%%%%%%%%%%%%%%%%%%%%%%%%%%%%
% \paragraph{Command Line Processing.}
%
% The following three command lines generate the output files
% |cdocscld|, |cdocscl1| and |cdocscl2|
% which should be identical to
% |cdocsdrf|, |cdocsch1| and |cdocsfn2|, respectively:
% \begin{center}
% \begin{tabular}{l}
% |latex -jobname cdocscld \|\\
% |  "\def\version{draft}\input{childdoc.def}\childdocforward{cdocsamp}"|\\
% |latex -jobname cdocscl1 \|\\
% |  "\input{childdoc.def}\childdocforward[cdocsamp]{cdocsch1}"|\\
% |latex -jobname cdocscl2 \|\\
% |  "\def\version{final}\input{childdoc.def}\childdocforward{cdocsch2}"|
% \end{tabular}
% \end{center}
% Note that the trailing backslash on each first line
% merely continues the input to the second line
% (for convenient cut ant paste).
% Furthermore, the command |latex| can be replaced by any
% of its alternative versions such as |pdflatex|.
%
% %%%%%%%%%%%%%%%%%%%%%%%%%%%%%%%%%%%%%%%%%%%%%%%%%%%%%%%%%%%%%%%%%%%%%%%%%%%%%%
% %%%%%%%%%%%%%%%%%%%%%%%%%%%%%%%%%%%%%%%%%%%%%%%%%%%%%%%%%%%%%%%%%%%%%%%%%%%%%%
% \section{Implementation}
%\iffalse
%<*package>
%\fi
%
% This section describes the definitions file |childdoc.def|.

% The definitions cannot be loaded using |\usepackage| or |\RequirePackage|
% which has a mechanism to prevent loading a style file more than once.
% When loading the definitions by means of |\input|
% multiple instances have to be prevented manually:
%\iffalse
%This code needs to be before the `\ProvidesFile' directive
%which is defined at the beginning of this file.
%Therefore it is also placed there and commented out here.
%</package>
%<*discard>
%\fi
%    \begin{macrocode}
\ifdefined\childdocmain\endinput\fi
%    \end{macrocode}
%\iffalse
%</discard>
%<*package>
%\fi
%
% \macro{\ifchilddoc}
% \macro{\ifchilddocmanual}
% The conditional |\ifchilddoc| tells whether a
% child (true) or main (false) document is being compiled.
% The conditional |\ifchilddocmanual| tells whether
% the |\includeonly| mechanism is used (false) or
% the selection of child files must be performed manually (true).
% The definitions initialise to false:
%    \begin{macrocode}
\newif\ifchilddoc
\newif\ifchilddocmanual
%    \end{macrocode}

% \macro{\childdocname}
% \macro{\childdocjob}
% The macro |\childdocname| stores the name of the main document
% to be compiled. The macro |\childdocjob| stores the name of
% the document on which the \LaTeX{} compiler was originally invoked.
% The content of |\jobname| cannot be compared
% to filenames specified in the source due to different catcodes.
% The following code rescans |\jobname|, stores the result
% in |\childdocname| and saves a copy in |\childdocjob|:
%    \begin{macrocode}
\edef\childdocname{\scantokens\expandafter{\jobname\noexpand}}
\let\childdocjob\childdocname
%    \end{macrocode}

% \macro{\childdocdisable}
% The macro |\childdocdisable| prevents the main file
% from being processed more than once.
% At this stage, the main document command |\childdocmain|
% is assumed to be called once again where it should do nothing.
% Any subsequent call to it should prevent
% a secondary processing of the main document
% It overwrites the forwarding commands
% |\childdocof| and |\childdocforward|
% with empty macros to prevent further inclusions of the main document:
%    \begin{macrocode}
\newcommand{\childdocdisable}
{
  \renewcommand{\childdocmain}[1]{\renewcommand{\childdocmain}[1]{\endinput}}
  \renewcommand{\childdocof}[1]{}
  \renewcommand{\childdocby}[2][]{}
  \renewcommand{\childdocforward}[2][]{}
  \renewcommand{\childdocdisable}{}
}
%    \end{macrocode}

% \macro{\childdocmain}
% The macro |\childdocmain| is to be called at the top of the main file
% with nothing or the main filename (without extension) as argument.
% First, it breaks loops.
% If the argument is not empty and does not match |\childdocname|
% (which is set by the first inclusion of |childdoc.def|),
% |\ifchilddoc| is set to true, |\includeonly| is applied to the child file
% and |\jobname| is set to the main file
% (for proper handling of |.aux| files):
%    \begin{macrocode}
\newcommand{\childdocmain}[1]
{
  \childdocdisable\childdocmain{}
  \if?#1?\else
    \begingroup
      \def\childdoctmp{#1}
      \ifx\childdoctmp\childdocname
        \def\childdoctmp{}
      \else
        \def\childdoctmp
        {
          \childdoctrue
          \includeonly{\childdocname}
          \def\childdocjob{#1}
          \def\jobname{#1}
        }
      \fi
      \expandafter
    \endgroup
    \childdoctmp
  \fi
}
%    \end{macrocode}

% \macro{\childdocof}
% The command |\childdocof| redirects
% compilation to the main file |#1|.
%    \begin{macrocode}
\newcommand{\childdocof}[1]
{
  \childdocdisable
  \childdoctrue
  \includeonly{\childdocname}
  \def\jobname{#1}
  \def\childdocjob{#1}
  \input{#1}
}
%    \end{macrocode}

% \macro{\childdocby}
% The command |\childdocby| ....
%    \begin{macrocode}
\newcommand{\childdocby}[2][]
{
  \childdocdisable
  \childdoctrue
  \childdocmanualtrue
  \if?#1?\else
    \def\jobname{#2}
  \fi
  \def\childdocjob{#2}
  \input{#2}
  \endinput
}
%    \end{macrocode}

% \macro{\childdocforward}
% The command |\childdocforward| redirects
% compilation to the main file or
% (if the optional argument is given) a child file.
% Parameters are set as if the main file
% or a child file starting with |\childdocof| was compiled.
% Then compilation is handed over to the main file:
%    \begin{macrocode}
\newcommand{\childdocforward}[2][]
{
  \begingroup
    \if?#1?
      \def\childdoctmp
      {
        \def\childdocname{#2}
        \def\childdocjob{#2}
        \def\jobname{#2}
        \input{#2}
        \endinput
      }
    \else
      \def\childdoctmp
      {
        \childdocdisable
        \def\childdocname{#2}
        \childdoctrue
        \includeonly{#2}
        \def\childdocjob{#1}
        \def\jobname{#1}
        \input{#1}
        \endinput
      }
    \fi
    \expandafter
  \endgroup
  \childdoctmp
}
%    \end{macrocode}

% \macro{\childdocforwardprefix}
% The command |\childdocforwardprefix| redirects
% compilation to the main or a child file by means of a pattern.
% The prefix |#1| in the current filename is replaced by |#2|
% and the suffix of the current filename is kept
% (it is assumed that the filename does not contain the substring `|~~~|'
% which is used as a delimiter).
% Compilation is handed over to the new file by |\childdocforward|:
%    \begin{macrocode}
\newcommand{\childdocforwardprefix}[3][]
{
  \begingroup
    \def\childdocextract #2##1~~~{\def\childdoctmp{\childdocforward[#1]{#3##1}}}
    \expandafter\childdocextract\childdocname~~~
    \expandafter
  \endgroup
  \childdoctmp
}
%    \end{macrocode}

% \macro{\childdoc}
% The deprecated macro |\childdoc| is a legacy version of |\childdocmain|:
%    \begin{macrocode}
\newcommand{\childdoc}{\childdocmain}
%    \end{macrocode}

% \macro{\childdocredirect}
% The deprecated macro |\childdocredirect| is a legacy version
% of |\childdocforward| and |\childdocforwardprefix|:
%    \begin{macrocode}
\newcommand{\childdocredirect}[2][]
{
  \begingroup
    \if?#1?
      \def\childdoctmp{\childdocforward{#2}}
    \else
      \def\childdoctmp{\childdocforwardprefix{#1}{#2}}
    \fi
    \expandafter
  \endgroup
  \childdoctmp
}
%    \end{macrocode}

%\iffalse
%</package>
%\fi
%
\endinput
|\\
|\childdocmain{|\textit{main}|}|\\
\end{tabular}
\end{center}
%
If |\jobname| does not match the argument \textit{main} of |\childdocmain|,
it is assumed that |\jobname| points to the child file to be compiled.
When using |\childdocmain| with the main file specified as argument,
it suffices to start a child file
with just |\input{|\textit{main}|}|
without loading of the package and using |\childdocof|.
If instead all processing is done
with the appropriate \textsf{childdoc} directives,
the argument of \textit{main} of |\childdocmain| can be empty.

An alternative version of the command line processing described
in \secref{sec:commandline} using the detection mechanism reads:
%
\begin{center}
|... -jobname "|\textit{target}|" "|[\textit{flags}]%
[|\def\jobname{|\textit{dest}|}|]|\input{|\textit{main}|}"|
\end{center}

%%%%%%%%%%%%%%%%%%%%%%%%%%%%%%%%%%%%%%%%%%%%%%%%%%%%%%%%%%%%%%%%%%%%%%%%%%%%%%%%
\subsection{Manual Code}
\label{sec:manual}

In case one cannot be certain whether the definitions file |childdoc.def|
is installed on the target \TeX{} distribution
and one prefers not to ship it,
it is conceivable to paste a few relevant commands into the sources.

To that end, drop all statements |% \iffalse
%
% childdoc.dtx Copyright (C) 2017-2018 Niklas Beisert
%
% This work may be distributed and/or modified under the
% conditions of the LaTeX Project Public License, either version 1.3
% of this license or (at your option) any later version.
% The latest version of this license is in
%   http://www.latex-project.org/lppl.txt
% and version 1.3 or later is part of all distributions of LaTeX
% version 2005/12/01 or later.
%
% This work has the LPPL maintenance status `maintained'.
%
% The Current Maintainer of this work is Niklas Beisert.
%
% This work consists of the files childdoc.dtx and childdoc.ins
% and the derived files childdoc.def and cdocsamp.tex with
% cdocsch1.tex, cdocsch2.tex, cdocsdrf.tex, cdocsfn1.tex, cdocsfn2.tex.
%
%<package>\ifdefined\childdocmain\endinput\fi
%<package>\ProvidesFile{childdoc.def}[2018/12/30 v2.0 child document driver]
%<samplemain>\ProvidesFile{cdocsamp.tex}[2018/12/30 v2.0 sample for childdoc]
%<*driver>
%\ProvidesFile{childdoc.drv}[2018/12/30 v2.0 childdoc reference manual file]
\PassOptionsToClass{10pt,a4paper}{article}
\documentclass{ltxdoc}

\usepackage[margin=35mm]{geometry}
\usepackage{hyperref}
\usepackage{hyperxmp}
\usepackage[usenames]{color}

\hypersetup{colorlinks=true}
\hypersetup{pdfstartview=FitH}
\hypersetup{pdfpagemode=UseNone}
\hypersetup{pdfsource={}}
\hypersetup{pdflang={en-UK}}
\hypersetup{pdfcopyright={Copyright 2017-2018 Niklas Beisert.
  This work may be distributed and/or modified under the
  conditions of the LaTeX Project Public License, either version 1.3
  of this license or (at your option) any later version.}}
\hypersetup{pdflicenseurl={http://www.latex-project.org/lppl.txt}}
\hypersetup{pdfcontactaddress={ETH Zurich, ITP, HIT K,
  Wolfgang-Pauli-Strasse 27}}
\hypersetup{pdfcontactpostcode={8093}}
\hypersetup{pdfcontactcity={Zurich}}
\hypersetup{pdfcontactcountry={Switzerland}}
\hypersetup{pdfcontactemail={nbeisert@itp.phys.ethz.ch}}
\hypersetup{pdfcontacturl={http://people.phys.ethz.ch/\xmptilde nbeisert/}}

\newcommand{\secref}[1]{\hyperref[#1]{section \ref*{#1}}}

\parskip1ex
\parindent0pt
\let\olditemize\itemize
\def\itemize{\olditemize\parskip0pt}

\begin{document}

\title{The \textsf{childdoc} Package}
\hypersetup{pdftitle={The childdoc Package}}
\author{Niklas Beisert\\[2ex]
  Institut f\"ur Theoretische Physik\\
  Eidgen\"ossische Technische Hochschule Z\"urich\\
  Wolfgang-Pauli-Strasse 27, 8093 Z\"urich, Switzerland\\[1ex]
  \href{mailto:nbeisert@itp.phys.ethz.ch}
  {\texttt{nbeisert@itp.phys.ethz.ch}}}
\hypersetup{pdfauthor={Niklas Beisert}}
\hypersetup{pdfsubject={Manual for the LaTeX2e Package childdoc}}
\date{30 December 2018, \textsf{v2.0}}
\maketitle

\begin{abstract}\noindent
\textsf{childdoc} is a \LaTeXe{} package
that enables the direct compilation
of document sections included by |\include|
to individual files.
\end{abstract}

\begingroup
\parskip0ex
\tableofcontents
\endgroup

%%%%%%%%%%%%%%%%%%%%%%%%%%%%%%%%%%%%%%%%%%%%%%%%%%%%%%%%%%%%%%%%%%%%%%%%%%%%%%%%
%%%%%%%%%%%%%%%%%%%%%%%%%%%%%%%%%%%%%%%%%%%%%%%%%%%%%%%%%%%%%%%%%%%%%%%%%%%%%%%%
\section{Introduction}

\LaTeX{} provides a mechanism to structure a large document (such as a book)
into a main file and several child files (containing the chapters)
using the |\include| command.
This mechanism is beneficial for documents
which span hundreds of pages in order to
make the source file(s) more manageable.
Moreover, compilation can be restricted to
selected child files by means of the |\includeonly| command.
The latter feature can be used to reduce the compilation time while editing
(this was significantly more useful in the earlier days of \LaTeX{})
or to generate a smaller document which is easier to navigate.
Another application of |\includeonly| is to generate
documents consisting of selected parts of the complete document.

However, there are a few drawbacks of the plain |\include| mechanism:
\begin{itemize}
\item
The child files cannot be compiled on their own,
they can only be compiled via the main file.
A naive editing environment
(such as a text editor with an option
to have the current file processed by \LaTeX)
may require one to switch to the main file before compiling;
attempting to compile the child file produces errors.
\item
The main file must be modified (each time)
to adjust the |\includeonly| command
to the present needs. This easily leaves the main file in a messy state.
\item
The generated document will always carry the filename
of the main document. This is inconvenient if
several child files are to be compiled and
to be kept for distribution.
\end{itemize}

The present package provides a simple interface
to make child files individually compilable by \LaTeX{}.
Compiling a child file then has the same effect as compiling
the main file with an |\includeonly| command
to select the appropriate child.
Moreover the generated document will carry the name of the child
rather than the main file.
This resolves all three above issues.

This feature is meant to make the editing of books,
thesis documents and lecture notes somewhat more convenient.
However, the package can also be used efficiently for
composing a series of documents (such as exercise sheets)
which are typically distributed individually.
It then assists the author in generating the individual documents
(potentially in different versions)
as well as a document containing the collected series.
Another application is in developing style files
or other kinds of included material
where compilation of the style file could redirect
to a sample or test file.

%%%%%%%%%%%%%%%%%%%%%%%%%%%%%%%%%%%%%%%%%%%%%%%%%%%%%%%%%%%%%%%%%%%%%%%%%%%%%%%%
%%%%%%%%%%%%%%%%%%%%%%%%%%%%%%%%%%%%%%%%%%%%%%%%%%%%%%%%%%%%%%%%%%%%%%%%%%%%%%%%
\section{Usage}

First of all, the package \textsf{childdoc} is \emph{not} a standard
\LaTeXe{} |.sty| style file! Therefore it needs to be invoked in
a non-standard way.

%%%%%%%%%%%%%%%%%%%%%%%%%%%%%%%%%%%%%%%%%%%%%%%%%%%%%%%%%%%%%%%%%%%%%%%%%%%%%%%%
\subsection{Included Files}
\label{sec:include}

%%%%%%%%%%%%%%%%%%%%%%%%%%%%%%%%%%%%%%%%
\DescribeMacro{\childdocmain}
To use the package, add the commands
\begin{center}
\begin{tabular}{l}
|\input{childdoc.def}|\\
|\childdocmain{}|\\
\end{tabular}
\end{center}
at the very top of the main \LaTeX{} file,
in particular \emph{before} the |\documentclass| statement!
The argument of |\childdocmain| should be left empty
(but it must be present).

%%%%%%%%%%%%%%%%%%%%%%%%%%%%%%%%%%%%%%%%
\DescribeMacro{\childdocof}
Furthermore, add the commands
\begin{center}
\begin{tabular}{l}
|\input{childdoc.def}|\\
|\childdocof{|\textit{main}|}|\\
\end{tabular}
\end{center}
at the top of every child file \textit{child}
which is included by |\include{|\textit{child}|}|
from within the main file
(or at least for those files to be compiled individually).
The argument \textit{main} must be the filename of the main file.

There are a couple of
considerations in setting up the main and child documents:

%%%%%%%%%%%%%%%%%%%%%%%%%%%%%%%%%%%%%%%%
\paragraph{Restrictions.}

Please note the following restrictions:
\begin{itemize}
\item
|\childdocmain| must be called with one argument \textit{main}
to ensure compatibility with earlier version of the package.
It must either be empty (|\childdocmain{}|)
or precisely match the filename of the main file in which it is specified.
See \secref{sec:detection} for further information.
\item
The filename \textit{main} must be specified without the |.tex| extension.
\item
The filename \textit{main} is case sensitive
(even in case-insensitive file systems)
due to internal string comparison.
\item
The argument \textit{main} should be fully expanded, it cannot be a macro.
\item
Subdirectories and special characters should be avoided in filenames.
\item
The command |\childdocmain{|\textit{main}|}| must be followed by a whitespace.
It should not be followed immediately by another command
or by a comment mark `|%|'.
This is because the \TeX{} parser reads the token immediately following
the argument of |\childdocmain| and puts it
at the beginning of every child section;
however, a white\-space is ignored.
\end{itemize}

%%%%%%%%%%%%%%%%%%%%%%%%%%%%%%%%%%%%%%%%
\paragraph{Content of Main File.}

It is advisable to place all content in the child files included by |\include|.
Any output contained in the main file will appear in all child documents
unless suppressed manually;
it cannot be suppressed automatically by the |\includeonly| directive
and thus should normally be avoided.
A method to include some content in the main file
by means of conditional processing is described in \secref{sec:conditional}.

%%%%%%%%%%%%%%%%%%%%%%%%%%%%%%%%%%%%%%%%
\paragraph{Page Numbering.}

When only a part of the document is compiled,
the appropriate numbering of pages
(as well as other status parameters)
is determined from the |.aux| files.
The latter contain information from previous passes.
However this information needs to propagate through
all intermediate child documents.
Therefore the page numbering in child documents may well
be inconsistent until the complete document is compiled at least once.

A useful (if unconventional) way to always ensure a consistent
page numbering is to restart the numbering in each child document
and denote the pages by `\textit{child}|.|\textit{page}'
where \textit{child} represents the chapter/section number of the child file.
This can be achieved by the command
|\numberwithin{page}{|\textit{child}|}|
of the \textsf{amsmath} package
where \textit{child} can be |chapter| or |section|
depending on the chosen structuring.
Alternatively, one can modify the macro |\thepage| appropriately
and reset the counter |page| at the start of each child file.

%%%%%%%%%%%%%%%%%%%%%%%%%%%%%%%%%%%%%%%%%%%%%%%%%%%%%%%%%%%%%%%%%%%%%%%%%%%%%%%%
\subsection{Conditional Processing}
\label{sec:conditional}

The package provides a mechanism to compile different versions
of a document. To customise the versions further some conditional processing
can come in handy to distinguish which version is being compiled.
The package provides two macros to describe the compilation context:

%%%%%%%%%%%%%%%%%%%%%%%%%%%%%%%%%%%%%%%%
\DescribeMacro{\ifchilddoc}
The conditional |\ifchilddoc| distinguishes between the compilation of
child documents and the main document:
%
\begin{center}
|\ifchilddoc |\textit{child-code}| |[|\||else |\textit{main-code}]| \||fi|
\end{center}

%%%%%%%%%%%%%%%%%%%%%%%%%%%%%%%%%%%%%%%%
\DescribeMacro{\childdocname}
\DescribeMacro{\childdocjob}
The macro |\childdocname| contains the filename (without extension)
of the main or child file being processed.
Note that |\childdocjob| will always contain the name of the main file.

%%%%%%%%%%%%%%%%%%%%%%%%%%%%%%%%%%%%%%%%
\paragraph{Title Page.}

Conditional processing can be used to include a title or banner page
in the main document when proper precautions are taken.
Importantly, the code in the main file should ensure that the page counter
(as well as other status parameters which are stored in the |.aux| files)
takes the same value after the conditional processing.
Otherwise the page numbers may take divergent values
depending on which part is compiled.

For example, a title page could be declared by:
%
\begin{center}
\begin{tabular}{l}
|\ifchilddoc\||else|\\
|\addtocounter{page}{-1}|\\
\textit{code for title page}\\
|\newpage|\\
|\||fi|
\end{tabular}
\end{center}
%
A banner page for the child documents can be generated by:
%
\begin{center}
\begin{tabular}{l}
|\ifchilddoc|\\
|\addtocounter{page}{-1}|\\
\textit{code for banner page}\\
|\newpage|\\
|\||fi|
\end{tabular}
\end{center}
%
Here one could write a message such as:
\begin{center}
|This is the part \childdocname{} of \childdocjob{}.|
\end{center}

%%%%%%%%%%%%%%%%%%%%%%%%%%%%%%%%%%%%%%%%%%%%%%%%%%%%%%%%%%%%%%%%%%%%%%%%%%%%%%%%
\subsection{Flags}
\label{sec:flags}

The package makes it easy to generate different versions
of the main or child documents.
To this end compilation flags can be defined
and assigned different default values.
They will be particularly useful in conjunction
with the forwarding mechanism described in \secref{sec:forward}.

For example, it may be useful to have a flag |\version|
which can be set to |draft| or |final|.
The document source will contain some conditional code
depending on the value of |\version|.
Suppose further, the flag should default to |final| for the main file
and to |draft| for child files
which is a natural assignment for editing the document.
This is achieved by placing the following code
in the preamble of the main document
(below the |\childdocmain| directive):
%
\begin{center}
\begin{tabular}{l}
|\ifchilddoc|\\
|\providecommand{\version}{draft}|\\
|\||else|\\
|\providecommand{\version}{final}|\\
|\||fi|
\end{tabular}
\end{center}
%
The definition by |\providecommand| makes sure
that previous definitions are not overwritten.
Further statements |\providecommand{\version}{...}|
can thus be added before the above code to override it.

For the main file, one might add a line
(between |\childdocmain| and the above block)
%
\begin{center}
|%\ifchilddoc\||else\providecommand{\version}{draft}\||fi|
\end{center}
%
which can be uncommented to produce a draft version.
Likewise one can add a line to the very top of a child file
(above the |\childdocof{|\textit{main}|}| directive)
%
\begin{center}
|%\providecommand{\version}{final}|
\end{center}
%
which can be uncommented to produce the final version of this child document.

%%%%%%%%%%%%%%%%%%%%%%%%%%%%%%%%%%%%%%%%%%%%%%%%%%%%%%%%%%%%%%%%%%%%%%%%%%%%%%%%
\subsection{Forwarding}
\label{sec:forward}

Different versions of the main or child documents
using compilation flags as described in \secref{sec:flags}
can be (permanently) stored in different files
for convenient compilation, viewing and distribution.
To this end, the package defines a command
to pass on compilation to a different file:

%%%%%%%%%%%%%%%%%%%%%%%%%%%%%%%%%%%%%%%%
\DescribeMacro{\childdocforward}
The command |\childdocforward| redirects processing to
another source file:
%
\begin{center}
\begin{tabular}{l}
|\input{childdoc.def}|\\
|\childdocforward[|\textit{main}|]{|\textit{dest}|}|\\
\end{tabular}
\end{center}
%
The argument \textit{dest} is the destination file
(without extension).
It should be the main file or one of the child files.
Note that further \textsf{childdoc} directives
such as |\childdocof| and |\childdocforward|
in the indicated file will be processed in this form.
The optional argument \textit{main}
passes on directly to the main file \textit{main}
while pretending to compile the child \textit{dest}.
This form behaves as if \textit{dest}
issues |\childdocof{|\textit{main}|}| right away,
and no further \textsf{childdoc} directives will be processed.

%%%%%%%%%%%%%%%%%%%%%%%%%%%%%%%%%%%%%%%%
\DescribeMacro{\...prefix}
In the alternative form |\childdocforwardprefix|,
%
\begin{center}
\begin{tabular}{l}
|\input{childdoc.def}|\\
|\childdocforwardprefix[|\textit{main}|]{|\textit{prefix}|}{|\textit{dest}|}|
\end{tabular}
\end{center}
%
the destination file is determined by a pattern
depending on the current file:
To make this work, the current file must be called
`{\textit{prefix}\hspace{0.2em}\textit{suffix}}'
with \textit{prefix} matching precisely the argument.
Processing is then passed on to the file
`{\textit{dest}\hspace{0.2em}\textit{suffix}}'.
Surely, the same effect is achieved by
directly specifying the
argument `{\textit{dest}\hspace{0.2em}\textit{suffix}}'
in the first form.
However, that requires to set up a different file
for each child. With the alternative form of the command
all these files can have exactly the same content
which simplifies setting them up and maintaining them.

For example, the following file |draft.tex|
with a compilation flag |\version| as described in \secref{sec:flags}
compiles the main document as a draft:
%
\begin{center}
\begin{tabular}{l}
|\def\version{draft}|\\
|\input{childdoc.def}|\\
|\childdocforward{|\textit{main}|}|
\end{tabular}
\end{center}
%
Likewise, the following files |final|\textit{nn}|.tex|
compile the final version of the child document
|child|\textit{nn}|.tex|:
%
\begin{center}
\begin{tabular}{l}
|\def\version{final}|\\
|\input{childdoc.def}|\\
|\childdocforwardprefix{final}{child}|
\end{tabular}
\end{center}
%

Note that when several versions of a main file and/or of each child file
are to be generated, it may be convenient to set up a |Makefile| or
shell script to automatise the process.

%%%%%%%%%%%%%%%%%%%%%%%%%%%%%%%%%%%%%%%%%%%%%%%%%%%%%%%%%%%%%%%%%%%%%%%%%%%%%%%%
\subsection{Command Line Processing}
\label{sec:commandline}

The effect of redirection files can also be achieved by invoking
the \LaTeX{} compiler with a more elaborate command line.
Most conveniently this should be done as part
of a shell script or a |Makefile|.

When using \textsf{childdoc} in the main file, the following
command lines effectively perform a redirection
(note that depending on the shell being used,
backslashes may have to be doubled: `|\|' $\to$ `|\\|'):
%
\begin{center}
|... -jobname "|\textit{target}|" |\\|"|[\textit{flags}]%
|\input{childdoc.def}\childdocforward[|\textit{main}|]{|\textit{dest}|}"|
\end{center}
%
Here \textit{target} is the name of the output file,
\textit{main} is the name of the main file
and \textit{dest} is the name of the main or child file to be processed
(all filenames without extensions).
The optional argument \textit{main} can be omitted
if \textit{main} matches \textit{dest}.
Optionally, compilation \textit{flags} can be defined via |\def| commands.
This command line makes the \TeX{} engine believe
it is compiling the file \textit{target}
whose content is specified as the latter parameter.
The provided code then forwards the processing to
\textit{main} or \textit{dest} as described in \secref{sec:forward}.

%%%%%%%%%%%%%%%%%%%%%%%%%%%%%%%%%%%%%%%%%%%%%%%%%%%%%%%%%%%%%%%%%%%%%%%%%%%%%%%%
\subsection{Include by Input}
\label{sec:input}

Including child documents by |\include| has some restrictions by design.
Most notably, the content of a child document always occupies
its own set of pages; pages cannot be shared between child documents.
Usually, this behaviour makes perfect sense
because each child document contain an essential part of the document.
However, in some situations it may be desirable to compose
a document from a collection of parts
without having mandatory page breaks between then.
For this case, the package
provides a mechanism to include parts
by |\input| which can also be processed individually.
However, by construction this mechanism
requires manual handling of the content to be output.

%%%%%%%%%%%%%%%%%%%%%%%%%%%%%%%%%%%%%%%%
\DescribeMacro{\ifchilddocmanual}
The main file should be prepared as usual, see \secref{sec:include}.
However, the document body must make a distinction
between processing of an individual part and of the main document, e.g.:
%
\begin{center}
\begin{tabular}{l}
|\ifchilddocmanual|\\
|\input{\childdocname}|\\
|\||else|\\
\textit{document body with }|\input{|\textit{part}|}|\\
|\||fi|
\end{tabular}
\end{center}
%
The conditional |\ifchilddocmanual| is true whenever
a part to be included by |\input| is being compiled,
and the name of the part is stored in |\childdocname|.

%%%%%%%%%%%%%%%%%%%%%%%%%%%%%%%%%%%%%%%%
\DescribeMacro{\childdocby}
Each part to be included by |\input| should start with:
%
\begin{center}
\begin{tabular}{l}
|\input{childdoc.def}|\\
|\childdocby{|\textit{main}|}|\\
\end{tabular}
\end{center}
%
The directive |\childdocby| is similar to |\childdocof|
described in \secref{sec:include},
but the subsequent selection of content must be done manually.
To that end, both |\ifchilddoc| and |\ifchilddocmanual|
will be true upon processing of a part,
and the name of the part is stored in |\childdocname|.
Note that |\jobname| will be set to the filename of the current part
so that each part receives an individual |.aux| file
that does not interfere with the |.aux| file(s) of the main document.
This behaviour can be altered by the alternative form
|\childdocby[*]{|\textit{main}|}| (with a non-empty optional argument)
which uses the |.aux| file of the main document
by setting |\jobname| to \textit{main}.

%%%%%%%%%%%%%%%%%%%%%%%%%%%%%%%%%%%%%%%%%%%%%%%%%%%%%%%%%%%%%%%%%%%%%%%%%%%%%%%%
\subsection{Driver Development}
\label{sec:driver}

The \textsf{childdoc} mechanism can also be use for the development
of definition files such as \LaTeX{} styles or classes.
This case differs from the above setup with multiple parts
included by |\include| in that no |\includeonly| should be invoked.
This can be achieved by starting the include file
(before |\ProvidesPackage|) with:
%
\begin{center}
\begin{tabular}{l}
|\input{childdoc.def}|\\
|\childdocforward{|\textit{main}|}|\\
\end{tabular}
\end{center}
%
or alternatively with:
%
\begin{center}
\begin{tabular}{l}
|\input{childdoc.def}|\\
|\childdocby{|\textit{main}|}|\\
\end{tabular}
\end{center}
%
Both forms have slightly different effects as described above.
The main file is prepared as usual, see \secref{sec:include}.

%%%%%%%%%%%%%%%%%%%%%%%%%%%%%%%%%%%%%%%%%%%%%%%%%%%%%%%%%%%%%%%%%%%%%%%%%%%%%%%%
\subsection{Legacy Detection}
\label{sec:detection}

The directive |\childdocmain| in the main file can detect
whether the complete document or merely a child is to be compiled
even without using the directive |\childdocof|.
This method is deprecated because it is less robust
and there is no compelling reason to use it;
it is merely provided for backward compatibility
and it may be removed in future versions.

If the detection mechanism is to be used,
it is mandatory to correctly specify
the filename of the main file as the argument of |\childdocmain|:
%
\begin{center}
\begin{tabular}{l}
|\input{childdoc.def}|\\
|\childdocmain{|\textit{main}|}|\\
\end{tabular}
\end{center}
%
If |\jobname| does not match the argument \textit{main} of |\childdocmain|,
it is assumed that |\jobname| points to the child file to be compiled.
When using |\childdocmain| with the main file specified as argument,
it suffices to start a child file
with just |\input{|\textit{main}|}|
without loading of the package and using |\childdocof|.
If instead all processing is done
with the appropriate \textsf{childdoc} directives,
the argument of \textit{main} of |\childdocmain| can be empty.

An alternative version of the command line processing described
in \secref{sec:commandline} using the detection mechanism reads:
%
\begin{center}
|... -jobname "|\textit{target}|" "|[\textit{flags}]%
[|\def\jobname{|\textit{dest}|}|]|\input{|\textit{main}|}"|
\end{center}

%%%%%%%%%%%%%%%%%%%%%%%%%%%%%%%%%%%%%%%%%%%%%%%%%%%%%%%%%%%%%%%%%%%%%%%%%%%%%%%%
\subsection{Manual Code}
\label{sec:manual}

In case one cannot be certain whether the definitions file |childdoc.def|
is installed on the target \TeX{} distribution
and one prefers not to ship it,
it is conceivable to paste a few relevant commands into the sources.

To that end, drop all statements |\input{childdoc.def}|
and perform the replacements as outlined below.
Instead of |\childdocmain{|\textit{main}|}| add the following code
to the top of the main file:
%
\begin{center}
\begin{tabular}{l}
|\||ifdefined\childdocname\endinput\||fi\newif\ifchilddoc|\\
|\edef\childdocname{\scantokens\expandafter{\jobname\noexpand}}|\\
|\def\childdocmain{|\textit{main}|}\||ifx\childdocmain\childdocname\||else|\\
|\childdoctrue\includeonly{\childdocname}\let\jobname\childdocmain\||fi|\\
\end{tabular}
\end{center}
%
Instead of |\childdocof{|\textit{main}|}| just include the main file
at the top of each child file:
%
\begin{center}
|\input{|\textit{main}|}|
\end{center}
%
A simple redirection |\childdocforward{|\textit{dest}|}| is achieved by:
%
\begin{center}
|\def\jobname{|\textit{dest}|}\input{\jobname}|
\end{center}
%
The redirection with prefix
|\childdocforwardprefix[|\textit{prefix}|]{|\textit{dest}|}|
is accomplished by:
%
\begin{center}
\begin{tabular}{l}
|{\edef\jobname{\scantokens\expandafter{\jobname\noexpand}}|\\
|\def\redirectjob |\textit{prefix}|#1~~~{\gdef\jobname{|\textit{dest}|#1}}|\\
|\expandafter\redirectjob\jobname~~~}\input{\jobname}|
\end{tabular}
\end{center}

In an alternative approach,
child documents can be compiled by a specific command line
without additional code or specific definitions:
%
\begin{center}
|... -jobname "|\textit{target}|" "|[\textit{flags}]%
|\includeonly{|\textit{dest}|}\input{|\textit{main}|}"|
\end{center}
%

%%%%%%%%%%%%%%%%%%%%%%%%%%%%%%%%%%%%%%%%%%%%%%%%%%%%%%%%%%%%%%%%%%%%%%%%%%%%%%%%
%%%%%%%%%%%%%%%%%%%%%%%%%%%%%%%%%%%%%%%%%%%%%%%%%%%%%%%%%%%%%%%%%%%%%%%%%%%%%%%%
\section{Information}

%%%%%%%%%%%%%%%%%%%%%%%%%%%%%%%%%%%%%%%%%%%%%%%%%%%%%%%%%%%%%%%%%%%%%%%%%%%%%%%%
\subsection{Copyright}

Copyright \copyright{} 2017--2018 Niklas Beisert

This work may be distributed and/or modified under the
conditions of the \LaTeX{} Project Public License, either version 1.3
of this license or (at your option) any later version.
The latest version of this license is in
  \url{http://www.latex-project.org/lppl.txt}
and version 1.3 or later is part of all distributions of \LaTeX{}
version 2005/12/01 or later.

This work has the LPPL maintenance status `maintained'.

The Current Maintainer of this work is Niklas Beisert.

This work consists of the files |README.txt|, |childdoc.ins| and |childdoc.dtx|
as well as the derived files |childdoc.def|, |cdocsamp.tex|
with |cdocsch1.tex|, |cdocsch2.tex|, |cdocspt3.tex|, |cdocspt4.tex|,
|cdocsdrf.tex|, |cdocsfn1.tex|, |cdocsfn2.tex|
as well as |childdoc.pdf|.

%%%%%%%%%%%%%%%%%%%%%%%%%%%%%%%%%%%%%%%%%%%%%%%%%%%%%%%%%%%%%%%%%%%%%%%%%%%%%%%%
\subsection{Files and Installation}

The package consists of the files:
%
\begin{center}
\begin{tabular}{ll}
    |README.txt|   & readme file \\
    |childdoc.ins| & installation file \\
    |childdoc.dtx| & source file \\
    |childdoc.def| & definition file \\
    |cdocsamp.tex| & sample main file \\
    |cdocsch1.tex| & sample include file \\
    |cdocsch2.tex| & sample include file \\
    |cdocspt3.tex| & sample part file \\
    |cdocspt4.tex| & sample part file \\
    |cdocsdrf.tex| & sample redirection file \\
    |cdocsfn1.tex| & sample redirection file \\
    |cdocsfn2.tex| & sample redirection file \\
    |childdoc.pdf| & manual
\end{tabular}
\end{center}
%
The distribution consists of the files
|README.txt|, |childdoc.ins| and |childdoc.dtx|.
%
\begin{itemize}
\item
Run (pdf)\LaTeX{} on |childdoc.dtx|
to compile the manual |childdoc.pdf| (this file).
\item
Run \LaTeX{} on |childdoc.ins| to create the definitions file |childdoc.def|
and the sample |cdocsamp.tex| with include files
|cdocsch1.tex|, |cdocsch2.tex|, |cdocspt3.tex|, |cdocspt4.tex|,
|cdocsdrf.tex|, |cdocsfn1.tex|, |cdocsfn2.tex|.
Then copy the file |childdoc.def| to an appropriate directory of your \LaTeX{}
distribution, e.g.\ \textit{texmf-root}|/tex/latex/childdoc|.
\end{itemize}

%%%%%%%%%%%%%%%%%%%%%%%%%%%%%%%%%%%%%%%%%%%%%%%%%%%%%%%%%%%%%%%%%%%%%%%%%%%%%%%%
\subsection{Related CTAN Packages}

There are several other packages which offer a similar functionality:
%
\begin{itemize}
\item
The packages
\href{http://ctan.org/pkg/docmute}{\textsf{docmute}},
\href{http://ctan.org/pkg/includex}{\textsf{includex}} and
\href{http://ctan.org/pkg/standalone}{\textsf{standalone}}
provide commands to include only the document body of
a child file thus allowing both files to be compiled individually.
\item
The packages \href{http://ctan.org/pkg/subdocs}{\textsf{subdocs}}
and \href{http://ctan.org/pkg/subfiles}{\textsf{subfiles}}
provide structures in which the main and child documents can be
encapsulated and allowing them to be compiled individually.
The inclusion mechanism is different from the conventional |\include|.
\item
The package \href{http://ctan.org/pkg/combine}{\textsf{combine}}
is an elaborate solution to combine several documents into one.
\end{itemize}
%
See also the CTAN topic \href{http://ctan.org/topic/subdocs}{\textsf{subdocs}}
for further related packages.
The present package differs from the above solutions in that
a document structure constructed with the conventional |\include| mechanism
just needs two extra commands at the top of every file
such that all constituent files can be compiled individually.

%%%%%%%%%%%%%%%%%%%%%%%%%%%%%%%%%%%%%%%%%%%%%%%%%%%%%%%%%%%%%%%%%%%%%%%%%%%%%%%%
%\subsection{Feature Suggestions}
%
%The following is a list of features which may be useful for future
%versions of this package:
%%
%\begin{itemize}
%\item
%\ldots
%\end{itemize}

%%%%%%%%%%%%%%%%%%%%%%%%%%%%%%%%%%%%%%%%%%%%%%%%%%%%%%%%%%%%%%%%%%%%%%%%%%%%%%%%
\subsection{Revision History}

%%%%%%%%%%%%%%%%%%%%%%%%%%%%%%%%%%%%%%%%
\paragraph{v2.0:} 2018/12/30

\begin{itemize}
\item
immediate forward processing
\item
added |\childdocby| mechanism
\item
manual restructured
\end{itemize}

%%%%%%%%%%%%%%%%%%%%%%%%%%%%%%%%%%%%%%%%
\paragraph{v1.6:} 2018/01/17

\begin{itemize}
\item
application for development of include files
\item
corrections to manual
\end{itemize}

%%%%%%%%%%%%%%%%%%%%%%%%%%%%%%%%%%%%%%%%
\paragraph{v1.5:} 2017/05/21

\begin{itemize}
\item
more complete structuring introduced
\item
|\childdocof| introduced
\item
|\childdoc| renamed to |\childdocmain|
\item
|\childredirect| renamed to |\childdocforward| and |\childdocforwardprefix|
and functionality expanded
\end{itemize}

%%%%%%%%%%%%%%%%%%%%%%%%%%%%%%%%%%%%%%%%
\paragraph{v1.0:} 2017/04/27

\begin{itemize}
\item
manual and install package
\item
first version published on CTAN
\end{itemize}

%%%%%%%%%%%%%%%%%%%%%%%%%%%%%%%%%%%%%%%%
\paragraph{v0.6:} 2017/04/26

\begin{itemize}
\item
redirection mechanism added
\end{itemize}

%%%%%%%%%%%%%%%%%%%%%%%%%%%%%%%%%%%%%%%%
\paragraph{v0.5:} 2017/04/26

\begin{itemize}
\item
functionality in definition file
\end{itemize}


%%%%%%%%%%%%%%%%%%%%%%%%%%%%%%%%%%%%%%%%%%%%%%%%%%%%%%%%%%%%%%%%%%%%%%%%%%%%%%%%
%%%%%%%%%%%%%%%%%%%%%%%%%%%%%%%%%%%%%%%%%%%%%%%%%%%%%%%%%%%%%%%%%%%%%%%%%%%%%%%%
%%%%%%%%%%%%%%%%%%%%%%%%%%%%%%%%%%%%%%%%%%%%%%%%%%%%%%%%%%%%%%%%%%%%%%%%%%%%%%%%
\appendix

\settowidth\MacroIndent{\rmfamily\scriptsize 000\ }

 \DocInput{childdoc.dtx}

\end{document}
%</driver>
% \fi
%
% %%%%%%%%%%%%%%%%%%%%%%%%%%%%%%%%%%%%%%%%%%%%%%%%%%%%%%%%%%%%%%%%%%%%%%%%%%%%%%
% %%%%%%%%%%%%%%%%%%%%%%%%%%%%%%%%%%%%%%%%%%%%%%%%%%%%%%%%%%%%%%%%%%%%%%%%%%%%%%
% \section{Sample}
%\iffalse
%<*samplemain>
%\fi
%
% The following presents a sample document
% with two chapters, two parts, a title page,
% a compile flag as well as three forwarding files to set the flag.
% It consists of eight |.tex| files:
% \begin{center}
% \begin{tabular}{ll}
% |cdocsamp.tex|&main file\\
% |cdocsch1.tex|&include file for chapter 1\\
% |cdocsch2.tex|&include file for chapter 2\\
% |cdocspt3.tex|&include file for part 3\\
% |cdocspt4.tex|&include file for part 4\\
% |cdocsdrf.tex|&forwarding file for main file in draft mode\\
% |cdocsfi1.tex|&forwarding file for final version of chapter 1\\
% |cdocsfi2.tex|&forwarding file for final version of chapter 2\\
% \end{tabular}
% \end{center}
% Each of the eight files can be compiled directly by the \LaTeX{} compiler.
%
% %%%%%%%%%%%%%%%%%%%%%%%%%%%%%%%%%%%%%%
% \paragraph{Main File.}
%
% The main file is called |cdocsamp.tex|.
%
% Load the \textsf{childdoc} definitions and
% declare the filename for the main document:
%    \begin{macrocode}
\input{childdoc.def}
\childdocmain{}
%    \end{macrocode}

% Optional override for |\version| flag:
%    \begin{macrocode}
%%\ifchilddoc\else\providecommand{\version}{draft}\fi
%    \end{macrocode}

% Define the default values for the |\version| flag
% (|final| for the main file and |draft| for childs):
%    \begin{macrocode}
\ifchilddoc
\providecommand{\version}{draft}
\else
\providecommand{\version}{final}
\fi
%    \end{macrocode}

% Load the standard document class:
%    \begin{macrocode}
\documentclass[12pt]{article}
%    \end{macrocode}

% Start the document body:
%    \begin{macrocode}
\begin{document}
%    \end{macrocode}

% Declare a title page.
% Print title, part of document being processed and version flag:
%    \begin{macrocode}
\addtocounter{page}{-1}
\begin{center}
{\LARGE\bfseries{}childdoc example\par}
\vspace{1cm}
\ifchilddoc
\ifchilddocmanual part\else chapter\fi:
`\childdocname' of `\childdocjob'\par
\else
main document: `\childdocjob'\par
\fi
version: \version\par
\end{center}
\newpage
%    \end{macrocode}

% Manually include selected file,
% otherwise process as usual:
%    \begin{macrocode}
\ifchilddocmanual
\section*{part `\childdocname'}
\input{\childdocname}
\else
%    \end{macrocode}

% Include the two chapters:
%    \begin{macrocode}
\include{cdocsch1}
\include{cdocsch2}
%    \end{macrocode}

% Include the two parts unless only chapters should be displayed:
%    \begin{macrocode}
\ifchilddoc\else
\section{part three}
\input{cdocspt3}
\section{part four}
\input{cdocspt4}
\fi
%    \end{macrocode}

% Process as usual until here:
%    \begin{macrocode}
\fi
%    \end{macrocode}

% End of document body:
%    \begin{macrocode}
\end{document}
%    \end{macrocode}
%\iffalse
%</samplemain>
%\fi
%
% %%%%%%%%%%%%%%%%%%%%%%%%%%%%%%%%%%%%%%
% \paragraph{Chapter Include Files.}
%
% The include files are called |cdocsch1.tex| and |cdocsch2.tex|.
%
%\iffalse
%<*samplechap1|samplechap2>
%\fi

% Optional override for |\version| flag:
%    \begin{macrocode}
%%\providecommand{\version}{final}
%    \end{macrocode}

% Include the main document:
%    \begin{macrocode}
\input{childdoc.def}
\childdocof{cdocsamp}
%    \end{macrocode}

%\iffalse
%</samplechap1|samplechap2>
%\fi
%
%\iffalse
%<*samplechap1>
%\fi
% Some text for chapter 1:
%    \begin{macrocode}
\section{one}
some text in chapter one
%    \end{macrocode}

%\iffalse
%</samplechap1>
%\fi
% Some text for chapter 2:
%\iffalse
%<*samplechap2>
%\fi
%    \begin{macrocode}
\section{two}
more text in chapter two
%    \end{macrocode}

%\iffalse
%</samplechap2>
%\fi
%
% %%%%%%%%%%%%%%%%%%%%%%%%%%%%%%%%%%%%%%
% \paragraph{Part Include Files.}
%
% The include files are called |cdocspt3.tex| and |cdocspt4.tex|.
%
%\iffalse
%<*samplepart3|samplepart4>
%\fi

% Optional override for |\version| flag:
%    \begin{macrocode}
%%\providecommand{\version}{final}
%    \end{macrocode}

% Include the main document:
%    \begin{macrocode}
\input{childdoc.def}
\childdocby{cdocsamp}
%    \end{macrocode}

%\iffalse
%</samplepart3|samplepart4>
%\fi
%
%\iffalse
%<*samplepart3>
%\fi
% Some text for part 3:
%    \begin{macrocode}
some text in part three
%    \end{macrocode}

%\iffalse
%</samplepart3>
%\fi
% Some text for part 4:
%\iffalse
%<*samplepart4>
%\fi
%    \begin{macrocode}
more text in part four
%    \end{macrocode}

%\iffalse
%</samplepart4>
%\fi
%
% %%%%%%%%%%%%%%%%%%%%%%%%%%%%%%%%%%%%%%
% \paragraph{Forwarding for a Complete Draft.}
%
% The following forwarding file |cdocsdrf.tex|
% compiles the main document in draft mode:
%\iffalse
%<*sampledraft>
%\fi
%    \begin{macrocode}
\def\version{draft}
\input{childdoc.def}
\childdocforward{cdocsamp}
%    \end{macrocode}

%\iffalse
%</sampledraft>
%\fi
%
% %%%%%%%%%%%%%%%%%%%%%%%%%%%%%%%%%%%%%%
% \paragraph{Forwarding for Final Version of the Chapters.}
%
% The following forwarding files |cdocsfn1.tex| and |cdocsfn2.tex|
% (with identical content)
% compile the final versions of the child documents
% |cdocsch1.tex| and |cdocsch2.tex|, respectively:
%\iffalse
%<*samplefinal>
%\fi
%    \begin{macrocode}
\def\version{final}
\input{childdoc.def}
\childdocforwardprefix[cdocsamp]{cdocsfn}{cdocsch}
%    \end{macrocode}

%\iffalse
%</samplefinal>
%\fi
%
% %%%%%%%%%%%%%%%%%%%%%%%%%%%%%%%%%%%%%%
% \paragraph{Command Line Processing.}
%
% The following three command lines generate the output files
% |cdocscld|, |cdocscl1| and |cdocscl2|
% which should be identical to
% |cdocsdrf|, |cdocsch1| and |cdocsfn2|, respectively:
% \begin{center}
% \begin{tabular}{l}
% |latex -jobname cdocscld \|\\
% |  "\def\version{draft}\input{childdoc.def}\childdocforward{cdocsamp}"|\\
% |latex -jobname cdocscl1 \|\\
% |  "\input{childdoc.def}\childdocforward[cdocsamp]{cdocsch1}"|\\
% |latex -jobname cdocscl2 \|\\
% |  "\def\version{final}\input{childdoc.def}\childdocforward{cdocsch2}"|
% \end{tabular}
% \end{center}
% Note that the trailing backslash on each first line
% merely continues the input to the second line
% (for convenient cut ant paste).
% Furthermore, the command |latex| can be replaced by any
% of its alternative versions such as |pdflatex|.
%
% %%%%%%%%%%%%%%%%%%%%%%%%%%%%%%%%%%%%%%%%%%%%%%%%%%%%%%%%%%%%%%%%%%%%%%%%%%%%%%
% %%%%%%%%%%%%%%%%%%%%%%%%%%%%%%%%%%%%%%%%%%%%%%%%%%%%%%%%%%%%%%%%%%%%%%%%%%%%%%
% \section{Implementation}
%\iffalse
%<*package>
%\fi
%
% This section describes the definitions file |childdoc.def|.

% The definitions cannot be loaded using |\usepackage| or |\RequirePackage|
% which has a mechanism to prevent loading a style file more than once.
% When loading the definitions by means of |\input|
% multiple instances have to be prevented manually:
%\iffalse
%This code needs to be before the `\ProvidesFile' directive
%which is defined at the beginning of this file.
%Therefore it is also placed there and commented out here.
%</package>
%<*discard>
%\fi
%    \begin{macrocode}
\ifdefined\childdocmain\endinput\fi
%    \end{macrocode}
%\iffalse
%</discard>
%<*package>
%\fi
%
% \macro{\ifchilddoc}
% \macro{\ifchilddocmanual}
% The conditional |\ifchilddoc| tells whether a
% child (true) or main (false) document is being compiled.
% The conditional |\ifchilddocmanual| tells whether
% the |\includeonly| mechanism is used (false) or
% the selection of child files must be performed manually (true).
% The definitions initialise to false:
%    \begin{macrocode}
\newif\ifchilddoc
\newif\ifchilddocmanual
%    \end{macrocode}

% \macro{\childdocname}
% \macro{\childdocjob}
% The macro |\childdocname| stores the name of the main document
% to be compiled. The macro |\childdocjob| stores the name of
% the document on which the \LaTeX{} compiler was originally invoked.
% The content of |\jobname| cannot be compared
% to filenames specified in the source due to different catcodes.
% The following code rescans |\jobname|, stores the result
% in |\childdocname| and saves a copy in |\childdocjob|:
%    \begin{macrocode}
\edef\childdocname{\scantokens\expandafter{\jobname\noexpand}}
\let\childdocjob\childdocname
%    \end{macrocode}

% \macro{\childdocdisable}
% The macro |\childdocdisable| prevents the main file
% from being processed more than once.
% At this stage, the main document command |\childdocmain|
% is assumed to be called once again where it should do nothing.
% Any subsequent call to it should prevent
% a secondary processing of the main document
% It overwrites the forwarding commands
% |\childdocof| and |\childdocforward|
% with empty macros to prevent further inclusions of the main document:
%    \begin{macrocode}
\newcommand{\childdocdisable}
{
  \renewcommand{\childdocmain}[1]{\renewcommand{\childdocmain}[1]{\endinput}}
  \renewcommand{\childdocof}[1]{}
  \renewcommand{\childdocby}[2][]{}
  \renewcommand{\childdocforward}[2][]{}
  \renewcommand{\childdocdisable}{}
}
%    \end{macrocode}

% \macro{\childdocmain}
% The macro |\childdocmain| is to be called at the top of the main file
% with nothing or the main filename (without extension) as argument.
% First, it breaks loops.
% If the argument is not empty and does not match |\childdocname|
% (which is set by the first inclusion of |childdoc.def|),
% |\ifchilddoc| is set to true, |\includeonly| is applied to the child file
% and |\jobname| is set to the main file
% (for proper handling of |.aux| files):
%    \begin{macrocode}
\newcommand{\childdocmain}[1]
{
  \childdocdisable\childdocmain{}
  \if?#1?\else
    \begingroup
      \def\childdoctmp{#1}
      \ifx\childdoctmp\childdocname
        \def\childdoctmp{}
      \else
        \def\childdoctmp
        {
          \childdoctrue
          \includeonly{\childdocname}
          \def\childdocjob{#1}
          \def\jobname{#1}
        }
      \fi
      \expandafter
    \endgroup
    \childdoctmp
  \fi
}
%    \end{macrocode}

% \macro{\childdocof}
% The command |\childdocof| redirects
% compilation to the main file |#1|.
%    \begin{macrocode}
\newcommand{\childdocof}[1]
{
  \childdocdisable
  \childdoctrue
  \includeonly{\childdocname}
  \def\jobname{#1}
  \def\childdocjob{#1}
  \input{#1}
}
%    \end{macrocode}

% \macro{\childdocby}
% The command |\childdocby| ....
%    \begin{macrocode}
\newcommand{\childdocby}[2][]
{
  \childdocdisable
  \childdoctrue
  \childdocmanualtrue
  \if?#1?\else
    \def\jobname{#2}
  \fi
  \def\childdocjob{#2}
  \input{#2}
  \endinput
}
%    \end{macrocode}

% \macro{\childdocforward}
% The command |\childdocforward| redirects
% compilation to the main file or
% (if the optional argument is given) a child file.
% Parameters are set as if the main file
% or a child file starting with |\childdocof| was compiled.
% Then compilation is handed over to the main file:
%    \begin{macrocode}
\newcommand{\childdocforward}[2][]
{
  \begingroup
    \if?#1?
      \def\childdoctmp
      {
        \def\childdocname{#2}
        \def\childdocjob{#2}
        \def\jobname{#2}
        \input{#2}
        \endinput
      }
    \else
      \def\childdoctmp
      {
        \childdocdisable
        \def\childdocname{#2}
        \childdoctrue
        \includeonly{#2}
        \def\childdocjob{#1}
        \def\jobname{#1}
        \input{#1}
        \endinput
      }
    \fi
    \expandafter
  \endgroup
  \childdoctmp
}
%    \end{macrocode}

% \macro{\childdocforwardprefix}
% The command |\childdocforwardprefix| redirects
% compilation to the main or a child file by means of a pattern.
% The prefix |#1| in the current filename is replaced by |#2|
% and the suffix of the current filename is kept
% (it is assumed that the filename does not contain the substring `|~~~|'
% which is used as a delimiter).
% Compilation is handed over to the new file by |\childdocforward|:
%    \begin{macrocode}
\newcommand{\childdocforwardprefix}[3][]
{
  \begingroup
    \def\childdocextract #2##1~~~{\def\childdoctmp{\childdocforward[#1]{#3##1}}}
    \expandafter\childdocextract\childdocname~~~
    \expandafter
  \endgroup
  \childdoctmp
}
%    \end{macrocode}

% \macro{\childdoc}
% The deprecated macro |\childdoc| is a legacy version of |\childdocmain|:
%    \begin{macrocode}
\newcommand{\childdoc}{\childdocmain}
%    \end{macrocode}

% \macro{\childdocredirect}
% The deprecated macro |\childdocredirect| is a legacy version
% of |\childdocforward| and |\childdocforwardprefix|:
%    \begin{macrocode}
\newcommand{\childdocredirect}[2][]
{
  \begingroup
    \if?#1?
      \def\childdoctmp{\childdocforward{#2}}
    \else
      \def\childdoctmp{\childdocforwardprefix{#1}{#2}}
    \fi
    \expandafter
  \endgroup
  \childdoctmp
}
%    \end{macrocode}

%\iffalse
%</package>
%\fi
%
\endinput
|
and perform the replacements as outlined below.
Instead of |\childdocmain{|\textit{main}|}| add the following code
to the top of the main file:
%
\begin{center}
\begin{tabular}{l}
|\||ifdefined\childdocname\endinput\||fi\newif\ifchilddoc|\\
|\edef\childdocname{\scantokens\expandafter{\jobname\noexpand}}|\\
|\def\childdocmain{|\textit{main}|}\||ifx\childdocmain\childdocname\||else|\\
|\childdoctrue\includeonly{\childdocname}\let\jobname\childdocmain\||fi|\\
\end{tabular}
\end{center}
%
Instead of |\childdocof{|\textit{main}|}| just include the main file
at the top of each child file:
%
\begin{center}
|\input{|\textit{main}|}|
\end{center}
%
A simple redirection |\childdocforward{|\textit{dest}|}| is achieved by:
%
\begin{center}
|\def\jobname{|\textit{dest}|}\input{\jobname}|
\end{center}
%
The redirection with prefix
|\childdocforwardprefix[|\textit{prefix}|]{|\textit{dest}|}|
is accomplished by:
%
\begin{center}
\begin{tabular}{l}
|{\edef\jobname{\scantokens\expandafter{\jobname\noexpand}}|\\
|\def\redirectjob |\textit{prefix}|#1~~~{\gdef\jobname{|\textit{dest}|#1}}|\\
|\expandafter\redirectjob\jobname~~~}\input{\jobname}|
\end{tabular}
\end{center}

In an alternative approach,
child documents can be compiled by a specific command line
without additional code or specific definitions:
%
\begin{center}
|... -jobname "|\textit{target}|" "|[\textit{flags}]%
|\includeonly{|\textit{dest}|}\input{|\textit{main}|}"|
\end{center}
%

%%%%%%%%%%%%%%%%%%%%%%%%%%%%%%%%%%%%%%%%%%%%%%%%%%%%%%%%%%%%%%%%%%%%%%%%%%%%%%%%
%%%%%%%%%%%%%%%%%%%%%%%%%%%%%%%%%%%%%%%%%%%%%%%%%%%%%%%%%%%%%%%%%%%%%%%%%%%%%%%%
\section{Information}

%%%%%%%%%%%%%%%%%%%%%%%%%%%%%%%%%%%%%%%%%%%%%%%%%%%%%%%%%%%%%%%%%%%%%%%%%%%%%%%%
\subsection{Copyright}

Copyright \copyright{} 2017--2018 Niklas Beisert

This work may be distributed and/or modified under the
conditions of the \LaTeX{} Project Public License, either version 1.3
of this license or (at your option) any later version.
The latest version of this license is in
  \url{http://www.latex-project.org/lppl.txt}
and version 1.3 or later is part of all distributions of \LaTeX{}
version 2005/12/01 or later.

This work has the LPPL maintenance status `maintained'.

The Current Maintainer of this work is Niklas Beisert.

This work consists of the files |README.txt|, |childdoc.ins| and |childdoc.dtx|
as well as the derived files |childdoc.def|, |cdocsamp.tex|
with |cdocsch1.tex|, |cdocsch2.tex|, |cdocspt3.tex|, |cdocspt4.tex|,
|cdocsdrf.tex|, |cdocsfn1.tex|, |cdocsfn2.tex|
as well as |childdoc.pdf|.

%%%%%%%%%%%%%%%%%%%%%%%%%%%%%%%%%%%%%%%%%%%%%%%%%%%%%%%%%%%%%%%%%%%%%%%%%%%%%%%%
\subsection{Files and Installation}

The package consists of the files:
%
\begin{center}
\begin{tabular}{ll}
    |README.txt|   & readme file \\
    |childdoc.ins| & installation file \\
    |childdoc.dtx| & source file \\
    |childdoc.def| & definition file \\
    |cdocsamp.tex| & sample main file \\
    |cdocsch1.tex| & sample include file \\
    |cdocsch2.tex| & sample include file \\
    |cdocspt3.tex| & sample part file \\
    |cdocspt4.tex| & sample part file \\
    |cdocsdrf.tex| & sample redirection file \\
    |cdocsfn1.tex| & sample redirection file \\
    |cdocsfn2.tex| & sample redirection file \\
    |childdoc.pdf| & manual
\end{tabular}
\end{center}
%
The distribution consists of the files
|README.txt|, |childdoc.ins| and |childdoc.dtx|.
%
\begin{itemize}
\item
Run (pdf)\LaTeX{} on |childdoc.dtx|
to compile the manual |childdoc.pdf| (this file).
\item
Run \LaTeX{} on |childdoc.ins| to create the definitions file |childdoc.def|
and the sample |cdocsamp.tex| with include files
|cdocsch1.tex|, |cdocsch2.tex|, |cdocspt3.tex|, |cdocspt4.tex|,
|cdocsdrf.tex|, |cdocsfn1.tex|, |cdocsfn2.tex|.
Then copy the file |childdoc.def| to an appropriate directory of your \LaTeX{}
distribution, e.g.\ \textit{texmf-root}|/tex/latex/childdoc|.
\end{itemize}

%%%%%%%%%%%%%%%%%%%%%%%%%%%%%%%%%%%%%%%%%%%%%%%%%%%%%%%%%%%%%%%%%%%%%%%%%%%%%%%%
\subsection{Related CTAN Packages}

There are several other packages which offer a similar functionality:
%
\begin{itemize}
\item
The packages
\href{http://ctan.org/pkg/docmute}{\textsf{docmute}},
\href{http://ctan.org/pkg/includex}{\textsf{includex}} and
\href{http://ctan.org/pkg/standalone}{\textsf{standalone}}
provide commands to include only the document body of
a child file thus allowing both files to be compiled individually.
\item
The packages \href{http://ctan.org/pkg/subdocs}{\textsf{subdocs}}
and \href{http://ctan.org/pkg/subfiles}{\textsf{subfiles}}
provide structures in which the main and child documents can be
encapsulated and allowing them to be compiled individually.
The inclusion mechanism is different from the conventional |\include|.
\item
The package \href{http://ctan.org/pkg/combine}{\textsf{combine}}
is an elaborate solution to combine several documents into one.
\end{itemize}
%
See also the CTAN topic \href{http://ctan.org/topic/subdocs}{\textsf{subdocs}}
for further related packages.
The present package differs from the above solutions in that
a document structure constructed with the conventional |\include| mechanism
just needs two extra commands at the top of every file
such that all constituent files can be compiled individually.

%%%%%%%%%%%%%%%%%%%%%%%%%%%%%%%%%%%%%%%%%%%%%%%%%%%%%%%%%%%%%%%%%%%%%%%%%%%%%%%%
%\subsection{Feature Suggestions}
%
%The following is a list of features which may be useful for future
%versions of this package:
%%
%\begin{itemize}
%\item
%\ldots
%\end{itemize}

%%%%%%%%%%%%%%%%%%%%%%%%%%%%%%%%%%%%%%%%%%%%%%%%%%%%%%%%%%%%%%%%%%%%%%%%%%%%%%%%
\subsection{Revision History}

%%%%%%%%%%%%%%%%%%%%%%%%%%%%%%%%%%%%%%%%
\paragraph{v2.0:} 2018/12/30

\begin{itemize}
\item
immediate forward processing
\item
added |\childdocby| mechanism
\item
manual restructured
\end{itemize}

%%%%%%%%%%%%%%%%%%%%%%%%%%%%%%%%%%%%%%%%
\paragraph{v1.6:} 2018/01/17

\begin{itemize}
\item
application for development of include files
\item
corrections to manual
\end{itemize}

%%%%%%%%%%%%%%%%%%%%%%%%%%%%%%%%%%%%%%%%
\paragraph{v1.5:} 2017/05/21

\begin{itemize}
\item
more complete structuring introduced
\item
|\childdocof| introduced
\item
|\childdoc| renamed to |\childdocmain|
\item
|\childredirect| renamed to |\childdocforward| and |\childdocforwardprefix|
and functionality expanded
\end{itemize}

%%%%%%%%%%%%%%%%%%%%%%%%%%%%%%%%%%%%%%%%
\paragraph{v1.0:} 2017/04/27

\begin{itemize}
\item
manual and install package
\item
first version published on CTAN
\end{itemize}

%%%%%%%%%%%%%%%%%%%%%%%%%%%%%%%%%%%%%%%%
\paragraph{v0.6:} 2017/04/26

\begin{itemize}
\item
redirection mechanism added
\end{itemize}

%%%%%%%%%%%%%%%%%%%%%%%%%%%%%%%%%%%%%%%%
\paragraph{v0.5:} 2017/04/26

\begin{itemize}
\item
functionality in definition file
\end{itemize}


%%%%%%%%%%%%%%%%%%%%%%%%%%%%%%%%%%%%%%%%%%%%%%%%%%%%%%%%%%%%%%%%%%%%%%%%%%%%%%%%
%%%%%%%%%%%%%%%%%%%%%%%%%%%%%%%%%%%%%%%%%%%%%%%%%%%%%%%%%%%%%%%%%%%%%%%%%%%%%%%%
%%%%%%%%%%%%%%%%%%%%%%%%%%%%%%%%%%%%%%%%%%%%%%%%%%%%%%%%%%%%%%%%%%%%%%%%%%%%%%%%
\appendix

\settowidth\MacroIndent{\rmfamily\scriptsize 000\ }

 \DocInput{childdoc.dtx}

\end{document}
%</driver>
% \fi
%
% %%%%%%%%%%%%%%%%%%%%%%%%%%%%%%%%%%%%%%%%%%%%%%%%%%%%%%%%%%%%%%%%%%%%%%%%%%%%%%
% %%%%%%%%%%%%%%%%%%%%%%%%%%%%%%%%%%%%%%%%%%%%%%%%%%%%%%%%%%%%%%%%%%%%%%%%%%%%%%
% \section{Sample}
%\iffalse
%<*samplemain>
%\fi
%
% The following presents a sample document
% with two chapters, two parts, a title page,
% a compile flag as well as three forwarding files to set the flag.
% It consists of eight |.tex| files:
% \begin{center}
% \begin{tabular}{ll}
% |cdocsamp.tex|&main file\\
% |cdocsch1.tex|&include file for chapter 1\\
% |cdocsch2.tex|&include file for chapter 2\\
% |cdocspt3.tex|&include file for part 3\\
% |cdocspt4.tex|&include file for part 4\\
% |cdocsdrf.tex|&forwarding file for main file in draft mode\\
% |cdocsfi1.tex|&forwarding file for final version of chapter 1\\
% |cdocsfi2.tex|&forwarding file for final version of chapter 2\\
% \end{tabular}
% \end{center}
% Each of the eight files can be compiled directly by the \LaTeX{} compiler.
%
% %%%%%%%%%%%%%%%%%%%%%%%%%%%%%%%%%%%%%%
% \paragraph{Main File.}
%
% The main file is called |cdocsamp.tex|.
%
% Load the \textsf{childdoc} definitions and
% declare the filename for the main document:
%    \begin{macrocode}
% \iffalse
%
% childdoc.dtx Copyright (C) 2017-2018 Niklas Beisert
%
% This work may be distributed and/or modified under the
% conditions of the LaTeX Project Public License, either version 1.3
% of this license or (at your option) any later version.
% The latest version of this license is in
%   http://www.latex-project.org/lppl.txt
% and version 1.3 or later is part of all distributions of LaTeX
% version 2005/12/01 or later.
%
% This work has the LPPL maintenance status `maintained'.
%
% The Current Maintainer of this work is Niklas Beisert.
%
% This work consists of the files childdoc.dtx and childdoc.ins
% and the derived files childdoc.def and cdocsamp.tex with
% cdocsch1.tex, cdocsch2.tex, cdocsdrf.tex, cdocsfn1.tex, cdocsfn2.tex.
%
%<package>\ifdefined\childdocmain\endinput\fi
%<package>\ProvidesFile{childdoc.def}[2018/12/30 v2.0 child document driver]
%<samplemain>\ProvidesFile{cdocsamp.tex}[2018/12/30 v2.0 sample for childdoc]
%<*driver>
%\ProvidesFile{childdoc.drv}[2018/12/30 v2.0 childdoc reference manual file]
\PassOptionsToClass{10pt,a4paper}{article}
\documentclass{ltxdoc}

\usepackage[margin=35mm]{geometry}
\usepackage{hyperref}
\usepackage{hyperxmp}
\usepackage[usenames]{color}

\hypersetup{colorlinks=true}
\hypersetup{pdfstartview=FitH}
\hypersetup{pdfpagemode=UseNone}
\hypersetup{pdfsource={}}
\hypersetup{pdflang={en-UK}}
\hypersetup{pdfcopyright={Copyright 2017-2018 Niklas Beisert.
  This work may be distributed and/or modified under the
  conditions of the LaTeX Project Public License, either version 1.3
  of this license or (at your option) any later version.}}
\hypersetup{pdflicenseurl={http://www.latex-project.org/lppl.txt}}
\hypersetup{pdfcontactaddress={ETH Zurich, ITP, HIT K,
  Wolfgang-Pauli-Strasse 27}}
\hypersetup{pdfcontactpostcode={8093}}
\hypersetup{pdfcontactcity={Zurich}}
\hypersetup{pdfcontactcountry={Switzerland}}
\hypersetup{pdfcontactemail={nbeisert@itp.phys.ethz.ch}}
\hypersetup{pdfcontacturl={http://people.phys.ethz.ch/\xmptilde nbeisert/}}

\newcommand{\secref}[1]{\hyperref[#1]{section \ref*{#1}}}

\parskip1ex
\parindent0pt
\let\olditemize\itemize
\def\itemize{\olditemize\parskip0pt}

\begin{document}

\title{The \textsf{childdoc} Package}
\hypersetup{pdftitle={The childdoc Package}}
\author{Niklas Beisert\\[2ex]
  Institut f\"ur Theoretische Physik\\
  Eidgen\"ossische Technische Hochschule Z\"urich\\
  Wolfgang-Pauli-Strasse 27, 8093 Z\"urich, Switzerland\\[1ex]
  \href{mailto:nbeisert@itp.phys.ethz.ch}
  {\texttt{nbeisert@itp.phys.ethz.ch}}}
\hypersetup{pdfauthor={Niklas Beisert}}
\hypersetup{pdfsubject={Manual for the LaTeX2e Package childdoc}}
\date{30 December 2018, \textsf{v2.0}}
\maketitle

\begin{abstract}\noindent
\textsf{childdoc} is a \LaTeXe{} package
that enables the direct compilation
of document sections included by |\include|
to individual files.
\end{abstract}

\begingroup
\parskip0ex
\tableofcontents
\endgroup

%%%%%%%%%%%%%%%%%%%%%%%%%%%%%%%%%%%%%%%%%%%%%%%%%%%%%%%%%%%%%%%%%%%%%%%%%%%%%%%%
%%%%%%%%%%%%%%%%%%%%%%%%%%%%%%%%%%%%%%%%%%%%%%%%%%%%%%%%%%%%%%%%%%%%%%%%%%%%%%%%
\section{Introduction}

\LaTeX{} provides a mechanism to structure a large document (such as a book)
into a main file and several child files (containing the chapters)
using the |\include| command.
This mechanism is beneficial for documents
which span hundreds of pages in order to
make the source file(s) more manageable.
Moreover, compilation can be restricted to
selected child files by means of the |\includeonly| command.
The latter feature can be used to reduce the compilation time while editing
(this was significantly more useful in the earlier days of \LaTeX{})
or to generate a smaller document which is easier to navigate.
Another application of |\includeonly| is to generate
documents consisting of selected parts of the complete document.

However, there are a few drawbacks of the plain |\include| mechanism:
\begin{itemize}
\item
The child files cannot be compiled on their own,
they can only be compiled via the main file.
A naive editing environment
(such as a text editor with an option
to have the current file processed by \LaTeX)
may require one to switch to the main file before compiling;
attempting to compile the child file produces errors.
\item
The main file must be modified (each time)
to adjust the |\includeonly| command
to the present needs. This easily leaves the main file in a messy state.
\item
The generated document will always carry the filename
of the main document. This is inconvenient if
several child files are to be compiled and
to be kept for distribution.
\end{itemize}

The present package provides a simple interface
to make child files individually compilable by \LaTeX{}.
Compiling a child file then has the same effect as compiling
the main file with an |\includeonly| command
to select the appropriate child.
Moreover the generated document will carry the name of the child
rather than the main file.
This resolves all three above issues.

This feature is meant to make the editing of books,
thesis documents and lecture notes somewhat more convenient.
However, the package can also be used efficiently for
composing a series of documents (such as exercise sheets)
which are typically distributed individually.
It then assists the author in generating the individual documents
(potentially in different versions)
as well as a document containing the collected series.
Another application is in developing style files
or other kinds of included material
where compilation of the style file could redirect
to a sample or test file.

%%%%%%%%%%%%%%%%%%%%%%%%%%%%%%%%%%%%%%%%%%%%%%%%%%%%%%%%%%%%%%%%%%%%%%%%%%%%%%%%
%%%%%%%%%%%%%%%%%%%%%%%%%%%%%%%%%%%%%%%%%%%%%%%%%%%%%%%%%%%%%%%%%%%%%%%%%%%%%%%%
\section{Usage}

First of all, the package \textsf{childdoc} is \emph{not} a standard
\LaTeXe{} |.sty| style file! Therefore it needs to be invoked in
a non-standard way.

%%%%%%%%%%%%%%%%%%%%%%%%%%%%%%%%%%%%%%%%%%%%%%%%%%%%%%%%%%%%%%%%%%%%%%%%%%%%%%%%
\subsection{Included Files}
\label{sec:include}

%%%%%%%%%%%%%%%%%%%%%%%%%%%%%%%%%%%%%%%%
\DescribeMacro{\childdocmain}
To use the package, add the commands
\begin{center}
\begin{tabular}{l}
|\input{childdoc.def}|\\
|\childdocmain{}|\\
\end{tabular}
\end{center}
at the very top of the main \LaTeX{} file,
in particular \emph{before} the |\documentclass| statement!
The argument of |\childdocmain| should be left empty
(but it must be present).

%%%%%%%%%%%%%%%%%%%%%%%%%%%%%%%%%%%%%%%%
\DescribeMacro{\childdocof}
Furthermore, add the commands
\begin{center}
\begin{tabular}{l}
|\input{childdoc.def}|\\
|\childdocof{|\textit{main}|}|\\
\end{tabular}
\end{center}
at the top of every child file \textit{child}
which is included by |\include{|\textit{child}|}|
from within the main file
(or at least for those files to be compiled individually).
The argument \textit{main} must be the filename of the main file.

There are a couple of
considerations in setting up the main and child documents:

%%%%%%%%%%%%%%%%%%%%%%%%%%%%%%%%%%%%%%%%
\paragraph{Restrictions.}

Please note the following restrictions:
\begin{itemize}
\item
|\childdocmain| must be called with one argument \textit{main}
to ensure compatibility with earlier version of the package.
It must either be empty (|\childdocmain{}|)
or precisely match the filename of the main file in which it is specified.
See \secref{sec:detection} for further information.
\item
The filename \textit{main} must be specified without the |.tex| extension.
\item
The filename \textit{main} is case sensitive
(even in case-insensitive file systems)
due to internal string comparison.
\item
The argument \textit{main} should be fully expanded, it cannot be a macro.
\item
Subdirectories and special characters should be avoided in filenames.
\item
The command |\childdocmain{|\textit{main}|}| must be followed by a whitespace.
It should not be followed immediately by another command
or by a comment mark `|%|'.
This is because the \TeX{} parser reads the token immediately following
the argument of |\childdocmain| and puts it
at the beginning of every child section;
however, a white\-space is ignored.
\end{itemize}

%%%%%%%%%%%%%%%%%%%%%%%%%%%%%%%%%%%%%%%%
\paragraph{Content of Main File.}

It is advisable to place all content in the child files included by |\include|.
Any output contained in the main file will appear in all child documents
unless suppressed manually;
it cannot be suppressed automatically by the |\includeonly| directive
and thus should normally be avoided.
A method to include some content in the main file
by means of conditional processing is described in \secref{sec:conditional}.

%%%%%%%%%%%%%%%%%%%%%%%%%%%%%%%%%%%%%%%%
\paragraph{Page Numbering.}

When only a part of the document is compiled,
the appropriate numbering of pages
(as well as other status parameters)
is determined from the |.aux| files.
The latter contain information from previous passes.
However this information needs to propagate through
all intermediate child documents.
Therefore the page numbering in child documents may well
be inconsistent until the complete document is compiled at least once.

A useful (if unconventional) way to always ensure a consistent
page numbering is to restart the numbering in each child document
and denote the pages by `\textit{child}|.|\textit{page}'
where \textit{child} represents the chapter/section number of the child file.
This can be achieved by the command
|\numberwithin{page}{|\textit{child}|}|
of the \textsf{amsmath} package
where \textit{child} can be |chapter| or |section|
depending on the chosen structuring.
Alternatively, one can modify the macro |\thepage| appropriately
and reset the counter |page| at the start of each child file.

%%%%%%%%%%%%%%%%%%%%%%%%%%%%%%%%%%%%%%%%%%%%%%%%%%%%%%%%%%%%%%%%%%%%%%%%%%%%%%%%
\subsection{Conditional Processing}
\label{sec:conditional}

The package provides a mechanism to compile different versions
of a document. To customise the versions further some conditional processing
can come in handy to distinguish which version is being compiled.
The package provides two macros to describe the compilation context:

%%%%%%%%%%%%%%%%%%%%%%%%%%%%%%%%%%%%%%%%
\DescribeMacro{\ifchilddoc}
The conditional |\ifchilddoc| distinguishes between the compilation of
child documents and the main document:
%
\begin{center}
|\ifchilddoc |\textit{child-code}| |[|\||else |\textit{main-code}]| \||fi|
\end{center}

%%%%%%%%%%%%%%%%%%%%%%%%%%%%%%%%%%%%%%%%
\DescribeMacro{\childdocname}
\DescribeMacro{\childdocjob}
The macro |\childdocname| contains the filename (without extension)
of the main or child file being processed.
Note that |\childdocjob| will always contain the name of the main file.

%%%%%%%%%%%%%%%%%%%%%%%%%%%%%%%%%%%%%%%%
\paragraph{Title Page.}

Conditional processing can be used to include a title or banner page
in the main document when proper precautions are taken.
Importantly, the code in the main file should ensure that the page counter
(as well as other status parameters which are stored in the |.aux| files)
takes the same value after the conditional processing.
Otherwise the page numbers may take divergent values
depending on which part is compiled.

For example, a title page could be declared by:
%
\begin{center}
\begin{tabular}{l}
|\ifchilddoc\||else|\\
|\addtocounter{page}{-1}|\\
\textit{code for title page}\\
|\newpage|\\
|\||fi|
\end{tabular}
\end{center}
%
A banner page for the child documents can be generated by:
%
\begin{center}
\begin{tabular}{l}
|\ifchilddoc|\\
|\addtocounter{page}{-1}|\\
\textit{code for banner page}\\
|\newpage|\\
|\||fi|
\end{tabular}
\end{center}
%
Here one could write a message such as:
\begin{center}
|This is the part \childdocname{} of \childdocjob{}.|
\end{center}

%%%%%%%%%%%%%%%%%%%%%%%%%%%%%%%%%%%%%%%%%%%%%%%%%%%%%%%%%%%%%%%%%%%%%%%%%%%%%%%%
\subsection{Flags}
\label{sec:flags}

The package makes it easy to generate different versions
of the main or child documents.
To this end compilation flags can be defined
and assigned different default values.
They will be particularly useful in conjunction
with the forwarding mechanism described in \secref{sec:forward}.

For example, it may be useful to have a flag |\version|
which can be set to |draft| or |final|.
The document source will contain some conditional code
depending on the value of |\version|.
Suppose further, the flag should default to |final| for the main file
and to |draft| for child files
which is a natural assignment for editing the document.
This is achieved by placing the following code
in the preamble of the main document
(below the |\childdocmain| directive):
%
\begin{center}
\begin{tabular}{l}
|\ifchilddoc|\\
|\providecommand{\version}{draft}|\\
|\||else|\\
|\providecommand{\version}{final}|\\
|\||fi|
\end{tabular}
\end{center}
%
The definition by |\providecommand| makes sure
that previous definitions are not overwritten.
Further statements |\providecommand{\version}{...}|
can thus be added before the above code to override it.

For the main file, one might add a line
(between |\childdocmain| and the above block)
%
\begin{center}
|%\ifchilddoc\||else\providecommand{\version}{draft}\||fi|
\end{center}
%
which can be uncommented to produce a draft version.
Likewise one can add a line to the very top of a child file
(above the |\childdocof{|\textit{main}|}| directive)
%
\begin{center}
|%\providecommand{\version}{final}|
\end{center}
%
which can be uncommented to produce the final version of this child document.

%%%%%%%%%%%%%%%%%%%%%%%%%%%%%%%%%%%%%%%%%%%%%%%%%%%%%%%%%%%%%%%%%%%%%%%%%%%%%%%%
\subsection{Forwarding}
\label{sec:forward}

Different versions of the main or child documents
using compilation flags as described in \secref{sec:flags}
can be (permanently) stored in different files
for convenient compilation, viewing and distribution.
To this end, the package defines a command
to pass on compilation to a different file:

%%%%%%%%%%%%%%%%%%%%%%%%%%%%%%%%%%%%%%%%
\DescribeMacro{\childdocforward}
The command |\childdocforward| redirects processing to
another source file:
%
\begin{center}
\begin{tabular}{l}
|\input{childdoc.def}|\\
|\childdocforward[|\textit{main}|]{|\textit{dest}|}|\\
\end{tabular}
\end{center}
%
The argument \textit{dest} is the destination file
(without extension).
It should be the main file or one of the child files.
Note that further \textsf{childdoc} directives
such as |\childdocof| and |\childdocforward|
in the indicated file will be processed in this form.
The optional argument \textit{main}
passes on directly to the main file \textit{main}
while pretending to compile the child \textit{dest}.
This form behaves as if \textit{dest}
issues |\childdocof{|\textit{main}|}| right away,
and no further \textsf{childdoc} directives will be processed.

%%%%%%%%%%%%%%%%%%%%%%%%%%%%%%%%%%%%%%%%
\DescribeMacro{\...prefix}
In the alternative form |\childdocforwardprefix|,
%
\begin{center}
\begin{tabular}{l}
|\input{childdoc.def}|\\
|\childdocforwardprefix[|\textit{main}|]{|\textit{prefix}|}{|\textit{dest}|}|
\end{tabular}
\end{center}
%
the destination file is determined by a pattern
depending on the current file:
To make this work, the current file must be called
`{\textit{prefix}\hspace{0.2em}\textit{suffix}}'
with \textit{prefix} matching precisely the argument.
Processing is then passed on to the file
`{\textit{dest}\hspace{0.2em}\textit{suffix}}'.
Surely, the same effect is achieved by
directly specifying the
argument `{\textit{dest}\hspace{0.2em}\textit{suffix}}'
in the first form.
However, that requires to set up a different file
for each child. With the alternative form of the command
all these files can have exactly the same content
which simplifies setting them up and maintaining them.

For example, the following file |draft.tex|
with a compilation flag |\version| as described in \secref{sec:flags}
compiles the main document as a draft:
%
\begin{center}
\begin{tabular}{l}
|\def\version{draft}|\\
|\input{childdoc.def}|\\
|\childdocforward{|\textit{main}|}|
\end{tabular}
\end{center}
%
Likewise, the following files |final|\textit{nn}|.tex|
compile the final version of the child document
|child|\textit{nn}|.tex|:
%
\begin{center}
\begin{tabular}{l}
|\def\version{final}|\\
|\input{childdoc.def}|\\
|\childdocforwardprefix{final}{child}|
\end{tabular}
\end{center}
%

Note that when several versions of a main file and/or of each child file
are to be generated, it may be convenient to set up a |Makefile| or
shell script to automatise the process.

%%%%%%%%%%%%%%%%%%%%%%%%%%%%%%%%%%%%%%%%%%%%%%%%%%%%%%%%%%%%%%%%%%%%%%%%%%%%%%%%
\subsection{Command Line Processing}
\label{sec:commandline}

The effect of redirection files can also be achieved by invoking
the \LaTeX{} compiler with a more elaborate command line.
Most conveniently this should be done as part
of a shell script or a |Makefile|.

When using \textsf{childdoc} in the main file, the following
command lines effectively perform a redirection
(note that depending on the shell being used,
backslashes may have to be doubled: `|\|' $\to$ `|\\|'):
%
\begin{center}
|... -jobname "|\textit{target}|" |\\|"|[\textit{flags}]%
|\input{childdoc.def}\childdocforward[|\textit{main}|]{|\textit{dest}|}"|
\end{center}
%
Here \textit{target} is the name of the output file,
\textit{main} is the name of the main file
and \textit{dest} is the name of the main or child file to be processed
(all filenames without extensions).
The optional argument \textit{main} can be omitted
if \textit{main} matches \textit{dest}.
Optionally, compilation \textit{flags} can be defined via |\def| commands.
This command line makes the \TeX{} engine believe
it is compiling the file \textit{target}
whose content is specified as the latter parameter.
The provided code then forwards the processing to
\textit{main} or \textit{dest} as described in \secref{sec:forward}.

%%%%%%%%%%%%%%%%%%%%%%%%%%%%%%%%%%%%%%%%%%%%%%%%%%%%%%%%%%%%%%%%%%%%%%%%%%%%%%%%
\subsection{Include by Input}
\label{sec:input}

Including child documents by |\include| has some restrictions by design.
Most notably, the content of a child document always occupies
its own set of pages; pages cannot be shared between child documents.
Usually, this behaviour makes perfect sense
because each child document contain an essential part of the document.
However, in some situations it may be desirable to compose
a document from a collection of parts
without having mandatory page breaks between then.
For this case, the package
provides a mechanism to include parts
by |\input| which can also be processed individually.
However, by construction this mechanism
requires manual handling of the content to be output.

%%%%%%%%%%%%%%%%%%%%%%%%%%%%%%%%%%%%%%%%
\DescribeMacro{\ifchilddocmanual}
The main file should be prepared as usual, see \secref{sec:include}.
However, the document body must make a distinction
between processing of an individual part and of the main document, e.g.:
%
\begin{center}
\begin{tabular}{l}
|\ifchilddocmanual|\\
|\input{\childdocname}|\\
|\||else|\\
\textit{document body with }|\input{|\textit{part}|}|\\
|\||fi|
\end{tabular}
\end{center}
%
The conditional |\ifchilddocmanual| is true whenever
a part to be included by |\input| is being compiled,
and the name of the part is stored in |\childdocname|.

%%%%%%%%%%%%%%%%%%%%%%%%%%%%%%%%%%%%%%%%
\DescribeMacro{\childdocby}
Each part to be included by |\input| should start with:
%
\begin{center}
\begin{tabular}{l}
|\input{childdoc.def}|\\
|\childdocby{|\textit{main}|}|\\
\end{tabular}
\end{center}
%
The directive |\childdocby| is similar to |\childdocof|
described in \secref{sec:include},
but the subsequent selection of content must be done manually.
To that end, both |\ifchilddoc| and |\ifchilddocmanual|
will be true upon processing of a part,
and the name of the part is stored in |\childdocname|.
Note that |\jobname| will be set to the filename of the current part
so that each part receives an individual |.aux| file
that does not interfere with the |.aux| file(s) of the main document.
This behaviour can be altered by the alternative form
|\childdocby[*]{|\textit{main}|}| (with a non-empty optional argument)
which uses the |.aux| file of the main document
by setting |\jobname| to \textit{main}.

%%%%%%%%%%%%%%%%%%%%%%%%%%%%%%%%%%%%%%%%%%%%%%%%%%%%%%%%%%%%%%%%%%%%%%%%%%%%%%%%
\subsection{Driver Development}
\label{sec:driver}

The \textsf{childdoc} mechanism can also be use for the development
of definition files such as \LaTeX{} styles or classes.
This case differs from the above setup with multiple parts
included by |\include| in that no |\includeonly| should be invoked.
This can be achieved by starting the include file
(before |\ProvidesPackage|) with:
%
\begin{center}
\begin{tabular}{l}
|\input{childdoc.def}|\\
|\childdocforward{|\textit{main}|}|\\
\end{tabular}
\end{center}
%
or alternatively with:
%
\begin{center}
\begin{tabular}{l}
|\input{childdoc.def}|\\
|\childdocby{|\textit{main}|}|\\
\end{tabular}
\end{center}
%
Both forms have slightly different effects as described above.
The main file is prepared as usual, see \secref{sec:include}.

%%%%%%%%%%%%%%%%%%%%%%%%%%%%%%%%%%%%%%%%%%%%%%%%%%%%%%%%%%%%%%%%%%%%%%%%%%%%%%%%
\subsection{Legacy Detection}
\label{sec:detection}

The directive |\childdocmain| in the main file can detect
whether the complete document or merely a child is to be compiled
even without using the directive |\childdocof|.
This method is deprecated because it is less robust
and there is no compelling reason to use it;
it is merely provided for backward compatibility
and it may be removed in future versions.

If the detection mechanism is to be used,
it is mandatory to correctly specify
the filename of the main file as the argument of |\childdocmain|:
%
\begin{center}
\begin{tabular}{l}
|\input{childdoc.def}|\\
|\childdocmain{|\textit{main}|}|\\
\end{tabular}
\end{center}
%
If |\jobname| does not match the argument \textit{main} of |\childdocmain|,
it is assumed that |\jobname| points to the child file to be compiled.
When using |\childdocmain| with the main file specified as argument,
it suffices to start a child file
with just |\input{|\textit{main}|}|
without loading of the package and using |\childdocof|.
If instead all processing is done
with the appropriate \textsf{childdoc} directives,
the argument of \textit{main} of |\childdocmain| can be empty.

An alternative version of the command line processing described
in \secref{sec:commandline} using the detection mechanism reads:
%
\begin{center}
|... -jobname "|\textit{target}|" "|[\textit{flags}]%
[|\def\jobname{|\textit{dest}|}|]|\input{|\textit{main}|}"|
\end{center}

%%%%%%%%%%%%%%%%%%%%%%%%%%%%%%%%%%%%%%%%%%%%%%%%%%%%%%%%%%%%%%%%%%%%%%%%%%%%%%%%
\subsection{Manual Code}
\label{sec:manual}

In case one cannot be certain whether the definitions file |childdoc.def|
is installed on the target \TeX{} distribution
and one prefers not to ship it,
it is conceivable to paste a few relevant commands into the sources.

To that end, drop all statements |\input{childdoc.def}|
and perform the replacements as outlined below.
Instead of |\childdocmain{|\textit{main}|}| add the following code
to the top of the main file:
%
\begin{center}
\begin{tabular}{l}
|\||ifdefined\childdocname\endinput\||fi\newif\ifchilddoc|\\
|\edef\childdocname{\scantokens\expandafter{\jobname\noexpand}}|\\
|\def\childdocmain{|\textit{main}|}\||ifx\childdocmain\childdocname\||else|\\
|\childdoctrue\includeonly{\childdocname}\let\jobname\childdocmain\||fi|\\
\end{tabular}
\end{center}
%
Instead of |\childdocof{|\textit{main}|}| just include the main file
at the top of each child file:
%
\begin{center}
|\input{|\textit{main}|}|
\end{center}
%
A simple redirection |\childdocforward{|\textit{dest}|}| is achieved by:
%
\begin{center}
|\def\jobname{|\textit{dest}|}\input{\jobname}|
\end{center}
%
The redirection with prefix
|\childdocforwardprefix[|\textit{prefix}|]{|\textit{dest}|}|
is accomplished by:
%
\begin{center}
\begin{tabular}{l}
|{\edef\jobname{\scantokens\expandafter{\jobname\noexpand}}|\\
|\def\redirectjob |\textit{prefix}|#1~~~{\gdef\jobname{|\textit{dest}|#1}}|\\
|\expandafter\redirectjob\jobname~~~}\input{\jobname}|
\end{tabular}
\end{center}

In an alternative approach,
child documents can be compiled by a specific command line
without additional code or specific definitions:
%
\begin{center}
|... -jobname "|\textit{target}|" "|[\textit{flags}]%
|\includeonly{|\textit{dest}|}\input{|\textit{main}|}"|
\end{center}
%

%%%%%%%%%%%%%%%%%%%%%%%%%%%%%%%%%%%%%%%%%%%%%%%%%%%%%%%%%%%%%%%%%%%%%%%%%%%%%%%%
%%%%%%%%%%%%%%%%%%%%%%%%%%%%%%%%%%%%%%%%%%%%%%%%%%%%%%%%%%%%%%%%%%%%%%%%%%%%%%%%
\section{Information}

%%%%%%%%%%%%%%%%%%%%%%%%%%%%%%%%%%%%%%%%%%%%%%%%%%%%%%%%%%%%%%%%%%%%%%%%%%%%%%%%
\subsection{Copyright}

Copyright \copyright{} 2017--2018 Niklas Beisert

This work may be distributed and/or modified under the
conditions of the \LaTeX{} Project Public License, either version 1.3
of this license or (at your option) any later version.
The latest version of this license is in
  \url{http://www.latex-project.org/lppl.txt}
and version 1.3 or later is part of all distributions of \LaTeX{}
version 2005/12/01 or later.

This work has the LPPL maintenance status `maintained'.

The Current Maintainer of this work is Niklas Beisert.

This work consists of the files |README.txt|, |childdoc.ins| and |childdoc.dtx|
as well as the derived files |childdoc.def|, |cdocsamp.tex|
with |cdocsch1.tex|, |cdocsch2.tex|, |cdocspt3.tex|, |cdocspt4.tex|,
|cdocsdrf.tex|, |cdocsfn1.tex|, |cdocsfn2.tex|
as well as |childdoc.pdf|.

%%%%%%%%%%%%%%%%%%%%%%%%%%%%%%%%%%%%%%%%%%%%%%%%%%%%%%%%%%%%%%%%%%%%%%%%%%%%%%%%
\subsection{Files and Installation}

The package consists of the files:
%
\begin{center}
\begin{tabular}{ll}
    |README.txt|   & readme file \\
    |childdoc.ins| & installation file \\
    |childdoc.dtx| & source file \\
    |childdoc.def| & definition file \\
    |cdocsamp.tex| & sample main file \\
    |cdocsch1.tex| & sample include file \\
    |cdocsch2.tex| & sample include file \\
    |cdocspt3.tex| & sample part file \\
    |cdocspt4.tex| & sample part file \\
    |cdocsdrf.tex| & sample redirection file \\
    |cdocsfn1.tex| & sample redirection file \\
    |cdocsfn2.tex| & sample redirection file \\
    |childdoc.pdf| & manual
\end{tabular}
\end{center}
%
The distribution consists of the files
|README.txt|, |childdoc.ins| and |childdoc.dtx|.
%
\begin{itemize}
\item
Run (pdf)\LaTeX{} on |childdoc.dtx|
to compile the manual |childdoc.pdf| (this file).
\item
Run \LaTeX{} on |childdoc.ins| to create the definitions file |childdoc.def|
and the sample |cdocsamp.tex| with include files
|cdocsch1.tex|, |cdocsch2.tex|, |cdocspt3.tex|, |cdocspt4.tex|,
|cdocsdrf.tex|, |cdocsfn1.tex|, |cdocsfn2.tex|.
Then copy the file |childdoc.def| to an appropriate directory of your \LaTeX{}
distribution, e.g.\ \textit{texmf-root}|/tex/latex/childdoc|.
\end{itemize}

%%%%%%%%%%%%%%%%%%%%%%%%%%%%%%%%%%%%%%%%%%%%%%%%%%%%%%%%%%%%%%%%%%%%%%%%%%%%%%%%
\subsection{Related CTAN Packages}

There are several other packages which offer a similar functionality:
%
\begin{itemize}
\item
The packages
\href{http://ctan.org/pkg/docmute}{\textsf{docmute}},
\href{http://ctan.org/pkg/includex}{\textsf{includex}} and
\href{http://ctan.org/pkg/standalone}{\textsf{standalone}}
provide commands to include only the document body of
a child file thus allowing both files to be compiled individually.
\item
The packages \href{http://ctan.org/pkg/subdocs}{\textsf{subdocs}}
and \href{http://ctan.org/pkg/subfiles}{\textsf{subfiles}}
provide structures in which the main and child documents can be
encapsulated and allowing them to be compiled individually.
The inclusion mechanism is different from the conventional |\include|.
\item
The package \href{http://ctan.org/pkg/combine}{\textsf{combine}}
is an elaborate solution to combine several documents into one.
\end{itemize}
%
See also the CTAN topic \href{http://ctan.org/topic/subdocs}{\textsf{subdocs}}
for further related packages.
The present package differs from the above solutions in that
a document structure constructed with the conventional |\include| mechanism
just needs two extra commands at the top of every file
such that all constituent files can be compiled individually.

%%%%%%%%%%%%%%%%%%%%%%%%%%%%%%%%%%%%%%%%%%%%%%%%%%%%%%%%%%%%%%%%%%%%%%%%%%%%%%%%
%\subsection{Feature Suggestions}
%
%The following is a list of features which may be useful for future
%versions of this package:
%%
%\begin{itemize}
%\item
%\ldots
%\end{itemize}

%%%%%%%%%%%%%%%%%%%%%%%%%%%%%%%%%%%%%%%%%%%%%%%%%%%%%%%%%%%%%%%%%%%%%%%%%%%%%%%%
\subsection{Revision History}

%%%%%%%%%%%%%%%%%%%%%%%%%%%%%%%%%%%%%%%%
\paragraph{v2.0:} 2018/12/30

\begin{itemize}
\item
immediate forward processing
\item
added |\childdocby| mechanism
\item
manual restructured
\end{itemize}

%%%%%%%%%%%%%%%%%%%%%%%%%%%%%%%%%%%%%%%%
\paragraph{v1.6:} 2018/01/17

\begin{itemize}
\item
application for development of include files
\item
corrections to manual
\end{itemize}

%%%%%%%%%%%%%%%%%%%%%%%%%%%%%%%%%%%%%%%%
\paragraph{v1.5:} 2017/05/21

\begin{itemize}
\item
more complete structuring introduced
\item
|\childdocof| introduced
\item
|\childdoc| renamed to |\childdocmain|
\item
|\childredirect| renamed to |\childdocforward| and |\childdocforwardprefix|
and functionality expanded
\end{itemize}

%%%%%%%%%%%%%%%%%%%%%%%%%%%%%%%%%%%%%%%%
\paragraph{v1.0:} 2017/04/27

\begin{itemize}
\item
manual and install package
\item
first version published on CTAN
\end{itemize}

%%%%%%%%%%%%%%%%%%%%%%%%%%%%%%%%%%%%%%%%
\paragraph{v0.6:} 2017/04/26

\begin{itemize}
\item
redirection mechanism added
\end{itemize}

%%%%%%%%%%%%%%%%%%%%%%%%%%%%%%%%%%%%%%%%
\paragraph{v0.5:} 2017/04/26

\begin{itemize}
\item
functionality in definition file
\end{itemize}


%%%%%%%%%%%%%%%%%%%%%%%%%%%%%%%%%%%%%%%%%%%%%%%%%%%%%%%%%%%%%%%%%%%%%%%%%%%%%%%%
%%%%%%%%%%%%%%%%%%%%%%%%%%%%%%%%%%%%%%%%%%%%%%%%%%%%%%%%%%%%%%%%%%%%%%%%%%%%%%%%
%%%%%%%%%%%%%%%%%%%%%%%%%%%%%%%%%%%%%%%%%%%%%%%%%%%%%%%%%%%%%%%%%%%%%%%%%%%%%%%%
\appendix

\settowidth\MacroIndent{\rmfamily\scriptsize 000\ }

 \DocInput{childdoc.dtx}

\end{document}
%</driver>
% \fi
%
% %%%%%%%%%%%%%%%%%%%%%%%%%%%%%%%%%%%%%%%%%%%%%%%%%%%%%%%%%%%%%%%%%%%%%%%%%%%%%%
% %%%%%%%%%%%%%%%%%%%%%%%%%%%%%%%%%%%%%%%%%%%%%%%%%%%%%%%%%%%%%%%%%%%%%%%%%%%%%%
% \section{Sample}
%\iffalse
%<*samplemain>
%\fi
%
% The following presents a sample document
% with two chapters, two parts, a title page,
% a compile flag as well as three forwarding files to set the flag.
% It consists of eight |.tex| files:
% \begin{center}
% \begin{tabular}{ll}
% |cdocsamp.tex|&main file\\
% |cdocsch1.tex|&include file for chapter 1\\
% |cdocsch2.tex|&include file for chapter 2\\
% |cdocspt3.tex|&include file for part 3\\
% |cdocspt4.tex|&include file for part 4\\
% |cdocsdrf.tex|&forwarding file for main file in draft mode\\
% |cdocsfi1.tex|&forwarding file for final version of chapter 1\\
% |cdocsfi2.tex|&forwarding file for final version of chapter 2\\
% \end{tabular}
% \end{center}
% Each of the eight files can be compiled directly by the \LaTeX{} compiler.
%
% %%%%%%%%%%%%%%%%%%%%%%%%%%%%%%%%%%%%%%
% \paragraph{Main File.}
%
% The main file is called |cdocsamp.tex|.
%
% Load the \textsf{childdoc} definitions and
% declare the filename for the main document:
%    \begin{macrocode}
\input{childdoc.def}
\childdocmain{}
%    \end{macrocode}

% Optional override for |\version| flag:
%    \begin{macrocode}
%%\ifchilddoc\else\providecommand{\version}{draft}\fi
%    \end{macrocode}

% Define the default values for the |\version| flag
% (|final| for the main file and |draft| for childs):
%    \begin{macrocode}
\ifchilddoc
\providecommand{\version}{draft}
\else
\providecommand{\version}{final}
\fi
%    \end{macrocode}

% Load the standard document class:
%    \begin{macrocode}
\documentclass[12pt]{article}
%    \end{macrocode}

% Start the document body:
%    \begin{macrocode}
\begin{document}
%    \end{macrocode}

% Declare a title page.
% Print title, part of document being processed and version flag:
%    \begin{macrocode}
\addtocounter{page}{-1}
\begin{center}
{\LARGE\bfseries{}childdoc example\par}
\vspace{1cm}
\ifchilddoc
\ifchilddocmanual part\else chapter\fi:
`\childdocname' of `\childdocjob'\par
\else
main document: `\childdocjob'\par
\fi
version: \version\par
\end{center}
\newpage
%    \end{macrocode}

% Manually include selected file,
% otherwise process as usual:
%    \begin{macrocode}
\ifchilddocmanual
\section*{part `\childdocname'}
\input{\childdocname}
\else
%    \end{macrocode}

% Include the two chapters:
%    \begin{macrocode}
\include{cdocsch1}
\include{cdocsch2}
%    \end{macrocode}

% Include the two parts unless only chapters should be displayed:
%    \begin{macrocode}
\ifchilddoc\else
\section{part three}
\input{cdocspt3}
\section{part four}
\input{cdocspt4}
\fi
%    \end{macrocode}

% Process as usual until here:
%    \begin{macrocode}
\fi
%    \end{macrocode}

% End of document body:
%    \begin{macrocode}
\end{document}
%    \end{macrocode}
%\iffalse
%</samplemain>
%\fi
%
% %%%%%%%%%%%%%%%%%%%%%%%%%%%%%%%%%%%%%%
% \paragraph{Chapter Include Files.}
%
% The include files are called |cdocsch1.tex| and |cdocsch2.tex|.
%
%\iffalse
%<*samplechap1|samplechap2>
%\fi

% Optional override for |\version| flag:
%    \begin{macrocode}
%%\providecommand{\version}{final}
%    \end{macrocode}

% Include the main document:
%    \begin{macrocode}
\input{childdoc.def}
\childdocof{cdocsamp}
%    \end{macrocode}

%\iffalse
%</samplechap1|samplechap2>
%\fi
%
%\iffalse
%<*samplechap1>
%\fi
% Some text for chapter 1:
%    \begin{macrocode}
\section{one}
some text in chapter one
%    \end{macrocode}

%\iffalse
%</samplechap1>
%\fi
% Some text for chapter 2:
%\iffalse
%<*samplechap2>
%\fi
%    \begin{macrocode}
\section{two}
more text in chapter two
%    \end{macrocode}

%\iffalse
%</samplechap2>
%\fi
%
% %%%%%%%%%%%%%%%%%%%%%%%%%%%%%%%%%%%%%%
% \paragraph{Part Include Files.}
%
% The include files are called |cdocspt3.tex| and |cdocspt4.tex|.
%
%\iffalse
%<*samplepart3|samplepart4>
%\fi

% Optional override for |\version| flag:
%    \begin{macrocode}
%%\providecommand{\version}{final}
%    \end{macrocode}

% Include the main document:
%    \begin{macrocode}
\input{childdoc.def}
\childdocby{cdocsamp}
%    \end{macrocode}

%\iffalse
%</samplepart3|samplepart4>
%\fi
%
%\iffalse
%<*samplepart3>
%\fi
% Some text for part 3:
%    \begin{macrocode}
some text in part three
%    \end{macrocode}

%\iffalse
%</samplepart3>
%\fi
% Some text for part 4:
%\iffalse
%<*samplepart4>
%\fi
%    \begin{macrocode}
more text in part four
%    \end{macrocode}

%\iffalse
%</samplepart4>
%\fi
%
% %%%%%%%%%%%%%%%%%%%%%%%%%%%%%%%%%%%%%%
% \paragraph{Forwarding for a Complete Draft.}
%
% The following forwarding file |cdocsdrf.tex|
% compiles the main document in draft mode:
%\iffalse
%<*sampledraft>
%\fi
%    \begin{macrocode}
\def\version{draft}
\input{childdoc.def}
\childdocforward{cdocsamp}
%    \end{macrocode}

%\iffalse
%</sampledraft>
%\fi
%
% %%%%%%%%%%%%%%%%%%%%%%%%%%%%%%%%%%%%%%
% \paragraph{Forwarding for Final Version of the Chapters.}
%
% The following forwarding files |cdocsfn1.tex| and |cdocsfn2.tex|
% (with identical content)
% compile the final versions of the child documents
% |cdocsch1.tex| and |cdocsch2.tex|, respectively:
%\iffalse
%<*samplefinal>
%\fi
%    \begin{macrocode}
\def\version{final}
\input{childdoc.def}
\childdocforwardprefix[cdocsamp]{cdocsfn}{cdocsch}
%    \end{macrocode}

%\iffalse
%</samplefinal>
%\fi
%
% %%%%%%%%%%%%%%%%%%%%%%%%%%%%%%%%%%%%%%
% \paragraph{Command Line Processing.}
%
% The following three command lines generate the output files
% |cdocscld|, |cdocscl1| and |cdocscl2|
% which should be identical to
% |cdocsdrf|, |cdocsch1| and |cdocsfn2|, respectively:
% \begin{center}
% \begin{tabular}{l}
% |latex -jobname cdocscld \|\\
% |  "\def\version{draft}\input{childdoc.def}\childdocforward{cdocsamp}"|\\
% |latex -jobname cdocscl1 \|\\
% |  "\input{childdoc.def}\childdocforward[cdocsamp]{cdocsch1}"|\\
% |latex -jobname cdocscl2 \|\\
% |  "\def\version{final}\input{childdoc.def}\childdocforward{cdocsch2}"|
% \end{tabular}
% \end{center}
% Note that the trailing backslash on each first line
% merely continues the input to the second line
% (for convenient cut ant paste).
% Furthermore, the command |latex| can be replaced by any
% of its alternative versions such as |pdflatex|.
%
% %%%%%%%%%%%%%%%%%%%%%%%%%%%%%%%%%%%%%%%%%%%%%%%%%%%%%%%%%%%%%%%%%%%%%%%%%%%%%%
% %%%%%%%%%%%%%%%%%%%%%%%%%%%%%%%%%%%%%%%%%%%%%%%%%%%%%%%%%%%%%%%%%%%%%%%%%%%%%%
% \section{Implementation}
%\iffalse
%<*package>
%\fi
%
% This section describes the definitions file |childdoc.def|.

% The definitions cannot be loaded using |\usepackage| or |\RequirePackage|
% which has a mechanism to prevent loading a style file more than once.
% When loading the definitions by means of |\input|
% multiple instances have to be prevented manually:
%\iffalse
%This code needs to be before the `\ProvidesFile' directive
%which is defined at the beginning of this file.
%Therefore it is also placed there and commented out here.
%</package>
%<*discard>
%\fi
%    \begin{macrocode}
\ifdefined\childdocmain\endinput\fi
%    \end{macrocode}
%\iffalse
%</discard>
%<*package>
%\fi
%
% \macro{\ifchilddoc}
% \macro{\ifchilddocmanual}
% The conditional |\ifchilddoc| tells whether a
% child (true) or main (false) document is being compiled.
% The conditional |\ifchilddocmanual| tells whether
% the |\includeonly| mechanism is used (false) or
% the selection of child files must be performed manually (true).
% The definitions initialise to false:
%    \begin{macrocode}
\newif\ifchilddoc
\newif\ifchilddocmanual
%    \end{macrocode}

% \macro{\childdocname}
% \macro{\childdocjob}
% The macro |\childdocname| stores the name of the main document
% to be compiled. The macro |\childdocjob| stores the name of
% the document on which the \LaTeX{} compiler was originally invoked.
% The content of |\jobname| cannot be compared
% to filenames specified in the source due to different catcodes.
% The following code rescans |\jobname|, stores the result
% in |\childdocname| and saves a copy in |\childdocjob|:
%    \begin{macrocode}
\edef\childdocname{\scantokens\expandafter{\jobname\noexpand}}
\let\childdocjob\childdocname
%    \end{macrocode}

% \macro{\childdocdisable}
% The macro |\childdocdisable| prevents the main file
% from being processed more than once.
% At this stage, the main document command |\childdocmain|
% is assumed to be called once again where it should do nothing.
% Any subsequent call to it should prevent
% a secondary processing of the main document
% It overwrites the forwarding commands
% |\childdocof| and |\childdocforward|
% with empty macros to prevent further inclusions of the main document:
%    \begin{macrocode}
\newcommand{\childdocdisable}
{
  \renewcommand{\childdocmain}[1]{\renewcommand{\childdocmain}[1]{\endinput}}
  \renewcommand{\childdocof}[1]{}
  \renewcommand{\childdocby}[2][]{}
  \renewcommand{\childdocforward}[2][]{}
  \renewcommand{\childdocdisable}{}
}
%    \end{macrocode}

% \macro{\childdocmain}
% The macro |\childdocmain| is to be called at the top of the main file
% with nothing or the main filename (without extension) as argument.
% First, it breaks loops.
% If the argument is not empty and does not match |\childdocname|
% (which is set by the first inclusion of |childdoc.def|),
% |\ifchilddoc| is set to true, |\includeonly| is applied to the child file
% and |\jobname| is set to the main file
% (for proper handling of |.aux| files):
%    \begin{macrocode}
\newcommand{\childdocmain}[1]
{
  \childdocdisable\childdocmain{}
  \if?#1?\else
    \begingroup
      \def\childdoctmp{#1}
      \ifx\childdoctmp\childdocname
        \def\childdoctmp{}
      \else
        \def\childdoctmp
        {
          \childdoctrue
          \includeonly{\childdocname}
          \def\childdocjob{#1}
          \def\jobname{#1}
        }
      \fi
      \expandafter
    \endgroup
    \childdoctmp
  \fi
}
%    \end{macrocode}

% \macro{\childdocof}
% The command |\childdocof| redirects
% compilation to the main file |#1|.
%    \begin{macrocode}
\newcommand{\childdocof}[1]
{
  \childdocdisable
  \childdoctrue
  \includeonly{\childdocname}
  \def\jobname{#1}
  \def\childdocjob{#1}
  \input{#1}
}
%    \end{macrocode}

% \macro{\childdocby}
% The command |\childdocby| ....
%    \begin{macrocode}
\newcommand{\childdocby}[2][]
{
  \childdocdisable
  \childdoctrue
  \childdocmanualtrue
  \if?#1?\else
    \def\jobname{#2}
  \fi
  \def\childdocjob{#2}
  \input{#2}
  \endinput
}
%    \end{macrocode}

% \macro{\childdocforward}
% The command |\childdocforward| redirects
% compilation to the main file or
% (if the optional argument is given) a child file.
% Parameters are set as if the main file
% or a child file starting with |\childdocof| was compiled.
% Then compilation is handed over to the main file:
%    \begin{macrocode}
\newcommand{\childdocforward}[2][]
{
  \begingroup
    \if?#1?
      \def\childdoctmp
      {
        \def\childdocname{#2}
        \def\childdocjob{#2}
        \def\jobname{#2}
        \input{#2}
        \endinput
      }
    \else
      \def\childdoctmp
      {
        \childdocdisable
        \def\childdocname{#2}
        \childdoctrue
        \includeonly{#2}
        \def\childdocjob{#1}
        \def\jobname{#1}
        \input{#1}
        \endinput
      }
    \fi
    \expandafter
  \endgroup
  \childdoctmp
}
%    \end{macrocode}

% \macro{\childdocforwardprefix}
% The command |\childdocforwardprefix| redirects
% compilation to the main or a child file by means of a pattern.
% The prefix |#1| in the current filename is replaced by |#2|
% and the suffix of the current filename is kept
% (it is assumed that the filename does not contain the substring `|~~~|'
% which is used as a delimiter).
% Compilation is handed over to the new file by |\childdocforward|:
%    \begin{macrocode}
\newcommand{\childdocforwardprefix}[3][]
{
  \begingroup
    \def\childdocextract #2##1~~~{\def\childdoctmp{\childdocforward[#1]{#3##1}}}
    \expandafter\childdocextract\childdocname~~~
    \expandafter
  \endgroup
  \childdoctmp
}
%    \end{macrocode}

% \macro{\childdoc}
% The deprecated macro |\childdoc| is a legacy version of |\childdocmain|:
%    \begin{macrocode}
\newcommand{\childdoc}{\childdocmain}
%    \end{macrocode}

% \macro{\childdocredirect}
% The deprecated macro |\childdocredirect| is a legacy version
% of |\childdocforward| and |\childdocforwardprefix|:
%    \begin{macrocode}
\newcommand{\childdocredirect}[2][]
{
  \begingroup
    \if?#1?
      \def\childdoctmp{\childdocforward{#2}}
    \else
      \def\childdoctmp{\childdocforwardprefix{#1}{#2}}
    \fi
    \expandafter
  \endgroup
  \childdoctmp
}
%    \end{macrocode}

%\iffalse
%</package>
%\fi
%
\endinput

\childdocmain{}
%    \end{macrocode}

% Optional override for |\version| flag:
%    \begin{macrocode}
%%\ifchilddoc\else\providecommand{\version}{draft}\fi
%    \end{macrocode}

% Define the default values for the |\version| flag
% (|final| for the main file and |draft| for childs):
%    \begin{macrocode}
\ifchilddoc
\providecommand{\version}{draft}
\else
\providecommand{\version}{final}
\fi
%    \end{macrocode}

% Load the standard document class:
%    \begin{macrocode}
\documentclass[12pt]{article}
%    \end{macrocode}

% Start the document body:
%    \begin{macrocode}
\begin{document}
%    \end{macrocode}

% Declare a title page.
% Print title, part of document being processed and version flag:
%    \begin{macrocode}
\addtocounter{page}{-1}
\begin{center}
{\LARGE\bfseries{}childdoc example\par}
\vspace{1cm}
\ifchilddoc
\ifchilddocmanual part\else chapter\fi:
`\childdocname' of `\childdocjob'\par
\else
main document: `\childdocjob'\par
\fi
version: \version\par
\end{center}
\newpage
%    \end{macrocode}

% Manually include selected file,
% otherwise process as usual:
%    \begin{macrocode}
\ifchilddocmanual
\section*{part `\childdocname'}
\input{\childdocname}
\else
%    \end{macrocode}

% Include the two chapters:
%    \begin{macrocode}
\include{cdocsch1}
\include{cdocsch2}
%    \end{macrocode}

% Include the two parts unless only chapters should be displayed:
%    \begin{macrocode}
\ifchilddoc\else
\section{part three}
\input{cdocspt3}
\section{part four}
\input{cdocspt4}
\fi
%    \end{macrocode}

% Process as usual until here:
%    \begin{macrocode}
\fi
%    \end{macrocode}

% End of document body:
%    \begin{macrocode}
\end{document}
%    \end{macrocode}
%\iffalse
%</samplemain>
%\fi
%
% %%%%%%%%%%%%%%%%%%%%%%%%%%%%%%%%%%%%%%
% \paragraph{Chapter Include Files.}
%
% The include files are called |cdocsch1.tex| and |cdocsch2.tex|.
%
%\iffalse
%<*samplechap1|samplechap2>
%\fi

% Optional override for |\version| flag:
%    \begin{macrocode}
%%\providecommand{\version}{final}
%    \end{macrocode}

% Include the main document:
%    \begin{macrocode}
% \iffalse
%
% childdoc.dtx Copyright (C) 2017-2018 Niklas Beisert
%
% This work may be distributed and/or modified under the
% conditions of the LaTeX Project Public License, either version 1.3
% of this license or (at your option) any later version.
% The latest version of this license is in
%   http://www.latex-project.org/lppl.txt
% and version 1.3 or later is part of all distributions of LaTeX
% version 2005/12/01 or later.
%
% This work has the LPPL maintenance status `maintained'.
%
% The Current Maintainer of this work is Niklas Beisert.
%
% This work consists of the files childdoc.dtx and childdoc.ins
% and the derived files childdoc.def and cdocsamp.tex with
% cdocsch1.tex, cdocsch2.tex, cdocsdrf.tex, cdocsfn1.tex, cdocsfn2.tex.
%
%<package>\ifdefined\childdocmain\endinput\fi
%<package>\ProvidesFile{childdoc.def}[2018/12/30 v2.0 child document driver]
%<samplemain>\ProvidesFile{cdocsamp.tex}[2018/12/30 v2.0 sample for childdoc]
%<*driver>
%\ProvidesFile{childdoc.drv}[2018/12/30 v2.0 childdoc reference manual file]
\PassOptionsToClass{10pt,a4paper}{article}
\documentclass{ltxdoc}

\usepackage[margin=35mm]{geometry}
\usepackage{hyperref}
\usepackage{hyperxmp}
\usepackage[usenames]{color}

\hypersetup{colorlinks=true}
\hypersetup{pdfstartview=FitH}
\hypersetup{pdfpagemode=UseNone}
\hypersetup{pdfsource={}}
\hypersetup{pdflang={en-UK}}
\hypersetup{pdfcopyright={Copyright 2017-2018 Niklas Beisert.
  This work may be distributed and/or modified under the
  conditions of the LaTeX Project Public License, either version 1.3
  of this license or (at your option) any later version.}}
\hypersetup{pdflicenseurl={http://www.latex-project.org/lppl.txt}}
\hypersetup{pdfcontactaddress={ETH Zurich, ITP, HIT K,
  Wolfgang-Pauli-Strasse 27}}
\hypersetup{pdfcontactpostcode={8093}}
\hypersetup{pdfcontactcity={Zurich}}
\hypersetup{pdfcontactcountry={Switzerland}}
\hypersetup{pdfcontactemail={nbeisert@itp.phys.ethz.ch}}
\hypersetup{pdfcontacturl={http://people.phys.ethz.ch/\xmptilde nbeisert/}}

\newcommand{\secref}[1]{\hyperref[#1]{section \ref*{#1}}}

\parskip1ex
\parindent0pt
\let\olditemize\itemize
\def\itemize{\olditemize\parskip0pt}

\begin{document}

\title{The \textsf{childdoc} Package}
\hypersetup{pdftitle={The childdoc Package}}
\author{Niklas Beisert\\[2ex]
  Institut f\"ur Theoretische Physik\\
  Eidgen\"ossische Technische Hochschule Z\"urich\\
  Wolfgang-Pauli-Strasse 27, 8093 Z\"urich, Switzerland\\[1ex]
  \href{mailto:nbeisert@itp.phys.ethz.ch}
  {\texttt{nbeisert@itp.phys.ethz.ch}}}
\hypersetup{pdfauthor={Niklas Beisert}}
\hypersetup{pdfsubject={Manual for the LaTeX2e Package childdoc}}
\date{30 December 2018, \textsf{v2.0}}
\maketitle

\begin{abstract}\noindent
\textsf{childdoc} is a \LaTeXe{} package
that enables the direct compilation
of document sections included by |\include|
to individual files.
\end{abstract}

\begingroup
\parskip0ex
\tableofcontents
\endgroup

%%%%%%%%%%%%%%%%%%%%%%%%%%%%%%%%%%%%%%%%%%%%%%%%%%%%%%%%%%%%%%%%%%%%%%%%%%%%%%%%
%%%%%%%%%%%%%%%%%%%%%%%%%%%%%%%%%%%%%%%%%%%%%%%%%%%%%%%%%%%%%%%%%%%%%%%%%%%%%%%%
\section{Introduction}

\LaTeX{} provides a mechanism to structure a large document (such as a book)
into a main file and several child files (containing the chapters)
using the |\include| command.
This mechanism is beneficial for documents
which span hundreds of pages in order to
make the source file(s) more manageable.
Moreover, compilation can be restricted to
selected child files by means of the |\includeonly| command.
The latter feature can be used to reduce the compilation time while editing
(this was significantly more useful in the earlier days of \LaTeX{})
or to generate a smaller document which is easier to navigate.
Another application of |\includeonly| is to generate
documents consisting of selected parts of the complete document.

However, there are a few drawbacks of the plain |\include| mechanism:
\begin{itemize}
\item
The child files cannot be compiled on their own,
they can only be compiled via the main file.
A naive editing environment
(such as a text editor with an option
to have the current file processed by \LaTeX)
may require one to switch to the main file before compiling;
attempting to compile the child file produces errors.
\item
The main file must be modified (each time)
to adjust the |\includeonly| command
to the present needs. This easily leaves the main file in a messy state.
\item
The generated document will always carry the filename
of the main document. This is inconvenient if
several child files are to be compiled and
to be kept for distribution.
\end{itemize}

The present package provides a simple interface
to make child files individually compilable by \LaTeX{}.
Compiling a child file then has the same effect as compiling
the main file with an |\includeonly| command
to select the appropriate child.
Moreover the generated document will carry the name of the child
rather than the main file.
This resolves all three above issues.

This feature is meant to make the editing of books,
thesis documents and lecture notes somewhat more convenient.
However, the package can also be used efficiently for
composing a series of documents (such as exercise sheets)
which are typically distributed individually.
It then assists the author in generating the individual documents
(potentially in different versions)
as well as a document containing the collected series.
Another application is in developing style files
or other kinds of included material
where compilation of the style file could redirect
to a sample or test file.

%%%%%%%%%%%%%%%%%%%%%%%%%%%%%%%%%%%%%%%%%%%%%%%%%%%%%%%%%%%%%%%%%%%%%%%%%%%%%%%%
%%%%%%%%%%%%%%%%%%%%%%%%%%%%%%%%%%%%%%%%%%%%%%%%%%%%%%%%%%%%%%%%%%%%%%%%%%%%%%%%
\section{Usage}

First of all, the package \textsf{childdoc} is \emph{not} a standard
\LaTeXe{} |.sty| style file! Therefore it needs to be invoked in
a non-standard way.

%%%%%%%%%%%%%%%%%%%%%%%%%%%%%%%%%%%%%%%%%%%%%%%%%%%%%%%%%%%%%%%%%%%%%%%%%%%%%%%%
\subsection{Included Files}
\label{sec:include}

%%%%%%%%%%%%%%%%%%%%%%%%%%%%%%%%%%%%%%%%
\DescribeMacro{\childdocmain}
To use the package, add the commands
\begin{center}
\begin{tabular}{l}
|\input{childdoc.def}|\\
|\childdocmain{}|\\
\end{tabular}
\end{center}
at the very top of the main \LaTeX{} file,
in particular \emph{before} the |\documentclass| statement!
The argument of |\childdocmain| should be left empty
(but it must be present).

%%%%%%%%%%%%%%%%%%%%%%%%%%%%%%%%%%%%%%%%
\DescribeMacro{\childdocof}
Furthermore, add the commands
\begin{center}
\begin{tabular}{l}
|\input{childdoc.def}|\\
|\childdocof{|\textit{main}|}|\\
\end{tabular}
\end{center}
at the top of every child file \textit{child}
which is included by |\include{|\textit{child}|}|
from within the main file
(or at least for those files to be compiled individually).
The argument \textit{main} must be the filename of the main file.

There are a couple of
considerations in setting up the main and child documents:

%%%%%%%%%%%%%%%%%%%%%%%%%%%%%%%%%%%%%%%%
\paragraph{Restrictions.}

Please note the following restrictions:
\begin{itemize}
\item
|\childdocmain| must be called with one argument \textit{main}
to ensure compatibility with earlier version of the package.
It must either be empty (|\childdocmain{}|)
or precisely match the filename of the main file in which it is specified.
See \secref{sec:detection} for further information.
\item
The filename \textit{main} must be specified without the |.tex| extension.
\item
The filename \textit{main} is case sensitive
(even in case-insensitive file systems)
due to internal string comparison.
\item
The argument \textit{main} should be fully expanded, it cannot be a macro.
\item
Subdirectories and special characters should be avoided in filenames.
\item
The command |\childdocmain{|\textit{main}|}| must be followed by a whitespace.
It should not be followed immediately by another command
or by a comment mark `|%|'.
This is because the \TeX{} parser reads the token immediately following
the argument of |\childdocmain| and puts it
at the beginning of every child section;
however, a white\-space is ignored.
\end{itemize}

%%%%%%%%%%%%%%%%%%%%%%%%%%%%%%%%%%%%%%%%
\paragraph{Content of Main File.}

It is advisable to place all content in the child files included by |\include|.
Any output contained in the main file will appear in all child documents
unless suppressed manually;
it cannot be suppressed automatically by the |\includeonly| directive
and thus should normally be avoided.
A method to include some content in the main file
by means of conditional processing is described in \secref{sec:conditional}.

%%%%%%%%%%%%%%%%%%%%%%%%%%%%%%%%%%%%%%%%
\paragraph{Page Numbering.}

When only a part of the document is compiled,
the appropriate numbering of pages
(as well as other status parameters)
is determined from the |.aux| files.
The latter contain information from previous passes.
However this information needs to propagate through
all intermediate child documents.
Therefore the page numbering in child documents may well
be inconsistent until the complete document is compiled at least once.

A useful (if unconventional) way to always ensure a consistent
page numbering is to restart the numbering in each child document
and denote the pages by `\textit{child}|.|\textit{page}'
where \textit{child} represents the chapter/section number of the child file.
This can be achieved by the command
|\numberwithin{page}{|\textit{child}|}|
of the \textsf{amsmath} package
where \textit{child} can be |chapter| or |section|
depending on the chosen structuring.
Alternatively, one can modify the macro |\thepage| appropriately
and reset the counter |page| at the start of each child file.

%%%%%%%%%%%%%%%%%%%%%%%%%%%%%%%%%%%%%%%%%%%%%%%%%%%%%%%%%%%%%%%%%%%%%%%%%%%%%%%%
\subsection{Conditional Processing}
\label{sec:conditional}

The package provides a mechanism to compile different versions
of a document. To customise the versions further some conditional processing
can come in handy to distinguish which version is being compiled.
The package provides two macros to describe the compilation context:

%%%%%%%%%%%%%%%%%%%%%%%%%%%%%%%%%%%%%%%%
\DescribeMacro{\ifchilddoc}
The conditional |\ifchilddoc| distinguishes between the compilation of
child documents and the main document:
%
\begin{center}
|\ifchilddoc |\textit{child-code}| |[|\||else |\textit{main-code}]| \||fi|
\end{center}

%%%%%%%%%%%%%%%%%%%%%%%%%%%%%%%%%%%%%%%%
\DescribeMacro{\childdocname}
\DescribeMacro{\childdocjob}
The macro |\childdocname| contains the filename (without extension)
of the main or child file being processed.
Note that |\childdocjob| will always contain the name of the main file.

%%%%%%%%%%%%%%%%%%%%%%%%%%%%%%%%%%%%%%%%
\paragraph{Title Page.}

Conditional processing can be used to include a title or banner page
in the main document when proper precautions are taken.
Importantly, the code in the main file should ensure that the page counter
(as well as other status parameters which are stored in the |.aux| files)
takes the same value after the conditional processing.
Otherwise the page numbers may take divergent values
depending on which part is compiled.

For example, a title page could be declared by:
%
\begin{center}
\begin{tabular}{l}
|\ifchilddoc\||else|\\
|\addtocounter{page}{-1}|\\
\textit{code for title page}\\
|\newpage|\\
|\||fi|
\end{tabular}
\end{center}
%
A banner page for the child documents can be generated by:
%
\begin{center}
\begin{tabular}{l}
|\ifchilddoc|\\
|\addtocounter{page}{-1}|\\
\textit{code for banner page}\\
|\newpage|\\
|\||fi|
\end{tabular}
\end{center}
%
Here one could write a message such as:
\begin{center}
|This is the part \childdocname{} of \childdocjob{}.|
\end{center}

%%%%%%%%%%%%%%%%%%%%%%%%%%%%%%%%%%%%%%%%%%%%%%%%%%%%%%%%%%%%%%%%%%%%%%%%%%%%%%%%
\subsection{Flags}
\label{sec:flags}

The package makes it easy to generate different versions
of the main or child documents.
To this end compilation flags can be defined
and assigned different default values.
They will be particularly useful in conjunction
with the forwarding mechanism described in \secref{sec:forward}.

For example, it may be useful to have a flag |\version|
which can be set to |draft| or |final|.
The document source will contain some conditional code
depending on the value of |\version|.
Suppose further, the flag should default to |final| for the main file
and to |draft| for child files
which is a natural assignment for editing the document.
This is achieved by placing the following code
in the preamble of the main document
(below the |\childdocmain| directive):
%
\begin{center}
\begin{tabular}{l}
|\ifchilddoc|\\
|\providecommand{\version}{draft}|\\
|\||else|\\
|\providecommand{\version}{final}|\\
|\||fi|
\end{tabular}
\end{center}
%
The definition by |\providecommand| makes sure
that previous definitions are not overwritten.
Further statements |\providecommand{\version}{...}|
can thus be added before the above code to override it.

For the main file, one might add a line
(between |\childdocmain| and the above block)
%
\begin{center}
|%\ifchilddoc\||else\providecommand{\version}{draft}\||fi|
\end{center}
%
which can be uncommented to produce a draft version.
Likewise one can add a line to the very top of a child file
(above the |\childdocof{|\textit{main}|}| directive)
%
\begin{center}
|%\providecommand{\version}{final}|
\end{center}
%
which can be uncommented to produce the final version of this child document.

%%%%%%%%%%%%%%%%%%%%%%%%%%%%%%%%%%%%%%%%%%%%%%%%%%%%%%%%%%%%%%%%%%%%%%%%%%%%%%%%
\subsection{Forwarding}
\label{sec:forward}

Different versions of the main or child documents
using compilation flags as described in \secref{sec:flags}
can be (permanently) stored in different files
for convenient compilation, viewing and distribution.
To this end, the package defines a command
to pass on compilation to a different file:

%%%%%%%%%%%%%%%%%%%%%%%%%%%%%%%%%%%%%%%%
\DescribeMacro{\childdocforward}
The command |\childdocforward| redirects processing to
another source file:
%
\begin{center}
\begin{tabular}{l}
|\input{childdoc.def}|\\
|\childdocforward[|\textit{main}|]{|\textit{dest}|}|\\
\end{tabular}
\end{center}
%
The argument \textit{dest} is the destination file
(without extension).
It should be the main file or one of the child files.
Note that further \textsf{childdoc} directives
such as |\childdocof| and |\childdocforward|
in the indicated file will be processed in this form.
The optional argument \textit{main}
passes on directly to the main file \textit{main}
while pretending to compile the child \textit{dest}.
This form behaves as if \textit{dest}
issues |\childdocof{|\textit{main}|}| right away,
and no further \textsf{childdoc} directives will be processed.

%%%%%%%%%%%%%%%%%%%%%%%%%%%%%%%%%%%%%%%%
\DescribeMacro{\...prefix}
In the alternative form |\childdocforwardprefix|,
%
\begin{center}
\begin{tabular}{l}
|\input{childdoc.def}|\\
|\childdocforwardprefix[|\textit{main}|]{|\textit{prefix}|}{|\textit{dest}|}|
\end{tabular}
\end{center}
%
the destination file is determined by a pattern
depending on the current file:
To make this work, the current file must be called
`{\textit{prefix}\hspace{0.2em}\textit{suffix}}'
with \textit{prefix} matching precisely the argument.
Processing is then passed on to the file
`{\textit{dest}\hspace{0.2em}\textit{suffix}}'.
Surely, the same effect is achieved by
directly specifying the
argument `{\textit{dest}\hspace{0.2em}\textit{suffix}}'
in the first form.
However, that requires to set up a different file
for each child. With the alternative form of the command
all these files can have exactly the same content
which simplifies setting them up and maintaining them.

For example, the following file |draft.tex|
with a compilation flag |\version| as described in \secref{sec:flags}
compiles the main document as a draft:
%
\begin{center}
\begin{tabular}{l}
|\def\version{draft}|\\
|\input{childdoc.def}|\\
|\childdocforward{|\textit{main}|}|
\end{tabular}
\end{center}
%
Likewise, the following files |final|\textit{nn}|.tex|
compile the final version of the child document
|child|\textit{nn}|.tex|:
%
\begin{center}
\begin{tabular}{l}
|\def\version{final}|\\
|\input{childdoc.def}|\\
|\childdocforwardprefix{final}{child}|
\end{tabular}
\end{center}
%

Note that when several versions of a main file and/or of each child file
are to be generated, it may be convenient to set up a |Makefile| or
shell script to automatise the process.

%%%%%%%%%%%%%%%%%%%%%%%%%%%%%%%%%%%%%%%%%%%%%%%%%%%%%%%%%%%%%%%%%%%%%%%%%%%%%%%%
\subsection{Command Line Processing}
\label{sec:commandline}

The effect of redirection files can also be achieved by invoking
the \LaTeX{} compiler with a more elaborate command line.
Most conveniently this should be done as part
of a shell script or a |Makefile|.

When using \textsf{childdoc} in the main file, the following
command lines effectively perform a redirection
(note that depending on the shell being used,
backslashes may have to be doubled: `|\|' $\to$ `|\\|'):
%
\begin{center}
|... -jobname "|\textit{target}|" |\\|"|[\textit{flags}]%
|\input{childdoc.def}\childdocforward[|\textit{main}|]{|\textit{dest}|}"|
\end{center}
%
Here \textit{target} is the name of the output file,
\textit{main} is the name of the main file
and \textit{dest} is the name of the main or child file to be processed
(all filenames without extensions).
The optional argument \textit{main} can be omitted
if \textit{main} matches \textit{dest}.
Optionally, compilation \textit{flags} can be defined via |\def| commands.
This command line makes the \TeX{} engine believe
it is compiling the file \textit{target}
whose content is specified as the latter parameter.
The provided code then forwards the processing to
\textit{main} or \textit{dest} as described in \secref{sec:forward}.

%%%%%%%%%%%%%%%%%%%%%%%%%%%%%%%%%%%%%%%%%%%%%%%%%%%%%%%%%%%%%%%%%%%%%%%%%%%%%%%%
\subsection{Include by Input}
\label{sec:input}

Including child documents by |\include| has some restrictions by design.
Most notably, the content of a child document always occupies
its own set of pages; pages cannot be shared between child documents.
Usually, this behaviour makes perfect sense
because each child document contain an essential part of the document.
However, in some situations it may be desirable to compose
a document from a collection of parts
without having mandatory page breaks between then.
For this case, the package
provides a mechanism to include parts
by |\input| which can also be processed individually.
However, by construction this mechanism
requires manual handling of the content to be output.

%%%%%%%%%%%%%%%%%%%%%%%%%%%%%%%%%%%%%%%%
\DescribeMacro{\ifchilddocmanual}
The main file should be prepared as usual, see \secref{sec:include}.
However, the document body must make a distinction
between processing of an individual part and of the main document, e.g.:
%
\begin{center}
\begin{tabular}{l}
|\ifchilddocmanual|\\
|\input{\childdocname}|\\
|\||else|\\
\textit{document body with }|\input{|\textit{part}|}|\\
|\||fi|
\end{tabular}
\end{center}
%
The conditional |\ifchilddocmanual| is true whenever
a part to be included by |\input| is being compiled,
and the name of the part is stored in |\childdocname|.

%%%%%%%%%%%%%%%%%%%%%%%%%%%%%%%%%%%%%%%%
\DescribeMacro{\childdocby}
Each part to be included by |\input| should start with:
%
\begin{center}
\begin{tabular}{l}
|\input{childdoc.def}|\\
|\childdocby{|\textit{main}|}|\\
\end{tabular}
\end{center}
%
The directive |\childdocby| is similar to |\childdocof|
described in \secref{sec:include},
but the subsequent selection of content must be done manually.
To that end, both |\ifchilddoc| and |\ifchilddocmanual|
will be true upon processing of a part,
and the name of the part is stored in |\childdocname|.
Note that |\jobname| will be set to the filename of the current part
so that each part receives an individual |.aux| file
that does not interfere with the |.aux| file(s) of the main document.
This behaviour can be altered by the alternative form
|\childdocby[*]{|\textit{main}|}| (with a non-empty optional argument)
which uses the |.aux| file of the main document
by setting |\jobname| to \textit{main}.

%%%%%%%%%%%%%%%%%%%%%%%%%%%%%%%%%%%%%%%%%%%%%%%%%%%%%%%%%%%%%%%%%%%%%%%%%%%%%%%%
\subsection{Driver Development}
\label{sec:driver}

The \textsf{childdoc} mechanism can also be use for the development
of definition files such as \LaTeX{} styles or classes.
This case differs from the above setup with multiple parts
included by |\include| in that no |\includeonly| should be invoked.
This can be achieved by starting the include file
(before |\ProvidesPackage|) with:
%
\begin{center}
\begin{tabular}{l}
|\input{childdoc.def}|\\
|\childdocforward{|\textit{main}|}|\\
\end{tabular}
\end{center}
%
or alternatively with:
%
\begin{center}
\begin{tabular}{l}
|\input{childdoc.def}|\\
|\childdocby{|\textit{main}|}|\\
\end{tabular}
\end{center}
%
Both forms have slightly different effects as described above.
The main file is prepared as usual, see \secref{sec:include}.

%%%%%%%%%%%%%%%%%%%%%%%%%%%%%%%%%%%%%%%%%%%%%%%%%%%%%%%%%%%%%%%%%%%%%%%%%%%%%%%%
\subsection{Legacy Detection}
\label{sec:detection}

The directive |\childdocmain| in the main file can detect
whether the complete document or merely a child is to be compiled
even without using the directive |\childdocof|.
This method is deprecated because it is less robust
and there is no compelling reason to use it;
it is merely provided for backward compatibility
and it may be removed in future versions.

If the detection mechanism is to be used,
it is mandatory to correctly specify
the filename of the main file as the argument of |\childdocmain|:
%
\begin{center}
\begin{tabular}{l}
|\input{childdoc.def}|\\
|\childdocmain{|\textit{main}|}|\\
\end{tabular}
\end{center}
%
If |\jobname| does not match the argument \textit{main} of |\childdocmain|,
it is assumed that |\jobname| points to the child file to be compiled.
When using |\childdocmain| with the main file specified as argument,
it suffices to start a child file
with just |\input{|\textit{main}|}|
without loading of the package and using |\childdocof|.
If instead all processing is done
with the appropriate \textsf{childdoc} directives,
the argument of \textit{main} of |\childdocmain| can be empty.

An alternative version of the command line processing described
in \secref{sec:commandline} using the detection mechanism reads:
%
\begin{center}
|... -jobname "|\textit{target}|" "|[\textit{flags}]%
[|\def\jobname{|\textit{dest}|}|]|\input{|\textit{main}|}"|
\end{center}

%%%%%%%%%%%%%%%%%%%%%%%%%%%%%%%%%%%%%%%%%%%%%%%%%%%%%%%%%%%%%%%%%%%%%%%%%%%%%%%%
\subsection{Manual Code}
\label{sec:manual}

In case one cannot be certain whether the definitions file |childdoc.def|
is installed on the target \TeX{} distribution
and one prefers not to ship it,
it is conceivable to paste a few relevant commands into the sources.

To that end, drop all statements |\input{childdoc.def}|
and perform the replacements as outlined below.
Instead of |\childdocmain{|\textit{main}|}| add the following code
to the top of the main file:
%
\begin{center}
\begin{tabular}{l}
|\||ifdefined\childdocname\endinput\||fi\newif\ifchilddoc|\\
|\edef\childdocname{\scantokens\expandafter{\jobname\noexpand}}|\\
|\def\childdocmain{|\textit{main}|}\||ifx\childdocmain\childdocname\||else|\\
|\childdoctrue\includeonly{\childdocname}\let\jobname\childdocmain\||fi|\\
\end{tabular}
\end{center}
%
Instead of |\childdocof{|\textit{main}|}| just include the main file
at the top of each child file:
%
\begin{center}
|\input{|\textit{main}|}|
\end{center}
%
A simple redirection |\childdocforward{|\textit{dest}|}| is achieved by:
%
\begin{center}
|\def\jobname{|\textit{dest}|}\input{\jobname}|
\end{center}
%
The redirection with prefix
|\childdocforwardprefix[|\textit{prefix}|]{|\textit{dest}|}|
is accomplished by:
%
\begin{center}
\begin{tabular}{l}
|{\edef\jobname{\scantokens\expandafter{\jobname\noexpand}}|\\
|\def\redirectjob |\textit{prefix}|#1~~~{\gdef\jobname{|\textit{dest}|#1}}|\\
|\expandafter\redirectjob\jobname~~~}\input{\jobname}|
\end{tabular}
\end{center}

In an alternative approach,
child documents can be compiled by a specific command line
without additional code or specific definitions:
%
\begin{center}
|... -jobname "|\textit{target}|" "|[\textit{flags}]%
|\includeonly{|\textit{dest}|}\input{|\textit{main}|}"|
\end{center}
%

%%%%%%%%%%%%%%%%%%%%%%%%%%%%%%%%%%%%%%%%%%%%%%%%%%%%%%%%%%%%%%%%%%%%%%%%%%%%%%%%
%%%%%%%%%%%%%%%%%%%%%%%%%%%%%%%%%%%%%%%%%%%%%%%%%%%%%%%%%%%%%%%%%%%%%%%%%%%%%%%%
\section{Information}

%%%%%%%%%%%%%%%%%%%%%%%%%%%%%%%%%%%%%%%%%%%%%%%%%%%%%%%%%%%%%%%%%%%%%%%%%%%%%%%%
\subsection{Copyright}

Copyright \copyright{} 2017--2018 Niklas Beisert

This work may be distributed and/or modified under the
conditions of the \LaTeX{} Project Public License, either version 1.3
of this license or (at your option) any later version.
The latest version of this license is in
  \url{http://www.latex-project.org/lppl.txt}
and version 1.3 or later is part of all distributions of \LaTeX{}
version 2005/12/01 or later.

This work has the LPPL maintenance status `maintained'.

The Current Maintainer of this work is Niklas Beisert.

This work consists of the files |README.txt|, |childdoc.ins| and |childdoc.dtx|
as well as the derived files |childdoc.def|, |cdocsamp.tex|
with |cdocsch1.tex|, |cdocsch2.tex|, |cdocspt3.tex|, |cdocspt4.tex|,
|cdocsdrf.tex|, |cdocsfn1.tex|, |cdocsfn2.tex|
as well as |childdoc.pdf|.

%%%%%%%%%%%%%%%%%%%%%%%%%%%%%%%%%%%%%%%%%%%%%%%%%%%%%%%%%%%%%%%%%%%%%%%%%%%%%%%%
\subsection{Files and Installation}

The package consists of the files:
%
\begin{center}
\begin{tabular}{ll}
    |README.txt|   & readme file \\
    |childdoc.ins| & installation file \\
    |childdoc.dtx| & source file \\
    |childdoc.def| & definition file \\
    |cdocsamp.tex| & sample main file \\
    |cdocsch1.tex| & sample include file \\
    |cdocsch2.tex| & sample include file \\
    |cdocspt3.tex| & sample part file \\
    |cdocspt4.tex| & sample part file \\
    |cdocsdrf.tex| & sample redirection file \\
    |cdocsfn1.tex| & sample redirection file \\
    |cdocsfn2.tex| & sample redirection file \\
    |childdoc.pdf| & manual
\end{tabular}
\end{center}
%
The distribution consists of the files
|README.txt|, |childdoc.ins| and |childdoc.dtx|.
%
\begin{itemize}
\item
Run (pdf)\LaTeX{} on |childdoc.dtx|
to compile the manual |childdoc.pdf| (this file).
\item
Run \LaTeX{} on |childdoc.ins| to create the definitions file |childdoc.def|
and the sample |cdocsamp.tex| with include files
|cdocsch1.tex|, |cdocsch2.tex|, |cdocspt3.tex|, |cdocspt4.tex|,
|cdocsdrf.tex|, |cdocsfn1.tex|, |cdocsfn2.tex|.
Then copy the file |childdoc.def| to an appropriate directory of your \LaTeX{}
distribution, e.g.\ \textit{texmf-root}|/tex/latex/childdoc|.
\end{itemize}

%%%%%%%%%%%%%%%%%%%%%%%%%%%%%%%%%%%%%%%%%%%%%%%%%%%%%%%%%%%%%%%%%%%%%%%%%%%%%%%%
\subsection{Related CTAN Packages}

There are several other packages which offer a similar functionality:
%
\begin{itemize}
\item
The packages
\href{http://ctan.org/pkg/docmute}{\textsf{docmute}},
\href{http://ctan.org/pkg/includex}{\textsf{includex}} and
\href{http://ctan.org/pkg/standalone}{\textsf{standalone}}
provide commands to include only the document body of
a child file thus allowing both files to be compiled individually.
\item
The packages \href{http://ctan.org/pkg/subdocs}{\textsf{subdocs}}
and \href{http://ctan.org/pkg/subfiles}{\textsf{subfiles}}
provide structures in which the main and child documents can be
encapsulated and allowing them to be compiled individually.
The inclusion mechanism is different from the conventional |\include|.
\item
The package \href{http://ctan.org/pkg/combine}{\textsf{combine}}
is an elaborate solution to combine several documents into one.
\end{itemize}
%
See also the CTAN topic \href{http://ctan.org/topic/subdocs}{\textsf{subdocs}}
for further related packages.
The present package differs from the above solutions in that
a document structure constructed with the conventional |\include| mechanism
just needs two extra commands at the top of every file
such that all constituent files can be compiled individually.

%%%%%%%%%%%%%%%%%%%%%%%%%%%%%%%%%%%%%%%%%%%%%%%%%%%%%%%%%%%%%%%%%%%%%%%%%%%%%%%%
%\subsection{Feature Suggestions}
%
%The following is a list of features which may be useful for future
%versions of this package:
%%
%\begin{itemize}
%\item
%\ldots
%\end{itemize}

%%%%%%%%%%%%%%%%%%%%%%%%%%%%%%%%%%%%%%%%%%%%%%%%%%%%%%%%%%%%%%%%%%%%%%%%%%%%%%%%
\subsection{Revision History}

%%%%%%%%%%%%%%%%%%%%%%%%%%%%%%%%%%%%%%%%
\paragraph{v2.0:} 2018/12/30

\begin{itemize}
\item
immediate forward processing
\item
added |\childdocby| mechanism
\item
manual restructured
\end{itemize}

%%%%%%%%%%%%%%%%%%%%%%%%%%%%%%%%%%%%%%%%
\paragraph{v1.6:} 2018/01/17

\begin{itemize}
\item
application for development of include files
\item
corrections to manual
\end{itemize}

%%%%%%%%%%%%%%%%%%%%%%%%%%%%%%%%%%%%%%%%
\paragraph{v1.5:} 2017/05/21

\begin{itemize}
\item
more complete structuring introduced
\item
|\childdocof| introduced
\item
|\childdoc| renamed to |\childdocmain|
\item
|\childredirect| renamed to |\childdocforward| and |\childdocforwardprefix|
and functionality expanded
\end{itemize}

%%%%%%%%%%%%%%%%%%%%%%%%%%%%%%%%%%%%%%%%
\paragraph{v1.0:} 2017/04/27

\begin{itemize}
\item
manual and install package
\item
first version published on CTAN
\end{itemize}

%%%%%%%%%%%%%%%%%%%%%%%%%%%%%%%%%%%%%%%%
\paragraph{v0.6:} 2017/04/26

\begin{itemize}
\item
redirection mechanism added
\end{itemize}

%%%%%%%%%%%%%%%%%%%%%%%%%%%%%%%%%%%%%%%%
\paragraph{v0.5:} 2017/04/26

\begin{itemize}
\item
functionality in definition file
\end{itemize}


%%%%%%%%%%%%%%%%%%%%%%%%%%%%%%%%%%%%%%%%%%%%%%%%%%%%%%%%%%%%%%%%%%%%%%%%%%%%%%%%
%%%%%%%%%%%%%%%%%%%%%%%%%%%%%%%%%%%%%%%%%%%%%%%%%%%%%%%%%%%%%%%%%%%%%%%%%%%%%%%%
%%%%%%%%%%%%%%%%%%%%%%%%%%%%%%%%%%%%%%%%%%%%%%%%%%%%%%%%%%%%%%%%%%%%%%%%%%%%%%%%
\appendix

\settowidth\MacroIndent{\rmfamily\scriptsize 000\ }

 \DocInput{childdoc.dtx}

\end{document}
%</driver>
% \fi
%
% %%%%%%%%%%%%%%%%%%%%%%%%%%%%%%%%%%%%%%%%%%%%%%%%%%%%%%%%%%%%%%%%%%%%%%%%%%%%%%
% %%%%%%%%%%%%%%%%%%%%%%%%%%%%%%%%%%%%%%%%%%%%%%%%%%%%%%%%%%%%%%%%%%%%%%%%%%%%%%
% \section{Sample}
%\iffalse
%<*samplemain>
%\fi
%
% The following presents a sample document
% with two chapters, two parts, a title page,
% a compile flag as well as three forwarding files to set the flag.
% It consists of eight |.tex| files:
% \begin{center}
% \begin{tabular}{ll}
% |cdocsamp.tex|&main file\\
% |cdocsch1.tex|&include file for chapter 1\\
% |cdocsch2.tex|&include file for chapter 2\\
% |cdocspt3.tex|&include file for part 3\\
% |cdocspt4.tex|&include file for part 4\\
% |cdocsdrf.tex|&forwarding file for main file in draft mode\\
% |cdocsfi1.tex|&forwarding file for final version of chapter 1\\
% |cdocsfi2.tex|&forwarding file for final version of chapter 2\\
% \end{tabular}
% \end{center}
% Each of the eight files can be compiled directly by the \LaTeX{} compiler.
%
% %%%%%%%%%%%%%%%%%%%%%%%%%%%%%%%%%%%%%%
% \paragraph{Main File.}
%
% The main file is called |cdocsamp.tex|.
%
% Load the \textsf{childdoc} definitions and
% declare the filename for the main document:
%    \begin{macrocode}
\input{childdoc.def}
\childdocmain{}
%    \end{macrocode}

% Optional override for |\version| flag:
%    \begin{macrocode}
%%\ifchilddoc\else\providecommand{\version}{draft}\fi
%    \end{macrocode}

% Define the default values for the |\version| flag
% (|final| for the main file and |draft| for childs):
%    \begin{macrocode}
\ifchilddoc
\providecommand{\version}{draft}
\else
\providecommand{\version}{final}
\fi
%    \end{macrocode}

% Load the standard document class:
%    \begin{macrocode}
\documentclass[12pt]{article}
%    \end{macrocode}

% Start the document body:
%    \begin{macrocode}
\begin{document}
%    \end{macrocode}

% Declare a title page.
% Print title, part of document being processed and version flag:
%    \begin{macrocode}
\addtocounter{page}{-1}
\begin{center}
{\LARGE\bfseries{}childdoc example\par}
\vspace{1cm}
\ifchilddoc
\ifchilddocmanual part\else chapter\fi:
`\childdocname' of `\childdocjob'\par
\else
main document: `\childdocjob'\par
\fi
version: \version\par
\end{center}
\newpage
%    \end{macrocode}

% Manually include selected file,
% otherwise process as usual:
%    \begin{macrocode}
\ifchilddocmanual
\section*{part `\childdocname'}
\input{\childdocname}
\else
%    \end{macrocode}

% Include the two chapters:
%    \begin{macrocode}
\include{cdocsch1}
\include{cdocsch2}
%    \end{macrocode}

% Include the two parts unless only chapters should be displayed:
%    \begin{macrocode}
\ifchilddoc\else
\section{part three}
\input{cdocspt3}
\section{part four}
\input{cdocspt4}
\fi
%    \end{macrocode}

% Process as usual until here:
%    \begin{macrocode}
\fi
%    \end{macrocode}

% End of document body:
%    \begin{macrocode}
\end{document}
%    \end{macrocode}
%\iffalse
%</samplemain>
%\fi
%
% %%%%%%%%%%%%%%%%%%%%%%%%%%%%%%%%%%%%%%
% \paragraph{Chapter Include Files.}
%
% The include files are called |cdocsch1.tex| and |cdocsch2.tex|.
%
%\iffalse
%<*samplechap1|samplechap2>
%\fi

% Optional override for |\version| flag:
%    \begin{macrocode}
%%\providecommand{\version}{final}
%    \end{macrocode}

% Include the main document:
%    \begin{macrocode}
\input{childdoc.def}
\childdocof{cdocsamp}
%    \end{macrocode}

%\iffalse
%</samplechap1|samplechap2>
%\fi
%
%\iffalse
%<*samplechap1>
%\fi
% Some text for chapter 1:
%    \begin{macrocode}
\section{one}
some text in chapter one
%    \end{macrocode}

%\iffalse
%</samplechap1>
%\fi
% Some text for chapter 2:
%\iffalse
%<*samplechap2>
%\fi
%    \begin{macrocode}
\section{two}
more text in chapter two
%    \end{macrocode}

%\iffalse
%</samplechap2>
%\fi
%
% %%%%%%%%%%%%%%%%%%%%%%%%%%%%%%%%%%%%%%
% \paragraph{Part Include Files.}
%
% The include files are called |cdocspt3.tex| and |cdocspt4.tex|.
%
%\iffalse
%<*samplepart3|samplepart4>
%\fi

% Optional override for |\version| flag:
%    \begin{macrocode}
%%\providecommand{\version}{final}
%    \end{macrocode}

% Include the main document:
%    \begin{macrocode}
\input{childdoc.def}
\childdocby{cdocsamp}
%    \end{macrocode}

%\iffalse
%</samplepart3|samplepart4>
%\fi
%
%\iffalse
%<*samplepart3>
%\fi
% Some text for part 3:
%    \begin{macrocode}
some text in part three
%    \end{macrocode}

%\iffalse
%</samplepart3>
%\fi
% Some text for part 4:
%\iffalse
%<*samplepart4>
%\fi
%    \begin{macrocode}
more text in part four
%    \end{macrocode}

%\iffalse
%</samplepart4>
%\fi
%
% %%%%%%%%%%%%%%%%%%%%%%%%%%%%%%%%%%%%%%
% \paragraph{Forwarding for a Complete Draft.}
%
% The following forwarding file |cdocsdrf.tex|
% compiles the main document in draft mode:
%\iffalse
%<*sampledraft>
%\fi
%    \begin{macrocode}
\def\version{draft}
\input{childdoc.def}
\childdocforward{cdocsamp}
%    \end{macrocode}

%\iffalse
%</sampledraft>
%\fi
%
% %%%%%%%%%%%%%%%%%%%%%%%%%%%%%%%%%%%%%%
% \paragraph{Forwarding for Final Version of the Chapters.}
%
% The following forwarding files |cdocsfn1.tex| and |cdocsfn2.tex|
% (with identical content)
% compile the final versions of the child documents
% |cdocsch1.tex| and |cdocsch2.tex|, respectively:
%\iffalse
%<*samplefinal>
%\fi
%    \begin{macrocode}
\def\version{final}
\input{childdoc.def}
\childdocforwardprefix[cdocsamp]{cdocsfn}{cdocsch}
%    \end{macrocode}

%\iffalse
%</samplefinal>
%\fi
%
% %%%%%%%%%%%%%%%%%%%%%%%%%%%%%%%%%%%%%%
% \paragraph{Command Line Processing.}
%
% The following three command lines generate the output files
% |cdocscld|, |cdocscl1| and |cdocscl2|
% which should be identical to
% |cdocsdrf|, |cdocsch1| and |cdocsfn2|, respectively:
% \begin{center}
% \begin{tabular}{l}
% |latex -jobname cdocscld \|\\
% |  "\def\version{draft}\input{childdoc.def}\childdocforward{cdocsamp}"|\\
% |latex -jobname cdocscl1 \|\\
% |  "\input{childdoc.def}\childdocforward[cdocsamp]{cdocsch1}"|\\
% |latex -jobname cdocscl2 \|\\
% |  "\def\version{final}\input{childdoc.def}\childdocforward{cdocsch2}"|
% \end{tabular}
% \end{center}
% Note that the trailing backslash on each first line
% merely continues the input to the second line
% (for convenient cut ant paste).
% Furthermore, the command |latex| can be replaced by any
% of its alternative versions such as |pdflatex|.
%
% %%%%%%%%%%%%%%%%%%%%%%%%%%%%%%%%%%%%%%%%%%%%%%%%%%%%%%%%%%%%%%%%%%%%%%%%%%%%%%
% %%%%%%%%%%%%%%%%%%%%%%%%%%%%%%%%%%%%%%%%%%%%%%%%%%%%%%%%%%%%%%%%%%%%%%%%%%%%%%
% \section{Implementation}
%\iffalse
%<*package>
%\fi
%
% This section describes the definitions file |childdoc.def|.

% The definitions cannot be loaded using |\usepackage| or |\RequirePackage|
% which has a mechanism to prevent loading a style file more than once.
% When loading the definitions by means of |\input|
% multiple instances have to be prevented manually:
%\iffalse
%This code needs to be before the `\ProvidesFile' directive
%which is defined at the beginning of this file.
%Therefore it is also placed there and commented out here.
%</package>
%<*discard>
%\fi
%    \begin{macrocode}
\ifdefined\childdocmain\endinput\fi
%    \end{macrocode}
%\iffalse
%</discard>
%<*package>
%\fi
%
% \macro{\ifchilddoc}
% \macro{\ifchilddocmanual}
% The conditional |\ifchilddoc| tells whether a
% child (true) or main (false) document is being compiled.
% The conditional |\ifchilddocmanual| tells whether
% the |\includeonly| mechanism is used (false) or
% the selection of child files must be performed manually (true).
% The definitions initialise to false:
%    \begin{macrocode}
\newif\ifchilddoc
\newif\ifchilddocmanual
%    \end{macrocode}

% \macro{\childdocname}
% \macro{\childdocjob}
% The macro |\childdocname| stores the name of the main document
% to be compiled. The macro |\childdocjob| stores the name of
% the document on which the \LaTeX{} compiler was originally invoked.
% The content of |\jobname| cannot be compared
% to filenames specified in the source due to different catcodes.
% The following code rescans |\jobname|, stores the result
% in |\childdocname| and saves a copy in |\childdocjob|:
%    \begin{macrocode}
\edef\childdocname{\scantokens\expandafter{\jobname\noexpand}}
\let\childdocjob\childdocname
%    \end{macrocode}

% \macro{\childdocdisable}
% The macro |\childdocdisable| prevents the main file
% from being processed more than once.
% At this stage, the main document command |\childdocmain|
% is assumed to be called once again where it should do nothing.
% Any subsequent call to it should prevent
% a secondary processing of the main document
% It overwrites the forwarding commands
% |\childdocof| and |\childdocforward|
% with empty macros to prevent further inclusions of the main document:
%    \begin{macrocode}
\newcommand{\childdocdisable}
{
  \renewcommand{\childdocmain}[1]{\renewcommand{\childdocmain}[1]{\endinput}}
  \renewcommand{\childdocof}[1]{}
  \renewcommand{\childdocby}[2][]{}
  \renewcommand{\childdocforward}[2][]{}
  \renewcommand{\childdocdisable}{}
}
%    \end{macrocode}

% \macro{\childdocmain}
% The macro |\childdocmain| is to be called at the top of the main file
% with nothing or the main filename (without extension) as argument.
% First, it breaks loops.
% If the argument is not empty and does not match |\childdocname|
% (which is set by the first inclusion of |childdoc.def|),
% |\ifchilddoc| is set to true, |\includeonly| is applied to the child file
% and |\jobname| is set to the main file
% (for proper handling of |.aux| files):
%    \begin{macrocode}
\newcommand{\childdocmain}[1]
{
  \childdocdisable\childdocmain{}
  \if?#1?\else
    \begingroup
      \def\childdoctmp{#1}
      \ifx\childdoctmp\childdocname
        \def\childdoctmp{}
      \else
        \def\childdoctmp
        {
          \childdoctrue
          \includeonly{\childdocname}
          \def\childdocjob{#1}
          \def\jobname{#1}
        }
      \fi
      \expandafter
    \endgroup
    \childdoctmp
  \fi
}
%    \end{macrocode}

% \macro{\childdocof}
% The command |\childdocof| redirects
% compilation to the main file |#1|.
%    \begin{macrocode}
\newcommand{\childdocof}[1]
{
  \childdocdisable
  \childdoctrue
  \includeonly{\childdocname}
  \def\jobname{#1}
  \def\childdocjob{#1}
  \input{#1}
}
%    \end{macrocode}

% \macro{\childdocby}
% The command |\childdocby| ....
%    \begin{macrocode}
\newcommand{\childdocby}[2][]
{
  \childdocdisable
  \childdoctrue
  \childdocmanualtrue
  \if?#1?\else
    \def\jobname{#2}
  \fi
  \def\childdocjob{#2}
  \input{#2}
  \endinput
}
%    \end{macrocode}

% \macro{\childdocforward}
% The command |\childdocforward| redirects
% compilation to the main file or
% (if the optional argument is given) a child file.
% Parameters are set as if the main file
% or a child file starting with |\childdocof| was compiled.
% Then compilation is handed over to the main file:
%    \begin{macrocode}
\newcommand{\childdocforward}[2][]
{
  \begingroup
    \if?#1?
      \def\childdoctmp
      {
        \def\childdocname{#2}
        \def\childdocjob{#2}
        \def\jobname{#2}
        \input{#2}
        \endinput
      }
    \else
      \def\childdoctmp
      {
        \childdocdisable
        \def\childdocname{#2}
        \childdoctrue
        \includeonly{#2}
        \def\childdocjob{#1}
        \def\jobname{#1}
        \input{#1}
        \endinput
      }
    \fi
    \expandafter
  \endgroup
  \childdoctmp
}
%    \end{macrocode}

% \macro{\childdocforwardprefix}
% The command |\childdocforwardprefix| redirects
% compilation to the main or a child file by means of a pattern.
% The prefix |#1| in the current filename is replaced by |#2|
% and the suffix of the current filename is kept
% (it is assumed that the filename does not contain the substring `|~~~|'
% which is used as a delimiter).
% Compilation is handed over to the new file by |\childdocforward|:
%    \begin{macrocode}
\newcommand{\childdocforwardprefix}[3][]
{
  \begingroup
    \def\childdocextract #2##1~~~{\def\childdoctmp{\childdocforward[#1]{#3##1}}}
    \expandafter\childdocextract\childdocname~~~
    \expandafter
  \endgroup
  \childdoctmp
}
%    \end{macrocode}

% \macro{\childdoc}
% The deprecated macro |\childdoc| is a legacy version of |\childdocmain|:
%    \begin{macrocode}
\newcommand{\childdoc}{\childdocmain}
%    \end{macrocode}

% \macro{\childdocredirect}
% The deprecated macro |\childdocredirect| is a legacy version
% of |\childdocforward| and |\childdocforwardprefix|:
%    \begin{macrocode}
\newcommand{\childdocredirect}[2][]
{
  \begingroup
    \if?#1?
      \def\childdoctmp{\childdocforward{#2}}
    \else
      \def\childdoctmp{\childdocforwardprefix{#1}{#2}}
    \fi
    \expandafter
  \endgroup
  \childdoctmp
}
%    \end{macrocode}

%\iffalse
%</package>
%\fi
%
\endinput

\childdocof{cdocsamp}
%    \end{macrocode}

%\iffalse
%</samplechap1|samplechap2>
%\fi
%
%\iffalse
%<*samplechap1>
%\fi
% Some text for chapter 1:
%    \begin{macrocode}
\section{one}
some text in chapter one
%    \end{macrocode}

%\iffalse
%</samplechap1>
%\fi
% Some text for chapter 2:
%\iffalse
%<*samplechap2>
%\fi
%    \begin{macrocode}
\section{two}
more text in chapter two
%    \end{macrocode}

%\iffalse
%</samplechap2>
%\fi
%
% %%%%%%%%%%%%%%%%%%%%%%%%%%%%%%%%%%%%%%
% \paragraph{Part Include Files.}
%
% The include files are called |cdocspt3.tex| and |cdocspt4.tex|.
%
%\iffalse
%<*samplepart3|samplepart4>
%\fi

% Optional override for |\version| flag:
%    \begin{macrocode}
%%\providecommand{\version}{final}
%    \end{macrocode}

% Include the main document:
%    \begin{macrocode}
% \iffalse
%
% childdoc.dtx Copyright (C) 2017-2018 Niklas Beisert
%
% This work may be distributed and/or modified under the
% conditions of the LaTeX Project Public License, either version 1.3
% of this license or (at your option) any later version.
% The latest version of this license is in
%   http://www.latex-project.org/lppl.txt
% and version 1.3 or later is part of all distributions of LaTeX
% version 2005/12/01 or later.
%
% This work has the LPPL maintenance status `maintained'.
%
% The Current Maintainer of this work is Niklas Beisert.
%
% This work consists of the files childdoc.dtx and childdoc.ins
% and the derived files childdoc.def and cdocsamp.tex with
% cdocsch1.tex, cdocsch2.tex, cdocsdrf.tex, cdocsfn1.tex, cdocsfn2.tex.
%
%<package>\ifdefined\childdocmain\endinput\fi
%<package>\ProvidesFile{childdoc.def}[2018/12/30 v2.0 child document driver]
%<samplemain>\ProvidesFile{cdocsamp.tex}[2018/12/30 v2.0 sample for childdoc]
%<*driver>
%\ProvidesFile{childdoc.drv}[2018/12/30 v2.0 childdoc reference manual file]
\PassOptionsToClass{10pt,a4paper}{article}
\documentclass{ltxdoc}

\usepackage[margin=35mm]{geometry}
\usepackage{hyperref}
\usepackage{hyperxmp}
\usepackage[usenames]{color}

\hypersetup{colorlinks=true}
\hypersetup{pdfstartview=FitH}
\hypersetup{pdfpagemode=UseNone}
\hypersetup{pdfsource={}}
\hypersetup{pdflang={en-UK}}
\hypersetup{pdfcopyright={Copyright 2017-2018 Niklas Beisert.
  This work may be distributed and/or modified under the
  conditions of the LaTeX Project Public License, either version 1.3
  of this license or (at your option) any later version.}}
\hypersetup{pdflicenseurl={http://www.latex-project.org/lppl.txt}}
\hypersetup{pdfcontactaddress={ETH Zurich, ITP, HIT K,
  Wolfgang-Pauli-Strasse 27}}
\hypersetup{pdfcontactpostcode={8093}}
\hypersetup{pdfcontactcity={Zurich}}
\hypersetup{pdfcontactcountry={Switzerland}}
\hypersetup{pdfcontactemail={nbeisert@itp.phys.ethz.ch}}
\hypersetup{pdfcontacturl={http://people.phys.ethz.ch/\xmptilde nbeisert/}}

\newcommand{\secref}[1]{\hyperref[#1]{section \ref*{#1}}}

\parskip1ex
\parindent0pt
\let\olditemize\itemize
\def\itemize{\olditemize\parskip0pt}

\begin{document}

\title{The \textsf{childdoc} Package}
\hypersetup{pdftitle={The childdoc Package}}
\author{Niklas Beisert\\[2ex]
  Institut f\"ur Theoretische Physik\\
  Eidgen\"ossische Technische Hochschule Z\"urich\\
  Wolfgang-Pauli-Strasse 27, 8093 Z\"urich, Switzerland\\[1ex]
  \href{mailto:nbeisert@itp.phys.ethz.ch}
  {\texttt{nbeisert@itp.phys.ethz.ch}}}
\hypersetup{pdfauthor={Niklas Beisert}}
\hypersetup{pdfsubject={Manual for the LaTeX2e Package childdoc}}
\date{30 December 2018, \textsf{v2.0}}
\maketitle

\begin{abstract}\noindent
\textsf{childdoc} is a \LaTeXe{} package
that enables the direct compilation
of document sections included by |\include|
to individual files.
\end{abstract}

\begingroup
\parskip0ex
\tableofcontents
\endgroup

%%%%%%%%%%%%%%%%%%%%%%%%%%%%%%%%%%%%%%%%%%%%%%%%%%%%%%%%%%%%%%%%%%%%%%%%%%%%%%%%
%%%%%%%%%%%%%%%%%%%%%%%%%%%%%%%%%%%%%%%%%%%%%%%%%%%%%%%%%%%%%%%%%%%%%%%%%%%%%%%%
\section{Introduction}

\LaTeX{} provides a mechanism to structure a large document (such as a book)
into a main file and several child files (containing the chapters)
using the |\include| command.
This mechanism is beneficial for documents
which span hundreds of pages in order to
make the source file(s) more manageable.
Moreover, compilation can be restricted to
selected child files by means of the |\includeonly| command.
The latter feature can be used to reduce the compilation time while editing
(this was significantly more useful in the earlier days of \LaTeX{})
or to generate a smaller document which is easier to navigate.
Another application of |\includeonly| is to generate
documents consisting of selected parts of the complete document.

However, there are a few drawbacks of the plain |\include| mechanism:
\begin{itemize}
\item
The child files cannot be compiled on their own,
they can only be compiled via the main file.
A naive editing environment
(such as a text editor with an option
to have the current file processed by \LaTeX)
may require one to switch to the main file before compiling;
attempting to compile the child file produces errors.
\item
The main file must be modified (each time)
to adjust the |\includeonly| command
to the present needs. This easily leaves the main file in a messy state.
\item
The generated document will always carry the filename
of the main document. This is inconvenient if
several child files are to be compiled and
to be kept for distribution.
\end{itemize}

The present package provides a simple interface
to make child files individually compilable by \LaTeX{}.
Compiling a child file then has the same effect as compiling
the main file with an |\includeonly| command
to select the appropriate child.
Moreover the generated document will carry the name of the child
rather than the main file.
This resolves all three above issues.

This feature is meant to make the editing of books,
thesis documents and lecture notes somewhat more convenient.
However, the package can also be used efficiently for
composing a series of documents (such as exercise sheets)
which are typically distributed individually.
It then assists the author in generating the individual documents
(potentially in different versions)
as well as a document containing the collected series.
Another application is in developing style files
or other kinds of included material
where compilation of the style file could redirect
to a sample or test file.

%%%%%%%%%%%%%%%%%%%%%%%%%%%%%%%%%%%%%%%%%%%%%%%%%%%%%%%%%%%%%%%%%%%%%%%%%%%%%%%%
%%%%%%%%%%%%%%%%%%%%%%%%%%%%%%%%%%%%%%%%%%%%%%%%%%%%%%%%%%%%%%%%%%%%%%%%%%%%%%%%
\section{Usage}

First of all, the package \textsf{childdoc} is \emph{not} a standard
\LaTeXe{} |.sty| style file! Therefore it needs to be invoked in
a non-standard way.

%%%%%%%%%%%%%%%%%%%%%%%%%%%%%%%%%%%%%%%%%%%%%%%%%%%%%%%%%%%%%%%%%%%%%%%%%%%%%%%%
\subsection{Included Files}
\label{sec:include}

%%%%%%%%%%%%%%%%%%%%%%%%%%%%%%%%%%%%%%%%
\DescribeMacro{\childdocmain}
To use the package, add the commands
\begin{center}
\begin{tabular}{l}
|\input{childdoc.def}|\\
|\childdocmain{}|\\
\end{tabular}
\end{center}
at the very top of the main \LaTeX{} file,
in particular \emph{before} the |\documentclass| statement!
The argument of |\childdocmain| should be left empty
(but it must be present).

%%%%%%%%%%%%%%%%%%%%%%%%%%%%%%%%%%%%%%%%
\DescribeMacro{\childdocof}
Furthermore, add the commands
\begin{center}
\begin{tabular}{l}
|\input{childdoc.def}|\\
|\childdocof{|\textit{main}|}|\\
\end{tabular}
\end{center}
at the top of every child file \textit{child}
which is included by |\include{|\textit{child}|}|
from within the main file
(or at least for those files to be compiled individually).
The argument \textit{main} must be the filename of the main file.

There are a couple of
considerations in setting up the main and child documents:

%%%%%%%%%%%%%%%%%%%%%%%%%%%%%%%%%%%%%%%%
\paragraph{Restrictions.}

Please note the following restrictions:
\begin{itemize}
\item
|\childdocmain| must be called with one argument \textit{main}
to ensure compatibility with earlier version of the package.
It must either be empty (|\childdocmain{}|)
or precisely match the filename of the main file in which it is specified.
See \secref{sec:detection} for further information.
\item
The filename \textit{main} must be specified without the |.tex| extension.
\item
The filename \textit{main} is case sensitive
(even in case-insensitive file systems)
due to internal string comparison.
\item
The argument \textit{main} should be fully expanded, it cannot be a macro.
\item
Subdirectories and special characters should be avoided in filenames.
\item
The command |\childdocmain{|\textit{main}|}| must be followed by a whitespace.
It should not be followed immediately by another command
or by a comment mark `|%|'.
This is because the \TeX{} parser reads the token immediately following
the argument of |\childdocmain| and puts it
at the beginning of every child section;
however, a white\-space is ignored.
\end{itemize}

%%%%%%%%%%%%%%%%%%%%%%%%%%%%%%%%%%%%%%%%
\paragraph{Content of Main File.}

It is advisable to place all content in the child files included by |\include|.
Any output contained in the main file will appear in all child documents
unless suppressed manually;
it cannot be suppressed automatically by the |\includeonly| directive
and thus should normally be avoided.
A method to include some content in the main file
by means of conditional processing is described in \secref{sec:conditional}.

%%%%%%%%%%%%%%%%%%%%%%%%%%%%%%%%%%%%%%%%
\paragraph{Page Numbering.}

When only a part of the document is compiled,
the appropriate numbering of pages
(as well as other status parameters)
is determined from the |.aux| files.
The latter contain information from previous passes.
However this information needs to propagate through
all intermediate child documents.
Therefore the page numbering in child documents may well
be inconsistent until the complete document is compiled at least once.

A useful (if unconventional) way to always ensure a consistent
page numbering is to restart the numbering in each child document
and denote the pages by `\textit{child}|.|\textit{page}'
where \textit{child} represents the chapter/section number of the child file.
This can be achieved by the command
|\numberwithin{page}{|\textit{child}|}|
of the \textsf{amsmath} package
where \textit{child} can be |chapter| or |section|
depending on the chosen structuring.
Alternatively, one can modify the macro |\thepage| appropriately
and reset the counter |page| at the start of each child file.

%%%%%%%%%%%%%%%%%%%%%%%%%%%%%%%%%%%%%%%%%%%%%%%%%%%%%%%%%%%%%%%%%%%%%%%%%%%%%%%%
\subsection{Conditional Processing}
\label{sec:conditional}

The package provides a mechanism to compile different versions
of a document. To customise the versions further some conditional processing
can come in handy to distinguish which version is being compiled.
The package provides two macros to describe the compilation context:

%%%%%%%%%%%%%%%%%%%%%%%%%%%%%%%%%%%%%%%%
\DescribeMacro{\ifchilddoc}
The conditional |\ifchilddoc| distinguishes between the compilation of
child documents and the main document:
%
\begin{center}
|\ifchilddoc |\textit{child-code}| |[|\||else |\textit{main-code}]| \||fi|
\end{center}

%%%%%%%%%%%%%%%%%%%%%%%%%%%%%%%%%%%%%%%%
\DescribeMacro{\childdocname}
\DescribeMacro{\childdocjob}
The macro |\childdocname| contains the filename (without extension)
of the main or child file being processed.
Note that |\childdocjob| will always contain the name of the main file.

%%%%%%%%%%%%%%%%%%%%%%%%%%%%%%%%%%%%%%%%
\paragraph{Title Page.}

Conditional processing can be used to include a title or banner page
in the main document when proper precautions are taken.
Importantly, the code in the main file should ensure that the page counter
(as well as other status parameters which are stored in the |.aux| files)
takes the same value after the conditional processing.
Otherwise the page numbers may take divergent values
depending on which part is compiled.

For example, a title page could be declared by:
%
\begin{center}
\begin{tabular}{l}
|\ifchilddoc\||else|\\
|\addtocounter{page}{-1}|\\
\textit{code for title page}\\
|\newpage|\\
|\||fi|
\end{tabular}
\end{center}
%
A banner page for the child documents can be generated by:
%
\begin{center}
\begin{tabular}{l}
|\ifchilddoc|\\
|\addtocounter{page}{-1}|\\
\textit{code for banner page}\\
|\newpage|\\
|\||fi|
\end{tabular}
\end{center}
%
Here one could write a message such as:
\begin{center}
|This is the part \childdocname{} of \childdocjob{}.|
\end{center}

%%%%%%%%%%%%%%%%%%%%%%%%%%%%%%%%%%%%%%%%%%%%%%%%%%%%%%%%%%%%%%%%%%%%%%%%%%%%%%%%
\subsection{Flags}
\label{sec:flags}

The package makes it easy to generate different versions
of the main or child documents.
To this end compilation flags can be defined
and assigned different default values.
They will be particularly useful in conjunction
with the forwarding mechanism described in \secref{sec:forward}.

For example, it may be useful to have a flag |\version|
which can be set to |draft| or |final|.
The document source will contain some conditional code
depending on the value of |\version|.
Suppose further, the flag should default to |final| for the main file
and to |draft| for child files
which is a natural assignment for editing the document.
This is achieved by placing the following code
in the preamble of the main document
(below the |\childdocmain| directive):
%
\begin{center}
\begin{tabular}{l}
|\ifchilddoc|\\
|\providecommand{\version}{draft}|\\
|\||else|\\
|\providecommand{\version}{final}|\\
|\||fi|
\end{tabular}
\end{center}
%
The definition by |\providecommand| makes sure
that previous definitions are not overwritten.
Further statements |\providecommand{\version}{...}|
can thus be added before the above code to override it.

For the main file, one might add a line
(between |\childdocmain| and the above block)
%
\begin{center}
|%\ifchilddoc\||else\providecommand{\version}{draft}\||fi|
\end{center}
%
which can be uncommented to produce a draft version.
Likewise one can add a line to the very top of a child file
(above the |\childdocof{|\textit{main}|}| directive)
%
\begin{center}
|%\providecommand{\version}{final}|
\end{center}
%
which can be uncommented to produce the final version of this child document.

%%%%%%%%%%%%%%%%%%%%%%%%%%%%%%%%%%%%%%%%%%%%%%%%%%%%%%%%%%%%%%%%%%%%%%%%%%%%%%%%
\subsection{Forwarding}
\label{sec:forward}

Different versions of the main or child documents
using compilation flags as described in \secref{sec:flags}
can be (permanently) stored in different files
for convenient compilation, viewing and distribution.
To this end, the package defines a command
to pass on compilation to a different file:

%%%%%%%%%%%%%%%%%%%%%%%%%%%%%%%%%%%%%%%%
\DescribeMacro{\childdocforward}
The command |\childdocforward| redirects processing to
another source file:
%
\begin{center}
\begin{tabular}{l}
|\input{childdoc.def}|\\
|\childdocforward[|\textit{main}|]{|\textit{dest}|}|\\
\end{tabular}
\end{center}
%
The argument \textit{dest} is the destination file
(without extension).
It should be the main file or one of the child files.
Note that further \textsf{childdoc} directives
such as |\childdocof| and |\childdocforward|
in the indicated file will be processed in this form.
The optional argument \textit{main}
passes on directly to the main file \textit{main}
while pretending to compile the child \textit{dest}.
This form behaves as if \textit{dest}
issues |\childdocof{|\textit{main}|}| right away,
and no further \textsf{childdoc} directives will be processed.

%%%%%%%%%%%%%%%%%%%%%%%%%%%%%%%%%%%%%%%%
\DescribeMacro{\...prefix}
In the alternative form |\childdocforwardprefix|,
%
\begin{center}
\begin{tabular}{l}
|\input{childdoc.def}|\\
|\childdocforwardprefix[|\textit{main}|]{|\textit{prefix}|}{|\textit{dest}|}|
\end{tabular}
\end{center}
%
the destination file is determined by a pattern
depending on the current file:
To make this work, the current file must be called
`{\textit{prefix}\hspace{0.2em}\textit{suffix}}'
with \textit{prefix} matching precisely the argument.
Processing is then passed on to the file
`{\textit{dest}\hspace{0.2em}\textit{suffix}}'.
Surely, the same effect is achieved by
directly specifying the
argument `{\textit{dest}\hspace{0.2em}\textit{suffix}}'
in the first form.
However, that requires to set up a different file
for each child. With the alternative form of the command
all these files can have exactly the same content
which simplifies setting them up and maintaining them.

For example, the following file |draft.tex|
with a compilation flag |\version| as described in \secref{sec:flags}
compiles the main document as a draft:
%
\begin{center}
\begin{tabular}{l}
|\def\version{draft}|\\
|\input{childdoc.def}|\\
|\childdocforward{|\textit{main}|}|
\end{tabular}
\end{center}
%
Likewise, the following files |final|\textit{nn}|.tex|
compile the final version of the child document
|child|\textit{nn}|.tex|:
%
\begin{center}
\begin{tabular}{l}
|\def\version{final}|\\
|\input{childdoc.def}|\\
|\childdocforwardprefix{final}{child}|
\end{tabular}
\end{center}
%

Note that when several versions of a main file and/or of each child file
are to be generated, it may be convenient to set up a |Makefile| or
shell script to automatise the process.

%%%%%%%%%%%%%%%%%%%%%%%%%%%%%%%%%%%%%%%%%%%%%%%%%%%%%%%%%%%%%%%%%%%%%%%%%%%%%%%%
\subsection{Command Line Processing}
\label{sec:commandline}

The effect of redirection files can also be achieved by invoking
the \LaTeX{} compiler with a more elaborate command line.
Most conveniently this should be done as part
of a shell script or a |Makefile|.

When using \textsf{childdoc} in the main file, the following
command lines effectively perform a redirection
(note that depending on the shell being used,
backslashes may have to be doubled: `|\|' $\to$ `|\\|'):
%
\begin{center}
|... -jobname "|\textit{target}|" |\\|"|[\textit{flags}]%
|\input{childdoc.def}\childdocforward[|\textit{main}|]{|\textit{dest}|}"|
\end{center}
%
Here \textit{target} is the name of the output file,
\textit{main} is the name of the main file
and \textit{dest} is the name of the main or child file to be processed
(all filenames without extensions).
The optional argument \textit{main} can be omitted
if \textit{main} matches \textit{dest}.
Optionally, compilation \textit{flags} can be defined via |\def| commands.
This command line makes the \TeX{} engine believe
it is compiling the file \textit{target}
whose content is specified as the latter parameter.
The provided code then forwards the processing to
\textit{main} or \textit{dest} as described in \secref{sec:forward}.

%%%%%%%%%%%%%%%%%%%%%%%%%%%%%%%%%%%%%%%%%%%%%%%%%%%%%%%%%%%%%%%%%%%%%%%%%%%%%%%%
\subsection{Include by Input}
\label{sec:input}

Including child documents by |\include| has some restrictions by design.
Most notably, the content of a child document always occupies
its own set of pages; pages cannot be shared between child documents.
Usually, this behaviour makes perfect sense
because each child document contain an essential part of the document.
However, in some situations it may be desirable to compose
a document from a collection of parts
without having mandatory page breaks between then.
For this case, the package
provides a mechanism to include parts
by |\input| which can also be processed individually.
However, by construction this mechanism
requires manual handling of the content to be output.

%%%%%%%%%%%%%%%%%%%%%%%%%%%%%%%%%%%%%%%%
\DescribeMacro{\ifchilddocmanual}
The main file should be prepared as usual, see \secref{sec:include}.
However, the document body must make a distinction
between processing of an individual part and of the main document, e.g.:
%
\begin{center}
\begin{tabular}{l}
|\ifchilddocmanual|\\
|\input{\childdocname}|\\
|\||else|\\
\textit{document body with }|\input{|\textit{part}|}|\\
|\||fi|
\end{tabular}
\end{center}
%
The conditional |\ifchilddocmanual| is true whenever
a part to be included by |\input| is being compiled,
and the name of the part is stored in |\childdocname|.

%%%%%%%%%%%%%%%%%%%%%%%%%%%%%%%%%%%%%%%%
\DescribeMacro{\childdocby}
Each part to be included by |\input| should start with:
%
\begin{center}
\begin{tabular}{l}
|\input{childdoc.def}|\\
|\childdocby{|\textit{main}|}|\\
\end{tabular}
\end{center}
%
The directive |\childdocby| is similar to |\childdocof|
described in \secref{sec:include},
but the subsequent selection of content must be done manually.
To that end, both |\ifchilddoc| and |\ifchilddocmanual|
will be true upon processing of a part,
and the name of the part is stored in |\childdocname|.
Note that |\jobname| will be set to the filename of the current part
so that each part receives an individual |.aux| file
that does not interfere with the |.aux| file(s) of the main document.
This behaviour can be altered by the alternative form
|\childdocby[*]{|\textit{main}|}| (with a non-empty optional argument)
which uses the |.aux| file of the main document
by setting |\jobname| to \textit{main}.

%%%%%%%%%%%%%%%%%%%%%%%%%%%%%%%%%%%%%%%%%%%%%%%%%%%%%%%%%%%%%%%%%%%%%%%%%%%%%%%%
\subsection{Driver Development}
\label{sec:driver}

The \textsf{childdoc} mechanism can also be use for the development
of definition files such as \LaTeX{} styles or classes.
This case differs from the above setup with multiple parts
included by |\include| in that no |\includeonly| should be invoked.
This can be achieved by starting the include file
(before |\ProvidesPackage|) with:
%
\begin{center}
\begin{tabular}{l}
|\input{childdoc.def}|\\
|\childdocforward{|\textit{main}|}|\\
\end{tabular}
\end{center}
%
or alternatively with:
%
\begin{center}
\begin{tabular}{l}
|\input{childdoc.def}|\\
|\childdocby{|\textit{main}|}|\\
\end{tabular}
\end{center}
%
Both forms have slightly different effects as described above.
The main file is prepared as usual, see \secref{sec:include}.

%%%%%%%%%%%%%%%%%%%%%%%%%%%%%%%%%%%%%%%%%%%%%%%%%%%%%%%%%%%%%%%%%%%%%%%%%%%%%%%%
\subsection{Legacy Detection}
\label{sec:detection}

The directive |\childdocmain| in the main file can detect
whether the complete document or merely a child is to be compiled
even without using the directive |\childdocof|.
This method is deprecated because it is less robust
and there is no compelling reason to use it;
it is merely provided for backward compatibility
and it may be removed in future versions.

If the detection mechanism is to be used,
it is mandatory to correctly specify
the filename of the main file as the argument of |\childdocmain|:
%
\begin{center}
\begin{tabular}{l}
|\input{childdoc.def}|\\
|\childdocmain{|\textit{main}|}|\\
\end{tabular}
\end{center}
%
If |\jobname| does not match the argument \textit{main} of |\childdocmain|,
it is assumed that |\jobname| points to the child file to be compiled.
When using |\childdocmain| with the main file specified as argument,
it suffices to start a child file
with just |\input{|\textit{main}|}|
without loading of the package and using |\childdocof|.
If instead all processing is done
with the appropriate \textsf{childdoc} directives,
the argument of \textit{main} of |\childdocmain| can be empty.

An alternative version of the command line processing described
in \secref{sec:commandline} using the detection mechanism reads:
%
\begin{center}
|... -jobname "|\textit{target}|" "|[\textit{flags}]%
[|\def\jobname{|\textit{dest}|}|]|\input{|\textit{main}|}"|
\end{center}

%%%%%%%%%%%%%%%%%%%%%%%%%%%%%%%%%%%%%%%%%%%%%%%%%%%%%%%%%%%%%%%%%%%%%%%%%%%%%%%%
\subsection{Manual Code}
\label{sec:manual}

In case one cannot be certain whether the definitions file |childdoc.def|
is installed on the target \TeX{} distribution
and one prefers not to ship it,
it is conceivable to paste a few relevant commands into the sources.

To that end, drop all statements |\input{childdoc.def}|
and perform the replacements as outlined below.
Instead of |\childdocmain{|\textit{main}|}| add the following code
to the top of the main file:
%
\begin{center}
\begin{tabular}{l}
|\||ifdefined\childdocname\endinput\||fi\newif\ifchilddoc|\\
|\edef\childdocname{\scantokens\expandafter{\jobname\noexpand}}|\\
|\def\childdocmain{|\textit{main}|}\||ifx\childdocmain\childdocname\||else|\\
|\childdoctrue\includeonly{\childdocname}\let\jobname\childdocmain\||fi|\\
\end{tabular}
\end{center}
%
Instead of |\childdocof{|\textit{main}|}| just include the main file
at the top of each child file:
%
\begin{center}
|\input{|\textit{main}|}|
\end{center}
%
A simple redirection |\childdocforward{|\textit{dest}|}| is achieved by:
%
\begin{center}
|\def\jobname{|\textit{dest}|}\input{\jobname}|
\end{center}
%
The redirection with prefix
|\childdocforwardprefix[|\textit{prefix}|]{|\textit{dest}|}|
is accomplished by:
%
\begin{center}
\begin{tabular}{l}
|{\edef\jobname{\scantokens\expandafter{\jobname\noexpand}}|\\
|\def\redirectjob |\textit{prefix}|#1~~~{\gdef\jobname{|\textit{dest}|#1}}|\\
|\expandafter\redirectjob\jobname~~~}\input{\jobname}|
\end{tabular}
\end{center}

In an alternative approach,
child documents can be compiled by a specific command line
without additional code or specific definitions:
%
\begin{center}
|... -jobname "|\textit{target}|" "|[\textit{flags}]%
|\includeonly{|\textit{dest}|}\input{|\textit{main}|}"|
\end{center}
%

%%%%%%%%%%%%%%%%%%%%%%%%%%%%%%%%%%%%%%%%%%%%%%%%%%%%%%%%%%%%%%%%%%%%%%%%%%%%%%%%
%%%%%%%%%%%%%%%%%%%%%%%%%%%%%%%%%%%%%%%%%%%%%%%%%%%%%%%%%%%%%%%%%%%%%%%%%%%%%%%%
\section{Information}

%%%%%%%%%%%%%%%%%%%%%%%%%%%%%%%%%%%%%%%%%%%%%%%%%%%%%%%%%%%%%%%%%%%%%%%%%%%%%%%%
\subsection{Copyright}

Copyright \copyright{} 2017--2018 Niklas Beisert

This work may be distributed and/or modified under the
conditions of the \LaTeX{} Project Public License, either version 1.3
of this license or (at your option) any later version.
The latest version of this license is in
  \url{http://www.latex-project.org/lppl.txt}
and version 1.3 or later is part of all distributions of \LaTeX{}
version 2005/12/01 or later.

This work has the LPPL maintenance status `maintained'.

The Current Maintainer of this work is Niklas Beisert.

This work consists of the files |README.txt|, |childdoc.ins| and |childdoc.dtx|
as well as the derived files |childdoc.def|, |cdocsamp.tex|
with |cdocsch1.tex|, |cdocsch2.tex|, |cdocspt3.tex|, |cdocspt4.tex|,
|cdocsdrf.tex|, |cdocsfn1.tex|, |cdocsfn2.tex|
as well as |childdoc.pdf|.

%%%%%%%%%%%%%%%%%%%%%%%%%%%%%%%%%%%%%%%%%%%%%%%%%%%%%%%%%%%%%%%%%%%%%%%%%%%%%%%%
\subsection{Files and Installation}

The package consists of the files:
%
\begin{center}
\begin{tabular}{ll}
    |README.txt|   & readme file \\
    |childdoc.ins| & installation file \\
    |childdoc.dtx| & source file \\
    |childdoc.def| & definition file \\
    |cdocsamp.tex| & sample main file \\
    |cdocsch1.tex| & sample include file \\
    |cdocsch2.tex| & sample include file \\
    |cdocspt3.tex| & sample part file \\
    |cdocspt4.tex| & sample part file \\
    |cdocsdrf.tex| & sample redirection file \\
    |cdocsfn1.tex| & sample redirection file \\
    |cdocsfn2.tex| & sample redirection file \\
    |childdoc.pdf| & manual
\end{tabular}
\end{center}
%
The distribution consists of the files
|README.txt|, |childdoc.ins| and |childdoc.dtx|.
%
\begin{itemize}
\item
Run (pdf)\LaTeX{} on |childdoc.dtx|
to compile the manual |childdoc.pdf| (this file).
\item
Run \LaTeX{} on |childdoc.ins| to create the definitions file |childdoc.def|
and the sample |cdocsamp.tex| with include files
|cdocsch1.tex|, |cdocsch2.tex|, |cdocspt3.tex|, |cdocspt4.tex|,
|cdocsdrf.tex|, |cdocsfn1.tex|, |cdocsfn2.tex|.
Then copy the file |childdoc.def| to an appropriate directory of your \LaTeX{}
distribution, e.g.\ \textit{texmf-root}|/tex/latex/childdoc|.
\end{itemize}

%%%%%%%%%%%%%%%%%%%%%%%%%%%%%%%%%%%%%%%%%%%%%%%%%%%%%%%%%%%%%%%%%%%%%%%%%%%%%%%%
\subsection{Related CTAN Packages}

There are several other packages which offer a similar functionality:
%
\begin{itemize}
\item
The packages
\href{http://ctan.org/pkg/docmute}{\textsf{docmute}},
\href{http://ctan.org/pkg/includex}{\textsf{includex}} and
\href{http://ctan.org/pkg/standalone}{\textsf{standalone}}
provide commands to include only the document body of
a child file thus allowing both files to be compiled individually.
\item
The packages \href{http://ctan.org/pkg/subdocs}{\textsf{subdocs}}
and \href{http://ctan.org/pkg/subfiles}{\textsf{subfiles}}
provide structures in which the main and child documents can be
encapsulated and allowing them to be compiled individually.
The inclusion mechanism is different from the conventional |\include|.
\item
The package \href{http://ctan.org/pkg/combine}{\textsf{combine}}
is an elaborate solution to combine several documents into one.
\end{itemize}
%
See also the CTAN topic \href{http://ctan.org/topic/subdocs}{\textsf{subdocs}}
for further related packages.
The present package differs from the above solutions in that
a document structure constructed with the conventional |\include| mechanism
just needs two extra commands at the top of every file
such that all constituent files can be compiled individually.

%%%%%%%%%%%%%%%%%%%%%%%%%%%%%%%%%%%%%%%%%%%%%%%%%%%%%%%%%%%%%%%%%%%%%%%%%%%%%%%%
%\subsection{Feature Suggestions}
%
%The following is a list of features which may be useful for future
%versions of this package:
%%
%\begin{itemize}
%\item
%\ldots
%\end{itemize}

%%%%%%%%%%%%%%%%%%%%%%%%%%%%%%%%%%%%%%%%%%%%%%%%%%%%%%%%%%%%%%%%%%%%%%%%%%%%%%%%
\subsection{Revision History}

%%%%%%%%%%%%%%%%%%%%%%%%%%%%%%%%%%%%%%%%
\paragraph{v2.0:} 2018/12/30

\begin{itemize}
\item
immediate forward processing
\item
added |\childdocby| mechanism
\item
manual restructured
\end{itemize}

%%%%%%%%%%%%%%%%%%%%%%%%%%%%%%%%%%%%%%%%
\paragraph{v1.6:} 2018/01/17

\begin{itemize}
\item
application for development of include files
\item
corrections to manual
\end{itemize}

%%%%%%%%%%%%%%%%%%%%%%%%%%%%%%%%%%%%%%%%
\paragraph{v1.5:} 2017/05/21

\begin{itemize}
\item
more complete structuring introduced
\item
|\childdocof| introduced
\item
|\childdoc| renamed to |\childdocmain|
\item
|\childredirect| renamed to |\childdocforward| and |\childdocforwardprefix|
and functionality expanded
\end{itemize}

%%%%%%%%%%%%%%%%%%%%%%%%%%%%%%%%%%%%%%%%
\paragraph{v1.0:} 2017/04/27

\begin{itemize}
\item
manual and install package
\item
first version published on CTAN
\end{itemize}

%%%%%%%%%%%%%%%%%%%%%%%%%%%%%%%%%%%%%%%%
\paragraph{v0.6:} 2017/04/26

\begin{itemize}
\item
redirection mechanism added
\end{itemize}

%%%%%%%%%%%%%%%%%%%%%%%%%%%%%%%%%%%%%%%%
\paragraph{v0.5:} 2017/04/26

\begin{itemize}
\item
functionality in definition file
\end{itemize}


%%%%%%%%%%%%%%%%%%%%%%%%%%%%%%%%%%%%%%%%%%%%%%%%%%%%%%%%%%%%%%%%%%%%%%%%%%%%%%%%
%%%%%%%%%%%%%%%%%%%%%%%%%%%%%%%%%%%%%%%%%%%%%%%%%%%%%%%%%%%%%%%%%%%%%%%%%%%%%%%%
%%%%%%%%%%%%%%%%%%%%%%%%%%%%%%%%%%%%%%%%%%%%%%%%%%%%%%%%%%%%%%%%%%%%%%%%%%%%%%%%
\appendix

\settowidth\MacroIndent{\rmfamily\scriptsize 000\ }

 \DocInput{childdoc.dtx}

\end{document}
%</driver>
% \fi
%
% %%%%%%%%%%%%%%%%%%%%%%%%%%%%%%%%%%%%%%%%%%%%%%%%%%%%%%%%%%%%%%%%%%%%%%%%%%%%%%
% %%%%%%%%%%%%%%%%%%%%%%%%%%%%%%%%%%%%%%%%%%%%%%%%%%%%%%%%%%%%%%%%%%%%%%%%%%%%%%
% \section{Sample}
%\iffalse
%<*samplemain>
%\fi
%
% The following presents a sample document
% with two chapters, two parts, a title page,
% a compile flag as well as three forwarding files to set the flag.
% It consists of eight |.tex| files:
% \begin{center}
% \begin{tabular}{ll}
% |cdocsamp.tex|&main file\\
% |cdocsch1.tex|&include file for chapter 1\\
% |cdocsch2.tex|&include file for chapter 2\\
% |cdocspt3.tex|&include file for part 3\\
% |cdocspt4.tex|&include file for part 4\\
% |cdocsdrf.tex|&forwarding file for main file in draft mode\\
% |cdocsfi1.tex|&forwarding file for final version of chapter 1\\
% |cdocsfi2.tex|&forwarding file for final version of chapter 2\\
% \end{tabular}
% \end{center}
% Each of the eight files can be compiled directly by the \LaTeX{} compiler.
%
% %%%%%%%%%%%%%%%%%%%%%%%%%%%%%%%%%%%%%%
% \paragraph{Main File.}
%
% The main file is called |cdocsamp.tex|.
%
% Load the \textsf{childdoc} definitions and
% declare the filename for the main document:
%    \begin{macrocode}
\input{childdoc.def}
\childdocmain{}
%    \end{macrocode}

% Optional override for |\version| flag:
%    \begin{macrocode}
%%\ifchilddoc\else\providecommand{\version}{draft}\fi
%    \end{macrocode}

% Define the default values for the |\version| flag
% (|final| for the main file and |draft| for childs):
%    \begin{macrocode}
\ifchilddoc
\providecommand{\version}{draft}
\else
\providecommand{\version}{final}
\fi
%    \end{macrocode}

% Load the standard document class:
%    \begin{macrocode}
\documentclass[12pt]{article}
%    \end{macrocode}

% Start the document body:
%    \begin{macrocode}
\begin{document}
%    \end{macrocode}

% Declare a title page.
% Print title, part of document being processed and version flag:
%    \begin{macrocode}
\addtocounter{page}{-1}
\begin{center}
{\LARGE\bfseries{}childdoc example\par}
\vspace{1cm}
\ifchilddoc
\ifchilddocmanual part\else chapter\fi:
`\childdocname' of `\childdocjob'\par
\else
main document: `\childdocjob'\par
\fi
version: \version\par
\end{center}
\newpage
%    \end{macrocode}

% Manually include selected file,
% otherwise process as usual:
%    \begin{macrocode}
\ifchilddocmanual
\section*{part `\childdocname'}
\input{\childdocname}
\else
%    \end{macrocode}

% Include the two chapters:
%    \begin{macrocode}
\include{cdocsch1}
\include{cdocsch2}
%    \end{macrocode}

% Include the two parts unless only chapters should be displayed:
%    \begin{macrocode}
\ifchilddoc\else
\section{part three}
\input{cdocspt3}
\section{part four}
\input{cdocspt4}
\fi
%    \end{macrocode}

% Process as usual until here:
%    \begin{macrocode}
\fi
%    \end{macrocode}

% End of document body:
%    \begin{macrocode}
\end{document}
%    \end{macrocode}
%\iffalse
%</samplemain>
%\fi
%
% %%%%%%%%%%%%%%%%%%%%%%%%%%%%%%%%%%%%%%
% \paragraph{Chapter Include Files.}
%
% The include files are called |cdocsch1.tex| and |cdocsch2.tex|.
%
%\iffalse
%<*samplechap1|samplechap2>
%\fi

% Optional override for |\version| flag:
%    \begin{macrocode}
%%\providecommand{\version}{final}
%    \end{macrocode}

% Include the main document:
%    \begin{macrocode}
\input{childdoc.def}
\childdocof{cdocsamp}
%    \end{macrocode}

%\iffalse
%</samplechap1|samplechap2>
%\fi
%
%\iffalse
%<*samplechap1>
%\fi
% Some text for chapter 1:
%    \begin{macrocode}
\section{one}
some text in chapter one
%    \end{macrocode}

%\iffalse
%</samplechap1>
%\fi
% Some text for chapter 2:
%\iffalse
%<*samplechap2>
%\fi
%    \begin{macrocode}
\section{two}
more text in chapter two
%    \end{macrocode}

%\iffalse
%</samplechap2>
%\fi
%
% %%%%%%%%%%%%%%%%%%%%%%%%%%%%%%%%%%%%%%
% \paragraph{Part Include Files.}
%
% The include files are called |cdocspt3.tex| and |cdocspt4.tex|.
%
%\iffalse
%<*samplepart3|samplepart4>
%\fi

% Optional override for |\version| flag:
%    \begin{macrocode}
%%\providecommand{\version}{final}
%    \end{macrocode}

% Include the main document:
%    \begin{macrocode}
\input{childdoc.def}
\childdocby{cdocsamp}
%    \end{macrocode}

%\iffalse
%</samplepart3|samplepart4>
%\fi
%
%\iffalse
%<*samplepart3>
%\fi
% Some text for part 3:
%    \begin{macrocode}
some text in part three
%    \end{macrocode}

%\iffalse
%</samplepart3>
%\fi
% Some text for part 4:
%\iffalse
%<*samplepart4>
%\fi
%    \begin{macrocode}
more text in part four
%    \end{macrocode}

%\iffalse
%</samplepart4>
%\fi
%
% %%%%%%%%%%%%%%%%%%%%%%%%%%%%%%%%%%%%%%
% \paragraph{Forwarding for a Complete Draft.}
%
% The following forwarding file |cdocsdrf.tex|
% compiles the main document in draft mode:
%\iffalse
%<*sampledraft>
%\fi
%    \begin{macrocode}
\def\version{draft}
\input{childdoc.def}
\childdocforward{cdocsamp}
%    \end{macrocode}

%\iffalse
%</sampledraft>
%\fi
%
% %%%%%%%%%%%%%%%%%%%%%%%%%%%%%%%%%%%%%%
% \paragraph{Forwarding for Final Version of the Chapters.}
%
% The following forwarding files |cdocsfn1.tex| and |cdocsfn2.tex|
% (with identical content)
% compile the final versions of the child documents
% |cdocsch1.tex| and |cdocsch2.tex|, respectively:
%\iffalse
%<*samplefinal>
%\fi
%    \begin{macrocode}
\def\version{final}
\input{childdoc.def}
\childdocforwardprefix[cdocsamp]{cdocsfn}{cdocsch}
%    \end{macrocode}

%\iffalse
%</samplefinal>
%\fi
%
% %%%%%%%%%%%%%%%%%%%%%%%%%%%%%%%%%%%%%%
% \paragraph{Command Line Processing.}
%
% The following three command lines generate the output files
% |cdocscld|, |cdocscl1| and |cdocscl2|
% which should be identical to
% |cdocsdrf|, |cdocsch1| and |cdocsfn2|, respectively:
% \begin{center}
% \begin{tabular}{l}
% |latex -jobname cdocscld \|\\
% |  "\def\version{draft}\input{childdoc.def}\childdocforward{cdocsamp}"|\\
% |latex -jobname cdocscl1 \|\\
% |  "\input{childdoc.def}\childdocforward[cdocsamp]{cdocsch1}"|\\
% |latex -jobname cdocscl2 \|\\
% |  "\def\version{final}\input{childdoc.def}\childdocforward{cdocsch2}"|
% \end{tabular}
% \end{center}
% Note that the trailing backslash on each first line
% merely continues the input to the second line
% (for convenient cut ant paste).
% Furthermore, the command |latex| can be replaced by any
% of its alternative versions such as |pdflatex|.
%
% %%%%%%%%%%%%%%%%%%%%%%%%%%%%%%%%%%%%%%%%%%%%%%%%%%%%%%%%%%%%%%%%%%%%%%%%%%%%%%
% %%%%%%%%%%%%%%%%%%%%%%%%%%%%%%%%%%%%%%%%%%%%%%%%%%%%%%%%%%%%%%%%%%%%%%%%%%%%%%
% \section{Implementation}
%\iffalse
%<*package>
%\fi
%
% This section describes the definitions file |childdoc.def|.

% The definitions cannot be loaded using |\usepackage| or |\RequirePackage|
% which has a mechanism to prevent loading a style file more than once.
% When loading the definitions by means of |\input|
% multiple instances have to be prevented manually:
%\iffalse
%This code needs to be before the `\ProvidesFile' directive
%which is defined at the beginning of this file.
%Therefore it is also placed there and commented out here.
%</package>
%<*discard>
%\fi
%    \begin{macrocode}
\ifdefined\childdocmain\endinput\fi
%    \end{macrocode}
%\iffalse
%</discard>
%<*package>
%\fi
%
% \macro{\ifchilddoc}
% \macro{\ifchilddocmanual}
% The conditional |\ifchilddoc| tells whether a
% child (true) or main (false) document is being compiled.
% The conditional |\ifchilddocmanual| tells whether
% the |\includeonly| mechanism is used (false) or
% the selection of child files must be performed manually (true).
% The definitions initialise to false:
%    \begin{macrocode}
\newif\ifchilddoc
\newif\ifchilddocmanual
%    \end{macrocode}

% \macro{\childdocname}
% \macro{\childdocjob}
% The macro |\childdocname| stores the name of the main document
% to be compiled. The macro |\childdocjob| stores the name of
% the document on which the \LaTeX{} compiler was originally invoked.
% The content of |\jobname| cannot be compared
% to filenames specified in the source due to different catcodes.
% The following code rescans |\jobname|, stores the result
% in |\childdocname| and saves a copy in |\childdocjob|:
%    \begin{macrocode}
\edef\childdocname{\scantokens\expandafter{\jobname\noexpand}}
\let\childdocjob\childdocname
%    \end{macrocode}

% \macro{\childdocdisable}
% The macro |\childdocdisable| prevents the main file
% from being processed more than once.
% At this stage, the main document command |\childdocmain|
% is assumed to be called once again where it should do nothing.
% Any subsequent call to it should prevent
% a secondary processing of the main document
% It overwrites the forwarding commands
% |\childdocof| and |\childdocforward|
% with empty macros to prevent further inclusions of the main document:
%    \begin{macrocode}
\newcommand{\childdocdisable}
{
  \renewcommand{\childdocmain}[1]{\renewcommand{\childdocmain}[1]{\endinput}}
  \renewcommand{\childdocof}[1]{}
  \renewcommand{\childdocby}[2][]{}
  \renewcommand{\childdocforward}[2][]{}
  \renewcommand{\childdocdisable}{}
}
%    \end{macrocode}

% \macro{\childdocmain}
% The macro |\childdocmain| is to be called at the top of the main file
% with nothing or the main filename (without extension) as argument.
% First, it breaks loops.
% If the argument is not empty and does not match |\childdocname|
% (which is set by the first inclusion of |childdoc.def|),
% |\ifchilddoc| is set to true, |\includeonly| is applied to the child file
% and |\jobname| is set to the main file
% (for proper handling of |.aux| files):
%    \begin{macrocode}
\newcommand{\childdocmain}[1]
{
  \childdocdisable\childdocmain{}
  \if?#1?\else
    \begingroup
      \def\childdoctmp{#1}
      \ifx\childdoctmp\childdocname
        \def\childdoctmp{}
      \else
        \def\childdoctmp
        {
          \childdoctrue
          \includeonly{\childdocname}
          \def\childdocjob{#1}
          \def\jobname{#1}
        }
      \fi
      \expandafter
    \endgroup
    \childdoctmp
  \fi
}
%    \end{macrocode}

% \macro{\childdocof}
% The command |\childdocof| redirects
% compilation to the main file |#1|.
%    \begin{macrocode}
\newcommand{\childdocof}[1]
{
  \childdocdisable
  \childdoctrue
  \includeonly{\childdocname}
  \def\jobname{#1}
  \def\childdocjob{#1}
  \input{#1}
}
%    \end{macrocode}

% \macro{\childdocby}
% The command |\childdocby| ....
%    \begin{macrocode}
\newcommand{\childdocby}[2][]
{
  \childdocdisable
  \childdoctrue
  \childdocmanualtrue
  \if?#1?\else
    \def\jobname{#2}
  \fi
  \def\childdocjob{#2}
  \input{#2}
  \endinput
}
%    \end{macrocode}

% \macro{\childdocforward}
% The command |\childdocforward| redirects
% compilation to the main file or
% (if the optional argument is given) a child file.
% Parameters are set as if the main file
% or a child file starting with |\childdocof| was compiled.
% Then compilation is handed over to the main file:
%    \begin{macrocode}
\newcommand{\childdocforward}[2][]
{
  \begingroup
    \if?#1?
      \def\childdoctmp
      {
        \def\childdocname{#2}
        \def\childdocjob{#2}
        \def\jobname{#2}
        \input{#2}
        \endinput
      }
    \else
      \def\childdoctmp
      {
        \childdocdisable
        \def\childdocname{#2}
        \childdoctrue
        \includeonly{#2}
        \def\childdocjob{#1}
        \def\jobname{#1}
        \input{#1}
        \endinput
      }
    \fi
    \expandafter
  \endgroup
  \childdoctmp
}
%    \end{macrocode}

% \macro{\childdocforwardprefix}
% The command |\childdocforwardprefix| redirects
% compilation to the main or a child file by means of a pattern.
% The prefix |#1| in the current filename is replaced by |#2|
% and the suffix of the current filename is kept
% (it is assumed that the filename does not contain the substring `|~~~|'
% which is used as a delimiter).
% Compilation is handed over to the new file by |\childdocforward|:
%    \begin{macrocode}
\newcommand{\childdocforwardprefix}[3][]
{
  \begingroup
    \def\childdocextract #2##1~~~{\def\childdoctmp{\childdocforward[#1]{#3##1}}}
    \expandafter\childdocextract\childdocname~~~
    \expandafter
  \endgroup
  \childdoctmp
}
%    \end{macrocode}

% \macro{\childdoc}
% The deprecated macro |\childdoc| is a legacy version of |\childdocmain|:
%    \begin{macrocode}
\newcommand{\childdoc}{\childdocmain}
%    \end{macrocode}

% \macro{\childdocredirect}
% The deprecated macro |\childdocredirect| is a legacy version
% of |\childdocforward| and |\childdocforwardprefix|:
%    \begin{macrocode}
\newcommand{\childdocredirect}[2][]
{
  \begingroup
    \if?#1?
      \def\childdoctmp{\childdocforward{#2}}
    \else
      \def\childdoctmp{\childdocforwardprefix{#1}{#2}}
    \fi
    \expandafter
  \endgroup
  \childdoctmp
}
%    \end{macrocode}

%\iffalse
%</package>
%\fi
%
\endinput

\childdocby{cdocsamp}
%    \end{macrocode}

%\iffalse
%</samplepart3|samplepart4>
%\fi
%
%\iffalse
%<*samplepart3>
%\fi
% Some text for part 3:
%    \begin{macrocode}
some text in part three
%    \end{macrocode}

%\iffalse
%</samplepart3>
%\fi
% Some text for part 4:
%\iffalse
%<*samplepart4>
%\fi
%    \begin{macrocode}
more text in part four
%    \end{macrocode}

%\iffalse
%</samplepart4>
%\fi
%
% %%%%%%%%%%%%%%%%%%%%%%%%%%%%%%%%%%%%%%
% \paragraph{Forwarding for a Complete Draft.}
%
% The following forwarding file |cdocsdrf.tex|
% compiles the main document in draft mode:
%\iffalse
%<*sampledraft>
%\fi
%    \begin{macrocode}
\def\version{draft}
% \iffalse
%
% childdoc.dtx Copyright (C) 2017-2018 Niklas Beisert
%
% This work may be distributed and/or modified under the
% conditions of the LaTeX Project Public License, either version 1.3
% of this license or (at your option) any later version.
% The latest version of this license is in
%   http://www.latex-project.org/lppl.txt
% and version 1.3 or later is part of all distributions of LaTeX
% version 2005/12/01 or later.
%
% This work has the LPPL maintenance status `maintained'.
%
% The Current Maintainer of this work is Niklas Beisert.
%
% This work consists of the files childdoc.dtx and childdoc.ins
% and the derived files childdoc.def and cdocsamp.tex with
% cdocsch1.tex, cdocsch2.tex, cdocsdrf.tex, cdocsfn1.tex, cdocsfn2.tex.
%
%<package>\ifdefined\childdocmain\endinput\fi
%<package>\ProvidesFile{childdoc.def}[2018/12/30 v2.0 child document driver]
%<samplemain>\ProvidesFile{cdocsamp.tex}[2018/12/30 v2.0 sample for childdoc]
%<*driver>
%\ProvidesFile{childdoc.drv}[2018/12/30 v2.0 childdoc reference manual file]
\PassOptionsToClass{10pt,a4paper}{article}
\documentclass{ltxdoc}

\usepackage[margin=35mm]{geometry}
\usepackage{hyperref}
\usepackage{hyperxmp}
\usepackage[usenames]{color}

\hypersetup{colorlinks=true}
\hypersetup{pdfstartview=FitH}
\hypersetup{pdfpagemode=UseNone}
\hypersetup{pdfsource={}}
\hypersetup{pdflang={en-UK}}
\hypersetup{pdfcopyright={Copyright 2017-2018 Niklas Beisert.
  This work may be distributed and/or modified under the
  conditions of the LaTeX Project Public License, either version 1.3
  of this license or (at your option) any later version.}}
\hypersetup{pdflicenseurl={http://www.latex-project.org/lppl.txt}}
\hypersetup{pdfcontactaddress={ETH Zurich, ITP, HIT K,
  Wolfgang-Pauli-Strasse 27}}
\hypersetup{pdfcontactpostcode={8093}}
\hypersetup{pdfcontactcity={Zurich}}
\hypersetup{pdfcontactcountry={Switzerland}}
\hypersetup{pdfcontactemail={nbeisert@itp.phys.ethz.ch}}
\hypersetup{pdfcontacturl={http://people.phys.ethz.ch/\xmptilde nbeisert/}}

\newcommand{\secref}[1]{\hyperref[#1]{section \ref*{#1}}}

\parskip1ex
\parindent0pt
\let\olditemize\itemize
\def\itemize{\olditemize\parskip0pt}

\begin{document}

\title{The \textsf{childdoc} Package}
\hypersetup{pdftitle={The childdoc Package}}
\author{Niklas Beisert\\[2ex]
  Institut f\"ur Theoretische Physik\\
  Eidgen\"ossische Technische Hochschule Z\"urich\\
  Wolfgang-Pauli-Strasse 27, 8093 Z\"urich, Switzerland\\[1ex]
  \href{mailto:nbeisert@itp.phys.ethz.ch}
  {\texttt{nbeisert@itp.phys.ethz.ch}}}
\hypersetup{pdfauthor={Niklas Beisert}}
\hypersetup{pdfsubject={Manual for the LaTeX2e Package childdoc}}
\date{30 December 2018, \textsf{v2.0}}
\maketitle

\begin{abstract}\noindent
\textsf{childdoc} is a \LaTeXe{} package
that enables the direct compilation
of document sections included by |\include|
to individual files.
\end{abstract}

\begingroup
\parskip0ex
\tableofcontents
\endgroup

%%%%%%%%%%%%%%%%%%%%%%%%%%%%%%%%%%%%%%%%%%%%%%%%%%%%%%%%%%%%%%%%%%%%%%%%%%%%%%%%
%%%%%%%%%%%%%%%%%%%%%%%%%%%%%%%%%%%%%%%%%%%%%%%%%%%%%%%%%%%%%%%%%%%%%%%%%%%%%%%%
\section{Introduction}

\LaTeX{} provides a mechanism to structure a large document (such as a book)
into a main file and several child files (containing the chapters)
using the |\include| command.
This mechanism is beneficial for documents
which span hundreds of pages in order to
make the source file(s) more manageable.
Moreover, compilation can be restricted to
selected child files by means of the |\includeonly| command.
The latter feature can be used to reduce the compilation time while editing
(this was significantly more useful in the earlier days of \LaTeX{})
or to generate a smaller document which is easier to navigate.
Another application of |\includeonly| is to generate
documents consisting of selected parts of the complete document.

However, there are a few drawbacks of the plain |\include| mechanism:
\begin{itemize}
\item
The child files cannot be compiled on their own,
they can only be compiled via the main file.
A naive editing environment
(such as a text editor with an option
to have the current file processed by \LaTeX)
may require one to switch to the main file before compiling;
attempting to compile the child file produces errors.
\item
The main file must be modified (each time)
to adjust the |\includeonly| command
to the present needs. This easily leaves the main file in a messy state.
\item
The generated document will always carry the filename
of the main document. This is inconvenient if
several child files are to be compiled and
to be kept for distribution.
\end{itemize}

The present package provides a simple interface
to make child files individually compilable by \LaTeX{}.
Compiling a child file then has the same effect as compiling
the main file with an |\includeonly| command
to select the appropriate child.
Moreover the generated document will carry the name of the child
rather than the main file.
This resolves all three above issues.

This feature is meant to make the editing of books,
thesis documents and lecture notes somewhat more convenient.
However, the package can also be used efficiently for
composing a series of documents (such as exercise sheets)
which are typically distributed individually.
It then assists the author in generating the individual documents
(potentially in different versions)
as well as a document containing the collected series.
Another application is in developing style files
or other kinds of included material
where compilation of the style file could redirect
to a sample or test file.

%%%%%%%%%%%%%%%%%%%%%%%%%%%%%%%%%%%%%%%%%%%%%%%%%%%%%%%%%%%%%%%%%%%%%%%%%%%%%%%%
%%%%%%%%%%%%%%%%%%%%%%%%%%%%%%%%%%%%%%%%%%%%%%%%%%%%%%%%%%%%%%%%%%%%%%%%%%%%%%%%
\section{Usage}

First of all, the package \textsf{childdoc} is \emph{not} a standard
\LaTeXe{} |.sty| style file! Therefore it needs to be invoked in
a non-standard way.

%%%%%%%%%%%%%%%%%%%%%%%%%%%%%%%%%%%%%%%%%%%%%%%%%%%%%%%%%%%%%%%%%%%%%%%%%%%%%%%%
\subsection{Included Files}
\label{sec:include}

%%%%%%%%%%%%%%%%%%%%%%%%%%%%%%%%%%%%%%%%
\DescribeMacro{\childdocmain}
To use the package, add the commands
\begin{center}
\begin{tabular}{l}
|\input{childdoc.def}|\\
|\childdocmain{}|\\
\end{tabular}
\end{center}
at the very top of the main \LaTeX{} file,
in particular \emph{before} the |\documentclass| statement!
The argument of |\childdocmain| should be left empty
(but it must be present).

%%%%%%%%%%%%%%%%%%%%%%%%%%%%%%%%%%%%%%%%
\DescribeMacro{\childdocof}
Furthermore, add the commands
\begin{center}
\begin{tabular}{l}
|\input{childdoc.def}|\\
|\childdocof{|\textit{main}|}|\\
\end{tabular}
\end{center}
at the top of every child file \textit{child}
which is included by |\include{|\textit{child}|}|
from within the main file
(or at least for those files to be compiled individually).
The argument \textit{main} must be the filename of the main file.

There are a couple of
considerations in setting up the main and child documents:

%%%%%%%%%%%%%%%%%%%%%%%%%%%%%%%%%%%%%%%%
\paragraph{Restrictions.}

Please note the following restrictions:
\begin{itemize}
\item
|\childdocmain| must be called with one argument \textit{main}
to ensure compatibility with earlier version of the package.
It must either be empty (|\childdocmain{}|)
or precisely match the filename of the main file in which it is specified.
See \secref{sec:detection} for further information.
\item
The filename \textit{main} must be specified without the |.tex| extension.
\item
The filename \textit{main} is case sensitive
(even in case-insensitive file systems)
due to internal string comparison.
\item
The argument \textit{main} should be fully expanded, it cannot be a macro.
\item
Subdirectories and special characters should be avoided in filenames.
\item
The command |\childdocmain{|\textit{main}|}| must be followed by a whitespace.
It should not be followed immediately by another command
or by a comment mark `|%|'.
This is because the \TeX{} parser reads the token immediately following
the argument of |\childdocmain| and puts it
at the beginning of every child section;
however, a white\-space is ignored.
\end{itemize}

%%%%%%%%%%%%%%%%%%%%%%%%%%%%%%%%%%%%%%%%
\paragraph{Content of Main File.}

It is advisable to place all content in the child files included by |\include|.
Any output contained in the main file will appear in all child documents
unless suppressed manually;
it cannot be suppressed automatically by the |\includeonly| directive
and thus should normally be avoided.
A method to include some content in the main file
by means of conditional processing is described in \secref{sec:conditional}.

%%%%%%%%%%%%%%%%%%%%%%%%%%%%%%%%%%%%%%%%
\paragraph{Page Numbering.}

When only a part of the document is compiled,
the appropriate numbering of pages
(as well as other status parameters)
is determined from the |.aux| files.
The latter contain information from previous passes.
However this information needs to propagate through
all intermediate child documents.
Therefore the page numbering in child documents may well
be inconsistent until the complete document is compiled at least once.

A useful (if unconventional) way to always ensure a consistent
page numbering is to restart the numbering in each child document
and denote the pages by `\textit{child}|.|\textit{page}'
where \textit{child} represents the chapter/section number of the child file.
This can be achieved by the command
|\numberwithin{page}{|\textit{child}|}|
of the \textsf{amsmath} package
where \textit{child} can be |chapter| or |section|
depending on the chosen structuring.
Alternatively, one can modify the macro |\thepage| appropriately
and reset the counter |page| at the start of each child file.

%%%%%%%%%%%%%%%%%%%%%%%%%%%%%%%%%%%%%%%%%%%%%%%%%%%%%%%%%%%%%%%%%%%%%%%%%%%%%%%%
\subsection{Conditional Processing}
\label{sec:conditional}

The package provides a mechanism to compile different versions
of a document. To customise the versions further some conditional processing
can come in handy to distinguish which version is being compiled.
The package provides two macros to describe the compilation context:

%%%%%%%%%%%%%%%%%%%%%%%%%%%%%%%%%%%%%%%%
\DescribeMacro{\ifchilddoc}
The conditional |\ifchilddoc| distinguishes between the compilation of
child documents and the main document:
%
\begin{center}
|\ifchilddoc |\textit{child-code}| |[|\||else |\textit{main-code}]| \||fi|
\end{center}

%%%%%%%%%%%%%%%%%%%%%%%%%%%%%%%%%%%%%%%%
\DescribeMacro{\childdocname}
\DescribeMacro{\childdocjob}
The macro |\childdocname| contains the filename (without extension)
of the main or child file being processed.
Note that |\childdocjob| will always contain the name of the main file.

%%%%%%%%%%%%%%%%%%%%%%%%%%%%%%%%%%%%%%%%
\paragraph{Title Page.}

Conditional processing can be used to include a title or banner page
in the main document when proper precautions are taken.
Importantly, the code in the main file should ensure that the page counter
(as well as other status parameters which are stored in the |.aux| files)
takes the same value after the conditional processing.
Otherwise the page numbers may take divergent values
depending on which part is compiled.

For example, a title page could be declared by:
%
\begin{center}
\begin{tabular}{l}
|\ifchilddoc\||else|\\
|\addtocounter{page}{-1}|\\
\textit{code for title page}\\
|\newpage|\\
|\||fi|
\end{tabular}
\end{center}
%
A banner page for the child documents can be generated by:
%
\begin{center}
\begin{tabular}{l}
|\ifchilddoc|\\
|\addtocounter{page}{-1}|\\
\textit{code for banner page}\\
|\newpage|\\
|\||fi|
\end{tabular}
\end{center}
%
Here one could write a message such as:
\begin{center}
|This is the part \childdocname{} of \childdocjob{}.|
\end{center}

%%%%%%%%%%%%%%%%%%%%%%%%%%%%%%%%%%%%%%%%%%%%%%%%%%%%%%%%%%%%%%%%%%%%%%%%%%%%%%%%
\subsection{Flags}
\label{sec:flags}

The package makes it easy to generate different versions
of the main or child documents.
To this end compilation flags can be defined
and assigned different default values.
They will be particularly useful in conjunction
with the forwarding mechanism described in \secref{sec:forward}.

For example, it may be useful to have a flag |\version|
which can be set to |draft| or |final|.
The document source will contain some conditional code
depending on the value of |\version|.
Suppose further, the flag should default to |final| for the main file
and to |draft| for child files
which is a natural assignment for editing the document.
This is achieved by placing the following code
in the preamble of the main document
(below the |\childdocmain| directive):
%
\begin{center}
\begin{tabular}{l}
|\ifchilddoc|\\
|\providecommand{\version}{draft}|\\
|\||else|\\
|\providecommand{\version}{final}|\\
|\||fi|
\end{tabular}
\end{center}
%
The definition by |\providecommand| makes sure
that previous definitions are not overwritten.
Further statements |\providecommand{\version}{...}|
can thus be added before the above code to override it.

For the main file, one might add a line
(between |\childdocmain| and the above block)
%
\begin{center}
|%\ifchilddoc\||else\providecommand{\version}{draft}\||fi|
\end{center}
%
which can be uncommented to produce a draft version.
Likewise one can add a line to the very top of a child file
(above the |\childdocof{|\textit{main}|}| directive)
%
\begin{center}
|%\providecommand{\version}{final}|
\end{center}
%
which can be uncommented to produce the final version of this child document.

%%%%%%%%%%%%%%%%%%%%%%%%%%%%%%%%%%%%%%%%%%%%%%%%%%%%%%%%%%%%%%%%%%%%%%%%%%%%%%%%
\subsection{Forwarding}
\label{sec:forward}

Different versions of the main or child documents
using compilation flags as described in \secref{sec:flags}
can be (permanently) stored in different files
for convenient compilation, viewing and distribution.
To this end, the package defines a command
to pass on compilation to a different file:

%%%%%%%%%%%%%%%%%%%%%%%%%%%%%%%%%%%%%%%%
\DescribeMacro{\childdocforward}
The command |\childdocforward| redirects processing to
another source file:
%
\begin{center}
\begin{tabular}{l}
|\input{childdoc.def}|\\
|\childdocforward[|\textit{main}|]{|\textit{dest}|}|\\
\end{tabular}
\end{center}
%
The argument \textit{dest} is the destination file
(without extension).
It should be the main file or one of the child files.
Note that further \textsf{childdoc} directives
such as |\childdocof| and |\childdocforward|
in the indicated file will be processed in this form.
The optional argument \textit{main}
passes on directly to the main file \textit{main}
while pretending to compile the child \textit{dest}.
This form behaves as if \textit{dest}
issues |\childdocof{|\textit{main}|}| right away,
and no further \textsf{childdoc} directives will be processed.

%%%%%%%%%%%%%%%%%%%%%%%%%%%%%%%%%%%%%%%%
\DescribeMacro{\...prefix}
In the alternative form |\childdocforwardprefix|,
%
\begin{center}
\begin{tabular}{l}
|\input{childdoc.def}|\\
|\childdocforwardprefix[|\textit{main}|]{|\textit{prefix}|}{|\textit{dest}|}|
\end{tabular}
\end{center}
%
the destination file is determined by a pattern
depending on the current file:
To make this work, the current file must be called
`{\textit{prefix}\hspace{0.2em}\textit{suffix}}'
with \textit{prefix} matching precisely the argument.
Processing is then passed on to the file
`{\textit{dest}\hspace{0.2em}\textit{suffix}}'.
Surely, the same effect is achieved by
directly specifying the
argument `{\textit{dest}\hspace{0.2em}\textit{suffix}}'
in the first form.
However, that requires to set up a different file
for each child. With the alternative form of the command
all these files can have exactly the same content
which simplifies setting them up and maintaining them.

For example, the following file |draft.tex|
with a compilation flag |\version| as described in \secref{sec:flags}
compiles the main document as a draft:
%
\begin{center}
\begin{tabular}{l}
|\def\version{draft}|\\
|\input{childdoc.def}|\\
|\childdocforward{|\textit{main}|}|
\end{tabular}
\end{center}
%
Likewise, the following files |final|\textit{nn}|.tex|
compile the final version of the child document
|child|\textit{nn}|.tex|:
%
\begin{center}
\begin{tabular}{l}
|\def\version{final}|\\
|\input{childdoc.def}|\\
|\childdocforwardprefix{final}{child}|
\end{tabular}
\end{center}
%

Note that when several versions of a main file and/or of each child file
are to be generated, it may be convenient to set up a |Makefile| or
shell script to automatise the process.

%%%%%%%%%%%%%%%%%%%%%%%%%%%%%%%%%%%%%%%%%%%%%%%%%%%%%%%%%%%%%%%%%%%%%%%%%%%%%%%%
\subsection{Command Line Processing}
\label{sec:commandline}

The effect of redirection files can also be achieved by invoking
the \LaTeX{} compiler with a more elaborate command line.
Most conveniently this should be done as part
of a shell script or a |Makefile|.

When using \textsf{childdoc} in the main file, the following
command lines effectively perform a redirection
(note that depending on the shell being used,
backslashes may have to be doubled: `|\|' $\to$ `|\\|'):
%
\begin{center}
|... -jobname "|\textit{target}|" |\\|"|[\textit{flags}]%
|\input{childdoc.def}\childdocforward[|\textit{main}|]{|\textit{dest}|}"|
\end{center}
%
Here \textit{target} is the name of the output file,
\textit{main} is the name of the main file
and \textit{dest} is the name of the main or child file to be processed
(all filenames without extensions).
The optional argument \textit{main} can be omitted
if \textit{main} matches \textit{dest}.
Optionally, compilation \textit{flags} can be defined via |\def| commands.
This command line makes the \TeX{} engine believe
it is compiling the file \textit{target}
whose content is specified as the latter parameter.
The provided code then forwards the processing to
\textit{main} or \textit{dest} as described in \secref{sec:forward}.

%%%%%%%%%%%%%%%%%%%%%%%%%%%%%%%%%%%%%%%%%%%%%%%%%%%%%%%%%%%%%%%%%%%%%%%%%%%%%%%%
\subsection{Include by Input}
\label{sec:input}

Including child documents by |\include| has some restrictions by design.
Most notably, the content of a child document always occupies
its own set of pages; pages cannot be shared between child documents.
Usually, this behaviour makes perfect sense
because each child document contain an essential part of the document.
However, in some situations it may be desirable to compose
a document from a collection of parts
without having mandatory page breaks between then.
For this case, the package
provides a mechanism to include parts
by |\input| which can also be processed individually.
However, by construction this mechanism
requires manual handling of the content to be output.

%%%%%%%%%%%%%%%%%%%%%%%%%%%%%%%%%%%%%%%%
\DescribeMacro{\ifchilddocmanual}
The main file should be prepared as usual, see \secref{sec:include}.
However, the document body must make a distinction
between processing of an individual part and of the main document, e.g.:
%
\begin{center}
\begin{tabular}{l}
|\ifchilddocmanual|\\
|\input{\childdocname}|\\
|\||else|\\
\textit{document body with }|\input{|\textit{part}|}|\\
|\||fi|
\end{tabular}
\end{center}
%
The conditional |\ifchilddocmanual| is true whenever
a part to be included by |\input| is being compiled,
and the name of the part is stored in |\childdocname|.

%%%%%%%%%%%%%%%%%%%%%%%%%%%%%%%%%%%%%%%%
\DescribeMacro{\childdocby}
Each part to be included by |\input| should start with:
%
\begin{center}
\begin{tabular}{l}
|\input{childdoc.def}|\\
|\childdocby{|\textit{main}|}|\\
\end{tabular}
\end{center}
%
The directive |\childdocby| is similar to |\childdocof|
described in \secref{sec:include},
but the subsequent selection of content must be done manually.
To that end, both |\ifchilddoc| and |\ifchilddocmanual|
will be true upon processing of a part,
and the name of the part is stored in |\childdocname|.
Note that |\jobname| will be set to the filename of the current part
so that each part receives an individual |.aux| file
that does not interfere with the |.aux| file(s) of the main document.
This behaviour can be altered by the alternative form
|\childdocby[*]{|\textit{main}|}| (with a non-empty optional argument)
which uses the |.aux| file of the main document
by setting |\jobname| to \textit{main}.

%%%%%%%%%%%%%%%%%%%%%%%%%%%%%%%%%%%%%%%%%%%%%%%%%%%%%%%%%%%%%%%%%%%%%%%%%%%%%%%%
\subsection{Driver Development}
\label{sec:driver}

The \textsf{childdoc} mechanism can also be use for the development
of definition files such as \LaTeX{} styles or classes.
This case differs from the above setup with multiple parts
included by |\include| in that no |\includeonly| should be invoked.
This can be achieved by starting the include file
(before |\ProvidesPackage|) with:
%
\begin{center}
\begin{tabular}{l}
|\input{childdoc.def}|\\
|\childdocforward{|\textit{main}|}|\\
\end{tabular}
\end{center}
%
or alternatively with:
%
\begin{center}
\begin{tabular}{l}
|\input{childdoc.def}|\\
|\childdocby{|\textit{main}|}|\\
\end{tabular}
\end{center}
%
Both forms have slightly different effects as described above.
The main file is prepared as usual, see \secref{sec:include}.

%%%%%%%%%%%%%%%%%%%%%%%%%%%%%%%%%%%%%%%%%%%%%%%%%%%%%%%%%%%%%%%%%%%%%%%%%%%%%%%%
\subsection{Legacy Detection}
\label{sec:detection}

The directive |\childdocmain| in the main file can detect
whether the complete document or merely a child is to be compiled
even without using the directive |\childdocof|.
This method is deprecated because it is less robust
and there is no compelling reason to use it;
it is merely provided for backward compatibility
and it may be removed in future versions.

If the detection mechanism is to be used,
it is mandatory to correctly specify
the filename of the main file as the argument of |\childdocmain|:
%
\begin{center}
\begin{tabular}{l}
|\input{childdoc.def}|\\
|\childdocmain{|\textit{main}|}|\\
\end{tabular}
\end{center}
%
If |\jobname| does not match the argument \textit{main} of |\childdocmain|,
it is assumed that |\jobname| points to the child file to be compiled.
When using |\childdocmain| with the main file specified as argument,
it suffices to start a child file
with just |\input{|\textit{main}|}|
without loading of the package and using |\childdocof|.
If instead all processing is done
with the appropriate \textsf{childdoc} directives,
the argument of \textit{main} of |\childdocmain| can be empty.

An alternative version of the command line processing described
in \secref{sec:commandline} using the detection mechanism reads:
%
\begin{center}
|... -jobname "|\textit{target}|" "|[\textit{flags}]%
[|\def\jobname{|\textit{dest}|}|]|\input{|\textit{main}|}"|
\end{center}

%%%%%%%%%%%%%%%%%%%%%%%%%%%%%%%%%%%%%%%%%%%%%%%%%%%%%%%%%%%%%%%%%%%%%%%%%%%%%%%%
\subsection{Manual Code}
\label{sec:manual}

In case one cannot be certain whether the definitions file |childdoc.def|
is installed on the target \TeX{} distribution
and one prefers not to ship it,
it is conceivable to paste a few relevant commands into the sources.

To that end, drop all statements |\input{childdoc.def}|
and perform the replacements as outlined below.
Instead of |\childdocmain{|\textit{main}|}| add the following code
to the top of the main file:
%
\begin{center}
\begin{tabular}{l}
|\||ifdefined\childdocname\endinput\||fi\newif\ifchilddoc|\\
|\edef\childdocname{\scantokens\expandafter{\jobname\noexpand}}|\\
|\def\childdocmain{|\textit{main}|}\||ifx\childdocmain\childdocname\||else|\\
|\childdoctrue\includeonly{\childdocname}\let\jobname\childdocmain\||fi|\\
\end{tabular}
\end{center}
%
Instead of |\childdocof{|\textit{main}|}| just include the main file
at the top of each child file:
%
\begin{center}
|\input{|\textit{main}|}|
\end{center}
%
A simple redirection |\childdocforward{|\textit{dest}|}| is achieved by:
%
\begin{center}
|\def\jobname{|\textit{dest}|}\input{\jobname}|
\end{center}
%
The redirection with prefix
|\childdocforwardprefix[|\textit{prefix}|]{|\textit{dest}|}|
is accomplished by:
%
\begin{center}
\begin{tabular}{l}
|{\edef\jobname{\scantokens\expandafter{\jobname\noexpand}}|\\
|\def\redirectjob |\textit{prefix}|#1~~~{\gdef\jobname{|\textit{dest}|#1}}|\\
|\expandafter\redirectjob\jobname~~~}\input{\jobname}|
\end{tabular}
\end{center}

In an alternative approach,
child documents can be compiled by a specific command line
without additional code or specific definitions:
%
\begin{center}
|... -jobname "|\textit{target}|" "|[\textit{flags}]%
|\includeonly{|\textit{dest}|}\input{|\textit{main}|}"|
\end{center}
%

%%%%%%%%%%%%%%%%%%%%%%%%%%%%%%%%%%%%%%%%%%%%%%%%%%%%%%%%%%%%%%%%%%%%%%%%%%%%%%%%
%%%%%%%%%%%%%%%%%%%%%%%%%%%%%%%%%%%%%%%%%%%%%%%%%%%%%%%%%%%%%%%%%%%%%%%%%%%%%%%%
\section{Information}

%%%%%%%%%%%%%%%%%%%%%%%%%%%%%%%%%%%%%%%%%%%%%%%%%%%%%%%%%%%%%%%%%%%%%%%%%%%%%%%%
\subsection{Copyright}

Copyright \copyright{} 2017--2018 Niklas Beisert

This work may be distributed and/or modified under the
conditions of the \LaTeX{} Project Public License, either version 1.3
of this license or (at your option) any later version.
The latest version of this license is in
  \url{http://www.latex-project.org/lppl.txt}
and version 1.3 or later is part of all distributions of \LaTeX{}
version 2005/12/01 or later.

This work has the LPPL maintenance status `maintained'.

The Current Maintainer of this work is Niklas Beisert.

This work consists of the files |README.txt|, |childdoc.ins| and |childdoc.dtx|
as well as the derived files |childdoc.def|, |cdocsamp.tex|
with |cdocsch1.tex|, |cdocsch2.tex|, |cdocspt3.tex|, |cdocspt4.tex|,
|cdocsdrf.tex|, |cdocsfn1.tex|, |cdocsfn2.tex|
as well as |childdoc.pdf|.

%%%%%%%%%%%%%%%%%%%%%%%%%%%%%%%%%%%%%%%%%%%%%%%%%%%%%%%%%%%%%%%%%%%%%%%%%%%%%%%%
\subsection{Files and Installation}

The package consists of the files:
%
\begin{center}
\begin{tabular}{ll}
    |README.txt|   & readme file \\
    |childdoc.ins| & installation file \\
    |childdoc.dtx| & source file \\
    |childdoc.def| & definition file \\
    |cdocsamp.tex| & sample main file \\
    |cdocsch1.tex| & sample include file \\
    |cdocsch2.tex| & sample include file \\
    |cdocspt3.tex| & sample part file \\
    |cdocspt4.tex| & sample part file \\
    |cdocsdrf.tex| & sample redirection file \\
    |cdocsfn1.tex| & sample redirection file \\
    |cdocsfn2.tex| & sample redirection file \\
    |childdoc.pdf| & manual
\end{tabular}
\end{center}
%
The distribution consists of the files
|README.txt|, |childdoc.ins| and |childdoc.dtx|.
%
\begin{itemize}
\item
Run (pdf)\LaTeX{} on |childdoc.dtx|
to compile the manual |childdoc.pdf| (this file).
\item
Run \LaTeX{} on |childdoc.ins| to create the definitions file |childdoc.def|
and the sample |cdocsamp.tex| with include files
|cdocsch1.tex|, |cdocsch2.tex|, |cdocspt3.tex|, |cdocspt4.tex|,
|cdocsdrf.tex|, |cdocsfn1.tex|, |cdocsfn2.tex|.
Then copy the file |childdoc.def| to an appropriate directory of your \LaTeX{}
distribution, e.g.\ \textit{texmf-root}|/tex/latex/childdoc|.
\end{itemize}

%%%%%%%%%%%%%%%%%%%%%%%%%%%%%%%%%%%%%%%%%%%%%%%%%%%%%%%%%%%%%%%%%%%%%%%%%%%%%%%%
\subsection{Related CTAN Packages}

There are several other packages which offer a similar functionality:
%
\begin{itemize}
\item
The packages
\href{http://ctan.org/pkg/docmute}{\textsf{docmute}},
\href{http://ctan.org/pkg/includex}{\textsf{includex}} and
\href{http://ctan.org/pkg/standalone}{\textsf{standalone}}
provide commands to include only the document body of
a child file thus allowing both files to be compiled individually.
\item
The packages \href{http://ctan.org/pkg/subdocs}{\textsf{subdocs}}
and \href{http://ctan.org/pkg/subfiles}{\textsf{subfiles}}
provide structures in which the main and child documents can be
encapsulated and allowing them to be compiled individually.
The inclusion mechanism is different from the conventional |\include|.
\item
The package \href{http://ctan.org/pkg/combine}{\textsf{combine}}
is an elaborate solution to combine several documents into one.
\end{itemize}
%
See also the CTAN topic \href{http://ctan.org/topic/subdocs}{\textsf{subdocs}}
for further related packages.
The present package differs from the above solutions in that
a document structure constructed with the conventional |\include| mechanism
just needs two extra commands at the top of every file
such that all constituent files can be compiled individually.

%%%%%%%%%%%%%%%%%%%%%%%%%%%%%%%%%%%%%%%%%%%%%%%%%%%%%%%%%%%%%%%%%%%%%%%%%%%%%%%%
%\subsection{Feature Suggestions}
%
%The following is a list of features which may be useful for future
%versions of this package:
%%
%\begin{itemize}
%\item
%\ldots
%\end{itemize}

%%%%%%%%%%%%%%%%%%%%%%%%%%%%%%%%%%%%%%%%%%%%%%%%%%%%%%%%%%%%%%%%%%%%%%%%%%%%%%%%
\subsection{Revision History}

%%%%%%%%%%%%%%%%%%%%%%%%%%%%%%%%%%%%%%%%
\paragraph{v2.0:} 2018/12/30

\begin{itemize}
\item
immediate forward processing
\item
added |\childdocby| mechanism
\item
manual restructured
\end{itemize}

%%%%%%%%%%%%%%%%%%%%%%%%%%%%%%%%%%%%%%%%
\paragraph{v1.6:} 2018/01/17

\begin{itemize}
\item
application for development of include files
\item
corrections to manual
\end{itemize}

%%%%%%%%%%%%%%%%%%%%%%%%%%%%%%%%%%%%%%%%
\paragraph{v1.5:} 2017/05/21

\begin{itemize}
\item
more complete structuring introduced
\item
|\childdocof| introduced
\item
|\childdoc| renamed to |\childdocmain|
\item
|\childredirect| renamed to |\childdocforward| and |\childdocforwardprefix|
and functionality expanded
\end{itemize}

%%%%%%%%%%%%%%%%%%%%%%%%%%%%%%%%%%%%%%%%
\paragraph{v1.0:} 2017/04/27

\begin{itemize}
\item
manual and install package
\item
first version published on CTAN
\end{itemize}

%%%%%%%%%%%%%%%%%%%%%%%%%%%%%%%%%%%%%%%%
\paragraph{v0.6:} 2017/04/26

\begin{itemize}
\item
redirection mechanism added
\end{itemize}

%%%%%%%%%%%%%%%%%%%%%%%%%%%%%%%%%%%%%%%%
\paragraph{v0.5:} 2017/04/26

\begin{itemize}
\item
functionality in definition file
\end{itemize}


%%%%%%%%%%%%%%%%%%%%%%%%%%%%%%%%%%%%%%%%%%%%%%%%%%%%%%%%%%%%%%%%%%%%%%%%%%%%%%%%
%%%%%%%%%%%%%%%%%%%%%%%%%%%%%%%%%%%%%%%%%%%%%%%%%%%%%%%%%%%%%%%%%%%%%%%%%%%%%%%%
%%%%%%%%%%%%%%%%%%%%%%%%%%%%%%%%%%%%%%%%%%%%%%%%%%%%%%%%%%%%%%%%%%%%%%%%%%%%%%%%
\appendix

\settowidth\MacroIndent{\rmfamily\scriptsize 000\ }

 \DocInput{childdoc.dtx}

\end{document}
%</driver>
% \fi
%
% %%%%%%%%%%%%%%%%%%%%%%%%%%%%%%%%%%%%%%%%%%%%%%%%%%%%%%%%%%%%%%%%%%%%%%%%%%%%%%
% %%%%%%%%%%%%%%%%%%%%%%%%%%%%%%%%%%%%%%%%%%%%%%%%%%%%%%%%%%%%%%%%%%%%%%%%%%%%%%
% \section{Sample}
%\iffalse
%<*samplemain>
%\fi
%
% The following presents a sample document
% with two chapters, two parts, a title page,
% a compile flag as well as three forwarding files to set the flag.
% It consists of eight |.tex| files:
% \begin{center}
% \begin{tabular}{ll}
% |cdocsamp.tex|&main file\\
% |cdocsch1.tex|&include file for chapter 1\\
% |cdocsch2.tex|&include file for chapter 2\\
% |cdocspt3.tex|&include file for part 3\\
% |cdocspt4.tex|&include file for part 4\\
% |cdocsdrf.tex|&forwarding file for main file in draft mode\\
% |cdocsfi1.tex|&forwarding file for final version of chapter 1\\
% |cdocsfi2.tex|&forwarding file for final version of chapter 2\\
% \end{tabular}
% \end{center}
% Each of the eight files can be compiled directly by the \LaTeX{} compiler.
%
% %%%%%%%%%%%%%%%%%%%%%%%%%%%%%%%%%%%%%%
% \paragraph{Main File.}
%
% The main file is called |cdocsamp.tex|.
%
% Load the \textsf{childdoc} definitions and
% declare the filename for the main document:
%    \begin{macrocode}
\input{childdoc.def}
\childdocmain{}
%    \end{macrocode}

% Optional override for |\version| flag:
%    \begin{macrocode}
%%\ifchilddoc\else\providecommand{\version}{draft}\fi
%    \end{macrocode}

% Define the default values for the |\version| flag
% (|final| for the main file and |draft| for childs):
%    \begin{macrocode}
\ifchilddoc
\providecommand{\version}{draft}
\else
\providecommand{\version}{final}
\fi
%    \end{macrocode}

% Load the standard document class:
%    \begin{macrocode}
\documentclass[12pt]{article}
%    \end{macrocode}

% Start the document body:
%    \begin{macrocode}
\begin{document}
%    \end{macrocode}

% Declare a title page.
% Print title, part of document being processed and version flag:
%    \begin{macrocode}
\addtocounter{page}{-1}
\begin{center}
{\LARGE\bfseries{}childdoc example\par}
\vspace{1cm}
\ifchilddoc
\ifchilddocmanual part\else chapter\fi:
`\childdocname' of `\childdocjob'\par
\else
main document: `\childdocjob'\par
\fi
version: \version\par
\end{center}
\newpage
%    \end{macrocode}

% Manually include selected file,
% otherwise process as usual:
%    \begin{macrocode}
\ifchilddocmanual
\section*{part `\childdocname'}
\input{\childdocname}
\else
%    \end{macrocode}

% Include the two chapters:
%    \begin{macrocode}
\include{cdocsch1}
\include{cdocsch2}
%    \end{macrocode}

% Include the two parts unless only chapters should be displayed:
%    \begin{macrocode}
\ifchilddoc\else
\section{part three}
\input{cdocspt3}
\section{part four}
\input{cdocspt4}
\fi
%    \end{macrocode}

% Process as usual until here:
%    \begin{macrocode}
\fi
%    \end{macrocode}

% End of document body:
%    \begin{macrocode}
\end{document}
%    \end{macrocode}
%\iffalse
%</samplemain>
%\fi
%
% %%%%%%%%%%%%%%%%%%%%%%%%%%%%%%%%%%%%%%
% \paragraph{Chapter Include Files.}
%
% The include files are called |cdocsch1.tex| and |cdocsch2.tex|.
%
%\iffalse
%<*samplechap1|samplechap2>
%\fi

% Optional override for |\version| flag:
%    \begin{macrocode}
%%\providecommand{\version}{final}
%    \end{macrocode}

% Include the main document:
%    \begin{macrocode}
\input{childdoc.def}
\childdocof{cdocsamp}
%    \end{macrocode}

%\iffalse
%</samplechap1|samplechap2>
%\fi
%
%\iffalse
%<*samplechap1>
%\fi
% Some text for chapter 1:
%    \begin{macrocode}
\section{one}
some text in chapter one
%    \end{macrocode}

%\iffalse
%</samplechap1>
%\fi
% Some text for chapter 2:
%\iffalse
%<*samplechap2>
%\fi
%    \begin{macrocode}
\section{two}
more text in chapter two
%    \end{macrocode}

%\iffalse
%</samplechap2>
%\fi
%
% %%%%%%%%%%%%%%%%%%%%%%%%%%%%%%%%%%%%%%
% \paragraph{Part Include Files.}
%
% The include files are called |cdocspt3.tex| and |cdocspt4.tex|.
%
%\iffalse
%<*samplepart3|samplepart4>
%\fi

% Optional override for |\version| flag:
%    \begin{macrocode}
%%\providecommand{\version}{final}
%    \end{macrocode}

% Include the main document:
%    \begin{macrocode}
\input{childdoc.def}
\childdocby{cdocsamp}
%    \end{macrocode}

%\iffalse
%</samplepart3|samplepart4>
%\fi
%
%\iffalse
%<*samplepart3>
%\fi
% Some text for part 3:
%    \begin{macrocode}
some text in part three
%    \end{macrocode}

%\iffalse
%</samplepart3>
%\fi
% Some text for part 4:
%\iffalse
%<*samplepart4>
%\fi
%    \begin{macrocode}
more text in part four
%    \end{macrocode}

%\iffalse
%</samplepart4>
%\fi
%
% %%%%%%%%%%%%%%%%%%%%%%%%%%%%%%%%%%%%%%
% \paragraph{Forwarding for a Complete Draft.}
%
% The following forwarding file |cdocsdrf.tex|
% compiles the main document in draft mode:
%\iffalse
%<*sampledraft>
%\fi
%    \begin{macrocode}
\def\version{draft}
\input{childdoc.def}
\childdocforward{cdocsamp}
%    \end{macrocode}

%\iffalse
%</sampledraft>
%\fi
%
% %%%%%%%%%%%%%%%%%%%%%%%%%%%%%%%%%%%%%%
% \paragraph{Forwarding for Final Version of the Chapters.}
%
% The following forwarding files |cdocsfn1.tex| and |cdocsfn2.tex|
% (with identical content)
% compile the final versions of the child documents
% |cdocsch1.tex| and |cdocsch2.tex|, respectively:
%\iffalse
%<*samplefinal>
%\fi
%    \begin{macrocode}
\def\version{final}
\input{childdoc.def}
\childdocforwardprefix[cdocsamp]{cdocsfn}{cdocsch}
%    \end{macrocode}

%\iffalse
%</samplefinal>
%\fi
%
% %%%%%%%%%%%%%%%%%%%%%%%%%%%%%%%%%%%%%%
% \paragraph{Command Line Processing.}
%
% The following three command lines generate the output files
% |cdocscld|, |cdocscl1| and |cdocscl2|
% which should be identical to
% |cdocsdrf|, |cdocsch1| and |cdocsfn2|, respectively:
% \begin{center}
% \begin{tabular}{l}
% |latex -jobname cdocscld \|\\
% |  "\def\version{draft}\input{childdoc.def}\childdocforward{cdocsamp}"|\\
% |latex -jobname cdocscl1 \|\\
% |  "\input{childdoc.def}\childdocforward[cdocsamp]{cdocsch1}"|\\
% |latex -jobname cdocscl2 \|\\
% |  "\def\version{final}\input{childdoc.def}\childdocforward{cdocsch2}"|
% \end{tabular}
% \end{center}
% Note that the trailing backslash on each first line
% merely continues the input to the second line
% (for convenient cut ant paste).
% Furthermore, the command |latex| can be replaced by any
% of its alternative versions such as |pdflatex|.
%
% %%%%%%%%%%%%%%%%%%%%%%%%%%%%%%%%%%%%%%%%%%%%%%%%%%%%%%%%%%%%%%%%%%%%%%%%%%%%%%
% %%%%%%%%%%%%%%%%%%%%%%%%%%%%%%%%%%%%%%%%%%%%%%%%%%%%%%%%%%%%%%%%%%%%%%%%%%%%%%
% \section{Implementation}
%\iffalse
%<*package>
%\fi
%
% This section describes the definitions file |childdoc.def|.

% The definitions cannot be loaded using |\usepackage| or |\RequirePackage|
% which has a mechanism to prevent loading a style file more than once.
% When loading the definitions by means of |\input|
% multiple instances have to be prevented manually:
%\iffalse
%This code needs to be before the `\ProvidesFile' directive
%which is defined at the beginning of this file.
%Therefore it is also placed there and commented out here.
%</package>
%<*discard>
%\fi
%    \begin{macrocode}
\ifdefined\childdocmain\endinput\fi
%    \end{macrocode}
%\iffalse
%</discard>
%<*package>
%\fi
%
% \macro{\ifchilddoc}
% \macro{\ifchilddocmanual}
% The conditional |\ifchilddoc| tells whether a
% child (true) or main (false) document is being compiled.
% The conditional |\ifchilddocmanual| tells whether
% the |\includeonly| mechanism is used (false) or
% the selection of child files must be performed manually (true).
% The definitions initialise to false:
%    \begin{macrocode}
\newif\ifchilddoc
\newif\ifchilddocmanual
%    \end{macrocode}

% \macro{\childdocname}
% \macro{\childdocjob}
% The macro |\childdocname| stores the name of the main document
% to be compiled. The macro |\childdocjob| stores the name of
% the document on which the \LaTeX{} compiler was originally invoked.
% The content of |\jobname| cannot be compared
% to filenames specified in the source due to different catcodes.
% The following code rescans |\jobname|, stores the result
% in |\childdocname| and saves a copy in |\childdocjob|:
%    \begin{macrocode}
\edef\childdocname{\scantokens\expandafter{\jobname\noexpand}}
\let\childdocjob\childdocname
%    \end{macrocode}

% \macro{\childdocdisable}
% The macro |\childdocdisable| prevents the main file
% from being processed more than once.
% At this stage, the main document command |\childdocmain|
% is assumed to be called once again where it should do nothing.
% Any subsequent call to it should prevent
% a secondary processing of the main document
% It overwrites the forwarding commands
% |\childdocof| and |\childdocforward|
% with empty macros to prevent further inclusions of the main document:
%    \begin{macrocode}
\newcommand{\childdocdisable}
{
  \renewcommand{\childdocmain}[1]{\renewcommand{\childdocmain}[1]{\endinput}}
  \renewcommand{\childdocof}[1]{}
  \renewcommand{\childdocby}[2][]{}
  \renewcommand{\childdocforward}[2][]{}
  \renewcommand{\childdocdisable}{}
}
%    \end{macrocode}

% \macro{\childdocmain}
% The macro |\childdocmain| is to be called at the top of the main file
% with nothing or the main filename (without extension) as argument.
% First, it breaks loops.
% If the argument is not empty and does not match |\childdocname|
% (which is set by the first inclusion of |childdoc.def|),
% |\ifchilddoc| is set to true, |\includeonly| is applied to the child file
% and |\jobname| is set to the main file
% (for proper handling of |.aux| files):
%    \begin{macrocode}
\newcommand{\childdocmain}[1]
{
  \childdocdisable\childdocmain{}
  \if?#1?\else
    \begingroup
      \def\childdoctmp{#1}
      \ifx\childdoctmp\childdocname
        \def\childdoctmp{}
      \else
        \def\childdoctmp
        {
          \childdoctrue
          \includeonly{\childdocname}
          \def\childdocjob{#1}
          \def\jobname{#1}
        }
      \fi
      \expandafter
    \endgroup
    \childdoctmp
  \fi
}
%    \end{macrocode}

% \macro{\childdocof}
% The command |\childdocof| redirects
% compilation to the main file |#1|.
%    \begin{macrocode}
\newcommand{\childdocof}[1]
{
  \childdocdisable
  \childdoctrue
  \includeonly{\childdocname}
  \def\jobname{#1}
  \def\childdocjob{#1}
  \input{#1}
}
%    \end{macrocode}

% \macro{\childdocby}
% The command |\childdocby| ....
%    \begin{macrocode}
\newcommand{\childdocby}[2][]
{
  \childdocdisable
  \childdoctrue
  \childdocmanualtrue
  \if?#1?\else
    \def\jobname{#2}
  \fi
  \def\childdocjob{#2}
  \input{#2}
  \endinput
}
%    \end{macrocode}

% \macro{\childdocforward}
% The command |\childdocforward| redirects
% compilation to the main file or
% (if the optional argument is given) a child file.
% Parameters are set as if the main file
% or a child file starting with |\childdocof| was compiled.
% Then compilation is handed over to the main file:
%    \begin{macrocode}
\newcommand{\childdocforward}[2][]
{
  \begingroup
    \if?#1?
      \def\childdoctmp
      {
        \def\childdocname{#2}
        \def\childdocjob{#2}
        \def\jobname{#2}
        \input{#2}
        \endinput
      }
    \else
      \def\childdoctmp
      {
        \childdocdisable
        \def\childdocname{#2}
        \childdoctrue
        \includeonly{#2}
        \def\childdocjob{#1}
        \def\jobname{#1}
        \input{#1}
        \endinput
      }
    \fi
    \expandafter
  \endgroup
  \childdoctmp
}
%    \end{macrocode}

% \macro{\childdocforwardprefix}
% The command |\childdocforwardprefix| redirects
% compilation to the main or a child file by means of a pattern.
% The prefix |#1| in the current filename is replaced by |#2|
% and the suffix of the current filename is kept
% (it is assumed that the filename does not contain the substring `|~~~|'
% which is used as a delimiter).
% Compilation is handed over to the new file by |\childdocforward|:
%    \begin{macrocode}
\newcommand{\childdocforwardprefix}[3][]
{
  \begingroup
    \def\childdocextract #2##1~~~{\def\childdoctmp{\childdocforward[#1]{#3##1}}}
    \expandafter\childdocextract\childdocname~~~
    \expandafter
  \endgroup
  \childdoctmp
}
%    \end{macrocode}

% \macro{\childdoc}
% The deprecated macro |\childdoc| is a legacy version of |\childdocmain|:
%    \begin{macrocode}
\newcommand{\childdoc}{\childdocmain}
%    \end{macrocode}

% \macro{\childdocredirect}
% The deprecated macro |\childdocredirect| is a legacy version
% of |\childdocforward| and |\childdocforwardprefix|:
%    \begin{macrocode}
\newcommand{\childdocredirect}[2][]
{
  \begingroup
    \if?#1?
      \def\childdoctmp{\childdocforward{#2}}
    \else
      \def\childdoctmp{\childdocforwardprefix{#1}{#2}}
    \fi
    \expandafter
  \endgroup
  \childdoctmp
}
%    \end{macrocode}

%\iffalse
%</package>
%\fi
%
\endinput

\childdocforward{cdocsamp}
%    \end{macrocode}

%\iffalse
%</sampledraft>
%\fi
%
% %%%%%%%%%%%%%%%%%%%%%%%%%%%%%%%%%%%%%%
% \paragraph{Forwarding for Final Version of the Chapters.}
%
% The following forwarding files |cdocsfn1.tex| and |cdocsfn2.tex|
% (with identical content)
% compile the final versions of the child documents
% |cdocsch1.tex| and |cdocsch2.tex|, respectively:
%\iffalse
%<*samplefinal>
%\fi
%    \begin{macrocode}
\def\version{final}
% \iffalse
%
% childdoc.dtx Copyright (C) 2017-2018 Niklas Beisert
%
% This work may be distributed and/or modified under the
% conditions of the LaTeX Project Public License, either version 1.3
% of this license or (at your option) any later version.
% The latest version of this license is in
%   http://www.latex-project.org/lppl.txt
% and version 1.3 or later is part of all distributions of LaTeX
% version 2005/12/01 or later.
%
% This work has the LPPL maintenance status `maintained'.
%
% The Current Maintainer of this work is Niklas Beisert.
%
% This work consists of the files childdoc.dtx and childdoc.ins
% and the derived files childdoc.def and cdocsamp.tex with
% cdocsch1.tex, cdocsch2.tex, cdocsdrf.tex, cdocsfn1.tex, cdocsfn2.tex.
%
%<package>\ifdefined\childdocmain\endinput\fi
%<package>\ProvidesFile{childdoc.def}[2018/12/30 v2.0 child document driver]
%<samplemain>\ProvidesFile{cdocsamp.tex}[2018/12/30 v2.0 sample for childdoc]
%<*driver>
%\ProvidesFile{childdoc.drv}[2018/12/30 v2.0 childdoc reference manual file]
\PassOptionsToClass{10pt,a4paper}{article}
\documentclass{ltxdoc}

\usepackage[margin=35mm]{geometry}
\usepackage{hyperref}
\usepackage{hyperxmp}
\usepackage[usenames]{color}

\hypersetup{colorlinks=true}
\hypersetup{pdfstartview=FitH}
\hypersetup{pdfpagemode=UseNone}
\hypersetup{pdfsource={}}
\hypersetup{pdflang={en-UK}}
\hypersetup{pdfcopyright={Copyright 2017-2018 Niklas Beisert.
  This work may be distributed and/or modified under the
  conditions of the LaTeX Project Public License, either version 1.3
  of this license or (at your option) any later version.}}
\hypersetup{pdflicenseurl={http://www.latex-project.org/lppl.txt}}
\hypersetup{pdfcontactaddress={ETH Zurich, ITP, HIT K,
  Wolfgang-Pauli-Strasse 27}}
\hypersetup{pdfcontactpostcode={8093}}
\hypersetup{pdfcontactcity={Zurich}}
\hypersetup{pdfcontactcountry={Switzerland}}
\hypersetup{pdfcontactemail={nbeisert@itp.phys.ethz.ch}}
\hypersetup{pdfcontacturl={http://people.phys.ethz.ch/\xmptilde nbeisert/}}

\newcommand{\secref}[1]{\hyperref[#1]{section \ref*{#1}}}

\parskip1ex
\parindent0pt
\let\olditemize\itemize
\def\itemize{\olditemize\parskip0pt}

\begin{document}

\title{The \textsf{childdoc} Package}
\hypersetup{pdftitle={The childdoc Package}}
\author{Niklas Beisert\\[2ex]
  Institut f\"ur Theoretische Physik\\
  Eidgen\"ossische Technische Hochschule Z\"urich\\
  Wolfgang-Pauli-Strasse 27, 8093 Z\"urich, Switzerland\\[1ex]
  \href{mailto:nbeisert@itp.phys.ethz.ch}
  {\texttt{nbeisert@itp.phys.ethz.ch}}}
\hypersetup{pdfauthor={Niklas Beisert}}
\hypersetup{pdfsubject={Manual for the LaTeX2e Package childdoc}}
\date{30 December 2018, \textsf{v2.0}}
\maketitle

\begin{abstract}\noindent
\textsf{childdoc} is a \LaTeXe{} package
that enables the direct compilation
of document sections included by |\include|
to individual files.
\end{abstract}

\begingroup
\parskip0ex
\tableofcontents
\endgroup

%%%%%%%%%%%%%%%%%%%%%%%%%%%%%%%%%%%%%%%%%%%%%%%%%%%%%%%%%%%%%%%%%%%%%%%%%%%%%%%%
%%%%%%%%%%%%%%%%%%%%%%%%%%%%%%%%%%%%%%%%%%%%%%%%%%%%%%%%%%%%%%%%%%%%%%%%%%%%%%%%
\section{Introduction}

\LaTeX{} provides a mechanism to structure a large document (such as a book)
into a main file and several child files (containing the chapters)
using the |\include| command.
This mechanism is beneficial for documents
which span hundreds of pages in order to
make the source file(s) more manageable.
Moreover, compilation can be restricted to
selected child files by means of the |\includeonly| command.
The latter feature can be used to reduce the compilation time while editing
(this was significantly more useful in the earlier days of \LaTeX{})
or to generate a smaller document which is easier to navigate.
Another application of |\includeonly| is to generate
documents consisting of selected parts of the complete document.

However, there are a few drawbacks of the plain |\include| mechanism:
\begin{itemize}
\item
The child files cannot be compiled on their own,
they can only be compiled via the main file.
A naive editing environment
(such as a text editor with an option
to have the current file processed by \LaTeX)
may require one to switch to the main file before compiling;
attempting to compile the child file produces errors.
\item
The main file must be modified (each time)
to adjust the |\includeonly| command
to the present needs. This easily leaves the main file in a messy state.
\item
The generated document will always carry the filename
of the main document. This is inconvenient if
several child files are to be compiled and
to be kept for distribution.
\end{itemize}

The present package provides a simple interface
to make child files individually compilable by \LaTeX{}.
Compiling a child file then has the same effect as compiling
the main file with an |\includeonly| command
to select the appropriate child.
Moreover the generated document will carry the name of the child
rather than the main file.
This resolves all three above issues.

This feature is meant to make the editing of books,
thesis documents and lecture notes somewhat more convenient.
However, the package can also be used efficiently for
composing a series of documents (such as exercise sheets)
which are typically distributed individually.
It then assists the author in generating the individual documents
(potentially in different versions)
as well as a document containing the collected series.
Another application is in developing style files
or other kinds of included material
where compilation of the style file could redirect
to a sample or test file.

%%%%%%%%%%%%%%%%%%%%%%%%%%%%%%%%%%%%%%%%%%%%%%%%%%%%%%%%%%%%%%%%%%%%%%%%%%%%%%%%
%%%%%%%%%%%%%%%%%%%%%%%%%%%%%%%%%%%%%%%%%%%%%%%%%%%%%%%%%%%%%%%%%%%%%%%%%%%%%%%%
\section{Usage}

First of all, the package \textsf{childdoc} is \emph{not} a standard
\LaTeXe{} |.sty| style file! Therefore it needs to be invoked in
a non-standard way.

%%%%%%%%%%%%%%%%%%%%%%%%%%%%%%%%%%%%%%%%%%%%%%%%%%%%%%%%%%%%%%%%%%%%%%%%%%%%%%%%
\subsection{Included Files}
\label{sec:include}

%%%%%%%%%%%%%%%%%%%%%%%%%%%%%%%%%%%%%%%%
\DescribeMacro{\childdocmain}
To use the package, add the commands
\begin{center}
\begin{tabular}{l}
|\input{childdoc.def}|\\
|\childdocmain{}|\\
\end{tabular}
\end{center}
at the very top of the main \LaTeX{} file,
in particular \emph{before} the |\documentclass| statement!
The argument of |\childdocmain| should be left empty
(but it must be present).

%%%%%%%%%%%%%%%%%%%%%%%%%%%%%%%%%%%%%%%%
\DescribeMacro{\childdocof}
Furthermore, add the commands
\begin{center}
\begin{tabular}{l}
|\input{childdoc.def}|\\
|\childdocof{|\textit{main}|}|\\
\end{tabular}
\end{center}
at the top of every child file \textit{child}
which is included by |\include{|\textit{child}|}|
from within the main file
(or at least for those files to be compiled individually).
The argument \textit{main} must be the filename of the main file.

There are a couple of
considerations in setting up the main and child documents:

%%%%%%%%%%%%%%%%%%%%%%%%%%%%%%%%%%%%%%%%
\paragraph{Restrictions.}

Please note the following restrictions:
\begin{itemize}
\item
|\childdocmain| must be called with one argument \textit{main}
to ensure compatibility with earlier version of the package.
It must either be empty (|\childdocmain{}|)
or precisely match the filename of the main file in which it is specified.
See \secref{sec:detection} for further information.
\item
The filename \textit{main} must be specified without the |.tex| extension.
\item
The filename \textit{main} is case sensitive
(even in case-insensitive file systems)
due to internal string comparison.
\item
The argument \textit{main} should be fully expanded, it cannot be a macro.
\item
Subdirectories and special characters should be avoided in filenames.
\item
The command |\childdocmain{|\textit{main}|}| must be followed by a whitespace.
It should not be followed immediately by another command
or by a comment mark `|%|'.
This is because the \TeX{} parser reads the token immediately following
the argument of |\childdocmain| and puts it
at the beginning of every child section;
however, a white\-space is ignored.
\end{itemize}

%%%%%%%%%%%%%%%%%%%%%%%%%%%%%%%%%%%%%%%%
\paragraph{Content of Main File.}

It is advisable to place all content in the child files included by |\include|.
Any output contained in the main file will appear in all child documents
unless suppressed manually;
it cannot be suppressed automatically by the |\includeonly| directive
and thus should normally be avoided.
A method to include some content in the main file
by means of conditional processing is described in \secref{sec:conditional}.

%%%%%%%%%%%%%%%%%%%%%%%%%%%%%%%%%%%%%%%%
\paragraph{Page Numbering.}

When only a part of the document is compiled,
the appropriate numbering of pages
(as well as other status parameters)
is determined from the |.aux| files.
The latter contain information from previous passes.
However this information needs to propagate through
all intermediate child documents.
Therefore the page numbering in child documents may well
be inconsistent until the complete document is compiled at least once.

A useful (if unconventional) way to always ensure a consistent
page numbering is to restart the numbering in each child document
and denote the pages by `\textit{child}|.|\textit{page}'
where \textit{child} represents the chapter/section number of the child file.
This can be achieved by the command
|\numberwithin{page}{|\textit{child}|}|
of the \textsf{amsmath} package
where \textit{child} can be |chapter| or |section|
depending on the chosen structuring.
Alternatively, one can modify the macro |\thepage| appropriately
and reset the counter |page| at the start of each child file.

%%%%%%%%%%%%%%%%%%%%%%%%%%%%%%%%%%%%%%%%%%%%%%%%%%%%%%%%%%%%%%%%%%%%%%%%%%%%%%%%
\subsection{Conditional Processing}
\label{sec:conditional}

The package provides a mechanism to compile different versions
of a document. To customise the versions further some conditional processing
can come in handy to distinguish which version is being compiled.
The package provides two macros to describe the compilation context:

%%%%%%%%%%%%%%%%%%%%%%%%%%%%%%%%%%%%%%%%
\DescribeMacro{\ifchilddoc}
The conditional |\ifchilddoc| distinguishes between the compilation of
child documents and the main document:
%
\begin{center}
|\ifchilddoc |\textit{child-code}| |[|\||else |\textit{main-code}]| \||fi|
\end{center}

%%%%%%%%%%%%%%%%%%%%%%%%%%%%%%%%%%%%%%%%
\DescribeMacro{\childdocname}
\DescribeMacro{\childdocjob}
The macro |\childdocname| contains the filename (without extension)
of the main or child file being processed.
Note that |\childdocjob| will always contain the name of the main file.

%%%%%%%%%%%%%%%%%%%%%%%%%%%%%%%%%%%%%%%%
\paragraph{Title Page.}

Conditional processing can be used to include a title or banner page
in the main document when proper precautions are taken.
Importantly, the code in the main file should ensure that the page counter
(as well as other status parameters which are stored in the |.aux| files)
takes the same value after the conditional processing.
Otherwise the page numbers may take divergent values
depending on which part is compiled.

For example, a title page could be declared by:
%
\begin{center}
\begin{tabular}{l}
|\ifchilddoc\||else|\\
|\addtocounter{page}{-1}|\\
\textit{code for title page}\\
|\newpage|\\
|\||fi|
\end{tabular}
\end{center}
%
A banner page for the child documents can be generated by:
%
\begin{center}
\begin{tabular}{l}
|\ifchilddoc|\\
|\addtocounter{page}{-1}|\\
\textit{code for banner page}\\
|\newpage|\\
|\||fi|
\end{tabular}
\end{center}
%
Here one could write a message such as:
\begin{center}
|This is the part \childdocname{} of \childdocjob{}.|
\end{center}

%%%%%%%%%%%%%%%%%%%%%%%%%%%%%%%%%%%%%%%%%%%%%%%%%%%%%%%%%%%%%%%%%%%%%%%%%%%%%%%%
\subsection{Flags}
\label{sec:flags}

The package makes it easy to generate different versions
of the main or child documents.
To this end compilation flags can be defined
and assigned different default values.
They will be particularly useful in conjunction
with the forwarding mechanism described in \secref{sec:forward}.

For example, it may be useful to have a flag |\version|
which can be set to |draft| or |final|.
The document source will contain some conditional code
depending on the value of |\version|.
Suppose further, the flag should default to |final| for the main file
and to |draft| for child files
which is a natural assignment for editing the document.
This is achieved by placing the following code
in the preamble of the main document
(below the |\childdocmain| directive):
%
\begin{center}
\begin{tabular}{l}
|\ifchilddoc|\\
|\providecommand{\version}{draft}|\\
|\||else|\\
|\providecommand{\version}{final}|\\
|\||fi|
\end{tabular}
\end{center}
%
The definition by |\providecommand| makes sure
that previous definitions are not overwritten.
Further statements |\providecommand{\version}{...}|
can thus be added before the above code to override it.

For the main file, one might add a line
(between |\childdocmain| and the above block)
%
\begin{center}
|%\ifchilddoc\||else\providecommand{\version}{draft}\||fi|
\end{center}
%
which can be uncommented to produce a draft version.
Likewise one can add a line to the very top of a child file
(above the |\childdocof{|\textit{main}|}| directive)
%
\begin{center}
|%\providecommand{\version}{final}|
\end{center}
%
which can be uncommented to produce the final version of this child document.

%%%%%%%%%%%%%%%%%%%%%%%%%%%%%%%%%%%%%%%%%%%%%%%%%%%%%%%%%%%%%%%%%%%%%%%%%%%%%%%%
\subsection{Forwarding}
\label{sec:forward}

Different versions of the main or child documents
using compilation flags as described in \secref{sec:flags}
can be (permanently) stored in different files
for convenient compilation, viewing and distribution.
To this end, the package defines a command
to pass on compilation to a different file:

%%%%%%%%%%%%%%%%%%%%%%%%%%%%%%%%%%%%%%%%
\DescribeMacro{\childdocforward}
The command |\childdocforward| redirects processing to
another source file:
%
\begin{center}
\begin{tabular}{l}
|\input{childdoc.def}|\\
|\childdocforward[|\textit{main}|]{|\textit{dest}|}|\\
\end{tabular}
\end{center}
%
The argument \textit{dest} is the destination file
(without extension).
It should be the main file or one of the child files.
Note that further \textsf{childdoc} directives
such as |\childdocof| and |\childdocforward|
in the indicated file will be processed in this form.
The optional argument \textit{main}
passes on directly to the main file \textit{main}
while pretending to compile the child \textit{dest}.
This form behaves as if \textit{dest}
issues |\childdocof{|\textit{main}|}| right away,
and no further \textsf{childdoc} directives will be processed.

%%%%%%%%%%%%%%%%%%%%%%%%%%%%%%%%%%%%%%%%
\DescribeMacro{\...prefix}
In the alternative form |\childdocforwardprefix|,
%
\begin{center}
\begin{tabular}{l}
|\input{childdoc.def}|\\
|\childdocforwardprefix[|\textit{main}|]{|\textit{prefix}|}{|\textit{dest}|}|
\end{tabular}
\end{center}
%
the destination file is determined by a pattern
depending on the current file:
To make this work, the current file must be called
`{\textit{prefix}\hspace{0.2em}\textit{suffix}}'
with \textit{prefix} matching precisely the argument.
Processing is then passed on to the file
`{\textit{dest}\hspace{0.2em}\textit{suffix}}'.
Surely, the same effect is achieved by
directly specifying the
argument `{\textit{dest}\hspace{0.2em}\textit{suffix}}'
in the first form.
However, that requires to set up a different file
for each child. With the alternative form of the command
all these files can have exactly the same content
which simplifies setting them up and maintaining them.

For example, the following file |draft.tex|
with a compilation flag |\version| as described in \secref{sec:flags}
compiles the main document as a draft:
%
\begin{center}
\begin{tabular}{l}
|\def\version{draft}|\\
|\input{childdoc.def}|\\
|\childdocforward{|\textit{main}|}|
\end{tabular}
\end{center}
%
Likewise, the following files |final|\textit{nn}|.tex|
compile the final version of the child document
|child|\textit{nn}|.tex|:
%
\begin{center}
\begin{tabular}{l}
|\def\version{final}|\\
|\input{childdoc.def}|\\
|\childdocforwardprefix{final}{child}|
\end{tabular}
\end{center}
%

Note that when several versions of a main file and/or of each child file
are to be generated, it may be convenient to set up a |Makefile| or
shell script to automatise the process.

%%%%%%%%%%%%%%%%%%%%%%%%%%%%%%%%%%%%%%%%%%%%%%%%%%%%%%%%%%%%%%%%%%%%%%%%%%%%%%%%
\subsection{Command Line Processing}
\label{sec:commandline}

The effect of redirection files can also be achieved by invoking
the \LaTeX{} compiler with a more elaborate command line.
Most conveniently this should be done as part
of a shell script or a |Makefile|.

When using \textsf{childdoc} in the main file, the following
command lines effectively perform a redirection
(note that depending on the shell being used,
backslashes may have to be doubled: `|\|' $\to$ `|\\|'):
%
\begin{center}
|... -jobname "|\textit{target}|" |\\|"|[\textit{flags}]%
|\input{childdoc.def}\childdocforward[|\textit{main}|]{|\textit{dest}|}"|
\end{center}
%
Here \textit{target} is the name of the output file,
\textit{main} is the name of the main file
and \textit{dest} is the name of the main or child file to be processed
(all filenames without extensions).
The optional argument \textit{main} can be omitted
if \textit{main} matches \textit{dest}.
Optionally, compilation \textit{flags} can be defined via |\def| commands.
This command line makes the \TeX{} engine believe
it is compiling the file \textit{target}
whose content is specified as the latter parameter.
The provided code then forwards the processing to
\textit{main} or \textit{dest} as described in \secref{sec:forward}.

%%%%%%%%%%%%%%%%%%%%%%%%%%%%%%%%%%%%%%%%%%%%%%%%%%%%%%%%%%%%%%%%%%%%%%%%%%%%%%%%
\subsection{Include by Input}
\label{sec:input}

Including child documents by |\include| has some restrictions by design.
Most notably, the content of a child document always occupies
its own set of pages; pages cannot be shared between child documents.
Usually, this behaviour makes perfect sense
because each child document contain an essential part of the document.
However, in some situations it may be desirable to compose
a document from a collection of parts
without having mandatory page breaks between then.
For this case, the package
provides a mechanism to include parts
by |\input| which can also be processed individually.
However, by construction this mechanism
requires manual handling of the content to be output.

%%%%%%%%%%%%%%%%%%%%%%%%%%%%%%%%%%%%%%%%
\DescribeMacro{\ifchilddocmanual}
The main file should be prepared as usual, see \secref{sec:include}.
However, the document body must make a distinction
between processing of an individual part and of the main document, e.g.:
%
\begin{center}
\begin{tabular}{l}
|\ifchilddocmanual|\\
|\input{\childdocname}|\\
|\||else|\\
\textit{document body with }|\input{|\textit{part}|}|\\
|\||fi|
\end{tabular}
\end{center}
%
The conditional |\ifchilddocmanual| is true whenever
a part to be included by |\input| is being compiled,
and the name of the part is stored in |\childdocname|.

%%%%%%%%%%%%%%%%%%%%%%%%%%%%%%%%%%%%%%%%
\DescribeMacro{\childdocby}
Each part to be included by |\input| should start with:
%
\begin{center}
\begin{tabular}{l}
|\input{childdoc.def}|\\
|\childdocby{|\textit{main}|}|\\
\end{tabular}
\end{center}
%
The directive |\childdocby| is similar to |\childdocof|
described in \secref{sec:include},
but the subsequent selection of content must be done manually.
To that end, both |\ifchilddoc| and |\ifchilddocmanual|
will be true upon processing of a part,
and the name of the part is stored in |\childdocname|.
Note that |\jobname| will be set to the filename of the current part
so that each part receives an individual |.aux| file
that does not interfere with the |.aux| file(s) of the main document.
This behaviour can be altered by the alternative form
|\childdocby[*]{|\textit{main}|}| (with a non-empty optional argument)
which uses the |.aux| file of the main document
by setting |\jobname| to \textit{main}.

%%%%%%%%%%%%%%%%%%%%%%%%%%%%%%%%%%%%%%%%%%%%%%%%%%%%%%%%%%%%%%%%%%%%%%%%%%%%%%%%
\subsection{Driver Development}
\label{sec:driver}

The \textsf{childdoc} mechanism can also be use for the development
of definition files such as \LaTeX{} styles or classes.
This case differs from the above setup with multiple parts
included by |\include| in that no |\includeonly| should be invoked.
This can be achieved by starting the include file
(before |\ProvidesPackage|) with:
%
\begin{center}
\begin{tabular}{l}
|\input{childdoc.def}|\\
|\childdocforward{|\textit{main}|}|\\
\end{tabular}
\end{center}
%
or alternatively with:
%
\begin{center}
\begin{tabular}{l}
|\input{childdoc.def}|\\
|\childdocby{|\textit{main}|}|\\
\end{tabular}
\end{center}
%
Both forms have slightly different effects as described above.
The main file is prepared as usual, see \secref{sec:include}.

%%%%%%%%%%%%%%%%%%%%%%%%%%%%%%%%%%%%%%%%%%%%%%%%%%%%%%%%%%%%%%%%%%%%%%%%%%%%%%%%
\subsection{Legacy Detection}
\label{sec:detection}

The directive |\childdocmain| in the main file can detect
whether the complete document or merely a child is to be compiled
even without using the directive |\childdocof|.
This method is deprecated because it is less robust
and there is no compelling reason to use it;
it is merely provided for backward compatibility
and it may be removed in future versions.

If the detection mechanism is to be used,
it is mandatory to correctly specify
the filename of the main file as the argument of |\childdocmain|:
%
\begin{center}
\begin{tabular}{l}
|\input{childdoc.def}|\\
|\childdocmain{|\textit{main}|}|\\
\end{tabular}
\end{center}
%
If |\jobname| does not match the argument \textit{main} of |\childdocmain|,
it is assumed that |\jobname| points to the child file to be compiled.
When using |\childdocmain| with the main file specified as argument,
it suffices to start a child file
with just |\input{|\textit{main}|}|
without loading of the package and using |\childdocof|.
If instead all processing is done
with the appropriate \textsf{childdoc} directives,
the argument of \textit{main} of |\childdocmain| can be empty.

An alternative version of the command line processing described
in \secref{sec:commandline} using the detection mechanism reads:
%
\begin{center}
|... -jobname "|\textit{target}|" "|[\textit{flags}]%
[|\def\jobname{|\textit{dest}|}|]|\input{|\textit{main}|}"|
\end{center}

%%%%%%%%%%%%%%%%%%%%%%%%%%%%%%%%%%%%%%%%%%%%%%%%%%%%%%%%%%%%%%%%%%%%%%%%%%%%%%%%
\subsection{Manual Code}
\label{sec:manual}

In case one cannot be certain whether the definitions file |childdoc.def|
is installed on the target \TeX{} distribution
and one prefers not to ship it,
it is conceivable to paste a few relevant commands into the sources.

To that end, drop all statements |\input{childdoc.def}|
and perform the replacements as outlined below.
Instead of |\childdocmain{|\textit{main}|}| add the following code
to the top of the main file:
%
\begin{center}
\begin{tabular}{l}
|\||ifdefined\childdocname\endinput\||fi\newif\ifchilddoc|\\
|\edef\childdocname{\scantokens\expandafter{\jobname\noexpand}}|\\
|\def\childdocmain{|\textit{main}|}\||ifx\childdocmain\childdocname\||else|\\
|\childdoctrue\includeonly{\childdocname}\let\jobname\childdocmain\||fi|\\
\end{tabular}
\end{center}
%
Instead of |\childdocof{|\textit{main}|}| just include the main file
at the top of each child file:
%
\begin{center}
|\input{|\textit{main}|}|
\end{center}
%
A simple redirection |\childdocforward{|\textit{dest}|}| is achieved by:
%
\begin{center}
|\def\jobname{|\textit{dest}|}\input{\jobname}|
\end{center}
%
The redirection with prefix
|\childdocforwardprefix[|\textit{prefix}|]{|\textit{dest}|}|
is accomplished by:
%
\begin{center}
\begin{tabular}{l}
|{\edef\jobname{\scantokens\expandafter{\jobname\noexpand}}|\\
|\def\redirectjob |\textit{prefix}|#1~~~{\gdef\jobname{|\textit{dest}|#1}}|\\
|\expandafter\redirectjob\jobname~~~}\input{\jobname}|
\end{tabular}
\end{center}

In an alternative approach,
child documents can be compiled by a specific command line
without additional code or specific definitions:
%
\begin{center}
|... -jobname "|\textit{target}|" "|[\textit{flags}]%
|\includeonly{|\textit{dest}|}\input{|\textit{main}|}"|
\end{center}
%

%%%%%%%%%%%%%%%%%%%%%%%%%%%%%%%%%%%%%%%%%%%%%%%%%%%%%%%%%%%%%%%%%%%%%%%%%%%%%%%%
%%%%%%%%%%%%%%%%%%%%%%%%%%%%%%%%%%%%%%%%%%%%%%%%%%%%%%%%%%%%%%%%%%%%%%%%%%%%%%%%
\section{Information}

%%%%%%%%%%%%%%%%%%%%%%%%%%%%%%%%%%%%%%%%%%%%%%%%%%%%%%%%%%%%%%%%%%%%%%%%%%%%%%%%
\subsection{Copyright}

Copyright \copyright{} 2017--2018 Niklas Beisert

This work may be distributed and/or modified under the
conditions of the \LaTeX{} Project Public License, either version 1.3
of this license or (at your option) any later version.
The latest version of this license is in
  \url{http://www.latex-project.org/lppl.txt}
and version 1.3 or later is part of all distributions of \LaTeX{}
version 2005/12/01 or later.

This work has the LPPL maintenance status `maintained'.

The Current Maintainer of this work is Niklas Beisert.

This work consists of the files |README.txt|, |childdoc.ins| and |childdoc.dtx|
as well as the derived files |childdoc.def|, |cdocsamp.tex|
with |cdocsch1.tex|, |cdocsch2.tex|, |cdocspt3.tex|, |cdocspt4.tex|,
|cdocsdrf.tex|, |cdocsfn1.tex|, |cdocsfn2.tex|
as well as |childdoc.pdf|.

%%%%%%%%%%%%%%%%%%%%%%%%%%%%%%%%%%%%%%%%%%%%%%%%%%%%%%%%%%%%%%%%%%%%%%%%%%%%%%%%
\subsection{Files and Installation}

The package consists of the files:
%
\begin{center}
\begin{tabular}{ll}
    |README.txt|   & readme file \\
    |childdoc.ins| & installation file \\
    |childdoc.dtx| & source file \\
    |childdoc.def| & definition file \\
    |cdocsamp.tex| & sample main file \\
    |cdocsch1.tex| & sample include file \\
    |cdocsch2.tex| & sample include file \\
    |cdocspt3.tex| & sample part file \\
    |cdocspt4.tex| & sample part file \\
    |cdocsdrf.tex| & sample redirection file \\
    |cdocsfn1.tex| & sample redirection file \\
    |cdocsfn2.tex| & sample redirection file \\
    |childdoc.pdf| & manual
\end{tabular}
\end{center}
%
The distribution consists of the files
|README.txt|, |childdoc.ins| and |childdoc.dtx|.
%
\begin{itemize}
\item
Run (pdf)\LaTeX{} on |childdoc.dtx|
to compile the manual |childdoc.pdf| (this file).
\item
Run \LaTeX{} on |childdoc.ins| to create the definitions file |childdoc.def|
and the sample |cdocsamp.tex| with include files
|cdocsch1.tex|, |cdocsch2.tex|, |cdocspt3.tex|, |cdocspt4.tex|,
|cdocsdrf.tex|, |cdocsfn1.tex|, |cdocsfn2.tex|.
Then copy the file |childdoc.def| to an appropriate directory of your \LaTeX{}
distribution, e.g.\ \textit{texmf-root}|/tex/latex/childdoc|.
\end{itemize}

%%%%%%%%%%%%%%%%%%%%%%%%%%%%%%%%%%%%%%%%%%%%%%%%%%%%%%%%%%%%%%%%%%%%%%%%%%%%%%%%
\subsection{Related CTAN Packages}

There are several other packages which offer a similar functionality:
%
\begin{itemize}
\item
The packages
\href{http://ctan.org/pkg/docmute}{\textsf{docmute}},
\href{http://ctan.org/pkg/includex}{\textsf{includex}} and
\href{http://ctan.org/pkg/standalone}{\textsf{standalone}}
provide commands to include only the document body of
a child file thus allowing both files to be compiled individually.
\item
The packages \href{http://ctan.org/pkg/subdocs}{\textsf{subdocs}}
and \href{http://ctan.org/pkg/subfiles}{\textsf{subfiles}}
provide structures in which the main and child documents can be
encapsulated and allowing them to be compiled individually.
The inclusion mechanism is different from the conventional |\include|.
\item
The package \href{http://ctan.org/pkg/combine}{\textsf{combine}}
is an elaborate solution to combine several documents into one.
\end{itemize}
%
See also the CTAN topic \href{http://ctan.org/topic/subdocs}{\textsf{subdocs}}
for further related packages.
The present package differs from the above solutions in that
a document structure constructed with the conventional |\include| mechanism
just needs two extra commands at the top of every file
such that all constituent files can be compiled individually.

%%%%%%%%%%%%%%%%%%%%%%%%%%%%%%%%%%%%%%%%%%%%%%%%%%%%%%%%%%%%%%%%%%%%%%%%%%%%%%%%
%\subsection{Feature Suggestions}
%
%The following is a list of features which may be useful for future
%versions of this package:
%%
%\begin{itemize}
%\item
%\ldots
%\end{itemize}

%%%%%%%%%%%%%%%%%%%%%%%%%%%%%%%%%%%%%%%%%%%%%%%%%%%%%%%%%%%%%%%%%%%%%%%%%%%%%%%%
\subsection{Revision History}

%%%%%%%%%%%%%%%%%%%%%%%%%%%%%%%%%%%%%%%%
\paragraph{v2.0:} 2018/12/30

\begin{itemize}
\item
immediate forward processing
\item
added |\childdocby| mechanism
\item
manual restructured
\end{itemize}

%%%%%%%%%%%%%%%%%%%%%%%%%%%%%%%%%%%%%%%%
\paragraph{v1.6:} 2018/01/17

\begin{itemize}
\item
application for development of include files
\item
corrections to manual
\end{itemize}

%%%%%%%%%%%%%%%%%%%%%%%%%%%%%%%%%%%%%%%%
\paragraph{v1.5:} 2017/05/21

\begin{itemize}
\item
more complete structuring introduced
\item
|\childdocof| introduced
\item
|\childdoc| renamed to |\childdocmain|
\item
|\childredirect| renamed to |\childdocforward| and |\childdocforwardprefix|
and functionality expanded
\end{itemize}

%%%%%%%%%%%%%%%%%%%%%%%%%%%%%%%%%%%%%%%%
\paragraph{v1.0:} 2017/04/27

\begin{itemize}
\item
manual and install package
\item
first version published on CTAN
\end{itemize}

%%%%%%%%%%%%%%%%%%%%%%%%%%%%%%%%%%%%%%%%
\paragraph{v0.6:} 2017/04/26

\begin{itemize}
\item
redirection mechanism added
\end{itemize}

%%%%%%%%%%%%%%%%%%%%%%%%%%%%%%%%%%%%%%%%
\paragraph{v0.5:} 2017/04/26

\begin{itemize}
\item
functionality in definition file
\end{itemize}


%%%%%%%%%%%%%%%%%%%%%%%%%%%%%%%%%%%%%%%%%%%%%%%%%%%%%%%%%%%%%%%%%%%%%%%%%%%%%%%%
%%%%%%%%%%%%%%%%%%%%%%%%%%%%%%%%%%%%%%%%%%%%%%%%%%%%%%%%%%%%%%%%%%%%%%%%%%%%%%%%
%%%%%%%%%%%%%%%%%%%%%%%%%%%%%%%%%%%%%%%%%%%%%%%%%%%%%%%%%%%%%%%%%%%%%%%%%%%%%%%%
\appendix

\settowidth\MacroIndent{\rmfamily\scriptsize 000\ }

 \DocInput{childdoc.dtx}

\end{document}
%</driver>
% \fi
%
% %%%%%%%%%%%%%%%%%%%%%%%%%%%%%%%%%%%%%%%%%%%%%%%%%%%%%%%%%%%%%%%%%%%%%%%%%%%%%%
% %%%%%%%%%%%%%%%%%%%%%%%%%%%%%%%%%%%%%%%%%%%%%%%%%%%%%%%%%%%%%%%%%%%%%%%%%%%%%%
% \section{Sample}
%\iffalse
%<*samplemain>
%\fi
%
% The following presents a sample document
% with two chapters, two parts, a title page,
% a compile flag as well as three forwarding files to set the flag.
% It consists of eight |.tex| files:
% \begin{center}
% \begin{tabular}{ll}
% |cdocsamp.tex|&main file\\
% |cdocsch1.tex|&include file for chapter 1\\
% |cdocsch2.tex|&include file for chapter 2\\
% |cdocspt3.tex|&include file for part 3\\
% |cdocspt4.tex|&include file for part 4\\
% |cdocsdrf.tex|&forwarding file for main file in draft mode\\
% |cdocsfi1.tex|&forwarding file for final version of chapter 1\\
% |cdocsfi2.tex|&forwarding file for final version of chapter 2\\
% \end{tabular}
% \end{center}
% Each of the eight files can be compiled directly by the \LaTeX{} compiler.
%
% %%%%%%%%%%%%%%%%%%%%%%%%%%%%%%%%%%%%%%
% \paragraph{Main File.}
%
% The main file is called |cdocsamp.tex|.
%
% Load the \textsf{childdoc} definitions and
% declare the filename for the main document:
%    \begin{macrocode}
\input{childdoc.def}
\childdocmain{}
%    \end{macrocode}

% Optional override for |\version| flag:
%    \begin{macrocode}
%%\ifchilddoc\else\providecommand{\version}{draft}\fi
%    \end{macrocode}

% Define the default values for the |\version| flag
% (|final| for the main file and |draft| for childs):
%    \begin{macrocode}
\ifchilddoc
\providecommand{\version}{draft}
\else
\providecommand{\version}{final}
\fi
%    \end{macrocode}

% Load the standard document class:
%    \begin{macrocode}
\documentclass[12pt]{article}
%    \end{macrocode}

% Start the document body:
%    \begin{macrocode}
\begin{document}
%    \end{macrocode}

% Declare a title page.
% Print title, part of document being processed and version flag:
%    \begin{macrocode}
\addtocounter{page}{-1}
\begin{center}
{\LARGE\bfseries{}childdoc example\par}
\vspace{1cm}
\ifchilddoc
\ifchilddocmanual part\else chapter\fi:
`\childdocname' of `\childdocjob'\par
\else
main document: `\childdocjob'\par
\fi
version: \version\par
\end{center}
\newpage
%    \end{macrocode}

% Manually include selected file,
% otherwise process as usual:
%    \begin{macrocode}
\ifchilddocmanual
\section*{part `\childdocname'}
\input{\childdocname}
\else
%    \end{macrocode}

% Include the two chapters:
%    \begin{macrocode}
\include{cdocsch1}
\include{cdocsch2}
%    \end{macrocode}

% Include the two parts unless only chapters should be displayed:
%    \begin{macrocode}
\ifchilddoc\else
\section{part three}
\input{cdocspt3}
\section{part four}
\input{cdocspt4}
\fi
%    \end{macrocode}

% Process as usual until here:
%    \begin{macrocode}
\fi
%    \end{macrocode}

% End of document body:
%    \begin{macrocode}
\end{document}
%    \end{macrocode}
%\iffalse
%</samplemain>
%\fi
%
% %%%%%%%%%%%%%%%%%%%%%%%%%%%%%%%%%%%%%%
% \paragraph{Chapter Include Files.}
%
% The include files are called |cdocsch1.tex| and |cdocsch2.tex|.
%
%\iffalse
%<*samplechap1|samplechap2>
%\fi

% Optional override for |\version| flag:
%    \begin{macrocode}
%%\providecommand{\version}{final}
%    \end{macrocode}

% Include the main document:
%    \begin{macrocode}
\input{childdoc.def}
\childdocof{cdocsamp}
%    \end{macrocode}

%\iffalse
%</samplechap1|samplechap2>
%\fi
%
%\iffalse
%<*samplechap1>
%\fi
% Some text for chapter 1:
%    \begin{macrocode}
\section{one}
some text in chapter one
%    \end{macrocode}

%\iffalse
%</samplechap1>
%\fi
% Some text for chapter 2:
%\iffalse
%<*samplechap2>
%\fi
%    \begin{macrocode}
\section{two}
more text in chapter two
%    \end{macrocode}

%\iffalse
%</samplechap2>
%\fi
%
% %%%%%%%%%%%%%%%%%%%%%%%%%%%%%%%%%%%%%%
% \paragraph{Part Include Files.}
%
% The include files are called |cdocspt3.tex| and |cdocspt4.tex|.
%
%\iffalse
%<*samplepart3|samplepart4>
%\fi

% Optional override for |\version| flag:
%    \begin{macrocode}
%%\providecommand{\version}{final}
%    \end{macrocode}

% Include the main document:
%    \begin{macrocode}
\input{childdoc.def}
\childdocby{cdocsamp}
%    \end{macrocode}

%\iffalse
%</samplepart3|samplepart4>
%\fi
%
%\iffalse
%<*samplepart3>
%\fi
% Some text for part 3:
%    \begin{macrocode}
some text in part three
%    \end{macrocode}

%\iffalse
%</samplepart3>
%\fi
% Some text for part 4:
%\iffalse
%<*samplepart4>
%\fi
%    \begin{macrocode}
more text in part four
%    \end{macrocode}

%\iffalse
%</samplepart4>
%\fi
%
% %%%%%%%%%%%%%%%%%%%%%%%%%%%%%%%%%%%%%%
% \paragraph{Forwarding for a Complete Draft.}
%
% The following forwarding file |cdocsdrf.tex|
% compiles the main document in draft mode:
%\iffalse
%<*sampledraft>
%\fi
%    \begin{macrocode}
\def\version{draft}
\input{childdoc.def}
\childdocforward{cdocsamp}
%    \end{macrocode}

%\iffalse
%</sampledraft>
%\fi
%
% %%%%%%%%%%%%%%%%%%%%%%%%%%%%%%%%%%%%%%
% \paragraph{Forwarding for Final Version of the Chapters.}
%
% The following forwarding files |cdocsfn1.tex| and |cdocsfn2.tex|
% (with identical content)
% compile the final versions of the child documents
% |cdocsch1.tex| and |cdocsch2.tex|, respectively:
%\iffalse
%<*samplefinal>
%\fi
%    \begin{macrocode}
\def\version{final}
\input{childdoc.def}
\childdocforwardprefix[cdocsamp]{cdocsfn}{cdocsch}
%    \end{macrocode}

%\iffalse
%</samplefinal>
%\fi
%
% %%%%%%%%%%%%%%%%%%%%%%%%%%%%%%%%%%%%%%
% \paragraph{Command Line Processing.}
%
% The following three command lines generate the output files
% |cdocscld|, |cdocscl1| and |cdocscl2|
% which should be identical to
% |cdocsdrf|, |cdocsch1| and |cdocsfn2|, respectively:
% \begin{center}
% \begin{tabular}{l}
% |latex -jobname cdocscld \|\\
% |  "\def\version{draft}\input{childdoc.def}\childdocforward{cdocsamp}"|\\
% |latex -jobname cdocscl1 \|\\
% |  "\input{childdoc.def}\childdocforward[cdocsamp]{cdocsch1}"|\\
% |latex -jobname cdocscl2 \|\\
% |  "\def\version{final}\input{childdoc.def}\childdocforward{cdocsch2}"|
% \end{tabular}
% \end{center}
% Note that the trailing backslash on each first line
% merely continues the input to the second line
% (for convenient cut ant paste).
% Furthermore, the command |latex| can be replaced by any
% of its alternative versions such as |pdflatex|.
%
% %%%%%%%%%%%%%%%%%%%%%%%%%%%%%%%%%%%%%%%%%%%%%%%%%%%%%%%%%%%%%%%%%%%%%%%%%%%%%%
% %%%%%%%%%%%%%%%%%%%%%%%%%%%%%%%%%%%%%%%%%%%%%%%%%%%%%%%%%%%%%%%%%%%%%%%%%%%%%%
% \section{Implementation}
%\iffalse
%<*package>
%\fi
%
% This section describes the definitions file |childdoc.def|.

% The definitions cannot be loaded using |\usepackage| or |\RequirePackage|
% which has a mechanism to prevent loading a style file more than once.
% When loading the definitions by means of |\input|
% multiple instances have to be prevented manually:
%\iffalse
%This code needs to be before the `\ProvidesFile' directive
%which is defined at the beginning of this file.
%Therefore it is also placed there and commented out here.
%</package>
%<*discard>
%\fi
%    \begin{macrocode}
\ifdefined\childdocmain\endinput\fi
%    \end{macrocode}
%\iffalse
%</discard>
%<*package>
%\fi
%
% \macro{\ifchilddoc}
% \macro{\ifchilddocmanual}
% The conditional |\ifchilddoc| tells whether a
% child (true) or main (false) document is being compiled.
% The conditional |\ifchilddocmanual| tells whether
% the |\includeonly| mechanism is used (false) or
% the selection of child files must be performed manually (true).
% The definitions initialise to false:
%    \begin{macrocode}
\newif\ifchilddoc
\newif\ifchilddocmanual
%    \end{macrocode}

% \macro{\childdocname}
% \macro{\childdocjob}
% The macro |\childdocname| stores the name of the main document
% to be compiled. The macro |\childdocjob| stores the name of
% the document on which the \LaTeX{} compiler was originally invoked.
% The content of |\jobname| cannot be compared
% to filenames specified in the source due to different catcodes.
% The following code rescans |\jobname|, stores the result
% in |\childdocname| and saves a copy in |\childdocjob|:
%    \begin{macrocode}
\edef\childdocname{\scantokens\expandafter{\jobname\noexpand}}
\let\childdocjob\childdocname
%    \end{macrocode}

% \macro{\childdocdisable}
% The macro |\childdocdisable| prevents the main file
% from being processed more than once.
% At this stage, the main document command |\childdocmain|
% is assumed to be called once again where it should do nothing.
% Any subsequent call to it should prevent
% a secondary processing of the main document
% It overwrites the forwarding commands
% |\childdocof| and |\childdocforward|
% with empty macros to prevent further inclusions of the main document:
%    \begin{macrocode}
\newcommand{\childdocdisable}
{
  \renewcommand{\childdocmain}[1]{\renewcommand{\childdocmain}[1]{\endinput}}
  \renewcommand{\childdocof}[1]{}
  \renewcommand{\childdocby}[2][]{}
  \renewcommand{\childdocforward}[2][]{}
  \renewcommand{\childdocdisable}{}
}
%    \end{macrocode}

% \macro{\childdocmain}
% The macro |\childdocmain| is to be called at the top of the main file
% with nothing or the main filename (without extension) as argument.
% First, it breaks loops.
% If the argument is not empty and does not match |\childdocname|
% (which is set by the first inclusion of |childdoc.def|),
% |\ifchilddoc| is set to true, |\includeonly| is applied to the child file
% and |\jobname| is set to the main file
% (for proper handling of |.aux| files):
%    \begin{macrocode}
\newcommand{\childdocmain}[1]
{
  \childdocdisable\childdocmain{}
  \if?#1?\else
    \begingroup
      \def\childdoctmp{#1}
      \ifx\childdoctmp\childdocname
        \def\childdoctmp{}
      \else
        \def\childdoctmp
        {
          \childdoctrue
          \includeonly{\childdocname}
          \def\childdocjob{#1}
          \def\jobname{#1}
        }
      \fi
      \expandafter
    \endgroup
    \childdoctmp
  \fi
}
%    \end{macrocode}

% \macro{\childdocof}
% The command |\childdocof| redirects
% compilation to the main file |#1|.
%    \begin{macrocode}
\newcommand{\childdocof}[1]
{
  \childdocdisable
  \childdoctrue
  \includeonly{\childdocname}
  \def\jobname{#1}
  \def\childdocjob{#1}
  \input{#1}
}
%    \end{macrocode}

% \macro{\childdocby}
% The command |\childdocby| ....
%    \begin{macrocode}
\newcommand{\childdocby}[2][]
{
  \childdocdisable
  \childdoctrue
  \childdocmanualtrue
  \if?#1?\else
    \def\jobname{#2}
  \fi
  \def\childdocjob{#2}
  \input{#2}
  \endinput
}
%    \end{macrocode}

% \macro{\childdocforward}
% The command |\childdocforward| redirects
% compilation to the main file or
% (if the optional argument is given) a child file.
% Parameters are set as if the main file
% or a child file starting with |\childdocof| was compiled.
% Then compilation is handed over to the main file:
%    \begin{macrocode}
\newcommand{\childdocforward}[2][]
{
  \begingroup
    \if?#1?
      \def\childdoctmp
      {
        \def\childdocname{#2}
        \def\childdocjob{#2}
        \def\jobname{#2}
        \input{#2}
        \endinput
      }
    \else
      \def\childdoctmp
      {
        \childdocdisable
        \def\childdocname{#2}
        \childdoctrue
        \includeonly{#2}
        \def\childdocjob{#1}
        \def\jobname{#1}
        \input{#1}
        \endinput
      }
    \fi
    \expandafter
  \endgroup
  \childdoctmp
}
%    \end{macrocode}

% \macro{\childdocforwardprefix}
% The command |\childdocforwardprefix| redirects
% compilation to the main or a child file by means of a pattern.
% The prefix |#1| in the current filename is replaced by |#2|
% and the suffix of the current filename is kept
% (it is assumed that the filename does not contain the substring `|~~~|'
% which is used as a delimiter).
% Compilation is handed over to the new file by |\childdocforward|:
%    \begin{macrocode}
\newcommand{\childdocforwardprefix}[3][]
{
  \begingroup
    \def\childdocextract #2##1~~~{\def\childdoctmp{\childdocforward[#1]{#3##1}}}
    \expandafter\childdocextract\childdocname~~~
    \expandafter
  \endgroup
  \childdoctmp
}
%    \end{macrocode}

% \macro{\childdoc}
% The deprecated macro |\childdoc| is a legacy version of |\childdocmain|:
%    \begin{macrocode}
\newcommand{\childdoc}{\childdocmain}
%    \end{macrocode}

% \macro{\childdocredirect}
% The deprecated macro |\childdocredirect| is a legacy version
% of |\childdocforward| and |\childdocforwardprefix|:
%    \begin{macrocode}
\newcommand{\childdocredirect}[2][]
{
  \begingroup
    \if?#1?
      \def\childdoctmp{\childdocforward{#2}}
    \else
      \def\childdoctmp{\childdocforwardprefix{#1}{#2}}
    \fi
    \expandafter
  \endgroup
  \childdoctmp
}
%    \end{macrocode}

%\iffalse
%</package>
%\fi
%
\endinput

\childdocforwardprefix[cdocsamp]{cdocsfn}{cdocsch}
%    \end{macrocode}

%\iffalse
%</samplefinal>
%\fi
%
% %%%%%%%%%%%%%%%%%%%%%%%%%%%%%%%%%%%%%%
% \paragraph{Command Line Processing.}
%
% The following three command lines generate the output files
% |cdocscld|, |cdocscl1| and |cdocscl2|
% which should be identical to
% |cdocsdrf|, |cdocsch1| and |cdocsfn2|, respectively:
% \begin{center}
% \begin{tabular}{l}
% |latex -jobname cdocscld \|\\
% |  "\def\version{draft}% \iffalse
%
% childdoc.dtx Copyright (C) 2017-2018 Niklas Beisert
%
% This work may be distributed and/or modified under the
% conditions of the LaTeX Project Public License, either version 1.3
% of this license or (at your option) any later version.
% The latest version of this license is in
%   http://www.latex-project.org/lppl.txt
% and version 1.3 or later is part of all distributions of LaTeX
% version 2005/12/01 or later.
%
% This work has the LPPL maintenance status `maintained'.
%
% The Current Maintainer of this work is Niklas Beisert.
%
% This work consists of the files childdoc.dtx and childdoc.ins
% and the derived files childdoc.def and cdocsamp.tex with
% cdocsch1.tex, cdocsch2.tex, cdocsdrf.tex, cdocsfn1.tex, cdocsfn2.tex.
%
%<package>\ifdefined\childdocmain\endinput\fi
%<package>\ProvidesFile{childdoc.def}[2018/12/30 v2.0 child document driver]
%<samplemain>\ProvidesFile{cdocsamp.tex}[2018/12/30 v2.0 sample for childdoc]
%<*driver>
%\ProvidesFile{childdoc.drv}[2018/12/30 v2.0 childdoc reference manual file]
\PassOptionsToClass{10pt,a4paper}{article}
\documentclass{ltxdoc}

\usepackage[margin=35mm]{geometry}
\usepackage{hyperref}
\usepackage{hyperxmp}
\usepackage[usenames]{color}

\hypersetup{colorlinks=true}
\hypersetup{pdfstartview=FitH}
\hypersetup{pdfpagemode=UseNone}
\hypersetup{pdfsource={}}
\hypersetup{pdflang={en-UK}}
\hypersetup{pdfcopyright={Copyright 2017-2018 Niklas Beisert.
  This work may be distributed and/or modified under the
  conditions of the LaTeX Project Public License, either version 1.3
  of this license or (at your option) any later version.}}
\hypersetup{pdflicenseurl={http://www.latex-project.org/lppl.txt}}
\hypersetup{pdfcontactaddress={ETH Zurich, ITP, HIT K,
  Wolfgang-Pauli-Strasse 27}}
\hypersetup{pdfcontactpostcode={8093}}
\hypersetup{pdfcontactcity={Zurich}}
\hypersetup{pdfcontactcountry={Switzerland}}
\hypersetup{pdfcontactemail={nbeisert@itp.phys.ethz.ch}}
\hypersetup{pdfcontacturl={http://people.phys.ethz.ch/\xmptilde nbeisert/}}

\newcommand{\secref}[1]{\hyperref[#1]{section \ref*{#1}}}

\parskip1ex
\parindent0pt
\let\olditemize\itemize
\def\itemize{\olditemize\parskip0pt}

\begin{document}

\title{The \textsf{childdoc} Package}
\hypersetup{pdftitle={The childdoc Package}}
\author{Niklas Beisert\\[2ex]
  Institut f\"ur Theoretische Physik\\
  Eidgen\"ossische Technische Hochschule Z\"urich\\
  Wolfgang-Pauli-Strasse 27, 8093 Z\"urich, Switzerland\\[1ex]
  \href{mailto:nbeisert@itp.phys.ethz.ch}
  {\texttt{nbeisert@itp.phys.ethz.ch}}}
\hypersetup{pdfauthor={Niklas Beisert}}
\hypersetup{pdfsubject={Manual for the LaTeX2e Package childdoc}}
\date{30 December 2018, \textsf{v2.0}}
\maketitle

\begin{abstract}\noindent
\textsf{childdoc} is a \LaTeXe{} package
that enables the direct compilation
of document sections included by |\include|
to individual files.
\end{abstract}

\begingroup
\parskip0ex
\tableofcontents
\endgroup

%%%%%%%%%%%%%%%%%%%%%%%%%%%%%%%%%%%%%%%%%%%%%%%%%%%%%%%%%%%%%%%%%%%%%%%%%%%%%%%%
%%%%%%%%%%%%%%%%%%%%%%%%%%%%%%%%%%%%%%%%%%%%%%%%%%%%%%%%%%%%%%%%%%%%%%%%%%%%%%%%
\section{Introduction}

\LaTeX{} provides a mechanism to structure a large document (such as a book)
into a main file and several child files (containing the chapters)
using the |\include| command.
This mechanism is beneficial for documents
which span hundreds of pages in order to
make the source file(s) more manageable.
Moreover, compilation can be restricted to
selected child files by means of the |\includeonly| command.
The latter feature can be used to reduce the compilation time while editing
(this was significantly more useful in the earlier days of \LaTeX{})
or to generate a smaller document which is easier to navigate.
Another application of |\includeonly| is to generate
documents consisting of selected parts of the complete document.

However, there are a few drawbacks of the plain |\include| mechanism:
\begin{itemize}
\item
The child files cannot be compiled on their own,
they can only be compiled via the main file.
A naive editing environment
(such as a text editor with an option
to have the current file processed by \LaTeX)
may require one to switch to the main file before compiling;
attempting to compile the child file produces errors.
\item
The main file must be modified (each time)
to adjust the |\includeonly| command
to the present needs. This easily leaves the main file in a messy state.
\item
The generated document will always carry the filename
of the main document. This is inconvenient if
several child files are to be compiled and
to be kept for distribution.
\end{itemize}

The present package provides a simple interface
to make child files individually compilable by \LaTeX{}.
Compiling a child file then has the same effect as compiling
the main file with an |\includeonly| command
to select the appropriate child.
Moreover the generated document will carry the name of the child
rather than the main file.
This resolves all three above issues.

This feature is meant to make the editing of books,
thesis documents and lecture notes somewhat more convenient.
However, the package can also be used efficiently for
composing a series of documents (such as exercise sheets)
which are typically distributed individually.
It then assists the author in generating the individual documents
(potentially in different versions)
as well as a document containing the collected series.
Another application is in developing style files
or other kinds of included material
where compilation of the style file could redirect
to a sample or test file.

%%%%%%%%%%%%%%%%%%%%%%%%%%%%%%%%%%%%%%%%%%%%%%%%%%%%%%%%%%%%%%%%%%%%%%%%%%%%%%%%
%%%%%%%%%%%%%%%%%%%%%%%%%%%%%%%%%%%%%%%%%%%%%%%%%%%%%%%%%%%%%%%%%%%%%%%%%%%%%%%%
\section{Usage}

First of all, the package \textsf{childdoc} is \emph{not} a standard
\LaTeXe{} |.sty| style file! Therefore it needs to be invoked in
a non-standard way.

%%%%%%%%%%%%%%%%%%%%%%%%%%%%%%%%%%%%%%%%%%%%%%%%%%%%%%%%%%%%%%%%%%%%%%%%%%%%%%%%
\subsection{Included Files}
\label{sec:include}

%%%%%%%%%%%%%%%%%%%%%%%%%%%%%%%%%%%%%%%%
\DescribeMacro{\childdocmain}
To use the package, add the commands
\begin{center}
\begin{tabular}{l}
|\input{childdoc.def}|\\
|\childdocmain{}|\\
\end{tabular}
\end{center}
at the very top of the main \LaTeX{} file,
in particular \emph{before} the |\documentclass| statement!
The argument of |\childdocmain| should be left empty
(but it must be present).

%%%%%%%%%%%%%%%%%%%%%%%%%%%%%%%%%%%%%%%%
\DescribeMacro{\childdocof}
Furthermore, add the commands
\begin{center}
\begin{tabular}{l}
|\input{childdoc.def}|\\
|\childdocof{|\textit{main}|}|\\
\end{tabular}
\end{center}
at the top of every child file \textit{child}
which is included by |\include{|\textit{child}|}|
from within the main file
(or at least for those files to be compiled individually).
The argument \textit{main} must be the filename of the main file.

There are a couple of
considerations in setting up the main and child documents:

%%%%%%%%%%%%%%%%%%%%%%%%%%%%%%%%%%%%%%%%
\paragraph{Restrictions.}

Please note the following restrictions:
\begin{itemize}
\item
|\childdocmain| must be called with one argument \textit{main}
to ensure compatibility with earlier version of the package.
It must either be empty (|\childdocmain{}|)
or precisely match the filename of the main file in which it is specified.
See \secref{sec:detection} for further information.
\item
The filename \textit{main} must be specified without the |.tex| extension.
\item
The filename \textit{main} is case sensitive
(even in case-insensitive file systems)
due to internal string comparison.
\item
The argument \textit{main} should be fully expanded, it cannot be a macro.
\item
Subdirectories and special characters should be avoided in filenames.
\item
The command |\childdocmain{|\textit{main}|}| must be followed by a whitespace.
It should not be followed immediately by another command
or by a comment mark `|%|'.
This is because the \TeX{} parser reads the token immediately following
the argument of |\childdocmain| and puts it
at the beginning of every child section;
however, a white\-space is ignored.
\end{itemize}

%%%%%%%%%%%%%%%%%%%%%%%%%%%%%%%%%%%%%%%%
\paragraph{Content of Main File.}

It is advisable to place all content in the child files included by |\include|.
Any output contained in the main file will appear in all child documents
unless suppressed manually;
it cannot be suppressed automatically by the |\includeonly| directive
and thus should normally be avoided.
A method to include some content in the main file
by means of conditional processing is described in \secref{sec:conditional}.

%%%%%%%%%%%%%%%%%%%%%%%%%%%%%%%%%%%%%%%%
\paragraph{Page Numbering.}

When only a part of the document is compiled,
the appropriate numbering of pages
(as well as other status parameters)
is determined from the |.aux| files.
The latter contain information from previous passes.
However this information needs to propagate through
all intermediate child documents.
Therefore the page numbering in child documents may well
be inconsistent until the complete document is compiled at least once.

A useful (if unconventional) way to always ensure a consistent
page numbering is to restart the numbering in each child document
and denote the pages by `\textit{child}|.|\textit{page}'
where \textit{child} represents the chapter/section number of the child file.
This can be achieved by the command
|\numberwithin{page}{|\textit{child}|}|
of the \textsf{amsmath} package
where \textit{child} can be |chapter| or |section|
depending on the chosen structuring.
Alternatively, one can modify the macro |\thepage| appropriately
and reset the counter |page| at the start of each child file.

%%%%%%%%%%%%%%%%%%%%%%%%%%%%%%%%%%%%%%%%%%%%%%%%%%%%%%%%%%%%%%%%%%%%%%%%%%%%%%%%
\subsection{Conditional Processing}
\label{sec:conditional}

The package provides a mechanism to compile different versions
of a document. To customise the versions further some conditional processing
can come in handy to distinguish which version is being compiled.
The package provides two macros to describe the compilation context:

%%%%%%%%%%%%%%%%%%%%%%%%%%%%%%%%%%%%%%%%
\DescribeMacro{\ifchilddoc}
The conditional |\ifchilddoc| distinguishes between the compilation of
child documents and the main document:
%
\begin{center}
|\ifchilddoc |\textit{child-code}| |[|\||else |\textit{main-code}]| \||fi|
\end{center}

%%%%%%%%%%%%%%%%%%%%%%%%%%%%%%%%%%%%%%%%
\DescribeMacro{\childdocname}
\DescribeMacro{\childdocjob}
The macro |\childdocname| contains the filename (without extension)
of the main or child file being processed.
Note that |\childdocjob| will always contain the name of the main file.

%%%%%%%%%%%%%%%%%%%%%%%%%%%%%%%%%%%%%%%%
\paragraph{Title Page.}

Conditional processing can be used to include a title or banner page
in the main document when proper precautions are taken.
Importantly, the code in the main file should ensure that the page counter
(as well as other status parameters which are stored in the |.aux| files)
takes the same value after the conditional processing.
Otherwise the page numbers may take divergent values
depending on which part is compiled.

For example, a title page could be declared by:
%
\begin{center}
\begin{tabular}{l}
|\ifchilddoc\||else|\\
|\addtocounter{page}{-1}|\\
\textit{code for title page}\\
|\newpage|\\
|\||fi|
\end{tabular}
\end{center}
%
A banner page for the child documents can be generated by:
%
\begin{center}
\begin{tabular}{l}
|\ifchilddoc|\\
|\addtocounter{page}{-1}|\\
\textit{code for banner page}\\
|\newpage|\\
|\||fi|
\end{tabular}
\end{center}
%
Here one could write a message such as:
\begin{center}
|This is the part \childdocname{} of \childdocjob{}.|
\end{center}

%%%%%%%%%%%%%%%%%%%%%%%%%%%%%%%%%%%%%%%%%%%%%%%%%%%%%%%%%%%%%%%%%%%%%%%%%%%%%%%%
\subsection{Flags}
\label{sec:flags}

The package makes it easy to generate different versions
of the main or child documents.
To this end compilation flags can be defined
and assigned different default values.
They will be particularly useful in conjunction
with the forwarding mechanism described in \secref{sec:forward}.

For example, it may be useful to have a flag |\version|
which can be set to |draft| or |final|.
The document source will contain some conditional code
depending on the value of |\version|.
Suppose further, the flag should default to |final| for the main file
and to |draft| for child files
which is a natural assignment for editing the document.
This is achieved by placing the following code
in the preamble of the main document
(below the |\childdocmain| directive):
%
\begin{center}
\begin{tabular}{l}
|\ifchilddoc|\\
|\providecommand{\version}{draft}|\\
|\||else|\\
|\providecommand{\version}{final}|\\
|\||fi|
\end{tabular}
\end{center}
%
The definition by |\providecommand| makes sure
that previous definitions are not overwritten.
Further statements |\providecommand{\version}{...}|
can thus be added before the above code to override it.

For the main file, one might add a line
(between |\childdocmain| and the above block)
%
\begin{center}
|%\ifchilddoc\||else\providecommand{\version}{draft}\||fi|
\end{center}
%
which can be uncommented to produce a draft version.
Likewise one can add a line to the very top of a child file
(above the |\childdocof{|\textit{main}|}| directive)
%
\begin{center}
|%\providecommand{\version}{final}|
\end{center}
%
which can be uncommented to produce the final version of this child document.

%%%%%%%%%%%%%%%%%%%%%%%%%%%%%%%%%%%%%%%%%%%%%%%%%%%%%%%%%%%%%%%%%%%%%%%%%%%%%%%%
\subsection{Forwarding}
\label{sec:forward}

Different versions of the main or child documents
using compilation flags as described in \secref{sec:flags}
can be (permanently) stored in different files
for convenient compilation, viewing and distribution.
To this end, the package defines a command
to pass on compilation to a different file:

%%%%%%%%%%%%%%%%%%%%%%%%%%%%%%%%%%%%%%%%
\DescribeMacro{\childdocforward}
The command |\childdocforward| redirects processing to
another source file:
%
\begin{center}
\begin{tabular}{l}
|\input{childdoc.def}|\\
|\childdocforward[|\textit{main}|]{|\textit{dest}|}|\\
\end{tabular}
\end{center}
%
The argument \textit{dest} is the destination file
(without extension).
It should be the main file or one of the child files.
Note that further \textsf{childdoc} directives
such as |\childdocof| and |\childdocforward|
in the indicated file will be processed in this form.
The optional argument \textit{main}
passes on directly to the main file \textit{main}
while pretending to compile the child \textit{dest}.
This form behaves as if \textit{dest}
issues |\childdocof{|\textit{main}|}| right away,
and no further \textsf{childdoc} directives will be processed.

%%%%%%%%%%%%%%%%%%%%%%%%%%%%%%%%%%%%%%%%
\DescribeMacro{\...prefix}
In the alternative form |\childdocforwardprefix|,
%
\begin{center}
\begin{tabular}{l}
|\input{childdoc.def}|\\
|\childdocforwardprefix[|\textit{main}|]{|\textit{prefix}|}{|\textit{dest}|}|
\end{tabular}
\end{center}
%
the destination file is determined by a pattern
depending on the current file:
To make this work, the current file must be called
`{\textit{prefix}\hspace{0.2em}\textit{suffix}}'
with \textit{prefix} matching precisely the argument.
Processing is then passed on to the file
`{\textit{dest}\hspace{0.2em}\textit{suffix}}'.
Surely, the same effect is achieved by
directly specifying the
argument `{\textit{dest}\hspace{0.2em}\textit{suffix}}'
in the first form.
However, that requires to set up a different file
for each child. With the alternative form of the command
all these files can have exactly the same content
which simplifies setting them up and maintaining them.

For example, the following file |draft.tex|
with a compilation flag |\version| as described in \secref{sec:flags}
compiles the main document as a draft:
%
\begin{center}
\begin{tabular}{l}
|\def\version{draft}|\\
|\input{childdoc.def}|\\
|\childdocforward{|\textit{main}|}|
\end{tabular}
\end{center}
%
Likewise, the following files |final|\textit{nn}|.tex|
compile the final version of the child document
|child|\textit{nn}|.tex|:
%
\begin{center}
\begin{tabular}{l}
|\def\version{final}|\\
|\input{childdoc.def}|\\
|\childdocforwardprefix{final}{child}|
\end{tabular}
\end{center}
%

Note that when several versions of a main file and/or of each child file
are to be generated, it may be convenient to set up a |Makefile| or
shell script to automatise the process.

%%%%%%%%%%%%%%%%%%%%%%%%%%%%%%%%%%%%%%%%%%%%%%%%%%%%%%%%%%%%%%%%%%%%%%%%%%%%%%%%
\subsection{Command Line Processing}
\label{sec:commandline}

The effect of redirection files can also be achieved by invoking
the \LaTeX{} compiler with a more elaborate command line.
Most conveniently this should be done as part
of a shell script or a |Makefile|.

When using \textsf{childdoc} in the main file, the following
command lines effectively perform a redirection
(note that depending on the shell being used,
backslashes may have to be doubled: `|\|' $\to$ `|\\|'):
%
\begin{center}
|... -jobname "|\textit{target}|" |\\|"|[\textit{flags}]%
|\input{childdoc.def}\childdocforward[|\textit{main}|]{|\textit{dest}|}"|
\end{center}
%
Here \textit{target} is the name of the output file,
\textit{main} is the name of the main file
and \textit{dest} is the name of the main or child file to be processed
(all filenames without extensions).
The optional argument \textit{main} can be omitted
if \textit{main} matches \textit{dest}.
Optionally, compilation \textit{flags} can be defined via |\def| commands.
This command line makes the \TeX{} engine believe
it is compiling the file \textit{target}
whose content is specified as the latter parameter.
The provided code then forwards the processing to
\textit{main} or \textit{dest} as described in \secref{sec:forward}.

%%%%%%%%%%%%%%%%%%%%%%%%%%%%%%%%%%%%%%%%%%%%%%%%%%%%%%%%%%%%%%%%%%%%%%%%%%%%%%%%
\subsection{Include by Input}
\label{sec:input}

Including child documents by |\include| has some restrictions by design.
Most notably, the content of a child document always occupies
its own set of pages; pages cannot be shared between child documents.
Usually, this behaviour makes perfect sense
because each child document contain an essential part of the document.
However, in some situations it may be desirable to compose
a document from a collection of parts
without having mandatory page breaks between then.
For this case, the package
provides a mechanism to include parts
by |\input| which can also be processed individually.
However, by construction this mechanism
requires manual handling of the content to be output.

%%%%%%%%%%%%%%%%%%%%%%%%%%%%%%%%%%%%%%%%
\DescribeMacro{\ifchilddocmanual}
The main file should be prepared as usual, see \secref{sec:include}.
However, the document body must make a distinction
between processing of an individual part and of the main document, e.g.:
%
\begin{center}
\begin{tabular}{l}
|\ifchilddocmanual|\\
|\input{\childdocname}|\\
|\||else|\\
\textit{document body with }|\input{|\textit{part}|}|\\
|\||fi|
\end{tabular}
\end{center}
%
The conditional |\ifchilddocmanual| is true whenever
a part to be included by |\input| is being compiled,
and the name of the part is stored in |\childdocname|.

%%%%%%%%%%%%%%%%%%%%%%%%%%%%%%%%%%%%%%%%
\DescribeMacro{\childdocby}
Each part to be included by |\input| should start with:
%
\begin{center}
\begin{tabular}{l}
|\input{childdoc.def}|\\
|\childdocby{|\textit{main}|}|\\
\end{tabular}
\end{center}
%
The directive |\childdocby| is similar to |\childdocof|
described in \secref{sec:include},
but the subsequent selection of content must be done manually.
To that end, both |\ifchilddoc| and |\ifchilddocmanual|
will be true upon processing of a part,
and the name of the part is stored in |\childdocname|.
Note that |\jobname| will be set to the filename of the current part
so that each part receives an individual |.aux| file
that does not interfere with the |.aux| file(s) of the main document.
This behaviour can be altered by the alternative form
|\childdocby[*]{|\textit{main}|}| (with a non-empty optional argument)
which uses the |.aux| file of the main document
by setting |\jobname| to \textit{main}.

%%%%%%%%%%%%%%%%%%%%%%%%%%%%%%%%%%%%%%%%%%%%%%%%%%%%%%%%%%%%%%%%%%%%%%%%%%%%%%%%
\subsection{Driver Development}
\label{sec:driver}

The \textsf{childdoc} mechanism can also be use for the development
of definition files such as \LaTeX{} styles or classes.
This case differs from the above setup with multiple parts
included by |\include| in that no |\includeonly| should be invoked.
This can be achieved by starting the include file
(before |\ProvidesPackage|) with:
%
\begin{center}
\begin{tabular}{l}
|\input{childdoc.def}|\\
|\childdocforward{|\textit{main}|}|\\
\end{tabular}
\end{center}
%
or alternatively with:
%
\begin{center}
\begin{tabular}{l}
|\input{childdoc.def}|\\
|\childdocby{|\textit{main}|}|\\
\end{tabular}
\end{center}
%
Both forms have slightly different effects as described above.
The main file is prepared as usual, see \secref{sec:include}.

%%%%%%%%%%%%%%%%%%%%%%%%%%%%%%%%%%%%%%%%%%%%%%%%%%%%%%%%%%%%%%%%%%%%%%%%%%%%%%%%
\subsection{Legacy Detection}
\label{sec:detection}

The directive |\childdocmain| in the main file can detect
whether the complete document or merely a child is to be compiled
even without using the directive |\childdocof|.
This method is deprecated because it is less robust
and there is no compelling reason to use it;
it is merely provided for backward compatibility
and it may be removed in future versions.

If the detection mechanism is to be used,
it is mandatory to correctly specify
the filename of the main file as the argument of |\childdocmain|:
%
\begin{center}
\begin{tabular}{l}
|\input{childdoc.def}|\\
|\childdocmain{|\textit{main}|}|\\
\end{tabular}
\end{center}
%
If |\jobname| does not match the argument \textit{main} of |\childdocmain|,
it is assumed that |\jobname| points to the child file to be compiled.
When using |\childdocmain| with the main file specified as argument,
it suffices to start a child file
with just |\input{|\textit{main}|}|
without loading of the package and using |\childdocof|.
If instead all processing is done
with the appropriate \textsf{childdoc} directives,
the argument of \textit{main} of |\childdocmain| can be empty.

An alternative version of the command line processing described
in \secref{sec:commandline} using the detection mechanism reads:
%
\begin{center}
|... -jobname "|\textit{target}|" "|[\textit{flags}]%
[|\def\jobname{|\textit{dest}|}|]|\input{|\textit{main}|}"|
\end{center}

%%%%%%%%%%%%%%%%%%%%%%%%%%%%%%%%%%%%%%%%%%%%%%%%%%%%%%%%%%%%%%%%%%%%%%%%%%%%%%%%
\subsection{Manual Code}
\label{sec:manual}

In case one cannot be certain whether the definitions file |childdoc.def|
is installed on the target \TeX{} distribution
and one prefers not to ship it,
it is conceivable to paste a few relevant commands into the sources.

To that end, drop all statements |\input{childdoc.def}|
and perform the replacements as outlined below.
Instead of |\childdocmain{|\textit{main}|}| add the following code
to the top of the main file:
%
\begin{center}
\begin{tabular}{l}
|\||ifdefined\childdocname\endinput\||fi\newif\ifchilddoc|\\
|\edef\childdocname{\scantokens\expandafter{\jobname\noexpand}}|\\
|\def\childdocmain{|\textit{main}|}\||ifx\childdocmain\childdocname\||else|\\
|\childdoctrue\includeonly{\childdocname}\let\jobname\childdocmain\||fi|\\
\end{tabular}
\end{center}
%
Instead of |\childdocof{|\textit{main}|}| just include the main file
at the top of each child file:
%
\begin{center}
|\input{|\textit{main}|}|
\end{center}
%
A simple redirection |\childdocforward{|\textit{dest}|}| is achieved by:
%
\begin{center}
|\def\jobname{|\textit{dest}|}\input{\jobname}|
\end{center}
%
The redirection with prefix
|\childdocforwardprefix[|\textit{prefix}|]{|\textit{dest}|}|
is accomplished by:
%
\begin{center}
\begin{tabular}{l}
|{\edef\jobname{\scantokens\expandafter{\jobname\noexpand}}|\\
|\def\redirectjob |\textit{prefix}|#1~~~{\gdef\jobname{|\textit{dest}|#1}}|\\
|\expandafter\redirectjob\jobname~~~}\input{\jobname}|
\end{tabular}
\end{center}

In an alternative approach,
child documents can be compiled by a specific command line
without additional code or specific definitions:
%
\begin{center}
|... -jobname "|\textit{target}|" "|[\textit{flags}]%
|\includeonly{|\textit{dest}|}\input{|\textit{main}|}"|
\end{center}
%

%%%%%%%%%%%%%%%%%%%%%%%%%%%%%%%%%%%%%%%%%%%%%%%%%%%%%%%%%%%%%%%%%%%%%%%%%%%%%%%%
%%%%%%%%%%%%%%%%%%%%%%%%%%%%%%%%%%%%%%%%%%%%%%%%%%%%%%%%%%%%%%%%%%%%%%%%%%%%%%%%
\section{Information}

%%%%%%%%%%%%%%%%%%%%%%%%%%%%%%%%%%%%%%%%%%%%%%%%%%%%%%%%%%%%%%%%%%%%%%%%%%%%%%%%
\subsection{Copyright}

Copyright \copyright{} 2017--2018 Niklas Beisert

This work may be distributed and/or modified under the
conditions of the \LaTeX{} Project Public License, either version 1.3
of this license or (at your option) any later version.
The latest version of this license is in
  \url{http://www.latex-project.org/lppl.txt}
and version 1.3 or later is part of all distributions of \LaTeX{}
version 2005/12/01 or later.

This work has the LPPL maintenance status `maintained'.

The Current Maintainer of this work is Niklas Beisert.

This work consists of the files |README.txt|, |childdoc.ins| and |childdoc.dtx|
as well as the derived files |childdoc.def|, |cdocsamp.tex|
with |cdocsch1.tex|, |cdocsch2.tex|, |cdocspt3.tex|, |cdocspt4.tex|,
|cdocsdrf.tex|, |cdocsfn1.tex|, |cdocsfn2.tex|
as well as |childdoc.pdf|.

%%%%%%%%%%%%%%%%%%%%%%%%%%%%%%%%%%%%%%%%%%%%%%%%%%%%%%%%%%%%%%%%%%%%%%%%%%%%%%%%
\subsection{Files and Installation}

The package consists of the files:
%
\begin{center}
\begin{tabular}{ll}
    |README.txt|   & readme file \\
    |childdoc.ins| & installation file \\
    |childdoc.dtx| & source file \\
    |childdoc.def| & definition file \\
    |cdocsamp.tex| & sample main file \\
    |cdocsch1.tex| & sample include file \\
    |cdocsch2.tex| & sample include file \\
    |cdocspt3.tex| & sample part file \\
    |cdocspt4.tex| & sample part file \\
    |cdocsdrf.tex| & sample redirection file \\
    |cdocsfn1.tex| & sample redirection file \\
    |cdocsfn2.tex| & sample redirection file \\
    |childdoc.pdf| & manual
\end{tabular}
\end{center}
%
The distribution consists of the files
|README.txt|, |childdoc.ins| and |childdoc.dtx|.
%
\begin{itemize}
\item
Run (pdf)\LaTeX{} on |childdoc.dtx|
to compile the manual |childdoc.pdf| (this file).
\item
Run \LaTeX{} on |childdoc.ins| to create the definitions file |childdoc.def|
and the sample |cdocsamp.tex| with include files
|cdocsch1.tex|, |cdocsch2.tex|, |cdocspt3.tex|, |cdocspt4.tex|,
|cdocsdrf.tex|, |cdocsfn1.tex|, |cdocsfn2.tex|.
Then copy the file |childdoc.def| to an appropriate directory of your \LaTeX{}
distribution, e.g.\ \textit{texmf-root}|/tex/latex/childdoc|.
\end{itemize}

%%%%%%%%%%%%%%%%%%%%%%%%%%%%%%%%%%%%%%%%%%%%%%%%%%%%%%%%%%%%%%%%%%%%%%%%%%%%%%%%
\subsection{Related CTAN Packages}

There are several other packages which offer a similar functionality:
%
\begin{itemize}
\item
The packages
\href{http://ctan.org/pkg/docmute}{\textsf{docmute}},
\href{http://ctan.org/pkg/includex}{\textsf{includex}} and
\href{http://ctan.org/pkg/standalone}{\textsf{standalone}}
provide commands to include only the document body of
a child file thus allowing both files to be compiled individually.
\item
The packages \href{http://ctan.org/pkg/subdocs}{\textsf{subdocs}}
and \href{http://ctan.org/pkg/subfiles}{\textsf{subfiles}}
provide structures in which the main and child documents can be
encapsulated and allowing them to be compiled individually.
The inclusion mechanism is different from the conventional |\include|.
\item
The package \href{http://ctan.org/pkg/combine}{\textsf{combine}}
is an elaborate solution to combine several documents into one.
\end{itemize}
%
See also the CTAN topic \href{http://ctan.org/topic/subdocs}{\textsf{subdocs}}
for further related packages.
The present package differs from the above solutions in that
a document structure constructed with the conventional |\include| mechanism
just needs two extra commands at the top of every file
such that all constituent files can be compiled individually.

%%%%%%%%%%%%%%%%%%%%%%%%%%%%%%%%%%%%%%%%%%%%%%%%%%%%%%%%%%%%%%%%%%%%%%%%%%%%%%%%
%\subsection{Feature Suggestions}
%
%The following is a list of features which may be useful for future
%versions of this package:
%%
%\begin{itemize}
%\item
%\ldots
%\end{itemize}

%%%%%%%%%%%%%%%%%%%%%%%%%%%%%%%%%%%%%%%%%%%%%%%%%%%%%%%%%%%%%%%%%%%%%%%%%%%%%%%%
\subsection{Revision History}

%%%%%%%%%%%%%%%%%%%%%%%%%%%%%%%%%%%%%%%%
\paragraph{v2.0:} 2018/12/30

\begin{itemize}
\item
immediate forward processing
\item
added |\childdocby| mechanism
\item
manual restructured
\end{itemize}

%%%%%%%%%%%%%%%%%%%%%%%%%%%%%%%%%%%%%%%%
\paragraph{v1.6:} 2018/01/17

\begin{itemize}
\item
application for development of include files
\item
corrections to manual
\end{itemize}

%%%%%%%%%%%%%%%%%%%%%%%%%%%%%%%%%%%%%%%%
\paragraph{v1.5:} 2017/05/21

\begin{itemize}
\item
more complete structuring introduced
\item
|\childdocof| introduced
\item
|\childdoc| renamed to |\childdocmain|
\item
|\childredirect| renamed to |\childdocforward| and |\childdocforwardprefix|
and functionality expanded
\end{itemize}

%%%%%%%%%%%%%%%%%%%%%%%%%%%%%%%%%%%%%%%%
\paragraph{v1.0:} 2017/04/27

\begin{itemize}
\item
manual and install package
\item
first version published on CTAN
\end{itemize}

%%%%%%%%%%%%%%%%%%%%%%%%%%%%%%%%%%%%%%%%
\paragraph{v0.6:} 2017/04/26

\begin{itemize}
\item
redirection mechanism added
\end{itemize}

%%%%%%%%%%%%%%%%%%%%%%%%%%%%%%%%%%%%%%%%
\paragraph{v0.5:} 2017/04/26

\begin{itemize}
\item
functionality in definition file
\end{itemize}


%%%%%%%%%%%%%%%%%%%%%%%%%%%%%%%%%%%%%%%%%%%%%%%%%%%%%%%%%%%%%%%%%%%%%%%%%%%%%%%%
%%%%%%%%%%%%%%%%%%%%%%%%%%%%%%%%%%%%%%%%%%%%%%%%%%%%%%%%%%%%%%%%%%%%%%%%%%%%%%%%
%%%%%%%%%%%%%%%%%%%%%%%%%%%%%%%%%%%%%%%%%%%%%%%%%%%%%%%%%%%%%%%%%%%%%%%%%%%%%%%%
\appendix

\settowidth\MacroIndent{\rmfamily\scriptsize 000\ }

 \DocInput{childdoc.dtx}

\end{document}
%</driver>
% \fi
%
% %%%%%%%%%%%%%%%%%%%%%%%%%%%%%%%%%%%%%%%%%%%%%%%%%%%%%%%%%%%%%%%%%%%%%%%%%%%%%%
% %%%%%%%%%%%%%%%%%%%%%%%%%%%%%%%%%%%%%%%%%%%%%%%%%%%%%%%%%%%%%%%%%%%%%%%%%%%%%%
% \section{Sample}
%\iffalse
%<*samplemain>
%\fi
%
% The following presents a sample document
% with two chapters, two parts, a title page,
% a compile flag as well as three forwarding files to set the flag.
% It consists of eight |.tex| files:
% \begin{center}
% \begin{tabular}{ll}
% |cdocsamp.tex|&main file\\
% |cdocsch1.tex|&include file for chapter 1\\
% |cdocsch2.tex|&include file for chapter 2\\
% |cdocspt3.tex|&include file for part 3\\
% |cdocspt4.tex|&include file for part 4\\
% |cdocsdrf.tex|&forwarding file for main file in draft mode\\
% |cdocsfi1.tex|&forwarding file for final version of chapter 1\\
% |cdocsfi2.tex|&forwarding file for final version of chapter 2\\
% \end{tabular}
% \end{center}
% Each of the eight files can be compiled directly by the \LaTeX{} compiler.
%
% %%%%%%%%%%%%%%%%%%%%%%%%%%%%%%%%%%%%%%
% \paragraph{Main File.}
%
% The main file is called |cdocsamp.tex|.
%
% Load the \textsf{childdoc} definitions and
% declare the filename for the main document:
%    \begin{macrocode}
\input{childdoc.def}
\childdocmain{}
%    \end{macrocode}

% Optional override for |\version| flag:
%    \begin{macrocode}
%%\ifchilddoc\else\providecommand{\version}{draft}\fi
%    \end{macrocode}

% Define the default values for the |\version| flag
% (|final| for the main file and |draft| for childs):
%    \begin{macrocode}
\ifchilddoc
\providecommand{\version}{draft}
\else
\providecommand{\version}{final}
\fi
%    \end{macrocode}

% Load the standard document class:
%    \begin{macrocode}
\documentclass[12pt]{article}
%    \end{macrocode}

% Start the document body:
%    \begin{macrocode}
\begin{document}
%    \end{macrocode}

% Declare a title page.
% Print title, part of document being processed and version flag:
%    \begin{macrocode}
\addtocounter{page}{-1}
\begin{center}
{\LARGE\bfseries{}childdoc example\par}
\vspace{1cm}
\ifchilddoc
\ifchilddocmanual part\else chapter\fi:
`\childdocname' of `\childdocjob'\par
\else
main document: `\childdocjob'\par
\fi
version: \version\par
\end{center}
\newpage
%    \end{macrocode}

% Manually include selected file,
% otherwise process as usual:
%    \begin{macrocode}
\ifchilddocmanual
\section*{part `\childdocname'}
\input{\childdocname}
\else
%    \end{macrocode}

% Include the two chapters:
%    \begin{macrocode}
\include{cdocsch1}
\include{cdocsch2}
%    \end{macrocode}

% Include the two parts unless only chapters should be displayed:
%    \begin{macrocode}
\ifchilddoc\else
\section{part three}
\input{cdocspt3}
\section{part four}
\input{cdocspt4}
\fi
%    \end{macrocode}

% Process as usual until here:
%    \begin{macrocode}
\fi
%    \end{macrocode}

% End of document body:
%    \begin{macrocode}
\end{document}
%    \end{macrocode}
%\iffalse
%</samplemain>
%\fi
%
% %%%%%%%%%%%%%%%%%%%%%%%%%%%%%%%%%%%%%%
% \paragraph{Chapter Include Files.}
%
% The include files are called |cdocsch1.tex| and |cdocsch2.tex|.
%
%\iffalse
%<*samplechap1|samplechap2>
%\fi

% Optional override for |\version| flag:
%    \begin{macrocode}
%%\providecommand{\version}{final}
%    \end{macrocode}

% Include the main document:
%    \begin{macrocode}
\input{childdoc.def}
\childdocof{cdocsamp}
%    \end{macrocode}

%\iffalse
%</samplechap1|samplechap2>
%\fi
%
%\iffalse
%<*samplechap1>
%\fi
% Some text for chapter 1:
%    \begin{macrocode}
\section{one}
some text in chapter one
%    \end{macrocode}

%\iffalse
%</samplechap1>
%\fi
% Some text for chapter 2:
%\iffalse
%<*samplechap2>
%\fi
%    \begin{macrocode}
\section{two}
more text in chapter two
%    \end{macrocode}

%\iffalse
%</samplechap2>
%\fi
%
% %%%%%%%%%%%%%%%%%%%%%%%%%%%%%%%%%%%%%%
% \paragraph{Part Include Files.}
%
% The include files are called |cdocspt3.tex| and |cdocspt4.tex|.
%
%\iffalse
%<*samplepart3|samplepart4>
%\fi

% Optional override for |\version| flag:
%    \begin{macrocode}
%%\providecommand{\version}{final}
%    \end{macrocode}

% Include the main document:
%    \begin{macrocode}
\input{childdoc.def}
\childdocby{cdocsamp}
%    \end{macrocode}

%\iffalse
%</samplepart3|samplepart4>
%\fi
%
%\iffalse
%<*samplepart3>
%\fi
% Some text for part 3:
%    \begin{macrocode}
some text in part three
%    \end{macrocode}

%\iffalse
%</samplepart3>
%\fi
% Some text for part 4:
%\iffalse
%<*samplepart4>
%\fi
%    \begin{macrocode}
more text in part four
%    \end{macrocode}

%\iffalse
%</samplepart4>
%\fi
%
% %%%%%%%%%%%%%%%%%%%%%%%%%%%%%%%%%%%%%%
% \paragraph{Forwarding for a Complete Draft.}
%
% The following forwarding file |cdocsdrf.tex|
% compiles the main document in draft mode:
%\iffalse
%<*sampledraft>
%\fi
%    \begin{macrocode}
\def\version{draft}
\input{childdoc.def}
\childdocforward{cdocsamp}
%    \end{macrocode}

%\iffalse
%</sampledraft>
%\fi
%
% %%%%%%%%%%%%%%%%%%%%%%%%%%%%%%%%%%%%%%
% \paragraph{Forwarding for Final Version of the Chapters.}
%
% The following forwarding files |cdocsfn1.tex| and |cdocsfn2.tex|
% (with identical content)
% compile the final versions of the child documents
% |cdocsch1.tex| and |cdocsch2.tex|, respectively:
%\iffalse
%<*samplefinal>
%\fi
%    \begin{macrocode}
\def\version{final}
\input{childdoc.def}
\childdocforwardprefix[cdocsamp]{cdocsfn}{cdocsch}
%    \end{macrocode}

%\iffalse
%</samplefinal>
%\fi
%
% %%%%%%%%%%%%%%%%%%%%%%%%%%%%%%%%%%%%%%
% \paragraph{Command Line Processing.}
%
% The following three command lines generate the output files
% |cdocscld|, |cdocscl1| and |cdocscl2|
% which should be identical to
% |cdocsdrf|, |cdocsch1| and |cdocsfn2|, respectively:
% \begin{center}
% \begin{tabular}{l}
% |latex -jobname cdocscld \|\\
% |  "\def\version{draft}\input{childdoc.def}\childdocforward{cdocsamp}"|\\
% |latex -jobname cdocscl1 \|\\
% |  "\input{childdoc.def}\childdocforward[cdocsamp]{cdocsch1}"|\\
% |latex -jobname cdocscl2 \|\\
% |  "\def\version{final}\input{childdoc.def}\childdocforward{cdocsch2}"|
% \end{tabular}
% \end{center}
% Note that the trailing backslash on each first line
% merely continues the input to the second line
% (for convenient cut ant paste).
% Furthermore, the command |latex| can be replaced by any
% of its alternative versions such as |pdflatex|.
%
% %%%%%%%%%%%%%%%%%%%%%%%%%%%%%%%%%%%%%%%%%%%%%%%%%%%%%%%%%%%%%%%%%%%%%%%%%%%%%%
% %%%%%%%%%%%%%%%%%%%%%%%%%%%%%%%%%%%%%%%%%%%%%%%%%%%%%%%%%%%%%%%%%%%%%%%%%%%%%%
% \section{Implementation}
%\iffalse
%<*package>
%\fi
%
% This section describes the definitions file |childdoc.def|.

% The definitions cannot be loaded using |\usepackage| or |\RequirePackage|
% which has a mechanism to prevent loading a style file more than once.
% When loading the definitions by means of |\input|
% multiple instances have to be prevented manually:
%\iffalse
%This code needs to be before the `\ProvidesFile' directive
%which is defined at the beginning of this file.
%Therefore it is also placed there and commented out here.
%</package>
%<*discard>
%\fi
%    \begin{macrocode}
\ifdefined\childdocmain\endinput\fi
%    \end{macrocode}
%\iffalse
%</discard>
%<*package>
%\fi
%
% \macro{\ifchilddoc}
% \macro{\ifchilddocmanual}
% The conditional |\ifchilddoc| tells whether a
% child (true) or main (false) document is being compiled.
% The conditional |\ifchilddocmanual| tells whether
% the |\includeonly| mechanism is used (false) or
% the selection of child files must be performed manually (true).
% The definitions initialise to false:
%    \begin{macrocode}
\newif\ifchilddoc
\newif\ifchilddocmanual
%    \end{macrocode}

% \macro{\childdocname}
% \macro{\childdocjob}
% The macro |\childdocname| stores the name of the main document
% to be compiled. The macro |\childdocjob| stores the name of
% the document on which the \LaTeX{} compiler was originally invoked.
% The content of |\jobname| cannot be compared
% to filenames specified in the source due to different catcodes.
% The following code rescans |\jobname|, stores the result
% in |\childdocname| and saves a copy in |\childdocjob|:
%    \begin{macrocode}
\edef\childdocname{\scantokens\expandafter{\jobname\noexpand}}
\let\childdocjob\childdocname
%    \end{macrocode}

% \macro{\childdocdisable}
% The macro |\childdocdisable| prevents the main file
% from being processed more than once.
% At this stage, the main document command |\childdocmain|
% is assumed to be called once again where it should do nothing.
% Any subsequent call to it should prevent
% a secondary processing of the main document
% It overwrites the forwarding commands
% |\childdocof| and |\childdocforward|
% with empty macros to prevent further inclusions of the main document:
%    \begin{macrocode}
\newcommand{\childdocdisable}
{
  \renewcommand{\childdocmain}[1]{\renewcommand{\childdocmain}[1]{\endinput}}
  \renewcommand{\childdocof}[1]{}
  \renewcommand{\childdocby}[2][]{}
  \renewcommand{\childdocforward}[2][]{}
  \renewcommand{\childdocdisable}{}
}
%    \end{macrocode}

% \macro{\childdocmain}
% The macro |\childdocmain| is to be called at the top of the main file
% with nothing or the main filename (without extension) as argument.
% First, it breaks loops.
% If the argument is not empty and does not match |\childdocname|
% (which is set by the first inclusion of |childdoc.def|),
% |\ifchilddoc| is set to true, |\includeonly| is applied to the child file
% and |\jobname| is set to the main file
% (for proper handling of |.aux| files):
%    \begin{macrocode}
\newcommand{\childdocmain}[1]
{
  \childdocdisable\childdocmain{}
  \if?#1?\else
    \begingroup
      \def\childdoctmp{#1}
      \ifx\childdoctmp\childdocname
        \def\childdoctmp{}
      \else
        \def\childdoctmp
        {
          \childdoctrue
          \includeonly{\childdocname}
          \def\childdocjob{#1}
          \def\jobname{#1}
        }
      \fi
      \expandafter
    \endgroup
    \childdoctmp
  \fi
}
%    \end{macrocode}

% \macro{\childdocof}
% The command |\childdocof| redirects
% compilation to the main file |#1|.
%    \begin{macrocode}
\newcommand{\childdocof}[1]
{
  \childdocdisable
  \childdoctrue
  \includeonly{\childdocname}
  \def\jobname{#1}
  \def\childdocjob{#1}
  \input{#1}
}
%    \end{macrocode}

% \macro{\childdocby}
% The command |\childdocby| ....
%    \begin{macrocode}
\newcommand{\childdocby}[2][]
{
  \childdocdisable
  \childdoctrue
  \childdocmanualtrue
  \if?#1?\else
    \def\jobname{#2}
  \fi
  \def\childdocjob{#2}
  \input{#2}
  \endinput
}
%    \end{macrocode}

% \macro{\childdocforward}
% The command |\childdocforward| redirects
% compilation to the main file or
% (if the optional argument is given) a child file.
% Parameters are set as if the main file
% or a child file starting with |\childdocof| was compiled.
% Then compilation is handed over to the main file:
%    \begin{macrocode}
\newcommand{\childdocforward}[2][]
{
  \begingroup
    \if?#1?
      \def\childdoctmp
      {
        \def\childdocname{#2}
        \def\childdocjob{#2}
        \def\jobname{#2}
        \input{#2}
        \endinput
      }
    \else
      \def\childdoctmp
      {
        \childdocdisable
        \def\childdocname{#2}
        \childdoctrue
        \includeonly{#2}
        \def\childdocjob{#1}
        \def\jobname{#1}
        \input{#1}
        \endinput
      }
    \fi
    \expandafter
  \endgroup
  \childdoctmp
}
%    \end{macrocode}

% \macro{\childdocforwardprefix}
% The command |\childdocforwardprefix| redirects
% compilation to the main or a child file by means of a pattern.
% The prefix |#1| in the current filename is replaced by |#2|
% and the suffix of the current filename is kept
% (it is assumed that the filename does not contain the substring `|~~~|'
% which is used as a delimiter).
% Compilation is handed over to the new file by |\childdocforward|:
%    \begin{macrocode}
\newcommand{\childdocforwardprefix}[3][]
{
  \begingroup
    \def\childdocextract #2##1~~~{\def\childdoctmp{\childdocforward[#1]{#3##1}}}
    \expandafter\childdocextract\childdocname~~~
    \expandafter
  \endgroup
  \childdoctmp
}
%    \end{macrocode}

% \macro{\childdoc}
% The deprecated macro |\childdoc| is a legacy version of |\childdocmain|:
%    \begin{macrocode}
\newcommand{\childdoc}{\childdocmain}
%    \end{macrocode}

% \macro{\childdocredirect}
% The deprecated macro |\childdocredirect| is a legacy version
% of |\childdocforward| and |\childdocforwardprefix|:
%    \begin{macrocode}
\newcommand{\childdocredirect}[2][]
{
  \begingroup
    \if?#1?
      \def\childdoctmp{\childdocforward{#2}}
    \else
      \def\childdoctmp{\childdocforwardprefix{#1}{#2}}
    \fi
    \expandafter
  \endgroup
  \childdoctmp
}
%    \end{macrocode}

%\iffalse
%</package>
%\fi
%
\endinput
\childdocforward{cdocsamp}"|\\
% |latex -jobname cdocscl1 \|\\
% |  "% \iffalse
%
% childdoc.dtx Copyright (C) 2017-2018 Niklas Beisert
%
% This work may be distributed and/or modified under the
% conditions of the LaTeX Project Public License, either version 1.3
% of this license or (at your option) any later version.
% The latest version of this license is in
%   http://www.latex-project.org/lppl.txt
% and version 1.3 or later is part of all distributions of LaTeX
% version 2005/12/01 or later.
%
% This work has the LPPL maintenance status `maintained'.
%
% The Current Maintainer of this work is Niklas Beisert.
%
% This work consists of the files childdoc.dtx and childdoc.ins
% and the derived files childdoc.def and cdocsamp.tex with
% cdocsch1.tex, cdocsch2.tex, cdocsdrf.tex, cdocsfn1.tex, cdocsfn2.tex.
%
%<package>\ifdefined\childdocmain\endinput\fi
%<package>\ProvidesFile{childdoc.def}[2018/12/30 v2.0 child document driver]
%<samplemain>\ProvidesFile{cdocsamp.tex}[2018/12/30 v2.0 sample for childdoc]
%<*driver>
%\ProvidesFile{childdoc.drv}[2018/12/30 v2.0 childdoc reference manual file]
\PassOptionsToClass{10pt,a4paper}{article}
\documentclass{ltxdoc}

\usepackage[margin=35mm]{geometry}
\usepackage{hyperref}
\usepackage{hyperxmp}
\usepackage[usenames]{color}

\hypersetup{colorlinks=true}
\hypersetup{pdfstartview=FitH}
\hypersetup{pdfpagemode=UseNone}
\hypersetup{pdfsource={}}
\hypersetup{pdflang={en-UK}}
\hypersetup{pdfcopyright={Copyright 2017-2018 Niklas Beisert.
  This work may be distributed and/or modified under the
  conditions of the LaTeX Project Public License, either version 1.3
  of this license or (at your option) any later version.}}
\hypersetup{pdflicenseurl={http://www.latex-project.org/lppl.txt}}
\hypersetup{pdfcontactaddress={ETH Zurich, ITP, HIT K,
  Wolfgang-Pauli-Strasse 27}}
\hypersetup{pdfcontactpostcode={8093}}
\hypersetup{pdfcontactcity={Zurich}}
\hypersetup{pdfcontactcountry={Switzerland}}
\hypersetup{pdfcontactemail={nbeisert@itp.phys.ethz.ch}}
\hypersetup{pdfcontacturl={http://people.phys.ethz.ch/\xmptilde nbeisert/}}

\newcommand{\secref}[1]{\hyperref[#1]{section \ref*{#1}}}

\parskip1ex
\parindent0pt
\let\olditemize\itemize
\def\itemize{\olditemize\parskip0pt}

\begin{document}

\title{The \textsf{childdoc} Package}
\hypersetup{pdftitle={The childdoc Package}}
\author{Niklas Beisert\\[2ex]
  Institut f\"ur Theoretische Physik\\
  Eidgen\"ossische Technische Hochschule Z\"urich\\
  Wolfgang-Pauli-Strasse 27, 8093 Z\"urich, Switzerland\\[1ex]
  \href{mailto:nbeisert@itp.phys.ethz.ch}
  {\texttt{nbeisert@itp.phys.ethz.ch}}}
\hypersetup{pdfauthor={Niklas Beisert}}
\hypersetup{pdfsubject={Manual for the LaTeX2e Package childdoc}}
\date{30 December 2018, \textsf{v2.0}}
\maketitle

\begin{abstract}\noindent
\textsf{childdoc} is a \LaTeXe{} package
that enables the direct compilation
of document sections included by |\include|
to individual files.
\end{abstract}

\begingroup
\parskip0ex
\tableofcontents
\endgroup

%%%%%%%%%%%%%%%%%%%%%%%%%%%%%%%%%%%%%%%%%%%%%%%%%%%%%%%%%%%%%%%%%%%%%%%%%%%%%%%%
%%%%%%%%%%%%%%%%%%%%%%%%%%%%%%%%%%%%%%%%%%%%%%%%%%%%%%%%%%%%%%%%%%%%%%%%%%%%%%%%
\section{Introduction}

\LaTeX{} provides a mechanism to structure a large document (such as a book)
into a main file and several child files (containing the chapters)
using the |\include| command.
This mechanism is beneficial for documents
which span hundreds of pages in order to
make the source file(s) more manageable.
Moreover, compilation can be restricted to
selected child files by means of the |\includeonly| command.
The latter feature can be used to reduce the compilation time while editing
(this was significantly more useful in the earlier days of \LaTeX{})
or to generate a smaller document which is easier to navigate.
Another application of |\includeonly| is to generate
documents consisting of selected parts of the complete document.

However, there are a few drawbacks of the plain |\include| mechanism:
\begin{itemize}
\item
The child files cannot be compiled on their own,
they can only be compiled via the main file.
A naive editing environment
(such as a text editor with an option
to have the current file processed by \LaTeX)
may require one to switch to the main file before compiling;
attempting to compile the child file produces errors.
\item
The main file must be modified (each time)
to adjust the |\includeonly| command
to the present needs. This easily leaves the main file in a messy state.
\item
The generated document will always carry the filename
of the main document. This is inconvenient if
several child files are to be compiled and
to be kept for distribution.
\end{itemize}

The present package provides a simple interface
to make child files individually compilable by \LaTeX{}.
Compiling a child file then has the same effect as compiling
the main file with an |\includeonly| command
to select the appropriate child.
Moreover the generated document will carry the name of the child
rather than the main file.
This resolves all three above issues.

This feature is meant to make the editing of books,
thesis documents and lecture notes somewhat more convenient.
However, the package can also be used efficiently for
composing a series of documents (such as exercise sheets)
which are typically distributed individually.
It then assists the author in generating the individual documents
(potentially in different versions)
as well as a document containing the collected series.
Another application is in developing style files
or other kinds of included material
where compilation of the style file could redirect
to a sample or test file.

%%%%%%%%%%%%%%%%%%%%%%%%%%%%%%%%%%%%%%%%%%%%%%%%%%%%%%%%%%%%%%%%%%%%%%%%%%%%%%%%
%%%%%%%%%%%%%%%%%%%%%%%%%%%%%%%%%%%%%%%%%%%%%%%%%%%%%%%%%%%%%%%%%%%%%%%%%%%%%%%%
\section{Usage}

First of all, the package \textsf{childdoc} is \emph{not} a standard
\LaTeXe{} |.sty| style file! Therefore it needs to be invoked in
a non-standard way.

%%%%%%%%%%%%%%%%%%%%%%%%%%%%%%%%%%%%%%%%%%%%%%%%%%%%%%%%%%%%%%%%%%%%%%%%%%%%%%%%
\subsection{Included Files}
\label{sec:include}

%%%%%%%%%%%%%%%%%%%%%%%%%%%%%%%%%%%%%%%%
\DescribeMacro{\childdocmain}
To use the package, add the commands
\begin{center}
\begin{tabular}{l}
|\input{childdoc.def}|\\
|\childdocmain{}|\\
\end{tabular}
\end{center}
at the very top of the main \LaTeX{} file,
in particular \emph{before} the |\documentclass| statement!
The argument of |\childdocmain| should be left empty
(but it must be present).

%%%%%%%%%%%%%%%%%%%%%%%%%%%%%%%%%%%%%%%%
\DescribeMacro{\childdocof}
Furthermore, add the commands
\begin{center}
\begin{tabular}{l}
|\input{childdoc.def}|\\
|\childdocof{|\textit{main}|}|\\
\end{tabular}
\end{center}
at the top of every child file \textit{child}
which is included by |\include{|\textit{child}|}|
from within the main file
(or at least for those files to be compiled individually).
The argument \textit{main} must be the filename of the main file.

There are a couple of
considerations in setting up the main and child documents:

%%%%%%%%%%%%%%%%%%%%%%%%%%%%%%%%%%%%%%%%
\paragraph{Restrictions.}

Please note the following restrictions:
\begin{itemize}
\item
|\childdocmain| must be called with one argument \textit{main}
to ensure compatibility with earlier version of the package.
It must either be empty (|\childdocmain{}|)
or precisely match the filename of the main file in which it is specified.
See \secref{sec:detection} for further information.
\item
The filename \textit{main} must be specified without the |.tex| extension.
\item
The filename \textit{main} is case sensitive
(even in case-insensitive file systems)
due to internal string comparison.
\item
The argument \textit{main} should be fully expanded, it cannot be a macro.
\item
Subdirectories and special characters should be avoided in filenames.
\item
The command |\childdocmain{|\textit{main}|}| must be followed by a whitespace.
It should not be followed immediately by another command
or by a comment mark `|%|'.
This is because the \TeX{} parser reads the token immediately following
the argument of |\childdocmain| and puts it
at the beginning of every child section;
however, a white\-space is ignored.
\end{itemize}

%%%%%%%%%%%%%%%%%%%%%%%%%%%%%%%%%%%%%%%%
\paragraph{Content of Main File.}

It is advisable to place all content in the child files included by |\include|.
Any output contained in the main file will appear in all child documents
unless suppressed manually;
it cannot be suppressed automatically by the |\includeonly| directive
and thus should normally be avoided.
A method to include some content in the main file
by means of conditional processing is described in \secref{sec:conditional}.

%%%%%%%%%%%%%%%%%%%%%%%%%%%%%%%%%%%%%%%%
\paragraph{Page Numbering.}

When only a part of the document is compiled,
the appropriate numbering of pages
(as well as other status parameters)
is determined from the |.aux| files.
The latter contain information from previous passes.
However this information needs to propagate through
all intermediate child documents.
Therefore the page numbering in child documents may well
be inconsistent until the complete document is compiled at least once.

A useful (if unconventional) way to always ensure a consistent
page numbering is to restart the numbering in each child document
and denote the pages by `\textit{child}|.|\textit{page}'
where \textit{child} represents the chapter/section number of the child file.
This can be achieved by the command
|\numberwithin{page}{|\textit{child}|}|
of the \textsf{amsmath} package
where \textit{child} can be |chapter| or |section|
depending on the chosen structuring.
Alternatively, one can modify the macro |\thepage| appropriately
and reset the counter |page| at the start of each child file.

%%%%%%%%%%%%%%%%%%%%%%%%%%%%%%%%%%%%%%%%%%%%%%%%%%%%%%%%%%%%%%%%%%%%%%%%%%%%%%%%
\subsection{Conditional Processing}
\label{sec:conditional}

The package provides a mechanism to compile different versions
of a document. To customise the versions further some conditional processing
can come in handy to distinguish which version is being compiled.
The package provides two macros to describe the compilation context:

%%%%%%%%%%%%%%%%%%%%%%%%%%%%%%%%%%%%%%%%
\DescribeMacro{\ifchilddoc}
The conditional |\ifchilddoc| distinguishes between the compilation of
child documents and the main document:
%
\begin{center}
|\ifchilddoc |\textit{child-code}| |[|\||else |\textit{main-code}]| \||fi|
\end{center}

%%%%%%%%%%%%%%%%%%%%%%%%%%%%%%%%%%%%%%%%
\DescribeMacro{\childdocname}
\DescribeMacro{\childdocjob}
The macro |\childdocname| contains the filename (without extension)
of the main or child file being processed.
Note that |\childdocjob| will always contain the name of the main file.

%%%%%%%%%%%%%%%%%%%%%%%%%%%%%%%%%%%%%%%%
\paragraph{Title Page.}

Conditional processing can be used to include a title or banner page
in the main document when proper precautions are taken.
Importantly, the code in the main file should ensure that the page counter
(as well as other status parameters which are stored in the |.aux| files)
takes the same value after the conditional processing.
Otherwise the page numbers may take divergent values
depending on which part is compiled.

For example, a title page could be declared by:
%
\begin{center}
\begin{tabular}{l}
|\ifchilddoc\||else|\\
|\addtocounter{page}{-1}|\\
\textit{code for title page}\\
|\newpage|\\
|\||fi|
\end{tabular}
\end{center}
%
A banner page for the child documents can be generated by:
%
\begin{center}
\begin{tabular}{l}
|\ifchilddoc|\\
|\addtocounter{page}{-1}|\\
\textit{code for banner page}\\
|\newpage|\\
|\||fi|
\end{tabular}
\end{center}
%
Here one could write a message such as:
\begin{center}
|This is the part \childdocname{} of \childdocjob{}.|
\end{center}

%%%%%%%%%%%%%%%%%%%%%%%%%%%%%%%%%%%%%%%%%%%%%%%%%%%%%%%%%%%%%%%%%%%%%%%%%%%%%%%%
\subsection{Flags}
\label{sec:flags}

The package makes it easy to generate different versions
of the main or child documents.
To this end compilation flags can be defined
and assigned different default values.
They will be particularly useful in conjunction
with the forwarding mechanism described in \secref{sec:forward}.

For example, it may be useful to have a flag |\version|
which can be set to |draft| or |final|.
The document source will contain some conditional code
depending on the value of |\version|.
Suppose further, the flag should default to |final| for the main file
and to |draft| for child files
which is a natural assignment for editing the document.
This is achieved by placing the following code
in the preamble of the main document
(below the |\childdocmain| directive):
%
\begin{center}
\begin{tabular}{l}
|\ifchilddoc|\\
|\providecommand{\version}{draft}|\\
|\||else|\\
|\providecommand{\version}{final}|\\
|\||fi|
\end{tabular}
\end{center}
%
The definition by |\providecommand| makes sure
that previous definitions are not overwritten.
Further statements |\providecommand{\version}{...}|
can thus be added before the above code to override it.

For the main file, one might add a line
(between |\childdocmain| and the above block)
%
\begin{center}
|%\ifchilddoc\||else\providecommand{\version}{draft}\||fi|
\end{center}
%
which can be uncommented to produce a draft version.
Likewise one can add a line to the very top of a child file
(above the |\childdocof{|\textit{main}|}| directive)
%
\begin{center}
|%\providecommand{\version}{final}|
\end{center}
%
which can be uncommented to produce the final version of this child document.

%%%%%%%%%%%%%%%%%%%%%%%%%%%%%%%%%%%%%%%%%%%%%%%%%%%%%%%%%%%%%%%%%%%%%%%%%%%%%%%%
\subsection{Forwarding}
\label{sec:forward}

Different versions of the main or child documents
using compilation flags as described in \secref{sec:flags}
can be (permanently) stored in different files
for convenient compilation, viewing and distribution.
To this end, the package defines a command
to pass on compilation to a different file:

%%%%%%%%%%%%%%%%%%%%%%%%%%%%%%%%%%%%%%%%
\DescribeMacro{\childdocforward}
The command |\childdocforward| redirects processing to
another source file:
%
\begin{center}
\begin{tabular}{l}
|\input{childdoc.def}|\\
|\childdocforward[|\textit{main}|]{|\textit{dest}|}|\\
\end{tabular}
\end{center}
%
The argument \textit{dest} is the destination file
(without extension).
It should be the main file or one of the child files.
Note that further \textsf{childdoc} directives
such as |\childdocof| and |\childdocforward|
in the indicated file will be processed in this form.
The optional argument \textit{main}
passes on directly to the main file \textit{main}
while pretending to compile the child \textit{dest}.
This form behaves as if \textit{dest}
issues |\childdocof{|\textit{main}|}| right away,
and no further \textsf{childdoc} directives will be processed.

%%%%%%%%%%%%%%%%%%%%%%%%%%%%%%%%%%%%%%%%
\DescribeMacro{\...prefix}
In the alternative form |\childdocforwardprefix|,
%
\begin{center}
\begin{tabular}{l}
|\input{childdoc.def}|\\
|\childdocforwardprefix[|\textit{main}|]{|\textit{prefix}|}{|\textit{dest}|}|
\end{tabular}
\end{center}
%
the destination file is determined by a pattern
depending on the current file:
To make this work, the current file must be called
`{\textit{prefix}\hspace{0.2em}\textit{suffix}}'
with \textit{prefix} matching precisely the argument.
Processing is then passed on to the file
`{\textit{dest}\hspace{0.2em}\textit{suffix}}'.
Surely, the same effect is achieved by
directly specifying the
argument `{\textit{dest}\hspace{0.2em}\textit{suffix}}'
in the first form.
However, that requires to set up a different file
for each child. With the alternative form of the command
all these files can have exactly the same content
which simplifies setting them up and maintaining them.

For example, the following file |draft.tex|
with a compilation flag |\version| as described in \secref{sec:flags}
compiles the main document as a draft:
%
\begin{center}
\begin{tabular}{l}
|\def\version{draft}|\\
|\input{childdoc.def}|\\
|\childdocforward{|\textit{main}|}|
\end{tabular}
\end{center}
%
Likewise, the following files |final|\textit{nn}|.tex|
compile the final version of the child document
|child|\textit{nn}|.tex|:
%
\begin{center}
\begin{tabular}{l}
|\def\version{final}|\\
|\input{childdoc.def}|\\
|\childdocforwardprefix{final}{child}|
\end{tabular}
\end{center}
%

Note that when several versions of a main file and/or of each child file
are to be generated, it may be convenient to set up a |Makefile| or
shell script to automatise the process.

%%%%%%%%%%%%%%%%%%%%%%%%%%%%%%%%%%%%%%%%%%%%%%%%%%%%%%%%%%%%%%%%%%%%%%%%%%%%%%%%
\subsection{Command Line Processing}
\label{sec:commandline}

The effect of redirection files can also be achieved by invoking
the \LaTeX{} compiler with a more elaborate command line.
Most conveniently this should be done as part
of a shell script or a |Makefile|.

When using \textsf{childdoc} in the main file, the following
command lines effectively perform a redirection
(note that depending on the shell being used,
backslashes may have to be doubled: `|\|' $\to$ `|\\|'):
%
\begin{center}
|... -jobname "|\textit{target}|" |\\|"|[\textit{flags}]%
|\input{childdoc.def}\childdocforward[|\textit{main}|]{|\textit{dest}|}"|
\end{center}
%
Here \textit{target} is the name of the output file,
\textit{main} is the name of the main file
and \textit{dest} is the name of the main or child file to be processed
(all filenames without extensions).
The optional argument \textit{main} can be omitted
if \textit{main} matches \textit{dest}.
Optionally, compilation \textit{flags} can be defined via |\def| commands.
This command line makes the \TeX{} engine believe
it is compiling the file \textit{target}
whose content is specified as the latter parameter.
The provided code then forwards the processing to
\textit{main} or \textit{dest} as described in \secref{sec:forward}.

%%%%%%%%%%%%%%%%%%%%%%%%%%%%%%%%%%%%%%%%%%%%%%%%%%%%%%%%%%%%%%%%%%%%%%%%%%%%%%%%
\subsection{Include by Input}
\label{sec:input}

Including child documents by |\include| has some restrictions by design.
Most notably, the content of a child document always occupies
its own set of pages; pages cannot be shared between child documents.
Usually, this behaviour makes perfect sense
because each child document contain an essential part of the document.
However, in some situations it may be desirable to compose
a document from a collection of parts
without having mandatory page breaks between then.
For this case, the package
provides a mechanism to include parts
by |\input| which can also be processed individually.
However, by construction this mechanism
requires manual handling of the content to be output.

%%%%%%%%%%%%%%%%%%%%%%%%%%%%%%%%%%%%%%%%
\DescribeMacro{\ifchilddocmanual}
The main file should be prepared as usual, see \secref{sec:include}.
However, the document body must make a distinction
between processing of an individual part and of the main document, e.g.:
%
\begin{center}
\begin{tabular}{l}
|\ifchilddocmanual|\\
|\input{\childdocname}|\\
|\||else|\\
\textit{document body with }|\input{|\textit{part}|}|\\
|\||fi|
\end{tabular}
\end{center}
%
The conditional |\ifchilddocmanual| is true whenever
a part to be included by |\input| is being compiled,
and the name of the part is stored in |\childdocname|.

%%%%%%%%%%%%%%%%%%%%%%%%%%%%%%%%%%%%%%%%
\DescribeMacro{\childdocby}
Each part to be included by |\input| should start with:
%
\begin{center}
\begin{tabular}{l}
|\input{childdoc.def}|\\
|\childdocby{|\textit{main}|}|\\
\end{tabular}
\end{center}
%
The directive |\childdocby| is similar to |\childdocof|
described in \secref{sec:include},
but the subsequent selection of content must be done manually.
To that end, both |\ifchilddoc| and |\ifchilddocmanual|
will be true upon processing of a part,
and the name of the part is stored in |\childdocname|.
Note that |\jobname| will be set to the filename of the current part
so that each part receives an individual |.aux| file
that does not interfere with the |.aux| file(s) of the main document.
This behaviour can be altered by the alternative form
|\childdocby[*]{|\textit{main}|}| (with a non-empty optional argument)
which uses the |.aux| file of the main document
by setting |\jobname| to \textit{main}.

%%%%%%%%%%%%%%%%%%%%%%%%%%%%%%%%%%%%%%%%%%%%%%%%%%%%%%%%%%%%%%%%%%%%%%%%%%%%%%%%
\subsection{Driver Development}
\label{sec:driver}

The \textsf{childdoc} mechanism can also be use for the development
of definition files such as \LaTeX{} styles or classes.
This case differs from the above setup with multiple parts
included by |\include| in that no |\includeonly| should be invoked.
This can be achieved by starting the include file
(before |\ProvidesPackage|) with:
%
\begin{center}
\begin{tabular}{l}
|\input{childdoc.def}|\\
|\childdocforward{|\textit{main}|}|\\
\end{tabular}
\end{center}
%
or alternatively with:
%
\begin{center}
\begin{tabular}{l}
|\input{childdoc.def}|\\
|\childdocby{|\textit{main}|}|\\
\end{tabular}
\end{center}
%
Both forms have slightly different effects as described above.
The main file is prepared as usual, see \secref{sec:include}.

%%%%%%%%%%%%%%%%%%%%%%%%%%%%%%%%%%%%%%%%%%%%%%%%%%%%%%%%%%%%%%%%%%%%%%%%%%%%%%%%
\subsection{Legacy Detection}
\label{sec:detection}

The directive |\childdocmain| in the main file can detect
whether the complete document or merely a child is to be compiled
even without using the directive |\childdocof|.
This method is deprecated because it is less robust
and there is no compelling reason to use it;
it is merely provided for backward compatibility
and it may be removed in future versions.

If the detection mechanism is to be used,
it is mandatory to correctly specify
the filename of the main file as the argument of |\childdocmain|:
%
\begin{center}
\begin{tabular}{l}
|\input{childdoc.def}|\\
|\childdocmain{|\textit{main}|}|\\
\end{tabular}
\end{center}
%
If |\jobname| does not match the argument \textit{main} of |\childdocmain|,
it is assumed that |\jobname| points to the child file to be compiled.
When using |\childdocmain| with the main file specified as argument,
it suffices to start a child file
with just |\input{|\textit{main}|}|
without loading of the package and using |\childdocof|.
If instead all processing is done
with the appropriate \textsf{childdoc} directives,
the argument of \textit{main} of |\childdocmain| can be empty.

An alternative version of the command line processing described
in \secref{sec:commandline} using the detection mechanism reads:
%
\begin{center}
|... -jobname "|\textit{target}|" "|[\textit{flags}]%
[|\def\jobname{|\textit{dest}|}|]|\input{|\textit{main}|}"|
\end{center}

%%%%%%%%%%%%%%%%%%%%%%%%%%%%%%%%%%%%%%%%%%%%%%%%%%%%%%%%%%%%%%%%%%%%%%%%%%%%%%%%
\subsection{Manual Code}
\label{sec:manual}

In case one cannot be certain whether the definitions file |childdoc.def|
is installed on the target \TeX{} distribution
and one prefers not to ship it,
it is conceivable to paste a few relevant commands into the sources.

To that end, drop all statements |\input{childdoc.def}|
and perform the replacements as outlined below.
Instead of |\childdocmain{|\textit{main}|}| add the following code
to the top of the main file:
%
\begin{center}
\begin{tabular}{l}
|\||ifdefined\childdocname\endinput\||fi\newif\ifchilddoc|\\
|\edef\childdocname{\scantokens\expandafter{\jobname\noexpand}}|\\
|\def\childdocmain{|\textit{main}|}\||ifx\childdocmain\childdocname\||else|\\
|\childdoctrue\includeonly{\childdocname}\let\jobname\childdocmain\||fi|\\
\end{tabular}
\end{center}
%
Instead of |\childdocof{|\textit{main}|}| just include the main file
at the top of each child file:
%
\begin{center}
|\input{|\textit{main}|}|
\end{center}
%
A simple redirection |\childdocforward{|\textit{dest}|}| is achieved by:
%
\begin{center}
|\def\jobname{|\textit{dest}|}\input{\jobname}|
\end{center}
%
The redirection with prefix
|\childdocforwardprefix[|\textit{prefix}|]{|\textit{dest}|}|
is accomplished by:
%
\begin{center}
\begin{tabular}{l}
|{\edef\jobname{\scantokens\expandafter{\jobname\noexpand}}|\\
|\def\redirectjob |\textit{prefix}|#1~~~{\gdef\jobname{|\textit{dest}|#1}}|\\
|\expandafter\redirectjob\jobname~~~}\input{\jobname}|
\end{tabular}
\end{center}

In an alternative approach,
child documents can be compiled by a specific command line
without additional code or specific definitions:
%
\begin{center}
|... -jobname "|\textit{target}|" "|[\textit{flags}]%
|\includeonly{|\textit{dest}|}\input{|\textit{main}|}"|
\end{center}
%

%%%%%%%%%%%%%%%%%%%%%%%%%%%%%%%%%%%%%%%%%%%%%%%%%%%%%%%%%%%%%%%%%%%%%%%%%%%%%%%%
%%%%%%%%%%%%%%%%%%%%%%%%%%%%%%%%%%%%%%%%%%%%%%%%%%%%%%%%%%%%%%%%%%%%%%%%%%%%%%%%
\section{Information}

%%%%%%%%%%%%%%%%%%%%%%%%%%%%%%%%%%%%%%%%%%%%%%%%%%%%%%%%%%%%%%%%%%%%%%%%%%%%%%%%
\subsection{Copyright}

Copyright \copyright{} 2017--2018 Niklas Beisert

This work may be distributed and/or modified under the
conditions of the \LaTeX{} Project Public License, either version 1.3
of this license or (at your option) any later version.
The latest version of this license is in
  \url{http://www.latex-project.org/lppl.txt}
and version 1.3 or later is part of all distributions of \LaTeX{}
version 2005/12/01 or later.

This work has the LPPL maintenance status `maintained'.

The Current Maintainer of this work is Niklas Beisert.

This work consists of the files |README.txt|, |childdoc.ins| and |childdoc.dtx|
as well as the derived files |childdoc.def|, |cdocsamp.tex|
with |cdocsch1.tex|, |cdocsch2.tex|, |cdocspt3.tex|, |cdocspt4.tex|,
|cdocsdrf.tex|, |cdocsfn1.tex|, |cdocsfn2.tex|
as well as |childdoc.pdf|.

%%%%%%%%%%%%%%%%%%%%%%%%%%%%%%%%%%%%%%%%%%%%%%%%%%%%%%%%%%%%%%%%%%%%%%%%%%%%%%%%
\subsection{Files and Installation}

The package consists of the files:
%
\begin{center}
\begin{tabular}{ll}
    |README.txt|   & readme file \\
    |childdoc.ins| & installation file \\
    |childdoc.dtx| & source file \\
    |childdoc.def| & definition file \\
    |cdocsamp.tex| & sample main file \\
    |cdocsch1.tex| & sample include file \\
    |cdocsch2.tex| & sample include file \\
    |cdocspt3.tex| & sample part file \\
    |cdocspt4.tex| & sample part file \\
    |cdocsdrf.tex| & sample redirection file \\
    |cdocsfn1.tex| & sample redirection file \\
    |cdocsfn2.tex| & sample redirection file \\
    |childdoc.pdf| & manual
\end{tabular}
\end{center}
%
The distribution consists of the files
|README.txt|, |childdoc.ins| and |childdoc.dtx|.
%
\begin{itemize}
\item
Run (pdf)\LaTeX{} on |childdoc.dtx|
to compile the manual |childdoc.pdf| (this file).
\item
Run \LaTeX{} on |childdoc.ins| to create the definitions file |childdoc.def|
and the sample |cdocsamp.tex| with include files
|cdocsch1.tex|, |cdocsch2.tex|, |cdocspt3.tex|, |cdocspt4.tex|,
|cdocsdrf.tex|, |cdocsfn1.tex|, |cdocsfn2.tex|.
Then copy the file |childdoc.def| to an appropriate directory of your \LaTeX{}
distribution, e.g.\ \textit{texmf-root}|/tex/latex/childdoc|.
\end{itemize}

%%%%%%%%%%%%%%%%%%%%%%%%%%%%%%%%%%%%%%%%%%%%%%%%%%%%%%%%%%%%%%%%%%%%%%%%%%%%%%%%
\subsection{Related CTAN Packages}

There are several other packages which offer a similar functionality:
%
\begin{itemize}
\item
The packages
\href{http://ctan.org/pkg/docmute}{\textsf{docmute}},
\href{http://ctan.org/pkg/includex}{\textsf{includex}} and
\href{http://ctan.org/pkg/standalone}{\textsf{standalone}}
provide commands to include only the document body of
a child file thus allowing both files to be compiled individually.
\item
The packages \href{http://ctan.org/pkg/subdocs}{\textsf{subdocs}}
and \href{http://ctan.org/pkg/subfiles}{\textsf{subfiles}}
provide structures in which the main and child documents can be
encapsulated and allowing them to be compiled individually.
The inclusion mechanism is different from the conventional |\include|.
\item
The package \href{http://ctan.org/pkg/combine}{\textsf{combine}}
is an elaborate solution to combine several documents into one.
\end{itemize}
%
See also the CTAN topic \href{http://ctan.org/topic/subdocs}{\textsf{subdocs}}
for further related packages.
The present package differs from the above solutions in that
a document structure constructed with the conventional |\include| mechanism
just needs two extra commands at the top of every file
such that all constituent files can be compiled individually.

%%%%%%%%%%%%%%%%%%%%%%%%%%%%%%%%%%%%%%%%%%%%%%%%%%%%%%%%%%%%%%%%%%%%%%%%%%%%%%%%
%\subsection{Feature Suggestions}
%
%The following is a list of features which may be useful for future
%versions of this package:
%%
%\begin{itemize}
%\item
%\ldots
%\end{itemize}

%%%%%%%%%%%%%%%%%%%%%%%%%%%%%%%%%%%%%%%%%%%%%%%%%%%%%%%%%%%%%%%%%%%%%%%%%%%%%%%%
\subsection{Revision History}

%%%%%%%%%%%%%%%%%%%%%%%%%%%%%%%%%%%%%%%%
\paragraph{v2.0:} 2018/12/30

\begin{itemize}
\item
immediate forward processing
\item
added |\childdocby| mechanism
\item
manual restructured
\end{itemize}

%%%%%%%%%%%%%%%%%%%%%%%%%%%%%%%%%%%%%%%%
\paragraph{v1.6:} 2018/01/17

\begin{itemize}
\item
application for development of include files
\item
corrections to manual
\end{itemize}

%%%%%%%%%%%%%%%%%%%%%%%%%%%%%%%%%%%%%%%%
\paragraph{v1.5:} 2017/05/21

\begin{itemize}
\item
more complete structuring introduced
\item
|\childdocof| introduced
\item
|\childdoc| renamed to |\childdocmain|
\item
|\childredirect| renamed to |\childdocforward| and |\childdocforwardprefix|
and functionality expanded
\end{itemize}

%%%%%%%%%%%%%%%%%%%%%%%%%%%%%%%%%%%%%%%%
\paragraph{v1.0:} 2017/04/27

\begin{itemize}
\item
manual and install package
\item
first version published on CTAN
\end{itemize}

%%%%%%%%%%%%%%%%%%%%%%%%%%%%%%%%%%%%%%%%
\paragraph{v0.6:} 2017/04/26

\begin{itemize}
\item
redirection mechanism added
\end{itemize}

%%%%%%%%%%%%%%%%%%%%%%%%%%%%%%%%%%%%%%%%
\paragraph{v0.5:} 2017/04/26

\begin{itemize}
\item
functionality in definition file
\end{itemize}


%%%%%%%%%%%%%%%%%%%%%%%%%%%%%%%%%%%%%%%%%%%%%%%%%%%%%%%%%%%%%%%%%%%%%%%%%%%%%%%%
%%%%%%%%%%%%%%%%%%%%%%%%%%%%%%%%%%%%%%%%%%%%%%%%%%%%%%%%%%%%%%%%%%%%%%%%%%%%%%%%
%%%%%%%%%%%%%%%%%%%%%%%%%%%%%%%%%%%%%%%%%%%%%%%%%%%%%%%%%%%%%%%%%%%%%%%%%%%%%%%%
\appendix

\settowidth\MacroIndent{\rmfamily\scriptsize 000\ }

 \DocInput{childdoc.dtx}

\end{document}
%</driver>
% \fi
%
% %%%%%%%%%%%%%%%%%%%%%%%%%%%%%%%%%%%%%%%%%%%%%%%%%%%%%%%%%%%%%%%%%%%%%%%%%%%%%%
% %%%%%%%%%%%%%%%%%%%%%%%%%%%%%%%%%%%%%%%%%%%%%%%%%%%%%%%%%%%%%%%%%%%%%%%%%%%%%%
% \section{Sample}
%\iffalse
%<*samplemain>
%\fi
%
% The following presents a sample document
% with two chapters, two parts, a title page,
% a compile flag as well as three forwarding files to set the flag.
% It consists of eight |.tex| files:
% \begin{center}
% \begin{tabular}{ll}
% |cdocsamp.tex|&main file\\
% |cdocsch1.tex|&include file for chapter 1\\
% |cdocsch2.tex|&include file for chapter 2\\
% |cdocspt3.tex|&include file for part 3\\
% |cdocspt4.tex|&include file for part 4\\
% |cdocsdrf.tex|&forwarding file for main file in draft mode\\
% |cdocsfi1.tex|&forwarding file for final version of chapter 1\\
% |cdocsfi2.tex|&forwarding file for final version of chapter 2\\
% \end{tabular}
% \end{center}
% Each of the eight files can be compiled directly by the \LaTeX{} compiler.
%
% %%%%%%%%%%%%%%%%%%%%%%%%%%%%%%%%%%%%%%
% \paragraph{Main File.}
%
% The main file is called |cdocsamp.tex|.
%
% Load the \textsf{childdoc} definitions and
% declare the filename for the main document:
%    \begin{macrocode}
\input{childdoc.def}
\childdocmain{}
%    \end{macrocode}

% Optional override for |\version| flag:
%    \begin{macrocode}
%%\ifchilddoc\else\providecommand{\version}{draft}\fi
%    \end{macrocode}

% Define the default values for the |\version| flag
% (|final| for the main file and |draft| for childs):
%    \begin{macrocode}
\ifchilddoc
\providecommand{\version}{draft}
\else
\providecommand{\version}{final}
\fi
%    \end{macrocode}

% Load the standard document class:
%    \begin{macrocode}
\documentclass[12pt]{article}
%    \end{macrocode}

% Start the document body:
%    \begin{macrocode}
\begin{document}
%    \end{macrocode}

% Declare a title page.
% Print title, part of document being processed and version flag:
%    \begin{macrocode}
\addtocounter{page}{-1}
\begin{center}
{\LARGE\bfseries{}childdoc example\par}
\vspace{1cm}
\ifchilddoc
\ifchilddocmanual part\else chapter\fi:
`\childdocname' of `\childdocjob'\par
\else
main document: `\childdocjob'\par
\fi
version: \version\par
\end{center}
\newpage
%    \end{macrocode}

% Manually include selected file,
% otherwise process as usual:
%    \begin{macrocode}
\ifchilddocmanual
\section*{part `\childdocname'}
\input{\childdocname}
\else
%    \end{macrocode}

% Include the two chapters:
%    \begin{macrocode}
\include{cdocsch1}
\include{cdocsch2}
%    \end{macrocode}

% Include the two parts unless only chapters should be displayed:
%    \begin{macrocode}
\ifchilddoc\else
\section{part three}
\input{cdocspt3}
\section{part four}
\input{cdocspt4}
\fi
%    \end{macrocode}

% Process as usual until here:
%    \begin{macrocode}
\fi
%    \end{macrocode}

% End of document body:
%    \begin{macrocode}
\end{document}
%    \end{macrocode}
%\iffalse
%</samplemain>
%\fi
%
% %%%%%%%%%%%%%%%%%%%%%%%%%%%%%%%%%%%%%%
% \paragraph{Chapter Include Files.}
%
% The include files are called |cdocsch1.tex| and |cdocsch2.tex|.
%
%\iffalse
%<*samplechap1|samplechap2>
%\fi

% Optional override for |\version| flag:
%    \begin{macrocode}
%%\providecommand{\version}{final}
%    \end{macrocode}

% Include the main document:
%    \begin{macrocode}
\input{childdoc.def}
\childdocof{cdocsamp}
%    \end{macrocode}

%\iffalse
%</samplechap1|samplechap2>
%\fi
%
%\iffalse
%<*samplechap1>
%\fi
% Some text for chapter 1:
%    \begin{macrocode}
\section{one}
some text in chapter one
%    \end{macrocode}

%\iffalse
%</samplechap1>
%\fi
% Some text for chapter 2:
%\iffalse
%<*samplechap2>
%\fi
%    \begin{macrocode}
\section{two}
more text in chapter two
%    \end{macrocode}

%\iffalse
%</samplechap2>
%\fi
%
% %%%%%%%%%%%%%%%%%%%%%%%%%%%%%%%%%%%%%%
% \paragraph{Part Include Files.}
%
% The include files are called |cdocspt3.tex| and |cdocspt4.tex|.
%
%\iffalse
%<*samplepart3|samplepart4>
%\fi

% Optional override for |\version| flag:
%    \begin{macrocode}
%%\providecommand{\version}{final}
%    \end{macrocode}

% Include the main document:
%    \begin{macrocode}
\input{childdoc.def}
\childdocby{cdocsamp}
%    \end{macrocode}

%\iffalse
%</samplepart3|samplepart4>
%\fi
%
%\iffalse
%<*samplepart3>
%\fi
% Some text for part 3:
%    \begin{macrocode}
some text in part three
%    \end{macrocode}

%\iffalse
%</samplepart3>
%\fi
% Some text for part 4:
%\iffalse
%<*samplepart4>
%\fi
%    \begin{macrocode}
more text in part four
%    \end{macrocode}

%\iffalse
%</samplepart4>
%\fi
%
% %%%%%%%%%%%%%%%%%%%%%%%%%%%%%%%%%%%%%%
% \paragraph{Forwarding for a Complete Draft.}
%
% The following forwarding file |cdocsdrf.tex|
% compiles the main document in draft mode:
%\iffalse
%<*sampledraft>
%\fi
%    \begin{macrocode}
\def\version{draft}
\input{childdoc.def}
\childdocforward{cdocsamp}
%    \end{macrocode}

%\iffalse
%</sampledraft>
%\fi
%
% %%%%%%%%%%%%%%%%%%%%%%%%%%%%%%%%%%%%%%
% \paragraph{Forwarding for Final Version of the Chapters.}
%
% The following forwarding files |cdocsfn1.tex| and |cdocsfn2.tex|
% (with identical content)
% compile the final versions of the child documents
% |cdocsch1.tex| and |cdocsch2.tex|, respectively:
%\iffalse
%<*samplefinal>
%\fi
%    \begin{macrocode}
\def\version{final}
\input{childdoc.def}
\childdocforwardprefix[cdocsamp]{cdocsfn}{cdocsch}
%    \end{macrocode}

%\iffalse
%</samplefinal>
%\fi
%
% %%%%%%%%%%%%%%%%%%%%%%%%%%%%%%%%%%%%%%
% \paragraph{Command Line Processing.}
%
% The following three command lines generate the output files
% |cdocscld|, |cdocscl1| and |cdocscl2|
% which should be identical to
% |cdocsdrf|, |cdocsch1| and |cdocsfn2|, respectively:
% \begin{center}
% \begin{tabular}{l}
% |latex -jobname cdocscld \|\\
% |  "\def\version{draft}\input{childdoc.def}\childdocforward{cdocsamp}"|\\
% |latex -jobname cdocscl1 \|\\
% |  "\input{childdoc.def}\childdocforward[cdocsamp]{cdocsch1}"|\\
% |latex -jobname cdocscl2 \|\\
% |  "\def\version{final}\input{childdoc.def}\childdocforward{cdocsch2}"|
% \end{tabular}
% \end{center}
% Note that the trailing backslash on each first line
% merely continues the input to the second line
% (for convenient cut ant paste).
% Furthermore, the command |latex| can be replaced by any
% of its alternative versions such as |pdflatex|.
%
% %%%%%%%%%%%%%%%%%%%%%%%%%%%%%%%%%%%%%%%%%%%%%%%%%%%%%%%%%%%%%%%%%%%%%%%%%%%%%%
% %%%%%%%%%%%%%%%%%%%%%%%%%%%%%%%%%%%%%%%%%%%%%%%%%%%%%%%%%%%%%%%%%%%%%%%%%%%%%%
% \section{Implementation}
%\iffalse
%<*package>
%\fi
%
% This section describes the definitions file |childdoc.def|.

% The definitions cannot be loaded using |\usepackage| or |\RequirePackage|
% which has a mechanism to prevent loading a style file more than once.
% When loading the definitions by means of |\input|
% multiple instances have to be prevented manually:
%\iffalse
%This code needs to be before the `\ProvidesFile' directive
%which is defined at the beginning of this file.
%Therefore it is also placed there and commented out here.
%</package>
%<*discard>
%\fi
%    \begin{macrocode}
\ifdefined\childdocmain\endinput\fi
%    \end{macrocode}
%\iffalse
%</discard>
%<*package>
%\fi
%
% \macro{\ifchilddoc}
% \macro{\ifchilddocmanual}
% The conditional |\ifchilddoc| tells whether a
% child (true) or main (false) document is being compiled.
% The conditional |\ifchilddocmanual| tells whether
% the |\includeonly| mechanism is used (false) or
% the selection of child files must be performed manually (true).
% The definitions initialise to false:
%    \begin{macrocode}
\newif\ifchilddoc
\newif\ifchilddocmanual
%    \end{macrocode}

% \macro{\childdocname}
% \macro{\childdocjob}
% The macro |\childdocname| stores the name of the main document
% to be compiled. The macro |\childdocjob| stores the name of
% the document on which the \LaTeX{} compiler was originally invoked.
% The content of |\jobname| cannot be compared
% to filenames specified in the source due to different catcodes.
% The following code rescans |\jobname|, stores the result
% in |\childdocname| and saves a copy in |\childdocjob|:
%    \begin{macrocode}
\edef\childdocname{\scantokens\expandafter{\jobname\noexpand}}
\let\childdocjob\childdocname
%    \end{macrocode}

% \macro{\childdocdisable}
% The macro |\childdocdisable| prevents the main file
% from being processed more than once.
% At this stage, the main document command |\childdocmain|
% is assumed to be called once again where it should do nothing.
% Any subsequent call to it should prevent
% a secondary processing of the main document
% It overwrites the forwarding commands
% |\childdocof| and |\childdocforward|
% with empty macros to prevent further inclusions of the main document:
%    \begin{macrocode}
\newcommand{\childdocdisable}
{
  \renewcommand{\childdocmain}[1]{\renewcommand{\childdocmain}[1]{\endinput}}
  \renewcommand{\childdocof}[1]{}
  \renewcommand{\childdocby}[2][]{}
  \renewcommand{\childdocforward}[2][]{}
  \renewcommand{\childdocdisable}{}
}
%    \end{macrocode}

% \macro{\childdocmain}
% The macro |\childdocmain| is to be called at the top of the main file
% with nothing or the main filename (without extension) as argument.
% First, it breaks loops.
% If the argument is not empty and does not match |\childdocname|
% (which is set by the first inclusion of |childdoc.def|),
% |\ifchilddoc| is set to true, |\includeonly| is applied to the child file
% and |\jobname| is set to the main file
% (for proper handling of |.aux| files):
%    \begin{macrocode}
\newcommand{\childdocmain}[1]
{
  \childdocdisable\childdocmain{}
  \if?#1?\else
    \begingroup
      \def\childdoctmp{#1}
      \ifx\childdoctmp\childdocname
        \def\childdoctmp{}
      \else
        \def\childdoctmp
        {
          \childdoctrue
          \includeonly{\childdocname}
          \def\childdocjob{#1}
          \def\jobname{#1}
        }
      \fi
      \expandafter
    \endgroup
    \childdoctmp
  \fi
}
%    \end{macrocode}

% \macro{\childdocof}
% The command |\childdocof| redirects
% compilation to the main file |#1|.
%    \begin{macrocode}
\newcommand{\childdocof}[1]
{
  \childdocdisable
  \childdoctrue
  \includeonly{\childdocname}
  \def\jobname{#1}
  \def\childdocjob{#1}
  \input{#1}
}
%    \end{macrocode}

% \macro{\childdocby}
% The command |\childdocby| ....
%    \begin{macrocode}
\newcommand{\childdocby}[2][]
{
  \childdocdisable
  \childdoctrue
  \childdocmanualtrue
  \if?#1?\else
    \def\jobname{#2}
  \fi
  \def\childdocjob{#2}
  \input{#2}
  \endinput
}
%    \end{macrocode}

% \macro{\childdocforward}
% The command |\childdocforward| redirects
% compilation to the main file or
% (if the optional argument is given) a child file.
% Parameters are set as if the main file
% or a child file starting with |\childdocof| was compiled.
% Then compilation is handed over to the main file:
%    \begin{macrocode}
\newcommand{\childdocforward}[2][]
{
  \begingroup
    \if?#1?
      \def\childdoctmp
      {
        \def\childdocname{#2}
        \def\childdocjob{#2}
        \def\jobname{#2}
        \input{#2}
        \endinput
      }
    \else
      \def\childdoctmp
      {
        \childdocdisable
        \def\childdocname{#2}
        \childdoctrue
        \includeonly{#2}
        \def\childdocjob{#1}
        \def\jobname{#1}
        \input{#1}
        \endinput
      }
    \fi
    \expandafter
  \endgroup
  \childdoctmp
}
%    \end{macrocode}

% \macro{\childdocforwardprefix}
% The command |\childdocforwardprefix| redirects
% compilation to the main or a child file by means of a pattern.
% The prefix |#1| in the current filename is replaced by |#2|
% and the suffix of the current filename is kept
% (it is assumed that the filename does not contain the substring `|~~~|'
% which is used as a delimiter).
% Compilation is handed over to the new file by |\childdocforward|:
%    \begin{macrocode}
\newcommand{\childdocforwardprefix}[3][]
{
  \begingroup
    \def\childdocextract #2##1~~~{\def\childdoctmp{\childdocforward[#1]{#3##1}}}
    \expandafter\childdocextract\childdocname~~~
    \expandafter
  \endgroup
  \childdoctmp
}
%    \end{macrocode}

% \macro{\childdoc}
% The deprecated macro |\childdoc| is a legacy version of |\childdocmain|:
%    \begin{macrocode}
\newcommand{\childdoc}{\childdocmain}
%    \end{macrocode}

% \macro{\childdocredirect}
% The deprecated macro |\childdocredirect| is a legacy version
% of |\childdocforward| and |\childdocforwardprefix|:
%    \begin{macrocode}
\newcommand{\childdocredirect}[2][]
{
  \begingroup
    \if?#1?
      \def\childdoctmp{\childdocforward{#2}}
    \else
      \def\childdoctmp{\childdocforwardprefix{#1}{#2}}
    \fi
    \expandafter
  \endgroup
  \childdoctmp
}
%    \end{macrocode}

%\iffalse
%</package>
%\fi
%
\endinput
\childdocforward[cdocsamp]{cdocsch1}"|\\
% |latex -jobname cdocscl2 \|\\
% |  "\def\version{final}% \iffalse
%
% childdoc.dtx Copyright (C) 2017-2018 Niklas Beisert
%
% This work may be distributed and/or modified under the
% conditions of the LaTeX Project Public License, either version 1.3
% of this license or (at your option) any later version.
% The latest version of this license is in
%   http://www.latex-project.org/lppl.txt
% and version 1.3 or later is part of all distributions of LaTeX
% version 2005/12/01 or later.
%
% This work has the LPPL maintenance status `maintained'.
%
% The Current Maintainer of this work is Niklas Beisert.
%
% This work consists of the files childdoc.dtx and childdoc.ins
% and the derived files childdoc.def and cdocsamp.tex with
% cdocsch1.tex, cdocsch2.tex, cdocsdrf.tex, cdocsfn1.tex, cdocsfn2.tex.
%
%<package>\ifdefined\childdocmain\endinput\fi
%<package>\ProvidesFile{childdoc.def}[2018/12/30 v2.0 child document driver]
%<samplemain>\ProvidesFile{cdocsamp.tex}[2018/12/30 v2.0 sample for childdoc]
%<*driver>
%\ProvidesFile{childdoc.drv}[2018/12/30 v2.0 childdoc reference manual file]
\PassOptionsToClass{10pt,a4paper}{article}
\documentclass{ltxdoc}

\usepackage[margin=35mm]{geometry}
\usepackage{hyperref}
\usepackage{hyperxmp}
\usepackage[usenames]{color}

\hypersetup{colorlinks=true}
\hypersetup{pdfstartview=FitH}
\hypersetup{pdfpagemode=UseNone}
\hypersetup{pdfsource={}}
\hypersetup{pdflang={en-UK}}
\hypersetup{pdfcopyright={Copyright 2017-2018 Niklas Beisert.
  This work may be distributed and/or modified under the
  conditions of the LaTeX Project Public License, either version 1.3
  of this license or (at your option) any later version.}}
\hypersetup{pdflicenseurl={http://www.latex-project.org/lppl.txt}}
\hypersetup{pdfcontactaddress={ETH Zurich, ITP, HIT K,
  Wolfgang-Pauli-Strasse 27}}
\hypersetup{pdfcontactpostcode={8093}}
\hypersetup{pdfcontactcity={Zurich}}
\hypersetup{pdfcontactcountry={Switzerland}}
\hypersetup{pdfcontactemail={nbeisert@itp.phys.ethz.ch}}
\hypersetup{pdfcontacturl={http://people.phys.ethz.ch/\xmptilde nbeisert/}}

\newcommand{\secref}[1]{\hyperref[#1]{section \ref*{#1}}}

\parskip1ex
\parindent0pt
\let\olditemize\itemize
\def\itemize{\olditemize\parskip0pt}

\begin{document}

\title{The \textsf{childdoc} Package}
\hypersetup{pdftitle={The childdoc Package}}
\author{Niklas Beisert\\[2ex]
  Institut f\"ur Theoretische Physik\\
  Eidgen\"ossische Technische Hochschule Z\"urich\\
  Wolfgang-Pauli-Strasse 27, 8093 Z\"urich, Switzerland\\[1ex]
  \href{mailto:nbeisert@itp.phys.ethz.ch}
  {\texttt{nbeisert@itp.phys.ethz.ch}}}
\hypersetup{pdfauthor={Niklas Beisert}}
\hypersetup{pdfsubject={Manual for the LaTeX2e Package childdoc}}
\date{30 December 2018, \textsf{v2.0}}
\maketitle

\begin{abstract}\noindent
\textsf{childdoc} is a \LaTeXe{} package
that enables the direct compilation
of document sections included by |\include|
to individual files.
\end{abstract}

\begingroup
\parskip0ex
\tableofcontents
\endgroup

%%%%%%%%%%%%%%%%%%%%%%%%%%%%%%%%%%%%%%%%%%%%%%%%%%%%%%%%%%%%%%%%%%%%%%%%%%%%%%%%
%%%%%%%%%%%%%%%%%%%%%%%%%%%%%%%%%%%%%%%%%%%%%%%%%%%%%%%%%%%%%%%%%%%%%%%%%%%%%%%%
\section{Introduction}

\LaTeX{} provides a mechanism to structure a large document (such as a book)
into a main file and several child files (containing the chapters)
using the |\include| command.
This mechanism is beneficial for documents
which span hundreds of pages in order to
make the source file(s) more manageable.
Moreover, compilation can be restricted to
selected child files by means of the |\includeonly| command.
The latter feature can be used to reduce the compilation time while editing
(this was significantly more useful in the earlier days of \LaTeX{})
or to generate a smaller document which is easier to navigate.
Another application of |\includeonly| is to generate
documents consisting of selected parts of the complete document.

However, there are a few drawbacks of the plain |\include| mechanism:
\begin{itemize}
\item
The child files cannot be compiled on their own,
they can only be compiled via the main file.
A naive editing environment
(such as a text editor with an option
to have the current file processed by \LaTeX)
may require one to switch to the main file before compiling;
attempting to compile the child file produces errors.
\item
The main file must be modified (each time)
to adjust the |\includeonly| command
to the present needs. This easily leaves the main file in a messy state.
\item
The generated document will always carry the filename
of the main document. This is inconvenient if
several child files are to be compiled and
to be kept for distribution.
\end{itemize}

The present package provides a simple interface
to make child files individually compilable by \LaTeX{}.
Compiling a child file then has the same effect as compiling
the main file with an |\includeonly| command
to select the appropriate child.
Moreover the generated document will carry the name of the child
rather than the main file.
This resolves all three above issues.

This feature is meant to make the editing of books,
thesis documents and lecture notes somewhat more convenient.
However, the package can also be used efficiently for
composing a series of documents (such as exercise sheets)
which are typically distributed individually.
It then assists the author in generating the individual documents
(potentially in different versions)
as well as a document containing the collected series.
Another application is in developing style files
or other kinds of included material
where compilation of the style file could redirect
to a sample or test file.

%%%%%%%%%%%%%%%%%%%%%%%%%%%%%%%%%%%%%%%%%%%%%%%%%%%%%%%%%%%%%%%%%%%%%%%%%%%%%%%%
%%%%%%%%%%%%%%%%%%%%%%%%%%%%%%%%%%%%%%%%%%%%%%%%%%%%%%%%%%%%%%%%%%%%%%%%%%%%%%%%
\section{Usage}

First of all, the package \textsf{childdoc} is \emph{not} a standard
\LaTeXe{} |.sty| style file! Therefore it needs to be invoked in
a non-standard way.

%%%%%%%%%%%%%%%%%%%%%%%%%%%%%%%%%%%%%%%%%%%%%%%%%%%%%%%%%%%%%%%%%%%%%%%%%%%%%%%%
\subsection{Included Files}
\label{sec:include}

%%%%%%%%%%%%%%%%%%%%%%%%%%%%%%%%%%%%%%%%
\DescribeMacro{\childdocmain}
To use the package, add the commands
\begin{center}
\begin{tabular}{l}
|\input{childdoc.def}|\\
|\childdocmain{}|\\
\end{tabular}
\end{center}
at the very top of the main \LaTeX{} file,
in particular \emph{before} the |\documentclass| statement!
The argument of |\childdocmain| should be left empty
(but it must be present).

%%%%%%%%%%%%%%%%%%%%%%%%%%%%%%%%%%%%%%%%
\DescribeMacro{\childdocof}
Furthermore, add the commands
\begin{center}
\begin{tabular}{l}
|\input{childdoc.def}|\\
|\childdocof{|\textit{main}|}|\\
\end{tabular}
\end{center}
at the top of every child file \textit{child}
which is included by |\include{|\textit{child}|}|
from within the main file
(or at least for those files to be compiled individually).
The argument \textit{main} must be the filename of the main file.

There are a couple of
considerations in setting up the main and child documents:

%%%%%%%%%%%%%%%%%%%%%%%%%%%%%%%%%%%%%%%%
\paragraph{Restrictions.}

Please note the following restrictions:
\begin{itemize}
\item
|\childdocmain| must be called with one argument \textit{main}
to ensure compatibility with earlier version of the package.
It must either be empty (|\childdocmain{}|)
or precisely match the filename of the main file in which it is specified.
See \secref{sec:detection} for further information.
\item
The filename \textit{main} must be specified without the |.tex| extension.
\item
The filename \textit{main} is case sensitive
(even in case-insensitive file systems)
due to internal string comparison.
\item
The argument \textit{main} should be fully expanded, it cannot be a macro.
\item
Subdirectories and special characters should be avoided in filenames.
\item
The command |\childdocmain{|\textit{main}|}| must be followed by a whitespace.
It should not be followed immediately by another command
or by a comment mark `|%|'.
This is because the \TeX{} parser reads the token immediately following
the argument of |\childdocmain| and puts it
at the beginning of every child section;
however, a white\-space is ignored.
\end{itemize}

%%%%%%%%%%%%%%%%%%%%%%%%%%%%%%%%%%%%%%%%
\paragraph{Content of Main File.}

It is advisable to place all content in the child files included by |\include|.
Any output contained in the main file will appear in all child documents
unless suppressed manually;
it cannot be suppressed automatically by the |\includeonly| directive
and thus should normally be avoided.
A method to include some content in the main file
by means of conditional processing is described in \secref{sec:conditional}.

%%%%%%%%%%%%%%%%%%%%%%%%%%%%%%%%%%%%%%%%
\paragraph{Page Numbering.}

When only a part of the document is compiled,
the appropriate numbering of pages
(as well as other status parameters)
is determined from the |.aux| files.
The latter contain information from previous passes.
However this information needs to propagate through
all intermediate child documents.
Therefore the page numbering in child documents may well
be inconsistent until the complete document is compiled at least once.

A useful (if unconventional) way to always ensure a consistent
page numbering is to restart the numbering in each child document
and denote the pages by `\textit{child}|.|\textit{page}'
where \textit{child} represents the chapter/section number of the child file.
This can be achieved by the command
|\numberwithin{page}{|\textit{child}|}|
of the \textsf{amsmath} package
where \textit{child} can be |chapter| or |section|
depending on the chosen structuring.
Alternatively, one can modify the macro |\thepage| appropriately
and reset the counter |page| at the start of each child file.

%%%%%%%%%%%%%%%%%%%%%%%%%%%%%%%%%%%%%%%%%%%%%%%%%%%%%%%%%%%%%%%%%%%%%%%%%%%%%%%%
\subsection{Conditional Processing}
\label{sec:conditional}

The package provides a mechanism to compile different versions
of a document. To customise the versions further some conditional processing
can come in handy to distinguish which version is being compiled.
The package provides two macros to describe the compilation context:

%%%%%%%%%%%%%%%%%%%%%%%%%%%%%%%%%%%%%%%%
\DescribeMacro{\ifchilddoc}
The conditional |\ifchilddoc| distinguishes between the compilation of
child documents and the main document:
%
\begin{center}
|\ifchilddoc |\textit{child-code}| |[|\||else |\textit{main-code}]| \||fi|
\end{center}

%%%%%%%%%%%%%%%%%%%%%%%%%%%%%%%%%%%%%%%%
\DescribeMacro{\childdocname}
\DescribeMacro{\childdocjob}
The macro |\childdocname| contains the filename (without extension)
of the main or child file being processed.
Note that |\childdocjob| will always contain the name of the main file.

%%%%%%%%%%%%%%%%%%%%%%%%%%%%%%%%%%%%%%%%
\paragraph{Title Page.}

Conditional processing can be used to include a title or banner page
in the main document when proper precautions are taken.
Importantly, the code in the main file should ensure that the page counter
(as well as other status parameters which are stored in the |.aux| files)
takes the same value after the conditional processing.
Otherwise the page numbers may take divergent values
depending on which part is compiled.

For example, a title page could be declared by:
%
\begin{center}
\begin{tabular}{l}
|\ifchilddoc\||else|\\
|\addtocounter{page}{-1}|\\
\textit{code for title page}\\
|\newpage|\\
|\||fi|
\end{tabular}
\end{center}
%
A banner page for the child documents can be generated by:
%
\begin{center}
\begin{tabular}{l}
|\ifchilddoc|\\
|\addtocounter{page}{-1}|\\
\textit{code for banner page}\\
|\newpage|\\
|\||fi|
\end{tabular}
\end{center}
%
Here one could write a message such as:
\begin{center}
|This is the part \childdocname{} of \childdocjob{}.|
\end{center}

%%%%%%%%%%%%%%%%%%%%%%%%%%%%%%%%%%%%%%%%%%%%%%%%%%%%%%%%%%%%%%%%%%%%%%%%%%%%%%%%
\subsection{Flags}
\label{sec:flags}

The package makes it easy to generate different versions
of the main or child documents.
To this end compilation flags can be defined
and assigned different default values.
They will be particularly useful in conjunction
with the forwarding mechanism described in \secref{sec:forward}.

For example, it may be useful to have a flag |\version|
which can be set to |draft| or |final|.
The document source will contain some conditional code
depending on the value of |\version|.
Suppose further, the flag should default to |final| for the main file
and to |draft| for child files
which is a natural assignment for editing the document.
This is achieved by placing the following code
in the preamble of the main document
(below the |\childdocmain| directive):
%
\begin{center}
\begin{tabular}{l}
|\ifchilddoc|\\
|\providecommand{\version}{draft}|\\
|\||else|\\
|\providecommand{\version}{final}|\\
|\||fi|
\end{tabular}
\end{center}
%
The definition by |\providecommand| makes sure
that previous definitions are not overwritten.
Further statements |\providecommand{\version}{...}|
can thus be added before the above code to override it.

For the main file, one might add a line
(between |\childdocmain| and the above block)
%
\begin{center}
|%\ifchilddoc\||else\providecommand{\version}{draft}\||fi|
\end{center}
%
which can be uncommented to produce a draft version.
Likewise one can add a line to the very top of a child file
(above the |\childdocof{|\textit{main}|}| directive)
%
\begin{center}
|%\providecommand{\version}{final}|
\end{center}
%
which can be uncommented to produce the final version of this child document.

%%%%%%%%%%%%%%%%%%%%%%%%%%%%%%%%%%%%%%%%%%%%%%%%%%%%%%%%%%%%%%%%%%%%%%%%%%%%%%%%
\subsection{Forwarding}
\label{sec:forward}

Different versions of the main or child documents
using compilation flags as described in \secref{sec:flags}
can be (permanently) stored in different files
for convenient compilation, viewing and distribution.
To this end, the package defines a command
to pass on compilation to a different file:

%%%%%%%%%%%%%%%%%%%%%%%%%%%%%%%%%%%%%%%%
\DescribeMacro{\childdocforward}
The command |\childdocforward| redirects processing to
another source file:
%
\begin{center}
\begin{tabular}{l}
|\input{childdoc.def}|\\
|\childdocforward[|\textit{main}|]{|\textit{dest}|}|\\
\end{tabular}
\end{center}
%
The argument \textit{dest} is the destination file
(without extension).
It should be the main file or one of the child files.
Note that further \textsf{childdoc} directives
such as |\childdocof| and |\childdocforward|
in the indicated file will be processed in this form.
The optional argument \textit{main}
passes on directly to the main file \textit{main}
while pretending to compile the child \textit{dest}.
This form behaves as if \textit{dest}
issues |\childdocof{|\textit{main}|}| right away,
and no further \textsf{childdoc} directives will be processed.

%%%%%%%%%%%%%%%%%%%%%%%%%%%%%%%%%%%%%%%%
\DescribeMacro{\...prefix}
In the alternative form |\childdocforwardprefix|,
%
\begin{center}
\begin{tabular}{l}
|\input{childdoc.def}|\\
|\childdocforwardprefix[|\textit{main}|]{|\textit{prefix}|}{|\textit{dest}|}|
\end{tabular}
\end{center}
%
the destination file is determined by a pattern
depending on the current file:
To make this work, the current file must be called
`{\textit{prefix}\hspace{0.2em}\textit{suffix}}'
with \textit{prefix} matching precisely the argument.
Processing is then passed on to the file
`{\textit{dest}\hspace{0.2em}\textit{suffix}}'.
Surely, the same effect is achieved by
directly specifying the
argument `{\textit{dest}\hspace{0.2em}\textit{suffix}}'
in the first form.
However, that requires to set up a different file
for each child. With the alternative form of the command
all these files can have exactly the same content
which simplifies setting them up and maintaining them.

For example, the following file |draft.tex|
with a compilation flag |\version| as described in \secref{sec:flags}
compiles the main document as a draft:
%
\begin{center}
\begin{tabular}{l}
|\def\version{draft}|\\
|\input{childdoc.def}|\\
|\childdocforward{|\textit{main}|}|
\end{tabular}
\end{center}
%
Likewise, the following files |final|\textit{nn}|.tex|
compile the final version of the child document
|child|\textit{nn}|.tex|:
%
\begin{center}
\begin{tabular}{l}
|\def\version{final}|\\
|\input{childdoc.def}|\\
|\childdocforwardprefix{final}{child}|
\end{tabular}
\end{center}
%

Note that when several versions of a main file and/or of each child file
are to be generated, it may be convenient to set up a |Makefile| or
shell script to automatise the process.

%%%%%%%%%%%%%%%%%%%%%%%%%%%%%%%%%%%%%%%%%%%%%%%%%%%%%%%%%%%%%%%%%%%%%%%%%%%%%%%%
\subsection{Command Line Processing}
\label{sec:commandline}

The effect of redirection files can also be achieved by invoking
the \LaTeX{} compiler with a more elaborate command line.
Most conveniently this should be done as part
of a shell script or a |Makefile|.

When using \textsf{childdoc} in the main file, the following
command lines effectively perform a redirection
(note that depending on the shell being used,
backslashes may have to be doubled: `|\|' $\to$ `|\\|'):
%
\begin{center}
|... -jobname "|\textit{target}|" |\\|"|[\textit{flags}]%
|\input{childdoc.def}\childdocforward[|\textit{main}|]{|\textit{dest}|}"|
\end{center}
%
Here \textit{target} is the name of the output file,
\textit{main} is the name of the main file
and \textit{dest} is the name of the main or child file to be processed
(all filenames without extensions).
The optional argument \textit{main} can be omitted
if \textit{main} matches \textit{dest}.
Optionally, compilation \textit{flags} can be defined via |\def| commands.
This command line makes the \TeX{} engine believe
it is compiling the file \textit{target}
whose content is specified as the latter parameter.
The provided code then forwards the processing to
\textit{main} or \textit{dest} as described in \secref{sec:forward}.

%%%%%%%%%%%%%%%%%%%%%%%%%%%%%%%%%%%%%%%%%%%%%%%%%%%%%%%%%%%%%%%%%%%%%%%%%%%%%%%%
\subsection{Include by Input}
\label{sec:input}

Including child documents by |\include| has some restrictions by design.
Most notably, the content of a child document always occupies
its own set of pages; pages cannot be shared between child documents.
Usually, this behaviour makes perfect sense
because each child document contain an essential part of the document.
However, in some situations it may be desirable to compose
a document from a collection of parts
without having mandatory page breaks between then.
For this case, the package
provides a mechanism to include parts
by |\input| which can also be processed individually.
However, by construction this mechanism
requires manual handling of the content to be output.

%%%%%%%%%%%%%%%%%%%%%%%%%%%%%%%%%%%%%%%%
\DescribeMacro{\ifchilddocmanual}
The main file should be prepared as usual, see \secref{sec:include}.
However, the document body must make a distinction
between processing of an individual part and of the main document, e.g.:
%
\begin{center}
\begin{tabular}{l}
|\ifchilddocmanual|\\
|\input{\childdocname}|\\
|\||else|\\
\textit{document body with }|\input{|\textit{part}|}|\\
|\||fi|
\end{tabular}
\end{center}
%
The conditional |\ifchilddocmanual| is true whenever
a part to be included by |\input| is being compiled,
and the name of the part is stored in |\childdocname|.

%%%%%%%%%%%%%%%%%%%%%%%%%%%%%%%%%%%%%%%%
\DescribeMacro{\childdocby}
Each part to be included by |\input| should start with:
%
\begin{center}
\begin{tabular}{l}
|\input{childdoc.def}|\\
|\childdocby{|\textit{main}|}|\\
\end{tabular}
\end{center}
%
The directive |\childdocby| is similar to |\childdocof|
described in \secref{sec:include},
but the subsequent selection of content must be done manually.
To that end, both |\ifchilddoc| and |\ifchilddocmanual|
will be true upon processing of a part,
and the name of the part is stored in |\childdocname|.
Note that |\jobname| will be set to the filename of the current part
so that each part receives an individual |.aux| file
that does not interfere with the |.aux| file(s) of the main document.
This behaviour can be altered by the alternative form
|\childdocby[*]{|\textit{main}|}| (with a non-empty optional argument)
which uses the |.aux| file of the main document
by setting |\jobname| to \textit{main}.

%%%%%%%%%%%%%%%%%%%%%%%%%%%%%%%%%%%%%%%%%%%%%%%%%%%%%%%%%%%%%%%%%%%%%%%%%%%%%%%%
\subsection{Driver Development}
\label{sec:driver}

The \textsf{childdoc} mechanism can also be use for the development
of definition files such as \LaTeX{} styles or classes.
This case differs from the above setup with multiple parts
included by |\include| in that no |\includeonly| should be invoked.
This can be achieved by starting the include file
(before |\ProvidesPackage|) with:
%
\begin{center}
\begin{tabular}{l}
|\input{childdoc.def}|\\
|\childdocforward{|\textit{main}|}|\\
\end{tabular}
\end{center}
%
or alternatively with:
%
\begin{center}
\begin{tabular}{l}
|\input{childdoc.def}|\\
|\childdocby{|\textit{main}|}|\\
\end{tabular}
\end{center}
%
Both forms have slightly different effects as described above.
The main file is prepared as usual, see \secref{sec:include}.

%%%%%%%%%%%%%%%%%%%%%%%%%%%%%%%%%%%%%%%%%%%%%%%%%%%%%%%%%%%%%%%%%%%%%%%%%%%%%%%%
\subsection{Legacy Detection}
\label{sec:detection}

The directive |\childdocmain| in the main file can detect
whether the complete document or merely a child is to be compiled
even without using the directive |\childdocof|.
This method is deprecated because it is less robust
and there is no compelling reason to use it;
it is merely provided for backward compatibility
and it may be removed in future versions.

If the detection mechanism is to be used,
it is mandatory to correctly specify
the filename of the main file as the argument of |\childdocmain|:
%
\begin{center}
\begin{tabular}{l}
|\input{childdoc.def}|\\
|\childdocmain{|\textit{main}|}|\\
\end{tabular}
\end{center}
%
If |\jobname| does not match the argument \textit{main} of |\childdocmain|,
it is assumed that |\jobname| points to the child file to be compiled.
When using |\childdocmain| with the main file specified as argument,
it suffices to start a child file
with just |\input{|\textit{main}|}|
without loading of the package and using |\childdocof|.
If instead all processing is done
with the appropriate \textsf{childdoc} directives,
the argument of \textit{main} of |\childdocmain| can be empty.

An alternative version of the command line processing described
in \secref{sec:commandline} using the detection mechanism reads:
%
\begin{center}
|... -jobname "|\textit{target}|" "|[\textit{flags}]%
[|\def\jobname{|\textit{dest}|}|]|\input{|\textit{main}|}"|
\end{center}

%%%%%%%%%%%%%%%%%%%%%%%%%%%%%%%%%%%%%%%%%%%%%%%%%%%%%%%%%%%%%%%%%%%%%%%%%%%%%%%%
\subsection{Manual Code}
\label{sec:manual}

In case one cannot be certain whether the definitions file |childdoc.def|
is installed on the target \TeX{} distribution
and one prefers not to ship it,
it is conceivable to paste a few relevant commands into the sources.

To that end, drop all statements |\input{childdoc.def}|
and perform the replacements as outlined below.
Instead of |\childdocmain{|\textit{main}|}| add the following code
to the top of the main file:
%
\begin{center}
\begin{tabular}{l}
|\||ifdefined\childdocname\endinput\||fi\newif\ifchilddoc|\\
|\edef\childdocname{\scantokens\expandafter{\jobname\noexpand}}|\\
|\def\childdocmain{|\textit{main}|}\||ifx\childdocmain\childdocname\||else|\\
|\childdoctrue\includeonly{\childdocname}\let\jobname\childdocmain\||fi|\\
\end{tabular}
\end{center}
%
Instead of |\childdocof{|\textit{main}|}| just include the main file
at the top of each child file:
%
\begin{center}
|\input{|\textit{main}|}|
\end{center}
%
A simple redirection |\childdocforward{|\textit{dest}|}| is achieved by:
%
\begin{center}
|\def\jobname{|\textit{dest}|}\input{\jobname}|
\end{center}
%
The redirection with prefix
|\childdocforwardprefix[|\textit{prefix}|]{|\textit{dest}|}|
is accomplished by:
%
\begin{center}
\begin{tabular}{l}
|{\edef\jobname{\scantokens\expandafter{\jobname\noexpand}}|\\
|\def\redirectjob |\textit{prefix}|#1~~~{\gdef\jobname{|\textit{dest}|#1}}|\\
|\expandafter\redirectjob\jobname~~~}\input{\jobname}|
\end{tabular}
\end{center}

In an alternative approach,
child documents can be compiled by a specific command line
without additional code or specific definitions:
%
\begin{center}
|... -jobname "|\textit{target}|" "|[\textit{flags}]%
|\includeonly{|\textit{dest}|}\input{|\textit{main}|}"|
\end{center}
%

%%%%%%%%%%%%%%%%%%%%%%%%%%%%%%%%%%%%%%%%%%%%%%%%%%%%%%%%%%%%%%%%%%%%%%%%%%%%%%%%
%%%%%%%%%%%%%%%%%%%%%%%%%%%%%%%%%%%%%%%%%%%%%%%%%%%%%%%%%%%%%%%%%%%%%%%%%%%%%%%%
\section{Information}

%%%%%%%%%%%%%%%%%%%%%%%%%%%%%%%%%%%%%%%%%%%%%%%%%%%%%%%%%%%%%%%%%%%%%%%%%%%%%%%%
\subsection{Copyright}

Copyright \copyright{} 2017--2018 Niklas Beisert

This work may be distributed and/or modified under the
conditions of the \LaTeX{} Project Public License, either version 1.3
of this license or (at your option) any later version.
The latest version of this license is in
  \url{http://www.latex-project.org/lppl.txt}
and version 1.3 or later is part of all distributions of \LaTeX{}
version 2005/12/01 or later.

This work has the LPPL maintenance status `maintained'.

The Current Maintainer of this work is Niklas Beisert.

This work consists of the files |README.txt|, |childdoc.ins| and |childdoc.dtx|
as well as the derived files |childdoc.def|, |cdocsamp.tex|
with |cdocsch1.tex|, |cdocsch2.tex|, |cdocspt3.tex|, |cdocspt4.tex|,
|cdocsdrf.tex|, |cdocsfn1.tex|, |cdocsfn2.tex|
as well as |childdoc.pdf|.

%%%%%%%%%%%%%%%%%%%%%%%%%%%%%%%%%%%%%%%%%%%%%%%%%%%%%%%%%%%%%%%%%%%%%%%%%%%%%%%%
\subsection{Files and Installation}

The package consists of the files:
%
\begin{center}
\begin{tabular}{ll}
    |README.txt|   & readme file \\
    |childdoc.ins| & installation file \\
    |childdoc.dtx| & source file \\
    |childdoc.def| & definition file \\
    |cdocsamp.tex| & sample main file \\
    |cdocsch1.tex| & sample include file \\
    |cdocsch2.tex| & sample include file \\
    |cdocspt3.tex| & sample part file \\
    |cdocspt4.tex| & sample part file \\
    |cdocsdrf.tex| & sample redirection file \\
    |cdocsfn1.tex| & sample redirection file \\
    |cdocsfn2.tex| & sample redirection file \\
    |childdoc.pdf| & manual
\end{tabular}
\end{center}
%
The distribution consists of the files
|README.txt|, |childdoc.ins| and |childdoc.dtx|.
%
\begin{itemize}
\item
Run (pdf)\LaTeX{} on |childdoc.dtx|
to compile the manual |childdoc.pdf| (this file).
\item
Run \LaTeX{} on |childdoc.ins| to create the definitions file |childdoc.def|
and the sample |cdocsamp.tex| with include files
|cdocsch1.tex|, |cdocsch2.tex|, |cdocspt3.tex|, |cdocspt4.tex|,
|cdocsdrf.tex|, |cdocsfn1.tex|, |cdocsfn2.tex|.
Then copy the file |childdoc.def| to an appropriate directory of your \LaTeX{}
distribution, e.g.\ \textit{texmf-root}|/tex/latex/childdoc|.
\end{itemize}

%%%%%%%%%%%%%%%%%%%%%%%%%%%%%%%%%%%%%%%%%%%%%%%%%%%%%%%%%%%%%%%%%%%%%%%%%%%%%%%%
\subsection{Related CTAN Packages}

There are several other packages which offer a similar functionality:
%
\begin{itemize}
\item
The packages
\href{http://ctan.org/pkg/docmute}{\textsf{docmute}},
\href{http://ctan.org/pkg/includex}{\textsf{includex}} and
\href{http://ctan.org/pkg/standalone}{\textsf{standalone}}
provide commands to include only the document body of
a child file thus allowing both files to be compiled individually.
\item
The packages \href{http://ctan.org/pkg/subdocs}{\textsf{subdocs}}
and \href{http://ctan.org/pkg/subfiles}{\textsf{subfiles}}
provide structures in which the main and child documents can be
encapsulated and allowing them to be compiled individually.
The inclusion mechanism is different from the conventional |\include|.
\item
The package \href{http://ctan.org/pkg/combine}{\textsf{combine}}
is an elaborate solution to combine several documents into one.
\end{itemize}
%
See also the CTAN topic \href{http://ctan.org/topic/subdocs}{\textsf{subdocs}}
for further related packages.
The present package differs from the above solutions in that
a document structure constructed with the conventional |\include| mechanism
just needs two extra commands at the top of every file
such that all constituent files can be compiled individually.

%%%%%%%%%%%%%%%%%%%%%%%%%%%%%%%%%%%%%%%%%%%%%%%%%%%%%%%%%%%%%%%%%%%%%%%%%%%%%%%%
%\subsection{Feature Suggestions}
%
%The following is a list of features which may be useful for future
%versions of this package:
%%
%\begin{itemize}
%\item
%\ldots
%\end{itemize}

%%%%%%%%%%%%%%%%%%%%%%%%%%%%%%%%%%%%%%%%%%%%%%%%%%%%%%%%%%%%%%%%%%%%%%%%%%%%%%%%
\subsection{Revision History}

%%%%%%%%%%%%%%%%%%%%%%%%%%%%%%%%%%%%%%%%
\paragraph{v2.0:} 2018/12/30

\begin{itemize}
\item
immediate forward processing
\item
added |\childdocby| mechanism
\item
manual restructured
\end{itemize}

%%%%%%%%%%%%%%%%%%%%%%%%%%%%%%%%%%%%%%%%
\paragraph{v1.6:} 2018/01/17

\begin{itemize}
\item
application for development of include files
\item
corrections to manual
\end{itemize}

%%%%%%%%%%%%%%%%%%%%%%%%%%%%%%%%%%%%%%%%
\paragraph{v1.5:} 2017/05/21

\begin{itemize}
\item
more complete structuring introduced
\item
|\childdocof| introduced
\item
|\childdoc| renamed to |\childdocmain|
\item
|\childredirect| renamed to |\childdocforward| and |\childdocforwardprefix|
and functionality expanded
\end{itemize}

%%%%%%%%%%%%%%%%%%%%%%%%%%%%%%%%%%%%%%%%
\paragraph{v1.0:} 2017/04/27

\begin{itemize}
\item
manual and install package
\item
first version published on CTAN
\end{itemize}

%%%%%%%%%%%%%%%%%%%%%%%%%%%%%%%%%%%%%%%%
\paragraph{v0.6:} 2017/04/26

\begin{itemize}
\item
redirection mechanism added
\end{itemize}

%%%%%%%%%%%%%%%%%%%%%%%%%%%%%%%%%%%%%%%%
\paragraph{v0.5:} 2017/04/26

\begin{itemize}
\item
functionality in definition file
\end{itemize}


%%%%%%%%%%%%%%%%%%%%%%%%%%%%%%%%%%%%%%%%%%%%%%%%%%%%%%%%%%%%%%%%%%%%%%%%%%%%%%%%
%%%%%%%%%%%%%%%%%%%%%%%%%%%%%%%%%%%%%%%%%%%%%%%%%%%%%%%%%%%%%%%%%%%%%%%%%%%%%%%%
%%%%%%%%%%%%%%%%%%%%%%%%%%%%%%%%%%%%%%%%%%%%%%%%%%%%%%%%%%%%%%%%%%%%%%%%%%%%%%%%
\appendix

\settowidth\MacroIndent{\rmfamily\scriptsize 000\ }

 \DocInput{childdoc.dtx}

\end{document}
%</driver>
% \fi
%
% %%%%%%%%%%%%%%%%%%%%%%%%%%%%%%%%%%%%%%%%%%%%%%%%%%%%%%%%%%%%%%%%%%%%%%%%%%%%%%
% %%%%%%%%%%%%%%%%%%%%%%%%%%%%%%%%%%%%%%%%%%%%%%%%%%%%%%%%%%%%%%%%%%%%%%%%%%%%%%
% \section{Sample}
%\iffalse
%<*samplemain>
%\fi
%
% The following presents a sample document
% with two chapters, two parts, a title page,
% a compile flag as well as three forwarding files to set the flag.
% It consists of eight |.tex| files:
% \begin{center}
% \begin{tabular}{ll}
% |cdocsamp.tex|&main file\\
% |cdocsch1.tex|&include file for chapter 1\\
% |cdocsch2.tex|&include file for chapter 2\\
% |cdocspt3.tex|&include file for part 3\\
% |cdocspt4.tex|&include file for part 4\\
% |cdocsdrf.tex|&forwarding file for main file in draft mode\\
% |cdocsfi1.tex|&forwarding file for final version of chapter 1\\
% |cdocsfi2.tex|&forwarding file for final version of chapter 2\\
% \end{tabular}
% \end{center}
% Each of the eight files can be compiled directly by the \LaTeX{} compiler.
%
% %%%%%%%%%%%%%%%%%%%%%%%%%%%%%%%%%%%%%%
% \paragraph{Main File.}
%
% The main file is called |cdocsamp.tex|.
%
% Load the \textsf{childdoc} definitions and
% declare the filename for the main document:
%    \begin{macrocode}
\input{childdoc.def}
\childdocmain{}
%    \end{macrocode}

% Optional override for |\version| flag:
%    \begin{macrocode}
%%\ifchilddoc\else\providecommand{\version}{draft}\fi
%    \end{macrocode}

% Define the default values for the |\version| flag
% (|final| for the main file and |draft| for childs):
%    \begin{macrocode}
\ifchilddoc
\providecommand{\version}{draft}
\else
\providecommand{\version}{final}
\fi
%    \end{macrocode}

% Load the standard document class:
%    \begin{macrocode}
\documentclass[12pt]{article}
%    \end{macrocode}

% Start the document body:
%    \begin{macrocode}
\begin{document}
%    \end{macrocode}

% Declare a title page.
% Print title, part of document being processed and version flag:
%    \begin{macrocode}
\addtocounter{page}{-1}
\begin{center}
{\LARGE\bfseries{}childdoc example\par}
\vspace{1cm}
\ifchilddoc
\ifchilddocmanual part\else chapter\fi:
`\childdocname' of `\childdocjob'\par
\else
main document: `\childdocjob'\par
\fi
version: \version\par
\end{center}
\newpage
%    \end{macrocode}

% Manually include selected file,
% otherwise process as usual:
%    \begin{macrocode}
\ifchilddocmanual
\section*{part `\childdocname'}
\input{\childdocname}
\else
%    \end{macrocode}

% Include the two chapters:
%    \begin{macrocode}
\include{cdocsch1}
\include{cdocsch2}
%    \end{macrocode}

% Include the two parts unless only chapters should be displayed:
%    \begin{macrocode}
\ifchilddoc\else
\section{part three}
\input{cdocspt3}
\section{part four}
\input{cdocspt4}
\fi
%    \end{macrocode}

% Process as usual until here:
%    \begin{macrocode}
\fi
%    \end{macrocode}

% End of document body:
%    \begin{macrocode}
\end{document}
%    \end{macrocode}
%\iffalse
%</samplemain>
%\fi
%
% %%%%%%%%%%%%%%%%%%%%%%%%%%%%%%%%%%%%%%
% \paragraph{Chapter Include Files.}
%
% The include files are called |cdocsch1.tex| and |cdocsch2.tex|.
%
%\iffalse
%<*samplechap1|samplechap2>
%\fi

% Optional override for |\version| flag:
%    \begin{macrocode}
%%\providecommand{\version}{final}
%    \end{macrocode}

% Include the main document:
%    \begin{macrocode}
\input{childdoc.def}
\childdocof{cdocsamp}
%    \end{macrocode}

%\iffalse
%</samplechap1|samplechap2>
%\fi
%
%\iffalse
%<*samplechap1>
%\fi
% Some text for chapter 1:
%    \begin{macrocode}
\section{one}
some text in chapter one
%    \end{macrocode}

%\iffalse
%</samplechap1>
%\fi
% Some text for chapter 2:
%\iffalse
%<*samplechap2>
%\fi
%    \begin{macrocode}
\section{two}
more text in chapter two
%    \end{macrocode}

%\iffalse
%</samplechap2>
%\fi
%
% %%%%%%%%%%%%%%%%%%%%%%%%%%%%%%%%%%%%%%
% \paragraph{Part Include Files.}
%
% The include files are called |cdocspt3.tex| and |cdocspt4.tex|.
%
%\iffalse
%<*samplepart3|samplepart4>
%\fi

% Optional override for |\version| flag:
%    \begin{macrocode}
%%\providecommand{\version}{final}
%    \end{macrocode}

% Include the main document:
%    \begin{macrocode}
\input{childdoc.def}
\childdocby{cdocsamp}
%    \end{macrocode}

%\iffalse
%</samplepart3|samplepart4>
%\fi
%
%\iffalse
%<*samplepart3>
%\fi
% Some text for part 3:
%    \begin{macrocode}
some text in part three
%    \end{macrocode}

%\iffalse
%</samplepart3>
%\fi
% Some text for part 4:
%\iffalse
%<*samplepart4>
%\fi
%    \begin{macrocode}
more text in part four
%    \end{macrocode}

%\iffalse
%</samplepart4>
%\fi
%
% %%%%%%%%%%%%%%%%%%%%%%%%%%%%%%%%%%%%%%
% \paragraph{Forwarding for a Complete Draft.}
%
% The following forwarding file |cdocsdrf.tex|
% compiles the main document in draft mode:
%\iffalse
%<*sampledraft>
%\fi
%    \begin{macrocode}
\def\version{draft}
\input{childdoc.def}
\childdocforward{cdocsamp}
%    \end{macrocode}

%\iffalse
%</sampledraft>
%\fi
%
% %%%%%%%%%%%%%%%%%%%%%%%%%%%%%%%%%%%%%%
% \paragraph{Forwarding for Final Version of the Chapters.}
%
% The following forwarding files |cdocsfn1.tex| and |cdocsfn2.tex|
% (with identical content)
% compile the final versions of the child documents
% |cdocsch1.tex| and |cdocsch2.tex|, respectively:
%\iffalse
%<*samplefinal>
%\fi
%    \begin{macrocode}
\def\version{final}
\input{childdoc.def}
\childdocforwardprefix[cdocsamp]{cdocsfn}{cdocsch}
%    \end{macrocode}

%\iffalse
%</samplefinal>
%\fi
%
% %%%%%%%%%%%%%%%%%%%%%%%%%%%%%%%%%%%%%%
% \paragraph{Command Line Processing.}
%
% The following three command lines generate the output files
% |cdocscld|, |cdocscl1| and |cdocscl2|
% which should be identical to
% |cdocsdrf|, |cdocsch1| and |cdocsfn2|, respectively:
% \begin{center}
% \begin{tabular}{l}
% |latex -jobname cdocscld \|\\
% |  "\def\version{draft}\input{childdoc.def}\childdocforward{cdocsamp}"|\\
% |latex -jobname cdocscl1 \|\\
% |  "\input{childdoc.def}\childdocforward[cdocsamp]{cdocsch1}"|\\
% |latex -jobname cdocscl2 \|\\
% |  "\def\version{final}\input{childdoc.def}\childdocforward{cdocsch2}"|
% \end{tabular}
% \end{center}
% Note that the trailing backslash on each first line
% merely continues the input to the second line
% (for convenient cut ant paste).
% Furthermore, the command |latex| can be replaced by any
% of its alternative versions such as |pdflatex|.
%
% %%%%%%%%%%%%%%%%%%%%%%%%%%%%%%%%%%%%%%%%%%%%%%%%%%%%%%%%%%%%%%%%%%%%%%%%%%%%%%
% %%%%%%%%%%%%%%%%%%%%%%%%%%%%%%%%%%%%%%%%%%%%%%%%%%%%%%%%%%%%%%%%%%%%%%%%%%%%%%
% \section{Implementation}
%\iffalse
%<*package>
%\fi
%
% This section describes the definitions file |childdoc.def|.

% The definitions cannot be loaded using |\usepackage| or |\RequirePackage|
% which has a mechanism to prevent loading a style file more than once.
% When loading the definitions by means of |\input|
% multiple instances have to be prevented manually:
%\iffalse
%This code needs to be before the `\ProvidesFile' directive
%which is defined at the beginning of this file.
%Therefore it is also placed there and commented out here.
%</package>
%<*discard>
%\fi
%    \begin{macrocode}
\ifdefined\childdocmain\endinput\fi
%    \end{macrocode}
%\iffalse
%</discard>
%<*package>
%\fi
%
% \macro{\ifchilddoc}
% \macro{\ifchilddocmanual}
% The conditional |\ifchilddoc| tells whether a
% child (true) or main (false) document is being compiled.
% The conditional |\ifchilddocmanual| tells whether
% the |\includeonly| mechanism is used (false) or
% the selection of child files must be performed manually (true).
% The definitions initialise to false:
%    \begin{macrocode}
\newif\ifchilddoc
\newif\ifchilddocmanual
%    \end{macrocode}

% \macro{\childdocname}
% \macro{\childdocjob}
% The macro |\childdocname| stores the name of the main document
% to be compiled. The macro |\childdocjob| stores the name of
% the document on which the \LaTeX{} compiler was originally invoked.
% The content of |\jobname| cannot be compared
% to filenames specified in the source due to different catcodes.
% The following code rescans |\jobname|, stores the result
% in |\childdocname| and saves a copy in |\childdocjob|:
%    \begin{macrocode}
\edef\childdocname{\scantokens\expandafter{\jobname\noexpand}}
\let\childdocjob\childdocname
%    \end{macrocode}

% \macro{\childdocdisable}
% The macro |\childdocdisable| prevents the main file
% from being processed more than once.
% At this stage, the main document command |\childdocmain|
% is assumed to be called once again where it should do nothing.
% Any subsequent call to it should prevent
% a secondary processing of the main document
% It overwrites the forwarding commands
% |\childdocof| and |\childdocforward|
% with empty macros to prevent further inclusions of the main document:
%    \begin{macrocode}
\newcommand{\childdocdisable}
{
  \renewcommand{\childdocmain}[1]{\renewcommand{\childdocmain}[1]{\endinput}}
  \renewcommand{\childdocof}[1]{}
  \renewcommand{\childdocby}[2][]{}
  \renewcommand{\childdocforward}[2][]{}
  \renewcommand{\childdocdisable}{}
}
%    \end{macrocode}

% \macro{\childdocmain}
% The macro |\childdocmain| is to be called at the top of the main file
% with nothing or the main filename (without extension) as argument.
% First, it breaks loops.
% If the argument is not empty and does not match |\childdocname|
% (which is set by the first inclusion of |childdoc.def|),
% |\ifchilddoc| is set to true, |\includeonly| is applied to the child file
% and |\jobname| is set to the main file
% (for proper handling of |.aux| files):
%    \begin{macrocode}
\newcommand{\childdocmain}[1]
{
  \childdocdisable\childdocmain{}
  \if?#1?\else
    \begingroup
      \def\childdoctmp{#1}
      \ifx\childdoctmp\childdocname
        \def\childdoctmp{}
      \else
        \def\childdoctmp
        {
          \childdoctrue
          \includeonly{\childdocname}
          \def\childdocjob{#1}
          \def\jobname{#1}
        }
      \fi
      \expandafter
    \endgroup
    \childdoctmp
  \fi
}
%    \end{macrocode}

% \macro{\childdocof}
% The command |\childdocof| redirects
% compilation to the main file |#1|.
%    \begin{macrocode}
\newcommand{\childdocof}[1]
{
  \childdocdisable
  \childdoctrue
  \includeonly{\childdocname}
  \def\jobname{#1}
  \def\childdocjob{#1}
  \input{#1}
}
%    \end{macrocode}

% \macro{\childdocby}
% The command |\childdocby| ....
%    \begin{macrocode}
\newcommand{\childdocby}[2][]
{
  \childdocdisable
  \childdoctrue
  \childdocmanualtrue
  \if?#1?\else
    \def\jobname{#2}
  \fi
  \def\childdocjob{#2}
  \input{#2}
  \endinput
}
%    \end{macrocode}

% \macro{\childdocforward}
% The command |\childdocforward| redirects
% compilation to the main file or
% (if the optional argument is given) a child file.
% Parameters are set as if the main file
% or a child file starting with |\childdocof| was compiled.
% Then compilation is handed over to the main file:
%    \begin{macrocode}
\newcommand{\childdocforward}[2][]
{
  \begingroup
    \if?#1?
      \def\childdoctmp
      {
        \def\childdocname{#2}
        \def\childdocjob{#2}
        \def\jobname{#2}
        \input{#2}
        \endinput
      }
    \else
      \def\childdoctmp
      {
        \childdocdisable
        \def\childdocname{#2}
        \childdoctrue
        \includeonly{#2}
        \def\childdocjob{#1}
        \def\jobname{#1}
        \input{#1}
        \endinput
      }
    \fi
    \expandafter
  \endgroup
  \childdoctmp
}
%    \end{macrocode}

% \macro{\childdocforwardprefix}
% The command |\childdocforwardprefix| redirects
% compilation to the main or a child file by means of a pattern.
% The prefix |#1| in the current filename is replaced by |#2|
% and the suffix of the current filename is kept
% (it is assumed that the filename does not contain the substring `|~~~|'
% which is used as a delimiter).
% Compilation is handed over to the new file by |\childdocforward|:
%    \begin{macrocode}
\newcommand{\childdocforwardprefix}[3][]
{
  \begingroup
    \def\childdocextract #2##1~~~{\def\childdoctmp{\childdocforward[#1]{#3##1}}}
    \expandafter\childdocextract\childdocname~~~
    \expandafter
  \endgroup
  \childdoctmp
}
%    \end{macrocode}

% \macro{\childdoc}
% The deprecated macro |\childdoc| is a legacy version of |\childdocmain|:
%    \begin{macrocode}
\newcommand{\childdoc}{\childdocmain}
%    \end{macrocode}

% \macro{\childdocredirect}
% The deprecated macro |\childdocredirect| is a legacy version
% of |\childdocforward| and |\childdocforwardprefix|:
%    \begin{macrocode}
\newcommand{\childdocredirect}[2][]
{
  \begingroup
    \if?#1?
      \def\childdoctmp{\childdocforward{#2}}
    \else
      \def\childdoctmp{\childdocforwardprefix{#1}{#2}}
    \fi
    \expandafter
  \endgroup
  \childdoctmp
}
%    \end{macrocode}

%\iffalse
%</package>
%\fi
%
\endinput
\childdocforward{cdocsch2}"|
% \end{tabular}
% \end{center}
% Note that the trailing backslash on each first line
% merely continues the input to the second line
% (for convenient cut ant paste).
% Furthermore, the command |latex| can be replaced by any
% of its alternative versions such as |pdflatex|.
%
% %%%%%%%%%%%%%%%%%%%%%%%%%%%%%%%%%%%%%%%%%%%%%%%%%%%%%%%%%%%%%%%%%%%%%%%%%%%%%%
% %%%%%%%%%%%%%%%%%%%%%%%%%%%%%%%%%%%%%%%%%%%%%%%%%%%%%%%%%%%%%%%%%%%%%%%%%%%%%%
% \section{Implementation}
%\iffalse
%<*package>
%\fi
%
% This section describes the definitions file |childdoc.def|.

% The definitions cannot be loaded using |\usepackage| or |\RequirePackage|
% which has a mechanism to prevent loading a style file more than once.
% When loading the definitions by means of |\input|
% multiple instances have to be prevented manually:
%\iffalse
%This code needs to be before the `\ProvidesFile' directive
%which is defined at the beginning of this file.
%Therefore it is also placed there and commented out here.
%</package>
%<*discard>
%\fi
%    \begin{macrocode}
\ifdefined\childdocmain\endinput\fi
%    \end{macrocode}
%\iffalse
%</discard>
%<*package>
%\fi
%
% \macro{\ifchilddoc}
% \macro{\ifchilddocmanual}
% The conditional |\ifchilddoc| tells whether a
% child (true) or main (false) document is being compiled.
% The conditional |\ifchilddocmanual| tells whether
% the |\includeonly| mechanism is used (false) or
% the selection of child files must be performed manually (true).
% The definitions initialise to false:
%    \begin{macrocode}
\newif\ifchilddoc
\newif\ifchilddocmanual
%    \end{macrocode}

% \macro{\childdocname}
% \macro{\childdocjob}
% The macro |\childdocname| stores the name of the main document
% to be compiled. The macro |\childdocjob| stores the name of
% the document on which the \LaTeX{} compiler was originally invoked.
% The content of |\jobname| cannot be compared
% to filenames specified in the source due to different catcodes.
% The following code rescans |\jobname|, stores the result
% in |\childdocname| and saves a copy in |\childdocjob|:
%    \begin{macrocode}
\edef\childdocname{\scantokens\expandafter{\jobname\noexpand}}
\let\childdocjob\childdocname
%    \end{macrocode}

% \macro{\childdocdisable}
% The macro |\childdocdisable| prevents the main file
% from being processed more than once.
% At this stage, the main document command |\childdocmain|
% is assumed to be called once again where it should do nothing.
% Any subsequent call to it should prevent
% a secondary processing of the main document
% It overwrites the forwarding commands
% |\childdocof| and |\childdocforward|
% with empty macros to prevent further inclusions of the main document:
%    \begin{macrocode}
\newcommand{\childdocdisable}
{
  \renewcommand{\childdocmain}[1]{\renewcommand{\childdocmain}[1]{\endinput}}
  \renewcommand{\childdocof}[1]{}
  \renewcommand{\childdocby}[2][]{}
  \renewcommand{\childdocforward}[2][]{}
  \renewcommand{\childdocdisable}{}
}
%    \end{macrocode}

% \macro{\childdocmain}
% The macro |\childdocmain| is to be called at the top of the main file
% with nothing or the main filename (without extension) as argument.
% First, it breaks loops.
% If the argument is not empty and does not match |\childdocname|
% (which is set by the first inclusion of |childdoc.def|),
% |\ifchilddoc| is set to true, |\includeonly| is applied to the child file
% and |\jobname| is set to the main file
% (for proper handling of |.aux| files):
%    \begin{macrocode}
\newcommand{\childdocmain}[1]
{
  \childdocdisable\childdocmain{}
  \if?#1?\else
    \begingroup
      \def\childdoctmp{#1}
      \ifx\childdoctmp\childdocname
        \def\childdoctmp{}
      \else
        \def\childdoctmp
        {
          \childdoctrue
          \includeonly{\childdocname}
          \def\childdocjob{#1}
          \def\jobname{#1}
        }
      \fi
      \expandafter
    \endgroup
    \childdoctmp
  \fi
}
%    \end{macrocode}

% \macro{\childdocof}
% The command |\childdocof| redirects
% compilation to the main file |#1|.
%    \begin{macrocode}
\newcommand{\childdocof}[1]
{
  \childdocdisable
  \childdoctrue
  \includeonly{\childdocname}
  \def\jobname{#1}
  \def\childdocjob{#1}
  \input{#1}
}
%    \end{macrocode}

% \macro{\childdocby}
% The command |\childdocby| ....
%    \begin{macrocode}
\newcommand{\childdocby}[2][]
{
  \childdocdisable
  \childdoctrue
  \childdocmanualtrue
  \if?#1?\else
    \def\jobname{#2}
  \fi
  \def\childdocjob{#2}
  \input{#2}
  \endinput
}
%    \end{macrocode}

% \macro{\childdocforward}
% The command |\childdocforward| redirects
% compilation to the main file or
% (if the optional argument is given) a child file.
% Parameters are set as if the main file
% or a child file starting with |\childdocof| was compiled.
% Then compilation is handed over to the main file:
%    \begin{macrocode}
\newcommand{\childdocforward}[2][]
{
  \begingroup
    \if?#1?
      \def\childdoctmp
      {
        \def\childdocname{#2}
        \def\childdocjob{#2}
        \def\jobname{#2}
        \input{#2}
        \endinput
      }
    \else
      \def\childdoctmp
      {
        \childdocdisable
        \def\childdocname{#2}
        \childdoctrue
        \includeonly{#2}
        \def\childdocjob{#1}
        \def\jobname{#1}
        \input{#1}
        \endinput
      }
    \fi
    \expandafter
  \endgroup
  \childdoctmp
}
%    \end{macrocode}

% \macro{\childdocforwardprefix}
% The command |\childdocforwardprefix| redirects
% compilation to the main or a child file by means of a pattern.
% The prefix |#1| in the current filename is replaced by |#2|
% and the suffix of the current filename is kept
% (it is assumed that the filename does not contain the substring `|~~~|'
% which is used as a delimiter).
% Compilation is handed over to the new file by |\childdocforward|:
%    \begin{macrocode}
\newcommand{\childdocforwardprefix}[3][]
{
  \begingroup
    \def\childdocextract #2##1~~~{\def\childdoctmp{\childdocforward[#1]{#3##1}}}
    \expandafter\childdocextract\childdocname~~~
    \expandafter
  \endgroup
  \childdoctmp
}
%    \end{macrocode}

% \macro{\childdoc}
% The deprecated macro |\childdoc| is a legacy version of |\childdocmain|:
%    \begin{macrocode}
\newcommand{\childdoc}{\childdocmain}
%    \end{macrocode}

% \macro{\childdocredirect}
% The deprecated macro |\childdocredirect| is a legacy version
% of |\childdocforward| and |\childdocforwardprefix|:
%    \begin{macrocode}
\newcommand{\childdocredirect}[2][]
{
  \begingroup
    \if?#1?
      \def\childdoctmp{\childdocforward{#2}}
    \else
      \def\childdoctmp{\childdocforwardprefix{#1}{#2}}
    \fi
    \expandafter
  \endgroup
  \childdoctmp
}
%    \end{macrocode}

%\iffalse
%</package>
%\fi
%
\endinput
\childdocforward[cdocsamp]{cdocsch1}"|\\
% |latex -jobname cdocscl2 \|\\
% |  "\def\version{final}% \iffalse
%
% childdoc.dtx Copyright (C) 2017-2018 Niklas Beisert
%
% This work may be distributed and/or modified under the
% conditions of the LaTeX Project Public License, either version 1.3
% of this license or (at your option) any later version.
% The latest version of this license is in
%   http://www.latex-project.org/lppl.txt
% and version 1.3 or later is part of all distributions of LaTeX
% version 2005/12/01 or later.
%
% This work has the LPPL maintenance status `maintained'.
%
% The Current Maintainer of this work is Niklas Beisert.
%
% This work consists of the files childdoc.dtx and childdoc.ins
% and the derived files childdoc.def and cdocsamp.tex with
% cdocsch1.tex, cdocsch2.tex, cdocsdrf.tex, cdocsfn1.tex, cdocsfn2.tex.
%
%<package>\ifdefined\childdocmain\endinput\fi
%<package>\ProvidesFile{childdoc.def}[2018/12/30 v2.0 child document driver]
%<samplemain>\ProvidesFile{cdocsamp.tex}[2018/12/30 v2.0 sample for childdoc]
%<*driver>
%\ProvidesFile{childdoc.drv}[2018/12/30 v2.0 childdoc reference manual file]
\PassOptionsToClass{10pt,a4paper}{article}
\documentclass{ltxdoc}

\usepackage[margin=35mm]{geometry}
\usepackage{hyperref}
\usepackage{hyperxmp}
\usepackage[usenames]{color}

\hypersetup{colorlinks=true}
\hypersetup{pdfstartview=FitH}
\hypersetup{pdfpagemode=UseNone}
\hypersetup{pdfsource={}}
\hypersetup{pdflang={en-UK}}
\hypersetup{pdfcopyright={Copyright 2017-2018 Niklas Beisert.
  This work may be distributed and/or modified under the
  conditions of the LaTeX Project Public License, either version 1.3
  of this license or (at your option) any later version.}}
\hypersetup{pdflicenseurl={http://www.latex-project.org/lppl.txt}}
\hypersetup{pdfcontactaddress={ETH Zurich, ITP, HIT K,
  Wolfgang-Pauli-Strasse 27}}
\hypersetup{pdfcontactpostcode={8093}}
\hypersetup{pdfcontactcity={Zurich}}
\hypersetup{pdfcontactcountry={Switzerland}}
\hypersetup{pdfcontactemail={nbeisert@itp.phys.ethz.ch}}
\hypersetup{pdfcontacturl={http://people.phys.ethz.ch/\xmptilde nbeisert/}}

\newcommand{\secref}[1]{\hyperref[#1]{section \ref*{#1}}}

\parskip1ex
\parindent0pt
\let\olditemize\itemize
\def\itemize{\olditemize\parskip0pt}

\begin{document}

\title{The \textsf{childdoc} Package}
\hypersetup{pdftitle={The childdoc Package}}
\author{Niklas Beisert\\[2ex]
  Institut f\"ur Theoretische Physik\\
  Eidgen\"ossische Technische Hochschule Z\"urich\\
  Wolfgang-Pauli-Strasse 27, 8093 Z\"urich, Switzerland\\[1ex]
  \href{mailto:nbeisert@itp.phys.ethz.ch}
  {\texttt{nbeisert@itp.phys.ethz.ch}}}
\hypersetup{pdfauthor={Niklas Beisert}}
\hypersetup{pdfsubject={Manual for the LaTeX2e Package childdoc}}
\date{30 December 2018, \textsf{v2.0}}
\maketitle

\begin{abstract}\noindent
\textsf{childdoc} is a \LaTeXe{} package
that enables the direct compilation
of document sections included by |\include|
to individual files.
\end{abstract}

\begingroup
\parskip0ex
\tableofcontents
\endgroup

%%%%%%%%%%%%%%%%%%%%%%%%%%%%%%%%%%%%%%%%%%%%%%%%%%%%%%%%%%%%%%%%%%%%%%%%%%%%%%%%
%%%%%%%%%%%%%%%%%%%%%%%%%%%%%%%%%%%%%%%%%%%%%%%%%%%%%%%%%%%%%%%%%%%%%%%%%%%%%%%%
\section{Introduction}

\LaTeX{} provides a mechanism to structure a large document (such as a book)
into a main file and several child files (containing the chapters)
using the |\include| command.
This mechanism is beneficial for documents
which span hundreds of pages in order to
make the source file(s) more manageable.
Moreover, compilation can be restricted to
selected child files by means of the |\includeonly| command.
The latter feature can be used to reduce the compilation time while editing
(this was significantly more useful in the earlier days of \LaTeX{})
or to generate a smaller document which is easier to navigate.
Another application of |\includeonly| is to generate
documents consisting of selected parts of the complete document.

However, there are a few drawbacks of the plain |\include| mechanism:
\begin{itemize}
\item
The child files cannot be compiled on their own,
they can only be compiled via the main file.
A naive editing environment
(such as a text editor with an option
to have the current file processed by \LaTeX)
may require one to switch to the main file before compiling;
attempting to compile the child file produces errors.
\item
The main file must be modified (each time)
to adjust the |\includeonly| command
to the present needs. This easily leaves the main file in a messy state.
\item
The generated document will always carry the filename
of the main document. This is inconvenient if
several child files are to be compiled and
to be kept for distribution.
\end{itemize}

The present package provides a simple interface
to make child files individually compilable by \LaTeX{}.
Compiling a child file then has the same effect as compiling
the main file with an |\includeonly| command
to select the appropriate child.
Moreover the generated document will carry the name of the child
rather than the main file.
This resolves all three above issues.

This feature is meant to make the editing of books,
thesis documents and lecture notes somewhat more convenient.
However, the package can also be used efficiently for
composing a series of documents (such as exercise sheets)
which are typically distributed individually.
It then assists the author in generating the individual documents
(potentially in different versions)
as well as a document containing the collected series.
Another application is in developing style files
or other kinds of included material
where compilation of the style file could redirect
to a sample or test file.

%%%%%%%%%%%%%%%%%%%%%%%%%%%%%%%%%%%%%%%%%%%%%%%%%%%%%%%%%%%%%%%%%%%%%%%%%%%%%%%%
%%%%%%%%%%%%%%%%%%%%%%%%%%%%%%%%%%%%%%%%%%%%%%%%%%%%%%%%%%%%%%%%%%%%%%%%%%%%%%%%
\section{Usage}

First of all, the package \textsf{childdoc} is \emph{not} a standard
\LaTeXe{} |.sty| style file! Therefore it needs to be invoked in
a non-standard way.

%%%%%%%%%%%%%%%%%%%%%%%%%%%%%%%%%%%%%%%%%%%%%%%%%%%%%%%%%%%%%%%%%%%%%%%%%%%%%%%%
\subsection{Included Files}
\label{sec:include}

%%%%%%%%%%%%%%%%%%%%%%%%%%%%%%%%%%%%%%%%
\DescribeMacro{\childdocmain}
To use the package, add the commands
\begin{center}
\begin{tabular}{l}
|% \iffalse
%
% childdoc.dtx Copyright (C) 2017-2018 Niklas Beisert
%
% This work may be distributed and/or modified under the
% conditions of the LaTeX Project Public License, either version 1.3
% of this license or (at your option) any later version.
% The latest version of this license is in
%   http://www.latex-project.org/lppl.txt
% and version 1.3 or later is part of all distributions of LaTeX
% version 2005/12/01 or later.
%
% This work has the LPPL maintenance status `maintained'.
%
% The Current Maintainer of this work is Niklas Beisert.
%
% This work consists of the files childdoc.dtx and childdoc.ins
% and the derived files childdoc.def and cdocsamp.tex with
% cdocsch1.tex, cdocsch2.tex, cdocsdrf.tex, cdocsfn1.tex, cdocsfn2.tex.
%
%<package>\ifdefined\childdocmain\endinput\fi
%<package>\ProvidesFile{childdoc.def}[2018/12/30 v2.0 child document driver]
%<samplemain>\ProvidesFile{cdocsamp.tex}[2018/12/30 v2.0 sample for childdoc]
%<*driver>
%\ProvidesFile{childdoc.drv}[2018/12/30 v2.0 childdoc reference manual file]
\PassOptionsToClass{10pt,a4paper}{article}
\documentclass{ltxdoc}

\usepackage[margin=35mm]{geometry}
\usepackage{hyperref}
\usepackage{hyperxmp}
\usepackage[usenames]{color}

\hypersetup{colorlinks=true}
\hypersetup{pdfstartview=FitH}
\hypersetup{pdfpagemode=UseNone}
\hypersetup{pdfsource={}}
\hypersetup{pdflang={en-UK}}
\hypersetup{pdfcopyright={Copyright 2017-2018 Niklas Beisert.
  This work may be distributed and/or modified under the
  conditions of the LaTeX Project Public License, either version 1.3
  of this license or (at your option) any later version.}}
\hypersetup{pdflicenseurl={http://www.latex-project.org/lppl.txt}}
\hypersetup{pdfcontactaddress={ETH Zurich, ITP, HIT K,
  Wolfgang-Pauli-Strasse 27}}
\hypersetup{pdfcontactpostcode={8093}}
\hypersetup{pdfcontactcity={Zurich}}
\hypersetup{pdfcontactcountry={Switzerland}}
\hypersetup{pdfcontactemail={nbeisert@itp.phys.ethz.ch}}
\hypersetup{pdfcontacturl={http://people.phys.ethz.ch/\xmptilde nbeisert/}}

\newcommand{\secref}[1]{\hyperref[#1]{section \ref*{#1}}}

\parskip1ex
\parindent0pt
\let\olditemize\itemize
\def\itemize{\olditemize\parskip0pt}

\begin{document}

\title{The \textsf{childdoc} Package}
\hypersetup{pdftitle={The childdoc Package}}
\author{Niklas Beisert\\[2ex]
  Institut f\"ur Theoretische Physik\\
  Eidgen\"ossische Technische Hochschule Z\"urich\\
  Wolfgang-Pauli-Strasse 27, 8093 Z\"urich, Switzerland\\[1ex]
  \href{mailto:nbeisert@itp.phys.ethz.ch}
  {\texttt{nbeisert@itp.phys.ethz.ch}}}
\hypersetup{pdfauthor={Niklas Beisert}}
\hypersetup{pdfsubject={Manual for the LaTeX2e Package childdoc}}
\date{30 December 2018, \textsf{v2.0}}
\maketitle

\begin{abstract}\noindent
\textsf{childdoc} is a \LaTeXe{} package
that enables the direct compilation
of document sections included by |\include|
to individual files.
\end{abstract}

\begingroup
\parskip0ex
\tableofcontents
\endgroup

%%%%%%%%%%%%%%%%%%%%%%%%%%%%%%%%%%%%%%%%%%%%%%%%%%%%%%%%%%%%%%%%%%%%%%%%%%%%%%%%
%%%%%%%%%%%%%%%%%%%%%%%%%%%%%%%%%%%%%%%%%%%%%%%%%%%%%%%%%%%%%%%%%%%%%%%%%%%%%%%%
\section{Introduction}

\LaTeX{} provides a mechanism to structure a large document (such as a book)
into a main file and several child files (containing the chapters)
using the |\include| command.
This mechanism is beneficial for documents
which span hundreds of pages in order to
make the source file(s) more manageable.
Moreover, compilation can be restricted to
selected child files by means of the |\includeonly| command.
The latter feature can be used to reduce the compilation time while editing
(this was significantly more useful in the earlier days of \LaTeX{})
or to generate a smaller document which is easier to navigate.
Another application of |\includeonly| is to generate
documents consisting of selected parts of the complete document.

However, there are a few drawbacks of the plain |\include| mechanism:
\begin{itemize}
\item
The child files cannot be compiled on their own,
they can only be compiled via the main file.
A naive editing environment
(such as a text editor with an option
to have the current file processed by \LaTeX)
may require one to switch to the main file before compiling;
attempting to compile the child file produces errors.
\item
The main file must be modified (each time)
to adjust the |\includeonly| command
to the present needs. This easily leaves the main file in a messy state.
\item
The generated document will always carry the filename
of the main document. This is inconvenient if
several child files are to be compiled and
to be kept for distribution.
\end{itemize}

The present package provides a simple interface
to make child files individually compilable by \LaTeX{}.
Compiling a child file then has the same effect as compiling
the main file with an |\includeonly| command
to select the appropriate child.
Moreover the generated document will carry the name of the child
rather than the main file.
This resolves all three above issues.

This feature is meant to make the editing of books,
thesis documents and lecture notes somewhat more convenient.
However, the package can also be used efficiently for
composing a series of documents (such as exercise sheets)
which are typically distributed individually.
It then assists the author in generating the individual documents
(potentially in different versions)
as well as a document containing the collected series.
Another application is in developing style files
or other kinds of included material
where compilation of the style file could redirect
to a sample or test file.

%%%%%%%%%%%%%%%%%%%%%%%%%%%%%%%%%%%%%%%%%%%%%%%%%%%%%%%%%%%%%%%%%%%%%%%%%%%%%%%%
%%%%%%%%%%%%%%%%%%%%%%%%%%%%%%%%%%%%%%%%%%%%%%%%%%%%%%%%%%%%%%%%%%%%%%%%%%%%%%%%
\section{Usage}

First of all, the package \textsf{childdoc} is \emph{not} a standard
\LaTeXe{} |.sty| style file! Therefore it needs to be invoked in
a non-standard way.

%%%%%%%%%%%%%%%%%%%%%%%%%%%%%%%%%%%%%%%%%%%%%%%%%%%%%%%%%%%%%%%%%%%%%%%%%%%%%%%%
\subsection{Included Files}
\label{sec:include}

%%%%%%%%%%%%%%%%%%%%%%%%%%%%%%%%%%%%%%%%
\DescribeMacro{\childdocmain}
To use the package, add the commands
\begin{center}
\begin{tabular}{l}
|\input{childdoc.def}|\\
|\childdocmain{}|\\
\end{tabular}
\end{center}
at the very top of the main \LaTeX{} file,
in particular \emph{before} the |\documentclass| statement!
The argument of |\childdocmain| should be left empty
(but it must be present).

%%%%%%%%%%%%%%%%%%%%%%%%%%%%%%%%%%%%%%%%
\DescribeMacro{\childdocof}
Furthermore, add the commands
\begin{center}
\begin{tabular}{l}
|\input{childdoc.def}|\\
|\childdocof{|\textit{main}|}|\\
\end{tabular}
\end{center}
at the top of every child file \textit{child}
which is included by |\include{|\textit{child}|}|
from within the main file
(or at least for those files to be compiled individually).
The argument \textit{main} must be the filename of the main file.

There are a couple of
considerations in setting up the main and child documents:

%%%%%%%%%%%%%%%%%%%%%%%%%%%%%%%%%%%%%%%%
\paragraph{Restrictions.}

Please note the following restrictions:
\begin{itemize}
\item
|\childdocmain| must be called with one argument \textit{main}
to ensure compatibility with earlier version of the package.
It must either be empty (|\childdocmain{}|)
or precisely match the filename of the main file in which it is specified.
See \secref{sec:detection} for further information.
\item
The filename \textit{main} must be specified without the |.tex| extension.
\item
The filename \textit{main} is case sensitive
(even in case-insensitive file systems)
due to internal string comparison.
\item
The argument \textit{main} should be fully expanded, it cannot be a macro.
\item
Subdirectories and special characters should be avoided in filenames.
\item
The command |\childdocmain{|\textit{main}|}| must be followed by a whitespace.
It should not be followed immediately by another command
or by a comment mark `|%|'.
This is because the \TeX{} parser reads the token immediately following
the argument of |\childdocmain| and puts it
at the beginning of every child section;
however, a white\-space is ignored.
\end{itemize}

%%%%%%%%%%%%%%%%%%%%%%%%%%%%%%%%%%%%%%%%
\paragraph{Content of Main File.}

It is advisable to place all content in the child files included by |\include|.
Any output contained in the main file will appear in all child documents
unless suppressed manually;
it cannot be suppressed automatically by the |\includeonly| directive
and thus should normally be avoided.
A method to include some content in the main file
by means of conditional processing is described in \secref{sec:conditional}.

%%%%%%%%%%%%%%%%%%%%%%%%%%%%%%%%%%%%%%%%
\paragraph{Page Numbering.}

When only a part of the document is compiled,
the appropriate numbering of pages
(as well as other status parameters)
is determined from the |.aux| files.
The latter contain information from previous passes.
However this information needs to propagate through
all intermediate child documents.
Therefore the page numbering in child documents may well
be inconsistent until the complete document is compiled at least once.

A useful (if unconventional) way to always ensure a consistent
page numbering is to restart the numbering in each child document
and denote the pages by `\textit{child}|.|\textit{page}'
where \textit{child} represents the chapter/section number of the child file.
This can be achieved by the command
|\numberwithin{page}{|\textit{child}|}|
of the \textsf{amsmath} package
where \textit{child} can be |chapter| or |section|
depending on the chosen structuring.
Alternatively, one can modify the macro |\thepage| appropriately
and reset the counter |page| at the start of each child file.

%%%%%%%%%%%%%%%%%%%%%%%%%%%%%%%%%%%%%%%%%%%%%%%%%%%%%%%%%%%%%%%%%%%%%%%%%%%%%%%%
\subsection{Conditional Processing}
\label{sec:conditional}

The package provides a mechanism to compile different versions
of a document. To customise the versions further some conditional processing
can come in handy to distinguish which version is being compiled.
The package provides two macros to describe the compilation context:

%%%%%%%%%%%%%%%%%%%%%%%%%%%%%%%%%%%%%%%%
\DescribeMacro{\ifchilddoc}
The conditional |\ifchilddoc| distinguishes between the compilation of
child documents and the main document:
%
\begin{center}
|\ifchilddoc |\textit{child-code}| |[|\||else |\textit{main-code}]| \||fi|
\end{center}

%%%%%%%%%%%%%%%%%%%%%%%%%%%%%%%%%%%%%%%%
\DescribeMacro{\childdocname}
\DescribeMacro{\childdocjob}
The macro |\childdocname| contains the filename (without extension)
of the main or child file being processed.
Note that |\childdocjob| will always contain the name of the main file.

%%%%%%%%%%%%%%%%%%%%%%%%%%%%%%%%%%%%%%%%
\paragraph{Title Page.}

Conditional processing can be used to include a title or banner page
in the main document when proper precautions are taken.
Importantly, the code in the main file should ensure that the page counter
(as well as other status parameters which are stored in the |.aux| files)
takes the same value after the conditional processing.
Otherwise the page numbers may take divergent values
depending on which part is compiled.

For example, a title page could be declared by:
%
\begin{center}
\begin{tabular}{l}
|\ifchilddoc\||else|\\
|\addtocounter{page}{-1}|\\
\textit{code for title page}\\
|\newpage|\\
|\||fi|
\end{tabular}
\end{center}
%
A banner page for the child documents can be generated by:
%
\begin{center}
\begin{tabular}{l}
|\ifchilddoc|\\
|\addtocounter{page}{-1}|\\
\textit{code for banner page}\\
|\newpage|\\
|\||fi|
\end{tabular}
\end{center}
%
Here one could write a message such as:
\begin{center}
|This is the part \childdocname{} of \childdocjob{}.|
\end{center}

%%%%%%%%%%%%%%%%%%%%%%%%%%%%%%%%%%%%%%%%%%%%%%%%%%%%%%%%%%%%%%%%%%%%%%%%%%%%%%%%
\subsection{Flags}
\label{sec:flags}

The package makes it easy to generate different versions
of the main or child documents.
To this end compilation flags can be defined
and assigned different default values.
They will be particularly useful in conjunction
with the forwarding mechanism described in \secref{sec:forward}.

For example, it may be useful to have a flag |\version|
which can be set to |draft| or |final|.
The document source will contain some conditional code
depending on the value of |\version|.
Suppose further, the flag should default to |final| for the main file
and to |draft| for child files
which is a natural assignment for editing the document.
This is achieved by placing the following code
in the preamble of the main document
(below the |\childdocmain| directive):
%
\begin{center}
\begin{tabular}{l}
|\ifchilddoc|\\
|\providecommand{\version}{draft}|\\
|\||else|\\
|\providecommand{\version}{final}|\\
|\||fi|
\end{tabular}
\end{center}
%
The definition by |\providecommand| makes sure
that previous definitions are not overwritten.
Further statements |\providecommand{\version}{...}|
can thus be added before the above code to override it.

For the main file, one might add a line
(between |\childdocmain| and the above block)
%
\begin{center}
|%\ifchilddoc\||else\providecommand{\version}{draft}\||fi|
\end{center}
%
which can be uncommented to produce a draft version.
Likewise one can add a line to the very top of a child file
(above the |\childdocof{|\textit{main}|}| directive)
%
\begin{center}
|%\providecommand{\version}{final}|
\end{center}
%
which can be uncommented to produce the final version of this child document.

%%%%%%%%%%%%%%%%%%%%%%%%%%%%%%%%%%%%%%%%%%%%%%%%%%%%%%%%%%%%%%%%%%%%%%%%%%%%%%%%
\subsection{Forwarding}
\label{sec:forward}

Different versions of the main or child documents
using compilation flags as described in \secref{sec:flags}
can be (permanently) stored in different files
for convenient compilation, viewing and distribution.
To this end, the package defines a command
to pass on compilation to a different file:

%%%%%%%%%%%%%%%%%%%%%%%%%%%%%%%%%%%%%%%%
\DescribeMacro{\childdocforward}
The command |\childdocforward| redirects processing to
another source file:
%
\begin{center}
\begin{tabular}{l}
|\input{childdoc.def}|\\
|\childdocforward[|\textit{main}|]{|\textit{dest}|}|\\
\end{tabular}
\end{center}
%
The argument \textit{dest} is the destination file
(without extension).
It should be the main file or one of the child files.
Note that further \textsf{childdoc} directives
such as |\childdocof| and |\childdocforward|
in the indicated file will be processed in this form.
The optional argument \textit{main}
passes on directly to the main file \textit{main}
while pretending to compile the child \textit{dest}.
This form behaves as if \textit{dest}
issues |\childdocof{|\textit{main}|}| right away,
and no further \textsf{childdoc} directives will be processed.

%%%%%%%%%%%%%%%%%%%%%%%%%%%%%%%%%%%%%%%%
\DescribeMacro{\...prefix}
In the alternative form |\childdocforwardprefix|,
%
\begin{center}
\begin{tabular}{l}
|\input{childdoc.def}|\\
|\childdocforwardprefix[|\textit{main}|]{|\textit{prefix}|}{|\textit{dest}|}|
\end{tabular}
\end{center}
%
the destination file is determined by a pattern
depending on the current file:
To make this work, the current file must be called
`{\textit{prefix}\hspace{0.2em}\textit{suffix}}'
with \textit{prefix} matching precisely the argument.
Processing is then passed on to the file
`{\textit{dest}\hspace{0.2em}\textit{suffix}}'.
Surely, the same effect is achieved by
directly specifying the
argument `{\textit{dest}\hspace{0.2em}\textit{suffix}}'
in the first form.
However, that requires to set up a different file
for each child. With the alternative form of the command
all these files can have exactly the same content
which simplifies setting them up and maintaining them.

For example, the following file |draft.tex|
with a compilation flag |\version| as described in \secref{sec:flags}
compiles the main document as a draft:
%
\begin{center}
\begin{tabular}{l}
|\def\version{draft}|\\
|\input{childdoc.def}|\\
|\childdocforward{|\textit{main}|}|
\end{tabular}
\end{center}
%
Likewise, the following files |final|\textit{nn}|.tex|
compile the final version of the child document
|child|\textit{nn}|.tex|:
%
\begin{center}
\begin{tabular}{l}
|\def\version{final}|\\
|\input{childdoc.def}|\\
|\childdocforwardprefix{final}{child}|
\end{tabular}
\end{center}
%

Note that when several versions of a main file and/or of each child file
are to be generated, it may be convenient to set up a |Makefile| or
shell script to automatise the process.

%%%%%%%%%%%%%%%%%%%%%%%%%%%%%%%%%%%%%%%%%%%%%%%%%%%%%%%%%%%%%%%%%%%%%%%%%%%%%%%%
\subsection{Command Line Processing}
\label{sec:commandline}

The effect of redirection files can also be achieved by invoking
the \LaTeX{} compiler with a more elaborate command line.
Most conveniently this should be done as part
of a shell script or a |Makefile|.

When using \textsf{childdoc} in the main file, the following
command lines effectively perform a redirection
(note that depending on the shell being used,
backslashes may have to be doubled: `|\|' $\to$ `|\\|'):
%
\begin{center}
|... -jobname "|\textit{target}|" |\\|"|[\textit{flags}]%
|\input{childdoc.def}\childdocforward[|\textit{main}|]{|\textit{dest}|}"|
\end{center}
%
Here \textit{target} is the name of the output file,
\textit{main} is the name of the main file
and \textit{dest} is the name of the main or child file to be processed
(all filenames without extensions).
The optional argument \textit{main} can be omitted
if \textit{main} matches \textit{dest}.
Optionally, compilation \textit{flags} can be defined via |\def| commands.
This command line makes the \TeX{} engine believe
it is compiling the file \textit{target}
whose content is specified as the latter parameter.
The provided code then forwards the processing to
\textit{main} or \textit{dest} as described in \secref{sec:forward}.

%%%%%%%%%%%%%%%%%%%%%%%%%%%%%%%%%%%%%%%%%%%%%%%%%%%%%%%%%%%%%%%%%%%%%%%%%%%%%%%%
\subsection{Include by Input}
\label{sec:input}

Including child documents by |\include| has some restrictions by design.
Most notably, the content of a child document always occupies
its own set of pages; pages cannot be shared between child documents.
Usually, this behaviour makes perfect sense
because each child document contain an essential part of the document.
However, in some situations it may be desirable to compose
a document from a collection of parts
without having mandatory page breaks between then.
For this case, the package
provides a mechanism to include parts
by |\input| which can also be processed individually.
However, by construction this mechanism
requires manual handling of the content to be output.

%%%%%%%%%%%%%%%%%%%%%%%%%%%%%%%%%%%%%%%%
\DescribeMacro{\ifchilddocmanual}
The main file should be prepared as usual, see \secref{sec:include}.
However, the document body must make a distinction
between processing of an individual part and of the main document, e.g.:
%
\begin{center}
\begin{tabular}{l}
|\ifchilddocmanual|\\
|\input{\childdocname}|\\
|\||else|\\
\textit{document body with }|\input{|\textit{part}|}|\\
|\||fi|
\end{tabular}
\end{center}
%
The conditional |\ifchilddocmanual| is true whenever
a part to be included by |\input| is being compiled,
and the name of the part is stored in |\childdocname|.

%%%%%%%%%%%%%%%%%%%%%%%%%%%%%%%%%%%%%%%%
\DescribeMacro{\childdocby}
Each part to be included by |\input| should start with:
%
\begin{center}
\begin{tabular}{l}
|\input{childdoc.def}|\\
|\childdocby{|\textit{main}|}|\\
\end{tabular}
\end{center}
%
The directive |\childdocby| is similar to |\childdocof|
described in \secref{sec:include},
but the subsequent selection of content must be done manually.
To that end, both |\ifchilddoc| and |\ifchilddocmanual|
will be true upon processing of a part,
and the name of the part is stored in |\childdocname|.
Note that |\jobname| will be set to the filename of the current part
so that each part receives an individual |.aux| file
that does not interfere with the |.aux| file(s) of the main document.
This behaviour can be altered by the alternative form
|\childdocby[*]{|\textit{main}|}| (with a non-empty optional argument)
which uses the |.aux| file of the main document
by setting |\jobname| to \textit{main}.

%%%%%%%%%%%%%%%%%%%%%%%%%%%%%%%%%%%%%%%%%%%%%%%%%%%%%%%%%%%%%%%%%%%%%%%%%%%%%%%%
\subsection{Driver Development}
\label{sec:driver}

The \textsf{childdoc} mechanism can also be use for the development
of definition files such as \LaTeX{} styles or classes.
This case differs from the above setup with multiple parts
included by |\include| in that no |\includeonly| should be invoked.
This can be achieved by starting the include file
(before |\ProvidesPackage|) with:
%
\begin{center}
\begin{tabular}{l}
|\input{childdoc.def}|\\
|\childdocforward{|\textit{main}|}|\\
\end{tabular}
\end{center}
%
or alternatively with:
%
\begin{center}
\begin{tabular}{l}
|\input{childdoc.def}|\\
|\childdocby{|\textit{main}|}|\\
\end{tabular}
\end{center}
%
Both forms have slightly different effects as described above.
The main file is prepared as usual, see \secref{sec:include}.

%%%%%%%%%%%%%%%%%%%%%%%%%%%%%%%%%%%%%%%%%%%%%%%%%%%%%%%%%%%%%%%%%%%%%%%%%%%%%%%%
\subsection{Legacy Detection}
\label{sec:detection}

The directive |\childdocmain| in the main file can detect
whether the complete document or merely a child is to be compiled
even without using the directive |\childdocof|.
This method is deprecated because it is less robust
and there is no compelling reason to use it;
it is merely provided for backward compatibility
and it may be removed in future versions.

If the detection mechanism is to be used,
it is mandatory to correctly specify
the filename of the main file as the argument of |\childdocmain|:
%
\begin{center}
\begin{tabular}{l}
|\input{childdoc.def}|\\
|\childdocmain{|\textit{main}|}|\\
\end{tabular}
\end{center}
%
If |\jobname| does not match the argument \textit{main} of |\childdocmain|,
it is assumed that |\jobname| points to the child file to be compiled.
When using |\childdocmain| with the main file specified as argument,
it suffices to start a child file
with just |\input{|\textit{main}|}|
without loading of the package and using |\childdocof|.
If instead all processing is done
with the appropriate \textsf{childdoc} directives,
the argument of \textit{main} of |\childdocmain| can be empty.

An alternative version of the command line processing described
in \secref{sec:commandline} using the detection mechanism reads:
%
\begin{center}
|... -jobname "|\textit{target}|" "|[\textit{flags}]%
[|\def\jobname{|\textit{dest}|}|]|\input{|\textit{main}|}"|
\end{center}

%%%%%%%%%%%%%%%%%%%%%%%%%%%%%%%%%%%%%%%%%%%%%%%%%%%%%%%%%%%%%%%%%%%%%%%%%%%%%%%%
\subsection{Manual Code}
\label{sec:manual}

In case one cannot be certain whether the definitions file |childdoc.def|
is installed on the target \TeX{} distribution
and one prefers not to ship it,
it is conceivable to paste a few relevant commands into the sources.

To that end, drop all statements |\input{childdoc.def}|
and perform the replacements as outlined below.
Instead of |\childdocmain{|\textit{main}|}| add the following code
to the top of the main file:
%
\begin{center}
\begin{tabular}{l}
|\||ifdefined\childdocname\endinput\||fi\newif\ifchilddoc|\\
|\edef\childdocname{\scantokens\expandafter{\jobname\noexpand}}|\\
|\def\childdocmain{|\textit{main}|}\||ifx\childdocmain\childdocname\||else|\\
|\childdoctrue\includeonly{\childdocname}\let\jobname\childdocmain\||fi|\\
\end{tabular}
\end{center}
%
Instead of |\childdocof{|\textit{main}|}| just include the main file
at the top of each child file:
%
\begin{center}
|\input{|\textit{main}|}|
\end{center}
%
A simple redirection |\childdocforward{|\textit{dest}|}| is achieved by:
%
\begin{center}
|\def\jobname{|\textit{dest}|}\input{\jobname}|
\end{center}
%
The redirection with prefix
|\childdocforwardprefix[|\textit{prefix}|]{|\textit{dest}|}|
is accomplished by:
%
\begin{center}
\begin{tabular}{l}
|{\edef\jobname{\scantokens\expandafter{\jobname\noexpand}}|\\
|\def\redirectjob |\textit{prefix}|#1~~~{\gdef\jobname{|\textit{dest}|#1}}|\\
|\expandafter\redirectjob\jobname~~~}\input{\jobname}|
\end{tabular}
\end{center}

In an alternative approach,
child documents can be compiled by a specific command line
without additional code or specific definitions:
%
\begin{center}
|... -jobname "|\textit{target}|" "|[\textit{flags}]%
|\includeonly{|\textit{dest}|}\input{|\textit{main}|}"|
\end{center}
%

%%%%%%%%%%%%%%%%%%%%%%%%%%%%%%%%%%%%%%%%%%%%%%%%%%%%%%%%%%%%%%%%%%%%%%%%%%%%%%%%
%%%%%%%%%%%%%%%%%%%%%%%%%%%%%%%%%%%%%%%%%%%%%%%%%%%%%%%%%%%%%%%%%%%%%%%%%%%%%%%%
\section{Information}

%%%%%%%%%%%%%%%%%%%%%%%%%%%%%%%%%%%%%%%%%%%%%%%%%%%%%%%%%%%%%%%%%%%%%%%%%%%%%%%%
\subsection{Copyright}

Copyright \copyright{} 2017--2018 Niklas Beisert

This work may be distributed and/or modified under the
conditions of the \LaTeX{} Project Public License, either version 1.3
of this license or (at your option) any later version.
The latest version of this license is in
  \url{http://www.latex-project.org/lppl.txt}
and version 1.3 or later is part of all distributions of \LaTeX{}
version 2005/12/01 or later.

This work has the LPPL maintenance status `maintained'.

The Current Maintainer of this work is Niklas Beisert.

This work consists of the files |README.txt|, |childdoc.ins| and |childdoc.dtx|
as well as the derived files |childdoc.def|, |cdocsamp.tex|
with |cdocsch1.tex|, |cdocsch2.tex|, |cdocspt3.tex|, |cdocspt4.tex|,
|cdocsdrf.tex|, |cdocsfn1.tex|, |cdocsfn2.tex|
as well as |childdoc.pdf|.

%%%%%%%%%%%%%%%%%%%%%%%%%%%%%%%%%%%%%%%%%%%%%%%%%%%%%%%%%%%%%%%%%%%%%%%%%%%%%%%%
\subsection{Files and Installation}

The package consists of the files:
%
\begin{center}
\begin{tabular}{ll}
    |README.txt|   & readme file \\
    |childdoc.ins| & installation file \\
    |childdoc.dtx| & source file \\
    |childdoc.def| & definition file \\
    |cdocsamp.tex| & sample main file \\
    |cdocsch1.tex| & sample include file \\
    |cdocsch2.tex| & sample include file \\
    |cdocspt3.tex| & sample part file \\
    |cdocspt4.tex| & sample part file \\
    |cdocsdrf.tex| & sample redirection file \\
    |cdocsfn1.tex| & sample redirection file \\
    |cdocsfn2.tex| & sample redirection file \\
    |childdoc.pdf| & manual
\end{tabular}
\end{center}
%
The distribution consists of the files
|README.txt|, |childdoc.ins| and |childdoc.dtx|.
%
\begin{itemize}
\item
Run (pdf)\LaTeX{} on |childdoc.dtx|
to compile the manual |childdoc.pdf| (this file).
\item
Run \LaTeX{} on |childdoc.ins| to create the definitions file |childdoc.def|
and the sample |cdocsamp.tex| with include files
|cdocsch1.tex|, |cdocsch2.tex|, |cdocspt3.tex|, |cdocspt4.tex|,
|cdocsdrf.tex|, |cdocsfn1.tex|, |cdocsfn2.tex|.
Then copy the file |childdoc.def| to an appropriate directory of your \LaTeX{}
distribution, e.g.\ \textit{texmf-root}|/tex/latex/childdoc|.
\end{itemize}

%%%%%%%%%%%%%%%%%%%%%%%%%%%%%%%%%%%%%%%%%%%%%%%%%%%%%%%%%%%%%%%%%%%%%%%%%%%%%%%%
\subsection{Related CTAN Packages}

There are several other packages which offer a similar functionality:
%
\begin{itemize}
\item
The packages
\href{http://ctan.org/pkg/docmute}{\textsf{docmute}},
\href{http://ctan.org/pkg/includex}{\textsf{includex}} and
\href{http://ctan.org/pkg/standalone}{\textsf{standalone}}
provide commands to include only the document body of
a child file thus allowing both files to be compiled individually.
\item
The packages \href{http://ctan.org/pkg/subdocs}{\textsf{subdocs}}
and \href{http://ctan.org/pkg/subfiles}{\textsf{subfiles}}
provide structures in which the main and child documents can be
encapsulated and allowing them to be compiled individually.
The inclusion mechanism is different from the conventional |\include|.
\item
The package \href{http://ctan.org/pkg/combine}{\textsf{combine}}
is an elaborate solution to combine several documents into one.
\end{itemize}
%
See also the CTAN topic \href{http://ctan.org/topic/subdocs}{\textsf{subdocs}}
for further related packages.
The present package differs from the above solutions in that
a document structure constructed with the conventional |\include| mechanism
just needs two extra commands at the top of every file
such that all constituent files can be compiled individually.

%%%%%%%%%%%%%%%%%%%%%%%%%%%%%%%%%%%%%%%%%%%%%%%%%%%%%%%%%%%%%%%%%%%%%%%%%%%%%%%%
%\subsection{Feature Suggestions}
%
%The following is a list of features which may be useful for future
%versions of this package:
%%
%\begin{itemize}
%\item
%\ldots
%\end{itemize}

%%%%%%%%%%%%%%%%%%%%%%%%%%%%%%%%%%%%%%%%%%%%%%%%%%%%%%%%%%%%%%%%%%%%%%%%%%%%%%%%
\subsection{Revision History}

%%%%%%%%%%%%%%%%%%%%%%%%%%%%%%%%%%%%%%%%
\paragraph{v2.0:} 2018/12/30

\begin{itemize}
\item
immediate forward processing
\item
added |\childdocby| mechanism
\item
manual restructured
\end{itemize}

%%%%%%%%%%%%%%%%%%%%%%%%%%%%%%%%%%%%%%%%
\paragraph{v1.6:} 2018/01/17

\begin{itemize}
\item
application for development of include files
\item
corrections to manual
\end{itemize}

%%%%%%%%%%%%%%%%%%%%%%%%%%%%%%%%%%%%%%%%
\paragraph{v1.5:} 2017/05/21

\begin{itemize}
\item
more complete structuring introduced
\item
|\childdocof| introduced
\item
|\childdoc| renamed to |\childdocmain|
\item
|\childredirect| renamed to |\childdocforward| and |\childdocforwardprefix|
and functionality expanded
\end{itemize}

%%%%%%%%%%%%%%%%%%%%%%%%%%%%%%%%%%%%%%%%
\paragraph{v1.0:} 2017/04/27

\begin{itemize}
\item
manual and install package
\item
first version published on CTAN
\end{itemize}

%%%%%%%%%%%%%%%%%%%%%%%%%%%%%%%%%%%%%%%%
\paragraph{v0.6:} 2017/04/26

\begin{itemize}
\item
redirection mechanism added
\end{itemize}

%%%%%%%%%%%%%%%%%%%%%%%%%%%%%%%%%%%%%%%%
\paragraph{v0.5:} 2017/04/26

\begin{itemize}
\item
functionality in definition file
\end{itemize}


%%%%%%%%%%%%%%%%%%%%%%%%%%%%%%%%%%%%%%%%%%%%%%%%%%%%%%%%%%%%%%%%%%%%%%%%%%%%%%%%
%%%%%%%%%%%%%%%%%%%%%%%%%%%%%%%%%%%%%%%%%%%%%%%%%%%%%%%%%%%%%%%%%%%%%%%%%%%%%%%%
%%%%%%%%%%%%%%%%%%%%%%%%%%%%%%%%%%%%%%%%%%%%%%%%%%%%%%%%%%%%%%%%%%%%%%%%%%%%%%%%
\appendix

\settowidth\MacroIndent{\rmfamily\scriptsize 000\ }

 \DocInput{childdoc.dtx}

\end{document}
%</driver>
% \fi
%
% %%%%%%%%%%%%%%%%%%%%%%%%%%%%%%%%%%%%%%%%%%%%%%%%%%%%%%%%%%%%%%%%%%%%%%%%%%%%%%
% %%%%%%%%%%%%%%%%%%%%%%%%%%%%%%%%%%%%%%%%%%%%%%%%%%%%%%%%%%%%%%%%%%%%%%%%%%%%%%
% \section{Sample}
%\iffalse
%<*samplemain>
%\fi
%
% The following presents a sample document
% with two chapters, two parts, a title page,
% a compile flag as well as three forwarding files to set the flag.
% It consists of eight |.tex| files:
% \begin{center}
% \begin{tabular}{ll}
% |cdocsamp.tex|&main file\\
% |cdocsch1.tex|&include file for chapter 1\\
% |cdocsch2.tex|&include file for chapter 2\\
% |cdocspt3.tex|&include file for part 3\\
% |cdocspt4.tex|&include file for part 4\\
% |cdocsdrf.tex|&forwarding file for main file in draft mode\\
% |cdocsfi1.tex|&forwarding file for final version of chapter 1\\
% |cdocsfi2.tex|&forwarding file for final version of chapter 2\\
% \end{tabular}
% \end{center}
% Each of the eight files can be compiled directly by the \LaTeX{} compiler.
%
% %%%%%%%%%%%%%%%%%%%%%%%%%%%%%%%%%%%%%%
% \paragraph{Main File.}
%
% The main file is called |cdocsamp.tex|.
%
% Load the \textsf{childdoc} definitions and
% declare the filename for the main document:
%    \begin{macrocode}
\input{childdoc.def}
\childdocmain{}
%    \end{macrocode}

% Optional override for |\version| flag:
%    \begin{macrocode}
%%\ifchilddoc\else\providecommand{\version}{draft}\fi
%    \end{macrocode}

% Define the default values for the |\version| flag
% (|final| for the main file and |draft| for childs):
%    \begin{macrocode}
\ifchilddoc
\providecommand{\version}{draft}
\else
\providecommand{\version}{final}
\fi
%    \end{macrocode}

% Load the standard document class:
%    \begin{macrocode}
\documentclass[12pt]{article}
%    \end{macrocode}

% Start the document body:
%    \begin{macrocode}
\begin{document}
%    \end{macrocode}

% Declare a title page.
% Print title, part of document being processed and version flag:
%    \begin{macrocode}
\addtocounter{page}{-1}
\begin{center}
{\LARGE\bfseries{}childdoc example\par}
\vspace{1cm}
\ifchilddoc
\ifchilddocmanual part\else chapter\fi:
`\childdocname' of `\childdocjob'\par
\else
main document: `\childdocjob'\par
\fi
version: \version\par
\end{center}
\newpage
%    \end{macrocode}

% Manually include selected file,
% otherwise process as usual:
%    \begin{macrocode}
\ifchilddocmanual
\section*{part `\childdocname'}
\input{\childdocname}
\else
%    \end{macrocode}

% Include the two chapters:
%    \begin{macrocode}
\include{cdocsch1}
\include{cdocsch2}
%    \end{macrocode}

% Include the two parts unless only chapters should be displayed:
%    \begin{macrocode}
\ifchilddoc\else
\section{part three}
\input{cdocspt3}
\section{part four}
\input{cdocspt4}
\fi
%    \end{macrocode}

% Process as usual until here:
%    \begin{macrocode}
\fi
%    \end{macrocode}

% End of document body:
%    \begin{macrocode}
\end{document}
%    \end{macrocode}
%\iffalse
%</samplemain>
%\fi
%
% %%%%%%%%%%%%%%%%%%%%%%%%%%%%%%%%%%%%%%
% \paragraph{Chapter Include Files.}
%
% The include files are called |cdocsch1.tex| and |cdocsch2.tex|.
%
%\iffalse
%<*samplechap1|samplechap2>
%\fi

% Optional override for |\version| flag:
%    \begin{macrocode}
%%\providecommand{\version}{final}
%    \end{macrocode}

% Include the main document:
%    \begin{macrocode}
\input{childdoc.def}
\childdocof{cdocsamp}
%    \end{macrocode}

%\iffalse
%</samplechap1|samplechap2>
%\fi
%
%\iffalse
%<*samplechap1>
%\fi
% Some text for chapter 1:
%    \begin{macrocode}
\section{one}
some text in chapter one
%    \end{macrocode}

%\iffalse
%</samplechap1>
%\fi
% Some text for chapter 2:
%\iffalse
%<*samplechap2>
%\fi
%    \begin{macrocode}
\section{two}
more text in chapter two
%    \end{macrocode}

%\iffalse
%</samplechap2>
%\fi
%
% %%%%%%%%%%%%%%%%%%%%%%%%%%%%%%%%%%%%%%
% \paragraph{Part Include Files.}
%
% The include files are called |cdocspt3.tex| and |cdocspt4.tex|.
%
%\iffalse
%<*samplepart3|samplepart4>
%\fi

% Optional override for |\version| flag:
%    \begin{macrocode}
%%\providecommand{\version}{final}
%    \end{macrocode}

% Include the main document:
%    \begin{macrocode}
\input{childdoc.def}
\childdocby{cdocsamp}
%    \end{macrocode}

%\iffalse
%</samplepart3|samplepart4>
%\fi
%
%\iffalse
%<*samplepart3>
%\fi
% Some text for part 3:
%    \begin{macrocode}
some text in part three
%    \end{macrocode}

%\iffalse
%</samplepart3>
%\fi
% Some text for part 4:
%\iffalse
%<*samplepart4>
%\fi
%    \begin{macrocode}
more text in part four
%    \end{macrocode}

%\iffalse
%</samplepart4>
%\fi
%
% %%%%%%%%%%%%%%%%%%%%%%%%%%%%%%%%%%%%%%
% \paragraph{Forwarding for a Complete Draft.}
%
% The following forwarding file |cdocsdrf.tex|
% compiles the main document in draft mode:
%\iffalse
%<*sampledraft>
%\fi
%    \begin{macrocode}
\def\version{draft}
\input{childdoc.def}
\childdocforward{cdocsamp}
%    \end{macrocode}

%\iffalse
%</sampledraft>
%\fi
%
% %%%%%%%%%%%%%%%%%%%%%%%%%%%%%%%%%%%%%%
% \paragraph{Forwarding for Final Version of the Chapters.}
%
% The following forwarding files |cdocsfn1.tex| and |cdocsfn2.tex|
% (with identical content)
% compile the final versions of the child documents
% |cdocsch1.tex| and |cdocsch2.tex|, respectively:
%\iffalse
%<*samplefinal>
%\fi
%    \begin{macrocode}
\def\version{final}
\input{childdoc.def}
\childdocforwardprefix[cdocsamp]{cdocsfn}{cdocsch}
%    \end{macrocode}

%\iffalse
%</samplefinal>
%\fi
%
% %%%%%%%%%%%%%%%%%%%%%%%%%%%%%%%%%%%%%%
% \paragraph{Command Line Processing.}
%
% The following three command lines generate the output files
% |cdocscld|, |cdocscl1| and |cdocscl2|
% which should be identical to
% |cdocsdrf|, |cdocsch1| and |cdocsfn2|, respectively:
% \begin{center}
% \begin{tabular}{l}
% |latex -jobname cdocscld \|\\
% |  "\def\version{draft}\input{childdoc.def}\childdocforward{cdocsamp}"|\\
% |latex -jobname cdocscl1 \|\\
% |  "\input{childdoc.def}\childdocforward[cdocsamp]{cdocsch1}"|\\
% |latex -jobname cdocscl2 \|\\
% |  "\def\version{final}\input{childdoc.def}\childdocforward{cdocsch2}"|
% \end{tabular}
% \end{center}
% Note that the trailing backslash on each first line
% merely continues the input to the second line
% (for convenient cut ant paste).
% Furthermore, the command |latex| can be replaced by any
% of its alternative versions such as |pdflatex|.
%
% %%%%%%%%%%%%%%%%%%%%%%%%%%%%%%%%%%%%%%%%%%%%%%%%%%%%%%%%%%%%%%%%%%%%%%%%%%%%%%
% %%%%%%%%%%%%%%%%%%%%%%%%%%%%%%%%%%%%%%%%%%%%%%%%%%%%%%%%%%%%%%%%%%%%%%%%%%%%%%
% \section{Implementation}
%\iffalse
%<*package>
%\fi
%
% This section describes the definitions file |childdoc.def|.

% The definitions cannot be loaded using |\usepackage| or |\RequirePackage|
% which has a mechanism to prevent loading a style file more than once.
% When loading the definitions by means of |\input|
% multiple instances have to be prevented manually:
%\iffalse
%This code needs to be before the `\ProvidesFile' directive
%which is defined at the beginning of this file.
%Therefore it is also placed there and commented out here.
%</package>
%<*discard>
%\fi
%    \begin{macrocode}
\ifdefined\childdocmain\endinput\fi
%    \end{macrocode}
%\iffalse
%</discard>
%<*package>
%\fi
%
% \macro{\ifchilddoc}
% \macro{\ifchilddocmanual}
% The conditional |\ifchilddoc| tells whether a
% child (true) or main (false) document is being compiled.
% The conditional |\ifchilddocmanual| tells whether
% the |\includeonly| mechanism is used (false) or
% the selection of child files must be performed manually (true).
% The definitions initialise to false:
%    \begin{macrocode}
\newif\ifchilddoc
\newif\ifchilddocmanual
%    \end{macrocode}

% \macro{\childdocname}
% \macro{\childdocjob}
% The macro |\childdocname| stores the name of the main document
% to be compiled. The macro |\childdocjob| stores the name of
% the document on which the \LaTeX{} compiler was originally invoked.
% The content of |\jobname| cannot be compared
% to filenames specified in the source due to different catcodes.
% The following code rescans |\jobname|, stores the result
% in |\childdocname| and saves a copy in |\childdocjob|:
%    \begin{macrocode}
\edef\childdocname{\scantokens\expandafter{\jobname\noexpand}}
\let\childdocjob\childdocname
%    \end{macrocode}

% \macro{\childdocdisable}
% The macro |\childdocdisable| prevents the main file
% from being processed more than once.
% At this stage, the main document command |\childdocmain|
% is assumed to be called once again where it should do nothing.
% Any subsequent call to it should prevent
% a secondary processing of the main document
% It overwrites the forwarding commands
% |\childdocof| and |\childdocforward|
% with empty macros to prevent further inclusions of the main document:
%    \begin{macrocode}
\newcommand{\childdocdisable}
{
  \renewcommand{\childdocmain}[1]{\renewcommand{\childdocmain}[1]{\endinput}}
  \renewcommand{\childdocof}[1]{}
  \renewcommand{\childdocby}[2][]{}
  \renewcommand{\childdocforward}[2][]{}
  \renewcommand{\childdocdisable}{}
}
%    \end{macrocode}

% \macro{\childdocmain}
% The macro |\childdocmain| is to be called at the top of the main file
% with nothing or the main filename (without extension) as argument.
% First, it breaks loops.
% If the argument is not empty and does not match |\childdocname|
% (which is set by the first inclusion of |childdoc.def|),
% |\ifchilddoc| is set to true, |\includeonly| is applied to the child file
% and |\jobname| is set to the main file
% (for proper handling of |.aux| files):
%    \begin{macrocode}
\newcommand{\childdocmain}[1]
{
  \childdocdisable\childdocmain{}
  \if?#1?\else
    \begingroup
      \def\childdoctmp{#1}
      \ifx\childdoctmp\childdocname
        \def\childdoctmp{}
      \else
        \def\childdoctmp
        {
          \childdoctrue
          \includeonly{\childdocname}
          \def\childdocjob{#1}
          \def\jobname{#1}
        }
      \fi
      \expandafter
    \endgroup
    \childdoctmp
  \fi
}
%    \end{macrocode}

% \macro{\childdocof}
% The command |\childdocof| redirects
% compilation to the main file |#1|.
%    \begin{macrocode}
\newcommand{\childdocof}[1]
{
  \childdocdisable
  \childdoctrue
  \includeonly{\childdocname}
  \def\jobname{#1}
  \def\childdocjob{#1}
  \input{#1}
}
%    \end{macrocode}

% \macro{\childdocby}
% The command |\childdocby| ....
%    \begin{macrocode}
\newcommand{\childdocby}[2][]
{
  \childdocdisable
  \childdoctrue
  \childdocmanualtrue
  \if?#1?\else
    \def\jobname{#2}
  \fi
  \def\childdocjob{#2}
  \input{#2}
  \endinput
}
%    \end{macrocode}

% \macro{\childdocforward}
% The command |\childdocforward| redirects
% compilation to the main file or
% (if the optional argument is given) a child file.
% Parameters are set as if the main file
% or a child file starting with |\childdocof| was compiled.
% Then compilation is handed over to the main file:
%    \begin{macrocode}
\newcommand{\childdocforward}[2][]
{
  \begingroup
    \if?#1?
      \def\childdoctmp
      {
        \def\childdocname{#2}
        \def\childdocjob{#2}
        \def\jobname{#2}
        \input{#2}
        \endinput
      }
    \else
      \def\childdoctmp
      {
        \childdocdisable
        \def\childdocname{#2}
        \childdoctrue
        \includeonly{#2}
        \def\childdocjob{#1}
        \def\jobname{#1}
        \input{#1}
        \endinput
      }
    \fi
    \expandafter
  \endgroup
  \childdoctmp
}
%    \end{macrocode}

% \macro{\childdocforwardprefix}
% The command |\childdocforwardprefix| redirects
% compilation to the main or a child file by means of a pattern.
% The prefix |#1| in the current filename is replaced by |#2|
% and the suffix of the current filename is kept
% (it is assumed that the filename does not contain the substring `|~~~|'
% which is used as a delimiter).
% Compilation is handed over to the new file by |\childdocforward|:
%    \begin{macrocode}
\newcommand{\childdocforwardprefix}[3][]
{
  \begingroup
    \def\childdocextract #2##1~~~{\def\childdoctmp{\childdocforward[#1]{#3##1}}}
    \expandafter\childdocextract\childdocname~~~
    \expandafter
  \endgroup
  \childdoctmp
}
%    \end{macrocode}

% \macro{\childdoc}
% The deprecated macro |\childdoc| is a legacy version of |\childdocmain|:
%    \begin{macrocode}
\newcommand{\childdoc}{\childdocmain}
%    \end{macrocode}

% \macro{\childdocredirect}
% The deprecated macro |\childdocredirect| is a legacy version
% of |\childdocforward| and |\childdocforwardprefix|:
%    \begin{macrocode}
\newcommand{\childdocredirect}[2][]
{
  \begingroup
    \if?#1?
      \def\childdoctmp{\childdocforward{#2}}
    \else
      \def\childdoctmp{\childdocforwardprefix{#1}{#2}}
    \fi
    \expandafter
  \endgroup
  \childdoctmp
}
%    \end{macrocode}

%\iffalse
%</package>
%\fi
%
\endinput
|\\
|\childdocmain{}|\\
\end{tabular}
\end{center}
at the very top of the main \LaTeX{} file,
in particular \emph{before} the |\documentclass| statement!
The argument of |\childdocmain| should be left empty
(but it must be present).

%%%%%%%%%%%%%%%%%%%%%%%%%%%%%%%%%%%%%%%%
\DescribeMacro{\childdocof}
Furthermore, add the commands
\begin{center}
\begin{tabular}{l}
|% \iffalse
%
% childdoc.dtx Copyright (C) 2017-2018 Niklas Beisert
%
% This work may be distributed and/or modified under the
% conditions of the LaTeX Project Public License, either version 1.3
% of this license or (at your option) any later version.
% The latest version of this license is in
%   http://www.latex-project.org/lppl.txt
% and version 1.3 or later is part of all distributions of LaTeX
% version 2005/12/01 or later.
%
% This work has the LPPL maintenance status `maintained'.
%
% The Current Maintainer of this work is Niklas Beisert.
%
% This work consists of the files childdoc.dtx and childdoc.ins
% and the derived files childdoc.def and cdocsamp.tex with
% cdocsch1.tex, cdocsch2.tex, cdocsdrf.tex, cdocsfn1.tex, cdocsfn2.tex.
%
%<package>\ifdefined\childdocmain\endinput\fi
%<package>\ProvidesFile{childdoc.def}[2018/12/30 v2.0 child document driver]
%<samplemain>\ProvidesFile{cdocsamp.tex}[2018/12/30 v2.0 sample for childdoc]
%<*driver>
%\ProvidesFile{childdoc.drv}[2018/12/30 v2.0 childdoc reference manual file]
\PassOptionsToClass{10pt,a4paper}{article}
\documentclass{ltxdoc}

\usepackage[margin=35mm]{geometry}
\usepackage{hyperref}
\usepackage{hyperxmp}
\usepackage[usenames]{color}

\hypersetup{colorlinks=true}
\hypersetup{pdfstartview=FitH}
\hypersetup{pdfpagemode=UseNone}
\hypersetup{pdfsource={}}
\hypersetup{pdflang={en-UK}}
\hypersetup{pdfcopyright={Copyright 2017-2018 Niklas Beisert.
  This work may be distributed and/or modified under the
  conditions of the LaTeX Project Public License, either version 1.3
  of this license or (at your option) any later version.}}
\hypersetup{pdflicenseurl={http://www.latex-project.org/lppl.txt}}
\hypersetup{pdfcontactaddress={ETH Zurich, ITP, HIT K,
  Wolfgang-Pauli-Strasse 27}}
\hypersetup{pdfcontactpostcode={8093}}
\hypersetup{pdfcontactcity={Zurich}}
\hypersetup{pdfcontactcountry={Switzerland}}
\hypersetup{pdfcontactemail={nbeisert@itp.phys.ethz.ch}}
\hypersetup{pdfcontacturl={http://people.phys.ethz.ch/\xmptilde nbeisert/}}

\newcommand{\secref}[1]{\hyperref[#1]{section \ref*{#1}}}

\parskip1ex
\parindent0pt
\let\olditemize\itemize
\def\itemize{\olditemize\parskip0pt}

\begin{document}

\title{The \textsf{childdoc} Package}
\hypersetup{pdftitle={The childdoc Package}}
\author{Niklas Beisert\\[2ex]
  Institut f\"ur Theoretische Physik\\
  Eidgen\"ossische Technische Hochschule Z\"urich\\
  Wolfgang-Pauli-Strasse 27, 8093 Z\"urich, Switzerland\\[1ex]
  \href{mailto:nbeisert@itp.phys.ethz.ch}
  {\texttt{nbeisert@itp.phys.ethz.ch}}}
\hypersetup{pdfauthor={Niklas Beisert}}
\hypersetup{pdfsubject={Manual for the LaTeX2e Package childdoc}}
\date{30 December 2018, \textsf{v2.0}}
\maketitle

\begin{abstract}\noindent
\textsf{childdoc} is a \LaTeXe{} package
that enables the direct compilation
of document sections included by |\include|
to individual files.
\end{abstract}

\begingroup
\parskip0ex
\tableofcontents
\endgroup

%%%%%%%%%%%%%%%%%%%%%%%%%%%%%%%%%%%%%%%%%%%%%%%%%%%%%%%%%%%%%%%%%%%%%%%%%%%%%%%%
%%%%%%%%%%%%%%%%%%%%%%%%%%%%%%%%%%%%%%%%%%%%%%%%%%%%%%%%%%%%%%%%%%%%%%%%%%%%%%%%
\section{Introduction}

\LaTeX{} provides a mechanism to structure a large document (such as a book)
into a main file and several child files (containing the chapters)
using the |\include| command.
This mechanism is beneficial for documents
which span hundreds of pages in order to
make the source file(s) more manageable.
Moreover, compilation can be restricted to
selected child files by means of the |\includeonly| command.
The latter feature can be used to reduce the compilation time while editing
(this was significantly more useful in the earlier days of \LaTeX{})
or to generate a smaller document which is easier to navigate.
Another application of |\includeonly| is to generate
documents consisting of selected parts of the complete document.

However, there are a few drawbacks of the plain |\include| mechanism:
\begin{itemize}
\item
The child files cannot be compiled on their own,
they can only be compiled via the main file.
A naive editing environment
(such as a text editor with an option
to have the current file processed by \LaTeX)
may require one to switch to the main file before compiling;
attempting to compile the child file produces errors.
\item
The main file must be modified (each time)
to adjust the |\includeonly| command
to the present needs. This easily leaves the main file in a messy state.
\item
The generated document will always carry the filename
of the main document. This is inconvenient if
several child files are to be compiled and
to be kept for distribution.
\end{itemize}

The present package provides a simple interface
to make child files individually compilable by \LaTeX{}.
Compiling a child file then has the same effect as compiling
the main file with an |\includeonly| command
to select the appropriate child.
Moreover the generated document will carry the name of the child
rather than the main file.
This resolves all three above issues.

This feature is meant to make the editing of books,
thesis documents and lecture notes somewhat more convenient.
However, the package can also be used efficiently for
composing a series of documents (such as exercise sheets)
which are typically distributed individually.
It then assists the author in generating the individual documents
(potentially in different versions)
as well as a document containing the collected series.
Another application is in developing style files
or other kinds of included material
where compilation of the style file could redirect
to a sample or test file.

%%%%%%%%%%%%%%%%%%%%%%%%%%%%%%%%%%%%%%%%%%%%%%%%%%%%%%%%%%%%%%%%%%%%%%%%%%%%%%%%
%%%%%%%%%%%%%%%%%%%%%%%%%%%%%%%%%%%%%%%%%%%%%%%%%%%%%%%%%%%%%%%%%%%%%%%%%%%%%%%%
\section{Usage}

First of all, the package \textsf{childdoc} is \emph{not} a standard
\LaTeXe{} |.sty| style file! Therefore it needs to be invoked in
a non-standard way.

%%%%%%%%%%%%%%%%%%%%%%%%%%%%%%%%%%%%%%%%%%%%%%%%%%%%%%%%%%%%%%%%%%%%%%%%%%%%%%%%
\subsection{Included Files}
\label{sec:include}

%%%%%%%%%%%%%%%%%%%%%%%%%%%%%%%%%%%%%%%%
\DescribeMacro{\childdocmain}
To use the package, add the commands
\begin{center}
\begin{tabular}{l}
|\input{childdoc.def}|\\
|\childdocmain{}|\\
\end{tabular}
\end{center}
at the very top of the main \LaTeX{} file,
in particular \emph{before} the |\documentclass| statement!
The argument of |\childdocmain| should be left empty
(but it must be present).

%%%%%%%%%%%%%%%%%%%%%%%%%%%%%%%%%%%%%%%%
\DescribeMacro{\childdocof}
Furthermore, add the commands
\begin{center}
\begin{tabular}{l}
|\input{childdoc.def}|\\
|\childdocof{|\textit{main}|}|\\
\end{tabular}
\end{center}
at the top of every child file \textit{child}
which is included by |\include{|\textit{child}|}|
from within the main file
(or at least for those files to be compiled individually).
The argument \textit{main} must be the filename of the main file.

There are a couple of
considerations in setting up the main and child documents:

%%%%%%%%%%%%%%%%%%%%%%%%%%%%%%%%%%%%%%%%
\paragraph{Restrictions.}

Please note the following restrictions:
\begin{itemize}
\item
|\childdocmain| must be called with one argument \textit{main}
to ensure compatibility with earlier version of the package.
It must either be empty (|\childdocmain{}|)
or precisely match the filename of the main file in which it is specified.
See \secref{sec:detection} for further information.
\item
The filename \textit{main} must be specified without the |.tex| extension.
\item
The filename \textit{main} is case sensitive
(even in case-insensitive file systems)
due to internal string comparison.
\item
The argument \textit{main} should be fully expanded, it cannot be a macro.
\item
Subdirectories and special characters should be avoided in filenames.
\item
The command |\childdocmain{|\textit{main}|}| must be followed by a whitespace.
It should not be followed immediately by another command
or by a comment mark `|%|'.
This is because the \TeX{} parser reads the token immediately following
the argument of |\childdocmain| and puts it
at the beginning of every child section;
however, a white\-space is ignored.
\end{itemize}

%%%%%%%%%%%%%%%%%%%%%%%%%%%%%%%%%%%%%%%%
\paragraph{Content of Main File.}

It is advisable to place all content in the child files included by |\include|.
Any output contained in the main file will appear in all child documents
unless suppressed manually;
it cannot be suppressed automatically by the |\includeonly| directive
and thus should normally be avoided.
A method to include some content in the main file
by means of conditional processing is described in \secref{sec:conditional}.

%%%%%%%%%%%%%%%%%%%%%%%%%%%%%%%%%%%%%%%%
\paragraph{Page Numbering.}

When only a part of the document is compiled,
the appropriate numbering of pages
(as well as other status parameters)
is determined from the |.aux| files.
The latter contain information from previous passes.
However this information needs to propagate through
all intermediate child documents.
Therefore the page numbering in child documents may well
be inconsistent until the complete document is compiled at least once.

A useful (if unconventional) way to always ensure a consistent
page numbering is to restart the numbering in each child document
and denote the pages by `\textit{child}|.|\textit{page}'
where \textit{child} represents the chapter/section number of the child file.
This can be achieved by the command
|\numberwithin{page}{|\textit{child}|}|
of the \textsf{amsmath} package
where \textit{child} can be |chapter| or |section|
depending on the chosen structuring.
Alternatively, one can modify the macro |\thepage| appropriately
and reset the counter |page| at the start of each child file.

%%%%%%%%%%%%%%%%%%%%%%%%%%%%%%%%%%%%%%%%%%%%%%%%%%%%%%%%%%%%%%%%%%%%%%%%%%%%%%%%
\subsection{Conditional Processing}
\label{sec:conditional}

The package provides a mechanism to compile different versions
of a document. To customise the versions further some conditional processing
can come in handy to distinguish which version is being compiled.
The package provides two macros to describe the compilation context:

%%%%%%%%%%%%%%%%%%%%%%%%%%%%%%%%%%%%%%%%
\DescribeMacro{\ifchilddoc}
The conditional |\ifchilddoc| distinguishes between the compilation of
child documents and the main document:
%
\begin{center}
|\ifchilddoc |\textit{child-code}| |[|\||else |\textit{main-code}]| \||fi|
\end{center}

%%%%%%%%%%%%%%%%%%%%%%%%%%%%%%%%%%%%%%%%
\DescribeMacro{\childdocname}
\DescribeMacro{\childdocjob}
The macro |\childdocname| contains the filename (without extension)
of the main or child file being processed.
Note that |\childdocjob| will always contain the name of the main file.

%%%%%%%%%%%%%%%%%%%%%%%%%%%%%%%%%%%%%%%%
\paragraph{Title Page.}

Conditional processing can be used to include a title or banner page
in the main document when proper precautions are taken.
Importantly, the code in the main file should ensure that the page counter
(as well as other status parameters which are stored in the |.aux| files)
takes the same value after the conditional processing.
Otherwise the page numbers may take divergent values
depending on which part is compiled.

For example, a title page could be declared by:
%
\begin{center}
\begin{tabular}{l}
|\ifchilddoc\||else|\\
|\addtocounter{page}{-1}|\\
\textit{code for title page}\\
|\newpage|\\
|\||fi|
\end{tabular}
\end{center}
%
A banner page for the child documents can be generated by:
%
\begin{center}
\begin{tabular}{l}
|\ifchilddoc|\\
|\addtocounter{page}{-1}|\\
\textit{code for banner page}\\
|\newpage|\\
|\||fi|
\end{tabular}
\end{center}
%
Here one could write a message such as:
\begin{center}
|This is the part \childdocname{} of \childdocjob{}.|
\end{center}

%%%%%%%%%%%%%%%%%%%%%%%%%%%%%%%%%%%%%%%%%%%%%%%%%%%%%%%%%%%%%%%%%%%%%%%%%%%%%%%%
\subsection{Flags}
\label{sec:flags}

The package makes it easy to generate different versions
of the main or child documents.
To this end compilation flags can be defined
and assigned different default values.
They will be particularly useful in conjunction
with the forwarding mechanism described in \secref{sec:forward}.

For example, it may be useful to have a flag |\version|
which can be set to |draft| or |final|.
The document source will contain some conditional code
depending on the value of |\version|.
Suppose further, the flag should default to |final| for the main file
and to |draft| for child files
which is a natural assignment for editing the document.
This is achieved by placing the following code
in the preamble of the main document
(below the |\childdocmain| directive):
%
\begin{center}
\begin{tabular}{l}
|\ifchilddoc|\\
|\providecommand{\version}{draft}|\\
|\||else|\\
|\providecommand{\version}{final}|\\
|\||fi|
\end{tabular}
\end{center}
%
The definition by |\providecommand| makes sure
that previous definitions are not overwritten.
Further statements |\providecommand{\version}{...}|
can thus be added before the above code to override it.

For the main file, one might add a line
(between |\childdocmain| and the above block)
%
\begin{center}
|%\ifchilddoc\||else\providecommand{\version}{draft}\||fi|
\end{center}
%
which can be uncommented to produce a draft version.
Likewise one can add a line to the very top of a child file
(above the |\childdocof{|\textit{main}|}| directive)
%
\begin{center}
|%\providecommand{\version}{final}|
\end{center}
%
which can be uncommented to produce the final version of this child document.

%%%%%%%%%%%%%%%%%%%%%%%%%%%%%%%%%%%%%%%%%%%%%%%%%%%%%%%%%%%%%%%%%%%%%%%%%%%%%%%%
\subsection{Forwarding}
\label{sec:forward}

Different versions of the main or child documents
using compilation flags as described in \secref{sec:flags}
can be (permanently) stored in different files
for convenient compilation, viewing and distribution.
To this end, the package defines a command
to pass on compilation to a different file:

%%%%%%%%%%%%%%%%%%%%%%%%%%%%%%%%%%%%%%%%
\DescribeMacro{\childdocforward}
The command |\childdocforward| redirects processing to
another source file:
%
\begin{center}
\begin{tabular}{l}
|\input{childdoc.def}|\\
|\childdocforward[|\textit{main}|]{|\textit{dest}|}|\\
\end{tabular}
\end{center}
%
The argument \textit{dest} is the destination file
(without extension).
It should be the main file or one of the child files.
Note that further \textsf{childdoc} directives
such as |\childdocof| and |\childdocforward|
in the indicated file will be processed in this form.
The optional argument \textit{main}
passes on directly to the main file \textit{main}
while pretending to compile the child \textit{dest}.
This form behaves as if \textit{dest}
issues |\childdocof{|\textit{main}|}| right away,
and no further \textsf{childdoc} directives will be processed.

%%%%%%%%%%%%%%%%%%%%%%%%%%%%%%%%%%%%%%%%
\DescribeMacro{\...prefix}
In the alternative form |\childdocforwardprefix|,
%
\begin{center}
\begin{tabular}{l}
|\input{childdoc.def}|\\
|\childdocforwardprefix[|\textit{main}|]{|\textit{prefix}|}{|\textit{dest}|}|
\end{tabular}
\end{center}
%
the destination file is determined by a pattern
depending on the current file:
To make this work, the current file must be called
`{\textit{prefix}\hspace{0.2em}\textit{suffix}}'
with \textit{prefix} matching precisely the argument.
Processing is then passed on to the file
`{\textit{dest}\hspace{0.2em}\textit{suffix}}'.
Surely, the same effect is achieved by
directly specifying the
argument `{\textit{dest}\hspace{0.2em}\textit{suffix}}'
in the first form.
However, that requires to set up a different file
for each child. With the alternative form of the command
all these files can have exactly the same content
which simplifies setting them up and maintaining them.

For example, the following file |draft.tex|
with a compilation flag |\version| as described in \secref{sec:flags}
compiles the main document as a draft:
%
\begin{center}
\begin{tabular}{l}
|\def\version{draft}|\\
|\input{childdoc.def}|\\
|\childdocforward{|\textit{main}|}|
\end{tabular}
\end{center}
%
Likewise, the following files |final|\textit{nn}|.tex|
compile the final version of the child document
|child|\textit{nn}|.tex|:
%
\begin{center}
\begin{tabular}{l}
|\def\version{final}|\\
|\input{childdoc.def}|\\
|\childdocforwardprefix{final}{child}|
\end{tabular}
\end{center}
%

Note that when several versions of a main file and/or of each child file
are to be generated, it may be convenient to set up a |Makefile| or
shell script to automatise the process.

%%%%%%%%%%%%%%%%%%%%%%%%%%%%%%%%%%%%%%%%%%%%%%%%%%%%%%%%%%%%%%%%%%%%%%%%%%%%%%%%
\subsection{Command Line Processing}
\label{sec:commandline}

The effect of redirection files can also be achieved by invoking
the \LaTeX{} compiler with a more elaborate command line.
Most conveniently this should be done as part
of a shell script or a |Makefile|.

When using \textsf{childdoc} in the main file, the following
command lines effectively perform a redirection
(note that depending on the shell being used,
backslashes may have to be doubled: `|\|' $\to$ `|\\|'):
%
\begin{center}
|... -jobname "|\textit{target}|" |\\|"|[\textit{flags}]%
|\input{childdoc.def}\childdocforward[|\textit{main}|]{|\textit{dest}|}"|
\end{center}
%
Here \textit{target} is the name of the output file,
\textit{main} is the name of the main file
and \textit{dest} is the name of the main or child file to be processed
(all filenames without extensions).
The optional argument \textit{main} can be omitted
if \textit{main} matches \textit{dest}.
Optionally, compilation \textit{flags} can be defined via |\def| commands.
This command line makes the \TeX{} engine believe
it is compiling the file \textit{target}
whose content is specified as the latter parameter.
The provided code then forwards the processing to
\textit{main} or \textit{dest} as described in \secref{sec:forward}.

%%%%%%%%%%%%%%%%%%%%%%%%%%%%%%%%%%%%%%%%%%%%%%%%%%%%%%%%%%%%%%%%%%%%%%%%%%%%%%%%
\subsection{Include by Input}
\label{sec:input}

Including child documents by |\include| has some restrictions by design.
Most notably, the content of a child document always occupies
its own set of pages; pages cannot be shared between child documents.
Usually, this behaviour makes perfect sense
because each child document contain an essential part of the document.
However, in some situations it may be desirable to compose
a document from a collection of parts
without having mandatory page breaks between then.
For this case, the package
provides a mechanism to include parts
by |\input| which can also be processed individually.
However, by construction this mechanism
requires manual handling of the content to be output.

%%%%%%%%%%%%%%%%%%%%%%%%%%%%%%%%%%%%%%%%
\DescribeMacro{\ifchilddocmanual}
The main file should be prepared as usual, see \secref{sec:include}.
However, the document body must make a distinction
between processing of an individual part and of the main document, e.g.:
%
\begin{center}
\begin{tabular}{l}
|\ifchilddocmanual|\\
|\input{\childdocname}|\\
|\||else|\\
\textit{document body with }|\input{|\textit{part}|}|\\
|\||fi|
\end{tabular}
\end{center}
%
The conditional |\ifchilddocmanual| is true whenever
a part to be included by |\input| is being compiled,
and the name of the part is stored in |\childdocname|.

%%%%%%%%%%%%%%%%%%%%%%%%%%%%%%%%%%%%%%%%
\DescribeMacro{\childdocby}
Each part to be included by |\input| should start with:
%
\begin{center}
\begin{tabular}{l}
|\input{childdoc.def}|\\
|\childdocby{|\textit{main}|}|\\
\end{tabular}
\end{center}
%
The directive |\childdocby| is similar to |\childdocof|
described in \secref{sec:include},
but the subsequent selection of content must be done manually.
To that end, both |\ifchilddoc| and |\ifchilddocmanual|
will be true upon processing of a part,
and the name of the part is stored in |\childdocname|.
Note that |\jobname| will be set to the filename of the current part
so that each part receives an individual |.aux| file
that does not interfere with the |.aux| file(s) of the main document.
This behaviour can be altered by the alternative form
|\childdocby[*]{|\textit{main}|}| (with a non-empty optional argument)
which uses the |.aux| file of the main document
by setting |\jobname| to \textit{main}.

%%%%%%%%%%%%%%%%%%%%%%%%%%%%%%%%%%%%%%%%%%%%%%%%%%%%%%%%%%%%%%%%%%%%%%%%%%%%%%%%
\subsection{Driver Development}
\label{sec:driver}

The \textsf{childdoc} mechanism can also be use for the development
of definition files such as \LaTeX{} styles or classes.
This case differs from the above setup with multiple parts
included by |\include| in that no |\includeonly| should be invoked.
This can be achieved by starting the include file
(before |\ProvidesPackage|) with:
%
\begin{center}
\begin{tabular}{l}
|\input{childdoc.def}|\\
|\childdocforward{|\textit{main}|}|\\
\end{tabular}
\end{center}
%
or alternatively with:
%
\begin{center}
\begin{tabular}{l}
|\input{childdoc.def}|\\
|\childdocby{|\textit{main}|}|\\
\end{tabular}
\end{center}
%
Both forms have slightly different effects as described above.
The main file is prepared as usual, see \secref{sec:include}.

%%%%%%%%%%%%%%%%%%%%%%%%%%%%%%%%%%%%%%%%%%%%%%%%%%%%%%%%%%%%%%%%%%%%%%%%%%%%%%%%
\subsection{Legacy Detection}
\label{sec:detection}

The directive |\childdocmain| in the main file can detect
whether the complete document or merely a child is to be compiled
even without using the directive |\childdocof|.
This method is deprecated because it is less robust
and there is no compelling reason to use it;
it is merely provided for backward compatibility
and it may be removed in future versions.

If the detection mechanism is to be used,
it is mandatory to correctly specify
the filename of the main file as the argument of |\childdocmain|:
%
\begin{center}
\begin{tabular}{l}
|\input{childdoc.def}|\\
|\childdocmain{|\textit{main}|}|\\
\end{tabular}
\end{center}
%
If |\jobname| does not match the argument \textit{main} of |\childdocmain|,
it is assumed that |\jobname| points to the child file to be compiled.
When using |\childdocmain| with the main file specified as argument,
it suffices to start a child file
with just |\input{|\textit{main}|}|
without loading of the package and using |\childdocof|.
If instead all processing is done
with the appropriate \textsf{childdoc} directives,
the argument of \textit{main} of |\childdocmain| can be empty.

An alternative version of the command line processing described
in \secref{sec:commandline} using the detection mechanism reads:
%
\begin{center}
|... -jobname "|\textit{target}|" "|[\textit{flags}]%
[|\def\jobname{|\textit{dest}|}|]|\input{|\textit{main}|}"|
\end{center}

%%%%%%%%%%%%%%%%%%%%%%%%%%%%%%%%%%%%%%%%%%%%%%%%%%%%%%%%%%%%%%%%%%%%%%%%%%%%%%%%
\subsection{Manual Code}
\label{sec:manual}

In case one cannot be certain whether the definitions file |childdoc.def|
is installed on the target \TeX{} distribution
and one prefers not to ship it,
it is conceivable to paste a few relevant commands into the sources.

To that end, drop all statements |\input{childdoc.def}|
and perform the replacements as outlined below.
Instead of |\childdocmain{|\textit{main}|}| add the following code
to the top of the main file:
%
\begin{center}
\begin{tabular}{l}
|\||ifdefined\childdocname\endinput\||fi\newif\ifchilddoc|\\
|\edef\childdocname{\scantokens\expandafter{\jobname\noexpand}}|\\
|\def\childdocmain{|\textit{main}|}\||ifx\childdocmain\childdocname\||else|\\
|\childdoctrue\includeonly{\childdocname}\let\jobname\childdocmain\||fi|\\
\end{tabular}
\end{center}
%
Instead of |\childdocof{|\textit{main}|}| just include the main file
at the top of each child file:
%
\begin{center}
|\input{|\textit{main}|}|
\end{center}
%
A simple redirection |\childdocforward{|\textit{dest}|}| is achieved by:
%
\begin{center}
|\def\jobname{|\textit{dest}|}\input{\jobname}|
\end{center}
%
The redirection with prefix
|\childdocforwardprefix[|\textit{prefix}|]{|\textit{dest}|}|
is accomplished by:
%
\begin{center}
\begin{tabular}{l}
|{\edef\jobname{\scantokens\expandafter{\jobname\noexpand}}|\\
|\def\redirectjob |\textit{prefix}|#1~~~{\gdef\jobname{|\textit{dest}|#1}}|\\
|\expandafter\redirectjob\jobname~~~}\input{\jobname}|
\end{tabular}
\end{center}

In an alternative approach,
child documents can be compiled by a specific command line
without additional code or specific definitions:
%
\begin{center}
|... -jobname "|\textit{target}|" "|[\textit{flags}]%
|\includeonly{|\textit{dest}|}\input{|\textit{main}|}"|
\end{center}
%

%%%%%%%%%%%%%%%%%%%%%%%%%%%%%%%%%%%%%%%%%%%%%%%%%%%%%%%%%%%%%%%%%%%%%%%%%%%%%%%%
%%%%%%%%%%%%%%%%%%%%%%%%%%%%%%%%%%%%%%%%%%%%%%%%%%%%%%%%%%%%%%%%%%%%%%%%%%%%%%%%
\section{Information}

%%%%%%%%%%%%%%%%%%%%%%%%%%%%%%%%%%%%%%%%%%%%%%%%%%%%%%%%%%%%%%%%%%%%%%%%%%%%%%%%
\subsection{Copyright}

Copyright \copyright{} 2017--2018 Niklas Beisert

This work may be distributed and/or modified under the
conditions of the \LaTeX{} Project Public License, either version 1.3
of this license or (at your option) any later version.
The latest version of this license is in
  \url{http://www.latex-project.org/lppl.txt}
and version 1.3 or later is part of all distributions of \LaTeX{}
version 2005/12/01 or later.

This work has the LPPL maintenance status `maintained'.

The Current Maintainer of this work is Niklas Beisert.

This work consists of the files |README.txt|, |childdoc.ins| and |childdoc.dtx|
as well as the derived files |childdoc.def|, |cdocsamp.tex|
with |cdocsch1.tex|, |cdocsch2.tex|, |cdocspt3.tex|, |cdocspt4.tex|,
|cdocsdrf.tex|, |cdocsfn1.tex|, |cdocsfn2.tex|
as well as |childdoc.pdf|.

%%%%%%%%%%%%%%%%%%%%%%%%%%%%%%%%%%%%%%%%%%%%%%%%%%%%%%%%%%%%%%%%%%%%%%%%%%%%%%%%
\subsection{Files and Installation}

The package consists of the files:
%
\begin{center}
\begin{tabular}{ll}
    |README.txt|   & readme file \\
    |childdoc.ins| & installation file \\
    |childdoc.dtx| & source file \\
    |childdoc.def| & definition file \\
    |cdocsamp.tex| & sample main file \\
    |cdocsch1.tex| & sample include file \\
    |cdocsch2.tex| & sample include file \\
    |cdocspt3.tex| & sample part file \\
    |cdocspt4.tex| & sample part file \\
    |cdocsdrf.tex| & sample redirection file \\
    |cdocsfn1.tex| & sample redirection file \\
    |cdocsfn2.tex| & sample redirection file \\
    |childdoc.pdf| & manual
\end{tabular}
\end{center}
%
The distribution consists of the files
|README.txt|, |childdoc.ins| and |childdoc.dtx|.
%
\begin{itemize}
\item
Run (pdf)\LaTeX{} on |childdoc.dtx|
to compile the manual |childdoc.pdf| (this file).
\item
Run \LaTeX{} on |childdoc.ins| to create the definitions file |childdoc.def|
and the sample |cdocsamp.tex| with include files
|cdocsch1.tex|, |cdocsch2.tex|, |cdocspt3.tex|, |cdocspt4.tex|,
|cdocsdrf.tex|, |cdocsfn1.tex|, |cdocsfn2.tex|.
Then copy the file |childdoc.def| to an appropriate directory of your \LaTeX{}
distribution, e.g.\ \textit{texmf-root}|/tex/latex/childdoc|.
\end{itemize}

%%%%%%%%%%%%%%%%%%%%%%%%%%%%%%%%%%%%%%%%%%%%%%%%%%%%%%%%%%%%%%%%%%%%%%%%%%%%%%%%
\subsection{Related CTAN Packages}

There are several other packages which offer a similar functionality:
%
\begin{itemize}
\item
The packages
\href{http://ctan.org/pkg/docmute}{\textsf{docmute}},
\href{http://ctan.org/pkg/includex}{\textsf{includex}} and
\href{http://ctan.org/pkg/standalone}{\textsf{standalone}}
provide commands to include only the document body of
a child file thus allowing both files to be compiled individually.
\item
The packages \href{http://ctan.org/pkg/subdocs}{\textsf{subdocs}}
and \href{http://ctan.org/pkg/subfiles}{\textsf{subfiles}}
provide structures in which the main and child documents can be
encapsulated and allowing them to be compiled individually.
The inclusion mechanism is different from the conventional |\include|.
\item
The package \href{http://ctan.org/pkg/combine}{\textsf{combine}}
is an elaborate solution to combine several documents into one.
\end{itemize}
%
See also the CTAN topic \href{http://ctan.org/topic/subdocs}{\textsf{subdocs}}
for further related packages.
The present package differs from the above solutions in that
a document structure constructed with the conventional |\include| mechanism
just needs two extra commands at the top of every file
such that all constituent files can be compiled individually.

%%%%%%%%%%%%%%%%%%%%%%%%%%%%%%%%%%%%%%%%%%%%%%%%%%%%%%%%%%%%%%%%%%%%%%%%%%%%%%%%
%\subsection{Feature Suggestions}
%
%The following is a list of features which may be useful for future
%versions of this package:
%%
%\begin{itemize}
%\item
%\ldots
%\end{itemize}

%%%%%%%%%%%%%%%%%%%%%%%%%%%%%%%%%%%%%%%%%%%%%%%%%%%%%%%%%%%%%%%%%%%%%%%%%%%%%%%%
\subsection{Revision History}

%%%%%%%%%%%%%%%%%%%%%%%%%%%%%%%%%%%%%%%%
\paragraph{v2.0:} 2018/12/30

\begin{itemize}
\item
immediate forward processing
\item
added |\childdocby| mechanism
\item
manual restructured
\end{itemize}

%%%%%%%%%%%%%%%%%%%%%%%%%%%%%%%%%%%%%%%%
\paragraph{v1.6:} 2018/01/17

\begin{itemize}
\item
application for development of include files
\item
corrections to manual
\end{itemize}

%%%%%%%%%%%%%%%%%%%%%%%%%%%%%%%%%%%%%%%%
\paragraph{v1.5:} 2017/05/21

\begin{itemize}
\item
more complete structuring introduced
\item
|\childdocof| introduced
\item
|\childdoc| renamed to |\childdocmain|
\item
|\childredirect| renamed to |\childdocforward| and |\childdocforwardprefix|
and functionality expanded
\end{itemize}

%%%%%%%%%%%%%%%%%%%%%%%%%%%%%%%%%%%%%%%%
\paragraph{v1.0:} 2017/04/27

\begin{itemize}
\item
manual and install package
\item
first version published on CTAN
\end{itemize}

%%%%%%%%%%%%%%%%%%%%%%%%%%%%%%%%%%%%%%%%
\paragraph{v0.6:} 2017/04/26

\begin{itemize}
\item
redirection mechanism added
\end{itemize}

%%%%%%%%%%%%%%%%%%%%%%%%%%%%%%%%%%%%%%%%
\paragraph{v0.5:} 2017/04/26

\begin{itemize}
\item
functionality in definition file
\end{itemize}


%%%%%%%%%%%%%%%%%%%%%%%%%%%%%%%%%%%%%%%%%%%%%%%%%%%%%%%%%%%%%%%%%%%%%%%%%%%%%%%%
%%%%%%%%%%%%%%%%%%%%%%%%%%%%%%%%%%%%%%%%%%%%%%%%%%%%%%%%%%%%%%%%%%%%%%%%%%%%%%%%
%%%%%%%%%%%%%%%%%%%%%%%%%%%%%%%%%%%%%%%%%%%%%%%%%%%%%%%%%%%%%%%%%%%%%%%%%%%%%%%%
\appendix

\settowidth\MacroIndent{\rmfamily\scriptsize 000\ }

 \DocInput{childdoc.dtx}

\end{document}
%</driver>
% \fi
%
% %%%%%%%%%%%%%%%%%%%%%%%%%%%%%%%%%%%%%%%%%%%%%%%%%%%%%%%%%%%%%%%%%%%%%%%%%%%%%%
% %%%%%%%%%%%%%%%%%%%%%%%%%%%%%%%%%%%%%%%%%%%%%%%%%%%%%%%%%%%%%%%%%%%%%%%%%%%%%%
% \section{Sample}
%\iffalse
%<*samplemain>
%\fi
%
% The following presents a sample document
% with two chapters, two parts, a title page,
% a compile flag as well as three forwarding files to set the flag.
% It consists of eight |.tex| files:
% \begin{center}
% \begin{tabular}{ll}
% |cdocsamp.tex|&main file\\
% |cdocsch1.tex|&include file for chapter 1\\
% |cdocsch2.tex|&include file for chapter 2\\
% |cdocspt3.tex|&include file for part 3\\
% |cdocspt4.tex|&include file for part 4\\
% |cdocsdrf.tex|&forwarding file for main file in draft mode\\
% |cdocsfi1.tex|&forwarding file for final version of chapter 1\\
% |cdocsfi2.tex|&forwarding file for final version of chapter 2\\
% \end{tabular}
% \end{center}
% Each of the eight files can be compiled directly by the \LaTeX{} compiler.
%
% %%%%%%%%%%%%%%%%%%%%%%%%%%%%%%%%%%%%%%
% \paragraph{Main File.}
%
% The main file is called |cdocsamp.tex|.
%
% Load the \textsf{childdoc} definitions and
% declare the filename for the main document:
%    \begin{macrocode}
\input{childdoc.def}
\childdocmain{}
%    \end{macrocode}

% Optional override for |\version| flag:
%    \begin{macrocode}
%%\ifchilddoc\else\providecommand{\version}{draft}\fi
%    \end{macrocode}

% Define the default values for the |\version| flag
% (|final| for the main file and |draft| for childs):
%    \begin{macrocode}
\ifchilddoc
\providecommand{\version}{draft}
\else
\providecommand{\version}{final}
\fi
%    \end{macrocode}

% Load the standard document class:
%    \begin{macrocode}
\documentclass[12pt]{article}
%    \end{macrocode}

% Start the document body:
%    \begin{macrocode}
\begin{document}
%    \end{macrocode}

% Declare a title page.
% Print title, part of document being processed and version flag:
%    \begin{macrocode}
\addtocounter{page}{-1}
\begin{center}
{\LARGE\bfseries{}childdoc example\par}
\vspace{1cm}
\ifchilddoc
\ifchilddocmanual part\else chapter\fi:
`\childdocname' of `\childdocjob'\par
\else
main document: `\childdocjob'\par
\fi
version: \version\par
\end{center}
\newpage
%    \end{macrocode}

% Manually include selected file,
% otherwise process as usual:
%    \begin{macrocode}
\ifchilddocmanual
\section*{part `\childdocname'}
\input{\childdocname}
\else
%    \end{macrocode}

% Include the two chapters:
%    \begin{macrocode}
\include{cdocsch1}
\include{cdocsch2}
%    \end{macrocode}

% Include the two parts unless only chapters should be displayed:
%    \begin{macrocode}
\ifchilddoc\else
\section{part three}
\input{cdocspt3}
\section{part four}
\input{cdocspt4}
\fi
%    \end{macrocode}

% Process as usual until here:
%    \begin{macrocode}
\fi
%    \end{macrocode}

% End of document body:
%    \begin{macrocode}
\end{document}
%    \end{macrocode}
%\iffalse
%</samplemain>
%\fi
%
% %%%%%%%%%%%%%%%%%%%%%%%%%%%%%%%%%%%%%%
% \paragraph{Chapter Include Files.}
%
% The include files are called |cdocsch1.tex| and |cdocsch2.tex|.
%
%\iffalse
%<*samplechap1|samplechap2>
%\fi

% Optional override for |\version| flag:
%    \begin{macrocode}
%%\providecommand{\version}{final}
%    \end{macrocode}

% Include the main document:
%    \begin{macrocode}
\input{childdoc.def}
\childdocof{cdocsamp}
%    \end{macrocode}

%\iffalse
%</samplechap1|samplechap2>
%\fi
%
%\iffalse
%<*samplechap1>
%\fi
% Some text for chapter 1:
%    \begin{macrocode}
\section{one}
some text in chapter one
%    \end{macrocode}

%\iffalse
%</samplechap1>
%\fi
% Some text for chapter 2:
%\iffalse
%<*samplechap2>
%\fi
%    \begin{macrocode}
\section{two}
more text in chapter two
%    \end{macrocode}

%\iffalse
%</samplechap2>
%\fi
%
% %%%%%%%%%%%%%%%%%%%%%%%%%%%%%%%%%%%%%%
% \paragraph{Part Include Files.}
%
% The include files are called |cdocspt3.tex| and |cdocspt4.tex|.
%
%\iffalse
%<*samplepart3|samplepart4>
%\fi

% Optional override for |\version| flag:
%    \begin{macrocode}
%%\providecommand{\version}{final}
%    \end{macrocode}

% Include the main document:
%    \begin{macrocode}
\input{childdoc.def}
\childdocby{cdocsamp}
%    \end{macrocode}

%\iffalse
%</samplepart3|samplepart4>
%\fi
%
%\iffalse
%<*samplepart3>
%\fi
% Some text for part 3:
%    \begin{macrocode}
some text in part three
%    \end{macrocode}

%\iffalse
%</samplepart3>
%\fi
% Some text for part 4:
%\iffalse
%<*samplepart4>
%\fi
%    \begin{macrocode}
more text in part four
%    \end{macrocode}

%\iffalse
%</samplepart4>
%\fi
%
% %%%%%%%%%%%%%%%%%%%%%%%%%%%%%%%%%%%%%%
% \paragraph{Forwarding for a Complete Draft.}
%
% The following forwarding file |cdocsdrf.tex|
% compiles the main document in draft mode:
%\iffalse
%<*sampledraft>
%\fi
%    \begin{macrocode}
\def\version{draft}
\input{childdoc.def}
\childdocforward{cdocsamp}
%    \end{macrocode}

%\iffalse
%</sampledraft>
%\fi
%
% %%%%%%%%%%%%%%%%%%%%%%%%%%%%%%%%%%%%%%
% \paragraph{Forwarding for Final Version of the Chapters.}
%
% The following forwarding files |cdocsfn1.tex| and |cdocsfn2.tex|
% (with identical content)
% compile the final versions of the child documents
% |cdocsch1.tex| and |cdocsch2.tex|, respectively:
%\iffalse
%<*samplefinal>
%\fi
%    \begin{macrocode}
\def\version{final}
\input{childdoc.def}
\childdocforwardprefix[cdocsamp]{cdocsfn}{cdocsch}
%    \end{macrocode}

%\iffalse
%</samplefinal>
%\fi
%
% %%%%%%%%%%%%%%%%%%%%%%%%%%%%%%%%%%%%%%
% \paragraph{Command Line Processing.}
%
% The following three command lines generate the output files
% |cdocscld|, |cdocscl1| and |cdocscl2|
% which should be identical to
% |cdocsdrf|, |cdocsch1| and |cdocsfn2|, respectively:
% \begin{center}
% \begin{tabular}{l}
% |latex -jobname cdocscld \|\\
% |  "\def\version{draft}\input{childdoc.def}\childdocforward{cdocsamp}"|\\
% |latex -jobname cdocscl1 \|\\
% |  "\input{childdoc.def}\childdocforward[cdocsamp]{cdocsch1}"|\\
% |latex -jobname cdocscl2 \|\\
% |  "\def\version{final}\input{childdoc.def}\childdocforward{cdocsch2}"|
% \end{tabular}
% \end{center}
% Note that the trailing backslash on each first line
% merely continues the input to the second line
% (for convenient cut ant paste).
% Furthermore, the command |latex| can be replaced by any
% of its alternative versions such as |pdflatex|.
%
% %%%%%%%%%%%%%%%%%%%%%%%%%%%%%%%%%%%%%%%%%%%%%%%%%%%%%%%%%%%%%%%%%%%%%%%%%%%%%%
% %%%%%%%%%%%%%%%%%%%%%%%%%%%%%%%%%%%%%%%%%%%%%%%%%%%%%%%%%%%%%%%%%%%%%%%%%%%%%%
% \section{Implementation}
%\iffalse
%<*package>
%\fi
%
% This section describes the definitions file |childdoc.def|.

% The definitions cannot be loaded using |\usepackage| or |\RequirePackage|
% which has a mechanism to prevent loading a style file more than once.
% When loading the definitions by means of |\input|
% multiple instances have to be prevented manually:
%\iffalse
%This code needs to be before the `\ProvidesFile' directive
%which is defined at the beginning of this file.
%Therefore it is also placed there and commented out here.
%</package>
%<*discard>
%\fi
%    \begin{macrocode}
\ifdefined\childdocmain\endinput\fi
%    \end{macrocode}
%\iffalse
%</discard>
%<*package>
%\fi
%
% \macro{\ifchilddoc}
% \macro{\ifchilddocmanual}
% The conditional |\ifchilddoc| tells whether a
% child (true) or main (false) document is being compiled.
% The conditional |\ifchilddocmanual| tells whether
% the |\includeonly| mechanism is used (false) or
% the selection of child files must be performed manually (true).
% The definitions initialise to false:
%    \begin{macrocode}
\newif\ifchilddoc
\newif\ifchilddocmanual
%    \end{macrocode}

% \macro{\childdocname}
% \macro{\childdocjob}
% The macro |\childdocname| stores the name of the main document
% to be compiled. The macro |\childdocjob| stores the name of
% the document on which the \LaTeX{} compiler was originally invoked.
% The content of |\jobname| cannot be compared
% to filenames specified in the source due to different catcodes.
% The following code rescans |\jobname|, stores the result
% in |\childdocname| and saves a copy in |\childdocjob|:
%    \begin{macrocode}
\edef\childdocname{\scantokens\expandafter{\jobname\noexpand}}
\let\childdocjob\childdocname
%    \end{macrocode}

% \macro{\childdocdisable}
% The macro |\childdocdisable| prevents the main file
% from being processed more than once.
% At this stage, the main document command |\childdocmain|
% is assumed to be called once again where it should do nothing.
% Any subsequent call to it should prevent
% a secondary processing of the main document
% It overwrites the forwarding commands
% |\childdocof| and |\childdocforward|
% with empty macros to prevent further inclusions of the main document:
%    \begin{macrocode}
\newcommand{\childdocdisable}
{
  \renewcommand{\childdocmain}[1]{\renewcommand{\childdocmain}[1]{\endinput}}
  \renewcommand{\childdocof}[1]{}
  \renewcommand{\childdocby}[2][]{}
  \renewcommand{\childdocforward}[2][]{}
  \renewcommand{\childdocdisable}{}
}
%    \end{macrocode}

% \macro{\childdocmain}
% The macro |\childdocmain| is to be called at the top of the main file
% with nothing or the main filename (without extension) as argument.
% First, it breaks loops.
% If the argument is not empty and does not match |\childdocname|
% (which is set by the first inclusion of |childdoc.def|),
% |\ifchilddoc| is set to true, |\includeonly| is applied to the child file
% and |\jobname| is set to the main file
% (for proper handling of |.aux| files):
%    \begin{macrocode}
\newcommand{\childdocmain}[1]
{
  \childdocdisable\childdocmain{}
  \if?#1?\else
    \begingroup
      \def\childdoctmp{#1}
      \ifx\childdoctmp\childdocname
        \def\childdoctmp{}
      \else
        \def\childdoctmp
        {
          \childdoctrue
          \includeonly{\childdocname}
          \def\childdocjob{#1}
          \def\jobname{#1}
        }
      \fi
      \expandafter
    \endgroup
    \childdoctmp
  \fi
}
%    \end{macrocode}

% \macro{\childdocof}
% The command |\childdocof| redirects
% compilation to the main file |#1|.
%    \begin{macrocode}
\newcommand{\childdocof}[1]
{
  \childdocdisable
  \childdoctrue
  \includeonly{\childdocname}
  \def\jobname{#1}
  \def\childdocjob{#1}
  \input{#1}
}
%    \end{macrocode}

% \macro{\childdocby}
% The command |\childdocby| ....
%    \begin{macrocode}
\newcommand{\childdocby}[2][]
{
  \childdocdisable
  \childdoctrue
  \childdocmanualtrue
  \if?#1?\else
    \def\jobname{#2}
  \fi
  \def\childdocjob{#2}
  \input{#2}
  \endinput
}
%    \end{macrocode}

% \macro{\childdocforward}
% The command |\childdocforward| redirects
% compilation to the main file or
% (if the optional argument is given) a child file.
% Parameters are set as if the main file
% or a child file starting with |\childdocof| was compiled.
% Then compilation is handed over to the main file:
%    \begin{macrocode}
\newcommand{\childdocforward}[2][]
{
  \begingroup
    \if?#1?
      \def\childdoctmp
      {
        \def\childdocname{#2}
        \def\childdocjob{#2}
        \def\jobname{#2}
        \input{#2}
        \endinput
      }
    \else
      \def\childdoctmp
      {
        \childdocdisable
        \def\childdocname{#2}
        \childdoctrue
        \includeonly{#2}
        \def\childdocjob{#1}
        \def\jobname{#1}
        \input{#1}
        \endinput
      }
    \fi
    \expandafter
  \endgroup
  \childdoctmp
}
%    \end{macrocode}

% \macro{\childdocforwardprefix}
% The command |\childdocforwardprefix| redirects
% compilation to the main or a child file by means of a pattern.
% The prefix |#1| in the current filename is replaced by |#2|
% and the suffix of the current filename is kept
% (it is assumed that the filename does not contain the substring `|~~~|'
% which is used as a delimiter).
% Compilation is handed over to the new file by |\childdocforward|:
%    \begin{macrocode}
\newcommand{\childdocforwardprefix}[3][]
{
  \begingroup
    \def\childdocextract #2##1~~~{\def\childdoctmp{\childdocforward[#1]{#3##1}}}
    \expandafter\childdocextract\childdocname~~~
    \expandafter
  \endgroup
  \childdoctmp
}
%    \end{macrocode}

% \macro{\childdoc}
% The deprecated macro |\childdoc| is a legacy version of |\childdocmain|:
%    \begin{macrocode}
\newcommand{\childdoc}{\childdocmain}
%    \end{macrocode}

% \macro{\childdocredirect}
% The deprecated macro |\childdocredirect| is a legacy version
% of |\childdocforward| and |\childdocforwardprefix|:
%    \begin{macrocode}
\newcommand{\childdocredirect}[2][]
{
  \begingroup
    \if?#1?
      \def\childdoctmp{\childdocforward{#2}}
    \else
      \def\childdoctmp{\childdocforwardprefix{#1}{#2}}
    \fi
    \expandafter
  \endgroup
  \childdoctmp
}
%    \end{macrocode}

%\iffalse
%</package>
%\fi
%
\endinput
|\\
|\childdocof{|\textit{main}|}|\\
\end{tabular}
\end{center}
at the top of every child file \textit{child}
which is included by |\include{|\textit{child}|}|
from within the main file
(or at least for those files to be compiled individually).
The argument \textit{main} must be the filename of the main file.

There are a couple of
considerations in setting up the main and child documents:

%%%%%%%%%%%%%%%%%%%%%%%%%%%%%%%%%%%%%%%%
\paragraph{Restrictions.}

Please note the following restrictions:
\begin{itemize}
\item
|\childdocmain| must be called with one argument \textit{main}
to ensure compatibility with earlier version of the package.
It must either be empty (|\childdocmain{}|)
or precisely match the filename of the main file in which it is specified.
See \secref{sec:detection} for further information.
\item
The filename \textit{main} must be specified without the |.tex| extension.
\item
The filename \textit{main} is case sensitive
(even in case-insensitive file systems)
due to internal string comparison.
\item
The argument \textit{main} should be fully expanded, it cannot be a macro.
\item
Subdirectories and special characters should be avoided in filenames.
\item
The command |\childdocmain{|\textit{main}|}| must be followed by a whitespace.
It should not be followed immediately by another command
or by a comment mark `|%|'.
This is because the \TeX{} parser reads the token immediately following
the argument of |\childdocmain| and puts it
at the beginning of every child section;
however, a white\-space is ignored.
\end{itemize}

%%%%%%%%%%%%%%%%%%%%%%%%%%%%%%%%%%%%%%%%
\paragraph{Content of Main File.}

It is advisable to place all content in the child files included by |\include|.
Any output contained in the main file will appear in all child documents
unless suppressed manually;
it cannot be suppressed automatically by the |\includeonly| directive
and thus should normally be avoided.
A method to include some content in the main file
by means of conditional processing is described in \secref{sec:conditional}.

%%%%%%%%%%%%%%%%%%%%%%%%%%%%%%%%%%%%%%%%
\paragraph{Page Numbering.}

When only a part of the document is compiled,
the appropriate numbering of pages
(as well as other status parameters)
is determined from the |.aux| files.
The latter contain information from previous passes.
However this information needs to propagate through
all intermediate child documents.
Therefore the page numbering in child documents may well
be inconsistent until the complete document is compiled at least once.

A useful (if unconventional) way to always ensure a consistent
page numbering is to restart the numbering in each child document
and denote the pages by `\textit{child}|.|\textit{page}'
where \textit{child} represents the chapter/section number of the child file.
This can be achieved by the command
|\numberwithin{page}{|\textit{child}|}|
of the \textsf{amsmath} package
where \textit{child} can be |chapter| or |section|
depending on the chosen structuring.
Alternatively, one can modify the macro |\thepage| appropriately
and reset the counter |page| at the start of each child file.

%%%%%%%%%%%%%%%%%%%%%%%%%%%%%%%%%%%%%%%%%%%%%%%%%%%%%%%%%%%%%%%%%%%%%%%%%%%%%%%%
\subsection{Conditional Processing}
\label{sec:conditional}

The package provides a mechanism to compile different versions
of a document. To customise the versions further some conditional processing
can come in handy to distinguish which version is being compiled.
The package provides two macros to describe the compilation context:

%%%%%%%%%%%%%%%%%%%%%%%%%%%%%%%%%%%%%%%%
\DescribeMacro{\ifchilddoc}
The conditional |\ifchilddoc| distinguishes between the compilation of
child documents and the main document:
%
\begin{center}
|\ifchilddoc |\textit{child-code}| |[|\||else |\textit{main-code}]| \||fi|
\end{center}

%%%%%%%%%%%%%%%%%%%%%%%%%%%%%%%%%%%%%%%%
\DescribeMacro{\childdocname}
\DescribeMacro{\childdocjob}
The macro |\childdocname| contains the filename (without extension)
of the main or child file being processed.
Note that |\childdocjob| will always contain the name of the main file.

%%%%%%%%%%%%%%%%%%%%%%%%%%%%%%%%%%%%%%%%
\paragraph{Title Page.}

Conditional processing can be used to include a title or banner page
in the main document when proper precautions are taken.
Importantly, the code in the main file should ensure that the page counter
(as well as other status parameters which are stored in the |.aux| files)
takes the same value after the conditional processing.
Otherwise the page numbers may take divergent values
depending on which part is compiled.

For example, a title page could be declared by:
%
\begin{center}
\begin{tabular}{l}
|\ifchilddoc\||else|\\
|\addtocounter{page}{-1}|\\
\textit{code for title page}\\
|\newpage|\\
|\||fi|
\end{tabular}
\end{center}
%
A banner page for the child documents can be generated by:
%
\begin{center}
\begin{tabular}{l}
|\ifchilddoc|\\
|\addtocounter{page}{-1}|\\
\textit{code for banner page}\\
|\newpage|\\
|\||fi|
\end{tabular}
\end{center}
%
Here one could write a message such as:
\begin{center}
|This is the part \childdocname{} of \childdocjob{}.|
\end{center}

%%%%%%%%%%%%%%%%%%%%%%%%%%%%%%%%%%%%%%%%%%%%%%%%%%%%%%%%%%%%%%%%%%%%%%%%%%%%%%%%
\subsection{Flags}
\label{sec:flags}

The package makes it easy to generate different versions
of the main or child documents.
To this end compilation flags can be defined
and assigned different default values.
They will be particularly useful in conjunction
with the forwarding mechanism described in \secref{sec:forward}.

For example, it may be useful to have a flag |\version|
which can be set to |draft| or |final|.
The document source will contain some conditional code
depending on the value of |\version|.
Suppose further, the flag should default to |final| for the main file
and to |draft| for child files
which is a natural assignment for editing the document.
This is achieved by placing the following code
in the preamble of the main document
(below the |\childdocmain| directive):
%
\begin{center}
\begin{tabular}{l}
|\ifchilddoc|\\
|\providecommand{\version}{draft}|\\
|\||else|\\
|\providecommand{\version}{final}|\\
|\||fi|
\end{tabular}
\end{center}
%
The definition by |\providecommand| makes sure
that previous definitions are not overwritten.
Further statements |\providecommand{\version}{...}|
can thus be added before the above code to override it.

For the main file, one might add a line
(between |\childdocmain| and the above block)
%
\begin{center}
|%\ifchilddoc\||else\providecommand{\version}{draft}\||fi|
\end{center}
%
which can be uncommented to produce a draft version.
Likewise one can add a line to the very top of a child file
(above the |\childdocof{|\textit{main}|}| directive)
%
\begin{center}
|%\providecommand{\version}{final}|
\end{center}
%
which can be uncommented to produce the final version of this child document.

%%%%%%%%%%%%%%%%%%%%%%%%%%%%%%%%%%%%%%%%%%%%%%%%%%%%%%%%%%%%%%%%%%%%%%%%%%%%%%%%
\subsection{Forwarding}
\label{sec:forward}

Different versions of the main or child documents
using compilation flags as described in \secref{sec:flags}
can be (permanently) stored in different files
for convenient compilation, viewing and distribution.
To this end, the package defines a command
to pass on compilation to a different file:

%%%%%%%%%%%%%%%%%%%%%%%%%%%%%%%%%%%%%%%%
\DescribeMacro{\childdocforward}
The command |\childdocforward| redirects processing to
another source file:
%
\begin{center}
\begin{tabular}{l}
|% \iffalse
%
% childdoc.dtx Copyright (C) 2017-2018 Niklas Beisert
%
% This work may be distributed and/or modified under the
% conditions of the LaTeX Project Public License, either version 1.3
% of this license or (at your option) any later version.
% The latest version of this license is in
%   http://www.latex-project.org/lppl.txt
% and version 1.3 or later is part of all distributions of LaTeX
% version 2005/12/01 or later.
%
% This work has the LPPL maintenance status `maintained'.
%
% The Current Maintainer of this work is Niklas Beisert.
%
% This work consists of the files childdoc.dtx and childdoc.ins
% and the derived files childdoc.def and cdocsamp.tex with
% cdocsch1.tex, cdocsch2.tex, cdocsdrf.tex, cdocsfn1.tex, cdocsfn2.tex.
%
%<package>\ifdefined\childdocmain\endinput\fi
%<package>\ProvidesFile{childdoc.def}[2018/12/30 v2.0 child document driver]
%<samplemain>\ProvidesFile{cdocsamp.tex}[2018/12/30 v2.0 sample for childdoc]
%<*driver>
%\ProvidesFile{childdoc.drv}[2018/12/30 v2.0 childdoc reference manual file]
\PassOptionsToClass{10pt,a4paper}{article}
\documentclass{ltxdoc}

\usepackage[margin=35mm]{geometry}
\usepackage{hyperref}
\usepackage{hyperxmp}
\usepackage[usenames]{color}

\hypersetup{colorlinks=true}
\hypersetup{pdfstartview=FitH}
\hypersetup{pdfpagemode=UseNone}
\hypersetup{pdfsource={}}
\hypersetup{pdflang={en-UK}}
\hypersetup{pdfcopyright={Copyright 2017-2018 Niklas Beisert.
  This work may be distributed and/or modified under the
  conditions of the LaTeX Project Public License, either version 1.3
  of this license or (at your option) any later version.}}
\hypersetup{pdflicenseurl={http://www.latex-project.org/lppl.txt}}
\hypersetup{pdfcontactaddress={ETH Zurich, ITP, HIT K,
  Wolfgang-Pauli-Strasse 27}}
\hypersetup{pdfcontactpostcode={8093}}
\hypersetup{pdfcontactcity={Zurich}}
\hypersetup{pdfcontactcountry={Switzerland}}
\hypersetup{pdfcontactemail={nbeisert@itp.phys.ethz.ch}}
\hypersetup{pdfcontacturl={http://people.phys.ethz.ch/\xmptilde nbeisert/}}

\newcommand{\secref}[1]{\hyperref[#1]{section \ref*{#1}}}

\parskip1ex
\parindent0pt
\let\olditemize\itemize
\def\itemize{\olditemize\parskip0pt}

\begin{document}

\title{The \textsf{childdoc} Package}
\hypersetup{pdftitle={The childdoc Package}}
\author{Niklas Beisert\\[2ex]
  Institut f\"ur Theoretische Physik\\
  Eidgen\"ossische Technische Hochschule Z\"urich\\
  Wolfgang-Pauli-Strasse 27, 8093 Z\"urich, Switzerland\\[1ex]
  \href{mailto:nbeisert@itp.phys.ethz.ch}
  {\texttt{nbeisert@itp.phys.ethz.ch}}}
\hypersetup{pdfauthor={Niklas Beisert}}
\hypersetup{pdfsubject={Manual for the LaTeX2e Package childdoc}}
\date{30 December 2018, \textsf{v2.0}}
\maketitle

\begin{abstract}\noindent
\textsf{childdoc} is a \LaTeXe{} package
that enables the direct compilation
of document sections included by |\include|
to individual files.
\end{abstract}

\begingroup
\parskip0ex
\tableofcontents
\endgroup

%%%%%%%%%%%%%%%%%%%%%%%%%%%%%%%%%%%%%%%%%%%%%%%%%%%%%%%%%%%%%%%%%%%%%%%%%%%%%%%%
%%%%%%%%%%%%%%%%%%%%%%%%%%%%%%%%%%%%%%%%%%%%%%%%%%%%%%%%%%%%%%%%%%%%%%%%%%%%%%%%
\section{Introduction}

\LaTeX{} provides a mechanism to structure a large document (such as a book)
into a main file and several child files (containing the chapters)
using the |\include| command.
This mechanism is beneficial for documents
which span hundreds of pages in order to
make the source file(s) more manageable.
Moreover, compilation can be restricted to
selected child files by means of the |\includeonly| command.
The latter feature can be used to reduce the compilation time while editing
(this was significantly more useful in the earlier days of \LaTeX{})
or to generate a smaller document which is easier to navigate.
Another application of |\includeonly| is to generate
documents consisting of selected parts of the complete document.

However, there are a few drawbacks of the plain |\include| mechanism:
\begin{itemize}
\item
The child files cannot be compiled on their own,
they can only be compiled via the main file.
A naive editing environment
(such as a text editor with an option
to have the current file processed by \LaTeX)
may require one to switch to the main file before compiling;
attempting to compile the child file produces errors.
\item
The main file must be modified (each time)
to adjust the |\includeonly| command
to the present needs. This easily leaves the main file in a messy state.
\item
The generated document will always carry the filename
of the main document. This is inconvenient if
several child files are to be compiled and
to be kept for distribution.
\end{itemize}

The present package provides a simple interface
to make child files individually compilable by \LaTeX{}.
Compiling a child file then has the same effect as compiling
the main file with an |\includeonly| command
to select the appropriate child.
Moreover the generated document will carry the name of the child
rather than the main file.
This resolves all three above issues.

This feature is meant to make the editing of books,
thesis documents and lecture notes somewhat more convenient.
However, the package can also be used efficiently for
composing a series of documents (such as exercise sheets)
which are typically distributed individually.
It then assists the author in generating the individual documents
(potentially in different versions)
as well as a document containing the collected series.
Another application is in developing style files
or other kinds of included material
where compilation of the style file could redirect
to a sample or test file.

%%%%%%%%%%%%%%%%%%%%%%%%%%%%%%%%%%%%%%%%%%%%%%%%%%%%%%%%%%%%%%%%%%%%%%%%%%%%%%%%
%%%%%%%%%%%%%%%%%%%%%%%%%%%%%%%%%%%%%%%%%%%%%%%%%%%%%%%%%%%%%%%%%%%%%%%%%%%%%%%%
\section{Usage}

First of all, the package \textsf{childdoc} is \emph{not} a standard
\LaTeXe{} |.sty| style file! Therefore it needs to be invoked in
a non-standard way.

%%%%%%%%%%%%%%%%%%%%%%%%%%%%%%%%%%%%%%%%%%%%%%%%%%%%%%%%%%%%%%%%%%%%%%%%%%%%%%%%
\subsection{Included Files}
\label{sec:include}

%%%%%%%%%%%%%%%%%%%%%%%%%%%%%%%%%%%%%%%%
\DescribeMacro{\childdocmain}
To use the package, add the commands
\begin{center}
\begin{tabular}{l}
|\input{childdoc.def}|\\
|\childdocmain{}|\\
\end{tabular}
\end{center}
at the very top of the main \LaTeX{} file,
in particular \emph{before} the |\documentclass| statement!
The argument of |\childdocmain| should be left empty
(but it must be present).

%%%%%%%%%%%%%%%%%%%%%%%%%%%%%%%%%%%%%%%%
\DescribeMacro{\childdocof}
Furthermore, add the commands
\begin{center}
\begin{tabular}{l}
|\input{childdoc.def}|\\
|\childdocof{|\textit{main}|}|\\
\end{tabular}
\end{center}
at the top of every child file \textit{child}
which is included by |\include{|\textit{child}|}|
from within the main file
(or at least for those files to be compiled individually).
The argument \textit{main} must be the filename of the main file.

There are a couple of
considerations in setting up the main and child documents:

%%%%%%%%%%%%%%%%%%%%%%%%%%%%%%%%%%%%%%%%
\paragraph{Restrictions.}

Please note the following restrictions:
\begin{itemize}
\item
|\childdocmain| must be called with one argument \textit{main}
to ensure compatibility with earlier version of the package.
It must either be empty (|\childdocmain{}|)
or precisely match the filename of the main file in which it is specified.
See \secref{sec:detection} for further information.
\item
The filename \textit{main} must be specified without the |.tex| extension.
\item
The filename \textit{main} is case sensitive
(even in case-insensitive file systems)
due to internal string comparison.
\item
The argument \textit{main} should be fully expanded, it cannot be a macro.
\item
Subdirectories and special characters should be avoided in filenames.
\item
The command |\childdocmain{|\textit{main}|}| must be followed by a whitespace.
It should not be followed immediately by another command
or by a comment mark `|%|'.
This is because the \TeX{} parser reads the token immediately following
the argument of |\childdocmain| and puts it
at the beginning of every child section;
however, a white\-space is ignored.
\end{itemize}

%%%%%%%%%%%%%%%%%%%%%%%%%%%%%%%%%%%%%%%%
\paragraph{Content of Main File.}

It is advisable to place all content in the child files included by |\include|.
Any output contained in the main file will appear in all child documents
unless suppressed manually;
it cannot be suppressed automatically by the |\includeonly| directive
and thus should normally be avoided.
A method to include some content in the main file
by means of conditional processing is described in \secref{sec:conditional}.

%%%%%%%%%%%%%%%%%%%%%%%%%%%%%%%%%%%%%%%%
\paragraph{Page Numbering.}

When only a part of the document is compiled,
the appropriate numbering of pages
(as well as other status parameters)
is determined from the |.aux| files.
The latter contain information from previous passes.
However this information needs to propagate through
all intermediate child documents.
Therefore the page numbering in child documents may well
be inconsistent until the complete document is compiled at least once.

A useful (if unconventional) way to always ensure a consistent
page numbering is to restart the numbering in each child document
and denote the pages by `\textit{child}|.|\textit{page}'
where \textit{child} represents the chapter/section number of the child file.
This can be achieved by the command
|\numberwithin{page}{|\textit{child}|}|
of the \textsf{amsmath} package
where \textit{child} can be |chapter| or |section|
depending on the chosen structuring.
Alternatively, one can modify the macro |\thepage| appropriately
and reset the counter |page| at the start of each child file.

%%%%%%%%%%%%%%%%%%%%%%%%%%%%%%%%%%%%%%%%%%%%%%%%%%%%%%%%%%%%%%%%%%%%%%%%%%%%%%%%
\subsection{Conditional Processing}
\label{sec:conditional}

The package provides a mechanism to compile different versions
of a document. To customise the versions further some conditional processing
can come in handy to distinguish which version is being compiled.
The package provides two macros to describe the compilation context:

%%%%%%%%%%%%%%%%%%%%%%%%%%%%%%%%%%%%%%%%
\DescribeMacro{\ifchilddoc}
The conditional |\ifchilddoc| distinguishes between the compilation of
child documents and the main document:
%
\begin{center}
|\ifchilddoc |\textit{child-code}| |[|\||else |\textit{main-code}]| \||fi|
\end{center}

%%%%%%%%%%%%%%%%%%%%%%%%%%%%%%%%%%%%%%%%
\DescribeMacro{\childdocname}
\DescribeMacro{\childdocjob}
The macro |\childdocname| contains the filename (without extension)
of the main or child file being processed.
Note that |\childdocjob| will always contain the name of the main file.

%%%%%%%%%%%%%%%%%%%%%%%%%%%%%%%%%%%%%%%%
\paragraph{Title Page.}

Conditional processing can be used to include a title or banner page
in the main document when proper precautions are taken.
Importantly, the code in the main file should ensure that the page counter
(as well as other status parameters which are stored in the |.aux| files)
takes the same value after the conditional processing.
Otherwise the page numbers may take divergent values
depending on which part is compiled.

For example, a title page could be declared by:
%
\begin{center}
\begin{tabular}{l}
|\ifchilddoc\||else|\\
|\addtocounter{page}{-1}|\\
\textit{code for title page}\\
|\newpage|\\
|\||fi|
\end{tabular}
\end{center}
%
A banner page for the child documents can be generated by:
%
\begin{center}
\begin{tabular}{l}
|\ifchilddoc|\\
|\addtocounter{page}{-1}|\\
\textit{code for banner page}\\
|\newpage|\\
|\||fi|
\end{tabular}
\end{center}
%
Here one could write a message such as:
\begin{center}
|This is the part \childdocname{} of \childdocjob{}.|
\end{center}

%%%%%%%%%%%%%%%%%%%%%%%%%%%%%%%%%%%%%%%%%%%%%%%%%%%%%%%%%%%%%%%%%%%%%%%%%%%%%%%%
\subsection{Flags}
\label{sec:flags}

The package makes it easy to generate different versions
of the main or child documents.
To this end compilation flags can be defined
and assigned different default values.
They will be particularly useful in conjunction
with the forwarding mechanism described in \secref{sec:forward}.

For example, it may be useful to have a flag |\version|
which can be set to |draft| or |final|.
The document source will contain some conditional code
depending on the value of |\version|.
Suppose further, the flag should default to |final| for the main file
and to |draft| for child files
which is a natural assignment for editing the document.
This is achieved by placing the following code
in the preamble of the main document
(below the |\childdocmain| directive):
%
\begin{center}
\begin{tabular}{l}
|\ifchilddoc|\\
|\providecommand{\version}{draft}|\\
|\||else|\\
|\providecommand{\version}{final}|\\
|\||fi|
\end{tabular}
\end{center}
%
The definition by |\providecommand| makes sure
that previous definitions are not overwritten.
Further statements |\providecommand{\version}{...}|
can thus be added before the above code to override it.

For the main file, one might add a line
(between |\childdocmain| and the above block)
%
\begin{center}
|%\ifchilddoc\||else\providecommand{\version}{draft}\||fi|
\end{center}
%
which can be uncommented to produce a draft version.
Likewise one can add a line to the very top of a child file
(above the |\childdocof{|\textit{main}|}| directive)
%
\begin{center}
|%\providecommand{\version}{final}|
\end{center}
%
which can be uncommented to produce the final version of this child document.

%%%%%%%%%%%%%%%%%%%%%%%%%%%%%%%%%%%%%%%%%%%%%%%%%%%%%%%%%%%%%%%%%%%%%%%%%%%%%%%%
\subsection{Forwarding}
\label{sec:forward}

Different versions of the main or child documents
using compilation flags as described in \secref{sec:flags}
can be (permanently) stored in different files
for convenient compilation, viewing and distribution.
To this end, the package defines a command
to pass on compilation to a different file:

%%%%%%%%%%%%%%%%%%%%%%%%%%%%%%%%%%%%%%%%
\DescribeMacro{\childdocforward}
The command |\childdocforward| redirects processing to
another source file:
%
\begin{center}
\begin{tabular}{l}
|\input{childdoc.def}|\\
|\childdocforward[|\textit{main}|]{|\textit{dest}|}|\\
\end{tabular}
\end{center}
%
The argument \textit{dest} is the destination file
(without extension).
It should be the main file or one of the child files.
Note that further \textsf{childdoc} directives
such as |\childdocof| and |\childdocforward|
in the indicated file will be processed in this form.
The optional argument \textit{main}
passes on directly to the main file \textit{main}
while pretending to compile the child \textit{dest}.
This form behaves as if \textit{dest}
issues |\childdocof{|\textit{main}|}| right away,
and no further \textsf{childdoc} directives will be processed.

%%%%%%%%%%%%%%%%%%%%%%%%%%%%%%%%%%%%%%%%
\DescribeMacro{\...prefix}
In the alternative form |\childdocforwardprefix|,
%
\begin{center}
\begin{tabular}{l}
|\input{childdoc.def}|\\
|\childdocforwardprefix[|\textit{main}|]{|\textit{prefix}|}{|\textit{dest}|}|
\end{tabular}
\end{center}
%
the destination file is determined by a pattern
depending on the current file:
To make this work, the current file must be called
`{\textit{prefix}\hspace{0.2em}\textit{suffix}}'
with \textit{prefix} matching precisely the argument.
Processing is then passed on to the file
`{\textit{dest}\hspace{0.2em}\textit{suffix}}'.
Surely, the same effect is achieved by
directly specifying the
argument `{\textit{dest}\hspace{0.2em}\textit{suffix}}'
in the first form.
However, that requires to set up a different file
for each child. With the alternative form of the command
all these files can have exactly the same content
which simplifies setting them up and maintaining them.

For example, the following file |draft.tex|
with a compilation flag |\version| as described in \secref{sec:flags}
compiles the main document as a draft:
%
\begin{center}
\begin{tabular}{l}
|\def\version{draft}|\\
|\input{childdoc.def}|\\
|\childdocforward{|\textit{main}|}|
\end{tabular}
\end{center}
%
Likewise, the following files |final|\textit{nn}|.tex|
compile the final version of the child document
|child|\textit{nn}|.tex|:
%
\begin{center}
\begin{tabular}{l}
|\def\version{final}|\\
|\input{childdoc.def}|\\
|\childdocforwardprefix{final}{child}|
\end{tabular}
\end{center}
%

Note that when several versions of a main file and/or of each child file
are to be generated, it may be convenient to set up a |Makefile| or
shell script to automatise the process.

%%%%%%%%%%%%%%%%%%%%%%%%%%%%%%%%%%%%%%%%%%%%%%%%%%%%%%%%%%%%%%%%%%%%%%%%%%%%%%%%
\subsection{Command Line Processing}
\label{sec:commandline}

The effect of redirection files can also be achieved by invoking
the \LaTeX{} compiler with a more elaborate command line.
Most conveniently this should be done as part
of a shell script or a |Makefile|.

When using \textsf{childdoc} in the main file, the following
command lines effectively perform a redirection
(note that depending on the shell being used,
backslashes may have to be doubled: `|\|' $\to$ `|\\|'):
%
\begin{center}
|... -jobname "|\textit{target}|" |\\|"|[\textit{flags}]%
|\input{childdoc.def}\childdocforward[|\textit{main}|]{|\textit{dest}|}"|
\end{center}
%
Here \textit{target} is the name of the output file,
\textit{main} is the name of the main file
and \textit{dest} is the name of the main or child file to be processed
(all filenames without extensions).
The optional argument \textit{main} can be omitted
if \textit{main} matches \textit{dest}.
Optionally, compilation \textit{flags} can be defined via |\def| commands.
This command line makes the \TeX{} engine believe
it is compiling the file \textit{target}
whose content is specified as the latter parameter.
The provided code then forwards the processing to
\textit{main} or \textit{dest} as described in \secref{sec:forward}.

%%%%%%%%%%%%%%%%%%%%%%%%%%%%%%%%%%%%%%%%%%%%%%%%%%%%%%%%%%%%%%%%%%%%%%%%%%%%%%%%
\subsection{Include by Input}
\label{sec:input}

Including child documents by |\include| has some restrictions by design.
Most notably, the content of a child document always occupies
its own set of pages; pages cannot be shared between child documents.
Usually, this behaviour makes perfect sense
because each child document contain an essential part of the document.
However, in some situations it may be desirable to compose
a document from a collection of parts
without having mandatory page breaks between then.
For this case, the package
provides a mechanism to include parts
by |\input| which can also be processed individually.
However, by construction this mechanism
requires manual handling of the content to be output.

%%%%%%%%%%%%%%%%%%%%%%%%%%%%%%%%%%%%%%%%
\DescribeMacro{\ifchilddocmanual}
The main file should be prepared as usual, see \secref{sec:include}.
However, the document body must make a distinction
between processing of an individual part and of the main document, e.g.:
%
\begin{center}
\begin{tabular}{l}
|\ifchilddocmanual|\\
|\input{\childdocname}|\\
|\||else|\\
\textit{document body with }|\input{|\textit{part}|}|\\
|\||fi|
\end{tabular}
\end{center}
%
The conditional |\ifchilddocmanual| is true whenever
a part to be included by |\input| is being compiled,
and the name of the part is stored in |\childdocname|.

%%%%%%%%%%%%%%%%%%%%%%%%%%%%%%%%%%%%%%%%
\DescribeMacro{\childdocby}
Each part to be included by |\input| should start with:
%
\begin{center}
\begin{tabular}{l}
|\input{childdoc.def}|\\
|\childdocby{|\textit{main}|}|\\
\end{tabular}
\end{center}
%
The directive |\childdocby| is similar to |\childdocof|
described in \secref{sec:include},
but the subsequent selection of content must be done manually.
To that end, both |\ifchilddoc| and |\ifchilddocmanual|
will be true upon processing of a part,
and the name of the part is stored in |\childdocname|.
Note that |\jobname| will be set to the filename of the current part
so that each part receives an individual |.aux| file
that does not interfere with the |.aux| file(s) of the main document.
This behaviour can be altered by the alternative form
|\childdocby[*]{|\textit{main}|}| (with a non-empty optional argument)
which uses the |.aux| file of the main document
by setting |\jobname| to \textit{main}.

%%%%%%%%%%%%%%%%%%%%%%%%%%%%%%%%%%%%%%%%%%%%%%%%%%%%%%%%%%%%%%%%%%%%%%%%%%%%%%%%
\subsection{Driver Development}
\label{sec:driver}

The \textsf{childdoc} mechanism can also be use for the development
of definition files such as \LaTeX{} styles or classes.
This case differs from the above setup with multiple parts
included by |\include| in that no |\includeonly| should be invoked.
This can be achieved by starting the include file
(before |\ProvidesPackage|) with:
%
\begin{center}
\begin{tabular}{l}
|\input{childdoc.def}|\\
|\childdocforward{|\textit{main}|}|\\
\end{tabular}
\end{center}
%
or alternatively with:
%
\begin{center}
\begin{tabular}{l}
|\input{childdoc.def}|\\
|\childdocby{|\textit{main}|}|\\
\end{tabular}
\end{center}
%
Both forms have slightly different effects as described above.
The main file is prepared as usual, see \secref{sec:include}.

%%%%%%%%%%%%%%%%%%%%%%%%%%%%%%%%%%%%%%%%%%%%%%%%%%%%%%%%%%%%%%%%%%%%%%%%%%%%%%%%
\subsection{Legacy Detection}
\label{sec:detection}

The directive |\childdocmain| in the main file can detect
whether the complete document or merely a child is to be compiled
even without using the directive |\childdocof|.
This method is deprecated because it is less robust
and there is no compelling reason to use it;
it is merely provided for backward compatibility
and it may be removed in future versions.

If the detection mechanism is to be used,
it is mandatory to correctly specify
the filename of the main file as the argument of |\childdocmain|:
%
\begin{center}
\begin{tabular}{l}
|\input{childdoc.def}|\\
|\childdocmain{|\textit{main}|}|\\
\end{tabular}
\end{center}
%
If |\jobname| does not match the argument \textit{main} of |\childdocmain|,
it is assumed that |\jobname| points to the child file to be compiled.
When using |\childdocmain| with the main file specified as argument,
it suffices to start a child file
with just |\input{|\textit{main}|}|
without loading of the package and using |\childdocof|.
If instead all processing is done
with the appropriate \textsf{childdoc} directives,
the argument of \textit{main} of |\childdocmain| can be empty.

An alternative version of the command line processing described
in \secref{sec:commandline} using the detection mechanism reads:
%
\begin{center}
|... -jobname "|\textit{target}|" "|[\textit{flags}]%
[|\def\jobname{|\textit{dest}|}|]|\input{|\textit{main}|}"|
\end{center}

%%%%%%%%%%%%%%%%%%%%%%%%%%%%%%%%%%%%%%%%%%%%%%%%%%%%%%%%%%%%%%%%%%%%%%%%%%%%%%%%
\subsection{Manual Code}
\label{sec:manual}

In case one cannot be certain whether the definitions file |childdoc.def|
is installed on the target \TeX{} distribution
and one prefers not to ship it,
it is conceivable to paste a few relevant commands into the sources.

To that end, drop all statements |\input{childdoc.def}|
and perform the replacements as outlined below.
Instead of |\childdocmain{|\textit{main}|}| add the following code
to the top of the main file:
%
\begin{center}
\begin{tabular}{l}
|\||ifdefined\childdocname\endinput\||fi\newif\ifchilddoc|\\
|\edef\childdocname{\scantokens\expandafter{\jobname\noexpand}}|\\
|\def\childdocmain{|\textit{main}|}\||ifx\childdocmain\childdocname\||else|\\
|\childdoctrue\includeonly{\childdocname}\let\jobname\childdocmain\||fi|\\
\end{tabular}
\end{center}
%
Instead of |\childdocof{|\textit{main}|}| just include the main file
at the top of each child file:
%
\begin{center}
|\input{|\textit{main}|}|
\end{center}
%
A simple redirection |\childdocforward{|\textit{dest}|}| is achieved by:
%
\begin{center}
|\def\jobname{|\textit{dest}|}\input{\jobname}|
\end{center}
%
The redirection with prefix
|\childdocforwardprefix[|\textit{prefix}|]{|\textit{dest}|}|
is accomplished by:
%
\begin{center}
\begin{tabular}{l}
|{\edef\jobname{\scantokens\expandafter{\jobname\noexpand}}|\\
|\def\redirectjob |\textit{prefix}|#1~~~{\gdef\jobname{|\textit{dest}|#1}}|\\
|\expandafter\redirectjob\jobname~~~}\input{\jobname}|
\end{tabular}
\end{center}

In an alternative approach,
child documents can be compiled by a specific command line
without additional code or specific definitions:
%
\begin{center}
|... -jobname "|\textit{target}|" "|[\textit{flags}]%
|\includeonly{|\textit{dest}|}\input{|\textit{main}|}"|
\end{center}
%

%%%%%%%%%%%%%%%%%%%%%%%%%%%%%%%%%%%%%%%%%%%%%%%%%%%%%%%%%%%%%%%%%%%%%%%%%%%%%%%%
%%%%%%%%%%%%%%%%%%%%%%%%%%%%%%%%%%%%%%%%%%%%%%%%%%%%%%%%%%%%%%%%%%%%%%%%%%%%%%%%
\section{Information}

%%%%%%%%%%%%%%%%%%%%%%%%%%%%%%%%%%%%%%%%%%%%%%%%%%%%%%%%%%%%%%%%%%%%%%%%%%%%%%%%
\subsection{Copyright}

Copyright \copyright{} 2017--2018 Niklas Beisert

This work may be distributed and/or modified under the
conditions of the \LaTeX{} Project Public License, either version 1.3
of this license or (at your option) any later version.
The latest version of this license is in
  \url{http://www.latex-project.org/lppl.txt}
and version 1.3 or later is part of all distributions of \LaTeX{}
version 2005/12/01 or later.

This work has the LPPL maintenance status `maintained'.

The Current Maintainer of this work is Niklas Beisert.

This work consists of the files |README.txt|, |childdoc.ins| and |childdoc.dtx|
as well as the derived files |childdoc.def|, |cdocsamp.tex|
with |cdocsch1.tex|, |cdocsch2.tex|, |cdocspt3.tex|, |cdocspt4.tex|,
|cdocsdrf.tex|, |cdocsfn1.tex|, |cdocsfn2.tex|
as well as |childdoc.pdf|.

%%%%%%%%%%%%%%%%%%%%%%%%%%%%%%%%%%%%%%%%%%%%%%%%%%%%%%%%%%%%%%%%%%%%%%%%%%%%%%%%
\subsection{Files and Installation}

The package consists of the files:
%
\begin{center}
\begin{tabular}{ll}
    |README.txt|   & readme file \\
    |childdoc.ins| & installation file \\
    |childdoc.dtx| & source file \\
    |childdoc.def| & definition file \\
    |cdocsamp.tex| & sample main file \\
    |cdocsch1.tex| & sample include file \\
    |cdocsch2.tex| & sample include file \\
    |cdocspt3.tex| & sample part file \\
    |cdocspt4.tex| & sample part file \\
    |cdocsdrf.tex| & sample redirection file \\
    |cdocsfn1.tex| & sample redirection file \\
    |cdocsfn2.tex| & sample redirection file \\
    |childdoc.pdf| & manual
\end{tabular}
\end{center}
%
The distribution consists of the files
|README.txt|, |childdoc.ins| and |childdoc.dtx|.
%
\begin{itemize}
\item
Run (pdf)\LaTeX{} on |childdoc.dtx|
to compile the manual |childdoc.pdf| (this file).
\item
Run \LaTeX{} on |childdoc.ins| to create the definitions file |childdoc.def|
and the sample |cdocsamp.tex| with include files
|cdocsch1.tex|, |cdocsch2.tex|, |cdocspt3.tex|, |cdocspt4.tex|,
|cdocsdrf.tex|, |cdocsfn1.tex|, |cdocsfn2.tex|.
Then copy the file |childdoc.def| to an appropriate directory of your \LaTeX{}
distribution, e.g.\ \textit{texmf-root}|/tex/latex/childdoc|.
\end{itemize}

%%%%%%%%%%%%%%%%%%%%%%%%%%%%%%%%%%%%%%%%%%%%%%%%%%%%%%%%%%%%%%%%%%%%%%%%%%%%%%%%
\subsection{Related CTAN Packages}

There are several other packages which offer a similar functionality:
%
\begin{itemize}
\item
The packages
\href{http://ctan.org/pkg/docmute}{\textsf{docmute}},
\href{http://ctan.org/pkg/includex}{\textsf{includex}} and
\href{http://ctan.org/pkg/standalone}{\textsf{standalone}}
provide commands to include only the document body of
a child file thus allowing both files to be compiled individually.
\item
The packages \href{http://ctan.org/pkg/subdocs}{\textsf{subdocs}}
and \href{http://ctan.org/pkg/subfiles}{\textsf{subfiles}}
provide structures in which the main and child documents can be
encapsulated and allowing them to be compiled individually.
The inclusion mechanism is different from the conventional |\include|.
\item
The package \href{http://ctan.org/pkg/combine}{\textsf{combine}}
is an elaborate solution to combine several documents into one.
\end{itemize}
%
See also the CTAN topic \href{http://ctan.org/topic/subdocs}{\textsf{subdocs}}
for further related packages.
The present package differs from the above solutions in that
a document structure constructed with the conventional |\include| mechanism
just needs two extra commands at the top of every file
such that all constituent files can be compiled individually.

%%%%%%%%%%%%%%%%%%%%%%%%%%%%%%%%%%%%%%%%%%%%%%%%%%%%%%%%%%%%%%%%%%%%%%%%%%%%%%%%
%\subsection{Feature Suggestions}
%
%The following is a list of features which may be useful for future
%versions of this package:
%%
%\begin{itemize}
%\item
%\ldots
%\end{itemize}

%%%%%%%%%%%%%%%%%%%%%%%%%%%%%%%%%%%%%%%%%%%%%%%%%%%%%%%%%%%%%%%%%%%%%%%%%%%%%%%%
\subsection{Revision History}

%%%%%%%%%%%%%%%%%%%%%%%%%%%%%%%%%%%%%%%%
\paragraph{v2.0:} 2018/12/30

\begin{itemize}
\item
immediate forward processing
\item
added |\childdocby| mechanism
\item
manual restructured
\end{itemize}

%%%%%%%%%%%%%%%%%%%%%%%%%%%%%%%%%%%%%%%%
\paragraph{v1.6:} 2018/01/17

\begin{itemize}
\item
application for development of include files
\item
corrections to manual
\end{itemize}

%%%%%%%%%%%%%%%%%%%%%%%%%%%%%%%%%%%%%%%%
\paragraph{v1.5:} 2017/05/21

\begin{itemize}
\item
more complete structuring introduced
\item
|\childdocof| introduced
\item
|\childdoc| renamed to |\childdocmain|
\item
|\childredirect| renamed to |\childdocforward| and |\childdocforwardprefix|
and functionality expanded
\end{itemize}

%%%%%%%%%%%%%%%%%%%%%%%%%%%%%%%%%%%%%%%%
\paragraph{v1.0:} 2017/04/27

\begin{itemize}
\item
manual and install package
\item
first version published on CTAN
\end{itemize}

%%%%%%%%%%%%%%%%%%%%%%%%%%%%%%%%%%%%%%%%
\paragraph{v0.6:} 2017/04/26

\begin{itemize}
\item
redirection mechanism added
\end{itemize}

%%%%%%%%%%%%%%%%%%%%%%%%%%%%%%%%%%%%%%%%
\paragraph{v0.5:} 2017/04/26

\begin{itemize}
\item
functionality in definition file
\end{itemize}


%%%%%%%%%%%%%%%%%%%%%%%%%%%%%%%%%%%%%%%%%%%%%%%%%%%%%%%%%%%%%%%%%%%%%%%%%%%%%%%%
%%%%%%%%%%%%%%%%%%%%%%%%%%%%%%%%%%%%%%%%%%%%%%%%%%%%%%%%%%%%%%%%%%%%%%%%%%%%%%%%
%%%%%%%%%%%%%%%%%%%%%%%%%%%%%%%%%%%%%%%%%%%%%%%%%%%%%%%%%%%%%%%%%%%%%%%%%%%%%%%%
\appendix

\settowidth\MacroIndent{\rmfamily\scriptsize 000\ }

 \DocInput{childdoc.dtx}

\end{document}
%</driver>
% \fi
%
% %%%%%%%%%%%%%%%%%%%%%%%%%%%%%%%%%%%%%%%%%%%%%%%%%%%%%%%%%%%%%%%%%%%%%%%%%%%%%%
% %%%%%%%%%%%%%%%%%%%%%%%%%%%%%%%%%%%%%%%%%%%%%%%%%%%%%%%%%%%%%%%%%%%%%%%%%%%%%%
% \section{Sample}
%\iffalse
%<*samplemain>
%\fi
%
% The following presents a sample document
% with two chapters, two parts, a title page,
% a compile flag as well as three forwarding files to set the flag.
% It consists of eight |.tex| files:
% \begin{center}
% \begin{tabular}{ll}
% |cdocsamp.tex|&main file\\
% |cdocsch1.tex|&include file for chapter 1\\
% |cdocsch2.tex|&include file for chapter 2\\
% |cdocspt3.tex|&include file for part 3\\
% |cdocspt4.tex|&include file for part 4\\
% |cdocsdrf.tex|&forwarding file for main file in draft mode\\
% |cdocsfi1.tex|&forwarding file for final version of chapter 1\\
% |cdocsfi2.tex|&forwarding file for final version of chapter 2\\
% \end{tabular}
% \end{center}
% Each of the eight files can be compiled directly by the \LaTeX{} compiler.
%
% %%%%%%%%%%%%%%%%%%%%%%%%%%%%%%%%%%%%%%
% \paragraph{Main File.}
%
% The main file is called |cdocsamp.tex|.
%
% Load the \textsf{childdoc} definitions and
% declare the filename for the main document:
%    \begin{macrocode}
\input{childdoc.def}
\childdocmain{}
%    \end{macrocode}

% Optional override for |\version| flag:
%    \begin{macrocode}
%%\ifchilddoc\else\providecommand{\version}{draft}\fi
%    \end{macrocode}

% Define the default values for the |\version| flag
% (|final| for the main file and |draft| for childs):
%    \begin{macrocode}
\ifchilddoc
\providecommand{\version}{draft}
\else
\providecommand{\version}{final}
\fi
%    \end{macrocode}

% Load the standard document class:
%    \begin{macrocode}
\documentclass[12pt]{article}
%    \end{macrocode}

% Start the document body:
%    \begin{macrocode}
\begin{document}
%    \end{macrocode}

% Declare a title page.
% Print title, part of document being processed and version flag:
%    \begin{macrocode}
\addtocounter{page}{-1}
\begin{center}
{\LARGE\bfseries{}childdoc example\par}
\vspace{1cm}
\ifchilddoc
\ifchilddocmanual part\else chapter\fi:
`\childdocname' of `\childdocjob'\par
\else
main document: `\childdocjob'\par
\fi
version: \version\par
\end{center}
\newpage
%    \end{macrocode}

% Manually include selected file,
% otherwise process as usual:
%    \begin{macrocode}
\ifchilddocmanual
\section*{part `\childdocname'}
\input{\childdocname}
\else
%    \end{macrocode}

% Include the two chapters:
%    \begin{macrocode}
\include{cdocsch1}
\include{cdocsch2}
%    \end{macrocode}

% Include the two parts unless only chapters should be displayed:
%    \begin{macrocode}
\ifchilddoc\else
\section{part three}
\input{cdocspt3}
\section{part four}
\input{cdocspt4}
\fi
%    \end{macrocode}

% Process as usual until here:
%    \begin{macrocode}
\fi
%    \end{macrocode}

% End of document body:
%    \begin{macrocode}
\end{document}
%    \end{macrocode}
%\iffalse
%</samplemain>
%\fi
%
% %%%%%%%%%%%%%%%%%%%%%%%%%%%%%%%%%%%%%%
% \paragraph{Chapter Include Files.}
%
% The include files are called |cdocsch1.tex| and |cdocsch2.tex|.
%
%\iffalse
%<*samplechap1|samplechap2>
%\fi

% Optional override for |\version| flag:
%    \begin{macrocode}
%%\providecommand{\version}{final}
%    \end{macrocode}

% Include the main document:
%    \begin{macrocode}
\input{childdoc.def}
\childdocof{cdocsamp}
%    \end{macrocode}

%\iffalse
%</samplechap1|samplechap2>
%\fi
%
%\iffalse
%<*samplechap1>
%\fi
% Some text for chapter 1:
%    \begin{macrocode}
\section{one}
some text in chapter one
%    \end{macrocode}

%\iffalse
%</samplechap1>
%\fi
% Some text for chapter 2:
%\iffalse
%<*samplechap2>
%\fi
%    \begin{macrocode}
\section{two}
more text in chapter two
%    \end{macrocode}

%\iffalse
%</samplechap2>
%\fi
%
% %%%%%%%%%%%%%%%%%%%%%%%%%%%%%%%%%%%%%%
% \paragraph{Part Include Files.}
%
% The include files are called |cdocspt3.tex| and |cdocspt4.tex|.
%
%\iffalse
%<*samplepart3|samplepart4>
%\fi

% Optional override for |\version| flag:
%    \begin{macrocode}
%%\providecommand{\version}{final}
%    \end{macrocode}

% Include the main document:
%    \begin{macrocode}
\input{childdoc.def}
\childdocby{cdocsamp}
%    \end{macrocode}

%\iffalse
%</samplepart3|samplepart4>
%\fi
%
%\iffalse
%<*samplepart3>
%\fi
% Some text for part 3:
%    \begin{macrocode}
some text in part three
%    \end{macrocode}

%\iffalse
%</samplepart3>
%\fi
% Some text for part 4:
%\iffalse
%<*samplepart4>
%\fi
%    \begin{macrocode}
more text in part four
%    \end{macrocode}

%\iffalse
%</samplepart4>
%\fi
%
% %%%%%%%%%%%%%%%%%%%%%%%%%%%%%%%%%%%%%%
% \paragraph{Forwarding for a Complete Draft.}
%
% The following forwarding file |cdocsdrf.tex|
% compiles the main document in draft mode:
%\iffalse
%<*sampledraft>
%\fi
%    \begin{macrocode}
\def\version{draft}
\input{childdoc.def}
\childdocforward{cdocsamp}
%    \end{macrocode}

%\iffalse
%</sampledraft>
%\fi
%
% %%%%%%%%%%%%%%%%%%%%%%%%%%%%%%%%%%%%%%
% \paragraph{Forwarding for Final Version of the Chapters.}
%
% The following forwarding files |cdocsfn1.tex| and |cdocsfn2.tex|
% (with identical content)
% compile the final versions of the child documents
% |cdocsch1.tex| and |cdocsch2.tex|, respectively:
%\iffalse
%<*samplefinal>
%\fi
%    \begin{macrocode}
\def\version{final}
\input{childdoc.def}
\childdocforwardprefix[cdocsamp]{cdocsfn}{cdocsch}
%    \end{macrocode}

%\iffalse
%</samplefinal>
%\fi
%
% %%%%%%%%%%%%%%%%%%%%%%%%%%%%%%%%%%%%%%
% \paragraph{Command Line Processing.}
%
% The following three command lines generate the output files
% |cdocscld|, |cdocscl1| and |cdocscl2|
% which should be identical to
% |cdocsdrf|, |cdocsch1| and |cdocsfn2|, respectively:
% \begin{center}
% \begin{tabular}{l}
% |latex -jobname cdocscld \|\\
% |  "\def\version{draft}\input{childdoc.def}\childdocforward{cdocsamp}"|\\
% |latex -jobname cdocscl1 \|\\
% |  "\input{childdoc.def}\childdocforward[cdocsamp]{cdocsch1}"|\\
% |latex -jobname cdocscl2 \|\\
% |  "\def\version{final}\input{childdoc.def}\childdocforward{cdocsch2}"|
% \end{tabular}
% \end{center}
% Note that the trailing backslash on each first line
% merely continues the input to the second line
% (for convenient cut ant paste).
% Furthermore, the command |latex| can be replaced by any
% of its alternative versions such as |pdflatex|.
%
% %%%%%%%%%%%%%%%%%%%%%%%%%%%%%%%%%%%%%%%%%%%%%%%%%%%%%%%%%%%%%%%%%%%%%%%%%%%%%%
% %%%%%%%%%%%%%%%%%%%%%%%%%%%%%%%%%%%%%%%%%%%%%%%%%%%%%%%%%%%%%%%%%%%%%%%%%%%%%%
% \section{Implementation}
%\iffalse
%<*package>
%\fi
%
% This section describes the definitions file |childdoc.def|.

% The definitions cannot be loaded using |\usepackage| or |\RequirePackage|
% which has a mechanism to prevent loading a style file more than once.
% When loading the definitions by means of |\input|
% multiple instances have to be prevented manually:
%\iffalse
%This code needs to be before the `\ProvidesFile' directive
%which is defined at the beginning of this file.
%Therefore it is also placed there and commented out here.
%</package>
%<*discard>
%\fi
%    \begin{macrocode}
\ifdefined\childdocmain\endinput\fi
%    \end{macrocode}
%\iffalse
%</discard>
%<*package>
%\fi
%
% \macro{\ifchilddoc}
% \macro{\ifchilddocmanual}
% The conditional |\ifchilddoc| tells whether a
% child (true) or main (false) document is being compiled.
% The conditional |\ifchilddocmanual| tells whether
% the |\includeonly| mechanism is used (false) or
% the selection of child files must be performed manually (true).
% The definitions initialise to false:
%    \begin{macrocode}
\newif\ifchilddoc
\newif\ifchilddocmanual
%    \end{macrocode}

% \macro{\childdocname}
% \macro{\childdocjob}
% The macro |\childdocname| stores the name of the main document
% to be compiled. The macro |\childdocjob| stores the name of
% the document on which the \LaTeX{} compiler was originally invoked.
% The content of |\jobname| cannot be compared
% to filenames specified in the source due to different catcodes.
% The following code rescans |\jobname|, stores the result
% in |\childdocname| and saves a copy in |\childdocjob|:
%    \begin{macrocode}
\edef\childdocname{\scantokens\expandafter{\jobname\noexpand}}
\let\childdocjob\childdocname
%    \end{macrocode}

% \macro{\childdocdisable}
% The macro |\childdocdisable| prevents the main file
% from being processed more than once.
% At this stage, the main document command |\childdocmain|
% is assumed to be called once again where it should do nothing.
% Any subsequent call to it should prevent
% a secondary processing of the main document
% It overwrites the forwarding commands
% |\childdocof| and |\childdocforward|
% with empty macros to prevent further inclusions of the main document:
%    \begin{macrocode}
\newcommand{\childdocdisable}
{
  \renewcommand{\childdocmain}[1]{\renewcommand{\childdocmain}[1]{\endinput}}
  \renewcommand{\childdocof}[1]{}
  \renewcommand{\childdocby}[2][]{}
  \renewcommand{\childdocforward}[2][]{}
  \renewcommand{\childdocdisable}{}
}
%    \end{macrocode}

% \macro{\childdocmain}
% The macro |\childdocmain| is to be called at the top of the main file
% with nothing or the main filename (without extension) as argument.
% First, it breaks loops.
% If the argument is not empty and does not match |\childdocname|
% (which is set by the first inclusion of |childdoc.def|),
% |\ifchilddoc| is set to true, |\includeonly| is applied to the child file
% and |\jobname| is set to the main file
% (for proper handling of |.aux| files):
%    \begin{macrocode}
\newcommand{\childdocmain}[1]
{
  \childdocdisable\childdocmain{}
  \if?#1?\else
    \begingroup
      \def\childdoctmp{#1}
      \ifx\childdoctmp\childdocname
        \def\childdoctmp{}
      \else
        \def\childdoctmp
        {
          \childdoctrue
          \includeonly{\childdocname}
          \def\childdocjob{#1}
          \def\jobname{#1}
        }
      \fi
      \expandafter
    \endgroup
    \childdoctmp
  \fi
}
%    \end{macrocode}

% \macro{\childdocof}
% The command |\childdocof| redirects
% compilation to the main file |#1|.
%    \begin{macrocode}
\newcommand{\childdocof}[1]
{
  \childdocdisable
  \childdoctrue
  \includeonly{\childdocname}
  \def\jobname{#1}
  \def\childdocjob{#1}
  \input{#1}
}
%    \end{macrocode}

% \macro{\childdocby}
% The command |\childdocby| ....
%    \begin{macrocode}
\newcommand{\childdocby}[2][]
{
  \childdocdisable
  \childdoctrue
  \childdocmanualtrue
  \if?#1?\else
    \def\jobname{#2}
  \fi
  \def\childdocjob{#2}
  \input{#2}
  \endinput
}
%    \end{macrocode}

% \macro{\childdocforward}
% The command |\childdocforward| redirects
% compilation to the main file or
% (if the optional argument is given) a child file.
% Parameters are set as if the main file
% or a child file starting with |\childdocof| was compiled.
% Then compilation is handed over to the main file:
%    \begin{macrocode}
\newcommand{\childdocforward}[2][]
{
  \begingroup
    \if?#1?
      \def\childdoctmp
      {
        \def\childdocname{#2}
        \def\childdocjob{#2}
        \def\jobname{#2}
        \input{#2}
        \endinput
      }
    \else
      \def\childdoctmp
      {
        \childdocdisable
        \def\childdocname{#2}
        \childdoctrue
        \includeonly{#2}
        \def\childdocjob{#1}
        \def\jobname{#1}
        \input{#1}
        \endinput
      }
    \fi
    \expandafter
  \endgroup
  \childdoctmp
}
%    \end{macrocode}

% \macro{\childdocforwardprefix}
% The command |\childdocforwardprefix| redirects
% compilation to the main or a child file by means of a pattern.
% The prefix |#1| in the current filename is replaced by |#2|
% and the suffix of the current filename is kept
% (it is assumed that the filename does not contain the substring `|~~~|'
% which is used as a delimiter).
% Compilation is handed over to the new file by |\childdocforward|:
%    \begin{macrocode}
\newcommand{\childdocforwardprefix}[3][]
{
  \begingroup
    \def\childdocextract #2##1~~~{\def\childdoctmp{\childdocforward[#1]{#3##1}}}
    \expandafter\childdocextract\childdocname~~~
    \expandafter
  \endgroup
  \childdoctmp
}
%    \end{macrocode}

% \macro{\childdoc}
% The deprecated macro |\childdoc| is a legacy version of |\childdocmain|:
%    \begin{macrocode}
\newcommand{\childdoc}{\childdocmain}
%    \end{macrocode}

% \macro{\childdocredirect}
% The deprecated macro |\childdocredirect| is a legacy version
% of |\childdocforward| and |\childdocforwardprefix|:
%    \begin{macrocode}
\newcommand{\childdocredirect}[2][]
{
  \begingroup
    \if?#1?
      \def\childdoctmp{\childdocforward{#2}}
    \else
      \def\childdoctmp{\childdocforwardprefix{#1}{#2}}
    \fi
    \expandafter
  \endgroup
  \childdoctmp
}
%    \end{macrocode}

%\iffalse
%</package>
%\fi
%
\endinput
|\\
|\childdocforward[|\textit{main}|]{|\textit{dest}|}|\\
\end{tabular}
\end{center}
%
The argument \textit{dest} is the destination file
(without extension).
It should be the main file or one of the child files.
Note that further \textsf{childdoc} directives
such as |\childdocof| and |\childdocforward|
in the indicated file will be processed in this form.
The optional argument \textit{main}
passes on directly to the main file \textit{main}
while pretending to compile the child \textit{dest}.
This form behaves as if \textit{dest}
issues |\childdocof{|\textit{main}|}| right away,
and no further \textsf{childdoc} directives will be processed.

%%%%%%%%%%%%%%%%%%%%%%%%%%%%%%%%%%%%%%%%
\DescribeMacro{\...prefix}
In the alternative form |\childdocforwardprefix|,
%
\begin{center}
\begin{tabular}{l}
|% \iffalse
%
% childdoc.dtx Copyright (C) 2017-2018 Niklas Beisert
%
% This work may be distributed and/or modified under the
% conditions of the LaTeX Project Public License, either version 1.3
% of this license or (at your option) any later version.
% The latest version of this license is in
%   http://www.latex-project.org/lppl.txt
% and version 1.3 or later is part of all distributions of LaTeX
% version 2005/12/01 or later.
%
% This work has the LPPL maintenance status `maintained'.
%
% The Current Maintainer of this work is Niklas Beisert.
%
% This work consists of the files childdoc.dtx and childdoc.ins
% and the derived files childdoc.def and cdocsamp.tex with
% cdocsch1.tex, cdocsch2.tex, cdocsdrf.tex, cdocsfn1.tex, cdocsfn2.tex.
%
%<package>\ifdefined\childdocmain\endinput\fi
%<package>\ProvidesFile{childdoc.def}[2018/12/30 v2.0 child document driver]
%<samplemain>\ProvidesFile{cdocsamp.tex}[2018/12/30 v2.0 sample for childdoc]
%<*driver>
%\ProvidesFile{childdoc.drv}[2018/12/30 v2.0 childdoc reference manual file]
\PassOptionsToClass{10pt,a4paper}{article}
\documentclass{ltxdoc}

\usepackage[margin=35mm]{geometry}
\usepackage{hyperref}
\usepackage{hyperxmp}
\usepackage[usenames]{color}

\hypersetup{colorlinks=true}
\hypersetup{pdfstartview=FitH}
\hypersetup{pdfpagemode=UseNone}
\hypersetup{pdfsource={}}
\hypersetup{pdflang={en-UK}}
\hypersetup{pdfcopyright={Copyright 2017-2018 Niklas Beisert.
  This work may be distributed and/or modified under the
  conditions of the LaTeX Project Public License, either version 1.3
  of this license or (at your option) any later version.}}
\hypersetup{pdflicenseurl={http://www.latex-project.org/lppl.txt}}
\hypersetup{pdfcontactaddress={ETH Zurich, ITP, HIT K,
  Wolfgang-Pauli-Strasse 27}}
\hypersetup{pdfcontactpostcode={8093}}
\hypersetup{pdfcontactcity={Zurich}}
\hypersetup{pdfcontactcountry={Switzerland}}
\hypersetup{pdfcontactemail={nbeisert@itp.phys.ethz.ch}}
\hypersetup{pdfcontacturl={http://people.phys.ethz.ch/\xmptilde nbeisert/}}

\newcommand{\secref}[1]{\hyperref[#1]{section \ref*{#1}}}

\parskip1ex
\parindent0pt
\let\olditemize\itemize
\def\itemize{\olditemize\parskip0pt}

\begin{document}

\title{The \textsf{childdoc} Package}
\hypersetup{pdftitle={The childdoc Package}}
\author{Niklas Beisert\\[2ex]
  Institut f\"ur Theoretische Physik\\
  Eidgen\"ossische Technische Hochschule Z\"urich\\
  Wolfgang-Pauli-Strasse 27, 8093 Z\"urich, Switzerland\\[1ex]
  \href{mailto:nbeisert@itp.phys.ethz.ch}
  {\texttt{nbeisert@itp.phys.ethz.ch}}}
\hypersetup{pdfauthor={Niklas Beisert}}
\hypersetup{pdfsubject={Manual for the LaTeX2e Package childdoc}}
\date{30 December 2018, \textsf{v2.0}}
\maketitle

\begin{abstract}\noindent
\textsf{childdoc} is a \LaTeXe{} package
that enables the direct compilation
of document sections included by |\include|
to individual files.
\end{abstract}

\begingroup
\parskip0ex
\tableofcontents
\endgroup

%%%%%%%%%%%%%%%%%%%%%%%%%%%%%%%%%%%%%%%%%%%%%%%%%%%%%%%%%%%%%%%%%%%%%%%%%%%%%%%%
%%%%%%%%%%%%%%%%%%%%%%%%%%%%%%%%%%%%%%%%%%%%%%%%%%%%%%%%%%%%%%%%%%%%%%%%%%%%%%%%
\section{Introduction}

\LaTeX{} provides a mechanism to structure a large document (such as a book)
into a main file and several child files (containing the chapters)
using the |\include| command.
This mechanism is beneficial for documents
which span hundreds of pages in order to
make the source file(s) more manageable.
Moreover, compilation can be restricted to
selected child files by means of the |\includeonly| command.
The latter feature can be used to reduce the compilation time while editing
(this was significantly more useful in the earlier days of \LaTeX{})
or to generate a smaller document which is easier to navigate.
Another application of |\includeonly| is to generate
documents consisting of selected parts of the complete document.

However, there are a few drawbacks of the plain |\include| mechanism:
\begin{itemize}
\item
The child files cannot be compiled on their own,
they can only be compiled via the main file.
A naive editing environment
(such as a text editor with an option
to have the current file processed by \LaTeX)
may require one to switch to the main file before compiling;
attempting to compile the child file produces errors.
\item
The main file must be modified (each time)
to adjust the |\includeonly| command
to the present needs. This easily leaves the main file in a messy state.
\item
The generated document will always carry the filename
of the main document. This is inconvenient if
several child files are to be compiled and
to be kept for distribution.
\end{itemize}

The present package provides a simple interface
to make child files individually compilable by \LaTeX{}.
Compiling a child file then has the same effect as compiling
the main file with an |\includeonly| command
to select the appropriate child.
Moreover the generated document will carry the name of the child
rather than the main file.
This resolves all three above issues.

This feature is meant to make the editing of books,
thesis documents and lecture notes somewhat more convenient.
However, the package can also be used efficiently for
composing a series of documents (such as exercise sheets)
which are typically distributed individually.
It then assists the author in generating the individual documents
(potentially in different versions)
as well as a document containing the collected series.
Another application is in developing style files
or other kinds of included material
where compilation of the style file could redirect
to a sample or test file.

%%%%%%%%%%%%%%%%%%%%%%%%%%%%%%%%%%%%%%%%%%%%%%%%%%%%%%%%%%%%%%%%%%%%%%%%%%%%%%%%
%%%%%%%%%%%%%%%%%%%%%%%%%%%%%%%%%%%%%%%%%%%%%%%%%%%%%%%%%%%%%%%%%%%%%%%%%%%%%%%%
\section{Usage}

First of all, the package \textsf{childdoc} is \emph{not} a standard
\LaTeXe{} |.sty| style file! Therefore it needs to be invoked in
a non-standard way.

%%%%%%%%%%%%%%%%%%%%%%%%%%%%%%%%%%%%%%%%%%%%%%%%%%%%%%%%%%%%%%%%%%%%%%%%%%%%%%%%
\subsection{Included Files}
\label{sec:include}

%%%%%%%%%%%%%%%%%%%%%%%%%%%%%%%%%%%%%%%%
\DescribeMacro{\childdocmain}
To use the package, add the commands
\begin{center}
\begin{tabular}{l}
|\input{childdoc.def}|\\
|\childdocmain{}|\\
\end{tabular}
\end{center}
at the very top of the main \LaTeX{} file,
in particular \emph{before} the |\documentclass| statement!
The argument of |\childdocmain| should be left empty
(but it must be present).

%%%%%%%%%%%%%%%%%%%%%%%%%%%%%%%%%%%%%%%%
\DescribeMacro{\childdocof}
Furthermore, add the commands
\begin{center}
\begin{tabular}{l}
|\input{childdoc.def}|\\
|\childdocof{|\textit{main}|}|\\
\end{tabular}
\end{center}
at the top of every child file \textit{child}
which is included by |\include{|\textit{child}|}|
from within the main file
(or at least for those files to be compiled individually).
The argument \textit{main} must be the filename of the main file.

There are a couple of
considerations in setting up the main and child documents:

%%%%%%%%%%%%%%%%%%%%%%%%%%%%%%%%%%%%%%%%
\paragraph{Restrictions.}

Please note the following restrictions:
\begin{itemize}
\item
|\childdocmain| must be called with one argument \textit{main}
to ensure compatibility with earlier version of the package.
It must either be empty (|\childdocmain{}|)
or precisely match the filename of the main file in which it is specified.
See \secref{sec:detection} for further information.
\item
The filename \textit{main} must be specified without the |.tex| extension.
\item
The filename \textit{main} is case sensitive
(even in case-insensitive file systems)
due to internal string comparison.
\item
The argument \textit{main} should be fully expanded, it cannot be a macro.
\item
Subdirectories and special characters should be avoided in filenames.
\item
The command |\childdocmain{|\textit{main}|}| must be followed by a whitespace.
It should not be followed immediately by another command
or by a comment mark `|%|'.
This is because the \TeX{} parser reads the token immediately following
the argument of |\childdocmain| and puts it
at the beginning of every child section;
however, a white\-space is ignored.
\end{itemize}

%%%%%%%%%%%%%%%%%%%%%%%%%%%%%%%%%%%%%%%%
\paragraph{Content of Main File.}

It is advisable to place all content in the child files included by |\include|.
Any output contained in the main file will appear in all child documents
unless suppressed manually;
it cannot be suppressed automatically by the |\includeonly| directive
and thus should normally be avoided.
A method to include some content in the main file
by means of conditional processing is described in \secref{sec:conditional}.

%%%%%%%%%%%%%%%%%%%%%%%%%%%%%%%%%%%%%%%%
\paragraph{Page Numbering.}

When only a part of the document is compiled,
the appropriate numbering of pages
(as well as other status parameters)
is determined from the |.aux| files.
The latter contain information from previous passes.
However this information needs to propagate through
all intermediate child documents.
Therefore the page numbering in child documents may well
be inconsistent until the complete document is compiled at least once.

A useful (if unconventional) way to always ensure a consistent
page numbering is to restart the numbering in each child document
and denote the pages by `\textit{child}|.|\textit{page}'
where \textit{child} represents the chapter/section number of the child file.
This can be achieved by the command
|\numberwithin{page}{|\textit{child}|}|
of the \textsf{amsmath} package
where \textit{child} can be |chapter| or |section|
depending on the chosen structuring.
Alternatively, one can modify the macro |\thepage| appropriately
and reset the counter |page| at the start of each child file.

%%%%%%%%%%%%%%%%%%%%%%%%%%%%%%%%%%%%%%%%%%%%%%%%%%%%%%%%%%%%%%%%%%%%%%%%%%%%%%%%
\subsection{Conditional Processing}
\label{sec:conditional}

The package provides a mechanism to compile different versions
of a document. To customise the versions further some conditional processing
can come in handy to distinguish which version is being compiled.
The package provides two macros to describe the compilation context:

%%%%%%%%%%%%%%%%%%%%%%%%%%%%%%%%%%%%%%%%
\DescribeMacro{\ifchilddoc}
The conditional |\ifchilddoc| distinguishes between the compilation of
child documents and the main document:
%
\begin{center}
|\ifchilddoc |\textit{child-code}| |[|\||else |\textit{main-code}]| \||fi|
\end{center}

%%%%%%%%%%%%%%%%%%%%%%%%%%%%%%%%%%%%%%%%
\DescribeMacro{\childdocname}
\DescribeMacro{\childdocjob}
The macro |\childdocname| contains the filename (without extension)
of the main or child file being processed.
Note that |\childdocjob| will always contain the name of the main file.

%%%%%%%%%%%%%%%%%%%%%%%%%%%%%%%%%%%%%%%%
\paragraph{Title Page.}

Conditional processing can be used to include a title or banner page
in the main document when proper precautions are taken.
Importantly, the code in the main file should ensure that the page counter
(as well as other status parameters which are stored in the |.aux| files)
takes the same value after the conditional processing.
Otherwise the page numbers may take divergent values
depending on which part is compiled.

For example, a title page could be declared by:
%
\begin{center}
\begin{tabular}{l}
|\ifchilddoc\||else|\\
|\addtocounter{page}{-1}|\\
\textit{code for title page}\\
|\newpage|\\
|\||fi|
\end{tabular}
\end{center}
%
A banner page for the child documents can be generated by:
%
\begin{center}
\begin{tabular}{l}
|\ifchilddoc|\\
|\addtocounter{page}{-1}|\\
\textit{code for banner page}\\
|\newpage|\\
|\||fi|
\end{tabular}
\end{center}
%
Here one could write a message such as:
\begin{center}
|This is the part \childdocname{} of \childdocjob{}.|
\end{center}

%%%%%%%%%%%%%%%%%%%%%%%%%%%%%%%%%%%%%%%%%%%%%%%%%%%%%%%%%%%%%%%%%%%%%%%%%%%%%%%%
\subsection{Flags}
\label{sec:flags}

The package makes it easy to generate different versions
of the main or child documents.
To this end compilation flags can be defined
and assigned different default values.
They will be particularly useful in conjunction
with the forwarding mechanism described in \secref{sec:forward}.

For example, it may be useful to have a flag |\version|
which can be set to |draft| or |final|.
The document source will contain some conditional code
depending on the value of |\version|.
Suppose further, the flag should default to |final| for the main file
and to |draft| for child files
which is a natural assignment for editing the document.
This is achieved by placing the following code
in the preamble of the main document
(below the |\childdocmain| directive):
%
\begin{center}
\begin{tabular}{l}
|\ifchilddoc|\\
|\providecommand{\version}{draft}|\\
|\||else|\\
|\providecommand{\version}{final}|\\
|\||fi|
\end{tabular}
\end{center}
%
The definition by |\providecommand| makes sure
that previous definitions are not overwritten.
Further statements |\providecommand{\version}{...}|
can thus be added before the above code to override it.

For the main file, one might add a line
(between |\childdocmain| and the above block)
%
\begin{center}
|%\ifchilddoc\||else\providecommand{\version}{draft}\||fi|
\end{center}
%
which can be uncommented to produce a draft version.
Likewise one can add a line to the very top of a child file
(above the |\childdocof{|\textit{main}|}| directive)
%
\begin{center}
|%\providecommand{\version}{final}|
\end{center}
%
which can be uncommented to produce the final version of this child document.

%%%%%%%%%%%%%%%%%%%%%%%%%%%%%%%%%%%%%%%%%%%%%%%%%%%%%%%%%%%%%%%%%%%%%%%%%%%%%%%%
\subsection{Forwarding}
\label{sec:forward}

Different versions of the main or child documents
using compilation flags as described in \secref{sec:flags}
can be (permanently) stored in different files
for convenient compilation, viewing and distribution.
To this end, the package defines a command
to pass on compilation to a different file:

%%%%%%%%%%%%%%%%%%%%%%%%%%%%%%%%%%%%%%%%
\DescribeMacro{\childdocforward}
The command |\childdocforward| redirects processing to
another source file:
%
\begin{center}
\begin{tabular}{l}
|\input{childdoc.def}|\\
|\childdocforward[|\textit{main}|]{|\textit{dest}|}|\\
\end{tabular}
\end{center}
%
The argument \textit{dest} is the destination file
(without extension).
It should be the main file or one of the child files.
Note that further \textsf{childdoc} directives
such as |\childdocof| and |\childdocforward|
in the indicated file will be processed in this form.
The optional argument \textit{main}
passes on directly to the main file \textit{main}
while pretending to compile the child \textit{dest}.
This form behaves as if \textit{dest}
issues |\childdocof{|\textit{main}|}| right away,
and no further \textsf{childdoc} directives will be processed.

%%%%%%%%%%%%%%%%%%%%%%%%%%%%%%%%%%%%%%%%
\DescribeMacro{\...prefix}
In the alternative form |\childdocforwardprefix|,
%
\begin{center}
\begin{tabular}{l}
|\input{childdoc.def}|\\
|\childdocforwardprefix[|\textit{main}|]{|\textit{prefix}|}{|\textit{dest}|}|
\end{tabular}
\end{center}
%
the destination file is determined by a pattern
depending on the current file:
To make this work, the current file must be called
`{\textit{prefix}\hspace{0.2em}\textit{suffix}}'
with \textit{prefix} matching precisely the argument.
Processing is then passed on to the file
`{\textit{dest}\hspace{0.2em}\textit{suffix}}'.
Surely, the same effect is achieved by
directly specifying the
argument `{\textit{dest}\hspace{0.2em}\textit{suffix}}'
in the first form.
However, that requires to set up a different file
for each child. With the alternative form of the command
all these files can have exactly the same content
which simplifies setting them up and maintaining them.

For example, the following file |draft.tex|
with a compilation flag |\version| as described in \secref{sec:flags}
compiles the main document as a draft:
%
\begin{center}
\begin{tabular}{l}
|\def\version{draft}|\\
|\input{childdoc.def}|\\
|\childdocforward{|\textit{main}|}|
\end{tabular}
\end{center}
%
Likewise, the following files |final|\textit{nn}|.tex|
compile the final version of the child document
|child|\textit{nn}|.tex|:
%
\begin{center}
\begin{tabular}{l}
|\def\version{final}|\\
|\input{childdoc.def}|\\
|\childdocforwardprefix{final}{child}|
\end{tabular}
\end{center}
%

Note that when several versions of a main file and/or of each child file
are to be generated, it may be convenient to set up a |Makefile| or
shell script to automatise the process.

%%%%%%%%%%%%%%%%%%%%%%%%%%%%%%%%%%%%%%%%%%%%%%%%%%%%%%%%%%%%%%%%%%%%%%%%%%%%%%%%
\subsection{Command Line Processing}
\label{sec:commandline}

The effect of redirection files can also be achieved by invoking
the \LaTeX{} compiler with a more elaborate command line.
Most conveniently this should be done as part
of a shell script or a |Makefile|.

When using \textsf{childdoc} in the main file, the following
command lines effectively perform a redirection
(note that depending on the shell being used,
backslashes may have to be doubled: `|\|' $\to$ `|\\|'):
%
\begin{center}
|... -jobname "|\textit{target}|" |\\|"|[\textit{flags}]%
|\input{childdoc.def}\childdocforward[|\textit{main}|]{|\textit{dest}|}"|
\end{center}
%
Here \textit{target} is the name of the output file,
\textit{main} is the name of the main file
and \textit{dest} is the name of the main or child file to be processed
(all filenames without extensions).
The optional argument \textit{main} can be omitted
if \textit{main} matches \textit{dest}.
Optionally, compilation \textit{flags} can be defined via |\def| commands.
This command line makes the \TeX{} engine believe
it is compiling the file \textit{target}
whose content is specified as the latter parameter.
The provided code then forwards the processing to
\textit{main} or \textit{dest} as described in \secref{sec:forward}.

%%%%%%%%%%%%%%%%%%%%%%%%%%%%%%%%%%%%%%%%%%%%%%%%%%%%%%%%%%%%%%%%%%%%%%%%%%%%%%%%
\subsection{Include by Input}
\label{sec:input}

Including child documents by |\include| has some restrictions by design.
Most notably, the content of a child document always occupies
its own set of pages; pages cannot be shared between child documents.
Usually, this behaviour makes perfect sense
because each child document contain an essential part of the document.
However, in some situations it may be desirable to compose
a document from a collection of parts
without having mandatory page breaks between then.
For this case, the package
provides a mechanism to include parts
by |\input| which can also be processed individually.
However, by construction this mechanism
requires manual handling of the content to be output.

%%%%%%%%%%%%%%%%%%%%%%%%%%%%%%%%%%%%%%%%
\DescribeMacro{\ifchilddocmanual}
The main file should be prepared as usual, see \secref{sec:include}.
However, the document body must make a distinction
between processing of an individual part and of the main document, e.g.:
%
\begin{center}
\begin{tabular}{l}
|\ifchilddocmanual|\\
|\input{\childdocname}|\\
|\||else|\\
\textit{document body with }|\input{|\textit{part}|}|\\
|\||fi|
\end{tabular}
\end{center}
%
The conditional |\ifchilddocmanual| is true whenever
a part to be included by |\input| is being compiled,
and the name of the part is stored in |\childdocname|.

%%%%%%%%%%%%%%%%%%%%%%%%%%%%%%%%%%%%%%%%
\DescribeMacro{\childdocby}
Each part to be included by |\input| should start with:
%
\begin{center}
\begin{tabular}{l}
|\input{childdoc.def}|\\
|\childdocby{|\textit{main}|}|\\
\end{tabular}
\end{center}
%
The directive |\childdocby| is similar to |\childdocof|
described in \secref{sec:include},
but the subsequent selection of content must be done manually.
To that end, both |\ifchilddoc| and |\ifchilddocmanual|
will be true upon processing of a part,
and the name of the part is stored in |\childdocname|.
Note that |\jobname| will be set to the filename of the current part
so that each part receives an individual |.aux| file
that does not interfere with the |.aux| file(s) of the main document.
This behaviour can be altered by the alternative form
|\childdocby[*]{|\textit{main}|}| (with a non-empty optional argument)
which uses the |.aux| file of the main document
by setting |\jobname| to \textit{main}.

%%%%%%%%%%%%%%%%%%%%%%%%%%%%%%%%%%%%%%%%%%%%%%%%%%%%%%%%%%%%%%%%%%%%%%%%%%%%%%%%
\subsection{Driver Development}
\label{sec:driver}

The \textsf{childdoc} mechanism can also be use for the development
of definition files such as \LaTeX{} styles or classes.
This case differs from the above setup with multiple parts
included by |\include| in that no |\includeonly| should be invoked.
This can be achieved by starting the include file
(before |\ProvidesPackage|) with:
%
\begin{center}
\begin{tabular}{l}
|\input{childdoc.def}|\\
|\childdocforward{|\textit{main}|}|\\
\end{tabular}
\end{center}
%
or alternatively with:
%
\begin{center}
\begin{tabular}{l}
|\input{childdoc.def}|\\
|\childdocby{|\textit{main}|}|\\
\end{tabular}
\end{center}
%
Both forms have slightly different effects as described above.
The main file is prepared as usual, see \secref{sec:include}.

%%%%%%%%%%%%%%%%%%%%%%%%%%%%%%%%%%%%%%%%%%%%%%%%%%%%%%%%%%%%%%%%%%%%%%%%%%%%%%%%
\subsection{Legacy Detection}
\label{sec:detection}

The directive |\childdocmain| in the main file can detect
whether the complete document or merely a child is to be compiled
even without using the directive |\childdocof|.
This method is deprecated because it is less robust
and there is no compelling reason to use it;
it is merely provided for backward compatibility
and it may be removed in future versions.

If the detection mechanism is to be used,
it is mandatory to correctly specify
the filename of the main file as the argument of |\childdocmain|:
%
\begin{center}
\begin{tabular}{l}
|\input{childdoc.def}|\\
|\childdocmain{|\textit{main}|}|\\
\end{tabular}
\end{center}
%
If |\jobname| does not match the argument \textit{main} of |\childdocmain|,
it is assumed that |\jobname| points to the child file to be compiled.
When using |\childdocmain| with the main file specified as argument,
it suffices to start a child file
with just |\input{|\textit{main}|}|
without loading of the package and using |\childdocof|.
If instead all processing is done
with the appropriate \textsf{childdoc} directives,
the argument of \textit{main} of |\childdocmain| can be empty.

An alternative version of the command line processing described
in \secref{sec:commandline} using the detection mechanism reads:
%
\begin{center}
|... -jobname "|\textit{target}|" "|[\textit{flags}]%
[|\def\jobname{|\textit{dest}|}|]|\input{|\textit{main}|}"|
\end{center}

%%%%%%%%%%%%%%%%%%%%%%%%%%%%%%%%%%%%%%%%%%%%%%%%%%%%%%%%%%%%%%%%%%%%%%%%%%%%%%%%
\subsection{Manual Code}
\label{sec:manual}

In case one cannot be certain whether the definitions file |childdoc.def|
is installed on the target \TeX{} distribution
and one prefers not to ship it,
it is conceivable to paste a few relevant commands into the sources.

To that end, drop all statements |\input{childdoc.def}|
and perform the replacements as outlined below.
Instead of |\childdocmain{|\textit{main}|}| add the following code
to the top of the main file:
%
\begin{center}
\begin{tabular}{l}
|\||ifdefined\childdocname\endinput\||fi\newif\ifchilddoc|\\
|\edef\childdocname{\scantokens\expandafter{\jobname\noexpand}}|\\
|\def\childdocmain{|\textit{main}|}\||ifx\childdocmain\childdocname\||else|\\
|\childdoctrue\includeonly{\childdocname}\let\jobname\childdocmain\||fi|\\
\end{tabular}
\end{center}
%
Instead of |\childdocof{|\textit{main}|}| just include the main file
at the top of each child file:
%
\begin{center}
|\input{|\textit{main}|}|
\end{center}
%
A simple redirection |\childdocforward{|\textit{dest}|}| is achieved by:
%
\begin{center}
|\def\jobname{|\textit{dest}|}\input{\jobname}|
\end{center}
%
The redirection with prefix
|\childdocforwardprefix[|\textit{prefix}|]{|\textit{dest}|}|
is accomplished by:
%
\begin{center}
\begin{tabular}{l}
|{\edef\jobname{\scantokens\expandafter{\jobname\noexpand}}|\\
|\def\redirectjob |\textit{prefix}|#1~~~{\gdef\jobname{|\textit{dest}|#1}}|\\
|\expandafter\redirectjob\jobname~~~}\input{\jobname}|
\end{tabular}
\end{center}

In an alternative approach,
child documents can be compiled by a specific command line
without additional code or specific definitions:
%
\begin{center}
|... -jobname "|\textit{target}|" "|[\textit{flags}]%
|\includeonly{|\textit{dest}|}\input{|\textit{main}|}"|
\end{center}
%

%%%%%%%%%%%%%%%%%%%%%%%%%%%%%%%%%%%%%%%%%%%%%%%%%%%%%%%%%%%%%%%%%%%%%%%%%%%%%%%%
%%%%%%%%%%%%%%%%%%%%%%%%%%%%%%%%%%%%%%%%%%%%%%%%%%%%%%%%%%%%%%%%%%%%%%%%%%%%%%%%
\section{Information}

%%%%%%%%%%%%%%%%%%%%%%%%%%%%%%%%%%%%%%%%%%%%%%%%%%%%%%%%%%%%%%%%%%%%%%%%%%%%%%%%
\subsection{Copyright}

Copyright \copyright{} 2017--2018 Niklas Beisert

This work may be distributed and/or modified under the
conditions of the \LaTeX{} Project Public License, either version 1.3
of this license or (at your option) any later version.
The latest version of this license is in
  \url{http://www.latex-project.org/lppl.txt}
and version 1.3 or later is part of all distributions of \LaTeX{}
version 2005/12/01 or later.

This work has the LPPL maintenance status `maintained'.

The Current Maintainer of this work is Niklas Beisert.

This work consists of the files |README.txt|, |childdoc.ins| and |childdoc.dtx|
as well as the derived files |childdoc.def|, |cdocsamp.tex|
with |cdocsch1.tex|, |cdocsch2.tex|, |cdocspt3.tex|, |cdocspt4.tex|,
|cdocsdrf.tex|, |cdocsfn1.tex|, |cdocsfn2.tex|
as well as |childdoc.pdf|.

%%%%%%%%%%%%%%%%%%%%%%%%%%%%%%%%%%%%%%%%%%%%%%%%%%%%%%%%%%%%%%%%%%%%%%%%%%%%%%%%
\subsection{Files and Installation}

The package consists of the files:
%
\begin{center}
\begin{tabular}{ll}
    |README.txt|   & readme file \\
    |childdoc.ins| & installation file \\
    |childdoc.dtx| & source file \\
    |childdoc.def| & definition file \\
    |cdocsamp.tex| & sample main file \\
    |cdocsch1.tex| & sample include file \\
    |cdocsch2.tex| & sample include file \\
    |cdocspt3.tex| & sample part file \\
    |cdocspt4.tex| & sample part file \\
    |cdocsdrf.tex| & sample redirection file \\
    |cdocsfn1.tex| & sample redirection file \\
    |cdocsfn2.tex| & sample redirection file \\
    |childdoc.pdf| & manual
\end{tabular}
\end{center}
%
The distribution consists of the files
|README.txt|, |childdoc.ins| and |childdoc.dtx|.
%
\begin{itemize}
\item
Run (pdf)\LaTeX{} on |childdoc.dtx|
to compile the manual |childdoc.pdf| (this file).
\item
Run \LaTeX{} on |childdoc.ins| to create the definitions file |childdoc.def|
and the sample |cdocsamp.tex| with include files
|cdocsch1.tex|, |cdocsch2.tex|, |cdocspt3.tex|, |cdocspt4.tex|,
|cdocsdrf.tex|, |cdocsfn1.tex|, |cdocsfn2.tex|.
Then copy the file |childdoc.def| to an appropriate directory of your \LaTeX{}
distribution, e.g.\ \textit{texmf-root}|/tex/latex/childdoc|.
\end{itemize}

%%%%%%%%%%%%%%%%%%%%%%%%%%%%%%%%%%%%%%%%%%%%%%%%%%%%%%%%%%%%%%%%%%%%%%%%%%%%%%%%
\subsection{Related CTAN Packages}

There are several other packages which offer a similar functionality:
%
\begin{itemize}
\item
The packages
\href{http://ctan.org/pkg/docmute}{\textsf{docmute}},
\href{http://ctan.org/pkg/includex}{\textsf{includex}} and
\href{http://ctan.org/pkg/standalone}{\textsf{standalone}}
provide commands to include only the document body of
a child file thus allowing both files to be compiled individually.
\item
The packages \href{http://ctan.org/pkg/subdocs}{\textsf{subdocs}}
and \href{http://ctan.org/pkg/subfiles}{\textsf{subfiles}}
provide structures in which the main and child documents can be
encapsulated and allowing them to be compiled individually.
The inclusion mechanism is different from the conventional |\include|.
\item
The package \href{http://ctan.org/pkg/combine}{\textsf{combine}}
is an elaborate solution to combine several documents into one.
\end{itemize}
%
See also the CTAN topic \href{http://ctan.org/topic/subdocs}{\textsf{subdocs}}
for further related packages.
The present package differs from the above solutions in that
a document structure constructed with the conventional |\include| mechanism
just needs two extra commands at the top of every file
such that all constituent files can be compiled individually.

%%%%%%%%%%%%%%%%%%%%%%%%%%%%%%%%%%%%%%%%%%%%%%%%%%%%%%%%%%%%%%%%%%%%%%%%%%%%%%%%
%\subsection{Feature Suggestions}
%
%The following is a list of features which may be useful for future
%versions of this package:
%%
%\begin{itemize}
%\item
%\ldots
%\end{itemize}

%%%%%%%%%%%%%%%%%%%%%%%%%%%%%%%%%%%%%%%%%%%%%%%%%%%%%%%%%%%%%%%%%%%%%%%%%%%%%%%%
\subsection{Revision History}

%%%%%%%%%%%%%%%%%%%%%%%%%%%%%%%%%%%%%%%%
\paragraph{v2.0:} 2018/12/30

\begin{itemize}
\item
immediate forward processing
\item
added |\childdocby| mechanism
\item
manual restructured
\end{itemize}

%%%%%%%%%%%%%%%%%%%%%%%%%%%%%%%%%%%%%%%%
\paragraph{v1.6:} 2018/01/17

\begin{itemize}
\item
application for development of include files
\item
corrections to manual
\end{itemize}

%%%%%%%%%%%%%%%%%%%%%%%%%%%%%%%%%%%%%%%%
\paragraph{v1.5:} 2017/05/21

\begin{itemize}
\item
more complete structuring introduced
\item
|\childdocof| introduced
\item
|\childdoc| renamed to |\childdocmain|
\item
|\childredirect| renamed to |\childdocforward| and |\childdocforwardprefix|
and functionality expanded
\end{itemize}

%%%%%%%%%%%%%%%%%%%%%%%%%%%%%%%%%%%%%%%%
\paragraph{v1.0:} 2017/04/27

\begin{itemize}
\item
manual and install package
\item
first version published on CTAN
\end{itemize}

%%%%%%%%%%%%%%%%%%%%%%%%%%%%%%%%%%%%%%%%
\paragraph{v0.6:} 2017/04/26

\begin{itemize}
\item
redirection mechanism added
\end{itemize}

%%%%%%%%%%%%%%%%%%%%%%%%%%%%%%%%%%%%%%%%
\paragraph{v0.5:} 2017/04/26

\begin{itemize}
\item
functionality in definition file
\end{itemize}


%%%%%%%%%%%%%%%%%%%%%%%%%%%%%%%%%%%%%%%%%%%%%%%%%%%%%%%%%%%%%%%%%%%%%%%%%%%%%%%%
%%%%%%%%%%%%%%%%%%%%%%%%%%%%%%%%%%%%%%%%%%%%%%%%%%%%%%%%%%%%%%%%%%%%%%%%%%%%%%%%
%%%%%%%%%%%%%%%%%%%%%%%%%%%%%%%%%%%%%%%%%%%%%%%%%%%%%%%%%%%%%%%%%%%%%%%%%%%%%%%%
\appendix

\settowidth\MacroIndent{\rmfamily\scriptsize 000\ }

 \DocInput{childdoc.dtx}

\end{document}
%</driver>
% \fi
%
% %%%%%%%%%%%%%%%%%%%%%%%%%%%%%%%%%%%%%%%%%%%%%%%%%%%%%%%%%%%%%%%%%%%%%%%%%%%%%%
% %%%%%%%%%%%%%%%%%%%%%%%%%%%%%%%%%%%%%%%%%%%%%%%%%%%%%%%%%%%%%%%%%%%%%%%%%%%%%%
% \section{Sample}
%\iffalse
%<*samplemain>
%\fi
%
% The following presents a sample document
% with two chapters, two parts, a title page,
% a compile flag as well as three forwarding files to set the flag.
% It consists of eight |.tex| files:
% \begin{center}
% \begin{tabular}{ll}
% |cdocsamp.tex|&main file\\
% |cdocsch1.tex|&include file for chapter 1\\
% |cdocsch2.tex|&include file for chapter 2\\
% |cdocspt3.tex|&include file for part 3\\
% |cdocspt4.tex|&include file for part 4\\
% |cdocsdrf.tex|&forwarding file for main file in draft mode\\
% |cdocsfi1.tex|&forwarding file for final version of chapter 1\\
% |cdocsfi2.tex|&forwarding file for final version of chapter 2\\
% \end{tabular}
% \end{center}
% Each of the eight files can be compiled directly by the \LaTeX{} compiler.
%
% %%%%%%%%%%%%%%%%%%%%%%%%%%%%%%%%%%%%%%
% \paragraph{Main File.}
%
% The main file is called |cdocsamp.tex|.
%
% Load the \textsf{childdoc} definitions and
% declare the filename for the main document:
%    \begin{macrocode}
\input{childdoc.def}
\childdocmain{}
%    \end{macrocode}

% Optional override for |\version| flag:
%    \begin{macrocode}
%%\ifchilddoc\else\providecommand{\version}{draft}\fi
%    \end{macrocode}

% Define the default values for the |\version| flag
% (|final| for the main file and |draft| for childs):
%    \begin{macrocode}
\ifchilddoc
\providecommand{\version}{draft}
\else
\providecommand{\version}{final}
\fi
%    \end{macrocode}

% Load the standard document class:
%    \begin{macrocode}
\documentclass[12pt]{article}
%    \end{macrocode}

% Start the document body:
%    \begin{macrocode}
\begin{document}
%    \end{macrocode}

% Declare a title page.
% Print title, part of document being processed and version flag:
%    \begin{macrocode}
\addtocounter{page}{-1}
\begin{center}
{\LARGE\bfseries{}childdoc example\par}
\vspace{1cm}
\ifchilddoc
\ifchilddocmanual part\else chapter\fi:
`\childdocname' of `\childdocjob'\par
\else
main document: `\childdocjob'\par
\fi
version: \version\par
\end{center}
\newpage
%    \end{macrocode}

% Manually include selected file,
% otherwise process as usual:
%    \begin{macrocode}
\ifchilddocmanual
\section*{part `\childdocname'}
\input{\childdocname}
\else
%    \end{macrocode}

% Include the two chapters:
%    \begin{macrocode}
\include{cdocsch1}
\include{cdocsch2}
%    \end{macrocode}

% Include the two parts unless only chapters should be displayed:
%    \begin{macrocode}
\ifchilddoc\else
\section{part three}
\input{cdocspt3}
\section{part four}
\input{cdocspt4}
\fi
%    \end{macrocode}

% Process as usual until here:
%    \begin{macrocode}
\fi
%    \end{macrocode}

% End of document body:
%    \begin{macrocode}
\end{document}
%    \end{macrocode}
%\iffalse
%</samplemain>
%\fi
%
% %%%%%%%%%%%%%%%%%%%%%%%%%%%%%%%%%%%%%%
% \paragraph{Chapter Include Files.}
%
% The include files are called |cdocsch1.tex| and |cdocsch2.tex|.
%
%\iffalse
%<*samplechap1|samplechap2>
%\fi

% Optional override for |\version| flag:
%    \begin{macrocode}
%%\providecommand{\version}{final}
%    \end{macrocode}

% Include the main document:
%    \begin{macrocode}
\input{childdoc.def}
\childdocof{cdocsamp}
%    \end{macrocode}

%\iffalse
%</samplechap1|samplechap2>
%\fi
%
%\iffalse
%<*samplechap1>
%\fi
% Some text for chapter 1:
%    \begin{macrocode}
\section{one}
some text in chapter one
%    \end{macrocode}

%\iffalse
%</samplechap1>
%\fi
% Some text for chapter 2:
%\iffalse
%<*samplechap2>
%\fi
%    \begin{macrocode}
\section{two}
more text in chapter two
%    \end{macrocode}

%\iffalse
%</samplechap2>
%\fi
%
% %%%%%%%%%%%%%%%%%%%%%%%%%%%%%%%%%%%%%%
% \paragraph{Part Include Files.}
%
% The include files are called |cdocspt3.tex| and |cdocspt4.tex|.
%
%\iffalse
%<*samplepart3|samplepart4>
%\fi

% Optional override for |\version| flag:
%    \begin{macrocode}
%%\providecommand{\version}{final}
%    \end{macrocode}

% Include the main document:
%    \begin{macrocode}
\input{childdoc.def}
\childdocby{cdocsamp}
%    \end{macrocode}

%\iffalse
%</samplepart3|samplepart4>
%\fi
%
%\iffalse
%<*samplepart3>
%\fi
% Some text for part 3:
%    \begin{macrocode}
some text in part three
%    \end{macrocode}

%\iffalse
%</samplepart3>
%\fi
% Some text for part 4:
%\iffalse
%<*samplepart4>
%\fi
%    \begin{macrocode}
more text in part four
%    \end{macrocode}

%\iffalse
%</samplepart4>
%\fi
%
% %%%%%%%%%%%%%%%%%%%%%%%%%%%%%%%%%%%%%%
% \paragraph{Forwarding for a Complete Draft.}
%
% The following forwarding file |cdocsdrf.tex|
% compiles the main document in draft mode:
%\iffalse
%<*sampledraft>
%\fi
%    \begin{macrocode}
\def\version{draft}
\input{childdoc.def}
\childdocforward{cdocsamp}
%    \end{macrocode}

%\iffalse
%</sampledraft>
%\fi
%
% %%%%%%%%%%%%%%%%%%%%%%%%%%%%%%%%%%%%%%
% \paragraph{Forwarding for Final Version of the Chapters.}
%
% The following forwarding files |cdocsfn1.tex| and |cdocsfn2.tex|
% (with identical content)
% compile the final versions of the child documents
% |cdocsch1.tex| and |cdocsch2.tex|, respectively:
%\iffalse
%<*samplefinal>
%\fi
%    \begin{macrocode}
\def\version{final}
\input{childdoc.def}
\childdocforwardprefix[cdocsamp]{cdocsfn}{cdocsch}
%    \end{macrocode}

%\iffalse
%</samplefinal>
%\fi
%
% %%%%%%%%%%%%%%%%%%%%%%%%%%%%%%%%%%%%%%
% \paragraph{Command Line Processing.}
%
% The following three command lines generate the output files
% |cdocscld|, |cdocscl1| and |cdocscl2|
% which should be identical to
% |cdocsdrf|, |cdocsch1| and |cdocsfn2|, respectively:
% \begin{center}
% \begin{tabular}{l}
% |latex -jobname cdocscld \|\\
% |  "\def\version{draft}\input{childdoc.def}\childdocforward{cdocsamp}"|\\
% |latex -jobname cdocscl1 \|\\
% |  "\input{childdoc.def}\childdocforward[cdocsamp]{cdocsch1}"|\\
% |latex -jobname cdocscl2 \|\\
% |  "\def\version{final}\input{childdoc.def}\childdocforward{cdocsch2}"|
% \end{tabular}
% \end{center}
% Note that the trailing backslash on each first line
% merely continues the input to the second line
% (for convenient cut ant paste).
% Furthermore, the command |latex| can be replaced by any
% of its alternative versions such as |pdflatex|.
%
% %%%%%%%%%%%%%%%%%%%%%%%%%%%%%%%%%%%%%%%%%%%%%%%%%%%%%%%%%%%%%%%%%%%%%%%%%%%%%%
% %%%%%%%%%%%%%%%%%%%%%%%%%%%%%%%%%%%%%%%%%%%%%%%%%%%%%%%%%%%%%%%%%%%%%%%%%%%%%%
% \section{Implementation}
%\iffalse
%<*package>
%\fi
%
% This section describes the definitions file |childdoc.def|.

% The definitions cannot be loaded using |\usepackage| or |\RequirePackage|
% which has a mechanism to prevent loading a style file more than once.
% When loading the definitions by means of |\input|
% multiple instances have to be prevented manually:
%\iffalse
%This code needs to be before the `\ProvidesFile' directive
%which is defined at the beginning of this file.
%Therefore it is also placed there and commented out here.
%</package>
%<*discard>
%\fi
%    \begin{macrocode}
\ifdefined\childdocmain\endinput\fi
%    \end{macrocode}
%\iffalse
%</discard>
%<*package>
%\fi
%
% \macro{\ifchilddoc}
% \macro{\ifchilddocmanual}
% The conditional |\ifchilddoc| tells whether a
% child (true) or main (false) document is being compiled.
% The conditional |\ifchilddocmanual| tells whether
% the |\includeonly| mechanism is used (false) or
% the selection of child files must be performed manually (true).
% The definitions initialise to false:
%    \begin{macrocode}
\newif\ifchilddoc
\newif\ifchilddocmanual
%    \end{macrocode}

% \macro{\childdocname}
% \macro{\childdocjob}
% The macro |\childdocname| stores the name of the main document
% to be compiled. The macro |\childdocjob| stores the name of
% the document on which the \LaTeX{} compiler was originally invoked.
% The content of |\jobname| cannot be compared
% to filenames specified in the source due to different catcodes.
% The following code rescans |\jobname|, stores the result
% in |\childdocname| and saves a copy in |\childdocjob|:
%    \begin{macrocode}
\edef\childdocname{\scantokens\expandafter{\jobname\noexpand}}
\let\childdocjob\childdocname
%    \end{macrocode}

% \macro{\childdocdisable}
% The macro |\childdocdisable| prevents the main file
% from being processed more than once.
% At this stage, the main document command |\childdocmain|
% is assumed to be called once again where it should do nothing.
% Any subsequent call to it should prevent
% a secondary processing of the main document
% It overwrites the forwarding commands
% |\childdocof| and |\childdocforward|
% with empty macros to prevent further inclusions of the main document:
%    \begin{macrocode}
\newcommand{\childdocdisable}
{
  \renewcommand{\childdocmain}[1]{\renewcommand{\childdocmain}[1]{\endinput}}
  \renewcommand{\childdocof}[1]{}
  \renewcommand{\childdocby}[2][]{}
  \renewcommand{\childdocforward}[2][]{}
  \renewcommand{\childdocdisable}{}
}
%    \end{macrocode}

% \macro{\childdocmain}
% The macro |\childdocmain| is to be called at the top of the main file
% with nothing or the main filename (without extension) as argument.
% First, it breaks loops.
% If the argument is not empty and does not match |\childdocname|
% (which is set by the first inclusion of |childdoc.def|),
% |\ifchilddoc| is set to true, |\includeonly| is applied to the child file
% and |\jobname| is set to the main file
% (for proper handling of |.aux| files):
%    \begin{macrocode}
\newcommand{\childdocmain}[1]
{
  \childdocdisable\childdocmain{}
  \if?#1?\else
    \begingroup
      \def\childdoctmp{#1}
      \ifx\childdoctmp\childdocname
        \def\childdoctmp{}
      \else
        \def\childdoctmp
        {
          \childdoctrue
          \includeonly{\childdocname}
          \def\childdocjob{#1}
          \def\jobname{#1}
        }
      \fi
      \expandafter
    \endgroup
    \childdoctmp
  \fi
}
%    \end{macrocode}

% \macro{\childdocof}
% The command |\childdocof| redirects
% compilation to the main file |#1|.
%    \begin{macrocode}
\newcommand{\childdocof}[1]
{
  \childdocdisable
  \childdoctrue
  \includeonly{\childdocname}
  \def\jobname{#1}
  \def\childdocjob{#1}
  \input{#1}
}
%    \end{macrocode}

% \macro{\childdocby}
% The command |\childdocby| ....
%    \begin{macrocode}
\newcommand{\childdocby}[2][]
{
  \childdocdisable
  \childdoctrue
  \childdocmanualtrue
  \if?#1?\else
    \def\jobname{#2}
  \fi
  \def\childdocjob{#2}
  \input{#2}
  \endinput
}
%    \end{macrocode}

% \macro{\childdocforward}
% The command |\childdocforward| redirects
% compilation to the main file or
% (if the optional argument is given) a child file.
% Parameters are set as if the main file
% or a child file starting with |\childdocof| was compiled.
% Then compilation is handed over to the main file:
%    \begin{macrocode}
\newcommand{\childdocforward}[2][]
{
  \begingroup
    \if?#1?
      \def\childdoctmp
      {
        \def\childdocname{#2}
        \def\childdocjob{#2}
        \def\jobname{#2}
        \input{#2}
        \endinput
      }
    \else
      \def\childdoctmp
      {
        \childdocdisable
        \def\childdocname{#2}
        \childdoctrue
        \includeonly{#2}
        \def\childdocjob{#1}
        \def\jobname{#1}
        \input{#1}
        \endinput
      }
    \fi
    \expandafter
  \endgroup
  \childdoctmp
}
%    \end{macrocode}

% \macro{\childdocforwardprefix}
% The command |\childdocforwardprefix| redirects
% compilation to the main or a child file by means of a pattern.
% The prefix |#1| in the current filename is replaced by |#2|
% and the suffix of the current filename is kept
% (it is assumed that the filename does not contain the substring `|~~~|'
% which is used as a delimiter).
% Compilation is handed over to the new file by |\childdocforward|:
%    \begin{macrocode}
\newcommand{\childdocforwardprefix}[3][]
{
  \begingroup
    \def\childdocextract #2##1~~~{\def\childdoctmp{\childdocforward[#1]{#3##1}}}
    \expandafter\childdocextract\childdocname~~~
    \expandafter
  \endgroup
  \childdoctmp
}
%    \end{macrocode}

% \macro{\childdoc}
% The deprecated macro |\childdoc| is a legacy version of |\childdocmain|:
%    \begin{macrocode}
\newcommand{\childdoc}{\childdocmain}
%    \end{macrocode}

% \macro{\childdocredirect}
% The deprecated macro |\childdocredirect| is a legacy version
% of |\childdocforward| and |\childdocforwardprefix|:
%    \begin{macrocode}
\newcommand{\childdocredirect}[2][]
{
  \begingroup
    \if?#1?
      \def\childdoctmp{\childdocforward{#2}}
    \else
      \def\childdoctmp{\childdocforwardprefix{#1}{#2}}
    \fi
    \expandafter
  \endgroup
  \childdoctmp
}
%    \end{macrocode}

%\iffalse
%</package>
%\fi
%
\endinput
|\\
|\childdocforwardprefix[|\textit{main}|]{|\textit{prefix}|}{|\textit{dest}|}|
\end{tabular}
\end{center}
%
the destination file is determined by a pattern
depending on the current file:
To make this work, the current file must be called
`{\textit{prefix}\hspace{0.2em}\textit{suffix}}'
with \textit{prefix} matching precisely the argument.
Processing is then passed on to the file
`{\textit{dest}\hspace{0.2em}\textit{suffix}}'.
Surely, the same effect is achieved by
directly specifying the
argument `{\textit{dest}\hspace{0.2em}\textit{suffix}}'
in the first form.
However, that requires to set up a different file
for each child. With the alternative form of the command
all these files can have exactly the same content
which simplifies setting them up and maintaining them.

For example, the following file |draft.tex|
with a compilation flag |\version| as described in \secref{sec:flags}
compiles the main document as a draft:
%
\begin{center}
\begin{tabular}{l}
|\def\version{draft}|\\
|% \iffalse
%
% childdoc.dtx Copyright (C) 2017-2018 Niklas Beisert
%
% This work may be distributed and/or modified under the
% conditions of the LaTeX Project Public License, either version 1.3
% of this license or (at your option) any later version.
% The latest version of this license is in
%   http://www.latex-project.org/lppl.txt
% and version 1.3 or later is part of all distributions of LaTeX
% version 2005/12/01 or later.
%
% This work has the LPPL maintenance status `maintained'.
%
% The Current Maintainer of this work is Niklas Beisert.
%
% This work consists of the files childdoc.dtx and childdoc.ins
% and the derived files childdoc.def and cdocsamp.tex with
% cdocsch1.tex, cdocsch2.tex, cdocsdrf.tex, cdocsfn1.tex, cdocsfn2.tex.
%
%<package>\ifdefined\childdocmain\endinput\fi
%<package>\ProvidesFile{childdoc.def}[2018/12/30 v2.0 child document driver]
%<samplemain>\ProvidesFile{cdocsamp.tex}[2018/12/30 v2.0 sample for childdoc]
%<*driver>
%\ProvidesFile{childdoc.drv}[2018/12/30 v2.0 childdoc reference manual file]
\PassOptionsToClass{10pt,a4paper}{article}
\documentclass{ltxdoc}

\usepackage[margin=35mm]{geometry}
\usepackage{hyperref}
\usepackage{hyperxmp}
\usepackage[usenames]{color}

\hypersetup{colorlinks=true}
\hypersetup{pdfstartview=FitH}
\hypersetup{pdfpagemode=UseNone}
\hypersetup{pdfsource={}}
\hypersetup{pdflang={en-UK}}
\hypersetup{pdfcopyright={Copyright 2017-2018 Niklas Beisert.
  This work may be distributed and/or modified under the
  conditions of the LaTeX Project Public License, either version 1.3
  of this license or (at your option) any later version.}}
\hypersetup{pdflicenseurl={http://www.latex-project.org/lppl.txt}}
\hypersetup{pdfcontactaddress={ETH Zurich, ITP, HIT K,
  Wolfgang-Pauli-Strasse 27}}
\hypersetup{pdfcontactpostcode={8093}}
\hypersetup{pdfcontactcity={Zurich}}
\hypersetup{pdfcontactcountry={Switzerland}}
\hypersetup{pdfcontactemail={nbeisert@itp.phys.ethz.ch}}
\hypersetup{pdfcontacturl={http://people.phys.ethz.ch/\xmptilde nbeisert/}}

\newcommand{\secref}[1]{\hyperref[#1]{section \ref*{#1}}}

\parskip1ex
\parindent0pt
\let\olditemize\itemize
\def\itemize{\olditemize\parskip0pt}

\begin{document}

\title{The \textsf{childdoc} Package}
\hypersetup{pdftitle={The childdoc Package}}
\author{Niklas Beisert\\[2ex]
  Institut f\"ur Theoretische Physik\\
  Eidgen\"ossische Technische Hochschule Z\"urich\\
  Wolfgang-Pauli-Strasse 27, 8093 Z\"urich, Switzerland\\[1ex]
  \href{mailto:nbeisert@itp.phys.ethz.ch}
  {\texttt{nbeisert@itp.phys.ethz.ch}}}
\hypersetup{pdfauthor={Niklas Beisert}}
\hypersetup{pdfsubject={Manual for the LaTeX2e Package childdoc}}
\date{30 December 2018, \textsf{v2.0}}
\maketitle

\begin{abstract}\noindent
\textsf{childdoc} is a \LaTeXe{} package
that enables the direct compilation
of document sections included by |\include|
to individual files.
\end{abstract}

\begingroup
\parskip0ex
\tableofcontents
\endgroup

%%%%%%%%%%%%%%%%%%%%%%%%%%%%%%%%%%%%%%%%%%%%%%%%%%%%%%%%%%%%%%%%%%%%%%%%%%%%%%%%
%%%%%%%%%%%%%%%%%%%%%%%%%%%%%%%%%%%%%%%%%%%%%%%%%%%%%%%%%%%%%%%%%%%%%%%%%%%%%%%%
\section{Introduction}

\LaTeX{} provides a mechanism to structure a large document (such as a book)
into a main file and several child files (containing the chapters)
using the |\include| command.
This mechanism is beneficial for documents
which span hundreds of pages in order to
make the source file(s) more manageable.
Moreover, compilation can be restricted to
selected child files by means of the |\includeonly| command.
The latter feature can be used to reduce the compilation time while editing
(this was significantly more useful in the earlier days of \LaTeX{})
or to generate a smaller document which is easier to navigate.
Another application of |\includeonly| is to generate
documents consisting of selected parts of the complete document.

However, there are a few drawbacks of the plain |\include| mechanism:
\begin{itemize}
\item
The child files cannot be compiled on their own,
they can only be compiled via the main file.
A naive editing environment
(such as a text editor with an option
to have the current file processed by \LaTeX)
may require one to switch to the main file before compiling;
attempting to compile the child file produces errors.
\item
The main file must be modified (each time)
to adjust the |\includeonly| command
to the present needs. This easily leaves the main file in a messy state.
\item
The generated document will always carry the filename
of the main document. This is inconvenient if
several child files are to be compiled and
to be kept for distribution.
\end{itemize}

The present package provides a simple interface
to make child files individually compilable by \LaTeX{}.
Compiling a child file then has the same effect as compiling
the main file with an |\includeonly| command
to select the appropriate child.
Moreover the generated document will carry the name of the child
rather than the main file.
This resolves all three above issues.

This feature is meant to make the editing of books,
thesis documents and lecture notes somewhat more convenient.
However, the package can also be used efficiently for
composing a series of documents (such as exercise sheets)
which are typically distributed individually.
It then assists the author in generating the individual documents
(potentially in different versions)
as well as a document containing the collected series.
Another application is in developing style files
or other kinds of included material
where compilation of the style file could redirect
to a sample or test file.

%%%%%%%%%%%%%%%%%%%%%%%%%%%%%%%%%%%%%%%%%%%%%%%%%%%%%%%%%%%%%%%%%%%%%%%%%%%%%%%%
%%%%%%%%%%%%%%%%%%%%%%%%%%%%%%%%%%%%%%%%%%%%%%%%%%%%%%%%%%%%%%%%%%%%%%%%%%%%%%%%
\section{Usage}

First of all, the package \textsf{childdoc} is \emph{not} a standard
\LaTeXe{} |.sty| style file! Therefore it needs to be invoked in
a non-standard way.

%%%%%%%%%%%%%%%%%%%%%%%%%%%%%%%%%%%%%%%%%%%%%%%%%%%%%%%%%%%%%%%%%%%%%%%%%%%%%%%%
\subsection{Included Files}
\label{sec:include}

%%%%%%%%%%%%%%%%%%%%%%%%%%%%%%%%%%%%%%%%
\DescribeMacro{\childdocmain}
To use the package, add the commands
\begin{center}
\begin{tabular}{l}
|\input{childdoc.def}|\\
|\childdocmain{}|\\
\end{tabular}
\end{center}
at the very top of the main \LaTeX{} file,
in particular \emph{before} the |\documentclass| statement!
The argument of |\childdocmain| should be left empty
(but it must be present).

%%%%%%%%%%%%%%%%%%%%%%%%%%%%%%%%%%%%%%%%
\DescribeMacro{\childdocof}
Furthermore, add the commands
\begin{center}
\begin{tabular}{l}
|\input{childdoc.def}|\\
|\childdocof{|\textit{main}|}|\\
\end{tabular}
\end{center}
at the top of every child file \textit{child}
which is included by |\include{|\textit{child}|}|
from within the main file
(or at least for those files to be compiled individually).
The argument \textit{main} must be the filename of the main file.

There are a couple of
considerations in setting up the main and child documents:

%%%%%%%%%%%%%%%%%%%%%%%%%%%%%%%%%%%%%%%%
\paragraph{Restrictions.}

Please note the following restrictions:
\begin{itemize}
\item
|\childdocmain| must be called with one argument \textit{main}
to ensure compatibility with earlier version of the package.
It must either be empty (|\childdocmain{}|)
or precisely match the filename of the main file in which it is specified.
See \secref{sec:detection} for further information.
\item
The filename \textit{main} must be specified without the |.tex| extension.
\item
The filename \textit{main} is case sensitive
(even in case-insensitive file systems)
due to internal string comparison.
\item
The argument \textit{main} should be fully expanded, it cannot be a macro.
\item
Subdirectories and special characters should be avoided in filenames.
\item
The command |\childdocmain{|\textit{main}|}| must be followed by a whitespace.
It should not be followed immediately by another command
or by a comment mark `|%|'.
This is because the \TeX{} parser reads the token immediately following
the argument of |\childdocmain| and puts it
at the beginning of every child section;
however, a white\-space is ignored.
\end{itemize}

%%%%%%%%%%%%%%%%%%%%%%%%%%%%%%%%%%%%%%%%
\paragraph{Content of Main File.}

It is advisable to place all content in the child files included by |\include|.
Any output contained in the main file will appear in all child documents
unless suppressed manually;
it cannot be suppressed automatically by the |\includeonly| directive
and thus should normally be avoided.
A method to include some content in the main file
by means of conditional processing is described in \secref{sec:conditional}.

%%%%%%%%%%%%%%%%%%%%%%%%%%%%%%%%%%%%%%%%
\paragraph{Page Numbering.}

When only a part of the document is compiled,
the appropriate numbering of pages
(as well as other status parameters)
is determined from the |.aux| files.
The latter contain information from previous passes.
However this information needs to propagate through
all intermediate child documents.
Therefore the page numbering in child documents may well
be inconsistent until the complete document is compiled at least once.

A useful (if unconventional) way to always ensure a consistent
page numbering is to restart the numbering in each child document
and denote the pages by `\textit{child}|.|\textit{page}'
where \textit{child} represents the chapter/section number of the child file.
This can be achieved by the command
|\numberwithin{page}{|\textit{child}|}|
of the \textsf{amsmath} package
where \textit{child} can be |chapter| or |section|
depending on the chosen structuring.
Alternatively, one can modify the macro |\thepage| appropriately
and reset the counter |page| at the start of each child file.

%%%%%%%%%%%%%%%%%%%%%%%%%%%%%%%%%%%%%%%%%%%%%%%%%%%%%%%%%%%%%%%%%%%%%%%%%%%%%%%%
\subsection{Conditional Processing}
\label{sec:conditional}

The package provides a mechanism to compile different versions
of a document. To customise the versions further some conditional processing
can come in handy to distinguish which version is being compiled.
The package provides two macros to describe the compilation context:

%%%%%%%%%%%%%%%%%%%%%%%%%%%%%%%%%%%%%%%%
\DescribeMacro{\ifchilddoc}
The conditional |\ifchilddoc| distinguishes between the compilation of
child documents and the main document:
%
\begin{center}
|\ifchilddoc |\textit{child-code}| |[|\||else |\textit{main-code}]| \||fi|
\end{center}

%%%%%%%%%%%%%%%%%%%%%%%%%%%%%%%%%%%%%%%%
\DescribeMacro{\childdocname}
\DescribeMacro{\childdocjob}
The macro |\childdocname| contains the filename (without extension)
of the main or child file being processed.
Note that |\childdocjob| will always contain the name of the main file.

%%%%%%%%%%%%%%%%%%%%%%%%%%%%%%%%%%%%%%%%
\paragraph{Title Page.}

Conditional processing can be used to include a title or banner page
in the main document when proper precautions are taken.
Importantly, the code in the main file should ensure that the page counter
(as well as other status parameters which are stored in the |.aux| files)
takes the same value after the conditional processing.
Otherwise the page numbers may take divergent values
depending on which part is compiled.

For example, a title page could be declared by:
%
\begin{center}
\begin{tabular}{l}
|\ifchilddoc\||else|\\
|\addtocounter{page}{-1}|\\
\textit{code for title page}\\
|\newpage|\\
|\||fi|
\end{tabular}
\end{center}
%
A banner page for the child documents can be generated by:
%
\begin{center}
\begin{tabular}{l}
|\ifchilddoc|\\
|\addtocounter{page}{-1}|\\
\textit{code for banner page}\\
|\newpage|\\
|\||fi|
\end{tabular}
\end{center}
%
Here one could write a message such as:
\begin{center}
|This is the part \childdocname{} of \childdocjob{}.|
\end{center}

%%%%%%%%%%%%%%%%%%%%%%%%%%%%%%%%%%%%%%%%%%%%%%%%%%%%%%%%%%%%%%%%%%%%%%%%%%%%%%%%
\subsection{Flags}
\label{sec:flags}

The package makes it easy to generate different versions
of the main or child documents.
To this end compilation flags can be defined
and assigned different default values.
They will be particularly useful in conjunction
with the forwarding mechanism described in \secref{sec:forward}.

For example, it may be useful to have a flag |\version|
which can be set to |draft| or |final|.
The document source will contain some conditional code
depending on the value of |\version|.
Suppose further, the flag should default to |final| for the main file
and to |draft| for child files
which is a natural assignment for editing the document.
This is achieved by placing the following code
in the preamble of the main document
(below the |\childdocmain| directive):
%
\begin{center}
\begin{tabular}{l}
|\ifchilddoc|\\
|\providecommand{\version}{draft}|\\
|\||else|\\
|\providecommand{\version}{final}|\\
|\||fi|
\end{tabular}
\end{center}
%
The definition by |\providecommand| makes sure
that previous definitions are not overwritten.
Further statements |\providecommand{\version}{...}|
can thus be added before the above code to override it.

For the main file, one might add a line
(between |\childdocmain| and the above block)
%
\begin{center}
|%\ifchilddoc\||else\providecommand{\version}{draft}\||fi|
\end{center}
%
which can be uncommented to produce a draft version.
Likewise one can add a line to the very top of a child file
(above the |\childdocof{|\textit{main}|}| directive)
%
\begin{center}
|%\providecommand{\version}{final}|
\end{center}
%
which can be uncommented to produce the final version of this child document.

%%%%%%%%%%%%%%%%%%%%%%%%%%%%%%%%%%%%%%%%%%%%%%%%%%%%%%%%%%%%%%%%%%%%%%%%%%%%%%%%
\subsection{Forwarding}
\label{sec:forward}

Different versions of the main or child documents
using compilation flags as described in \secref{sec:flags}
can be (permanently) stored in different files
for convenient compilation, viewing and distribution.
To this end, the package defines a command
to pass on compilation to a different file:

%%%%%%%%%%%%%%%%%%%%%%%%%%%%%%%%%%%%%%%%
\DescribeMacro{\childdocforward}
The command |\childdocforward| redirects processing to
another source file:
%
\begin{center}
\begin{tabular}{l}
|\input{childdoc.def}|\\
|\childdocforward[|\textit{main}|]{|\textit{dest}|}|\\
\end{tabular}
\end{center}
%
The argument \textit{dest} is the destination file
(without extension).
It should be the main file or one of the child files.
Note that further \textsf{childdoc} directives
such as |\childdocof| and |\childdocforward|
in the indicated file will be processed in this form.
The optional argument \textit{main}
passes on directly to the main file \textit{main}
while pretending to compile the child \textit{dest}.
This form behaves as if \textit{dest}
issues |\childdocof{|\textit{main}|}| right away,
and no further \textsf{childdoc} directives will be processed.

%%%%%%%%%%%%%%%%%%%%%%%%%%%%%%%%%%%%%%%%
\DescribeMacro{\...prefix}
In the alternative form |\childdocforwardprefix|,
%
\begin{center}
\begin{tabular}{l}
|\input{childdoc.def}|\\
|\childdocforwardprefix[|\textit{main}|]{|\textit{prefix}|}{|\textit{dest}|}|
\end{tabular}
\end{center}
%
the destination file is determined by a pattern
depending on the current file:
To make this work, the current file must be called
`{\textit{prefix}\hspace{0.2em}\textit{suffix}}'
with \textit{prefix} matching precisely the argument.
Processing is then passed on to the file
`{\textit{dest}\hspace{0.2em}\textit{suffix}}'.
Surely, the same effect is achieved by
directly specifying the
argument `{\textit{dest}\hspace{0.2em}\textit{suffix}}'
in the first form.
However, that requires to set up a different file
for each child. With the alternative form of the command
all these files can have exactly the same content
which simplifies setting them up and maintaining them.

For example, the following file |draft.tex|
with a compilation flag |\version| as described in \secref{sec:flags}
compiles the main document as a draft:
%
\begin{center}
\begin{tabular}{l}
|\def\version{draft}|\\
|\input{childdoc.def}|\\
|\childdocforward{|\textit{main}|}|
\end{tabular}
\end{center}
%
Likewise, the following files |final|\textit{nn}|.tex|
compile the final version of the child document
|child|\textit{nn}|.tex|:
%
\begin{center}
\begin{tabular}{l}
|\def\version{final}|\\
|\input{childdoc.def}|\\
|\childdocforwardprefix{final}{child}|
\end{tabular}
\end{center}
%

Note that when several versions of a main file and/or of each child file
are to be generated, it may be convenient to set up a |Makefile| or
shell script to automatise the process.

%%%%%%%%%%%%%%%%%%%%%%%%%%%%%%%%%%%%%%%%%%%%%%%%%%%%%%%%%%%%%%%%%%%%%%%%%%%%%%%%
\subsection{Command Line Processing}
\label{sec:commandline}

The effect of redirection files can also be achieved by invoking
the \LaTeX{} compiler with a more elaborate command line.
Most conveniently this should be done as part
of a shell script or a |Makefile|.

When using \textsf{childdoc} in the main file, the following
command lines effectively perform a redirection
(note that depending on the shell being used,
backslashes may have to be doubled: `|\|' $\to$ `|\\|'):
%
\begin{center}
|... -jobname "|\textit{target}|" |\\|"|[\textit{flags}]%
|\input{childdoc.def}\childdocforward[|\textit{main}|]{|\textit{dest}|}"|
\end{center}
%
Here \textit{target} is the name of the output file,
\textit{main} is the name of the main file
and \textit{dest} is the name of the main or child file to be processed
(all filenames without extensions).
The optional argument \textit{main} can be omitted
if \textit{main} matches \textit{dest}.
Optionally, compilation \textit{flags} can be defined via |\def| commands.
This command line makes the \TeX{} engine believe
it is compiling the file \textit{target}
whose content is specified as the latter parameter.
The provided code then forwards the processing to
\textit{main} or \textit{dest} as described in \secref{sec:forward}.

%%%%%%%%%%%%%%%%%%%%%%%%%%%%%%%%%%%%%%%%%%%%%%%%%%%%%%%%%%%%%%%%%%%%%%%%%%%%%%%%
\subsection{Include by Input}
\label{sec:input}

Including child documents by |\include| has some restrictions by design.
Most notably, the content of a child document always occupies
its own set of pages; pages cannot be shared between child documents.
Usually, this behaviour makes perfect sense
because each child document contain an essential part of the document.
However, in some situations it may be desirable to compose
a document from a collection of parts
without having mandatory page breaks between then.
For this case, the package
provides a mechanism to include parts
by |\input| which can also be processed individually.
However, by construction this mechanism
requires manual handling of the content to be output.

%%%%%%%%%%%%%%%%%%%%%%%%%%%%%%%%%%%%%%%%
\DescribeMacro{\ifchilddocmanual}
The main file should be prepared as usual, see \secref{sec:include}.
However, the document body must make a distinction
between processing of an individual part and of the main document, e.g.:
%
\begin{center}
\begin{tabular}{l}
|\ifchilddocmanual|\\
|\input{\childdocname}|\\
|\||else|\\
\textit{document body with }|\input{|\textit{part}|}|\\
|\||fi|
\end{tabular}
\end{center}
%
The conditional |\ifchilddocmanual| is true whenever
a part to be included by |\input| is being compiled,
and the name of the part is stored in |\childdocname|.

%%%%%%%%%%%%%%%%%%%%%%%%%%%%%%%%%%%%%%%%
\DescribeMacro{\childdocby}
Each part to be included by |\input| should start with:
%
\begin{center}
\begin{tabular}{l}
|\input{childdoc.def}|\\
|\childdocby{|\textit{main}|}|\\
\end{tabular}
\end{center}
%
The directive |\childdocby| is similar to |\childdocof|
described in \secref{sec:include},
but the subsequent selection of content must be done manually.
To that end, both |\ifchilddoc| and |\ifchilddocmanual|
will be true upon processing of a part,
and the name of the part is stored in |\childdocname|.
Note that |\jobname| will be set to the filename of the current part
so that each part receives an individual |.aux| file
that does not interfere with the |.aux| file(s) of the main document.
This behaviour can be altered by the alternative form
|\childdocby[*]{|\textit{main}|}| (with a non-empty optional argument)
which uses the |.aux| file of the main document
by setting |\jobname| to \textit{main}.

%%%%%%%%%%%%%%%%%%%%%%%%%%%%%%%%%%%%%%%%%%%%%%%%%%%%%%%%%%%%%%%%%%%%%%%%%%%%%%%%
\subsection{Driver Development}
\label{sec:driver}

The \textsf{childdoc} mechanism can also be use for the development
of definition files such as \LaTeX{} styles or classes.
This case differs from the above setup with multiple parts
included by |\include| in that no |\includeonly| should be invoked.
This can be achieved by starting the include file
(before |\ProvidesPackage|) with:
%
\begin{center}
\begin{tabular}{l}
|\input{childdoc.def}|\\
|\childdocforward{|\textit{main}|}|\\
\end{tabular}
\end{center}
%
or alternatively with:
%
\begin{center}
\begin{tabular}{l}
|\input{childdoc.def}|\\
|\childdocby{|\textit{main}|}|\\
\end{tabular}
\end{center}
%
Both forms have slightly different effects as described above.
The main file is prepared as usual, see \secref{sec:include}.

%%%%%%%%%%%%%%%%%%%%%%%%%%%%%%%%%%%%%%%%%%%%%%%%%%%%%%%%%%%%%%%%%%%%%%%%%%%%%%%%
\subsection{Legacy Detection}
\label{sec:detection}

The directive |\childdocmain| in the main file can detect
whether the complete document or merely a child is to be compiled
even without using the directive |\childdocof|.
This method is deprecated because it is less robust
and there is no compelling reason to use it;
it is merely provided for backward compatibility
and it may be removed in future versions.

If the detection mechanism is to be used,
it is mandatory to correctly specify
the filename of the main file as the argument of |\childdocmain|:
%
\begin{center}
\begin{tabular}{l}
|\input{childdoc.def}|\\
|\childdocmain{|\textit{main}|}|\\
\end{tabular}
\end{center}
%
If |\jobname| does not match the argument \textit{main} of |\childdocmain|,
it is assumed that |\jobname| points to the child file to be compiled.
When using |\childdocmain| with the main file specified as argument,
it suffices to start a child file
with just |\input{|\textit{main}|}|
without loading of the package and using |\childdocof|.
If instead all processing is done
with the appropriate \textsf{childdoc} directives,
the argument of \textit{main} of |\childdocmain| can be empty.

An alternative version of the command line processing described
in \secref{sec:commandline} using the detection mechanism reads:
%
\begin{center}
|... -jobname "|\textit{target}|" "|[\textit{flags}]%
[|\def\jobname{|\textit{dest}|}|]|\input{|\textit{main}|}"|
\end{center}

%%%%%%%%%%%%%%%%%%%%%%%%%%%%%%%%%%%%%%%%%%%%%%%%%%%%%%%%%%%%%%%%%%%%%%%%%%%%%%%%
\subsection{Manual Code}
\label{sec:manual}

In case one cannot be certain whether the definitions file |childdoc.def|
is installed on the target \TeX{} distribution
and one prefers not to ship it,
it is conceivable to paste a few relevant commands into the sources.

To that end, drop all statements |\input{childdoc.def}|
and perform the replacements as outlined below.
Instead of |\childdocmain{|\textit{main}|}| add the following code
to the top of the main file:
%
\begin{center}
\begin{tabular}{l}
|\||ifdefined\childdocname\endinput\||fi\newif\ifchilddoc|\\
|\edef\childdocname{\scantokens\expandafter{\jobname\noexpand}}|\\
|\def\childdocmain{|\textit{main}|}\||ifx\childdocmain\childdocname\||else|\\
|\childdoctrue\includeonly{\childdocname}\let\jobname\childdocmain\||fi|\\
\end{tabular}
\end{center}
%
Instead of |\childdocof{|\textit{main}|}| just include the main file
at the top of each child file:
%
\begin{center}
|\input{|\textit{main}|}|
\end{center}
%
A simple redirection |\childdocforward{|\textit{dest}|}| is achieved by:
%
\begin{center}
|\def\jobname{|\textit{dest}|}\input{\jobname}|
\end{center}
%
The redirection with prefix
|\childdocforwardprefix[|\textit{prefix}|]{|\textit{dest}|}|
is accomplished by:
%
\begin{center}
\begin{tabular}{l}
|{\edef\jobname{\scantokens\expandafter{\jobname\noexpand}}|\\
|\def\redirectjob |\textit{prefix}|#1~~~{\gdef\jobname{|\textit{dest}|#1}}|\\
|\expandafter\redirectjob\jobname~~~}\input{\jobname}|
\end{tabular}
\end{center}

In an alternative approach,
child documents can be compiled by a specific command line
without additional code or specific definitions:
%
\begin{center}
|... -jobname "|\textit{target}|" "|[\textit{flags}]%
|\includeonly{|\textit{dest}|}\input{|\textit{main}|}"|
\end{center}
%

%%%%%%%%%%%%%%%%%%%%%%%%%%%%%%%%%%%%%%%%%%%%%%%%%%%%%%%%%%%%%%%%%%%%%%%%%%%%%%%%
%%%%%%%%%%%%%%%%%%%%%%%%%%%%%%%%%%%%%%%%%%%%%%%%%%%%%%%%%%%%%%%%%%%%%%%%%%%%%%%%
\section{Information}

%%%%%%%%%%%%%%%%%%%%%%%%%%%%%%%%%%%%%%%%%%%%%%%%%%%%%%%%%%%%%%%%%%%%%%%%%%%%%%%%
\subsection{Copyright}

Copyright \copyright{} 2017--2018 Niklas Beisert

This work may be distributed and/or modified under the
conditions of the \LaTeX{} Project Public License, either version 1.3
of this license or (at your option) any later version.
The latest version of this license is in
  \url{http://www.latex-project.org/lppl.txt}
and version 1.3 or later is part of all distributions of \LaTeX{}
version 2005/12/01 or later.

This work has the LPPL maintenance status `maintained'.

The Current Maintainer of this work is Niklas Beisert.

This work consists of the files |README.txt|, |childdoc.ins| and |childdoc.dtx|
as well as the derived files |childdoc.def|, |cdocsamp.tex|
with |cdocsch1.tex|, |cdocsch2.tex|, |cdocspt3.tex|, |cdocspt4.tex|,
|cdocsdrf.tex|, |cdocsfn1.tex|, |cdocsfn2.tex|
as well as |childdoc.pdf|.

%%%%%%%%%%%%%%%%%%%%%%%%%%%%%%%%%%%%%%%%%%%%%%%%%%%%%%%%%%%%%%%%%%%%%%%%%%%%%%%%
\subsection{Files and Installation}

The package consists of the files:
%
\begin{center}
\begin{tabular}{ll}
    |README.txt|   & readme file \\
    |childdoc.ins| & installation file \\
    |childdoc.dtx| & source file \\
    |childdoc.def| & definition file \\
    |cdocsamp.tex| & sample main file \\
    |cdocsch1.tex| & sample include file \\
    |cdocsch2.tex| & sample include file \\
    |cdocspt3.tex| & sample part file \\
    |cdocspt4.tex| & sample part file \\
    |cdocsdrf.tex| & sample redirection file \\
    |cdocsfn1.tex| & sample redirection file \\
    |cdocsfn2.tex| & sample redirection file \\
    |childdoc.pdf| & manual
\end{tabular}
\end{center}
%
The distribution consists of the files
|README.txt|, |childdoc.ins| and |childdoc.dtx|.
%
\begin{itemize}
\item
Run (pdf)\LaTeX{} on |childdoc.dtx|
to compile the manual |childdoc.pdf| (this file).
\item
Run \LaTeX{} on |childdoc.ins| to create the definitions file |childdoc.def|
and the sample |cdocsamp.tex| with include files
|cdocsch1.tex|, |cdocsch2.tex|, |cdocspt3.tex|, |cdocspt4.tex|,
|cdocsdrf.tex|, |cdocsfn1.tex|, |cdocsfn2.tex|.
Then copy the file |childdoc.def| to an appropriate directory of your \LaTeX{}
distribution, e.g.\ \textit{texmf-root}|/tex/latex/childdoc|.
\end{itemize}

%%%%%%%%%%%%%%%%%%%%%%%%%%%%%%%%%%%%%%%%%%%%%%%%%%%%%%%%%%%%%%%%%%%%%%%%%%%%%%%%
\subsection{Related CTAN Packages}

There are several other packages which offer a similar functionality:
%
\begin{itemize}
\item
The packages
\href{http://ctan.org/pkg/docmute}{\textsf{docmute}},
\href{http://ctan.org/pkg/includex}{\textsf{includex}} and
\href{http://ctan.org/pkg/standalone}{\textsf{standalone}}
provide commands to include only the document body of
a child file thus allowing both files to be compiled individually.
\item
The packages \href{http://ctan.org/pkg/subdocs}{\textsf{subdocs}}
and \href{http://ctan.org/pkg/subfiles}{\textsf{subfiles}}
provide structures in which the main and child documents can be
encapsulated and allowing them to be compiled individually.
The inclusion mechanism is different from the conventional |\include|.
\item
The package \href{http://ctan.org/pkg/combine}{\textsf{combine}}
is an elaborate solution to combine several documents into one.
\end{itemize}
%
See also the CTAN topic \href{http://ctan.org/topic/subdocs}{\textsf{subdocs}}
for further related packages.
The present package differs from the above solutions in that
a document structure constructed with the conventional |\include| mechanism
just needs two extra commands at the top of every file
such that all constituent files can be compiled individually.

%%%%%%%%%%%%%%%%%%%%%%%%%%%%%%%%%%%%%%%%%%%%%%%%%%%%%%%%%%%%%%%%%%%%%%%%%%%%%%%%
%\subsection{Feature Suggestions}
%
%The following is a list of features which may be useful for future
%versions of this package:
%%
%\begin{itemize}
%\item
%\ldots
%\end{itemize}

%%%%%%%%%%%%%%%%%%%%%%%%%%%%%%%%%%%%%%%%%%%%%%%%%%%%%%%%%%%%%%%%%%%%%%%%%%%%%%%%
\subsection{Revision History}

%%%%%%%%%%%%%%%%%%%%%%%%%%%%%%%%%%%%%%%%
\paragraph{v2.0:} 2018/12/30

\begin{itemize}
\item
immediate forward processing
\item
added |\childdocby| mechanism
\item
manual restructured
\end{itemize}

%%%%%%%%%%%%%%%%%%%%%%%%%%%%%%%%%%%%%%%%
\paragraph{v1.6:} 2018/01/17

\begin{itemize}
\item
application for development of include files
\item
corrections to manual
\end{itemize}

%%%%%%%%%%%%%%%%%%%%%%%%%%%%%%%%%%%%%%%%
\paragraph{v1.5:} 2017/05/21

\begin{itemize}
\item
more complete structuring introduced
\item
|\childdocof| introduced
\item
|\childdoc| renamed to |\childdocmain|
\item
|\childredirect| renamed to |\childdocforward| and |\childdocforwardprefix|
and functionality expanded
\end{itemize}

%%%%%%%%%%%%%%%%%%%%%%%%%%%%%%%%%%%%%%%%
\paragraph{v1.0:} 2017/04/27

\begin{itemize}
\item
manual and install package
\item
first version published on CTAN
\end{itemize}

%%%%%%%%%%%%%%%%%%%%%%%%%%%%%%%%%%%%%%%%
\paragraph{v0.6:} 2017/04/26

\begin{itemize}
\item
redirection mechanism added
\end{itemize}

%%%%%%%%%%%%%%%%%%%%%%%%%%%%%%%%%%%%%%%%
\paragraph{v0.5:} 2017/04/26

\begin{itemize}
\item
functionality in definition file
\end{itemize}


%%%%%%%%%%%%%%%%%%%%%%%%%%%%%%%%%%%%%%%%%%%%%%%%%%%%%%%%%%%%%%%%%%%%%%%%%%%%%%%%
%%%%%%%%%%%%%%%%%%%%%%%%%%%%%%%%%%%%%%%%%%%%%%%%%%%%%%%%%%%%%%%%%%%%%%%%%%%%%%%%
%%%%%%%%%%%%%%%%%%%%%%%%%%%%%%%%%%%%%%%%%%%%%%%%%%%%%%%%%%%%%%%%%%%%%%%%%%%%%%%%
\appendix

\settowidth\MacroIndent{\rmfamily\scriptsize 000\ }

 \DocInput{childdoc.dtx}

\end{document}
%</driver>
% \fi
%
% %%%%%%%%%%%%%%%%%%%%%%%%%%%%%%%%%%%%%%%%%%%%%%%%%%%%%%%%%%%%%%%%%%%%%%%%%%%%%%
% %%%%%%%%%%%%%%%%%%%%%%%%%%%%%%%%%%%%%%%%%%%%%%%%%%%%%%%%%%%%%%%%%%%%%%%%%%%%%%
% \section{Sample}
%\iffalse
%<*samplemain>
%\fi
%
% The following presents a sample document
% with two chapters, two parts, a title page,
% a compile flag as well as three forwarding files to set the flag.
% It consists of eight |.tex| files:
% \begin{center}
% \begin{tabular}{ll}
% |cdocsamp.tex|&main file\\
% |cdocsch1.tex|&include file for chapter 1\\
% |cdocsch2.tex|&include file for chapter 2\\
% |cdocspt3.tex|&include file for part 3\\
% |cdocspt4.tex|&include file for part 4\\
% |cdocsdrf.tex|&forwarding file for main file in draft mode\\
% |cdocsfi1.tex|&forwarding file for final version of chapter 1\\
% |cdocsfi2.tex|&forwarding file for final version of chapter 2\\
% \end{tabular}
% \end{center}
% Each of the eight files can be compiled directly by the \LaTeX{} compiler.
%
% %%%%%%%%%%%%%%%%%%%%%%%%%%%%%%%%%%%%%%
% \paragraph{Main File.}
%
% The main file is called |cdocsamp.tex|.
%
% Load the \textsf{childdoc} definitions and
% declare the filename for the main document:
%    \begin{macrocode}
\input{childdoc.def}
\childdocmain{}
%    \end{macrocode}

% Optional override for |\version| flag:
%    \begin{macrocode}
%%\ifchilddoc\else\providecommand{\version}{draft}\fi
%    \end{macrocode}

% Define the default values for the |\version| flag
% (|final| for the main file and |draft| for childs):
%    \begin{macrocode}
\ifchilddoc
\providecommand{\version}{draft}
\else
\providecommand{\version}{final}
\fi
%    \end{macrocode}

% Load the standard document class:
%    \begin{macrocode}
\documentclass[12pt]{article}
%    \end{macrocode}

% Start the document body:
%    \begin{macrocode}
\begin{document}
%    \end{macrocode}

% Declare a title page.
% Print title, part of document being processed and version flag:
%    \begin{macrocode}
\addtocounter{page}{-1}
\begin{center}
{\LARGE\bfseries{}childdoc example\par}
\vspace{1cm}
\ifchilddoc
\ifchilddocmanual part\else chapter\fi:
`\childdocname' of `\childdocjob'\par
\else
main document: `\childdocjob'\par
\fi
version: \version\par
\end{center}
\newpage
%    \end{macrocode}

% Manually include selected file,
% otherwise process as usual:
%    \begin{macrocode}
\ifchilddocmanual
\section*{part `\childdocname'}
\input{\childdocname}
\else
%    \end{macrocode}

% Include the two chapters:
%    \begin{macrocode}
\include{cdocsch1}
\include{cdocsch2}
%    \end{macrocode}

% Include the two parts unless only chapters should be displayed:
%    \begin{macrocode}
\ifchilddoc\else
\section{part three}
\input{cdocspt3}
\section{part four}
\input{cdocspt4}
\fi
%    \end{macrocode}

% Process as usual until here:
%    \begin{macrocode}
\fi
%    \end{macrocode}

% End of document body:
%    \begin{macrocode}
\end{document}
%    \end{macrocode}
%\iffalse
%</samplemain>
%\fi
%
% %%%%%%%%%%%%%%%%%%%%%%%%%%%%%%%%%%%%%%
% \paragraph{Chapter Include Files.}
%
% The include files are called |cdocsch1.tex| and |cdocsch2.tex|.
%
%\iffalse
%<*samplechap1|samplechap2>
%\fi

% Optional override for |\version| flag:
%    \begin{macrocode}
%%\providecommand{\version}{final}
%    \end{macrocode}

% Include the main document:
%    \begin{macrocode}
\input{childdoc.def}
\childdocof{cdocsamp}
%    \end{macrocode}

%\iffalse
%</samplechap1|samplechap2>
%\fi
%
%\iffalse
%<*samplechap1>
%\fi
% Some text for chapter 1:
%    \begin{macrocode}
\section{one}
some text in chapter one
%    \end{macrocode}

%\iffalse
%</samplechap1>
%\fi
% Some text for chapter 2:
%\iffalse
%<*samplechap2>
%\fi
%    \begin{macrocode}
\section{two}
more text in chapter two
%    \end{macrocode}

%\iffalse
%</samplechap2>
%\fi
%
% %%%%%%%%%%%%%%%%%%%%%%%%%%%%%%%%%%%%%%
% \paragraph{Part Include Files.}
%
% The include files are called |cdocspt3.tex| and |cdocspt4.tex|.
%
%\iffalse
%<*samplepart3|samplepart4>
%\fi

% Optional override for |\version| flag:
%    \begin{macrocode}
%%\providecommand{\version}{final}
%    \end{macrocode}

% Include the main document:
%    \begin{macrocode}
\input{childdoc.def}
\childdocby{cdocsamp}
%    \end{macrocode}

%\iffalse
%</samplepart3|samplepart4>
%\fi
%
%\iffalse
%<*samplepart3>
%\fi
% Some text for part 3:
%    \begin{macrocode}
some text in part three
%    \end{macrocode}

%\iffalse
%</samplepart3>
%\fi
% Some text for part 4:
%\iffalse
%<*samplepart4>
%\fi
%    \begin{macrocode}
more text in part four
%    \end{macrocode}

%\iffalse
%</samplepart4>
%\fi
%
% %%%%%%%%%%%%%%%%%%%%%%%%%%%%%%%%%%%%%%
% \paragraph{Forwarding for a Complete Draft.}
%
% The following forwarding file |cdocsdrf.tex|
% compiles the main document in draft mode:
%\iffalse
%<*sampledraft>
%\fi
%    \begin{macrocode}
\def\version{draft}
\input{childdoc.def}
\childdocforward{cdocsamp}
%    \end{macrocode}

%\iffalse
%</sampledraft>
%\fi
%
% %%%%%%%%%%%%%%%%%%%%%%%%%%%%%%%%%%%%%%
% \paragraph{Forwarding for Final Version of the Chapters.}
%
% The following forwarding files |cdocsfn1.tex| and |cdocsfn2.tex|
% (with identical content)
% compile the final versions of the child documents
% |cdocsch1.tex| and |cdocsch2.tex|, respectively:
%\iffalse
%<*samplefinal>
%\fi
%    \begin{macrocode}
\def\version{final}
\input{childdoc.def}
\childdocforwardprefix[cdocsamp]{cdocsfn}{cdocsch}
%    \end{macrocode}

%\iffalse
%</samplefinal>
%\fi
%
% %%%%%%%%%%%%%%%%%%%%%%%%%%%%%%%%%%%%%%
% \paragraph{Command Line Processing.}
%
% The following three command lines generate the output files
% |cdocscld|, |cdocscl1| and |cdocscl2|
% which should be identical to
% |cdocsdrf|, |cdocsch1| and |cdocsfn2|, respectively:
% \begin{center}
% \begin{tabular}{l}
% |latex -jobname cdocscld \|\\
% |  "\def\version{draft}\input{childdoc.def}\childdocforward{cdocsamp}"|\\
% |latex -jobname cdocscl1 \|\\
% |  "\input{childdoc.def}\childdocforward[cdocsamp]{cdocsch1}"|\\
% |latex -jobname cdocscl2 \|\\
% |  "\def\version{final}\input{childdoc.def}\childdocforward{cdocsch2}"|
% \end{tabular}
% \end{center}
% Note that the trailing backslash on each first line
% merely continues the input to the second line
% (for convenient cut ant paste).
% Furthermore, the command |latex| can be replaced by any
% of its alternative versions such as |pdflatex|.
%
% %%%%%%%%%%%%%%%%%%%%%%%%%%%%%%%%%%%%%%%%%%%%%%%%%%%%%%%%%%%%%%%%%%%%%%%%%%%%%%
% %%%%%%%%%%%%%%%%%%%%%%%%%%%%%%%%%%%%%%%%%%%%%%%%%%%%%%%%%%%%%%%%%%%%%%%%%%%%%%
% \section{Implementation}
%\iffalse
%<*package>
%\fi
%
% This section describes the definitions file |childdoc.def|.

% The definitions cannot be loaded using |\usepackage| or |\RequirePackage|
% which has a mechanism to prevent loading a style file more than once.
% When loading the definitions by means of |\input|
% multiple instances have to be prevented manually:
%\iffalse
%This code needs to be before the `\ProvidesFile' directive
%which is defined at the beginning of this file.
%Therefore it is also placed there and commented out here.
%</package>
%<*discard>
%\fi
%    \begin{macrocode}
\ifdefined\childdocmain\endinput\fi
%    \end{macrocode}
%\iffalse
%</discard>
%<*package>
%\fi
%
% \macro{\ifchilddoc}
% \macro{\ifchilddocmanual}
% The conditional |\ifchilddoc| tells whether a
% child (true) or main (false) document is being compiled.
% The conditional |\ifchilddocmanual| tells whether
% the |\includeonly| mechanism is used (false) or
% the selection of child files must be performed manually (true).
% The definitions initialise to false:
%    \begin{macrocode}
\newif\ifchilddoc
\newif\ifchilddocmanual
%    \end{macrocode}

% \macro{\childdocname}
% \macro{\childdocjob}
% The macro |\childdocname| stores the name of the main document
% to be compiled. The macro |\childdocjob| stores the name of
% the document on which the \LaTeX{} compiler was originally invoked.
% The content of |\jobname| cannot be compared
% to filenames specified in the source due to different catcodes.
% The following code rescans |\jobname|, stores the result
% in |\childdocname| and saves a copy in |\childdocjob|:
%    \begin{macrocode}
\edef\childdocname{\scantokens\expandafter{\jobname\noexpand}}
\let\childdocjob\childdocname
%    \end{macrocode}

% \macro{\childdocdisable}
% The macro |\childdocdisable| prevents the main file
% from being processed more than once.
% At this stage, the main document command |\childdocmain|
% is assumed to be called once again where it should do nothing.
% Any subsequent call to it should prevent
% a secondary processing of the main document
% It overwrites the forwarding commands
% |\childdocof| and |\childdocforward|
% with empty macros to prevent further inclusions of the main document:
%    \begin{macrocode}
\newcommand{\childdocdisable}
{
  \renewcommand{\childdocmain}[1]{\renewcommand{\childdocmain}[1]{\endinput}}
  \renewcommand{\childdocof}[1]{}
  \renewcommand{\childdocby}[2][]{}
  \renewcommand{\childdocforward}[2][]{}
  \renewcommand{\childdocdisable}{}
}
%    \end{macrocode}

% \macro{\childdocmain}
% The macro |\childdocmain| is to be called at the top of the main file
% with nothing or the main filename (without extension) as argument.
% First, it breaks loops.
% If the argument is not empty and does not match |\childdocname|
% (which is set by the first inclusion of |childdoc.def|),
% |\ifchilddoc| is set to true, |\includeonly| is applied to the child file
% and |\jobname| is set to the main file
% (for proper handling of |.aux| files):
%    \begin{macrocode}
\newcommand{\childdocmain}[1]
{
  \childdocdisable\childdocmain{}
  \if?#1?\else
    \begingroup
      \def\childdoctmp{#1}
      \ifx\childdoctmp\childdocname
        \def\childdoctmp{}
      \else
        \def\childdoctmp
        {
          \childdoctrue
          \includeonly{\childdocname}
          \def\childdocjob{#1}
          \def\jobname{#1}
        }
      \fi
      \expandafter
    \endgroup
    \childdoctmp
  \fi
}
%    \end{macrocode}

% \macro{\childdocof}
% The command |\childdocof| redirects
% compilation to the main file |#1|.
%    \begin{macrocode}
\newcommand{\childdocof}[1]
{
  \childdocdisable
  \childdoctrue
  \includeonly{\childdocname}
  \def\jobname{#1}
  \def\childdocjob{#1}
  \input{#1}
}
%    \end{macrocode}

% \macro{\childdocby}
% The command |\childdocby| ....
%    \begin{macrocode}
\newcommand{\childdocby}[2][]
{
  \childdocdisable
  \childdoctrue
  \childdocmanualtrue
  \if?#1?\else
    \def\jobname{#2}
  \fi
  \def\childdocjob{#2}
  \input{#2}
  \endinput
}
%    \end{macrocode}

% \macro{\childdocforward}
% The command |\childdocforward| redirects
% compilation to the main file or
% (if the optional argument is given) a child file.
% Parameters are set as if the main file
% or a child file starting with |\childdocof| was compiled.
% Then compilation is handed over to the main file:
%    \begin{macrocode}
\newcommand{\childdocforward}[2][]
{
  \begingroup
    \if?#1?
      \def\childdoctmp
      {
        \def\childdocname{#2}
        \def\childdocjob{#2}
        \def\jobname{#2}
        \input{#2}
        \endinput
      }
    \else
      \def\childdoctmp
      {
        \childdocdisable
        \def\childdocname{#2}
        \childdoctrue
        \includeonly{#2}
        \def\childdocjob{#1}
        \def\jobname{#1}
        \input{#1}
        \endinput
      }
    \fi
    \expandafter
  \endgroup
  \childdoctmp
}
%    \end{macrocode}

% \macro{\childdocforwardprefix}
% The command |\childdocforwardprefix| redirects
% compilation to the main or a child file by means of a pattern.
% The prefix |#1| in the current filename is replaced by |#2|
% and the suffix of the current filename is kept
% (it is assumed that the filename does not contain the substring `|~~~|'
% which is used as a delimiter).
% Compilation is handed over to the new file by |\childdocforward|:
%    \begin{macrocode}
\newcommand{\childdocforwardprefix}[3][]
{
  \begingroup
    \def\childdocextract #2##1~~~{\def\childdoctmp{\childdocforward[#1]{#3##1}}}
    \expandafter\childdocextract\childdocname~~~
    \expandafter
  \endgroup
  \childdoctmp
}
%    \end{macrocode}

% \macro{\childdoc}
% The deprecated macro |\childdoc| is a legacy version of |\childdocmain|:
%    \begin{macrocode}
\newcommand{\childdoc}{\childdocmain}
%    \end{macrocode}

% \macro{\childdocredirect}
% The deprecated macro |\childdocredirect| is a legacy version
% of |\childdocforward| and |\childdocforwardprefix|:
%    \begin{macrocode}
\newcommand{\childdocredirect}[2][]
{
  \begingroup
    \if?#1?
      \def\childdoctmp{\childdocforward{#2}}
    \else
      \def\childdoctmp{\childdocforwardprefix{#1}{#2}}
    \fi
    \expandafter
  \endgroup
  \childdoctmp
}
%    \end{macrocode}

%\iffalse
%</package>
%\fi
%
\endinput
|\\
|\childdocforward{|\textit{main}|}|
\end{tabular}
\end{center}
%
Likewise, the following files |final|\textit{nn}|.tex|
compile the final version of the child document
|child|\textit{nn}|.tex|:
%
\begin{center}
\begin{tabular}{l}
|\def\version{final}|\\
|% \iffalse
%
% childdoc.dtx Copyright (C) 2017-2018 Niklas Beisert
%
% This work may be distributed and/or modified under the
% conditions of the LaTeX Project Public License, either version 1.3
% of this license or (at your option) any later version.
% The latest version of this license is in
%   http://www.latex-project.org/lppl.txt
% and version 1.3 or later is part of all distributions of LaTeX
% version 2005/12/01 or later.
%
% This work has the LPPL maintenance status `maintained'.
%
% The Current Maintainer of this work is Niklas Beisert.
%
% This work consists of the files childdoc.dtx and childdoc.ins
% and the derived files childdoc.def and cdocsamp.tex with
% cdocsch1.tex, cdocsch2.tex, cdocsdrf.tex, cdocsfn1.tex, cdocsfn2.tex.
%
%<package>\ifdefined\childdocmain\endinput\fi
%<package>\ProvidesFile{childdoc.def}[2018/12/30 v2.0 child document driver]
%<samplemain>\ProvidesFile{cdocsamp.tex}[2018/12/30 v2.0 sample for childdoc]
%<*driver>
%\ProvidesFile{childdoc.drv}[2018/12/30 v2.0 childdoc reference manual file]
\PassOptionsToClass{10pt,a4paper}{article}
\documentclass{ltxdoc}

\usepackage[margin=35mm]{geometry}
\usepackage{hyperref}
\usepackage{hyperxmp}
\usepackage[usenames]{color}

\hypersetup{colorlinks=true}
\hypersetup{pdfstartview=FitH}
\hypersetup{pdfpagemode=UseNone}
\hypersetup{pdfsource={}}
\hypersetup{pdflang={en-UK}}
\hypersetup{pdfcopyright={Copyright 2017-2018 Niklas Beisert.
  This work may be distributed and/or modified under the
  conditions of the LaTeX Project Public License, either version 1.3
  of this license or (at your option) any later version.}}
\hypersetup{pdflicenseurl={http://www.latex-project.org/lppl.txt}}
\hypersetup{pdfcontactaddress={ETH Zurich, ITP, HIT K,
  Wolfgang-Pauli-Strasse 27}}
\hypersetup{pdfcontactpostcode={8093}}
\hypersetup{pdfcontactcity={Zurich}}
\hypersetup{pdfcontactcountry={Switzerland}}
\hypersetup{pdfcontactemail={nbeisert@itp.phys.ethz.ch}}
\hypersetup{pdfcontacturl={http://people.phys.ethz.ch/\xmptilde nbeisert/}}

\newcommand{\secref}[1]{\hyperref[#1]{section \ref*{#1}}}

\parskip1ex
\parindent0pt
\let\olditemize\itemize
\def\itemize{\olditemize\parskip0pt}

\begin{document}

\title{The \textsf{childdoc} Package}
\hypersetup{pdftitle={The childdoc Package}}
\author{Niklas Beisert\\[2ex]
  Institut f\"ur Theoretische Physik\\
  Eidgen\"ossische Technische Hochschule Z\"urich\\
  Wolfgang-Pauli-Strasse 27, 8093 Z\"urich, Switzerland\\[1ex]
  \href{mailto:nbeisert@itp.phys.ethz.ch}
  {\texttt{nbeisert@itp.phys.ethz.ch}}}
\hypersetup{pdfauthor={Niklas Beisert}}
\hypersetup{pdfsubject={Manual for the LaTeX2e Package childdoc}}
\date{30 December 2018, \textsf{v2.0}}
\maketitle

\begin{abstract}\noindent
\textsf{childdoc} is a \LaTeXe{} package
that enables the direct compilation
of document sections included by |\include|
to individual files.
\end{abstract}

\begingroup
\parskip0ex
\tableofcontents
\endgroup

%%%%%%%%%%%%%%%%%%%%%%%%%%%%%%%%%%%%%%%%%%%%%%%%%%%%%%%%%%%%%%%%%%%%%%%%%%%%%%%%
%%%%%%%%%%%%%%%%%%%%%%%%%%%%%%%%%%%%%%%%%%%%%%%%%%%%%%%%%%%%%%%%%%%%%%%%%%%%%%%%
\section{Introduction}

\LaTeX{} provides a mechanism to structure a large document (such as a book)
into a main file and several child files (containing the chapters)
using the |\include| command.
This mechanism is beneficial for documents
which span hundreds of pages in order to
make the source file(s) more manageable.
Moreover, compilation can be restricted to
selected child files by means of the |\includeonly| command.
The latter feature can be used to reduce the compilation time while editing
(this was significantly more useful in the earlier days of \LaTeX{})
or to generate a smaller document which is easier to navigate.
Another application of |\includeonly| is to generate
documents consisting of selected parts of the complete document.

However, there are a few drawbacks of the plain |\include| mechanism:
\begin{itemize}
\item
The child files cannot be compiled on their own,
they can only be compiled via the main file.
A naive editing environment
(such as a text editor with an option
to have the current file processed by \LaTeX)
may require one to switch to the main file before compiling;
attempting to compile the child file produces errors.
\item
The main file must be modified (each time)
to adjust the |\includeonly| command
to the present needs. This easily leaves the main file in a messy state.
\item
The generated document will always carry the filename
of the main document. This is inconvenient if
several child files are to be compiled and
to be kept for distribution.
\end{itemize}

The present package provides a simple interface
to make child files individually compilable by \LaTeX{}.
Compiling a child file then has the same effect as compiling
the main file with an |\includeonly| command
to select the appropriate child.
Moreover the generated document will carry the name of the child
rather than the main file.
This resolves all three above issues.

This feature is meant to make the editing of books,
thesis documents and lecture notes somewhat more convenient.
However, the package can also be used efficiently for
composing a series of documents (such as exercise sheets)
which are typically distributed individually.
It then assists the author in generating the individual documents
(potentially in different versions)
as well as a document containing the collected series.
Another application is in developing style files
or other kinds of included material
where compilation of the style file could redirect
to a sample or test file.

%%%%%%%%%%%%%%%%%%%%%%%%%%%%%%%%%%%%%%%%%%%%%%%%%%%%%%%%%%%%%%%%%%%%%%%%%%%%%%%%
%%%%%%%%%%%%%%%%%%%%%%%%%%%%%%%%%%%%%%%%%%%%%%%%%%%%%%%%%%%%%%%%%%%%%%%%%%%%%%%%
\section{Usage}

First of all, the package \textsf{childdoc} is \emph{not} a standard
\LaTeXe{} |.sty| style file! Therefore it needs to be invoked in
a non-standard way.

%%%%%%%%%%%%%%%%%%%%%%%%%%%%%%%%%%%%%%%%%%%%%%%%%%%%%%%%%%%%%%%%%%%%%%%%%%%%%%%%
\subsection{Included Files}
\label{sec:include}

%%%%%%%%%%%%%%%%%%%%%%%%%%%%%%%%%%%%%%%%
\DescribeMacro{\childdocmain}
To use the package, add the commands
\begin{center}
\begin{tabular}{l}
|\input{childdoc.def}|\\
|\childdocmain{}|\\
\end{tabular}
\end{center}
at the very top of the main \LaTeX{} file,
in particular \emph{before} the |\documentclass| statement!
The argument of |\childdocmain| should be left empty
(but it must be present).

%%%%%%%%%%%%%%%%%%%%%%%%%%%%%%%%%%%%%%%%
\DescribeMacro{\childdocof}
Furthermore, add the commands
\begin{center}
\begin{tabular}{l}
|\input{childdoc.def}|\\
|\childdocof{|\textit{main}|}|\\
\end{tabular}
\end{center}
at the top of every child file \textit{child}
which is included by |\include{|\textit{child}|}|
from within the main file
(or at least for those files to be compiled individually).
The argument \textit{main} must be the filename of the main file.

There are a couple of
considerations in setting up the main and child documents:

%%%%%%%%%%%%%%%%%%%%%%%%%%%%%%%%%%%%%%%%
\paragraph{Restrictions.}

Please note the following restrictions:
\begin{itemize}
\item
|\childdocmain| must be called with one argument \textit{main}
to ensure compatibility with earlier version of the package.
It must either be empty (|\childdocmain{}|)
or precisely match the filename of the main file in which it is specified.
See \secref{sec:detection} for further information.
\item
The filename \textit{main} must be specified without the |.tex| extension.
\item
The filename \textit{main} is case sensitive
(even in case-insensitive file systems)
due to internal string comparison.
\item
The argument \textit{main} should be fully expanded, it cannot be a macro.
\item
Subdirectories and special characters should be avoided in filenames.
\item
The command |\childdocmain{|\textit{main}|}| must be followed by a whitespace.
It should not be followed immediately by another command
or by a comment mark `|%|'.
This is because the \TeX{} parser reads the token immediately following
the argument of |\childdocmain| and puts it
at the beginning of every child section;
however, a white\-space is ignored.
\end{itemize}

%%%%%%%%%%%%%%%%%%%%%%%%%%%%%%%%%%%%%%%%
\paragraph{Content of Main File.}

It is advisable to place all content in the child files included by |\include|.
Any output contained in the main file will appear in all child documents
unless suppressed manually;
it cannot be suppressed automatically by the |\includeonly| directive
and thus should normally be avoided.
A method to include some content in the main file
by means of conditional processing is described in \secref{sec:conditional}.

%%%%%%%%%%%%%%%%%%%%%%%%%%%%%%%%%%%%%%%%
\paragraph{Page Numbering.}

When only a part of the document is compiled,
the appropriate numbering of pages
(as well as other status parameters)
is determined from the |.aux| files.
The latter contain information from previous passes.
However this information needs to propagate through
all intermediate child documents.
Therefore the page numbering in child documents may well
be inconsistent until the complete document is compiled at least once.

A useful (if unconventional) way to always ensure a consistent
page numbering is to restart the numbering in each child document
and denote the pages by `\textit{child}|.|\textit{page}'
where \textit{child} represents the chapter/section number of the child file.
This can be achieved by the command
|\numberwithin{page}{|\textit{child}|}|
of the \textsf{amsmath} package
where \textit{child} can be |chapter| or |section|
depending on the chosen structuring.
Alternatively, one can modify the macro |\thepage| appropriately
and reset the counter |page| at the start of each child file.

%%%%%%%%%%%%%%%%%%%%%%%%%%%%%%%%%%%%%%%%%%%%%%%%%%%%%%%%%%%%%%%%%%%%%%%%%%%%%%%%
\subsection{Conditional Processing}
\label{sec:conditional}

The package provides a mechanism to compile different versions
of a document. To customise the versions further some conditional processing
can come in handy to distinguish which version is being compiled.
The package provides two macros to describe the compilation context:

%%%%%%%%%%%%%%%%%%%%%%%%%%%%%%%%%%%%%%%%
\DescribeMacro{\ifchilddoc}
The conditional |\ifchilddoc| distinguishes between the compilation of
child documents and the main document:
%
\begin{center}
|\ifchilddoc |\textit{child-code}| |[|\||else |\textit{main-code}]| \||fi|
\end{center}

%%%%%%%%%%%%%%%%%%%%%%%%%%%%%%%%%%%%%%%%
\DescribeMacro{\childdocname}
\DescribeMacro{\childdocjob}
The macro |\childdocname| contains the filename (without extension)
of the main or child file being processed.
Note that |\childdocjob| will always contain the name of the main file.

%%%%%%%%%%%%%%%%%%%%%%%%%%%%%%%%%%%%%%%%
\paragraph{Title Page.}

Conditional processing can be used to include a title or banner page
in the main document when proper precautions are taken.
Importantly, the code in the main file should ensure that the page counter
(as well as other status parameters which are stored in the |.aux| files)
takes the same value after the conditional processing.
Otherwise the page numbers may take divergent values
depending on which part is compiled.

For example, a title page could be declared by:
%
\begin{center}
\begin{tabular}{l}
|\ifchilddoc\||else|\\
|\addtocounter{page}{-1}|\\
\textit{code for title page}\\
|\newpage|\\
|\||fi|
\end{tabular}
\end{center}
%
A banner page for the child documents can be generated by:
%
\begin{center}
\begin{tabular}{l}
|\ifchilddoc|\\
|\addtocounter{page}{-1}|\\
\textit{code for banner page}\\
|\newpage|\\
|\||fi|
\end{tabular}
\end{center}
%
Here one could write a message such as:
\begin{center}
|This is the part \childdocname{} of \childdocjob{}.|
\end{center}

%%%%%%%%%%%%%%%%%%%%%%%%%%%%%%%%%%%%%%%%%%%%%%%%%%%%%%%%%%%%%%%%%%%%%%%%%%%%%%%%
\subsection{Flags}
\label{sec:flags}

The package makes it easy to generate different versions
of the main or child documents.
To this end compilation flags can be defined
and assigned different default values.
They will be particularly useful in conjunction
with the forwarding mechanism described in \secref{sec:forward}.

For example, it may be useful to have a flag |\version|
which can be set to |draft| or |final|.
The document source will contain some conditional code
depending on the value of |\version|.
Suppose further, the flag should default to |final| for the main file
and to |draft| for child files
which is a natural assignment for editing the document.
This is achieved by placing the following code
in the preamble of the main document
(below the |\childdocmain| directive):
%
\begin{center}
\begin{tabular}{l}
|\ifchilddoc|\\
|\providecommand{\version}{draft}|\\
|\||else|\\
|\providecommand{\version}{final}|\\
|\||fi|
\end{tabular}
\end{center}
%
The definition by |\providecommand| makes sure
that previous definitions are not overwritten.
Further statements |\providecommand{\version}{...}|
can thus be added before the above code to override it.

For the main file, one might add a line
(between |\childdocmain| and the above block)
%
\begin{center}
|%\ifchilddoc\||else\providecommand{\version}{draft}\||fi|
\end{center}
%
which can be uncommented to produce a draft version.
Likewise one can add a line to the very top of a child file
(above the |\childdocof{|\textit{main}|}| directive)
%
\begin{center}
|%\providecommand{\version}{final}|
\end{center}
%
which can be uncommented to produce the final version of this child document.

%%%%%%%%%%%%%%%%%%%%%%%%%%%%%%%%%%%%%%%%%%%%%%%%%%%%%%%%%%%%%%%%%%%%%%%%%%%%%%%%
\subsection{Forwarding}
\label{sec:forward}

Different versions of the main or child documents
using compilation flags as described in \secref{sec:flags}
can be (permanently) stored in different files
for convenient compilation, viewing and distribution.
To this end, the package defines a command
to pass on compilation to a different file:

%%%%%%%%%%%%%%%%%%%%%%%%%%%%%%%%%%%%%%%%
\DescribeMacro{\childdocforward}
The command |\childdocforward| redirects processing to
another source file:
%
\begin{center}
\begin{tabular}{l}
|\input{childdoc.def}|\\
|\childdocforward[|\textit{main}|]{|\textit{dest}|}|\\
\end{tabular}
\end{center}
%
The argument \textit{dest} is the destination file
(without extension).
It should be the main file or one of the child files.
Note that further \textsf{childdoc} directives
such as |\childdocof| and |\childdocforward|
in the indicated file will be processed in this form.
The optional argument \textit{main}
passes on directly to the main file \textit{main}
while pretending to compile the child \textit{dest}.
This form behaves as if \textit{dest}
issues |\childdocof{|\textit{main}|}| right away,
and no further \textsf{childdoc} directives will be processed.

%%%%%%%%%%%%%%%%%%%%%%%%%%%%%%%%%%%%%%%%
\DescribeMacro{\...prefix}
In the alternative form |\childdocforwardprefix|,
%
\begin{center}
\begin{tabular}{l}
|\input{childdoc.def}|\\
|\childdocforwardprefix[|\textit{main}|]{|\textit{prefix}|}{|\textit{dest}|}|
\end{tabular}
\end{center}
%
the destination file is determined by a pattern
depending on the current file:
To make this work, the current file must be called
`{\textit{prefix}\hspace{0.2em}\textit{suffix}}'
with \textit{prefix} matching precisely the argument.
Processing is then passed on to the file
`{\textit{dest}\hspace{0.2em}\textit{suffix}}'.
Surely, the same effect is achieved by
directly specifying the
argument `{\textit{dest}\hspace{0.2em}\textit{suffix}}'
in the first form.
However, that requires to set up a different file
for each child. With the alternative form of the command
all these files can have exactly the same content
which simplifies setting them up and maintaining them.

For example, the following file |draft.tex|
with a compilation flag |\version| as described in \secref{sec:flags}
compiles the main document as a draft:
%
\begin{center}
\begin{tabular}{l}
|\def\version{draft}|\\
|\input{childdoc.def}|\\
|\childdocforward{|\textit{main}|}|
\end{tabular}
\end{center}
%
Likewise, the following files |final|\textit{nn}|.tex|
compile the final version of the child document
|child|\textit{nn}|.tex|:
%
\begin{center}
\begin{tabular}{l}
|\def\version{final}|\\
|\input{childdoc.def}|\\
|\childdocforwardprefix{final}{child}|
\end{tabular}
\end{center}
%

Note that when several versions of a main file and/or of each child file
are to be generated, it may be convenient to set up a |Makefile| or
shell script to automatise the process.

%%%%%%%%%%%%%%%%%%%%%%%%%%%%%%%%%%%%%%%%%%%%%%%%%%%%%%%%%%%%%%%%%%%%%%%%%%%%%%%%
\subsection{Command Line Processing}
\label{sec:commandline}

The effect of redirection files can also be achieved by invoking
the \LaTeX{} compiler with a more elaborate command line.
Most conveniently this should be done as part
of a shell script or a |Makefile|.

When using \textsf{childdoc} in the main file, the following
command lines effectively perform a redirection
(note that depending on the shell being used,
backslashes may have to be doubled: `|\|' $\to$ `|\\|'):
%
\begin{center}
|... -jobname "|\textit{target}|" |\\|"|[\textit{flags}]%
|\input{childdoc.def}\childdocforward[|\textit{main}|]{|\textit{dest}|}"|
\end{center}
%
Here \textit{target} is the name of the output file,
\textit{main} is the name of the main file
and \textit{dest} is the name of the main or child file to be processed
(all filenames without extensions).
The optional argument \textit{main} can be omitted
if \textit{main} matches \textit{dest}.
Optionally, compilation \textit{flags} can be defined via |\def| commands.
This command line makes the \TeX{} engine believe
it is compiling the file \textit{target}
whose content is specified as the latter parameter.
The provided code then forwards the processing to
\textit{main} or \textit{dest} as described in \secref{sec:forward}.

%%%%%%%%%%%%%%%%%%%%%%%%%%%%%%%%%%%%%%%%%%%%%%%%%%%%%%%%%%%%%%%%%%%%%%%%%%%%%%%%
\subsection{Include by Input}
\label{sec:input}

Including child documents by |\include| has some restrictions by design.
Most notably, the content of a child document always occupies
its own set of pages; pages cannot be shared between child documents.
Usually, this behaviour makes perfect sense
because each child document contain an essential part of the document.
However, in some situations it may be desirable to compose
a document from a collection of parts
without having mandatory page breaks between then.
For this case, the package
provides a mechanism to include parts
by |\input| which can also be processed individually.
However, by construction this mechanism
requires manual handling of the content to be output.

%%%%%%%%%%%%%%%%%%%%%%%%%%%%%%%%%%%%%%%%
\DescribeMacro{\ifchilddocmanual}
The main file should be prepared as usual, see \secref{sec:include}.
However, the document body must make a distinction
between processing of an individual part and of the main document, e.g.:
%
\begin{center}
\begin{tabular}{l}
|\ifchilddocmanual|\\
|\input{\childdocname}|\\
|\||else|\\
\textit{document body with }|\input{|\textit{part}|}|\\
|\||fi|
\end{tabular}
\end{center}
%
The conditional |\ifchilddocmanual| is true whenever
a part to be included by |\input| is being compiled,
and the name of the part is stored in |\childdocname|.

%%%%%%%%%%%%%%%%%%%%%%%%%%%%%%%%%%%%%%%%
\DescribeMacro{\childdocby}
Each part to be included by |\input| should start with:
%
\begin{center}
\begin{tabular}{l}
|\input{childdoc.def}|\\
|\childdocby{|\textit{main}|}|\\
\end{tabular}
\end{center}
%
The directive |\childdocby| is similar to |\childdocof|
described in \secref{sec:include},
but the subsequent selection of content must be done manually.
To that end, both |\ifchilddoc| and |\ifchilddocmanual|
will be true upon processing of a part,
and the name of the part is stored in |\childdocname|.
Note that |\jobname| will be set to the filename of the current part
so that each part receives an individual |.aux| file
that does not interfere with the |.aux| file(s) of the main document.
This behaviour can be altered by the alternative form
|\childdocby[*]{|\textit{main}|}| (with a non-empty optional argument)
which uses the |.aux| file of the main document
by setting |\jobname| to \textit{main}.

%%%%%%%%%%%%%%%%%%%%%%%%%%%%%%%%%%%%%%%%%%%%%%%%%%%%%%%%%%%%%%%%%%%%%%%%%%%%%%%%
\subsection{Driver Development}
\label{sec:driver}

The \textsf{childdoc} mechanism can also be use for the development
of definition files such as \LaTeX{} styles or classes.
This case differs from the above setup with multiple parts
included by |\include| in that no |\includeonly| should be invoked.
This can be achieved by starting the include file
(before |\ProvidesPackage|) with:
%
\begin{center}
\begin{tabular}{l}
|\input{childdoc.def}|\\
|\childdocforward{|\textit{main}|}|\\
\end{tabular}
\end{center}
%
or alternatively with:
%
\begin{center}
\begin{tabular}{l}
|\input{childdoc.def}|\\
|\childdocby{|\textit{main}|}|\\
\end{tabular}
\end{center}
%
Both forms have slightly different effects as described above.
The main file is prepared as usual, see \secref{sec:include}.

%%%%%%%%%%%%%%%%%%%%%%%%%%%%%%%%%%%%%%%%%%%%%%%%%%%%%%%%%%%%%%%%%%%%%%%%%%%%%%%%
\subsection{Legacy Detection}
\label{sec:detection}

The directive |\childdocmain| in the main file can detect
whether the complete document or merely a child is to be compiled
even without using the directive |\childdocof|.
This method is deprecated because it is less robust
and there is no compelling reason to use it;
it is merely provided for backward compatibility
and it may be removed in future versions.

If the detection mechanism is to be used,
it is mandatory to correctly specify
the filename of the main file as the argument of |\childdocmain|:
%
\begin{center}
\begin{tabular}{l}
|\input{childdoc.def}|\\
|\childdocmain{|\textit{main}|}|\\
\end{tabular}
\end{center}
%
If |\jobname| does not match the argument \textit{main} of |\childdocmain|,
it is assumed that |\jobname| points to the child file to be compiled.
When using |\childdocmain| with the main file specified as argument,
it suffices to start a child file
with just |\input{|\textit{main}|}|
without loading of the package and using |\childdocof|.
If instead all processing is done
with the appropriate \textsf{childdoc} directives,
the argument of \textit{main} of |\childdocmain| can be empty.

An alternative version of the command line processing described
in \secref{sec:commandline} using the detection mechanism reads:
%
\begin{center}
|... -jobname "|\textit{target}|" "|[\textit{flags}]%
[|\def\jobname{|\textit{dest}|}|]|\input{|\textit{main}|}"|
\end{center}

%%%%%%%%%%%%%%%%%%%%%%%%%%%%%%%%%%%%%%%%%%%%%%%%%%%%%%%%%%%%%%%%%%%%%%%%%%%%%%%%
\subsection{Manual Code}
\label{sec:manual}

In case one cannot be certain whether the definitions file |childdoc.def|
is installed on the target \TeX{} distribution
and one prefers not to ship it,
it is conceivable to paste a few relevant commands into the sources.

To that end, drop all statements |\input{childdoc.def}|
and perform the replacements as outlined below.
Instead of |\childdocmain{|\textit{main}|}| add the following code
to the top of the main file:
%
\begin{center}
\begin{tabular}{l}
|\||ifdefined\childdocname\endinput\||fi\newif\ifchilddoc|\\
|\edef\childdocname{\scantokens\expandafter{\jobname\noexpand}}|\\
|\def\childdocmain{|\textit{main}|}\||ifx\childdocmain\childdocname\||else|\\
|\childdoctrue\includeonly{\childdocname}\let\jobname\childdocmain\||fi|\\
\end{tabular}
\end{center}
%
Instead of |\childdocof{|\textit{main}|}| just include the main file
at the top of each child file:
%
\begin{center}
|\input{|\textit{main}|}|
\end{center}
%
A simple redirection |\childdocforward{|\textit{dest}|}| is achieved by:
%
\begin{center}
|\def\jobname{|\textit{dest}|}\input{\jobname}|
\end{center}
%
The redirection with prefix
|\childdocforwardprefix[|\textit{prefix}|]{|\textit{dest}|}|
is accomplished by:
%
\begin{center}
\begin{tabular}{l}
|{\edef\jobname{\scantokens\expandafter{\jobname\noexpand}}|\\
|\def\redirectjob |\textit{prefix}|#1~~~{\gdef\jobname{|\textit{dest}|#1}}|\\
|\expandafter\redirectjob\jobname~~~}\input{\jobname}|
\end{tabular}
\end{center}

In an alternative approach,
child documents can be compiled by a specific command line
without additional code or specific definitions:
%
\begin{center}
|... -jobname "|\textit{target}|" "|[\textit{flags}]%
|\includeonly{|\textit{dest}|}\input{|\textit{main}|}"|
\end{center}
%

%%%%%%%%%%%%%%%%%%%%%%%%%%%%%%%%%%%%%%%%%%%%%%%%%%%%%%%%%%%%%%%%%%%%%%%%%%%%%%%%
%%%%%%%%%%%%%%%%%%%%%%%%%%%%%%%%%%%%%%%%%%%%%%%%%%%%%%%%%%%%%%%%%%%%%%%%%%%%%%%%
\section{Information}

%%%%%%%%%%%%%%%%%%%%%%%%%%%%%%%%%%%%%%%%%%%%%%%%%%%%%%%%%%%%%%%%%%%%%%%%%%%%%%%%
\subsection{Copyright}

Copyright \copyright{} 2017--2018 Niklas Beisert

This work may be distributed and/or modified under the
conditions of the \LaTeX{} Project Public License, either version 1.3
of this license or (at your option) any later version.
The latest version of this license is in
  \url{http://www.latex-project.org/lppl.txt}
and version 1.3 or later is part of all distributions of \LaTeX{}
version 2005/12/01 or later.

This work has the LPPL maintenance status `maintained'.

The Current Maintainer of this work is Niklas Beisert.

This work consists of the files |README.txt|, |childdoc.ins| and |childdoc.dtx|
as well as the derived files |childdoc.def|, |cdocsamp.tex|
with |cdocsch1.tex|, |cdocsch2.tex|, |cdocspt3.tex|, |cdocspt4.tex|,
|cdocsdrf.tex|, |cdocsfn1.tex|, |cdocsfn2.tex|
as well as |childdoc.pdf|.

%%%%%%%%%%%%%%%%%%%%%%%%%%%%%%%%%%%%%%%%%%%%%%%%%%%%%%%%%%%%%%%%%%%%%%%%%%%%%%%%
\subsection{Files and Installation}

The package consists of the files:
%
\begin{center}
\begin{tabular}{ll}
    |README.txt|   & readme file \\
    |childdoc.ins| & installation file \\
    |childdoc.dtx| & source file \\
    |childdoc.def| & definition file \\
    |cdocsamp.tex| & sample main file \\
    |cdocsch1.tex| & sample include file \\
    |cdocsch2.tex| & sample include file \\
    |cdocspt3.tex| & sample part file \\
    |cdocspt4.tex| & sample part file \\
    |cdocsdrf.tex| & sample redirection file \\
    |cdocsfn1.tex| & sample redirection file \\
    |cdocsfn2.tex| & sample redirection file \\
    |childdoc.pdf| & manual
\end{tabular}
\end{center}
%
The distribution consists of the files
|README.txt|, |childdoc.ins| and |childdoc.dtx|.
%
\begin{itemize}
\item
Run (pdf)\LaTeX{} on |childdoc.dtx|
to compile the manual |childdoc.pdf| (this file).
\item
Run \LaTeX{} on |childdoc.ins| to create the definitions file |childdoc.def|
and the sample |cdocsamp.tex| with include files
|cdocsch1.tex|, |cdocsch2.tex|, |cdocspt3.tex|, |cdocspt4.tex|,
|cdocsdrf.tex|, |cdocsfn1.tex|, |cdocsfn2.tex|.
Then copy the file |childdoc.def| to an appropriate directory of your \LaTeX{}
distribution, e.g.\ \textit{texmf-root}|/tex/latex/childdoc|.
\end{itemize}

%%%%%%%%%%%%%%%%%%%%%%%%%%%%%%%%%%%%%%%%%%%%%%%%%%%%%%%%%%%%%%%%%%%%%%%%%%%%%%%%
\subsection{Related CTAN Packages}

There are several other packages which offer a similar functionality:
%
\begin{itemize}
\item
The packages
\href{http://ctan.org/pkg/docmute}{\textsf{docmute}},
\href{http://ctan.org/pkg/includex}{\textsf{includex}} and
\href{http://ctan.org/pkg/standalone}{\textsf{standalone}}
provide commands to include only the document body of
a child file thus allowing both files to be compiled individually.
\item
The packages \href{http://ctan.org/pkg/subdocs}{\textsf{subdocs}}
and \href{http://ctan.org/pkg/subfiles}{\textsf{subfiles}}
provide structures in which the main and child documents can be
encapsulated and allowing them to be compiled individually.
The inclusion mechanism is different from the conventional |\include|.
\item
The package \href{http://ctan.org/pkg/combine}{\textsf{combine}}
is an elaborate solution to combine several documents into one.
\end{itemize}
%
See also the CTAN topic \href{http://ctan.org/topic/subdocs}{\textsf{subdocs}}
for further related packages.
The present package differs from the above solutions in that
a document structure constructed with the conventional |\include| mechanism
just needs two extra commands at the top of every file
such that all constituent files can be compiled individually.

%%%%%%%%%%%%%%%%%%%%%%%%%%%%%%%%%%%%%%%%%%%%%%%%%%%%%%%%%%%%%%%%%%%%%%%%%%%%%%%%
%\subsection{Feature Suggestions}
%
%The following is a list of features which may be useful for future
%versions of this package:
%%
%\begin{itemize}
%\item
%\ldots
%\end{itemize}

%%%%%%%%%%%%%%%%%%%%%%%%%%%%%%%%%%%%%%%%%%%%%%%%%%%%%%%%%%%%%%%%%%%%%%%%%%%%%%%%
\subsection{Revision History}

%%%%%%%%%%%%%%%%%%%%%%%%%%%%%%%%%%%%%%%%
\paragraph{v2.0:} 2018/12/30

\begin{itemize}
\item
immediate forward processing
\item
added |\childdocby| mechanism
\item
manual restructured
\end{itemize}

%%%%%%%%%%%%%%%%%%%%%%%%%%%%%%%%%%%%%%%%
\paragraph{v1.6:} 2018/01/17

\begin{itemize}
\item
application for development of include files
\item
corrections to manual
\end{itemize}

%%%%%%%%%%%%%%%%%%%%%%%%%%%%%%%%%%%%%%%%
\paragraph{v1.5:} 2017/05/21

\begin{itemize}
\item
more complete structuring introduced
\item
|\childdocof| introduced
\item
|\childdoc| renamed to |\childdocmain|
\item
|\childredirect| renamed to |\childdocforward| and |\childdocforwardprefix|
and functionality expanded
\end{itemize}

%%%%%%%%%%%%%%%%%%%%%%%%%%%%%%%%%%%%%%%%
\paragraph{v1.0:} 2017/04/27

\begin{itemize}
\item
manual and install package
\item
first version published on CTAN
\end{itemize}

%%%%%%%%%%%%%%%%%%%%%%%%%%%%%%%%%%%%%%%%
\paragraph{v0.6:} 2017/04/26

\begin{itemize}
\item
redirection mechanism added
\end{itemize}

%%%%%%%%%%%%%%%%%%%%%%%%%%%%%%%%%%%%%%%%
\paragraph{v0.5:} 2017/04/26

\begin{itemize}
\item
functionality in definition file
\end{itemize}


%%%%%%%%%%%%%%%%%%%%%%%%%%%%%%%%%%%%%%%%%%%%%%%%%%%%%%%%%%%%%%%%%%%%%%%%%%%%%%%%
%%%%%%%%%%%%%%%%%%%%%%%%%%%%%%%%%%%%%%%%%%%%%%%%%%%%%%%%%%%%%%%%%%%%%%%%%%%%%%%%
%%%%%%%%%%%%%%%%%%%%%%%%%%%%%%%%%%%%%%%%%%%%%%%%%%%%%%%%%%%%%%%%%%%%%%%%%%%%%%%%
\appendix

\settowidth\MacroIndent{\rmfamily\scriptsize 000\ }

 \DocInput{childdoc.dtx}

\end{document}
%</driver>
% \fi
%
% %%%%%%%%%%%%%%%%%%%%%%%%%%%%%%%%%%%%%%%%%%%%%%%%%%%%%%%%%%%%%%%%%%%%%%%%%%%%%%
% %%%%%%%%%%%%%%%%%%%%%%%%%%%%%%%%%%%%%%%%%%%%%%%%%%%%%%%%%%%%%%%%%%%%%%%%%%%%%%
% \section{Sample}
%\iffalse
%<*samplemain>
%\fi
%
% The following presents a sample document
% with two chapters, two parts, a title page,
% a compile flag as well as three forwarding files to set the flag.
% It consists of eight |.tex| files:
% \begin{center}
% \begin{tabular}{ll}
% |cdocsamp.tex|&main file\\
% |cdocsch1.tex|&include file for chapter 1\\
% |cdocsch2.tex|&include file for chapter 2\\
% |cdocspt3.tex|&include file for part 3\\
% |cdocspt4.tex|&include file for part 4\\
% |cdocsdrf.tex|&forwarding file for main file in draft mode\\
% |cdocsfi1.tex|&forwarding file for final version of chapter 1\\
% |cdocsfi2.tex|&forwarding file for final version of chapter 2\\
% \end{tabular}
% \end{center}
% Each of the eight files can be compiled directly by the \LaTeX{} compiler.
%
% %%%%%%%%%%%%%%%%%%%%%%%%%%%%%%%%%%%%%%
% \paragraph{Main File.}
%
% The main file is called |cdocsamp.tex|.
%
% Load the \textsf{childdoc} definitions and
% declare the filename for the main document:
%    \begin{macrocode}
\input{childdoc.def}
\childdocmain{}
%    \end{macrocode}

% Optional override for |\version| flag:
%    \begin{macrocode}
%%\ifchilddoc\else\providecommand{\version}{draft}\fi
%    \end{macrocode}

% Define the default values for the |\version| flag
% (|final| for the main file and |draft| for childs):
%    \begin{macrocode}
\ifchilddoc
\providecommand{\version}{draft}
\else
\providecommand{\version}{final}
\fi
%    \end{macrocode}

% Load the standard document class:
%    \begin{macrocode}
\documentclass[12pt]{article}
%    \end{macrocode}

% Start the document body:
%    \begin{macrocode}
\begin{document}
%    \end{macrocode}

% Declare a title page.
% Print title, part of document being processed and version flag:
%    \begin{macrocode}
\addtocounter{page}{-1}
\begin{center}
{\LARGE\bfseries{}childdoc example\par}
\vspace{1cm}
\ifchilddoc
\ifchilddocmanual part\else chapter\fi:
`\childdocname' of `\childdocjob'\par
\else
main document: `\childdocjob'\par
\fi
version: \version\par
\end{center}
\newpage
%    \end{macrocode}

% Manually include selected file,
% otherwise process as usual:
%    \begin{macrocode}
\ifchilddocmanual
\section*{part `\childdocname'}
\input{\childdocname}
\else
%    \end{macrocode}

% Include the two chapters:
%    \begin{macrocode}
\include{cdocsch1}
\include{cdocsch2}
%    \end{macrocode}

% Include the two parts unless only chapters should be displayed:
%    \begin{macrocode}
\ifchilddoc\else
\section{part three}
\input{cdocspt3}
\section{part four}
\input{cdocspt4}
\fi
%    \end{macrocode}

% Process as usual until here:
%    \begin{macrocode}
\fi
%    \end{macrocode}

% End of document body:
%    \begin{macrocode}
\end{document}
%    \end{macrocode}
%\iffalse
%</samplemain>
%\fi
%
% %%%%%%%%%%%%%%%%%%%%%%%%%%%%%%%%%%%%%%
% \paragraph{Chapter Include Files.}
%
% The include files are called |cdocsch1.tex| and |cdocsch2.tex|.
%
%\iffalse
%<*samplechap1|samplechap2>
%\fi

% Optional override for |\version| flag:
%    \begin{macrocode}
%%\providecommand{\version}{final}
%    \end{macrocode}

% Include the main document:
%    \begin{macrocode}
\input{childdoc.def}
\childdocof{cdocsamp}
%    \end{macrocode}

%\iffalse
%</samplechap1|samplechap2>
%\fi
%
%\iffalse
%<*samplechap1>
%\fi
% Some text for chapter 1:
%    \begin{macrocode}
\section{one}
some text in chapter one
%    \end{macrocode}

%\iffalse
%</samplechap1>
%\fi
% Some text for chapter 2:
%\iffalse
%<*samplechap2>
%\fi
%    \begin{macrocode}
\section{two}
more text in chapter two
%    \end{macrocode}

%\iffalse
%</samplechap2>
%\fi
%
% %%%%%%%%%%%%%%%%%%%%%%%%%%%%%%%%%%%%%%
% \paragraph{Part Include Files.}
%
% The include files are called |cdocspt3.tex| and |cdocspt4.tex|.
%
%\iffalse
%<*samplepart3|samplepart4>
%\fi

% Optional override for |\version| flag:
%    \begin{macrocode}
%%\providecommand{\version}{final}
%    \end{macrocode}

% Include the main document:
%    \begin{macrocode}
\input{childdoc.def}
\childdocby{cdocsamp}
%    \end{macrocode}

%\iffalse
%</samplepart3|samplepart4>
%\fi
%
%\iffalse
%<*samplepart3>
%\fi
% Some text for part 3:
%    \begin{macrocode}
some text in part three
%    \end{macrocode}

%\iffalse
%</samplepart3>
%\fi
% Some text for part 4:
%\iffalse
%<*samplepart4>
%\fi
%    \begin{macrocode}
more text in part four
%    \end{macrocode}

%\iffalse
%</samplepart4>
%\fi
%
% %%%%%%%%%%%%%%%%%%%%%%%%%%%%%%%%%%%%%%
% \paragraph{Forwarding for a Complete Draft.}
%
% The following forwarding file |cdocsdrf.tex|
% compiles the main document in draft mode:
%\iffalse
%<*sampledraft>
%\fi
%    \begin{macrocode}
\def\version{draft}
\input{childdoc.def}
\childdocforward{cdocsamp}
%    \end{macrocode}

%\iffalse
%</sampledraft>
%\fi
%
% %%%%%%%%%%%%%%%%%%%%%%%%%%%%%%%%%%%%%%
% \paragraph{Forwarding for Final Version of the Chapters.}
%
% The following forwarding files |cdocsfn1.tex| and |cdocsfn2.tex|
% (with identical content)
% compile the final versions of the child documents
% |cdocsch1.tex| and |cdocsch2.tex|, respectively:
%\iffalse
%<*samplefinal>
%\fi
%    \begin{macrocode}
\def\version{final}
\input{childdoc.def}
\childdocforwardprefix[cdocsamp]{cdocsfn}{cdocsch}
%    \end{macrocode}

%\iffalse
%</samplefinal>
%\fi
%
% %%%%%%%%%%%%%%%%%%%%%%%%%%%%%%%%%%%%%%
% \paragraph{Command Line Processing.}
%
% The following three command lines generate the output files
% |cdocscld|, |cdocscl1| and |cdocscl2|
% which should be identical to
% |cdocsdrf|, |cdocsch1| and |cdocsfn2|, respectively:
% \begin{center}
% \begin{tabular}{l}
% |latex -jobname cdocscld \|\\
% |  "\def\version{draft}\input{childdoc.def}\childdocforward{cdocsamp}"|\\
% |latex -jobname cdocscl1 \|\\
% |  "\input{childdoc.def}\childdocforward[cdocsamp]{cdocsch1}"|\\
% |latex -jobname cdocscl2 \|\\
% |  "\def\version{final}\input{childdoc.def}\childdocforward{cdocsch2}"|
% \end{tabular}
% \end{center}
% Note that the trailing backslash on each first line
% merely continues the input to the second line
% (for convenient cut ant paste).
% Furthermore, the command |latex| can be replaced by any
% of its alternative versions such as |pdflatex|.
%
% %%%%%%%%%%%%%%%%%%%%%%%%%%%%%%%%%%%%%%%%%%%%%%%%%%%%%%%%%%%%%%%%%%%%%%%%%%%%%%
% %%%%%%%%%%%%%%%%%%%%%%%%%%%%%%%%%%%%%%%%%%%%%%%%%%%%%%%%%%%%%%%%%%%%%%%%%%%%%%
% \section{Implementation}
%\iffalse
%<*package>
%\fi
%
% This section describes the definitions file |childdoc.def|.

% The definitions cannot be loaded using |\usepackage| or |\RequirePackage|
% which has a mechanism to prevent loading a style file more than once.
% When loading the definitions by means of |\input|
% multiple instances have to be prevented manually:
%\iffalse
%This code needs to be before the `\ProvidesFile' directive
%which is defined at the beginning of this file.
%Therefore it is also placed there and commented out here.
%</package>
%<*discard>
%\fi
%    \begin{macrocode}
\ifdefined\childdocmain\endinput\fi
%    \end{macrocode}
%\iffalse
%</discard>
%<*package>
%\fi
%
% \macro{\ifchilddoc}
% \macro{\ifchilddocmanual}
% The conditional |\ifchilddoc| tells whether a
% child (true) or main (false) document is being compiled.
% The conditional |\ifchilddocmanual| tells whether
% the |\includeonly| mechanism is used (false) or
% the selection of child files must be performed manually (true).
% The definitions initialise to false:
%    \begin{macrocode}
\newif\ifchilddoc
\newif\ifchilddocmanual
%    \end{macrocode}

% \macro{\childdocname}
% \macro{\childdocjob}
% The macro |\childdocname| stores the name of the main document
% to be compiled. The macro |\childdocjob| stores the name of
% the document on which the \LaTeX{} compiler was originally invoked.
% The content of |\jobname| cannot be compared
% to filenames specified in the source due to different catcodes.
% The following code rescans |\jobname|, stores the result
% in |\childdocname| and saves a copy in |\childdocjob|:
%    \begin{macrocode}
\edef\childdocname{\scantokens\expandafter{\jobname\noexpand}}
\let\childdocjob\childdocname
%    \end{macrocode}

% \macro{\childdocdisable}
% The macro |\childdocdisable| prevents the main file
% from being processed more than once.
% At this stage, the main document command |\childdocmain|
% is assumed to be called once again where it should do nothing.
% Any subsequent call to it should prevent
% a secondary processing of the main document
% It overwrites the forwarding commands
% |\childdocof| and |\childdocforward|
% with empty macros to prevent further inclusions of the main document:
%    \begin{macrocode}
\newcommand{\childdocdisable}
{
  \renewcommand{\childdocmain}[1]{\renewcommand{\childdocmain}[1]{\endinput}}
  \renewcommand{\childdocof}[1]{}
  \renewcommand{\childdocby}[2][]{}
  \renewcommand{\childdocforward}[2][]{}
  \renewcommand{\childdocdisable}{}
}
%    \end{macrocode}

% \macro{\childdocmain}
% The macro |\childdocmain| is to be called at the top of the main file
% with nothing or the main filename (without extension) as argument.
% First, it breaks loops.
% If the argument is not empty and does not match |\childdocname|
% (which is set by the first inclusion of |childdoc.def|),
% |\ifchilddoc| is set to true, |\includeonly| is applied to the child file
% and |\jobname| is set to the main file
% (for proper handling of |.aux| files):
%    \begin{macrocode}
\newcommand{\childdocmain}[1]
{
  \childdocdisable\childdocmain{}
  \if?#1?\else
    \begingroup
      \def\childdoctmp{#1}
      \ifx\childdoctmp\childdocname
        \def\childdoctmp{}
      \else
        \def\childdoctmp
        {
          \childdoctrue
          \includeonly{\childdocname}
          \def\childdocjob{#1}
          \def\jobname{#1}
        }
      \fi
      \expandafter
    \endgroup
    \childdoctmp
  \fi
}
%    \end{macrocode}

% \macro{\childdocof}
% The command |\childdocof| redirects
% compilation to the main file |#1|.
%    \begin{macrocode}
\newcommand{\childdocof}[1]
{
  \childdocdisable
  \childdoctrue
  \includeonly{\childdocname}
  \def\jobname{#1}
  \def\childdocjob{#1}
  \input{#1}
}
%    \end{macrocode}

% \macro{\childdocby}
% The command |\childdocby| ....
%    \begin{macrocode}
\newcommand{\childdocby}[2][]
{
  \childdocdisable
  \childdoctrue
  \childdocmanualtrue
  \if?#1?\else
    \def\jobname{#2}
  \fi
  \def\childdocjob{#2}
  \input{#2}
  \endinput
}
%    \end{macrocode}

% \macro{\childdocforward}
% The command |\childdocforward| redirects
% compilation to the main file or
% (if the optional argument is given) a child file.
% Parameters are set as if the main file
% or a child file starting with |\childdocof| was compiled.
% Then compilation is handed over to the main file:
%    \begin{macrocode}
\newcommand{\childdocforward}[2][]
{
  \begingroup
    \if?#1?
      \def\childdoctmp
      {
        \def\childdocname{#2}
        \def\childdocjob{#2}
        \def\jobname{#2}
        \input{#2}
        \endinput
      }
    \else
      \def\childdoctmp
      {
        \childdocdisable
        \def\childdocname{#2}
        \childdoctrue
        \includeonly{#2}
        \def\childdocjob{#1}
        \def\jobname{#1}
        \input{#1}
        \endinput
      }
    \fi
    \expandafter
  \endgroup
  \childdoctmp
}
%    \end{macrocode}

% \macro{\childdocforwardprefix}
% The command |\childdocforwardprefix| redirects
% compilation to the main or a child file by means of a pattern.
% The prefix |#1| in the current filename is replaced by |#2|
% and the suffix of the current filename is kept
% (it is assumed that the filename does not contain the substring `|~~~|'
% which is used as a delimiter).
% Compilation is handed over to the new file by |\childdocforward|:
%    \begin{macrocode}
\newcommand{\childdocforwardprefix}[3][]
{
  \begingroup
    \def\childdocextract #2##1~~~{\def\childdoctmp{\childdocforward[#1]{#3##1}}}
    \expandafter\childdocextract\childdocname~~~
    \expandafter
  \endgroup
  \childdoctmp
}
%    \end{macrocode}

% \macro{\childdoc}
% The deprecated macro |\childdoc| is a legacy version of |\childdocmain|:
%    \begin{macrocode}
\newcommand{\childdoc}{\childdocmain}
%    \end{macrocode}

% \macro{\childdocredirect}
% The deprecated macro |\childdocredirect| is a legacy version
% of |\childdocforward| and |\childdocforwardprefix|:
%    \begin{macrocode}
\newcommand{\childdocredirect}[2][]
{
  \begingroup
    \if?#1?
      \def\childdoctmp{\childdocforward{#2}}
    \else
      \def\childdoctmp{\childdocforwardprefix{#1}{#2}}
    \fi
    \expandafter
  \endgroup
  \childdoctmp
}
%    \end{macrocode}

%\iffalse
%</package>
%\fi
%
\endinput
|\\
|\childdocforwardprefix{final}{child}|
\end{tabular}
\end{center}
%

Note that when several versions of a main file and/or of each child file
are to be generated, it may be convenient to set up a |Makefile| or
shell script to automatise the process.

%%%%%%%%%%%%%%%%%%%%%%%%%%%%%%%%%%%%%%%%%%%%%%%%%%%%%%%%%%%%%%%%%%%%%%%%%%%%%%%%
\subsection{Command Line Processing}
\label{sec:commandline}

The effect of redirection files can also be achieved by invoking
the \LaTeX{} compiler with a more elaborate command line.
Most conveniently this should be done as part
of a shell script or a |Makefile|.

When using \textsf{childdoc} in the main file, the following
command lines effectively perform a redirection
(note that depending on the shell being used,
backslashes may have to be doubled: `|\|' $\to$ `|\\|'):
%
\begin{center}
|... -jobname "|\textit{target}|" |\\|"|[\textit{flags}]%
|% \iffalse
%
% childdoc.dtx Copyright (C) 2017-2018 Niklas Beisert
%
% This work may be distributed and/or modified under the
% conditions of the LaTeX Project Public License, either version 1.3
% of this license or (at your option) any later version.
% The latest version of this license is in
%   http://www.latex-project.org/lppl.txt
% and version 1.3 or later is part of all distributions of LaTeX
% version 2005/12/01 or later.
%
% This work has the LPPL maintenance status `maintained'.
%
% The Current Maintainer of this work is Niklas Beisert.
%
% This work consists of the files childdoc.dtx and childdoc.ins
% and the derived files childdoc.def and cdocsamp.tex with
% cdocsch1.tex, cdocsch2.tex, cdocsdrf.tex, cdocsfn1.tex, cdocsfn2.tex.
%
%<package>\ifdefined\childdocmain\endinput\fi
%<package>\ProvidesFile{childdoc.def}[2018/12/30 v2.0 child document driver]
%<samplemain>\ProvidesFile{cdocsamp.tex}[2018/12/30 v2.0 sample for childdoc]
%<*driver>
%\ProvidesFile{childdoc.drv}[2018/12/30 v2.0 childdoc reference manual file]
\PassOptionsToClass{10pt,a4paper}{article}
\documentclass{ltxdoc}

\usepackage[margin=35mm]{geometry}
\usepackage{hyperref}
\usepackage{hyperxmp}
\usepackage[usenames]{color}

\hypersetup{colorlinks=true}
\hypersetup{pdfstartview=FitH}
\hypersetup{pdfpagemode=UseNone}
\hypersetup{pdfsource={}}
\hypersetup{pdflang={en-UK}}
\hypersetup{pdfcopyright={Copyright 2017-2018 Niklas Beisert.
  This work may be distributed and/or modified under the
  conditions of the LaTeX Project Public License, either version 1.3
  of this license or (at your option) any later version.}}
\hypersetup{pdflicenseurl={http://www.latex-project.org/lppl.txt}}
\hypersetup{pdfcontactaddress={ETH Zurich, ITP, HIT K,
  Wolfgang-Pauli-Strasse 27}}
\hypersetup{pdfcontactpostcode={8093}}
\hypersetup{pdfcontactcity={Zurich}}
\hypersetup{pdfcontactcountry={Switzerland}}
\hypersetup{pdfcontactemail={nbeisert@itp.phys.ethz.ch}}
\hypersetup{pdfcontacturl={http://people.phys.ethz.ch/\xmptilde nbeisert/}}

\newcommand{\secref}[1]{\hyperref[#1]{section \ref*{#1}}}

\parskip1ex
\parindent0pt
\let\olditemize\itemize
\def\itemize{\olditemize\parskip0pt}

\begin{document}

\title{The \textsf{childdoc} Package}
\hypersetup{pdftitle={The childdoc Package}}
\author{Niklas Beisert\\[2ex]
  Institut f\"ur Theoretische Physik\\
  Eidgen\"ossische Technische Hochschule Z\"urich\\
  Wolfgang-Pauli-Strasse 27, 8093 Z\"urich, Switzerland\\[1ex]
  \href{mailto:nbeisert@itp.phys.ethz.ch}
  {\texttt{nbeisert@itp.phys.ethz.ch}}}
\hypersetup{pdfauthor={Niklas Beisert}}
\hypersetup{pdfsubject={Manual for the LaTeX2e Package childdoc}}
\date{30 December 2018, \textsf{v2.0}}
\maketitle

\begin{abstract}\noindent
\textsf{childdoc} is a \LaTeXe{} package
that enables the direct compilation
of document sections included by |\include|
to individual files.
\end{abstract}

\begingroup
\parskip0ex
\tableofcontents
\endgroup

%%%%%%%%%%%%%%%%%%%%%%%%%%%%%%%%%%%%%%%%%%%%%%%%%%%%%%%%%%%%%%%%%%%%%%%%%%%%%%%%
%%%%%%%%%%%%%%%%%%%%%%%%%%%%%%%%%%%%%%%%%%%%%%%%%%%%%%%%%%%%%%%%%%%%%%%%%%%%%%%%
\section{Introduction}

\LaTeX{} provides a mechanism to structure a large document (such as a book)
into a main file and several child files (containing the chapters)
using the |\include| command.
This mechanism is beneficial for documents
which span hundreds of pages in order to
make the source file(s) more manageable.
Moreover, compilation can be restricted to
selected child files by means of the |\includeonly| command.
The latter feature can be used to reduce the compilation time while editing
(this was significantly more useful in the earlier days of \LaTeX{})
or to generate a smaller document which is easier to navigate.
Another application of |\includeonly| is to generate
documents consisting of selected parts of the complete document.

However, there are a few drawbacks of the plain |\include| mechanism:
\begin{itemize}
\item
The child files cannot be compiled on their own,
they can only be compiled via the main file.
A naive editing environment
(such as a text editor with an option
to have the current file processed by \LaTeX)
may require one to switch to the main file before compiling;
attempting to compile the child file produces errors.
\item
The main file must be modified (each time)
to adjust the |\includeonly| command
to the present needs. This easily leaves the main file in a messy state.
\item
The generated document will always carry the filename
of the main document. This is inconvenient if
several child files are to be compiled and
to be kept for distribution.
\end{itemize}

The present package provides a simple interface
to make child files individually compilable by \LaTeX{}.
Compiling a child file then has the same effect as compiling
the main file with an |\includeonly| command
to select the appropriate child.
Moreover the generated document will carry the name of the child
rather than the main file.
This resolves all three above issues.

This feature is meant to make the editing of books,
thesis documents and lecture notes somewhat more convenient.
However, the package can also be used efficiently for
composing a series of documents (such as exercise sheets)
which are typically distributed individually.
It then assists the author in generating the individual documents
(potentially in different versions)
as well as a document containing the collected series.
Another application is in developing style files
or other kinds of included material
where compilation of the style file could redirect
to a sample or test file.

%%%%%%%%%%%%%%%%%%%%%%%%%%%%%%%%%%%%%%%%%%%%%%%%%%%%%%%%%%%%%%%%%%%%%%%%%%%%%%%%
%%%%%%%%%%%%%%%%%%%%%%%%%%%%%%%%%%%%%%%%%%%%%%%%%%%%%%%%%%%%%%%%%%%%%%%%%%%%%%%%
\section{Usage}

First of all, the package \textsf{childdoc} is \emph{not} a standard
\LaTeXe{} |.sty| style file! Therefore it needs to be invoked in
a non-standard way.

%%%%%%%%%%%%%%%%%%%%%%%%%%%%%%%%%%%%%%%%%%%%%%%%%%%%%%%%%%%%%%%%%%%%%%%%%%%%%%%%
\subsection{Included Files}
\label{sec:include}

%%%%%%%%%%%%%%%%%%%%%%%%%%%%%%%%%%%%%%%%
\DescribeMacro{\childdocmain}
To use the package, add the commands
\begin{center}
\begin{tabular}{l}
|\input{childdoc.def}|\\
|\childdocmain{}|\\
\end{tabular}
\end{center}
at the very top of the main \LaTeX{} file,
in particular \emph{before} the |\documentclass| statement!
The argument of |\childdocmain| should be left empty
(but it must be present).

%%%%%%%%%%%%%%%%%%%%%%%%%%%%%%%%%%%%%%%%
\DescribeMacro{\childdocof}
Furthermore, add the commands
\begin{center}
\begin{tabular}{l}
|\input{childdoc.def}|\\
|\childdocof{|\textit{main}|}|\\
\end{tabular}
\end{center}
at the top of every child file \textit{child}
which is included by |\include{|\textit{child}|}|
from within the main file
(or at least for those files to be compiled individually).
The argument \textit{main} must be the filename of the main file.

There are a couple of
considerations in setting up the main and child documents:

%%%%%%%%%%%%%%%%%%%%%%%%%%%%%%%%%%%%%%%%
\paragraph{Restrictions.}

Please note the following restrictions:
\begin{itemize}
\item
|\childdocmain| must be called with one argument \textit{main}
to ensure compatibility with earlier version of the package.
It must either be empty (|\childdocmain{}|)
or precisely match the filename of the main file in which it is specified.
See \secref{sec:detection} for further information.
\item
The filename \textit{main} must be specified without the |.tex| extension.
\item
The filename \textit{main} is case sensitive
(even in case-insensitive file systems)
due to internal string comparison.
\item
The argument \textit{main} should be fully expanded, it cannot be a macro.
\item
Subdirectories and special characters should be avoided in filenames.
\item
The command |\childdocmain{|\textit{main}|}| must be followed by a whitespace.
It should not be followed immediately by another command
or by a comment mark `|%|'.
This is because the \TeX{} parser reads the token immediately following
the argument of |\childdocmain| and puts it
at the beginning of every child section;
however, a white\-space is ignored.
\end{itemize}

%%%%%%%%%%%%%%%%%%%%%%%%%%%%%%%%%%%%%%%%
\paragraph{Content of Main File.}

It is advisable to place all content in the child files included by |\include|.
Any output contained in the main file will appear in all child documents
unless suppressed manually;
it cannot be suppressed automatically by the |\includeonly| directive
and thus should normally be avoided.
A method to include some content in the main file
by means of conditional processing is described in \secref{sec:conditional}.

%%%%%%%%%%%%%%%%%%%%%%%%%%%%%%%%%%%%%%%%
\paragraph{Page Numbering.}

When only a part of the document is compiled,
the appropriate numbering of pages
(as well as other status parameters)
is determined from the |.aux| files.
The latter contain information from previous passes.
However this information needs to propagate through
all intermediate child documents.
Therefore the page numbering in child documents may well
be inconsistent until the complete document is compiled at least once.

A useful (if unconventional) way to always ensure a consistent
page numbering is to restart the numbering in each child document
and denote the pages by `\textit{child}|.|\textit{page}'
where \textit{child} represents the chapter/section number of the child file.
This can be achieved by the command
|\numberwithin{page}{|\textit{child}|}|
of the \textsf{amsmath} package
where \textit{child} can be |chapter| or |section|
depending on the chosen structuring.
Alternatively, one can modify the macro |\thepage| appropriately
and reset the counter |page| at the start of each child file.

%%%%%%%%%%%%%%%%%%%%%%%%%%%%%%%%%%%%%%%%%%%%%%%%%%%%%%%%%%%%%%%%%%%%%%%%%%%%%%%%
\subsection{Conditional Processing}
\label{sec:conditional}

The package provides a mechanism to compile different versions
of a document. To customise the versions further some conditional processing
can come in handy to distinguish which version is being compiled.
The package provides two macros to describe the compilation context:

%%%%%%%%%%%%%%%%%%%%%%%%%%%%%%%%%%%%%%%%
\DescribeMacro{\ifchilddoc}
The conditional |\ifchilddoc| distinguishes between the compilation of
child documents and the main document:
%
\begin{center}
|\ifchilddoc |\textit{child-code}| |[|\||else |\textit{main-code}]| \||fi|
\end{center}

%%%%%%%%%%%%%%%%%%%%%%%%%%%%%%%%%%%%%%%%
\DescribeMacro{\childdocname}
\DescribeMacro{\childdocjob}
The macro |\childdocname| contains the filename (without extension)
of the main or child file being processed.
Note that |\childdocjob| will always contain the name of the main file.

%%%%%%%%%%%%%%%%%%%%%%%%%%%%%%%%%%%%%%%%
\paragraph{Title Page.}

Conditional processing can be used to include a title or banner page
in the main document when proper precautions are taken.
Importantly, the code in the main file should ensure that the page counter
(as well as other status parameters which are stored in the |.aux| files)
takes the same value after the conditional processing.
Otherwise the page numbers may take divergent values
depending on which part is compiled.

For example, a title page could be declared by:
%
\begin{center}
\begin{tabular}{l}
|\ifchilddoc\||else|\\
|\addtocounter{page}{-1}|\\
\textit{code for title page}\\
|\newpage|\\
|\||fi|
\end{tabular}
\end{center}
%
A banner page for the child documents can be generated by:
%
\begin{center}
\begin{tabular}{l}
|\ifchilddoc|\\
|\addtocounter{page}{-1}|\\
\textit{code for banner page}\\
|\newpage|\\
|\||fi|
\end{tabular}
\end{center}
%
Here one could write a message such as:
\begin{center}
|This is the part \childdocname{} of \childdocjob{}.|
\end{center}

%%%%%%%%%%%%%%%%%%%%%%%%%%%%%%%%%%%%%%%%%%%%%%%%%%%%%%%%%%%%%%%%%%%%%%%%%%%%%%%%
\subsection{Flags}
\label{sec:flags}

The package makes it easy to generate different versions
of the main or child documents.
To this end compilation flags can be defined
and assigned different default values.
They will be particularly useful in conjunction
with the forwarding mechanism described in \secref{sec:forward}.

For example, it may be useful to have a flag |\version|
which can be set to |draft| or |final|.
The document source will contain some conditional code
depending on the value of |\version|.
Suppose further, the flag should default to |final| for the main file
and to |draft| for child files
which is a natural assignment for editing the document.
This is achieved by placing the following code
in the preamble of the main document
(below the |\childdocmain| directive):
%
\begin{center}
\begin{tabular}{l}
|\ifchilddoc|\\
|\providecommand{\version}{draft}|\\
|\||else|\\
|\providecommand{\version}{final}|\\
|\||fi|
\end{tabular}
\end{center}
%
The definition by |\providecommand| makes sure
that previous definitions are not overwritten.
Further statements |\providecommand{\version}{...}|
can thus be added before the above code to override it.

For the main file, one might add a line
(between |\childdocmain| and the above block)
%
\begin{center}
|%\ifchilddoc\||else\providecommand{\version}{draft}\||fi|
\end{center}
%
which can be uncommented to produce a draft version.
Likewise one can add a line to the very top of a child file
(above the |\childdocof{|\textit{main}|}| directive)
%
\begin{center}
|%\providecommand{\version}{final}|
\end{center}
%
which can be uncommented to produce the final version of this child document.

%%%%%%%%%%%%%%%%%%%%%%%%%%%%%%%%%%%%%%%%%%%%%%%%%%%%%%%%%%%%%%%%%%%%%%%%%%%%%%%%
\subsection{Forwarding}
\label{sec:forward}

Different versions of the main or child documents
using compilation flags as described in \secref{sec:flags}
can be (permanently) stored in different files
for convenient compilation, viewing and distribution.
To this end, the package defines a command
to pass on compilation to a different file:

%%%%%%%%%%%%%%%%%%%%%%%%%%%%%%%%%%%%%%%%
\DescribeMacro{\childdocforward}
The command |\childdocforward| redirects processing to
another source file:
%
\begin{center}
\begin{tabular}{l}
|\input{childdoc.def}|\\
|\childdocforward[|\textit{main}|]{|\textit{dest}|}|\\
\end{tabular}
\end{center}
%
The argument \textit{dest} is the destination file
(without extension).
It should be the main file or one of the child files.
Note that further \textsf{childdoc} directives
such as |\childdocof| and |\childdocforward|
in the indicated file will be processed in this form.
The optional argument \textit{main}
passes on directly to the main file \textit{main}
while pretending to compile the child \textit{dest}.
This form behaves as if \textit{dest}
issues |\childdocof{|\textit{main}|}| right away,
and no further \textsf{childdoc} directives will be processed.

%%%%%%%%%%%%%%%%%%%%%%%%%%%%%%%%%%%%%%%%
\DescribeMacro{\...prefix}
In the alternative form |\childdocforwardprefix|,
%
\begin{center}
\begin{tabular}{l}
|\input{childdoc.def}|\\
|\childdocforwardprefix[|\textit{main}|]{|\textit{prefix}|}{|\textit{dest}|}|
\end{tabular}
\end{center}
%
the destination file is determined by a pattern
depending on the current file:
To make this work, the current file must be called
`{\textit{prefix}\hspace{0.2em}\textit{suffix}}'
with \textit{prefix} matching precisely the argument.
Processing is then passed on to the file
`{\textit{dest}\hspace{0.2em}\textit{suffix}}'.
Surely, the same effect is achieved by
directly specifying the
argument `{\textit{dest}\hspace{0.2em}\textit{suffix}}'
in the first form.
However, that requires to set up a different file
for each child. With the alternative form of the command
all these files can have exactly the same content
which simplifies setting them up and maintaining them.

For example, the following file |draft.tex|
with a compilation flag |\version| as described in \secref{sec:flags}
compiles the main document as a draft:
%
\begin{center}
\begin{tabular}{l}
|\def\version{draft}|\\
|\input{childdoc.def}|\\
|\childdocforward{|\textit{main}|}|
\end{tabular}
\end{center}
%
Likewise, the following files |final|\textit{nn}|.tex|
compile the final version of the child document
|child|\textit{nn}|.tex|:
%
\begin{center}
\begin{tabular}{l}
|\def\version{final}|\\
|\input{childdoc.def}|\\
|\childdocforwardprefix{final}{child}|
\end{tabular}
\end{center}
%

Note that when several versions of a main file and/or of each child file
are to be generated, it may be convenient to set up a |Makefile| or
shell script to automatise the process.

%%%%%%%%%%%%%%%%%%%%%%%%%%%%%%%%%%%%%%%%%%%%%%%%%%%%%%%%%%%%%%%%%%%%%%%%%%%%%%%%
\subsection{Command Line Processing}
\label{sec:commandline}

The effect of redirection files can also be achieved by invoking
the \LaTeX{} compiler with a more elaborate command line.
Most conveniently this should be done as part
of a shell script or a |Makefile|.

When using \textsf{childdoc} in the main file, the following
command lines effectively perform a redirection
(note that depending on the shell being used,
backslashes may have to be doubled: `|\|' $\to$ `|\\|'):
%
\begin{center}
|... -jobname "|\textit{target}|" |\\|"|[\textit{flags}]%
|\input{childdoc.def}\childdocforward[|\textit{main}|]{|\textit{dest}|}"|
\end{center}
%
Here \textit{target} is the name of the output file,
\textit{main} is the name of the main file
and \textit{dest} is the name of the main or child file to be processed
(all filenames without extensions).
The optional argument \textit{main} can be omitted
if \textit{main} matches \textit{dest}.
Optionally, compilation \textit{flags} can be defined via |\def| commands.
This command line makes the \TeX{} engine believe
it is compiling the file \textit{target}
whose content is specified as the latter parameter.
The provided code then forwards the processing to
\textit{main} or \textit{dest} as described in \secref{sec:forward}.

%%%%%%%%%%%%%%%%%%%%%%%%%%%%%%%%%%%%%%%%%%%%%%%%%%%%%%%%%%%%%%%%%%%%%%%%%%%%%%%%
\subsection{Include by Input}
\label{sec:input}

Including child documents by |\include| has some restrictions by design.
Most notably, the content of a child document always occupies
its own set of pages; pages cannot be shared between child documents.
Usually, this behaviour makes perfect sense
because each child document contain an essential part of the document.
However, in some situations it may be desirable to compose
a document from a collection of parts
without having mandatory page breaks between then.
For this case, the package
provides a mechanism to include parts
by |\input| which can also be processed individually.
However, by construction this mechanism
requires manual handling of the content to be output.

%%%%%%%%%%%%%%%%%%%%%%%%%%%%%%%%%%%%%%%%
\DescribeMacro{\ifchilddocmanual}
The main file should be prepared as usual, see \secref{sec:include}.
However, the document body must make a distinction
between processing of an individual part and of the main document, e.g.:
%
\begin{center}
\begin{tabular}{l}
|\ifchilddocmanual|\\
|\input{\childdocname}|\\
|\||else|\\
\textit{document body with }|\input{|\textit{part}|}|\\
|\||fi|
\end{tabular}
\end{center}
%
The conditional |\ifchilddocmanual| is true whenever
a part to be included by |\input| is being compiled,
and the name of the part is stored in |\childdocname|.

%%%%%%%%%%%%%%%%%%%%%%%%%%%%%%%%%%%%%%%%
\DescribeMacro{\childdocby}
Each part to be included by |\input| should start with:
%
\begin{center}
\begin{tabular}{l}
|\input{childdoc.def}|\\
|\childdocby{|\textit{main}|}|\\
\end{tabular}
\end{center}
%
The directive |\childdocby| is similar to |\childdocof|
described in \secref{sec:include},
but the subsequent selection of content must be done manually.
To that end, both |\ifchilddoc| and |\ifchilddocmanual|
will be true upon processing of a part,
and the name of the part is stored in |\childdocname|.
Note that |\jobname| will be set to the filename of the current part
so that each part receives an individual |.aux| file
that does not interfere with the |.aux| file(s) of the main document.
This behaviour can be altered by the alternative form
|\childdocby[*]{|\textit{main}|}| (with a non-empty optional argument)
which uses the |.aux| file of the main document
by setting |\jobname| to \textit{main}.

%%%%%%%%%%%%%%%%%%%%%%%%%%%%%%%%%%%%%%%%%%%%%%%%%%%%%%%%%%%%%%%%%%%%%%%%%%%%%%%%
\subsection{Driver Development}
\label{sec:driver}

The \textsf{childdoc} mechanism can also be use for the development
of definition files such as \LaTeX{} styles or classes.
This case differs from the above setup with multiple parts
included by |\include| in that no |\includeonly| should be invoked.
This can be achieved by starting the include file
(before |\ProvidesPackage|) with:
%
\begin{center}
\begin{tabular}{l}
|\input{childdoc.def}|\\
|\childdocforward{|\textit{main}|}|\\
\end{tabular}
\end{center}
%
or alternatively with:
%
\begin{center}
\begin{tabular}{l}
|\input{childdoc.def}|\\
|\childdocby{|\textit{main}|}|\\
\end{tabular}
\end{center}
%
Both forms have slightly different effects as described above.
The main file is prepared as usual, see \secref{sec:include}.

%%%%%%%%%%%%%%%%%%%%%%%%%%%%%%%%%%%%%%%%%%%%%%%%%%%%%%%%%%%%%%%%%%%%%%%%%%%%%%%%
\subsection{Legacy Detection}
\label{sec:detection}

The directive |\childdocmain| in the main file can detect
whether the complete document or merely a child is to be compiled
even without using the directive |\childdocof|.
This method is deprecated because it is less robust
and there is no compelling reason to use it;
it is merely provided for backward compatibility
and it may be removed in future versions.

If the detection mechanism is to be used,
it is mandatory to correctly specify
the filename of the main file as the argument of |\childdocmain|:
%
\begin{center}
\begin{tabular}{l}
|\input{childdoc.def}|\\
|\childdocmain{|\textit{main}|}|\\
\end{tabular}
\end{center}
%
If |\jobname| does not match the argument \textit{main} of |\childdocmain|,
it is assumed that |\jobname| points to the child file to be compiled.
When using |\childdocmain| with the main file specified as argument,
it suffices to start a child file
with just |\input{|\textit{main}|}|
without loading of the package and using |\childdocof|.
If instead all processing is done
with the appropriate \textsf{childdoc} directives,
the argument of \textit{main} of |\childdocmain| can be empty.

An alternative version of the command line processing described
in \secref{sec:commandline} using the detection mechanism reads:
%
\begin{center}
|... -jobname "|\textit{target}|" "|[\textit{flags}]%
[|\def\jobname{|\textit{dest}|}|]|\input{|\textit{main}|}"|
\end{center}

%%%%%%%%%%%%%%%%%%%%%%%%%%%%%%%%%%%%%%%%%%%%%%%%%%%%%%%%%%%%%%%%%%%%%%%%%%%%%%%%
\subsection{Manual Code}
\label{sec:manual}

In case one cannot be certain whether the definitions file |childdoc.def|
is installed on the target \TeX{} distribution
and one prefers not to ship it,
it is conceivable to paste a few relevant commands into the sources.

To that end, drop all statements |\input{childdoc.def}|
and perform the replacements as outlined below.
Instead of |\childdocmain{|\textit{main}|}| add the following code
to the top of the main file:
%
\begin{center}
\begin{tabular}{l}
|\||ifdefined\childdocname\endinput\||fi\newif\ifchilddoc|\\
|\edef\childdocname{\scantokens\expandafter{\jobname\noexpand}}|\\
|\def\childdocmain{|\textit{main}|}\||ifx\childdocmain\childdocname\||else|\\
|\childdoctrue\includeonly{\childdocname}\let\jobname\childdocmain\||fi|\\
\end{tabular}
\end{center}
%
Instead of |\childdocof{|\textit{main}|}| just include the main file
at the top of each child file:
%
\begin{center}
|\input{|\textit{main}|}|
\end{center}
%
A simple redirection |\childdocforward{|\textit{dest}|}| is achieved by:
%
\begin{center}
|\def\jobname{|\textit{dest}|}\input{\jobname}|
\end{center}
%
The redirection with prefix
|\childdocforwardprefix[|\textit{prefix}|]{|\textit{dest}|}|
is accomplished by:
%
\begin{center}
\begin{tabular}{l}
|{\edef\jobname{\scantokens\expandafter{\jobname\noexpand}}|\\
|\def\redirectjob |\textit{prefix}|#1~~~{\gdef\jobname{|\textit{dest}|#1}}|\\
|\expandafter\redirectjob\jobname~~~}\input{\jobname}|
\end{tabular}
\end{center}

In an alternative approach,
child documents can be compiled by a specific command line
without additional code or specific definitions:
%
\begin{center}
|... -jobname "|\textit{target}|" "|[\textit{flags}]%
|\includeonly{|\textit{dest}|}\input{|\textit{main}|}"|
\end{center}
%

%%%%%%%%%%%%%%%%%%%%%%%%%%%%%%%%%%%%%%%%%%%%%%%%%%%%%%%%%%%%%%%%%%%%%%%%%%%%%%%%
%%%%%%%%%%%%%%%%%%%%%%%%%%%%%%%%%%%%%%%%%%%%%%%%%%%%%%%%%%%%%%%%%%%%%%%%%%%%%%%%
\section{Information}

%%%%%%%%%%%%%%%%%%%%%%%%%%%%%%%%%%%%%%%%%%%%%%%%%%%%%%%%%%%%%%%%%%%%%%%%%%%%%%%%
\subsection{Copyright}

Copyright \copyright{} 2017--2018 Niklas Beisert

This work may be distributed and/or modified under the
conditions of the \LaTeX{} Project Public License, either version 1.3
of this license or (at your option) any later version.
The latest version of this license is in
  \url{http://www.latex-project.org/lppl.txt}
and version 1.3 or later is part of all distributions of \LaTeX{}
version 2005/12/01 or later.

This work has the LPPL maintenance status `maintained'.

The Current Maintainer of this work is Niklas Beisert.

This work consists of the files |README.txt|, |childdoc.ins| and |childdoc.dtx|
as well as the derived files |childdoc.def|, |cdocsamp.tex|
with |cdocsch1.tex|, |cdocsch2.tex|, |cdocspt3.tex|, |cdocspt4.tex|,
|cdocsdrf.tex|, |cdocsfn1.tex|, |cdocsfn2.tex|
as well as |childdoc.pdf|.

%%%%%%%%%%%%%%%%%%%%%%%%%%%%%%%%%%%%%%%%%%%%%%%%%%%%%%%%%%%%%%%%%%%%%%%%%%%%%%%%
\subsection{Files and Installation}

The package consists of the files:
%
\begin{center}
\begin{tabular}{ll}
    |README.txt|   & readme file \\
    |childdoc.ins| & installation file \\
    |childdoc.dtx| & source file \\
    |childdoc.def| & definition file \\
    |cdocsamp.tex| & sample main file \\
    |cdocsch1.tex| & sample include file \\
    |cdocsch2.tex| & sample include file \\
    |cdocspt3.tex| & sample part file \\
    |cdocspt4.tex| & sample part file \\
    |cdocsdrf.tex| & sample redirection file \\
    |cdocsfn1.tex| & sample redirection file \\
    |cdocsfn2.tex| & sample redirection file \\
    |childdoc.pdf| & manual
\end{tabular}
\end{center}
%
The distribution consists of the files
|README.txt|, |childdoc.ins| and |childdoc.dtx|.
%
\begin{itemize}
\item
Run (pdf)\LaTeX{} on |childdoc.dtx|
to compile the manual |childdoc.pdf| (this file).
\item
Run \LaTeX{} on |childdoc.ins| to create the definitions file |childdoc.def|
and the sample |cdocsamp.tex| with include files
|cdocsch1.tex|, |cdocsch2.tex|, |cdocspt3.tex|, |cdocspt4.tex|,
|cdocsdrf.tex|, |cdocsfn1.tex|, |cdocsfn2.tex|.
Then copy the file |childdoc.def| to an appropriate directory of your \LaTeX{}
distribution, e.g.\ \textit{texmf-root}|/tex/latex/childdoc|.
\end{itemize}

%%%%%%%%%%%%%%%%%%%%%%%%%%%%%%%%%%%%%%%%%%%%%%%%%%%%%%%%%%%%%%%%%%%%%%%%%%%%%%%%
\subsection{Related CTAN Packages}

There are several other packages which offer a similar functionality:
%
\begin{itemize}
\item
The packages
\href{http://ctan.org/pkg/docmute}{\textsf{docmute}},
\href{http://ctan.org/pkg/includex}{\textsf{includex}} and
\href{http://ctan.org/pkg/standalone}{\textsf{standalone}}
provide commands to include only the document body of
a child file thus allowing both files to be compiled individually.
\item
The packages \href{http://ctan.org/pkg/subdocs}{\textsf{subdocs}}
and \href{http://ctan.org/pkg/subfiles}{\textsf{subfiles}}
provide structures in which the main and child documents can be
encapsulated and allowing them to be compiled individually.
The inclusion mechanism is different from the conventional |\include|.
\item
The package \href{http://ctan.org/pkg/combine}{\textsf{combine}}
is an elaborate solution to combine several documents into one.
\end{itemize}
%
See also the CTAN topic \href{http://ctan.org/topic/subdocs}{\textsf{subdocs}}
for further related packages.
The present package differs from the above solutions in that
a document structure constructed with the conventional |\include| mechanism
just needs two extra commands at the top of every file
such that all constituent files can be compiled individually.

%%%%%%%%%%%%%%%%%%%%%%%%%%%%%%%%%%%%%%%%%%%%%%%%%%%%%%%%%%%%%%%%%%%%%%%%%%%%%%%%
%\subsection{Feature Suggestions}
%
%The following is a list of features which may be useful for future
%versions of this package:
%%
%\begin{itemize}
%\item
%\ldots
%\end{itemize}

%%%%%%%%%%%%%%%%%%%%%%%%%%%%%%%%%%%%%%%%%%%%%%%%%%%%%%%%%%%%%%%%%%%%%%%%%%%%%%%%
\subsection{Revision History}

%%%%%%%%%%%%%%%%%%%%%%%%%%%%%%%%%%%%%%%%
\paragraph{v2.0:} 2018/12/30

\begin{itemize}
\item
immediate forward processing
\item
added |\childdocby| mechanism
\item
manual restructured
\end{itemize}

%%%%%%%%%%%%%%%%%%%%%%%%%%%%%%%%%%%%%%%%
\paragraph{v1.6:} 2018/01/17

\begin{itemize}
\item
application for development of include files
\item
corrections to manual
\end{itemize}

%%%%%%%%%%%%%%%%%%%%%%%%%%%%%%%%%%%%%%%%
\paragraph{v1.5:} 2017/05/21

\begin{itemize}
\item
more complete structuring introduced
\item
|\childdocof| introduced
\item
|\childdoc| renamed to |\childdocmain|
\item
|\childredirect| renamed to |\childdocforward| and |\childdocforwardprefix|
and functionality expanded
\end{itemize}

%%%%%%%%%%%%%%%%%%%%%%%%%%%%%%%%%%%%%%%%
\paragraph{v1.0:} 2017/04/27

\begin{itemize}
\item
manual and install package
\item
first version published on CTAN
\end{itemize}

%%%%%%%%%%%%%%%%%%%%%%%%%%%%%%%%%%%%%%%%
\paragraph{v0.6:} 2017/04/26

\begin{itemize}
\item
redirection mechanism added
\end{itemize}

%%%%%%%%%%%%%%%%%%%%%%%%%%%%%%%%%%%%%%%%
\paragraph{v0.5:} 2017/04/26

\begin{itemize}
\item
functionality in definition file
\end{itemize}


%%%%%%%%%%%%%%%%%%%%%%%%%%%%%%%%%%%%%%%%%%%%%%%%%%%%%%%%%%%%%%%%%%%%%%%%%%%%%%%%
%%%%%%%%%%%%%%%%%%%%%%%%%%%%%%%%%%%%%%%%%%%%%%%%%%%%%%%%%%%%%%%%%%%%%%%%%%%%%%%%
%%%%%%%%%%%%%%%%%%%%%%%%%%%%%%%%%%%%%%%%%%%%%%%%%%%%%%%%%%%%%%%%%%%%%%%%%%%%%%%%
\appendix

\settowidth\MacroIndent{\rmfamily\scriptsize 000\ }

 \DocInput{childdoc.dtx}

\end{document}
%</driver>
% \fi
%
% %%%%%%%%%%%%%%%%%%%%%%%%%%%%%%%%%%%%%%%%%%%%%%%%%%%%%%%%%%%%%%%%%%%%%%%%%%%%%%
% %%%%%%%%%%%%%%%%%%%%%%%%%%%%%%%%%%%%%%%%%%%%%%%%%%%%%%%%%%%%%%%%%%%%%%%%%%%%%%
% \section{Sample}
%\iffalse
%<*samplemain>
%\fi
%
% The following presents a sample document
% with two chapters, two parts, a title page,
% a compile flag as well as three forwarding files to set the flag.
% It consists of eight |.tex| files:
% \begin{center}
% \begin{tabular}{ll}
% |cdocsamp.tex|&main file\\
% |cdocsch1.tex|&include file for chapter 1\\
% |cdocsch2.tex|&include file for chapter 2\\
% |cdocspt3.tex|&include file for part 3\\
% |cdocspt4.tex|&include file for part 4\\
% |cdocsdrf.tex|&forwarding file for main file in draft mode\\
% |cdocsfi1.tex|&forwarding file for final version of chapter 1\\
% |cdocsfi2.tex|&forwarding file for final version of chapter 2\\
% \end{tabular}
% \end{center}
% Each of the eight files can be compiled directly by the \LaTeX{} compiler.
%
% %%%%%%%%%%%%%%%%%%%%%%%%%%%%%%%%%%%%%%
% \paragraph{Main File.}
%
% The main file is called |cdocsamp.tex|.
%
% Load the \textsf{childdoc} definitions and
% declare the filename for the main document:
%    \begin{macrocode}
\input{childdoc.def}
\childdocmain{}
%    \end{macrocode}

% Optional override for |\version| flag:
%    \begin{macrocode}
%%\ifchilddoc\else\providecommand{\version}{draft}\fi
%    \end{macrocode}

% Define the default values for the |\version| flag
% (|final| for the main file and |draft| for childs):
%    \begin{macrocode}
\ifchilddoc
\providecommand{\version}{draft}
\else
\providecommand{\version}{final}
\fi
%    \end{macrocode}

% Load the standard document class:
%    \begin{macrocode}
\documentclass[12pt]{article}
%    \end{macrocode}

% Start the document body:
%    \begin{macrocode}
\begin{document}
%    \end{macrocode}

% Declare a title page.
% Print title, part of document being processed and version flag:
%    \begin{macrocode}
\addtocounter{page}{-1}
\begin{center}
{\LARGE\bfseries{}childdoc example\par}
\vspace{1cm}
\ifchilddoc
\ifchilddocmanual part\else chapter\fi:
`\childdocname' of `\childdocjob'\par
\else
main document: `\childdocjob'\par
\fi
version: \version\par
\end{center}
\newpage
%    \end{macrocode}

% Manually include selected file,
% otherwise process as usual:
%    \begin{macrocode}
\ifchilddocmanual
\section*{part `\childdocname'}
\input{\childdocname}
\else
%    \end{macrocode}

% Include the two chapters:
%    \begin{macrocode}
\include{cdocsch1}
\include{cdocsch2}
%    \end{macrocode}

% Include the two parts unless only chapters should be displayed:
%    \begin{macrocode}
\ifchilddoc\else
\section{part three}
\input{cdocspt3}
\section{part four}
\input{cdocspt4}
\fi
%    \end{macrocode}

% Process as usual until here:
%    \begin{macrocode}
\fi
%    \end{macrocode}

% End of document body:
%    \begin{macrocode}
\end{document}
%    \end{macrocode}
%\iffalse
%</samplemain>
%\fi
%
% %%%%%%%%%%%%%%%%%%%%%%%%%%%%%%%%%%%%%%
% \paragraph{Chapter Include Files.}
%
% The include files are called |cdocsch1.tex| and |cdocsch2.tex|.
%
%\iffalse
%<*samplechap1|samplechap2>
%\fi

% Optional override for |\version| flag:
%    \begin{macrocode}
%%\providecommand{\version}{final}
%    \end{macrocode}

% Include the main document:
%    \begin{macrocode}
\input{childdoc.def}
\childdocof{cdocsamp}
%    \end{macrocode}

%\iffalse
%</samplechap1|samplechap2>
%\fi
%
%\iffalse
%<*samplechap1>
%\fi
% Some text for chapter 1:
%    \begin{macrocode}
\section{one}
some text in chapter one
%    \end{macrocode}

%\iffalse
%</samplechap1>
%\fi
% Some text for chapter 2:
%\iffalse
%<*samplechap2>
%\fi
%    \begin{macrocode}
\section{two}
more text in chapter two
%    \end{macrocode}

%\iffalse
%</samplechap2>
%\fi
%
% %%%%%%%%%%%%%%%%%%%%%%%%%%%%%%%%%%%%%%
% \paragraph{Part Include Files.}
%
% The include files are called |cdocspt3.tex| and |cdocspt4.tex|.
%
%\iffalse
%<*samplepart3|samplepart4>
%\fi

% Optional override for |\version| flag:
%    \begin{macrocode}
%%\providecommand{\version}{final}
%    \end{macrocode}

% Include the main document:
%    \begin{macrocode}
\input{childdoc.def}
\childdocby{cdocsamp}
%    \end{macrocode}

%\iffalse
%</samplepart3|samplepart4>
%\fi
%
%\iffalse
%<*samplepart3>
%\fi
% Some text for part 3:
%    \begin{macrocode}
some text in part three
%    \end{macrocode}

%\iffalse
%</samplepart3>
%\fi
% Some text for part 4:
%\iffalse
%<*samplepart4>
%\fi
%    \begin{macrocode}
more text in part four
%    \end{macrocode}

%\iffalse
%</samplepart4>
%\fi
%
% %%%%%%%%%%%%%%%%%%%%%%%%%%%%%%%%%%%%%%
% \paragraph{Forwarding for a Complete Draft.}
%
% The following forwarding file |cdocsdrf.tex|
% compiles the main document in draft mode:
%\iffalse
%<*sampledraft>
%\fi
%    \begin{macrocode}
\def\version{draft}
\input{childdoc.def}
\childdocforward{cdocsamp}
%    \end{macrocode}

%\iffalse
%</sampledraft>
%\fi
%
% %%%%%%%%%%%%%%%%%%%%%%%%%%%%%%%%%%%%%%
% \paragraph{Forwarding for Final Version of the Chapters.}
%
% The following forwarding files |cdocsfn1.tex| and |cdocsfn2.tex|
% (with identical content)
% compile the final versions of the child documents
% |cdocsch1.tex| and |cdocsch2.tex|, respectively:
%\iffalse
%<*samplefinal>
%\fi
%    \begin{macrocode}
\def\version{final}
\input{childdoc.def}
\childdocforwardprefix[cdocsamp]{cdocsfn}{cdocsch}
%    \end{macrocode}

%\iffalse
%</samplefinal>
%\fi
%
% %%%%%%%%%%%%%%%%%%%%%%%%%%%%%%%%%%%%%%
% \paragraph{Command Line Processing.}
%
% The following three command lines generate the output files
% |cdocscld|, |cdocscl1| and |cdocscl2|
% which should be identical to
% |cdocsdrf|, |cdocsch1| and |cdocsfn2|, respectively:
% \begin{center}
% \begin{tabular}{l}
% |latex -jobname cdocscld \|\\
% |  "\def\version{draft}\input{childdoc.def}\childdocforward{cdocsamp}"|\\
% |latex -jobname cdocscl1 \|\\
% |  "\input{childdoc.def}\childdocforward[cdocsamp]{cdocsch1}"|\\
% |latex -jobname cdocscl2 \|\\
% |  "\def\version{final}\input{childdoc.def}\childdocforward{cdocsch2}"|
% \end{tabular}
% \end{center}
% Note that the trailing backslash on each first line
% merely continues the input to the second line
% (for convenient cut ant paste).
% Furthermore, the command |latex| can be replaced by any
% of its alternative versions such as |pdflatex|.
%
% %%%%%%%%%%%%%%%%%%%%%%%%%%%%%%%%%%%%%%%%%%%%%%%%%%%%%%%%%%%%%%%%%%%%%%%%%%%%%%
% %%%%%%%%%%%%%%%%%%%%%%%%%%%%%%%%%%%%%%%%%%%%%%%%%%%%%%%%%%%%%%%%%%%%%%%%%%%%%%
% \section{Implementation}
%\iffalse
%<*package>
%\fi
%
% This section describes the definitions file |childdoc.def|.

% The definitions cannot be loaded using |\usepackage| or |\RequirePackage|
% which has a mechanism to prevent loading a style file more than once.
% When loading the definitions by means of |\input|
% multiple instances have to be prevented manually:
%\iffalse
%This code needs to be before the `\ProvidesFile' directive
%which is defined at the beginning of this file.
%Therefore it is also placed there and commented out here.
%</package>
%<*discard>
%\fi
%    \begin{macrocode}
\ifdefined\childdocmain\endinput\fi
%    \end{macrocode}
%\iffalse
%</discard>
%<*package>
%\fi
%
% \macro{\ifchilddoc}
% \macro{\ifchilddocmanual}
% The conditional |\ifchilddoc| tells whether a
% child (true) or main (false) document is being compiled.
% The conditional |\ifchilddocmanual| tells whether
% the |\includeonly| mechanism is used (false) or
% the selection of child files must be performed manually (true).
% The definitions initialise to false:
%    \begin{macrocode}
\newif\ifchilddoc
\newif\ifchilddocmanual
%    \end{macrocode}

% \macro{\childdocname}
% \macro{\childdocjob}
% The macro |\childdocname| stores the name of the main document
% to be compiled. The macro |\childdocjob| stores the name of
% the document on which the \LaTeX{} compiler was originally invoked.
% The content of |\jobname| cannot be compared
% to filenames specified in the source due to different catcodes.
% The following code rescans |\jobname|, stores the result
% in |\childdocname| and saves a copy in |\childdocjob|:
%    \begin{macrocode}
\edef\childdocname{\scantokens\expandafter{\jobname\noexpand}}
\let\childdocjob\childdocname
%    \end{macrocode}

% \macro{\childdocdisable}
% The macro |\childdocdisable| prevents the main file
% from being processed more than once.
% At this stage, the main document command |\childdocmain|
% is assumed to be called once again where it should do nothing.
% Any subsequent call to it should prevent
% a secondary processing of the main document
% It overwrites the forwarding commands
% |\childdocof| and |\childdocforward|
% with empty macros to prevent further inclusions of the main document:
%    \begin{macrocode}
\newcommand{\childdocdisable}
{
  \renewcommand{\childdocmain}[1]{\renewcommand{\childdocmain}[1]{\endinput}}
  \renewcommand{\childdocof}[1]{}
  \renewcommand{\childdocby}[2][]{}
  \renewcommand{\childdocforward}[2][]{}
  \renewcommand{\childdocdisable}{}
}
%    \end{macrocode}

% \macro{\childdocmain}
% The macro |\childdocmain| is to be called at the top of the main file
% with nothing or the main filename (without extension) as argument.
% First, it breaks loops.
% If the argument is not empty and does not match |\childdocname|
% (which is set by the first inclusion of |childdoc.def|),
% |\ifchilddoc| is set to true, |\includeonly| is applied to the child file
% and |\jobname| is set to the main file
% (for proper handling of |.aux| files):
%    \begin{macrocode}
\newcommand{\childdocmain}[1]
{
  \childdocdisable\childdocmain{}
  \if?#1?\else
    \begingroup
      \def\childdoctmp{#1}
      \ifx\childdoctmp\childdocname
        \def\childdoctmp{}
      \else
        \def\childdoctmp
        {
          \childdoctrue
          \includeonly{\childdocname}
          \def\childdocjob{#1}
          \def\jobname{#1}
        }
      \fi
      \expandafter
    \endgroup
    \childdoctmp
  \fi
}
%    \end{macrocode}

% \macro{\childdocof}
% The command |\childdocof| redirects
% compilation to the main file |#1|.
%    \begin{macrocode}
\newcommand{\childdocof}[1]
{
  \childdocdisable
  \childdoctrue
  \includeonly{\childdocname}
  \def\jobname{#1}
  \def\childdocjob{#1}
  \input{#1}
}
%    \end{macrocode}

% \macro{\childdocby}
% The command |\childdocby| ....
%    \begin{macrocode}
\newcommand{\childdocby}[2][]
{
  \childdocdisable
  \childdoctrue
  \childdocmanualtrue
  \if?#1?\else
    \def\jobname{#2}
  \fi
  \def\childdocjob{#2}
  \input{#2}
  \endinput
}
%    \end{macrocode}

% \macro{\childdocforward}
% The command |\childdocforward| redirects
% compilation to the main file or
% (if the optional argument is given) a child file.
% Parameters are set as if the main file
% or a child file starting with |\childdocof| was compiled.
% Then compilation is handed over to the main file:
%    \begin{macrocode}
\newcommand{\childdocforward}[2][]
{
  \begingroup
    \if?#1?
      \def\childdoctmp
      {
        \def\childdocname{#2}
        \def\childdocjob{#2}
        \def\jobname{#2}
        \input{#2}
        \endinput
      }
    \else
      \def\childdoctmp
      {
        \childdocdisable
        \def\childdocname{#2}
        \childdoctrue
        \includeonly{#2}
        \def\childdocjob{#1}
        \def\jobname{#1}
        \input{#1}
        \endinput
      }
    \fi
    \expandafter
  \endgroup
  \childdoctmp
}
%    \end{macrocode}

% \macro{\childdocforwardprefix}
% The command |\childdocforwardprefix| redirects
% compilation to the main or a child file by means of a pattern.
% The prefix |#1| in the current filename is replaced by |#2|
% and the suffix of the current filename is kept
% (it is assumed that the filename does not contain the substring `|~~~|'
% which is used as a delimiter).
% Compilation is handed over to the new file by |\childdocforward|:
%    \begin{macrocode}
\newcommand{\childdocforwardprefix}[3][]
{
  \begingroup
    \def\childdocextract #2##1~~~{\def\childdoctmp{\childdocforward[#1]{#3##1}}}
    \expandafter\childdocextract\childdocname~~~
    \expandafter
  \endgroup
  \childdoctmp
}
%    \end{macrocode}

% \macro{\childdoc}
% The deprecated macro |\childdoc| is a legacy version of |\childdocmain|:
%    \begin{macrocode}
\newcommand{\childdoc}{\childdocmain}
%    \end{macrocode}

% \macro{\childdocredirect}
% The deprecated macro |\childdocredirect| is a legacy version
% of |\childdocforward| and |\childdocforwardprefix|:
%    \begin{macrocode}
\newcommand{\childdocredirect}[2][]
{
  \begingroup
    \if?#1?
      \def\childdoctmp{\childdocforward{#2}}
    \else
      \def\childdoctmp{\childdocforwardprefix{#1}{#2}}
    \fi
    \expandafter
  \endgroup
  \childdoctmp
}
%    \end{macrocode}

%\iffalse
%</package>
%\fi
%
\endinput
\childdocforward[|\textit{main}|]{|\textit{dest}|}"|
\end{center}
%
Here \textit{target} is the name of the output file,
\textit{main} is the name of the main file
and \textit{dest} is the name of the main or child file to be processed
(all filenames without extensions).
The optional argument \textit{main} can be omitted
if \textit{main} matches \textit{dest}.
Optionally, compilation \textit{flags} can be defined via |\def| commands.
This command line makes the \TeX{} engine believe
it is compiling the file \textit{target}
whose content is specified as the latter parameter.
The provided code then forwards the processing to
\textit{main} or \textit{dest} as described in \secref{sec:forward}.

%%%%%%%%%%%%%%%%%%%%%%%%%%%%%%%%%%%%%%%%%%%%%%%%%%%%%%%%%%%%%%%%%%%%%%%%%%%%%%%%
\subsection{Include by Input}
\label{sec:input}

Including child documents by |\include| has some restrictions by design.
Most notably, the content of a child document always occupies
its own set of pages; pages cannot be shared between child documents.
Usually, this behaviour makes perfect sense
because each child document contain an essential part of the document.
However, in some situations it may be desirable to compose
a document from a collection of parts
without having mandatory page breaks between then.
For this case, the package
provides a mechanism to include parts
by |\input| which can also be processed individually.
However, by construction this mechanism
requires manual handling of the content to be output.

%%%%%%%%%%%%%%%%%%%%%%%%%%%%%%%%%%%%%%%%
\DescribeMacro{\ifchilddocmanual}
The main file should be prepared as usual, see \secref{sec:include}.
However, the document body must make a distinction
between processing of an individual part and of the main document, e.g.:
%
\begin{center}
\begin{tabular}{l}
|\ifchilddocmanual|\\
|\input{\childdocname}|\\
|\||else|\\
\textit{document body with }|\input{|\textit{part}|}|\\
|\||fi|
\end{tabular}
\end{center}
%
The conditional |\ifchilddocmanual| is true whenever
a part to be included by |\input| is being compiled,
and the name of the part is stored in |\childdocname|.

%%%%%%%%%%%%%%%%%%%%%%%%%%%%%%%%%%%%%%%%
\DescribeMacro{\childdocby}
Each part to be included by |\input| should start with:
%
\begin{center}
\begin{tabular}{l}
|% \iffalse
%
% childdoc.dtx Copyright (C) 2017-2018 Niklas Beisert
%
% This work may be distributed and/or modified under the
% conditions of the LaTeX Project Public License, either version 1.3
% of this license or (at your option) any later version.
% The latest version of this license is in
%   http://www.latex-project.org/lppl.txt
% and version 1.3 or later is part of all distributions of LaTeX
% version 2005/12/01 or later.
%
% This work has the LPPL maintenance status `maintained'.
%
% The Current Maintainer of this work is Niklas Beisert.
%
% This work consists of the files childdoc.dtx and childdoc.ins
% and the derived files childdoc.def and cdocsamp.tex with
% cdocsch1.tex, cdocsch2.tex, cdocsdrf.tex, cdocsfn1.tex, cdocsfn2.tex.
%
%<package>\ifdefined\childdocmain\endinput\fi
%<package>\ProvidesFile{childdoc.def}[2018/12/30 v2.0 child document driver]
%<samplemain>\ProvidesFile{cdocsamp.tex}[2018/12/30 v2.0 sample for childdoc]
%<*driver>
%\ProvidesFile{childdoc.drv}[2018/12/30 v2.0 childdoc reference manual file]
\PassOptionsToClass{10pt,a4paper}{article}
\documentclass{ltxdoc}

\usepackage[margin=35mm]{geometry}
\usepackage{hyperref}
\usepackage{hyperxmp}
\usepackage[usenames]{color}

\hypersetup{colorlinks=true}
\hypersetup{pdfstartview=FitH}
\hypersetup{pdfpagemode=UseNone}
\hypersetup{pdfsource={}}
\hypersetup{pdflang={en-UK}}
\hypersetup{pdfcopyright={Copyright 2017-2018 Niklas Beisert.
  This work may be distributed and/or modified under the
  conditions of the LaTeX Project Public License, either version 1.3
  of this license or (at your option) any later version.}}
\hypersetup{pdflicenseurl={http://www.latex-project.org/lppl.txt}}
\hypersetup{pdfcontactaddress={ETH Zurich, ITP, HIT K,
  Wolfgang-Pauli-Strasse 27}}
\hypersetup{pdfcontactpostcode={8093}}
\hypersetup{pdfcontactcity={Zurich}}
\hypersetup{pdfcontactcountry={Switzerland}}
\hypersetup{pdfcontactemail={nbeisert@itp.phys.ethz.ch}}
\hypersetup{pdfcontacturl={http://people.phys.ethz.ch/\xmptilde nbeisert/}}

\newcommand{\secref}[1]{\hyperref[#1]{section \ref*{#1}}}

\parskip1ex
\parindent0pt
\let\olditemize\itemize
\def\itemize{\olditemize\parskip0pt}

\begin{document}

\title{The \textsf{childdoc} Package}
\hypersetup{pdftitle={The childdoc Package}}
\author{Niklas Beisert\\[2ex]
  Institut f\"ur Theoretische Physik\\
  Eidgen\"ossische Technische Hochschule Z\"urich\\
  Wolfgang-Pauli-Strasse 27, 8093 Z\"urich, Switzerland\\[1ex]
  \href{mailto:nbeisert@itp.phys.ethz.ch}
  {\texttt{nbeisert@itp.phys.ethz.ch}}}
\hypersetup{pdfauthor={Niklas Beisert}}
\hypersetup{pdfsubject={Manual for the LaTeX2e Package childdoc}}
\date{30 December 2018, \textsf{v2.0}}
\maketitle

\begin{abstract}\noindent
\textsf{childdoc} is a \LaTeXe{} package
that enables the direct compilation
of document sections included by |\include|
to individual files.
\end{abstract}

\begingroup
\parskip0ex
\tableofcontents
\endgroup

%%%%%%%%%%%%%%%%%%%%%%%%%%%%%%%%%%%%%%%%%%%%%%%%%%%%%%%%%%%%%%%%%%%%%%%%%%%%%%%%
%%%%%%%%%%%%%%%%%%%%%%%%%%%%%%%%%%%%%%%%%%%%%%%%%%%%%%%%%%%%%%%%%%%%%%%%%%%%%%%%
\section{Introduction}

\LaTeX{} provides a mechanism to structure a large document (such as a book)
into a main file and several child files (containing the chapters)
using the |\include| command.
This mechanism is beneficial for documents
which span hundreds of pages in order to
make the source file(s) more manageable.
Moreover, compilation can be restricted to
selected child files by means of the |\includeonly| command.
The latter feature can be used to reduce the compilation time while editing
(this was significantly more useful in the earlier days of \LaTeX{})
or to generate a smaller document which is easier to navigate.
Another application of |\includeonly| is to generate
documents consisting of selected parts of the complete document.

However, there are a few drawbacks of the plain |\include| mechanism:
\begin{itemize}
\item
The child files cannot be compiled on their own,
they can only be compiled via the main file.
A naive editing environment
(such as a text editor with an option
to have the current file processed by \LaTeX)
may require one to switch to the main file before compiling;
attempting to compile the child file produces errors.
\item
The main file must be modified (each time)
to adjust the |\includeonly| command
to the present needs. This easily leaves the main file in a messy state.
\item
The generated document will always carry the filename
of the main document. This is inconvenient if
several child files are to be compiled and
to be kept for distribution.
\end{itemize}

The present package provides a simple interface
to make child files individually compilable by \LaTeX{}.
Compiling a child file then has the same effect as compiling
the main file with an |\includeonly| command
to select the appropriate child.
Moreover the generated document will carry the name of the child
rather than the main file.
This resolves all three above issues.

This feature is meant to make the editing of books,
thesis documents and lecture notes somewhat more convenient.
However, the package can also be used efficiently for
composing a series of documents (such as exercise sheets)
which are typically distributed individually.
It then assists the author in generating the individual documents
(potentially in different versions)
as well as a document containing the collected series.
Another application is in developing style files
or other kinds of included material
where compilation of the style file could redirect
to a sample or test file.

%%%%%%%%%%%%%%%%%%%%%%%%%%%%%%%%%%%%%%%%%%%%%%%%%%%%%%%%%%%%%%%%%%%%%%%%%%%%%%%%
%%%%%%%%%%%%%%%%%%%%%%%%%%%%%%%%%%%%%%%%%%%%%%%%%%%%%%%%%%%%%%%%%%%%%%%%%%%%%%%%
\section{Usage}

First of all, the package \textsf{childdoc} is \emph{not} a standard
\LaTeXe{} |.sty| style file! Therefore it needs to be invoked in
a non-standard way.

%%%%%%%%%%%%%%%%%%%%%%%%%%%%%%%%%%%%%%%%%%%%%%%%%%%%%%%%%%%%%%%%%%%%%%%%%%%%%%%%
\subsection{Included Files}
\label{sec:include}

%%%%%%%%%%%%%%%%%%%%%%%%%%%%%%%%%%%%%%%%
\DescribeMacro{\childdocmain}
To use the package, add the commands
\begin{center}
\begin{tabular}{l}
|\input{childdoc.def}|\\
|\childdocmain{}|\\
\end{tabular}
\end{center}
at the very top of the main \LaTeX{} file,
in particular \emph{before} the |\documentclass| statement!
The argument of |\childdocmain| should be left empty
(but it must be present).

%%%%%%%%%%%%%%%%%%%%%%%%%%%%%%%%%%%%%%%%
\DescribeMacro{\childdocof}
Furthermore, add the commands
\begin{center}
\begin{tabular}{l}
|\input{childdoc.def}|\\
|\childdocof{|\textit{main}|}|\\
\end{tabular}
\end{center}
at the top of every child file \textit{child}
which is included by |\include{|\textit{child}|}|
from within the main file
(or at least for those files to be compiled individually).
The argument \textit{main} must be the filename of the main file.

There are a couple of
considerations in setting up the main and child documents:

%%%%%%%%%%%%%%%%%%%%%%%%%%%%%%%%%%%%%%%%
\paragraph{Restrictions.}

Please note the following restrictions:
\begin{itemize}
\item
|\childdocmain| must be called with one argument \textit{main}
to ensure compatibility with earlier version of the package.
It must either be empty (|\childdocmain{}|)
or precisely match the filename of the main file in which it is specified.
See \secref{sec:detection} for further information.
\item
The filename \textit{main} must be specified without the |.tex| extension.
\item
The filename \textit{main} is case sensitive
(even in case-insensitive file systems)
due to internal string comparison.
\item
The argument \textit{main} should be fully expanded, it cannot be a macro.
\item
Subdirectories and special characters should be avoided in filenames.
\item
The command |\childdocmain{|\textit{main}|}| must be followed by a whitespace.
It should not be followed immediately by another command
or by a comment mark `|%|'.
This is because the \TeX{} parser reads the token immediately following
the argument of |\childdocmain| and puts it
at the beginning of every child section;
however, a white\-space is ignored.
\end{itemize}

%%%%%%%%%%%%%%%%%%%%%%%%%%%%%%%%%%%%%%%%
\paragraph{Content of Main File.}

It is advisable to place all content in the child files included by |\include|.
Any output contained in the main file will appear in all child documents
unless suppressed manually;
it cannot be suppressed automatically by the |\includeonly| directive
and thus should normally be avoided.
A method to include some content in the main file
by means of conditional processing is described in \secref{sec:conditional}.

%%%%%%%%%%%%%%%%%%%%%%%%%%%%%%%%%%%%%%%%
\paragraph{Page Numbering.}

When only a part of the document is compiled,
the appropriate numbering of pages
(as well as other status parameters)
is determined from the |.aux| files.
The latter contain information from previous passes.
However this information needs to propagate through
all intermediate child documents.
Therefore the page numbering in child documents may well
be inconsistent until the complete document is compiled at least once.

A useful (if unconventional) way to always ensure a consistent
page numbering is to restart the numbering in each child document
and denote the pages by `\textit{child}|.|\textit{page}'
where \textit{child} represents the chapter/section number of the child file.
This can be achieved by the command
|\numberwithin{page}{|\textit{child}|}|
of the \textsf{amsmath} package
where \textit{child} can be |chapter| or |section|
depending on the chosen structuring.
Alternatively, one can modify the macro |\thepage| appropriately
and reset the counter |page| at the start of each child file.

%%%%%%%%%%%%%%%%%%%%%%%%%%%%%%%%%%%%%%%%%%%%%%%%%%%%%%%%%%%%%%%%%%%%%%%%%%%%%%%%
\subsection{Conditional Processing}
\label{sec:conditional}

The package provides a mechanism to compile different versions
of a document. To customise the versions further some conditional processing
can come in handy to distinguish which version is being compiled.
The package provides two macros to describe the compilation context:

%%%%%%%%%%%%%%%%%%%%%%%%%%%%%%%%%%%%%%%%
\DescribeMacro{\ifchilddoc}
The conditional |\ifchilddoc| distinguishes between the compilation of
child documents and the main document:
%
\begin{center}
|\ifchilddoc |\textit{child-code}| |[|\||else |\textit{main-code}]| \||fi|
\end{center}

%%%%%%%%%%%%%%%%%%%%%%%%%%%%%%%%%%%%%%%%
\DescribeMacro{\childdocname}
\DescribeMacro{\childdocjob}
The macro |\childdocname| contains the filename (without extension)
of the main or child file being processed.
Note that |\childdocjob| will always contain the name of the main file.

%%%%%%%%%%%%%%%%%%%%%%%%%%%%%%%%%%%%%%%%
\paragraph{Title Page.}

Conditional processing can be used to include a title or banner page
in the main document when proper precautions are taken.
Importantly, the code in the main file should ensure that the page counter
(as well as other status parameters which are stored in the |.aux| files)
takes the same value after the conditional processing.
Otherwise the page numbers may take divergent values
depending on which part is compiled.

For example, a title page could be declared by:
%
\begin{center}
\begin{tabular}{l}
|\ifchilddoc\||else|\\
|\addtocounter{page}{-1}|\\
\textit{code for title page}\\
|\newpage|\\
|\||fi|
\end{tabular}
\end{center}
%
A banner page for the child documents can be generated by:
%
\begin{center}
\begin{tabular}{l}
|\ifchilddoc|\\
|\addtocounter{page}{-1}|\\
\textit{code for banner page}\\
|\newpage|\\
|\||fi|
\end{tabular}
\end{center}
%
Here one could write a message such as:
\begin{center}
|This is the part \childdocname{} of \childdocjob{}.|
\end{center}

%%%%%%%%%%%%%%%%%%%%%%%%%%%%%%%%%%%%%%%%%%%%%%%%%%%%%%%%%%%%%%%%%%%%%%%%%%%%%%%%
\subsection{Flags}
\label{sec:flags}

The package makes it easy to generate different versions
of the main or child documents.
To this end compilation flags can be defined
and assigned different default values.
They will be particularly useful in conjunction
with the forwarding mechanism described in \secref{sec:forward}.

For example, it may be useful to have a flag |\version|
which can be set to |draft| or |final|.
The document source will contain some conditional code
depending on the value of |\version|.
Suppose further, the flag should default to |final| for the main file
and to |draft| for child files
which is a natural assignment for editing the document.
This is achieved by placing the following code
in the preamble of the main document
(below the |\childdocmain| directive):
%
\begin{center}
\begin{tabular}{l}
|\ifchilddoc|\\
|\providecommand{\version}{draft}|\\
|\||else|\\
|\providecommand{\version}{final}|\\
|\||fi|
\end{tabular}
\end{center}
%
The definition by |\providecommand| makes sure
that previous definitions are not overwritten.
Further statements |\providecommand{\version}{...}|
can thus be added before the above code to override it.

For the main file, one might add a line
(between |\childdocmain| and the above block)
%
\begin{center}
|%\ifchilddoc\||else\providecommand{\version}{draft}\||fi|
\end{center}
%
which can be uncommented to produce a draft version.
Likewise one can add a line to the very top of a child file
(above the |\childdocof{|\textit{main}|}| directive)
%
\begin{center}
|%\providecommand{\version}{final}|
\end{center}
%
which can be uncommented to produce the final version of this child document.

%%%%%%%%%%%%%%%%%%%%%%%%%%%%%%%%%%%%%%%%%%%%%%%%%%%%%%%%%%%%%%%%%%%%%%%%%%%%%%%%
\subsection{Forwarding}
\label{sec:forward}

Different versions of the main or child documents
using compilation flags as described in \secref{sec:flags}
can be (permanently) stored in different files
for convenient compilation, viewing and distribution.
To this end, the package defines a command
to pass on compilation to a different file:

%%%%%%%%%%%%%%%%%%%%%%%%%%%%%%%%%%%%%%%%
\DescribeMacro{\childdocforward}
The command |\childdocforward| redirects processing to
another source file:
%
\begin{center}
\begin{tabular}{l}
|\input{childdoc.def}|\\
|\childdocforward[|\textit{main}|]{|\textit{dest}|}|\\
\end{tabular}
\end{center}
%
The argument \textit{dest} is the destination file
(without extension).
It should be the main file or one of the child files.
Note that further \textsf{childdoc} directives
such as |\childdocof| and |\childdocforward|
in the indicated file will be processed in this form.
The optional argument \textit{main}
passes on directly to the main file \textit{main}
while pretending to compile the child \textit{dest}.
This form behaves as if \textit{dest}
issues |\childdocof{|\textit{main}|}| right away,
and no further \textsf{childdoc} directives will be processed.

%%%%%%%%%%%%%%%%%%%%%%%%%%%%%%%%%%%%%%%%
\DescribeMacro{\...prefix}
In the alternative form |\childdocforwardprefix|,
%
\begin{center}
\begin{tabular}{l}
|\input{childdoc.def}|\\
|\childdocforwardprefix[|\textit{main}|]{|\textit{prefix}|}{|\textit{dest}|}|
\end{tabular}
\end{center}
%
the destination file is determined by a pattern
depending on the current file:
To make this work, the current file must be called
`{\textit{prefix}\hspace{0.2em}\textit{suffix}}'
with \textit{prefix} matching precisely the argument.
Processing is then passed on to the file
`{\textit{dest}\hspace{0.2em}\textit{suffix}}'.
Surely, the same effect is achieved by
directly specifying the
argument `{\textit{dest}\hspace{0.2em}\textit{suffix}}'
in the first form.
However, that requires to set up a different file
for each child. With the alternative form of the command
all these files can have exactly the same content
which simplifies setting them up and maintaining them.

For example, the following file |draft.tex|
with a compilation flag |\version| as described in \secref{sec:flags}
compiles the main document as a draft:
%
\begin{center}
\begin{tabular}{l}
|\def\version{draft}|\\
|\input{childdoc.def}|\\
|\childdocforward{|\textit{main}|}|
\end{tabular}
\end{center}
%
Likewise, the following files |final|\textit{nn}|.tex|
compile the final version of the child document
|child|\textit{nn}|.tex|:
%
\begin{center}
\begin{tabular}{l}
|\def\version{final}|\\
|\input{childdoc.def}|\\
|\childdocforwardprefix{final}{child}|
\end{tabular}
\end{center}
%

Note that when several versions of a main file and/or of each child file
are to be generated, it may be convenient to set up a |Makefile| or
shell script to automatise the process.

%%%%%%%%%%%%%%%%%%%%%%%%%%%%%%%%%%%%%%%%%%%%%%%%%%%%%%%%%%%%%%%%%%%%%%%%%%%%%%%%
\subsection{Command Line Processing}
\label{sec:commandline}

The effect of redirection files can also be achieved by invoking
the \LaTeX{} compiler with a more elaborate command line.
Most conveniently this should be done as part
of a shell script or a |Makefile|.

When using \textsf{childdoc} in the main file, the following
command lines effectively perform a redirection
(note that depending on the shell being used,
backslashes may have to be doubled: `|\|' $\to$ `|\\|'):
%
\begin{center}
|... -jobname "|\textit{target}|" |\\|"|[\textit{flags}]%
|\input{childdoc.def}\childdocforward[|\textit{main}|]{|\textit{dest}|}"|
\end{center}
%
Here \textit{target} is the name of the output file,
\textit{main} is the name of the main file
and \textit{dest} is the name of the main or child file to be processed
(all filenames without extensions).
The optional argument \textit{main} can be omitted
if \textit{main} matches \textit{dest}.
Optionally, compilation \textit{flags} can be defined via |\def| commands.
This command line makes the \TeX{} engine believe
it is compiling the file \textit{target}
whose content is specified as the latter parameter.
The provided code then forwards the processing to
\textit{main} or \textit{dest} as described in \secref{sec:forward}.

%%%%%%%%%%%%%%%%%%%%%%%%%%%%%%%%%%%%%%%%%%%%%%%%%%%%%%%%%%%%%%%%%%%%%%%%%%%%%%%%
\subsection{Include by Input}
\label{sec:input}

Including child documents by |\include| has some restrictions by design.
Most notably, the content of a child document always occupies
its own set of pages; pages cannot be shared between child documents.
Usually, this behaviour makes perfect sense
because each child document contain an essential part of the document.
However, in some situations it may be desirable to compose
a document from a collection of parts
without having mandatory page breaks between then.
For this case, the package
provides a mechanism to include parts
by |\input| which can also be processed individually.
However, by construction this mechanism
requires manual handling of the content to be output.

%%%%%%%%%%%%%%%%%%%%%%%%%%%%%%%%%%%%%%%%
\DescribeMacro{\ifchilddocmanual}
The main file should be prepared as usual, see \secref{sec:include}.
However, the document body must make a distinction
between processing of an individual part and of the main document, e.g.:
%
\begin{center}
\begin{tabular}{l}
|\ifchilddocmanual|\\
|\input{\childdocname}|\\
|\||else|\\
\textit{document body with }|\input{|\textit{part}|}|\\
|\||fi|
\end{tabular}
\end{center}
%
The conditional |\ifchilddocmanual| is true whenever
a part to be included by |\input| is being compiled,
and the name of the part is stored in |\childdocname|.

%%%%%%%%%%%%%%%%%%%%%%%%%%%%%%%%%%%%%%%%
\DescribeMacro{\childdocby}
Each part to be included by |\input| should start with:
%
\begin{center}
\begin{tabular}{l}
|\input{childdoc.def}|\\
|\childdocby{|\textit{main}|}|\\
\end{tabular}
\end{center}
%
The directive |\childdocby| is similar to |\childdocof|
described in \secref{sec:include},
but the subsequent selection of content must be done manually.
To that end, both |\ifchilddoc| and |\ifchilddocmanual|
will be true upon processing of a part,
and the name of the part is stored in |\childdocname|.
Note that |\jobname| will be set to the filename of the current part
so that each part receives an individual |.aux| file
that does not interfere with the |.aux| file(s) of the main document.
This behaviour can be altered by the alternative form
|\childdocby[*]{|\textit{main}|}| (with a non-empty optional argument)
which uses the |.aux| file of the main document
by setting |\jobname| to \textit{main}.

%%%%%%%%%%%%%%%%%%%%%%%%%%%%%%%%%%%%%%%%%%%%%%%%%%%%%%%%%%%%%%%%%%%%%%%%%%%%%%%%
\subsection{Driver Development}
\label{sec:driver}

The \textsf{childdoc} mechanism can also be use for the development
of definition files such as \LaTeX{} styles or classes.
This case differs from the above setup with multiple parts
included by |\include| in that no |\includeonly| should be invoked.
This can be achieved by starting the include file
(before |\ProvidesPackage|) with:
%
\begin{center}
\begin{tabular}{l}
|\input{childdoc.def}|\\
|\childdocforward{|\textit{main}|}|\\
\end{tabular}
\end{center}
%
or alternatively with:
%
\begin{center}
\begin{tabular}{l}
|\input{childdoc.def}|\\
|\childdocby{|\textit{main}|}|\\
\end{tabular}
\end{center}
%
Both forms have slightly different effects as described above.
The main file is prepared as usual, see \secref{sec:include}.

%%%%%%%%%%%%%%%%%%%%%%%%%%%%%%%%%%%%%%%%%%%%%%%%%%%%%%%%%%%%%%%%%%%%%%%%%%%%%%%%
\subsection{Legacy Detection}
\label{sec:detection}

The directive |\childdocmain| in the main file can detect
whether the complete document or merely a child is to be compiled
even without using the directive |\childdocof|.
This method is deprecated because it is less robust
and there is no compelling reason to use it;
it is merely provided for backward compatibility
and it may be removed in future versions.

If the detection mechanism is to be used,
it is mandatory to correctly specify
the filename of the main file as the argument of |\childdocmain|:
%
\begin{center}
\begin{tabular}{l}
|\input{childdoc.def}|\\
|\childdocmain{|\textit{main}|}|\\
\end{tabular}
\end{center}
%
If |\jobname| does not match the argument \textit{main} of |\childdocmain|,
it is assumed that |\jobname| points to the child file to be compiled.
When using |\childdocmain| with the main file specified as argument,
it suffices to start a child file
with just |\input{|\textit{main}|}|
without loading of the package and using |\childdocof|.
If instead all processing is done
with the appropriate \textsf{childdoc} directives,
the argument of \textit{main} of |\childdocmain| can be empty.

An alternative version of the command line processing described
in \secref{sec:commandline} using the detection mechanism reads:
%
\begin{center}
|... -jobname "|\textit{target}|" "|[\textit{flags}]%
[|\def\jobname{|\textit{dest}|}|]|\input{|\textit{main}|}"|
\end{center}

%%%%%%%%%%%%%%%%%%%%%%%%%%%%%%%%%%%%%%%%%%%%%%%%%%%%%%%%%%%%%%%%%%%%%%%%%%%%%%%%
\subsection{Manual Code}
\label{sec:manual}

In case one cannot be certain whether the definitions file |childdoc.def|
is installed on the target \TeX{} distribution
and one prefers not to ship it,
it is conceivable to paste a few relevant commands into the sources.

To that end, drop all statements |\input{childdoc.def}|
and perform the replacements as outlined below.
Instead of |\childdocmain{|\textit{main}|}| add the following code
to the top of the main file:
%
\begin{center}
\begin{tabular}{l}
|\||ifdefined\childdocname\endinput\||fi\newif\ifchilddoc|\\
|\edef\childdocname{\scantokens\expandafter{\jobname\noexpand}}|\\
|\def\childdocmain{|\textit{main}|}\||ifx\childdocmain\childdocname\||else|\\
|\childdoctrue\includeonly{\childdocname}\let\jobname\childdocmain\||fi|\\
\end{tabular}
\end{center}
%
Instead of |\childdocof{|\textit{main}|}| just include the main file
at the top of each child file:
%
\begin{center}
|\input{|\textit{main}|}|
\end{center}
%
A simple redirection |\childdocforward{|\textit{dest}|}| is achieved by:
%
\begin{center}
|\def\jobname{|\textit{dest}|}\input{\jobname}|
\end{center}
%
The redirection with prefix
|\childdocforwardprefix[|\textit{prefix}|]{|\textit{dest}|}|
is accomplished by:
%
\begin{center}
\begin{tabular}{l}
|{\edef\jobname{\scantokens\expandafter{\jobname\noexpand}}|\\
|\def\redirectjob |\textit{prefix}|#1~~~{\gdef\jobname{|\textit{dest}|#1}}|\\
|\expandafter\redirectjob\jobname~~~}\input{\jobname}|
\end{tabular}
\end{center}

In an alternative approach,
child documents can be compiled by a specific command line
without additional code or specific definitions:
%
\begin{center}
|... -jobname "|\textit{target}|" "|[\textit{flags}]%
|\includeonly{|\textit{dest}|}\input{|\textit{main}|}"|
\end{center}
%

%%%%%%%%%%%%%%%%%%%%%%%%%%%%%%%%%%%%%%%%%%%%%%%%%%%%%%%%%%%%%%%%%%%%%%%%%%%%%%%%
%%%%%%%%%%%%%%%%%%%%%%%%%%%%%%%%%%%%%%%%%%%%%%%%%%%%%%%%%%%%%%%%%%%%%%%%%%%%%%%%
\section{Information}

%%%%%%%%%%%%%%%%%%%%%%%%%%%%%%%%%%%%%%%%%%%%%%%%%%%%%%%%%%%%%%%%%%%%%%%%%%%%%%%%
\subsection{Copyright}

Copyright \copyright{} 2017--2018 Niklas Beisert

This work may be distributed and/or modified under the
conditions of the \LaTeX{} Project Public License, either version 1.3
of this license or (at your option) any later version.
The latest version of this license is in
  \url{http://www.latex-project.org/lppl.txt}
and version 1.3 or later is part of all distributions of \LaTeX{}
version 2005/12/01 or later.

This work has the LPPL maintenance status `maintained'.

The Current Maintainer of this work is Niklas Beisert.

This work consists of the files |README.txt|, |childdoc.ins| and |childdoc.dtx|
as well as the derived files |childdoc.def|, |cdocsamp.tex|
with |cdocsch1.tex|, |cdocsch2.tex|, |cdocspt3.tex|, |cdocspt4.tex|,
|cdocsdrf.tex|, |cdocsfn1.tex|, |cdocsfn2.tex|
as well as |childdoc.pdf|.

%%%%%%%%%%%%%%%%%%%%%%%%%%%%%%%%%%%%%%%%%%%%%%%%%%%%%%%%%%%%%%%%%%%%%%%%%%%%%%%%
\subsection{Files and Installation}

The package consists of the files:
%
\begin{center}
\begin{tabular}{ll}
    |README.txt|   & readme file \\
    |childdoc.ins| & installation file \\
    |childdoc.dtx| & source file \\
    |childdoc.def| & definition file \\
    |cdocsamp.tex| & sample main file \\
    |cdocsch1.tex| & sample include file \\
    |cdocsch2.tex| & sample include file \\
    |cdocspt3.tex| & sample part file \\
    |cdocspt4.tex| & sample part file \\
    |cdocsdrf.tex| & sample redirection file \\
    |cdocsfn1.tex| & sample redirection file \\
    |cdocsfn2.tex| & sample redirection file \\
    |childdoc.pdf| & manual
\end{tabular}
\end{center}
%
The distribution consists of the files
|README.txt|, |childdoc.ins| and |childdoc.dtx|.
%
\begin{itemize}
\item
Run (pdf)\LaTeX{} on |childdoc.dtx|
to compile the manual |childdoc.pdf| (this file).
\item
Run \LaTeX{} on |childdoc.ins| to create the definitions file |childdoc.def|
and the sample |cdocsamp.tex| with include files
|cdocsch1.tex|, |cdocsch2.tex|, |cdocspt3.tex|, |cdocspt4.tex|,
|cdocsdrf.tex|, |cdocsfn1.tex|, |cdocsfn2.tex|.
Then copy the file |childdoc.def| to an appropriate directory of your \LaTeX{}
distribution, e.g.\ \textit{texmf-root}|/tex/latex/childdoc|.
\end{itemize}

%%%%%%%%%%%%%%%%%%%%%%%%%%%%%%%%%%%%%%%%%%%%%%%%%%%%%%%%%%%%%%%%%%%%%%%%%%%%%%%%
\subsection{Related CTAN Packages}

There are several other packages which offer a similar functionality:
%
\begin{itemize}
\item
The packages
\href{http://ctan.org/pkg/docmute}{\textsf{docmute}},
\href{http://ctan.org/pkg/includex}{\textsf{includex}} and
\href{http://ctan.org/pkg/standalone}{\textsf{standalone}}
provide commands to include only the document body of
a child file thus allowing both files to be compiled individually.
\item
The packages \href{http://ctan.org/pkg/subdocs}{\textsf{subdocs}}
and \href{http://ctan.org/pkg/subfiles}{\textsf{subfiles}}
provide structures in which the main and child documents can be
encapsulated and allowing them to be compiled individually.
The inclusion mechanism is different from the conventional |\include|.
\item
The package \href{http://ctan.org/pkg/combine}{\textsf{combine}}
is an elaborate solution to combine several documents into one.
\end{itemize}
%
See also the CTAN topic \href{http://ctan.org/topic/subdocs}{\textsf{subdocs}}
for further related packages.
The present package differs from the above solutions in that
a document structure constructed with the conventional |\include| mechanism
just needs two extra commands at the top of every file
such that all constituent files can be compiled individually.

%%%%%%%%%%%%%%%%%%%%%%%%%%%%%%%%%%%%%%%%%%%%%%%%%%%%%%%%%%%%%%%%%%%%%%%%%%%%%%%%
%\subsection{Feature Suggestions}
%
%The following is a list of features which may be useful for future
%versions of this package:
%%
%\begin{itemize}
%\item
%\ldots
%\end{itemize}

%%%%%%%%%%%%%%%%%%%%%%%%%%%%%%%%%%%%%%%%%%%%%%%%%%%%%%%%%%%%%%%%%%%%%%%%%%%%%%%%
\subsection{Revision History}

%%%%%%%%%%%%%%%%%%%%%%%%%%%%%%%%%%%%%%%%
\paragraph{v2.0:} 2018/12/30

\begin{itemize}
\item
immediate forward processing
\item
added |\childdocby| mechanism
\item
manual restructured
\end{itemize}

%%%%%%%%%%%%%%%%%%%%%%%%%%%%%%%%%%%%%%%%
\paragraph{v1.6:} 2018/01/17

\begin{itemize}
\item
application for development of include files
\item
corrections to manual
\end{itemize}

%%%%%%%%%%%%%%%%%%%%%%%%%%%%%%%%%%%%%%%%
\paragraph{v1.5:} 2017/05/21

\begin{itemize}
\item
more complete structuring introduced
\item
|\childdocof| introduced
\item
|\childdoc| renamed to |\childdocmain|
\item
|\childredirect| renamed to |\childdocforward| and |\childdocforwardprefix|
and functionality expanded
\end{itemize}

%%%%%%%%%%%%%%%%%%%%%%%%%%%%%%%%%%%%%%%%
\paragraph{v1.0:} 2017/04/27

\begin{itemize}
\item
manual and install package
\item
first version published on CTAN
\end{itemize}

%%%%%%%%%%%%%%%%%%%%%%%%%%%%%%%%%%%%%%%%
\paragraph{v0.6:} 2017/04/26

\begin{itemize}
\item
redirection mechanism added
\end{itemize}

%%%%%%%%%%%%%%%%%%%%%%%%%%%%%%%%%%%%%%%%
\paragraph{v0.5:} 2017/04/26

\begin{itemize}
\item
functionality in definition file
\end{itemize}


%%%%%%%%%%%%%%%%%%%%%%%%%%%%%%%%%%%%%%%%%%%%%%%%%%%%%%%%%%%%%%%%%%%%%%%%%%%%%%%%
%%%%%%%%%%%%%%%%%%%%%%%%%%%%%%%%%%%%%%%%%%%%%%%%%%%%%%%%%%%%%%%%%%%%%%%%%%%%%%%%
%%%%%%%%%%%%%%%%%%%%%%%%%%%%%%%%%%%%%%%%%%%%%%%%%%%%%%%%%%%%%%%%%%%%%%%%%%%%%%%%
\appendix

\settowidth\MacroIndent{\rmfamily\scriptsize 000\ }

 \DocInput{childdoc.dtx}

\end{document}
%</driver>
% \fi
%
% %%%%%%%%%%%%%%%%%%%%%%%%%%%%%%%%%%%%%%%%%%%%%%%%%%%%%%%%%%%%%%%%%%%%%%%%%%%%%%
% %%%%%%%%%%%%%%%%%%%%%%%%%%%%%%%%%%%%%%%%%%%%%%%%%%%%%%%%%%%%%%%%%%%%%%%%%%%%%%
% \section{Sample}
%\iffalse
%<*samplemain>
%\fi
%
% The following presents a sample document
% with two chapters, two parts, a title page,
% a compile flag as well as three forwarding files to set the flag.
% It consists of eight |.tex| files:
% \begin{center}
% \begin{tabular}{ll}
% |cdocsamp.tex|&main file\\
% |cdocsch1.tex|&include file for chapter 1\\
% |cdocsch2.tex|&include file for chapter 2\\
% |cdocspt3.tex|&include file for part 3\\
% |cdocspt4.tex|&include file for part 4\\
% |cdocsdrf.tex|&forwarding file for main file in draft mode\\
% |cdocsfi1.tex|&forwarding file for final version of chapter 1\\
% |cdocsfi2.tex|&forwarding file for final version of chapter 2\\
% \end{tabular}
% \end{center}
% Each of the eight files can be compiled directly by the \LaTeX{} compiler.
%
% %%%%%%%%%%%%%%%%%%%%%%%%%%%%%%%%%%%%%%
% \paragraph{Main File.}
%
% The main file is called |cdocsamp.tex|.
%
% Load the \textsf{childdoc} definitions and
% declare the filename for the main document:
%    \begin{macrocode}
\input{childdoc.def}
\childdocmain{}
%    \end{macrocode}

% Optional override for |\version| flag:
%    \begin{macrocode}
%%\ifchilddoc\else\providecommand{\version}{draft}\fi
%    \end{macrocode}

% Define the default values for the |\version| flag
% (|final| for the main file and |draft| for childs):
%    \begin{macrocode}
\ifchilddoc
\providecommand{\version}{draft}
\else
\providecommand{\version}{final}
\fi
%    \end{macrocode}

% Load the standard document class:
%    \begin{macrocode}
\documentclass[12pt]{article}
%    \end{macrocode}

% Start the document body:
%    \begin{macrocode}
\begin{document}
%    \end{macrocode}

% Declare a title page.
% Print title, part of document being processed and version flag:
%    \begin{macrocode}
\addtocounter{page}{-1}
\begin{center}
{\LARGE\bfseries{}childdoc example\par}
\vspace{1cm}
\ifchilddoc
\ifchilddocmanual part\else chapter\fi:
`\childdocname' of `\childdocjob'\par
\else
main document: `\childdocjob'\par
\fi
version: \version\par
\end{center}
\newpage
%    \end{macrocode}

% Manually include selected file,
% otherwise process as usual:
%    \begin{macrocode}
\ifchilddocmanual
\section*{part `\childdocname'}
\input{\childdocname}
\else
%    \end{macrocode}

% Include the two chapters:
%    \begin{macrocode}
\include{cdocsch1}
\include{cdocsch2}
%    \end{macrocode}

% Include the two parts unless only chapters should be displayed:
%    \begin{macrocode}
\ifchilddoc\else
\section{part three}
\input{cdocspt3}
\section{part four}
\input{cdocspt4}
\fi
%    \end{macrocode}

% Process as usual until here:
%    \begin{macrocode}
\fi
%    \end{macrocode}

% End of document body:
%    \begin{macrocode}
\end{document}
%    \end{macrocode}
%\iffalse
%</samplemain>
%\fi
%
% %%%%%%%%%%%%%%%%%%%%%%%%%%%%%%%%%%%%%%
% \paragraph{Chapter Include Files.}
%
% The include files are called |cdocsch1.tex| and |cdocsch2.tex|.
%
%\iffalse
%<*samplechap1|samplechap2>
%\fi

% Optional override for |\version| flag:
%    \begin{macrocode}
%%\providecommand{\version}{final}
%    \end{macrocode}

% Include the main document:
%    \begin{macrocode}
\input{childdoc.def}
\childdocof{cdocsamp}
%    \end{macrocode}

%\iffalse
%</samplechap1|samplechap2>
%\fi
%
%\iffalse
%<*samplechap1>
%\fi
% Some text for chapter 1:
%    \begin{macrocode}
\section{one}
some text in chapter one
%    \end{macrocode}

%\iffalse
%</samplechap1>
%\fi
% Some text for chapter 2:
%\iffalse
%<*samplechap2>
%\fi
%    \begin{macrocode}
\section{two}
more text in chapter two
%    \end{macrocode}

%\iffalse
%</samplechap2>
%\fi
%
% %%%%%%%%%%%%%%%%%%%%%%%%%%%%%%%%%%%%%%
% \paragraph{Part Include Files.}
%
% The include files are called |cdocspt3.tex| and |cdocspt4.tex|.
%
%\iffalse
%<*samplepart3|samplepart4>
%\fi

% Optional override for |\version| flag:
%    \begin{macrocode}
%%\providecommand{\version}{final}
%    \end{macrocode}

% Include the main document:
%    \begin{macrocode}
\input{childdoc.def}
\childdocby{cdocsamp}
%    \end{macrocode}

%\iffalse
%</samplepart3|samplepart4>
%\fi
%
%\iffalse
%<*samplepart3>
%\fi
% Some text for part 3:
%    \begin{macrocode}
some text in part three
%    \end{macrocode}

%\iffalse
%</samplepart3>
%\fi
% Some text for part 4:
%\iffalse
%<*samplepart4>
%\fi
%    \begin{macrocode}
more text in part four
%    \end{macrocode}

%\iffalse
%</samplepart4>
%\fi
%
% %%%%%%%%%%%%%%%%%%%%%%%%%%%%%%%%%%%%%%
% \paragraph{Forwarding for a Complete Draft.}
%
% The following forwarding file |cdocsdrf.tex|
% compiles the main document in draft mode:
%\iffalse
%<*sampledraft>
%\fi
%    \begin{macrocode}
\def\version{draft}
\input{childdoc.def}
\childdocforward{cdocsamp}
%    \end{macrocode}

%\iffalse
%</sampledraft>
%\fi
%
% %%%%%%%%%%%%%%%%%%%%%%%%%%%%%%%%%%%%%%
% \paragraph{Forwarding for Final Version of the Chapters.}
%
% The following forwarding files |cdocsfn1.tex| and |cdocsfn2.tex|
% (with identical content)
% compile the final versions of the child documents
% |cdocsch1.tex| and |cdocsch2.tex|, respectively:
%\iffalse
%<*samplefinal>
%\fi
%    \begin{macrocode}
\def\version{final}
\input{childdoc.def}
\childdocforwardprefix[cdocsamp]{cdocsfn}{cdocsch}
%    \end{macrocode}

%\iffalse
%</samplefinal>
%\fi
%
% %%%%%%%%%%%%%%%%%%%%%%%%%%%%%%%%%%%%%%
% \paragraph{Command Line Processing.}
%
% The following three command lines generate the output files
% |cdocscld|, |cdocscl1| and |cdocscl2|
% which should be identical to
% |cdocsdrf|, |cdocsch1| and |cdocsfn2|, respectively:
% \begin{center}
% \begin{tabular}{l}
% |latex -jobname cdocscld \|\\
% |  "\def\version{draft}\input{childdoc.def}\childdocforward{cdocsamp}"|\\
% |latex -jobname cdocscl1 \|\\
% |  "\input{childdoc.def}\childdocforward[cdocsamp]{cdocsch1}"|\\
% |latex -jobname cdocscl2 \|\\
% |  "\def\version{final}\input{childdoc.def}\childdocforward{cdocsch2}"|
% \end{tabular}
% \end{center}
% Note that the trailing backslash on each first line
% merely continues the input to the second line
% (for convenient cut ant paste).
% Furthermore, the command |latex| can be replaced by any
% of its alternative versions such as |pdflatex|.
%
% %%%%%%%%%%%%%%%%%%%%%%%%%%%%%%%%%%%%%%%%%%%%%%%%%%%%%%%%%%%%%%%%%%%%%%%%%%%%%%
% %%%%%%%%%%%%%%%%%%%%%%%%%%%%%%%%%%%%%%%%%%%%%%%%%%%%%%%%%%%%%%%%%%%%%%%%%%%%%%
% \section{Implementation}
%\iffalse
%<*package>
%\fi
%
% This section describes the definitions file |childdoc.def|.

% The definitions cannot be loaded using |\usepackage| or |\RequirePackage|
% which has a mechanism to prevent loading a style file more than once.
% When loading the definitions by means of |\input|
% multiple instances have to be prevented manually:
%\iffalse
%This code needs to be before the `\ProvidesFile' directive
%which is defined at the beginning of this file.
%Therefore it is also placed there and commented out here.
%</package>
%<*discard>
%\fi
%    \begin{macrocode}
\ifdefined\childdocmain\endinput\fi
%    \end{macrocode}
%\iffalse
%</discard>
%<*package>
%\fi
%
% \macro{\ifchilddoc}
% \macro{\ifchilddocmanual}
% The conditional |\ifchilddoc| tells whether a
% child (true) or main (false) document is being compiled.
% The conditional |\ifchilddocmanual| tells whether
% the |\includeonly| mechanism is used (false) or
% the selection of child files must be performed manually (true).
% The definitions initialise to false:
%    \begin{macrocode}
\newif\ifchilddoc
\newif\ifchilddocmanual
%    \end{macrocode}

% \macro{\childdocname}
% \macro{\childdocjob}
% The macro |\childdocname| stores the name of the main document
% to be compiled. The macro |\childdocjob| stores the name of
% the document on which the \LaTeX{} compiler was originally invoked.
% The content of |\jobname| cannot be compared
% to filenames specified in the source due to different catcodes.
% The following code rescans |\jobname|, stores the result
% in |\childdocname| and saves a copy in |\childdocjob|:
%    \begin{macrocode}
\edef\childdocname{\scantokens\expandafter{\jobname\noexpand}}
\let\childdocjob\childdocname
%    \end{macrocode}

% \macro{\childdocdisable}
% The macro |\childdocdisable| prevents the main file
% from being processed more than once.
% At this stage, the main document command |\childdocmain|
% is assumed to be called once again where it should do nothing.
% Any subsequent call to it should prevent
% a secondary processing of the main document
% It overwrites the forwarding commands
% |\childdocof| and |\childdocforward|
% with empty macros to prevent further inclusions of the main document:
%    \begin{macrocode}
\newcommand{\childdocdisable}
{
  \renewcommand{\childdocmain}[1]{\renewcommand{\childdocmain}[1]{\endinput}}
  \renewcommand{\childdocof}[1]{}
  \renewcommand{\childdocby}[2][]{}
  \renewcommand{\childdocforward}[2][]{}
  \renewcommand{\childdocdisable}{}
}
%    \end{macrocode}

% \macro{\childdocmain}
% The macro |\childdocmain| is to be called at the top of the main file
% with nothing or the main filename (without extension) as argument.
% First, it breaks loops.
% If the argument is not empty and does not match |\childdocname|
% (which is set by the first inclusion of |childdoc.def|),
% |\ifchilddoc| is set to true, |\includeonly| is applied to the child file
% and |\jobname| is set to the main file
% (for proper handling of |.aux| files):
%    \begin{macrocode}
\newcommand{\childdocmain}[1]
{
  \childdocdisable\childdocmain{}
  \if?#1?\else
    \begingroup
      \def\childdoctmp{#1}
      \ifx\childdoctmp\childdocname
        \def\childdoctmp{}
      \else
        \def\childdoctmp
        {
          \childdoctrue
          \includeonly{\childdocname}
          \def\childdocjob{#1}
          \def\jobname{#1}
        }
      \fi
      \expandafter
    \endgroup
    \childdoctmp
  \fi
}
%    \end{macrocode}

% \macro{\childdocof}
% The command |\childdocof| redirects
% compilation to the main file |#1|.
%    \begin{macrocode}
\newcommand{\childdocof}[1]
{
  \childdocdisable
  \childdoctrue
  \includeonly{\childdocname}
  \def\jobname{#1}
  \def\childdocjob{#1}
  \input{#1}
}
%    \end{macrocode}

% \macro{\childdocby}
% The command |\childdocby| ....
%    \begin{macrocode}
\newcommand{\childdocby}[2][]
{
  \childdocdisable
  \childdoctrue
  \childdocmanualtrue
  \if?#1?\else
    \def\jobname{#2}
  \fi
  \def\childdocjob{#2}
  \input{#2}
  \endinput
}
%    \end{macrocode}

% \macro{\childdocforward}
% The command |\childdocforward| redirects
% compilation to the main file or
% (if the optional argument is given) a child file.
% Parameters are set as if the main file
% or a child file starting with |\childdocof| was compiled.
% Then compilation is handed over to the main file:
%    \begin{macrocode}
\newcommand{\childdocforward}[2][]
{
  \begingroup
    \if?#1?
      \def\childdoctmp
      {
        \def\childdocname{#2}
        \def\childdocjob{#2}
        \def\jobname{#2}
        \input{#2}
        \endinput
      }
    \else
      \def\childdoctmp
      {
        \childdocdisable
        \def\childdocname{#2}
        \childdoctrue
        \includeonly{#2}
        \def\childdocjob{#1}
        \def\jobname{#1}
        \input{#1}
        \endinput
      }
    \fi
    \expandafter
  \endgroup
  \childdoctmp
}
%    \end{macrocode}

% \macro{\childdocforwardprefix}
% The command |\childdocforwardprefix| redirects
% compilation to the main or a child file by means of a pattern.
% The prefix |#1| in the current filename is replaced by |#2|
% and the suffix of the current filename is kept
% (it is assumed that the filename does not contain the substring `|~~~|'
% which is used as a delimiter).
% Compilation is handed over to the new file by |\childdocforward|:
%    \begin{macrocode}
\newcommand{\childdocforwardprefix}[3][]
{
  \begingroup
    \def\childdocextract #2##1~~~{\def\childdoctmp{\childdocforward[#1]{#3##1}}}
    \expandafter\childdocextract\childdocname~~~
    \expandafter
  \endgroup
  \childdoctmp
}
%    \end{macrocode}

% \macro{\childdoc}
% The deprecated macro |\childdoc| is a legacy version of |\childdocmain|:
%    \begin{macrocode}
\newcommand{\childdoc}{\childdocmain}
%    \end{macrocode}

% \macro{\childdocredirect}
% The deprecated macro |\childdocredirect| is a legacy version
% of |\childdocforward| and |\childdocforwardprefix|:
%    \begin{macrocode}
\newcommand{\childdocredirect}[2][]
{
  \begingroup
    \if?#1?
      \def\childdoctmp{\childdocforward{#2}}
    \else
      \def\childdoctmp{\childdocforwardprefix{#1}{#2}}
    \fi
    \expandafter
  \endgroup
  \childdoctmp
}
%    \end{macrocode}

%\iffalse
%</package>
%\fi
%
\endinput
|\\
|\childdocby{|\textit{main}|}|\\
\end{tabular}
\end{center}
%
The directive |\childdocby| is similar to |\childdocof|
described in \secref{sec:include},
but the subsequent selection of content must be done manually.
To that end, both |\ifchilddoc| and |\ifchilddocmanual|
will be true upon processing of a part,
and the name of the part is stored in |\childdocname|.
Note that |\jobname| will be set to the filename of the current part
so that each part receives an individual |.aux| file
that does not interfere with the |.aux| file(s) of the main document.
This behaviour can be altered by the alternative form
|\childdocby[*]{|\textit{main}|}| (with a non-empty optional argument)
which uses the |.aux| file of the main document
by setting |\jobname| to \textit{main}.

%%%%%%%%%%%%%%%%%%%%%%%%%%%%%%%%%%%%%%%%%%%%%%%%%%%%%%%%%%%%%%%%%%%%%%%%%%%%%%%%
\subsection{Driver Development}
\label{sec:driver}

The \textsf{childdoc} mechanism can also be use for the development
of definition files such as \LaTeX{} styles or classes.
This case differs from the above setup with multiple parts
included by |\include| in that no |\includeonly| should be invoked.
This can be achieved by starting the include file
(before |\ProvidesPackage|) with:
%
\begin{center}
\begin{tabular}{l}
|% \iffalse
%
% childdoc.dtx Copyright (C) 2017-2018 Niklas Beisert
%
% This work may be distributed and/or modified under the
% conditions of the LaTeX Project Public License, either version 1.3
% of this license or (at your option) any later version.
% The latest version of this license is in
%   http://www.latex-project.org/lppl.txt
% and version 1.3 or later is part of all distributions of LaTeX
% version 2005/12/01 or later.
%
% This work has the LPPL maintenance status `maintained'.
%
% The Current Maintainer of this work is Niklas Beisert.
%
% This work consists of the files childdoc.dtx and childdoc.ins
% and the derived files childdoc.def and cdocsamp.tex with
% cdocsch1.tex, cdocsch2.tex, cdocsdrf.tex, cdocsfn1.tex, cdocsfn2.tex.
%
%<package>\ifdefined\childdocmain\endinput\fi
%<package>\ProvidesFile{childdoc.def}[2018/12/30 v2.0 child document driver]
%<samplemain>\ProvidesFile{cdocsamp.tex}[2018/12/30 v2.0 sample for childdoc]
%<*driver>
%\ProvidesFile{childdoc.drv}[2018/12/30 v2.0 childdoc reference manual file]
\PassOptionsToClass{10pt,a4paper}{article}
\documentclass{ltxdoc}

\usepackage[margin=35mm]{geometry}
\usepackage{hyperref}
\usepackage{hyperxmp}
\usepackage[usenames]{color}

\hypersetup{colorlinks=true}
\hypersetup{pdfstartview=FitH}
\hypersetup{pdfpagemode=UseNone}
\hypersetup{pdfsource={}}
\hypersetup{pdflang={en-UK}}
\hypersetup{pdfcopyright={Copyright 2017-2018 Niklas Beisert.
  This work may be distributed and/or modified under the
  conditions of the LaTeX Project Public License, either version 1.3
  of this license or (at your option) any later version.}}
\hypersetup{pdflicenseurl={http://www.latex-project.org/lppl.txt}}
\hypersetup{pdfcontactaddress={ETH Zurich, ITP, HIT K,
  Wolfgang-Pauli-Strasse 27}}
\hypersetup{pdfcontactpostcode={8093}}
\hypersetup{pdfcontactcity={Zurich}}
\hypersetup{pdfcontactcountry={Switzerland}}
\hypersetup{pdfcontactemail={nbeisert@itp.phys.ethz.ch}}
\hypersetup{pdfcontacturl={http://people.phys.ethz.ch/\xmptilde nbeisert/}}

\newcommand{\secref}[1]{\hyperref[#1]{section \ref*{#1}}}

\parskip1ex
\parindent0pt
\let\olditemize\itemize
\def\itemize{\olditemize\parskip0pt}

\begin{document}

\title{The \textsf{childdoc} Package}
\hypersetup{pdftitle={The childdoc Package}}
\author{Niklas Beisert\\[2ex]
  Institut f\"ur Theoretische Physik\\
  Eidgen\"ossische Technische Hochschule Z\"urich\\
  Wolfgang-Pauli-Strasse 27, 8093 Z\"urich, Switzerland\\[1ex]
  \href{mailto:nbeisert@itp.phys.ethz.ch}
  {\texttt{nbeisert@itp.phys.ethz.ch}}}
\hypersetup{pdfauthor={Niklas Beisert}}
\hypersetup{pdfsubject={Manual for the LaTeX2e Package childdoc}}
\date{30 December 2018, \textsf{v2.0}}
\maketitle

\begin{abstract}\noindent
\textsf{childdoc} is a \LaTeXe{} package
that enables the direct compilation
of document sections included by |\include|
to individual files.
\end{abstract}

\begingroup
\parskip0ex
\tableofcontents
\endgroup

%%%%%%%%%%%%%%%%%%%%%%%%%%%%%%%%%%%%%%%%%%%%%%%%%%%%%%%%%%%%%%%%%%%%%%%%%%%%%%%%
%%%%%%%%%%%%%%%%%%%%%%%%%%%%%%%%%%%%%%%%%%%%%%%%%%%%%%%%%%%%%%%%%%%%%%%%%%%%%%%%
\section{Introduction}

\LaTeX{} provides a mechanism to structure a large document (such as a book)
into a main file and several child files (containing the chapters)
using the |\include| command.
This mechanism is beneficial for documents
which span hundreds of pages in order to
make the source file(s) more manageable.
Moreover, compilation can be restricted to
selected child files by means of the |\includeonly| command.
The latter feature can be used to reduce the compilation time while editing
(this was significantly more useful in the earlier days of \LaTeX{})
or to generate a smaller document which is easier to navigate.
Another application of |\includeonly| is to generate
documents consisting of selected parts of the complete document.

However, there are a few drawbacks of the plain |\include| mechanism:
\begin{itemize}
\item
The child files cannot be compiled on their own,
they can only be compiled via the main file.
A naive editing environment
(such as a text editor with an option
to have the current file processed by \LaTeX)
may require one to switch to the main file before compiling;
attempting to compile the child file produces errors.
\item
The main file must be modified (each time)
to adjust the |\includeonly| command
to the present needs. This easily leaves the main file in a messy state.
\item
The generated document will always carry the filename
of the main document. This is inconvenient if
several child files are to be compiled and
to be kept for distribution.
\end{itemize}

The present package provides a simple interface
to make child files individually compilable by \LaTeX{}.
Compiling a child file then has the same effect as compiling
the main file with an |\includeonly| command
to select the appropriate child.
Moreover the generated document will carry the name of the child
rather than the main file.
This resolves all three above issues.

This feature is meant to make the editing of books,
thesis documents and lecture notes somewhat more convenient.
However, the package can also be used efficiently for
composing a series of documents (such as exercise sheets)
which are typically distributed individually.
It then assists the author in generating the individual documents
(potentially in different versions)
as well as a document containing the collected series.
Another application is in developing style files
or other kinds of included material
where compilation of the style file could redirect
to a sample or test file.

%%%%%%%%%%%%%%%%%%%%%%%%%%%%%%%%%%%%%%%%%%%%%%%%%%%%%%%%%%%%%%%%%%%%%%%%%%%%%%%%
%%%%%%%%%%%%%%%%%%%%%%%%%%%%%%%%%%%%%%%%%%%%%%%%%%%%%%%%%%%%%%%%%%%%%%%%%%%%%%%%
\section{Usage}

First of all, the package \textsf{childdoc} is \emph{not} a standard
\LaTeXe{} |.sty| style file! Therefore it needs to be invoked in
a non-standard way.

%%%%%%%%%%%%%%%%%%%%%%%%%%%%%%%%%%%%%%%%%%%%%%%%%%%%%%%%%%%%%%%%%%%%%%%%%%%%%%%%
\subsection{Included Files}
\label{sec:include}

%%%%%%%%%%%%%%%%%%%%%%%%%%%%%%%%%%%%%%%%
\DescribeMacro{\childdocmain}
To use the package, add the commands
\begin{center}
\begin{tabular}{l}
|\input{childdoc.def}|\\
|\childdocmain{}|\\
\end{tabular}
\end{center}
at the very top of the main \LaTeX{} file,
in particular \emph{before} the |\documentclass| statement!
The argument of |\childdocmain| should be left empty
(but it must be present).

%%%%%%%%%%%%%%%%%%%%%%%%%%%%%%%%%%%%%%%%
\DescribeMacro{\childdocof}
Furthermore, add the commands
\begin{center}
\begin{tabular}{l}
|\input{childdoc.def}|\\
|\childdocof{|\textit{main}|}|\\
\end{tabular}
\end{center}
at the top of every child file \textit{child}
which is included by |\include{|\textit{child}|}|
from within the main file
(or at least for those files to be compiled individually).
The argument \textit{main} must be the filename of the main file.

There are a couple of
considerations in setting up the main and child documents:

%%%%%%%%%%%%%%%%%%%%%%%%%%%%%%%%%%%%%%%%
\paragraph{Restrictions.}

Please note the following restrictions:
\begin{itemize}
\item
|\childdocmain| must be called with one argument \textit{main}
to ensure compatibility with earlier version of the package.
It must either be empty (|\childdocmain{}|)
or precisely match the filename of the main file in which it is specified.
See \secref{sec:detection} for further information.
\item
The filename \textit{main} must be specified without the |.tex| extension.
\item
The filename \textit{main} is case sensitive
(even in case-insensitive file systems)
due to internal string comparison.
\item
The argument \textit{main} should be fully expanded, it cannot be a macro.
\item
Subdirectories and special characters should be avoided in filenames.
\item
The command |\childdocmain{|\textit{main}|}| must be followed by a whitespace.
It should not be followed immediately by another command
or by a comment mark `|%|'.
This is because the \TeX{} parser reads the token immediately following
the argument of |\childdocmain| and puts it
at the beginning of every child section;
however, a white\-space is ignored.
\end{itemize}

%%%%%%%%%%%%%%%%%%%%%%%%%%%%%%%%%%%%%%%%
\paragraph{Content of Main File.}

It is advisable to place all content in the child files included by |\include|.
Any output contained in the main file will appear in all child documents
unless suppressed manually;
it cannot be suppressed automatically by the |\includeonly| directive
and thus should normally be avoided.
A method to include some content in the main file
by means of conditional processing is described in \secref{sec:conditional}.

%%%%%%%%%%%%%%%%%%%%%%%%%%%%%%%%%%%%%%%%
\paragraph{Page Numbering.}

When only a part of the document is compiled,
the appropriate numbering of pages
(as well as other status parameters)
is determined from the |.aux| files.
The latter contain information from previous passes.
However this information needs to propagate through
all intermediate child documents.
Therefore the page numbering in child documents may well
be inconsistent until the complete document is compiled at least once.

A useful (if unconventional) way to always ensure a consistent
page numbering is to restart the numbering in each child document
and denote the pages by `\textit{child}|.|\textit{page}'
where \textit{child} represents the chapter/section number of the child file.
This can be achieved by the command
|\numberwithin{page}{|\textit{child}|}|
of the \textsf{amsmath} package
where \textit{child} can be |chapter| or |section|
depending on the chosen structuring.
Alternatively, one can modify the macro |\thepage| appropriately
and reset the counter |page| at the start of each child file.

%%%%%%%%%%%%%%%%%%%%%%%%%%%%%%%%%%%%%%%%%%%%%%%%%%%%%%%%%%%%%%%%%%%%%%%%%%%%%%%%
\subsection{Conditional Processing}
\label{sec:conditional}

The package provides a mechanism to compile different versions
of a document. To customise the versions further some conditional processing
can come in handy to distinguish which version is being compiled.
The package provides two macros to describe the compilation context:

%%%%%%%%%%%%%%%%%%%%%%%%%%%%%%%%%%%%%%%%
\DescribeMacro{\ifchilddoc}
The conditional |\ifchilddoc| distinguishes between the compilation of
child documents and the main document:
%
\begin{center}
|\ifchilddoc |\textit{child-code}| |[|\||else |\textit{main-code}]| \||fi|
\end{center}

%%%%%%%%%%%%%%%%%%%%%%%%%%%%%%%%%%%%%%%%
\DescribeMacro{\childdocname}
\DescribeMacro{\childdocjob}
The macro |\childdocname| contains the filename (without extension)
of the main or child file being processed.
Note that |\childdocjob| will always contain the name of the main file.

%%%%%%%%%%%%%%%%%%%%%%%%%%%%%%%%%%%%%%%%
\paragraph{Title Page.}

Conditional processing can be used to include a title or banner page
in the main document when proper precautions are taken.
Importantly, the code in the main file should ensure that the page counter
(as well as other status parameters which are stored in the |.aux| files)
takes the same value after the conditional processing.
Otherwise the page numbers may take divergent values
depending on which part is compiled.

For example, a title page could be declared by:
%
\begin{center}
\begin{tabular}{l}
|\ifchilddoc\||else|\\
|\addtocounter{page}{-1}|\\
\textit{code for title page}\\
|\newpage|\\
|\||fi|
\end{tabular}
\end{center}
%
A banner page for the child documents can be generated by:
%
\begin{center}
\begin{tabular}{l}
|\ifchilddoc|\\
|\addtocounter{page}{-1}|\\
\textit{code for banner page}\\
|\newpage|\\
|\||fi|
\end{tabular}
\end{center}
%
Here one could write a message such as:
\begin{center}
|This is the part \childdocname{} of \childdocjob{}.|
\end{center}

%%%%%%%%%%%%%%%%%%%%%%%%%%%%%%%%%%%%%%%%%%%%%%%%%%%%%%%%%%%%%%%%%%%%%%%%%%%%%%%%
\subsection{Flags}
\label{sec:flags}

The package makes it easy to generate different versions
of the main or child documents.
To this end compilation flags can be defined
and assigned different default values.
They will be particularly useful in conjunction
with the forwarding mechanism described in \secref{sec:forward}.

For example, it may be useful to have a flag |\version|
which can be set to |draft| or |final|.
The document source will contain some conditional code
depending on the value of |\version|.
Suppose further, the flag should default to |final| for the main file
and to |draft| for child files
which is a natural assignment for editing the document.
This is achieved by placing the following code
in the preamble of the main document
(below the |\childdocmain| directive):
%
\begin{center}
\begin{tabular}{l}
|\ifchilddoc|\\
|\providecommand{\version}{draft}|\\
|\||else|\\
|\providecommand{\version}{final}|\\
|\||fi|
\end{tabular}
\end{center}
%
The definition by |\providecommand| makes sure
that previous definitions are not overwritten.
Further statements |\providecommand{\version}{...}|
can thus be added before the above code to override it.

For the main file, one might add a line
(between |\childdocmain| and the above block)
%
\begin{center}
|%\ifchilddoc\||else\providecommand{\version}{draft}\||fi|
\end{center}
%
which can be uncommented to produce a draft version.
Likewise one can add a line to the very top of a child file
(above the |\childdocof{|\textit{main}|}| directive)
%
\begin{center}
|%\providecommand{\version}{final}|
\end{center}
%
which can be uncommented to produce the final version of this child document.

%%%%%%%%%%%%%%%%%%%%%%%%%%%%%%%%%%%%%%%%%%%%%%%%%%%%%%%%%%%%%%%%%%%%%%%%%%%%%%%%
\subsection{Forwarding}
\label{sec:forward}

Different versions of the main or child documents
using compilation flags as described in \secref{sec:flags}
can be (permanently) stored in different files
for convenient compilation, viewing and distribution.
To this end, the package defines a command
to pass on compilation to a different file:

%%%%%%%%%%%%%%%%%%%%%%%%%%%%%%%%%%%%%%%%
\DescribeMacro{\childdocforward}
The command |\childdocforward| redirects processing to
another source file:
%
\begin{center}
\begin{tabular}{l}
|\input{childdoc.def}|\\
|\childdocforward[|\textit{main}|]{|\textit{dest}|}|\\
\end{tabular}
\end{center}
%
The argument \textit{dest} is the destination file
(without extension).
It should be the main file or one of the child files.
Note that further \textsf{childdoc} directives
such as |\childdocof| and |\childdocforward|
in the indicated file will be processed in this form.
The optional argument \textit{main}
passes on directly to the main file \textit{main}
while pretending to compile the child \textit{dest}.
This form behaves as if \textit{dest}
issues |\childdocof{|\textit{main}|}| right away,
and no further \textsf{childdoc} directives will be processed.

%%%%%%%%%%%%%%%%%%%%%%%%%%%%%%%%%%%%%%%%
\DescribeMacro{\...prefix}
In the alternative form |\childdocforwardprefix|,
%
\begin{center}
\begin{tabular}{l}
|\input{childdoc.def}|\\
|\childdocforwardprefix[|\textit{main}|]{|\textit{prefix}|}{|\textit{dest}|}|
\end{tabular}
\end{center}
%
the destination file is determined by a pattern
depending on the current file:
To make this work, the current file must be called
`{\textit{prefix}\hspace{0.2em}\textit{suffix}}'
with \textit{prefix} matching precisely the argument.
Processing is then passed on to the file
`{\textit{dest}\hspace{0.2em}\textit{suffix}}'.
Surely, the same effect is achieved by
directly specifying the
argument `{\textit{dest}\hspace{0.2em}\textit{suffix}}'
in the first form.
However, that requires to set up a different file
for each child. With the alternative form of the command
all these files can have exactly the same content
which simplifies setting them up and maintaining them.

For example, the following file |draft.tex|
with a compilation flag |\version| as described in \secref{sec:flags}
compiles the main document as a draft:
%
\begin{center}
\begin{tabular}{l}
|\def\version{draft}|\\
|\input{childdoc.def}|\\
|\childdocforward{|\textit{main}|}|
\end{tabular}
\end{center}
%
Likewise, the following files |final|\textit{nn}|.tex|
compile the final version of the child document
|child|\textit{nn}|.tex|:
%
\begin{center}
\begin{tabular}{l}
|\def\version{final}|\\
|\input{childdoc.def}|\\
|\childdocforwardprefix{final}{child}|
\end{tabular}
\end{center}
%

Note that when several versions of a main file and/or of each child file
are to be generated, it may be convenient to set up a |Makefile| or
shell script to automatise the process.

%%%%%%%%%%%%%%%%%%%%%%%%%%%%%%%%%%%%%%%%%%%%%%%%%%%%%%%%%%%%%%%%%%%%%%%%%%%%%%%%
\subsection{Command Line Processing}
\label{sec:commandline}

The effect of redirection files can also be achieved by invoking
the \LaTeX{} compiler with a more elaborate command line.
Most conveniently this should be done as part
of a shell script or a |Makefile|.

When using \textsf{childdoc} in the main file, the following
command lines effectively perform a redirection
(note that depending on the shell being used,
backslashes may have to be doubled: `|\|' $\to$ `|\\|'):
%
\begin{center}
|... -jobname "|\textit{target}|" |\\|"|[\textit{flags}]%
|\input{childdoc.def}\childdocforward[|\textit{main}|]{|\textit{dest}|}"|
\end{center}
%
Here \textit{target} is the name of the output file,
\textit{main} is the name of the main file
and \textit{dest} is the name of the main or child file to be processed
(all filenames without extensions).
The optional argument \textit{main} can be omitted
if \textit{main} matches \textit{dest}.
Optionally, compilation \textit{flags} can be defined via |\def| commands.
This command line makes the \TeX{} engine believe
it is compiling the file \textit{target}
whose content is specified as the latter parameter.
The provided code then forwards the processing to
\textit{main} or \textit{dest} as described in \secref{sec:forward}.

%%%%%%%%%%%%%%%%%%%%%%%%%%%%%%%%%%%%%%%%%%%%%%%%%%%%%%%%%%%%%%%%%%%%%%%%%%%%%%%%
\subsection{Include by Input}
\label{sec:input}

Including child documents by |\include| has some restrictions by design.
Most notably, the content of a child document always occupies
its own set of pages; pages cannot be shared between child documents.
Usually, this behaviour makes perfect sense
because each child document contain an essential part of the document.
However, in some situations it may be desirable to compose
a document from a collection of parts
without having mandatory page breaks between then.
For this case, the package
provides a mechanism to include parts
by |\input| which can also be processed individually.
However, by construction this mechanism
requires manual handling of the content to be output.

%%%%%%%%%%%%%%%%%%%%%%%%%%%%%%%%%%%%%%%%
\DescribeMacro{\ifchilddocmanual}
The main file should be prepared as usual, see \secref{sec:include}.
However, the document body must make a distinction
between processing of an individual part and of the main document, e.g.:
%
\begin{center}
\begin{tabular}{l}
|\ifchilddocmanual|\\
|\input{\childdocname}|\\
|\||else|\\
\textit{document body with }|\input{|\textit{part}|}|\\
|\||fi|
\end{tabular}
\end{center}
%
The conditional |\ifchilddocmanual| is true whenever
a part to be included by |\input| is being compiled,
and the name of the part is stored in |\childdocname|.

%%%%%%%%%%%%%%%%%%%%%%%%%%%%%%%%%%%%%%%%
\DescribeMacro{\childdocby}
Each part to be included by |\input| should start with:
%
\begin{center}
\begin{tabular}{l}
|\input{childdoc.def}|\\
|\childdocby{|\textit{main}|}|\\
\end{tabular}
\end{center}
%
The directive |\childdocby| is similar to |\childdocof|
described in \secref{sec:include},
but the subsequent selection of content must be done manually.
To that end, both |\ifchilddoc| and |\ifchilddocmanual|
will be true upon processing of a part,
and the name of the part is stored in |\childdocname|.
Note that |\jobname| will be set to the filename of the current part
so that each part receives an individual |.aux| file
that does not interfere with the |.aux| file(s) of the main document.
This behaviour can be altered by the alternative form
|\childdocby[*]{|\textit{main}|}| (with a non-empty optional argument)
which uses the |.aux| file of the main document
by setting |\jobname| to \textit{main}.

%%%%%%%%%%%%%%%%%%%%%%%%%%%%%%%%%%%%%%%%%%%%%%%%%%%%%%%%%%%%%%%%%%%%%%%%%%%%%%%%
\subsection{Driver Development}
\label{sec:driver}

The \textsf{childdoc} mechanism can also be use for the development
of definition files such as \LaTeX{} styles or classes.
This case differs from the above setup with multiple parts
included by |\include| in that no |\includeonly| should be invoked.
This can be achieved by starting the include file
(before |\ProvidesPackage|) with:
%
\begin{center}
\begin{tabular}{l}
|\input{childdoc.def}|\\
|\childdocforward{|\textit{main}|}|\\
\end{tabular}
\end{center}
%
or alternatively with:
%
\begin{center}
\begin{tabular}{l}
|\input{childdoc.def}|\\
|\childdocby{|\textit{main}|}|\\
\end{tabular}
\end{center}
%
Both forms have slightly different effects as described above.
The main file is prepared as usual, see \secref{sec:include}.

%%%%%%%%%%%%%%%%%%%%%%%%%%%%%%%%%%%%%%%%%%%%%%%%%%%%%%%%%%%%%%%%%%%%%%%%%%%%%%%%
\subsection{Legacy Detection}
\label{sec:detection}

The directive |\childdocmain| in the main file can detect
whether the complete document or merely a child is to be compiled
even without using the directive |\childdocof|.
This method is deprecated because it is less robust
and there is no compelling reason to use it;
it is merely provided for backward compatibility
and it may be removed in future versions.

If the detection mechanism is to be used,
it is mandatory to correctly specify
the filename of the main file as the argument of |\childdocmain|:
%
\begin{center}
\begin{tabular}{l}
|\input{childdoc.def}|\\
|\childdocmain{|\textit{main}|}|\\
\end{tabular}
\end{center}
%
If |\jobname| does not match the argument \textit{main} of |\childdocmain|,
it is assumed that |\jobname| points to the child file to be compiled.
When using |\childdocmain| with the main file specified as argument,
it suffices to start a child file
with just |\input{|\textit{main}|}|
without loading of the package and using |\childdocof|.
If instead all processing is done
with the appropriate \textsf{childdoc} directives,
the argument of \textit{main} of |\childdocmain| can be empty.

An alternative version of the command line processing described
in \secref{sec:commandline} using the detection mechanism reads:
%
\begin{center}
|... -jobname "|\textit{target}|" "|[\textit{flags}]%
[|\def\jobname{|\textit{dest}|}|]|\input{|\textit{main}|}"|
\end{center}

%%%%%%%%%%%%%%%%%%%%%%%%%%%%%%%%%%%%%%%%%%%%%%%%%%%%%%%%%%%%%%%%%%%%%%%%%%%%%%%%
\subsection{Manual Code}
\label{sec:manual}

In case one cannot be certain whether the definitions file |childdoc.def|
is installed on the target \TeX{} distribution
and one prefers not to ship it,
it is conceivable to paste a few relevant commands into the sources.

To that end, drop all statements |\input{childdoc.def}|
and perform the replacements as outlined below.
Instead of |\childdocmain{|\textit{main}|}| add the following code
to the top of the main file:
%
\begin{center}
\begin{tabular}{l}
|\||ifdefined\childdocname\endinput\||fi\newif\ifchilddoc|\\
|\edef\childdocname{\scantokens\expandafter{\jobname\noexpand}}|\\
|\def\childdocmain{|\textit{main}|}\||ifx\childdocmain\childdocname\||else|\\
|\childdoctrue\includeonly{\childdocname}\let\jobname\childdocmain\||fi|\\
\end{tabular}
\end{center}
%
Instead of |\childdocof{|\textit{main}|}| just include the main file
at the top of each child file:
%
\begin{center}
|\input{|\textit{main}|}|
\end{center}
%
A simple redirection |\childdocforward{|\textit{dest}|}| is achieved by:
%
\begin{center}
|\def\jobname{|\textit{dest}|}\input{\jobname}|
\end{center}
%
The redirection with prefix
|\childdocforwardprefix[|\textit{prefix}|]{|\textit{dest}|}|
is accomplished by:
%
\begin{center}
\begin{tabular}{l}
|{\edef\jobname{\scantokens\expandafter{\jobname\noexpand}}|\\
|\def\redirectjob |\textit{prefix}|#1~~~{\gdef\jobname{|\textit{dest}|#1}}|\\
|\expandafter\redirectjob\jobname~~~}\input{\jobname}|
\end{tabular}
\end{center}

In an alternative approach,
child documents can be compiled by a specific command line
without additional code or specific definitions:
%
\begin{center}
|... -jobname "|\textit{target}|" "|[\textit{flags}]%
|\includeonly{|\textit{dest}|}\input{|\textit{main}|}"|
\end{center}
%

%%%%%%%%%%%%%%%%%%%%%%%%%%%%%%%%%%%%%%%%%%%%%%%%%%%%%%%%%%%%%%%%%%%%%%%%%%%%%%%%
%%%%%%%%%%%%%%%%%%%%%%%%%%%%%%%%%%%%%%%%%%%%%%%%%%%%%%%%%%%%%%%%%%%%%%%%%%%%%%%%
\section{Information}

%%%%%%%%%%%%%%%%%%%%%%%%%%%%%%%%%%%%%%%%%%%%%%%%%%%%%%%%%%%%%%%%%%%%%%%%%%%%%%%%
\subsection{Copyright}

Copyright \copyright{} 2017--2018 Niklas Beisert

This work may be distributed and/or modified under the
conditions of the \LaTeX{} Project Public License, either version 1.3
of this license or (at your option) any later version.
The latest version of this license is in
  \url{http://www.latex-project.org/lppl.txt}
and version 1.3 or later is part of all distributions of \LaTeX{}
version 2005/12/01 or later.

This work has the LPPL maintenance status `maintained'.

The Current Maintainer of this work is Niklas Beisert.

This work consists of the files |README.txt|, |childdoc.ins| and |childdoc.dtx|
as well as the derived files |childdoc.def|, |cdocsamp.tex|
with |cdocsch1.tex|, |cdocsch2.tex|, |cdocspt3.tex|, |cdocspt4.tex|,
|cdocsdrf.tex|, |cdocsfn1.tex|, |cdocsfn2.tex|
as well as |childdoc.pdf|.

%%%%%%%%%%%%%%%%%%%%%%%%%%%%%%%%%%%%%%%%%%%%%%%%%%%%%%%%%%%%%%%%%%%%%%%%%%%%%%%%
\subsection{Files and Installation}

The package consists of the files:
%
\begin{center}
\begin{tabular}{ll}
    |README.txt|   & readme file \\
    |childdoc.ins| & installation file \\
    |childdoc.dtx| & source file \\
    |childdoc.def| & definition file \\
    |cdocsamp.tex| & sample main file \\
    |cdocsch1.tex| & sample include file \\
    |cdocsch2.tex| & sample include file \\
    |cdocspt3.tex| & sample part file \\
    |cdocspt4.tex| & sample part file \\
    |cdocsdrf.tex| & sample redirection file \\
    |cdocsfn1.tex| & sample redirection file \\
    |cdocsfn2.tex| & sample redirection file \\
    |childdoc.pdf| & manual
\end{tabular}
\end{center}
%
The distribution consists of the files
|README.txt|, |childdoc.ins| and |childdoc.dtx|.
%
\begin{itemize}
\item
Run (pdf)\LaTeX{} on |childdoc.dtx|
to compile the manual |childdoc.pdf| (this file).
\item
Run \LaTeX{} on |childdoc.ins| to create the definitions file |childdoc.def|
and the sample |cdocsamp.tex| with include files
|cdocsch1.tex|, |cdocsch2.tex|, |cdocspt3.tex|, |cdocspt4.tex|,
|cdocsdrf.tex|, |cdocsfn1.tex|, |cdocsfn2.tex|.
Then copy the file |childdoc.def| to an appropriate directory of your \LaTeX{}
distribution, e.g.\ \textit{texmf-root}|/tex/latex/childdoc|.
\end{itemize}

%%%%%%%%%%%%%%%%%%%%%%%%%%%%%%%%%%%%%%%%%%%%%%%%%%%%%%%%%%%%%%%%%%%%%%%%%%%%%%%%
\subsection{Related CTAN Packages}

There are several other packages which offer a similar functionality:
%
\begin{itemize}
\item
The packages
\href{http://ctan.org/pkg/docmute}{\textsf{docmute}},
\href{http://ctan.org/pkg/includex}{\textsf{includex}} and
\href{http://ctan.org/pkg/standalone}{\textsf{standalone}}
provide commands to include only the document body of
a child file thus allowing both files to be compiled individually.
\item
The packages \href{http://ctan.org/pkg/subdocs}{\textsf{subdocs}}
and \href{http://ctan.org/pkg/subfiles}{\textsf{subfiles}}
provide structures in which the main and child documents can be
encapsulated and allowing them to be compiled individually.
The inclusion mechanism is different from the conventional |\include|.
\item
The package \href{http://ctan.org/pkg/combine}{\textsf{combine}}
is an elaborate solution to combine several documents into one.
\end{itemize}
%
See also the CTAN topic \href{http://ctan.org/topic/subdocs}{\textsf{subdocs}}
for further related packages.
The present package differs from the above solutions in that
a document structure constructed with the conventional |\include| mechanism
just needs two extra commands at the top of every file
such that all constituent files can be compiled individually.

%%%%%%%%%%%%%%%%%%%%%%%%%%%%%%%%%%%%%%%%%%%%%%%%%%%%%%%%%%%%%%%%%%%%%%%%%%%%%%%%
%\subsection{Feature Suggestions}
%
%The following is a list of features which may be useful for future
%versions of this package:
%%
%\begin{itemize}
%\item
%\ldots
%\end{itemize}

%%%%%%%%%%%%%%%%%%%%%%%%%%%%%%%%%%%%%%%%%%%%%%%%%%%%%%%%%%%%%%%%%%%%%%%%%%%%%%%%
\subsection{Revision History}

%%%%%%%%%%%%%%%%%%%%%%%%%%%%%%%%%%%%%%%%
\paragraph{v2.0:} 2018/12/30

\begin{itemize}
\item
immediate forward processing
\item
added |\childdocby| mechanism
\item
manual restructured
\end{itemize}

%%%%%%%%%%%%%%%%%%%%%%%%%%%%%%%%%%%%%%%%
\paragraph{v1.6:} 2018/01/17

\begin{itemize}
\item
application for development of include files
\item
corrections to manual
\end{itemize}

%%%%%%%%%%%%%%%%%%%%%%%%%%%%%%%%%%%%%%%%
\paragraph{v1.5:} 2017/05/21

\begin{itemize}
\item
more complete structuring introduced
\item
|\childdocof| introduced
\item
|\childdoc| renamed to |\childdocmain|
\item
|\childredirect| renamed to |\childdocforward| and |\childdocforwardprefix|
and functionality expanded
\end{itemize}

%%%%%%%%%%%%%%%%%%%%%%%%%%%%%%%%%%%%%%%%
\paragraph{v1.0:} 2017/04/27

\begin{itemize}
\item
manual and install package
\item
first version published on CTAN
\end{itemize}

%%%%%%%%%%%%%%%%%%%%%%%%%%%%%%%%%%%%%%%%
\paragraph{v0.6:} 2017/04/26

\begin{itemize}
\item
redirection mechanism added
\end{itemize}

%%%%%%%%%%%%%%%%%%%%%%%%%%%%%%%%%%%%%%%%
\paragraph{v0.5:} 2017/04/26

\begin{itemize}
\item
functionality in definition file
\end{itemize}


%%%%%%%%%%%%%%%%%%%%%%%%%%%%%%%%%%%%%%%%%%%%%%%%%%%%%%%%%%%%%%%%%%%%%%%%%%%%%%%%
%%%%%%%%%%%%%%%%%%%%%%%%%%%%%%%%%%%%%%%%%%%%%%%%%%%%%%%%%%%%%%%%%%%%%%%%%%%%%%%%
%%%%%%%%%%%%%%%%%%%%%%%%%%%%%%%%%%%%%%%%%%%%%%%%%%%%%%%%%%%%%%%%%%%%%%%%%%%%%%%%
\appendix

\settowidth\MacroIndent{\rmfamily\scriptsize 000\ }

 \DocInput{childdoc.dtx}

\end{document}
%</driver>
% \fi
%
% %%%%%%%%%%%%%%%%%%%%%%%%%%%%%%%%%%%%%%%%%%%%%%%%%%%%%%%%%%%%%%%%%%%%%%%%%%%%%%
% %%%%%%%%%%%%%%%%%%%%%%%%%%%%%%%%%%%%%%%%%%%%%%%%%%%%%%%%%%%%%%%%%%%%%%%%%%%%%%
% \section{Sample}
%\iffalse
%<*samplemain>
%\fi
%
% The following presents a sample document
% with two chapters, two parts, a title page,
% a compile flag as well as three forwarding files to set the flag.
% It consists of eight |.tex| files:
% \begin{center}
% \begin{tabular}{ll}
% |cdocsamp.tex|&main file\\
% |cdocsch1.tex|&include file for chapter 1\\
% |cdocsch2.tex|&include file for chapter 2\\
% |cdocspt3.tex|&include file for part 3\\
% |cdocspt4.tex|&include file for part 4\\
% |cdocsdrf.tex|&forwarding file for main file in draft mode\\
% |cdocsfi1.tex|&forwarding file for final version of chapter 1\\
% |cdocsfi2.tex|&forwarding file for final version of chapter 2\\
% \end{tabular}
% \end{center}
% Each of the eight files can be compiled directly by the \LaTeX{} compiler.
%
% %%%%%%%%%%%%%%%%%%%%%%%%%%%%%%%%%%%%%%
% \paragraph{Main File.}
%
% The main file is called |cdocsamp.tex|.
%
% Load the \textsf{childdoc} definitions and
% declare the filename for the main document:
%    \begin{macrocode}
\input{childdoc.def}
\childdocmain{}
%    \end{macrocode}

% Optional override for |\version| flag:
%    \begin{macrocode}
%%\ifchilddoc\else\providecommand{\version}{draft}\fi
%    \end{macrocode}

% Define the default values for the |\version| flag
% (|final| for the main file and |draft| for childs):
%    \begin{macrocode}
\ifchilddoc
\providecommand{\version}{draft}
\else
\providecommand{\version}{final}
\fi
%    \end{macrocode}

% Load the standard document class:
%    \begin{macrocode}
\documentclass[12pt]{article}
%    \end{macrocode}

% Start the document body:
%    \begin{macrocode}
\begin{document}
%    \end{macrocode}

% Declare a title page.
% Print title, part of document being processed and version flag:
%    \begin{macrocode}
\addtocounter{page}{-1}
\begin{center}
{\LARGE\bfseries{}childdoc example\par}
\vspace{1cm}
\ifchilddoc
\ifchilddocmanual part\else chapter\fi:
`\childdocname' of `\childdocjob'\par
\else
main document: `\childdocjob'\par
\fi
version: \version\par
\end{center}
\newpage
%    \end{macrocode}

% Manually include selected file,
% otherwise process as usual:
%    \begin{macrocode}
\ifchilddocmanual
\section*{part `\childdocname'}
\input{\childdocname}
\else
%    \end{macrocode}

% Include the two chapters:
%    \begin{macrocode}
\include{cdocsch1}
\include{cdocsch2}
%    \end{macrocode}

% Include the two parts unless only chapters should be displayed:
%    \begin{macrocode}
\ifchilddoc\else
\section{part three}
\input{cdocspt3}
\section{part four}
\input{cdocspt4}
\fi
%    \end{macrocode}

% Process as usual until here:
%    \begin{macrocode}
\fi
%    \end{macrocode}

% End of document body:
%    \begin{macrocode}
\end{document}
%    \end{macrocode}
%\iffalse
%</samplemain>
%\fi
%
% %%%%%%%%%%%%%%%%%%%%%%%%%%%%%%%%%%%%%%
% \paragraph{Chapter Include Files.}
%
% The include files are called |cdocsch1.tex| and |cdocsch2.tex|.
%
%\iffalse
%<*samplechap1|samplechap2>
%\fi

% Optional override for |\version| flag:
%    \begin{macrocode}
%%\providecommand{\version}{final}
%    \end{macrocode}

% Include the main document:
%    \begin{macrocode}
\input{childdoc.def}
\childdocof{cdocsamp}
%    \end{macrocode}

%\iffalse
%</samplechap1|samplechap2>
%\fi
%
%\iffalse
%<*samplechap1>
%\fi
% Some text for chapter 1:
%    \begin{macrocode}
\section{one}
some text in chapter one
%    \end{macrocode}

%\iffalse
%</samplechap1>
%\fi
% Some text for chapter 2:
%\iffalse
%<*samplechap2>
%\fi
%    \begin{macrocode}
\section{two}
more text in chapter two
%    \end{macrocode}

%\iffalse
%</samplechap2>
%\fi
%
% %%%%%%%%%%%%%%%%%%%%%%%%%%%%%%%%%%%%%%
% \paragraph{Part Include Files.}
%
% The include files are called |cdocspt3.tex| and |cdocspt4.tex|.
%
%\iffalse
%<*samplepart3|samplepart4>
%\fi

% Optional override for |\version| flag:
%    \begin{macrocode}
%%\providecommand{\version}{final}
%    \end{macrocode}

% Include the main document:
%    \begin{macrocode}
\input{childdoc.def}
\childdocby{cdocsamp}
%    \end{macrocode}

%\iffalse
%</samplepart3|samplepart4>
%\fi
%
%\iffalse
%<*samplepart3>
%\fi
% Some text for part 3:
%    \begin{macrocode}
some text in part three
%    \end{macrocode}

%\iffalse
%</samplepart3>
%\fi
% Some text for part 4:
%\iffalse
%<*samplepart4>
%\fi
%    \begin{macrocode}
more text in part four
%    \end{macrocode}

%\iffalse
%</samplepart4>
%\fi
%
% %%%%%%%%%%%%%%%%%%%%%%%%%%%%%%%%%%%%%%
% \paragraph{Forwarding for a Complete Draft.}
%
% The following forwarding file |cdocsdrf.tex|
% compiles the main document in draft mode:
%\iffalse
%<*sampledraft>
%\fi
%    \begin{macrocode}
\def\version{draft}
\input{childdoc.def}
\childdocforward{cdocsamp}
%    \end{macrocode}

%\iffalse
%</sampledraft>
%\fi
%
% %%%%%%%%%%%%%%%%%%%%%%%%%%%%%%%%%%%%%%
% \paragraph{Forwarding for Final Version of the Chapters.}
%
% The following forwarding files |cdocsfn1.tex| and |cdocsfn2.tex|
% (with identical content)
% compile the final versions of the child documents
% |cdocsch1.tex| and |cdocsch2.tex|, respectively:
%\iffalse
%<*samplefinal>
%\fi
%    \begin{macrocode}
\def\version{final}
\input{childdoc.def}
\childdocforwardprefix[cdocsamp]{cdocsfn}{cdocsch}
%    \end{macrocode}

%\iffalse
%</samplefinal>
%\fi
%
% %%%%%%%%%%%%%%%%%%%%%%%%%%%%%%%%%%%%%%
% \paragraph{Command Line Processing.}
%
% The following three command lines generate the output files
% |cdocscld|, |cdocscl1| and |cdocscl2|
% which should be identical to
% |cdocsdrf|, |cdocsch1| and |cdocsfn2|, respectively:
% \begin{center}
% \begin{tabular}{l}
% |latex -jobname cdocscld \|\\
% |  "\def\version{draft}\input{childdoc.def}\childdocforward{cdocsamp}"|\\
% |latex -jobname cdocscl1 \|\\
% |  "\input{childdoc.def}\childdocforward[cdocsamp]{cdocsch1}"|\\
% |latex -jobname cdocscl2 \|\\
% |  "\def\version{final}\input{childdoc.def}\childdocforward{cdocsch2}"|
% \end{tabular}
% \end{center}
% Note that the trailing backslash on each first line
% merely continues the input to the second line
% (for convenient cut ant paste).
% Furthermore, the command |latex| can be replaced by any
% of its alternative versions such as |pdflatex|.
%
% %%%%%%%%%%%%%%%%%%%%%%%%%%%%%%%%%%%%%%%%%%%%%%%%%%%%%%%%%%%%%%%%%%%%%%%%%%%%%%
% %%%%%%%%%%%%%%%%%%%%%%%%%%%%%%%%%%%%%%%%%%%%%%%%%%%%%%%%%%%%%%%%%%%%%%%%%%%%%%
% \section{Implementation}
%\iffalse
%<*package>
%\fi
%
% This section describes the definitions file |childdoc.def|.

% The definitions cannot be loaded using |\usepackage| or |\RequirePackage|
% which has a mechanism to prevent loading a style file more than once.
% When loading the definitions by means of |\input|
% multiple instances have to be prevented manually:
%\iffalse
%This code needs to be before the `\ProvidesFile' directive
%which is defined at the beginning of this file.
%Therefore it is also placed there and commented out here.
%</package>
%<*discard>
%\fi
%    \begin{macrocode}
\ifdefined\childdocmain\endinput\fi
%    \end{macrocode}
%\iffalse
%</discard>
%<*package>
%\fi
%
% \macro{\ifchilddoc}
% \macro{\ifchilddocmanual}
% The conditional |\ifchilddoc| tells whether a
% child (true) or main (false) document is being compiled.
% The conditional |\ifchilddocmanual| tells whether
% the |\includeonly| mechanism is used (false) or
% the selection of child files must be performed manually (true).
% The definitions initialise to false:
%    \begin{macrocode}
\newif\ifchilddoc
\newif\ifchilddocmanual
%    \end{macrocode}

% \macro{\childdocname}
% \macro{\childdocjob}
% The macro |\childdocname| stores the name of the main document
% to be compiled. The macro |\childdocjob| stores the name of
% the document on which the \LaTeX{} compiler was originally invoked.
% The content of |\jobname| cannot be compared
% to filenames specified in the source due to different catcodes.
% The following code rescans |\jobname|, stores the result
% in |\childdocname| and saves a copy in |\childdocjob|:
%    \begin{macrocode}
\edef\childdocname{\scantokens\expandafter{\jobname\noexpand}}
\let\childdocjob\childdocname
%    \end{macrocode}

% \macro{\childdocdisable}
% The macro |\childdocdisable| prevents the main file
% from being processed more than once.
% At this stage, the main document command |\childdocmain|
% is assumed to be called once again where it should do nothing.
% Any subsequent call to it should prevent
% a secondary processing of the main document
% It overwrites the forwarding commands
% |\childdocof| and |\childdocforward|
% with empty macros to prevent further inclusions of the main document:
%    \begin{macrocode}
\newcommand{\childdocdisable}
{
  \renewcommand{\childdocmain}[1]{\renewcommand{\childdocmain}[1]{\endinput}}
  \renewcommand{\childdocof}[1]{}
  \renewcommand{\childdocby}[2][]{}
  \renewcommand{\childdocforward}[2][]{}
  \renewcommand{\childdocdisable}{}
}
%    \end{macrocode}

% \macro{\childdocmain}
% The macro |\childdocmain| is to be called at the top of the main file
% with nothing or the main filename (without extension) as argument.
% First, it breaks loops.
% If the argument is not empty and does not match |\childdocname|
% (which is set by the first inclusion of |childdoc.def|),
% |\ifchilddoc| is set to true, |\includeonly| is applied to the child file
% and |\jobname| is set to the main file
% (for proper handling of |.aux| files):
%    \begin{macrocode}
\newcommand{\childdocmain}[1]
{
  \childdocdisable\childdocmain{}
  \if?#1?\else
    \begingroup
      \def\childdoctmp{#1}
      \ifx\childdoctmp\childdocname
        \def\childdoctmp{}
      \else
        \def\childdoctmp
        {
          \childdoctrue
          \includeonly{\childdocname}
          \def\childdocjob{#1}
          \def\jobname{#1}
        }
      \fi
      \expandafter
    \endgroup
    \childdoctmp
  \fi
}
%    \end{macrocode}

% \macro{\childdocof}
% The command |\childdocof| redirects
% compilation to the main file |#1|.
%    \begin{macrocode}
\newcommand{\childdocof}[1]
{
  \childdocdisable
  \childdoctrue
  \includeonly{\childdocname}
  \def\jobname{#1}
  \def\childdocjob{#1}
  \input{#1}
}
%    \end{macrocode}

% \macro{\childdocby}
% The command |\childdocby| ....
%    \begin{macrocode}
\newcommand{\childdocby}[2][]
{
  \childdocdisable
  \childdoctrue
  \childdocmanualtrue
  \if?#1?\else
    \def\jobname{#2}
  \fi
  \def\childdocjob{#2}
  \input{#2}
  \endinput
}
%    \end{macrocode}

% \macro{\childdocforward}
% The command |\childdocforward| redirects
% compilation to the main file or
% (if the optional argument is given) a child file.
% Parameters are set as if the main file
% or a child file starting with |\childdocof| was compiled.
% Then compilation is handed over to the main file:
%    \begin{macrocode}
\newcommand{\childdocforward}[2][]
{
  \begingroup
    \if?#1?
      \def\childdoctmp
      {
        \def\childdocname{#2}
        \def\childdocjob{#2}
        \def\jobname{#2}
        \input{#2}
        \endinput
      }
    \else
      \def\childdoctmp
      {
        \childdocdisable
        \def\childdocname{#2}
        \childdoctrue
        \includeonly{#2}
        \def\childdocjob{#1}
        \def\jobname{#1}
        \input{#1}
        \endinput
      }
    \fi
    \expandafter
  \endgroup
  \childdoctmp
}
%    \end{macrocode}

% \macro{\childdocforwardprefix}
% The command |\childdocforwardprefix| redirects
% compilation to the main or a child file by means of a pattern.
% The prefix |#1| in the current filename is replaced by |#2|
% and the suffix of the current filename is kept
% (it is assumed that the filename does not contain the substring `|~~~|'
% which is used as a delimiter).
% Compilation is handed over to the new file by |\childdocforward|:
%    \begin{macrocode}
\newcommand{\childdocforwardprefix}[3][]
{
  \begingroup
    \def\childdocextract #2##1~~~{\def\childdoctmp{\childdocforward[#1]{#3##1}}}
    \expandafter\childdocextract\childdocname~~~
    \expandafter
  \endgroup
  \childdoctmp
}
%    \end{macrocode}

% \macro{\childdoc}
% The deprecated macro |\childdoc| is a legacy version of |\childdocmain|:
%    \begin{macrocode}
\newcommand{\childdoc}{\childdocmain}
%    \end{macrocode}

% \macro{\childdocredirect}
% The deprecated macro |\childdocredirect| is a legacy version
% of |\childdocforward| and |\childdocforwardprefix|:
%    \begin{macrocode}
\newcommand{\childdocredirect}[2][]
{
  \begingroup
    \if?#1?
      \def\childdoctmp{\childdocforward{#2}}
    \else
      \def\childdoctmp{\childdocforwardprefix{#1}{#2}}
    \fi
    \expandafter
  \endgroup
  \childdoctmp
}
%    \end{macrocode}

%\iffalse
%</package>
%\fi
%
\endinput
|\\
|\childdocforward{|\textit{main}|}|\\
\end{tabular}
\end{center}
%
or alternatively with:
%
\begin{center}
\begin{tabular}{l}
|% \iffalse
%
% childdoc.dtx Copyright (C) 2017-2018 Niklas Beisert
%
% This work may be distributed and/or modified under the
% conditions of the LaTeX Project Public License, either version 1.3
% of this license or (at your option) any later version.
% The latest version of this license is in
%   http://www.latex-project.org/lppl.txt
% and version 1.3 or later is part of all distributions of LaTeX
% version 2005/12/01 or later.
%
% This work has the LPPL maintenance status `maintained'.
%
% The Current Maintainer of this work is Niklas Beisert.
%
% This work consists of the files childdoc.dtx and childdoc.ins
% and the derived files childdoc.def and cdocsamp.tex with
% cdocsch1.tex, cdocsch2.tex, cdocsdrf.tex, cdocsfn1.tex, cdocsfn2.tex.
%
%<package>\ifdefined\childdocmain\endinput\fi
%<package>\ProvidesFile{childdoc.def}[2018/12/30 v2.0 child document driver]
%<samplemain>\ProvidesFile{cdocsamp.tex}[2018/12/30 v2.0 sample for childdoc]
%<*driver>
%\ProvidesFile{childdoc.drv}[2018/12/30 v2.0 childdoc reference manual file]
\PassOptionsToClass{10pt,a4paper}{article}
\documentclass{ltxdoc}

\usepackage[margin=35mm]{geometry}
\usepackage{hyperref}
\usepackage{hyperxmp}
\usepackage[usenames]{color}

\hypersetup{colorlinks=true}
\hypersetup{pdfstartview=FitH}
\hypersetup{pdfpagemode=UseNone}
\hypersetup{pdfsource={}}
\hypersetup{pdflang={en-UK}}
\hypersetup{pdfcopyright={Copyright 2017-2018 Niklas Beisert.
  This work may be distributed and/or modified under the
  conditions of the LaTeX Project Public License, either version 1.3
  of this license or (at your option) any later version.}}
\hypersetup{pdflicenseurl={http://www.latex-project.org/lppl.txt}}
\hypersetup{pdfcontactaddress={ETH Zurich, ITP, HIT K,
  Wolfgang-Pauli-Strasse 27}}
\hypersetup{pdfcontactpostcode={8093}}
\hypersetup{pdfcontactcity={Zurich}}
\hypersetup{pdfcontactcountry={Switzerland}}
\hypersetup{pdfcontactemail={nbeisert@itp.phys.ethz.ch}}
\hypersetup{pdfcontacturl={http://people.phys.ethz.ch/\xmptilde nbeisert/}}

\newcommand{\secref}[1]{\hyperref[#1]{section \ref*{#1}}}

\parskip1ex
\parindent0pt
\let\olditemize\itemize
\def\itemize{\olditemize\parskip0pt}

\begin{document}

\title{The \textsf{childdoc} Package}
\hypersetup{pdftitle={The childdoc Package}}
\author{Niklas Beisert\\[2ex]
  Institut f\"ur Theoretische Physik\\
  Eidgen\"ossische Technische Hochschule Z\"urich\\
  Wolfgang-Pauli-Strasse 27, 8093 Z\"urich, Switzerland\\[1ex]
  \href{mailto:nbeisert@itp.phys.ethz.ch}
  {\texttt{nbeisert@itp.phys.ethz.ch}}}
\hypersetup{pdfauthor={Niklas Beisert}}
\hypersetup{pdfsubject={Manual for the LaTeX2e Package childdoc}}
\date{30 December 2018, \textsf{v2.0}}
\maketitle

\begin{abstract}\noindent
\textsf{childdoc} is a \LaTeXe{} package
that enables the direct compilation
of document sections included by |\include|
to individual files.
\end{abstract}

\begingroup
\parskip0ex
\tableofcontents
\endgroup

%%%%%%%%%%%%%%%%%%%%%%%%%%%%%%%%%%%%%%%%%%%%%%%%%%%%%%%%%%%%%%%%%%%%%%%%%%%%%%%%
%%%%%%%%%%%%%%%%%%%%%%%%%%%%%%%%%%%%%%%%%%%%%%%%%%%%%%%%%%%%%%%%%%%%%%%%%%%%%%%%
\section{Introduction}

\LaTeX{} provides a mechanism to structure a large document (such as a book)
into a main file and several child files (containing the chapters)
using the |\include| command.
This mechanism is beneficial for documents
which span hundreds of pages in order to
make the source file(s) more manageable.
Moreover, compilation can be restricted to
selected child files by means of the |\includeonly| command.
The latter feature can be used to reduce the compilation time while editing
(this was significantly more useful in the earlier days of \LaTeX{})
or to generate a smaller document which is easier to navigate.
Another application of |\includeonly| is to generate
documents consisting of selected parts of the complete document.

However, there are a few drawbacks of the plain |\include| mechanism:
\begin{itemize}
\item
The child files cannot be compiled on their own,
they can only be compiled via the main file.
A naive editing environment
(such as a text editor with an option
to have the current file processed by \LaTeX)
may require one to switch to the main file before compiling;
attempting to compile the child file produces errors.
\item
The main file must be modified (each time)
to adjust the |\includeonly| command
to the present needs. This easily leaves the main file in a messy state.
\item
The generated document will always carry the filename
of the main document. This is inconvenient if
several child files are to be compiled and
to be kept for distribution.
\end{itemize}

The present package provides a simple interface
to make child files individually compilable by \LaTeX{}.
Compiling a child file then has the same effect as compiling
the main file with an |\includeonly| command
to select the appropriate child.
Moreover the generated document will carry the name of the child
rather than the main file.
This resolves all three above issues.

This feature is meant to make the editing of books,
thesis documents and lecture notes somewhat more convenient.
However, the package can also be used efficiently for
composing a series of documents (such as exercise sheets)
which are typically distributed individually.
It then assists the author in generating the individual documents
(potentially in different versions)
as well as a document containing the collected series.
Another application is in developing style files
or other kinds of included material
where compilation of the style file could redirect
to a sample or test file.

%%%%%%%%%%%%%%%%%%%%%%%%%%%%%%%%%%%%%%%%%%%%%%%%%%%%%%%%%%%%%%%%%%%%%%%%%%%%%%%%
%%%%%%%%%%%%%%%%%%%%%%%%%%%%%%%%%%%%%%%%%%%%%%%%%%%%%%%%%%%%%%%%%%%%%%%%%%%%%%%%
\section{Usage}

First of all, the package \textsf{childdoc} is \emph{not} a standard
\LaTeXe{} |.sty| style file! Therefore it needs to be invoked in
a non-standard way.

%%%%%%%%%%%%%%%%%%%%%%%%%%%%%%%%%%%%%%%%%%%%%%%%%%%%%%%%%%%%%%%%%%%%%%%%%%%%%%%%
\subsection{Included Files}
\label{sec:include}

%%%%%%%%%%%%%%%%%%%%%%%%%%%%%%%%%%%%%%%%
\DescribeMacro{\childdocmain}
To use the package, add the commands
\begin{center}
\begin{tabular}{l}
|\input{childdoc.def}|\\
|\childdocmain{}|\\
\end{tabular}
\end{center}
at the very top of the main \LaTeX{} file,
in particular \emph{before} the |\documentclass| statement!
The argument of |\childdocmain| should be left empty
(but it must be present).

%%%%%%%%%%%%%%%%%%%%%%%%%%%%%%%%%%%%%%%%
\DescribeMacro{\childdocof}
Furthermore, add the commands
\begin{center}
\begin{tabular}{l}
|\input{childdoc.def}|\\
|\childdocof{|\textit{main}|}|\\
\end{tabular}
\end{center}
at the top of every child file \textit{child}
which is included by |\include{|\textit{child}|}|
from within the main file
(or at least for those files to be compiled individually).
The argument \textit{main} must be the filename of the main file.

There are a couple of
considerations in setting up the main and child documents:

%%%%%%%%%%%%%%%%%%%%%%%%%%%%%%%%%%%%%%%%
\paragraph{Restrictions.}

Please note the following restrictions:
\begin{itemize}
\item
|\childdocmain| must be called with one argument \textit{main}
to ensure compatibility with earlier version of the package.
It must either be empty (|\childdocmain{}|)
or precisely match the filename of the main file in which it is specified.
See \secref{sec:detection} for further information.
\item
The filename \textit{main} must be specified without the |.tex| extension.
\item
The filename \textit{main} is case sensitive
(even in case-insensitive file systems)
due to internal string comparison.
\item
The argument \textit{main} should be fully expanded, it cannot be a macro.
\item
Subdirectories and special characters should be avoided in filenames.
\item
The command |\childdocmain{|\textit{main}|}| must be followed by a whitespace.
It should not be followed immediately by another command
or by a comment mark `|%|'.
This is because the \TeX{} parser reads the token immediately following
the argument of |\childdocmain| and puts it
at the beginning of every child section;
however, a white\-space is ignored.
\end{itemize}

%%%%%%%%%%%%%%%%%%%%%%%%%%%%%%%%%%%%%%%%
\paragraph{Content of Main File.}

It is advisable to place all content in the child files included by |\include|.
Any output contained in the main file will appear in all child documents
unless suppressed manually;
it cannot be suppressed automatically by the |\includeonly| directive
and thus should normally be avoided.
A method to include some content in the main file
by means of conditional processing is described in \secref{sec:conditional}.

%%%%%%%%%%%%%%%%%%%%%%%%%%%%%%%%%%%%%%%%
\paragraph{Page Numbering.}

When only a part of the document is compiled,
the appropriate numbering of pages
(as well as other status parameters)
is determined from the |.aux| files.
The latter contain information from previous passes.
However this information needs to propagate through
all intermediate child documents.
Therefore the page numbering in child documents may well
be inconsistent until the complete document is compiled at least once.

A useful (if unconventional) way to always ensure a consistent
page numbering is to restart the numbering in each child document
and denote the pages by `\textit{child}|.|\textit{page}'
where \textit{child} represents the chapter/section number of the child file.
This can be achieved by the command
|\numberwithin{page}{|\textit{child}|}|
of the \textsf{amsmath} package
where \textit{child} can be |chapter| or |section|
depending on the chosen structuring.
Alternatively, one can modify the macro |\thepage| appropriately
and reset the counter |page| at the start of each child file.

%%%%%%%%%%%%%%%%%%%%%%%%%%%%%%%%%%%%%%%%%%%%%%%%%%%%%%%%%%%%%%%%%%%%%%%%%%%%%%%%
\subsection{Conditional Processing}
\label{sec:conditional}

The package provides a mechanism to compile different versions
of a document. To customise the versions further some conditional processing
can come in handy to distinguish which version is being compiled.
The package provides two macros to describe the compilation context:

%%%%%%%%%%%%%%%%%%%%%%%%%%%%%%%%%%%%%%%%
\DescribeMacro{\ifchilddoc}
The conditional |\ifchilddoc| distinguishes between the compilation of
child documents and the main document:
%
\begin{center}
|\ifchilddoc |\textit{child-code}| |[|\||else |\textit{main-code}]| \||fi|
\end{center}

%%%%%%%%%%%%%%%%%%%%%%%%%%%%%%%%%%%%%%%%
\DescribeMacro{\childdocname}
\DescribeMacro{\childdocjob}
The macro |\childdocname| contains the filename (without extension)
of the main or child file being processed.
Note that |\childdocjob| will always contain the name of the main file.

%%%%%%%%%%%%%%%%%%%%%%%%%%%%%%%%%%%%%%%%
\paragraph{Title Page.}

Conditional processing can be used to include a title or banner page
in the main document when proper precautions are taken.
Importantly, the code in the main file should ensure that the page counter
(as well as other status parameters which are stored in the |.aux| files)
takes the same value after the conditional processing.
Otherwise the page numbers may take divergent values
depending on which part is compiled.

For example, a title page could be declared by:
%
\begin{center}
\begin{tabular}{l}
|\ifchilddoc\||else|\\
|\addtocounter{page}{-1}|\\
\textit{code for title page}\\
|\newpage|\\
|\||fi|
\end{tabular}
\end{center}
%
A banner page for the child documents can be generated by:
%
\begin{center}
\begin{tabular}{l}
|\ifchilddoc|\\
|\addtocounter{page}{-1}|\\
\textit{code for banner page}\\
|\newpage|\\
|\||fi|
\end{tabular}
\end{center}
%
Here one could write a message such as:
\begin{center}
|This is the part \childdocname{} of \childdocjob{}.|
\end{center}

%%%%%%%%%%%%%%%%%%%%%%%%%%%%%%%%%%%%%%%%%%%%%%%%%%%%%%%%%%%%%%%%%%%%%%%%%%%%%%%%
\subsection{Flags}
\label{sec:flags}

The package makes it easy to generate different versions
of the main or child documents.
To this end compilation flags can be defined
and assigned different default values.
They will be particularly useful in conjunction
with the forwarding mechanism described in \secref{sec:forward}.

For example, it may be useful to have a flag |\version|
which can be set to |draft| or |final|.
The document source will contain some conditional code
depending on the value of |\version|.
Suppose further, the flag should default to |final| for the main file
and to |draft| for child files
which is a natural assignment for editing the document.
This is achieved by placing the following code
in the preamble of the main document
(below the |\childdocmain| directive):
%
\begin{center}
\begin{tabular}{l}
|\ifchilddoc|\\
|\providecommand{\version}{draft}|\\
|\||else|\\
|\providecommand{\version}{final}|\\
|\||fi|
\end{tabular}
\end{center}
%
The definition by |\providecommand| makes sure
that previous definitions are not overwritten.
Further statements |\providecommand{\version}{...}|
can thus be added before the above code to override it.

For the main file, one might add a line
(between |\childdocmain| and the above block)
%
\begin{center}
|%\ifchilddoc\||else\providecommand{\version}{draft}\||fi|
\end{center}
%
which can be uncommented to produce a draft version.
Likewise one can add a line to the very top of a child file
(above the |\childdocof{|\textit{main}|}| directive)
%
\begin{center}
|%\providecommand{\version}{final}|
\end{center}
%
which can be uncommented to produce the final version of this child document.

%%%%%%%%%%%%%%%%%%%%%%%%%%%%%%%%%%%%%%%%%%%%%%%%%%%%%%%%%%%%%%%%%%%%%%%%%%%%%%%%
\subsection{Forwarding}
\label{sec:forward}

Different versions of the main or child documents
using compilation flags as described in \secref{sec:flags}
can be (permanently) stored in different files
for convenient compilation, viewing and distribution.
To this end, the package defines a command
to pass on compilation to a different file:

%%%%%%%%%%%%%%%%%%%%%%%%%%%%%%%%%%%%%%%%
\DescribeMacro{\childdocforward}
The command |\childdocforward| redirects processing to
another source file:
%
\begin{center}
\begin{tabular}{l}
|\input{childdoc.def}|\\
|\childdocforward[|\textit{main}|]{|\textit{dest}|}|\\
\end{tabular}
\end{center}
%
The argument \textit{dest} is the destination file
(without extension).
It should be the main file or one of the child files.
Note that further \textsf{childdoc} directives
such as |\childdocof| and |\childdocforward|
in the indicated file will be processed in this form.
The optional argument \textit{main}
passes on directly to the main file \textit{main}
while pretending to compile the child \textit{dest}.
This form behaves as if \textit{dest}
issues |\childdocof{|\textit{main}|}| right away,
and no further \textsf{childdoc} directives will be processed.

%%%%%%%%%%%%%%%%%%%%%%%%%%%%%%%%%%%%%%%%
\DescribeMacro{\...prefix}
In the alternative form |\childdocforwardprefix|,
%
\begin{center}
\begin{tabular}{l}
|\input{childdoc.def}|\\
|\childdocforwardprefix[|\textit{main}|]{|\textit{prefix}|}{|\textit{dest}|}|
\end{tabular}
\end{center}
%
the destination file is determined by a pattern
depending on the current file:
To make this work, the current file must be called
`{\textit{prefix}\hspace{0.2em}\textit{suffix}}'
with \textit{prefix} matching precisely the argument.
Processing is then passed on to the file
`{\textit{dest}\hspace{0.2em}\textit{suffix}}'.
Surely, the same effect is achieved by
directly specifying the
argument `{\textit{dest}\hspace{0.2em}\textit{suffix}}'
in the first form.
However, that requires to set up a different file
for each child. With the alternative form of the command
all these files can have exactly the same content
which simplifies setting them up and maintaining them.

For example, the following file |draft.tex|
with a compilation flag |\version| as described in \secref{sec:flags}
compiles the main document as a draft:
%
\begin{center}
\begin{tabular}{l}
|\def\version{draft}|\\
|\input{childdoc.def}|\\
|\childdocforward{|\textit{main}|}|
\end{tabular}
\end{center}
%
Likewise, the following files |final|\textit{nn}|.tex|
compile the final version of the child document
|child|\textit{nn}|.tex|:
%
\begin{center}
\begin{tabular}{l}
|\def\version{final}|\\
|\input{childdoc.def}|\\
|\childdocforwardprefix{final}{child}|
\end{tabular}
\end{center}
%

Note that when several versions of a main file and/or of each child file
are to be generated, it may be convenient to set up a |Makefile| or
shell script to automatise the process.

%%%%%%%%%%%%%%%%%%%%%%%%%%%%%%%%%%%%%%%%%%%%%%%%%%%%%%%%%%%%%%%%%%%%%%%%%%%%%%%%
\subsection{Command Line Processing}
\label{sec:commandline}

The effect of redirection files can also be achieved by invoking
the \LaTeX{} compiler with a more elaborate command line.
Most conveniently this should be done as part
of a shell script or a |Makefile|.

When using \textsf{childdoc} in the main file, the following
command lines effectively perform a redirection
(note that depending on the shell being used,
backslashes may have to be doubled: `|\|' $\to$ `|\\|'):
%
\begin{center}
|... -jobname "|\textit{target}|" |\\|"|[\textit{flags}]%
|\input{childdoc.def}\childdocforward[|\textit{main}|]{|\textit{dest}|}"|
\end{center}
%
Here \textit{target} is the name of the output file,
\textit{main} is the name of the main file
and \textit{dest} is the name of the main or child file to be processed
(all filenames without extensions).
The optional argument \textit{main} can be omitted
if \textit{main} matches \textit{dest}.
Optionally, compilation \textit{flags} can be defined via |\def| commands.
This command line makes the \TeX{} engine believe
it is compiling the file \textit{target}
whose content is specified as the latter parameter.
The provided code then forwards the processing to
\textit{main} or \textit{dest} as described in \secref{sec:forward}.

%%%%%%%%%%%%%%%%%%%%%%%%%%%%%%%%%%%%%%%%%%%%%%%%%%%%%%%%%%%%%%%%%%%%%%%%%%%%%%%%
\subsection{Include by Input}
\label{sec:input}

Including child documents by |\include| has some restrictions by design.
Most notably, the content of a child document always occupies
its own set of pages; pages cannot be shared between child documents.
Usually, this behaviour makes perfect sense
because each child document contain an essential part of the document.
However, in some situations it may be desirable to compose
a document from a collection of parts
without having mandatory page breaks between then.
For this case, the package
provides a mechanism to include parts
by |\input| which can also be processed individually.
However, by construction this mechanism
requires manual handling of the content to be output.

%%%%%%%%%%%%%%%%%%%%%%%%%%%%%%%%%%%%%%%%
\DescribeMacro{\ifchilddocmanual}
The main file should be prepared as usual, see \secref{sec:include}.
However, the document body must make a distinction
between processing of an individual part and of the main document, e.g.:
%
\begin{center}
\begin{tabular}{l}
|\ifchilddocmanual|\\
|\input{\childdocname}|\\
|\||else|\\
\textit{document body with }|\input{|\textit{part}|}|\\
|\||fi|
\end{tabular}
\end{center}
%
The conditional |\ifchilddocmanual| is true whenever
a part to be included by |\input| is being compiled,
and the name of the part is stored in |\childdocname|.

%%%%%%%%%%%%%%%%%%%%%%%%%%%%%%%%%%%%%%%%
\DescribeMacro{\childdocby}
Each part to be included by |\input| should start with:
%
\begin{center}
\begin{tabular}{l}
|\input{childdoc.def}|\\
|\childdocby{|\textit{main}|}|\\
\end{tabular}
\end{center}
%
The directive |\childdocby| is similar to |\childdocof|
described in \secref{sec:include},
but the subsequent selection of content must be done manually.
To that end, both |\ifchilddoc| and |\ifchilddocmanual|
will be true upon processing of a part,
and the name of the part is stored in |\childdocname|.
Note that |\jobname| will be set to the filename of the current part
so that each part receives an individual |.aux| file
that does not interfere with the |.aux| file(s) of the main document.
This behaviour can be altered by the alternative form
|\childdocby[*]{|\textit{main}|}| (with a non-empty optional argument)
which uses the |.aux| file of the main document
by setting |\jobname| to \textit{main}.

%%%%%%%%%%%%%%%%%%%%%%%%%%%%%%%%%%%%%%%%%%%%%%%%%%%%%%%%%%%%%%%%%%%%%%%%%%%%%%%%
\subsection{Driver Development}
\label{sec:driver}

The \textsf{childdoc} mechanism can also be use for the development
of definition files such as \LaTeX{} styles or classes.
This case differs from the above setup with multiple parts
included by |\include| in that no |\includeonly| should be invoked.
This can be achieved by starting the include file
(before |\ProvidesPackage|) with:
%
\begin{center}
\begin{tabular}{l}
|\input{childdoc.def}|\\
|\childdocforward{|\textit{main}|}|\\
\end{tabular}
\end{center}
%
or alternatively with:
%
\begin{center}
\begin{tabular}{l}
|\input{childdoc.def}|\\
|\childdocby{|\textit{main}|}|\\
\end{tabular}
\end{center}
%
Both forms have slightly different effects as described above.
The main file is prepared as usual, see \secref{sec:include}.

%%%%%%%%%%%%%%%%%%%%%%%%%%%%%%%%%%%%%%%%%%%%%%%%%%%%%%%%%%%%%%%%%%%%%%%%%%%%%%%%
\subsection{Legacy Detection}
\label{sec:detection}

The directive |\childdocmain| in the main file can detect
whether the complete document or merely a child is to be compiled
even without using the directive |\childdocof|.
This method is deprecated because it is less robust
and there is no compelling reason to use it;
it is merely provided for backward compatibility
and it may be removed in future versions.

If the detection mechanism is to be used,
it is mandatory to correctly specify
the filename of the main file as the argument of |\childdocmain|:
%
\begin{center}
\begin{tabular}{l}
|\input{childdoc.def}|\\
|\childdocmain{|\textit{main}|}|\\
\end{tabular}
\end{center}
%
If |\jobname| does not match the argument \textit{main} of |\childdocmain|,
it is assumed that |\jobname| points to the child file to be compiled.
When using |\childdocmain| with the main file specified as argument,
it suffices to start a child file
with just |\input{|\textit{main}|}|
without loading of the package and using |\childdocof|.
If instead all processing is done
with the appropriate \textsf{childdoc} directives,
the argument of \textit{main} of |\childdocmain| can be empty.

An alternative version of the command line processing described
in \secref{sec:commandline} using the detection mechanism reads:
%
\begin{center}
|... -jobname "|\textit{target}|" "|[\textit{flags}]%
[|\def\jobname{|\textit{dest}|}|]|\input{|\textit{main}|}"|
\end{center}

%%%%%%%%%%%%%%%%%%%%%%%%%%%%%%%%%%%%%%%%%%%%%%%%%%%%%%%%%%%%%%%%%%%%%%%%%%%%%%%%
\subsection{Manual Code}
\label{sec:manual}

In case one cannot be certain whether the definitions file |childdoc.def|
is installed on the target \TeX{} distribution
and one prefers not to ship it,
it is conceivable to paste a few relevant commands into the sources.

To that end, drop all statements |\input{childdoc.def}|
and perform the replacements as outlined below.
Instead of |\childdocmain{|\textit{main}|}| add the following code
to the top of the main file:
%
\begin{center}
\begin{tabular}{l}
|\||ifdefined\childdocname\endinput\||fi\newif\ifchilddoc|\\
|\edef\childdocname{\scantokens\expandafter{\jobname\noexpand}}|\\
|\def\childdocmain{|\textit{main}|}\||ifx\childdocmain\childdocname\||else|\\
|\childdoctrue\includeonly{\childdocname}\let\jobname\childdocmain\||fi|\\
\end{tabular}
\end{center}
%
Instead of |\childdocof{|\textit{main}|}| just include the main file
at the top of each child file:
%
\begin{center}
|\input{|\textit{main}|}|
\end{center}
%
A simple redirection |\childdocforward{|\textit{dest}|}| is achieved by:
%
\begin{center}
|\def\jobname{|\textit{dest}|}\input{\jobname}|
\end{center}
%
The redirection with prefix
|\childdocforwardprefix[|\textit{prefix}|]{|\textit{dest}|}|
is accomplished by:
%
\begin{center}
\begin{tabular}{l}
|{\edef\jobname{\scantokens\expandafter{\jobname\noexpand}}|\\
|\def\redirectjob |\textit{prefix}|#1~~~{\gdef\jobname{|\textit{dest}|#1}}|\\
|\expandafter\redirectjob\jobname~~~}\input{\jobname}|
\end{tabular}
\end{center}

In an alternative approach,
child documents can be compiled by a specific command line
without additional code or specific definitions:
%
\begin{center}
|... -jobname "|\textit{target}|" "|[\textit{flags}]%
|\includeonly{|\textit{dest}|}\input{|\textit{main}|}"|
\end{center}
%

%%%%%%%%%%%%%%%%%%%%%%%%%%%%%%%%%%%%%%%%%%%%%%%%%%%%%%%%%%%%%%%%%%%%%%%%%%%%%%%%
%%%%%%%%%%%%%%%%%%%%%%%%%%%%%%%%%%%%%%%%%%%%%%%%%%%%%%%%%%%%%%%%%%%%%%%%%%%%%%%%
\section{Information}

%%%%%%%%%%%%%%%%%%%%%%%%%%%%%%%%%%%%%%%%%%%%%%%%%%%%%%%%%%%%%%%%%%%%%%%%%%%%%%%%
\subsection{Copyright}

Copyright \copyright{} 2017--2018 Niklas Beisert

This work may be distributed and/or modified under the
conditions of the \LaTeX{} Project Public License, either version 1.3
of this license or (at your option) any later version.
The latest version of this license is in
  \url{http://www.latex-project.org/lppl.txt}
and version 1.3 or later is part of all distributions of \LaTeX{}
version 2005/12/01 or later.

This work has the LPPL maintenance status `maintained'.

The Current Maintainer of this work is Niklas Beisert.

This work consists of the files |README.txt|, |childdoc.ins| and |childdoc.dtx|
as well as the derived files |childdoc.def|, |cdocsamp.tex|
with |cdocsch1.tex|, |cdocsch2.tex|, |cdocspt3.tex|, |cdocspt4.tex|,
|cdocsdrf.tex|, |cdocsfn1.tex|, |cdocsfn2.tex|
as well as |childdoc.pdf|.

%%%%%%%%%%%%%%%%%%%%%%%%%%%%%%%%%%%%%%%%%%%%%%%%%%%%%%%%%%%%%%%%%%%%%%%%%%%%%%%%
\subsection{Files and Installation}

The package consists of the files:
%
\begin{center}
\begin{tabular}{ll}
    |README.txt|   & readme file \\
    |childdoc.ins| & installation file \\
    |childdoc.dtx| & source file \\
    |childdoc.def| & definition file \\
    |cdocsamp.tex| & sample main file \\
    |cdocsch1.tex| & sample include file \\
    |cdocsch2.tex| & sample include file \\
    |cdocspt3.tex| & sample part file \\
    |cdocspt4.tex| & sample part file \\
    |cdocsdrf.tex| & sample redirection file \\
    |cdocsfn1.tex| & sample redirection file \\
    |cdocsfn2.tex| & sample redirection file \\
    |childdoc.pdf| & manual
\end{tabular}
\end{center}
%
The distribution consists of the files
|README.txt|, |childdoc.ins| and |childdoc.dtx|.
%
\begin{itemize}
\item
Run (pdf)\LaTeX{} on |childdoc.dtx|
to compile the manual |childdoc.pdf| (this file).
\item
Run \LaTeX{} on |childdoc.ins| to create the definitions file |childdoc.def|
and the sample |cdocsamp.tex| with include files
|cdocsch1.tex|, |cdocsch2.tex|, |cdocspt3.tex|, |cdocspt4.tex|,
|cdocsdrf.tex|, |cdocsfn1.tex|, |cdocsfn2.tex|.
Then copy the file |childdoc.def| to an appropriate directory of your \LaTeX{}
distribution, e.g.\ \textit{texmf-root}|/tex/latex/childdoc|.
\end{itemize}

%%%%%%%%%%%%%%%%%%%%%%%%%%%%%%%%%%%%%%%%%%%%%%%%%%%%%%%%%%%%%%%%%%%%%%%%%%%%%%%%
\subsection{Related CTAN Packages}

There are several other packages which offer a similar functionality:
%
\begin{itemize}
\item
The packages
\href{http://ctan.org/pkg/docmute}{\textsf{docmute}},
\href{http://ctan.org/pkg/includex}{\textsf{includex}} and
\href{http://ctan.org/pkg/standalone}{\textsf{standalone}}
provide commands to include only the document body of
a child file thus allowing both files to be compiled individually.
\item
The packages \href{http://ctan.org/pkg/subdocs}{\textsf{subdocs}}
and \href{http://ctan.org/pkg/subfiles}{\textsf{subfiles}}
provide structures in which the main and child documents can be
encapsulated and allowing them to be compiled individually.
The inclusion mechanism is different from the conventional |\include|.
\item
The package \href{http://ctan.org/pkg/combine}{\textsf{combine}}
is an elaborate solution to combine several documents into one.
\end{itemize}
%
See also the CTAN topic \href{http://ctan.org/topic/subdocs}{\textsf{subdocs}}
for further related packages.
The present package differs from the above solutions in that
a document structure constructed with the conventional |\include| mechanism
just needs two extra commands at the top of every file
such that all constituent files can be compiled individually.

%%%%%%%%%%%%%%%%%%%%%%%%%%%%%%%%%%%%%%%%%%%%%%%%%%%%%%%%%%%%%%%%%%%%%%%%%%%%%%%%
%\subsection{Feature Suggestions}
%
%The following is a list of features which may be useful for future
%versions of this package:
%%
%\begin{itemize}
%\item
%\ldots
%\end{itemize}

%%%%%%%%%%%%%%%%%%%%%%%%%%%%%%%%%%%%%%%%%%%%%%%%%%%%%%%%%%%%%%%%%%%%%%%%%%%%%%%%
\subsection{Revision History}

%%%%%%%%%%%%%%%%%%%%%%%%%%%%%%%%%%%%%%%%
\paragraph{v2.0:} 2018/12/30

\begin{itemize}
\item
immediate forward processing
\item
added |\childdocby| mechanism
\item
manual restructured
\end{itemize}

%%%%%%%%%%%%%%%%%%%%%%%%%%%%%%%%%%%%%%%%
\paragraph{v1.6:} 2018/01/17

\begin{itemize}
\item
application for development of include files
\item
corrections to manual
\end{itemize}

%%%%%%%%%%%%%%%%%%%%%%%%%%%%%%%%%%%%%%%%
\paragraph{v1.5:} 2017/05/21

\begin{itemize}
\item
more complete structuring introduced
\item
|\childdocof| introduced
\item
|\childdoc| renamed to |\childdocmain|
\item
|\childredirect| renamed to |\childdocforward| and |\childdocforwardprefix|
and functionality expanded
\end{itemize}

%%%%%%%%%%%%%%%%%%%%%%%%%%%%%%%%%%%%%%%%
\paragraph{v1.0:} 2017/04/27

\begin{itemize}
\item
manual and install package
\item
first version published on CTAN
\end{itemize}

%%%%%%%%%%%%%%%%%%%%%%%%%%%%%%%%%%%%%%%%
\paragraph{v0.6:} 2017/04/26

\begin{itemize}
\item
redirection mechanism added
\end{itemize}

%%%%%%%%%%%%%%%%%%%%%%%%%%%%%%%%%%%%%%%%
\paragraph{v0.5:} 2017/04/26

\begin{itemize}
\item
functionality in definition file
\end{itemize}


%%%%%%%%%%%%%%%%%%%%%%%%%%%%%%%%%%%%%%%%%%%%%%%%%%%%%%%%%%%%%%%%%%%%%%%%%%%%%%%%
%%%%%%%%%%%%%%%%%%%%%%%%%%%%%%%%%%%%%%%%%%%%%%%%%%%%%%%%%%%%%%%%%%%%%%%%%%%%%%%%
%%%%%%%%%%%%%%%%%%%%%%%%%%%%%%%%%%%%%%%%%%%%%%%%%%%%%%%%%%%%%%%%%%%%%%%%%%%%%%%%
\appendix

\settowidth\MacroIndent{\rmfamily\scriptsize 000\ }

 \DocInput{childdoc.dtx}

\end{document}
%</driver>
% \fi
%
% %%%%%%%%%%%%%%%%%%%%%%%%%%%%%%%%%%%%%%%%%%%%%%%%%%%%%%%%%%%%%%%%%%%%%%%%%%%%%%
% %%%%%%%%%%%%%%%%%%%%%%%%%%%%%%%%%%%%%%%%%%%%%%%%%%%%%%%%%%%%%%%%%%%%%%%%%%%%%%
% \section{Sample}
%\iffalse
%<*samplemain>
%\fi
%
% The following presents a sample document
% with two chapters, two parts, a title page,
% a compile flag as well as three forwarding files to set the flag.
% It consists of eight |.tex| files:
% \begin{center}
% \begin{tabular}{ll}
% |cdocsamp.tex|&main file\\
% |cdocsch1.tex|&include file for chapter 1\\
% |cdocsch2.tex|&include file for chapter 2\\
% |cdocspt3.tex|&include file for part 3\\
% |cdocspt4.tex|&include file for part 4\\
% |cdocsdrf.tex|&forwarding file for main file in draft mode\\
% |cdocsfi1.tex|&forwarding file for final version of chapter 1\\
% |cdocsfi2.tex|&forwarding file for final version of chapter 2\\
% \end{tabular}
% \end{center}
% Each of the eight files can be compiled directly by the \LaTeX{} compiler.
%
% %%%%%%%%%%%%%%%%%%%%%%%%%%%%%%%%%%%%%%
% \paragraph{Main File.}
%
% The main file is called |cdocsamp.tex|.
%
% Load the \textsf{childdoc} definitions and
% declare the filename for the main document:
%    \begin{macrocode}
\input{childdoc.def}
\childdocmain{}
%    \end{macrocode}

% Optional override for |\version| flag:
%    \begin{macrocode}
%%\ifchilddoc\else\providecommand{\version}{draft}\fi
%    \end{macrocode}

% Define the default values for the |\version| flag
% (|final| for the main file and |draft| for childs):
%    \begin{macrocode}
\ifchilddoc
\providecommand{\version}{draft}
\else
\providecommand{\version}{final}
\fi
%    \end{macrocode}

% Load the standard document class:
%    \begin{macrocode}
\documentclass[12pt]{article}
%    \end{macrocode}

% Start the document body:
%    \begin{macrocode}
\begin{document}
%    \end{macrocode}

% Declare a title page.
% Print title, part of document being processed and version flag:
%    \begin{macrocode}
\addtocounter{page}{-1}
\begin{center}
{\LARGE\bfseries{}childdoc example\par}
\vspace{1cm}
\ifchilddoc
\ifchilddocmanual part\else chapter\fi:
`\childdocname' of `\childdocjob'\par
\else
main document: `\childdocjob'\par
\fi
version: \version\par
\end{center}
\newpage
%    \end{macrocode}

% Manually include selected file,
% otherwise process as usual:
%    \begin{macrocode}
\ifchilddocmanual
\section*{part `\childdocname'}
\input{\childdocname}
\else
%    \end{macrocode}

% Include the two chapters:
%    \begin{macrocode}
\include{cdocsch1}
\include{cdocsch2}
%    \end{macrocode}

% Include the two parts unless only chapters should be displayed:
%    \begin{macrocode}
\ifchilddoc\else
\section{part three}
\input{cdocspt3}
\section{part four}
\input{cdocspt4}
\fi
%    \end{macrocode}

% Process as usual until here:
%    \begin{macrocode}
\fi
%    \end{macrocode}

% End of document body:
%    \begin{macrocode}
\end{document}
%    \end{macrocode}
%\iffalse
%</samplemain>
%\fi
%
% %%%%%%%%%%%%%%%%%%%%%%%%%%%%%%%%%%%%%%
% \paragraph{Chapter Include Files.}
%
% The include files are called |cdocsch1.tex| and |cdocsch2.tex|.
%
%\iffalse
%<*samplechap1|samplechap2>
%\fi

% Optional override for |\version| flag:
%    \begin{macrocode}
%%\providecommand{\version}{final}
%    \end{macrocode}

% Include the main document:
%    \begin{macrocode}
\input{childdoc.def}
\childdocof{cdocsamp}
%    \end{macrocode}

%\iffalse
%</samplechap1|samplechap2>
%\fi
%
%\iffalse
%<*samplechap1>
%\fi
% Some text for chapter 1:
%    \begin{macrocode}
\section{one}
some text in chapter one
%    \end{macrocode}

%\iffalse
%</samplechap1>
%\fi
% Some text for chapter 2:
%\iffalse
%<*samplechap2>
%\fi
%    \begin{macrocode}
\section{two}
more text in chapter two
%    \end{macrocode}

%\iffalse
%</samplechap2>
%\fi
%
% %%%%%%%%%%%%%%%%%%%%%%%%%%%%%%%%%%%%%%
% \paragraph{Part Include Files.}
%
% The include files are called |cdocspt3.tex| and |cdocspt4.tex|.
%
%\iffalse
%<*samplepart3|samplepart4>
%\fi

% Optional override for |\version| flag:
%    \begin{macrocode}
%%\providecommand{\version}{final}
%    \end{macrocode}

% Include the main document:
%    \begin{macrocode}
\input{childdoc.def}
\childdocby{cdocsamp}
%    \end{macrocode}

%\iffalse
%</samplepart3|samplepart4>
%\fi
%
%\iffalse
%<*samplepart3>
%\fi
% Some text for part 3:
%    \begin{macrocode}
some text in part three
%    \end{macrocode}

%\iffalse
%</samplepart3>
%\fi
% Some text for part 4:
%\iffalse
%<*samplepart4>
%\fi
%    \begin{macrocode}
more text in part four
%    \end{macrocode}

%\iffalse
%</samplepart4>
%\fi
%
% %%%%%%%%%%%%%%%%%%%%%%%%%%%%%%%%%%%%%%
% \paragraph{Forwarding for a Complete Draft.}
%
% The following forwarding file |cdocsdrf.tex|
% compiles the main document in draft mode:
%\iffalse
%<*sampledraft>
%\fi
%    \begin{macrocode}
\def\version{draft}
\input{childdoc.def}
\childdocforward{cdocsamp}
%    \end{macrocode}

%\iffalse
%</sampledraft>
%\fi
%
% %%%%%%%%%%%%%%%%%%%%%%%%%%%%%%%%%%%%%%
% \paragraph{Forwarding for Final Version of the Chapters.}
%
% The following forwarding files |cdocsfn1.tex| and |cdocsfn2.tex|
% (with identical content)
% compile the final versions of the child documents
% |cdocsch1.tex| and |cdocsch2.tex|, respectively:
%\iffalse
%<*samplefinal>
%\fi
%    \begin{macrocode}
\def\version{final}
\input{childdoc.def}
\childdocforwardprefix[cdocsamp]{cdocsfn}{cdocsch}
%    \end{macrocode}

%\iffalse
%</samplefinal>
%\fi
%
% %%%%%%%%%%%%%%%%%%%%%%%%%%%%%%%%%%%%%%
% \paragraph{Command Line Processing.}
%
% The following three command lines generate the output files
% |cdocscld|, |cdocscl1| and |cdocscl2|
% which should be identical to
% |cdocsdrf|, |cdocsch1| and |cdocsfn2|, respectively:
% \begin{center}
% \begin{tabular}{l}
% |latex -jobname cdocscld \|\\
% |  "\def\version{draft}\input{childdoc.def}\childdocforward{cdocsamp}"|\\
% |latex -jobname cdocscl1 \|\\
% |  "\input{childdoc.def}\childdocforward[cdocsamp]{cdocsch1}"|\\
% |latex -jobname cdocscl2 \|\\
% |  "\def\version{final}\input{childdoc.def}\childdocforward{cdocsch2}"|
% \end{tabular}
% \end{center}
% Note that the trailing backslash on each first line
% merely continues the input to the second line
% (for convenient cut ant paste).
% Furthermore, the command |latex| can be replaced by any
% of its alternative versions such as |pdflatex|.
%
% %%%%%%%%%%%%%%%%%%%%%%%%%%%%%%%%%%%%%%%%%%%%%%%%%%%%%%%%%%%%%%%%%%%%%%%%%%%%%%
% %%%%%%%%%%%%%%%%%%%%%%%%%%%%%%%%%%%%%%%%%%%%%%%%%%%%%%%%%%%%%%%%%%%%%%%%%%%%%%
% \section{Implementation}
%\iffalse
%<*package>
%\fi
%
% This section describes the definitions file |childdoc.def|.

% The definitions cannot be loaded using |\usepackage| or |\RequirePackage|
% which has a mechanism to prevent loading a style file more than once.
% When loading the definitions by means of |\input|
% multiple instances have to be prevented manually:
%\iffalse
%This code needs to be before the `\ProvidesFile' directive
%which is defined at the beginning of this file.
%Therefore it is also placed there and commented out here.
%</package>
%<*discard>
%\fi
%    \begin{macrocode}
\ifdefined\childdocmain\endinput\fi
%    \end{macrocode}
%\iffalse
%</discard>
%<*package>
%\fi
%
% \macro{\ifchilddoc}
% \macro{\ifchilddocmanual}
% The conditional |\ifchilddoc| tells whether a
% child (true) or main (false) document is being compiled.
% The conditional |\ifchilddocmanual| tells whether
% the |\includeonly| mechanism is used (false) or
% the selection of child files must be performed manually (true).
% The definitions initialise to false:
%    \begin{macrocode}
\newif\ifchilddoc
\newif\ifchilddocmanual
%    \end{macrocode}

% \macro{\childdocname}
% \macro{\childdocjob}
% The macro |\childdocname| stores the name of the main document
% to be compiled. The macro |\childdocjob| stores the name of
% the document on which the \LaTeX{} compiler was originally invoked.
% The content of |\jobname| cannot be compared
% to filenames specified in the source due to different catcodes.
% The following code rescans |\jobname|, stores the result
% in |\childdocname| and saves a copy in |\childdocjob|:
%    \begin{macrocode}
\edef\childdocname{\scantokens\expandafter{\jobname\noexpand}}
\let\childdocjob\childdocname
%    \end{macrocode}

% \macro{\childdocdisable}
% The macro |\childdocdisable| prevents the main file
% from being processed more than once.
% At this stage, the main document command |\childdocmain|
% is assumed to be called once again where it should do nothing.
% Any subsequent call to it should prevent
% a secondary processing of the main document
% It overwrites the forwarding commands
% |\childdocof| and |\childdocforward|
% with empty macros to prevent further inclusions of the main document:
%    \begin{macrocode}
\newcommand{\childdocdisable}
{
  \renewcommand{\childdocmain}[1]{\renewcommand{\childdocmain}[1]{\endinput}}
  \renewcommand{\childdocof}[1]{}
  \renewcommand{\childdocby}[2][]{}
  \renewcommand{\childdocforward}[2][]{}
  \renewcommand{\childdocdisable}{}
}
%    \end{macrocode}

% \macro{\childdocmain}
% The macro |\childdocmain| is to be called at the top of the main file
% with nothing or the main filename (without extension) as argument.
% First, it breaks loops.
% If the argument is not empty and does not match |\childdocname|
% (which is set by the first inclusion of |childdoc.def|),
% |\ifchilddoc| is set to true, |\includeonly| is applied to the child file
% and |\jobname| is set to the main file
% (for proper handling of |.aux| files):
%    \begin{macrocode}
\newcommand{\childdocmain}[1]
{
  \childdocdisable\childdocmain{}
  \if?#1?\else
    \begingroup
      \def\childdoctmp{#1}
      \ifx\childdoctmp\childdocname
        \def\childdoctmp{}
      \else
        \def\childdoctmp
        {
          \childdoctrue
          \includeonly{\childdocname}
          \def\childdocjob{#1}
          \def\jobname{#1}
        }
      \fi
      \expandafter
    \endgroup
    \childdoctmp
  \fi
}
%    \end{macrocode}

% \macro{\childdocof}
% The command |\childdocof| redirects
% compilation to the main file |#1|.
%    \begin{macrocode}
\newcommand{\childdocof}[1]
{
  \childdocdisable
  \childdoctrue
  \includeonly{\childdocname}
  \def\jobname{#1}
  \def\childdocjob{#1}
  \input{#1}
}
%    \end{macrocode}

% \macro{\childdocby}
% The command |\childdocby| ....
%    \begin{macrocode}
\newcommand{\childdocby}[2][]
{
  \childdocdisable
  \childdoctrue
  \childdocmanualtrue
  \if?#1?\else
    \def\jobname{#2}
  \fi
  \def\childdocjob{#2}
  \input{#2}
  \endinput
}
%    \end{macrocode}

% \macro{\childdocforward}
% The command |\childdocforward| redirects
% compilation to the main file or
% (if the optional argument is given) a child file.
% Parameters are set as if the main file
% or a child file starting with |\childdocof| was compiled.
% Then compilation is handed over to the main file:
%    \begin{macrocode}
\newcommand{\childdocforward}[2][]
{
  \begingroup
    \if?#1?
      \def\childdoctmp
      {
        \def\childdocname{#2}
        \def\childdocjob{#2}
        \def\jobname{#2}
        \input{#2}
        \endinput
      }
    \else
      \def\childdoctmp
      {
        \childdocdisable
        \def\childdocname{#2}
        \childdoctrue
        \includeonly{#2}
        \def\childdocjob{#1}
        \def\jobname{#1}
        \input{#1}
        \endinput
      }
    \fi
    \expandafter
  \endgroup
  \childdoctmp
}
%    \end{macrocode}

% \macro{\childdocforwardprefix}
% The command |\childdocforwardprefix| redirects
% compilation to the main or a child file by means of a pattern.
% The prefix |#1| in the current filename is replaced by |#2|
% and the suffix of the current filename is kept
% (it is assumed that the filename does not contain the substring `|~~~|'
% which is used as a delimiter).
% Compilation is handed over to the new file by |\childdocforward|:
%    \begin{macrocode}
\newcommand{\childdocforwardprefix}[3][]
{
  \begingroup
    \def\childdocextract #2##1~~~{\def\childdoctmp{\childdocforward[#1]{#3##1}}}
    \expandafter\childdocextract\childdocname~~~
    \expandafter
  \endgroup
  \childdoctmp
}
%    \end{macrocode}

% \macro{\childdoc}
% The deprecated macro |\childdoc| is a legacy version of |\childdocmain|:
%    \begin{macrocode}
\newcommand{\childdoc}{\childdocmain}
%    \end{macrocode}

% \macro{\childdocredirect}
% The deprecated macro |\childdocredirect| is a legacy version
% of |\childdocforward| and |\childdocforwardprefix|:
%    \begin{macrocode}
\newcommand{\childdocredirect}[2][]
{
  \begingroup
    \if?#1?
      \def\childdoctmp{\childdocforward{#2}}
    \else
      \def\childdoctmp{\childdocforwardprefix{#1}{#2}}
    \fi
    \expandafter
  \endgroup
  \childdoctmp
}
%    \end{macrocode}

%\iffalse
%</package>
%\fi
%
\endinput
|\\
|\childdocby{|\textit{main}|}|\\
\end{tabular}
\end{center}
%
Both forms have slightly different effects as described above.
The main file is prepared as usual, see \secref{sec:include}.

%%%%%%%%%%%%%%%%%%%%%%%%%%%%%%%%%%%%%%%%%%%%%%%%%%%%%%%%%%%%%%%%%%%%%%%%%%%%%%%%
\subsection{Legacy Detection}
\label{sec:detection}

The directive |\childdocmain| in the main file can detect
whether the complete document or merely a child is to be compiled
even without using the directive |\childdocof|.
This method is deprecated because it is less robust
and there is no compelling reason to use it;
it is merely provided for backward compatibility
and it may be removed in future versions.

If the detection mechanism is to be used,
it is mandatory to correctly specify
the filename of the main file as the argument of |\childdocmain|:
%
\begin{center}
\begin{tabular}{l}
|% \iffalse
%
% childdoc.dtx Copyright (C) 2017-2018 Niklas Beisert
%
% This work may be distributed and/or modified under the
% conditions of the LaTeX Project Public License, either version 1.3
% of this license or (at your option) any later version.
% The latest version of this license is in
%   http://www.latex-project.org/lppl.txt
% and version 1.3 or later is part of all distributions of LaTeX
% version 2005/12/01 or later.
%
% This work has the LPPL maintenance status `maintained'.
%
% The Current Maintainer of this work is Niklas Beisert.
%
% This work consists of the files childdoc.dtx and childdoc.ins
% and the derived files childdoc.def and cdocsamp.tex with
% cdocsch1.tex, cdocsch2.tex, cdocsdrf.tex, cdocsfn1.tex, cdocsfn2.tex.
%
%<package>\ifdefined\childdocmain\endinput\fi
%<package>\ProvidesFile{childdoc.def}[2018/12/30 v2.0 child document driver]
%<samplemain>\ProvidesFile{cdocsamp.tex}[2018/12/30 v2.0 sample for childdoc]
%<*driver>
%\ProvidesFile{childdoc.drv}[2018/12/30 v2.0 childdoc reference manual file]
\PassOptionsToClass{10pt,a4paper}{article}
\documentclass{ltxdoc}

\usepackage[margin=35mm]{geometry}
\usepackage{hyperref}
\usepackage{hyperxmp}
\usepackage[usenames]{color}

\hypersetup{colorlinks=true}
\hypersetup{pdfstartview=FitH}
\hypersetup{pdfpagemode=UseNone}
\hypersetup{pdfsource={}}
\hypersetup{pdflang={en-UK}}
\hypersetup{pdfcopyright={Copyright 2017-2018 Niklas Beisert.
  This work may be distributed and/or modified under the
  conditions of the LaTeX Project Public License, either version 1.3
  of this license or (at your option) any later version.}}
\hypersetup{pdflicenseurl={http://www.latex-project.org/lppl.txt}}
\hypersetup{pdfcontactaddress={ETH Zurich, ITP, HIT K,
  Wolfgang-Pauli-Strasse 27}}
\hypersetup{pdfcontactpostcode={8093}}
\hypersetup{pdfcontactcity={Zurich}}
\hypersetup{pdfcontactcountry={Switzerland}}
\hypersetup{pdfcontactemail={nbeisert@itp.phys.ethz.ch}}
\hypersetup{pdfcontacturl={http://people.phys.ethz.ch/\xmptilde nbeisert/}}

\newcommand{\secref}[1]{\hyperref[#1]{section \ref*{#1}}}

\parskip1ex
\parindent0pt
\let\olditemize\itemize
\def\itemize{\olditemize\parskip0pt}

\begin{document}

\title{The \textsf{childdoc} Package}
\hypersetup{pdftitle={The childdoc Package}}
\author{Niklas Beisert\\[2ex]
  Institut f\"ur Theoretische Physik\\
  Eidgen\"ossische Technische Hochschule Z\"urich\\
  Wolfgang-Pauli-Strasse 27, 8093 Z\"urich, Switzerland\\[1ex]
  \href{mailto:nbeisert@itp.phys.ethz.ch}
  {\texttt{nbeisert@itp.phys.ethz.ch}}}
\hypersetup{pdfauthor={Niklas Beisert}}
\hypersetup{pdfsubject={Manual for the LaTeX2e Package childdoc}}
\date{30 December 2018, \textsf{v2.0}}
\maketitle

\begin{abstract}\noindent
\textsf{childdoc} is a \LaTeXe{} package
that enables the direct compilation
of document sections included by |\include|
to individual files.
\end{abstract}

\begingroup
\parskip0ex
\tableofcontents
\endgroup

%%%%%%%%%%%%%%%%%%%%%%%%%%%%%%%%%%%%%%%%%%%%%%%%%%%%%%%%%%%%%%%%%%%%%%%%%%%%%%%%
%%%%%%%%%%%%%%%%%%%%%%%%%%%%%%%%%%%%%%%%%%%%%%%%%%%%%%%%%%%%%%%%%%%%%%%%%%%%%%%%
\section{Introduction}

\LaTeX{} provides a mechanism to structure a large document (such as a book)
into a main file and several child files (containing the chapters)
using the |\include| command.
This mechanism is beneficial for documents
which span hundreds of pages in order to
make the source file(s) more manageable.
Moreover, compilation can be restricted to
selected child files by means of the |\includeonly| command.
The latter feature can be used to reduce the compilation time while editing
(this was significantly more useful in the earlier days of \LaTeX{})
or to generate a smaller document which is easier to navigate.
Another application of |\includeonly| is to generate
documents consisting of selected parts of the complete document.

However, there are a few drawbacks of the plain |\include| mechanism:
\begin{itemize}
\item
The child files cannot be compiled on their own,
they can only be compiled via the main file.
A naive editing environment
(such as a text editor with an option
to have the current file processed by \LaTeX)
may require one to switch to the main file before compiling;
attempting to compile the child file produces errors.
\item
The main file must be modified (each time)
to adjust the |\includeonly| command
to the present needs. This easily leaves the main file in a messy state.
\item
The generated document will always carry the filename
of the main document. This is inconvenient if
several child files are to be compiled and
to be kept for distribution.
\end{itemize}

The present package provides a simple interface
to make child files individually compilable by \LaTeX{}.
Compiling a child file then has the same effect as compiling
the main file with an |\includeonly| command
to select the appropriate child.
Moreover the generated document will carry the name of the child
rather than the main file.
This resolves all three above issues.

This feature is meant to make the editing of books,
thesis documents and lecture notes somewhat more convenient.
However, the package can also be used efficiently for
composing a series of documents (such as exercise sheets)
which are typically distributed individually.
It then assists the author in generating the individual documents
(potentially in different versions)
as well as a document containing the collected series.
Another application is in developing style files
or other kinds of included material
where compilation of the style file could redirect
to a sample or test file.

%%%%%%%%%%%%%%%%%%%%%%%%%%%%%%%%%%%%%%%%%%%%%%%%%%%%%%%%%%%%%%%%%%%%%%%%%%%%%%%%
%%%%%%%%%%%%%%%%%%%%%%%%%%%%%%%%%%%%%%%%%%%%%%%%%%%%%%%%%%%%%%%%%%%%%%%%%%%%%%%%
\section{Usage}

First of all, the package \textsf{childdoc} is \emph{not} a standard
\LaTeXe{} |.sty| style file! Therefore it needs to be invoked in
a non-standard way.

%%%%%%%%%%%%%%%%%%%%%%%%%%%%%%%%%%%%%%%%%%%%%%%%%%%%%%%%%%%%%%%%%%%%%%%%%%%%%%%%
\subsection{Included Files}
\label{sec:include}

%%%%%%%%%%%%%%%%%%%%%%%%%%%%%%%%%%%%%%%%
\DescribeMacro{\childdocmain}
To use the package, add the commands
\begin{center}
\begin{tabular}{l}
|\input{childdoc.def}|\\
|\childdocmain{}|\\
\end{tabular}
\end{center}
at the very top of the main \LaTeX{} file,
in particular \emph{before} the |\documentclass| statement!
The argument of |\childdocmain| should be left empty
(but it must be present).

%%%%%%%%%%%%%%%%%%%%%%%%%%%%%%%%%%%%%%%%
\DescribeMacro{\childdocof}
Furthermore, add the commands
\begin{center}
\begin{tabular}{l}
|\input{childdoc.def}|\\
|\childdocof{|\textit{main}|}|\\
\end{tabular}
\end{center}
at the top of every child file \textit{child}
which is included by |\include{|\textit{child}|}|
from within the main file
(or at least for those files to be compiled individually).
The argument \textit{main} must be the filename of the main file.

There are a couple of
considerations in setting up the main and child documents:

%%%%%%%%%%%%%%%%%%%%%%%%%%%%%%%%%%%%%%%%
\paragraph{Restrictions.}

Please note the following restrictions:
\begin{itemize}
\item
|\childdocmain| must be called with one argument \textit{main}
to ensure compatibility with earlier version of the package.
It must either be empty (|\childdocmain{}|)
or precisely match the filename of the main file in which it is specified.
See \secref{sec:detection} for further information.
\item
The filename \textit{main} must be specified without the |.tex| extension.
\item
The filename \textit{main} is case sensitive
(even in case-insensitive file systems)
due to internal string comparison.
\item
The argument \textit{main} should be fully expanded, it cannot be a macro.
\item
Subdirectories and special characters should be avoided in filenames.
\item
The command |\childdocmain{|\textit{main}|}| must be followed by a whitespace.
It should not be followed immediately by another command
or by a comment mark `|%|'.
This is because the \TeX{} parser reads the token immediately following
the argument of |\childdocmain| and puts it
at the beginning of every child section;
however, a white\-space is ignored.
\end{itemize}

%%%%%%%%%%%%%%%%%%%%%%%%%%%%%%%%%%%%%%%%
\paragraph{Content of Main File.}

It is advisable to place all content in the child files included by |\include|.
Any output contained in the main file will appear in all child documents
unless suppressed manually;
it cannot be suppressed automatically by the |\includeonly| directive
and thus should normally be avoided.
A method to include some content in the main file
by means of conditional processing is described in \secref{sec:conditional}.

%%%%%%%%%%%%%%%%%%%%%%%%%%%%%%%%%%%%%%%%
\paragraph{Page Numbering.}

When only a part of the document is compiled,
the appropriate numbering of pages
(as well as other status parameters)
is determined from the |.aux| files.
The latter contain information from previous passes.
However this information needs to propagate through
all intermediate child documents.
Therefore the page numbering in child documents may well
be inconsistent until the complete document is compiled at least once.

A useful (if unconventional) way to always ensure a consistent
page numbering is to restart the numbering in each child document
and denote the pages by `\textit{child}|.|\textit{page}'
where \textit{child} represents the chapter/section number of the child file.
This can be achieved by the command
|\numberwithin{page}{|\textit{child}|}|
of the \textsf{amsmath} package
where \textit{child} can be |chapter| or |section|
depending on the chosen structuring.
Alternatively, one can modify the macro |\thepage| appropriately
and reset the counter |page| at the start of each child file.

%%%%%%%%%%%%%%%%%%%%%%%%%%%%%%%%%%%%%%%%%%%%%%%%%%%%%%%%%%%%%%%%%%%%%%%%%%%%%%%%
\subsection{Conditional Processing}
\label{sec:conditional}

The package provides a mechanism to compile different versions
of a document. To customise the versions further some conditional processing
can come in handy to distinguish which version is being compiled.
The package provides two macros to describe the compilation context:

%%%%%%%%%%%%%%%%%%%%%%%%%%%%%%%%%%%%%%%%
\DescribeMacro{\ifchilddoc}
The conditional |\ifchilddoc| distinguishes between the compilation of
child documents and the main document:
%
\begin{center}
|\ifchilddoc |\textit{child-code}| |[|\||else |\textit{main-code}]| \||fi|
\end{center}

%%%%%%%%%%%%%%%%%%%%%%%%%%%%%%%%%%%%%%%%
\DescribeMacro{\childdocname}
\DescribeMacro{\childdocjob}
The macro |\childdocname| contains the filename (without extension)
of the main or child file being processed.
Note that |\childdocjob| will always contain the name of the main file.

%%%%%%%%%%%%%%%%%%%%%%%%%%%%%%%%%%%%%%%%
\paragraph{Title Page.}

Conditional processing can be used to include a title or banner page
in the main document when proper precautions are taken.
Importantly, the code in the main file should ensure that the page counter
(as well as other status parameters which are stored in the |.aux| files)
takes the same value after the conditional processing.
Otherwise the page numbers may take divergent values
depending on which part is compiled.

For example, a title page could be declared by:
%
\begin{center}
\begin{tabular}{l}
|\ifchilddoc\||else|\\
|\addtocounter{page}{-1}|\\
\textit{code for title page}\\
|\newpage|\\
|\||fi|
\end{tabular}
\end{center}
%
A banner page for the child documents can be generated by:
%
\begin{center}
\begin{tabular}{l}
|\ifchilddoc|\\
|\addtocounter{page}{-1}|\\
\textit{code for banner page}\\
|\newpage|\\
|\||fi|
\end{tabular}
\end{center}
%
Here one could write a message such as:
\begin{center}
|This is the part \childdocname{} of \childdocjob{}.|
\end{center}

%%%%%%%%%%%%%%%%%%%%%%%%%%%%%%%%%%%%%%%%%%%%%%%%%%%%%%%%%%%%%%%%%%%%%%%%%%%%%%%%
\subsection{Flags}
\label{sec:flags}

The package makes it easy to generate different versions
of the main or child documents.
To this end compilation flags can be defined
and assigned different default values.
They will be particularly useful in conjunction
with the forwarding mechanism described in \secref{sec:forward}.

For example, it may be useful to have a flag |\version|
which can be set to |draft| or |final|.
The document source will contain some conditional code
depending on the value of |\version|.
Suppose further, the flag should default to |final| for the main file
and to |draft| for child files
which is a natural assignment for editing the document.
This is achieved by placing the following code
in the preamble of the main document
(below the |\childdocmain| directive):
%
\begin{center}
\begin{tabular}{l}
|\ifchilddoc|\\
|\providecommand{\version}{draft}|\\
|\||else|\\
|\providecommand{\version}{final}|\\
|\||fi|
\end{tabular}
\end{center}
%
The definition by |\providecommand| makes sure
that previous definitions are not overwritten.
Further statements |\providecommand{\version}{...}|
can thus be added before the above code to override it.

For the main file, one might add a line
(between |\childdocmain| and the above block)
%
\begin{center}
|%\ifchilddoc\||else\providecommand{\version}{draft}\||fi|
\end{center}
%
which can be uncommented to produce a draft version.
Likewise one can add a line to the very top of a child file
(above the |\childdocof{|\textit{main}|}| directive)
%
\begin{center}
|%\providecommand{\version}{final}|
\end{center}
%
which can be uncommented to produce the final version of this child document.

%%%%%%%%%%%%%%%%%%%%%%%%%%%%%%%%%%%%%%%%%%%%%%%%%%%%%%%%%%%%%%%%%%%%%%%%%%%%%%%%
\subsection{Forwarding}
\label{sec:forward}

Different versions of the main or child documents
using compilation flags as described in \secref{sec:flags}
can be (permanently) stored in different files
for convenient compilation, viewing and distribution.
To this end, the package defines a command
to pass on compilation to a different file:

%%%%%%%%%%%%%%%%%%%%%%%%%%%%%%%%%%%%%%%%
\DescribeMacro{\childdocforward}
The command |\childdocforward| redirects processing to
another source file:
%
\begin{center}
\begin{tabular}{l}
|\input{childdoc.def}|\\
|\childdocforward[|\textit{main}|]{|\textit{dest}|}|\\
\end{tabular}
\end{center}
%
The argument \textit{dest} is the destination file
(without extension).
It should be the main file or one of the child files.
Note that further \textsf{childdoc} directives
such as |\childdocof| and |\childdocforward|
in the indicated file will be processed in this form.
The optional argument \textit{main}
passes on directly to the main file \textit{main}
while pretending to compile the child \textit{dest}.
This form behaves as if \textit{dest}
issues |\childdocof{|\textit{main}|}| right away,
and no further \textsf{childdoc} directives will be processed.

%%%%%%%%%%%%%%%%%%%%%%%%%%%%%%%%%%%%%%%%
\DescribeMacro{\...prefix}
In the alternative form |\childdocforwardprefix|,
%
\begin{center}
\begin{tabular}{l}
|\input{childdoc.def}|\\
|\childdocforwardprefix[|\textit{main}|]{|\textit{prefix}|}{|\textit{dest}|}|
\end{tabular}
\end{center}
%
the destination file is determined by a pattern
depending on the current file:
To make this work, the current file must be called
`{\textit{prefix}\hspace{0.2em}\textit{suffix}}'
with \textit{prefix} matching precisely the argument.
Processing is then passed on to the file
`{\textit{dest}\hspace{0.2em}\textit{suffix}}'.
Surely, the same effect is achieved by
directly specifying the
argument `{\textit{dest}\hspace{0.2em}\textit{suffix}}'
in the first form.
However, that requires to set up a different file
for each child. With the alternative form of the command
all these files can have exactly the same content
which simplifies setting them up and maintaining them.

For example, the following file |draft.tex|
with a compilation flag |\version| as described in \secref{sec:flags}
compiles the main document as a draft:
%
\begin{center}
\begin{tabular}{l}
|\def\version{draft}|\\
|\input{childdoc.def}|\\
|\childdocforward{|\textit{main}|}|
\end{tabular}
\end{center}
%
Likewise, the following files |final|\textit{nn}|.tex|
compile the final version of the child document
|child|\textit{nn}|.tex|:
%
\begin{center}
\begin{tabular}{l}
|\def\version{final}|\\
|\input{childdoc.def}|\\
|\childdocforwardprefix{final}{child}|
\end{tabular}
\end{center}
%

Note that when several versions of a main file and/or of each child file
are to be generated, it may be convenient to set up a |Makefile| or
shell script to automatise the process.

%%%%%%%%%%%%%%%%%%%%%%%%%%%%%%%%%%%%%%%%%%%%%%%%%%%%%%%%%%%%%%%%%%%%%%%%%%%%%%%%
\subsection{Command Line Processing}
\label{sec:commandline}

The effect of redirection files can also be achieved by invoking
the \LaTeX{} compiler with a more elaborate command line.
Most conveniently this should be done as part
of a shell script or a |Makefile|.

When using \textsf{childdoc} in the main file, the following
command lines effectively perform a redirection
(note that depending on the shell being used,
backslashes may have to be doubled: `|\|' $\to$ `|\\|'):
%
\begin{center}
|... -jobname "|\textit{target}|" |\\|"|[\textit{flags}]%
|\input{childdoc.def}\childdocforward[|\textit{main}|]{|\textit{dest}|}"|
\end{center}
%
Here \textit{target} is the name of the output file,
\textit{main} is the name of the main file
and \textit{dest} is the name of the main or child file to be processed
(all filenames without extensions).
The optional argument \textit{main} can be omitted
if \textit{main} matches \textit{dest}.
Optionally, compilation \textit{flags} can be defined via |\def| commands.
This command line makes the \TeX{} engine believe
it is compiling the file \textit{target}
whose content is specified as the latter parameter.
The provided code then forwards the processing to
\textit{main} or \textit{dest} as described in \secref{sec:forward}.

%%%%%%%%%%%%%%%%%%%%%%%%%%%%%%%%%%%%%%%%%%%%%%%%%%%%%%%%%%%%%%%%%%%%%%%%%%%%%%%%
\subsection{Include by Input}
\label{sec:input}

Including child documents by |\include| has some restrictions by design.
Most notably, the content of a child document always occupies
its own set of pages; pages cannot be shared between child documents.
Usually, this behaviour makes perfect sense
because each child document contain an essential part of the document.
However, in some situations it may be desirable to compose
a document from a collection of parts
without having mandatory page breaks between then.
For this case, the package
provides a mechanism to include parts
by |\input| which can also be processed individually.
However, by construction this mechanism
requires manual handling of the content to be output.

%%%%%%%%%%%%%%%%%%%%%%%%%%%%%%%%%%%%%%%%
\DescribeMacro{\ifchilddocmanual}
The main file should be prepared as usual, see \secref{sec:include}.
However, the document body must make a distinction
between processing of an individual part and of the main document, e.g.:
%
\begin{center}
\begin{tabular}{l}
|\ifchilddocmanual|\\
|\input{\childdocname}|\\
|\||else|\\
\textit{document body with }|\input{|\textit{part}|}|\\
|\||fi|
\end{tabular}
\end{center}
%
The conditional |\ifchilddocmanual| is true whenever
a part to be included by |\input| is being compiled,
and the name of the part is stored in |\childdocname|.

%%%%%%%%%%%%%%%%%%%%%%%%%%%%%%%%%%%%%%%%
\DescribeMacro{\childdocby}
Each part to be included by |\input| should start with:
%
\begin{center}
\begin{tabular}{l}
|\input{childdoc.def}|\\
|\childdocby{|\textit{main}|}|\\
\end{tabular}
\end{center}
%
The directive |\childdocby| is similar to |\childdocof|
described in \secref{sec:include},
but the subsequent selection of content must be done manually.
To that end, both |\ifchilddoc| and |\ifchilddocmanual|
will be true upon processing of a part,
and the name of the part is stored in |\childdocname|.
Note that |\jobname| will be set to the filename of the current part
so that each part receives an individual |.aux| file
that does not interfere with the |.aux| file(s) of the main document.
This behaviour can be altered by the alternative form
|\childdocby[*]{|\textit{main}|}| (with a non-empty optional argument)
which uses the |.aux| file of the main document
by setting |\jobname| to \textit{main}.

%%%%%%%%%%%%%%%%%%%%%%%%%%%%%%%%%%%%%%%%%%%%%%%%%%%%%%%%%%%%%%%%%%%%%%%%%%%%%%%%
\subsection{Driver Development}
\label{sec:driver}

The \textsf{childdoc} mechanism can also be use for the development
of definition files such as \LaTeX{} styles or classes.
This case differs from the above setup with multiple parts
included by |\include| in that no |\includeonly| should be invoked.
This can be achieved by starting the include file
(before |\ProvidesPackage|) with:
%
\begin{center}
\begin{tabular}{l}
|\input{childdoc.def}|\\
|\childdocforward{|\textit{main}|}|\\
\end{tabular}
\end{center}
%
or alternatively with:
%
\begin{center}
\begin{tabular}{l}
|\input{childdoc.def}|\\
|\childdocby{|\textit{main}|}|\\
\end{tabular}
\end{center}
%
Both forms have slightly different effects as described above.
The main file is prepared as usual, see \secref{sec:include}.

%%%%%%%%%%%%%%%%%%%%%%%%%%%%%%%%%%%%%%%%%%%%%%%%%%%%%%%%%%%%%%%%%%%%%%%%%%%%%%%%
\subsection{Legacy Detection}
\label{sec:detection}

The directive |\childdocmain| in the main file can detect
whether the complete document or merely a child is to be compiled
even without using the directive |\childdocof|.
This method is deprecated because it is less robust
and there is no compelling reason to use it;
it is merely provided for backward compatibility
and it may be removed in future versions.

If the detection mechanism is to be used,
it is mandatory to correctly specify
the filename of the main file as the argument of |\childdocmain|:
%
\begin{center}
\begin{tabular}{l}
|\input{childdoc.def}|\\
|\childdocmain{|\textit{main}|}|\\
\end{tabular}
\end{center}
%
If |\jobname| does not match the argument \textit{main} of |\childdocmain|,
it is assumed that |\jobname| points to the child file to be compiled.
When using |\childdocmain| with the main file specified as argument,
it suffices to start a child file
with just |\input{|\textit{main}|}|
without loading of the package and using |\childdocof|.
If instead all processing is done
with the appropriate \textsf{childdoc} directives,
the argument of \textit{main} of |\childdocmain| can be empty.

An alternative version of the command line processing described
in \secref{sec:commandline} using the detection mechanism reads:
%
\begin{center}
|... -jobname "|\textit{target}|" "|[\textit{flags}]%
[|\def\jobname{|\textit{dest}|}|]|\input{|\textit{main}|}"|
\end{center}

%%%%%%%%%%%%%%%%%%%%%%%%%%%%%%%%%%%%%%%%%%%%%%%%%%%%%%%%%%%%%%%%%%%%%%%%%%%%%%%%
\subsection{Manual Code}
\label{sec:manual}

In case one cannot be certain whether the definitions file |childdoc.def|
is installed on the target \TeX{} distribution
and one prefers not to ship it,
it is conceivable to paste a few relevant commands into the sources.

To that end, drop all statements |\input{childdoc.def}|
and perform the replacements as outlined below.
Instead of |\childdocmain{|\textit{main}|}| add the following code
to the top of the main file:
%
\begin{center}
\begin{tabular}{l}
|\||ifdefined\childdocname\endinput\||fi\newif\ifchilddoc|\\
|\edef\childdocname{\scantokens\expandafter{\jobname\noexpand}}|\\
|\def\childdocmain{|\textit{main}|}\||ifx\childdocmain\childdocname\||else|\\
|\childdoctrue\includeonly{\childdocname}\let\jobname\childdocmain\||fi|\\
\end{tabular}
\end{center}
%
Instead of |\childdocof{|\textit{main}|}| just include the main file
at the top of each child file:
%
\begin{center}
|\input{|\textit{main}|}|
\end{center}
%
A simple redirection |\childdocforward{|\textit{dest}|}| is achieved by:
%
\begin{center}
|\def\jobname{|\textit{dest}|}\input{\jobname}|
\end{center}
%
The redirection with prefix
|\childdocforwardprefix[|\textit{prefix}|]{|\textit{dest}|}|
is accomplished by:
%
\begin{center}
\begin{tabular}{l}
|{\edef\jobname{\scantokens\expandafter{\jobname\noexpand}}|\\
|\def\redirectjob |\textit{prefix}|#1~~~{\gdef\jobname{|\textit{dest}|#1}}|\\
|\expandafter\redirectjob\jobname~~~}\input{\jobname}|
\end{tabular}
\end{center}

In an alternative approach,
child documents can be compiled by a specific command line
without additional code or specific definitions:
%
\begin{center}
|... -jobname "|\textit{target}|" "|[\textit{flags}]%
|\includeonly{|\textit{dest}|}\input{|\textit{main}|}"|
\end{center}
%

%%%%%%%%%%%%%%%%%%%%%%%%%%%%%%%%%%%%%%%%%%%%%%%%%%%%%%%%%%%%%%%%%%%%%%%%%%%%%%%%
%%%%%%%%%%%%%%%%%%%%%%%%%%%%%%%%%%%%%%%%%%%%%%%%%%%%%%%%%%%%%%%%%%%%%%%%%%%%%%%%
\section{Information}

%%%%%%%%%%%%%%%%%%%%%%%%%%%%%%%%%%%%%%%%%%%%%%%%%%%%%%%%%%%%%%%%%%%%%%%%%%%%%%%%
\subsection{Copyright}

Copyright \copyright{} 2017--2018 Niklas Beisert

This work may be distributed and/or modified under the
conditions of the \LaTeX{} Project Public License, either version 1.3
of this license or (at your option) any later version.
The latest version of this license is in
  \url{http://www.latex-project.org/lppl.txt}
and version 1.3 or later is part of all distributions of \LaTeX{}
version 2005/12/01 or later.

This work has the LPPL maintenance status `maintained'.

The Current Maintainer of this work is Niklas Beisert.

This work consists of the files |README.txt|, |childdoc.ins| and |childdoc.dtx|
as well as the derived files |childdoc.def|, |cdocsamp.tex|
with |cdocsch1.tex|, |cdocsch2.tex|, |cdocspt3.tex|, |cdocspt4.tex|,
|cdocsdrf.tex|, |cdocsfn1.tex|, |cdocsfn2.tex|
as well as |childdoc.pdf|.

%%%%%%%%%%%%%%%%%%%%%%%%%%%%%%%%%%%%%%%%%%%%%%%%%%%%%%%%%%%%%%%%%%%%%%%%%%%%%%%%
\subsection{Files and Installation}

The package consists of the files:
%
\begin{center}
\begin{tabular}{ll}
    |README.txt|   & readme file \\
    |childdoc.ins| & installation file \\
    |childdoc.dtx| & source file \\
    |childdoc.def| & definition file \\
    |cdocsamp.tex| & sample main file \\
    |cdocsch1.tex| & sample include file \\
    |cdocsch2.tex| & sample include file \\
    |cdocspt3.tex| & sample part file \\
    |cdocspt4.tex| & sample part file \\
    |cdocsdrf.tex| & sample redirection file \\
    |cdocsfn1.tex| & sample redirection file \\
    |cdocsfn2.tex| & sample redirection file \\
    |childdoc.pdf| & manual
\end{tabular}
\end{center}
%
The distribution consists of the files
|README.txt|, |childdoc.ins| and |childdoc.dtx|.
%
\begin{itemize}
\item
Run (pdf)\LaTeX{} on |childdoc.dtx|
to compile the manual |childdoc.pdf| (this file).
\item
Run \LaTeX{} on |childdoc.ins| to create the definitions file |childdoc.def|
and the sample |cdocsamp.tex| with include files
|cdocsch1.tex|, |cdocsch2.tex|, |cdocspt3.tex|, |cdocspt4.tex|,
|cdocsdrf.tex|, |cdocsfn1.tex|, |cdocsfn2.tex|.
Then copy the file |childdoc.def| to an appropriate directory of your \LaTeX{}
distribution, e.g.\ \textit{texmf-root}|/tex/latex/childdoc|.
\end{itemize}

%%%%%%%%%%%%%%%%%%%%%%%%%%%%%%%%%%%%%%%%%%%%%%%%%%%%%%%%%%%%%%%%%%%%%%%%%%%%%%%%
\subsection{Related CTAN Packages}

There are several other packages which offer a similar functionality:
%
\begin{itemize}
\item
The packages
\href{http://ctan.org/pkg/docmute}{\textsf{docmute}},
\href{http://ctan.org/pkg/includex}{\textsf{includex}} and
\href{http://ctan.org/pkg/standalone}{\textsf{standalone}}
provide commands to include only the document body of
a child file thus allowing both files to be compiled individually.
\item
The packages \href{http://ctan.org/pkg/subdocs}{\textsf{subdocs}}
and \href{http://ctan.org/pkg/subfiles}{\textsf{subfiles}}
provide structures in which the main and child documents can be
encapsulated and allowing them to be compiled individually.
The inclusion mechanism is different from the conventional |\include|.
\item
The package \href{http://ctan.org/pkg/combine}{\textsf{combine}}
is an elaborate solution to combine several documents into one.
\end{itemize}
%
See also the CTAN topic \href{http://ctan.org/topic/subdocs}{\textsf{subdocs}}
for further related packages.
The present package differs from the above solutions in that
a document structure constructed with the conventional |\include| mechanism
just needs two extra commands at the top of every file
such that all constituent files can be compiled individually.

%%%%%%%%%%%%%%%%%%%%%%%%%%%%%%%%%%%%%%%%%%%%%%%%%%%%%%%%%%%%%%%%%%%%%%%%%%%%%%%%
%\subsection{Feature Suggestions}
%
%The following is a list of features which may be useful for future
%versions of this package:
%%
%\begin{itemize}
%\item
%\ldots
%\end{itemize}

%%%%%%%%%%%%%%%%%%%%%%%%%%%%%%%%%%%%%%%%%%%%%%%%%%%%%%%%%%%%%%%%%%%%%%%%%%%%%%%%
\subsection{Revision History}

%%%%%%%%%%%%%%%%%%%%%%%%%%%%%%%%%%%%%%%%
\paragraph{v2.0:} 2018/12/30

\begin{itemize}
\item
immediate forward processing
\item
added |\childdocby| mechanism
\item
manual restructured
\end{itemize}

%%%%%%%%%%%%%%%%%%%%%%%%%%%%%%%%%%%%%%%%
\paragraph{v1.6:} 2018/01/17

\begin{itemize}
\item
application for development of include files
\item
corrections to manual
\end{itemize}

%%%%%%%%%%%%%%%%%%%%%%%%%%%%%%%%%%%%%%%%
\paragraph{v1.5:} 2017/05/21

\begin{itemize}
\item
more complete structuring introduced
\item
|\childdocof| introduced
\item
|\childdoc| renamed to |\childdocmain|
\item
|\childredirect| renamed to |\childdocforward| and |\childdocforwardprefix|
and functionality expanded
\end{itemize}

%%%%%%%%%%%%%%%%%%%%%%%%%%%%%%%%%%%%%%%%
\paragraph{v1.0:} 2017/04/27

\begin{itemize}
\item
manual and install package
\item
first version published on CTAN
\end{itemize}

%%%%%%%%%%%%%%%%%%%%%%%%%%%%%%%%%%%%%%%%
\paragraph{v0.6:} 2017/04/26

\begin{itemize}
\item
redirection mechanism added
\end{itemize}

%%%%%%%%%%%%%%%%%%%%%%%%%%%%%%%%%%%%%%%%
\paragraph{v0.5:} 2017/04/26

\begin{itemize}
\item
functionality in definition file
\end{itemize}


%%%%%%%%%%%%%%%%%%%%%%%%%%%%%%%%%%%%%%%%%%%%%%%%%%%%%%%%%%%%%%%%%%%%%%%%%%%%%%%%
%%%%%%%%%%%%%%%%%%%%%%%%%%%%%%%%%%%%%%%%%%%%%%%%%%%%%%%%%%%%%%%%%%%%%%%%%%%%%%%%
%%%%%%%%%%%%%%%%%%%%%%%%%%%%%%%%%%%%%%%%%%%%%%%%%%%%%%%%%%%%%%%%%%%%%%%%%%%%%%%%
\appendix

\settowidth\MacroIndent{\rmfamily\scriptsize 000\ }

 \DocInput{childdoc.dtx}

\end{document}
%</driver>
% \fi
%
% %%%%%%%%%%%%%%%%%%%%%%%%%%%%%%%%%%%%%%%%%%%%%%%%%%%%%%%%%%%%%%%%%%%%%%%%%%%%%%
% %%%%%%%%%%%%%%%%%%%%%%%%%%%%%%%%%%%%%%%%%%%%%%%%%%%%%%%%%%%%%%%%%%%%%%%%%%%%%%
% \section{Sample}
%\iffalse
%<*samplemain>
%\fi
%
% The following presents a sample document
% with two chapters, two parts, a title page,
% a compile flag as well as three forwarding files to set the flag.
% It consists of eight |.tex| files:
% \begin{center}
% \begin{tabular}{ll}
% |cdocsamp.tex|&main file\\
% |cdocsch1.tex|&include file for chapter 1\\
% |cdocsch2.tex|&include file for chapter 2\\
% |cdocspt3.tex|&include file for part 3\\
% |cdocspt4.tex|&include file for part 4\\
% |cdocsdrf.tex|&forwarding file for main file in draft mode\\
% |cdocsfi1.tex|&forwarding file for final version of chapter 1\\
% |cdocsfi2.tex|&forwarding file for final version of chapter 2\\
% \end{tabular}
% \end{center}
% Each of the eight files can be compiled directly by the \LaTeX{} compiler.
%
% %%%%%%%%%%%%%%%%%%%%%%%%%%%%%%%%%%%%%%
% \paragraph{Main File.}
%
% The main file is called |cdocsamp.tex|.
%
% Load the \textsf{childdoc} definitions and
% declare the filename for the main document:
%    \begin{macrocode}
\input{childdoc.def}
\childdocmain{}
%    \end{macrocode}

% Optional override for |\version| flag:
%    \begin{macrocode}
%%\ifchilddoc\else\providecommand{\version}{draft}\fi
%    \end{macrocode}

% Define the default values for the |\version| flag
% (|final| for the main file and |draft| for childs):
%    \begin{macrocode}
\ifchilddoc
\providecommand{\version}{draft}
\else
\providecommand{\version}{final}
\fi
%    \end{macrocode}

% Load the standard document class:
%    \begin{macrocode}
\documentclass[12pt]{article}
%    \end{macrocode}

% Start the document body:
%    \begin{macrocode}
\begin{document}
%    \end{macrocode}

% Declare a title page.
% Print title, part of document being processed and version flag:
%    \begin{macrocode}
\addtocounter{page}{-1}
\begin{center}
{\LARGE\bfseries{}childdoc example\par}
\vspace{1cm}
\ifchilddoc
\ifchilddocmanual part\else chapter\fi:
`\childdocname' of `\childdocjob'\par
\else
main document: `\childdocjob'\par
\fi
version: \version\par
\end{center}
\newpage
%    \end{macrocode}

% Manually include selected file,
% otherwise process as usual:
%    \begin{macrocode}
\ifchilddocmanual
\section*{part `\childdocname'}
\input{\childdocname}
\else
%    \end{macrocode}

% Include the two chapters:
%    \begin{macrocode}
\include{cdocsch1}
\include{cdocsch2}
%    \end{macrocode}

% Include the two parts unless only chapters should be displayed:
%    \begin{macrocode}
\ifchilddoc\else
\section{part three}
\input{cdocspt3}
\section{part four}
\input{cdocspt4}
\fi
%    \end{macrocode}

% Process as usual until here:
%    \begin{macrocode}
\fi
%    \end{macrocode}

% End of document body:
%    \begin{macrocode}
\end{document}
%    \end{macrocode}
%\iffalse
%</samplemain>
%\fi
%
% %%%%%%%%%%%%%%%%%%%%%%%%%%%%%%%%%%%%%%
% \paragraph{Chapter Include Files.}
%
% The include files are called |cdocsch1.tex| and |cdocsch2.tex|.
%
%\iffalse
%<*samplechap1|samplechap2>
%\fi

% Optional override for |\version| flag:
%    \begin{macrocode}
%%\providecommand{\version}{final}
%    \end{macrocode}

% Include the main document:
%    \begin{macrocode}
\input{childdoc.def}
\childdocof{cdocsamp}
%    \end{macrocode}

%\iffalse
%</samplechap1|samplechap2>
%\fi
%
%\iffalse
%<*samplechap1>
%\fi
% Some text for chapter 1:
%    \begin{macrocode}
\section{one}
some text in chapter one
%    \end{macrocode}

%\iffalse
%</samplechap1>
%\fi
% Some text for chapter 2:
%\iffalse
%<*samplechap2>
%\fi
%    \begin{macrocode}
\section{two}
more text in chapter two
%    \end{macrocode}

%\iffalse
%</samplechap2>
%\fi
%
% %%%%%%%%%%%%%%%%%%%%%%%%%%%%%%%%%%%%%%
% \paragraph{Part Include Files.}
%
% The include files are called |cdocspt3.tex| and |cdocspt4.tex|.
%
%\iffalse
%<*samplepart3|samplepart4>
%\fi

% Optional override for |\version| flag:
%    \begin{macrocode}
%%\providecommand{\version}{final}
%    \end{macrocode}

% Include the main document:
%    \begin{macrocode}
\input{childdoc.def}
\childdocby{cdocsamp}
%    \end{macrocode}

%\iffalse
%</samplepart3|samplepart4>
%\fi
%
%\iffalse
%<*samplepart3>
%\fi
% Some text for part 3:
%    \begin{macrocode}
some text in part three
%    \end{macrocode}

%\iffalse
%</samplepart3>
%\fi
% Some text for part 4:
%\iffalse
%<*samplepart4>
%\fi
%    \begin{macrocode}
more text in part four
%    \end{macrocode}

%\iffalse
%</samplepart4>
%\fi
%
% %%%%%%%%%%%%%%%%%%%%%%%%%%%%%%%%%%%%%%
% \paragraph{Forwarding for a Complete Draft.}
%
% The following forwarding file |cdocsdrf.tex|
% compiles the main document in draft mode:
%\iffalse
%<*sampledraft>
%\fi
%    \begin{macrocode}
\def\version{draft}
\input{childdoc.def}
\childdocforward{cdocsamp}
%    \end{macrocode}

%\iffalse
%</sampledraft>
%\fi
%
% %%%%%%%%%%%%%%%%%%%%%%%%%%%%%%%%%%%%%%
% \paragraph{Forwarding for Final Version of the Chapters.}
%
% The following forwarding files |cdocsfn1.tex| and |cdocsfn2.tex|
% (with identical content)
% compile the final versions of the child documents
% |cdocsch1.tex| and |cdocsch2.tex|, respectively:
%\iffalse
%<*samplefinal>
%\fi
%    \begin{macrocode}
\def\version{final}
\input{childdoc.def}
\childdocforwardprefix[cdocsamp]{cdocsfn}{cdocsch}
%    \end{macrocode}

%\iffalse
%</samplefinal>
%\fi
%
% %%%%%%%%%%%%%%%%%%%%%%%%%%%%%%%%%%%%%%
% \paragraph{Command Line Processing.}
%
% The following three command lines generate the output files
% |cdocscld|, |cdocscl1| and |cdocscl2|
% which should be identical to
% |cdocsdrf|, |cdocsch1| and |cdocsfn2|, respectively:
% \begin{center}
% \begin{tabular}{l}
% |latex -jobname cdocscld \|\\
% |  "\def\version{draft}\input{childdoc.def}\childdocforward{cdocsamp}"|\\
% |latex -jobname cdocscl1 \|\\
% |  "\input{childdoc.def}\childdocforward[cdocsamp]{cdocsch1}"|\\
% |latex -jobname cdocscl2 \|\\
% |  "\def\version{final}\input{childdoc.def}\childdocforward{cdocsch2}"|
% \end{tabular}
% \end{center}
% Note that the trailing backslash on each first line
% merely continues the input to the second line
% (for convenient cut ant paste).
% Furthermore, the command |latex| can be replaced by any
% of its alternative versions such as |pdflatex|.
%
% %%%%%%%%%%%%%%%%%%%%%%%%%%%%%%%%%%%%%%%%%%%%%%%%%%%%%%%%%%%%%%%%%%%%%%%%%%%%%%
% %%%%%%%%%%%%%%%%%%%%%%%%%%%%%%%%%%%%%%%%%%%%%%%%%%%%%%%%%%%%%%%%%%%%%%%%%%%%%%
% \section{Implementation}
%\iffalse
%<*package>
%\fi
%
% This section describes the definitions file |childdoc.def|.

% The definitions cannot be loaded using |\usepackage| or |\RequirePackage|
% which has a mechanism to prevent loading a style file more than once.
% When loading the definitions by means of |\input|
% multiple instances have to be prevented manually:
%\iffalse
%This code needs to be before the `\ProvidesFile' directive
%which is defined at the beginning of this file.
%Therefore it is also placed there and commented out here.
%</package>
%<*discard>
%\fi
%    \begin{macrocode}
\ifdefined\childdocmain\endinput\fi
%    \end{macrocode}
%\iffalse
%</discard>
%<*package>
%\fi
%
% \macro{\ifchilddoc}
% \macro{\ifchilddocmanual}
% The conditional |\ifchilddoc| tells whether a
% child (true) or main (false) document is being compiled.
% The conditional |\ifchilddocmanual| tells whether
% the |\includeonly| mechanism is used (false) or
% the selection of child files must be performed manually (true).
% The definitions initialise to false:
%    \begin{macrocode}
\newif\ifchilddoc
\newif\ifchilddocmanual
%    \end{macrocode}

% \macro{\childdocname}
% \macro{\childdocjob}
% The macro |\childdocname| stores the name of the main document
% to be compiled. The macro |\childdocjob| stores the name of
% the document on which the \LaTeX{} compiler was originally invoked.
% The content of |\jobname| cannot be compared
% to filenames specified in the source due to different catcodes.
% The following code rescans |\jobname|, stores the result
% in |\childdocname| and saves a copy in |\childdocjob|:
%    \begin{macrocode}
\edef\childdocname{\scantokens\expandafter{\jobname\noexpand}}
\let\childdocjob\childdocname
%    \end{macrocode}

% \macro{\childdocdisable}
% The macro |\childdocdisable| prevents the main file
% from being processed more than once.
% At this stage, the main document command |\childdocmain|
% is assumed to be called once again where it should do nothing.
% Any subsequent call to it should prevent
% a secondary processing of the main document
% It overwrites the forwarding commands
% |\childdocof| and |\childdocforward|
% with empty macros to prevent further inclusions of the main document:
%    \begin{macrocode}
\newcommand{\childdocdisable}
{
  \renewcommand{\childdocmain}[1]{\renewcommand{\childdocmain}[1]{\endinput}}
  \renewcommand{\childdocof}[1]{}
  \renewcommand{\childdocby}[2][]{}
  \renewcommand{\childdocforward}[2][]{}
  \renewcommand{\childdocdisable}{}
}
%    \end{macrocode}

% \macro{\childdocmain}
% The macro |\childdocmain| is to be called at the top of the main file
% with nothing or the main filename (without extension) as argument.
% First, it breaks loops.
% If the argument is not empty and does not match |\childdocname|
% (which is set by the first inclusion of |childdoc.def|),
% |\ifchilddoc| is set to true, |\includeonly| is applied to the child file
% and |\jobname| is set to the main file
% (for proper handling of |.aux| files):
%    \begin{macrocode}
\newcommand{\childdocmain}[1]
{
  \childdocdisable\childdocmain{}
  \if?#1?\else
    \begingroup
      \def\childdoctmp{#1}
      \ifx\childdoctmp\childdocname
        \def\childdoctmp{}
      \else
        \def\childdoctmp
        {
          \childdoctrue
          \includeonly{\childdocname}
          \def\childdocjob{#1}
          \def\jobname{#1}
        }
      \fi
      \expandafter
    \endgroup
    \childdoctmp
  \fi
}
%    \end{macrocode}

% \macro{\childdocof}
% The command |\childdocof| redirects
% compilation to the main file |#1|.
%    \begin{macrocode}
\newcommand{\childdocof}[1]
{
  \childdocdisable
  \childdoctrue
  \includeonly{\childdocname}
  \def\jobname{#1}
  \def\childdocjob{#1}
  \input{#1}
}
%    \end{macrocode}

% \macro{\childdocby}
% The command |\childdocby| ....
%    \begin{macrocode}
\newcommand{\childdocby}[2][]
{
  \childdocdisable
  \childdoctrue
  \childdocmanualtrue
  \if?#1?\else
    \def\jobname{#2}
  \fi
  \def\childdocjob{#2}
  \input{#2}
  \endinput
}
%    \end{macrocode}

% \macro{\childdocforward}
% The command |\childdocforward| redirects
% compilation to the main file or
% (if the optional argument is given) a child file.
% Parameters are set as if the main file
% or a child file starting with |\childdocof| was compiled.
% Then compilation is handed over to the main file:
%    \begin{macrocode}
\newcommand{\childdocforward}[2][]
{
  \begingroup
    \if?#1?
      \def\childdoctmp
      {
        \def\childdocname{#2}
        \def\childdocjob{#2}
        \def\jobname{#2}
        \input{#2}
        \endinput
      }
    \else
      \def\childdoctmp
      {
        \childdocdisable
        \def\childdocname{#2}
        \childdoctrue
        \includeonly{#2}
        \def\childdocjob{#1}
        \def\jobname{#1}
        \input{#1}
        \endinput
      }
    \fi
    \expandafter
  \endgroup
  \childdoctmp
}
%    \end{macrocode}

% \macro{\childdocforwardprefix}
% The command |\childdocforwardprefix| redirects
% compilation to the main or a child file by means of a pattern.
% The prefix |#1| in the current filename is replaced by |#2|
% and the suffix of the current filename is kept
% (it is assumed that the filename does not contain the substring `|~~~|'
% which is used as a delimiter).
% Compilation is handed over to the new file by |\childdocforward|:
%    \begin{macrocode}
\newcommand{\childdocforwardprefix}[3][]
{
  \begingroup
    \def\childdocextract #2##1~~~{\def\childdoctmp{\childdocforward[#1]{#3##1}}}
    \expandafter\childdocextract\childdocname~~~
    \expandafter
  \endgroup
  \childdoctmp
}
%    \end{macrocode}

% \macro{\childdoc}
% The deprecated macro |\childdoc| is a legacy version of |\childdocmain|:
%    \begin{macrocode}
\newcommand{\childdoc}{\childdocmain}
%    \end{macrocode}

% \macro{\childdocredirect}
% The deprecated macro |\childdocredirect| is a legacy version
% of |\childdocforward| and |\childdocforwardprefix|:
%    \begin{macrocode}
\newcommand{\childdocredirect}[2][]
{
  \begingroup
    \if?#1?
      \def\childdoctmp{\childdocforward{#2}}
    \else
      \def\childdoctmp{\childdocforwardprefix{#1}{#2}}
    \fi
    \expandafter
  \endgroup
  \childdoctmp
}
%    \end{macrocode}

%\iffalse
%</package>
%\fi
%
\endinput
|\\
|\childdocmain{|\textit{main}|}|\\
\end{tabular}
\end{center}
%
If |\jobname| does not match the argument \textit{main} of |\childdocmain|,
it is assumed that |\jobname| points to the child file to be compiled.
When using |\childdocmain| with the main file specified as argument,
it suffices to start a child file
with just |\input{|\textit{main}|}|
without loading of the package and using |\childdocof|.
If instead all processing is done
with the appropriate \textsf{childdoc} directives,
the argument of \textit{main} of |\childdocmain| can be empty.

An alternative version of the command line processing described
in \secref{sec:commandline} using the detection mechanism reads:
%
\begin{center}
|... -jobname "|\textit{target}|" "|[\textit{flags}]%
[|\def\jobname{|\textit{dest}|}|]|\input{|\textit{main}|}"|
\end{center}

%%%%%%%%%%%%%%%%%%%%%%%%%%%%%%%%%%%%%%%%%%%%%%%%%%%%%%%%%%%%%%%%%%%%%%%%%%%%%%%%
\subsection{Manual Code}
\label{sec:manual}

In case one cannot be certain whether the definitions file |childdoc.def|
is installed on the target \TeX{} distribution
and one prefers not to ship it,
it is conceivable to paste a few relevant commands into the sources.

To that end, drop all statements |% \iffalse
%
% childdoc.dtx Copyright (C) 2017-2018 Niklas Beisert
%
% This work may be distributed and/or modified under the
% conditions of the LaTeX Project Public License, either version 1.3
% of this license or (at your option) any later version.
% The latest version of this license is in
%   http://www.latex-project.org/lppl.txt
% and version 1.3 or later is part of all distributions of LaTeX
% version 2005/12/01 or later.
%
% This work has the LPPL maintenance status `maintained'.
%
% The Current Maintainer of this work is Niklas Beisert.
%
% This work consists of the files childdoc.dtx and childdoc.ins
% and the derived files childdoc.def and cdocsamp.tex with
% cdocsch1.tex, cdocsch2.tex, cdocsdrf.tex, cdocsfn1.tex, cdocsfn2.tex.
%
%<package>\ifdefined\childdocmain\endinput\fi
%<package>\ProvidesFile{childdoc.def}[2018/12/30 v2.0 child document driver]
%<samplemain>\ProvidesFile{cdocsamp.tex}[2018/12/30 v2.0 sample for childdoc]
%<*driver>
%\ProvidesFile{childdoc.drv}[2018/12/30 v2.0 childdoc reference manual file]
\PassOptionsToClass{10pt,a4paper}{article}
\documentclass{ltxdoc}

\usepackage[margin=35mm]{geometry}
\usepackage{hyperref}
\usepackage{hyperxmp}
\usepackage[usenames]{color}

\hypersetup{colorlinks=true}
\hypersetup{pdfstartview=FitH}
\hypersetup{pdfpagemode=UseNone}
\hypersetup{pdfsource={}}
\hypersetup{pdflang={en-UK}}
\hypersetup{pdfcopyright={Copyright 2017-2018 Niklas Beisert.
  This work may be distributed and/or modified under the
  conditions of the LaTeX Project Public License, either version 1.3
  of this license or (at your option) any later version.}}
\hypersetup{pdflicenseurl={http://www.latex-project.org/lppl.txt}}
\hypersetup{pdfcontactaddress={ETH Zurich, ITP, HIT K,
  Wolfgang-Pauli-Strasse 27}}
\hypersetup{pdfcontactpostcode={8093}}
\hypersetup{pdfcontactcity={Zurich}}
\hypersetup{pdfcontactcountry={Switzerland}}
\hypersetup{pdfcontactemail={nbeisert@itp.phys.ethz.ch}}
\hypersetup{pdfcontacturl={http://people.phys.ethz.ch/\xmptilde nbeisert/}}

\newcommand{\secref}[1]{\hyperref[#1]{section \ref*{#1}}}

\parskip1ex
\parindent0pt
\let\olditemize\itemize
\def\itemize{\olditemize\parskip0pt}

\begin{document}

\title{The \textsf{childdoc} Package}
\hypersetup{pdftitle={The childdoc Package}}
\author{Niklas Beisert\\[2ex]
  Institut f\"ur Theoretische Physik\\
  Eidgen\"ossische Technische Hochschule Z\"urich\\
  Wolfgang-Pauli-Strasse 27, 8093 Z\"urich, Switzerland\\[1ex]
  \href{mailto:nbeisert@itp.phys.ethz.ch}
  {\texttt{nbeisert@itp.phys.ethz.ch}}}
\hypersetup{pdfauthor={Niklas Beisert}}
\hypersetup{pdfsubject={Manual for the LaTeX2e Package childdoc}}
\date{30 December 2018, \textsf{v2.0}}
\maketitle

\begin{abstract}\noindent
\textsf{childdoc} is a \LaTeXe{} package
that enables the direct compilation
of document sections included by |\include|
to individual files.
\end{abstract}

\begingroup
\parskip0ex
\tableofcontents
\endgroup

%%%%%%%%%%%%%%%%%%%%%%%%%%%%%%%%%%%%%%%%%%%%%%%%%%%%%%%%%%%%%%%%%%%%%%%%%%%%%%%%
%%%%%%%%%%%%%%%%%%%%%%%%%%%%%%%%%%%%%%%%%%%%%%%%%%%%%%%%%%%%%%%%%%%%%%%%%%%%%%%%
\section{Introduction}

\LaTeX{} provides a mechanism to structure a large document (such as a book)
into a main file and several child files (containing the chapters)
using the |\include| command.
This mechanism is beneficial for documents
which span hundreds of pages in order to
make the source file(s) more manageable.
Moreover, compilation can be restricted to
selected child files by means of the |\includeonly| command.
The latter feature can be used to reduce the compilation time while editing
(this was significantly more useful in the earlier days of \LaTeX{})
or to generate a smaller document which is easier to navigate.
Another application of |\includeonly| is to generate
documents consisting of selected parts of the complete document.

However, there are a few drawbacks of the plain |\include| mechanism:
\begin{itemize}
\item
The child files cannot be compiled on their own,
they can only be compiled via the main file.
A naive editing environment
(such as a text editor with an option
to have the current file processed by \LaTeX)
may require one to switch to the main file before compiling;
attempting to compile the child file produces errors.
\item
The main file must be modified (each time)
to adjust the |\includeonly| command
to the present needs. This easily leaves the main file in a messy state.
\item
The generated document will always carry the filename
of the main document. This is inconvenient if
several child files are to be compiled and
to be kept for distribution.
\end{itemize}

The present package provides a simple interface
to make child files individually compilable by \LaTeX{}.
Compiling a child file then has the same effect as compiling
the main file with an |\includeonly| command
to select the appropriate child.
Moreover the generated document will carry the name of the child
rather than the main file.
This resolves all three above issues.

This feature is meant to make the editing of books,
thesis documents and lecture notes somewhat more convenient.
However, the package can also be used efficiently for
composing a series of documents (such as exercise sheets)
which are typically distributed individually.
It then assists the author in generating the individual documents
(potentially in different versions)
as well as a document containing the collected series.
Another application is in developing style files
or other kinds of included material
where compilation of the style file could redirect
to a sample or test file.

%%%%%%%%%%%%%%%%%%%%%%%%%%%%%%%%%%%%%%%%%%%%%%%%%%%%%%%%%%%%%%%%%%%%%%%%%%%%%%%%
%%%%%%%%%%%%%%%%%%%%%%%%%%%%%%%%%%%%%%%%%%%%%%%%%%%%%%%%%%%%%%%%%%%%%%%%%%%%%%%%
\section{Usage}

First of all, the package \textsf{childdoc} is \emph{not} a standard
\LaTeXe{} |.sty| style file! Therefore it needs to be invoked in
a non-standard way.

%%%%%%%%%%%%%%%%%%%%%%%%%%%%%%%%%%%%%%%%%%%%%%%%%%%%%%%%%%%%%%%%%%%%%%%%%%%%%%%%
\subsection{Included Files}
\label{sec:include}

%%%%%%%%%%%%%%%%%%%%%%%%%%%%%%%%%%%%%%%%
\DescribeMacro{\childdocmain}
To use the package, add the commands
\begin{center}
\begin{tabular}{l}
|\input{childdoc.def}|\\
|\childdocmain{}|\\
\end{tabular}
\end{center}
at the very top of the main \LaTeX{} file,
in particular \emph{before} the |\documentclass| statement!
The argument of |\childdocmain| should be left empty
(but it must be present).

%%%%%%%%%%%%%%%%%%%%%%%%%%%%%%%%%%%%%%%%
\DescribeMacro{\childdocof}
Furthermore, add the commands
\begin{center}
\begin{tabular}{l}
|\input{childdoc.def}|\\
|\childdocof{|\textit{main}|}|\\
\end{tabular}
\end{center}
at the top of every child file \textit{child}
which is included by |\include{|\textit{child}|}|
from within the main file
(or at least for those files to be compiled individually).
The argument \textit{main} must be the filename of the main file.

There are a couple of
considerations in setting up the main and child documents:

%%%%%%%%%%%%%%%%%%%%%%%%%%%%%%%%%%%%%%%%
\paragraph{Restrictions.}

Please note the following restrictions:
\begin{itemize}
\item
|\childdocmain| must be called with one argument \textit{main}
to ensure compatibility with earlier version of the package.
It must either be empty (|\childdocmain{}|)
or precisely match the filename of the main file in which it is specified.
See \secref{sec:detection} for further information.
\item
The filename \textit{main} must be specified without the |.tex| extension.
\item
The filename \textit{main} is case sensitive
(even in case-insensitive file systems)
due to internal string comparison.
\item
The argument \textit{main} should be fully expanded, it cannot be a macro.
\item
Subdirectories and special characters should be avoided in filenames.
\item
The command |\childdocmain{|\textit{main}|}| must be followed by a whitespace.
It should not be followed immediately by another command
or by a comment mark `|%|'.
This is because the \TeX{} parser reads the token immediately following
the argument of |\childdocmain| and puts it
at the beginning of every child section;
however, a white\-space is ignored.
\end{itemize}

%%%%%%%%%%%%%%%%%%%%%%%%%%%%%%%%%%%%%%%%
\paragraph{Content of Main File.}

It is advisable to place all content in the child files included by |\include|.
Any output contained in the main file will appear in all child documents
unless suppressed manually;
it cannot be suppressed automatically by the |\includeonly| directive
and thus should normally be avoided.
A method to include some content in the main file
by means of conditional processing is described in \secref{sec:conditional}.

%%%%%%%%%%%%%%%%%%%%%%%%%%%%%%%%%%%%%%%%
\paragraph{Page Numbering.}

When only a part of the document is compiled,
the appropriate numbering of pages
(as well as other status parameters)
is determined from the |.aux| files.
The latter contain information from previous passes.
However this information needs to propagate through
all intermediate child documents.
Therefore the page numbering in child documents may well
be inconsistent until the complete document is compiled at least once.

A useful (if unconventional) way to always ensure a consistent
page numbering is to restart the numbering in each child document
and denote the pages by `\textit{child}|.|\textit{page}'
where \textit{child} represents the chapter/section number of the child file.
This can be achieved by the command
|\numberwithin{page}{|\textit{child}|}|
of the \textsf{amsmath} package
where \textit{child} can be |chapter| or |section|
depending on the chosen structuring.
Alternatively, one can modify the macro |\thepage| appropriately
and reset the counter |page| at the start of each child file.

%%%%%%%%%%%%%%%%%%%%%%%%%%%%%%%%%%%%%%%%%%%%%%%%%%%%%%%%%%%%%%%%%%%%%%%%%%%%%%%%
\subsection{Conditional Processing}
\label{sec:conditional}

The package provides a mechanism to compile different versions
of a document. To customise the versions further some conditional processing
can come in handy to distinguish which version is being compiled.
The package provides two macros to describe the compilation context:

%%%%%%%%%%%%%%%%%%%%%%%%%%%%%%%%%%%%%%%%
\DescribeMacro{\ifchilddoc}
The conditional |\ifchilddoc| distinguishes between the compilation of
child documents and the main document:
%
\begin{center}
|\ifchilddoc |\textit{child-code}| |[|\||else |\textit{main-code}]| \||fi|
\end{center}

%%%%%%%%%%%%%%%%%%%%%%%%%%%%%%%%%%%%%%%%
\DescribeMacro{\childdocname}
\DescribeMacro{\childdocjob}
The macro |\childdocname| contains the filename (without extension)
of the main or child file being processed.
Note that |\childdocjob| will always contain the name of the main file.

%%%%%%%%%%%%%%%%%%%%%%%%%%%%%%%%%%%%%%%%
\paragraph{Title Page.}

Conditional processing can be used to include a title or banner page
in the main document when proper precautions are taken.
Importantly, the code in the main file should ensure that the page counter
(as well as other status parameters which are stored in the |.aux| files)
takes the same value after the conditional processing.
Otherwise the page numbers may take divergent values
depending on which part is compiled.

For example, a title page could be declared by:
%
\begin{center}
\begin{tabular}{l}
|\ifchilddoc\||else|\\
|\addtocounter{page}{-1}|\\
\textit{code for title page}\\
|\newpage|\\
|\||fi|
\end{tabular}
\end{center}
%
A banner page for the child documents can be generated by:
%
\begin{center}
\begin{tabular}{l}
|\ifchilddoc|\\
|\addtocounter{page}{-1}|\\
\textit{code for banner page}\\
|\newpage|\\
|\||fi|
\end{tabular}
\end{center}
%
Here one could write a message such as:
\begin{center}
|This is the part \childdocname{} of \childdocjob{}.|
\end{center}

%%%%%%%%%%%%%%%%%%%%%%%%%%%%%%%%%%%%%%%%%%%%%%%%%%%%%%%%%%%%%%%%%%%%%%%%%%%%%%%%
\subsection{Flags}
\label{sec:flags}

The package makes it easy to generate different versions
of the main or child documents.
To this end compilation flags can be defined
and assigned different default values.
They will be particularly useful in conjunction
with the forwarding mechanism described in \secref{sec:forward}.

For example, it may be useful to have a flag |\version|
which can be set to |draft| or |final|.
The document source will contain some conditional code
depending on the value of |\version|.
Suppose further, the flag should default to |final| for the main file
and to |draft| for child files
which is a natural assignment for editing the document.
This is achieved by placing the following code
in the preamble of the main document
(below the |\childdocmain| directive):
%
\begin{center}
\begin{tabular}{l}
|\ifchilddoc|\\
|\providecommand{\version}{draft}|\\
|\||else|\\
|\providecommand{\version}{final}|\\
|\||fi|
\end{tabular}
\end{center}
%
The definition by |\providecommand| makes sure
that previous definitions are not overwritten.
Further statements |\providecommand{\version}{...}|
can thus be added before the above code to override it.

For the main file, one might add a line
(between |\childdocmain| and the above block)
%
\begin{center}
|%\ifchilddoc\||else\providecommand{\version}{draft}\||fi|
\end{center}
%
which can be uncommented to produce a draft version.
Likewise one can add a line to the very top of a child file
(above the |\childdocof{|\textit{main}|}| directive)
%
\begin{center}
|%\providecommand{\version}{final}|
\end{center}
%
which can be uncommented to produce the final version of this child document.

%%%%%%%%%%%%%%%%%%%%%%%%%%%%%%%%%%%%%%%%%%%%%%%%%%%%%%%%%%%%%%%%%%%%%%%%%%%%%%%%
\subsection{Forwarding}
\label{sec:forward}

Different versions of the main or child documents
using compilation flags as described in \secref{sec:flags}
can be (permanently) stored in different files
for convenient compilation, viewing and distribution.
To this end, the package defines a command
to pass on compilation to a different file:

%%%%%%%%%%%%%%%%%%%%%%%%%%%%%%%%%%%%%%%%
\DescribeMacro{\childdocforward}
The command |\childdocforward| redirects processing to
another source file:
%
\begin{center}
\begin{tabular}{l}
|\input{childdoc.def}|\\
|\childdocforward[|\textit{main}|]{|\textit{dest}|}|\\
\end{tabular}
\end{center}
%
The argument \textit{dest} is the destination file
(without extension).
It should be the main file or one of the child files.
Note that further \textsf{childdoc} directives
such as |\childdocof| and |\childdocforward|
in the indicated file will be processed in this form.
The optional argument \textit{main}
passes on directly to the main file \textit{main}
while pretending to compile the child \textit{dest}.
This form behaves as if \textit{dest}
issues |\childdocof{|\textit{main}|}| right away,
and no further \textsf{childdoc} directives will be processed.

%%%%%%%%%%%%%%%%%%%%%%%%%%%%%%%%%%%%%%%%
\DescribeMacro{\...prefix}
In the alternative form |\childdocforwardprefix|,
%
\begin{center}
\begin{tabular}{l}
|\input{childdoc.def}|\\
|\childdocforwardprefix[|\textit{main}|]{|\textit{prefix}|}{|\textit{dest}|}|
\end{tabular}
\end{center}
%
the destination file is determined by a pattern
depending on the current file:
To make this work, the current file must be called
`{\textit{prefix}\hspace{0.2em}\textit{suffix}}'
with \textit{prefix} matching precisely the argument.
Processing is then passed on to the file
`{\textit{dest}\hspace{0.2em}\textit{suffix}}'.
Surely, the same effect is achieved by
directly specifying the
argument `{\textit{dest}\hspace{0.2em}\textit{suffix}}'
in the first form.
However, that requires to set up a different file
for each child. With the alternative form of the command
all these files can have exactly the same content
which simplifies setting them up and maintaining them.

For example, the following file |draft.tex|
with a compilation flag |\version| as described in \secref{sec:flags}
compiles the main document as a draft:
%
\begin{center}
\begin{tabular}{l}
|\def\version{draft}|\\
|\input{childdoc.def}|\\
|\childdocforward{|\textit{main}|}|
\end{tabular}
\end{center}
%
Likewise, the following files |final|\textit{nn}|.tex|
compile the final version of the child document
|child|\textit{nn}|.tex|:
%
\begin{center}
\begin{tabular}{l}
|\def\version{final}|\\
|\input{childdoc.def}|\\
|\childdocforwardprefix{final}{child}|
\end{tabular}
\end{center}
%

Note that when several versions of a main file and/or of each child file
are to be generated, it may be convenient to set up a |Makefile| or
shell script to automatise the process.

%%%%%%%%%%%%%%%%%%%%%%%%%%%%%%%%%%%%%%%%%%%%%%%%%%%%%%%%%%%%%%%%%%%%%%%%%%%%%%%%
\subsection{Command Line Processing}
\label{sec:commandline}

The effect of redirection files can also be achieved by invoking
the \LaTeX{} compiler with a more elaborate command line.
Most conveniently this should be done as part
of a shell script or a |Makefile|.

When using \textsf{childdoc} in the main file, the following
command lines effectively perform a redirection
(note that depending on the shell being used,
backslashes may have to be doubled: `|\|' $\to$ `|\\|'):
%
\begin{center}
|... -jobname "|\textit{target}|" |\\|"|[\textit{flags}]%
|\input{childdoc.def}\childdocforward[|\textit{main}|]{|\textit{dest}|}"|
\end{center}
%
Here \textit{target} is the name of the output file,
\textit{main} is the name of the main file
and \textit{dest} is the name of the main or child file to be processed
(all filenames without extensions).
The optional argument \textit{main} can be omitted
if \textit{main} matches \textit{dest}.
Optionally, compilation \textit{flags} can be defined via |\def| commands.
This command line makes the \TeX{} engine believe
it is compiling the file \textit{target}
whose content is specified as the latter parameter.
The provided code then forwards the processing to
\textit{main} or \textit{dest} as described in \secref{sec:forward}.

%%%%%%%%%%%%%%%%%%%%%%%%%%%%%%%%%%%%%%%%%%%%%%%%%%%%%%%%%%%%%%%%%%%%%%%%%%%%%%%%
\subsection{Include by Input}
\label{sec:input}

Including child documents by |\include| has some restrictions by design.
Most notably, the content of a child document always occupies
its own set of pages; pages cannot be shared between child documents.
Usually, this behaviour makes perfect sense
because each child document contain an essential part of the document.
However, in some situations it may be desirable to compose
a document from a collection of parts
without having mandatory page breaks between then.
For this case, the package
provides a mechanism to include parts
by |\input| which can also be processed individually.
However, by construction this mechanism
requires manual handling of the content to be output.

%%%%%%%%%%%%%%%%%%%%%%%%%%%%%%%%%%%%%%%%
\DescribeMacro{\ifchilddocmanual}
The main file should be prepared as usual, see \secref{sec:include}.
However, the document body must make a distinction
between processing of an individual part and of the main document, e.g.:
%
\begin{center}
\begin{tabular}{l}
|\ifchilddocmanual|\\
|\input{\childdocname}|\\
|\||else|\\
\textit{document body with }|\input{|\textit{part}|}|\\
|\||fi|
\end{tabular}
\end{center}
%
The conditional |\ifchilddocmanual| is true whenever
a part to be included by |\input| is being compiled,
and the name of the part is stored in |\childdocname|.

%%%%%%%%%%%%%%%%%%%%%%%%%%%%%%%%%%%%%%%%
\DescribeMacro{\childdocby}
Each part to be included by |\input| should start with:
%
\begin{center}
\begin{tabular}{l}
|\input{childdoc.def}|\\
|\childdocby{|\textit{main}|}|\\
\end{tabular}
\end{center}
%
The directive |\childdocby| is similar to |\childdocof|
described in \secref{sec:include},
but the subsequent selection of content must be done manually.
To that end, both |\ifchilddoc| and |\ifchilddocmanual|
will be true upon processing of a part,
and the name of the part is stored in |\childdocname|.
Note that |\jobname| will be set to the filename of the current part
so that each part receives an individual |.aux| file
that does not interfere with the |.aux| file(s) of the main document.
This behaviour can be altered by the alternative form
|\childdocby[*]{|\textit{main}|}| (with a non-empty optional argument)
which uses the |.aux| file of the main document
by setting |\jobname| to \textit{main}.

%%%%%%%%%%%%%%%%%%%%%%%%%%%%%%%%%%%%%%%%%%%%%%%%%%%%%%%%%%%%%%%%%%%%%%%%%%%%%%%%
\subsection{Driver Development}
\label{sec:driver}

The \textsf{childdoc} mechanism can also be use for the development
of definition files such as \LaTeX{} styles or classes.
This case differs from the above setup with multiple parts
included by |\include| in that no |\includeonly| should be invoked.
This can be achieved by starting the include file
(before |\ProvidesPackage|) with:
%
\begin{center}
\begin{tabular}{l}
|\input{childdoc.def}|\\
|\childdocforward{|\textit{main}|}|\\
\end{tabular}
\end{center}
%
or alternatively with:
%
\begin{center}
\begin{tabular}{l}
|\input{childdoc.def}|\\
|\childdocby{|\textit{main}|}|\\
\end{tabular}
\end{center}
%
Both forms have slightly different effects as described above.
The main file is prepared as usual, see \secref{sec:include}.

%%%%%%%%%%%%%%%%%%%%%%%%%%%%%%%%%%%%%%%%%%%%%%%%%%%%%%%%%%%%%%%%%%%%%%%%%%%%%%%%
\subsection{Legacy Detection}
\label{sec:detection}

The directive |\childdocmain| in the main file can detect
whether the complete document or merely a child is to be compiled
even without using the directive |\childdocof|.
This method is deprecated because it is less robust
and there is no compelling reason to use it;
it is merely provided for backward compatibility
and it may be removed in future versions.

If the detection mechanism is to be used,
it is mandatory to correctly specify
the filename of the main file as the argument of |\childdocmain|:
%
\begin{center}
\begin{tabular}{l}
|\input{childdoc.def}|\\
|\childdocmain{|\textit{main}|}|\\
\end{tabular}
\end{center}
%
If |\jobname| does not match the argument \textit{main} of |\childdocmain|,
it is assumed that |\jobname| points to the child file to be compiled.
When using |\childdocmain| with the main file specified as argument,
it suffices to start a child file
with just |\input{|\textit{main}|}|
without loading of the package and using |\childdocof|.
If instead all processing is done
with the appropriate \textsf{childdoc} directives,
the argument of \textit{main} of |\childdocmain| can be empty.

An alternative version of the command line processing described
in \secref{sec:commandline} using the detection mechanism reads:
%
\begin{center}
|... -jobname "|\textit{target}|" "|[\textit{flags}]%
[|\def\jobname{|\textit{dest}|}|]|\input{|\textit{main}|}"|
\end{center}

%%%%%%%%%%%%%%%%%%%%%%%%%%%%%%%%%%%%%%%%%%%%%%%%%%%%%%%%%%%%%%%%%%%%%%%%%%%%%%%%
\subsection{Manual Code}
\label{sec:manual}

In case one cannot be certain whether the definitions file |childdoc.def|
is installed on the target \TeX{} distribution
and one prefers not to ship it,
it is conceivable to paste a few relevant commands into the sources.

To that end, drop all statements |\input{childdoc.def}|
and perform the replacements as outlined below.
Instead of |\childdocmain{|\textit{main}|}| add the following code
to the top of the main file:
%
\begin{center}
\begin{tabular}{l}
|\||ifdefined\childdocname\endinput\||fi\newif\ifchilddoc|\\
|\edef\childdocname{\scantokens\expandafter{\jobname\noexpand}}|\\
|\def\childdocmain{|\textit{main}|}\||ifx\childdocmain\childdocname\||else|\\
|\childdoctrue\includeonly{\childdocname}\let\jobname\childdocmain\||fi|\\
\end{tabular}
\end{center}
%
Instead of |\childdocof{|\textit{main}|}| just include the main file
at the top of each child file:
%
\begin{center}
|\input{|\textit{main}|}|
\end{center}
%
A simple redirection |\childdocforward{|\textit{dest}|}| is achieved by:
%
\begin{center}
|\def\jobname{|\textit{dest}|}\input{\jobname}|
\end{center}
%
The redirection with prefix
|\childdocforwardprefix[|\textit{prefix}|]{|\textit{dest}|}|
is accomplished by:
%
\begin{center}
\begin{tabular}{l}
|{\edef\jobname{\scantokens\expandafter{\jobname\noexpand}}|\\
|\def\redirectjob |\textit{prefix}|#1~~~{\gdef\jobname{|\textit{dest}|#1}}|\\
|\expandafter\redirectjob\jobname~~~}\input{\jobname}|
\end{tabular}
\end{center}

In an alternative approach,
child documents can be compiled by a specific command line
without additional code or specific definitions:
%
\begin{center}
|... -jobname "|\textit{target}|" "|[\textit{flags}]%
|\includeonly{|\textit{dest}|}\input{|\textit{main}|}"|
\end{center}
%

%%%%%%%%%%%%%%%%%%%%%%%%%%%%%%%%%%%%%%%%%%%%%%%%%%%%%%%%%%%%%%%%%%%%%%%%%%%%%%%%
%%%%%%%%%%%%%%%%%%%%%%%%%%%%%%%%%%%%%%%%%%%%%%%%%%%%%%%%%%%%%%%%%%%%%%%%%%%%%%%%
\section{Information}

%%%%%%%%%%%%%%%%%%%%%%%%%%%%%%%%%%%%%%%%%%%%%%%%%%%%%%%%%%%%%%%%%%%%%%%%%%%%%%%%
\subsection{Copyright}

Copyright \copyright{} 2017--2018 Niklas Beisert

This work may be distributed and/or modified under the
conditions of the \LaTeX{} Project Public License, either version 1.3
of this license or (at your option) any later version.
The latest version of this license is in
  \url{http://www.latex-project.org/lppl.txt}
and version 1.3 or later is part of all distributions of \LaTeX{}
version 2005/12/01 or later.

This work has the LPPL maintenance status `maintained'.

The Current Maintainer of this work is Niklas Beisert.

This work consists of the files |README.txt|, |childdoc.ins| and |childdoc.dtx|
as well as the derived files |childdoc.def|, |cdocsamp.tex|
with |cdocsch1.tex|, |cdocsch2.tex|, |cdocspt3.tex|, |cdocspt4.tex|,
|cdocsdrf.tex|, |cdocsfn1.tex|, |cdocsfn2.tex|
as well as |childdoc.pdf|.

%%%%%%%%%%%%%%%%%%%%%%%%%%%%%%%%%%%%%%%%%%%%%%%%%%%%%%%%%%%%%%%%%%%%%%%%%%%%%%%%
\subsection{Files and Installation}

The package consists of the files:
%
\begin{center}
\begin{tabular}{ll}
    |README.txt|   & readme file \\
    |childdoc.ins| & installation file \\
    |childdoc.dtx| & source file \\
    |childdoc.def| & definition file \\
    |cdocsamp.tex| & sample main file \\
    |cdocsch1.tex| & sample include file \\
    |cdocsch2.tex| & sample include file \\
    |cdocspt3.tex| & sample part file \\
    |cdocspt4.tex| & sample part file \\
    |cdocsdrf.tex| & sample redirection file \\
    |cdocsfn1.tex| & sample redirection file \\
    |cdocsfn2.tex| & sample redirection file \\
    |childdoc.pdf| & manual
\end{tabular}
\end{center}
%
The distribution consists of the files
|README.txt|, |childdoc.ins| and |childdoc.dtx|.
%
\begin{itemize}
\item
Run (pdf)\LaTeX{} on |childdoc.dtx|
to compile the manual |childdoc.pdf| (this file).
\item
Run \LaTeX{} on |childdoc.ins| to create the definitions file |childdoc.def|
and the sample |cdocsamp.tex| with include files
|cdocsch1.tex|, |cdocsch2.tex|, |cdocspt3.tex|, |cdocspt4.tex|,
|cdocsdrf.tex|, |cdocsfn1.tex|, |cdocsfn2.tex|.
Then copy the file |childdoc.def| to an appropriate directory of your \LaTeX{}
distribution, e.g.\ \textit{texmf-root}|/tex/latex/childdoc|.
\end{itemize}

%%%%%%%%%%%%%%%%%%%%%%%%%%%%%%%%%%%%%%%%%%%%%%%%%%%%%%%%%%%%%%%%%%%%%%%%%%%%%%%%
\subsection{Related CTAN Packages}

There are several other packages which offer a similar functionality:
%
\begin{itemize}
\item
The packages
\href{http://ctan.org/pkg/docmute}{\textsf{docmute}},
\href{http://ctan.org/pkg/includex}{\textsf{includex}} and
\href{http://ctan.org/pkg/standalone}{\textsf{standalone}}
provide commands to include only the document body of
a child file thus allowing both files to be compiled individually.
\item
The packages \href{http://ctan.org/pkg/subdocs}{\textsf{subdocs}}
and \href{http://ctan.org/pkg/subfiles}{\textsf{subfiles}}
provide structures in which the main and child documents can be
encapsulated and allowing them to be compiled individually.
The inclusion mechanism is different from the conventional |\include|.
\item
The package \href{http://ctan.org/pkg/combine}{\textsf{combine}}
is an elaborate solution to combine several documents into one.
\end{itemize}
%
See also the CTAN topic \href{http://ctan.org/topic/subdocs}{\textsf{subdocs}}
for further related packages.
The present package differs from the above solutions in that
a document structure constructed with the conventional |\include| mechanism
just needs two extra commands at the top of every file
such that all constituent files can be compiled individually.

%%%%%%%%%%%%%%%%%%%%%%%%%%%%%%%%%%%%%%%%%%%%%%%%%%%%%%%%%%%%%%%%%%%%%%%%%%%%%%%%
%\subsection{Feature Suggestions}
%
%The following is a list of features which may be useful for future
%versions of this package:
%%
%\begin{itemize}
%\item
%\ldots
%\end{itemize}

%%%%%%%%%%%%%%%%%%%%%%%%%%%%%%%%%%%%%%%%%%%%%%%%%%%%%%%%%%%%%%%%%%%%%%%%%%%%%%%%
\subsection{Revision History}

%%%%%%%%%%%%%%%%%%%%%%%%%%%%%%%%%%%%%%%%
\paragraph{v2.0:} 2018/12/30

\begin{itemize}
\item
immediate forward processing
\item
added |\childdocby| mechanism
\item
manual restructured
\end{itemize}

%%%%%%%%%%%%%%%%%%%%%%%%%%%%%%%%%%%%%%%%
\paragraph{v1.6:} 2018/01/17

\begin{itemize}
\item
application for development of include files
\item
corrections to manual
\end{itemize}

%%%%%%%%%%%%%%%%%%%%%%%%%%%%%%%%%%%%%%%%
\paragraph{v1.5:} 2017/05/21

\begin{itemize}
\item
more complete structuring introduced
\item
|\childdocof| introduced
\item
|\childdoc| renamed to |\childdocmain|
\item
|\childredirect| renamed to |\childdocforward| and |\childdocforwardprefix|
and functionality expanded
\end{itemize}

%%%%%%%%%%%%%%%%%%%%%%%%%%%%%%%%%%%%%%%%
\paragraph{v1.0:} 2017/04/27

\begin{itemize}
\item
manual and install package
\item
first version published on CTAN
\end{itemize}

%%%%%%%%%%%%%%%%%%%%%%%%%%%%%%%%%%%%%%%%
\paragraph{v0.6:} 2017/04/26

\begin{itemize}
\item
redirection mechanism added
\end{itemize}

%%%%%%%%%%%%%%%%%%%%%%%%%%%%%%%%%%%%%%%%
\paragraph{v0.5:} 2017/04/26

\begin{itemize}
\item
functionality in definition file
\end{itemize}


%%%%%%%%%%%%%%%%%%%%%%%%%%%%%%%%%%%%%%%%%%%%%%%%%%%%%%%%%%%%%%%%%%%%%%%%%%%%%%%%
%%%%%%%%%%%%%%%%%%%%%%%%%%%%%%%%%%%%%%%%%%%%%%%%%%%%%%%%%%%%%%%%%%%%%%%%%%%%%%%%
%%%%%%%%%%%%%%%%%%%%%%%%%%%%%%%%%%%%%%%%%%%%%%%%%%%%%%%%%%%%%%%%%%%%%%%%%%%%%%%%
\appendix

\settowidth\MacroIndent{\rmfamily\scriptsize 000\ }

 \DocInput{childdoc.dtx}

\end{document}
%</driver>
% \fi
%
% %%%%%%%%%%%%%%%%%%%%%%%%%%%%%%%%%%%%%%%%%%%%%%%%%%%%%%%%%%%%%%%%%%%%%%%%%%%%%%
% %%%%%%%%%%%%%%%%%%%%%%%%%%%%%%%%%%%%%%%%%%%%%%%%%%%%%%%%%%%%%%%%%%%%%%%%%%%%%%
% \section{Sample}
%\iffalse
%<*samplemain>
%\fi
%
% The following presents a sample document
% with two chapters, two parts, a title page,
% a compile flag as well as three forwarding files to set the flag.
% It consists of eight |.tex| files:
% \begin{center}
% \begin{tabular}{ll}
% |cdocsamp.tex|&main file\\
% |cdocsch1.tex|&include file for chapter 1\\
% |cdocsch2.tex|&include file for chapter 2\\
% |cdocspt3.tex|&include file for part 3\\
% |cdocspt4.tex|&include file for part 4\\
% |cdocsdrf.tex|&forwarding file for main file in draft mode\\
% |cdocsfi1.tex|&forwarding file for final version of chapter 1\\
% |cdocsfi2.tex|&forwarding file for final version of chapter 2\\
% \end{tabular}
% \end{center}
% Each of the eight files can be compiled directly by the \LaTeX{} compiler.
%
% %%%%%%%%%%%%%%%%%%%%%%%%%%%%%%%%%%%%%%
% \paragraph{Main File.}
%
% The main file is called |cdocsamp.tex|.
%
% Load the \textsf{childdoc} definitions and
% declare the filename for the main document:
%    \begin{macrocode}
\input{childdoc.def}
\childdocmain{}
%    \end{macrocode}

% Optional override for |\version| flag:
%    \begin{macrocode}
%%\ifchilddoc\else\providecommand{\version}{draft}\fi
%    \end{macrocode}

% Define the default values for the |\version| flag
% (|final| for the main file and |draft| for childs):
%    \begin{macrocode}
\ifchilddoc
\providecommand{\version}{draft}
\else
\providecommand{\version}{final}
\fi
%    \end{macrocode}

% Load the standard document class:
%    \begin{macrocode}
\documentclass[12pt]{article}
%    \end{macrocode}

% Start the document body:
%    \begin{macrocode}
\begin{document}
%    \end{macrocode}

% Declare a title page.
% Print title, part of document being processed and version flag:
%    \begin{macrocode}
\addtocounter{page}{-1}
\begin{center}
{\LARGE\bfseries{}childdoc example\par}
\vspace{1cm}
\ifchilddoc
\ifchilddocmanual part\else chapter\fi:
`\childdocname' of `\childdocjob'\par
\else
main document: `\childdocjob'\par
\fi
version: \version\par
\end{center}
\newpage
%    \end{macrocode}

% Manually include selected file,
% otherwise process as usual:
%    \begin{macrocode}
\ifchilddocmanual
\section*{part `\childdocname'}
\input{\childdocname}
\else
%    \end{macrocode}

% Include the two chapters:
%    \begin{macrocode}
\include{cdocsch1}
\include{cdocsch2}
%    \end{macrocode}

% Include the two parts unless only chapters should be displayed:
%    \begin{macrocode}
\ifchilddoc\else
\section{part three}
\input{cdocspt3}
\section{part four}
\input{cdocspt4}
\fi
%    \end{macrocode}

% Process as usual until here:
%    \begin{macrocode}
\fi
%    \end{macrocode}

% End of document body:
%    \begin{macrocode}
\end{document}
%    \end{macrocode}
%\iffalse
%</samplemain>
%\fi
%
% %%%%%%%%%%%%%%%%%%%%%%%%%%%%%%%%%%%%%%
% \paragraph{Chapter Include Files.}
%
% The include files are called |cdocsch1.tex| and |cdocsch2.tex|.
%
%\iffalse
%<*samplechap1|samplechap2>
%\fi

% Optional override for |\version| flag:
%    \begin{macrocode}
%%\providecommand{\version}{final}
%    \end{macrocode}

% Include the main document:
%    \begin{macrocode}
\input{childdoc.def}
\childdocof{cdocsamp}
%    \end{macrocode}

%\iffalse
%</samplechap1|samplechap2>
%\fi
%
%\iffalse
%<*samplechap1>
%\fi
% Some text for chapter 1:
%    \begin{macrocode}
\section{one}
some text in chapter one
%    \end{macrocode}

%\iffalse
%</samplechap1>
%\fi
% Some text for chapter 2:
%\iffalse
%<*samplechap2>
%\fi
%    \begin{macrocode}
\section{two}
more text in chapter two
%    \end{macrocode}

%\iffalse
%</samplechap2>
%\fi
%
% %%%%%%%%%%%%%%%%%%%%%%%%%%%%%%%%%%%%%%
% \paragraph{Part Include Files.}
%
% The include files are called |cdocspt3.tex| and |cdocspt4.tex|.
%
%\iffalse
%<*samplepart3|samplepart4>
%\fi

% Optional override for |\version| flag:
%    \begin{macrocode}
%%\providecommand{\version}{final}
%    \end{macrocode}

% Include the main document:
%    \begin{macrocode}
\input{childdoc.def}
\childdocby{cdocsamp}
%    \end{macrocode}

%\iffalse
%</samplepart3|samplepart4>
%\fi
%
%\iffalse
%<*samplepart3>
%\fi
% Some text for part 3:
%    \begin{macrocode}
some text in part three
%    \end{macrocode}

%\iffalse
%</samplepart3>
%\fi
% Some text for part 4:
%\iffalse
%<*samplepart4>
%\fi
%    \begin{macrocode}
more text in part four
%    \end{macrocode}

%\iffalse
%</samplepart4>
%\fi
%
% %%%%%%%%%%%%%%%%%%%%%%%%%%%%%%%%%%%%%%
% \paragraph{Forwarding for a Complete Draft.}
%
% The following forwarding file |cdocsdrf.tex|
% compiles the main document in draft mode:
%\iffalse
%<*sampledraft>
%\fi
%    \begin{macrocode}
\def\version{draft}
\input{childdoc.def}
\childdocforward{cdocsamp}
%    \end{macrocode}

%\iffalse
%</sampledraft>
%\fi
%
% %%%%%%%%%%%%%%%%%%%%%%%%%%%%%%%%%%%%%%
% \paragraph{Forwarding for Final Version of the Chapters.}
%
% The following forwarding files |cdocsfn1.tex| and |cdocsfn2.tex|
% (with identical content)
% compile the final versions of the child documents
% |cdocsch1.tex| and |cdocsch2.tex|, respectively:
%\iffalse
%<*samplefinal>
%\fi
%    \begin{macrocode}
\def\version{final}
\input{childdoc.def}
\childdocforwardprefix[cdocsamp]{cdocsfn}{cdocsch}
%    \end{macrocode}

%\iffalse
%</samplefinal>
%\fi
%
% %%%%%%%%%%%%%%%%%%%%%%%%%%%%%%%%%%%%%%
% \paragraph{Command Line Processing.}
%
% The following three command lines generate the output files
% |cdocscld|, |cdocscl1| and |cdocscl2|
% which should be identical to
% |cdocsdrf|, |cdocsch1| and |cdocsfn2|, respectively:
% \begin{center}
% \begin{tabular}{l}
% |latex -jobname cdocscld \|\\
% |  "\def\version{draft}\input{childdoc.def}\childdocforward{cdocsamp}"|\\
% |latex -jobname cdocscl1 \|\\
% |  "\input{childdoc.def}\childdocforward[cdocsamp]{cdocsch1}"|\\
% |latex -jobname cdocscl2 \|\\
% |  "\def\version{final}\input{childdoc.def}\childdocforward{cdocsch2}"|
% \end{tabular}
% \end{center}
% Note that the trailing backslash on each first line
% merely continues the input to the second line
% (for convenient cut ant paste).
% Furthermore, the command |latex| can be replaced by any
% of its alternative versions such as |pdflatex|.
%
% %%%%%%%%%%%%%%%%%%%%%%%%%%%%%%%%%%%%%%%%%%%%%%%%%%%%%%%%%%%%%%%%%%%%%%%%%%%%%%
% %%%%%%%%%%%%%%%%%%%%%%%%%%%%%%%%%%%%%%%%%%%%%%%%%%%%%%%%%%%%%%%%%%%%%%%%%%%%%%
% \section{Implementation}
%\iffalse
%<*package>
%\fi
%
% This section describes the definitions file |childdoc.def|.

% The definitions cannot be loaded using |\usepackage| or |\RequirePackage|
% which has a mechanism to prevent loading a style file more than once.
% When loading the definitions by means of |\input|
% multiple instances have to be prevented manually:
%\iffalse
%This code needs to be before the `\ProvidesFile' directive
%which is defined at the beginning of this file.
%Therefore it is also placed there and commented out here.
%</package>
%<*discard>
%\fi
%    \begin{macrocode}
\ifdefined\childdocmain\endinput\fi
%    \end{macrocode}
%\iffalse
%</discard>
%<*package>
%\fi
%
% \macro{\ifchilddoc}
% \macro{\ifchilddocmanual}
% The conditional |\ifchilddoc| tells whether a
% child (true) or main (false) document is being compiled.
% The conditional |\ifchilddocmanual| tells whether
% the |\includeonly| mechanism is used (false) or
% the selection of child files must be performed manually (true).
% The definitions initialise to false:
%    \begin{macrocode}
\newif\ifchilddoc
\newif\ifchilddocmanual
%    \end{macrocode}

% \macro{\childdocname}
% \macro{\childdocjob}
% The macro |\childdocname| stores the name of the main document
% to be compiled. The macro |\childdocjob| stores the name of
% the document on which the \LaTeX{} compiler was originally invoked.
% The content of |\jobname| cannot be compared
% to filenames specified in the source due to different catcodes.
% The following code rescans |\jobname|, stores the result
% in |\childdocname| and saves a copy in |\childdocjob|:
%    \begin{macrocode}
\edef\childdocname{\scantokens\expandafter{\jobname\noexpand}}
\let\childdocjob\childdocname
%    \end{macrocode}

% \macro{\childdocdisable}
% The macro |\childdocdisable| prevents the main file
% from being processed more than once.
% At this stage, the main document command |\childdocmain|
% is assumed to be called once again where it should do nothing.
% Any subsequent call to it should prevent
% a secondary processing of the main document
% It overwrites the forwarding commands
% |\childdocof| and |\childdocforward|
% with empty macros to prevent further inclusions of the main document:
%    \begin{macrocode}
\newcommand{\childdocdisable}
{
  \renewcommand{\childdocmain}[1]{\renewcommand{\childdocmain}[1]{\endinput}}
  \renewcommand{\childdocof}[1]{}
  \renewcommand{\childdocby}[2][]{}
  \renewcommand{\childdocforward}[2][]{}
  \renewcommand{\childdocdisable}{}
}
%    \end{macrocode}

% \macro{\childdocmain}
% The macro |\childdocmain| is to be called at the top of the main file
% with nothing or the main filename (without extension) as argument.
% First, it breaks loops.
% If the argument is not empty and does not match |\childdocname|
% (which is set by the first inclusion of |childdoc.def|),
% |\ifchilddoc| is set to true, |\includeonly| is applied to the child file
% and |\jobname| is set to the main file
% (for proper handling of |.aux| files):
%    \begin{macrocode}
\newcommand{\childdocmain}[1]
{
  \childdocdisable\childdocmain{}
  \if?#1?\else
    \begingroup
      \def\childdoctmp{#1}
      \ifx\childdoctmp\childdocname
        \def\childdoctmp{}
      \else
        \def\childdoctmp
        {
          \childdoctrue
          \includeonly{\childdocname}
          \def\childdocjob{#1}
          \def\jobname{#1}
        }
      \fi
      \expandafter
    \endgroup
    \childdoctmp
  \fi
}
%    \end{macrocode}

% \macro{\childdocof}
% The command |\childdocof| redirects
% compilation to the main file |#1|.
%    \begin{macrocode}
\newcommand{\childdocof}[1]
{
  \childdocdisable
  \childdoctrue
  \includeonly{\childdocname}
  \def\jobname{#1}
  \def\childdocjob{#1}
  \input{#1}
}
%    \end{macrocode}

% \macro{\childdocby}
% The command |\childdocby| ....
%    \begin{macrocode}
\newcommand{\childdocby}[2][]
{
  \childdocdisable
  \childdoctrue
  \childdocmanualtrue
  \if?#1?\else
    \def\jobname{#2}
  \fi
  \def\childdocjob{#2}
  \input{#2}
  \endinput
}
%    \end{macrocode}

% \macro{\childdocforward}
% The command |\childdocforward| redirects
% compilation to the main file or
% (if the optional argument is given) a child file.
% Parameters are set as if the main file
% or a child file starting with |\childdocof| was compiled.
% Then compilation is handed over to the main file:
%    \begin{macrocode}
\newcommand{\childdocforward}[2][]
{
  \begingroup
    \if?#1?
      \def\childdoctmp
      {
        \def\childdocname{#2}
        \def\childdocjob{#2}
        \def\jobname{#2}
        \input{#2}
        \endinput
      }
    \else
      \def\childdoctmp
      {
        \childdocdisable
        \def\childdocname{#2}
        \childdoctrue
        \includeonly{#2}
        \def\childdocjob{#1}
        \def\jobname{#1}
        \input{#1}
        \endinput
      }
    \fi
    \expandafter
  \endgroup
  \childdoctmp
}
%    \end{macrocode}

% \macro{\childdocforwardprefix}
% The command |\childdocforwardprefix| redirects
% compilation to the main or a child file by means of a pattern.
% The prefix |#1| in the current filename is replaced by |#2|
% and the suffix of the current filename is kept
% (it is assumed that the filename does not contain the substring `|~~~|'
% which is used as a delimiter).
% Compilation is handed over to the new file by |\childdocforward|:
%    \begin{macrocode}
\newcommand{\childdocforwardprefix}[3][]
{
  \begingroup
    \def\childdocextract #2##1~~~{\def\childdoctmp{\childdocforward[#1]{#3##1}}}
    \expandafter\childdocextract\childdocname~~~
    \expandafter
  \endgroup
  \childdoctmp
}
%    \end{macrocode}

% \macro{\childdoc}
% The deprecated macro |\childdoc| is a legacy version of |\childdocmain|:
%    \begin{macrocode}
\newcommand{\childdoc}{\childdocmain}
%    \end{macrocode}

% \macro{\childdocredirect}
% The deprecated macro |\childdocredirect| is a legacy version
% of |\childdocforward| and |\childdocforwardprefix|:
%    \begin{macrocode}
\newcommand{\childdocredirect}[2][]
{
  \begingroup
    \if?#1?
      \def\childdoctmp{\childdocforward{#2}}
    \else
      \def\childdoctmp{\childdocforwardprefix{#1}{#2}}
    \fi
    \expandafter
  \endgroup
  \childdoctmp
}
%    \end{macrocode}

%\iffalse
%</package>
%\fi
%
\endinput
|
and perform the replacements as outlined below.
Instead of |\childdocmain{|\textit{main}|}| add the following code
to the top of the main file:
%
\begin{center}
\begin{tabular}{l}
|\||ifdefined\childdocname\endinput\||fi\newif\ifchilddoc|\\
|\edef\childdocname{\scantokens\expandafter{\jobname\noexpand}}|\\
|\def\childdocmain{|\textit{main}|}\||ifx\childdocmain\childdocname\||else|\\
|\childdoctrue\includeonly{\childdocname}\let\jobname\childdocmain\||fi|\\
\end{tabular}
\end{center}
%
Instead of |\childdocof{|\textit{main}|}| just include the main file
at the top of each child file:
%
\begin{center}
|\input{|\textit{main}|}|
\end{center}
%
A simple redirection |\childdocforward{|\textit{dest}|}| is achieved by:
%
\begin{center}
|\def\jobname{|\textit{dest}|}\input{\jobname}|
\end{center}
%
The redirection with prefix
|\childdocforwardprefix[|\textit{prefix}|]{|\textit{dest}|}|
is accomplished by:
%
\begin{center}
\begin{tabular}{l}
|{\edef\jobname{\scantokens\expandafter{\jobname\noexpand}}|\\
|\def\redirectjob |\textit{prefix}|#1~~~{\gdef\jobname{|\textit{dest}|#1}}|\\
|\expandafter\redirectjob\jobname~~~}\input{\jobname}|
\end{tabular}
\end{center}

In an alternative approach,
child documents can be compiled by a specific command line
without additional code or specific definitions:
%
\begin{center}
|... -jobname "|\textit{target}|" "|[\textit{flags}]%
|\includeonly{|\textit{dest}|}\input{|\textit{main}|}"|
\end{center}
%

%%%%%%%%%%%%%%%%%%%%%%%%%%%%%%%%%%%%%%%%%%%%%%%%%%%%%%%%%%%%%%%%%%%%%%%%%%%%%%%%
%%%%%%%%%%%%%%%%%%%%%%%%%%%%%%%%%%%%%%%%%%%%%%%%%%%%%%%%%%%%%%%%%%%%%%%%%%%%%%%%
\section{Information}

%%%%%%%%%%%%%%%%%%%%%%%%%%%%%%%%%%%%%%%%%%%%%%%%%%%%%%%%%%%%%%%%%%%%%%%%%%%%%%%%
\subsection{Copyright}

Copyright \copyright{} 2017--2018 Niklas Beisert

This work may be distributed and/or modified under the
conditions of the \LaTeX{} Project Public License, either version 1.3
of this license or (at your option) any later version.
The latest version of this license is in
  \url{http://www.latex-project.org/lppl.txt}
and version 1.3 or later is part of all distributions of \LaTeX{}
version 2005/12/01 or later.

This work has the LPPL maintenance status `maintained'.

The Current Maintainer of this work is Niklas Beisert.

This work consists of the files |README.txt|, |childdoc.ins| and |childdoc.dtx|
as well as the derived files |childdoc.def|, |cdocsamp.tex|
with |cdocsch1.tex|, |cdocsch2.tex|, |cdocspt3.tex|, |cdocspt4.tex|,
|cdocsdrf.tex|, |cdocsfn1.tex|, |cdocsfn2.tex|
as well as |childdoc.pdf|.

%%%%%%%%%%%%%%%%%%%%%%%%%%%%%%%%%%%%%%%%%%%%%%%%%%%%%%%%%%%%%%%%%%%%%%%%%%%%%%%%
\subsection{Files and Installation}

The package consists of the files:
%
\begin{center}
\begin{tabular}{ll}
    |README.txt|   & readme file \\
    |childdoc.ins| & installation file \\
    |childdoc.dtx| & source file \\
    |childdoc.def| & definition file \\
    |cdocsamp.tex| & sample main file \\
    |cdocsch1.tex| & sample include file \\
    |cdocsch2.tex| & sample include file \\
    |cdocspt3.tex| & sample part file \\
    |cdocspt4.tex| & sample part file \\
    |cdocsdrf.tex| & sample redirection file \\
    |cdocsfn1.tex| & sample redirection file \\
    |cdocsfn2.tex| & sample redirection file \\
    |childdoc.pdf| & manual
\end{tabular}
\end{center}
%
The distribution consists of the files
|README.txt|, |childdoc.ins| and |childdoc.dtx|.
%
\begin{itemize}
\item
Run (pdf)\LaTeX{} on |childdoc.dtx|
to compile the manual |childdoc.pdf| (this file).
\item
Run \LaTeX{} on |childdoc.ins| to create the definitions file |childdoc.def|
and the sample |cdocsamp.tex| with include files
|cdocsch1.tex|, |cdocsch2.tex|, |cdocspt3.tex|, |cdocspt4.tex|,
|cdocsdrf.tex|, |cdocsfn1.tex|, |cdocsfn2.tex|.
Then copy the file |childdoc.def| to an appropriate directory of your \LaTeX{}
distribution, e.g.\ \textit{texmf-root}|/tex/latex/childdoc|.
\end{itemize}

%%%%%%%%%%%%%%%%%%%%%%%%%%%%%%%%%%%%%%%%%%%%%%%%%%%%%%%%%%%%%%%%%%%%%%%%%%%%%%%%
\subsection{Related CTAN Packages}

There are several other packages which offer a similar functionality:
%
\begin{itemize}
\item
The packages
\href{http://ctan.org/pkg/docmute}{\textsf{docmute}},
\href{http://ctan.org/pkg/includex}{\textsf{includex}} and
\href{http://ctan.org/pkg/standalone}{\textsf{standalone}}
provide commands to include only the document body of
a child file thus allowing both files to be compiled individually.
\item
The packages \href{http://ctan.org/pkg/subdocs}{\textsf{subdocs}}
and \href{http://ctan.org/pkg/subfiles}{\textsf{subfiles}}
provide structures in which the main and child documents can be
encapsulated and allowing them to be compiled individually.
The inclusion mechanism is different from the conventional |\include|.
\item
The package \href{http://ctan.org/pkg/combine}{\textsf{combine}}
is an elaborate solution to combine several documents into one.
\end{itemize}
%
See also the CTAN topic \href{http://ctan.org/topic/subdocs}{\textsf{subdocs}}
for further related packages.
The present package differs from the above solutions in that
a document structure constructed with the conventional |\include| mechanism
just needs two extra commands at the top of every file
such that all constituent files can be compiled individually.

%%%%%%%%%%%%%%%%%%%%%%%%%%%%%%%%%%%%%%%%%%%%%%%%%%%%%%%%%%%%%%%%%%%%%%%%%%%%%%%%
%\subsection{Feature Suggestions}
%
%The following is a list of features which may be useful for future
%versions of this package:
%%
%\begin{itemize}
%\item
%\ldots
%\end{itemize}

%%%%%%%%%%%%%%%%%%%%%%%%%%%%%%%%%%%%%%%%%%%%%%%%%%%%%%%%%%%%%%%%%%%%%%%%%%%%%%%%
\subsection{Revision History}

%%%%%%%%%%%%%%%%%%%%%%%%%%%%%%%%%%%%%%%%
\paragraph{v2.0:} 2018/12/30

\begin{itemize}
\item
immediate forward processing
\item
added |\childdocby| mechanism
\item
manual restructured
\end{itemize}

%%%%%%%%%%%%%%%%%%%%%%%%%%%%%%%%%%%%%%%%
\paragraph{v1.6:} 2018/01/17

\begin{itemize}
\item
application for development of include files
\item
corrections to manual
\end{itemize}

%%%%%%%%%%%%%%%%%%%%%%%%%%%%%%%%%%%%%%%%
\paragraph{v1.5:} 2017/05/21

\begin{itemize}
\item
more complete structuring introduced
\item
|\childdocof| introduced
\item
|\childdoc| renamed to |\childdocmain|
\item
|\childredirect| renamed to |\childdocforward| and |\childdocforwardprefix|
and functionality expanded
\end{itemize}

%%%%%%%%%%%%%%%%%%%%%%%%%%%%%%%%%%%%%%%%
\paragraph{v1.0:} 2017/04/27

\begin{itemize}
\item
manual and install package
\item
first version published on CTAN
\end{itemize}

%%%%%%%%%%%%%%%%%%%%%%%%%%%%%%%%%%%%%%%%
\paragraph{v0.6:} 2017/04/26

\begin{itemize}
\item
redirection mechanism added
\end{itemize}

%%%%%%%%%%%%%%%%%%%%%%%%%%%%%%%%%%%%%%%%
\paragraph{v0.5:} 2017/04/26

\begin{itemize}
\item
functionality in definition file
\end{itemize}


%%%%%%%%%%%%%%%%%%%%%%%%%%%%%%%%%%%%%%%%%%%%%%%%%%%%%%%%%%%%%%%%%%%%%%%%%%%%%%%%
%%%%%%%%%%%%%%%%%%%%%%%%%%%%%%%%%%%%%%%%%%%%%%%%%%%%%%%%%%%%%%%%%%%%%%%%%%%%%%%%
%%%%%%%%%%%%%%%%%%%%%%%%%%%%%%%%%%%%%%%%%%%%%%%%%%%%%%%%%%%%%%%%%%%%%%%%%%%%%%%%
\appendix

\settowidth\MacroIndent{\rmfamily\scriptsize 000\ }

 \DocInput{childdoc.dtx}

\end{document}
%</driver>
% \fi
%
% %%%%%%%%%%%%%%%%%%%%%%%%%%%%%%%%%%%%%%%%%%%%%%%%%%%%%%%%%%%%%%%%%%%%%%%%%%%%%%
% %%%%%%%%%%%%%%%%%%%%%%%%%%%%%%%%%%%%%%%%%%%%%%%%%%%%%%%%%%%%%%%%%%%%%%%%%%%%%%
% \section{Sample}
%\iffalse
%<*samplemain>
%\fi
%
% The following presents a sample document
% with two chapters, two parts, a title page,
% a compile flag as well as three forwarding files to set the flag.
% It consists of eight |.tex| files:
% \begin{center}
% \begin{tabular}{ll}
% |cdocsamp.tex|&main file\\
% |cdocsch1.tex|&include file for chapter 1\\
% |cdocsch2.tex|&include file for chapter 2\\
% |cdocspt3.tex|&include file for part 3\\
% |cdocspt4.tex|&include file for part 4\\
% |cdocsdrf.tex|&forwarding file for main file in draft mode\\
% |cdocsfi1.tex|&forwarding file for final version of chapter 1\\
% |cdocsfi2.tex|&forwarding file for final version of chapter 2\\
% \end{tabular}
% \end{center}
% Each of the eight files can be compiled directly by the \LaTeX{} compiler.
%
% %%%%%%%%%%%%%%%%%%%%%%%%%%%%%%%%%%%%%%
% \paragraph{Main File.}
%
% The main file is called |cdocsamp.tex|.
%
% Load the \textsf{childdoc} definitions and
% declare the filename for the main document:
%    \begin{macrocode}
% \iffalse
%
% childdoc.dtx Copyright (C) 2017-2018 Niklas Beisert
%
% This work may be distributed and/or modified under the
% conditions of the LaTeX Project Public License, either version 1.3
% of this license or (at your option) any later version.
% The latest version of this license is in
%   http://www.latex-project.org/lppl.txt
% and version 1.3 or later is part of all distributions of LaTeX
% version 2005/12/01 or later.
%
% This work has the LPPL maintenance status `maintained'.
%
% The Current Maintainer of this work is Niklas Beisert.
%
% This work consists of the files childdoc.dtx and childdoc.ins
% and the derived files childdoc.def and cdocsamp.tex with
% cdocsch1.tex, cdocsch2.tex, cdocsdrf.tex, cdocsfn1.tex, cdocsfn2.tex.
%
%<package>\ifdefined\childdocmain\endinput\fi
%<package>\ProvidesFile{childdoc.def}[2018/12/30 v2.0 child document driver]
%<samplemain>\ProvidesFile{cdocsamp.tex}[2018/12/30 v2.0 sample for childdoc]
%<*driver>
%\ProvidesFile{childdoc.drv}[2018/12/30 v2.0 childdoc reference manual file]
\PassOptionsToClass{10pt,a4paper}{article}
\documentclass{ltxdoc}

\usepackage[margin=35mm]{geometry}
\usepackage{hyperref}
\usepackage{hyperxmp}
\usepackage[usenames]{color}

\hypersetup{colorlinks=true}
\hypersetup{pdfstartview=FitH}
\hypersetup{pdfpagemode=UseNone}
\hypersetup{pdfsource={}}
\hypersetup{pdflang={en-UK}}
\hypersetup{pdfcopyright={Copyright 2017-2018 Niklas Beisert.
  This work may be distributed and/or modified under the
  conditions of the LaTeX Project Public License, either version 1.3
  of this license or (at your option) any later version.}}
\hypersetup{pdflicenseurl={http://www.latex-project.org/lppl.txt}}
\hypersetup{pdfcontactaddress={ETH Zurich, ITP, HIT K,
  Wolfgang-Pauli-Strasse 27}}
\hypersetup{pdfcontactpostcode={8093}}
\hypersetup{pdfcontactcity={Zurich}}
\hypersetup{pdfcontactcountry={Switzerland}}
\hypersetup{pdfcontactemail={nbeisert@itp.phys.ethz.ch}}
\hypersetup{pdfcontacturl={http://people.phys.ethz.ch/\xmptilde nbeisert/}}

\newcommand{\secref}[1]{\hyperref[#1]{section \ref*{#1}}}

\parskip1ex
\parindent0pt
\let\olditemize\itemize
\def\itemize{\olditemize\parskip0pt}

\begin{document}

\title{The \textsf{childdoc} Package}
\hypersetup{pdftitle={The childdoc Package}}
\author{Niklas Beisert\\[2ex]
  Institut f\"ur Theoretische Physik\\
  Eidgen\"ossische Technische Hochschule Z\"urich\\
  Wolfgang-Pauli-Strasse 27, 8093 Z\"urich, Switzerland\\[1ex]
  \href{mailto:nbeisert@itp.phys.ethz.ch}
  {\texttt{nbeisert@itp.phys.ethz.ch}}}
\hypersetup{pdfauthor={Niklas Beisert}}
\hypersetup{pdfsubject={Manual for the LaTeX2e Package childdoc}}
\date{30 December 2018, \textsf{v2.0}}
\maketitle

\begin{abstract}\noindent
\textsf{childdoc} is a \LaTeXe{} package
that enables the direct compilation
of document sections included by |\include|
to individual files.
\end{abstract}

\begingroup
\parskip0ex
\tableofcontents
\endgroup

%%%%%%%%%%%%%%%%%%%%%%%%%%%%%%%%%%%%%%%%%%%%%%%%%%%%%%%%%%%%%%%%%%%%%%%%%%%%%%%%
%%%%%%%%%%%%%%%%%%%%%%%%%%%%%%%%%%%%%%%%%%%%%%%%%%%%%%%%%%%%%%%%%%%%%%%%%%%%%%%%
\section{Introduction}

\LaTeX{} provides a mechanism to structure a large document (such as a book)
into a main file and several child files (containing the chapters)
using the |\include| command.
This mechanism is beneficial for documents
which span hundreds of pages in order to
make the source file(s) more manageable.
Moreover, compilation can be restricted to
selected child files by means of the |\includeonly| command.
The latter feature can be used to reduce the compilation time while editing
(this was significantly more useful in the earlier days of \LaTeX{})
or to generate a smaller document which is easier to navigate.
Another application of |\includeonly| is to generate
documents consisting of selected parts of the complete document.

However, there are a few drawbacks of the plain |\include| mechanism:
\begin{itemize}
\item
The child files cannot be compiled on their own,
they can only be compiled via the main file.
A naive editing environment
(such as a text editor with an option
to have the current file processed by \LaTeX)
may require one to switch to the main file before compiling;
attempting to compile the child file produces errors.
\item
The main file must be modified (each time)
to adjust the |\includeonly| command
to the present needs. This easily leaves the main file in a messy state.
\item
The generated document will always carry the filename
of the main document. This is inconvenient if
several child files are to be compiled and
to be kept for distribution.
\end{itemize}

The present package provides a simple interface
to make child files individually compilable by \LaTeX{}.
Compiling a child file then has the same effect as compiling
the main file with an |\includeonly| command
to select the appropriate child.
Moreover the generated document will carry the name of the child
rather than the main file.
This resolves all three above issues.

This feature is meant to make the editing of books,
thesis documents and lecture notes somewhat more convenient.
However, the package can also be used efficiently for
composing a series of documents (such as exercise sheets)
which are typically distributed individually.
It then assists the author in generating the individual documents
(potentially in different versions)
as well as a document containing the collected series.
Another application is in developing style files
or other kinds of included material
where compilation of the style file could redirect
to a sample or test file.

%%%%%%%%%%%%%%%%%%%%%%%%%%%%%%%%%%%%%%%%%%%%%%%%%%%%%%%%%%%%%%%%%%%%%%%%%%%%%%%%
%%%%%%%%%%%%%%%%%%%%%%%%%%%%%%%%%%%%%%%%%%%%%%%%%%%%%%%%%%%%%%%%%%%%%%%%%%%%%%%%
\section{Usage}

First of all, the package \textsf{childdoc} is \emph{not} a standard
\LaTeXe{} |.sty| style file! Therefore it needs to be invoked in
a non-standard way.

%%%%%%%%%%%%%%%%%%%%%%%%%%%%%%%%%%%%%%%%%%%%%%%%%%%%%%%%%%%%%%%%%%%%%%%%%%%%%%%%
\subsection{Included Files}
\label{sec:include}

%%%%%%%%%%%%%%%%%%%%%%%%%%%%%%%%%%%%%%%%
\DescribeMacro{\childdocmain}
To use the package, add the commands
\begin{center}
\begin{tabular}{l}
|\input{childdoc.def}|\\
|\childdocmain{}|\\
\end{tabular}
\end{center}
at the very top of the main \LaTeX{} file,
in particular \emph{before} the |\documentclass| statement!
The argument of |\childdocmain| should be left empty
(but it must be present).

%%%%%%%%%%%%%%%%%%%%%%%%%%%%%%%%%%%%%%%%
\DescribeMacro{\childdocof}
Furthermore, add the commands
\begin{center}
\begin{tabular}{l}
|\input{childdoc.def}|\\
|\childdocof{|\textit{main}|}|\\
\end{tabular}
\end{center}
at the top of every child file \textit{child}
which is included by |\include{|\textit{child}|}|
from within the main file
(or at least for those files to be compiled individually).
The argument \textit{main} must be the filename of the main file.

There are a couple of
considerations in setting up the main and child documents:

%%%%%%%%%%%%%%%%%%%%%%%%%%%%%%%%%%%%%%%%
\paragraph{Restrictions.}

Please note the following restrictions:
\begin{itemize}
\item
|\childdocmain| must be called with one argument \textit{main}
to ensure compatibility with earlier version of the package.
It must either be empty (|\childdocmain{}|)
or precisely match the filename of the main file in which it is specified.
See \secref{sec:detection} for further information.
\item
The filename \textit{main} must be specified without the |.tex| extension.
\item
The filename \textit{main} is case sensitive
(even in case-insensitive file systems)
due to internal string comparison.
\item
The argument \textit{main} should be fully expanded, it cannot be a macro.
\item
Subdirectories and special characters should be avoided in filenames.
\item
The command |\childdocmain{|\textit{main}|}| must be followed by a whitespace.
It should not be followed immediately by another command
or by a comment mark `|%|'.
This is because the \TeX{} parser reads the token immediately following
the argument of |\childdocmain| and puts it
at the beginning of every child section;
however, a white\-space is ignored.
\end{itemize}

%%%%%%%%%%%%%%%%%%%%%%%%%%%%%%%%%%%%%%%%
\paragraph{Content of Main File.}

It is advisable to place all content in the child files included by |\include|.
Any output contained in the main file will appear in all child documents
unless suppressed manually;
it cannot be suppressed automatically by the |\includeonly| directive
and thus should normally be avoided.
A method to include some content in the main file
by means of conditional processing is described in \secref{sec:conditional}.

%%%%%%%%%%%%%%%%%%%%%%%%%%%%%%%%%%%%%%%%
\paragraph{Page Numbering.}

When only a part of the document is compiled,
the appropriate numbering of pages
(as well as other status parameters)
is determined from the |.aux| files.
The latter contain information from previous passes.
However this information needs to propagate through
all intermediate child documents.
Therefore the page numbering in child documents may well
be inconsistent until the complete document is compiled at least once.

A useful (if unconventional) way to always ensure a consistent
page numbering is to restart the numbering in each child document
and denote the pages by `\textit{child}|.|\textit{page}'
where \textit{child} represents the chapter/section number of the child file.
This can be achieved by the command
|\numberwithin{page}{|\textit{child}|}|
of the \textsf{amsmath} package
where \textit{child} can be |chapter| or |section|
depending on the chosen structuring.
Alternatively, one can modify the macro |\thepage| appropriately
and reset the counter |page| at the start of each child file.

%%%%%%%%%%%%%%%%%%%%%%%%%%%%%%%%%%%%%%%%%%%%%%%%%%%%%%%%%%%%%%%%%%%%%%%%%%%%%%%%
\subsection{Conditional Processing}
\label{sec:conditional}

The package provides a mechanism to compile different versions
of a document. To customise the versions further some conditional processing
can come in handy to distinguish which version is being compiled.
The package provides two macros to describe the compilation context:

%%%%%%%%%%%%%%%%%%%%%%%%%%%%%%%%%%%%%%%%
\DescribeMacro{\ifchilddoc}
The conditional |\ifchilddoc| distinguishes between the compilation of
child documents and the main document:
%
\begin{center}
|\ifchilddoc |\textit{child-code}| |[|\||else |\textit{main-code}]| \||fi|
\end{center}

%%%%%%%%%%%%%%%%%%%%%%%%%%%%%%%%%%%%%%%%
\DescribeMacro{\childdocname}
\DescribeMacro{\childdocjob}
The macro |\childdocname| contains the filename (without extension)
of the main or child file being processed.
Note that |\childdocjob| will always contain the name of the main file.

%%%%%%%%%%%%%%%%%%%%%%%%%%%%%%%%%%%%%%%%
\paragraph{Title Page.}

Conditional processing can be used to include a title or banner page
in the main document when proper precautions are taken.
Importantly, the code in the main file should ensure that the page counter
(as well as other status parameters which are stored in the |.aux| files)
takes the same value after the conditional processing.
Otherwise the page numbers may take divergent values
depending on which part is compiled.

For example, a title page could be declared by:
%
\begin{center}
\begin{tabular}{l}
|\ifchilddoc\||else|\\
|\addtocounter{page}{-1}|\\
\textit{code for title page}\\
|\newpage|\\
|\||fi|
\end{tabular}
\end{center}
%
A banner page for the child documents can be generated by:
%
\begin{center}
\begin{tabular}{l}
|\ifchilddoc|\\
|\addtocounter{page}{-1}|\\
\textit{code for banner page}\\
|\newpage|\\
|\||fi|
\end{tabular}
\end{center}
%
Here one could write a message such as:
\begin{center}
|This is the part \childdocname{} of \childdocjob{}.|
\end{center}

%%%%%%%%%%%%%%%%%%%%%%%%%%%%%%%%%%%%%%%%%%%%%%%%%%%%%%%%%%%%%%%%%%%%%%%%%%%%%%%%
\subsection{Flags}
\label{sec:flags}

The package makes it easy to generate different versions
of the main or child documents.
To this end compilation flags can be defined
and assigned different default values.
They will be particularly useful in conjunction
with the forwarding mechanism described in \secref{sec:forward}.

For example, it may be useful to have a flag |\version|
which can be set to |draft| or |final|.
The document source will contain some conditional code
depending on the value of |\version|.
Suppose further, the flag should default to |final| for the main file
and to |draft| for child files
which is a natural assignment for editing the document.
This is achieved by placing the following code
in the preamble of the main document
(below the |\childdocmain| directive):
%
\begin{center}
\begin{tabular}{l}
|\ifchilddoc|\\
|\providecommand{\version}{draft}|\\
|\||else|\\
|\providecommand{\version}{final}|\\
|\||fi|
\end{tabular}
\end{center}
%
The definition by |\providecommand| makes sure
that previous definitions are not overwritten.
Further statements |\providecommand{\version}{...}|
can thus be added before the above code to override it.

For the main file, one might add a line
(between |\childdocmain| and the above block)
%
\begin{center}
|%\ifchilddoc\||else\providecommand{\version}{draft}\||fi|
\end{center}
%
which can be uncommented to produce a draft version.
Likewise one can add a line to the very top of a child file
(above the |\childdocof{|\textit{main}|}| directive)
%
\begin{center}
|%\providecommand{\version}{final}|
\end{center}
%
which can be uncommented to produce the final version of this child document.

%%%%%%%%%%%%%%%%%%%%%%%%%%%%%%%%%%%%%%%%%%%%%%%%%%%%%%%%%%%%%%%%%%%%%%%%%%%%%%%%
\subsection{Forwarding}
\label{sec:forward}

Different versions of the main or child documents
using compilation flags as described in \secref{sec:flags}
can be (permanently) stored in different files
for convenient compilation, viewing and distribution.
To this end, the package defines a command
to pass on compilation to a different file:

%%%%%%%%%%%%%%%%%%%%%%%%%%%%%%%%%%%%%%%%
\DescribeMacro{\childdocforward}
The command |\childdocforward| redirects processing to
another source file:
%
\begin{center}
\begin{tabular}{l}
|\input{childdoc.def}|\\
|\childdocforward[|\textit{main}|]{|\textit{dest}|}|\\
\end{tabular}
\end{center}
%
The argument \textit{dest} is the destination file
(without extension).
It should be the main file or one of the child files.
Note that further \textsf{childdoc} directives
such as |\childdocof| and |\childdocforward|
in the indicated file will be processed in this form.
The optional argument \textit{main}
passes on directly to the main file \textit{main}
while pretending to compile the child \textit{dest}.
This form behaves as if \textit{dest}
issues |\childdocof{|\textit{main}|}| right away,
and no further \textsf{childdoc} directives will be processed.

%%%%%%%%%%%%%%%%%%%%%%%%%%%%%%%%%%%%%%%%
\DescribeMacro{\...prefix}
In the alternative form |\childdocforwardprefix|,
%
\begin{center}
\begin{tabular}{l}
|\input{childdoc.def}|\\
|\childdocforwardprefix[|\textit{main}|]{|\textit{prefix}|}{|\textit{dest}|}|
\end{tabular}
\end{center}
%
the destination file is determined by a pattern
depending on the current file:
To make this work, the current file must be called
`{\textit{prefix}\hspace{0.2em}\textit{suffix}}'
with \textit{prefix} matching precisely the argument.
Processing is then passed on to the file
`{\textit{dest}\hspace{0.2em}\textit{suffix}}'.
Surely, the same effect is achieved by
directly specifying the
argument `{\textit{dest}\hspace{0.2em}\textit{suffix}}'
in the first form.
However, that requires to set up a different file
for each child. With the alternative form of the command
all these files can have exactly the same content
which simplifies setting them up and maintaining them.

For example, the following file |draft.tex|
with a compilation flag |\version| as described in \secref{sec:flags}
compiles the main document as a draft:
%
\begin{center}
\begin{tabular}{l}
|\def\version{draft}|\\
|\input{childdoc.def}|\\
|\childdocforward{|\textit{main}|}|
\end{tabular}
\end{center}
%
Likewise, the following files |final|\textit{nn}|.tex|
compile the final version of the child document
|child|\textit{nn}|.tex|:
%
\begin{center}
\begin{tabular}{l}
|\def\version{final}|\\
|\input{childdoc.def}|\\
|\childdocforwardprefix{final}{child}|
\end{tabular}
\end{center}
%

Note that when several versions of a main file and/or of each child file
are to be generated, it may be convenient to set up a |Makefile| or
shell script to automatise the process.

%%%%%%%%%%%%%%%%%%%%%%%%%%%%%%%%%%%%%%%%%%%%%%%%%%%%%%%%%%%%%%%%%%%%%%%%%%%%%%%%
\subsection{Command Line Processing}
\label{sec:commandline}

The effect of redirection files can also be achieved by invoking
the \LaTeX{} compiler with a more elaborate command line.
Most conveniently this should be done as part
of a shell script or a |Makefile|.

When using \textsf{childdoc} in the main file, the following
command lines effectively perform a redirection
(note that depending on the shell being used,
backslashes may have to be doubled: `|\|' $\to$ `|\\|'):
%
\begin{center}
|... -jobname "|\textit{target}|" |\\|"|[\textit{flags}]%
|\input{childdoc.def}\childdocforward[|\textit{main}|]{|\textit{dest}|}"|
\end{center}
%
Here \textit{target} is the name of the output file,
\textit{main} is the name of the main file
and \textit{dest} is the name of the main or child file to be processed
(all filenames without extensions).
The optional argument \textit{main} can be omitted
if \textit{main} matches \textit{dest}.
Optionally, compilation \textit{flags} can be defined via |\def| commands.
This command line makes the \TeX{} engine believe
it is compiling the file \textit{target}
whose content is specified as the latter parameter.
The provided code then forwards the processing to
\textit{main} or \textit{dest} as described in \secref{sec:forward}.

%%%%%%%%%%%%%%%%%%%%%%%%%%%%%%%%%%%%%%%%%%%%%%%%%%%%%%%%%%%%%%%%%%%%%%%%%%%%%%%%
\subsection{Include by Input}
\label{sec:input}

Including child documents by |\include| has some restrictions by design.
Most notably, the content of a child document always occupies
its own set of pages; pages cannot be shared between child documents.
Usually, this behaviour makes perfect sense
because each child document contain an essential part of the document.
However, in some situations it may be desirable to compose
a document from a collection of parts
without having mandatory page breaks between then.
For this case, the package
provides a mechanism to include parts
by |\input| which can also be processed individually.
However, by construction this mechanism
requires manual handling of the content to be output.

%%%%%%%%%%%%%%%%%%%%%%%%%%%%%%%%%%%%%%%%
\DescribeMacro{\ifchilddocmanual}
The main file should be prepared as usual, see \secref{sec:include}.
However, the document body must make a distinction
between processing of an individual part and of the main document, e.g.:
%
\begin{center}
\begin{tabular}{l}
|\ifchilddocmanual|\\
|\input{\childdocname}|\\
|\||else|\\
\textit{document body with }|\input{|\textit{part}|}|\\
|\||fi|
\end{tabular}
\end{center}
%
The conditional |\ifchilddocmanual| is true whenever
a part to be included by |\input| is being compiled,
and the name of the part is stored in |\childdocname|.

%%%%%%%%%%%%%%%%%%%%%%%%%%%%%%%%%%%%%%%%
\DescribeMacro{\childdocby}
Each part to be included by |\input| should start with:
%
\begin{center}
\begin{tabular}{l}
|\input{childdoc.def}|\\
|\childdocby{|\textit{main}|}|\\
\end{tabular}
\end{center}
%
The directive |\childdocby| is similar to |\childdocof|
described in \secref{sec:include},
but the subsequent selection of content must be done manually.
To that end, both |\ifchilddoc| and |\ifchilddocmanual|
will be true upon processing of a part,
and the name of the part is stored in |\childdocname|.
Note that |\jobname| will be set to the filename of the current part
so that each part receives an individual |.aux| file
that does not interfere with the |.aux| file(s) of the main document.
This behaviour can be altered by the alternative form
|\childdocby[*]{|\textit{main}|}| (with a non-empty optional argument)
which uses the |.aux| file of the main document
by setting |\jobname| to \textit{main}.

%%%%%%%%%%%%%%%%%%%%%%%%%%%%%%%%%%%%%%%%%%%%%%%%%%%%%%%%%%%%%%%%%%%%%%%%%%%%%%%%
\subsection{Driver Development}
\label{sec:driver}

The \textsf{childdoc} mechanism can also be use for the development
of definition files such as \LaTeX{} styles or classes.
This case differs from the above setup with multiple parts
included by |\include| in that no |\includeonly| should be invoked.
This can be achieved by starting the include file
(before |\ProvidesPackage|) with:
%
\begin{center}
\begin{tabular}{l}
|\input{childdoc.def}|\\
|\childdocforward{|\textit{main}|}|\\
\end{tabular}
\end{center}
%
or alternatively with:
%
\begin{center}
\begin{tabular}{l}
|\input{childdoc.def}|\\
|\childdocby{|\textit{main}|}|\\
\end{tabular}
\end{center}
%
Both forms have slightly different effects as described above.
The main file is prepared as usual, see \secref{sec:include}.

%%%%%%%%%%%%%%%%%%%%%%%%%%%%%%%%%%%%%%%%%%%%%%%%%%%%%%%%%%%%%%%%%%%%%%%%%%%%%%%%
\subsection{Legacy Detection}
\label{sec:detection}

The directive |\childdocmain| in the main file can detect
whether the complete document or merely a child is to be compiled
even without using the directive |\childdocof|.
This method is deprecated because it is less robust
and there is no compelling reason to use it;
it is merely provided for backward compatibility
and it may be removed in future versions.

If the detection mechanism is to be used,
it is mandatory to correctly specify
the filename of the main file as the argument of |\childdocmain|:
%
\begin{center}
\begin{tabular}{l}
|\input{childdoc.def}|\\
|\childdocmain{|\textit{main}|}|\\
\end{tabular}
\end{center}
%
If |\jobname| does not match the argument \textit{main} of |\childdocmain|,
it is assumed that |\jobname| points to the child file to be compiled.
When using |\childdocmain| with the main file specified as argument,
it suffices to start a child file
with just |\input{|\textit{main}|}|
without loading of the package and using |\childdocof|.
If instead all processing is done
with the appropriate \textsf{childdoc} directives,
the argument of \textit{main} of |\childdocmain| can be empty.

An alternative version of the command line processing described
in \secref{sec:commandline} using the detection mechanism reads:
%
\begin{center}
|... -jobname "|\textit{target}|" "|[\textit{flags}]%
[|\def\jobname{|\textit{dest}|}|]|\input{|\textit{main}|}"|
\end{center}

%%%%%%%%%%%%%%%%%%%%%%%%%%%%%%%%%%%%%%%%%%%%%%%%%%%%%%%%%%%%%%%%%%%%%%%%%%%%%%%%
\subsection{Manual Code}
\label{sec:manual}

In case one cannot be certain whether the definitions file |childdoc.def|
is installed on the target \TeX{} distribution
and one prefers not to ship it,
it is conceivable to paste a few relevant commands into the sources.

To that end, drop all statements |\input{childdoc.def}|
and perform the replacements as outlined below.
Instead of |\childdocmain{|\textit{main}|}| add the following code
to the top of the main file:
%
\begin{center}
\begin{tabular}{l}
|\||ifdefined\childdocname\endinput\||fi\newif\ifchilddoc|\\
|\edef\childdocname{\scantokens\expandafter{\jobname\noexpand}}|\\
|\def\childdocmain{|\textit{main}|}\||ifx\childdocmain\childdocname\||else|\\
|\childdoctrue\includeonly{\childdocname}\let\jobname\childdocmain\||fi|\\
\end{tabular}
\end{center}
%
Instead of |\childdocof{|\textit{main}|}| just include the main file
at the top of each child file:
%
\begin{center}
|\input{|\textit{main}|}|
\end{center}
%
A simple redirection |\childdocforward{|\textit{dest}|}| is achieved by:
%
\begin{center}
|\def\jobname{|\textit{dest}|}\input{\jobname}|
\end{center}
%
The redirection with prefix
|\childdocforwardprefix[|\textit{prefix}|]{|\textit{dest}|}|
is accomplished by:
%
\begin{center}
\begin{tabular}{l}
|{\edef\jobname{\scantokens\expandafter{\jobname\noexpand}}|\\
|\def\redirectjob |\textit{prefix}|#1~~~{\gdef\jobname{|\textit{dest}|#1}}|\\
|\expandafter\redirectjob\jobname~~~}\input{\jobname}|
\end{tabular}
\end{center}

In an alternative approach,
child documents can be compiled by a specific command line
without additional code or specific definitions:
%
\begin{center}
|... -jobname "|\textit{target}|" "|[\textit{flags}]%
|\includeonly{|\textit{dest}|}\input{|\textit{main}|}"|
\end{center}
%

%%%%%%%%%%%%%%%%%%%%%%%%%%%%%%%%%%%%%%%%%%%%%%%%%%%%%%%%%%%%%%%%%%%%%%%%%%%%%%%%
%%%%%%%%%%%%%%%%%%%%%%%%%%%%%%%%%%%%%%%%%%%%%%%%%%%%%%%%%%%%%%%%%%%%%%%%%%%%%%%%
\section{Information}

%%%%%%%%%%%%%%%%%%%%%%%%%%%%%%%%%%%%%%%%%%%%%%%%%%%%%%%%%%%%%%%%%%%%%%%%%%%%%%%%
\subsection{Copyright}

Copyright \copyright{} 2017--2018 Niklas Beisert

This work may be distributed and/or modified under the
conditions of the \LaTeX{} Project Public License, either version 1.3
of this license or (at your option) any later version.
The latest version of this license is in
  \url{http://www.latex-project.org/lppl.txt}
and version 1.3 or later is part of all distributions of \LaTeX{}
version 2005/12/01 or later.

This work has the LPPL maintenance status `maintained'.

The Current Maintainer of this work is Niklas Beisert.

This work consists of the files |README.txt|, |childdoc.ins| and |childdoc.dtx|
as well as the derived files |childdoc.def|, |cdocsamp.tex|
with |cdocsch1.tex|, |cdocsch2.tex|, |cdocspt3.tex|, |cdocspt4.tex|,
|cdocsdrf.tex|, |cdocsfn1.tex|, |cdocsfn2.tex|
as well as |childdoc.pdf|.

%%%%%%%%%%%%%%%%%%%%%%%%%%%%%%%%%%%%%%%%%%%%%%%%%%%%%%%%%%%%%%%%%%%%%%%%%%%%%%%%
\subsection{Files and Installation}

The package consists of the files:
%
\begin{center}
\begin{tabular}{ll}
    |README.txt|   & readme file \\
    |childdoc.ins| & installation file \\
    |childdoc.dtx| & source file \\
    |childdoc.def| & definition file \\
    |cdocsamp.tex| & sample main file \\
    |cdocsch1.tex| & sample include file \\
    |cdocsch2.tex| & sample include file \\
    |cdocspt3.tex| & sample part file \\
    |cdocspt4.tex| & sample part file \\
    |cdocsdrf.tex| & sample redirection file \\
    |cdocsfn1.tex| & sample redirection file \\
    |cdocsfn2.tex| & sample redirection file \\
    |childdoc.pdf| & manual
\end{tabular}
\end{center}
%
The distribution consists of the files
|README.txt|, |childdoc.ins| and |childdoc.dtx|.
%
\begin{itemize}
\item
Run (pdf)\LaTeX{} on |childdoc.dtx|
to compile the manual |childdoc.pdf| (this file).
\item
Run \LaTeX{} on |childdoc.ins| to create the definitions file |childdoc.def|
and the sample |cdocsamp.tex| with include files
|cdocsch1.tex|, |cdocsch2.tex|, |cdocspt3.tex|, |cdocspt4.tex|,
|cdocsdrf.tex|, |cdocsfn1.tex|, |cdocsfn2.tex|.
Then copy the file |childdoc.def| to an appropriate directory of your \LaTeX{}
distribution, e.g.\ \textit{texmf-root}|/tex/latex/childdoc|.
\end{itemize}

%%%%%%%%%%%%%%%%%%%%%%%%%%%%%%%%%%%%%%%%%%%%%%%%%%%%%%%%%%%%%%%%%%%%%%%%%%%%%%%%
\subsection{Related CTAN Packages}

There are several other packages which offer a similar functionality:
%
\begin{itemize}
\item
The packages
\href{http://ctan.org/pkg/docmute}{\textsf{docmute}},
\href{http://ctan.org/pkg/includex}{\textsf{includex}} and
\href{http://ctan.org/pkg/standalone}{\textsf{standalone}}
provide commands to include only the document body of
a child file thus allowing both files to be compiled individually.
\item
The packages \href{http://ctan.org/pkg/subdocs}{\textsf{subdocs}}
and \href{http://ctan.org/pkg/subfiles}{\textsf{subfiles}}
provide structures in which the main and child documents can be
encapsulated and allowing them to be compiled individually.
The inclusion mechanism is different from the conventional |\include|.
\item
The package \href{http://ctan.org/pkg/combine}{\textsf{combine}}
is an elaborate solution to combine several documents into one.
\end{itemize}
%
See also the CTAN topic \href{http://ctan.org/topic/subdocs}{\textsf{subdocs}}
for further related packages.
The present package differs from the above solutions in that
a document structure constructed with the conventional |\include| mechanism
just needs two extra commands at the top of every file
such that all constituent files can be compiled individually.

%%%%%%%%%%%%%%%%%%%%%%%%%%%%%%%%%%%%%%%%%%%%%%%%%%%%%%%%%%%%%%%%%%%%%%%%%%%%%%%%
%\subsection{Feature Suggestions}
%
%The following is a list of features which may be useful for future
%versions of this package:
%%
%\begin{itemize}
%\item
%\ldots
%\end{itemize}

%%%%%%%%%%%%%%%%%%%%%%%%%%%%%%%%%%%%%%%%%%%%%%%%%%%%%%%%%%%%%%%%%%%%%%%%%%%%%%%%
\subsection{Revision History}

%%%%%%%%%%%%%%%%%%%%%%%%%%%%%%%%%%%%%%%%
\paragraph{v2.0:} 2018/12/30

\begin{itemize}
\item
immediate forward processing
\item
added |\childdocby| mechanism
\item
manual restructured
\end{itemize}

%%%%%%%%%%%%%%%%%%%%%%%%%%%%%%%%%%%%%%%%
\paragraph{v1.6:} 2018/01/17

\begin{itemize}
\item
application for development of include files
\item
corrections to manual
\end{itemize}

%%%%%%%%%%%%%%%%%%%%%%%%%%%%%%%%%%%%%%%%
\paragraph{v1.5:} 2017/05/21

\begin{itemize}
\item
more complete structuring introduced
\item
|\childdocof| introduced
\item
|\childdoc| renamed to |\childdocmain|
\item
|\childredirect| renamed to |\childdocforward| and |\childdocforwardprefix|
and functionality expanded
\end{itemize}

%%%%%%%%%%%%%%%%%%%%%%%%%%%%%%%%%%%%%%%%
\paragraph{v1.0:} 2017/04/27

\begin{itemize}
\item
manual and install package
\item
first version published on CTAN
\end{itemize}

%%%%%%%%%%%%%%%%%%%%%%%%%%%%%%%%%%%%%%%%
\paragraph{v0.6:} 2017/04/26

\begin{itemize}
\item
redirection mechanism added
\end{itemize}

%%%%%%%%%%%%%%%%%%%%%%%%%%%%%%%%%%%%%%%%
\paragraph{v0.5:} 2017/04/26

\begin{itemize}
\item
functionality in definition file
\end{itemize}


%%%%%%%%%%%%%%%%%%%%%%%%%%%%%%%%%%%%%%%%%%%%%%%%%%%%%%%%%%%%%%%%%%%%%%%%%%%%%%%%
%%%%%%%%%%%%%%%%%%%%%%%%%%%%%%%%%%%%%%%%%%%%%%%%%%%%%%%%%%%%%%%%%%%%%%%%%%%%%%%%
%%%%%%%%%%%%%%%%%%%%%%%%%%%%%%%%%%%%%%%%%%%%%%%%%%%%%%%%%%%%%%%%%%%%%%%%%%%%%%%%
\appendix

\settowidth\MacroIndent{\rmfamily\scriptsize 000\ }

 \DocInput{childdoc.dtx}

\end{document}
%</driver>
% \fi
%
% %%%%%%%%%%%%%%%%%%%%%%%%%%%%%%%%%%%%%%%%%%%%%%%%%%%%%%%%%%%%%%%%%%%%%%%%%%%%%%
% %%%%%%%%%%%%%%%%%%%%%%%%%%%%%%%%%%%%%%%%%%%%%%%%%%%%%%%%%%%%%%%%%%%%%%%%%%%%%%
% \section{Sample}
%\iffalse
%<*samplemain>
%\fi
%
% The following presents a sample document
% with two chapters, two parts, a title page,
% a compile flag as well as three forwarding files to set the flag.
% It consists of eight |.tex| files:
% \begin{center}
% \begin{tabular}{ll}
% |cdocsamp.tex|&main file\\
% |cdocsch1.tex|&include file for chapter 1\\
% |cdocsch2.tex|&include file for chapter 2\\
% |cdocspt3.tex|&include file for part 3\\
% |cdocspt4.tex|&include file for part 4\\
% |cdocsdrf.tex|&forwarding file for main file in draft mode\\
% |cdocsfi1.tex|&forwarding file for final version of chapter 1\\
% |cdocsfi2.tex|&forwarding file for final version of chapter 2\\
% \end{tabular}
% \end{center}
% Each of the eight files can be compiled directly by the \LaTeX{} compiler.
%
% %%%%%%%%%%%%%%%%%%%%%%%%%%%%%%%%%%%%%%
% \paragraph{Main File.}
%
% The main file is called |cdocsamp.tex|.
%
% Load the \textsf{childdoc} definitions and
% declare the filename for the main document:
%    \begin{macrocode}
\input{childdoc.def}
\childdocmain{}
%    \end{macrocode}

% Optional override for |\version| flag:
%    \begin{macrocode}
%%\ifchilddoc\else\providecommand{\version}{draft}\fi
%    \end{macrocode}

% Define the default values for the |\version| flag
% (|final| for the main file and |draft| for childs):
%    \begin{macrocode}
\ifchilddoc
\providecommand{\version}{draft}
\else
\providecommand{\version}{final}
\fi
%    \end{macrocode}

% Load the standard document class:
%    \begin{macrocode}
\documentclass[12pt]{article}
%    \end{macrocode}

% Start the document body:
%    \begin{macrocode}
\begin{document}
%    \end{macrocode}

% Declare a title page.
% Print title, part of document being processed and version flag:
%    \begin{macrocode}
\addtocounter{page}{-1}
\begin{center}
{\LARGE\bfseries{}childdoc example\par}
\vspace{1cm}
\ifchilddoc
\ifchilddocmanual part\else chapter\fi:
`\childdocname' of `\childdocjob'\par
\else
main document: `\childdocjob'\par
\fi
version: \version\par
\end{center}
\newpage
%    \end{macrocode}

% Manually include selected file,
% otherwise process as usual:
%    \begin{macrocode}
\ifchilddocmanual
\section*{part `\childdocname'}
\input{\childdocname}
\else
%    \end{macrocode}

% Include the two chapters:
%    \begin{macrocode}
\include{cdocsch1}
\include{cdocsch2}
%    \end{macrocode}

% Include the two parts unless only chapters should be displayed:
%    \begin{macrocode}
\ifchilddoc\else
\section{part three}
\input{cdocspt3}
\section{part four}
\input{cdocspt4}
\fi
%    \end{macrocode}

% Process as usual until here:
%    \begin{macrocode}
\fi
%    \end{macrocode}

% End of document body:
%    \begin{macrocode}
\end{document}
%    \end{macrocode}
%\iffalse
%</samplemain>
%\fi
%
% %%%%%%%%%%%%%%%%%%%%%%%%%%%%%%%%%%%%%%
% \paragraph{Chapter Include Files.}
%
% The include files are called |cdocsch1.tex| and |cdocsch2.tex|.
%
%\iffalse
%<*samplechap1|samplechap2>
%\fi

% Optional override for |\version| flag:
%    \begin{macrocode}
%%\providecommand{\version}{final}
%    \end{macrocode}

% Include the main document:
%    \begin{macrocode}
\input{childdoc.def}
\childdocof{cdocsamp}
%    \end{macrocode}

%\iffalse
%</samplechap1|samplechap2>
%\fi
%
%\iffalse
%<*samplechap1>
%\fi
% Some text for chapter 1:
%    \begin{macrocode}
\section{one}
some text in chapter one
%    \end{macrocode}

%\iffalse
%</samplechap1>
%\fi
% Some text for chapter 2:
%\iffalse
%<*samplechap2>
%\fi
%    \begin{macrocode}
\section{two}
more text in chapter two
%    \end{macrocode}

%\iffalse
%</samplechap2>
%\fi
%
% %%%%%%%%%%%%%%%%%%%%%%%%%%%%%%%%%%%%%%
% \paragraph{Part Include Files.}
%
% The include files are called |cdocspt3.tex| and |cdocspt4.tex|.
%
%\iffalse
%<*samplepart3|samplepart4>
%\fi

% Optional override for |\version| flag:
%    \begin{macrocode}
%%\providecommand{\version}{final}
%    \end{macrocode}

% Include the main document:
%    \begin{macrocode}
\input{childdoc.def}
\childdocby{cdocsamp}
%    \end{macrocode}

%\iffalse
%</samplepart3|samplepart4>
%\fi
%
%\iffalse
%<*samplepart3>
%\fi
% Some text for part 3:
%    \begin{macrocode}
some text in part three
%    \end{macrocode}

%\iffalse
%</samplepart3>
%\fi
% Some text for part 4:
%\iffalse
%<*samplepart4>
%\fi
%    \begin{macrocode}
more text in part four
%    \end{macrocode}

%\iffalse
%</samplepart4>
%\fi
%
% %%%%%%%%%%%%%%%%%%%%%%%%%%%%%%%%%%%%%%
% \paragraph{Forwarding for a Complete Draft.}
%
% The following forwarding file |cdocsdrf.tex|
% compiles the main document in draft mode:
%\iffalse
%<*sampledraft>
%\fi
%    \begin{macrocode}
\def\version{draft}
\input{childdoc.def}
\childdocforward{cdocsamp}
%    \end{macrocode}

%\iffalse
%</sampledraft>
%\fi
%
% %%%%%%%%%%%%%%%%%%%%%%%%%%%%%%%%%%%%%%
% \paragraph{Forwarding for Final Version of the Chapters.}
%
% The following forwarding files |cdocsfn1.tex| and |cdocsfn2.tex|
% (with identical content)
% compile the final versions of the child documents
% |cdocsch1.tex| and |cdocsch2.tex|, respectively:
%\iffalse
%<*samplefinal>
%\fi
%    \begin{macrocode}
\def\version{final}
\input{childdoc.def}
\childdocforwardprefix[cdocsamp]{cdocsfn}{cdocsch}
%    \end{macrocode}

%\iffalse
%</samplefinal>
%\fi
%
% %%%%%%%%%%%%%%%%%%%%%%%%%%%%%%%%%%%%%%
% \paragraph{Command Line Processing.}
%
% The following three command lines generate the output files
% |cdocscld|, |cdocscl1| and |cdocscl2|
% which should be identical to
% |cdocsdrf|, |cdocsch1| and |cdocsfn2|, respectively:
% \begin{center}
% \begin{tabular}{l}
% |latex -jobname cdocscld \|\\
% |  "\def\version{draft}\input{childdoc.def}\childdocforward{cdocsamp}"|\\
% |latex -jobname cdocscl1 \|\\
% |  "\input{childdoc.def}\childdocforward[cdocsamp]{cdocsch1}"|\\
% |latex -jobname cdocscl2 \|\\
% |  "\def\version{final}\input{childdoc.def}\childdocforward{cdocsch2}"|
% \end{tabular}
% \end{center}
% Note that the trailing backslash on each first line
% merely continues the input to the second line
% (for convenient cut ant paste).
% Furthermore, the command |latex| can be replaced by any
% of its alternative versions such as |pdflatex|.
%
% %%%%%%%%%%%%%%%%%%%%%%%%%%%%%%%%%%%%%%%%%%%%%%%%%%%%%%%%%%%%%%%%%%%%%%%%%%%%%%
% %%%%%%%%%%%%%%%%%%%%%%%%%%%%%%%%%%%%%%%%%%%%%%%%%%%%%%%%%%%%%%%%%%%%%%%%%%%%%%
% \section{Implementation}
%\iffalse
%<*package>
%\fi
%
% This section describes the definitions file |childdoc.def|.

% The definitions cannot be loaded using |\usepackage| or |\RequirePackage|
% which has a mechanism to prevent loading a style file more than once.
% When loading the definitions by means of |\input|
% multiple instances have to be prevented manually:
%\iffalse
%This code needs to be before the `\ProvidesFile' directive
%which is defined at the beginning of this file.
%Therefore it is also placed there and commented out here.
%</package>
%<*discard>
%\fi
%    \begin{macrocode}
\ifdefined\childdocmain\endinput\fi
%    \end{macrocode}
%\iffalse
%</discard>
%<*package>
%\fi
%
% \macro{\ifchilddoc}
% \macro{\ifchilddocmanual}
% The conditional |\ifchilddoc| tells whether a
% child (true) or main (false) document is being compiled.
% The conditional |\ifchilddocmanual| tells whether
% the |\includeonly| mechanism is used (false) or
% the selection of child files must be performed manually (true).
% The definitions initialise to false:
%    \begin{macrocode}
\newif\ifchilddoc
\newif\ifchilddocmanual
%    \end{macrocode}

% \macro{\childdocname}
% \macro{\childdocjob}
% The macro |\childdocname| stores the name of the main document
% to be compiled. The macro |\childdocjob| stores the name of
% the document on which the \LaTeX{} compiler was originally invoked.
% The content of |\jobname| cannot be compared
% to filenames specified in the source due to different catcodes.
% The following code rescans |\jobname|, stores the result
% in |\childdocname| and saves a copy in |\childdocjob|:
%    \begin{macrocode}
\edef\childdocname{\scantokens\expandafter{\jobname\noexpand}}
\let\childdocjob\childdocname
%    \end{macrocode}

% \macro{\childdocdisable}
% The macro |\childdocdisable| prevents the main file
% from being processed more than once.
% At this stage, the main document command |\childdocmain|
% is assumed to be called once again where it should do nothing.
% Any subsequent call to it should prevent
% a secondary processing of the main document
% It overwrites the forwarding commands
% |\childdocof| and |\childdocforward|
% with empty macros to prevent further inclusions of the main document:
%    \begin{macrocode}
\newcommand{\childdocdisable}
{
  \renewcommand{\childdocmain}[1]{\renewcommand{\childdocmain}[1]{\endinput}}
  \renewcommand{\childdocof}[1]{}
  \renewcommand{\childdocby}[2][]{}
  \renewcommand{\childdocforward}[2][]{}
  \renewcommand{\childdocdisable}{}
}
%    \end{macrocode}

% \macro{\childdocmain}
% The macro |\childdocmain| is to be called at the top of the main file
% with nothing or the main filename (without extension) as argument.
% First, it breaks loops.
% If the argument is not empty and does not match |\childdocname|
% (which is set by the first inclusion of |childdoc.def|),
% |\ifchilddoc| is set to true, |\includeonly| is applied to the child file
% and |\jobname| is set to the main file
% (for proper handling of |.aux| files):
%    \begin{macrocode}
\newcommand{\childdocmain}[1]
{
  \childdocdisable\childdocmain{}
  \if?#1?\else
    \begingroup
      \def\childdoctmp{#1}
      \ifx\childdoctmp\childdocname
        \def\childdoctmp{}
      \else
        \def\childdoctmp
        {
          \childdoctrue
          \includeonly{\childdocname}
          \def\childdocjob{#1}
          \def\jobname{#1}
        }
      \fi
      \expandafter
    \endgroup
    \childdoctmp
  \fi
}
%    \end{macrocode}

% \macro{\childdocof}
% The command |\childdocof| redirects
% compilation to the main file |#1|.
%    \begin{macrocode}
\newcommand{\childdocof}[1]
{
  \childdocdisable
  \childdoctrue
  \includeonly{\childdocname}
  \def\jobname{#1}
  \def\childdocjob{#1}
  \input{#1}
}
%    \end{macrocode}

% \macro{\childdocby}
% The command |\childdocby| ....
%    \begin{macrocode}
\newcommand{\childdocby}[2][]
{
  \childdocdisable
  \childdoctrue
  \childdocmanualtrue
  \if?#1?\else
    \def\jobname{#2}
  \fi
  \def\childdocjob{#2}
  \input{#2}
  \endinput
}
%    \end{macrocode}

% \macro{\childdocforward}
% The command |\childdocforward| redirects
% compilation to the main file or
% (if the optional argument is given) a child file.
% Parameters are set as if the main file
% or a child file starting with |\childdocof| was compiled.
% Then compilation is handed over to the main file:
%    \begin{macrocode}
\newcommand{\childdocforward}[2][]
{
  \begingroup
    \if?#1?
      \def\childdoctmp
      {
        \def\childdocname{#2}
        \def\childdocjob{#2}
        \def\jobname{#2}
        \input{#2}
        \endinput
      }
    \else
      \def\childdoctmp
      {
        \childdocdisable
        \def\childdocname{#2}
        \childdoctrue
        \includeonly{#2}
        \def\childdocjob{#1}
        \def\jobname{#1}
        \input{#1}
        \endinput
      }
    \fi
    \expandafter
  \endgroup
  \childdoctmp
}
%    \end{macrocode}

% \macro{\childdocforwardprefix}
% The command |\childdocforwardprefix| redirects
% compilation to the main or a child file by means of a pattern.
% The prefix |#1| in the current filename is replaced by |#2|
% and the suffix of the current filename is kept
% (it is assumed that the filename does not contain the substring `|~~~|'
% which is used as a delimiter).
% Compilation is handed over to the new file by |\childdocforward|:
%    \begin{macrocode}
\newcommand{\childdocforwardprefix}[3][]
{
  \begingroup
    \def\childdocextract #2##1~~~{\def\childdoctmp{\childdocforward[#1]{#3##1}}}
    \expandafter\childdocextract\childdocname~~~
    \expandafter
  \endgroup
  \childdoctmp
}
%    \end{macrocode}

% \macro{\childdoc}
% The deprecated macro |\childdoc| is a legacy version of |\childdocmain|:
%    \begin{macrocode}
\newcommand{\childdoc}{\childdocmain}
%    \end{macrocode}

% \macro{\childdocredirect}
% The deprecated macro |\childdocredirect| is a legacy version
% of |\childdocforward| and |\childdocforwardprefix|:
%    \begin{macrocode}
\newcommand{\childdocredirect}[2][]
{
  \begingroup
    \if?#1?
      \def\childdoctmp{\childdocforward{#2}}
    \else
      \def\childdoctmp{\childdocforwardprefix{#1}{#2}}
    \fi
    \expandafter
  \endgroup
  \childdoctmp
}
%    \end{macrocode}

%\iffalse
%</package>
%\fi
%
\endinput

\childdocmain{}
%    \end{macrocode}

% Optional override for |\version| flag:
%    \begin{macrocode}
%%\ifchilddoc\else\providecommand{\version}{draft}\fi
%    \end{macrocode}

% Define the default values for the |\version| flag
% (|final| for the main file and |draft| for childs):
%    \begin{macrocode}
\ifchilddoc
\providecommand{\version}{draft}
\else
\providecommand{\version}{final}
\fi
%    \end{macrocode}

% Load the standard document class:
%    \begin{macrocode}
\documentclass[12pt]{article}
%    \end{macrocode}

% Start the document body:
%    \begin{macrocode}
\begin{document}
%    \end{macrocode}

% Declare a title page.
% Print title, part of document being processed and version flag:
%    \begin{macrocode}
\addtocounter{page}{-1}
\begin{center}
{\LARGE\bfseries{}childdoc example\par}
\vspace{1cm}
\ifchilddoc
\ifchilddocmanual part\else chapter\fi:
`\childdocname' of `\childdocjob'\par
\else
main document: `\childdocjob'\par
\fi
version: \version\par
\end{center}
\newpage
%    \end{macrocode}

% Manually include selected file,
% otherwise process as usual:
%    \begin{macrocode}
\ifchilddocmanual
\section*{part `\childdocname'}
\input{\childdocname}
\else
%    \end{macrocode}

% Include the two chapters:
%    \begin{macrocode}
\include{cdocsch1}
\include{cdocsch2}
%    \end{macrocode}

% Include the two parts unless only chapters should be displayed:
%    \begin{macrocode}
\ifchilddoc\else
\section{part three}
\input{cdocspt3}
\section{part four}
\input{cdocspt4}
\fi
%    \end{macrocode}

% Process as usual until here:
%    \begin{macrocode}
\fi
%    \end{macrocode}

% End of document body:
%    \begin{macrocode}
\end{document}
%    \end{macrocode}
%\iffalse
%</samplemain>
%\fi
%
% %%%%%%%%%%%%%%%%%%%%%%%%%%%%%%%%%%%%%%
% \paragraph{Chapter Include Files.}
%
% The include files are called |cdocsch1.tex| and |cdocsch2.tex|.
%
%\iffalse
%<*samplechap1|samplechap2>
%\fi

% Optional override for |\version| flag:
%    \begin{macrocode}
%%\providecommand{\version}{final}
%    \end{macrocode}

% Include the main document:
%    \begin{macrocode}
% \iffalse
%
% childdoc.dtx Copyright (C) 2017-2018 Niklas Beisert
%
% This work may be distributed and/or modified under the
% conditions of the LaTeX Project Public License, either version 1.3
% of this license or (at your option) any later version.
% The latest version of this license is in
%   http://www.latex-project.org/lppl.txt
% and version 1.3 or later is part of all distributions of LaTeX
% version 2005/12/01 or later.
%
% This work has the LPPL maintenance status `maintained'.
%
% The Current Maintainer of this work is Niklas Beisert.
%
% This work consists of the files childdoc.dtx and childdoc.ins
% and the derived files childdoc.def and cdocsamp.tex with
% cdocsch1.tex, cdocsch2.tex, cdocsdrf.tex, cdocsfn1.tex, cdocsfn2.tex.
%
%<package>\ifdefined\childdocmain\endinput\fi
%<package>\ProvidesFile{childdoc.def}[2018/12/30 v2.0 child document driver]
%<samplemain>\ProvidesFile{cdocsamp.tex}[2018/12/30 v2.0 sample for childdoc]
%<*driver>
%\ProvidesFile{childdoc.drv}[2018/12/30 v2.0 childdoc reference manual file]
\PassOptionsToClass{10pt,a4paper}{article}
\documentclass{ltxdoc}

\usepackage[margin=35mm]{geometry}
\usepackage{hyperref}
\usepackage{hyperxmp}
\usepackage[usenames]{color}

\hypersetup{colorlinks=true}
\hypersetup{pdfstartview=FitH}
\hypersetup{pdfpagemode=UseNone}
\hypersetup{pdfsource={}}
\hypersetup{pdflang={en-UK}}
\hypersetup{pdfcopyright={Copyright 2017-2018 Niklas Beisert.
  This work may be distributed and/or modified under the
  conditions of the LaTeX Project Public License, either version 1.3
  of this license or (at your option) any later version.}}
\hypersetup{pdflicenseurl={http://www.latex-project.org/lppl.txt}}
\hypersetup{pdfcontactaddress={ETH Zurich, ITP, HIT K,
  Wolfgang-Pauli-Strasse 27}}
\hypersetup{pdfcontactpostcode={8093}}
\hypersetup{pdfcontactcity={Zurich}}
\hypersetup{pdfcontactcountry={Switzerland}}
\hypersetup{pdfcontactemail={nbeisert@itp.phys.ethz.ch}}
\hypersetup{pdfcontacturl={http://people.phys.ethz.ch/\xmptilde nbeisert/}}

\newcommand{\secref}[1]{\hyperref[#1]{section \ref*{#1}}}

\parskip1ex
\parindent0pt
\let\olditemize\itemize
\def\itemize{\olditemize\parskip0pt}

\begin{document}

\title{The \textsf{childdoc} Package}
\hypersetup{pdftitle={The childdoc Package}}
\author{Niklas Beisert\\[2ex]
  Institut f\"ur Theoretische Physik\\
  Eidgen\"ossische Technische Hochschule Z\"urich\\
  Wolfgang-Pauli-Strasse 27, 8093 Z\"urich, Switzerland\\[1ex]
  \href{mailto:nbeisert@itp.phys.ethz.ch}
  {\texttt{nbeisert@itp.phys.ethz.ch}}}
\hypersetup{pdfauthor={Niklas Beisert}}
\hypersetup{pdfsubject={Manual for the LaTeX2e Package childdoc}}
\date{30 December 2018, \textsf{v2.0}}
\maketitle

\begin{abstract}\noindent
\textsf{childdoc} is a \LaTeXe{} package
that enables the direct compilation
of document sections included by |\include|
to individual files.
\end{abstract}

\begingroup
\parskip0ex
\tableofcontents
\endgroup

%%%%%%%%%%%%%%%%%%%%%%%%%%%%%%%%%%%%%%%%%%%%%%%%%%%%%%%%%%%%%%%%%%%%%%%%%%%%%%%%
%%%%%%%%%%%%%%%%%%%%%%%%%%%%%%%%%%%%%%%%%%%%%%%%%%%%%%%%%%%%%%%%%%%%%%%%%%%%%%%%
\section{Introduction}

\LaTeX{} provides a mechanism to structure a large document (such as a book)
into a main file and several child files (containing the chapters)
using the |\include| command.
This mechanism is beneficial for documents
which span hundreds of pages in order to
make the source file(s) more manageable.
Moreover, compilation can be restricted to
selected child files by means of the |\includeonly| command.
The latter feature can be used to reduce the compilation time while editing
(this was significantly more useful in the earlier days of \LaTeX{})
or to generate a smaller document which is easier to navigate.
Another application of |\includeonly| is to generate
documents consisting of selected parts of the complete document.

However, there are a few drawbacks of the plain |\include| mechanism:
\begin{itemize}
\item
The child files cannot be compiled on their own,
they can only be compiled via the main file.
A naive editing environment
(such as a text editor with an option
to have the current file processed by \LaTeX)
may require one to switch to the main file before compiling;
attempting to compile the child file produces errors.
\item
The main file must be modified (each time)
to adjust the |\includeonly| command
to the present needs. This easily leaves the main file in a messy state.
\item
The generated document will always carry the filename
of the main document. This is inconvenient if
several child files are to be compiled and
to be kept for distribution.
\end{itemize}

The present package provides a simple interface
to make child files individually compilable by \LaTeX{}.
Compiling a child file then has the same effect as compiling
the main file with an |\includeonly| command
to select the appropriate child.
Moreover the generated document will carry the name of the child
rather than the main file.
This resolves all three above issues.

This feature is meant to make the editing of books,
thesis documents and lecture notes somewhat more convenient.
However, the package can also be used efficiently for
composing a series of documents (such as exercise sheets)
which are typically distributed individually.
It then assists the author in generating the individual documents
(potentially in different versions)
as well as a document containing the collected series.
Another application is in developing style files
or other kinds of included material
where compilation of the style file could redirect
to a sample or test file.

%%%%%%%%%%%%%%%%%%%%%%%%%%%%%%%%%%%%%%%%%%%%%%%%%%%%%%%%%%%%%%%%%%%%%%%%%%%%%%%%
%%%%%%%%%%%%%%%%%%%%%%%%%%%%%%%%%%%%%%%%%%%%%%%%%%%%%%%%%%%%%%%%%%%%%%%%%%%%%%%%
\section{Usage}

First of all, the package \textsf{childdoc} is \emph{not} a standard
\LaTeXe{} |.sty| style file! Therefore it needs to be invoked in
a non-standard way.

%%%%%%%%%%%%%%%%%%%%%%%%%%%%%%%%%%%%%%%%%%%%%%%%%%%%%%%%%%%%%%%%%%%%%%%%%%%%%%%%
\subsection{Included Files}
\label{sec:include}

%%%%%%%%%%%%%%%%%%%%%%%%%%%%%%%%%%%%%%%%
\DescribeMacro{\childdocmain}
To use the package, add the commands
\begin{center}
\begin{tabular}{l}
|\input{childdoc.def}|\\
|\childdocmain{}|\\
\end{tabular}
\end{center}
at the very top of the main \LaTeX{} file,
in particular \emph{before} the |\documentclass| statement!
The argument of |\childdocmain| should be left empty
(but it must be present).

%%%%%%%%%%%%%%%%%%%%%%%%%%%%%%%%%%%%%%%%
\DescribeMacro{\childdocof}
Furthermore, add the commands
\begin{center}
\begin{tabular}{l}
|\input{childdoc.def}|\\
|\childdocof{|\textit{main}|}|\\
\end{tabular}
\end{center}
at the top of every child file \textit{child}
which is included by |\include{|\textit{child}|}|
from within the main file
(or at least for those files to be compiled individually).
The argument \textit{main} must be the filename of the main file.

There are a couple of
considerations in setting up the main and child documents:

%%%%%%%%%%%%%%%%%%%%%%%%%%%%%%%%%%%%%%%%
\paragraph{Restrictions.}

Please note the following restrictions:
\begin{itemize}
\item
|\childdocmain| must be called with one argument \textit{main}
to ensure compatibility with earlier version of the package.
It must either be empty (|\childdocmain{}|)
or precisely match the filename of the main file in which it is specified.
See \secref{sec:detection} for further information.
\item
The filename \textit{main} must be specified without the |.tex| extension.
\item
The filename \textit{main} is case sensitive
(even in case-insensitive file systems)
due to internal string comparison.
\item
The argument \textit{main} should be fully expanded, it cannot be a macro.
\item
Subdirectories and special characters should be avoided in filenames.
\item
The command |\childdocmain{|\textit{main}|}| must be followed by a whitespace.
It should not be followed immediately by another command
or by a comment mark `|%|'.
This is because the \TeX{} parser reads the token immediately following
the argument of |\childdocmain| and puts it
at the beginning of every child section;
however, a white\-space is ignored.
\end{itemize}

%%%%%%%%%%%%%%%%%%%%%%%%%%%%%%%%%%%%%%%%
\paragraph{Content of Main File.}

It is advisable to place all content in the child files included by |\include|.
Any output contained in the main file will appear in all child documents
unless suppressed manually;
it cannot be suppressed automatically by the |\includeonly| directive
and thus should normally be avoided.
A method to include some content in the main file
by means of conditional processing is described in \secref{sec:conditional}.

%%%%%%%%%%%%%%%%%%%%%%%%%%%%%%%%%%%%%%%%
\paragraph{Page Numbering.}

When only a part of the document is compiled,
the appropriate numbering of pages
(as well as other status parameters)
is determined from the |.aux| files.
The latter contain information from previous passes.
However this information needs to propagate through
all intermediate child documents.
Therefore the page numbering in child documents may well
be inconsistent until the complete document is compiled at least once.

A useful (if unconventional) way to always ensure a consistent
page numbering is to restart the numbering in each child document
and denote the pages by `\textit{child}|.|\textit{page}'
where \textit{child} represents the chapter/section number of the child file.
This can be achieved by the command
|\numberwithin{page}{|\textit{child}|}|
of the \textsf{amsmath} package
where \textit{child} can be |chapter| or |section|
depending on the chosen structuring.
Alternatively, one can modify the macro |\thepage| appropriately
and reset the counter |page| at the start of each child file.

%%%%%%%%%%%%%%%%%%%%%%%%%%%%%%%%%%%%%%%%%%%%%%%%%%%%%%%%%%%%%%%%%%%%%%%%%%%%%%%%
\subsection{Conditional Processing}
\label{sec:conditional}

The package provides a mechanism to compile different versions
of a document. To customise the versions further some conditional processing
can come in handy to distinguish which version is being compiled.
The package provides two macros to describe the compilation context:

%%%%%%%%%%%%%%%%%%%%%%%%%%%%%%%%%%%%%%%%
\DescribeMacro{\ifchilddoc}
The conditional |\ifchilddoc| distinguishes between the compilation of
child documents and the main document:
%
\begin{center}
|\ifchilddoc |\textit{child-code}| |[|\||else |\textit{main-code}]| \||fi|
\end{center}

%%%%%%%%%%%%%%%%%%%%%%%%%%%%%%%%%%%%%%%%
\DescribeMacro{\childdocname}
\DescribeMacro{\childdocjob}
The macro |\childdocname| contains the filename (without extension)
of the main or child file being processed.
Note that |\childdocjob| will always contain the name of the main file.

%%%%%%%%%%%%%%%%%%%%%%%%%%%%%%%%%%%%%%%%
\paragraph{Title Page.}

Conditional processing can be used to include a title or banner page
in the main document when proper precautions are taken.
Importantly, the code in the main file should ensure that the page counter
(as well as other status parameters which are stored in the |.aux| files)
takes the same value after the conditional processing.
Otherwise the page numbers may take divergent values
depending on which part is compiled.

For example, a title page could be declared by:
%
\begin{center}
\begin{tabular}{l}
|\ifchilddoc\||else|\\
|\addtocounter{page}{-1}|\\
\textit{code for title page}\\
|\newpage|\\
|\||fi|
\end{tabular}
\end{center}
%
A banner page for the child documents can be generated by:
%
\begin{center}
\begin{tabular}{l}
|\ifchilddoc|\\
|\addtocounter{page}{-1}|\\
\textit{code for banner page}\\
|\newpage|\\
|\||fi|
\end{tabular}
\end{center}
%
Here one could write a message such as:
\begin{center}
|This is the part \childdocname{} of \childdocjob{}.|
\end{center}

%%%%%%%%%%%%%%%%%%%%%%%%%%%%%%%%%%%%%%%%%%%%%%%%%%%%%%%%%%%%%%%%%%%%%%%%%%%%%%%%
\subsection{Flags}
\label{sec:flags}

The package makes it easy to generate different versions
of the main or child documents.
To this end compilation flags can be defined
and assigned different default values.
They will be particularly useful in conjunction
with the forwarding mechanism described in \secref{sec:forward}.

For example, it may be useful to have a flag |\version|
which can be set to |draft| or |final|.
The document source will contain some conditional code
depending on the value of |\version|.
Suppose further, the flag should default to |final| for the main file
and to |draft| for child files
which is a natural assignment for editing the document.
This is achieved by placing the following code
in the preamble of the main document
(below the |\childdocmain| directive):
%
\begin{center}
\begin{tabular}{l}
|\ifchilddoc|\\
|\providecommand{\version}{draft}|\\
|\||else|\\
|\providecommand{\version}{final}|\\
|\||fi|
\end{tabular}
\end{center}
%
The definition by |\providecommand| makes sure
that previous definitions are not overwritten.
Further statements |\providecommand{\version}{...}|
can thus be added before the above code to override it.

For the main file, one might add a line
(between |\childdocmain| and the above block)
%
\begin{center}
|%\ifchilddoc\||else\providecommand{\version}{draft}\||fi|
\end{center}
%
which can be uncommented to produce a draft version.
Likewise one can add a line to the very top of a child file
(above the |\childdocof{|\textit{main}|}| directive)
%
\begin{center}
|%\providecommand{\version}{final}|
\end{center}
%
which can be uncommented to produce the final version of this child document.

%%%%%%%%%%%%%%%%%%%%%%%%%%%%%%%%%%%%%%%%%%%%%%%%%%%%%%%%%%%%%%%%%%%%%%%%%%%%%%%%
\subsection{Forwarding}
\label{sec:forward}

Different versions of the main or child documents
using compilation flags as described in \secref{sec:flags}
can be (permanently) stored in different files
for convenient compilation, viewing and distribution.
To this end, the package defines a command
to pass on compilation to a different file:

%%%%%%%%%%%%%%%%%%%%%%%%%%%%%%%%%%%%%%%%
\DescribeMacro{\childdocforward}
The command |\childdocforward| redirects processing to
another source file:
%
\begin{center}
\begin{tabular}{l}
|\input{childdoc.def}|\\
|\childdocforward[|\textit{main}|]{|\textit{dest}|}|\\
\end{tabular}
\end{center}
%
The argument \textit{dest} is the destination file
(without extension).
It should be the main file or one of the child files.
Note that further \textsf{childdoc} directives
such as |\childdocof| and |\childdocforward|
in the indicated file will be processed in this form.
The optional argument \textit{main}
passes on directly to the main file \textit{main}
while pretending to compile the child \textit{dest}.
This form behaves as if \textit{dest}
issues |\childdocof{|\textit{main}|}| right away,
and no further \textsf{childdoc} directives will be processed.

%%%%%%%%%%%%%%%%%%%%%%%%%%%%%%%%%%%%%%%%
\DescribeMacro{\...prefix}
In the alternative form |\childdocforwardprefix|,
%
\begin{center}
\begin{tabular}{l}
|\input{childdoc.def}|\\
|\childdocforwardprefix[|\textit{main}|]{|\textit{prefix}|}{|\textit{dest}|}|
\end{tabular}
\end{center}
%
the destination file is determined by a pattern
depending on the current file:
To make this work, the current file must be called
`{\textit{prefix}\hspace{0.2em}\textit{suffix}}'
with \textit{prefix} matching precisely the argument.
Processing is then passed on to the file
`{\textit{dest}\hspace{0.2em}\textit{suffix}}'.
Surely, the same effect is achieved by
directly specifying the
argument `{\textit{dest}\hspace{0.2em}\textit{suffix}}'
in the first form.
However, that requires to set up a different file
for each child. With the alternative form of the command
all these files can have exactly the same content
which simplifies setting them up and maintaining them.

For example, the following file |draft.tex|
with a compilation flag |\version| as described in \secref{sec:flags}
compiles the main document as a draft:
%
\begin{center}
\begin{tabular}{l}
|\def\version{draft}|\\
|\input{childdoc.def}|\\
|\childdocforward{|\textit{main}|}|
\end{tabular}
\end{center}
%
Likewise, the following files |final|\textit{nn}|.tex|
compile the final version of the child document
|child|\textit{nn}|.tex|:
%
\begin{center}
\begin{tabular}{l}
|\def\version{final}|\\
|\input{childdoc.def}|\\
|\childdocforwardprefix{final}{child}|
\end{tabular}
\end{center}
%

Note that when several versions of a main file and/or of each child file
are to be generated, it may be convenient to set up a |Makefile| or
shell script to automatise the process.

%%%%%%%%%%%%%%%%%%%%%%%%%%%%%%%%%%%%%%%%%%%%%%%%%%%%%%%%%%%%%%%%%%%%%%%%%%%%%%%%
\subsection{Command Line Processing}
\label{sec:commandline}

The effect of redirection files can also be achieved by invoking
the \LaTeX{} compiler with a more elaborate command line.
Most conveniently this should be done as part
of a shell script or a |Makefile|.

When using \textsf{childdoc} in the main file, the following
command lines effectively perform a redirection
(note that depending on the shell being used,
backslashes may have to be doubled: `|\|' $\to$ `|\\|'):
%
\begin{center}
|... -jobname "|\textit{target}|" |\\|"|[\textit{flags}]%
|\input{childdoc.def}\childdocforward[|\textit{main}|]{|\textit{dest}|}"|
\end{center}
%
Here \textit{target} is the name of the output file,
\textit{main} is the name of the main file
and \textit{dest} is the name of the main or child file to be processed
(all filenames without extensions).
The optional argument \textit{main} can be omitted
if \textit{main} matches \textit{dest}.
Optionally, compilation \textit{flags} can be defined via |\def| commands.
This command line makes the \TeX{} engine believe
it is compiling the file \textit{target}
whose content is specified as the latter parameter.
The provided code then forwards the processing to
\textit{main} or \textit{dest} as described in \secref{sec:forward}.

%%%%%%%%%%%%%%%%%%%%%%%%%%%%%%%%%%%%%%%%%%%%%%%%%%%%%%%%%%%%%%%%%%%%%%%%%%%%%%%%
\subsection{Include by Input}
\label{sec:input}

Including child documents by |\include| has some restrictions by design.
Most notably, the content of a child document always occupies
its own set of pages; pages cannot be shared between child documents.
Usually, this behaviour makes perfect sense
because each child document contain an essential part of the document.
However, in some situations it may be desirable to compose
a document from a collection of parts
without having mandatory page breaks between then.
For this case, the package
provides a mechanism to include parts
by |\input| which can also be processed individually.
However, by construction this mechanism
requires manual handling of the content to be output.

%%%%%%%%%%%%%%%%%%%%%%%%%%%%%%%%%%%%%%%%
\DescribeMacro{\ifchilddocmanual}
The main file should be prepared as usual, see \secref{sec:include}.
However, the document body must make a distinction
between processing of an individual part and of the main document, e.g.:
%
\begin{center}
\begin{tabular}{l}
|\ifchilddocmanual|\\
|\input{\childdocname}|\\
|\||else|\\
\textit{document body with }|\input{|\textit{part}|}|\\
|\||fi|
\end{tabular}
\end{center}
%
The conditional |\ifchilddocmanual| is true whenever
a part to be included by |\input| is being compiled,
and the name of the part is stored in |\childdocname|.

%%%%%%%%%%%%%%%%%%%%%%%%%%%%%%%%%%%%%%%%
\DescribeMacro{\childdocby}
Each part to be included by |\input| should start with:
%
\begin{center}
\begin{tabular}{l}
|\input{childdoc.def}|\\
|\childdocby{|\textit{main}|}|\\
\end{tabular}
\end{center}
%
The directive |\childdocby| is similar to |\childdocof|
described in \secref{sec:include},
but the subsequent selection of content must be done manually.
To that end, both |\ifchilddoc| and |\ifchilddocmanual|
will be true upon processing of a part,
and the name of the part is stored in |\childdocname|.
Note that |\jobname| will be set to the filename of the current part
so that each part receives an individual |.aux| file
that does not interfere with the |.aux| file(s) of the main document.
This behaviour can be altered by the alternative form
|\childdocby[*]{|\textit{main}|}| (with a non-empty optional argument)
which uses the |.aux| file of the main document
by setting |\jobname| to \textit{main}.

%%%%%%%%%%%%%%%%%%%%%%%%%%%%%%%%%%%%%%%%%%%%%%%%%%%%%%%%%%%%%%%%%%%%%%%%%%%%%%%%
\subsection{Driver Development}
\label{sec:driver}

The \textsf{childdoc} mechanism can also be use for the development
of definition files such as \LaTeX{} styles or classes.
This case differs from the above setup with multiple parts
included by |\include| in that no |\includeonly| should be invoked.
This can be achieved by starting the include file
(before |\ProvidesPackage|) with:
%
\begin{center}
\begin{tabular}{l}
|\input{childdoc.def}|\\
|\childdocforward{|\textit{main}|}|\\
\end{tabular}
\end{center}
%
or alternatively with:
%
\begin{center}
\begin{tabular}{l}
|\input{childdoc.def}|\\
|\childdocby{|\textit{main}|}|\\
\end{tabular}
\end{center}
%
Both forms have slightly different effects as described above.
The main file is prepared as usual, see \secref{sec:include}.

%%%%%%%%%%%%%%%%%%%%%%%%%%%%%%%%%%%%%%%%%%%%%%%%%%%%%%%%%%%%%%%%%%%%%%%%%%%%%%%%
\subsection{Legacy Detection}
\label{sec:detection}

The directive |\childdocmain| in the main file can detect
whether the complete document or merely a child is to be compiled
even without using the directive |\childdocof|.
This method is deprecated because it is less robust
and there is no compelling reason to use it;
it is merely provided for backward compatibility
and it may be removed in future versions.

If the detection mechanism is to be used,
it is mandatory to correctly specify
the filename of the main file as the argument of |\childdocmain|:
%
\begin{center}
\begin{tabular}{l}
|\input{childdoc.def}|\\
|\childdocmain{|\textit{main}|}|\\
\end{tabular}
\end{center}
%
If |\jobname| does not match the argument \textit{main} of |\childdocmain|,
it is assumed that |\jobname| points to the child file to be compiled.
When using |\childdocmain| with the main file specified as argument,
it suffices to start a child file
with just |\input{|\textit{main}|}|
without loading of the package and using |\childdocof|.
If instead all processing is done
with the appropriate \textsf{childdoc} directives,
the argument of \textit{main} of |\childdocmain| can be empty.

An alternative version of the command line processing described
in \secref{sec:commandline} using the detection mechanism reads:
%
\begin{center}
|... -jobname "|\textit{target}|" "|[\textit{flags}]%
[|\def\jobname{|\textit{dest}|}|]|\input{|\textit{main}|}"|
\end{center}

%%%%%%%%%%%%%%%%%%%%%%%%%%%%%%%%%%%%%%%%%%%%%%%%%%%%%%%%%%%%%%%%%%%%%%%%%%%%%%%%
\subsection{Manual Code}
\label{sec:manual}

In case one cannot be certain whether the definitions file |childdoc.def|
is installed on the target \TeX{} distribution
and one prefers not to ship it,
it is conceivable to paste a few relevant commands into the sources.

To that end, drop all statements |\input{childdoc.def}|
and perform the replacements as outlined below.
Instead of |\childdocmain{|\textit{main}|}| add the following code
to the top of the main file:
%
\begin{center}
\begin{tabular}{l}
|\||ifdefined\childdocname\endinput\||fi\newif\ifchilddoc|\\
|\edef\childdocname{\scantokens\expandafter{\jobname\noexpand}}|\\
|\def\childdocmain{|\textit{main}|}\||ifx\childdocmain\childdocname\||else|\\
|\childdoctrue\includeonly{\childdocname}\let\jobname\childdocmain\||fi|\\
\end{tabular}
\end{center}
%
Instead of |\childdocof{|\textit{main}|}| just include the main file
at the top of each child file:
%
\begin{center}
|\input{|\textit{main}|}|
\end{center}
%
A simple redirection |\childdocforward{|\textit{dest}|}| is achieved by:
%
\begin{center}
|\def\jobname{|\textit{dest}|}\input{\jobname}|
\end{center}
%
The redirection with prefix
|\childdocforwardprefix[|\textit{prefix}|]{|\textit{dest}|}|
is accomplished by:
%
\begin{center}
\begin{tabular}{l}
|{\edef\jobname{\scantokens\expandafter{\jobname\noexpand}}|\\
|\def\redirectjob |\textit{prefix}|#1~~~{\gdef\jobname{|\textit{dest}|#1}}|\\
|\expandafter\redirectjob\jobname~~~}\input{\jobname}|
\end{tabular}
\end{center}

In an alternative approach,
child documents can be compiled by a specific command line
without additional code or specific definitions:
%
\begin{center}
|... -jobname "|\textit{target}|" "|[\textit{flags}]%
|\includeonly{|\textit{dest}|}\input{|\textit{main}|}"|
\end{center}
%

%%%%%%%%%%%%%%%%%%%%%%%%%%%%%%%%%%%%%%%%%%%%%%%%%%%%%%%%%%%%%%%%%%%%%%%%%%%%%%%%
%%%%%%%%%%%%%%%%%%%%%%%%%%%%%%%%%%%%%%%%%%%%%%%%%%%%%%%%%%%%%%%%%%%%%%%%%%%%%%%%
\section{Information}

%%%%%%%%%%%%%%%%%%%%%%%%%%%%%%%%%%%%%%%%%%%%%%%%%%%%%%%%%%%%%%%%%%%%%%%%%%%%%%%%
\subsection{Copyright}

Copyright \copyright{} 2017--2018 Niklas Beisert

This work may be distributed and/or modified under the
conditions of the \LaTeX{} Project Public License, either version 1.3
of this license or (at your option) any later version.
The latest version of this license is in
  \url{http://www.latex-project.org/lppl.txt}
and version 1.3 or later is part of all distributions of \LaTeX{}
version 2005/12/01 or later.

This work has the LPPL maintenance status `maintained'.

The Current Maintainer of this work is Niklas Beisert.

This work consists of the files |README.txt|, |childdoc.ins| and |childdoc.dtx|
as well as the derived files |childdoc.def|, |cdocsamp.tex|
with |cdocsch1.tex|, |cdocsch2.tex|, |cdocspt3.tex|, |cdocspt4.tex|,
|cdocsdrf.tex|, |cdocsfn1.tex|, |cdocsfn2.tex|
as well as |childdoc.pdf|.

%%%%%%%%%%%%%%%%%%%%%%%%%%%%%%%%%%%%%%%%%%%%%%%%%%%%%%%%%%%%%%%%%%%%%%%%%%%%%%%%
\subsection{Files and Installation}

The package consists of the files:
%
\begin{center}
\begin{tabular}{ll}
    |README.txt|   & readme file \\
    |childdoc.ins| & installation file \\
    |childdoc.dtx| & source file \\
    |childdoc.def| & definition file \\
    |cdocsamp.tex| & sample main file \\
    |cdocsch1.tex| & sample include file \\
    |cdocsch2.tex| & sample include file \\
    |cdocspt3.tex| & sample part file \\
    |cdocspt4.tex| & sample part file \\
    |cdocsdrf.tex| & sample redirection file \\
    |cdocsfn1.tex| & sample redirection file \\
    |cdocsfn2.tex| & sample redirection file \\
    |childdoc.pdf| & manual
\end{tabular}
\end{center}
%
The distribution consists of the files
|README.txt|, |childdoc.ins| and |childdoc.dtx|.
%
\begin{itemize}
\item
Run (pdf)\LaTeX{} on |childdoc.dtx|
to compile the manual |childdoc.pdf| (this file).
\item
Run \LaTeX{} on |childdoc.ins| to create the definitions file |childdoc.def|
and the sample |cdocsamp.tex| with include files
|cdocsch1.tex|, |cdocsch2.tex|, |cdocspt3.tex|, |cdocspt4.tex|,
|cdocsdrf.tex|, |cdocsfn1.tex|, |cdocsfn2.tex|.
Then copy the file |childdoc.def| to an appropriate directory of your \LaTeX{}
distribution, e.g.\ \textit{texmf-root}|/tex/latex/childdoc|.
\end{itemize}

%%%%%%%%%%%%%%%%%%%%%%%%%%%%%%%%%%%%%%%%%%%%%%%%%%%%%%%%%%%%%%%%%%%%%%%%%%%%%%%%
\subsection{Related CTAN Packages}

There are several other packages which offer a similar functionality:
%
\begin{itemize}
\item
The packages
\href{http://ctan.org/pkg/docmute}{\textsf{docmute}},
\href{http://ctan.org/pkg/includex}{\textsf{includex}} and
\href{http://ctan.org/pkg/standalone}{\textsf{standalone}}
provide commands to include only the document body of
a child file thus allowing both files to be compiled individually.
\item
The packages \href{http://ctan.org/pkg/subdocs}{\textsf{subdocs}}
and \href{http://ctan.org/pkg/subfiles}{\textsf{subfiles}}
provide structures in which the main and child documents can be
encapsulated and allowing them to be compiled individually.
The inclusion mechanism is different from the conventional |\include|.
\item
The package \href{http://ctan.org/pkg/combine}{\textsf{combine}}
is an elaborate solution to combine several documents into one.
\end{itemize}
%
See also the CTAN topic \href{http://ctan.org/topic/subdocs}{\textsf{subdocs}}
for further related packages.
The present package differs from the above solutions in that
a document structure constructed with the conventional |\include| mechanism
just needs two extra commands at the top of every file
such that all constituent files can be compiled individually.

%%%%%%%%%%%%%%%%%%%%%%%%%%%%%%%%%%%%%%%%%%%%%%%%%%%%%%%%%%%%%%%%%%%%%%%%%%%%%%%%
%\subsection{Feature Suggestions}
%
%The following is a list of features which may be useful for future
%versions of this package:
%%
%\begin{itemize}
%\item
%\ldots
%\end{itemize}

%%%%%%%%%%%%%%%%%%%%%%%%%%%%%%%%%%%%%%%%%%%%%%%%%%%%%%%%%%%%%%%%%%%%%%%%%%%%%%%%
\subsection{Revision History}

%%%%%%%%%%%%%%%%%%%%%%%%%%%%%%%%%%%%%%%%
\paragraph{v2.0:} 2018/12/30

\begin{itemize}
\item
immediate forward processing
\item
added |\childdocby| mechanism
\item
manual restructured
\end{itemize}

%%%%%%%%%%%%%%%%%%%%%%%%%%%%%%%%%%%%%%%%
\paragraph{v1.6:} 2018/01/17

\begin{itemize}
\item
application for development of include files
\item
corrections to manual
\end{itemize}

%%%%%%%%%%%%%%%%%%%%%%%%%%%%%%%%%%%%%%%%
\paragraph{v1.5:} 2017/05/21

\begin{itemize}
\item
more complete structuring introduced
\item
|\childdocof| introduced
\item
|\childdoc| renamed to |\childdocmain|
\item
|\childredirect| renamed to |\childdocforward| and |\childdocforwardprefix|
and functionality expanded
\end{itemize}

%%%%%%%%%%%%%%%%%%%%%%%%%%%%%%%%%%%%%%%%
\paragraph{v1.0:} 2017/04/27

\begin{itemize}
\item
manual and install package
\item
first version published on CTAN
\end{itemize}

%%%%%%%%%%%%%%%%%%%%%%%%%%%%%%%%%%%%%%%%
\paragraph{v0.6:} 2017/04/26

\begin{itemize}
\item
redirection mechanism added
\end{itemize}

%%%%%%%%%%%%%%%%%%%%%%%%%%%%%%%%%%%%%%%%
\paragraph{v0.5:} 2017/04/26

\begin{itemize}
\item
functionality in definition file
\end{itemize}


%%%%%%%%%%%%%%%%%%%%%%%%%%%%%%%%%%%%%%%%%%%%%%%%%%%%%%%%%%%%%%%%%%%%%%%%%%%%%%%%
%%%%%%%%%%%%%%%%%%%%%%%%%%%%%%%%%%%%%%%%%%%%%%%%%%%%%%%%%%%%%%%%%%%%%%%%%%%%%%%%
%%%%%%%%%%%%%%%%%%%%%%%%%%%%%%%%%%%%%%%%%%%%%%%%%%%%%%%%%%%%%%%%%%%%%%%%%%%%%%%%
\appendix

\settowidth\MacroIndent{\rmfamily\scriptsize 000\ }

 \DocInput{childdoc.dtx}

\end{document}
%</driver>
% \fi
%
% %%%%%%%%%%%%%%%%%%%%%%%%%%%%%%%%%%%%%%%%%%%%%%%%%%%%%%%%%%%%%%%%%%%%%%%%%%%%%%
% %%%%%%%%%%%%%%%%%%%%%%%%%%%%%%%%%%%%%%%%%%%%%%%%%%%%%%%%%%%%%%%%%%%%%%%%%%%%%%
% \section{Sample}
%\iffalse
%<*samplemain>
%\fi
%
% The following presents a sample document
% with two chapters, two parts, a title page,
% a compile flag as well as three forwarding files to set the flag.
% It consists of eight |.tex| files:
% \begin{center}
% \begin{tabular}{ll}
% |cdocsamp.tex|&main file\\
% |cdocsch1.tex|&include file for chapter 1\\
% |cdocsch2.tex|&include file for chapter 2\\
% |cdocspt3.tex|&include file for part 3\\
% |cdocspt4.tex|&include file for part 4\\
% |cdocsdrf.tex|&forwarding file for main file in draft mode\\
% |cdocsfi1.tex|&forwarding file for final version of chapter 1\\
% |cdocsfi2.tex|&forwarding file for final version of chapter 2\\
% \end{tabular}
% \end{center}
% Each of the eight files can be compiled directly by the \LaTeX{} compiler.
%
% %%%%%%%%%%%%%%%%%%%%%%%%%%%%%%%%%%%%%%
% \paragraph{Main File.}
%
% The main file is called |cdocsamp.tex|.
%
% Load the \textsf{childdoc} definitions and
% declare the filename for the main document:
%    \begin{macrocode}
\input{childdoc.def}
\childdocmain{}
%    \end{macrocode}

% Optional override for |\version| flag:
%    \begin{macrocode}
%%\ifchilddoc\else\providecommand{\version}{draft}\fi
%    \end{macrocode}

% Define the default values for the |\version| flag
% (|final| for the main file and |draft| for childs):
%    \begin{macrocode}
\ifchilddoc
\providecommand{\version}{draft}
\else
\providecommand{\version}{final}
\fi
%    \end{macrocode}

% Load the standard document class:
%    \begin{macrocode}
\documentclass[12pt]{article}
%    \end{macrocode}

% Start the document body:
%    \begin{macrocode}
\begin{document}
%    \end{macrocode}

% Declare a title page.
% Print title, part of document being processed and version flag:
%    \begin{macrocode}
\addtocounter{page}{-1}
\begin{center}
{\LARGE\bfseries{}childdoc example\par}
\vspace{1cm}
\ifchilddoc
\ifchilddocmanual part\else chapter\fi:
`\childdocname' of `\childdocjob'\par
\else
main document: `\childdocjob'\par
\fi
version: \version\par
\end{center}
\newpage
%    \end{macrocode}

% Manually include selected file,
% otherwise process as usual:
%    \begin{macrocode}
\ifchilddocmanual
\section*{part `\childdocname'}
\input{\childdocname}
\else
%    \end{macrocode}

% Include the two chapters:
%    \begin{macrocode}
\include{cdocsch1}
\include{cdocsch2}
%    \end{macrocode}

% Include the two parts unless only chapters should be displayed:
%    \begin{macrocode}
\ifchilddoc\else
\section{part three}
\input{cdocspt3}
\section{part four}
\input{cdocspt4}
\fi
%    \end{macrocode}

% Process as usual until here:
%    \begin{macrocode}
\fi
%    \end{macrocode}

% End of document body:
%    \begin{macrocode}
\end{document}
%    \end{macrocode}
%\iffalse
%</samplemain>
%\fi
%
% %%%%%%%%%%%%%%%%%%%%%%%%%%%%%%%%%%%%%%
% \paragraph{Chapter Include Files.}
%
% The include files are called |cdocsch1.tex| and |cdocsch2.tex|.
%
%\iffalse
%<*samplechap1|samplechap2>
%\fi

% Optional override for |\version| flag:
%    \begin{macrocode}
%%\providecommand{\version}{final}
%    \end{macrocode}

% Include the main document:
%    \begin{macrocode}
\input{childdoc.def}
\childdocof{cdocsamp}
%    \end{macrocode}

%\iffalse
%</samplechap1|samplechap2>
%\fi
%
%\iffalse
%<*samplechap1>
%\fi
% Some text for chapter 1:
%    \begin{macrocode}
\section{one}
some text in chapter one
%    \end{macrocode}

%\iffalse
%</samplechap1>
%\fi
% Some text for chapter 2:
%\iffalse
%<*samplechap2>
%\fi
%    \begin{macrocode}
\section{two}
more text in chapter two
%    \end{macrocode}

%\iffalse
%</samplechap2>
%\fi
%
% %%%%%%%%%%%%%%%%%%%%%%%%%%%%%%%%%%%%%%
% \paragraph{Part Include Files.}
%
% The include files are called |cdocspt3.tex| and |cdocspt4.tex|.
%
%\iffalse
%<*samplepart3|samplepart4>
%\fi

% Optional override for |\version| flag:
%    \begin{macrocode}
%%\providecommand{\version}{final}
%    \end{macrocode}

% Include the main document:
%    \begin{macrocode}
\input{childdoc.def}
\childdocby{cdocsamp}
%    \end{macrocode}

%\iffalse
%</samplepart3|samplepart4>
%\fi
%
%\iffalse
%<*samplepart3>
%\fi
% Some text for part 3:
%    \begin{macrocode}
some text in part three
%    \end{macrocode}

%\iffalse
%</samplepart3>
%\fi
% Some text for part 4:
%\iffalse
%<*samplepart4>
%\fi
%    \begin{macrocode}
more text in part four
%    \end{macrocode}

%\iffalse
%</samplepart4>
%\fi
%
% %%%%%%%%%%%%%%%%%%%%%%%%%%%%%%%%%%%%%%
% \paragraph{Forwarding for a Complete Draft.}
%
% The following forwarding file |cdocsdrf.tex|
% compiles the main document in draft mode:
%\iffalse
%<*sampledraft>
%\fi
%    \begin{macrocode}
\def\version{draft}
\input{childdoc.def}
\childdocforward{cdocsamp}
%    \end{macrocode}

%\iffalse
%</sampledraft>
%\fi
%
% %%%%%%%%%%%%%%%%%%%%%%%%%%%%%%%%%%%%%%
% \paragraph{Forwarding for Final Version of the Chapters.}
%
% The following forwarding files |cdocsfn1.tex| and |cdocsfn2.tex|
% (with identical content)
% compile the final versions of the child documents
% |cdocsch1.tex| and |cdocsch2.tex|, respectively:
%\iffalse
%<*samplefinal>
%\fi
%    \begin{macrocode}
\def\version{final}
\input{childdoc.def}
\childdocforwardprefix[cdocsamp]{cdocsfn}{cdocsch}
%    \end{macrocode}

%\iffalse
%</samplefinal>
%\fi
%
% %%%%%%%%%%%%%%%%%%%%%%%%%%%%%%%%%%%%%%
% \paragraph{Command Line Processing.}
%
% The following three command lines generate the output files
% |cdocscld|, |cdocscl1| and |cdocscl2|
% which should be identical to
% |cdocsdrf|, |cdocsch1| and |cdocsfn2|, respectively:
% \begin{center}
% \begin{tabular}{l}
% |latex -jobname cdocscld \|\\
% |  "\def\version{draft}\input{childdoc.def}\childdocforward{cdocsamp}"|\\
% |latex -jobname cdocscl1 \|\\
% |  "\input{childdoc.def}\childdocforward[cdocsamp]{cdocsch1}"|\\
% |latex -jobname cdocscl2 \|\\
% |  "\def\version{final}\input{childdoc.def}\childdocforward{cdocsch2}"|
% \end{tabular}
% \end{center}
% Note that the trailing backslash on each first line
% merely continues the input to the second line
% (for convenient cut ant paste).
% Furthermore, the command |latex| can be replaced by any
% of its alternative versions such as |pdflatex|.
%
% %%%%%%%%%%%%%%%%%%%%%%%%%%%%%%%%%%%%%%%%%%%%%%%%%%%%%%%%%%%%%%%%%%%%%%%%%%%%%%
% %%%%%%%%%%%%%%%%%%%%%%%%%%%%%%%%%%%%%%%%%%%%%%%%%%%%%%%%%%%%%%%%%%%%%%%%%%%%%%
% \section{Implementation}
%\iffalse
%<*package>
%\fi
%
% This section describes the definitions file |childdoc.def|.

% The definitions cannot be loaded using |\usepackage| or |\RequirePackage|
% which has a mechanism to prevent loading a style file more than once.
% When loading the definitions by means of |\input|
% multiple instances have to be prevented manually:
%\iffalse
%This code needs to be before the `\ProvidesFile' directive
%which is defined at the beginning of this file.
%Therefore it is also placed there and commented out here.
%</package>
%<*discard>
%\fi
%    \begin{macrocode}
\ifdefined\childdocmain\endinput\fi
%    \end{macrocode}
%\iffalse
%</discard>
%<*package>
%\fi
%
% \macro{\ifchilddoc}
% \macro{\ifchilddocmanual}
% The conditional |\ifchilddoc| tells whether a
% child (true) or main (false) document is being compiled.
% The conditional |\ifchilddocmanual| tells whether
% the |\includeonly| mechanism is used (false) or
% the selection of child files must be performed manually (true).
% The definitions initialise to false:
%    \begin{macrocode}
\newif\ifchilddoc
\newif\ifchilddocmanual
%    \end{macrocode}

% \macro{\childdocname}
% \macro{\childdocjob}
% The macro |\childdocname| stores the name of the main document
% to be compiled. The macro |\childdocjob| stores the name of
% the document on which the \LaTeX{} compiler was originally invoked.
% The content of |\jobname| cannot be compared
% to filenames specified in the source due to different catcodes.
% The following code rescans |\jobname|, stores the result
% in |\childdocname| and saves a copy in |\childdocjob|:
%    \begin{macrocode}
\edef\childdocname{\scantokens\expandafter{\jobname\noexpand}}
\let\childdocjob\childdocname
%    \end{macrocode}

% \macro{\childdocdisable}
% The macro |\childdocdisable| prevents the main file
% from being processed more than once.
% At this stage, the main document command |\childdocmain|
% is assumed to be called once again where it should do nothing.
% Any subsequent call to it should prevent
% a secondary processing of the main document
% It overwrites the forwarding commands
% |\childdocof| and |\childdocforward|
% with empty macros to prevent further inclusions of the main document:
%    \begin{macrocode}
\newcommand{\childdocdisable}
{
  \renewcommand{\childdocmain}[1]{\renewcommand{\childdocmain}[1]{\endinput}}
  \renewcommand{\childdocof}[1]{}
  \renewcommand{\childdocby}[2][]{}
  \renewcommand{\childdocforward}[2][]{}
  \renewcommand{\childdocdisable}{}
}
%    \end{macrocode}

% \macro{\childdocmain}
% The macro |\childdocmain| is to be called at the top of the main file
% with nothing or the main filename (without extension) as argument.
% First, it breaks loops.
% If the argument is not empty and does not match |\childdocname|
% (which is set by the first inclusion of |childdoc.def|),
% |\ifchilddoc| is set to true, |\includeonly| is applied to the child file
% and |\jobname| is set to the main file
% (for proper handling of |.aux| files):
%    \begin{macrocode}
\newcommand{\childdocmain}[1]
{
  \childdocdisable\childdocmain{}
  \if?#1?\else
    \begingroup
      \def\childdoctmp{#1}
      \ifx\childdoctmp\childdocname
        \def\childdoctmp{}
      \else
        \def\childdoctmp
        {
          \childdoctrue
          \includeonly{\childdocname}
          \def\childdocjob{#1}
          \def\jobname{#1}
        }
      \fi
      \expandafter
    \endgroup
    \childdoctmp
  \fi
}
%    \end{macrocode}

% \macro{\childdocof}
% The command |\childdocof| redirects
% compilation to the main file |#1|.
%    \begin{macrocode}
\newcommand{\childdocof}[1]
{
  \childdocdisable
  \childdoctrue
  \includeonly{\childdocname}
  \def\jobname{#1}
  \def\childdocjob{#1}
  \input{#1}
}
%    \end{macrocode}

% \macro{\childdocby}
% The command |\childdocby| ....
%    \begin{macrocode}
\newcommand{\childdocby}[2][]
{
  \childdocdisable
  \childdoctrue
  \childdocmanualtrue
  \if?#1?\else
    \def\jobname{#2}
  \fi
  \def\childdocjob{#2}
  \input{#2}
  \endinput
}
%    \end{macrocode}

% \macro{\childdocforward}
% The command |\childdocforward| redirects
% compilation to the main file or
% (if the optional argument is given) a child file.
% Parameters are set as if the main file
% or a child file starting with |\childdocof| was compiled.
% Then compilation is handed over to the main file:
%    \begin{macrocode}
\newcommand{\childdocforward}[2][]
{
  \begingroup
    \if?#1?
      \def\childdoctmp
      {
        \def\childdocname{#2}
        \def\childdocjob{#2}
        \def\jobname{#2}
        \input{#2}
        \endinput
      }
    \else
      \def\childdoctmp
      {
        \childdocdisable
        \def\childdocname{#2}
        \childdoctrue
        \includeonly{#2}
        \def\childdocjob{#1}
        \def\jobname{#1}
        \input{#1}
        \endinput
      }
    \fi
    \expandafter
  \endgroup
  \childdoctmp
}
%    \end{macrocode}

% \macro{\childdocforwardprefix}
% The command |\childdocforwardprefix| redirects
% compilation to the main or a child file by means of a pattern.
% The prefix |#1| in the current filename is replaced by |#2|
% and the suffix of the current filename is kept
% (it is assumed that the filename does not contain the substring `|~~~|'
% which is used as a delimiter).
% Compilation is handed over to the new file by |\childdocforward|:
%    \begin{macrocode}
\newcommand{\childdocforwardprefix}[3][]
{
  \begingroup
    \def\childdocextract #2##1~~~{\def\childdoctmp{\childdocforward[#1]{#3##1}}}
    \expandafter\childdocextract\childdocname~~~
    \expandafter
  \endgroup
  \childdoctmp
}
%    \end{macrocode}

% \macro{\childdoc}
% The deprecated macro |\childdoc| is a legacy version of |\childdocmain|:
%    \begin{macrocode}
\newcommand{\childdoc}{\childdocmain}
%    \end{macrocode}

% \macro{\childdocredirect}
% The deprecated macro |\childdocredirect| is a legacy version
% of |\childdocforward| and |\childdocforwardprefix|:
%    \begin{macrocode}
\newcommand{\childdocredirect}[2][]
{
  \begingroup
    \if?#1?
      \def\childdoctmp{\childdocforward{#2}}
    \else
      \def\childdoctmp{\childdocforwardprefix{#1}{#2}}
    \fi
    \expandafter
  \endgroup
  \childdoctmp
}
%    \end{macrocode}

%\iffalse
%</package>
%\fi
%
\endinput

\childdocof{cdocsamp}
%    \end{macrocode}

%\iffalse
%</samplechap1|samplechap2>
%\fi
%
%\iffalse
%<*samplechap1>
%\fi
% Some text for chapter 1:
%    \begin{macrocode}
\section{one}
some text in chapter one
%    \end{macrocode}

%\iffalse
%</samplechap1>
%\fi
% Some text for chapter 2:
%\iffalse
%<*samplechap2>
%\fi
%    \begin{macrocode}
\section{two}
more text in chapter two
%    \end{macrocode}

%\iffalse
%</samplechap2>
%\fi
%
% %%%%%%%%%%%%%%%%%%%%%%%%%%%%%%%%%%%%%%
% \paragraph{Part Include Files.}
%
% The include files are called |cdocspt3.tex| and |cdocspt4.tex|.
%
%\iffalse
%<*samplepart3|samplepart4>
%\fi

% Optional override for |\version| flag:
%    \begin{macrocode}
%%\providecommand{\version}{final}
%    \end{macrocode}

% Include the main document:
%    \begin{macrocode}
% \iffalse
%
% childdoc.dtx Copyright (C) 2017-2018 Niklas Beisert
%
% This work may be distributed and/or modified under the
% conditions of the LaTeX Project Public License, either version 1.3
% of this license or (at your option) any later version.
% The latest version of this license is in
%   http://www.latex-project.org/lppl.txt
% and version 1.3 or later is part of all distributions of LaTeX
% version 2005/12/01 or later.
%
% This work has the LPPL maintenance status `maintained'.
%
% The Current Maintainer of this work is Niklas Beisert.
%
% This work consists of the files childdoc.dtx and childdoc.ins
% and the derived files childdoc.def and cdocsamp.tex with
% cdocsch1.tex, cdocsch2.tex, cdocsdrf.tex, cdocsfn1.tex, cdocsfn2.tex.
%
%<package>\ifdefined\childdocmain\endinput\fi
%<package>\ProvidesFile{childdoc.def}[2018/12/30 v2.0 child document driver]
%<samplemain>\ProvidesFile{cdocsamp.tex}[2018/12/30 v2.0 sample for childdoc]
%<*driver>
%\ProvidesFile{childdoc.drv}[2018/12/30 v2.0 childdoc reference manual file]
\PassOptionsToClass{10pt,a4paper}{article}
\documentclass{ltxdoc}

\usepackage[margin=35mm]{geometry}
\usepackage{hyperref}
\usepackage{hyperxmp}
\usepackage[usenames]{color}

\hypersetup{colorlinks=true}
\hypersetup{pdfstartview=FitH}
\hypersetup{pdfpagemode=UseNone}
\hypersetup{pdfsource={}}
\hypersetup{pdflang={en-UK}}
\hypersetup{pdfcopyright={Copyright 2017-2018 Niklas Beisert.
  This work may be distributed and/or modified under the
  conditions of the LaTeX Project Public License, either version 1.3
  of this license or (at your option) any later version.}}
\hypersetup{pdflicenseurl={http://www.latex-project.org/lppl.txt}}
\hypersetup{pdfcontactaddress={ETH Zurich, ITP, HIT K,
  Wolfgang-Pauli-Strasse 27}}
\hypersetup{pdfcontactpostcode={8093}}
\hypersetup{pdfcontactcity={Zurich}}
\hypersetup{pdfcontactcountry={Switzerland}}
\hypersetup{pdfcontactemail={nbeisert@itp.phys.ethz.ch}}
\hypersetup{pdfcontacturl={http://people.phys.ethz.ch/\xmptilde nbeisert/}}

\newcommand{\secref}[1]{\hyperref[#1]{section \ref*{#1}}}

\parskip1ex
\parindent0pt
\let\olditemize\itemize
\def\itemize{\olditemize\parskip0pt}

\begin{document}

\title{The \textsf{childdoc} Package}
\hypersetup{pdftitle={The childdoc Package}}
\author{Niklas Beisert\\[2ex]
  Institut f\"ur Theoretische Physik\\
  Eidgen\"ossische Technische Hochschule Z\"urich\\
  Wolfgang-Pauli-Strasse 27, 8093 Z\"urich, Switzerland\\[1ex]
  \href{mailto:nbeisert@itp.phys.ethz.ch}
  {\texttt{nbeisert@itp.phys.ethz.ch}}}
\hypersetup{pdfauthor={Niklas Beisert}}
\hypersetup{pdfsubject={Manual for the LaTeX2e Package childdoc}}
\date{30 December 2018, \textsf{v2.0}}
\maketitle

\begin{abstract}\noindent
\textsf{childdoc} is a \LaTeXe{} package
that enables the direct compilation
of document sections included by |\include|
to individual files.
\end{abstract}

\begingroup
\parskip0ex
\tableofcontents
\endgroup

%%%%%%%%%%%%%%%%%%%%%%%%%%%%%%%%%%%%%%%%%%%%%%%%%%%%%%%%%%%%%%%%%%%%%%%%%%%%%%%%
%%%%%%%%%%%%%%%%%%%%%%%%%%%%%%%%%%%%%%%%%%%%%%%%%%%%%%%%%%%%%%%%%%%%%%%%%%%%%%%%
\section{Introduction}

\LaTeX{} provides a mechanism to structure a large document (such as a book)
into a main file and several child files (containing the chapters)
using the |\include| command.
This mechanism is beneficial for documents
which span hundreds of pages in order to
make the source file(s) more manageable.
Moreover, compilation can be restricted to
selected child files by means of the |\includeonly| command.
The latter feature can be used to reduce the compilation time while editing
(this was significantly more useful in the earlier days of \LaTeX{})
or to generate a smaller document which is easier to navigate.
Another application of |\includeonly| is to generate
documents consisting of selected parts of the complete document.

However, there are a few drawbacks of the plain |\include| mechanism:
\begin{itemize}
\item
The child files cannot be compiled on their own,
they can only be compiled via the main file.
A naive editing environment
(such as a text editor with an option
to have the current file processed by \LaTeX)
may require one to switch to the main file before compiling;
attempting to compile the child file produces errors.
\item
The main file must be modified (each time)
to adjust the |\includeonly| command
to the present needs. This easily leaves the main file in a messy state.
\item
The generated document will always carry the filename
of the main document. This is inconvenient if
several child files are to be compiled and
to be kept for distribution.
\end{itemize}

The present package provides a simple interface
to make child files individually compilable by \LaTeX{}.
Compiling a child file then has the same effect as compiling
the main file with an |\includeonly| command
to select the appropriate child.
Moreover the generated document will carry the name of the child
rather than the main file.
This resolves all three above issues.

This feature is meant to make the editing of books,
thesis documents and lecture notes somewhat more convenient.
However, the package can also be used efficiently for
composing a series of documents (such as exercise sheets)
which are typically distributed individually.
It then assists the author in generating the individual documents
(potentially in different versions)
as well as a document containing the collected series.
Another application is in developing style files
or other kinds of included material
where compilation of the style file could redirect
to a sample or test file.

%%%%%%%%%%%%%%%%%%%%%%%%%%%%%%%%%%%%%%%%%%%%%%%%%%%%%%%%%%%%%%%%%%%%%%%%%%%%%%%%
%%%%%%%%%%%%%%%%%%%%%%%%%%%%%%%%%%%%%%%%%%%%%%%%%%%%%%%%%%%%%%%%%%%%%%%%%%%%%%%%
\section{Usage}

First of all, the package \textsf{childdoc} is \emph{not} a standard
\LaTeXe{} |.sty| style file! Therefore it needs to be invoked in
a non-standard way.

%%%%%%%%%%%%%%%%%%%%%%%%%%%%%%%%%%%%%%%%%%%%%%%%%%%%%%%%%%%%%%%%%%%%%%%%%%%%%%%%
\subsection{Included Files}
\label{sec:include}

%%%%%%%%%%%%%%%%%%%%%%%%%%%%%%%%%%%%%%%%
\DescribeMacro{\childdocmain}
To use the package, add the commands
\begin{center}
\begin{tabular}{l}
|\input{childdoc.def}|\\
|\childdocmain{}|\\
\end{tabular}
\end{center}
at the very top of the main \LaTeX{} file,
in particular \emph{before} the |\documentclass| statement!
The argument of |\childdocmain| should be left empty
(but it must be present).

%%%%%%%%%%%%%%%%%%%%%%%%%%%%%%%%%%%%%%%%
\DescribeMacro{\childdocof}
Furthermore, add the commands
\begin{center}
\begin{tabular}{l}
|\input{childdoc.def}|\\
|\childdocof{|\textit{main}|}|\\
\end{tabular}
\end{center}
at the top of every child file \textit{child}
which is included by |\include{|\textit{child}|}|
from within the main file
(or at least for those files to be compiled individually).
The argument \textit{main} must be the filename of the main file.

There are a couple of
considerations in setting up the main and child documents:

%%%%%%%%%%%%%%%%%%%%%%%%%%%%%%%%%%%%%%%%
\paragraph{Restrictions.}

Please note the following restrictions:
\begin{itemize}
\item
|\childdocmain| must be called with one argument \textit{main}
to ensure compatibility with earlier version of the package.
It must either be empty (|\childdocmain{}|)
or precisely match the filename of the main file in which it is specified.
See \secref{sec:detection} for further information.
\item
The filename \textit{main} must be specified without the |.tex| extension.
\item
The filename \textit{main} is case sensitive
(even in case-insensitive file systems)
due to internal string comparison.
\item
The argument \textit{main} should be fully expanded, it cannot be a macro.
\item
Subdirectories and special characters should be avoided in filenames.
\item
The command |\childdocmain{|\textit{main}|}| must be followed by a whitespace.
It should not be followed immediately by another command
or by a comment mark `|%|'.
This is because the \TeX{} parser reads the token immediately following
the argument of |\childdocmain| and puts it
at the beginning of every child section;
however, a white\-space is ignored.
\end{itemize}

%%%%%%%%%%%%%%%%%%%%%%%%%%%%%%%%%%%%%%%%
\paragraph{Content of Main File.}

It is advisable to place all content in the child files included by |\include|.
Any output contained in the main file will appear in all child documents
unless suppressed manually;
it cannot be suppressed automatically by the |\includeonly| directive
and thus should normally be avoided.
A method to include some content in the main file
by means of conditional processing is described in \secref{sec:conditional}.

%%%%%%%%%%%%%%%%%%%%%%%%%%%%%%%%%%%%%%%%
\paragraph{Page Numbering.}

When only a part of the document is compiled,
the appropriate numbering of pages
(as well as other status parameters)
is determined from the |.aux| files.
The latter contain information from previous passes.
However this information needs to propagate through
all intermediate child documents.
Therefore the page numbering in child documents may well
be inconsistent until the complete document is compiled at least once.

A useful (if unconventional) way to always ensure a consistent
page numbering is to restart the numbering in each child document
and denote the pages by `\textit{child}|.|\textit{page}'
where \textit{child} represents the chapter/section number of the child file.
This can be achieved by the command
|\numberwithin{page}{|\textit{child}|}|
of the \textsf{amsmath} package
where \textit{child} can be |chapter| or |section|
depending on the chosen structuring.
Alternatively, one can modify the macro |\thepage| appropriately
and reset the counter |page| at the start of each child file.

%%%%%%%%%%%%%%%%%%%%%%%%%%%%%%%%%%%%%%%%%%%%%%%%%%%%%%%%%%%%%%%%%%%%%%%%%%%%%%%%
\subsection{Conditional Processing}
\label{sec:conditional}

The package provides a mechanism to compile different versions
of a document. To customise the versions further some conditional processing
can come in handy to distinguish which version is being compiled.
The package provides two macros to describe the compilation context:

%%%%%%%%%%%%%%%%%%%%%%%%%%%%%%%%%%%%%%%%
\DescribeMacro{\ifchilddoc}
The conditional |\ifchilddoc| distinguishes between the compilation of
child documents and the main document:
%
\begin{center}
|\ifchilddoc |\textit{child-code}| |[|\||else |\textit{main-code}]| \||fi|
\end{center}

%%%%%%%%%%%%%%%%%%%%%%%%%%%%%%%%%%%%%%%%
\DescribeMacro{\childdocname}
\DescribeMacro{\childdocjob}
The macro |\childdocname| contains the filename (without extension)
of the main or child file being processed.
Note that |\childdocjob| will always contain the name of the main file.

%%%%%%%%%%%%%%%%%%%%%%%%%%%%%%%%%%%%%%%%
\paragraph{Title Page.}

Conditional processing can be used to include a title or banner page
in the main document when proper precautions are taken.
Importantly, the code in the main file should ensure that the page counter
(as well as other status parameters which are stored in the |.aux| files)
takes the same value after the conditional processing.
Otherwise the page numbers may take divergent values
depending on which part is compiled.

For example, a title page could be declared by:
%
\begin{center}
\begin{tabular}{l}
|\ifchilddoc\||else|\\
|\addtocounter{page}{-1}|\\
\textit{code for title page}\\
|\newpage|\\
|\||fi|
\end{tabular}
\end{center}
%
A banner page for the child documents can be generated by:
%
\begin{center}
\begin{tabular}{l}
|\ifchilddoc|\\
|\addtocounter{page}{-1}|\\
\textit{code for banner page}\\
|\newpage|\\
|\||fi|
\end{tabular}
\end{center}
%
Here one could write a message such as:
\begin{center}
|This is the part \childdocname{} of \childdocjob{}.|
\end{center}

%%%%%%%%%%%%%%%%%%%%%%%%%%%%%%%%%%%%%%%%%%%%%%%%%%%%%%%%%%%%%%%%%%%%%%%%%%%%%%%%
\subsection{Flags}
\label{sec:flags}

The package makes it easy to generate different versions
of the main or child documents.
To this end compilation flags can be defined
and assigned different default values.
They will be particularly useful in conjunction
with the forwarding mechanism described in \secref{sec:forward}.

For example, it may be useful to have a flag |\version|
which can be set to |draft| or |final|.
The document source will contain some conditional code
depending on the value of |\version|.
Suppose further, the flag should default to |final| for the main file
and to |draft| for child files
which is a natural assignment for editing the document.
This is achieved by placing the following code
in the preamble of the main document
(below the |\childdocmain| directive):
%
\begin{center}
\begin{tabular}{l}
|\ifchilddoc|\\
|\providecommand{\version}{draft}|\\
|\||else|\\
|\providecommand{\version}{final}|\\
|\||fi|
\end{tabular}
\end{center}
%
The definition by |\providecommand| makes sure
that previous definitions are not overwritten.
Further statements |\providecommand{\version}{...}|
can thus be added before the above code to override it.

For the main file, one might add a line
(between |\childdocmain| and the above block)
%
\begin{center}
|%\ifchilddoc\||else\providecommand{\version}{draft}\||fi|
\end{center}
%
which can be uncommented to produce a draft version.
Likewise one can add a line to the very top of a child file
(above the |\childdocof{|\textit{main}|}| directive)
%
\begin{center}
|%\providecommand{\version}{final}|
\end{center}
%
which can be uncommented to produce the final version of this child document.

%%%%%%%%%%%%%%%%%%%%%%%%%%%%%%%%%%%%%%%%%%%%%%%%%%%%%%%%%%%%%%%%%%%%%%%%%%%%%%%%
\subsection{Forwarding}
\label{sec:forward}

Different versions of the main or child documents
using compilation flags as described in \secref{sec:flags}
can be (permanently) stored in different files
for convenient compilation, viewing and distribution.
To this end, the package defines a command
to pass on compilation to a different file:

%%%%%%%%%%%%%%%%%%%%%%%%%%%%%%%%%%%%%%%%
\DescribeMacro{\childdocforward}
The command |\childdocforward| redirects processing to
another source file:
%
\begin{center}
\begin{tabular}{l}
|\input{childdoc.def}|\\
|\childdocforward[|\textit{main}|]{|\textit{dest}|}|\\
\end{tabular}
\end{center}
%
The argument \textit{dest} is the destination file
(without extension).
It should be the main file or one of the child files.
Note that further \textsf{childdoc} directives
such as |\childdocof| and |\childdocforward|
in the indicated file will be processed in this form.
The optional argument \textit{main}
passes on directly to the main file \textit{main}
while pretending to compile the child \textit{dest}.
This form behaves as if \textit{dest}
issues |\childdocof{|\textit{main}|}| right away,
and no further \textsf{childdoc} directives will be processed.

%%%%%%%%%%%%%%%%%%%%%%%%%%%%%%%%%%%%%%%%
\DescribeMacro{\...prefix}
In the alternative form |\childdocforwardprefix|,
%
\begin{center}
\begin{tabular}{l}
|\input{childdoc.def}|\\
|\childdocforwardprefix[|\textit{main}|]{|\textit{prefix}|}{|\textit{dest}|}|
\end{tabular}
\end{center}
%
the destination file is determined by a pattern
depending on the current file:
To make this work, the current file must be called
`{\textit{prefix}\hspace{0.2em}\textit{suffix}}'
with \textit{prefix} matching precisely the argument.
Processing is then passed on to the file
`{\textit{dest}\hspace{0.2em}\textit{suffix}}'.
Surely, the same effect is achieved by
directly specifying the
argument `{\textit{dest}\hspace{0.2em}\textit{suffix}}'
in the first form.
However, that requires to set up a different file
for each child. With the alternative form of the command
all these files can have exactly the same content
which simplifies setting them up and maintaining them.

For example, the following file |draft.tex|
with a compilation flag |\version| as described in \secref{sec:flags}
compiles the main document as a draft:
%
\begin{center}
\begin{tabular}{l}
|\def\version{draft}|\\
|\input{childdoc.def}|\\
|\childdocforward{|\textit{main}|}|
\end{tabular}
\end{center}
%
Likewise, the following files |final|\textit{nn}|.tex|
compile the final version of the child document
|child|\textit{nn}|.tex|:
%
\begin{center}
\begin{tabular}{l}
|\def\version{final}|\\
|\input{childdoc.def}|\\
|\childdocforwardprefix{final}{child}|
\end{tabular}
\end{center}
%

Note that when several versions of a main file and/or of each child file
are to be generated, it may be convenient to set up a |Makefile| or
shell script to automatise the process.

%%%%%%%%%%%%%%%%%%%%%%%%%%%%%%%%%%%%%%%%%%%%%%%%%%%%%%%%%%%%%%%%%%%%%%%%%%%%%%%%
\subsection{Command Line Processing}
\label{sec:commandline}

The effect of redirection files can also be achieved by invoking
the \LaTeX{} compiler with a more elaborate command line.
Most conveniently this should be done as part
of a shell script or a |Makefile|.

When using \textsf{childdoc} in the main file, the following
command lines effectively perform a redirection
(note that depending on the shell being used,
backslashes may have to be doubled: `|\|' $\to$ `|\\|'):
%
\begin{center}
|... -jobname "|\textit{target}|" |\\|"|[\textit{flags}]%
|\input{childdoc.def}\childdocforward[|\textit{main}|]{|\textit{dest}|}"|
\end{center}
%
Here \textit{target} is the name of the output file,
\textit{main} is the name of the main file
and \textit{dest} is the name of the main or child file to be processed
(all filenames without extensions).
The optional argument \textit{main} can be omitted
if \textit{main} matches \textit{dest}.
Optionally, compilation \textit{flags} can be defined via |\def| commands.
This command line makes the \TeX{} engine believe
it is compiling the file \textit{target}
whose content is specified as the latter parameter.
The provided code then forwards the processing to
\textit{main} or \textit{dest} as described in \secref{sec:forward}.

%%%%%%%%%%%%%%%%%%%%%%%%%%%%%%%%%%%%%%%%%%%%%%%%%%%%%%%%%%%%%%%%%%%%%%%%%%%%%%%%
\subsection{Include by Input}
\label{sec:input}

Including child documents by |\include| has some restrictions by design.
Most notably, the content of a child document always occupies
its own set of pages; pages cannot be shared between child documents.
Usually, this behaviour makes perfect sense
because each child document contain an essential part of the document.
However, in some situations it may be desirable to compose
a document from a collection of parts
without having mandatory page breaks between then.
For this case, the package
provides a mechanism to include parts
by |\input| which can also be processed individually.
However, by construction this mechanism
requires manual handling of the content to be output.

%%%%%%%%%%%%%%%%%%%%%%%%%%%%%%%%%%%%%%%%
\DescribeMacro{\ifchilddocmanual}
The main file should be prepared as usual, see \secref{sec:include}.
However, the document body must make a distinction
between processing of an individual part and of the main document, e.g.:
%
\begin{center}
\begin{tabular}{l}
|\ifchilddocmanual|\\
|\input{\childdocname}|\\
|\||else|\\
\textit{document body with }|\input{|\textit{part}|}|\\
|\||fi|
\end{tabular}
\end{center}
%
The conditional |\ifchilddocmanual| is true whenever
a part to be included by |\input| is being compiled,
and the name of the part is stored in |\childdocname|.

%%%%%%%%%%%%%%%%%%%%%%%%%%%%%%%%%%%%%%%%
\DescribeMacro{\childdocby}
Each part to be included by |\input| should start with:
%
\begin{center}
\begin{tabular}{l}
|\input{childdoc.def}|\\
|\childdocby{|\textit{main}|}|\\
\end{tabular}
\end{center}
%
The directive |\childdocby| is similar to |\childdocof|
described in \secref{sec:include},
but the subsequent selection of content must be done manually.
To that end, both |\ifchilddoc| and |\ifchilddocmanual|
will be true upon processing of a part,
and the name of the part is stored in |\childdocname|.
Note that |\jobname| will be set to the filename of the current part
so that each part receives an individual |.aux| file
that does not interfere with the |.aux| file(s) of the main document.
This behaviour can be altered by the alternative form
|\childdocby[*]{|\textit{main}|}| (with a non-empty optional argument)
which uses the |.aux| file of the main document
by setting |\jobname| to \textit{main}.

%%%%%%%%%%%%%%%%%%%%%%%%%%%%%%%%%%%%%%%%%%%%%%%%%%%%%%%%%%%%%%%%%%%%%%%%%%%%%%%%
\subsection{Driver Development}
\label{sec:driver}

The \textsf{childdoc} mechanism can also be use for the development
of definition files such as \LaTeX{} styles or classes.
This case differs from the above setup with multiple parts
included by |\include| in that no |\includeonly| should be invoked.
This can be achieved by starting the include file
(before |\ProvidesPackage|) with:
%
\begin{center}
\begin{tabular}{l}
|\input{childdoc.def}|\\
|\childdocforward{|\textit{main}|}|\\
\end{tabular}
\end{center}
%
or alternatively with:
%
\begin{center}
\begin{tabular}{l}
|\input{childdoc.def}|\\
|\childdocby{|\textit{main}|}|\\
\end{tabular}
\end{center}
%
Both forms have slightly different effects as described above.
The main file is prepared as usual, see \secref{sec:include}.

%%%%%%%%%%%%%%%%%%%%%%%%%%%%%%%%%%%%%%%%%%%%%%%%%%%%%%%%%%%%%%%%%%%%%%%%%%%%%%%%
\subsection{Legacy Detection}
\label{sec:detection}

The directive |\childdocmain| in the main file can detect
whether the complete document or merely a child is to be compiled
even without using the directive |\childdocof|.
This method is deprecated because it is less robust
and there is no compelling reason to use it;
it is merely provided for backward compatibility
and it may be removed in future versions.

If the detection mechanism is to be used,
it is mandatory to correctly specify
the filename of the main file as the argument of |\childdocmain|:
%
\begin{center}
\begin{tabular}{l}
|\input{childdoc.def}|\\
|\childdocmain{|\textit{main}|}|\\
\end{tabular}
\end{center}
%
If |\jobname| does not match the argument \textit{main} of |\childdocmain|,
it is assumed that |\jobname| points to the child file to be compiled.
When using |\childdocmain| with the main file specified as argument,
it suffices to start a child file
with just |\input{|\textit{main}|}|
without loading of the package and using |\childdocof|.
If instead all processing is done
with the appropriate \textsf{childdoc} directives,
the argument of \textit{main} of |\childdocmain| can be empty.

An alternative version of the command line processing described
in \secref{sec:commandline} using the detection mechanism reads:
%
\begin{center}
|... -jobname "|\textit{target}|" "|[\textit{flags}]%
[|\def\jobname{|\textit{dest}|}|]|\input{|\textit{main}|}"|
\end{center}

%%%%%%%%%%%%%%%%%%%%%%%%%%%%%%%%%%%%%%%%%%%%%%%%%%%%%%%%%%%%%%%%%%%%%%%%%%%%%%%%
\subsection{Manual Code}
\label{sec:manual}

In case one cannot be certain whether the definitions file |childdoc.def|
is installed on the target \TeX{} distribution
and one prefers not to ship it,
it is conceivable to paste a few relevant commands into the sources.

To that end, drop all statements |\input{childdoc.def}|
and perform the replacements as outlined below.
Instead of |\childdocmain{|\textit{main}|}| add the following code
to the top of the main file:
%
\begin{center}
\begin{tabular}{l}
|\||ifdefined\childdocname\endinput\||fi\newif\ifchilddoc|\\
|\edef\childdocname{\scantokens\expandafter{\jobname\noexpand}}|\\
|\def\childdocmain{|\textit{main}|}\||ifx\childdocmain\childdocname\||else|\\
|\childdoctrue\includeonly{\childdocname}\let\jobname\childdocmain\||fi|\\
\end{tabular}
\end{center}
%
Instead of |\childdocof{|\textit{main}|}| just include the main file
at the top of each child file:
%
\begin{center}
|\input{|\textit{main}|}|
\end{center}
%
A simple redirection |\childdocforward{|\textit{dest}|}| is achieved by:
%
\begin{center}
|\def\jobname{|\textit{dest}|}\input{\jobname}|
\end{center}
%
The redirection with prefix
|\childdocforwardprefix[|\textit{prefix}|]{|\textit{dest}|}|
is accomplished by:
%
\begin{center}
\begin{tabular}{l}
|{\edef\jobname{\scantokens\expandafter{\jobname\noexpand}}|\\
|\def\redirectjob |\textit{prefix}|#1~~~{\gdef\jobname{|\textit{dest}|#1}}|\\
|\expandafter\redirectjob\jobname~~~}\input{\jobname}|
\end{tabular}
\end{center}

In an alternative approach,
child documents can be compiled by a specific command line
without additional code or specific definitions:
%
\begin{center}
|... -jobname "|\textit{target}|" "|[\textit{flags}]%
|\includeonly{|\textit{dest}|}\input{|\textit{main}|}"|
\end{center}
%

%%%%%%%%%%%%%%%%%%%%%%%%%%%%%%%%%%%%%%%%%%%%%%%%%%%%%%%%%%%%%%%%%%%%%%%%%%%%%%%%
%%%%%%%%%%%%%%%%%%%%%%%%%%%%%%%%%%%%%%%%%%%%%%%%%%%%%%%%%%%%%%%%%%%%%%%%%%%%%%%%
\section{Information}

%%%%%%%%%%%%%%%%%%%%%%%%%%%%%%%%%%%%%%%%%%%%%%%%%%%%%%%%%%%%%%%%%%%%%%%%%%%%%%%%
\subsection{Copyright}

Copyright \copyright{} 2017--2018 Niklas Beisert

This work may be distributed and/or modified under the
conditions of the \LaTeX{} Project Public License, either version 1.3
of this license or (at your option) any later version.
The latest version of this license is in
  \url{http://www.latex-project.org/lppl.txt}
and version 1.3 or later is part of all distributions of \LaTeX{}
version 2005/12/01 or later.

This work has the LPPL maintenance status `maintained'.

The Current Maintainer of this work is Niklas Beisert.

This work consists of the files |README.txt|, |childdoc.ins| and |childdoc.dtx|
as well as the derived files |childdoc.def|, |cdocsamp.tex|
with |cdocsch1.tex|, |cdocsch2.tex|, |cdocspt3.tex|, |cdocspt4.tex|,
|cdocsdrf.tex|, |cdocsfn1.tex|, |cdocsfn2.tex|
as well as |childdoc.pdf|.

%%%%%%%%%%%%%%%%%%%%%%%%%%%%%%%%%%%%%%%%%%%%%%%%%%%%%%%%%%%%%%%%%%%%%%%%%%%%%%%%
\subsection{Files and Installation}

The package consists of the files:
%
\begin{center}
\begin{tabular}{ll}
    |README.txt|   & readme file \\
    |childdoc.ins| & installation file \\
    |childdoc.dtx| & source file \\
    |childdoc.def| & definition file \\
    |cdocsamp.tex| & sample main file \\
    |cdocsch1.tex| & sample include file \\
    |cdocsch2.tex| & sample include file \\
    |cdocspt3.tex| & sample part file \\
    |cdocspt4.tex| & sample part file \\
    |cdocsdrf.tex| & sample redirection file \\
    |cdocsfn1.tex| & sample redirection file \\
    |cdocsfn2.tex| & sample redirection file \\
    |childdoc.pdf| & manual
\end{tabular}
\end{center}
%
The distribution consists of the files
|README.txt|, |childdoc.ins| and |childdoc.dtx|.
%
\begin{itemize}
\item
Run (pdf)\LaTeX{} on |childdoc.dtx|
to compile the manual |childdoc.pdf| (this file).
\item
Run \LaTeX{} on |childdoc.ins| to create the definitions file |childdoc.def|
and the sample |cdocsamp.tex| with include files
|cdocsch1.tex|, |cdocsch2.tex|, |cdocspt3.tex|, |cdocspt4.tex|,
|cdocsdrf.tex|, |cdocsfn1.tex|, |cdocsfn2.tex|.
Then copy the file |childdoc.def| to an appropriate directory of your \LaTeX{}
distribution, e.g.\ \textit{texmf-root}|/tex/latex/childdoc|.
\end{itemize}

%%%%%%%%%%%%%%%%%%%%%%%%%%%%%%%%%%%%%%%%%%%%%%%%%%%%%%%%%%%%%%%%%%%%%%%%%%%%%%%%
\subsection{Related CTAN Packages}

There are several other packages which offer a similar functionality:
%
\begin{itemize}
\item
The packages
\href{http://ctan.org/pkg/docmute}{\textsf{docmute}},
\href{http://ctan.org/pkg/includex}{\textsf{includex}} and
\href{http://ctan.org/pkg/standalone}{\textsf{standalone}}
provide commands to include only the document body of
a child file thus allowing both files to be compiled individually.
\item
The packages \href{http://ctan.org/pkg/subdocs}{\textsf{subdocs}}
and \href{http://ctan.org/pkg/subfiles}{\textsf{subfiles}}
provide structures in which the main and child documents can be
encapsulated and allowing them to be compiled individually.
The inclusion mechanism is different from the conventional |\include|.
\item
The package \href{http://ctan.org/pkg/combine}{\textsf{combine}}
is an elaborate solution to combine several documents into one.
\end{itemize}
%
See also the CTAN topic \href{http://ctan.org/topic/subdocs}{\textsf{subdocs}}
for further related packages.
The present package differs from the above solutions in that
a document structure constructed with the conventional |\include| mechanism
just needs two extra commands at the top of every file
such that all constituent files can be compiled individually.

%%%%%%%%%%%%%%%%%%%%%%%%%%%%%%%%%%%%%%%%%%%%%%%%%%%%%%%%%%%%%%%%%%%%%%%%%%%%%%%%
%\subsection{Feature Suggestions}
%
%The following is a list of features which may be useful for future
%versions of this package:
%%
%\begin{itemize}
%\item
%\ldots
%\end{itemize}

%%%%%%%%%%%%%%%%%%%%%%%%%%%%%%%%%%%%%%%%%%%%%%%%%%%%%%%%%%%%%%%%%%%%%%%%%%%%%%%%
\subsection{Revision History}

%%%%%%%%%%%%%%%%%%%%%%%%%%%%%%%%%%%%%%%%
\paragraph{v2.0:} 2018/12/30

\begin{itemize}
\item
immediate forward processing
\item
added |\childdocby| mechanism
\item
manual restructured
\end{itemize}

%%%%%%%%%%%%%%%%%%%%%%%%%%%%%%%%%%%%%%%%
\paragraph{v1.6:} 2018/01/17

\begin{itemize}
\item
application for development of include files
\item
corrections to manual
\end{itemize}

%%%%%%%%%%%%%%%%%%%%%%%%%%%%%%%%%%%%%%%%
\paragraph{v1.5:} 2017/05/21

\begin{itemize}
\item
more complete structuring introduced
\item
|\childdocof| introduced
\item
|\childdoc| renamed to |\childdocmain|
\item
|\childredirect| renamed to |\childdocforward| and |\childdocforwardprefix|
and functionality expanded
\end{itemize}

%%%%%%%%%%%%%%%%%%%%%%%%%%%%%%%%%%%%%%%%
\paragraph{v1.0:} 2017/04/27

\begin{itemize}
\item
manual and install package
\item
first version published on CTAN
\end{itemize}

%%%%%%%%%%%%%%%%%%%%%%%%%%%%%%%%%%%%%%%%
\paragraph{v0.6:} 2017/04/26

\begin{itemize}
\item
redirection mechanism added
\end{itemize}

%%%%%%%%%%%%%%%%%%%%%%%%%%%%%%%%%%%%%%%%
\paragraph{v0.5:} 2017/04/26

\begin{itemize}
\item
functionality in definition file
\end{itemize}


%%%%%%%%%%%%%%%%%%%%%%%%%%%%%%%%%%%%%%%%%%%%%%%%%%%%%%%%%%%%%%%%%%%%%%%%%%%%%%%%
%%%%%%%%%%%%%%%%%%%%%%%%%%%%%%%%%%%%%%%%%%%%%%%%%%%%%%%%%%%%%%%%%%%%%%%%%%%%%%%%
%%%%%%%%%%%%%%%%%%%%%%%%%%%%%%%%%%%%%%%%%%%%%%%%%%%%%%%%%%%%%%%%%%%%%%%%%%%%%%%%
\appendix

\settowidth\MacroIndent{\rmfamily\scriptsize 000\ }

 \DocInput{childdoc.dtx}

\end{document}
%</driver>
% \fi
%
% %%%%%%%%%%%%%%%%%%%%%%%%%%%%%%%%%%%%%%%%%%%%%%%%%%%%%%%%%%%%%%%%%%%%%%%%%%%%%%
% %%%%%%%%%%%%%%%%%%%%%%%%%%%%%%%%%%%%%%%%%%%%%%%%%%%%%%%%%%%%%%%%%%%%%%%%%%%%%%
% \section{Sample}
%\iffalse
%<*samplemain>
%\fi
%
% The following presents a sample document
% with two chapters, two parts, a title page,
% a compile flag as well as three forwarding files to set the flag.
% It consists of eight |.tex| files:
% \begin{center}
% \begin{tabular}{ll}
% |cdocsamp.tex|&main file\\
% |cdocsch1.tex|&include file for chapter 1\\
% |cdocsch2.tex|&include file for chapter 2\\
% |cdocspt3.tex|&include file for part 3\\
% |cdocspt4.tex|&include file for part 4\\
% |cdocsdrf.tex|&forwarding file for main file in draft mode\\
% |cdocsfi1.tex|&forwarding file for final version of chapter 1\\
% |cdocsfi2.tex|&forwarding file for final version of chapter 2\\
% \end{tabular}
% \end{center}
% Each of the eight files can be compiled directly by the \LaTeX{} compiler.
%
% %%%%%%%%%%%%%%%%%%%%%%%%%%%%%%%%%%%%%%
% \paragraph{Main File.}
%
% The main file is called |cdocsamp.tex|.
%
% Load the \textsf{childdoc} definitions and
% declare the filename for the main document:
%    \begin{macrocode}
\input{childdoc.def}
\childdocmain{}
%    \end{macrocode}

% Optional override for |\version| flag:
%    \begin{macrocode}
%%\ifchilddoc\else\providecommand{\version}{draft}\fi
%    \end{macrocode}

% Define the default values for the |\version| flag
% (|final| for the main file and |draft| for childs):
%    \begin{macrocode}
\ifchilddoc
\providecommand{\version}{draft}
\else
\providecommand{\version}{final}
\fi
%    \end{macrocode}

% Load the standard document class:
%    \begin{macrocode}
\documentclass[12pt]{article}
%    \end{macrocode}

% Start the document body:
%    \begin{macrocode}
\begin{document}
%    \end{macrocode}

% Declare a title page.
% Print title, part of document being processed and version flag:
%    \begin{macrocode}
\addtocounter{page}{-1}
\begin{center}
{\LARGE\bfseries{}childdoc example\par}
\vspace{1cm}
\ifchilddoc
\ifchilddocmanual part\else chapter\fi:
`\childdocname' of `\childdocjob'\par
\else
main document: `\childdocjob'\par
\fi
version: \version\par
\end{center}
\newpage
%    \end{macrocode}

% Manually include selected file,
% otherwise process as usual:
%    \begin{macrocode}
\ifchilddocmanual
\section*{part `\childdocname'}
\input{\childdocname}
\else
%    \end{macrocode}

% Include the two chapters:
%    \begin{macrocode}
\include{cdocsch1}
\include{cdocsch2}
%    \end{macrocode}

% Include the two parts unless only chapters should be displayed:
%    \begin{macrocode}
\ifchilddoc\else
\section{part three}
\input{cdocspt3}
\section{part four}
\input{cdocspt4}
\fi
%    \end{macrocode}

% Process as usual until here:
%    \begin{macrocode}
\fi
%    \end{macrocode}

% End of document body:
%    \begin{macrocode}
\end{document}
%    \end{macrocode}
%\iffalse
%</samplemain>
%\fi
%
% %%%%%%%%%%%%%%%%%%%%%%%%%%%%%%%%%%%%%%
% \paragraph{Chapter Include Files.}
%
% The include files are called |cdocsch1.tex| and |cdocsch2.tex|.
%
%\iffalse
%<*samplechap1|samplechap2>
%\fi

% Optional override for |\version| flag:
%    \begin{macrocode}
%%\providecommand{\version}{final}
%    \end{macrocode}

% Include the main document:
%    \begin{macrocode}
\input{childdoc.def}
\childdocof{cdocsamp}
%    \end{macrocode}

%\iffalse
%</samplechap1|samplechap2>
%\fi
%
%\iffalse
%<*samplechap1>
%\fi
% Some text for chapter 1:
%    \begin{macrocode}
\section{one}
some text in chapter one
%    \end{macrocode}

%\iffalse
%</samplechap1>
%\fi
% Some text for chapter 2:
%\iffalse
%<*samplechap2>
%\fi
%    \begin{macrocode}
\section{two}
more text in chapter two
%    \end{macrocode}

%\iffalse
%</samplechap2>
%\fi
%
% %%%%%%%%%%%%%%%%%%%%%%%%%%%%%%%%%%%%%%
% \paragraph{Part Include Files.}
%
% The include files are called |cdocspt3.tex| and |cdocspt4.tex|.
%
%\iffalse
%<*samplepart3|samplepart4>
%\fi

% Optional override for |\version| flag:
%    \begin{macrocode}
%%\providecommand{\version}{final}
%    \end{macrocode}

% Include the main document:
%    \begin{macrocode}
\input{childdoc.def}
\childdocby{cdocsamp}
%    \end{macrocode}

%\iffalse
%</samplepart3|samplepart4>
%\fi
%
%\iffalse
%<*samplepart3>
%\fi
% Some text for part 3:
%    \begin{macrocode}
some text in part three
%    \end{macrocode}

%\iffalse
%</samplepart3>
%\fi
% Some text for part 4:
%\iffalse
%<*samplepart4>
%\fi
%    \begin{macrocode}
more text in part four
%    \end{macrocode}

%\iffalse
%</samplepart4>
%\fi
%
% %%%%%%%%%%%%%%%%%%%%%%%%%%%%%%%%%%%%%%
% \paragraph{Forwarding for a Complete Draft.}
%
% The following forwarding file |cdocsdrf.tex|
% compiles the main document in draft mode:
%\iffalse
%<*sampledraft>
%\fi
%    \begin{macrocode}
\def\version{draft}
\input{childdoc.def}
\childdocforward{cdocsamp}
%    \end{macrocode}

%\iffalse
%</sampledraft>
%\fi
%
% %%%%%%%%%%%%%%%%%%%%%%%%%%%%%%%%%%%%%%
% \paragraph{Forwarding for Final Version of the Chapters.}
%
% The following forwarding files |cdocsfn1.tex| and |cdocsfn2.tex|
% (with identical content)
% compile the final versions of the child documents
% |cdocsch1.tex| and |cdocsch2.tex|, respectively:
%\iffalse
%<*samplefinal>
%\fi
%    \begin{macrocode}
\def\version{final}
\input{childdoc.def}
\childdocforwardprefix[cdocsamp]{cdocsfn}{cdocsch}
%    \end{macrocode}

%\iffalse
%</samplefinal>
%\fi
%
% %%%%%%%%%%%%%%%%%%%%%%%%%%%%%%%%%%%%%%
% \paragraph{Command Line Processing.}
%
% The following three command lines generate the output files
% |cdocscld|, |cdocscl1| and |cdocscl2|
% which should be identical to
% |cdocsdrf|, |cdocsch1| and |cdocsfn2|, respectively:
% \begin{center}
% \begin{tabular}{l}
% |latex -jobname cdocscld \|\\
% |  "\def\version{draft}\input{childdoc.def}\childdocforward{cdocsamp}"|\\
% |latex -jobname cdocscl1 \|\\
% |  "\input{childdoc.def}\childdocforward[cdocsamp]{cdocsch1}"|\\
% |latex -jobname cdocscl2 \|\\
% |  "\def\version{final}\input{childdoc.def}\childdocforward{cdocsch2}"|
% \end{tabular}
% \end{center}
% Note that the trailing backslash on each first line
% merely continues the input to the second line
% (for convenient cut ant paste).
% Furthermore, the command |latex| can be replaced by any
% of its alternative versions such as |pdflatex|.
%
% %%%%%%%%%%%%%%%%%%%%%%%%%%%%%%%%%%%%%%%%%%%%%%%%%%%%%%%%%%%%%%%%%%%%%%%%%%%%%%
% %%%%%%%%%%%%%%%%%%%%%%%%%%%%%%%%%%%%%%%%%%%%%%%%%%%%%%%%%%%%%%%%%%%%%%%%%%%%%%
% \section{Implementation}
%\iffalse
%<*package>
%\fi
%
% This section describes the definitions file |childdoc.def|.

% The definitions cannot be loaded using |\usepackage| or |\RequirePackage|
% which has a mechanism to prevent loading a style file more than once.
% When loading the definitions by means of |\input|
% multiple instances have to be prevented manually:
%\iffalse
%This code needs to be before the `\ProvidesFile' directive
%which is defined at the beginning of this file.
%Therefore it is also placed there and commented out here.
%</package>
%<*discard>
%\fi
%    \begin{macrocode}
\ifdefined\childdocmain\endinput\fi
%    \end{macrocode}
%\iffalse
%</discard>
%<*package>
%\fi
%
% \macro{\ifchilddoc}
% \macro{\ifchilddocmanual}
% The conditional |\ifchilddoc| tells whether a
% child (true) or main (false) document is being compiled.
% The conditional |\ifchilddocmanual| tells whether
% the |\includeonly| mechanism is used (false) or
% the selection of child files must be performed manually (true).
% The definitions initialise to false:
%    \begin{macrocode}
\newif\ifchilddoc
\newif\ifchilddocmanual
%    \end{macrocode}

% \macro{\childdocname}
% \macro{\childdocjob}
% The macro |\childdocname| stores the name of the main document
% to be compiled. The macro |\childdocjob| stores the name of
% the document on which the \LaTeX{} compiler was originally invoked.
% The content of |\jobname| cannot be compared
% to filenames specified in the source due to different catcodes.
% The following code rescans |\jobname|, stores the result
% in |\childdocname| and saves a copy in |\childdocjob|:
%    \begin{macrocode}
\edef\childdocname{\scantokens\expandafter{\jobname\noexpand}}
\let\childdocjob\childdocname
%    \end{macrocode}

% \macro{\childdocdisable}
% The macro |\childdocdisable| prevents the main file
% from being processed more than once.
% At this stage, the main document command |\childdocmain|
% is assumed to be called once again where it should do nothing.
% Any subsequent call to it should prevent
% a secondary processing of the main document
% It overwrites the forwarding commands
% |\childdocof| and |\childdocforward|
% with empty macros to prevent further inclusions of the main document:
%    \begin{macrocode}
\newcommand{\childdocdisable}
{
  \renewcommand{\childdocmain}[1]{\renewcommand{\childdocmain}[1]{\endinput}}
  \renewcommand{\childdocof}[1]{}
  \renewcommand{\childdocby}[2][]{}
  \renewcommand{\childdocforward}[2][]{}
  \renewcommand{\childdocdisable}{}
}
%    \end{macrocode}

% \macro{\childdocmain}
% The macro |\childdocmain| is to be called at the top of the main file
% with nothing or the main filename (without extension) as argument.
% First, it breaks loops.
% If the argument is not empty and does not match |\childdocname|
% (which is set by the first inclusion of |childdoc.def|),
% |\ifchilddoc| is set to true, |\includeonly| is applied to the child file
% and |\jobname| is set to the main file
% (for proper handling of |.aux| files):
%    \begin{macrocode}
\newcommand{\childdocmain}[1]
{
  \childdocdisable\childdocmain{}
  \if?#1?\else
    \begingroup
      \def\childdoctmp{#1}
      \ifx\childdoctmp\childdocname
        \def\childdoctmp{}
      \else
        \def\childdoctmp
        {
          \childdoctrue
          \includeonly{\childdocname}
          \def\childdocjob{#1}
          \def\jobname{#1}
        }
      \fi
      \expandafter
    \endgroup
    \childdoctmp
  \fi
}
%    \end{macrocode}

% \macro{\childdocof}
% The command |\childdocof| redirects
% compilation to the main file |#1|.
%    \begin{macrocode}
\newcommand{\childdocof}[1]
{
  \childdocdisable
  \childdoctrue
  \includeonly{\childdocname}
  \def\jobname{#1}
  \def\childdocjob{#1}
  \input{#1}
}
%    \end{macrocode}

% \macro{\childdocby}
% The command |\childdocby| ....
%    \begin{macrocode}
\newcommand{\childdocby}[2][]
{
  \childdocdisable
  \childdoctrue
  \childdocmanualtrue
  \if?#1?\else
    \def\jobname{#2}
  \fi
  \def\childdocjob{#2}
  \input{#2}
  \endinput
}
%    \end{macrocode}

% \macro{\childdocforward}
% The command |\childdocforward| redirects
% compilation to the main file or
% (if the optional argument is given) a child file.
% Parameters are set as if the main file
% or a child file starting with |\childdocof| was compiled.
% Then compilation is handed over to the main file:
%    \begin{macrocode}
\newcommand{\childdocforward}[2][]
{
  \begingroup
    \if?#1?
      \def\childdoctmp
      {
        \def\childdocname{#2}
        \def\childdocjob{#2}
        \def\jobname{#2}
        \input{#2}
        \endinput
      }
    \else
      \def\childdoctmp
      {
        \childdocdisable
        \def\childdocname{#2}
        \childdoctrue
        \includeonly{#2}
        \def\childdocjob{#1}
        \def\jobname{#1}
        \input{#1}
        \endinput
      }
    \fi
    \expandafter
  \endgroup
  \childdoctmp
}
%    \end{macrocode}

% \macro{\childdocforwardprefix}
% The command |\childdocforwardprefix| redirects
% compilation to the main or a child file by means of a pattern.
% The prefix |#1| in the current filename is replaced by |#2|
% and the suffix of the current filename is kept
% (it is assumed that the filename does not contain the substring `|~~~|'
% which is used as a delimiter).
% Compilation is handed over to the new file by |\childdocforward|:
%    \begin{macrocode}
\newcommand{\childdocforwardprefix}[3][]
{
  \begingroup
    \def\childdocextract #2##1~~~{\def\childdoctmp{\childdocforward[#1]{#3##1}}}
    \expandafter\childdocextract\childdocname~~~
    \expandafter
  \endgroup
  \childdoctmp
}
%    \end{macrocode}

% \macro{\childdoc}
% The deprecated macro |\childdoc| is a legacy version of |\childdocmain|:
%    \begin{macrocode}
\newcommand{\childdoc}{\childdocmain}
%    \end{macrocode}

% \macro{\childdocredirect}
% The deprecated macro |\childdocredirect| is a legacy version
% of |\childdocforward| and |\childdocforwardprefix|:
%    \begin{macrocode}
\newcommand{\childdocredirect}[2][]
{
  \begingroup
    \if?#1?
      \def\childdoctmp{\childdocforward{#2}}
    \else
      \def\childdoctmp{\childdocforwardprefix{#1}{#2}}
    \fi
    \expandafter
  \endgroup
  \childdoctmp
}
%    \end{macrocode}

%\iffalse
%</package>
%\fi
%
\endinput

\childdocby{cdocsamp}
%    \end{macrocode}

%\iffalse
%</samplepart3|samplepart4>
%\fi
%
%\iffalse
%<*samplepart3>
%\fi
% Some text for part 3:
%    \begin{macrocode}
some text in part three
%    \end{macrocode}

%\iffalse
%</samplepart3>
%\fi
% Some text for part 4:
%\iffalse
%<*samplepart4>
%\fi
%    \begin{macrocode}
more text in part four
%    \end{macrocode}

%\iffalse
%</samplepart4>
%\fi
%
% %%%%%%%%%%%%%%%%%%%%%%%%%%%%%%%%%%%%%%
% \paragraph{Forwarding for a Complete Draft.}
%
% The following forwarding file |cdocsdrf.tex|
% compiles the main document in draft mode:
%\iffalse
%<*sampledraft>
%\fi
%    \begin{macrocode}
\def\version{draft}
% \iffalse
%
% childdoc.dtx Copyright (C) 2017-2018 Niklas Beisert
%
% This work may be distributed and/or modified under the
% conditions of the LaTeX Project Public License, either version 1.3
% of this license or (at your option) any later version.
% The latest version of this license is in
%   http://www.latex-project.org/lppl.txt
% and version 1.3 or later is part of all distributions of LaTeX
% version 2005/12/01 or later.
%
% This work has the LPPL maintenance status `maintained'.
%
% The Current Maintainer of this work is Niklas Beisert.
%
% This work consists of the files childdoc.dtx and childdoc.ins
% and the derived files childdoc.def and cdocsamp.tex with
% cdocsch1.tex, cdocsch2.tex, cdocsdrf.tex, cdocsfn1.tex, cdocsfn2.tex.
%
%<package>\ifdefined\childdocmain\endinput\fi
%<package>\ProvidesFile{childdoc.def}[2018/12/30 v2.0 child document driver]
%<samplemain>\ProvidesFile{cdocsamp.tex}[2018/12/30 v2.0 sample for childdoc]
%<*driver>
%\ProvidesFile{childdoc.drv}[2018/12/30 v2.0 childdoc reference manual file]
\PassOptionsToClass{10pt,a4paper}{article}
\documentclass{ltxdoc}

\usepackage[margin=35mm]{geometry}
\usepackage{hyperref}
\usepackage{hyperxmp}
\usepackage[usenames]{color}

\hypersetup{colorlinks=true}
\hypersetup{pdfstartview=FitH}
\hypersetup{pdfpagemode=UseNone}
\hypersetup{pdfsource={}}
\hypersetup{pdflang={en-UK}}
\hypersetup{pdfcopyright={Copyright 2017-2018 Niklas Beisert.
  This work may be distributed and/or modified under the
  conditions of the LaTeX Project Public License, either version 1.3
  of this license or (at your option) any later version.}}
\hypersetup{pdflicenseurl={http://www.latex-project.org/lppl.txt}}
\hypersetup{pdfcontactaddress={ETH Zurich, ITP, HIT K,
  Wolfgang-Pauli-Strasse 27}}
\hypersetup{pdfcontactpostcode={8093}}
\hypersetup{pdfcontactcity={Zurich}}
\hypersetup{pdfcontactcountry={Switzerland}}
\hypersetup{pdfcontactemail={nbeisert@itp.phys.ethz.ch}}
\hypersetup{pdfcontacturl={http://people.phys.ethz.ch/\xmptilde nbeisert/}}

\newcommand{\secref}[1]{\hyperref[#1]{section \ref*{#1}}}

\parskip1ex
\parindent0pt
\let\olditemize\itemize
\def\itemize{\olditemize\parskip0pt}

\begin{document}

\title{The \textsf{childdoc} Package}
\hypersetup{pdftitle={The childdoc Package}}
\author{Niklas Beisert\\[2ex]
  Institut f\"ur Theoretische Physik\\
  Eidgen\"ossische Technische Hochschule Z\"urich\\
  Wolfgang-Pauli-Strasse 27, 8093 Z\"urich, Switzerland\\[1ex]
  \href{mailto:nbeisert@itp.phys.ethz.ch}
  {\texttt{nbeisert@itp.phys.ethz.ch}}}
\hypersetup{pdfauthor={Niklas Beisert}}
\hypersetup{pdfsubject={Manual for the LaTeX2e Package childdoc}}
\date{30 December 2018, \textsf{v2.0}}
\maketitle

\begin{abstract}\noindent
\textsf{childdoc} is a \LaTeXe{} package
that enables the direct compilation
of document sections included by |\include|
to individual files.
\end{abstract}

\begingroup
\parskip0ex
\tableofcontents
\endgroup

%%%%%%%%%%%%%%%%%%%%%%%%%%%%%%%%%%%%%%%%%%%%%%%%%%%%%%%%%%%%%%%%%%%%%%%%%%%%%%%%
%%%%%%%%%%%%%%%%%%%%%%%%%%%%%%%%%%%%%%%%%%%%%%%%%%%%%%%%%%%%%%%%%%%%%%%%%%%%%%%%
\section{Introduction}

\LaTeX{} provides a mechanism to structure a large document (such as a book)
into a main file and several child files (containing the chapters)
using the |\include| command.
This mechanism is beneficial for documents
which span hundreds of pages in order to
make the source file(s) more manageable.
Moreover, compilation can be restricted to
selected child files by means of the |\includeonly| command.
The latter feature can be used to reduce the compilation time while editing
(this was significantly more useful in the earlier days of \LaTeX{})
or to generate a smaller document which is easier to navigate.
Another application of |\includeonly| is to generate
documents consisting of selected parts of the complete document.

However, there are a few drawbacks of the plain |\include| mechanism:
\begin{itemize}
\item
The child files cannot be compiled on their own,
they can only be compiled via the main file.
A naive editing environment
(such as a text editor with an option
to have the current file processed by \LaTeX)
may require one to switch to the main file before compiling;
attempting to compile the child file produces errors.
\item
The main file must be modified (each time)
to adjust the |\includeonly| command
to the present needs. This easily leaves the main file in a messy state.
\item
The generated document will always carry the filename
of the main document. This is inconvenient if
several child files are to be compiled and
to be kept for distribution.
\end{itemize}

The present package provides a simple interface
to make child files individually compilable by \LaTeX{}.
Compiling a child file then has the same effect as compiling
the main file with an |\includeonly| command
to select the appropriate child.
Moreover the generated document will carry the name of the child
rather than the main file.
This resolves all three above issues.

This feature is meant to make the editing of books,
thesis documents and lecture notes somewhat more convenient.
However, the package can also be used efficiently for
composing a series of documents (such as exercise sheets)
which are typically distributed individually.
It then assists the author in generating the individual documents
(potentially in different versions)
as well as a document containing the collected series.
Another application is in developing style files
or other kinds of included material
where compilation of the style file could redirect
to a sample or test file.

%%%%%%%%%%%%%%%%%%%%%%%%%%%%%%%%%%%%%%%%%%%%%%%%%%%%%%%%%%%%%%%%%%%%%%%%%%%%%%%%
%%%%%%%%%%%%%%%%%%%%%%%%%%%%%%%%%%%%%%%%%%%%%%%%%%%%%%%%%%%%%%%%%%%%%%%%%%%%%%%%
\section{Usage}

First of all, the package \textsf{childdoc} is \emph{not} a standard
\LaTeXe{} |.sty| style file! Therefore it needs to be invoked in
a non-standard way.

%%%%%%%%%%%%%%%%%%%%%%%%%%%%%%%%%%%%%%%%%%%%%%%%%%%%%%%%%%%%%%%%%%%%%%%%%%%%%%%%
\subsection{Included Files}
\label{sec:include}

%%%%%%%%%%%%%%%%%%%%%%%%%%%%%%%%%%%%%%%%
\DescribeMacro{\childdocmain}
To use the package, add the commands
\begin{center}
\begin{tabular}{l}
|\input{childdoc.def}|\\
|\childdocmain{}|\\
\end{tabular}
\end{center}
at the very top of the main \LaTeX{} file,
in particular \emph{before} the |\documentclass| statement!
The argument of |\childdocmain| should be left empty
(but it must be present).

%%%%%%%%%%%%%%%%%%%%%%%%%%%%%%%%%%%%%%%%
\DescribeMacro{\childdocof}
Furthermore, add the commands
\begin{center}
\begin{tabular}{l}
|\input{childdoc.def}|\\
|\childdocof{|\textit{main}|}|\\
\end{tabular}
\end{center}
at the top of every child file \textit{child}
which is included by |\include{|\textit{child}|}|
from within the main file
(or at least for those files to be compiled individually).
The argument \textit{main} must be the filename of the main file.

There are a couple of
considerations in setting up the main and child documents:

%%%%%%%%%%%%%%%%%%%%%%%%%%%%%%%%%%%%%%%%
\paragraph{Restrictions.}

Please note the following restrictions:
\begin{itemize}
\item
|\childdocmain| must be called with one argument \textit{main}
to ensure compatibility with earlier version of the package.
It must either be empty (|\childdocmain{}|)
or precisely match the filename of the main file in which it is specified.
See \secref{sec:detection} for further information.
\item
The filename \textit{main} must be specified without the |.tex| extension.
\item
The filename \textit{main} is case sensitive
(even in case-insensitive file systems)
due to internal string comparison.
\item
The argument \textit{main} should be fully expanded, it cannot be a macro.
\item
Subdirectories and special characters should be avoided in filenames.
\item
The command |\childdocmain{|\textit{main}|}| must be followed by a whitespace.
It should not be followed immediately by another command
or by a comment mark `|%|'.
This is because the \TeX{} parser reads the token immediately following
the argument of |\childdocmain| and puts it
at the beginning of every child section;
however, a white\-space is ignored.
\end{itemize}

%%%%%%%%%%%%%%%%%%%%%%%%%%%%%%%%%%%%%%%%
\paragraph{Content of Main File.}

It is advisable to place all content in the child files included by |\include|.
Any output contained in the main file will appear in all child documents
unless suppressed manually;
it cannot be suppressed automatically by the |\includeonly| directive
and thus should normally be avoided.
A method to include some content in the main file
by means of conditional processing is described in \secref{sec:conditional}.

%%%%%%%%%%%%%%%%%%%%%%%%%%%%%%%%%%%%%%%%
\paragraph{Page Numbering.}

When only a part of the document is compiled,
the appropriate numbering of pages
(as well as other status parameters)
is determined from the |.aux| files.
The latter contain information from previous passes.
However this information needs to propagate through
all intermediate child documents.
Therefore the page numbering in child documents may well
be inconsistent until the complete document is compiled at least once.

A useful (if unconventional) way to always ensure a consistent
page numbering is to restart the numbering in each child document
and denote the pages by `\textit{child}|.|\textit{page}'
where \textit{child} represents the chapter/section number of the child file.
This can be achieved by the command
|\numberwithin{page}{|\textit{child}|}|
of the \textsf{amsmath} package
where \textit{child} can be |chapter| or |section|
depending on the chosen structuring.
Alternatively, one can modify the macro |\thepage| appropriately
and reset the counter |page| at the start of each child file.

%%%%%%%%%%%%%%%%%%%%%%%%%%%%%%%%%%%%%%%%%%%%%%%%%%%%%%%%%%%%%%%%%%%%%%%%%%%%%%%%
\subsection{Conditional Processing}
\label{sec:conditional}

The package provides a mechanism to compile different versions
of a document. To customise the versions further some conditional processing
can come in handy to distinguish which version is being compiled.
The package provides two macros to describe the compilation context:

%%%%%%%%%%%%%%%%%%%%%%%%%%%%%%%%%%%%%%%%
\DescribeMacro{\ifchilddoc}
The conditional |\ifchilddoc| distinguishes between the compilation of
child documents and the main document:
%
\begin{center}
|\ifchilddoc |\textit{child-code}| |[|\||else |\textit{main-code}]| \||fi|
\end{center}

%%%%%%%%%%%%%%%%%%%%%%%%%%%%%%%%%%%%%%%%
\DescribeMacro{\childdocname}
\DescribeMacro{\childdocjob}
The macro |\childdocname| contains the filename (without extension)
of the main or child file being processed.
Note that |\childdocjob| will always contain the name of the main file.

%%%%%%%%%%%%%%%%%%%%%%%%%%%%%%%%%%%%%%%%
\paragraph{Title Page.}

Conditional processing can be used to include a title or banner page
in the main document when proper precautions are taken.
Importantly, the code in the main file should ensure that the page counter
(as well as other status parameters which are stored in the |.aux| files)
takes the same value after the conditional processing.
Otherwise the page numbers may take divergent values
depending on which part is compiled.

For example, a title page could be declared by:
%
\begin{center}
\begin{tabular}{l}
|\ifchilddoc\||else|\\
|\addtocounter{page}{-1}|\\
\textit{code for title page}\\
|\newpage|\\
|\||fi|
\end{tabular}
\end{center}
%
A banner page for the child documents can be generated by:
%
\begin{center}
\begin{tabular}{l}
|\ifchilddoc|\\
|\addtocounter{page}{-1}|\\
\textit{code for banner page}\\
|\newpage|\\
|\||fi|
\end{tabular}
\end{center}
%
Here one could write a message such as:
\begin{center}
|This is the part \childdocname{} of \childdocjob{}.|
\end{center}

%%%%%%%%%%%%%%%%%%%%%%%%%%%%%%%%%%%%%%%%%%%%%%%%%%%%%%%%%%%%%%%%%%%%%%%%%%%%%%%%
\subsection{Flags}
\label{sec:flags}

The package makes it easy to generate different versions
of the main or child documents.
To this end compilation flags can be defined
and assigned different default values.
They will be particularly useful in conjunction
with the forwarding mechanism described in \secref{sec:forward}.

For example, it may be useful to have a flag |\version|
which can be set to |draft| or |final|.
The document source will contain some conditional code
depending on the value of |\version|.
Suppose further, the flag should default to |final| for the main file
and to |draft| for child files
which is a natural assignment for editing the document.
This is achieved by placing the following code
in the preamble of the main document
(below the |\childdocmain| directive):
%
\begin{center}
\begin{tabular}{l}
|\ifchilddoc|\\
|\providecommand{\version}{draft}|\\
|\||else|\\
|\providecommand{\version}{final}|\\
|\||fi|
\end{tabular}
\end{center}
%
The definition by |\providecommand| makes sure
that previous definitions are not overwritten.
Further statements |\providecommand{\version}{...}|
can thus be added before the above code to override it.

For the main file, one might add a line
(between |\childdocmain| and the above block)
%
\begin{center}
|%\ifchilddoc\||else\providecommand{\version}{draft}\||fi|
\end{center}
%
which can be uncommented to produce a draft version.
Likewise one can add a line to the very top of a child file
(above the |\childdocof{|\textit{main}|}| directive)
%
\begin{center}
|%\providecommand{\version}{final}|
\end{center}
%
which can be uncommented to produce the final version of this child document.

%%%%%%%%%%%%%%%%%%%%%%%%%%%%%%%%%%%%%%%%%%%%%%%%%%%%%%%%%%%%%%%%%%%%%%%%%%%%%%%%
\subsection{Forwarding}
\label{sec:forward}

Different versions of the main or child documents
using compilation flags as described in \secref{sec:flags}
can be (permanently) stored in different files
for convenient compilation, viewing and distribution.
To this end, the package defines a command
to pass on compilation to a different file:

%%%%%%%%%%%%%%%%%%%%%%%%%%%%%%%%%%%%%%%%
\DescribeMacro{\childdocforward}
The command |\childdocforward| redirects processing to
another source file:
%
\begin{center}
\begin{tabular}{l}
|\input{childdoc.def}|\\
|\childdocforward[|\textit{main}|]{|\textit{dest}|}|\\
\end{tabular}
\end{center}
%
The argument \textit{dest} is the destination file
(without extension).
It should be the main file or one of the child files.
Note that further \textsf{childdoc} directives
such as |\childdocof| and |\childdocforward|
in the indicated file will be processed in this form.
The optional argument \textit{main}
passes on directly to the main file \textit{main}
while pretending to compile the child \textit{dest}.
This form behaves as if \textit{dest}
issues |\childdocof{|\textit{main}|}| right away,
and no further \textsf{childdoc} directives will be processed.

%%%%%%%%%%%%%%%%%%%%%%%%%%%%%%%%%%%%%%%%
\DescribeMacro{\...prefix}
In the alternative form |\childdocforwardprefix|,
%
\begin{center}
\begin{tabular}{l}
|\input{childdoc.def}|\\
|\childdocforwardprefix[|\textit{main}|]{|\textit{prefix}|}{|\textit{dest}|}|
\end{tabular}
\end{center}
%
the destination file is determined by a pattern
depending on the current file:
To make this work, the current file must be called
`{\textit{prefix}\hspace{0.2em}\textit{suffix}}'
with \textit{prefix} matching precisely the argument.
Processing is then passed on to the file
`{\textit{dest}\hspace{0.2em}\textit{suffix}}'.
Surely, the same effect is achieved by
directly specifying the
argument `{\textit{dest}\hspace{0.2em}\textit{suffix}}'
in the first form.
However, that requires to set up a different file
for each child. With the alternative form of the command
all these files can have exactly the same content
which simplifies setting them up and maintaining them.

For example, the following file |draft.tex|
with a compilation flag |\version| as described in \secref{sec:flags}
compiles the main document as a draft:
%
\begin{center}
\begin{tabular}{l}
|\def\version{draft}|\\
|\input{childdoc.def}|\\
|\childdocforward{|\textit{main}|}|
\end{tabular}
\end{center}
%
Likewise, the following files |final|\textit{nn}|.tex|
compile the final version of the child document
|child|\textit{nn}|.tex|:
%
\begin{center}
\begin{tabular}{l}
|\def\version{final}|\\
|\input{childdoc.def}|\\
|\childdocforwardprefix{final}{child}|
\end{tabular}
\end{center}
%

Note that when several versions of a main file and/or of each child file
are to be generated, it may be convenient to set up a |Makefile| or
shell script to automatise the process.

%%%%%%%%%%%%%%%%%%%%%%%%%%%%%%%%%%%%%%%%%%%%%%%%%%%%%%%%%%%%%%%%%%%%%%%%%%%%%%%%
\subsection{Command Line Processing}
\label{sec:commandline}

The effect of redirection files can also be achieved by invoking
the \LaTeX{} compiler with a more elaborate command line.
Most conveniently this should be done as part
of a shell script or a |Makefile|.

When using \textsf{childdoc} in the main file, the following
command lines effectively perform a redirection
(note that depending on the shell being used,
backslashes may have to be doubled: `|\|' $\to$ `|\\|'):
%
\begin{center}
|... -jobname "|\textit{target}|" |\\|"|[\textit{flags}]%
|\input{childdoc.def}\childdocforward[|\textit{main}|]{|\textit{dest}|}"|
\end{center}
%
Here \textit{target} is the name of the output file,
\textit{main} is the name of the main file
and \textit{dest} is the name of the main or child file to be processed
(all filenames without extensions).
The optional argument \textit{main} can be omitted
if \textit{main} matches \textit{dest}.
Optionally, compilation \textit{flags} can be defined via |\def| commands.
This command line makes the \TeX{} engine believe
it is compiling the file \textit{target}
whose content is specified as the latter parameter.
The provided code then forwards the processing to
\textit{main} or \textit{dest} as described in \secref{sec:forward}.

%%%%%%%%%%%%%%%%%%%%%%%%%%%%%%%%%%%%%%%%%%%%%%%%%%%%%%%%%%%%%%%%%%%%%%%%%%%%%%%%
\subsection{Include by Input}
\label{sec:input}

Including child documents by |\include| has some restrictions by design.
Most notably, the content of a child document always occupies
its own set of pages; pages cannot be shared between child documents.
Usually, this behaviour makes perfect sense
because each child document contain an essential part of the document.
However, in some situations it may be desirable to compose
a document from a collection of parts
without having mandatory page breaks between then.
For this case, the package
provides a mechanism to include parts
by |\input| which can also be processed individually.
However, by construction this mechanism
requires manual handling of the content to be output.

%%%%%%%%%%%%%%%%%%%%%%%%%%%%%%%%%%%%%%%%
\DescribeMacro{\ifchilddocmanual}
The main file should be prepared as usual, see \secref{sec:include}.
However, the document body must make a distinction
between processing of an individual part and of the main document, e.g.:
%
\begin{center}
\begin{tabular}{l}
|\ifchilddocmanual|\\
|\input{\childdocname}|\\
|\||else|\\
\textit{document body with }|\input{|\textit{part}|}|\\
|\||fi|
\end{tabular}
\end{center}
%
The conditional |\ifchilddocmanual| is true whenever
a part to be included by |\input| is being compiled,
and the name of the part is stored in |\childdocname|.

%%%%%%%%%%%%%%%%%%%%%%%%%%%%%%%%%%%%%%%%
\DescribeMacro{\childdocby}
Each part to be included by |\input| should start with:
%
\begin{center}
\begin{tabular}{l}
|\input{childdoc.def}|\\
|\childdocby{|\textit{main}|}|\\
\end{tabular}
\end{center}
%
The directive |\childdocby| is similar to |\childdocof|
described in \secref{sec:include},
but the subsequent selection of content must be done manually.
To that end, both |\ifchilddoc| and |\ifchilddocmanual|
will be true upon processing of a part,
and the name of the part is stored in |\childdocname|.
Note that |\jobname| will be set to the filename of the current part
so that each part receives an individual |.aux| file
that does not interfere with the |.aux| file(s) of the main document.
This behaviour can be altered by the alternative form
|\childdocby[*]{|\textit{main}|}| (with a non-empty optional argument)
which uses the |.aux| file of the main document
by setting |\jobname| to \textit{main}.

%%%%%%%%%%%%%%%%%%%%%%%%%%%%%%%%%%%%%%%%%%%%%%%%%%%%%%%%%%%%%%%%%%%%%%%%%%%%%%%%
\subsection{Driver Development}
\label{sec:driver}

The \textsf{childdoc} mechanism can also be use for the development
of definition files such as \LaTeX{} styles or classes.
This case differs from the above setup with multiple parts
included by |\include| in that no |\includeonly| should be invoked.
This can be achieved by starting the include file
(before |\ProvidesPackage|) with:
%
\begin{center}
\begin{tabular}{l}
|\input{childdoc.def}|\\
|\childdocforward{|\textit{main}|}|\\
\end{tabular}
\end{center}
%
or alternatively with:
%
\begin{center}
\begin{tabular}{l}
|\input{childdoc.def}|\\
|\childdocby{|\textit{main}|}|\\
\end{tabular}
\end{center}
%
Both forms have slightly different effects as described above.
The main file is prepared as usual, see \secref{sec:include}.

%%%%%%%%%%%%%%%%%%%%%%%%%%%%%%%%%%%%%%%%%%%%%%%%%%%%%%%%%%%%%%%%%%%%%%%%%%%%%%%%
\subsection{Legacy Detection}
\label{sec:detection}

The directive |\childdocmain| in the main file can detect
whether the complete document or merely a child is to be compiled
even without using the directive |\childdocof|.
This method is deprecated because it is less robust
and there is no compelling reason to use it;
it is merely provided for backward compatibility
and it may be removed in future versions.

If the detection mechanism is to be used,
it is mandatory to correctly specify
the filename of the main file as the argument of |\childdocmain|:
%
\begin{center}
\begin{tabular}{l}
|\input{childdoc.def}|\\
|\childdocmain{|\textit{main}|}|\\
\end{tabular}
\end{center}
%
If |\jobname| does not match the argument \textit{main} of |\childdocmain|,
it is assumed that |\jobname| points to the child file to be compiled.
When using |\childdocmain| with the main file specified as argument,
it suffices to start a child file
with just |\input{|\textit{main}|}|
without loading of the package and using |\childdocof|.
If instead all processing is done
with the appropriate \textsf{childdoc} directives,
the argument of \textit{main} of |\childdocmain| can be empty.

An alternative version of the command line processing described
in \secref{sec:commandline} using the detection mechanism reads:
%
\begin{center}
|... -jobname "|\textit{target}|" "|[\textit{flags}]%
[|\def\jobname{|\textit{dest}|}|]|\input{|\textit{main}|}"|
\end{center}

%%%%%%%%%%%%%%%%%%%%%%%%%%%%%%%%%%%%%%%%%%%%%%%%%%%%%%%%%%%%%%%%%%%%%%%%%%%%%%%%
\subsection{Manual Code}
\label{sec:manual}

In case one cannot be certain whether the definitions file |childdoc.def|
is installed on the target \TeX{} distribution
and one prefers not to ship it,
it is conceivable to paste a few relevant commands into the sources.

To that end, drop all statements |\input{childdoc.def}|
and perform the replacements as outlined below.
Instead of |\childdocmain{|\textit{main}|}| add the following code
to the top of the main file:
%
\begin{center}
\begin{tabular}{l}
|\||ifdefined\childdocname\endinput\||fi\newif\ifchilddoc|\\
|\edef\childdocname{\scantokens\expandafter{\jobname\noexpand}}|\\
|\def\childdocmain{|\textit{main}|}\||ifx\childdocmain\childdocname\||else|\\
|\childdoctrue\includeonly{\childdocname}\let\jobname\childdocmain\||fi|\\
\end{tabular}
\end{center}
%
Instead of |\childdocof{|\textit{main}|}| just include the main file
at the top of each child file:
%
\begin{center}
|\input{|\textit{main}|}|
\end{center}
%
A simple redirection |\childdocforward{|\textit{dest}|}| is achieved by:
%
\begin{center}
|\def\jobname{|\textit{dest}|}\input{\jobname}|
\end{center}
%
The redirection with prefix
|\childdocforwardprefix[|\textit{prefix}|]{|\textit{dest}|}|
is accomplished by:
%
\begin{center}
\begin{tabular}{l}
|{\edef\jobname{\scantokens\expandafter{\jobname\noexpand}}|\\
|\def\redirectjob |\textit{prefix}|#1~~~{\gdef\jobname{|\textit{dest}|#1}}|\\
|\expandafter\redirectjob\jobname~~~}\input{\jobname}|
\end{tabular}
\end{center}

In an alternative approach,
child documents can be compiled by a specific command line
without additional code or specific definitions:
%
\begin{center}
|... -jobname "|\textit{target}|" "|[\textit{flags}]%
|\includeonly{|\textit{dest}|}\input{|\textit{main}|}"|
\end{center}
%

%%%%%%%%%%%%%%%%%%%%%%%%%%%%%%%%%%%%%%%%%%%%%%%%%%%%%%%%%%%%%%%%%%%%%%%%%%%%%%%%
%%%%%%%%%%%%%%%%%%%%%%%%%%%%%%%%%%%%%%%%%%%%%%%%%%%%%%%%%%%%%%%%%%%%%%%%%%%%%%%%
\section{Information}

%%%%%%%%%%%%%%%%%%%%%%%%%%%%%%%%%%%%%%%%%%%%%%%%%%%%%%%%%%%%%%%%%%%%%%%%%%%%%%%%
\subsection{Copyright}

Copyright \copyright{} 2017--2018 Niklas Beisert

This work may be distributed and/or modified under the
conditions of the \LaTeX{} Project Public License, either version 1.3
of this license or (at your option) any later version.
The latest version of this license is in
  \url{http://www.latex-project.org/lppl.txt}
and version 1.3 or later is part of all distributions of \LaTeX{}
version 2005/12/01 or later.

This work has the LPPL maintenance status `maintained'.

The Current Maintainer of this work is Niklas Beisert.

This work consists of the files |README.txt|, |childdoc.ins| and |childdoc.dtx|
as well as the derived files |childdoc.def|, |cdocsamp.tex|
with |cdocsch1.tex|, |cdocsch2.tex|, |cdocspt3.tex|, |cdocspt4.tex|,
|cdocsdrf.tex|, |cdocsfn1.tex|, |cdocsfn2.tex|
as well as |childdoc.pdf|.

%%%%%%%%%%%%%%%%%%%%%%%%%%%%%%%%%%%%%%%%%%%%%%%%%%%%%%%%%%%%%%%%%%%%%%%%%%%%%%%%
\subsection{Files and Installation}

The package consists of the files:
%
\begin{center}
\begin{tabular}{ll}
    |README.txt|   & readme file \\
    |childdoc.ins| & installation file \\
    |childdoc.dtx| & source file \\
    |childdoc.def| & definition file \\
    |cdocsamp.tex| & sample main file \\
    |cdocsch1.tex| & sample include file \\
    |cdocsch2.tex| & sample include file \\
    |cdocspt3.tex| & sample part file \\
    |cdocspt4.tex| & sample part file \\
    |cdocsdrf.tex| & sample redirection file \\
    |cdocsfn1.tex| & sample redirection file \\
    |cdocsfn2.tex| & sample redirection file \\
    |childdoc.pdf| & manual
\end{tabular}
\end{center}
%
The distribution consists of the files
|README.txt|, |childdoc.ins| and |childdoc.dtx|.
%
\begin{itemize}
\item
Run (pdf)\LaTeX{} on |childdoc.dtx|
to compile the manual |childdoc.pdf| (this file).
\item
Run \LaTeX{} on |childdoc.ins| to create the definitions file |childdoc.def|
and the sample |cdocsamp.tex| with include files
|cdocsch1.tex|, |cdocsch2.tex|, |cdocspt3.tex|, |cdocspt4.tex|,
|cdocsdrf.tex|, |cdocsfn1.tex|, |cdocsfn2.tex|.
Then copy the file |childdoc.def| to an appropriate directory of your \LaTeX{}
distribution, e.g.\ \textit{texmf-root}|/tex/latex/childdoc|.
\end{itemize}

%%%%%%%%%%%%%%%%%%%%%%%%%%%%%%%%%%%%%%%%%%%%%%%%%%%%%%%%%%%%%%%%%%%%%%%%%%%%%%%%
\subsection{Related CTAN Packages}

There are several other packages which offer a similar functionality:
%
\begin{itemize}
\item
The packages
\href{http://ctan.org/pkg/docmute}{\textsf{docmute}},
\href{http://ctan.org/pkg/includex}{\textsf{includex}} and
\href{http://ctan.org/pkg/standalone}{\textsf{standalone}}
provide commands to include only the document body of
a child file thus allowing both files to be compiled individually.
\item
The packages \href{http://ctan.org/pkg/subdocs}{\textsf{subdocs}}
and \href{http://ctan.org/pkg/subfiles}{\textsf{subfiles}}
provide structures in which the main and child documents can be
encapsulated and allowing them to be compiled individually.
The inclusion mechanism is different from the conventional |\include|.
\item
The package \href{http://ctan.org/pkg/combine}{\textsf{combine}}
is an elaborate solution to combine several documents into one.
\end{itemize}
%
See also the CTAN topic \href{http://ctan.org/topic/subdocs}{\textsf{subdocs}}
for further related packages.
The present package differs from the above solutions in that
a document structure constructed with the conventional |\include| mechanism
just needs two extra commands at the top of every file
such that all constituent files can be compiled individually.

%%%%%%%%%%%%%%%%%%%%%%%%%%%%%%%%%%%%%%%%%%%%%%%%%%%%%%%%%%%%%%%%%%%%%%%%%%%%%%%%
%\subsection{Feature Suggestions}
%
%The following is a list of features which may be useful for future
%versions of this package:
%%
%\begin{itemize}
%\item
%\ldots
%\end{itemize}

%%%%%%%%%%%%%%%%%%%%%%%%%%%%%%%%%%%%%%%%%%%%%%%%%%%%%%%%%%%%%%%%%%%%%%%%%%%%%%%%
\subsection{Revision History}

%%%%%%%%%%%%%%%%%%%%%%%%%%%%%%%%%%%%%%%%
\paragraph{v2.0:} 2018/12/30

\begin{itemize}
\item
immediate forward processing
\item
added |\childdocby| mechanism
\item
manual restructured
\end{itemize}

%%%%%%%%%%%%%%%%%%%%%%%%%%%%%%%%%%%%%%%%
\paragraph{v1.6:} 2018/01/17

\begin{itemize}
\item
application for development of include files
\item
corrections to manual
\end{itemize}

%%%%%%%%%%%%%%%%%%%%%%%%%%%%%%%%%%%%%%%%
\paragraph{v1.5:} 2017/05/21

\begin{itemize}
\item
more complete structuring introduced
\item
|\childdocof| introduced
\item
|\childdoc| renamed to |\childdocmain|
\item
|\childredirect| renamed to |\childdocforward| and |\childdocforwardprefix|
and functionality expanded
\end{itemize}

%%%%%%%%%%%%%%%%%%%%%%%%%%%%%%%%%%%%%%%%
\paragraph{v1.0:} 2017/04/27

\begin{itemize}
\item
manual and install package
\item
first version published on CTAN
\end{itemize}

%%%%%%%%%%%%%%%%%%%%%%%%%%%%%%%%%%%%%%%%
\paragraph{v0.6:} 2017/04/26

\begin{itemize}
\item
redirection mechanism added
\end{itemize}

%%%%%%%%%%%%%%%%%%%%%%%%%%%%%%%%%%%%%%%%
\paragraph{v0.5:} 2017/04/26

\begin{itemize}
\item
functionality in definition file
\end{itemize}


%%%%%%%%%%%%%%%%%%%%%%%%%%%%%%%%%%%%%%%%%%%%%%%%%%%%%%%%%%%%%%%%%%%%%%%%%%%%%%%%
%%%%%%%%%%%%%%%%%%%%%%%%%%%%%%%%%%%%%%%%%%%%%%%%%%%%%%%%%%%%%%%%%%%%%%%%%%%%%%%%
%%%%%%%%%%%%%%%%%%%%%%%%%%%%%%%%%%%%%%%%%%%%%%%%%%%%%%%%%%%%%%%%%%%%%%%%%%%%%%%%
\appendix

\settowidth\MacroIndent{\rmfamily\scriptsize 000\ }

 \DocInput{childdoc.dtx}

\end{document}
%</driver>
% \fi
%
% %%%%%%%%%%%%%%%%%%%%%%%%%%%%%%%%%%%%%%%%%%%%%%%%%%%%%%%%%%%%%%%%%%%%%%%%%%%%%%
% %%%%%%%%%%%%%%%%%%%%%%%%%%%%%%%%%%%%%%%%%%%%%%%%%%%%%%%%%%%%%%%%%%%%%%%%%%%%%%
% \section{Sample}
%\iffalse
%<*samplemain>
%\fi
%
% The following presents a sample document
% with two chapters, two parts, a title page,
% a compile flag as well as three forwarding files to set the flag.
% It consists of eight |.tex| files:
% \begin{center}
% \begin{tabular}{ll}
% |cdocsamp.tex|&main file\\
% |cdocsch1.tex|&include file for chapter 1\\
% |cdocsch2.tex|&include file for chapter 2\\
% |cdocspt3.tex|&include file for part 3\\
% |cdocspt4.tex|&include file for part 4\\
% |cdocsdrf.tex|&forwarding file for main file in draft mode\\
% |cdocsfi1.tex|&forwarding file for final version of chapter 1\\
% |cdocsfi2.tex|&forwarding file for final version of chapter 2\\
% \end{tabular}
% \end{center}
% Each of the eight files can be compiled directly by the \LaTeX{} compiler.
%
% %%%%%%%%%%%%%%%%%%%%%%%%%%%%%%%%%%%%%%
% \paragraph{Main File.}
%
% The main file is called |cdocsamp.tex|.
%
% Load the \textsf{childdoc} definitions and
% declare the filename for the main document:
%    \begin{macrocode}
\input{childdoc.def}
\childdocmain{}
%    \end{macrocode}

% Optional override for |\version| flag:
%    \begin{macrocode}
%%\ifchilddoc\else\providecommand{\version}{draft}\fi
%    \end{macrocode}

% Define the default values for the |\version| flag
% (|final| for the main file and |draft| for childs):
%    \begin{macrocode}
\ifchilddoc
\providecommand{\version}{draft}
\else
\providecommand{\version}{final}
\fi
%    \end{macrocode}

% Load the standard document class:
%    \begin{macrocode}
\documentclass[12pt]{article}
%    \end{macrocode}

% Start the document body:
%    \begin{macrocode}
\begin{document}
%    \end{macrocode}

% Declare a title page.
% Print title, part of document being processed and version flag:
%    \begin{macrocode}
\addtocounter{page}{-1}
\begin{center}
{\LARGE\bfseries{}childdoc example\par}
\vspace{1cm}
\ifchilddoc
\ifchilddocmanual part\else chapter\fi:
`\childdocname' of `\childdocjob'\par
\else
main document: `\childdocjob'\par
\fi
version: \version\par
\end{center}
\newpage
%    \end{macrocode}

% Manually include selected file,
% otherwise process as usual:
%    \begin{macrocode}
\ifchilddocmanual
\section*{part `\childdocname'}
\input{\childdocname}
\else
%    \end{macrocode}

% Include the two chapters:
%    \begin{macrocode}
\include{cdocsch1}
\include{cdocsch2}
%    \end{macrocode}

% Include the two parts unless only chapters should be displayed:
%    \begin{macrocode}
\ifchilddoc\else
\section{part three}
\input{cdocspt3}
\section{part four}
\input{cdocspt4}
\fi
%    \end{macrocode}

% Process as usual until here:
%    \begin{macrocode}
\fi
%    \end{macrocode}

% End of document body:
%    \begin{macrocode}
\end{document}
%    \end{macrocode}
%\iffalse
%</samplemain>
%\fi
%
% %%%%%%%%%%%%%%%%%%%%%%%%%%%%%%%%%%%%%%
% \paragraph{Chapter Include Files.}
%
% The include files are called |cdocsch1.tex| and |cdocsch2.tex|.
%
%\iffalse
%<*samplechap1|samplechap2>
%\fi

% Optional override for |\version| flag:
%    \begin{macrocode}
%%\providecommand{\version}{final}
%    \end{macrocode}

% Include the main document:
%    \begin{macrocode}
\input{childdoc.def}
\childdocof{cdocsamp}
%    \end{macrocode}

%\iffalse
%</samplechap1|samplechap2>
%\fi
%
%\iffalse
%<*samplechap1>
%\fi
% Some text for chapter 1:
%    \begin{macrocode}
\section{one}
some text in chapter one
%    \end{macrocode}

%\iffalse
%</samplechap1>
%\fi
% Some text for chapter 2:
%\iffalse
%<*samplechap2>
%\fi
%    \begin{macrocode}
\section{two}
more text in chapter two
%    \end{macrocode}

%\iffalse
%</samplechap2>
%\fi
%
% %%%%%%%%%%%%%%%%%%%%%%%%%%%%%%%%%%%%%%
% \paragraph{Part Include Files.}
%
% The include files are called |cdocspt3.tex| and |cdocspt4.tex|.
%
%\iffalse
%<*samplepart3|samplepart4>
%\fi

% Optional override for |\version| flag:
%    \begin{macrocode}
%%\providecommand{\version}{final}
%    \end{macrocode}

% Include the main document:
%    \begin{macrocode}
\input{childdoc.def}
\childdocby{cdocsamp}
%    \end{macrocode}

%\iffalse
%</samplepart3|samplepart4>
%\fi
%
%\iffalse
%<*samplepart3>
%\fi
% Some text for part 3:
%    \begin{macrocode}
some text in part three
%    \end{macrocode}

%\iffalse
%</samplepart3>
%\fi
% Some text for part 4:
%\iffalse
%<*samplepart4>
%\fi
%    \begin{macrocode}
more text in part four
%    \end{macrocode}

%\iffalse
%</samplepart4>
%\fi
%
% %%%%%%%%%%%%%%%%%%%%%%%%%%%%%%%%%%%%%%
% \paragraph{Forwarding for a Complete Draft.}
%
% The following forwarding file |cdocsdrf.tex|
% compiles the main document in draft mode:
%\iffalse
%<*sampledraft>
%\fi
%    \begin{macrocode}
\def\version{draft}
\input{childdoc.def}
\childdocforward{cdocsamp}
%    \end{macrocode}

%\iffalse
%</sampledraft>
%\fi
%
% %%%%%%%%%%%%%%%%%%%%%%%%%%%%%%%%%%%%%%
% \paragraph{Forwarding for Final Version of the Chapters.}
%
% The following forwarding files |cdocsfn1.tex| and |cdocsfn2.tex|
% (with identical content)
% compile the final versions of the child documents
% |cdocsch1.tex| and |cdocsch2.tex|, respectively:
%\iffalse
%<*samplefinal>
%\fi
%    \begin{macrocode}
\def\version{final}
\input{childdoc.def}
\childdocforwardprefix[cdocsamp]{cdocsfn}{cdocsch}
%    \end{macrocode}

%\iffalse
%</samplefinal>
%\fi
%
% %%%%%%%%%%%%%%%%%%%%%%%%%%%%%%%%%%%%%%
% \paragraph{Command Line Processing.}
%
% The following three command lines generate the output files
% |cdocscld|, |cdocscl1| and |cdocscl2|
% which should be identical to
% |cdocsdrf|, |cdocsch1| and |cdocsfn2|, respectively:
% \begin{center}
% \begin{tabular}{l}
% |latex -jobname cdocscld \|\\
% |  "\def\version{draft}\input{childdoc.def}\childdocforward{cdocsamp}"|\\
% |latex -jobname cdocscl1 \|\\
% |  "\input{childdoc.def}\childdocforward[cdocsamp]{cdocsch1}"|\\
% |latex -jobname cdocscl2 \|\\
% |  "\def\version{final}\input{childdoc.def}\childdocforward{cdocsch2}"|
% \end{tabular}
% \end{center}
% Note that the trailing backslash on each first line
% merely continues the input to the second line
% (for convenient cut ant paste).
% Furthermore, the command |latex| can be replaced by any
% of its alternative versions such as |pdflatex|.
%
% %%%%%%%%%%%%%%%%%%%%%%%%%%%%%%%%%%%%%%%%%%%%%%%%%%%%%%%%%%%%%%%%%%%%%%%%%%%%%%
% %%%%%%%%%%%%%%%%%%%%%%%%%%%%%%%%%%%%%%%%%%%%%%%%%%%%%%%%%%%%%%%%%%%%%%%%%%%%%%
% \section{Implementation}
%\iffalse
%<*package>
%\fi
%
% This section describes the definitions file |childdoc.def|.

% The definitions cannot be loaded using |\usepackage| or |\RequirePackage|
% which has a mechanism to prevent loading a style file more than once.
% When loading the definitions by means of |\input|
% multiple instances have to be prevented manually:
%\iffalse
%This code needs to be before the `\ProvidesFile' directive
%which is defined at the beginning of this file.
%Therefore it is also placed there and commented out here.
%</package>
%<*discard>
%\fi
%    \begin{macrocode}
\ifdefined\childdocmain\endinput\fi
%    \end{macrocode}
%\iffalse
%</discard>
%<*package>
%\fi
%
% \macro{\ifchilddoc}
% \macro{\ifchilddocmanual}
% The conditional |\ifchilddoc| tells whether a
% child (true) or main (false) document is being compiled.
% The conditional |\ifchilddocmanual| tells whether
% the |\includeonly| mechanism is used (false) or
% the selection of child files must be performed manually (true).
% The definitions initialise to false:
%    \begin{macrocode}
\newif\ifchilddoc
\newif\ifchilddocmanual
%    \end{macrocode}

% \macro{\childdocname}
% \macro{\childdocjob}
% The macro |\childdocname| stores the name of the main document
% to be compiled. The macro |\childdocjob| stores the name of
% the document on which the \LaTeX{} compiler was originally invoked.
% The content of |\jobname| cannot be compared
% to filenames specified in the source due to different catcodes.
% The following code rescans |\jobname|, stores the result
% in |\childdocname| and saves a copy in |\childdocjob|:
%    \begin{macrocode}
\edef\childdocname{\scantokens\expandafter{\jobname\noexpand}}
\let\childdocjob\childdocname
%    \end{macrocode}

% \macro{\childdocdisable}
% The macro |\childdocdisable| prevents the main file
% from being processed more than once.
% At this stage, the main document command |\childdocmain|
% is assumed to be called once again where it should do nothing.
% Any subsequent call to it should prevent
% a secondary processing of the main document
% It overwrites the forwarding commands
% |\childdocof| and |\childdocforward|
% with empty macros to prevent further inclusions of the main document:
%    \begin{macrocode}
\newcommand{\childdocdisable}
{
  \renewcommand{\childdocmain}[1]{\renewcommand{\childdocmain}[1]{\endinput}}
  \renewcommand{\childdocof}[1]{}
  \renewcommand{\childdocby}[2][]{}
  \renewcommand{\childdocforward}[2][]{}
  \renewcommand{\childdocdisable}{}
}
%    \end{macrocode}

% \macro{\childdocmain}
% The macro |\childdocmain| is to be called at the top of the main file
% with nothing or the main filename (without extension) as argument.
% First, it breaks loops.
% If the argument is not empty and does not match |\childdocname|
% (which is set by the first inclusion of |childdoc.def|),
% |\ifchilddoc| is set to true, |\includeonly| is applied to the child file
% and |\jobname| is set to the main file
% (for proper handling of |.aux| files):
%    \begin{macrocode}
\newcommand{\childdocmain}[1]
{
  \childdocdisable\childdocmain{}
  \if?#1?\else
    \begingroup
      \def\childdoctmp{#1}
      \ifx\childdoctmp\childdocname
        \def\childdoctmp{}
      \else
        \def\childdoctmp
        {
          \childdoctrue
          \includeonly{\childdocname}
          \def\childdocjob{#1}
          \def\jobname{#1}
        }
      \fi
      \expandafter
    \endgroup
    \childdoctmp
  \fi
}
%    \end{macrocode}

% \macro{\childdocof}
% The command |\childdocof| redirects
% compilation to the main file |#1|.
%    \begin{macrocode}
\newcommand{\childdocof}[1]
{
  \childdocdisable
  \childdoctrue
  \includeonly{\childdocname}
  \def\jobname{#1}
  \def\childdocjob{#1}
  \input{#1}
}
%    \end{macrocode}

% \macro{\childdocby}
% The command |\childdocby| ....
%    \begin{macrocode}
\newcommand{\childdocby}[2][]
{
  \childdocdisable
  \childdoctrue
  \childdocmanualtrue
  \if?#1?\else
    \def\jobname{#2}
  \fi
  \def\childdocjob{#2}
  \input{#2}
  \endinput
}
%    \end{macrocode}

% \macro{\childdocforward}
% The command |\childdocforward| redirects
% compilation to the main file or
% (if the optional argument is given) a child file.
% Parameters are set as if the main file
% or a child file starting with |\childdocof| was compiled.
% Then compilation is handed over to the main file:
%    \begin{macrocode}
\newcommand{\childdocforward}[2][]
{
  \begingroup
    \if?#1?
      \def\childdoctmp
      {
        \def\childdocname{#2}
        \def\childdocjob{#2}
        \def\jobname{#2}
        \input{#2}
        \endinput
      }
    \else
      \def\childdoctmp
      {
        \childdocdisable
        \def\childdocname{#2}
        \childdoctrue
        \includeonly{#2}
        \def\childdocjob{#1}
        \def\jobname{#1}
        \input{#1}
        \endinput
      }
    \fi
    \expandafter
  \endgroup
  \childdoctmp
}
%    \end{macrocode}

% \macro{\childdocforwardprefix}
% The command |\childdocforwardprefix| redirects
% compilation to the main or a child file by means of a pattern.
% The prefix |#1| in the current filename is replaced by |#2|
% and the suffix of the current filename is kept
% (it is assumed that the filename does not contain the substring `|~~~|'
% which is used as a delimiter).
% Compilation is handed over to the new file by |\childdocforward|:
%    \begin{macrocode}
\newcommand{\childdocforwardprefix}[3][]
{
  \begingroup
    \def\childdocextract #2##1~~~{\def\childdoctmp{\childdocforward[#1]{#3##1}}}
    \expandafter\childdocextract\childdocname~~~
    \expandafter
  \endgroup
  \childdoctmp
}
%    \end{macrocode}

% \macro{\childdoc}
% The deprecated macro |\childdoc| is a legacy version of |\childdocmain|:
%    \begin{macrocode}
\newcommand{\childdoc}{\childdocmain}
%    \end{macrocode}

% \macro{\childdocredirect}
% The deprecated macro |\childdocredirect| is a legacy version
% of |\childdocforward| and |\childdocforwardprefix|:
%    \begin{macrocode}
\newcommand{\childdocredirect}[2][]
{
  \begingroup
    \if?#1?
      \def\childdoctmp{\childdocforward{#2}}
    \else
      \def\childdoctmp{\childdocforwardprefix{#1}{#2}}
    \fi
    \expandafter
  \endgroup
  \childdoctmp
}
%    \end{macrocode}

%\iffalse
%</package>
%\fi
%
\endinput

\childdocforward{cdocsamp}
%    \end{macrocode}

%\iffalse
%</sampledraft>
%\fi
%
% %%%%%%%%%%%%%%%%%%%%%%%%%%%%%%%%%%%%%%
% \paragraph{Forwarding for Final Version of the Chapters.}
%
% The following forwarding files |cdocsfn1.tex| and |cdocsfn2.tex|
% (with identical content)
% compile the final versions of the child documents
% |cdocsch1.tex| and |cdocsch2.tex|, respectively:
%\iffalse
%<*samplefinal>
%\fi
%    \begin{macrocode}
\def\version{final}
% \iffalse
%
% childdoc.dtx Copyright (C) 2017-2018 Niklas Beisert
%
% This work may be distributed and/or modified under the
% conditions of the LaTeX Project Public License, either version 1.3
% of this license or (at your option) any later version.
% The latest version of this license is in
%   http://www.latex-project.org/lppl.txt
% and version 1.3 or later is part of all distributions of LaTeX
% version 2005/12/01 or later.
%
% This work has the LPPL maintenance status `maintained'.
%
% The Current Maintainer of this work is Niklas Beisert.
%
% This work consists of the files childdoc.dtx and childdoc.ins
% and the derived files childdoc.def and cdocsamp.tex with
% cdocsch1.tex, cdocsch2.tex, cdocsdrf.tex, cdocsfn1.tex, cdocsfn2.tex.
%
%<package>\ifdefined\childdocmain\endinput\fi
%<package>\ProvidesFile{childdoc.def}[2018/12/30 v2.0 child document driver]
%<samplemain>\ProvidesFile{cdocsamp.tex}[2018/12/30 v2.0 sample for childdoc]
%<*driver>
%\ProvidesFile{childdoc.drv}[2018/12/30 v2.0 childdoc reference manual file]
\PassOptionsToClass{10pt,a4paper}{article}
\documentclass{ltxdoc}

\usepackage[margin=35mm]{geometry}
\usepackage{hyperref}
\usepackage{hyperxmp}
\usepackage[usenames]{color}

\hypersetup{colorlinks=true}
\hypersetup{pdfstartview=FitH}
\hypersetup{pdfpagemode=UseNone}
\hypersetup{pdfsource={}}
\hypersetup{pdflang={en-UK}}
\hypersetup{pdfcopyright={Copyright 2017-2018 Niklas Beisert.
  This work may be distributed and/or modified under the
  conditions of the LaTeX Project Public License, either version 1.3
  of this license or (at your option) any later version.}}
\hypersetup{pdflicenseurl={http://www.latex-project.org/lppl.txt}}
\hypersetup{pdfcontactaddress={ETH Zurich, ITP, HIT K,
  Wolfgang-Pauli-Strasse 27}}
\hypersetup{pdfcontactpostcode={8093}}
\hypersetup{pdfcontactcity={Zurich}}
\hypersetup{pdfcontactcountry={Switzerland}}
\hypersetup{pdfcontactemail={nbeisert@itp.phys.ethz.ch}}
\hypersetup{pdfcontacturl={http://people.phys.ethz.ch/\xmptilde nbeisert/}}

\newcommand{\secref}[1]{\hyperref[#1]{section \ref*{#1}}}

\parskip1ex
\parindent0pt
\let\olditemize\itemize
\def\itemize{\olditemize\parskip0pt}

\begin{document}

\title{The \textsf{childdoc} Package}
\hypersetup{pdftitle={The childdoc Package}}
\author{Niklas Beisert\\[2ex]
  Institut f\"ur Theoretische Physik\\
  Eidgen\"ossische Technische Hochschule Z\"urich\\
  Wolfgang-Pauli-Strasse 27, 8093 Z\"urich, Switzerland\\[1ex]
  \href{mailto:nbeisert@itp.phys.ethz.ch}
  {\texttt{nbeisert@itp.phys.ethz.ch}}}
\hypersetup{pdfauthor={Niklas Beisert}}
\hypersetup{pdfsubject={Manual for the LaTeX2e Package childdoc}}
\date{30 December 2018, \textsf{v2.0}}
\maketitle

\begin{abstract}\noindent
\textsf{childdoc} is a \LaTeXe{} package
that enables the direct compilation
of document sections included by |\include|
to individual files.
\end{abstract}

\begingroup
\parskip0ex
\tableofcontents
\endgroup

%%%%%%%%%%%%%%%%%%%%%%%%%%%%%%%%%%%%%%%%%%%%%%%%%%%%%%%%%%%%%%%%%%%%%%%%%%%%%%%%
%%%%%%%%%%%%%%%%%%%%%%%%%%%%%%%%%%%%%%%%%%%%%%%%%%%%%%%%%%%%%%%%%%%%%%%%%%%%%%%%
\section{Introduction}

\LaTeX{} provides a mechanism to structure a large document (such as a book)
into a main file and several child files (containing the chapters)
using the |\include| command.
This mechanism is beneficial for documents
which span hundreds of pages in order to
make the source file(s) more manageable.
Moreover, compilation can be restricted to
selected child files by means of the |\includeonly| command.
The latter feature can be used to reduce the compilation time while editing
(this was significantly more useful in the earlier days of \LaTeX{})
or to generate a smaller document which is easier to navigate.
Another application of |\includeonly| is to generate
documents consisting of selected parts of the complete document.

However, there are a few drawbacks of the plain |\include| mechanism:
\begin{itemize}
\item
The child files cannot be compiled on their own,
they can only be compiled via the main file.
A naive editing environment
(such as a text editor with an option
to have the current file processed by \LaTeX)
may require one to switch to the main file before compiling;
attempting to compile the child file produces errors.
\item
The main file must be modified (each time)
to adjust the |\includeonly| command
to the present needs. This easily leaves the main file in a messy state.
\item
The generated document will always carry the filename
of the main document. This is inconvenient if
several child files are to be compiled and
to be kept for distribution.
\end{itemize}

The present package provides a simple interface
to make child files individually compilable by \LaTeX{}.
Compiling a child file then has the same effect as compiling
the main file with an |\includeonly| command
to select the appropriate child.
Moreover the generated document will carry the name of the child
rather than the main file.
This resolves all three above issues.

This feature is meant to make the editing of books,
thesis documents and lecture notes somewhat more convenient.
However, the package can also be used efficiently for
composing a series of documents (such as exercise sheets)
which are typically distributed individually.
It then assists the author in generating the individual documents
(potentially in different versions)
as well as a document containing the collected series.
Another application is in developing style files
or other kinds of included material
where compilation of the style file could redirect
to a sample or test file.

%%%%%%%%%%%%%%%%%%%%%%%%%%%%%%%%%%%%%%%%%%%%%%%%%%%%%%%%%%%%%%%%%%%%%%%%%%%%%%%%
%%%%%%%%%%%%%%%%%%%%%%%%%%%%%%%%%%%%%%%%%%%%%%%%%%%%%%%%%%%%%%%%%%%%%%%%%%%%%%%%
\section{Usage}

First of all, the package \textsf{childdoc} is \emph{not} a standard
\LaTeXe{} |.sty| style file! Therefore it needs to be invoked in
a non-standard way.

%%%%%%%%%%%%%%%%%%%%%%%%%%%%%%%%%%%%%%%%%%%%%%%%%%%%%%%%%%%%%%%%%%%%%%%%%%%%%%%%
\subsection{Included Files}
\label{sec:include}

%%%%%%%%%%%%%%%%%%%%%%%%%%%%%%%%%%%%%%%%
\DescribeMacro{\childdocmain}
To use the package, add the commands
\begin{center}
\begin{tabular}{l}
|\input{childdoc.def}|\\
|\childdocmain{}|\\
\end{tabular}
\end{center}
at the very top of the main \LaTeX{} file,
in particular \emph{before} the |\documentclass| statement!
The argument of |\childdocmain| should be left empty
(but it must be present).

%%%%%%%%%%%%%%%%%%%%%%%%%%%%%%%%%%%%%%%%
\DescribeMacro{\childdocof}
Furthermore, add the commands
\begin{center}
\begin{tabular}{l}
|\input{childdoc.def}|\\
|\childdocof{|\textit{main}|}|\\
\end{tabular}
\end{center}
at the top of every child file \textit{child}
which is included by |\include{|\textit{child}|}|
from within the main file
(or at least for those files to be compiled individually).
The argument \textit{main} must be the filename of the main file.

There are a couple of
considerations in setting up the main and child documents:

%%%%%%%%%%%%%%%%%%%%%%%%%%%%%%%%%%%%%%%%
\paragraph{Restrictions.}

Please note the following restrictions:
\begin{itemize}
\item
|\childdocmain| must be called with one argument \textit{main}
to ensure compatibility with earlier version of the package.
It must either be empty (|\childdocmain{}|)
or precisely match the filename of the main file in which it is specified.
See \secref{sec:detection} for further information.
\item
The filename \textit{main} must be specified without the |.tex| extension.
\item
The filename \textit{main} is case sensitive
(even in case-insensitive file systems)
due to internal string comparison.
\item
The argument \textit{main} should be fully expanded, it cannot be a macro.
\item
Subdirectories and special characters should be avoided in filenames.
\item
The command |\childdocmain{|\textit{main}|}| must be followed by a whitespace.
It should not be followed immediately by another command
or by a comment mark `|%|'.
This is because the \TeX{} parser reads the token immediately following
the argument of |\childdocmain| and puts it
at the beginning of every child section;
however, a white\-space is ignored.
\end{itemize}

%%%%%%%%%%%%%%%%%%%%%%%%%%%%%%%%%%%%%%%%
\paragraph{Content of Main File.}

It is advisable to place all content in the child files included by |\include|.
Any output contained in the main file will appear in all child documents
unless suppressed manually;
it cannot be suppressed automatically by the |\includeonly| directive
and thus should normally be avoided.
A method to include some content in the main file
by means of conditional processing is described in \secref{sec:conditional}.

%%%%%%%%%%%%%%%%%%%%%%%%%%%%%%%%%%%%%%%%
\paragraph{Page Numbering.}

When only a part of the document is compiled,
the appropriate numbering of pages
(as well as other status parameters)
is determined from the |.aux| files.
The latter contain information from previous passes.
However this information needs to propagate through
all intermediate child documents.
Therefore the page numbering in child documents may well
be inconsistent until the complete document is compiled at least once.

A useful (if unconventional) way to always ensure a consistent
page numbering is to restart the numbering in each child document
and denote the pages by `\textit{child}|.|\textit{page}'
where \textit{child} represents the chapter/section number of the child file.
This can be achieved by the command
|\numberwithin{page}{|\textit{child}|}|
of the \textsf{amsmath} package
where \textit{child} can be |chapter| or |section|
depending on the chosen structuring.
Alternatively, one can modify the macro |\thepage| appropriately
and reset the counter |page| at the start of each child file.

%%%%%%%%%%%%%%%%%%%%%%%%%%%%%%%%%%%%%%%%%%%%%%%%%%%%%%%%%%%%%%%%%%%%%%%%%%%%%%%%
\subsection{Conditional Processing}
\label{sec:conditional}

The package provides a mechanism to compile different versions
of a document. To customise the versions further some conditional processing
can come in handy to distinguish which version is being compiled.
The package provides two macros to describe the compilation context:

%%%%%%%%%%%%%%%%%%%%%%%%%%%%%%%%%%%%%%%%
\DescribeMacro{\ifchilddoc}
The conditional |\ifchilddoc| distinguishes between the compilation of
child documents and the main document:
%
\begin{center}
|\ifchilddoc |\textit{child-code}| |[|\||else |\textit{main-code}]| \||fi|
\end{center}

%%%%%%%%%%%%%%%%%%%%%%%%%%%%%%%%%%%%%%%%
\DescribeMacro{\childdocname}
\DescribeMacro{\childdocjob}
The macro |\childdocname| contains the filename (without extension)
of the main or child file being processed.
Note that |\childdocjob| will always contain the name of the main file.

%%%%%%%%%%%%%%%%%%%%%%%%%%%%%%%%%%%%%%%%
\paragraph{Title Page.}

Conditional processing can be used to include a title or banner page
in the main document when proper precautions are taken.
Importantly, the code in the main file should ensure that the page counter
(as well as other status parameters which are stored in the |.aux| files)
takes the same value after the conditional processing.
Otherwise the page numbers may take divergent values
depending on which part is compiled.

For example, a title page could be declared by:
%
\begin{center}
\begin{tabular}{l}
|\ifchilddoc\||else|\\
|\addtocounter{page}{-1}|\\
\textit{code for title page}\\
|\newpage|\\
|\||fi|
\end{tabular}
\end{center}
%
A banner page for the child documents can be generated by:
%
\begin{center}
\begin{tabular}{l}
|\ifchilddoc|\\
|\addtocounter{page}{-1}|\\
\textit{code for banner page}\\
|\newpage|\\
|\||fi|
\end{tabular}
\end{center}
%
Here one could write a message such as:
\begin{center}
|This is the part \childdocname{} of \childdocjob{}.|
\end{center}

%%%%%%%%%%%%%%%%%%%%%%%%%%%%%%%%%%%%%%%%%%%%%%%%%%%%%%%%%%%%%%%%%%%%%%%%%%%%%%%%
\subsection{Flags}
\label{sec:flags}

The package makes it easy to generate different versions
of the main or child documents.
To this end compilation flags can be defined
and assigned different default values.
They will be particularly useful in conjunction
with the forwarding mechanism described in \secref{sec:forward}.

For example, it may be useful to have a flag |\version|
which can be set to |draft| or |final|.
The document source will contain some conditional code
depending on the value of |\version|.
Suppose further, the flag should default to |final| for the main file
and to |draft| for child files
which is a natural assignment for editing the document.
This is achieved by placing the following code
in the preamble of the main document
(below the |\childdocmain| directive):
%
\begin{center}
\begin{tabular}{l}
|\ifchilddoc|\\
|\providecommand{\version}{draft}|\\
|\||else|\\
|\providecommand{\version}{final}|\\
|\||fi|
\end{tabular}
\end{center}
%
The definition by |\providecommand| makes sure
that previous definitions are not overwritten.
Further statements |\providecommand{\version}{...}|
can thus be added before the above code to override it.

For the main file, one might add a line
(between |\childdocmain| and the above block)
%
\begin{center}
|%\ifchilddoc\||else\providecommand{\version}{draft}\||fi|
\end{center}
%
which can be uncommented to produce a draft version.
Likewise one can add a line to the very top of a child file
(above the |\childdocof{|\textit{main}|}| directive)
%
\begin{center}
|%\providecommand{\version}{final}|
\end{center}
%
which can be uncommented to produce the final version of this child document.

%%%%%%%%%%%%%%%%%%%%%%%%%%%%%%%%%%%%%%%%%%%%%%%%%%%%%%%%%%%%%%%%%%%%%%%%%%%%%%%%
\subsection{Forwarding}
\label{sec:forward}

Different versions of the main or child documents
using compilation flags as described in \secref{sec:flags}
can be (permanently) stored in different files
for convenient compilation, viewing and distribution.
To this end, the package defines a command
to pass on compilation to a different file:

%%%%%%%%%%%%%%%%%%%%%%%%%%%%%%%%%%%%%%%%
\DescribeMacro{\childdocforward}
The command |\childdocforward| redirects processing to
another source file:
%
\begin{center}
\begin{tabular}{l}
|\input{childdoc.def}|\\
|\childdocforward[|\textit{main}|]{|\textit{dest}|}|\\
\end{tabular}
\end{center}
%
The argument \textit{dest} is the destination file
(without extension).
It should be the main file or one of the child files.
Note that further \textsf{childdoc} directives
such as |\childdocof| and |\childdocforward|
in the indicated file will be processed in this form.
The optional argument \textit{main}
passes on directly to the main file \textit{main}
while pretending to compile the child \textit{dest}.
This form behaves as if \textit{dest}
issues |\childdocof{|\textit{main}|}| right away,
and no further \textsf{childdoc} directives will be processed.

%%%%%%%%%%%%%%%%%%%%%%%%%%%%%%%%%%%%%%%%
\DescribeMacro{\...prefix}
In the alternative form |\childdocforwardprefix|,
%
\begin{center}
\begin{tabular}{l}
|\input{childdoc.def}|\\
|\childdocforwardprefix[|\textit{main}|]{|\textit{prefix}|}{|\textit{dest}|}|
\end{tabular}
\end{center}
%
the destination file is determined by a pattern
depending on the current file:
To make this work, the current file must be called
`{\textit{prefix}\hspace{0.2em}\textit{suffix}}'
with \textit{prefix} matching precisely the argument.
Processing is then passed on to the file
`{\textit{dest}\hspace{0.2em}\textit{suffix}}'.
Surely, the same effect is achieved by
directly specifying the
argument `{\textit{dest}\hspace{0.2em}\textit{suffix}}'
in the first form.
However, that requires to set up a different file
for each child. With the alternative form of the command
all these files can have exactly the same content
which simplifies setting them up and maintaining them.

For example, the following file |draft.tex|
with a compilation flag |\version| as described in \secref{sec:flags}
compiles the main document as a draft:
%
\begin{center}
\begin{tabular}{l}
|\def\version{draft}|\\
|\input{childdoc.def}|\\
|\childdocforward{|\textit{main}|}|
\end{tabular}
\end{center}
%
Likewise, the following files |final|\textit{nn}|.tex|
compile the final version of the child document
|child|\textit{nn}|.tex|:
%
\begin{center}
\begin{tabular}{l}
|\def\version{final}|\\
|\input{childdoc.def}|\\
|\childdocforwardprefix{final}{child}|
\end{tabular}
\end{center}
%

Note that when several versions of a main file and/or of each child file
are to be generated, it may be convenient to set up a |Makefile| or
shell script to automatise the process.

%%%%%%%%%%%%%%%%%%%%%%%%%%%%%%%%%%%%%%%%%%%%%%%%%%%%%%%%%%%%%%%%%%%%%%%%%%%%%%%%
\subsection{Command Line Processing}
\label{sec:commandline}

The effect of redirection files can also be achieved by invoking
the \LaTeX{} compiler with a more elaborate command line.
Most conveniently this should be done as part
of a shell script or a |Makefile|.

When using \textsf{childdoc} in the main file, the following
command lines effectively perform a redirection
(note that depending on the shell being used,
backslashes may have to be doubled: `|\|' $\to$ `|\\|'):
%
\begin{center}
|... -jobname "|\textit{target}|" |\\|"|[\textit{flags}]%
|\input{childdoc.def}\childdocforward[|\textit{main}|]{|\textit{dest}|}"|
\end{center}
%
Here \textit{target} is the name of the output file,
\textit{main} is the name of the main file
and \textit{dest} is the name of the main or child file to be processed
(all filenames without extensions).
The optional argument \textit{main} can be omitted
if \textit{main} matches \textit{dest}.
Optionally, compilation \textit{flags} can be defined via |\def| commands.
This command line makes the \TeX{} engine believe
it is compiling the file \textit{target}
whose content is specified as the latter parameter.
The provided code then forwards the processing to
\textit{main} or \textit{dest} as described in \secref{sec:forward}.

%%%%%%%%%%%%%%%%%%%%%%%%%%%%%%%%%%%%%%%%%%%%%%%%%%%%%%%%%%%%%%%%%%%%%%%%%%%%%%%%
\subsection{Include by Input}
\label{sec:input}

Including child documents by |\include| has some restrictions by design.
Most notably, the content of a child document always occupies
its own set of pages; pages cannot be shared between child documents.
Usually, this behaviour makes perfect sense
because each child document contain an essential part of the document.
However, in some situations it may be desirable to compose
a document from a collection of parts
without having mandatory page breaks between then.
For this case, the package
provides a mechanism to include parts
by |\input| which can also be processed individually.
However, by construction this mechanism
requires manual handling of the content to be output.

%%%%%%%%%%%%%%%%%%%%%%%%%%%%%%%%%%%%%%%%
\DescribeMacro{\ifchilddocmanual}
The main file should be prepared as usual, see \secref{sec:include}.
However, the document body must make a distinction
between processing of an individual part and of the main document, e.g.:
%
\begin{center}
\begin{tabular}{l}
|\ifchilddocmanual|\\
|\input{\childdocname}|\\
|\||else|\\
\textit{document body with }|\input{|\textit{part}|}|\\
|\||fi|
\end{tabular}
\end{center}
%
The conditional |\ifchilddocmanual| is true whenever
a part to be included by |\input| is being compiled,
and the name of the part is stored in |\childdocname|.

%%%%%%%%%%%%%%%%%%%%%%%%%%%%%%%%%%%%%%%%
\DescribeMacro{\childdocby}
Each part to be included by |\input| should start with:
%
\begin{center}
\begin{tabular}{l}
|\input{childdoc.def}|\\
|\childdocby{|\textit{main}|}|\\
\end{tabular}
\end{center}
%
The directive |\childdocby| is similar to |\childdocof|
described in \secref{sec:include},
but the subsequent selection of content must be done manually.
To that end, both |\ifchilddoc| and |\ifchilddocmanual|
will be true upon processing of a part,
and the name of the part is stored in |\childdocname|.
Note that |\jobname| will be set to the filename of the current part
so that each part receives an individual |.aux| file
that does not interfere with the |.aux| file(s) of the main document.
This behaviour can be altered by the alternative form
|\childdocby[*]{|\textit{main}|}| (with a non-empty optional argument)
which uses the |.aux| file of the main document
by setting |\jobname| to \textit{main}.

%%%%%%%%%%%%%%%%%%%%%%%%%%%%%%%%%%%%%%%%%%%%%%%%%%%%%%%%%%%%%%%%%%%%%%%%%%%%%%%%
\subsection{Driver Development}
\label{sec:driver}

The \textsf{childdoc} mechanism can also be use for the development
of definition files such as \LaTeX{} styles or classes.
This case differs from the above setup with multiple parts
included by |\include| in that no |\includeonly| should be invoked.
This can be achieved by starting the include file
(before |\ProvidesPackage|) with:
%
\begin{center}
\begin{tabular}{l}
|\input{childdoc.def}|\\
|\childdocforward{|\textit{main}|}|\\
\end{tabular}
\end{center}
%
or alternatively with:
%
\begin{center}
\begin{tabular}{l}
|\input{childdoc.def}|\\
|\childdocby{|\textit{main}|}|\\
\end{tabular}
\end{center}
%
Both forms have slightly different effects as described above.
The main file is prepared as usual, see \secref{sec:include}.

%%%%%%%%%%%%%%%%%%%%%%%%%%%%%%%%%%%%%%%%%%%%%%%%%%%%%%%%%%%%%%%%%%%%%%%%%%%%%%%%
\subsection{Legacy Detection}
\label{sec:detection}

The directive |\childdocmain| in the main file can detect
whether the complete document or merely a child is to be compiled
even without using the directive |\childdocof|.
This method is deprecated because it is less robust
and there is no compelling reason to use it;
it is merely provided for backward compatibility
and it may be removed in future versions.

If the detection mechanism is to be used,
it is mandatory to correctly specify
the filename of the main file as the argument of |\childdocmain|:
%
\begin{center}
\begin{tabular}{l}
|\input{childdoc.def}|\\
|\childdocmain{|\textit{main}|}|\\
\end{tabular}
\end{center}
%
If |\jobname| does not match the argument \textit{main} of |\childdocmain|,
it is assumed that |\jobname| points to the child file to be compiled.
When using |\childdocmain| with the main file specified as argument,
it suffices to start a child file
with just |\input{|\textit{main}|}|
without loading of the package and using |\childdocof|.
If instead all processing is done
with the appropriate \textsf{childdoc} directives,
the argument of \textit{main} of |\childdocmain| can be empty.

An alternative version of the command line processing described
in \secref{sec:commandline} using the detection mechanism reads:
%
\begin{center}
|... -jobname "|\textit{target}|" "|[\textit{flags}]%
[|\def\jobname{|\textit{dest}|}|]|\input{|\textit{main}|}"|
\end{center}

%%%%%%%%%%%%%%%%%%%%%%%%%%%%%%%%%%%%%%%%%%%%%%%%%%%%%%%%%%%%%%%%%%%%%%%%%%%%%%%%
\subsection{Manual Code}
\label{sec:manual}

In case one cannot be certain whether the definitions file |childdoc.def|
is installed on the target \TeX{} distribution
and one prefers not to ship it,
it is conceivable to paste a few relevant commands into the sources.

To that end, drop all statements |\input{childdoc.def}|
and perform the replacements as outlined below.
Instead of |\childdocmain{|\textit{main}|}| add the following code
to the top of the main file:
%
\begin{center}
\begin{tabular}{l}
|\||ifdefined\childdocname\endinput\||fi\newif\ifchilddoc|\\
|\edef\childdocname{\scantokens\expandafter{\jobname\noexpand}}|\\
|\def\childdocmain{|\textit{main}|}\||ifx\childdocmain\childdocname\||else|\\
|\childdoctrue\includeonly{\childdocname}\let\jobname\childdocmain\||fi|\\
\end{tabular}
\end{center}
%
Instead of |\childdocof{|\textit{main}|}| just include the main file
at the top of each child file:
%
\begin{center}
|\input{|\textit{main}|}|
\end{center}
%
A simple redirection |\childdocforward{|\textit{dest}|}| is achieved by:
%
\begin{center}
|\def\jobname{|\textit{dest}|}\input{\jobname}|
\end{center}
%
The redirection with prefix
|\childdocforwardprefix[|\textit{prefix}|]{|\textit{dest}|}|
is accomplished by:
%
\begin{center}
\begin{tabular}{l}
|{\edef\jobname{\scantokens\expandafter{\jobname\noexpand}}|\\
|\def\redirectjob |\textit{prefix}|#1~~~{\gdef\jobname{|\textit{dest}|#1}}|\\
|\expandafter\redirectjob\jobname~~~}\input{\jobname}|
\end{tabular}
\end{center}

In an alternative approach,
child documents can be compiled by a specific command line
without additional code or specific definitions:
%
\begin{center}
|... -jobname "|\textit{target}|" "|[\textit{flags}]%
|\includeonly{|\textit{dest}|}\input{|\textit{main}|}"|
\end{center}
%

%%%%%%%%%%%%%%%%%%%%%%%%%%%%%%%%%%%%%%%%%%%%%%%%%%%%%%%%%%%%%%%%%%%%%%%%%%%%%%%%
%%%%%%%%%%%%%%%%%%%%%%%%%%%%%%%%%%%%%%%%%%%%%%%%%%%%%%%%%%%%%%%%%%%%%%%%%%%%%%%%
\section{Information}

%%%%%%%%%%%%%%%%%%%%%%%%%%%%%%%%%%%%%%%%%%%%%%%%%%%%%%%%%%%%%%%%%%%%%%%%%%%%%%%%
\subsection{Copyright}

Copyright \copyright{} 2017--2018 Niklas Beisert

This work may be distributed and/or modified under the
conditions of the \LaTeX{} Project Public License, either version 1.3
of this license or (at your option) any later version.
The latest version of this license is in
  \url{http://www.latex-project.org/lppl.txt}
and version 1.3 or later is part of all distributions of \LaTeX{}
version 2005/12/01 or later.

This work has the LPPL maintenance status `maintained'.

The Current Maintainer of this work is Niklas Beisert.

This work consists of the files |README.txt|, |childdoc.ins| and |childdoc.dtx|
as well as the derived files |childdoc.def|, |cdocsamp.tex|
with |cdocsch1.tex|, |cdocsch2.tex|, |cdocspt3.tex|, |cdocspt4.tex|,
|cdocsdrf.tex|, |cdocsfn1.tex|, |cdocsfn2.tex|
as well as |childdoc.pdf|.

%%%%%%%%%%%%%%%%%%%%%%%%%%%%%%%%%%%%%%%%%%%%%%%%%%%%%%%%%%%%%%%%%%%%%%%%%%%%%%%%
\subsection{Files and Installation}

The package consists of the files:
%
\begin{center}
\begin{tabular}{ll}
    |README.txt|   & readme file \\
    |childdoc.ins| & installation file \\
    |childdoc.dtx| & source file \\
    |childdoc.def| & definition file \\
    |cdocsamp.tex| & sample main file \\
    |cdocsch1.tex| & sample include file \\
    |cdocsch2.tex| & sample include file \\
    |cdocspt3.tex| & sample part file \\
    |cdocspt4.tex| & sample part file \\
    |cdocsdrf.tex| & sample redirection file \\
    |cdocsfn1.tex| & sample redirection file \\
    |cdocsfn2.tex| & sample redirection file \\
    |childdoc.pdf| & manual
\end{tabular}
\end{center}
%
The distribution consists of the files
|README.txt|, |childdoc.ins| and |childdoc.dtx|.
%
\begin{itemize}
\item
Run (pdf)\LaTeX{} on |childdoc.dtx|
to compile the manual |childdoc.pdf| (this file).
\item
Run \LaTeX{} on |childdoc.ins| to create the definitions file |childdoc.def|
and the sample |cdocsamp.tex| with include files
|cdocsch1.tex|, |cdocsch2.tex|, |cdocspt3.tex|, |cdocspt4.tex|,
|cdocsdrf.tex|, |cdocsfn1.tex|, |cdocsfn2.tex|.
Then copy the file |childdoc.def| to an appropriate directory of your \LaTeX{}
distribution, e.g.\ \textit{texmf-root}|/tex/latex/childdoc|.
\end{itemize}

%%%%%%%%%%%%%%%%%%%%%%%%%%%%%%%%%%%%%%%%%%%%%%%%%%%%%%%%%%%%%%%%%%%%%%%%%%%%%%%%
\subsection{Related CTAN Packages}

There are several other packages which offer a similar functionality:
%
\begin{itemize}
\item
The packages
\href{http://ctan.org/pkg/docmute}{\textsf{docmute}},
\href{http://ctan.org/pkg/includex}{\textsf{includex}} and
\href{http://ctan.org/pkg/standalone}{\textsf{standalone}}
provide commands to include only the document body of
a child file thus allowing both files to be compiled individually.
\item
The packages \href{http://ctan.org/pkg/subdocs}{\textsf{subdocs}}
and \href{http://ctan.org/pkg/subfiles}{\textsf{subfiles}}
provide structures in which the main and child documents can be
encapsulated and allowing them to be compiled individually.
The inclusion mechanism is different from the conventional |\include|.
\item
The package \href{http://ctan.org/pkg/combine}{\textsf{combine}}
is an elaborate solution to combine several documents into one.
\end{itemize}
%
See also the CTAN topic \href{http://ctan.org/topic/subdocs}{\textsf{subdocs}}
for further related packages.
The present package differs from the above solutions in that
a document structure constructed with the conventional |\include| mechanism
just needs two extra commands at the top of every file
such that all constituent files can be compiled individually.

%%%%%%%%%%%%%%%%%%%%%%%%%%%%%%%%%%%%%%%%%%%%%%%%%%%%%%%%%%%%%%%%%%%%%%%%%%%%%%%%
%\subsection{Feature Suggestions}
%
%The following is a list of features which may be useful for future
%versions of this package:
%%
%\begin{itemize}
%\item
%\ldots
%\end{itemize}

%%%%%%%%%%%%%%%%%%%%%%%%%%%%%%%%%%%%%%%%%%%%%%%%%%%%%%%%%%%%%%%%%%%%%%%%%%%%%%%%
\subsection{Revision History}

%%%%%%%%%%%%%%%%%%%%%%%%%%%%%%%%%%%%%%%%
\paragraph{v2.0:} 2018/12/30

\begin{itemize}
\item
immediate forward processing
\item
added |\childdocby| mechanism
\item
manual restructured
\end{itemize}

%%%%%%%%%%%%%%%%%%%%%%%%%%%%%%%%%%%%%%%%
\paragraph{v1.6:} 2018/01/17

\begin{itemize}
\item
application for development of include files
\item
corrections to manual
\end{itemize}

%%%%%%%%%%%%%%%%%%%%%%%%%%%%%%%%%%%%%%%%
\paragraph{v1.5:} 2017/05/21

\begin{itemize}
\item
more complete structuring introduced
\item
|\childdocof| introduced
\item
|\childdoc| renamed to |\childdocmain|
\item
|\childredirect| renamed to |\childdocforward| and |\childdocforwardprefix|
and functionality expanded
\end{itemize}

%%%%%%%%%%%%%%%%%%%%%%%%%%%%%%%%%%%%%%%%
\paragraph{v1.0:} 2017/04/27

\begin{itemize}
\item
manual and install package
\item
first version published on CTAN
\end{itemize}

%%%%%%%%%%%%%%%%%%%%%%%%%%%%%%%%%%%%%%%%
\paragraph{v0.6:} 2017/04/26

\begin{itemize}
\item
redirection mechanism added
\end{itemize}

%%%%%%%%%%%%%%%%%%%%%%%%%%%%%%%%%%%%%%%%
\paragraph{v0.5:} 2017/04/26

\begin{itemize}
\item
functionality in definition file
\end{itemize}


%%%%%%%%%%%%%%%%%%%%%%%%%%%%%%%%%%%%%%%%%%%%%%%%%%%%%%%%%%%%%%%%%%%%%%%%%%%%%%%%
%%%%%%%%%%%%%%%%%%%%%%%%%%%%%%%%%%%%%%%%%%%%%%%%%%%%%%%%%%%%%%%%%%%%%%%%%%%%%%%%
%%%%%%%%%%%%%%%%%%%%%%%%%%%%%%%%%%%%%%%%%%%%%%%%%%%%%%%%%%%%%%%%%%%%%%%%%%%%%%%%
\appendix

\settowidth\MacroIndent{\rmfamily\scriptsize 000\ }

 \DocInput{childdoc.dtx}

\end{document}
%</driver>
% \fi
%
% %%%%%%%%%%%%%%%%%%%%%%%%%%%%%%%%%%%%%%%%%%%%%%%%%%%%%%%%%%%%%%%%%%%%%%%%%%%%%%
% %%%%%%%%%%%%%%%%%%%%%%%%%%%%%%%%%%%%%%%%%%%%%%%%%%%%%%%%%%%%%%%%%%%%%%%%%%%%%%
% \section{Sample}
%\iffalse
%<*samplemain>
%\fi
%
% The following presents a sample document
% with two chapters, two parts, a title page,
% a compile flag as well as three forwarding files to set the flag.
% It consists of eight |.tex| files:
% \begin{center}
% \begin{tabular}{ll}
% |cdocsamp.tex|&main file\\
% |cdocsch1.tex|&include file for chapter 1\\
% |cdocsch2.tex|&include file for chapter 2\\
% |cdocspt3.tex|&include file for part 3\\
% |cdocspt4.tex|&include file for part 4\\
% |cdocsdrf.tex|&forwarding file for main file in draft mode\\
% |cdocsfi1.tex|&forwarding file for final version of chapter 1\\
% |cdocsfi2.tex|&forwarding file for final version of chapter 2\\
% \end{tabular}
% \end{center}
% Each of the eight files can be compiled directly by the \LaTeX{} compiler.
%
% %%%%%%%%%%%%%%%%%%%%%%%%%%%%%%%%%%%%%%
% \paragraph{Main File.}
%
% The main file is called |cdocsamp.tex|.
%
% Load the \textsf{childdoc} definitions and
% declare the filename for the main document:
%    \begin{macrocode}
\input{childdoc.def}
\childdocmain{}
%    \end{macrocode}

% Optional override for |\version| flag:
%    \begin{macrocode}
%%\ifchilddoc\else\providecommand{\version}{draft}\fi
%    \end{macrocode}

% Define the default values for the |\version| flag
% (|final| for the main file and |draft| for childs):
%    \begin{macrocode}
\ifchilddoc
\providecommand{\version}{draft}
\else
\providecommand{\version}{final}
\fi
%    \end{macrocode}

% Load the standard document class:
%    \begin{macrocode}
\documentclass[12pt]{article}
%    \end{macrocode}

% Start the document body:
%    \begin{macrocode}
\begin{document}
%    \end{macrocode}

% Declare a title page.
% Print title, part of document being processed and version flag:
%    \begin{macrocode}
\addtocounter{page}{-1}
\begin{center}
{\LARGE\bfseries{}childdoc example\par}
\vspace{1cm}
\ifchilddoc
\ifchilddocmanual part\else chapter\fi:
`\childdocname' of `\childdocjob'\par
\else
main document: `\childdocjob'\par
\fi
version: \version\par
\end{center}
\newpage
%    \end{macrocode}

% Manually include selected file,
% otherwise process as usual:
%    \begin{macrocode}
\ifchilddocmanual
\section*{part `\childdocname'}
\input{\childdocname}
\else
%    \end{macrocode}

% Include the two chapters:
%    \begin{macrocode}
\include{cdocsch1}
\include{cdocsch2}
%    \end{macrocode}

% Include the two parts unless only chapters should be displayed:
%    \begin{macrocode}
\ifchilddoc\else
\section{part three}
\input{cdocspt3}
\section{part four}
\input{cdocspt4}
\fi
%    \end{macrocode}

% Process as usual until here:
%    \begin{macrocode}
\fi
%    \end{macrocode}

% End of document body:
%    \begin{macrocode}
\end{document}
%    \end{macrocode}
%\iffalse
%</samplemain>
%\fi
%
% %%%%%%%%%%%%%%%%%%%%%%%%%%%%%%%%%%%%%%
% \paragraph{Chapter Include Files.}
%
% The include files are called |cdocsch1.tex| and |cdocsch2.tex|.
%
%\iffalse
%<*samplechap1|samplechap2>
%\fi

% Optional override for |\version| flag:
%    \begin{macrocode}
%%\providecommand{\version}{final}
%    \end{macrocode}

% Include the main document:
%    \begin{macrocode}
\input{childdoc.def}
\childdocof{cdocsamp}
%    \end{macrocode}

%\iffalse
%</samplechap1|samplechap2>
%\fi
%
%\iffalse
%<*samplechap1>
%\fi
% Some text for chapter 1:
%    \begin{macrocode}
\section{one}
some text in chapter one
%    \end{macrocode}

%\iffalse
%</samplechap1>
%\fi
% Some text for chapter 2:
%\iffalse
%<*samplechap2>
%\fi
%    \begin{macrocode}
\section{two}
more text in chapter two
%    \end{macrocode}

%\iffalse
%</samplechap2>
%\fi
%
% %%%%%%%%%%%%%%%%%%%%%%%%%%%%%%%%%%%%%%
% \paragraph{Part Include Files.}
%
% The include files are called |cdocspt3.tex| and |cdocspt4.tex|.
%
%\iffalse
%<*samplepart3|samplepart4>
%\fi

% Optional override for |\version| flag:
%    \begin{macrocode}
%%\providecommand{\version}{final}
%    \end{macrocode}

% Include the main document:
%    \begin{macrocode}
\input{childdoc.def}
\childdocby{cdocsamp}
%    \end{macrocode}

%\iffalse
%</samplepart3|samplepart4>
%\fi
%
%\iffalse
%<*samplepart3>
%\fi
% Some text for part 3:
%    \begin{macrocode}
some text in part three
%    \end{macrocode}

%\iffalse
%</samplepart3>
%\fi
% Some text for part 4:
%\iffalse
%<*samplepart4>
%\fi
%    \begin{macrocode}
more text in part four
%    \end{macrocode}

%\iffalse
%</samplepart4>
%\fi
%
% %%%%%%%%%%%%%%%%%%%%%%%%%%%%%%%%%%%%%%
% \paragraph{Forwarding for a Complete Draft.}
%
% The following forwarding file |cdocsdrf.tex|
% compiles the main document in draft mode:
%\iffalse
%<*sampledraft>
%\fi
%    \begin{macrocode}
\def\version{draft}
\input{childdoc.def}
\childdocforward{cdocsamp}
%    \end{macrocode}

%\iffalse
%</sampledraft>
%\fi
%
% %%%%%%%%%%%%%%%%%%%%%%%%%%%%%%%%%%%%%%
% \paragraph{Forwarding for Final Version of the Chapters.}
%
% The following forwarding files |cdocsfn1.tex| and |cdocsfn2.tex|
% (with identical content)
% compile the final versions of the child documents
% |cdocsch1.tex| and |cdocsch2.tex|, respectively:
%\iffalse
%<*samplefinal>
%\fi
%    \begin{macrocode}
\def\version{final}
\input{childdoc.def}
\childdocforwardprefix[cdocsamp]{cdocsfn}{cdocsch}
%    \end{macrocode}

%\iffalse
%</samplefinal>
%\fi
%
% %%%%%%%%%%%%%%%%%%%%%%%%%%%%%%%%%%%%%%
% \paragraph{Command Line Processing.}
%
% The following three command lines generate the output files
% |cdocscld|, |cdocscl1| and |cdocscl2|
% which should be identical to
% |cdocsdrf|, |cdocsch1| and |cdocsfn2|, respectively:
% \begin{center}
% \begin{tabular}{l}
% |latex -jobname cdocscld \|\\
% |  "\def\version{draft}\input{childdoc.def}\childdocforward{cdocsamp}"|\\
% |latex -jobname cdocscl1 \|\\
% |  "\input{childdoc.def}\childdocforward[cdocsamp]{cdocsch1}"|\\
% |latex -jobname cdocscl2 \|\\
% |  "\def\version{final}\input{childdoc.def}\childdocforward{cdocsch2}"|
% \end{tabular}
% \end{center}
% Note that the trailing backslash on each first line
% merely continues the input to the second line
% (for convenient cut ant paste).
% Furthermore, the command |latex| can be replaced by any
% of its alternative versions such as |pdflatex|.
%
% %%%%%%%%%%%%%%%%%%%%%%%%%%%%%%%%%%%%%%%%%%%%%%%%%%%%%%%%%%%%%%%%%%%%%%%%%%%%%%
% %%%%%%%%%%%%%%%%%%%%%%%%%%%%%%%%%%%%%%%%%%%%%%%%%%%%%%%%%%%%%%%%%%%%%%%%%%%%%%
% \section{Implementation}
%\iffalse
%<*package>
%\fi
%
% This section describes the definitions file |childdoc.def|.

% The definitions cannot be loaded using |\usepackage| or |\RequirePackage|
% which has a mechanism to prevent loading a style file more than once.
% When loading the definitions by means of |\input|
% multiple instances have to be prevented manually:
%\iffalse
%This code needs to be before the `\ProvidesFile' directive
%which is defined at the beginning of this file.
%Therefore it is also placed there and commented out here.
%</package>
%<*discard>
%\fi
%    \begin{macrocode}
\ifdefined\childdocmain\endinput\fi
%    \end{macrocode}
%\iffalse
%</discard>
%<*package>
%\fi
%
% \macro{\ifchilddoc}
% \macro{\ifchilddocmanual}
% The conditional |\ifchilddoc| tells whether a
% child (true) or main (false) document is being compiled.
% The conditional |\ifchilddocmanual| tells whether
% the |\includeonly| mechanism is used (false) or
% the selection of child files must be performed manually (true).
% The definitions initialise to false:
%    \begin{macrocode}
\newif\ifchilddoc
\newif\ifchilddocmanual
%    \end{macrocode}

% \macro{\childdocname}
% \macro{\childdocjob}
% The macro |\childdocname| stores the name of the main document
% to be compiled. The macro |\childdocjob| stores the name of
% the document on which the \LaTeX{} compiler was originally invoked.
% The content of |\jobname| cannot be compared
% to filenames specified in the source due to different catcodes.
% The following code rescans |\jobname|, stores the result
% in |\childdocname| and saves a copy in |\childdocjob|:
%    \begin{macrocode}
\edef\childdocname{\scantokens\expandafter{\jobname\noexpand}}
\let\childdocjob\childdocname
%    \end{macrocode}

% \macro{\childdocdisable}
% The macro |\childdocdisable| prevents the main file
% from being processed more than once.
% At this stage, the main document command |\childdocmain|
% is assumed to be called once again where it should do nothing.
% Any subsequent call to it should prevent
% a secondary processing of the main document
% It overwrites the forwarding commands
% |\childdocof| and |\childdocforward|
% with empty macros to prevent further inclusions of the main document:
%    \begin{macrocode}
\newcommand{\childdocdisable}
{
  \renewcommand{\childdocmain}[1]{\renewcommand{\childdocmain}[1]{\endinput}}
  \renewcommand{\childdocof}[1]{}
  \renewcommand{\childdocby}[2][]{}
  \renewcommand{\childdocforward}[2][]{}
  \renewcommand{\childdocdisable}{}
}
%    \end{macrocode}

% \macro{\childdocmain}
% The macro |\childdocmain| is to be called at the top of the main file
% with nothing or the main filename (without extension) as argument.
% First, it breaks loops.
% If the argument is not empty and does not match |\childdocname|
% (which is set by the first inclusion of |childdoc.def|),
% |\ifchilddoc| is set to true, |\includeonly| is applied to the child file
% and |\jobname| is set to the main file
% (for proper handling of |.aux| files):
%    \begin{macrocode}
\newcommand{\childdocmain}[1]
{
  \childdocdisable\childdocmain{}
  \if?#1?\else
    \begingroup
      \def\childdoctmp{#1}
      \ifx\childdoctmp\childdocname
        \def\childdoctmp{}
      \else
        \def\childdoctmp
        {
          \childdoctrue
          \includeonly{\childdocname}
          \def\childdocjob{#1}
          \def\jobname{#1}
        }
      \fi
      \expandafter
    \endgroup
    \childdoctmp
  \fi
}
%    \end{macrocode}

% \macro{\childdocof}
% The command |\childdocof| redirects
% compilation to the main file |#1|.
%    \begin{macrocode}
\newcommand{\childdocof}[1]
{
  \childdocdisable
  \childdoctrue
  \includeonly{\childdocname}
  \def\jobname{#1}
  \def\childdocjob{#1}
  \input{#1}
}
%    \end{macrocode}

% \macro{\childdocby}
% The command |\childdocby| ....
%    \begin{macrocode}
\newcommand{\childdocby}[2][]
{
  \childdocdisable
  \childdoctrue
  \childdocmanualtrue
  \if?#1?\else
    \def\jobname{#2}
  \fi
  \def\childdocjob{#2}
  \input{#2}
  \endinput
}
%    \end{macrocode}

% \macro{\childdocforward}
% The command |\childdocforward| redirects
% compilation to the main file or
% (if the optional argument is given) a child file.
% Parameters are set as if the main file
% or a child file starting with |\childdocof| was compiled.
% Then compilation is handed over to the main file:
%    \begin{macrocode}
\newcommand{\childdocforward}[2][]
{
  \begingroup
    \if?#1?
      \def\childdoctmp
      {
        \def\childdocname{#2}
        \def\childdocjob{#2}
        \def\jobname{#2}
        \input{#2}
        \endinput
      }
    \else
      \def\childdoctmp
      {
        \childdocdisable
        \def\childdocname{#2}
        \childdoctrue
        \includeonly{#2}
        \def\childdocjob{#1}
        \def\jobname{#1}
        \input{#1}
        \endinput
      }
    \fi
    \expandafter
  \endgroup
  \childdoctmp
}
%    \end{macrocode}

% \macro{\childdocforwardprefix}
% The command |\childdocforwardprefix| redirects
% compilation to the main or a child file by means of a pattern.
% The prefix |#1| in the current filename is replaced by |#2|
% and the suffix of the current filename is kept
% (it is assumed that the filename does not contain the substring `|~~~|'
% which is used as a delimiter).
% Compilation is handed over to the new file by |\childdocforward|:
%    \begin{macrocode}
\newcommand{\childdocforwardprefix}[3][]
{
  \begingroup
    \def\childdocextract #2##1~~~{\def\childdoctmp{\childdocforward[#1]{#3##1}}}
    \expandafter\childdocextract\childdocname~~~
    \expandafter
  \endgroup
  \childdoctmp
}
%    \end{macrocode}

% \macro{\childdoc}
% The deprecated macro |\childdoc| is a legacy version of |\childdocmain|:
%    \begin{macrocode}
\newcommand{\childdoc}{\childdocmain}
%    \end{macrocode}

% \macro{\childdocredirect}
% The deprecated macro |\childdocredirect| is a legacy version
% of |\childdocforward| and |\childdocforwardprefix|:
%    \begin{macrocode}
\newcommand{\childdocredirect}[2][]
{
  \begingroup
    \if?#1?
      \def\childdoctmp{\childdocforward{#2}}
    \else
      \def\childdoctmp{\childdocforwardprefix{#1}{#2}}
    \fi
    \expandafter
  \endgroup
  \childdoctmp
}
%    \end{macrocode}

%\iffalse
%</package>
%\fi
%
\endinput

\childdocforwardprefix[cdocsamp]{cdocsfn}{cdocsch}
%    \end{macrocode}

%\iffalse
%</samplefinal>
%\fi
%
% %%%%%%%%%%%%%%%%%%%%%%%%%%%%%%%%%%%%%%
% \paragraph{Command Line Processing.}
%
% The following three command lines generate the output files
% |cdocscld|, |cdocscl1| and |cdocscl2|
% which should be identical to
% |cdocsdrf|, |cdocsch1| and |cdocsfn2|, respectively:
% \begin{center}
% \begin{tabular}{l}
% |latex -jobname cdocscld \|\\
% |  "\def\version{draft}% \iffalse
%
% childdoc.dtx Copyright (C) 2017-2018 Niklas Beisert
%
% This work may be distributed and/or modified under the
% conditions of the LaTeX Project Public License, either version 1.3
% of this license or (at your option) any later version.
% The latest version of this license is in
%   http://www.latex-project.org/lppl.txt
% and version 1.3 or later is part of all distributions of LaTeX
% version 2005/12/01 or later.
%
% This work has the LPPL maintenance status `maintained'.
%
% The Current Maintainer of this work is Niklas Beisert.
%
% This work consists of the files childdoc.dtx and childdoc.ins
% and the derived files childdoc.def and cdocsamp.tex with
% cdocsch1.tex, cdocsch2.tex, cdocsdrf.tex, cdocsfn1.tex, cdocsfn2.tex.
%
%<package>\ifdefined\childdocmain\endinput\fi
%<package>\ProvidesFile{childdoc.def}[2018/12/30 v2.0 child document driver]
%<samplemain>\ProvidesFile{cdocsamp.tex}[2018/12/30 v2.0 sample for childdoc]
%<*driver>
%\ProvidesFile{childdoc.drv}[2018/12/30 v2.0 childdoc reference manual file]
\PassOptionsToClass{10pt,a4paper}{article}
\documentclass{ltxdoc}

\usepackage[margin=35mm]{geometry}
\usepackage{hyperref}
\usepackage{hyperxmp}
\usepackage[usenames]{color}

\hypersetup{colorlinks=true}
\hypersetup{pdfstartview=FitH}
\hypersetup{pdfpagemode=UseNone}
\hypersetup{pdfsource={}}
\hypersetup{pdflang={en-UK}}
\hypersetup{pdfcopyright={Copyright 2017-2018 Niklas Beisert.
  This work may be distributed and/or modified under the
  conditions of the LaTeX Project Public License, either version 1.3
  of this license or (at your option) any later version.}}
\hypersetup{pdflicenseurl={http://www.latex-project.org/lppl.txt}}
\hypersetup{pdfcontactaddress={ETH Zurich, ITP, HIT K,
  Wolfgang-Pauli-Strasse 27}}
\hypersetup{pdfcontactpostcode={8093}}
\hypersetup{pdfcontactcity={Zurich}}
\hypersetup{pdfcontactcountry={Switzerland}}
\hypersetup{pdfcontactemail={nbeisert@itp.phys.ethz.ch}}
\hypersetup{pdfcontacturl={http://people.phys.ethz.ch/\xmptilde nbeisert/}}

\newcommand{\secref}[1]{\hyperref[#1]{section \ref*{#1}}}

\parskip1ex
\parindent0pt
\let\olditemize\itemize
\def\itemize{\olditemize\parskip0pt}

\begin{document}

\title{The \textsf{childdoc} Package}
\hypersetup{pdftitle={The childdoc Package}}
\author{Niklas Beisert\\[2ex]
  Institut f\"ur Theoretische Physik\\
  Eidgen\"ossische Technische Hochschule Z\"urich\\
  Wolfgang-Pauli-Strasse 27, 8093 Z\"urich, Switzerland\\[1ex]
  \href{mailto:nbeisert@itp.phys.ethz.ch}
  {\texttt{nbeisert@itp.phys.ethz.ch}}}
\hypersetup{pdfauthor={Niklas Beisert}}
\hypersetup{pdfsubject={Manual for the LaTeX2e Package childdoc}}
\date{30 December 2018, \textsf{v2.0}}
\maketitle

\begin{abstract}\noindent
\textsf{childdoc} is a \LaTeXe{} package
that enables the direct compilation
of document sections included by |\include|
to individual files.
\end{abstract}

\begingroup
\parskip0ex
\tableofcontents
\endgroup

%%%%%%%%%%%%%%%%%%%%%%%%%%%%%%%%%%%%%%%%%%%%%%%%%%%%%%%%%%%%%%%%%%%%%%%%%%%%%%%%
%%%%%%%%%%%%%%%%%%%%%%%%%%%%%%%%%%%%%%%%%%%%%%%%%%%%%%%%%%%%%%%%%%%%%%%%%%%%%%%%
\section{Introduction}

\LaTeX{} provides a mechanism to structure a large document (such as a book)
into a main file and several child files (containing the chapters)
using the |\include| command.
This mechanism is beneficial for documents
which span hundreds of pages in order to
make the source file(s) more manageable.
Moreover, compilation can be restricted to
selected child files by means of the |\includeonly| command.
The latter feature can be used to reduce the compilation time while editing
(this was significantly more useful in the earlier days of \LaTeX{})
or to generate a smaller document which is easier to navigate.
Another application of |\includeonly| is to generate
documents consisting of selected parts of the complete document.

However, there are a few drawbacks of the plain |\include| mechanism:
\begin{itemize}
\item
The child files cannot be compiled on their own,
they can only be compiled via the main file.
A naive editing environment
(such as a text editor with an option
to have the current file processed by \LaTeX)
may require one to switch to the main file before compiling;
attempting to compile the child file produces errors.
\item
The main file must be modified (each time)
to adjust the |\includeonly| command
to the present needs. This easily leaves the main file in a messy state.
\item
The generated document will always carry the filename
of the main document. This is inconvenient if
several child files are to be compiled and
to be kept for distribution.
\end{itemize}

The present package provides a simple interface
to make child files individually compilable by \LaTeX{}.
Compiling a child file then has the same effect as compiling
the main file with an |\includeonly| command
to select the appropriate child.
Moreover the generated document will carry the name of the child
rather than the main file.
This resolves all three above issues.

This feature is meant to make the editing of books,
thesis documents and lecture notes somewhat more convenient.
However, the package can also be used efficiently for
composing a series of documents (such as exercise sheets)
which are typically distributed individually.
It then assists the author in generating the individual documents
(potentially in different versions)
as well as a document containing the collected series.
Another application is in developing style files
or other kinds of included material
where compilation of the style file could redirect
to a sample or test file.

%%%%%%%%%%%%%%%%%%%%%%%%%%%%%%%%%%%%%%%%%%%%%%%%%%%%%%%%%%%%%%%%%%%%%%%%%%%%%%%%
%%%%%%%%%%%%%%%%%%%%%%%%%%%%%%%%%%%%%%%%%%%%%%%%%%%%%%%%%%%%%%%%%%%%%%%%%%%%%%%%
\section{Usage}

First of all, the package \textsf{childdoc} is \emph{not} a standard
\LaTeXe{} |.sty| style file! Therefore it needs to be invoked in
a non-standard way.

%%%%%%%%%%%%%%%%%%%%%%%%%%%%%%%%%%%%%%%%%%%%%%%%%%%%%%%%%%%%%%%%%%%%%%%%%%%%%%%%
\subsection{Included Files}
\label{sec:include}

%%%%%%%%%%%%%%%%%%%%%%%%%%%%%%%%%%%%%%%%
\DescribeMacro{\childdocmain}
To use the package, add the commands
\begin{center}
\begin{tabular}{l}
|\input{childdoc.def}|\\
|\childdocmain{}|\\
\end{tabular}
\end{center}
at the very top of the main \LaTeX{} file,
in particular \emph{before} the |\documentclass| statement!
The argument of |\childdocmain| should be left empty
(but it must be present).

%%%%%%%%%%%%%%%%%%%%%%%%%%%%%%%%%%%%%%%%
\DescribeMacro{\childdocof}
Furthermore, add the commands
\begin{center}
\begin{tabular}{l}
|\input{childdoc.def}|\\
|\childdocof{|\textit{main}|}|\\
\end{tabular}
\end{center}
at the top of every child file \textit{child}
which is included by |\include{|\textit{child}|}|
from within the main file
(or at least for those files to be compiled individually).
The argument \textit{main} must be the filename of the main file.

There are a couple of
considerations in setting up the main and child documents:

%%%%%%%%%%%%%%%%%%%%%%%%%%%%%%%%%%%%%%%%
\paragraph{Restrictions.}

Please note the following restrictions:
\begin{itemize}
\item
|\childdocmain| must be called with one argument \textit{main}
to ensure compatibility with earlier version of the package.
It must either be empty (|\childdocmain{}|)
or precisely match the filename of the main file in which it is specified.
See \secref{sec:detection} for further information.
\item
The filename \textit{main} must be specified without the |.tex| extension.
\item
The filename \textit{main} is case sensitive
(even in case-insensitive file systems)
due to internal string comparison.
\item
The argument \textit{main} should be fully expanded, it cannot be a macro.
\item
Subdirectories and special characters should be avoided in filenames.
\item
The command |\childdocmain{|\textit{main}|}| must be followed by a whitespace.
It should not be followed immediately by another command
or by a comment mark `|%|'.
This is because the \TeX{} parser reads the token immediately following
the argument of |\childdocmain| and puts it
at the beginning of every child section;
however, a white\-space is ignored.
\end{itemize}

%%%%%%%%%%%%%%%%%%%%%%%%%%%%%%%%%%%%%%%%
\paragraph{Content of Main File.}

It is advisable to place all content in the child files included by |\include|.
Any output contained in the main file will appear in all child documents
unless suppressed manually;
it cannot be suppressed automatically by the |\includeonly| directive
and thus should normally be avoided.
A method to include some content in the main file
by means of conditional processing is described in \secref{sec:conditional}.

%%%%%%%%%%%%%%%%%%%%%%%%%%%%%%%%%%%%%%%%
\paragraph{Page Numbering.}

When only a part of the document is compiled,
the appropriate numbering of pages
(as well as other status parameters)
is determined from the |.aux| files.
The latter contain information from previous passes.
However this information needs to propagate through
all intermediate child documents.
Therefore the page numbering in child documents may well
be inconsistent until the complete document is compiled at least once.

A useful (if unconventional) way to always ensure a consistent
page numbering is to restart the numbering in each child document
and denote the pages by `\textit{child}|.|\textit{page}'
where \textit{child} represents the chapter/section number of the child file.
This can be achieved by the command
|\numberwithin{page}{|\textit{child}|}|
of the \textsf{amsmath} package
where \textit{child} can be |chapter| or |section|
depending on the chosen structuring.
Alternatively, one can modify the macro |\thepage| appropriately
and reset the counter |page| at the start of each child file.

%%%%%%%%%%%%%%%%%%%%%%%%%%%%%%%%%%%%%%%%%%%%%%%%%%%%%%%%%%%%%%%%%%%%%%%%%%%%%%%%
\subsection{Conditional Processing}
\label{sec:conditional}

The package provides a mechanism to compile different versions
of a document. To customise the versions further some conditional processing
can come in handy to distinguish which version is being compiled.
The package provides two macros to describe the compilation context:

%%%%%%%%%%%%%%%%%%%%%%%%%%%%%%%%%%%%%%%%
\DescribeMacro{\ifchilddoc}
The conditional |\ifchilddoc| distinguishes between the compilation of
child documents and the main document:
%
\begin{center}
|\ifchilddoc |\textit{child-code}| |[|\||else |\textit{main-code}]| \||fi|
\end{center}

%%%%%%%%%%%%%%%%%%%%%%%%%%%%%%%%%%%%%%%%
\DescribeMacro{\childdocname}
\DescribeMacro{\childdocjob}
The macro |\childdocname| contains the filename (without extension)
of the main or child file being processed.
Note that |\childdocjob| will always contain the name of the main file.

%%%%%%%%%%%%%%%%%%%%%%%%%%%%%%%%%%%%%%%%
\paragraph{Title Page.}

Conditional processing can be used to include a title or banner page
in the main document when proper precautions are taken.
Importantly, the code in the main file should ensure that the page counter
(as well as other status parameters which are stored in the |.aux| files)
takes the same value after the conditional processing.
Otherwise the page numbers may take divergent values
depending on which part is compiled.

For example, a title page could be declared by:
%
\begin{center}
\begin{tabular}{l}
|\ifchilddoc\||else|\\
|\addtocounter{page}{-1}|\\
\textit{code for title page}\\
|\newpage|\\
|\||fi|
\end{tabular}
\end{center}
%
A banner page for the child documents can be generated by:
%
\begin{center}
\begin{tabular}{l}
|\ifchilddoc|\\
|\addtocounter{page}{-1}|\\
\textit{code for banner page}\\
|\newpage|\\
|\||fi|
\end{tabular}
\end{center}
%
Here one could write a message such as:
\begin{center}
|This is the part \childdocname{} of \childdocjob{}.|
\end{center}

%%%%%%%%%%%%%%%%%%%%%%%%%%%%%%%%%%%%%%%%%%%%%%%%%%%%%%%%%%%%%%%%%%%%%%%%%%%%%%%%
\subsection{Flags}
\label{sec:flags}

The package makes it easy to generate different versions
of the main or child documents.
To this end compilation flags can be defined
and assigned different default values.
They will be particularly useful in conjunction
with the forwarding mechanism described in \secref{sec:forward}.

For example, it may be useful to have a flag |\version|
which can be set to |draft| or |final|.
The document source will contain some conditional code
depending on the value of |\version|.
Suppose further, the flag should default to |final| for the main file
and to |draft| for child files
which is a natural assignment for editing the document.
This is achieved by placing the following code
in the preamble of the main document
(below the |\childdocmain| directive):
%
\begin{center}
\begin{tabular}{l}
|\ifchilddoc|\\
|\providecommand{\version}{draft}|\\
|\||else|\\
|\providecommand{\version}{final}|\\
|\||fi|
\end{tabular}
\end{center}
%
The definition by |\providecommand| makes sure
that previous definitions are not overwritten.
Further statements |\providecommand{\version}{...}|
can thus be added before the above code to override it.

For the main file, one might add a line
(between |\childdocmain| and the above block)
%
\begin{center}
|%\ifchilddoc\||else\providecommand{\version}{draft}\||fi|
\end{center}
%
which can be uncommented to produce a draft version.
Likewise one can add a line to the very top of a child file
(above the |\childdocof{|\textit{main}|}| directive)
%
\begin{center}
|%\providecommand{\version}{final}|
\end{center}
%
which can be uncommented to produce the final version of this child document.

%%%%%%%%%%%%%%%%%%%%%%%%%%%%%%%%%%%%%%%%%%%%%%%%%%%%%%%%%%%%%%%%%%%%%%%%%%%%%%%%
\subsection{Forwarding}
\label{sec:forward}

Different versions of the main or child documents
using compilation flags as described in \secref{sec:flags}
can be (permanently) stored in different files
for convenient compilation, viewing and distribution.
To this end, the package defines a command
to pass on compilation to a different file:

%%%%%%%%%%%%%%%%%%%%%%%%%%%%%%%%%%%%%%%%
\DescribeMacro{\childdocforward}
The command |\childdocforward| redirects processing to
another source file:
%
\begin{center}
\begin{tabular}{l}
|\input{childdoc.def}|\\
|\childdocforward[|\textit{main}|]{|\textit{dest}|}|\\
\end{tabular}
\end{center}
%
The argument \textit{dest} is the destination file
(without extension).
It should be the main file or one of the child files.
Note that further \textsf{childdoc} directives
such as |\childdocof| and |\childdocforward|
in the indicated file will be processed in this form.
The optional argument \textit{main}
passes on directly to the main file \textit{main}
while pretending to compile the child \textit{dest}.
This form behaves as if \textit{dest}
issues |\childdocof{|\textit{main}|}| right away,
and no further \textsf{childdoc} directives will be processed.

%%%%%%%%%%%%%%%%%%%%%%%%%%%%%%%%%%%%%%%%
\DescribeMacro{\...prefix}
In the alternative form |\childdocforwardprefix|,
%
\begin{center}
\begin{tabular}{l}
|\input{childdoc.def}|\\
|\childdocforwardprefix[|\textit{main}|]{|\textit{prefix}|}{|\textit{dest}|}|
\end{tabular}
\end{center}
%
the destination file is determined by a pattern
depending on the current file:
To make this work, the current file must be called
`{\textit{prefix}\hspace{0.2em}\textit{suffix}}'
with \textit{prefix} matching precisely the argument.
Processing is then passed on to the file
`{\textit{dest}\hspace{0.2em}\textit{suffix}}'.
Surely, the same effect is achieved by
directly specifying the
argument `{\textit{dest}\hspace{0.2em}\textit{suffix}}'
in the first form.
However, that requires to set up a different file
for each child. With the alternative form of the command
all these files can have exactly the same content
which simplifies setting them up and maintaining them.

For example, the following file |draft.tex|
with a compilation flag |\version| as described in \secref{sec:flags}
compiles the main document as a draft:
%
\begin{center}
\begin{tabular}{l}
|\def\version{draft}|\\
|\input{childdoc.def}|\\
|\childdocforward{|\textit{main}|}|
\end{tabular}
\end{center}
%
Likewise, the following files |final|\textit{nn}|.tex|
compile the final version of the child document
|child|\textit{nn}|.tex|:
%
\begin{center}
\begin{tabular}{l}
|\def\version{final}|\\
|\input{childdoc.def}|\\
|\childdocforwardprefix{final}{child}|
\end{tabular}
\end{center}
%

Note that when several versions of a main file and/or of each child file
are to be generated, it may be convenient to set up a |Makefile| or
shell script to automatise the process.

%%%%%%%%%%%%%%%%%%%%%%%%%%%%%%%%%%%%%%%%%%%%%%%%%%%%%%%%%%%%%%%%%%%%%%%%%%%%%%%%
\subsection{Command Line Processing}
\label{sec:commandline}

The effect of redirection files can also be achieved by invoking
the \LaTeX{} compiler with a more elaborate command line.
Most conveniently this should be done as part
of a shell script or a |Makefile|.

When using \textsf{childdoc} in the main file, the following
command lines effectively perform a redirection
(note that depending on the shell being used,
backslashes may have to be doubled: `|\|' $\to$ `|\\|'):
%
\begin{center}
|... -jobname "|\textit{target}|" |\\|"|[\textit{flags}]%
|\input{childdoc.def}\childdocforward[|\textit{main}|]{|\textit{dest}|}"|
\end{center}
%
Here \textit{target} is the name of the output file,
\textit{main} is the name of the main file
and \textit{dest} is the name of the main or child file to be processed
(all filenames without extensions).
The optional argument \textit{main} can be omitted
if \textit{main} matches \textit{dest}.
Optionally, compilation \textit{flags} can be defined via |\def| commands.
This command line makes the \TeX{} engine believe
it is compiling the file \textit{target}
whose content is specified as the latter parameter.
The provided code then forwards the processing to
\textit{main} or \textit{dest} as described in \secref{sec:forward}.

%%%%%%%%%%%%%%%%%%%%%%%%%%%%%%%%%%%%%%%%%%%%%%%%%%%%%%%%%%%%%%%%%%%%%%%%%%%%%%%%
\subsection{Include by Input}
\label{sec:input}

Including child documents by |\include| has some restrictions by design.
Most notably, the content of a child document always occupies
its own set of pages; pages cannot be shared between child documents.
Usually, this behaviour makes perfect sense
because each child document contain an essential part of the document.
However, in some situations it may be desirable to compose
a document from a collection of parts
without having mandatory page breaks between then.
For this case, the package
provides a mechanism to include parts
by |\input| which can also be processed individually.
However, by construction this mechanism
requires manual handling of the content to be output.

%%%%%%%%%%%%%%%%%%%%%%%%%%%%%%%%%%%%%%%%
\DescribeMacro{\ifchilddocmanual}
The main file should be prepared as usual, see \secref{sec:include}.
However, the document body must make a distinction
between processing of an individual part and of the main document, e.g.:
%
\begin{center}
\begin{tabular}{l}
|\ifchilddocmanual|\\
|\input{\childdocname}|\\
|\||else|\\
\textit{document body with }|\input{|\textit{part}|}|\\
|\||fi|
\end{tabular}
\end{center}
%
The conditional |\ifchilddocmanual| is true whenever
a part to be included by |\input| is being compiled,
and the name of the part is stored in |\childdocname|.

%%%%%%%%%%%%%%%%%%%%%%%%%%%%%%%%%%%%%%%%
\DescribeMacro{\childdocby}
Each part to be included by |\input| should start with:
%
\begin{center}
\begin{tabular}{l}
|\input{childdoc.def}|\\
|\childdocby{|\textit{main}|}|\\
\end{tabular}
\end{center}
%
The directive |\childdocby| is similar to |\childdocof|
described in \secref{sec:include},
but the subsequent selection of content must be done manually.
To that end, both |\ifchilddoc| and |\ifchilddocmanual|
will be true upon processing of a part,
and the name of the part is stored in |\childdocname|.
Note that |\jobname| will be set to the filename of the current part
so that each part receives an individual |.aux| file
that does not interfere with the |.aux| file(s) of the main document.
This behaviour can be altered by the alternative form
|\childdocby[*]{|\textit{main}|}| (with a non-empty optional argument)
which uses the |.aux| file of the main document
by setting |\jobname| to \textit{main}.

%%%%%%%%%%%%%%%%%%%%%%%%%%%%%%%%%%%%%%%%%%%%%%%%%%%%%%%%%%%%%%%%%%%%%%%%%%%%%%%%
\subsection{Driver Development}
\label{sec:driver}

The \textsf{childdoc} mechanism can also be use for the development
of definition files such as \LaTeX{} styles or classes.
This case differs from the above setup with multiple parts
included by |\include| in that no |\includeonly| should be invoked.
This can be achieved by starting the include file
(before |\ProvidesPackage|) with:
%
\begin{center}
\begin{tabular}{l}
|\input{childdoc.def}|\\
|\childdocforward{|\textit{main}|}|\\
\end{tabular}
\end{center}
%
or alternatively with:
%
\begin{center}
\begin{tabular}{l}
|\input{childdoc.def}|\\
|\childdocby{|\textit{main}|}|\\
\end{tabular}
\end{center}
%
Both forms have slightly different effects as described above.
The main file is prepared as usual, see \secref{sec:include}.

%%%%%%%%%%%%%%%%%%%%%%%%%%%%%%%%%%%%%%%%%%%%%%%%%%%%%%%%%%%%%%%%%%%%%%%%%%%%%%%%
\subsection{Legacy Detection}
\label{sec:detection}

The directive |\childdocmain| in the main file can detect
whether the complete document or merely a child is to be compiled
even without using the directive |\childdocof|.
This method is deprecated because it is less robust
and there is no compelling reason to use it;
it is merely provided for backward compatibility
and it may be removed in future versions.

If the detection mechanism is to be used,
it is mandatory to correctly specify
the filename of the main file as the argument of |\childdocmain|:
%
\begin{center}
\begin{tabular}{l}
|\input{childdoc.def}|\\
|\childdocmain{|\textit{main}|}|\\
\end{tabular}
\end{center}
%
If |\jobname| does not match the argument \textit{main} of |\childdocmain|,
it is assumed that |\jobname| points to the child file to be compiled.
When using |\childdocmain| with the main file specified as argument,
it suffices to start a child file
with just |\input{|\textit{main}|}|
without loading of the package and using |\childdocof|.
If instead all processing is done
with the appropriate \textsf{childdoc} directives,
the argument of \textit{main} of |\childdocmain| can be empty.

An alternative version of the command line processing described
in \secref{sec:commandline} using the detection mechanism reads:
%
\begin{center}
|... -jobname "|\textit{target}|" "|[\textit{flags}]%
[|\def\jobname{|\textit{dest}|}|]|\input{|\textit{main}|}"|
\end{center}

%%%%%%%%%%%%%%%%%%%%%%%%%%%%%%%%%%%%%%%%%%%%%%%%%%%%%%%%%%%%%%%%%%%%%%%%%%%%%%%%
\subsection{Manual Code}
\label{sec:manual}

In case one cannot be certain whether the definitions file |childdoc.def|
is installed on the target \TeX{} distribution
and one prefers not to ship it,
it is conceivable to paste a few relevant commands into the sources.

To that end, drop all statements |\input{childdoc.def}|
and perform the replacements as outlined below.
Instead of |\childdocmain{|\textit{main}|}| add the following code
to the top of the main file:
%
\begin{center}
\begin{tabular}{l}
|\||ifdefined\childdocname\endinput\||fi\newif\ifchilddoc|\\
|\edef\childdocname{\scantokens\expandafter{\jobname\noexpand}}|\\
|\def\childdocmain{|\textit{main}|}\||ifx\childdocmain\childdocname\||else|\\
|\childdoctrue\includeonly{\childdocname}\let\jobname\childdocmain\||fi|\\
\end{tabular}
\end{center}
%
Instead of |\childdocof{|\textit{main}|}| just include the main file
at the top of each child file:
%
\begin{center}
|\input{|\textit{main}|}|
\end{center}
%
A simple redirection |\childdocforward{|\textit{dest}|}| is achieved by:
%
\begin{center}
|\def\jobname{|\textit{dest}|}\input{\jobname}|
\end{center}
%
The redirection with prefix
|\childdocforwardprefix[|\textit{prefix}|]{|\textit{dest}|}|
is accomplished by:
%
\begin{center}
\begin{tabular}{l}
|{\edef\jobname{\scantokens\expandafter{\jobname\noexpand}}|\\
|\def\redirectjob |\textit{prefix}|#1~~~{\gdef\jobname{|\textit{dest}|#1}}|\\
|\expandafter\redirectjob\jobname~~~}\input{\jobname}|
\end{tabular}
\end{center}

In an alternative approach,
child documents can be compiled by a specific command line
without additional code or specific definitions:
%
\begin{center}
|... -jobname "|\textit{target}|" "|[\textit{flags}]%
|\includeonly{|\textit{dest}|}\input{|\textit{main}|}"|
\end{center}
%

%%%%%%%%%%%%%%%%%%%%%%%%%%%%%%%%%%%%%%%%%%%%%%%%%%%%%%%%%%%%%%%%%%%%%%%%%%%%%%%%
%%%%%%%%%%%%%%%%%%%%%%%%%%%%%%%%%%%%%%%%%%%%%%%%%%%%%%%%%%%%%%%%%%%%%%%%%%%%%%%%
\section{Information}

%%%%%%%%%%%%%%%%%%%%%%%%%%%%%%%%%%%%%%%%%%%%%%%%%%%%%%%%%%%%%%%%%%%%%%%%%%%%%%%%
\subsection{Copyright}

Copyright \copyright{} 2017--2018 Niklas Beisert

This work may be distributed and/or modified under the
conditions of the \LaTeX{} Project Public License, either version 1.3
of this license or (at your option) any later version.
The latest version of this license is in
  \url{http://www.latex-project.org/lppl.txt}
and version 1.3 or later is part of all distributions of \LaTeX{}
version 2005/12/01 or later.

This work has the LPPL maintenance status `maintained'.

The Current Maintainer of this work is Niklas Beisert.

This work consists of the files |README.txt|, |childdoc.ins| and |childdoc.dtx|
as well as the derived files |childdoc.def|, |cdocsamp.tex|
with |cdocsch1.tex|, |cdocsch2.tex|, |cdocspt3.tex|, |cdocspt4.tex|,
|cdocsdrf.tex|, |cdocsfn1.tex|, |cdocsfn2.tex|
as well as |childdoc.pdf|.

%%%%%%%%%%%%%%%%%%%%%%%%%%%%%%%%%%%%%%%%%%%%%%%%%%%%%%%%%%%%%%%%%%%%%%%%%%%%%%%%
\subsection{Files and Installation}

The package consists of the files:
%
\begin{center}
\begin{tabular}{ll}
    |README.txt|   & readme file \\
    |childdoc.ins| & installation file \\
    |childdoc.dtx| & source file \\
    |childdoc.def| & definition file \\
    |cdocsamp.tex| & sample main file \\
    |cdocsch1.tex| & sample include file \\
    |cdocsch2.tex| & sample include file \\
    |cdocspt3.tex| & sample part file \\
    |cdocspt4.tex| & sample part file \\
    |cdocsdrf.tex| & sample redirection file \\
    |cdocsfn1.tex| & sample redirection file \\
    |cdocsfn2.tex| & sample redirection file \\
    |childdoc.pdf| & manual
\end{tabular}
\end{center}
%
The distribution consists of the files
|README.txt|, |childdoc.ins| and |childdoc.dtx|.
%
\begin{itemize}
\item
Run (pdf)\LaTeX{} on |childdoc.dtx|
to compile the manual |childdoc.pdf| (this file).
\item
Run \LaTeX{} on |childdoc.ins| to create the definitions file |childdoc.def|
and the sample |cdocsamp.tex| with include files
|cdocsch1.tex|, |cdocsch2.tex|, |cdocspt3.tex|, |cdocspt4.tex|,
|cdocsdrf.tex|, |cdocsfn1.tex|, |cdocsfn2.tex|.
Then copy the file |childdoc.def| to an appropriate directory of your \LaTeX{}
distribution, e.g.\ \textit{texmf-root}|/tex/latex/childdoc|.
\end{itemize}

%%%%%%%%%%%%%%%%%%%%%%%%%%%%%%%%%%%%%%%%%%%%%%%%%%%%%%%%%%%%%%%%%%%%%%%%%%%%%%%%
\subsection{Related CTAN Packages}

There are several other packages which offer a similar functionality:
%
\begin{itemize}
\item
The packages
\href{http://ctan.org/pkg/docmute}{\textsf{docmute}},
\href{http://ctan.org/pkg/includex}{\textsf{includex}} and
\href{http://ctan.org/pkg/standalone}{\textsf{standalone}}
provide commands to include only the document body of
a child file thus allowing both files to be compiled individually.
\item
The packages \href{http://ctan.org/pkg/subdocs}{\textsf{subdocs}}
and \href{http://ctan.org/pkg/subfiles}{\textsf{subfiles}}
provide structures in which the main and child documents can be
encapsulated and allowing them to be compiled individually.
The inclusion mechanism is different from the conventional |\include|.
\item
The package \href{http://ctan.org/pkg/combine}{\textsf{combine}}
is an elaborate solution to combine several documents into one.
\end{itemize}
%
See also the CTAN topic \href{http://ctan.org/topic/subdocs}{\textsf{subdocs}}
for further related packages.
The present package differs from the above solutions in that
a document structure constructed with the conventional |\include| mechanism
just needs two extra commands at the top of every file
such that all constituent files can be compiled individually.

%%%%%%%%%%%%%%%%%%%%%%%%%%%%%%%%%%%%%%%%%%%%%%%%%%%%%%%%%%%%%%%%%%%%%%%%%%%%%%%%
%\subsection{Feature Suggestions}
%
%The following is a list of features which may be useful for future
%versions of this package:
%%
%\begin{itemize}
%\item
%\ldots
%\end{itemize}

%%%%%%%%%%%%%%%%%%%%%%%%%%%%%%%%%%%%%%%%%%%%%%%%%%%%%%%%%%%%%%%%%%%%%%%%%%%%%%%%
\subsection{Revision History}

%%%%%%%%%%%%%%%%%%%%%%%%%%%%%%%%%%%%%%%%
\paragraph{v2.0:} 2018/12/30

\begin{itemize}
\item
immediate forward processing
\item
added |\childdocby| mechanism
\item
manual restructured
\end{itemize}

%%%%%%%%%%%%%%%%%%%%%%%%%%%%%%%%%%%%%%%%
\paragraph{v1.6:} 2018/01/17

\begin{itemize}
\item
application for development of include files
\item
corrections to manual
\end{itemize}

%%%%%%%%%%%%%%%%%%%%%%%%%%%%%%%%%%%%%%%%
\paragraph{v1.5:} 2017/05/21

\begin{itemize}
\item
more complete structuring introduced
\item
|\childdocof| introduced
\item
|\childdoc| renamed to |\childdocmain|
\item
|\childredirect| renamed to |\childdocforward| and |\childdocforwardprefix|
and functionality expanded
\end{itemize}

%%%%%%%%%%%%%%%%%%%%%%%%%%%%%%%%%%%%%%%%
\paragraph{v1.0:} 2017/04/27

\begin{itemize}
\item
manual and install package
\item
first version published on CTAN
\end{itemize}

%%%%%%%%%%%%%%%%%%%%%%%%%%%%%%%%%%%%%%%%
\paragraph{v0.6:} 2017/04/26

\begin{itemize}
\item
redirection mechanism added
\end{itemize}

%%%%%%%%%%%%%%%%%%%%%%%%%%%%%%%%%%%%%%%%
\paragraph{v0.5:} 2017/04/26

\begin{itemize}
\item
functionality in definition file
\end{itemize}


%%%%%%%%%%%%%%%%%%%%%%%%%%%%%%%%%%%%%%%%%%%%%%%%%%%%%%%%%%%%%%%%%%%%%%%%%%%%%%%%
%%%%%%%%%%%%%%%%%%%%%%%%%%%%%%%%%%%%%%%%%%%%%%%%%%%%%%%%%%%%%%%%%%%%%%%%%%%%%%%%
%%%%%%%%%%%%%%%%%%%%%%%%%%%%%%%%%%%%%%%%%%%%%%%%%%%%%%%%%%%%%%%%%%%%%%%%%%%%%%%%
\appendix

\settowidth\MacroIndent{\rmfamily\scriptsize 000\ }

 \DocInput{childdoc.dtx}

\end{document}
%</driver>
% \fi
%
% %%%%%%%%%%%%%%%%%%%%%%%%%%%%%%%%%%%%%%%%%%%%%%%%%%%%%%%%%%%%%%%%%%%%%%%%%%%%%%
% %%%%%%%%%%%%%%%%%%%%%%%%%%%%%%%%%%%%%%%%%%%%%%%%%%%%%%%%%%%%%%%%%%%%%%%%%%%%%%
% \section{Sample}
%\iffalse
%<*samplemain>
%\fi
%
% The following presents a sample document
% with two chapters, two parts, a title page,
% a compile flag as well as three forwarding files to set the flag.
% It consists of eight |.tex| files:
% \begin{center}
% \begin{tabular}{ll}
% |cdocsamp.tex|&main file\\
% |cdocsch1.tex|&include file for chapter 1\\
% |cdocsch2.tex|&include file for chapter 2\\
% |cdocspt3.tex|&include file for part 3\\
% |cdocspt4.tex|&include file for part 4\\
% |cdocsdrf.tex|&forwarding file for main file in draft mode\\
% |cdocsfi1.tex|&forwarding file for final version of chapter 1\\
% |cdocsfi2.tex|&forwarding file for final version of chapter 2\\
% \end{tabular}
% \end{center}
% Each of the eight files can be compiled directly by the \LaTeX{} compiler.
%
% %%%%%%%%%%%%%%%%%%%%%%%%%%%%%%%%%%%%%%
% \paragraph{Main File.}
%
% The main file is called |cdocsamp.tex|.
%
% Load the \textsf{childdoc} definitions and
% declare the filename for the main document:
%    \begin{macrocode}
\input{childdoc.def}
\childdocmain{}
%    \end{macrocode}

% Optional override for |\version| flag:
%    \begin{macrocode}
%%\ifchilddoc\else\providecommand{\version}{draft}\fi
%    \end{macrocode}

% Define the default values for the |\version| flag
% (|final| for the main file and |draft| for childs):
%    \begin{macrocode}
\ifchilddoc
\providecommand{\version}{draft}
\else
\providecommand{\version}{final}
\fi
%    \end{macrocode}

% Load the standard document class:
%    \begin{macrocode}
\documentclass[12pt]{article}
%    \end{macrocode}

% Start the document body:
%    \begin{macrocode}
\begin{document}
%    \end{macrocode}

% Declare a title page.
% Print title, part of document being processed and version flag:
%    \begin{macrocode}
\addtocounter{page}{-1}
\begin{center}
{\LARGE\bfseries{}childdoc example\par}
\vspace{1cm}
\ifchilddoc
\ifchilddocmanual part\else chapter\fi:
`\childdocname' of `\childdocjob'\par
\else
main document: `\childdocjob'\par
\fi
version: \version\par
\end{center}
\newpage
%    \end{macrocode}

% Manually include selected file,
% otherwise process as usual:
%    \begin{macrocode}
\ifchilddocmanual
\section*{part `\childdocname'}
\input{\childdocname}
\else
%    \end{macrocode}

% Include the two chapters:
%    \begin{macrocode}
\include{cdocsch1}
\include{cdocsch2}
%    \end{macrocode}

% Include the two parts unless only chapters should be displayed:
%    \begin{macrocode}
\ifchilddoc\else
\section{part three}
\input{cdocspt3}
\section{part four}
\input{cdocspt4}
\fi
%    \end{macrocode}

% Process as usual until here:
%    \begin{macrocode}
\fi
%    \end{macrocode}

% End of document body:
%    \begin{macrocode}
\end{document}
%    \end{macrocode}
%\iffalse
%</samplemain>
%\fi
%
% %%%%%%%%%%%%%%%%%%%%%%%%%%%%%%%%%%%%%%
% \paragraph{Chapter Include Files.}
%
% The include files are called |cdocsch1.tex| and |cdocsch2.tex|.
%
%\iffalse
%<*samplechap1|samplechap2>
%\fi

% Optional override for |\version| flag:
%    \begin{macrocode}
%%\providecommand{\version}{final}
%    \end{macrocode}

% Include the main document:
%    \begin{macrocode}
\input{childdoc.def}
\childdocof{cdocsamp}
%    \end{macrocode}

%\iffalse
%</samplechap1|samplechap2>
%\fi
%
%\iffalse
%<*samplechap1>
%\fi
% Some text for chapter 1:
%    \begin{macrocode}
\section{one}
some text in chapter one
%    \end{macrocode}

%\iffalse
%</samplechap1>
%\fi
% Some text for chapter 2:
%\iffalse
%<*samplechap2>
%\fi
%    \begin{macrocode}
\section{two}
more text in chapter two
%    \end{macrocode}

%\iffalse
%</samplechap2>
%\fi
%
% %%%%%%%%%%%%%%%%%%%%%%%%%%%%%%%%%%%%%%
% \paragraph{Part Include Files.}
%
% The include files are called |cdocspt3.tex| and |cdocspt4.tex|.
%
%\iffalse
%<*samplepart3|samplepart4>
%\fi

% Optional override for |\version| flag:
%    \begin{macrocode}
%%\providecommand{\version}{final}
%    \end{macrocode}

% Include the main document:
%    \begin{macrocode}
\input{childdoc.def}
\childdocby{cdocsamp}
%    \end{macrocode}

%\iffalse
%</samplepart3|samplepart4>
%\fi
%
%\iffalse
%<*samplepart3>
%\fi
% Some text for part 3:
%    \begin{macrocode}
some text in part three
%    \end{macrocode}

%\iffalse
%</samplepart3>
%\fi
% Some text for part 4:
%\iffalse
%<*samplepart4>
%\fi
%    \begin{macrocode}
more text in part four
%    \end{macrocode}

%\iffalse
%</samplepart4>
%\fi
%
% %%%%%%%%%%%%%%%%%%%%%%%%%%%%%%%%%%%%%%
% \paragraph{Forwarding for a Complete Draft.}
%
% The following forwarding file |cdocsdrf.tex|
% compiles the main document in draft mode:
%\iffalse
%<*sampledraft>
%\fi
%    \begin{macrocode}
\def\version{draft}
\input{childdoc.def}
\childdocforward{cdocsamp}
%    \end{macrocode}

%\iffalse
%</sampledraft>
%\fi
%
% %%%%%%%%%%%%%%%%%%%%%%%%%%%%%%%%%%%%%%
% \paragraph{Forwarding for Final Version of the Chapters.}
%
% The following forwarding files |cdocsfn1.tex| and |cdocsfn2.tex|
% (with identical content)
% compile the final versions of the child documents
% |cdocsch1.tex| and |cdocsch2.tex|, respectively:
%\iffalse
%<*samplefinal>
%\fi
%    \begin{macrocode}
\def\version{final}
\input{childdoc.def}
\childdocforwardprefix[cdocsamp]{cdocsfn}{cdocsch}
%    \end{macrocode}

%\iffalse
%</samplefinal>
%\fi
%
% %%%%%%%%%%%%%%%%%%%%%%%%%%%%%%%%%%%%%%
% \paragraph{Command Line Processing.}
%
% The following three command lines generate the output files
% |cdocscld|, |cdocscl1| and |cdocscl2|
% which should be identical to
% |cdocsdrf|, |cdocsch1| and |cdocsfn2|, respectively:
% \begin{center}
% \begin{tabular}{l}
% |latex -jobname cdocscld \|\\
% |  "\def\version{draft}\input{childdoc.def}\childdocforward{cdocsamp}"|\\
% |latex -jobname cdocscl1 \|\\
% |  "\input{childdoc.def}\childdocforward[cdocsamp]{cdocsch1}"|\\
% |latex -jobname cdocscl2 \|\\
% |  "\def\version{final}\input{childdoc.def}\childdocforward{cdocsch2}"|
% \end{tabular}
% \end{center}
% Note that the trailing backslash on each first line
% merely continues the input to the second line
% (for convenient cut ant paste).
% Furthermore, the command |latex| can be replaced by any
% of its alternative versions such as |pdflatex|.
%
% %%%%%%%%%%%%%%%%%%%%%%%%%%%%%%%%%%%%%%%%%%%%%%%%%%%%%%%%%%%%%%%%%%%%%%%%%%%%%%
% %%%%%%%%%%%%%%%%%%%%%%%%%%%%%%%%%%%%%%%%%%%%%%%%%%%%%%%%%%%%%%%%%%%%%%%%%%%%%%
% \section{Implementation}
%\iffalse
%<*package>
%\fi
%
% This section describes the definitions file |childdoc.def|.

% The definitions cannot be loaded using |\usepackage| or |\RequirePackage|
% which has a mechanism to prevent loading a style file more than once.
% When loading the definitions by means of |\input|
% multiple instances have to be prevented manually:
%\iffalse
%This code needs to be before the `\ProvidesFile' directive
%which is defined at the beginning of this file.
%Therefore it is also placed there and commented out here.
%</package>
%<*discard>
%\fi
%    \begin{macrocode}
\ifdefined\childdocmain\endinput\fi
%    \end{macrocode}
%\iffalse
%</discard>
%<*package>
%\fi
%
% \macro{\ifchilddoc}
% \macro{\ifchilddocmanual}
% The conditional |\ifchilddoc| tells whether a
% child (true) or main (false) document is being compiled.
% The conditional |\ifchilddocmanual| tells whether
% the |\includeonly| mechanism is used (false) or
% the selection of child files must be performed manually (true).
% The definitions initialise to false:
%    \begin{macrocode}
\newif\ifchilddoc
\newif\ifchilddocmanual
%    \end{macrocode}

% \macro{\childdocname}
% \macro{\childdocjob}
% The macro |\childdocname| stores the name of the main document
% to be compiled. The macro |\childdocjob| stores the name of
% the document on which the \LaTeX{} compiler was originally invoked.
% The content of |\jobname| cannot be compared
% to filenames specified in the source due to different catcodes.
% The following code rescans |\jobname|, stores the result
% in |\childdocname| and saves a copy in |\childdocjob|:
%    \begin{macrocode}
\edef\childdocname{\scantokens\expandafter{\jobname\noexpand}}
\let\childdocjob\childdocname
%    \end{macrocode}

% \macro{\childdocdisable}
% The macro |\childdocdisable| prevents the main file
% from being processed more than once.
% At this stage, the main document command |\childdocmain|
% is assumed to be called once again where it should do nothing.
% Any subsequent call to it should prevent
% a secondary processing of the main document
% It overwrites the forwarding commands
% |\childdocof| and |\childdocforward|
% with empty macros to prevent further inclusions of the main document:
%    \begin{macrocode}
\newcommand{\childdocdisable}
{
  \renewcommand{\childdocmain}[1]{\renewcommand{\childdocmain}[1]{\endinput}}
  \renewcommand{\childdocof}[1]{}
  \renewcommand{\childdocby}[2][]{}
  \renewcommand{\childdocforward}[2][]{}
  \renewcommand{\childdocdisable}{}
}
%    \end{macrocode}

% \macro{\childdocmain}
% The macro |\childdocmain| is to be called at the top of the main file
% with nothing or the main filename (without extension) as argument.
% First, it breaks loops.
% If the argument is not empty and does not match |\childdocname|
% (which is set by the first inclusion of |childdoc.def|),
% |\ifchilddoc| is set to true, |\includeonly| is applied to the child file
% and |\jobname| is set to the main file
% (for proper handling of |.aux| files):
%    \begin{macrocode}
\newcommand{\childdocmain}[1]
{
  \childdocdisable\childdocmain{}
  \if?#1?\else
    \begingroup
      \def\childdoctmp{#1}
      \ifx\childdoctmp\childdocname
        \def\childdoctmp{}
      \else
        \def\childdoctmp
        {
          \childdoctrue
          \includeonly{\childdocname}
          \def\childdocjob{#1}
          \def\jobname{#1}
        }
      \fi
      \expandafter
    \endgroup
    \childdoctmp
  \fi
}
%    \end{macrocode}

% \macro{\childdocof}
% The command |\childdocof| redirects
% compilation to the main file |#1|.
%    \begin{macrocode}
\newcommand{\childdocof}[1]
{
  \childdocdisable
  \childdoctrue
  \includeonly{\childdocname}
  \def\jobname{#1}
  \def\childdocjob{#1}
  \input{#1}
}
%    \end{macrocode}

% \macro{\childdocby}
% The command |\childdocby| ....
%    \begin{macrocode}
\newcommand{\childdocby}[2][]
{
  \childdocdisable
  \childdoctrue
  \childdocmanualtrue
  \if?#1?\else
    \def\jobname{#2}
  \fi
  \def\childdocjob{#2}
  \input{#2}
  \endinput
}
%    \end{macrocode}

% \macro{\childdocforward}
% The command |\childdocforward| redirects
% compilation to the main file or
% (if the optional argument is given) a child file.
% Parameters are set as if the main file
% or a child file starting with |\childdocof| was compiled.
% Then compilation is handed over to the main file:
%    \begin{macrocode}
\newcommand{\childdocforward}[2][]
{
  \begingroup
    \if?#1?
      \def\childdoctmp
      {
        \def\childdocname{#2}
        \def\childdocjob{#2}
        \def\jobname{#2}
        \input{#2}
        \endinput
      }
    \else
      \def\childdoctmp
      {
        \childdocdisable
        \def\childdocname{#2}
        \childdoctrue
        \includeonly{#2}
        \def\childdocjob{#1}
        \def\jobname{#1}
        \input{#1}
        \endinput
      }
    \fi
    \expandafter
  \endgroup
  \childdoctmp
}
%    \end{macrocode}

% \macro{\childdocforwardprefix}
% The command |\childdocforwardprefix| redirects
% compilation to the main or a child file by means of a pattern.
% The prefix |#1| in the current filename is replaced by |#2|
% and the suffix of the current filename is kept
% (it is assumed that the filename does not contain the substring `|~~~|'
% which is used as a delimiter).
% Compilation is handed over to the new file by |\childdocforward|:
%    \begin{macrocode}
\newcommand{\childdocforwardprefix}[3][]
{
  \begingroup
    \def\childdocextract #2##1~~~{\def\childdoctmp{\childdocforward[#1]{#3##1}}}
    \expandafter\childdocextract\childdocname~~~
    \expandafter
  \endgroup
  \childdoctmp
}
%    \end{macrocode}

% \macro{\childdoc}
% The deprecated macro |\childdoc| is a legacy version of |\childdocmain|:
%    \begin{macrocode}
\newcommand{\childdoc}{\childdocmain}
%    \end{macrocode}

% \macro{\childdocredirect}
% The deprecated macro |\childdocredirect| is a legacy version
% of |\childdocforward| and |\childdocforwardprefix|:
%    \begin{macrocode}
\newcommand{\childdocredirect}[2][]
{
  \begingroup
    \if?#1?
      \def\childdoctmp{\childdocforward{#2}}
    \else
      \def\childdoctmp{\childdocforwardprefix{#1}{#2}}
    \fi
    \expandafter
  \endgroup
  \childdoctmp
}
%    \end{macrocode}

%\iffalse
%</package>
%\fi
%
\endinput
\childdocforward{cdocsamp}"|\\
% |latex -jobname cdocscl1 \|\\
% |  "% \iffalse
%
% childdoc.dtx Copyright (C) 2017-2018 Niklas Beisert
%
% This work may be distributed and/or modified under the
% conditions of the LaTeX Project Public License, either version 1.3
% of this license or (at your option) any later version.
% The latest version of this license is in
%   http://www.latex-project.org/lppl.txt
% and version 1.3 or later is part of all distributions of LaTeX
% version 2005/12/01 or later.
%
% This work has the LPPL maintenance status `maintained'.
%
% The Current Maintainer of this work is Niklas Beisert.
%
% This work consists of the files childdoc.dtx and childdoc.ins
% and the derived files childdoc.def and cdocsamp.tex with
% cdocsch1.tex, cdocsch2.tex, cdocsdrf.tex, cdocsfn1.tex, cdocsfn2.tex.
%
%<package>\ifdefined\childdocmain\endinput\fi
%<package>\ProvidesFile{childdoc.def}[2018/12/30 v2.0 child document driver]
%<samplemain>\ProvidesFile{cdocsamp.tex}[2018/12/30 v2.0 sample for childdoc]
%<*driver>
%\ProvidesFile{childdoc.drv}[2018/12/30 v2.0 childdoc reference manual file]
\PassOptionsToClass{10pt,a4paper}{article}
\documentclass{ltxdoc}

\usepackage[margin=35mm]{geometry}
\usepackage{hyperref}
\usepackage{hyperxmp}
\usepackage[usenames]{color}

\hypersetup{colorlinks=true}
\hypersetup{pdfstartview=FitH}
\hypersetup{pdfpagemode=UseNone}
\hypersetup{pdfsource={}}
\hypersetup{pdflang={en-UK}}
\hypersetup{pdfcopyright={Copyright 2017-2018 Niklas Beisert.
  This work may be distributed and/or modified under the
  conditions of the LaTeX Project Public License, either version 1.3
  of this license or (at your option) any later version.}}
\hypersetup{pdflicenseurl={http://www.latex-project.org/lppl.txt}}
\hypersetup{pdfcontactaddress={ETH Zurich, ITP, HIT K,
  Wolfgang-Pauli-Strasse 27}}
\hypersetup{pdfcontactpostcode={8093}}
\hypersetup{pdfcontactcity={Zurich}}
\hypersetup{pdfcontactcountry={Switzerland}}
\hypersetup{pdfcontactemail={nbeisert@itp.phys.ethz.ch}}
\hypersetup{pdfcontacturl={http://people.phys.ethz.ch/\xmptilde nbeisert/}}

\newcommand{\secref}[1]{\hyperref[#1]{section \ref*{#1}}}

\parskip1ex
\parindent0pt
\let\olditemize\itemize
\def\itemize{\olditemize\parskip0pt}

\begin{document}

\title{The \textsf{childdoc} Package}
\hypersetup{pdftitle={The childdoc Package}}
\author{Niklas Beisert\\[2ex]
  Institut f\"ur Theoretische Physik\\
  Eidgen\"ossische Technische Hochschule Z\"urich\\
  Wolfgang-Pauli-Strasse 27, 8093 Z\"urich, Switzerland\\[1ex]
  \href{mailto:nbeisert@itp.phys.ethz.ch}
  {\texttt{nbeisert@itp.phys.ethz.ch}}}
\hypersetup{pdfauthor={Niklas Beisert}}
\hypersetup{pdfsubject={Manual for the LaTeX2e Package childdoc}}
\date{30 December 2018, \textsf{v2.0}}
\maketitle

\begin{abstract}\noindent
\textsf{childdoc} is a \LaTeXe{} package
that enables the direct compilation
of document sections included by |\include|
to individual files.
\end{abstract}

\begingroup
\parskip0ex
\tableofcontents
\endgroup

%%%%%%%%%%%%%%%%%%%%%%%%%%%%%%%%%%%%%%%%%%%%%%%%%%%%%%%%%%%%%%%%%%%%%%%%%%%%%%%%
%%%%%%%%%%%%%%%%%%%%%%%%%%%%%%%%%%%%%%%%%%%%%%%%%%%%%%%%%%%%%%%%%%%%%%%%%%%%%%%%
\section{Introduction}

\LaTeX{} provides a mechanism to structure a large document (such as a book)
into a main file and several child files (containing the chapters)
using the |\include| command.
This mechanism is beneficial for documents
which span hundreds of pages in order to
make the source file(s) more manageable.
Moreover, compilation can be restricted to
selected child files by means of the |\includeonly| command.
The latter feature can be used to reduce the compilation time while editing
(this was significantly more useful in the earlier days of \LaTeX{})
or to generate a smaller document which is easier to navigate.
Another application of |\includeonly| is to generate
documents consisting of selected parts of the complete document.

However, there are a few drawbacks of the plain |\include| mechanism:
\begin{itemize}
\item
The child files cannot be compiled on their own,
they can only be compiled via the main file.
A naive editing environment
(such as a text editor with an option
to have the current file processed by \LaTeX)
may require one to switch to the main file before compiling;
attempting to compile the child file produces errors.
\item
The main file must be modified (each time)
to adjust the |\includeonly| command
to the present needs. This easily leaves the main file in a messy state.
\item
The generated document will always carry the filename
of the main document. This is inconvenient if
several child files are to be compiled and
to be kept for distribution.
\end{itemize}

The present package provides a simple interface
to make child files individually compilable by \LaTeX{}.
Compiling a child file then has the same effect as compiling
the main file with an |\includeonly| command
to select the appropriate child.
Moreover the generated document will carry the name of the child
rather than the main file.
This resolves all three above issues.

This feature is meant to make the editing of books,
thesis documents and lecture notes somewhat more convenient.
However, the package can also be used efficiently for
composing a series of documents (such as exercise sheets)
which are typically distributed individually.
It then assists the author in generating the individual documents
(potentially in different versions)
as well as a document containing the collected series.
Another application is in developing style files
or other kinds of included material
where compilation of the style file could redirect
to a sample or test file.

%%%%%%%%%%%%%%%%%%%%%%%%%%%%%%%%%%%%%%%%%%%%%%%%%%%%%%%%%%%%%%%%%%%%%%%%%%%%%%%%
%%%%%%%%%%%%%%%%%%%%%%%%%%%%%%%%%%%%%%%%%%%%%%%%%%%%%%%%%%%%%%%%%%%%%%%%%%%%%%%%
\section{Usage}

First of all, the package \textsf{childdoc} is \emph{not} a standard
\LaTeXe{} |.sty| style file! Therefore it needs to be invoked in
a non-standard way.

%%%%%%%%%%%%%%%%%%%%%%%%%%%%%%%%%%%%%%%%%%%%%%%%%%%%%%%%%%%%%%%%%%%%%%%%%%%%%%%%
\subsection{Included Files}
\label{sec:include}

%%%%%%%%%%%%%%%%%%%%%%%%%%%%%%%%%%%%%%%%
\DescribeMacro{\childdocmain}
To use the package, add the commands
\begin{center}
\begin{tabular}{l}
|\input{childdoc.def}|\\
|\childdocmain{}|\\
\end{tabular}
\end{center}
at the very top of the main \LaTeX{} file,
in particular \emph{before} the |\documentclass| statement!
The argument of |\childdocmain| should be left empty
(but it must be present).

%%%%%%%%%%%%%%%%%%%%%%%%%%%%%%%%%%%%%%%%
\DescribeMacro{\childdocof}
Furthermore, add the commands
\begin{center}
\begin{tabular}{l}
|\input{childdoc.def}|\\
|\childdocof{|\textit{main}|}|\\
\end{tabular}
\end{center}
at the top of every child file \textit{child}
which is included by |\include{|\textit{child}|}|
from within the main file
(or at least for those files to be compiled individually).
The argument \textit{main} must be the filename of the main file.

There are a couple of
considerations in setting up the main and child documents:

%%%%%%%%%%%%%%%%%%%%%%%%%%%%%%%%%%%%%%%%
\paragraph{Restrictions.}

Please note the following restrictions:
\begin{itemize}
\item
|\childdocmain| must be called with one argument \textit{main}
to ensure compatibility with earlier version of the package.
It must either be empty (|\childdocmain{}|)
or precisely match the filename of the main file in which it is specified.
See \secref{sec:detection} for further information.
\item
The filename \textit{main} must be specified without the |.tex| extension.
\item
The filename \textit{main} is case sensitive
(even in case-insensitive file systems)
due to internal string comparison.
\item
The argument \textit{main} should be fully expanded, it cannot be a macro.
\item
Subdirectories and special characters should be avoided in filenames.
\item
The command |\childdocmain{|\textit{main}|}| must be followed by a whitespace.
It should not be followed immediately by another command
or by a comment mark `|%|'.
This is because the \TeX{} parser reads the token immediately following
the argument of |\childdocmain| and puts it
at the beginning of every child section;
however, a white\-space is ignored.
\end{itemize}

%%%%%%%%%%%%%%%%%%%%%%%%%%%%%%%%%%%%%%%%
\paragraph{Content of Main File.}

It is advisable to place all content in the child files included by |\include|.
Any output contained in the main file will appear in all child documents
unless suppressed manually;
it cannot be suppressed automatically by the |\includeonly| directive
and thus should normally be avoided.
A method to include some content in the main file
by means of conditional processing is described in \secref{sec:conditional}.

%%%%%%%%%%%%%%%%%%%%%%%%%%%%%%%%%%%%%%%%
\paragraph{Page Numbering.}

When only a part of the document is compiled,
the appropriate numbering of pages
(as well as other status parameters)
is determined from the |.aux| files.
The latter contain information from previous passes.
However this information needs to propagate through
all intermediate child documents.
Therefore the page numbering in child documents may well
be inconsistent until the complete document is compiled at least once.

A useful (if unconventional) way to always ensure a consistent
page numbering is to restart the numbering in each child document
and denote the pages by `\textit{child}|.|\textit{page}'
where \textit{child} represents the chapter/section number of the child file.
This can be achieved by the command
|\numberwithin{page}{|\textit{child}|}|
of the \textsf{amsmath} package
where \textit{child} can be |chapter| or |section|
depending on the chosen structuring.
Alternatively, one can modify the macro |\thepage| appropriately
and reset the counter |page| at the start of each child file.

%%%%%%%%%%%%%%%%%%%%%%%%%%%%%%%%%%%%%%%%%%%%%%%%%%%%%%%%%%%%%%%%%%%%%%%%%%%%%%%%
\subsection{Conditional Processing}
\label{sec:conditional}

The package provides a mechanism to compile different versions
of a document. To customise the versions further some conditional processing
can come in handy to distinguish which version is being compiled.
The package provides two macros to describe the compilation context:

%%%%%%%%%%%%%%%%%%%%%%%%%%%%%%%%%%%%%%%%
\DescribeMacro{\ifchilddoc}
The conditional |\ifchilddoc| distinguishes between the compilation of
child documents and the main document:
%
\begin{center}
|\ifchilddoc |\textit{child-code}| |[|\||else |\textit{main-code}]| \||fi|
\end{center}

%%%%%%%%%%%%%%%%%%%%%%%%%%%%%%%%%%%%%%%%
\DescribeMacro{\childdocname}
\DescribeMacro{\childdocjob}
The macro |\childdocname| contains the filename (without extension)
of the main or child file being processed.
Note that |\childdocjob| will always contain the name of the main file.

%%%%%%%%%%%%%%%%%%%%%%%%%%%%%%%%%%%%%%%%
\paragraph{Title Page.}

Conditional processing can be used to include a title or banner page
in the main document when proper precautions are taken.
Importantly, the code in the main file should ensure that the page counter
(as well as other status parameters which are stored in the |.aux| files)
takes the same value after the conditional processing.
Otherwise the page numbers may take divergent values
depending on which part is compiled.

For example, a title page could be declared by:
%
\begin{center}
\begin{tabular}{l}
|\ifchilddoc\||else|\\
|\addtocounter{page}{-1}|\\
\textit{code for title page}\\
|\newpage|\\
|\||fi|
\end{tabular}
\end{center}
%
A banner page for the child documents can be generated by:
%
\begin{center}
\begin{tabular}{l}
|\ifchilddoc|\\
|\addtocounter{page}{-1}|\\
\textit{code for banner page}\\
|\newpage|\\
|\||fi|
\end{tabular}
\end{center}
%
Here one could write a message such as:
\begin{center}
|This is the part \childdocname{} of \childdocjob{}.|
\end{center}

%%%%%%%%%%%%%%%%%%%%%%%%%%%%%%%%%%%%%%%%%%%%%%%%%%%%%%%%%%%%%%%%%%%%%%%%%%%%%%%%
\subsection{Flags}
\label{sec:flags}

The package makes it easy to generate different versions
of the main or child documents.
To this end compilation flags can be defined
and assigned different default values.
They will be particularly useful in conjunction
with the forwarding mechanism described in \secref{sec:forward}.

For example, it may be useful to have a flag |\version|
which can be set to |draft| or |final|.
The document source will contain some conditional code
depending on the value of |\version|.
Suppose further, the flag should default to |final| for the main file
and to |draft| for child files
which is a natural assignment for editing the document.
This is achieved by placing the following code
in the preamble of the main document
(below the |\childdocmain| directive):
%
\begin{center}
\begin{tabular}{l}
|\ifchilddoc|\\
|\providecommand{\version}{draft}|\\
|\||else|\\
|\providecommand{\version}{final}|\\
|\||fi|
\end{tabular}
\end{center}
%
The definition by |\providecommand| makes sure
that previous definitions are not overwritten.
Further statements |\providecommand{\version}{...}|
can thus be added before the above code to override it.

For the main file, one might add a line
(between |\childdocmain| and the above block)
%
\begin{center}
|%\ifchilddoc\||else\providecommand{\version}{draft}\||fi|
\end{center}
%
which can be uncommented to produce a draft version.
Likewise one can add a line to the very top of a child file
(above the |\childdocof{|\textit{main}|}| directive)
%
\begin{center}
|%\providecommand{\version}{final}|
\end{center}
%
which can be uncommented to produce the final version of this child document.

%%%%%%%%%%%%%%%%%%%%%%%%%%%%%%%%%%%%%%%%%%%%%%%%%%%%%%%%%%%%%%%%%%%%%%%%%%%%%%%%
\subsection{Forwarding}
\label{sec:forward}

Different versions of the main or child documents
using compilation flags as described in \secref{sec:flags}
can be (permanently) stored in different files
for convenient compilation, viewing and distribution.
To this end, the package defines a command
to pass on compilation to a different file:

%%%%%%%%%%%%%%%%%%%%%%%%%%%%%%%%%%%%%%%%
\DescribeMacro{\childdocforward}
The command |\childdocforward| redirects processing to
another source file:
%
\begin{center}
\begin{tabular}{l}
|\input{childdoc.def}|\\
|\childdocforward[|\textit{main}|]{|\textit{dest}|}|\\
\end{tabular}
\end{center}
%
The argument \textit{dest} is the destination file
(without extension).
It should be the main file or one of the child files.
Note that further \textsf{childdoc} directives
such as |\childdocof| and |\childdocforward|
in the indicated file will be processed in this form.
The optional argument \textit{main}
passes on directly to the main file \textit{main}
while pretending to compile the child \textit{dest}.
This form behaves as if \textit{dest}
issues |\childdocof{|\textit{main}|}| right away,
and no further \textsf{childdoc} directives will be processed.

%%%%%%%%%%%%%%%%%%%%%%%%%%%%%%%%%%%%%%%%
\DescribeMacro{\...prefix}
In the alternative form |\childdocforwardprefix|,
%
\begin{center}
\begin{tabular}{l}
|\input{childdoc.def}|\\
|\childdocforwardprefix[|\textit{main}|]{|\textit{prefix}|}{|\textit{dest}|}|
\end{tabular}
\end{center}
%
the destination file is determined by a pattern
depending on the current file:
To make this work, the current file must be called
`{\textit{prefix}\hspace{0.2em}\textit{suffix}}'
with \textit{prefix} matching precisely the argument.
Processing is then passed on to the file
`{\textit{dest}\hspace{0.2em}\textit{suffix}}'.
Surely, the same effect is achieved by
directly specifying the
argument `{\textit{dest}\hspace{0.2em}\textit{suffix}}'
in the first form.
However, that requires to set up a different file
for each child. With the alternative form of the command
all these files can have exactly the same content
which simplifies setting them up and maintaining them.

For example, the following file |draft.tex|
with a compilation flag |\version| as described in \secref{sec:flags}
compiles the main document as a draft:
%
\begin{center}
\begin{tabular}{l}
|\def\version{draft}|\\
|\input{childdoc.def}|\\
|\childdocforward{|\textit{main}|}|
\end{tabular}
\end{center}
%
Likewise, the following files |final|\textit{nn}|.tex|
compile the final version of the child document
|child|\textit{nn}|.tex|:
%
\begin{center}
\begin{tabular}{l}
|\def\version{final}|\\
|\input{childdoc.def}|\\
|\childdocforwardprefix{final}{child}|
\end{tabular}
\end{center}
%

Note that when several versions of a main file and/or of each child file
are to be generated, it may be convenient to set up a |Makefile| or
shell script to automatise the process.

%%%%%%%%%%%%%%%%%%%%%%%%%%%%%%%%%%%%%%%%%%%%%%%%%%%%%%%%%%%%%%%%%%%%%%%%%%%%%%%%
\subsection{Command Line Processing}
\label{sec:commandline}

The effect of redirection files can also be achieved by invoking
the \LaTeX{} compiler with a more elaborate command line.
Most conveniently this should be done as part
of a shell script or a |Makefile|.

When using \textsf{childdoc} in the main file, the following
command lines effectively perform a redirection
(note that depending on the shell being used,
backslashes may have to be doubled: `|\|' $\to$ `|\\|'):
%
\begin{center}
|... -jobname "|\textit{target}|" |\\|"|[\textit{flags}]%
|\input{childdoc.def}\childdocforward[|\textit{main}|]{|\textit{dest}|}"|
\end{center}
%
Here \textit{target} is the name of the output file,
\textit{main} is the name of the main file
and \textit{dest} is the name of the main or child file to be processed
(all filenames without extensions).
The optional argument \textit{main} can be omitted
if \textit{main} matches \textit{dest}.
Optionally, compilation \textit{flags} can be defined via |\def| commands.
This command line makes the \TeX{} engine believe
it is compiling the file \textit{target}
whose content is specified as the latter parameter.
The provided code then forwards the processing to
\textit{main} or \textit{dest} as described in \secref{sec:forward}.

%%%%%%%%%%%%%%%%%%%%%%%%%%%%%%%%%%%%%%%%%%%%%%%%%%%%%%%%%%%%%%%%%%%%%%%%%%%%%%%%
\subsection{Include by Input}
\label{sec:input}

Including child documents by |\include| has some restrictions by design.
Most notably, the content of a child document always occupies
its own set of pages; pages cannot be shared between child documents.
Usually, this behaviour makes perfect sense
because each child document contain an essential part of the document.
However, in some situations it may be desirable to compose
a document from a collection of parts
without having mandatory page breaks between then.
For this case, the package
provides a mechanism to include parts
by |\input| which can also be processed individually.
However, by construction this mechanism
requires manual handling of the content to be output.

%%%%%%%%%%%%%%%%%%%%%%%%%%%%%%%%%%%%%%%%
\DescribeMacro{\ifchilddocmanual}
The main file should be prepared as usual, see \secref{sec:include}.
However, the document body must make a distinction
between processing of an individual part and of the main document, e.g.:
%
\begin{center}
\begin{tabular}{l}
|\ifchilddocmanual|\\
|\input{\childdocname}|\\
|\||else|\\
\textit{document body with }|\input{|\textit{part}|}|\\
|\||fi|
\end{tabular}
\end{center}
%
The conditional |\ifchilddocmanual| is true whenever
a part to be included by |\input| is being compiled,
and the name of the part is stored in |\childdocname|.

%%%%%%%%%%%%%%%%%%%%%%%%%%%%%%%%%%%%%%%%
\DescribeMacro{\childdocby}
Each part to be included by |\input| should start with:
%
\begin{center}
\begin{tabular}{l}
|\input{childdoc.def}|\\
|\childdocby{|\textit{main}|}|\\
\end{tabular}
\end{center}
%
The directive |\childdocby| is similar to |\childdocof|
described in \secref{sec:include},
but the subsequent selection of content must be done manually.
To that end, both |\ifchilddoc| and |\ifchilddocmanual|
will be true upon processing of a part,
and the name of the part is stored in |\childdocname|.
Note that |\jobname| will be set to the filename of the current part
so that each part receives an individual |.aux| file
that does not interfere with the |.aux| file(s) of the main document.
This behaviour can be altered by the alternative form
|\childdocby[*]{|\textit{main}|}| (with a non-empty optional argument)
which uses the |.aux| file of the main document
by setting |\jobname| to \textit{main}.

%%%%%%%%%%%%%%%%%%%%%%%%%%%%%%%%%%%%%%%%%%%%%%%%%%%%%%%%%%%%%%%%%%%%%%%%%%%%%%%%
\subsection{Driver Development}
\label{sec:driver}

The \textsf{childdoc} mechanism can also be use for the development
of definition files such as \LaTeX{} styles or classes.
This case differs from the above setup with multiple parts
included by |\include| in that no |\includeonly| should be invoked.
This can be achieved by starting the include file
(before |\ProvidesPackage|) with:
%
\begin{center}
\begin{tabular}{l}
|\input{childdoc.def}|\\
|\childdocforward{|\textit{main}|}|\\
\end{tabular}
\end{center}
%
or alternatively with:
%
\begin{center}
\begin{tabular}{l}
|\input{childdoc.def}|\\
|\childdocby{|\textit{main}|}|\\
\end{tabular}
\end{center}
%
Both forms have slightly different effects as described above.
The main file is prepared as usual, see \secref{sec:include}.

%%%%%%%%%%%%%%%%%%%%%%%%%%%%%%%%%%%%%%%%%%%%%%%%%%%%%%%%%%%%%%%%%%%%%%%%%%%%%%%%
\subsection{Legacy Detection}
\label{sec:detection}

The directive |\childdocmain| in the main file can detect
whether the complete document or merely a child is to be compiled
even without using the directive |\childdocof|.
This method is deprecated because it is less robust
and there is no compelling reason to use it;
it is merely provided for backward compatibility
and it may be removed in future versions.

If the detection mechanism is to be used,
it is mandatory to correctly specify
the filename of the main file as the argument of |\childdocmain|:
%
\begin{center}
\begin{tabular}{l}
|\input{childdoc.def}|\\
|\childdocmain{|\textit{main}|}|\\
\end{tabular}
\end{center}
%
If |\jobname| does not match the argument \textit{main} of |\childdocmain|,
it is assumed that |\jobname| points to the child file to be compiled.
When using |\childdocmain| with the main file specified as argument,
it suffices to start a child file
with just |\input{|\textit{main}|}|
without loading of the package and using |\childdocof|.
If instead all processing is done
with the appropriate \textsf{childdoc} directives,
the argument of \textit{main} of |\childdocmain| can be empty.

An alternative version of the command line processing described
in \secref{sec:commandline} using the detection mechanism reads:
%
\begin{center}
|... -jobname "|\textit{target}|" "|[\textit{flags}]%
[|\def\jobname{|\textit{dest}|}|]|\input{|\textit{main}|}"|
\end{center}

%%%%%%%%%%%%%%%%%%%%%%%%%%%%%%%%%%%%%%%%%%%%%%%%%%%%%%%%%%%%%%%%%%%%%%%%%%%%%%%%
\subsection{Manual Code}
\label{sec:manual}

In case one cannot be certain whether the definitions file |childdoc.def|
is installed on the target \TeX{} distribution
and one prefers not to ship it,
it is conceivable to paste a few relevant commands into the sources.

To that end, drop all statements |\input{childdoc.def}|
and perform the replacements as outlined below.
Instead of |\childdocmain{|\textit{main}|}| add the following code
to the top of the main file:
%
\begin{center}
\begin{tabular}{l}
|\||ifdefined\childdocname\endinput\||fi\newif\ifchilddoc|\\
|\edef\childdocname{\scantokens\expandafter{\jobname\noexpand}}|\\
|\def\childdocmain{|\textit{main}|}\||ifx\childdocmain\childdocname\||else|\\
|\childdoctrue\includeonly{\childdocname}\let\jobname\childdocmain\||fi|\\
\end{tabular}
\end{center}
%
Instead of |\childdocof{|\textit{main}|}| just include the main file
at the top of each child file:
%
\begin{center}
|\input{|\textit{main}|}|
\end{center}
%
A simple redirection |\childdocforward{|\textit{dest}|}| is achieved by:
%
\begin{center}
|\def\jobname{|\textit{dest}|}\input{\jobname}|
\end{center}
%
The redirection with prefix
|\childdocforwardprefix[|\textit{prefix}|]{|\textit{dest}|}|
is accomplished by:
%
\begin{center}
\begin{tabular}{l}
|{\edef\jobname{\scantokens\expandafter{\jobname\noexpand}}|\\
|\def\redirectjob |\textit{prefix}|#1~~~{\gdef\jobname{|\textit{dest}|#1}}|\\
|\expandafter\redirectjob\jobname~~~}\input{\jobname}|
\end{tabular}
\end{center}

In an alternative approach,
child documents can be compiled by a specific command line
without additional code or specific definitions:
%
\begin{center}
|... -jobname "|\textit{target}|" "|[\textit{flags}]%
|\includeonly{|\textit{dest}|}\input{|\textit{main}|}"|
\end{center}
%

%%%%%%%%%%%%%%%%%%%%%%%%%%%%%%%%%%%%%%%%%%%%%%%%%%%%%%%%%%%%%%%%%%%%%%%%%%%%%%%%
%%%%%%%%%%%%%%%%%%%%%%%%%%%%%%%%%%%%%%%%%%%%%%%%%%%%%%%%%%%%%%%%%%%%%%%%%%%%%%%%
\section{Information}

%%%%%%%%%%%%%%%%%%%%%%%%%%%%%%%%%%%%%%%%%%%%%%%%%%%%%%%%%%%%%%%%%%%%%%%%%%%%%%%%
\subsection{Copyright}

Copyright \copyright{} 2017--2018 Niklas Beisert

This work may be distributed and/or modified under the
conditions of the \LaTeX{} Project Public License, either version 1.3
of this license or (at your option) any later version.
The latest version of this license is in
  \url{http://www.latex-project.org/lppl.txt}
and version 1.3 or later is part of all distributions of \LaTeX{}
version 2005/12/01 or later.

This work has the LPPL maintenance status `maintained'.

The Current Maintainer of this work is Niklas Beisert.

This work consists of the files |README.txt|, |childdoc.ins| and |childdoc.dtx|
as well as the derived files |childdoc.def|, |cdocsamp.tex|
with |cdocsch1.tex|, |cdocsch2.tex|, |cdocspt3.tex|, |cdocspt4.tex|,
|cdocsdrf.tex|, |cdocsfn1.tex|, |cdocsfn2.tex|
as well as |childdoc.pdf|.

%%%%%%%%%%%%%%%%%%%%%%%%%%%%%%%%%%%%%%%%%%%%%%%%%%%%%%%%%%%%%%%%%%%%%%%%%%%%%%%%
\subsection{Files and Installation}

The package consists of the files:
%
\begin{center}
\begin{tabular}{ll}
    |README.txt|   & readme file \\
    |childdoc.ins| & installation file \\
    |childdoc.dtx| & source file \\
    |childdoc.def| & definition file \\
    |cdocsamp.tex| & sample main file \\
    |cdocsch1.tex| & sample include file \\
    |cdocsch2.tex| & sample include file \\
    |cdocspt3.tex| & sample part file \\
    |cdocspt4.tex| & sample part file \\
    |cdocsdrf.tex| & sample redirection file \\
    |cdocsfn1.tex| & sample redirection file \\
    |cdocsfn2.tex| & sample redirection file \\
    |childdoc.pdf| & manual
\end{tabular}
\end{center}
%
The distribution consists of the files
|README.txt|, |childdoc.ins| and |childdoc.dtx|.
%
\begin{itemize}
\item
Run (pdf)\LaTeX{} on |childdoc.dtx|
to compile the manual |childdoc.pdf| (this file).
\item
Run \LaTeX{} on |childdoc.ins| to create the definitions file |childdoc.def|
and the sample |cdocsamp.tex| with include files
|cdocsch1.tex|, |cdocsch2.tex|, |cdocspt3.tex|, |cdocspt4.tex|,
|cdocsdrf.tex|, |cdocsfn1.tex|, |cdocsfn2.tex|.
Then copy the file |childdoc.def| to an appropriate directory of your \LaTeX{}
distribution, e.g.\ \textit{texmf-root}|/tex/latex/childdoc|.
\end{itemize}

%%%%%%%%%%%%%%%%%%%%%%%%%%%%%%%%%%%%%%%%%%%%%%%%%%%%%%%%%%%%%%%%%%%%%%%%%%%%%%%%
\subsection{Related CTAN Packages}

There are several other packages which offer a similar functionality:
%
\begin{itemize}
\item
The packages
\href{http://ctan.org/pkg/docmute}{\textsf{docmute}},
\href{http://ctan.org/pkg/includex}{\textsf{includex}} and
\href{http://ctan.org/pkg/standalone}{\textsf{standalone}}
provide commands to include only the document body of
a child file thus allowing both files to be compiled individually.
\item
The packages \href{http://ctan.org/pkg/subdocs}{\textsf{subdocs}}
and \href{http://ctan.org/pkg/subfiles}{\textsf{subfiles}}
provide structures in which the main and child documents can be
encapsulated and allowing them to be compiled individually.
The inclusion mechanism is different from the conventional |\include|.
\item
The package \href{http://ctan.org/pkg/combine}{\textsf{combine}}
is an elaborate solution to combine several documents into one.
\end{itemize}
%
See also the CTAN topic \href{http://ctan.org/topic/subdocs}{\textsf{subdocs}}
for further related packages.
The present package differs from the above solutions in that
a document structure constructed with the conventional |\include| mechanism
just needs two extra commands at the top of every file
such that all constituent files can be compiled individually.

%%%%%%%%%%%%%%%%%%%%%%%%%%%%%%%%%%%%%%%%%%%%%%%%%%%%%%%%%%%%%%%%%%%%%%%%%%%%%%%%
%\subsection{Feature Suggestions}
%
%The following is a list of features which may be useful for future
%versions of this package:
%%
%\begin{itemize}
%\item
%\ldots
%\end{itemize}

%%%%%%%%%%%%%%%%%%%%%%%%%%%%%%%%%%%%%%%%%%%%%%%%%%%%%%%%%%%%%%%%%%%%%%%%%%%%%%%%
\subsection{Revision History}

%%%%%%%%%%%%%%%%%%%%%%%%%%%%%%%%%%%%%%%%
\paragraph{v2.0:} 2018/12/30

\begin{itemize}
\item
immediate forward processing
\item
added |\childdocby| mechanism
\item
manual restructured
\end{itemize}

%%%%%%%%%%%%%%%%%%%%%%%%%%%%%%%%%%%%%%%%
\paragraph{v1.6:} 2018/01/17

\begin{itemize}
\item
application for development of include files
\item
corrections to manual
\end{itemize}

%%%%%%%%%%%%%%%%%%%%%%%%%%%%%%%%%%%%%%%%
\paragraph{v1.5:} 2017/05/21

\begin{itemize}
\item
more complete structuring introduced
\item
|\childdocof| introduced
\item
|\childdoc| renamed to |\childdocmain|
\item
|\childredirect| renamed to |\childdocforward| and |\childdocforwardprefix|
and functionality expanded
\end{itemize}

%%%%%%%%%%%%%%%%%%%%%%%%%%%%%%%%%%%%%%%%
\paragraph{v1.0:} 2017/04/27

\begin{itemize}
\item
manual and install package
\item
first version published on CTAN
\end{itemize}

%%%%%%%%%%%%%%%%%%%%%%%%%%%%%%%%%%%%%%%%
\paragraph{v0.6:} 2017/04/26

\begin{itemize}
\item
redirection mechanism added
\end{itemize}

%%%%%%%%%%%%%%%%%%%%%%%%%%%%%%%%%%%%%%%%
\paragraph{v0.5:} 2017/04/26

\begin{itemize}
\item
functionality in definition file
\end{itemize}


%%%%%%%%%%%%%%%%%%%%%%%%%%%%%%%%%%%%%%%%%%%%%%%%%%%%%%%%%%%%%%%%%%%%%%%%%%%%%%%%
%%%%%%%%%%%%%%%%%%%%%%%%%%%%%%%%%%%%%%%%%%%%%%%%%%%%%%%%%%%%%%%%%%%%%%%%%%%%%%%%
%%%%%%%%%%%%%%%%%%%%%%%%%%%%%%%%%%%%%%%%%%%%%%%%%%%%%%%%%%%%%%%%%%%%%%%%%%%%%%%%
\appendix

\settowidth\MacroIndent{\rmfamily\scriptsize 000\ }

 \DocInput{childdoc.dtx}

\end{document}
%</driver>
% \fi
%
% %%%%%%%%%%%%%%%%%%%%%%%%%%%%%%%%%%%%%%%%%%%%%%%%%%%%%%%%%%%%%%%%%%%%%%%%%%%%%%
% %%%%%%%%%%%%%%%%%%%%%%%%%%%%%%%%%%%%%%%%%%%%%%%%%%%%%%%%%%%%%%%%%%%%%%%%%%%%%%
% \section{Sample}
%\iffalse
%<*samplemain>
%\fi
%
% The following presents a sample document
% with two chapters, two parts, a title page,
% a compile flag as well as three forwarding files to set the flag.
% It consists of eight |.tex| files:
% \begin{center}
% \begin{tabular}{ll}
% |cdocsamp.tex|&main file\\
% |cdocsch1.tex|&include file for chapter 1\\
% |cdocsch2.tex|&include file for chapter 2\\
% |cdocspt3.tex|&include file for part 3\\
% |cdocspt4.tex|&include file for part 4\\
% |cdocsdrf.tex|&forwarding file for main file in draft mode\\
% |cdocsfi1.tex|&forwarding file for final version of chapter 1\\
% |cdocsfi2.tex|&forwarding file for final version of chapter 2\\
% \end{tabular}
% \end{center}
% Each of the eight files can be compiled directly by the \LaTeX{} compiler.
%
% %%%%%%%%%%%%%%%%%%%%%%%%%%%%%%%%%%%%%%
% \paragraph{Main File.}
%
% The main file is called |cdocsamp.tex|.
%
% Load the \textsf{childdoc} definitions and
% declare the filename for the main document:
%    \begin{macrocode}
\input{childdoc.def}
\childdocmain{}
%    \end{macrocode}

% Optional override for |\version| flag:
%    \begin{macrocode}
%%\ifchilddoc\else\providecommand{\version}{draft}\fi
%    \end{macrocode}

% Define the default values for the |\version| flag
% (|final| for the main file and |draft| for childs):
%    \begin{macrocode}
\ifchilddoc
\providecommand{\version}{draft}
\else
\providecommand{\version}{final}
\fi
%    \end{macrocode}

% Load the standard document class:
%    \begin{macrocode}
\documentclass[12pt]{article}
%    \end{macrocode}

% Start the document body:
%    \begin{macrocode}
\begin{document}
%    \end{macrocode}

% Declare a title page.
% Print title, part of document being processed and version flag:
%    \begin{macrocode}
\addtocounter{page}{-1}
\begin{center}
{\LARGE\bfseries{}childdoc example\par}
\vspace{1cm}
\ifchilddoc
\ifchilddocmanual part\else chapter\fi:
`\childdocname' of `\childdocjob'\par
\else
main document: `\childdocjob'\par
\fi
version: \version\par
\end{center}
\newpage
%    \end{macrocode}

% Manually include selected file,
% otherwise process as usual:
%    \begin{macrocode}
\ifchilddocmanual
\section*{part `\childdocname'}
\input{\childdocname}
\else
%    \end{macrocode}

% Include the two chapters:
%    \begin{macrocode}
\include{cdocsch1}
\include{cdocsch2}
%    \end{macrocode}

% Include the two parts unless only chapters should be displayed:
%    \begin{macrocode}
\ifchilddoc\else
\section{part three}
\input{cdocspt3}
\section{part four}
\input{cdocspt4}
\fi
%    \end{macrocode}

% Process as usual until here:
%    \begin{macrocode}
\fi
%    \end{macrocode}

% End of document body:
%    \begin{macrocode}
\end{document}
%    \end{macrocode}
%\iffalse
%</samplemain>
%\fi
%
% %%%%%%%%%%%%%%%%%%%%%%%%%%%%%%%%%%%%%%
% \paragraph{Chapter Include Files.}
%
% The include files are called |cdocsch1.tex| and |cdocsch2.tex|.
%
%\iffalse
%<*samplechap1|samplechap2>
%\fi

% Optional override for |\version| flag:
%    \begin{macrocode}
%%\providecommand{\version}{final}
%    \end{macrocode}

% Include the main document:
%    \begin{macrocode}
\input{childdoc.def}
\childdocof{cdocsamp}
%    \end{macrocode}

%\iffalse
%</samplechap1|samplechap2>
%\fi
%
%\iffalse
%<*samplechap1>
%\fi
% Some text for chapter 1:
%    \begin{macrocode}
\section{one}
some text in chapter one
%    \end{macrocode}

%\iffalse
%</samplechap1>
%\fi
% Some text for chapter 2:
%\iffalse
%<*samplechap2>
%\fi
%    \begin{macrocode}
\section{two}
more text in chapter two
%    \end{macrocode}

%\iffalse
%</samplechap2>
%\fi
%
% %%%%%%%%%%%%%%%%%%%%%%%%%%%%%%%%%%%%%%
% \paragraph{Part Include Files.}
%
% The include files are called |cdocspt3.tex| and |cdocspt4.tex|.
%
%\iffalse
%<*samplepart3|samplepart4>
%\fi

% Optional override for |\version| flag:
%    \begin{macrocode}
%%\providecommand{\version}{final}
%    \end{macrocode}

% Include the main document:
%    \begin{macrocode}
\input{childdoc.def}
\childdocby{cdocsamp}
%    \end{macrocode}

%\iffalse
%</samplepart3|samplepart4>
%\fi
%
%\iffalse
%<*samplepart3>
%\fi
% Some text for part 3:
%    \begin{macrocode}
some text in part three
%    \end{macrocode}

%\iffalse
%</samplepart3>
%\fi
% Some text for part 4:
%\iffalse
%<*samplepart4>
%\fi
%    \begin{macrocode}
more text in part four
%    \end{macrocode}

%\iffalse
%</samplepart4>
%\fi
%
% %%%%%%%%%%%%%%%%%%%%%%%%%%%%%%%%%%%%%%
% \paragraph{Forwarding for a Complete Draft.}
%
% The following forwarding file |cdocsdrf.tex|
% compiles the main document in draft mode:
%\iffalse
%<*sampledraft>
%\fi
%    \begin{macrocode}
\def\version{draft}
\input{childdoc.def}
\childdocforward{cdocsamp}
%    \end{macrocode}

%\iffalse
%</sampledraft>
%\fi
%
% %%%%%%%%%%%%%%%%%%%%%%%%%%%%%%%%%%%%%%
% \paragraph{Forwarding for Final Version of the Chapters.}
%
% The following forwarding files |cdocsfn1.tex| and |cdocsfn2.tex|
% (with identical content)
% compile the final versions of the child documents
% |cdocsch1.tex| and |cdocsch2.tex|, respectively:
%\iffalse
%<*samplefinal>
%\fi
%    \begin{macrocode}
\def\version{final}
\input{childdoc.def}
\childdocforwardprefix[cdocsamp]{cdocsfn}{cdocsch}
%    \end{macrocode}

%\iffalse
%</samplefinal>
%\fi
%
% %%%%%%%%%%%%%%%%%%%%%%%%%%%%%%%%%%%%%%
% \paragraph{Command Line Processing.}
%
% The following three command lines generate the output files
% |cdocscld|, |cdocscl1| and |cdocscl2|
% which should be identical to
% |cdocsdrf|, |cdocsch1| and |cdocsfn2|, respectively:
% \begin{center}
% \begin{tabular}{l}
% |latex -jobname cdocscld \|\\
% |  "\def\version{draft}\input{childdoc.def}\childdocforward{cdocsamp}"|\\
% |latex -jobname cdocscl1 \|\\
% |  "\input{childdoc.def}\childdocforward[cdocsamp]{cdocsch1}"|\\
% |latex -jobname cdocscl2 \|\\
% |  "\def\version{final}\input{childdoc.def}\childdocforward{cdocsch2}"|
% \end{tabular}
% \end{center}
% Note that the trailing backslash on each first line
% merely continues the input to the second line
% (for convenient cut ant paste).
% Furthermore, the command |latex| can be replaced by any
% of its alternative versions such as |pdflatex|.
%
% %%%%%%%%%%%%%%%%%%%%%%%%%%%%%%%%%%%%%%%%%%%%%%%%%%%%%%%%%%%%%%%%%%%%%%%%%%%%%%
% %%%%%%%%%%%%%%%%%%%%%%%%%%%%%%%%%%%%%%%%%%%%%%%%%%%%%%%%%%%%%%%%%%%%%%%%%%%%%%
% \section{Implementation}
%\iffalse
%<*package>
%\fi
%
% This section describes the definitions file |childdoc.def|.

% The definitions cannot be loaded using |\usepackage| or |\RequirePackage|
% which has a mechanism to prevent loading a style file more than once.
% When loading the definitions by means of |\input|
% multiple instances have to be prevented manually:
%\iffalse
%This code needs to be before the `\ProvidesFile' directive
%which is defined at the beginning of this file.
%Therefore it is also placed there and commented out here.
%</package>
%<*discard>
%\fi
%    \begin{macrocode}
\ifdefined\childdocmain\endinput\fi
%    \end{macrocode}
%\iffalse
%</discard>
%<*package>
%\fi
%
% \macro{\ifchilddoc}
% \macro{\ifchilddocmanual}
% The conditional |\ifchilddoc| tells whether a
% child (true) or main (false) document is being compiled.
% The conditional |\ifchilddocmanual| tells whether
% the |\includeonly| mechanism is used (false) or
% the selection of child files must be performed manually (true).
% The definitions initialise to false:
%    \begin{macrocode}
\newif\ifchilddoc
\newif\ifchilddocmanual
%    \end{macrocode}

% \macro{\childdocname}
% \macro{\childdocjob}
% The macro |\childdocname| stores the name of the main document
% to be compiled. The macro |\childdocjob| stores the name of
% the document on which the \LaTeX{} compiler was originally invoked.
% The content of |\jobname| cannot be compared
% to filenames specified in the source due to different catcodes.
% The following code rescans |\jobname|, stores the result
% in |\childdocname| and saves a copy in |\childdocjob|:
%    \begin{macrocode}
\edef\childdocname{\scantokens\expandafter{\jobname\noexpand}}
\let\childdocjob\childdocname
%    \end{macrocode}

% \macro{\childdocdisable}
% The macro |\childdocdisable| prevents the main file
% from being processed more than once.
% At this stage, the main document command |\childdocmain|
% is assumed to be called once again where it should do nothing.
% Any subsequent call to it should prevent
% a secondary processing of the main document
% It overwrites the forwarding commands
% |\childdocof| and |\childdocforward|
% with empty macros to prevent further inclusions of the main document:
%    \begin{macrocode}
\newcommand{\childdocdisable}
{
  \renewcommand{\childdocmain}[1]{\renewcommand{\childdocmain}[1]{\endinput}}
  \renewcommand{\childdocof}[1]{}
  \renewcommand{\childdocby}[2][]{}
  \renewcommand{\childdocforward}[2][]{}
  \renewcommand{\childdocdisable}{}
}
%    \end{macrocode}

% \macro{\childdocmain}
% The macro |\childdocmain| is to be called at the top of the main file
% with nothing or the main filename (without extension) as argument.
% First, it breaks loops.
% If the argument is not empty and does not match |\childdocname|
% (which is set by the first inclusion of |childdoc.def|),
% |\ifchilddoc| is set to true, |\includeonly| is applied to the child file
% and |\jobname| is set to the main file
% (for proper handling of |.aux| files):
%    \begin{macrocode}
\newcommand{\childdocmain}[1]
{
  \childdocdisable\childdocmain{}
  \if?#1?\else
    \begingroup
      \def\childdoctmp{#1}
      \ifx\childdoctmp\childdocname
        \def\childdoctmp{}
      \else
        \def\childdoctmp
        {
          \childdoctrue
          \includeonly{\childdocname}
          \def\childdocjob{#1}
          \def\jobname{#1}
        }
      \fi
      \expandafter
    \endgroup
    \childdoctmp
  \fi
}
%    \end{macrocode}

% \macro{\childdocof}
% The command |\childdocof| redirects
% compilation to the main file |#1|.
%    \begin{macrocode}
\newcommand{\childdocof}[1]
{
  \childdocdisable
  \childdoctrue
  \includeonly{\childdocname}
  \def\jobname{#1}
  \def\childdocjob{#1}
  \input{#1}
}
%    \end{macrocode}

% \macro{\childdocby}
% The command |\childdocby| ....
%    \begin{macrocode}
\newcommand{\childdocby}[2][]
{
  \childdocdisable
  \childdoctrue
  \childdocmanualtrue
  \if?#1?\else
    \def\jobname{#2}
  \fi
  \def\childdocjob{#2}
  \input{#2}
  \endinput
}
%    \end{macrocode}

% \macro{\childdocforward}
% The command |\childdocforward| redirects
% compilation to the main file or
% (if the optional argument is given) a child file.
% Parameters are set as if the main file
% or a child file starting with |\childdocof| was compiled.
% Then compilation is handed over to the main file:
%    \begin{macrocode}
\newcommand{\childdocforward}[2][]
{
  \begingroup
    \if?#1?
      \def\childdoctmp
      {
        \def\childdocname{#2}
        \def\childdocjob{#2}
        \def\jobname{#2}
        \input{#2}
        \endinput
      }
    \else
      \def\childdoctmp
      {
        \childdocdisable
        \def\childdocname{#2}
        \childdoctrue
        \includeonly{#2}
        \def\childdocjob{#1}
        \def\jobname{#1}
        \input{#1}
        \endinput
      }
    \fi
    \expandafter
  \endgroup
  \childdoctmp
}
%    \end{macrocode}

% \macro{\childdocforwardprefix}
% The command |\childdocforwardprefix| redirects
% compilation to the main or a child file by means of a pattern.
% The prefix |#1| in the current filename is replaced by |#2|
% and the suffix of the current filename is kept
% (it is assumed that the filename does not contain the substring `|~~~|'
% which is used as a delimiter).
% Compilation is handed over to the new file by |\childdocforward|:
%    \begin{macrocode}
\newcommand{\childdocforwardprefix}[3][]
{
  \begingroup
    \def\childdocextract #2##1~~~{\def\childdoctmp{\childdocforward[#1]{#3##1}}}
    \expandafter\childdocextract\childdocname~~~
    \expandafter
  \endgroup
  \childdoctmp
}
%    \end{macrocode}

% \macro{\childdoc}
% The deprecated macro |\childdoc| is a legacy version of |\childdocmain|:
%    \begin{macrocode}
\newcommand{\childdoc}{\childdocmain}
%    \end{macrocode}

% \macro{\childdocredirect}
% The deprecated macro |\childdocredirect| is a legacy version
% of |\childdocforward| and |\childdocforwardprefix|:
%    \begin{macrocode}
\newcommand{\childdocredirect}[2][]
{
  \begingroup
    \if?#1?
      \def\childdoctmp{\childdocforward{#2}}
    \else
      \def\childdoctmp{\childdocforwardprefix{#1}{#2}}
    \fi
    \expandafter
  \endgroup
  \childdoctmp
}
%    \end{macrocode}

%\iffalse
%</package>
%\fi
%
\endinput
\childdocforward[cdocsamp]{cdocsch1}"|\\
% |latex -jobname cdocscl2 \|\\
% |  "\def\version{final}% \iffalse
%
% childdoc.dtx Copyright (C) 2017-2018 Niklas Beisert
%
% This work may be distributed and/or modified under the
% conditions of the LaTeX Project Public License, either version 1.3
% of this license or (at your option) any later version.
% The latest version of this license is in
%   http://www.latex-project.org/lppl.txt
% and version 1.3 or later is part of all distributions of LaTeX
% version 2005/12/01 or later.
%
% This work has the LPPL maintenance status `maintained'.
%
% The Current Maintainer of this work is Niklas Beisert.
%
% This work consists of the files childdoc.dtx and childdoc.ins
% and the derived files childdoc.def and cdocsamp.tex with
% cdocsch1.tex, cdocsch2.tex, cdocsdrf.tex, cdocsfn1.tex, cdocsfn2.tex.
%
%<package>\ifdefined\childdocmain\endinput\fi
%<package>\ProvidesFile{childdoc.def}[2018/12/30 v2.0 child document driver]
%<samplemain>\ProvidesFile{cdocsamp.tex}[2018/12/30 v2.0 sample for childdoc]
%<*driver>
%\ProvidesFile{childdoc.drv}[2018/12/30 v2.0 childdoc reference manual file]
\PassOptionsToClass{10pt,a4paper}{article}
\documentclass{ltxdoc}

\usepackage[margin=35mm]{geometry}
\usepackage{hyperref}
\usepackage{hyperxmp}
\usepackage[usenames]{color}

\hypersetup{colorlinks=true}
\hypersetup{pdfstartview=FitH}
\hypersetup{pdfpagemode=UseNone}
\hypersetup{pdfsource={}}
\hypersetup{pdflang={en-UK}}
\hypersetup{pdfcopyright={Copyright 2017-2018 Niklas Beisert.
  This work may be distributed and/or modified under the
  conditions of the LaTeX Project Public License, either version 1.3
  of this license or (at your option) any later version.}}
\hypersetup{pdflicenseurl={http://www.latex-project.org/lppl.txt}}
\hypersetup{pdfcontactaddress={ETH Zurich, ITP, HIT K,
  Wolfgang-Pauli-Strasse 27}}
\hypersetup{pdfcontactpostcode={8093}}
\hypersetup{pdfcontactcity={Zurich}}
\hypersetup{pdfcontactcountry={Switzerland}}
\hypersetup{pdfcontactemail={nbeisert@itp.phys.ethz.ch}}
\hypersetup{pdfcontacturl={http://people.phys.ethz.ch/\xmptilde nbeisert/}}

\newcommand{\secref}[1]{\hyperref[#1]{section \ref*{#1}}}

\parskip1ex
\parindent0pt
\let\olditemize\itemize
\def\itemize{\olditemize\parskip0pt}

\begin{document}

\title{The \textsf{childdoc} Package}
\hypersetup{pdftitle={The childdoc Package}}
\author{Niklas Beisert\\[2ex]
  Institut f\"ur Theoretische Physik\\
  Eidgen\"ossische Technische Hochschule Z\"urich\\
  Wolfgang-Pauli-Strasse 27, 8093 Z\"urich, Switzerland\\[1ex]
  \href{mailto:nbeisert@itp.phys.ethz.ch}
  {\texttt{nbeisert@itp.phys.ethz.ch}}}
\hypersetup{pdfauthor={Niklas Beisert}}
\hypersetup{pdfsubject={Manual for the LaTeX2e Package childdoc}}
\date{30 December 2018, \textsf{v2.0}}
\maketitle

\begin{abstract}\noindent
\textsf{childdoc} is a \LaTeXe{} package
that enables the direct compilation
of document sections included by |\include|
to individual files.
\end{abstract}

\begingroup
\parskip0ex
\tableofcontents
\endgroup

%%%%%%%%%%%%%%%%%%%%%%%%%%%%%%%%%%%%%%%%%%%%%%%%%%%%%%%%%%%%%%%%%%%%%%%%%%%%%%%%
%%%%%%%%%%%%%%%%%%%%%%%%%%%%%%%%%%%%%%%%%%%%%%%%%%%%%%%%%%%%%%%%%%%%%%%%%%%%%%%%
\section{Introduction}

\LaTeX{} provides a mechanism to structure a large document (such as a book)
into a main file and several child files (containing the chapters)
using the |\include| command.
This mechanism is beneficial for documents
which span hundreds of pages in order to
make the source file(s) more manageable.
Moreover, compilation can be restricted to
selected child files by means of the |\includeonly| command.
The latter feature can be used to reduce the compilation time while editing
(this was significantly more useful in the earlier days of \LaTeX{})
or to generate a smaller document which is easier to navigate.
Another application of |\includeonly| is to generate
documents consisting of selected parts of the complete document.

However, there are a few drawbacks of the plain |\include| mechanism:
\begin{itemize}
\item
The child files cannot be compiled on their own,
they can only be compiled via the main file.
A naive editing environment
(such as a text editor with an option
to have the current file processed by \LaTeX)
may require one to switch to the main file before compiling;
attempting to compile the child file produces errors.
\item
The main file must be modified (each time)
to adjust the |\includeonly| command
to the present needs. This easily leaves the main file in a messy state.
\item
The generated document will always carry the filename
of the main document. This is inconvenient if
several child files are to be compiled and
to be kept for distribution.
\end{itemize}

The present package provides a simple interface
to make child files individually compilable by \LaTeX{}.
Compiling a child file then has the same effect as compiling
the main file with an |\includeonly| command
to select the appropriate child.
Moreover the generated document will carry the name of the child
rather than the main file.
This resolves all three above issues.

This feature is meant to make the editing of books,
thesis documents and lecture notes somewhat more convenient.
However, the package can also be used efficiently for
composing a series of documents (such as exercise sheets)
which are typically distributed individually.
It then assists the author in generating the individual documents
(potentially in different versions)
as well as a document containing the collected series.
Another application is in developing style files
or other kinds of included material
where compilation of the style file could redirect
to a sample or test file.

%%%%%%%%%%%%%%%%%%%%%%%%%%%%%%%%%%%%%%%%%%%%%%%%%%%%%%%%%%%%%%%%%%%%%%%%%%%%%%%%
%%%%%%%%%%%%%%%%%%%%%%%%%%%%%%%%%%%%%%%%%%%%%%%%%%%%%%%%%%%%%%%%%%%%%%%%%%%%%%%%
\section{Usage}

First of all, the package \textsf{childdoc} is \emph{not} a standard
\LaTeXe{} |.sty| style file! Therefore it needs to be invoked in
a non-standard way.

%%%%%%%%%%%%%%%%%%%%%%%%%%%%%%%%%%%%%%%%%%%%%%%%%%%%%%%%%%%%%%%%%%%%%%%%%%%%%%%%
\subsection{Included Files}
\label{sec:include}

%%%%%%%%%%%%%%%%%%%%%%%%%%%%%%%%%%%%%%%%
\DescribeMacro{\childdocmain}
To use the package, add the commands
\begin{center}
\begin{tabular}{l}
|\input{childdoc.def}|\\
|\childdocmain{}|\\
\end{tabular}
\end{center}
at the very top of the main \LaTeX{} file,
in particular \emph{before} the |\documentclass| statement!
The argument of |\childdocmain| should be left empty
(but it must be present).

%%%%%%%%%%%%%%%%%%%%%%%%%%%%%%%%%%%%%%%%
\DescribeMacro{\childdocof}
Furthermore, add the commands
\begin{center}
\begin{tabular}{l}
|\input{childdoc.def}|\\
|\childdocof{|\textit{main}|}|\\
\end{tabular}
\end{center}
at the top of every child file \textit{child}
which is included by |\include{|\textit{child}|}|
from within the main file
(or at least for those files to be compiled individually).
The argument \textit{main} must be the filename of the main file.

There are a couple of
considerations in setting up the main and child documents:

%%%%%%%%%%%%%%%%%%%%%%%%%%%%%%%%%%%%%%%%
\paragraph{Restrictions.}

Please note the following restrictions:
\begin{itemize}
\item
|\childdocmain| must be called with one argument \textit{main}
to ensure compatibility with earlier version of the package.
It must either be empty (|\childdocmain{}|)
or precisely match the filename of the main file in which it is specified.
See \secref{sec:detection} for further information.
\item
The filename \textit{main} must be specified without the |.tex| extension.
\item
The filename \textit{main} is case sensitive
(even in case-insensitive file systems)
due to internal string comparison.
\item
The argument \textit{main} should be fully expanded, it cannot be a macro.
\item
Subdirectories and special characters should be avoided in filenames.
\item
The command |\childdocmain{|\textit{main}|}| must be followed by a whitespace.
It should not be followed immediately by another command
or by a comment mark `|%|'.
This is because the \TeX{} parser reads the token immediately following
the argument of |\childdocmain| and puts it
at the beginning of every child section;
however, a white\-space is ignored.
\end{itemize}

%%%%%%%%%%%%%%%%%%%%%%%%%%%%%%%%%%%%%%%%
\paragraph{Content of Main File.}

It is advisable to place all content in the child files included by |\include|.
Any output contained in the main file will appear in all child documents
unless suppressed manually;
it cannot be suppressed automatically by the |\includeonly| directive
and thus should normally be avoided.
A method to include some content in the main file
by means of conditional processing is described in \secref{sec:conditional}.

%%%%%%%%%%%%%%%%%%%%%%%%%%%%%%%%%%%%%%%%
\paragraph{Page Numbering.}

When only a part of the document is compiled,
the appropriate numbering of pages
(as well as other status parameters)
is determined from the |.aux| files.
The latter contain information from previous passes.
However this information needs to propagate through
all intermediate child documents.
Therefore the page numbering in child documents may well
be inconsistent until the complete document is compiled at least once.

A useful (if unconventional) way to always ensure a consistent
page numbering is to restart the numbering in each child document
and denote the pages by `\textit{child}|.|\textit{page}'
where \textit{child} represents the chapter/section number of the child file.
This can be achieved by the command
|\numberwithin{page}{|\textit{child}|}|
of the \textsf{amsmath} package
where \textit{child} can be |chapter| or |section|
depending on the chosen structuring.
Alternatively, one can modify the macro |\thepage| appropriately
and reset the counter |page| at the start of each child file.

%%%%%%%%%%%%%%%%%%%%%%%%%%%%%%%%%%%%%%%%%%%%%%%%%%%%%%%%%%%%%%%%%%%%%%%%%%%%%%%%
\subsection{Conditional Processing}
\label{sec:conditional}

The package provides a mechanism to compile different versions
of a document. To customise the versions further some conditional processing
can come in handy to distinguish which version is being compiled.
The package provides two macros to describe the compilation context:

%%%%%%%%%%%%%%%%%%%%%%%%%%%%%%%%%%%%%%%%
\DescribeMacro{\ifchilddoc}
The conditional |\ifchilddoc| distinguishes between the compilation of
child documents and the main document:
%
\begin{center}
|\ifchilddoc |\textit{child-code}| |[|\||else |\textit{main-code}]| \||fi|
\end{center}

%%%%%%%%%%%%%%%%%%%%%%%%%%%%%%%%%%%%%%%%
\DescribeMacro{\childdocname}
\DescribeMacro{\childdocjob}
The macro |\childdocname| contains the filename (without extension)
of the main or child file being processed.
Note that |\childdocjob| will always contain the name of the main file.

%%%%%%%%%%%%%%%%%%%%%%%%%%%%%%%%%%%%%%%%
\paragraph{Title Page.}

Conditional processing can be used to include a title or banner page
in the main document when proper precautions are taken.
Importantly, the code in the main file should ensure that the page counter
(as well as other status parameters which are stored in the |.aux| files)
takes the same value after the conditional processing.
Otherwise the page numbers may take divergent values
depending on which part is compiled.

For example, a title page could be declared by:
%
\begin{center}
\begin{tabular}{l}
|\ifchilddoc\||else|\\
|\addtocounter{page}{-1}|\\
\textit{code for title page}\\
|\newpage|\\
|\||fi|
\end{tabular}
\end{center}
%
A banner page for the child documents can be generated by:
%
\begin{center}
\begin{tabular}{l}
|\ifchilddoc|\\
|\addtocounter{page}{-1}|\\
\textit{code for banner page}\\
|\newpage|\\
|\||fi|
\end{tabular}
\end{center}
%
Here one could write a message such as:
\begin{center}
|This is the part \childdocname{} of \childdocjob{}.|
\end{center}

%%%%%%%%%%%%%%%%%%%%%%%%%%%%%%%%%%%%%%%%%%%%%%%%%%%%%%%%%%%%%%%%%%%%%%%%%%%%%%%%
\subsection{Flags}
\label{sec:flags}

The package makes it easy to generate different versions
of the main or child documents.
To this end compilation flags can be defined
and assigned different default values.
They will be particularly useful in conjunction
with the forwarding mechanism described in \secref{sec:forward}.

For example, it may be useful to have a flag |\version|
which can be set to |draft| or |final|.
The document source will contain some conditional code
depending on the value of |\version|.
Suppose further, the flag should default to |final| for the main file
and to |draft| for child files
which is a natural assignment for editing the document.
This is achieved by placing the following code
in the preamble of the main document
(below the |\childdocmain| directive):
%
\begin{center}
\begin{tabular}{l}
|\ifchilddoc|\\
|\providecommand{\version}{draft}|\\
|\||else|\\
|\providecommand{\version}{final}|\\
|\||fi|
\end{tabular}
\end{center}
%
The definition by |\providecommand| makes sure
that previous definitions are not overwritten.
Further statements |\providecommand{\version}{...}|
can thus be added before the above code to override it.

For the main file, one might add a line
(between |\childdocmain| and the above block)
%
\begin{center}
|%\ifchilddoc\||else\providecommand{\version}{draft}\||fi|
\end{center}
%
which can be uncommented to produce a draft version.
Likewise one can add a line to the very top of a child file
(above the |\childdocof{|\textit{main}|}| directive)
%
\begin{center}
|%\providecommand{\version}{final}|
\end{center}
%
which can be uncommented to produce the final version of this child document.

%%%%%%%%%%%%%%%%%%%%%%%%%%%%%%%%%%%%%%%%%%%%%%%%%%%%%%%%%%%%%%%%%%%%%%%%%%%%%%%%
\subsection{Forwarding}
\label{sec:forward}

Different versions of the main or child documents
using compilation flags as described in \secref{sec:flags}
can be (permanently) stored in different files
for convenient compilation, viewing and distribution.
To this end, the package defines a command
to pass on compilation to a different file:

%%%%%%%%%%%%%%%%%%%%%%%%%%%%%%%%%%%%%%%%
\DescribeMacro{\childdocforward}
The command |\childdocforward| redirects processing to
another source file:
%
\begin{center}
\begin{tabular}{l}
|\input{childdoc.def}|\\
|\childdocforward[|\textit{main}|]{|\textit{dest}|}|\\
\end{tabular}
\end{center}
%
The argument \textit{dest} is the destination file
(without extension).
It should be the main file or one of the child files.
Note that further \textsf{childdoc} directives
such as |\childdocof| and |\childdocforward|
in the indicated file will be processed in this form.
The optional argument \textit{main}
passes on directly to the main file \textit{main}
while pretending to compile the child \textit{dest}.
This form behaves as if \textit{dest}
issues |\childdocof{|\textit{main}|}| right away,
and no further \textsf{childdoc} directives will be processed.

%%%%%%%%%%%%%%%%%%%%%%%%%%%%%%%%%%%%%%%%
\DescribeMacro{\...prefix}
In the alternative form |\childdocforwardprefix|,
%
\begin{center}
\begin{tabular}{l}
|\input{childdoc.def}|\\
|\childdocforwardprefix[|\textit{main}|]{|\textit{prefix}|}{|\textit{dest}|}|
\end{tabular}
\end{center}
%
the destination file is determined by a pattern
depending on the current file:
To make this work, the current file must be called
`{\textit{prefix}\hspace{0.2em}\textit{suffix}}'
with \textit{prefix} matching precisely the argument.
Processing is then passed on to the file
`{\textit{dest}\hspace{0.2em}\textit{suffix}}'.
Surely, the same effect is achieved by
directly specifying the
argument `{\textit{dest}\hspace{0.2em}\textit{suffix}}'
in the first form.
However, that requires to set up a different file
for each child. With the alternative form of the command
all these files can have exactly the same content
which simplifies setting them up and maintaining them.

For example, the following file |draft.tex|
with a compilation flag |\version| as described in \secref{sec:flags}
compiles the main document as a draft:
%
\begin{center}
\begin{tabular}{l}
|\def\version{draft}|\\
|\input{childdoc.def}|\\
|\childdocforward{|\textit{main}|}|
\end{tabular}
\end{center}
%
Likewise, the following files |final|\textit{nn}|.tex|
compile the final version of the child document
|child|\textit{nn}|.tex|:
%
\begin{center}
\begin{tabular}{l}
|\def\version{final}|\\
|\input{childdoc.def}|\\
|\childdocforwardprefix{final}{child}|
\end{tabular}
\end{center}
%

Note that when several versions of a main file and/or of each child file
are to be generated, it may be convenient to set up a |Makefile| or
shell script to automatise the process.

%%%%%%%%%%%%%%%%%%%%%%%%%%%%%%%%%%%%%%%%%%%%%%%%%%%%%%%%%%%%%%%%%%%%%%%%%%%%%%%%
\subsection{Command Line Processing}
\label{sec:commandline}

The effect of redirection files can also be achieved by invoking
the \LaTeX{} compiler with a more elaborate command line.
Most conveniently this should be done as part
of a shell script or a |Makefile|.

When using \textsf{childdoc} in the main file, the following
command lines effectively perform a redirection
(note that depending on the shell being used,
backslashes may have to be doubled: `|\|' $\to$ `|\\|'):
%
\begin{center}
|... -jobname "|\textit{target}|" |\\|"|[\textit{flags}]%
|\input{childdoc.def}\childdocforward[|\textit{main}|]{|\textit{dest}|}"|
\end{center}
%
Here \textit{target} is the name of the output file,
\textit{main} is the name of the main file
and \textit{dest} is the name of the main or child file to be processed
(all filenames without extensions).
The optional argument \textit{main} can be omitted
if \textit{main} matches \textit{dest}.
Optionally, compilation \textit{flags} can be defined via |\def| commands.
This command line makes the \TeX{} engine believe
it is compiling the file \textit{target}
whose content is specified as the latter parameter.
The provided code then forwards the processing to
\textit{main} or \textit{dest} as described in \secref{sec:forward}.

%%%%%%%%%%%%%%%%%%%%%%%%%%%%%%%%%%%%%%%%%%%%%%%%%%%%%%%%%%%%%%%%%%%%%%%%%%%%%%%%
\subsection{Include by Input}
\label{sec:input}

Including child documents by |\include| has some restrictions by design.
Most notably, the content of a child document always occupies
its own set of pages; pages cannot be shared between child documents.
Usually, this behaviour makes perfect sense
because each child document contain an essential part of the document.
However, in some situations it may be desirable to compose
a document from a collection of parts
without having mandatory page breaks between then.
For this case, the package
provides a mechanism to include parts
by |\input| which can also be processed individually.
However, by construction this mechanism
requires manual handling of the content to be output.

%%%%%%%%%%%%%%%%%%%%%%%%%%%%%%%%%%%%%%%%
\DescribeMacro{\ifchilddocmanual}
The main file should be prepared as usual, see \secref{sec:include}.
However, the document body must make a distinction
between processing of an individual part and of the main document, e.g.:
%
\begin{center}
\begin{tabular}{l}
|\ifchilddocmanual|\\
|\input{\childdocname}|\\
|\||else|\\
\textit{document body with }|\input{|\textit{part}|}|\\
|\||fi|
\end{tabular}
\end{center}
%
The conditional |\ifchilddocmanual| is true whenever
a part to be included by |\input| is being compiled,
and the name of the part is stored in |\childdocname|.

%%%%%%%%%%%%%%%%%%%%%%%%%%%%%%%%%%%%%%%%
\DescribeMacro{\childdocby}
Each part to be included by |\input| should start with:
%
\begin{center}
\begin{tabular}{l}
|\input{childdoc.def}|\\
|\childdocby{|\textit{main}|}|\\
\end{tabular}
\end{center}
%
The directive |\childdocby| is similar to |\childdocof|
described in \secref{sec:include},
but the subsequent selection of content must be done manually.
To that end, both |\ifchilddoc| and |\ifchilddocmanual|
will be true upon processing of a part,
and the name of the part is stored in |\childdocname|.
Note that |\jobname| will be set to the filename of the current part
so that each part receives an individual |.aux| file
that does not interfere with the |.aux| file(s) of the main document.
This behaviour can be altered by the alternative form
|\childdocby[*]{|\textit{main}|}| (with a non-empty optional argument)
which uses the |.aux| file of the main document
by setting |\jobname| to \textit{main}.

%%%%%%%%%%%%%%%%%%%%%%%%%%%%%%%%%%%%%%%%%%%%%%%%%%%%%%%%%%%%%%%%%%%%%%%%%%%%%%%%
\subsection{Driver Development}
\label{sec:driver}

The \textsf{childdoc} mechanism can also be use for the development
of definition files such as \LaTeX{} styles or classes.
This case differs from the above setup with multiple parts
included by |\include| in that no |\includeonly| should be invoked.
This can be achieved by starting the include file
(before |\ProvidesPackage|) with:
%
\begin{center}
\begin{tabular}{l}
|\input{childdoc.def}|\\
|\childdocforward{|\textit{main}|}|\\
\end{tabular}
\end{center}
%
or alternatively with:
%
\begin{center}
\begin{tabular}{l}
|\input{childdoc.def}|\\
|\childdocby{|\textit{main}|}|\\
\end{tabular}
\end{center}
%
Both forms have slightly different effects as described above.
The main file is prepared as usual, see \secref{sec:include}.

%%%%%%%%%%%%%%%%%%%%%%%%%%%%%%%%%%%%%%%%%%%%%%%%%%%%%%%%%%%%%%%%%%%%%%%%%%%%%%%%
\subsection{Legacy Detection}
\label{sec:detection}

The directive |\childdocmain| in the main file can detect
whether the complete document or merely a child is to be compiled
even without using the directive |\childdocof|.
This method is deprecated because it is less robust
and there is no compelling reason to use it;
it is merely provided for backward compatibility
and it may be removed in future versions.

If the detection mechanism is to be used,
it is mandatory to correctly specify
the filename of the main file as the argument of |\childdocmain|:
%
\begin{center}
\begin{tabular}{l}
|\input{childdoc.def}|\\
|\childdocmain{|\textit{main}|}|\\
\end{tabular}
\end{center}
%
If |\jobname| does not match the argument \textit{main} of |\childdocmain|,
it is assumed that |\jobname| points to the child file to be compiled.
When using |\childdocmain| with the main file specified as argument,
it suffices to start a child file
with just |\input{|\textit{main}|}|
without loading of the package and using |\childdocof|.
If instead all processing is done
with the appropriate \textsf{childdoc} directives,
the argument of \textit{main} of |\childdocmain| can be empty.

An alternative version of the command line processing described
in \secref{sec:commandline} using the detection mechanism reads:
%
\begin{center}
|... -jobname "|\textit{target}|" "|[\textit{flags}]%
[|\def\jobname{|\textit{dest}|}|]|\input{|\textit{main}|}"|
\end{center}

%%%%%%%%%%%%%%%%%%%%%%%%%%%%%%%%%%%%%%%%%%%%%%%%%%%%%%%%%%%%%%%%%%%%%%%%%%%%%%%%
\subsection{Manual Code}
\label{sec:manual}

In case one cannot be certain whether the definitions file |childdoc.def|
is installed on the target \TeX{} distribution
and one prefers not to ship it,
it is conceivable to paste a few relevant commands into the sources.

To that end, drop all statements |\input{childdoc.def}|
and perform the replacements as outlined below.
Instead of |\childdocmain{|\textit{main}|}| add the following code
to the top of the main file:
%
\begin{center}
\begin{tabular}{l}
|\||ifdefined\childdocname\endinput\||fi\newif\ifchilddoc|\\
|\edef\childdocname{\scantokens\expandafter{\jobname\noexpand}}|\\
|\def\childdocmain{|\textit{main}|}\||ifx\childdocmain\childdocname\||else|\\
|\childdoctrue\includeonly{\childdocname}\let\jobname\childdocmain\||fi|\\
\end{tabular}
\end{center}
%
Instead of |\childdocof{|\textit{main}|}| just include the main file
at the top of each child file:
%
\begin{center}
|\input{|\textit{main}|}|
\end{center}
%
A simple redirection |\childdocforward{|\textit{dest}|}| is achieved by:
%
\begin{center}
|\def\jobname{|\textit{dest}|}\input{\jobname}|
\end{center}
%
The redirection with prefix
|\childdocforwardprefix[|\textit{prefix}|]{|\textit{dest}|}|
is accomplished by:
%
\begin{center}
\begin{tabular}{l}
|{\edef\jobname{\scantokens\expandafter{\jobname\noexpand}}|\\
|\def\redirectjob |\textit{prefix}|#1~~~{\gdef\jobname{|\textit{dest}|#1}}|\\
|\expandafter\redirectjob\jobname~~~}\input{\jobname}|
\end{tabular}
\end{center}

In an alternative approach,
child documents can be compiled by a specific command line
without additional code or specific definitions:
%
\begin{center}
|... -jobname "|\textit{target}|" "|[\textit{flags}]%
|\includeonly{|\textit{dest}|}\input{|\textit{main}|}"|
\end{center}
%

%%%%%%%%%%%%%%%%%%%%%%%%%%%%%%%%%%%%%%%%%%%%%%%%%%%%%%%%%%%%%%%%%%%%%%%%%%%%%%%%
%%%%%%%%%%%%%%%%%%%%%%%%%%%%%%%%%%%%%%%%%%%%%%%%%%%%%%%%%%%%%%%%%%%%%%%%%%%%%%%%
\section{Information}

%%%%%%%%%%%%%%%%%%%%%%%%%%%%%%%%%%%%%%%%%%%%%%%%%%%%%%%%%%%%%%%%%%%%%%%%%%%%%%%%
\subsection{Copyright}

Copyright \copyright{} 2017--2018 Niklas Beisert

This work may be distributed and/or modified under the
conditions of the \LaTeX{} Project Public License, either version 1.3
of this license or (at your option) any later version.
The latest version of this license is in
  \url{http://www.latex-project.org/lppl.txt}
and version 1.3 or later is part of all distributions of \LaTeX{}
version 2005/12/01 or later.

This work has the LPPL maintenance status `maintained'.

The Current Maintainer of this work is Niklas Beisert.

This work consists of the files |README.txt|, |childdoc.ins| and |childdoc.dtx|
as well as the derived files |childdoc.def|, |cdocsamp.tex|
with |cdocsch1.tex|, |cdocsch2.tex|, |cdocspt3.tex|, |cdocspt4.tex|,
|cdocsdrf.tex|, |cdocsfn1.tex|, |cdocsfn2.tex|
as well as |childdoc.pdf|.

%%%%%%%%%%%%%%%%%%%%%%%%%%%%%%%%%%%%%%%%%%%%%%%%%%%%%%%%%%%%%%%%%%%%%%%%%%%%%%%%
\subsection{Files and Installation}

The package consists of the files:
%
\begin{center}
\begin{tabular}{ll}
    |README.txt|   & readme file \\
    |childdoc.ins| & installation file \\
    |childdoc.dtx| & source file \\
    |childdoc.def| & definition file \\
    |cdocsamp.tex| & sample main file \\
    |cdocsch1.tex| & sample include file \\
    |cdocsch2.tex| & sample include file \\
    |cdocspt3.tex| & sample part file \\
    |cdocspt4.tex| & sample part file \\
    |cdocsdrf.tex| & sample redirection file \\
    |cdocsfn1.tex| & sample redirection file \\
    |cdocsfn2.tex| & sample redirection file \\
    |childdoc.pdf| & manual
\end{tabular}
\end{center}
%
The distribution consists of the files
|README.txt|, |childdoc.ins| and |childdoc.dtx|.
%
\begin{itemize}
\item
Run (pdf)\LaTeX{} on |childdoc.dtx|
to compile the manual |childdoc.pdf| (this file).
\item
Run \LaTeX{} on |childdoc.ins| to create the definitions file |childdoc.def|
and the sample |cdocsamp.tex| with include files
|cdocsch1.tex|, |cdocsch2.tex|, |cdocspt3.tex|, |cdocspt4.tex|,
|cdocsdrf.tex|, |cdocsfn1.tex|, |cdocsfn2.tex|.
Then copy the file |childdoc.def| to an appropriate directory of your \LaTeX{}
distribution, e.g.\ \textit{texmf-root}|/tex/latex/childdoc|.
\end{itemize}

%%%%%%%%%%%%%%%%%%%%%%%%%%%%%%%%%%%%%%%%%%%%%%%%%%%%%%%%%%%%%%%%%%%%%%%%%%%%%%%%
\subsection{Related CTAN Packages}

There are several other packages which offer a similar functionality:
%
\begin{itemize}
\item
The packages
\href{http://ctan.org/pkg/docmute}{\textsf{docmute}},
\href{http://ctan.org/pkg/includex}{\textsf{includex}} and
\href{http://ctan.org/pkg/standalone}{\textsf{standalone}}
provide commands to include only the document body of
a child file thus allowing both files to be compiled individually.
\item
The packages \href{http://ctan.org/pkg/subdocs}{\textsf{subdocs}}
and \href{http://ctan.org/pkg/subfiles}{\textsf{subfiles}}
provide structures in which the main and child documents can be
encapsulated and allowing them to be compiled individually.
The inclusion mechanism is different from the conventional |\include|.
\item
The package \href{http://ctan.org/pkg/combine}{\textsf{combine}}
is an elaborate solution to combine several documents into one.
\end{itemize}
%
See also the CTAN topic \href{http://ctan.org/topic/subdocs}{\textsf{subdocs}}
for further related packages.
The present package differs from the above solutions in that
a document structure constructed with the conventional |\include| mechanism
just needs two extra commands at the top of every file
such that all constituent files can be compiled individually.

%%%%%%%%%%%%%%%%%%%%%%%%%%%%%%%%%%%%%%%%%%%%%%%%%%%%%%%%%%%%%%%%%%%%%%%%%%%%%%%%
%\subsection{Feature Suggestions}
%
%The following is a list of features which may be useful for future
%versions of this package:
%%
%\begin{itemize}
%\item
%\ldots
%\end{itemize}

%%%%%%%%%%%%%%%%%%%%%%%%%%%%%%%%%%%%%%%%%%%%%%%%%%%%%%%%%%%%%%%%%%%%%%%%%%%%%%%%
\subsection{Revision History}

%%%%%%%%%%%%%%%%%%%%%%%%%%%%%%%%%%%%%%%%
\paragraph{v2.0:} 2018/12/30

\begin{itemize}
\item
immediate forward processing
\item
added |\childdocby| mechanism
\item
manual restructured
\end{itemize}

%%%%%%%%%%%%%%%%%%%%%%%%%%%%%%%%%%%%%%%%
\paragraph{v1.6:} 2018/01/17

\begin{itemize}
\item
application for development of include files
\item
corrections to manual
\end{itemize}

%%%%%%%%%%%%%%%%%%%%%%%%%%%%%%%%%%%%%%%%
\paragraph{v1.5:} 2017/05/21

\begin{itemize}
\item
more complete structuring introduced
\item
|\childdocof| introduced
\item
|\childdoc| renamed to |\childdocmain|
\item
|\childredirect| renamed to |\childdocforward| and |\childdocforwardprefix|
and functionality expanded
\end{itemize}

%%%%%%%%%%%%%%%%%%%%%%%%%%%%%%%%%%%%%%%%
\paragraph{v1.0:} 2017/04/27

\begin{itemize}
\item
manual and install package
\item
first version published on CTAN
\end{itemize}

%%%%%%%%%%%%%%%%%%%%%%%%%%%%%%%%%%%%%%%%
\paragraph{v0.6:} 2017/04/26

\begin{itemize}
\item
redirection mechanism added
\end{itemize}

%%%%%%%%%%%%%%%%%%%%%%%%%%%%%%%%%%%%%%%%
\paragraph{v0.5:} 2017/04/26

\begin{itemize}
\item
functionality in definition file
\end{itemize}


%%%%%%%%%%%%%%%%%%%%%%%%%%%%%%%%%%%%%%%%%%%%%%%%%%%%%%%%%%%%%%%%%%%%%%%%%%%%%%%%
%%%%%%%%%%%%%%%%%%%%%%%%%%%%%%%%%%%%%%%%%%%%%%%%%%%%%%%%%%%%%%%%%%%%%%%%%%%%%%%%
%%%%%%%%%%%%%%%%%%%%%%%%%%%%%%%%%%%%%%%%%%%%%%%%%%%%%%%%%%%%%%%%%%%%%%%%%%%%%%%%
\appendix

\settowidth\MacroIndent{\rmfamily\scriptsize 000\ }

 \DocInput{childdoc.dtx}

\end{document}
%</driver>
% \fi
%
% %%%%%%%%%%%%%%%%%%%%%%%%%%%%%%%%%%%%%%%%%%%%%%%%%%%%%%%%%%%%%%%%%%%%%%%%%%%%%%
% %%%%%%%%%%%%%%%%%%%%%%%%%%%%%%%%%%%%%%%%%%%%%%%%%%%%%%%%%%%%%%%%%%%%%%%%%%%%%%
% \section{Sample}
%\iffalse
%<*samplemain>
%\fi
%
% The following presents a sample document
% with two chapters, two parts, a title page,
% a compile flag as well as three forwarding files to set the flag.
% It consists of eight |.tex| files:
% \begin{center}
% \begin{tabular}{ll}
% |cdocsamp.tex|&main file\\
% |cdocsch1.tex|&include file for chapter 1\\
% |cdocsch2.tex|&include file for chapter 2\\
% |cdocspt3.tex|&include file for part 3\\
% |cdocspt4.tex|&include file for part 4\\
% |cdocsdrf.tex|&forwarding file for main file in draft mode\\
% |cdocsfi1.tex|&forwarding file for final version of chapter 1\\
% |cdocsfi2.tex|&forwarding file for final version of chapter 2\\
% \end{tabular}
% \end{center}
% Each of the eight files can be compiled directly by the \LaTeX{} compiler.
%
% %%%%%%%%%%%%%%%%%%%%%%%%%%%%%%%%%%%%%%
% \paragraph{Main File.}
%
% The main file is called |cdocsamp.tex|.
%
% Load the \textsf{childdoc} definitions and
% declare the filename for the main document:
%    \begin{macrocode}
\input{childdoc.def}
\childdocmain{}
%    \end{macrocode}

% Optional override for |\version| flag:
%    \begin{macrocode}
%%\ifchilddoc\else\providecommand{\version}{draft}\fi
%    \end{macrocode}

% Define the default values for the |\version| flag
% (|final| for the main file and |draft| for childs):
%    \begin{macrocode}
\ifchilddoc
\providecommand{\version}{draft}
\else
\providecommand{\version}{final}
\fi
%    \end{macrocode}

% Load the standard document class:
%    \begin{macrocode}
\documentclass[12pt]{article}
%    \end{macrocode}

% Start the document body:
%    \begin{macrocode}
\begin{document}
%    \end{macrocode}

% Declare a title page.
% Print title, part of document being processed and version flag:
%    \begin{macrocode}
\addtocounter{page}{-1}
\begin{center}
{\LARGE\bfseries{}childdoc example\par}
\vspace{1cm}
\ifchilddoc
\ifchilddocmanual part\else chapter\fi:
`\childdocname' of `\childdocjob'\par
\else
main document: `\childdocjob'\par
\fi
version: \version\par
\end{center}
\newpage
%    \end{macrocode}

% Manually include selected file,
% otherwise process as usual:
%    \begin{macrocode}
\ifchilddocmanual
\section*{part `\childdocname'}
\input{\childdocname}
\else
%    \end{macrocode}

% Include the two chapters:
%    \begin{macrocode}
\include{cdocsch1}
\include{cdocsch2}
%    \end{macrocode}

% Include the two parts unless only chapters should be displayed:
%    \begin{macrocode}
\ifchilddoc\else
\section{part three}
\input{cdocspt3}
\section{part four}
\input{cdocspt4}
\fi
%    \end{macrocode}

% Process as usual until here:
%    \begin{macrocode}
\fi
%    \end{macrocode}

% End of document body:
%    \begin{macrocode}
\end{document}
%    \end{macrocode}
%\iffalse
%</samplemain>
%\fi
%
% %%%%%%%%%%%%%%%%%%%%%%%%%%%%%%%%%%%%%%
% \paragraph{Chapter Include Files.}
%
% The include files are called |cdocsch1.tex| and |cdocsch2.tex|.
%
%\iffalse
%<*samplechap1|samplechap2>
%\fi

% Optional override for |\version| flag:
%    \begin{macrocode}
%%\providecommand{\version}{final}
%    \end{macrocode}

% Include the main document:
%    \begin{macrocode}
\input{childdoc.def}
\childdocof{cdocsamp}
%    \end{macrocode}

%\iffalse
%</samplechap1|samplechap2>
%\fi
%
%\iffalse
%<*samplechap1>
%\fi
% Some text for chapter 1:
%    \begin{macrocode}
\section{one}
some text in chapter one
%    \end{macrocode}

%\iffalse
%</samplechap1>
%\fi
% Some text for chapter 2:
%\iffalse
%<*samplechap2>
%\fi
%    \begin{macrocode}
\section{two}
more text in chapter two
%    \end{macrocode}

%\iffalse
%</samplechap2>
%\fi
%
% %%%%%%%%%%%%%%%%%%%%%%%%%%%%%%%%%%%%%%
% \paragraph{Part Include Files.}
%
% The include files are called |cdocspt3.tex| and |cdocspt4.tex|.
%
%\iffalse
%<*samplepart3|samplepart4>
%\fi

% Optional override for |\version| flag:
%    \begin{macrocode}
%%\providecommand{\version}{final}
%    \end{macrocode}

% Include the main document:
%    \begin{macrocode}
\input{childdoc.def}
\childdocby{cdocsamp}
%    \end{macrocode}

%\iffalse
%</samplepart3|samplepart4>
%\fi
%
%\iffalse
%<*samplepart3>
%\fi
% Some text for part 3:
%    \begin{macrocode}
some text in part three
%    \end{macrocode}

%\iffalse
%</samplepart3>
%\fi
% Some text for part 4:
%\iffalse
%<*samplepart4>
%\fi
%    \begin{macrocode}
more text in part four
%    \end{macrocode}

%\iffalse
%</samplepart4>
%\fi
%
% %%%%%%%%%%%%%%%%%%%%%%%%%%%%%%%%%%%%%%
% \paragraph{Forwarding for a Complete Draft.}
%
% The following forwarding file |cdocsdrf.tex|
% compiles the main document in draft mode:
%\iffalse
%<*sampledraft>
%\fi
%    \begin{macrocode}
\def\version{draft}
\input{childdoc.def}
\childdocforward{cdocsamp}
%    \end{macrocode}

%\iffalse
%</sampledraft>
%\fi
%
% %%%%%%%%%%%%%%%%%%%%%%%%%%%%%%%%%%%%%%
% \paragraph{Forwarding for Final Version of the Chapters.}
%
% The following forwarding files |cdocsfn1.tex| and |cdocsfn2.tex|
% (with identical content)
% compile the final versions of the child documents
% |cdocsch1.tex| and |cdocsch2.tex|, respectively:
%\iffalse
%<*samplefinal>
%\fi
%    \begin{macrocode}
\def\version{final}
\input{childdoc.def}
\childdocforwardprefix[cdocsamp]{cdocsfn}{cdocsch}
%    \end{macrocode}

%\iffalse
%</samplefinal>
%\fi
%
% %%%%%%%%%%%%%%%%%%%%%%%%%%%%%%%%%%%%%%
% \paragraph{Command Line Processing.}
%
% The following three command lines generate the output files
% |cdocscld|, |cdocscl1| and |cdocscl2|
% which should be identical to
% |cdocsdrf|, |cdocsch1| and |cdocsfn2|, respectively:
% \begin{center}
% \begin{tabular}{l}
% |latex -jobname cdocscld \|\\
% |  "\def\version{draft}\input{childdoc.def}\childdocforward{cdocsamp}"|\\
% |latex -jobname cdocscl1 \|\\
% |  "\input{childdoc.def}\childdocforward[cdocsamp]{cdocsch1}"|\\
% |latex -jobname cdocscl2 \|\\
% |  "\def\version{final}\input{childdoc.def}\childdocforward{cdocsch2}"|
% \end{tabular}
% \end{center}
% Note that the trailing backslash on each first line
% merely continues the input to the second line
% (for convenient cut ant paste).
% Furthermore, the command |latex| can be replaced by any
% of its alternative versions such as |pdflatex|.
%
% %%%%%%%%%%%%%%%%%%%%%%%%%%%%%%%%%%%%%%%%%%%%%%%%%%%%%%%%%%%%%%%%%%%%%%%%%%%%%%
% %%%%%%%%%%%%%%%%%%%%%%%%%%%%%%%%%%%%%%%%%%%%%%%%%%%%%%%%%%%%%%%%%%%%%%%%%%%%%%
% \section{Implementation}
%\iffalse
%<*package>
%\fi
%
% This section describes the definitions file |childdoc.def|.

% The definitions cannot be loaded using |\usepackage| or |\RequirePackage|
% which has a mechanism to prevent loading a style file more than once.
% When loading the definitions by means of |\input|
% multiple instances have to be prevented manually:
%\iffalse
%This code needs to be before the `\ProvidesFile' directive
%which is defined at the beginning of this file.
%Therefore it is also placed there and commented out here.
%</package>
%<*discard>
%\fi
%    \begin{macrocode}
\ifdefined\childdocmain\endinput\fi
%    \end{macrocode}
%\iffalse
%</discard>
%<*package>
%\fi
%
% \macro{\ifchilddoc}
% \macro{\ifchilddocmanual}
% The conditional |\ifchilddoc| tells whether a
% child (true) or main (false) document is being compiled.
% The conditional |\ifchilddocmanual| tells whether
% the |\includeonly| mechanism is used (false) or
% the selection of child files must be performed manually (true).
% The definitions initialise to false:
%    \begin{macrocode}
\newif\ifchilddoc
\newif\ifchilddocmanual
%    \end{macrocode}

% \macro{\childdocname}
% \macro{\childdocjob}
% The macro |\childdocname| stores the name of the main document
% to be compiled. The macro |\childdocjob| stores the name of
% the document on which the \LaTeX{} compiler was originally invoked.
% The content of |\jobname| cannot be compared
% to filenames specified in the source due to different catcodes.
% The following code rescans |\jobname|, stores the result
% in |\childdocname| and saves a copy in |\childdocjob|:
%    \begin{macrocode}
\edef\childdocname{\scantokens\expandafter{\jobname\noexpand}}
\let\childdocjob\childdocname
%    \end{macrocode}

% \macro{\childdocdisable}
% The macro |\childdocdisable| prevents the main file
% from being processed more than once.
% At this stage, the main document command |\childdocmain|
% is assumed to be called once again where it should do nothing.
% Any subsequent call to it should prevent
% a secondary processing of the main document
% It overwrites the forwarding commands
% |\childdocof| and |\childdocforward|
% with empty macros to prevent further inclusions of the main document:
%    \begin{macrocode}
\newcommand{\childdocdisable}
{
  \renewcommand{\childdocmain}[1]{\renewcommand{\childdocmain}[1]{\endinput}}
  \renewcommand{\childdocof}[1]{}
  \renewcommand{\childdocby}[2][]{}
  \renewcommand{\childdocforward}[2][]{}
  \renewcommand{\childdocdisable}{}
}
%    \end{macrocode}

% \macro{\childdocmain}
% The macro |\childdocmain| is to be called at the top of the main file
% with nothing or the main filename (without extension) as argument.
% First, it breaks loops.
% If the argument is not empty and does not match |\childdocname|
% (which is set by the first inclusion of |childdoc.def|),
% |\ifchilddoc| is set to true, |\includeonly| is applied to the child file
% and |\jobname| is set to the main file
% (for proper handling of |.aux| files):
%    \begin{macrocode}
\newcommand{\childdocmain}[1]
{
  \childdocdisable\childdocmain{}
  \if?#1?\else
    \begingroup
      \def\childdoctmp{#1}
      \ifx\childdoctmp\childdocname
        \def\childdoctmp{}
      \else
        \def\childdoctmp
        {
          \childdoctrue
          \includeonly{\childdocname}
          \def\childdocjob{#1}
          \def\jobname{#1}
        }
      \fi
      \expandafter
    \endgroup
    \childdoctmp
  \fi
}
%    \end{macrocode}

% \macro{\childdocof}
% The command |\childdocof| redirects
% compilation to the main file |#1|.
%    \begin{macrocode}
\newcommand{\childdocof}[1]
{
  \childdocdisable
  \childdoctrue
  \includeonly{\childdocname}
  \def\jobname{#1}
  \def\childdocjob{#1}
  \input{#1}
}
%    \end{macrocode}

% \macro{\childdocby}
% The command |\childdocby| ....
%    \begin{macrocode}
\newcommand{\childdocby}[2][]
{
  \childdocdisable
  \childdoctrue
  \childdocmanualtrue
  \if?#1?\else
    \def\jobname{#2}
  \fi
  \def\childdocjob{#2}
  \input{#2}
  \endinput
}
%    \end{macrocode}

% \macro{\childdocforward}
% The command |\childdocforward| redirects
% compilation to the main file or
% (if the optional argument is given) a child file.
% Parameters are set as if the main file
% or a child file starting with |\childdocof| was compiled.
% Then compilation is handed over to the main file:
%    \begin{macrocode}
\newcommand{\childdocforward}[2][]
{
  \begingroup
    \if?#1?
      \def\childdoctmp
      {
        \def\childdocname{#2}
        \def\childdocjob{#2}
        \def\jobname{#2}
        \input{#2}
        \endinput
      }
    \else
      \def\childdoctmp
      {
        \childdocdisable
        \def\childdocname{#2}
        \childdoctrue
        \includeonly{#2}
        \def\childdocjob{#1}
        \def\jobname{#1}
        \input{#1}
        \endinput
      }
    \fi
    \expandafter
  \endgroup
  \childdoctmp
}
%    \end{macrocode}

% \macro{\childdocforwardprefix}
% The command |\childdocforwardprefix| redirects
% compilation to the main or a child file by means of a pattern.
% The prefix |#1| in the current filename is replaced by |#2|
% and the suffix of the current filename is kept
% (it is assumed that the filename does not contain the substring `|~~~|'
% which is used as a delimiter).
% Compilation is handed over to the new file by |\childdocforward|:
%    \begin{macrocode}
\newcommand{\childdocforwardprefix}[3][]
{
  \begingroup
    \def\childdocextract #2##1~~~{\def\childdoctmp{\childdocforward[#1]{#3##1}}}
    \expandafter\childdocextract\childdocname~~~
    \expandafter
  \endgroup
  \childdoctmp
}
%    \end{macrocode}

% \macro{\childdoc}
% The deprecated macro |\childdoc| is a legacy version of |\childdocmain|:
%    \begin{macrocode}
\newcommand{\childdoc}{\childdocmain}
%    \end{macrocode}

% \macro{\childdocredirect}
% The deprecated macro |\childdocredirect| is a legacy version
% of |\childdocforward| and |\childdocforwardprefix|:
%    \begin{macrocode}
\newcommand{\childdocredirect}[2][]
{
  \begingroup
    \if?#1?
      \def\childdoctmp{\childdocforward{#2}}
    \else
      \def\childdoctmp{\childdocforwardprefix{#1}{#2}}
    \fi
    \expandafter
  \endgroup
  \childdoctmp
}
%    \end{macrocode}

%\iffalse
%</package>
%\fi
%
\endinput
\childdocforward{cdocsch2}"|
% \end{tabular}
% \end{center}
% Note that the trailing backslash on each first line
% merely continues the input to the second line
% (for convenient cut ant paste).
% Furthermore, the command |latex| can be replaced by any
% of its alternative versions such as |pdflatex|.
%
% %%%%%%%%%%%%%%%%%%%%%%%%%%%%%%%%%%%%%%%%%%%%%%%%%%%%%%%%%%%%%%%%%%%%%%%%%%%%%%
% %%%%%%%%%%%%%%%%%%%%%%%%%%%%%%%%%%%%%%%%%%%%%%%%%%%%%%%%%%%%%%%%%%%%%%%%%%%%%%
% \section{Implementation}
%\iffalse
%<*package>
%\fi
%
% This section describes the definitions file |childdoc.def|.

% The definitions cannot be loaded using |\usepackage| or |\RequirePackage|
% which has a mechanism to prevent loading a style file more than once.
% When loading the definitions by means of |\input|
% multiple instances have to be prevented manually:
%\iffalse
%This code needs to be before the `\ProvidesFile' directive
%which is defined at the beginning of this file.
%Therefore it is also placed there and commented out here.
%</package>
%<*discard>
%\fi
%    \begin{macrocode}
\ifdefined\childdocmain\endinput\fi
%    \end{macrocode}
%\iffalse
%</discard>
%<*package>
%\fi
%
% \macro{\ifchilddoc}
% \macro{\ifchilddocmanual}
% The conditional |\ifchilddoc| tells whether a
% child (true) or main (false) document is being compiled.
% The conditional |\ifchilddocmanual| tells whether
% the |\includeonly| mechanism is used (false) or
% the selection of child files must be performed manually (true).
% The definitions initialise to false:
%    \begin{macrocode}
\newif\ifchilddoc
\newif\ifchilddocmanual
%    \end{macrocode}

% \macro{\childdocname}
% \macro{\childdocjob}
% The macro |\childdocname| stores the name of the main document
% to be compiled. The macro |\childdocjob| stores the name of
% the document on which the \LaTeX{} compiler was originally invoked.
% The content of |\jobname| cannot be compared
% to filenames specified in the source due to different catcodes.
% The following code rescans |\jobname|, stores the result
% in |\childdocname| and saves a copy in |\childdocjob|:
%    \begin{macrocode}
\edef\childdocname{\scantokens\expandafter{\jobname\noexpand}}
\let\childdocjob\childdocname
%    \end{macrocode}

% \macro{\childdocdisable}
% The macro |\childdocdisable| prevents the main file
% from being processed more than once.
% At this stage, the main document command |\childdocmain|
% is assumed to be called once again where it should do nothing.
% Any subsequent call to it should prevent
% a secondary processing of the main document
% It overwrites the forwarding commands
% |\childdocof| and |\childdocforward|
% with empty macros to prevent further inclusions of the main document:
%    \begin{macrocode}
\newcommand{\childdocdisable}
{
  \renewcommand{\childdocmain}[1]{\renewcommand{\childdocmain}[1]{\endinput}}
  \renewcommand{\childdocof}[1]{}
  \renewcommand{\childdocby}[2][]{}
  \renewcommand{\childdocforward}[2][]{}
  \renewcommand{\childdocdisable}{}
}
%    \end{macrocode}

% \macro{\childdocmain}
% The macro |\childdocmain| is to be called at the top of the main file
% with nothing or the main filename (without extension) as argument.
% First, it breaks loops.
% If the argument is not empty and does not match |\childdocname|
% (which is set by the first inclusion of |childdoc.def|),
% |\ifchilddoc| is set to true, |\includeonly| is applied to the child file
% and |\jobname| is set to the main file
% (for proper handling of |.aux| files):
%    \begin{macrocode}
\newcommand{\childdocmain}[1]
{
  \childdocdisable\childdocmain{}
  \if?#1?\else
    \begingroup
      \def\childdoctmp{#1}
      \ifx\childdoctmp\childdocname
        \def\childdoctmp{}
      \else
        \def\childdoctmp
        {
          \childdoctrue
          \includeonly{\childdocname}
          \def\childdocjob{#1}
          \def\jobname{#1}
        }
      \fi
      \expandafter
    \endgroup
    \childdoctmp
  \fi
}
%    \end{macrocode}

% \macro{\childdocof}
% The command |\childdocof| redirects
% compilation to the main file |#1|.
%    \begin{macrocode}
\newcommand{\childdocof}[1]
{
  \childdocdisable
  \childdoctrue
  \includeonly{\childdocname}
  \def\jobname{#1}
  \def\childdocjob{#1}
  \input{#1}
}
%    \end{macrocode}

% \macro{\childdocby}
% The command |\childdocby| ....
%    \begin{macrocode}
\newcommand{\childdocby}[2][]
{
  \childdocdisable
  \childdoctrue
  \childdocmanualtrue
  \if?#1?\else
    \def\jobname{#2}
  \fi
  \def\childdocjob{#2}
  \input{#2}
  \endinput
}
%    \end{macrocode}

% \macro{\childdocforward}
% The command |\childdocforward| redirects
% compilation to the main file or
% (if the optional argument is given) a child file.
% Parameters are set as if the main file
% or a child file starting with |\childdocof| was compiled.
% Then compilation is handed over to the main file:
%    \begin{macrocode}
\newcommand{\childdocforward}[2][]
{
  \begingroup
    \if?#1?
      \def\childdoctmp
      {
        \def\childdocname{#2}
        \def\childdocjob{#2}
        \def\jobname{#2}
        \input{#2}
        \endinput
      }
    \else
      \def\childdoctmp
      {
        \childdocdisable
        \def\childdocname{#2}
        \childdoctrue
        \includeonly{#2}
        \def\childdocjob{#1}
        \def\jobname{#1}
        \input{#1}
        \endinput
      }
    \fi
    \expandafter
  \endgroup
  \childdoctmp
}
%    \end{macrocode}

% \macro{\childdocforwardprefix}
% The command |\childdocforwardprefix| redirects
% compilation to the main or a child file by means of a pattern.
% The prefix |#1| in the current filename is replaced by |#2|
% and the suffix of the current filename is kept
% (it is assumed that the filename does not contain the substring `|~~~|'
% which is used as a delimiter).
% Compilation is handed over to the new file by |\childdocforward|:
%    \begin{macrocode}
\newcommand{\childdocforwardprefix}[3][]
{
  \begingroup
    \def\childdocextract #2##1~~~{\def\childdoctmp{\childdocforward[#1]{#3##1}}}
    \expandafter\childdocextract\childdocname~~~
    \expandafter
  \endgroup
  \childdoctmp
}
%    \end{macrocode}

% \macro{\childdoc}
% The deprecated macro |\childdoc| is a legacy version of |\childdocmain|:
%    \begin{macrocode}
\newcommand{\childdoc}{\childdocmain}
%    \end{macrocode}

% \macro{\childdocredirect}
% The deprecated macro |\childdocredirect| is a legacy version
% of |\childdocforward| and |\childdocforwardprefix|:
%    \begin{macrocode}
\newcommand{\childdocredirect}[2][]
{
  \begingroup
    \if?#1?
      \def\childdoctmp{\childdocforward{#2}}
    \else
      \def\childdoctmp{\childdocforwardprefix{#1}{#2}}
    \fi
    \expandafter
  \endgroup
  \childdoctmp
}
%    \end{macrocode}

%\iffalse
%</package>
%\fi
%
\endinput
\childdocforward{cdocsch2}"|
% \end{tabular}
% \end{center}
% Note that the trailing backslash on each first line
% merely continues the input to the second line
% (for convenient cut ant paste).
% Furthermore, the command |latex| can be replaced by any
% of its alternative versions such as |pdflatex|.
%
% %%%%%%%%%%%%%%%%%%%%%%%%%%%%%%%%%%%%%%%%%%%%%%%%%%%%%%%%%%%%%%%%%%%%%%%%%%%%%%
% %%%%%%%%%%%%%%%%%%%%%%%%%%%%%%%%%%%%%%%%%%%%%%%%%%%%%%%%%%%%%%%%%%%%%%%%%%%%%%
% \section{Implementation}
%\iffalse
%<*package>
%\fi
%
% This section describes the definitions file |childdoc.def|.

% The definitions cannot be loaded using |\usepackage| or |\RequirePackage|
% which has a mechanism to prevent loading a style file more than once.
% When loading the definitions by means of |\input|
% multiple instances have to be prevented manually:
%\iffalse
%This code needs to be before the `\ProvidesFile' directive
%which is defined at the beginning of this file.
%Therefore it is also placed there and commented out here.
%</package>
%<*discard>
%\fi
%    \begin{macrocode}
\ifdefined\childdocmain\endinput\fi
%    \end{macrocode}
%\iffalse
%</discard>
%<*package>
%\fi
%
% \macro{\ifchilddoc}
% \macro{\ifchilddocmanual}
% The conditional |\ifchilddoc| tells whether a
% child (true) or main (false) document is being compiled.
% The conditional |\ifchilddocmanual| tells whether
% the |\includeonly| mechanism is used (false) or
% the selection of child files must be performed manually (true).
% The definitions initialise to false:
%    \begin{macrocode}
\newif\ifchilddoc
\newif\ifchilddocmanual
%    \end{macrocode}

% \macro{\childdocname}
% \macro{\childdocjob}
% The macro |\childdocname| stores the name of the main document
% to be compiled. The macro |\childdocjob| stores the name of
% the document on which the \LaTeX{} compiler was originally invoked.
% The content of |\jobname| cannot be compared
% to filenames specified in the source due to different catcodes.
% The following code rescans |\jobname|, stores the result
% in |\childdocname| and saves a copy in |\childdocjob|:
%    \begin{macrocode}
\edef\childdocname{\scantokens\expandafter{\jobname\noexpand}}
\let\childdocjob\childdocname
%    \end{macrocode}

% \macro{\childdocdisable}
% The macro |\childdocdisable| prevents the main file
% from being processed more than once.
% At this stage, the main document command |\childdocmain|
% is assumed to be called once again where it should do nothing.
% Any subsequent call to it should prevent
% a secondary processing of the main document
% It overwrites the forwarding commands
% |\childdocof| and |\childdocforward|
% with empty macros to prevent further inclusions of the main document:
%    \begin{macrocode}
\newcommand{\childdocdisable}
{
  \renewcommand{\childdocmain}[1]{\renewcommand{\childdocmain}[1]{\endinput}}
  \renewcommand{\childdocof}[1]{}
  \renewcommand{\childdocby}[2][]{}
  \renewcommand{\childdocforward}[2][]{}
  \renewcommand{\childdocdisable}{}
}
%    \end{macrocode}

% \macro{\childdocmain}
% The macro |\childdocmain| is to be called at the top of the main file
% with nothing or the main filename (without extension) as argument.
% First, it breaks loops.
% If the argument is not empty and does not match |\childdocname|
% (which is set by the first inclusion of |childdoc.def|),
% |\ifchilddoc| is set to true, |\includeonly| is applied to the child file
% and |\jobname| is set to the main file
% (for proper handling of |.aux| files):
%    \begin{macrocode}
\newcommand{\childdocmain}[1]
{
  \childdocdisable\childdocmain{}
  \if?#1?\else
    \begingroup
      \def\childdoctmp{#1}
      \ifx\childdoctmp\childdocname
        \def\childdoctmp{}
      \else
        \def\childdoctmp
        {
          \childdoctrue
          \includeonly{\childdocname}
          \def\childdocjob{#1}
          \def\jobname{#1}
        }
      \fi
      \expandafter
    \endgroup
    \childdoctmp
  \fi
}
%    \end{macrocode}

% \macro{\childdocof}
% The command |\childdocof| redirects
% compilation to the main file |#1|.
%    \begin{macrocode}
\newcommand{\childdocof}[1]
{
  \childdocdisable
  \childdoctrue
  \includeonly{\childdocname}
  \def\jobname{#1}
  \def\childdocjob{#1}
  \input{#1}
}
%    \end{macrocode}

% \macro{\childdocby}
% The command |\childdocby| ....
%    \begin{macrocode}
\newcommand{\childdocby}[2][]
{
  \childdocdisable
  \childdoctrue
  \childdocmanualtrue
  \if?#1?\else
    \def\jobname{#2}
  \fi
  \def\childdocjob{#2}
  \input{#2}
  \endinput
}
%    \end{macrocode}

% \macro{\childdocforward}
% The command |\childdocforward| redirects
% compilation to the main file or
% (if the optional argument is given) a child file.
% Parameters are set as if the main file
% or a child file starting with |\childdocof| was compiled.
% Then compilation is handed over to the main file:
%    \begin{macrocode}
\newcommand{\childdocforward}[2][]
{
  \begingroup
    \if?#1?
      \def\childdoctmp
      {
        \def\childdocname{#2}
        \def\childdocjob{#2}
        \def\jobname{#2}
        \input{#2}
        \endinput
      }
    \else
      \def\childdoctmp
      {
        \childdocdisable
        \def\childdocname{#2}
        \childdoctrue
        \includeonly{#2}
        \def\childdocjob{#1}
        \def\jobname{#1}
        \input{#1}
        \endinput
      }
    \fi
    \expandafter
  \endgroup
  \childdoctmp
}
%    \end{macrocode}

% \macro{\childdocforwardprefix}
% The command |\childdocforwardprefix| redirects
% compilation to the main or a child file by means of a pattern.
% The prefix |#1| in the current filename is replaced by |#2|
% and the suffix of the current filename is kept
% (it is assumed that the filename does not contain the substring `|~~~|'
% which is used as a delimiter).
% Compilation is handed over to the new file by |\childdocforward|:
%    \begin{macrocode}
\newcommand{\childdocforwardprefix}[3][]
{
  \begingroup
    \def\childdocextract #2##1~~~{\def\childdoctmp{\childdocforward[#1]{#3##1}}}
    \expandafter\childdocextract\childdocname~~~
    \expandafter
  \endgroup
  \childdoctmp
}
%    \end{macrocode}

% \macro{\childdoc}
% The deprecated macro |\childdoc| is a legacy version of |\childdocmain|:
%    \begin{macrocode}
\newcommand{\childdoc}{\childdocmain}
%    \end{macrocode}

% \macro{\childdocredirect}
% The deprecated macro |\childdocredirect| is a legacy version
% of |\childdocforward| and |\childdocforwardprefix|:
%    \begin{macrocode}
\newcommand{\childdocredirect}[2][]
{
  \begingroup
    \if?#1?
      \def\childdoctmp{\childdocforward{#2}}
    \else
      \def\childdoctmp{\childdocforwardprefix{#1}{#2}}
    \fi
    \expandafter
  \endgroup
  \childdoctmp
}
%    \end{macrocode}

%\iffalse
%</package>
%\fi
%
\endinput

\childdocof{exfserm}
%    \end{macrocode}
%\iffalse
%</samplemultisheet1|samplemultisheet2|samplemultisheet3|samplemultisheeta>
%<samplemultiprobleme|samplemultiproblemf>%%\providecommand{\printsol}{n}
%<samplemultiprobleme|samplemultiproblemf>% \iffalse
%
% childdoc.dtx Copyright (C) 2017-2018 Niklas Beisert
%
% This work may be distributed and/or modified under the
% conditions of the LaTeX Project Public License, either version 1.3
% of this license or (at your option) any later version.
% The latest version of this license is in
%   http://www.latex-project.org/lppl.txt
% and version 1.3 or later is part of all distributions of LaTeX
% version 2005/12/01 or later.
%
% This work has the LPPL maintenance status `maintained'.
%
% The Current Maintainer of this work is Niklas Beisert.
%
% This work consists of the files childdoc.dtx and childdoc.ins
% and the derived files childdoc.def and cdocsamp.tex with
% cdocsch1.tex, cdocsch2.tex, cdocsdrf.tex, cdocsfn1.tex, cdocsfn2.tex.
%
%<package>\ifdefined\childdocmain\endinput\fi
%<package>\ProvidesFile{childdoc.def}[2018/12/30 v2.0 child document driver]
%<samplemain>\ProvidesFile{cdocsamp.tex}[2018/12/30 v2.0 sample for childdoc]
%<*driver>
%\ProvidesFile{childdoc.drv}[2018/12/30 v2.0 childdoc reference manual file]
\PassOptionsToClass{10pt,a4paper}{article}
\documentclass{ltxdoc}

\usepackage[margin=35mm]{geometry}
\usepackage{hyperref}
\usepackage{hyperxmp}
\usepackage[usenames]{color}

\hypersetup{colorlinks=true}
\hypersetup{pdfstartview=FitH}
\hypersetup{pdfpagemode=UseNone}
\hypersetup{pdfsource={}}
\hypersetup{pdflang={en-UK}}
\hypersetup{pdfcopyright={Copyright 2017-2018 Niklas Beisert.
  This work may be distributed and/or modified under the
  conditions of the LaTeX Project Public License, either version 1.3
  of this license or (at your option) any later version.}}
\hypersetup{pdflicenseurl={http://www.latex-project.org/lppl.txt}}
\hypersetup{pdfcontactaddress={ETH Zurich, ITP, HIT K,
  Wolfgang-Pauli-Strasse 27}}
\hypersetup{pdfcontactpostcode={8093}}
\hypersetup{pdfcontactcity={Zurich}}
\hypersetup{pdfcontactcountry={Switzerland}}
\hypersetup{pdfcontactemail={nbeisert@itp.phys.ethz.ch}}
\hypersetup{pdfcontacturl={http://people.phys.ethz.ch/\xmptilde nbeisert/}}

\newcommand{\secref}[1]{\hyperref[#1]{section \ref*{#1}}}

\parskip1ex
\parindent0pt
\let\olditemize\itemize
\def\itemize{\olditemize\parskip0pt}

\begin{document}

\title{The \textsf{childdoc} Package}
\hypersetup{pdftitle={The childdoc Package}}
\author{Niklas Beisert\\[2ex]
  Institut f\"ur Theoretische Physik\\
  Eidgen\"ossische Technische Hochschule Z\"urich\\
  Wolfgang-Pauli-Strasse 27, 8093 Z\"urich, Switzerland\\[1ex]
  \href{mailto:nbeisert@itp.phys.ethz.ch}
  {\texttt{nbeisert@itp.phys.ethz.ch}}}
\hypersetup{pdfauthor={Niklas Beisert}}
\hypersetup{pdfsubject={Manual for the LaTeX2e Package childdoc}}
\date{30 December 2018, \textsf{v2.0}}
\maketitle

\begin{abstract}\noindent
\textsf{childdoc} is a \LaTeXe{} package
that enables the direct compilation
of document sections included by |\include|
to individual files.
\end{abstract}

\begingroup
\parskip0ex
\tableofcontents
\endgroup

%%%%%%%%%%%%%%%%%%%%%%%%%%%%%%%%%%%%%%%%%%%%%%%%%%%%%%%%%%%%%%%%%%%%%%%%%%%%%%%%
%%%%%%%%%%%%%%%%%%%%%%%%%%%%%%%%%%%%%%%%%%%%%%%%%%%%%%%%%%%%%%%%%%%%%%%%%%%%%%%%
\section{Introduction}

\LaTeX{} provides a mechanism to structure a large document (such as a book)
into a main file and several child files (containing the chapters)
using the |\include| command.
This mechanism is beneficial for documents
which span hundreds of pages in order to
make the source file(s) more manageable.
Moreover, compilation can be restricted to
selected child files by means of the |\includeonly| command.
The latter feature can be used to reduce the compilation time while editing
(this was significantly more useful in the earlier days of \LaTeX{})
or to generate a smaller document which is easier to navigate.
Another application of |\includeonly| is to generate
documents consisting of selected parts of the complete document.

However, there are a few drawbacks of the plain |\include| mechanism:
\begin{itemize}
\item
The child files cannot be compiled on their own,
they can only be compiled via the main file.
A naive editing environment
(such as a text editor with an option
to have the current file processed by \LaTeX)
may require one to switch to the main file before compiling;
attempting to compile the child file produces errors.
\item
The main file must be modified (each time)
to adjust the |\includeonly| command
to the present needs. This easily leaves the main file in a messy state.
\item
The generated document will always carry the filename
of the main document. This is inconvenient if
several child files are to be compiled and
to be kept for distribution.
\end{itemize}

The present package provides a simple interface
to make child files individually compilable by \LaTeX{}.
Compiling a child file then has the same effect as compiling
the main file with an |\includeonly| command
to select the appropriate child.
Moreover the generated document will carry the name of the child
rather than the main file.
This resolves all three above issues.

This feature is meant to make the editing of books,
thesis documents and lecture notes somewhat more convenient.
However, the package can also be used efficiently for
composing a series of documents (such as exercise sheets)
which are typically distributed individually.
It then assists the author in generating the individual documents
(potentially in different versions)
as well as a document containing the collected series.
Another application is in developing style files
or other kinds of included material
where compilation of the style file could redirect
to a sample or test file.

%%%%%%%%%%%%%%%%%%%%%%%%%%%%%%%%%%%%%%%%%%%%%%%%%%%%%%%%%%%%%%%%%%%%%%%%%%%%%%%%
%%%%%%%%%%%%%%%%%%%%%%%%%%%%%%%%%%%%%%%%%%%%%%%%%%%%%%%%%%%%%%%%%%%%%%%%%%%%%%%%
\section{Usage}

First of all, the package \textsf{childdoc} is \emph{not} a standard
\LaTeXe{} |.sty| style file! Therefore it needs to be invoked in
a non-standard way.

%%%%%%%%%%%%%%%%%%%%%%%%%%%%%%%%%%%%%%%%%%%%%%%%%%%%%%%%%%%%%%%%%%%%%%%%%%%%%%%%
\subsection{Included Files}
\label{sec:include}

%%%%%%%%%%%%%%%%%%%%%%%%%%%%%%%%%%%%%%%%
\DescribeMacro{\childdocmain}
To use the package, add the commands
\begin{center}
\begin{tabular}{l}
|% \iffalse
%
% childdoc.dtx Copyright (C) 2017-2018 Niklas Beisert
%
% This work may be distributed and/or modified under the
% conditions of the LaTeX Project Public License, either version 1.3
% of this license or (at your option) any later version.
% The latest version of this license is in
%   http://www.latex-project.org/lppl.txt
% and version 1.3 or later is part of all distributions of LaTeX
% version 2005/12/01 or later.
%
% This work has the LPPL maintenance status `maintained'.
%
% The Current Maintainer of this work is Niklas Beisert.
%
% This work consists of the files childdoc.dtx and childdoc.ins
% and the derived files childdoc.def and cdocsamp.tex with
% cdocsch1.tex, cdocsch2.tex, cdocsdrf.tex, cdocsfn1.tex, cdocsfn2.tex.
%
%<package>\ifdefined\childdocmain\endinput\fi
%<package>\ProvidesFile{childdoc.def}[2018/12/30 v2.0 child document driver]
%<samplemain>\ProvidesFile{cdocsamp.tex}[2018/12/30 v2.0 sample for childdoc]
%<*driver>
%\ProvidesFile{childdoc.drv}[2018/12/30 v2.0 childdoc reference manual file]
\PassOptionsToClass{10pt,a4paper}{article}
\documentclass{ltxdoc}

\usepackage[margin=35mm]{geometry}
\usepackage{hyperref}
\usepackage{hyperxmp}
\usepackage[usenames]{color}

\hypersetup{colorlinks=true}
\hypersetup{pdfstartview=FitH}
\hypersetup{pdfpagemode=UseNone}
\hypersetup{pdfsource={}}
\hypersetup{pdflang={en-UK}}
\hypersetup{pdfcopyright={Copyright 2017-2018 Niklas Beisert.
  This work may be distributed and/or modified under the
  conditions of the LaTeX Project Public License, either version 1.3
  of this license or (at your option) any later version.}}
\hypersetup{pdflicenseurl={http://www.latex-project.org/lppl.txt}}
\hypersetup{pdfcontactaddress={ETH Zurich, ITP, HIT K,
  Wolfgang-Pauli-Strasse 27}}
\hypersetup{pdfcontactpostcode={8093}}
\hypersetup{pdfcontactcity={Zurich}}
\hypersetup{pdfcontactcountry={Switzerland}}
\hypersetup{pdfcontactemail={nbeisert@itp.phys.ethz.ch}}
\hypersetup{pdfcontacturl={http://people.phys.ethz.ch/\xmptilde nbeisert/}}

\newcommand{\secref}[1]{\hyperref[#1]{section \ref*{#1}}}

\parskip1ex
\parindent0pt
\let\olditemize\itemize
\def\itemize{\olditemize\parskip0pt}

\begin{document}

\title{The \textsf{childdoc} Package}
\hypersetup{pdftitle={The childdoc Package}}
\author{Niklas Beisert\\[2ex]
  Institut f\"ur Theoretische Physik\\
  Eidgen\"ossische Technische Hochschule Z\"urich\\
  Wolfgang-Pauli-Strasse 27, 8093 Z\"urich, Switzerland\\[1ex]
  \href{mailto:nbeisert@itp.phys.ethz.ch}
  {\texttt{nbeisert@itp.phys.ethz.ch}}}
\hypersetup{pdfauthor={Niklas Beisert}}
\hypersetup{pdfsubject={Manual for the LaTeX2e Package childdoc}}
\date{30 December 2018, \textsf{v2.0}}
\maketitle

\begin{abstract}\noindent
\textsf{childdoc} is a \LaTeXe{} package
that enables the direct compilation
of document sections included by |\include|
to individual files.
\end{abstract}

\begingroup
\parskip0ex
\tableofcontents
\endgroup

%%%%%%%%%%%%%%%%%%%%%%%%%%%%%%%%%%%%%%%%%%%%%%%%%%%%%%%%%%%%%%%%%%%%%%%%%%%%%%%%
%%%%%%%%%%%%%%%%%%%%%%%%%%%%%%%%%%%%%%%%%%%%%%%%%%%%%%%%%%%%%%%%%%%%%%%%%%%%%%%%
\section{Introduction}

\LaTeX{} provides a mechanism to structure a large document (such as a book)
into a main file and several child files (containing the chapters)
using the |\include| command.
This mechanism is beneficial for documents
which span hundreds of pages in order to
make the source file(s) more manageable.
Moreover, compilation can be restricted to
selected child files by means of the |\includeonly| command.
The latter feature can be used to reduce the compilation time while editing
(this was significantly more useful in the earlier days of \LaTeX{})
or to generate a smaller document which is easier to navigate.
Another application of |\includeonly| is to generate
documents consisting of selected parts of the complete document.

However, there are a few drawbacks of the plain |\include| mechanism:
\begin{itemize}
\item
The child files cannot be compiled on their own,
they can only be compiled via the main file.
A naive editing environment
(such as a text editor with an option
to have the current file processed by \LaTeX)
may require one to switch to the main file before compiling;
attempting to compile the child file produces errors.
\item
The main file must be modified (each time)
to adjust the |\includeonly| command
to the present needs. This easily leaves the main file in a messy state.
\item
The generated document will always carry the filename
of the main document. This is inconvenient if
several child files are to be compiled and
to be kept for distribution.
\end{itemize}

The present package provides a simple interface
to make child files individually compilable by \LaTeX{}.
Compiling a child file then has the same effect as compiling
the main file with an |\includeonly| command
to select the appropriate child.
Moreover the generated document will carry the name of the child
rather than the main file.
This resolves all three above issues.

This feature is meant to make the editing of books,
thesis documents and lecture notes somewhat more convenient.
However, the package can also be used efficiently for
composing a series of documents (such as exercise sheets)
which are typically distributed individually.
It then assists the author in generating the individual documents
(potentially in different versions)
as well as a document containing the collected series.
Another application is in developing style files
or other kinds of included material
where compilation of the style file could redirect
to a sample or test file.

%%%%%%%%%%%%%%%%%%%%%%%%%%%%%%%%%%%%%%%%%%%%%%%%%%%%%%%%%%%%%%%%%%%%%%%%%%%%%%%%
%%%%%%%%%%%%%%%%%%%%%%%%%%%%%%%%%%%%%%%%%%%%%%%%%%%%%%%%%%%%%%%%%%%%%%%%%%%%%%%%
\section{Usage}

First of all, the package \textsf{childdoc} is \emph{not} a standard
\LaTeXe{} |.sty| style file! Therefore it needs to be invoked in
a non-standard way.

%%%%%%%%%%%%%%%%%%%%%%%%%%%%%%%%%%%%%%%%%%%%%%%%%%%%%%%%%%%%%%%%%%%%%%%%%%%%%%%%
\subsection{Included Files}
\label{sec:include}

%%%%%%%%%%%%%%%%%%%%%%%%%%%%%%%%%%%%%%%%
\DescribeMacro{\childdocmain}
To use the package, add the commands
\begin{center}
\begin{tabular}{l}
|% \iffalse
%
% childdoc.dtx Copyright (C) 2017-2018 Niklas Beisert
%
% This work may be distributed and/or modified under the
% conditions of the LaTeX Project Public License, either version 1.3
% of this license or (at your option) any later version.
% The latest version of this license is in
%   http://www.latex-project.org/lppl.txt
% and version 1.3 or later is part of all distributions of LaTeX
% version 2005/12/01 or later.
%
% This work has the LPPL maintenance status `maintained'.
%
% The Current Maintainer of this work is Niklas Beisert.
%
% This work consists of the files childdoc.dtx and childdoc.ins
% and the derived files childdoc.def and cdocsamp.tex with
% cdocsch1.tex, cdocsch2.tex, cdocsdrf.tex, cdocsfn1.tex, cdocsfn2.tex.
%
%<package>\ifdefined\childdocmain\endinput\fi
%<package>\ProvidesFile{childdoc.def}[2018/12/30 v2.0 child document driver]
%<samplemain>\ProvidesFile{cdocsamp.tex}[2018/12/30 v2.0 sample for childdoc]
%<*driver>
%\ProvidesFile{childdoc.drv}[2018/12/30 v2.0 childdoc reference manual file]
\PassOptionsToClass{10pt,a4paper}{article}
\documentclass{ltxdoc}

\usepackage[margin=35mm]{geometry}
\usepackage{hyperref}
\usepackage{hyperxmp}
\usepackage[usenames]{color}

\hypersetup{colorlinks=true}
\hypersetup{pdfstartview=FitH}
\hypersetup{pdfpagemode=UseNone}
\hypersetup{pdfsource={}}
\hypersetup{pdflang={en-UK}}
\hypersetup{pdfcopyright={Copyright 2017-2018 Niklas Beisert.
  This work may be distributed and/or modified under the
  conditions of the LaTeX Project Public License, either version 1.3
  of this license or (at your option) any later version.}}
\hypersetup{pdflicenseurl={http://www.latex-project.org/lppl.txt}}
\hypersetup{pdfcontactaddress={ETH Zurich, ITP, HIT K,
  Wolfgang-Pauli-Strasse 27}}
\hypersetup{pdfcontactpostcode={8093}}
\hypersetup{pdfcontactcity={Zurich}}
\hypersetup{pdfcontactcountry={Switzerland}}
\hypersetup{pdfcontactemail={nbeisert@itp.phys.ethz.ch}}
\hypersetup{pdfcontacturl={http://people.phys.ethz.ch/\xmptilde nbeisert/}}

\newcommand{\secref}[1]{\hyperref[#1]{section \ref*{#1}}}

\parskip1ex
\parindent0pt
\let\olditemize\itemize
\def\itemize{\olditemize\parskip0pt}

\begin{document}

\title{The \textsf{childdoc} Package}
\hypersetup{pdftitle={The childdoc Package}}
\author{Niklas Beisert\\[2ex]
  Institut f\"ur Theoretische Physik\\
  Eidgen\"ossische Technische Hochschule Z\"urich\\
  Wolfgang-Pauli-Strasse 27, 8093 Z\"urich, Switzerland\\[1ex]
  \href{mailto:nbeisert@itp.phys.ethz.ch}
  {\texttt{nbeisert@itp.phys.ethz.ch}}}
\hypersetup{pdfauthor={Niklas Beisert}}
\hypersetup{pdfsubject={Manual for the LaTeX2e Package childdoc}}
\date{30 December 2018, \textsf{v2.0}}
\maketitle

\begin{abstract}\noindent
\textsf{childdoc} is a \LaTeXe{} package
that enables the direct compilation
of document sections included by |\include|
to individual files.
\end{abstract}

\begingroup
\parskip0ex
\tableofcontents
\endgroup

%%%%%%%%%%%%%%%%%%%%%%%%%%%%%%%%%%%%%%%%%%%%%%%%%%%%%%%%%%%%%%%%%%%%%%%%%%%%%%%%
%%%%%%%%%%%%%%%%%%%%%%%%%%%%%%%%%%%%%%%%%%%%%%%%%%%%%%%%%%%%%%%%%%%%%%%%%%%%%%%%
\section{Introduction}

\LaTeX{} provides a mechanism to structure a large document (such as a book)
into a main file and several child files (containing the chapters)
using the |\include| command.
This mechanism is beneficial for documents
which span hundreds of pages in order to
make the source file(s) more manageable.
Moreover, compilation can be restricted to
selected child files by means of the |\includeonly| command.
The latter feature can be used to reduce the compilation time while editing
(this was significantly more useful in the earlier days of \LaTeX{})
or to generate a smaller document which is easier to navigate.
Another application of |\includeonly| is to generate
documents consisting of selected parts of the complete document.

However, there are a few drawbacks of the plain |\include| mechanism:
\begin{itemize}
\item
The child files cannot be compiled on their own,
they can only be compiled via the main file.
A naive editing environment
(such as a text editor with an option
to have the current file processed by \LaTeX)
may require one to switch to the main file before compiling;
attempting to compile the child file produces errors.
\item
The main file must be modified (each time)
to adjust the |\includeonly| command
to the present needs. This easily leaves the main file in a messy state.
\item
The generated document will always carry the filename
of the main document. This is inconvenient if
several child files are to be compiled and
to be kept for distribution.
\end{itemize}

The present package provides a simple interface
to make child files individually compilable by \LaTeX{}.
Compiling a child file then has the same effect as compiling
the main file with an |\includeonly| command
to select the appropriate child.
Moreover the generated document will carry the name of the child
rather than the main file.
This resolves all three above issues.

This feature is meant to make the editing of books,
thesis documents and lecture notes somewhat more convenient.
However, the package can also be used efficiently for
composing a series of documents (such as exercise sheets)
which are typically distributed individually.
It then assists the author in generating the individual documents
(potentially in different versions)
as well as a document containing the collected series.
Another application is in developing style files
or other kinds of included material
where compilation of the style file could redirect
to a sample or test file.

%%%%%%%%%%%%%%%%%%%%%%%%%%%%%%%%%%%%%%%%%%%%%%%%%%%%%%%%%%%%%%%%%%%%%%%%%%%%%%%%
%%%%%%%%%%%%%%%%%%%%%%%%%%%%%%%%%%%%%%%%%%%%%%%%%%%%%%%%%%%%%%%%%%%%%%%%%%%%%%%%
\section{Usage}

First of all, the package \textsf{childdoc} is \emph{not} a standard
\LaTeXe{} |.sty| style file! Therefore it needs to be invoked in
a non-standard way.

%%%%%%%%%%%%%%%%%%%%%%%%%%%%%%%%%%%%%%%%%%%%%%%%%%%%%%%%%%%%%%%%%%%%%%%%%%%%%%%%
\subsection{Included Files}
\label{sec:include}

%%%%%%%%%%%%%%%%%%%%%%%%%%%%%%%%%%%%%%%%
\DescribeMacro{\childdocmain}
To use the package, add the commands
\begin{center}
\begin{tabular}{l}
|\input{childdoc.def}|\\
|\childdocmain{}|\\
\end{tabular}
\end{center}
at the very top of the main \LaTeX{} file,
in particular \emph{before} the |\documentclass| statement!
The argument of |\childdocmain| should be left empty
(but it must be present).

%%%%%%%%%%%%%%%%%%%%%%%%%%%%%%%%%%%%%%%%
\DescribeMacro{\childdocof}
Furthermore, add the commands
\begin{center}
\begin{tabular}{l}
|\input{childdoc.def}|\\
|\childdocof{|\textit{main}|}|\\
\end{tabular}
\end{center}
at the top of every child file \textit{child}
which is included by |\include{|\textit{child}|}|
from within the main file
(or at least for those files to be compiled individually).
The argument \textit{main} must be the filename of the main file.

There are a couple of
considerations in setting up the main and child documents:

%%%%%%%%%%%%%%%%%%%%%%%%%%%%%%%%%%%%%%%%
\paragraph{Restrictions.}

Please note the following restrictions:
\begin{itemize}
\item
|\childdocmain| must be called with one argument \textit{main}
to ensure compatibility with earlier version of the package.
It must either be empty (|\childdocmain{}|)
or precisely match the filename of the main file in which it is specified.
See \secref{sec:detection} for further information.
\item
The filename \textit{main} must be specified without the |.tex| extension.
\item
The filename \textit{main} is case sensitive
(even in case-insensitive file systems)
due to internal string comparison.
\item
The argument \textit{main} should be fully expanded, it cannot be a macro.
\item
Subdirectories and special characters should be avoided in filenames.
\item
The command |\childdocmain{|\textit{main}|}| must be followed by a whitespace.
It should not be followed immediately by another command
or by a comment mark `|%|'.
This is because the \TeX{} parser reads the token immediately following
the argument of |\childdocmain| and puts it
at the beginning of every child section;
however, a white\-space is ignored.
\end{itemize}

%%%%%%%%%%%%%%%%%%%%%%%%%%%%%%%%%%%%%%%%
\paragraph{Content of Main File.}

It is advisable to place all content in the child files included by |\include|.
Any output contained in the main file will appear in all child documents
unless suppressed manually;
it cannot be suppressed automatically by the |\includeonly| directive
and thus should normally be avoided.
A method to include some content in the main file
by means of conditional processing is described in \secref{sec:conditional}.

%%%%%%%%%%%%%%%%%%%%%%%%%%%%%%%%%%%%%%%%
\paragraph{Page Numbering.}

When only a part of the document is compiled,
the appropriate numbering of pages
(as well as other status parameters)
is determined from the |.aux| files.
The latter contain information from previous passes.
However this information needs to propagate through
all intermediate child documents.
Therefore the page numbering in child documents may well
be inconsistent until the complete document is compiled at least once.

A useful (if unconventional) way to always ensure a consistent
page numbering is to restart the numbering in each child document
and denote the pages by `\textit{child}|.|\textit{page}'
where \textit{child} represents the chapter/section number of the child file.
This can be achieved by the command
|\numberwithin{page}{|\textit{child}|}|
of the \textsf{amsmath} package
where \textit{child} can be |chapter| or |section|
depending on the chosen structuring.
Alternatively, one can modify the macro |\thepage| appropriately
and reset the counter |page| at the start of each child file.

%%%%%%%%%%%%%%%%%%%%%%%%%%%%%%%%%%%%%%%%%%%%%%%%%%%%%%%%%%%%%%%%%%%%%%%%%%%%%%%%
\subsection{Conditional Processing}
\label{sec:conditional}

The package provides a mechanism to compile different versions
of a document. To customise the versions further some conditional processing
can come in handy to distinguish which version is being compiled.
The package provides two macros to describe the compilation context:

%%%%%%%%%%%%%%%%%%%%%%%%%%%%%%%%%%%%%%%%
\DescribeMacro{\ifchilddoc}
The conditional |\ifchilddoc| distinguishes between the compilation of
child documents and the main document:
%
\begin{center}
|\ifchilddoc |\textit{child-code}| |[|\||else |\textit{main-code}]| \||fi|
\end{center}

%%%%%%%%%%%%%%%%%%%%%%%%%%%%%%%%%%%%%%%%
\DescribeMacro{\childdocname}
\DescribeMacro{\childdocjob}
The macro |\childdocname| contains the filename (without extension)
of the main or child file being processed.
Note that |\childdocjob| will always contain the name of the main file.

%%%%%%%%%%%%%%%%%%%%%%%%%%%%%%%%%%%%%%%%
\paragraph{Title Page.}

Conditional processing can be used to include a title or banner page
in the main document when proper precautions are taken.
Importantly, the code in the main file should ensure that the page counter
(as well as other status parameters which are stored in the |.aux| files)
takes the same value after the conditional processing.
Otherwise the page numbers may take divergent values
depending on which part is compiled.

For example, a title page could be declared by:
%
\begin{center}
\begin{tabular}{l}
|\ifchilddoc\||else|\\
|\addtocounter{page}{-1}|\\
\textit{code for title page}\\
|\newpage|\\
|\||fi|
\end{tabular}
\end{center}
%
A banner page for the child documents can be generated by:
%
\begin{center}
\begin{tabular}{l}
|\ifchilddoc|\\
|\addtocounter{page}{-1}|\\
\textit{code for banner page}\\
|\newpage|\\
|\||fi|
\end{tabular}
\end{center}
%
Here one could write a message such as:
\begin{center}
|This is the part \childdocname{} of \childdocjob{}.|
\end{center}

%%%%%%%%%%%%%%%%%%%%%%%%%%%%%%%%%%%%%%%%%%%%%%%%%%%%%%%%%%%%%%%%%%%%%%%%%%%%%%%%
\subsection{Flags}
\label{sec:flags}

The package makes it easy to generate different versions
of the main or child documents.
To this end compilation flags can be defined
and assigned different default values.
They will be particularly useful in conjunction
with the forwarding mechanism described in \secref{sec:forward}.

For example, it may be useful to have a flag |\version|
which can be set to |draft| or |final|.
The document source will contain some conditional code
depending on the value of |\version|.
Suppose further, the flag should default to |final| for the main file
and to |draft| for child files
which is a natural assignment for editing the document.
This is achieved by placing the following code
in the preamble of the main document
(below the |\childdocmain| directive):
%
\begin{center}
\begin{tabular}{l}
|\ifchilddoc|\\
|\providecommand{\version}{draft}|\\
|\||else|\\
|\providecommand{\version}{final}|\\
|\||fi|
\end{tabular}
\end{center}
%
The definition by |\providecommand| makes sure
that previous definitions are not overwritten.
Further statements |\providecommand{\version}{...}|
can thus be added before the above code to override it.

For the main file, one might add a line
(between |\childdocmain| and the above block)
%
\begin{center}
|%\ifchilddoc\||else\providecommand{\version}{draft}\||fi|
\end{center}
%
which can be uncommented to produce a draft version.
Likewise one can add a line to the very top of a child file
(above the |\childdocof{|\textit{main}|}| directive)
%
\begin{center}
|%\providecommand{\version}{final}|
\end{center}
%
which can be uncommented to produce the final version of this child document.

%%%%%%%%%%%%%%%%%%%%%%%%%%%%%%%%%%%%%%%%%%%%%%%%%%%%%%%%%%%%%%%%%%%%%%%%%%%%%%%%
\subsection{Forwarding}
\label{sec:forward}

Different versions of the main or child documents
using compilation flags as described in \secref{sec:flags}
can be (permanently) stored in different files
for convenient compilation, viewing and distribution.
To this end, the package defines a command
to pass on compilation to a different file:

%%%%%%%%%%%%%%%%%%%%%%%%%%%%%%%%%%%%%%%%
\DescribeMacro{\childdocforward}
The command |\childdocforward| redirects processing to
another source file:
%
\begin{center}
\begin{tabular}{l}
|\input{childdoc.def}|\\
|\childdocforward[|\textit{main}|]{|\textit{dest}|}|\\
\end{tabular}
\end{center}
%
The argument \textit{dest} is the destination file
(without extension).
It should be the main file or one of the child files.
Note that further \textsf{childdoc} directives
such as |\childdocof| and |\childdocforward|
in the indicated file will be processed in this form.
The optional argument \textit{main}
passes on directly to the main file \textit{main}
while pretending to compile the child \textit{dest}.
This form behaves as if \textit{dest}
issues |\childdocof{|\textit{main}|}| right away,
and no further \textsf{childdoc} directives will be processed.

%%%%%%%%%%%%%%%%%%%%%%%%%%%%%%%%%%%%%%%%
\DescribeMacro{\...prefix}
In the alternative form |\childdocforwardprefix|,
%
\begin{center}
\begin{tabular}{l}
|\input{childdoc.def}|\\
|\childdocforwardprefix[|\textit{main}|]{|\textit{prefix}|}{|\textit{dest}|}|
\end{tabular}
\end{center}
%
the destination file is determined by a pattern
depending on the current file:
To make this work, the current file must be called
`{\textit{prefix}\hspace{0.2em}\textit{suffix}}'
with \textit{prefix} matching precisely the argument.
Processing is then passed on to the file
`{\textit{dest}\hspace{0.2em}\textit{suffix}}'.
Surely, the same effect is achieved by
directly specifying the
argument `{\textit{dest}\hspace{0.2em}\textit{suffix}}'
in the first form.
However, that requires to set up a different file
for each child. With the alternative form of the command
all these files can have exactly the same content
which simplifies setting them up and maintaining them.

For example, the following file |draft.tex|
with a compilation flag |\version| as described in \secref{sec:flags}
compiles the main document as a draft:
%
\begin{center}
\begin{tabular}{l}
|\def\version{draft}|\\
|\input{childdoc.def}|\\
|\childdocforward{|\textit{main}|}|
\end{tabular}
\end{center}
%
Likewise, the following files |final|\textit{nn}|.tex|
compile the final version of the child document
|child|\textit{nn}|.tex|:
%
\begin{center}
\begin{tabular}{l}
|\def\version{final}|\\
|\input{childdoc.def}|\\
|\childdocforwardprefix{final}{child}|
\end{tabular}
\end{center}
%

Note that when several versions of a main file and/or of each child file
are to be generated, it may be convenient to set up a |Makefile| or
shell script to automatise the process.

%%%%%%%%%%%%%%%%%%%%%%%%%%%%%%%%%%%%%%%%%%%%%%%%%%%%%%%%%%%%%%%%%%%%%%%%%%%%%%%%
\subsection{Command Line Processing}
\label{sec:commandline}

The effect of redirection files can also be achieved by invoking
the \LaTeX{} compiler with a more elaborate command line.
Most conveniently this should be done as part
of a shell script or a |Makefile|.

When using \textsf{childdoc} in the main file, the following
command lines effectively perform a redirection
(note that depending on the shell being used,
backslashes may have to be doubled: `|\|' $\to$ `|\\|'):
%
\begin{center}
|... -jobname "|\textit{target}|" |\\|"|[\textit{flags}]%
|\input{childdoc.def}\childdocforward[|\textit{main}|]{|\textit{dest}|}"|
\end{center}
%
Here \textit{target} is the name of the output file,
\textit{main} is the name of the main file
and \textit{dest} is the name of the main or child file to be processed
(all filenames without extensions).
The optional argument \textit{main} can be omitted
if \textit{main} matches \textit{dest}.
Optionally, compilation \textit{flags} can be defined via |\def| commands.
This command line makes the \TeX{} engine believe
it is compiling the file \textit{target}
whose content is specified as the latter parameter.
The provided code then forwards the processing to
\textit{main} or \textit{dest} as described in \secref{sec:forward}.

%%%%%%%%%%%%%%%%%%%%%%%%%%%%%%%%%%%%%%%%%%%%%%%%%%%%%%%%%%%%%%%%%%%%%%%%%%%%%%%%
\subsection{Include by Input}
\label{sec:input}

Including child documents by |\include| has some restrictions by design.
Most notably, the content of a child document always occupies
its own set of pages; pages cannot be shared between child documents.
Usually, this behaviour makes perfect sense
because each child document contain an essential part of the document.
However, in some situations it may be desirable to compose
a document from a collection of parts
without having mandatory page breaks between then.
For this case, the package
provides a mechanism to include parts
by |\input| which can also be processed individually.
However, by construction this mechanism
requires manual handling of the content to be output.

%%%%%%%%%%%%%%%%%%%%%%%%%%%%%%%%%%%%%%%%
\DescribeMacro{\ifchilddocmanual}
The main file should be prepared as usual, see \secref{sec:include}.
However, the document body must make a distinction
between processing of an individual part and of the main document, e.g.:
%
\begin{center}
\begin{tabular}{l}
|\ifchilddocmanual|\\
|\input{\childdocname}|\\
|\||else|\\
\textit{document body with }|\input{|\textit{part}|}|\\
|\||fi|
\end{tabular}
\end{center}
%
The conditional |\ifchilddocmanual| is true whenever
a part to be included by |\input| is being compiled,
and the name of the part is stored in |\childdocname|.

%%%%%%%%%%%%%%%%%%%%%%%%%%%%%%%%%%%%%%%%
\DescribeMacro{\childdocby}
Each part to be included by |\input| should start with:
%
\begin{center}
\begin{tabular}{l}
|\input{childdoc.def}|\\
|\childdocby{|\textit{main}|}|\\
\end{tabular}
\end{center}
%
The directive |\childdocby| is similar to |\childdocof|
described in \secref{sec:include},
but the subsequent selection of content must be done manually.
To that end, both |\ifchilddoc| and |\ifchilddocmanual|
will be true upon processing of a part,
and the name of the part is stored in |\childdocname|.
Note that |\jobname| will be set to the filename of the current part
so that each part receives an individual |.aux| file
that does not interfere with the |.aux| file(s) of the main document.
This behaviour can be altered by the alternative form
|\childdocby[*]{|\textit{main}|}| (with a non-empty optional argument)
which uses the |.aux| file of the main document
by setting |\jobname| to \textit{main}.

%%%%%%%%%%%%%%%%%%%%%%%%%%%%%%%%%%%%%%%%%%%%%%%%%%%%%%%%%%%%%%%%%%%%%%%%%%%%%%%%
\subsection{Driver Development}
\label{sec:driver}

The \textsf{childdoc} mechanism can also be use for the development
of definition files such as \LaTeX{} styles or classes.
This case differs from the above setup with multiple parts
included by |\include| in that no |\includeonly| should be invoked.
This can be achieved by starting the include file
(before |\ProvidesPackage|) with:
%
\begin{center}
\begin{tabular}{l}
|\input{childdoc.def}|\\
|\childdocforward{|\textit{main}|}|\\
\end{tabular}
\end{center}
%
or alternatively with:
%
\begin{center}
\begin{tabular}{l}
|\input{childdoc.def}|\\
|\childdocby{|\textit{main}|}|\\
\end{tabular}
\end{center}
%
Both forms have slightly different effects as described above.
The main file is prepared as usual, see \secref{sec:include}.

%%%%%%%%%%%%%%%%%%%%%%%%%%%%%%%%%%%%%%%%%%%%%%%%%%%%%%%%%%%%%%%%%%%%%%%%%%%%%%%%
\subsection{Legacy Detection}
\label{sec:detection}

The directive |\childdocmain| in the main file can detect
whether the complete document or merely a child is to be compiled
even without using the directive |\childdocof|.
This method is deprecated because it is less robust
and there is no compelling reason to use it;
it is merely provided for backward compatibility
and it may be removed in future versions.

If the detection mechanism is to be used,
it is mandatory to correctly specify
the filename of the main file as the argument of |\childdocmain|:
%
\begin{center}
\begin{tabular}{l}
|\input{childdoc.def}|\\
|\childdocmain{|\textit{main}|}|\\
\end{tabular}
\end{center}
%
If |\jobname| does not match the argument \textit{main} of |\childdocmain|,
it is assumed that |\jobname| points to the child file to be compiled.
When using |\childdocmain| with the main file specified as argument,
it suffices to start a child file
with just |\input{|\textit{main}|}|
without loading of the package and using |\childdocof|.
If instead all processing is done
with the appropriate \textsf{childdoc} directives,
the argument of \textit{main} of |\childdocmain| can be empty.

An alternative version of the command line processing described
in \secref{sec:commandline} using the detection mechanism reads:
%
\begin{center}
|... -jobname "|\textit{target}|" "|[\textit{flags}]%
[|\def\jobname{|\textit{dest}|}|]|\input{|\textit{main}|}"|
\end{center}

%%%%%%%%%%%%%%%%%%%%%%%%%%%%%%%%%%%%%%%%%%%%%%%%%%%%%%%%%%%%%%%%%%%%%%%%%%%%%%%%
\subsection{Manual Code}
\label{sec:manual}

In case one cannot be certain whether the definitions file |childdoc.def|
is installed on the target \TeX{} distribution
and one prefers not to ship it,
it is conceivable to paste a few relevant commands into the sources.

To that end, drop all statements |\input{childdoc.def}|
and perform the replacements as outlined below.
Instead of |\childdocmain{|\textit{main}|}| add the following code
to the top of the main file:
%
\begin{center}
\begin{tabular}{l}
|\||ifdefined\childdocname\endinput\||fi\newif\ifchilddoc|\\
|\edef\childdocname{\scantokens\expandafter{\jobname\noexpand}}|\\
|\def\childdocmain{|\textit{main}|}\||ifx\childdocmain\childdocname\||else|\\
|\childdoctrue\includeonly{\childdocname}\let\jobname\childdocmain\||fi|\\
\end{tabular}
\end{center}
%
Instead of |\childdocof{|\textit{main}|}| just include the main file
at the top of each child file:
%
\begin{center}
|\input{|\textit{main}|}|
\end{center}
%
A simple redirection |\childdocforward{|\textit{dest}|}| is achieved by:
%
\begin{center}
|\def\jobname{|\textit{dest}|}\input{\jobname}|
\end{center}
%
The redirection with prefix
|\childdocforwardprefix[|\textit{prefix}|]{|\textit{dest}|}|
is accomplished by:
%
\begin{center}
\begin{tabular}{l}
|{\edef\jobname{\scantokens\expandafter{\jobname\noexpand}}|\\
|\def\redirectjob |\textit{prefix}|#1~~~{\gdef\jobname{|\textit{dest}|#1}}|\\
|\expandafter\redirectjob\jobname~~~}\input{\jobname}|
\end{tabular}
\end{center}

In an alternative approach,
child documents can be compiled by a specific command line
without additional code or specific definitions:
%
\begin{center}
|... -jobname "|\textit{target}|" "|[\textit{flags}]%
|\includeonly{|\textit{dest}|}\input{|\textit{main}|}"|
\end{center}
%

%%%%%%%%%%%%%%%%%%%%%%%%%%%%%%%%%%%%%%%%%%%%%%%%%%%%%%%%%%%%%%%%%%%%%%%%%%%%%%%%
%%%%%%%%%%%%%%%%%%%%%%%%%%%%%%%%%%%%%%%%%%%%%%%%%%%%%%%%%%%%%%%%%%%%%%%%%%%%%%%%
\section{Information}

%%%%%%%%%%%%%%%%%%%%%%%%%%%%%%%%%%%%%%%%%%%%%%%%%%%%%%%%%%%%%%%%%%%%%%%%%%%%%%%%
\subsection{Copyright}

Copyright \copyright{} 2017--2018 Niklas Beisert

This work may be distributed and/or modified under the
conditions of the \LaTeX{} Project Public License, either version 1.3
of this license or (at your option) any later version.
The latest version of this license is in
  \url{http://www.latex-project.org/lppl.txt}
and version 1.3 or later is part of all distributions of \LaTeX{}
version 2005/12/01 or later.

This work has the LPPL maintenance status `maintained'.

The Current Maintainer of this work is Niklas Beisert.

This work consists of the files |README.txt|, |childdoc.ins| and |childdoc.dtx|
as well as the derived files |childdoc.def|, |cdocsamp.tex|
with |cdocsch1.tex|, |cdocsch2.tex|, |cdocspt3.tex|, |cdocspt4.tex|,
|cdocsdrf.tex|, |cdocsfn1.tex|, |cdocsfn2.tex|
as well as |childdoc.pdf|.

%%%%%%%%%%%%%%%%%%%%%%%%%%%%%%%%%%%%%%%%%%%%%%%%%%%%%%%%%%%%%%%%%%%%%%%%%%%%%%%%
\subsection{Files and Installation}

The package consists of the files:
%
\begin{center}
\begin{tabular}{ll}
    |README.txt|   & readme file \\
    |childdoc.ins| & installation file \\
    |childdoc.dtx| & source file \\
    |childdoc.def| & definition file \\
    |cdocsamp.tex| & sample main file \\
    |cdocsch1.tex| & sample include file \\
    |cdocsch2.tex| & sample include file \\
    |cdocspt3.tex| & sample part file \\
    |cdocspt4.tex| & sample part file \\
    |cdocsdrf.tex| & sample redirection file \\
    |cdocsfn1.tex| & sample redirection file \\
    |cdocsfn2.tex| & sample redirection file \\
    |childdoc.pdf| & manual
\end{tabular}
\end{center}
%
The distribution consists of the files
|README.txt|, |childdoc.ins| and |childdoc.dtx|.
%
\begin{itemize}
\item
Run (pdf)\LaTeX{} on |childdoc.dtx|
to compile the manual |childdoc.pdf| (this file).
\item
Run \LaTeX{} on |childdoc.ins| to create the definitions file |childdoc.def|
and the sample |cdocsamp.tex| with include files
|cdocsch1.tex|, |cdocsch2.tex|, |cdocspt3.tex|, |cdocspt4.tex|,
|cdocsdrf.tex|, |cdocsfn1.tex|, |cdocsfn2.tex|.
Then copy the file |childdoc.def| to an appropriate directory of your \LaTeX{}
distribution, e.g.\ \textit{texmf-root}|/tex/latex/childdoc|.
\end{itemize}

%%%%%%%%%%%%%%%%%%%%%%%%%%%%%%%%%%%%%%%%%%%%%%%%%%%%%%%%%%%%%%%%%%%%%%%%%%%%%%%%
\subsection{Related CTAN Packages}

There are several other packages which offer a similar functionality:
%
\begin{itemize}
\item
The packages
\href{http://ctan.org/pkg/docmute}{\textsf{docmute}},
\href{http://ctan.org/pkg/includex}{\textsf{includex}} and
\href{http://ctan.org/pkg/standalone}{\textsf{standalone}}
provide commands to include only the document body of
a child file thus allowing both files to be compiled individually.
\item
The packages \href{http://ctan.org/pkg/subdocs}{\textsf{subdocs}}
and \href{http://ctan.org/pkg/subfiles}{\textsf{subfiles}}
provide structures in which the main and child documents can be
encapsulated and allowing them to be compiled individually.
The inclusion mechanism is different from the conventional |\include|.
\item
The package \href{http://ctan.org/pkg/combine}{\textsf{combine}}
is an elaborate solution to combine several documents into one.
\end{itemize}
%
See also the CTAN topic \href{http://ctan.org/topic/subdocs}{\textsf{subdocs}}
for further related packages.
The present package differs from the above solutions in that
a document structure constructed with the conventional |\include| mechanism
just needs two extra commands at the top of every file
such that all constituent files can be compiled individually.

%%%%%%%%%%%%%%%%%%%%%%%%%%%%%%%%%%%%%%%%%%%%%%%%%%%%%%%%%%%%%%%%%%%%%%%%%%%%%%%%
%\subsection{Feature Suggestions}
%
%The following is a list of features which may be useful for future
%versions of this package:
%%
%\begin{itemize}
%\item
%\ldots
%\end{itemize}

%%%%%%%%%%%%%%%%%%%%%%%%%%%%%%%%%%%%%%%%%%%%%%%%%%%%%%%%%%%%%%%%%%%%%%%%%%%%%%%%
\subsection{Revision History}

%%%%%%%%%%%%%%%%%%%%%%%%%%%%%%%%%%%%%%%%
\paragraph{v2.0:} 2018/12/30

\begin{itemize}
\item
immediate forward processing
\item
added |\childdocby| mechanism
\item
manual restructured
\end{itemize}

%%%%%%%%%%%%%%%%%%%%%%%%%%%%%%%%%%%%%%%%
\paragraph{v1.6:} 2018/01/17

\begin{itemize}
\item
application for development of include files
\item
corrections to manual
\end{itemize}

%%%%%%%%%%%%%%%%%%%%%%%%%%%%%%%%%%%%%%%%
\paragraph{v1.5:} 2017/05/21

\begin{itemize}
\item
more complete structuring introduced
\item
|\childdocof| introduced
\item
|\childdoc| renamed to |\childdocmain|
\item
|\childredirect| renamed to |\childdocforward| and |\childdocforwardprefix|
and functionality expanded
\end{itemize}

%%%%%%%%%%%%%%%%%%%%%%%%%%%%%%%%%%%%%%%%
\paragraph{v1.0:} 2017/04/27

\begin{itemize}
\item
manual and install package
\item
first version published on CTAN
\end{itemize}

%%%%%%%%%%%%%%%%%%%%%%%%%%%%%%%%%%%%%%%%
\paragraph{v0.6:} 2017/04/26

\begin{itemize}
\item
redirection mechanism added
\end{itemize}

%%%%%%%%%%%%%%%%%%%%%%%%%%%%%%%%%%%%%%%%
\paragraph{v0.5:} 2017/04/26

\begin{itemize}
\item
functionality in definition file
\end{itemize}


%%%%%%%%%%%%%%%%%%%%%%%%%%%%%%%%%%%%%%%%%%%%%%%%%%%%%%%%%%%%%%%%%%%%%%%%%%%%%%%%
%%%%%%%%%%%%%%%%%%%%%%%%%%%%%%%%%%%%%%%%%%%%%%%%%%%%%%%%%%%%%%%%%%%%%%%%%%%%%%%%
%%%%%%%%%%%%%%%%%%%%%%%%%%%%%%%%%%%%%%%%%%%%%%%%%%%%%%%%%%%%%%%%%%%%%%%%%%%%%%%%
\appendix

\settowidth\MacroIndent{\rmfamily\scriptsize 000\ }

 \DocInput{childdoc.dtx}

\end{document}
%</driver>
% \fi
%
% %%%%%%%%%%%%%%%%%%%%%%%%%%%%%%%%%%%%%%%%%%%%%%%%%%%%%%%%%%%%%%%%%%%%%%%%%%%%%%
% %%%%%%%%%%%%%%%%%%%%%%%%%%%%%%%%%%%%%%%%%%%%%%%%%%%%%%%%%%%%%%%%%%%%%%%%%%%%%%
% \section{Sample}
%\iffalse
%<*samplemain>
%\fi
%
% The following presents a sample document
% with two chapters, two parts, a title page,
% a compile flag as well as three forwarding files to set the flag.
% It consists of eight |.tex| files:
% \begin{center}
% \begin{tabular}{ll}
% |cdocsamp.tex|&main file\\
% |cdocsch1.tex|&include file for chapter 1\\
% |cdocsch2.tex|&include file for chapter 2\\
% |cdocspt3.tex|&include file for part 3\\
% |cdocspt4.tex|&include file for part 4\\
% |cdocsdrf.tex|&forwarding file for main file in draft mode\\
% |cdocsfi1.tex|&forwarding file for final version of chapter 1\\
% |cdocsfi2.tex|&forwarding file for final version of chapter 2\\
% \end{tabular}
% \end{center}
% Each of the eight files can be compiled directly by the \LaTeX{} compiler.
%
% %%%%%%%%%%%%%%%%%%%%%%%%%%%%%%%%%%%%%%
% \paragraph{Main File.}
%
% The main file is called |cdocsamp.tex|.
%
% Load the \textsf{childdoc} definitions and
% declare the filename for the main document:
%    \begin{macrocode}
\input{childdoc.def}
\childdocmain{}
%    \end{macrocode}

% Optional override for |\version| flag:
%    \begin{macrocode}
%%\ifchilddoc\else\providecommand{\version}{draft}\fi
%    \end{macrocode}

% Define the default values for the |\version| flag
% (|final| for the main file and |draft| for childs):
%    \begin{macrocode}
\ifchilddoc
\providecommand{\version}{draft}
\else
\providecommand{\version}{final}
\fi
%    \end{macrocode}

% Load the standard document class:
%    \begin{macrocode}
\documentclass[12pt]{article}
%    \end{macrocode}

% Start the document body:
%    \begin{macrocode}
\begin{document}
%    \end{macrocode}

% Declare a title page.
% Print title, part of document being processed and version flag:
%    \begin{macrocode}
\addtocounter{page}{-1}
\begin{center}
{\LARGE\bfseries{}childdoc example\par}
\vspace{1cm}
\ifchilddoc
\ifchilddocmanual part\else chapter\fi:
`\childdocname' of `\childdocjob'\par
\else
main document: `\childdocjob'\par
\fi
version: \version\par
\end{center}
\newpage
%    \end{macrocode}

% Manually include selected file,
% otherwise process as usual:
%    \begin{macrocode}
\ifchilddocmanual
\section*{part `\childdocname'}
\input{\childdocname}
\else
%    \end{macrocode}

% Include the two chapters:
%    \begin{macrocode}
\include{cdocsch1}
\include{cdocsch2}
%    \end{macrocode}

% Include the two parts unless only chapters should be displayed:
%    \begin{macrocode}
\ifchilddoc\else
\section{part three}
\input{cdocspt3}
\section{part four}
\input{cdocspt4}
\fi
%    \end{macrocode}

% Process as usual until here:
%    \begin{macrocode}
\fi
%    \end{macrocode}

% End of document body:
%    \begin{macrocode}
\end{document}
%    \end{macrocode}
%\iffalse
%</samplemain>
%\fi
%
% %%%%%%%%%%%%%%%%%%%%%%%%%%%%%%%%%%%%%%
% \paragraph{Chapter Include Files.}
%
% The include files are called |cdocsch1.tex| and |cdocsch2.tex|.
%
%\iffalse
%<*samplechap1|samplechap2>
%\fi

% Optional override for |\version| flag:
%    \begin{macrocode}
%%\providecommand{\version}{final}
%    \end{macrocode}

% Include the main document:
%    \begin{macrocode}
\input{childdoc.def}
\childdocof{cdocsamp}
%    \end{macrocode}

%\iffalse
%</samplechap1|samplechap2>
%\fi
%
%\iffalse
%<*samplechap1>
%\fi
% Some text for chapter 1:
%    \begin{macrocode}
\section{one}
some text in chapter one
%    \end{macrocode}

%\iffalse
%</samplechap1>
%\fi
% Some text for chapter 2:
%\iffalse
%<*samplechap2>
%\fi
%    \begin{macrocode}
\section{two}
more text in chapter two
%    \end{macrocode}

%\iffalse
%</samplechap2>
%\fi
%
% %%%%%%%%%%%%%%%%%%%%%%%%%%%%%%%%%%%%%%
% \paragraph{Part Include Files.}
%
% The include files are called |cdocspt3.tex| and |cdocspt4.tex|.
%
%\iffalse
%<*samplepart3|samplepart4>
%\fi

% Optional override for |\version| flag:
%    \begin{macrocode}
%%\providecommand{\version}{final}
%    \end{macrocode}

% Include the main document:
%    \begin{macrocode}
\input{childdoc.def}
\childdocby{cdocsamp}
%    \end{macrocode}

%\iffalse
%</samplepart3|samplepart4>
%\fi
%
%\iffalse
%<*samplepart3>
%\fi
% Some text for part 3:
%    \begin{macrocode}
some text in part three
%    \end{macrocode}

%\iffalse
%</samplepart3>
%\fi
% Some text for part 4:
%\iffalse
%<*samplepart4>
%\fi
%    \begin{macrocode}
more text in part four
%    \end{macrocode}

%\iffalse
%</samplepart4>
%\fi
%
% %%%%%%%%%%%%%%%%%%%%%%%%%%%%%%%%%%%%%%
% \paragraph{Forwarding for a Complete Draft.}
%
% The following forwarding file |cdocsdrf.tex|
% compiles the main document in draft mode:
%\iffalse
%<*sampledraft>
%\fi
%    \begin{macrocode}
\def\version{draft}
\input{childdoc.def}
\childdocforward{cdocsamp}
%    \end{macrocode}

%\iffalse
%</sampledraft>
%\fi
%
% %%%%%%%%%%%%%%%%%%%%%%%%%%%%%%%%%%%%%%
% \paragraph{Forwarding for Final Version of the Chapters.}
%
% The following forwarding files |cdocsfn1.tex| and |cdocsfn2.tex|
% (with identical content)
% compile the final versions of the child documents
% |cdocsch1.tex| and |cdocsch2.tex|, respectively:
%\iffalse
%<*samplefinal>
%\fi
%    \begin{macrocode}
\def\version{final}
\input{childdoc.def}
\childdocforwardprefix[cdocsamp]{cdocsfn}{cdocsch}
%    \end{macrocode}

%\iffalse
%</samplefinal>
%\fi
%
% %%%%%%%%%%%%%%%%%%%%%%%%%%%%%%%%%%%%%%
% \paragraph{Command Line Processing.}
%
% The following three command lines generate the output files
% |cdocscld|, |cdocscl1| and |cdocscl2|
% which should be identical to
% |cdocsdrf|, |cdocsch1| and |cdocsfn2|, respectively:
% \begin{center}
% \begin{tabular}{l}
% |latex -jobname cdocscld \|\\
% |  "\def\version{draft}\input{childdoc.def}\childdocforward{cdocsamp}"|\\
% |latex -jobname cdocscl1 \|\\
% |  "\input{childdoc.def}\childdocforward[cdocsamp]{cdocsch1}"|\\
% |latex -jobname cdocscl2 \|\\
% |  "\def\version{final}\input{childdoc.def}\childdocforward{cdocsch2}"|
% \end{tabular}
% \end{center}
% Note that the trailing backslash on each first line
% merely continues the input to the second line
% (for convenient cut ant paste).
% Furthermore, the command |latex| can be replaced by any
% of its alternative versions such as |pdflatex|.
%
% %%%%%%%%%%%%%%%%%%%%%%%%%%%%%%%%%%%%%%%%%%%%%%%%%%%%%%%%%%%%%%%%%%%%%%%%%%%%%%
% %%%%%%%%%%%%%%%%%%%%%%%%%%%%%%%%%%%%%%%%%%%%%%%%%%%%%%%%%%%%%%%%%%%%%%%%%%%%%%
% \section{Implementation}
%\iffalse
%<*package>
%\fi
%
% This section describes the definitions file |childdoc.def|.

% The definitions cannot be loaded using |\usepackage| or |\RequirePackage|
% which has a mechanism to prevent loading a style file more than once.
% When loading the definitions by means of |\input|
% multiple instances have to be prevented manually:
%\iffalse
%This code needs to be before the `\ProvidesFile' directive
%which is defined at the beginning of this file.
%Therefore it is also placed there and commented out here.
%</package>
%<*discard>
%\fi
%    \begin{macrocode}
\ifdefined\childdocmain\endinput\fi
%    \end{macrocode}
%\iffalse
%</discard>
%<*package>
%\fi
%
% \macro{\ifchilddoc}
% \macro{\ifchilddocmanual}
% The conditional |\ifchilddoc| tells whether a
% child (true) or main (false) document is being compiled.
% The conditional |\ifchilddocmanual| tells whether
% the |\includeonly| mechanism is used (false) or
% the selection of child files must be performed manually (true).
% The definitions initialise to false:
%    \begin{macrocode}
\newif\ifchilddoc
\newif\ifchilddocmanual
%    \end{macrocode}

% \macro{\childdocname}
% \macro{\childdocjob}
% The macro |\childdocname| stores the name of the main document
% to be compiled. The macro |\childdocjob| stores the name of
% the document on which the \LaTeX{} compiler was originally invoked.
% The content of |\jobname| cannot be compared
% to filenames specified in the source due to different catcodes.
% The following code rescans |\jobname|, stores the result
% in |\childdocname| and saves a copy in |\childdocjob|:
%    \begin{macrocode}
\edef\childdocname{\scantokens\expandafter{\jobname\noexpand}}
\let\childdocjob\childdocname
%    \end{macrocode}

% \macro{\childdocdisable}
% The macro |\childdocdisable| prevents the main file
% from being processed more than once.
% At this stage, the main document command |\childdocmain|
% is assumed to be called once again where it should do nothing.
% Any subsequent call to it should prevent
% a secondary processing of the main document
% It overwrites the forwarding commands
% |\childdocof| and |\childdocforward|
% with empty macros to prevent further inclusions of the main document:
%    \begin{macrocode}
\newcommand{\childdocdisable}
{
  \renewcommand{\childdocmain}[1]{\renewcommand{\childdocmain}[1]{\endinput}}
  \renewcommand{\childdocof}[1]{}
  \renewcommand{\childdocby}[2][]{}
  \renewcommand{\childdocforward}[2][]{}
  \renewcommand{\childdocdisable}{}
}
%    \end{macrocode}

% \macro{\childdocmain}
% The macro |\childdocmain| is to be called at the top of the main file
% with nothing or the main filename (without extension) as argument.
% First, it breaks loops.
% If the argument is not empty and does not match |\childdocname|
% (which is set by the first inclusion of |childdoc.def|),
% |\ifchilddoc| is set to true, |\includeonly| is applied to the child file
% and |\jobname| is set to the main file
% (for proper handling of |.aux| files):
%    \begin{macrocode}
\newcommand{\childdocmain}[1]
{
  \childdocdisable\childdocmain{}
  \if?#1?\else
    \begingroup
      \def\childdoctmp{#1}
      \ifx\childdoctmp\childdocname
        \def\childdoctmp{}
      \else
        \def\childdoctmp
        {
          \childdoctrue
          \includeonly{\childdocname}
          \def\childdocjob{#1}
          \def\jobname{#1}
        }
      \fi
      \expandafter
    \endgroup
    \childdoctmp
  \fi
}
%    \end{macrocode}

% \macro{\childdocof}
% The command |\childdocof| redirects
% compilation to the main file |#1|.
%    \begin{macrocode}
\newcommand{\childdocof}[1]
{
  \childdocdisable
  \childdoctrue
  \includeonly{\childdocname}
  \def\jobname{#1}
  \def\childdocjob{#1}
  \input{#1}
}
%    \end{macrocode}

% \macro{\childdocby}
% The command |\childdocby| ....
%    \begin{macrocode}
\newcommand{\childdocby}[2][]
{
  \childdocdisable
  \childdoctrue
  \childdocmanualtrue
  \if?#1?\else
    \def\jobname{#2}
  \fi
  \def\childdocjob{#2}
  \input{#2}
  \endinput
}
%    \end{macrocode}

% \macro{\childdocforward}
% The command |\childdocforward| redirects
% compilation to the main file or
% (if the optional argument is given) a child file.
% Parameters are set as if the main file
% or a child file starting with |\childdocof| was compiled.
% Then compilation is handed over to the main file:
%    \begin{macrocode}
\newcommand{\childdocforward}[2][]
{
  \begingroup
    \if?#1?
      \def\childdoctmp
      {
        \def\childdocname{#2}
        \def\childdocjob{#2}
        \def\jobname{#2}
        \input{#2}
        \endinput
      }
    \else
      \def\childdoctmp
      {
        \childdocdisable
        \def\childdocname{#2}
        \childdoctrue
        \includeonly{#2}
        \def\childdocjob{#1}
        \def\jobname{#1}
        \input{#1}
        \endinput
      }
    \fi
    \expandafter
  \endgroup
  \childdoctmp
}
%    \end{macrocode}

% \macro{\childdocforwardprefix}
% The command |\childdocforwardprefix| redirects
% compilation to the main or a child file by means of a pattern.
% The prefix |#1| in the current filename is replaced by |#2|
% and the suffix of the current filename is kept
% (it is assumed that the filename does not contain the substring `|~~~|'
% which is used as a delimiter).
% Compilation is handed over to the new file by |\childdocforward|:
%    \begin{macrocode}
\newcommand{\childdocforwardprefix}[3][]
{
  \begingroup
    \def\childdocextract #2##1~~~{\def\childdoctmp{\childdocforward[#1]{#3##1}}}
    \expandafter\childdocextract\childdocname~~~
    \expandafter
  \endgroup
  \childdoctmp
}
%    \end{macrocode}

% \macro{\childdoc}
% The deprecated macro |\childdoc| is a legacy version of |\childdocmain|:
%    \begin{macrocode}
\newcommand{\childdoc}{\childdocmain}
%    \end{macrocode}

% \macro{\childdocredirect}
% The deprecated macro |\childdocredirect| is a legacy version
% of |\childdocforward| and |\childdocforwardprefix|:
%    \begin{macrocode}
\newcommand{\childdocredirect}[2][]
{
  \begingroup
    \if?#1?
      \def\childdoctmp{\childdocforward{#2}}
    \else
      \def\childdoctmp{\childdocforwardprefix{#1}{#2}}
    \fi
    \expandafter
  \endgroup
  \childdoctmp
}
%    \end{macrocode}

%\iffalse
%</package>
%\fi
%
\endinput
|\\
|\childdocmain{}|\\
\end{tabular}
\end{center}
at the very top of the main \LaTeX{} file,
in particular \emph{before} the |\documentclass| statement!
The argument of |\childdocmain| should be left empty
(but it must be present).

%%%%%%%%%%%%%%%%%%%%%%%%%%%%%%%%%%%%%%%%
\DescribeMacro{\childdocof}
Furthermore, add the commands
\begin{center}
\begin{tabular}{l}
|% \iffalse
%
% childdoc.dtx Copyright (C) 2017-2018 Niklas Beisert
%
% This work may be distributed and/or modified under the
% conditions of the LaTeX Project Public License, either version 1.3
% of this license or (at your option) any later version.
% The latest version of this license is in
%   http://www.latex-project.org/lppl.txt
% and version 1.3 or later is part of all distributions of LaTeX
% version 2005/12/01 or later.
%
% This work has the LPPL maintenance status `maintained'.
%
% The Current Maintainer of this work is Niklas Beisert.
%
% This work consists of the files childdoc.dtx and childdoc.ins
% and the derived files childdoc.def and cdocsamp.tex with
% cdocsch1.tex, cdocsch2.tex, cdocsdrf.tex, cdocsfn1.tex, cdocsfn2.tex.
%
%<package>\ifdefined\childdocmain\endinput\fi
%<package>\ProvidesFile{childdoc.def}[2018/12/30 v2.0 child document driver]
%<samplemain>\ProvidesFile{cdocsamp.tex}[2018/12/30 v2.0 sample for childdoc]
%<*driver>
%\ProvidesFile{childdoc.drv}[2018/12/30 v2.0 childdoc reference manual file]
\PassOptionsToClass{10pt,a4paper}{article}
\documentclass{ltxdoc}

\usepackage[margin=35mm]{geometry}
\usepackage{hyperref}
\usepackage{hyperxmp}
\usepackage[usenames]{color}

\hypersetup{colorlinks=true}
\hypersetup{pdfstartview=FitH}
\hypersetup{pdfpagemode=UseNone}
\hypersetup{pdfsource={}}
\hypersetup{pdflang={en-UK}}
\hypersetup{pdfcopyright={Copyright 2017-2018 Niklas Beisert.
  This work may be distributed and/or modified under the
  conditions of the LaTeX Project Public License, either version 1.3
  of this license or (at your option) any later version.}}
\hypersetup{pdflicenseurl={http://www.latex-project.org/lppl.txt}}
\hypersetup{pdfcontactaddress={ETH Zurich, ITP, HIT K,
  Wolfgang-Pauli-Strasse 27}}
\hypersetup{pdfcontactpostcode={8093}}
\hypersetup{pdfcontactcity={Zurich}}
\hypersetup{pdfcontactcountry={Switzerland}}
\hypersetup{pdfcontactemail={nbeisert@itp.phys.ethz.ch}}
\hypersetup{pdfcontacturl={http://people.phys.ethz.ch/\xmptilde nbeisert/}}

\newcommand{\secref}[1]{\hyperref[#1]{section \ref*{#1}}}

\parskip1ex
\parindent0pt
\let\olditemize\itemize
\def\itemize{\olditemize\parskip0pt}

\begin{document}

\title{The \textsf{childdoc} Package}
\hypersetup{pdftitle={The childdoc Package}}
\author{Niklas Beisert\\[2ex]
  Institut f\"ur Theoretische Physik\\
  Eidgen\"ossische Technische Hochschule Z\"urich\\
  Wolfgang-Pauli-Strasse 27, 8093 Z\"urich, Switzerland\\[1ex]
  \href{mailto:nbeisert@itp.phys.ethz.ch}
  {\texttt{nbeisert@itp.phys.ethz.ch}}}
\hypersetup{pdfauthor={Niklas Beisert}}
\hypersetup{pdfsubject={Manual for the LaTeX2e Package childdoc}}
\date{30 December 2018, \textsf{v2.0}}
\maketitle

\begin{abstract}\noindent
\textsf{childdoc} is a \LaTeXe{} package
that enables the direct compilation
of document sections included by |\include|
to individual files.
\end{abstract}

\begingroup
\parskip0ex
\tableofcontents
\endgroup

%%%%%%%%%%%%%%%%%%%%%%%%%%%%%%%%%%%%%%%%%%%%%%%%%%%%%%%%%%%%%%%%%%%%%%%%%%%%%%%%
%%%%%%%%%%%%%%%%%%%%%%%%%%%%%%%%%%%%%%%%%%%%%%%%%%%%%%%%%%%%%%%%%%%%%%%%%%%%%%%%
\section{Introduction}

\LaTeX{} provides a mechanism to structure a large document (such as a book)
into a main file and several child files (containing the chapters)
using the |\include| command.
This mechanism is beneficial for documents
which span hundreds of pages in order to
make the source file(s) more manageable.
Moreover, compilation can be restricted to
selected child files by means of the |\includeonly| command.
The latter feature can be used to reduce the compilation time while editing
(this was significantly more useful in the earlier days of \LaTeX{})
or to generate a smaller document which is easier to navigate.
Another application of |\includeonly| is to generate
documents consisting of selected parts of the complete document.

However, there are a few drawbacks of the plain |\include| mechanism:
\begin{itemize}
\item
The child files cannot be compiled on their own,
they can only be compiled via the main file.
A naive editing environment
(such as a text editor with an option
to have the current file processed by \LaTeX)
may require one to switch to the main file before compiling;
attempting to compile the child file produces errors.
\item
The main file must be modified (each time)
to adjust the |\includeonly| command
to the present needs. This easily leaves the main file in a messy state.
\item
The generated document will always carry the filename
of the main document. This is inconvenient if
several child files are to be compiled and
to be kept for distribution.
\end{itemize}

The present package provides a simple interface
to make child files individually compilable by \LaTeX{}.
Compiling a child file then has the same effect as compiling
the main file with an |\includeonly| command
to select the appropriate child.
Moreover the generated document will carry the name of the child
rather than the main file.
This resolves all three above issues.

This feature is meant to make the editing of books,
thesis documents and lecture notes somewhat more convenient.
However, the package can also be used efficiently for
composing a series of documents (such as exercise sheets)
which are typically distributed individually.
It then assists the author in generating the individual documents
(potentially in different versions)
as well as a document containing the collected series.
Another application is in developing style files
or other kinds of included material
where compilation of the style file could redirect
to a sample or test file.

%%%%%%%%%%%%%%%%%%%%%%%%%%%%%%%%%%%%%%%%%%%%%%%%%%%%%%%%%%%%%%%%%%%%%%%%%%%%%%%%
%%%%%%%%%%%%%%%%%%%%%%%%%%%%%%%%%%%%%%%%%%%%%%%%%%%%%%%%%%%%%%%%%%%%%%%%%%%%%%%%
\section{Usage}

First of all, the package \textsf{childdoc} is \emph{not} a standard
\LaTeXe{} |.sty| style file! Therefore it needs to be invoked in
a non-standard way.

%%%%%%%%%%%%%%%%%%%%%%%%%%%%%%%%%%%%%%%%%%%%%%%%%%%%%%%%%%%%%%%%%%%%%%%%%%%%%%%%
\subsection{Included Files}
\label{sec:include}

%%%%%%%%%%%%%%%%%%%%%%%%%%%%%%%%%%%%%%%%
\DescribeMacro{\childdocmain}
To use the package, add the commands
\begin{center}
\begin{tabular}{l}
|\input{childdoc.def}|\\
|\childdocmain{}|\\
\end{tabular}
\end{center}
at the very top of the main \LaTeX{} file,
in particular \emph{before} the |\documentclass| statement!
The argument of |\childdocmain| should be left empty
(but it must be present).

%%%%%%%%%%%%%%%%%%%%%%%%%%%%%%%%%%%%%%%%
\DescribeMacro{\childdocof}
Furthermore, add the commands
\begin{center}
\begin{tabular}{l}
|\input{childdoc.def}|\\
|\childdocof{|\textit{main}|}|\\
\end{tabular}
\end{center}
at the top of every child file \textit{child}
which is included by |\include{|\textit{child}|}|
from within the main file
(or at least for those files to be compiled individually).
The argument \textit{main} must be the filename of the main file.

There are a couple of
considerations in setting up the main and child documents:

%%%%%%%%%%%%%%%%%%%%%%%%%%%%%%%%%%%%%%%%
\paragraph{Restrictions.}

Please note the following restrictions:
\begin{itemize}
\item
|\childdocmain| must be called with one argument \textit{main}
to ensure compatibility with earlier version of the package.
It must either be empty (|\childdocmain{}|)
or precisely match the filename of the main file in which it is specified.
See \secref{sec:detection} for further information.
\item
The filename \textit{main} must be specified without the |.tex| extension.
\item
The filename \textit{main} is case sensitive
(even in case-insensitive file systems)
due to internal string comparison.
\item
The argument \textit{main} should be fully expanded, it cannot be a macro.
\item
Subdirectories and special characters should be avoided in filenames.
\item
The command |\childdocmain{|\textit{main}|}| must be followed by a whitespace.
It should not be followed immediately by another command
or by a comment mark `|%|'.
This is because the \TeX{} parser reads the token immediately following
the argument of |\childdocmain| and puts it
at the beginning of every child section;
however, a white\-space is ignored.
\end{itemize}

%%%%%%%%%%%%%%%%%%%%%%%%%%%%%%%%%%%%%%%%
\paragraph{Content of Main File.}

It is advisable to place all content in the child files included by |\include|.
Any output contained in the main file will appear in all child documents
unless suppressed manually;
it cannot be suppressed automatically by the |\includeonly| directive
and thus should normally be avoided.
A method to include some content in the main file
by means of conditional processing is described in \secref{sec:conditional}.

%%%%%%%%%%%%%%%%%%%%%%%%%%%%%%%%%%%%%%%%
\paragraph{Page Numbering.}

When only a part of the document is compiled,
the appropriate numbering of pages
(as well as other status parameters)
is determined from the |.aux| files.
The latter contain information from previous passes.
However this information needs to propagate through
all intermediate child documents.
Therefore the page numbering in child documents may well
be inconsistent until the complete document is compiled at least once.

A useful (if unconventional) way to always ensure a consistent
page numbering is to restart the numbering in each child document
and denote the pages by `\textit{child}|.|\textit{page}'
where \textit{child} represents the chapter/section number of the child file.
This can be achieved by the command
|\numberwithin{page}{|\textit{child}|}|
of the \textsf{amsmath} package
where \textit{child} can be |chapter| or |section|
depending on the chosen structuring.
Alternatively, one can modify the macro |\thepage| appropriately
and reset the counter |page| at the start of each child file.

%%%%%%%%%%%%%%%%%%%%%%%%%%%%%%%%%%%%%%%%%%%%%%%%%%%%%%%%%%%%%%%%%%%%%%%%%%%%%%%%
\subsection{Conditional Processing}
\label{sec:conditional}

The package provides a mechanism to compile different versions
of a document. To customise the versions further some conditional processing
can come in handy to distinguish which version is being compiled.
The package provides two macros to describe the compilation context:

%%%%%%%%%%%%%%%%%%%%%%%%%%%%%%%%%%%%%%%%
\DescribeMacro{\ifchilddoc}
The conditional |\ifchilddoc| distinguishes between the compilation of
child documents and the main document:
%
\begin{center}
|\ifchilddoc |\textit{child-code}| |[|\||else |\textit{main-code}]| \||fi|
\end{center}

%%%%%%%%%%%%%%%%%%%%%%%%%%%%%%%%%%%%%%%%
\DescribeMacro{\childdocname}
\DescribeMacro{\childdocjob}
The macro |\childdocname| contains the filename (without extension)
of the main or child file being processed.
Note that |\childdocjob| will always contain the name of the main file.

%%%%%%%%%%%%%%%%%%%%%%%%%%%%%%%%%%%%%%%%
\paragraph{Title Page.}

Conditional processing can be used to include a title or banner page
in the main document when proper precautions are taken.
Importantly, the code in the main file should ensure that the page counter
(as well as other status parameters which are stored in the |.aux| files)
takes the same value after the conditional processing.
Otherwise the page numbers may take divergent values
depending on which part is compiled.

For example, a title page could be declared by:
%
\begin{center}
\begin{tabular}{l}
|\ifchilddoc\||else|\\
|\addtocounter{page}{-1}|\\
\textit{code for title page}\\
|\newpage|\\
|\||fi|
\end{tabular}
\end{center}
%
A banner page for the child documents can be generated by:
%
\begin{center}
\begin{tabular}{l}
|\ifchilddoc|\\
|\addtocounter{page}{-1}|\\
\textit{code for banner page}\\
|\newpage|\\
|\||fi|
\end{tabular}
\end{center}
%
Here one could write a message such as:
\begin{center}
|This is the part \childdocname{} of \childdocjob{}.|
\end{center}

%%%%%%%%%%%%%%%%%%%%%%%%%%%%%%%%%%%%%%%%%%%%%%%%%%%%%%%%%%%%%%%%%%%%%%%%%%%%%%%%
\subsection{Flags}
\label{sec:flags}

The package makes it easy to generate different versions
of the main or child documents.
To this end compilation flags can be defined
and assigned different default values.
They will be particularly useful in conjunction
with the forwarding mechanism described in \secref{sec:forward}.

For example, it may be useful to have a flag |\version|
which can be set to |draft| or |final|.
The document source will contain some conditional code
depending on the value of |\version|.
Suppose further, the flag should default to |final| for the main file
and to |draft| for child files
which is a natural assignment for editing the document.
This is achieved by placing the following code
in the preamble of the main document
(below the |\childdocmain| directive):
%
\begin{center}
\begin{tabular}{l}
|\ifchilddoc|\\
|\providecommand{\version}{draft}|\\
|\||else|\\
|\providecommand{\version}{final}|\\
|\||fi|
\end{tabular}
\end{center}
%
The definition by |\providecommand| makes sure
that previous definitions are not overwritten.
Further statements |\providecommand{\version}{...}|
can thus be added before the above code to override it.

For the main file, one might add a line
(between |\childdocmain| and the above block)
%
\begin{center}
|%\ifchilddoc\||else\providecommand{\version}{draft}\||fi|
\end{center}
%
which can be uncommented to produce a draft version.
Likewise one can add a line to the very top of a child file
(above the |\childdocof{|\textit{main}|}| directive)
%
\begin{center}
|%\providecommand{\version}{final}|
\end{center}
%
which can be uncommented to produce the final version of this child document.

%%%%%%%%%%%%%%%%%%%%%%%%%%%%%%%%%%%%%%%%%%%%%%%%%%%%%%%%%%%%%%%%%%%%%%%%%%%%%%%%
\subsection{Forwarding}
\label{sec:forward}

Different versions of the main or child documents
using compilation flags as described in \secref{sec:flags}
can be (permanently) stored in different files
for convenient compilation, viewing and distribution.
To this end, the package defines a command
to pass on compilation to a different file:

%%%%%%%%%%%%%%%%%%%%%%%%%%%%%%%%%%%%%%%%
\DescribeMacro{\childdocforward}
The command |\childdocforward| redirects processing to
another source file:
%
\begin{center}
\begin{tabular}{l}
|\input{childdoc.def}|\\
|\childdocforward[|\textit{main}|]{|\textit{dest}|}|\\
\end{tabular}
\end{center}
%
The argument \textit{dest} is the destination file
(without extension).
It should be the main file or one of the child files.
Note that further \textsf{childdoc} directives
such as |\childdocof| and |\childdocforward|
in the indicated file will be processed in this form.
The optional argument \textit{main}
passes on directly to the main file \textit{main}
while pretending to compile the child \textit{dest}.
This form behaves as if \textit{dest}
issues |\childdocof{|\textit{main}|}| right away,
and no further \textsf{childdoc} directives will be processed.

%%%%%%%%%%%%%%%%%%%%%%%%%%%%%%%%%%%%%%%%
\DescribeMacro{\...prefix}
In the alternative form |\childdocforwardprefix|,
%
\begin{center}
\begin{tabular}{l}
|\input{childdoc.def}|\\
|\childdocforwardprefix[|\textit{main}|]{|\textit{prefix}|}{|\textit{dest}|}|
\end{tabular}
\end{center}
%
the destination file is determined by a pattern
depending on the current file:
To make this work, the current file must be called
`{\textit{prefix}\hspace{0.2em}\textit{suffix}}'
with \textit{prefix} matching precisely the argument.
Processing is then passed on to the file
`{\textit{dest}\hspace{0.2em}\textit{suffix}}'.
Surely, the same effect is achieved by
directly specifying the
argument `{\textit{dest}\hspace{0.2em}\textit{suffix}}'
in the first form.
However, that requires to set up a different file
for each child. With the alternative form of the command
all these files can have exactly the same content
which simplifies setting them up and maintaining them.

For example, the following file |draft.tex|
with a compilation flag |\version| as described in \secref{sec:flags}
compiles the main document as a draft:
%
\begin{center}
\begin{tabular}{l}
|\def\version{draft}|\\
|\input{childdoc.def}|\\
|\childdocforward{|\textit{main}|}|
\end{tabular}
\end{center}
%
Likewise, the following files |final|\textit{nn}|.tex|
compile the final version of the child document
|child|\textit{nn}|.tex|:
%
\begin{center}
\begin{tabular}{l}
|\def\version{final}|\\
|\input{childdoc.def}|\\
|\childdocforwardprefix{final}{child}|
\end{tabular}
\end{center}
%

Note that when several versions of a main file and/or of each child file
are to be generated, it may be convenient to set up a |Makefile| or
shell script to automatise the process.

%%%%%%%%%%%%%%%%%%%%%%%%%%%%%%%%%%%%%%%%%%%%%%%%%%%%%%%%%%%%%%%%%%%%%%%%%%%%%%%%
\subsection{Command Line Processing}
\label{sec:commandline}

The effect of redirection files can also be achieved by invoking
the \LaTeX{} compiler with a more elaborate command line.
Most conveniently this should be done as part
of a shell script or a |Makefile|.

When using \textsf{childdoc} in the main file, the following
command lines effectively perform a redirection
(note that depending on the shell being used,
backslashes may have to be doubled: `|\|' $\to$ `|\\|'):
%
\begin{center}
|... -jobname "|\textit{target}|" |\\|"|[\textit{flags}]%
|\input{childdoc.def}\childdocforward[|\textit{main}|]{|\textit{dest}|}"|
\end{center}
%
Here \textit{target} is the name of the output file,
\textit{main} is the name of the main file
and \textit{dest} is the name of the main or child file to be processed
(all filenames without extensions).
The optional argument \textit{main} can be omitted
if \textit{main} matches \textit{dest}.
Optionally, compilation \textit{flags} can be defined via |\def| commands.
This command line makes the \TeX{} engine believe
it is compiling the file \textit{target}
whose content is specified as the latter parameter.
The provided code then forwards the processing to
\textit{main} or \textit{dest} as described in \secref{sec:forward}.

%%%%%%%%%%%%%%%%%%%%%%%%%%%%%%%%%%%%%%%%%%%%%%%%%%%%%%%%%%%%%%%%%%%%%%%%%%%%%%%%
\subsection{Include by Input}
\label{sec:input}

Including child documents by |\include| has some restrictions by design.
Most notably, the content of a child document always occupies
its own set of pages; pages cannot be shared between child documents.
Usually, this behaviour makes perfect sense
because each child document contain an essential part of the document.
However, in some situations it may be desirable to compose
a document from a collection of parts
without having mandatory page breaks between then.
For this case, the package
provides a mechanism to include parts
by |\input| which can also be processed individually.
However, by construction this mechanism
requires manual handling of the content to be output.

%%%%%%%%%%%%%%%%%%%%%%%%%%%%%%%%%%%%%%%%
\DescribeMacro{\ifchilddocmanual}
The main file should be prepared as usual, see \secref{sec:include}.
However, the document body must make a distinction
between processing of an individual part and of the main document, e.g.:
%
\begin{center}
\begin{tabular}{l}
|\ifchilddocmanual|\\
|\input{\childdocname}|\\
|\||else|\\
\textit{document body with }|\input{|\textit{part}|}|\\
|\||fi|
\end{tabular}
\end{center}
%
The conditional |\ifchilddocmanual| is true whenever
a part to be included by |\input| is being compiled,
and the name of the part is stored in |\childdocname|.

%%%%%%%%%%%%%%%%%%%%%%%%%%%%%%%%%%%%%%%%
\DescribeMacro{\childdocby}
Each part to be included by |\input| should start with:
%
\begin{center}
\begin{tabular}{l}
|\input{childdoc.def}|\\
|\childdocby{|\textit{main}|}|\\
\end{tabular}
\end{center}
%
The directive |\childdocby| is similar to |\childdocof|
described in \secref{sec:include},
but the subsequent selection of content must be done manually.
To that end, both |\ifchilddoc| and |\ifchilddocmanual|
will be true upon processing of a part,
and the name of the part is stored in |\childdocname|.
Note that |\jobname| will be set to the filename of the current part
so that each part receives an individual |.aux| file
that does not interfere with the |.aux| file(s) of the main document.
This behaviour can be altered by the alternative form
|\childdocby[*]{|\textit{main}|}| (with a non-empty optional argument)
which uses the |.aux| file of the main document
by setting |\jobname| to \textit{main}.

%%%%%%%%%%%%%%%%%%%%%%%%%%%%%%%%%%%%%%%%%%%%%%%%%%%%%%%%%%%%%%%%%%%%%%%%%%%%%%%%
\subsection{Driver Development}
\label{sec:driver}

The \textsf{childdoc} mechanism can also be use for the development
of definition files such as \LaTeX{} styles or classes.
This case differs from the above setup with multiple parts
included by |\include| in that no |\includeonly| should be invoked.
This can be achieved by starting the include file
(before |\ProvidesPackage|) with:
%
\begin{center}
\begin{tabular}{l}
|\input{childdoc.def}|\\
|\childdocforward{|\textit{main}|}|\\
\end{tabular}
\end{center}
%
or alternatively with:
%
\begin{center}
\begin{tabular}{l}
|\input{childdoc.def}|\\
|\childdocby{|\textit{main}|}|\\
\end{tabular}
\end{center}
%
Both forms have slightly different effects as described above.
The main file is prepared as usual, see \secref{sec:include}.

%%%%%%%%%%%%%%%%%%%%%%%%%%%%%%%%%%%%%%%%%%%%%%%%%%%%%%%%%%%%%%%%%%%%%%%%%%%%%%%%
\subsection{Legacy Detection}
\label{sec:detection}

The directive |\childdocmain| in the main file can detect
whether the complete document or merely a child is to be compiled
even without using the directive |\childdocof|.
This method is deprecated because it is less robust
and there is no compelling reason to use it;
it is merely provided for backward compatibility
and it may be removed in future versions.

If the detection mechanism is to be used,
it is mandatory to correctly specify
the filename of the main file as the argument of |\childdocmain|:
%
\begin{center}
\begin{tabular}{l}
|\input{childdoc.def}|\\
|\childdocmain{|\textit{main}|}|\\
\end{tabular}
\end{center}
%
If |\jobname| does not match the argument \textit{main} of |\childdocmain|,
it is assumed that |\jobname| points to the child file to be compiled.
When using |\childdocmain| with the main file specified as argument,
it suffices to start a child file
with just |\input{|\textit{main}|}|
without loading of the package and using |\childdocof|.
If instead all processing is done
with the appropriate \textsf{childdoc} directives,
the argument of \textit{main} of |\childdocmain| can be empty.

An alternative version of the command line processing described
in \secref{sec:commandline} using the detection mechanism reads:
%
\begin{center}
|... -jobname "|\textit{target}|" "|[\textit{flags}]%
[|\def\jobname{|\textit{dest}|}|]|\input{|\textit{main}|}"|
\end{center}

%%%%%%%%%%%%%%%%%%%%%%%%%%%%%%%%%%%%%%%%%%%%%%%%%%%%%%%%%%%%%%%%%%%%%%%%%%%%%%%%
\subsection{Manual Code}
\label{sec:manual}

In case one cannot be certain whether the definitions file |childdoc.def|
is installed on the target \TeX{} distribution
and one prefers not to ship it,
it is conceivable to paste a few relevant commands into the sources.

To that end, drop all statements |\input{childdoc.def}|
and perform the replacements as outlined below.
Instead of |\childdocmain{|\textit{main}|}| add the following code
to the top of the main file:
%
\begin{center}
\begin{tabular}{l}
|\||ifdefined\childdocname\endinput\||fi\newif\ifchilddoc|\\
|\edef\childdocname{\scantokens\expandafter{\jobname\noexpand}}|\\
|\def\childdocmain{|\textit{main}|}\||ifx\childdocmain\childdocname\||else|\\
|\childdoctrue\includeonly{\childdocname}\let\jobname\childdocmain\||fi|\\
\end{tabular}
\end{center}
%
Instead of |\childdocof{|\textit{main}|}| just include the main file
at the top of each child file:
%
\begin{center}
|\input{|\textit{main}|}|
\end{center}
%
A simple redirection |\childdocforward{|\textit{dest}|}| is achieved by:
%
\begin{center}
|\def\jobname{|\textit{dest}|}\input{\jobname}|
\end{center}
%
The redirection with prefix
|\childdocforwardprefix[|\textit{prefix}|]{|\textit{dest}|}|
is accomplished by:
%
\begin{center}
\begin{tabular}{l}
|{\edef\jobname{\scantokens\expandafter{\jobname\noexpand}}|\\
|\def\redirectjob |\textit{prefix}|#1~~~{\gdef\jobname{|\textit{dest}|#1}}|\\
|\expandafter\redirectjob\jobname~~~}\input{\jobname}|
\end{tabular}
\end{center}

In an alternative approach,
child documents can be compiled by a specific command line
without additional code or specific definitions:
%
\begin{center}
|... -jobname "|\textit{target}|" "|[\textit{flags}]%
|\includeonly{|\textit{dest}|}\input{|\textit{main}|}"|
\end{center}
%

%%%%%%%%%%%%%%%%%%%%%%%%%%%%%%%%%%%%%%%%%%%%%%%%%%%%%%%%%%%%%%%%%%%%%%%%%%%%%%%%
%%%%%%%%%%%%%%%%%%%%%%%%%%%%%%%%%%%%%%%%%%%%%%%%%%%%%%%%%%%%%%%%%%%%%%%%%%%%%%%%
\section{Information}

%%%%%%%%%%%%%%%%%%%%%%%%%%%%%%%%%%%%%%%%%%%%%%%%%%%%%%%%%%%%%%%%%%%%%%%%%%%%%%%%
\subsection{Copyright}

Copyright \copyright{} 2017--2018 Niklas Beisert

This work may be distributed and/or modified under the
conditions of the \LaTeX{} Project Public License, either version 1.3
of this license or (at your option) any later version.
The latest version of this license is in
  \url{http://www.latex-project.org/lppl.txt}
and version 1.3 or later is part of all distributions of \LaTeX{}
version 2005/12/01 or later.

This work has the LPPL maintenance status `maintained'.

The Current Maintainer of this work is Niklas Beisert.

This work consists of the files |README.txt|, |childdoc.ins| and |childdoc.dtx|
as well as the derived files |childdoc.def|, |cdocsamp.tex|
with |cdocsch1.tex|, |cdocsch2.tex|, |cdocspt3.tex|, |cdocspt4.tex|,
|cdocsdrf.tex|, |cdocsfn1.tex|, |cdocsfn2.tex|
as well as |childdoc.pdf|.

%%%%%%%%%%%%%%%%%%%%%%%%%%%%%%%%%%%%%%%%%%%%%%%%%%%%%%%%%%%%%%%%%%%%%%%%%%%%%%%%
\subsection{Files and Installation}

The package consists of the files:
%
\begin{center}
\begin{tabular}{ll}
    |README.txt|   & readme file \\
    |childdoc.ins| & installation file \\
    |childdoc.dtx| & source file \\
    |childdoc.def| & definition file \\
    |cdocsamp.tex| & sample main file \\
    |cdocsch1.tex| & sample include file \\
    |cdocsch2.tex| & sample include file \\
    |cdocspt3.tex| & sample part file \\
    |cdocspt4.tex| & sample part file \\
    |cdocsdrf.tex| & sample redirection file \\
    |cdocsfn1.tex| & sample redirection file \\
    |cdocsfn2.tex| & sample redirection file \\
    |childdoc.pdf| & manual
\end{tabular}
\end{center}
%
The distribution consists of the files
|README.txt|, |childdoc.ins| and |childdoc.dtx|.
%
\begin{itemize}
\item
Run (pdf)\LaTeX{} on |childdoc.dtx|
to compile the manual |childdoc.pdf| (this file).
\item
Run \LaTeX{} on |childdoc.ins| to create the definitions file |childdoc.def|
and the sample |cdocsamp.tex| with include files
|cdocsch1.tex|, |cdocsch2.tex|, |cdocspt3.tex|, |cdocspt4.tex|,
|cdocsdrf.tex|, |cdocsfn1.tex|, |cdocsfn2.tex|.
Then copy the file |childdoc.def| to an appropriate directory of your \LaTeX{}
distribution, e.g.\ \textit{texmf-root}|/tex/latex/childdoc|.
\end{itemize}

%%%%%%%%%%%%%%%%%%%%%%%%%%%%%%%%%%%%%%%%%%%%%%%%%%%%%%%%%%%%%%%%%%%%%%%%%%%%%%%%
\subsection{Related CTAN Packages}

There are several other packages which offer a similar functionality:
%
\begin{itemize}
\item
The packages
\href{http://ctan.org/pkg/docmute}{\textsf{docmute}},
\href{http://ctan.org/pkg/includex}{\textsf{includex}} and
\href{http://ctan.org/pkg/standalone}{\textsf{standalone}}
provide commands to include only the document body of
a child file thus allowing both files to be compiled individually.
\item
The packages \href{http://ctan.org/pkg/subdocs}{\textsf{subdocs}}
and \href{http://ctan.org/pkg/subfiles}{\textsf{subfiles}}
provide structures in which the main and child documents can be
encapsulated and allowing them to be compiled individually.
The inclusion mechanism is different from the conventional |\include|.
\item
The package \href{http://ctan.org/pkg/combine}{\textsf{combine}}
is an elaborate solution to combine several documents into one.
\end{itemize}
%
See also the CTAN topic \href{http://ctan.org/topic/subdocs}{\textsf{subdocs}}
for further related packages.
The present package differs from the above solutions in that
a document structure constructed with the conventional |\include| mechanism
just needs two extra commands at the top of every file
such that all constituent files can be compiled individually.

%%%%%%%%%%%%%%%%%%%%%%%%%%%%%%%%%%%%%%%%%%%%%%%%%%%%%%%%%%%%%%%%%%%%%%%%%%%%%%%%
%\subsection{Feature Suggestions}
%
%The following is a list of features which may be useful for future
%versions of this package:
%%
%\begin{itemize}
%\item
%\ldots
%\end{itemize}

%%%%%%%%%%%%%%%%%%%%%%%%%%%%%%%%%%%%%%%%%%%%%%%%%%%%%%%%%%%%%%%%%%%%%%%%%%%%%%%%
\subsection{Revision History}

%%%%%%%%%%%%%%%%%%%%%%%%%%%%%%%%%%%%%%%%
\paragraph{v2.0:} 2018/12/30

\begin{itemize}
\item
immediate forward processing
\item
added |\childdocby| mechanism
\item
manual restructured
\end{itemize}

%%%%%%%%%%%%%%%%%%%%%%%%%%%%%%%%%%%%%%%%
\paragraph{v1.6:} 2018/01/17

\begin{itemize}
\item
application for development of include files
\item
corrections to manual
\end{itemize}

%%%%%%%%%%%%%%%%%%%%%%%%%%%%%%%%%%%%%%%%
\paragraph{v1.5:} 2017/05/21

\begin{itemize}
\item
more complete structuring introduced
\item
|\childdocof| introduced
\item
|\childdoc| renamed to |\childdocmain|
\item
|\childredirect| renamed to |\childdocforward| and |\childdocforwardprefix|
and functionality expanded
\end{itemize}

%%%%%%%%%%%%%%%%%%%%%%%%%%%%%%%%%%%%%%%%
\paragraph{v1.0:} 2017/04/27

\begin{itemize}
\item
manual and install package
\item
first version published on CTAN
\end{itemize}

%%%%%%%%%%%%%%%%%%%%%%%%%%%%%%%%%%%%%%%%
\paragraph{v0.6:} 2017/04/26

\begin{itemize}
\item
redirection mechanism added
\end{itemize}

%%%%%%%%%%%%%%%%%%%%%%%%%%%%%%%%%%%%%%%%
\paragraph{v0.5:} 2017/04/26

\begin{itemize}
\item
functionality in definition file
\end{itemize}


%%%%%%%%%%%%%%%%%%%%%%%%%%%%%%%%%%%%%%%%%%%%%%%%%%%%%%%%%%%%%%%%%%%%%%%%%%%%%%%%
%%%%%%%%%%%%%%%%%%%%%%%%%%%%%%%%%%%%%%%%%%%%%%%%%%%%%%%%%%%%%%%%%%%%%%%%%%%%%%%%
%%%%%%%%%%%%%%%%%%%%%%%%%%%%%%%%%%%%%%%%%%%%%%%%%%%%%%%%%%%%%%%%%%%%%%%%%%%%%%%%
\appendix

\settowidth\MacroIndent{\rmfamily\scriptsize 000\ }

 \DocInput{childdoc.dtx}

\end{document}
%</driver>
% \fi
%
% %%%%%%%%%%%%%%%%%%%%%%%%%%%%%%%%%%%%%%%%%%%%%%%%%%%%%%%%%%%%%%%%%%%%%%%%%%%%%%
% %%%%%%%%%%%%%%%%%%%%%%%%%%%%%%%%%%%%%%%%%%%%%%%%%%%%%%%%%%%%%%%%%%%%%%%%%%%%%%
% \section{Sample}
%\iffalse
%<*samplemain>
%\fi
%
% The following presents a sample document
% with two chapters, two parts, a title page,
% a compile flag as well as three forwarding files to set the flag.
% It consists of eight |.tex| files:
% \begin{center}
% \begin{tabular}{ll}
% |cdocsamp.tex|&main file\\
% |cdocsch1.tex|&include file for chapter 1\\
% |cdocsch2.tex|&include file for chapter 2\\
% |cdocspt3.tex|&include file for part 3\\
% |cdocspt4.tex|&include file for part 4\\
% |cdocsdrf.tex|&forwarding file for main file in draft mode\\
% |cdocsfi1.tex|&forwarding file for final version of chapter 1\\
% |cdocsfi2.tex|&forwarding file for final version of chapter 2\\
% \end{tabular}
% \end{center}
% Each of the eight files can be compiled directly by the \LaTeX{} compiler.
%
% %%%%%%%%%%%%%%%%%%%%%%%%%%%%%%%%%%%%%%
% \paragraph{Main File.}
%
% The main file is called |cdocsamp.tex|.
%
% Load the \textsf{childdoc} definitions and
% declare the filename for the main document:
%    \begin{macrocode}
\input{childdoc.def}
\childdocmain{}
%    \end{macrocode}

% Optional override for |\version| flag:
%    \begin{macrocode}
%%\ifchilddoc\else\providecommand{\version}{draft}\fi
%    \end{macrocode}

% Define the default values for the |\version| flag
% (|final| for the main file and |draft| for childs):
%    \begin{macrocode}
\ifchilddoc
\providecommand{\version}{draft}
\else
\providecommand{\version}{final}
\fi
%    \end{macrocode}

% Load the standard document class:
%    \begin{macrocode}
\documentclass[12pt]{article}
%    \end{macrocode}

% Start the document body:
%    \begin{macrocode}
\begin{document}
%    \end{macrocode}

% Declare a title page.
% Print title, part of document being processed and version flag:
%    \begin{macrocode}
\addtocounter{page}{-1}
\begin{center}
{\LARGE\bfseries{}childdoc example\par}
\vspace{1cm}
\ifchilddoc
\ifchilddocmanual part\else chapter\fi:
`\childdocname' of `\childdocjob'\par
\else
main document: `\childdocjob'\par
\fi
version: \version\par
\end{center}
\newpage
%    \end{macrocode}

% Manually include selected file,
% otherwise process as usual:
%    \begin{macrocode}
\ifchilddocmanual
\section*{part `\childdocname'}
\input{\childdocname}
\else
%    \end{macrocode}

% Include the two chapters:
%    \begin{macrocode}
\include{cdocsch1}
\include{cdocsch2}
%    \end{macrocode}

% Include the two parts unless only chapters should be displayed:
%    \begin{macrocode}
\ifchilddoc\else
\section{part three}
\input{cdocspt3}
\section{part four}
\input{cdocspt4}
\fi
%    \end{macrocode}

% Process as usual until here:
%    \begin{macrocode}
\fi
%    \end{macrocode}

% End of document body:
%    \begin{macrocode}
\end{document}
%    \end{macrocode}
%\iffalse
%</samplemain>
%\fi
%
% %%%%%%%%%%%%%%%%%%%%%%%%%%%%%%%%%%%%%%
% \paragraph{Chapter Include Files.}
%
% The include files are called |cdocsch1.tex| and |cdocsch2.tex|.
%
%\iffalse
%<*samplechap1|samplechap2>
%\fi

% Optional override for |\version| flag:
%    \begin{macrocode}
%%\providecommand{\version}{final}
%    \end{macrocode}

% Include the main document:
%    \begin{macrocode}
\input{childdoc.def}
\childdocof{cdocsamp}
%    \end{macrocode}

%\iffalse
%</samplechap1|samplechap2>
%\fi
%
%\iffalse
%<*samplechap1>
%\fi
% Some text for chapter 1:
%    \begin{macrocode}
\section{one}
some text in chapter one
%    \end{macrocode}

%\iffalse
%</samplechap1>
%\fi
% Some text for chapter 2:
%\iffalse
%<*samplechap2>
%\fi
%    \begin{macrocode}
\section{two}
more text in chapter two
%    \end{macrocode}

%\iffalse
%</samplechap2>
%\fi
%
% %%%%%%%%%%%%%%%%%%%%%%%%%%%%%%%%%%%%%%
% \paragraph{Part Include Files.}
%
% The include files are called |cdocspt3.tex| and |cdocspt4.tex|.
%
%\iffalse
%<*samplepart3|samplepart4>
%\fi

% Optional override for |\version| flag:
%    \begin{macrocode}
%%\providecommand{\version}{final}
%    \end{macrocode}

% Include the main document:
%    \begin{macrocode}
\input{childdoc.def}
\childdocby{cdocsamp}
%    \end{macrocode}

%\iffalse
%</samplepart3|samplepart4>
%\fi
%
%\iffalse
%<*samplepart3>
%\fi
% Some text for part 3:
%    \begin{macrocode}
some text in part three
%    \end{macrocode}

%\iffalse
%</samplepart3>
%\fi
% Some text for part 4:
%\iffalse
%<*samplepart4>
%\fi
%    \begin{macrocode}
more text in part four
%    \end{macrocode}

%\iffalse
%</samplepart4>
%\fi
%
% %%%%%%%%%%%%%%%%%%%%%%%%%%%%%%%%%%%%%%
% \paragraph{Forwarding for a Complete Draft.}
%
% The following forwarding file |cdocsdrf.tex|
% compiles the main document in draft mode:
%\iffalse
%<*sampledraft>
%\fi
%    \begin{macrocode}
\def\version{draft}
\input{childdoc.def}
\childdocforward{cdocsamp}
%    \end{macrocode}

%\iffalse
%</sampledraft>
%\fi
%
% %%%%%%%%%%%%%%%%%%%%%%%%%%%%%%%%%%%%%%
% \paragraph{Forwarding for Final Version of the Chapters.}
%
% The following forwarding files |cdocsfn1.tex| and |cdocsfn2.tex|
% (with identical content)
% compile the final versions of the child documents
% |cdocsch1.tex| and |cdocsch2.tex|, respectively:
%\iffalse
%<*samplefinal>
%\fi
%    \begin{macrocode}
\def\version{final}
\input{childdoc.def}
\childdocforwardprefix[cdocsamp]{cdocsfn}{cdocsch}
%    \end{macrocode}

%\iffalse
%</samplefinal>
%\fi
%
% %%%%%%%%%%%%%%%%%%%%%%%%%%%%%%%%%%%%%%
% \paragraph{Command Line Processing.}
%
% The following three command lines generate the output files
% |cdocscld|, |cdocscl1| and |cdocscl2|
% which should be identical to
% |cdocsdrf|, |cdocsch1| and |cdocsfn2|, respectively:
% \begin{center}
% \begin{tabular}{l}
% |latex -jobname cdocscld \|\\
% |  "\def\version{draft}\input{childdoc.def}\childdocforward{cdocsamp}"|\\
% |latex -jobname cdocscl1 \|\\
% |  "\input{childdoc.def}\childdocforward[cdocsamp]{cdocsch1}"|\\
% |latex -jobname cdocscl2 \|\\
% |  "\def\version{final}\input{childdoc.def}\childdocforward{cdocsch2}"|
% \end{tabular}
% \end{center}
% Note that the trailing backslash on each first line
% merely continues the input to the second line
% (for convenient cut ant paste).
% Furthermore, the command |latex| can be replaced by any
% of its alternative versions such as |pdflatex|.
%
% %%%%%%%%%%%%%%%%%%%%%%%%%%%%%%%%%%%%%%%%%%%%%%%%%%%%%%%%%%%%%%%%%%%%%%%%%%%%%%
% %%%%%%%%%%%%%%%%%%%%%%%%%%%%%%%%%%%%%%%%%%%%%%%%%%%%%%%%%%%%%%%%%%%%%%%%%%%%%%
% \section{Implementation}
%\iffalse
%<*package>
%\fi
%
% This section describes the definitions file |childdoc.def|.

% The definitions cannot be loaded using |\usepackage| or |\RequirePackage|
% which has a mechanism to prevent loading a style file more than once.
% When loading the definitions by means of |\input|
% multiple instances have to be prevented manually:
%\iffalse
%This code needs to be before the `\ProvidesFile' directive
%which is defined at the beginning of this file.
%Therefore it is also placed there and commented out here.
%</package>
%<*discard>
%\fi
%    \begin{macrocode}
\ifdefined\childdocmain\endinput\fi
%    \end{macrocode}
%\iffalse
%</discard>
%<*package>
%\fi
%
% \macro{\ifchilddoc}
% \macro{\ifchilddocmanual}
% The conditional |\ifchilddoc| tells whether a
% child (true) or main (false) document is being compiled.
% The conditional |\ifchilddocmanual| tells whether
% the |\includeonly| mechanism is used (false) or
% the selection of child files must be performed manually (true).
% The definitions initialise to false:
%    \begin{macrocode}
\newif\ifchilddoc
\newif\ifchilddocmanual
%    \end{macrocode}

% \macro{\childdocname}
% \macro{\childdocjob}
% The macro |\childdocname| stores the name of the main document
% to be compiled. The macro |\childdocjob| stores the name of
% the document on which the \LaTeX{} compiler was originally invoked.
% The content of |\jobname| cannot be compared
% to filenames specified in the source due to different catcodes.
% The following code rescans |\jobname|, stores the result
% in |\childdocname| and saves a copy in |\childdocjob|:
%    \begin{macrocode}
\edef\childdocname{\scantokens\expandafter{\jobname\noexpand}}
\let\childdocjob\childdocname
%    \end{macrocode}

% \macro{\childdocdisable}
% The macro |\childdocdisable| prevents the main file
% from being processed more than once.
% At this stage, the main document command |\childdocmain|
% is assumed to be called once again where it should do nothing.
% Any subsequent call to it should prevent
% a secondary processing of the main document
% It overwrites the forwarding commands
% |\childdocof| and |\childdocforward|
% with empty macros to prevent further inclusions of the main document:
%    \begin{macrocode}
\newcommand{\childdocdisable}
{
  \renewcommand{\childdocmain}[1]{\renewcommand{\childdocmain}[1]{\endinput}}
  \renewcommand{\childdocof}[1]{}
  \renewcommand{\childdocby}[2][]{}
  \renewcommand{\childdocforward}[2][]{}
  \renewcommand{\childdocdisable}{}
}
%    \end{macrocode}

% \macro{\childdocmain}
% The macro |\childdocmain| is to be called at the top of the main file
% with nothing or the main filename (without extension) as argument.
% First, it breaks loops.
% If the argument is not empty and does not match |\childdocname|
% (which is set by the first inclusion of |childdoc.def|),
% |\ifchilddoc| is set to true, |\includeonly| is applied to the child file
% and |\jobname| is set to the main file
% (for proper handling of |.aux| files):
%    \begin{macrocode}
\newcommand{\childdocmain}[1]
{
  \childdocdisable\childdocmain{}
  \if?#1?\else
    \begingroup
      \def\childdoctmp{#1}
      \ifx\childdoctmp\childdocname
        \def\childdoctmp{}
      \else
        \def\childdoctmp
        {
          \childdoctrue
          \includeonly{\childdocname}
          \def\childdocjob{#1}
          \def\jobname{#1}
        }
      \fi
      \expandafter
    \endgroup
    \childdoctmp
  \fi
}
%    \end{macrocode}

% \macro{\childdocof}
% The command |\childdocof| redirects
% compilation to the main file |#1|.
%    \begin{macrocode}
\newcommand{\childdocof}[1]
{
  \childdocdisable
  \childdoctrue
  \includeonly{\childdocname}
  \def\jobname{#1}
  \def\childdocjob{#1}
  \input{#1}
}
%    \end{macrocode}

% \macro{\childdocby}
% The command |\childdocby| ....
%    \begin{macrocode}
\newcommand{\childdocby}[2][]
{
  \childdocdisable
  \childdoctrue
  \childdocmanualtrue
  \if?#1?\else
    \def\jobname{#2}
  \fi
  \def\childdocjob{#2}
  \input{#2}
  \endinput
}
%    \end{macrocode}

% \macro{\childdocforward}
% The command |\childdocforward| redirects
% compilation to the main file or
% (if the optional argument is given) a child file.
% Parameters are set as if the main file
% or a child file starting with |\childdocof| was compiled.
% Then compilation is handed over to the main file:
%    \begin{macrocode}
\newcommand{\childdocforward}[2][]
{
  \begingroup
    \if?#1?
      \def\childdoctmp
      {
        \def\childdocname{#2}
        \def\childdocjob{#2}
        \def\jobname{#2}
        \input{#2}
        \endinput
      }
    \else
      \def\childdoctmp
      {
        \childdocdisable
        \def\childdocname{#2}
        \childdoctrue
        \includeonly{#2}
        \def\childdocjob{#1}
        \def\jobname{#1}
        \input{#1}
        \endinput
      }
    \fi
    \expandafter
  \endgroup
  \childdoctmp
}
%    \end{macrocode}

% \macro{\childdocforwardprefix}
% The command |\childdocforwardprefix| redirects
% compilation to the main or a child file by means of a pattern.
% The prefix |#1| in the current filename is replaced by |#2|
% and the suffix of the current filename is kept
% (it is assumed that the filename does not contain the substring `|~~~|'
% which is used as a delimiter).
% Compilation is handed over to the new file by |\childdocforward|:
%    \begin{macrocode}
\newcommand{\childdocforwardprefix}[3][]
{
  \begingroup
    \def\childdocextract #2##1~~~{\def\childdoctmp{\childdocforward[#1]{#3##1}}}
    \expandafter\childdocextract\childdocname~~~
    \expandafter
  \endgroup
  \childdoctmp
}
%    \end{macrocode}

% \macro{\childdoc}
% The deprecated macro |\childdoc| is a legacy version of |\childdocmain|:
%    \begin{macrocode}
\newcommand{\childdoc}{\childdocmain}
%    \end{macrocode}

% \macro{\childdocredirect}
% The deprecated macro |\childdocredirect| is a legacy version
% of |\childdocforward| and |\childdocforwardprefix|:
%    \begin{macrocode}
\newcommand{\childdocredirect}[2][]
{
  \begingroup
    \if?#1?
      \def\childdoctmp{\childdocforward{#2}}
    \else
      \def\childdoctmp{\childdocforwardprefix{#1}{#2}}
    \fi
    \expandafter
  \endgroup
  \childdoctmp
}
%    \end{macrocode}

%\iffalse
%</package>
%\fi
%
\endinput
|\\
|\childdocof{|\textit{main}|}|\\
\end{tabular}
\end{center}
at the top of every child file \textit{child}
which is included by |\include{|\textit{child}|}|
from within the main file
(or at least for those files to be compiled individually).
The argument \textit{main} must be the filename of the main file.

There are a couple of
considerations in setting up the main and child documents:

%%%%%%%%%%%%%%%%%%%%%%%%%%%%%%%%%%%%%%%%
\paragraph{Restrictions.}

Please note the following restrictions:
\begin{itemize}
\item
|\childdocmain| must be called with one argument \textit{main}
to ensure compatibility with earlier version of the package.
It must either be empty (|\childdocmain{}|)
or precisely match the filename of the main file in which it is specified.
See \secref{sec:detection} for further information.
\item
The filename \textit{main} must be specified without the |.tex| extension.
\item
The filename \textit{main} is case sensitive
(even in case-insensitive file systems)
due to internal string comparison.
\item
The argument \textit{main} should be fully expanded, it cannot be a macro.
\item
Subdirectories and special characters should be avoided in filenames.
\item
The command |\childdocmain{|\textit{main}|}| must be followed by a whitespace.
It should not be followed immediately by another command
or by a comment mark `|%|'.
This is because the \TeX{} parser reads the token immediately following
the argument of |\childdocmain| and puts it
at the beginning of every child section;
however, a white\-space is ignored.
\end{itemize}

%%%%%%%%%%%%%%%%%%%%%%%%%%%%%%%%%%%%%%%%
\paragraph{Content of Main File.}

It is advisable to place all content in the child files included by |\include|.
Any output contained in the main file will appear in all child documents
unless suppressed manually;
it cannot be suppressed automatically by the |\includeonly| directive
and thus should normally be avoided.
A method to include some content in the main file
by means of conditional processing is described in \secref{sec:conditional}.

%%%%%%%%%%%%%%%%%%%%%%%%%%%%%%%%%%%%%%%%
\paragraph{Page Numbering.}

When only a part of the document is compiled,
the appropriate numbering of pages
(as well as other status parameters)
is determined from the |.aux| files.
The latter contain information from previous passes.
However this information needs to propagate through
all intermediate child documents.
Therefore the page numbering in child documents may well
be inconsistent until the complete document is compiled at least once.

A useful (if unconventional) way to always ensure a consistent
page numbering is to restart the numbering in each child document
and denote the pages by `\textit{child}|.|\textit{page}'
where \textit{child} represents the chapter/section number of the child file.
This can be achieved by the command
|\numberwithin{page}{|\textit{child}|}|
of the \textsf{amsmath} package
where \textit{child} can be |chapter| or |section|
depending on the chosen structuring.
Alternatively, one can modify the macro |\thepage| appropriately
and reset the counter |page| at the start of each child file.

%%%%%%%%%%%%%%%%%%%%%%%%%%%%%%%%%%%%%%%%%%%%%%%%%%%%%%%%%%%%%%%%%%%%%%%%%%%%%%%%
\subsection{Conditional Processing}
\label{sec:conditional}

The package provides a mechanism to compile different versions
of a document. To customise the versions further some conditional processing
can come in handy to distinguish which version is being compiled.
The package provides two macros to describe the compilation context:

%%%%%%%%%%%%%%%%%%%%%%%%%%%%%%%%%%%%%%%%
\DescribeMacro{\ifchilddoc}
The conditional |\ifchilddoc| distinguishes between the compilation of
child documents and the main document:
%
\begin{center}
|\ifchilddoc |\textit{child-code}| |[|\||else |\textit{main-code}]| \||fi|
\end{center}

%%%%%%%%%%%%%%%%%%%%%%%%%%%%%%%%%%%%%%%%
\DescribeMacro{\childdocname}
\DescribeMacro{\childdocjob}
The macro |\childdocname| contains the filename (without extension)
of the main or child file being processed.
Note that |\childdocjob| will always contain the name of the main file.

%%%%%%%%%%%%%%%%%%%%%%%%%%%%%%%%%%%%%%%%
\paragraph{Title Page.}

Conditional processing can be used to include a title or banner page
in the main document when proper precautions are taken.
Importantly, the code in the main file should ensure that the page counter
(as well as other status parameters which are stored in the |.aux| files)
takes the same value after the conditional processing.
Otherwise the page numbers may take divergent values
depending on which part is compiled.

For example, a title page could be declared by:
%
\begin{center}
\begin{tabular}{l}
|\ifchilddoc\||else|\\
|\addtocounter{page}{-1}|\\
\textit{code for title page}\\
|\newpage|\\
|\||fi|
\end{tabular}
\end{center}
%
A banner page for the child documents can be generated by:
%
\begin{center}
\begin{tabular}{l}
|\ifchilddoc|\\
|\addtocounter{page}{-1}|\\
\textit{code for banner page}\\
|\newpage|\\
|\||fi|
\end{tabular}
\end{center}
%
Here one could write a message such as:
\begin{center}
|This is the part \childdocname{} of \childdocjob{}.|
\end{center}

%%%%%%%%%%%%%%%%%%%%%%%%%%%%%%%%%%%%%%%%%%%%%%%%%%%%%%%%%%%%%%%%%%%%%%%%%%%%%%%%
\subsection{Flags}
\label{sec:flags}

The package makes it easy to generate different versions
of the main or child documents.
To this end compilation flags can be defined
and assigned different default values.
They will be particularly useful in conjunction
with the forwarding mechanism described in \secref{sec:forward}.

For example, it may be useful to have a flag |\version|
which can be set to |draft| or |final|.
The document source will contain some conditional code
depending on the value of |\version|.
Suppose further, the flag should default to |final| for the main file
and to |draft| for child files
which is a natural assignment for editing the document.
This is achieved by placing the following code
in the preamble of the main document
(below the |\childdocmain| directive):
%
\begin{center}
\begin{tabular}{l}
|\ifchilddoc|\\
|\providecommand{\version}{draft}|\\
|\||else|\\
|\providecommand{\version}{final}|\\
|\||fi|
\end{tabular}
\end{center}
%
The definition by |\providecommand| makes sure
that previous definitions are not overwritten.
Further statements |\providecommand{\version}{...}|
can thus be added before the above code to override it.

For the main file, one might add a line
(between |\childdocmain| and the above block)
%
\begin{center}
|%\ifchilddoc\||else\providecommand{\version}{draft}\||fi|
\end{center}
%
which can be uncommented to produce a draft version.
Likewise one can add a line to the very top of a child file
(above the |\childdocof{|\textit{main}|}| directive)
%
\begin{center}
|%\providecommand{\version}{final}|
\end{center}
%
which can be uncommented to produce the final version of this child document.

%%%%%%%%%%%%%%%%%%%%%%%%%%%%%%%%%%%%%%%%%%%%%%%%%%%%%%%%%%%%%%%%%%%%%%%%%%%%%%%%
\subsection{Forwarding}
\label{sec:forward}

Different versions of the main or child documents
using compilation flags as described in \secref{sec:flags}
can be (permanently) stored in different files
for convenient compilation, viewing and distribution.
To this end, the package defines a command
to pass on compilation to a different file:

%%%%%%%%%%%%%%%%%%%%%%%%%%%%%%%%%%%%%%%%
\DescribeMacro{\childdocforward}
The command |\childdocforward| redirects processing to
another source file:
%
\begin{center}
\begin{tabular}{l}
|% \iffalse
%
% childdoc.dtx Copyright (C) 2017-2018 Niklas Beisert
%
% This work may be distributed and/or modified under the
% conditions of the LaTeX Project Public License, either version 1.3
% of this license or (at your option) any later version.
% The latest version of this license is in
%   http://www.latex-project.org/lppl.txt
% and version 1.3 or later is part of all distributions of LaTeX
% version 2005/12/01 or later.
%
% This work has the LPPL maintenance status `maintained'.
%
% The Current Maintainer of this work is Niklas Beisert.
%
% This work consists of the files childdoc.dtx and childdoc.ins
% and the derived files childdoc.def and cdocsamp.tex with
% cdocsch1.tex, cdocsch2.tex, cdocsdrf.tex, cdocsfn1.tex, cdocsfn2.tex.
%
%<package>\ifdefined\childdocmain\endinput\fi
%<package>\ProvidesFile{childdoc.def}[2018/12/30 v2.0 child document driver]
%<samplemain>\ProvidesFile{cdocsamp.tex}[2018/12/30 v2.0 sample for childdoc]
%<*driver>
%\ProvidesFile{childdoc.drv}[2018/12/30 v2.0 childdoc reference manual file]
\PassOptionsToClass{10pt,a4paper}{article}
\documentclass{ltxdoc}

\usepackage[margin=35mm]{geometry}
\usepackage{hyperref}
\usepackage{hyperxmp}
\usepackage[usenames]{color}

\hypersetup{colorlinks=true}
\hypersetup{pdfstartview=FitH}
\hypersetup{pdfpagemode=UseNone}
\hypersetup{pdfsource={}}
\hypersetup{pdflang={en-UK}}
\hypersetup{pdfcopyright={Copyright 2017-2018 Niklas Beisert.
  This work may be distributed and/or modified under the
  conditions of the LaTeX Project Public License, either version 1.3
  of this license or (at your option) any later version.}}
\hypersetup{pdflicenseurl={http://www.latex-project.org/lppl.txt}}
\hypersetup{pdfcontactaddress={ETH Zurich, ITP, HIT K,
  Wolfgang-Pauli-Strasse 27}}
\hypersetup{pdfcontactpostcode={8093}}
\hypersetup{pdfcontactcity={Zurich}}
\hypersetup{pdfcontactcountry={Switzerland}}
\hypersetup{pdfcontactemail={nbeisert@itp.phys.ethz.ch}}
\hypersetup{pdfcontacturl={http://people.phys.ethz.ch/\xmptilde nbeisert/}}

\newcommand{\secref}[1]{\hyperref[#1]{section \ref*{#1}}}

\parskip1ex
\parindent0pt
\let\olditemize\itemize
\def\itemize{\olditemize\parskip0pt}

\begin{document}

\title{The \textsf{childdoc} Package}
\hypersetup{pdftitle={The childdoc Package}}
\author{Niklas Beisert\\[2ex]
  Institut f\"ur Theoretische Physik\\
  Eidgen\"ossische Technische Hochschule Z\"urich\\
  Wolfgang-Pauli-Strasse 27, 8093 Z\"urich, Switzerland\\[1ex]
  \href{mailto:nbeisert@itp.phys.ethz.ch}
  {\texttt{nbeisert@itp.phys.ethz.ch}}}
\hypersetup{pdfauthor={Niklas Beisert}}
\hypersetup{pdfsubject={Manual for the LaTeX2e Package childdoc}}
\date{30 December 2018, \textsf{v2.0}}
\maketitle

\begin{abstract}\noindent
\textsf{childdoc} is a \LaTeXe{} package
that enables the direct compilation
of document sections included by |\include|
to individual files.
\end{abstract}

\begingroup
\parskip0ex
\tableofcontents
\endgroup

%%%%%%%%%%%%%%%%%%%%%%%%%%%%%%%%%%%%%%%%%%%%%%%%%%%%%%%%%%%%%%%%%%%%%%%%%%%%%%%%
%%%%%%%%%%%%%%%%%%%%%%%%%%%%%%%%%%%%%%%%%%%%%%%%%%%%%%%%%%%%%%%%%%%%%%%%%%%%%%%%
\section{Introduction}

\LaTeX{} provides a mechanism to structure a large document (such as a book)
into a main file and several child files (containing the chapters)
using the |\include| command.
This mechanism is beneficial for documents
which span hundreds of pages in order to
make the source file(s) more manageable.
Moreover, compilation can be restricted to
selected child files by means of the |\includeonly| command.
The latter feature can be used to reduce the compilation time while editing
(this was significantly more useful in the earlier days of \LaTeX{})
or to generate a smaller document which is easier to navigate.
Another application of |\includeonly| is to generate
documents consisting of selected parts of the complete document.

However, there are a few drawbacks of the plain |\include| mechanism:
\begin{itemize}
\item
The child files cannot be compiled on their own,
they can only be compiled via the main file.
A naive editing environment
(such as a text editor with an option
to have the current file processed by \LaTeX)
may require one to switch to the main file before compiling;
attempting to compile the child file produces errors.
\item
The main file must be modified (each time)
to adjust the |\includeonly| command
to the present needs. This easily leaves the main file in a messy state.
\item
The generated document will always carry the filename
of the main document. This is inconvenient if
several child files are to be compiled and
to be kept for distribution.
\end{itemize}

The present package provides a simple interface
to make child files individually compilable by \LaTeX{}.
Compiling a child file then has the same effect as compiling
the main file with an |\includeonly| command
to select the appropriate child.
Moreover the generated document will carry the name of the child
rather than the main file.
This resolves all three above issues.

This feature is meant to make the editing of books,
thesis documents and lecture notes somewhat more convenient.
However, the package can also be used efficiently for
composing a series of documents (such as exercise sheets)
which are typically distributed individually.
It then assists the author in generating the individual documents
(potentially in different versions)
as well as a document containing the collected series.
Another application is in developing style files
or other kinds of included material
where compilation of the style file could redirect
to a sample or test file.

%%%%%%%%%%%%%%%%%%%%%%%%%%%%%%%%%%%%%%%%%%%%%%%%%%%%%%%%%%%%%%%%%%%%%%%%%%%%%%%%
%%%%%%%%%%%%%%%%%%%%%%%%%%%%%%%%%%%%%%%%%%%%%%%%%%%%%%%%%%%%%%%%%%%%%%%%%%%%%%%%
\section{Usage}

First of all, the package \textsf{childdoc} is \emph{not} a standard
\LaTeXe{} |.sty| style file! Therefore it needs to be invoked in
a non-standard way.

%%%%%%%%%%%%%%%%%%%%%%%%%%%%%%%%%%%%%%%%%%%%%%%%%%%%%%%%%%%%%%%%%%%%%%%%%%%%%%%%
\subsection{Included Files}
\label{sec:include}

%%%%%%%%%%%%%%%%%%%%%%%%%%%%%%%%%%%%%%%%
\DescribeMacro{\childdocmain}
To use the package, add the commands
\begin{center}
\begin{tabular}{l}
|\input{childdoc.def}|\\
|\childdocmain{}|\\
\end{tabular}
\end{center}
at the very top of the main \LaTeX{} file,
in particular \emph{before} the |\documentclass| statement!
The argument of |\childdocmain| should be left empty
(but it must be present).

%%%%%%%%%%%%%%%%%%%%%%%%%%%%%%%%%%%%%%%%
\DescribeMacro{\childdocof}
Furthermore, add the commands
\begin{center}
\begin{tabular}{l}
|\input{childdoc.def}|\\
|\childdocof{|\textit{main}|}|\\
\end{tabular}
\end{center}
at the top of every child file \textit{child}
which is included by |\include{|\textit{child}|}|
from within the main file
(or at least for those files to be compiled individually).
The argument \textit{main} must be the filename of the main file.

There are a couple of
considerations in setting up the main and child documents:

%%%%%%%%%%%%%%%%%%%%%%%%%%%%%%%%%%%%%%%%
\paragraph{Restrictions.}

Please note the following restrictions:
\begin{itemize}
\item
|\childdocmain| must be called with one argument \textit{main}
to ensure compatibility with earlier version of the package.
It must either be empty (|\childdocmain{}|)
or precisely match the filename of the main file in which it is specified.
See \secref{sec:detection} for further information.
\item
The filename \textit{main} must be specified without the |.tex| extension.
\item
The filename \textit{main} is case sensitive
(even in case-insensitive file systems)
due to internal string comparison.
\item
The argument \textit{main} should be fully expanded, it cannot be a macro.
\item
Subdirectories and special characters should be avoided in filenames.
\item
The command |\childdocmain{|\textit{main}|}| must be followed by a whitespace.
It should not be followed immediately by another command
or by a comment mark `|%|'.
This is because the \TeX{} parser reads the token immediately following
the argument of |\childdocmain| and puts it
at the beginning of every child section;
however, a white\-space is ignored.
\end{itemize}

%%%%%%%%%%%%%%%%%%%%%%%%%%%%%%%%%%%%%%%%
\paragraph{Content of Main File.}

It is advisable to place all content in the child files included by |\include|.
Any output contained in the main file will appear in all child documents
unless suppressed manually;
it cannot be suppressed automatically by the |\includeonly| directive
and thus should normally be avoided.
A method to include some content in the main file
by means of conditional processing is described in \secref{sec:conditional}.

%%%%%%%%%%%%%%%%%%%%%%%%%%%%%%%%%%%%%%%%
\paragraph{Page Numbering.}

When only a part of the document is compiled,
the appropriate numbering of pages
(as well as other status parameters)
is determined from the |.aux| files.
The latter contain information from previous passes.
However this information needs to propagate through
all intermediate child documents.
Therefore the page numbering in child documents may well
be inconsistent until the complete document is compiled at least once.

A useful (if unconventional) way to always ensure a consistent
page numbering is to restart the numbering in each child document
and denote the pages by `\textit{child}|.|\textit{page}'
where \textit{child} represents the chapter/section number of the child file.
This can be achieved by the command
|\numberwithin{page}{|\textit{child}|}|
of the \textsf{amsmath} package
where \textit{child} can be |chapter| or |section|
depending on the chosen structuring.
Alternatively, one can modify the macro |\thepage| appropriately
and reset the counter |page| at the start of each child file.

%%%%%%%%%%%%%%%%%%%%%%%%%%%%%%%%%%%%%%%%%%%%%%%%%%%%%%%%%%%%%%%%%%%%%%%%%%%%%%%%
\subsection{Conditional Processing}
\label{sec:conditional}

The package provides a mechanism to compile different versions
of a document. To customise the versions further some conditional processing
can come in handy to distinguish which version is being compiled.
The package provides two macros to describe the compilation context:

%%%%%%%%%%%%%%%%%%%%%%%%%%%%%%%%%%%%%%%%
\DescribeMacro{\ifchilddoc}
The conditional |\ifchilddoc| distinguishes between the compilation of
child documents and the main document:
%
\begin{center}
|\ifchilddoc |\textit{child-code}| |[|\||else |\textit{main-code}]| \||fi|
\end{center}

%%%%%%%%%%%%%%%%%%%%%%%%%%%%%%%%%%%%%%%%
\DescribeMacro{\childdocname}
\DescribeMacro{\childdocjob}
The macro |\childdocname| contains the filename (without extension)
of the main or child file being processed.
Note that |\childdocjob| will always contain the name of the main file.

%%%%%%%%%%%%%%%%%%%%%%%%%%%%%%%%%%%%%%%%
\paragraph{Title Page.}

Conditional processing can be used to include a title or banner page
in the main document when proper precautions are taken.
Importantly, the code in the main file should ensure that the page counter
(as well as other status parameters which are stored in the |.aux| files)
takes the same value after the conditional processing.
Otherwise the page numbers may take divergent values
depending on which part is compiled.

For example, a title page could be declared by:
%
\begin{center}
\begin{tabular}{l}
|\ifchilddoc\||else|\\
|\addtocounter{page}{-1}|\\
\textit{code for title page}\\
|\newpage|\\
|\||fi|
\end{tabular}
\end{center}
%
A banner page for the child documents can be generated by:
%
\begin{center}
\begin{tabular}{l}
|\ifchilddoc|\\
|\addtocounter{page}{-1}|\\
\textit{code for banner page}\\
|\newpage|\\
|\||fi|
\end{tabular}
\end{center}
%
Here one could write a message such as:
\begin{center}
|This is the part \childdocname{} of \childdocjob{}.|
\end{center}

%%%%%%%%%%%%%%%%%%%%%%%%%%%%%%%%%%%%%%%%%%%%%%%%%%%%%%%%%%%%%%%%%%%%%%%%%%%%%%%%
\subsection{Flags}
\label{sec:flags}

The package makes it easy to generate different versions
of the main or child documents.
To this end compilation flags can be defined
and assigned different default values.
They will be particularly useful in conjunction
with the forwarding mechanism described in \secref{sec:forward}.

For example, it may be useful to have a flag |\version|
which can be set to |draft| or |final|.
The document source will contain some conditional code
depending on the value of |\version|.
Suppose further, the flag should default to |final| for the main file
and to |draft| for child files
which is a natural assignment for editing the document.
This is achieved by placing the following code
in the preamble of the main document
(below the |\childdocmain| directive):
%
\begin{center}
\begin{tabular}{l}
|\ifchilddoc|\\
|\providecommand{\version}{draft}|\\
|\||else|\\
|\providecommand{\version}{final}|\\
|\||fi|
\end{tabular}
\end{center}
%
The definition by |\providecommand| makes sure
that previous definitions are not overwritten.
Further statements |\providecommand{\version}{...}|
can thus be added before the above code to override it.

For the main file, one might add a line
(between |\childdocmain| and the above block)
%
\begin{center}
|%\ifchilddoc\||else\providecommand{\version}{draft}\||fi|
\end{center}
%
which can be uncommented to produce a draft version.
Likewise one can add a line to the very top of a child file
(above the |\childdocof{|\textit{main}|}| directive)
%
\begin{center}
|%\providecommand{\version}{final}|
\end{center}
%
which can be uncommented to produce the final version of this child document.

%%%%%%%%%%%%%%%%%%%%%%%%%%%%%%%%%%%%%%%%%%%%%%%%%%%%%%%%%%%%%%%%%%%%%%%%%%%%%%%%
\subsection{Forwarding}
\label{sec:forward}

Different versions of the main or child documents
using compilation flags as described in \secref{sec:flags}
can be (permanently) stored in different files
for convenient compilation, viewing and distribution.
To this end, the package defines a command
to pass on compilation to a different file:

%%%%%%%%%%%%%%%%%%%%%%%%%%%%%%%%%%%%%%%%
\DescribeMacro{\childdocforward}
The command |\childdocforward| redirects processing to
another source file:
%
\begin{center}
\begin{tabular}{l}
|\input{childdoc.def}|\\
|\childdocforward[|\textit{main}|]{|\textit{dest}|}|\\
\end{tabular}
\end{center}
%
The argument \textit{dest} is the destination file
(without extension).
It should be the main file or one of the child files.
Note that further \textsf{childdoc} directives
such as |\childdocof| and |\childdocforward|
in the indicated file will be processed in this form.
The optional argument \textit{main}
passes on directly to the main file \textit{main}
while pretending to compile the child \textit{dest}.
This form behaves as if \textit{dest}
issues |\childdocof{|\textit{main}|}| right away,
and no further \textsf{childdoc} directives will be processed.

%%%%%%%%%%%%%%%%%%%%%%%%%%%%%%%%%%%%%%%%
\DescribeMacro{\...prefix}
In the alternative form |\childdocforwardprefix|,
%
\begin{center}
\begin{tabular}{l}
|\input{childdoc.def}|\\
|\childdocforwardprefix[|\textit{main}|]{|\textit{prefix}|}{|\textit{dest}|}|
\end{tabular}
\end{center}
%
the destination file is determined by a pattern
depending on the current file:
To make this work, the current file must be called
`{\textit{prefix}\hspace{0.2em}\textit{suffix}}'
with \textit{prefix} matching precisely the argument.
Processing is then passed on to the file
`{\textit{dest}\hspace{0.2em}\textit{suffix}}'.
Surely, the same effect is achieved by
directly specifying the
argument `{\textit{dest}\hspace{0.2em}\textit{suffix}}'
in the first form.
However, that requires to set up a different file
for each child. With the alternative form of the command
all these files can have exactly the same content
which simplifies setting them up and maintaining them.

For example, the following file |draft.tex|
with a compilation flag |\version| as described in \secref{sec:flags}
compiles the main document as a draft:
%
\begin{center}
\begin{tabular}{l}
|\def\version{draft}|\\
|\input{childdoc.def}|\\
|\childdocforward{|\textit{main}|}|
\end{tabular}
\end{center}
%
Likewise, the following files |final|\textit{nn}|.tex|
compile the final version of the child document
|child|\textit{nn}|.tex|:
%
\begin{center}
\begin{tabular}{l}
|\def\version{final}|\\
|\input{childdoc.def}|\\
|\childdocforwardprefix{final}{child}|
\end{tabular}
\end{center}
%

Note that when several versions of a main file and/or of each child file
are to be generated, it may be convenient to set up a |Makefile| or
shell script to automatise the process.

%%%%%%%%%%%%%%%%%%%%%%%%%%%%%%%%%%%%%%%%%%%%%%%%%%%%%%%%%%%%%%%%%%%%%%%%%%%%%%%%
\subsection{Command Line Processing}
\label{sec:commandline}

The effect of redirection files can also be achieved by invoking
the \LaTeX{} compiler with a more elaborate command line.
Most conveniently this should be done as part
of a shell script or a |Makefile|.

When using \textsf{childdoc} in the main file, the following
command lines effectively perform a redirection
(note that depending on the shell being used,
backslashes may have to be doubled: `|\|' $\to$ `|\\|'):
%
\begin{center}
|... -jobname "|\textit{target}|" |\\|"|[\textit{flags}]%
|\input{childdoc.def}\childdocforward[|\textit{main}|]{|\textit{dest}|}"|
\end{center}
%
Here \textit{target} is the name of the output file,
\textit{main} is the name of the main file
and \textit{dest} is the name of the main or child file to be processed
(all filenames without extensions).
The optional argument \textit{main} can be omitted
if \textit{main} matches \textit{dest}.
Optionally, compilation \textit{flags} can be defined via |\def| commands.
This command line makes the \TeX{} engine believe
it is compiling the file \textit{target}
whose content is specified as the latter parameter.
The provided code then forwards the processing to
\textit{main} or \textit{dest} as described in \secref{sec:forward}.

%%%%%%%%%%%%%%%%%%%%%%%%%%%%%%%%%%%%%%%%%%%%%%%%%%%%%%%%%%%%%%%%%%%%%%%%%%%%%%%%
\subsection{Include by Input}
\label{sec:input}

Including child documents by |\include| has some restrictions by design.
Most notably, the content of a child document always occupies
its own set of pages; pages cannot be shared between child documents.
Usually, this behaviour makes perfect sense
because each child document contain an essential part of the document.
However, in some situations it may be desirable to compose
a document from a collection of parts
without having mandatory page breaks between then.
For this case, the package
provides a mechanism to include parts
by |\input| which can also be processed individually.
However, by construction this mechanism
requires manual handling of the content to be output.

%%%%%%%%%%%%%%%%%%%%%%%%%%%%%%%%%%%%%%%%
\DescribeMacro{\ifchilddocmanual}
The main file should be prepared as usual, see \secref{sec:include}.
However, the document body must make a distinction
between processing of an individual part and of the main document, e.g.:
%
\begin{center}
\begin{tabular}{l}
|\ifchilddocmanual|\\
|\input{\childdocname}|\\
|\||else|\\
\textit{document body with }|\input{|\textit{part}|}|\\
|\||fi|
\end{tabular}
\end{center}
%
The conditional |\ifchilddocmanual| is true whenever
a part to be included by |\input| is being compiled,
and the name of the part is stored in |\childdocname|.

%%%%%%%%%%%%%%%%%%%%%%%%%%%%%%%%%%%%%%%%
\DescribeMacro{\childdocby}
Each part to be included by |\input| should start with:
%
\begin{center}
\begin{tabular}{l}
|\input{childdoc.def}|\\
|\childdocby{|\textit{main}|}|\\
\end{tabular}
\end{center}
%
The directive |\childdocby| is similar to |\childdocof|
described in \secref{sec:include},
but the subsequent selection of content must be done manually.
To that end, both |\ifchilddoc| and |\ifchilddocmanual|
will be true upon processing of a part,
and the name of the part is stored in |\childdocname|.
Note that |\jobname| will be set to the filename of the current part
so that each part receives an individual |.aux| file
that does not interfere with the |.aux| file(s) of the main document.
This behaviour can be altered by the alternative form
|\childdocby[*]{|\textit{main}|}| (with a non-empty optional argument)
which uses the |.aux| file of the main document
by setting |\jobname| to \textit{main}.

%%%%%%%%%%%%%%%%%%%%%%%%%%%%%%%%%%%%%%%%%%%%%%%%%%%%%%%%%%%%%%%%%%%%%%%%%%%%%%%%
\subsection{Driver Development}
\label{sec:driver}

The \textsf{childdoc} mechanism can also be use for the development
of definition files such as \LaTeX{} styles or classes.
This case differs from the above setup with multiple parts
included by |\include| in that no |\includeonly| should be invoked.
This can be achieved by starting the include file
(before |\ProvidesPackage|) with:
%
\begin{center}
\begin{tabular}{l}
|\input{childdoc.def}|\\
|\childdocforward{|\textit{main}|}|\\
\end{tabular}
\end{center}
%
or alternatively with:
%
\begin{center}
\begin{tabular}{l}
|\input{childdoc.def}|\\
|\childdocby{|\textit{main}|}|\\
\end{tabular}
\end{center}
%
Both forms have slightly different effects as described above.
The main file is prepared as usual, see \secref{sec:include}.

%%%%%%%%%%%%%%%%%%%%%%%%%%%%%%%%%%%%%%%%%%%%%%%%%%%%%%%%%%%%%%%%%%%%%%%%%%%%%%%%
\subsection{Legacy Detection}
\label{sec:detection}

The directive |\childdocmain| in the main file can detect
whether the complete document or merely a child is to be compiled
even without using the directive |\childdocof|.
This method is deprecated because it is less robust
and there is no compelling reason to use it;
it is merely provided for backward compatibility
and it may be removed in future versions.

If the detection mechanism is to be used,
it is mandatory to correctly specify
the filename of the main file as the argument of |\childdocmain|:
%
\begin{center}
\begin{tabular}{l}
|\input{childdoc.def}|\\
|\childdocmain{|\textit{main}|}|\\
\end{tabular}
\end{center}
%
If |\jobname| does not match the argument \textit{main} of |\childdocmain|,
it is assumed that |\jobname| points to the child file to be compiled.
When using |\childdocmain| with the main file specified as argument,
it suffices to start a child file
with just |\input{|\textit{main}|}|
without loading of the package and using |\childdocof|.
If instead all processing is done
with the appropriate \textsf{childdoc} directives,
the argument of \textit{main} of |\childdocmain| can be empty.

An alternative version of the command line processing described
in \secref{sec:commandline} using the detection mechanism reads:
%
\begin{center}
|... -jobname "|\textit{target}|" "|[\textit{flags}]%
[|\def\jobname{|\textit{dest}|}|]|\input{|\textit{main}|}"|
\end{center}

%%%%%%%%%%%%%%%%%%%%%%%%%%%%%%%%%%%%%%%%%%%%%%%%%%%%%%%%%%%%%%%%%%%%%%%%%%%%%%%%
\subsection{Manual Code}
\label{sec:manual}

In case one cannot be certain whether the definitions file |childdoc.def|
is installed on the target \TeX{} distribution
and one prefers not to ship it,
it is conceivable to paste a few relevant commands into the sources.

To that end, drop all statements |\input{childdoc.def}|
and perform the replacements as outlined below.
Instead of |\childdocmain{|\textit{main}|}| add the following code
to the top of the main file:
%
\begin{center}
\begin{tabular}{l}
|\||ifdefined\childdocname\endinput\||fi\newif\ifchilddoc|\\
|\edef\childdocname{\scantokens\expandafter{\jobname\noexpand}}|\\
|\def\childdocmain{|\textit{main}|}\||ifx\childdocmain\childdocname\||else|\\
|\childdoctrue\includeonly{\childdocname}\let\jobname\childdocmain\||fi|\\
\end{tabular}
\end{center}
%
Instead of |\childdocof{|\textit{main}|}| just include the main file
at the top of each child file:
%
\begin{center}
|\input{|\textit{main}|}|
\end{center}
%
A simple redirection |\childdocforward{|\textit{dest}|}| is achieved by:
%
\begin{center}
|\def\jobname{|\textit{dest}|}\input{\jobname}|
\end{center}
%
The redirection with prefix
|\childdocforwardprefix[|\textit{prefix}|]{|\textit{dest}|}|
is accomplished by:
%
\begin{center}
\begin{tabular}{l}
|{\edef\jobname{\scantokens\expandafter{\jobname\noexpand}}|\\
|\def\redirectjob |\textit{prefix}|#1~~~{\gdef\jobname{|\textit{dest}|#1}}|\\
|\expandafter\redirectjob\jobname~~~}\input{\jobname}|
\end{tabular}
\end{center}

In an alternative approach,
child documents can be compiled by a specific command line
without additional code or specific definitions:
%
\begin{center}
|... -jobname "|\textit{target}|" "|[\textit{flags}]%
|\includeonly{|\textit{dest}|}\input{|\textit{main}|}"|
\end{center}
%

%%%%%%%%%%%%%%%%%%%%%%%%%%%%%%%%%%%%%%%%%%%%%%%%%%%%%%%%%%%%%%%%%%%%%%%%%%%%%%%%
%%%%%%%%%%%%%%%%%%%%%%%%%%%%%%%%%%%%%%%%%%%%%%%%%%%%%%%%%%%%%%%%%%%%%%%%%%%%%%%%
\section{Information}

%%%%%%%%%%%%%%%%%%%%%%%%%%%%%%%%%%%%%%%%%%%%%%%%%%%%%%%%%%%%%%%%%%%%%%%%%%%%%%%%
\subsection{Copyright}

Copyright \copyright{} 2017--2018 Niklas Beisert

This work may be distributed and/or modified under the
conditions of the \LaTeX{} Project Public License, either version 1.3
of this license or (at your option) any later version.
The latest version of this license is in
  \url{http://www.latex-project.org/lppl.txt}
and version 1.3 or later is part of all distributions of \LaTeX{}
version 2005/12/01 or later.

This work has the LPPL maintenance status `maintained'.

The Current Maintainer of this work is Niklas Beisert.

This work consists of the files |README.txt|, |childdoc.ins| and |childdoc.dtx|
as well as the derived files |childdoc.def|, |cdocsamp.tex|
with |cdocsch1.tex|, |cdocsch2.tex|, |cdocspt3.tex|, |cdocspt4.tex|,
|cdocsdrf.tex|, |cdocsfn1.tex|, |cdocsfn2.tex|
as well as |childdoc.pdf|.

%%%%%%%%%%%%%%%%%%%%%%%%%%%%%%%%%%%%%%%%%%%%%%%%%%%%%%%%%%%%%%%%%%%%%%%%%%%%%%%%
\subsection{Files and Installation}

The package consists of the files:
%
\begin{center}
\begin{tabular}{ll}
    |README.txt|   & readme file \\
    |childdoc.ins| & installation file \\
    |childdoc.dtx| & source file \\
    |childdoc.def| & definition file \\
    |cdocsamp.tex| & sample main file \\
    |cdocsch1.tex| & sample include file \\
    |cdocsch2.tex| & sample include file \\
    |cdocspt3.tex| & sample part file \\
    |cdocspt4.tex| & sample part file \\
    |cdocsdrf.tex| & sample redirection file \\
    |cdocsfn1.tex| & sample redirection file \\
    |cdocsfn2.tex| & sample redirection file \\
    |childdoc.pdf| & manual
\end{tabular}
\end{center}
%
The distribution consists of the files
|README.txt|, |childdoc.ins| and |childdoc.dtx|.
%
\begin{itemize}
\item
Run (pdf)\LaTeX{} on |childdoc.dtx|
to compile the manual |childdoc.pdf| (this file).
\item
Run \LaTeX{} on |childdoc.ins| to create the definitions file |childdoc.def|
and the sample |cdocsamp.tex| with include files
|cdocsch1.tex|, |cdocsch2.tex|, |cdocspt3.tex|, |cdocspt4.tex|,
|cdocsdrf.tex|, |cdocsfn1.tex|, |cdocsfn2.tex|.
Then copy the file |childdoc.def| to an appropriate directory of your \LaTeX{}
distribution, e.g.\ \textit{texmf-root}|/tex/latex/childdoc|.
\end{itemize}

%%%%%%%%%%%%%%%%%%%%%%%%%%%%%%%%%%%%%%%%%%%%%%%%%%%%%%%%%%%%%%%%%%%%%%%%%%%%%%%%
\subsection{Related CTAN Packages}

There are several other packages which offer a similar functionality:
%
\begin{itemize}
\item
The packages
\href{http://ctan.org/pkg/docmute}{\textsf{docmute}},
\href{http://ctan.org/pkg/includex}{\textsf{includex}} and
\href{http://ctan.org/pkg/standalone}{\textsf{standalone}}
provide commands to include only the document body of
a child file thus allowing both files to be compiled individually.
\item
The packages \href{http://ctan.org/pkg/subdocs}{\textsf{subdocs}}
and \href{http://ctan.org/pkg/subfiles}{\textsf{subfiles}}
provide structures in which the main and child documents can be
encapsulated and allowing them to be compiled individually.
The inclusion mechanism is different from the conventional |\include|.
\item
The package \href{http://ctan.org/pkg/combine}{\textsf{combine}}
is an elaborate solution to combine several documents into one.
\end{itemize}
%
See also the CTAN topic \href{http://ctan.org/topic/subdocs}{\textsf{subdocs}}
for further related packages.
The present package differs from the above solutions in that
a document structure constructed with the conventional |\include| mechanism
just needs two extra commands at the top of every file
such that all constituent files can be compiled individually.

%%%%%%%%%%%%%%%%%%%%%%%%%%%%%%%%%%%%%%%%%%%%%%%%%%%%%%%%%%%%%%%%%%%%%%%%%%%%%%%%
%\subsection{Feature Suggestions}
%
%The following is a list of features which may be useful for future
%versions of this package:
%%
%\begin{itemize}
%\item
%\ldots
%\end{itemize}

%%%%%%%%%%%%%%%%%%%%%%%%%%%%%%%%%%%%%%%%%%%%%%%%%%%%%%%%%%%%%%%%%%%%%%%%%%%%%%%%
\subsection{Revision History}

%%%%%%%%%%%%%%%%%%%%%%%%%%%%%%%%%%%%%%%%
\paragraph{v2.0:} 2018/12/30

\begin{itemize}
\item
immediate forward processing
\item
added |\childdocby| mechanism
\item
manual restructured
\end{itemize}

%%%%%%%%%%%%%%%%%%%%%%%%%%%%%%%%%%%%%%%%
\paragraph{v1.6:} 2018/01/17

\begin{itemize}
\item
application for development of include files
\item
corrections to manual
\end{itemize}

%%%%%%%%%%%%%%%%%%%%%%%%%%%%%%%%%%%%%%%%
\paragraph{v1.5:} 2017/05/21

\begin{itemize}
\item
more complete structuring introduced
\item
|\childdocof| introduced
\item
|\childdoc| renamed to |\childdocmain|
\item
|\childredirect| renamed to |\childdocforward| and |\childdocforwardprefix|
and functionality expanded
\end{itemize}

%%%%%%%%%%%%%%%%%%%%%%%%%%%%%%%%%%%%%%%%
\paragraph{v1.0:} 2017/04/27

\begin{itemize}
\item
manual and install package
\item
first version published on CTAN
\end{itemize}

%%%%%%%%%%%%%%%%%%%%%%%%%%%%%%%%%%%%%%%%
\paragraph{v0.6:} 2017/04/26

\begin{itemize}
\item
redirection mechanism added
\end{itemize}

%%%%%%%%%%%%%%%%%%%%%%%%%%%%%%%%%%%%%%%%
\paragraph{v0.5:} 2017/04/26

\begin{itemize}
\item
functionality in definition file
\end{itemize}


%%%%%%%%%%%%%%%%%%%%%%%%%%%%%%%%%%%%%%%%%%%%%%%%%%%%%%%%%%%%%%%%%%%%%%%%%%%%%%%%
%%%%%%%%%%%%%%%%%%%%%%%%%%%%%%%%%%%%%%%%%%%%%%%%%%%%%%%%%%%%%%%%%%%%%%%%%%%%%%%%
%%%%%%%%%%%%%%%%%%%%%%%%%%%%%%%%%%%%%%%%%%%%%%%%%%%%%%%%%%%%%%%%%%%%%%%%%%%%%%%%
\appendix

\settowidth\MacroIndent{\rmfamily\scriptsize 000\ }

 \DocInput{childdoc.dtx}

\end{document}
%</driver>
% \fi
%
% %%%%%%%%%%%%%%%%%%%%%%%%%%%%%%%%%%%%%%%%%%%%%%%%%%%%%%%%%%%%%%%%%%%%%%%%%%%%%%
% %%%%%%%%%%%%%%%%%%%%%%%%%%%%%%%%%%%%%%%%%%%%%%%%%%%%%%%%%%%%%%%%%%%%%%%%%%%%%%
% \section{Sample}
%\iffalse
%<*samplemain>
%\fi
%
% The following presents a sample document
% with two chapters, two parts, a title page,
% a compile flag as well as three forwarding files to set the flag.
% It consists of eight |.tex| files:
% \begin{center}
% \begin{tabular}{ll}
% |cdocsamp.tex|&main file\\
% |cdocsch1.tex|&include file for chapter 1\\
% |cdocsch2.tex|&include file for chapter 2\\
% |cdocspt3.tex|&include file for part 3\\
% |cdocspt4.tex|&include file for part 4\\
% |cdocsdrf.tex|&forwarding file for main file in draft mode\\
% |cdocsfi1.tex|&forwarding file for final version of chapter 1\\
% |cdocsfi2.tex|&forwarding file for final version of chapter 2\\
% \end{tabular}
% \end{center}
% Each of the eight files can be compiled directly by the \LaTeX{} compiler.
%
% %%%%%%%%%%%%%%%%%%%%%%%%%%%%%%%%%%%%%%
% \paragraph{Main File.}
%
% The main file is called |cdocsamp.tex|.
%
% Load the \textsf{childdoc} definitions and
% declare the filename for the main document:
%    \begin{macrocode}
\input{childdoc.def}
\childdocmain{}
%    \end{macrocode}

% Optional override for |\version| flag:
%    \begin{macrocode}
%%\ifchilddoc\else\providecommand{\version}{draft}\fi
%    \end{macrocode}

% Define the default values for the |\version| flag
% (|final| for the main file and |draft| for childs):
%    \begin{macrocode}
\ifchilddoc
\providecommand{\version}{draft}
\else
\providecommand{\version}{final}
\fi
%    \end{macrocode}

% Load the standard document class:
%    \begin{macrocode}
\documentclass[12pt]{article}
%    \end{macrocode}

% Start the document body:
%    \begin{macrocode}
\begin{document}
%    \end{macrocode}

% Declare a title page.
% Print title, part of document being processed and version flag:
%    \begin{macrocode}
\addtocounter{page}{-1}
\begin{center}
{\LARGE\bfseries{}childdoc example\par}
\vspace{1cm}
\ifchilddoc
\ifchilddocmanual part\else chapter\fi:
`\childdocname' of `\childdocjob'\par
\else
main document: `\childdocjob'\par
\fi
version: \version\par
\end{center}
\newpage
%    \end{macrocode}

% Manually include selected file,
% otherwise process as usual:
%    \begin{macrocode}
\ifchilddocmanual
\section*{part `\childdocname'}
\input{\childdocname}
\else
%    \end{macrocode}

% Include the two chapters:
%    \begin{macrocode}
\include{cdocsch1}
\include{cdocsch2}
%    \end{macrocode}

% Include the two parts unless only chapters should be displayed:
%    \begin{macrocode}
\ifchilddoc\else
\section{part three}
\input{cdocspt3}
\section{part four}
\input{cdocspt4}
\fi
%    \end{macrocode}

% Process as usual until here:
%    \begin{macrocode}
\fi
%    \end{macrocode}

% End of document body:
%    \begin{macrocode}
\end{document}
%    \end{macrocode}
%\iffalse
%</samplemain>
%\fi
%
% %%%%%%%%%%%%%%%%%%%%%%%%%%%%%%%%%%%%%%
% \paragraph{Chapter Include Files.}
%
% The include files are called |cdocsch1.tex| and |cdocsch2.tex|.
%
%\iffalse
%<*samplechap1|samplechap2>
%\fi

% Optional override for |\version| flag:
%    \begin{macrocode}
%%\providecommand{\version}{final}
%    \end{macrocode}

% Include the main document:
%    \begin{macrocode}
\input{childdoc.def}
\childdocof{cdocsamp}
%    \end{macrocode}

%\iffalse
%</samplechap1|samplechap2>
%\fi
%
%\iffalse
%<*samplechap1>
%\fi
% Some text for chapter 1:
%    \begin{macrocode}
\section{one}
some text in chapter one
%    \end{macrocode}

%\iffalse
%</samplechap1>
%\fi
% Some text for chapter 2:
%\iffalse
%<*samplechap2>
%\fi
%    \begin{macrocode}
\section{two}
more text in chapter two
%    \end{macrocode}

%\iffalse
%</samplechap2>
%\fi
%
% %%%%%%%%%%%%%%%%%%%%%%%%%%%%%%%%%%%%%%
% \paragraph{Part Include Files.}
%
% The include files are called |cdocspt3.tex| and |cdocspt4.tex|.
%
%\iffalse
%<*samplepart3|samplepart4>
%\fi

% Optional override for |\version| flag:
%    \begin{macrocode}
%%\providecommand{\version}{final}
%    \end{macrocode}

% Include the main document:
%    \begin{macrocode}
\input{childdoc.def}
\childdocby{cdocsamp}
%    \end{macrocode}

%\iffalse
%</samplepart3|samplepart4>
%\fi
%
%\iffalse
%<*samplepart3>
%\fi
% Some text for part 3:
%    \begin{macrocode}
some text in part three
%    \end{macrocode}

%\iffalse
%</samplepart3>
%\fi
% Some text for part 4:
%\iffalse
%<*samplepart4>
%\fi
%    \begin{macrocode}
more text in part four
%    \end{macrocode}

%\iffalse
%</samplepart4>
%\fi
%
% %%%%%%%%%%%%%%%%%%%%%%%%%%%%%%%%%%%%%%
% \paragraph{Forwarding for a Complete Draft.}
%
% The following forwarding file |cdocsdrf.tex|
% compiles the main document in draft mode:
%\iffalse
%<*sampledraft>
%\fi
%    \begin{macrocode}
\def\version{draft}
\input{childdoc.def}
\childdocforward{cdocsamp}
%    \end{macrocode}

%\iffalse
%</sampledraft>
%\fi
%
% %%%%%%%%%%%%%%%%%%%%%%%%%%%%%%%%%%%%%%
% \paragraph{Forwarding for Final Version of the Chapters.}
%
% The following forwarding files |cdocsfn1.tex| and |cdocsfn2.tex|
% (with identical content)
% compile the final versions of the child documents
% |cdocsch1.tex| and |cdocsch2.tex|, respectively:
%\iffalse
%<*samplefinal>
%\fi
%    \begin{macrocode}
\def\version{final}
\input{childdoc.def}
\childdocforwardprefix[cdocsamp]{cdocsfn}{cdocsch}
%    \end{macrocode}

%\iffalse
%</samplefinal>
%\fi
%
% %%%%%%%%%%%%%%%%%%%%%%%%%%%%%%%%%%%%%%
% \paragraph{Command Line Processing.}
%
% The following three command lines generate the output files
% |cdocscld|, |cdocscl1| and |cdocscl2|
% which should be identical to
% |cdocsdrf|, |cdocsch1| and |cdocsfn2|, respectively:
% \begin{center}
% \begin{tabular}{l}
% |latex -jobname cdocscld \|\\
% |  "\def\version{draft}\input{childdoc.def}\childdocforward{cdocsamp}"|\\
% |latex -jobname cdocscl1 \|\\
% |  "\input{childdoc.def}\childdocforward[cdocsamp]{cdocsch1}"|\\
% |latex -jobname cdocscl2 \|\\
% |  "\def\version{final}\input{childdoc.def}\childdocforward{cdocsch2}"|
% \end{tabular}
% \end{center}
% Note that the trailing backslash on each first line
% merely continues the input to the second line
% (for convenient cut ant paste).
% Furthermore, the command |latex| can be replaced by any
% of its alternative versions such as |pdflatex|.
%
% %%%%%%%%%%%%%%%%%%%%%%%%%%%%%%%%%%%%%%%%%%%%%%%%%%%%%%%%%%%%%%%%%%%%%%%%%%%%%%
% %%%%%%%%%%%%%%%%%%%%%%%%%%%%%%%%%%%%%%%%%%%%%%%%%%%%%%%%%%%%%%%%%%%%%%%%%%%%%%
% \section{Implementation}
%\iffalse
%<*package>
%\fi
%
% This section describes the definitions file |childdoc.def|.

% The definitions cannot be loaded using |\usepackage| or |\RequirePackage|
% which has a mechanism to prevent loading a style file more than once.
% When loading the definitions by means of |\input|
% multiple instances have to be prevented manually:
%\iffalse
%This code needs to be before the `\ProvidesFile' directive
%which is defined at the beginning of this file.
%Therefore it is also placed there and commented out here.
%</package>
%<*discard>
%\fi
%    \begin{macrocode}
\ifdefined\childdocmain\endinput\fi
%    \end{macrocode}
%\iffalse
%</discard>
%<*package>
%\fi
%
% \macro{\ifchilddoc}
% \macro{\ifchilddocmanual}
% The conditional |\ifchilddoc| tells whether a
% child (true) or main (false) document is being compiled.
% The conditional |\ifchilddocmanual| tells whether
% the |\includeonly| mechanism is used (false) or
% the selection of child files must be performed manually (true).
% The definitions initialise to false:
%    \begin{macrocode}
\newif\ifchilddoc
\newif\ifchilddocmanual
%    \end{macrocode}

% \macro{\childdocname}
% \macro{\childdocjob}
% The macro |\childdocname| stores the name of the main document
% to be compiled. The macro |\childdocjob| stores the name of
% the document on which the \LaTeX{} compiler was originally invoked.
% The content of |\jobname| cannot be compared
% to filenames specified in the source due to different catcodes.
% The following code rescans |\jobname|, stores the result
% in |\childdocname| and saves a copy in |\childdocjob|:
%    \begin{macrocode}
\edef\childdocname{\scantokens\expandafter{\jobname\noexpand}}
\let\childdocjob\childdocname
%    \end{macrocode}

% \macro{\childdocdisable}
% The macro |\childdocdisable| prevents the main file
% from being processed more than once.
% At this stage, the main document command |\childdocmain|
% is assumed to be called once again where it should do nothing.
% Any subsequent call to it should prevent
% a secondary processing of the main document
% It overwrites the forwarding commands
% |\childdocof| and |\childdocforward|
% with empty macros to prevent further inclusions of the main document:
%    \begin{macrocode}
\newcommand{\childdocdisable}
{
  \renewcommand{\childdocmain}[1]{\renewcommand{\childdocmain}[1]{\endinput}}
  \renewcommand{\childdocof}[1]{}
  \renewcommand{\childdocby}[2][]{}
  \renewcommand{\childdocforward}[2][]{}
  \renewcommand{\childdocdisable}{}
}
%    \end{macrocode}

% \macro{\childdocmain}
% The macro |\childdocmain| is to be called at the top of the main file
% with nothing or the main filename (without extension) as argument.
% First, it breaks loops.
% If the argument is not empty and does not match |\childdocname|
% (which is set by the first inclusion of |childdoc.def|),
% |\ifchilddoc| is set to true, |\includeonly| is applied to the child file
% and |\jobname| is set to the main file
% (for proper handling of |.aux| files):
%    \begin{macrocode}
\newcommand{\childdocmain}[1]
{
  \childdocdisable\childdocmain{}
  \if?#1?\else
    \begingroup
      \def\childdoctmp{#1}
      \ifx\childdoctmp\childdocname
        \def\childdoctmp{}
      \else
        \def\childdoctmp
        {
          \childdoctrue
          \includeonly{\childdocname}
          \def\childdocjob{#1}
          \def\jobname{#1}
        }
      \fi
      \expandafter
    \endgroup
    \childdoctmp
  \fi
}
%    \end{macrocode}

% \macro{\childdocof}
% The command |\childdocof| redirects
% compilation to the main file |#1|.
%    \begin{macrocode}
\newcommand{\childdocof}[1]
{
  \childdocdisable
  \childdoctrue
  \includeonly{\childdocname}
  \def\jobname{#1}
  \def\childdocjob{#1}
  \input{#1}
}
%    \end{macrocode}

% \macro{\childdocby}
% The command |\childdocby| ....
%    \begin{macrocode}
\newcommand{\childdocby}[2][]
{
  \childdocdisable
  \childdoctrue
  \childdocmanualtrue
  \if?#1?\else
    \def\jobname{#2}
  \fi
  \def\childdocjob{#2}
  \input{#2}
  \endinput
}
%    \end{macrocode}

% \macro{\childdocforward}
% The command |\childdocforward| redirects
% compilation to the main file or
% (if the optional argument is given) a child file.
% Parameters are set as if the main file
% or a child file starting with |\childdocof| was compiled.
% Then compilation is handed over to the main file:
%    \begin{macrocode}
\newcommand{\childdocforward}[2][]
{
  \begingroup
    \if?#1?
      \def\childdoctmp
      {
        \def\childdocname{#2}
        \def\childdocjob{#2}
        \def\jobname{#2}
        \input{#2}
        \endinput
      }
    \else
      \def\childdoctmp
      {
        \childdocdisable
        \def\childdocname{#2}
        \childdoctrue
        \includeonly{#2}
        \def\childdocjob{#1}
        \def\jobname{#1}
        \input{#1}
        \endinput
      }
    \fi
    \expandafter
  \endgroup
  \childdoctmp
}
%    \end{macrocode}

% \macro{\childdocforwardprefix}
% The command |\childdocforwardprefix| redirects
% compilation to the main or a child file by means of a pattern.
% The prefix |#1| in the current filename is replaced by |#2|
% and the suffix of the current filename is kept
% (it is assumed that the filename does not contain the substring `|~~~|'
% which is used as a delimiter).
% Compilation is handed over to the new file by |\childdocforward|:
%    \begin{macrocode}
\newcommand{\childdocforwardprefix}[3][]
{
  \begingroup
    \def\childdocextract #2##1~~~{\def\childdoctmp{\childdocforward[#1]{#3##1}}}
    \expandafter\childdocextract\childdocname~~~
    \expandafter
  \endgroup
  \childdoctmp
}
%    \end{macrocode}

% \macro{\childdoc}
% The deprecated macro |\childdoc| is a legacy version of |\childdocmain|:
%    \begin{macrocode}
\newcommand{\childdoc}{\childdocmain}
%    \end{macrocode}

% \macro{\childdocredirect}
% The deprecated macro |\childdocredirect| is a legacy version
% of |\childdocforward| and |\childdocforwardprefix|:
%    \begin{macrocode}
\newcommand{\childdocredirect}[2][]
{
  \begingroup
    \if?#1?
      \def\childdoctmp{\childdocforward{#2}}
    \else
      \def\childdoctmp{\childdocforwardprefix{#1}{#2}}
    \fi
    \expandafter
  \endgroup
  \childdoctmp
}
%    \end{macrocode}

%\iffalse
%</package>
%\fi
%
\endinput
|\\
|\childdocforward[|\textit{main}|]{|\textit{dest}|}|\\
\end{tabular}
\end{center}
%
The argument \textit{dest} is the destination file
(without extension).
It should be the main file or one of the child files.
Note that further \textsf{childdoc} directives
such as |\childdocof| and |\childdocforward|
in the indicated file will be processed in this form.
The optional argument \textit{main}
passes on directly to the main file \textit{main}
while pretending to compile the child \textit{dest}.
This form behaves as if \textit{dest}
issues |\childdocof{|\textit{main}|}| right away,
and no further \textsf{childdoc} directives will be processed.

%%%%%%%%%%%%%%%%%%%%%%%%%%%%%%%%%%%%%%%%
\DescribeMacro{\...prefix}
In the alternative form |\childdocforwardprefix|,
%
\begin{center}
\begin{tabular}{l}
|% \iffalse
%
% childdoc.dtx Copyright (C) 2017-2018 Niklas Beisert
%
% This work may be distributed and/or modified under the
% conditions of the LaTeX Project Public License, either version 1.3
% of this license or (at your option) any later version.
% The latest version of this license is in
%   http://www.latex-project.org/lppl.txt
% and version 1.3 or later is part of all distributions of LaTeX
% version 2005/12/01 or later.
%
% This work has the LPPL maintenance status `maintained'.
%
% The Current Maintainer of this work is Niklas Beisert.
%
% This work consists of the files childdoc.dtx and childdoc.ins
% and the derived files childdoc.def and cdocsamp.tex with
% cdocsch1.tex, cdocsch2.tex, cdocsdrf.tex, cdocsfn1.tex, cdocsfn2.tex.
%
%<package>\ifdefined\childdocmain\endinput\fi
%<package>\ProvidesFile{childdoc.def}[2018/12/30 v2.0 child document driver]
%<samplemain>\ProvidesFile{cdocsamp.tex}[2018/12/30 v2.0 sample for childdoc]
%<*driver>
%\ProvidesFile{childdoc.drv}[2018/12/30 v2.0 childdoc reference manual file]
\PassOptionsToClass{10pt,a4paper}{article}
\documentclass{ltxdoc}

\usepackage[margin=35mm]{geometry}
\usepackage{hyperref}
\usepackage{hyperxmp}
\usepackage[usenames]{color}

\hypersetup{colorlinks=true}
\hypersetup{pdfstartview=FitH}
\hypersetup{pdfpagemode=UseNone}
\hypersetup{pdfsource={}}
\hypersetup{pdflang={en-UK}}
\hypersetup{pdfcopyright={Copyright 2017-2018 Niklas Beisert.
  This work may be distributed and/or modified under the
  conditions of the LaTeX Project Public License, either version 1.3
  of this license or (at your option) any later version.}}
\hypersetup{pdflicenseurl={http://www.latex-project.org/lppl.txt}}
\hypersetup{pdfcontactaddress={ETH Zurich, ITP, HIT K,
  Wolfgang-Pauli-Strasse 27}}
\hypersetup{pdfcontactpostcode={8093}}
\hypersetup{pdfcontactcity={Zurich}}
\hypersetup{pdfcontactcountry={Switzerland}}
\hypersetup{pdfcontactemail={nbeisert@itp.phys.ethz.ch}}
\hypersetup{pdfcontacturl={http://people.phys.ethz.ch/\xmptilde nbeisert/}}

\newcommand{\secref}[1]{\hyperref[#1]{section \ref*{#1}}}

\parskip1ex
\parindent0pt
\let\olditemize\itemize
\def\itemize{\olditemize\parskip0pt}

\begin{document}

\title{The \textsf{childdoc} Package}
\hypersetup{pdftitle={The childdoc Package}}
\author{Niklas Beisert\\[2ex]
  Institut f\"ur Theoretische Physik\\
  Eidgen\"ossische Technische Hochschule Z\"urich\\
  Wolfgang-Pauli-Strasse 27, 8093 Z\"urich, Switzerland\\[1ex]
  \href{mailto:nbeisert@itp.phys.ethz.ch}
  {\texttt{nbeisert@itp.phys.ethz.ch}}}
\hypersetup{pdfauthor={Niklas Beisert}}
\hypersetup{pdfsubject={Manual for the LaTeX2e Package childdoc}}
\date{30 December 2018, \textsf{v2.0}}
\maketitle

\begin{abstract}\noindent
\textsf{childdoc} is a \LaTeXe{} package
that enables the direct compilation
of document sections included by |\include|
to individual files.
\end{abstract}

\begingroup
\parskip0ex
\tableofcontents
\endgroup

%%%%%%%%%%%%%%%%%%%%%%%%%%%%%%%%%%%%%%%%%%%%%%%%%%%%%%%%%%%%%%%%%%%%%%%%%%%%%%%%
%%%%%%%%%%%%%%%%%%%%%%%%%%%%%%%%%%%%%%%%%%%%%%%%%%%%%%%%%%%%%%%%%%%%%%%%%%%%%%%%
\section{Introduction}

\LaTeX{} provides a mechanism to structure a large document (such as a book)
into a main file and several child files (containing the chapters)
using the |\include| command.
This mechanism is beneficial for documents
which span hundreds of pages in order to
make the source file(s) more manageable.
Moreover, compilation can be restricted to
selected child files by means of the |\includeonly| command.
The latter feature can be used to reduce the compilation time while editing
(this was significantly more useful in the earlier days of \LaTeX{})
or to generate a smaller document which is easier to navigate.
Another application of |\includeonly| is to generate
documents consisting of selected parts of the complete document.

However, there are a few drawbacks of the plain |\include| mechanism:
\begin{itemize}
\item
The child files cannot be compiled on their own,
they can only be compiled via the main file.
A naive editing environment
(such as a text editor with an option
to have the current file processed by \LaTeX)
may require one to switch to the main file before compiling;
attempting to compile the child file produces errors.
\item
The main file must be modified (each time)
to adjust the |\includeonly| command
to the present needs. This easily leaves the main file in a messy state.
\item
The generated document will always carry the filename
of the main document. This is inconvenient if
several child files are to be compiled and
to be kept for distribution.
\end{itemize}

The present package provides a simple interface
to make child files individually compilable by \LaTeX{}.
Compiling a child file then has the same effect as compiling
the main file with an |\includeonly| command
to select the appropriate child.
Moreover the generated document will carry the name of the child
rather than the main file.
This resolves all three above issues.

This feature is meant to make the editing of books,
thesis documents and lecture notes somewhat more convenient.
However, the package can also be used efficiently for
composing a series of documents (such as exercise sheets)
which are typically distributed individually.
It then assists the author in generating the individual documents
(potentially in different versions)
as well as a document containing the collected series.
Another application is in developing style files
or other kinds of included material
where compilation of the style file could redirect
to a sample or test file.

%%%%%%%%%%%%%%%%%%%%%%%%%%%%%%%%%%%%%%%%%%%%%%%%%%%%%%%%%%%%%%%%%%%%%%%%%%%%%%%%
%%%%%%%%%%%%%%%%%%%%%%%%%%%%%%%%%%%%%%%%%%%%%%%%%%%%%%%%%%%%%%%%%%%%%%%%%%%%%%%%
\section{Usage}

First of all, the package \textsf{childdoc} is \emph{not} a standard
\LaTeXe{} |.sty| style file! Therefore it needs to be invoked in
a non-standard way.

%%%%%%%%%%%%%%%%%%%%%%%%%%%%%%%%%%%%%%%%%%%%%%%%%%%%%%%%%%%%%%%%%%%%%%%%%%%%%%%%
\subsection{Included Files}
\label{sec:include}

%%%%%%%%%%%%%%%%%%%%%%%%%%%%%%%%%%%%%%%%
\DescribeMacro{\childdocmain}
To use the package, add the commands
\begin{center}
\begin{tabular}{l}
|\input{childdoc.def}|\\
|\childdocmain{}|\\
\end{tabular}
\end{center}
at the very top of the main \LaTeX{} file,
in particular \emph{before} the |\documentclass| statement!
The argument of |\childdocmain| should be left empty
(but it must be present).

%%%%%%%%%%%%%%%%%%%%%%%%%%%%%%%%%%%%%%%%
\DescribeMacro{\childdocof}
Furthermore, add the commands
\begin{center}
\begin{tabular}{l}
|\input{childdoc.def}|\\
|\childdocof{|\textit{main}|}|\\
\end{tabular}
\end{center}
at the top of every child file \textit{child}
which is included by |\include{|\textit{child}|}|
from within the main file
(or at least for those files to be compiled individually).
The argument \textit{main} must be the filename of the main file.

There are a couple of
considerations in setting up the main and child documents:

%%%%%%%%%%%%%%%%%%%%%%%%%%%%%%%%%%%%%%%%
\paragraph{Restrictions.}

Please note the following restrictions:
\begin{itemize}
\item
|\childdocmain| must be called with one argument \textit{main}
to ensure compatibility with earlier version of the package.
It must either be empty (|\childdocmain{}|)
or precisely match the filename of the main file in which it is specified.
See \secref{sec:detection} for further information.
\item
The filename \textit{main} must be specified without the |.tex| extension.
\item
The filename \textit{main} is case sensitive
(even in case-insensitive file systems)
due to internal string comparison.
\item
The argument \textit{main} should be fully expanded, it cannot be a macro.
\item
Subdirectories and special characters should be avoided in filenames.
\item
The command |\childdocmain{|\textit{main}|}| must be followed by a whitespace.
It should not be followed immediately by another command
or by a comment mark `|%|'.
This is because the \TeX{} parser reads the token immediately following
the argument of |\childdocmain| and puts it
at the beginning of every child section;
however, a white\-space is ignored.
\end{itemize}

%%%%%%%%%%%%%%%%%%%%%%%%%%%%%%%%%%%%%%%%
\paragraph{Content of Main File.}

It is advisable to place all content in the child files included by |\include|.
Any output contained in the main file will appear in all child documents
unless suppressed manually;
it cannot be suppressed automatically by the |\includeonly| directive
and thus should normally be avoided.
A method to include some content in the main file
by means of conditional processing is described in \secref{sec:conditional}.

%%%%%%%%%%%%%%%%%%%%%%%%%%%%%%%%%%%%%%%%
\paragraph{Page Numbering.}

When only a part of the document is compiled,
the appropriate numbering of pages
(as well as other status parameters)
is determined from the |.aux| files.
The latter contain information from previous passes.
However this information needs to propagate through
all intermediate child documents.
Therefore the page numbering in child documents may well
be inconsistent until the complete document is compiled at least once.

A useful (if unconventional) way to always ensure a consistent
page numbering is to restart the numbering in each child document
and denote the pages by `\textit{child}|.|\textit{page}'
where \textit{child} represents the chapter/section number of the child file.
This can be achieved by the command
|\numberwithin{page}{|\textit{child}|}|
of the \textsf{amsmath} package
where \textit{child} can be |chapter| or |section|
depending on the chosen structuring.
Alternatively, one can modify the macro |\thepage| appropriately
and reset the counter |page| at the start of each child file.

%%%%%%%%%%%%%%%%%%%%%%%%%%%%%%%%%%%%%%%%%%%%%%%%%%%%%%%%%%%%%%%%%%%%%%%%%%%%%%%%
\subsection{Conditional Processing}
\label{sec:conditional}

The package provides a mechanism to compile different versions
of a document. To customise the versions further some conditional processing
can come in handy to distinguish which version is being compiled.
The package provides two macros to describe the compilation context:

%%%%%%%%%%%%%%%%%%%%%%%%%%%%%%%%%%%%%%%%
\DescribeMacro{\ifchilddoc}
The conditional |\ifchilddoc| distinguishes between the compilation of
child documents and the main document:
%
\begin{center}
|\ifchilddoc |\textit{child-code}| |[|\||else |\textit{main-code}]| \||fi|
\end{center}

%%%%%%%%%%%%%%%%%%%%%%%%%%%%%%%%%%%%%%%%
\DescribeMacro{\childdocname}
\DescribeMacro{\childdocjob}
The macro |\childdocname| contains the filename (without extension)
of the main or child file being processed.
Note that |\childdocjob| will always contain the name of the main file.

%%%%%%%%%%%%%%%%%%%%%%%%%%%%%%%%%%%%%%%%
\paragraph{Title Page.}

Conditional processing can be used to include a title or banner page
in the main document when proper precautions are taken.
Importantly, the code in the main file should ensure that the page counter
(as well as other status parameters which are stored in the |.aux| files)
takes the same value after the conditional processing.
Otherwise the page numbers may take divergent values
depending on which part is compiled.

For example, a title page could be declared by:
%
\begin{center}
\begin{tabular}{l}
|\ifchilddoc\||else|\\
|\addtocounter{page}{-1}|\\
\textit{code for title page}\\
|\newpage|\\
|\||fi|
\end{tabular}
\end{center}
%
A banner page for the child documents can be generated by:
%
\begin{center}
\begin{tabular}{l}
|\ifchilddoc|\\
|\addtocounter{page}{-1}|\\
\textit{code for banner page}\\
|\newpage|\\
|\||fi|
\end{tabular}
\end{center}
%
Here one could write a message such as:
\begin{center}
|This is the part \childdocname{} of \childdocjob{}.|
\end{center}

%%%%%%%%%%%%%%%%%%%%%%%%%%%%%%%%%%%%%%%%%%%%%%%%%%%%%%%%%%%%%%%%%%%%%%%%%%%%%%%%
\subsection{Flags}
\label{sec:flags}

The package makes it easy to generate different versions
of the main or child documents.
To this end compilation flags can be defined
and assigned different default values.
They will be particularly useful in conjunction
with the forwarding mechanism described in \secref{sec:forward}.

For example, it may be useful to have a flag |\version|
which can be set to |draft| or |final|.
The document source will contain some conditional code
depending on the value of |\version|.
Suppose further, the flag should default to |final| for the main file
and to |draft| for child files
which is a natural assignment for editing the document.
This is achieved by placing the following code
in the preamble of the main document
(below the |\childdocmain| directive):
%
\begin{center}
\begin{tabular}{l}
|\ifchilddoc|\\
|\providecommand{\version}{draft}|\\
|\||else|\\
|\providecommand{\version}{final}|\\
|\||fi|
\end{tabular}
\end{center}
%
The definition by |\providecommand| makes sure
that previous definitions are not overwritten.
Further statements |\providecommand{\version}{...}|
can thus be added before the above code to override it.

For the main file, one might add a line
(between |\childdocmain| and the above block)
%
\begin{center}
|%\ifchilddoc\||else\providecommand{\version}{draft}\||fi|
\end{center}
%
which can be uncommented to produce a draft version.
Likewise one can add a line to the very top of a child file
(above the |\childdocof{|\textit{main}|}| directive)
%
\begin{center}
|%\providecommand{\version}{final}|
\end{center}
%
which can be uncommented to produce the final version of this child document.

%%%%%%%%%%%%%%%%%%%%%%%%%%%%%%%%%%%%%%%%%%%%%%%%%%%%%%%%%%%%%%%%%%%%%%%%%%%%%%%%
\subsection{Forwarding}
\label{sec:forward}

Different versions of the main or child documents
using compilation flags as described in \secref{sec:flags}
can be (permanently) stored in different files
for convenient compilation, viewing and distribution.
To this end, the package defines a command
to pass on compilation to a different file:

%%%%%%%%%%%%%%%%%%%%%%%%%%%%%%%%%%%%%%%%
\DescribeMacro{\childdocforward}
The command |\childdocforward| redirects processing to
another source file:
%
\begin{center}
\begin{tabular}{l}
|\input{childdoc.def}|\\
|\childdocforward[|\textit{main}|]{|\textit{dest}|}|\\
\end{tabular}
\end{center}
%
The argument \textit{dest} is the destination file
(without extension).
It should be the main file or one of the child files.
Note that further \textsf{childdoc} directives
such as |\childdocof| and |\childdocforward|
in the indicated file will be processed in this form.
The optional argument \textit{main}
passes on directly to the main file \textit{main}
while pretending to compile the child \textit{dest}.
This form behaves as if \textit{dest}
issues |\childdocof{|\textit{main}|}| right away,
and no further \textsf{childdoc} directives will be processed.

%%%%%%%%%%%%%%%%%%%%%%%%%%%%%%%%%%%%%%%%
\DescribeMacro{\...prefix}
In the alternative form |\childdocforwardprefix|,
%
\begin{center}
\begin{tabular}{l}
|\input{childdoc.def}|\\
|\childdocforwardprefix[|\textit{main}|]{|\textit{prefix}|}{|\textit{dest}|}|
\end{tabular}
\end{center}
%
the destination file is determined by a pattern
depending on the current file:
To make this work, the current file must be called
`{\textit{prefix}\hspace{0.2em}\textit{suffix}}'
with \textit{prefix} matching precisely the argument.
Processing is then passed on to the file
`{\textit{dest}\hspace{0.2em}\textit{suffix}}'.
Surely, the same effect is achieved by
directly specifying the
argument `{\textit{dest}\hspace{0.2em}\textit{suffix}}'
in the first form.
However, that requires to set up a different file
for each child. With the alternative form of the command
all these files can have exactly the same content
which simplifies setting them up and maintaining them.

For example, the following file |draft.tex|
with a compilation flag |\version| as described in \secref{sec:flags}
compiles the main document as a draft:
%
\begin{center}
\begin{tabular}{l}
|\def\version{draft}|\\
|\input{childdoc.def}|\\
|\childdocforward{|\textit{main}|}|
\end{tabular}
\end{center}
%
Likewise, the following files |final|\textit{nn}|.tex|
compile the final version of the child document
|child|\textit{nn}|.tex|:
%
\begin{center}
\begin{tabular}{l}
|\def\version{final}|\\
|\input{childdoc.def}|\\
|\childdocforwardprefix{final}{child}|
\end{tabular}
\end{center}
%

Note that when several versions of a main file and/or of each child file
are to be generated, it may be convenient to set up a |Makefile| or
shell script to automatise the process.

%%%%%%%%%%%%%%%%%%%%%%%%%%%%%%%%%%%%%%%%%%%%%%%%%%%%%%%%%%%%%%%%%%%%%%%%%%%%%%%%
\subsection{Command Line Processing}
\label{sec:commandline}

The effect of redirection files can also be achieved by invoking
the \LaTeX{} compiler with a more elaborate command line.
Most conveniently this should be done as part
of a shell script or a |Makefile|.

When using \textsf{childdoc} in the main file, the following
command lines effectively perform a redirection
(note that depending on the shell being used,
backslashes may have to be doubled: `|\|' $\to$ `|\\|'):
%
\begin{center}
|... -jobname "|\textit{target}|" |\\|"|[\textit{flags}]%
|\input{childdoc.def}\childdocforward[|\textit{main}|]{|\textit{dest}|}"|
\end{center}
%
Here \textit{target} is the name of the output file,
\textit{main} is the name of the main file
and \textit{dest} is the name of the main or child file to be processed
(all filenames without extensions).
The optional argument \textit{main} can be omitted
if \textit{main} matches \textit{dest}.
Optionally, compilation \textit{flags} can be defined via |\def| commands.
This command line makes the \TeX{} engine believe
it is compiling the file \textit{target}
whose content is specified as the latter parameter.
The provided code then forwards the processing to
\textit{main} or \textit{dest} as described in \secref{sec:forward}.

%%%%%%%%%%%%%%%%%%%%%%%%%%%%%%%%%%%%%%%%%%%%%%%%%%%%%%%%%%%%%%%%%%%%%%%%%%%%%%%%
\subsection{Include by Input}
\label{sec:input}

Including child documents by |\include| has some restrictions by design.
Most notably, the content of a child document always occupies
its own set of pages; pages cannot be shared between child documents.
Usually, this behaviour makes perfect sense
because each child document contain an essential part of the document.
However, in some situations it may be desirable to compose
a document from a collection of parts
without having mandatory page breaks between then.
For this case, the package
provides a mechanism to include parts
by |\input| which can also be processed individually.
However, by construction this mechanism
requires manual handling of the content to be output.

%%%%%%%%%%%%%%%%%%%%%%%%%%%%%%%%%%%%%%%%
\DescribeMacro{\ifchilddocmanual}
The main file should be prepared as usual, see \secref{sec:include}.
However, the document body must make a distinction
between processing of an individual part and of the main document, e.g.:
%
\begin{center}
\begin{tabular}{l}
|\ifchilddocmanual|\\
|\input{\childdocname}|\\
|\||else|\\
\textit{document body with }|\input{|\textit{part}|}|\\
|\||fi|
\end{tabular}
\end{center}
%
The conditional |\ifchilddocmanual| is true whenever
a part to be included by |\input| is being compiled,
and the name of the part is stored in |\childdocname|.

%%%%%%%%%%%%%%%%%%%%%%%%%%%%%%%%%%%%%%%%
\DescribeMacro{\childdocby}
Each part to be included by |\input| should start with:
%
\begin{center}
\begin{tabular}{l}
|\input{childdoc.def}|\\
|\childdocby{|\textit{main}|}|\\
\end{tabular}
\end{center}
%
The directive |\childdocby| is similar to |\childdocof|
described in \secref{sec:include},
but the subsequent selection of content must be done manually.
To that end, both |\ifchilddoc| and |\ifchilddocmanual|
will be true upon processing of a part,
and the name of the part is stored in |\childdocname|.
Note that |\jobname| will be set to the filename of the current part
so that each part receives an individual |.aux| file
that does not interfere with the |.aux| file(s) of the main document.
This behaviour can be altered by the alternative form
|\childdocby[*]{|\textit{main}|}| (with a non-empty optional argument)
which uses the |.aux| file of the main document
by setting |\jobname| to \textit{main}.

%%%%%%%%%%%%%%%%%%%%%%%%%%%%%%%%%%%%%%%%%%%%%%%%%%%%%%%%%%%%%%%%%%%%%%%%%%%%%%%%
\subsection{Driver Development}
\label{sec:driver}

The \textsf{childdoc} mechanism can also be use for the development
of definition files such as \LaTeX{} styles or classes.
This case differs from the above setup with multiple parts
included by |\include| in that no |\includeonly| should be invoked.
This can be achieved by starting the include file
(before |\ProvidesPackage|) with:
%
\begin{center}
\begin{tabular}{l}
|\input{childdoc.def}|\\
|\childdocforward{|\textit{main}|}|\\
\end{tabular}
\end{center}
%
or alternatively with:
%
\begin{center}
\begin{tabular}{l}
|\input{childdoc.def}|\\
|\childdocby{|\textit{main}|}|\\
\end{tabular}
\end{center}
%
Both forms have slightly different effects as described above.
The main file is prepared as usual, see \secref{sec:include}.

%%%%%%%%%%%%%%%%%%%%%%%%%%%%%%%%%%%%%%%%%%%%%%%%%%%%%%%%%%%%%%%%%%%%%%%%%%%%%%%%
\subsection{Legacy Detection}
\label{sec:detection}

The directive |\childdocmain| in the main file can detect
whether the complete document or merely a child is to be compiled
even without using the directive |\childdocof|.
This method is deprecated because it is less robust
and there is no compelling reason to use it;
it is merely provided for backward compatibility
and it may be removed in future versions.

If the detection mechanism is to be used,
it is mandatory to correctly specify
the filename of the main file as the argument of |\childdocmain|:
%
\begin{center}
\begin{tabular}{l}
|\input{childdoc.def}|\\
|\childdocmain{|\textit{main}|}|\\
\end{tabular}
\end{center}
%
If |\jobname| does not match the argument \textit{main} of |\childdocmain|,
it is assumed that |\jobname| points to the child file to be compiled.
When using |\childdocmain| with the main file specified as argument,
it suffices to start a child file
with just |\input{|\textit{main}|}|
without loading of the package and using |\childdocof|.
If instead all processing is done
with the appropriate \textsf{childdoc} directives,
the argument of \textit{main} of |\childdocmain| can be empty.

An alternative version of the command line processing described
in \secref{sec:commandline} using the detection mechanism reads:
%
\begin{center}
|... -jobname "|\textit{target}|" "|[\textit{flags}]%
[|\def\jobname{|\textit{dest}|}|]|\input{|\textit{main}|}"|
\end{center}

%%%%%%%%%%%%%%%%%%%%%%%%%%%%%%%%%%%%%%%%%%%%%%%%%%%%%%%%%%%%%%%%%%%%%%%%%%%%%%%%
\subsection{Manual Code}
\label{sec:manual}

In case one cannot be certain whether the definitions file |childdoc.def|
is installed on the target \TeX{} distribution
and one prefers not to ship it,
it is conceivable to paste a few relevant commands into the sources.

To that end, drop all statements |\input{childdoc.def}|
and perform the replacements as outlined below.
Instead of |\childdocmain{|\textit{main}|}| add the following code
to the top of the main file:
%
\begin{center}
\begin{tabular}{l}
|\||ifdefined\childdocname\endinput\||fi\newif\ifchilddoc|\\
|\edef\childdocname{\scantokens\expandafter{\jobname\noexpand}}|\\
|\def\childdocmain{|\textit{main}|}\||ifx\childdocmain\childdocname\||else|\\
|\childdoctrue\includeonly{\childdocname}\let\jobname\childdocmain\||fi|\\
\end{tabular}
\end{center}
%
Instead of |\childdocof{|\textit{main}|}| just include the main file
at the top of each child file:
%
\begin{center}
|\input{|\textit{main}|}|
\end{center}
%
A simple redirection |\childdocforward{|\textit{dest}|}| is achieved by:
%
\begin{center}
|\def\jobname{|\textit{dest}|}\input{\jobname}|
\end{center}
%
The redirection with prefix
|\childdocforwardprefix[|\textit{prefix}|]{|\textit{dest}|}|
is accomplished by:
%
\begin{center}
\begin{tabular}{l}
|{\edef\jobname{\scantokens\expandafter{\jobname\noexpand}}|\\
|\def\redirectjob |\textit{prefix}|#1~~~{\gdef\jobname{|\textit{dest}|#1}}|\\
|\expandafter\redirectjob\jobname~~~}\input{\jobname}|
\end{tabular}
\end{center}

In an alternative approach,
child documents can be compiled by a specific command line
without additional code or specific definitions:
%
\begin{center}
|... -jobname "|\textit{target}|" "|[\textit{flags}]%
|\includeonly{|\textit{dest}|}\input{|\textit{main}|}"|
\end{center}
%

%%%%%%%%%%%%%%%%%%%%%%%%%%%%%%%%%%%%%%%%%%%%%%%%%%%%%%%%%%%%%%%%%%%%%%%%%%%%%%%%
%%%%%%%%%%%%%%%%%%%%%%%%%%%%%%%%%%%%%%%%%%%%%%%%%%%%%%%%%%%%%%%%%%%%%%%%%%%%%%%%
\section{Information}

%%%%%%%%%%%%%%%%%%%%%%%%%%%%%%%%%%%%%%%%%%%%%%%%%%%%%%%%%%%%%%%%%%%%%%%%%%%%%%%%
\subsection{Copyright}

Copyright \copyright{} 2017--2018 Niklas Beisert

This work may be distributed and/or modified under the
conditions of the \LaTeX{} Project Public License, either version 1.3
of this license or (at your option) any later version.
The latest version of this license is in
  \url{http://www.latex-project.org/lppl.txt}
and version 1.3 or later is part of all distributions of \LaTeX{}
version 2005/12/01 or later.

This work has the LPPL maintenance status `maintained'.

The Current Maintainer of this work is Niklas Beisert.

This work consists of the files |README.txt|, |childdoc.ins| and |childdoc.dtx|
as well as the derived files |childdoc.def|, |cdocsamp.tex|
with |cdocsch1.tex|, |cdocsch2.tex|, |cdocspt3.tex|, |cdocspt4.tex|,
|cdocsdrf.tex|, |cdocsfn1.tex|, |cdocsfn2.tex|
as well as |childdoc.pdf|.

%%%%%%%%%%%%%%%%%%%%%%%%%%%%%%%%%%%%%%%%%%%%%%%%%%%%%%%%%%%%%%%%%%%%%%%%%%%%%%%%
\subsection{Files and Installation}

The package consists of the files:
%
\begin{center}
\begin{tabular}{ll}
    |README.txt|   & readme file \\
    |childdoc.ins| & installation file \\
    |childdoc.dtx| & source file \\
    |childdoc.def| & definition file \\
    |cdocsamp.tex| & sample main file \\
    |cdocsch1.tex| & sample include file \\
    |cdocsch2.tex| & sample include file \\
    |cdocspt3.tex| & sample part file \\
    |cdocspt4.tex| & sample part file \\
    |cdocsdrf.tex| & sample redirection file \\
    |cdocsfn1.tex| & sample redirection file \\
    |cdocsfn2.tex| & sample redirection file \\
    |childdoc.pdf| & manual
\end{tabular}
\end{center}
%
The distribution consists of the files
|README.txt|, |childdoc.ins| and |childdoc.dtx|.
%
\begin{itemize}
\item
Run (pdf)\LaTeX{} on |childdoc.dtx|
to compile the manual |childdoc.pdf| (this file).
\item
Run \LaTeX{} on |childdoc.ins| to create the definitions file |childdoc.def|
and the sample |cdocsamp.tex| with include files
|cdocsch1.tex|, |cdocsch2.tex|, |cdocspt3.tex|, |cdocspt4.tex|,
|cdocsdrf.tex|, |cdocsfn1.tex|, |cdocsfn2.tex|.
Then copy the file |childdoc.def| to an appropriate directory of your \LaTeX{}
distribution, e.g.\ \textit{texmf-root}|/tex/latex/childdoc|.
\end{itemize}

%%%%%%%%%%%%%%%%%%%%%%%%%%%%%%%%%%%%%%%%%%%%%%%%%%%%%%%%%%%%%%%%%%%%%%%%%%%%%%%%
\subsection{Related CTAN Packages}

There are several other packages which offer a similar functionality:
%
\begin{itemize}
\item
The packages
\href{http://ctan.org/pkg/docmute}{\textsf{docmute}},
\href{http://ctan.org/pkg/includex}{\textsf{includex}} and
\href{http://ctan.org/pkg/standalone}{\textsf{standalone}}
provide commands to include only the document body of
a child file thus allowing both files to be compiled individually.
\item
The packages \href{http://ctan.org/pkg/subdocs}{\textsf{subdocs}}
and \href{http://ctan.org/pkg/subfiles}{\textsf{subfiles}}
provide structures in which the main and child documents can be
encapsulated and allowing them to be compiled individually.
The inclusion mechanism is different from the conventional |\include|.
\item
The package \href{http://ctan.org/pkg/combine}{\textsf{combine}}
is an elaborate solution to combine several documents into one.
\end{itemize}
%
See also the CTAN topic \href{http://ctan.org/topic/subdocs}{\textsf{subdocs}}
for further related packages.
The present package differs from the above solutions in that
a document structure constructed with the conventional |\include| mechanism
just needs two extra commands at the top of every file
such that all constituent files can be compiled individually.

%%%%%%%%%%%%%%%%%%%%%%%%%%%%%%%%%%%%%%%%%%%%%%%%%%%%%%%%%%%%%%%%%%%%%%%%%%%%%%%%
%\subsection{Feature Suggestions}
%
%The following is a list of features which may be useful for future
%versions of this package:
%%
%\begin{itemize}
%\item
%\ldots
%\end{itemize}

%%%%%%%%%%%%%%%%%%%%%%%%%%%%%%%%%%%%%%%%%%%%%%%%%%%%%%%%%%%%%%%%%%%%%%%%%%%%%%%%
\subsection{Revision History}

%%%%%%%%%%%%%%%%%%%%%%%%%%%%%%%%%%%%%%%%
\paragraph{v2.0:} 2018/12/30

\begin{itemize}
\item
immediate forward processing
\item
added |\childdocby| mechanism
\item
manual restructured
\end{itemize}

%%%%%%%%%%%%%%%%%%%%%%%%%%%%%%%%%%%%%%%%
\paragraph{v1.6:} 2018/01/17

\begin{itemize}
\item
application for development of include files
\item
corrections to manual
\end{itemize}

%%%%%%%%%%%%%%%%%%%%%%%%%%%%%%%%%%%%%%%%
\paragraph{v1.5:} 2017/05/21

\begin{itemize}
\item
more complete structuring introduced
\item
|\childdocof| introduced
\item
|\childdoc| renamed to |\childdocmain|
\item
|\childredirect| renamed to |\childdocforward| and |\childdocforwardprefix|
and functionality expanded
\end{itemize}

%%%%%%%%%%%%%%%%%%%%%%%%%%%%%%%%%%%%%%%%
\paragraph{v1.0:} 2017/04/27

\begin{itemize}
\item
manual and install package
\item
first version published on CTAN
\end{itemize}

%%%%%%%%%%%%%%%%%%%%%%%%%%%%%%%%%%%%%%%%
\paragraph{v0.6:} 2017/04/26

\begin{itemize}
\item
redirection mechanism added
\end{itemize}

%%%%%%%%%%%%%%%%%%%%%%%%%%%%%%%%%%%%%%%%
\paragraph{v0.5:} 2017/04/26

\begin{itemize}
\item
functionality in definition file
\end{itemize}


%%%%%%%%%%%%%%%%%%%%%%%%%%%%%%%%%%%%%%%%%%%%%%%%%%%%%%%%%%%%%%%%%%%%%%%%%%%%%%%%
%%%%%%%%%%%%%%%%%%%%%%%%%%%%%%%%%%%%%%%%%%%%%%%%%%%%%%%%%%%%%%%%%%%%%%%%%%%%%%%%
%%%%%%%%%%%%%%%%%%%%%%%%%%%%%%%%%%%%%%%%%%%%%%%%%%%%%%%%%%%%%%%%%%%%%%%%%%%%%%%%
\appendix

\settowidth\MacroIndent{\rmfamily\scriptsize 000\ }

 \DocInput{childdoc.dtx}

\end{document}
%</driver>
% \fi
%
% %%%%%%%%%%%%%%%%%%%%%%%%%%%%%%%%%%%%%%%%%%%%%%%%%%%%%%%%%%%%%%%%%%%%%%%%%%%%%%
% %%%%%%%%%%%%%%%%%%%%%%%%%%%%%%%%%%%%%%%%%%%%%%%%%%%%%%%%%%%%%%%%%%%%%%%%%%%%%%
% \section{Sample}
%\iffalse
%<*samplemain>
%\fi
%
% The following presents a sample document
% with two chapters, two parts, a title page,
% a compile flag as well as three forwarding files to set the flag.
% It consists of eight |.tex| files:
% \begin{center}
% \begin{tabular}{ll}
% |cdocsamp.tex|&main file\\
% |cdocsch1.tex|&include file for chapter 1\\
% |cdocsch2.tex|&include file for chapter 2\\
% |cdocspt3.tex|&include file for part 3\\
% |cdocspt4.tex|&include file for part 4\\
% |cdocsdrf.tex|&forwarding file for main file in draft mode\\
% |cdocsfi1.tex|&forwarding file for final version of chapter 1\\
% |cdocsfi2.tex|&forwarding file for final version of chapter 2\\
% \end{tabular}
% \end{center}
% Each of the eight files can be compiled directly by the \LaTeX{} compiler.
%
% %%%%%%%%%%%%%%%%%%%%%%%%%%%%%%%%%%%%%%
% \paragraph{Main File.}
%
% The main file is called |cdocsamp.tex|.
%
% Load the \textsf{childdoc} definitions and
% declare the filename for the main document:
%    \begin{macrocode}
\input{childdoc.def}
\childdocmain{}
%    \end{macrocode}

% Optional override for |\version| flag:
%    \begin{macrocode}
%%\ifchilddoc\else\providecommand{\version}{draft}\fi
%    \end{macrocode}

% Define the default values for the |\version| flag
% (|final| for the main file and |draft| for childs):
%    \begin{macrocode}
\ifchilddoc
\providecommand{\version}{draft}
\else
\providecommand{\version}{final}
\fi
%    \end{macrocode}

% Load the standard document class:
%    \begin{macrocode}
\documentclass[12pt]{article}
%    \end{macrocode}

% Start the document body:
%    \begin{macrocode}
\begin{document}
%    \end{macrocode}

% Declare a title page.
% Print title, part of document being processed and version flag:
%    \begin{macrocode}
\addtocounter{page}{-1}
\begin{center}
{\LARGE\bfseries{}childdoc example\par}
\vspace{1cm}
\ifchilddoc
\ifchilddocmanual part\else chapter\fi:
`\childdocname' of `\childdocjob'\par
\else
main document: `\childdocjob'\par
\fi
version: \version\par
\end{center}
\newpage
%    \end{macrocode}

% Manually include selected file,
% otherwise process as usual:
%    \begin{macrocode}
\ifchilddocmanual
\section*{part `\childdocname'}
\input{\childdocname}
\else
%    \end{macrocode}

% Include the two chapters:
%    \begin{macrocode}
\include{cdocsch1}
\include{cdocsch2}
%    \end{macrocode}

% Include the two parts unless only chapters should be displayed:
%    \begin{macrocode}
\ifchilddoc\else
\section{part three}
\input{cdocspt3}
\section{part four}
\input{cdocspt4}
\fi
%    \end{macrocode}

% Process as usual until here:
%    \begin{macrocode}
\fi
%    \end{macrocode}

% End of document body:
%    \begin{macrocode}
\end{document}
%    \end{macrocode}
%\iffalse
%</samplemain>
%\fi
%
% %%%%%%%%%%%%%%%%%%%%%%%%%%%%%%%%%%%%%%
% \paragraph{Chapter Include Files.}
%
% The include files are called |cdocsch1.tex| and |cdocsch2.tex|.
%
%\iffalse
%<*samplechap1|samplechap2>
%\fi

% Optional override for |\version| flag:
%    \begin{macrocode}
%%\providecommand{\version}{final}
%    \end{macrocode}

% Include the main document:
%    \begin{macrocode}
\input{childdoc.def}
\childdocof{cdocsamp}
%    \end{macrocode}

%\iffalse
%</samplechap1|samplechap2>
%\fi
%
%\iffalse
%<*samplechap1>
%\fi
% Some text for chapter 1:
%    \begin{macrocode}
\section{one}
some text in chapter one
%    \end{macrocode}

%\iffalse
%</samplechap1>
%\fi
% Some text for chapter 2:
%\iffalse
%<*samplechap2>
%\fi
%    \begin{macrocode}
\section{two}
more text in chapter two
%    \end{macrocode}

%\iffalse
%</samplechap2>
%\fi
%
% %%%%%%%%%%%%%%%%%%%%%%%%%%%%%%%%%%%%%%
% \paragraph{Part Include Files.}
%
% The include files are called |cdocspt3.tex| and |cdocspt4.tex|.
%
%\iffalse
%<*samplepart3|samplepart4>
%\fi

% Optional override for |\version| flag:
%    \begin{macrocode}
%%\providecommand{\version}{final}
%    \end{macrocode}

% Include the main document:
%    \begin{macrocode}
\input{childdoc.def}
\childdocby{cdocsamp}
%    \end{macrocode}

%\iffalse
%</samplepart3|samplepart4>
%\fi
%
%\iffalse
%<*samplepart3>
%\fi
% Some text for part 3:
%    \begin{macrocode}
some text in part three
%    \end{macrocode}

%\iffalse
%</samplepart3>
%\fi
% Some text for part 4:
%\iffalse
%<*samplepart4>
%\fi
%    \begin{macrocode}
more text in part four
%    \end{macrocode}

%\iffalse
%</samplepart4>
%\fi
%
% %%%%%%%%%%%%%%%%%%%%%%%%%%%%%%%%%%%%%%
% \paragraph{Forwarding for a Complete Draft.}
%
% The following forwarding file |cdocsdrf.tex|
% compiles the main document in draft mode:
%\iffalse
%<*sampledraft>
%\fi
%    \begin{macrocode}
\def\version{draft}
\input{childdoc.def}
\childdocforward{cdocsamp}
%    \end{macrocode}

%\iffalse
%</sampledraft>
%\fi
%
% %%%%%%%%%%%%%%%%%%%%%%%%%%%%%%%%%%%%%%
% \paragraph{Forwarding for Final Version of the Chapters.}
%
% The following forwarding files |cdocsfn1.tex| and |cdocsfn2.tex|
% (with identical content)
% compile the final versions of the child documents
% |cdocsch1.tex| and |cdocsch2.tex|, respectively:
%\iffalse
%<*samplefinal>
%\fi
%    \begin{macrocode}
\def\version{final}
\input{childdoc.def}
\childdocforwardprefix[cdocsamp]{cdocsfn}{cdocsch}
%    \end{macrocode}

%\iffalse
%</samplefinal>
%\fi
%
% %%%%%%%%%%%%%%%%%%%%%%%%%%%%%%%%%%%%%%
% \paragraph{Command Line Processing.}
%
% The following three command lines generate the output files
% |cdocscld|, |cdocscl1| and |cdocscl2|
% which should be identical to
% |cdocsdrf|, |cdocsch1| and |cdocsfn2|, respectively:
% \begin{center}
% \begin{tabular}{l}
% |latex -jobname cdocscld \|\\
% |  "\def\version{draft}\input{childdoc.def}\childdocforward{cdocsamp}"|\\
% |latex -jobname cdocscl1 \|\\
% |  "\input{childdoc.def}\childdocforward[cdocsamp]{cdocsch1}"|\\
% |latex -jobname cdocscl2 \|\\
% |  "\def\version{final}\input{childdoc.def}\childdocforward{cdocsch2}"|
% \end{tabular}
% \end{center}
% Note that the trailing backslash on each first line
% merely continues the input to the second line
% (for convenient cut ant paste).
% Furthermore, the command |latex| can be replaced by any
% of its alternative versions such as |pdflatex|.
%
% %%%%%%%%%%%%%%%%%%%%%%%%%%%%%%%%%%%%%%%%%%%%%%%%%%%%%%%%%%%%%%%%%%%%%%%%%%%%%%
% %%%%%%%%%%%%%%%%%%%%%%%%%%%%%%%%%%%%%%%%%%%%%%%%%%%%%%%%%%%%%%%%%%%%%%%%%%%%%%
% \section{Implementation}
%\iffalse
%<*package>
%\fi
%
% This section describes the definitions file |childdoc.def|.

% The definitions cannot be loaded using |\usepackage| or |\RequirePackage|
% which has a mechanism to prevent loading a style file more than once.
% When loading the definitions by means of |\input|
% multiple instances have to be prevented manually:
%\iffalse
%This code needs to be before the `\ProvidesFile' directive
%which is defined at the beginning of this file.
%Therefore it is also placed there and commented out here.
%</package>
%<*discard>
%\fi
%    \begin{macrocode}
\ifdefined\childdocmain\endinput\fi
%    \end{macrocode}
%\iffalse
%</discard>
%<*package>
%\fi
%
% \macro{\ifchilddoc}
% \macro{\ifchilddocmanual}
% The conditional |\ifchilddoc| tells whether a
% child (true) or main (false) document is being compiled.
% The conditional |\ifchilddocmanual| tells whether
% the |\includeonly| mechanism is used (false) or
% the selection of child files must be performed manually (true).
% The definitions initialise to false:
%    \begin{macrocode}
\newif\ifchilddoc
\newif\ifchilddocmanual
%    \end{macrocode}

% \macro{\childdocname}
% \macro{\childdocjob}
% The macro |\childdocname| stores the name of the main document
% to be compiled. The macro |\childdocjob| stores the name of
% the document on which the \LaTeX{} compiler was originally invoked.
% The content of |\jobname| cannot be compared
% to filenames specified in the source due to different catcodes.
% The following code rescans |\jobname|, stores the result
% in |\childdocname| and saves a copy in |\childdocjob|:
%    \begin{macrocode}
\edef\childdocname{\scantokens\expandafter{\jobname\noexpand}}
\let\childdocjob\childdocname
%    \end{macrocode}

% \macro{\childdocdisable}
% The macro |\childdocdisable| prevents the main file
% from being processed more than once.
% At this stage, the main document command |\childdocmain|
% is assumed to be called once again where it should do nothing.
% Any subsequent call to it should prevent
% a secondary processing of the main document
% It overwrites the forwarding commands
% |\childdocof| and |\childdocforward|
% with empty macros to prevent further inclusions of the main document:
%    \begin{macrocode}
\newcommand{\childdocdisable}
{
  \renewcommand{\childdocmain}[1]{\renewcommand{\childdocmain}[1]{\endinput}}
  \renewcommand{\childdocof}[1]{}
  \renewcommand{\childdocby}[2][]{}
  \renewcommand{\childdocforward}[2][]{}
  \renewcommand{\childdocdisable}{}
}
%    \end{macrocode}

% \macro{\childdocmain}
% The macro |\childdocmain| is to be called at the top of the main file
% with nothing or the main filename (without extension) as argument.
% First, it breaks loops.
% If the argument is not empty and does not match |\childdocname|
% (which is set by the first inclusion of |childdoc.def|),
% |\ifchilddoc| is set to true, |\includeonly| is applied to the child file
% and |\jobname| is set to the main file
% (for proper handling of |.aux| files):
%    \begin{macrocode}
\newcommand{\childdocmain}[1]
{
  \childdocdisable\childdocmain{}
  \if?#1?\else
    \begingroup
      \def\childdoctmp{#1}
      \ifx\childdoctmp\childdocname
        \def\childdoctmp{}
      \else
        \def\childdoctmp
        {
          \childdoctrue
          \includeonly{\childdocname}
          \def\childdocjob{#1}
          \def\jobname{#1}
        }
      \fi
      \expandafter
    \endgroup
    \childdoctmp
  \fi
}
%    \end{macrocode}

% \macro{\childdocof}
% The command |\childdocof| redirects
% compilation to the main file |#1|.
%    \begin{macrocode}
\newcommand{\childdocof}[1]
{
  \childdocdisable
  \childdoctrue
  \includeonly{\childdocname}
  \def\jobname{#1}
  \def\childdocjob{#1}
  \input{#1}
}
%    \end{macrocode}

% \macro{\childdocby}
% The command |\childdocby| ....
%    \begin{macrocode}
\newcommand{\childdocby}[2][]
{
  \childdocdisable
  \childdoctrue
  \childdocmanualtrue
  \if?#1?\else
    \def\jobname{#2}
  \fi
  \def\childdocjob{#2}
  \input{#2}
  \endinput
}
%    \end{macrocode}

% \macro{\childdocforward}
% The command |\childdocforward| redirects
% compilation to the main file or
% (if the optional argument is given) a child file.
% Parameters are set as if the main file
% or a child file starting with |\childdocof| was compiled.
% Then compilation is handed over to the main file:
%    \begin{macrocode}
\newcommand{\childdocforward}[2][]
{
  \begingroup
    \if?#1?
      \def\childdoctmp
      {
        \def\childdocname{#2}
        \def\childdocjob{#2}
        \def\jobname{#2}
        \input{#2}
        \endinput
      }
    \else
      \def\childdoctmp
      {
        \childdocdisable
        \def\childdocname{#2}
        \childdoctrue
        \includeonly{#2}
        \def\childdocjob{#1}
        \def\jobname{#1}
        \input{#1}
        \endinput
      }
    \fi
    \expandafter
  \endgroup
  \childdoctmp
}
%    \end{macrocode}

% \macro{\childdocforwardprefix}
% The command |\childdocforwardprefix| redirects
% compilation to the main or a child file by means of a pattern.
% The prefix |#1| in the current filename is replaced by |#2|
% and the suffix of the current filename is kept
% (it is assumed that the filename does not contain the substring `|~~~|'
% which is used as a delimiter).
% Compilation is handed over to the new file by |\childdocforward|:
%    \begin{macrocode}
\newcommand{\childdocforwardprefix}[3][]
{
  \begingroup
    \def\childdocextract #2##1~~~{\def\childdoctmp{\childdocforward[#1]{#3##1}}}
    \expandafter\childdocextract\childdocname~~~
    \expandafter
  \endgroup
  \childdoctmp
}
%    \end{macrocode}

% \macro{\childdoc}
% The deprecated macro |\childdoc| is a legacy version of |\childdocmain|:
%    \begin{macrocode}
\newcommand{\childdoc}{\childdocmain}
%    \end{macrocode}

% \macro{\childdocredirect}
% The deprecated macro |\childdocredirect| is a legacy version
% of |\childdocforward| and |\childdocforwardprefix|:
%    \begin{macrocode}
\newcommand{\childdocredirect}[2][]
{
  \begingroup
    \if?#1?
      \def\childdoctmp{\childdocforward{#2}}
    \else
      \def\childdoctmp{\childdocforwardprefix{#1}{#2}}
    \fi
    \expandafter
  \endgroup
  \childdoctmp
}
%    \end{macrocode}

%\iffalse
%</package>
%\fi
%
\endinput
|\\
|\childdocforwardprefix[|\textit{main}|]{|\textit{prefix}|}{|\textit{dest}|}|
\end{tabular}
\end{center}
%
the destination file is determined by a pattern
depending on the current file:
To make this work, the current file must be called
`{\textit{prefix}\hspace{0.2em}\textit{suffix}}'
with \textit{prefix} matching precisely the argument.
Processing is then passed on to the file
`{\textit{dest}\hspace{0.2em}\textit{suffix}}'.
Surely, the same effect is achieved by
directly specifying the
argument `{\textit{dest}\hspace{0.2em}\textit{suffix}}'
in the first form.
However, that requires to set up a different file
for each child. With the alternative form of the command
all these files can have exactly the same content
which simplifies setting them up and maintaining them.

For example, the following file |draft.tex|
with a compilation flag |\version| as described in \secref{sec:flags}
compiles the main document as a draft:
%
\begin{center}
\begin{tabular}{l}
|\def\version{draft}|\\
|% \iffalse
%
% childdoc.dtx Copyright (C) 2017-2018 Niklas Beisert
%
% This work may be distributed and/or modified under the
% conditions of the LaTeX Project Public License, either version 1.3
% of this license or (at your option) any later version.
% The latest version of this license is in
%   http://www.latex-project.org/lppl.txt
% and version 1.3 or later is part of all distributions of LaTeX
% version 2005/12/01 or later.
%
% This work has the LPPL maintenance status `maintained'.
%
% The Current Maintainer of this work is Niklas Beisert.
%
% This work consists of the files childdoc.dtx and childdoc.ins
% and the derived files childdoc.def and cdocsamp.tex with
% cdocsch1.tex, cdocsch2.tex, cdocsdrf.tex, cdocsfn1.tex, cdocsfn2.tex.
%
%<package>\ifdefined\childdocmain\endinput\fi
%<package>\ProvidesFile{childdoc.def}[2018/12/30 v2.0 child document driver]
%<samplemain>\ProvidesFile{cdocsamp.tex}[2018/12/30 v2.0 sample for childdoc]
%<*driver>
%\ProvidesFile{childdoc.drv}[2018/12/30 v2.0 childdoc reference manual file]
\PassOptionsToClass{10pt,a4paper}{article}
\documentclass{ltxdoc}

\usepackage[margin=35mm]{geometry}
\usepackage{hyperref}
\usepackage{hyperxmp}
\usepackage[usenames]{color}

\hypersetup{colorlinks=true}
\hypersetup{pdfstartview=FitH}
\hypersetup{pdfpagemode=UseNone}
\hypersetup{pdfsource={}}
\hypersetup{pdflang={en-UK}}
\hypersetup{pdfcopyright={Copyright 2017-2018 Niklas Beisert.
  This work may be distributed and/or modified under the
  conditions of the LaTeX Project Public License, either version 1.3
  of this license or (at your option) any later version.}}
\hypersetup{pdflicenseurl={http://www.latex-project.org/lppl.txt}}
\hypersetup{pdfcontactaddress={ETH Zurich, ITP, HIT K,
  Wolfgang-Pauli-Strasse 27}}
\hypersetup{pdfcontactpostcode={8093}}
\hypersetup{pdfcontactcity={Zurich}}
\hypersetup{pdfcontactcountry={Switzerland}}
\hypersetup{pdfcontactemail={nbeisert@itp.phys.ethz.ch}}
\hypersetup{pdfcontacturl={http://people.phys.ethz.ch/\xmptilde nbeisert/}}

\newcommand{\secref}[1]{\hyperref[#1]{section \ref*{#1}}}

\parskip1ex
\parindent0pt
\let\olditemize\itemize
\def\itemize{\olditemize\parskip0pt}

\begin{document}

\title{The \textsf{childdoc} Package}
\hypersetup{pdftitle={The childdoc Package}}
\author{Niklas Beisert\\[2ex]
  Institut f\"ur Theoretische Physik\\
  Eidgen\"ossische Technische Hochschule Z\"urich\\
  Wolfgang-Pauli-Strasse 27, 8093 Z\"urich, Switzerland\\[1ex]
  \href{mailto:nbeisert@itp.phys.ethz.ch}
  {\texttt{nbeisert@itp.phys.ethz.ch}}}
\hypersetup{pdfauthor={Niklas Beisert}}
\hypersetup{pdfsubject={Manual for the LaTeX2e Package childdoc}}
\date{30 December 2018, \textsf{v2.0}}
\maketitle

\begin{abstract}\noindent
\textsf{childdoc} is a \LaTeXe{} package
that enables the direct compilation
of document sections included by |\include|
to individual files.
\end{abstract}

\begingroup
\parskip0ex
\tableofcontents
\endgroup

%%%%%%%%%%%%%%%%%%%%%%%%%%%%%%%%%%%%%%%%%%%%%%%%%%%%%%%%%%%%%%%%%%%%%%%%%%%%%%%%
%%%%%%%%%%%%%%%%%%%%%%%%%%%%%%%%%%%%%%%%%%%%%%%%%%%%%%%%%%%%%%%%%%%%%%%%%%%%%%%%
\section{Introduction}

\LaTeX{} provides a mechanism to structure a large document (such as a book)
into a main file and several child files (containing the chapters)
using the |\include| command.
This mechanism is beneficial for documents
which span hundreds of pages in order to
make the source file(s) more manageable.
Moreover, compilation can be restricted to
selected child files by means of the |\includeonly| command.
The latter feature can be used to reduce the compilation time while editing
(this was significantly more useful in the earlier days of \LaTeX{})
or to generate a smaller document which is easier to navigate.
Another application of |\includeonly| is to generate
documents consisting of selected parts of the complete document.

However, there are a few drawbacks of the plain |\include| mechanism:
\begin{itemize}
\item
The child files cannot be compiled on their own,
they can only be compiled via the main file.
A naive editing environment
(such as a text editor with an option
to have the current file processed by \LaTeX)
may require one to switch to the main file before compiling;
attempting to compile the child file produces errors.
\item
The main file must be modified (each time)
to adjust the |\includeonly| command
to the present needs. This easily leaves the main file in a messy state.
\item
The generated document will always carry the filename
of the main document. This is inconvenient if
several child files are to be compiled and
to be kept for distribution.
\end{itemize}

The present package provides a simple interface
to make child files individually compilable by \LaTeX{}.
Compiling a child file then has the same effect as compiling
the main file with an |\includeonly| command
to select the appropriate child.
Moreover the generated document will carry the name of the child
rather than the main file.
This resolves all three above issues.

This feature is meant to make the editing of books,
thesis documents and lecture notes somewhat more convenient.
However, the package can also be used efficiently for
composing a series of documents (such as exercise sheets)
which are typically distributed individually.
It then assists the author in generating the individual documents
(potentially in different versions)
as well as a document containing the collected series.
Another application is in developing style files
or other kinds of included material
where compilation of the style file could redirect
to a sample or test file.

%%%%%%%%%%%%%%%%%%%%%%%%%%%%%%%%%%%%%%%%%%%%%%%%%%%%%%%%%%%%%%%%%%%%%%%%%%%%%%%%
%%%%%%%%%%%%%%%%%%%%%%%%%%%%%%%%%%%%%%%%%%%%%%%%%%%%%%%%%%%%%%%%%%%%%%%%%%%%%%%%
\section{Usage}

First of all, the package \textsf{childdoc} is \emph{not} a standard
\LaTeXe{} |.sty| style file! Therefore it needs to be invoked in
a non-standard way.

%%%%%%%%%%%%%%%%%%%%%%%%%%%%%%%%%%%%%%%%%%%%%%%%%%%%%%%%%%%%%%%%%%%%%%%%%%%%%%%%
\subsection{Included Files}
\label{sec:include}

%%%%%%%%%%%%%%%%%%%%%%%%%%%%%%%%%%%%%%%%
\DescribeMacro{\childdocmain}
To use the package, add the commands
\begin{center}
\begin{tabular}{l}
|\input{childdoc.def}|\\
|\childdocmain{}|\\
\end{tabular}
\end{center}
at the very top of the main \LaTeX{} file,
in particular \emph{before} the |\documentclass| statement!
The argument of |\childdocmain| should be left empty
(but it must be present).

%%%%%%%%%%%%%%%%%%%%%%%%%%%%%%%%%%%%%%%%
\DescribeMacro{\childdocof}
Furthermore, add the commands
\begin{center}
\begin{tabular}{l}
|\input{childdoc.def}|\\
|\childdocof{|\textit{main}|}|\\
\end{tabular}
\end{center}
at the top of every child file \textit{child}
which is included by |\include{|\textit{child}|}|
from within the main file
(or at least for those files to be compiled individually).
The argument \textit{main} must be the filename of the main file.

There are a couple of
considerations in setting up the main and child documents:

%%%%%%%%%%%%%%%%%%%%%%%%%%%%%%%%%%%%%%%%
\paragraph{Restrictions.}

Please note the following restrictions:
\begin{itemize}
\item
|\childdocmain| must be called with one argument \textit{main}
to ensure compatibility with earlier version of the package.
It must either be empty (|\childdocmain{}|)
or precisely match the filename of the main file in which it is specified.
See \secref{sec:detection} for further information.
\item
The filename \textit{main} must be specified without the |.tex| extension.
\item
The filename \textit{main} is case sensitive
(even in case-insensitive file systems)
due to internal string comparison.
\item
The argument \textit{main} should be fully expanded, it cannot be a macro.
\item
Subdirectories and special characters should be avoided in filenames.
\item
The command |\childdocmain{|\textit{main}|}| must be followed by a whitespace.
It should not be followed immediately by another command
or by a comment mark `|%|'.
This is because the \TeX{} parser reads the token immediately following
the argument of |\childdocmain| and puts it
at the beginning of every child section;
however, a white\-space is ignored.
\end{itemize}

%%%%%%%%%%%%%%%%%%%%%%%%%%%%%%%%%%%%%%%%
\paragraph{Content of Main File.}

It is advisable to place all content in the child files included by |\include|.
Any output contained in the main file will appear in all child documents
unless suppressed manually;
it cannot be suppressed automatically by the |\includeonly| directive
and thus should normally be avoided.
A method to include some content in the main file
by means of conditional processing is described in \secref{sec:conditional}.

%%%%%%%%%%%%%%%%%%%%%%%%%%%%%%%%%%%%%%%%
\paragraph{Page Numbering.}

When only a part of the document is compiled,
the appropriate numbering of pages
(as well as other status parameters)
is determined from the |.aux| files.
The latter contain information from previous passes.
However this information needs to propagate through
all intermediate child documents.
Therefore the page numbering in child documents may well
be inconsistent until the complete document is compiled at least once.

A useful (if unconventional) way to always ensure a consistent
page numbering is to restart the numbering in each child document
and denote the pages by `\textit{child}|.|\textit{page}'
where \textit{child} represents the chapter/section number of the child file.
This can be achieved by the command
|\numberwithin{page}{|\textit{child}|}|
of the \textsf{amsmath} package
where \textit{child} can be |chapter| or |section|
depending on the chosen structuring.
Alternatively, one can modify the macro |\thepage| appropriately
and reset the counter |page| at the start of each child file.

%%%%%%%%%%%%%%%%%%%%%%%%%%%%%%%%%%%%%%%%%%%%%%%%%%%%%%%%%%%%%%%%%%%%%%%%%%%%%%%%
\subsection{Conditional Processing}
\label{sec:conditional}

The package provides a mechanism to compile different versions
of a document. To customise the versions further some conditional processing
can come in handy to distinguish which version is being compiled.
The package provides two macros to describe the compilation context:

%%%%%%%%%%%%%%%%%%%%%%%%%%%%%%%%%%%%%%%%
\DescribeMacro{\ifchilddoc}
The conditional |\ifchilddoc| distinguishes between the compilation of
child documents and the main document:
%
\begin{center}
|\ifchilddoc |\textit{child-code}| |[|\||else |\textit{main-code}]| \||fi|
\end{center}

%%%%%%%%%%%%%%%%%%%%%%%%%%%%%%%%%%%%%%%%
\DescribeMacro{\childdocname}
\DescribeMacro{\childdocjob}
The macro |\childdocname| contains the filename (without extension)
of the main or child file being processed.
Note that |\childdocjob| will always contain the name of the main file.

%%%%%%%%%%%%%%%%%%%%%%%%%%%%%%%%%%%%%%%%
\paragraph{Title Page.}

Conditional processing can be used to include a title or banner page
in the main document when proper precautions are taken.
Importantly, the code in the main file should ensure that the page counter
(as well as other status parameters which are stored in the |.aux| files)
takes the same value after the conditional processing.
Otherwise the page numbers may take divergent values
depending on which part is compiled.

For example, a title page could be declared by:
%
\begin{center}
\begin{tabular}{l}
|\ifchilddoc\||else|\\
|\addtocounter{page}{-1}|\\
\textit{code for title page}\\
|\newpage|\\
|\||fi|
\end{tabular}
\end{center}
%
A banner page for the child documents can be generated by:
%
\begin{center}
\begin{tabular}{l}
|\ifchilddoc|\\
|\addtocounter{page}{-1}|\\
\textit{code for banner page}\\
|\newpage|\\
|\||fi|
\end{tabular}
\end{center}
%
Here one could write a message such as:
\begin{center}
|This is the part \childdocname{} of \childdocjob{}.|
\end{center}

%%%%%%%%%%%%%%%%%%%%%%%%%%%%%%%%%%%%%%%%%%%%%%%%%%%%%%%%%%%%%%%%%%%%%%%%%%%%%%%%
\subsection{Flags}
\label{sec:flags}

The package makes it easy to generate different versions
of the main or child documents.
To this end compilation flags can be defined
and assigned different default values.
They will be particularly useful in conjunction
with the forwarding mechanism described in \secref{sec:forward}.

For example, it may be useful to have a flag |\version|
which can be set to |draft| or |final|.
The document source will contain some conditional code
depending on the value of |\version|.
Suppose further, the flag should default to |final| for the main file
and to |draft| for child files
which is a natural assignment for editing the document.
This is achieved by placing the following code
in the preamble of the main document
(below the |\childdocmain| directive):
%
\begin{center}
\begin{tabular}{l}
|\ifchilddoc|\\
|\providecommand{\version}{draft}|\\
|\||else|\\
|\providecommand{\version}{final}|\\
|\||fi|
\end{tabular}
\end{center}
%
The definition by |\providecommand| makes sure
that previous definitions are not overwritten.
Further statements |\providecommand{\version}{...}|
can thus be added before the above code to override it.

For the main file, one might add a line
(between |\childdocmain| and the above block)
%
\begin{center}
|%\ifchilddoc\||else\providecommand{\version}{draft}\||fi|
\end{center}
%
which can be uncommented to produce a draft version.
Likewise one can add a line to the very top of a child file
(above the |\childdocof{|\textit{main}|}| directive)
%
\begin{center}
|%\providecommand{\version}{final}|
\end{center}
%
which can be uncommented to produce the final version of this child document.

%%%%%%%%%%%%%%%%%%%%%%%%%%%%%%%%%%%%%%%%%%%%%%%%%%%%%%%%%%%%%%%%%%%%%%%%%%%%%%%%
\subsection{Forwarding}
\label{sec:forward}

Different versions of the main or child documents
using compilation flags as described in \secref{sec:flags}
can be (permanently) stored in different files
for convenient compilation, viewing and distribution.
To this end, the package defines a command
to pass on compilation to a different file:

%%%%%%%%%%%%%%%%%%%%%%%%%%%%%%%%%%%%%%%%
\DescribeMacro{\childdocforward}
The command |\childdocforward| redirects processing to
another source file:
%
\begin{center}
\begin{tabular}{l}
|\input{childdoc.def}|\\
|\childdocforward[|\textit{main}|]{|\textit{dest}|}|\\
\end{tabular}
\end{center}
%
The argument \textit{dest} is the destination file
(without extension).
It should be the main file or one of the child files.
Note that further \textsf{childdoc} directives
such as |\childdocof| and |\childdocforward|
in the indicated file will be processed in this form.
The optional argument \textit{main}
passes on directly to the main file \textit{main}
while pretending to compile the child \textit{dest}.
This form behaves as if \textit{dest}
issues |\childdocof{|\textit{main}|}| right away,
and no further \textsf{childdoc} directives will be processed.

%%%%%%%%%%%%%%%%%%%%%%%%%%%%%%%%%%%%%%%%
\DescribeMacro{\...prefix}
In the alternative form |\childdocforwardprefix|,
%
\begin{center}
\begin{tabular}{l}
|\input{childdoc.def}|\\
|\childdocforwardprefix[|\textit{main}|]{|\textit{prefix}|}{|\textit{dest}|}|
\end{tabular}
\end{center}
%
the destination file is determined by a pattern
depending on the current file:
To make this work, the current file must be called
`{\textit{prefix}\hspace{0.2em}\textit{suffix}}'
with \textit{prefix} matching precisely the argument.
Processing is then passed on to the file
`{\textit{dest}\hspace{0.2em}\textit{suffix}}'.
Surely, the same effect is achieved by
directly specifying the
argument `{\textit{dest}\hspace{0.2em}\textit{suffix}}'
in the first form.
However, that requires to set up a different file
for each child. With the alternative form of the command
all these files can have exactly the same content
which simplifies setting them up and maintaining them.

For example, the following file |draft.tex|
with a compilation flag |\version| as described in \secref{sec:flags}
compiles the main document as a draft:
%
\begin{center}
\begin{tabular}{l}
|\def\version{draft}|\\
|\input{childdoc.def}|\\
|\childdocforward{|\textit{main}|}|
\end{tabular}
\end{center}
%
Likewise, the following files |final|\textit{nn}|.tex|
compile the final version of the child document
|child|\textit{nn}|.tex|:
%
\begin{center}
\begin{tabular}{l}
|\def\version{final}|\\
|\input{childdoc.def}|\\
|\childdocforwardprefix{final}{child}|
\end{tabular}
\end{center}
%

Note that when several versions of a main file and/or of each child file
are to be generated, it may be convenient to set up a |Makefile| or
shell script to automatise the process.

%%%%%%%%%%%%%%%%%%%%%%%%%%%%%%%%%%%%%%%%%%%%%%%%%%%%%%%%%%%%%%%%%%%%%%%%%%%%%%%%
\subsection{Command Line Processing}
\label{sec:commandline}

The effect of redirection files can also be achieved by invoking
the \LaTeX{} compiler with a more elaborate command line.
Most conveniently this should be done as part
of a shell script or a |Makefile|.

When using \textsf{childdoc} in the main file, the following
command lines effectively perform a redirection
(note that depending on the shell being used,
backslashes may have to be doubled: `|\|' $\to$ `|\\|'):
%
\begin{center}
|... -jobname "|\textit{target}|" |\\|"|[\textit{flags}]%
|\input{childdoc.def}\childdocforward[|\textit{main}|]{|\textit{dest}|}"|
\end{center}
%
Here \textit{target} is the name of the output file,
\textit{main} is the name of the main file
and \textit{dest} is the name of the main or child file to be processed
(all filenames without extensions).
The optional argument \textit{main} can be omitted
if \textit{main} matches \textit{dest}.
Optionally, compilation \textit{flags} can be defined via |\def| commands.
This command line makes the \TeX{} engine believe
it is compiling the file \textit{target}
whose content is specified as the latter parameter.
The provided code then forwards the processing to
\textit{main} or \textit{dest} as described in \secref{sec:forward}.

%%%%%%%%%%%%%%%%%%%%%%%%%%%%%%%%%%%%%%%%%%%%%%%%%%%%%%%%%%%%%%%%%%%%%%%%%%%%%%%%
\subsection{Include by Input}
\label{sec:input}

Including child documents by |\include| has some restrictions by design.
Most notably, the content of a child document always occupies
its own set of pages; pages cannot be shared between child documents.
Usually, this behaviour makes perfect sense
because each child document contain an essential part of the document.
However, in some situations it may be desirable to compose
a document from a collection of parts
without having mandatory page breaks between then.
For this case, the package
provides a mechanism to include parts
by |\input| which can also be processed individually.
However, by construction this mechanism
requires manual handling of the content to be output.

%%%%%%%%%%%%%%%%%%%%%%%%%%%%%%%%%%%%%%%%
\DescribeMacro{\ifchilddocmanual}
The main file should be prepared as usual, see \secref{sec:include}.
However, the document body must make a distinction
between processing of an individual part and of the main document, e.g.:
%
\begin{center}
\begin{tabular}{l}
|\ifchilddocmanual|\\
|\input{\childdocname}|\\
|\||else|\\
\textit{document body with }|\input{|\textit{part}|}|\\
|\||fi|
\end{tabular}
\end{center}
%
The conditional |\ifchilddocmanual| is true whenever
a part to be included by |\input| is being compiled,
and the name of the part is stored in |\childdocname|.

%%%%%%%%%%%%%%%%%%%%%%%%%%%%%%%%%%%%%%%%
\DescribeMacro{\childdocby}
Each part to be included by |\input| should start with:
%
\begin{center}
\begin{tabular}{l}
|\input{childdoc.def}|\\
|\childdocby{|\textit{main}|}|\\
\end{tabular}
\end{center}
%
The directive |\childdocby| is similar to |\childdocof|
described in \secref{sec:include},
but the subsequent selection of content must be done manually.
To that end, both |\ifchilddoc| and |\ifchilddocmanual|
will be true upon processing of a part,
and the name of the part is stored in |\childdocname|.
Note that |\jobname| will be set to the filename of the current part
so that each part receives an individual |.aux| file
that does not interfere with the |.aux| file(s) of the main document.
This behaviour can be altered by the alternative form
|\childdocby[*]{|\textit{main}|}| (with a non-empty optional argument)
which uses the |.aux| file of the main document
by setting |\jobname| to \textit{main}.

%%%%%%%%%%%%%%%%%%%%%%%%%%%%%%%%%%%%%%%%%%%%%%%%%%%%%%%%%%%%%%%%%%%%%%%%%%%%%%%%
\subsection{Driver Development}
\label{sec:driver}

The \textsf{childdoc} mechanism can also be use for the development
of definition files such as \LaTeX{} styles or classes.
This case differs from the above setup with multiple parts
included by |\include| in that no |\includeonly| should be invoked.
This can be achieved by starting the include file
(before |\ProvidesPackage|) with:
%
\begin{center}
\begin{tabular}{l}
|\input{childdoc.def}|\\
|\childdocforward{|\textit{main}|}|\\
\end{tabular}
\end{center}
%
or alternatively with:
%
\begin{center}
\begin{tabular}{l}
|\input{childdoc.def}|\\
|\childdocby{|\textit{main}|}|\\
\end{tabular}
\end{center}
%
Both forms have slightly different effects as described above.
The main file is prepared as usual, see \secref{sec:include}.

%%%%%%%%%%%%%%%%%%%%%%%%%%%%%%%%%%%%%%%%%%%%%%%%%%%%%%%%%%%%%%%%%%%%%%%%%%%%%%%%
\subsection{Legacy Detection}
\label{sec:detection}

The directive |\childdocmain| in the main file can detect
whether the complete document or merely a child is to be compiled
even without using the directive |\childdocof|.
This method is deprecated because it is less robust
and there is no compelling reason to use it;
it is merely provided for backward compatibility
and it may be removed in future versions.

If the detection mechanism is to be used,
it is mandatory to correctly specify
the filename of the main file as the argument of |\childdocmain|:
%
\begin{center}
\begin{tabular}{l}
|\input{childdoc.def}|\\
|\childdocmain{|\textit{main}|}|\\
\end{tabular}
\end{center}
%
If |\jobname| does not match the argument \textit{main} of |\childdocmain|,
it is assumed that |\jobname| points to the child file to be compiled.
When using |\childdocmain| with the main file specified as argument,
it suffices to start a child file
with just |\input{|\textit{main}|}|
without loading of the package and using |\childdocof|.
If instead all processing is done
with the appropriate \textsf{childdoc} directives,
the argument of \textit{main} of |\childdocmain| can be empty.

An alternative version of the command line processing described
in \secref{sec:commandline} using the detection mechanism reads:
%
\begin{center}
|... -jobname "|\textit{target}|" "|[\textit{flags}]%
[|\def\jobname{|\textit{dest}|}|]|\input{|\textit{main}|}"|
\end{center}

%%%%%%%%%%%%%%%%%%%%%%%%%%%%%%%%%%%%%%%%%%%%%%%%%%%%%%%%%%%%%%%%%%%%%%%%%%%%%%%%
\subsection{Manual Code}
\label{sec:manual}

In case one cannot be certain whether the definitions file |childdoc.def|
is installed on the target \TeX{} distribution
and one prefers not to ship it,
it is conceivable to paste a few relevant commands into the sources.

To that end, drop all statements |\input{childdoc.def}|
and perform the replacements as outlined below.
Instead of |\childdocmain{|\textit{main}|}| add the following code
to the top of the main file:
%
\begin{center}
\begin{tabular}{l}
|\||ifdefined\childdocname\endinput\||fi\newif\ifchilddoc|\\
|\edef\childdocname{\scantokens\expandafter{\jobname\noexpand}}|\\
|\def\childdocmain{|\textit{main}|}\||ifx\childdocmain\childdocname\||else|\\
|\childdoctrue\includeonly{\childdocname}\let\jobname\childdocmain\||fi|\\
\end{tabular}
\end{center}
%
Instead of |\childdocof{|\textit{main}|}| just include the main file
at the top of each child file:
%
\begin{center}
|\input{|\textit{main}|}|
\end{center}
%
A simple redirection |\childdocforward{|\textit{dest}|}| is achieved by:
%
\begin{center}
|\def\jobname{|\textit{dest}|}\input{\jobname}|
\end{center}
%
The redirection with prefix
|\childdocforwardprefix[|\textit{prefix}|]{|\textit{dest}|}|
is accomplished by:
%
\begin{center}
\begin{tabular}{l}
|{\edef\jobname{\scantokens\expandafter{\jobname\noexpand}}|\\
|\def\redirectjob |\textit{prefix}|#1~~~{\gdef\jobname{|\textit{dest}|#1}}|\\
|\expandafter\redirectjob\jobname~~~}\input{\jobname}|
\end{tabular}
\end{center}

In an alternative approach,
child documents can be compiled by a specific command line
without additional code or specific definitions:
%
\begin{center}
|... -jobname "|\textit{target}|" "|[\textit{flags}]%
|\includeonly{|\textit{dest}|}\input{|\textit{main}|}"|
\end{center}
%

%%%%%%%%%%%%%%%%%%%%%%%%%%%%%%%%%%%%%%%%%%%%%%%%%%%%%%%%%%%%%%%%%%%%%%%%%%%%%%%%
%%%%%%%%%%%%%%%%%%%%%%%%%%%%%%%%%%%%%%%%%%%%%%%%%%%%%%%%%%%%%%%%%%%%%%%%%%%%%%%%
\section{Information}

%%%%%%%%%%%%%%%%%%%%%%%%%%%%%%%%%%%%%%%%%%%%%%%%%%%%%%%%%%%%%%%%%%%%%%%%%%%%%%%%
\subsection{Copyright}

Copyright \copyright{} 2017--2018 Niklas Beisert

This work may be distributed and/or modified under the
conditions of the \LaTeX{} Project Public License, either version 1.3
of this license or (at your option) any later version.
The latest version of this license is in
  \url{http://www.latex-project.org/lppl.txt}
and version 1.3 or later is part of all distributions of \LaTeX{}
version 2005/12/01 or later.

This work has the LPPL maintenance status `maintained'.

The Current Maintainer of this work is Niklas Beisert.

This work consists of the files |README.txt|, |childdoc.ins| and |childdoc.dtx|
as well as the derived files |childdoc.def|, |cdocsamp.tex|
with |cdocsch1.tex|, |cdocsch2.tex|, |cdocspt3.tex|, |cdocspt4.tex|,
|cdocsdrf.tex|, |cdocsfn1.tex|, |cdocsfn2.tex|
as well as |childdoc.pdf|.

%%%%%%%%%%%%%%%%%%%%%%%%%%%%%%%%%%%%%%%%%%%%%%%%%%%%%%%%%%%%%%%%%%%%%%%%%%%%%%%%
\subsection{Files and Installation}

The package consists of the files:
%
\begin{center}
\begin{tabular}{ll}
    |README.txt|   & readme file \\
    |childdoc.ins| & installation file \\
    |childdoc.dtx| & source file \\
    |childdoc.def| & definition file \\
    |cdocsamp.tex| & sample main file \\
    |cdocsch1.tex| & sample include file \\
    |cdocsch2.tex| & sample include file \\
    |cdocspt3.tex| & sample part file \\
    |cdocspt4.tex| & sample part file \\
    |cdocsdrf.tex| & sample redirection file \\
    |cdocsfn1.tex| & sample redirection file \\
    |cdocsfn2.tex| & sample redirection file \\
    |childdoc.pdf| & manual
\end{tabular}
\end{center}
%
The distribution consists of the files
|README.txt|, |childdoc.ins| and |childdoc.dtx|.
%
\begin{itemize}
\item
Run (pdf)\LaTeX{} on |childdoc.dtx|
to compile the manual |childdoc.pdf| (this file).
\item
Run \LaTeX{} on |childdoc.ins| to create the definitions file |childdoc.def|
and the sample |cdocsamp.tex| with include files
|cdocsch1.tex|, |cdocsch2.tex|, |cdocspt3.tex|, |cdocspt4.tex|,
|cdocsdrf.tex|, |cdocsfn1.tex|, |cdocsfn2.tex|.
Then copy the file |childdoc.def| to an appropriate directory of your \LaTeX{}
distribution, e.g.\ \textit{texmf-root}|/tex/latex/childdoc|.
\end{itemize}

%%%%%%%%%%%%%%%%%%%%%%%%%%%%%%%%%%%%%%%%%%%%%%%%%%%%%%%%%%%%%%%%%%%%%%%%%%%%%%%%
\subsection{Related CTAN Packages}

There are several other packages which offer a similar functionality:
%
\begin{itemize}
\item
The packages
\href{http://ctan.org/pkg/docmute}{\textsf{docmute}},
\href{http://ctan.org/pkg/includex}{\textsf{includex}} and
\href{http://ctan.org/pkg/standalone}{\textsf{standalone}}
provide commands to include only the document body of
a child file thus allowing both files to be compiled individually.
\item
The packages \href{http://ctan.org/pkg/subdocs}{\textsf{subdocs}}
and \href{http://ctan.org/pkg/subfiles}{\textsf{subfiles}}
provide structures in which the main and child documents can be
encapsulated and allowing them to be compiled individually.
The inclusion mechanism is different from the conventional |\include|.
\item
The package \href{http://ctan.org/pkg/combine}{\textsf{combine}}
is an elaborate solution to combine several documents into one.
\end{itemize}
%
See also the CTAN topic \href{http://ctan.org/topic/subdocs}{\textsf{subdocs}}
for further related packages.
The present package differs from the above solutions in that
a document structure constructed with the conventional |\include| mechanism
just needs two extra commands at the top of every file
such that all constituent files can be compiled individually.

%%%%%%%%%%%%%%%%%%%%%%%%%%%%%%%%%%%%%%%%%%%%%%%%%%%%%%%%%%%%%%%%%%%%%%%%%%%%%%%%
%\subsection{Feature Suggestions}
%
%The following is a list of features which may be useful for future
%versions of this package:
%%
%\begin{itemize}
%\item
%\ldots
%\end{itemize}

%%%%%%%%%%%%%%%%%%%%%%%%%%%%%%%%%%%%%%%%%%%%%%%%%%%%%%%%%%%%%%%%%%%%%%%%%%%%%%%%
\subsection{Revision History}

%%%%%%%%%%%%%%%%%%%%%%%%%%%%%%%%%%%%%%%%
\paragraph{v2.0:} 2018/12/30

\begin{itemize}
\item
immediate forward processing
\item
added |\childdocby| mechanism
\item
manual restructured
\end{itemize}

%%%%%%%%%%%%%%%%%%%%%%%%%%%%%%%%%%%%%%%%
\paragraph{v1.6:} 2018/01/17

\begin{itemize}
\item
application for development of include files
\item
corrections to manual
\end{itemize}

%%%%%%%%%%%%%%%%%%%%%%%%%%%%%%%%%%%%%%%%
\paragraph{v1.5:} 2017/05/21

\begin{itemize}
\item
more complete structuring introduced
\item
|\childdocof| introduced
\item
|\childdoc| renamed to |\childdocmain|
\item
|\childredirect| renamed to |\childdocforward| and |\childdocforwardprefix|
and functionality expanded
\end{itemize}

%%%%%%%%%%%%%%%%%%%%%%%%%%%%%%%%%%%%%%%%
\paragraph{v1.0:} 2017/04/27

\begin{itemize}
\item
manual and install package
\item
first version published on CTAN
\end{itemize}

%%%%%%%%%%%%%%%%%%%%%%%%%%%%%%%%%%%%%%%%
\paragraph{v0.6:} 2017/04/26

\begin{itemize}
\item
redirection mechanism added
\end{itemize}

%%%%%%%%%%%%%%%%%%%%%%%%%%%%%%%%%%%%%%%%
\paragraph{v0.5:} 2017/04/26

\begin{itemize}
\item
functionality in definition file
\end{itemize}


%%%%%%%%%%%%%%%%%%%%%%%%%%%%%%%%%%%%%%%%%%%%%%%%%%%%%%%%%%%%%%%%%%%%%%%%%%%%%%%%
%%%%%%%%%%%%%%%%%%%%%%%%%%%%%%%%%%%%%%%%%%%%%%%%%%%%%%%%%%%%%%%%%%%%%%%%%%%%%%%%
%%%%%%%%%%%%%%%%%%%%%%%%%%%%%%%%%%%%%%%%%%%%%%%%%%%%%%%%%%%%%%%%%%%%%%%%%%%%%%%%
\appendix

\settowidth\MacroIndent{\rmfamily\scriptsize 000\ }

 \DocInput{childdoc.dtx}

\end{document}
%</driver>
% \fi
%
% %%%%%%%%%%%%%%%%%%%%%%%%%%%%%%%%%%%%%%%%%%%%%%%%%%%%%%%%%%%%%%%%%%%%%%%%%%%%%%
% %%%%%%%%%%%%%%%%%%%%%%%%%%%%%%%%%%%%%%%%%%%%%%%%%%%%%%%%%%%%%%%%%%%%%%%%%%%%%%
% \section{Sample}
%\iffalse
%<*samplemain>
%\fi
%
% The following presents a sample document
% with two chapters, two parts, a title page,
% a compile flag as well as three forwarding files to set the flag.
% It consists of eight |.tex| files:
% \begin{center}
% \begin{tabular}{ll}
% |cdocsamp.tex|&main file\\
% |cdocsch1.tex|&include file for chapter 1\\
% |cdocsch2.tex|&include file for chapter 2\\
% |cdocspt3.tex|&include file for part 3\\
% |cdocspt4.tex|&include file for part 4\\
% |cdocsdrf.tex|&forwarding file for main file in draft mode\\
% |cdocsfi1.tex|&forwarding file for final version of chapter 1\\
% |cdocsfi2.tex|&forwarding file for final version of chapter 2\\
% \end{tabular}
% \end{center}
% Each of the eight files can be compiled directly by the \LaTeX{} compiler.
%
% %%%%%%%%%%%%%%%%%%%%%%%%%%%%%%%%%%%%%%
% \paragraph{Main File.}
%
% The main file is called |cdocsamp.tex|.
%
% Load the \textsf{childdoc} definitions and
% declare the filename for the main document:
%    \begin{macrocode}
\input{childdoc.def}
\childdocmain{}
%    \end{macrocode}

% Optional override for |\version| flag:
%    \begin{macrocode}
%%\ifchilddoc\else\providecommand{\version}{draft}\fi
%    \end{macrocode}

% Define the default values for the |\version| flag
% (|final| for the main file and |draft| for childs):
%    \begin{macrocode}
\ifchilddoc
\providecommand{\version}{draft}
\else
\providecommand{\version}{final}
\fi
%    \end{macrocode}

% Load the standard document class:
%    \begin{macrocode}
\documentclass[12pt]{article}
%    \end{macrocode}

% Start the document body:
%    \begin{macrocode}
\begin{document}
%    \end{macrocode}

% Declare a title page.
% Print title, part of document being processed and version flag:
%    \begin{macrocode}
\addtocounter{page}{-1}
\begin{center}
{\LARGE\bfseries{}childdoc example\par}
\vspace{1cm}
\ifchilddoc
\ifchilddocmanual part\else chapter\fi:
`\childdocname' of `\childdocjob'\par
\else
main document: `\childdocjob'\par
\fi
version: \version\par
\end{center}
\newpage
%    \end{macrocode}

% Manually include selected file,
% otherwise process as usual:
%    \begin{macrocode}
\ifchilddocmanual
\section*{part `\childdocname'}
\input{\childdocname}
\else
%    \end{macrocode}

% Include the two chapters:
%    \begin{macrocode}
\include{cdocsch1}
\include{cdocsch2}
%    \end{macrocode}

% Include the two parts unless only chapters should be displayed:
%    \begin{macrocode}
\ifchilddoc\else
\section{part three}
\input{cdocspt3}
\section{part four}
\input{cdocspt4}
\fi
%    \end{macrocode}

% Process as usual until here:
%    \begin{macrocode}
\fi
%    \end{macrocode}

% End of document body:
%    \begin{macrocode}
\end{document}
%    \end{macrocode}
%\iffalse
%</samplemain>
%\fi
%
% %%%%%%%%%%%%%%%%%%%%%%%%%%%%%%%%%%%%%%
% \paragraph{Chapter Include Files.}
%
% The include files are called |cdocsch1.tex| and |cdocsch2.tex|.
%
%\iffalse
%<*samplechap1|samplechap2>
%\fi

% Optional override for |\version| flag:
%    \begin{macrocode}
%%\providecommand{\version}{final}
%    \end{macrocode}

% Include the main document:
%    \begin{macrocode}
\input{childdoc.def}
\childdocof{cdocsamp}
%    \end{macrocode}

%\iffalse
%</samplechap1|samplechap2>
%\fi
%
%\iffalse
%<*samplechap1>
%\fi
% Some text for chapter 1:
%    \begin{macrocode}
\section{one}
some text in chapter one
%    \end{macrocode}

%\iffalse
%</samplechap1>
%\fi
% Some text for chapter 2:
%\iffalse
%<*samplechap2>
%\fi
%    \begin{macrocode}
\section{two}
more text in chapter two
%    \end{macrocode}

%\iffalse
%</samplechap2>
%\fi
%
% %%%%%%%%%%%%%%%%%%%%%%%%%%%%%%%%%%%%%%
% \paragraph{Part Include Files.}
%
% The include files are called |cdocspt3.tex| and |cdocspt4.tex|.
%
%\iffalse
%<*samplepart3|samplepart4>
%\fi

% Optional override for |\version| flag:
%    \begin{macrocode}
%%\providecommand{\version}{final}
%    \end{macrocode}

% Include the main document:
%    \begin{macrocode}
\input{childdoc.def}
\childdocby{cdocsamp}
%    \end{macrocode}

%\iffalse
%</samplepart3|samplepart4>
%\fi
%
%\iffalse
%<*samplepart3>
%\fi
% Some text for part 3:
%    \begin{macrocode}
some text in part three
%    \end{macrocode}

%\iffalse
%</samplepart3>
%\fi
% Some text for part 4:
%\iffalse
%<*samplepart4>
%\fi
%    \begin{macrocode}
more text in part four
%    \end{macrocode}

%\iffalse
%</samplepart4>
%\fi
%
% %%%%%%%%%%%%%%%%%%%%%%%%%%%%%%%%%%%%%%
% \paragraph{Forwarding for a Complete Draft.}
%
% The following forwarding file |cdocsdrf.tex|
% compiles the main document in draft mode:
%\iffalse
%<*sampledraft>
%\fi
%    \begin{macrocode}
\def\version{draft}
\input{childdoc.def}
\childdocforward{cdocsamp}
%    \end{macrocode}

%\iffalse
%</sampledraft>
%\fi
%
% %%%%%%%%%%%%%%%%%%%%%%%%%%%%%%%%%%%%%%
% \paragraph{Forwarding for Final Version of the Chapters.}
%
% The following forwarding files |cdocsfn1.tex| and |cdocsfn2.tex|
% (with identical content)
% compile the final versions of the child documents
% |cdocsch1.tex| and |cdocsch2.tex|, respectively:
%\iffalse
%<*samplefinal>
%\fi
%    \begin{macrocode}
\def\version{final}
\input{childdoc.def}
\childdocforwardprefix[cdocsamp]{cdocsfn}{cdocsch}
%    \end{macrocode}

%\iffalse
%</samplefinal>
%\fi
%
% %%%%%%%%%%%%%%%%%%%%%%%%%%%%%%%%%%%%%%
% \paragraph{Command Line Processing.}
%
% The following three command lines generate the output files
% |cdocscld|, |cdocscl1| and |cdocscl2|
% which should be identical to
% |cdocsdrf|, |cdocsch1| and |cdocsfn2|, respectively:
% \begin{center}
% \begin{tabular}{l}
% |latex -jobname cdocscld \|\\
% |  "\def\version{draft}\input{childdoc.def}\childdocforward{cdocsamp}"|\\
% |latex -jobname cdocscl1 \|\\
% |  "\input{childdoc.def}\childdocforward[cdocsamp]{cdocsch1}"|\\
% |latex -jobname cdocscl2 \|\\
% |  "\def\version{final}\input{childdoc.def}\childdocforward{cdocsch2}"|
% \end{tabular}
% \end{center}
% Note that the trailing backslash on each first line
% merely continues the input to the second line
% (for convenient cut ant paste).
% Furthermore, the command |latex| can be replaced by any
% of its alternative versions such as |pdflatex|.
%
% %%%%%%%%%%%%%%%%%%%%%%%%%%%%%%%%%%%%%%%%%%%%%%%%%%%%%%%%%%%%%%%%%%%%%%%%%%%%%%
% %%%%%%%%%%%%%%%%%%%%%%%%%%%%%%%%%%%%%%%%%%%%%%%%%%%%%%%%%%%%%%%%%%%%%%%%%%%%%%
% \section{Implementation}
%\iffalse
%<*package>
%\fi
%
% This section describes the definitions file |childdoc.def|.

% The definitions cannot be loaded using |\usepackage| or |\RequirePackage|
% which has a mechanism to prevent loading a style file more than once.
% When loading the definitions by means of |\input|
% multiple instances have to be prevented manually:
%\iffalse
%This code needs to be before the `\ProvidesFile' directive
%which is defined at the beginning of this file.
%Therefore it is also placed there and commented out here.
%</package>
%<*discard>
%\fi
%    \begin{macrocode}
\ifdefined\childdocmain\endinput\fi
%    \end{macrocode}
%\iffalse
%</discard>
%<*package>
%\fi
%
% \macro{\ifchilddoc}
% \macro{\ifchilddocmanual}
% The conditional |\ifchilddoc| tells whether a
% child (true) or main (false) document is being compiled.
% The conditional |\ifchilddocmanual| tells whether
% the |\includeonly| mechanism is used (false) or
% the selection of child files must be performed manually (true).
% The definitions initialise to false:
%    \begin{macrocode}
\newif\ifchilddoc
\newif\ifchilddocmanual
%    \end{macrocode}

% \macro{\childdocname}
% \macro{\childdocjob}
% The macro |\childdocname| stores the name of the main document
% to be compiled. The macro |\childdocjob| stores the name of
% the document on which the \LaTeX{} compiler was originally invoked.
% The content of |\jobname| cannot be compared
% to filenames specified in the source due to different catcodes.
% The following code rescans |\jobname|, stores the result
% in |\childdocname| and saves a copy in |\childdocjob|:
%    \begin{macrocode}
\edef\childdocname{\scantokens\expandafter{\jobname\noexpand}}
\let\childdocjob\childdocname
%    \end{macrocode}

% \macro{\childdocdisable}
% The macro |\childdocdisable| prevents the main file
% from being processed more than once.
% At this stage, the main document command |\childdocmain|
% is assumed to be called once again where it should do nothing.
% Any subsequent call to it should prevent
% a secondary processing of the main document
% It overwrites the forwarding commands
% |\childdocof| and |\childdocforward|
% with empty macros to prevent further inclusions of the main document:
%    \begin{macrocode}
\newcommand{\childdocdisable}
{
  \renewcommand{\childdocmain}[1]{\renewcommand{\childdocmain}[1]{\endinput}}
  \renewcommand{\childdocof}[1]{}
  \renewcommand{\childdocby}[2][]{}
  \renewcommand{\childdocforward}[2][]{}
  \renewcommand{\childdocdisable}{}
}
%    \end{macrocode}

% \macro{\childdocmain}
% The macro |\childdocmain| is to be called at the top of the main file
% with nothing or the main filename (without extension) as argument.
% First, it breaks loops.
% If the argument is not empty and does not match |\childdocname|
% (which is set by the first inclusion of |childdoc.def|),
% |\ifchilddoc| is set to true, |\includeonly| is applied to the child file
% and |\jobname| is set to the main file
% (for proper handling of |.aux| files):
%    \begin{macrocode}
\newcommand{\childdocmain}[1]
{
  \childdocdisable\childdocmain{}
  \if?#1?\else
    \begingroup
      \def\childdoctmp{#1}
      \ifx\childdoctmp\childdocname
        \def\childdoctmp{}
      \else
        \def\childdoctmp
        {
          \childdoctrue
          \includeonly{\childdocname}
          \def\childdocjob{#1}
          \def\jobname{#1}
        }
      \fi
      \expandafter
    \endgroup
    \childdoctmp
  \fi
}
%    \end{macrocode}

% \macro{\childdocof}
% The command |\childdocof| redirects
% compilation to the main file |#1|.
%    \begin{macrocode}
\newcommand{\childdocof}[1]
{
  \childdocdisable
  \childdoctrue
  \includeonly{\childdocname}
  \def\jobname{#1}
  \def\childdocjob{#1}
  \input{#1}
}
%    \end{macrocode}

% \macro{\childdocby}
% The command |\childdocby| ....
%    \begin{macrocode}
\newcommand{\childdocby}[2][]
{
  \childdocdisable
  \childdoctrue
  \childdocmanualtrue
  \if?#1?\else
    \def\jobname{#2}
  \fi
  \def\childdocjob{#2}
  \input{#2}
  \endinput
}
%    \end{macrocode}

% \macro{\childdocforward}
% The command |\childdocforward| redirects
% compilation to the main file or
% (if the optional argument is given) a child file.
% Parameters are set as if the main file
% or a child file starting with |\childdocof| was compiled.
% Then compilation is handed over to the main file:
%    \begin{macrocode}
\newcommand{\childdocforward}[2][]
{
  \begingroup
    \if?#1?
      \def\childdoctmp
      {
        \def\childdocname{#2}
        \def\childdocjob{#2}
        \def\jobname{#2}
        \input{#2}
        \endinput
      }
    \else
      \def\childdoctmp
      {
        \childdocdisable
        \def\childdocname{#2}
        \childdoctrue
        \includeonly{#2}
        \def\childdocjob{#1}
        \def\jobname{#1}
        \input{#1}
        \endinput
      }
    \fi
    \expandafter
  \endgroup
  \childdoctmp
}
%    \end{macrocode}

% \macro{\childdocforwardprefix}
% The command |\childdocforwardprefix| redirects
% compilation to the main or a child file by means of a pattern.
% The prefix |#1| in the current filename is replaced by |#2|
% and the suffix of the current filename is kept
% (it is assumed that the filename does not contain the substring `|~~~|'
% which is used as a delimiter).
% Compilation is handed over to the new file by |\childdocforward|:
%    \begin{macrocode}
\newcommand{\childdocforwardprefix}[3][]
{
  \begingroup
    \def\childdocextract #2##1~~~{\def\childdoctmp{\childdocforward[#1]{#3##1}}}
    \expandafter\childdocextract\childdocname~~~
    \expandafter
  \endgroup
  \childdoctmp
}
%    \end{macrocode}

% \macro{\childdoc}
% The deprecated macro |\childdoc| is a legacy version of |\childdocmain|:
%    \begin{macrocode}
\newcommand{\childdoc}{\childdocmain}
%    \end{macrocode}

% \macro{\childdocredirect}
% The deprecated macro |\childdocredirect| is a legacy version
% of |\childdocforward| and |\childdocforwardprefix|:
%    \begin{macrocode}
\newcommand{\childdocredirect}[2][]
{
  \begingroup
    \if?#1?
      \def\childdoctmp{\childdocforward{#2}}
    \else
      \def\childdoctmp{\childdocforwardprefix{#1}{#2}}
    \fi
    \expandafter
  \endgroup
  \childdoctmp
}
%    \end{macrocode}

%\iffalse
%</package>
%\fi
%
\endinput
|\\
|\childdocforward{|\textit{main}|}|
\end{tabular}
\end{center}
%
Likewise, the following files |final|\textit{nn}|.tex|
compile the final version of the child document
|child|\textit{nn}|.tex|:
%
\begin{center}
\begin{tabular}{l}
|\def\version{final}|\\
|% \iffalse
%
% childdoc.dtx Copyright (C) 2017-2018 Niklas Beisert
%
% This work may be distributed and/or modified under the
% conditions of the LaTeX Project Public License, either version 1.3
% of this license or (at your option) any later version.
% The latest version of this license is in
%   http://www.latex-project.org/lppl.txt
% and version 1.3 or later is part of all distributions of LaTeX
% version 2005/12/01 or later.
%
% This work has the LPPL maintenance status `maintained'.
%
% The Current Maintainer of this work is Niklas Beisert.
%
% This work consists of the files childdoc.dtx and childdoc.ins
% and the derived files childdoc.def and cdocsamp.tex with
% cdocsch1.tex, cdocsch2.tex, cdocsdrf.tex, cdocsfn1.tex, cdocsfn2.tex.
%
%<package>\ifdefined\childdocmain\endinput\fi
%<package>\ProvidesFile{childdoc.def}[2018/12/30 v2.0 child document driver]
%<samplemain>\ProvidesFile{cdocsamp.tex}[2018/12/30 v2.0 sample for childdoc]
%<*driver>
%\ProvidesFile{childdoc.drv}[2018/12/30 v2.0 childdoc reference manual file]
\PassOptionsToClass{10pt,a4paper}{article}
\documentclass{ltxdoc}

\usepackage[margin=35mm]{geometry}
\usepackage{hyperref}
\usepackage{hyperxmp}
\usepackage[usenames]{color}

\hypersetup{colorlinks=true}
\hypersetup{pdfstartview=FitH}
\hypersetup{pdfpagemode=UseNone}
\hypersetup{pdfsource={}}
\hypersetup{pdflang={en-UK}}
\hypersetup{pdfcopyright={Copyright 2017-2018 Niklas Beisert.
  This work may be distributed and/or modified under the
  conditions of the LaTeX Project Public License, either version 1.3
  of this license or (at your option) any later version.}}
\hypersetup{pdflicenseurl={http://www.latex-project.org/lppl.txt}}
\hypersetup{pdfcontactaddress={ETH Zurich, ITP, HIT K,
  Wolfgang-Pauli-Strasse 27}}
\hypersetup{pdfcontactpostcode={8093}}
\hypersetup{pdfcontactcity={Zurich}}
\hypersetup{pdfcontactcountry={Switzerland}}
\hypersetup{pdfcontactemail={nbeisert@itp.phys.ethz.ch}}
\hypersetup{pdfcontacturl={http://people.phys.ethz.ch/\xmptilde nbeisert/}}

\newcommand{\secref}[1]{\hyperref[#1]{section \ref*{#1}}}

\parskip1ex
\parindent0pt
\let\olditemize\itemize
\def\itemize{\olditemize\parskip0pt}

\begin{document}

\title{The \textsf{childdoc} Package}
\hypersetup{pdftitle={The childdoc Package}}
\author{Niklas Beisert\\[2ex]
  Institut f\"ur Theoretische Physik\\
  Eidgen\"ossische Technische Hochschule Z\"urich\\
  Wolfgang-Pauli-Strasse 27, 8093 Z\"urich, Switzerland\\[1ex]
  \href{mailto:nbeisert@itp.phys.ethz.ch}
  {\texttt{nbeisert@itp.phys.ethz.ch}}}
\hypersetup{pdfauthor={Niklas Beisert}}
\hypersetup{pdfsubject={Manual for the LaTeX2e Package childdoc}}
\date{30 December 2018, \textsf{v2.0}}
\maketitle

\begin{abstract}\noindent
\textsf{childdoc} is a \LaTeXe{} package
that enables the direct compilation
of document sections included by |\include|
to individual files.
\end{abstract}

\begingroup
\parskip0ex
\tableofcontents
\endgroup

%%%%%%%%%%%%%%%%%%%%%%%%%%%%%%%%%%%%%%%%%%%%%%%%%%%%%%%%%%%%%%%%%%%%%%%%%%%%%%%%
%%%%%%%%%%%%%%%%%%%%%%%%%%%%%%%%%%%%%%%%%%%%%%%%%%%%%%%%%%%%%%%%%%%%%%%%%%%%%%%%
\section{Introduction}

\LaTeX{} provides a mechanism to structure a large document (such as a book)
into a main file and several child files (containing the chapters)
using the |\include| command.
This mechanism is beneficial for documents
which span hundreds of pages in order to
make the source file(s) more manageable.
Moreover, compilation can be restricted to
selected child files by means of the |\includeonly| command.
The latter feature can be used to reduce the compilation time while editing
(this was significantly more useful in the earlier days of \LaTeX{})
or to generate a smaller document which is easier to navigate.
Another application of |\includeonly| is to generate
documents consisting of selected parts of the complete document.

However, there are a few drawbacks of the plain |\include| mechanism:
\begin{itemize}
\item
The child files cannot be compiled on their own,
they can only be compiled via the main file.
A naive editing environment
(such as a text editor with an option
to have the current file processed by \LaTeX)
may require one to switch to the main file before compiling;
attempting to compile the child file produces errors.
\item
The main file must be modified (each time)
to adjust the |\includeonly| command
to the present needs. This easily leaves the main file in a messy state.
\item
The generated document will always carry the filename
of the main document. This is inconvenient if
several child files are to be compiled and
to be kept for distribution.
\end{itemize}

The present package provides a simple interface
to make child files individually compilable by \LaTeX{}.
Compiling a child file then has the same effect as compiling
the main file with an |\includeonly| command
to select the appropriate child.
Moreover the generated document will carry the name of the child
rather than the main file.
This resolves all three above issues.

This feature is meant to make the editing of books,
thesis documents and lecture notes somewhat more convenient.
However, the package can also be used efficiently for
composing a series of documents (such as exercise sheets)
which are typically distributed individually.
It then assists the author in generating the individual documents
(potentially in different versions)
as well as a document containing the collected series.
Another application is in developing style files
or other kinds of included material
where compilation of the style file could redirect
to a sample or test file.

%%%%%%%%%%%%%%%%%%%%%%%%%%%%%%%%%%%%%%%%%%%%%%%%%%%%%%%%%%%%%%%%%%%%%%%%%%%%%%%%
%%%%%%%%%%%%%%%%%%%%%%%%%%%%%%%%%%%%%%%%%%%%%%%%%%%%%%%%%%%%%%%%%%%%%%%%%%%%%%%%
\section{Usage}

First of all, the package \textsf{childdoc} is \emph{not} a standard
\LaTeXe{} |.sty| style file! Therefore it needs to be invoked in
a non-standard way.

%%%%%%%%%%%%%%%%%%%%%%%%%%%%%%%%%%%%%%%%%%%%%%%%%%%%%%%%%%%%%%%%%%%%%%%%%%%%%%%%
\subsection{Included Files}
\label{sec:include}

%%%%%%%%%%%%%%%%%%%%%%%%%%%%%%%%%%%%%%%%
\DescribeMacro{\childdocmain}
To use the package, add the commands
\begin{center}
\begin{tabular}{l}
|\input{childdoc.def}|\\
|\childdocmain{}|\\
\end{tabular}
\end{center}
at the very top of the main \LaTeX{} file,
in particular \emph{before} the |\documentclass| statement!
The argument of |\childdocmain| should be left empty
(but it must be present).

%%%%%%%%%%%%%%%%%%%%%%%%%%%%%%%%%%%%%%%%
\DescribeMacro{\childdocof}
Furthermore, add the commands
\begin{center}
\begin{tabular}{l}
|\input{childdoc.def}|\\
|\childdocof{|\textit{main}|}|\\
\end{tabular}
\end{center}
at the top of every child file \textit{child}
which is included by |\include{|\textit{child}|}|
from within the main file
(or at least for those files to be compiled individually).
The argument \textit{main} must be the filename of the main file.

There are a couple of
considerations in setting up the main and child documents:

%%%%%%%%%%%%%%%%%%%%%%%%%%%%%%%%%%%%%%%%
\paragraph{Restrictions.}

Please note the following restrictions:
\begin{itemize}
\item
|\childdocmain| must be called with one argument \textit{main}
to ensure compatibility with earlier version of the package.
It must either be empty (|\childdocmain{}|)
or precisely match the filename of the main file in which it is specified.
See \secref{sec:detection} for further information.
\item
The filename \textit{main} must be specified without the |.tex| extension.
\item
The filename \textit{main} is case sensitive
(even in case-insensitive file systems)
due to internal string comparison.
\item
The argument \textit{main} should be fully expanded, it cannot be a macro.
\item
Subdirectories and special characters should be avoided in filenames.
\item
The command |\childdocmain{|\textit{main}|}| must be followed by a whitespace.
It should not be followed immediately by another command
or by a comment mark `|%|'.
This is because the \TeX{} parser reads the token immediately following
the argument of |\childdocmain| and puts it
at the beginning of every child section;
however, a white\-space is ignored.
\end{itemize}

%%%%%%%%%%%%%%%%%%%%%%%%%%%%%%%%%%%%%%%%
\paragraph{Content of Main File.}

It is advisable to place all content in the child files included by |\include|.
Any output contained in the main file will appear in all child documents
unless suppressed manually;
it cannot be suppressed automatically by the |\includeonly| directive
and thus should normally be avoided.
A method to include some content in the main file
by means of conditional processing is described in \secref{sec:conditional}.

%%%%%%%%%%%%%%%%%%%%%%%%%%%%%%%%%%%%%%%%
\paragraph{Page Numbering.}

When only a part of the document is compiled,
the appropriate numbering of pages
(as well as other status parameters)
is determined from the |.aux| files.
The latter contain information from previous passes.
However this information needs to propagate through
all intermediate child documents.
Therefore the page numbering in child documents may well
be inconsistent until the complete document is compiled at least once.

A useful (if unconventional) way to always ensure a consistent
page numbering is to restart the numbering in each child document
and denote the pages by `\textit{child}|.|\textit{page}'
where \textit{child} represents the chapter/section number of the child file.
This can be achieved by the command
|\numberwithin{page}{|\textit{child}|}|
of the \textsf{amsmath} package
where \textit{child} can be |chapter| or |section|
depending on the chosen structuring.
Alternatively, one can modify the macro |\thepage| appropriately
and reset the counter |page| at the start of each child file.

%%%%%%%%%%%%%%%%%%%%%%%%%%%%%%%%%%%%%%%%%%%%%%%%%%%%%%%%%%%%%%%%%%%%%%%%%%%%%%%%
\subsection{Conditional Processing}
\label{sec:conditional}

The package provides a mechanism to compile different versions
of a document. To customise the versions further some conditional processing
can come in handy to distinguish which version is being compiled.
The package provides two macros to describe the compilation context:

%%%%%%%%%%%%%%%%%%%%%%%%%%%%%%%%%%%%%%%%
\DescribeMacro{\ifchilddoc}
The conditional |\ifchilddoc| distinguishes between the compilation of
child documents and the main document:
%
\begin{center}
|\ifchilddoc |\textit{child-code}| |[|\||else |\textit{main-code}]| \||fi|
\end{center}

%%%%%%%%%%%%%%%%%%%%%%%%%%%%%%%%%%%%%%%%
\DescribeMacro{\childdocname}
\DescribeMacro{\childdocjob}
The macro |\childdocname| contains the filename (without extension)
of the main or child file being processed.
Note that |\childdocjob| will always contain the name of the main file.

%%%%%%%%%%%%%%%%%%%%%%%%%%%%%%%%%%%%%%%%
\paragraph{Title Page.}

Conditional processing can be used to include a title or banner page
in the main document when proper precautions are taken.
Importantly, the code in the main file should ensure that the page counter
(as well as other status parameters which are stored in the |.aux| files)
takes the same value after the conditional processing.
Otherwise the page numbers may take divergent values
depending on which part is compiled.

For example, a title page could be declared by:
%
\begin{center}
\begin{tabular}{l}
|\ifchilddoc\||else|\\
|\addtocounter{page}{-1}|\\
\textit{code for title page}\\
|\newpage|\\
|\||fi|
\end{tabular}
\end{center}
%
A banner page for the child documents can be generated by:
%
\begin{center}
\begin{tabular}{l}
|\ifchilddoc|\\
|\addtocounter{page}{-1}|\\
\textit{code for banner page}\\
|\newpage|\\
|\||fi|
\end{tabular}
\end{center}
%
Here one could write a message such as:
\begin{center}
|This is the part \childdocname{} of \childdocjob{}.|
\end{center}

%%%%%%%%%%%%%%%%%%%%%%%%%%%%%%%%%%%%%%%%%%%%%%%%%%%%%%%%%%%%%%%%%%%%%%%%%%%%%%%%
\subsection{Flags}
\label{sec:flags}

The package makes it easy to generate different versions
of the main or child documents.
To this end compilation flags can be defined
and assigned different default values.
They will be particularly useful in conjunction
with the forwarding mechanism described in \secref{sec:forward}.

For example, it may be useful to have a flag |\version|
which can be set to |draft| or |final|.
The document source will contain some conditional code
depending on the value of |\version|.
Suppose further, the flag should default to |final| for the main file
and to |draft| for child files
which is a natural assignment for editing the document.
This is achieved by placing the following code
in the preamble of the main document
(below the |\childdocmain| directive):
%
\begin{center}
\begin{tabular}{l}
|\ifchilddoc|\\
|\providecommand{\version}{draft}|\\
|\||else|\\
|\providecommand{\version}{final}|\\
|\||fi|
\end{tabular}
\end{center}
%
The definition by |\providecommand| makes sure
that previous definitions are not overwritten.
Further statements |\providecommand{\version}{...}|
can thus be added before the above code to override it.

For the main file, one might add a line
(between |\childdocmain| and the above block)
%
\begin{center}
|%\ifchilddoc\||else\providecommand{\version}{draft}\||fi|
\end{center}
%
which can be uncommented to produce a draft version.
Likewise one can add a line to the very top of a child file
(above the |\childdocof{|\textit{main}|}| directive)
%
\begin{center}
|%\providecommand{\version}{final}|
\end{center}
%
which can be uncommented to produce the final version of this child document.

%%%%%%%%%%%%%%%%%%%%%%%%%%%%%%%%%%%%%%%%%%%%%%%%%%%%%%%%%%%%%%%%%%%%%%%%%%%%%%%%
\subsection{Forwarding}
\label{sec:forward}

Different versions of the main or child documents
using compilation flags as described in \secref{sec:flags}
can be (permanently) stored in different files
for convenient compilation, viewing and distribution.
To this end, the package defines a command
to pass on compilation to a different file:

%%%%%%%%%%%%%%%%%%%%%%%%%%%%%%%%%%%%%%%%
\DescribeMacro{\childdocforward}
The command |\childdocforward| redirects processing to
another source file:
%
\begin{center}
\begin{tabular}{l}
|\input{childdoc.def}|\\
|\childdocforward[|\textit{main}|]{|\textit{dest}|}|\\
\end{tabular}
\end{center}
%
The argument \textit{dest} is the destination file
(without extension).
It should be the main file or one of the child files.
Note that further \textsf{childdoc} directives
such as |\childdocof| and |\childdocforward|
in the indicated file will be processed in this form.
The optional argument \textit{main}
passes on directly to the main file \textit{main}
while pretending to compile the child \textit{dest}.
This form behaves as if \textit{dest}
issues |\childdocof{|\textit{main}|}| right away,
and no further \textsf{childdoc} directives will be processed.

%%%%%%%%%%%%%%%%%%%%%%%%%%%%%%%%%%%%%%%%
\DescribeMacro{\...prefix}
In the alternative form |\childdocforwardprefix|,
%
\begin{center}
\begin{tabular}{l}
|\input{childdoc.def}|\\
|\childdocforwardprefix[|\textit{main}|]{|\textit{prefix}|}{|\textit{dest}|}|
\end{tabular}
\end{center}
%
the destination file is determined by a pattern
depending on the current file:
To make this work, the current file must be called
`{\textit{prefix}\hspace{0.2em}\textit{suffix}}'
with \textit{prefix} matching precisely the argument.
Processing is then passed on to the file
`{\textit{dest}\hspace{0.2em}\textit{suffix}}'.
Surely, the same effect is achieved by
directly specifying the
argument `{\textit{dest}\hspace{0.2em}\textit{suffix}}'
in the first form.
However, that requires to set up a different file
for each child. With the alternative form of the command
all these files can have exactly the same content
which simplifies setting them up and maintaining them.

For example, the following file |draft.tex|
with a compilation flag |\version| as described in \secref{sec:flags}
compiles the main document as a draft:
%
\begin{center}
\begin{tabular}{l}
|\def\version{draft}|\\
|\input{childdoc.def}|\\
|\childdocforward{|\textit{main}|}|
\end{tabular}
\end{center}
%
Likewise, the following files |final|\textit{nn}|.tex|
compile the final version of the child document
|child|\textit{nn}|.tex|:
%
\begin{center}
\begin{tabular}{l}
|\def\version{final}|\\
|\input{childdoc.def}|\\
|\childdocforwardprefix{final}{child}|
\end{tabular}
\end{center}
%

Note that when several versions of a main file and/or of each child file
are to be generated, it may be convenient to set up a |Makefile| or
shell script to automatise the process.

%%%%%%%%%%%%%%%%%%%%%%%%%%%%%%%%%%%%%%%%%%%%%%%%%%%%%%%%%%%%%%%%%%%%%%%%%%%%%%%%
\subsection{Command Line Processing}
\label{sec:commandline}

The effect of redirection files can also be achieved by invoking
the \LaTeX{} compiler with a more elaborate command line.
Most conveniently this should be done as part
of a shell script or a |Makefile|.

When using \textsf{childdoc} in the main file, the following
command lines effectively perform a redirection
(note that depending on the shell being used,
backslashes may have to be doubled: `|\|' $\to$ `|\\|'):
%
\begin{center}
|... -jobname "|\textit{target}|" |\\|"|[\textit{flags}]%
|\input{childdoc.def}\childdocforward[|\textit{main}|]{|\textit{dest}|}"|
\end{center}
%
Here \textit{target} is the name of the output file,
\textit{main} is the name of the main file
and \textit{dest} is the name of the main or child file to be processed
(all filenames without extensions).
The optional argument \textit{main} can be omitted
if \textit{main} matches \textit{dest}.
Optionally, compilation \textit{flags} can be defined via |\def| commands.
This command line makes the \TeX{} engine believe
it is compiling the file \textit{target}
whose content is specified as the latter parameter.
The provided code then forwards the processing to
\textit{main} or \textit{dest} as described in \secref{sec:forward}.

%%%%%%%%%%%%%%%%%%%%%%%%%%%%%%%%%%%%%%%%%%%%%%%%%%%%%%%%%%%%%%%%%%%%%%%%%%%%%%%%
\subsection{Include by Input}
\label{sec:input}

Including child documents by |\include| has some restrictions by design.
Most notably, the content of a child document always occupies
its own set of pages; pages cannot be shared between child documents.
Usually, this behaviour makes perfect sense
because each child document contain an essential part of the document.
However, in some situations it may be desirable to compose
a document from a collection of parts
without having mandatory page breaks between then.
For this case, the package
provides a mechanism to include parts
by |\input| which can also be processed individually.
However, by construction this mechanism
requires manual handling of the content to be output.

%%%%%%%%%%%%%%%%%%%%%%%%%%%%%%%%%%%%%%%%
\DescribeMacro{\ifchilddocmanual}
The main file should be prepared as usual, see \secref{sec:include}.
However, the document body must make a distinction
between processing of an individual part and of the main document, e.g.:
%
\begin{center}
\begin{tabular}{l}
|\ifchilddocmanual|\\
|\input{\childdocname}|\\
|\||else|\\
\textit{document body with }|\input{|\textit{part}|}|\\
|\||fi|
\end{tabular}
\end{center}
%
The conditional |\ifchilddocmanual| is true whenever
a part to be included by |\input| is being compiled,
and the name of the part is stored in |\childdocname|.

%%%%%%%%%%%%%%%%%%%%%%%%%%%%%%%%%%%%%%%%
\DescribeMacro{\childdocby}
Each part to be included by |\input| should start with:
%
\begin{center}
\begin{tabular}{l}
|\input{childdoc.def}|\\
|\childdocby{|\textit{main}|}|\\
\end{tabular}
\end{center}
%
The directive |\childdocby| is similar to |\childdocof|
described in \secref{sec:include},
but the subsequent selection of content must be done manually.
To that end, both |\ifchilddoc| and |\ifchilddocmanual|
will be true upon processing of a part,
and the name of the part is stored in |\childdocname|.
Note that |\jobname| will be set to the filename of the current part
so that each part receives an individual |.aux| file
that does not interfere with the |.aux| file(s) of the main document.
This behaviour can be altered by the alternative form
|\childdocby[*]{|\textit{main}|}| (with a non-empty optional argument)
which uses the |.aux| file of the main document
by setting |\jobname| to \textit{main}.

%%%%%%%%%%%%%%%%%%%%%%%%%%%%%%%%%%%%%%%%%%%%%%%%%%%%%%%%%%%%%%%%%%%%%%%%%%%%%%%%
\subsection{Driver Development}
\label{sec:driver}

The \textsf{childdoc} mechanism can also be use for the development
of definition files such as \LaTeX{} styles or classes.
This case differs from the above setup with multiple parts
included by |\include| in that no |\includeonly| should be invoked.
This can be achieved by starting the include file
(before |\ProvidesPackage|) with:
%
\begin{center}
\begin{tabular}{l}
|\input{childdoc.def}|\\
|\childdocforward{|\textit{main}|}|\\
\end{tabular}
\end{center}
%
or alternatively with:
%
\begin{center}
\begin{tabular}{l}
|\input{childdoc.def}|\\
|\childdocby{|\textit{main}|}|\\
\end{tabular}
\end{center}
%
Both forms have slightly different effects as described above.
The main file is prepared as usual, see \secref{sec:include}.

%%%%%%%%%%%%%%%%%%%%%%%%%%%%%%%%%%%%%%%%%%%%%%%%%%%%%%%%%%%%%%%%%%%%%%%%%%%%%%%%
\subsection{Legacy Detection}
\label{sec:detection}

The directive |\childdocmain| in the main file can detect
whether the complete document or merely a child is to be compiled
even without using the directive |\childdocof|.
This method is deprecated because it is less robust
and there is no compelling reason to use it;
it is merely provided for backward compatibility
and it may be removed in future versions.

If the detection mechanism is to be used,
it is mandatory to correctly specify
the filename of the main file as the argument of |\childdocmain|:
%
\begin{center}
\begin{tabular}{l}
|\input{childdoc.def}|\\
|\childdocmain{|\textit{main}|}|\\
\end{tabular}
\end{center}
%
If |\jobname| does not match the argument \textit{main} of |\childdocmain|,
it is assumed that |\jobname| points to the child file to be compiled.
When using |\childdocmain| with the main file specified as argument,
it suffices to start a child file
with just |\input{|\textit{main}|}|
without loading of the package and using |\childdocof|.
If instead all processing is done
with the appropriate \textsf{childdoc} directives,
the argument of \textit{main} of |\childdocmain| can be empty.

An alternative version of the command line processing described
in \secref{sec:commandline} using the detection mechanism reads:
%
\begin{center}
|... -jobname "|\textit{target}|" "|[\textit{flags}]%
[|\def\jobname{|\textit{dest}|}|]|\input{|\textit{main}|}"|
\end{center}

%%%%%%%%%%%%%%%%%%%%%%%%%%%%%%%%%%%%%%%%%%%%%%%%%%%%%%%%%%%%%%%%%%%%%%%%%%%%%%%%
\subsection{Manual Code}
\label{sec:manual}

In case one cannot be certain whether the definitions file |childdoc.def|
is installed on the target \TeX{} distribution
and one prefers not to ship it,
it is conceivable to paste a few relevant commands into the sources.

To that end, drop all statements |\input{childdoc.def}|
and perform the replacements as outlined below.
Instead of |\childdocmain{|\textit{main}|}| add the following code
to the top of the main file:
%
\begin{center}
\begin{tabular}{l}
|\||ifdefined\childdocname\endinput\||fi\newif\ifchilddoc|\\
|\edef\childdocname{\scantokens\expandafter{\jobname\noexpand}}|\\
|\def\childdocmain{|\textit{main}|}\||ifx\childdocmain\childdocname\||else|\\
|\childdoctrue\includeonly{\childdocname}\let\jobname\childdocmain\||fi|\\
\end{tabular}
\end{center}
%
Instead of |\childdocof{|\textit{main}|}| just include the main file
at the top of each child file:
%
\begin{center}
|\input{|\textit{main}|}|
\end{center}
%
A simple redirection |\childdocforward{|\textit{dest}|}| is achieved by:
%
\begin{center}
|\def\jobname{|\textit{dest}|}\input{\jobname}|
\end{center}
%
The redirection with prefix
|\childdocforwardprefix[|\textit{prefix}|]{|\textit{dest}|}|
is accomplished by:
%
\begin{center}
\begin{tabular}{l}
|{\edef\jobname{\scantokens\expandafter{\jobname\noexpand}}|\\
|\def\redirectjob |\textit{prefix}|#1~~~{\gdef\jobname{|\textit{dest}|#1}}|\\
|\expandafter\redirectjob\jobname~~~}\input{\jobname}|
\end{tabular}
\end{center}

In an alternative approach,
child documents can be compiled by a specific command line
without additional code or specific definitions:
%
\begin{center}
|... -jobname "|\textit{target}|" "|[\textit{flags}]%
|\includeonly{|\textit{dest}|}\input{|\textit{main}|}"|
\end{center}
%

%%%%%%%%%%%%%%%%%%%%%%%%%%%%%%%%%%%%%%%%%%%%%%%%%%%%%%%%%%%%%%%%%%%%%%%%%%%%%%%%
%%%%%%%%%%%%%%%%%%%%%%%%%%%%%%%%%%%%%%%%%%%%%%%%%%%%%%%%%%%%%%%%%%%%%%%%%%%%%%%%
\section{Information}

%%%%%%%%%%%%%%%%%%%%%%%%%%%%%%%%%%%%%%%%%%%%%%%%%%%%%%%%%%%%%%%%%%%%%%%%%%%%%%%%
\subsection{Copyright}

Copyright \copyright{} 2017--2018 Niklas Beisert

This work may be distributed and/or modified under the
conditions of the \LaTeX{} Project Public License, either version 1.3
of this license or (at your option) any later version.
The latest version of this license is in
  \url{http://www.latex-project.org/lppl.txt}
and version 1.3 or later is part of all distributions of \LaTeX{}
version 2005/12/01 or later.

This work has the LPPL maintenance status `maintained'.

The Current Maintainer of this work is Niklas Beisert.

This work consists of the files |README.txt|, |childdoc.ins| and |childdoc.dtx|
as well as the derived files |childdoc.def|, |cdocsamp.tex|
with |cdocsch1.tex|, |cdocsch2.tex|, |cdocspt3.tex|, |cdocspt4.tex|,
|cdocsdrf.tex|, |cdocsfn1.tex|, |cdocsfn2.tex|
as well as |childdoc.pdf|.

%%%%%%%%%%%%%%%%%%%%%%%%%%%%%%%%%%%%%%%%%%%%%%%%%%%%%%%%%%%%%%%%%%%%%%%%%%%%%%%%
\subsection{Files and Installation}

The package consists of the files:
%
\begin{center}
\begin{tabular}{ll}
    |README.txt|   & readme file \\
    |childdoc.ins| & installation file \\
    |childdoc.dtx| & source file \\
    |childdoc.def| & definition file \\
    |cdocsamp.tex| & sample main file \\
    |cdocsch1.tex| & sample include file \\
    |cdocsch2.tex| & sample include file \\
    |cdocspt3.tex| & sample part file \\
    |cdocspt4.tex| & sample part file \\
    |cdocsdrf.tex| & sample redirection file \\
    |cdocsfn1.tex| & sample redirection file \\
    |cdocsfn2.tex| & sample redirection file \\
    |childdoc.pdf| & manual
\end{tabular}
\end{center}
%
The distribution consists of the files
|README.txt|, |childdoc.ins| and |childdoc.dtx|.
%
\begin{itemize}
\item
Run (pdf)\LaTeX{} on |childdoc.dtx|
to compile the manual |childdoc.pdf| (this file).
\item
Run \LaTeX{} on |childdoc.ins| to create the definitions file |childdoc.def|
and the sample |cdocsamp.tex| with include files
|cdocsch1.tex|, |cdocsch2.tex|, |cdocspt3.tex|, |cdocspt4.tex|,
|cdocsdrf.tex|, |cdocsfn1.tex|, |cdocsfn2.tex|.
Then copy the file |childdoc.def| to an appropriate directory of your \LaTeX{}
distribution, e.g.\ \textit{texmf-root}|/tex/latex/childdoc|.
\end{itemize}

%%%%%%%%%%%%%%%%%%%%%%%%%%%%%%%%%%%%%%%%%%%%%%%%%%%%%%%%%%%%%%%%%%%%%%%%%%%%%%%%
\subsection{Related CTAN Packages}

There are several other packages which offer a similar functionality:
%
\begin{itemize}
\item
The packages
\href{http://ctan.org/pkg/docmute}{\textsf{docmute}},
\href{http://ctan.org/pkg/includex}{\textsf{includex}} and
\href{http://ctan.org/pkg/standalone}{\textsf{standalone}}
provide commands to include only the document body of
a child file thus allowing both files to be compiled individually.
\item
The packages \href{http://ctan.org/pkg/subdocs}{\textsf{subdocs}}
and \href{http://ctan.org/pkg/subfiles}{\textsf{subfiles}}
provide structures in which the main and child documents can be
encapsulated and allowing them to be compiled individually.
The inclusion mechanism is different from the conventional |\include|.
\item
The package \href{http://ctan.org/pkg/combine}{\textsf{combine}}
is an elaborate solution to combine several documents into one.
\end{itemize}
%
See also the CTAN topic \href{http://ctan.org/topic/subdocs}{\textsf{subdocs}}
for further related packages.
The present package differs from the above solutions in that
a document structure constructed with the conventional |\include| mechanism
just needs two extra commands at the top of every file
such that all constituent files can be compiled individually.

%%%%%%%%%%%%%%%%%%%%%%%%%%%%%%%%%%%%%%%%%%%%%%%%%%%%%%%%%%%%%%%%%%%%%%%%%%%%%%%%
%\subsection{Feature Suggestions}
%
%The following is a list of features which may be useful for future
%versions of this package:
%%
%\begin{itemize}
%\item
%\ldots
%\end{itemize}

%%%%%%%%%%%%%%%%%%%%%%%%%%%%%%%%%%%%%%%%%%%%%%%%%%%%%%%%%%%%%%%%%%%%%%%%%%%%%%%%
\subsection{Revision History}

%%%%%%%%%%%%%%%%%%%%%%%%%%%%%%%%%%%%%%%%
\paragraph{v2.0:} 2018/12/30

\begin{itemize}
\item
immediate forward processing
\item
added |\childdocby| mechanism
\item
manual restructured
\end{itemize}

%%%%%%%%%%%%%%%%%%%%%%%%%%%%%%%%%%%%%%%%
\paragraph{v1.6:} 2018/01/17

\begin{itemize}
\item
application for development of include files
\item
corrections to manual
\end{itemize}

%%%%%%%%%%%%%%%%%%%%%%%%%%%%%%%%%%%%%%%%
\paragraph{v1.5:} 2017/05/21

\begin{itemize}
\item
more complete structuring introduced
\item
|\childdocof| introduced
\item
|\childdoc| renamed to |\childdocmain|
\item
|\childredirect| renamed to |\childdocforward| and |\childdocforwardprefix|
and functionality expanded
\end{itemize}

%%%%%%%%%%%%%%%%%%%%%%%%%%%%%%%%%%%%%%%%
\paragraph{v1.0:} 2017/04/27

\begin{itemize}
\item
manual and install package
\item
first version published on CTAN
\end{itemize}

%%%%%%%%%%%%%%%%%%%%%%%%%%%%%%%%%%%%%%%%
\paragraph{v0.6:} 2017/04/26

\begin{itemize}
\item
redirection mechanism added
\end{itemize}

%%%%%%%%%%%%%%%%%%%%%%%%%%%%%%%%%%%%%%%%
\paragraph{v0.5:} 2017/04/26

\begin{itemize}
\item
functionality in definition file
\end{itemize}


%%%%%%%%%%%%%%%%%%%%%%%%%%%%%%%%%%%%%%%%%%%%%%%%%%%%%%%%%%%%%%%%%%%%%%%%%%%%%%%%
%%%%%%%%%%%%%%%%%%%%%%%%%%%%%%%%%%%%%%%%%%%%%%%%%%%%%%%%%%%%%%%%%%%%%%%%%%%%%%%%
%%%%%%%%%%%%%%%%%%%%%%%%%%%%%%%%%%%%%%%%%%%%%%%%%%%%%%%%%%%%%%%%%%%%%%%%%%%%%%%%
\appendix

\settowidth\MacroIndent{\rmfamily\scriptsize 000\ }

 \DocInput{childdoc.dtx}

\end{document}
%</driver>
% \fi
%
% %%%%%%%%%%%%%%%%%%%%%%%%%%%%%%%%%%%%%%%%%%%%%%%%%%%%%%%%%%%%%%%%%%%%%%%%%%%%%%
% %%%%%%%%%%%%%%%%%%%%%%%%%%%%%%%%%%%%%%%%%%%%%%%%%%%%%%%%%%%%%%%%%%%%%%%%%%%%%%
% \section{Sample}
%\iffalse
%<*samplemain>
%\fi
%
% The following presents a sample document
% with two chapters, two parts, a title page,
% a compile flag as well as three forwarding files to set the flag.
% It consists of eight |.tex| files:
% \begin{center}
% \begin{tabular}{ll}
% |cdocsamp.tex|&main file\\
% |cdocsch1.tex|&include file for chapter 1\\
% |cdocsch2.tex|&include file for chapter 2\\
% |cdocspt3.tex|&include file for part 3\\
% |cdocspt4.tex|&include file for part 4\\
% |cdocsdrf.tex|&forwarding file for main file in draft mode\\
% |cdocsfi1.tex|&forwarding file for final version of chapter 1\\
% |cdocsfi2.tex|&forwarding file for final version of chapter 2\\
% \end{tabular}
% \end{center}
% Each of the eight files can be compiled directly by the \LaTeX{} compiler.
%
% %%%%%%%%%%%%%%%%%%%%%%%%%%%%%%%%%%%%%%
% \paragraph{Main File.}
%
% The main file is called |cdocsamp.tex|.
%
% Load the \textsf{childdoc} definitions and
% declare the filename for the main document:
%    \begin{macrocode}
\input{childdoc.def}
\childdocmain{}
%    \end{macrocode}

% Optional override for |\version| flag:
%    \begin{macrocode}
%%\ifchilddoc\else\providecommand{\version}{draft}\fi
%    \end{macrocode}

% Define the default values for the |\version| flag
% (|final| for the main file and |draft| for childs):
%    \begin{macrocode}
\ifchilddoc
\providecommand{\version}{draft}
\else
\providecommand{\version}{final}
\fi
%    \end{macrocode}

% Load the standard document class:
%    \begin{macrocode}
\documentclass[12pt]{article}
%    \end{macrocode}

% Start the document body:
%    \begin{macrocode}
\begin{document}
%    \end{macrocode}

% Declare a title page.
% Print title, part of document being processed and version flag:
%    \begin{macrocode}
\addtocounter{page}{-1}
\begin{center}
{\LARGE\bfseries{}childdoc example\par}
\vspace{1cm}
\ifchilddoc
\ifchilddocmanual part\else chapter\fi:
`\childdocname' of `\childdocjob'\par
\else
main document: `\childdocjob'\par
\fi
version: \version\par
\end{center}
\newpage
%    \end{macrocode}

% Manually include selected file,
% otherwise process as usual:
%    \begin{macrocode}
\ifchilddocmanual
\section*{part `\childdocname'}
\input{\childdocname}
\else
%    \end{macrocode}

% Include the two chapters:
%    \begin{macrocode}
\include{cdocsch1}
\include{cdocsch2}
%    \end{macrocode}

% Include the two parts unless only chapters should be displayed:
%    \begin{macrocode}
\ifchilddoc\else
\section{part three}
\input{cdocspt3}
\section{part four}
\input{cdocspt4}
\fi
%    \end{macrocode}

% Process as usual until here:
%    \begin{macrocode}
\fi
%    \end{macrocode}

% End of document body:
%    \begin{macrocode}
\end{document}
%    \end{macrocode}
%\iffalse
%</samplemain>
%\fi
%
% %%%%%%%%%%%%%%%%%%%%%%%%%%%%%%%%%%%%%%
% \paragraph{Chapter Include Files.}
%
% The include files are called |cdocsch1.tex| and |cdocsch2.tex|.
%
%\iffalse
%<*samplechap1|samplechap2>
%\fi

% Optional override for |\version| flag:
%    \begin{macrocode}
%%\providecommand{\version}{final}
%    \end{macrocode}

% Include the main document:
%    \begin{macrocode}
\input{childdoc.def}
\childdocof{cdocsamp}
%    \end{macrocode}

%\iffalse
%</samplechap1|samplechap2>
%\fi
%
%\iffalse
%<*samplechap1>
%\fi
% Some text for chapter 1:
%    \begin{macrocode}
\section{one}
some text in chapter one
%    \end{macrocode}

%\iffalse
%</samplechap1>
%\fi
% Some text for chapter 2:
%\iffalse
%<*samplechap2>
%\fi
%    \begin{macrocode}
\section{two}
more text in chapter two
%    \end{macrocode}

%\iffalse
%</samplechap2>
%\fi
%
% %%%%%%%%%%%%%%%%%%%%%%%%%%%%%%%%%%%%%%
% \paragraph{Part Include Files.}
%
% The include files are called |cdocspt3.tex| and |cdocspt4.tex|.
%
%\iffalse
%<*samplepart3|samplepart4>
%\fi

% Optional override for |\version| flag:
%    \begin{macrocode}
%%\providecommand{\version}{final}
%    \end{macrocode}

% Include the main document:
%    \begin{macrocode}
\input{childdoc.def}
\childdocby{cdocsamp}
%    \end{macrocode}

%\iffalse
%</samplepart3|samplepart4>
%\fi
%
%\iffalse
%<*samplepart3>
%\fi
% Some text for part 3:
%    \begin{macrocode}
some text in part three
%    \end{macrocode}

%\iffalse
%</samplepart3>
%\fi
% Some text for part 4:
%\iffalse
%<*samplepart4>
%\fi
%    \begin{macrocode}
more text in part four
%    \end{macrocode}

%\iffalse
%</samplepart4>
%\fi
%
% %%%%%%%%%%%%%%%%%%%%%%%%%%%%%%%%%%%%%%
% \paragraph{Forwarding for a Complete Draft.}
%
% The following forwarding file |cdocsdrf.tex|
% compiles the main document in draft mode:
%\iffalse
%<*sampledraft>
%\fi
%    \begin{macrocode}
\def\version{draft}
\input{childdoc.def}
\childdocforward{cdocsamp}
%    \end{macrocode}

%\iffalse
%</sampledraft>
%\fi
%
% %%%%%%%%%%%%%%%%%%%%%%%%%%%%%%%%%%%%%%
% \paragraph{Forwarding for Final Version of the Chapters.}
%
% The following forwarding files |cdocsfn1.tex| and |cdocsfn2.tex|
% (with identical content)
% compile the final versions of the child documents
% |cdocsch1.tex| and |cdocsch2.tex|, respectively:
%\iffalse
%<*samplefinal>
%\fi
%    \begin{macrocode}
\def\version{final}
\input{childdoc.def}
\childdocforwardprefix[cdocsamp]{cdocsfn}{cdocsch}
%    \end{macrocode}

%\iffalse
%</samplefinal>
%\fi
%
% %%%%%%%%%%%%%%%%%%%%%%%%%%%%%%%%%%%%%%
% \paragraph{Command Line Processing.}
%
% The following three command lines generate the output files
% |cdocscld|, |cdocscl1| and |cdocscl2|
% which should be identical to
% |cdocsdrf|, |cdocsch1| and |cdocsfn2|, respectively:
% \begin{center}
% \begin{tabular}{l}
% |latex -jobname cdocscld \|\\
% |  "\def\version{draft}\input{childdoc.def}\childdocforward{cdocsamp}"|\\
% |latex -jobname cdocscl1 \|\\
% |  "\input{childdoc.def}\childdocforward[cdocsamp]{cdocsch1}"|\\
% |latex -jobname cdocscl2 \|\\
% |  "\def\version{final}\input{childdoc.def}\childdocforward{cdocsch2}"|
% \end{tabular}
% \end{center}
% Note that the trailing backslash on each first line
% merely continues the input to the second line
% (for convenient cut ant paste).
% Furthermore, the command |latex| can be replaced by any
% of its alternative versions such as |pdflatex|.
%
% %%%%%%%%%%%%%%%%%%%%%%%%%%%%%%%%%%%%%%%%%%%%%%%%%%%%%%%%%%%%%%%%%%%%%%%%%%%%%%
% %%%%%%%%%%%%%%%%%%%%%%%%%%%%%%%%%%%%%%%%%%%%%%%%%%%%%%%%%%%%%%%%%%%%%%%%%%%%%%
% \section{Implementation}
%\iffalse
%<*package>
%\fi
%
% This section describes the definitions file |childdoc.def|.

% The definitions cannot be loaded using |\usepackage| or |\RequirePackage|
% which has a mechanism to prevent loading a style file more than once.
% When loading the definitions by means of |\input|
% multiple instances have to be prevented manually:
%\iffalse
%This code needs to be before the `\ProvidesFile' directive
%which is defined at the beginning of this file.
%Therefore it is also placed there and commented out here.
%</package>
%<*discard>
%\fi
%    \begin{macrocode}
\ifdefined\childdocmain\endinput\fi
%    \end{macrocode}
%\iffalse
%</discard>
%<*package>
%\fi
%
% \macro{\ifchilddoc}
% \macro{\ifchilddocmanual}
% The conditional |\ifchilddoc| tells whether a
% child (true) or main (false) document is being compiled.
% The conditional |\ifchilddocmanual| tells whether
% the |\includeonly| mechanism is used (false) or
% the selection of child files must be performed manually (true).
% The definitions initialise to false:
%    \begin{macrocode}
\newif\ifchilddoc
\newif\ifchilddocmanual
%    \end{macrocode}

% \macro{\childdocname}
% \macro{\childdocjob}
% The macro |\childdocname| stores the name of the main document
% to be compiled. The macro |\childdocjob| stores the name of
% the document on which the \LaTeX{} compiler was originally invoked.
% The content of |\jobname| cannot be compared
% to filenames specified in the source due to different catcodes.
% The following code rescans |\jobname|, stores the result
% in |\childdocname| and saves a copy in |\childdocjob|:
%    \begin{macrocode}
\edef\childdocname{\scantokens\expandafter{\jobname\noexpand}}
\let\childdocjob\childdocname
%    \end{macrocode}

% \macro{\childdocdisable}
% The macro |\childdocdisable| prevents the main file
% from being processed more than once.
% At this stage, the main document command |\childdocmain|
% is assumed to be called once again where it should do nothing.
% Any subsequent call to it should prevent
% a secondary processing of the main document
% It overwrites the forwarding commands
% |\childdocof| and |\childdocforward|
% with empty macros to prevent further inclusions of the main document:
%    \begin{macrocode}
\newcommand{\childdocdisable}
{
  \renewcommand{\childdocmain}[1]{\renewcommand{\childdocmain}[1]{\endinput}}
  \renewcommand{\childdocof}[1]{}
  \renewcommand{\childdocby}[2][]{}
  \renewcommand{\childdocforward}[2][]{}
  \renewcommand{\childdocdisable}{}
}
%    \end{macrocode}

% \macro{\childdocmain}
% The macro |\childdocmain| is to be called at the top of the main file
% with nothing or the main filename (without extension) as argument.
% First, it breaks loops.
% If the argument is not empty and does not match |\childdocname|
% (which is set by the first inclusion of |childdoc.def|),
% |\ifchilddoc| is set to true, |\includeonly| is applied to the child file
% and |\jobname| is set to the main file
% (for proper handling of |.aux| files):
%    \begin{macrocode}
\newcommand{\childdocmain}[1]
{
  \childdocdisable\childdocmain{}
  \if?#1?\else
    \begingroup
      \def\childdoctmp{#1}
      \ifx\childdoctmp\childdocname
        \def\childdoctmp{}
      \else
        \def\childdoctmp
        {
          \childdoctrue
          \includeonly{\childdocname}
          \def\childdocjob{#1}
          \def\jobname{#1}
        }
      \fi
      \expandafter
    \endgroup
    \childdoctmp
  \fi
}
%    \end{macrocode}

% \macro{\childdocof}
% The command |\childdocof| redirects
% compilation to the main file |#1|.
%    \begin{macrocode}
\newcommand{\childdocof}[1]
{
  \childdocdisable
  \childdoctrue
  \includeonly{\childdocname}
  \def\jobname{#1}
  \def\childdocjob{#1}
  \input{#1}
}
%    \end{macrocode}

% \macro{\childdocby}
% The command |\childdocby| ....
%    \begin{macrocode}
\newcommand{\childdocby}[2][]
{
  \childdocdisable
  \childdoctrue
  \childdocmanualtrue
  \if?#1?\else
    \def\jobname{#2}
  \fi
  \def\childdocjob{#2}
  \input{#2}
  \endinput
}
%    \end{macrocode}

% \macro{\childdocforward}
% The command |\childdocforward| redirects
% compilation to the main file or
% (if the optional argument is given) a child file.
% Parameters are set as if the main file
% or a child file starting with |\childdocof| was compiled.
% Then compilation is handed over to the main file:
%    \begin{macrocode}
\newcommand{\childdocforward}[2][]
{
  \begingroup
    \if?#1?
      \def\childdoctmp
      {
        \def\childdocname{#2}
        \def\childdocjob{#2}
        \def\jobname{#2}
        \input{#2}
        \endinput
      }
    \else
      \def\childdoctmp
      {
        \childdocdisable
        \def\childdocname{#2}
        \childdoctrue
        \includeonly{#2}
        \def\childdocjob{#1}
        \def\jobname{#1}
        \input{#1}
        \endinput
      }
    \fi
    \expandafter
  \endgroup
  \childdoctmp
}
%    \end{macrocode}

% \macro{\childdocforwardprefix}
% The command |\childdocforwardprefix| redirects
% compilation to the main or a child file by means of a pattern.
% The prefix |#1| in the current filename is replaced by |#2|
% and the suffix of the current filename is kept
% (it is assumed that the filename does not contain the substring `|~~~|'
% which is used as a delimiter).
% Compilation is handed over to the new file by |\childdocforward|:
%    \begin{macrocode}
\newcommand{\childdocforwardprefix}[3][]
{
  \begingroup
    \def\childdocextract #2##1~~~{\def\childdoctmp{\childdocforward[#1]{#3##1}}}
    \expandafter\childdocextract\childdocname~~~
    \expandafter
  \endgroup
  \childdoctmp
}
%    \end{macrocode}

% \macro{\childdoc}
% The deprecated macro |\childdoc| is a legacy version of |\childdocmain|:
%    \begin{macrocode}
\newcommand{\childdoc}{\childdocmain}
%    \end{macrocode}

% \macro{\childdocredirect}
% The deprecated macro |\childdocredirect| is a legacy version
% of |\childdocforward| and |\childdocforwardprefix|:
%    \begin{macrocode}
\newcommand{\childdocredirect}[2][]
{
  \begingroup
    \if?#1?
      \def\childdoctmp{\childdocforward{#2}}
    \else
      \def\childdoctmp{\childdocforwardprefix{#1}{#2}}
    \fi
    \expandafter
  \endgroup
  \childdoctmp
}
%    \end{macrocode}

%\iffalse
%</package>
%\fi
%
\endinput
|\\
|\childdocforwardprefix{final}{child}|
\end{tabular}
\end{center}
%

Note that when several versions of a main file and/or of each child file
are to be generated, it may be convenient to set up a |Makefile| or
shell script to automatise the process.

%%%%%%%%%%%%%%%%%%%%%%%%%%%%%%%%%%%%%%%%%%%%%%%%%%%%%%%%%%%%%%%%%%%%%%%%%%%%%%%%
\subsection{Command Line Processing}
\label{sec:commandline}

The effect of redirection files can also be achieved by invoking
the \LaTeX{} compiler with a more elaborate command line.
Most conveniently this should be done as part
of a shell script or a |Makefile|.

When using \textsf{childdoc} in the main file, the following
command lines effectively perform a redirection
(note that depending on the shell being used,
backslashes may have to be doubled: `|\|' $\to$ `|\\|'):
%
\begin{center}
|... -jobname "|\textit{target}|" |\\|"|[\textit{flags}]%
|% \iffalse
%
% childdoc.dtx Copyright (C) 2017-2018 Niklas Beisert
%
% This work may be distributed and/or modified under the
% conditions of the LaTeX Project Public License, either version 1.3
% of this license or (at your option) any later version.
% The latest version of this license is in
%   http://www.latex-project.org/lppl.txt
% and version 1.3 or later is part of all distributions of LaTeX
% version 2005/12/01 or later.
%
% This work has the LPPL maintenance status `maintained'.
%
% The Current Maintainer of this work is Niklas Beisert.
%
% This work consists of the files childdoc.dtx and childdoc.ins
% and the derived files childdoc.def and cdocsamp.tex with
% cdocsch1.tex, cdocsch2.tex, cdocsdrf.tex, cdocsfn1.tex, cdocsfn2.tex.
%
%<package>\ifdefined\childdocmain\endinput\fi
%<package>\ProvidesFile{childdoc.def}[2018/12/30 v2.0 child document driver]
%<samplemain>\ProvidesFile{cdocsamp.tex}[2018/12/30 v2.0 sample for childdoc]
%<*driver>
%\ProvidesFile{childdoc.drv}[2018/12/30 v2.0 childdoc reference manual file]
\PassOptionsToClass{10pt,a4paper}{article}
\documentclass{ltxdoc}

\usepackage[margin=35mm]{geometry}
\usepackage{hyperref}
\usepackage{hyperxmp}
\usepackage[usenames]{color}

\hypersetup{colorlinks=true}
\hypersetup{pdfstartview=FitH}
\hypersetup{pdfpagemode=UseNone}
\hypersetup{pdfsource={}}
\hypersetup{pdflang={en-UK}}
\hypersetup{pdfcopyright={Copyright 2017-2018 Niklas Beisert.
  This work may be distributed and/or modified under the
  conditions of the LaTeX Project Public License, either version 1.3
  of this license or (at your option) any later version.}}
\hypersetup{pdflicenseurl={http://www.latex-project.org/lppl.txt}}
\hypersetup{pdfcontactaddress={ETH Zurich, ITP, HIT K,
  Wolfgang-Pauli-Strasse 27}}
\hypersetup{pdfcontactpostcode={8093}}
\hypersetup{pdfcontactcity={Zurich}}
\hypersetup{pdfcontactcountry={Switzerland}}
\hypersetup{pdfcontactemail={nbeisert@itp.phys.ethz.ch}}
\hypersetup{pdfcontacturl={http://people.phys.ethz.ch/\xmptilde nbeisert/}}

\newcommand{\secref}[1]{\hyperref[#1]{section \ref*{#1}}}

\parskip1ex
\parindent0pt
\let\olditemize\itemize
\def\itemize{\olditemize\parskip0pt}

\begin{document}

\title{The \textsf{childdoc} Package}
\hypersetup{pdftitle={The childdoc Package}}
\author{Niklas Beisert\\[2ex]
  Institut f\"ur Theoretische Physik\\
  Eidgen\"ossische Technische Hochschule Z\"urich\\
  Wolfgang-Pauli-Strasse 27, 8093 Z\"urich, Switzerland\\[1ex]
  \href{mailto:nbeisert@itp.phys.ethz.ch}
  {\texttt{nbeisert@itp.phys.ethz.ch}}}
\hypersetup{pdfauthor={Niklas Beisert}}
\hypersetup{pdfsubject={Manual for the LaTeX2e Package childdoc}}
\date{30 December 2018, \textsf{v2.0}}
\maketitle

\begin{abstract}\noindent
\textsf{childdoc} is a \LaTeXe{} package
that enables the direct compilation
of document sections included by |\include|
to individual files.
\end{abstract}

\begingroup
\parskip0ex
\tableofcontents
\endgroup

%%%%%%%%%%%%%%%%%%%%%%%%%%%%%%%%%%%%%%%%%%%%%%%%%%%%%%%%%%%%%%%%%%%%%%%%%%%%%%%%
%%%%%%%%%%%%%%%%%%%%%%%%%%%%%%%%%%%%%%%%%%%%%%%%%%%%%%%%%%%%%%%%%%%%%%%%%%%%%%%%
\section{Introduction}

\LaTeX{} provides a mechanism to structure a large document (such as a book)
into a main file and several child files (containing the chapters)
using the |\include| command.
This mechanism is beneficial for documents
which span hundreds of pages in order to
make the source file(s) more manageable.
Moreover, compilation can be restricted to
selected child files by means of the |\includeonly| command.
The latter feature can be used to reduce the compilation time while editing
(this was significantly more useful in the earlier days of \LaTeX{})
or to generate a smaller document which is easier to navigate.
Another application of |\includeonly| is to generate
documents consisting of selected parts of the complete document.

However, there are a few drawbacks of the plain |\include| mechanism:
\begin{itemize}
\item
The child files cannot be compiled on their own,
they can only be compiled via the main file.
A naive editing environment
(such as a text editor with an option
to have the current file processed by \LaTeX)
may require one to switch to the main file before compiling;
attempting to compile the child file produces errors.
\item
The main file must be modified (each time)
to adjust the |\includeonly| command
to the present needs. This easily leaves the main file in a messy state.
\item
The generated document will always carry the filename
of the main document. This is inconvenient if
several child files are to be compiled and
to be kept for distribution.
\end{itemize}

The present package provides a simple interface
to make child files individually compilable by \LaTeX{}.
Compiling a child file then has the same effect as compiling
the main file with an |\includeonly| command
to select the appropriate child.
Moreover the generated document will carry the name of the child
rather than the main file.
This resolves all three above issues.

This feature is meant to make the editing of books,
thesis documents and lecture notes somewhat more convenient.
However, the package can also be used efficiently for
composing a series of documents (such as exercise sheets)
which are typically distributed individually.
It then assists the author in generating the individual documents
(potentially in different versions)
as well as a document containing the collected series.
Another application is in developing style files
or other kinds of included material
where compilation of the style file could redirect
to a sample or test file.

%%%%%%%%%%%%%%%%%%%%%%%%%%%%%%%%%%%%%%%%%%%%%%%%%%%%%%%%%%%%%%%%%%%%%%%%%%%%%%%%
%%%%%%%%%%%%%%%%%%%%%%%%%%%%%%%%%%%%%%%%%%%%%%%%%%%%%%%%%%%%%%%%%%%%%%%%%%%%%%%%
\section{Usage}

First of all, the package \textsf{childdoc} is \emph{not} a standard
\LaTeXe{} |.sty| style file! Therefore it needs to be invoked in
a non-standard way.

%%%%%%%%%%%%%%%%%%%%%%%%%%%%%%%%%%%%%%%%%%%%%%%%%%%%%%%%%%%%%%%%%%%%%%%%%%%%%%%%
\subsection{Included Files}
\label{sec:include}

%%%%%%%%%%%%%%%%%%%%%%%%%%%%%%%%%%%%%%%%
\DescribeMacro{\childdocmain}
To use the package, add the commands
\begin{center}
\begin{tabular}{l}
|\input{childdoc.def}|\\
|\childdocmain{}|\\
\end{tabular}
\end{center}
at the very top of the main \LaTeX{} file,
in particular \emph{before} the |\documentclass| statement!
The argument of |\childdocmain| should be left empty
(but it must be present).

%%%%%%%%%%%%%%%%%%%%%%%%%%%%%%%%%%%%%%%%
\DescribeMacro{\childdocof}
Furthermore, add the commands
\begin{center}
\begin{tabular}{l}
|\input{childdoc.def}|\\
|\childdocof{|\textit{main}|}|\\
\end{tabular}
\end{center}
at the top of every child file \textit{child}
which is included by |\include{|\textit{child}|}|
from within the main file
(or at least for those files to be compiled individually).
The argument \textit{main} must be the filename of the main file.

There are a couple of
considerations in setting up the main and child documents:

%%%%%%%%%%%%%%%%%%%%%%%%%%%%%%%%%%%%%%%%
\paragraph{Restrictions.}

Please note the following restrictions:
\begin{itemize}
\item
|\childdocmain| must be called with one argument \textit{main}
to ensure compatibility with earlier version of the package.
It must either be empty (|\childdocmain{}|)
or precisely match the filename of the main file in which it is specified.
See \secref{sec:detection} for further information.
\item
The filename \textit{main} must be specified without the |.tex| extension.
\item
The filename \textit{main} is case sensitive
(even in case-insensitive file systems)
due to internal string comparison.
\item
The argument \textit{main} should be fully expanded, it cannot be a macro.
\item
Subdirectories and special characters should be avoided in filenames.
\item
The command |\childdocmain{|\textit{main}|}| must be followed by a whitespace.
It should not be followed immediately by another command
or by a comment mark `|%|'.
This is because the \TeX{} parser reads the token immediately following
the argument of |\childdocmain| and puts it
at the beginning of every child section;
however, a white\-space is ignored.
\end{itemize}

%%%%%%%%%%%%%%%%%%%%%%%%%%%%%%%%%%%%%%%%
\paragraph{Content of Main File.}

It is advisable to place all content in the child files included by |\include|.
Any output contained in the main file will appear in all child documents
unless suppressed manually;
it cannot be suppressed automatically by the |\includeonly| directive
and thus should normally be avoided.
A method to include some content in the main file
by means of conditional processing is described in \secref{sec:conditional}.

%%%%%%%%%%%%%%%%%%%%%%%%%%%%%%%%%%%%%%%%
\paragraph{Page Numbering.}

When only a part of the document is compiled,
the appropriate numbering of pages
(as well as other status parameters)
is determined from the |.aux| files.
The latter contain information from previous passes.
However this information needs to propagate through
all intermediate child documents.
Therefore the page numbering in child documents may well
be inconsistent until the complete document is compiled at least once.

A useful (if unconventional) way to always ensure a consistent
page numbering is to restart the numbering in each child document
and denote the pages by `\textit{child}|.|\textit{page}'
where \textit{child} represents the chapter/section number of the child file.
This can be achieved by the command
|\numberwithin{page}{|\textit{child}|}|
of the \textsf{amsmath} package
where \textit{child} can be |chapter| or |section|
depending on the chosen structuring.
Alternatively, one can modify the macro |\thepage| appropriately
and reset the counter |page| at the start of each child file.

%%%%%%%%%%%%%%%%%%%%%%%%%%%%%%%%%%%%%%%%%%%%%%%%%%%%%%%%%%%%%%%%%%%%%%%%%%%%%%%%
\subsection{Conditional Processing}
\label{sec:conditional}

The package provides a mechanism to compile different versions
of a document. To customise the versions further some conditional processing
can come in handy to distinguish which version is being compiled.
The package provides two macros to describe the compilation context:

%%%%%%%%%%%%%%%%%%%%%%%%%%%%%%%%%%%%%%%%
\DescribeMacro{\ifchilddoc}
The conditional |\ifchilddoc| distinguishes between the compilation of
child documents and the main document:
%
\begin{center}
|\ifchilddoc |\textit{child-code}| |[|\||else |\textit{main-code}]| \||fi|
\end{center}

%%%%%%%%%%%%%%%%%%%%%%%%%%%%%%%%%%%%%%%%
\DescribeMacro{\childdocname}
\DescribeMacro{\childdocjob}
The macro |\childdocname| contains the filename (without extension)
of the main or child file being processed.
Note that |\childdocjob| will always contain the name of the main file.

%%%%%%%%%%%%%%%%%%%%%%%%%%%%%%%%%%%%%%%%
\paragraph{Title Page.}

Conditional processing can be used to include a title or banner page
in the main document when proper precautions are taken.
Importantly, the code in the main file should ensure that the page counter
(as well as other status parameters which are stored in the |.aux| files)
takes the same value after the conditional processing.
Otherwise the page numbers may take divergent values
depending on which part is compiled.

For example, a title page could be declared by:
%
\begin{center}
\begin{tabular}{l}
|\ifchilddoc\||else|\\
|\addtocounter{page}{-1}|\\
\textit{code for title page}\\
|\newpage|\\
|\||fi|
\end{tabular}
\end{center}
%
A banner page for the child documents can be generated by:
%
\begin{center}
\begin{tabular}{l}
|\ifchilddoc|\\
|\addtocounter{page}{-1}|\\
\textit{code for banner page}\\
|\newpage|\\
|\||fi|
\end{tabular}
\end{center}
%
Here one could write a message such as:
\begin{center}
|This is the part \childdocname{} of \childdocjob{}.|
\end{center}

%%%%%%%%%%%%%%%%%%%%%%%%%%%%%%%%%%%%%%%%%%%%%%%%%%%%%%%%%%%%%%%%%%%%%%%%%%%%%%%%
\subsection{Flags}
\label{sec:flags}

The package makes it easy to generate different versions
of the main or child documents.
To this end compilation flags can be defined
and assigned different default values.
They will be particularly useful in conjunction
with the forwarding mechanism described in \secref{sec:forward}.

For example, it may be useful to have a flag |\version|
which can be set to |draft| or |final|.
The document source will contain some conditional code
depending on the value of |\version|.
Suppose further, the flag should default to |final| for the main file
and to |draft| for child files
which is a natural assignment for editing the document.
This is achieved by placing the following code
in the preamble of the main document
(below the |\childdocmain| directive):
%
\begin{center}
\begin{tabular}{l}
|\ifchilddoc|\\
|\providecommand{\version}{draft}|\\
|\||else|\\
|\providecommand{\version}{final}|\\
|\||fi|
\end{tabular}
\end{center}
%
The definition by |\providecommand| makes sure
that previous definitions are not overwritten.
Further statements |\providecommand{\version}{...}|
can thus be added before the above code to override it.

For the main file, one might add a line
(between |\childdocmain| and the above block)
%
\begin{center}
|%\ifchilddoc\||else\providecommand{\version}{draft}\||fi|
\end{center}
%
which can be uncommented to produce a draft version.
Likewise one can add a line to the very top of a child file
(above the |\childdocof{|\textit{main}|}| directive)
%
\begin{center}
|%\providecommand{\version}{final}|
\end{center}
%
which can be uncommented to produce the final version of this child document.

%%%%%%%%%%%%%%%%%%%%%%%%%%%%%%%%%%%%%%%%%%%%%%%%%%%%%%%%%%%%%%%%%%%%%%%%%%%%%%%%
\subsection{Forwarding}
\label{sec:forward}

Different versions of the main or child documents
using compilation flags as described in \secref{sec:flags}
can be (permanently) stored in different files
for convenient compilation, viewing and distribution.
To this end, the package defines a command
to pass on compilation to a different file:

%%%%%%%%%%%%%%%%%%%%%%%%%%%%%%%%%%%%%%%%
\DescribeMacro{\childdocforward}
The command |\childdocforward| redirects processing to
another source file:
%
\begin{center}
\begin{tabular}{l}
|\input{childdoc.def}|\\
|\childdocforward[|\textit{main}|]{|\textit{dest}|}|\\
\end{tabular}
\end{center}
%
The argument \textit{dest} is the destination file
(without extension).
It should be the main file or one of the child files.
Note that further \textsf{childdoc} directives
such as |\childdocof| and |\childdocforward|
in the indicated file will be processed in this form.
The optional argument \textit{main}
passes on directly to the main file \textit{main}
while pretending to compile the child \textit{dest}.
This form behaves as if \textit{dest}
issues |\childdocof{|\textit{main}|}| right away,
and no further \textsf{childdoc} directives will be processed.

%%%%%%%%%%%%%%%%%%%%%%%%%%%%%%%%%%%%%%%%
\DescribeMacro{\...prefix}
In the alternative form |\childdocforwardprefix|,
%
\begin{center}
\begin{tabular}{l}
|\input{childdoc.def}|\\
|\childdocforwardprefix[|\textit{main}|]{|\textit{prefix}|}{|\textit{dest}|}|
\end{tabular}
\end{center}
%
the destination file is determined by a pattern
depending on the current file:
To make this work, the current file must be called
`{\textit{prefix}\hspace{0.2em}\textit{suffix}}'
with \textit{prefix} matching precisely the argument.
Processing is then passed on to the file
`{\textit{dest}\hspace{0.2em}\textit{suffix}}'.
Surely, the same effect is achieved by
directly specifying the
argument `{\textit{dest}\hspace{0.2em}\textit{suffix}}'
in the first form.
However, that requires to set up a different file
for each child. With the alternative form of the command
all these files can have exactly the same content
which simplifies setting them up and maintaining them.

For example, the following file |draft.tex|
with a compilation flag |\version| as described in \secref{sec:flags}
compiles the main document as a draft:
%
\begin{center}
\begin{tabular}{l}
|\def\version{draft}|\\
|\input{childdoc.def}|\\
|\childdocforward{|\textit{main}|}|
\end{tabular}
\end{center}
%
Likewise, the following files |final|\textit{nn}|.tex|
compile the final version of the child document
|child|\textit{nn}|.tex|:
%
\begin{center}
\begin{tabular}{l}
|\def\version{final}|\\
|\input{childdoc.def}|\\
|\childdocforwardprefix{final}{child}|
\end{tabular}
\end{center}
%

Note that when several versions of a main file and/or of each child file
are to be generated, it may be convenient to set up a |Makefile| or
shell script to automatise the process.

%%%%%%%%%%%%%%%%%%%%%%%%%%%%%%%%%%%%%%%%%%%%%%%%%%%%%%%%%%%%%%%%%%%%%%%%%%%%%%%%
\subsection{Command Line Processing}
\label{sec:commandline}

The effect of redirection files can also be achieved by invoking
the \LaTeX{} compiler with a more elaborate command line.
Most conveniently this should be done as part
of a shell script or a |Makefile|.

When using \textsf{childdoc} in the main file, the following
command lines effectively perform a redirection
(note that depending on the shell being used,
backslashes may have to be doubled: `|\|' $\to$ `|\\|'):
%
\begin{center}
|... -jobname "|\textit{target}|" |\\|"|[\textit{flags}]%
|\input{childdoc.def}\childdocforward[|\textit{main}|]{|\textit{dest}|}"|
\end{center}
%
Here \textit{target} is the name of the output file,
\textit{main} is the name of the main file
and \textit{dest} is the name of the main or child file to be processed
(all filenames without extensions).
The optional argument \textit{main} can be omitted
if \textit{main} matches \textit{dest}.
Optionally, compilation \textit{flags} can be defined via |\def| commands.
This command line makes the \TeX{} engine believe
it is compiling the file \textit{target}
whose content is specified as the latter parameter.
The provided code then forwards the processing to
\textit{main} or \textit{dest} as described in \secref{sec:forward}.

%%%%%%%%%%%%%%%%%%%%%%%%%%%%%%%%%%%%%%%%%%%%%%%%%%%%%%%%%%%%%%%%%%%%%%%%%%%%%%%%
\subsection{Include by Input}
\label{sec:input}

Including child documents by |\include| has some restrictions by design.
Most notably, the content of a child document always occupies
its own set of pages; pages cannot be shared between child documents.
Usually, this behaviour makes perfect sense
because each child document contain an essential part of the document.
However, in some situations it may be desirable to compose
a document from a collection of parts
without having mandatory page breaks between then.
For this case, the package
provides a mechanism to include parts
by |\input| which can also be processed individually.
However, by construction this mechanism
requires manual handling of the content to be output.

%%%%%%%%%%%%%%%%%%%%%%%%%%%%%%%%%%%%%%%%
\DescribeMacro{\ifchilddocmanual}
The main file should be prepared as usual, see \secref{sec:include}.
However, the document body must make a distinction
between processing of an individual part and of the main document, e.g.:
%
\begin{center}
\begin{tabular}{l}
|\ifchilddocmanual|\\
|\input{\childdocname}|\\
|\||else|\\
\textit{document body with }|\input{|\textit{part}|}|\\
|\||fi|
\end{tabular}
\end{center}
%
The conditional |\ifchilddocmanual| is true whenever
a part to be included by |\input| is being compiled,
and the name of the part is stored in |\childdocname|.

%%%%%%%%%%%%%%%%%%%%%%%%%%%%%%%%%%%%%%%%
\DescribeMacro{\childdocby}
Each part to be included by |\input| should start with:
%
\begin{center}
\begin{tabular}{l}
|\input{childdoc.def}|\\
|\childdocby{|\textit{main}|}|\\
\end{tabular}
\end{center}
%
The directive |\childdocby| is similar to |\childdocof|
described in \secref{sec:include},
but the subsequent selection of content must be done manually.
To that end, both |\ifchilddoc| and |\ifchilddocmanual|
will be true upon processing of a part,
and the name of the part is stored in |\childdocname|.
Note that |\jobname| will be set to the filename of the current part
so that each part receives an individual |.aux| file
that does not interfere with the |.aux| file(s) of the main document.
This behaviour can be altered by the alternative form
|\childdocby[*]{|\textit{main}|}| (with a non-empty optional argument)
which uses the |.aux| file of the main document
by setting |\jobname| to \textit{main}.

%%%%%%%%%%%%%%%%%%%%%%%%%%%%%%%%%%%%%%%%%%%%%%%%%%%%%%%%%%%%%%%%%%%%%%%%%%%%%%%%
\subsection{Driver Development}
\label{sec:driver}

The \textsf{childdoc} mechanism can also be use for the development
of definition files such as \LaTeX{} styles or classes.
This case differs from the above setup with multiple parts
included by |\include| in that no |\includeonly| should be invoked.
This can be achieved by starting the include file
(before |\ProvidesPackage|) with:
%
\begin{center}
\begin{tabular}{l}
|\input{childdoc.def}|\\
|\childdocforward{|\textit{main}|}|\\
\end{tabular}
\end{center}
%
or alternatively with:
%
\begin{center}
\begin{tabular}{l}
|\input{childdoc.def}|\\
|\childdocby{|\textit{main}|}|\\
\end{tabular}
\end{center}
%
Both forms have slightly different effects as described above.
The main file is prepared as usual, see \secref{sec:include}.

%%%%%%%%%%%%%%%%%%%%%%%%%%%%%%%%%%%%%%%%%%%%%%%%%%%%%%%%%%%%%%%%%%%%%%%%%%%%%%%%
\subsection{Legacy Detection}
\label{sec:detection}

The directive |\childdocmain| in the main file can detect
whether the complete document or merely a child is to be compiled
even without using the directive |\childdocof|.
This method is deprecated because it is less robust
and there is no compelling reason to use it;
it is merely provided for backward compatibility
and it may be removed in future versions.

If the detection mechanism is to be used,
it is mandatory to correctly specify
the filename of the main file as the argument of |\childdocmain|:
%
\begin{center}
\begin{tabular}{l}
|\input{childdoc.def}|\\
|\childdocmain{|\textit{main}|}|\\
\end{tabular}
\end{center}
%
If |\jobname| does not match the argument \textit{main} of |\childdocmain|,
it is assumed that |\jobname| points to the child file to be compiled.
When using |\childdocmain| with the main file specified as argument,
it suffices to start a child file
with just |\input{|\textit{main}|}|
without loading of the package and using |\childdocof|.
If instead all processing is done
with the appropriate \textsf{childdoc} directives,
the argument of \textit{main} of |\childdocmain| can be empty.

An alternative version of the command line processing described
in \secref{sec:commandline} using the detection mechanism reads:
%
\begin{center}
|... -jobname "|\textit{target}|" "|[\textit{flags}]%
[|\def\jobname{|\textit{dest}|}|]|\input{|\textit{main}|}"|
\end{center}

%%%%%%%%%%%%%%%%%%%%%%%%%%%%%%%%%%%%%%%%%%%%%%%%%%%%%%%%%%%%%%%%%%%%%%%%%%%%%%%%
\subsection{Manual Code}
\label{sec:manual}

In case one cannot be certain whether the definitions file |childdoc.def|
is installed on the target \TeX{} distribution
and one prefers not to ship it,
it is conceivable to paste a few relevant commands into the sources.

To that end, drop all statements |\input{childdoc.def}|
and perform the replacements as outlined below.
Instead of |\childdocmain{|\textit{main}|}| add the following code
to the top of the main file:
%
\begin{center}
\begin{tabular}{l}
|\||ifdefined\childdocname\endinput\||fi\newif\ifchilddoc|\\
|\edef\childdocname{\scantokens\expandafter{\jobname\noexpand}}|\\
|\def\childdocmain{|\textit{main}|}\||ifx\childdocmain\childdocname\||else|\\
|\childdoctrue\includeonly{\childdocname}\let\jobname\childdocmain\||fi|\\
\end{tabular}
\end{center}
%
Instead of |\childdocof{|\textit{main}|}| just include the main file
at the top of each child file:
%
\begin{center}
|\input{|\textit{main}|}|
\end{center}
%
A simple redirection |\childdocforward{|\textit{dest}|}| is achieved by:
%
\begin{center}
|\def\jobname{|\textit{dest}|}\input{\jobname}|
\end{center}
%
The redirection with prefix
|\childdocforwardprefix[|\textit{prefix}|]{|\textit{dest}|}|
is accomplished by:
%
\begin{center}
\begin{tabular}{l}
|{\edef\jobname{\scantokens\expandafter{\jobname\noexpand}}|\\
|\def\redirectjob |\textit{prefix}|#1~~~{\gdef\jobname{|\textit{dest}|#1}}|\\
|\expandafter\redirectjob\jobname~~~}\input{\jobname}|
\end{tabular}
\end{center}

In an alternative approach,
child documents can be compiled by a specific command line
without additional code or specific definitions:
%
\begin{center}
|... -jobname "|\textit{target}|" "|[\textit{flags}]%
|\includeonly{|\textit{dest}|}\input{|\textit{main}|}"|
\end{center}
%

%%%%%%%%%%%%%%%%%%%%%%%%%%%%%%%%%%%%%%%%%%%%%%%%%%%%%%%%%%%%%%%%%%%%%%%%%%%%%%%%
%%%%%%%%%%%%%%%%%%%%%%%%%%%%%%%%%%%%%%%%%%%%%%%%%%%%%%%%%%%%%%%%%%%%%%%%%%%%%%%%
\section{Information}

%%%%%%%%%%%%%%%%%%%%%%%%%%%%%%%%%%%%%%%%%%%%%%%%%%%%%%%%%%%%%%%%%%%%%%%%%%%%%%%%
\subsection{Copyright}

Copyright \copyright{} 2017--2018 Niklas Beisert

This work may be distributed and/or modified under the
conditions of the \LaTeX{} Project Public License, either version 1.3
of this license or (at your option) any later version.
The latest version of this license is in
  \url{http://www.latex-project.org/lppl.txt}
and version 1.3 or later is part of all distributions of \LaTeX{}
version 2005/12/01 or later.

This work has the LPPL maintenance status `maintained'.

The Current Maintainer of this work is Niklas Beisert.

This work consists of the files |README.txt|, |childdoc.ins| and |childdoc.dtx|
as well as the derived files |childdoc.def|, |cdocsamp.tex|
with |cdocsch1.tex|, |cdocsch2.tex|, |cdocspt3.tex|, |cdocspt4.tex|,
|cdocsdrf.tex|, |cdocsfn1.tex|, |cdocsfn2.tex|
as well as |childdoc.pdf|.

%%%%%%%%%%%%%%%%%%%%%%%%%%%%%%%%%%%%%%%%%%%%%%%%%%%%%%%%%%%%%%%%%%%%%%%%%%%%%%%%
\subsection{Files and Installation}

The package consists of the files:
%
\begin{center}
\begin{tabular}{ll}
    |README.txt|   & readme file \\
    |childdoc.ins| & installation file \\
    |childdoc.dtx| & source file \\
    |childdoc.def| & definition file \\
    |cdocsamp.tex| & sample main file \\
    |cdocsch1.tex| & sample include file \\
    |cdocsch2.tex| & sample include file \\
    |cdocspt3.tex| & sample part file \\
    |cdocspt4.tex| & sample part file \\
    |cdocsdrf.tex| & sample redirection file \\
    |cdocsfn1.tex| & sample redirection file \\
    |cdocsfn2.tex| & sample redirection file \\
    |childdoc.pdf| & manual
\end{tabular}
\end{center}
%
The distribution consists of the files
|README.txt|, |childdoc.ins| and |childdoc.dtx|.
%
\begin{itemize}
\item
Run (pdf)\LaTeX{} on |childdoc.dtx|
to compile the manual |childdoc.pdf| (this file).
\item
Run \LaTeX{} on |childdoc.ins| to create the definitions file |childdoc.def|
and the sample |cdocsamp.tex| with include files
|cdocsch1.tex|, |cdocsch2.tex|, |cdocspt3.tex|, |cdocspt4.tex|,
|cdocsdrf.tex|, |cdocsfn1.tex|, |cdocsfn2.tex|.
Then copy the file |childdoc.def| to an appropriate directory of your \LaTeX{}
distribution, e.g.\ \textit{texmf-root}|/tex/latex/childdoc|.
\end{itemize}

%%%%%%%%%%%%%%%%%%%%%%%%%%%%%%%%%%%%%%%%%%%%%%%%%%%%%%%%%%%%%%%%%%%%%%%%%%%%%%%%
\subsection{Related CTAN Packages}

There are several other packages which offer a similar functionality:
%
\begin{itemize}
\item
The packages
\href{http://ctan.org/pkg/docmute}{\textsf{docmute}},
\href{http://ctan.org/pkg/includex}{\textsf{includex}} and
\href{http://ctan.org/pkg/standalone}{\textsf{standalone}}
provide commands to include only the document body of
a child file thus allowing both files to be compiled individually.
\item
The packages \href{http://ctan.org/pkg/subdocs}{\textsf{subdocs}}
and \href{http://ctan.org/pkg/subfiles}{\textsf{subfiles}}
provide structures in which the main and child documents can be
encapsulated and allowing them to be compiled individually.
The inclusion mechanism is different from the conventional |\include|.
\item
The package \href{http://ctan.org/pkg/combine}{\textsf{combine}}
is an elaborate solution to combine several documents into one.
\end{itemize}
%
See also the CTAN topic \href{http://ctan.org/topic/subdocs}{\textsf{subdocs}}
for further related packages.
The present package differs from the above solutions in that
a document structure constructed with the conventional |\include| mechanism
just needs two extra commands at the top of every file
such that all constituent files can be compiled individually.

%%%%%%%%%%%%%%%%%%%%%%%%%%%%%%%%%%%%%%%%%%%%%%%%%%%%%%%%%%%%%%%%%%%%%%%%%%%%%%%%
%\subsection{Feature Suggestions}
%
%The following is a list of features which may be useful for future
%versions of this package:
%%
%\begin{itemize}
%\item
%\ldots
%\end{itemize}

%%%%%%%%%%%%%%%%%%%%%%%%%%%%%%%%%%%%%%%%%%%%%%%%%%%%%%%%%%%%%%%%%%%%%%%%%%%%%%%%
\subsection{Revision History}

%%%%%%%%%%%%%%%%%%%%%%%%%%%%%%%%%%%%%%%%
\paragraph{v2.0:} 2018/12/30

\begin{itemize}
\item
immediate forward processing
\item
added |\childdocby| mechanism
\item
manual restructured
\end{itemize}

%%%%%%%%%%%%%%%%%%%%%%%%%%%%%%%%%%%%%%%%
\paragraph{v1.6:} 2018/01/17

\begin{itemize}
\item
application for development of include files
\item
corrections to manual
\end{itemize}

%%%%%%%%%%%%%%%%%%%%%%%%%%%%%%%%%%%%%%%%
\paragraph{v1.5:} 2017/05/21

\begin{itemize}
\item
more complete structuring introduced
\item
|\childdocof| introduced
\item
|\childdoc| renamed to |\childdocmain|
\item
|\childredirect| renamed to |\childdocforward| and |\childdocforwardprefix|
and functionality expanded
\end{itemize}

%%%%%%%%%%%%%%%%%%%%%%%%%%%%%%%%%%%%%%%%
\paragraph{v1.0:} 2017/04/27

\begin{itemize}
\item
manual and install package
\item
first version published on CTAN
\end{itemize}

%%%%%%%%%%%%%%%%%%%%%%%%%%%%%%%%%%%%%%%%
\paragraph{v0.6:} 2017/04/26

\begin{itemize}
\item
redirection mechanism added
\end{itemize}

%%%%%%%%%%%%%%%%%%%%%%%%%%%%%%%%%%%%%%%%
\paragraph{v0.5:} 2017/04/26

\begin{itemize}
\item
functionality in definition file
\end{itemize}


%%%%%%%%%%%%%%%%%%%%%%%%%%%%%%%%%%%%%%%%%%%%%%%%%%%%%%%%%%%%%%%%%%%%%%%%%%%%%%%%
%%%%%%%%%%%%%%%%%%%%%%%%%%%%%%%%%%%%%%%%%%%%%%%%%%%%%%%%%%%%%%%%%%%%%%%%%%%%%%%%
%%%%%%%%%%%%%%%%%%%%%%%%%%%%%%%%%%%%%%%%%%%%%%%%%%%%%%%%%%%%%%%%%%%%%%%%%%%%%%%%
\appendix

\settowidth\MacroIndent{\rmfamily\scriptsize 000\ }

 \DocInput{childdoc.dtx}

\end{document}
%</driver>
% \fi
%
% %%%%%%%%%%%%%%%%%%%%%%%%%%%%%%%%%%%%%%%%%%%%%%%%%%%%%%%%%%%%%%%%%%%%%%%%%%%%%%
% %%%%%%%%%%%%%%%%%%%%%%%%%%%%%%%%%%%%%%%%%%%%%%%%%%%%%%%%%%%%%%%%%%%%%%%%%%%%%%
% \section{Sample}
%\iffalse
%<*samplemain>
%\fi
%
% The following presents a sample document
% with two chapters, two parts, a title page,
% a compile flag as well as three forwarding files to set the flag.
% It consists of eight |.tex| files:
% \begin{center}
% \begin{tabular}{ll}
% |cdocsamp.tex|&main file\\
% |cdocsch1.tex|&include file for chapter 1\\
% |cdocsch2.tex|&include file for chapter 2\\
% |cdocspt3.tex|&include file for part 3\\
% |cdocspt4.tex|&include file for part 4\\
% |cdocsdrf.tex|&forwarding file for main file in draft mode\\
% |cdocsfi1.tex|&forwarding file for final version of chapter 1\\
% |cdocsfi2.tex|&forwarding file for final version of chapter 2\\
% \end{tabular}
% \end{center}
% Each of the eight files can be compiled directly by the \LaTeX{} compiler.
%
% %%%%%%%%%%%%%%%%%%%%%%%%%%%%%%%%%%%%%%
% \paragraph{Main File.}
%
% The main file is called |cdocsamp.tex|.
%
% Load the \textsf{childdoc} definitions and
% declare the filename for the main document:
%    \begin{macrocode}
\input{childdoc.def}
\childdocmain{}
%    \end{macrocode}

% Optional override for |\version| flag:
%    \begin{macrocode}
%%\ifchilddoc\else\providecommand{\version}{draft}\fi
%    \end{macrocode}

% Define the default values for the |\version| flag
% (|final| for the main file and |draft| for childs):
%    \begin{macrocode}
\ifchilddoc
\providecommand{\version}{draft}
\else
\providecommand{\version}{final}
\fi
%    \end{macrocode}

% Load the standard document class:
%    \begin{macrocode}
\documentclass[12pt]{article}
%    \end{macrocode}

% Start the document body:
%    \begin{macrocode}
\begin{document}
%    \end{macrocode}

% Declare a title page.
% Print title, part of document being processed and version flag:
%    \begin{macrocode}
\addtocounter{page}{-1}
\begin{center}
{\LARGE\bfseries{}childdoc example\par}
\vspace{1cm}
\ifchilddoc
\ifchilddocmanual part\else chapter\fi:
`\childdocname' of `\childdocjob'\par
\else
main document: `\childdocjob'\par
\fi
version: \version\par
\end{center}
\newpage
%    \end{macrocode}

% Manually include selected file,
% otherwise process as usual:
%    \begin{macrocode}
\ifchilddocmanual
\section*{part `\childdocname'}
\input{\childdocname}
\else
%    \end{macrocode}

% Include the two chapters:
%    \begin{macrocode}
\include{cdocsch1}
\include{cdocsch2}
%    \end{macrocode}

% Include the two parts unless only chapters should be displayed:
%    \begin{macrocode}
\ifchilddoc\else
\section{part three}
\input{cdocspt3}
\section{part four}
\input{cdocspt4}
\fi
%    \end{macrocode}

% Process as usual until here:
%    \begin{macrocode}
\fi
%    \end{macrocode}

% End of document body:
%    \begin{macrocode}
\end{document}
%    \end{macrocode}
%\iffalse
%</samplemain>
%\fi
%
% %%%%%%%%%%%%%%%%%%%%%%%%%%%%%%%%%%%%%%
% \paragraph{Chapter Include Files.}
%
% The include files are called |cdocsch1.tex| and |cdocsch2.tex|.
%
%\iffalse
%<*samplechap1|samplechap2>
%\fi

% Optional override for |\version| flag:
%    \begin{macrocode}
%%\providecommand{\version}{final}
%    \end{macrocode}

% Include the main document:
%    \begin{macrocode}
\input{childdoc.def}
\childdocof{cdocsamp}
%    \end{macrocode}

%\iffalse
%</samplechap1|samplechap2>
%\fi
%
%\iffalse
%<*samplechap1>
%\fi
% Some text for chapter 1:
%    \begin{macrocode}
\section{one}
some text in chapter one
%    \end{macrocode}

%\iffalse
%</samplechap1>
%\fi
% Some text for chapter 2:
%\iffalse
%<*samplechap2>
%\fi
%    \begin{macrocode}
\section{two}
more text in chapter two
%    \end{macrocode}

%\iffalse
%</samplechap2>
%\fi
%
% %%%%%%%%%%%%%%%%%%%%%%%%%%%%%%%%%%%%%%
% \paragraph{Part Include Files.}
%
% The include files are called |cdocspt3.tex| and |cdocspt4.tex|.
%
%\iffalse
%<*samplepart3|samplepart4>
%\fi

% Optional override for |\version| flag:
%    \begin{macrocode}
%%\providecommand{\version}{final}
%    \end{macrocode}

% Include the main document:
%    \begin{macrocode}
\input{childdoc.def}
\childdocby{cdocsamp}
%    \end{macrocode}

%\iffalse
%</samplepart3|samplepart4>
%\fi
%
%\iffalse
%<*samplepart3>
%\fi
% Some text for part 3:
%    \begin{macrocode}
some text in part three
%    \end{macrocode}

%\iffalse
%</samplepart3>
%\fi
% Some text for part 4:
%\iffalse
%<*samplepart4>
%\fi
%    \begin{macrocode}
more text in part four
%    \end{macrocode}

%\iffalse
%</samplepart4>
%\fi
%
% %%%%%%%%%%%%%%%%%%%%%%%%%%%%%%%%%%%%%%
% \paragraph{Forwarding for a Complete Draft.}
%
% The following forwarding file |cdocsdrf.tex|
% compiles the main document in draft mode:
%\iffalse
%<*sampledraft>
%\fi
%    \begin{macrocode}
\def\version{draft}
\input{childdoc.def}
\childdocforward{cdocsamp}
%    \end{macrocode}

%\iffalse
%</sampledraft>
%\fi
%
% %%%%%%%%%%%%%%%%%%%%%%%%%%%%%%%%%%%%%%
% \paragraph{Forwarding for Final Version of the Chapters.}
%
% The following forwarding files |cdocsfn1.tex| and |cdocsfn2.tex|
% (with identical content)
% compile the final versions of the child documents
% |cdocsch1.tex| and |cdocsch2.tex|, respectively:
%\iffalse
%<*samplefinal>
%\fi
%    \begin{macrocode}
\def\version{final}
\input{childdoc.def}
\childdocforwardprefix[cdocsamp]{cdocsfn}{cdocsch}
%    \end{macrocode}

%\iffalse
%</samplefinal>
%\fi
%
% %%%%%%%%%%%%%%%%%%%%%%%%%%%%%%%%%%%%%%
% \paragraph{Command Line Processing.}
%
% The following three command lines generate the output files
% |cdocscld|, |cdocscl1| and |cdocscl2|
% which should be identical to
% |cdocsdrf|, |cdocsch1| and |cdocsfn2|, respectively:
% \begin{center}
% \begin{tabular}{l}
% |latex -jobname cdocscld \|\\
% |  "\def\version{draft}\input{childdoc.def}\childdocforward{cdocsamp}"|\\
% |latex -jobname cdocscl1 \|\\
% |  "\input{childdoc.def}\childdocforward[cdocsamp]{cdocsch1}"|\\
% |latex -jobname cdocscl2 \|\\
% |  "\def\version{final}\input{childdoc.def}\childdocforward{cdocsch2}"|
% \end{tabular}
% \end{center}
% Note that the trailing backslash on each first line
% merely continues the input to the second line
% (for convenient cut ant paste).
% Furthermore, the command |latex| can be replaced by any
% of its alternative versions such as |pdflatex|.
%
% %%%%%%%%%%%%%%%%%%%%%%%%%%%%%%%%%%%%%%%%%%%%%%%%%%%%%%%%%%%%%%%%%%%%%%%%%%%%%%
% %%%%%%%%%%%%%%%%%%%%%%%%%%%%%%%%%%%%%%%%%%%%%%%%%%%%%%%%%%%%%%%%%%%%%%%%%%%%%%
% \section{Implementation}
%\iffalse
%<*package>
%\fi
%
% This section describes the definitions file |childdoc.def|.

% The definitions cannot be loaded using |\usepackage| or |\RequirePackage|
% which has a mechanism to prevent loading a style file more than once.
% When loading the definitions by means of |\input|
% multiple instances have to be prevented manually:
%\iffalse
%This code needs to be before the `\ProvidesFile' directive
%which is defined at the beginning of this file.
%Therefore it is also placed there and commented out here.
%</package>
%<*discard>
%\fi
%    \begin{macrocode}
\ifdefined\childdocmain\endinput\fi
%    \end{macrocode}
%\iffalse
%</discard>
%<*package>
%\fi
%
% \macro{\ifchilddoc}
% \macro{\ifchilddocmanual}
% The conditional |\ifchilddoc| tells whether a
% child (true) or main (false) document is being compiled.
% The conditional |\ifchilddocmanual| tells whether
% the |\includeonly| mechanism is used (false) or
% the selection of child files must be performed manually (true).
% The definitions initialise to false:
%    \begin{macrocode}
\newif\ifchilddoc
\newif\ifchilddocmanual
%    \end{macrocode}

% \macro{\childdocname}
% \macro{\childdocjob}
% The macro |\childdocname| stores the name of the main document
% to be compiled. The macro |\childdocjob| stores the name of
% the document on which the \LaTeX{} compiler was originally invoked.
% The content of |\jobname| cannot be compared
% to filenames specified in the source due to different catcodes.
% The following code rescans |\jobname|, stores the result
% in |\childdocname| and saves a copy in |\childdocjob|:
%    \begin{macrocode}
\edef\childdocname{\scantokens\expandafter{\jobname\noexpand}}
\let\childdocjob\childdocname
%    \end{macrocode}

% \macro{\childdocdisable}
% The macro |\childdocdisable| prevents the main file
% from being processed more than once.
% At this stage, the main document command |\childdocmain|
% is assumed to be called once again where it should do nothing.
% Any subsequent call to it should prevent
% a secondary processing of the main document
% It overwrites the forwarding commands
% |\childdocof| and |\childdocforward|
% with empty macros to prevent further inclusions of the main document:
%    \begin{macrocode}
\newcommand{\childdocdisable}
{
  \renewcommand{\childdocmain}[1]{\renewcommand{\childdocmain}[1]{\endinput}}
  \renewcommand{\childdocof}[1]{}
  \renewcommand{\childdocby}[2][]{}
  \renewcommand{\childdocforward}[2][]{}
  \renewcommand{\childdocdisable}{}
}
%    \end{macrocode}

% \macro{\childdocmain}
% The macro |\childdocmain| is to be called at the top of the main file
% with nothing or the main filename (without extension) as argument.
% First, it breaks loops.
% If the argument is not empty and does not match |\childdocname|
% (which is set by the first inclusion of |childdoc.def|),
% |\ifchilddoc| is set to true, |\includeonly| is applied to the child file
% and |\jobname| is set to the main file
% (for proper handling of |.aux| files):
%    \begin{macrocode}
\newcommand{\childdocmain}[1]
{
  \childdocdisable\childdocmain{}
  \if?#1?\else
    \begingroup
      \def\childdoctmp{#1}
      \ifx\childdoctmp\childdocname
        \def\childdoctmp{}
      \else
        \def\childdoctmp
        {
          \childdoctrue
          \includeonly{\childdocname}
          \def\childdocjob{#1}
          \def\jobname{#1}
        }
      \fi
      \expandafter
    \endgroup
    \childdoctmp
  \fi
}
%    \end{macrocode}

% \macro{\childdocof}
% The command |\childdocof| redirects
% compilation to the main file |#1|.
%    \begin{macrocode}
\newcommand{\childdocof}[1]
{
  \childdocdisable
  \childdoctrue
  \includeonly{\childdocname}
  \def\jobname{#1}
  \def\childdocjob{#1}
  \input{#1}
}
%    \end{macrocode}

% \macro{\childdocby}
% The command |\childdocby| ....
%    \begin{macrocode}
\newcommand{\childdocby}[2][]
{
  \childdocdisable
  \childdoctrue
  \childdocmanualtrue
  \if?#1?\else
    \def\jobname{#2}
  \fi
  \def\childdocjob{#2}
  \input{#2}
  \endinput
}
%    \end{macrocode}

% \macro{\childdocforward}
% The command |\childdocforward| redirects
% compilation to the main file or
% (if the optional argument is given) a child file.
% Parameters are set as if the main file
% or a child file starting with |\childdocof| was compiled.
% Then compilation is handed over to the main file:
%    \begin{macrocode}
\newcommand{\childdocforward}[2][]
{
  \begingroup
    \if?#1?
      \def\childdoctmp
      {
        \def\childdocname{#2}
        \def\childdocjob{#2}
        \def\jobname{#2}
        \input{#2}
        \endinput
      }
    \else
      \def\childdoctmp
      {
        \childdocdisable
        \def\childdocname{#2}
        \childdoctrue
        \includeonly{#2}
        \def\childdocjob{#1}
        \def\jobname{#1}
        \input{#1}
        \endinput
      }
    \fi
    \expandafter
  \endgroup
  \childdoctmp
}
%    \end{macrocode}

% \macro{\childdocforwardprefix}
% The command |\childdocforwardprefix| redirects
% compilation to the main or a child file by means of a pattern.
% The prefix |#1| in the current filename is replaced by |#2|
% and the suffix of the current filename is kept
% (it is assumed that the filename does not contain the substring `|~~~|'
% which is used as a delimiter).
% Compilation is handed over to the new file by |\childdocforward|:
%    \begin{macrocode}
\newcommand{\childdocforwardprefix}[3][]
{
  \begingroup
    \def\childdocextract #2##1~~~{\def\childdoctmp{\childdocforward[#1]{#3##1}}}
    \expandafter\childdocextract\childdocname~~~
    \expandafter
  \endgroup
  \childdoctmp
}
%    \end{macrocode}

% \macro{\childdoc}
% The deprecated macro |\childdoc| is a legacy version of |\childdocmain|:
%    \begin{macrocode}
\newcommand{\childdoc}{\childdocmain}
%    \end{macrocode}

% \macro{\childdocredirect}
% The deprecated macro |\childdocredirect| is a legacy version
% of |\childdocforward| and |\childdocforwardprefix|:
%    \begin{macrocode}
\newcommand{\childdocredirect}[2][]
{
  \begingroup
    \if?#1?
      \def\childdoctmp{\childdocforward{#2}}
    \else
      \def\childdoctmp{\childdocforwardprefix{#1}{#2}}
    \fi
    \expandafter
  \endgroup
  \childdoctmp
}
%    \end{macrocode}

%\iffalse
%</package>
%\fi
%
\endinput
\childdocforward[|\textit{main}|]{|\textit{dest}|}"|
\end{center}
%
Here \textit{target} is the name of the output file,
\textit{main} is the name of the main file
and \textit{dest} is the name of the main or child file to be processed
(all filenames without extensions).
The optional argument \textit{main} can be omitted
if \textit{main} matches \textit{dest}.
Optionally, compilation \textit{flags} can be defined via |\def| commands.
This command line makes the \TeX{} engine believe
it is compiling the file \textit{target}
whose content is specified as the latter parameter.
The provided code then forwards the processing to
\textit{main} or \textit{dest} as described in \secref{sec:forward}.

%%%%%%%%%%%%%%%%%%%%%%%%%%%%%%%%%%%%%%%%%%%%%%%%%%%%%%%%%%%%%%%%%%%%%%%%%%%%%%%%
\subsection{Include by Input}
\label{sec:input}

Including child documents by |\include| has some restrictions by design.
Most notably, the content of a child document always occupies
its own set of pages; pages cannot be shared between child documents.
Usually, this behaviour makes perfect sense
because each child document contain an essential part of the document.
However, in some situations it may be desirable to compose
a document from a collection of parts
without having mandatory page breaks between then.
For this case, the package
provides a mechanism to include parts
by |\input| which can also be processed individually.
However, by construction this mechanism
requires manual handling of the content to be output.

%%%%%%%%%%%%%%%%%%%%%%%%%%%%%%%%%%%%%%%%
\DescribeMacro{\ifchilddocmanual}
The main file should be prepared as usual, see \secref{sec:include}.
However, the document body must make a distinction
between processing of an individual part and of the main document, e.g.:
%
\begin{center}
\begin{tabular}{l}
|\ifchilddocmanual|\\
|\input{\childdocname}|\\
|\||else|\\
\textit{document body with }|\input{|\textit{part}|}|\\
|\||fi|
\end{tabular}
\end{center}
%
The conditional |\ifchilddocmanual| is true whenever
a part to be included by |\input| is being compiled,
and the name of the part is stored in |\childdocname|.

%%%%%%%%%%%%%%%%%%%%%%%%%%%%%%%%%%%%%%%%
\DescribeMacro{\childdocby}
Each part to be included by |\input| should start with:
%
\begin{center}
\begin{tabular}{l}
|% \iffalse
%
% childdoc.dtx Copyright (C) 2017-2018 Niklas Beisert
%
% This work may be distributed and/or modified under the
% conditions of the LaTeX Project Public License, either version 1.3
% of this license or (at your option) any later version.
% The latest version of this license is in
%   http://www.latex-project.org/lppl.txt
% and version 1.3 or later is part of all distributions of LaTeX
% version 2005/12/01 or later.
%
% This work has the LPPL maintenance status `maintained'.
%
% The Current Maintainer of this work is Niklas Beisert.
%
% This work consists of the files childdoc.dtx and childdoc.ins
% and the derived files childdoc.def and cdocsamp.tex with
% cdocsch1.tex, cdocsch2.tex, cdocsdrf.tex, cdocsfn1.tex, cdocsfn2.tex.
%
%<package>\ifdefined\childdocmain\endinput\fi
%<package>\ProvidesFile{childdoc.def}[2018/12/30 v2.0 child document driver]
%<samplemain>\ProvidesFile{cdocsamp.tex}[2018/12/30 v2.0 sample for childdoc]
%<*driver>
%\ProvidesFile{childdoc.drv}[2018/12/30 v2.0 childdoc reference manual file]
\PassOptionsToClass{10pt,a4paper}{article}
\documentclass{ltxdoc}

\usepackage[margin=35mm]{geometry}
\usepackage{hyperref}
\usepackage{hyperxmp}
\usepackage[usenames]{color}

\hypersetup{colorlinks=true}
\hypersetup{pdfstartview=FitH}
\hypersetup{pdfpagemode=UseNone}
\hypersetup{pdfsource={}}
\hypersetup{pdflang={en-UK}}
\hypersetup{pdfcopyright={Copyright 2017-2018 Niklas Beisert.
  This work may be distributed and/or modified under the
  conditions of the LaTeX Project Public License, either version 1.3
  of this license or (at your option) any later version.}}
\hypersetup{pdflicenseurl={http://www.latex-project.org/lppl.txt}}
\hypersetup{pdfcontactaddress={ETH Zurich, ITP, HIT K,
  Wolfgang-Pauli-Strasse 27}}
\hypersetup{pdfcontactpostcode={8093}}
\hypersetup{pdfcontactcity={Zurich}}
\hypersetup{pdfcontactcountry={Switzerland}}
\hypersetup{pdfcontactemail={nbeisert@itp.phys.ethz.ch}}
\hypersetup{pdfcontacturl={http://people.phys.ethz.ch/\xmptilde nbeisert/}}

\newcommand{\secref}[1]{\hyperref[#1]{section \ref*{#1}}}

\parskip1ex
\parindent0pt
\let\olditemize\itemize
\def\itemize{\olditemize\parskip0pt}

\begin{document}

\title{The \textsf{childdoc} Package}
\hypersetup{pdftitle={The childdoc Package}}
\author{Niklas Beisert\\[2ex]
  Institut f\"ur Theoretische Physik\\
  Eidgen\"ossische Technische Hochschule Z\"urich\\
  Wolfgang-Pauli-Strasse 27, 8093 Z\"urich, Switzerland\\[1ex]
  \href{mailto:nbeisert@itp.phys.ethz.ch}
  {\texttt{nbeisert@itp.phys.ethz.ch}}}
\hypersetup{pdfauthor={Niklas Beisert}}
\hypersetup{pdfsubject={Manual for the LaTeX2e Package childdoc}}
\date{30 December 2018, \textsf{v2.0}}
\maketitle

\begin{abstract}\noindent
\textsf{childdoc} is a \LaTeXe{} package
that enables the direct compilation
of document sections included by |\include|
to individual files.
\end{abstract}

\begingroup
\parskip0ex
\tableofcontents
\endgroup

%%%%%%%%%%%%%%%%%%%%%%%%%%%%%%%%%%%%%%%%%%%%%%%%%%%%%%%%%%%%%%%%%%%%%%%%%%%%%%%%
%%%%%%%%%%%%%%%%%%%%%%%%%%%%%%%%%%%%%%%%%%%%%%%%%%%%%%%%%%%%%%%%%%%%%%%%%%%%%%%%
\section{Introduction}

\LaTeX{} provides a mechanism to structure a large document (such as a book)
into a main file and several child files (containing the chapters)
using the |\include| command.
This mechanism is beneficial for documents
which span hundreds of pages in order to
make the source file(s) more manageable.
Moreover, compilation can be restricted to
selected child files by means of the |\includeonly| command.
The latter feature can be used to reduce the compilation time while editing
(this was significantly more useful in the earlier days of \LaTeX{})
or to generate a smaller document which is easier to navigate.
Another application of |\includeonly| is to generate
documents consisting of selected parts of the complete document.

However, there are a few drawbacks of the plain |\include| mechanism:
\begin{itemize}
\item
The child files cannot be compiled on their own,
they can only be compiled via the main file.
A naive editing environment
(such as a text editor with an option
to have the current file processed by \LaTeX)
may require one to switch to the main file before compiling;
attempting to compile the child file produces errors.
\item
The main file must be modified (each time)
to adjust the |\includeonly| command
to the present needs. This easily leaves the main file in a messy state.
\item
The generated document will always carry the filename
of the main document. This is inconvenient if
several child files are to be compiled and
to be kept for distribution.
\end{itemize}

The present package provides a simple interface
to make child files individually compilable by \LaTeX{}.
Compiling a child file then has the same effect as compiling
the main file with an |\includeonly| command
to select the appropriate child.
Moreover the generated document will carry the name of the child
rather than the main file.
This resolves all three above issues.

This feature is meant to make the editing of books,
thesis documents and lecture notes somewhat more convenient.
However, the package can also be used efficiently for
composing a series of documents (such as exercise sheets)
which are typically distributed individually.
It then assists the author in generating the individual documents
(potentially in different versions)
as well as a document containing the collected series.
Another application is in developing style files
or other kinds of included material
where compilation of the style file could redirect
to a sample or test file.

%%%%%%%%%%%%%%%%%%%%%%%%%%%%%%%%%%%%%%%%%%%%%%%%%%%%%%%%%%%%%%%%%%%%%%%%%%%%%%%%
%%%%%%%%%%%%%%%%%%%%%%%%%%%%%%%%%%%%%%%%%%%%%%%%%%%%%%%%%%%%%%%%%%%%%%%%%%%%%%%%
\section{Usage}

First of all, the package \textsf{childdoc} is \emph{not} a standard
\LaTeXe{} |.sty| style file! Therefore it needs to be invoked in
a non-standard way.

%%%%%%%%%%%%%%%%%%%%%%%%%%%%%%%%%%%%%%%%%%%%%%%%%%%%%%%%%%%%%%%%%%%%%%%%%%%%%%%%
\subsection{Included Files}
\label{sec:include}

%%%%%%%%%%%%%%%%%%%%%%%%%%%%%%%%%%%%%%%%
\DescribeMacro{\childdocmain}
To use the package, add the commands
\begin{center}
\begin{tabular}{l}
|\input{childdoc.def}|\\
|\childdocmain{}|\\
\end{tabular}
\end{center}
at the very top of the main \LaTeX{} file,
in particular \emph{before} the |\documentclass| statement!
The argument of |\childdocmain| should be left empty
(but it must be present).

%%%%%%%%%%%%%%%%%%%%%%%%%%%%%%%%%%%%%%%%
\DescribeMacro{\childdocof}
Furthermore, add the commands
\begin{center}
\begin{tabular}{l}
|\input{childdoc.def}|\\
|\childdocof{|\textit{main}|}|\\
\end{tabular}
\end{center}
at the top of every child file \textit{child}
which is included by |\include{|\textit{child}|}|
from within the main file
(or at least for those files to be compiled individually).
The argument \textit{main} must be the filename of the main file.

There are a couple of
considerations in setting up the main and child documents:

%%%%%%%%%%%%%%%%%%%%%%%%%%%%%%%%%%%%%%%%
\paragraph{Restrictions.}

Please note the following restrictions:
\begin{itemize}
\item
|\childdocmain| must be called with one argument \textit{main}
to ensure compatibility with earlier version of the package.
It must either be empty (|\childdocmain{}|)
or precisely match the filename of the main file in which it is specified.
See \secref{sec:detection} for further information.
\item
The filename \textit{main} must be specified without the |.tex| extension.
\item
The filename \textit{main} is case sensitive
(even in case-insensitive file systems)
due to internal string comparison.
\item
The argument \textit{main} should be fully expanded, it cannot be a macro.
\item
Subdirectories and special characters should be avoided in filenames.
\item
The command |\childdocmain{|\textit{main}|}| must be followed by a whitespace.
It should not be followed immediately by another command
or by a comment mark `|%|'.
This is because the \TeX{} parser reads the token immediately following
the argument of |\childdocmain| and puts it
at the beginning of every child section;
however, a white\-space is ignored.
\end{itemize}

%%%%%%%%%%%%%%%%%%%%%%%%%%%%%%%%%%%%%%%%
\paragraph{Content of Main File.}

It is advisable to place all content in the child files included by |\include|.
Any output contained in the main file will appear in all child documents
unless suppressed manually;
it cannot be suppressed automatically by the |\includeonly| directive
and thus should normally be avoided.
A method to include some content in the main file
by means of conditional processing is described in \secref{sec:conditional}.

%%%%%%%%%%%%%%%%%%%%%%%%%%%%%%%%%%%%%%%%
\paragraph{Page Numbering.}

When only a part of the document is compiled,
the appropriate numbering of pages
(as well as other status parameters)
is determined from the |.aux| files.
The latter contain information from previous passes.
However this information needs to propagate through
all intermediate child documents.
Therefore the page numbering in child documents may well
be inconsistent until the complete document is compiled at least once.

A useful (if unconventional) way to always ensure a consistent
page numbering is to restart the numbering in each child document
and denote the pages by `\textit{child}|.|\textit{page}'
where \textit{child} represents the chapter/section number of the child file.
This can be achieved by the command
|\numberwithin{page}{|\textit{child}|}|
of the \textsf{amsmath} package
where \textit{child} can be |chapter| or |section|
depending on the chosen structuring.
Alternatively, one can modify the macro |\thepage| appropriately
and reset the counter |page| at the start of each child file.

%%%%%%%%%%%%%%%%%%%%%%%%%%%%%%%%%%%%%%%%%%%%%%%%%%%%%%%%%%%%%%%%%%%%%%%%%%%%%%%%
\subsection{Conditional Processing}
\label{sec:conditional}

The package provides a mechanism to compile different versions
of a document. To customise the versions further some conditional processing
can come in handy to distinguish which version is being compiled.
The package provides two macros to describe the compilation context:

%%%%%%%%%%%%%%%%%%%%%%%%%%%%%%%%%%%%%%%%
\DescribeMacro{\ifchilddoc}
The conditional |\ifchilddoc| distinguishes between the compilation of
child documents and the main document:
%
\begin{center}
|\ifchilddoc |\textit{child-code}| |[|\||else |\textit{main-code}]| \||fi|
\end{center}

%%%%%%%%%%%%%%%%%%%%%%%%%%%%%%%%%%%%%%%%
\DescribeMacro{\childdocname}
\DescribeMacro{\childdocjob}
The macro |\childdocname| contains the filename (without extension)
of the main or child file being processed.
Note that |\childdocjob| will always contain the name of the main file.

%%%%%%%%%%%%%%%%%%%%%%%%%%%%%%%%%%%%%%%%
\paragraph{Title Page.}

Conditional processing can be used to include a title or banner page
in the main document when proper precautions are taken.
Importantly, the code in the main file should ensure that the page counter
(as well as other status parameters which are stored in the |.aux| files)
takes the same value after the conditional processing.
Otherwise the page numbers may take divergent values
depending on which part is compiled.

For example, a title page could be declared by:
%
\begin{center}
\begin{tabular}{l}
|\ifchilddoc\||else|\\
|\addtocounter{page}{-1}|\\
\textit{code for title page}\\
|\newpage|\\
|\||fi|
\end{tabular}
\end{center}
%
A banner page for the child documents can be generated by:
%
\begin{center}
\begin{tabular}{l}
|\ifchilddoc|\\
|\addtocounter{page}{-1}|\\
\textit{code for banner page}\\
|\newpage|\\
|\||fi|
\end{tabular}
\end{center}
%
Here one could write a message such as:
\begin{center}
|This is the part \childdocname{} of \childdocjob{}.|
\end{center}

%%%%%%%%%%%%%%%%%%%%%%%%%%%%%%%%%%%%%%%%%%%%%%%%%%%%%%%%%%%%%%%%%%%%%%%%%%%%%%%%
\subsection{Flags}
\label{sec:flags}

The package makes it easy to generate different versions
of the main or child documents.
To this end compilation flags can be defined
and assigned different default values.
They will be particularly useful in conjunction
with the forwarding mechanism described in \secref{sec:forward}.

For example, it may be useful to have a flag |\version|
which can be set to |draft| or |final|.
The document source will contain some conditional code
depending on the value of |\version|.
Suppose further, the flag should default to |final| for the main file
and to |draft| for child files
which is a natural assignment for editing the document.
This is achieved by placing the following code
in the preamble of the main document
(below the |\childdocmain| directive):
%
\begin{center}
\begin{tabular}{l}
|\ifchilddoc|\\
|\providecommand{\version}{draft}|\\
|\||else|\\
|\providecommand{\version}{final}|\\
|\||fi|
\end{tabular}
\end{center}
%
The definition by |\providecommand| makes sure
that previous definitions are not overwritten.
Further statements |\providecommand{\version}{...}|
can thus be added before the above code to override it.

For the main file, one might add a line
(between |\childdocmain| and the above block)
%
\begin{center}
|%\ifchilddoc\||else\providecommand{\version}{draft}\||fi|
\end{center}
%
which can be uncommented to produce a draft version.
Likewise one can add a line to the very top of a child file
(above the |\childdocof{|\textit{main}|}| directive)
%
\begin{center}
|%\providecommand{\version}{final}|
\end{center}
%
which can be uncommented to produce the final version of this child document.

%%%%%%%%%%%%%%%%%%%%%%%%%%%%%%%%%%%%%%%%%%%%%%%%%%%%%%%%%%%%%%%%%%%%%%%%%%%%%%%%
\subsection{Forwarding}
\label{sec:forward}

Different versions of the main or child documents
using compilation flags as described in \secref{sec:flags}
can be (permanently) stored in different files
for convenient compilation, viewing and distribution.
To this end, the package defines a command
to pass on compilation to a different file:

%%%%%%%%%%%%%%%%%%%%%%%%%%%%%%%%%%%%%%%%
\DescribeMacro{\childdocforward}
The command |\childdocforward| redirects processing to
another source file:
%
\begin{center}
\begin{tabular}{l}
|\input{childdoc.def}|\\
|\childdocforward[|\textit{main}|]{|\textit{dest}|}|\\
\end{tabular}
\end{center}
%
The argument \textit{dest} is the destination file
(without extension).
It should be the main file or one of the child files.
Note that further \textsf{childdoc} directives
such as |\childdocof| and |\childdocforward|
in the indicated file will be processed in this form.
The optional argument \textit{main}
passes on directly to the main file \textit{main}
while pretending to compile the child \textit{dest}.
This form behaves as if \textit{dest}
issues |\childdocof{|\textit{main}|}| right away,
and no further \textsf{childdoc} directives will be processed.

%%%%%%%%%%%%%%%%%%%%%%%%%%%%%%%%%%%%%%%%
\DescribeMacro{\...prefix}
In the alternative form |\childdocforwardprefix|,
%
\begin{center}
\begin{tabular}{l}
|\input{childdoc.def}|\\
|\childdocforwardprefix[|\textit{main}|]{|\textit{prefix}|}{|\textit{dest}|}|
\end{tabular}
\end{center}
%
the destination file is determined by a pattern
depending on the current file:
To make this work, the current file must be called
`{\textit{prefix}\hspace{0.2em}\textit{suffix}}'
with \textit{prefix} matching precisely the argument.
Processing is then passed on to the file
`{\textit{dest}\hspace{0.2em}\textit{suffix}}'.
Surely, the same effect is achieved by
directly specifying the
argument `{\textit{dest}\hspace{0.2em}\textit{suffix}}'
in the first form.
However, that requires to set up a different file
for each child. With the alternative form of the command
all these files can have exactly the same content
which simplifies setting them up and maintaining them.

For example, the following file |draft.tex|
with a compilation flag |\version| as described in \secref{sec:flags}
compiles the main document as a draft:
%
\begin{center}
\begin{tabular}{l}
|\def\version{draft}|\\
|\input{childdoc.def}|\\
|\childdocforward{|\textit{main}|}|
\end{tabular}
\end{center}
%
Likewise, the following files |final|\textit{nn}|.tex|
compile the final version of the child document
|child|\textit{nn}|.tex|:
%
\begin{center}
\begin{tabular}{l}
|\def\version{final}|\\
|\input{childdoc.def}|\\
|\childdocforwardprefix{final}{child}|
\end{tabular}
\end{center}
%

Note that when several versions of a main file and/or of each child file
are to be generated, it may be convenient to set up a |Makefile| or
shell script to automatise the process.

%%%%%%%%%%%%%%%%%%%%%%%%%%%%%%%%%%%%%%%%%%%%%%%%%%%%%%%%%%%%%%%%%%%%%%%%%%%%%%%%
\subsection{Command Line Processing}
\label{sec:commandline}

The effect of redirection files can also be achieved by invoking
the \LaTeX{} compiler with a more elaborate command line.
Most conveniently this should be done as part
of a shell script or a |Makefile|.

When using \textsf{childdoc} in the main file, the following
command lines effectively perform a redirection
(note that depending on the shell being used,
backslashes may have to be doubled: `|\|' $\to$ `|\\|'):
%
\begin{center}
|... -jobname "|\textit{target}|" |\\|"|[\textit{flags}]%
|\input{childdoc.def}\childdocforward[|\textit{main}|]{|\textit{dest}|}"|
\end{center}
%
Here \textit{target} is the name of the output file,
\textit{main} is the name of the main file
and \textit{dest} is the name of the main or child file to be processed
(all filenames without extensions).
The optional argument \textit{main} can be omitted
if \textit{main} matches \textit{dest}.
Optionally, compilation \textit{flags} can be defined via |\def| commands.
This command line makes the \TeX{} engine believe
it is compiling the file \textit{target}
whose content is specified as the latter parameter.
The provided code then forwards the processing to
\textit{main} or \textit{dest} as described in \secref{sec:forward}.

%%%%%%%%%%%%%%%%%%%%%%%%%%%%%%%%%%%%%%%%%%%%%%%%%%%%%%%%%%%%%%%%%%%%%%%%%%%%%%%%
\subsection{Include by Input}
\label{sec:input}

Including child documents by |\include| has some restrictions by design.
Most notably, the content of a child document always occupies
its own set of pages; pages cannot be shared between child documents.
Usually, this behaviour makes perfect sense
because each child document contain an essential part of the document.
However, in some situations it may be desirable to compose
a document from a collection of parts
without having mandatory page breaks between then.
For this case, the package
provides a mechanism to include parts
by |\input| which can also be processed individually.
However, by construction this mechanism
requires manual handling of the content to be output.

%%%%%%%%%%%%%%%%%%%%%%%%%%%%%%%%%%%%%%%%
\DescribeMacro{\ifchilddocmanual}
The main file should be prepared as usual, see \secref{sec:include}.
However, the document body must make a distinction
between processing of an individual part and of the main document, e.g.:
%
\begin{center}
\begin{tabular}{l}
|\ifchilddocmanual|\\
|\input{\childdocname}|\\
|\||else|\\
\textit{document body with }|\input{|\textit{part}|}|\\
|\||fi|
\end{tabular}
\end{center}
%
The conditional |\ifchilddocmanual| is true whenever
a part to be included by |\input| is being compiled,
and the name of the part is stored in |\childdocname|.

%%%%%%%%%%%%%%%%%%%%%%%%%%%%%%%%%%%%%%%%
\DescribeMacro{\childdocby}
Each part to be included by |\input| should start with:
%
\begin{center}
\begin{tabular}{l}
|\input{childdoc.def}|\\
|\childdocby{|\textit{main}|}|\\
\end{tabular}
\end{center}
%
The directive |\childdocby| is similar to |\childdocof|
described in \secref{sec:include},
but the subsequent selection of content must be done manually.
To that end, both |\ifchilddoc| and |\ifchilddocmanual|
will be true upon processing of a part,
and the name of the part is stored in |\childdocname|.
Note that |\jobname| will be set to the filename of the current part
so that each part receives an individual |.aux| file
that does not interfere with the |.aux| file(s) of the main document.
This behaviour can be altered by the alternative form
|\childdocby[*]{|\textit{main}|}| (with a non-empty optional argument)
which uses the |.aux| file of the main document
by setting |\jobname| to \textit{main}.

%%%%%%%%%%%%%%%%%%%%%%%%%%%%%%%%%%%%%%%%%%%%%%%%%%%%%%%%%%%%%%%%%%%%%%%%%%%%%%%%
\subsection{Driver Development}
\label{sec:driver}

The \textsf{childdoc} mechanism can also be use for the development
of definition files such as \LaTeX{} styles or classes.
This case differs from the above setup with multiple parts
included by |\include| in that no |\includeonly| should be invoked.
This can be achieved by starting the include file
(before |\ProvidesPackage|) with:
%
\begin{center}
\begin{tabular}{l}
|\input{childdoc.def}|\\
|\childdocforward{|\textit{main}|}|\\
\end{tabular}
\end{center}
%
or alternatively with:
%
\begin{center}
\begin{tabular}{l}
|\input{childdoc.def}|\\
|\childdocby{|\textit{main}|}|\\
\end{tabular}
\end{center}
%
Both forms have slightly different effects as described above.
The main file is prepared as usual, see \secref{sec:include}.

%%%%%%%%%%%%%%%%%%%%%%%%%%%%%%%%%%%%%%%%%%%%%%%%%%%%%%%%%%%%%%%%%%%%%%%%%%%%%%%%
\subsection{Legacy Detection}
\label{sec:detection}

The directive |\childdocmain| in the main file can detect
whether the complete document or merely a child is to be compiled
even without using the directive |\childdocof|.
This method is deprecated because it is less robust
and there is no compelling reason to use it;
it is merely provided for backward compatibility
and it may be removed in future versions.

If the detection mechanism is to be used,
it is mandatory to correctly specify
the filename of the main file as the argument of |\childdocmain|:
%
\begin{center}
\begin{tabular}{l}
|\input{childdoc.def}|\\
|\childdocmain{|\textit{main}|}|\\
\end{tabular}
\end{center}
%
If |\jobname| does not match the argument \textit{main} of |\childdocmain|,
it is assumed that |\jobname| points to the child file to be compiled.
When using |\childdocmain| with the main file specified as argument,
it suffices to start a child file
with just |\input{|\textit{main}|}|
without loading of the package and using |\childdocof|.
If instead all processing is done
with the appropriate \textsf{childdoc} directives,
the argument of \textit{main} of |\childdocmain| can be empty.

An alternative version of the command line processing described
in \secref{sec:commandline} using the detection mechanism reads:
%
\begin{center}
|... -jobname "|\textit{target}|" "|[\textit{flags}]%
[|\def\jobname{|\textit{dest}|}|]|\input{|\textit{main}|}"|
\end{center}

%%%%%%%%%%%%%%%%%%%%%%%%%%%%%%%%%%%%%%%%%%%%%%%%%%%%%%%%%%%%%%%%%%%%%%%%%%%%%%%%
\subsection{Manual Code}
\label{sec:manual}

In case one cannot be certain whether the definitions file |childdoc.def|
is installed on the target \TeX{} distribution
and one prefers not to ship it,
it is conceivable to paste a few relevant commands into the sources.

To that end, drop all statements |\input{childdoc.def}|
and perform the replacements as outlined below.
Instead of |\childdocmain{|\textit{main}|}| add the following code
to the top of the main file:
%
\begin{center}
\begin{tabular}{l}
|\||ifdefined\childdocname\endinput\||fi\newif\ifchilddoc|\\
|\edef\childdocname{\scantokens\expandafter{\jobname\noexpand}}|\\
|\def\childdocmain{|\textit{main}|}\||ifx\childdocmain\childdocname\||else|\\
|\childdoctrue\includeonly{\childdocname}\let\jobname\childdocmain\||fi|\\
\end{tabular}
\end{center}
%
Instead of |\childdocof{|\textit{main}|}| just include the main file
at the top of each child file:
%
\begin{center}
|\input{|\textit{main}|}|
\end{center}
%
A simple redirection |\childdocforward{|\textit{dest}|}| is achieved by:
%
\begin{center}
|\def\jobname{|\textit{dest}|}\input{\jobname}|
\end{center}
%
The redirection with prefix
|\childdocforwardprefix[|\textit{prefix}|]{|\textit{dest}|}|
is accomplished by:
%
\begin{center}
\begin{tabular}{l}
|{\edef\jobname{\scantokens\expandafter{\jobname\noexpand}}|\\
|\def\redirectjob |\textit{prefix}|#1~~~{\gdef\jobname{|\textit{dest}|#1}}|\\
|\expandafter\redirectjob\jobname~~~}\input{\jobname}|
\end{tabular}
\end{center}

In an alternative approach,
child documents can be compiled by a specific command line
without additional code or specific definitions:
%
\begin{center}
|... -jobname "|\textit{target}|" "|[\textit{flags}]%
|\includeonly{|\textit{dest}|}\input{|\textit{main}|}"|
\end{center}
%

%%%%%%%%%%%%%%%%%%%%%%%%%%%%%%%%%%%%%%%%%%%%%%%%%%%%%%%%%%%%%%%%%%%%%%%%%%%%%%%%
%%%%%%%%%%%%%%%%%%%%%%%%%%%%%%%%%%%%%%%%%%%%%%%%%%%%%%%%%%%%%%%%%%%%%%%%%%%%%%%%
\section{Information}

%%%%%%%%%%%%%%%%%%%%%%%%%%%%%%%%%%%%%%%%%%%%%%%%%%%%%%%%%%%%%%%%%%%%%%%%%%%%%%%%
\subsection{Copyright}

Copyright \copyright{} 2017--2018 Niklas Beisert

This work may be distributed and/or modified under the
conditions of the \LaTeX{} Project Public License, either version 1.3
of this license or (at your option) any later version.
The latest version of this license is in
  \url{http://www.latex-project.org/lppl.txt}
and version 1.3 or later is part of all distributions of \LaTeX{}
version 2005/12/01 or later.

This work has the LPPL maintenance status `maintained'.

The Current Maintainer of this work is Niklas Beisert.

This work consists of the files |README.txt|, |childdoc.ins| and |childdoc.dtx|
as well as the derived files |childdoc.def|, |cdocsamp.tex|
with |cdocsch1.tex|, |cdocsch2.tex|, |cdocspt3.tex|, |cdocspt4.tex|,
|cdocsdrf.tex|, |cdocsfn1.tex|, |cdocsfn2.tex|
as well as |childdoc.pdf|.

%%%%%%%%%%%%%%%%%%%%%%%%%%%%%%%%%%%%%%%%%%%%%%%%%%%%%%%%%%%%%%%%%%%%%%%%%%%%%%%%
\subsection{Files and Installation}

The package consists of the files:
%
\begin{center}
\begin{tabular}{ll}
    |README.txt|   & readme file \\
    |childdoc.ins| & installation file \\
    |childdoc.dtx| & source file \\
    |childdoc.def| & definition file \\
    |cdocsamp.tex| & sample main file \\
    |cdocsch1.tex| & sample include file \\
    |cdocsch2.tex| & sample include file \\
    |cdocspt3.tex| & sample part file \\
    |cdocspt4.tex| & sample part file \\
    |cdocsdrf.tex| & sample redirection file \\
    |cdocsfn1.tex| & sample redirection file \\
    |cdocsfn2.tex| & sample redirection file \\
    |childdoc.pdf| & manual
\end{tabular}
\end{center}
%
The distribution consists of the files
|README.txt|, |childdoc.ins| and |childdoc.dtx|.
%
\begin{itemize}
\item
Run (pdf)\LaTeX{} on |childdoc.dtx|
to compile the manual |childdoc.pdf| (this file).
\item
Run \LaTeX{} on |childdoc.ins| to create the definitions file |childdoc.def|
and the sample |cdocsamp.tex| with include files
|cdocsch1.tex|, |cdocsch2.tex|, |cdocspt3.tex|, |cdocspt4.tex|,
|cdocsdrf.tex|, |cdocsfn1.tex|, |cdocsfn2.tex|.
Then copy the file |childdoc.def| to an appropriate directory of your \LaTeX{}
distribution, e.g.\ \textit{texmf-root}|/tex/latex/childdoc|.
\end{itemize}

%%%%%%%%%%%%%%%%%%%%%%%%%%%%%%%%%%%%%%%%%%%%%%%%%%%%%%%%%%%%%%%%%%%%%%%%%%%%%%%%
\subsection{Related CTAN Packages}

There are several other packages which offer a similar functionality:
%
\begin{itemize}
\item
The packages
\href{http://ctan.org/pkg/docmute}{\textsf{docmute}},
\href{http://ctan.org/pkg/includex}{\textsf{includex}} and
\href{http://ctan.org/pkg/standalone}{\textsf{standalone}}
provide commands to include only the document body of
a child file thus allowing both files to be compiled individually.
\item
The packages \href{http://ctan.org/pkg/subdocs}{\textsf{subdocs}}
and \href{http://ctan.org/pkg/subfiles}{\textsf{subfiles}}
provide structures in which the main and child documents can be
encapsulated and allowing them to be compiled individually.
The inclusion mechanism is different from the conventional |\include|.
\item
The package \href{http://ctan.org/pkg/combine}{\textsf{combine}}
is an elaborate solution to combine several documents into one.
\end{itemize}
%
See also the CTAN topic \href{http://ctan.org/topic/subdocs}{\textsf{subdocs}}
for further related packages.
The present package differs from the above solutions in that
a document structure constructed with the conventional |\include| mechanism
just needs two extra commands at the top of every file
such that all constituent files can be compiled individually.

%%%%%%%%%%%%%%%%%%%%%%%%%%%%%%%%%%%%%%%%%%%%%%%%%%%%%%%%%%%%%%%%%%%%%%%%%%%%%%%%
%\subsection{Feature Suggestions}
%
%The following is a list of features which may be useful for future
%versions of this package:
%%
%\begin{itemize}
%\item
%\ldots
%\end{itemize}

%%%%%%%%%%%%%%%%%%%%%%%%%%%%%%%%%%%%%%%%%%%%%%%%%%%%%%%%%%%%%%%%%%%%%%%%%%%%%%%%
\subsection{Revision History}

%%%%%%%%%%%%%%%%%%%%%%%%%%%%%%%%%%%%%%%%
\paragraph{v2.0:} 2018/12/30

\begin{itemize}
\item
immediate forward processing
\item
added |\childdocby| mechanism
\item
manual restructured
\end{itemize}

%%%%%%%%%%%%%%%%%%%%%%%%%%%%%%%%%%%%%%%%
\paragraph{v1.6:} 2018/01/17

\begin{itemize}
\item
application for development of include files
\item
corrections to manual
\end{itemize}

%%%%%%%%%%%%%%%%%%%%%%%%%%%%%%%%%%%%%%%%
\paragraph{v1.5:} 2017/05/21

\begin{itemize}
\item
more complete structuring introduced
\item
|\childdocof| introduced
\item
|\childdoc| renamed to |\childdocmain|
\item
|\childredirect| renamed to |\childdocforward| and |\childdocforwardprefix|
and functionality expanded
\end{itemize}

%%%%%%%%%%%%%%%%%%%%%%%%%%%%%%%%%%%%%%%%
\paragraph{v1.0:} 2017/04/27

\begin{itemize}
\item
manual and install package
\item
first version published on CTAN
\end{itemize}

%%%%%%%%%%%%%%%%%%%%%%%%%%%%%%%%%%%%%%%%
\paragraph{v0.6:} 2017/04/26

\begin{itemize}
\item
redirection mechanism added
\end{itemize}

%%%%%%%%%%%%%%%%%%%%%%%%%%%%%%%%%%%%%%%%
\paragraph{v0.5:} 2017/04/26

\begin{itemize}
\item
functionality in definition file
\end{itemize}


%%%%%%%%%%%%%%%%%%%%%%%%%%%%%%%%%%%%%%%%%%%%%%%%%%%%%%%%%%%%%%%%%%%%%%%%%%%%%%%%
%%%%%%%%%%%%%%%%%%%%%%%%%%%%%%%%%%%%%%%%%%%%%%%%%%%%%%%%%%%%%%%%%%%%%%%%%%%%%%%%
%%%%%%%%%%%%%%%%%%%%%%%%%%%%%%%%%%%%%%%%%%%%%%%%%%%%%%%%%%%%%%%%%%%%%%%%%%%%%%%%
\appendix

\settowidth\MacroIndent{\rmfamily\scriptsize 000\ }

 \DocInput{childdoc.dtx}

\end{document}
%</driver>
% \fi
%
% %%%%%%%%%%%%%%%%%%%%%%%%%%%%%%%%%%%%%%%%%%%%%%%%%%%%%%%%%%%%%%%%%%%%%%%%%%%%%%
% %%%%%%%%%%%%%%%%%%%%%%%%%%%%%%%%%%%%%%%%%%%%%%%%%%%%%%%%%%%%%%%%%%%%%%%%%%%%%%
% \section{Sample}
%\iffalse
%<*samplemain>
%\fi
%
% The following presents a sample document
% with two chapters, two parts, a title page,
% a compile flag as well as three forwarding files to set the flag.
% It consists of eight |.tex| files:
% \begin{center}
% \begin{tabular}{ll}
% |cdocsamp.tex|&main file\\
% |cdocsch1.tex|&include file for chapter 1\\
% |cdocsch2.tex|&include file for chapter 2\\
% |cdocspt3.tex|&include file for part 3\\
% |cdocspt4.tex|&include file for part 4\\
% |cdocsdrf.tex|&forwarding file for main file in draft mode\\
% |cdocsfi1.tex|&forwarding file for final version of chapter 1\\
% |cdocsfi2.tex|&forwarding file for final version of chapter 2\\
% \end{tabular}
% \end{center}
% Each of the eight files can be compiled directly by the \LaTeX{} compiler.
%
% %%%%%%%%%%%%%%%%%%%%%%%%%%%%%%%%%%%%%%
% \paragraph{Main File.}
%
% The main file is called |cdocsamp.tex|.
%
% Load the \textsf{childdoc} definitions and
% declare the filename for the main document:
%    \begin{macrocode}
\input{childdoc.def}
\childdocmain{}
%    \end{macrocode}

% Optional override for |\version| flag:
%    \begin{macrocode}
%%\ifchilddoc\else\providecommand{\version}{draft}\fi
%    \end{macrocode}

% Define the default values for the |\version| flag
% (|final| for the main file and |draft| for childs):
%    \begin{macrocode}
\ifchilddoc
\providecommand{\version}{draft}
\else
\providecommand{\version}{final}
\fi
%    \end{macrocode}

% Load the standard document class:
%    \begin{macrocode}
\documentclass[12pt]{article}
%    \end{macrocode}

% Start the document body:
%    \begin{macrocode}
\begin{document}
%    \end{macrocode}

% Declare a title page.
% Print title, part of document being processed and version flag:
%    \begin{macrocode}
\addtocounter{page}{-1}
\begin{center}
{\LARGE\bfseries{}childdoc example\par}
\vspace{1cm}
\ifchilddoc
\ifchilddocmanual part\else chapter\fi:
`\childdocname' of `\childdocjob'\par
\else
main document: `\childdocjob'\par
\fi
version: \version\par
\end{center}
\newpage
%    \end{macrocode}

% Manually include selected file,
% otherwise process as usual:
%    \begin{macrocode}
\ifchilddocmanual
\section*{part `\childdocname'}
\input{\childdocname}
\else
%    \end{macrocode}

% Include the two chapters:
%    \begin{macrocode}
\include{cdocsch1}
\include{cdocsch2}
%    \end{macrocode}

% Include the two parts unless only chapters should be displayed:
%    \begin{macrocode}
\ifchilddoc\else
\section{part three}
\input{cdocspt3}
\section{part four}
\input{cdocspt4}
\fi
%    \end{macrocode}

% Process as usual until here:
%    \begin{macrocode}
\fi
%    \end{macrocode}

% End of document body:
%    \begin{macrocode}
\end{document}
%    \end{macrocode}
%\iffalse
%</samplemain>
%\fi
%
% %%%%%%%%%%%%%%%%%%%%%%%%%%%%%%%%%%%%%%
% \paragraph{Chapter Include Files.}
%
% The include files are called |cdocsch1.tex| and |cdocsch2.tex|.
%
%\iffalse
%<*samplechap1|samplechap2>
%\fi

% Optional override for |\version| flag:
%    \begin{macrocode}
%%\providecommand{\version}{final}
%    \end{macrocode}

% Include the main document:
%    \begin{macrocode}
\input{childdoc.def}
\childdocof{cdocsamp}
%    \end{macrocode}

%\iffalse
%</samplechap1|samplechap2>
%\fi
%
%\iffalse
%<*samplechap1>
%\fi
% Some text for chapter 1:
%    \begin{macrocode}
\section{one}
some text in chapter one
%    \end{macrocode}

%\iffalse
%</samplechap1>
%\fi
% Some text for chapter 2:
%\iffalse
%<*samplechap2>
%\fi
%    \begin{macrocode}
\section{two}
more text in chapter two
%    \end{macrocode}

%\iffalse
%</samplechap2>
%\fi
%
% %%%%%%%%%%%%%%%%%%%%%%%%%%%%%%%%%%%%%%
% \paragraph{Part Include Files.}
%
% The include files are called |cdocspt3.tex| and |cdocspt4.tex|.
%
%\iffalse
%<*samplepart3|samplepart4>
%\fi

% Optional override for |\version| flag:
%    \begin{macrocode}
%%\providecommand{\version}{final}
%    \end{macrocode}

% Include the main document:
%    \begin{macrocode}
\input{childdoc.def}
\childdocby{cdocsamp}
%    \end{macrocode}

%\iffalse
%</samplepart3|samplepart4>
%\fi
%
%\iffalse
%<*samplepart3>
%\fi
% Some text for part 3:
%    \begin{macrocode}
some text in part three
%    \end{macrocode}

%\iffalse
%</samplepart3>
%\fi
% Some text for part 4:
%\iffalse
%<*samplepart4>
%\fi
%    \begin{macrocode}
more text in part four
%    \end{macrocode}

%\iffalse
%</samplepart4>
%\fi
%
% %%%%%%%%%%%%%%%%%%%%%%%%%%%%%%%%%%%%%%
% \paragraph{Forwarding for a Complete Draft.}
%
% The following forwarding file |cdocsdrf.tex|
% compiles the main document in draft mode:
%\iffalse
%<*sampledraft>
%\fi
%    \begin{macrocode}
\def\version{draft}
\input{childdoc.def}
\childdocforward{cdocsamp}
%    \end{macrocode}

%\iffalse
%</sampledraft>
%\fi
%
% %%%%%%%%%%%%%%%%%%%%%%%%%%%%%%%%%%%%%%
% \paragraph{Forwarding for Final Version of the Chapters.}
%
% The following forwarding files |cdocsfn1.tex| and |cdocsfn2.tex|
% (with identical content)
% compile the final versions of the child documents
% |cdocsch1.tex| and |cdocsch2.tex|, respectively:
%\iffalse
%<*samplefinal>
%\fi
%    \begin{macrocode}
\def\version{final}
\input{childdoc.def}
\childdocforwardprefix[cdocsamp]{cdocsfn}{cdocsch}
%    \end{macrocode}

%\iffalse
%</samplefinal>
%\fi
%
% %%%%%%%%%%%%%%%%%%%%%%%%%%%%%%%%%%%%%%
% \paragraph{Command Line Processing.}
%
% The following three command lines generate the output files
% |cdocscld|, |cdocscl1| and |cdocscl2|
% which should be identical to
% |cdocsdrf|, |cdocsch1| and |cdocsfn2|, respectively:
% \begin{center}
% \begin{tabular}{l}
% |latex -jobname cdocscld \|\\
% |  "\def\version{draft}\input{childdoc.def}\childdocforward{cdocsamp}"|\\
% |latex -jobname cdocscl1 \|\\
% |  "\input{childdoc.def}\childdocforward[cdocsamp]{cdocsch1}"|\\
% |latex -jobname cdocscl2 \|\\
% |  "\def\version{final}\input{childdoc.def}\childdocforward{cdocsch2}"|
% \end{tabular}
% \end{center}
% Note that the trailing backslash on each first line
% merely continues the input to the second line
% (for convenient cut ant paste).
% Furthermore, the command |latex| can be replaced by any
% of its alternative versions such as |pdflatex|.
%
% %%%%%%%%%%%%%%%%%%%%%%%%%%%%%%%%%%%%%%%%%%%%%%%%%%%%%%%%%%%%%%%%%%%%%%%%%%%%%%
% %%%%%%%%%%%%%%%%%%%%%%%%%%%%%%%%%%%%%%%%%%%%%%%%%%%%%%%%%%%%%%%%%%%%%%%%%%%%%%
% \section{Implementation}
%\iffalse
%<*package>
%\fi
%
% This section describes the definitions file |childdoc.def|.

% The definitions cannot be loaded using |\usepackage| or |\RequirePackage|
% which has a mechanism to prevent loading a style file more than once.
% When loading the definitions by means of |\input|
% multiple instances have to be prevented manually:
%\iffalse
%This code needs to be before the `\ProvidesFile' directive
%which is defined at the beginning of this file.
%Therefore it is also placed there and commented out here.
%</package>
%<*discard>
%\fi
%    \begin{macrocode}
\ifdefined\childdocmain\endinput\fi
%    \end{macrocode}
%\iffalse
%</discard>
%<*package>
%\fi
%
% \macro{\ifchilddoc}
% \macro{\ifchilddocmanual}
% The conditional |\ifchilddoc| tells whether a
% child (true) or main (false) document is being compiled.
% The conditional |\ifchilddocmanual| tells whether
% the |\includeonly| mechanism is used (false) or
% the selection of child files must be performed manually (true).
% The definitions initialise to false:
%    \begin{macrocode}
\newif\ifchilddoc
\newif\ifchilddocmanual
%    \end{macrocode}

% \macro{\childdocname}
% \macro{\childdocjob}
% The macro |\childdocname| stores the name of the main document
% to be compiled. The macro |\childdocjob| stores the name of
% the document on which the \LaTeX{} compiler was originally invoked.
% The content of |\jobname| cannot be compared
% to filenames specified in the source due to different catcodes.
% The following code rescans |\jobname|, stores the result
% in |\childdocname| and saves a copy in |\childdocjob|:
%    \begin{macrocode}
\edef\childdocname{\scantokens\expandafter{\jobname\noexpand}}
\let\childdocjob\childdocname
%    \end{macrocode}

% \macro{\childdocdisable}
% The macro |\childdocdisable| prevents the main file
% from being processed more than once.
% At this stage, the main document command |\childdocmain|
% is assumed to be called once again where it should do nothing.
% Any subsequent call to it should prevent
% a secondary processing of the main document
% It overwrites the forwarding commands
% |\childdocof| and |\childdocforward|
% with empty macros to prevent further inclusions of the main document:
%    \begin{macrocode}
\newcommand{\childdocdisable}
{
  \renewcommand{\childdocmain}[1]{\renewcommand{\childdocmain}[1]{\endinput}}
  \renewcommand{\childdocof}[1]{}
  \renewcommand{\childdocby}[2][]{}
  \renewcommand{\childdocforward}[2][]{}
  \renewcommand{\childdocdisable}{}
}
%    \end{macrocode}

% \macro{\childdocmain}
% The macro |\childdocmain| is to be called at the top of the main file
% with nothing or the main filename (without extension) as argument.
% First, it breaks loops.
% If the argument is not empty and does not match |\childdocname|
% (which is set by the first inclusion of |childdoc.def|),
% |\ifchilddoc| is set to true, |\includeonly| is applied to the child file
% and |\jobname| is set to the main file
% (for proper handling of |.aux| files):
%    \begin{macrocode}
\newcommand{\childdocmain}[1]
{
  \childdocdisable\childdocmain{}
  \if?#1?\else
    \begingroup
      \def\childdoctmp{#1}
      \ifx\childdoctmp\childdocname
        \def\childdoctmp{}
      \else
        \def\childdoctmp
        {
          \childdoctrue
          \includeonly{\childdocname}
          \def\childdocjob{#1}
          \def\jobname{#1}
        }
      \fi
      \expandafter
    \endgroup
    \childdoctmp
  \fi
}
%    \end{macrocode}

% \macro{\childdocof}
% The command |\childdocof| redirects
% compilation to the main file |#1|.
%    \begin{macrocode}
\newcommand{\childdocof}[1]
{
  \childdocdisable
  \childdoctrue
  \includeonly{\childdocname}
  \def\jobname{#1}
  \def\childdocjob{#1}
  \input{#1}
}
%    \end{macrocode}

% \macro{\childdocby}
% The command |\childdocby| ....
%    \begin{macrocode}
\newcommand{\childdocby}[2][]
{
  \childdocdisable
  \childdoctrue
  \childdocmanualtrue
  \if?#1?\else
    \def\jobname{#2}
  \fi
  \def\childdocjob{#2}
  \input{#2}
  \endinput
}
%    \end{macrocode}

% \macro{\childdocforward}
% The command |\childdocforward| redirects
% compilation to the main file or
% (if the optional argument is given) a child file.
% Parameters are set as if the main file
% or a child file starting with |\childdocof| was compiled.
% Then compilation is handed over to the main file:
%    \begin{macrocode}
\newcommand{\childdocforward}[2][]
{
  \begingroup
    \if?#1?
      \def\childdoctmp
      {
        \def\childdocname{#2}
        \def\childdocjob{#2}
        \def\jobname{#2}
        \input{#2}
        \endinput
      }
    \else
      \def\childdoctmp
      {
        \childdocdisable
        \def\childdocname{#2}
        \childdoctrue
        \includeonly{#2}
        \def\childdocjob{#1}
        \def\jobname{#1}
        \input{#1}
        \endinput
      }
    \fi
    \expandafter
  \endgroup
  \childdoctmp
}
%    \end{macrocode}

% \macro{\childdocforwardprefix}
% The command |\childdocforwardprefix| redirects
% compilation to the main or a child file by means of a pattern.
% The prefix |#1| in the current filename is replaced by |#2|
% and the suffix of the current filename is kept
% (it is assumed that the filename does not contain the substring `|~~~|'
% which is used as a delimiter).
% Compilation is handed over to the new file by |\childdocforward|:
%    \begin{macrocode}
\newcommand{\childdocforwardprefix}[3][]
{
  \begingroup
    \def\childdocextract #2##1~~~{\def\childdoctmp{\childdocforward[#1]{#3##1}}}
    \expandafter\childdocextract\childdocname~~~
    \expandafter
  \endgroup
  \childdoctmp
}
%    \end{macrocode}

% \macro{\childdoc}
% The deprecated macro |\childdoc| is a legacy version of |\childdocmain|:
%    \begin{macrocode}
\newcommand{\childdoc}{\childdocmain}
%    \end{macrocode}

% \macro{\childdocredirect}
% The deprecated macro |\childdocredirect| is a legacy version
% of |\childdocforward| and |\childdocforwardprefix|:
%    \begin{macrocode}
\newcommand{\childdocredirect}[2][]
{
  \begingroup
    \if?#1?
      \def\childdoctmp{\childdocforward{#2}}
    \else
      \def\childdoctmp{\childdocforwardprefix{#1}{#2}}
    \fi
    \expandafter
  \endgroup
  \childdoctmp
}
%    \end{macrocode}

%\iffalse
%</package>
%\fi
%
\endinput
|\\
|\childdocby{|\textit{main}|}|\\
\end{tabular}
\end{center}
%
The directive |\childdocby| is similar to |\childdocof|
described in \secref{sec:include},
but the subsequent selection of content must be done manually.
To that end, both |\ifchilddoc| and |\ifchilddocmanual|
will be true upon processing of a part,
and the name of the part is stored in |\childdocname|.
Note that |\jobname| will be set to the filename of the current part
so that each part receives an individual |.aux| file
that does not interfere with the |.aux| file(s) of the main document.
This behaviour can be altered by the alternative form
|\childdocby[*]{|\textit{main}|}| (with a non-empty optional argument)
which uses the |.aux| file of the main document
by setting |\jobname| to \textit{main}.

%%%%%%%%%%%%%%%%%%%%%%%%%%%%%%%%%%%%%%%%%%%%%%%%%%%%%%%%%%%%%%%%%%%%%%%%%%%%%%%%
\subsection{Driver Development}
\label{sec:driver}

The \textsf{childdoc} mechanism can also be use for the development
of definition files such as \LaTeX{} styles or classes.
This case differs from the above setup with multiple parts
included by |\include| in that no |\includeonly| should be invoked.
This can be achieved by starting the include file
(before |\ProvidesPackage|) with:
%
\begin{center}
\begin{tabular}{l}
|% \iffalse
%
% childdoc.dtx Copyright (C) 2017-2018 Niklas Beisert
%
% This work may be distributed and/or modified under the
% conditions of the LaTeX Project Public License, either version 1.3
% of this license or (at your option) any later version.
% The latest version of this license is in
%   http://www.latex-project.org/lppl.txt
% and version 1.3 or later is part of all distributions of LaTeX
% version 2005/12/01 or later.
%
% This work has the LPPL maintenance status `maintained'.
%
% The Current Maintainer of this work is Niklas Beisert.
%
% This work consists of the files childdoc.dtx and childdoc.ins
% and the derived files childdoc.def and cdocsamp.tex with
% cdocsch1.tex, cdocsch2.tex, cdocsdrf.tex, cdocsfn1.tex, cdocsfn2.tex.
%
%<package>\ifdefined\childdocmain\endinput\fi
%<package>\ProvidesFile{childdoc.def}[2018/12/30 v2.0 child document driver]
%<samplemain>\ProvidesFile{cdocsamp.tex}[2018/12/30 v2.0 sample for childdoc]
%<*driver>
%\ProvidesFile{childdoc.drv}[2018/12/30 v2.0 childdoc reference manual file]
\PassOptionsToClass{10pt,a4paper}{article}
\documentclass{ltxdoc}

\usepackage[margin=35mm]{geometry}
\usepackage{hyperref}
\usepackage{hyperxmp}
\usepackage[usenames]{color}

\hypersetup{colorlinks=true}
\hypersetup{pdfstartview=FitH}
\hypersetup{pdfpagemode=UseNone}
\hypersetup{pdfsource={}}
\hypersetup{pdflang={en-UK}}
\hypersetup{pdfcopyright={Copyright 2017-2018 Niklas Beisert.
  This work may be distributed and/or modified under the
  conditions of the LaTeX Project Public License, either version 1.3
  of this license or (at your option) any later version.}}
\hypersetup{pdflicenseurl={http://www.latex-project.org/lppl.txt}}
\hypersetup{pdfcontactaddress={ETH Zurich, ITP, HIT K,
  Wolfgang-Pauli-Strasse 27}}
\hypersetup{pdfcontactpostcode={8093}}
\hypersetup{pdfcontactcity={Zurich}}
\hypersetup{pdfcontactcountry={Switzerland}}
\hypersetup{pdfcontactemail={nbeisert@itp.phys.ethz.ch}}
\hypersetup{pdfcontacturl={http://people.phys.ethz.ch/\xmptilde nbeisert/}}

\newcommand{\secref}[1]{\hyperref[#1]{section \ref*{#1}}}

\parskip1ex
\parindent0pt
\let\olditemize\itemize
\def\itemize{\olditemize\parskip0pt}

\begin{document}

\title{The \textsf{childdoc} Package}
\hypersetup{pdftitle={The childdoc Package}}
\author{Niklas Beisert\\[2ex]
  Institut f\"ur Theoretische Physik\\
  Eidgen\"ossische Technische Hochschule Z\"urich\\
  Wolfgang-Pauli-Strasse 27, 8093 Z\"urich, Switzerland\\[1ex]
  \href{mailto:nbeisert@itp.phys.ethz.ch}
  {\texttt{nbeisert@itp.phys.ethz.ch}}}
\hypersetup{pdfauthor={Niklas Beisert}}
\hypersetup{pdfsubject={Manual for the LaTeX2e Package childdoc}}
\date{30 December 2018, \textsf{v2.0}}
\maketitle

\begin{abstract}\noindent
\textsf{childdoc} is a \LaTeXe{} package
that enables the direct compilation
of document sections included by |\include|
to individual files.
\end{abstract}

\begingroup
\parskip0ex
\tableofcontents
\endgroup

%%%%%%%%%%%%%%%%%%%%%%%%%%%%%%%%%%%%%%%%%%%%%%%%%%%%%%%%%%%%%%%%%%%%%%%%%%%%%%%%
%%%%%%%%%%%%%%%%%%%%%%%%%%%%%%%%%%%%%%%%%%%%%%%%%%%%%%%%%%%%%%%%%%%%%%%%%%%%%%%%
\section{Introduction}

\LaTeX{} provides a mechanism to structure a large document (such as a book)
into a main file and several child files (containing the chapters)
using the |\include| command.
This mechanism is beneficial for documents
which span hundreds of pages in order to
make the source file(s) more manageable.
Moreover, compilation can be restricted to
selected child files by means of the |\includeonly| command.
The latter feature can be used to reduce the compilation time while editing
(this was significantly more useful in the earlier days of \LaTeX{})
or to generate a smaller document which is easier to navigate.
Another application of |\includeonly| is to generate
documents consisting of selected parts of the complete document.

However, there are a few drawbacks of the plain |\include| mechanism:
\begin{itemize}
\item
The child files cannot be compiled on their own,
they can only be compiled via the main file.
A naive editing environment
(such as a text editor with an option
to have the current file processed by \LaTeX)
may require one to switch to the main file before compiling;
attempting to compile the child file produces errors.
\item
The main file must be modified (each time)
to adjust the |\includeonly| command
to the present needs. This easily leaves the main file in a messy state.
\item
The generated document will always carry the filename
of the main document. This is inconvenient if
several child files are to be compiled and
to be kept for distribution.
\end{itemize}

The present package provides a simple interface
to make child files individually compilable by \LaTeX{}.
Compiling a child file then has the same effect as compiling
the main file with an |\includeonly| command
to select the appropriate child.
Moreover the generated document will carry the name of the child
rather than the main file.
This resolves all three above issues.

This feature is meant to make the editing of books,
thesis documents and lecture notes somewhat more convenient.
However, the package can also be used efficiently for
composing a series of documents (such as exercise sheets)
which are typically distributed individually.
It then assists the author in generating the individual documents
(potentially in different versions)
as well as a document containing the collected series.
Another application is in developing style files
or other kinds of included material
where compilation of the style file could redirect
to a sample or test file.

%%%%%%%%%%%%%%%%%%%%%%%%%%%%%%%%%%%%%%%%%%%%%%%%%%%%%%%%%%%%%%%%%%%%%%%%%%%%%%%%
%%%%%%%%%%%%%%%%%%%%%%%%%%%%%%%%%%%%%%%%%%%%%%%%%%%%%%%%%%%%%%%%%%%%%%%%%%%%%%%%
\section{Usage}

First of all, the package \textsf{childdoc} is \emph{not} a standard
\LaTeXe{} |.sty| style file! Therefore it needs to be invoked in
a non-standard way.

%%%%%%%%%%%%%%%%%%%%%%%%%%%%%%%%%%%%%%%%%%%%%%%%%%%%%%%%%%%%%%%%%%%%%%%%%%%%%%%%
\subsection{Included Files}
\label{sec:include}

%%%%%%%%%%%%%%%%%%%%%%%%%%%%%%%%%%%%%%%%
\DescribeMacro{\childdocmain}
To use the package, add the commands
\begin{center}
\begin{tabular}{l}
|\input{childdoc.def}|\\
|\childdocmain{}|\\
\end{tabular}
\end{center}
at the very top of the main \LaTeX{} file,
in particular \emph{before} the |\documentclass| statement!
The argument of |\childdocmain| should be left empty
(but it must be present).

%%%%%%%%%%%%%%%%%%%%%%%%%%%%%%%%%%%%%%%%
\DescribeMacro{\childdocof}
Furthermore, add the commands
\begin{center}
\begin{tabular}{l}
|\input{childdoc.def}|\\
|\childdocof{|\textit{main}|}|\\
\end{tabular}
\end{center}
at the top of every child file \textit{child}
which is included by |\include{|\textit{child}|}|
from within the main file
(or at least for those files to be compiled individually).
The argument \textit{main} must be the filename of the main file.

There are a couple of
considerations in setting up the main and child documents:

%%%%%%%%%%%%%%%%%%%%%%%%%%%%%%%%%%%%%%%%
\paragraph{Restrictions.}

Please note the following restrictions:
\begin{itemize}
\item
|\childdocmain| must be called with one argument \textit{main}
to ensure compatibility with earlier version of the package.
It must either be empty (|\childdocmain{}|)
or precisely match the filename of the main file in which it is specified.
See \secref{sec:detection} for further information.
\item
The filename \textit{main} must be specified without the |.tex| extension.
\item
The filename \textit{main} is case sensitive
(even in case-insensitive file systems)
due to internal string comparison.
\item
The argument \textit{main} should be fully expanded, it cannot be a macro.
\item
Subdirectories and special characters should be avoided in filenames.
\item
The command |\childdocmain{|\textit{main}|}| must be followed by a whitespace.
It should not be followed immediately by another command
or by a comment mark `|%|'.
This is because the \TeX{} parser reads the token immediately following
the argument of |\childdocmain| and puts it
at the beginning of every child section;
however, a white\-space is ignored.
\end{itemize}

%%%%%%%%%%%%%%%%%%%%%%%%%%%%%%%%%%%%%%%%
\paragraph{Content of Main File.}

It is advisable to place all content in the child files included by |\include|.
Any output contained in the main file will appear in all child documents
unless suppressed manually;
it cannot be suppressed automatically by the |\includeonly| directive
and thus should normally be avoided.
A method to include some content in the main file
by means of conditional processing is described in \secref{sec:conditional}.

%%%%%%%%%%%%%%%%%%%%%%%%%%%%%%%%%%%%%%%%
\paragraph{Page Numbering.}

When only a part of the document is compiled,
the appropriate numbering of pages
(as well as other status parameters)
is determined from the |.aux| files.
The latter contain information from previous passes.
However this information needs to propagate through
all intermediate child documents.
Therefore the page numbering in child documents may well
be inconsistent until the complete document is compiled at least once.

A useful (if unconventional) way to always ensure a consistent
page numbering is to restart the numbering in each child document
and denote the pages by `\textit{child}|.|\textit{page}'
where \textit{child} represents the chapter/section number of the child file.
This can be achieved by the command
|\numberwithin{page}{|\textit{child}|}|
of the \textsf{amsmath} package
where \textit{child} can be |chapter| or |section|
depending on the chosen structuring.
Alternatively, one can modify the macro |\thepage| appropriately
and reset the counter |page| at the start of each child file.

%%%%%%%%%%%%%%%%%%%%%%%%%%%%%%%%%%%%%%%%%%%%%%%%%%%%%%%%%%%%%%%%%%%%%%%%%%%%%%%%
\subsection{Conditional Processing}
\label{sec:conditional}

The package provides a mechanism to compile different versions
of a document. To customise the versions further some conditional processing
can come in handy to distinguish which version is being compiled.
The package provides two macros to describe the compilation context:

%%%%%%%%%%%%%%%%%%%%%%%%%%%%%%%%%%%%%%%%
\DescribeMacro{\ifchilddoc}
The conditional |\ifchilddoc| distinguishes between the compilation of
child documents and the main document:
%
\begin{center}
|\ifchilddoc |\textit{child-code}| |[|\||else |\textit{main-code}]| \||fi|
\end{center}

%%%%%%%%%%%%%%%%%%%%%%%%%%%%%%%%%%%%%%%%
\DescribeMacro{\childdocname}
\DescribeMacro{\childdocjob}
The macro |\childdocname| contains the filename (without extension)
of the main or child file being processed.
Note that |\childdocjob| will always contain the name of the main file.

%%%%%%%%%%%%%%%%%%%%%%%%%%%%%%%%%%%%%%%%
\paragraph{Title Page.}

Conditional processing can be used to include a title or banner page
in the main document when proper precautions are taken.
Importantly, the code in the main file should ensure that the page counter
(as well as other status parameters which are stored in the |.aux| files)
takes the same value after the conditional processing.
Otherwise the page numbers may take divergent values
depending on which part is compiled.

For example, a title page could be declared by:
%
\begin{center}
\begin{tabular}{l}
|\ifchilddoc\||else|\\
|\addtocounter{page}{-1}|\\
\textit{code for title page}\\
|\newpage|\\
|\||fi|
\end{tabular}
\end{center}
%
A banner page for the child documents can be generated by:
%
\begin{center}
\begin{tabular}{l}
|\ifchilddoc|\\
|\addtocounter{page}{-1}|\\
\textit{code for banner page}\\
|\newpage|\\
|\||fi|
\end{tabular}
\end{center}
%
Here one could write a message such as:
\begin{center}
|This is the part \childdocname{} of \childdocjob{}.|
\end{center}

%%%%%%%%%%%%%%%%%%%%%%%%%%%%%%%%%%%%%%%%%%%%%%%%%%%%%%%%%%%%%%%%%%%%%%%%%%%%%%%%
\subsection{Flags}
\label{sec:flags}

The package makes it easy to generate different versions
of the main or child documents.
To this end compilation flags can be defined
and assigned different default values.
They will be particularly useful in conjunction
with the forwarding mechanism described in \secref{sec:forward}.

For example, it may be useful to have a flag |\version|
which can be set to |draft| or |final|.
The document source will contain some conditional code
depending on the value of |\version|.
Suppose further, the flag should default to |final| for the main file
and to |draft| for child files
which is a natural assignment for editing the document.
This is achieved by placing the following code
in the preamble of the main document
(below the |\childdocmain| directive):
%
\begin{center}
\begin{tabular}{l}
|\ifchilddoc|\\
|\providecommand{\version}{draft}|\\
|\||else|\\
|\providecommand{\version}{final}|\\
|\||fi|
\end{tabular}
\end{center}
%
The definition by |\providecommand| makes sure
that previous definitions are not overwritten.
Further statements |\providecommand{\version}{...}|
can thus be added before the above code to override it.

For the main file, one might add a line
(between |\childdocmain| and the above block)
%
\begin{center}
|%\ifchilddoc\||else\providecommand{\version}{draft}\||fi|
\end{center}
%
which can be uncommented to produce a draft version.
Likewise one can add a line to the very top of a child file
(above the |\childdocof{|\textit{main}|}| directive)
%
\begin{center}
|%\providecommand{\version}{final}|
\end{center}
%
which can be uncommented to produce the final version of this child document.

%%%%%%%%%%%%%%%%%%%%%%%%%%%%%%%%%%%%%%%%%%%%%%%%%%%%%%%%%%%%%%%%%%%%%%%%%%%%%%%%
\subsection{Forwarding}
\label{sec:forward}

Different versions of the main or child documents
using compilation flags as described in \secref{sec:flags}
can be (permanently) stored in different files
for convenient compilation, viewing and distribution.
To this end, the package defines a command
to pass on compilation to a different file:

%%%%%%%%%%%%%%%%%%%%%%%%%%%%%%%%%%%%%%%%
\DescribeMacro{\childdocforward}
The command |\childdocforward| redirects processing to
another source file:
%
\begin{center}
\begin{tabular}{l}
|\input{childdoc.def}|\\
|\childdocforward[|\textit{main}|]{|\textit{dest}|}|\\
\end{tabular}
\end{center}
%
The argument \textit{dest} is the destination file
(without extension).
It should be the main file or one of the child files.
Note that further \textsf{childdoc} directives
such as |\childdocof| and |\childdocforward|
in the indicated file will be processed in this form.
The optional argument \textit{main}
passes on directly to the main file \textit{main}
while pretending to compile the child \textit{dest}.
This form behaves as if \textit{dest}
issues |\childdocof{|\textit{main}|}| right away,
and no further \textsf{childdoc} directives will be processed.

%%%%%%%%%%%%%%%%%%%%%%%%%%%%%%%%%%%%%%%%
\DescribeMacro{\...prefix}
In the alternative form |\childdocforwardprefix|,
%
\begin{center}
\begin{tabular}{l}
|\input{childdoc.def}|\\
|\childdocforwardprefix[|\textit{main}|]{|\textit{prefix}|}{|\textit{dest}|}|
\end{tabular}
\end{center}
%
the destination file is determined by a pattern
depending on the current file:
To make this work, the current file must be called
`{\textit{prefix}\hspace{0.2em}\textit{suffix}}'
with \textit{prefix} matching precisely the argument.
Processing is then passed on to the file
`{\textit{dest}\hspace{0.2em}\textit{suffix}}'.
Surely, the same effect is achieved by
directly specifying the
argument `{\textit{dest}\hspace{0.2em}\textit{suffix}}'
in the first form.
However, that requires to set up a different file
for each child. With the alternative form of the command
all these files can have exactly the same content
which simplifies setting them up and maintaining them.

For example, the following file |draft.tex|
with a compilation flag |\version| as described in \secref{sec:flags}
compiles the main document as a draft:
%
\begin{center}
\begin{tabular}{l}
|\def\version{draft}|\\
|\input{childdoc.def}|\\
|\childdocforward{|\textit{main}|}|
\end{tabular}
\end{center}
%
Likewise, the following files |final|\textit{nn}|.tex|
compile the final version of the child document
|child|\textit{nn}|.tex|:
%
\begin{center}
\begin{tabular}{l}
|\def\version{final}|\\
|\input{childdoc.def}|\\
|\childdocforwardprefix{final}{child}|
\end{tabular}
\end{center}
%

Note that when several versions of a main file and/or of each child file
are to be generated, it may be convenient to set up a |Makefile| or
shell script to automatise the process.

%%%%%%%%%%%%%%%%%%%%%%%%%%%%%%%%%%%%%%%%%%%%%%%%%%%%%%%%%%%%%%%%%%%%%%%%%%%%%%%%
\subsection{Command Line Processing}
\label{sec:commandline}

The effect of redirection files can also be achieved by invoking
the \LaTeX{} compiler with a more elaborate command line.
Most conveniently this should be done as part
of a shell script or a |Makefile|.

When using \textsf{childdoc} in the main file, the following
command lines effectively perform a redirection
(note that depending on the shell being used,
backslashes may have to be doubled: `|\|' $\to$ `|\\|'):
%
\begin{center}
|... -jobname "|\textit{target}|" |\\|"|[\textit{flags}]%
|\input{childdoc.def}\childdocforward[|\textit{main}|]{|\textit{dest}|}"|
\end{center}
%
Here \textit{target} is the name of the output file,
\textit{main} is the name of the main file
and \textit{dest} is the name of the main or child file to be processed
(all filenames without extensions).
The optional argument \textit{main} can be omitted
if \textit{main} matches \textit{dest}.
Optionally, compilation \textit{flags} can be defined via |\def| commands.
This command line makes the \TeX{} engine believe
it is compiling the file \textit{target}
whose content is specified as the latter parameter.
The provided code then forwards the processing to
\textit{main} or \textit{dest} as described in \secref{sec:forward}.

%%%%%%%%%%%%%%%%%%%%%%%%%%%%%%%%%%%%%%%%%%%%%%%%%%%%%%%%%%%%%%%%%%%%%%%%%%%%%%%%
\subsection{Include by Input}
\label{sec:input}

Including child documents by |\include| has some restrictions by design.
Most notably, the content of a child document always occupies
its own set of pages; pages cannot be shared between child documents.
Usually, this behaviour makes perfect sense
because each child document contain an essential part of the document.
However, in some situations it may be desirable to compose
a document from a collection of parts
without having mandatory page breaks between then.
For this case, the package
provides a mechanism to include parts
by |\input| which can also be processed individually.
However, by construction this mechanism
requires manual handling of the content to be output.

%%%%%%%%%%%%%%%%%%%%%%%%%%%%%%%%%%%%%%%%
\DescribeMacro{\ifchilddocmanual}
The main file should be prepared as usual, see \secref{sec:include}.
However, the document body must make a distinction
between processing of an individual part and of the main document, e.g.:
%
\begin{center}
\begin{tabular}{l}
|\ifchilddocmanual|\\
|\input{\childdocname}|\\
|\||else|\\
\textit{document body with }|\input{|\textit{part}|}|\\
|\||fi|
\end{tabular}
\end{center}
%
The conditional |\ifchilddocmanual| is true whenever
a part to be included by |\input| is being compiled,
and the name of the part is stored in |\childdocname|.

%%%%%%%%%%%%%%%%%%%%%%%%%%%%%%%%%%%%%%%%
\DescribeMacro{\childdocby}
Each part to be included by |\input| should start with:
%
\begin{center}
\begin{tabular}{l}
|\input{childdoc.def}|\\
|\childdocby{|\textit{main}|}|\\
\end{tabular}
\end{center}
%
The directive |\childdocby| is similar to |\childdocof|
described in \secref{sec:include},
but the subsequent selection of content must be done manually.
To that end, both |\ifchilddoc| and |\ifchilddocmanual|
will be true upon processing of a part,
and the name of the part is stored in |\childdocname|.
Note that |\jobname| will be set to the filename of the current part
so that each part receives an individual |.aux| file
that does not interfere with the |.aux| file(s) of the main document.
This behaviour can be altered by the alternative form
|\childdocby[*]{|\textit{main}|}| (with a non-empty optional argument)
which uses the |.aux| file of the main document
by setting |\jobname| to \textit{main}.

%%%%%%%%%%%%%%%%%%%%%%%%%%%%%%%%%%%%%%%%%%%%%%%%%%%%%%%%%%%%%%%%%%%%%%%%%%%%%%%%
\subsection{Driver Development}
\label{sec:driver}

The \textsf{childdoc} mechanism can also be use for the development
of definition files such as \LaTeX{} styles or classes.
This case differs from the above setup with multiple parts
included by |\include| in that no |\includeonly| should be invoked.
This can be achieved by starting the include file
(before |\ProvidesPackage|) with:
%
\begin{center}
\begin{tabular}{l}
|\input{childdoc.def}|\\
|\childdocforward{|\textit{main}|}|\\
\end{tabular}
\end{center}
%
or alternatively with:
%
\begin{center}
\begin{tabular}{l}
|\input{childdoc.def}|\\
|\childdocby{|\textit{main}|}|\\
\end{tabular}
\end{center}
%
Both forms have slightly different effects as described above.
The main file is prepared as usual, see \secref{sec:include}.

%%%%%%%%%%%%%%%%%%%%%%%%%%%%%%%%%%%%%%%%%%%%%%%%%%%%%%%%%%%%%%%%%%%%%%%%%%%%%%%%
\subsection{Legacy Detection}
\label{sec:detection}

The directive |\childdocmain| in the main file can detect
whether the complete document or merely a child is to be compiled
even without using the directive |\childdocof|.
This method is deprecated because it is less robust
and there is no compelling reason to use it;
it is merely provided for backward compatibility
and it may be removed in future versions.

If the detection mechanism is to be used,
it is mandatory to correctly specify
the filename of the main file as the argument of |\childdocmain|:
%
\begin{center}
\begin{tabular}{l}
|\input{childdoc.def}|\\
|\childdocmain{|\textit{main}|}|\\
\end{tabular}
\end{center}
%
If |\jobname| does not match the argument \textit{main} of |\childdocmain|,
it is assumed that |\jobname| points to the child file to be compiled.
When using |\childdocmain| with the main file specified as argument,
it suffices to start a child file
with just |\input{|\textit{main}|}|
without loading of the package and using |\childdocof|.
If instead all processing is done
with the appropriate \textsf{childdoc} directives,
the argument of \textit{main} of |\childdocmain| can be empty.

An alternative version of the command line processing described
in \secref{sec:commandline} using the detection mechanism reads:
%
\begin{center}
|... -jobname "|\textit{target}|" "|[\textit{flags}]%
[|\def\jobname{|\textit{dest}|}|]|\input{|\textit{main}|}"|
\end{center}

%%%%%%%%%%%%%%%%%%%%%%%%%%%%%%%%%%%%%%%%%%%%%%%%%%%%%%%%%%%%%%%%%%%%%%%%%%%%%%%%
\subsection{Manual Code}
\label{sec:manual}

In case one cannot be certain whether the definitions file |childdoc.def|
is installed on the target \TeX{} distribution
and one prefers not to ship it,
it is conceivable to paste a few relevant commands into the sources.

To that end, drop all statements |\input{childdoc.def}|
and perform the replacements as outlined below.
Instead of |\childdocmain{|\textit{main}|}| add the following code
to the top of the main file:
%
\begin{center}
\begin{tabular}{l}
|\||ifdefined\childdocname\endinput\||fi\newif\ifchilddoc|\\
|\edef\childdocname{\scantokens\expandafter{\jobname\noexpand}}|\\
|\def\childdocmain{|\textit{main}|}\||ifx\childdocmain\childdocname\||else|\\
|\childdoctrue\includeonly{\childdocname}\let\jobname\childdocmain\||fi|\\
\end{tabular}
\end{center}
%
Instead of |\childdocof{|\textit{main}|}| just include the main file
at the top of each child file:
%
\begin{center}
|\input{|\textit{main}|}|
\end{center}
%
A simple redirection |\childdocforward{|\textit{dest}|}| is achieved by:
%
\begin{center}
|\def\jobname{|\textit{dest}|}\input{\jobname}|
\end{center}
%
The redirection with prefix
|\childdocforwardprefix[|\textit{prefix}|]{|\textit{dest}|}|
is accomplished by:
%
\begin{center}
\begin{tabular}{l}
|{\edef\jobname{\scantokens\expandafter{\jobname\noexpand}}|\\
|\def\redirectjob |\textit{prefix}|#1~~~{\gdef\jobname{|\textit{dest}|#1}}|\\
|\expandafter\redirectjob\jobname~~~}\input{\jobname}|
\end{tabular}
\end{center}

In an alternative approach,
child documents can be compiled by a specific command line
without additional code or specific definitions:
%
\begin{center}
|... -jobname "|\textit{target}|" "|[\textit{flags}]%
|\includeonly{|\textit{dest}|}\input{|\textit{main}|}"|
\end{center}
%

%%%%%%%%%%%%%%%%%%%%%%%%%%%%%%%%%%%%%%%%%%%%%%%%%%%%%%%%%%%%%%%%%%%%%%%%%%%%%%%%
%%%%%%%%%%%%%%%%%%%%%%%%%%%%%%%%%%%%%%%%%%%%%%%%%%%%%%%%%%%%%%%%%%%%%%%%%%%%%%%%
\section{Information}

%%%%%%%%%%%%%%%%%%%%%%%%%%%%%%%%%%%%%%%%%%%%%%%%%%%%%%%%%%%%%%%%%%%%%%%%%%%%%%%%
\subsection{Copyright}

Copyright \copyright{} 2017--2018 Niklas Beisert

This work may be distributed and/or modified under the
conditions of the \LaTeX{} Project Public License, either version 1.3
of this license or (at your option) any later version.
The latest version of this license is in
  \url{http://www.latex-project.org/lppl.txt}
and version 1.3 or later is part of all distributions of \LaTeX{}
version 2005/12/01 or later.

This work has the LPPL maintenance status `maintained'.

The Current Maintainer of this work is Niklas Beisert.

This work consists of the files |README.txt|, |childdoc.ins| and |childdoc.dtx|
as well as the derived files |childdoc.def|, |cdocsamp.tex|
with |cdocsch1.tex|, |cdocsch2.tex|, |cdocspt3.tex|, |cdocspt4.tex|,
|cdocsdrf.tex|, |cdocsfn1.tex|, |cdocsfn2.tex|
as well as |childdoc.pdf|.

%%%%%%%%%%%%%%%%%%%%%%%%%%%%%%%%%%%%%%%%%%%%%%%%%%%%%%%%%%%%%%%%%%%%%%%%%%%%%%%%
\subsection{Files and Installation}

The package consists of the files:
%
\begin{center}
\begin{tabular}{ll}
    |README.txt|   & readme file \\
    |childdoc.ins| & installation file \\
    |childdoc.dtx| & source file \\
    |childdoc.def| & definition file \\
    |cdocsamp.tex| & sample main file \\
    |cdocsch1.tex| & sample include file \\
    |cdocsch2.tex| & sample include file \\
    |cdocspt3.tex| & sample part file \\
    |cdocspt4.tex| & sample part file \\
    |cdocsdrf.tex| & sample redirection file \\
    |cdocsfn1.tex| & sample redirection file \\
    |cdocsfn2.tex| & sample redirection file \\
    |childdoc.pdf| & manual
\end{tabular}
\end{center}
%
The distribution consists of the files
|README.txt|, |childdoc.ins| and |childdoc.dtx|.
%
\begin{itemize}
\item
Run (pdf)\LaTeX{} on |childdoc.dtx|
to compile the manual |childdoc.pdf| (this file).
\item
Run \LaTeX{} on |childdoc.ins| to create the definitions file |childdoc.def|
and the sample |cdocsamp.tex| with include files
|cdocsch1.tex|, |cdocsch2.tex|, |cdocspt3.tex|, |cdocspt4.tex|,
|cdocsdrf.tex|, |cdocsfn1.tex|, |cdocsfn2.tex|.
Then copy the file |childdoc.def| to an appropriate directory of your \LaTeX{}
distribution, e.g.\ \textit{texmf-root}|/tex/latex/childdoc|.
\end{itemize}

%%%%%%%%%%%%%%%%%%%%%%%%%%%%%%%%%%%%%%%%%%%%%%%%%%%%%%%%%%%%%%%%%%%%%%%%%%%%%%%%
\subsection{Related CTAN Packages}

There are several other packages which offer a similar functionality:
%
\begin{itemize}
\item
The packages
\href{http://ctan.org/pkg/docmute}{\textsf{docmute}},
\href{http://ctan.org/pkg/includex}{\textsf{includex}} and
\href{http://ctan.org/pkg/standalone}{\textsf{standalone}}
provide commands to include only the document body of
a child file thus allowing both files to be compiled individually.
\item
The packages \href{http://ctan.org/pkg/subdocs}{\textsf{subdocs}}
and \href{http://ctan.org/pkg/subfiles}{\textsf{subfiles}}
provide structures in which the main and child documents can be
encapsulated and allowing them to be compiled individually.
The inclusion mechanism is different from the conventional |\include|.
\item
The package \href{http://ctan.org/pkg/combine}{\textsf{combine}}
is an elaborate solution to combine several documents into one.
\end{itemize}
%
See also the CTAN topic \href{http://ctan.org/topic/subdocs}{\textsf{subdocs}}
for further related packages.
The present package differs from the above solutions in that
a document structure constructed with the conventional |\include| mechanism
just needs two extra commands at the top of every file
such that all constituent files can be compiled individually.

%%%%%%%%%%%%%%%%%%%%%%%%%%%%%%%%%%%%%%%%%%%%%%%%%%%%%%%%%%%%%%%%%%%%%%%%%%%%%%%%
%\subsection{Feature Suggestions}
%
%The following is a list of features which may be useful for future
%versions of this package:
%%
%\begin{itemize}
%\item
%\ldots
%\end{itemize}

%%%%%%%%%%%%%%%%%%%%%%%%%%%%%%%%%%%%%%%%%%%%%%%%%%%%%%%%%%%%%%%%%%%%%%%%%%%%%%%%
\subsection{Revision History}

%%%%%%%%%%%%%%%%%%%%%%%%%%%%%%%%%%%%%%%%
\paragraph{v2.0:} 2018/12/30

\begin{itemize}
\item
immediate forward processing
\item
added |\childdocby| mechanism
\item
manual restructured
\end{itemize}

%%%%%%%%%%%%%%%%%%%%%%%%%%%%%%%%%%%%%%%%
\paragraph{v1.6:} 2018/01/17

\begin{itemize}
\item
application for development of include files
\item
corrections to manual
\end{itemize}

%%%%%%%%%%%%%%%%%%%%%%%%%%%%%%%%%%%%%%%%
\paragraph{v1.5:} 2017/05/21

\begin{itemize}
\item
more complete structuring introduced
\item
|\childdocof| introduced
\item
|\childdoc| renamed to |\childdocmain|
\item
|\childredirect| renamed to |\childdocforward| and |\childdocforwardprefix|
and functionality expanded
\end{itemize}

%%%%%%%%%%%%%%%%%%%%%%%%%%%%%%%%%%%%%%%%
\paragraph{v1.0:} 2017/04/27

\begin{itemize}
\item
manual and install package
\item
first version published on CTAN
\end{itemize}

%%%%%%%%%%%%%%%%%%%%%%%%%%%%%%%%%%%%%%%%
\paragraph{v0.6:} 2017/04/26

\begin{itemize}
\item
redirection mechanism added
\end{itemize}

%%%%%%%%%%%%%%%%%%%%%%%%%%%%%%%%%%%%%%%%
\paragraph{v0.5:} 2017/04/26

\begin{itemize}
\item
functionality in definition file
\end{itemize}


%%%%%%%%%%%%%%%%%%%%%%%%%%%%%%%%%%%%%%%%%%%%%%%%%%%%%%%%%%%%%%%%%%%%%%%%%%%%%%%%
%%%%%%%%%%%%%%%%%%%%%%%%%%%%%%%%%%%%%%%%%%%%%%%%%%%%%%%%%%%%%%%%%%%%%%%%%%%%%%%%
%%%%%%%%%%%%%%%%%%%%%%%%%%%%%%%%%%%%%%%%%%%%%%%%%%%%%%%%%%%%%%%%%%%%%%%%%%%%%%%%
\appendix

\settowidth\MacroIndent{\rmfamily\scriptsize 000\ }

 \DocInput{childdoc.dtx}

\end{document}
%</driver>
% \fi
%
% %%%%%%%%%%%%%%%%%%%%%%%%%%%%%%%%%%%%%%%%%%%%%%%%%%%%%%%%%%%%%%%%%%%%%%%%%%%%%%
% %%%%%%%%%%%%%%%%%%%%%%%%%%%%%%%%%%%%%%%%%%%%%%%%%%%%%%%%%%%%%%%%%%%%%%%%%%%%%%
% \section{Sample}
%\iffalse
%<*samplemain>
%\fi
%
% The following presents a sample document
% with two chapters, two parts, a title page,
% a compile flag as well as three forwarding files to set the flag.
% It consists of eight |.tex| files:
% \begin{center}
% \begin{tabular}{ll}
% |cdocsamp.tex|&main file\\
% |cdocsch1.tex|&include file for chapter 1\\
% |cdocsch2.tex|&include file for chapter 2\\
% |cdocspt3.tex|&include file for part 3\\
% |cdocspt4.tex|&include file for part 4\\
% |cdocsdrf.tex|&forwarding file for main file in draft mode\\
% |cdocsfi1.tex|&forwarding file for final version of chapter 1\\
% |cdocsfi2.tex|&forwarding file for final version of chapter 2\\
% \end{tabular}
% \end{center}
% Each of the eight files can be compiled directly by the \LaTeX{} compiler.
%
% %%%%%%%%%%%%%%%%%%%%%%%%%%%%%%%%%%%%%%
% \paragraph{Main File.}
%
% The main file is called |cdocsamp.tex|.
%
% Load the \textsf{childdoc} definitions and
% declare the filename for the main document:
%    \begin{macrocode}
\input{childdoc.def}
\childdocmain{}
%    \end{macrocode}

% Optional override for |\version| flag:
%    \begin{macrocode}
%%\ifchilddoc\else\providecommand{\version}{draft}\fi
%    \end{macrocode}

% Define the default values for the |\version| flag
% (|final| for the main file and |draft| for childs):
%    \begin{macrocode}
\ifchilddoc
\providecommand{\version}{draft}
\else
\providecommand{\version}{final}
\fi
%    \end{macrocode}

% Load the standard document class:
%    \begin{macrocode}
\documentclass[12pt]{article}
%    \end{macrocode}

% Start the document body:
%    \begin{macrocode}
\begin{document}
%    \end{macrocode}

% Declare a title page.
% Print title, part of document being processed and version flag:
%    \begin{macrocode}
\addtocounter{page}{-1}
\begin{center}
{\LARGE\bfseries{}childdoc example\par}
\vspace{1cm}
\ifchilddoc
\ifchilddocmanual part\else chapter\fi:
`\childdocname' of `\childdocjob'\par
\else
main document: `\childdocjob'\par
\fi
version: \version\par
\end{center}
\newpage
%    \end{macrocode}

% Manually include selected file,
% otherwise process as usual:
%    \begin{macrocode}
\ifchilddocmanual
\section*{part `\childdocname'}
\input{\childdocname}
\else
%    \end{macrocode}

% Include the two chapters:
%    \begin{macrocode}
\include{cdocsch1}
\include{cdocsch2}
%    \end{macrocode}

% Include the two parts unless only chapters should be displayed:
%    \begin{macrocode}
\ifchilddoc\else
\section{part three}
\input{cdocspt3}
\section{part four}
\input{cdocspt4}
\fi
%    \end{macrocode}

% Process as usual until here:
%    \begin{macrocode}
\fi
%    \end{macrocode}

% End of document body:
%    \begin{macrocode}
\end{document}
%    \end{macrocode}
%\iffalse
%</samplemain>
%\fi
%
% %%%%%%%%%%%%%%%%%%%%%%%%%%%%%%%%%%%%%%
% \paragraph{Chapter Include Files.}
%
% The include files are called |cdocsch1.tex| and |cdocsch2.tex|.
%
%\iffalse
%<*samplechap1|samplechap2>
%\fi

% Optional override for |\version| flag:
%    \begin{macrocode}
%%\providecommand{\version}{final}
%    \end{macrocode}

% Include the main document:
%    \begin{macrocode}
\input{childdoc.def}
\childdocof{cdocsamp}
%    \end{macrocode}

%\iffalse
%</samplechap1|samplechap2>
%\fi
%
%\iffalse
%<*samplechap1>
%\fi
% Some text for chapter 1:
%    \begin{macrocode}
\section{one}
some text in chapter one
%    \end{macrocode}

%\iffalse
%</samplechap1>
%\fi
% Some text for chapter 2:
%\iffalse
%<*samplechap2>
%\fi
%    \begin{macrocode}
\section{two}
more text in chapter two
%    \end{macrocode}

%\iffalse
%</samplechap2>
%\fi
%
% %%%%%%%%%%%%%%%%%%%%%%%%%%%%%%%%%%%%%%
% \paragraph{Part Include Files.}
%
% The include files are called |cdocspt3.tex| and |cdocspt4.tex|.
%
%\iffalse
%<*samplepart3|samplepart4>
%\fi

% Optional override for |\version| flag:
%    \begin{macrocode}
%%\providecommand{\version}{final}
%    \end{macrocode}

% Include the main document:
%    \begin{macrocode}
\input{childdoc.def}
\childdocby{cdocsamp}
%    \end{macrocode}

%\iffalse
%</samplepart3|samplepart4>
%\fi
%
%\iffalse
%<*samplepart3>
%\fi
% Some text for part 3:
%    \begin{macrocode}
some text in part three
%    \end{macrocode}

%\iffalse
%</samplepart3>
%\fi
% Some text for part 4:
%\iffalse
%<*samplepart4>
%\fi
%    \begin{macrocode}
more text in part four
%    \end{macrocode}

%\iffalse
%</samplepart4>
%\fi
%
% %%%%%%%%%%%%%%%%%%%%%%%%%%%%%%%%%%%%%%
% \paragraph{Forwarding for a Complete Draft.}
%
% The following forwarding file |cdocsdrf.tex|
% compiles the main document in draft mode:
%\iffalse
%<*sampledraft>
%\fi
%    \begin{macrocode}
\def\version{draft}
\input{childdoc.def}
\childdocforward{cdocsamp}
%    \end{macrocode}

%\iffalse
%</sampledraft>
%\fi
%
% %%%%%%%%%%%%%%%%%%%%%%%%%%%%%%%%%%%%%%
% \paragraph{Forwarding for Final Version of the Chapters.}
%
% The following forwarding files |cdocsfn1.tex| and |cdocsfn2.tex|
% (with identical content)
% compile the final versions of the child documents
% |cdocsch1.tex| and |cdocsch2.tex|, respectively:
%\iffalse
%<*samplefinal>
%\fi
%    \begin{macrocode}
\def\version{final}
\input{childdoc.def}
\childdocforwardprefix[cdocsamp]{cdocsfn}{cdocsch}
%    \end{macrocode}

%\iffalse
%</samplefinal>
%\fi
%
% %%%%%%%%%%%%%%%%%%%%%%%%%%%%%%%%%%%%%%
% \paragraph{Command Line Processing.}
%
% The following three command lines generate the output files
% |cdocscld|, |cdocscl1| and |cdocscl2|
% which should be identical to
% |cdocsdrf|, |cdocsch1| and |cdocsfn2|, respectively:
% \begin{center}
% \begin{tabular}{l}
% |latex -jobname cdocscld \|\\
% |  "\def\version{draft}\input{childdoc.def}\childdocforward{cdocsamp}"|\\
% |latex -jobname cdocscl1 \|\\
% |  "\input{childdoc.def}\childdocforward[cdocsamp]{cdocsch1}"|\\
% |latex -jobname cdocscl2 \|\\
% |  "\def\version{final}\input{childdoc.def}\childdocforward{cdocsch2}"|
% \end{tabular}
% \end{center}
% Note that the trailing backslash on each first line
% merely continues the input to the second line
% (for convenient cut ant paste).
% Furthermore, the command |latex| can be replaced by any
% of its alternative versions such as |pdflatex|.
%
% %%%%%%%%%%%%%%%%%%%%%%%%%%%%%%%%%%%%%%%%%%%%%%%%%%%%%%%%%%%%%%%%%%%%%%%%%%%%%%
% %%%%%%%%%%%%%%%%%%%%%%%%%%%%%%%%%%%%%%%%%%%%%%%%%%%%%%%%%%%%%%%%%%%%%%%%%%%%%%
% \section{Implementation}
%\iffalse
%<*package>
%\fi
%
% This section describes the definitions file |childdoc.def|.

% The definitions cannot be loaded using |\usepackage| or |\RequirePackage|
% which has a mechanism to prevent loading a style file more than once.
% When loading the definitions by means of |\input|
% multiple instances have to be prevented manually:
%\iffalse
%This code needs to be before the `\ProvidesFile' directive
%which is defined at the beginning of this file.
%Therefore it is also placed there and commented out here.
%</package>
%<*discard>
%\fi
%    \begin{macrocode}
\ifdefined\childdocmain\endinput\fi
%    \end{macrocode}
%\iffalse
%</discard>
%<*package>
%\fi
%
% \macro{\ifchilddoc}
% \macro{\ifchilddocmanual}
% The conditional |\ifchilddoc| tells whether a
% child (true) or main (false) document is being compiled.
% The conditional |\ifchilddocmanual| tells whether
% the |\includeonly| mechanism is used (false) or
% the selection of child files must be performed manually (true).
% The definitions initialise to false:
%    \begin{macrocode}
\newif\ifchilddoc
\newif\ifchilddocmanual
%    \end{macrocode}

% \macro{\childdocname}
% \macro{\childdocjob}
% The macro |\childdocname| stores the name of the main document
% to be compiled. The macro |\childdocjob| stores the name of
% the document on which the \LaTeX{} compiler was originally invoked.
% The content of |\jobname| cannot be compared
% to filenames specified in the source due to different catcodes.
% The following code rescans |\jobname|, stores the result
% in |\childdocname| and saves a copy in |\childdocjob|:
%    \begin{macrocode}
\edef\childdocname{\scantokens\expandafter{\jobname\noexpand}}
\let\childdocjob\childdocname
%    \end{macrocode}

% \macro{\childdocdisable}
% The macro |\childdocdisable| prevents the main file
% from being processed more than once.
% At this stage, the main document command |\childdocmain|
% is assumed to be called once again where it should do nothing.
% Any subsequent call to it should prevent
% a secondary processing of the main document
% It overwrites the forwarding commands
% |\childdocof| and |\childdocforward|
% with empty macros to prevent further inclusions of the main document:
%    \begin{macrocode}
\newcommand{\childdocdisable}
{
  \renewcommand{\childdocmain}[1]{\renewcommand{\childdocmain}[1]{\endinput}}
  \renewcommand{\childdocof}[1]{}
  \renewcommand{\childdocby}[2][]{}
  \renewcommand{\childdocforward}[2][]{}
  \renewcommand{\childdocdisable}{}
}
%    \end{macrocode}

% \macro{\childdocmain}
% The macro |\childdocmain| is to be called at the top of the main file
% with nothing or the main filename (without extension) as argument.
% First, it breaks loops.
% If the argument is not empty and does not match |\childdocname|
% (which is set by the first inclusion of |childdoc.def|),
% |\ifchilddoc| is set to true, |\includeonly| is applied to the child file
% and |\jobname| is set to the main file
% (for proper handling of |.aux| files):
%    \begin{macrocode}
\newcommand{\childdocmain}[1]
{
  \childdocdisable\childdocmain{}
  \if?#1?\else
    \begingroup
      \def\childdoctmp{#1}
      \ifx\childdoctmp\childdocname
        \def\childdoctmp{}
      \else
        \def\childdoctmp
        {
          \childdoctrue
          \includeonly{\childdocname}
          \def\childdocjob{#1}
          \def\jobname{#1}
        }
      \fi
      \expandafter
    \endgroup
    \childdoctmp
  \fi
}
%    \end{macrocode}

% \macro{\childdocof}
% The command |\childdocof| redirects
% compilation to the main file |#1|.
%    \begin{macrocode}
\newcommand{\childdocof}[1]
{
  \childdocdisable
  \childdoctrue
  \includeonly{\childdocname}
  \def\jobname{#1}
  \def\childdocjob{#1}
  \input{#1}
}
%    \end{macrocode}

% \macro{\childdocby}
% The command |\childdocby| ....
%    \begin{macrocode}
\newcommand{\childdocby}[2][]
{
  \childdocdisable
  \childdoctrue
  \childdocmanualtrue
  \if?#1?\else
    \def\jobname{#2}
  \fi
  \def\childdocjob{#2}
  \input{#2}
  \endinput
}
%    \end{macrocode}

% \macro{\childdocforward}
% The command |\childdocforward| redirects
% compilation to the main file or
% (if the optional argument is given) a child file.
% Parameters are set as if the main file
% or a child file starting with |\childdocof| was compiled.
% Then compilation is handed over to the main file:
%    \begin{macrocode}
\newcommand{\childdocforward}[2][]
{
  \begingroup
    \if?#1?
      \def\childdoctmp
      {
        \def\childdocname{#2}
        \def\childdocjob{#2}
        \def\jobname{#2}
        \input{#2}
        \endinput
      }
    \else
      \def\childdoctmp
      {
        \childdocdisable
        \def\childdocname{#2}
        \childdoctrue
        \includeonly{#2}
        \def\childdocjob{#1}
        \def\jobname{#1}
        \input{#1}
        \endinput
      }
    \fi
    \expandafter
  \endgroup
  \childdoctmp
}
%    \end{macrocode}

% \macro{\childdocforwardprefix}
% The command |\childdocforwardprefix| redirects
% compilation to the main or a child file by means of a pattern.
% The prefix |#1| in the current filename is replaced by |#2|
% and the suffix of the current filename is kept
% (it is assumed that the filename does not contain the substring `|~~~|'
% which is used as a delimiter).
% Compilation is handed over to the new file by |\childdocforward|:
%    \begin{macrocode}
\newcommand{\childdocforwardprefix}[3][]
{
  \begingroup
    \def\childdocextract #2##1~~~{\def\childdoctmp{\childdocforward[#1]{#3##1}}}
    \expandafter\childdocextract\childdocname~~~
    \expandafter
  \endgroup
  \childdoctmp
}
%    \end{macrocode}

% \macro{\childdoc}
% The deprecated macro |\childdoc| is a legacy version of |\childdocmain|:
%    \begin{macrocode}
\newcommand{\childdoc}{\childdocmain}
%    \end{macrocode}

% \macro{\childdocredirect}
% The deprecated macro |\childdocredirect| is a legacy version
% of |\childdocforward| and |\childdocforwardprefix|:
%    \begin{macrocode}
\newcommand{\childdocredirect}[2][]
{
  \begingroup
    \if?#1?
      \def\childdoctmp{\childdocforward{#2}}
    \else
      \def\childdoctmp{\childdocforwardprefix{#1}{#2}}
    \fi
    \expandafter
  \endgroup
  \childdoctmp
}
%    \end{macrocode}

%\iffalse
%</package>
%\fi
%
\endinput
|\\
|\childdocforward{|\textit{main}|}|\\
\end{tabular}
\end{center}
%
or alternatively with:
%
\begin{center}
\begin{tabular}{l}
|% \iffalse
%
% childdoc.dtx Copyright (C) 2017-2018 Niklas Beisert
%
% This work may be distributed and/or modified under the
% conditions of the LaTeX Project Public License, either version 1.3
% of this license or (at your option) any later version.
% The latest version of this license is in
%   http://www.latex-project.org/lppl.txt
% and version 1.3 or later is part of all distributions of LaTeX
% version 2005/12/01 or later.
%
% This work has the LPPL maintenance status `maintained'.
%
% The Current Maintainer of this work is Niklas Beisert.
%
% This work consists of the files childdoc.dtx and childdoc.ins
% and the derived files childdoc.def and cdocsamp.tex with
% cdocsch1.tex, cdocsch2.tex, cdocsdrf.tex, cdocsfn1.tex, cdocsfn2.tex.
%
%<package>\ifdefined\childdocmain\endinput\fi
%<package>\ProvidesFile{childdoc.def}[2018/12/30 v2.0 child document driver]
%<samplemain>\ProvidesFile{cdocsamp.tex}[2018/12/30 v2.0 sample for childdoc]
%<*driver>
%\ProvidesFile{childdoc.drv}[2018/12/30 v2.0 childdoc reference manual file]
\PassOptionsToClass{10pt,a4paper}{article}
\documentclass{ltxdoc}

\usepackage[margin=35mm]{geometry}
\usepackage{hyperref}
\usepackage{hyperxmp}
\usepackage[usenames]{color}

\hypersetup{colorlinks=true}
\hypersetup{pdfstartview=FitH}
\hypersetup{pdfpagemode=UseNone}
\hypersetup{pdfsource={}}
\hypersetup{pdflang={en-UK}}
\hypersetup{pdfcopyright={Copyright 2017-2018 Niklas Beisert.
  This work may be distributed and/or modified under the
  conditions of the LaTeX Project Public License, either version 1.3
  of this license or (at your option) any later version.}}
\hypersetup{pdflicenseurl={http://www.latex-project.org/lppl.txt}}
\hypersetup{pdfcontactaddress={ETH Zurich, ITP, HIT K,
  Wolfgang-Pauli-Strasse 27}}
\hypersetup{pdfcontactpostcode={8093}}
\hypersetup{pdfcontactcity={Zurich}}
\hypersetup{pdfcontactcountry={Switzerland}}
\hypersetup{pdfcontactemail={nbeisert@itp.phys.ethz.ch}}
\hypersetup{pdfcontacturl={http://people.phys.ethz.ch/\xmptilde nbeisert/}}

\newcommand{\secref}[1]{\hyperref[#1]{section \ref*{#1}}}

\parskip1ex
\parindent0pt
\let\olditemize\itemize
\def\itemize{\olditemize\parskip0pt}

\begin{document}

\title{The \textsf{childdoc} Package}
\hypersetup{pdftitle={The childdoc Package}}
\author{Niklas Beisert\\[2ex]
  Institut f\"ur Theoretische Physik\\
  Eidgen\"ossische Technische Hochschule Z\"urich\\
  Wolfgang-Pauli-Strasse 27, 8093 Z\"urich, Switzerland\\[1ex]
  \href{mailto:nbeisert@itp.phys.ethz.ch}
  {\texttt{nbeisert@itp.phys.ethz.ch}}}
\hypersetup{pdfauthor={Niklas Beisert}}
\hypersetup{pdfsubject={Manual for the LaTeX2e Package childdoc}}
\date{30 December 2018, \textsf{v2.0}}
\maketitle

\begin{abstract}\noindent
\textsf{childdoc} is a \LaTeXe{} package
that enables the direct compilation
of document sections included by |\include|
to individual files.
\end{abstract}

\begingroup
\parskip0ex
\tableofcontents
\endgroup

%%%%%%%%%%%%%%%%%%%%%%%%%%%%%%%%%%%%%%%%%%%%%%%%%%%%%%%%%%%%%%%%%%%%%%%%%%%%%%%%
%%%%%%%%%%%%%%%%%%%%%%%%%%%%%%%%%%%%%%%%%%%%%%%%%%%%%%%%%%%%%%%%%%%%%%%%%%%%%%%%
\section{Introduction}

\LaTeX{} provides a mechanism to structure a large document (such as a book)
into a main file and several child files (containing the chapters)
using the |\include| command.
This mechanism is beneficial for documents
which span hundreds of pages in order to
make the source file(s) more manageable.
Moreover, compilation can be restricted to
selected child files by means of the |\includeonly| command.
The latter feature can be used to reduce the compilation time while editing
(this was significantly more useful in the earlier days of \LaTeX{})
or to generate a smaller document which is easier to navigate.
Another application of |\includeonly| is to generate
documents consisting of selected parts of the complete document.

However, there are a few drawbacks of the plain |\include| mechanism:
\begin{itemize}
\item
The child files cannot be compiled on their own,
they can only be compiled via the main file.
A naive editing environment
(such as a text editor with an option
to have the current file processed by \LaTeX)
may require one to switch to the main file before compiling;
attempting to compile the child file produces errors.
\item
The main file must be modified (each time)
to adjust the |\includeonly| command
to the present needs. This easily leaves the main file in a messy state.
\item
The generated document will always carry the filename
of the main document. This is inconvenient if
several child files are to be compiled and
to be kept for distribution.
\end{itemize}

The present package provides a simple interface
to make child files individually compilable by \LaTeX{}.
Compiling a child file then has the same effect as compiling
the main file with an |\includeonly| command
to select the appropriate child.
Moreover the generated document will carry the name of the child
rather than the main file.
This resolves all three above issues.

This feature is meant to make the editing of books,
thesis documents and lecture notes somewhat more convenient.
However, the package can also be used efficiently for
composing a series of documents (such as exercise sheets)
which are typically distributed individually.
It then assists the author in generating the individual documents
(potentially in different versions)
as well as a document containing the collected series.
Another application is in developing style files
or other kinds of included material
where compilation of the style file could redirect
to a sample or test file.

%%%%%%%%%%%%%%%%%%%%%%%%%%%%%%%%%%%%%%%%%%%%%%%%%%%%%%%%%%%%%%%%%%%%%%%%%%%%%%%%
%%%%%%%%%%%%%%%%%%%%%%%%%%%%%%%%%%%%%%%%%%%%%%%%%%%%%%%%%%%%%%%%%%%%%%%%%%%%%%%%
\section{Usage}

First of all, the package \textsf{childdoc} is \emph{not} a standard
\LaTeXe{} |.sty| style file! Therefore it needs to be invoked in
a non-standard way.

%%%%%%%%%%%%%%%%%%%%%%%%%%%%%%%%%%%%%%%%%%%%%%%%%%%%%%%%%%%%%%%%%%%%%%%%%%%%%%%%
\subsection{Included Files}
\label{sec:include}

%%%%%%%%%%%%%%%%%%%%%%%%%%%%%%%%%%%%%%%%
\DescribeMacro{\childdocmain}
To use the package, add the commands
\begin{center}
\begin{tabular}{l}
|\input{childdoc.def}|\\
|\childdocmain{}|\\
\end{tabular}
\end{center}
at the very top of the main \LaTeX{} file,
in particular \emph{before} the |\documentclass| statement!
The argument of |\childdocmain| should be left empty
(but it must be present).

%%%%%%%%%%%%%%%%%%%%%%%%%%%%%%%%%%%%%%%%
\DescribeMacro{\childdocof}
Furthermore, add the commands
\begin{center}
\begin{tabular}{l}
|\input{childdoc.def}|\\
|\childdocof{|\textit{main}|}|\\
\end{tabular}
\end{center}
at the top of every child file \textit{child}
which is included by |\include{|\textit{child}|}|
from within the main file
(or at least for those files to be compiled individually).
The argument \textit{main} must be the filename of the main file.

There are a couple of
considerations in setting up the main and child documents:

%%%%%%%%%%%%%%%%%%%%%%%%%%%%%%%%%%%%%%%%
\paragraph{Restrictions.}

Please note the following restrictions:
\begin{itemize}
\item
|\childdocmain| must be called with one argument \textit{main}
to ensure compatibility with earlier version of the package.
It must either be empty (|\childdocmain{}|)
or precisely match the filename of the main file in which it is specified.
See \secref{sec:detection} for further information.
\item
The filename \textit{main} must be specified without the |.tex| extension.
\item
The filename \textit{main} is case sensitive
(even in case-insensitive file systems)
due to internal string comparison.
\item
The argument \textit{main} should be fully expanded, it cannot be a macro.
\item
Subdirectories and special characters should be avoided in filenames.
\item
The command |\childdocmain{|\textit{main}|}| must be followed by a whitespace.
It should not be followed immediately by another command
or by a comment mark `|%|'.
This is because the \TeX{} parser reads the token immediately following
the argument of |\childdocmain| and puts it
at the beginning of every child section;
however, a white\-space is ignored.
\end{itemize}

%%%%%%%%%%%%%%%%%%%%%%%%%%%%%%%%%%%%%%%%
\paragraph{Content of Main File.}

It is advisable to place all content in the child files included by |\include|.
Any output contained in the main file will appear in all child documents
unless suppressed manually;
it cannot be suppressed automatically by the |\includeonly| directive
and thus should normally be avoided.
A method to include some content in the main file
by means of conditional processing is described in \secref{sec:conditional}.

%%%%%%%%%%%%%%%%%%%%%%%%%%%%%%%%%%%%%%%%
\paragraph{Page Numbering.}

When only a part of the document is compiled,
the appropriate numbering of pages
(as well as other status parameters)
is determined from the |.aux| files.
The latter contain information from previous passes.
However this information needs to propagate through
all intermediate child documents.
Therefore the page numbering in child documents may well
be inconsistent until the complete document is compiled at least once.

A useful (if unconventional) way to always ensure a consistent
page numbering is to restart the numbering in each child document
and denote the pages by `\textit{child}|.|\textit{page}'
where \textit{child} represents the chapter/section number of the child file.
This can be achieved by the command
|\numberwithin{page}{|\textit{child}|}|
of the \textsf{amsmath} package
where \textit{child} can be |chapter| or |section|
depending on the chosen structuring.
Alternatively, one can modify the macro |\thepage| appropriately
and reset the counter |page| at the start of each child file.

%%%%%%%%%%%%%%%%%%%%%%%%%%%%%%%%%%%%%%%%%%%%%%%%%%%%%%%%%%%%%%%%%%%%%%%%%%%%%%%%
\subsection{Conditional Processing}
\label{sec:conditional}

The package provides a mechanism to compile different versions
of a document. To customise the versions further some conditional processing
can come in handy to distinguish which version is being compiled.
The package provides two macros to describe the compilation context:

%%%%%%%%%%%%%%%%%%%%%%%%%%%%%%%%%%%%%%%%
\DescribeMacro{\ifchilddoc}
The conditional |\ifchilddoc| distinguishes between the compilation of
child documents and the main document:
%
\begin{center}
|\ifchilddoc |\textit{child-code}| |[|\||else |\textit{main-code}]| \||fi|
\end{center}

%%%%%%%%%%%%%%%%%%%%%%%%%%%%%%%%%%%%%%%%
\DescribeMacro{\childdocname}
\DescribeMacro{\childdocjob}
The macro |\childdocname| contains the filename (without extension)
of the main or child file being processed.
Note that |\childdocjob| will always contain the name of the main file.

%%%%%%%%%%%%%%%%%%%%%%%%%%%%%%%%%%%%%%%%
\paragraph{Title Page.}

Conditional processing can be used to include a title or banner page
in the main document when proper precautions are taken.
Importantly, the code in the main file should ensure that the page counter
(as well as other status parameters which are stored in the |.aux| files)
takes the same value after the conditional processing.
Otherwise the page numbers may take divergent values
depending on which part is compiled.

For example, a title page could be declared by:
%
\begin{center}
\begin{tabular}{l}
|\ifchilddoc\||else|\\
|\addtocounter{page}{-1}|\\
\textit{code for title page}\\
|\newpage|\\
|\||fi|
\end{tabular}
\end{center}
%
A banner page for the child documents can be generated by:
%
\begin{center}
\begin{tabular}{l}
|\ifchilddoc|\\
|\addtocounter{page}{-1}|\\
\textit{code for banner page}\\
|\newpage|\\
|\||fi|
\end{tabular}
\end{center}
%
Here one could write a message such as:
\begin{center}
|This is the part \childdocname{} of \childdocjob{}.|
\end{center}

%%%%%%%%%%%%%%%%%%%%%%%%%%%%%%%%%%%%%%%%%%%%%%%%%%%%%%%%%%%%%%%%%%%%%%%%%%%%%%%%
\subsection{Flags}
\label{sec:flags}

The package makes it easy to generate different versions
of the main or child documents.
To this end compilation flags can be defined
and assigned different default values.
They will be particularly useful in conjunction
with the forwarding mechanism described in \secref{sec:forward}.

For example, it may be useful to have a flag |\version|
which can be set to |draft| or |final|.
The document source will contain some conditional code
depending on the value of |\version|.
Suppose further, the flag should default to |final| for the main file
and to |draft| for child files
which is a natural assignment for editing the document.
This is achieved by placing the following code
in the preamble of the main document
(below the |\childdocmain| directive):
%
\begin{center}
\begin{tabular}{l}
|\ifchilddoc|\\
|\providecommand{\version}{draft}|\\
|\||else|\\
|\providecommand{\version}{final}|\\
|\||fi|
\end{tabular}
\end{center}
%
The definition by |\providecommand| makes sure
that previous definitions are not overwritten.
Further statements |\providecommand{\version}{...}|
can thus be added before the above code to override it.

For the main file, one might add a line
(between |\childdocmain| and the above block)
%
\begin{center}
|%\ifchilddoc\||else\providecommand{\version}{draft}\||fi|
\end{center}
%
which can be uncommented to produce a draft version.
Likewise one can add a line to the very top of a child file
(above the |\childdocof{|\textit{main}|}| directive)
%
\begin{center}
|%\providecommand{\version}{final}|
\end{center}
%
which can be uncommented to produce the final version of this child document.

%%%%%%%%%%%%%%%%%%%%%%%%%%%%%%%%%%%%%%%%%%%%%%%%%%%%%%%%%%%%%%%%%%%%%%%%%%%%%%%%
\subsection{Forwarding}
\label{sec:forward}

Different versions of the main or child documents
using compilation flags as described in \secref{sec:flags}
can be (permanently) stored in different files
for convenient compilation, viewing and distribution.
To this end, the package defines a command
to pass on compilation to a different file:

%%%%%%%%%%%%%%%%%%%%%%%%%%%%%%%%%%%%%%%%
\DescribeMacro{\childdocforward}
The command |\childdocforward| redirects processing to
another source file:
%
\begin{center}
\begin{tabular}{l}
|\input{childdoc.def}|\\
|\childdocforward[|\textit{main}|]{|\textit{dest}|}|\\
\end{tabular}
\end{center}
%
The argument \textit{dest} is the destination file
(without extension).
It should be the main file or one of the child files.
Note that further \textsf{childdoc} directives
such as |\childdocof| and |\childdocforward|
in the indicated file will be processed in this form.
The optional argument \textit{main}
passes on directly to the main file \textit{main}
while pretending to compile the child \textit{dest}.
This form behaves as if \textit{dest}
issues |\childdocof{|\textit{main}|}| right away,
and no further \textsf{childdoc} directives will be processed.

%%%%%%%%%%%%%%%%%%%%%%%%%%%%%%%%%%%%%%%%
\DescribeMacro{\...prefix}
In the alternative form |\childdocforwardprefix|,
%
\begin{center}
\begin{tabular}{l}
|\input{childdoc.def}|\\
|\childdocforwardprefix[|\textit{main}|]{|\textit{prefix}|}{|\textit{dest}|}|
\end{tabular}
\end{center}
%
the destination file is determined by a pattern
depending on the current file:
To make this work, the current file must be called
`{\textit{prefix}\hspace{0.2em}\textit{suffix}}'
with \textit{prefix} matching precisely the argument.
Processing is then passed on to the file
`{\textit{dest}\hspace{0.2em}\textit{suffix}}'.
Surely, the same effect is achieved by
directly specifying the
argument `{\textit{dest}\hspace{0.2em}\textit{suffix}}'
in the first form.
However, that requires to set up a different file
for each child. With the alternative form of the command
all these files can have exactly the same content
which simplifies setting them up and maintaining them.

For example, the following file |draft.tex|
with a compilation flag |\version| as described in \secref{sec:flags}
compiles the main document as a draft:
%
\begin{center}
\begin{tabular}{l}
|\def\version{draft}|\\
|\input{childdoc.def}|\\
|\childdocforward{|\textit{main}|}|
\end{tabular}
\end{center}
%
Likewise, the following files |final|\textit{nn}|.tex|
compile the final version of the child document
|child|\textit{nn}|.tex|:
%
\begin{center}
\begin{tabular}{l}
|\def\version{final}|\\
|\input{childdoc.def}|\\
|\childdocforwardprefix{final}{child}|
\end{tabular}
\end{center}
%

Note that when several versions of a main file and/or of each child file
are to be generated, it may be convenient to set up a |Makefile| or
shell script to automatise the process.

%%%%%%%%%%%%%%%%%%%%%%%%%%%%%%%%%%%%%%%%%%%%%%%%%%%%%%%%%%%%%%%%%%%%%%%%%%%%%%%%
\subsection{Command Line Processing}
\label{sec:commandline}

The effect of redirection files can also be achieved by invoking
the \LaTeX{} compiler with a more elaborate command line.
Most conveniently this should be done as part
of a shell script or a |Makefile|.

When using \textsf{childdoc} in the main file, the following
command lines effectively perform a redirection
(note that depending on the shell being used,
backslashes may have to be doubled: `|\|' $\to$ `|\\|'):
%
\begin{center}
|... -jobname "|\textit{target}|" |\\|"|[\textit{flags}]%
|\input{childdoc.def}\childdocforward[|\textit{main}|]{|\textit{dest}|}"|
\end{center}
%
Here \textit{target} is the name of the output file,
\textit{main} is the name of the main file
and \textit{dest} is the name of the main or child file to be processed
(all filenames without extensions).
The optional argument \textit{main} can be omitted
if \textit{main} matches \textit{dest}.
Optionally, compilation \textit{flags} can be defined via |\def| commands.
This command line makes the \TeX{} engine believe
it is compiling the file \textit{target}
whose content is specified as the latter parameter.
The provided code then forwards the processing to
\textit{main} or \textit{dest} as described in \secref{sec:forward}.

%%%%%%%%%%%%%%%%%%%%%%%%%%%%%%%%%%%%%%%%%%%%%%%%%%%%%%%%%%%%%%%%%%%%%%%%%%%%%%%%
\subsection{Include by Input}
\label{sec:input}

Including child documents by |\include| has some restrictions by design.
Most notably, the content of a child document always occupies
its own set of pages; pages cannot be shared between child documents.
Usually, this behaviour makes perfect sense
because each child document contain an essential part of the document.
However, in some situations it may be desirable to compose
a document from a collection of parts
without having mandatory page breaks between then.
For this case, the package
provides a mechanism to include parts
by |\input| which can also be processed individually.
However, by construction this mechanism
requires manual handling of the content to be output.

%%%%%%%%%%%%%%%%%%%%%%%%%%%%%%%%%%%%%%%%
\DescribeMacro{\ifchilddocmanual}
The main file should be prepared as usual, see \secref{sec:include}.
However, the document body must make a distinction
between processing of an individual part and of the main document, e.g.:
%
\begin{center}
\begin{tabular}{l}
|\ifchilddocmanual|\\
|\input{\childdocname}|\\
|\||else|\\
\textit{document body with }|\input{|\textit{part}|}|\\
|\||fi|
\end{tabular}
\end{center}
%
The conditional |\ifchilddocmanual| is true whenever
a part to be included by |\input| is being compiled,
and the name of the part is stored in |\childdocname|.

%%%%%%%%%%%%%%%%%%%%%%%%%%%%%%%%%%%%%%%%
\DescribeMacro{\childdocby}
Each part to be included by |\input| should start with:
%
\begin{center}
\begin{tabular}{l}
|\input{childdoc.def}|\\
|\childdocby{|\textit{main}|}|\\
\end{tabular}
\end{center}
%
The directive |\childdocby| is similar to |\childdocof|
described in \secref{sec:include},
but the subsequent selection of content must be done manually.
To that end, both |\ifchilddoc| and |\ifchilddocmanual|
will be true upon processing of a part,
and the name of the part is stored in |\childdocname|.
Note that |\jobname| will be set to the filename of the current part
so that each part receives an individual |.aux| file
that does not interfere with the |.aux| file(s) of the main document.
This behaviour can be altered by the alternative form
|\childdocby[*]{|\textit{main}|}| (with a non-empty optional argument)
which uses the |.aux| file of the main document
by setting |\jobname| to \textit{main}.

%%%%%%%%%%%%%%%%%%%%%%%%%%%%%%%%%%%%%%%%%%%%%%%%%%%%%%%%%%%%%%%%%%%%%%%%%%%%%%%%
\subsection{Driver Development}
\label{sec:driver}

The \textsf{childdoc} mechanism can also be use for the development
of definition files such as \LaTeX{} styles or classes.
This case differs from the above setup with multiple parts
included by |\include| in that no |\includeonly| should be invoked.
This can be achieved by starting the include file
(before |\ProvidesPackage|) with:
%
\begin{center}
\begin{tabular}{l}
|\input{childdoc.def}|\\
|\childdocforward{|\textit{main}|}|\\
\end{tabular}
\end{center}
%
or alternatively with:
%
\begin{center}
\begin{tabular}{l}
|\input{childdoc.def}|\\
|\childdocby{|\textit{main}|}|\\
\end{tabular}
\end{center}
%
Both forms have slightly different effects as described above.
The main file is prepared as usual, see \secref{sec:include}.

%%%%%%%%%%%%%%%%%%%%%%%%%%%%%%%%%%%%%%%%%%%%%%%%%%%%%%%%%%%%%%%%%%%%%%%%%%%%%%%%
\subsection{Legacy Detection}
\label{sec:detection}

The directive |\childdocmain| in the main file can detect
whether the complete document or merely a child is to be compiled
even without using the directive |\childdocof|.
This method is deprecated because it is less robust
and there is no compelling reason to use it;
it is merely provided for backward compatibility
and it may be removed in future versions.

If the detection mechanism is to be used,
it is mandatory to correctly specify
the filename of the main file as the argument of |\childdocmain|:
%
\begin{center}
\begin{tabular}{l}
|\input{childdoc.def}|\\
|\childdocmain{|\textit{main}|}|\\
\end{tabular}
\end{center}
%
If |\jobname| does not match the argument \textit{main} of |\childdocmain|,
it is assumed that |\jobname| points to the child file to be compiled.
When using |\childdocmain| with the main file specified as argument,
it suffices to start a child file
with just |\input{|\textit{main}|}|
without loading of the package and using |\childdocof|.
If instead all processing is done
with the appropriate \textsf{childdoc} directives,
the argument of \textit{main} of |\childdocmain| can be empty.

An alternative version of the command line processing described
in \secref{sec:commandline} using the detection mechanism reads:
%
\begin{center}
|... -jobname "|\textit{target}|" "|[\textit{flags}]%
[|\def\jobname{|\textit{dest}|}|]|\input{|\textit{main}|}"|
\end{center}

%%%%%%%%%%%%%%%%%%%%%%%%%%%%%%%%%%%%%%%%%%%%%%%%%%%%%%%%%%%%%%%%%%%%%%%%%%%%%%%%
\subsection{Manual Code}
\label{sec:manual}

In case one cannot be certain whether the definitions file |childdoc.def|
is installed on the target \TeX{} distribution
and one prefers not to ship it,
it is conceivable to paste a few relevant commands into the sources.

To that end, drop all statements |\input{childdoc.def}|
and perform the replacements as outlined below.
Instead of |\childdocmain{|\textit{main}|}| add the following code
to the top of the main file:
%
\begin{center}
\begin{tabular}{l}
|\||ifdefined\childdocname\endinput\||fi\newif\ifchilddoc|\\
|\edef\childdocname{\scantokens\expandafter{\jobname\noexpand}}|\\
|\def\childdocmain{|\textit{main}|}\||ifx\childdocmain\childdocname\||else|\\
|\childdoctrue\includeonly{\childdocname}\let\jobname\childdocmain\||fi|\\
\end{tabular}
\end{center}
%
Instead of |\childdocof{|\textit{main}|}| just include the main file
at the top of each child file:
%
\begin{center}
|\input{|\textit{main}|}|
\end{center}
%
A simple redirection |\childdocforward{|\textit{dest}|}| is achieved by:
%
\begin{center}
|\def\jobname{|\textit{dest}|}\input{\jobname}|
\end{center}
%
The redirection with prefix
|\childdocforwardprefix[|\textit{prefix}|]{|\textit{dest}|}|
is accomplished by:
%
\begin{center}
\begin{tabular}{l}
|{\edef\jobname{\scantokens\expandafter{\jobname\noexpand}}|\\
|\def\redirectjob |\textit{prefix}|#1~~~{\gdef\jobname{|\textit{dest}|#1}}|\\
|\expandafter\redirectjob\jobname~~~}\input{\jobname}|
\end{tabular}
\end{center}

In an alternative approach,
child documents can be compiled by a specific command line
without additional code or specific definitions:
%
\begin{center}
|... -jobname "|\textit{target}|" "|[\textit{flags}]%
|\includeonly{|\textit{dest}|}\input{|\textit{main}|}"|
\end{center}
%

%%%%%%%%%%%%%%%%%%%%%%%%%%%%%%%%%%%%%%%%%%%%%%%%%%%%%%%%%%%%%%%%%%%%%%%%%%%%%%%%
%%%%%%%%%%%%%%%%%%%%%%%%%%%%%%%%%%%%%%%%%%%%%%%%%%%%%%%%%%%%%%%%%%%%%%%%%%%%%%%%
\section{Information}

%%%%%%%%%%%%%%%%%%%%%%%%%%%%%%%%%%%%%%%%%%%%%%%%%%%%%%%%%%%%%%%%%%%%%%%%%%%%%%%%
\subsection{Copyright}

Copyright \copyright{} 2017--2018 Niklas Beisert

This work may be distributed and/or modified under the
conditions of the \LaTeX{} Project Public License, either version 1.3
of this license or (at your option) any later version.
The latest version of this license is in
  \url{http://www.latex-project.org/lppl.txt}
and version 1.3 or later is part of all distributions of \LaTeX{}
version 2005/12/01 or later.

This work has the LPPL maintenance status `maintained'.

The Current Maintainer of this work is Niklas Beisert.

This work consists of the files |README.txt|, |childdoc.ins| and |childdoc.dtx|
as well as the derived files |childdoc.def|, |cdocsamp.tex|
with |cdocsch1.tex|, |cdocsch2.tex|, |cdocspt3.tex|, |cdocspt4.tex|,
|cdocsdrf.tex|, |cdocsfn1.tex|, |cdocsfn2.tex|
as well as |childdoc.pdf|.

%%%%%%%%%%%%%%%%%%%%%%%%%%%%%%%%%%%%%%%%%%%%%%%%%%%%%%%%%%%%%%%%%%%%%%%%%%%%%%%%
\subsection{Files and Installation}

The package consists of the files:
%
\begin{center}
\begin{tabular}{ll}
    |README.txt|   & readme file \\
    |childdoc.ins| & installation file \\
    |childdoc.dtx| & source file \\
    |childdoc.def| & definition file \\
    |cdocsamp.tex| & sample main file \\
    |cdocsch1.tex| & sample include file \\
    |cdocsch2.tex| & sample include file \\
    |cdocspt3.tex| & sample part file \\
    |cdocspt4.tex| & sample part file \\
    |cdocsdrf.tex| & sample redirection file \\
    |cdocsfn1.tex| & sample redirection file \\
    |cdocsfn2.tex| & sample redirection file \\
    |childdoc.pdf| & manual
\end{tabular}
\end{center}
%
The distribution consists of the files
|README.txt|, |childdoc.ins| and |childdoc.dtx|.
%
\begin{itemize}
\item
Run (pdf)\LaTeX{} on |childdoc.dtx|
to compile the manual |childdoc.pdf| (this file).
\item
Run \LaTeX{} on |childdoc.ins| to create the definitions file |childdoc.def|
and the sample |cdocsamp.tex| with include files
|cdocsch1.tex|, |cdocsch2.tex|, |cdocspt3.tex|, |cdocspt4.tex|,
|cdocsdrf.tex|, |cdocsfn1.tex|, |cdocsfn2.tex|.
Then copy the file |childdoc.def| to an appropriate directory of your \LaTeX{}
distribution, e.g.\ \textit{texmf-root}|/tex/latex/childdoc|.
\end{itemize}

%%%%%%%%%%%%%%%%%%%%%%%%%%%%%%%%%%%%%%%%%%%%%%%%%%%%%%%%%%%%%%%%%%%%%%%%%%%%%%%%
\subsection{Related CTAN Packages}

There are several other packages which offer a similar functionality:
%
\begin{itemize}
\item
The packages
\href{http://ctan.org/pkg/docmute}{\textsf{docmute}},
\href{http://ctan.org/pkg/includex}{\textsf{includex}} and
\href{http://ctan.org/pkg/standalone}{\textsf{standalone}}
provide commands to include only the document body of
a child file thus allowing both files to be compiled individually.
\item
The packages \href{http://ctan.org/pkg/subdocs}{\textsf{subdocs}}
and \href{http://ctan.org/pkg/subfiles}{\textsf{subfiles}}
provide structures in which the main and child documents can be
encapsulated and allowing them to be compiled individually.
The inclusion mechanism is different from the conventional |\include|.
\item
The package \href{http://ctan.org/pkg/combine}{\textsf{combine}}
is an elaborate solution to combine several documents into one.
\end{itemize}
%
See also the CTAN topic \href{http://ctan.org/topic/subdocs}{\textsf{subdocs}}
for further related packages.
The present package differs from the above solutions in that
a document structure constructed with the conventional |\include| mechanism
just needs two extra commands at the top of every file
such that all constituent files can be compiled individually.

%%%%%%%%%%%%%%%%%%%%%%%%%%%%%%%%%%%%%%%%%%%%%%%%%%%%%%%%%%%%%%%%%%%%%%%%%%%%%%%%
%\subsection{Feature Suggestions}
%
%The following is a list of features which may be useful for future
%versions of this package:
%%
%\begin{itemize}
%\item
%\ldots
%\end{itemize}

%%%%%%%%%%%%%%%%%%%%%%%%%%%%%%%%%%%%%%%%%%%%%%%%%%%%%%%%%%%%%%%%%%%%%%%%%%%%%%%%
\subsection{Revision History}

%%%%%%%%%%%%%%%%%%%%%%%%%%%%%%%%%%%%%%%%
\paragraph{v2.0:} 2018/12/30

\begin{itemize}
\item
immediate forward processing
\item
added |\childdocby| mechanism
\item
manual restructured
\end{itemize}

%%%%%%%%%%%%%%%%%%%%%%%%%%%%%%%%%%%%%%%%
\paragraph{v1.6:} 2018/01/17

\begin{itemize}
\item
application for development of include files
\item
corrections to manual
\end{itemize}

%%%%%%%%%%%%%%%%%%%%%%%%%%%%%%%%%%%%%%%%
\paragraph{v1.5:} 2017/05/21

\begin{itemize}
\item
more complete structuring introduced
\item
|\childdocof| introduced
\item
|\childdoc| renamed to |\childdocmain|
\item
|\childredirect| renamed to |\childdocforward| and |\childdocforwardprefix|
and functionality expanded
\end{itemize}

%%%%%%%%%%%%%%%%%%%%%%%%%%%%%%%%%%%%%%%%
\paragraph{v1.0:} 2017/04/27

\begin{itemize}
\item
manual and install package
\item
first version published on CTAN
\end{itemize}

%%%%%%%%%%%%%%%%%%%%%%%%%%%%%%%%%%%%%%%%
\paragraph{v0.6:} 2017/04/26

\begin{itemize}
\item
redirection mechanism added
\end{itemize}

%%%%%%%%%%%%%%%%%%%%%%%%%%%%%%%%%%%%%%%%
\paragraph{v0.5:} 2017/04/26

\begin{itemize}
\item
functionality in definition file
\end{itemize}


%%%%%%%%%%%%%%%%%%%%%%%%%%%%%%%%%%%%%%%%%%%%%%%%%%%%%%%%%%%%%%%%%%%%%%%%%%%%%%%%
%%%%%%%%%%%%%%%%%%%%%%%%%%%%%%%%%%%%%%%%%%%%%%%%%%%%%%%%%%%%%%%%%%%%%%%%%%%%%%%%
%%%%%%%%%%%%%%%%%%%%%%%%%%%%%%%%%%%%%%%%%%%%%%%%%%%%%%%%%%%%%%%%%%%%%%%%%%%%%%%%
\appendix

\settowidth\MacroIndent{\rmfamily\scriptsize 000\ }

 \DocInput{childdoc.dtx}

\end{document}
%</driver>
% \fi
%
% %%%%%%%%%%%%%%%%%%%%%%%%%%%%%%%%%%%%%%%%%%%%%%%%%%%%%%%%%%%%%%%%%%%%%%%%%%%%%%
% %%%%%%%%%%%%%%%%%%%%%%%%%%%%%%%%%%%%%%%%%%%%%%%%%%%%%%%%%%%%%%%%%%%%%%%%%%%%%%
% \section{Sample}
%\iffalse
%<*samplemain>
%\fi
%
% The following presents a sample document
% with two chapters, two parts, a title page,
% a compile flag as well as three forwarding files to set the flag.
% It consists of eight |.tex| files:
% \begin{center}
% \begin{tabular}{ll}
% |cdocsamp.tex|&main file\\
% |cdocsch1.tex|&include file for chapter 1\\
% |cdocsch2.tex|&include file for chapter 2\\
% |cdocspt3.tex|&include file for part 3\\
% |cdocspt4.tex|&include file for part 4\\
% |cdocsdrf.tex|&forwarding file for main file in draft mode\\
% |cdocsfi1.tex|&forwarding file for final version of chapter 1\\
% |cdocsfi2.tex|&forwarding file for final version of chapter 2\\
% \end{tabular}
% \end{center}
% Each of the eight files can be compiled directly by the \LaTeX{} compiler.
%
% %%%%%%%%%%%%%%%%%%%%%%%%%%%%%%%%%%%%%%
% \paragraph{Main File.}
%
% The main file is called |cdocsamp.tex|.
%
% Load the \textsf{childdoc} definitions and
% declare the filename for the main document:
%    \begin{macrocode}
\input{childdoc.def}
\childdocmain{}
%    \end{macrocode}

% Optional override for |\version| flag:
%    \begin{macrocode}
%%\ifchilddoc\else\providecommand{\version}{draft}\fi
%    \end{macrocode}

% Define the default values for the |\version| flag
% (|final| for the main file and |draft| for childs):
%    \begin{macrocode}
\ifchilddoc
\providecommand{\version}{draft}
\else
\providecommand{\version}{final}
\fi
%    \end{macrocode}

% Load the standard document class:
%    \begin{macrocode}
\documentclass[12pt]{article}
%    \end{macrocode}

% Start the document body:
%    \begin{macrocode}
\begin{document}
%    \end{macrocode}

% Declare a title page.
% Print title, part of document being processed and version flag:
%    \begin{macrocode}
\addtocounter{page}{-1}
\begin{center}
{\LARGE\bfseries{}childdoc example\par}
\vspace{1cm}
\ifchilddoc
\ifchilddocmanual part\else chapter\fi:
`\childdocname' of `\childdocjob'\par
\else
main document: `\childdocjob'\par
\fi
version: \version\par
\end{center}
\newpage
%    \end{macrocode}

% Manually include selected file,
% otherwise process as usual:
%    \begin{macrocode}
\ifchilddocmanual
\section*{part `\childdocname'}
\input{\childdocname}
\else
%    \end{macrocode}

% Include the two chapters:
%    \begin{macrocode}
\include{cdocsch1}
\include{cdocsch2}
%    \end{macrocode}

% Include the two parts unless only chapters should be displayed:
%    \begin{macrocode}
\ifchilddoc\else
\section{part three}
\input{cdocspt3}
\section{part four}
\input{cdocspt4}
\fi
%    \end{macrocode}

% Process as usual until here:
%    \begin{macrocode}
\fi
%    \end{macrocode}

% End of document body:
%    \begin{macrocode}
\end{document}
%    \end{macrocode}
%\iffalse
%</samplemain>
%\fi
%
% %%%%%%%%%%%%%%%%%%%%%%%%%%%%%%%%%%%%%%
% \paragraph{Chapter Include Files.}
%
% The include files are called |cdocsch1.tex| and |cdocsch2.tex|.
%
%\iffalse
%<*samplechap1|samplechap2>
%\fi

% Optional override for |\version| flag:
%    \begin{macrocode}
%%\providecommand{\version}{final}
%    \end{macrocode}

% Include the main document:
%    \begin{macrocode}
\input{childdoc.def}
\childdocof{cdocsamp}
%    \end{macrocode}

%\iffalse
%</samplechap1|samplechap2>
%\fi
%
%\iffalse
%<*samplechap1>
%\fi
% Some text for chapter 1:
%    \begin{macrocode}
\section{one}
some text in chapter one
%    \end{macrocode}

%\iffalse
%</samplechap1>
%\fi
% Some text for chapter 2:
%\iffalse
%<*samplechap2>
%\fi
%    \begin{macrocode}
\section{two}
more text in chapter two
%    \end{macrocode}

%\iffalse
%</samplechap2>
%\fi
%
% %%%%%%%%%%%%%%%%%%%%%%%%%%%%%%%%%%%%%%
% \paragraph{Part Include Files.}
%
% The include files are called |cdocspt3.tex| and |cdocspt4.tex|.
%
%\iffalse
%<*samplepart3|samplepart4>
%\fi

% Optional override for |\version| flag:
%    \begin{macrocode}
%%\providecommand{\version}{final}
%    \end{macrocode}

% Include the main document:
%    \begin{macrocode}
\input{childdoc.def}
\childdocby{cdocsamp}
%    \end{macrocode}

%\iffalse
%</samplepart3|samplepart4>
%\fi
%
%\iffalse
%<*samplepart3>
%\fi
% Some text for part 3:
%    \begin{macrocode}
some text in part three
%    \end{macrocode}

%\iffalse
%</samplepart3>
%\fi
% Some text for part 4:
%\iffalse
%<*samplepart4>
%\fi
%    \begin{macrocode}
more text in part four
%    \end{macrocode}

%\iffalse
%</samplepart4>
%\fi
%
% %%%%%%%%%%%%%%%%%%%%%%%%%%%%%%%%%%%%%%
% \paragraph{Forwarding for a Complete Draft.}
%
% The following forwarding file |cdocsdrf.tex|
% compiles the main document in draft mode:
%\iffalse
%<*sampledraft>
%\fi
%    \begin{macrocode}
\def\version{draft}
\input{childdoc.def}
\childdocforward{cdocsamp}
%    \end{macrocode}

%\iffalse
%</sampledraft>
%\fi
%
% %%%%%%%%%%%%%%%%%%%%%%%%%%%%%%%%%%%%%%
% \paragraph{Forwarding for Final Version of the Chapters.}
%
% The following forwarding files |cdocsfn1.tex| and |cdocsfn2.tex|
% (with identical content)
% compile the final versions of the child documents
% |cdocsch1.tex| and |cdocsch2.tex|, respectively:
%\iffalse
%<*samplefinal>
%\fi
%    \begin{macrocode}
\def\version{final}
\input{childdoc.def}
\childdocforwardprefix[cdocsamp]{cdocsfn}{cdocsch}
%    \end{macrocode}

%\iffalse
%</samplefinal>
%\fi
%
% %%%%%%%%%%%%%%%%%%%%%%%%%%%%%%%%%%%%%%
% \paragraph{Command Line Processing.}
%
% The following three command lines generate the output files
% |cdocscld|, |cdocscl1| and |cdocscl2|
% which should be identical to
% |cdocsdrf|, |cdocsch1| and |cdocsfn2|, respectively:
% \begin{center}
% \begin{tabular}{l}
% |latex -jobname cdocscld \|\\
% |  "\def\version{draft}\input{childdoc.def}\childdocforward{cdocsamp}"|\\
% |latex -jobname cdocscl1 \|\\
% |  "\input{childdoc.def}\childdocforward[cdocsamp]{cdocsch1}"|\\
% |latex -jobname cdocscl2 \|\\
% |  "\def\version{final}\input{childdoc.def}\childdocforward{cdocsch2}"|
% \end{tabular}
% \end{center}
% Note that the trailing backslash on each first line
% merely continues the input to the second line
% (for convenient cut ant paste).
% Furthermore, the command |latex| can be replaced by any
% of its alternative versions such as |pdflatex|.
%
% %%%%%%%%%%%%%%%%%%%%%%%%%%%%%%%%%%%%%%%%%%%%%%%%%%%%%%%%%%%%%%%%%%%%%%%%%%%%%%
% %%%%%%%%%%%%%%%%%%%%%%%%%%%%%%%%%%%%%%%%%%%%%%%%%%%%%%%%%%%%%%%%%%%%%%%%%%%%%%
% \section{Implementation}
%\iffalse
%<*package>
%\fi
%
% This section describes the definitions file |childdoc.def|.

% The definitions cannot be loaded using |\usepackage| or |\RequirePackage|
% which has a mechanism to prevent loading a style file more than once.
% When loading the definitions by means of |\input|
% multiple instances have to be prevented manually:
%\iffalse
%This code needs to be before the `\ProvidesFile' directive
%which is defined at the beginning of this file.
%Therefore it is also placed there and commented out here.
%</package>
%<*discard>
%\fi
%    \begin{macrocode}
\ifdefined\childdocmain\endinput\fi
%    \end{macrocode}
%\iffalse
%</discard>
%<*package>
%\fi
%
% \macro{\ifchilddoc}
% \macro{\ifchilddocmanual}
% The conditional |\ifchilddoc| tells whether a
% child (true) or main (false) document is being compiled.
% The conditional |\ifchilddocmanual| tells whether
% the |\includeonly| mechanism is used (false) or
% the selection of child files must be performed manually (true).
% The definitions initialise to false:
%    \begin{macrocode}
\newif\ifchilddoc
\newif\ifchilddocmanual
%    \end{macrocode}

% \macro{\childdocname}
% \macro{\childdocjob}
% The macro |\childdocname| stores the name of the main document
% to be compiled. The macro |\childdocjob| stores the name of
% the document on which the \LaTeX{} compiler was originally invoked.
% The content of |\jobname| cannot be compared
% to filenames specified in the source due to different catcodes.
% The following code rescans |\jobname|, stores the result
% in |\childdocname| and saves a copy in |\childdocjob|:
%    \begin{macrocode}
\edef\childdocname{\scantokens\expandafter{\jobname\noexpand}}
\let\childdocjob\childdocname
%    \end{macrocode}

% \macro{\childdocdisable}
% The macro |\childdocdisable| prevents the main file
% from being processed more than once.
% At this stage, the main document command |\childdocmain|
% is assumed to be called once again where it should do nothing.
% Any subsequent call to it should prevent
% a secondary processing of the main document
% It overwrites the forwarding commands
% |\childdocof| and |\childdocforward|
% with empty macros to prevent further inclusions of the main document:
%    \begin{macrocode}
\newcommand{\childdocdisable}
{
  \renewcommand{\childdocmain}[1]{\renewcommand{\childdocmain}[1]{\endinput}}
  \renewcommand{\childdocof}[1]{}
  \renewcommand{\childdocby}[2][]{}
  \renewcommand{\childdocforward}[2][]{}
  \renewcommand{\childdocdisable}{}
}
%    \end{macrocode}

% \macro{\childdocmain}
% The macro |\childdocmain| is to be called at the top of the main file
% with nothing or the main filename (without extension) as argument.
% First, it breaks loops.
% If the argument is not empty and does not match |\childdocname|
% (which is set by the first inclusion of |childdoc.def|),
% |\ifchilddoc| is set to true, |\includeonly| is applied to the child file
% and |\jobname| is set to the main file
% (for proper handling of |.aux| files):
%    \begin{macrocode}
\newcommand{\childdocmain}[1]
{
  \childdocdisable\childdocmain{}
  \if?#1?\else
    \begingroup
      \def\childdoctmp{#1}
      \ifx\childdoctmp\childdocname
        \def\childdoctmp{}
      \else
        \def\childdoctmp
        {
          \childdoctrue
          \includeonly{\childdocname}
          \def\childdocjob{#1}
          \def\jobname{#1}
        }
      \fi
      \expandafter
    \endgroup
    \childdoctmp
  \fi
}
%    \end{macrocode}

% \macro{\childdocof}
% The command |\childdocof| redirects
% compilation to the main file |#1|.
%    \begin{macrocode}
\newcommand{\childdocof}[1]
{
  \childdocdisable
  \childdoctrue
  \includeonly{\childdocname}
  \def\jobname{#1}
  \def\childdocjob{#1}
  \input{#1}
}
%    \end{macrocode}

% \macro{\childdocby}
% The command |\childdocby| ....
%    \begin{macrocode}
\newcommand{\childdocby}[2][]
{
  \childdocdisable
  \childdoctrue
  \childdocmanualtrue
  \if?#1?\else
    \def\jobname{#2}
  \fi
  \def\childdocjob{#2}
  \input{#2}
  \endinput
}
%    \end{macrocode}

% \macro{\childdocforward}
% The command |\childdocforward| redirects
% compilation to the main file or
% (if the optional argument is given) a child file.
% Parameters are set as if the main file
% or a child file starting with |\childdocof| was compiled.
% Then compilation is handed over to the main file:
%    \begin{macrocode}
\newcommand{\childdocforward}[2][]
{
  \begingroup
    \if?#1?
      \def\childdoctmp
      {
        \def\childdocname{#2}
        \def\childdocjob{#2}
        \def\jobname{#2}
        \input{#2}
        \endinput
      }
    \else
      \def\childdoctmp
      {
        \childdocdisable
        \def\childdocname{#2}
        \childdoctrue
        \includeonly{#2}
        \def\childdocjob{#1}
        \def\jobname{#1}
        \input{#1}
        \endinput
      }
    \fi
    \expandafter
  \endgroup
  \childdoctmp
}
%    \end{macrocode}

% \macro{\childdocforwardprefix}
% The command |\childdocforwardprefix| redirects
% compilation to the main or a child file by means of a pattern.
% The prefix |#1| in the current filename is replaced by |#2|
% and the suffix of the current filename is kept
% (it is assumed that the filename does not contain the substring `|~~~|'
% which is used as a delimiter).
% Compilation is handed over to the new file by |\childdocforward|:
%    \begin{macrocode}
\newcommand{\childdocforwardprefix}[3][]
{
  \begingroup
    \def\childdocextract #2##1~~~{\def\childdoctmp{\childdocforward[#1]{#3##1}}}
    \expandafter\childdocextract\childdocname~~~
    \expandafter
  \endgroup
  \childdoctmp
}
%    \end{macrocode}

% \macro{\childdoc}
% The deprecated macro |\childdoc| is a legacy version of |\childdocmain|:
%    \begin{macrocode}
\newcommand{\childdoc}{\childdocmain}
%    \end{macrocode}

% \macro{\childdocredirect}
% The deprecated macro |\childdocredirect| is a legacy version
% of |\childdocforward| and |\childdocforwardprefix|:
%    \begin{macrocode}
\newcommand{\childdocredirect}[2][]
{
  \begingroup
    \if?#1?
      \def\childdoctmp{\childdocforward{#2}}
    \else
      \def\childdoctmp{\childdocforwardprefix{#1}{#2}}
    \fi
    \expandafter
  \endgroup
  \childdoctmp
}
%    \end{macrocode}

%\iffalse
%</package>
%\fi
%
\endinput
|\\
|\childdocby{|\textit{main}|}|\\
\end{tabular}
\end{center}
%
Both forms have slightly different effects as described above.
The main file is prepared as usual, see \secref{sec:include}.

%%%%%%%%%%%%%%%%%%%%%%%%%%%%%%%%%%%%%%%%%%%%%%%%%%%%%%%%%%%%%%%%%%%%%%%%%%%%%%%%
\subsection{Legacy Detection}
\label{sec:detection}

The directive |\childdocmain| in the main file can detect
whether the complete document or merely a child is to be compiled
even without using the directive |\childdocof|.
This method is deprecated because it is less robust
and there is no compelling reason to use it;
it is merely provided for backward compatibility
and it may be removed in future versions.

If the detection mechanism is to be used,
it is mandatory to correctly specify
the filename of the main file as the argument of |\childdocmain|:
%
\begin{center}
\begin{tabular}{l}
|% \iffalse
%
% childdoc.dtx Copyright (C) 2017-2018 Niklas Beisert
%
% This work may be distributed and/or modified under the
% conditions of the LaTeX Project Public License, either version 1.3
% of this license or (at your option) any later version.
% The latest version of this license is in
%   http://www.latex-project.org/lppl.txt
% and version 1.3 or later is part of all distributions of LaTeX
% version 2005/12/01 or later.
%
% This work has the LPPL maintenance status `maintained'.
%
% The Current Maintainer of this work is Niklas Beisert.
%
% This work consists of the files childdoc.dtx and childdoc.ins
% and the derived files childdoc.def and cdocsamp.tex with
% cdocsch1.tex, cdocsch2.tex, cdocsdrf.tex, cdocsfn1.tex, cdocsfn2.tex.
%
%<package>\ifdefined\childdocmain\endinput\fi
%<package>\ProvidesFile{childdoc.def}[2018/12/30 v2.0 child document driver]
%<samplemain>\ProvidesFile{cdocsamp.tex}[2018/12/30 v2.0 sample for childdoc]
%<*driver>
%\ProvidesFile{childdoc.drv}[2018/12/30 v2.0 childdoc reference manual file]
\PassOptionsToClass{10pt,a4paper}{article}
\documentclass{ltxdoc}

\usepackage[margin=35mm]{geometry}
\usepackage{hyperref}
\usepackage{hyperxmp}
\usepackage[usenames]{color}

\hypersetup{colorlinks=true}
\hypersetup{pdfstartview=FitH}
\hypersetup{pdfpagemode=UseNone}
\hypersetup{pdfsource={}}
\hypersetup{pdflang={en-UK}}
\hypersetup{pdfcopyright={Copyright 2017-2018 Niklas Beisert.
  This work may be distributed and/or modified under the
  conditions of the LaTeX Project Public License, either version 1.3
  of this license or (at your option) any later version.}}
\hypersetup{pdflicenseurl={http://www.latex-project.org/lppl.txt}}
\hypersetup{pdfcontactaddress={ETH Zurich, ITP, HIT K,
  Wolfgang-Pauli-Strasse 27}}
\hypersetup{pdfcontactpostcode={8093}}
\hypersetup{pdfcontactcity={Zurich}}
\hypersetup{pdfcontactcountry={Switzerland}}
\hypersetup{pdfcontactemail={nbeisert@itp.phys.ethz.ch}}
\hypersetup{pdfcontacturl={http://people.phys.ethz.ch/\xmptilde nbeisert/}}

\newcommand{\secref}[1]{\hyperref[#1]{section \ref*{#1}}}

\parskip1ex
\parindent0pt
\let\olditemize\itemize
\def\itemize{\olditemize\parskip0pt}

\begin{document}

\title{The \textsf{childdoc} Package}
\hypersetup{pdftitle={The childdoc Package}}
\author{Niklas Beisert\\[2ex]
  Institut f\"ur Theoretische Physik\\
  Eidgen\"ossische Technische Hochschule Z\"urich\\
  Wolfgang-Pauli-Strasse 27, 8093 Z\"urich, Switzerland\\[1ex]
  \href{mailto:nbeisert@itp.phys.ethz.ch}
  {\texttt{nbeisert@itp.phys.ethz.ch}}}
\hypersetup{pdfauthor={Niklas Beisert}}
\hypersetup{pdfsubject={Manual for the LaTeX2e Package childdoc}}
\date{30 December 2018, \textsf{v2.0}}
\maketitle

\begin{abstract}\noindent
\textsf{childdoc} is a \LaTeXe{} package
that enables the direct compilation
of document sections included by |\include|
to individual files.
\end{abstract}

\begingroup
\parskip0ex
\tableofcontents
\endgroup

%%%%%%%%%%%%%%%%%%%%%%%%%%%%%%%%%%%%%%%%%%%%%%%%%%%%%%%%%%%%%%%%%%%%%%%%%%%%%%%%
%%%%%%%%%%%%%%%%%%%%%%%%%%%%%%%%%%%%%%%%%%%%%%%%%%%%%%%%%%%%%%%%%%%%%%%%%%%%%%%%
\section{Introduction}

\LaTeX{} provides a mechanism to structure a large document (such as a book)
into a main file and several child files (containing the chapters)
using the |\include| command.
This mechanism is beneficial for documents
which span hundreds of pages in order to
make the source file(s) more manageable.
Moreover, compilation can be restricted to
selected child files by means of the |\includeonly| command.
The latter feature can be used to reduce the compilation time while editing
(this was significantly more useful in the earlier days of \LaTeX{})
or to generate a smaller document which is easier to navigate.
Another application of |\includeonly| is to generate
documents consisting of selected parts of the complete document.

However, there are a few drawbacks of the plain |\include| mechanism:
\begin{itemize}
\item
The child files cannot be compiled on their own,
they can only be compiled via the main file.
A naive editing environment
(such as a text editor with an option
to have the current file processed by \LaTeX)
may require one to switch to the main file before compiling;
attempting to compile the child file produces errors.
\item
The main file must be modified (each time)
to adjust the |\includeonly| command
to the present needs. This easily leaves the main file in a messy state.
\item
The generated document will always carry the filename
of the main document. This is inconvenient if
several child files are to be compiled and
to be kept for distribution.
\end{itemize}

The present package provides a simple interface
to make child files individually compilable by \LaTeX{}.
Compiling a child file then has the same effect as compiling
the main file with an |\includeonly| command
to select the appropriate child.
Moreover the generated document will carry the name of the child
rather than the main file.
This resolves all three above issues.

This feature is meant to make the editing of books,
thesis documents and lecture notes somewhat more convenient.
However, the package can also be used efficiently for
composing a series of documents (such as exercise sheets)
which are typically distributed individually.
It then assists the author in generating the individual documents
(potentially in different versions)
as well as a document containing the collected series.
Another application is in developing style files
or other kinds of included material
where compilation of the style file could redirect
to a sample or test file.

%%%%%%%%%%%%%%%%%%%%%%%%%%%%%%%%%%%%%%%%%%%%%%%%%%%%%%%%%%%%%%%%%%%%%%%%%%%%%%%%
%%%%%%%%%%%%%%%%%%%%%%%%%%%%%%%%%%%%%%%%%%%%%%%%%%%%%%%%%%%%%%%%%%%%%%%%%%%%%%%%
\section{Usage}

First of all, the package \textsf{childdoc} is \emph{not} a standard
\LaTeXe{} |.sty| style file! Therefore it needs to be invoked in
a non-standard way.

%%%%%%%%%%%%%%%%%%%%%%%%%%%%%%%%%%%%%%%%%%%%%%%%%%%%%%%%%%%%%%%%%%%%%%%%%%%%%%%%
\subsection{Included Files}
\label{sec:include}

%%%%%%%%%%%%%%%%%%%%%%%%%%%%%%%%%%%%%%%%
\DescribeMacro{\childdocmain}
To use the package, add the commands
\begin{center}
\begin{tabular}{l}
|\input{childdoc.def}|\\
|\childdocmain{}|\\
\end{tabular}
\end{center}
at the very top of the main \LaTeX{} file,
in particular \emph{before} the |\documentclass| statement!
The argument of |\childdocmain| should be left empty
(but it must be present).

%%%%%%%%%%%%%%%%%%%%%%%%%%%%%%%%%%%%%%%%
\DescribeMacro{\childdocof}
Furthermore, add the commands
\begin{center}
\begin{tabular}{l}
|\input{childdoc.def}|\\
|\childdocof{|\textit{main}|}|\\
\end{tabular}
\end{center}
at the top of every child file \textit{child}
which is included by |\include{|\textit{child}|}|
from within the main file
(or at least for those files to be compiled individually).
The argument \textit{main} must be the filename of the main file.

There are a couple of
considerations in setting up the main and child documents:

%%%%%%%%%%%%%%%%%%%%%%%%%%%%%%%%%%%%%%%%
\paragraph{Restrictions.}

Please note the following restrictions:
\begin{itemize}
\item
|\childdocmain| must be called with one argument \textit{main}
to ensure compatibility with earlier version of the package.
It must either be empty (|\childdocmain{}|)
or precisely match the filename of the main file in which it is specified.
See \secref{sec:detection} for further information.
\item
The filename \textit{main} must be specified without the |.tex| extension.
\item
The filename \textit{main} is case sensitive
(even in case-insensitive file systems)
due to internal string comparison.
\item
The argument \textit{main} should be fully expanded, it cannot be a macro.
\item
Subdirectories and special characters should be avoided in filenames.
\item
The command |\childdocmain{|\textit{main}|}| must be followed by a whitespace.
It should not be followed immediately by another command
or by a comment mark `|%|'.
This is because the \TeX{} parser reads the token immediately following
the argument of |\childdocmain| and puts it
at the beginning of every child section;
however, a white\-space is ignored.
\end{itemize}

%%%%%%%%%%%%%%%%%%%%%%%%%%%%%%%%%%%%%%%%
\paragraph{Content of Main File.}

It is advisable to place all content in the child files included by |\include|.
Any output contained in the main file will appear in all child documents
unless suppressed manually;
it cannot be suppressed automatically by the |\includeonly| directive
and thus should normally be avoided.
A method to include some content in the main file
by means of conditional processing is described in \secref{sec:conditional}.

%%%%%%%%%%%%%%%%%%%%%%%%%%%%%%%%%%%%%%%%
\paragraph{Page Numbering.}

When only a part of the document is compiled,
the appropriate numbering of pages
(as well as other status parameters)
is determined from the |.aux| files.
The latter contain information from previous passes.
However this information needs to propagate through
all intermediate child documents.
Therefore the page numbering in child documents may well
be inconsistent until the complete document is compiled at least once.

A useful (if unconventional) way to always ensure a consistent
page numbering is to restart the numbering in each child document
and denote the pages by `\textit{child}|.|\textit{page}'
where \textit{child} represents the chapter/section number of the child file.
This can be achieved by the command
|\numberwithin{page}{|\textit{child}|}|
of the \textsf{amsmath} package
where \textit{child} can be |chapter| or |section|
depending on the chosen structuring.
Alternatively, one can modify the macro |\thepage| appropriately
and reset the counter |page| at the start of each child file.

%%%%%%%%%%%%%%%%%%%%%%%%%%%%%%%%%%%%%%%%%%%%%%%%%%%%%%%%%%%%%%%%%%%%%%%%%%%%%%%%
\subsection{Conditional Processing}
\label{sec:conditional}

The package provides a mechanism to compile different versions
of a document. To customise the versions further some conditional processing
can come in handy to distinguish which version is being compiled.
The package provides two macros to describe the compilation context:

%%%%%%%%%%%%%%%%%%%%%%%%%%%%%%%%%%%%%%%%
\DescribeMacro{\ifchilddoc}
The conditional |\ifchilddoc| distinguishes between the compilation of
child documents and the main document:
%
\begin{center}
|\ifchilddoc |\textit{child-code}| |[|\||else |\textit{main-code}]| \||fi|
\end{center}

%%%%%%%%%%%%%%%%%%%%%%%%%%%%%%%%%%%%%%%%
\DescribeMacro{\childdocname}
\DescribeMacro{\childdocjob}
The macro |\childdocname| contains the filename (without extension)
of the main or child file being processed.
Note that |\childdocjob| will always contain the name of the main file.

%%%%%%%%%%%%%%%%%%%%%%%%%%%%%%%%%%%%%%%%
\paragraph{Title Page.}

Conditional processing can be used to include a title or banner page
in the main document when proper precautions are taken.
Importantly, the code in the main file should ensure that the page counter
(as well as other status parameters which are stored in the |.aux| files)
takes the same value after the conditional processing.
Otherwise the page numbers may take divergent values
depending on which part is compiled.

For example, a title page could be declared by:
%
\begin{center}
\begin{tabular}{l}
|\ifchilddoc\||else|\\
|\addtocounter{page}{-1}|\\
\textit{code for title page}\\
|\newpage|\\
|\||fi|
\end{tabular}
\end{center}
%
A banner page for the child documents can be generated by:
%
\begin{center}
\begin{tabular}{l}
|\ifchilddoc|\\
|\addtocounter{page}{-1}|\\
\textit{code for banner page}\\
|\newpage|\\
|\||fi|
\end{tabular}
\end{center}
%
Here one could write a message such as:
\begin{center}
|This is the part \childdocname{} of \childdocjob{}.|
\end{center}

%%%%%%%%%%%%%%%%%%%%%%%%%%%%%%%%%%%%%%%%%%%%%%%%%%%%%%%%%%%%%%%%%%%%%%%%%%%%%%%%
\subsection{Flags}
\label{sec:flags}

The package makes it easy to generate different versions
of the main or child documents.
To this end compilation flags can be defined
and assigned different default values.
They will be particularly useful in conjunction
with the forwarding mechanism described in \secref{sec:forward}.

For example, it may be useful to have a flag |\version|
which can be set to |draft| or |final|.
The document source will contain some conditional code
depending on the value of |\version|.
Suppose further, the flag should default to |final| for the main file
and to |draft| for child files
which is a natural assignment for editing the document.
This is achieved by placing the following code
in the preamble of the main document
(below the |\childdocmain| directive):
%
\begin{center}
\begin{tabular}{l}
|\ifchilddoc|\\
|\providecommand{\version}{draft}|\\
|\||else|\\
|\providecommand{\version}{final}|\\
|\||fi|
\end{tabular}
\end{center}
%
The definition by |\providecommand| makes sure
that previous definitions are not overwritten.
Further statements |\providecommand{\version}{...}|
can thus be added before the above code to override it.

For the main file, one might add a line
(between |\childdocmain| and the above block)
%
\begin{center}
|%\ifchilddoc\||else\providecommand{\version}{draft}\||fi|
\end{center}
%
which can be uncommented to produce a draft version.
Likewise one can add a line to the very top of a child file
(above the |\childdocof{|\textit{main}|}| directive)
%
\begin{center}
|%\providecommand{\version}{final}|
\end{center}
%
which can be uncommented to produce the final version of this child document.

%%%%%%%%%%%%%%%%%%%%%%%%%%%%%%%%%%%%%%%%%%%%%%%%%%%%%%%%%%%%%%%%%%%%%%%%%%%%%%%%
\subsection{Forwarding}
\label{sec:forward}

Different versions of the main or child documents
using compilation flags as described in \secref{sec:flags}
can be (permanently) stored in different files
for convenient compilation, viewing and distribution.
To this end, the package defines a command
to pass on compilation to a different file:

%%%%%%%%%%%%%%%%%%%%%%%%%%%%%%%%%%%%%%%%
\DescribeMacro{\childdocforward}
The command |\childdocforward| redirects processing to
another source file:
%
\begin{center}
\begin{tabular}{l}
|\input{childdoc.def}|\\
|\childdocforward[|\textit{main}|]{|\textit{dest}|}|\\
\end{tabular}
\end{center}
%
The argument \textit{dest} is the destination file
(without extension).
It should be the main file or one of the child files.
Note that further \textsf{childdoc} directives
such as |\childdocof| and |\childdocforward|
in the indicated file will be processed in this form.
The optional argument \textit{main}
passes on directly to the main file \textit{main}
while pretending to compile the child \textit{dest}.
This form behaves as if \textit{dest}
issues |\childdocof{|\textit{main}|}| right away,
and no further \textsf{childdoc} directives will be processed.

%%%%%%%%%%%%%%%%%%%%%%%%%%%%%%%%%%%%%%%%
\DescribeMacro{\...prefix}
In the alternative form |\childdocforwardprefix|,
%
\begin{center}
\begin{tabular}{l}
|\input{childdoc.def}|\\
|\childdocforwardprefix[|\textit{main}|]{|\textit{prefix}|}{|\textit{dest}|}|
\end{tabular}
\end{center}
%
the destination file is determined by a pattern
depending on the current file:
To make this work, the current file must be called
`{\textit{prefix}\hspace{0.2em}\textit{suffix}}'
with \textit{prefix} matching precisely the argument.
Processing is then passed on to the file
`{\textit{dest}\hspace{0.2em}\textit{suffix}}'.
Surely, the same effect is achieved by
directly specifying the
argument `{\textit{dest}\hspace{0.2em}\textit{suffix}}'
in the first form.
However, that requires to set up a different file
for each child. With the alternative form of the command
all these files can have exactly the same content
which simplifies setting them up and maintaining them.

For example, the following file |draft.tex|
with a compilation flag |\version| as described in \secref{sec:flags}
compiles the main document as a draft:
%
\begin{center}
\begin{tabular}{l}
|\def\version{draft}|\\
|\input{childdoc.def}|\\
|\childdocforward{|\textit{main}|}|
\end{tabular}
\end{center}
%
Likewise, the following files |final|\textit{nn}|.tex|
compile the final version of the child document
|child|\textit{nn}|.tex|:
%
\begin{center}
\begin{tabular}{l}
|\def\version{final}|\\
|\input{childdoc.def}|\\
|\childdocforwardprefix{final}{child}|
\end{tabular}
\end{center}
%

Note that when several versions of a main file and/or of each child file
are to be generated, it may be convenient to set up a |Makefile| or
shell script to automatise the process.

%%%%%%%%%%%%%%%%%%%%%%%%%%%%%%%%%%%%%%%%%%%%%%%%%%%%%%%%%%%%%%%%%%%%%%%%%%%%%%%%
\subsection{Command Line Processing}
\label{sec:commandline}

The effect of redirection files can also be achieved by invoking
the \LaTeX{} compiler with a more elaborate command line.
Most conveniently this should be done as part
of a shell script or a |Makefile|.

When using \textsf{childdoc} in the main file, the following
command lines effectively perform a redirection
(note that depending on the shell being used,
backslashes may have to be doubled: `|\|' $\to$ `|\\|'):
%
\begin{center}
|... -jobname "|\textit{target}|" |\\|"|[\textit{flags}]%
|\input{childdoc.def}\childdocforward[|\textit{main}|]{|\textit{dest}|}"|
\end{center}
%
Here \textit{target} is the name of the output file,
\textit{main} is the name of the main file
and \textit{dest} is the name of the main or child file to be processed
(all filenames without extensions).
The optional argument \textit{main} can be omitted
if \textit{main} matches \textit{dest}.
Optionally, compilation \textit{flags} can be defined via |\def| commands.
This command line makes the \TeX{} engine believe
it is compiling the file \textit{target}
whose content is specified as the latter parameter.
The provided code then forwards the processing to
\textit{main} or \textit{dest} as described in \secref{sec:forward}.

%%%%%%%%%%%%%%%%%%%%%%%%%%%%%%%%%%%%%%%%%%%%%%%%%%%%%%%%%%%%%%%%%%%%%%%%%%%%%%%%
\subsection{Include by Input}
\label{sec:input}

Including child documents by |\include| has some restrictions by design.
Most notably, the content of a child document always occupies
its own set of pages; pages cannot be shared between child documents.
Usually, this behaviour makes perfect sense
because each child document contain an essential part of the document.
However, in some situations it may be desirable to compose
a document from a collection of parts
without having mandatory page breaks between then.
For this case, the package
provides a mechanism to include parts
by |\input| which can also be processed individually.
However, by construction this mechanism
requires manual handling of the content to be output.

%%%%%%%%%%%%%%%%%%%%%%%%%%%%%%%%%%%%%%%%
\DescribeMacro{\ifchilddocmanual}
The main file should be prepared as usual, see \secref{sec:include}.
However, the document body must make a distinction
between processing of an individual part and of the main document, e.g.:
%
\begin{center}
\begin{tabular}{l}
|\ifchilddocmanual|\\
|\input{\childdocname}|\\
|\||else|\\
\textit{document body with }|\input{|\textit{part}|}|\\
|\||fi|
\end{tabular}
\end{center}
%
The conditional |\ifchilddocmanual| is true whenever
a part to be included by |\input| is being compiled,
and the name of the part is stored in |\childdocname|.

%%%%%%%%%%%%%%%%%%%%%%%%%%%%%%%%%%%%%%%%
\DescribeMacro{\childdocby}
Each part to be included by |\input| should start with:
%
\begin{center}
\begin{tabular}{l}
|\input{childdoc.def}|\\
|\childdocby{|\textit{main}|}|\\
\end{tabular}
\end{center}
%
The directive |\childdocby| is similar to |\childdocof|
described in \secref{sec:include},
but the subsequent selection of content must be done manually.
To that end, both |\ifchilddoc| and |\ifchilddocmanual|
will be true upon processing of a part,
and the name of the part is stored in |\childdocname|.
Note that |\jobname| will be set to the filename of the current part
so that each part receives an individual |.aux| file
that does not interfere with the |.aux| file(s) of the main document.
This behaviour can be altered by the alternative form
|\childdocby[*]{|\textit{main}|}| (with a non-empty optional argument)
which uses the |.aux| file of the main document
by setting |\jobname| to \textit{main}.

%%%%%%%%%%%%%%%%%%%%%%%%%%%%%%%%%%%%%%%%%%%%%%%%%%%%%%%%%%%%%%%%%%%%%%%%%%%%%%%%
\subsection{Driver Development}
\label{sec:driver}

The \textsf{childdoc} mechanism can also be use for the development
of definition files such as \LaTeX{} styles or classes.
This case differs from the above setup with multiple parts
included by |\include| in that no |\includeonly| should be invoked.
This can be achieved by starting the include file
(before |\ProvidesPackage|) with:
%
\begin{center}
\begin{tabular}{l}
|\input{childdoc.def}|\\
|\childdocforward{|\textit{main}|}|\\
\end{tabular}
\end{center}
%
or alternatively with:
%
\begin{center}
\begin{tabular}{l}
|\input{childdoc.def}|\\
|\childdocby{|\textit{main}|}|\\
\end{tabular}
\end{center}
%
Both forms have slightly different effects as described above.
The main file is prepared as usual, see \secref{sec:include}.

%%%%%%%%%%%%%%%%%%%%%%%%%%%%%%%%%%%%%%%%%%%%%%%%%%%%%%%%%%%%%%%%%%%%%%%%%%%%%%%%
\subsection{Legacy Detection}
\label{sec:detection}

The directive |\childdocmain| in the main file can detect
whether the complete document or merely a child is to be compiled
even without using the directive |\childdocof|.
This method is deprecated because it is less robust
and there is no compelling reason to use it;
it is merely provided for backward compatibility
and it may be removed in future versions.

If the detection mechanism is to be used,
it is mandatory to correctly specify
the filename of the main file as the argument of |\childdocmain|:
%
\begin{center}
\begin{tabular}{l}
|\input{childdoc.def}|\\
|\childdocmain{|\textit{main}|}|\\
\end{tabular}
\end{center}
%
If |\jobname| does not match the argument \textit{main} of |\childdocmain|,
it is assumed that |\jobname| points to the child file to be compiled.
When using |\childdocmain| with the main file specified as argument,
it suffices to start a child file
with just |\input{|\textit{main}|}|
without loading of the package and using |\childdocof|.
If instead all processing is done
with the appropriate \textsf{childdoc} directives,
the argument of \textit{main} of |\childdocmain| can be empty.

An alternative version of the command line processing described
in \secref{sec:commandline} using the detection mechanism reads:
%
\begin{center}
|... -jobname "|\textit{target}|" "|[\textit{flags}]%
[|\def\jobname{|\textit{dest}|}|]|\input{|\textit{main}|}"|
\end{center}

%%%%%%%%%%%%%%%%%%%%%%%%%%%%%%%%%%%%%%%%%%%%%%%%%%%%%%%%%%%%%%%%%%%%%%%%%%%%%%%%
\subsection{Manual Code}
\label{sec:manual}

In case one cannot be certain whether the definitions file |childdoc.def|
is installed on the target \TeX{} distribution
and one prefers not to ship it,
it is conceivable to paste a few relevant commands into the sources.

To that end, drop all statements |\input{childdoc.def}|
and perform the replacements as outlined below.
Instead of |\childdocmain{|\textit{main}|}| add the following code
to the top of the main file:
%
\begin{center}
\begin{tabular}{l}
|\||ifdefined\childdocname\endinput\||fi\newif\ifchilddoc|\\
|\edef\childdocname{\scantokens\expandafter{\jobname\noexpand}}|\\
|\def\childdocmain{|\textit{main}|}\||ifx\childdocmain\childdocname\||else|\\
|\childdoctrue\includeonly{\childdocname}\let\jobname\childdocmain\||fi|\\
\end{tabular}
\end{center}
%
Instead of |\childdocof{|\textit{main}|}| just include the main file
at the top of each child file:
%
\begin{center}
|\input{|\textit{main}|}|
\end{center}
%
A simple redirection |\childdocforward{|\textit{dest}|}| is achieved by:
%
\begin{center}
|\def\jobname{|\textit{dest}|}\input{\jobname}|
\end{center}
%
The redirection with prefix
|\childdocforwardprefix[|\textit{prefix}|]{|\textit{dest}|}|
is accomplished by:
%
\begin{center}
\begin{tabular}{l}
|{\edef\jobname{\scantokens\expandafter{\jobname\noexpand}}|\\
|\def\redirectjob |\textit{prefix}|#1~~~{\gdef\jobname{|\textit{dest}|#1}}|\\
|\expandafter\redirectjob\jobname~~~}\input{\jobname}|
\end{tabular}
\end{center}

In an alternative approach,
child documents can be compiled by a specific command line
without additional code or specific definitions:
%
\begin{center}
|... -jobname "|\textit{target}|" "|[\textit{flags}]%
|\includeonly{|\textit{dest}|}\input{|\textit{main}|}"|
\end{center}
%

%%%%%%%%%%%%%%%%%%%%%%%%%%%%%%%%%%%%%%%%%%%%%%%%%%%%%%%%%%%%%%%%%%%%%%%%%%%%%%%%
%%%%%%%%%%%%%%%%%%%%%%%%%%%%%%%%%%%%%%%%%%%%%%%%%%%%%%%%%%%%%%%%%%%%%%%%%%%%%%%%
\section{Information}

%%%%%%%%%%%%%%%%%%%%%%%%%%%%%%%%%%%%%%%%%%%%%%%%%%%%%%%%%%%%%%%%%%%%%%%%%%%%%%%%
\subsection{Copyright}

Copyright \copyright{} 2017--2018 Niklas Beisert

This work may be distributed and/or modified under the
conditions of the \LaTeX{} Project Public License, either version 1.3
of this license or (at your option) any later version.
The latest version of this license is in
  \url{http://www.latex-project.org/lppl.txt}
and version 1.3 or later is part of all distributions of \LaTeX{}
version 2005/12/01 or later.

This work has the LPPL maintenance status `maintained'.

The Current Maintainer of this work is Niklas Beisert.

This work consists of the files |README.txt|, |childdoc.ins| and |childdoc.dtx|
as well as the derived files |childdoc.def|, |cdocsamp.tex|
with |cdocsch1.tex|, |cdocsch2.tex|, |cdocspt3.tex|, |cdocspt4.tex|,
|cdocsdrf.tex|, |cdocsfn1.tex|, |cdocsfn2.tex|
as well as |childdoc.pdf|.

%%%%%%%%%%%%%%%%%%%%%%%%%%%%%%%%%%%%%%%%%%%%%%%%%%%%%%%%%%%%%%%%%%%%%%%%%%%%%%%%
\subsection{Files and Installation}

The package consists of the files:
%
\begin{center}
\begin{tabular}{ll}
    |README.txt|   & readme file \\
    |childdoc.ins| & installation file \\
    |childdoc.dtx| & source file \\
    |childdoc.def| & definition file \\
    |cdocsamp.tex| & sample main file \\
    |cdocsch1.tex| & sample include file \\
    |cdocsch2.tex| & sample include file \\
    |cdocspt3.tex| & sample part file \\
    |cdocspt4.tex| & sample part file \\
    |cdocsdrf.tex| & sample redirection file \\
    |cdocsfn1.tex| & sample redirection file \\
    |cdocsfn2.tex| & sample redirection file \\
    |childdoc.pdf| & manual
\end{tabular}
\end{center}
%
The distribution consists of the files
|README.txt|, |childdoc.ins| and |childdoc.dtx|.
%
\begin{itemize}
\item
Run (pdf)\LaTeX{} on |childdoc.dtx|
to compile the manual |childdoc.pdf| (this file).
\item
Run \LaTeX{} on |childdoc.ins| to create the definitions file |childdoc.def|
and the sample |cdocsamp.tex| with include files
|cdocsch1.tex|, |cdocsch2.tex|, |cdocspt3.tex|, |cdocspt4.tex|,
|cdocsdrf.tex|, |cdocsfn1.tex|, |cdocsfn2.tex|.
Then copy the file |childdoc.def| to an appropriate directory of your \LaTeX{}
distribution, e.g.\ \textit{texmf-root}|/tex/latex/childdoc|.
\end{itemize}

%%%%%%%%%%%%%%%%%%%%%%%%%%%%%%%%%%%%%%%%%%%%%%%%%%%%%%%%%%%%%%%%%%%%%%%%%%%%%%%%
\subsection{Related CTAN Packages}

There are several other packages which offer a similar functionality:
%
\begin{itemize}
\item
The packages
\href{http://ctan.org/pkg/docmute}{\textsf{docmute}},
\href{http://ctan.org/pkg/includex}{\textsf{includex}} and
\href{http://ctan.org/pkg/standalone}{\textsf{standalone}}
provide commands to include only the document body of
a child file thus allowing both files to be compiled individually.
\item
The packages \href{http://ctan.org/pkg/subdocs}{\textsf{subdocs}}
and \href{http://ctan.org/pkg/subfiles}{\textsf{subfiles}}
provide structures in which the main and child documents can be
encapsulated and allowing them to be compiled individually.
The inclusion mechanism is different from the conventional |\include|.
\item
The package \href{http://ctan.org/pkg/combine}{\textsf{combine}}
is an elaborate solution to combine several documents into one.
\end{itemize}
%
See also the CTAN topic \href{http://ctan.org/topic/subdocs}{\textsf{subdocs}}
for further related packages.
The present package differs from the above solutions in that
a document structure constructed with the conventional |\include| mechanism
just needs two extra commands at the top of every file
such that all constituent files can be compiled individually.

%%%%%%%%%%%%%%%%%%%%%%%%%%%%%%%%%%%%%%%%%%%%%%%%%%%%%%%%%%%%%%%%%%%%%%%%%%%%%%%%
%\subsection{Feature Suggestions}
%
%The following is a list of features which may be useful for future
%versions of this package:
%%
%\begin{itemize}
%\item
%\ldots
%\end{itemize}

%%%%%%%%%%%%%%%%%%%%%%%%%%%%%%%%%%%%%%%%%%%%%%%%%%%%%%%%%%%%%%%%%%%%%%%%%%%%%%%%
\subsection{Revision History}

%%%%%%%%%%%%%%%%%%%%%%%%%%%%%%%%%%%%%%%%
\paragraph{v2.0:} 2018/12/30

\begin{itemize}
\item
immediate forward processing
\item
added |\childdocby| mechanism
\item
manual restructured
\end{itemize}

%%%%%%%%%%%%%%%%%%%%%%%%%%%%%%%%%%%%%%%%
\paragraph{v1.6:} 2018/01/17

\begin{itemize}
\item
application for development of include files
\item
corrections to manual
\end{itemize}

%%%%%%%%%%%%%%%%%%%%%%%%%%%%%%%%%%%%%%%%
\paragraph{v1.5:} 2017/05/21

\begin{itemize}
\item
more complete structuring introduced
\item
|\childdocof| introduced
\item
|\childdoc| renamed to |\childdocmain|
\item
|\childredirect| renamed to |\childdocforward| and |\childdocforwardprefix|
and functionality expanded
\end{itemize}

%%%%%%%%%%%%%%%%%%%%%%%%%%%%%%%%%%%%%%%%
\paragraph{v1.0:} 2017/04/27

\begin{itemize}
\item
manual and install package
\item
first version published on CTAN
\end{itemize}

%%%%%%%%%%%%%%%%%%%%%%%%%%%%%%%%%%%%%%%%
\paragraph{v0.6:} 2017/04/26

\begin{itemize}
\item
redirection mechanism added
\end{itemize}

%%%%%%%%%%%%%%%%%%%%%%%%%%%%%%%%%%%%%%%%
\paragraph{v0.5:} 2017/04/26

\begin{itemize}
\item
functionality in definition file
\end{itemize}


%%%%%%%%%%%%%%%%%%%%%%%%%%%%%%%%%%%%%%%%%%%%%%%%%%%%%%%%%%%%%%%%%%%%%%%%%%%%%%%%
%%%%%%%%%%%%%%%%%%%%%%%%%%%%%%%%%%%%%%%%%%%%%%%%%%%%%%%%%%%%%%%%%%%%%%%%%%%%%%%%
%%%%%%%%%%%%%%%%%%%%%%%%%%%%%%%%%%%%%%%%%%%%%%%%%%%%%%%%%%%%%%%%%%%%%%%%%%%%%%%%
\appendix

\settowidth\MacroIndent{\rmfamily\scriptsize 000\ }

 \DocInput{childdoc.dtx}

\end{document}
%</driver>
% \fi
%
% %%%%%%%%%%%%%%%%%%%%%%%%%%%%%%%%%%%%%%%%%%%%%%%%%%%%%%%%%%%%%%%%%%%%%%%%%%%%%%
% %%%%%%%%%%%%%%%%%%%%%%%%%%%%%%%%%%%%%%%%%%%%%%%%%%%%%%%%%%%%%%%%%%%%%%%%%%%%%%
% \section{Sample}
%\iffalse
%<*samplemain>
%\fi
%
% The following presents a sample document
% with two chapters, two parts, a title page,
% a compile flag as well as three forwarding files to set the flag.
% It consists of eight |.tex| files:
% \begin{center}
% \begin{tabular}{ll}
% |cdocsamp.tex|&main file\\
% |cdocsch1.tex|&include file for chapter 1\\
% |cdocsch2.tex|&include file for chapter 2\\
% |cdocspt3.tex|&include file for part 3\\
% |cdocspt4.tex|&include file for part 4\\
% |cdocsdrf.tex|&forwarding file for main file in draft mode\\
% |cdocsfi1.tex|&forwarding file for final version of chapter 1\\
% |cdocsfi2.tex|&forwarding file for final version of chapter 2\\
% \end{tabular}
% \end{center}
% Each of the eight files can be compiled directly by the \LaTeX{} compiler.
%
% %%%%%%%%%%%%%%%%%%%%%%%%%%%%%%%%%%%%%%
% \paragraph{Main File.}
%
% The main file is called |cdocsamp.tex|.
%
% Load the \textsf{childdoc} definitions and
% declare the filename for the main document:
%    \begin{macrocode}
\input{childdoc.def}
\childdocmain{}
%    \end{macrocode}

% Optional override for |\version| flag:
%    \begin{macrocode}
%%\ifchilddoc\else\providecommand{\version}{draft}\fi
%    \end{macrocode}

% Define the default values for the |\version| flag
% (|final| for the main file and |draft| for childs):
%    \begin{macrocode}
\ifchilddoc
\providecommand{\version}{draft}
\else
\providecommand{\version}{final}
\fi
%    \end{macrocode}

% Load the standard document class:
%    \begin{macrocode}
\documentclass[12pt]{article}
%    \end{macrocode}

% Start the document body:
%    \begin{macrocode}
\begin{document}
%    \end{macrocode}

% Declare a title page.
% Print title, part of document being processed and version flag:
%    \begin{macrocode}
\addtocounter{page}{-1}
\begin{center}
{\LARGE\bfseries{}childdoc example\par}
\vspace{1cm}
\ifchilddoc
\ifchilddocmanual part\else chapter\fi:
`\childdocname' of `\childdocjob'\par
\else
main document: `\childdocjob'\par
\fi
version: \version\par
\end{center}
\newpage
%    \end{macrocode}

% Manually include selected file,
% otherwise process as usual:
%    \begin{macrocode}
\ifchilddocmanual
\section*{part `\childdocname'}
\input{\childdocname}
\else
%    \end{macrocode}

% Include the two chapters:
%    \begin{macrocode}
\include{cdocsch1}
\include{cdocsch2}
%    \end{macrocode}

% Include the two parts unless only chapters should be displayed:
%    \begin{macrocode}
\ifchilddoc\else
\section{part three}
\input{cdocspt3}
\section{part four}
\input{cdocspt4}
\fi
%    \end{macrocode}

% Process as usual until here:
%    \begin{macrocode}
\fi
%    \end{macrocode}

% End of document body:
%    \begin{macrocode}
\end{document}
%    \end{macrocode}
%\iffalse
%</samplemain>
%\fi
%
% %%%%%%%%%%%%%%%%%%%%%%%%%%%%%%%%%%%%%%
% \paragraph{Chapter Include Files.}
%
% The include files are called |cdocsch1.tex| and |cdocsch2.tex|.
%
%\iffalse
%<*samplechap1|samplechap2>
%\fi

% Optional override for |\version| flag:
%    \begin{macrocode}
%%\providecommand{\version}{final}
%    \end{macrocode}

% Include the main document:
%    \begin{macrocode}
\input{childdoc.def}
\childdocof{cdocsamp}
%    \end{macrocode}

%\iffalse
%</samplechap1|samplechap2>
%\fi
%
%\iffalse
%<*samplechap1>
%\fi
% Some text for chapter 1:
%    \begin{macrocode}
\section{one}
some text in chapter one
%    \end{macrocode}

%\iffalse
%</samplechap1>
%\fi
% Some text for chapter 2:
%\iffalse
%<*samplechap2>
%\fi
%    \begin{macrocode}
\section{two}
more text in chapter two
%    \end{macrocode}

%\iffalse
%</samplechap2>
%\fi
%
% %%%%%%%%%%%%%%%%%%%%%%%%%%%%%%%%%%%%%%
% \paragraph{Part Include Files.}
%
% The include files are called |cdocspt3.tex| and |cdocspt4.tex|.
%
%\iffalse
%<*samplepart3|samplepart4>
%\fi

% Optional override for |\version| flag:
%    \begin{macrocode}
%%\providecommand{\version}{final}
%    \end{macrocode}

% Include the main document:
%    \begin{macrocode}
\input{childdoc.def}
\childdocby{cdocsamp}
%    \end{macrocode}

%\iffalse
%</samplepart3|samplepart4>
%\fi
%
%\iffalse
%<*samplepart3>
%\fi
% Some text for part 3:
%    \begin{macrocode}
some text in part three
%    \end{macrocode}

%\iffalse
%</samplepart3>
%\fi
% Some text for part 4:
%\iffalse
%<*samplepart4>
%\fi
%    \begin{macrocode}
more text in part four
%    \end{macrocode}

%\iffalse
%</samplepart4>
%\fi
%
% %%%%%%%%%%%%%%%%%%%%%%%%%%%%%%%%%%%%%%
% \paragraph{Forwarding for a Complete Draft.}
%
% The following forwarding file |cdocsdrf.tex|
% compiles the main document in draft mode:
%\iffalse
%<*sampledraft>
%\fi
%    \begin{macrocode}
\def\version{draft}
\input{childdoc.def}
\childdocforward{cdocsamp}
%    \end{macrocode}

%\iffalse
%</sampledraft>
%\fi
%
% %%%%%%%%%%%%%%%%%%%%%%%%%%%%%%%%%%%%%%
% \paragraph{Forwarding for Final Version of the Chapters.}
%
% The following forwarding files |cdocsfn1.tex| and |cdocsfn2.tex|
% (with identical content)
% compile the final versions of the child documents
% |cdocsch1.tex| and |cdocsch2.tex|, respectively:
%\iffalse
%<*samplefinal>
%\fi
%    \begin{macrocode}
\def\version{final}
\input{childdoc.def}
\childdocforwardprefix[cdocsamp]{cdocsfn}{cdocsch}
%    \end{macrocode}

%\iffalse
%</samplefinal>
%\fi
%
% %%%%%%%%%%%%%%%%%%%%%%%%%%%%%%%%%%%%%%
% \paragraph{Command Line Processing.}
%
% The following three command lines generate the output files
% |cdocscld|, |cdocscl1| and |cdocscl2|
% which should be identical to
% |cdocsdrf|, |cdocsch1| and |cdocsfn2|, respectively:
% \begin{center}
% \begin{tabular}{l}
% |latex -jobname cdocscld \|\\
% |  "\def\version{draft}\input{childdoc.def}\childdocforward{cdocsamp}"|\\
% |latex -jobname cdocscl1 \|\\
% |  "\input{childdoc.def}\childdocforward[cdocsamp]{cdocsch1}"|\\
% |latex -jobname cdocscl2 \|\\
% |  "\def\version{final}\input{childdoc.def}\childdocforward{cdocsch2}"|
% \end{tabular}
% \end{center}
% Note that the trailing backslash on each first line
% merely continues the input to the second line
% (for convenient cut ant paste).
% Furthermore, the command |latex| can be replaced by any
% of its alternative versions such as |pdflatex|.
%
% %%%%%%%%%%%%%%%%%%%%%%%%%%%%%%%%%%%%%%%%%%%%%%%%%%%%%%%%%%%%%%%%%%%%%%%%%%%%%%
% %%%%%%%%%%%%%%%%%%%%%%%%%%%%%%%%%%%%%%%%%%%%%%%%%%%%%%%%%%%%%%%%%%%%%%%%%%%%%%
% \section{Implementation}
%\iffalse
%<*package>
%\fi
%
% This section describes the definitions file |childdoc.def|.

% The definitions cannot be loaded using |\usepackage| or |\RequirePackage|
% which has a mechanism to prevent loading a style file more than once.
% When loading the definitions by means of |\input|
% multiple instances have to be prevented manually:
%\iffalse
%This code needs to be before the `\ProvidesFile' directive
%which is defined at the beginning of this file.
%Therefore it is also placed there and commented out here.
%</package>
%<*discard>
%\fi
%    \begin{macrocode}
\ifdefined\childdocmain\endinput\fi
%    \end{macrocode}
%\iffalse
%</discard>
%<*package>
%\fi
%
% \macro{\ifchilddoc}
% \macro{\ifchilddocmanual}
% The conditional |\ifchilddoc| tells whether a
% child (true) or main (false) document is being compiled.
% The conditional |\ifchilddocmanual| tells whether
% the |\includeonly| mechanism is used (false) or
% the selection of child files must be performed manually (true).
% The definitions initialise to false:
%    \begin{macrocode}
\newif\ifchilddoc
\newif\ifchilddocmanual
%    \end{macrocode}

% \macro{\childdocname}
% \macro{\childdocjob}
% The macro |\childdocname| stores the name of the main document
% to be compiled. The macro |\childdocjob| stores the name of
% the document on which the \LaTeX{} compiler was originally invoked.
% The content of |\jobname| cannot be compared
% to filenames specified in the source due to different catcodes.
% The following code rescans |\jobname|, stores the result
% in |\childdocname| and saves a copy in |\childdocjob|:
%    \begin{macrocode}
\edef\childdocname{\scantokens\expandafter{\jobname\noexpand}}
\let\childdocjob\childdocname
%    \end{macrocode}

% \macro{\childdocdisable}
% The macro |\childdocdisable| prevents the main file
% from being processed more than once.
% At this stage, the main document command |\childdocmain|
% is assumed to be called once again where it should do nothing.
% Any subsequent call to it should prevent
% a secondary processing of the main document
% It overwrites the forwarding commands
% |\childdocof| and |\childdocforward|
% with empty macros to prevent further inclusions of the main document:
%    \begin{macrocode}
\newcommand{\childdocdisable}
{
  \renewcommand{\childdocmain}[1]{\renewcommand{\childdocmain}[1]{\endinput}}
  \renewcommand{\childdocof}[1]{}
  \renewcommand{\childdocby}[2][]{}
  \renewcommand{\childdocforward}[2][]{}
  \renewcommand{\childdocdisable}{}
}
%    \end{macrocode}

% \macro{\childdocmain}
% The macro |\childdocmain| is to be called at the top of the main file
% with nothing or the main filename (without extension) as argument.
% First, it breaks loops.
% If the argument is not empty and does not match |\childdocname|
% (which is set by the first inclusion of |childdoc.def|),
% |\ifchilddoc| is set to true, |\includeonly| is applied to the child file
% and |\jobname| is set to the main file
% (for proper handling of |.aux| files):
%    \begin{macrocode}
\newcommand{\childdocmain}[1]
{
  \childdocdisable\childdocmain{}
  \if?#1?\else
    \begingroup
      \def\childdoctmp{#1}
      \ifx\childdoctmp\childdocname
        \def\childdoctmp{}
      \else
        \def\childdoctmp
        {
          \childdoctrue
          \includeonly{\childdocname}
          \def\childdocjob{#1}
          \def\jobname{#1}
        }
      \fi
      \expandafter
    \endgroup
    \childdoctmp
  \fi
}
%    \end{macrocode}

% \macro{\childdocof}
% The command |\childdocof| redirects
% compilation to the main file |#1|.
%    \begin{macrocode}
\newcommand{\childdocof}[1]
{
  \childdocdisable
  \childdoctrue
  \includeonly{\childdocname}
  \def\jobname{#1}
  \def\childdocjob{#1}
  \input{#1}
}
%    \end{macrocode}

% \macro{\childdocby}
% The command |\childdocby| ....
%    \begin{macrocode}
\newcommand{\childdocby}[2][]
{
  \childdocdisable
  \childdoctrue
  \childdocmanualtrue
  \if?#1?\else
    \def\jobname{#2}
  \fi
  \def\childdocjob{#2}
  \input{#2}
  \endinput
}
%    \end{macrocode}

% \macro{\childdocforward}
% The command |\childdocforward| redirects
% compilation to the main file or
% (if the optional argument is given) a child file.
% Parameters are set as if the main file
% or a child file starting with |\childdocof| was compiled.
% Then compilation is handed over to the main file:
%    \begin{macrocode}
\newcommand{\childdocforward}[2][]
{
  \begingroup
    \if?#1?
      \def\childdoctmp
      {
        \def\childdocname{#2}
        \def\childdocjob{#2}
        \def\jobname{#2}
        \input{#2}
        \endinput
      }
    \else
      \def\childdoctmp
      {
        \childdocdisable
        \def\childdocname{#2}
        \childdoctrue
        \includeonly{#2}
        \def\childdocjob{#1}
        \def\jobname{#1}
        \input{#1}
        \endinput
      }
    \fi
    \expandafter
  \endgroup
  \childdoctmp
}
%    \end{macrocode}

% \macro{\childdocforwardprefix}
% The command |\childdocforwardprefix| redirects
% compilation to the main or a child file by means of a pattern.
% The prefix |#1| in the current filename is replaced by |#2|
% and the suffix of the current filename is kept
% (it is assumed that the filename does not contain the substring `|~~~|'
% which is used as a delimiter).
% Compilation is handed over to the new file by |\childdocforward|:
%    \begin{macrocode}
\newcommand{\childdocforwardprefix}[3][]
{
  \begingroup
    \def\childdocextract #2##1~~~{\def\childdoctmp{\childdocforward[#1]{#3##1}}}
    \expandafter\childdocextract\childdocname~~~
    \expandafter
  \endgroup
  \childdoctmp
}
%    \end{macrocode}

% \macro{\childdoc}
% The deprecated macro |\childdoc| is a legacy version of |\childdocmain|:
%    \begin{macrocode}
\newcommand{\childdoc}{\childdocmain}
%    \end{macrocode}

% \macro{\childdocredirect}
% The deprecated macro |\childdocredirect| is a legacy version
% of |\childdocforward| and |\childdocforwardprefix|:
%    \begin{macrocode}
\newcommand{\childdocredirect}[2][]
{
  \begingroup
    \if?#1?
      \def\childdoctmp{\childdocforward{#2}}
    \else
      \def\childdoctmp{\childdocforwardprefix{#1}{#2}}
    \fi
    \expandafter
  \endgroup
  \childdoctmp
}
%    \end{macrocode}

%\iffalse
%</package>
%\fi
%
\endinput
|\\
|\childdocmain{|\textit{main}|}|\\
\end{tabular}
\end{center}
%
If |\jobname| does not match the argument \textit{main} of |\childdocmain|,
it is assumed that |\jobname| points to the child file to be compiled.
When using |\childdocmain| with the main file specified as argument,
it suffices to start a child file
with just |\input{|\textit{main}|}|
without loading of the package and using |\childdocof|.
If instead all processing is done
with the appropriate \textsf{childdoc} directives,
the argument of \textit{main} of |\childdocmain| can be empty.

An alternative version of the command line processing described
in \secref{sec:commandline} using the detection mechanism reads:
%
\begin{center}
|... -jobname "|\textit{target}|" "|[\textit{flags}]%
[|\def\jobname{|\textit{dest}|}|]|\input{|\textit{main}|}"|
\end{center}

%%%%%%%%%%%%%%%%%%%%%%%%%%%%%%%%%%%%%%%%%%%%%%%%%%%%%%%%%%%%%%%%%%%%%%%%%%%%%%%%
\subsection{Manual Code}
\label{sec:manual}

In case one cannot be certain whether the definitions file |childdoc.def|
is installed on the target \TeX{} distribution
and one prefers not to ship it,
it is conceivable to paste a few relevant commands into the sources.

To that end, drop all statements |% \iffalse
%
% childdoc.dtx Copyright (C) 2017-2018 Niklas Beisert
%
% This work may be distributed and/or modified under the
% conditions of the LaTeX Project Public License, either version 1.3
% of this license or (at your option) any later version.
% The latest version of this license is in
%   http://www.latex-project.org/lppl.txt
% and version 1.3 or later is part of all distributions of LaTeX
% version 2005/12/01 or later.
%
% This work has the LPPL maintenance status `maintained'.
%
% The Current Maintainer of this work is Niklas Beisert.
%
% This work consists of the files childdoc.dtx and childdoc.ins
% and the derived files childdoc.def and cdocsamp.tex with
% cdocsch1.tex, cdocsch2.tex, cdocsdrf.tex, cdocsfn1.tex, cdocsfn2.tex.
%
%<package>\ifdefined\childdocmain\endinput\fi
%<package>\ProvidesFile{childdoc.def}[2018/12/30 v2.0 child document driver]
%<samplemain>\ProvidesFile{cdocsamp.tex}[2018/12/30 v2.0 sample for childdoc]
%<*driver>
%\ProvidesFile{childdoc.drv}[2018/12/30 v2.0 childdoc reference manual file]
\PassOptionsToClass{10pt,a4paper}{article}
\documentclass{ltxdoc}

\usepackage[margin=35mm]{geometry}
\usepackage{hyperref}
\usepackage{hyperxmp}
\usepackage[usenames]{color}

\hypersetup{colorlinks=true}
\hypersetup{pdfstartview=FitH}
\hypersetup{pdfpagemode=UseNone}
\hypersetup{pdfsource={}}
\hypersetup{pdflang={en-UK}}
\hypersetup{pdfcopyright={Copyright 2017-2018 Niklas Beisert.
  This work may be distributed and/or modified under the
  conditions of the LaTeX Project Public License, either version 1.3
  of this license or (at your option) any later version.}}
\hypersetup{pdflicenseurl={http://www.latex-project.org/lppl.txt}}
\hypersetup{pdfcontactaddress={ETH Zurich, ITP, HIT K,
  Wolfgang-Pauli-Strasse 27}}
\hypersetup{pdfcontactpostcode={8093}}
\hypersetup{pdfcontactcity={Zurich}}
\hypersetup{pdfcontactcountry={Switzerland}}
\hypersetup{pdfcontactemail={nbeisert@itp.phys.ethz.ch}}
\hypersetup{pdfcontacturl={http://people.phys.ethz.ch/\xmptilde nbeisert/}}

\newcommand{\secref}[1]{\hyperref[#1]{section \ref*{#1}}}

\parskip1ex
\parindent0pt
\let\olditemize\itemize
\def\itemize{\olditemize\parskip0pt}

\begin{document}

\title{The \textsf{childdoc} Package}
\hypersetup{pdftitle={The childdoc Package}}
\author{Niklas Beisert\\[2ex]
  Institut f\"ur Theoretische Physik\\
  Eidgen\"ossische Technische Hochschule Z\"urich\\
  Wolfgang-Pauli-Strasse 27, 8093 Z\"urich, Switzerland\\[1ex]
  \href{mailto:nbeisert@itp.phys.ethz.ch}
  {\texttt{nbeisert@itp.phys.ethz.ch}}}
\hypersetup{pdfauthor={Niklas Beisert}}
\hypersetup{pdfsubject={Manual for the LaTeX2e Package childdoc}}
\date{30 December 2018, \textsf{v2.0}}
\maketitle

\begin{abstract}\noindent
\textsf{childdoc} is a \LaTeXe{} package
that enables the direct compilation
of document sections included by |\include|
to individual files.
\end{abstract}

\begingroup
\parskip0ex
\tableofcontents
\endgroup

%%%%%%%%%%%%%%%%%%%%%%%%%%%%%%%%%%%%%%%%%%%%%%%%%%%%%%%%%%%%%%%%%%%%%%%%%%%%%%%%
%%%%%%%%%%%%%%%%%%%%%%%%%%%%%%%%%%%%%%%%%%%%%%%%%%%%%%%%%%%%%%%%%%%%%%%%%%%%%%%%
\section{Introduction}

\LaTeX{} provides a mechanism to structure a large document (such as a book)
into a main file and several child files (containing the chapters)
using the |\include| command.
This mechanism is beneficial for documents
which span hundreds of pages in order to
make the source file(s) more manageable.
Moreover, compilation can be restricted to
selected child files by means of the |\includeonly| command.
The latter feature can be used to reduce the compilation time while editing
(this was significantly more useful in the earlier days of \LaTeX{})
or to generate a smaller document which is easier to navigate.
Another application of |\includeonly| is to generate
documents consisting of selected parts of the complete document.

However, there are a few drawbacks of the plain |\include| mechanism:
\begin{itemize}
\item
The child files cannot be compiled on their own,
they can only be compiled via the main file.
A naive editing environment
(such as a text editor with an option
to have the current file processed by \LaTeX)
may require one to switch to the main file before compiling;
attempting to compile the child file produces errors.
\item
The main file must be modified (each time)
to adjust the |\includeonly| command
to the present needs. This easily leaves the main file in a messy state.
\item
The generated document will always carry the filename
of the main document. This is inconvenient if
several child files are to be compiled and
to be kept for distribution.
\end{itemize}

The present package provides a simple interface
to make child files individually compilable by \LaTeX{}.
Compiling a child file then has the same effect as compiling
the main file with an |\includeonly| command
to select the appropriate child.
Moreover the generated document will carry the name of the child
rather than the main file.
This resolves all three above issues.

This feature is meant to make the editing of books,
thesis documents and lecture notes somewhat more convenient.
However, the package can also be used efficiently for
composing a series of documents (such as exercise sheets)
which are typically distributed individually.
It then assists the author in generating the individual documents
(potentially in different versions)
as well as a document containing the collected series.
Another application is in developing style files
or other kinds of included material
where compilation of the style file could redirect
to a sample or test file.

%%%%%%%%%%%%%%%%%%%%%%%%%%%%%%%%%%%%%%%%%%%%%%%%%%%%%%%%%%%%%%%%%%%%%%%%%%%%%%%%
%%%%%%%%%%%%%%%%%%%%%%%%%%%%%%%%%%%%%%%%%%%%%%%%%%%%%%%%%%%%%%%%%%%%%%%%%%%%%%%%
\section{Usage}

First of all, the package \textsf{childdoc} is \emph{not} a standard
\LaTeXe{} |.sty| style file! Therefore it needs to be invoked in
a non-standard way.

%%%%%%%%%%%%%%%%%%%%%%%%%%%%%%%%%%%%%%%%%%%%%%%%%%%%%%%%%%%%%%%%%%%%%%%%%%%%%%%%
\subsection{Included Files}
\label{sec:include}

%%%%%%%%%%%%%%%%%%%%%%%%%%%%%%%%%%%%%%%%
\DescribeMacro{\childdocmain}
To use the package, add the commands
\begin{center}
\begin{tabular}{l}
|\input{childdoc.def}|\\
|\childdocmain{}|\\
\end{tabular}
\end{center}
at the very top of the main \LaTeX{} file,
in particular \emph{before} the |\documentclass| statement!
The argument of |\childdocmain| should be left empty
(but it must be present).

%%%%%%%%%%%%%%%%%%%%%%%%%%%%%%%%%%%%%%%%
\DescribeMacro{\childdocof}
Furthermore, add the commands
\begin{center}
\begin{tabular}{l}
|\input{childdoc.def}|\\
|\childdocof{|\textit{main}|}|\\
\end{tabular}
\end{center}
at the top of every child file \textit{child}
which is included by |\include{|\textit{child}|}|
from within the main file
(or at least for those files to be compiled individually).
The argument \textit{main} must be the filename of the main file.

There are a couple of
considerations in setting up the main and child documents:

%%%%%%%%%%%%%%%%%%%%%%%%%%%%%%%%%%%%%%%%
\paragraph{Restrictions.}

Please note the following restrictions:
\begin{itemize}
\item
|\childdocmain| must be called with one argument \textit{main}
to ensure compatibility with earlier version of the package.
It must either be empty (|\childdocmain{}|)
or precisely match the filename of the main file in which it is specified.
See \secref{sec:detection} for further information.
\item
The filename \textit{main} must be specified without the |.tex| extension.
\item
The filename \textit{main} is case sensitive
(even in case-insensitive file systems)
due to internal string comparison.
\item
The argument \textit{main} should be fully expanded, it cannot be a macro.
\item
Subdirectories and special characters should be avoided in filenames.
\item
The command |\childdocmain{|\textit{main}|}| must be followed by a whitespace.
It should not be followed immediately by another command
or by a comment mark `|%|'.
This is because the \TeX{} parser reads the token immediately following
the argument of |\childdocmain| and puts it
at the beginning of every child section;
however, a white\-space is ignored.
\end{itemize}

%%%%%%%%%%%%%%%%%%%%%%%%%%%%%%%%%%%%%%%%
\paragraph{Content of Main File.}

It is advisable to place all content in the child files included by |\include|.
Any output contained in the main file will appear in all child documents
unless suppressed manually;
it cannot be suppressed automatically by the |\includeonly| directive
and thus should normally be avoided.
A method to include some content in the main file
by means of conditional processing is described in \secref{sec:conditional}.

%%%%%%%%%%%%%%%%%%%%%%%%%%%%%%%%%%%%%%%%
\paragraph{Page Numbering.}

When only a part of the document is compiled,
the appropriate numbering of pages
(as well as other status parameters)
is determined from the |.aux| files.
The latter contain information from previous passes.
However this information needs to propagate through
all intermediate child documents.
Therefore the page numbering in child documents may well
be inconsistent until the complete document is compiled at least once.

A useful (if unconventional) way to always ensure a consistent
page numbering is to restart the numbering in each child document
and denote the pages by `\textit{child}|.|\textit{page}'
where \textit{child} represents the chapter/section number of the child file.
This can be achieved by the command
|\numberwithin{page}{|\textit{child}|}|
of the \textsf{amsmath} package
where \textit{child} can be |chapter| or |section|
depending on the chosen structuring.
Alternatively, one can modify the macro |\thepage| appropriately
and reset the counter |page| at the start of each child file.

%%%%%%%%%%%%%%%%%%%%%%%%%%%%%%%%%%%%%%%%%%%%%%%%%%%%%%%%%%%%%%%%%%%%%%%%%%%%%%%%
\subsection{Conditional Processing}
\label{sec:conditional}

The package provides a mechanism to compile different versions
of a document. To customise the versions further some conditional processing
can come in handy to distinguish which version is being compiled.
The package provides two macros to describe the compilation context:

%%%%%%%%%%%%%%%%%%%%%%%%%%%%%%%%%%%%%%%%
\DescribeMacro{\ifchilddoc}
The conditional |\ifchilddoc| distinguishes between the compilation of
child documents and the main document:
%
\begin{center}
|\ifchilddoc |\textit{child-code}| |[|\||else |\textit{main-code}]| \||fi|
\end{center}

%%%%%%%%%%%%%%%%%%%%%%%%%%%%%%%%%%%%%%%%
\DescribeMacro{\childdocname}
\DescribeMacro{\childdocjob}
The macro |\childdocname| contains the filename (without extension)
of the main or child file being processed.
Note that |\childdocjob| will always contain the name of the main file.

%%%%%%%%%%%%%%%%%%%%%%%%%%%%%%%%%%%%%%%%
\paragraph{Title Page.}

Conditional processing can be used to include a title or banner page
in the main document when proper precautions are taken.
Importantly, the code in the main file should ensure that the page counter
(as well as other status parameters which are stored in the |.aux| files)
takes the same value after the conditional processing.
Otherwise the page numbers may take divergent values
depending on which part is compiled.

For example, a title page could be declared by:
%
\begin{center}
\begin{tabular}{l}
|\ifchilddoc\||else|\\
|\addtocounter{page}{-1}|\\
\textit{code for title page}\\
|\newpage|\\
|\||fi|
\end{tabular}
\end{center}
%
A banner page for the child documents can be generated by:
%
\begin{center}
\begin{tabular}{l}
|\ifchilddoc|\\
|\addtocounter{page}{-1}|\\
\textit{code for banner page}\\
|\newpage|\\
|\||fi|
\end{tabular}
\end{center}
%
Here one could write a message such as:
\begin{center}
|This is the part \childdocname{} of \childdocjob{}.|
\end{center}

%%%%%%%%%%%%%%%%%%%%%%%%%%%%%%%%%%%%%%%%%%%%%%%%%%%%%%%%%%%%%%%%%%%%%%%%%%%%%%%%
\subsection{Flags}
\label{sec:flags}

The package makes it easy to generate different versions
of the main or child documents.
To this end compilation flags can be defined
and assigned different default values.
They will be particularly useful in conjunction
with the forwarding mechanism described in \secref{sec:forward}.

For example, it may be useful to have a flag |\version|
which can be set to |draft| or |final|.
The document source will contain some conditional code
depending on the value of |\version|.
Suppose further, the flag should default to |final| for the main file
and to |draft| for child files
which is a natural assignment for editing the document.
This is achieved by placing the following code
in the preamble of the main document
(below the |\childdocmain| directive):
%
\begin{center}
\begin{tabular}{l}
|\ifchilddoc|\\
|\providecommand{\version}{draft}|\\
|\||else|\\
|\providecommand{\version}{final}|\\
|\||fi|
\end{tabular}
\end{center}
%
The definition by |\providecommand| makes sure
that previous definitions are not overwritten.
Further statements |\providecommand{\version}{...}|
can thus be added before the above code to override it.

For the main file, one might add a line
(between |\childdocmain| and the above block)
%
\begin{center}
|%\ifchilddoc\||else\providecommand{\version}{draft}\||fi|
\end{center}
%
which can be uncommented to produce a draft version.
Likewise one can add a line to the very top of a child file
(above the |\childdocof{|\textit{main}|}| directive)
%
\begin{center}
|%\providecommand{\version}{final}|
\end{center}
%
which can be uncommented to produce the final version of this child document.

%%%%%%%%%%%%%%%%%%%%%%%%%%%%%%%%%%%%%%%%%%%%%%%%%%%%%%%%%%%%%%%%%%%%%%%%%%%%%%%%
\subsection{Forwarding}
\label{sec:forward}

Different versions of the main or child documents
using compilation flags as described in \secref{sec:flags}
can be (permanently) stored in different files
for convenient compilation, viewing and distribution.
To this end, the package defines a command
to pass on compilation to a different file:

%%%%%%%%%%%%%%%%%%%%%%%%%%%%%%%%%%%%%%%%
\DescribeMacro{\childdocforward}
The command |\childdocforward| redirects processing to
another source file:
%
\begin{center}
\begin{tabular}{l}
|\input{childdoc.def}|\\
|\childdocforward[|\textit{main}|]{|\textit{dest}|}|\\
\end{tabular}
\end{center}
%
The argument \textit{dest} is the destination file
(without extension).
It should be the main file or one of the child files.
Note that further \textsf{childdoc} directives
such as |\childdocof| and |\childdocforward|
in the indicated file will be processed in this form.
The optional argument \textit{main}
passes on directly to the main file \textit{main}
while pretending to compile the child \textit{dest}.
This form behaves as if \textit{dest}
issues |\childdocof{|\textit{main}|}| right away,
and no further \textsf{childdoc} directives will be processed.

%%%%%%%%%%%%%%%%%%%%%%%%%%%%%%%%%%%%%%%%
\DescribeMacro{\...prefix}
In the alternative form |\childdocforwardprefix|,
%
\begin{center}
\begin{tabular}{l}
|\input{childdoc.def}|\\
|\childdocforwardprefix[|\textit{main}|]{|\textit{prefix}|}{|\textit{dest}|}|
\end{tabular}
\end{center}
%
the destination file is determined by a pattern
depending on the current file:
To make this work, the current file must be called
`{\textit{prefix}\hspace{0.2em}\textit{suffix}}'
with \textit{prefix} matching precisely the argument.
Processing is then passed on to the file
`{\textit{dest}\hspace{0.2em}\textit{suffix}}'.
Surely, the same effect is achieved by
directly specifying the
argument `{\textit{dest}\hspace{0.2em}\textit{suffix}}'
in the first form.
However, that requires to set up a different file
for each child. With the alternative form of the command
all these files can have exactly the same content
which simplifies setting them up and maintaining them.

For example, the following file |draft.tex|
with a compilation flag |\version| as described in \secref{sec:flags}
compiles the main document as a draft:
%
\begin{center}
\begin{tabular}{l}
|\def\version{draft}|\\
|\input{childdoc.def}|\\
|\childdocforward{|\textit{main}|}|
\end{tabular}
\end{center}
%
Likewise, the following files |final|\textit{nn}|.tex|
compile the final version of the child document
|child|\textit{nn}|.tex|:
%
\begin{center}
\begin{tabular}{l}
|\def\version{final}|\\
|\input{childdoc.def}|\\
|\childdocforwardprefix{final}{child}|
\end{tabular}
\end{center}
%

Note that when several versions of a main file and/or of each child file
are to be generated, it may be convenient to set up a |Makefile| or
shell script to automatise the process.

%%%%%%%%%%%%%%%%%%%%%%%%%%%%%%%%%%%%%%%%%%%%%%%%%%%%%%%%%%%%%%%%%%%%%%%%%%%%%%%%
\subsection{Command Line Processing}
\label{sec:commandline}

The effect of redirection files can also be achieved by invoking
the \LaTeX{} compiler with a more elaborate command line.
Most conveniently this should be done as part
of a shell script or a |Makefile|.

When using \textsf{childdoc} in the main file, the following
command lines effectively perform a redirection
(note that depending on the shell being used,
backslashes may have to be doubled: `|\|' $\to$ `|\\|'):
%
\begin{center}
|... -jobname "|\textit{target}|" |\\|"|[\textit{flags}]%
|\input{childdoc.def}\childdocforward[|\textit{main}|]{|\textit{dest}|}"|
\end{center}
%
Here \textit{target} is the name of the output file,
\textit{main} is the name of the main file
and \textit{dest} is the name of the main or child file to be processed
(all filenames without extensions).
The optional argument \textit{main} can be omitted
if \textit{main} matches \textit{dest}.
Optionally, compilation \textit{flags} can be defined via |\def| commands.
This command line makes the \TeX{} engine believe
it is compiling the file \textit{target}
whose content is specified as the latter parameter.
The provided code then forwards the processing to
\textit{main} or \textit{dest} as described in \secref{sec:forward}.

%%%%%%%%%%%%%%%%%%%%%%%%%%%%%%%%%%%%%%%%%%%%%%%%%%%%%%%%%%%%%%%%%%%%%%%%%%%%%%%%
\subsection{Include by Input}
\label{sec:input}

Including child documents by |\include| has some restrictions by design.
Most notably, the content of a child document always occupies
its own set of pages; pages cannot be shared between child documents.
Usually, this behaviour makes perfect sense
because each child document contain an essential part of the document.
However, in some situations it may be desirable to compose
a document from a collection of parts
without having mandatory page breaks between then.
For this case, the package
provides a mechanism to include parts
by |\input| which can also be processed individually.
However, by construction this mechanism
requires manual handling of the content to be output.

%%%%%%%%%%%%%%%%%%%%%%%%%%%%%%%%%%%%%%%%
\DescribeMacro{\ifchilddocmanual}
The main file should be prepared as usual, see \secref{sec:include}.
However, the document body must make a distinction
between processing of an individual part and of the main document, e.g.:
%
\begin{center}
\begin{tabular}{l}
|\ifchilddocmanual|\\
|\input{\childdocname}|\\
|\||else|\\
\textit{document body with }|\input{|\textit{part}|}|\\
|\||fi|
\end{tabular}
\end{center}
%
The conditional |\ifchilddocmanual| is true whenever
a part to be included by |\input| is being compiled,
and the name of the part is stored in |\childdocname|.

%%%%%%%%%%%%%%%%%%%%%%%%%%%%%%%%%%%%%%%%
\DescribeMacro{\childdocby}
Each part to be included by |\input| should start with:
%
\begin{center}
\begin{tabular}{l}
|\input{childdoc.def}|\\
|\childdocby{|\textit{main}|}|\\
\end{tabular}
\end{center}
%
The directive |\childdocby| is similar to |\childdocof|
described in \secref{sec:include},
but the subsequent selection of content must be done manually.
To that end, both |\ifchilddoc| and |\ifchilddocmanual|
will be true upon processing of a part,
and the name of the part is stored in |\childdocname|.
Note that |\jobname| will be set to the filename of the current part
so that each part receives an individual |.aux| file
that does not interfere with the |.aux| file(s) of the main document.
This behaviour can be altered by the alternative form
|\childdocby[*]{|\textit{main}|}| (with a non-empty optional argument)
which uses the |.aux| file of the main document
by setting |\jobname| to \textit{main}.

%%%%%%%%%%%%%%%%%%%%%%%%%%%%%%%%%%%%%%%%%%%%%%%%%%%%%%%%%%%%%%%%%%%%%%%%%%%%%%%%
\subsection{Driver Development}
\label{sec:driver}

The \textsf{childdoc} mechanism can also be use for the development
of definition files such as \LaTeX{} styles or classes.
This case differs from the above setup with multiple parts
included by |\include| in that no |\includeonly| should be invoked.
This can be achieved by starting the include file
(before |\ProvidesPackage|) with:
%
\begin{center}
\begin{tabular}{l}
|\input{childdoc.def}|\\
|\childdocforward{|\textit{main}|}|\\
\end{tabular}
\end{center}
%
or alternatively with:
%
\begin{center}
\begin{tabular}{l}
|\input{childdoc.def}|\\
|\childdocby{|\textit{main}|}|\\
\end{tabular}
\end{center}
%
Both forms have slightly different effects as described above.
The main file is prepared as usual, see \secref{sec:include}.

%%%%%%%%%%%%%%%%%%%%%%%%%%%%%%%%%%%%%%%%%%%%%%%%%%%%%%%%%%%%%%%%%%%%%%%%%%%%%%%%
\subsection{Legacy Detection}
\label{sec:detection}

The directive |\childdocmain| in the main file can detect
whether the complete document or merely a child is to be compiled
even without using the directive |\childdocof|.
This method is deprecated because it is less robust
and there is no compelling reason to use it;
it is merely provided for backward compatibility
and it may be removed in future versions.

If the detection mechanism is to be used,
it is mandatory to correctly specify
the filename of the main file as the argument of |\childdocmain|:
%
\begin{center}
\begin{tabular}{l}
|\input{childdoc.def}|\\
|\childdocmain{|\textit{main}|}|\\
\end{tabular}
\end{center}
%
If |\jobname| does not match the argument \textit{main} of |\childdocmain|,
it is assumed that |\jobname| points to the child file to be compiled.
When using |\childdocmain| with the main file specified as argument,
it suffices to start a child file
with just |\input{|\textit{main}|}|
without loading of the package and using |\childdocof|.
If instead all processing is done
with the appropriate \textsf{childdoc} directives,
the argument of \textit{main} of |\childdocmain| can be empty.

An alternative version of the command line processing described
in \secref{sec:commandline} using the detection mechanism reads:
%
\begin{center}
|... -jobname "|\textit{target}|" "|[\textit{flags}]%
[|\def\jobname{|\textit{dest}|}|]|\input{|\textit{main}|}"|
\end{center}

%%%%%%%%%%%%%%%%%%%%%%%%%%%%%%%%%%%%%%%%%%%%%%%%%%%%%%%%%%%%%%%%%%%%%%%%%%%%%%%%
\subsection{Manual Code}
\label{sec:manual}

In case one cannot be certain whether the definitions file |childdoc.def|
is installed on the target \TeX{} distribution
and one prefers not to ship it,
it is conceivable to paste a few relevant commands into the sources.

To that end, drop all statements |\input{childdoc.def}|
and perform the replacements as outlined below.
Instead of |\childdocmain{|\textit{main}|}| add the following code
to the top of the main file:
%
\begin{center}
\begin{tabular}{l}
|\||ifdefined\childdocname\endinput\||fi\newif\ifchilddoc|\\
|\edef\childdocname{\scantokens\expandafter{\jobname\noexpand}}|\\
|\def\childdocmain{|\textit{main}|}\||ifx\childdocmain\childdocname\||else|\\
|\childdoctrue\includeonly{\childdocname}\let\jobname\childdocmain\||fi|\\
\end{tabular}
\end{center}
%
Instead of |\childdocof{|\textit{main}|}| just include the main file
at the top of each child file:
%
\begin{center}
|\input{|\textit{main}|}|
\end{center}
%
A simple redirection |\childdocforward{|\textit{dest}|}| is achieved by:
%
\begin{center}
|\def\jobname{|\textit{dest}|}\input{\jobname}|
\end{center}
%
The redirection with prefix
|\childdocforwardprefix[|\textit{prefix}|]{|\textit{dest}|}|
is accomplished by:
%
\begin{center}
\begin{tabular}{l}
|{\edef\jobname{\scantokens\expandafter{\jobname\noexpand}}|\\
|\def\redirectjob |\textit{prefix}|#1~~~{\gdef\jobname{|\textit{dest}|#1}}|\\
|\expandafter\redirectjob\jobname~~~}\input{\jobname}|
\end{tabular}
\end{center}

In an alternative approach,
child documents can be compiled by a specific command line
without additional code or specific definitions:
%
\begin{center}
|... -jobname "|\textit{target}|" "|[\textit{flags}]%
|\includeonly{|\textit{dest}|}\input{|\textit{main}|}"|
\end{center}
%

%%%%%%%%%%%%%%%%%%%%%%%%%%%%%%%%%%%%%%%%%%%%%%%%%%%%%%%%%%%%%%%%%%%%%%%%%%%%%%%%
%%%%%%%%%%%%%%%%%%%%%%%%%%%%%%%%%%%%%%%%%%%%%%%%%%%%%%%%%%%%%%%%%%%%%%%%%%%%%%%%
\section{Information}

%%%%%%%%%%%%%%%%%%%%%%%%%%%%%%%%%%%%%%%%%%%%%%%%%%%%%%%%%%%%%%%%%%%%%%%%%%%%%%%%
\subsection{Copyright}

Copyright \copyright{} 2017--2018 Niklas Beisert

This work may be distributed and/or modified under the
conditions of the \LaTeX{} Project Public License, either version 1.3
of this license or (at your option) any later version.
The latest version of this license is in
  \url{http://www.latex-project.org/lppl.txt}
and version 1.3 or later is part of all distributions of \LaTeX{}
version 2005/12/01 or later.

This work has the LPPL maintenance status `maintained'.

The Current Maintainer of this work is Niklas Beisert.

This work consists of the files |README.txt|, |childdoc.ins| and |childdoc.dtx|
as well as the derived files |childdoc.def|, |cdocsamp.tex|
with |cdocsch1.tex|, |cdocsch2.tex|, |cdocspt3.tex|, |cdocspt4.tex|,
|cdocsdrf.tex|, |cdocsfn1.tex|, |cdocsfn2.tex|
as well as |childdoc.pdf|.

%%%%%%%%%%%%%%%%%%%%%%%%%%%%%%%%%%%%%%%%%%%%%%%%%%%%%%%%%%%%%%%%%%%%%%%%%%%%%%%%
\subsection{Files and Installation}

The package consists of the files:
%
\begin{center}
\begin{tabular}{ll}
    |README.txt|   & readme file \\
    |childdoc.ins| & installation file \\
    |childdoc.dtx| & source file \\
    |childdoc.def| & definition file \\
    |cdocsamp.tex| & sample main file \\
    |cdocsch1.tex| & sample include file \\
    |cdocsch2.tex| & sample include file \\
    |cdocspt3.tex| & sample part file \\
    |cdocspt4.tex| & sample part file \\
    |cdocsdrf.tex| & sample redirection file \\
    |cdocsfn1.tex| & sample redirection file \\
    |cdocsfn2.tex| & sample redirection file \\
    |childdoc.pdf| & manual
\end{tabular}
\end{center}
%
The distribution consists of the files
|README.txt|, |childdoc.ins| and |childdoc.dtx|.
%
\begin{itemize}
\item
Run (pdf)\LaTeX{} on |childdoc.dtx|
to compile the manual |childdoc.pdf| (this file).
\item
Run \LaTeX{} on |childdoc.ins| to create the definitions file |childdoc.def|
and the sample |cdocsamp.tex| with include files
|cdocsch1.tex|, |cdocsch2.tex|, |cdocspt3.tex|, |cdocspt4.tex|,
|cdocsdrf.tex|, |cdocsfn1.tex|, |cdocsfn2.tex|.
Then copy the file |childdoc.def| to an appropriate directory of your \LaTeX{}
distribution, e.g.\ \textit{texmf-root}|/tex/latex/childdoc|.
\end{itemize}

%%%%%%%%%%%%%%%%%%%%%%%%%%%%%%%%%%%%%%%%%%%%%%%%%%%%%%%%%%%%%%%%%%%%%%%%%%%%%%%%
\subsection{Related CTAN Packages}

There are several other packages which offer a similar functionality:
%
\begin{itemize}
\item
The packages
\href{http://ctan.org/pkg/docmute}{\textsf{docmute}},
\href{http://ctan.org/pkg/includex}{\textsf{includex}} and
\href{http://ctan.org/pkg/standalone}{\textsf{standalone}}
provide commands to include only the document body of
a child file thus allowing both files to be compiled individually.
\item
The packages \href{http://ctan.org/pkg/subdocs}{\textsf{subdocs}}
and \href{http://ctan.org/pkg/subfiles}{\textsf{subfiles}}
provide structures in which the main and child documents can be
encapsulated and allowing them to be compiled individually.
The inclusion mechanism is different from the conventional |\include|.
\item
The package \href{http://ctan.org/pkg/combine}{\textsf{combine}}
is an elaborate solution to combine several documents into one.
\end{itemize}
%
See also the CTAN topic \href{http://ctan.org/topic/subdocs}{\textsf{subdocs}}
for further related packages.
The present package differs from the above solutions in that
a document structure constructed with the conventional |\include| mechanism
just needs two extra commands at the top of every file
such that all constituent files can be compiled individually.

%%%%%%%%%%%%%%%%%%%%%%%%%%%%%%%%%%%%%%%%%%%%%%%%%%%%%%%%%%%%%%%%%%%%%%%%%%%%%%%%
%\subsection{Feature Suggestions}
%
%The following is a list of features which may be useful for future
%versions of this package:
%%
%\begin{itemize}
%\item
%\ldots
%\end{itemize}

%%%%%%%%%%%%%%%%%%%%%%%%%%%%%%%%%%%%%%%%%%%%%%%%%%%%%%%%%%%%%%%%%%%%%%%%%%%%%%%%
\subsection{Revision History}

%%%%%%%%%%%%%%%%%%%%%%%%%%%%%%%%%%%%%%%%
\paragraph{v2.0:} 2018/12/30

\begin{itemize}
\item
immediate forward processing
\item
added |\childdocby| mechanism
\item
manual restructured
\end{itemize}

%%%%%%%%%%%%%%%%%%%%%%%%%%%%%%%%%%%%%%%%
\paragraph{v1.6:} 2018/01/17

\begin{itemize}
\item
application for development of include files
\item
corrections to manual
\end{itemize}

%%%%%%%%%%%%%%%%%%%%%%%%%%%%%%%%%%%%%%%%
\paragraph{v1.5:} 2017/05/21

\begin{itemize}
\item
more complete structuring introduced
\item
|\childdocof| introduced
\item
|\childdoc| renamed to |\childdocmain|
\item
|\childredirect| renamed to |\childdocforward| and |\childdocforwardprefix|
and functionality expanded
\end{itemize}

%%%%%%%%%%%%%%%%%%%%%%%%%%%%%%%%%%%%%%%%
\paragraph{v1.0:} 2017/04/27

\begin{itemize}
\item
manual and install package
\item
first version published on CTAN
\end{itemize}

%%%%%%%%%%%%%%%%%%%%%%%%%%%%%%%%%%%%%%%%
\paragraph{v0.6:} 2017/04/26

\begin{itemize}
\item
redirection mechanism added
\end{itemize}

%%%%%%%%%%%%%%%%%%%%%%%%%%%%%%%%%%%%%%%%
\paragraph{v0.5:} 2017/04/26

\begin{itemize}
\item
functionality in definition file
\end{itemize}


%%%%%%%%%%%%%%%%%%%%%%%%%%%%%%%%%%%%%%%%%%%%%%%%%%%%%%%%%%%%%%%%%%%%%%%%%%%%%%%%
%%%%%%%%%%%%%%%%%%%%%%%%%%%%%%%%%%%%%%%%%%%%%%%%%%%%%%%%%%%%%%%%%%%%%%%%%%%%%%%%
%%%%%%%%%%%%%%%%%%%%%%%%%%%%%%%%%%%%%%%%%%%%%%%%%%%%%%%%%%%%%%%%%%%%%%%%%%%%%%%%
\appendix

\settowidth\MacroIndent{\rmfamily\scriptsize 000\ }

 \DocInput{childdoc.dtx}

\end{document}
%</driver>
% \fi
%
% %%%%%%%%%%%%%%%%%%%%%%%%%%%%%%%%%%%%%%%%%%%%%%%%%%%%%%%%%%%%%%%%%%%%%%%%%%%%%%
% %%%%%%%%%%%%%%%%%%%%%%%%%%%%%%%%%%%%%%%%%%%%%%%%%%%%%%%%%%%%%%%%%%%%%%%%%%%%%%
% \section{Sample}
%\iffalse
%<*samplemain>
%\fi
%
% The following presents a sample document
% with two chapters, two parts, a title page,
% a compile flag as well as three forwarding files to set the flag.
% It consists of eight |.tex| files:
% \begin{center}
% \begin{tabular}{ll}
% |cdocsamp.tex|&main file\\
% |cdocsch1.tex|&include file for chapter 1\\
% |cdocsch2.tex|&include file for chapter 2\\
% |cdocspt3.tex|&include file for part 3\\
% |cdocspt4.tex|&include file for part 4\\
% |cdocsdrf.tex|&forwarding file for main file in draft mode\\
% |cdocsfi1.tex|&forwarding file for final version of chapter 1\\
% |cdocsfi2.tex|&forwarding file for final version of chapter 2\\
% \end{tabular}
% \end{center}
% Each of the eight files can be compiled directly by the \LaTeX{} compiler.
%
% %%%%%%%%%%%%%%%%%%%%%%%%%%%%%%%%%%%%%%
% \paragraph{Main File.}
%
% The main file is called |cdocsamp.tex|.
%
% Load the \textsf{childdoc} definitions and
% declare the filename for the main document:
%    \begin{macrocode}
\input{childdoc.def}
\childdocmain{}
%    \end{macrocode}

% Optional override for |\version| flag:
%    \begin{macrocode}
%%\ifchilddoc\else\providecommand{\version}{draft}\fi
%    \end{macrocode}

% Define the default values for the |\version| flag
% (|final| for the main file and |draft| for childs):
%    \begin{macrocode}
\ifchilddoc
\providecommand{\version}{draft}
\else
\providecommand{\version}{final}
\fi
%    \end{macrocode}

% Load the standard document class:
%    \begin{macrocode}
\documentclass[12pt]{article}
%    \end{macrocode}

% Start the document body:
%    \begin{macrocode}
\begin{document}
%    \end{macrocode}

% Declare a title page.
% Print title, part of document being processed and version flag:
%    \begin{macrocode}
\addtocounter{page}{-1}
\begin{center}
{\LARGE\bfseries{}childdoc example\par}
\vspace{1cm}
\ifchilddoc
\ifchilddocmanual part\else chapter\fi:
`\childdocname' of `\childdocjob'\par
\else
main document: `\childdocjob'\par
\fi
version: \version\par
\end{center}
\newpage
%    \end{macrocode}

% Manually include selected file,
% otherwise process as usual:
%    \begin{macrocode}
\ifchilddocmanual
\section*{part `\childdocname'}
\input{\childdocname}
\else
%    \end{macrocode}

% Include the two chapters:
%    \begin{macrocode}
\include{cdocsch1}
\include{cdocsch2}
%    \end{macrocode}

% Include the two parts unless only chapters should be displayed:
%    \begin{macrocode}
\ifchilddoc\else
\section{part three}
\input{cdocspt3}
\section{part four}
\input{cdocspt4}
\fi
%    \end{macrocode}

% Process as usual until here:
%    \begin{macrocode}
\fi
%    \end{macrocode}

% End of document body:
%    \begin{macrocode}
\end{document}
%    \end{macrocode}
%\iffalse
%</samplemain>
%\fi
%
% %%%%%%%%%%%%%%%%%%%%%%%%%%%%%%%%%%%%%%
% \paragraph{Chapter Include Files.}
%
% The include files are called |cdocsch1.tex| and |cdocsch2.tex|.
%
%\iffalse
%<*samplechap1|samplechap2>
%\fi

% Optional override for |\version| flag:
%    \begin{macrocode}
%%\providecommand{\version}{final}
%    \end{macrocode}

% Include the main document:
%    \begin{macrocode}
\input{childdoc.def}
\childdocof{cdocsamp}
%    \end{macrocode}

%\iffalse
%</samplechap1|samplechap2>
%\fi
%
%\iffalse
%<*samplechap1>
%\fi
% Some text for chapter 1:
%    \begin{macrocode}
\section{one}
some text in chapter one
%    \end{macrocode}

%\iffalse
%</samplechap1>
%\fi
% Some text for chapter 2:
%\iffalse
%<*samplechap2>
%\fi
%    \begin{macrocode}
\section{two}
more text in chapter two
%    \end{macrocode}

%\iffalse
%</samplechap2>
%\fi
%
% %%%%%%%%%%%%%%%%%%%%%%%%%%%%%%%%%%%%%%
% \paragraph{Part Include Files.}
%
% The include files are called |cdocspt3.tex| and |cdocspt4.tex|.
%
%\iffalse
%<*samplepart3|samplepart4>
%\fi

% Optional override for |\version| flag:
%    \begin{macrocode}
%%\providecommand{\version}{final}
%    \end{macrocode}

% Include the main document:
%    \begin{macrocode}
\input{childdoc.def}
\childdocby{cdocsamp}
%    \end{macrocode}

%\iffalse
%</samplepart3|samplepart4>
%\fi
%
%\iffalse
%<*samplepart3>
%\fi
% Some text for part 3:
%    \begin{macrocode}
some text in part three
%    \end{macrocode}

%\iffalse
%</samplepart3>
%\fi
% Some text for part 4:
%\iffalse
%<*samplepart4>
%\fi
%    \begin{macrocode}
more text in part four
%    \end{macrocode}

%\iffalse
%</samplepart4>
%\fi
%
% %%%%%%%%%%%%%%%%%%%%%%%%%%%%%%%%%%%%%%
% \paragraph{Forwarding for a Complete Draft.}
%
% The following forwarding file |cdocsdrf.tex|
% compiles the main document in draft mode:
%\iffalse
%<*sampledraft>
%\fi
%    \begin{macrocode}
\def\version{draft}
\input{childdoc.def}
\childdocforward{cdocsamp}
%    \end{macrocode}

%\iffalse
%</sampledraft>
%\fi
%
% %%%%%%%%%%%%%%%%%%%%%%%%%%%%%%%%%%%%%%
% \paragraph{Forwarding for Final Version of the Chapters.}
%
% The following forwarding files |cdocsfn1.tex| and |cdocsfn2.tex|
% (with identical content)
% compile the final versions of the child documents
% |cdocsch1.tex| and |cdocsch2.tex|, respectively:
%\iffalse
%<*samplefinal>
%\fi
%    \begin{macrocode}
\def\version{final}
\input{childdoc.def}
\childdocforwardprefix[cdocsamp]{cdocsfn}{cdocsch}
%    \end{macrocode}

%\iffalse
%</samplefinal>
%\fi
%
% %%%%%%%%%%%%%%%%%%%%%%%%%%%%%%%%%%%%%%
% \paragraph{Command Line Processing.}
%
% The following three command lines generate the output files
% |cdocscld|, |cdocscl1| and |cdocscl2|
% which should be identical to
% |cdocsdrf|, |cdocsch1| and |cdocsfn2|, respectively:
% \begin{center}
% \begin{tabular}{l}
% |latex -jobname cdocscld \|\\
% |  "\def\version{draft}\input{childdoc.def}\childdocforward{cdocsamp}"|\\
% |latex -jobname cdocscl1 \|\\
% |  "\input{childdoc.def}\childdocforward[cdocsamp]{cdocsch1}"|\\
% |latex -jobname cdocscl2 \|\\
% |  "\def\version{final}\input{childdoc.def}\childdocforward{cdocsch2}"|
% \end{tabular}
% \end{center}
% Note that the trailing backslash on each first line
% merely continues the input to the second line
% (for convenient cut ant paste).
% Furthermore, the command |latex| can be replaced by any
% of its alternative versions such as |pdflatex|.
%
% %%%%%%%%%%%%%%%%%%%%%%%%%%%%%%%%%%%%%%%%%%%%%%%%%%%%%%%%%%%%%%%%%%%%%%%%%%%%%%
% %%%%%%%%%%%%%%%%%%%%%%%%%%%%%%%%%%%%%%%%%%%%%%%%%%%%%%%%%%%%%%%%%%%%%%%%%%%%%%
% \section{Implementation}
%\iffalse
%<*package>
%\fi
%
% This section describes the definitions file |childdoc.def|.

% The definitions cannot be loaded using |\usepackage| or |\RequirePackage|
% which has a mechanism to prevent loading a style file more than once.
% When loading the definitions by means of |\input|
% multiple instances have to be prevented manually:
%\iffalse
%This code needs to be before the `\ProvidesFile' directive
%which is defined at the beginning of this file.
%Therefore it is also placed there and commented out here.
%</package>
%<*discard>
%\fi
%    \begin{macrocode}
\ifdefined\childdocmain\endinput\fi
%    \end{macrocode}
%\iffalse
%</discard>
%<*package>
%\fi
%
% \macro{\ifchilddoc}
% \macro{\ifchilddocmanual}
% The conditional |\ifchilddoc| tells whether a
% child (true) or main (false) document is being compiled.
% The conditional |\ifchilddocmanual| tells whether
% the |\includeonly| mechanism is used (false) or
% the selection of child files must be performed manually (true).
% The definitions initialise to false:
%    \begin{macrocode}
\newif\ifchilddoc
\newif\ifchilddocmanual
%    \end{macrocode}

% \macro{\childdocname}
% \macro{\childdocjob}
% The macro |\childdocname| stores the name of the main document
% to be compiled. The macro |\childdocjob| stores the name of
% the document on which the \LaTeX{} compiler was originally invoked.
% The content of |\jobname| cannot be compared
% to filenames specified in the source due to different catcodes.
% The following code rescans |\jobname|, stores the result
% in |\childdocname| and saves a copy in |\childdocjob|:
%    \begin{macrocode}
\edef\childdocname{\scantokens\expandafter{\jobname\noexpand}}
\let\childdocjob\childdocname
%    \end{macrocode}

% \macro{\childdocdisable}
% The macro |\childdocdisable| prevents the main file
% from being processed more than once.
% At this stage, the main document command |\childdocmain|
% is assumed to be called once again where it should do nothing.
% Any subsequent call to it should prevent
% a secondary processing of the main document
% It overwrites the forwarding commands
% |\childdocof| and |\childdocforward|
% with empty macros to prevent further inclusions of the main document:
%    \begin{macrocode}
\newcommand{\childdocdisable}
{
  \renewcommand{\childdocmain}[1]{\renewcommand{\childdocmain}[1]{\endinput}}
  \renewcommand{\childdocof}[1]{}
  \renewcommand{\childdocby}[2][]{}
  \renewcommand{\childdocforward}[2][]{}
  \renewcommand{\childdocdisable}{}
}
%    \end{macrocode}

% \macro{\childdocmain}
% The macro |\childdocmain| is to be called at the top of the main file
% with nothing or the main filename (without extension) as argument.
% First, it breaks loops.
% If the argument is not empty and does not match |\childdocname|
% (which is set by the first inclusion of |childdoc.def|),
% |\ifchilddoc| is set to true, |\includeonly| is applied to the child file
% and |\jobname| is set to the main file
% (for proper handling of |.aux| files):
%    \begin{macrocode}
\newcommand{\childdocmain}[1]
{
  \childdocdisable\childdocmain{}
  \if?#1?\else
    \begingroup
      \def\childdoctmp{#1}
      \ifx\childdoctmp\childdocname
        \def\childdoctmp{}
      \else
        \def\childdoctmp
        {
          \childdoctrue
          \includeonly{\childdocname}
          \def\childdocjob{#1}
          \def\jobname{#1}
        }
      \fi
      \expandafter
    \endgroup
    \childdoctmp
  \fi
}
%    \end{macrocode}

% \macro{\childdocof}
% The command |\childdocof| redirects
% compilation to the main file |#1|.
%    \begin{macrocode}
\newcommand{\childdocof}[1]
{
  \childdocdisable
  \childdoctrue
  \includeonly{\childdocname}
  \def\jobname{#1}
  \def\childdocjob{#1}
  \input{#1}
}
%    \end{macrocode}

% \macro{\childdocby}
% The command |\childdocby| ....
%    \begin{macrocode}
\newcommand{\childdocby}[2][]
{
  \childdocdisable
  \childdoctrue
  \childdocmanualtrue
  \if?#1?\else
    \def\jobname{#2}
  \fi
  \def\childdocjob{#2}
  \input{#2}
  \endinput
}
%    \end{macrocode}

% \macro{\childdocforward}
% The command |\childdocforward| redirects
% compilation to the main file or
% (if the optional argument is given) a child file.
% Parameters are set as if the main file
% or a child file starting with |\childdocof| was compiled.
% Then compilation is handed over to the main file:
%    \begin{macrocode}
\newcommand{\childdocforward}[2][]
{
  \begingroup
    \if?#1?
      \def\childdoctmp
      {
        \def\childdocname{#2}
        \def\childdocjob{#2}
        \def\jobname{#2}
        \input{#2}
        \endinput
      }
    \else
      \def\childdoctmp
      {
        \childdocdisable
        \def\childdocname{#2}
        \childdoctrue
        \includeonly{#2}
        \def\childdocjob{#1}
        \def\jobname{#1}
        \input{#1}
        \endinput
      }
    \fi
    \expandafter
  \endgroup
  \childdoctmp
}
%    \end{macrocode}

% \macro{\childdocforwardprefix}
% The command |\childdocforwardprefix| redirects
% compilation to the main or a child file by means of a pattern.
% The prefix |#1| in the current filename is replaced by |#2|
% and the suffix of the current filename is kept
% (it is assumed that the filename does not contain the substring `|~~~|'
% which is used as a delimiter).
% Compilation is handed over to the new file by |\childdocforward|:
%    \begin{macrocode}
\newcommand{\childdocforwardprefix}[3][]
{
  \begingroup
    \def\childdocextract #2##1~~~{\def\childdoctmp{\childdocforward[#1]{#3##1}}}
    \expandafter\childdocextract\childdocname~~~
    \expandafter
  \endgroup
  \childdoctmp
}
%    \end{macrocode}

% \macro{\childdoc}
% The deprecated macro |\childdoc| is a legacy version of |\childdocmain|:
%    \begin{macrocode}
\newcommand{\childdoc}{\childdocmain}
%    \end{macrocode}

% \macro{\childdocredirect}
% The deprecated macro |\childdocredirect| is a legacy version
% of |\childdocforward| and |\childdocforwardprefix|:
%    \begin{macrocode}
\newcommand{\childdocredirect}[2][]
{
  \begingroup
    \if?#1?
      \def\childdoctmp{\childdocforward{#2}}
    \else
      \def\childdoctmp{\childdocforwardprefix{#1}{#2}}
    \fi
    \expandafter
  \endgroup
  \childdoctmp
}
%    \end{macrocode}

%\iffalse
%</package>
%\fi
%
\endinput
|
and perform the replacements as outlined below.
Instead of |\childdocmain{|\textit{main}|}| add the following code
to the top of the main file:
%
\begin{center}
\begin{tabular}{l}
|\||ifdefined\childdocname\endinput\||fi\newif\ifchilddoc|\\
|\edef\childdocname{\scantokens\expandafter{\jobname\noexpand}}|\\
|\def\childdocmain{|\textit{main}|}\||ifx\childdocmain\childdocname\||else|\\
|\childdoctrue\includeonly{\childdocname}\let\jobname\childdocmain\||fi|\\
\end{tabular}
\end{center}
%
Instead of |\childdocof{|\textit{main}|}| just include the main file
at the top of each child file:
%
\begin{center}
|\input{|\textit{main}|}|
\end{center}
%
A simple redirection |\childdocforward{|\textit{dest}|}| is achieved by:
%
\begin{center}
|\def\jobname{|\textit{dest}|}\input{\jobname}|
\end{center}
%
The redirection with prefix
|\childdocforwardprefix[|\textit{prefix}|]{|\textit{dest}|}|
is accomplished by:
%
\begin{center}
\begin{tabular}{l}
|{\edef\jobname{\scantokens\expandafter{\jobname\noexpand}}|\\
|\def\redirectjob |\textit{prefix}|#1~~~{\gdef\jobname{|\textit{dest}|#1}}|\\
|\expandafter\redirectjob\jobname~~~}\input{\jobname}|
\end{tabular}
\end{center}

In an alternative approach,
child documents can be compiled by a specific command line
without additional code or specific definitions:
%
\begin{center}
|... -jobname "|\textit{target}|" "|[\textit{flags}]%
|\includeonly{|\textit{dest}|}\input{|\textit{main}|}"|
\end{center}
%

%%%%%%%%%%%%%%%%%%%%%%%%%%%%%%%%%%%%%%%%%%%%%%%%%%%%%%%%%%%%%%%%%%%%%%%%%%%%%%%%
%%%%%%%%%%%%%%%%%%%%%%%%%%%%%%%%%%%%%%%%%%%%%%%%%%%%%%%%%%%%%%%%%%%%%%%%%%%%%%%%
\section{Information}

%%%%%%%%%%%%%%%%%%%%%%%%%%%%%%%%%%%%%%%%%%%%%%%%%%%%%%%%%%%%%%%%%%%%%%%%%%%%%%%%
\subsection{Copyright}

Copyright \copyright{} 2017--2018 Niklas Beisert

This work may be distributed and/or modified under the
conditions of the \LaTeX{} Project Public License, either version 1.3
of this license or (at your option) any later version.
The latest version of this license is in
  \url{http://www.latex-project.org/lppl.txt}
and version 1.3 or later is part of all distributions of \LaTeX{}
version 2005/12/01 or later.

This work has the LPPL maintenance status `maintained'.

The Current Maintainer of this work is Niklas Beisert.

This work consists of the files |README.txt|, |childdoc.ins| and |childdoc.dtx|
as well as the derived files |childdoc.def|, |cdocsamp.tex|
with |cdocsch1.tex|, |cdocsch2.tex|, |cdocspt3.tex|, |cdocspt4.tex|,
|cdocsdrf.tex|, |cdocsfn1.tex|, |cdocsfn2.tex|
as well as |childdoc.pdf|.

%%%%%%%%%%%%%%%%%%%%%%%%%%%%%%%%%%%%%%%%%%%%%%%%%%%%%%%%%%%%%%%%%%%%%%%%%%%%%%%%
\subsection{Files and Installation}

The package consists of the files:
%
\begin{center}
\begin{tabular}{ll}
    |README.txt|   & readme file \\
    |childdoc.ins| & installation file \\
    |childdoc.dtx| & source file \\
    |childdoc.def| & definition file \\
    |cdocsamp.tex| & sample main file \\
    |cdocsch1.tex| & sample include file \\
    |cdocsch2.tex| & sample include file \\
    |cdocspt3.tex| & sample part file \\
    |cdocspt4.tex| & sample part file \\
    |cdocsdrf.tex| & sample redirection file \\
    |cdocsfn1.tex| & sample redirection file \\
    |cdocsfn2.tex| & sample redirection file \\
    |childdoc.pdf| & manual
\end{tabular}
\end{center}
%
The distribution consists of the files
|README.txt|, |childdoc.ins| and |childdoc.dtx|.
%
\begin{itemize}
\item
Run (pdf)\LaTeX{} on |childdoc.dtx|
to compile the manual |childdoc.pdf| (this file).
\item
Run \LaTeX{} on |childdoc.ins| to create the definitions file |childdoc.def|
and the sample |cdocsamp.tex| with include files
|cdocsch1.tex|, |cdocsch2.tex|, |cdocspt3.tex|, |cdocspt4.tex|,
|cdocsdrf.tex|, |cdocsfn1.tex|, |cdocsfn2.tex|.
Then copy the file |childdoc.def| to an appropriate directory of your \LaTeX{}
distribution, e.g.\ \textit{texmf-root}|/tex/latex/childdoc|.
\end{itemize}

%%%%%%%%%%%%%%%%%%%%%%%%%%%%%%%%%%%%%%%%%%%%%%%%%%%%%%%%%%%%%%%%%%%%%%%%%%%%%%%%
\subsection{Related CTAN Packages}

There are several other packages which offer a similar functionality:
%
\begin{itemize}
\item
The packages
\href{http://ctan.org/pkg/docmute}{\textsf{docmute}},
\href{http://ctan.org/pkg/includex}{\textsf{includex}} and
\href{http://ctan.org/pkg/standalone}{\textsf{standalone}}
provide commands to include only the document body of
a child file thus allowing both files to be compiled individually.
\item
The packages \href{http://ctan.org/pkg/subdocs}{\textsf{subdocs}}
and \href{http://ctan.org/pkg/subfiles}{\textsf{subfiles}}
provide structures in which the main and child documents can be
encapsulated and allowing them to be compiled individually.
The inclusion mechanism is different from the conventional |\include|.
\item
The package \href{http://ctan.org/pkg/combine}{\textsf{combine}}
is an elaborate solution to combine several documents into one.
\end{itemize}
%
See also the CTAN topic \href{http://ctan.org/topic/subdocs}{\textsf{subdocs}}
for further related packages.
The present package differs from the above solutions in that
a document structure constructed with the conventional |\include| mechanism
just needs two extra commands at the top of every file
such that all constituent files can be compiled individually.

%%%%%%%%%%%%%%%%%%%%%%%%%%%%%%%%%%%%%%%%%%%%%%%%%%%%%%%%%%%%%%%%%%%%%%%%%%%%%%%%
%\subsection{Feature Suggestions}
%
%The following is a list of features which may be useful for future
%versions of this package:
%%
%\begin{itemize}
%\item
%\ldots
%\end{itemize}

%%%%%%%%%%%%%%%%%%%%%%%%%%%%%%%%%%%%%%%%%%%%%%%%%%%%%%%%%%%%%%%%%%%%%%%%%%%%%%%%
\subsection{Revision History}

%%%%%%%%%%%%%%%%%%%%%%%%%%%%%%%%%%%%%%%%
\paragraph{v2.0:} 2018/12/30

\begin{itemize}
\item
immediate forward processing
\item
added |\childdocby| mechanism
\item
manual restructured
\end{itemize}

%%%%%%%%%%%%%%%%%%%%%%%%%%%%%%%%%%%%%%%%
\paragraph{v1.6:} 2018/01/17

\begin{itemize}
\item
application for development of include files
\item
corrections to manual
\end{itemize}

%%%%%%%%%%%%%%%%%%%%%%%%%%%%%%%%%%%%%%%%
\paragraph{v1.5:} 2017/05/21

\begin{itemize}
\item
more complete structuring introduced
\item
|\childdocof| introduced
\item
|\childdoc| renamed to |\childdocmain|
\item
|\childredirect| renamed to |\childdocforward| and |\childdocforwardprefix|
and functionality expanded
\end{itemize}

%%%%%%%%%%%%%%%%%%%%%%%%%%%%%%%%%%%%%%%%
\paragraph{v1.0:} 2017/04/27

\begin{itemize}
\item
manual and install package
\item
first version published on CTAN
\end{itemize}

%%%%%%%%%%%%%%%%%%%%%%%%%%%%%%%%%%%%%%%%
\paragraph{v0.6:} 2017/04/26

\begin{itemize}
\item
redirection mechanism added
\end{itemize}

%%%%%%%%%%%%%%%%%%%%%%%%%%%%%%%%%%%%%%%%
\paragraph{v0.5:} 2017/04/26

\begin{itemize}
\item
functionality in definition file
\end{itemize}


%%%%%%%%%%%%%%%%%%%%%%%%%%%%%%%%%%%%%%%%%%%%%%%%%%%%%%%%%%%%%%%%%%%%%%%%%%%%%%%%
%%%%%%%%%%%%%%%%%%%%%%%%%%%%%%%%%%%%%%%%%%%%%%%%%%%%%%%%%%%%%%%%%%%%%%%%%%%%%%%%
%%%%%%%%%%%%%%%%%%%%%%%%%%%%%%%%%%%%%%%%%%%%%%%%%%%%%%%%%%%%%%%%%%%%%%%%%%%%%%%%
\appendix

\settowidth\MacroIndent{\rmfamily\scriptsize 000\ }

 \DocInput{childdoc.dtx}

\end{document}
%</driver>
% \fi
%
% %%%%%%%%%%%%%%%%%%%%%%%%%%%%%%%%%%%%%%%%%%%%%%%%%%%%%%%%%%%%%%%%%%%%%%%%%%%%%%
% %%%%%%%%%%%%%%%%%%%%%%%%%%%%%%%%%%%%%%%%%%%%%%%%%%%%%%%%%%%%%%%%%%%%%%%%%%%%%%
% \section{Sample}
%\iffalse
%<*samplemain>
%\fi
%
% The following presents a sample document
% with two chapters, two parts, a title page,
% a compile flag as well as three forwarding files to set the flag.
% It consists of eight |.tex| files:
% \begin{center}
% \begin{tabular}{ll}
% |cdocsamp.tex|&main file\\
% |cdocsch1.tex|&include file for chapter 1\\
% |cdocsch2.tex|&include file for chapter 2\\
% |cdocspt3.tex|&include file for part 3\\
% |cdocspt4.tex|&include file for part 4\\
% |cdocsdrf.tex|&forwarding file for main file in draft mode\\
% |cdocsfi1.tex|&forwarding file for final version of chapter 1\\
% |cdocsfi2.tex|&forwarding file for final version of chapter 2\\
% \end{tabular}
% \end{center}
% Each of the eight files can be compiled directly by the \LaTeX{} compiler.
%
% %%%%%%%%%%%%%%%%%%%%%%%%%%%%%%%%%%%%%%
% \paragraph{Main File.}
%
% The main file is called |cdocsamp.tex|.
%
% Load the \textsf{childdoc} definitions and
% declare the filename for the main document:
%    \begin{macrocode}
% \iffalse
%
% childdoc.dtx Copyright (C) 2017-2018 Niklas Beisert
%
% This work may be distributed and/or modified under the
% conditions of the LaTeX Project Public License, either version 1.3
% of this license or (at your option) any later version.
% The latest version of this license is in
%   http://www.latex-project.org/lppl.txt
% and version 1.3 or later is part of all distributions of LaTeX
% version 2005/12/01 or later.
%
% This work has the LPPL maintenance status `maintained'.
%
% The Current Maintainer of this work is Niklas Beisert.
%
% This work consists of the files childdoc.dtx and childdoc.ins
% and the derived files childdoc.def and cdocsamp.tex with
% cdocsch1.tex, cdocsch2.tex, cdocsdrf.tex, cdocsfn1.tex, cdocsfn2.tex.
%
%<package>\ifdefined\childdocmain\endinput\fi
%<package>\ProvidesFile{childdoc.def}[2018/12/30 v2.0 child document driver]
%<samplemain>\ProvidesFile{cdocsamp.tex}[2018/12/30 v2.0 sample for childdoc]
%<*driver>
%\ProvidesFile{childdoc.drv}[2018/12/30 v2.0 childdoc reference manual file]
\PassOptionsToClass{10pt,a4paper}{article}
\documentclass{ltxdoc}

\usepackage[margin=35mm]{geometry}
\usepackage{hyperref}
\usepackage{hyperxmp}
\usepackage[usenames]{color}

\hypersetup{colorlinks=true}
\hypersetup{pdfstartview=FitH}
\hypersetup{pdfpagemode=UseNone}
\hypersetup{pdfsource={}}
\hypersetup{pdflang={en-UK}}
\hypersetup{pdfcopyright={Copyright 2017-2018 Niklas Beisert.
  This work may be distributed and/or modified under the
  conditions of the LaTeX Project Public License, either version 1.3
  of this license or (at your option) any later version.}}
\hypersetup{pdflicenseurl={http://www.latex-project.org/lppl.txt}}
\hypersetup{pdfcontactaddress={ETH Zurich, ITP, HIT K,
  Wolfgang-Pauli-Strasse 27}}
\hypersetup{pdfcontactpostcode={8093}}
\hypersetup{pdfcontactcity={Zurich}}
\hypersetup{pdfcontactcountry={Switzerland}}
\hypersetup{pdfcontactemail={nbeisert@itp.phys.ethz.ch}}
\hypersetup{pdfcontacturl={http://people.phys.ethz.ch/\xmptilde nbeisert/}}

\newcommand{\secref}[1]{\hyperref[#1]{section \ref*{#1}}}

\parskip1ex
\parindent0pt
\let\olditemize\itemize
\def\itemize{\olditemize\parskip0pt}

\begin{document}

\title{The \textsf{childdoc} Package}
\hypersetup{pdftitle={The childdoc Package}}
\author{Niklas Beisert\\[2ex]
  Institut f\"ur Theoretische Physik\\
  Eidgen\"ossische Technische Hochschule Z\"urich\\
  Wolfgang-Pauli-Strasse 27, 8093 Z\"urich, Switzerland\\[1ex]
  \href{mailto:nbeisert@itp.phys.ethz.ch}
  {\texttt{nbeisert@itp.phys.ethz.ch}}}
\hypersetup{pdfauthor={Niklas Beisert}}
\hypersetup{pdfsubject={Manual for the LaTeX2e Package childdoc}}
\date{30 December 2018, \textsf{v2.0}}
\maketitle

\begin{abstract}\noindent
\textsf{childdoc} is a \LaTeXe{} package
that enables the direct compilation
of document sections included by |\include|
to individual files.
\end{abstract}

\begingroup
\parskip0ex
\tableofcontents
\endgroup

%%%%%%%%%%%%%%%%%%%%%%%%%%%%%%%%%%%%%%%%%%%%%%%%%%%%%%%%%%%%%%%%%%%%%%%%%%%%%%%%
%%%%%%%%%%%%%%%%%%%%%%%%%%%%%%%%%%%%%%%%%%%%%%%%%%%%%%%%%%%%%%%%%%%%%%%%%%%%%%%%
\section{Introduction}

\LaTeX{} provides a mechanism to structure a large document (such as a book)
into a main file and several child files (containing the chapters)
using the |\include| command.
This mechanism is beneficial for documents
which span hundreds of pages in order to
make the source file(s) more manageable.
Moreover, compilation can be restricted to
selected child files by means of the |\includeonly| command.
The latter feature can be used to reduce the compilation time while editing
(this was significantly more useful in the earlier days of \LaTeX{})
or to generate a smaller document which is easier to navigate.
Another application of |\includeonly| is to generate
documents consisting of selected parts of the complete document.

However, there are a few drawbacks of the plain |\include| mechanism:
\begin{itemize}
\item
The child files cannot be compiled on their own,
they can only be compiled via the main file.
A naive editing environment
(such as a text editor with an option
to have the current file processed by \LaTeX)
may require one to switch to the main file before compiling;
attempting to compile the child file produces errors.
\item
The main file must be modified (each time)
to adjust the |\includeonly| command
to the present needs. This easily leaves the main file in a messy state.
\item
The generated document will always carry the filename
of the main document. This is inconvenient if
several child files are to be compiled and
to be kept for distribution.
\end{itemize}

The present package provides a simple interface
to make child files individually compilable by \LaTeX{}.
Compiling a child file then has the same effect as compiling
the main file with an |\includeonly| command
to select the appropriate child.
Moreover the generated document will carry the name of the child
rather than the main file.
This resolves all three above issues.

This feature is meant to make the editing of books,
thesis documents and lecture notes somewhat more convenient.
However, the package can also be used efficiently for
composing a series of documents (such as exercise sheets)
which are typically distributed individually.
It then assists the author in generating the individual documents
(potentially in different versions)
as well as a document containing the collected series.
Another application is in developing style files
or other kinds of included material
where compilation of the style file could redirect
to a sample or test file.

%%%%%%%%%%%%%%%%%%%%%%%%%%%%%%%%%%%%%%%%%%%%%%%%%%%%%%%%%%%%%%%%%%%%%%%%%%%%%%%%
%%%%%%%%%%%%%%%%%%%%%%%%%%%%%%%%%%%%%%%%%%%%%%%%%%%%%%%%%%%%%%%%%%%%%%%%%%%%%%%%
\section{Usage}

First of all, the package \textsf{childdoc} is \emph{not} a standard
\LaTeXe{} |.sty| style file! Therefore it needs to be invoked in
a non-standard way.

%%%%%%%%%%%%%%%%%%%%%%%%%%%%%%%%%%%%%%%%%%%%%%%%%%%%%%%%%%%%%%%%%%%%%%%%%%%%%%%%
\subsection{Included Files}
\label{sec:include}

%%%%%%%%%%%%%%%%%%%%%%%%%%%%%%%%%%%%%%%%
\DescribeMacro{\childdocmain}
To use the package, add the commands
\begin{center}
\begin{tabular}{l}
|\input{childdoc.def}|\\
|\childdocmain{}|\\
\end{tabular}
\end{center}
at the very top of the main \LaTeX{} file,
in particular \emph{before} the |\documentclass| statement!
The argument of |\childdocmain| should be left empty
(but it must be present).

%%%%%%%%%%%%%%%%%%%%%%%%%%%%%%%%%%%%%%%%
\DescribeMacro{\childdocof}
Furthermore, add the commands
\begin{center}
\begin{tabular}{l}
|\input{childdoc.def}|\\
|\childdocof{|\textit{main}|}|\\
\end{tabular}
\end{center}
at the top of every child file \textit{child}
which is included by |\include{|\textit{child}|}|
from within the main file
(or at least for those files to be compiled individually).
The argument \textit{main} must be the filename of the main file.

There are a couple of
considerations in setting up the main and child documents:

%%%%%%%%%%%%%%%%%%%%%%%%%%%%%%%%%%%%%%%%
\paragraph{Restrictions.}

Please note the following restrictions:
\begin{itemize}
\item
|\childdocmain| must be called with one argument \textit{main}
to ensure compatibility with earlier version of the package.
It must either be empty (|\childdocmain{}|)
or precisely match the filename of the main file in which it is specified.
See \secref{sec:detection} for further information.
\item
The filename \textit{main} must be specified without the |.tex| extension.
\item
The filename \textit{main} is case sensitive
(even in case-insensitive file systems)
due to internal string comparison.
\item
The argument \textit{main} should be fully expanded, it cannot be a macro.
\item
Subdirectories and special characters should be avoided in filenames.
\item
The command |\childdocmain{|\textit{main}|}| must be followed by a whitespace.
It should not be followed immediately by another command
or by a comment mark `|%|'.
This is because the \TeX{} parser reads the token immediately following
the argument of |\childdocmain| and puts it
at the beginning of every child section;
however, a white\-space is ignored.
\end{itemize}

%%%%%%%%%%%%%%%%%%%%%%%%%%%%%%%%%%%%%%%%
\paragraph{Content of Main File.}

It is advisable to place all content in the child files included by |\include|.
Any output contained in the main file will appear in all child documents
unless suppressed manually;
it cannot be suppressed automatically by the |\includeonly| directive
and thus should normally be avoided.
A method to include some content in the main file
by means of conditional processing is described in \secref{sec:conditional}.

%%%%%%%%%%%%%%%%%%%%%%%%%%%%%%%%%%%%%%%%
\paragraph{Page Numbering.}

When only a part of the document is compiled,
the appropriate numbering of pages
(as well as other status parameters)
is determined from the |.aux| files.
The latter contain information from previous passes.
However this information needs to propagate through
all intermediate child documents.
Therefore the page numbering in child documents may well
be inconsistent until the complete document is compiled at least once.

A useful (if unconventional) way to always ensure a consistent
page numbering is to restart the numbering in each child document
and denote the pages by `\textit{child}|.|\textit{page}'
where \textit{child} represents the chapter/section number of the child file.
This can be achieved by the command
|\numberwithin{page}{|\textit{child}|}|
of the \textsf{amsmath} package
where \textit{child} can be |chapter| or |section|
depending on the chosen structuring.
Alternatively, one can modify the macro |\thepage| appropriately
and reset the counter |page| at the start of each child file.

%%%%%%%%%%%%%%%%%%%%%%%%%%%%%%%%%%%%%%%%%%%%%%%%%%%%%%%%%%%%%%%%%%%%%%%%%%%%%%%%
\subsection{Conditional Processing}
\label{sec:conditional}

The package provides a mechanism to compile different versions
of a document. To customise the versions further some conditional processing
can come in handy to distinguish which version is being compiled.
The package provides two macros to describe the compilation context:

%%%%%%%%%%%%%%%%%%%%%%%%%%%%%%%%%%%%%%%%
\DescribeMacro{\ifchilddoc}
The conditional |\ifchilddoc| distinguishes between the compilation of
child documents and the main document:
%
\begin{center}
|\ifchilddoc |\textit{child-code}| |[|\||else |\textit{main-code}]| \||fi|
\end{center}

%%%%%%%%%%%%%%%%%%%%%%%%%%%%%%%%%%%%%%%%
\DescribeMacro{\childdocname}
\DescribeMacro{\childdocjob}
The macro |\childdocname| contains the filename (without extension)
of the main or child file being processed.
Note that |\childdocjob| will always contain the name of the main file.

%%%%%%%%%%%%%%%%%%%%%%%%%%%%%%%%%%%%%%%%
\paragraph{Title Page.}

Conditional processing can be used to include a title or banner page
in the main document when proper precautions are taken.
Importantly, the code in the main file should ensure that the page counter
(as well as other status parameters which are stored in the |.aux| files)
takes the same value after the conditional processing.
Otherwise the page numbers may take divergent values
depending on which part is compiled.

For example, a title page could be declared by:
%
\begin{center}
\begin{tabular}{l}
|\ifchilddoc\||else|\\
|\addtocounter{page}{-1}|\\
\textit{code for title page}\\
|\newpage|\\
|\||fi|
\end{tabular}
\end{center}
%
A banner page for the child documents can be generated by:
%
\begin{center}
\begin{tabular}{l}
|\ifchilddoc|\\
|\addtocounter{page}{-1}|\\
\textit{code for banner page}\\
|\newpage|\\
|\||fi|
\end{tabular}
\end{center}
%
Here one could write a message such as:
\begin{center}
|This is the part \childdocname{} of \childdocjob{}.|
\end{center}

%%%%%%%%%%%%%%%%%%%%%%%%%%%%%%%%%%%%%%%%%%%%%%%%%%%%%%%%%%%%%%%%%%%%%%%%%%%%%%%%
\subsection{Flags}
\label{sec:flags}

The package makes it easy to generate different versions
of the main or child documents.
To this end compilation flags can be defined
and assigned different default values.
They will be particularly useful in conjunction
with the forwarding mechanism described in \secref{sec:forward}.

For example, it may be useful to have a flag |\version|
which can be set to |draft| or |final|.
The document source will contain some conditional code
depending on the value of |\version|.
Suppose further, the flag should default to |final| for the main file
and to |draft| for child files
which is a natural assignment for editing the document.
This is achieved by placing the following code
in the preamble of the main document
(below the |\childdocmain| directive):
%
\begin{center}
\begin{tabular}{l}
|\ifchilddoc|\\
|\providecommand{\version}{draft}|\\
|\||else|\\
|\providecommand{\version}{final}|\\
|\||fi|
\end{tabular}
\end{center}
%
The definition by |\providecommand| makes sure
that previous definitions are not overwritten.
Further statements |\providecommand{\version}{...}|
can thus be added before the above code to override it.

For the main file, one might add a line
(between |\childdocmain| and the above block)
%
\begin{center}
|%\ifchilddoc\||else\providecommand{\version}{draft}\||fi|
\end{center}
%
which can be uncommented to produce a draft version.
Likewise one can add a line to the very top of a child file
(above the |\childdocof{|\textit{main}|}| directive)
%
\begin{center}
|%\providecommand{\version}{final}|
\end{center}
%
which can be uncommented to produce the final version of this child document.

%%%%%%%%%%%%%%%%%%%%%%%%%%%%%%%%%%%%%%%%%%%%%%%%%%%%%%%%%%%%%%%%%%%%%%%%%%%%%%%%
\subsection{Forwarding}
\label{sec:forward}

Different versions of the main or child documents
using compilation flags as described in \secref{sec:flags}
can be (permanently) stored in different files
for convenient compilation, viewing and distribution.
To this end, the package defines a command
to pass on compilation to a different file:

%%%%%%%%%%%%%%%%%%%%%%%%%%%%%%%%%%%%%%%%
\DescribeMacro{\childdocforward}
The command |\childdocforward| redirects processing to
another source file:
%
\begin{center}
\begin{tabular}{l}
|\input{childdoc.def}|\\
|\childdocforward[|\textit{main}|]{|\textit{dest}|}|\\
\end{tabular}
\end{center}
%
The argument \textit{dest} is the destination file
(without extension).
It should be the main file or one of the child files.
Note that further \textsf{childdoc} directives
such as |\childdocof| and |\childdocforward|
in the indicated file will be processed in this form.
The optional argument \textit{main}
passes on directly to the main file \textit{main}
while pretending to compile the child \textit{dest}.
This form behaves as if \textit{dest}
issues |\childdocof{|\textit{main}|}| right away,
and no further \textsf{childdoc} directives will be processed.

%%%%%%%%%%%%%%%%%%%%%%%%%%%%%%%%%%%%%%%%
\DescribeMacro{\...prefix}
In the alternative form |\childdocforwardprefix|,
%
\begin{center}
\begin{tabular}{l}
|\input{childdoc.def}|\\
|\childdocforwardprefix[|\textit{main}|]{|\textit{prefix}|}{|\textit{dest}|}|
\end{tabular}
\end{center}
%
the destination file is determined by a pattern
depending on the current file:
To make this work, the current file must be called
`{\textit{prefix}\hspace{0.2em}\textit{suffix}}'
with \textit{prefix} matching precisely the argument.
Processing is then passed on to the file
`{\textit{dest}\hspace{0.2em}\textit{suffix}}'.
Surely, the same effect is achieved by
directly specifying the
argument `{\textit{dest}\hspace{0.2em}\textit{suffix}}'
in the first form.
However, that requires to set up a different file
for each child. With the alternative form of the command
all these files can have exactly the same content
which simplifies setting them up and maintaining them.

For example, the following file |draft.tex|
with a compilation flag |\version| as described in \secref{sec:flags}
compiles the main document as a draft:
%
\begin{center}
\begin{tabular}{l}
|\def\version{draft}|\\
|\input{childdoc.def}|\\
|\childdocforward{|\textit{main}|}|
\end{tabular}
\end{center}
%
Likewise, the following files |final|\textit{nn}|.tex|
compile the final version of the child document
|child|\textit{nn}|.tex|:
%
\begin{center}
\begin{tabular}{l}
|\def\version{final}|\\
|\input{childdoc.def}|\\
|\childdocforwardprefix{final}{child}|
\end{tabular}
\end{center}
%

Note that when several versions of a main file and/or of each child file
are to be generated, it may be convenient to set up a |Makefile| or
shell script to automatise the process.

%%%%%%%%%%%%%%%%%%%%%%%%%%%%%%%%%%%%%%%%%%%%%%%%%%%%%%%%%%%%%%%%%%%%%%%%%%%%%%%%
\subsection{Command Line Processing}
\label{sec:commandline}

The effect of redirection files can also be achieved by invoking
the \LaTeX{} compiler with a more elaborate command line.
Most conveniently this should be done as part
of a shell script or a |Makefile|.

When using \textsf{childdoc} in the main file, the following
command lines effectively perform a redirection
(note that depending on the shell being used,
backslashes may have to be doubled: `|\|' $\to$ `|\\|'):
%
\begin{center}
|... -jobname "|\textit{target}|" |\\|"|[\textit{flags}]%
|\input{childdoc.def}\childdocforward[|\textit{main}|]{|\textit{dest}|}"|
\end{center}
%
Here \textit{target} is the name of the output file,
\textit{main} is the name of the main file
and \textit{dest} is the name of the main or child file to be processed
(all filenames without extensions).
The optional argument \textit{main} can be omitted
if \textit{main} matches \textit{dest}.
Optionally, compilation \textit{flags} can be defined via |\def| commands.
This command line makes the \TeX{} engine believe
it is compiling the file \textit{target}
whose content is specified as the latter parameter.
The provided code then forwards the processing to
\textit{main} or \textit{dest} as described in \secref{sec:forward}.

%%%%%%%%%%%%%%%%%%%%%%%%%%%%%%%%%%%%%%%%%%%%%%%%%%%%%%%%%%%%%%%%%%%%%%%%%%%%%%%%
\subsection{Include by Input}
\label{sec:input}

Including child documents by |\include| has some restrictions by design.
Most notably, the content of a child document always occupies
its own set of pages; pages cannot be shared between child documents.
Usually, this behaviour makes perfect sense
because each child document contain an essential part of the document.
However, in some situations it may be desirable to compose
a document from a collection of parts
without having mandatory page breaks between then.
For this case, the package
provides a mechanism to include parts
by |\input| which can also be processed individually.
However, by construction this mechanism
requires manual handling of the content to be output.

%%%%%%%%%%%%%%%%%%%%%%%%%%%%%%%%%%%%%%%%
\DescribeMacro{\ifchilddocmanual}
The main file should be prepared as usual, see \secref{sec:include}.
However, the document body must make a distinction
between processing of an individual part and of the main document, e.g.:
%
\begin{center}
\begin{tabular}{l}
|\ifchilddocmanual|\\
|\input{\childdocname}|\\
|\||else|\\
\textit{document body with }|\input{|\textit{part}|}|\\
|\||fi|
\end{tabular}
\end{center}
%
The conditional |\ifchilddocmanual| is true whenever
a part to be included by |\input| is being compiled,
and the name of the part is stored in |\childdocname|.

%%%%%%%%%%%%%%%%%%%%%%%%%%%%%%%%%%%%%%%%
\DescribeMacro{\childdocby}
Each part to be included by |\input| should start with:
%
\begin{center}
\begin{tabular}{l}
|\input{childdoc.def}|\\
|\childdocby{|\textit{main}|}|\\
\end{tabular}
\end{center}
%
The directive |\childdocby| is similar to |\childdocof|
described in \secref{sec:include},
but the subsequent selection of content must be done manually.
To that end, both |\ifchilddoc| and |\ifchilddocmanual|
will be true upon processing of a part,
and the name of the part is stored in |\childdocname|.
Note that |\jobname| will be set to the filename of the current part
so that each part receives an individual |.aux| file
that does not interfere with the |.aux| file(s) of the main document.
This behaviour can be altered by the alternative form
|\childdocby[*]{|\textit{main}|}| (with a non-empty optional argument)
which uses the |.aux| file of the main document
by setting |\jobname| to \textit{main}.

%%%%%%%%%%%%%%%%%%%%%%%%%%%%%%%%%%%%%%%%%%%%%%%%%%%%%%%%%%%%%%%%%%%%%%%%%%%%%%%%
\subsection{Driver Development}
\label{sec:driver}

The \textsf{childdoc} mechanism can also be use for the development
of definition files such as \LaTeX{} styles or classes.
This case differs from the above setup with multiple parts
included by |\include| in that no |\includeonly| should be invoked.
This can be achieved by starting the include file
(before |\ProvidesPackage|) with:
%
\begin{center}
\begin{tabular}{l}
|\input{childdoc.def}|\\
|\childdocforward{|\textit{main}|}|\\
\end{tabular}
\end{center}
%
or alternatively with:
%
\begin{center}
\begin{tabular}{l}
|\input{childdoc.def}|\\
|\childdocby{|\textit{main}|}|\\
\end{tabular}
\end{center}
%
Both forms have slightly different effects as described above.
The main file is prepared as usual, see \secref{sec:include}.

%%%%%%%%%%%%%%%%%%%%%%%%%%%%%%%%%%%%%%%%%%%%%%%%%%%%%%%%%%%%%%%%%%%%%%%%%%%%%%%%
\subsection{Legacy Detection}
\label{sec:detection}

The directive |\childdocmain| in the main file can detect
whether the complete document or merely a child is to be compiled
even without using the directive |\childdocof|.
This method is deprecated because it is less robust
and there is no compelling reason to use it;
it is merely provided for backward compatibility
and it may be removed in future versions.

If the detection mechanism is to be used,
it is mandatory to correctly specify
the filename of the main file as the argument of |\childdocmain|:
%
\begin{center}
\begin{tabular}{l}
|\input{childdoc.def}|\\
|\childdocmain{|\textit{main}|}|\\
\end{tabular}
\end{center}
%
If |\jobname| does not match the argument \textit{main} of |\childdocmain|,
it is assumed that |\jobname| points to the child file to be compiled.
When using |\childdocmain| with the main file specified as argument,
it suffices to start a child file
with just |\input{|\textit{main}|}|
without loading of the package and using |\childdocof|.
If instead all processing is done
with the appropriate \textsf{childdoc} directives,
the argument of \textit{main} of |\childdocmain| can be empty.

An alternative version of the command line processing described
in \secref{sec:commandline} using the detection mechanism reads:
%
\begin{center}
|... -jobname "|\textit{target}|" "|[\textit{flags}]%
[|\def\jobname{|\textit{dest}|}|]|\input{|\textit{main}|}"|
\end{center}

%%%%%%%%%%%%%%%%%%%%%%%%%%%%%%%%%%%%%%%%%%%%%%%%%%%%%%%%%%%%%%%%%%%%%%%%%%%%%%%%
\subsection{Manual Code}
\label{sec:manual}

In case one cannot be certain whether the definitions file |childdoc.def|
is installed on the target \TeX{} distribution
and one prefers not to ship it,
it is conceivable to paste a few relevant commands into the sources.

To that end, drop all statements |\input{childdoc.def}|
and perform the replacements as outlined below.
Instead of |\childdocmain{|\textit{main}|}| add the following code
to the top of the main file:
%
\begin{center}
\begin{tabular}{l}
|\||ifdefined\childdocname\endinput\||fi\newif\ifchilddoc|\\
|\edef\childdocname{\scantokens\expandafter{\jobname\noexpand}}|\\
|\def\childdocmain{|\textit{main}|}\||ifx\childdocmain\childdocname\||else|\\
|\childdoctrue\includeonly{\childdocname}\let\jobname\childdocmain\||fi|\\
\end{tabular}
\end{center}
%
Instead of |\childdocof{|\textit{main}|}| just include the main file
at the top of each child file:
%
\begin{center}
|\input{|\textit{main}|}|
\end{center}
%
A simple redirection |\childdocforward{|\textit{dest}|}| is achieved by:
%
\begin{center}
|\def\jobname{|\textit{dest}|}\input{\jobname}|
\end{center}
%
The redirection with prefix
|\childdocforwardprefix[|\textit{prefix}|]{|\textit{dest}|}|
is accomplished by:
%
\begin{center}
\begin{tabular}{l}
|{\edef\jobname{\scantokens\expandafter{\jobname\noexpand}}|\\
|\def\redirectjob |\textit{prefix}|#1~~~{\gdef\jobname{|\textit{dest}|#1}}|\\
|\expandafter\redirectjob\jobname~~~}\input{\jobname}|
\end{tabular}
\end{center}

In an alternative approach,
child documents can be compiled by a specific command line
without additional code or specific definitions:
%
\begin{center}
|... -jobname "|\textit{target}|" "|[\textit{flags}]%
|\includeonly{|\textit{dest}|}\input{|\textit{main}|}"|
\end{center}
%

%%%%%%%%%%%%%%%%%%%%%%%%%%%%%%%%%%%%%%%%%%%%%%%%%%%%%%%%%%%%%%%%%%%%%%%%%%%%%%%%
%%%%%%%%%%%%%%%%%%%%%%%%%%%%%%%%%%%%%%%%%%%%%%%%%%%%%%%%%%%%%%%%%%%%%%%%%%%%%%%%
\section{Information}

%%%%%%%%%%%%%%%%%%%%%%%%%%%%%%%%%%%%%%%%%%%%%%%%%%%%%%%%%%%%%%%%%%%%%%%%%%%%%%%%
\subsection{Copyright}

Copyright \copyright{} 2017--2018 Niklas Beisert

This work may be distributed and/or modified under the
conditions of the \LaTeX{} Project Public License, either version 1.3
of this license or (at your option) any later version.
The latest version of this license is in
  \url{http://www.latex-project.org/lppl.txt}
and version 1.3 or later is part of all distributions of \LaTeX{}
version 2005/12/01 or later.

This work has the LPPL maintenance status `maintained'.

The Current Maintainer of this work is Niklas Beisert.

This work consists of the files |README.txt|, |childdoc.ins| and |childdoc.dtx|
as well as the derived files |childdoc.def|, |cdocsamp.tex|
with |cdocsch1.tex|, |cdocsch2.tex|, |cdocspt3.tex|, |cdocspt4.tex|,
|cdocsdrf.tex|, |cdocsfn1.tex|, |cdocsfn2.tex|
as well as |childdoc.pdf|.

%%%%%%%%%%%%%%%%%%%%%%%%%%%%%%%%%%%%%%%%%%%%%%%%%%%%%%%%%%%%%%%%%%%%%%%%%%%%%%%%
\subsection{Files and Installation}

The package consists of the files:
%
\begin{center}
\begin{tabular}{ll}
    |README.txt|   & readme file \\
    |childdoc.ins| & installation file \\
    |childdoc.dtx| & source file \\
    |childdoc.def| & definition file \\
    |cdocsamp.tex| & sample main file \\
    |cdocsch1.tex| & sample include file \\
    |cdocsch2.tex| & sample include file \\
    |cdocspt3.tex| & sample part file \\
    |cdocspt4.tex| & sample part file \\
    |cdocsdrf.tex| & sample redirection file \\
    |cdocsfn1.tex| & sample redirection file \\
    |cdocsfn2.tex| & sample redirection file \\
    |childdoc.pdf| & manual
\end{tabular}
\end{center}
%
The distribution consists of the files
|README.txt|, |childdoc.ins| and |childdoc.dtx|.
%
\begin{itemize}
\item
Run (pdf)\LaTeX{} on |childdoc.dtx|
to compile the manual |childdoc.pdf| (this file).
\item
Run \LaTeX{} on |childdoc.ins| to create the definitions file |childdoc.def|
and the sample |cdocsamp.tex| with include files
|cdocsch1.tex|, |cdocsch2.tex|, |cdocspt3.tex|, |cdocspt4.tex|,
|cdocsdrf.tex|, |cdocsfn1.tex|, |cdocsfn2.tex|.
Then copy the file |childdoc.def| to an appropriate directory of your \LaTeX{}
distribution, e.g.\ \textit{texmf-root}|/tex/latex/childdoc|.
\end{itemize}

%%%%%%%%%%%%%%%%%%%%%%%%%%%%%%%%%%%%%%%%%%%%%%%%%%%%%%%%%%%%%%%%%%%%%%%%%%%%%%%%
\subsection{Related CTAN Packages}

There are several other packages which offer a similar functionality:
%
\begin{itemize}
\item
The packages
\href{http://ctan.org/pkg/docmute}{\textsf{docmute}},
\href{http://ctan.org/pkg/includex}{\textsf{includex}} and
\href{http://ctan.org/pkg/standalone}{\textsf{standalone}}
provide commands to include only the document body of
a child file thus allowing both files to be compiled individually.
\item
The packages \href{http://ctan.org/pkg/subdocs}{\textsf{subdocs}}
and \href{http://ctan.org/pkg/subfiles}{\textsf{subfiles}}
provide structures in which the main and child documents can be
encapsulated and allowing them to be compiled individually.
The inclusion mechanism is different from the conventional |\include|.
\item
The package \href{http://ctan.org/pkg/combine}{\textsf{combine}}
is an elaborate solution to combine several documents into one.
\end{itemize}
%
See also the CTAN topic \href{http://ctan.org/topic/subdocs}{\textsf{subdocs}}
for further related packages.
The present package differs from the above solutions in that
a document structure constructed with the conventional |\include| mechanism
just needs two extra commands at the top of every file
such that all constituent files can be compiled individually.

%%%%%%%%%%%%%%%%%%%%%%%%%%%%%%%%%%%%%%%%%%%%%%%%%%%%%%%%%%%%%%%%%%%%%%%%%%%%%%%%
%\subsection{Feature Suggestions}
%
%The following is a list of features which may be useful for future
%versions of this package:
%%
%\begin{itemize}
%\item
%\ldots
%\end{itemize}

%%%%%%%%%%%%%%%%%%%%%%%%%%%%%%%%%%%%%%%%%%%%%%%%%%%%%%%%%%%%%%%%%%%%%%%%%%%%%%%%
\subsection{Revision History}

%%%%%%%%%%%%%%%%%%%%%%%%%%%%%%%%%%%%%%%%
\paragraph{v2.0:} 2018/12/30

\begin{itemize}
\item
immediate forward processing
\item
added |\childdocby| mechanism
\item
manual restructured
\end{itemize}

%%%%%%%%%%%%%%%%%%%%%%%%%%%%%%%%%%%%%%%%
\paragraph{v1.6:} 2018/01/17

\begin{itemize}
\item
application for development of include files
\item
corrections to manual
\end{itemize}

%%%%%%%%%%%%%%%%%%%%%%%%%%%%%%%%%%%%%%%%
\paragraph{v1.5:} 2017/05/21

\begin{itemize}
\item
more complete structuring introduced
\item
|\childdocof| introduced
\item
|\childdoc| renamed to |\childdocmain|
\item
|\childredirect| renamed to |\childdocforward| and |\childdocforwardprefix|
and functionality expanded
\end{itemize}

%%%%%%%%%%%%%%%%%%%%%%%%%%%%%%%%%%%%%%%%
\paragraph{v1.0:} 2017/04/27

\begin{itemize}
\item
manual and install package
\item
first version published on CTAN
\end{itemize}

%%%%%%%%%%%%%%%%%%%%%%%%%%%%%%%%%%%%%%%%
\paragraph{v0.6:} 2017/04/26

\begin{itemize}
\item
redirection mechanism added
\end{itemize}

%%%%%%%%%%%%%%%%%%%%%%%%%%%%%%%%%%%%%%%%
\paragraph{v0.5:} 2017/04/26

\begin{itemize}
\item
functionality in definition file
\end{itemize}


%%%%%%%%%%%%%%%%%%%%%%%%%%%%%%%%%%%%%%%%%%%%%%%%%%%%%%%%%%%%%%%%%%%%%%%%%%%%%%%%
%%%%%%%%%%%%%%%%%%%%%%%%%%%%%%%%%%%%%%%%%%%%%%%%%%%%%%%%%%%%%%%%%%%%%%%%%%%%%%%%
%%%%%%%%%%%%%%%%%%%%%%%%%%%%%%%%%%%%%%%%%%%%%%%%%%%%%%%%%%%%%%%%%%%%%%%%%%%%%%%%
\appendix

\settowidth\MacroIndent{\rmfamily\scriptsize 000\ }

 \DocInput{childdoc.dtx}

\end{document}
%</driver>
% \fi
%
% %%%%%%%%%%%%%%%%%%%%%%%%%%%%%%%%%%%%%%%%%%%%%%%%%%%%%%%%%%%%%%%%%%%%%%%%%%%%%%
% %%%%%%%%%%%%%%%%%%%%%%%%%%%%%%%%%%%%%%%%%%%%%%%%%%%%%%%%%%%%%%%%%%%%%%%%%%%%%%
% \section{Sample}
%\iffalse
%<*samplemain>
%\fi
%
% The following presents a sample document
% with two chapters, two parts, a title page,
% a compile flag as well as three forwarding files to set the flag.
% It consists of eight |.tex| files:
% \begin{center}
% \begin{tabular}{ll}
% |cdocsamp.tex|&main file\\
% |cdocsch1.tex|&include file for chapter 1\\
% |cdocsch2.tex|&include file for chapter 2\\
% |cdocspt3.tex|&include file for part 3\\
% |cdocspt4.tex|&include file for part 4\\
% |cdocsdrf.tex|&forwarding file for main file in draft mode\\
% |cdocsfi1.tex|&forwarding file for final version of chapter 1\\
% |cdocsfi2.tex|&forwarding file for final version of chapter 2\\
% \end{tabular}
% \end{center}
% Each of the eight files can be compiled directly by the \LaTeX{} compiler.
%
% %%%%%%%%%%%%%%%%%%%%%%%%%%%%%%%%%%%%%%
% \paragraph{Main File.}
%
% The main file is called |cdocsamp.tex|.
%
% Load the \textsf{childdoc} definitions and
% declare the filename for the main document:
%    \begin{macrocode}
\input{childdoc.def}
\childdocmain{}
%    \end{macrocode}

% Optional override for |\version| flag:
%    \begin{macrocode}
%%\ifchilddoc\else\providecommand{\version}{draft}\fi
%    \end{macrocode}

% Define the default values for the |\version| flag
% (|final| for the main file and |draft| for childs):
%    \begin{macrocode}
\ifchilddoc
\providecommand{\version}{draft}
\else
\providecommand{\version}{final}
\fi
%    \end{macrocode}

% Load the standard document class:
%    \begin{macrocode}
\documentclass[12pt]{article}
%    \end{macrocode}

% Start the document body:
%    \begin{macrocode}
\begin{document}
%    \end{macrocode}

% Declare a title page.
% Print title, part of document being processed and version flag:
%    \begin{macrocode}
\addtocounter{page}{-1}
\begin{center}
{\LARGE\bfseries{}childdoc example\par}
\vspace{1cm}
\ifchilddoc
\ifchilddocmanual part\else chapter\fi:
`\childdocname' of `\childdocjob'\par
\else
main document: `\childdocjob'\par
\fi
version: \version\par
\end{center}
\newpage
%    \end{macrocode}

% Manually include selected file,
% otherwise process as usual:
%    \begin{macrocode}
\ifchilddocmanual
\section*{part `\childdocname'}
\input{\childdocname}
\else
%    \end{macrocode}

% Include the two chapters:
%    \begin{macrocode}
\include{cdocsch1}
\include{cdocsch2}
%    \end{macrocode}

% Include the two parts unless only chapters should be displayed:
%    \begin{macrocode}
\ifchilddoc\else
\section{part three}
\input{cdocspt3}
\section{part four}
\input{cdocspt4}
\fi
%    \end{macrocode}

% Process as usual until here:
%    \begin{macrocode}
\fi
%    \end{macrocode}

% End of document body:
%    \begin{macrocode}
\end{document}
%    \end{macrocode}
%\iffalse
%</samplemain>
%\fi
%
% %%%%%%%%%%%%%%%%%%%%%%%%%%%%%%%%%%%%%%
% \paragraph{Chapter Include Files.}
%
% The include files are called |cdocsch1.tex| and |cdocsch2.tex|.
%
%\iffalse
%<*samplechap1|samplechap2>
%\fi

% Optional override for |\version| flag:
%    \begin{macrocode}
%%\providecommand{\version}{final}
%    \end{macrocode}

% Include the main document:
%    \begin{macrocode}
\input{childdoc.def}
\childdocof{cdocsamp}
%    \end{macrocode}

%\iffalse
%</samplechap1|samplechap2>
%\fi
%
%\iffalse
%<*samplechap1>
%\fi
% Some text for chapter 1:
%    \begin{macrocode}
\section{one}
some text in chapter one
%    \end{macrocode}

%\iffalse
%</samplechap1>
%\fi
% Some text for chapter 2:
%\iffalse
%<*samplechap2>
%\fi
%    \begin{macrocode}
\section{two}
more text in chapter two
%    \end{macrocode}

%\iffalse
%</samplechap2>
%\fi
%
% %%%%%%%%%%%%%%%%%%%%%%%%%%%%%%%%%%%%%%
% \paragraph{Part Include Files.}
%
% The include files are called |cdocspt3.tex| and |cdocspt4.tex|.
%
%\iffalse
%<*samplepart3|samplepart4>
%\fi

% Optional override for |\version| flag:
%    \begin{macrocode}
%%\providecommand{\version}{final}
%    \end{macrocode}

% Include the main document:
%    \begin{macrocode}
\input{childdoc.def}
\childdocby{cdocsamp}
%    \end{macrocode}

%\iffalse
%</samplepart3|samplepart4>
%\fi
%
%\iffalse
%<*samplepart3>
%\fi
% Some text for part 3:
%    \begin{macrocode}
some text in part three
%    \end{macrocode}

%\iffalse
%</samplepart3>
%\fi
% Some text for part 4:
%\iffalse
%<*samplepart4>
%\fi
%    \begin{macrocode}
more text in part four
%    \end{macrocode}

%\iffalse
%</samplepart4>
%\fi
%
% %%%%%%%%%%%%%%%%%%%%%%%%%%%%%%%%%%%%%%
% \paragraph{Forwarding for a Complete Draft.}
%
% The following forwarding file |cdocsdrf.tex|
% compiles the main document in draft mode:
%\iffalse
%<*sampledraft>
%\fi
%    \begin{macrocode}
\def\version{draft}
\input{childdoc.def}
\childdocforward{cdocsamp}
%    \end{macrocode}

%\iffalse
%</sampledraft>
%\fi
%
% %%%%%%%%%%%%%%%%%%%%%%%%%%%%%%%%%%%%%%
% \paragraph{Forwarding for Final Version of the Chapters.}
%
% The following forwarding files |cdocsfn1.tex| and |cdocsfn2.tex|
% (with identical content)
% compile the final versions of the child documents
% |cdocsch1.tex| and |cdocsch2.tex|, respectively:
%\iffalse
%<*samplefinal>
%\fi
%    \begin{macrocode}
\def\version{final}
\input{childdoc.def}
\childdocforwardprefix[cdocsamp]{cdocsfn}{cdocsch}
%    \end{macrocode}

%\iffalse
%</samplefinal>
%\fi
%
% %%%%%%%%%%%%%%%%%%%%%%%%%%%%%%%%%%%%%%
% \paragraph{Command Line Processing.}
%
% The following three command lines generate the output files
% |cdocscld|, |cdocscl1| and |cdocscl2|
% which should be identical to
% |cdocsdrf|, |cdocsch1| and |cdocsfn2|, respectively:
% \begin{center}
% \begin{tabular}{l}
% |latex -jobname cdocscld \|\\
% |  "\def\version{draft}\input{childdoc.def}\childdocforward{cdocsamp}"|\\
% |latex -jobname cdocscl1 \|\\
% |  "\input{childdoc.def}\childdocforward[cdocsamp]{cdocsch1}"|\\
% |latex -jobname cdocscl2 \|\\
% |  "\def\version{final}\input{childdoc.def}\childdocforward{cdocsch2}"|
% \end{tabular}
% \end{center}
% Note that the trailing backslash on each first line
% merely continues the input to the second line
% (for convenient cut ant paste).
% Furthermore, the command |latex| can be replaced by any
% of its alternative versions such as |pdflatex|.
%
% %%%%%%%%%%%%%%%%%%%%%%%%%%%%%%%%%%%%%%%%%%%%%%%%%%%%%%%%%%%%%%%%%%%%%%%%%%%%%%
% %%%%%%%%%%%%%%%%%%%%%%%%%%%%%%%%%%%%%%%%%%%%%%%%%%%%%%%%%%%%%%%%%%%%%%%%%%%%%%
% \section{Implementation}
%\iffalse
%<*package>
%\fi
%
% This section describes the definitions file |childdoc.def|.

% The definitions cannot be loaded using |\usepackage| or |\RequirePackage|
% which has a mechanism to prevent loading a style file more than once.
% When loading the definitions by means of |\input|
% multiple instances have to be prevented manually:
%\iffalse
%This code needs to be before the `\ProvidesFile' directive
%which is defined at the beginning of this file.
%Therefore it is also placed there and commented out here.
%</package>
%<*discard>
%\fi
%    \begin{macrocode}
\ifdefined\childdocmain\endinput\fi
%    \end{macrocode}
%\iffalse
%</discard>
%<*package>
%\fi
%
% \macro{\ifchilddoc}
% \macro{\ifchilddocmanual}
% The conditional |\ifchilddoc| tells whether a
% child (true) or main (false) document is being compiled.
% The conditional |\ifchilddocmanual| tells whether
% the |\includeonly| mechanism is used (false) or
% the selection of child files must be performed manually (true).
% The definitions initialise to false:
%    \begin{macrocode}
\newif\ifchilddoc
\newif\ifchilddocmanual
%    \end{macrocode}

% \macro{\childdocname}
% \macro{\childdocjob}
% The macro |\childdocname| stores the name of the main document
% to be compiled. The macro |\childdocjob| stores the name of
% the document on which the \LaTeX{} compiler was originally invoked.
% The content of |\jobname| cannot be compared
% to filenames specified in the source due to different catcodes.
% The following code rescans |\jobname|, stores the result
% in |\childdocname| and saves a copy in |\childdocjob|:
%    \begin{macrocode}
\edef\childdocname{\scantokens\expandafter{\jobname\noexpand}}
\let\childdocjob\childdocname
%    \end{macrocode}

% \macro{\childdocdisable}
% The macro |\childdocdisable| prevents the main file
% from being processed more than once.
% At this stage, the main document command |\childdocmain|
% is assumed to be called once again where it should do nothing.
% Any subsequent call to it should prevent
% a secondary processing of the main document
% It overwrites the forwarding commands
% |\childdocof| and |\childdocforward|
% with empty macros to prevent further inclusions of the main document:
%    \begin{macrocode}
\newcommand{\childdocdisable}
{
  \renewcommand{\childdocmain}[1]{\renewcommand{\childdocmain}[1]{\endinput}}
  \renewcommand{\childdocof}[1]{}
  \renewcommand{\childdocby}[2][]{}
  \renewcommand{\childdocforward}[2][]{}
  \renewcommand{\childdocdisable}{}
}
%    \end{macrocode}

% \macro{\childdocmain}
% The macro |\childdocmain| is to be called at the top of the main file
% with nothing or the main filename (without extension) as argument.
% First, it breaks loops.
% If the argument is not empty and does not match |\childdocname|
% (which is set by the first inclusion of |childdoc.def|),
% |\ifchilddoc| is set to true, |\includeonly| is applied to the child file
% and |\jobname| is set to the main file
% (for proper handling of |.aux| files):
%    \begin{macrocode}
\newcommand{\childdocmain}[1]
{
  \childdocdisable\childdocmain{}
  \if?#1?\else
    \begingroup
      \def\childdoctmp{#1}
      \ifx\childdoctmp\childdocname
        \def\childdoctmp{}
      \else
        \def\childdoctmp
        {
          \childdoctrue
          \includeonly{\childdocname}
          \def\childdocjob{#1}
          \def\jobname{#1}
        }
      \fi
      \expandafter
    \endgroup
    \childdoctmp
  \fi
}
%    \end{macrocode}

% \macro{\childdocof}
% The command |\childdocof| redirects
% compilation to the main file |#1|.
%    \begin{macrocode}
\newcommand{\childdocof}[1]
{
  \childdocdisable
  \childdoctrue
  \includeonly{\childdocname}
  \def\jobname{#1}
  \def\childdocjob{#1}
  \input{#1}
}
%    \end{macrocode}

% \macro{\childdocby}
% The command |\childdocby| ....
%    \begin{macrocode}
\newcommand{\childdocby}[2][]
{
  \childdocdisable
  \childdoctrue
  \childdocmanualtrue
  \if?#1?\else
    \def\jobname{#2}
  \fi
  \def\childdocjob{#2}
  \input{#2}
  \endinput
}
%    \end{macrocode}

% \macro{\childdocforward}
% The command |\childdocforward| redirects
% compilation to the main file or
% (if the optional argument is given) a child file.
% Parameters are set as if the main file
% or a child file starting with |\childdocof| was compiled.
% Then compilation is handed over to the main file:
%    \begin{macrocode}
\newcommand{\childdocforward}[2][]
{
  \begingroup
    \if?#1?
      \def\childdoctmp
      {
        \def\childdocname{#2}
        \def\childdocjob{#2}
        \def\jobname{#2}
        \input{#2}
        \endinput
      }
    \else
      \def\childdoctmp
      {
        \childdocdisable
        \def\childdocname{#2}
        \childdoctrue
        \includeonly{#2}
        \def\childdocjob{#1}
        \def\jobname{#1}
        \input{#1}
        \endinput
      }
    \fi
    \expandafter
  \endgroup
  \childdoctmp
}
%    \end{macrocode}

% \macro{\childdocforwardprefix}
% The command |\childdocforwardprefix| redirects
% compilation to the main or a child file by means of a pattern.
% The prefix |#1| in the current filename is replaced by |#2|
% and the suffix of the current filename is kept
% (it is assumed that the filename does not contain the substring `|~~~|'
% which is used as a delimiter).
% Compilation is handed over to the new file by |\childdocforward|:
%    \begin{macrocode}
\newcommand{\childdocforwardprefix}[3][]
{
  \begingroup
    \def\childdocextract #2##1~~~{\def\childdoctmp{\childdocforward[#1]{#3##1}}}
    \expandafter\childdocextract\childdocname~~~
    \expandafter
  \endgroup
  \childdoctmp
}
%    \end{macrocode}

% \macro{\childdoc}
% The deprecated macro |\childdoc| is a legacy version of |\childdocmain|:
%    \begin{macrocode}
\newcommand{\childdoc}{\childdocmain}
%    \end{macrocode}

% \macro{\childdocredirect}
% The deprecated macro |\childdocredirect| is a legacy version
% of |\childdocforward| and |\childdocforwardprefix|:
%    \begin{macrocode}
\newcommand{\childdocredirect}[2][]
{
  \begingroup
    \if?#1?
      \def\childdoctmp{\childdocforward{#2}}
    \else
      \def\childdoctmp{\childdocforwardprefix{#1}{#2}}
    \fi
    \expandafter
  \endgroup
  \childdoctmp
}
%    \end{macrocode}

%\iffalse
%</package>
%\fi
%
\endinput

\childdocmain{}
%    \end{macrocode}

% Optional override for |\version| flag:
%    \begin{macrocode}
%%\ifchilddoc\else\providecommand{\version}{draft}\fi
%    \end{macrocode}

% Define the default values for the |\version| flag
% (|final| for the main file and |draft| for childs):
%    \begin{macrocode}
\ifchilddoc
\providecommand{\version}{draft}
\else
\providecommand{\version}{final}
\fi
%    \end{macrocode}

% Load the standard document class:
%    \begin{macrocode}
\documentclass[12pt]{article}
%    \end{macrocode}

% Start the document body:
%    \begin{macrocode}
\begin{document}
%    \end{macrocode}

% Declare a title page.
% Print title, part of document being processed and version flag:
%    \begin{macrocode}
\addtocounter{page}{-1}
\begin{center}
{\LARGE\bfseries{}childdoc example\par}
\vspace{1cm}
\ifchilddoc
\ifchilddocmanual part\else chapter\fi:
`\childdocname' of `\childdocjob'\par
\else
main document: `\childdocjob'\par
\fi
version: \version\par
\end{center}
\newpage
%    \end{macrocode}

% Manually include selected file,
% otherwise process as usual:
%    \begin{macrocode}
\ifchilddocmanual
\section*{part `\childdocname'}
\input{\childdocname}
\else
%    \end{macrocode}

% Include the two chapters:
%    \begin{macrocode}
\include{cdocsch1}
\include{cdocsch2}
%    \end{macrocode}

% Include the two parts unless only chapters should be displayed:
%    \begin{macrocode}
\ifchilddoc\else
\section{part three}
\input{cdocspt3}
\section{part four}
\input{cdocspt4}
\fi
%    \end{macrocode}

% Process as usual until here:
%    \begin{macrocode}
\fi
%    \end{macrocode}

% End of document body:
%    \begin{macrocode}
\end{document}
%    \end{macrocode}
%\iffalse
%</samplemain>
%\fi
%
% %%%%%%%%%%%%%%%%%%%%%%%%%%%%%%%%%%%%%%
% \paragraph{Chapter Include Files.}
%
% The include files are called |cdocsch1.tex| and |cdocsch2.tex|.
%
%\iffalse
%<*samplechap1|samplechap2>
%\fi

% Optional override for |\version| flag:
%    \begin{macrocode}
%%\providecommand{\version}{final}
%    \end{macrocode}

% Include the main document:
%    \begin{macrocode}
% \iffalse
%
% childdoc.dtx Copyright (C) 2017-2018 Niklas Beisert
%
% This work may be distributed and/or modified under the
% conditions of the LaTeX Project Public License, either version 1.3
% of this license or (at your option) any later version.
% The latest version of this license is in
%   http://www.latex-project.org/lppl.txt
% and version 1.3 or later is part of all distributions of LaTeX
% version 2005/12/01 or later.
%
% This work has the LPPL maintenance status `maintained'.
%
% The Current Maintainer of this work is Niklas Beisert.
%
% This work consists of the files childdoc.dtx and childdoc.ins
% and the derived files childdoc.def and cdocsamp.tex with
% cdocsch1.tex, cdocsch2.tex, cdocsdrf.tex, cdocsfn1.tex, cdocsfn2.tex.
%
%<package>\ifdefined\childdocmain\endinput\fi
%<package>\ProvidesFile{childdoc.def}[2018/12/30 v2.0 child document driver]
%<samplemain>\ProvidesFile{cdocsamp.tex}[2018/12/30 v2.0 sample for childdoc]
%<*driver>
%\ProvidesFile{childdoc.drv}[2018/12/30 v2.0 childdoc reference manual file]
\PassOptionsToClass{10pt,a4paper}{article}
\documentclass{ltxdoc}

\usepackage[margin=35mm]{geometry}
\usepackage{hyperref}
\usepackage{hyperxmp}
\usepackage[usenames]{color}

\hypersetup{colorlinks=true}
\hypersetup{pdfstartview=FitH}
\hypersetup{pdfpagemode=UseNone}
\hypersetup{pdfsource={}}
\hypersetup{pdflang={en-UK}}
\hypersetup{pdfcopyright={Copyright 2017-2018 Niklas Beisert.
  This work may be distributed and/or modified under the
  conditions of the LaTeX Project Public License, either version 1.3
  of this license or (at your option) any later version.}}
\hypersetup{pdflicenseurl={http://www.latex-project.org/lppl.txt}}
\hypersetup{pdfcontactaddress={ETH Zurich, ITP, HIT K,
  Wolfgang-Pauli-Strasse 27}}
\hypersetup{pdfcontactpostcode={8093}}
\hypersetup{pdfcontactcity={Zurich}}
\hypersetup{pdfcontactcountry={Switzerland}}
\hypersetup{pdfcontactemail={nbeisert@itp.phys.ethz.ch}}
\hypersetup{pdfcontacturl={http://people.phys.ethz.ch/\xmptilde nbeisert/}}

\newcommand{\secref}[1]{\hyperref[#1]{section \ref*{#1}}}

\parskip1ex
\parindent0pt
\let\olditemize\itemize
\def\itemize{\olditemize\parskip0pt}

\begin{document}

\title{The \textsf{childdoc} Package}
\hypersetup{pdftitle={The childdoc Package}}
\author{Niklas Beisert\\[2ex]
  Institut f\"ur Theoretische Physik\\
  Eidgen\"ossische Technische Hochschule Z\"urich\\
  Wolfgang-Pauli-Strasse 27, 8093 Z\"urich, Switzerland\\[1ex]
  \href{mailto:nbeisert@itp.phys.ethz.ch}
  {\texttt{nbeisert@itp.phys.ethz.ch}}}
\hypersetup{pdfauthor={Niklas Beisert}}
\hypersetup{pdfsubject={Manual for the LaTeX2e Package childdoc}}
\date{30 December 2018, \textsf{v2.0}}
\maketitle

\begin{abstract}\noindent
\textsf{childdoc} is a \LaTeXe{} package
that enables the direct compilation
of document sections included by |\include|
to individual files.
\end{abstract}

\begingroup
\parskip0ex
\tableofcontents
\endgroup

%%%%%%%%%%%%%%%%%%%%%%%%%%%%%%%%%%%%%%%%%%%%%%%%%%%%%%%%%%%%%%%%%%%%%%%%%%%%%%%%
%%%%%%%%%%%%%%%%%%%%%%%%%%%%%%%%%%%%%%%%%%%%%%%%%%%%%%%%%%%%%%%%%%%%%%%%%%%%%%%%
\section{Introduction}

\LaTeX{} provides a mechanism to structure a large document (such as a book)
into a main file and several child files (containing the chapters)
using the |\include| command.
This mechanism is beneficial for documents
which span hundreds of pages in order to
make the source file(s) more manageable.
Moreover, compilation can be restricted to
selected child files by means of the |\includeonly| command.
The latter feature can be used to reduce the compilation time while editing
(this was significantly more useful in the earlier days of \LaTeX{})
or to generate a smaller document which is easier to navigate.
Another application of |\includeonly| is to generate
documents consisting of selected parts of the complete document.

However, there are a few drawbacks of the plain |\include| mechanism:
\begin{itemize}
\item
The child files cannot be compiled on their own,
they can only be compiled via the main file.
A naive editing environment
(such as a text editor with an option
to have the current file processed by \LaTeX)
may require one to switch to the main file before compiling;
attempting to compile the child file produces errors.
\item
The main file must be modified (each time)
to adjust the |\includeonly| command
to the present needs. This easily leaves the main file in a messy state.
\item
The generated document will always carry the filename
of the main document. This is inconvenient if
several child files are to be compiled and
to be kept for distribution.
\end{itemize}

The present package provides a simple interface
to make child files individually compilable by \LaTeX{}.
Compiling a child file then has the same effect as compiling
the main file with an |\includeonly| command
to select the appropriate child.
Moreover the generated document will carry the name of the child
rather than the main file.
This resolves all three above issues.

This feature is meant to make the editing of books,
thesis documents and lecture notes somewhat more convenient.
However, the package can also be used efficiently for
composing a series of documents (such as exercise sheets)
which are typically distributed individually.
It then assists the author in generating the individual documents
(potentially in different versions)
as well as a document containing the collected series.
Another application is in developing style files
or other kinds of included material
where compilation of the style file could redirect
to a sample or test file.

%%%%%%%%%%%%%%%%%%%%%%%%%%%%%%%%%%%%%%%%%%%%%%%%%%%%%%%%%%%%%%%%%%%%%%%%%%%%%%%%
%%%%%%%%%%%%%%%%%%%%%%%%%%%%%%%%%%%%%%%%%%%%%%%%%%%%%%%%%%%%%%%%%%%%%%%%%%%%%%%%
\section{Usage}

First of all, the package \textsf{childdoc} is \emph{not} a standard
\LaTeXe{} |.sty| style file! Therefore it needs to be invoked in
a non-standard way.

%%%%%%%%%%%%%%%%%%%%%%%%%%%%%%%%%%%%%%%%%%%%%%%%%%%%%%%%%%%%%%%%%%%%%%%%%%%%%%%%
\subsection{Included Files}
\label{sec:include}

%%%%%%%%%%%%%%%%%%%%%%%%%%%%%%%%%%%%%%%%
\DescribeMacro{\childdocmain}
To use the package, add the commands
\begin{center}
\begin{tabular}{l}
|\input{childdoc.def}|\\
|\childdocmain{}|\\
\end{tabular}
\end{center}
at the very top of the main \LaTeX{} file,
in particular \emph{before} the |\documentclass| statement!
The argument of |\childdocmain| should be left empty
(but it must be present).

%%%%%%%%%%%%%%%%%%%%%%%%%%%%%%%%%%%%%%%%
\DescribeMacro{\childdocof}
Furthermore, add the commands
\begin{center}
\begin{tabular}{l}
|\input{childdoc.def}|\\
|\childdocof{|\textit{main}|}|\\
\end{tabular}
\end{center}
at the top of every child file \textit{child}
which is included by |\include{|\textit{child}|}|
from within the main file
(or at least for those files to be compiled individually).
The argument \textit{main} must be the filename of the main file.

There are a couple of
considerations in setting up the main and child documents:

%%%%%%%%%%%%%%%%%%%%%%%%%%%%%%%%%%%%%%%%
\paragraph{Restrictions.}

Please note the following restrictions:
\begin{itemize}
\item
|\childdocmain| must be called with one argument \textit{main}
to ensure compatibility with earlier version of the package.
It must either be empty (|\childdocmain{}|)
or precisely match the filename of the main file in which it is specified.
See \secref{sec:detection} for further information.
\item
The filename \textit{main} must be specified without the |.tex| extension.
\item
The filename \textit{main} is case sensitive
(even in case-insensitive file systems)
due to internal string comparison.
\item
The argument \textit{main} should be fully expanded, it cannot be a macro.
\item
Subdirectories and special characters should be avoided in filenames.
\item
The command |\childdocmain{|\textit{main}|}| must be followed by a whitespace.
It should not be followed immediately by another command
or by a comment mark `|%|'.
This is because the \TeX{} parser reads the token immediately following
the argument of |\childdocmain| and puts it
at the beginning of every child section;
however, a white\-space is ignored.
\end{itemize}

%%%%%%%%%%%%%%%%%%%%%%%%%%%%%%%%%%%%%%%%
\paragraph{Content of Main File.}

It is advisable to place all content in the child files included by |\include|.
Any output contained in the main file will appear in all child documents
unless suppressed manually;
it cannot be suppressed automatically by the |\includeonly| directive
and thus should normally be avoided.
A method to include some content in the main file
by means of conditional processing is described in \secref{sec:conditional}.

%%%%%%%%%%%%%%%%%%%%%%%%%%%%%%%%%%%%%%%%
\paragraph{Page Numbering.}

When only a part of the document is compiled,
the appropriate numbering of pages
(as well as other status parameters)
is determined from the |.aux| files.
The latter contain information from previous passes.
However this information needs to propagate through
all intermediate child documents.
Therefore the page numbering in child documents may well
be inconsistent until the complete document is compiled at least once.

A useful (if unconventional) way to always ensure a consistent
page numbering is to restart the numbering in each child document
and denote the pages by `\textit{child}|.|\textit{page}'
where \textit{child} represents the chapter/section number of the child file.
This can be achieved by the command
|\numberwithin{page}{|\textit{child}|}|
of the \textsf{amsmath} package
where \textit{child} can be |chapter| or |section|
depending on the chosen structuring.
Alternatively, one can modify the macro |\thepage| appropriately
and reset the counter |page| at the start of each child file.

%%%%%%%%%%%%%%%%%%%%%%%%%%%%%%%%%%%%%%%%%%%%%%%%%%%%%%%%%%%%%%%%%%%%%%%%%%%%%%%%
\subsection{Conditional Processing}
\label{sec:conditional}

The package provides a mechanism to compile different versions
of a document. To customise the versions further some conditional processing
can come in handy to distinguish which version is being compiled.
The package provides two macros to describe the compilation context:

%%%%%%%%%%%%%%%%%%%%%%%%%%%%%%%%%%%%%%%%
\DescribeMacro{\ifchilddoc}
The conditional |\ifchilddoc| distinguishes between the compilation of
child documents and the main document:
%
\begin{center}
|\ifchilddoc |\textit{child-code}| |[|\||else |\textit{main-code}]| \||fi|
\end{center}

%%%%%%%%%%%%%%%%%%%%%%%%%%%%%%%%%%%%%%%%
\DescribeMacro{\childdocname}
\DescribeMacro{\childdocjob}
The macro |\childdocname| contains the filename (without extension)
of the main or child file being processed.
Note that |\childdocjob| will always contain the name of the main file.

%%%%%%%%%%%%%%%%%%%%%%%%%%%%%%%%%%%%%%%%
\paragraph{Title Page.}

Conditional processing can be used to include a title or banner page
in the main document when proper precautions are taken.
Importantly, the code in the main file should ensure that the page counter
(as well as other status parameters which are stored in the |.aux| files)
takes the same value after the conditional processing.
Otherwise the page numbers may take divergent values
depending on which part is compiled.

For example, a title page could be declared by:
%
\begin{center}
\begin{tabular}{l}
|\ifchilddoc\||else|\\
|\addtocounter{page}{-1}|\\
\textit{code for title page}\\
|\newpage|\\
|\||fi|
\end{tabular}
\end{center}
%
A banner page for the child documents can be generated by:
%
\begin{center}
\begin{tabular}{l}
|\ifchilddoc|\\
|\addtocounter{page}{-1}|\\
\textit{code for banner page}\\
|\newpage|\\
|\||fi|
\end{tabular}
\end{center}
%
Here one could write a message such as:
\begin{center}
|This is the part \childdocname{} of \childdocjob{}.|
\end{center}

%%%%%%%%%%%%%%%%%%%%%%%%%%%%%%%%%%%%%%%%%%%%%%%%%%%%%%%%%%%%%%%%%%%%%%%%%%%%%%%%
\subsection{Flags}
\label{sec:flags}

The package makes it easy to generate different versions
of the main or child documents.
To this end compilation flags can be defined
and assigned different default values.
They will be particularly useful in conjunction
with the forwarding mechanism described in \secref{sec:forward}.

For example, it may be useful to have a flag |\version|
which can be set to |draft| or |final|.
The document source will contain some conditional code
depending on the value of |\version|.
Suppose further, the flag should default to |final| for the main file
and to |draft| for child files
which is a natural assignment for editing the document.
This is achieved by placing the following code
in the preamble of the main document
(below the |\childdocmain| directive):
%
\begin{center}
\begin{tabular}{l}
|\ifchilddoc|\\
|\providecommand{\version}{draft}|\\
|\||else|\\
|\providecommand{\version}{final}|\\
|\||fi|
\end{tabular}
\end{center}
%
The definition by |\providecommand| makes sure
that previous definitions are not overwritten.
Further statements |\providecommand{\version}{...}|
can thus be added before the above code to override it.

For the main file, one might add a line
(between |\childdocmain| and the above block)
%
\begin{center}
|%\ifchilddoc\||else\providecommand{\version}{draft}\||fi|
\end{center}
%
which can be uncommented to produce a draft version.
Likewise one can add a line to the very top of a child file
(above the |\childdocof{|\textit{main}|}| directive)
%
\begin{center}
|%\providecommand{\version}{final}|
\end{center}
%
which can be uncommented to produce the final version of this child document.

%%%%%%%%%%%%%%%%%%%%%%%%%%%%%%%%%%%%%%%%%%%%%%%%%%%%%%%%%%%%%%%%%%%%%%%%%%%%%%%%
\subsection{Forwarding}
\label{sec:forward}

Different versions of the main or child documents
using compilation flags as described in \secref{sec:flags}
can be (permanently) stored in different files
for convenient compilation, viewing and distribution.
To this end, the package defines a command
to pass on compilation to a different file:

%%%%%%%%%%%%%%%%%%%%%%%%%%%%%%%%%%%%%%%%
\DescribeMacro{\childdocforward}
The command |\childdocforward| redirects processing to
another source file:
%
\begin{center}
\begin{tabular}{l}
|\input{childdoc.def}|\\
|\childdocforward[|\textit{main}|]{|\textit{dest}|}|\\
\end{tabular}
\end{center}
%
The argument \textit{dest} is the destination file
(without extension).
It should be the main file or one of the child files.
Note that further \textsf{childdoc} directives
such as |\childdocof| and |\childdocforward|
in the indicated file will be processed in this form.
The optional argument \textit{main}
passes on directly to the main file \textit{main}
while pretending to compile the child \textit{dest}.
This form behaves as if \textit{dest}
issues |\childdocof{|\textit{main}|}| right away,
and no further \textsf{childdoc} directives will be processed.

%%%%%%%%%%%%%%%%%%%%%%%%%%%%%%%%%%%%%%%%
\DescribeMacro{\...prefix}
In the alternative form |\childdocforwardprefix|,
%
\begin{center}
\begin{tabular}{l}
|\input{childdoc.def}|\\
|\childdocforwardprefix[|\textit{main}|]{|\textit{prefix}|}{|\textit{dest}|}|
\end{tabular}
\end{center}
%
the destination file is determined by a pattern
depending on the current file:
To make this work, the current file must be called
`{\textit{prefix}\hspace{0.2em}\textit{suffix}}'
with \textit{prefix} matching precisely the argument.
Processing is then passed on to the file
`{\textit{dest}\hspace{0.2em}\textit{suffix}}'.
Surely, the same effect is achieved by
directly specifying the
argument `{\textit{dest}\hspace{0.2em}\textit{suffix}}'
in the first form.
However, that requires to set up a different file
for each child. With the alternative form of the command
all these files can have exactly the same content
which simplifies setting them up and maintaining them.

For example, the following file |draft.tex|
with a compilation flag |\version| as described in \secref{sec:flags}
compiles the main document as a draft:
%
\begin{center}
\begin{tabular}{l}
|\def\version{draft}|\\
|\input{childdoc.def}|\\
|\childdocforward{|\textit{main}|}|
\end{tabular}
\end{center}
%
Likewise, the following files |final|\textit{nn}|.tex|
compile the final version of the child document
|child|\textit{nn}|.tex|:
%
\begin{center}
\begin{tabular}{l}
|\def\version{final}|\\
|\input{childdoc.def}|\\
|\childdocforwardprefix{final}{child}|
\end{tabular}
\end{center}
%

Note that when several versions of a main file and/or of each child file
are to be generated, it may be convenient to set up a |Makefile| or
shell script to automatise the process.

%%%%%%%%%%%%%%%%%%%%%%%%%%%%%%%%%%%%%%%%%%%%%%%%%%%%%%%%%%%%%%%%%%%%%%%%%%%%%%%%
\subsection{Command Line Processing}
\label{sec:commandline}

The effect of redirection files can also be achieved by invoking
the \LaTeX{} compiler with a more elaborate command line.
Most conveniently this should be done as part
of a shell script or a |Makefile|.

When using \textsf{childdoc} in the main file, the following
command lines effectively perform a redirection
(note that depending on the shell being used,
backslashes may have to be doubled: `|\|' $\to$ `|\\|'):
%
\begin{center}
|... -jobname "|\textit{target}|" |\\|"|[\textit{flags}]%
|\input{childdoc.def}\childdocforward[|\textit{main}|]{|\textit{dest}|}"|
\end{center}
%
Here \textit{target} is the name of the output file,
\textit{main} is the name of the main file
and \textit{dest} is the name of the main or child file to be processed
(all filenames without extensions).
The optional argument \textit{main} can be omitted
if \textit{main} matches \textit{dest}.
Optionally, compilation \textit{flags} can be defined via |\def| commands.
This command line makes the \TeX{} engine believe
it is compiling the file \textit{target}
whose content is specified as the latter parameter.
The provided code then forwards the processing to
\textit{main} or \textit{dest} as described in \secref{sec:forward}.

%%%%%%%%%%%%%%%%%%%%%%%%%%%%%%%%%%%%%%%%%%%%%%%%%%%%%%%%%%%%%%%%%%%%%%%%%%%%%%%%
\subsection{Include by Input}
\label{sec:input}

Including child documents by |\include| has some restrictions by design.
Most notably, the content of a child document always occupies
its own set of pages; pages cannot be shared between child documents.
Usually, this behaviour makes perfect sense
because each child document contain an essential part of the document.
However, in some situations it may be desirable to compose
a document from a collection of parts
without having mandatory page breaks between then.
For this case, the package
provides a mechanism to include parts
by |\input| which can also be processed individually.
However, by construction this mechanism
requires manual handling of the content to be output.

%%%%%%%%%%%%%%%%%%%%%%%%%%%%%%%%%%%%%%%%
\DescribeMacro{\ifchilddocmanual}
The main file should be prepared as usual, see \secref{sec:include}.
However, the document body must make a distinction
between processing of an individual part and of the main document, e.g.:
%
\begin{center}
\begin{tabular}{l}
|\ifchilddocmanual|\\
|\input{\childdocname}|\\
|\||else|\\
\textit{document body with }|\input{|\textit{part}|}|\\
|\||fi|
\end{tabular}
\end{center}
%
The conditional |\ifchilddocmanual| is true whenever
a part to be included by |\input| is being compiled,
and the name of the part is stored in |\childdocname|.

%%%%%%%%%%%%%%%%%%%%%%%%%%%%%%%%%%%%%%%%
\DescribeMacro{\childdocby}
Each part to be included by |\input| should start with:
%
\begin{center}
\begin{tabular}{l}
|\input{childdoc.def}|\\
|\childdocby{|\textit{main}|}|\\
\end{tabular}
\end{center}
%
The directive |\childdocby| is similar to |\childdocof|
described in \secref{sec:include},
but the subsequent selection of content must be done manually.
To that end, both |\ifchilddoc| and |\ifchilddocmanual|
will be true upon processing of a part,
and the name of the part is stored in |\childdocname|.
Note that |\jobname| will be set to the filename of the current part
so that each part receives an individual |.aux| file
that does not interfere with the |.aux| file(s) of the main document.
This behaviour can be altered by the alternative form
|\childdocby[*]{|\textit{main}|}| (with a non-empty optional argument)
which uses the |.aux| file of the main document
by setting |\jobname| to \textit{main}.

%%%%%%%%%%%%%%%%%%%%%%%%%%%%%%%%%%%%%%%%%%%%%%%%%%%%%%%%%%%%%%%%%%%%%%%%%%%%%%%%
\subsection{Driver Development}
\label{sec:driver}

The \textsf{childdoc} mechanism can also be use for the development
of definition files such as \LaTeX{} styles or classes.
This case differs from the above setup with multiple parts
included by |\include| in that no |\includeonly| should be invoked.
This can be achieved by starting the include file
(before |\ProvidesPackage|) with:
%
\begin{center}
\begin{tabular}{l}
|\input{childdoc.def}|\\
|\childdocforward{|\textit{main}|}|\\
\end{tabular}
\end{center}
%
or alternatively with:
%
\begin{center}
\begin{tabular}{l}
|\input{childdoc.def}|\\
|\childdocby{|\textit{main}|}|\\
\end{tabular}
\end{center}
%
Both forms have slightly different effects as described above.
The main file is prepared as usual, see \secref{sec:include}.

%%%%%%%%%%%%%%%%%%%%%%%%%%%%%%%%%%%%%%%%%%%%%%%%%%%%%%%%%%%%%%%%%%%%%%%%%%%%%%%%
\subsection{Legacy Detection}
\label{sec:detection}

The directive |\childdocmain| in the main file can detect
whether the complete document or merely a child is to be compiled
even without using the directive |\childdocof|.
This method is deprecated because it is less robust
and there is no compelling reason to use it;
it is merely provided for backward compatibility
and it may be removed in future versions.

If the detection mechanism is to be used,
it is mandatory to correctly specify
the filename of the main file as the argument of |\childdocmain|:
%
\begin{center}
\begin{tabular}{l}
|\input{childdoc.def}|\\
|\childdocmain{|\textit{main}|}|\\
\end{tabular}
\end{center}
%
If |\jobname| does not match the argument \textit{main} of |\childdocmain|,
it is assumed that |\jobname| points to the child file to be compiled.
When using |\childdocmain| with the main file specified as argument,
it suffices to start a child file
with just |\input{|\textit{main}|}|
without loading of the package and using |\childdocof|.
If instead all processing is done
with the appropriate \textsf{childdoc} directives,
the argument of \textit{main} of |\childdocmain| can be empty.

An alternative version of the command line processing described
in \secref{sec:commandline} using the detection mechanism reads:
%
\begin{center}
|... -jobname "|\textit{target}|" "|[\textit{flags}]%
[|\def\jobname{|\textit{dest}|}|]|\input{|\textit{main}|}"|
\end{center}

%%%%%%%%%%%%%%%%%%%%%%%%%%%%%%%%%%%%%%%%%%%%%%%%%%%%%%%%%%%%%%%%%%%%%%%%%%%%%%%%
\subsection{Manual Code}
\label{sec:manual}

In case one cannot be certain whether the definitions file |childdoc.def|
is installed on the target \TeX{} distribution
and one prefers not to ship it,
it is conceivable to paste a few relevant commands into the sources.

To that end, drop all statements |\input{childdoc.def}|
and perform the replacements as outlined below.
Instead of |\childdocmain{|\textit{main}|}| add the following code
to the top of the main file:
%
\begin{center}
\begin{tabular}{l}
|\||ifdefined\childdocname\endinput\||fi\newif\ifchilddoc|\\
|\edef\childdocname{\scantokens\expandafter{\jobname\noexpand}}|\\
|\def\childdocmain{|\textit{main}|}\||ifx\childdocmain\childdocname\||else|\\
|\childdoctrue\includeonly{\childdocname}\let\jobname\childdocmain\||fi|\\
\end{tabular}
\end{center}
%
Instead of |\childdocof{|\textit{main}|}| just include the main file
at the top of each child file:
%
\begin{center}
|\input{|\textit{main}|}|
\end{center}
%
A simple redirection |\childdocforward{|\textit{dest}|}| is achieved by:
%
\begin{center}
|\def\jobname{|\textit{dest}|}\input{\jobname}|
\end{center}
%
The redirection with prefix
|\childdocforwardprefix[|\textit{prefix}|]{|\textit{dest}|}|
is accomplished by:
%
\begin{center}
\begin{tabular}{l}
|{\edef\jobname{\scantokens\expandafter{\jobname\noexpand}}|\\
|\def\redirectjob |\textit{prefix}|#1~~~{\gdef\jobname{|\textit{dest}|#1}}|\\
|\expandafter\redirectjob\jobname~~~}\input{\jobname}|
\end{tabular}
\end{center}

In an alternative approach,
child documents can be compiled by a specific command line
without additional code or specific definitions:
%
\begin{center}
|... -jobname "|\textit{target}|" "|[\textit{flags}]%
|\includeonly{|\textit{dest}|}\input{|\textit{main}|}"|
\end{center}
%

%%%%%%%%%%%%%%%%%%%%%%%%%%%%%%%%%%%%%%%%%%%%%%%%%%%%%%%%%%%%%%%%%%%%%%%%%%%%%%%%
%%%%%%%%%%%%%%%%%%%%%%%%%%%%%%%%%%%%%%%%%%%%%%%%%%%%%%%%%%%%%%%%%%%%%%%%%%%%%%%%
\section{Information}

%%%%%%%%%%%%%%%%%%%%%%%%%%%%%%%%%%%%%%%%%%%%%%%%%%%%%%%%%%%%%%%%%%%%%%%%%%%%%%%%
\subsection{Copyright}

Copyright \copyright{} 2017--2018 Niklas Beisert

This work may be distributed and/or modified under the
conditions of the \LaTeX{} Project Public License, either version 1.3
of this license or (at your option) any later version.
The latest version of this license is in
  \url{http://www.latex-project.org/lppl.txt}
and version 1.3 or later is part of all distributions of \LaTeX{}
version 2005/12/01 or later.

This work has the LPPL maintenance status `maintained'.

The Current Maintainer of this work is Niklas Beisert.

This work consists of the files |README.txt|, |childdoc.ins| and |childdoc.dtx|
as well as the derived files |childdoc.def|, |cdocsamp.tex|
with |cdocsch1.tex|, |cdocsch2.tex|, |cdocspt3.tex|, |cdocspt4.tex|,
|cdocsdrf.tex|, |cdocsfn1.tex|, |cdocsfn2.tex|
as well as |childdoc.pdf|.

%%%%%%%%%%%%%%%%%%%%%%%%%%%%%%%%%%%%%%%%%%%%%%%%%%%%%%%%%%%%%%%%%%%%%%%%%%%%%%%%
\subsection{Files and Installation}

The package consists of the files:
%
\begin{center}
\begin{tabular}{ll}
    |README.txt|   & readme file \\
    |childdoc.ins| & installation file \\
    |childdoc.dtx| & source file \\
    |childdoc.def| & definition file \\
    |cdocsamp.tex| & sample main file \\
    |cdocsch1.tex| & sample include file \\
    |cdocsch2.tex| & sample include file \\
    |cdocspt3.tex| & sample part file \\
    |cdocspt4.tex| & sample part file \\
    |cdocsdrf.tex| & sample redirection file \\
    |cdocsfn1.tex| & sample redirection file \\
    |cdocsfn2.tex| & sample redirection file \\
    |childdoc.pdf| & manual
\end{tabular}
\end{center}
%
The distribution consists of the files
|README.txt|, |childdoc.ins| and |childdoc.dtx|.
%
\begin{itemize}
\item
Run (pdf)\LaTeX{} on |childdoc.dtx|
to compile the manual |childdoc.pdf| (this file).
\item
Run \LaTeX{} on |childdoc.ins| to create the definitions file |childdoc.def|
and the sample |cdocsamp.tex| with include files
|cdocsch1.tex|, |cdocsch2.tex|, |cdocspt3.tex|, |cdocspt4.tex|,
|cdocsdrf.tex|, |cdocsfn1.tex|, |cdocsfn2.tex|.
Then copy the file |childdoc.def| to an appropriate directory of your \LaTeX{}
distribution, e.g.\ \textit{texmf-root}|/tex/latex/childdoc|.
\end{itemize}

%%%%%%%%%%%%%%%%%%%%%%%%%%%%%%%%%%%%%%%%%%%%%%%%%%%%%%%%%%%%%%%%%%%%%%%%%%%%%%%%
\subsection{Related CTAN Packages}

There are several other packages which offer a similar functionality:
%
\begin{itemize}
\item
The packages
\href{http://ctan.org/pkg/docmute}{\textsf{docmute}},
\href{http://ctan.org/pkg/includex}{\textsf{includex}} and
\href{http://ctan.org/pkg/standalone}{\textsf{standalone}}
provide commands to include only the document body of
a child file thus allowing both files to be compiled individually.
\item
The packages \href{http://ctan.org/pkg/subdocs}{\textsf{subdocs}}
and \href{http://ctan.org/pkg/subfiles}{\textsf{subfiles}}
provide structures in which the main and child documents can be
encapsulated and allowing them to be compiled individually.
The inclusion mechanism is different from the conventional |\include|.
\item
The package \href{http://ctan.org/pkg/combine}{\textsf{combine}}
is an elaborate solution to combine several documents into one.
\end{itemize}
%
See also the CTAN topic \href{http://ctan.org/topic/subdocs}{\textsf{subdocs}}
for further related packages.
The present package differs from the above solutions in that
a document structure constructed with the conventional |\include| mechanism
just needs two extra commands at the top of every file
such that all constituent files can be compiled individually.

%%%%%%%%%%%%%%%%%%%%%%%%%%%%%%%%%%%%%%%%%%%%%%%%%%%%%%%%%%%%%%%%%%%%%%%%%%%%%%%%
%\subsection{Feature Suggestions}
%
%The following is a list of features which may be useful for future
%versions of this package:
%%
%\begin{itemize}
%\item
%\ldots
%\end{itemize}

%%%%%%%%%%%%%%%%%%%%%%%%%%%%%%%%%%%%%%%%%%%%%%%%%%%%%%%%%%%%%%%%%%%%%%%%%%%%%%%%
\subsection{Revision History}

%%%%%%%%%%%%%%%%%%%%%%%%%%%%%%%%%%%%%%%%
\paragraph{v2.0:} 2018/12/30

\begin{itemize}
\item
immediate forward processing
\item
added |\childdocby| mechanism
\item
manual restructured
\end{itemize}

%%%%%%%%%%%%%%%%%%%%%%%%%%%%%%%%%%%%%%%%
\paragraph{v1.6:} 2018/01/17

\begin{itemize}
\item
application for development of include files
\item
corrections to manual
\end{itemize}

%%%%%%%%%%%%%%%%%%%%%%%%%%%%%%%%%%%%%%%%
\paragraph{v1.5:} 2017/05/21

\begin{itemize}
\item
more complete structuring introduced
\item
|\childdocof| introduced
\item
|\childdoc| renamed to |\childdocmain|
\item
|\childredirect| renamed to |\childdocforward| and |\childdocforwardprefix|
and functionality expanded
\end{itemize}

%%%%%%%%%%%%%%%%%%%%%%%%%%%%%%%%%%%%%%%%
\paragraph{v1.0:} 2017/04/27

\begin{itemize}
\item
manual and install package
\item
first version published on CTAN
\end{itemize}

%%%%%%%%%%%%%%%%%%%%%%%%%%%%%%%%%%%%%%%%
\paragraph{v0.6:} 2017/04/26

\begin{itemize}
\item
redirection mechanism added
\end{itemize}

%%%%%%%%%%%%%%%%%%%%%%%%%%%%%%%%%%%%%%%%
\paragraph{v0.5:} 2017/04/26

\begin{itemize}
\item
functionality in definition file
\end{itemize}


%%%%%%%%%%%%%%%%%%%%%%%%%%%%%%%%%%%%%%%%%%%%%%%%%%%%%%%%%%%%%%%%%%%%%%%%%%%%%%%%
%%%%%%%%%%%%%%%%%%%%%%%%%%%%%%%%%%%%%%%%%%%%%%%%%%%%%%%%%%%%%%%%%%%%%%%%%%%%%%%%
%%%%%%%%%%%%%%%%%%%%%%%%%%%%%%%%%%%%%%%%%%%%%%%%%%%%%%%%%%%%%%%%%%%%%%%%%%%%%%%%
\appendix

\settowidth\MacroIndent{\rmfamily\scriptsize 000\ }

 \DocInput{childdoc.dtx}

\end{document}
%</driver>
% \fi
%
% %%%%%%%%%%%%%%%%%%%%%%%%%%%%%%%%%%%%%%%%%%%%%%%%%%%%%%%%%%%%%%%%%%%%%%%%%%%%%%
% %%%%%%%%%%%%%%%%%%%%%%%%%%%%%%%%%%%%%%%%%%%%%%%%%%%%%%%%%%%%%%%%%%%%%%%%%%%%%%
% \section{Sample}
%\iffalse
%<*samplemain>
%\fi
%
% The following presents a sample document
% with two chapters, two parts, a title page,
% a compile flag as well as three forwarding files to set the flag.
% It consists of eight |.tex| files:
% \begin{center}
% \begin{tabular}{ll}
% |cdocsamp.tex|&main file\\
% |cdocsch1.tex|&include file for chapter 1\\
% |cdocsch2.tex|&include file for chapter 2\\
% |cdocspt3.tex|&include file for part 3\\
% |cdocspt4.tex|&include file for part 4\\
% |cdocsdrf.tex|&forwarding file for main file in draft mode\\
% |cdocsfi1.tex|&forwarding file for final version of chapter 1\\
% |cdocsfi2.tex|&forwarding file for final version of chapter 2\\
% \end{tabular}
% \end{center}
% Each of the eight files can be compiled directly by the \LaTeX{} compiler.
%
% %%%%%%%%%%%%%%%%%%%%%%%%%%%%%%%%%%%%%%
% \paragraph{Main File.}
%
% The main file is called |cdocsamp.tex|.
%
% Load the \textsf{childdoc} definitions and
% declare the filename for the main document:
%    \begin{macrocode}
\input{childdoc.def}
\childdocmain{}
%    \end{macrocode}

% Optional override for |\version| flag:
%    \begin{macrocode}
%%\ifchilddoc\else\providecommand{\version}{draft}\fi
%    \end{macrocode}

% Define the default values for the |\version| flag
% (|final| for the main file and |draft| for childs):
%    \begin{macrocode}
\ifchilddoc
\providecommand{\version}{draft}
\else
\providecommand{\version}{final}
\fi
%    \end{macrocode}

% Load the standard document class:
%    \begin{macrocode}
\documentclass[12pt]{article}
%    \end{macrocode}

% Start the document body:
%    \begin{macrocode}
\begin{document}
%    \end{macrocode}

% Declare a title page.
% Print title, part of document being processed and version flag:
%    \begin{macrocode}
\addtocounter{page}{-1}
\begin{center}
{\LARGE\bfseries{}childdoc example\par}
\vspace{1cm}
\ifchilddoc
\ifchilddocmanual part\else chapter\fi:
`\childdocname' of `\childdocjob'\par
\else
main document: `\childdocjob'\par
\fi
version: \version\par
\end{center}
\newpage
%    \end{macrocode}

% Manually include selected file,
% otherwise process as usual:
%    \begin{macrocode}
\ifchilddocmanual
\section*{part `\childdocname'}
\input{\childdocname}
\else
%    \end{macrocode}

% Include the two chapters:
%    \begin{macrocode}
\include{cdocsch1}
\include{cdocsch2}
%    \end{macrocode}

% Include the two parts unless only chapters should be displayed:
%    \begin{macrocode}
\ifchilddoc\else
\section{part three}
\input{cdocspt3}
\section{part four}
\input{cdocspt4}
\fi
%    \end{macrocode}

% Process as usual until here:
%    \begin{macrocode}
\fi
%    \end{macrocode}

% End of document body:
%    \begin{macrocode}
\end{document}
%    \end{macrocode}
%\iffalse
%</samplemain>
%\fi
%
% %%%%%%%%%%%%%%%%%%%%%%%%%%%%%%%%%%%%%%
% \paragraph{Chapter Include Files.}
%
% The include files are called |cdocsch1.tex| and |cdocsch2.tex|.
%
%\iffalse
%<*samplechap1|samplechap2>
%\fi

% Optional override for |\version| flag:
%    \begin{macrocode}
%%\providecommand{\version}{final}
%    \end{macrocode}

% Include the main document:
%    \begin{macrocode}
\input{childdoc.def}
\childdocof{cdocsamp}
%    \end{macrocode}

%\iffalse
%</samplechap1|samplechap2>
%\fi
%
%\iffalse
%<*samplechap1>
%\fi
% Some text for chapter 1:
%    \begin{macrocode}
\section{one}
some text in chapter one
%    \end{macrocode}

%\iffalse
%</samplechap1>
%\fi
% Some text for chapter 2:
%\iffalse
%<*samplechap2>
%\fi
%    \begin{macrocode}
\section{two}
more text in chapter two
%    \end{macrocode}

%\iffalse
%</samplechap2>
%\fi
%
% %%%%%%%%%%%%%%%%%%%%%%%%%%%%%%%%%%%%%%
% \paragraph{Part Include Files.}
%
% The include files are called |cdocspt3.tex| and |cdocspt4.tex|.
%
%\iffalse
%<*samplepart3|samplepart4>
%\fi

% Optional override for |\version| flag:
%    \begin{macrocode}
%%\providecommand{\version}{final}
%    \end{macrocode}

% Include the main document:
%    \begin{macrocode}
\input{childdoc.def}
\childdocby{cdocsamp}
%    \end{macrocode}

%\iffalse
%</samplepart3|samplepart4>
%\fi
%
%\iffalse
%<*samplepart3>
%\fi
% Some text for part 3:
%    \begin{macrocode}
some text in part three
%    \end{macrocode}

%\iffalse
%</samplepart3>
%\fi
% Some text for part 4:
%\iffalse
%<*samplepart4>
%\fi
%    \begin{macrocode}
more text in part four
%    \end{macrocode}

%\iffalse
%</samplepart4>
%\fi
%
% %%%%%%%%%%%%%%%%%%%%%%%%%%%%%%%%%%%%%%
% \paragraph{Forwarding for a Complete Draft.}
%
% The following forwarding file |cdocsdrf.tex|
% compiles the main document in draft mode:
%\iffalse
%<*sampledraft>
%\fi
%    \begin{macrocode}
\def\version{draft}
\input{childdoc.def}
\childdocforward{cdocsamp}
%    \end{macrocode}

%\iffalse
%</sampledraft>
%\fi
%
% %%%%%%%%%%%%%%%%%%%%%%%%%%%%%%%%%%%%%%
% \paragraph{Forwarding for Final Version of the Chapters.}
%
% The following forwarding files |cdocsfn1.tex| and |cdocsfn2.tex|
% (with identical content)
% compile the final versions of the child documents
% |cdocsch1.tex| and |cdocsch2.tex|, respectively:
%\iffalse
%<*samplefinal>
%\fi
%    \begin{macrocode}
\def\version{final}
\input{childdoc.def}
\childdocforwardprefix[cdocsamp]{cdocsfn}{cdocsch}
%    \end{macrocode}

%\iffalse
%</samplefinal>
%\fi
%
% %%%%%%%%%%%%%%%%%%%%%%%%%%%%%%%%%%%%%%
% \paragraph{Command Line Processing.}
%
% The following three command lines generate the output files
% |cdocscld|, |cdocscl1| and |cdocscl2|
% which should be identical to
% |cdocsdrf|, |cdocsch1| and |cdocsfn2|, respectively:
% \begin{center}
% \begin{tabular}{l}
% |latex -jobname cdocscld \|\\
% |  "\def\version{draft}\input{childdoc.def}\childdocforward{cdocsamp}"|\\
% |latex -jobname cdocscl1 \|\\
% |  "\input{childdoc.def}\childdocforward[cdocsamp]{cdocsch1}"|\\
% |latex -jobname cdocscl2 \|\\
% |  "\def\version{final}\input{childdoc.def}\childdocforward{cdocsch2}"|
% \end{tabular}
% \end{center}
% Note that the trailing backslash on each first line
% merely continues the input to the second line
% (for convenient cut ant paste).
% Furthermore, the command |latex| can be replaced by any
% of its alternative versions such as |pdflatex|.
%
% %%%%%%%%%%%%%%%%%%%%%%%%%%%%%%%%%%%%%%%%%%%%%%%%%%%%%%%%%%%%%%%%%%%%%%%%%%%%%%
% %%%%%%%%%%%%%%%%%%%%%%%%%%%%%%%%%%%%%%%%%%%%%%%%%%%%%%%%%%%%%%%%%%%%%%%%%%%%%%
% \section{Implementation}
%\iffalse
%<*package>
%\fi
%
% This section describes the definitions file |childdoc.def|.

% The definitions cannot be loaded using |\usepackage| or |\RequirePackage|
% which has a mechanism to prevent loading a style file more than once.
% When loading the definitions by means of |\input|
% multiple instances have to be prevented manually:
%\iffalse
%This code needs to be before the `\ProvidesFile' directive
%which is defined at the beginning of this file.
%Therefore it is also placed there and commented out here.
%</package>
%<*discard>
%\fi
%    \begin{macrocode}
\ifdefined\childdocmain\endinput\fi
%    \end{macrocode}
%\iffalse
%</discard>
%<*package>
%\fi
%
% \macro{\ifchilddoc}
% \macro{\ifchilddocmanual}
% The conditional |\ifchilddoc| tells whether a
% child (true) or main (false) document is being compiled.
% The conditional |\ifchilddocmanual| tells whether
% the |\includeonly| mechanism is used (false) or
% the selection of child files must be performed manually (true).
% The definitions initialise to false:
%    \begin{macrocode}
\newif\ifchilddoc
\newif\ifchilddocmanual
%    \end{macrocode}

% \macro{\childdocname}
% \macro{\childdocjob}
% The macro |\childdocname| stores the name of the main document
% to be compiled. The macro |\childdocjob| stores the name of
% the document on which the \LaTeX{} compiler was originally invoked.
% The content of |\jobname| cannot be compared
% to filenames specified in the source due to different catcodes.
% The following code rescans |\jobname|, stores the result
% in |\childdocname| and saves a copy in |\childdocjob|:
%    \begin{macrocode}
\edef\childdocname{\scantokens\expandafter{\jobname\noexpand}}
\let\childdocjob\childdocname
%    \end{macrocode}

% \macro{\childdocdisable}
% The macro |\childdocdisable| prevents the main file
% from being processed more than once.
% At this stage, the main document command |\childdocmain|
% is assumed to be called once again where it should do nothing.
% Any subsequent call to it should prevent
% a secondary processing of the main document
% It overwrites the forwarding commands
% |\childdocof| and |\childdocforward|
% with empty macros to prevent further inclusions of the main document:
%    \begin{macrocode}
\newcommand{\childdocdisable}
{
  \renewcommand{\childdocmain}[1]{\renewcommand{\childdocmain}[1]{\endinput}}
  \renewcommand{\childdocof}[1]{}
  \renewcommand{\childdocby}[2][]{}
  \renewcommand{\childdocforward}[2][]{}
  \renewcommand{\childdocdisable}{}
}
%    \end{macrocode}

% \macro{\childdocmain}
% The macro |\childdocmain| is to be called at the top of the main file
% with nothing or the main filename (without extension) as argument.
% First, it breaks loops.
% If the argument is not empty and does not match |\childdocname|
% (which is set by the first inclusion of |childdoc.def|),
% |\ifchilddoc| is set to true, |\includeonly| is applied to the child file
% and |\jobname| is set to the main file
% (for proper handling of |.aux| files):
%    \begin{macrocode}
\newcommand{\childdocmain}[1]
{
  \childdocdisable\childdocmain{}
  \if?#1?\else
    \begingroup
      \def\childdoctmp{#1}
      \ifx\childdoctmp\childdocname
        \def\childdoctmp{}
      \else
        \def\childdoctmp
        {
          \childdoctrue
          \includeonly{\childdocname}
          \def\childdocjob{#1}
          \def\jobname{#1}
        }
      \fi
      \expandafter
    \endgroup
    \childdoctmp
  \fi
}
%    \end{macrocode}

% \macro{\childdocof}
% The command |\childdocof| redirects
% compilation to the main file |#1|.
%    \begin{macrocode}
\newcommand{\childdocof}[1]
{
  \childdocdisable
  \childdoctrue
  \includeonly{\childdocname}
  \def\jobname{#1}
  \def\childdocjob{#1}
  \input{#1}
}
%    \end{macrocode}

% \macro{\childdocby}
% The command |\childdocby| ....
%    \begin{macrocode}
\newcommand{\childdocby}[2][]
{
  \childdocdisable
  \childdoctrue
  \childdocmanualtrue
  \if?#1?\else
    \def\jobname{#2}
  \fi
  \def\childdocjob{#2}
  \input{#2}
  \endinput
}
%    \end{macrocode}

% \macro{\childdocforward}
% The command |\childdocforward| redirects
% compilation to the main file or
% (if the optional argument is given) a child file.
% Parameters are set as if the main file
% or a child file starting with |\childdocof| was compiled.
% Then compilation is handed over to the main file:
%    \begin{macrocode}
\newcommand{\childdocforward}[2][]
{
  \begingroup
    \if?#1?
      \def\childdoctmp
      {
        \def\childdocname{#2}
        \def\childdocjob{#2}
        \def\jobname{#2}
        \input{#2}
        \endinput
      }
    \else
      \def\childdoctmp
      {
        \childdocdisable
        \def\childdocname{#2}
        \childdoctrue
        \includeonly{#2}
        \def\childdocjob{#1}
        \def\jobname{#1}
        \input{#1}
        \endinput
      }
    \fi
    \expandafter
  \endgroup
  \childdoctmp
}
%    \end{macrocode}

% \macro{\childdocforwardprefix}
% The command |\childdocforwardprefix| redirects
% compilation to the main or a child file by means of a pattern.
% The prefix |#1| in the current filename is replaced by |#2|
% and the suffix of the current filename is kept
% (it is assumed that the filename does not contain the substring `|~~~|'
% which is used as a delimiter).
% Compilation is handed over to the new file by |\childdocforward|:
%    \begin{macrocode}
\newcommand{\childdocforwardprefix}[3][]
{
  \begingroup
    \def\childdocextract #2##1~~~{\def\childdoctmp{\childdocforward[#1]{#3##1}}}
    \expandafter\childdocextract\childdocname~~~
    \expandafter
  \endgroup
  \childdoctmp
}
%    \end{macrocode}

% \macro{\childdoc}
% The deprecated macro |\childdoc| is a legacy version of |\childdocmain|:
%    \begin{macrocode}
\newcommand{\childdoc}{\childdocmain}
%    \end{macrocode}

% \macro{\childdocredirect}
% The deprecated macro |\childdocredirect| is a legacy version
% of |\childdocforward| and |\childdocforwardprefix|:
%    \begin{macrocode}
\newcommand{\childdocredirect}[2][]
{
  \begingroup
    \if?#1?
      \def\childdoctmp{\childdocforward{#2}}
    \else
      \def\childdoctmp{\childdocforwardprefix{#1}{#2}}
    \fi
    \expandafter
  \endgroup
  \childdoctmp
}
%    \end{macrocode}

%\iffalse
%</package>
%\fi
%
\endinput

\childdocof{cdocsamp}
%    \end{macrocode}

%\iffalse
%</samplechap1|samplechap2>
%\fi
%
%\iffalse
%<*samplechap1>
%\fi
% Some text for chapter 1:
%    \begin{macrocode}
\section{one}
some text in chapter one
%    \end{macrocode}

%\iffalse
%</samplechap1>
%\fi
% Some text for chapter 2:
%\iffalse
%<*samplechap2>
%\fi
%    \begin{macrocode}
\section{two}
more text in chapter two
%    \end{macrocode}

%\iffalse
%</samplechap2>
%\fi
%
% %%%%%%%%%%%%%%%%%%%%%%%%%%%%%%%%%%%%%%
% \paragraph{Part Include Files.}
%
% The include files are called |cdocspt3.tex| and |cdocspt4.tex|.
%
%\iffalse
%<*samplepart3|samplepart4>
%\fi

% Optional override for |\version| flag:
%    \begin{macrocode}
%%\providecommand{\version}{final}
%    \end{macrocode}

% Include the main document:
%    \begin{macrocode}
% \iffalse
%
% childdoc.dtx Copyright (C) 2017-2018 Niklas Beisert
%
% This work may be distributed and/or modified under the
% conditions of the LaTeX Project Public License, either version 1.3
% of this license or (at your option) any later version.
% The latest version of this license is in
%   http://www.latex-project.org/lppl.txt
% and version 1.3 or later is part of all distributions of LaTeX
% version 2005/12/01 or later.
%
% This work has the LPPL maintenance status `maintained'.
%
% The Current Maintainer of this work is Niklas Beisert.
%
% This work consists of the files childdoc.dtx and childdoc.ins
% and the derived files childdoc.def and cdocsamp.tex with
% cdocsch1.tex, cdocsch2.tex, cdocsdrf.tex, cdocsfn1.tex, cdocsfn2.tex.
%
%<package>\ifdefined\childdocmain\endinput\fi
%<package>\ProvidesFile{childdoc.def}[2018/12/30 v2.0 child document driver]
%<samplemain>\ProvidesFile{cdocsamp.tex}[2018/12/30 v2.0 sample for childdoc]
%<*driver>
%\ProvidesFile{childdoc.drv}[2018/12/30 v2.0 childdoc reference manual file]
\PassOptionsToClass{10pt,a4paper}{article}
\documentclass{ltxdoc}

\usepackage[margin=35mm]{geometry}
\usepackage{hyperref}
\usepackage{hyperxmp}
\usepackage[usenames]{color}

\hypersetup{colorlinks=true}
\hypersetup{pdfstartview=FitH}
\hypersetup{pdfpagemode=UseNone}
\hypersetup{pdfsource={}}
\hypersetup{pdflang={en-UK}}
\hypersetup{pdfcopyright={Copyright 2017-2018 Niklas Beisert.
  This work may be distributed and/or modified under the
  conditions of the LaTeX Project Public License, either version 1.3
  of this license or (at your option) any later version.}}
\hypersetup{pdflicenseurl={http://www.latex-project.org/lppl.txt}}
\hypersetup{pdfcontactaddress={ETH Zurich, ITP, HIT K,
  Wolfgang-Pauli-Strasse 27}}
\hypersetup{pdfcontactpostcode={8093}}
\hypersetup{pdfcontactcity={Zurich}}
\hypersetup{pdfcontactcountry={Switzerland}}
\hypersetup{pdfcontactemail={nbeisert@itp.phys.ethz.ch}}
\hypersetup{pdfcontacturl={http://people.phys.ethz.ch/\xmptilde nbeisert/}}

\newcommand{\secref}[1]{\hyperref[#1]{section \ref*{#1}}}

\parskip1ex
\parindent0pt
\let\olditemize\itemize
\def\itemize{\olditemize\parskip0pt}

\begin{document}

\title{The \textsf{childdoc} Package}
\hypersetup{pdftitle={The childdoc Package}}
\author{Niklas Beisert\\[2ex]
  Institut f\"ur Theoretische Physik\\
  Eidgen\"ossische Technische Hochschule Z\"urich\\
  Wolfgang-Pauli-Strasse 27, 8093 Z\"urich, Switzerland\\[1ex]
  \href{mailto:nbeisert@itp.phys.ethz.ch}
  {\texttt{nbeisert@itp.phys.ethz.ch}}}
\hypersetup{pdfauthor={Niklas Beisert}}
\hypersetup{pdfsubject={Manual for the LaTeX2e Package childdoc}}
\date{30 December 2018, \textsf{v2.0}}
\maketitle

\begin{abstract}\noindent
\textsf{childdoc} is a \LaTeXe{} package
that enables the direct compilation
of document sections included by |\include|
to individual files.
\end{abstract}

\begingroup
\parskip0ex
\tableofcontents
\endgroup

%%%%%%%%%%%%%%%%%%%%%%%%%%%%%%%%%%%%%%%%%%%%%%%%%%%%%%%%%%%%%%%%%%%%%%%%%%%%%%%%
%%%%%%%%%%%%%%%%%%%%%%%%%%%%%%%%%%%%%%%%%%%%%%%%%%%%%%%%%%%%%%%%%%%%%%%%%%%%%%%%
\section{Introduction}

\LaTeX{} provides a mechanism to structure a large document (such as a book)
into a main file and several child files (containing the chapters)
using the |\include| command.
This mechanism is beneficial for documents
which span hundreds of pages in order to
make the source file(s) more manageable.
Moreover, compilation can be restricted to
selected child files by means of the |\includeonly| command.
The latter feature can be used to reduce the compilation time while editing
(this was significantly more useful in the earlier days of \LaTeX{})
or to generate a smaller document which is easier to navigate.
Another application of |\includeonly| is to generate
documents consisting of selected parts of the complete document.

However, there are a few drawbacks of the plain |\include| mechanism:
\begin{itemize}
\item
The child files cannot be compiled on their own,
they can only be compiled via the main file.
A naive editing environment
(such as a text editor with an option
to have the current file processed by \LaTeX)
may require one to switch to the main file before compiling;
attempting to compile the child file produces errors.
\item
The main file must be modified (each time)
to adjust the |\includeonly| command
to the present needs. This easily leaves the main file in a messy state.
\item
The generated document will always carry the filename
of the main document. This is inconvenient if
several child files are to be compiled and
to be kept for distribution.
\end{itemize}

The present package provides a simple interface
to make child files individually compilable by \LaTeX{}.
Compiling a child file then has the same effect as compiling
the main file with an |\includeonly| command
to select the appropriate child.
Moreover the generated document will carry the name of the child
rather than the main file.
This resolves all three above issues.

This feature is meant to make the editing of books,
thesis documents and lecture notes somewhat more convenient.
However, the package can also be used efficiently for
composing a series of documents (such as exercise sheets)
which are typically distributed individually.
It then assists the author in generating the individual documents
(potentially in different versions)
as well as a document containing the collected series.
Another application is in developing style files
or other kinds of included material
where compilation of the style file could redirect
to a sample or test file.

%%%%%%%%%%%%%%%%%%%%%%%%%%%%%%%%%%%%%%%%%%%%%%%%%%%%%%%%%%%%%%%%%%%%%%%%%%%%%%%%
%%%%%%%%%%%%%%%%%%%%%%%%%%%%%%%%%%%%%%%%%%%%%%%%%%%%%%%%%%%%%%%%%%%%%%%%%%%%%%%%
\section{Usage}

First of all, the package \textsf{childdoc} is \emph{not} a standard
\LaTeXe{} |.sty| style file! Therefore it needs to be invoked in
a non-standard way.

%%%%%%%%%%%%%%%%%%%%%%%%%%%%%%%%%%%%%%%%%%%%%%%%%%%%%%%%%%%%%%%%%%%%%%%%%%%%%%%%
\subsection{Included Files}
\label{sec:include}

%%%%%%%%%%%%%%%%%%%%%%%%%%%%%%%%%%%%%%%%
\DescribeMacro{\childdocmain}
To use the package, add the commands
\begin{center}
\begin{tabular}{l}
|\input{childdoc.def}|\\
|\childdocmain{}|\\
\end{tabular}
\end{center}
at the very top of the main \LaTeX{} file,
in particular \emph{before} the |\documentclass| statement!
The argument of |\childdocmain| should be left empty
(but it must be present).

%%%%%%%%%%%%%%%%%%%%%%%%%%%%%%%%%%%%%%%%
\DescribeMacro{\childdocof}
Furthermore, add the commands
\begin{center}
\begin{tabular}{l}
|\input{childdoc.def}|\\
|\childdocof{|\textit{main}|}|\\
\end{tabular}
\end{center}
at the top of every child file \textit{child}
which is included by |\include{|\textit{child}|}|
from within the main file
(or at least for those files to be compiled individually).
The argument \textit{main} must be the filename of the main file.

There are a couple of
considerations in setting up the main and child documents:

%%%%%%%%%%%%%%%%%%%%%%%%%%%%%%%%%%%%%%%%
\paragraph{Restrictions.}

Please note the following restrictions:
\begin{itemize}
\item
|\childdocmain| must be called with one argument \textit{main}
to ensure compatibility with earlier version of the package.
It must either be empty (|\childdocmain{}|)
or precisely match the filename of the main file in which it is specified.
See \secref{sec:detection} for further information.
\item
The filename \textit{main} must be specified without the |.tex| extension.
\item
The filename \textit{main} is case sensitive
(even in case-insensitive file systems)
due to internal string comparison.
\item
The argument \textit{main} should be fully expanded, it cannot be a macro.
\item
Subdirectories and special characters should be avoided in filenames.
\item
The command |\childdocmain{|\textit{main}|}| must be followed by a whitespace.
It should not be followed immediately by another command
or by a comment mark `|%|'.
This is because the \TeX{} parser reads the token immediately following
the argument of |\childdocmain| and puts it
at the beginning of every child section;
however, a white\-space is ignored.
\end{itemize}

%%%%%%%%%%%%%%%%%%%%%%%%%%%%%%%%%%%%%%%%
\paragraph{Content of Main File.}

It is advisable to place all content in the child files included by |\include|.
Any output contained in the main file will appear in all child documents
unless suppressed manually;
it cannot be suppressed automatically by the |\includeonly| directive
and thus should normally be avoided.
A method to include some content in the main file
by means of conditional processing is described in \secref{sec:conditional}.

%%%%%%%%%%%%%%%%%%%%%%%%%%%%%%%%%%%%%%%%
\paragraph{Page Numbering.}

When only a part of the document is compiled,
the appropriate numbering of pages
(as well as other status parameters)
is determined from the |.aux| files.
The latter contain information from previous passes.
However this information needs to propagate through
all intermediate child documents.
Therefore the page numbering in child documents may well
be inconsistent until the complete document is compiled at least once.

A useful (if unconventional) way to always ensure a consistent
page numbering is to restart the numbering in each child document
and denote the pages by `\textit{child}|.|\textit{page}'
where \textit{child} represents the chapter/section number of the child file.
This can be achieved by the command
|\numberwithin{page}{|\textit{child}|}|
of the \textsf{amsmath} package
where \textit{child} can be |chapter| or |section|
depending on the chosen structuring.
Alternatively, one can modify the macro |\thepage| appropriately
and reset the counter |page| at the start of each child file.

%%%%%%%%%%%%%%%%%%%%%%%%%%%%%%%%%%%%%%%%%%%%%%%%%%%%%%%%%%%%%%%%%%%%%%%%%%%%%%%%
\subsection{Conditional Processing}
\label{sec:conditional}

The package provides a mechanism to compile different versions
of a document. To customise the versions further some conditional processing
can come in handy to distinguish which version is being compiled.
The package provides two macros to describe the compilation context:

%%%%%%%%%%%%%%%%%%%%%%%%%%%%%%%%%%%%%%%%
\DescribeMacro{\ifchilddoc}
The conditional |\ifchilddoc| distinguishes between the compilation of
child documents and the main document:
%
\begin{center}
|\ifchilddoc |\textit{child-code}| |[|\||else |\textit{main-code}]| \||fi|
\end{center}

%%%%%%%%%%%%%%%%%%%%%%%%%%%%%%%%%%%%%%%%
\DescribeMacro{\childdocname}
\DescribeMacro{\childdocjob}
The macro |\childdocname| contains the filename (without extension)
of the main or child file being processed.
Note that |\childdocjob| will always contain the name of the main file.

%%%%%%%%%%%%%%%%%%%%%%%%%%%%%%%%%%%%%%%%
\paragraph{Title Page.}

Conditional processing can be used to include a title or banner page
in the main document when proper precautions are taken.
Importantly, the code in the main file should ensure that the page counter
(as well as other status parameters which are stored in the |.aux| files)
takes the same value after the conditional processing.
Otherwise the page numbers may take divergent values
depending on which part is compiled.

For example, a title page could be declared by:
%
\begin{center}
\begin{tabular}{l}
|\ifchilddoc\||else|\\
|\addtocounter{page}{-1}|\\
\textit{code for title page}\\
|\newpage|\\
|\||fi|
\end{tabular}
\end{center}
%
A banner page for the child documents can be generated by:
%
\begin{center}
\begin{tabular}{l}
|\ifchilddoc|\\
|\addtocounter{page}{-1}|\\
\textit{code for banner page}\\
|\newpage|\\
|\||fi|
\end{tabular}
\end{center}
%
Here one could write a message such as:
\begin{center}
|This is the part \childdocname{} of \childdocjob{}.|
\end{center}

%%%%%%%%%%%%%%%%%%%%%%%%%%%%%%%%%%%%%%%%%%%%%%%%%%%%%%%%%%%%%%%%%%%%%%%%%%%%%%%%
\subsection{Flags}
\label{sec:flags}

The package makes it easy to generate different versions
of the main or child documents.
To this end compilation flags can be defined
and assigned different default values.
They will be particularly useful in conjunction
with the forwarding mechanism described in \secref{sec:forward}.

For example, it may be useful to have a flag |\version|
which can be set to |draft| or |final|.
The document source will contain some conditional code
depending on the value of |\version|.
Suppose further, the flag should default to |final| for the main file
and to |draft| for child files
which is a natural assignment for editing the document.
This is achieved by placing the following code
in the preamble of the main document
(below the |\childdocmain| directive):
%
\begin{center}
\begin{tabular}{l}
|\ifchilddoc|\\
|\providecommand{\version}{draft}|\\
|\||else|\\
|\providecommand{\version}{final}|\\
|\||fi|
\end{tabular}
\end{center}
%
The definition by |\providecommand| makes sure
that previous definitions are not overwritten.
Further statements |\providecommand{\version}{...}|
can thus be added before the above code to override it.

For the main file, one might add a line
(between |\childdocmain| and the above block)
%
\begin{center}
|%\ifchilddoc\||else\providecommand{\version}{draft}\||fi|
\end{center}
%
which can be uncommented to produce a draft version.
Likewise one can add a line to the very top of a child file
(above the |\childdocof{|\textit{main}|}| directive)
%
\begin{center}
|%\providecommand{\version}{final}|
\end{center}
%
which can be uncommented to produce the final version of this child document.

%%%%%%%%%%%%%%%%%%%%%%%%%%%%%%%%%%%%%%%%%%%%%%%%%%%%%%%%%%%%%%%%%%%%%%%%%%%%%%%%
\subsection{Forwarding}
\label{sec:forward}

Different versions of the main or child documents
using compilation flags as described in \secref{sec:flags}
can be (permanently) stored in different files
for convenient compilation, viewing and distribution.
To this end, the package defines a command
to pass on compilation to a different file:

%%%%%%%%%%%%%%%%%%%%%%%%%%%%%%%%%%%%%%%%
\DescribeMacro{\childdocforward}
The command |\childdocforward| redirects processing to
another source file:
%
\begin{center}
\begin{tabular}{l}
|\input{childdoc.def}|\\
|\childdocforward[|\textit{main}|]{|\textit{dest}|}|\\
\end{tabular}
\end{center}
%
The argument \textit{dest} is the destination file
(without extension).
It should be the main file or one of the child files.
Note that further \textsf{childdoc} directives
such as |\childdocof| and |\childdocforward|
in the indicated file will be processed in this form.
The optional argument \textit{main}
passes on directly to the main file \textit{main}
while pretending to compile the child \textit{dest}.
This form behaves as if \textit{dest}
issues |\childdocof{|\textit{main}|}| right away,
and no further \textsf{childdoc} directives will be processed.

%%%%%%%%%%%%%%%%%%%%%%%%%%%%%%%%%%%%%%%%
\DescribeMacro{\...prefix}
In the alternative form |\childdocforwardprefix|,
%
\begin{center}
\begin{tabular}{l}
|\input{childdoc.def}|\\
|\childdocforwardprefix[|\textit{main}|]{|\textit{prefix}|}{|\textit{dest}|}|
\end{tabular}
\end{center}
%
the destination file is determined by a pattern
depending on the current file:
To make this work, the current file must be called
`{\textit{prefix}\hspace{0.2em}\textit{suffix}}'
with \textit{prefix} matching precisely the argument.
Processing is then passed on to the file
`{\textit{dest}\hspace{0.2em}\textit{suffix}}'.
Surely, the same effect is achieved by
directly specifying the
argument `{\textit{dest}\hspace{0.2em}\textit{suffix}}'
in the first form.
However, that requires to set up a different file
for each child. With the alternative form of the command
all these files can have exactly the same content
which simplifies setting them up and maintaining them.

For example, the following file |draft.tex|
with a compilation flag |\version| as described in \secref{sec:flags}
compiles the main document as a draft:
%
\begin{center}
\begin{tabular}{l}
|\def\version{draft}|\\
|\input{childdoc.def}|\\
|\childdocforward{|\textit{main}|}|
\end{tabular}
\end{center}
%
Likewise, the following files |final|\textit{nn}|.tex|
compile the final version of the child document
|child|\textit{nn}|.tex|:
%
\begin{center}
\begin{tabular}{l}
|\def\version{final}|\\
|\input{childdoc.def}|\\
|\childdocforwardprefix{final}{child}|
\end{tabular}
\end{center}
%

Note that when several versions of a main file and/or of each child file
are to be generated, it may be convenient to set up a |Makefile| or
shell script to automatise the process.

%%%%%%%%%%%%%%%%%%%%%%%%%%%%%%%%%%%%%%%%%%%%%%%%%%%%%%%%%%%%%%%%%%%%%%%%%%%%%%%%
\subsection{Command Line Processing}
\label{sec:commandline}

The effect of redirection files can also be achieved by invoking
the \LaTeX{} compiler with a more elaborate command line.
Most conveniently this should be done as part
of a shell script or a |Makefile|.

When using \textsf{childdoc} in the main file, the following
command lines effectively perform a redirection
(note that depending on the shell being used,
backslashes may have to be doubled: `|\|' $\to$ `|\\|'):
%
\begin{center}
|... -jobname "|\textit{target}|" |\\|"|[\textit{flags}]%
|\input{childdoc.def}\childdocforward[|\textit{main}|]{|\textit{dest}|}"|
\end{center}
%
Here \textit{target} is the name of the output file,
\textit{main} is the name of the main file
and \textit{dest} is the name of the main or child file to be processed
(all filenames without extensions).
The optional argument \textit{main} can be omitted
if \textit{main} matches \textit{dest}.
Optionally, compilation \textit{flags} can be defined via |\def| commands.
This command line makes the \TeX{} engine believe
it is compiling the file \textit{target}
whose content is specified as the latter parameter.
The provided code then forwards the processing to
\textit{main} or \textit{dest} as described in \secref{sec:forward}.

%%%%%%%%%%%%%%%%%%%%%%%%%%%%%%%%%%%%%%%%%%%%%%%%%%%%%%%%%%%%%%%%%%%%%%%%%%%%%%%%
\subsection{Include by Input}
\label{sec:input}

Including child documents by |\include| has some restrictions by design.
Most notably, the content of a child document always occupies
its own set of pages; pages cannot be shared between child documents.
Usually, this behaviour makes perfect sense
because each child document contain an essential part of the document.
However, in some situations it may be desirable to compose
a document from a collection of parts
without having mandatory page breaks between then.
For this case, the package
provides a mechanism to include parts
by |\input| which can also be processed individually.
However, by construction this mechanism
requires manual handling of the content to be output.

%%%%%%%%%%%%%%%%%%%%%%%%%%%%%%%%%%%%%%%%
\DescribeMacro{\ifchilddocmanual}
The main file should be prepared as usual, see \secref{sec:include}.
However, the document body must make a distinction
between processing of an individual part and of the main document, e.g.:
%
\begin{center}
\begin{tabular}{l}
|\ifchilddocmanual|\\
|\input{\childdocname}|\\
|\||else|\\
\textit{document body with }|\input{|\textit{part}|}|\\
|\||fi|
\end{tabular}
\end{center}
%
The conditional |\ifchilddocmanual| is true whenever
a part to be included by |\input| is being compiled,
and the name of the part is stored in |\childdocname|.

%%%%%%%%%%%%%%%%%%%%%%%%%%%%%%%%%%%%%%%%
\DescribeMacro{\childdocby}
Each part to be included by |\input| should start with:
%
\begin{center}
\begin{tabular}{l}
|\input{childdoc.def}|\\
|\childdocby{|\textit{main}|}|\\
\end{tabular}
\end{center}
%
The directive |\childdocby| is similar to |\childdocof|
described in \secref{sec:include},
but the subsequent selection of content must be done manually.
To that end, both |\ifchilddoc| and |\ifchilddocmanual|
will be true upon processing of a part,
and the name of the part is stored in |\childdocname|.
Note that |\jobname| will be set to the filename of the current part
so that each part receives an individual |.aux| file
that does not interfere with the |.aux| file(s) of the main document.
This behaviour can be altered by the alternative form
|\childdocby[*]{|\textit{main}|}| (with a non-empty optional argument)
which uses the |.aux| file of the main document
by setting |\jobname| to \textit{main}.

%%%%%%%%%%%%%%%%%%%%%%%%%%%%%%%%%%%%%%%%%%%%%%%%%%%%%%%%%%%%%%%%%%%%%%%%%%%%%%%%
\subsection{Driver Development}
\label{sec:driver}

The \textsf{childdoc} mechanism can also be use for the development
of definition files such as \LaTeX{} styles or classes.
This case differs from the above setup with multiple parts
included by |\include| in that no |\includeonly| should be invoked.
This can be achieved by starting the include file
(before |\ProvidesPackage|) with:
%
\begin{center}
\begin{tabular}{l}
|\input{childdoc.def}|\\
|\childdocforward{|\textit{main}|}|\\
\end{tabular}
\end{center}
%
or alternatively with:
%
\begin{center}
\begin{tabular}{l}
|\input{childdoc.def}|\\
|\childdocby{|\textit{main}|}|\\
\end{tabular}
\end{center}
%
Both forms have slightly different effects as described above.
The main file is prepared as usual, see \secref{sec:include}.

%%%%%%%%%%%%%%%%%%%%%%%%%%%%%%%%%%%%%%%%%%%%%%%%%%%%%%%%%%%%%%%%%%%%%%%%%%%%%%%%
\subsection{Legacy Detection}
\label{sec:detection}

The directive |\childdocmain| in the main file can detect
whether the complete document or merely a child is to be compiled
even without using the directive |\childdocof|.
This method is deprecated because it is less robust
and there is no compelling reason to use it;
it is merely provided for backward compatibility
and it may be removed in future versions.

If the detection mechanism is to be used,
it is mandatory to correctly specify
the filename of the main file as the argument of |\childdocmain|:
%
\begin{center}
\begin{tabular}{l}
|\input{childdoc.def}|\\
|\childdocmain{|\textit{main}|}|\\
\end{tabular}
\end{center}
%
If |\jobname| does not match the argument \textit{main} of |\childdocmain|,
it is assumed that |\jobname| points to the child file to be compiled.
When using |\childdocmain| with the main file specified as argument,
it suffices to start a child file
with just |\input{|\textit{main}|}|
without loading of the package and using |\childdocof|.
If instead all processing is done
with the appropriate \textsf{childdoc} directives,
the argument of \textit{main} of |\childdocmain| can be empty.

An alternative version of the command line processing described
in \secref{sec:commandline} using the detection mechanism reads:
%
\begin{center}
|... -jobname "|\textit{target}|" "|[\textit{flags}]%
[|\def\jobname{|\textit{dest}|}|]|\input{|\textit{main}|}"|
\end{center}

%%%%%%%%%%%%%%%%%%%%%%%%%%%%%%%%%%%%%%%%%%%%%%%%%%%%%%%%%%%%%%%%%%%%%%%%%%%%%%%%
\subsection{Manual Code}
\label{sec:manual}

In case one cannot be certain whether the definitions file |childdoc.def|
is installed on the target \TeX{} distribution
and one prefers not to ship it,
it is conceivable to paste a few relevant commands into the sources.

To that end, drop all statements |\input{childdoc.def}|
and perform the replacements as outlined below.
Instead of |\childdocmain{|\textit{main}|}| add the following code
to the top of the main file:
%
\begin{center}
\begin{tabular}{l}
|\||ifdefined\childdocname\endinput\||fi\newif\ifchilddoc|\\
|\edef\childdocname{\scantokens\expandafter{\jobname\noexpand}}|\\
|\def\childdocmain{|\textit{main}|}\||ifx\childdocmain\childdocname\||else|\\
|\childdoctrue\includeonly{\childdocname}\let\jobname\childdocmain\||fi|\\
\end{tabular}
\end{center}
%
Instead of |\childdocof{|\textit{main}|}| just include the main file
at the top of each child file:
%
\begin{center}
|\input{|\textit{main}|}|
\end{center}
%
A simple redirection |\childdocforward{|\textit{dest}|}| is achieved by:
%
\begin{center}
|\def\jobname{|\textit{dest}|}\input{\jobname}|
\end{center}
%
The redirection with prefix
|\childdocforwardprefix[|\textit{prefix}|]{|\textit{dest}|}|
is accomplished by:
%
\begin{center}
\begin{tabular}{l}
|{\edef\jobname{\scantokens\expandafter{\jobname\noexpand}}|\\
|\def\redirectjob |\textit{prefix}|#1~~~{\gdef\jobname{|\textit{dest}|#1}}|\\
|\expandafter\redirectjob\jobname~~~}\input{\jobname}|
\end{tabular}
\end{center}

In an alternative approach,
child documents can be compiled by a specific command line
without additional code or specific definitions:
%
\begin{center}
|... -jobname "|\textit{target}|" "|[\textit{flags}]%
|\includeonly{|\textit{dest}|}\input{|\textit{main}|}"|
\end{center}
%

%%%%%%%%%%%%%%%%%%%%%%%%%%%%%%%%%%%%%%%%%%%%%%%%%%%%%%%%%%%%%%%%%%%%%%%%%%%%%%%%
%%%%%%%%%%%%%%%%%%%%%%%%%%%%%%%%%%%%%%%%%%%%%%%%%%%%%%%%%%%%%%%%%%%%%%%%%%%%%%%%
\section{Information}

%%%%%%%%%%%%%%%%%%%%%%%%%%%%%%%%%%%%%%%%%%%%%%%%%%%%%%%%%%%%%%%%%%%%%%%%%%%%%%%%
\subsection{Copyright}

Copyright \copyright{} 2017--2018 Niklas Beisert

This work may be distributed and/or modified under the
conditions of the \LaTeX{} Project Public License, either version 1.3
of this license or (at your option) any later version.
The latest version of this license is in
  \url{http://www.latex-project.org/lppl.txt}
and version 1.3 or later is part of all distributions of \LaTeX{}
version 2005/12/01 or later.

This work has the LPPL maintenance status `maintained'.

The Current Maintainer of this work is Niklas Beisert.

This work consists of the files |README.txt|, |childdoc.ins| and |childdoc.dtx|
as well as the derived files |childdoc.def|, |cdocsamp.tex|
with |cdocsch1.tex|, |cdocsch2.tex|, |cdocspt3.tex|, |cdocspt4.tex|,
|cdocsdrf.tex|, |cdocsfn1.tex|, |cdocsfn2.tex|
as well as |childdoc.pdf|.

%%%%%%%%%%%%%%%%%%%%%%%%%%%%%%%%%%%%%%%%%%%%%%%%%%%%%%%%%%%%%%%%%%%%%%%%%%%%%%%%
\subsection{Files and Installation}

The package consists of the files:
%
\begin{center}
\begin{tabular}{ll}
    |README.txt|   & readme file \\
    |childdoc.ins| & installation file \\
    |childdoc.dtx| & source file \\
    |childdoc.def| & definition file \\
    |cdocsamp.tex| & sample main file \\
    |cdocsch1.tex| & sample include file \\
    |cdocsch2.tex| & sample include file \\
    |cdocspt3.tex| & sample part file \\
    |cdocspt4.tex| & sample part file \\
    |cdocsdrf.tex| & sample redirection file \\
    |cdocsfn1.tex| & sample redirection file \\
    |cdocsfn2.tex| & sample redirection file \\
    |childdoc.pdf| & manual
\end{tabular}
\end{center}
%
The distribution consists of the files
|README.txt|, |childdoc.ins| and |childdoc.dtx|.
%
\begin{itemize}
\item
Run (pdf)\LaTeX{} on |childdoc.dtx|
to compile the manual |childdoc.pdf| (this file).
\item
Run \LaTeX{} on |childdoc.ins| to create the definitions file |childdoc.def|
and the sample |cdocsamp.tex| with include files
|cdocsch1.tex|, |cdocsch2.tex|, |cdocspt3.tex|, |cdocspt4.tex|,
|cdocsdrf.tex|, |cdocsfn1.tex|, |cdocsfn2.tex|.
Then copy the file |childdoc.def| to an appropriate directory of your \LaTeX{}
distribution, e.g.\ \textit{texmf-root}|/tex/latex/childdoc|.
\end{itemize}

%%%%%%%%%%%%%%%%%%%%%%%%%%%%%%%%%%%%%%%%%%%%%%%%%%%%%%%%%%%%%%%%%%%%%%%%%%%%%%%%
\subsection{Related CTAN Packages}

There are several other packages which offer a similar functionality:
%
\begin{itemize}
\item
The packages
\href{http://ctan.org/pkg/docmute}{\textsf{docmute}},
\href{http://ctan.org/pkg/includex}{\textsf{includex}} and
\href{http://ctan.org/pkg/standalone}{\textsf{standalone}}
provide commands to include only the document body of
a child file thus allowing both files to be compiled individually.
\item
The packages \href{http://ctan.org/pkg/subdocs}{\textsf{subdocs}}
and \href{http://ctan.org/pkg/subfiles}{\textsf{subfiles}}
provide structures in which the main and child documents can be
encapsulated and allowing them to be compiled individually.
The inclusion mechanism is different from the conventional |\include|.
\item
The package \href{http://ctan.org/pkg/combine}{\textsf{combine}}
is an elaborate solution to combine several documents into one.
\end{itemize}
%
See also the CTAN topic \href{http://ctan.org/topic/subdocs}{\textsf{subdocs}}
for further related packages.
The present package differs from the above solutions in that
a document structure constructed with the conventional |\include| mechanism
just needs two extra commands at the top of every file
such that all constituent files can be compiled individually.

%%%%%%%%%%%%%%%%%%%%%%%%%%%%%%%%%%%%%%%%%%%%%%%%%%%%%%%%%%%%%%%%%%%%%%%%%%%%%%%%
%\subsection{Feature Suggestions}
%
%The following is a list of features which may be useful for future
%versions of this package:
%%
%\begin{itemize}
%\item
%\ldots
%\end{itemize}

%%%%%%%%%%%%%%%%%%%%%%%%%%%%%%%%%%%%%%%%%%%%%%%%%%%%%%%%%%%%%%%%%%%%%%%%%%%%%%%%
\subsection{Revision History}

%%%%%%%%%%%%%%%%%%%%%%%%%%%%%%%%%%%%%%%%
\paragraph{v2.0:} 2018/12/30

\begin{itemize}
\item
immediate forward processing
\item
added |\childdocby| mechanism
\item
manual restructured
\end{itemize}

%%%%%%%%%%%%%%%%%%%%%%%%%%%%%%%%%%%%%%%%
\paragraph{v1.6:} 2018/01/17

\begin{itemize}
\item
application for development of include files
\item
corrections to manual
\end{itemize}

%%%%%%%%%%%%%%%%%%%%%%%%%%%%%%%%%%%%%%%%
\paragraph{v1.5:} 2017/05/21

\begin{itemize}
\item
more complete structuring introduced
\item
|\childdocof| introduced
\item
|\childdoc| renamed to |\childdocmain|
\item
|\childredirect| renamed to |\childdocforward| and |\childdocforwardprefix|
and functionality expanded
\end{itemize}

%%%%%%%%%%%%%%%%%%%%%%%%%%%%%%%%%%%%%%%%
\paragraph{v1.0:} 2017/04/27

\begin{itemize}
\item
manual and install package
\item
first version published on CTAN
\end{itemize}

%%%%%%%%%%%%%%%%%%%%%%%%%%%%%%%%%%%%%%%%
\paragraph{v0.6:} 2017/04/26

\begin{itemize}
\item
redirection mechanism added
\end{itemize}

%%%%%%%%%%%%%%%%%%%%%%%%%%%%%%%%%%%%%%%%
\paragraph{v0.5:} 2017/04/26

\begin{itemize}
\item
functionality in definition file
\end{itemize}


%%%%%%%%%%%%%%%%%%%%%%%%%%%%%%%%%%%%%%%%%%%%%%%%%%%%%%%%%%%%%%%%%%%%%%%%%%%%%%%%
%%%%%%%%%%%%%%%%%%%%%%%%%%%%%%%%%%%%%%%%%%%%%%%%%%%%%%%%%%%%%%%%%%%%%%%%%%%%%%%%
%%%%%%%%%%%%%%%%%%%%%%%%%%%%%%%%%%%%%%%%%%%%%%%%%%%%%%%%%%%%%%%%%%%%%%%%%%%%%%%%
\appendix

\settowidth\MacroIndent{\rmfamily\scriptsize 000\ }

 \DocInput{childdoc.dtx}

\end{document}
%</driver>
% \fi
%
% %%%%%%%%%%%%%%%%%%%%%%%%%%%%%%%%%%%%%%%%%%%%%%%%%%%%%%%%%%%%%%%%%%%%%%%%%%%%%%
% %%%%%%%%%%%%%%%%%%%%%%%%%%%%%%%%%%%%%%%%%%%%%%%%%%%%%%%%%%%%%%%%%%%%%%%%%%%%%%
% \section{Sample}
%\iffalse
%<*samplemain>
%\fi
%
% The following presents a sample document
% with two chapters, two parts, a title page,
% a compile flag as well as three forwarding files to set the flag.
% It consists of eight |.tex| files:
% \begin{center}
% \begin{tabular}{ll}
% |cdocsamp.tex|&main file\\
% |cdocsch1.tex|&include file for chapter 1\\
% |cdocsch2.tex|&include file for chapter 2\\
% |cdocspt3.tex|&include file for part 3\\
% |cdocspt4.tex|&include file for part 4\\
% |cdocsdrf.tex|&forwarding file for main file in draft mode\\
% |cdocsfi1.tex|&forwarding file for final version of chapter 1\\
% |cdocsfi2.tex|&forwarding file for final version of chapter 2\\
% \end{tabular}
% \end{center}
% Each of the eight files can be compiled directly by the \LaTeX{} compiler.
%
% %%%%%%%%%%%%%%%%%%%%%%%%%%%%%%%%%%%%%%
% \paragraph{Main File.}
%
% The main file is called |cdocsamp.tex|.
%
% Load the \textsf{childdoc} definitions and
% declare the filename for the main document:
%    \begin{macrocode}
\input{childdoc.def}
\childdocmain{}
%    \end{macrocode}

% Optional override for |\version| flag:
%    \begin{macrocode}
%%\ifchilddoc\else\providecommand{\version}{draft}\fi
%    \end{macrocode}

% Define the default values for the |\version| flag
% (|final| for the main file and |draft| for childs):
%    \begin{macrocode}
\ifchilddoc
\providecommand{\version}{draft}
\else
\providecommand{\version}{final}
\fi
%    \end{macrocode}

% Load the standard document class:
%    \begin{macrocode}
\documentclass[12pt]{article}
%    \end{macrocode}

% Start the document body:
%    \begin{macrocode}
\begin{document}
%    \end{macrocode}

% Declare a title page.
% Print title, part of document being processed and version flag:
%    \begin{macrocode}
\addtocounter{page}{-1}
\begin{center}
{\LARGE\bfseries{}childdoc example\par}
\vspace{1cm}
\ifchilddoc
\ifchilddocmanual part\else chapter\fi:
`\childdocname' of `\childdocjob'\par
\else
main document: `\childdocjob'\par
\fi
version: \version\par
\end{center}
\newpage
%    \end{macrocode}

% Manually include selected file,
% otherwise process as usual:
%    \begin{macrocode}
\ifchilddocmanual
\section*{part `\childdocname'}
\input{\childdocname}
\else
%    \end{macrocode}

% Include the two chapters:
%    \begin{macrocode}
\include{cdocsch1}
\include{cdocsch2}
%    \end{macrocode}

% Include the two parts unless only chapters should be displayed:
%    \begin{macrocode}
\ifchilddoc\else
\section{part three}
\input{cdocspt3}
\section{part four}
\input{cdocspt4}
\fi
%    \end{macrocode}

% Process as usual until here:
%    \begin{macrocode}
\fi
%    \end{macrocode}

% End of document body:
%    \begin{macrocode}
\end{document}
%    \end{macrocode}
%\iffalse
%</samplemain>
%\fi
%
% %%%%%%%%%%%%%%%%%%%%%%%%%%%%%%%%%%%%%%
% \paragraph{Chapter Include Files.}
%
% The include files are called |cdocsch1.tex| and |cdocsch2.tex|.
%
%\iffalse
%<*samplechap1|samplechap2>
%\fi

% Optional override for |\version| flag:
%    \begin{macrocode}
%%\providecommand{\version}{final}
%    \end{macrocode}

% Include the main document:
%    \begin{macrocode}
\input{childdoc.def}
\childdocof{cdocsamp}
%    \end{macrocode}

%\iffalse
%</samplechap1|samplechap2>
%\fi
%
%\iffalse
%<*samplechap1>
%\fi
% Some text for chapter 1:
%    \begin{macrocode}
\section{one}
some text in chapter one
%    \end{macrocode}

%\iffalse
%</samplechap1>
%\fi
% Some text for chapter 2:
%\iffalse
%<*samplechap2>
%\fi
%    \begin{macrocode}
\section{two}
more text in chapter two
%    \end{macrocode}

%\iffalse
%</samplechap2>
%\fi
%
% %%%%%%%%%%%%%%%%%%%%%%%%%%%%%%%%%%%%%%
% \paragraph{Part Include Files.}
%
% The include files are called |cdocspt3.tex| and |cdocspt4.tex|.
%
%\iffalse
%<*samplepart3|samplepart4>
%\fi

% Optional override for |\version| flag:
%    \begin{macrocode}
%%\providecommand{\version}{final}
%    \end{macrocode}

% Include the main document:
%    \begin{macrocode}
\input{childdoc.def}
\childdocby{cdocsamp}
%    \end{macrocode}

%\iffalse
%</samplepart3|samplepart4>
%\fi
%
%\iffalse
%<*samplepart3>
%\fi
% Some text for part 3:
%    \begin{macrocode}
some text in part three
%    \end{macrocode}

%\iffalse
%</samplepart3>
%\fi
% Some text for part 4:
%\iffalse
%<*samplepart4>
%\fi
%    \begin{macrocode}
more text in part four
%    \end{macrocode}

%\iffalse
%</samplepart4>
%\fi
%
% %%%%%%%%%%%%%%%%%%%%%%%%%%%%%%%%%%%%%%
% \paragraph{Forwarding for a Complete Draft.}
%
% The following forwarding file |cdocsdrf.tex|
% compiles the main document in draft mode:
%\iffalse
%<*sampledraft>
%\fi
%    \begin{macrocode}
\def\version{draft}
\input{childdoc.def}
\childdocforward{cdocsamp}
%    \end{macrocode}

%\iffalse
%</sampledraft>
%\fi
%
% %%%%%%%%%%%%%%%%%%%%%%%%%%%%%%%%%%%%%%
% \paragraph{Forwarding for Final Version of the Chapters.}
%
% The following forwarding files |cdocsfn1.tex| and |cdocsfn2.tex|
% (with identical content)
% compile the final versions of the child documents
% |cdocsch1.tex| and |cdocsch2.tex|, respectively:
%\iffalse
%<*samplefinal>
%\fi
%    \begin{macrocode}
\def\version{final}
\input{childdoc.def}
\childdocforwardprefix[cdocsamp]{cdocsfn}{cdocsch}
%    \end{macrocode}

%\iffalse
%</samplefinal>
%\fi
%
% %%%%%%%%%%%%%%%%%%%%%%%%%%%%%%%%%%%%%%
% \paragraph{Command Line Processing.}
%
% The following three command lines generate the output files
% |cdocscld|, |cdocscl1| and |cdocscl2|
% which should be identical to
% |cdocsdrf|, |cdocsch1| and |cdocsfn2|, respectively:
% \begin{center}
% \begin{tabular}{l}
% |latex -jobname cdocscld \|\\
% |  "\def\version{draft}\input{childdoc.def}\childdocforward{cdocsamp}"|\\
% |latex -jobname cdocscl1 \|\\
% |  "\input{childdoc.def}\childdocforward[cdocsamp]{cdocsch1}"|\\
% |latex -jobname cdocscl2 \|\\
% |  "\def\version{final}\input{childdoc.def}\childdocforward{cdocsch2}"|
% \end{tabular}
% \end{center}
% Note that the trailing backslash on each first line
% merely continues the input to the second line
% (for convenient cut ant paste).
% Furthermore, the command |latex| can be replaced by any
% of its alternative versions such as |pdflatex|.
%
% %%%%%%%%%%%%%%%%%%%%%%%%%%%%%%%%%%%%%%%%%%%%%%%%%%%%%%%%%%%%%%%%%%%%%%%%%%%%%%
% %%%%%%%%%%%%%%%%%%%%%%%%%%%%%%%%%%%%%%%%%%%%%%%%%%%%%%%%%%%%%%%%%%%%%%%%%%%%%%
% \section{Implementation}
%\iffalse
%<*package>
%\fi
%
% This section describes the definitions file |childdoc.def|.

% The definitions cannot be loaded using |\usepackage| or |\RequirePackage|
% which has a mechanism to prevent loading a style file more than once.
% When loading the definitions by means of |\input|
% multiple instances have to be prevented manually:
%\iffalse
%This code needs to be before the `\ProvidesFile' directive
%which is defined at the beginning of this file.
%Therefore it is also placed there and commented out here.
%</package>
%<*discard>
%\fi
%    \begin{macrocode}
\ifdefined\childdocmain\endinput\fi
%    \end{macrocode}
%\iffalse
%</discard>
%<*package>
%\fi
%
% \macro{\ifchilddoc}
% \macro{\ifchilddocmanual}
% The conditional |\ifchilddoc| tells whether a
% child (true) or main (false) document is being compiled.
% The conditional |\ifchilddocmanual| tells whether
% the |\includeonly| mechanism is used (false) or
% the selection of child files must be performed manually (true).
% The definitions initialise to false:
%    \begin{macrocode}
\newif\ifchilddoc
\newif\ifchilddocmanual
%    \end{macrocode}

% \macro{\childdocname}
% \macro{\childdocjob}
% The macro |\childdocname| stores the name of the main document
% to be compiled. The macro |\childdocjob| stores the name of
% the document on which the \LaTeX{} compiler was originally invoked.
% The content of |\jobname| cannot be compared
% to filenames specified in the source due to different catcodes.
% The following code rescans |\jobname|, stores the result
% in |\childdocname| and saves a copy in |\childdocjob|:
%    \begin{macrocode}
\edef\childdocname{\scantokens\expandafter{\jobname\noexpand}}
\let\childdocjob\childdocname
%    \end{macrocode}

% \macro{\childdocdisable}
% The macro |\childdocdisable| prevents the main file
% from being processed more than once.
% At this stage, the main document command |\childdocmain|
% is assumed to be called once again where it should do nothing.
% Any subsequent call to it should prevent
% a secondary processing of the main document
% It overwrites the forwarding commands
% |\childdocof| and |\childdocforward|
% with empty macros to prevent further inclusions of the main document:
%    \begin{macrocode}
\newcommand{\childdocdisable}
{
  \renewcommand{\childdocmain}[1]{\renewcommand{\childdocmain}[1]{\endinput}}
  \renewcommand{\childdocof}[1]{}
  \renewcommand{\childdocby}[2][]{}
  \renewcommand{\childdocforward}[2][]{}
  \renewcommand{\childdocdisable}{}
}
%    \end{macrocode}

% \macro{\childdocmain}
% The macro |\childdocmain| is to be called at the top of the main file
% with nothing or the main filename (without extension) as argument.
% First, it breaks loops.
% If the argument is not empty and does not match |\childdocname|
% (which is set by the first inclusion of |childdoc.def|),
% |\ifchilddoc| is set to true, |\includeonly| is applied to the child file
% and |\jobname| is set to the main file
% (for proper handling of |.aux| files):
%    \begin{macrocode}
\newcommand{\childdocmain}[1]
{
  \childdocdisable\childdocmain{}
  \if?#1?\else
    \begingroup
      \def\childdoctmp{#1}
      \ifx\childdoctmp\childdocname
        \def\childdoctmp{}
      \else
        \def\childdoctmp
        {
          \childdoctrue
          \includeonly{\childdocname}
          \def\childdocjob{#1}
          \def\jobname{#1}
        }
      \fi
      \expandafter
    \endgroup
    \childdoctmp
  \fi
}
%    \end{macrocode}

% \macro{\childdocof}
% The command |\childdocof| redirects
% compilation to the main file |#1|.
%    \begin{macrocode}
\newcommand{\childdocof}[1]
{
  \childdocdisable
  \childdoctrue
  \includeonly{\childdocname}
  \def\jobname{#1}
  \def\childdocjob{#1}
  \input{#1}
}
%    \end{macrocode}

% \macro{\childdocby}
% The command |\childdocby| ....
%    \begin{macrocode}
\newcommand{\childdocby}[2][]
{
  \childdocdisable
  \childdoctrue
  \childdocmanualtrue
  \if?#1?\else
    \def\jobname{#2}
  \fi
  \def\childdocjob{#2}
  \input{#2}
  \endinput
}
%    \end{macrocode}

% \macro{\childdocforward}
% The command |\childdocforward| redirects
% compilation to the main file or
% (if the optional argument is given) a child file.
% Parameters are set as if the main file
% or a child file starting with |\childdocof| was compiled.
% Then compilation is handed over to the main file:
%    \begin{macrocode}
\newcommand{\childdocforward}[2][]
{
  \begingroup
    \if?#1?
      \def\childdoctmp
      {
        \def\childdocname{#2}
        \def\childdocjob{#2}
        \def\jobname{#2}
        \input{#2}
        \endinput
      }
    \else
      \def\childdoctmp
      {
        \childdocdisable
        \def\childdocname{#2}
        \childdoctrue
        \includeonly{#2}
        \def\childdocjob{#1}
        \def\jobname{#1}
        \input{#1}
        \endinput
      }
    \fi
    \expandafter
  \endgroup
  \childdoctmp
}
%    \end{macrocode}

% \macro{\childdocforwardprefix}
% The command |\childdocforwardprefix| redirects
% compilation to the main or a child file by means of a pattern.
% The prefix |#1| in the current filename is replaced by |#2|
% and the suffix of the current filename is kept
% (it is assumed that the filename does not contain the substring `|~~~|'
% which is used as a delimiter).
% Compilation is handed over to the new file by |\childdocforward|:
%    \begin{macrocode}
\newcommand{\childdocforwardprefix}[3][]
{
  \begingroup
    \def\childdocextract #2##1~~~{\def\childdoctmp{\childdocforward[#1]{#3##1}}}
    \expandafter\childdocextract\childdocname~~~
    \expandafter
  \endgroup
  \childdoctmp
}
%    \end{macrocode}

% \macro{\childdoc}
% The deprecated macro |\childdoc| is a legacy version of |\childdocmain|:
%    \begin{macrocode}
\newcommand{\childdoc}{\childdocmain}
%    \end{macrocode}

% \macro{\childdocredirect}
% The deprecated macro |\childdocredirect| is a legacy version
% of |\childdocforward| and |\childdocforwardprefix|:
%    \begin{macrocode}
\newcommand{\childdocredirect}[2][]
{
  \begingroup
    \if?#1?
      \def\childdoctmp{\childdocforward{#2}}
    \else
      \def\childdoctmp{\childdocforwardprefix{#1}{#2}}
    \fi
    \expandafter
  \endgroup
  \childdoctmp
}
%    \end{macrocode}

%\iffalse
%</package>
%\fi
%
\endinput

\childdocby{cdocsamp}
%    \end{macrocode}

%\iffalse
%</samplepart3|samplepart4>
%\fi
%
%\iffalse
%<*samplepart3>
%\fi
% Some text for part 3:
%    \begin{macrocode}
some text in part three
%    \end{macrocode}

%\iffalse
%</samplepart3>
%\fi
% Some text for part 4:
%\iffalse
%<*samplepart4>
%\fi
%    \begin{macrocode}
more text in part four
%    \end{macrocode}

%\iffalse
%</samplepart4>
%\fi
%
% %%%%%%%%%%%%%%%%%%%%%%%%%%%%%%%%%%%%%%
% \paragraph{Forwarding for a Complete Draft.}
%
% The following forwarding file |cdocsdrf.tex|
% compiles the main document in draft mode:
%\iffalse
%<*sampledraft>
%\fi
%    \begin{macrocode}
\def\version{draft}
% \iffalse
%
% childdoc.dtx Copyright (C) 2017-2018 Niklas Beisert
%
% This work may be distributed and/or modified under the
% conditions of the LaTeX Project Public License, either version 1.3
% of this license or (at your option) any later version.
% The latest version of this license is in
%   http://www.latex-project.org/lppl.txt
% and version 1.3 or later is part of all distributions of LaTeX
% version 2005/12/01 or later.
%
% This work has the LPPL maintenance status `maintained'.
%
% The Current Maintainer of this work is Niklas Beisert.
%
% This work consists of the files childdoc.dtx and childdoc.ins
% and the derived files childdoc.def and cdocsamp.tex with
% cdocsch1.tex, cdocsch2.tex, cdocsdrf.tex, cdocsfn1.tex, cdocsfn2.tex.
%
%<package>\ifdefined\childdocmain\endinput\fi
%<package>\ProvidesFile{childdoc.def}[2018/12/30 v2.0 child document driver]
%<samplemain>\ProvidesFile{cdocsamp.tex}[2018/12/30 v2.0 sample for childdoc]
%<*driver>
%\ProvidesFile{childdoc.drv}[2018/12/30 v2.0 childdoc reference manual file]
\PassOptionsToClass{10pt,a4paper}{article}
\documentclass{ltxdoc}

\usepackage[margin=35mm]{geometry}
\usepackage{hyperref}
\usepackage{hyperxmp}
\usepackage[usenames]{color}

\hypersetup{colorlinks=true}
\hypersetup{pdfstartview=FitH}
\hypersetup{pdfpagemode=UseNone}
\hypersetup{pdfsource={}}
\hypersetup{pdflang={en-UK}}
\hypersetup{pdfcopyright={Copyright 2017-2018 Niklas Beisert.
  This work may be distributed and/or modified under the
  conditions of the LaTeX Project Public License, either version 1.3
  of this license or (at your option) any later version.}}
\hypersetup{pdflicenseurl={http://www.latex-project.org/lppl.txt}}
\hypersetup{pdfcontactaddress={ETH Zurich, ITP, HIT K,
  Wolfgang-Pauli-Strasse 27}}
\hypersetup{pdfcontactpostcode={8093}}
\hypersetup{pdfcontactcity={Zurich}}
\hypersetup{pdfcontactcountry={Switzerland}}
\hypersetup{pdfcontactemail={nbeisert@itp.phys.ethz.ch}}
\hypersetup{pdfcontacturl={http://people.phys.ethz.ch/\xmptilde nbeisert/}}

\newcommand{\secref}[1]{\hyperref[#1]{section \ref*{#1}}}

\parskip1ex
\parindent0pt
\let\olditemize\itemize
\def\itemize{\olditemize\parskip0pt}

\begin{document}

\title{The \textsf{childdoc} Package}
\hypersetup{pdftitle={The childdoc Package}}
\author{Niklas Beisert\\[2ex]
  Institut f\"ur Theoretische Physik\\
  Eidgen\"ossische Technische Hochschule Z\"urich\\
  Wolfgang-Pauli-Strasse 27, 8093 Z\"urich, Switzerland\\[1ex]
  \href{mailto:nbeisert@itp.phys.ethz.ch}
  {\texttt{nbeisert@itp.phys.ethz.ch}}}
\hypersetup{pdfauthor={Niklas Beisert}}
\hypersetup{pdfsubject={Manual for the LaTeX2e Package childdoc}}
\date{30 December 2018, \textsf{v2.0}}
\maketitle

\begin{abstract}\noindent
\textsf{childdoc} is a \LaTeXe{} package
that enables the direct compilation
of document sections included by |\include|
to individual files.
\end{abstract}

\begingroup
\parskip0ex
\tableofcontents
\endgroup

%%%%%%%%%%%%%%%%%%%%%%%%%%%%%%%%%%%%%%%%%%%%%%%%%%%%%%%%%%%%%%%%%%%%%%%%%%%%%%%%
%%%%%%%%%%%%%%%%%%%%%%%%%%%%%%%%%%%%%%%%%%%%%%%%%%%%%%%%%%%%%%%%%%%%%%%%%%%%%%%%
\section{Introduction}

\LaTeX{} provides a mechanism to structure a large document (such as a book)
into a main file and several child files (containing the chapters)
using the |\include| command.
This mechanism is beneficial for documents
which span hundreds of pages in order to
make the source file(s) more manageable.
Moreover, compilation can be restricted to
selected child files by means of the |\includeonly| command.
The latter feature can be used to reduce the compilation time while editing
(this was significantly more useful in the earlier days of \LaTeX{})
or to generate a smaller document which is easier to navigate.
Another application of |\includeonly| is to generate
documents consisting of selected parts of the complete document.

However, there are a few drawbacks of the plain |\include| mechanism:
\begin{itemize}
\item
The child files cannot be compiled on their own,
they can only be compiled via the main file.
A naive editing environment
(such as a text editor with an option
to have the current file processed by \LaTeX)
may require one to switch to the main file before compiling;
attempting to compile the child file produces errors.
\item
The main file must be modified (each time)
to adjust the |\includeonly| command
to the present needs. This easily leaves the main file in a messy state.
\item
The generated document will always carry the filename
of the main document. This is inconvenient if
several child files are to be compiled and
to be kept for distribution.
\end{itemize}

The present package provides a simple interface
to make child files individually compilable by \LaTeX{}.
Compiling a child file then has the same effect as compiling
the main file with an |\includeonly| command
to select the appropriate child.
Moreover the generated document will carry the name of the child
rather than the main file.
This resolves all three above issues.

This feature is meant to make the editing of books,
thesis documents and lecture notes somewhat more convenient.
However, the package can also be used efficiently for
composing a series of documents (such as exercise sheets)
which are typically distributed individually.
It then assists the author in generating the individual documents
(potentially in different versions)
as well as a document containing the collected series.
Another application is in developing style files
or other kinds of included material
where compilation of the style file could redirect
to a sample or test file.

%%%%%%%%%%%%%%%%%%%%%%%%%%%%%%%%%%%%%%%%%%%%%%%%%%%%%%%%%%%%%%%%%%%%%%%%%%%%%%%%
%%%%%%%%%%%%%%%%%%%%%%%%%%%%%%%%%%%%%%%%%%%%%%%%%%%%%%%%%%%%%%%%%%%%%%%%%%%%%%%%
\section{Usage}

First of all, the package \textsf{childdoc} is \emph{not} a standard
\LaTeXe{} |.sty| style file! Therefore it needs to be invoked in
a non-standard way.

%%%%%%%%%%%%%%%%%%%%%%%%%%%%%%%%%%%%%%%%%%%%%%%%%%%%%%%%%%%%%%%%%%%%%%%%%%%%%%%%
\subsection{Included Files}
\label{sec:include}

%%%%%%%%%%%%%%%%%%%%%%%%%%%%%%%%%%%%%%%%
\DescribeMacro{\childdocmain}
To use the package, add the commands
\begin{center}
\begin{tabular}{l}
|\input{childdoc.def}|\\
|\childdocmain{}|\\
\end{tabular}
\end{center}
at the very top of the main \LaTeX{} file,
in particular \emph{before} the |\documentclass| statement!
The argument of |\childdocmain| should be left empty
(but it must be present).

%%%%%%%%%%%%%%%%%%%%%%%%%%%%%%%%%%%%%%%%
\DescribeMacro{\childdocof}
Furthermore, add the commands
\begin{center}
\begin{tabular}{l}
|\input{childdoc.def}|\\
|\childdocof{|\textit{main}|}|\\
\end{tabular}
\end{center}
at the top of every child file \textit{child}
which is included by |\include{|\textit{child}|}|
from within the main file
(or at least for those files to be compiled individually).
The argument \textit{main} must be the filename of the main file.

There are a couple of
considerations in setting up the main and child documents:

%%%%%%%%%%%%%%%%%%%%%%%%%%%%%%%%%%%%%%%%
\paragraph{Restrictions.}

Please note the following restrictions:
\begin{itemize}
\item
|\childdocmain| must be called with one argument \textit{main}
to ensure compatibility with earlier version of the package.
It must either be empty (|\childdocmain{}|)
or precisely match the filename of the main file in which it is specified.
See \secref{sec:detection} for further information.
\item
The filename \textit{main} must be specified without the |.tex| extension.
\item
The filename \textit{main} is case sensitive
(even in case-insensitive file systems)
due to internal string comparison.
\item
The argument \textit{main} should be fully expanded, it cannot be a macro.
\item
Subdirectories and special characters should be avoided in filenames.
\item
The command |\childdocmain{|\textit{main}|}| must be followed by a whitespace.
It should not be followed immediately by another command
or by a comment mark `|%|'.
This is because the \TeX{} parser reads the token immediately following
the argument of |\childdocmain| and puts it
at the beginning of every child section;
however, a white\-space is ignored.
\end{itemize}

%%%%%%%%%%%%%%%%%%%%%%%%%%%%%%%%%%%%%%%%
\paragraph{Content of Main File.}

It is advisable to place all content in the child files included by |\include|.
Any output contained in the main file will appear in all child documents
unless suppressed manually;
it cannot be suppressed automatically by the |\includeonly| directive
and thus should normally be avoided.
A method to include some content in the main file
by means of conditional processing is described in \secref{sec:conditional}.

%%%%%%%%%%%%%%%%%%%%%%%%%%%%%%%%%%%%%%%%
\paragraph{Page Numbering.}

When only a part of the document is compiled,
the appropriate numbering of pages
(as well as other status parameters)
is determined from the |.aux| files.
The latter contain information from previous passes.
However this information needs to propagate through
all intermediate child documents.
Therefore the page numbering in child documents may well
be inconsistent until the complete document is compiled at least once.

A useful (if unconventional) way to always ensure a consistent
page numbering is to restart the numbering in each child document
and denote the pages by `\textit{child}|.|\textit{page}'
where \textit{child} represents the chapter/section number of the child file.
This can be achieved by the command
|\numberwithin{page}{|\textit{child}|}|
of the \textsf{amsmath} package
where \textit{child} can be |chapter| or |section|
depending on the chosen structuring.
Alternatively, one can modify the macro |\thepage| appropriately
and reset the counter |page| at the start of each child file.

%%%%%%%%%%%%%%%%%%%%%%%%%%%%%%%%%%%%%%%%%%%%%%%%%%%%%%%%%%%%%%%%%%%%%%%%%%%%%%%%
\subsection{Conditional Processing}
\label{sec:conditional}

The package provides a mechanism to compile different versions
of a document. To customise the versions further some conditional processing
can come in handy to distinguish which version is being compiled.
The package provides two macros to describe the compilation context:

%%%%%%%%%%%%%%%%%%%%%%%%%%%%%%%%%%%%%%%%
\DescribeMacro{\ifchilddoc}
The conditional |\ifchilddoc| distinguishes between the compilation of
child documents and the main document:
%
\begin{center}
|\ifchilddoc |\textit{child-code}| |[|\||else |\textit{main-code}]| \||fi|
\end{center}

%%%%%%%%%%%%%%%%%%%%%%%%%%%%%%%%%%%%%%%%
\DescribeMacro{\childdocname}
\DescribeMacro{\childdocjob}
The macro |\childdocname| contains the filename (without extension)
of the main or child file being processed.
Note that |\childdocjob| will always contain the name of the main file.

%%%%%%%%%%%%%%%%%%%%%%%%%%%%%%%%%%%%%%%%
\paragraph{Title Page.}

Conditional processing can be used to include a title or banner page
in the main document when proper precautions are taken.
Importantly, the code in the main file should ensure that the page counter
(as well as other status parameters which are stored in the |.aux| files)
takes the same value after the conditional processing.
Otherwise the page numbers may take divergent values
depending on which part is compiled.

For example, a title page could be declared by:
%
\begin{center}
\begin{tabular}{l}
|\ifchilddoc\||else|\\
|\addtocounter{page}{-1}|\\
\textit{code for title page}\\
|\newpage|\\
|\||fi|
\end{tabular}
\end{center}
%
A banner page for the child documents can be generated by:
%
\begin{center}
\begin{tabular}{l}
|\ifchilddoc|\\
|\addtocounter{page}{-1}|\\
\textit{code for banner page}\\
|\newpage|\\
|\||fi|
\end{tabular}
\end{center}
%
Here one could write a message such as:
\begin{center}
|This is the part \childdocname{} of \childdocjob{}.|
\end{center}

%%%%%%%%%%%%%%%%%%%%%%%%%%%%%%%%%%%%%%%%%%%%%%%%%%%%%%%%%%%%%%%%%%%%%%%%%%%%%%%%
\subsection{Flags}
\label{sec:flags}

The package makes it easy to generate different versions
of the main or child documents.
To this end compilation flags can be defined
and assigned different default values.
They will be particularly useful in conjunction
with the forwarding mechanism described in \secref{sec:forward}.

For example, it may be useful to have a flag |\version|
which can be set to |draft| or |final|.
The document source will contain some conditional code
depending on the value of |\version|.
Suppose further, the flag should default to |final| for the main file
and to |draft| for child files
which is a natural assignment for editing the document.
This is achieved by placing the following code
in the preamble of the main document
(below the |\childdocmain| directive):
%
\begin{center}
\begin{tabular}{l}
|\ifchilddoc|\\
|\providecommand{\version}{draft}|\\
|\||else|\\
|\providecommand{\version}{final}|\\
|\||fi|
\end{tabular}
\end{center}
%
The definition by |\providecommand| makes sure
that previous definitions are not overwritten.
Further statements |\providecommand{\version}{...}|
can thus be added before the above code to override it.

For the main file, one might add a line
(between |\childdocmain| and the above block)
%
\begin{center}
|%\ifchilddoc\||else\providecommand{\version}{draft}\||fi|
\end{center}
%
which can be uncommented to produce a draft version.
Likewise one can add a line to the very top of a child file
(above the |\childdocof{|\textit{main}|}| directive)
%
\begin{center}
|%\providecommand{\version}{final}|
\end{center}
%
which can be uncommented to produce the final version of this child document.

%%%%%%%%%%%%%%%%%%%%%%%%%%%%%%%%%%%%%%%%%%%%%%%%%%%%%%%%%%%%%%%%%%%%%%%%%%%%%%%%
\subsection{Forwarding}
\label{sec:forward}

Different versions of the main or child documents
using compilation flags as described in \secref{sec:flags}
can be (permanently) stored in different files
for convenient compilation, viewing and distribution.
To this end, the package defines a command
to pass on compilation to a different file:

%%%%%%%%%%%%%%%%%%%%%%%%%%%%%%%%%%%%%%%%
\DescribeMacro{\childdocforward}
The command |\childdocforward| redirects processing to
another source file:
%
\begin{center}
\begin{tabular}{l}
|\input{childdoc.def}|\\
|\childdocforward[|\textit{main}|]{|\textit{dest}|}|\\
\end{tabular}
\end{center}
%
The argument \textit{dest} is the destination file
(without extension).
It should be the main file or one of the child files.
Note that further \textsf{childdoc} directives
such as |\childdocof| and |\childdocforward|
in the indicated file will be processed in this form.
The optional argument \textit{main}
passes on directly to the main file \textit{main}
while pretending to compile the child \textit{dest}.
This form behaves as if \textit{dest}
issues |\childdocof{|\textit{main}|}| right away,
and no further \textsf{childdoc} directives will be processed.

%%%%%%%%%%%%%%%%%%%%%%%%%%%%%%%%%%%%%%%%
\DescribeMacro{\...prefix}
In the alternative form |\childdocforwardprefix|,
%
\begin{center}
\begin{tabular}{l}
|\input{childdoc.def}|\\
|\childdocforwardprefix[|\textit{main}|]{|\textit{prefix}|}{|\textit{dest}|}|
\end{tabular}
\end{center}
%
the destination file is determined by a pattern
depending on the current file:
To make this work, the current file must be called
`{\textit{prefix}\hspace{0.2em}\textit{suffix}}'
with \textit{prefix} matching precisely the argument.
Processing is then passed on to the file
`{\textit{dest}\hspace{0.2em}\textit{suffix}}'.
Surely, the same effect is achieved by
directly specifying the
argument `{\textit{dest}\hspace{0.2em}\textit{suffix}}'
in the first form.
However, that requires to set up a different file
for each child. With the alternative form of the command
all these files can have exactly the same content
which simplifies setting them up and maintaining them.

For example, the following file |draft.tex|
with a compilation flag |\version| as described in \secref{sec:flags}
compiles the main document as a draft:
%
\begin{center}
\begin{tabular}{l}
|\def\version{draft}|\\
|\input{childdoc.def}|\\
|\childdocforward{|\textit{main}|}|
\end{tabular}
\end{center}
%
Likewise, the following files |final|\textit{nn}|.tex|
compile the final version of the child document
|child|\textit{nn}|.tex|:
%
\begin{center}
\begin{tabular}{l}
|\def\version{final}|\\
|\input{childdoc.def}|\\
|\childdocforwardprefix{final}{child}|
\end{tabular}
\end{center}
%

Note that when several versions of a main file and/or of each child file
are to be generated, it may be convenient to set up a |Makefile| or
shell script to automatise the process.

%%%%%%%%%%%%%%%%%%%%%%%%%%%%%%%%%%%%%%%%%%%%%%%%%%%%%%%%%%%%%%%%%%%%%%%%%%%%%%%%
\subsection{Command Line Processing}
\label{sec:commandline}

The effect of redirection files can also be achieved by invoking
the \LaTeX{} compiler with a more elaborate command line.
Most conveniently this should be done as part
of a shell script or a |Makefile|.

When using \textsf{childdoc} in the main file, the following
command lines effectively perform a redirection
(note that depending on the shell being used,
backslashes may have to be doubled: `|\|' $\to$ `|\\|'):
%
\begin{center}
|... -jobname "|\textit{target}|" |\\|"|[\textit{flags}]%
|\input{childdoc.def}\childdocforward[|\textit{main}|]{|\textit{dest}|}"|
\end{center}
%
Here \textit{target} is the name of the output file,
\textit{main} is the name of the main file
and \textit{dest} is the name of the main or child file to be processed
(all filenames without extensions).
The optional argument \textit{main} can be omitted
if \textit{main} matches \textit{dest}.
Optionally, compilation \textit{flags} can be defined via |\def| commands.
This command line makes the \TeX{} engine believe
it is compiling the file \textit{target}
whose content is specified as the latter parameter.
The provided code then forwards the processing to
\textit{main} or \textit{dest} as described in \secref{sec:forward}.

%%%%%%%%%%%%%%%%%%%%%%%%%%%%%%%%%%%%%%%%%%%%%%%%%%%%%%%%%%%%%%%%%%%%%%%%%%%%%%%%
\subsection{Include by Input}
\label{sec:input}

Including child documents by |\include| has some restrictions by design.
Most notably, the content of a child document always occupies
its own set of pages; pages cannot be shared between child documents.
Usually, this behaviour makes perfect sense
because each child document contain an essential part of the document.
However, in some situations it may be desirable to compose
a document from a collection of parts
without having mandatory page breaks between then.
For this case, the package
provides a mechanism to include parts
by |\input| which can also be processed individually.
However, by construction this mechanism
requires manual handling of the content to be output.

%%%%%%%%%%%%%%%%%%%%%%%%%%%%%%%%%%%%%%%%
\DescribeMacro{\ifchilddocmanual}
The main file should be prepared as usual, see \secref{sec:include}.
However, the document body must make a distinction
between processing of an individual part and of the main document, e.g.:
%
\begin{center}
\begin{tabular}{l}
|\ifchilddocmanual|\\
|\input{\childdocname}|\\
|\||else|\\
\textit{document body with }|\input{|\textit{part}|}|\\
|\||fi|
\end{tabular}
\end{center}
%
The conditional |\ifchilddocmanual| is true whenever
a part to be included by |\input| is being compiled,
and the name of the part is stored in |\childdocname|.

%%%%%%%%%%%%%%%%%%%%%%%%%%%%%%%%%%%%%%%%
\DescribeMacro{\childdocby}
Each part to be included by |\input| should start with:
%
\begin{center}
\begin{tabular}{l}
|\input{childdoc.def}|\\
|\childdocby{|\textit{main}|}|\\
\end{tabular}
\end{center}
%
The directive |\childdocby| is similar to |\childdocof|
described in \secref{sec:include},
but the subsequent selection of content must be done manually.
To that end, both |\ifchilddoc| and |\ifchilddocmanual|
will be true upon processing of a part,
and the name of the part is stored in |\childdocname|.
Note that |\jobname| will be set to the filename of the current part
so that each part receives an individual |.aux| file
that does not interfere with the |.aux| file(s) of the main document.
This behaviour can be altered by the alternative form
|\childdocby[*]{|\textit{main}|}| (with a non-empty optional argument)
which uses the |.aux| file of the main document
by setting |\jobname| to \textit{main}.

%%%%%%%%%%%%%%%%%%%%%%%%%%%%%%%%%%%%%%%%%%%%%%%%%%%%%%%%%%%%%%%%%%%%%%%%%%%%%%%%
\subsection{Driver Development}
\label{sec:driver}

The \textsf{childdoc} mechanism can also be use for the development
of definition files such as \LaTeX{} styles or classes.
This case differs from the above setup with multiple parts
included by |\include| in that no |\includeonly| should be invoked.
This can be achieved by starting the include file
(before |\ProvidesPackage|) with:
%
\begin{center}
\begin{tabular}{l}
|\input{childdoc.def}|\\
|\childdocforward{|\textit{main}|}|\\
\end{tabular}
\end{center}
%
or alternatively with:
%
\begin{center}
\begin{tabular}{l}
|\input{childdoc.def}|\\
|\childdocby{|\textit{main}|}|\\
\end{tabular}
\end{center}
%
Both forms have slightly different effects as described above.
The main file is prepared as usual, see \secref{sec:include}.

%%%%%%%%%%%%%%%%%%%%%%%%%%%%%%%%%%%%%%%%%%%%%%%%%%%%%%%%%%%%%%%%%%%%%%%%%%%%%%%%
\subsection{Legacy Detection}
\label{sec:detection}

The directive |\childdocmain| in the main file can detect
whether the complete document or merely a child is to be compiled
even without using the directive |\childdocof|.
This method is deprecated because it is less robust
and there is no compelling reason to use it;
it is merely provided for backward compatibility
and it may be removed in future versions.

If the detection mechanism is to be used,
it is mandatory to correctly specify
the filename of the main file as the argument of |\childdocmain|:
%
\begin{center}
\begin{tabular}{l}
|\input{childdoc.def}|\\
|\childdocmain{|\textit{main}|}|\\
\end{tabular}
\end{center}
%
If |\jobname| does not match the argument \textit{main} of |\childdocmain|,
it is assumed that |\jobname| points to the child file to be compiled.
When using |\childdocmain| with the main file specified as argument,
it suffices to start a child file
with just |\input{|\textit{main}|}|
without loading of the package and using |\childdocof|.
If instead all processing is done
with the appropriate \textsf{childdoc} directives,
the argument of \textit{main} of |\childdocmain| can be empty.

An alternative version of the command line processing described
in \secref{sec:commandline} using the detection mechanism reads:
%
\begin{center}
|... -jobname "|\textit{target}|" "|[\textit{flags}]%
[|\def\jobname{|\textit{dest}|}|]|\input{|\textit{main}|}"|
\end{center}

%%%%%%%%%%%%%%%%%%%%%%%%%%%%%%%%%%%%%%%%%%%%%%%%%%%%%%%%%%%%%%%%%%%%%%%%%%%%%%%%
\subsection{Manual Code}
\label{sec:manual}

In case one cannot be certain whether the definitions file |childdoc.def|
is installed on the target \TeX{} distribution
and one prefers not to ship it,
it is conceivable to paste a few relevant commands into the sources.

To that end, drop all statements |\input{childdoc.def}|
and perform the replacements as outlined below.
Instead of |\childdocmain{|\textit{main}|}| add the following code
to the top of the main file:
%
\begin{center}
\begin{tabular}{l}
|\||ifdefined\childdocname\endinput\||fi\newif\ifchilddoc|\\
|\edef\childdocname{\scantokens\expandafter{\jobname\noexpand}}|\\
|\def\childdocmain{|\textit{main}|}\||ifx\childdocmain\childdocname\||else|\\
|\childdoctrue\includeonly{\childdocname}\let\jobname\childdocmain\||fi|\\
\end{tabular}
\end{center}
%
Instead of |\childdocof{|\textit{main}|}| just include the main file
at the top of each child file:
%
\begin{center}
|\input{|\textit{main}|}|
\end{center}
%
A simple redirection |\childdocforward{|\textit{dest}|}| is achieved by:
%
\begin{center}
|\def\jobname{|\textit{dest}|}\input{\jobname}|
\end{center}
%
The redirection with prefix
|\childdocforwardprefix[|\textit{prefix}|]{|\textit{dest}|}|
is accomplished by:
%
\begin{center}
\begin{tabular}{l}
|{\edef\jobname{\scantokens\expandafter{\jobname\noexpand}}|\\
|\def\redirectjob |\textit{prefix}|#1~~~{\gdef\jobname{|\textit{dest}|#1}}|\\
|\expandafter\redirectjob\jobname~~~}\input{\jobname}|
\end{tabular}
\end{center}

In an alternative approach,
child documents can be compiled by a specific command line
without additional code or specific definitions:
%
\begin{center}
|... -jobname "|\textit{target}|" "|[\textit{flags}]%
|\includeonly{|\textit{dest}|}\input{|\textit{main}|}"|
\end{center}
%

%%%%%%%%%%%%%%%%%%%%%%%%%%%%%%%%%%%%%%%%%%%%%%%%%%%%%%%%%%%%%%%%%%%%%%%%%%%%%%%%
%%%%%%%%%%%%%%%%%%%%%%%%%%%%%%%%%%%%%%%%%%%%%%%%%%%%%%%%%%%%%%%%%%%%%%%%%%%%%%%%
\section{Information}

%%%%%%%%%%%%%%%%%%%%%%%%%%%%%%%%%%%%%%%%%%%%%%%%%%%%%%%%%%%%%%%%%%%%%%%%%%%%%%%%
\subsection{Copyright}

Copyright \copyright{} 2017--2018 Niklas Beisert

This work may be distributed and/or modified under the
conditions of the \LaTeX{} Project Public License, either version 1.3
of this license or (at your option) any later version.
The latest version of this license is in
  \url{http://www.latex-project.org/lppl.txt}
and version 1.3 or later is part of all distributions of \LaTeX{}
version 2005/12/01 or later.

This work has the LPPL maintenance status `maintained'.

The Current Maintainer of this work is Niklas Beisert.

This work consists of the files |README.txt|, |childdoc.ins| and |childdoc.dtx|
as well as the derived files |childdoc.def|, |cdocsamp.tex|
with |cdocsch1.tex|, |cdocsch2.tex|, |cdocspt3.tex|, |cdocspt4.tex|,
|cdocsdrf.tex|, |cdocsfn1.tex|, |cdocsfn2.tex|
as well as |childdoc.pdf|.

%%%%%%%%%%%%%%%%%%%%%%%%%%%%%%%%%%%%%%%%%%%%%%%%%%%%%%%%%%%%%%%%%%%%%%%%%%%%%%%%
\subsection{Files and Installation}

The package consists of the files:
%
\begin{center}
\begin{tabular}{ll}
    |README.txt|   & readme file \\
    |childdoc.ins| & installation file \\
    |childdoc.dtx| & source file \\
    |childdoc.def| & definition file \\
    |cdocsamp.tex| & sample main file \\
    |cdocsch1.tex| & sample include file \\
    |cdocsch2.tex| & sample include file \\
    |cdocspt3.tex| & sample part file \\
    |cdocspt4.tex| & sample part file \\
    |cdocsdrf.tex| & sample redirection file \\
    |cdocsfn1.tex| & sample redirection file \\
    |cdocsfn2.tex| & sample redirection file \\
    |childdoc.pdf| & manual
\end{tabular}
\end{center}
%
The distribution consists of the files
|README.txt|, |childdoc.ins| and |childdoc.dtx|.
%
\begin{itemize}
\item
Run (pdf)\LaTeX{} on |childdoc.dtx|
to compile the manual |childdoc.pdf| (this file).
\item
Run \LaTeX{} on |childdoc.ins| to create the definitions file |childdoc.def|
and the sample |cdocsamp.tex| with include files
|cdocsch1.tex|, |cdocsch2.tex|, |cdocspt3.tex|, |cdocspt4.tex|,
|cdocsdrf.tex|, |cdocsfn1.tex|, |cdocsfn2.tex|.
Then copy the file |childdoc.def| to an appropriate directory of your \LaTeX{}
distribution, e.g.\ \textit{texmf-root}|/tex/latex/childdoc|.
\end{itemize}

%%%%%%%%%%%%%%%%%%%%%%%%%%%%%%%%%%%%%%%%%%%%%%%%%%%%%%%%%%%%%%%%%%%%%%%%%%%%%%%%
\subsection{Related CTAN Packages}

There are several other packages which offer a similar functionality:
%
\begin{itemize}
\item
The packages
\href{http://ctan.org/pkg/docmute}{\textsf{docmute}},
\href{http://ctan.org/pkg/includex}{\textsf{includex}} and
\href{http://ctan.org/pkg/standalone}{\textsf{standalone}}
provide commands to include only the document body of
a child file thus allowing both files to be compiled individually.
\item
The packages \href{http://ctan.org/pkg/subdocs}{\textsf{subdocs}}
and \href{http://ctan.org/pkg/subfiles}{\textsf{subfiles}}
provide structures in which the main and child documents can be
encapsulated and allowing them to be compiled individually.
The inclusion mechanism is different from the conventional |\include|.
\item
The package \href{http://ctan.org/pkg/combine}{\textsf{combine}}
is an elaborate solution to combine several documents into one.
\end{itemize}
%
See also the CTAN topic \href{http://ctan.org/topic/subdocs}{\textsf{subdocs}}
for further related packages.
The present package differs from the above solutions in that
a document structure constructed with the conventional |\include| mechanism
just needs two extra commands at the top of every file
such that all constituent files can be compiled individually.

%%%%%%%%%%%%%%%%%%%%%%%%%%%%%%%%%%%%%%%%%%%%%%%%%%%%%%%%%%%%%%%%%%%%%%%%%%%%%%%%
%\subsection{Feature Suggestions}
%
%The following is a list of features which may be useful for future
%versions of this package:
%%
%\begin{itemize}
%\item
%\ldots
%\end{itemize}

%%%%%%%%%%%%%%%%%%%%%%%%%%%%%%%%%%%%%%%%%%%%%%%%%%%%%%%%%%%%%%%%%%%%%%%%%%%%%%%%
\subsection{Revision History}

%%%%%%%%%%%%%%%%%%%%%%%%%%%%%%%%%%%%%%%%
\paragraph{v2.0:} 2018/12/30

\begin{itemize}
\item
immediate forward processing
\item
added |\childdocby| mechanism
\item
manual restructured
\end{itemize}

%%%%%%%%%%%%%%%%%%%%%%%%%%%%%%%%%%%%%%%%
\paragraph{v1.6:} 2018/01/17

\begin{itemize}
\item
application for development of include files
\item
corrections to manual
\end{itemize}

%%%%%%%%%%%%%%%%%%%%%%%%%%%%%%%%%%%%%%%%
\paragraph{v1.5:} 2017/05/21

\begin{itemize}
\item
more complete structuring introduced
\item
|\childdocof| introduced
\item
|\childdoc| renamed to |\childdocmain|
\item
|\childredirect| renamed to |\childdocforward| and |\childdocforwardprefix|
and functionality expanded
\end{itemize}

%%%%%%%%%%%%%%%%%%%%%%%%%%%%%%%%%%%%%%%%
\paragraph{v1.0:} 2017/04/27

\begin{itemize}
\item
manual and install package
\item
first version published on CTAN
\end{itemize}

%%%%%%%%%%%%%%%%%%%%%%%%%%%%%%%%%%%%%%%%
\paragraph{v0.6:} 2017/04/26

\begin{itemize}
\item
redirection mechanism added
\end{itemize}

%%%%%%%%%%%%%%%%%%%%%%%%%%%%%%%%%%%%%%%%
\paragraph{v0.5:} 2017/04/26

\begin{itemize}
\item
functionality in definition file
\end{itemize}


%%%%%%%%%%%%%%%%%%%%%%%%%%%%%%%%%%%%%%%%%%%%%%%%%%%%%%%%%%%%%%%%%%%%%%%%%%%%%%%%
%%%%%%%%%%%%%%%%%%%%%%%%%%%%%%%%%%%%%%%%%%%%%%%%%%%%%%%%%%%%%%%%%%%%%%%%%%%%%%%%
%%%%%%%%%%%%%%%%%%%%%%%%%%%%%%%%%%%%%%%%%%%%%%%%%%%%%%%%%%%%%%%%%%%%%%%%%%%%%%%%
\appendix

\settowidth\MacroIndent{\rmfamily\scriptsize 000\ }

 \DocInput{childdoc.dtx}

\end{document}
%</driver>
% \fi
%
% %%%%%%%%%%%%%%%%%%%%%%%%%%%%%%%%%%%%%%%%%%%%%%%%%%%%%%%%%%%%%%%%%%%%%%%%%%%%%%
% %%%%%%%%%%%%%%%%%%%%%%%%%%%%%%%%%%%%%%%%%%%%%%%%%%%%%%%%%%%%%%%%%%%%%%%%%%%%%%
% \section{Sample}
%\iffalse
%<*samplemain>
%\fi
%
% The following presents a sample document
% with two chapters, two parts, a title page,
% a compile flag as well as three forwarding files to set the flag.
% It consists of eight |.tex| files:
% \begin{center}
% \begin{tabular}{ll}
% |cdocsamp.tex|&main file\\
% |cdocsch1.tex|&include file for chapter 1\\
% |cdocsch2.tex|&include file for chapter 2\\
% |cdocspt3.tex|&include file for part 3\\
% |cdocspt4.tex|&include file for part 4\\
% |cdocsdrf.tex|&forwarding file for main file in draft mode\\
% |cdocsfi1.tex|&forwarding file for final version of chapter 1\\
% |cdocsfi2.tex|&forwarding file for final version of chapter 2\\
% \end{tabular}
% \end{center}
% Each of the eight files can be compiled directly by the \LaTeX{} compiler.
%
% %%%%%%%%%%%%%%%%%%%%%%%%%%%%%%%%%%%%%%
% \paragraph{Main File.}
%
% The main file is called |cdocsamp.tex|.
%
% Load the \textsf{childdoc} definitions and
% declare the filename for the main document:
%    \begin{macrocode}
\input{childdoc.def}
\childdocmain{}
%    \end{macrocode}

% Optional override for |\version| flag:
%    \begin{macrocode}
%%\ifchilddoc\else\providecommand{\version}{draft}\fi
%    \end{macrocode}

% Define the default values for the |\version| flag
% (|final| for the main file and |draft| for childs):
%    \begin{macrocode}
\ifchilddoc
\providecommand{\version}{draft}
\else
\providecommand{\version}{final}
\fi
%    \end{macrocode}

% Load the standard document class:
%    \begin{macrocode}
\documentclass[12pt]{article}
%    \end{macrocode}

% Start the document body:
%    \begin{macrocode}
\begin{document}
%    \end{macrocode}

% Declare a title page.
% Print title, part of document being processed and version flag:
%    \begin{macrocode}
\addtocounter{page}{-1}
\begin{center}
{\LARGE\bfseries{}childdoc example\par}
\vspace{1cm}
\ifchilddoc
\ifchilddocmanual part\else chapter\fi:
`\childdocname' of `\childdocjob'\par
\else
main document: `\childdocjob'\par
\fi
version: \version\par
\end{center}
\newpage
%    \end{macrocode}

% Manually include selected file,
% otherwise process as usual:
%    \begin{macrocode}
\ifchilddocmanual
\section*{part `\childdocname'}
\input{\childdocname}
\else
%    \end{macrocode}

% Include the two chapters:
%    \begin{macrocode}
\include{cdocsch1}
\include{cdocsch2}
%    \end{macrocode}

% Include the two parts unless only chapters should be displayed:
%    \begin{macrocode}
\ifchilddoc\else
\section{part three}
\input{cdocspt3}
\section{part four}
\input{cdocspt4}
\fi
%    \end{macrocode}

% Process as usual until here:
%    \begin{macrocode}
\fi
%    \end{macrocode}

% End of document body:
%    \begin{macrocode}
\end{document}
%    \end{macrocode}
%\iffalse
%</samplemain>
%\fi
%
% %%%%%%%%%%%%%%%%%%%%%%%%%%%%%%%%%%%%%%
% \paragraph{Chapter Include Files.}
%
% The include files are called |cdocsch1.tex| and |cdocsch2.tex|.
%
%\iffalse
%<*samplechap1|samplechap2>
%\fi

% Optional override for |\version| flag:
%    \begin{macrocode}
%%\providecommand{\version}{final}
%    \end{macrocode}

% Include the main document:
%    \begin{macrocode}
\input{childdoc.def}
\childdocof{cdocsamp}
%    \end{macrocode}

%\iffalse
%</samplechap1|samplechap2>
%\fi
%
%\iffalse
%<*samplechap1>
%\fi
% Some text for chapter 1:
%    \begin{macrocode}
\section{one}
some text in chapter one
%    \end{macrocode}

%\iffalse
%</samplechap1>
%\fi
% Some text for chapter 2:
%\iffalse
%<*samplechap2>
%\fi
%    \begin{macrocode}
\section{two}
more text in chapter two
%    \end{macrocode}

%\iffalse
%</samplechap2>
%\fi
%
% %%%%%%%%%%%%%%%%%%%%%%%%%%%%%%%%%%%%%%
% \paragraph{Part Include Files.}
%
% The include files are called |cdocspt3.tex| and |cdocspt4.tex|.
%
%\iffalse
%<*samplepart3|samplepart4>
%\fi

% Optional override for |\version| flag:
%    \begin{macrocode}
%%\providecommand{\version}{final}
%    \end{macrocode}

% Include the main document:
%    \begin{macrocode}
\input{childdoc.def}
\childdocby{cdocsamp}
%    \end{macrocode}

%\iffalse
%</samplepart3|samplepart4>
%\fi
%
%\iffalse
%<*samplepart3>
%\fi
% Some text for part 3:
%    \begin{macrocode}
some text in part three
%    \end{macrocode}

%\iffalse
%</samplepart3>
%\fi
% Some text for part 4:
%\iffalse
%<*samplepart4>
%\fi
%    \begin{macrocode}
more text in part four
%    \end{macrocode}

%\iffalse
%</samplepart4>
%\fi
%
% %%%%%%%%%%%%%%%%%%%%%%%%%%%%%%%%%%%%%%
% \paragraph{Forwarding for a Complete Draft.}
%
% The following forwarding file |cdocsdrf.tex|
% compiles the main document in draft mode:
%\iffalse
%<*sampledraft>
%\fi
%    \begin{macrocode}
\def\version{draft}
\input{childdoc.def}
\childdocforward{cdocsamp}
%    \end{macrocode}

%\iffalse
%</sampledraft>
%\fi
%
% %%%%%%%%%%%%%%%%%%%%%%%%%%%%%%%%%%%%%%
% \paragraph{Forwarding for Final Version of the Chapters.}
%
% The following forwarding files |cdocsfn1.tex| and |cdocsfn2.tex|
% (with identical content)
% compile the final versions of the child documents
% |cdocsch1.tex| and |cdocsch2.tex|, respectively:
%\iffalse
%<*samplefinal>
%\fi
%    \begin{macrocode}
\def\version{final}
\input{childdoc.def}
\childdocforwardprefix[cdocsamp]{cdocsfn}{cdocsch}
%    \end{macrocode}

%\iffalse
%</samplefinal>
%\fi
%
% %%%%%%%%%%%%%%%%%%%%%%%%%%%%%%%%%%%%%%
% \paragraph{Command Line Processing.}
%
% The following three command lines generate the output files
% |cdocscld|, |cdocscl1| and |cdocscl2|
% which should be identical to
% |cdocsdrf|, |cdocsch1| and |cdocsfn2|, respectively:
% \begin{center}
% \begin{tabular}{l}
% |latex -jobname cdocscld \|\\
% |  "\def\version{draft}\input{childdoc.def}\childdocforward{cdocsamp}"|\\
% |latex -jobname cdocscl1 \|\\
% |  "\input{childdoc.def}\childdocforward[cdocsamp]{cdocsch1}"|\\
% |latex -jobname cdocscl2 \|\\
% |  "\def\version{final}\input{childdoc.def}\childdocforward{cdocsch2}"|
% \end{tabular}
% \end{center}
% Note that the trailing backslash on each first line
% merely continues the input to the second line
% (for convenient cut ant paste).
% Furthermore, the command |latex| can be replaced by any
% of its alternative versions such as |pdflatex|.
%
% %%%%%%%%%%%%%%%%%%%%%%%%%%%%%%%%%%%%%%%%%%%%%%%%%%%%%%%%%%%%%%%%%%%%%%%%%%%%%%
% %%%%%%%%%%%%%%%%%%%%%%%%%%%%%%%%%%%%%%%%%%%%%%%%%%%%%%%%%%%%%%%%%%%%%%%%%%%%%%
% \section{Implementation}
%\iffalse
%<*package>
%\fi
%
% This section describes the definitions file |childdoc.def|.

% The definitions cannot be loaded using |\usepackage| or |\RequirePackage|
% which has a mechanism to prevent loading a style file more than once.
% When loading the definitions by means of |\input|
% multiple instances have to be prevented manually:
%\iffalse
%This code needs to be before the `\ProvidesFile' directive
%which is defined at the beginning of this file.
%Therefore it is also placed there and commented out here.
%</package>
%<*discard>
%\fi
%    \begin{macrocode}
\ifdefined\childdocmain\endinput\fi
%    \end{macrocode}
%\iffalse
%</discard>
%<*package>
%\fi
%
% \macro{\ifchilddoc}
% \macro{\ifchilddocmanual}
% The conditional |\ifchilddoc| tells whether a
% child (true) or main (false) document is being compiled.
% The conditional |\ifchilddocmanual| tells whether
% the |\includeonly| mechanism is used (false) or
% the selection of child files must be performed manually (true).
% The definitions initialise to false:
%    \begin{macrocode}
\newif\ifchilddoc
\newif\ifchilddocmanual
%    \end{macrocode}

% \macro{\childdocname}
% \macro{\childdocjob}
% The macro |\childdocname| stores the name of the main document
% to be compiled. The macro |\childdocjob| stores the name of
% the document on which the \LaTeX{} compiler was originally invoked.
% The content of |\jobname| cannot be compared
% to filenames specified in the source due to different catcodes.
% The following code rescans |\jobname|, stores the result
% in |\childdocname| and saves a copy in |\childdocjob|:
%    \begin{macrocode}
\edef\childdocname{\scantokens\expandafter{\jobname\noexpand}}
\let\childdocjob\childdocname
%    \end{macrocode}

% \macro{\childdocdisable}
% The macro |\childdocdisable| prevents the main file
% from being processed more than once.
% At this stage, the main document command |\childdocmain|
% is assumed to be called once again where it should do nothing.
% Any subsequent call to it should prevent
% a secondary processing of the main document
% It overwrites the forwarding commands
% |\childdocof| and |\childdocforward|
% with empty macros to prevent further inclusions of the main document:
%    \begin{macrocode}
\newcommand{\childdocdisable}
{
  \renewcommand{\childdocmain}[1]{\renewcommand{\childdocmain}[1]{\endinput}}
  \renewcommand{\childdocof}[1]{}
  \renewcommand{\childdocby}[2][]{}
  \renewcommand{\childdocforward}[2][]{}
  \renewcommand{\childdocdisable}{}
}
%    \end{macrocode}

% \macro{\childdocmain}
% The macro |\childdocmain| is to be called at the top of the main file
% with nothing or the main filename (without extension) as argument.
% First, it breaks loops.
% If the argument is not empty and does not match |\childdocname|
% (which is set by the first inclusion of |childdoc.def|),
% |\ifchilddoc| is set to true, |\includeonly| is applied to the child file
% and |\jobname| is set to the main file
% (for proper handling of |.aux| files):
%    \begin{macrocode}
\newcommand{\childdocmain}[1]
{
  \childdocdisable\childdocmain{}
  \if?#1?\else
    \begingroup
      \def\childdoctmp{#1}
      \ifx\childdoctmp\childdocname
        \def\childdoctmp{}
      \else
        \def\childdoctmp
        {
          \childdoctrue
          \includeonly{\childdocname}
          \def\childdocjob{#1}
          \def\jobname{#1}
        }
      \fi
      \expandafter
    \endgroup
    \childdoctmp
  \fi
}
%    \end{macrocode}

% \macro{\childdocof}
% The command |\childdocof| redirects
% compilation to the main file |#1|.
%    \begin{macrocode}
\newcommand{\childdocof}[1]
{
  \childdocdisable
  \childdoctrue
  \includeonly{\childdocname}
  \def\jobname{#1}
  \def\childdocjob{#1}
  \input{#1}
}
%    \end{macrocode}

% \macro{\childdocby}
% The command |\childdocby| ....
%    \begin{macrocode}
\newcommand{\childdocby}[2][]
{
  \childdocdisable
  \childdoctrue
  \childdocmanualtrue
  \if?#1?\else
    \def\jobname{#2}
  \fi
  \def\childdocjob{#2}
  \input{#2}
  \endinput
}
%    \end{macrocode}

% \macro{\childdocforward}
% The command |\childdocforward| redirects
% compilation to the main file or
% (if the optional argument is given) a child file.
% Parameters are set as if the main file
% or a child file starting with |\childdocof| was compiled.
% Then compilation is handed over to the main file:
%    \begin{macrocode}
\newcommand{\childdocforward}[2][]
{
  \begingroup
    \if?#1?
      \def\childdoctmp
      {
        \def\childdocname{#2}
        \def\childdocjob{#2}
        \def\jobname{#2}
        \input{#2}
        \endinput
      }
    \else
      \def\childdoctmp
      {
        \childdocdisable
        \def\childdocname{#2}
        \childdoctrue
        \includeonly{#2}
        \def\childdocjob{#1}
        \def\jobname{#1}
        \input{#1}
        \endinput
      }
    \fi
    \expandafter
  \endgroup
  \childdoctmp
}
%    \end{macrocode}

% \macro{\childdocforwardprefix}
% The command |\childdocforwardprefix| redirects
% compilation to the main or a child file by means of a pattern.
% The prefix |#1| in the current filename is replaced by |#2|
% and the suffix of the current filename is kept
% (it is assumed that the filename does not contain the substring `|~~~|'
% which is used as a delimiter).
% Compilation is handed over to the new file by |\childdocforward|:
%    \begin{macrocode}
\newcommand{\childdocforwardprefix}[3][]
{
  \begingroup
    \def\childdocextract #2##1~~~{\def\childdoctmp{\childdocforward[#1]{#3##1}}}
    \expandafter\childdocextract\childdocname~~~
    \expandafter
  \endgroup
  \childdoctmp
}
%    \end{macrocode}

% \macro{\childdoc}
% The deprecated macro |\childdoc| is a legacy version of |\childdocmain|:
%    \begin{macrocode}
\newcommand{\childdoc}{\childdocmain}
%    \end{macrocode}

% \macro{\childdocredirect}
% The deprecated macro |\childdocredirect| is a legacy version
% of |\childdocforward| and |\childdocforwardprefix|:
%    \begin{macrocode}
\newcommand{\childdocredirect}[2][]
{
  \begingroup
    \if?#1?
      \def\childdoctmp{\childdocforward{#2}}
    \else
      \def\childdoctmp{\childdocforwardprefix{#1}{#2}}
    \fi
    \expandafter
  \endgroup
  \childdoctmp
}
%    \end{macrocode}

%\iffalse
%</package>
%\fi
%
\endinput

\childdocforward{cdocsamp}
%    \end{macrocode}

%\iffalse
%</sampledraft>
%\fi
%
% %%%%%%%%%%%%%%%%%%%%%%%%%%%%%%%%%%%%%%
% \paragraph{Forwarding for Final Version of the Chapters.}
%
% The following forwarding files |cdocsfn1.tex| and |cdocsfn2.tex|
% (with identical content)
% compile the final versions of the child documents
% |cdocsch1.tex| and |cdocsch2.tex|, respectively:
%\iffalse
%<*samplefinal>
%\fi
%    \begin{macrocode}
\def\version{final}
% \iffalse
%
% childdoc.dtx Copyright (C) 2017-2018 Niklas Beisert
%
% This work may be distributed and/or modified under the
% conditions of the LaTeX Project Public License, either version 1.3
% of this license or (at your option) any later version.
% The latest version of this license is in
%   http://www.latex-project.org/lppl.txt
% and version 1.3 or later is part of all distributions of LaTeX
% version 2005/12/01 or later.
%
% This work has the LPPL maintenance status `maintained'.
%
% The Current Maintainer of this work is Niklas Beisert.
%
% This work consists of the files childdoc.dtx and childdoc.ins
% and the derived files childdoc.def and cdocsamp.tex with
% cdocsch1.tex, cdocsch2.tex, cdocsdrf.tex, cdocsfn1.tex, cdocsfn2.tex.
%
%<package>\ifdefined\childdocmain\endinput\fi
%<package>\ProvidesFile{childdoc.def}[2018/12/30 v2.0 child document driver]
%<samplemain>\ProvidesFile{cdocsamp.tex}[2018/12/30 v2.0 sample for childdoc]
%<*driver>
%\ProvidesFile{childdoc.drv}[2018/12/30 v2.0 childdoc reference manual file]
\PassOptionsToClass{10pt,a4paper}{article}
\documentclass{ltxdoc}

\usepackage[margin=35mm]{geometry}
\usepackage{hyperref}
\usepackage{hyperxmp}
\usepackage[usenames]{color}

\hypersetup{colorlinks=true}
\hypersetup{pdfstartview=FitH}
\hypersetup{pdfpagemode=UseNone}
\hypersetup{pdfsource={}}
\hypersetup{pdflang={en-UK}}
\hypersetup{pdfcopyright={Copyright 2017-2018 Niklas Beisert.
  This work may be distributed and/or modified under the
  conditions of the LaTeX Project Public License, either version 1.3
  of this license or (at your option) any later version.}}
\hypersetup{pdflicenseurl={http://www.latex-project.org/lppl.txt}}
\hypersetup{pdfcontactaddress={ETH Zurich, ITP, HIT K,
  Wolfgang-Pauli-Strasse 27}}
\hypersetup{pdfcontactpostcode={8093}}
\hypersetup{pdfcontactcity={Zurich}}
\hypersetup{pdfcontactcountry={Switzerland}}
\hypersetup{pdfcontactemail={nbeisert@itp.phys.ethz.ch}}
\hypersetup{pdfcontacturl={http://people.phys.ethz.ch/\xmptilde nbeisert/}}

\newcommand{\secref}[1]{\hyperref[#1]{section \ref*{#1}}}

\parskip1ex
\parindent0pt
\let\olditemize\itemize
\def\itemize{\olditemize\parskip0pt}

\begin{document}

\title{The \textsf{childdoc} Package}
\hypersetup{pdftitle={The childdoc Package}}
\author{Niklas Beisert\\[2ex]
  Institut f\"ur Theoretische Physik\\
  Eidgen\"ossische Technische Hochschule Z\"urich\\
  Wolfgang-Pauli-Strasse 27, 8093 Z\"urich, Switzerland\\[1ex]
  \href{mailto:nbeisert@itp.phys.ethz.ch}
  {\texttt{nbeisert@itp.phys.ethz.ch}}}
\hypersetup{pdfauthor={Niklas Beisert}}
\hypersetup{pdfsubject={Manual for the LaTeX2e Package childdoc}}
\date{30 December 2018, \textsf{v2.0}}
\maketitle

\begin{abstract}\noindent
\textsf{childdoc} is a \LaTeXe{} package
that enables the direct compilation
of document sections included by |\include|
to individual files.
\end{abstract}

\begingroup
\parskip0ex
\tableofcontents
\endgroup

%%%%%%%%%%%%%%%%%%%%%%%%%%%%%%%%%%%%%%%%%%%%%%%%%%%%%%%%%%%%%%%%%%%%%%%%%%%%%%%%
%%%%%%%%%%%%%%%%%%%%%%%%%%%%%%%%%%%%%%%%%%%%%%%%%%%%%%%%%%%%%%%%%%%%%%%%%%%%%%%%
\section{Introduction}

\LaTeX{} provides a mechanism to structure a large document (such as a book)
into a main file and several child files (containing the chapters)
using the |\include| command.
This mechanism is beneficial for documents
which span hundreds of pages in order to
make the source file(s) more manageable.
Moreover, compilation can be restricted to
selected child files by means of the |\includeonly| command.
The latter feature can be used to reduce the compilation time while editing
(this was significantly more useful in the earlier days of \LaTeX{})
or to generate a smaller document which is easier to navigate.
Another application of |\includeonly| is to generate
documents consisting of selected parts of the complete document.

However, there are a few drawbacks of the plain |\include| mechanism:
\begin{itemize}
\item
The child files cannot be compiled on their own,
they can only be compiled via the main file.
A naive editing environment
(such as a text editor with an option
to have the current file processed by \LaTeX)
may require one to switch to the main file before compiling;
attempting to compile the child file produces errors.
\item
The main file must be modified (each time)
to adjust the |\includeonly| command
to the present needs. This easily leaves the main file in a messy state.
\item
The generated document will always carry the filename
of the main document. This is inconvenient if
several child files are to be compiled and
to be kept for distribution.
\end{itemize}

The present package provides a simple interface
to make child files individually compilable by \LaTeX{}.
Compiling a child file then has the same effect as compiling
the main file with an |\includeonly| command
to select the appropriate child.
Moreover the generated document will carry the name of the child
rather than the main file.
This resolves all three above issues.

This feature is meant to make the editing of books,
thesis documents and lecture notes somewhat more convenient.
However, the package can also be used efficiently for
composing a series of documents (such as exercise sheets)
which are typically distributed individually.
It then assists the author in generating the individual documents
(potentially in different versions)
as well as a document containing the collected series.
Another application is in developing style files
or other kinds of included material
where compilation of the style file could redirect
to a sample or test file.

%%%%%%%%%%%%%%%%%%%%%%%%%%%%%%%%%%%%%%%%%%%%%%%%%%%%%%%%%%%%%%%%%%%%%%%%%%%%%%%%
%%%%%%%%%%%%%%%%%%%%%%%%%%%%%%%%%%%%%%%%%%%%%%%%%%%%%%%%%%%%%%%%%%%%%%%%%%%%%%%%
\section{Usage}

First of all, the package \textsf{childdoc} is \emph{not} a standard
\LaTeXe{} |.sty| style file! Therefore it needs to be invoked in
a non-standard way.

%%%%%%%%%%%%%%%%%%%%%%%%%%%%%%%%%%%%%%%%%%%%%%%%%%%%%%%%%%%%%%%%%%%%%%%%%%%%%%%%
\subsection{Included Files}
\label{sec:include}

%%%%%%%%%%%%%%%%%%%%%%%%%%%%%%%%%%%%%%%%
\DescribeMacro{\childdocmain}
To use the package, add the commands
\begin{center}
\begin{tabular}{l}
|\input{childdoc.def}|\\
|\childdocmain{}|\\
\end{tabular}
\end{center}
at the very top of the main \LaTeX{} file,
in particular \emph{before} the |\documentclass| statement!
The argument of |\childdocmain| should be left empty
(but it must be present).

%%%%%%%%%%%%%%%%%%%%%%%%%%%%%%%%%%%%%%%%
\DescribeMacro{\childdocof}
Furthermore, add the commands
\begin{center}
\begin{tabular}{l}
|\input{childdoc.def}|\\
|\childdocof{|\textit{main}|}|\\
\end{tabular}
\end{center}
at the top of every child file \textit{child}
which is included by |\include{|\textit{child}|}|
from within the main file
(or at least for those files to be compiled individually).
The argument \textit{main} must be the filename of the main file.

There are a couple of
considerations in setting up the main and child documents:

%%%%%%%%%%%%%%%%%%%%%%%%%%%%%%%%%%%%%%%%
\paragraph{Restrictions.}

Please note the following restrictions:
\begin{itemize}
\item
|\childdocmain| must be called with one argument \textit{main}
to ensure compatibility with earlier version of the package.
It must either be empty (|\childdocmain{}|)
or precisely match the filename of the main file in which it is specified.
See \secref{sec:detection} for further information.
\item
The filename \textit{main} must be specified without the |.tex| extension.
\item
The filename \textit{main} is case sensitive
(even in case-insensitive file systems)
due to internal string comparison.
\item
The argument \textit{main} should be fully expanded, it cannot be a macro.
\item
Subdirectories and special characters should be avoided in filenames.
\item
The command |\childdocmain{|\textit{main}|}| must be followed by a whitespace.
It should not be followed immediately by another command
or by a comment mark `|%|'.
This is because the \TeX{} parser reads the token immediately following
the argument of |\childdocmain| and puts it
at the beginning of every child section;
however, a white\-space is ignored.
\end{itemize}

%%%%%%%%%%%%%%%%%%%%%%%%%%%%%%%%%%%%%%%%
\paragraph{Content of Main File.}

It is advisable to place all content in the child files included by |\include|.
Any output contained in the main file will appear in all child documents
unless suppressed manually;
it cannot be suppressed automatically by the |\includeonly| directive
and thus should normally be avoided.
A method to include some content in the main file
by means of conditional processing is described in \secref{sec:conditional}.

%%%%%%%%%%%%%%%%%%%%%%%%%%%%%%%%%%%%%%%%
\paragraph{Page Numbering.}

When only a part of the document is compiled,
the appropriate numbering of pages
(as well as other status parameters)
is determined from the |.aux| files.
The latter contain information from previous passes.
However this information needs to propagate through
all intermediate child documents.
Therefore the page numbering in child documents may well
be inconsistent until the complete document is compiled at least once.

A useful (if unconventional) way to always ensure a consistent
page numbering is to restart the numbering in each child document
and denote the pages by `\textit{child}|.|\textit{page}'
where \textit{child} represents the chapter/section number of the child file.
This can be achieved by the command
|\numberwithin{page}{|\textit{child}|}|
of the \textsf{amsmath} package
where \textit{child} can be |chapter| or |section|
depending on the chosen structuring.
Alternatively, one can modify the macro |\thepage| appropriately
and reset the counter |page| at the start of each child file.

%%%%%%%%%%%%%%%%%%%%%%%%%%%%%%%%%%%%%%%%%%%%%%%%%%%%%%%%%%%%%%%%%%%%%%%%%%%%%%%%
\subsection{Conditional Processing}
\label{sec:conditional}

The package provides a mechanism to compile different versions
of a document. To customise the versions further some conditional processing
can come in handy to distinguish which version is being compiled.
The package provides two macros to describe the compilation context:

%%%%%%%%%%%%%%%%%%%%%%%%%%%%%%%%%%%%%%%%
\DescribeMacro{\ifchilddoc}
The conditional |\ifchilddoc| distinguishes between the compilation of
child documents and the main document:
%
\begin{center}
|\ifchilddoc |\textit{child-code}| |[|\||else |\textit{main-code}]| \||fi|
\end{center}

%%%%%%%%%%%%%%%%%%%%%%%%%%%%%%%%%%%%%%%%
\DescribeMacro{\childdocname}
\DescribeMacro{\childdocjob}
The macro |\childdocname| contains the filename (without extension)
of the main or child file being processed.
Note that |\childdocjob| will always contain the name of the main file.

%%%%%%%%%%%%%%%%%%%%%%%%%%%%%%%%%%%%%%%%
\paragraph{Title Page.}

Conditional processing can be used to include a title or banner page
in the main document when proper precautions are taken.
Importantly, the code in the main file should ensure that the page counter
(as well as other status parameters which are stored in the |.aux| files)
takes the same value after the conditional processing.
Otherwise the page numbers may take divergent values
depending on which part is compiled.

For example, a title page could be declared by:
%
\begin{center}
\begin{tabular}{l}
|\ifchilddoc\||else|\\
|\addtocounter{page}{-1}|\\
\textit{code for title page}\\
|\newpage|\\
|\||fi|
\end{tabular}
\end{center}
%
A banner page for the child documents can be generated by:
%
\begin{center}
\begin{tabular}{l}
|\ifchilddoc|\\
|\addtocounter{page}{-1}|\\
\textit{code for banner page}\\
|\newpage|\\
|\||fi|
\end{tabular}
\end{center}
%
Here one could write a message such as:
\begin{center}
|This is the part \childdocname{} of \childdocjob{}.|
\end{center}

%%%%%%%%%%%%%%%%%%%%%%%%%%%%%%%%%%%%%%%%%%%%%%%%%%%%%%%%%%%%%%%%%%%%%%%%%%%%%%%%
\subsection{Flags}
\label{sec:flags}

The package makes it easy to generate different versions
of the main or child documents.
To this end compilation flags can be defined
and assigned different default values.
They will be particularly useful in conjunction
with the forwarding mechanism described in \secref{sec:forward}.

For example, it may be useful to have a flag |\version|
which can be set to |draft| or |final|.
The document source will contain some conditional code
depending on the value of |\version|.
Suppose further, the flag should default to |final| for the main file
and to |draft| for child files
which is a natural assignment for editing the document.
This is achieved by placing the following code
in the preamble of the main document
(below the |\childdocmain| directive):
%
\begin{center}
\begin{tabular}{l}
|\ifchilddoc|\\
|\providecommand{\version}{draft}|\\
|\||else|\\
|\providecommand{\version}{final}|\\
|\||fi|
\end{tabular}
\end{center}
%
The definition by |\providecommand| makes sure
that previous definitions are not overwritten.
Further statements |\providecommand{\version}{...}|
can thus be added before the above code to override it.

For the main file, one might add a line
(between |\childdocmain| and the above block)
%
\begin{center}
|%\ifchilddoc\||else\providecommand{\version}{draft}\||fi|
\end{center}
%
which can be uncommented to produce a draft version.
Likewise one can add a line to the very top of a child file
(above the |\childdocof{|\textit{main}|}| directive)
%
\begin{center}
|%\providecommand{\version}{final}|
\end{center}
%
which can be uncommented to produce the final version of this child document.

%%%%%%%%%%%%%%%%%%%%%%%%%%%%%%%%%%%%%%%%%%%%%%%%%%%%%%%%%%%%%%%%%%%%%%%%%%%%%%%%
\subsection{Forwarding}
\label{sec:forward}

Different versions of the main or child documents
using compilation flags as described in \secref{sec:flags}
can be (permanently) stored in different files
for convenient compilation, viewing and distribution.
To this end, the package defines a command
to pass on compilation to a different file:

%%%%%%%%%%%%%%%%%%%%%%%%%%%%%%%%%%%%%%%%
\DescribeMacro{\childdocforward}
The command |\childdocforward| redirects processing to
another source file:
%
\begin{center}
\begin{tabular}{l}
|\input{childdoc.def}|\\
|\childdocforward[|\textit{main}|]{|\textit{dest}|}|\\
\end{tabular}
\end{center}
%
The argument \textit{dest} is the destination file
(without extension).
It should be the main file or one of the child files.
Note that further \textsf{childdoc} directives
such as |\childdocof| and |\childdocforward|
in the indicated file will be processed in this form.
The optional argument \textit{main}
passes on directly to the main file \textit{main}
while pretending to compile the child \textit{dest}.
This form behaves as if \textit{dest}
issues |\childdocof{|\textit{main}|}| right away,
and no further \textsf{childdoc} directives will be processed.

%%%%%%%%%%%%%%%%%%%%%%%%%%%%%%%%%%%%%%%%
\DescribeMacro{\...prefix}
In the alternative form |\childdocforwardprefix|,
%
\begin{center}
\begin{tabular}{l}
|\input{childdoc.def}|\\
|\childdocforwardprefix[|\textit{main}|]{|\textit{prefix}|}{|\textit{dest}|}|
\end{tabular}
\end{center}
%
the destination file is determined by a pattern
depending on the current file:
To make this work, the current file must be called
`{\textit{prefix}\hspace{0.2em}\textit{suffix}}'
with \textit{prefix} matching precisely the argument.
Processing is then passed on to the file
`{\textit{dest}\hspace{0.2em}\textit{suffix}}'.
Surely, the same effect is achieved by
directly specifying the
argument `{\textit{dest}\hspace{0.2em}\textit{suffix}}'
in the first form.
However, that requires to set up a different file
for each child. With the alternative form of the command
all these files can have exactly the same content
which simplifies setting them up and maintaining them.

For example, the following file |draft.tex|
with a compilation flag |\version| as described in \secref{sec:flags}
compiles the main document as a draft:
%
\begin{center}
\begin{tabular}{l}
|\def\version{draft}|\\
|\input{childdoc.def}|\\
|\childdocforward{|\textit{main}|}|
\end{tabular}
\end{center}
%
Likewise, the following files |final|\textit{nn}|.tex|
compile the final version of the child document
|child|\textit{nn}|.tex|:
%
\begin{center}
\begin{tabular}{l}
|\def\version{final}|\\
|\input{childdoc.def}|\\
|\childdocforwardprefix{final}{child}|
\end{tabular}
\end{center}
%

Note that when several versions of a main file and/or of each child file
are to be generated, it may be convenient to set up a |Makefile| or
shell script to automatise the process.

%%%%%%%%%%%%%%%%%%%%%%%%%%%%%%%%%%%%%%%%%%%%%%%%%%%%%%%%%%%%%%%%%%%%%%%%%%%%%%%%
\subsection{Command Line Processing}
\label{sec:commandline}

The effect of redirection files can also be achieved by invoking
the \LaTeX{} compiler with a more elaborate command line.
Most conveniently this should be done as part
of a shell script or a |Makefile|.

When using \textsf{childdoc} in the main file, the following
command lines effectively perform a redirection
(note that depending on the shell being used,
backslashes may have to be doubled: `|\|' $\to$ `|\\|'):
%
\begin{center}
|... -jobname "|\textit{target}|" |\\|"|[\textit{flags}]%
|\input{childdoc.def}\childdocforward[|\textit{main}|]{|\textit{dest}|}"|
\end{center}
%
Here \textit{target} is the name of the output file,
\textit{main} is the name of the main file
and \textit{dest} is the name of the main or child file to be processed
(all filenames without extensions).
The optional argument \textit{main} can be omitted
if \textit{main} matches \textit{dest}.
Optionally, compilation \textit{flags} can be defined via |\def| commands.
This command line makes the \TeX{} engine believe
it is compiling the file \textit{target}
whose content is specified as the latter parameter.
The provided code then forwards the processing to
\textit{main} or \textit{dest} as described in \secref{sec:forward}.

%%%%%%%%%%%%%%%%%%%%%%%%%%%%%%%%%%%%%%%%%%%%%%%%%%%%%%%%%%%%%%%%%%%%%%%%%%%%%%%%
\subsection{Include by Input}
\label{sec:input}

Including child documents by |\include| has some restrictions by design.
Most notably, the content of a child document always occupies
its own set of pages; pages cannot be shared between child documents.
Usually, this behaviour makes perfect sense
because each child document contain an essential part of the document.
However, in some situations it may be desirable to compose
a document from a collection of parts
without having mandatory page breaks between then.
For this case, the package
provides a mechanism to include parts
by |\input| which can also be processed individually.
However, by construction this mechanism
requires manual handling of the content to be output.

%%%%%%%%%%%%%%%%%%%%%%%%%%%%%%%%%%%%%%%%
\DescribeMacro{\ifchilddocmanual}
The main file should be prepared as usual, see \secref{sec:include}.
However, the document body must make a distinction
between processing of an individual part and of the main document, e.g.:
%
\begin{center}
\begin{tabular}{l}
|\ifchilddocmanual|\\
|\input{\childdocname}|\\
|\||else|\\
\textit{document body with }|\input{|\textit{part}|}|\\
|\||fi|
\end{tabular}
\end{center}
%
The conditional |\ifchilddocmanual| is true whenever
a part to be included by |\input| is being compiled,
and the name of the part is stored in |\childdocname|.

%%%%%%%%%%%%%%%%%%%%%%%%%%%%%%%%%%%%%%%%
\DescribeMacro{\childdocby}
Each part to be included by |\input| should start with:
%
\begin{center}
\begin{tabular}{l}
|\input{childdoc.def}|\\
|\childdocby{|\textit{main}|}|\\
\end{tabular}
\end{center}
%
The directive |\childdocby| is similar to |\childdocof|
described in \secref{sec:include},
but the subsequent selection of content must be done manually.
To that end, both |\ifchilddoc| and |\ifchilddocmanual|
will be true upon processing of a part,
and the name of the part is stored in |\childdocname|.
Note that |\jobname| will be set to the filename of the current part
so that each part receives an individual |.aux| file
that does not interfere with the |.aux| file(s) of the main document.
This behaviour can be altered by the alternative form
|\childdocby[*]{|\textit{main}|}| (with a non-empty optional argument)
which uses the |.aux| file of the main document
by setting |\jobname| to \textit{main}.

%%%%%%%%%%%%%%%%%%%%%%%%%%%%%%%%%%%%%%%%%%%%%%%%%%%%%%%%%%%%%%%%%%%%%%%%%%%%%%%%
\subsection{Driver Development}
\label{sec:driver}

The \textsf{childdoc} mechanism can also be use for the development
of definition files such as \LaTeX{} styles or classes.
This case differs from the above setup with multiple parts
included by |\include| in that no |\includeonly| should be invoked.
This can be achieved by starting the include file
(before |\ProvidesPackage|) with:
%
\begin{center}
\begin{tabular}{l}
|\input{childdoc.def}|\\
|\childdocforward{|\textit{main}|}|\\
\end{tabular}
\end{center}
%
or alternatively with:
%
\begin{center}
\begin{tabular}{l}
|\input{childdoc.def}|\\
|\childdocby{|\textit{main}|}|\\
\end{tabular}
\end{center}
%
Both forms have slightly different effects as described above.
The main file is prepared as usual, see \secref{sec:include}.

%%%%%%%%%%%%%%%%%%%%%%%%%%%%%%%%%%%%%%%%%%%%%%%%%%%%%%%%%%%%%%%%%%%%%%%%%%%%%%%%
\subsection{Legacy Detection}
\label{sec:detection}

The directive |\childdocmain| in the main file can detect
whether the complete document or merely a child is to be compiled
even without using the directive |\childdocof|.
This method is deprecated because it is less robust
and there is no compelling reason to use it;
it is merely provided for backward compatibility
and it may be removed in future versions.

If the detection mechanism is to be used,
it is mandatory to correctly specify
the filename of the main file as the argument of |\childdocmain|:
%
\begin{center}
\begin{tabular}{l}
|\input{childdoc.def}|\\
|\childdocmain{|\textit{main}|}|\\
\end{tabular}
\end{center}
%
If |\jobname| does not match the argument \textit{main} of |\childdocmain|,
it is assumed that |\jobname| points to the child file to be compiled.
When using |\childdocmain| with the main file specified as argument,
it suffices to start a child file
with just |\input{|\textit{main}|}|
without loading of the package and using |\childdocof|.
If instead all processing is done
with the appropriate \textsf{childdoc} directives,
the argument of \textit{main} of |\childdocmain| can be empty.

An alternative version of the command line processing described
in \secref{sec:commandline} using the detection mechanism reads:
%
\begin{center}
|... -jobname "|\textit{target}|" "|[\textit{flags}]%
[|\def\jobname{|\textit{dest}|}|]|\input{|\textit{main}|}"|
\end{center}

%%%%%%%%%%%%%%%%%%%%%%%%%%%%%%%%%%%%%%%%%%%%%%%%%%%%%%%%%%%%%%%%%%%%%%%%%%%%%%%%
\subsection{Manual Code}
\label{sec:manual}

In case one cannot be certain whether the definitions file |childdoc.def|
is installed on the target \TeX{} distribution
and one prefers not to ship it,
it is conceivable to paste a few relevant commands into the sources.

To that end, drop all statements |\input{childdoc.def}|
and perform the replacements as outlined below.
Instead of |\childdocmain{|\textit{main}|}| add the following code
to the top of the main file:
%
\begin{center}
\begin{tabular}{l}
|\||ifdefined\childdocname\endinput\||fi\newif\ifchilddoc|\\
|\edef\childdocname{\scantokens\expandafter{\jobname\noexpand}}|\\
|\def\childdocmain{|\textit{main}|}\||ifx\childdocmain\childdocname\||else|\\
|\childdoctrue\includeonly{\childdocname}\let\jobname\childdocmain\||fi|\\
\end{tabular}
\end{center}
%
Instead of |\childdocof{|\textit{main}|}| just include the main file
at the top of each child file:
%
\begin{center}
|\input{|\textit{main}|}|
\end{center}
%
A simple redirection |\childdocforward{|\textit{dest}|}| is achieved by:
%
\begin{center}
|\def\jobname{|\textit{dest}|}\input{\jobname}|
\end{center}
%
The redirection with prefix
|\childdocforwardprefix[|\textit{prefix}|]{|\textit{dest}|}|
is accomplished by:
%
\begin{center}
\begin{tabular}{l}
|{\edef\jobname{\scantokens\expandafter{\jobname\noexpand}}|\\
|\def\redirectjob |\textit{prefix}|#1~~~{\gdef\jobname{|\textit{dest}|#1}}|\\
|\expandafter\redirectjob\jobname~~~}\input{\jobname}|
\end{tabular}
\end{center}

In an alternative approach,
child documents can be compiled by a specific command line
without additional code or specific definitions:
%
\begin{center}
|... -jobname "|\textit{target}|" "|[\textit{flags}]%
|\includeonly{|\textit{dest}|}\input{|\textit{main}|}"|
\end{center}
%

%%%%%%%%%%%%%%%%%%%%%%%%%%%%%%%%%%%%%%%%%%%%%%%%%%%%%%%%%%%%%%%%%%%%%%%%%%%%%%%%
%%%%%%%%%%%%%%%%%%%%%%%%%%%%%%%%%%%%%%%%%%%%%%%%%%%%%%%%%%%%%%%%%%%%%%%%%%%%%%%%
\section{Information}

%%%%%%%%%%%%%%%%%%%%%%%%%%%%%%%%%%%%%%%%%%%%%%%%%%%%%%%%%%%%%%%%%%%%%%%%%%%%%%%%
\subsection{Copyright}

Copyright \copyright{} 2017--2018 Niklas Beisert

This work may be distributed and/or modified under the
conditions of the \LaTeX{} Project Public License, either version 1.3
of this license or (at your option) any later version.
The latest version of this license is in
  \url{http://www.latex-project.org/lppl.txt}
and version 1.3 or later is part of all distributions of \LaTeX{}
version 2005/12/01 or later.

This work has the LPPL maintenance status `maintained'.

The Current Maintainer of this work is Niklas Beisert.

This work consists of the files |README.txt|, |childdoc.ins| and |childdoc.dtx|
as well as the derived files |childdoc.def|, |cdocsamp.tex|
with |cdocsch1.tex|, |cdocsch2.tex|, |cdocspt3.tex|, |cdocspt4.tex|,
|cdocsdrf.tex|, |cdocsfn1.tex|, |cdocsfn2.tex|
as well as |childdoc.pdf|.

%%%%%%%%%%%%%%%%%%%%%%%%%%%%%%%%%%%%%%%%%%%%%%%%%%%%%%%%%%%%%%%%%%%%%%%%%%%%%%%%
\subsection{Files and Installation}

The package consists of the files:
%
\begin{center}
\begin{tabular}{ll}
    |README.txt|   & readme file \\
    |childdoc.ins| & installation file \\
    |childdoc.dtx| & source file \\
    |childdoc.def| & definition file \\
    |cdocsamp.tex| & sample main file \\
    |cdocsch1.tex| & sample include file \\
    |cdocsch2.tex| & sample include file \\
    |cdocspt3.tex| & sample part file \\
    |cdocspt4.tex| & sample part file \\
    |cdocsdrf.tex| & sample redirection file \\
    |cdocsfn1.tex| & sample redirection file \\
    |cdocsfn2.tex| & sample redirection file \\
    |childdoc.pdf| & manual
\end{tabular}
\end{center}
%
The distribution consists of the files
|README.txt|, |childdoc.ins| and |childdoc.dtx|.
%
\begin{itemize}
\item
Run (pdf)\LaTeX{} on |childdoc.dtx|
to compile the manual |childdoc.pdf| (this file).
\item
Run \LaTeX{} on |childdoc.ins| to create the definitions file |childdoc.def|
and the sample |cdocsamp.tex| with include files
|cdocsch1.tex|, |cdocsch2.tex|, |cdocspt3.tex|, |cdocspt4.tex|,
|cdocsdrf.tex|, |cdocsfn1.tex|, |cdocsfn2.tex|.
Then copy the file |childdoc.def| to an appropriate directory of your \LaTeX{}
distribution, e.g.\ \textit{texmf-root}|/tex/latex/childdoc|.
\end{itemize}

%%%%%%%%%%%%%%%%%%%%%%%%%%%%%%%%%%%%%%%%%%%%%%%%%%%%%%%%%%%%%%%%%%%%%%%%%%%%%%%%
\subsection{Related CTAN Packages}

There are several other packages which offer a similar functionality:
%
\begin{itemize}
\item
The packages
\href{http://ctan.org/pkg/docmute}{\textsf{docmute}},
\href{http://ctan.org/pkg/includex}{\textsf{includex}} and
\href{http://ctan.org/pkg/standalone}{\textsf{standalone}}
provide commands to include only the document body of
a child file thus allowing both files to be compiled individually.
\item
The packages \href{http://ctan.org/pkg/subdocs}{\textsf{subdocs}}
and \href{http://ctan.org/pkg/subfiles}{\textsf{subfiles}}
provide structures in which the main and child documents can be
encapsulated and allowing them to be compiled individually.
The inclusion mechanism is different from the conventional |\include|.
\item
The package \href{http://ctan.org/pkg/combine}{\textsf{combine}}
is an elaborate solution to combine several documents into one.
\end{itemize}
%
See also the CTAN topic \href{http://ctan.org/topic/subdocs}{\textsf{subdocs}}
for further related packages.
The present package differs from the above solutions in that
a document structure constructed with the conventional |\include| mechanism
just needs two extra commands at the top of every file
such that all constituent files can be compiled individually.

%%%%%%%%%%%%%%%%%%%%%%%%%%%%%%%%%%%%%%%%%%%%%%%%%%%%%%%%%%%%%%%%%%%%%%%%%%%%%%%%
%\subsection{Feature Suggestions}
%
%The following is a list of features which may be useful for future
%versions of this package:
%%
%\begin{itemize}
%\item
%\ldots
%\end{itemize}

%%%%%%%%%%%%%%%%%%%%%%%%%%%%%%%%%%%%%%%%%%%%%%%%%%%%%%%%%%%%%%%%%%%%%%%%%%%%%%%%
\subsection{Revision History}

%%%%%%%%%%%%%%%%%%%%%%%%%%%%%%%%%%%%%%%%
\paragraph{v2.0:} 2018/12/30

\begin{itemize}
\item
immediate forward processing
\item
added |\childdocby| mechanism
\item
manual restructured
\end{itemize}

%%%%%%%%%%%%%%%%%%%%%%%%%%%%%%%%%%%%%%%%
\paragraph{v1.6:} 2018/01/17

\begin{itemize}
\item
application for development of include files
\item
corrections to manual
\end{itemize}

%%%%%%%%%%%%%%%%%%%%%%%%%%%%%%%%%%%%%%%%
\paragraph{v1.5:} 2017/05/21

\begin{itemize}
\item
more complete structuring introduced
\item
|\childdocof| introduced
\item
|\childdoc| renamed to |\childdocmain|
\item
|\childredirect| renamed to |\childdocforward| and |\childdocforwardprefix|
and functionality expanded
\end{itemize}

%%%%%%%%%%%%%%%%%%%%%%%%%%%%%%%%%%%%%%%%
\paragraph{v1.0:} 2017/04/27

\begin{itemize}
\item
manual and install package
\item
first version published on CTAN
\end{itemize}

%%%%%%%%%%%%%%%%%%%%%%%%%%%%%%%%%%%%%%%%
\paragraph{v0.6:} 2017/04/26

\begin{itemize}
\item
redirection mechanism added
\end{itemize}

%%%%%%%%%%%%%%%%%%%%%%%%%%%%%%%%%%%%%%%%
\paragraph{v0.5:} 2017/04/26

\begin{itemize}
\item
functionality in definition file
\end{itemize}


%%%%%%%%%%%%%%%%%%%%%%%%%%%%%%%%%%%%%%%%%%%%%%%%%%%%%%%%%%%%%%%%%%%%%%%%%%%%%%%%
%%%%%%%%%%%%%%%%%%%%%%%%%%%%%%%%%%%%%%%%%%%%%%%%%%%%%%%%%%%%%%%%%%%%%%%%%%%%%%%%
%%%%%%%%%%%%%%%%%%%%%%%%%%%%%%%%%%%%%%%%%%%%%%%%%%%%%%%%%%%%%%%%%%%%%%%%%%%%%%%%
\appendix

\settowidth\MacroIndent{\rmfamily\scriptsize 000\ }

 \DocInput{childdoc.dtx}

\end{document}
%</driver>
% \fi
%
% %%%%%%%%%%%%%%%%%%%%%%%%%%%%%%%%%%%%%%%%%%%%%%%%%%%%%%%%%%%%%%%%%%%%%%%%%%%%%%
% %%%%%%%%%%%%%%%%%%%%%%%%%%%%%%%%%%%%%%%%%%%%%%%%%%%%%%%%%%%%%%%%%%%%%%%%%%%%%%
% \section{Sample}
%\iffalse
%<*samplemain>
%\fi
%
% The following presents a sample document
% with two chapters, two parts, a title page,
% a compile flag as well as three forwarding files to set the flag.
% It consists of eight |.tex| files:
% \begin{center}
% \begin{tabular}{ll}
% |cdocsamp.tex|&main file\\
% |cdocsch1.tex|&include file for chapter 1\\
% |cdocsch2.tex|&include file for chapter 2\\
% |cdocspt3.tex|&include file for part 3\\
% |cdocspt4.tex|&include file for part 4\\
% |cdocsdrf.tex|&forwarding file for main file in draft mode\\
% |cdocsfi1.tex|&forwarding file for final version of chapter 1\\
% |cdocsfi2.tex|&forwarding file for final version of chapter 2\\
% \end{tabular}
% \end{center}
% Each of the eight files can be compiled directly by the \LaTeX{} compiler.
%
% %%%%%%%%%%%%%%%%%%%%%%%%%%%%%%%%%%%%%%
% \paragraph{Main File.}
%
% The main file is called |cdocsamp.tex|.
%
% Load the \textsf{childdoc} definitions and
% declare the filename for the main document:
%    \begin{macrocode}
\input{childdoc.def}
\childdocmain{}
%    \end{macrocode}

% Optional override for |\version| flag:
%    \begin{macrocode}
%%\ifchilddoc\else\providecommand{\version}{draft}\fi
%    \end{macrocode}

% Define the default values for the |\version| flag
% (|final| for the main file and |draft| for childs):
%    \begin{macrocode}
\ifchilddoc
\providecommand{\version}{draft}
\else
\providecommand{\version}{final}
\fi
%    \end{macrocode}

% Load the standard document class:
%    \begin{macrocode}
\documentclass[12pt]{article}
%    \end{macrocode}

% Start the document body:
%    \begin{macrocode}
\begin{document}
%    \end{macrocode}

% Declare a title page.
% Print title, part of document being processed and version flag:
%    \begin{macrocode}
\addtocounter{page}{-1}
\begin{center}
{\LARGE\bfseries{}childdoc example\par}
\vspace{1cm}
\ifchilddoc
\ifchilddocmanual part\else chapter\fi:
`\childdocname' of `\childdocjob'\par
\else
main document: `\childdocjob'\par
\fi
version: \version\par
\end{center}
\newpage
%    \end{macrocode}

% Manually include selected file,
% otherwise process as usual:
%    \begin{macrocode}
\ifchilddocmanual
\section*{part `\childdocname'}
\input{\childdocname}
\else
%    \end{macrocode}

% Include the two chapters:
%    \begin{macrocode}
\include{cdocsch1}
\include{cdocsch2}
%    \end{macrocode}

% Include the two parts unless only chapters should be displayed:
%    \begin{macrocode}
\ifchilddoc\else
\section{part three}
\input{cdocspt3}
\section{part four}
\input{cdocspt4}
\fi
%    \end{macrocode}

% Process as usual until here:
%    \begin{macrocode}
\fi
%    \end{macrocode}

% End of document body:
%    \begin{macrocode}
\end{document}
%    \end{macrocode}
%\iffalse
%</samplemain>
%\fi
%
% %%%%%%%%%%%%%%%%%%%%%%%%%%%%%%%%%%%%%%
% \paragraph{Chapter Include Files.}
%
% The include files are called |cdocsch1.tex| and |cdocsch2.tex|.
%
%\iffalse
%<*samplechap1|samplechap2>
%\fi

% Optional override for |\version| flag:
%    \begin{macrocode}
%%\providecommand{\version}{final}
%    \end{macrocode}

% Include the main document:
%    \begin{macrocode}
\input{childdoc.def}
\childdocof{cdocsamp}
%    \end{macrocode}

%\iffalse
%</samplechap1|samplechap2>
%\fi
%
%\iffalse
%<*samplechap1>
%\fi
% Some text for chapter 1:
%    \begin{macrocode}
\section{one}
some text in chapter one
%    \end{macrocode}

%\iffalse
%</samplechap1>
%\fi
% Some text for chapter 2:
%\iffalse
%<*samplechap2>
%\fi
%    \begin{macrocode}
\section{two}
more text in chapter two
%    \end{macrocode}

%\iffalse
%</samplechap2>
%\fi
%
% %%%%%%%%%%%%%%%%%%%%%%%%%%%%%%%%%%%%%%
% \paragraph{Part Include Files.}
%
% The include files are called |cdocspt3.tex| and |cdocspt4.tex|.
%
%\iffalse
%<*samplepart3|samplepart4>
%\fi

% Optional override for |\version| flag:
%    \begin{macrocode}
%%\providecommand{\version}{final}
%    \end{macrocode}

% Include the main document:
%    \begin{macrocode}
\input{childdoc.def}
\childdocby{cdocsamp}
%    \end{macrocode}

%\iffalse
%</samplepart3|samplepart4>
%\fi
%
%\iffalse
%<*samplepart3>
%\fi
% Some text for part 3:
%    \begin{macrocode}
some text in part three
%    \end{macrocode}

%\iffalse
%</samplepart3>
%\fi
% Some text for part 4:
%\iffalse
%<*samplepart4>
%\fi
%    \begin{macrocode}
more text in part four
%    \end{macrocode}

%\iffalse
%</samplepart4>
%\fi
%
% %%%%%%%%%%%%%%%%%%%%%%%%%%%%%%%%%%%%%%
% \paragraph{Forwarding for a Complete Draft.}
%
% The following forwarding file |cdocsdrf.tex|
% compiles the main document in draft mode:
%\iffalse
%<*sampledraft>
%\fi
%    \begin{macrocode}
\def\version{draft}
\input{childdoc.def}
\childdocforward{cdocsamp}
%    \end{macrocode}

%\iffalse
%</sampledraft>
%\fi
%
% %%%%%%%%%%%%%%%%%%%%%%%%%%%%%%%%%%%%%%
% \paragraph{Forwarding for Final Version of the Chapters.}
%
% The following forwarding files |cdocsfn1.tex| and |cdocsfn2.tex|
% (with identical content)
% compile the final versions of the child documents
% |cdocsch1.tex| and |cdocsch2.tex|, respectively:
%\iffalse
%<*samplefinal>
%\fi
%    \begin{macrocode}
\def\version{final}
\input{childdoc.def}
\childdocforwardprefix[cdocsamp]{cdocsfn}{cdocsch}
%    \end{macrocode}

%\iffalse
%</samplefinal>
%\fi
%
% %%%%%%%%%%%%%%%%%%%%%%%%%%%%%%%%%%%%%%
% \paragraph{Command Line Processing.}
%
% The following three command lines generate the output files
% |cdocscld|, |cdocscl1| and |cdocscl2|
% which should be identical to
% |cdocsdrf|, |cdocsch1| and |cdocsfn2|, respectively:
% \begin{center}
% \begin{tabular}{l}
% |latex -jobname cdocscld \|\\
% |  "\def\version{draft}\input{childdoc.def}\childdocforward{cdocsamp}"|\\
% |latex -jobname cdocscl1 \|\\
% |  "\input{childdoc.def}\childdocforward[cdocsamp]{cdocsch1}"|\\
% |latex -jobname cdocscl2 \|\\
% |  "\def\version{final}\input{childdoc.def}\childdocforward{cdocsch2}"|
% \end{tabular}
% \end{center}
% Note that the trailing backslash on each first line
% merely continues the input to the second line
% (for convenient cut ant paste).
% Furthermore, the command |latex| can be replaced by any
% of its alternative versions such as |pdflatex|.
%
% %%%%%%%%%%%%%%%%%%%%%%%%%%%%%%%%%%%%%%%%%%%%%%%%%%%%%%%%%%%%%%%%%%%%%%%%%%%%%%
% %%%%%%%%%%%%%%%%%%%%%%%%%%%%%%%%%%%%%%%%%%%%%%%%%%%%%%%%%%%%%%%%%%%%%%%%%%%%%%
% \section{Implementation}
%\iffalse
%<*package>
%\fi
%
% This section describes the definitions file |childdoc.def|.

% The definitions cannot be loaded using |\usepackage| or |\RequirePackage|
% which has a mechanism to prevent loading a style file more than once.
% When loading the definitions by means of |\input|
% multiple instances have to be prevented manually:
%\iffalse
%This code needs to be before the `\ProvidesFile' directive
%which is defined at the beginning of this file.
%Therefore it is also placed there and commented out here.
%</package>
%<*discard>
%\fi
%    \begin{macrocode}
\ifdefined\childdocmain\endinput\fi
%    \end{macrocode}
%\iffalse
%</discard>
%<*package>
%\fi
%
% \macro{\ifchilddoc}
% \macro{\ifchilddocmanual}
% The conditional |\ifchilddoc| tells whether a
% child (true) or main (false) document is being compiled.
% The conditional |\ifchilddocmanual| tells whether
% the |\includeonly| mechanism is used (false) or
% the selection of child files must be performed manually (true).
% The definitions initialise to false:
%    \begin{macrocode}
\newif\ifchilddoc
\newif\ifchilddocmanual
%    \end{macrocode}

% \macro{\childdocname}
% \macro{\childdocjob}
% The macro |\childdocname| stores the name of the main document
% to be compiled. The macro |\childdocjob| stores the name of
% the document on which the \LaTeX{} compiler was originally invoked.
% The content of |\jobname| cannot be compared
% to filenames specified in the source due to different catcodes.
% The following code rescans |\jobname|, stores the result
% in |\childdocname| and saves a copy in |\childdocjob|:
%    \begin{macrocode}
\edef\childdocname{\scantokens\expandafter{\jobname\noexpand}}
\let\childdocjob\childdocname
%    \end{macrocode}

% \macro{\childdocdisable}
% The macro |\childdocdisable| prevents the main file
% from being processed more than once.
% At this stage, the main document command |\childdocmain|
% is assumed to be called once again where it should do nothing.
% Any subsequent call to it should prevent
% a secondary processing of the main document
% It overwrites the forwarding commands
% |\childdocof| and |\childdocforward|
% with empty macros to prevent further inclusions of the main document:
%    \begin{macrocode}
\newcommand{\childdocdisable}
{
  \renewcommand{\childdocmain}[1]{\renewcommand{\childdocmain}[1]{\endinput}}
  \renewcommand{\childdocof}[1]{}
  \renewcommand{\childdocby}[2][]{}
  \renewcommand{\childdocforward}[2][]{}
  \renewcommand{\childdocdisable}{}
}
%    \end{macrocode}

% \macro{\childdocmain}
% The macro |\childdocmain| is to be called at the top of the main file
% with nothing or the main filename (without extension) as argument.
% First, it breaks loops.
% If the argument is not empty and does not match |\childdocname|
% (which is set by the first inclusion of |childdoc.def|),
% |\ifchilddoc| is set to true, |\includeonly| is applied to the child file
% and |\jobname| is set to the main file
% (for proper handling of |.aux| files):
%    \begin{macrocode}
\newcommand{\childdocmain}[1]
{
  \childdocdisable\childdocmain{}
  \if?#1?\else
    \begingroup
      \def\childdoctmp{#1}
      \ifx\childdoctmp\childdocname
        \def\childdoctmp{}
      \else
        \def\childdoctmp
        {
          \childdoctrue
          \includeonly{\childdocname}
          \def\childdocjob{#1}
          \def\jobname{#1}
        }
      \fi
      \expandafter
    \endgroup
    \childdoctmp
  \fi
}
%    \end{macrocode}

% \macro{\childdocof}
% The command |\childdocof| redirects
% compilation to the main file |#1|.
%    \begin{macrocode}
\newcommand{\childdocof}[1]
{
  \childdocdisable
  \childdoctrue
  \includeonly{\childdocname}
  \def\jobname{#1}
  \def\childdocjob{#1}
  \input{#1}
}
%    \end{macrocode}

% \macro{\childdocby}
% The command |\childdocby| ....
%    \begin{macrocode}
\newcommand{\childdocby}[2][]
{
  \childdocdisable
  \childdoctrue
  \childdocmanualtrue
  \if?#1?\else
    \def\jobname{#2}
  \fi
  \def\childdocjob{#2}
  \input{#2}
  \endinput
}
%    \end{macrocode}

% \macro{\childdocforward}
% The command |\childdocforward| redirects
% compilation to the main file or
% (if the optional argument is given) a child file.
% Parameters are set as if the main file
% or a child file starting with |\childdocof| was compiled.
% Then compilation is handed over to the main file:
%    \begin{macrocode}
\newcommand{\childdocforward}[2][]
{
  \begingroup
    \if?#1?
      \def\childdoctmp
      {
        \def\childdocname{#2}
        \def\childdocjob{#2}
        \def\jobname{#2}
        \input{#2}
        \endinput
      }
    \else
      \def\childdoctmp
      {
        \childdocdisable
        \def\childdocname{#2}
        \childdoctrue
        \includeonly{#2}
        \def\childdocjob{#1}
        \def\jobname{#1}
        \input{#1}
        \endinput
      }
    \fi
    \expandafter
  \endgroup
  \childdoctmp
}
%    \end{macrocode}

% \macro{\childdocforwardprefix}
% The command |\childdocforwardprefix| redirects
% compilation to the main or a child file by means of a pattern.
% The prefix |#1| in the current filename is replaced by |#2|
% and the suffix of the current filename is kept
% (it is assumed that the filename does not contain the substring `|~~~|'
% which is used as a delimiter).
% Compilation is handed over to the new file by |\childdocforward|:
%    \begin{macrocode}
\newcommand{\childdocforwardprefix}[3][]
{
  \begingroup
    \def\childdocextract #2##1~~~{\def\childdoctmp{\childdocforward[#1]{#3##1}}}
    \expandafter\childdocextract\childdocname~~~
    \expandafter
  \endgroup
  \childdoctmp
}
%    \end{macrocode}

% \macro{\childdoc}
% The deprecated macro |\childdoc| is a legacy version of |\childdocmain|:
%    \begin{macrocode}
\newcommand{\childdoc}{\childdocmain}
%    \end{macrocode}

% \macro{\childdocredirect}
% The deprecated macro |\childdocredirect| is a legacy version
% of |\childdocforward| and |\childdocforwardprefix|:
%    \begin{macrocode}
\newcommand{\childdocredirect}[2][]
{
  \begingroup
    \if?#1?
      \def\childdoctmp{\childdocforward{#2}}
    \else
      \def\childdoctmp{\childdocforwardprefix{#1}{#2}}
    \fi
    \expandafter
  \endgroup
  \childdoctmp
}
%    \end{macrocode}

%\iffalse
%</package>
%\fi
%
\endinput

\childdocforwardprefix[cdocsamp]{cdocsfn}{cdocsch}
%    \end{macrocode}

%\iffalse
%</samplefinal>
%\fi
%
% %%%%%%%%%%%%%%%%%%%%%%%%%%%%%%%%%%%%%%
% \paragraph{Command Line Processing.}
%
% The following three command lines generate the output files
% |cdocscld|, |cdocscl1| and |cdocscl2|
% which should be identical to
% |cdocsdrf|, |cdocsch1| and |cdocsfn2|, respectively:
% \begin{center}
% \begin{tabular}{l}
% |latex -jobname cdocscld \|\\
% |  "\def\version{draft}% \iffalse
%
% childdoc.dtx Copyright (C) 2017-2018 Niklas Beisert
%
% This work may be distributed and/or modified under the
% conditions of the LaTeX Project Public License, either version 1.3
% of this license or (at your option) any later version.
% The latest version of this license is in
%   http://www.latex-project.org/lppl.txt
% and version 1.3 or later is part of all distributions of LaTeX
% version 2005/12/01 or later.
%
% This work has the LPPL maintenance status `maintained'.
%
% The Current Maintainer of this work is Niklas Beisert.
%
% This work consists of the files childdoc.dtx and childdoc.ins
% and the derived files childdoc.def and cdocsamp.tex with
% cdocsch1.tex, cdocsch2.tex, cdocsdrf.tex, cdocsfn1.tex, cdocsfn2.tex.
%
%<package>\ifdefined\childdocmain\endinput\fi
%<package>\ProvidesFile{childdoc.def}[2018/12/30 v2.0 child document driver]
%<samplemain>\ProvidesFile{cdocsamp.tex}[2018/12/30 v2.0 sample for childdoc]
%<*driver>
%\ProvidesFile{childdoc.drv}[2018/12/30 v2.0 childdoc reference manual file]
\PassOptionsToClass{10pt,a4paper}{article}
\documentclass{ltxdoc}

\usepackage[margin=35mm]{geometry}
\usepackage{hyperref}
\usepackage{hyperxmp}
\usepackage[usenames]{color}

\hypersetup{colorlinks=true}
\hypersetup{pdfstartview=FitH}
\hypersetup{pdfpagemode=UseNone}
\hypersetup{pdfsource={}}
\hypersetup{pdflang={en-UK}}
\hypersetup{pdfcopyright={Copyright 2017-2018 Niklas Beisert.
  This work may be distributed and/or modified under the
  conditions of the LaTeX Project Public License, either version 1.3
  of this license or (at your option) any later version.}}
\hypersetup{pdflicenseurl={http://www.latex-project.org/lppl.txt}}
\hypersetup{pdfcontactaddress={ETH Zurich, ITP, HIT K,
  Wolfgang-Pauli-Strasse 27}}
\hypersetup{pdfcontactpostcode={8093}}
\hypersetup{pdfcontactcity={Zurich}}
\hypersetup{pdfcontactcountry={Switzerland}}
\hypersetup{pdfcontactemail={nbeisert@itp.phys.ethz.ch}}
\hypersetup{pdfcontacturl={http://people.phys.ethz.ch/\xmptilde nbeisert/}}

\newcommand{\secref}[1]{\hyperref[#1]{section \ref*{#1}}}

\parskip1ex
\parindent0pt
\let\olditemize\itemize
\def\itemize{\olditemize\parskip0pt}

\begin{document}

\title{The \textsf{childdoc} Package}
\hypersetup{pdftitle={The childdoc Package}}
\author{Niklas Beisert\\[2ex]
  Institut f\"ur Theoretische Physik\\
  Eidgen\"ossische Technische Hochschule Z\"urich\\
  Wolfgang-Pauli-Strasse 27, 8093 Z\"urich, Switzerland\\[1ex]
  \href{mailto:nbeisert@itp.phys.ethz.ch}
  {\texttt{nbeisert@itp.phys.ethz.ch}}}
\hypersetup{pdfauthor={Niklas Beisert}}
\hypersetup{pdfsubject={Manual for the LaTeX2e Package childdoc}}
\date{30 December 2018, \textsf{v2.0}}
\maketitle

\begin{abstract}\noindent
\textsf{childdoc} is a \LaTeXe{} package
that enables the direct compilation
of document sections included by |\include|
to individual files.
\end{abstract}

\begingroup
\parskip0ex
\tableofcontents
\endgroup

%%%%%%%%%%%%%%%%%%%%%%%%%%%%%%%%%%%%%%%%%%%%%%%%%%%%%%%%%%%%%%%%%%%%%%%%%%%%%%%%
%%%%%%%%%%%%%%%%%%%%%%%%%%%%%%%%%%%%%%%%%%%%%%%%%%%%%%%%%%%%%%%%%%%%%%%%%%%%%%%%
\section{Introduction}

\LaTeX{} provides a mechanism to structure a large document (such as a book)
into a main file and several child files (containing the chapters)
using the |\include| command.
This mechanism is beneficial for documents
which span hundreds of pages in order to
make the source file(s) more manageable.
Moreover, compilation can be restricted to
selected child files by means of the |\includeonly| command.
The latter feature can be used to reduce the compilation time while editing
(this was significantly more useful in the earlier days of \LaTeX{})
or to generate a smaller document which is easier to navigate.
Another application of |\includeonly| is to generate
documents consisting of selected parts of the complete document.

However, there are a few drawbacks of the plain |\include| mechanism:
\begin{itemize}
\item
The child files cannot be compiled on their own,
they can only be compiled via the main file.
A naive editing environment
(such as a text editor with an option
to have the current file processed by \LaTeX)
may require one to switch to the main file before compiling;
attempting to compile the child file produces errors.
\item
The main file must be modified (each time)
to adjust the |\includeonly| command
to the present needs. This easily leaves the main file in a messy state.
\item
The generated document will always carry the filename
of the main document. This is inconvenient if
several child files are to be compiled and
to be kept for distribution.
\end{itemize}

The present package provides a simple interface
to make child files individually compilable by \LaTeX{}.
Compiling a child file then has the same effect as compiling
the main file with an |\includeonly| command
to select the appropriate child.
Moreover the generated document will carry the name of the child
rather than the main file.
This resolves all three above issues.

This feature is meant to make the editing of books,
thesis documents and lecture notes somewhat more convenient.
However, the package can also be used efficiently for
composing a series of documents (such as exercise sheets)
which are typically distributed individually.
It then assists the author in generating the individual documents
(potentially in different versions)
as well as a document containing the collected series.
Another application is in developing style files
or other kinds of included material
where compilation of the style file could redirect
to a sample or test file.

%%%%%%%%%%%%%%%%%%%%%%%%%%%%%%%%%%%%%%%%%%%%%%%%%%%%%%%%%%%%%%%%%%%%%%%%%%%%%%%%
%%%%%%%%%%%%%%%%%%%%%%%%%%%%%%%%%%%%%%%%%%%%%%%%%%%%%%%%%%%%%%%%%%%%%%%%%%%%%%%%
\section{Usage}

First of all, the package \textsf{childdoc} is \emph{not} a standard
\LaTeXe{} |.sty| style file! Therefore it needs to be invoked in
a non-standard way.

%%%%%%%%%%%%%%%%%%%%%%%%%%%%%%%%%%%%%%%%%%%%%%%%%%%%%%%%%%%%%%%%%%%%%%%%%%%%%%%%
\subsection{Included Files}
\label{sec:include}

%%%%%%%%%%%%%%%%%%%%%%%%%%%%%%%%%%%%%%%%
\DescribeMacro{\childdocmain}
To use the package, add the commands
\begin{center}
\begin{tabular}{l}
|\input{childdoc.def}|\\
|\childdocmain{}|\\
\end{tabular}
\end{center}
at the very top of the main \LaTeX{} file,
in particular \emph{before} the |\documentclass| statement!
The argument of |\childdocmain| should be left empty
(but it must be present).

%%%%%%%%%%%%%%%%%%%%%%%%%%%%%%%%%%%%%%%%
\DescribeMacro{\childdocof}
Furthermore, add the commands
\begin{center}
\begin{tabular}{l}
|\input{childdoc.def}|\\
|\childdocof{|\textit{main}|}|\\
\end{tabular}
\end{center}
at the top of every child file \textit{child}
which is included by |\include{|\textit{child}|}|
from within the main file
(or at least for those files to be compiled individually).
The argument \textit{main} must be the filename of the main file.

There are a couple of
considerations in setting up the main and child documents:

%%%%%%%%%%%%%%%%%%%%%%%%%%%%%%%%%%%%%%%%
\paragraph{Restrictions.}

Please note the following restrictions:
\begin{itemize}
\item
|\childdocmain| must be called with one argument \textit{main}
to ensure compatibility with earlier version of the package.
It must either be empty (|\childdocmain{}|)
or precisely match the filename of the main file in which it is specified.
See \secref{sec:detection} for further information.
\item
The filename \textit{main} must be specified without the |.tex| extension.
\item
The filename \textit{main} is case sensitive
(even in case-insensitive file systems)
due to internal string comparison.
\item
The argument \textit{main} should be fully expanded, it cannot be a macro.
\item
Subdirectories and special characters should be avoided in filenames.
\item
The command |\childdocmain{|\textit{main}|}| must be followed by a whitespace.
It should not be followed immediately by another command
or by a comment mark `|%|'.
This is because the \TeX{} parser reads the token immediately following
the argument of |\childdocmain| and puts it
at the beginning of every child section;
however, a white\-space is ignored.
\end{itemize}

%%%%%%%%%%%%%%%%%%%%%%%%%%%%%%%%%%%%%%%%
\paragraph{Content of Main File.}

It is advisable to place all content in the child files included by |\include|.
Any output contained in the main file will appear in all child documents
unless suppressed manually;
it cannot be suppressed automatically by the |\includeonly| directive
and thus should normally be avoided.
A method to include some content in the main file
by means of conditional processing is described in \secref{sec:conditional}.

%%%%%%%%%%%%%%%%%%%%%%%%%%%%%%%%%%%%%%%%
\paragraph{Page Numbering.}

When only a part of the document is compiled,
the appropriate numbering of pages
(as well as other status parameters)
is determined from the |.aux| files.
The latter contain information from previous passes.
However this information needs to propagate through
all intermediate child documents.
Therefore the page numbering in child documents may well
be inconsistent until the complete document is compiled at least once.

A useful (if unconventional) way to always ensure a consistent
page numbering is to restart the numbering in each child document
and denote the pages by `\textit{child}|.|\textit{page}'
where \textit{child} represents the chapter/section number of the child file.
This can be achieved by the command
|\numberwithin{page}{|\textit{child}|}|
of the \textsf{amsmath} package
where \textit{child} can be |chapter| or |section|
depending on the chosen structuring.
Alternatively, one can modify the macro |\thepage| appropriately
and reset the counter |page| at the start of each child file.

%%%%%%%%%%%%%%%%%%%%%%%%%%%%%%%%%%%%%%%%%%%%%%%%%%%%%%%%%%%%%%%%%%%%%%%%%%%%%%%%
\subsection{Conditional Processing}
\label{sec:conditional}

The package provides a mechanism to compile different versions
of a document. To customise the versions further some conditional processing
can come in handy to distinguish which version is being compiled.
The package provides two macros to describe the compilation context:

%%%%%%%%%%%%%%%%%%%%%%%%%%%%%%%%%%%%%%%%
\DescribeMacro{\ifchilddoc}
The conditional |\ifchilddoc| distinguishes between the compilation of
child documents and the main document:
%
\begin{center}
|\ifchilddoc |\textit{child-code}| |[|\||else |\textit{main-code}]| \||fi|
\end{center}

%%%%%%%%%%%%%%%%%%%%%%%%%%%%%%%%%%%%%%%%
\DescribeMacro{\childdocname}
\DescribeMacro{\childdocjob}
The macro |\childdocname| contains the filename (without extension)
of the main or child file being processed.
Note that |\childdocjob| will always contain the name of the main file.

%%%%%%%%%%%%%%%%%%%%%%%%%%%%%%%%%%%%%%%%
\paragraph{Title Page.}

Conditional processing can be used to include a title or banner page
in the main document when proper precautions are taken.
Importantly, the code in the main file should ensure that the page counter
(as well as other status parameters which are stored in the |.aux| files)
takes the same value after the conditional processing.
Otherwise the page numbers may take divergent values
depending on which part is compiled.

For example, a title page could be declared by:
%
\begin{center}
\begin{tabular}{l}
|\ifchilddoc\||else|\\
|\addtocounter{page}{-1}|\\
\textit{code for title page}\\
|\newpage|\\
|\||fi|
\end{tabular}
\end{center}
%
A banner page for the child documents can be generated by:
%
\begin{center}
\begin{tabular}{l}
|\ifchilddoc|\\
|\addtocounter{page}{-1}|\\
\textit{code for banner page}\\
|\newpage|\\
|\||fi|
\end{tabular}
\end{center}
%
Here one could write a message such as:
\begin{center}
|This is the part \childdocname{} of \childdocjob{}.|
\end{center}

%%%%%%%%%%%%%%%%%%%%%%%%%%%%%%%%%%%%%%%%%%%%%%%%%%%%%%%%%%%%%%%%%%%%%%%%%%%%%%%%
\subsection{Flags}
\label{sec:flags}

The package makes it easy to generate different versions
of the main or child documents.
To this end compilation flags can be defined
and assigned different default values.
They will be particularly useful in conjunction
with the forwarding mechanism described in \secref{sec:forward}.

For example, it may be useful to have a flag |\version|
which can be set to |draft| or |final|.
The document source will contain some conditional code
depending on the value of |\version|.
Suppose further, the flag should default to |final| for the main file
and to |draft| for child files
which is a natural assignment for editing the document.
This is achieved by placing the following code
in the preamble of the main document
(below the |\childdocmain| directive):
%
\begin{center}
\begin{tabular}{l}
|\ifchilddoc|\\
|\providecommand{\version}{draft}|\\
|\||else|\\
|\providecommand{\version}{final}|\\
|\||fi|
\end{tabular}
\end{center}
%
The definition by |\providecommand| makes sure
that previous definitions are not overwritten.
Further statements |\providecommand{\version}{...}|
can thus be added before the above code to override it.

For the main file, one might add a line
(between |\childdocmain| and the above block)
%
\begin{center}
|%\ifchilddoc\||else\providecommand{\version}{draft}\||fi|
\end{center}
%
which can be uncommented to produce a draft version.
Likewise one can add a line to the very top of a child file
(above the |\childdocof{|\textit{main}|}| directive)
%
\begin{center}
|%\providecommand{\version}{final}|
\end{center}
%
which can be uncommented to produce the final version of this child document.

%%%%%%%%%%%%%%%%%%%%%%%%%%%%%%%%%%%%%%%%%%%%%%%%%%%%%%%%%%%%%%%%%%%%%%%%%%%%%%%%
\subsection{Forwarding}
\label{sec:forward}

Different versions of the main or child documents
using compilation flags as described in \secref{sec:flags}
can be (permanently) stored in different files
for convenient compilation, viewing and distribution.
To this end, the package defines a command
to pass on compilation to a different file:

%%%%%%%%%%%%%%%%%%%%%%%%%%%%%%%%%%%%%%%%
\DescribeMacro{\childdocforward}
The command |\childdocforward| redirects processing to
another source file:
%
\begin{center}
\begin{tabular}{l}
|\input{childdoc.def}|\\
|\childdocforward[|\textit{main}|]{|\textit{dest}|}|\\
\end{tabular}
\end{center}
%
The argument \textit{dest} is the destination file
(without extension).
It should be the main file or one of the child files.
Note that further \textsf{childdoc} directives
such as |\childdocof| and |\childdocforward|
in the indicated file will be processed in this form.
The optional argument \textit{main}
passes on directly to the main file \textit{main}
while pretending to compile the child \textit{dest}.
This form behaves as if \textit{dest}
issues |\childdocof{|\textit{main}|}| right away,
and no further \textsf{childdoc} directives will be processed.

%%%%%%%%%%%%%%%%%%%%%%%%%%%%%%%%%%%%%%%%
\DescribeMacro{\...prefix}
In the alternative form |\childdocforwardprefix|,
%
\begin{center}
\begin{tabular}{l}
|\input{childdoc.def}|\\
|\childdocforwardprefix[|\textit{main}|]{|\textit{prefix}|}{|\textit{dest}|}|
\end{tabular}
\end{center}
%
the destination file is determined by a pattern
depending on the current file:
To make this work, the current file must be called
`{\textit{prefix}\hspace{0.2em}\textit{suffix}}'
with \textit{prefix} matching precisely the argument.
Processing is then passed on to the file
`{\textit{dest}\hspace{0.2em}\textit{suffix}}'.
Surely, the same effect is achieved by
directly specifying the
argument `{\textit{dest}\hspace{0.2em}\textit{suffix}}'
in the first form.
However, that requires to set up a different file
for each child. With the alternative form of the command
all these files can have exactly the same content
which simplifies setting them up and maintaining them.

For example, the following file |draft.tex|
with a compilation flag |\version| as described in \secref{sec:flags}
compiles the main document as a draft:
%
\begin{center}
\begin{tabular}{l}
|\def\version{draft}|\\
|\input{childdoc.def}|\\
|\childdocforward{|\textit{main}|}|
\end{tabular}
\end{center}
%
Likewise, the following files |final|\textit{nn}|.tex|
compile the final version of the child document
|child|\textit{nn}|.tex|:
%
\begin{center}
\begin{tabular}{l}
|\def\version{final}|\\
|\input{childdoc.def}|\\
|\childdocforwardprefix{final}{child}|
\end{tabular}
\end{center}
%

Note that when several versions of a main file and/or of each child file
are to be generated, it may be convenient to set up a |Makefile| or
shell script to automatise the process.

%%%%%%%%%%%%%%%%%%%%%%%%%%%%%%%%%%%%%%%%%%%%%%%%%%%%%%%%%%%%%%%%%%%%%%%%%%%%%%%%
\subsection{Command Line Processing}
\label{sec:commandline}

The effect of redirection files can also be achieved by invoking
the \LaTeX{} compiler with a more elaborate command line.
Most conveniently this should be done as part
of a shell script or a |Makefile|.

When using \textsf{childdoc} in the main file, the following
command lines effectively perform a redirection
(note that depending on the shell being used,
backslashes may have to be doubled: `|\|' $\to$ `|\\|'):
%
\begin{center}
|... -jobname "|\textit{target}|" |\\|"|[\textit{flags}]%
|\input{childdoc.def}\childdocforward[|\textit{main}|]{|\textit{dest}|}"|
\end{center}
%
Here \textit{target} is the name of the output file,
\textit{main} is the name of the main file
and \textit{dest} is the name of the main or child file to be processed
(all filenames without extensions).
The optional argument \textit{main} can be omitted
if \textit{main} matches \textit{dest}.
Optionally, compilation \textit{flags} can be defined via |\def| commands.
This command line makes the \TeX{} engine believe
it is compiling the file \textit{target}
whose content is specified as the latter parameter.
The provided code then forwards the processing to
\textit{main} or \textit{dest} as described in \secref{sec:forward}.

%%%%%%%%%%%%%%%%%%%%%%%%%%%%%%%%%%%%%%%%%%%%%%%%%%%%%%%%%%%%%%%%%%%%%%%%%%%%%%%%
\subsection{Include by Input}
\label{sec:input}

Including child documents by |\include| has some restrictions by design.
Most notably, the content of a child document always occupies
its own set of pages; pages cannot be shared between child documents.
Usually, this behaviour makes perfect sense
because each child document contain an essential part of the document.
However, in some situations it may be desirable to compose
a document from a collection of parts
without having mandatory page breaks between then.
For this case, the package
provides a mechanism to include parts
by |\input| which can also be processed individually.
However, by construction this mechanism
requires manual handling of the content to be output.

%%%%%%%%%%%%%%%%%%%%%%%%%%%%%%%%%%%%%%%%
\DescribeMacro{\ifchilddocmanual}
The main file should be prepared as usual, see \secref{sec:include}.
However, the document body must make a distinction
between processing of an individual part and of the main document, e.g.:
%
\begin{center}
\begin{tabular}{l}
|\ifchilddocmanual|\\
|\input{\childdocname}|\\
|\||else|\\
\textit{document body with }|\input{|\textit{part}|}|\\
|\||fi|
\end{tabular}
\end{center}
%
The conditional |\ifchilddocmanual| is true whenever
a part to be included by |\input| is being compiled,
and the name of the part is stored in |\childdocname|.

%%%%%%%%%%%%%%%%%%%%%%%%%%%%%%%%%%%%%%%%
\DescribeMacro{\childdocby}
Each part to be included by |\input| should start with:
%
\begin{center}
\begin{tabular}{l}
|\input{childdoc.def}|\\
|\childdocby{|\textit{main}|}|\\
\end{tabular}
\end{center}
%
The directive |\childdocby| is similar to |\childdocof|
described in \secref{sec:include},
but the subsequent selection of content must be done manually.
To that end, both |\ifchilddoc| and |\ifchilddocmanual|
will be true upon processing of a part,
and the name of the part is stored in |\childdocname|.
Note that |\jobname| will be set to the filename of the current part
so that each part receives an individual |.aux| file
that does not interfere with the |.aux| file(s) of the main document.
This behaviour can be altered by the alternative form
|\childdocby[*]{|\textit{main}|}| (with a non-empty optional argument)
which uses the |.aux| file of the main document
by setting |\jobname| to \textit{main}.

%%%%%%%%%%%%%%%%%%%%%%%%%%%%%%%%%%%%%%%%%%%%%%%%%%%%%%%%%%%%%%%%%%%%%%%%%%%%%%%%
\subsection{Driver Development}
\label{sec:driver}

The \textsf{childdoc} mechanism can also be use for the development
of definition files such as \LaTeX{} styles or classes.
This case differs from the above setup with multiple parts
included by |\include| in that no |\includeonly| should be invoked.
This can be achieved by starting the include file
(before |\ProvidesPackage|) with:
%
\begin{center}
\begin{tabular}{l}
|\input{childdoc.def}|\\
|\childdocforward{|\textit{main}|}|\\
\end{tabular}
\end{center}
%
or alternatively with:
%
\begin{center}
\begin{tabular}{l}
|\input{childdoc.def}|\\
|\childdocby{|\textit{main}|}|\\
\end{tabular}
\end{center}
%
Both forms have slightly different effects as described above.
The main file is prepared as usual, see \secref{sec:include}.

%%%%%%%%%%%%%%%%%%%%%%%%%%%%%%%%%%%%%%%%%%%%%%%%%%%%%%%%%%%%%%%%%%%%%%%%%%%%%%%%
\subsection{Legacy Detection}
\label{sec:detection}

The directive |\childdocmain| in the main file can detect
whether the complete document or merely a child is to be compiled
even without using the directive |\childdocof|.
This method is deprecated because it is less robust
and there is no compelling reason to use it;
it is merely provided for backward compatibility
and it may be removed in future versions.

If the detection mechanism is to be used,
it is mandatory to correctly specify
the filename of the main file as the argument of |\childdocmain|:
%
\begin{center}
\begin{tabular}{l}
|\input{childdoc.def}|\\
|\childdocmain{|\textit{main}|}|\\
\end{tabular}
\end{center}
%
If |\jobname| does not match the argument \textit{main} of |\childdocmain|,
it is assumed that |\jobname| points to the child file to be compiled.
When using |\childdocmain| with the main file specified as argument,
it suffices to start a child file
with just |\input{|\textit{main}|}|
without loading of the package and using |\childdocof|.
If instead all processing is done
with the appropriate \textsf{childdoc} directives,
the argument of \textit{main} of |\childdocmain| can be empty.

An alternative version of the command line processing described
in \secref{sec:commandline} using the detection mechanism reads:
%
\begin{center}
|... -jobname "|\textit{target}|" "|[\textit{flags}]%
[|\def\jobname{|\textit{dest}|}|]|\input{|\textit{main}|}"|
\end{center}

%%%%%%%%%%%%%%%%%%%%%%%%%%%%%%%%%%%%%%%%%%%%%%%%%%%%%%%%%%%%%%%%%%%%%%%%%%%%%%%%
\subsection{Manual Code}
\label{sec:manual}

In case one cannot be certain whether the definitions file |childdoc.def|
is installed on the target \TeX{} distribution
and one prefers not to ship it,
it is conceivable to paste a few relevant commands into the sources.

To that end, drop all statements |\input{childdoc.def}|
and perform the replacements as outlined below.
Instead of |\childdocmain{|\textit{main}|}| add the following code
to the top of the main file:
%
\begin{center}
\begin{tabular}{l}
|\||ifdefined\childdocname\endinput\||fi\newif\ifchilddoc|\\
|\edef\childdocname{\scantokens\expandafter{\jobname\noexpand}}|\\
|\def\childdocmain{|\textit{main}|}\||ifx\childdocmain\childdocname\||else|\\
|\childdoctrue\includeonly{\childdocname}\let\jobname\childdocmain\||fi|\\
\end{tabular}
\end{center}
%
Instead of |\childdocof{|\textit{main}|}| just include the main file
at the top of each child file:
%
\begin{center}
|\input{|\textit{main}|}|
\end{center}
%
A simple redirection |\childdocforward{|\textit{dest}|}| is achieved by:
%
\begin{center}
|\def\jobname{|\textit{dest}|}\input{\jobname}|
\end{center}
%
The redirection with prefix
|\childdocforwardprefix[|\textit{prefix}|]{|\textit{dest}|}|
is accomplished by:
%
\begin{center}
\begin{tabular}{l}
|{\edef\jobname{\scantokens\expandafter{\jobname\noexpand}}|\\
|\def\redirectjob |\textit{prefix}|#1~~~{\gdef\jobname{|\textit{dest}|#1}}|\\
|\expandafter\redirectjob\jobname~~~}\input{\jobname}|
\end{tabular}
\end{center}

In an alternative approach,
child documents can be compiled by a specific command line
without additional code or specific definitions:
%
\begin{center}
|... -jobname "|\textit{target}|" "|[\textit{flags}]%
|\includeonly{|\textit{dest}|}\input{|\textit{main}|}"|
\end{center}
%

%%%%%%%%%%%%%%%%%%%%%%%%%%%%%%%%%%%%%%%%%%%%%%%%%%%%%%%%%%%%%%%%%%%%%%%%%%%%%%%%
%%%%%%%%%%%%%%%%%%%%%%%%%%%%%%%%%%%%%%%%%%%%%%%%%%%%%%%%%%%%%%%%%%%%%%%%%%%%%%%%
\section{Information}

%%%%%%%%%%%%%%%%%%%%%%%%%%%%%%%%%%%%%%%%%%%%%%%%%%%%%%%%%%%%%%%%%%%%%%%%%%%%%%%%
\subsection{Copyright}

Copyright \copyright{} 2017--2018 Niklas Beisert

This work may be distributed and/or modified under the
conditions of the \LaTeX{} Project Public License, either version 1.3
of this license or (at your option) any later version.
The latest version of this license is in
  \url{http://www.latex-project.org/lppl.txt}
and version 1.3 or later is part of all distributions of \LaTeX{}
version 2005/12/01 or later.

This work has the LPPL maintenance status `maintained'.

The Current Maintainer of this work is Niklas Beisert.

This work consists of the files |README.txt|, |childdoc.ins| and |childdoc.dtx|
as well as the derived files |childdoc.def|, |cdocsamp.tex|
with |cdocsch1.tex|, |cdocsch2.tex|, |cdocspt3.tex|, |cdocspt4.tex|,
|cdocsdrf.tex|, |cdocsfn1.tex|, |cdocsfn2.tex|
as well as |childdoc.pdf|.

%%%%%%%%%%%%%%%%%%%%%%%%%%%%%%%%%%%%%%%%%%%%%%%%%%%%%%%%%%%%%%%%%%%%%%%%%%%%%%%%
\subsection{Files and Installation}

The package consists of the files:
%
\begin{center}
\begin{tabular}{ll}
    |README.txt|   & readme file \\
    |childdoc.ins| & installation file \\
    |childdoc.dtx| & source file \\
    |childdoc.def| & definition file \\
    |cdocsamp.tex| & sample main file \\
    |cdocsch1.tex| & sample include file \\
    |cdocsch2.tex| & sample include file \\
    |cdocspt3.tex| & sample part file \\
    |cdocspt4.tex| & sample part file \\
    |cdocsdrf.tex| & sample redirection file \\
    |cdocsfn1.tex| & sample redirection file \\
    |cdocsfn2.tex| & sample redirection file \\
    |childdoc.pdf| & manual
\end{tabular}
\end{center}
%
The distribution consists of the files
|README.txt|, |childdoc.ins| and |childdoc.dtx|.
%
\begin{itemize}
\item
Run (pdf)\LaTeX{} on |childdoc.dtx|
to compile the manual |childdoc.pdf| (this file).
\item
Run \LaTeX{} on |childdoc.ins| to create the definitions file |childdoc.def|
and the sample |cdocsamp.tex| with include files
|cdocsch1.tex|, |cdocsch2.tex|, |cdocspt3.tex|, |cdocspt4.tex|,
|cdocsdrf.tex|, |cdocsfn1.tex|, |cdocsfn2.tex|.
Then copy the file |childdoc.def| to an appropriate directory of your \LaTeX{}
distribution, e.g.\ \textit{texmf-root}|/tex/latex/childdoc|.
\end{itemize}

%%%%%%%%%%%%%%%%%%%%%%%%%%%%%%%%%%%%%%%%%%%%%%%%%%%%%%%%%%%%%%%%%%%%%%%%%%%%%%%%
\subsection{Related CTAN Packages}

There are several other packages which offer a similar functionality:
%
\begin{itemize}
\item
The packages
\href{http://ctan.org/pkg/docmute}{\textsf{docmute}},
\href{http://ctan.org/pkg/includex}{\textsf{includex}} and
\href{http://ctan.org/pkg/standalone}{\textsf{standalone}}
provide commands to include only the document body of
a child file thus allowing both files to be compiled individually.
\item
The packages \href{http://ctan.org/pkg/subdocs}{\textsf{subdocs}}
and \href{http://ctan.org/pkg/subfiles}{\textsf{subfiles}}
provide structures in which the main and child documents can be
encapsulated and allowing them to be compiled individually.
The inclusion mechanism is different from the conventional |\include|.
\item
The package \href{http://ctan.org/pkg/combine}{\textsf{combine}}
is an elaborate solution to combine several documents into one.
\end{itemize}
%
See also the CTAN topic \href{http://ctan.org/topic/subdocs}{\textsf{subdocs}}
for further related packages.
The present package differs from the above solutions in that
a document structure constructed with the conventional |\include| mechanism
just needs two extra commands at the top of every file
such that all constituent files can be compiled individually.

%%%%%%%%%%%%%%%%%%%%%%%%%%%%%%%%%%%%%%%%%%%%%%%%%%%%%%%%%%%%%%%%%%%%%%%%%%%%%%%%
%\subsection{Feature Suggestions}
%
%The following is a list of features which may be useful for future
%versions of this package:
%%
%\begin{itemize}
%\item
%\ldots
%\end{itemize}

%%%%%%%%%%%%%%%%%%%%%%%%%%%%%%%%%%%%%%%%%%%%%%%%%%%%%%%%%%%%%%%%%%%%%%%%%%%%%%%%
\subsection{Revision History}

%%%%%%%%%%%%%%%%%%%%%%%%%%%%%%%%%%%%%%%%
\paragraph{v2.0:} 2018/12/30

\begin{itemize}
\item
immediate forward processing
\item
added |\childdocby| mechanism
\item
manual restructured
\end{itemize}

%%%%%%%%%%%%%%%%%%%%%%%%%%%%%%%%%%%%%%%%
\paragraph{v1.6:} 2018/01/17

\begin{itemize}
\item
application for development of include files
\item
corrections to manual
\end{itemize}

%%%%%%%%%%%%%%%%%%%%%%%%%%%%%%%%%%%%%%%%
\paragraph{v1.5:} 2017/05/21

\begin{itemize}
\item
more complete structuring introduced
\item
|\childdocof| introduced
\item
|\childdoc| renamed to |\childdocmain|
\item
|\childredirect| renamed to |\childdocforward| and |\childdocforwardprefix|
and functionality expanded
\end{itemize}

%%%%%%%%%%%%%%%%%%%%%%%%%%%%%%%%%%%%%%%%
\paragraph{v1.0:} 2017/04/27

\begin{itemize}
\item
manual and install package
\item
first version published on CTAN
\end{itemize}

%%%%%%%%%%%%%%%%%%%%%%%%%%%%%%%%%%%%%%%%
\paragraph{v0.6:} 2017/04/26

\begin{itemize}
\item
redirection mechanism added
\end{itemize}

%%%%%%%%%%%%%%%%%%%%%%%%%%%%%%%%%%%%%%%%
\paragraph{v0.5:} 2017/04/26

\begin{itemize}
\item
functionality in definition file
\end{itemize}


%%%%%%%%%%%%%%%%%%%%%%%%%%%%%%%%%%%%%%%%%%%%%%%%%%%%%%%%%%%%%%%%%%%%%%%%%%%%%%%%
%%%%%%%%%%%%%%%%%%%%%%%%%%%%%%%%%%%%%%%%%%%%%%%%%%%%%%%%%%%%%%%%%%%%%%%%%%%%%%%%
%%%%%%%%%%%%%%%%%%%%%%%%%%%%%%%%%%%%%%%%%%%%%%%%%%%%%%%%%%%%%%%%%%%%%%%%%%%%%%%%
\appendix

\settowidth\MacroIndent{\rmfamily\scriptsize 000\ }

 \DocInput{childdoc.dtx}

\end{document}
%</driver>
% \fi
%
% %%%%%%%%%%%%%%%%%%%%%%%%%%%%%%%%%%%%%%%%%%%%%%%%%%%%%%%%%%%%%%%%%%%%%%%%%%%%%%
% %%%%%%%%%%%%%%%%%%%%%%%%%%%%%%%%%%%%%%%%%%%%%%%%%%%%%%%%%%%%%%%%%%%%%%%%%%%%%%
% \section{Sample}
%\iffalse
%<*samplemain>
%\fi
%
% The following presents a sample document
% with two chapters, two parts, a title page,
% a compile flag as well as three forwarding files to set the flag.
% It consists of eight |.tex| files:
% \begin{center}
% \begin{tabular}{ll}
% |cdocsamp.tex|&main file\\
% |cdocsch1.tex|&include file for chapter 1\\
% |cdocsch2.tex|&include file for chapter 2\\
% |cdocspt3.tex|&include file for part 3\\
% |cdocspt4.tex|&include file for part 4\\
% |cdocsdrf.tex|&forwarding file for main file in draft mode\\
% |cdocsfi1.tex|&forwarding file for final version of chapter 1\\
% |cdocsfi2.tex|&forwarding file for final version of chapter 2\\
% \end{tabular}
% \end{center}
% Each of the eight files can be compiled directly by the \LaTeX{} compiler.
%
% %%%%%%%%%%%%%%%%%%%%%%%%%%%%%%%%%%%%%%
% \paragraph{Main File.}
%
% The main file is called |cdocsamp.tex|.
%
% Load the \textsf{childdoc} definitions and
% declare the filename for the main document:
%    \begin{macrocode}
\input{childdoc.def}
\childdocmain{}
%    \end{macrocode}

% Optional override for |\version| flag:
%    \begin{macrocode}
%%\ifchilddoc\else\providecommand{\version}{draft}\fi
%    \end{macrocode}

% Define the default values for the |\version| flag
% (|final| for the main file and |draft| for childs):
%    \begin{macrocode}
\ifchilddoc
\providecommand{\version}{draft}
\else
\providecommand{\version}{final}
\fi
%    \end{macrocode}

% Load the standard document class:
%    \begin{macrocode}
\documentclass[12pt]{article}
%    \end{macrocode}

% Start the document body:
%    \begin{macrocode}
\begin{document}
%    \end{macrocode}

% Declare a title page.
% Print title, part of document being processed and version flag:
%    \begin{macrocode}
\addtocounter{page}{-1}
\begin{center}
{\LARGE\bfseries{}childdoc example\par}
\vspace{1cm}
\ifchilddoc
\ifchilddocmanual part\else chapter\fi:
`\childdocname' of `\childdocjob'\par
\else
main document: `\childdocjob'\par
\fi
version: \version\par
\end{center}
\newpage
%    \end{macrocode}

% Manually include selected file,
% otherwise process as usual:
%    \begin{macrocode}
\ifchilddocmanual
\section*{part `\childdocname'}
\input{\childdocname}
\else
%    \end{macrocode}

% Include the two chapters:
%    \begin{macrocode}
\include{cdocsch1}
\include{cdocsch2}
%    \end{macrocode}

% Include the two parts unless only chapters should be displayed:
%    \begin{macrocode}
\ifchilddoc\else
\section{part three}
\input{cdocspt3}
\section{part four}
\input{cdocspt4}
\fi
%    \end{macrocode}

% Process as usual until here:
%    \begin{macrocode}
\fi
%    \end{macrocode}

% End of document body:
%    \begin{macrocode}
\end{document}
%    \end{macrocode}
%\iffalse
%</samplemain>
%\fi
%
% %%%%%%%%%%%%%%%%%%%%%%%%%%%%%%%%%%%%%%
% \paragraph{Chapter Include Files.}
%
% The include files are called |cdocsch1.tex| and |cdocsch2.tex|.
%
%\iffalse
%<*samplechap1|samplechap2>
%\fi

% Optional override for |\version| flag:
%    \begin{macrocode}
%%\providecommand{\version}{final}
%    \end{macrocode}

% Include the main document:
%    \begin{macrocode}
\input{childdoc.def}
\childdocof{cdocsamp}
%    \end{macrocode}

%\iffalse
%</samplechap1|samplechap2>
%\fi
%
%\iffalse
%<*samplechap1>
%\fi
% Some text for chapter 1:
%    \begin{macrocode}
\section{one}
some text in chapter one
%    \end{macrocode}

%\iffalse
%</samplechap1>
%\fi
% Some text for chapter 2:
%\iffalse
%<*samplechap2>
%\fi
%    \begin{macrocode}
\section{two}
more text in chapter two
%    \end{macrocode}

%\iffalse
%</samplechap2>
%\fi
%
% %%%%%%%%%%%%%%%%%%%%%%%%%%%%%%%%%%%%%%
% \paragraph{Part Include Files.}
%
% The include files are called |cdocspt3.tex| and |cdocspt4.tex|.
%
%\iffalse
%<*samplepart3|samplepart4>
%\fi

% Optional override for |\version| flag:
%    \begin{macrocode}
%%\providecommand{\version}{final}
%    \end{macrocode}

% Include the main document:
%    \begin{macrocode}
\input{childdoc.def}
\childdocby{cdocsamp}
%    \end{macrocode}

%\iffalse
%</samplepart3|samplepart4>
%\fi
%
%\iffalse
%<*samplepart3>
%\fi
% Some text for part 3:
%    \begin{macrocode}
some text in part three
%    \end{macrocode}

%\iffalse
%</samplepart3>
%\fi
% Some text for part 4:
%\iffalse
%<*samplepart4>
%\fi
%    \begin{macrocode}
more text in part four
%    \end{macrocode}

%\iffalse
%</samplepart4>
%\fi
%
% %%%%%%%%%%%%%%%%%%%%%%%%%%%%%%%%%%%%%%
% \paragraph{Forwarding for a Complete Draft.}
%
% The following forwarding file |cdocsdrf.tex|
% compiles the main document in draft mode:
%\iffalse
%<*sampledraft>
%\fi
%    \begin{macrocode}
\def\version{draft}
\input{childdoc.def}
\childdocforward{cdocsamp}
%    \end{macrocode}

%\iffalse
%</sampledraft>
%\fi
%
% %%%%%%%%%%%%%%%%%%%%%%%%%%%%%%%%%%%%%%
% \paragraph{Forwarding for Final Version of the Chapters.}
%
% The following forwarding files |cdocsfn1.tex| and |cdocsfn2.tex|
% (with identical content)
% compile the final versions of the child documents
% |cdocsch1.tex| and |cdocsch2.tex|, respectively:
%\iffalse
%<*samplefinal>
%\fi
%    \begin{macrocode}
\def\version{final}
\input{childdoc.def}
\childdocforwardprefix[cdocsamp]{cdocsfn}{cdocsch}
%    \end{macrocode}

%\iffalse
%</samplefinal>
%\fi
%
% %%%%%%%%%%%%%%%%%%%%%%%%%%%%%%%%%%%%%%
% \paragraph{Command Line Processing.}
%
% The following three command lines generate the output files
% |cdocscld|, |cdocscl1| and |cdocscl2|
% which should be identical to
% |cdocsdrf|, |cdocsch1| and |cdocsfn2|, respectively:
% \begin{center}
% \begin{tabular}{l}
% |latex -jobname cdocscld \|\\
% |  "\def\version{draft}\input{childdoc.def}\childdocforward{cdocsamp}"|\\
% |latex -jobname cdocscl1 \|\\
% |  "\input{childdoc.def}\childdocforward[cdocsamp]{cdocsch1}"|\\
% |latex -jobname cdocscl2 \|\\
% |  "\def\version{final}\input{childdoc.def}\childdocforward{cdocsch2}"|
% \end{tabular}
% \end{center}
% Note that the trailing backslash on each first line
% merely continues the input to the second line
% (for convenient cut ant paste).
% Furthermore, the command |latex| can be replaced by any
% of its alternative versions such as |pdflatex|.
%
% %%%%%%%%%%%%%%%%%%%%%%%%%%%%%%%%%%%%%%%%%%%%%%%%%%%%%%%%%%%%%%%%%%%%%%%%%%%%%%
% %%%%%%%%%%%%%%%%%%%%%%%%%%%%%%%%%%%%%%%%%%%%%%%%%%%%%%%%%%%%%%%%%%%%%%%%%%%%%%
% \section{Implementation}
%\iffalse
%<*package>
%\fi
%
% This section describes the definitions file |childdoc.def|.

% The definitions cannot be loaded using |\usepackage| or |\RequirePackage|
% which has a mechanism to prevent loading a style file more than once.
% When loading the definitions by means of |\input|
% multiple instances have to be prevented manually:
%\iffalse
%This code needs to be before the `\ProvidesFile' directive
%which is defined at the beginning of this file.
%Therefore it is also placed there and commented out here.
%</package>
%<*discard>
%\fi
%    \begin{macrocode}
\ifdefined\childdocmain\endinput\fi
%    \end{macrocode}
%\iffalse
%</discard>
%<*package>
%\fi
%
% \macro{\ifchilddoc}
% \macro{\ifchilddocmanual}
% The conditional |\ifchilddoc| tells whether a
% child (true) or main (false) document is being compiled.
% The conditional |\ifchilddocmanual| tells whether
% the |\includeonly| mechanism is used (false) or
% the selection of child files must be performed manually (true).
% The definitions initialise to false:
%    \begin{macrocode}
\newif\ifchilddoc
\newif\ifchilddocmanual
%    \end{macrocode}

% \macro{\childdocname}
% \macro{\childdocjob}
% The macro |\childdocname| stores the name of the main document
% to be compiled. The macro |\childdocjob| stores the name of
% the document on which the \LaTeX{} compiler was originally invoked.
% The content of |\jobname| cannot be compared
% to filenames specified in the source due to different catcodes.
% The following code rescans |\jobname|, stores the result
% in |\childdocname| and saves a copy in |\childdocjob|:
%    \begin{macrocode}
\edef\childdocname{\scantokens\expandafter{\jobname\noexpand}}
\let\childdocjob\childdocname
%    \end{macrocode}

% \macro{\childdocdisable}
% The macro |\childdocdisable| prevents the main file
% from being processed more than once.
% At this stage, the main document command |\childdocmain|
% is assumed to be called once again where it should do nothing.
% Any subsequent call to it should prevent
% a secondary processing of the main document
% It overwrites the forwarding commands
% |\childdocof| and |\childdocforward|
% with empty macros to prevent further inclusions of the main document:
%    \begin{macrocode}
\newcommand{\childdocdisable}
{
  \renewcommand{\childdocmain}[1]{\renewcommand{\childdocmain}[1]{\endinput}}
  \renewcommand{\childdocof}[1]{}
  \renewcommand{\childdocby}[2][]{}
  \renewcommand{\childdocforward}[2][]{}
  \renewcommand{\childdocdisable}{}
}
%    \end{macrocode}

% \macro{\childdocmain}
% The macro |\childdocmain| is to be called at the top of the main file
% with nothing or the main filename (without extension) as argument.
% First, it breaks loops.
% If the argument is not empty and does not match |\childdocname|
% (which is set by the first inclusion of |childdoc.def|),
% |\ifchilddoc| is set to true, |\includeonly| is applied to the child file
% and |\jobname| is set to the main file
% (for proper handling of |.aux| files):
%    \begin{macrocode}
\newcommand{\childdocmain}[1]
{
  \childdocdisable\childdocmain{}
  \if?#1?\else
    \begingroup
      \def\childdoctmp{#1}
      \ifx\childdoctmp\childdocname
        \def\childdoctmp{}
      \else
        \def\childdoctmp
        {
          \childdoctrue
          \includeonly{\childdocname}
          \def\childdocjob{#1}
          \def\jobname{#1}
        }
      \fi
      \expandafter
    \endgroup
    \childdoctmp
  \fi
}
%    \end{macrocode}

% \macro{\childdocof}
% The command |\childdocof| redirects
% compilation to the main file |#1|.
%    \begin{macrocode}
\newcommand{\childdocof}[1]
{
  \childdocdisable
  \childdoctrue
  \includeonly{\childdocname}
  \def\jobname{#1}
  \def\childdocjob{#1}
  \input{#1}
}
%    \end{macrocode}

% \macro{\childdocby}
% The command |\childdocby| ....
%    \begin{macrocode}
\newcommand{\childdocby}[2][]
{
  \childdocdisable
  \childdoctrue
  \childdocmanualtrue
  \if?#1?\else
    \def\jobname{#2}
  \fi
  \def\childdocjob{#2}
  \input{#2}
  \endinput
}
%    \end{macrocode}

% \macro{\childdocforward}
% The command |\childdocforward| redirects
% compilation to the main file or
% (if the optional argument is given) a child file.
% Parameters are set as if the main file
% or a child file starting with |\childdocof| was compiled.
% Then compilation is handed over to the main file:
%    \begin{macrocode}
\newcommand{\childdocforward}[2][]
{
  \begingroup
    \if?#1?
      \def\childdoctmp
      {
        \def\childdocname{#2}
        \def\childdocjob{#2}
        \def\jobname{#2}
        \input{#2}
        \endinput
      }
    \else
      \def\childdoctmp
      {
        \childdocdisable
        \def\childdocname{#2}
        \childdoctrue
        \includeonly{#2}
        \def\childdocjob{#1}
        \def\jobname{#1}
        \input{#1}
        \endinput
      }
    \fi
    \expandafter
  \endgroup
  \childdoctmp
}
%    \end{macrocode}

% \macro{\childdocforwardprefix}
% The command |\childdocforwardprefix| redirects
% compilation to the main or a child file by means of a pattern.
% The prefix |#1| in the current filename is replaced by |#2|
% and the suffix of the current filename is kept
% (it is assumed that the filename does not contain the substring `|~~~|'
% which is used as a delimiter).
% Compilation is handed over to the new file by |\childdocforward|:
%    \begin{macrocode}
\newcommand{\childdocforwardprefix}[3][]
{
  \begingroup
    \def\childdocextract #2##1~~~{\def\childdoctmp{\childdocforward[#1]{#3##1}}}
    \expandafter\childdocextract\childdocname~~~
    \expandafter
  \endgroup
  \childdoctmp
}
%    \end{macrocode}

% \macro{\childdoc}
% The deprecated macro |\childdoc| is a legacy version of |\childdocmain|:
%    \begin{macrocode}
\newcommand{\childdoc}{\childdocmain}
%    \end{macrocode}

% \macro{\childdocredirect}
% The deprecated macro |\childdocredirect| is a legacy version
% of |\childdocforward| and |\childdocforwardprefix|:
%    \begin{macrocode}
\newcommand{\childdocredirect}[2][]
{
  \begingroup
    \if?#1?
      \def\childdoctmp{\childdocforward{#2}}
    \else
      \def\childdoctmp{\childdocforwardprefix{#1}{#2}}
    \fi
    \expandafter
  \endgroup
  \childdoctmp
}
%    \end{macrocode}

%\iffalse
%</package>
%\fi
%
\endinput
\childdocforward{cdocsamp}"|\\
% |latex -jobname cdocscl1 \|\\
% |  "% \iffalse
%
% childdoc.dtx Copyright (C) 2017-2018 Niklas Beisert
%
% This work may be distributed and/or modified under the
% conditions of the LaTeX Project Public License, either version 1.3
% of this license or (at your option) any later version.
% The latest version of this license is in
%   http://www.latex-project.org/lppl.txt
% and version 1.3 or later is part of all distributions of LaTeX
% version 2005/12/01 or later.
%
% This work has the LPPL maintenance status `maintained'.
%
% The Current Maintainer of this work is Niklas Beisert.
%
% This work consists of the files childdoc.dtx and childdoc.ins
% and the derived files childdoc.def and cdocsamp.tex with
% cdocsch1.tex, cdocsch2.tex, cdocsdrf.tex, cdocsfn1.tex, cdocsfn2.tex.
%
%<package>\ifdefined\childdocmain\endinput\fi
%<package>\ProvidesFile{childdoc.def}[2018/12/30 v2.0 child document driver]
%<samplemain>\ProvidesFile{cdocsamp.tex}[2018/12/30 v2.0 sample for childdoc]
%<*driver>
%\ProvidesFile{childdoc.drv}[2018/12/30 v2.0 childdoc reference manual file]
\PassOptionsToClass{10pt,a4paper}{article}
\documentclass{ltxdoc}

\usepackage[margin=35mm]{geometry}
\usepackage{hyperref}
\usepackage{hyperxmp}
\usepackage[usenames]{color}

\hypersetup{colorlinks=true}
\hypersetup{pdfstartview=FitH}
\hypersetup{pdfpagemode=UseNone}
\hypersetup{pdfsource={}}
\hypersetup{pdflang={en-UK}}
\hypersetup{pdfcopyright={Copyright 2017-2018 Niklas Beisert.
  This work may be distributed and/or modified under the
  conditions of the LaTeX Project Public License, either version 1.3
  of this license or (at your option) any later version.}}
\hypersetup{pdflicenseurl={http://www.latex-project.org/lppl.txt}}
\hypersetup{pdfcontactaddress={ETH Zurich, ITP, HIT K,
  Wolfgang-Pauli-Strasse 27}}
\hypersetup{pdfcontactpostcode={8093}}
\hypersetup{pdfcontactcity={Zurich}}
\hypersetup{pdfcontactcountry={Switzerland}}
\hypersetup{pdfcontactemail={nbeisert@itp.phys.ethz.ch}}
\hypersetup{pdfcontacturl={http://people.phys.ethz.ch/\xmptilde nbeisert/}}

\newcommand{\secref}[1]{\hyperref[#1]{section \ref*{#1}}}

\parskip1ex
\parindent0pt
\let\olditemize\itemize
\def\itemize{\olditemize\parskip0pt}

\begin{document}

\title{The \textsf{childdoc} Package}
\hypersetup{pdftitle={The childdoc Package}}
\author{Niklas Beisert\\[2ex]
  Institut f\"ur Theoretische Physik\\
  Eidgen\"ossische Technische Hochschule Z\"urich\\
  Wolfgang-Pauli-Strasse 27, 8093 Z\"urich, Switzerland\\[1ex]
  \href{mailto:nbeisert@itp.phys.ethz.ch}
  {\texttt{nbeisert@itp.phys.ethz.ch}}}
\hypersetup{pdfauthor={Niklas Beisert}}
\hypersetup{pdfsubject={Manual for the LaTeX2e Package childdoc}}
\date{30 December 2018, \textsf{v2.0}}
\maketitle

\begin{abstract}\noindent
\textsf{childdoc} is a \LaTeXe{} package
that enables the direct compilation
of document sections included by |\include|
to individual files.
\end{abstract}

\begingroup
\parskip0ex
\tableofcontents
\endgroup

%%%%%%%%%%%%%%%%%%%%%%%%%%%%%%%%%%%%%%%%%%%%%%%%%%%%%%%%%%%%%%%%%%%%%%%%%%%%%%%%
%%%%%%%%%%%%%%%%%%%%%%%%%%%%%%%%%%%%%%%%%%%%%%%%%%%%%%%%%%%%%%%%%%%%%%%%%%%%%%%%
\section{Introduction}

\LaTeX{} provides a mechanism to structure a large document (such as a book)
into a main file and several child files (containing the chapters)
using the |\include| command.
This mechanism is beneficial for documents
which span hundreds of pages in order to
make the source file(s) more manageable.
Moreover, compilation can be restricted to
selected child files by means of the |\includeonly| command.
The latter feature can be used to reduce the compilation time while editing
(this was significantly more useful in the earlier days of \LaTeX{})
or to generate a smaller document which is easier to navigate.
Another application of |\includeonly| is to generate
documents consisting of selected parts of the complete document.

However, there are a few drawbacks of the plain |\include| mechanism:
\begin{itemize}
\item
The child files cannot be compiled on their own,
they can only be compiled via the main file.
A naive editing environment
(such as a text editor with an option
to have the current file processed by \LaTeX)
may require one to switch to the main file before compiling;
attempting to compile the child file produces errors.
\item
The main file must be modified (each time)
to adjust the |\includeonly| command
to the present needs. This easily leaves the main file in a messy state.
\item
The generated document will always carry the filename
of the main document. This is inconvenient if
several child files are to be compiled and
to be kept for distribution.
\end{itemize}

The present package provides a simple interface
to make child files individually compilable by \LaTeX{}.
Compiling a child file then has the same effect as compiling
the main file with an |\includeonly| command
to select the appropriate child.
Moreover the generated document will carry the name of the child
rather than the main file.
This resolves all three above issues.

This feature is meant to make the editing of books,
thesis documents and lecture notes somewhat more convenient.
However, the package can also be used efficiently for
composing a series of documents (such as exercise sheets)
which are typically distributed individually.
It then assists the author in generating the individual documents
(potentially in different versions)
as well as a document containing the collected series.
Another application is in developing style files
or other kinds of included material
where compilation of the style file could redirect
to a sample or test file.

%%%%%%%%%%%%%%%%%%%%%%%%%%%%%%%%%%%%%%%%%%%%%%%%%%%%%%%%%%%%%%%%%%%%%%%%%%%%%%%%
%%%%%%%%%%%%%%%%%%%%%%%%%%%%%%%%%%%%%%%%%%%%%%%%%%%%%%%%%%%%%%%%%%%%%%%%%%%%%%%%
\section{Usage}

First of all, the package \textsf{childdoc} is \emph{not} a standard
\LaTeXe{} |.sty| style file! Therefore it needs to be invoked in
a non-standard way.

%%%%%%%%%%%%%%%%%%%%%%%%%%%%%%%%%%%%%%%%%%%%%%%%%%%%%%%%%%%%%%%%%%%%%%%%%%%%%%%%
\subsection{Included Files}
\label{sec:include}

%%%%%%%%%%%%%%%%%%%%%%%%%%%%%%%%%%%%%%%%
\DescribeMacro{\childdocmain}
To use the package, add the commands
\begin{center}
\begin{tabular}{l}
|\input{childdoc.def}|\\
|\childdocmain{}|\\
\end{tabular}
\end{center}
at the very top of the main \LaTeX{} file,
in particular \emph{before} the |\documentclass| statement!
The argument of |\childdocmain| should be left empty
(but it must be present).

%%%%%%%%%%%%%%%%%%%%%%%%%%%%%%%%%%%%%%%%
\DescribeMacro{\childdocof}
Furthermore, add the commands
\begin{center}
\begin{tabular}{l}
|\input{childdoc.def}|\\
|\childdocof{|\textit{main}|}|\\
\end{tabular}
\end{center}
at the top of every child file \textit{child}
which is included by |\include{|\textit{child}|}|
from within the main file
(or at least for those files to be compiled individually).
The argument \textit{main} must be the filename of the main file.

There are a couple of
considerations in setting up the main and child documents:

%%%%%%%%%%%%%%%%%%%%%%%%%%%%%%%%%%%%%%%%
\paragraph{Restrictions.}

Please note the following restrictions:
\begin{itemize}
\item
|\childdocmain| must be called with one argument \textit{main}
to ensure compatibility with earlier version of the package.
It must either be empty (|\childdocmain{}|)
or precisely match the filename of the main file in which it is specified.
See \secref{sec:detection} for further information.
\item
The filename \textit{main} must be specified without the |.tex| extension.
\item
The filename \textit{main} is case sensitive
(even in case-insensitive file systems)
due to internal string comparison.
\item
The argument \textit{main} should be fully expanded, it cannot be a macro.
\item
Subdirectories and special characters should be avoided in filenames.
\item
The command |\childdocmain{|\textit{main}|}| must be followed by a whitespace.
It should not be followed immediately by another command
or by a comment mark `|%|'.
This is because the \TeX{} parser reads the token immediately following
the argument of |\childdocmain| and puts it
at the beginning of every child section;
however, a white\-space is ignored.
\end{itemize}

%%%%%%%%%%%%%%%%%%%%%%%%%%%%%%%%%%%%%%%%
\paragraph{Content of Main File.}

It is advisable to place all content in the child files included by |\include|.
Any output contained in the main file will appear in all child documents
unless suppressed manually;
it cannot be suppressed automatically by the |\includeonly| directive
and thus should normally be avoided.
A method to include some content in the main file
by means of conditional processing is described in \secref{sec:conditional}.

%%%%%%%%%%%%%%%%%%%%%%%%%%%%%%%%%%%%%%%%
\paragraph{Page Numbering.}

When only a part of the document is compiled,
the appropriate numbering of pages
(as well as other status parameters)
is determined from the |.aux| files.
The latter contain information from previous passes.
However this information needs to propagate through
all intermediate child documents.
Therefore the page numbering in child documents may well
be inconsistent until the complete document is compiled at least once.

A useful (if unconventional) way to always ensure a consistent
page numbering is to restart the numbering in each child document
and denote the pages by `\textit{child}|.|\textit{page}'
where \textit{child} represents the chapter/section number of the child file.
This can be achieved by the command
|\numberwithin{page}{|\textit{child}|}|
of the \textsf{amsmath} package
where \textit{child} can be |chapter| or |section|
depending on the chosen structuring.
Alternatively, one can modify the macro |\thepage| appropriately
and reset the counter |page| at the start of each child file.

%%%%%%%%%%%%%%%%%%%%%%%%%%%%%%%%%%%%%%%%%%%%%%%%%%%%%%%%%%%%%%%%%%%%%%%%%%%%%%%%
\subsection{Conditional Processing}
\label{sec:conditional}

The package provides a mechanism to compile different versions
of a document. To customise the versions further some conditional processing
can come in handy to distinguish which version is being compiled.
The package provides two macros to describe the compilation context:

%%%%%%%%%%%%%%%%%%%%%%%%%%%%%%%%%%%%%%%%
\DescribeMacro{\ifchilddoc}
The conditional |\ifchilddoc| distinguishes between the compilation of
child documents and the main document:
%
\begin{center}
|\ifchilddoc |\textit{child-code}| |[|\||else |\textit{main-code}]| \||fi|
\end{center}

%%%%%%%%%%%%%%%%%%%%%%%%%%%%%%%%%%%%%%%%
\DescribeMacro{\childdocname}
\DescribeMacro{\childdocjob}
The macro |\childdocname| contains the filename (without extension)
of the main or child file being processed.
Note that |\childdocjob| will always contain the name of the main file.

%%%%%%%%%%%%%%%%%%%%%%%%%%%%%%%%%%%%%%%%
\paragraph{Title Page.}

Conditional processing can be used to include a title or banner page
in the main document when proper precautions are taken.
Importantly, the code in the main file should ensure that the page counter
(as well as other status parameters which are stored in the |.aux| files)
takes the same value after the conditional processing.
Otherwise the page numbers may take divergent values
depending on which part is compiled.

For example, a title page could be declared by:
%
\begin{center}
\begin{tabular}{l}
|\ifchilddoc\||else|\\
|\addtocounter{page}{-1}|\\
\textit{code for title page}\\
|\newpage|\\
|\||fi|
\end{tabular}
\end{center}
%
A banner page for the child documents can be generated by:
%
\begin{center}
\begin{tabular}{l}
|\ifchilddoc|\\
|\addtocounter{page}{-1}|\\
\textit{code for banner page}\\
|\newpage|\\
|\||fi|
\end{tabular}
\end{center}
%
Here one could write a message such as:
\begin{center}
|This is the part \childdocname{} of \childdocjob{}.|
\end{center}

%%%%%%%%%%%%%%%%%%%%%%%%%%%%%%%%%%%%%%%%%%%%%%%%%%%%%%%%%%%%%%%%%%%%%%%%%%%%%%%%
\subsection{Flags}
\label{sec:flags}

The package makes it easy to generate different versions
of the main or child documents.
To this end compilation flags can be defined
and assigned different default values.
They will be particularly useful in conjunction
with the forwarding mechanism described in \secref{sec:forward}.

For example, it may be useful to have a flag |\version|
which can be set to |draft| or |final|.
The document source will contain some conditional code
depending on the value of |\version|.
Suppose further, the flag should default to |final| for the main file
and to |draft| for child files
which is a natural assignment for editing the document.
This is achieved by placing the following code
in the preamble of the main document
(below the |\childdocmain| directive):
%
\begin{center}
\begin{tabular}{l}
|\ifchilddoc|\\
|\providecommand{\version}{draft}|\\
|\||else|\\
|\providecommand{\version}{final}|\\
|\||fi|
\end{tabular}
\end{center}
%
The definition by |\providecommand| makes sure
that previous definitions are not overwritten.
Further statements |\providecommand{\version}{...}|
can thus be added before the above code to override it.

For the main file, one might add a line
(between |\childdocmain| and the above block)
%
\begin{center}
|%\ifchilddoc\||else\providecommand{\version}{draft}\||fi|
\end{center}
%
which can be uncommented to produce a draft version.
Likewise one can add a line to the very top of a child file
(above the |\childdocof{|\textit{main}|}| directive)
%
\begin{center}
|%\providecommand{\version}{final}|
\end{center}
%
which can be uncommented to produce the final version of this child document.

%%%%%%%%%%%%%%%%%%%%%%%%%%%%%%%%%%%%%%%%%%%%%%%%%%%%%%%%%%%%%%%%%%%%%%%%%%%%%%%%
\subsection{Forwarding}
\label{sec:forward}

Different versions of the main or child documents
using compilation flags as described in \secref{sec:flags}
can be (permanently) stored in different files
for convenient compilation, viewing and distribution.
To this end, the package defines a command
to pass on compilation to a different file:

%%%%%%%%%%%%%%%%%%%%%%%%%%%%%%%%%%%%%%%%
\DescribeMacro{\childdocforward}
The command |\childdocforward| redirects processing to
another source file:
%
\begin{center}
\begin{tabular}{l}
|\input{childdoc.def}|\\
|\childdocforward[|\textit{main}|]{|\textit{dest}|}|\\
\end{tabular}
\end{center}
%
The argument \textit{dest} is the destination file
(without extension).
It should be the main file or one of the child files.
Note that further \textsf{childdoc} directives
such as |\childdocof| and |\childdocforward|
in the indicated file will be processed in this form.
The optional argument \textit{main}
passes on directly to the main file \textit{main}
while pretending to compile the child \textit{dest}.
This form behaves as if \textit{dest}
issues |\childdocof{|\textit{main}|}| right away,
and no further \textsf{childdoc} directives will be processed.

%%%%%%%%%%%%%%%%%%%%%%%%%%%%%%%%%%%%%%%%
\DescribeMacro{\...prefix}
In the alternative form |\childdocforwardprefix|,
%
\begin{center}
\begin{tabular}{l}
|\input{childdoc.def}|\\
|\childdocforwardprefix[|\textit{main}|]{|\textit{prefix}|}{|\textit{dest}|}|
\end{tabular}
\end{center}
%
the destination file is determined by a pattern
depending on the current file:
To make this work, the current file must be called
`{\textit{prefix}\hspace{0.2em}\textit{suffix}}'
with \textit{prefix} matching precisely the argument.
Processing is then passed on to the file
`{\textit{dest}\hspace{0.2em}\textit{suffix}}'.
Surely, the same effect is achieved by
directly specifying the
argument `{\textit{dest}\hspace{0.2em}\textit{suffix}}'
in the first form.
However, that requires to set up a different file
for each child. With the alternative form of the command
all these files can have exactly the same content
which simplifies setting them up and maintaining them.

For example, the following file |draft.tex|
with a compilation flag |\version| as described in \secref{sec:flags}
compiles the main document as a draft:
%
\begin{center}
\begin{tabular}{l}
|\def\version{draft}|\\
|\input{childdoc.def}|\\
|\childdocforward{|\textit{main}|}|
\end{tabular}
\end{center}
%
Likewise, the following files |final|\textit{nn}|.tex|
compile the final version of the child document
|child|\textit{nn}|.tex|:
%
\begin{center}
\begin{tabular}{l}
|\def\version{final}|\\
|\input{childdoc.def}|\\
|\childdocforwardprefix{final}{child}|
\end{tabular}
\end{center}
%

Note that when several versions of a main file and/or of each child file
are to be generated, it may be convenient to set up a |Makefile| or
shell script to automatise the process.

%%%%%%%%%%%%%%%%%%%%%%%%%%%%%%%%%%%%%%%%%%%%%%%%%%%%%%%%%%%%%%%%%%%%%%%%%%%%%%%%
\subsection{Command Line Processing}
\label{sec:commandline}

The effect of redirection files can also be achieved by invoking
the \LaTeX{} compiler with a more elaborate command line.
Most conveniently this should be done as part
of a shell script or a |Makefile|.

When using \textsf{childdoc} in the main file, the following
command lines effectively perform a redirection
(note that depending on the shell being used,
backslashes may have to be doubled: `|\|' $\to$ `|\\|'):
%
\begin{center}
|... -jobname "|\textit{target}|" |\\|"|[\textit{flags}]%
|\input{childdoc.def}\childdocforward[|\textit{main}|]{|\textit{dest}|}"|
\end{center}
%
Here \textit{target} is the name of the output file,
\textit{main} is the name of the main file
and \textit{dest} is the name of the main or child file to be processed
(all filenames without extensions).
The optional argument \textit{main} can be omitted
if \textit{main} matches \textit{dest}.
Optionally, compilation \textit{flags} can be defined via |\def| commands.
This command line makes the \TeX{} engine believe
it is compiling the file \textit{target}
whose content is specified as the latter parameter.
The provided code then forwards the processing to
\textit{main} or \textit{dest} as described in \secref{sec:forward}.

%%%%%%%%%%%%%%%%%%%%%%%%%%%%%%%%%%%%%%%%%%%%%%%%%%%%%%%%%%%%%%%%%%%%%%%%%%%%%%%%
\subsection{Include by Input}
\label{sec:input}

Including child documents by |\include| has some restrictions by design.
Most notably, the content of a child document always occupies
its own set of pages; pages cannot be shared between child documents.
Usually, this behaviour makes perfect sense
because each child document contain an essential part of the document.
However, in some situations it may be desirable to compose
a document from a collection of parts
without having mandatory page breaks between then.
For this case, the package
provides a mechanism to include parts
by |\input| which can also be processed individually.
However, by construction this mechanism
requires manual handling of the content to be output.

%%%%%%%%%%%%%%%%%%%%%%%%%%%%%%%%%%%%%%%%
\DescribeMacro{\ifchilddocmanual}
The main file should be prepared as usual, see \secref{sec:include}.
However, the document body must make a distinction
between processing of an individual part and of the main document, e.g.:
%
\begin{center}
\begin{tabular}{l}
|\ifchilddocmanual|\\
|\input{\childdocname}|\\
|\||else|\\
\textit{document body with }|\input{|\textit{part}|}|\\
|\||fi|
\end{tabular}
\end{center}
%
The conditional |\ifchilddocmanual| is true whenever
a part to be included by |\input| is being compiled,
and the name of the part is stored in |\childdocname|.

%%%%%%%%%%%%%%%%%%%%%%%%%%%%%%%%%%%%%%%%
\DescribeMacro{\childdocby}
Each part to be included by |\input| should start with:
%
\begin{center}
\begin{tabular}{l}
|\input{childdoc.def}|\\
|\childdocby{|\textit{main}|}|\\
\end{tabular}
\end{center}
%
The directive |\childdocby| is similar to |\childdocof|
described in \secref{sec:include},
but the subsequent selection of content must be done manually.
To that end, both |\ifchilddoc| and |\ifchilddocmanual|
will be true upon processing of a part,
and the name of the part is stored in |\childdocname|.
Note that |\jobname| will be set to the filename of the current part
so that each part receives an individual |.aux| file
that does not interfere with the |.aux| file(s) of the main document.
This behaviour can be altered by the alternative form
|\childdocby[*]{|\textit{main}|}| (with a non-empty optional argument)
which uses the |.aux| file of the main document
by setting |\jobname| to \textit{main}.

%%%%%%%%%%%%%%%%%%%%%%%%%%%%%%%%%%%%%%%%%%%%%%%%%%%%%%%%%%%%%%%%%%%%%%%%%%%%%%%%
\subsection{Driver Development}
\label{sec:driver}

The \textsf{childdoc} mechanism can also be use for the development
of definition files such as \LaTeX{} styles or classes.
This case differs from the above setup with multiple parts
included by |\include| in that no |\includeonly| should be invoked.
This can be achieved by starting the include file
(before |\ProvidesPackage|) with:
%
\begin{center}
\begin{tabular}{l}
|\input{childdoc.def}|\\
|\childdocforward{|\textit{main}|}|\\
\end{tabular}
\end{center}
%
or alternatively with:
%
\begin{center}
\begin{tabular}{l}
|\input{childdoc.def}|\\
|\childdocby{|\textit{main}|}|\\
\end{tabular}
\end{center}
%
Both forms have slightly different effects as described above.
The main file is prepared as usual, see \secref{sec:include}.

%%%%%%%%%%%%%%%%%%%%%%%%%%%%%%%%%%%%%%%%%%%%%%%%%%%%%%%%%%%%%%%%%%%%%%%%%%%%%%%%
\subsection{Legacy Detection}
\label{sec:detection}

The directive |\childdocmain| in the main file can detect
whether the complete document or merely a child is to be compiled
even without using the directive |\childdocof|.
This method is deprecated because it is less robust
and there is no compelling reason to use it;
it is merely provided for backward compatibility
and it may be removed in future versions.

If the detection mechanism is to be used,
it is mandatory to correctly specify
the filename of the main file as the argument of |\childdocmain|:
%
\begin{center}
\begin{tabular}{l}
|\input{childdoc.def}|\\
|\childdocmain{|\textit{main}|}|\\
\end{tabular}
\end{center}
%
If |\jobname| does not match the argument \textit{main} of |\childdocmain|,
it is assumed that |\jobname| points to the child file to be compiled.
When using |\childdocmain| with the main file specified as argument,
it suffices to start a child file
with just |\input{|\textit{main}|}|
without loading of the package and using |\childdocof|.
If instead all processing is done
with the appropriate \textsf{childdoc} directives,
the argument of \textit{main} of |\childdocmain| can be empty.

An alternative version of the command line processing described
in \secref{sec:commandline} using the detection mechanism reads:
%
\begin{center}
|... -jobname "|\textit{target}|" "|[\textit{flags}]%
[|\def\jobname{|\textit{dest}|}|]|\input{|\textit{main}|}"|
\end{center}

%%%%%%%%%%%%%%%%%%%%%%%%%%%%%%%%%%%%%%%%%%%%%%%%%%%%%%%%%%%%%%%%%%%%%%%%%%%%%%%%
\subsection{Manual Code}
\label{sec:manual}

In case one cannot be certain whether the definitions file |childdoc.def|
is installed on the target \TeX{} distribution
and one prefers not to ship it,
it is conceivable to paste a few relevant commands into the sources.

To that end, drop all statements |\input{childdoc.def}|
and perform the replacements as outlined below.
Instead of |\childdocmain{|\textit{main}|}| add the following code
to the top of the main file:
%
\begin{center}
\begin{tabular}{l}
|\||ifdefined\childdocname\endinput\||fi\newif\ifchilddoc|\\
|\edef\childdocname{\scantokens\expandafter{\jobname\noexpand}}|\\
|\def\childdocmain{|\textit{main}|}\||ifx\childdocmain\childdocname\||else|\\
|\childdoctrue\includeonly{\childdocname}\let\jobname\childdocmain\||fi|\\
\end{tabular}
\end{center}
%
Instead of |\childdocof{|\textit{main}|}| just include the main file
at the top of each child file:
%
\begin{center}
|\input{|\textit{main}|}|
\end{center}
%
A simple redirection |\childdocforward{|\textit{dest}|}| is achieved by:
%
\begin{center}
|\def\jobname{|\textit{dest}|}\input{\jobname}|
\end{center}
%
The redirection with prefix
|\childdocforwardprefix[|\textit{prefix}|]{|\textit{dest}|}|
is accomplished by:
%
\begin{center}
\begin{tabular}{l}
|{\edef\jobname{\scantokens\expandafter{\jobname\noexpand}}|\\
|\def\redirectjob |\textit{prefix}|#1~~~{\gdef\jobname{|\textit{dest}|#1}}|\\
|\expandafter\redirectjob\jobname~~~}\input{\jobname}|
\end{tabular}
\end{center}

In an alternative approach,
child documents can be compiled by a specific command line
without additional code or specific definitions:
%
\begin{center}
|... -jobname "|\textit{target}|" "|[\textit{flags}]%
|\includeonly{|\textit{dest}|}\input{|\textit{main}|}"|
\end{center}
%

%%%%%%%%%%%%%%%%%%%%%%%%%%%%%%%%%%%%%%%%%%%%%%%%%%%%%%%%%%%%%%%%%%%%%%%%%%%%%%%%
%%%%%%%%%%%%%%%%%%%%%%%%%%%%%%%%%%%%%%%%%%%%%%%%%%%%%%%%%%%%%%%%%%%%%%%%%%%%%%%%
\section{Information}

%%%%%%%%%%%%%%%%%%%%%%%%%%%%%%%%%%%%%%%%%%%%%%%%%%%%%%%%%%%%%%%%%%%%%%%%%%%%%%%%
\subsection{Copyright}

Copyright \copyright{} 2017--2018 Niklas Beisert

This work may be distributed and/or modified under the
conditions of the \LaTeX{} Project Public License, either version 1.3
of this license or (at your option) any later version.
The latest version of this license is in
  \url{http://www.latex-project.org/lppl.txt}
and version 1.3 or later is part of all distributions of \LaTeX{}
version 2005/12/01 or later.

This work has the LPPL maintenance status `maintained'.

The Current Maintainer of this work is Niklas Beisert.

This work consists of the files |README.txt|, |childdoc.ins| and |childdoc.dtx|
as well as the derived files |childdoc.def|, |cdocsamp.tex|
with |cdocsch1.tex|, |cdocsch2.tex|, |cdocspt3.tex|, |cdocspt4.tex|,
|cdocsdrf.tex|, |cdocsfn1.tex|, |cdocsfn2.tex|
as well as |childdoc.pdf|.

%%%%%%%%%%%%%%%%%%%%%%%%%%%%%%%%%%%%%%%%%%%%%%%%%%%%%%%%%%%%%%%%%%%%%%%%%%%%%%%%
\subsection{Files and Installation}

The package consists of the files:
%
\begin{center}
\begin{tabular}{ll}
    |README.txt|   & readme file \\
    |childdoc.ins| & installation file \\
    |childdoc.dtx| & source file \\
    |childdoc.def| & definition file \\
    |cdocsamp.tex| & sample main file \\
    |cdocsch1.tex| & sample include file \\
    |cdocsch2.tex| & sample include file \\
    |cdocspt3.tex| & sample part file \\
    |cdocspt4.tex| & sample part file \\
    |cdocsdrf.tex| & sample redirection file \\
    |cdocsfn1.tex| & sample redirection file \\
    |cdocsfn2.tex| & sample redirection file \\
    |childdoc.pdf| & manual
\end{tabular}
\end{center}
%
The distribution consists of the files
|README.txt|, |childdoc.ins| and |childdoc.dtx|.
%
\begin{itemize}
\item
Run (pdf)\LaTeX{} on |childdoc.dtx|
to compile the manual |childdoc.pdf| (this file).
\item
Run \LaTeX{} on |childdoc.ins| to create the definitions file |childdoc.def|
and the sample |cdocsamp.tex| with include files
|cdocsch1.tex|, |cdocsch2.tex|, |cdocspt3.tex|, |cdocspt4.tex|,
|cdocsdrf.tex|, |cdocsfn1.tex|, |cdocsfn2.tex|.
Then copy the file |childdoc.def| to an appropriate directory of your \LaTeX{}
distribution, e.g.\ \textit{texmf-root}|/tex/latex/childdoc|.
\end{itemize}

%%%%%%%%%%%%%%%%%%%%%%%%%%%%%%%%%%%%%%%%%%%%%%%%%%%%%%%%%%%%%%%%%%%%%%%%%%%%%%%%
\subsection{Related CTAN Packages}

There are several other packages which offer a similar functionality:
%
\begin{itemize}
\item
The packages
\href{http://ctan.org/pkg/docmute}{\textsf{docmute}},
\href{http://ctan.org/pkg/includex}{\textsf{includex}} and
\href{http://ctan.org/pkg/standalone}{\textsf{standalone}}
provide commands to include only the document body of
a child file thus allowing both files to be compiled individually.
\item
The packages \href{http://ctan.org/pkg/subdocs}{\textsf{subdocs}}
and \href{http://ctan.org/pkg/subfiles}{\textsf{subfiles}}
provide structures in which the main and child documents can be
encapsulated and allowing them to be compiled individually.
The inclusion mechanism is different from the conventional |\include|.
\item
The package \href{http://ctan.org/pkg/combine}{\textsf{combine}}
is an elaborate solution to combine several documents into one.
\end{itemize}
%
See also the CTAN topic \href{http://ctan.org/topic/subdocs}{\textsf{subdocs}}
for further related packages.
The present package differs from the above solutions in that
a document structure constructed with the conventional |\include| mechanism
just needs two extra commands at the top of every file
such that all constituent files can be compiled individually.

%%%%%%%%%%%%%%%%%%%%%%%%%%%%%%%%%%%%%%%%%%%%%%%%%%%%%%%%%%%%%%%%%%%%%%%%%%%%%%%%
%\subsection{Feature Suggestions}
%
%The following is a list of features which may be useful for future
%versions of this package:
%%
%\begin{itemize}
%\item
%\ldots
%\end{itemize}

%%%%%%%%%%%%%%%%%%%%%%%%%%%%%%%%%%%%%%%%%%%%%%%%%%%%%%%%%%%%%%%%%%%%%%%%%%%%%%%%
\subsection{Revision History}

%%%%%%%%%%%%%%%%%%%%%%%%%%%%%%%%%%%%%%%%
\paragraph{v2.0:} 2018/12/30

\begin{itemize}
\item
immediate forward processing
\item
added |\childdocby| mechanism
\item
manual restructured
\end{itemize}

%%%%%%%%%%%%%%%%%%%%%%%%%%%%%%%%%%%%%%%%
\paragraph{v1.6:} 2018/01/17

\begin{itemize}
\item
application for development of include files
\item
corrections to manual
\end{itemize}

%%%%%%%%%%%%%%%%%%%%%%%%%%%%%%%%%%%%%%%%
\paragraph{v1.5:} 2017/05/21

\begin{itemize}
\item
more complete structuring introduced
\item
|\childdocof| introduced
\item
|\childdoc| renamed to |\childdocmain|
\item
|\childredirect| renamed to |\childdocforward| and |\childdocforwardprefix|
and functionality expanded
\end{itemize}

%%%%%%%%%%%%%%%%%%%%%%%%%%%%%%%%%%%%%%%%
\paragraph{v1.0:} 2017/04/27

\begin{itemize}
\item
manual and install package
\item
first version published on CTAN
\end{itemize}

%%%%%%%%%%%%%%%%%%%%%%%%%%%%%%%%%%%%%%%%
\paragraph{v0.6:} 2017/04/26

\begin{itemize}
\item
redirection mechanism added
\end{itemize}

%%%%%%%%%%%%%%%%%%%%%%%%%%%%%%%%%%%%%%%%
\paragraph{v0.5:} 2017/04/26

\begin{itemize}
\item
functionality in definition file
\end{itemize}


%%%%%%%%%%%%%%%%%%%%%%%%%%%%%%%%%%%%%%%%%%%%%%%%%%%%%%%%%%%%%%%%%%%%%%%%%%%%%%%%
%%%%%%%%%%%%%%%%%%%%%%%%%%%%%%%%%%%%%%%%%%%%%%%%%%%%%%%%%%%%%%%%%%%%%%%%%%%%%%%%
%%%%%%%%%%%%%%%%%%%%%%%%%%%%%%%%%%%%%%%%%%%%%%%%%%%%%%%%%%%%%%%%%%%%%%%%%%%%%%%%
\appendix

\settowidth\MacroIndent{\rmfamily\scriptsize 000\ }

 \DocInput{childdoc.dtx}

\end{document}
%</driver>
% \fi
%
% %%%%%%%%%%%%%%%%%%%%%%%%%%%%%%%%%%%%%%%%%%%%%%%%%%%%%%%%%%%%%%%%%%%%%%%%%%%%%%
% %%%%%%%%%%%%%%%%%%%%%%%%%%%%%%%%%%%%%%%%%%%%%%%%%%%%%%%%%%%%%%%%%%%%%%%%%%%%%%
% \section{Sample}
%\iffalse
%<*samplemain>
%\fi
%
% The following presents a sample document
% with two chapters, two parts, a title page,
% a compile flag as well as three forwarding files to set the flag.
% It consists of eight |.tex| files:
% \begin{center}
% \begin{tabular}{ll}
% |cdocsamp.tex|&main file\\
% |cdocsch1.tex|&include file for chapter 1\\
% |cdocsch2.tex|&include file for chapter 2\\
% |cdocspt3.tex|&include file for part 3\\
% |cdocspt4.tex|&include file for part 4\\
% |cdocsdrf.tex|&forwarding file for main file in draft mode\\
% |cdocsfi1.tex|&forwarding file for final version of chapter 1\\
% |cdocsfi2.tex|&forwarding file for final version of chapter 2\\
% \end{tabular}
% \end{center}
% Each of the eight files can be compiled directly by the \LaTeX{} compiler.
%
% %%%%%%%%%%%%%%%%%%%%%%%%%%%%%%%%%%%%%%
% \paragraph{Main File.}
%
% The main file is called |cdocsamp.tex|.
%
% Load the \textsf{childdoc} definitions and
% declare the filename for the main document:
%    \begin{macrocode}
\input{childdoc.def}
\childdocmain{}
%    \end{macrocode}

% Optional override for |\version| flag:
%    \begin{macrocode}
%%\ifchilddoc\else\providecommand{\version}{draft}\fi
%    \end{macrocode}

% Define the default values for the |\version| flag
% (|final| for the main file and |draft| for childs):
%    \begin{macrocode}
\ifchilddoc
\providecommand{\version}{draft}
\else
\providecommand{\version}{final}
\fi
%    \end{macrocode}

% Load the standard document class:
%    \begin{macrocode}
\documentclass[12pt]{article}
%    \end{macrocode}

% Start the document body:
%    \begin{macrocode}
\begin{document}
%    \end{macrocode}

% Declare a title page.
% Print title, part of document being processed and version flag:
%    \begin{macrocode}
\addtocounter{page}{-1}
\begin{center}
{\LARGE\bfseries{}childdoc example\par}
\vspace{1cm}
\ifchilddoc
\ifchilddocmanual part\else chapter\fi:
`\childdocname' of `\childdocjob'\par
\else
main document: `\childdocjob'\par
\fi
version: \version\par
\end{center}
\newpage
%    \end{macrocode}

% Manually include selected file,
% otherwise process as usual:
%    \begin{macrocode}
\ifchilddocmanual
\section*{part `\childdocname'}
\input{\childdocname}
\else
%    \end{macrocode}

% Include the two chapters:
%    \begin{macrocode}
\include{cdocsch1}
\include{cdocsch2}
%    \end{macrocode}

% Include the two parts unless only chapters should be displayed:
%    \begin{macrocode}
\ifchilddoc\else
\section{part three}
\input{cdocspt3}
\section{part four}
\input{cdocspt4}
\fi
%    \end{macrocode}

% Process as usual until here:
%    \begin{macrocode}
\fi
%    \end{macrocode}

% End of document body:
%    \begin{macrocode}
\end{document}
%    \end{macrocode}
%\iffalse
%</samplemain>
%\fi
%
% %%%%%%%%%%%%%%%%%%%%%%%%%%%%%%%%%%%%%%
% \paragraph{Chapter Include Files.}
%
% The include files are called |cdocsch1.tex| and |cdocsch2.tex|.
%
%\iffalse
%<*samplechap1|samplechap2>
%\fi

% Optional override for |\version| flag:
%    \begin{macrocode}
%%\providecommand{\version}{final}
%    \end{macrocode}

% Include the main document:
%    \begin{macrocode}
\input{childdoc.def}
\childdocof{cdocsamp}
%    \end{macrocode}

%\iffalse
%</samplechap1|samplechap2>
%\fi
%
%\iffalse
%<*samplechap1>
%\fi
% Some text for chapter 1:
%    \begin{macrocode}
\section{one}
some text in chapter one
%    \end{macrocode}

%\iffalse
%</samplechap1>
%\fi
% Some text for chapter 2:
%\iffalse
%<*samplechap2>
%\fi
%    \begin{macrocode}
\section{two}
more text in chapter two
%    \end{macrocode}

%\iffalse
%</samplechap2>
%\fi
%
% %%%%%%%%%%%%%%%%%%%%%%%%%%%%%%%%%%%%%%
% \paragraph{Part Include Files.}
%
% The include files are called |cdocspt3.tex| and |cdocspt4.tex|.
%
%\iffalse
%<*samplepart3|samplepart4>
%\fi

% Optional override for |\version| flag:
%    \begin{macrocode}
%%\providecommand{\version}{final}
%    \end{macrocode}

% Include the main document:
%    \begin{macrocode}
\input{childdoc.def}
\childdocby{cdocsamp}
%    \end{macrocode}

%\iffalse
%</samplepart3|samplepart4>
%\fi
%
%\iffalse
%<*samplepart3>
%\fi
% Some text for part 3:
%    \begin{macrocode}
some text in part three
%    \end{macrocode}

%\iffalse
%</samplepart3>
%\fi
% Some text for part 4:
%\iffalse
%<*samplepart4>
%\fi
%    \begin{macrocode}
more text in part four
%    \end{macrocode}

%\iffalse
%</samplepart4>
%\fi
%
% %%%%%%%%%%%%%%%%%%%%%%%%%%%%%%%%%%%%%%
% \paragraph{Forwarding for a Complete Draft.}
%
% The following forwarding file |cdocsdrf.tex|
% compiles the main document in draft mode:
%\iffalse
%<*sampledraft>
%\fi
%    \begin{macrocode}
\def\version{draft}
\input{childdoc.def}
\childdocforward{cdocsamp}
%    \end{macrocode}

%\iffalse
%</sampledraft>
%\fi
%
% %%%%%%%%%%%%%%%%%%%%%%%%%%%%%%%%%%%%%%
% \paragraph{Forwarding for Final Version of the Chapters.}
%
% The following forwarding files |cdocsfn1.tex| and |cdocsfn2.tex|
% (with identical content)
% compile the final versions of the child documents
% |cdocsch1.tex| and |cdocsch2.tex|, respectively:
%\iffalse
%<*samplefinal>
%\fi
%    \begin{macrocode}
\def\version{final}
\input{childdoc.def}
\childdocforwardprefix[cdocsamp]{cdocsfn}{cdocsch}
%    \end{macrocode}

%\iffalse
%</samplefinal>
%\fi
%
% %%%%%%%%%%%%%%%%%%%%%%%%%%%%%%%%%%%%%%
% \paragraph{Command Line Processing.}
%
% The following three command lines generate the output files
% |cdocscld|, |cdocscl1| and |cdocscl2|
% which should be identical to
% |cdocsdrf|, |cdocsch1| and |cdocsfn2|, respectively:
% \begin{center}
% \begin{tabular}{l}
% |latex -jobname cdocscld \|\\
% |  "\def\version{draft}\input{childdoc.def}\childdocforward{cdocsamp}"|\\
% |latex -jobname cdocscl1 \|\\
% |  "\input{childdoc.def}\childdocforward[cdocsamp]{cdocsch1}"|\\
% |latex -jobname cdocscl2 \|\\
% |  "\def\version{final}\input{childdoc.def}\childdocforward{cdocsch2}"|
% \end{tabular}
% \end{center}
% Note that the trailing backslash on each first line
% merely continues the input to the second line
% (for convenient cut ant paste).
% Furthermore, the command |latex| can be replaced by any
% of its alternative versions such as |pdflatex|.
%
% %%%%%%%%%%%%%%%%%%%%%%%%%%%%%%%%%%%%%%%%%%%%%%%%%%%%%%%%%%%%%%%%%%%%%%%%%%%%%%
% %%%%%%%%%%%%%%%%%%%%%%%%%%%%%%%%%%%%%%%%%%%%%%%%%%%%%%%%%%%%%%%%%%%%%%%%%%%%%%
% \section{Implementation}
%\iffalse
%<*package>
%\fi
%
% This section describes the definitions file |childdoc.def|.

% The definitions cannot be loaded using |\usepackage| or |\RequirePackage|
% which has a mechanism to prevent loading a style file more than once.
% When loading the definitions by means of |\input|
% multiple instances have to be prevented manually:
%\iffalse
%This code needs to be before the `\ProvidesFile' directive
%which is defined at the beginning of this file.
%Therefore it is also placed there and commented out here.
%</package>
%<*discard>
%\fi
%    \begin{macrocode}
\ifdefined\childdocmain\endinput\fi
%    \end{macrocode}
%\iffalse
%</discard>
%<*package>
%\fi
%
% \macro{\ifchilddoc}
% \macro{\ifchilddocmanual}
% The conditional |\ifchilddoc| tells whether a
% child (true) or main (false) document is being compiled.
% The conditional |\ifchilddocmanual| tells whether
% the |\includeonly| mechanism is used (false) or
% the selection of child files must be performed manually (true).
% The definitions initialise to false:
%    \begin{macrocode}
\newif\ifchilddoc
\newif\ifchilddocmanual
%    \end{macrocode}

% \macro{\childdocname}
% \macro{\childdocjob}
% The macro |\childdocname| stores the name of the main document
% to be compiled. The macro |\childdocjob| stores the name of
% the document on which the \LaTeX{} compiler was originally invoked.
% The content of |\jobname| cannot be compared
% to filenames specified in the source due to different catcodes.
% The following code rescans |\jobname|, stores the result
% in |\childdocname| and saves a copy in |\childdocjob|:
%    \begin{macrocode}
\edef\childdocname{\scantokens\expandafter{\jobname\noexpand}}
\let\childdocjob\childdocname
%    \end{macrocode}

% \macro{\childdocdisable}
% The macro |\childdocdisable| prevents the main file
% from being processed more than once.
% At this stage, the main document command |\childdocmain|
% is assumed to be called once again where it should do nothing.
% Any subsequent call to it should prevent
% a secondary processing of the main document
% It overwrites the forwarding commands
% |\childdocof| and |\childdocforward|
% with empty macros to prevent further inclusions of the main document:
%    \begin{macrocode}
\newcommand{\childdocdisable}
{
  \renewcommand{\childdocmain}[1]{\renewcommand{\childdocmain}[1]{\endinput}}
  \renewcommand{\childdocof}[1]{}
  \renewcommand{\childdocby}[2][]{}
  \renewcommand{\childdocforward}[2][]{}
  \renewcommand{\childdocdisable}{}
}
%    \end{macrocode}

% \macro{\childdocmain}
% The macro |\childdocmain| is to be called at the top of the main file
% with nothing or the main filename (without extension) as argument.
% First, it breaks loops.
% If the argument is not empty and does not match |\childdocname|
% (which is set by the first inclusion of |childdoc.def|),
% |\ifchilddoc| is set to true, |\includeonly| is applied to the child file
% and |\jobname| is set to the main file
% (for proper handling of |.aux| files):
%    \begin{macrocode}
\newcommand{\childdocmain}[1]
{
  \childdocdisable\childdocmain{}
  \if?#1?\else
    \begingroup
      \def\childdoctmp{#1}
      \ifx\childdoctmp\childdocname
        \def\childdoctmp{}
      \else
        \def\childdoctmp
        {
          \childdoctrue
          \includeonly{\childdocname}
          \def\childdocjob{#1}
          \def\jobname{#1}
        }
      \fi
      \expandafter
    \endgroup
    \childdoctmp
  \fi
}
%    \end{macrocode}

% \macro{\childdocof}
% The command |\childdocof| redirects
% compilation to the main file |#1|.
%    \begin{macrocode}
\newcommand{\childdocof}[1]
{
  \childdocdisable
  \childdoctrue
  \includeonly{\childdocname}
  \def\jobname{#1}
  \def\childdocjob{#1}
  \input{#1}
}
%    \end{macrocode}

% \macro{\childdocby}
% The command |\childdocby| ....
%    \begin{macrocode}
\newcommand{\childdocby}[2][]
{
  \childdocdisable
  \childdoctrue
  \childdocmanualtrue
  \if?#1?\else
    \def\jobname{#2}
  \fi
  \def\childdocjob{#2}
  \input{#2}
  \endinput
}
%    \end{macrocode}

% \macro{\childdocforward}
% The command |\childdocforward| redirects
% compilation to the main file or
% (if the optional argument is given) a child file.
% Parameters are set as if the main file
% or a child file starting with |\childdocof| was compiled.
% Then compilation is handed over to the main file:
%    \begin{macrocode}
\newcommand{\childdocforward}[2][]
{
  \begingroup
    \if?#1?
      \def\childdoctmp
      {
        \def\childdocname{#2}
        \def\childdocjob{#2}
        \def\jobname{#2}
        \input{#2}
        \endinput
      }
    \else
      \def\childdoctmp
      {
        \childdocdisable
        \def\childdocname{#2}
        \childdoctrue
        \includeonly{#2}
        \def\childdocjob{#1}
        \def\jobname{#1}
        \input{#1}
        \endinput
      }
    \fi
    \expandafter
  \endgroup
  \childdoctmp
}
%    \end{macrocode}

% \macro{\childdocforwardprefix}
% The command |\childdocforwardprefix| redirects
% compilation to the main or a child file by means of a pattern.
% The prefix |#1| in the current filename is replaced by |#2|
% and the suffix of the current filename is kept
% (it is assumed that the filename does not contain the substring `|~~~|'
% which is used as a delimiter).
% Compilation is handed over to the new file by |\childdocforward|:
%    \begin{macrocode}
\newcommand{\childdocforwardprefix}[3][]
{
  \begingroup
    \def\childdocextract #2##1~~~{\def\childdoctmp{\childdocforward[#1]{#3##1}}}
    \expandafter\childdocextract\childdocname~~~
    \expandafter
  \endgroup
  \childdoctmp
}
%    \end{macrocode}

% \macro{\childdoc}
% The deprecated macro |\childdoc| is a legacy version of |\childdocmain|:
%    \begin{macrocode}
\newcommand{\childdoc}{\childdocmain}
%    \end{macrocode}

% \macro{\childdocredirect}
% The deprecated macro |\childdocredirect| is a legacy version
% of |\childdocforward| and |\childdocforwardprefix|:
%    \begin{macrocode}
\newcommand{\childdocredirect}[2][]
{
  \begingroup
    \if?#1?
      \def\childdoctmp{\childdocforward{#2}}
    \else
      \def\childdoctmp{\childdocforwardprefix{#1}{#2}}
    \fi
    \expandafter
  \endgroup
  \childdoctmp
}
%    \end{macrocode}

%\iffalse
%</package>
%\fi
%
\endinput
\childdocforward[cdocsamp]{cdocsch1}"|\\
% |latex -jobname cdocscl2 \|\\
% |  "\def\version{final}% \iffalse
%
% childdoc.dtx Copyright (C) 2017-2018 Niklas Beisert
%
% This work may be distributed and/or modified under the
% conditions of the LaTeX Project Public License, either version 1.3
% of this license or (at your option) any later version.
% The latest version of this license is in
%   http://www.latex-project.org/lppl.txt
% and version 1.3 or later is part of all distributions of LaTeX
% version 2005/12/01 or later.
%
% This work has the LPPL maintenance status `maintained'.
%
% The Current Maintainer of this work is Niklas Beisert.
%
% This work consists of the files childdoc.dtx and childdoc.ins
% and the derived files childdoc.def and cdocsamp.tex with
% cdocsch1.tex, cdocsch2.tex, cdocsdrf.tex, cdocsfn1.tex, cdocsfn2.tex.
%
%<package>\ifdefined\childdocmain\endinput\fi
%<package>\ProvidesFile{childdoc.def}[2018/12/30 v2.0 child document driver]
%<samplemain>\ProvidesFile{cdocsamp.tex}[2018/12/30 v2.0 sample for childdoc]
%<*driver>
%\ProvidesFile{childdoc.drv}[2018/12/30 v2.0 childdoc reference manual file]
\PassOptionsToClass{10pt,a4paper}{article}
\documentclass{ltxdoc}

\usepackage[margin=35mm]{geometry}
\usepackage{hyperref}
\usepackage{hyperxmp}
\usepackage[usenames]{color}

\hypersetup{colorlinks=true}
\hypersetup{pdfstartview=FitH}
\hypersetup{pdfpagemode=UseNone}
\hypersetup{pdfsource={}}
\hypersetup{pdflang={en-UK}}
\hypersetup{pdfcopyright={Copyright 2017-2018 Niklas Beisert.
  This work may be distributed and/or modified under the
  conditions of the LaTeX Project Public License, either version 1.3
  of this license or (at your option) any later version.}}
\hypersetup{pdflicenseurl={http://www.latex-project.org/lppl.txt}}
\hypersetup{pdfcontactaddress={ETH Zurich, ITP, HIT K,
  Wolfgang-Pauli-Strasse 27}}
\hypersetup{pdfcontactpostcode={8093}}
\hypersetup{pdfcontactcity={Zurich}}
\hypersetup{pdfcontactcountry={Switzerland}}
\hypersetup{pdfcontactemail={nbeisert@itp.phys.ethz.ch}}
\hypersetup{pdfcontacturl={http://people.phys.ethz.ch/\xmptilde nbeisert/}}

\newcommand{\secref}[1]{\hyperref[#1]{section \ref*{#1}}}

\parskip1ex
\parindent0pt
\let\olditemize\itemize
\def\itemize{\olditemize\parskip0pt}

\begin{document}

\title{The \textsf{childdoc} Package}
\hypersetup{pdftitle={The childdoc Package}}
\author{Niklas Beisert\\[2ex]
  Institut f\"ur Theoretische Physik\\
  Eidgen\"ossische Technische Hochschule Z\"urich\\
  Wolfgang-Pauli-Strasse 27, 8093 Z\"urich, Switzerland\\[1ex]
  \href{mailto:nbeisert@itp.phys.ethz.ch}
  {\texttt{nbeisert@itp.phys.ethz.ch}}}
\hypersetup{pdfauthor={Niklas Beisert}}
\hypersetup{pdfsubject={Manual for the LaTeX2e Package childdoc}}
\date{30 December 2018, \textsf{v2.0}}
\maketitle

\begin{abstract}\noindent
\textsf{childdoc} is a \LaTeXe{} package
that enables the direct compilation
of document sections included by |\include|
to individual files.
\end{abstract}

\begingroup
\parskip0ex
\tableofcontents
\endgroup

%%%%%%%%%%%%%%%%%%%%%%%%%%%%%%%%%%%%%%%%%%%%%%%%%%%%%%%%%%%%%%%%%%%%%%%%%%%%%%%%
%%%%%%%%%%%%%%%%%%%%%%%%%%%%%%%%%%%%%%%%%%%%%%%%%%%%%%%%%%%%%%%%%%%%%%%%%%%%%%%%
\section{Introduction}

\LaTeX{} provides a mechanism to structure a large document (such as a book)
into a main file and several child files (containing the chapters)
using the |\include| command.
This mechanism is beneficial for documents
which span hundreds of pages in order to
make the source file(s) more manageable.
Moreover, compilation can be restricted to
selected child files by means of the |\includeonly| command.
The latter feature can be used to reduce the compilation time while editing
(this was significantly more useful in the earlier days of \LaTeX{})
or to generate a smaller document which is easier to navigate.
Another application of |\includeonly| is to generate
documents consisting of selected parts of the complete document.

However, there are a few drawbacks of the plain |\include| mechanism:
\begin{itemize}
\item
The child files cannot be compiled on their own,
they can only be compiled via the main file.
A naive editing environment
(such as a text editor with an option
to have the current file processed by \LaTeX)
may require one to switch to the main file before compiling;
attempting to compile the child file produces errors.
\item
The main file must be modified (each time)
to adjust the |\includeonly| command
to the present needs. This easily leaves the main file in a messy state.
\item
The generated document will always carry the filename
of the main document. This is inconvenient if
several child files are to be compiled and
to be kept for distribution.
\end{itemize}

The present package provides a simple interface
to make child files individually compilable by \LaTeX{}.
Compiling a child file then has the same effect as compiling
the main file with an |\includeonly| command
to select the appropriate child.
Moreover the generated document will carry the name of the child
rather than the main file.
This resolves all three above issues.

This feature is meant to make the editing of books,
thesis documents and lecture notes somewhat more convenient.
However, the package can also be used efficiently for
composing a series of documents (such as exercise sheets)
which are typically distributed individually.
It then assists the author in generating the individual documents
(potentially in different versions)
as well as a document containing the collected series.
Another application is in developing style files
or other kinds of included material
where compilation of the style file could redirect
to a sample or test file.

%%%%%%%%%%%%%%%%%%%%%%%%%%%%%%%%%%%%%%%%%%%%%%%%%%%%%%%%%%%%%%%%%%%%%%%%%%%%%%%%
%%%%%%%%%%%%%%%%%%%%%%%%%%%%%%%%%%%%%%%%%%%%%%%%%%%%%%%%%%%%%%%%%%%%%%%%%%%%%%%%
\section{Usage}

First of all, the package \textsf{childdoc} is \emph{not} a standard
\LaTeXe{} |.sty| style file! Therefore it needs to be invoked in
a non-standard way.

%%%%%%%%%%%%%%%%%%%%%%%%%%%%%%%%%%%%%%%%%%%%%%%%%%%%%%%%%%%%%%%%%%%%%%%%%%%%%%%%
\subsection{Included Files}
\label{sec:include}

%%%%%%%%%%%%%%%%%%%%%%%%%%%%%%%%%%%%%%%%
\DescribeMacro{\childdocmain}
To use the package, add the commands
\begin{center}
\begin{tabular}{l}
|\input{childdoc.def}|\\
|\childdocmain{}|\\
\end{tabular}
\end{center}
at the very top of the main \LaTeX{} file,
in particular \emph{before} the |\documentclass| statement!
The argument of |\childdocmain| should be left empty
(but it must be present).

%%%%%%%%%%%%%%%%%%%%%%%%%%%%%%%%%%%%%%%%
\DescribeMacro{\childdocof}
Furthermore, add the commands
\begin{center}
\begin{tabular}{l}
|\input{childdoc.def}|\\
|\childdocof{|\textit{main}|}|\\
\end{tabular}
\end{center}
at the top of every child file \textit{child}
which is included by |\include{|\textit{child}|}|
from within the main file
(or at least for those files to be compiled individually).
The argument \textit{main} must be the filename of the main file.

There are a couple of
considerations in setting up the main and child documents:

%%%%%%%%%%%%%%%%%%%%%%%%%%%%%%%%%%%%%%%%
\paragraph{Restrictions.}

Please note the following restrictions:
\begin{itemize}
\item
|\childdocmain| must be called with one argument \textit{main}
to ensure compatibility with earlier version of the package.
It must either be empty (|\childdocmain{}|)
or precisely match the filename of the main file in which it is specified.
See \secref{sec:detection} for further information.
\item
The filename \textit{main} must be specified without the |.tex| extension.
\item
The filename \textit{main} is case sensitive
(even in case-insensitive file systems)
due to internal string comparison.
\item
The argument \textit{main} should be fully expanded, it cannot be a macro.
\item
Subdirectories and special characters should be avoided in filenames.
\item
The command |\childdocmain{|\textit{main}|}| must be followed by a whitespace.
It should not be followed immediately by another command
or by a comment mark `|%|'.
This is because the \TeX{} parser reads the token immediately following
the argument of |\childdocmain| and puts it
at the beginning of every child section;
however, a white\-space is ignored.
\end{itemize}

%%%%%%%%%%%%%%%%%%%%%%%%%%%%%%%%%%%%%%%%
\paragraph{Content of Main File.}

It is advisable to place all content in the child files included by |\include|.
Any output contained in the main file will appear in all child documents
unless suppressed manually;
it cannot be suppressed automatically by the |\includeonly| directive
and thus should normally be avoided.
A method to include some content in the main file
by means of conditional processing is described in \secref{sec:conditional}.

%%%%%%%%%%%%%%%%%%%%%%%%%%%%%%%%%%%%%%%%
\paragraph{Page Numbering.}

When only a part of the document is compiled,
the appropriate numbering of pages
(as well as other status parameters)
is determined from the |.aux| files.
The latter contain information from previous passes.
However this information needs to propagate through
all intermediate child documents.
Therefore the page numbering in child documents may well
be inconsistent until the complete document is compiled at least once.

A useful (if unconventional) way to always ensure a consistent
page numbering is to restart the numbering in each child document
and denote the pages by `\textit{child}|.|\textit{page}'
where \textit{child} represents the chapter/section number of the child file.
This can be achieved by the command
|\numberwithin{page}{|\textit{child}|}|
of the \textsf{amsmath} package
where \textit{child} can be |chapter| or |section|
depending on the chosen structuring.
Alternatively, one can modify the macro |\thepage| appropriately
and reset the counter |page| at the start of each child file.

%%%%%%%%%%%%%%%%%%%%%%%%%%%%%%%%%%%%%%%%%%%%%%%%%%%%%%%%%%%%%%%%%%%%%%%%%%%%%%%%
\subsection{Conditional Processing}
\label{sec:conditional}

The package provides a mechanism to compile different versions
of a document. To customise the versions further some conditional processing
can come in handy to distinguish which version is being compiled.
The package provides two macros to describe the compilation context:

%%%%%%%%%%%%%%%%%%%%%%%%%%%%%%%%%%%%%%%%
\DescribeMacro{\ifchilddoc}
The conditional |\ifchilddoc| distinguishes between the compilation of
child documents and the main document:
%
\begin{center}
|\ifchilddoc |\textit{child-code}| |[|\||else |\textit{main-code}]| \||fi|
\end{center}

%%%%%%%%%%%%%%%%%%%%%%%%%%%%%%%%%%%%%%%%
\DescribeMacro{\childdocname}
\DescribeMacro{\childdocjob}
The macro |\childdocname| contains the filename (without extension)
of the main or child file being processed.
Note that |\childdocjob| will always contain the name of the main file.

%%%%%%%%%%%%%%%%%%%%%%%%%%%%%%%%%%%%%%%%
\paragraph{Title Page.}

Conditional processing can be used to include a title or banner page
in the main document when proper precautions are taken.
Importantly, the code in the main file should ensure that the page counter
(as well as other status parameters which are stored in the |.aux| files)
takes the same value after the conditional processing.
Otherwise the page numbers may take divergent values
depending on which part is compiled.

For example, a title page could be declared by:
%
\begin{center}
\begin{tabular}{l}
|\ifchilddoc\||else|\\
|\addtocounter{page}{-1}|\\
\textit{code for title page}\\
|\newpage|\\
|\||fi|
\end{tabular}
\end{center}
%
A banner page for the child documents can be generated by:
%
\begin{center}
\begin{tabular}{l}
|\ifchilddoc|\\
|\addtocounter{page}{-1}|\\
\textit{code for banner page}\\
|\newpage|\\
|\||fi|
\end{tabular}
\end{center}
%
Here one could write a message such as:
\begin{center}
|This is the part \childdocname{} of \childdocjob{}.|
\end{center}

%%%%%%%%%%%%%%%%%%%%%%%%%%%%%%%%%%%%%%%%%%%%%%%%%%%%%%%%%%%%%%%%%%%%%%%%%%%%%%%%
\subsection{Flags}
\label{sec:flags}

The package makes it easy to generate different versions
of the main or child documents.
To this end compilation flags can be defined
and assigned different default values.
They will be particularly useful in conjunction
with the forwarding mechanism described in \secref{sec:forward}.

For example, it may be useful to have a flag |\version|
which can be set to |draft| or |final|.
The document source will contain some conditional code
depending on the value of |\version|.
Suppose further, the flag should default to |final| for the main file
and to |draft| for child files
which is a natural assignment for editing the document.
This is achieved by placing the following code
in the preamble of the main document
(below the |\childdocmain| directive):
%
\begin{center}
\begin{tabular}{l}
|\ifchilddoc|\\
|\providecommand{\version}{draft}|\\
|\||else|\\
|\providecommand{\version}{final}|\\
|\||fi|
\end{tabular}
\end{center}
%
The definition by |\providecommand| makes sure
that previous definitions are not overwritten.
Further statements |\providecommand{\version}{...}|
can thus be added before the above code to override it.

For the main file, one might add a line
(between |\childdocmain| and the above block)
%
\begin{center}
|%\ifchilddoc\||else\providecommand{\version}{draft}\||fi|
\end{center}
%
which can be uncommented to produce a draft version.
Likewise one can add a line to the very top of a child file
(above the |\childdocof{|\textit{main}|}| directive)
%
\begin{center}
|%\providecommand{\version}{final}|
\end{center}
%
which can be uncommented to produce the final version of this child document.

%%%%%%%%%%%%%%%%%%%%%%%%%%%%%%%%%%%%%%%%%%%%%%%%%%%%%%%%%%%%%%%%%%%%%%%%%%%%%%%%
\subsection{Forwarding}
\label{sec:forward}

Different versions of the main or child documents
using compilation flags as described in \secref{sec:flags}
can be (permanently) stored in different files
for convenient compilation, viewing and distribution.
To this end, the package defines a command
to pass on compilation to a different file:

%%%%%%%%%%%%%%%%%%%%%%%%%%%%%%%%%%%%%%%%
\DescribeMacro{\childdocforward}
The command |\childdocforward| redirects processing to
another source file:
%
\begin{center}
\begin{tabular}{l}
|\input{childdoc.def}|\\
|\childdocforward[|\textit{main}|]{|\textit{dest}|}|\\
\end{tabular}
\end{center}
%
The argument \textit{dest} is the destination file
(without extension).
It should be the main file or one of the child files.
Note that further \textsf{childdoc} directives
such as |\childdocof| and |\childdocforward|
in the indicated file will be processed in this form.
The optional argument \textit{main}
passes on directly to the main file \textit{main}
while pretending to compile the child \textit{dest}.
This form behaves as if \textit{dest}
issues |\childdocof{|\textit{main}|}| right away,
and no further \textsf{childdoc} directives will be processed.

%%%%%%%%%%%%%%%%%%%%%%%%%%%%%%%%%%%%%%%%
\DescribeMacro{\...prefix}
In the alternative form |\childdocforwardprefix|,
%
\begin{center}
\begin{tabular}{l}
|\input{childdoc.def}|\\
|\childdocforwardprefix[|\textit{main}|]{|\textit{prefix}|}{|\textit{dest}|}|
\end{tabular}
\end{center}
%
the destination file is determined by a pattern
depending on the current file:
To make this work, the current file must be called
`{\textit{prefix}\hspace{0.2em}\textit{suffix}}'
with \textit{prefix} matching precisely the argument.
Processing is then passed on to the file
`{\textit{dest}\hspace{0.2em}\textit{suffix}}'.
Surely, the same effect is achieved by
directly specifying the
argument `{\textit{dest}\hspace{0.2em}\textit{suffix}}'
in the first form.
However, that requires to set up a different file
for each child. With the alternative form of the command
all these files can have exactly the same content
which simplifies setting them up and maintaining them.

For example, the following file |draft.tex|
with a compilation flag |\version| as described in \secref{sec:flags}
compiles the main document as a draft:
%
\begin{center}
\begin{tabular}{l}
|\def\version{draft}|\\
|\input{childdoc.def}|\\
|\childdocforward{|\textit{main}|}|
\end{tabular}
\end{center}
%
Likewise, the following files |final|\textit{nn}|.tex|
compile the final version of the child document
|child|\textit{nn}|.tex|:
%
\begin{center}
\begin{tabular}{l}
|\def\version{final}|\\
|\input{childdoc.def}|\\
|\childdocforwardprefix{final}{child}|
\end{tabular}
\end{center}
%

Note that when several versions of a main file and/or of each child file
are to be generated, it may be convenient to set up a |Makefile| or
shell script to automatise the process.

%%%%%%%%%%%%%%%%%%%%%%%%%%%%%%%%%%%%%%%%%%%%%%%%%%%%%%%%%%%%%%%%%%%%%%%%%%%%%%%%
\subsection{Command Line Processing}
\label{sec:commandline}

The effect of redirection files can also be achieved by invoking
the \LaTeX{} compiler with a more elaborate command line.
Most conveniently this should be done as part
of a shell script or a |Makefile|.

When using \textsf{childdoc} in the main file, the following
command lines effectively perform a redirection
(note that depending on the shell being used,
backslashes may have to be doubled: `|\|' $\to$ `|\\|'):
%
\begin{center}
|... -jobname "|\textit{target}|" |\\|"|[\textit{flags}]%
|\input{childdoc.def}\childdocforward[|\textit{main}|]{|\textit{dest}|}"|
\end{center}
%
Here \textit{target} is the name of the output file,
\textit{main} is the name of the main file
and \textit{dest} is the name of the main or child file to be processed
(all filenames without extensions).
The optional argument \textit{main} can be omitted
if \textit{main} matches \textit{dest}.
Optionally, compilation \textit{flags} can be defined via |\def| commands.
This command line makes the \TeX{} engine believe
it is compiling the file \textit{target}
whose content is specified as the latter parameter.
The provided code then forwards the processing to
\textit{main} or \textit{dest} as described in \secref{sec:forward}.

%%%%%%%%%%%%%%%%%%%%%%%%%%%%%%%%%%%%%%%%%%%%%%%%%%%%%%%%%%%%%%%%%%%%%%%%%%%%%%%%
\subsection{Include by Input}
\label{sec:input}

Including child documents by |\include| has some restrictions by design.
Most notably, the content of a child document always occupies
its own set of pages; pages cannot be shared between child documents.
Usually, this behaviour makes perfect sense
because each child document contain an essential part of the document.
However, in some situations it may be desirable to compose
a document from a collection of parts
without having mandatory page breaks between then.
For this case, the package
provides a mechanism to include parts
by |\input| which can also be processed individually.
However, by construction this mechanism
requires manual handling of the content to be output.

%%%%%%%%%%%%%%%%%%%%%%%%%%%%%%%%%%%%%%%%
\DescribeMacro{\ifchilddocmanual}
The main file should be prepared as usual, see \secref{sec:include}.
However, the document body must make a distinction
between processing of an individual part and of the main document, e.g.:
%
\begin{center}
\begin{tabular}{l}
|\ifchilddocmanual|\\
|\input{\childdocname}|\\
|\||else|\\
\textit{document body with }|\input{|\textit{part}|}|\\
|\||fi|
\end{tabular}
\end{center}
%
The conditional |\ifchilddocmanual| is true whenever
a part to be included by |\input| is being compiled,
and the name of the part is stored in |\childdocname|.

%%%%%%%%%%%%%%%%%%%%%%%%%%%%%%%%%%%%%%%%
\DescribeMacro{\childdocby}
Each part to be included by |\input| should start with:
%
\begin{center}
\begin{tabular}{l}
|\input{childdoc.def}|\\
|\childdocby{|\textit{main}|}|\\
\end{tabular}
\end{center}
%
The directive |\childdocby| is similar to |\childdocof|
described in \secref{sec:include},
but the subsequent selection of content must be done manually.
To that end, both |\ifchilddoc| and |\ifchilddocmanual|
will be true upon processing of a part,
and the name of the part is stored in |\childdocname|.
Note that |\jobname| will be set to the filename of the current part
so that each part receives an individual |.aux| file
that does not interfere with the |.aux| file(s) of the main document.
This behaviour can be altered by the alternative form
|\childdocby[*]{|\textit{main}|}| (with a non-empty optional argument)
which uses the |.aux| file of the main document
by setting |\jobname| to \textit{main}.

%%%%%%%%%%%%%%%%%%%%%%%%%%%%%%%%%%%%%%%%%%%%%%%%%%%%%%%%%%%%%%%%%%%%%%%%%%%%%%%%
\subsection{Driver Development}
\label{sec:driver}

The \textsf{childdoc} mechanism can also be use for the development
of definition files such as \LaTeX{} styles or classes.
This case differs from the above setup with multiple parts
included by |\include| in that no |\includeonly| should be invoked.
This can be achieved by starting the include file
(before |\ProvidesPackage|) with:
%
\begin{center}
\begin{tabular}{l}
|\input{childdoc.def}|\\
|\childdocforward{|\textit{main}|}|\\
\end{tabular}
\end{center}
%
or alternatively with:
%
\begin{center}
\begin{tabular}{l}
|\input{childdoc.def}|\\
|\childdocby{|\textit{main}|}|\\
\end{tabular}
\end{center}
%
Both forms have slightly different effects as described above.
The main file is prepared as usual, see \secref{sec:include}.

%%%%%%%%%%%%%%%%%%%%%%%%%%%%%%%%%%%%%%%%%%%%%%%%%%%%%%%%%%%%%%%%%%%%%%%%%%%%%%%%
\subsection{Legacy Detection}
\label{sec:detection}

The directive |\childdocmain| in the main file can detect
whether the complete document or merely a child is to be compiled
even without using the directive |\childdocof|.
This method is deprecated because it is less robust
and there is no compelling reason to use it;
it is merely provided for backward compatibility
and it may be removed in future versions.

If the detection mechanism is to be used,
it is mandatory to correctly specify
the filename of the main file as the argument of |\childdocmain|:
%
\begin{center}
\begin{tabular}{l}
|\input{childdoc.def}|\\
|\childdocmain{|\textit{main}|}|\\
\end{tabular}
\end{center}
%
If |\jobname| does not match the argument \textit{main} of |\childdocmain|,
it is assumed that |\jobname| points to the child file to be compiled.
When using |\childdocmain| with the main file specified as argument,
it suffices to start a child file
with just |\input{|\textit{main}|}|
without loading of the package and using |\childdocof|.
If instead all processing is done
with the appropriate \textsf{childdoc} directives,
the argument of \textit{main} of |\childdocmain| can be empty.

An alternative version of the command line processing described
in \secref{sec:commandline} using the detection mechanism reads:
%
\begin{center}
|... -jobname "|\textit{target}|" "|[\textit{flags}]%
[|\def\jobname{|\textit{dest}|}|]|\input{|\textit{main}|}"|
\end{center}

%%%%%%%%%%%%%%%%%%%%%%%%%%%%%%%%%%%%%%%%%%%%%%%%%%%%%%%%%%%%%%%%%%%%%%%%%%%%%%%%
\subsection{Manual Code}
\label{sec:manual}

In case one cannot be certain whether the definitions file |childdoc.def|
is installed on the target \TeX{} distribution
and one prefers not to ship it,
it is conceivable to paste a few relevant commands into the sources.

To that end, drop all statements |\input{childdoc.def}|
and perform the replacements as outlined below.
Instead of |\childdocmain{|\textit{main}|}| add the following code
to the top of the main file:
%
\begin{center}
\begin{tabular}{l}
|\||ifdefined\childdocname\endinput\||fi\newif\ifchilddoc|\\
|\edef\childdocname{\scantokens\expandafter{\jobname\noexpand}}|\\
|\def\childdocmain{|\textit{main}|}\||ifx\childdocmain\childdocname\||else|\\
|\childdoctrue\includeonly{\childdocname}\let\jobname\childdocmain\||fi|\\
\end{tabular}
\end{center}
%
Instead of |\childdocof{|\textit{main}|}| just include the main file
at the top of each child file:
%
\begin{center}
|\input{|\textit{main}|}|
\end{center}
%
A simple redirection |\childdocforward{|\textit{dest}|}| is achieved by:
%
\begin{center}
|\def\jobname{|\textit{dest}|}\input{\jobname}|
\end{center}
%
The redirection with prefix
|\childdocforwardprefix[|\textit{prefix}|]{|\textit{dest}|}|
is accomplished by:
%
\begin{center}
\begin{tabular}{l}
|{\edef\jobname{\scantokens\expandafter{\jobname\noexpand}}|\\
|\def\redirectjob |\textit{prefix}|#1~~~{\gdef\jobname{|\textit{dest}|#1}}|\\
|\expandafter\redirectjob\jobname~~~}\input{\jobname}|
\end{tabular}
\end{center}

In an alternative approach,
child documents can be compiled by a specific command line
without additional code or specific definitions:
%
\begin{center}
|... -jobname "|\textit{target}|" "|[\textit{flags}]%
|\includeonly{|\textit{dest}|}\input{|\textit{main}|}"|
\end{center}
%

%%%%%%%%%%%%%%%%%%%%%%%%%%%%%%%%%%%%%%%%%%%%%%%%%%%%%%%%%%%%%%%%%%%%%%%%%%%%%%%%
%%%%%%%%%%%%%%%%%%%%%%%%%%%%%%%%%%%%%%%%%%%%%%%%%%%%%%%%%%%%%%%%%%%%%%%%%%%%%%%%
\section{Information}

%%%%%%%%%%%%%%%%%%%%%%%%%%%%%%%%%%%%%%%%%%%%%%%%%%%%%%%%%%%%%%%%%%%%%%%%%%%%%%%%
\subsection{Copyright}

Copyright \copyright{} 2017--2018 Niklas Beisert

This work may be distributed and/or modified under the
conditions of the \LaTeX{} Project Public License, either version 1.3
of this license or (at your option) any later version.
The latest version of this license is in
  \url{http://www.latex-project.org/lppl.txt}
and version 1.3 or later is part of all distributions of \LaTeX{}
version 2005/12/01 or later.

This work has the LPPL maintenance status `maintained'.

The Current Maintainer of this work is Niklas Beisert.

This work consists of the files |README.txt|, |childdoc.ins| and |childdoc.dtx|
as well as the derived files |childdoc.def|, |cdocsamp.tex|
with |cdocsch1.tex|, |cdocsch2.tex|, |cdocspt3.tex|, |cdocspt4.tex|,
|cdocsdrf.tex|, |cdocsfn1.tex|, |cdocsfn2.tex|
as well as |childdoc.pdf|.

%%%%%%%%%%%%%%%%%%%%%%%%%%%%%%%%%%%%%%%%%%%%%%%%%%%%%%%%%%%%%%%%%%%%%%%%%%%%%%%%
\subsection{Files and Installation}

The package consists of the files:
%
\begin{center}
\begin{tabular}{ll}
    |README.txt|   & readme file \\
    |childdoc.ins| & installation file \\
    |childdoc.dtx| & source file \\
    |childdoc.def| & definition file \\
    |cdocsamp.tex| & sample main file \\
    |cdocsch1.tex| & sample include file \\
    |cdocsch2.tex| & sample include file \\
    |cdocspt3.tex| & sample part file \\
    |cdocspt4.tex| & sample part file \\
    |cdocsdrf.tex| & sample redirection file \\
    |cdocsfn1.tex| & sample redirection file \\
    |cdocsfn2.tex| & sample redirection file \\
    |childdoc.pdf| & manual
\end{tabular}
\end{center}
%
The distribution consists of the files
|README.txt|, |childdoc.ins| and |childdoc.dtx|.
%
\begin{itemize}
\item
Run (pdf)\LaTeX{} on |childdoc.dtx|
to compile the manual |childdoc.pdf| (this file).
\item
Run \LaTeX{} on |childdoc.ins| to create the definitions file |childdoc.def|
and the sample |cdocsamp.tex| with include files
|cdocsch1.tex|, |cdocsch2.tex|, |cdocspt3.tex|, |cdocspt4.tex|,
|cdocsdrf.tex|, |cdocsfn1.tex|, |cdocsfn2.tex|.
Then copy the file |childdoc.def| to an appropriate directory of your \LaTeX{}
distribution, e.g.\ \textit{texmf-root}|/tex/latex/childdoc|.
\end{itemize}

%%%%%%%%%%%%%%%%%%%%%%%%%%%%%%%%%%%%%%%%%%%%%%%%%%%%%%%%%%%%%%%%%%%%%%%%%%%%%%%%
\subsection{Related CTAN Packages}

There are several other packages which offer a similar functionality:
%
\begin{itemize}
\item
The packages
\href{http://ctan.org/pkg/docmute}{\textsf{docmute}},
\href{http://ctan.org/pkg/includex}{\textsf{includex}} and
\href{http://ctan.org/pkg/standalone}{\textsf{standalone}}
provide commands to include only the document body of
a child file thus allowing both files to be compiled individually.
\item
The packages \href{http://ctan.org/pkg/subdocs}{\textsf{subdocs}}
and \href{http://ctan.org/pkg/subfiles}{\textsf{subfiles}}
provide structures in which the main and child documents can be
encapsulated and allowing them to be compiled individually.
The inclusion mechanism is different from the conventional |\include|.
\item
The package \href{http://ctan.org/pkg/combine}{\textsf{combine}}
is an elaborate solution to combine several documents into one.
\end{itemize}
%
See also the CTAN topic \href{http://ctan.org/topic/subdocs}{\textsf{subdocs}}
for further related packages.
The present package differs from the above solutions in that
a document structure constructed with the conventional |\include| mechanism
just needs two extra commands at the top of every file
such that all constituent files can be compiled individually.

%%%%%%%%%%%%%%%%%%%%%%%%%%%%%%%%%%%%%%%%%%%%%%%%%%%%%%%%%%%%%%%%%%%%%%%%%%%%%%%%
%\subsection{Feature Suggestions}
%
%The following is a list of features which may be useful for future
%versions of this package:
%%
%\begin{itemize}
%\item
%\ldots
%\end{itemize}

%%%%%%%%%%%%%%%%%%%%%%%%%%%%%%%%%%%%%%%%%%%%%%%%%%%%%%%%%%%%%%%%%%%%%%%%%%%%%%%%
\subsection{Revision History}

%%%%%%%%%%%%%%%%%%%%%%%%%%%%%%%%%%%%%%%%
\paragraph{v2.0:} 2018/12/30

\begin{itemize}
\item
immediate forward processing
\item
added |\childdocby| mechanism
\item
manual restructured
\end{itemize}

%%%%%%%%%%%%%%%%%%%%%%%%%%%%%%%%%%%%%%%%
\paragraph{v1.6:} 2018/01/17

\begin{itemize}
\item
application for development of include files
\item
corrections to manual
\end{itemize}

%%%%%%%%%%%%%%%%%%%%%%%%%%%%%%%%%%%%%%%%
\paragraph{v1.5:} 2017/05/21

\begin{itemize}
\item
more complete structuring introduced
\item
|\childdocof| introduced
\item
|\childdoc| renamed to |\childdocmain|
\item
|\childredirect| renamed to |\childdocforward| and |\childdocforwardprefix|
and functionality expanded
\end{itemize}

%%%%%%%%%%%%%%%%%%%%%%%%%%%%%%%%%%%%%%%%
\paragraph{v1.0:} 2017/04/27

\begin{itemize}
\item
manual and install package
\item
first version published on CTAN
\end{itemize}

%%%%%%%%%%%%%%%%%%%%%%%%%%%%%%%%%%%%%%%%
\paragraph{v0.6:} 2017/04/26

\begin{itemize}
\item
redirection mechanism added
\end{itemize}

%%%%%%%%%%%%%%%%%%%%%%%%%%%%%%%%%%%%%%%%
\paragraph{v0.5:} 2017/04/26

\begin{itemize}
\item
functionality in definition file
\end{itemize}


%%%%%%%%%%%%%%%%%%%%%%%%%%%%%%%%%%%%%%%%%%%%%%%%%%%%%%%%%%%%%%%%%%%%%%%%%%%%%%%%
%%%%%%%%%%%%%%%%%%%%%%%%%%%%%%%%%%%%%%%%%%%%%%%%%%%%%%%%%%%%%%%%%%%%%%%%%%%%%%%%
%%%%%%%%%%%%%%%%%%%%%%%%%%%%%%%%%%%%%%%%%%%%%%%%%%%%%%%%%%%%%%%%%%%%%%%%%%%%%%%%
\appendix

\settowidth\MacroIndent{\rmfamily\scriptsize 000\ }

 \DocInput{childdoc.dtx}

\end{document}
%</driver>
% \fi
%
% %%%%%%%%%%%%%%%%%%%%%%%%%%%%%%%%%%%%%%%%%%%%%%%%%%%%%%%%%%%%%%%%%%%%%%%%%%%%%%
% %%%%%%%%%%%%%%%%%%%%%%%%%%%%%%%%%%%%%%%%%%%%%%%%%%%%%%%%%%%%%%%%%%%%%%%%%%%%%%
% \section{Sample}
%\iffalse
%<*samplemain>
%\fi
%
% The following presents a sample document
% with two chapters, two parts, a title page,
% a compile flag as well as three forwarding files to set the flag.
% It consists of eight |.tex| files:
% \begin{center}
% \begin{tabular}{ll}
% |cdocsamp.tex|&main file\\
% |cdocsch1.tex|&include file for chapter 1\\
% |cdocsch2.tex|&include file for chapter 2\\
% |cdocspt3.tex|&include file for part 3\\
% |cdocspt4.tex|&include file for part 4\\
% |cdocsdrf.tex|&forwarding file for main file in draft mode\\
% |cdocsfi1.tex|&forwarding file for final version of chapter 1\\
% |cdocsfi2.tex|&forwarding file for final version of chapter 2\\
% \end{tabular}
% \end{center}
% Each of the eight files can be compiled directly by the \LaTeX{} compiler.
%
% %%%%%%%%%%%%%%%%%%%%%%%%%%%%%%%%%%%%%%
% \paragraph{Main File.}
%
% The main file is called |cdocsamp.tex|.
%
% Load the \textsf{childdoc} definitions and
% declare the filename for the main document:
%    \begin{macrocode}
\input{childdoc.def}
\childdocmain{}
%    \end{macrocode}

% Optional override for |\version| flag:
%    \begin{macrocode}
%%\ifchilddoc\else\providecommand{\version}{draft}\fi
%    \end{macrocode}

% Define the default values for the |\version| flag
% (|final| for the main file and |draft| for childs):
%    \begin{macrocode}
\ifchilddoc
\providecommand{\version}{draft}
\else
\providecommand{\version}{final}
\fi
%    \end{macrocode}

% Load the standard document class:
%    \begin{macrocode}
\documentclass[12pt]{article}
%    \end{macrocode}

% Start the document body:
%    \begin{macrocode}
\begin{document}
%    \end{macrocode}

% Declare a title page.
% Print title, part of document being processed and version flag:
%    \begin{macrocode}
\addtocounter{page}{-1}
\begin{center}
{\LARGE\bfseries{}childdoc example\par}
\vspace{1cm}
\ifchilddoc
\ifchilddocmanual part\else chapter\fi:
`\childdocname' of `\childdocjob'\par
\else
main document: `\childdocjob'\par
\fi
version: \version\par
\end{center}
\newpage
%    \end{macrocode}

% Manually include selected file,
% otherwise process as usual:
%    \begin{macrocode}
\ifchilddocmanual
\section*{part `\childdocname'}
\input{\childdocname}
\else
%    \end{macrocode}

% Include the two chapters:
%    \begin{macrocode}
\include{cdocsch1}
\include{cdocsch2}
%    \end{macrocode}

% Include the two parts unless only chapters should be displayed:
%    \begin{macrocode}
\ifchilddoc\else
\section{part three}
\input{cdocspt3}
\section{part four}
\input{cdocspt4}
\fi
%    \end{macrocode}

% Process as usual until here:
%    \begin{macrocode}
\fi
%    \end{macrocode}

% End of document body:
%    \begin{macrocode}
\end{document}
%    \end{macrocode}
%\iffalse
%</samplemain>
%\fi
%
% %%%%%%%%%%%%%%%%%%%%%%%%%%%%%%%%%%%%%%
% \paragraph{Chapter Include Files.}
%
% The include files are called |cdocsch1.tex| and |cdocsch2.tex|.
%
%\iffalse
%<*samplechap1|samplechap2>
%\fi

% Optional override for |\version| flag:
%    \begin{macrocode}
%%\providecommand{\version}{final}
%    \end{macrocode}

% Include the main document:
%    \begin{macrocode}
\input{childdoc.def}
\childdocof{cdocsamp}
%    \end{macrocode}

%\iffalse
%</samplechap1|samplechap2>
%\fi
%
%\iffalse
%<*samplechap1>
%\fi
% Some text for chapter 1:
%    \begin{macrocode}
\section{one}
some text in chapter one
%    \end{macrocode}

%\iffalse
%</samplechap1>
%\fi
% Some text for chapter 2:
%\iffalse
%<*samplechap2>
%\fi
%    \begin{macrocode}
\section{two}
more text in chapter two
%    \end{macrocode}

%\iffalse
%</samplechap2>
%\fi
%
% %%%%%%%%%%%%%%%%%%%%%%%%%%%%%%%%%%%%%%
% \paragraph{Part Include Files.}
%
% The include files are called |cdocspt3.tex| and |cdocspt4.tex|.
%
%\iffalse
%<*samplepart3|samplepart4>
%\fi

% Optional override for |\version| flag:
%    \begin{macrocode}
%%\providecommand{\version}{final}
%    \end{macrocode}

% Include the main document:
%    \begin{macrocode}
\input{childdoc.def}
\childdocby{cdocsamp}
%    \end{macrocode}

%\iffalse
%</samplepart3|samplepart4>
%\fi
%
%\iffalse
%<*samplepart3>
%\fi
% Some text for part 3:
%    \begin{macrocode}
some text in part three
%    \end{macrocode}

%\iffalse
%</samplepart3>
%\fi
% Some text for part 4:
%\iffalse
%<*samplepart4>
%\fi
%    \begin{macrocode}
more text in part four
%    \end{macrocode}

%\iffalse
%</samplepart4>
%\fi
%
% %%%%%%%%%%%%%%%%%%%%%%%%%%%%%%%%%%%%%%
% \paragraph{Forwarding for a Complete Draft.}
%
% The following forwarding file |cdocsdrf.tex|
% compiles the main document in draft mode:
%\iffalse
%<*sampledraft>
%\fi
%    \begin{macrocode}
\def\version{draft}
\input{childdoc.def}
\childdocforward{cdocsamp}
%    \end{macrocode}

%\iffalse
%</sampledraft>
%\fi
%
% %%%%%%%%%%%%%%%%%%%%%%%%%%%%%%%%%%%%%%
% \paragraph{Forwarding for Final Version of the Chapters.}
%
% The following forwarding files |cdocsfn1.tex| and |cdocsfn2.tex|
% (with identical content)
% compile the final versions of the child documents
% |cdocsch1.tex| and |cdocsch2.tex|, respectively:
%\iffalse
%<*samplefinal>
%\fi
%    \begin{macrocode}
\def\version{final}
\input{childdoc.def}
\childdocforwardprefix[cdocsamp]{cdocsfn}{cdocsch}
%    \end{macrocode}

%\iffalse
%</samplefinal>
%\fi
%
% %%%%%%%%%%%%%%%%%%%%%%%%%%%%%%%%%%%%%%
% \paragraph{Command Line Processing.}
%
% The following three command lines generate the output files
% |cdocscld|, |cdocscl1| and |cdocscl2|
% which should be identical to
% |cdocsdrf|, |cdocsch1| and |cdocsfn2|, respectively:
% \begin{center}
% \begin{tabular}{l}
% |latex -jobname cdocscld \|\\
% |  "\def\version{draft}\input{childdoc.def}\childdocforward{cdocsamp}"|\\
% |latex -jobname cdocscl1 \|\\
% |  "\input{childdoc.def}\childdocforward[cdocsamp]{cdocsch1}"|\\
% |latex -jobname cdocscl2 \|\\
% |  "\def\version{final}\input{childdoc.def}\childdocforward{cdocsch2}"|
% \end{tabular}
% \end{center}
% Note that the trailing backslash on each first line
% merely continues the input to the second line
% (for convenient cut ant paste).
% Furthermore, the command |latex| can be replaced by any
% of its alternative versions such as |pdflatex|.
%
% %%%%%%%%%%%%%%%%%%%%%%%%%%%%%%%%%%%%%%%%%%%%%%%%%%%%%%%%%%%%%%%%%%%%%%%%%%%%%%
% %%%%%%%%%%%%%%%%%%%%%%%%%%%%%%%%%%%%%%%%%%%%%%%%%%%%%%%%%%%%%%%%%%%%%%%%%%%%%%
% \section{Implementation}
%\iffalse
%<*package>
%\fi
%
% This section describes the definitions file |childdoc.def|.

% The definitions cannot be loaded using |\usepackage| or |\RequirePackage|
% which has a mechanism to prevent loading a style file more than once.
% When loading the definitions by means of |\input|
% multiple instances have to be prevented manually:
%\iffalse
%This code needs to be before the `\ProvidesFile' directive
%which is defined at the beginning of this file.
%Therefore it is also placed there and commented out here.
%</package>
%<*discard>
%\fi
%    \begin{macrocode}
\ifdefined\childdocmain\endinput\fi
%    \end{macrocode}
%\iffalse
%</discard>
%<*package>
%\fi
%
% \macro{\ifchilddoc}
% \macro{\ifchilddocmanual}
% The conditional |\ifchilddoc| tells whether a
% child (true) or main (false) document is being compiled.
% The conditional |\ifchilddocmanual| tells whether
% the |\includeonly| mechanism is used (false) or
% the selection of child files must be performed manually (true).
% The definitions initialise to false:
%    \begin{macrocode}
\newif\ifchilddoc
\newif\ifchilddocmanual
%    \end{macrocode}

% \macro{\childdocname}
% \macro{\childdocjob}
% The macro |\childdocname| stores the name of the main document
% to be compiled. The macro |\childdocjob| stores the name of
% the document on which the \LaTeX{} compiler was originally invoked.
% The content of |\jobname| cannot be compared
% to filenames specified in the source due to different catcodes.
% The following code rescans |\jobname|, stores the result
% in |\childdocname| and saves a copy in |\childdocjob|:
%    \begin{macrocode}
\edef\childdocname{\scantokens\expandafter{\jobname\noexpand}}
\let\childdocjob\childdocname
%    \end{macrocode}

% \macro{\childdocdisable}
% The macro |\childdocdisable| prevents the main file
% from being processed more than once.
% At this stage, the main document command |\childdocmain|
% is assumed to be called once again where it should do nothing.
% Any subsequent call to it should prevent
% a secondary processing of the main document
% It overwrites the forwarding commands
% |\childdocof| and |\childdocforward|
% with empty macros to prevent further inclusions of the main document:
%    \begin{macrocode}
\newcommand{\childdocdisable}
{
  \renewcommand{\childdocmain}[1]{\renewcommand{\childdocmain}[1]{\endinput}}
  \renewcommand{\childdocof}[1]{}
  \renewcommand{\childdocby}[2][]{}
  \renewcommand{\childdocforward}[2][]{}
  \renewcommand{\childdocdisable}{}
}
%    \end{macrocode}

% \macro{\childdocmain}
% The macro |\childdocmain| is to be called at the top of the main file
% with nothing or the main filename (without extension) as argument.
% First, it breaks loops.
% If the argument is not empty and does not match |\childdocname|
% (which is set by the first inclusion of |childdoc.def|),
% |\ifchilddoc| is set to true, |\includeonly| is applied to the child file
% and |\jobname| is set to the main file
% (for proper handling of |.aux| files):
%    \begin{macrocode}
\newcommand{\childdocmain}[1]
{
  \childdocdisable\childdocmain{}
  \if?#1?\else
    \begingroup
      \def\childdoctmp{#1}
      \ifx\childdoctmp\childdocname
        \def\childdoctmp{}
      \else
        \def\childdoctmp
        {
          \childdoctrue
          \includeonly{\childdocname}
          \def\childdocjob{#1}
          \def\jobname{#1}
        }
      \fi
      \expandafter
    \endgroup
    \childdoctmp
  \fi
}
%    \end{macrocode}

% \macro{\childdocof}
% The command |\childdocof| redirects
% compilation to the main file |#1|.
%    \begin{macrocode}
\newcommand{\childdocof}[1]
{
  \childdocdisable
  \childdoctrue
  \includeonly{\childdocname}
  \def\jobname{#1}
  \def\childdocjob{#1}
  \input{#1}
}
%    \end{macrocode}

% \macro{\childdocby}
% The command |\childdocby| ....
%    \begin{macrocode}
\newcommand{\childdocby}[2][]
{
  \childdocdisable
  \childdoctrue
  \childdocmanualtrue
  \if?#1?\else
    \def\jobname{#2}
  \fi
  \def\childdocjob{#2}
  \input{#2}
  \endinput
}
%    \end{macrocode}

% \macro{\childdocforward}
% The command |\childdocforward| redirects
% compilation to the main file or
% (if the optional argument is given) a child file.
% Parameters are set as if the main file
% or a child file starting with |\childdocof| was compiled.
% Then compilation is handed over to the main file:
%    \begin{macrocode}
\newcommand{\childdocforward}[2][]
{
  \begingroup
    \if?#1?
      \def\childdoctmp
      {
        \def\childdocname{#2}
        \def\childdocjob{#2}
        \def\jobname{#2}
        \input{#2}
        \endinput
      }
    \else
      \def\childdoctmp
      {
        \childdocdisable
        \def\childdocname{#2}
        \childdoctrue
        \includeonly{#2}
        \def\childdocjob{#1}
        \def\jobname{#1}
        \input{#1}
        \endinput
      }
    \fi
    \expandafter
  \endgroup
  \childdoctmp
}
%    \end{macrocode}

% \macro{\childdocforwardprefix}
% The command |\childdocforwardprefix| redirects
% compilation to the main or a child file by means of a pattern.
% The prefix |#1| in the current filename is replaced by |#2|
% and the suffix of the current filename is kept
% (it is assumed that the filename does not contain the substring `|~~~|'
% which is used as a delimiter).
% Compilation is handed over to the new file by |\childdocforward|:
%    \begin{macrocode}
\newcommand{\childdocforwardprefix}[3][]
{
  \begingroup
    \def\childdocextract #2##1~~~{\def\childdoctmp{\childdocforward[#1]{#3##1}}}
    \expandafter\childdocextract\childdocname~~~
    \expandafter
  \endgroup
  \childdoctmp
}
%    \end{macrocode}

% \macro{\childdoc}
% The deprecated macro |\childdoc| is a legacy version of |\childdocmain|:
%    \begin{macrocode}
\newcommand{\childdoc}{\childdocmain}
%    \end{macrocode}

% \macro{\childdocredirect}
% The deprecated macro |\childdocredirect| is a legacy version
% of |\childdocforward| and |\childdocforwardprefix|:
%    \begin{macrocode}
\newcommand{\childdocredirect}[2][]
{
  \begingroup
    \if?#1?
      \def\childdoctmp{\childdocforward{#2}}
    \else
      \def\childdoctmp{\childdocforwardprefix{#1}{#2}}
    \fi
    \expandafter
  \endgroup
  \childdoctmp
}
%    \end{macrocode}

%\iffalse
%</package>
%\fi
%
\endinput
\childdocforward{cdocsch2}"|
% \end{tabular}
% \end{center}
% Note that the trailing backslash on each first line
% merely continues the input to the second line
% (for convenient cut ant paste).
% Furthermore, the command |latex| can be replaced by any
% of its alternative versions such as |pdflatex|.
%
% %%%%%%%%%%%%%%%%%%%%%%%%%%%%%%%%%%%%%%%%%%%%%%%%%%%%%%%%%%%%%%%%%%%%%%%%%%%%%%
% %%%%%%%%%%%%%%%%%%%%%%%%%%%%%%%%%%%%%%%%%%%%%%%%%%%%%%%%%%%%%%%%%%%%%%%%%%%%%%
% \section{Implementation}
%\iffalse
%<*package>
%\fi
%
% This section describes the definitions file |childdoc.def|.

% The definitions cannot be loaded using |\usepackage| or |\RequirePackage|
% which has a mechanism to prevent loading a style file more than once.
% When loading the definitions by means of |\input|
% multiple instances have to be prevented manually:
%\iffalse
%This code needs to be before the `\ProvidesFile' directive
%which is defined at the beginning of this file.
%Therefore it is also placed there and commented out here.
%</package>
%<*discard>
%\fi
%    \begin{macrocode}
\ifdefined\childdocmain\endinput\fi
%    \end{macrocode}
%\iffalse
%</discard>
%<*package>
%\fi
%
% \macro{\ifchilddoc}
% \macro{\ifchilddocmanual}
% The conditional |\ifchilddoc| tells whether a
% child (true) or main (false) document is being compiled.
% The conditional |\ifchilddocmanual| tells whether
% the |\includeonly| mechanism is used (false) or
% the selection of child files must be performed manually (true).
% The definitions initialise to false:
%    \begin{macrocode}
\newif\ifchilddoc
\newif\ifchilddocmanual
%    \end{macrocode}

% \macro{\childdocname}
% \macro{\childdocjob}
% The macro |\childdocname| stores the name of the main document
% to be compiled. The macro |\childdocjob| stores the name of
% the document on which the \LaTeX{} compiler was originally invoked.
% The content of |\jobname| cannot be compared
% to filenames specified in the source due to different catcodes.
% The following code rescans |\jobname|, stores the result
% in |\childdocname| and saves a copy in |\childdocjob|:
%    \begin{macrocode}
\edef\childdocname{\scantokens\expandafter{\jobname\noexpand}}
\let\childdocjob\childdocname
%    \end{macrocode}

% \macro{\childdocdisable}
% The macro |\childdocdisable| prevents the main file
% from being processed more than once.
% At this stage, the main document command |\childdocmain|
% is assumed to be called once again where it should do nothing.
% Any subsequent call to it should prevent
% a secondary processing of the main document
% It overwrites the forwarding commands
% |\childdocof| and |\childdocforward|
% with empty macros to prevent further inclusions of the main document:
%    \begin{macrocode}
\newcommand{\childdocdisable}
{
  \renewcommand{\childdocmain}[1]{\renewcommand{\childdocmain}[1]{\endinput}}
  \renewcommand{\childdocof}[1]{}
  \renewcommand{\childdocby}[2][]{}
  \renewcommand{\childdocforward}[2][]{}
  \renewcommand{\childdocdisable}{}
}
%    \end{macrocode}

% \macro{\childdocmain}
% The macro |\childdocmain| is to be called at the top of the main file
% with nothing or the main filename (without extension) as argument.
% First, it breaks loops.
% If the argument is not empty and does not match |\childdocname|
% (which is set by the first inclusion of |childdoc.def|),
% |\ifchilddoc| is set to true, |\includeonly| is applied to the child file
% and |\jobname| is set to the main file
% (for proper handling of |.aux| files):
%    \begin{macrocode}
\newcommand{\childdocmain}[1]
{
  \childdocdisable\childdocmain{}
  \if?#1?\else
    \begingroup
      \def\childdoctmp{#1}
      \ifx\childdoctmp\childdocname
        \def\childdoctmp{}
      \else
        \def\childdoctmp
        {
          \childdoctrue
          \includeonly{\childdocname}
          \def\childdocjob{#1}
          \def\jobname{#1}
        }
      \fi
      \expandafter
    \endgroup
    \childdoctmp
  \fi
}
%    \end{macrocode}

% \macro{\childdocof}
% The command |\childdocof| redirects
% compilation to the main file |#1|.
%    \begin{macrocode}
\newcommand{\childdocof}[1]
{
  \childdocdisable
  \childdoctrue
  \includeonly{\childdocname}
  \def\jobname{#1}
  \def\childdocjob{#1}
  \input{#1}
}
%    \end{macrocode}

% \macro{\childdocby}
% The command |\childdocby| ....
%    \begin{macrocode}
\newcommand{\childdocby}[2][]
{
  \childdocdisable
  \childdoctrue
  \childdocmanualtrue
  \if?#1?\else
    \def\jobname{#2}
  \fi
  \def\childdocjob{#2}
  \input{#2}
  \endinput
}
%    \end{macrocode}

% \macro{\childdocforward}
% The command |\childdocforward| redirects
% compilation to the main file or
% (if the optional argument is given) a child file.
% Parameters are set as if the main file
% or a child file starting with |\childdocof| was compiled.
% Then compilation is handed over to the main file:
%    \begin{macrocode}
\newcommand{\childdocforward}[2][]
{
  \begingroup
    \if?#1?
      \def\childdoctmp
      {
        \def\childdocname{#2}
        \def\childdocjob{#2}
        \def\jobname{#2}
        \input{#2}
        \endinput
      }
    \else
      \def\childdoctmp
      {
        \childdocdisable
        \def\childdocname{#2}
        \childdoctrue
        \includeonly{#2}
        \def\childdocjob{#1}
        \def\jobname{#1}
        \input{#1}
        \endinput
      }
    \fi
    \expandafter
  \endgroup
  \childdoctmp
}
%    \end{macrocode}

% \macro{\childdocforwardprefix}
% The command |\childdocforwardprefix| redirects
% compilation to the main or a child file by means of a pattern.
% The prefix |#1| in the current filename is replaced by |#2|
% and the suffix of the current filename is kept
% (it is assumed that the filename does not contain the substring `|~~~|'
% which is used as a delimiter).
% Compilation is handed over to the new file by |\childdocforward|:
%    \begin{macrocode}
\newcommand{\childdocforwardprefix}[3][]
{
  \begingroup
    \def\childdocextract #2##1~~~{\def\childdoctmp{\childdocforward[#1]{#3##1}}}
    \expandafter\childdocextract\childdocname~~~
    \expandafter
  \endgroup
  \childdoctmp
}
%    \end{macrocode}

% \macro{\childdoc}
% The deprecated macro |\childdoc| is a legacy version of |\childdocmain|:
%    \begin{macrocode}
\newcommand{\childdoc}{\childdocmain}
%    \end{macrocode}

% \macro{\childdocredirect}
% The deprecated macro |\childdocredirect| is a legacy version
% of |\childdocforward| and |\childdocforwardprefix|:
%    \begin{macrocode}
\newcommand{\childdocredirect}[2][]
{
  \begingroup
    \if?#1?
      \def\childdoctmp{\childdocforward{#2}}
    \else
      \def\childdoctmp{\childdocforwardprefix{#1}{#2}}
    \fi
    \expandafter
  \endgroup
  \childdoctmp
}
%    \end{macrocode}

%\iffalse
%</package>
%\fi
%
\endinput
|\\
|\childdocmain{}|\\
\end{tabular}
\end{center}
at the very top of the main \LaTeX{} file,
in particular \emph{before} the |\documentclass| statement!
The argument of |\childdocmain| should be left empty
(but it must be present).

%%%%%%%%%%%%%%%%%%%%%%%%%%%%%%%%%%%%%%%%
\DescribeMacro{\childdocof}
Furthermore, add the commands
\begin{center}
\begin{tabular}{l}
|% \iffalse
%
% childdoc.dtx Copyright (C) 2017-2018 Niklas Beisert
%
% This work may be distributed and/or modified under the
% conditions of the LaTeX Project Public License, either version 1.3
% of this license or (at your option) any later version.
% The latest version of this license is in
%   http://www.latex-project.org/lppl.txt
% and version 1.3 or later is part of all distributions of LaTeX
% version 2005/12/01 or later.
%
% This work has the LPPL maintenance status `maintained'.
%
% The Current Maintainer of this work is Niklas Beisert.
%
% This work consists of the files childdoc.dtx and childdoc.ins
% and the derived files childdoc.def and cdocsamp.tex with
% cdocsch1.tex, cdocsch2.tex, cdocsdrf.tex, cdocsfn1.tex, cdocsfn2.tex.
%
%<package>\ifdefined\childdocmain\endinput\fi
%<package>\ProvidesFile{childdoc.def}[2018/12/30 v2.0 child document driver]
%<samplemain>\ProvidesFile{cdocsamp.tex}[2018/12/30 v2.0 sample for childdoc]
%<*driver>
%\ProvidesFile{childdoc.drv}[2018/12/30 v2.0 childdoc reference manual file]
\PassOptionsToClass{10pt,a4paper}{article}
\documentclass{ltxdoc}

\usepackage[margin=35mm]{geometry}
\usepackage{hyperref}
\usepackage{hyperxmp}
\usepackage[usenames]{color}

\hypersetup{colorlinks=true}
\hypersetup{pdfstartview=FitH}
\hypersetup{pdfpagemode=UseNone}
\hypersetup{pdfsource={}}
\hypersetup{pdflang={en-UK}}
\hypersetup{pdfcopyright={Copyright 2017-2018 Niklas Beisert.
  This work may be distributed and/or modified under the
  conditions of the LaTeX Project Public License, either version 1.3
  of this license or (at your option) any later version.}}
\hypersetup{pdflicenseurl={http://www.latex-project.org/lppl.txt}}
\hypersetup{pdfcontactaddress={ETH Zurich, ITP, HIT K,
  Wolfgang-Pauli-Strasse 27}}
\hypersetup{pdfcontactpostcode={8093}}
\hypersetup{pdfcontactcity={Zurich}}
\hypersetup{pdfcontactcountry={Switzerland}}
\hypersetup{pdfcontactemail={nbeisert@itp.phys.ethz.ch}}
\hypersetup{pdfcontacturl={http://people.phys.ethz.ch/\xmptilde nbeisert/}}

\newcommand{\secref}[1]{\hyperref[#1]{section \ref*{#1}}}

\parskip1ex
\parindent0pt
\let\olditemize\itemize
\def\itemize{\olditemize\parskip0pt}

\begin{document}

\title{The \textsf{childdoc} Package}
\hypersetup{pdftitle={The childdoc Package}}
\author{Niklas Beisert\\[2ex]
  Institut f\"ur Theoretische Physik\\
  Eidgen\"ossische Technische Hochschule Z\"urich\\
  Wolfgang-Pauli-Strasse 27, 8093 Z\"urich, Switzerland\\[1ex]
  \href{mailto:nbeisert@itp.phys.ethz.ch}
  {\texttt{nbeisert@itp.phys.ethz.ch}}}
\hypersetup{pdfauthor={Niklas Beisert}}
\hypersetup{pdfsubject={Manual for the LaTeX2e Package childdoc}}
\date{30 December 2018, \textsf{v2.0}}
\maketitle

\begin{abstract}\noindent
\textsf{childdoc} is a \LaTeXe{} package
that enables the direct compilation
of document sections included by |\include|
to individual files.
\end{abstract}

\begingroup
\parskip0ex
\tableofcontents
\endgroup

%%%%%%%%%%%%%%%%%%%%%%%%%%%%%%%%%%%%%%%%%%%%%%%%%%%%%%%%%%%%%%%%%%%%%%%%%%%%%%%%
%%%%%%%%%%%%%%%%%%%%%%%%%%%%%%%%%%%%%%%%%%%%%%%%%%%%%%%%%%%%%%%%%%%%%%%%%%%%%%%%
\section{Introduction}

\LaTeX{} provides a mechanism to structure a large document (such as a book)
into a main file and several child files (containing the chapters)
using the |\include| command.
This mechanism is beneficial for documents
which span hundreds of pages in order to
make the source file(s) more manageable.
Moreover, compilation can be restricted to
selected child files by means of the |\includeonly| command.
The latter feature can be used to reduce the compilation time while editing
(this was significantly more useful in the earlier days of \LaTeX{})
or to generate a smaller document which is easier to navigate.
Another application of |\includeonly| is to generate
documents consisting of selected parts of the complete document.

However, there are a few drawbacks of the plain |\include| mechanism:
\begin{itemize}
\item
The child files cannot be compiled on their own,
they can only be compiled via the main file.
A naive editing environment
(such as a text editor with an option
to have the current file processed by \LaTeX)
may require one to switch to the main file before compiling;
attempting to compile the child file produces errors.
\item
The main file must be modified (each time)
to adjust the |\includeonly| command
to the present needs. This easily leaves the main file in a messy state.
\item
The generated document will always carry the filename
of the main document. This is inconvenient if
several child files are to be compiled and
to be kept for distribution.
\end{itemize}

The present package provides a simple interface
to make child files individually compilable by \LaTeX{}.
Compiling a child file then has the same effect as compiling
the main file with an |\includeonly| command
to select the appropriate child.
Moreover the generated document will carry the name of the child
rather than the main file.
This resolves all three above issues.

This feature is meant to make the editing of books,
thesis documents and lecture notes somewhat more convenient.
However, the package can also be used efficiently for
composing a series of documents (such as exercise sheets)
which are typically distributed individually.
It then assists the author in generating the individual documents
(potentially in different versions)
as well as a document containing the collected series.
Another application is in developing style files
or other kinds of included material
where compilation of the style file could redirect
to a sample or test file.

%%%%%%%%%%%%%%%%%%%%%%%%%%%%%%%%%%%%%%%%%%%%%%%%%%%%%%%%%%%%%%%%%%%%%%%%%%%%%%%%
%%%%%%%%%%%%%%%%%%%%%%%%%%%%%%%%%%%%%%%%%%%%%%%%%%%%%%%%%%%%%%%%%%%%%%%%%%%%%%%%
\section{Usage}

First of all, the package \textsf{childdoc} is \emph{not} a standard
\LaTeXe{} |.sty| style file! Therefore it needs to be invoked in
a non-standard way.

%%%%%%%%%%%%%%%%%%%%%%%%%%%%%%%%%%%%%%%%%%%%%%%%%%%%%%%%%%%%%%%%%%%%%%%%%%%%%%%%
\subsection{Included Files}
\label{sec:include}

%%%%%%%%%%%%%%%%%%%%%%%%%%%%%%%%%%%%%%%%
\DescribeMacro{\childdocmain}
To use the package, add the commands
\begin{center}
\begin{tabular}{l}
|% \iffalse
%
% childdoc.dtx Copyright (C) 2017-2018 Niklas Beisert
%
% This work may be distributed and/or modified under the
% conditions of the LaTeX Project Public License, either version 1.3
% of this license or (at your option) any later version.
% The latest version of this license is in
%   http://www.latex-project.org/lppl.txt
% and version 1.3 or later is part of all distributions of LaTeX
% version 2005/12/01 or later.
%
% This work has the LPPL maintenance status `maintained'.
%
% The Current Maintainer of this work is Niklas Beisert.
%
% This work consists of the files childdoc.dtx and childdoc.ins
% and the derived files childdoc.def and cdocsamp.tex with
% cdocsch1.tex, cdocsch2.tex, cdocsdrf.tex, cdocsfn1.tex, cdocsfn2.tex.
%
%<package>\ifdefined\childdocmain\endinput\fi
%<package>\ProvidesFile{childdoc.def}[2018/12/30 v2.0 child document driver]
%<samplemain>\ProvidesFile{cdocsamp.tex}[2018/12/30 v2.0 sample for childdoc]
%<*driver>
%\ProvidesFile{childdoc.drv}[2018/12/30 v2.0 childdoc reference manual file]
\PassOptionsToClass{10pt,a4paper}{article}
\documentclass{ltxdoc}

\usepackage[margin=35mm]{geometry}
\usepackage{hyperref}
\usepackage{hyperxmp}
\usepackage[usenames]{color}

\hypersetup{colorlinks=true}
\hypersetup{pdfstartview=FitH}
\hypersetup{pdfpagemode=UseNone}
\hypersetup{pdfsource={}}
\hypersetup{pdflang={en-UK}}
\hypersetup{pdfcopyright={Copyright 2017-2018 Niklas Beisert.
  This work may be distributed and/or modified under the
  conditions of the LaTeX Project Public License, either version 1.3
  of this license or (at your option) any later version.}}
\hypersetup{pdflicenseurl={http://www.latex-project.org/lppl.txt}}
\hypersetup{pdfcontactaddress={ETH Zurich, ITP, HIT K,
  Wolfgang-Pauli-Strasse 27}}
\hypersetup{pdfcontactpostcode={8093}}
\hypersetup{pdfcontactcity={Zurich}}
\hypersetup{pdfcontactcountry={Switzerland}}
\hypersetup{pdfcontactemail={nbeisert@itp.phys.ethz.ch}}
\hypersetup{pdfcontacturl={http://people.phys.ethz.ch/\xmptilde nbeisert/}}

\newcommand{\secref}[1]{\hyperref[#1]{section \ref*{#1}}}

\parskip1ex
\parindent0pt
\let\olditemize\itemize
\def\itemize{\olditemize\parskip0pt}

\begin{document}

\title{The \textsf{childdoc} Package}
\hypersetup{pdftitle={The childdoc Package}}
\author{Niklas Beisert\\[2ex]
  Institut f\"ur Theoretische Physik\\
  Eidgen\"ossische Technische Hochschule Z\"urich\\
  Wolfgang-Pauli-Strasse 27, 8093 Z\"urich, Switzerland\\[1ex]
  \href{mailto:nbeisert@itp.phys.ethz.ch}
  {\texttt{nbeisert@itp.phys.ethz.ch}}}
\hypersetup{pdfauthor={Niklas Beisert}}
\hypersetup{pdfsubject={Manual for the LaTeX2e Package childdoc}}
\date{30 December 2018, \textsf{v2.0}}
\maketitle

\begin{abstract}\noindent
\textsf{childdoc} is a \LaTeXe{} package
that enables the direct compilation
of document sections included by |\include|
to individual files.
\end{abstract}

\begingroup
\parskip0ex
\tableofcontents
\endgroup

%%%%%%%%%%%%%%%%%%%%%%%%%%%%%%%%%%%%%%%%%%%%%%%%%%%%%%%%%%%%%%%%%%%%%%%%%%%%%%%%
%%%%%%%%%%%%%%%%%%%%%%%%%%%%%%%%%%%%%%%%%%%%%%%%%%%%%%%%%%%%%%%%%%%%%%%%%%%%%%%%
\section{Introduction}

\LaTeX{} provides a mechanism to structure a large document (such as a book)
into a main file and several child files (containing the chapters)
using the |\include| command.
This mechanism is beneficial for documents
which span hundreds of pages in order to
make the source file(s) more manageable.
Moreover, compilation can be restricted to
selected child files by means of the |\includeonly| command.
The latter feature can be used to reduce the compilation time while editing
(this was significantly more useful in the earlier days of \LaTeX{})
or to generate a smaller document which is easier to navigate.
Another application of |\includeonly| is to generate
documents consisting of selected parts of the complete document.

However, there are a few drawbacks of the plain |\include| mechanism:
\begin{itemize}
\item
The child files cannot be compiled on their own,
they can only be compiled via the main file.
A naive editing environment
(such as a text editor with an option
to have the current file processed by \LaTeX)
may require one to switch to the main file before compiling;
attempting to compile the child file produces errors.
\item
The main file must be modified (each time)
to adjust the |\includeonly| command
to the present needs. This easily leaves the main file in a messy state.
\item
The generated document will always carry the filename
of the main document. This is inconvenient if
several child files are to be compiled and
to be kept for distribution.
\end{itemize}

The present package provides a simple interface
to make child files individually compilable by \LaTeX{}.
Compiling a child file then has the same effect as compiling
the main file with an |\includeonly| command
to select the appropriate child.
Moreover the generated document will carry the name of the child
rather than the main file.
This resolves all three above issues.

This feature is meant to make the editing of books,
thesis documents and lecture notes somewhat more convenient.
However, the package can also be used efficiently for
composing a series of documents (such as exercise sheets)
which are typically distributed individually.
It then assists the author in generating the individual documents
(potentially in different versions)
as well as a document containing the collected series.
Another application is in developing style files
or other kinds of included material
where compilation of the style file could redirect
to a sample or test file.

%%%%%%%%%%%%%%%%%%%%%%%%%%%%%%%%%%%%%%%%%%%%%%%%%%%%%%%%%%%%%%%%%%%%%%%%%%%%%%%%
%%%%%%%%%%%%%%%%%%%%%%%%%%%%%%%%%%%%%%%%%%%%%%%%%%%%%%%%%%%%%%%%%%%%%%%%%%%%%%%%
\section{Usage}

First of all, the package \textsf{childdoc} is \emph{not} a standard
\LaTeXe{} |.sty| style file! Therefore it needs to be invoked in
a non-standard way.

%%%%%%%%%%%%%%%%%%%%%%%%%%%%%%%%%%%%%%%%%%%%%%%%%%%%%%%%%%%%%%%%%%%%%%%%%%%%%%%%
\subsection{Included Files}
\label{sec:include}

%%%%%%%%%%%%%%%%%%%%%%%%%%%%%%%%%%%%%%%%
\DescribeMacro{\childdocmain}
To use the package, add the commands
\begin{center}
\begin{tabular}{l}
|\input{childdoc.def}|\\
|\childdocmain{}|\\
\end{tabular}
\end{center}
at the very top of the main \LaTeX{} file,
in particular \emph{before} the |\documentclass| statement!
The argument of |\childdocmain| should be left empty
(but it must be present).

%%%%%%%%%%%%%%%%%%%%%%%%%%%%%%%%%%%%%%%%
\DescribeMacro{\childdocof}
Furthermore, add the commands
\begin{center}
\begin{tabular}{l}
|\input{childdoc.def}|\\
|\childdocof{|\textit{main}|}|\\
\end{tabular}
\end{center}
at the top of every child file \textit{child}
which is included by |\include{|\textit{child}|}|
from within the main file
(or at least for those files to be compiled individually).
The argument \textit{main} must be the filename of the main file.

There are a couple of
considerations in setting up the main and child documents:

%%%%%%%%%%%%%%%%%%%%%%%%%%%%%%%%%%%%%%%%
\paragraph{Restrictions.}

Please note the following restrictions:
\begin{itemize}
\item
|\childdocmain| must be called with one argument \textit{main}
to ensure compatibility with earlier version of the package.
It must either be empty (|\childdocmain{}|)
or precisely match the filename of the main file in which it is specified.
See \secref{sec:detection} for further information.
\item
The filename \textit{main} must be specified without the |.tex| extension.
\item
The filename \textit{main} is case sensitive
(even in case-insensitive file systems)
due to internal string comparison.
\item
The argument \textit{main} should be fully expanded, it cannot be a macro.
\item
Subdirectories and special characters should be avoided in filenames.
\item
The command |\childdocmain{|\textit{main}|}| must be followed by a whitespace.
It should not be followed immediately by another command
or by a comment mark `|%|'.
This is because the \TeX{} parser reads the token immediately following
the argument of |\childdocmain| and puts it
at the beginning of every child section;
however, a white\-space is ignored.
\end{itemize}

%%%%%%%%%%%%%%%%%%%%%%%%%%%%%%%%%%%%%%%%
\paragraph{Content of Main File.}

It is advisable to place all content in the child files included by |\include|.
Any output contained in the main file will appear in all child documents
unless suppressed manually;
it cannot be suppressed automatically by the |\includeonly| directive
and thus should normally be avoided.
A method to include some content in the main file
by means of conditional processing is described in \secref{sec:conditional}.

%%%%%%%%%%%%%%%%%%%%%%%%%%%%%%%%%%%%%%%%
\paragraph{Page Numbering.}

When only a part of the document is compiled,
the appropriate numbering of pages
(as well as other status parameters)
is determined from the |.aux| files.
The latter contain information from previous passes.
However this information needs to propagate through
all intermediate child documents.
Therefore the page numbering in child documents may well
be inconsistent until the complete document is compiled at least once.

A useful (if unconventional) way to always ensure a consistent
page numbering is to restart the numbering in each child document
and denote the pages by `\textit{child}|.|\textit{page}'
where \textit{child} represents the chapter/section number of the child file.
This can be achieved by the command
|\numberwithin{page}{|\textit{child}|}|
of the \textsf{amsmath} package
where \textit{child} can be |chapter| or |section|
depending on the chosen structuring.
Alternatively, one can modify the macro |\thepage| appropriately
and reset the counter |page| at the start of each child file.

%%%%%%%%%%%%%%%%%%%%%%%%%%%%%%%%%%%%%%%%%%%%%%%%%%%%%%%%%%%%%%%%%%%%%%%%%%%%%%%%
\subsection{Conditional Processing}
\label{sec:conditional}

The package provides a mechanism to compile different versions
of a document. To customise the versions further some conditional processing
can come in handy to distinguish which version is being compiled.
The package provides two macros to describe the compilation context:

%%%%%%%%%%%%%%%%%%%%%%%%%%%%%%%%%%%%%%%%
\DescribeMacro{\ifchilddoc}
The conditional |\ifchilddoc| distinguishes between the compilation of
child documents and the main document:
%
\begin{center}
|\ifchilddoc |\textit{child-code}| |[|\||else |\textit{main-code}]| \||fi|
\end{center}

%%%%%%%%%%%%%%%%%%%%%%%%%%%%%%%%%%%%%%%%
\DescribeMacro{\childdocname}
\DescribeMacro{\childdocjob}
The macro |\childdocname| contains the filename (without extension)
of the main or child file being processed.
Note that |\childdocjob| will always contain the name of the main file.

%%%%%%%%%%%%%%%%%%%%%%%%%%%%%%%%%%%%%%%%
\paragraph{Title Page.}

Conditional processing can be used to include a title or banner page
in the main document when proper precautions are taken.
Importantly, the code in the main file should ensure that the page counter
(as well as other status parameters which are stored in the |.aux| files)
takes the same value after the conditional processing.
Otherwise the page numbers may take divergent values
depending on which part is compiled.

For example, a title page could be declared by:
%
\begin{center}
\begin{tabular}{l}
|\ifchilddoc\||else|\\
|\addtocounter{page}{-1}|\\
\textit{code for title page}\\
|\newpage|\\
|\||fi|
\end{tabular}
\end{center}
%
A banner page for the child documents can be generated by:
%
\begin{center}
\begin{tabular}{l}
|\ifchilddoc|\\
|\addtocounter{page}{-1}|\\
\textit{code for banner page}\\
|\newpage|\\
|\||fi|
\end{tabular}
\end{center}
%
Here one could write a message such as:
\begin{center}
|This is the part \childdocname{} of \childdocjob{}.|
\end{center}

%%%%%%%%%%%%%%%%%%%%%%%%%%%%%%%%%%%%%%%%%%%%%%%%%%%%%%%%%%%%%%%%%%%%%%%%%%%%%%%%
\subsection{Flags}
\label{sec:flags}

The package makes it easy to generate different versions
of the main or child documents.
To this end compilation flags can be defined
and assigned different default values.
They will be particularly useful in conjunction
with the forwarding mechanism described in \secref{sec:forward}.

For example, it may be useful to have a flag |\version|
which can be set to |draft| or |final|.
The document source will contain some conditional code
depending on the value of |\version|.
Suppose further, the flag should default to |final| for the main file
and to |draft| for child files
which is a natural assignment for editing the document.
This is achieved by placing the following code
in the preamble of the main document
(below the |\childdocmain| directive):
%
\begin{center}
\begin{tabular}{l}
|\ifchilddoc|\\
|\providecommand{\version}{draft}|\\
|\||else|\\
|\providecommand{\version}{final}|\\
|\||fi|
\end{tabular}
\end{center}
%
The definition by |\providecommand| makes sure
that previous definitions are not overwritten.
Further statements |\providecommand{\version}{...}|
can thus be added before the above code to override it.

For the main file, one might add a line
(between |\childdocmain| and the above block)
%
\begin{center}
|%\ifchilddoc\||else\providecommand{\version}{draft}\||fi|
\end{center}
%
which can be uncommented to produce a draft version.
Likewise one can add a line to the very top of a child file
(above the |\childdocof{|\textit{main}|}| directive)
%
\begin{center}
|%\providecommand{\version}{final}|
\end{center}
%
which can be uncommented to produce the final version of this child document.

%%%%%%%%%%%%%%%%%%%%%%%%%%%%%%%%%%%%%%%%%%%%%%%%%%%%%%%%%%%%%%%%%%%%%%%%%%%%%%%%
\subsection{Forwarding}
\label{sec:forward}

Different versions of the main or child documents
using compilation flags as described in \secref{sec:flags}
can be (permanently) stored in different files
for convenient compilation, viewing and distribution.
To this end, the package defines a command
to pass on compilation to a different file:

%%%%%%%%%%%%%%%%%%%%%%%%%%%%%%%%%%%%%%%%
\DescribeMacro{\childdocforward}
The command |\childdocforward| redirects processing to
another source file:
%
\begin{center}
\begin{tabular}{l}
|\input{childdoc.def}|\\
|\childdocforward[|\textit{main}|]{|\textit{dest}|}|\\
\end{tabular}
\end{center}
%
The argument \textit{dest} is the destination file
(without extension).
It should be the main file or one of the child files.
Note that further \textsf{childdoc} directives
such as |\childdocof| and |\childdocforward|
in the indicated file will be processed in this form.
The optional argument \textit{main}
passes on directly to the main file \textit{main}
while pretending to compile the child \textit{dest}.
This form behaves as if \textit{dest}
issues |\childdocof{|\textit{main}|}| right away,
and no further \textsf{childdoc} directives will be processed.

%%%%%%%%%%%%%%%%%%%%%%%%%%%%%%%%%%%%%%%%
\DescribeMacro{\...prefix}
In the alternative form |\childdocforwardprefix|,
%
\begin{center}
\begin{tabular}{l}
|\input{childdoc.def}|\\
|\childdocforwardprefix[|\textit{main}|]{|\textit{prefix}|}{|\textit{dest}|}|
\end{tabular}
\end{center}
%
the destination file is determined by a pattern
depending on the current file:
To make this work, the current file must be called
`{\textit{prefix}\hspace{0.2em}\textit{suffix}}'
with \textit{prefix} matching precisely the argument.
Processing is then passed on to the file
`{\textit{dest}\hspace{0.2em}\textit{suffix}}'.
Surely, the same effect is achieved by
directly specifying the
argument `{\textit{dest}\hspace{0.2em}\textit{suffix}}'
in the first form.
However, that requires to set up a different file
for each child. With the alternative form of the command
all these files can have exactly the same content
which simplifies setting them up and maintaining them.

For example, the following file |draft.tex|
with a compilation flag |\version| as described in \secref{sec:flags}
compiles the main document as a draft:
%
\begin{center}
\begin{tabular}{l}
|\def\version{draft}|\\
|\input{childdoc.def}|\\
|\childdocforward{|\textit{main}|}|
\end{tabular}
\end{center}
%
Likewise, the following files |final|\textit{nn}|.tex|
compile the final version of the child document
|child|\textit{nn}|.tex|:
%
\begin{center}
\begin{tabular}{l}
|\def\version{final}|\\
|\input{childdoc.def}|\\
|\childdocforwardprefix{final}{child}|
\end{tabular}
\end{center}
%

Note that when several versions of a main file and/or of each child file
are to be generated, it may be convenient to set up a |Makefile| or
shell script to automatise the process.

%%%%%%%%%%%%%%%%%%%%%%%%%%%%%%%%%%%%%%%%%%%%%%%%%%%%%%%%%%%%%%%%%%%%%%%%%%%%%%%%
\subsection{Command Line Processing}
\label{sec:commandline}

The effect of redirection files can also be achieved by invoking
the \LaTeX{} compiler with a more elaborate command line.
Most conveniently this should be done as part
of a shell script or a |Makefile|.

When using \textsf{childdoc} in the main file, the following
command lines effectively perform a redirection
(note that depending on the shell being used,
backslashes may have to be doubled: `|\|' $\to$ `|\\|'):
%
\begin{center}
|... -jobname "|\textit{target}|" |\\|"|[\textit{flags}]%
|\input{childdoc.def}\childdocforward[|\textit{main}|]{|\textit{dest}|}"|
\end{center}
%
Here \textit{target} is the name of the output file,
\textit{main} is the name of the main file
and \textit{dest} is the name of the main or child file to be processed
(all filenames without extensions).
The optional argument \textit{main} can be omitted
if \textit{main} matches \textit{dest}.
Optionally, compilation \textit{flags} can be defined via |\def| commands.
This command line makes the \TeX{} engine believe
it is compiling the file \textit{target}
whose content is specified as the latter parameter.
The provided code then forwards the processing to
\textit{main} or \textit{dest} as described in \secref{sec:forward}.

%%%%%%%%%%%%%%%%%%%%%%%%%%%%%%%%%%%%%%%%%%%%%%%%%%%%%%%%%%%%%%%%%%%%%%%%%%%%%%%%
\subsection{Include by Input}
\label{sec:input}

Including child documents by |\include| has some restrictions by design.
Most notably, the content of a child document always occupies
its own set of pages; pages cannot be shared between child documents.
Usually, this behaviour makes perfect sense
because each child document contain an essential part of the document.
However, in some situations it may be desirable to compose
a document from a collection of parts
without having mandatory page breaks between then.
For this case, the package
provides a mechanism to include parts
by |\input| which can also be processed individually.
However, by construction this mechanism
requires manual handling of the content to be output.

%%%%%%%%%%%%%%%%%%%%%%%%%%%%%%%%%%%%%%%%
\DescribeMacro{\ifchilddocmanual}
The main file should be prepared as usual, see \secref{sec:include}.
However, the document body must make a distinction
between processing of an individual part and of the main document, e.g.:
%
\begin{center}
\begin{tabular}{l}
|\ifchilddocmanual|\\
|\input{\childdocname}|\\
|\||else|\\
\textit{document body with }|\input{|\textit{part}|}|\\
|\||fi|
\end{tabular}
\end{center}
%
The conditional |\ifchilddocmanual| is true whenever
a part to be included by |\input| is being compiled,
and the name of the part is stored in |\childdocname|.

%%%%%%%%%%%%%%%%%%%%%%%%%%%%%%%%%%%%%%%%
\DescribeMacro{\childdocby}
Each part to be included by |\input| should start with:
%
\begin{center}
\begin{tabular}{l}
|\input{childdoc.def}|\\
|\childdocby{|\textit{main}|}|\\
\end{tabular}
\end{center}
%
The directive |\childdocby| is similar to |\childdocof|
described in \secref{sec:include},
but the subsequent selection of content must be done manually.
To that end, both |\ifchilddoc| and |\ifchilddocmanual|
will be true upon processing of a part,
and the name of the part is stored in |\childdocname|.
Note that |\jobname| will be set to the filename of the current part
so that each part receives an individual |.aux| file
that does not interfere with the |.aux| file(s) of the main document.
This behaviour can be altered by the alternative form
|\childdocby[*]{|\textit{main}|}| (with a non-empty optional argument)
which uses the |.aux| file of the main document
by setting |\jobname| to \textit{main}.

%%%%%%%%%%%%%%%%%%%%%%%%%%%%%%%%%%%%%%%%%%%%%%%%%%%%%%%%%%%%%%%%%%%%%%%%%%%%%%%%
\subsection{Driver Development}
\label{sec:driver}

The \textsf{childdoc} mechanism can also be use for the development
of definition files such as \LaTeX{} styles or classes.
This case differs from the above setup with multiple parts
included by |\include| in that no |\includeonly| should be invoked.
This can be achieved by starting the include file
(before |\ProvidesPackage|) with:
%
\begin{center}
\begin{tabular}{l}
|\input{childdoc.def}|\\
|\childdocforward{|\textit{main}|}|\\
\end{tabular}
\end{center}
%
or alternatively with:
%
\begin{center}
\begin{tabular}{l}
|\input{childdoc.def}|\\
|\childdocby{|\textit{main}|}|\\
\end{tabular}
\end{center}
%
Both forms have slightly different effects as described above.
The main file is prepared as usual, see \secref{sec:include}.

%%%%%%%%%%%%%%%%%%%%%%%%%%%%%%%%%%%%%%%%%%%%%%%%%%%%%%%%%%%%%%%%%%%%%%%%%%%%%%%%
\subsection{Legacy Detection}
\label{sec:detection}

The directive |\childdocmain| in the main file can detect
whether the complete document or merely a child is to be compiled
even without using the directive |\childdocof|.
This method is deprecated because it is less robust
and there is no compelling reason to use it;
it is merely provided for backward compatibility
and it may be removed in future versions.

If the detection mechanism is to be used,
it is mandatory to correctly specify
the filename of the main file as the argument of |\childdocmain|:
%
\begin{center}
\begin{tabular}{l}
|\input{childdoc.def}|\\
|\childdocmain{|\textit{main}|}|\\
\end{tabular}
\end{center}
%
If |\jobname| does not match the argument \textit{main} of |\childdocmain|,
it is assumed that |\jobname| points to the child file to be compiled.
When using |\childdocmain| with the main file specified as argument,
it suffices to start a child file
with just |\input{|\textit{main}|}|
without loading of the package and using |\childdocof|.
If instead all processing is done
with the appropriate \textsf{childdoc} directives,
the argument of \textit{main} of |\childdocmain| can be empty.

An alternative version of the command line processing described
in \secref{sec:commandline} using the detection mechanism reads:
%
\begin{center}
|... -jobname "|\textit{target}|" "|[\textit{flags}]%
[|\def\jobname{|\textit{dest}|}|]|\input{|\textit{main}|}"|
\end{center}

%%%%%%%%%%%%%%%%%%%%%%%%%%%%%%%%%%%%%%%%%%%%%%%%%%%%%%%%%%%%%%%%%%%%%%%%%%%%%%%%
\subsection{Manual Code}
\label{sec:manual}

In case one cannot be certain whether the definitions file |childdoc.def|
is installed on the target \TeX{} distribution
and one prefers not to ship it,
it is conceivable to paste a few relevant commands into the sources.

To that end, drop all statements |\input{childdoc.def}|
and perform the replacements as outlined below.
Instead of |\childdocmain{|\textit{main}|}| add the following code
to the top of the main file:
%
\begin{center}
\begin{tabular}{l}
|\||ifdefined\childdocname\endinput\||fi\newif\ifchilddoc|\\
|\edef\childdocname{\scantokens\expandafter{\jobname\noexpand}}|\\
|\def\childdocmain{|\textit{main}|}\||ifx\childdocmain\childdocname\||else|\\
|\childdoctrue\includeonly{\childdocname}\let\jobname\childdocmain\||fi|\\
\end{tabular}
\end{center}
%
Instead of |\childdocof{|\textit{main}|}| just include the main file
at the top of each child file:
%
\begin{center}
|\input{|\textit{main}|}|
\end{center}
%
A simple redirection |\childdocforward{|\textit{dest}|}| is achieved by:
%
\begin{center}
|\def\jobname{|\textit{dest}|}\input{\jobname}|
\end{center}
%
The redirection with prefix
|\childdocforwardprefix[|\textit{prefix}|]{|\textit{dest}|}|
is accomplished by:
%
\begin{center}
\begin{tabular}{l}
|{\edef\jobname{\scantokens\expandafter{\jobname\noexpand}}|\\
|\def\redirectjob |\textit{prefix}|#1~~~{\gdef\jobname{|\textit{dest}|#1}}|\\
|\expandafter\redirectjob\jobname~~~}\input{\jobname}|
\end{tabular}
\end{center}

In an alternative approach,
child documents can be compiled by a specific command line
without additional code or specific definitions:
%
\begin{center}
|... -jobname "|\textit{target}|" "|[\textit{flags}]%
|\includeonly{|\textit{dest}|}\input{|\textit{main}|}"|
\end{center}
%

%%%%%%%%%%%%%%%%%%%%%%%%%%%%%%%%%%%%%%%%%%%%%%%%%%%%%%%%%%%%%%%%%%%%%%%%%%%%%%%%
%%%%%%%%%%%%%%%%%%%%%%%%%%%%%%%%%%%%%%%%%%%%%%%%%%%%%%%%%%%%%%%%%%%%%%%%%%%%%%%%
\section{Information}

%%%%%%%%%%%%%%%%%%%%%%%%%%%%%%%%%%%%%%%%%%%%%%%%%%%%%%%%%%%%%%%%%%%%%%%%%%%%%%%%
\subsection{Copyright}

Copyright \copyright{} 2017--2018 Niklas Beisert

This work may be distributed and/or modified under the
conditions of the \LaTeX{} Project Public License, either version 1.3
of this license or (at your option) any later version.
The latest version of this license is in
  \url{http://www.latex-project.org/lppl.txt}
and version 1.3 or later is part of all distributions of \LaTeX{}
version 2005/12/01 or later.

This work has the LPPL maintenance status `maintained'.

The Current Maintainer of this work is Niklas Beisert.

This work consists of the files |README.txt|, |childdoc.ins| and |childdoc.dtx|
as well as the derived files |childdoc.def|, |cdocsamp.tex|
with |cdocsch1.tex|, |cdocsch2.tex|, |cdocspt3.tex|, |cdocspt4.tex|,
|cdocsdrf.tex|, |cdocsfn1.tex|, |cdocsfn2.tex|
as well as |childdoc.pdf|.

%%%%%%%%%%%%%%%%%%%%%%%%%%%%%%%%%%%%%%%%%%%%%%%%%%%%%%%%%%%%%%%%%%%%%%%%%%%%%%%%
\subsection{Files and Installation}

The package consists of the files:
%
\begin{center}
\begin{tabular}{ll}
    |README.txt|   & readme file \\
    |childdoc.ins| & installation file \\
    |childdoc.dtx| & source file \\
    |childdoc.def| & definition file \\
    |cdocsamp.tex| & sample main file \\
    |cdocsch1.tex| & sample include file \\
    |cdocsch2.tex| & sample include file \\
    |cdocspt3.tex| & sample part file \\
    |cdocspt4.tex| & sample part file \\
    |cdocsdrf.tex| & sample redirection file \\
    |cdocsfn1.tex| & sample redirection file \\
    |cdocsfn2.tex| & sample redirection file \\
    |childdoc.pdf| & manual
\end{tabular}
\end{center}
%
The distribution consists of the files
|README.txt|, |childdoc.ins| and |childdoc.dtx|.
%
\begin{itemize}
\item
Run (pdf)\LaTeX{} on |childdoc.dtx|
to compile the manual |childdoc.pdf| (this file).
\item
Run \LaTeX{} on |childdoc.ins| to create the definitions file |childdoc.def|
and the sample |cdocsamp.tex| with include files
|cdocsch1.tex|, |cdocsch2.tex|, |cdocspt3.tex|, |cdocspt4.tex|,
|cdocsdrf.tex|, |cdocsfn1.tex|, |cdocsfn2.tex|.
Then copy the file |childdoc.def| to an appropriate directory of your \LaTeX{}
distribution, e.g.\ \textit{texmf-root}|/tex/latex/childdoc|.
\end{itemize}

%%%%%%%%%%%%%%%%%%%%%%%%%%%%%%%%%%%%%%%%%%%%%%%%%%%%%%%%%%%%%%%%%%%%%%%%%%%%%%%%
\subsection{Related CTAN Packages}

There are several other packages which offer a similar functionality:
%
\begin{itemize}
\item
The packages
\href{http://ctan.org/pkg/docmute}{\textsf{docmute}},
\href{http://ctan.org/pkg/includex}{\textsf{includex}} and
\href{http://ctan.org/pkg/standalone}{\textsf{standalone}}
provide commands to include only the document body of
a child file thus allowing both files to be compiled individually.
\item
The packages \href{http://ctan.org/pkg/subdocs}{\textsf{subdocs}}
and \href{http://ctan.org/pkg/subfiles}{\textsf{subfiles}}
provide structures in which the main and child documents can be
encapsulated and allowing them to be compiled individually.
The inclusion mechanism is different from the conventional |\include|.
\item
The package \href{http://ctan.org/pkg/combine}{\textsf{combine}}
is an elaborate solution to combine several documents into one.
\end{itemize}
%
See also the CTAN topic \href{http://ctan.org/topic/subdocs}{\textsf{subdocs}}
for further related packages.
The present package differs from the above solutions in that
a document structure constructed with the conventional |\include| mechanism
just needs two extra commands at the top of every file
such that all constituent files can be compiled individually.

%%%%%%%%%%%%%%%%%%%%%%%%%%%%%%%%%%%%%%%%%%%%%%%%%%%%%%%%%%%%%%%%%%%%%%%%%%%%%%%%
%\subsection{Feature Suggestions}
%
%The following is a list of features which may be useful for future
%versions of this package:
%%
%\begin{itemize}
%\item
%\ldots
%\end{itemize}

%%%%%%%%%%%%%%%%%%%%%%%%%%%%%%%%%%%%%%%%%%%%%%%%%%%%%%%%%%%%%%%%%%%%%%%%%%%%%%%%
\subsection{Revision History}

%%%%%%%%%%%%%%%%%%%%%%%%%%%%%%%%%%%%%%%%
\paragraph{v2.0:} 2018/12/30

\begin{itemize}
\item
immediate forward processing
\item
added |\childdocby| mechanism
\item
manual restructured
\end{itemize}

%%%%%%%%%%%%%%%%%%%%%%%%%%%%%%%%%%%%%%%%
\paragraph{v1.6:} 2018/01/17

\begin{itemize}
\item
application for development of include files
\item
corrections to manual
\end{itemize}

%%%%%%%%%%%%%%%%%%%%%%%%%%%%%%%%%%%%%%%%
\paragraph{v1.5:} 2017/05/21

\begin{itemize}
\item
more complete structuring introduced
\item
|\childdocof| introduced
\item
|\childdoc| renamed to |\childdocmain|
\item
|\childredirect| renamed to |\childdocforward| and |\childdocforwardprefix|
and functionality expanded
\end{itemize}

%%%%%%%%%%%%%%%%%%%%%%%%%%%%%%%%%%%%%%%%
\paragraph{v1.0:} 2017/04/27

\begin{itemize}
\item
manual and install package
\item
first version published on CTAN
\end{itemize}

%%%%%%%%%%%%%%%%%%%%%%%%%%%%%%%%%%%%%%%%
\paragraph{v0.6:} 2017/04/26

\begin{itemize}
\item
redirection mechanism added
\end{itemize}

%%%%%%%%%%%%%%%%%%%%%%%%%%%%%%%%%%%%%%%%
\paragraph{v0.5:} 2017/04/26

\begin{itemize}
\item
functionality in definition file
\end{itemize}


%%%%%%%%%%%%%%%%%%%%%%%%%%%%%%%%%%%%%%%%%%%%%%%%%%%%%%%%%%%%%%%%%%%%%%%%%%%%%%%%
%%%%%%%%%%%%%%%%%%%%%%%%%%%%%%%%%%%%%%%%%%%%%%%%%%%%%%%%%%%%%%%%%%%%%%%%%%%%%%%%
%%%%%%%%%%%%%%%%%%%%%%%%%%%%%%%%%%%%%%%%%%%%%%%%%%%%%%%%%%%%%%%%%%%%%%%%%%%%%%%%
\appendix

\settowidth\MacroIndent{\rmfamily\scriptsize 000\ }

 \DocInput{childdoc.dtx}

\end{document}
%</driver>
% \fi
%
% %%%%%%%%%%%%%%%%%%%%%%%%%%%%%%%%%%%%%%%%%%%%%%%%%%%%%%%%%%%%%%%%%%%%%%%%%%%%%%
% %%%%%%%%%%%%%%%%%%%%%%%%%%%%%%%%%%%%%%%%%%%%%%%%%%%%%%%%%%%%%%%%%%%%%%%%%%%%%%
% \section{Sample}
%\iffalse
%<*samplemain>
%\fi
%
% The following presents a sample document
% with two chapters, two parts, a title page,
% a compile flag as well as three forwarding files to set the flag.
% It consists of eight |.tex| files:
% \begin{center}
% \begin{tabular}{ll}
% |cdocsamp.tex|&main file\\
% |cdocsch1.tex|&include file for chapter 1\\
% |cdocsch2.tex|&include file for chapter 2\\
% |cdocspt3.tex|&include file for part 3\\
% |cdocspt4.tex|&include file for part 4\\
% |cdocsdrf.tex|&forwarding file for main file in draft mode\\
% |cdocsfi1.tex|&forwarding file for final version of chapter 1\\
% |cdocsfi2.tex|&forwarding file for final version of chapter 2\\
% \end{tabular}
% \end{center}
% Each of the eight files can be compiled directly by the \LaTeX{} compiler.
%
% %%%%%%%%%%%%%%%%%%%%%%%%%%%%%%%%%%%%%%
% \paragraph{Main File.}
%
% The main file is called |cdocsamp.tex|.
%
% Load the \textsf{childdoc} definitions and
% declare the filename for the main document:
%    \begin{macrocode}
\input{childdoc.def}
\childdocmain{}
%    \end{macrocode}

% Optional override for |\version| flag:
%    \begin{macrocode}
%%\ifchilddoc\else\providecommand{\version}{draft}\fi
%    \end{macrocode}

% Define the default values for the |\version| flag
% (|final| for the main file and |draft| for childs):
%    \begin{macrocode}
\ifchilddoc
\providecommand{\version}{draft}
\else
\providecommand{\version}{final}
\fi
%    \end{macrocode}

% Load the standard document class:
%    \begin{macrocode}
\documentclass[12pt]{article}
%    \end{macrocode}

% Start the document body:
%    \begin{macrocode}
\begin{document}
%    \end{macrocode}

% Declare a title page.
% Print title, part of document being processed and version flag:
%    \begin{macrocode}
\addtocounter{page}{-1}
\begin{center}
{\LARGE\bfseries{}childdoc example\par}
\vspace{1cm}
\ifchilddoc
\ifchilddocmanual part\else chapter\fi:
`\childdocname' of `\childdocjob'\par
\else
main document: `\childdocjob'\par
\fi
version: \version\par
\end{center}
\newpage
%    \end{macrocode}

% Manually include selected file,
% otherwise process as usual:
%    \begin{macrocode}
\ifchilddocmanual
\section*{part `\childdocname'}
\input{\childdocname}
\else
%    \end{macrocode}

% Include the two chapters:
%    \begin{macrocode}
\include{cdocsch1}
\include{cdocsch2}
%    \end{macrocode}

% Include the two parts unless only chapters should be displayed:
%    \begin{macrocode}
\ifchilddoc\else
\section{part three}
\input{cdocspt3}
\section{part four}
\input{cdocspt4}
\fi
%    \end{macrocode}

% Process as usual until here:
%    \begin{macrocode}
\fi
%    \end{macrocode}

% End of document body:
%    \begin{macrocode}
\end{document}
%    \end{macrocode}
%\iffalse
%</samplemain>
%\fi
%
% %%%%%%%%%%%%%%%%%%%%%%%%%%%%%%%%%%%%%%
% \paragraph{Chapter Include Files.}
%
% The include files are called |cdocsch1.tex| and |cdocsch2.tex|.
%
%\iffalse
%<*samplechap1|samplechap2>
%\fi

% Optional override for |\version| flag:
%    \begin{macrocode}
%%\providecommand{\version}{final}
%    \end{macrocode}

% Include the main document:
%    \begin{macrocode}
\input{childdoc.def}
\childdocof{cdocsamp}
%    \end{macrocode}

%\iffalse
%</samplechap1|samplechap2>
%\fi
%
%\iffalse
%<*samplechap1>
%\fi
% Some text for chapter 1:
%    \begin{macrocode}
\section{one}
some text in chapter one
%    \end{macrocode}

%\iffalse
%</samplechap1>
%\fi
% Some text for chapter 2:
%\iffalse
%<*samplechap2>
%\fi
%    \begin{macrocode}
\section{two}
more text in chapter two
%    \end{macrocode}

%\iffalse
%</samplechap2>
%\fi
%
% %%%%%%%%%%%%%%%%%%%%%%%%%%%%%%%%%%%%%%
% \paragraph{Part Include Files.}
%
% The include files are called |cdocspt3.tex| and |cdocspt4.tex|.
%
%\iffalse
%<*samplepart3|samplepart4>
%\fi

% Optional override for |\version| flag:
%    \begin{macrocode}
%%\providecommand{\version}{final}
%    \end{macrocode}

% Include the main document:
%    \begin{macrocode}
\input{childdoc.def}
\childdocby{cdocsamp}
%    \end{macrocode}

%\iffalse
%</samplepart3|samplepart4>
%\fi
%
%\iffalse
%<*samplepart3>
%\fi
% Some text for part 3:
%    \begin{macrocode}
some text in part three
%    \end{macrocode}

%\iffalse
%</samplepart3>
%\fi
% Some text for part 4:
%\iffalse
%<*samplepart4>
%\fi
%    \begin{macrocode}
more text in part four
%    \end{macrocode}

%\iffalse
%</samplepart4>
%\fi
%
% %%%%%%%%%%%%%%%%%%%%%%%%%%%%%%%%%%%%%%
% \paragraph{Forwarding for a Complete Draft.}
%
% The following forwarding file |cdocsdrf.tex|
% compiles the main document in draft mode:
%\iffalse
%<*sampledraft>
%\fi
%    \begin{macrocode}
\def\version{draft}
\input{childdoc.def}
\childdocforward{cdocsamp}
%    \end{macrocode}

%\iffalse
%</sampledraft>
%\fi
%
% %%%%%%%%%%%%%%%%%%%%%%%%%%%%%%%%%%%%%%
% \paragraph{Forwarding for Final Version of the Chapters.}
%
% The following forwarding files |cdocsfn1.tex| and |cdocsfn2.tex|
% (with identical content)
% compile the final versions of the child documents
% |cdocsch1.tex| and |cdocsch2.tex|, respectively:
%\iffalse
%<*samplefinal>
%\fi
%    \begin{macrocode}
\def\version{final}
\input{childdoc.def}
\childdocforwardprefix[cdocsamp]{cdocsfn}{cdocsch}
%    \end{macrocode}

%\iffalse
%</samplefinal>
%\fi
%
% %%%%%%%%%%%%%%%%%%%%%%%%%%%%%%%%%%%%%%
% \paragraph{Command Line Processing.}
%
% The following three command lines generate the output files
% |cdocscld|, |cdocscl1| and |cdocscl2|
% which should be identical to
% |cdocsdrf|, |cdocsch1| and |cdocsfn2|, respectively:
% \begin{center}
% \begin{tabular}{l}
% |latex -jobname cdocscld \|\\
% |  "\def\version{draft}\input{childdoc.def}\childdocforward{cdocsamp}"|\\
% |latex -jobname cdocscl1 \|\\
% |  "\input{childdoc.def}\childdocforward[cdocsamp]{cdocsch1}"|\\
% |latex -jobname cdocscl2 \|\\
% |  "\def\version{final}\input{childdoc.def}\childdocforward{cdocsch2}"|
% \end{tabular}
% \end{center}
% Note that the trailing backslash on each first line
% merely continues the input to the second line
% (for convenient cut ant paste).
% Furthermore, the command |latex| can be replaced by any
% of its alternative versions such as |pdflatex|.
%
% %%%%%%%%%%%%%%%%%%%%%%%%%%%%%%%%%%%%%%%%%%%%%%%%%%%%%%%%%%%%%%%%%%%%%%%%%%%%%%
% %%%%%%%%%%%%%%%%%%%%%%%%%%%%%%%%%%%%%%%%%%%%%%%%%%%%%%%%%%%%%%%%%%%%%%%%%%%%%%
% \section{Implementation}
%\iffalse
%<*package>
%\fi
%
% This section describes the definitions file |childdoc.def|.

% The definitions cannot be loaded using |\usepackage| or |\RequirePackage|
% which has a mechanism to prevent loading a style file more than once.
% When loading the definitions by means of |\input|
% multiple instances have to be prevented manually:
%\iffalse
%This code needs to be before the `\ProvidesFile' directive
%which is defined at the beginning of this file.
%Therefore it is also placed there and commented out here.
%</package>
%<*discard>
%\fi
%    \begin{macrocode}
\ifdefined\childdocmain\endinput\fi
%    \end{macrocode}
%\iffalse
%</discard>
%<*package>
%\fi
%
% \macro{\ifchilddoc}
% \macro{\ifchilddocmanual}
% The conditional |\ifchilddoc| tells whether a
% child (true) or main (false) document is being compiled.
% The conditional |\ifchilddocmanual| tells whether
% the |\includeonly| mechanism is used (false) or
% the selection of child files must be performed manually (true).
% The definitions initialise to false:
%    \begin{macrocode}
\newif\ifchilddoc
\newif\ifchilddocmanual
%    \end{macrocode}

% \macro{\childdocname}
% \macro{\childdocjob}
% The macro |\childdocname| stores the name of the main document
% to be compiled. The macro |\childdocjob| stores the name of
% the document on which the \LaTeX{} compiler was originally invoked.
% The content of |\jobname| cannot be compared
% to filenames specified in the source due to different catcodes.
% The following code rescans |\jobname|, stores the result
% in |\childdocname| and saves a copy in |\childdocjob|:
%    \begin{macrocode}
\edef\childdocname{\scantokens\expandafter{\jobname\noexpand}}
\let\childdocjob\childdocname
%    \end{macrocode}

% \macro{\childdocdisable}
% The macro |\childdocdisable| prevents the main file
% from being processed more than once.
% At this stage, the main document command |\childdocmain|
% is assumed to be called once again where it should do nothing.
% Any subsequent call to it should prevent
% a secondary processing of the main document
% It overwrites the forwarding commands
% |\childdocof| and |\childdocforward|
% with empty macros to prevent further inclusions of the main document:
%    \begin{macrocode}
\newcommand{\childdocdisable}
{
  \renewcommand{\childdocmain}[1]{\renewcommand{\childdocmain}[1]{\endinput}}
  \renewcommand{\childdocof}[1]{}
  \renewcommand{\childdocby}[2][]{}
  \renewcommand{\childdocforward}[2][]{}
  \renewcommand{\childdocdisable}{}
}
%    \end{macrocode}

% \macro{\childdocmain}
% The macro |\childdocmain| is to be called at the top of the main file
% with nothing or the main filename (without extension) as argument.
% First, it breaks loops.
% If the argument is not empty and does not match |\childdocname|
% (which is set by the first inclusion of |childdoc.def|),
% |\ifchilddoc| is set to true, |\includeonly| is applied to the child file
% and |\jobname| is set to the main file
% (for proper handling of |.aux| files):
%    \begin{macrocode}
\newcommand{\childdocmain}[1]
{
  \childdocdisable\childdocmain{}
  \if?#1?\else
    \begingroup
      \def\childdoctmp{#1}
      \ifx\childdoctmp\childdocname
        \def\childdoctmp{}
      \else
        \def\childdoctmp
        {
          \childdoctrue
          \includeonly{\childdocname}
          \def\childdocjob{#1}
          \def\jobname{#1}
        }
      \fi
      \expandafter
    \endgroup
    \childdoctmp
  \fi
}
%    \end{macrocode}

% \macro{\childdocof}
% The command |\childdocof| redirects
% compilation to the main file |#1|.
%    \begin{macrocode}
\newcommand{\childdocof}[1]
{
  \childdocdisable
  \childdoctrue
  \includeonly{\childdocname}
  \def\jobname{#1}
  \def\childdocjob{#1}
  \input{#1}
}
%    \end{macrocode}

% \macro{\childdocby}
% The command |\childdocby| ....
%    \begin{macrocode}
\newcommand{\childdocby}[2][]
{
  \childdocdisable
  \childdoctrue
  \childdocmanualtrue
  \if?#1?\else
    \def\jobname{#2}
  \fi
  \def\childdocjob{#2}
  \input{#2}
  \endinput
}
%    \end{macrocode}

% \macro{\childdocforward}
% The command |\childdocforward| redirects
% compilation to the main file or
% (if the optional argument is given) a child file.
% Parameters are set as if the main file
% or a child file starting with |\childdocof| was compiled.
% Then compilation is handed over to the main file:
%    \begin{macrocode}
\newcommand{\childdocforward}[2][]
{
  \begingroup
    \if?#1?
      \def\childdoctmp
      {
        \def\childdocname{#2}
        \def\childdocjob{#2}
        \def\jobname{#2}
        \input{#2}
        \endinput
      }
    \else
      \def\childdoctmp
      {
        \childdocdisable
        \def\childdocname{#2}
        \childdoctrue
        \includeonly{#2}
        \def\childdocjob{#1}
        \def\jobname{#1}
        \input{#1}
        \endinput
      }
    \fi
    \expandafter
  \endgroup
  \childdoctmp
}
%    \end{macrocode}

% \macro{\childdocforwardprefix}
% The command |\childdocforwardprefix| redirects
% compilation to the main or a child file by means of a pattern.
% The prefix |#1| in the current filename is replaced by |#2|
% and the suffix of the current filename is kept
% (it is assumed that the filename does not contain the substring `|~~~|'
% which is used as a delimiter).
% Compilation is handed over to the new file by |\childdocforward|:
%    \begin{macrocode}
\newcommand{\childdocforwardprefix}[3][]
{
  \begingroup
    \def\childdocextract #2##1~~~{\def\childdoctmp{\childdocforward[#1]{#3##1}}}
    \expandafter\childdocextract\childdocname~~~
    \expandafter
  \endgroup
  \childdoctmp
}
%    \end{macrocode}

% \macro{\childdoc}
% The deprecated macro |\childdoc| is a legacy version of |\childdocmain|:
%    \begin{macrocode}
\newcommand{\childdoc}{\childdocmain}
%    \end{macrocode}

% \macro{\childdocredirect}
% The deprecated macro |\childdocredirect| is a legacy version
% of |\childdocforward| and |\childdocforwardprefix|:
%    \begin{macrocode}
\newcommand{\childdocredirect}[2][]
{
  \begingroup
    \if?#1?
      \def\childdoctmp{\childdocforward{#2}}
    \else
      \def\childdoctmp{\childdocforwardprefix{#1}{#2}}
    \fi
    \expandafter
  \endgroup
  \childdoctmp
}
%    \end{macrocode}

%\iffalse
%</package>
%\fi
%
\endinput
|\\
|\childdocmain{}|\\
\end{tabular}
\end{center}
at the very top of the main \LaTeX{} file,
in particular \emph{before} the |\documentclass| statement!
The argument of |\childdocmain| should be left empty
(but it must be present).

%%%%%%%%%%%%%%%%%%%%%%%%%%%%%%%%%%%%%%%%
\DescribeMacro{\childdocof}
Furthermore, add the commands
\begin{center}
\begin{tabular}{l}
|% \iffalse
%
% childdoc.dtx Copyright (C) 2017-2018 Niklas Beisert
%
% This work may be distributed and/or modified under the
% conditions of the LaTeX Project Public License, either version 1.3
% of this license or (at your option) any later version.
% The latest version of this license is in
%   http://www.latex-project.org/lppl.txt
% and version 1.3 or later is part of all distributions of LaTeX
% version 2005/12/01 or later.
%
% This work has the LPPL maintenance status `maintained'.
%
% The Current Maintainer of this work is Niklas Beisert.
%
% This work consists of the files childdoc.dtx and childdoc.ins
% and the derived files childdoc.def and cdocsamp.tex with
% cdocsch1.tex, cdocsch2.tex, cdocsdrf.tex, cdocsfn1.tex, cdocsfn2.tex.
%
%<package>\ifdefined\childdocmain\endinput\fi
%<package>\ProvidesFile{childdoc.def}[2018/12/30 v2.0 child document driver]
%<samplemain>\ProvidesFile{cdocsamp.tex}[2018/12/30 v2.0 sample for childdoc]
%<*driver>
%\ProvidesFile{childdoc.drv}[2018/12/30 v2.0 childdoc reference manual file]
\PassOptionsToClass{10pt,a4paper}{article}
\documentclass{ltxdoc}

\usepackage[margin=35mm]{geometry}
\usepackage{hyperref}
\usepackage{hyperxmp}
\usepackage[usenames]{color}

\hypersetup{colorlinks=true}
\hypersetup{pdfstartview=FitH}
\hypersetup{pdfpagemode=UseNone}
\hypersetup{pdfsource={}}
\hypersetup{pdflang={en-UK}}
\hypersetup{pdfcopyright={Copyright 2017-2018 Niklas Beisert.
  This work may be distributed and/or modified under the
  conditions of the LaTeX Project Public License, either version 1.3
  of this license or (at your option) any later version.}}
\hypersetup{pdflicenseurl={http://www.latex-project.org/lppl.txt}}
\hypersetup{pdfcontactaddress={ETH Zurich, ITP, HIT K,
  Wolfgang-Pauli-Strasse 27}}
\hypersetup{pdfcontactpostcode={8093}}
\hypersetup{pdfcontactcity={Zurich}}
\hypersetup{pdfcontactcountry={Switzerland}}
\hypersetup{pdfcontactemail={nbeisert@itp.phys.ethz.ch}}
\hypersetup{pdfcontacturl={http://people.phys.ethz.ch/\xmptilde nbeisert/}}

\newcommand{\secref}[1]{\hyperref[#1]{section \ref*{#1}}}

\parskip1ex
\parindent0pt
\let\olditemize\itemize
\def\itemize{\olditemize\parskip0pt}

\begin{document}

\title{The \textsf{childdoc} Package}
\hypersetup{pdftitle={The childdoc Package}}
\author{Niklas Beisert\\[2ex]
  Institut f\"ur Theoretische Physik\\
  Eidgen\"ossische Technische Hochschule Z\"urich\\
  Wolfgang-Pauli-Strasse 27, 8093 Z\"urich, Switzerland\\[1ex]
  \href{mailto:nbeisert@itp.phys.ethz.ch}
  {\texttt{nbeisert@itp.phys.ethz.ch}}}
\hypersetup{pdfauthor={Niklas Beisert}}
\hypersetup{pdfsubject={Manual for the LaTeX2e Package childdoc}}
\date{30 December 2018, \textsf{v2.0}}
\maketitle

\begin{abstract}\noindent
\textsf{childdoc} is a \LaTeXe{} package
that enables the direct compilation
of document sections included by |\include|
to individual files.
\end{abstract}

\begingroup
\parskip0ex
\tableofcontents
\endgroup

%%%%%%%%%%%%%%%%%%%%%%%%%%%%%%%%%%%%%%%%%%%%%%%%%%%%%%%%%%%%%%%%%%%%%%%%%%%%%%%%
%%%%%%%%%%%%%%%%%%%%%%%%%%%%%%%%%%%%%%%%%%%%%%%%%%%%%%%%%%%%%%%%%%%%%%%%%%%%%%%%
\section{Introduction}

\LaTeX{} provides a mechanism to structure a large document (such as a book)
into a main file and several child files (containing the chapters)
using the |\include| command.
This mechanism is beneficial for documents
which span hundreds of pages in order to
make the source file(s) more manageable.
Moreover, compilation can be restricted to
selected child files by means of the |\includeonly| command.
The latter feature can be used to reduce the compilation time while editing
(this was significantly more useful in the earlier days of \LaTeX{})
or to generate a smaller document which is easier to navigate.
Another application of |\includeonly| is to generate
documents consisting of selected parts of the complete document.

However, there are a few drawbacks of the plain |\include| mechanism:
\begin{itemize}
\item
The child files cannot be compiled on their own,
they can only be compiled via the main file.
A naive editing environment
(such as a text editor with an option
to have the current file processed by \LaTeX)
may require one to switch to the main file before compiling;
attempting to compile the child file produces errors.
\item
The main file must be modified (each time)
to adjust the |\includeonly| command
to the present needs. This easily leaves the main file in a messy state.
\item
The generated document will always carry the filename
of the main document. This is inconvenient if
several child files are to be compiled and
to be kept for distribution.
\end{itemize}

The present package provides a simple interface
to make child files individually compilable by \LaTeX{}.
Compiling a child file then has the same effect as compiling
the main file with an |\includeonly| command
to select the appropriate child.
Moreover the generated document will carry the name of the child
rather than the main file.
This resolves all three above issues.

This feature is meant to make the editing of books,
thesis documents and lecture notes somewhat more convenient.
However, the package can also be used efficiently for
composing a series of documents (such as exercise sheets)
which are typically distributed individually.
It then assists the author in generating the individual documents
(potentially in different versions)
as well as a document containing the collected series.
Another application is in developing style files
or other kinds of included material
where compilation of the style file could redirect
to a sample or test file.

%%%%%%%%%%%%%%%%%%%%%%%%%%%%%%%%%%%%%%%%%%%%%%%%%%%%%%%%%%%%%%%%%%%%%%%%%%%%%%%%
%%%%%%%%%%%%%%%%%%%%%%%%%%%%%%%%%%%%%%%%%%%%%%%%%%%%%%%%%%%%%%%%%%%%%%%%%%%%%%%%
\section{Usage}

First of all, the package \textsf{childdoc} is \emph{not} a standard
\LaTeXe{} |.sty| style file! Therefore it needs to be invoked in
a non-standard way.

%%%%%%%%%%%%%%%%%%%%%%%%%%%%%%%%%%%%%%%%%%%%%%%%%%%%%%%%%%%%%%%%%%%%%%%%%%%%%%%%
\subsection{Included Files}
\label{sec:include}

%%%%%%%%%%%%%%%%%%%%%%%%%%%%%%%%%%%%%%%%
\DescribeMacro{\childdocmain}
To use the package, add the commands
\begin{center}
\begin{tabular}{l}
|\input{childdoc.def}|\\
|\childdocmain{}|\\
\end{tabular}
\end{center}
at the very top of the main \LaTeX{} file,
in particular \emph{before} the |\documentclass| statement!
The argument of |\childdocmain| should be left empty
(but it must be present).

%%%%%%%%%%%%%%%%%%%%%%%%%%%%%%%%%%%%%%%%
\DescribeMacro{\childdocof}
Furthermore, add the commands
\begin{center}
\begin{tabular}{l}
|\input{childdoc.def}|\\
|\childdocof{|\textit{main}|}|\\
\end{tabular}
\end{center}
at the top of every child file \textit{child}
which is included by |\include{|\textit{child}|}|
from within the main file
(or at least for those files to be compiled individually).
The argument \textit{main} must be the filename of the main file.

There are a couple of
considerations in setting up the main and child documents:

%%%%%%%%%%%%%%%%%%%%%%%%%%%%%%%%%%%%%%%%
\paragraph{Restrictions.}

Please note the following restrictions:
\begin{itemize}
\item
|\childdocmain| must be called with one argument \textit{main}
to ensure compatibility with earlier version of the package.
It must either be empty (|\childdocmain{}|)
or precisely match the filename of the main file in which it is specified.
See \secref{sec:detection} for further information.
\item
The filename \textit{main} must be specified without the |.tex| extension.
\item
The filename \textit{main} is case sensitive
(even in case-insensitive file systems)
due to internal string comparison.
\item
The argument \textit{main} should be fully expanded, it cannot be a macro.
\item
Subdirectories and special characters should be avoided in filenames.
\item
The command |\childdocmain{|\textit{main}|}| must be followed by a whitespace.
It should not be followed immediately by another command
or by a comment mark `|%|'.
This is because the \TeX{} parser reads the token immediately following
the argument of |\childdocmain| and puts it
at the beginning of every child section;
however, a white\-space is ignored.
\end{itemize}

%%%%%%%%%%%%%%%%%%%%%%%%%%%%%%%%%%%%%%%%
\paragraph{Content of Main File.}

It is advisable to place all content in the child files included by |\include|.
Any output contained in the main file will appear in all child documents
unless suppressed manually;
it cannot be suppressed automatically by the |\includeonly| directive
and thus should normally be avoided.
A method to include some content in the main file
by means of conditional processing is described in \secref{sec:conditional}.

%%%%%%%%%%%%%%%%%%%%%%%%%%%%%%%%%%%%%%%%
\paragraph{Page Numbering.}

When only a part of the document is compiled,
the appropriate numbering of pages
(as well as other status parameters)
is determined from the |.aux| files.
The latter contain information from previous passes.
However this information needs to propagate through
all intermediate child documents.
Therefore the page numbering in child documents may well
be inconsistent until the complete document is compiled at least once.

A useful (if unconventional) way to always ensure a consistent
page numbering is to restart the numbering in each child document
and denote the pages by `\textit{child}|.|\textit{page}'
where \textit{child} represents the chapter/section number of the child file.
This can be achieved by the command
|\numberwithin{page}{|\textit{child}|}|
of the \textsf{amsmath} package
where \textit{child} can be |chapter| or |section|
depending on the chosen structuring.
Alternatively, one can modify the macro |\thepage| appropriately
and reset the counter |page| at the start of each child file.

%%%%%%%%%%%%%%%%%%%%%%%%%%%%%%%%%%%%%%%%%%%%%%%%%%%%%%%%%%%%%%%%%%%%%%%%%%%%%%%%
\subsection{Conditional Processing}
\label{sec:conditional}

The package provides a mechanism to compile different versions
of a document. To customise the versions further some conditional processing
can come in handy to distinguish which version is being compiled.
The package provides two macros to describe the compilation context:

%%%%%%%%%%%%%%%%%%%%%%%%%%%%%%%%%%%%%%%%
\DescribeMacro{\ifchilddoc}
The conditional |\ifchilddoc| distinguishes between the compilation of
child documents and the main document:
%
\begin{center}
|\ifchilddoc |\textit{child-code}| |[|\||else |\textit{main-code}]| \||fi|
\end{center}

%%%%%%%%%%%%%%%%%%%%%%%%%%%%%%%%%%%%%%%%
\DescribeMacro{\childdocname}
\DescribeMacro{\childdocjob}
The macro |\childdocname| contains the filename (without extension)
of the main or child file being processed.
Note that |\childdocjob| will always contain the name of the main file.

%%%%%%%%%%%%%%%%%%%%%%%%%%%%%%%%%%%%%%%%
\paragraph{Title Page.}

Conditional processing can be used to include a title or banner page
in the main document when proper precautions are taken.
Importantly, the code in the main file should ensure that the page counter
(as well as other status parameters which are stored in the |.aux| files)
takes the same value after the conditional processing.
Otherwise the page numbers may take divergent values
depending on which part is compiled.

For example, a title page could be declared by:
%
\begin{center}
\begin{tabular}{l}
|\ifchilddoc\||else|\\
|\addtocounter{page}{-1}|\\
\textit{code for title page}\\
|\newpage|\\
|\||fi|
\end{tabular}
\end{center}
%
A banner page for the child documents can be generated by:
%
\begin{center}
\begin{tabular}{l}
|\ifchilddoc|\\
|\addtocounter{page}{-1}|\\
\textit{code for banner page}\\
|\newpage|\\
|\||fi|
\end{tabular}
\end{center}
%
Here one could write a message such as:
\begin{center}
|This is the part \childdocname{} of \childdocjob{}.|
\end{center}

%%%%%%%%%%%%%%%%%%%%%%%%%%%%%%%%%%%%%%%%%%%%%%%%%%%%%%%%%%%%%%%%%%%%%%%%%%%%%%%%
\subsection{Flags}
\label{sec:flags}

The package makes it easy to generate different versions
of the main or child documents.
To this end compilation flags can be defined
and assigned different default values.
They will be particularly useful in conjunction
with the forwarding mechanism described in \secref{sec:forward}.

For example, it may be useful to have a flag |\version|
which can be set to |draft| or |final|.
The document source will contain some conditional code
depending on the value of |\version|.
Suppose further, the flag should default to |final| for the main file
and to |draft| for child files
which is a natural assignment for editing the document.
This is achieved by placing the following code
in the preamble of the main document
(below the |\childdocmain| directive):
%
\begin{center}
\begin{tabular}{l}
|\ifchilddoc|\\
|\providecommand{\version}{draft}|\\
|\||else|\\
|\providecommand{\version}{final}|\\
|\||fi|
\end{tabular}
\end{center}
%
The definition by |\providecommand| makes sure
that previous definitions are not overwritten.
Further statements |\providecommand{\version}{...}|
can thus be added before the above code to override it.

For the main file, one might add a line
(between |\childdocmain| and the above block)
%
\begin{center}
|%\ifchilddoc\||else\providecommand{\version}{draft}\||fi|
\end{center}
%
which can be uncommented to produce a draft version.
Likewise one can add a line to the very top of a child file
(above the |\childdocof{|\textit{main}|}| directive)
%
\begin{center}
|%\providecommand{\version}{final}|
\end{center}
%
which can be uncommented to produce the final version of this child document.

%%%%%%%%%%%%%%%%%%%%%%%%%%%%%%%%%%%%%%%%%%%%%%%%%%%%%%%%%%%%%%%%%%%%%%%%%%%%%%%%
\subsection{Forwarding}
\label{sec:forward}

Different versions of the main or child documents
using compilation flags as described in \secref{sec:flags}
can be (permanently) stored in different files
for convenient compilation, viewing and distribution.
To this end, the package defines a command
to pass on compilation to a different file:

%%%%%%%%%%%%%%%%%%%%%%%%%%%%%%%%%%%%%%%%
\DescribeMacro{\childdocforward}
The command |\childdocforward| redirects processing to
another source file:
%
\begin{center}
\begin{tabular}{l}
|\input{childdoc.def}|\\
|\childdocforward[|\textit{main}|]{|\textit{dest}|}|\\
\end{tabular}
\end{center}
%
The argument \textit{dest} is the destination file
(without extension).
It should be the main file or one of the child files.
Note that further \textsf{childdoc} directives
such as |\childdocof| and |\childdocforward|
in the indicated file will be processed in this form.
The optional argument \textit{main}
passes on directly to the main file \textit{main}
while pretending to compile the child \textit{dest}.
This form behaves as if \textit{dest}
issues |\childdocof{|\textit{main}|}| right away,
and no further \textsf{childdoc} directives will be processed.

%%%%%%%%%%%%%%%%%%%%%%%%%%%%%%%%%%%%%%%%
\DescribeMacro{\...prefix}
In the alternative form |\childdocforwardprefix|,
%
\begin{center}
\begin{tabular}{l}
|\input{childdoc.def}|\\
|\childdocforwardprefix[|\textit{main}|]{|\textit{prefix}|}{|\textit{dest}|}|
\end{tabular}
\end{center}
%
the destination file is determined by a pattern
depending on the current file:
To make this work, the current file must be called
`{\textit{prefix}\hspace{0.2em}\textit{suffix}}'
with \textit{prefix} matching precisely the argument.
Processing is then passed on to the file
`{\textit{dest}\hspace{0.2em}\textit{suffix}}'.
Surely, the same effect is achieved by
directly specifying the
argument `{\textit{dest}\hspace{0.2em}\textit{suffix}}'
in the first form.
However, that requires to set up a different file
for each child. With the alternative form of the command
all these files can have exactly the same content
which simplifies setting them up and maintaining them.

For example, the following file |draft.tex|
with a compilation flag |\version| as described in \secref{sec:flags}
compiles the main document as a draft:
%
\begin{center}
\begin{tabular}{l}
|\def\version{draft}|\\
|\input{childdoc.def}|\\
|\childdocforward{|\textit{main}|}|
\end{tabular}
\end{center}
%
Likewise, the following files |final|\textit{nn}|.tex|
compile the final version of the child document
|child|\textit{nn}|.tex|:
%
\begin{center}
\begin{tabular}{l}
|\def\version{final}|\\
|\input{childdoc.def}|\\
|\childdocforwardprefix{final}{child}|
\end{tabular}
\end{center}
%

Note that when several versions of a main file and/or of each child file
are to be generated, it may be convenient to set up a |Makefile| or
shell script to automatise the process.

%%%%%%%%%%%%%%%%%%%%%%%%%%%%%%%%%%%%%%%%%%%%%%%%%%%%%%%%%%%%%%%%%%%%%%%%%%%%%%%%
\subsection{Command Line Processing}
\label{sec:commandline}

The effect of redirection files can also be achieved by invoking
the \LaTeX{} compiler with a more elaborate command line.
Most conveniently this should be done as part
of a shell script or a |Makefile|.

When using \textsf{childdoc} in the main file, the following
command lines effectively perform a redirection
(note that depending on the shell being used,
backslashes may have to be doubled: `|\|' $\to$ `|\\|'):
%
\begin{center}
|... -jobname "|\textit{target}|" |\\|"|[\textit{flags}]%
|\input{childdoc.def}\childdocforward[|\textit{main}|]{|\textit{dest}|}"|
\end{center}
%
Here \textit{target} is the name of the output file,
\textit{main} is the name of the main file
and \textit{dest} is the name of the main or child file to be processed
(all filenames without extensions).
The optional argument \textit{main} can be omitted
if \textit{main} matches \textit{dest}.
Optionally, compilation \textit{flags} can be defined via |\def| commands.
This command line makes the \TeX{} engine believe
it is compiling the file \textit{target}
whose content is specified as the latter parameter.
The provided code then forwards the processing to
\textit{main} or \textit{dest} as described in \secref{sec:forward}.

%%%%%%%%%%%%%%%%%%%%%%%%%%%%%%%%%%%%%%%%%%%%%%%%%%%%%%%%%%%%%%%%%%%%%%%%%%%%%%%%
\subsection{Include by Input}
\label{sec:input}

Including child documents by |\include| has some restrictions by design.
Most notably, the content of a child document always occupies
its own set of pages; pages cannot be shared between child documents.
Usually, this behaviour makes perfect sense
because each child document contain an essential part of the document.
However, in some situations it may be desirable to compose
a document from a collection of parts
without having mandatory page breaks between then.
For this case, the package
provides a mechanism to include parts
by |\input| which can also be processed individually.
However, by construction this mechanism
requires manual handling of the content to be output.

%%%%%%%%%%%%%%%%%%%%%%%%%%%%%%%%%%%%%%%%
\DescribeMacro{\ifchilddocmanual}
The main file should be prepared as usual, see \secref{sec:include}.
However, the document body must make a distinction
between processing of an individual part and of the main document, e.g.:
%
\begin{center}
\begin{tabular}{l}
|\ifchilddocmanual|\\
|\input{\childdocname}|\\
|\||else|\\
\textit{document body with }|\input{|\textit{part}|}|\\
|\||fi|
\end{tabular}
\end{center}
%
The conditional |\ifchilddocmanual| is true whenever
a part to be included by |\input| is being compiled,
and the name of the part is stored in |\childdocname|.

%%%%%%%%%%%%%%%%%%%%%%%%%%%%%%%%%%%%%%%%
\DescribeMacro{\childdocby}
Each part to be included by |\input| should start with:
%
\begin{center}
\begin{tabular}{l}
|\input{childdoc.def}|\\
|\childdocby{|\textit{main}|}|\\
\end{tabular}
\end{center}
%
The directive |\childdocby| is similar to |\childdocof|
described in \secref{sec:include},
but the subsequent selection of content must be done manually.
To that end, both |\ifchilddoc| and |\ifchilddocmanual|
will be true upon processing of a part,
and the name of the part is stored in |\childdocname|.
Note that |\jobname| will be set to the filename of the current part
so that each part receives an individual |.aux| file
that does not interfere with the |.aux| file(s) of the main document.
This behaviour can be altered by the alternative form
|\childdocby[*]{|\textit{main}|}| (with a non-empty optional argument)
which uses the |.aux| file of the main document
by setting |\jobname| to \textit{main}.

%%%%%%%%%%%%%%%%%%%%%%%%%%%%%%%%%%%%%%%%%%%%%%%%%%%%%%%%%%%%%%%%%%%%%%%%%%%%%%%%
\subsection{Driver Development}
\label{sec:driver}

The \textsf{childdoc} mechanism can also be use for the development
of definition files such as \LaTeX{} styles or classes.
This case differs from the above setup with multiple parts
included by |\include| in that no |\includeonly| should be invoked.
This can be achieved by starting the include file
(before |\ProvidesPackage|) with:
%
\begin{center}
\begin{tabular}{l}
|\input{childdoc.def}|\\
|\childdocforward{|\textit{main}|}|\\
\end{tabular}
\end{center}
%
or alternatively with:
%
\begin{center}
\begin{tabular}{l}
|\input{childdoc.def}|\\
|\childdocby{|\textit{main}|}|\\
\end{tabular}
\end{center}
%
Both forms have slightly different effects as described above.
The main file is prepared as usual, see \secref{sec:include}.

%%%%%%%%%%%%%%%%%%%%%%%%%%%%%%%%%%%%%%%%%%%%%%%%%%%%%%%%%%%%%%%%%%%%%%%%%%%%%%%%
\subsection{Legacy Detection}
\label{sec:detection}

The directive |\childdocmain| in the main file can detect
whether the complete document or merely a child is to be compiled
even without using the directive |\childdocof|.
This method is deprecated because it is less robust
and there is no compelling reason to use it;
it is merely provided for backward compatibility
and it may be removed in future versions.

If the detection mechanism is to be used,
it is mandatory to correctly specify
the filename of the main file as the argument of |\childdocmain|:
%
\begin{center}
\begin{tabular}{l}
|\input{childdoc.def}|\\
|\childdocmain{|\textit{main}|}|\\
\end{tabular}
\end{center}
%
If |\jobname| does not match the argument \textit{main} of |\childdocmain|,
it is assumed that |\jobname| points to the child file to be compiled.
When using |\childdocmain| with the main file specified as argument,
it suffices to start a child file
with just |\input{|\textit{main}|}|
without loading of the package and using |\childdocof|.
If instead all processing is done
with the appropriate \textsf{childdoc} directives,
the argument of \textit{main} of |\childdocmain| can be empty.

An alternative version of the command line processing described
in \secref{sec:commandline} using the detection mechanism reads:
%
\begin{center}
|... -jobname "|\textit{target}|" "|[\textit{flags}]%
[|\def\jobname{|\textit{dest}|}|]|\input{|\textit{main}|}"|
\end{center}

%%%%%%%%%%%%%%%%%%%%%%%%%%%%%%%%%%%%%%%%%%%%%%%%%%%%%%%%%%%%%%%%%%%%%%%%%%%%%%%%
\subsection{Manual Code}
\label{sec:manual}

In case one cannot be certain whether the definitions file |childdoc.def|
is installed on the target \TeX{} distribution
and one prefers not to ship it,
it is conceivable to paste a few relevant commands into the sources.

To that end, drop all statements |\input{childdoc.def}|
and perform the replacements as outlined below.
Instead of |\childdocmain{|\textit{main}|}| add the following code
to the top of the main file:
%
\begin{center}
\begin{tabular}{l}
|\||ifdefined\childdocname\endinput\||fi\newif\ifchilddoc|\\
|\edef\childdocname{\scantokens\expandafter{\jobname\noexpand}}|\\
|\def\childdocmain{|\textit{main}|}\||ifx\childdocmain\childdocname\||else|\\
|\childdoctrue\includeonly{\childdocname}\let\jobname\childdocmain\||fi|\\
\end{tabular}
\end{center}
%
Instead of |\childdocof{|\textit{main}|}| just include the main file
at the top of each child file:
%
\begin{center}
|\input{|\textit{main}|}|
\end{center}
%
A simple redirection |\childdocforward{|\textit{dest}|}| is achieved by:
%
\begin{center}
|\def\jobname{|\textit{dest}|}\input{\jobname}|
\end{center}
%
The redirection with prefix
|\childdocforwardprefix[|\textit{prefix}|]{|\textit{dest}|}|
is accomplished by:
%
\begin{center}
\begin{tabular}{l}
|{\edef\jobname{\scantokens\expandafter{\jobname\noexpand}}|\\
|\def\redirectjob |\textit{prefix}|#1~~~{\gdef\jobname{|\textit{dest}|#1}}|\\
|\expandafter\redirectjob\jobname~~~}\input{\jobname}|
\end{tabular}
\end{center}

In an alternative approach,
child documents can be compiled by a specific command line
without additional code or specific definitions:
%
\begin{center}
|... -jobname "|\textit{target}|" "|[\textit{flags}]%
|\includeonly{|\textit{dest}|}\input{|\textit{main}|}"|
\end{center}
%

%%%%%%%%%%%%%%%%%%%%%%%%%%%%%%%%%%%%%%%%%%%%%%%%%%%%%%%%%%%%%%%%%%%%%%%%%%%%%%%%
%%%%%%%%%%%%%%%%%%%%%%%%%%%%%%%%%%%%%%%%%%%%%%%%%%%%%%%%%%%%%%%%%%%%%%%%%%%%%%%%
\section{Information}

%%%%%%%%%%%%%%%%%%%%%%%%%%%%%%%%%%%%%%%%%%%%%%%%%%%%%%%%%%%%%%%%%%%%%%%%%%%%%%%%
\subsection{Copyright}

Copyright \copyright{} 2017--2018 Niklas Beisert

This work may be distributed and/or modified under the
conditions of the \LaTeX{} Project Public License, either version 1.3
of this license or (at your option) any later version.
The latest version of this license is in
  \url{http://www.latex-project.org/lppl.txt}
and version 1.3 or later is part of all distributions of \LaTeX{}
version 2005/12/01 or later.

This work has the LPPL maintenance status `maintained'.

The Current Maintainer of this work is Niklas Beisert.

This work consists of the files |README.txt|, |childdoc.ins| and |childdoc.dtx|
as well as the derived files |childdoc.def|, |cdocsamp.tex|
with |cdocsch1.tex|, |cdocsch2.tex|, |cdocspt3.tex|, |cdocspt4.tex|,
|cdocsdrf.tex|, |cdocsfn1.tex|, |cdocsfn2.tex|
as well as |childdoc.pdf|.

%%%%%%%%%%%%%%%%%%%%%%%%%%%%%%%%%%%%%%%%%%%%%%%%%%%%%%%%%%%%%%%%%%%%%%%%%%%%%%%%
\subsection{Files and Installation}

The package consists of the files:
%
\begin{center}
\begin{tabular}{ll}
    |README.txt|   & readme file \\
    |childdoc.ins| & installation file \\
    |childdoc.dtx| & source file \\
    |childdoc.def| & definition file \\
    |cdocsamp.tex| & sample main file \\
    |cdocsch1.tex| & sample include file \\
    |cdocsch2.tex| & sample include file \\
    |cdocspt3.tex| & sample part file \\
    |cdocspt4.tex| & sample part file \\
    |cdocsdrf.tex| & sample redirection file \\
    |cdocsfn1.tex| & sample redirection file \\
    |cdocsfn2.tex| & sample redirection file \\
    |childdoc.pdf| & manual
\end{tabular}
\end{center}
%
The distribution consists of the files
|README.txt|, |childdoc.ins| and |childdoc.dtx|.
%
\begin{itemize}
\item
Run (pdf)\LaTeX{} on |childdoc.dtx|
to compile the manual |childdoc.pdf| (this file).
\item
Run \LaTeX{} on |childdoc.ins| to create the definitions file |childdoc.def|
and the sample |cdocsamp.tex| with include files
|cdocsch1.tex|, |cdocsch2.tex|, |cdocspt3.tex|, |cdocspt4.tex|,
|cdocsdrf.tex|, |cdocsfn1.tex|, |cdocsfn2.tex|.
Then copy the file |childdoc.def| to an appropriate directory of your \LaTeX{}
distribution, e.g.\ \textit{texmf-root}|/tex/latex/childdoc|.
\end{itemize}

%%%%%%%%%%%%%%%%%%%%%%%%%%%%%%%%%%%%%%%%%%%%%%%%%%%%%%%%%%%%%%%%%%%%%%%%%%%%%%%%
\subsection{Related CTAN Packages}

There are several other packages which offer a similar functionality:
%
\begin{itemize}
\item
The packages
\href{http://ctan.org/pkg/docmute}{\textsf{docmute}},
\href{http://ctan.org/pkg/includex}{\textsf{includex}} and
\href{http://ctan.org/pkg/standalone}{\textsf{standalone}}
provide commands to include only the document body of
a child file thus allowing both files to be compiled individually.
\item
The packages \href{http://ctan.org/pkg/subdocs}{\textsf{subdocs}}
and \href{http://ctan.org/pkg/subfiles}{\textsf{subfiles}}
provide structures in which the main and child documents can be
encapsulated and allowing them to be compiled individually.
The inclusion mechanism is different from the conventional |\include|.
\item
The package \href{http://ctan.org/pkg/combine}{\textsf{combine}}
is an elaborate solution to combine several documents into one.
\end{itemize}
%
See also the CTAN topic \href{http://ctan.org/topic/subdocs}{\textsf{subdocs}}
for further related packages.
The present package differs from the above solutions in that
a document structure constructed with the conventional |\include| mechanism
just needs two extra commands at the top of every file
such that all constituent files can be compiled individually.

%%%%%%%%%%%%%%%%%%%%%%%%%%%%%%%%%%%%%%%%%%%%%%%%%%%%%%%%%%%%%%%%%%%%%%%%%%%%%%%%
%\subsection{Feature Suggestions}
%
%The following is a list of features which may be useful for future
%versions of this package:
%%
%\begin{itemize}
%\item
%\ldots
%\end{itemize}

%%%%%%%%%%%%%%%%%%%%%%%%%%%%%%%%%%%%%%%%%%%%%%%%%%%%%%%%%%%%%%%%%%%%%%%%%%%%%%%%
\subsection{Revision History}

%%%%%%%%%%%%%%%%%%%%%%%%%%%%%%%%%%%%%%%%
\paragraph{v2.0:} 2018/12/30

\begin{itemize}
\item
immediate forward processing
\item
added |\childdocby| mechanism
\item
manual restructured
\end{itemize}

%%%%%%%%%%%%%%%%%%%%%%%%%%%%%%%%%%%%%%%%
\paragraph{v1.6:} 2018/01/17

\begin{itemize}
\item
application for development of include files
\item
corrections to manual
\end{itemize}

%%%%%%%%%%%%%%%%%%%%%%%%%%%%%%%%%%%%%%%%
\paragraph{v1.5:} 2017/05/21

\begin{itemize}
\item
more complete structuring introduced
\item
|\childdocof| introduced
\item
|\childdoc| renamed to |\childdocmain|
\item
|\childredirect| renamed to |\childdocforward| and |\childdocforwardprefix|
and functionality expanded
\end{itemize}

%%%%%%%%%%%%%%%%%%%%%%%%%%%%%%%%%%%%%%%%
\paragraph{v1.0:} 2017/04/27

\begin{itemize}
\item
manual and install package
\item
first version published on CTAN
\end{itemize}

%%%%%%%%%%%%%%%%%%%%%%%%%%%%%%%%%%%%%%%%
\paragraph{v0.6:} 2017/04/26

\begin{itemize}
\item
redirection mechanism added
\end{itemize}

%%%%%%%%%%%%%%%%%%%%%%%%%%%%%%%%%%%%%%%%
\paragraph{v0.5:} 2017/04/26

\begin{itemize}
\item
functionality in definition file
\end{itemize}


%%%%%%%%%%%%%%%%%%%%%%%%%%%%%%%%%%%%%%%%%%%%%%%%%%%%%%%%%%%%%%%%%%%%%%%%%%%%%%%%
%%%%%%%%%%%%%%%%%%%%%%%%%%%%%%%%%%%%%%%%%%%%%%%%%%%%%%%%%%%%%%%%%%%%%%%%%%%%%%%%
%%%%%%%%%%%%%%%%%%%%%%%%%%%%%%%%%%%%%%%%%%%%%%%%%%%%%%%%%%%%%%%%%%%%%%%%%%%%%%%%
\appendix

\settowidth\MacroIndent{\rmfamily\scriptsize 000\ }

 \DocInput{childdoc.dtx}

\end{document}
%</driver>
% \fi
%
% %%%%%%%%%%%%%%%%%%%%%%%%%%%%%%%%%%%%%%%%%%%%%%%%%%%%%%%%%%%%%%%%%%%%%%%%%%%%%%
% %%%%%%%%%%%%%%%%%%%%%%%%%%%%%%%%%%%%%%%%%%%%%%%%%%%%%%%%%%%%%%%%%%%%%%%%%%%%%%
% \section{Sample}
%\iffalse
%<*samplemain>
%\fi
%
% The following presents a sample document
% with two chapters, two parts, a title page,
% a compile flag as well as three forwarding files to set the flag.
% It consists of eight |.tex| files:
% \begin{center}
% \begin{tabular}{ll}
% |cdocsamp.tex|&main file\\
% |cdocsch1.tex|&include file for chapter 1\\
% |cdocsch2.tex|&include file for chapter 2\\
% |cdocspt3.tex|&include file for part 3\\
% |cdocspt4.tex|&include file for part 4\\
% |cdocsdrf.tex|&forwarding file for main file in draft mode\\
% |cdocsfi1.tex|&forwarding file for final version of chapter 1\\
% |cdocsfi2.tex|&forwarding file for final version of chapter 2\\
% \end{tabular}
% \end{center}
% Each of the eight files can be compiled directly by the \LaTeX{} compiler.
%
% %%%%%%%%%%%%%%%%%%%%%%%%%%%%%%%%%%%%%%
% \paragraph{Main File.}
%
% The main file is called |cdocsamp.tex|.
%
% Load the \textsf{childdoc} definitions and
% declare the filename for the main document:
%    \begin{macrocode}
\input{childdoc.def}
\childdocmain{}
%    \end{macrocode}

% Optional override for |\version| flag:
%    \begin{macrocode}
%%\ifchilddoc\else\providecommand{\version}{draft}\fi
%    \end{macrocode}

% Define the default values for the |\version| flag
% (|final| for the main file and |draft| for childs):
%    \begin{macrocode}
\ifchilddoc
\providecommand{\version}{draft}
\else
\providecommand{\version}{final}
\fi
%    \end{macrocode}

% Load the standard document class:
%    \begin{macrocode}
\documentclass[12pt]{article}
%    \end{macrocode}

% Start the document body:
%    \begin{macrocode}
\begin{document}
%    \end{macrocode}

% Declare a title page.
% Print title, part of document being processed and version flag:
%    \begin{macrocode}
\addtocounter{page}{-1}
\begin{center}
{\LARGE\bfseries{}childdoc example\par}
\vspace{1cm}
\ifchilddoc
\ifchilddocmanual part\else chapter\fi:
`\childdocname' of `\childdocjob'\par
\else
main document: `\childdocjob'\par
\fi
version: \version\par
\end{center}
\newpage
%    \end{macrocode}

% Manually include selected file,
% otherwise process as usual:
%    \begin{macrocode}
\ifchilddocmanual
\section*{part `\childdocname'}
\input{\childdocname}
\else
%    \end{macrocode}

% Include the two chapters:
%    \begin{macrocode}
\include{cdocsch1}
\include{cdocsch2}
%    \end{macrocode}

% Include the two parts unless only chapters should be displayed:
%    \begin{macrocode}
\ifchilddoc\else
\section{part three}
\input{cdocspt3}
\section{part four}
\input{cdocspt4}
\fi
%    \end{macrocode}

% Process as usual until here:
%    \begin{macrocode}
\fi
%    \end{macrocode}

% End of document body:
%    \begin{macrocode}
\end{document}
%    \end{macrocode}
%\iffalse
%</samplemain>
%\fi
%
% %%%%%%%%%%%%%%%%%%%%%%%%%%%%%%%%%%%%%%
% \paragraph{Chapter Include Files.}
%
% The include files are called |cdocsch1.tex| and |cdocsch2.tex|.
%
%\iffalse
%<*samplechap1|samplechap2>
%\fi

% Optional override for |\version| flag:
%    \begin{macrocode}
%%\providecommand{\version}{final}
%    \end{macrocode}

% Include the main document:
%    \begin{macrocode}
\input{childdoc.def}
\childdocof{cdocsamp}
%    \end{macrocode}

%\iffalse
%</samplechap1|samplechap2>
%\fi
%
%\iffalse
%<*samplechap1>
%\fi
% Some text for chapter 1:
%    \begin{macrocode}
\section{one}
some text in chapter one
%    \end{macrocode}

%\iffalse
%</samplechap1>
%\fi
% Some text for chapter 2:
%\iffalse
%<*samplechap2>
%\fi
%    \begin{macrocode}
\section{two}
more text in chapter two
%    \end{macrocode}

%\iffalse
%</samplechap2>
%\fi
%
% %%%%%%%%%%%%%%%%%%%%%%%%%%%%%%%%%%%%%%
% \paragraph{Part Include Files.}
%
% The include files are called |cdocspt3.tex| and |cdocspt4.tex|.
%
%\iffalse
%<*samplepart3|samplepart4>
%\fi

% Optional override for |\version| flag:
%    \begin{macrocode}
%%\providecommand{\version}{final}
%    \end{macrocode}

% Include the main document:
%    \begin{macrocode}
\input{childdoc.def}
\childdocby{cdocsamp}
%    \end{macrocode}

%\iffalse
%</samplepart3|samplepart4>
%\fi
%
%\iffalse
%<*samplepart3>
%\fi
% Some text for part 3:
%    \begin{macrocode}
some text in part three
%    \end{macrocode}

%\iffalse
%</samplepart3>
%\fi
% Some text for part 4:
%\iffalse
%<*samplepart4>
%\fi
%    \begin{macrocode}
more text in part four
%    \end{macrocode}

%\iffalse
%</samplepart4>
%\fi
%
% %%%%%%%%%%%%%%%%%%%%%%%%%%%%%%%%%%%%%%
% \paragraph{Forwarding for a Complete Draft.}
%
% The following forwarding file |cdocsdrf.tex|
% compiles the main document in draft mode:
%\iffalse
%<*sampledraft>
%\fi
%    \begin{macrocode}
\def\version{draft}
\input{childdoc.def}
\childdocforward{cdocsamp}
%    \end{macrocode}

%\iffalse
%</sampledraft>
%\fi
%
% %%%%%%%%%%%%%%%%%%%%%%%%%%%%%%%%%%%%%%
% \paragraph{Forwarding for Final Version of the Chapters.}
%
% The following forwarding files |cdocsfn1.tex| and |cdocsfn2.tex|
% (with identical content)
% compile the final versions of the child documents
% |cdocsch1.tex| and |cdocsch2.tex|, respectively:
%\iffalse
%<*samplefinal>
%\fi
%    \begin{macrocode}
\def\version{final}
\input{childdoc.def}
\childdocforwardprefix[cdocsamp]{cdocsfn}{cdocsch}
%    \end{macrocode}

%\iffalse
%</samplefinal>
%\fi
%
% %%%%%%%%%%%%%%%%%%%%%%%%%%%%%%%%%%%%%%
% \paragraph{Command Line Processing.}
%
% The following three command lines generate the output files
% |cdocscld|, |cdocscl1| and |cdocscl2|
% which should be identical to
% |cdocsdrf|, |cdocsch1| and |cdocsfn2|, respectively:
% \begin{center}
% \begin{tabular}{l}
% |latex -jobname cdocscld \|\\
% |  "\def\version{draft}\input{childdoc.def}\childdocforward{cdocsamp}"|\\
% |latex -jobname cdocscl1 \|\\
% |  "\input{childdoc.def}\childdocforward[cdocsamp]{cdocsch1}"|\\
% |latex -jobname cdocscl2 \|\\
% |  "\def\version{final}\input{childdoc.def}\childdocforward{cdocsch2}"|
% \end{tabular}
% \end{center}
% Note that the trailing backslash on each first line
% merely continues the input to the second line
% (for convenient cut ant paste).
% Furthermore, the command |latex| can be replaced by any
% of its alternative versions such as |pdflatex|.
%
% %%%%%%%%%%%%%%%%%%%%%%%%%%%%%%%%%%%%%%%%%%%%%%%%%%%%%%%%%%%%%%%%%%%%%%%%%%%%%%
% %%%%%%%%%%%%%%%%%%%%%%%%%%%%%%%%%%%%%%%%%%%%%%%%%%%%%%%%%%%%%%%%%%%%%%%%%%%%%%
% \section{Implementation}
%\iffalse
%<*package>
%\fi
%
% This section describes the definitions file |childdoc.def|.

% The definitions cannot be loaded using |\usepackage| or |\RequirePackage|
% which has a mechanism to prevent loading a style file more than once.
% When loading the definitions by means of |\input|
% multiple instances have to be prevented manually:
%\iffalse
%This code needs to be before the `\ProvidesFile' directive
%which is defined at the beginning of this file.
%Therefore it is also placed there and commented out here.
%</package>
%<*discard>
%\fi
%    \begin{macrocode}
\ifdefined\childdocmain\endinput\fi
%    \end{macrocode}
%\iffalse
%</discard>
%<*package>
%\fi
%
% \macro{\ifchilddoc}
% \macro{\ifchilddocmanual}
% The conditional |\ifchilddoc| tells whether a
% child (true) or main (false) document is being compiled.
% The conditional |\ifchilddocmanual| tells whether
% the |\includeonly| mechanism is used (false) or
% the selection of child files must be performed manually (true).
% The definitions initialise to false:
%    \begin{macrocode}
\newif\ifchilddoc
\newif\ifchilddocmanual
%    \end{macrocode}

% \macro{\childdocname}
% \macro{\childdocjob}
% The macro |\childdocname| stores the name of the main document
% to be compiled. The macro |\childdocjob| stores the name of
% the document on which the \LaTeX{} compiler was originally invoked.
% The content of |\jobname| cannot be compared
% to filenames specified in the source due to different catcodes.
% The following code rescans |\jobname|, stores the result
% in |\childdocname| and saves a copy in |\childdocjob|:
%    \begin{macrocode}
\edef\childdocname{\scantokens\expandafter{\jobname\noexpand}}
\let\childdocjob\childdocname
%    \end{macrocode}

% \macro{\childdocdisable}
% The macro |\childdocdisable| prevents the main file
% from being processed more than once.
% At this stage, the main document command |\childdocmain|
% is assumed to be called once again where it should do nothing.
% Any subsequent call to it should prevent
% a secondary processing of the main document
% It overwrites the forwarding commands
% |\childdocof| and |\childdocforward|
% with empty macros to prevent further inclusions of the main document:
%    \begin{macrocode}
\newcommand{\childdocdisable}
{
  \renewcommand{\childdocmain}[1]{\renewcommand{\childdocmain}[1]{\endinput}}
  \renewcommand{\childdocof}[1]{}
  \renewcommand{\childdocby}[2][]{}
  \renewcommand{\childdocforward}[2][]{}
  \renewcommand{\childdocdisable}{}
}
%    \end{macrocode}

% \macro{\childdocmain}
% The macro |\childdocmain| is to be called at the top of the main file
% with nothing or the main filename (without extension) as argument.
% First, it breaks loops.
% If the argument is not empty and does not match |\childdocname|
% (which is set by the first inclusion of |childdoc.def|),
% |\ifchilddoc| is set to true, |\includeonly| is applied to the child file
% and |\jobname| is set to the main file
% (for proper handling of |.aux| files):
%    \begin{macrocode}
\newcommand{\childdocmain}[1]
{
  \childdocdisable\childdocmain{}
  \if?#1?\else
    \begingroup
      \def\childdoctmp{#1}
      \ifx\childdoctmp\childdocname
        \def\childdoctmp{}
      \else
        \def\childdoctmp
        {
          \childdoctrue
          \includeonly{\childdocname}
          \def\childdocjob{#1}
          \def\jobname{#1}
        }
      \fi
      \expandafter
    \endgroup
    \childdoctmp
  \fi
}
%    \end{macrocode}

% \macro{\childdocof}
% The command |\childdocof| redirects
% compilation to the main file |#1|.
%    \begin{macrocode}
\newcommand{\childdocof}[1]
{
  \childdocdisable
  \childdoctrue
  \includeonly{\childdocname}
  \def\jobname{#1}
  \def\childdocjob{#1}
  \input{#1}
}
%    \end{macrocode}

% \macro{\childdocby}
% The command |\childdocby| ....
%    \begin{macrocode}
\newcommand{\childdocby}[2][]
{
  \childdocdisable
  \childdoctrue
  \childdocmanualtrue
  \if?#1?\else
    \def\jobname{#2}
  \fi
  \def\childdocjob{#2}
  \input{#2}
  \endinput
}
%    \end{macrocode}

% \macro{\childdocforward}
% The command |\childdocforward| redirects
% compilation to the main file or
% (if the optional argument is given) a child file.
% Parameters are set as if the main file
% or a child file starting with |\childdocof| was compiled.
% Then compilation is handed over to the main file:
%    \begin{macrocode}
\newcommand{\childdocforward}[2][]
{
  \begingroup
    \if?#1?
      \def\childdoctmp
      {
        \def\childdocname{#2}
        \def\childdocjob{#2}
        \def\jobname{#2}
        \input{#2}
        \endinput
      }
    \else
      \def\childdoctmp
      {
        \childdocdisable
        \def\childdocname{#2}
        \childdoctrue
        \includeonly{#2}
        \def\childdocjob{#1}
        \def\jobname{#1}
        \input{#1}
        \endinput
      }
    \fi
    \expandafter
  \endgroup
  \childdoctmp
}
%    \end{macrocode}

% \macro{\childdocforwardprefix}
% The command |\childdocforwardprefix| redirects
% compilation to the main or a child file by means of a pattern.
% The prefix |#1| in the current filename is replaced by |#2|
% and the suffix of the current filename is kept
% (it is assumed that the filename does not contain the substring `|~~~|'
% which is used as a delimiter).
% Compilation is handed over to the new file by |\childdocforward|:
%    \begin{macrocode}
\newcommand{\childdocforwardprefix}[3][]
{
  \begingroup
    \def\childdocextract #2##1~~~{\def\childdoctmp{\childdocforward[#1]{#3##1}}}
    \expandafter\childdocextract\childdocname~~~
    \expandafter
  \endgroup
  \childdoctmp
}
%    \end{macrocode}

% \macro{\childdoc}
% The deprecated macro |\childdoc| is a legacy version of |\childdocmain|:
%    \begin{macrocode}
\newcommand{\childdoc}{\childdocmain}
%    \end{macrocode}

% \macro{\childdocredirect}
% The deprecated macro |\childdocredirect| is a legacy version
% of |\childdocforward| and |\childdocforwardprefix|:
%    \begin{macrocode}
\newcommand{\childdocredirect}[2][]
{
  \begingroup
    \if?#1?
      \def\childdoctmp{\childdocforward{#2}}
    \else
      \def\childdoctmp{\childdocforwardprefix{#1}{#2}}
    \fi
    \expandafter
  \endgroup
  \childdoctmp
}
%    \end{macrocode}

%\iffalse
%</package>
%\fi
%
\endinput
|\\
|\childdocof{|\textit{main}|}|\\
\end{tabular}
\end{center}
at the top of every child file \textit{child}
which is included by |\include{|\textit{child}|}|
from within the main file
(or at least for those files to be compiled individually).
The argument \textit{main} must be the filename of the main file.

There are a couple of
considerations in setting up the main and child documents:

%%%%%%%%%%%%%%%%%%%%%%%%%%%%%%%%%%%%%%%%
\paragraph{Restrictions.}

Please note the following restrictions:
\begin{itemize}
\item
|\childdocmain| must be called with one argument \textit{main}
to ensure compatibility with earlier version of the package.
It must either be empty (|\childdocmain{}|)
or precisely match the filename of the main file in which it is specified.
See \secref{sec:detection} for further information.
\item
The filename \textit{main} must be specified without the |.tex| extension.
\item
The filename \textit{main} is case sensitive
(even in case-insensitive file systems)
due to internal string comparison.
\item
The argument \textit{main} should be fully expanded, it cannot be a macro.
\item
Subdirectories and special characters should be avoided in filenames.
\item
The command |\childdocmain{|\textit{main}|}| must be followed by a whitespace.
It should not be followed immediately by another command
or by a comment mark `|%|'.
This is because the \TeX{} parser reads the token immediately following
the argument of |\childdocmain| and puts it
at the beginning of every child section;
however, a white\-space is ignored.
\end{itemize}

%%%%%%%%%%%%%%%%%%%%%%%%%%%%%%%%%%%%%%%%
\paragraph{Content of Main File.}

It is advisable to place all content in the child files included by |\include|.
Any output contained in the main file will appear in all child documents
unless suppressed manually;
it cannot be suppressed automatically by the |\includeonly| directive
and thus should normally be avoided.
A method to include some content in the main file
by means of conditional processing is described in \secref{sec:conditional}.

%%%%%%%%%%%%%%%%%%%%%%%%%%%%%%%%%%%%%%%%
\paragraph{Page Numbering.}

When only a part of the document is compiled,
the appropriate numbering of pages
(as well as other status parameters)
is determined from the |.aux| files.
The latter contain information from previous passes.
However this information needs to propagate through
all intermediate child documents.
Therefore the page numbering in child documents may well
be inconsistent until the complete document is compiled at least once.

A useful (if unconventional) way to always ensure a consistent
page numbering is to restart the numbering in each child document
and denote the pages by `\textit{child}|.|\textit{page}'
where \textit{child} represents the chapter/section number of the child file.
This can be achieved by the command
|\numberwithin{page}{|\textit{child}|}|
of the \textsf{amsmath} package
where \textit{child} can be |chapter| or |section|
depending on the chosen structuring.
Alternatively, one can modify the macro |\thepage| appropriately
and reset the counter |page| at the start of each child file.

%%%%%%%%%%%%%%%%%%%%%%%%%%%%%%%%%%%%%%%%%%%%%%%%%%%%%%%%%%%%%%%%%%%%%%%%%%%%%%%%
\subsection{Conditional Processing}
\label{sec:conditional}

The package provides a mechanism to compile different versions
of a document. To customise the versions further some conditional processing
can come in handy to distinguish which version is being compiled.
The package provides two macros to describe the compilation context:

%%%%%%%%%%%%%%%%%%%%%%%%%%%%%%%%%%%%%%%%
\DescribeMacro{\ifchilddoc}
The conditional |\ifchilddoc| distinguishes between the compilation of
child documents and the main document:
%
\begin{center}
|\ifchilddoc |\textit{child-code}| |[|\||else |\textit{main-code}]| \||fi|
\end{center}

%%%%%%%%%%%%%%%%%%%%%%%%%%%%%%%%%%%%%%%%
\DescribeMacro{\childdocname}
\DescribeMacro{\childdocjob}
The macro |\childdocname| contains the filename (without extension)
of the main or child file being processed.
Note that |\childdocjob| will always contain the name of the main file.

%%%%%%%%%%%%%%%%%%%%%%%%%%%%%%%%%%%%%%%%
\paragraph{Title Page.}

Conditional processing can be used to include a title or banner page
in the main document when proper precautions are taken.
Importantly, the code in the main file should ensure that the page counter
(as well as other status parameters which are stored in the |.aux| files)
takes the same value after the conditional processing.
Otherwise the page numbers may take divergent values
depending on which part is compiled.

For example, a title page could be declared by:
%
\begin{center}
\begin{tabular}{l}
|\ifchilddoc\||else|\\
|\addtocounter{page}{-1}|\\
\textit{code for title page}\\
|\newpage|\\
|\||fi|
\end{tabular}
\end{center}
%
A banner page for the child documents can be generated by:
%
\begin{center}
\begin{tabular}{l}
|\ifchilddoc|\\
|\addtocounter{page}{-1}|\\
\textit{code for banner page}\\
|\newpage|\\
|\||fi|
\end{tabular}
\end{center}
%
Here one could write a message such as:
\begin{center}
|This is the part \childdocname{} of \childdocjob{}.|
\end{center}

%%%%%%%%%%%%%%%%%%%%%%%%%%%%%%%%%%%%%%%%%%%%%%%%%%%%%%%%%%%%%%%%%%%%%%%%%%%%%%%%
\subsection{Flags}
\label{sec:flags}

The package makes it easy to generate different versions
of the main or child documents.
To this end compilation flags can be defined
and assigned different default values.
They will be particularly useful in conjunction
with the forwarding mechanism described in \secref{sec:forward}.

For example, it may be useful to have a flag |\version|
which can be set to |draft| or |final|.
The document source will contain some conditional code
depending on the value of |\version|.
Suppose further, the flag should default to |final| for the main file
and to |draft| for child files
which is a natural assignment for editing the document.
This is achieved by placing the following code
in the preamble of the main document
(below the |\childdocmain| directive):
%
\begin{center}
\begin{tabular}{l}
|\ifchilddoc|\\
|\providecommand{\version}{draft}|\\
|\||else|\\
|\providecommand{\version}{final}|\\
|\||fi|
\end{tabular}
\end{center}
%
The definition by |\providecommand| makes sure
that previous definitions are not overwritten.
Further statements |\providecommand{\version}{...}|
can thus be added before the above code to override it.

For the main file, one might add a line
(between |\childdocmain| and the above block)
%
\begin{center}
|%\ifchilddoc\||else\providecommand{\version}{draft}\||fi|
\end{center}
%
which can be uncommented to produce a draft version.
Likewise one can add a line to the very top of a child file
(above the |\childdocof{|\textit{main}|}| directive)
%
\begin{center}
|%\providecommand{\version}{final}|
\end{center}
%
which can be uncommented to produce the final version of this child document.

%%%%%%%%%%%%%%%%%%%%%%%%%%%%%%%%%%%%%%%%%%%%%%%%%%%%%%%%%%%%%%%%%%%%%%%%%%%%%%%%
\subsection{Forwarding}
\label{sec:forward}

Different versions of the main or child documents
using compilation flags as described in \secref{sec:flags}
can be (permanently) stored in different files
for convenient compilation, viewing and distribution.
To this end, the package defines a command
to pass on compilation to a different file:

%%%%%%%%%%%%%%%%%%%%%%%%%%%%%%%%%%%%%%%%
\DescribeMacro{\childdocforward}
The command |\childdocforward| redirects processing to
another source file:
%
\begin{center}
\begin{tabular}{l}
|% \iffalse
%
% childdoc.dtx Copyright (C) 2017-2018 Niklas Beisert
%
% This work may be distributed and/or modified under the
% conditions of the LaTeX Project Public License, either version 1.3
% of this license or (at your option) any later version.
% The latest version of this license is in
%   http://www.latex-project.org/lppl.txt
% and version 1.3 or later is part of all distributions of LaTeX
% version 2005/12/01 or later.
%
% This work has the LPPL maintenance status `maintained'.
%
% The Current Maintainer of this work is Niklas Beisert.
%
% This work consists of the files childdoc.dtx and childdoc.ins
% and the derived files childdoc.def and cdocsamp.tex with
% cdocsch1.tex, cdocsch2.tex, cdocsdrf.tex, cdocsfn1.tex, cdocsfn2.tex.
%
%<package>\ifdefined\childdocmain\endinput\fi
%<package>\ProvidesFile{childdoc.def}[2018/12/30 v2.0 child document driver]
%<samplemain>\ProvidesFile{cdocsamp.tex}[2018/12/30 v2.0 sample for childdoc]
%<*driver>
%\ProvidesFile{childdoc.drv}[2018/12/30 v2.0 childdoc reference manual file]
\PassOptionsToClass{10pt,a4paper}{article}
\documentclass{ltxdoc}

\usepackage[margin=35mm]{geometry}
\usepackage{hyperref}
\usepackage{hyperxmp}
\usepackage[usenames]{color}

\hypersetup{colorlinks=true}
\hypersetup{pdfstartview=FitH}
\hypersetup{pdfpagemode=UseNone}
\hypersetup{pdfsource={}}
\hypersetup{pdflang={en-UK}}
\hypersetup{pdfcopyright={Copyright 2017-2018 Niklas Beisert.
  This work may be distributed and/or modified under the
  conditions of the LaTeX Project Public License, either version 1.3
  of this license or (at your option) any later version.}}
\hypersetup{pdflicenseurl={http://www.latex-project.org/lppl.txt}}
\hypersetup{pdfcontactaddress={ETH Zurich, ITP, HIT K,
  Wolfgang-Pauli-Strasse 27}}
\hypersetup{pdfcontactpostcode={8093}}
\hypersetup{pdfcontactcity={Zurich}}
\hypersetup{pdfcontactcountry={Switzerland}}
\hypersetup{pdfcontactemail={nbeisert@itp.phys.ethz.ch}}
\hypersetup{pdfcontacturl={http://people.phys.ethz.ch/\xmptilde nbeisert/}}

\newcommand{\secref}[1]{\hyperref[#1]{section \ref*{#1}}}

\parskip1ex
\parindent0pt
\let\olditemize\itemize
\def\itemize{\olditemize\parskip0pt}

\begin{document}

\title{The \textsf{childdoc} Package}
\hypersetup{pdftitle={The childdoc Package}}
\author{Niklas Beisert\\[2ex]
  Institut f\"ur Theoretische Physik\\
  Eidgen\"ossische Technische Hochschule Z\"urich\\
  Wolfgang-Pauli-Strasse 27, 8093 Z\"urich, Switzerland\\[1ex]
  \href{mailto:nbeisert@itp.phys.ethz.ch}
  {\texttt{nbeisert@itp.phys.ethz.ch}}}
\hypersetup{pdfauthor={Niklas Beisert}}
\hypersetup{pdfsubject={Manual for the LaTeX2e Package childdoc}}
\date{30 December 2018, \textsf{v2.0}}
\maketitle

\begin{abstract}\noindent
\textsf{childdoc} is a \LaTeXe{} package
that enables the direct compilation
of document sections included by |\include|
to individual files.
\end{abstract}

\begingroup
\parskip0ex
\tableofcontents
\endgroup

%%%%%%%%%%%%%%%%%%%%%%%%%%%%%%%%%%%%%%%%%%%%%%%%%%%%%%%%%%%%%%%%%%%%%%%%%%%%%%%%
%%%%%%%%%%%%%%%%%%%%%%%%%%%%%%%%%%%%%%%%%%%%%%%%%%%%%%%%%%%%%%%%%%%%%%%%%%%%%%%%
\section{Introduction}

\LaTeX{} provides a mechanism to structure a large document (such as a book)
into a main file and several child files (containing the chapters)
using the |\include| command.
This mechanism is beneficial for documents
which span hundreds of pages in order to
make the source file(s) more manageable.
Moreover, compilation can be restricted to
selected child files by means of the |\includeonly| command.
The latter feature can be used to reduce the compilation time while editing
(this was significantly more useful in the earlier days of \LaTeX{})
or to generate a smaller document which is easier to navigate.
Another application of |\includeonly| is to generate
documents consisting of selected parts of the complete document.

However, there are a few drawbacks of the plain |\include| mechanism:
\begin{itemize}
\item
The child files cannot be compiled on their own,
they can only be compiled via the main file.
A naive editing environment
(such as a text editor with an option
to have the current file processed by \LaTeX)
may require one to switch to the main file before compiling;
attempting to compile the child file produces errors.
\item
The main file must be modified (each time)
to adjust the |\includeonly| command
to the present needs. This easily leaves the main file in a messy state.
\item
The generated document will always carry the filename
of the main document. This is inconvenient if
several child files are to be compiled and
to be kept for distribution.
\end{itemize}

The present package provides a simple interface
to make child files individually compilable by \LaTeX{}.
Compiling a child file then has the same effect as compiling
the main file with an |\includeonly| command
to select the appropriate child.
Moreover the generated document will carry the name of the child
rather than the main file.
This resolves all three above issues.

This feature is meant to make the editing of books,
thesis documents and lecture notes somewhat more convenient.
However, the package can also be used efficiently for
composing a series of documents (such as exercise sheets)
which are typically distributed individually.
It then assists the author in generating the individual documents
(potentially in different versions)
as well as a document containing the collected series.
Another application is in developing style files
or other kinds of included material
where compilation of the style file could redirect
to a sample or test file.

%%%%%%%%%%%%%%%%%%%%%%%%%%%%%%%%%%%%%%%%%%%%%%%%%%%%%%%%%%%%%%%%%%%%%%%%%%%%%%%%
%%%%%%%%%%%%%%%%%%%%%%%%%%%%%%%%%%%%%%%%%%%%%%%%%%%%%%%%%%%%%%%%%%%%%%%%%%%%%%%%
\section{Usage}

First of all, the package \textsf{childdoc} is \emph{not} a standard
\LaTeXe{} |.sty| style file! Therefore it needs to be invoked in
a non-standard way.

%%%%%%%%%%%%%%%%%%%%%%%%%%%%%%%%%%%%%%%%%%%%%%%%%%%%%%%%%%%%%%%%%%%%%%%%%%%%%%%%
\subsection{Included Files}
\label{sec:include}

%%%%%%%%%%%%%%%%%%%%%%%%%%%%%%%%%%%%%%%%
\DescribeMacro{\childdocmain}
To use the package, add the commands
\begin{center}
\begin{tabular}{l}
|\input{childdoc.def}|\\
|\childdocmain{}|\\
\end{tabular}
\end{center}
at the very top of the main \LaTeX{} file,
in particular \emph{before} the |\documentclass| statement!
The argument of |\childdocmain| should be left empty
(but it must be present).

%%%%%%%%%%%%%%%%%%%%%%%%%%%%%%%%%%%%%%%%
\DescribeMacro{\childdocof}
Furthermore, add the commands
\begin{center}
\begin{tabular}{l}
|\input{childdoc.def}|\\
|\childdocof{|\textit{main}|}|\\
\end{tabular}
\end{center}
at the top of every child file \textit{child}
which is included by |\include{|\textit{child}|}|
from within the main file
(or at least for those files to be compiled individually).
The argument \textit{main} must be the filename of the main file.

There are a couple of
considerations in setting up the main and child documents:

%%%%%%%%%%%%%%%%%%%%%%%%%%%%%%%%%%%%%%%%
\paragraph{Restrictions.}

Please note the following restrictions:
\begin{itemize}
\item
|\childdocmain| must be called with one argument \textit{main}
to ensure compatibility with earlier version of the package.
It must either be empty (|\childdocmain{}|)
or precisely match the filename of the main file in which it is specified.
See \secref{sec:detection} for further information.
\item
The filename \textit{main} must be specified without the |.tex| extension.
\item
The filename \textit{main} is case sensitive
(even in case-insensitive file systems)
due to internal string comparison.
\item
The argument \textit{main} should be fully expanded, it cannot be a macro.
\item
Subdirectories and special characters should be avoided in filenames.
\item
The command |\childdocmain{|\textit{main}|}| must be followed by a whitespace.
It should not be followed immediately by another command
or by a comment mark `|%|'.
This is because the \TeX{} parser reads the token immediately following
the argument of |\childdocmain| and puts it
at the beginning of every child section;
however, a white\-space is ignored.
\end{itemize}

%%%%%%%%%%%%%%%%%%%%%%%%%%%%%%%%%%%%%%%%
\paragraph{Content of Main File.}

It is advisable to place all content in the child files included by |\include|.
Any output contained in the main file will appear in all child documents
unless suppressed manually;
it cannot be suppressed automatically by the |\includeonly| directive
and thus should normally be avoided.
A method to include some content in the main file
by means of conditional processing is described in \secref{sec:conditional}.

%%%%%%%%%%%%%%%%%%%%%%%%%%%%%%%%%%%%%%%%
\paragraph{Page Numbering.}

When only a part of the document is compiled,
the appropriate numbering of pages
(as well as other status parameters)
is determined from the |.aux| files.
The latter contain information from previous passes.
However this information needs to propagate through
all intermediate child documents.
Therefore the page numbering in child documents may well
be inconsistent until the complete document is compiled at least once.

A useful (if unconventional) way to always ensure a consistent
page numbering is to restart the numbering in each child document
and denote the pages by `\textit{child}|.|\textit{page}'
where \textit{child} represents the chapter/section number of the child file.
This can be achieved by the command
|\numberwithin{page}{|\textit{child}|}|
of the \textsf{amsmath} package
where \textit{child} can be |chapter| or |section|
depending on the chosen structuring.
Alternatively, one can modify the macro |\thepage| appropriately
and reset the counter |page| at the start of each child file.

%%%%%%%%%%%%%%%%%%%%%%%%%%%%%%%%%%%%%%%%%%%%%%%%%%%%%%%%%%%%%%%%%%%%%%%%%%%%%%%%
\subsection{Conditional Processing}
\label{sec:conditional}

The package provides a mechanism to compile different versions
of a document. To customise the versions further some conditional processing
can come in handy to distinguish which version is being compiled.
The package provides two macros to describe the compilation context:

%%%%%%%%%%%%%%%%%%%%%%%%%%%%%%%%%%%%%%%%
\DescribeMacro{\ifchilddoc}
The conditional |\ifchilddoc| distinguishes between the compilation of
child documents and the main document:
%
\begin{center}
|\ifchilddoc |\textit{child-code}| |[|\||else |\textit{main-code}]| \||fi|
\end{center}

%%%%%%%%%%%%%%%%%%%%%%%%%%%%%%%%%%%%%%%%
\DescribeMacro{\childdocname}
\DescribeMacro{\childdocjob}
The macro |\childdocname| contains the filename (without extension)
of the main or child file being processed.
Note that |\childdocjob| will always contain the name of the main file.

%%%%%%%%%%%%%%%%%%%%%%%%%%%%%%%%%%%%%%%%
\paragraph{Title Page.}

Conditional processing can be used to include a title or banner page
in the main document when proper precautions are taken.
Importantly, the code in the main file should ensure that the page counter
(as well as other status parameters which are stored in the |.aux| files)
takes the same value after the conditional processing.
Otherwise the page numbers may take divergent values
depending on which part is compiled.

For example, a title page could be declared by:
%
\begin{center}
\begin{tabular}{l}
|\ifchilddoc\||else|\\
|\addtocounter{page}{-1}|\\
\textit{code for title page}\\
|\newpage|\\
|\||fi|
\end{tabular}
\end{center}
%
A banner page for the child documents can be generated by:
%
\begin{center}
\begin{tabular}{l}
|\ifchilddoc|\\
|\addtocounter{page}{-1}|\\
\textit{code for banner page}\\
|\newpage|\\
|\||fi|
\end{tabular}
\end{center}
%
Here one could write a message such as:
\begin{center}
|This is the part \childdocname{} of \childdocjob{}.|
\end{center}

%%%%%%%%%%%%%%%%%%%%%%%%%%%%%%%%%%%%%%%%%%%%%%%%%%%%%%%%%%%%%%%%%%%%%%%%%%%%%%%%
\subsection{Flags}
\label{sec:flags}

The package makes it easy to generate different versions
of the main or child documents.
To this end compilation flags can be defined
and assigned different default values.
They will be particularly useful in conjunction
with the forwarding mechanism described in \secref{sec:forward}.

For example, it may be useful to have a flag |\version|
which can be set to |draft| or |final|.
The document source will contain some conditional code
depending on the value of |\version|.
Suppose further, the flag should default to |final| for the main file
and to |draft| for child files
which is a natural assignment for editing the document.
This is achieved by placing the following code
in the preamble of the main document
(below the |\childdocmain| directive):
%
\begin{center}
\begin{tabular}{l}
|\ifchilddoc|\\
|\providecommand{\version}{draft}|\\
|\||else|\\
|\providecommand{\version}{final}|\\
|\||fi|
\end{tabular}
\end{center}
%
The definition by |\providecommand| makes sure
that previous definitions are not overwritten.
Further statements |\providecommand{\version}{...}|
can thus be added before the above code to override it.

For the main file, one might add a line
(between |\childdocmain| and the above block)
%
\begin{center}
|%\ifchilddoc\||else\providecommand{\version}{draft}\||fi|
\end{center}
%
which can be uncommented to produce a draft version.
Likewise one can add a line to the very top of a child file
(above the |\childdocof{|\textit{main}|}| directive)
%
\begin{center}
|%\providecommand{\version}{final}|
\end{center}
%
which can be uncommented to produce the final version of this child document.

%%%%%%%%%%%%%%%%%%%%%%%%%%%%%%%%%%%%%%%%%%%%%%%%%%%%%%%%%%%%%%%%%%%%%%%%%%%%%%%%
\subsection{Forwarding}
\label{sec:forward}

Different versions of the main or child documents
using compilation flags as described in \secref{sec:flags}
can be (permanently) stored in different files
for convenient compilation, viewing and distribution.
To this end, the package defines a command
to pass on compilation to a different file:

%%%%%%%%%%%%%%%%%%%%%%%%%%%%%%%%%%%%%%%%
\DescribeMacro{\childdocforward}
The command |\childdocforward| redirects processing to
another source file:
%
\begin{center}
\begin{tabular}{l}
|\input{childdoc.def}|\\
|\childdocforward[|\textit{main}|]{|\textit{dest}|}|\\
\end{tabular}
\end{center}
%
The argument \textit{dest} is the destination file
(without extension).
It should be the main file or one of the child files.
Note that further \textsf{childdoc} directives
such as |\childdocof| and |\childdocforward|
in the indicated file will be processed in this form.
The optional argument \textit{main}
passes on directly to the main file \textit{main}
while pretending to compile the child \textit{dest}.
This form behaves as if \textit{dest}
issues |\childdocof{|\textit{main}|}| right away,
and no further \textsf{childdoc} directives will be processed.

%%%%%%%%%%%%%%%%%%%%%%%%%%%%%%%%%%%%%%%%
\DescribeMacro{\...prefix}
In the alternative form |\childdocforwardprefix|,
%
\begin{center}
\begin{tabular}{l}
|\input{childdoc.def}|\\
|\childdocforwardprefix[|\textit{main}|]{|\textit{prefix}|}{|\textit{dest}|}|
\end{tabular}
\end{center}
%
the destination file is determined by a pattern
depending on the current file:
To make this work, the current file must be called
`{\textit{prefix}\hspace{0.2em}\textit{suffix}}'
with \textit{prefix} matching precisely the argument.
Processing is then passed on to the file
`{\textit{dest}\hspace{0.2em}\textit{suffix}}'.
Surely, the same effect is achieved by
directly specifying the
argument `{\textit{dest}\hspace{0.2em}\textit{suffix}}'
in the first form.
However, that requires to set up a different file
for each child. With the alternative form of the command
all these files can have exactly the same content
which simplifies setting them up and maintaining them.

For example, the following file |draft.tex|
with a compilation flag |\version| as described in \secref{sec:flags}
compiles the main document as a draft:
%
\begin{center}
\begin{tabular}{l}
|\def\version{draft}|\\
|\input{childdoc.def}|\\
|\childdocforward{|\textit{main}|}|
\end{tabular}
\end{center}
%
Likewise, the following files |final|\textit{nn}|.tex|
compile the final version of the child document
|child|\textit{nn}|.tex|:
%
\begin{center}
\begin{tabular}{l}
|\def\version{final}|\\
|\input{childdoc.def}|\\
|\childdocforwardprefix{final}{child}|
\end{tabular}
\end{center}
%

Note that when several versions of a main file and/or of each child file
are to be generated, it may be convenient to set up a |Makefile| or
shell script to automatise the process.

%%%%%%%%%%%%%%%%%%%%%%%%%%%%%%%%%%%%%%%%%%%%%%%%%%%%%%%%%%%%%%%%%%%%%%%%%%%%%%%%
\subsection{Command Line Processing}
\label{sec:commandline}

The effect of redirection files can also be achieved by invoking
the \LaTeX{} compiler with a more elaborate command line.
Most conveniently this should be done as part
of a shell script or a |Makefile|.

When using \textsf{childdoc} in the main file, the following
command lines effectively perform a redirection
(note that depending on the shell being used,
backslashes may have to be doubled: `|\|' $\to$ `|\\|'):
%
\begin{center}
|... -jobname "|\textit{target}|" |\\|"|[\textit{flags}]%
|\input{childdoc.def}\childdocforward[|\textit{main}|]{|\textit{dest}|}"|
\end{center}
%
Here \textit{target} is the name of the output file,
\textit{main} is the name of the main file
and \textit{dest} is the name of the main or child file to be processed
(all filenames without extensions).
The optional argument \textit{main} can be omitted
if \textit{main} matches \textit{dest}.
Optionally, compilation \textit{flags} can be defined via |\def| commands.
This command line makes the \TeX{} engine believe
it is compiling the file \textit{target}
whose content is specified as the latter parameter.
The provided code then forwards the processing to
\textit{main} or \textit{dest} as described in \secref{sec:forward}.

%%%%%%%%%%%%%%%%%%%%%%%%%%%%%%%%%%%%%%%%%%%%%%%%%%%%%%%%%%%%%%%%%%%%%%%%%%%%%%%%
\subsection{Include by Input}
\label{sec:input}

Including child documents by |\include| has some restrictions by design.
Most notably, the content of a child document always occupies
its own set of pages; pages cannot be shared between child documents.
Usually, this behaviour makes perfect sense
because each child document contain an essential part of the document.
However, in some situations it may be desirable to compose
a document from a collection of parts
without having mandatory page breaks between then.
For this case, the package
provides a mechanism to include parts
by |\input| which can also be processed individually.
However, by construction this mechanism
requires manual handling of the content to be output.

%%%%%%%%%%%%%%%%%%%%%%%%%%%%%%%%%%%%%%%%
\DescribeMacro{\ifchilddocmanual}
The main file should be prepared as usual, see \secref{sec:include}.
However, the document body must make a distinction
between processing of an individual part and of the main document, e.g.:
%
\begin{center}
\begin{tabular}{l}
|\ifchilddocmanual|\\
|\input{\childdocname}|\\
|\||else|\\
\textit{document body with }|\input{|\textit{part}|}|\\
|\||fi|
\end{tabular}
\end{center}
%
The conditional |\ifchilddocmanual| is true whenever
a part to be included by |\input| is being compiled,
and the name of the part is stored in |\childdocname|.

%%%%%%%%%%%%%%%%%%%%%%%%%%%%%%%%%%%%%%%%
\DescribeMacro{\childdocby}
Each part to be included by |\input| should start with:
%
\begin{center}
\begin{tabular}{l}
|\input{childdoc.def}|\\
|\childdocby{|\textit{main}|}|\\
\end{tabular}
\end{center}
%
The directive |\childdocby| is similar to |\childdocof|
described in \secref{sec:include},
but the subsequent selection of content must be done manually.
To that end, both |\ifchilddoc| and |\ifchilddocmanual|
will be true upon processing of a part,
and the name of the part is stored in |\childdocname|.
Note that |\jobname| will be set to the filename of the current part
so that each part receives an individual |.aux| file
that does not interfere with the |.aux| file(s) of the main document.
This behaviour can be altered by the alternative form
|\childdocby[*]{|\textit{main}|}| (with a non-empty optional argument)
which uses the |.aux| file of the main document
by setting |\jobname| to \textit{main}.

%%%%%%%%%%%%%%%%%%%%%%%%%%%%%%%%%%%%%%%%%%%%%%%%%%%%%%%%%%%%%%%%%%%%%%%%%%%%%%%%
\subsection{Driver Development}
\label{sec:driver}

The \textsf{childdoc} mechanism can also be use for the development
of definition files such as \LaTeX{} styles or classes.
This case differs from the above setup with multiple parts
included by |\include| in that no |\includeonly| should be invoked.
This can be achieved by starting the include file
(before |\ProvidesPackage|) with:
%
\begin{center}
\begin{tabular}{l}
|\input{childdoc.def}|\\
|\childdocforward{|\textit{main}|}|\\
\end{tabular}
\end{center}
%
or alternatively with:
%
\begin{center}
\begin{tabular}{l}
|\input{childdoc.def}|\\
|\childdocby{|\textit{main}|}|\\
\end{tabular}
\end{center}
%
Both forms have slightly different effects as described above.
The main file is prepared as usual, see \secref{sec:include}.

%%%%%%%%%%%%%%%%%%%%%%%%%%%%%%%%%%%%%%%%%%%%%%%%%%%%%%%%%%%%%%%%%%%%%%%%%%%%%%%%
\subsection{Legacy Detection}
\label{sec:detection}

The directive |\childdocmain| in the main file can detect
whether the complete document or merely a child is to be compiled
even without using the directive |\childdocof|.
This method is deprecated because it is less robust
and there is no compelling reason to use it;
it is merely provided for backward compatibility
and it may be removed in future versions.

If the detection mechanism is to be used,
it is mandatory to correctly specify
the filename of the main file as the argument of |\childdocmain|:
%
\begin{center}
\begin{tabular}{l}
|\input{childdoc.def}|\\
|\childdocmain{|\textit{main}|}|\\
\end{tabular}
\end{center}
%
If |\jobname| does not match the argument \textit{main} of |\childdocmain|,
it is assumed that |\jobname| points to the child file to be compiled.
When using |\childdocmain| with the main file specified as argument,
it suffices to start a child file
with just |\input{|\textit{main}|}|
without loading of the package and using |\childdocof|.
If instead all processing is done
with the appropriate \textsf{childdoc} directives,
the argument of \textit{main} of |\childdocmain| can be empty.

An alternative version of the command line processing described
in \secref{sec:commandline} using the detection mechanism reads:
%
\begin{center}
|... -jobname "|\textit{target}|" "|[\textit{flags}]%
[|\def\jobname{|\textit{dest}|}|]|\input{|\textit{main}|}"|
\end{center}

%%%%%%%%%%%%%%%%%%%%%%%%%%%%%%%%%%%%%%%%%%%%%%%%%%%%%%%%%%%%%%%%%%%%%%%%%%%%%%%%
\subsection{Manual Code}
\label{sec:manual}

In case one cannot be certain whether the definitions file |childdoc.def|
is installed on the target \TeX{} distribution
and one prefers not to ship it,
it is conceivable to paste a few relevant commands into the sources.

To that end, drop all statements |\input{childdoc.def}|
and perform the replacements as outlined below.
Instead of |\childdocmain{|\textit{main}|}| add the following code
to the top of the main file:
%
\begin{center}
\begin{tabular}{l}
|\||ifdefined\childdocname\endinput\||fi\newif\ifchilddoc|\\
|\edef\childdocname{\scantokens\expandafter{\jobname\noexpand}}|\\
|\def\childdocmain{|\textit{main}|}\||ifx\childdocmain\childdocname\||else|\\
|\childdoctrue\includeonly{\childdocname}\let\jobname\childdocmain\||fi|\\
\end{tabular}
\end{center}
%
Instead of |\childdocof{|\textit{main}|}| just include the main file
at the top of each child file:
%
\begin{center}
|\input{|\textit{main}|}|
\end{center}
%
A simple redirection |\childdocforward{|\textit{dest}|}| is achieved by:
%
\begin{center}
|\def\jobname{|\textit{dest}|}\input{\jobname}|
\end{center}
%
The redirection with prefix
|\childdocforwardprefix[|\textit{prefix}|]{|\textit{dest}|}|
is accomplished by:
%
\begin{center}
\begin{tabular}{l}
|{\edef\jobname{\scantokens\expandafter{\jobname\noexpand}}|\\
|\def\redirectjob |\textit{prefix}|#1~~~{\gdef\jobname{|\textit{dest}|#1}}|\\
|\expandafter\redirectjob\jobname~~~}\input{\jobname}|
\end{tabular}
\end{center}

In an alternative approach,
child documents can be compiled by a specific command line
without additional code or specific definitions:
%
\begin{center}
|... -jobname "|\textit{target}|" "|[\textit{flags}]%
|\includeonly{|\textit{dest}|}\input{|\textit{main}|}"|
\end{center}
%

%%%%%%%%%%%%%%%%%%%%%%%%%%%%%%%%%%%%%%%%%%%%%%%%%%%%%%%%%%%%%%%%%%%%%%%%%%%%%%%%
%%%%%%%%%%%%%%%%%%%%%%%%%%%%%%%%%%%%%%%%%%%%%%%%%%%%%%%%%%%%%%%%%%%%%%%%%%%%%%%%
\section{Information}

%%%%%%%%%%%%%%%%%%%%%%%%%%%%%%%%%%%%%%%%%%%%%%%%%%%%%%%%%%%%%%%%%%%%%%%%%%%%%%%%
\subsection{Copyright}

Copyright \copyright{} 2017--2018 Niklas Beisert

This work may be distributed and/or modified under the
conditions of the \LaTeX{} Project Public License, either version 1.3
of this license or (at your option) any later version.
The latest version of this license is in
  \url{http://www.latex-project.org/lppl.txt}
and version 1.3 or later is part of all distributions of \LaTeX{}
version 2005/12/01 or later.

This work has the LPPL maintenance status `maintained'.

The Current Maintainer of this work is Niklas Beisert.

This work consists of the files |README.txt|, |childdoc.ins| and |childdoc.dtx|
as well as the derived files |childdoc.def|, |cdocsamp.tex|
with |cdocsch1.tex|, |cdocsch2.tex|, |cdocspt3.tex|, |cdocspt4.tex|,
|cdocsdrf.tex|, |cdocsfn1.tex|, |cdocsfn2.tex|
as well as |childdoc.pdf|.

%%%%%%%%%%%%%%%%%%%%%%%%%%%%%%%%%%%%%%%%%%%%%%%%%%%%%%%%%%%%%%%%%%%%%%%%%%%%%%%%
\subsection{Files and Installation}

The package consists of the files:
%
\begin{center}
\begin{tabular}{ll}
    |README.txt|   & readme file \\
    |childdoc.ins| & installation file \\
    |childdoc.dtx| & source file \\
    |childdoc.def| & definition file \\
    |cdocsamp.tex| & sample main file \\
    |cdocsch1.tex| & sample include file \\
    |cdocsch2.tex| & sample include file \\
    |cdocspt3.tex| & sample part file \\
    |cdocspt4.tex| & sample part file \\
    |cdocsdrf.tex| & sample redirection file \\
    |cdocsfn1.tex| & sample redirection file \\
    |cdocsfn2.tex| & sample redirection file \\
    |childdoc.pdf| & manual
\end{tabular}
\end{center}
%
The distribution consists of the files
|README.txt|, |childdoc.ins| and |childdoc.dtx|.
%
\begin{itemize}
\item
Run (pdf)\LaTeX{} on |childdoc.dtx|
to compile the manual |childdoc.pdf| (this file).
\item
Run \LaTeX{} on |childdoc.ins| to create the definitions file |childdoc.def|
and the sample |cdocsamp.tex| with include files
|cdocsch1.tex|, |cdocsch2.tex|, |cdocspt3.tex|, |cdocspt4.tex|,
|cdocsdrf.tex|, |cdocsfn1.tex|, |cdocsfn2.tex|.
Then copy the file |childdoc.def| to an appropriate directory of your \LaTeX{}
distribution, e.g.\ \textit{texmf-root}|/tex/latex/childdoc|.
\end{itemize}

%%%%%%%%%%%%%%%%%%%%%%%%%%%%%%%%%%%%%%%%%%%%%%%%%%%%%%%%%%%%%%%%%%%%%%%%%%%%%%%%
\subsection{Related CTAN Packages}

There are several other packages which offer a similar functionality:
%
\begin{itemize}
\item
The packages
\href{http://ctan.org/pkg/docmute}{\textsf{docmute}},
\href{http://ctan.org/pkg/includex}{\textsf{includex}} and
\href{http://ctan.org/pkg/standalone}{\textsf{standalone}}
provide commands to include only the document body of
a child file thus allowing both files to be compiled individually.
\item
The packages \href{http://ctan.org/pkg/subdocs}{\textsf{subdocs}}
and \href{http://ctan.org/pkg/subfiles}{\textsf{subfiles}}
provide structures in which the main and child documents can be
encapsulated and allowing them to be compiled individually.
The inclusion mechanism is different from the conventional |\include|.
\item
The package \href{http://ctan.org/pkg/combine}{\textsf{combine}}
is an elaborate solution to combine several documents into one.
\end{itemize}
%
See also the CTAN topic \href{http://ctan.org/topic/subdocs}{\textsf{subdocs}}
for further related packages.
The present package differs from the above solutions in that
a document structure constructed with the conventional |\include| mechanism
just needs two extra commands at the top of every file
such that all constituent files can be compiled individually.

%%%%%%%%%%%%%%%%%%%%%%%%%%%%%%%%%%%%%%%%%%%%%%%%%%%%%%%%%%%%%%%%%%%%%%%%%%%%%%%%
%\subsection{Feature Suggestions}
%
%The following is a list of features which may be useful for future
%versions of this package:
%%
%\begin{itemize}
%\item
%\ldots
%\end{itemize}

%%%%%%%%%%%%%%%%%%%%%%%%%%%%%%%%%%%%%%%%%%%%%%%%%%%%%%%%%%%%%%%%%%%%%%%%%%%%%%%%
\subsection{Revision History}

%%%%%%%%%%%%%%%%%%%%%%%%%%%%%%%%%%%%%%%%
\paragraph{v2.0:} 2018/12/30

\begin{itemize}
\item
immediate forward processing
\item
added |\childdocby| mechanism
\item
manual restructured
\end{itemize}

%%%%%%%%%%%%%%%%%%%%%%%%%%%%%%%%%%%%%%%%
\paragraph{v1.6:} 2018/01/17

\begin{itemize}
\item
application for development of include files
\item
corrections to manual
\end{itemize}

%%%%%%%%%%%%%%%%%%%%%%%%%%%%%%%%%%%%%%%%
\paragraph{v1.5:} 2017/05/21

\begin{itemize}
\item
more complete structuring introduced
\item
|\childdocof| introduced
\item
|\childdoc| renamed to |\childdocmain|
\item
|\childredirect| renamed to |\childdocforward| and |\childdocforwardprefix|
and functionality expanded
\end{itemize}

%%%%%%%%%%%%%%%%%%%%%%%%%%%%%%%%%%%%%%%%
\paragraph{v1.0:} 2017/04/27

\begin{itemize}
\item
manual and install package
\item
first version published on CTAN
\end{itemize}

%%%%%%%%%%%%%%%%%%%%%%%%%%%%%%%%%%%%%%%%
\paragraph{v0.6:} 2017/04/26

\begin{itemize}
\item
redirection mechanism added
\end{itemize}

%%%%%%%%%%%%%%%%%%%%%%%%%%%%%%%%%%%%%%%%
\paragraph{v0.5:} 2017/04/26

\begin{itemize}
\item
functionality in definition file
\end{itemize}


%%%%%%%%%%%%%%%%%%%%%%%%%%%%%%%%%%%%%%%%%%%%%%%%%%%%%%%%%%%%%%%%%%%%%%%%%%%%%%%%
%%%%%%%%%%%%%%%%%%%%%%%%%%%%%%%%%%%%%%%%%%%%%%%%%%%%%%%%%%%%%%%%%%%%%%%%%%%%%%%%
%%%%%%%%%%%%%%%%%%%%%%%%%%%%%%%%%%%%%%%%%%%%%%%%%%%%%%%%%%%%%%%%%%%%%%%%%%%%%%%%
\appendix

\settowidth\MacroIndent{\rmfamily\scriptsize 000\ }

 \DocInput{childdoc.dtx}

\end{document}
%</driver>
% \fi
%
% %%%%%%%%%%%%%%%%%%%%%%%%%%%%%%%%%%%%%%%%%%%%%%%%%%%%%%%%%%%%%%%%%%%%%%%%%%%%%%
% %%%%%%%%%%%%%%%%%%%%%%%%%%%%%%%%%%%%%%%%%%%%%%%%%%%%%%%%%%%%%%%%%%%%%%%%%%%%%%
% \section{Sample}
%\iffalse
%<*samplemain>
%\fi
%
% The following presents a sample document
% with two chapters, two parts, a title page,
% a compile flag as well as three forwarding files to set the flag.
% It consists of eight |.tex| files:
% \begin{center}
% \begin{tabular}{ll}
% |cdocsamp.tex|&main file\\
% |cdocsch1.tex|&include file for chapter 1\\
% |cdocsch2.tex|&include file for chapter 2\\
% |cdocspt3.tex|&include file for part 3\\
% |cdocspt4.tex|&include file for part 4\\
% |cdocsdrf.tex|&forwarding file for main file in draft mode\\
% |cdocsfi1.tex|&forwarding file for final version of chapter 1\\
% |cdocsfi2.tex|&forwarding file for final version of chapter 2\\
% \end{tabular}
% \end{center}
% Each of the eight files can be compiled directly by the \LaTeX{} compiler.
%
% %%%%%%%%%%%%%%%%%%%%%%%%%%%%%%%%%%%%%%
% \paragraph{Main File.}
%
% The main file is called |cdocsamp.tex|.
%
% Load the \textsf{childdoc} definitions and
% declare the filename for the main document:
%    \begin{macrocode}
\input{childdoc.def}
\childdocmain{}
%    \end{macrocode}

% Optional override for |\version| flag:
%    \begin{macrocode}
%%\ifchilddoc\else\providecommand{\version}{draft}\fi
%    \end{macrocode}

% Define the default values for the |\version| flag
% (|final| for the main file and |draft| for childs):
%    \begin{macrocode}
\ifchilddoc
\providecommand{\version}{draft}
\else
\providecommand{\version}{final}
\fi
%    \end{macrocode}

% Load the standard document class:
%    \begin{macrocode}
\documentclass[12pt]{article}
%    \end{macrocode}

% Start the document body:
%    \begin{macrocode}
\begin{document}
%    \end{macrocode}

% Declare a title page.
% Print title, part of document being processed and version flag:
%    \begin{macrocode}
\addtocounter{page}{-1}
\begin{center}
{\LARGE\bfseries{}childdoc example\par}
\vspace{1cm}
\ifchilddoc
\ifchilddocmanual part\else chapter\fi:
`\childdocname' of `\childdocjob'\par
\else
main document: `\childdocjob'\par
\fi
version: \version\par
\end{center}
\newpage
%    \end{macrocode}

% Manually include selected file,
% otherwise process as usual:
%    \begin{macrocode}
\ifchilddocmanual
\section*{part `\childdocname'}
\input{\childdocname}
\else
%    \end{macrocode}

% Include the two chapters:
%    \begin{macrocode}
\include{cdocsch1}
\include{cdocsch2}
%    \end{macrocode}

% Include the two parts unless only chapters should be displayed:
%    \begin{macrocode}
\ifchilddoc\else
\section{part three}
\input{cdocspt3}
\section{part four}
\input{cdocspt4}
\fi
%    \end{macrocode}

% Process as usual until here:
%    \begin{macrocode}
\fi
%    \end{macrocode}

% End of document body:
%    \begin{macrocode}
\end{document}
%    \end{macrocode}
%\iffalse
%</samplemain>
%\fi
%
% %%%%%%%%%%%%%%%%%%%%%%%%%%%%%%%%%%%%%%
% \paragraph{Chapter Include Files.}
%
% The include files are called |cdocsch1.tex| and |cdocsch2.tex|.
%
%\iffalse
%<*samplechap1|samplechap2>
%\fi

% Optional override for |\version| flag:
%    \begin{macrocode}
%%\providecommand{\version}{final}
%    \end{macrocode}

% Include the main document:
%    \begin{macrocode}
\input{childdoc.def}
\childdocof{cdocsamp}
%    \end{macrocode}

%\iffalse
%</samplechap1|samplechap2>
%\fi
%
%\iffalse
%<*samplechap1>
%\fi
% Some text for chapter 1:
%    \begin{macrocode}
\section{one}
some text in chapter one
%    \end{macrocode}

%\iffalse
%</samplechap1>
%\fi
% Some text for chapter 2:
%\iffalse
%<*samplechap2>
%\fi
%    \begin{macrocode}
\section{two}
more text in chapter two
%    \end{macrocode}

%\iffalse
%</samplechap2>
%\fi
%
% %%%%%%%%%%%%%%%%%%%%%%%%%%%%%%%%%%%%%%
% \paragraph{Part Include Files.}
%
% The include files are called |cdocspt3.tex| and |cdocspt4.tex|.
%
%\iffalse
%<*samplepart3|samplepart4>
%\fi

% Optional override for |\version| flag:
%    \begin{macrocode}
%%\providecommand{\version}{final}
%    \end{macrocode}

% Include the main document:
%    \begin{macrocode}
\input{childdoc.def}
\childdocby{cdocsamp}
%    \end{macrocode}

%\iffalse
%</samplepart3|samplepart4>
%\fi
%
%\iffalse
%<*samplepart3>
%\fi
% Some text for part 3:
%    \begin{macrocode}
some text in part three
%    \end{macrocode}

%\iffalse
%</samplepart3>
%\fi
% Some text for part 4:
%\iffalse
%<*samplepart4>
%\fi
%    \begin{macrocode}
more text in part four
%    \end{macrocode}

%\iffalse
%</samplepart4>
%\fi
%
% %%%%%%%%%%%%%%%%%%%%%%%%%%%%%%%%%%%%%%
% \paragraph{Forwarding for a Complete Draft.}
%
% The following forwarding file |cdocsdrf.tex|
% compiles the main document in draft mode:
%\iffalse
%<*sampledraft>
%\fi
%    \begin{macrocode}
\def\version{draft}
\input{childdoc.def}
\childdocforward{cdocsamp}
%    \end{macrocode}

%\iffalse
%</sampledraft>
%\fi
%
% %%%%%%%%%%%%%%%%%%%%%%%%%%%%%%%%%%%%%%
% \paragraph{Forwarding for Final Version of the Chapters.}
%
% The following forwarding files |cdocsfn1.tex| and |cdocsfn2.tex|
% (with identical content)
% compile the final versions of the child documents
% |cdocsch1.tex| and |cdocsch2.tex|, respectively:
%\iffalse
%<*samplefinal>
%\fi
%    \begin{macrocode}
\def\version{final}
\input{childdoc.def}
\childdocforwardprefix[cdocsamp]{cdocsfn}{cdocsch}
%    \end{macrocode}

%\iffalse
%</samplefinal>
%\fi
%
% %%%%%%%%%%%%%%%%%%%%%%%%%%%%%%%%%%%%%%
% \paragraph{Command Line Processing.}
%
% The following three command lines generate the output files
% |cdocscld|, |cdocscl1| and |cdocscl2|
% which should be identical to
% |cdocsdrf|, |cdocsch1| and |cdocsfn2|, respectively:
% \begin{center}
% \begin{tabular}{l}
% |latex -jobname cdocscld \|\\
% |  "\def\version{draft}\input{childdoc.def}\childdocforward{cdocsamp}"|\\
% |latex -jobname cdocscl1 \|\\
% |  "\input{childdoc.def}\childdocforward[cdocsamp]{cdocsch1}"|\\
% |latex -jobname cdocscl2 \|\\
% |  "\def\version{final}\input{childdoc.def}\childdocforward{cdocsch2}"|
% \end{tabular}
% \end{center}
% Note that the trailing backslash on each first line
% merely continues the input to the second line
% (for convenient cut ant paste).
% Furthermore, the command |latex| can be replaced by any
% of its alternative versions such as |pdflatex|.
%
% %%%%%%%%%%%%%%%%%%%%%%%%%%%%%%%%%%%%%%%%%%%%%%%%%%%%%%%%%%%%%%%%%%%%%%%%%%%%%%
% %%%%%%%%%%%%%%%%%%%%%%%%%%%%%%%%%%%%%%%%%%%%%%%%%%%%%%%%%%%%%%%%%%%%%%%%%%%%%%
% \section{Implementation}
%\iffalse
%<*package>
%\fi
%
% This section describes the definitions file |childdoc.def|.

% The definitions cannot be loaded using |\usepackage| or |\RequirePackage|
% which has a mechanism to prevent loading a style file more than once.
% When loading the definitions by means of |\input|
% multiple instances have to be prevented manually:
%\iffalse
%This code needs to be before the `\ProvidesFile' directive
%which is defined at the beginning of this file.
%Therefore it is also placed there and commented out here.
%</package>
%<*discard>
%\fi
%    \begin{macrocode}
\ifdefined\childdocmain\endinput\fi
%    \end{macrocode}
%\iffalse
%</discard>
%<*package>
%\fi
%
% \macro{\ifchilddoc}
% \macro{\ifchilddocmanual}
% The conditional |\ifchilddoc| tells whether a
% child (true) or main (false) document is being compiled.
% The conditional |\ifchilddocmanual| tells whether
% the |\includeonly| mechanism is used (false) or
% the selection of child files must be performed manually (true).
% The definitions initialise to false:
%    \begin{macrocode}
\newif\ifchilddoc
\newif\ifchilddocmanual
%    \end{macrocode}

% \macro{\childdocname}
% \macro{\childdocjob}
% The macro |\childdocname| stores the name of the main document
% to be compiled. The macro |\childdocjob| stores the name of
% the document on which the \LaTeX{} compiler was originally invoked.
% The content of |\jobname| cannot be compared
% to filenames specified in the source due to different catcodes.
% The following code rescans |\jobname|, stores the result
% in |\childdocname| and saves a copy in |\childdocjob|:
%    \begin{macrocode}
\edef\childdocname{\scantokens\expandafter{\jobname\noexpand}}
\let\childdocjob\childdocname
%    \end{macrocode}

% \macro{\childdocdisable}
% The macro |\childdocdisable| prevents the main file
% from being processed more than once.
% At this stage, the main document command |\childdocmain|
% is assumed to be called once again where it should do nothing.
% Any subsequent call to it should prevent
% a secondary processing of the main document
% It overwrites the forwarding commands
% |\childdocof| and |\childdocforward|
% with empty macros to prevent further inclusions of the main document:
%    \begin{macrocode}
\newcommand{\childdocdisable}
{
  \renewcommand{\childdocmain}[1]{\renewcommand{\childdocmain}[1]{\endinput}}
  \renewcommand{\childdocof}[1]{}
  \renewcommand{\childdocby}[2][]{}
  \renewcommand{\childdocforward}[2][]{}
  \renewcommand{\childdocdisable}{}
}
%    \end{macrocode}

% \macro{\childdocmain}
% The macro |\childdocmain| is to be called at the top of the main file
% with nothing or the main filename (without extension) as argument.
% First, it breaks loops.
% If the argument is not empty and does not match |\childdocname|
% (which is set by the first inclusion of |childdoc.def|),
% |\ifchilddoc| is set to true, |\includeonly| is applied to the child file
% and |\jobname| is set to the main file
% (for proper handling of |.aux| files):
%    \begin{macrocode}
\newcommand{\childdocmain}[1]
{
  \childdocdisable\childdocmain{}
  \if?#1?\else
    \begingroup
      \def\childdoctmp{#1}
      \ifx\childdoctmp\childdocname
        \def\childdoctmp{}
      \else
        \def\childdoctmp
        {
          \childdoctrue
          \includeonly{\childdocname}
          \def\childdocjob{#1}
          \def\jobname{#1}
        }
      \fi
      \expandafter
    \endgroup
    \childdoctmp
  \fi
}
%    \end{macrocode}

% \macro{\childdocof}
% The command |\childdocof| redirects
% compilation to the main file |#1|.
%    \begin{macrocode}
\newcommand{\childdocof}[1]
{
  \childdocdisable
  \childdoctrue
  \includeonly{\childdocname}
  \def\jobname{#1}
  \def\childdocjob{#1}
  \input{#1}
}
%    \end{macrocode}

% \macro{\childdocby}
% The command |\childdocby| ....
%    \begin{macrocode}
\newcommand{\childdocby}[2][]
{
  \childdocdisable
  \childdoctrue
  \childdocmanualtrue
  \if?#1?\else
    \def\jobname{#2}
  \fi
  \def\childdocjob{#2}
  \input{#2}
  \endinput
}
%    \end{macrocode}

% \macro{\childdocforward}
% The command |\childdocforward| redirects
% compilation to the main file or
% (if the optional argument is given) a child file.
% Parameters are set as if the main file
% or a child file starting with |\childdocof| was compiled.
% Then compilation is handed over to the main file:
%    \begin{macrocode}
\newcommand{\childdocforward}[2][]
{
  \begingroup
    \if?#1?
      \def\childdoctmp
      {
        \def\childdocname{#2}
        \def\childdocjob{#2}
        \def\jobname{#2}
        \input{#2}
        \endinput
      }
    \else
      \def\childdoctmp
      {
        \childdocdisable
        \def\childdocname{#2}
        \childdoctrue
        \includeonly{#2}
        \def\childdocjob{#1}
        \def\jobname{#1}
        \input{#1}
        \endinput
      }
    \fi
    \expandafter
  \endgroup
  \childdoctmp
}
%    \end{macrocode}

% \macro{\childdocforwardprefix}
% The command |\childdocforwardprefix| redirects
% compilation to the main or a child file by means of a pattern.
% The prefix |#1| in the current filename is replaced by |#2|
% and the suffix of the current filename is kept
% (it is assumed that the filename does not contain the substring `|~~~|'
% which is used as a delimiter).
% Compilation is handed over to the new file by |\childdocforward|:
%    \begin{macrocode}
\newcommand{\childdocforwardprefix}[3][]
{
  \begingroup
    \def\childdocextract #2##1~~~{\def\childdoctmp{\childdocforward[#1]{#3##1}}}
    \expandafter\childdocextract\childdocname~~~
    \expandafter
  \endgroup
  \childdoctmp
}
%    \end{macrocode}

% \macro{\childdoc}
% The deprecated macro |\childdoc| is a legacy version of |\childdocmain|:
%    \begin{macrocode}
\newcommand{\childdoc}{\childdocmain}
%    \end{macrocode}

% \macro{\childdocredirect}
% The deprecated macro |\childdocredirect| is a legacy version
% of |\childdocforward| and |\childdocforwardprefix|:
%    \begin{macrocode}
\newcommand{\childdocredirect}[2][]
{
  \begingroup
    \if?#1?
      \def\childdoctmp{\childdocforward{#2}}
    \else
      \def\childdoctmp{\childdocforwardprefix{#1}{#2}}
    \fi
    \expandafter
  \endgroup
  \childdoctmp
}
%    \end{macrocode}

%\iffalse
%</package>
%\fi
%
\endinput
|\\
|\childdocforward[|\textit{main}|]{|\textit{dest}|}|\\
\end{tabular}
\end{center}
%
The argument \textit{dest} is the destination file
(without extension).
It should be the main file or one of the child files.
Note that further \textsf{childdoc} directives
such as |\childdocof| and |\childdocforward|
in the indicated file will be processed in this form.
The optional argument \textit{main}
passes on directly to the main file \textit{main}
while pretending to compile the child \textit{dest}.
This form behaves as if \textit{dest}
issues |\childdocof{|\textit{main}|}| right away,
and no further \textsf{childdoc} directives will be processed.

%%%%%%%%%%%%%%%%%%%%%%%%%%%%%%%%%%%%%%%%
\DescribeMacro{\...prefix}
In the alternative form |\childdocforwardprefix|,
%
\begin{center}
\begin{tabular}{l}
|% \iffalse
%
% childdoc.dtx Copyright (C) 2017-2018 Niklas Beisert
%
% This work may be distributed and/or modified under the
% conditions of the LaTeX Project Public License, either version 1.3
% of this license or (at your option) any later version.
% The latest version of this license is in
%   http://www.latex-project.org/lppl.txt
% and version 1.3 or later is part of all distributions of LaTeX
% version 2005/12/01 or later.
%
% This work has the LPPL maintenance status `maintained'.
%
% The Current Maintainer of this work is Niklas Beisert.
%
% This work consists of the files childdoc.dtx and childdoc.ins
% and the derived files childdoc.def and cdocsamp.tex with
% cdocsch1.tex, cdocsch2.tex, cdocsdrf.tex, cdocsfn1.tex, cdocsfn2.tex.
%
%<package>\ifdefined\childdocmain\endinput\fi
%<package>\ProvidesFile{childdoc.def}[2018/12/30 v2.0 child document driver]
%<samplemain>\ProvidesFile{cdocsamp.tex}[2018/12/30 v2.0 sample for childdoc]
%<*driver>
%\ProvidesFile{childdoc.drv}[2018/12/30 v2.0 childdoc reference manual file]
\PassOptionsToClass{10pt,a4paper}{article}
\documentclass{ltxdoc}

\usepackage[margin=35mm]{geometry}
\usepackage{hyperref}
\usepackage{hyperxmp}
\usepackage[usenames]{color}

\hypersetup{colorlinks=true}
\hypersetup{pdfstartview=FitH}
\hypersetup{pdfpagemode=UseNone}
\hypersetup{pdfsource={}}
\hypersetup{pdflang={en-UK}}
\hypersetup{pdfcopyright={Copyright 2017-2018 Niklas Beisert.
  This work may be distributed and/or modified under the
  conditions of the LaTeX Project Public License, either version 1.3
  of this license or (at your option) any later version.}}
\hypersetup{pdflicenseurl={http://www.latex-project.org/lppl.txt}}
\hypersetup{pdfcontactaddress={ETH Zurich, ITP, HIT K,
  Wolfgang-Pauli-Strasse 27}}
\hypersetup{pdfcontactpostcode={8093}}
\hypersetup{pdfcontactcity={Zurich}}
\hypersetup{pdfcontactcountry={Switzerland}}
\hypersetup{pdfcontactemail={nbeisert@itp.phys.ethz.ch}}
\hypersetup{pdfcontacturl={http://people.phys.ethz.ch/\xmptilde nbeisert/}}

\newcommand{\secref}[1]{\hyperref[#1]{section \ref*{#1}}}

\parskip1ex
\parindent0pt
\let\olditemize\itemize
\def\itemize{\olditemize\parskip0pt}

\begin{document}

\title{The \textsf{childdoc} Package}
\hypersetup{pdftitle={The childdoc Package}}
\author{Niklas Beisert\\[2ex]
  Institut f\"ur Theoretische Physik\\
  Eidgen\"ossische Technische Hochschule Z\"urich\\
  Wolfgang-Pauli-Strasse 27, 8093 Z\"urich, Switzerland\\[1ex]
  \href{mailto:nbeisert@itp.phys.ethz.ch}
  {\texttt{nbeisert@itp.phys.ethz.ch}}}
\hypersetup{pdfauthor={Niklas Beisert}}
\hypersetup{pdfsubject={Manual for the LaTeX2e Package childdoc}}
\date{30 December 2018, \textsf{v2.0}}
\maketitle

\begin{abstract}\noindent
\textsf{childdoc} is a \LaTeXe{} package
that enables the direct compilation
of document sections included by |\include|
to individual files.
\end{abstract}

\begingroup
\parskip0ex
\tableofcontents
\endgroup

%%%%%%%%%%%%%%%%%%%%%%%%%%%%%%%%%%%%%%%%%%%%%%%%%%%%%%%%%%%%%%%%%%%%%%%%%%%%%%%%
%%%%%%%%%%%%%%%%%%%%%%%%%%%%%%%%%%%%%%%%%%%%%%%%%%%%%%%%%%%%%%%%%%%%%%%%%%%%%%%%
\section{Introduction}

\LaTeX{} provides a mechanism to structure a large document (such as a book)
into a main file and several child files (containing the chapters)
using the |\include| command.
This mechanism is beneficial for documents
which span hundreds of pages in order to
make the source file(s) more manageable.
Moreover, compilation can be restricted to
selected child files by means of the |\includeonly| command.
The latter feature can be used to reduce the compilation time while editing
(this was significantly more useful in the earlier days of \LaTeX{})
or to generate a smaller document which is easier to navigate.
Another application of |\includeonly| is to generate
documents consisting of selected parts of the complete document.

However, there are a few drawbacks of the plain |\include| mechanism:
\begin{itemize}
\item
The child files cannot be compiled on their own,
they can only be compiled via the main file.
A naive editing environment
(such as a text editor with an option
to have the current file processed by \LaTeX)
may require one to switch to the main file before compiling;
attempting to compile the child file produces errors.
\item
The main file must be modified (each time)
to adjust the |\includeonly| command
to the present needs. This easily leaves the main file in a messy state.
\item
The generated document will always carry the filename
of the main document. This is inconvenient if
several child files are to be compiled and
to be kept for distribution.
\end{itemize}

The present package provides a simple interface
to make child files individually compilable by \LaTeX{}.
Compiling a child file then has the same effect as compiling
the main file with an |\includeonly| command
to select the appropriate child.
Moreover the generated document will carry the name of the child
rather than the main file.
This resolves all three above issues.

This feature is meant to make the editing of books,
thesis documents and lecture notes somewhat more convenient.
However, the package can also be used efficiently for
composing a series of documents (such as exercise sheets)
which are typically distributed individually.
It then assists the author in generating the individual documents
(potentially in different versions)
as well as a document containing the collected series.
Another application is in developing style files
or other kinds of included material
where compilation of the style file could redirect
to a sample or test file.

%%%%%%%%%%%%%%%%%%%%%%%%%%%%%%%%%%%%%%%%%%%%%%%%%%%%%%%%%%%%%%%%%%%%%%%%%%%%%%%%
%%%%%%%%%%%%%%%%%%%%%%%%%%%%%%%%%%%%%%%%%%%%%%%%%%%%%%%%%%%%%%%%%%%%%%%%%%%%%%%%
\section{Usage}

First of all, the package \textsf{childdoc} is \emph{not} a standard
\LaTeXe{} |.sty| style file! Therefore it needs to be invoked in
a non-standard way.

%%%%%%%%%%%%%%%%%%%%%%%%%%%%%%%%%%%%%%%%%%%%%%%%%%%%%%%%%%%%%%%%%%%%%%%%%%%%%%%%
\subsection{Included Files}
\label{sec:include}

%%%%%%%%%%%%%%%%%%%%%%%%%%%%%%%%%%%%%%%%
\DescribeMacro{\childdocmain}
To use the package, add the commands
\begin{center}
\begin{tabular}{l}
|\input{childdoc.def}|\\
|\childdocmain{}|\\
\end{tabular}
\end{center}
at the very top of the main \LaTeX{} file,
in particular \emph{before} the |\documentclass| statement!
The argument of |\childdocmain| should be left empty
(but it must be present).

%%%%%%%%%%%%%%%%%%%%%%%%%%%%%%%%%%%%%%%%
\DescribeMacro{\childdocof}
Furthermore, add the commands
\begin{center}
\begin{tabular}{l}
|\input{childdoc.def}|\\
|\childdocof{|\textit{main}|}|\\
\end{tabular}
\end{center}
at the top of every child file \textit{child}
which is included by |\include{|\textit{child}|}|
from within the main file
(or at least for those files to be compiled individually).
The argument \textit{main} must be the filename of the main file.

There are a couple of
considerations in setting up the main and child documents:

%%%%%%%%%%%%%%%%%%%%%%%%%%%%%%%%%%%%%%%%
\paragraph{Restrictions.}

Please note the following restrictions:
\begin{itemize}
\item
|\childdocmain| must be called with one argument \textit{main}
to ensure compatibility with earlier version of the package.
It must either be empty (|\childdocmain{}|)
or precisely match the filename of the main file in which it is specified.
See \secref{sec:detection} for further information.
\item
The filename \textit{main} must be specified without the |.tex| extension.
\item
The filename \textit{main} is case sensitive
(even in case-insensitive file systems)
due to internal string comparison.
\item
The argument \textit{main} should be fully expanded, it cannot be a macro.
\item
Subdirectories and special characters should be avoided in filenames.
\item
The command |\childdocmain{|\textit{main}|}| must be followed by a whitespace.
It should not be followed immediately by another command
or by a comment mark `|%|'.
This is because the \TeX{} parser reads the token immediately following
the argument of |\childdocmain| and puts it
at the beginning of every child section;
however, a white\-space is ignored.
\end{itemize}

%%%%%%%%%%%%%%%%%%%%%%%%%%%%%%%%%%%%%%%%
\paragraph{Content of Main File.}

It is advisable to place all content in the child files included by |\include|.
Any output contained in the main file will appear in all child documents
unless suppressed manually;
it cannot be suppressed automatically by the |\includeonly| directive
and thus should normally be avoided.
A method to include some content in the main file
by means of conditional processing is described in \secref{sec:conditional}.

%%%%%%%%%%%%%%%%%%%%%%%%%%%%%%%%%%%%%%%%
\paragraph{Page Numbering.}

When only a part of the document is compiled,
the appropriate numbering of pages
(as well as other status parameters)
is determined from the |.aux| files.
The latter contain information from previous passes.
However this information needs to propagate through
all intermediate child documents.
Therefore the page numbering in child documents may well
be inconsistent until the complete document is compiled at least once.

A useful (if unconventional) way to always ensure a consistent
page numbering is to restart the numbering in each child document
and denote the pages by `\textit{child}|.|\textit{page}'
where \textit{child} represents the chapter/section number of the child file.
This can be achieved by the command
|\numberwithin{page}{|\textit{child}|}|
of the \textsf{amsmath} package
where \textit{child} can be |chapter| or |section|
depending on the chosen structuring.
Alternatively, one can modify the macro |\thepage| appropriately
and reset the counter |page| at the start of each child file.

%%%%%%%%%%%%%%%%%%%%%%%%%%%%%%%%%%%%%%%%%%%%%%%%%%%%%%%%%%%%%%%%%%%%%%%%%%%%%%%%
\subsection{Conditional Processing}
\label{sec:conditional}

The package provides a mechanism to compile different versions
of a document. To customise the versions further some conditional processing
can come in handy to distinguish which version is being compiled.
The package provides two macros to describe the compilation context:

%%%%%%%%%%%%%%%%%%%%%%%%%%%%%%%%%%%%%%%%
\DescribeMacro{\ifchilddoc}
The conditional |\ifchilddoc| distinguishes between the compilation of
child documents and the main document:
%
\begin{center}
|\ifchilddoc |\textit{child-code}| |[|\||else |\textit{main-code}]| \||fi|
\end{center}

%%%%%%%%%%%%%%%%%%%%%%%%%%%%%%%%%%%%%%%%
\DescribeMacro{\childdocname}
\DescribeMacro{\childdocjob}
The macro |\childdocname| contains the filename (without extension)
of the main or child file being processed.
Note that |\childdocjob| will always contain the name of the main file.

%%%%%%%%%%%%%%%%%%%%%%%%%%%%%%%%%%%%%%%%
\paragraph{Title Page.}

Conditional processing can be used to include a title or banner page
in the main document when proper precautions are taken.
Importantly, the code in the main file should ensure that the page counter
(as well as other status parameters which are stored in the |.aux| files)
takes the same value after the conditional processing.
Otherwise the page numbers may take divergent values
depending on which part is compiled.

For example, a title page could be declared by:
%
\begin{center}
\begin{tabular}{l}
|\ifchilddoc\||else|\\
|\addtocounter{page}{-1}|\\
\textit{code for title page}\\
|\newpage|\\
|\||fi|
\end{tabular}
\end{center}
%
A banner page for the child documents can be generated by:
%
\begin{center}
\begin{tabular}{l}
|\ifchilddoc|\\
|\addtocounter{page}{-1}|\\
\textit{code for banner page}\\
|\newpage|\\
|\||fi|
\end{tabular}
\end{center}
%
Here one could write a message such as:
\begin{center}
|This is the part \childdocname{} of \childdocjob{}.|
\end{center}

%%%%%%%%%%%%%%%%%%%%%%%%%%%%%%%%%%%%%%%%%%%%%%%%%%%%%%%%%%%%%%%%%%%%%%%%%%%%%%%%
\subsection{Flags}
\label{sec:flags}

The package makes it easy to generate different versions
of the main or child documents.
To this end compilation flags can be defined
and assigned different default values.
They will be particularly useful in conjunction
with the forwarding mechanism described in \secref{sec:forward}.

For example, it may be useful to have a flag |\version|
which can be set to |draft| or |final|.
The document source will contain some conditional code
depending on the value of |\version|.
Suppose further, the flag should default to |final| for the main file
and to |draft| for child files
which is a natural assignment for editing the document.
This is achieved by placing the following code
in the preamble of the main document
(below the |\childdocmain| directive):
%
\begin{center}
\begin{tabular}{l}
|\ifchilddoc|\\
|\providecommand{\version}{draft}|\\
|\||else|\\
|\providecommand{\version}{final}|\\
|\||fi|
\end{tabular}
\end{center}
%
The definition by |\providecommand| makes sure
that previous definitions are not overwritten.
Further statements |\providecommand{\version}{...}|
can thus be added before the above code to override it.

For the main file, one might add a line
(between |\childdocmain| and the above block)
%
\begin{center}
|%\ifchilddoc\||else\providecommand{\version}{draft}\||fi|
\end{center}
%
which can be uncommented to produce a draft version.
Likewise one can add a line to the very top of a child file
(above the |\childdocof{|\textit{main}|}| directive)
%
\begin{center}
|%\providecommand{\version}{final}|
\end{center}
%
which can be uncommented to produce the final version of this child document.

%%%%%%%%%%%%%%%%%%%%%%%%%%%%%%%%%%%%%%%%%%%%%%%%%%%%%%%%%%%%%%%%%%%%%%%%%%%%%%%%
\subsection{Forwarding}
\label{sec:forward}

Different versions of the main or child documents
using compilation flags as described in \secref{sec:flags}
can be (permanently) stored in different files
for convenient compilation, viewing and distribution.
To this end, the package defines a command
to pass on compilation to a different file:

%%%%%%%%%%%%%%%%%%%%%%%%%%%%%%%%%%%%%%%%
\DescribeMacro{\childdocforward}
The command |\childdocforward| redirects processing to
another source file:
%
\begin{center}
\begin{tabular}{l}
|\input{childdoc.def}|\\
|\childdocforward[|\textit{main}|]{|\textit{dest}|}|\\
\end{tabular}
\end{center}
%
The argument \textit{dest} is the destination file
(without extension).
It should be the main file or one of the child files.
Note that further \textsf{childdoc} directives
such as |\childdocof| and |\childdocforward|
in the indicated file will be processed in this form.
The optional argument \textit{main}
passes on directly to the main file \textit{main}
while pretending to compile the child \textit{dest}.
This form behaves as if \textit{dest}
issues |\childdocof{|\textit{main}|}| right away,
and no further \textsf{childdoc} directives will be processed.

%%%%%%%%%%%%%%%%%%%%%%%%%%%%%%%%%%%%%%%%
\DescribeMacro{\...prefix}
In the alternative form |\childdocforwardprefix|,
%
\begin{center}
\begin{tabular}{l}
|\input{childdoc.def}|\\
|\childdocforwardprefix[|\textit{main}|]{|\textit{prefix}|}{|\textit{dest}|}|
\end{tabular}
\end{center}
%
the destination file is determined by a pattern
depending on the current file:
To make this work, the current file must be called
`{\textit{prefix}\hspace{0.2em}\textit{suffix}}'
with \textit{prefix} matching precisely the argument.
Processing is then passed on to the file
`{\textit{dest}\hspace{0.2em}\textit{suffix}}'.
Surely, the same effect is achieved by
directly specifying the
argument `{\textit{dest}\hspace{0.2em}\textit{suffix}}'
in the first form.
However, that requires to set up a different file
for each child. With the alternative form of the command
all these files can have exactly the same content
which simplifies setting them up and maintaining them.

For example, the following file |draft.tex|
with a compilation flag |\version| as described in \secref{sec:flags}
compiles the main document as a draft:
%
\begin{center}
\begin{tabular}{l}
|\def\version{draft}|\\
|\input{childdoc.def}|\\
|\childdocforward{|\textit{main}|}|
\end{tabular}
\end{center}
%
Likewise, the following files |final|\textit{nn}|.tex|
compile the final version of the child document
|child|\textit{nn}|.tex|:
%
\begin{center}
\begin{tabular}{l}
|\def\version{final}|\\
|\input{childdoc.def}|\\
|\childdocforwardprefix{final}{child}|
\end{tabular}
\end{center}
%

Note that when several versions of a main file and/or of each child file
are to be generated, it may be convenient to set up a |Makefile| or
shell script to automatise the process.

%%%%%%%%%%%%%%%%%%%%%%%%%%%%%%%%%%%%%%%%%%%%%%%%%%%%%%%%%%%%%%%%%%%%%%%%%%%%%%%%
\subsection{Command Line Processing}
\label{sec:commandline}

The effect of redirection files can also be achieved by invoking
the \LaTeX{} compiler with a more elaborate command line.
Most conveniently this should be done as part
of a shell script or a |Makefile|.

When using \textsf{childdoc} in the main file, the following
command lines effectively perform a redirection
(note that depending on the shell being used,
backslashes may have to be doubled: `|\|' $\to$ `|\\|'):
%
\begin{center}
|... -jobname "|\textit{target}|" |\\|"|[\textit{flags}]%
|\input{childdoc.def}\childdocforward[|\textit{main}|]{|\textit{dest}|}"|
\end{center}
%
Here \textit{target} is the name of the output file,
\textit{main} is the name of the main file
and \textit{dest} is the name of the main or child file to be processed
(all filenames without extensions).
The optional argument \textit{main} can be omitted
if \textit{main} matches \textit{dest}.
Optionally, compilation \textit{flags} can be defined via |\def| commands.
This command line makes the \TeX{} engine believe
it is compiling the file \textit{target}
whose content is specified as the latter parameter.
The provided code then forwards the processing to
\textit{main} or \textit{dest} as described in \secref{sec:forward}.

%%%%%%%%%%%%%%%%%%%%%%%%%%%%%%%%%%%%%%%%%%%%%%%%%%%%%%%%%%%%%%%%%%%%%%%%%%%%%%%%
\subsection{Include by Input}
\label{sec:input}

Including child documents by |\include| has some restrictions by design.
Most notably, the content of a child document always occupies
its own set of pages; pages cannot be shared between child documents.
Usually, this behaviour makes perfect sense
because each child document contain an essential part of the document.
However, in some situations it may be desirable to compose
a document from a collection of parts
without having mandatory page breaks between then.
For this case, the package
provides a mechanism to include parts
by |\input| which can also be processed individually.
However, by construction this mechanism
requires manual handling of the content to be output.

%%%%%%%%%%%%%%%%%%%%%%%%%%%%%%%%%%%%%%%%
\DescribeMacro{\ifchilddocmanual}
The main file should be prepared as usual, see \secref{sec:include}.
However, the document body must make a distinction
between processing of an individual part and of the main document, e.g.:
%
\begin{center}
\begin{tabular}{l}
|\ifchilddocmanual|\\
|\input{\childdocname}|\\
|\||else|\\
\textit{document body with }|\input{|\textit{part}|}|\\
|\||fi|
\end{tabular}
\end{center}
%
The conditional |\ifchilddocmanual| is true whenever
a part to be included by |\input| is being compiled,
and the name of the part is stored in |\childdocname|.

%%%%%%%%%%%%%%%%%%%%%%%%%%%%%%%%%%%%%%%%
\DescribeMacro{\childdocby}
Each part to be included by |\input| should start with:
%
\begin{center}
\begin{tabular}{l}
|\input{childdoc.def}|\\
|\childdocby{|\textit{main}|}|\\
\end{tabular}
\end{center}
%
The directive |\childdocby| is similar to |\childdocof|
described in \secref{sec:include},
but the subsequent selection of content must be done manually.
To that end, both |\ifchilddoc| and |\ifchilddocmanual|
will be true upon processing of a part,
and the name of the part is stored in |\childdocname|.
Note that |\jobname| will be set to the filename of the current part
so that each part receives an individual |.aux| file
that does not interfere with the |.aux| file(s) of the main document.
This behaviour can be altered by the alternative form
|\childdocby[*]{|\textit{main}|}| (with a non-empty optional argument)
which uses the |.aux| file of the main document
by setting |\jobname| to \textit{main}.

%%%%%%%%%%%%%%%%%%%%%%%%%%%%%%%%%%%%%%%%%%%%%%%%%%%%%%%%%%%%%%%%%%%%%%%%%%%%%%%%
\subsection{Driver Development}
\label{sec:driver}

The \textsf{childdoc} mechanism can also be use for the development
of definition files such as \LaTeX{} styles or classes.
This case differs from the above setup with multiple parts
included by |\include| in that no |\includeonly| should be invoked.
This can be achieved by starting the include file
(before |\ProvidesPackage|) with:
%
\begin{center}
\begin{tabular}{l}
|\input{childdoc.def}|\\
|\childdocforward{|\textit{main}|}|\\
\end{tabular}
\end{center}
%
or alternatively with:
%
\begin{center}
\begin{tabular}{l}
|\input{childdoc.def}|\\
|\childdocby{|\textit{main}|}|\\
\end{tabular}
\end{center}
%
Both forms have slightly different effects as described above.
The main file is prepared as usual, see \secref{sec:include}.

%%%%%%%%%%%%%%%%%%%%%%%%%%%%%%%%%%%%%%%%%%%%%%%%%%%%%%%%%%%%%%%%%%%%%%%%%%%%%%%%
\subsection{Legacy Detection}
\label{sec:detection}

The directive |\childdocmain| in the main file can detect
whether the complete document or merely a child is to be compiled
even without using the directive |\childdocof|.
This method is deprecated because it is less robust
and there is no compelling reason to use it;
it is merely provided for backward compatibility
and it may be removed in future versions.

If the detection mechanism is to be used,
it is mandatory to correctly specify
the filename of the main file as the argument of |\childdocmain|:
%
\begin{center}
\begin{tabular}{l}
|\input{childdoc.def}|\\
|\childdocmain{|\textit{main}|}|\\
\end{tabular}
\end{center}
%
If |\jobname| does not match the argument \textit{main} of |\childdocmain|,
it is assumed that |\jobname| points to the child file to be compiled.
When using |\childdocmain| with the main file specified as argument,
it suffices to start a child file
with just |\input{|\textit{main}|}|
without loading of the package and using |\childdocof|.
If instead all processing is done
with the appropriate \textsf{childdoc} directives,
the argument of \textit{main} of |\childdocmain| can be empty.

An alternative version of the command line processing described
in \secref{sec:commandline} using the detection mechanism reads:
%
\begin{center}
|... -jobname "|\textit{target}|" "|[\textit{flags}]%
[|\def\jobname{|\textit{dest}|}|]|\input{|\textit{main}|}"|
\end{center}

%%%%%%%%%%%%%%%%%%%%%%%%%%%%%%%%%%%%%%%%%%%%%%%%%%%%%%%%%%%%%%%%%%%%%%%%%%%%%%%%
\subsection{Manual Code}
\label{sec:manual}

In case one cannot be certain whether the definitions file |childdoc.def|
is installed on the target \TeX{} distribution
and one prefers not to ship it,
it is conceivable to paste a few relevant commands into the sources.

To that end, drop all statements |\input{childdoc.def}|
and perform the replacements as outlined below.
Instead of |\childdocmain{|\textit{main}|}| add the following code
to the top of the main file:
%
\begin{center}
\begin{tabular}{l}
|\||ifdefined\childdocname\endinput\||fi\newif\ifchilddoc|\\
|\edef\childdocname{\scantokens\expandafter{\jobname\noexpand}}|\\
|\def\childdocmain{|\textit{main}|}\||ifx\childdocmain\childdocname\||else|\\
|\childdoctrue\includeonly{\childdocname}\let\jobname\childdocmain\||fi|\\
\end{tabular}
\end{center}
%
Instead of |\childdocof{|\textit{main}|}| just include the main file
at the top of each child file:
%
\begin{center}
|\input{|\textit{main}|}|
\end{center}
%
A simple redirection |\childdocforward{|\textit{dest}|}| is achieved by:
%
\begin{center}
|\def\jobname{|\textit{dest}|}\input{\jobname}|
\end{center}
%
The redirection with prefix
|\childdocforwardprefix[|\textit{prefix}|]{|\textit{dest}|}|
is accomplished by:
%
\begin{center}
\begin{tabular}{l}
|{\edef\jobname{\scantokens\expandafter{\jobname\noexpand}}|\\
|\def\redirectjob |\textit{prefix}|#1~~~{\gdef\jobname{|\textit{dest}|#1}}|\\
|\expandafter\redirectjob\jobname~~~}\input{\jobname}|
\end{tabular}
\end{center}

In an alternative approach,
child documents can be compiled by a specific command line
without additional code or specific definitions:
%
\begin{center}
|... -jobname "|\textit{target}|" "|[\textit{flags}]%
|\includeonly{|\textit{dest}|}\input{|\textit{main}|}"|
\end{center}
%

%%%%%%%%%%%%%%%%%%%%%%%%%%%%%%%%%%%%%%%%%%%%%%%%%%%%%%%%%%%%%%%%%%%%%%%%%%%%%%%%
%%%%%%%%%%%%%%%%%%%%%%%%%%%%%%%%%%%%%%%%%%%%%%%%%%%%%%%%%%%%%%%%%%%%%%%%%%%%%%%%
\section{Information}

%%%%%%%%%%%%%%%%%%%%%%%%%%%%%%%%%%%%%%%%%%%%%%%%%%%%%%%%%%%%%%%%%%%%%%%%%%%%%%%%
\subsection{Copyright}

Copyright \copyright{} 2017--2018 Niklas Beisert

This work may be distributed and/or modified under the
conditions of the \LaTeX{} Project Public License, either version 1.3
of this license or (at your option) any later version.
The latest version of this license is in
  \url{http://www.latex-project.org/lppl.txt}
and version 1.3 or later is part of all distributions of \LaTeX{}
version 2005/12/01 or later.

This work has the LPPL maintenance status `maintained'.

The Current Maintainer of this work is Niklas Beisert.

This work consists of the files |README.txt|, |childdoc.ins| and |childdoc.dtx|
as well as the derived files |childdoc.def|, |cdocsamp.tex|
with |cdocsch1.tex|, |cdocsch2.tex|, |cdocspt3.tex|, |cdocspt4.tex|,
|cdocsdrf.tex|, |cdocsfn1.tex|, |cdocsfn2.tex|
as well as |childdoc.pdf|.

%%%%%%%%%%%%%%%%%%%%%%%%%%%%%%%%%%%%%%%%%%%%%%%%%%%%%%%%%%%%%%%%%%%%%%%%%%%%%%%%
\subsection{Files and Installation}

The package consists of the files:
%
\begin{center}
\begin{tabular}{ll}
    |README.txt|   & readme file \\
    |childdoc.ins| & installation file \\
    |childdoc.dtx| & source file \\
    |childdoc.def| & definition file \\
    |cdocsamp.tex| & sample main file \\
    |cdocsch1.tex| & sample include file \\
    |cdocsch2.tex| & sample include file \\
    |cdocspt3.tex| & sample part file \\
    |cdocspt4.tex| & sample part file \\
    |cdocsdrf.tex| & sample redirection file \\
    |cdocsfn1.tex| & sample redirection file \\
    |cdocsfn2.tex| & sample redirection file \\
    |childdoc.pdf| & manual
\end{tabular}
\end{center}
%
The distribution consists of the files
|README.txt|, |childdoc.ins| and |childdoc.dtx|.
%
\begin{itemize}
\item
Run (pdf)\LaTeX{} on |childdoc.dtx|
to compile the manual |childdoc.pdf| (this file).
\item
Run \LaTeX{} on |childdoc.ins| to create the definitions file |childdoc.def|
and the sample |cdocsamp.tex| with include files
|cdocsch1.tex|, |cdocsch2.tex|, |cdocspt3.tex|, |cdocspt4.tex|,
|cdocsdrf.tex|, |cdocsfn1.tex|, |cdocsfn2.tex|.
Then copy the file |childdoc.def| to an appropriate directory of your \LaTeX{}
distribution, e.g.\ \textit{texmf-root}|/tex/latex/childdoc|.
\end{itemize}

%%%%%%%%%%%%%%%%%%%%%%%%%%%%%%%%%%%%%%%%%%%%%%%%%%%%%%%%%%%%%%%%%%%%%%%%%%%%%%%%
\subsection{Related CTAN Packages}

There are several other packages which offer a similar functionality:
%
\begin{itemize}
\item
The packages
\href{http://ctan.org/pkg/docmute}{\textsf{docmute}},
\href{http://ctan.org/pkg/includex}{\textsf{includex}} and
\href{http://ctan.org/pkg/standalone}{\textsf{standalone}}
provide commands to include only the document body of
a child file thus allowing both files to be compiled individually.
\item
The packages \href{http://ctan.org/pkg/subdocs}{\textsf{subdocs}}
and \href{http://ctan.org/pkg/subfiles}{\textsf{subfiles}}
provide structures in which the main and child documents can be
encapsulated and allowing them to be compiled individually.
The inclusion mechanism is different from the conventional |\include|.
\item
The package \href{http://ctan.org/pkg/combine}{\textsf{combine}}
is an elaborate solution to combine several documents into one.
\end{itemize}
%
See also the CTAN topic \href{http://ctan.org/topic/subdocs}{\textsf{subdocs}}
for further related packages.
The present package differs from the above solutions in that
a document structure constructed with the conventional |\include| mechanism
just needs two extra commands at the top of every file
such that all constituent files can be compiled individually.

%%%%%%%%%%%%%%%%%%%%%%%%%%%%%%%%%%%%%%%%%%%%%%%%%%%%%%%%%%%%%%%%%%%%%%%%%%%%%%%%
%\subsection{Feature Suggestions}
%
%The following is a list of features which may be useful for future
%versions of this package:
%%
%\begin{itemize}
%\item
%\ldots
%\end{itemize}

%%%%%%%%%%%%%%%%%%%%%%%%%%%%%%%%%%%%%%%%%%%%%%%%%%%%%%%%%%%%%%%%%%%%%%%%%%%%%%%%
\subsection{Revision History}

%%%%%%%%%%%%%%%%%%%%%%%%%%%%%%%%%%%%%%%%
\paragraph{v2.0:} 2018/12/30

\begin{itemize}
\item
immediate forward processing
\item
added |\childdocby| mechanism
\item
manual restructured
\end{itemize}

%%%%%%%%%%%%%%%%%%%%%%%%%%%%%%%%%%%%%%%%
\paragraph{v1.6:} 2018/01/17

\begin{itemize}
\item
application for development of include files
\item
corrections to manual
\end{itemize}

%%%%%%%%%%%%%%%%%%%%%%%%%%%%%%%%%%%%%%%%
\paragraph{v1.5:} 2017/05/21

\begin{itemize}
\item
more complete structuring introduced
\item
|\childdocof| introduced
\item
|\childdoc| renamed to |\childdocmain|
\item
|\childredirect| renamed to |\childdocforward| and |\childdocforwardprefix|
and functionality expanded
\end{itemize}

%%%%%%%%%%%%%%%%%%%%%%%%%%%%%%%%%%%%%%%%
\paragraph{v1.0:} 2017/04/27

\begin{itemize}
\item
manual and install package
\item
first version published on CTAN
\end{itemize}

%%%%%%%%%%%%%%%%%%%%%%%%%%%%%%%%%%%%%%%%
\paragraph{v0.6:} 2017/04/26

\begin{itemize}
\item
redirection mechanism added
\end{itemize}

%%%%%%%%%%%%%%%%%%%%%%%%%%%%%%%%%%%%%%%%
\paragraph{v0.5:} 2017/04/26

\begin{itemize}
\item
functionality in definition file
\end{itemize}


%%%%%%%%%%%%%%%%%%%%%%%%%%%%%%%%%%%%%%%%%%%%%%%%%%%%%%%%%%%%%%%%%%%%%%%%%%%%%%%%
%%%%%%%%%%%%%%%%%%%%%%%%%%%%%%%%%%%%%%%%%%%%%%%%%%%%%%%%%%%%%%%%%%%%%%%%%%%%%%%%
%%%%%%%%%%%%%%%%%%%%%%%%%%%%%%%%%%%%%%%%%%%%%%%%%%%%%%%%%%%%%%%%%%%%%%%%%%%%%%%%
\appendix

\settowidth\MacroIndent{\rmfamily\scriptsize 000\ }

 \DocInput{childdoc.dtx}

\end{document}
%</driver>
% \fi
%
% %%%%%%%%%%%%%%%%%%%%%%%%%%%%%%%%%%%%%%%%%%%%%%%%%%%%%%%%%%%%%%%%%%%%%%%%%%%%%%
% %%%%%%%%%%%%%%%%%%%%%%%%%%%%%%%%%%%%%%%%%%%%%%%%%%%%%%%%%%%%%%%%%%%%%%%%%%%%%%
% \section{Sample}
%\iffalse
%<*samplemain>
%\fi
%
% The following presents a sample document
% with two chapters, two parts, a title page,
% a compile flag as well as three forwarding files to set the flag.
% It consists of eight |.tex| files:
% \begin{center}
% \begin{tabular}{ll}
% |cdocsamp.tex|&main file\\
% |cdocsch1.tex|&include file for chapter 1\\
% |cdocsch2.tex|&include file for chapter 2\\
% |cdocspt3.tex|&include file for part 3\\
% |cdocspt4.tex|&include file for part 4\\
% |cdocsdrf.tex|&forwarding file for main file in draft mode\\
% |cdocsfi1.tex|&forwarding file for final version of chapter 1\\
% |cdocsfi2.tex|&forwarding file for final version of chapter 2\\
% \end{tabular}
% \end{center}
% Each of the eight files can be compiled directly by the \LaTeX{} compiler.
%
% %%%%%%%%%%%%%%%%%%%%%%%%%%%%%%%%%%%%%%
% \paragraph{Main File.}
%
% The main file is called |cdocsamp.tex|.
%
% Load the \textsf{childdoc} definitions and
% declare the filename for the main document:
%    \begin{macrocode}
\input{childdoc.def}
\childdocmain{}
%    \end{macrocode}

% Optional override for |\version| flag:
%    \begin{macrocode}
%%\ifchilddoc\else\providecommand{\version}{draft}\fi
%    \end{macrocode}

% Define the default values for the |\version| flag
% (|final| for the main file and |draft| for childs):
%    \begin{macrocode}
\ifchilddoc
\providecommand{\version}{draft}
\else
\providecommand{\version}{final}
\fi
%    \end{macrocode}

% Load the standard document class:
%    \begin{macrocode}
\documentclass[12pt]{article}
%    \end{macrocode}

% Start the document body:
%    \begin{macrocode}
\begin{document}
%    \end{macrocode}

% Declare a title page.
% Print title, part of document being processed and version flag:
%    \begin{macrocode}
\addtocounter{page}{-1}
\begin{center}
{\LARGE\bfseries{}childdoc example\par}
\vspace{1cm}
\ifchilddoc
\ifchilddocmanual part\else chapter\fi:
`\childdocname' of `\childdocjob'\par
\else
main document: `\childdocjob'\par
\fi
version: \version\par
\end{center}
\newpage
%    \end{macrocode}

% Manually include selected file,
% otherwise process as usual:
%    \begin{macrocode}
\ifchilddocmanual
\section*{part `\childdocname'}
\input{\childdocname}
\else
%    \end{macrocode}

% Include the two chapters:
%    \begin{macrocode}
\include{cdocsch1}
\include{cdocsch2}
%    \end{macrocode}

% Include the two parts unless only chapters should be displayed:
%    \begin{macrocode}
\ifchilddoc\else
\section{part three}
\input{cdocspt3}
\section{part four}
\input{cdocspt4}
\fi
%    \end{macrocode}

% Process as usual until here:
%    \begin{macrocode}
\fi
%    \end{macrocode}

% End of document body:
%    \begin{macrocode}
\end{document}
%    \end{macrocode}
%\iffalse
%</samplemain>
%\fi
%
% %%%%%%%%%%%%%%%%%%%%%%%%%%%%%%%%%%%%%%
% \paragraph{Chapter Include Files.}
%
% The include files are called |cdocsch1.tex| and |cdocsch2.tex|.
%
%\iffalse
%<*samplechap1|samplechap2>
%\fi

% Optional override for |\version| flag:
%    \begin{macrocode}
%%\providecommand{\version}{final}
%    \end{macrocode}

% Include the main document:
%    \begin{macrocode}
\input{childdoc.def}
\childdocof{cdocsamp}
%    \end{macrocode}

%\iffalse
%</samplechap1|samplechap2>
%\fi
%
%\iffalse
%<*samplechap1>
%\fi
% Some text for chapter 1:
%    \begin{macrocode}
\section{one}
some text in chapter one
%    \end{macrocode}

%\iffalse
%</samplechap1>
%\fi
% Some text for chapter 2:
%\iffalse
%<*samplechap2>
%\fi
%    \begin{macrocode}
\section{two}
more text in chapter two
%    \end{macrocode}

%\iffalse
%</samplechap2>
%\fi
%
% %%%%%%%%%%%%%%%%%%%%%%%%%%%%%%%%%%%%%%
% \paragraph{Part Include Files.}
%
% The include files are called |cdocspt3.tex| and |cdocspt4.tex|.
%
%\iffalse
%<*samplepart3|samplepart4>
%\fi

% Optional override for |\version| flag:
%    \begin{macrocode}
%%\providecommand{\version}{final}
%    \end{macrocode}

% Include the main document:
%    \begin{macrocode}
\input{childdoc.def}
\childdocby{cdocsamp}
%    \end{macrocode}

%\iffalse
%</samplepart3|samplepart4>
%\fi
%
%\iffalse
%<*samplepart3>
%\fi
% Some text for part 3:
%    \begin{macrocode}
some text in part three
%    \end{macrocode}

%\iffalse
%</samplepart3>
%\fi
% Some text for part 4:
%\iffalse
%<*samplepart4>
%\fi
%    \begin{macrocode}
more text in part four
%    \end{macrocode}

%\iffalse
%</samplepart4>
%\fi
%
% %%%%%%%%%%%%%%%%%%%%%%%%%%%%%%%%%%%%%%
% \paragraph{Forwarding for a Complete Draft.}
%
% The following forwarding file |cdocsdrf.tex|
% compiles the main document in draft mode:
%\iffalse
%<*sampledraft>
%\fi
%    \begin{macrocode}
\def\version{draft}
\input{childdoc.def}
\childdocforward{cdocsamp}
%    \end{macrocode}

%\iffalse
%</sampledraft>
%\fi
%
% %%%%%%%%%%%%%%%%%%%%%%%%%%%%%%%%%%%%%%
% \paragraph{Forwarding for Final Version of the Chapters.}
%
% The following forwarding files |cdocsfn1.tex| and |cdocsfn2.tex|
% (with identical content)
% compile the final versions of the child documents
% |cdocsch1.tex| and |cdocsch2.tex|, respectively:
%\iffalse
%<*samplefinal>
%\fi
%    \begin{macrocode}
\def\version{final}
\input{childdoc.def}
\childdocforwardprefix[cdocsamp]{cdocsfn}{cdocsch}
%    \end{macrocode}

%\iffalse
%</samplefinal>
%\fi
%
% %%%%%%%%%%%%%%%%%%%%%%%%%%%%%%%%%%%%%%
% \paragraph{Command Line Processing.}
%
% The following three command lines generate the output files
% |cdocscld|, |cdocscl1| and |cdocscl2|
% which should be identical to
% |cdocsdrf|, |cdocsch1| and |cdocsfn2|, respectively:
% \begin{center}
% \begin{tabular}{l}
% |latex -jobname cdocscld \|\\
% |  "\def\version{draft}\input{childdoc.def}\childdocforward{cdocsamp}"|\\
% |latex -jobname cdocscl1 \|\\
% |  "\input{childdoc.def}\childdocforward[cdocsamp]{cdocsch1}"|\\
% |latex -jobname cdocscl2 \|\\
% |  "\def\version{final}\input{childdoc.def}\childdocforward{cdocsch2}"|
% \end{tabular}
% \end{center}
% Note that the trailing backslash on each first line
% merely continues the input to the second line
% (for convenient cut ant paste).
% Furthermore, the command |latex| can be replaced by any
% of its alternative versions such as |pdflatex|.
%
% %%%%%%%%%%%%%%%%%%%%%%%%%%%%%%%%%%%%%%%%%%%%%%%%%%%%%%%%%%%%%%%%%%%%%%%%%%%%%%
% %%%%%%%%%%%%%%%%%%%%%%%%%%%%%%%%%%%%%%%%%%%%%%%%%%%%%%%%%%%%%%%%%%%%%%%%%%%%%%
% \section{Implementation}
%\iffalse
%<*package>
%\fi
%
% This section describes the definitions file |childdoc.def|.

% The definitions cannot be loaded using |\usepackage| or |\RequirePackage|
% which has a mechanism to prevent loading a style file more than once.
% When loading the definitions by means of |\input|
% multiple instances have to be prevented manually:
%\iffalse
%This code needs to be before the `\ProvidesFile' directive
%which is defined at the beginning of this file.
%Therefore it is also placed there and commented out here.
%</package>
%<*discard>
%\fi
%    \begin{macrocode}
\ifdefined\childdocmain\endinput\fi
%    \end{macrocode}
%\iffalse
%</discard>
%<*package>
%\fi
%
% \macro{\ifchilddoc}
% \macro{\ifchilddocmanual}
% The conditional |\ifchilddoc| tells whether a
% child (true) or main (false) document is being compiled.
% The conditional |\ifchilddocmanual| tells whether
% the |\includeonly| mechanism is used (false) or
% the selection of child files must be performed manually (true).
% The definitions initialise to false:
%    \begin{macrocode}
\newif\ifchilddoc
\newif\ifchilddocmanual
%    \end{macrocode}

% \macro{\childdocname}
% \macro{\childdocjob}
% The macro |\childdocname| stores the name of the main document
% to be compiled. The macro |\childdocjob| stores the name of
% the document on which the \LaTeX{} compiler was originally invoked.
% The content of |\jobname| cannot be compared
% to filenames specified in the source due to different catcodes.
% The following code rescans |\jobname|, stores the result
% in |\childdocname| and saves a copy in |\childdocjob|:
%    \begin{macrocode}
\edef\childdocname{\scantokens\expandafter{\jobname\noexpand}}
\let\childdocjob\childdocname
%    \end{macrocode}

% \macro{\childdocdisable}
% The macro |\childdocdisable| prevents the main file
% from being processed more than once.
% At this stage, the main document command |\childdocmain|
% is assumed to be called once again where it should do nothing.
% Any subsequent call to it should prevent
% a secondary processing of the main document
% It overwrites the forwarding commands
% |\childdocof| and |\childdocforward|
% with empty macros to prevent further inclusions of the main document:
%    \begin{macrocode}
\newcommand{\childdocdisable}
{
  \renewcommand{\childdocmain}[1]{\renewcommand{\childdocmain}[1]{\endinput}}
  \renewcommand{\childdocof}[1]{}
  \renewcommand{\childdocby}[2][]{}
  \renewcommand{\childdocforward}[2][]{}
  \renewcommand{\childdocdisable}{}
}
%    \end{macrocode}

% \macro{\childdocmain}
% The macro |\childdocmain| is to be called at the top of the main file
% with nothing or the main filename (without extension) as argument.
% First, it breaks loops.
% If the argument is not empty and does not match |\childdocname|
% (which is set by the first inclusion of |childdoc.def|),
% |\ifchilddoc| is set to true, |\includeonly| is applied to the child file
% and |\jobname| is set to the main file
% (for proper handling of |.aux| files):
%    \begin{macrocode}
\newcommand{\childdocmain}[1]
{
  \childdocdisable\childdocmain{}
  \if?#1?\else
    \begingroup
      \def\childdoctmp{#1}
      \ifx\childdoctmp\childdocname
        \def\childdoctmp{}
      \else
        \def\childdoctmp
        {
          \childdoctrue
          \includeonly{\childdocname}
          \def\childdocjob{#1}
          \def\jobname{#1}
        }
      \fi
      \expandafter
    \endgroup
    \childdoctmp
  \fi
}
%    \end{macrocode}

% \macro{\childdocof}
% The command |\childdocof| redirects
% compilation to the main file |#1|.
%    \begin{macrocode}
\newcommand{\childdocof}[1]
{
  \childdocdisable
  \childdoctrue
  \includeonly{\childdocname}
  \def\jobname{#1}
  \def\childdocjob{#1}
  \input{#1}
}
%    \end{macrocode}

% \macro{\childdocby}
% The command |\childdocby| ....
%    \begin{macrocode}
\newcommand{\childdocby}[2][]
{
  \childdocdisable
  \childdoctrue
  \childdocmanualtrue
  \if?#1?\else
    \def\jobname{#2}
  \fi
  \def\childdocjob{#2}
  \input{#2}
  \endinput
}
%    \end{macrocode}

% \macro{\childdocforward}
% The command |\childdocforward| redirects
% compilation to the main file or
% (if the optional argument is given) a child file.
% Parameters are set as if the main file
% or a child file starting with |\childdocof| was compiled.
% Then compilation is handed over to the main file:
%    \begin{macrocode}
\newcommand{\childdocforward}[2][]
{
  \begingroup
    \if?#1?
      \def\childdoctmp
      {
        \def\childdocname{#2}
        \def\childdocjob{#2}
        \def\jobname{#2}
        \input{#2}
        \endinput
      }
    \else
      \def\childdoctmp
      {
        \childdocdisable
        \def\childdocname{#2}
        \childdoctrue
        \includeonly{#2}
        \def\childdocjob{#1}
        \def\jobname{#1}
        \input{#1}
        \endinput
      }
    \fi
    \expandafter
  \endgroup
  \childdoctmp
}
%    \end{macrocode}

% \macro{\childdocforwardprefix}
% The command |\childdocforwardprefix| redirects
% compilation to the main or a child file by means of a pattern.
% The prefix |#1| in the current filename is replaced by |#2|
% and the suffix of the current filename is kept
% (it is assumed that the filename does not contain the substring `|~~~|'
% which is used as a delimiter).
% Compilation is handed over to the new file by |\childdocforward|:
%    \begin{macrocode}
\newcommand{\childdocforwardprefix}[3][]
{
  \begingroup
    \def\childdocextract #2##1~~~{\def\childdoctmp{\childdocforward[#1]{#3##1}}}
    \expandafter\childdocextract\childdocname~~~
    \expandafter
  \endgroup
  \childdoctmp
}
%    \end{macrocode}

% \macro{\childdoc}
% The deprecated macro |\childdoc| is a legacy version of |\childdocmain|:
%    \begin{macrocode}
\newcommand{\childdoc}{\childdocmain}
%    \end{macrocode}

% \macro{\childdocredirect}
% The deprecated macro |\childdocredirect| is a legacy version
% of |\childdocforward| and |\childdocforwardprefix|:
%    \begin{macrocode}
\newcommand{\childdocredirect}[2][]
{
  \begingroup
    \if?#1?
      \def\childdoctmp{\childdocforward{#2}}
    \else
      \def\childdoctmp{\childdocforwardprefix{#1}{#2}}
    \fi
    \expandafter
  \endgroup
  \childdoctmp
}
%    \end{macrocode}

%\iffalse
%</package>
%\fi
%
\endinput
|\\
|\childdocforwardprefix[|\textit{main}|]{|\textit{prefix}|}{|\textit{dest}|}|
\end{tabular}
\end{center}
%
the destination file is determined by a pattern
depending on the current file:
To make this work, the current file must be called
`{\textit{prefix}\hspace{0.2em}\textit{suffix}}'
with \textit{prefix} matching precisely the argument.
Processing is then passed on to the file
`{\textit{dest}\hspace{0.2em}\textit{suffix}}'.
Surely, the same effect is achieved by
directly specifying the
argument `{\textit{dest}\hspace{0.2em}\textit{suffix}}'
in the first form.
However, that requires to set up a different file
for each child. With the alternative form of the command
all these files can have exactly the same content
which simplifies setting them up and maintaining them.

For example, the following file |draft.tex|
with a compilation flag |\version| as described in \secref{sec:flags}
compiles the main document as a draft:
%
\begin{center}
\begin{tabular}{l}
|\def\version{draft}|\\
|% \iffalse
%
% childdoc.dtx Copyright (C) 2017-2018 Niklas Beisert
%
% This work may be distributed and/or modified under the
% conditions of the LaTeX Project Public License, either version 1.3
% of this license or (at your option) any later version.
% The latest version of this license is in
%   http://www.latex-project.org/lppl.txt
% and version 1.3 or later is part of all distributions of LaTeX
% version 2005/12/01 or later.
%
% This work has the LPPL maintenance status `maintained'.
%
% The Current Maintainer of this work is Niklas Beisert.
%
% This work consists of the files childdoc.dtx and childdoc.ins
% and the derived files childdoc.def and cdocsamp.tex with
% cdocsch1.tex, cdocsch2.tex, cdocsdrf.tex, cdocsfn1.tex, cdocsfn2.tex.
%
%<package>\ifdefined\childdocmain\endinput\fi
%<package>\ProvidesFile{childdoc.def}[2018/12/30 v2.0 child document driver]
%<samplemain>\ProvidesFile{cdocsamp.tex}[2018/12/30 v2.0 sample for childdoc]
%<*driver>
%\ProvidesFile{childdoc.drv}[2018/12/30 v2.0 childdoc reference manual file]
\PassOptionsToClass{10pt,a4paper}{article}
\documentclass{ltxdoc}

\usepackage[margin=35mm]{geometry}
\usepackage{hyperref}
\usepackage{hyperxmp}
\usepackage[usenames]{color}

\hypersetup{colorlinks=true}
\hypersetup{pdfstartview=FitH}
\hypersetup{pdfpagemode=UseNone}
\hypersetup{pdfsource={}}
\hypersetup{pdflang={en-UK}}
\hypersetup{pdfcopyright={Copyright 2017-2018 Niklas Beisert.
  This work may be distributed and/or modified under the
  conditions of the LaTeX Project Public License, either version 1.3
  of this license or (at your option) any later version.}}
\hypersetup{pdflicenseurl={http://www.latex-project.org/lppl.txt}}
\hypersetup{pdfcontactaddress={ETH Zurich, ITP, HIT K,
  Wolfgang-Pauli-Strasse 27}}
\hypersetup{pdfcontactpostcode={8093}}
\hypersetup{pdfcontactcity={Zurich}}
\hypersetup{pdfcontactcountry={Switzerland}}
\hypersetup{pdfcontactemail={nbeisert@itp.phys.ethz.ch}}
\hypersetup{pdfcontacturl={http://people.phys.ethz.ch/\xmptilde nbeisert/}}

\newcommand{\secref}[1]{\hyperref[#1]{section \ref*{#1}}}

\parskip1ex
\parindent0pt
\let\olditemize\itemize
\def\itemize{\olditemize\parskip0pt}

\begin{document}

\title{The \textsf{childdoc} Package}
\hypersetup{pdftitle={The childdoc Package}}
\author{Niklas Beisert\\[2ex]
  Institut f\"ur Theoretische Physik\\
  Eidgen\"ossische Technische Hochschule Z\"urich\\
  Wolfgang-Pauli-Strasse 27, 8093 Z\"urich, Switzerland\\[1ex]
  \href{mailto:nbeisert@itp.phys.ethz.ch}
  {\texttt{nbeisert@itp.phys.ethz.ch}}}
\hypersetup{pdfauthor={Niklas Beisert}}
\hypersetup{pdfsubject={Manual for the LaTeX2e Package childdoc}}
\date{30 December 2018, \textsf{v2.0}}
\maketitle

\begin{abstract}\noindent
\textsf{childdoc} is a \LaTeXe{} package
that enables the direct compilation
of document sections included by |\include|
to individual files.
\end{abstract}

\begingroup
\parskip0ex
\tableofcontents
\endgroup

%%%%%%%%%%%%%%%%%%%%%%%%%%%%%%%%%%%%%%%%%%%%%%%%%%%%%%%%%%%%%%%%%%%%%%%%%%%%%%%%
%%%%%%%%%%%%%%%%%%%%%%%%%%%%%%%%%%%%%%%%%%%%%%%%%%%%%%%%%%%%%%%%%%%%%%%%%%%%%%%%
\section{Introduction}

\LaTeX{} provides a mechanism to structure a large document (such as a book)
into a main file and several child files (containing the chapters)
using the |\include| command.
This mechanism is beneficial for documents
which span hundreds of pages in order to
make the source file(s) more manageable.
Moreover, compilation can be restricted to
selected child files by means of the |\includeonly| command.
The latter feature can be used to reduce the compilation time while editing
(this was significantly more useful in the earlier days of \LaTeX{})
or to generate a smaller document which is easier to navigate.
Another application of |\includeonly| is to generate
documents consisting of selected parts of the complete document.

However, there are a few drawbacks of the plain |\include| mechanism:
\begin{itemize}
\item
The child files cannot be compiled on their own,
they can only be compiled via the main file.
A naive editing environment
(such as a text editor with an option
to have the current file processed by \LaTeX)
may require one to switch to the main file before compiling;
attempting to compile the child file produces errors.
\item
The main file must be modified (each time)
to adjust the |\includeonly| command
to the present needs. This easily leaves the main file in a messy state.
\item
The generated document will always carry the filename
of the main document. This is inconvenient if
several child files are to be compiled and
to be kept for distribution.
\end{itemize}

The present package provides a simple interface
to make child files individually compilable by \LaTeX{}.
Compiling a child file then has the same effect as compiling
the main file with an |\includeonly| command
to select the appropriate child.
Moreover the generated document will carry the name of the child
rather than the main file.
This resolves all three above issues.

This feature is meant to make the editing of books,
thesis documents and lecture notes somewhat more convenient.
However, the package can also be used efficiently for
composing a series of documents (such as exercise sheets)
which are typically distributed individually.
It then assists the author in generating the individual documents
(potentially in different versions)
as well as a document containing the collected series.
Another application is in developing style files
or other kinds of included material
where compilation of the style file could redirect
to a sample or test file.

%%%%%%%%%%%%%%%%%%%%%%%%%%%%%%%%%%%%%%%%%%%%%%%%%%%%%%%%%%%%%%%%%%%%%%%%%%%%%%%%
%%%%%%%%%%%%%%%%%%%%%%%%%%%%%%%%%%%%%%%%%%%%%%%%%%%%%%%%%%%%%%%%%%%%%%%%%%%%%%%%
\section{Usage}

First of all, the package \textsf{childdoc} is \emph{not} a standard
\LaTeXe{} |.sty| style file! Therefore it needs to be invoked in
a non-standard way.

%%%%%%%%%%%%%%%%%%%%%%%%%%%%%%%%%%%%%%%%%%%%%%%%%%%%%%%%%%%%%%%%%%%%%%%%%%%%%%%%
\subsection{Included Files}
\label{sec:include}

%%%%%%%%%%%%%%%%%%%%%%%%%%%%%%%%%%%%%%%%
\DescribeMacro{\childdocmain}
To use the package, add the commands
\begin{center}
\begin{tabular}{l}
|\input{childdoc.def}|\\
|\childdocmain{}|\\
\end{tabular}
\end{center}
at the very top of the main \LaTeX{} file,
in particular \emph{before} the |\documentclass| statement!
The argument of |\childdocmain| should be left empty
(but it must be present).

%%%%%%%%%%%%%%%%%%%%%%%%%%%%%%%%%%%%%%%%
\DescribeMacro{\childdocof}
Furthermore, add the commands
\begin{center}
\begin{tabular}{l}
|\input{childdoc.def}|\\
|\childdocof{|\textit{main}|}|\\
\end{tabular}
\end{center}
at the top of every child file \textit{child}
which is included by |\include{|\textit{child}|}|
from within the main file
(or at least for those files to be compiled individually).
The argument \textit{main} must be the filename of the main file.

There are a couple of
considerations in setting up the main and child documents:

%%%%%%%%%%%%%%%%%%%%%%%%%%%%%%%%%%%%%%%%
\paragraph{Restrictions.}

Please note the following restrictions:
\begin{itemize}
\item
|\childdocmain| must be called with one argument \textit{main}
to ensure compatibility with earlier version of the package.
It must either be empty (|\childdocmain{}|)
or precisely match the filename of the main file in which it is specified.
See \secref{sec:detection} for further information.
\item
The filename \textit{main} must be specified without the |.tex| extension.
\item
The filename \textit{main} is case sensitive
(even in case-insensitive file systems)
due to internal string comparison.
\item
The argument \textit{main} should be fully expanded, it cannot be a macro.
\item
Subdirectories and special characters should be avoided in filenames.
\item
The command |\childdocmain{|\textit{main}|}| must be followed by a whitespace.
It should not be followed immediately by another command
or by a comment mark `|%|'.
This is because the \TeX{} parser reads the token immediately following
the argument of |\childdocmain| and puts it
at the beginning of every child section;
however, a white\-space is ignored.
\end{itemize}

%%%%%%%%%%%%%%%%%%%%%%%%%%%%%%%%%%%%%%%%
\paragraph{Content of Main File.}

It is advisable to place all content in the child files included by |\include|.
Any output contained in the main file will appear in all child documents
unless suppressed manually;
it cannot be suppressed automatically by the |\includeonly| directive
and thus should normally be avoided.
A method to include some content in the main file
by means of conditional processing is described in \secref{sec:conditional}.

%%%%%%%%%%%%%%%%%%%%%%%%%%%%%%%%%%%%%%%%
\paragraph{Page Numbering.}

When only a part of the document is compiled,
the appropriate numbering of pages
(as well as other status parameters)
is determined from the |.aux| files.
The latter contain information from previous passes.
However this information needs to propagate through
all intermediate child documents.
Therefore the page numbering in child documents may well
be inconsistent until the complete document is compiled at least once.

A useful (if unconventional) way to always ensure a consistent
page numbering is to restart the numbering in each child document
and denote the pages by `\textit{child}|.|\textit{page}'
where \textit{child} represents the chapter/section number of the child file.
This can be achieved by the command
|\numberwithin{page}{|\textit{child}|}|
of the \textsf{amsmath} package
where \textit{child} can be |chapter| or |section|
depending on the chosen structuring.
Alternatively, one can modify the macro |\thepage| appropriately
and reset the counter |page| at the start of each child file.

%%%%%%%%%%%%%%%%%%%%%%%%%%%%%%%%%%%%%%%%%%%%%%%%%%%%%%%%%%%%%%%%%%%%%%%%%%%%%%%%
\subsection{Conditional Processing}
\label{sec:conditional}

The package provides a mechanism to compile different versions
of a document. To customise the versions further some conditional processing
can come in handy to distinguish which version is being compiled.
The package provides two macros to describe the compilation context:

%%%%%%%%%%%%%%%%%%%%%%%%%%%%%%%%%%%%%%%%
\DescribeMacro{\ifchilddoc}
The conditional |\ifchilddoc| distinguishes between the compilation of
child documents and the main document:
%
\begin{center}
|\ifchilddoc |\textit{child-code}| |[|\||else |\textit{main-code}]| \||fi|
\end{center}

%%%%%%%%%%%%%%%%%%%%%%%%%%%%%%%%%%%%%%%%
\DescribeMacro{\childdocname}
\DescribeMacro{\childdocjob}
The macro |\childdocname| contains the filename (without extension)
of the main or child file being processed.
Note that |\childdocjob| will always contain the name of the main file.

%%%%%%%%%%%%%%%%%%%%%%%%%%%%%%%%%%%%%%%%
\paragraph{Title Page.}

Conditional processing can be used to include a title or banner page
in the main document when proper precautions are taken.
Importantly, the code in the main file should ensure that the page counter
(as well as other status parameters which are stored in the |.aux| files)
takes the same value after the conditional processing.
Otherwise the page numbers may take divergent values
depending on which part is compiled.

For example, a title page could be declared by:
%
\begin{center}
\begin{tabular}{l}
|\ifchilddoc\||else|\\
|\addtocounter{page}{-1}|\\
\textit{code for title page}\\
|\newpage|\\
|\||fi|
\end{tabular}
\end{center}
%
A banner page for the child documents can be generated by:
%
\begin{center}
\begin{tabular}{l}
|\ifchilddoc|\\
|\addtocounter{page}{-1}|\\
\textit{code for banner page}\\
|\newpage|\\
|\||fi|
\end{tabular}
\end{center}
%
Here one could write a message such as:
\begin{center}
|This is the part \childdocname{} of \childdocjob{}.|
\end{center}

%%%%%%%%%%%%%%%%%%%%%%%%%%%%%%%%%%%%%%%%%%%%%%%%%%%%%%%%%%%%%%%%%%%%%%%%%%%%%%%%
\subsection{Flags}
\label{sec:flags}

The package makes it easy to generate different versions
of the main or child documents.
To this end compilation flags can be defined
and assigned different default values.
They will be particularly useful in conjunction
with the forwarding mechanism described in \secref{sec:forward}.

For example, it may be useful to have a flag |\version|
which can be set to |draft| or |final|.
The document source will contain some conditional code
depending on the value of |\version|.
Suppose further, the flag should default to |final| for the main file
and to |draft| for child files
which is a natural assignment for editing the document.
This is achieved by placing the following code
in the preamble of the main document
(below the |\childdocmain| directive):
%
\begin{center}
\begin{tabular}{l}
|\ifchilddoc|\\
|\providecommand{\version}{draft}|\\
|\||else|\\
|\providecommand{\version}{final}|\\
|\||fi|
\end{tabular}
\end{center}
%
The definition by |\providecommand| makes sure
that previous definitions are not overwritten.
Further statements |\providecommand{\version}{...}|
can thus be added before the above code to override it.

For the main file, one might add a line
(between |\childdocmain| and the above block)
%
\begin{center}
|%\ifchilddoc\||else\providecommand{\version}{draft}\||fi|
\end{center}
%
which can be uncommented to produce a draft version.
Likewise one can add a line to the very top of a child file
(above the |\childdocof{|\textit{main}|}| directive)
%
\begin{center}
|%\providecommand{\version}{final}|
\end{center}
%
which can be uncommented to produce the final version of this child document.

%%%%%%%%%%%%%%%%%%%%%%%%%%%%%%%%%%%%%%%%%%%%%%%%%%%%%%%%%%%%%%%%%%%%%%%%%%%%%%%%
\subsection{Forwarding}
\label{sec:forward}

Different versions of the main or child documents
using compilation flags as described in \secref{sec:flags}
can be (permanently) stored in different files
for convenient compilation, viewing and distribution.
To this end, the package defines a command
to pass on compilation to a different file:

%%%%%%%%%%%%%%%%%%%%%%%%%%%%%%%%%%%%%%%%
\DescribeMacro{\childdocforward}
The command |\childdocforward| redirects processing to
another source file:
%
\begin{center}
\begin{tabular}{l}
|\input{childdoc.def}|\\
|\childdocforward[|\textit{main}|]{|\textit{dest}|}|\\
\end{tabular}
\end{center}
%
The argument \textit{dest} is the destination file
(without extension).
It should be the main file or one of the child files.
Note that further \textsf{childdoc} directives
such as |\childdocof| and |\childdocforward|
in the indicated file will be processed in this form.
The optional argument \textit{main}
passes on directly to the main file \textit{main}
while pretending to compile the child \textit{dest}.
This form behaves as if \textit{dest}
issues |\childdocof{|\textit{main}|}| right away,
and no further \textsf{childdoc} directives will be processed.

%%%%%%%%%%%%%%%%%%%%%%%%%%%%%%%%%%%%%%%%
\DescribeMacro{\...prefix}
In the alternative form |\childdocforwardprefix|,
%
\begin{center}
\begin{tabular}{l}
|\input{childdoc.def}|\\
|\childdocforwardprefix[|\textit{main}|]{|\textit{prefix}|}{|\textit{dest}|}|
\end{tabular}
\end{center}
%
the destination file is determined by a pattern
depending on the current file:
To make this work, the current file must be called
`{\textit{prefix}\hspace{0.2em}\textit{suffix}}'
with \textit{prefix} matching precisely the argument.
Processing is then passed on to the file
`{\textit{dest}\hspace{0.2em}\textit{suffix}}'.
Surely, the same effect is achieved by
directly specifying the
argument `{\textit{dest}\hspace{0.2em}\textit{suffix}}'
in the first form.
However, that requires to set up a different file
for each child. With the alternative form of the command
all these files can have exactly the same content
which simplifies setting them up and maintaining them.

For example, the following file |draft.tex|
with a compilation flag |\version| as described in \secref{sec:flags}
compiles the main document as a draft:
%
\begin{center}
\begin{tabular}{l}
|\def\version{draft}|\\
|\input{childdoc.def}|\\
|\childdocforward{|\textit{main}|}|
\end{tabular}
\end{center}
%
Likewise, the following files |final|\textit{nn}|.tex|
compile the final version of the child document
|child|\textit{nn}|.tex|:
%
\begin{center}
\begin{tabular}{l}
|\def\version{final}|\\
|\input{childdoc.def}|\\
|\childdocforwardprefix{final}{child}|
\end{tabular}
\end{center}
%

Note that when several versions of a main file and/or of each child file
are to be generated, it may be convenient to set up a |Makefile| or
shell script to automatise the process.

%%%%%%%%%%%%%%%%%%%%%%%%%%%%%%%%%%%%%%%%%%%%%%%%%%%%%%%%%%%%%%%%%%%%%%%%%%%%%%%%
\subsection{Command Line Processing}
\label{sec:commandline}

The effect of redirection files can also be achieved by invoking
the \LaTeX{} compiler with a more elaborate command line.
Most conveniently this should be done as part
of a shell script or a |Makefile|.

When using \textsf{childdoc} in the main file, the following
command lines effectively perform a redirection
(note that depending on the shell being used,
backslashes may have to be doubled: `|\|' $\to$ `|\\|'):
%
\begin{center}
|... -jobname "|\textit{target}|" |\\|"|[\textit{flags}]%
|\input{childdoc.def}\childdocforward[|\textit{main}|]{|\textit{dest}|}"|
\end{center}
%
Here \textit{target} is the name of the output file,
\textit{main} is the name of the main file
and \textit{dest} is the name of the main or child file to be processed
(all filenames without extensions).
The optional argument \textit{main} can be omitted
if \textit{main} matches \textit{dest}.
Optionally, compilation \textit{flags} can be defined via |\def| commands.
This command line makes the \TeX{} engine believe
it is compiling the file \textit{target}
whose content is specified as the latter parameter.
The provided code then forwards the processing to
\textit{main} or \textit{dest} as described in \secref{sec:forward}.

%%%%%%%%%%%%%%%%%%%%%%%%%%%%%%%%%%%%%%%%%%%%%%%%%%%%%%%%%%%%%%%%%%%%%%%%%%%%%%%%
\subsection{Include by Input}
\label{sec:input}

Including child documents by |\include| has some restrictions by design.
Most notably, the content of a child document always occupies
its own set of pages; pages cannot be shared between child documents.
Usually, this behaviour makes perfect sense
because each child document contain an essential part of the document.
However, in some situations it may be desirable to compose
a document from a collection of parts
without having mandatory page breaks between then.
For this case, the package
provides a mechanism to include parts
by |\input| which can also be processed individually.
However, by construction this mechanism
requires manual handling of the content to be output.

%%%%%%%%%%%%%%%%%%%%%%%%%%%%%%%%%%%%%%%%
\DescribeMacro{\ifchilddocmanual}
The main file should be prepared as usual, see \secref{sec:include}.
However, the document body must make a distinction
between processing of an individual part and of the main document, e.g.:
%
\begin{center}
\begin{tabular}{l}
|\ifchilddocmanual|\\
|\input{\childdocname}|\\
|\||else|\\
\textit{document body with }|\input{|\textit{part}|}|\\
|\||fi|
\end{tabular}
\end{center}
%
The conditional |\ifchilddocmanual| is true whenever
a part to be included by |\input| is being compiled,
and the name of the part is stored in |\childdocname|.

%%%%%%%%%%%%%%%%%%%%%%%%%%%%%%%%%%%%%%%%
\DescribeMacro{\childdocby}
Each part to be included by |\input| should start with:
%
\begin{center}
\begin{tabular}{l}
|\input{childdoc.def}|\\
|\childdocby{|\textit{main}|}|\\
\end{tabular}
\end{center}
%
The directive |\childdocby| is similar to |\childdocof|
described in \secref{sec:include},
but the subsequent selection of content must be done manually.
To that end, both |\ifchilddoc| and |\ifchilddocmanual|
will be true upon processing of a part,
and the name of the part is stored in |\childdocname|.
Note that |\jobname| will be set to the filename of the current part
so that each part receives an individual |.aux| file
that does not interfere with the |.aux| file(s) of the main document.
This behaviour can be altered by the alternative form
|\childdocby[*]{|\textit{main}|}| (with a non-empty optional argument)
which uses the |.aux| file of the main document
by setting |\jobname| to \textit{main}.

%%%%%%%%%%%%%%%%%%%%%%%%%%%%%%%%%%%%%%%%%%%%%%%%%%%%%%%%%%%%%%%%%%%%%%%%%%%%%%%%
\subsection{Driver Development}
\label{sec:driver}

The \textsf{childdoc} mechanism can also be use for the development
of definition files such as \LaTeX{} styles or classes.
This case differs from the above setup with multiple parts
included by |\include| in that no |\includeonly| should be invoked.
This can be achieved by starting the include file
(before |\ProvidesPackage|) with:
%
\begin{center}
\begin{tabular}{l}
|\input{childdoc.def}|\\
|\childdocforward{|\textit{main}|}|\\
\end{tabular}
\end{center}
%
or alternatively with:
%
\begin{center}
\begin{tabular}{l}
|\input{childdoc.def}|\\
|\childdocby{|\textit{main}|}|\\
\end{tabular}
\end{center}
%
Both forms have slightly different effects as described above.
The main file is prepared as usual, see \secref{sec:include}.

%%%%%%%%%%%%%%%%%%%%%%%%%%%%%%%%%%%%%%%%%%%%%%%%%%%%%%%%%%%%%%%%%%%%%%%%%%%%%%%%
\subsection{Legacy Detection}
\label{sec:detection}

The directive |\childdocmain| in the main file can detect
whether the complete document or merely a child is to be compiled
even without using the directive |\childdocof|.
This method is deprecated because it is less robust
and there is no compelling reason to use it;
it is merely provided for backward compatibility
and it may be removed in future versions.

If the detection mechanism is to be used,
it is mandatory to correctly specify
the filename of the main file as the argument of |\childdocmain|:
%
\begin{center}
\begin{tabular}{l}
|\input{childdoc.def}|\\
|\childdocmain{|\textit{main}|}|\\
\end{tabular}
\end{center}
%
If |\jobname| does not match the argument \textit{main} of |\childdocmain|,
it is assumed that |\jobname| points to the child file to be compiled.
When using |\childdocmain| with the main file specified as argument,
it suffices to start a child file
with just |\input{|\textit{main}|}|
without loading of the package and using |\childdocof|.
If instead all processing is done
with the appropriate \textsf{childdoc} directives,
the argument of \textit{main} of |\childdocmain| can be empty.

An alternative version of the command line processing described
in \secref{sec:commandline} using the detection mechanism reads:
%
\begin{center}
|... -jobname "|\textit{target}|" "|[\textit{flags}]%
[|\def\jobname{|\textit{dest}|}|]|\input{|\textit{main}|}"|
\end{center}

%%%%%%%%%%%%%%%%%%%%%%%%%%%%%%%%%%%%%%%%%%%%%%%%%%%%%%%%%%%%%%%%%%%%%%%%%%%%%%%%
\subsection{Manual Code}
\label{sec:manual}

In case one cannot be certain whether the definitions file |childdoc.def|
is installed on the target \TeX{} distribution
and one prefers not to ship it,
it is conceivable to paste a few relevant commands into the sources.

To that end, drop all statements |\input{childdoc.def}|
and perform the replacements as outlined below.
Instead of |\childdocmain{|\textit{main}|}| add the following code
to the top of the main file:
%
\begin{center}
\begin{tabular}{l}
|\||ifdefined\childdocname\endinput\||fi\newif\ifchilddoc|\\
|\edef\childdocname{\scantokens\expandafter{\jobname\noexpand}}|\\
|\def\childdocmain{|\textit{main}|}\||ifx\childdocmain\childdocname\||else|\\
|\childdoctrue\includeonly{\childdocname}\let\jobname\childdocmain\||fi|\\
\end{tabular}
\end{center}
%
Instead of |\childdocof{|\textit{main}|}| just include the main file
at the top of each child file:
%
\begin{center}
|\input{|\textit{main}|}|
\end{center}
%
A simple redirection |\childdocforward{|\textit{dest}|}| is achieved by:
%
\begin{center}
|\def\jobname{|\textit{dest}|}\input{\jobname}|
\end{center}
%
The redirection with prefix
|\childdocforwardprefix[|\textit{prefix}|]{|\textit{dest}|}|
is accomplished by:
%
\begin{center}
\begin{tabular}{l}
|{\edef\jobname{\scantokens\expandafter{\jobname\noexpand}}|\\
|\def\redirectjob |\textit{prefix}|#1~~~{\gdef\jobname{|\textit{dest}|#1}}|\\
|\expandafter\redirectjob\jobname~~~}\input{\jobname}|
\end{tabular}
\end{center}

In an alternative approach,
child documents can be compiled by a specific command line
without additional code or specific definitions:
%
\begin{center}
|... -jobname "|\textit{target}|" "|[\textit{flags}]%
|\includeonly{|\textit{dest}|}\input{|\textit{main}|}"|
\end{center}
%

%%%%%%%%%%%%%%%%%%%%%%%%%%%%%%%%%%%%%%%%%%%%%%%%%%%%%%%%%%%%%%%%%%%%%%%%%%%%%%%%
%%%%%%%%%%%%%%%%%%%%%%%%%%%%%%%%%%%%%%%%%%%%%%%%%%%%%%%%%%%%%%%%%%%%%%%%%%%%%%%%
\section{Information}

%%%%%%%%%%%%%%%%%%%%%%%%%%%%%%%%%%%%%%%%%%%%%%%%%%%%%%%%%%%%%%%%%%%%%%%%%%%%%%%%
\subsection{Copyright}

Copyright \copyright{} 2017--2018 Niklas Beisert

This work may be distributed and/or modified under the
conditions of the \LaTeX{} Project Public License, either version 1.3
of this license or (at your option) any later version.
The latest version of this license is in
  \url{http://www.latex-project.org/lppl.txt}
and version 1.3 or later is part of all distributions of \LaTeX{}
version 2005/12/01 or later.

This work has the LPPL maintenance status `maintained'.

The Current Maintainer of this work is Niklas Beisert.

This work consists of the files |README.txt|, |childdoc.ins| and |childdoc.dtx|
as well as the derived files |childdoc.def|, |cdocsamp.tex|
with |cdocsch1.tex|, |cdocsch2.tex|, |cdocspt3.tex|, |cdocspt4.tex|,
|cdocsdrf.tex|, |cdocsfn1.tex|, |cdocsfn2.tex|
as well as |childdoc.pdf|.

%%%%%%%%%%%%%%%%%%%%%%%%%%%%%%%%%%%%%%%%%%%%%%%%%%%%%%%%%%%%%%%%%%%%%%%%%%%%%%%%
\subsection{Files and Installation}

The package consists of the files:
%
\begin{center}
\begin{tabular}{ll}
    |README.txt|   & readme file \\
    |childdoc.ins| & installation file \\
    |childdoc.dtx| & source file \\
    |childdoc.def| & definition file \\
    |cdocsamp.tex| & sample main file \\
    |cdocsch1.tex| & sample include file \\
    |cdocsch2.tex| & sample include file \\
    |cdocspt3.tex| & sample part file \\
    |cdocspt4.tex| & sample part file \\
    |cdocsdrf.tex| & sample redirection file \\
    |cdocsfn1.tex| & sample redirection file \\
    |cdocsfn2.tex| & sample redirection file \\
    |childdoc.pdf| & manual
\end{tabular}
\end{center}
%
The distribution consists of the files
|README.txt|, |childdoc.ins| and |childdoc.dtx|.
%
\begin{itemize}
\item
Run (pdf)\LaTeX{} on |childdoc.dtx|
to compile the manual |childdoc.pdf| (this file).
\item
Run \LaTeX{} on |childdoc.ins| to create the definitions file |childdoc.def|
and the sample |cdocsamp.tex| with include files
|cdocsch1.tex|, |cdocsch2.tex|, |cdocspt3.tex|, |cdocspt4.tex|,
|cdocsdrf.tex|, |cdocsfn1.tex|, |cdocsfn2.tex|.
Then copy the file |childdoc.def| to an appropriate directory of your \LaTeX{}
distribution, e.g.\ \textit{texmf-root}|/tex/latex/childdoc|.
\end{itemize}

%%%%%%%%%%%%%%%%%%%%%%%%%%%%%%%%%%%%%%%%%%%%%%%%%%%%%%%%%%%%%%%%%%%%%%%%%%%%%%%%
\subsection{Related CTAN Packages}

There are several other packages which offer a similar functionality:
%
\begin{itemize}
\item
The packages
\href{http://ctan.org/pkg/docmute}{\textsf{docmute}},
\href{http://ctan.org/pkg/includex}{\textsf{includex}} and
\href{http://ctan.org/pkg/standalone}{\textsf{standalone}}
provide commands to include only the document body of
a child file thus allowing both files to be compiled individually.
\item
The packages \href{http://ctan.org/pkg/subdocs}{\textsf{subdocs}}
and \href{http://ctan.org/pkg/subfiles}{\textsf{subfiles}}
provide structures in which the main and child documents can be
encapsulated and allowing them to be compiled individually.
The inclusion mechanism is different from the conventional |\include|.
\item
The package \href{http://ctan.org/pkg/combine}{\textsf{combine}}
is an elaborate solution to combine several documents into one.
\end{itemize}
%
See also the CTAN topic \href{http://ctan.org/topic/subdocs}{\textsf{subdocs}}
for further related packages.
The present package differs from the above solutions in that
a document structure constructed with the conventional |\include| mechanism
just needs two extra commands at the top of every file
such that all constituent files can be compiled individually.

%%%%%%%%%%%%%%%%%%%%%%%%%%%%%%%%%%%%%%%%%%%%%%%%%%%%%%%%%%%%%%%%%%%%%%%%%%%%%%%%
%\subsection{Feature Suggestions}
%
%The following is a list of features which may be useful for future
%versions of this package:
%%
%\begin{itemize}
%\item
%\ldots
%\end{itemize}

%%%%%%%%%%%%%%%%%%%%%%%%%%%%%%%%%%%%%%%%%%%%%%%%%%%%%%%%%%%%%%%%%%%%%%%%%%%%%%%%
\subsection{Revision History}

%%%%%%%%%%%%%%%%%%%%%%%%%%%%%%%%%%%%%%%%
\paragraph{v2.0:} 2018/12/30

\begin{itemize}
\item
immediate forward processing
\item
added |\childdocby| mechanism
\item
manual restructured
\end{itemize}

%%%%%%%%%%%%%%%%%%%%%%%%%%%%%%%%%%%%%%%%
\paragraph{v1.6:} 2018/01/17

\begin{itemize}
\item
application for development of include files
\item
corrections to manual
\end{itemize}

%%%%%%%%%%%%%%%%%%%%%%%%%%%%%%%%%%%%%%%%
\paragraph{v1.5:} 2017/05/21

\begin{itemize}
\item
more complete structuring introduced
\item
|\childdocof| introduced
\item
|\childdoc| renamed to |\childdocmain|
\item
|\childredirect| renamed to |\childdocforward| and |\childdocforwardprefix|
and functionality expanded
\end{itemize}

%%%%%%%%%%%%%%%%%%%%%%%%%%%%%%%%%%%%%%%%
\paragraph{v1.0:} 2017/04/27

\begin{itemize}
\item
manual and install package
\item
first version published on CTAN
\end{itemize}

%%%%%%%%%%%%%%%%%%%%%%%%%%%%%%%%%%%%%%%%
\paragraph{v0.6:} 2017/04/26

\begin{itemize}
\item
redirection mechanism added
\end{itemize}

%%%%%%%%%%%%%%%%%%%%%%%%%%%%%%%%%%%%%%%%
\paragraph{v0.5:} 2017/04/26

\begin{itemize}
\item
functionality in definition file
\end{itemize}


%%%%%%%%%%%%%%%%%%%%%%%%%%%%%%%%%%%%%%%%%%%%%%%%%%%%%%%%%%%%%%%%%%%%%%%%%%%%%%%%
%%%%%%%%%%%%%%%%%%%%%%%%%%%%%%%%%%%%%%%%%%%%%%%%%%%%%%%%%%%%%%%%%%%%%%%%%%%%%%%%
%%%%%%%%%%%%%%%%%%%%%%%%%%%%%%%%%%%%%%%%%%%%%%%%%%%%%%%%%%%%%%%%%%%%%%%%%%%%%%%%
\appendix

\settowidth\MacroIndent{\rmfamily\scriptsize 000\ }

 \DocInput{childdoc.dtx}

\end{document}
%</driver>
% \fi
%
% %%%%%%%%%%%%%%%%%%%%%%%%%%%%%%%%%%%%%%%%%%%%%%%%%%%%%%%%%%%%%%%%%%%%%%%%%%%%%%
% %%%%%%%%%%%%%%%%%%%%%%%%%%%%%%%%%%%%%%%%%%%%%%%%%%%%%%%%%%%%%%%%%%%%%%%%%%%%%%
% \section{Sample}
%\iffalse
%<*samplemain>
%\fi
%
% The following presents a sample document
% with two chapters, two parts, a title page,
% a compile flag as well as three forwarding files to set the flag.
% It consists of eight |.tex| files:
% \begin{center}
% \begin{tabular}{ll}
% |cdocsamp.tex|&main file\\
% |cdocsch1.tex|&include file for chapter 1\\
% |cdocsch2.tex|&include file for chapter 2\\
% |cdocspt3.tex|&include file for part 3\\
% |cdocspt4.tex|&include file for part 4\\
% |cdocsdrf.tex|&forwarding file for main file in draft mode\\
% |cdocsfi1.tex|&forwarding file for final version of chapter 1\\
% |cdocsfi2.tex|&forwarding file for final version of chapter 2\\
% \end{tabular}
% \end{center}
% Each of the eight files can be compiled directly by the \LaTeX{} compiler.
%
% %%%%%%%%%%%%%%%%%%%%%%%%%%%%%%%%%%%%%%
% \paragraph{Main File.}
%
% The main file is called |cdocsamp.tex|.
%
% Load the \textsf{childdoc} definitions and
% declare the filename for the main document:
%    \begin{macrocode}
\input{childdoc.def}
\childdocmain{}
%    \end{macrocode}

% Optional override for |\version| flag:
%    \begin{macrocode}
%%\ifchilddoc\else\providecommand{\version}{draft}\fi
%    \end{macrocode}

% Define the default values for the |\version| flag
% (|final| for the main file and |draft| for childs):
%    \begin{macrocode}
\ifchilddoc
\providecommand{\version}{draft}
\else
\providecommand{\version}{final}
\fi
%    \end{macrocode}

% Load the standard document class:
%    \begin{macrocode}
\documentclass[12pt]{article}
%    \end{macrocode}

% Start the document body:
%    \begin{macrocode}
\begin{document}
%    \end{macrocode}

% Declare a title page.
% Print title, part of document being processed and version flag:
%    \begin{macrocode}
\addtocounter{page}{-1}
\begin{center}
{\LARGE\bfseries{}childdoc example\par}
\vspace{1cm}
\ifchilddoc
\ifchilddocmanual part\else chapter\fi:
`\childdocname' of `\childdocjob'\par
\else
main document: `\childdocjob'\par
\fi
version: \version\par
\end{center}
\newpage
%    \end{macrocode}

% Manually include selected file,
% otherwise process as usual:
%    \begin{macrocode}
\ifchilddocmanual
\section*{part `\childdocname'}
\input{\childdocname}
\else
%    \end{macrocode}

% Include the two chapters:
%    \begin{macrocode}
\include{cdocsch1}
\include{cdocsch2}
%    \end{macrocode}

% Include the two parts unless only chapters should be displayed:
%    \begin{macrocode}
\ifchilddoc\else
\section{part three}
\input{cdocspt3}
\section{part four}
\input{cdocspt4}
\fi
%    \end{macrocode}

% Process as usual until here:
%    \begin{macrocode}
\fi
%    \end{macrocode}

% End of document body:
%    \begin{macrocode}
\end{document}
%    \end{macrocode}
%\iffalse
%</samplemain>
%\fi
%
% %%%%%%%%%%%%%%%%%%%%%%%%%%%%%%%%%%%%%%
% \paragraph{Chapter Include Files.}
%
% The include files are called |cdocsch1.tex| and |cdocsch2.tex|.
%
%\iffalse
%<*samplechap1|samplechap2>
%\fi

% Optional override for |\version| flag:
%    \begin{macrocode}
%%\providecommand{\version}{final}
%    \end{macrocode}

% Include the main document:
%    \begin{macrocode}
\input{childdoc.def}
\childdocof{cdocsamp}
%    \end{macrocode}

%\iffalse
%</samplechap1|samplechap2>
%\fi
%
%\iffalse
%<*samplechap1>
%\fi
% Some text for chapter 1:
%    \begin{macrocode}
\section{one}
some text in chapter one
%    \end{macrocode}

%\iffalse
%</samplechap1>
%\fi
% Some text for chapter 2:
%\iffalse
%<*samplechap2>
%\fi
%    \begin{macrocode}
\section{two}
more text in chapter two
%    \end{macrocode}

%\iffalse
%</samplechap2>
%\fi
%
% %%%%%%%%%%%%%%%%%%%%%%%%%%%%%%%%%%%%%%
% \paragraph{Part Include Files.}
%
% The include files are called |cdocspt3.tex| and |cdocspt4.tex|.
%
%\iffalse
%<*samplepart3|samplepart4>
%\fi

% Optional override for |\version| flag:
%    \begin{macrocode}
%%\providecommand{\version}{final}
%    \end{macrocode}

% Include the main document:
%    \begin{macrocode}
\input{childdoc.def}
\childdocby{cdocsamp}
%    \end{macrocode}

%\iffalse
%</samplepart3|samplepart4>
%\fi
%
%\iffalse
%<*samplepart3>
%\fi
% Some text for part 3:
%    \begin{macrocode}
some text in part three
%    \end{macrocode}

%\iffalse
%</samplepart3>
%\fi
% Some text for part 4:
%\iffalse
%<*samplepart4>
%\fi
%    \begin{macrocode}
more text in part four
%    \end{macrocode}

%\iffalse
%</samplepart4>
%\fi
%
% %%%%%%%%%%%%%%%%%%%%%%%%%%%%%%%%%%%%%%
% \paragraph{Forwarding for a Complete Draft.}
%
% The following forwarding file |cdocsdrf.tex|
% compiles the main document in draft mode:
%\iffalse
%<*sampledraft>
%\fi
%    \begin{macrocode}
\def\version{draft}
\input{childdoc.def}
\childdocforward{cdocsamp}
%    \end{macrocode}

%\iffalse
%</sampledraft>
%\fi
%
% %%%%%%%%%%%%%%%%%%%%%%%%%%%%%%%%%%%%%%
% \paragraph{Forwarding for Final Version of the Chapters.}
%
% The following forwarding files |cdocsfn1.tex| and |cdocsfn2.tex|
% (with identical content)
% compile the final versions of the child documents
% |cdocsch1.tex| and |cdocsch2.tex|, respectively:
%\iffalse
%<*samplefinal>
%\fi
%    \begin{macrocode}
\def\version{final}
\input{childdoc.def}
\childdocforwardprefix[cdocsamp]{cdocsfn}{cdocsch}
%    \end{macrocode}

%\iffalse
%</samplefinal>
%\fi
%
% %%%%%%%%%%%%%%%%%%%%%%%%%%%%%%%%%%%%%%
% \paragraph{Command Line Processing.}
%
% The following three command lines generate the output files
% |cdocscld|, |cdocscl1| and |cdocscl2|
% which should be identical to
% |cdocsdrf|, |cdocsch1| and |cdocsfn2|, respectively:
% \begin{center}
% \begin{tabular}{l}
% |latex -jobname cdocscld \|\\
% |  "\def\version{draft}\input{childdoc.def}\childdocforward{cdocsamp}"|\\
% |latex -jobname cdocscl1 \|\\
% |  "\input{childdoc.def}\childdocforward[cdocsamp]{cdocsch1}"|\\
% |latex -jobname cdocscl2 \|\\
% |  "\def\version{final}\input{childdoc.def}\childdocforward{cdocsch2}"|
% \end{tabular}
% \end{center}
% Note that the trailing backslash on each first line
% merely continues the input to the second line
% (for convenient cut ant paste).
% Furthermore, the command |latex| can be replaced by any
% of its alternative versions such as |pdflatex|.
%
% %%%%%%%%%%%%%%%%%%%%%%%%%%%%%%%%%%%%%%%%%%%%%%%%%%%%%%%%%%%%%%%%%%%%%%%%%%%%%%
% %%%%%%%%%%%%%%%%%%%%%%%%%%%%%%%%%%%%%%%%%%%%%%%%%%%%%%%%%%%%%%%%%%%%%%%%%%%%%%
% \section{Implementation}
%\iffalse
%<*package>
%\fi
%
% This section describes the definitions file |childdoc.def|.

% The definitions cannot be loaded using |\usepackage| or |\RequirePackage|
% which has a mechanism to prevent loading a style file more than once.
% When loading the definitions by means of |\input|
% multiple instances have to be prevented manually:
%\iffalse
%This code needs to be before the `\ProvidesFile' directive
%which is defined at the beginning of this file.
%Therefore it is also placed there and commented out here.
%</package>
%<*discard>
%\fi
%    \begin{macrocode}
\ifdefined\childdocmain\endinput\fi
%    \end{macrocode}
%\iffalse
%</discard>
%<*package>
%\fi
%
% \macro{\ifchilddoc}
% \macro{\ifchilddocmanual}
% The conditional |\ifchilddoc| tells whether a
% child (true) or main (false) document is being compiled.
% The conditional |\ifchilddocmanual| tells whether
% the |\includeonly| mechanism is used (false) or
% the selection of child files must be performed manually (true).
% The definitions initialise to false:
%    \begin{macrocode}
\newif\ifchilddoc
\newif\ifchilddocmanual
%    \end{macrocode}

% \macro{\childdocname}
% \macro{\childdocjob}
% The macro |\childdocname| stores the name of the main document
% to be compiled. The macro |\childdocjob| stores the name of
% the document on which the \LaTeX{} compiler was originally invoked.
% The content of |\jobname| cannot be compared
% to filenames specified in the source due to different catcodes.
% The following code rescans |\jobname|, stores the result
% in |\childdocname| and saves a copy in |\childdocjob|:
%    \begin{macrocode}
\edef\childdocname{\scantokens\expandafter{\jobname\noexpand}}
\let\childdocjob\childdocname
%    \end{macrocode}

% \macro{\childdocdisable}
% The macro |\childdocdisable| prevents the main file
% from being processed more than once.
% At this stage, the main document command |\childdocmain|
% is assumed to be called once again where it should do nothing.
% Any subsequent call to it should prevent
% a secondary processing of the main document
% It overwrites the forwarding commands
% |\childdocof| and |\childdocforward|
% with empty macros to prevent further inclusions of the main document:
%    \begin{macrocode}
\newcommand{\childdocdisable}
{
  \renewcommand{\childdocmain}[1]{\renewcommand{\childdocmain}[1]{\endinput}}
  \renewcommand{\childdocof}[1]{}
  \renewcommand{\childdocby}[2][]{}
  \renewcommand{\childdocforward}[2][]{}
  \renewcommand{\childdocdisable}{}
}
%    \end{macrocode}

% \macro{\childdocmain}
% The macro |\childdocmain| is to be called at the top of the main file
% with nothing or the main filename (without extension) as argument.
% First, it breaks loops.
% If the argument is not empty and does not match |\childdocname|
% (which is set by the first inclusion of |childdoc.def|),
% |\ifchilddoc| is set to true, |\includeonly| is applied to the child file
% and |\jobname| is set to the main file
% (for proper handling of |.aux| files):
%    \begin{macrocode}
\newcommand{\childdocmain}[1]
{
  \childdocdisable\childdocmain{}
  \if?#1?\else
    \begingroup
      \def\childdoctmp{#1}
      \ifx\childdoctmp\childdocname
        \def\childdoctmp{}
      \else
        \def\childdoctmp
        {
          \childdoctrue
          \includeonly{\childdocname}
          \def\childdocjob{#1}
          \def\jobname{#1}
        }
      \fi
      \expandafter
    \endgroup
    \childdoctmp
  \fi
}
%    \end{macrocode}

% \macro{\childdocof}
% The command |\childdocof| redirects
% compilation to the main file |#1|.
%    \begin{macrocode}
\newcommand{\childdocof}[1]
{
  \childdocdisable
  \childdoctrue
  \includeonly{\childdocname}
  \def\jobname{#1}
  \def\childdocjob{#1}
  \input{#1}
}
%    \end{macrocode}

% \macro{\childdocby}
% The command |\childdocby| ....
%    \begin{macrocode}
\newcommand{\childdocby}[2][]
{
  \childdocdisable
  \childdoctrue
  \childdocmanualtrue
  \if?#1?\else
    \def\jobname{#2}
  \fi
  \def\childdocjob{#2}
  \input{#2}
  \endinput
}
%    \end{macrocode}

% \macro{\childdocforward}
% The command |\childdocforward| redirects
% compilation to the main file or
% (if the optional argument is given) a child file.
% Parameters are set as if the main file
% or a child file starting with |\childdocof| was compiled.
% Then compilation is handed over to the main file:
%    \begin{macrocode}
\newcommand{\childdocforward}[2][]
{
  \begingroup
    \if?#1?
      \def\childdoctmp
      {
        \def\childdocname{#2}
        \def\childdocjob{#2}
        \def\jobname{#2}
        \input{#2}
        \endinput
      }
    \else
      \def\childdoctmp
      {
        \childdocdisable
        \def\childdocname{#2}
        \childdoctrue
        \includeonly{#2}
        \def\childdocjob{#1}
        \def\jobname{#1}
        \input{#1}
        \endinput
      }
    \fi
    \expandafter
  \endgroup
  \childdoctmp
}
%    \end{macrocode}

% \macro{\childdocforwardprefix}
% The command |\childdocforwardprefix| redirects
% compilation to the main or a child file by means of a pattern.
% The prefix |#1| in the current filename is replaced by |#2|
% and the suffix of the current filename is kept
% (it is assumed that the filename does not contain the substring `|~~~|'
% which is used as a delimiter).
% Compilation is handed over to the new file by |\childdocforward|:
%    \begin{macrocode}
\newcommand{\childdocforwardprefix}[3][]
{
  \begingroup
    \def\childdocextract #2##1~~~{\def\childdoctmp{\childdocforward[#1]{#3##1}}}
    \expandafter\childdocextract\childdocname~~~
    \expandafter
  \endgroup
  \childdoctmp
}
%    \end{macrocode}

% \macro{\childdoc}
% The deprecated macro |\childdoc| is a legacy version of |\childdocmain|:
%    \begin{macrocode}
\newcommand{\childdoc}{\childdocmain}
%    \end{macrocode}

% \macro{\childdocredirect}
% The deprecated macro |\childdocredirect| is a legacy version
% of |\childdocforward| and |\childdocforwardprefix|:
%    \begin{macrocode}
\newcommand{\childdocredirect}[2][]
{
  \begingroup
    \if?#1?
      \def\childdoctmp{\childdocforward{#2}}
    \else
      \def\childdoctmp{\childdocforwardprefix{#1}{#2}}
    \fi
    \expandafter
  \endgroup
  \childdoctmp
}
%    \end{macrocode}

%\iffalse
%</package>
%\fi
%
\endinput
|\\
|\childdocforward{|\textit{main}|}|
\end{tabular}
\end{center}
%
Likewise, the following files |final|\textit{nn}|.tex|
compile the final version of the child document
|child|\textit{nn}|.tex|:
%
\begin{center}
\begin{tabular}{l}
|\def\version{final}|\\
|% \iffalse
%
% childdoc.dtx Copyright (C) 2017-2018 Niklas Beisert
%
% This work may be distributed and/or modified under the
% conditions of the LaTeX Project Public License, either version 1.3
% of this license or (at your option) any later version.
% The latest version of this license is in
%   http://www.latex-project.org/lppl.txt
% and version 1.3 or later is part of all distributions of LaTeX
% version 2005/12/01 or later.
%
% This work has the LPPL maintenance status `maintained'.
%
% The Current Maintainer of this work is Niklas Beisert.
%
% This work consists of the files childdoc.dtx and childdoc.ins
% and the derived files childdoc.def and cdocsamp.tex with
% cdocsch1.tex, cdocsch2.tex, cdocsdrf.tex, cdocsfn1.tex, cdocsfn2.tex.
%
%<package>\ifdefined\childdocmain\endinput\fi
%<package>\ProvidesFile{childdoc.def}[2018/12/30 v2.0 child document driver]
%<samplemain>\ProvidesFile{cdocsamp.tex}[2018/12/30 v2.0 sample for childdoc]
%<*driver>
%\ProvidesFile{childdoc.drv}[2018/12/30 v2.0 childdoc reference manual file]
\PassOptionsToClass{10pt,a4paper}{article}
\documentclass{ltxdoc}

\usepackage[margin=35mm]{geometry}
\usepackage{hyperref}
\usepackage{hyperxmp}
\usepackage[usenames]{color}

\hypersetup{colorlinks=true}
\hypersetup{pdfstartview=FitH}
\hypersetup{pdfpagemode=UseNone}
\hypersetup{pdfsource={}}
\hypersetup{pdflang={en-UK}}
\hypersetup{pdfcopyright={Copyright 2017-2018 Niklas Beisert.
  This work may be distributed and/or modified under the
  conditions of the LaTeX Project Public License, either version 1.3
  of this license or (at your option) any later version.}}
\hypersetup{pdflicenseurl={http://www.latex-project.org/lppl.txt}}
\hypersetup{pdfcontactaddress={ETH Zurich, ITP, HIT K,
  Wolfgang-Pauli-Strasse 27}}
\hypersetup{pdfcontactpostcode={8093}}
\hypersetup{pdfcontactcity={Zurich}}
\hypersetup{pdfcontactcountry={Switzerland}}
\hypersetup{pdfcontactemail={nbeisert@itp.phys.ethz.ch}}
\hypersetup{pdfcontacturl={http://people.phys.ethz.ch/\xmptilde nbeisert/}}

\newcommand{\secref}[1]{\hyperref[#1]{section \ref*{#1}}}

\parskip1ex
\parindent0pt
\let\olditemize\itemize
\def\itemize{\olditemize\parskip0pt}

\begin{document}

\title{The \textsf{childdoc} Package}
\hypersetup{pdftitle={The childdoc Package}}
\author{Niklas Beisert\\[2ex]
  Institut f\"ur Theoretische Physik\\
  Eidgen\"ossische Technische Hochschule Z\"urich\\
  Wolfgang-Pauli-Strasse 27, 8093 Z\"urich, Switzerland\\[1ex]
  \href{mailto:nbeisert@itp.phys.ethz.ch}
  {\texttt{nbeisert@itp.phys.ethz.ch}}}
\hypersetup{pdfauthor={Niklas Beisert}}
\hypersetup{pdfsubject={Manual for the LaTeX2e Package childdoc}}
\date{30 December 2018, \textsf{v2.0}}
\maketitle

\begin{abstract}\noindent
\textsf{childdoc} is a \LaTeXe{} package
that enables the direct compilation
of document sections included by |\include|
to individual files.
\end{abstract}

\begingroup
\parskip0ex
\tableofcontents
\endgroup

%%%%%%%%%%%%%%%%%%%%%%%%%%%%%%%%%%%%%%%%%%%%%%%%%%%%%%%%%%%%%%%%%%%%%%%%%%%%%%%%
%%%%%%%%%%%%%%%%%%%%%%%%%%%%%%%%%%%%%%%%%%%%%%%%%%%%%%%%%%%%%%%%%%%%%%%%%%%%%%%%
\section{Introduction}

\LaTeX{} provides a mechanism to structure a large document (such as a book)
into a main file and several child files (containing the chapters)
using the |\include| command.
This mechanism is beneficial for documents
which span hundreds of pages in order to
make the source file(s) more manageable.
Moreover, compilation can be restricted to
selected child files by means of the |\includeonly| command.
The latter feature can be used to reduce the compilation time while editing
(this was significantly more useful in the earlier days of \LaTeX{})
or to generate a smaller document which is easier to navigate.
Another application of |\includeonly| is to generate
documents consisting of selected parts of the complete document.

However, there are a few drawbacks of the plain |\include| mechanism:
\begin{itemize}
\item
The child files cannot be compiled on their own,
they can only be compiled via the main file.
A naive editing environment
(such as a text editor with an option
to have the current file processed by \LaTeX)
may require one to switch to the main file before compiling;
attempting to compile the child file produces errors.
\item
The main file must be modified (each time)
to adjust the |\includeonly| command
to the present needs. This easily leaves the main file in a messy state.
\item
The generated document will always carry the filename
of the main document. This is inconvenient if
several child files are to be compiled and
to be kept for distribution.
\end{itemize}

The present package provides a simple interface
to make child files individually compilable by \LaTeX{}.
Compiling a child file then has the same effect as compiling
the main file with an |\includeonly| command
to select the appropriate child.
Moreover the generated document will carry the name of the child
rather than the main file.
This resolves all three above issues.

This feature is meant to make the editing of books,
thesis documents and lecture notes somewhat more convenient.
However, the package can also be used efficiently for
composing a series of documents (such as exercise sheets)
which are typically distributed individually.
It then assists the author in generating the individual documents
(potentially in different versions)
as well as a document containing the collected series.
Another application is in developing style files
or other kinds of included material
where compilation of the style file could redirect
to a sample or test file.

%%%%%%%%%%%%%%%%%%%%%%%%%%%%%%%%%%%%%%%%%%%%%%%%%%%%%%%%%%%%%%%%%%%%%%%%%%%%%%%%
%%%%%%%%%%%%%%%%%%%%%%%%%%%%%%%%%%%%%%%%%%%%%%%%%%%%%%%%%%%%%%%%%%%%%%%%%%%%%%%%
\section{Usage}

First of all, the package \textsf{childdoc} is \emph{not} a standard
\LaTeXe{} |.sty| style file! Therefore it needs to be invoked in
a non-standard way.

%%%%%%%%%%%%%%%%%%%%%%%%%%%%%%%%%%%%%%%%%%%%%%%%%%%%%%%%%%%%%%%%%%%%%%%%%%%%%%%%
\subsection{Included Files}
\label{sec:include}

%%%%%%%%%%%%%%%%%%%%%%%%%%%%%%%%%%%%%%%%
\DescribeMacro{\childdocmain}
To use the package, add the commands
\begin{center}
\begin{tabular}{l}
|\input{childdoc.def}|\\
|\childdocmain{}|\\
\end{tabular}
\end{center}
at the very top of the main \LaTeX{} file,
in particular \emph{before} the |\documentclass| statement!
The argument of |\childdocmain| should be left empty
(but it must be present).

%%%%%%%%%%%%%%%%%%%%%%%%%%%%%%%%%%%%%%%%
\DescribeMacro{\childdocof}
Furthermore, add the commands
\begin{center}
\begin{tabular}{l}
|\input{childdoc.def}|\\
|\childdocof{|\textit{main}|}|\\
\end{tabular}
\end{center}
at the top of every child file \textit{child}
which is included by |\include{|\textit{child}|}|
from within the main file
(or at least for those files to be compiled individually).
The argument \textit{main} must be the filename of the main file.

There are a couple of
considerations in setting up the main and child documents:

%%%%%%%%%%%%%%%%%%%%%%%%%%%%%%%%%%%%%%%%
\paragraph{Restrictions.}

Please note the following restrictions:
\begin{itemize}
\item
|\childdocmain| must be called with one argument \textit{main}
to ensure compatibility with earlier version of the package.
It must either be empty (|\childdocmain{}|)
or precisely match the filename of the main file in which it is specified.
See \secref{sec:detection} for further information.
\item
The filename \textit{main} must be specified without the |.tex| extension.
\item
The filename \textit{main} is case sensitive
(even in case-insensitive file systems)
due to internal string comparison.
\item
The argument \textit{main} should be fully expanded, it cannot be a macro.
\item
Subdirectories and special characters should be avoided in filenames.
\item
The command |\childdocmain{|\textit{main}|}| must be followed by a whitespace.
It should not be followed immediately by another command
or by a comment mark `|%|'.
This is because the \TeX{} parser reads the token immediately following
the argument of |\childdocmain| and puts it
at the beginning of every child section;
however, a white\-space is ignored.
\end{itemize}

%%%%%%%%%%%%%%%%%%%%%%%%%%%%%%%%%%%%%%%%
\paragraph{Content of Main File.}

It is advisable to place all content in the child files included by |\include|.
Any output contained in the main file will appear in all child documents
unless suppressed manually;
it cannot be suppressed automatically by the |\includeonly| directive
and thus should normally be avoided.
A method to include some content in the main file
by means of conditional processing is described in \secref{sec:conditional}.

%%%%%%%%%%%%%%%%%%%%%%%%%%%%%%%%%%%%%%%%
\paragraph{Page Numbering.}

When only a part of the document is compiled,
the appropriate numbering of pages
(as well as other status parameters)
is determined from the |.aux| files.
The latter contain information from previous passes.
However this information needs to propagate through
all intermediate child documents.
Therefore the page numbering in child documents may well
be inconsistent until the complete document is compiled at least once.

A useful (if unconventional) way to always ensure a consistent
page numbering is to restart the numbering in each child document
and denote the pages by `\textit{child}|.|\textit{page}'
where \textit{child} represents the chapter/section number of the child file.
This can be achieved by the command
|\numberwithin{page}{|\textit{child}|}|
of the \textsf{amsmath} package
where \textit{child} can be |chapter| or |section|
depending on the chosen structuring.
Alternatively, one can modify the macro |\thepage| appropriately
and reset the counter |page| at the start of each child file.

%%%%%%%%%%%%%%%%%%%%%%%%%%%%%%%%%%%%%%%%%%%%%%%%%%%%%%%%%%%%%%%%%%%%%%%%%%%%%%%%
\subsection{Conditional Processing}
\label{sec:conditional}

The package provides a mechanism to compile different versions
of a document. To customise the versions further some conditional processing
can come in handy to distinguish which version is being compiled.
The package provides two macros to describe the compilation context:

%%%%%%%%%%%%%%%%%%%%%%%%%%%%%%%%%%%%%%%%
\DescribeMacro{\ifchilddoc}
The conditional |\ifchilddoc| distinguishes between the compilation of
child documents and the main document:
%
\begin{center}
|\ifchilddoc |\textit{child-code}| |[|\||else |\textit{main-code}]| \||fi|
\end{center}

%%%%%%%%%%%%%%%%%%%%%%%%%%%%%%%%%%%%%%%%
\DescribeMacro{\childdocname}
\DescribeMacro{\childdocjob}
The macro |\childdocname| contains the filename (without extension)
of the main or child file being processed.
Note that |\childdocjob| will always contain the name of the main file.

%%%%%%%%%%%%%%%%%%%%%%%%%%%%%%%%%%%%%%%%
\paragraph{Title Page.}

Conditional processing can be used to include a title or banner page
in the main document when proper precautions are taken.
Importantly, the code in the main file should ensure that the page counter
(as well as other status parameters which are stored in the |.aux| files)
takes the same value after the conditional processing.
Otherwise the page numbers may take divergent values
depending on which part is compiled.

For example, a title page could be declared by:
%
\begin{center}
\begin{tabular}{l}
|\ifchilddoc\||else|\\
|\addtocounter{page}{-1}|\\
\textit{code for title page}\\
|\newpage|\\
|\||fi|
\end{tabular}
\end{center}
%
A banner page for the child documents can be generated by:
%
\begin{center}
\begin{tabular}{l}
|\ifchilddoc|\\
|\addtocounter{page}{-1}|\\
\textit{code for banner page}\\
|\newpage|\\
|\||fi|
\end{tabular}
\end{center}
%
Here one could write a message such as:
\begin{center}
|This is the part \childdocname{} of \childdocjob{}.|
\end{center}

%%%%%%%%%%%%%%%%%%%%%%%%%%%%%%%%%%%%%%%%%%%%%%%%%%%%%%%%%%%%%%%%%%%%%%%%%%%%%%%%
\subsection{Flags}
\label{sec:flags}

The package makes it easy to generate different versions
of the main or child documents.
To this end compilation flags can be defined
and assigned different default values.
They will be particularly useful in conjunction
with the forwarding mechanism described in \secref{sec:forward}.

For example, it may be useful to have a flag |\version|
which can be set to |draft| or |final|.
The document source will contain some conditional code
depending on the value of |\version|.
Suppose further, the flag should default to |final| for the main file
and to |draft| for child files
which is a natural assignment for editing the document.
This is achieved by placing the following code
in the preamble of the main document
(below the |\childdocmain| directive):
%
\begin{center}
\begin{tabular}{l}
|\ifchilddoc|\\
|\providecommand{\version}{draft}|\\
|\||else|\\
|\providecommand{\version}{final}|\\
|\||fi|
\end{tabular}
\end{center}
%
The definition by |\providecommand| makes sure
that previous definitions are not overwritten.
Further statements |\providecommand{\version}{...}|
can thus be added before the above code to override it.

For the main file, one might add a line
(between |\childdocmain| and the above block)
%
\begin{center}
|%\ifchilddoc\||else\providecommand{\version}{draft}\||fi|
\end{center}
%
which can be uncommented to produce a draft version.
Likewise one can add a line to the very top of a child file
(above the |\childdocof{|\textit{main}|}| directive)
%
\begin{center}
|%\providecommand{\version}{final}|
\end{center}
%
which can be uncommented to produce the final version of this child document.

%%%%%%%%%%%%%%%%%%%%%%%%%%%%%%%%%%%%%%%%%%%%%%%%%%%%%%%%%%%%%%%%%%%%%%%%%%%%%%%%
\subsection{Forwarding}
\label{sec:forward}

Different versions of the main or child documents
using compilation flags as described in \secref{sec:flags}
can be (permanently) stored in different files
for convenient compilation, viewing and distribution.
To this end, the package defines a command
to pass on compilation to a different file:

%%%%%%%%%%%%%%%%%%%%%%%%%%%%%%%%%%%%%%%%
\DescribeMacro{\childdocforward}
The command |\childdocforward| redirects processing to
another source file:
%
\begin{center}
\begin{tabular}{l}
|\input{childdoc.def}|\\
|\childdocforward[|\textit{main}|]{|\textit{dest}|}|\\
\end{tabular}
\end{center}
%
The argument \textit{dest} is the destination file
(without extension).
It should be the main file or one of the child files.
Note that further \textsf{childdoc} directives
such as |\childdocof| and |\childdocforward|
in the indicated file will be processed in this form.
The optional argument \textit{main}
passes on directly to the main file \textit{main}
while pretending to compile the child \textit{dest}.
This form behaves as if \textit{dest}
issues |\childdocof{|\textit{main}|}| right away,
and no further \textsf{childdoc} directives will be processed.

%%%%%%%%%%%%%%%%%%%%%%%%%%%%%%%%%%%%%%%%
\DescribeMacro{\...prefix}
In the alternative form |\childdocforwardprefix|,
%
\begin{center}
\begin{tabular}{l}
|\input{childdoc.def}|\\
|\childdocforwardprefix[|\textit{main}|]{|\textit{prefix}|}{|\textit{dest}|}|
\end{tabular}
\end{center}
%
the destination file is determined by a pattern
depending on the current file:
To make this work, the current file must be called
`{\textit{prefix}\hspace{0.2em}\textit{suffix}}'
with \textit{prefix} matching precisely the argument.
Processing is then passed on to the file
`{\textit{dest}\hspace{0.2em}\textit{suffix}}'.
Surely, the same effect is achieved by
directly specifying the
argument `{\textit{dest}\hspace{0.2em}\textit{suffix}}'
in the first form.
However, that requires to set up a different file
for each child. With the alternative form of the command
all these files can have exactly the same content
which simplifies setting them up and maintaining them.

For example, the following file |draft.tex|
with a compilation flag |\version| as described in \secref{sec:flags}
compiles the main document as a draft:
%
\begin{center}
\begin{tabular}{l}
|\def\version{draft}|\\
|\input{childdoc.def}|\\
|\childdocforward{|\textit{main}|}|
\end{tabular}
\end{center}
%
Likewise, the following files |final|\textit{nn}|.tex|
compile the final version of the child document
|child|\textit{nn}|.tex|:
%
\begin{center}
\begin{tabular}{l}
|\def\version{final}|\\
|\input{childdoc.def}|\\
|\childdocforwardprefix{final}{child}|
\end{tabular}
\end{center}
%

Note that when several versions of a main file and/or of each child file
are to be generated, it may be convenient to set up a |Makefile| or
shell script to automatise the process.

%%%%%%%%%%%%%%%%%%%%%%%%%%%%%%%%%%%%%%%%%%%%%%%%%%%%%%%%%%%%%%%%%%%%%%%%%%%%%%%%
\subsection{Command Line Processing}
\label{sec:commandline}

The effect of redirection files can also be achieved by invoking
the \LaTeX{} compiler with a more elaborate command line.
Most conveniently this should be done as part
of a shell script or a |Makefile|.

When using \textsf{childdoc} in the main file, the following
command lines effectively perform a redirection
(note that depending on the shell being used,
backslashes may have to be doubled: `|\|' $\to$ `|\\|'):
%
\begin{center}
|... -jobname "|\textit{target}|" |\\|"|[\textit{flags}]%
|\input{childdoc.def}\childdocforward[|\textit{main}|]{|\textit{dest}|}"|
\end{center}
%
Here \textit{target} is the name of the output file,
\textit{main} is the name of the main file
and \textit{dest} is the name of the main or child file to be processed
(all filenames without extensions).
The optional argument \textit{main} can be omitted
if \textit{main} matches \textit{dest}.
Optionally, compilation \textit{flags} can be defined via |\def| commands.
This command line makes the \TeX{} engine believe
it is compiling the file \textit{target}
whose content is specified as the latter parameter.
The provided code then forwards the processing to
\textit{main} or \textit{dest} as described in \secref{sec:forward}.

%%%%%%%%%%%%%%%%%%%%%%%%%%%%%%%%%%%%%%%%%%%%%%%%%%%%%%%%%%%%%%%%%%%%%%%%%%%%%%%%
\subsection{Include by Input}
\label{sec:input}

Including child documents by |\include| has some restrictions by design.
Most notably, the content of a child document always occupies
its own set of pages; pages cannot be shared between child documents.
Usually, this behaviour makes perfect sense
because each child document contain an essential part of the document.
However, in some situations it may be desirable to compose
a document from a collection of parts
without having mandatory page breaks between then.
For this case, the package
provides a mechanism to include parts
by |\input| which can also be processed individually.
However, by construction this mechanism
requires manual handling of the content to be output.

%%%%%%%%%%%%%%%%%%%%%%%%%%%%%%%%%%%%%%%%
\DescribeMacro{\ifchilddocmanual}
The main file should be prepared as usual, see \secref{sec:include}.
However, the document body must make a distinction
between processing of an individual part and of the main document, e.g.:
%
\begin{center}
\begin{tabular}{l}
|\ifchilddocmanual|\\
|\input{\childdocname}|\\
|\||else|\\
\textit{document body with }|\input{|\textit{part}|}|\\
|\||fi|
\end{tabular}
\end{center}
%
The conditional |\ifchilddocmanual| is true whenever
a part to be included by |\input| is being compiled,
and the name of the part is stored in |\childdocname|.

%%%%%%%%%%%%%%%%%%%%%%%%%%%%%%%%%%%%%%%%
\DescribeMacro{\childdocby}
Each part to be included by |\input| should start with:
%
\begin{center}
\begin{tabular}{l}
|\input{childdoc.def}|\\
|\childdocby{|\textit{main}|}|\\
\end{tabular}
\end{center}
%
The directive |\childdocby| is similar to |\childdocof|
described in \secref{sec:include},
but the subsequent selection of content must be done manually.
To that end, both |\ifchilddoc| and |\ifchilddocmanual|
will be true upon processing of a part,
and the name of the part is stored in |\childdocname|.
Note that |\jobname| will be set to the filename of the current part
so that each part receives an individual |.aux| file
that does not interfere with the |.aux| file(s) of the main document.
This behaviour can be altered by the alternative form
|\childdocby[*]{|\textit{main}|}| (with a non-empty optional argument)
which uses the |.aux| file of the main document
by setting |\jobname| to \textit{main}.

%%%%%%%%%%%%%%%%%%%%%%%%%%%%%%%%%%%%%%%%%%%%%%%%%%%%%%%%%%%%%%%%%%%%%%%%%%%%%%%%
\subsection{Driver Development}
\label{sec:driver}

The \textsf{childdoc} mechanism can also be use for the development
of definition files such as \LaTeX{} styles or classes.
This case differs from the above setup with multiple parts
included by |\include| in that no |\includeonly| should be invoked.
This can be achieved by starting the include file
(before |\ProvidesPackage|) with:
%
\begin{center}
\begin{tabular}{l}
|\input{childdoc.def}|\\
|\childdocforward{|\textit{main}|}|\\
\end{tabular}
\end{center}
%
or alternatively with:
%
\begin{center}
\begin{tabular}{l}
|\input{childdoc.def}|\\
|\childdocby{|\textit{main}|}|\\
\end{tabular}
\end{center}
%
Both forms have slightly different effects as described above.
The main file is prepared as usual, see \secref{sec:include}.

%%%%%%%%%%%%%%%%%%%%%%%%%%%%%%%%%%%%%%%%%%%%%%%%%%%%%%%%%%%%%%%%%%%%%%%%%%%%%%%%
\subsection{Legacy Detection}
\label{sec:detection}

The directive |\childdocmain| in the main file can detect
whether the complete document or merely a child is to be compiled
even without using the directive |\childdocof|.
This method is deprecated because it is less robust
and there is no compelling reason to use it;
it is merely provided for backward compatibility
and it may be removed in future versions.

If the detection mechanism is to be used,
it is mandatory to correctly specify
the filename of the main file as the argument of |\childdocmain|:
%
\begin{center}
\begin{tabular}{l}
|\input{childdoc.def}|\\
|\childdocmain{|\textit{main}|}|\\
\end{tabular}
\end{center}
%
If |\jobname| does not match the argument \textit{main} of |\childdocmain|,
it is assumed that |\jobname| points to the child file to be compiled.
When using |\childdocmain| with the main file specified as argument,
it suffices to start a child file
with just |\input{|\textit{main}|}|
without loading of the package and using |\childdocof|.
If instead all processing is done
with the appropriate \textsf{childdoc} directives,
the argument of \textit{main} of |\childdocmain| can be empty.

An alternative version of the command line processing described
in \secref{sec:commandline} using the detection mechanism reads:
%
\begin{center}
|... -jobname "|\textit{target}|" "|[\textit{flags}]%
[|\def\jobname{|\textit{dest}|}|]|\input{|\textit{main}|}"|
\end{center}

%%%%%%%%%%%%%%%%%%%%%%%%%%%%%%%%%%%%%%%%%%%%%%%%%%%%%%%%%%%%%%%%%%%%%%%%%%%%%%%%
\subsection{Manual Code}
\label{sec:manual}

In case one cannot be certain whether the definitions file |childdoc.def|
is installed on the target \TeX{} distribution
and one prefers not to ship it,
it is conceivable to paste a few relevant commands into the sources.

To that end, drop all statements |\input{childdoc.def}|
and perform the replacements as outlined below.
Instead of |\childdocmain{|\textit{main}|}| add the following code
to the top of the main file:
%
\begin{center}
\begin{tabular}{l}
|\||ifdefined\childdocname\endinput\||fi\newif\ifchilddoc|\\
|\edef\childdocname{\scantokens\expandafter{\jobname\noexpand}}|\\
|\def\childdocmain{|\textit{main}|}\||ifx\childdocmain\childdocname\||else|\\
|\childdoctrue\includeonly{\childdocname}\let\jobname\childdocmain\||fi|\\
\end{tabular}
\end{center}
%
Instead of |\childdocof{|\textit{main}|}| just include the main file
at the top of each child file:
%
\begin{center}
|\input{|\textit{main}|}|
\end{center}
%
A simple redirection |\childdocforward{|\textit{dest}|}| is achieved by:
%
\begin{center}
|\def\jobname{|\textit{dest}|}\input{\jobname}|
\end{center}
%
The redirection with prefix
|\childdocforwardprefix[|\textit{prefix}|]{|\textit{dest}|}|
is accomplished by:
%
\begin{center}
\begin{tabular}{l}
|{\edef\jobname{\scantokens\expandafter{\jobname\noexpand}}|\\
|\def\redirectjob |\textit{prefix}|#1~~~{\gdef\jobname{|\textit{dest}|#1}}|\\
|\expandafter\redirectjob\jobname~~~}\input{\jobname}|
\end{tabular}
\end{center}

In an alternative approach,
child documents can be compiled by a specific command line
without additional code or specific definitions:
%
\begin{center}
|... -jobname "|\textit{target}|" "|[\textit{flags}]%
|\includeonly{|\textit{dest}|}\input{|\textit{main}|}"|
\end{center}
%

%%%%%%%%%%%%%%%%%%%%%%%%%%%%%%%%%%%%%%%%%%%%%%%%%%%%%%%%%%%%%%%%%%%%%%%%%%%%%%%%
%%%%%%%%%%%%%%%%%%%%%%%%%%%%%%%%%%%%%%%%%%%%%%%%%%%%%%%%%%%%%%%%%%%%%%%%%%%%%%%%
\section{Information}

%%%%%%%%%%%%%%%%%%%%%%%%%%%%%%%%%%%%%%%%%%%%%%%%%%%%%%%%%%%%%%%%%%%%%%%%%%%%%%%%
\subsection{Copyright}

Copyright \copyright{} 2017--2018 Niklas Beisert

This work may be distributed and/or modified under the
conditions of the \LaTeX{} Project Public License, either version 1.3
of this license or (at your option) any later version.
The latest version of this license is in
  \url{http://www.latex-project.org/lppl.txt}
and version 1.3 or later is part of all distributions of \LaTeX{}
version 2005/12/01 or later.

This work has the LPPL maintenance status `maintained'.

The Current Maintainer of this work is Niklas Beisert.

This work consists of the files |README.txt|, |childdoc.ins| and |childdoc.dtx|
as well as the derived files |childdoc.def|, |cdocsamp.tex|
with |cdocsch1.tex|, |cdocsch2.tex|, |cdocspt3.tex|, |cdocspt4.tex|,
|cdocsdrf.tex|, |cdocsfn1.tex|, |cdocsfn2.tex|
as well as |childdoc.pdf|.

%%%%%%%%%%%%%%%%%%%%%%%%%%%%%%%%%%%%%%%%%%%%%%%%%%%%%%%%%%%%%%%%%%%%%%%%%%%%%%%%
\subsection{Files and Installation}

The package consists of the files:
%
\begin{center}
\begin{tabular}{ll}
    |README.txt|   & readme file \\
    |childdoc.ins| & installation file \\
    |childdoc.dtx| & source file \\
    |childdoc.def| & definition file \\
    |cdocsamp.tex| & sample main file \\
    |cdocsch1.tex| & sample include file \\
    |cdocsch2.tex| & sample include file \\
    |cdocspt3.tex| & sample part file \\
    |cdocspt4.tex| & sample part file \\
    |cdocsdrf.tex| & sample redirection file \\
    |cdocsfn1.tex| & sample redirection file \\
    |cdocsfn2.tex| & sample redirection file \\
    |childdoc.pdf| & manual
\end{tabular}
\end{center}
%
The distribution consists of the files
|README.txt|, |childdoc.ins| and |childdoc.dtx|.
%
\begin{itemize}
\item
Run (pdf)\LaTeX{} on |childdoc.dtx|
to compile the manual |childdoc.pdf| (this file).
\item
Run \LaTeX{} on |childdoc.ins| to create the definitions file |childdoc.def|
and the sample |cdocsamp.tex| with include files
|cdocsch1.tex|, |cdocsch2.tex|, |cdocspt3.tex|, |cdocspt4.tex|,
|cdocsdrf.tex|, |cdocsfn1.tex|, |cdocsfn2.tex|.
Then copy the file |childdoc.def| to an appropriate directory of your \LaTeX{}
distribution, e.g.\ \textit{texmf-root}|/tex/latex/childdoc|.
\end{itemize}

%%%%%%%%%%%%%%%%%%%%%%%%%%%%%%%%%%%%%%%%%%%%%%%%%%%%%%%%%%%%%%%%%%%%%%%%%%%%%%%%
\subsection{Related CTAN Packages}

There are several other packages which offer a similar functionality:
%
\begin{itemize}
\item
The packages
\href{http://ctan.org/pkg/docmute}{\textsf{docmute}},
\href{http://ctan.org/pkg/includex}{\textsf{includex}} and
\href{http://ctan.org/pkg/standalone}{\textsf{standalone}}
provide commands to include only the document body of
a child file thus allowing both files to be compiled individually.
\item
The packages \href{http://ctan.org/pkg/subdocs}{\textsf{subdocs}}
and \href{http://ctan.org/pkg/subfiles}{\textsf{subfiles}}
provide structures in which the main and child documents can be
encapsulated and allowing them to be compiled individually.
The inclusion mechanism is different from the conventional |\include|.
\item
The package \href{http://ctan.org/pkg/combine}{\textsf{combine}}
is an elaborate solution to combine several documents into one.
\end{itemize}
%
See also the CTAN topic \href{http://ctan.org/topic/subdocs}{\textsf{subdocs}}
for further related packages.
The present package differs from the above solutions in that
a document structure constructed with the conventional |\include| mechanism
just needs two extra commands at the top of every file
such that all constituent files can be compiled individually.

%%%%%%%%%%%%%%%%%%%%%%%%%%%%%%%%%%%%%%%%%%%%%%%%%%%%%%%%%%%%%%%%%%%%%%%%%%%%%%%%
%\subsection{Feature Suggestions}
%
%The following is a list of features which may be useful for future
%versions of this package:
%%
%\begin{itemize}
%\item
%\ldots
%\end{itemize}

%%%%%%%%%%%%%%%%%%%%%%%%%%%%%%%%%%%%%%%%%%%%%%%%%%%%%%%%%%%%%%%%%%%%%%%%%%%%%%%%
\subsection{Revision History}

%%%%%%%%%%%%%%%%%%%%%%%%%%%%%%%%%%%%%%%%
\paragraph{v2.0:} 2018/12/30

\begin{itemize}
\item
immediate forward processing
\item
added |\childdocby| mechanism
\item
manual restructured
\end{itemize}

%%%%%%%%%%%%%%%%%%%%%%%%%%%%%%%%%%%%%%%%
\paragraph{v1.6:} 2018/01/17

\begin{itemize}
\item
application for development of include files
\item
corrections to manual
\end{itemize}

%%%%%%%%%%%%%%%%%%%%%%%%%%%%%%%%%%%%%%%%
\paragraph{v1.5:} 2017/05/21

\begin{itemize}
\item
more complete structuring introduced
\item
|\childdocof| introduced
\item
|\childdoc| renamed to |\childdocmain|
\item
|\childredirect| renamed to |\childdocforward| and |\childdocforwardprefix|
and functionality expanded
\end{itemize}

%%%%%%%%%%%%%%%%%%%%%%%%%%%%%%%%%%%%%%%%
\paragraph{v1.0:} 2017/04/27

\begin{itemize}
\item
manual and install package
\item
first version published on CTAN
\end{itemize}

%%%%%%%%%%%%%%%%%%%%%%%%%%%%%%%%%%%%%%%%
\paragraph{v0.6:} 2017/04/26

\begin{itemize}
\item
redirection mechanism added
\end{itemize}

%%%%%%%%%%%%%%%%%%%%%%%%%%%%%%%%%%%%%%%%
\paragraph{v0.5:} 2017/04/26

\begin{itemize}
\item
functionality in definition file
\end{itemize}


%%%%%%%%%%%%%%%%%%%%%%%%%%%%%%%%%%%%%%%%%%%%%%%%%%%%%%%%%%%%%%%%%%%%%%%%%%%%%%%%
%%%%%%%%%%%%%%%%%%%%%%%%%%%%%%%%%%%%%%%%%%%%%%%%%%%%%%%%%%%%%%%%%%%%%%%%%%%%%%%%
%%%%%%%%%%%%%%%%%%%%%%%%%%%%%%%%%%%%%%%%%%%%%%%%%%%%%%%%%%%%%%%%%%%%%%%%%%%%%%%%
\appendix

\settowidth\MacroIndent{\rmfamily\scriptsize 000\ }

 \DocInput{childdoc.dtx}

\end{document}
%</driver>
% \fi
%
% %%%%%%%%%%%%%%%%%%%%%%%%%%%%%%%%%%%%%%%%%%%%%%%%%%%%%%%%%%%%%%%%%%%%%%%%%%%%%%
% %%%%%%%%%%%%%%%%%%%%%%%%%%%%%%%%%%%%%%%%%%%%%%%%%%%%%%%%%%%%%%%%%%%%%%%%%%%%%%
% \section{Sample}
%\iffalse
%<*samplemain>
%\fi
%
% The following presents a sample document
% with two chapters, two parts, a title page,
% a compile flag as well as three forwarding files to set the flag.
% It consists of eight |.tex| files:
% \begin{center}
% \begin{tabular}{ll}
% |cdocsamp.tex|&main file\\
% |cdocsch1.tex|&include file for chapter 1\\
% |cdocsch2.tex|&include file for chapter 2\\
% |cdocspt3.tex|&include file for part 3\\
% |cdocspt4.tex|&include file for part 4\\
% |cdocsdrf.tex|&forwarding file for main file in draft mode\\
% |cdocsfi1.tex|&forwarding file for final version of chapter 1\\
% |cdocsfi2.tex|&forwarding file for final version of chapter 2\\
% \end{tabular}
% \end{center}
% Each of the eight files can be compiled directly by the \LaTeX{} compiler.
%
% %%%%%%%%%%%%%%%%%%%%%%%%%%%%%%%%%%%%%%
% \paragraph{Main File.}
%
% The main file is called |cdocsamp.tex|.
%
% Load the \textsf{childdoc} definitions and
% declare the filename for the main document:
%    \begin{macrocode}
\input{childdoc.def}
\childdocmain{}
%    \end{macrocode}

% Optional override for |\version| flag:
%    \begin{macrocode}
%%\ifchilddoc\else\providecommand{\version}{draft}\fi
%    \end{macrocode}

% Define the default values for the |\version| flag
% (|final| for the main file and |draft| for childs):
%    \begin{macrocode}
\ifchilddoc
\providecommand{\version}{draft}
\else
\providecommand{\version}{final}
\fi
%    \end{macrocode}

% Load the standard document class:
%    \begin{macrocode}
\documentclass[12pt]{article}
%    \end{macrocode}

% Start the document body:
%    \begin{macrocode}
\begin{document}
%    \end{macrocode}

% Declare a title page.
% Print title, part of document being processed and version flag:
%    \begin{macrocode}
\addtocounter{page}{-1}
\begin{center}
{\LARGE\bfseries{}childdoc example\par}
\vspace{1cm}
\ifchilddoc
\ifchilddocmanual part\else chapter\fi:
`\childdocname' of `\childdocjob'\par
\else
main document: `\childdocjob'\par
\fi
version: \version\par
\end{center}
\newpage
%    \end{macrocode}

% Manually include selected file,
% otherwise process as usual:
%    \begin{macrocode}
\ifchilddocmanual
\section*{part `\childdocname'}
\input{\childdocname}
\else
%    \end{macrocode}

% Include the two chapters:
%    \begin{macrocode}
\include{cdocsch1}
\include{cdocsch2}
%    \end{macrocode}

% Include the two parts unless only chapters should be displayed:
%    \begin{macrocode}
\ifchilddoc\else
\section{part three}
\input{cdocspt3}
\section{part four}
\input{cdocspt4}
\fi
%    \end{macrocode}

% Process as usual until here:
%    \begin{macrocode}
\fi
%    \end{macrocode}

% End of document body:
%    \begin{macrocode}
\end{document}
%    \end{macrocode}
%\iffalse
%</samplemain>
%\fi
%
% %%%%%%%%%%%%%%%%%%%%%%%%%%%%%%%%%%%%%%
% \paragraph{Chapter Include Files.}
%
% The include files are called |cdocsch1.tex| and |cdocsch2.tex|.
%
%\iffalse
%<*samplechap1|samplechap2>
%\fi

% Optional override for |\version| flag:
%    \begin{macrocode}
%%\providecommand{\version}{final}
%    \end{macrocode}

% Include the main document:
%    \begin{macrocode}
\input{childdoc.def}
\childdocof{cdocsamp}
%    \end{macrocode}

%\iffalse
%</samplechap1|samplechap2>
%\fi
%
%\iffalse
%<*samplechap1>
%\fi
% Some text for chapter 1:
%    \begin{macrocode}
\section{one}
some text in chapter one
%    \end{macrocode}

%\iffalse
%</samplechap1>
%\fi
% Some text for chapter 2:
%\iffalse
%<*samplechap2>
%\fi
%    \begin{macrocode}
\section{two}
more text in chapter two
%    \end{macrocode}

%\iffalse
%</samplechap2>
%\fi
%
% %%%%%%%%%%%%%%%%%%%%%%%%%%%%%%%%%%%%%%
% \paragraph{Part Include Files.}
%
% The include files are called |cdocspt3.tex| and |cdocspt4.tex|.
%
%\iffalse
%<*samplepart3|samplepart4>
%\fi

% Optional override for |\version| flag:
%    \begin{macrocode}
%%\providecommand{\version}{final}
%    \end{macrocode}

% Include the main document:
%    \begin{macrocode}
\input{childdoc.def}
\childdocby{cdocsamp}
%    \end{macrocode}

%\iffalse
%</samplepart3|samplepart4>
%\fi
%
%\iffalse
%<*samplepart3>
%\fi
% Some text for part 3:
%    \begin{macrocode}
some text in part three
%    \end{macrocode}

%\iffalse
%</samplepart3>
%\fi
% Some text for part 4:
%\iffalse
%<*samplepart4>
%\fi
%    \begin{macrocode}
more text in part four
%    \end{macrocode}

%\iffalse
%</samplepart4>
%\fi
%
% %%%%%%%%%%%%%%%%%%%%%%%%%%%%%%%%%%%%%%
% \paragraph{Forwarding for a Complete Draft.}
%
% The following forwarding file |cdocsdrf.tex|
% compiles the main document in draft mode:
%\iffalse
%<*sampledraft>
%\fi
%    \begin{macrocode}
\def\version{draft}
\input{childdoc.def}
\childdocforward{cdocsamp}
%    \end{macrocode}

%\iffalse
%</sampledraft>
%\fi
%
% %%%%%%%%%%%%%%%%%%%%%%%%%%%%%%%%%%%%%%
% \paragraph{Forwarding for Final Version of the Chapters.}
%
% The following forwarding files |cdocsfn1.tex| and |cdocsfn2.tex|
% (with identical content)
% compile the final versions of the child documents
% |cdocsch1.tex| and |cdocsch2.tex|, respectively:
%\iffalse
%<*samplefinal>
%\fi
%    \begin{macrocode}
\def\version{final}
\input{childdoc.def}
\childdocforwardprefix[cdocsamp]{cdocsfn}{cdocsch}
%    \end{macrocode}

%\iffalse
%</samplefinal>
%\fi
%
% %%%%%%%%%%%%%%%%%%%%%%%%%%%%%%%%%%%%%%
% \paragraph{Command Line Processing.}
%
% The following three command lines generate the output files
% |cdocscld|, |cdocscl1| and |cdocscl2|
% which should be identical to
% |cdocsdrf|, |cdocsch1| and |cdocsfn2|, respectively:
% \begin{center}
% \begin{tabular}{l}
% |latex -jobname cdocscld \|\\
% |  "\def\version{draft}\input{childdoc.def}\childdocforward{cdocsamp}"|\\
% |latex -jobname cdocscl1 \|\\
% |  "\input{childdoc.def}\childdocforward[cdocsamp]{cdocsch1}"|\\
% |latex -jobname cdocscl2 \|\\
% |  "\def\version{final}\input{childdoc.def}\childdocforward{cdocsch2}"|
% \end{tabular}
% \end{center}
% Note that the trailing backslash on each first line
% merely continues the input to the second line
% (for convenient cut ant paste).
% Furthermore, the command |latex| can be replaced by any
% of its alternative versions such as |pdflatex|.
%
% %%%%%%%%%%%%%%%%%%%%%%%%%%%%%%%%%%%%%%%%%%%%%%%%%%%%%%%%%%%%%%%%%%%%%%%%%%%%%%
% %%%%%%%%%%%%%%%%%%%%%%%%%%%%%%%%%%%%%%%%%%%%%%%%%%%%%%%%%%%%%%%%%%%%%%%%%%%%%%
% \section{Implementation}
%\iffalse
%<*package>
%\fi
%
% This section describes the definitions file |childdoc.def|.

% The definitions cannot be loaded using |\usepackage| or |\RequirePackage|
% which has a mechanism to prevent loading a style file more than once.
% When loading the definitions by means of |\input|
% multiple instances have to be prevented manually:
%\iffalse
%This code needs to be before the `\ProvidesFile' directive
%which is defined at the beginning of this file.
%Therefore it is also placed there and commented out here.
%</package>
%<*discard>
%\fi
%    \begin{macrocode}
\ifdefined\childdocmain\endinput\fi
%    \end{macrocode}
%\iffalse
%</discard>
%<*package>
%\fi
%
% \macro{\ifchilddoc}
% \macro{\ifchilddocmanual}
% The conditional |\ifchilddoc| tells whether a
% child (true) or main (false) document is being compiled.
% The conditional |\ifchilddocmanual| tells whether
% the |\includeonly| mechanism is used (false) or
% the selection of child files must be performed manually (true).
% The definitions initialise to false:
%    \begin{macrocode}
\newif\ifchilddoc
\newif\ifchilddocmanual
%    \end{macrocode}

% \macro{\childdocname}
% \macro{\childdocjob}
% The macro |\childdocname| stores the name of the main document
% to be compiled. The macro |\childdocjob| stores the name of
% the document on which the \LaTeX{} compiler was originally invoked.
% The content of |\jobname| cannot be compared
% to filenames specified in the source due to different catcodes.
% The following code rescans |\jobname|, stores the result
% in |\childdocname| and saves a copy in |\childdocjob|:
%    \begin{macrocode}
\edef\childdocname{\scantokens\expandafter{\jobname\noexpand}}
\let\childdocjob\childdocname
%    \end{macrocode}

% \macro{\childdocdisable}
% The macro |\childdocdisable| prevents the main file
% from being processed more than once.
% At this stage, the main document command |\childdocmain|
% is assumed to be called once again where it should do nothing.
% Any subsequent call to it should prevent
% a secondary processing of the main document
% It overwrites the forwarding commands
% |\childdocof| and |\childdocforward|
% with empty macros to prevent further inclusions of the main document:
%    \begin{macrocode}
\newcommand{\childdocdisable}
{
  \renewcommand{\childdocmain}[1]{\renewcommand{\childdocmain}[1]{\endinput}}
  \renewcommand{\childdocof}[1]{}
  \renewcommand{\childdocby}[2][]{}
  \renewcommand{\childdocforward}[2][]{}
  \renewcommand{\childdocdisable}{}
}
%    \end{macrocode}

% \macro{\childdocmain}
% The macro |\childdocmain| is to be called at the top of the main file
% with nothing or the main filename (without extension) as argument.
% First, it breaks loops.
% If the argument is not empty and does not match |\childdocname|
% (which is set by the first inclusion of |childdoc.def|),
% |\ifchilddoc| is set to true, |\includeonly| is applied to the child file
% and |\jobname| is set to the main file
% (for proper handling of |.aux| files):
%    \begin{macrocode}
\newcommand{\childdocmain}[1]
{
  \childdocdisable\childdocmain{}
  \if?#1?\else
    \begingroup
      \def\childdoctmp{#1}
      \ifx\childdoctmp\childdocname
        \def\childdoctmp{}
      \else
        \def\childdoctmp
        {
          \childdoctrue
          \includeonly{\childdocname}
          \def\childdocjob{#1}
          \def\jobname{#1}
        }
      \fi
      \expandafter
    \endgroup
    \childdoctmp
  \fi
}
%    \end{macrocode}

% \macro{\childdocof}
% The command |\childdocof| redirects
% compilation to the main file |#1|.
%    \begin{macrocode}
\newcommand{\childdocof}[1]
{
  \childdocdisable
  \childdoctrue
  \includeonly{\childdocname}
  \def\jobname{#1}
  \def\childdocjob{#1}
  \input{#1}
}
%    \end{macrocode}

% \macro{\childdocby}
% The command |\childdocby| ....
%    \begin{macrocode}
\newcommand{\childdocby}[2][]
{
  \childdocdisable
  \childdoctrue
  \childdocmanualtrue
  \if?#1?\else
    \def\jobname{#2}
  \fi
  \def\childdocjob{#2}
  \input{#2}
  \endinput
}
%    \end{macrocode}

% \macro{\childdocforward}
% The command |\childdocforward| redirects
% compilation to the main file or
% (if the optional argument is given) a child file.
% Parameters are set as if the main file
% or a child file starting with |\childdocof| was compiled.
% Then compilation is handed over to the main file:
%    \begin{macrocode}
\newcommand{\childdocforward}[2][]
{
  \begingroup
    \if?#1?
      \def\childdoctmp
      {
        \def\childdocname{#2}
        \def\childdocjob{#2}
        \def\jobname{#2}
        \input{#2}
        \endinput
      }
    \else
      \def\childdoctmp
      {
        \childdocdisable
        \def\childdocname{#2}
        \childdoctrue
        \includeonly{#2}
        \def\childdocjob{#1}
        \def\jobname{#1}
        \input{#1}
        \endinput
      }
    \fi
    \expandafter
  \endgroup
  \childdoctmp
}
%    \end{macrocode}

% \macro{\childdocforwardprefix}
% The command |\childdocforwardprefix| redirects
% compilation to the main or a child file by means of a pattern.
% The prefix |#1| in the current filename is replaced by |#2|
% and the suffix of the current filename is kept
% (it is assumed that the filename does not contain the substring `|~~~|'
% which is used as a delimiter).
% Compilation is handed over to the new file by |\childdocforward|:
%    \begin{macrocode}
\newcommand{\childdocforwardprefix}[3][]
{
  \begingroup
    \def\childdocextract #2##1~~~{\def\childdoctmp{\childdocforward[#1]{#3##1}}}
    \expandafter\childdocextract\childdocname~~~
    \expandafter
  \endgroup
  \childdoctmp
}
%    \end{macrocode}

% \macro{\childdoc}
% The deprecated macro |\childdoc| is a legacy version of |\childdocmain|:
%    \begin{macrocode}
\newcommand{\childdoc}{\childdocmain}
%    \end{macrocode}

% \macro{\childdocredirect}
% The deprecated macro |\childdocredirect| is a legacy version
% of |\childdocforward| and |\childdocforwardprefix|:
%    \begin{macrocode}
\newcommand{\childdocredirect}[2][]
{
  \begingroup
    \if?#1?
      \def\childdoctmp{\childdocforward{#2}}
    \else
      \def\childdoctmp{\childdocforwardprefix{#1}{#2}}
    \fi
    \expandafter
  \endgroup
  \childdoctmp
}
%    \end{macrocode}

%\iffalse
%</package>
%\fi
%
\endinput
|\\
|\childdocforwardprefix{final}{child}|
\end{tabular}
\end{center}
%

Note that when several versions of a main file and/or of each child file
are to be generated, it may be convenient to set up a |Makefile| or
shell script to automatise the process.

%%%%%%%%%%%%%%%%%%%%%%%%%%%%%%%%%%%%%%%%%%%%%%%%%%%%%%%%%%%%%%%%%%%%%%%%%%%%%%%%
\subsection{Command Line Processing}
\label{sec:commandline}

The effect of redirection files can also be achieved by invoking
the \LaTeX{} compiler with a more elaborate command line.
Most conveniently this should be done as part
of a shell script or a |Makefile|.

When using \textsf{childdoc} in the main file, the following
command lines effectively perform a redirection
(note that depending on the shell being used,
backslashes may have to be doubled: `|\|' $\to$ `|\\|'):
%
\begin{center}
|... -jobname "|\textit{target}|" |\\|"|[\textit{flags}]%
|% \iffalse
%
% childdoc.dtx Copyright (C) 2017-2018 Niklas Beisert
%
% This work may be distributed and/or modified under the
% conditions of the LaTeX Project Public License, either version 1.3
% of this license or (at your option) any later version.
% The latest version of this license is in
%   http://www.latex-project.org/lppl.txt
% and version 1.3 or later is part of all distributions of LaTeX
% version 2005/12/01 or later.
%
% This work has the LPPL maintenance status `maintained'.
%
% The Current Maintainer of this work is Niklas Beisert.
%
% This work consists of the files childdoc.dtx and childdoc.ins
% and the derived files childdoc.def and cdocsamp.tex with
% cdocsch1.tex, cdocsch2.tex, cdocsdrf.tex, cdocsfn1.tex, cdocsfn2.tex.
%
%<package>\ifdefined\childdocmain\endinput\fi
%<package>\ProvidesFile{childdoc.def}[2018/12/30 v2.0 child document driver]
%<samplemain>\ProvidesFile{cdocsamp.tex}[2018/12/30 v2.0 sample for childdoc]
%<*driver>
%\ProvidesFile{childdoc.drv}[2018/12/30 v2.0 childdoc reference manual file]
\PassOptionsToClass{10pt,a4paper}{article}
\documentclass{ltxdoc}

\usepackage[margin=35mm]{geometry}
\usepackage{hyperref}
\usepackage{hyperxmp}
\usepackage[usenames]{color}

\hypersetup{colorlinks=true}
\hypersetup{pdfstartview=FitH}
\hypersetup{pdfpagemode=UseNone}
\hypersetup{pdfsource={}}
\hypersetup{pdflang={en-UK}}
\hypersetup{pdfcopyright={Copyright 2017-2018 Niklas Beisert.
  This work may be distributed and/or modified under the
  conditions of the LaTeX Project Public License, either version 1.3
  of this license or (at your option) any later version.}}
\hypersetup{pdflicenseurl={http://www.latex-project.org/lppl.txt}}
\hypersetup{pdfcontactaddress={ETH Zurich, ITP, HIT K,
  Wolfgang-Pauli-Strasse 27}}
\hypersetup{pdfcontactpostcode={8093}}
\hypersetup{pdfcontactcity={Zurich}}
\hypersetup{pdfcontactcountry={Switzerland}}
\hypersetup{pdfcontactemail={nbeisert@itp.phys.ethz.ch}}
\hypersetup{pdfcontacturl={http://people.phys.ethz.ch/\xmptilde nbeisert/}}

\newcommand{\secref}[1]{\hyperref[#1]{section \ref*{#1}}}

\parskip1ex
\parindent0pt
\let\olditemize\itemize
\def\itemize{\olditemize\parskip0pt}

\begin{document}

\title{The \textsf{childdoc} Package}
\hypersetup{pdftitle={The childdoc Package}}
\author{Niklas Beisert\\[2ex]
  Institut f\"ur Theoretische Physik\\
  Eidgen\"ossische Technische Hochschule Z\"urich\\
  Wolfgang-Pauli-Strasse 27, 8093 Z\"urich, Switzerland\\[1ex]
  \href{mailto:nbeisert@itp.phys.ethz.ch}
  {\texttt{nbeisert@itp.phys.ethz.ch}}}
\hypersetup{pdfauthor={Niklas Beisert}}
\hypersetup{pdfsubject={Manual for the LaTeX2e Package childdoc}}
\date{30 December 2018, \textsf{v2.0}}
\maketitle

\begin{abstract}\noindent
\textsf{childdoc} is a \LaTeXe{} package
that enables the direct compilation
of document sections included by |\include|
to individual files.
\end{abstract}

\begingroup
\parskip0ex
\tableofcontents
\endgroup

%%%%%%%%%%%%%%%%%%%%%%%%%%%%%%%%%%%%%%%%%%%%%%%%%%%%%%%%%%%%%%%%%%%%%%%%%%%%%%%%
%%%%%%%%%%%%%%%%%%%%%%%%%%%%%%%%%%%%%%%%%%%%%%%%%%%%%%%%%%%%%%%%%%%%%%%%%%%%%%%%
\section{Introduction}

\LaTeX{} provides a mechanism to structure a large document (such as a book)
into a main file and several child files (containing the chapters)
using the |\include| command.
This mechanism is beneficial for documents
which span hundreds of pages in order to
make the source file(s) more manageable.
Moreover, compilation can be restricted to
selected child files by means of the |\includeonly| command.
The latter feature can be used to reduce the compilation time while editing
(this was significantly more useful in the earlier days of \LaTeX{})
or to generate a smaller document which is easier to navigate.
Another application of |\includeonly| is to generate
documents consisting of selected parts of the complete document.

However, there are a few drawbacks of the plain |\include| mechanism:
\begin{itemize}
\item
The child files cannot be compiled on their own,
they can only be compiled via the main file.
A naive editing environment
(such as a text editor with an option
to have the current file processed by \LaTeX)
may require one to switch to the main file before compiling;
attempting to compile the child file produces errors.
\item
The main file must be modified (each time)
to adjust the |\includeonly| command
to the present needs. This easily leaves the main file in a messy state.
\item
The generated document will always carry the filename
of the main document. This is inconvenient if
several child files are to be compiled and
to be kept for distribution.
\end{itemize}

The present package provides a simple interface
to make child files individually compilable by \LaTeX{}.
Compiling a child file then has the same effect as compiling
the main file with an |\includeonly| command
to select the appropriate child.
Moreover the generated document will carry the name of the child
rather than the main file.
This resolves all three above issues.

This feature is meant to make the editing of books,
thesis documents and lecture notes somewhat more convenient.
However, the package can also be used efficiently for
composing a series of documents (such as exercise sheets)
which are typically distributed individually.
It then assists the author in generating the individual documents
(potentially in different versions)
as well as a document containing the collected series.
Another application is in developing style files
or other kinds of included material
where compilation of the style file could redirect
to a sample or test file.

%%%%%%%%%%%%%%%%%%%%%%%%%%%%%%%%%%%%%%%%%%%%%%%%%%%%%%%%%%%%%%%%%%%%%%%%%%%%%%%%
%%%%%%%%%%%%%%%%%%%%%%%%%%%%%%%%%%%%%%%%%%%%%%%%%%%%%%%%%%%%%%%%%%%%%%%%%%%%%%%%
\section{Usage}

First of all, the package \textsf{childdoc} is \emph{not} a standard
\LaTeXe{} |.sty| style file! Therefore it needs to be invoked in
a non-standard way.

%%%%%%%%%%%%%%%%%%%%%%%%%%%%%%%%%%%%%%%%%%%%%%%%%%%%%%%%%%%%%%%%%%%%%%%%%%%%%%%%
\subsection{Included Files}
\label{sec:include}

%%%%%%%%%%%%%%%%%%%%%%%%%%%%%%%%%%%%%%%%
\DescribeMacro{\childdocmain}
To use the package, add the commands
\begin{center}
\begin{tabular}{l}
|\input{childdoc.def}|\\
|\childdocmain{}|\\
\end{tabular}
\end{center}
at the very top of the main \LaTeX{} file,
in particular \emph{before} the |\documentclass| statement!
The argument of |\childdocmain| should be left empty
(but it must be present).

%%%%%%%%%%%%%%%%%%%%%%%%%%%%%%%%%%%%%%%%
\DescribeMacro{\childdocof}
Furthermore, add the commands
\begin{center}
\begin{tabular}{l}
|\input{childdoc.def}|\\
|\childdocof{|\textit{main}|}|\\
\end{tabular}
\end{center}
at the top of every child file \textit{child}
which is included by |\include{|\textit{child}|}|
from within the main file
(or at least for those files to be compiled individually).
The argument \textit{main} must be the filename of the main file.

There are a couple of
considerations in setting up the main and child documents:

%%%%%%%%%%%%%%%%%%%%%%%%%%%%%%%%%%%%%%%%
\paragraph{Restrictions.}

Please note the following restrictions:
\begin{itemize}
\item
|\childdocmain| must be called with one argument \textit{main}
to ensure compatibility with earlier version of the package.
It must either be empty (|\childdocmain{}|)
or precisely match the filename of the main file in which it is specified.
See \secref{sec:detection} for further information.
\item
The filename \textit{main} must be specified without the |.tex| extension.
\item
The filename \textit{main} is case sensitive
(even in case-insensitive file systems)
due to internal string comparison.
\item
The argument \textit{main} should be fully expanded, it cannot be a macro.
\item
Subdirectories and special characters should be avoided in filenames.
\item
The command |\childdocmain{|\textit{main}|}| must be followed by a whitespace.
It should not be followed immediately by another command
or by a comment mark `|%|'.
This is because the \TeX{} parser reads the token immediately following
the argument of |\childdocmain| and puts it
at the beginning of every child section;
however, a white\-space is ignored.
\end{itemize}

%%%%%%%%%%%%%%%%%%%%%%%%%%%%%%%%%%%%%%%%
\paragraph{Content of Main File.}

It is advisable to place all content in the child files included by |\include|.
Any output contained in the main file will appear in all child documents
unless suppressed manually;
it cannot be suppressed automatically by the |\includeonly| directive
and thus should normally be avoided.
A method to include some content in the main file
by means of conditional processing is described in \secref{sec:conditional}.

%%%%%%%%%%%%%%%%%%%%%%%%%%%%%%%%%%%%%%%%
\paragraph{Page Numbering.}

When only a part of the document is compiled,
the appropriate numbering of pages
(as well as other status parameters)
is determined from the |.aux| files.
The latter contain information from previous passes.
However this information needs to propagate through
all intermediate child documents.
Therefore the page numbering in child documents may well
be inconsistent until the complete document is compiled at least once.

A useful (if unconventional) way to always ensure a consistent
page numbering is to restart the numbering in each child document
and denote the pages by `\textit{child}|.|\textit{page}'
where \textit{child} represents the chapter/section number of the child file.
This can be achieved by the command
|\numberwithin{page}{|\textit{child}|}|
of the \textsf{amsmath} package
where \textit{child} can be |chapter| or |section|
depending on the chosen structuring.
Alternatively, one can modify the macro |\thepage| appropriately
and reset the counter |page| at the start of each child file.

%%%%%%%%%%%%%%%%%%%%%%%%%%%%%%%%%%%%%%%%%%%%%%%%%%%%%%%%%%%%%%%%%%%%%%%%%%%%%%%%
\subsection{Conditional Processing}
\label{sec:conditional}

The package provides a mechanism to compile different versions
of a document. To customise the versions further some conditional processing
can come in handy to distinguish which version is being compiled.
The package provides two macros to describe the compilation context:

%%%%%%%%%%%%%%%%%%%%%%%%%%%%%%%%%%%%%%%%
\DescribeMacro{\ifchilddoc}
The conditional |\ifchilddoc| distinguishes between the compilation of
child documents and the main document:
%
\begin{center}
|\ifchilddoc |\textit{child-code}| |[|\||else |\textit{main-code}]| \||fi|
\end{center}

%%%%%%%%%%%%%%%%%%%%%%%%%%%%%%%%%%%%%%%%
\DescribeMacro{\childdocname}
\DescribeMacro{\childdocjob}
The macro |\childdocname| contains the filename (without extension)
of the main or child file being processed.
Note that |\childdocjob| will always contain the name of the main file.

%%%%%%%%%%%%%%%%%%%%%%%%%%%%%%%%%%%%%%%%
\paragraph{Title Page.}

Conditional processing can be used to include a title or banner page
in the main document when proper precautions are taken.
Importantly, the code in the main file should ensure that the page counter
(as well as other status parameters which are stored in the |.aux| files)
takes the same value after the conditional processing.
Otherwise the page numbers may take divergent values
depending on which part is compiled.

For example, a title page could be declared by:
%
\begin{center}
\begin{tabular}{l}
|\ifchilddoc\||else|\\
|\addtocounter{page}{-1}|\\
\textit{code for title page}\\
|\newpage|\\
|\||fi|
\end{tabular}
\end{center}
%
A banner page for the child documents can be generated by:
%
\begin{center}
\begin{tabular}{l}
|\ifchilddoc|\\
|\addtocounter{page}{-1}|\\
\textit{code for banner page}\\
|\newpage|\\
|\||fi|
\end{tabular}
\end{center}
%
Here one could write a message such as:
\begin{center}
|This is the part \childdocname{} of \childdocjob{}.|
\end{center}

%%%%%%%%%%%%%%%%%%%%%%%%%%%%%%%%%%%%%%%%%%%%%%%%%%%%%%%%%%%%%%%%%%%%%%%%%%%%%%%%
\subsection{Flags}
\label{sec:flags}

The package makes it easy to generate different versions
of the main or child documents.
To this end compilation flags can be defined
and assigned different default values.
They will be particularly useful in conjunction
with the forwarding mechanism described in \secref{sec:forward}.

For example, it may be useful to have a flag |\version|
which can be set to |draft| or |final|.
The document source will contain some conditional code
depending on the value of |\version|.
Suppose further, the flag should default to |final| for the main file
and to |draft| for child files
which is a natural assignment for editing the document.
This is achieved by placing the following code
in the preamble of the main document
(below the |\childdocmain| directive):
%
\begin{center}
\begin{tabular}{l}
|\ifchilddoc|\\
|\providecommand{\version}{draft}|\\
|\||else|\\
|\providecommand{\version}{final}|\\
|\||fi|
\end{tabular}
\end{center}
%
The definition by |\providecommand| makes sure
that previous definitions are not overwritten.
Further statements |\providecommand{\version}{...}|
can thus be added before the above code to override it.

For the main file, one might add a line
(between |\childdocmain| and the above block)
%
\begin{center}
|%\ifchilddoc\||else\providecommand{\version}{draft}\||fi|
\end{center}
%
which can be uncommented to produce a draft version.
Likewise one can add a line to the very top of a child file
(above the |\childdocof{|\textit{main}|}| directive)
%
\begin{center}
|%\providecommand{\version}{final}|
\end{center}
%
which can be uncommented to produce the final version of this child document.

%%%%%%%%%%%%%%%%%%%%%%%%%%%%%%%%%%%%%%%%%%%%%%%%%%%%%%%%%%%%%%%%%%%%%%%%%%%%%%%%
\subsection{Forwarding}
\label{sec:forward}

Different versions of the main or child documents
using compilation flags as described in \secref{sec:flags}
can be (permanently) stored in different files
for convenient compilation, viewing and distribution.
To this end, the package defines a command
to pass on compilation to a different file:

%%%%%%%%%%%%%%%%%%%%%%%%%%%%%%%%%%%%%%%%
\DescribeMacro{\childdocforward}
The command |\childdocforward| redirects processing to
another source file:
%
\begin{center}
\begin{tabular}{l}
|\input{childdoc.def}|\\
|\childdocforward[|\textit{main}|]{|\textit{dest}|}|\\
\end{tabular}
\end{center}
%
The argument \textit{dest} is the destination file
(without extension).
It should be the main file or one of the child files.
Note that further \textsf{childdoc} directives
such as |\childdocof| and |\childdocforward|
in the indicated file will be processed in this form.
The optional argument \textit{main}
passes on directly to the main file \textit{main}
while pretending to compile the child \textit{dest}.
This form behaves as if \textit{dest}
issues |\childdocof{|\textit{main}|}| right away,
and no further \textsf{childdoc} directives will be processed.

%%%%%%%%%%%%%%%%%%%%%%%%%%%%%%%%%%%%%%%%
\DescribeMacro{\...prefix}
In the alternative form |\childdocforwardprefix|,
%
\begin{center}
\begin{tabular}{l}
|\input{childdoc.def}|\\
|\childdocforwardprefix[|\textit{main}|]{|\textit{prefix}|}{|\textit{dest}|}|
\end{tabular}
\end{center}
%
the destination file is determined by a pattern
depending on the current file:
To make this work, the current file must be called
`{\textit{prefix}\hspace{0.2em}\textit{suffix}}'
with \textit{prefix} matching precisely the argument.
Processing is then passed on to the file
`{\textit{dest}\hspace{0.2em}\textit{suffix}}'.
Surely, the same effect is achieved by
directly specifying the
argument `{\textit{dest}\hspace{0.2em}\textit{suffix}}'
in the first form.
However, that requires to set up a different file
for each child. With the alternative form of the command
all these files can have exactly the same content
which simplifies setting them up and maintaining them.

For example, the following file |draft.tex|
with a compilation flag |\version| as described in \secref{sec:flags}
compiles the main document as a draft:
%
\begin{center}
\begin{tabular}{l}
|\def\version{draft}|\\
|\input{childdoc.def}|\\
|\childdocforward{|\textit{main}|}|
\end{tabular}
\end{center}
%
Likewise, the following files |final|\textit{nn}|.tex|
compile the final version of the child document
|child|\textit{nn}|.tex|:
%
\begin{center}
\begin{tabular}{l}
|\def\version{final}|\\
|\input{childdoc.def}|\\
|\childdocforwardprefix{final}{child}|
\end{tabular}
\end{center}
%

Note that when several versions of a main file and/or of each child file
are to be generated, it may be convenient to set up a |Makefile| or
shell script to automatise the process.

%%%%%%%%%%%%%%%%%%%%%%%%%%%%%%%%%%%%%%%%%%%%%%%%%%%%%%%%%%%%%%%%%%%%%%%%%%%%%%%%
\subsection{Command Line Processing}
\label{sec:commandline}

The effect of redirection files can also be achieved by invoking
the \LaTeX{} compiler with a more elaborate command line.
Most conveniently this should be done as part
of a shell script or a |Makefile|.

When using \textsf{childdoc} in the main file, the following
command lines effectively perform a redirection
(note that depending on the shell being used,
backslashes may have to be doubled: `|\|' $\to$ `|\\|'):
%
\begin{center}
|... -jobname "|\textit{target}|" |\\|"|[\textit{flags}]%
|\input{childdoc.def}\childdocforward[|\textit{main}|]{|\textit{dest}|}"|
\end{center}
%
Here \textit{target} is the name of the output file,
\textit{main} is the name of the main file
and \textit{dest} is the name of the main or child file to be processed
(all filenames without extensions).
The optional argument \textit{main} can be omitted
if \textit{main} matches \textit{dest}.
Optionally, compilation \textit{flags} can be defined via |\def| commands.
This command line makes the \TeX{} engine believe
it is compiling the file \textit{target}
whose content is specified as the latter parameter.
The provided code then forwards the processing to
\textit{main} or \textit{dest} as described in \secref{sec:forward}.

%%%%%%%%%%%%%%%%%%%%%%%%%%%%%%%%%%%%%%%%%%%%%%%%%%%%%%%%%%%%%%%%%%%%%%%%%%%%%%%%
\subsection{Include by Input}
\label{sec:input}

Including child documents by |\include| has some restrictions by design.
Most notably, the content of a child document always occupies
its own set of pages; pages cannot be shared between child documents.
Usually, this behaviour makes perfect sense
because each child document contain an essential part of the document.
However, in some situations it may be desirable to compose
a document from a collection of parts
without having mandatory page breaks between then.
For this case, the package
provides a mechanism to include parts
by |\input| which can also be processed individually.
However, by construction this mechanism
requires manual handling of the content to be output.

%%%%%%%%%%%%%%%%%%%%%%%%%%%%%%%%%%%%%%%%
\DescribeMacro{\ifchilddocmanual}
The main file should be prepared as usual, see \secref{sec:include}.
However, the document body must make a distinction
between processing of an individual part and of the main document, e.g.:
%
\begin{center}
\begin{tabular}{l}
|\ifchilddocmanual|\\
|\input{\childdocname}|\\
|\||else|\\
\textit{document body with }|\input{|\textit{part}|}|\\
|\||fi|
\end{tabular}
\end{center}
%
The conditional |\ifchilddocmanual| is true whenever
a part to be included by |\input| is being compiled,
and the name of the part is stored in |\childdocname|.

%%%%%%%%%%%%%%%%%%%%%%%%%%%%%%%%%%%%%%%%
\DescribeMacro{\childdocby}
Each part to be included by |\input| should start with:
%
\begin{center}
\begin{tabular}{l}
|\input{childdoc.def}|\\
|\childdocby{|\textit{main}|}|\\
\end{tabular}
\end{center}
%
The directive |\childdocby| is similar to |\childdocof|
described in \secref{sec:include},
but the subsequent selection of content must be done manually.
To that end, both |\ifchilddoc| and |\ifchilddocmanual|
will be true upon processing of a part,
and the name of the part is stored in |\childdocname|.
Note that |\jobname| will be set to the filename of the current part
so that each part receives an individual |.aux| file
that does not interfere with the |.aux| file(s) of the main document.
This behaviour can be altered by the alternative form
|\childdocby[*]{|\textit{main}|}| (with a non-empty optional argument)
which uses the |.aux| file of the main document
by setting |\jobname| to \textit{main}.

%%%%%%%%%%%%%%%%%%%%%%%%%%%%%%%%%%%%%%%%%%%%%%%%%%%%%%%%%%%%%%%%%%%%%%%%%%%%%%%%
\subsection{Driver Development}
\label{sec:driver}

The \textsf{childdoc} mechanism can also be use for the development
of definition files such as \LaTeX{} styles or classes.
This case differs from the above setup with multiple parts
included by |\include| in that no |\includeonly| should be invoked.
This can be achieved by starting the include file
(before |\ProvidesPackage|) with:
%
\begin{center}
\begin{tabular}{l}
|\input{childdoc.def}|\\
|\childdocforward{|\textit{main}|}|\\
\end{tabular}
\end{center}
%
or alternatively with:
%
\begin{center}
\begin{tabular}{l}
|\input{childdoc.def}|\\
|\childdocby{|\textit{main}|}|\\
\end{tabular}
\end{center}
%
Both forms have slightly different effects as described above.
The main file is prepared as usual, see \secref{sec:include}.

%%%%%%%%%%%%%%%%%%%%%%%%%%%%%%%%%%%%%%%%%%%%%%%%%%%%%%%%%%%%%%%%%%%%%%%%%%%%%%%%
\subsection{Legacy Detection}
\label{sec:detection}

The directive |\childdocmain| in the main file can detect
whether the complete document or merely a child is to be compiled
even without using the directive |\childdocof|.
This method is deprecated because it is less robust
and there is no compelling reason to use it;
it is merely provided for backward compatibility
and it may be removed in future versions.

If the detection mechanism is to be used,
it is mandatory to correctly specify
the filename of the main file as the argument of |\childdocmain|:
%
\begin{center}
\begin{tabular}{l}
|\input{childdoc.def}|\\
|\childdocmain{|\textit{main}|}|\\
\end{tabular}
\end{center}
%
If |\jobname| does not match the argument \textit{main} of |\childdocmain|,
it is assumed that |\jobname| points to the child file to be compiled.
When using |\childdocmain| with the main file specified as argument,
it suffices to start a child file
with just |\input{|\textit{main}|}|
without loading of the package and using |\childdocof|.
If instead all processing is done
with the appropriate \textsf{childdoc} directives,
the argument of \textit{main} of |\childdocmain| can be empty.

An alternative version of the command line processing described
in \secref{sec:commandline} using the detection mechanism reads:
%
\begin{center}
|... -jobname "|\textit{target}|" "|[\textit{flags}]%
[|\def\jobname{|\textit{dest}|}|]|\input{|\textit{main}|}"|
\end{center}

%%%%%%%%%%%%%%%%%%%%%%%%%%%%%%%%%%%%%%%%%%%%%%%%%%%%%%%%%%%%%%%%%%%%%%%%%%%%%%%%
\subsection{Manual Code}
\label{sec:manual}

In case one cannot be certain whether the definitions file |childdoc.def|
is installed on the target \TeX{} distribution
and one prefers not to ship it,
it is conceivable to paste a few relevant commands into the sources.

To that end, drop all statements |\input{childdoc.def}|
and perform the replacements as outlined below.
Instead of |\childdocmain{|\textit{main}|}| add the following code
to the top of the main file:
%
\begin{center}
\begin{tabular}{l}
|\||ifdefined\childdocname\endinput\||fi\newif\ifchilddoc|\\
|\edef\childdocname{\scantokens\expandafter{\jobname\noexpand}}|\\
|\def\childdocmain{|\textit{main}|}\||ifx\childdocmain\childdocname\||else|\\
|\childdoctrue\includeonly{\childdocname}\let\jobname\childdocmain\||fi|\\
\end{tabular}
\end{center}
%
Instead of |\childdocof{|\textit{main}|}| just include the main file
at the top of each child file:
%
\begin{center}
|\input{|\textit{main}|}|
\end{center}
%
A simple redirection |\childdocforward{|\textit{dest}|}| is achieved by:
%
\begin{center}
|\def\jobname{|\textit{dest}|}\input{\jobname}|
\end{center}
%
The redirection with prefix
|\childdocforwardprefix[|\textit{prefix}|]{|\textit{dest}|}|
is accomplished by:
%
\begin{center}
\begin{tabular}{l}
|{\edef\jobname{\scantokens\expandafter{\jobname\noexpand}}|\\
|\def\redirectjob |\textit{prefix}|#1~~~{\gdef\jobname{|\textit{dest}|#1}}|\\
|\expandafter\redirectjob\jobname~~~}\input{\jobname}|
\end{tabular}
\end{center}

In an alternative approach,
child documents can be compiled by a specific command line
without additional code or specific definitions:
%
\begin{center}
|... -jobname "|\textit{target}|" "|[\textit{flags}]%
|\includeonly{|\textit{dest}|}\input{|\textit{main}|}"|
\end{center}
%

%%%%%%%%%%%%%%%%%%%%%%%%%%%%%%%%%%%%%%%%%%%%%%%%%%%%%%%%%%%%%%%%%%%%%%%%%%%%%%%%
%%%%%%%%%%%%%%%%%%%%%%%%%%%%%%%%%%%%%%%%%%%%%%%%%%%%%%%%%%%%%%%%%%%%%%%%%%%%%%%%
\section{Information}

%%%%%%%%%%%%%%%%%%%%%%%%%%%%%%%%%%%%%%%%%%%%%%%%%%%%%%%%%%%%%%%%%%%%%%%%%%%%%%%%
\subsection{Copyright}

Copyright \copyright{} 2017--2018 Niklas Beisert

This work may be distributed and/or modified under the
conditions of the \LaTeX{} Project Public License, either version 1.3
of this license or (at your option) any later version.
The latest version of this license is in
  \url{http://www.latex-project.org/lppl.txt}
and version 1.3 or later is part of all distributions of \LaTeX{}
version 2005/12/01 or later.

This work has the LPPL maintenance status `maintained'.

The Current Maintainer of this work is Niklas Beisert.

This work consists of the files |README.txt|, |childdoc.ins| and |childdoc.dtx|
as well as the derived files |childdoc.def|, |cdocsamp.tex|
with |cdocsch1.tex|, |cdocsch2.tex|, |cdocspt3.tex|, |cdocspt4.tex|,
|cdocsdrf.tex|, |cdocsfn1.tex|, |cdocsfn2.tex|
as well as |childdoc.pdf|.

%%%%%%%%%%%%%%%%%%%%%%%%%%%%%%%%%%%%%%%%%%%%%%%%%%%%%%%%%%%%%%%%%%%%%%%%%%%%%%%%
\subsection{Files and Installation}

The package consists of the files:
%
\begin{center}
\begin{tabular}{ll}
    |README.txt|   & readme file \\
    |childdoc.ins| & installation file \\
    |childdoc.dtx| & source file \\
    |childdoc.def| & definition file \\
    |cdocsamp.tex| & sample main file \\
    |cdocsch1.tex| & sample include file \\
    |cdocsch2.tex| & sample include file \\
    |cdocspt3.tex| & sample part file \\
    |cdocspt4.tex| & sample part file \\
    |cdocsdrf.tex| & sample redirection file \\
    |cdocsfn1.tex| & sample redirection file \\
    |cdocsfn2.tex| & sample redirection file \\
    |childdoc.pdf| & manual
\end{tabular}
\end{center}
%
The distribution consists of the files
|README.txt|, |childdoc.ins| and |childdoc.dtx|.
%
\begin{itemize}
\item
Run (pdf)\LaTeX{} on |childdoc.dtx|
to compile the manual |childdoc.pdf| (this file).
\item
Run \LaTeX{} on |childdoc.ins| to create the definitions file |childdoc.def|
and the sample |cdocsamp.tex| with include files
|cdocsch1.tex|, |cdocsch2.tex|, |cdocspt3.tex|, |cdocspt4.tex|,
|cdocsdrf.tex|, |cdocsfn1.tex|, |cdocsfn2.tex|.
Then copy the file |childdoc.def| to an appropriate directory of your \LaTeX{}
distribution, e.g.\ \textit{texmf-root}|/tex/latex/childdoc|.
\end{itemize}

%%%%%%%%%%%%%%%%%%%%%%%%%%%%%%%%%%%%%%%%%%%%%%%%%%%%%%%%%%%%%%%%%%%%%%%%%%%%%%%%
\subsection{Related CTAN Packages}

There are several other packages which offer a similar functionality:
%
\begin{itemize}
\item
The packages
\href{http://ctan.org/pkg/docmute}{\textsf{docmute}},
\href{http://ctan.org/pkg/includex}{\textsf{includex}} and
\href{http://ctan.org/pkg/standalone}{\textsf{standalone}}
provide commands to include only the document body of
a child file thus allowing both files to be compiled individually.
\item
The packages \href{http://ctan.org/pkg/subdocs}{\textsf{subdocs}}
and \href{http://ctan.org/pkg/subfiles}{\textsf{subfiles}}
provide structures in which the main and child documents can be
encapsulated and allowing them to be compiled individually.
The inclusion mechanism is different from the conventional |\include|.
\item
The package \href{http://ctan.org/pkg/combine}{\textsf{combine}}
is an elaborate solution to combine several documents into one.
\end{itemize}
%
See also the CTAN topic \href{http://ctan.org/topic/subdocs}{\textsf{subdocs}}
for further related packages.
The present package differs from the above solutions in that
a document structure constructed with the conventional |\include| mechanism
just needs two extra commands at the top of every file
such that all constituent files can be compiled individually.

%%%%%%%%%%%%%%%%%%%%%%%%%%%%%%%%%%%%%%%%%%%%%%%%%%%%%%%%%%%%%%%%%%%%%%%%%%%%%%%%
%\subsection{Feature Suggestions}
%
%The following is a list of features which may be useful for future
%versions of this package:
%%
%\begin{itemize}
%\item
%\ldots
%\end{itemize}

%%%%%%%%%%%%%%%%%%%%%%%%%%%%%%%%%%%%%%%%%%%%%%%%%%%%%%%%%%%%%%%%%%%%%%%%%%%%%%%%
\subsection{Revision History}

%%%%%%%%%%%%%%%%%%%%%%%%%%%%%%%%%%%%%%%%
\paragraph{v2.0:} 2018/12/30

\begin{itemize}
\item
immediate forward processing
\item
added |\childdocby| mechanism
\item
manual restructured
\end{itemize}

%%%%%%%%%%%%%%%%%%%%%%%%%%%%%%%%%%%%%%%%
\paragraph{v1.6:} 2018/01/17

\begin{itemize}
\item
application for development of include files
\item
corrections to manual
\end{itemize}

%%%%%%%%%%%%%%%%%%%%%%%%%%%%%%%%%%%%%%%%
\paragraph{v1.5:} 2017/05/21

\begin{itemize}
\item
more complete structuring introduced
\item
|\childdocof| introduced
\item
|\childdoc| renamed to |\childdocmain|
\item
|\childredirect| renamed to |\childdocforward| and |\childdocforwardprefix|
and functionality expanded
\end{itemize}

%%%%%%%%%%%%%%%%%%%%%%%%%%%%%%%%%%%%%%%%
\paragraph{v1.0:} 2017/04/27

\begin{itemize}
\item
manual and install package
\item
first version published on CTAN
\end{itemize}

%%%%%%%%%%%%%%%%%%%%%%%%%%%%%%%%%%%%%%%%
\paragraph{v0.6:} 2017/04/26

\begin{itemize}
\item
redirection mechanism added
\end{itemize}

%%%%%%%%%%%%%%%%%%%%%%%%%%%%%%%%%%%%%%%%
\paragraph{v0.5:} 2017/04/26

\begin{itemize}
\item
functionality in definition file
\end{itemize}


%%%%%%%%%%%%%%%%%%%%%%%%%%%%%%%%%%%%%%%%%%%%%%%%%%%%%%%%%%%%%%%%%%%%%%%%%%%%%%%%
%%%%%%%%%%%%%%%%%%%%%%%%%%%%%%%%%%%%%%%%%%%%%%%%%%%%%%%%%%%%%%%%%%%%%%%%%%%%%%%%
%%%%%%%%%%%%%%%%%%%%%%%%%%%%%%%%%%%%%%%%%%%%%%%%%%%%%%%%%%%%%%%%%%%%%%%%%%%%%%%%
\appendix

\settowidth\MacroIndent{\rmfamily\scriptsize 000\ }

 \DocInput{childdoc.dtx}

\end{document}
%</driver>
% \fi
%
% %%%%%%%%%%%%%%%%%%%%%%%%%%%%%%%%%%%%%%%%%%%%%%%%%%%%%%%%%%%%%%%%%%%%%%%%%%%%%%
% %%%%%%%%%%%%%%%%%%%%%%%%%%%%%%%%%%%%%%%%%%%%%%%%%%%%%%%%%%%%%%%%%%%%%%%%%%%%%%
% \section{Sample}
%\iffalse
%<*samplemain>
%\fi
%
% The following presents a sample document
% with two chapters, two parts, a title page,
% a compile flag as well as three forwarding files to set the flag.
% It consists of eight |.tex| files:
% \begin{center}
% \begin{tabular}{ll}
% |cdocsamp.tex|&main file\\
% |cdocsch1.tex|&include file for chapter 1\\
% |cdocsch2.tex|&include file for chapter 2\\
% |cdocspt3.tex|&include file for part 3\\
% |cdocspt4.tex|&include file for part 4\\
% |cdocsdrf.tex|&forwarding file for main file in draft mode\\
% |cdocsfi1.tex|&forwarding file for final version of chapter 1\\
% |cdocsfi2.tex|&forwarding file for final version of chapter 2\\
% \end{tabular}
% \end{center}
% Each of the eight files can be compiled directly by the \LaTeX{} compiler.
%
% %%%%%%%%%%%%%%%%%%%%%%%%%%%%%%%%%%%%%%
% \paragraph{Main File.}
%
% The main file is called |cdocsamp.tex|.
%
% Load the \textsf{childdoc} definitions and
% declare the filename for the main document:
%    \begin{macrocode}
\input{childdoc.def}
\childdocmain{}
%    \end{macrocode}

% Optional override for |\version| flag:
%    \begin{macrocode}
%%\ifchilddoc\else\providecommand{\version}{draft}\fi
%    \end{macrocode}

% Define the default values for the |\version| flag
% (|final| for the main file and |draft| for childs):
%    \begin{macrocode}
\ifchilddoc
\providecommand{\version}{draft}
\else
\providecommand{\version}{final}
\fi
%    \end{macrocode}

% Load the standard document class:
%    \begin{macrocode}
\documentclass[12pt]{article}
%    \end{macrocode}

% Start the document body:
%    \begin{macrocode}
\begin{document}
%    \end{macrocode}

% Declare a title page.
% Print title, part of document being processed and version flag:
%    \begin{macrocode}
\addtocounter{page}{-1}
\begin{center}
{\LARGE\bfseries{}childdoc example\par}
\vspace{1cm}
\ifchilddoc
\ifchilddocmanual part\else chapter\fi:
`\childdocname' of `\childdocjob'\par
\else
main document: `\childdocjob'\par
\fi
version: \version\par
\end{center}
\newpage
%    \end{macrocode}

% Manually include selected file,
% otherwise process as usual:
%    \begin{macrocode}
\ifchilddocmanual
\section*{part `\childdocname'}
\input{\childdocname}
\else
%    \end{macrocode}

% Include the two chapters:
%    \begin{macrocode}
\include{cdocsch1}
\include{cdocsch2}
%    \end{macrocode}

% Include the two parts unless only chapters should be displayed:
%    \begin{macrocode}
\ifchilddoc\else
\section{part three}
\input{cdocspt3}
\section{part four}
\input{cdocspt4}
\fi
%    \end{macrocode}

% Process as usual until here:
%    \begin{macrocode}
\fi
%    \end{macrocode}

% End of document body:
%    \begin{macrocode}
\end{document}
%    \end{macrocode}
%\iffalse
%</samplemain>
%\fi
%
% %%%%%%%%%%%%%%%%%%%%%%%%%%%%%%%%%%%%%%
% \paragraph{Chapter Include Files.}
%
% The include files are called |cdocsch1.tex| and |cdocsch2.tex|.
%
%\iffalse
%<*samplechap1|samplechap2>
%\fi

% Optional override for |\version| flag:
%    \begin{macrocode}
%%\providecommand{\version}{final}
%    \end{macrocode}

% Include the main document:
%    \begin{macrocode}
\input{childdoc.def}
\childdocof{cdocsamp}
%    \end{macrocode}

%\iffalse
%</samplechap1|samplechap2>
%\fi
%
%\iffalse
%<*samplechap1>
%\fi
% Some text for chapter 1:
%    \begin{macrocode}
\section{one}
some text in chapter one
%    \end{macrocode}

%\iffalse
%</samplechap1>
%\fi
% Some text for chapter 2:
%\iffalse
%<*samplechap2>
%\fi
%    \begin{macrocode}
\section{two}
more text in chapter two
%    \end{macrocode}

%\iffalse
%</samplechap2>
%\fi
%
% %%%%%%%%%%%%%%%%%%%%%%%%%%%%%%%%%%%%%%
% \paragraph{Part Include Files.}
%
% The include files are called |cdocspt3.tex| and |cdocspt4.tex|.
%
%\iffalse
%<*samplepart3|samplepart4>
%\fi

% Optional override for |\version| flag:
%    \begin{macrocode}
%%\providecommand{\version}{final}
%    \end{macrocode}

% Include the main document:
%    \begin{macrocode}
\input{childdoc.def}
\childdocby{cdocsamp}
%    \end{macrocode}

%\iffalse
%</samplepart3|samplepart4>
%\fi
%
%\iffalse
%<*samplepart3>
%\fi
% Some text for part 3:
%    \begin{macrocode}
some text in part three
%    \end{macrocode}

%\iffalse
%</samplepart3>
%\fi
% Some text for part 4:
%\iffalse
%<*samplepart4>
%\fi
%    \begin{macrocode}
more text in part four
%    \end{macrocode}

%\iffalse
%</samplepart4>
%\fi
%
% %%%%%%%%%%%%%%%%%%%%%%%%%%%%%%%%%%%%%%
% \paragraph{Forwarding for a Complete Draft.}
%
% The following forwarding file |cdocsdrf.tex|
% compiles the main document in draft mode:
%\iffalse
%<*sampledraft>
%\fi
%    \begin{macrocode}
\def\version{draft}
\input{childdoc.def}
\childdocforward{cdocsamp}
%    \end{macrocode}

%\iffalse
%</sampledraft>
%\fi
%
% %%%%%%%%%%%%%%%%%%%%%%%%%%%%%%%%%%%%%%
% \paragraph{Forwarding for Final Version of the Chapters.}
%
% The following forwarding files |cdocsfn1.tex| and |cdocsfn2.tex|
% (with identical content)
% compile the final versions of the child documents
% |cdocsch1.tex| and |cdocsch2.tex|, respectively:
%\iffalse
%<*samplefinal>
%\fi
%    \begin{macrocode}
\def\version{final}
\input{childdoc.def}
\childdocforwardprefix[cdocsamp]{cdocsfn}{cdocsch}
%    \end{macrocode}

%\iffalse
%</samplefinal>
%\fi
%
% %%%%%%%%%%%%%%%%%%%%%%%%%%%%%%%%%%%%%%
% \paragraph{Command Line Processing.}
%
% The following three command lines generate the output files
% |cdocscld|, |cdocscl1| and |cdocscl2|
% which should be identical to
% |cdocsdrf|, |cdocsch1| and |cdocsfn2|, respectively:
% \begin{center}
% \begin{tabular}{l}
% |latex -jobname cdocscld \|\\
% |  "\def\version{draft}\input{childdoc.def}\childdocforward{cdocsamp}"|\\
% |latex -jobname cdocscl1 \|\\
% |  "\input{childdoc.def}\childdocforward[cdocsamp]{cdocsch1}"|\\
% |latex -jobname cdocscl2 \|\\
% |  "\def\version{final}\input{childdoc.def}\childdocforward{cdocsch2}"|
% \end{tabular}
% \end{center}
% Note that the trailing backslash on each first line
% merely continues the input to the second line
% (for convenient cut ant paste).
% Furthermore, the command |latex| can be replaced by any
% of its alternative versions such as |pdflatex|.
%
% %%%%%%%%%%%%%%%%%%%%%%%%%%%%%%%%%%%%%%%%%%%%%%%%%%%%%%%%%%%%%%%%%%%%%%%%%%%%%%
% %%%%%%%%%%%%%%%%%%%%%%%%%%%%%%%%%%%%%%%%%%%%%%%%%%%%%%%%%%%%%%%%%%%%%%%%%%%%%%
% \section{Implementation}
%\iffalse
%<*package>
%\fi
%
% This section describes the definitions file |childdoc.def|.

% The definitions cannot be loaded using |\usepackage| or |\RequirePackage|
% which has a mechanism to prevent loading a style file more than once.
% When loading the definitions by means of |\input|
% multiple instances have to be prevented manually:
%\iffalse
%This code needs to be before the `\ProvidesFile' directive
%which is defined at the beginning of this file.
%Therefore it is also placed there and commented out here.
%</package>
%<*discard>
%\fi
%    \begin{macrocode}
\ifdefined\childdocmain\endinput\fi
%    \end{macrocode}
%\iffalse
%</discard>
%<*package>
%\fi
%
% \macro{\ifchilddoc}
% \macro{\ifchilddocmanual}
% The conditional |\ifchilddoc| tells whether a
% child (true) or main (false) document is being compiled.
% The conditional |\ifchilddocmanual| tells whether
% the |\includeonly| mechanism is used (false) or
% the selection of child files must be performed manually (true).
% The definitions initialise to false:
%    \begin{macrocode}
\newif\ifchilddoc
\newif\ifchilddocmanual
%    \end{macrocode}

% \macro{\childdocname}
% \macro{\childdocjob}
% The macro |\childdocname| stores the name of the main document
% to be compiled. The macro |\childdocjob| stores the name of
% the document on which the \LaTeX{} compiler was originally invoked.
% The content of |\jobname| cannot be compared
% to filenames specified in the source due to different catcodes.
% The following code rescans |\jobname|, stores the result
% in |\childdocname| and saves a copy in |\childdocjob|:
%    \begin{macrocode}
\edef\childdocname{\scantokens\expandafter{\jobname\noexpand}}
\let\childdocjob\childdocname
%    \end{macrocode}

% \macro{\childdocdisable}
% The macro |\childdocdisable| prevents the main file
% from being processed more than once.
% At this stage, the main document command |\childdocmain|
% is assumed to be called once again where it should do nothing.
% Any subsequent call to it should prevent
% a secondary processing of the main document
% It overwrites the forwarding commands
% |\childdocof| and |\childdocforward|
% with empty macros to prevent further inclusions of the main document:
%    \begin{macrocode}
\newcommand{\childdocdisable}
{
  \renewcommand{\childdocmain}[1]{\renewcommand{\childdocmain}[1]{\endinput}}
  \renewcommand{\childdocof}[1]{}
  \renewcommand{\childdocby}[2][]{}
  \renewcommand{\childdocforward}[2][]{}
  \renewcommand{\childdocdisable}{}
}
%    \end{macrocode}

% \macro{\childdocmain}
% The macro |\childdocmain| is to be called at the top of the main file
% with nothing or the main filename (without extension) as argument.
% First, it breaks loops.
% If the argument is not empty and does not match |\childdocname|
% (which is set by the first inclusion of |childdoc.def|),
% |\ifchilddoc| is set to true, |\includeonly| is applied to the child file
% and |\jobname| is set to the main file
% (for proper handling of |.aux| files):
%    \begin{macrocode}
\newcommand{\childdocmain}[1]
{
  \childdocdisable\childdocmain{}
  \if?#1?\else
    \begingroup
      \def\childdoctmp{#1}
      \ifx\childdoctmp\childdocname
        \def\childdoctmp{}
      \else
        \def\childdoctmp
        {
          \childdoctrue
          \includeonly{\childdocname}
          \def\childdocjob{#1}
          \def\jobname{#1}
        }
      \fi
      \expandafter
    \endgroup
    \childdoctmp
  \fi
}
%    \end{macrocode}

% \macro{\childdocof}
% The command |\childdocof| redirects
% compilation to the main file |#1|.
%    \begin{macrocode}
\newcommand{\childdocof}[1]
{
  \childdocdisable
  \childdoctrue
  \includeonly{\childdocname}
  \def\jobname{#1}
  \def\childdocjob{#1}
  \input{#1}
}
%    \end{macrocode}

% \macro{\childdocby}
% The command |\childdocby| ....
%    \begin{macrocode}
\newcommand{\childdocby}[2][]
{
  \childdocdisable
  \childdoctrue
  \childdocmanualtrue
  \if?#1?\else
    \def\jobname{#2}
  \fi
  \def\childdocjob{#2}
  \input{#2}
  \endinput
}
%    \end{macrocode}

% \macro{\childdocforward}
% The command |\childdocforward| redirects
% compilation to the main file or
% (if the optional argument is given) a child file.
% Parameters are set as if the main file
% or a child file starting with |\childdocof| was compiled.
% Then compilation is handed over to the main file:
%    \begin{macrocode}
\newcommand{\childdocforward}[2][]
{
  \begingroup
    \if?#1?
      \def\childdoctmp
      {
        \def\childdocname{#2}
        \def\childdocjob{#2}
        \def\jobname{#2}
        \input{#2}
        \endinput
      }
    \else
      \def\childdoctmp
      {
        \childdocdisable
        \def\childdocname{#2}
        \childdoctrue
        \includeonly{#2}
        \def\childdocjob{#1}
        \def\jobname{#1}
        \input{#1}
        \endinput
      }
    \fi
    \expandafter
  \endgroup
  \childdoctmp
}
%    \end{macrocode}

% \macro{\childdocforwardprefix}
% The command |\childdocforwardprefix| redirects
% compilation to the main or a child file by means of a pattern.
% The prefix |#1| in the current filename is replaced by |#2|
% and the suffix of the current filename is kept
% (it is assumed that the filename does not contain the substring `|~~~|'
% which is used as a delimiter).
% Compilation is handed over to the new file by |\childdocforward|:
%    \begin{macrocode}
\newcommand{\childdocforwardprefix}[3][]
{
  \begingroup
    \def\childdocextract #2##1~~~{\def\childdoctmp{\childdocforward[#1]{#3##1}}}
    \expandafter\childdocextract\childdocname~~~
    \expandafter
  \endgroup
  \childdoctmp
}
%    \end{macrocode}

% \macro{\childdoc}
% The deprecated macro |\childdoc| is a legacy version of |\childdocmain|:
%    \begin{macrocode}
\newcommand{\childdoc}{\childdocmain}
%    \end{macrocode}

% \macro{\childdocredirect}
% The deprecated macro |\childdocredirect| is a legacy version
% of |\childdocforward| and |\childdocforwardprefix|:
%    \begin{macrocode}
\newcommand{\childdocredirect}[2][]
{
  \begingroup
    \if?#1?
      \def\childdoctmp{\childdocforward{#2}}
    \else
      \def\childdoctmp{\childdocforwardprefix{#1}{#2}}
    \fi
    \expandafter
  \endgroup
  \childdoctmp
}
%    \end{macrocode}

%\iffalse
%</package>
%\fi
%
\endinput
\childdocforward[|\textit{main}|]{|\textit{dest}|}"|
\end{center}
%
Here \textit{target} is the name of the output file,
\textit{main} is the name of the main file
and \textit{dest} is the name of the main or child file to be processed
(all filenames without extensions).
The optional argument \textit{main} can be omitted
if \textit{main} matches \textit{dest}.
Optionally, compilation \textit{flags} can be defined via |\def| commands.
This command line makes the \TeX{} engine believe
it is compiling the file \textit{target}
whose content is specified as the latter parameter.
The provided code then forwards the processing to
\textit{main} or \textit{dest} as described in \secref{sec:forward}.

%%%%%%%%%%%%%%%%%%%%%%%%%%%%%%%%%%%%%%%%%%%%%%%%%%%%%%%%%%%%%%%%%%%%%%%%%%%%%%%%
\subsection{Include by Input}
\label{sec:input}

Including child documents by |\include| has some restrictions by design.
Most notably, the content of a child document always occupies
its own set of pages; pages cannot be shared between child documents.
Usually, this behaviour makes perfect sense
because each child document contain an essential part of the document.
However, in some situations it may be desirable to compose
a document from a collection of parts
without having mandatory page breaks between then.
For this case, the package
provides a mechanism to include parts
by |\input| which can also be processed individually.
However, by construction this mechanism
requires manual handling of the content to be output.

%%%%%%%%%%%%%%%%%%%%%%%%%%%%%%%%%%%%%%%%
\DescribeMacro{\ifchilddocmanual}
The main file should be prepared as usual, see \secref{sec:include}.
However, the document body must make a distinction
between processing of an individual part and of the main document, e.g.:
%
\begin{center}
\begin{tabular}{l}
|\ifchilddocmanual|\\
|\input{\childdocname}|\\
|\||else|\\
\textit{document body with }|\input{|\textit{part}|}|\\
|\||fi|
\end{tabular}
\end{center}
%
The conditional |\ifchilddocmanual| is true whenever
a part to be included by |\input| is being compiled,
and the name of the part is stored in |\childdocname|.

%%%%%%%%%%%%%%%%%%%%%%%%%%%%%%%%%%%%%%%%
\DescribeMacro{\childdocby}
Each part to be included by |\input| should start with:
%
\begin{center}
\begin{tabular}{l}
|% \iffalse
%
% childdoc.dtx Copyright (C) 2017-2018 Niklas Beisert
%
% This work may be distributed and/or modified under the
% conditions of the LaTeX Project Public License, either version 1.3
% of this license or (at your option) any later version.
% The latest version of this license is in
%   http://www.latex-project.org/lppl.txt
% and version 1.3 or later is part of all distributions of LaTeX
% version 2005/12/01 or later.
%
% This work has the LPPL maintenance status `maintained'.
%
% The Current Maintainer of this work is Niklas Beisert.
%
% This work consists of the files childdoc.dtx and childdoc.ins
% and the derived files childdoc.def and cdocsamp.tex with
% cdocsch1.tex, cdocsch2.tex, cdocsdrf.tex, cdocsfn1.tex, cdocsfn2.tex.
%
%<package>\ifdefined\childdocmain\endinput\fi
%<package>\ProvidesFile{childdoc.def}[2018/12/30 v2.0 child document driver]
%<samplemain>\ProvidesFile{cdocsamp.tex}[2018/12/30 v2.0 sample for childdoc]
%<*driver>
%\ProvidesFile{childdoc.drv}[2018/12/30 v2.0 childdoc reference manual file]
\PassOptionsToClass{10pt,a4paper}{article}
\documentclass{ltxdoc}

\usepackage[margin=35mm]{geometry}
\usepackage{hyperref}
\usepackage{hyperxmp}
\usepackage[usenames]{color}

\hypersetup{colorlinks=true}
\hypersetup{pdfstartview=FitH}
\hypersetup{pdfpagemode=UseNone}
\hypersetup{pdfsource={}}
\hypersetup{pdflang={en-UK}}
\hypersetup{pdfcopyright={Copyright 2017-2018 Niklas Beisert.
  This work may be distributed and/or modified under the
  conditions of the LaTeX Project Public License, either version 1.3
  of this license or (at your option) any later version.}}
\hypersetup{pdflicenseurl={http://www.latex-project.org/lppl.txt}}
\hypersetup{pdfcontactaddress={ETH Zurich, ITP, HIT K,
  Wolfgang-Pauli-Strasse 27}}
\hypersetup{pdfcontactpostcode={8093}}
\hypersetup{pdfcontactcity={Zurich}}
\hypersetup{pdfcontactcountry={Switzerland}}
\hypersetup{pdfcontactemail={nbeisert@itp.phys.ethz.ch}}
\hypersetup{pdfcontacturl={http://people.phys.ethz.ch/\xmptilde nbeisert/}}

\newcommand{\secref}[1]{\hyperref[#1]{section \ref*{#1}}}

\parskip1ex
\parindent0pt
\let\olditemize\itemize
\def\itemize{\olditemize\parskip0pt}

\begin{document}

\title{The \textsf{childdoc} Package}
\hypersetup{pdftitle={The childdoc Package}}
\author{Niklas Beisert\\[2ex]
  Institut f\"ur Theoretische Physik\\
  Eidgen\"ossische Technische Hochschule Z\"urich\\
  Wolfgang-Pauli-Strasse 27, 8093 Z\"urich, Switzerland\\[1ex]
  \href{mailto:nbeisert@itp.phys.ethz.ch}
  {\texttt{nbeisert@itp.phys.ethz.ch}}}
\hypersetup{pdfauthor={Niklas Beisert}}
\hypersetup{pdfsubject={Manual for the LaTeX2e Package childdoc}}
\date{30 December 2018, \textsf{v2.0}}
\maketitle

\begin{abstract}\noindent
\textsf{childdoc} is a \LaTeXe{} package
that enables the direct compilation
of document sections included by |\include|
to individual files.
\end{abstract}

\begingroup
\parskip0ex
\tableofcontents
\endgroup

%%%%%%%%%%%%%%%%%%%%%%%%%%%%%%%%%%%%%%%%%%%%%%%%%%%%%%%%%%%%%%%%%%%%%%%%%%%%%%%%
%%%%%%%%%%%%%%%%%%%%%%%%%%%%%%%%%%%%%%%%%%%%%%%%%%%%%%%%%%%%%%%%%%%%%%%%%%%%%%%%
\section{Introduction}

\LaTeX{} provides a mechanism to structure a large document (such as a book)
into a main file and several child files (containing the chapters)
using the |\include| command.
This mechanism is beneficial for documents
which span hundreds of pages in order to
make the source file(s) more manageable.
Moreover, compilation can be restricted to
selected child files by means of the |\includeonly| command.
The latter feature can be used to reduce the compilation time while editing
(this was significantly more useful in the earlier days of \LaTeX{})
or to generate a smaller document which is easier to navigate.
Another application of |\includeonly| is to generate
documents consisting of selected parts of the complete document.

However, there are a few drawbacks of the plain |\include| mechanism:
\begin{itemize}
\item
The child files cannot be compiled on their own,
they can only be compiled via the main file.
A naive editing environment
(such as a text editor with an option
to have the current file processed by \LaTeX)
may require one to switch to the main file before compiling;
attempting to compile the child file produces errors.
\item
The main file must be modified (each time)
to adjust the |\includeonly| command
to the present needs. This easily leaves the main file in a messy state.
\item
The generated document will always carry the filename
of the main document. This is inconvenient if
several child files are to be compiled and
to be kept for distribution.
\end{itemize}

The present package provides a simple interface
to make child files individually compilable by \LaTeX{}.
Compiling a child file then has the same effect as compiling
the main file with an |\includeonly| command
to select the appropriate child.
Moreover the generated document will carry the name of the child
rather than the main file.
This resolves all three above issues.

This feature is meant to make the editing of books,
thesis documents and lecture notes somewhat more convenient.
However, the package can also be used efficiently for
composing a series of documents (such as exercise sheets)
which are typically distributed individually.
It then assists the author in generating the individual documents
(potentially in different versions)
as well as a document containing the collected series.
Another application is in developing style files
or other kinds of included material
where compilation of the style file could redirect
to a sample or test file.

%%%%%%%%%%%%%%%%%%%%%%%%%%%%%%%%%%%%%%%%%%%%%%%%%%%%%%%%%%%%%%%%%%%%%%%%%%%%%%%%
%%%%%%%%%%%%%%%%%%%%%%%%%%%%%%%%%%%%%%%%%%%%%%%%%%%%%%%%%%%%%%%%%%%%%%%%%%%%%%%%
\section{Usage}

First of all, the package \textsf{childdoc} is \emph{not} a standard
\LaTeXe{} |.sty| style file! Therefore it needs to be invoked in
a non-standard way.

%%%%%%%%%%%%%%%%%%%%%%%%%%%%%%%%%%%%%%%%%%%%%%%%%%%%%%%%%%%%%%%%%%%%%%%%%%%%%%%%
\subsection{Included Files}
\label{sec:include}

%%%%%%%%%%%%%%%%%%%%%%%%%%%%%%%%%%%%%%%%
\DescribeMacro{\childdocmain}
To use the package, add the commands
\begin{center}
\begin{tabular}{l}
|\input{childdoc.def}|\\
|\childdocmain{}|\\
\end{tabular}
\end{center}
at the very top of the main \LaTeX{} file,
in particular \emph{before} the |\documentclass| statement!
The argument of |\childdocmain| should be left empty
(but it must be present).

%%%%%%%%%%%%%%%%%%%%%%%%%%%%%%%%%%%%%%%%
\DescribeMacro{\childdocof}
Furthermore, add the commands
\begin{center}
\begin{tabular}{l}
|\input{childdoc.def}|\\
|\childdocof{|\textit{main}|}|\\
\end{tabular}
\end{center}
at the top of every child file \textit{child}
which is included by |\include{|\textit{child}|}|
from within the main file
(or at least for those files to be compiled individually).
The argument \textit{main} must be the filename of the main file.

There are a couple of
considerations in setting up the main and child documents:

%%%%%%%%%%%%%%%%%%%%%%%%%%%%%%%%%%%%%%%%
\paragraph{Restrictions.}

Please note the following restrictions:
\begin{itemize}
\item
|\childdocmain| must be called with one argument \textit{main}
to ensure compatibility with earlier version of the package.
It must either be empty (|\childdocmain{}|)
or precisely match the filename of the main file in which it is specified.
See \secref{sec:detection} for further information.
\item
The filename \textit{main} must be specified without the |.tex| extension.
\item
The filename \textit{main} is case sensitive
(even in case-insensitive file systems)
due to internal string comparison.
\item
The argument \textit{main} should be fully expanded, it cannot be a macro.
\item
Subdirectories and special characters should be avoided in filenames.
\item
The command |\childdocmain{|\textit{main}|}| must be followed by a whitespace.
It should not be followed immediately by another command
or by a comment mark `|%|'.
This is because the \TeX{} parser reads the token immediately following
the argument of |\childdocmain| and puts it
at the beginning of every child section;
however, a white\-space is ignored.
\end{itemize}

%%%%%%%%%%%%%%%%%%%%%%%%%%%%%%%%%%%%%%%%
\paragraph{Content of Main File.}

It is advisable to place all content in the child files included by |\include|.
Any output contained in the main file will appear in all child documents
unless suppressed manually;
it cannot be suppressed automatically by the |\includeonly| directive
and thus should normally be avoided.
A method to include some content in the main file
by means of conditional processing is described in \secref{sec:conditional}.

%%%%%%%%%%%%%%%%%%%%%%%%%%%%%%%%%%%%%%%%
\paragraph{Page Numbering.}

When only a part of the document is compiled,
the appropriate numbering of pages
(as well as other status parameters)
is determined from the |.aux| files.
The latter contain information from previous passes.
However this information needs to propagate through
all intermediate child documents.
Therefore the page numbering in child documents may well
be inconsistent until the complete document is compiled at least once.

A useful (if unconventional) way to always ensure a consistent
page numbering is to restart the numbering in each child document
and denote the pages by `\textit{child}|.|\textit{page}'
where \textit{child} represents the chapter/section number of the child file.
This can be achieved by the command
|\numberwithin{page}{|\textit{child}|}|
of the \textsf{amsmath} package
where \textit{child} can be |chapter| or |section|
depending on the chosen structuring.
Alternatively, one can modify the macro |\thepage| appropriately
and reset the counter |page| at the start of each child file.

%%%%%%%%%%%%%%%%%%%%%%%%%%%%%%%%%%%%%%%%%%%%%%%%%%%%%%%%%%%%%%%%%%%%%%%%%%%%%%%%
\subsection{Conditional Processing}
\label{sec:conditional}

The package provides a mechanism to compile different versions
of a document. To customise the versions further some conditional processing
can come in handy to distinguish which version is being compiled.
The package provides two macros to describe the compilation context:

%%%%%%%%%%%%%%%%%%%%%%%%%%%%%%%%%%%%%%%%
\DescribeMacro{\ifchilddoc}
The conditional |\ifchilddoc| distinguishes between the compilation of
child documents and the main document:
%
\begin{center}
|\ifchilddoc |\textit{child-code}| |[|\||else |\textit{main-code}]| \||fi|
\end{center}

%%%%%%%%%%%%%%%%%%%%%%%%%%%%%%%%%%%%%%%%
\DescribeMacro{\childdocname}
\DescribeMacro{\childdocjob}
The macro |\childdocname| contains the filename (without extension)
of the main or child file being processed.
Note that |\childdocjob| will always contain the name of the main file.

%%%%%%%%%%%%%%%%%%%%%%%%%%%%%%%%%%%%%%%%
\paragraph{Title Page.}

Conditional processing can be used to include a title or banner page
in the main document when proper precautions are taken.
Importantly, the code in the main file should ensure that the page counter
(as well as other status parameters which are stored in the |.aux| files)
takes the same value after the conditional processing.
Otherwise the page numbers may take divergent values
depending on which part is compiled.

For example, a title page could be declared by:
%
\begin{center}
\begin{tabular}{l}
|\ifchilddoc\||else|\\
|\addtocounter{page}{-1}|\\
\textit{code for title page}\\
|\newpage|\\
|\||fi|
\end{tabular}
\end{center}
%
A banner page for the child documents can be generated by:
%
\begin{center}
\begin{tabular}{l}
|\ifchilddoc|\\
|\addtocounter{page}{-1}|\\
\textit{code for banner page}\\
|\newpage|\\
|\||fi|
\end{tabular}
\end{center}
%
Here one could write a message such as:
\begin{center}
|This is the part \childdocname{} of \childdocjob{}.|
\end{center}

%%%%%%%%%%%%%%%%%%%%%%%%%%%%%%%%%%%%%%%%%%%%%%%%%%%%%%%%%%%%%%%%%%%%%%%%%%%%%%%%
\subsection{Flags}
\label{sec:flags}

The package makes it easy to generate different versions
of the main or child documents.
To this end compilation flags can be defined
and assigned different default values.
They will be particularly useful in conjunction
with the forwarding mechanism described in \secref{sec:forward}.

For example, it may be useful to have a flag |\version|
which can be set to |draft| or |final|.
The document source will contain some conditional code
depending on the value of |\version|.
Suppose further, the flag should default to |final| for the main file
and to |draft| for child files
which is a natural assignment for editing the document.
This is achieved by placing the following code
in the preamble of the main document
(below the |\childdocmain| directive):
%
\begin{center}
\begin{tabular}{l}
|\ifchilddoc|\\
|\providecommand{\version}{draft}|\\
|\||else|\\
|\providecommand{\version}{final}|\\
|\||fi|
\end{tabular}
\end{center}
%
The definition by |\providecommand| makes sure
that previous definitions are not overwritten.
Further statements |\providecommand{\version}{...}|
can thus be added before the above code to override it.

For the main file, one might add a line
(between |\childdocmain| and the above block)
%
\begin{center}
|%\ifchilddoc\||else\providecommand{\version}{draft}\||fi|
\end{center}
%
which can be uncommented to produce a draft version.
Likewise one can add a line to the very top of a child file
(above the |\childdocof{|\textit{main}|}| directive)
%
\begin{center}
|%\providecommand{\version}{final}|
\end{center}
%
which can be uncommented to produce the final version of this child document.

%%%%%%%%%%%%%%%%%%%%%%%%%%%%%%%%%%%%%%%%%%%%%%%%%%%%%%%%%%%%%%%%%%%%%%%%%%%%%%%%
\subsection{Forwarding}
\label{sec:forward}

Different versions of the main or child documents
using compilation flags as described in \secref{sec:flags}
can be (permanently) stored in different files
for convenient compilation, viewing and distribution.
To this end, the package defines a command
to pass on compilation to a different file:

%%%%%%%%%%%%%%%%%%%%%%%%%%%%%%%%%%%%%%%%
\DescribeMacro{\childdocforward}
The command |\childdocforward| redirects processing to
another source file:
%
\begin{center}
\begin{tabular}{l}
|\input{childdoc.def}|\\
|\childdocforward[|\textit{main}|]{|\textit{dest}|}|\\
\end{tabular}
\end{center}
%
The argument \textit{dest} is the destination file
(without extension).
It should be the main file or one of the child files.
Note that further \textsf{childdoc} directives
such as |\childdocof| and |\childdocforward|
in the indicated file will be processed in this form.
The optional argument \textit{main}
passes on directly to the main file \textit{main}
while pretending to compile the child \textit{dest}.
This form behaves as if \textit{dest}
issues |\childdocof{|\textit{main}|}| right away,
and no further \textsf{childdoc} directives will be processed.

%%%%%%%%%%%%%%%%%%%%%%%%%%%%%%%%%%%%%%%%
\DescribeMacro{\...prefix}
In the alternative form |\childdocforwardprefix|,
%
\begin{center}
\begin{tabular}{l}
|\input{childdoc.def}|\\
|\childdocforwardprefix[|\textit{main}|]{|\textit{prefix}|}{|\textit{dest}|}|
\end{tabular}
\end{center}
%
the destination file is determined by a pattern
depending on the current file:
To make this work, the current file must be called
`{\textit{prefix}\hspace{0.2em}\textit{suffix}}'
with \textit{prefix} matching precisely the argument.
Processing is then passed on to the file
`{\textit{dest}\hspace{0.2em}\textit{suffix}}'.
Surely, the same effect is achieved by
directly specifying the
argument `{\textit{dest}\hspace{0.2em}\textit{suffix}}'
in the first form.
However, that requires to set up a different file
for each child. With the alternative form of the command
all these files can have exactly the same content
which simplifies setting them up and maintaining them.

For example, the following file |draft.tex|
with a compilation flag |\version| as described in \secref{sec:flags}
compiles the main document as a draft:
%
\begin{center}
\begin{tabular}{l}
|\def\version{draft}|\\
|\input{childdoc.def}|\\
|\childdocforward{|\textit{main}|}|
\end{tabular}
\end{center}
%
Likewise, the following files |final|\textit{nn}|.tex|
compile the final version of the child document
|child|\textit{nn}|.tex|:
%
\begin{center}
\begin{tabular}{l}
|\def\version{final}|\\
|\input{childdoc.def}|\\
|\childdocforwardprefix{final}{child}|
\end{tabular}
\end{center}
%

Note that when several versions of a main file and/or of each child file
are to be generated, it may be convenient to set up a |Makefile| or
shell script to automatise the process.

%%%%%%%%%%%%%%%%%%%%%%%%%%%%%%%%%%%%%%%%%%%%%%%%%%%%%%%%%%%%%%%%%%%%%%%%%%%%%%%%
\subsection{Command Line Processing}
\label{sec:commandline}

The effect of redirection files can also be achieved by invoking
the \LaTeX{} compiler with a more elaborate command line.
Most conveniently this should be done as part
of a shell script or a |Makefile|.

When using \textsf{childdoc} in the main file, the following
command lines effectively perform a redirection
(note that depending on the shell being used,
backslashes may have to be doubled: `|\|' $\to$ `|\\|'):
%
\begin{center}
|... -jobname "|\textit{target}|" |\\|"|[\textit{flags}]%
|\input{childdoc.def}\childdocforward[|\textit{main}|]{|\textit{dest}|}"|
\end{center}
%
Here \textit{target} is the name of the output file,
\textit{main} is the name of the main file
and \textit{dest} is the name of the main or child file to be processed
(all filenames without extensions).
The optional argument \textit{main} can be omitted
if \textit{main} matches \textit{dest}.
Optionally, compilation \textit{flags} can be defined via |\def| commands.
This command line makes the \TeX{} engine believe
it is compiling the file \textit{target}
whose content is specified as the latter parameter.
The provided code then forwards the processing to
\textit{main} or \textit{dest} as described in \secref{sec:forward}.

%%%%%%%%%%%%%%%%%%%%%%%%%%%%%%%%%%%%%%%%%%%%%%%%%%%%%%%%%%%%%%%%%%%%%%%%%%%%%%%%
\subsection{Include by Input}
\label{sec:input}

Including child documents by |\include| has some restrictions by design.
Most notably, the content of a child document always occupies
its own set of pages; pages cannot be shared between child documents.
Usually, this behaviour makes perfect sense
because each child document contain an essential part of the document.
However, in some situations it may be desirable to compose
a document from a collection of parts
without having mandatory page breaks between then.
For this case, the package
provides a mechanism to include parts
by |\input| which can also be processed individually.
However, by construction this mechanism
requires manual handling of the content to be output.

%%%%%%%%%%%%%%%%%%%%%%%%%%%%%%%%%%%%%%%%
\DescribeMacro{\ifchilddocmanual}
The main file should be prepared as usual, see \secref{sec:include}.
However, the document body must make a distinction
between processing of an individual part and of the main document, e.g.:
%
\begin{center}
\begin{tabular}{l}
|\ifchilddocmanual|\\
|\input{\childdocname}|\\
|\||else|\\
\textit{document body with }|\input{|\textit{part}|}|\\
|\||fi|
\end{tabular}
\end{center}
%
The conditional |\ifchilddocmanual| is true whenever
a part to be included by |\input| is being compiled,
and the name of the part is stored in |\childdocname|.

%%%%%%%%%%%%%%%%%%%%%%%%%%%%%%%%%%%%%%%%
\DescribeMacro{\childdocby}
Each part to be included by |\input| should start with:
%
\begin{center}
\begin{tabular}{l}
|\input{childdoc.def}|\\
|\childdocby{|\textit{main}|}|\\
\end{tabular}
\end{center}
%
The directive |\childdocby| is similar to |\childdocof|
described in \secref{sec:include},
but the subsequent selection of content must be done manually.
To that end, both |\ifchilddoc| and |\ifchilddocmanual|
will be true upon processing of a part,
and the name of the part is stored in |\childdocname|.
Note that |\jobname| will be set to the filename of the current part
so that each part receives an individual |.aux| file
that does not interfere with the |.aux| file(s) of the main document.
This behaviour can be altered by the alternative form
|\childdocby[*]{|\textit{main}|}| (with a non-empty optional argument)
which uses the |.aux| file of the main document
by setting |\jobname| to \textit{main}.

%%%%%%%%%%%%%%%%%%%%%%%%%%%%%%%%%%%%%%%%%%%%%%%%%%%%%%%%%%%%%%%%%%%%%%%%%%%%%%%%
\subsection{Driver Development}
\label{sec:driver}

The \textsf{childdoc} mechanism can also be use for the development
of definition files such as \LaTeX{} styles or classes.
This case differs from the above setup with multiple parts
included by |\include| in that no |\includeonly| should be invoked.
This can be achieved by starting the include file
(before |\ProvidesPackage|) with:
%
\begin{center}
\begin{tabular}{l}
|\input{childdoc.def}|\\
|\childdocforward{|\textit{main}|}|\\
\end{tabular}
\end{center}
%
or alternatively with:
%
\begin{center}
\begin{tabular}{l}
|\input{childdoc.def}|\\
|\childdocby{|\textit{main}|}|\\
\end{tabular}
\end{center}
%
Both forms have slightly different effects as described above.
The main file is prepared as usual, see \secref{sec:include}.

%%%%%%%%%%%%%%%%%%%%%%%%%%%%%%%%%%%%%%%%%%%%%%%%%%%%%%%%%%%%%%%%%%%%%%%%%%%%%%%%
\subsection{Legacy Detection}
\label{sec:detection}

The directive |\childdocmain| in the main file can detect
whether the complete document or merely a child is to be compiled
even without using the directive |\childdocof|.
This method is deprecated because it is less robust
and there is no compelling reason to use it;
it is merely provided for backward compatibility
and it may be removed in future versions.

If the detection mechanism is to be used,
it is mandatory to correctly specify
the filename of the main file as the argument of |\childdocmain|:
%
\begin{center}
\begin{tabular}{l}
|\input{childdoc.def}|\\
|\childdocmain{|\textit{main}|}|\\
\end{tabular}
\end{center}
%
If |\jobname| does not match the argument \textit{main} of |\childdocmain|,
it is assumed that |\jobname| points to the child file to be compiled.
When using |\childdocmain| with the main file specified as argument,
it suffices to start a child file
with just |\input{|\textit{main}|}|
without loading of the package and using |\childdocof|.
If instead all processing is done
with the appropriate \textsf{childdoc} directives,
the argument of \textit{main} of |\childdocmain| can be empty.

An alternative version of the command line processing described
in \secref{sec:commandline} using the detection mechanism reads:
%
\begin{center}
|... -jobname "|\textit{target}|" "|[\textit{flags}]%
[|\def\jobname{|\textit{dest}|}|]|\input{|\textit{main}|}"|
\end{center}

%%%%%%%%%%%%%%%%%%%%%%%%%%%%%%%%%%%%%%%%%%%%%%%%%%%%%%%%%%%%%%%%%%%%%%%%%%%%%%%%
\subsection{Manual Code}
\label{sec:manual}

In case one cannot be certain whether the definitions file |childdoc.def|
is installed on the target \TeX{} distribution
and one prefers not to ship it,
it is conceivable to paste a few relevant commands into the sources.

To that end, drop all statements |\input{childdoc.def}|
and perform the replacements as outlined below.
Instead of |\childdocmain{|\textit{main}|}| add the following code
to the top of the main file:
%
\begin{center}
\begin{tabular}{l}
|\||ifdefined\childdocname\endinput\||fi\newif\ifchilddoc|\\
|\edef\childdocname{\scantokens\expandafter{\jobname\noexpand}}|\\
|\def\childdocmain{|\textit{main}|}\||ifx\childdocmain\childdocname\||else|\\
|\childdoctrue\includeonly{\childdocname}\let\jobname\childdocmain\||fi|\\
\end{tabular}
\end{center}
%
Instead of |\childdocof{|\textit{main}|}| just include the main file
at the top of each child file:
%
\begin{center}
|\input{|\textit{main}|}|
\end{center}
%
A simple redirection |\childdocforward{|\textit{dest}|}| is achieved by:
%
\begin{center}
|\def\jobname{|\textit{dest}|}\input{\jobname}|
\end{center}
%
The redirection with prefix
|\childdocforwardprefix[|\textit{prefix}|]{|\textit{dest}|}|
is accomplished by:
%
\begin{center}
\begin{tabular}{l}
|{\edef\jobname{\scantokens\expandafter{\jobname\noexpand}}|\\
|\def\redirectjob |\textit{prefix}|#1~~~{\gdef\jobname{|\textit{dest}|#1}}|\\
|\expandafter\redirectjob\jobname~~~}\input{\jobname}|
\end{tabular}
\end{center}

In an alternative approach,
child documents can be compiled by a specific command line
without additional code or specific definitions:
%
\begin{center}
|... -jobname "|\textit{target}|" "|[\textit{flags}]%
|\includeonly{|\textit{dest}|}\input{|\textit{main}|}"|
\end{center}
%

%%%%%%%%%%%%%%%%%%%%%%%%%%%%%%%%%%%%%%%%%%%%%%%%%%%%%%%%%%%%%%%%%%%%%%%%%%%%%%%%
%%%%%%%%%%%%%%%%%%%%%%%%%%%%%%%%%%%%%%%%%%%%%%%%%%%%%%%%%%%%%%%%%%%%%%%%%%%%%%%%
\section{Information}

%%%%%%%%%%%%%%%%%%%%%%%%%%%%%%%%%%%%%%%%%%%%%%%%%%%%%%%%%%%%%%%%%%%%%%%%%%%%%%%%
\subsection{Copyright}

Copyright \copyright{} 2017--2018 Niklas Beisert

This work may be distributed and/or modified under the
conditions of the \LaTeX{} Project Public License, either version 1.3
of this license or (at your option) any later version.
The latest version of this license is in
  \url{http://www.latex-project.org/lppl.txt}
and version 1.3 or later is part of all distributions of \LaTeX{}
version 2005/12/01 or later.

This work has the LPPL maintenance status `maintained'.

The Current Maintainer of this work is Niklas Beisert.

This work consists of the files |README.txt|, |childdoc.ins| and |childdoc.dtx|
as well as the derived files |childdoc.def|, |cdocsamp.tex|
with |cdocsch1.tex|, |cdocsch2.tex|, |cdocspt3.tex|, |cdocspt4.tex|,
|cdocsdrf.tex|, |cdocsfn1.tex|, |cdocsfn2.tex|
as well as |childdoc.pdf|.

%%%%%%%%%%%%%%%%%%%%%%%%%%%%%%%%%%%%%%%%%%%%%%%%%%%%%%%%%%%%%%%%%%%%%%%%%%%%%%%%
\subsection{Files and Installation}

The package consists of the files:
%
\begin{center}
\begin{tabular}{ll}
    |README.txt|   & readme file \\
    |childdoc.ins| & installation file \\
    |childdoc.dtx| & source file \\
    |childdoc.def| & definition file \\
    |cdocsamp.tex| & sample main file \\
    |cdocsch1.tex| & sample include file \\
    |cdocsch2.tex| & sample include file \\
    |cdocspt3.tex| & sample part file \\
    |cdocspt4.tex| & sample part file \\
    |cdocsdrf.tex| & sample redirection file \\
    |cdocsfn1.tex| & sample redirection file \\
    |cdocsfn2.tex| & sample redirection file \\
    |childdoc.pdf| & manual
\end{tabular}
\end{center}
%
The distribution consists of the files
|README.txt|, |childdoc.ins| and |childdoc.dtx|.
%
\begin{itemize}
\item
Run (pdf)\LaTeX{} on |childdoc.dtx|
to compile the manual |childdoc.pdf| (this file).
\item
Run \LaTeX{} on |childdoc.ins| to create the definitions file |childdoc.def|
and the sample |cdocsamp.tex| with include files
|cdocsch1.tex|, |cdocsch2.tex|, |cdocspt3.tex|, |cdocspt4.tex|,
|cdocsdrf.tex|, |cdocsfn1.tex|, |cdocsfn2.tex|.
Then copy the file |childdoc.def| to an appropriate directory of your \LaTeX{}
distribution, e.g.\ \textit{texmf-root}|/tex/latex/childdoc|.
\end{itemize}

%%%%%%%%%%%%%%%%%%%%%%%%%%%%%%%%%%%%%%%%%%%%%%%%%%%%%%%%%%%%%%%%%%%%%%%%%%%%%%%%
\subsection{Related CTAN Packages}

There are several other packages which offer a similar functionality:
%
\begin{itemize}
\item
The packages
\href{http://ctan.org/pkg/docmute}{\textsf{docmute}},
\href{http://ctan.org/pkg/includex}{\textsf{includex}} and
\href{http://ctan.org/pkg/standalone}{\textsf{standalone}}
provide commands to include only the document body of
a child file thus allowing both files to be compiled individually.
\item
The packages \href{http://ctan.org/pkg/subdocs}{\textsf{subdocs}}
and \href{http://ctan.org/pkg/subfiles}{\textsf{subfiles}}
provide structures in which the main and child documents can be
encapsulated and allowing them to be compiled individually.
The inclusion mechanism is different from the conventional |\include|.
\item
The package \href{http://ctan.org/pkg/combine}{\textsf{combine}}
is an elaborate solution to combine several documents into one.
\end{itemize}
%
See also the CTAN topic \href{http://ctan.org/topic/subdocs}{\textsf{subdocs}}
for further related packages.
The present package differs from the above solutions in that
a document structure constructed with the conventional |\include| mechanism
just needs two extra commands at the top of every file
such that all constituent files can be compiled individually.

%%%%%%%%%%%%%%%%%%%%%%%%%%%%%%%%%%%%%%%%%%%%%%%%%%%%%%%%%%%%%%%%%%%%%%%%%%%%%%%%
%\subsection{Feature Suggestions}
%
%The following is a list of features which may be useful for future
%versions of this package:
%%
%\begin{itemize}
%\item
%\ldots
%\end{itemize}

%%%%%%%%%%%%%%%%%%%%%%%%%%%%%%%%%%%%%%%%%%%%%%%%%%%%%%%%%%%%%%%%%%%%%%%%%%%%%%%%
\subsection{Revision History}

%%%%%%%%%%%%%%%%%%%%%%%%%%%%%%%%%%%%%%%%
\paragraph{v2.0:} 2018/12/30

\begin{itemize}
\item
immediate forward processing
\item
added |\childdocby| mechanism
\item
manual restructured
\end{itemize}

%%%%%%%%%%%%%%%%%%%%%%%%%%%%%%%%%%%%%%%%
\paragraph{v1.6:} 2018/01/17

\begin{itemize}
\item
application for development of include files
\item
corrections to manual
\end{itemize}

%%%%%%%%%%%%%%%%%%%%%%%%%%%%%%%%%%%%%%%%
\paragraph{v1.5:} 2017/05/21

\begin{itemize}
\item
more complete structuring introduced
\item
|\childdocof| introduced
\item
|\childdoc| renamed to |\childdocmain|
\item
|\childredirect| renamed to |\childdocforward| and |\childdocforwardprefix|
and functionality expanded
\end{itemize}

%%%%%%%%%%%%%%%%%%%%%%%%%%%%%%%%%%%%%%%%
\paragraph{v1.0:} 2017/04/27

\begin{itemize}
\item
manual and install package
\item
first version published on CTAN
\end{itemize}

%%%%%%%%%%%%%%%%%%%%%%%%%%%%%%%%%%%%%%%%
\paragraph{v0.6:} 2017/04/26

\begin{itemize}
\item
redirection mechanism added
\end{itemize}

%%%%%%%%%%%%%%%%%%%%%%%%%%%%%%%%%%%%%%%%
\paragraph{v0.5:} 2017/04/26

\begin{itemize}
\item
functionality in definition file
\end{itemize}


%%%%%%%%%%%%%%%%%%%%%%%%%%%%%%%%%%%%%%%%%%%%%%%%%%%%%%%%%%%%%%%%%%%%%%%%%%%%%%%%
%%%%%%%%%%%%%%%%%%%%%%%%%%%%%%%%%%%%%%%%%%%%%%%%%%%%%%%%%%%%%%%%%%%%%%%%%%%%%%%%
%%%%%%%%%%%%%%%%%%%%%%%%%%%%%%%%%%%%%%%%%%%%%%%%%%%%%%%%%%%%%%%%%%%%%%%%%%%%%%%%
\appendix

\settowidth\MacroIndent{\rmfamily\scriptsize 000\ }

 \DocInput{childdoc.dtx}

\end{document}
%</driver>
% \fi
%
% %%%%%%%%%%%%%%%%%%%%%%%%%%%%%%%%%%%%%%%%%%%%%%%%%%%%%%%%%%%%%%%%%%%%%%%%%%%%%%
% %%%%%%%%%%%%%%%%%%%%%%%%%%%%%%%%%%%%%%%%%%%%%%%%%%%%%%%%%%%%%%%%%%%%%%%%%%%%%%
% \section{Sample}
%\iffalse
%<*samplemain>
%\fi
%
% The following presents a sample document
% with two chapters, two parts, a title page,
% a compile flag as well as three forwarding files to set the flag.
% It consists of eight |.tex| files:
% \begin{center}
% \begin{tabular}{ll}
% |cdocsamp.tex|&main file\\
% |cdocsch1.tex|&include file for chapter 1\\
% |cdocsch2.tex|&include file for chapter 2\\
% |cdocspt3.tex|&include file for part 3\\
% |cdocspt4.tex|&include file for part 4\\
% |cdocsdrf.tex|&forwarding file for main file in draft mode\\
% |cdocsfi1.tex|&forwarding file for final version of chapter 1\\
% |cdocsfi2.tex|&forwarding file for final version of chapter 2\\
% \end{tabular}
% \end{center}
% Each of the eight files can be compiled directly by the \LaTeX{} compiler.
%
% %%%%%%%%%%%%%%%%%%%%%%%%%%%%%%%%%%%%%%
% \paragraph{Main File.}
%
% The main file is called |cdocsamp.tex|.
%
% Load the \textsf{childdoc} definitions and
% declare the filename for the main document:
%    \begin{macrocode}
\input{childdoc.def}
\childdocmain{}
%    \end{macrocode}

% Optional override for |\version| flag:
%    \begin{macrocode}
%%\ifchilddoc\else\providecommand{\version}{draft}\fi
%    \end{macrocode}

% Define the default values for the |\version| flag
% (|final| for the main file and |draft| for childs):
%    \begin{macrocode}
\ifchilddoc
\providecommand{\version}{draft}
\else
\providecommand{\version}{final}
\fi
%    \end{macrocode}

% Load the standard document class:
%    \begin{macrocode}
\documentclass[12pt]{article}
%    \end{macrocode}

% Start the document body:
%    \begin{macrocode}
\begin{document}
%    \end{macrocode}

% Declare a title page.
% Print title, part of document being processed and version flag:
%    \begin{macrocode}
\addtocounter{page}{-1}
\begin{center}
{\LARGE\bfseries{}childdoc example\par}
\vspace{1cm}
\ifchilddoc
\ifchilddocmanual part\else chapter\fi:
`\childdocname' of `\childdocjob'\par
\else
main document: `\childdocjob'\par
\fi
version: \version\par
\end{center}
\newpage
%    \end{macrocode}

% Manually include selected file,
% otherwise process as usual:
%    \begin{macrocode}
\ifchilddocmanual
\section*{part `\childdocname'}
\input{\childdocname}
\else
%    \end{macrocode}

% Include the two chapters:
%    \begin{macrocode}
\include{cdocsch1}
\include{cdocsch2}
%    \end{macrocode}

% Include the two parts unless only chapters should be displayed:
%    \begin{macrocode}
\ifchilddoc\else
\section{part three}
\input{cdocspt3}
\section{part four}
\input{cdocspt4}
\fi
%    \end{macrocode}

% Process as usual until here:
%    \begin{macrocode}
\fi
%    \end{macrocode}

% End of document body:
%    \begin{macrocode}
\end{document}
%    \end{macrocode}
%\iffalse
%</samplemain>
%\fi
%
% %%%%%%%%%%%%%%%%%%%%%%%%%%%%%%%%%%%%%%
% \paragraph{Chapter Include Files.}
%
% The include files are called |cdocsch1.tex| and |cdocsch2.tex|.
%
%\iffalse
%<*samplechap1|samplechap2>
%\fi

% Optional override for |\version| flag:
%    \begin{macrocode}
%%\providecommand{\version}{final}
%    \end{macrocode}

% Include the main document:
%    \begin{macrocode}
\input{childdoc.def}
\childdocof{cdocsamp}
%    \end{macrocode}

%\iffalse
%</samplechap1|samplechap2>
%\fi
%
%\iffalse
%<*samplechap1>
%\fi
% Some text for chapter 1:
%    \begin{macrocode}
\section{one}
some text in chapter one
%    \end{macrocode}

%\iffalse
%</samplechap1>
%\fi
% Some text for chapter 2:
%\iffalse
%<*samplechap2>
%\fi
%    \begin{macrocode}
\section{two}
more text in chapter two
%    \end{macrocode}

%\iffalse
%</samplechap2>
%\fi
%
% %%%%%%%%%%%%%%%%%%%%%%%%%%%%%%%%%%%%%%
% \paragraph{Part Include Files.}
%
% The include files are called |cdocspt3.tex| and |cdocspt4.tex|.
%
%\iffalse
%<*samplepart3|samplepart4>
%\fi

% Optional override for |\version| flag:
%    \begin{macrocode}
%%\providecommand{\version}{final}
%    \end{macrocode}

% Include the main document:
%    \begin{macrocode}
\input{childdoc.def}
\childdocby{cdocsamp}
%    \end{macrocode}

%\iffalse
%</samplepart3|samplepart4>
%\fi
%
%\iffalse
%<*samplepart3>
%\fi
% Some text for part 3:
%    \begin{macrocode}
some text in part three
%    \end{macrocode}

%\iffalse
%</samplepart3>
%\fi
% Some text for part 4:
%\iffalse
%<*samplepart4>
%\fi
%    \begin{macrocode}
more text in part four
%    \end{macrocode}

%\iffalse
%</samplepart4>
%\fi
%
% %%%%%%%%%%%%%%%%%%%%%%%%%%%%%%%%%%%%%%
% \paragraph{Forwarding for a Complete Draft.}
%
% The following forwarding file |cdocsdrf.tex|
% compiles the main document in draft mode:
%\iffalse
%<*sampledraft>
%\fi
%    \begin{macrocode}
\def\version{draft}
\input{childdoc.def}
\childdocforward{cdocsamp}
%    \end{macrocode}

%\iffalse
%</sampledraft>
%\fi
%
% %%%%%%%%%%%%%%%%%%%%%%%%%%%%%%%%%%%%%%
% \paragraph{Forwarding for Final Version of the Chapters.}
%
% The following forwarding files |cdocsfn1.tex| and |cdocsfn2.tex|
% (with identical content)
% compile the final versions of the child documents
% |cdocsch1.tex| and |cdocsch2.tex|, respectively:
%\iffalse
%<*samplefinal>
%\fi
%    \begin{macrocode}
\def\version{final}
\input{childdoc.def}
\childdocforwardprefix[cdocsamp]{cdocsfn}{cdocsch}
%    \end{macrocode}

%\iffalse
%</samplefinal>
%\fi
%
% %%%%%%%%%%%%%%%%%%%%%%%%%%%%%%%%%%%%%%
% \paragraph{Command Line Processing.}
%
% The following three command lines generate the output files
% |cdocscld|, |cdocscl1| and |cdocscl2|
% which should be identical to
% |cdocsdrf|, |cdocsch1| and |cdocsfn2|, respectively:
% \begin{center}
% \begin{tabular}{l}
% |latex -jobname cdocscld \|\\
% |  "\def\version{draft}\input{childdoc.def}\childdocforward{cdocsamp}"|\\
% |latex -jobname cdocscl1 \|\\
% |  "\input{childdoc.def}\childdocforward[cdocsamp]{cdocsch1}"|\\
% |latex -jobname cdocscl2 \|\\
% |  "\def\version{final}\input{childdoc.def}\childdocforward{cdocsch2}"|
% \end{tabular}
% \end{center}
% Note that the trailing backslash on each first line
% merely continues the input to the second line
% (for convenient cut ant paste).
% Furthermore, the command |latex| can be replaced by any
% of its alternative versions such as |pdflatex|.
%
% %%%%%%%%%%%%%%%%%%%%%%%%%%%%%%%%%%%%%%%%%%%%%%%%%%%%%%%%%%%%%%%%%%%%%%%%%%%%%%
% %%%%%%%%%%%%%%%%%%%%%%%%%%%%%%%%%%%%%%%%%%%%%%%%%%%%%%%%%%%%%%%%%%%%%%%%%%%%%%
% \section{Implementation}
%\iffalse
%<*package>
%\fi
%
% This section describes the definitions file |childdoc.def|.

% The definitions cannot be loaded using |\usepackage| or |\RequirePackage|
% which has a mechanism to prevent loading a style file more than once.
% When loading the definitions by means of |\input|
% multiple instances have to be prevented manually:
%\iffalse
%This code needs to be before the `\ProvidesFile' directive
%which is defined at the beginning of this file.
%Therefore it is also placed there and commented out here.
%</package>
%<*discard>
%\fi
%    \begin{macrocode}
\ifdefined\childdocmain\endinput\fi
%    \end{macrocode}
%\iffalse
%</discard>
%<*package>
%\fi
%
% \macro{\ifchilddoc}
% \macro{\ifchilddocmanual}
% The conditional |\ifchilddoc| tells whether a
% child (true) or main (false) document is being compiled.
% The conditional |\ifchilddocmanual| tells whether
% the |\includeonly| mechanism is used (false) or
% the selection of child files must be performed manually (true).
% The definitions initialise to false:
%    \begin{macrocode}
\newif\ifchilddoc
\newif\ifchilddocmanual
%    \end{macrocode}

% \macro{\childdocname}
% \macro{\childdocjob}
% The macro |\childdocname| stores the name of the main document
% to be compiled. The macro |\childdocjob| stores the name of
% the document on which the \LaTeX{} compiler was originally invoked.
% The content of |\jobname| cannot be compared
% to filenames specified in the source due to different catcodes.
% The following code rescans |\jobname|, stores the result
% in |\childdocname| and saves a copy in |\childdocjob|:
%    \begin{macrocode}
\edef\childdocname{\scantokens\expandafter{\jobname\noexpand}}
\let\childdocjob\childdocname
%    \end{macrocode}

% \macro{\childdocdisable}
% The macro |\childdocdisable| prevents the main file
% from being processed more than once.
% At this stage, the main document command |\childdocmain|
% is assumed to be called once again where it should do nothing.
% Any subsequent call to it should prevent
% a secondary processing of the main document
% It overwrites the forwarding commands
% |\childdocof| and |\childdocforward|
% with empty macros to prevent further inclusions of the main document:
%    \begin{macrocode}
\newcommand{\childdocdisable}
{
  \renewcommand{\childdocmain}[1]{\renewcommand{\childdocmain}[1]{\endinput}}
  \renewcommand{\childdocof}[1]{}
  \renewcommand{\childdocby}[2][]{}
  \renewcommand{\childdocforward}[2][]{}
  \renewcommand{\childdocdisable}{}
}
%    \end{macrocode}

% \macro{\childdocmain}
% The macro |\childdocmain| is to be called at the top of the main file
% with nothing or the main filename (without extension) as argument.
% First, it breaks loops.
% If the argument is not empty and does not match |\childdocname|
% (which is set by the first inclusion of |childdoc.def|),
% |\ifchilddoc| is set to true, |\includeonly| is applied to the child file
% and |\jobname| is set to the main file
% (for proper handling of |.aux| files):
%    \begin{macrocode}
\newcommand{\childdocmain}[1]
{
  \childdocdisable\childdocmain{}
  \if?#1?\else
    \begingroup
      \def\childdoctmp{#1}
      \ifx\childdoctmp\childdocname
        \def\childdoctmp{}
      \else
        \def\childdoctmp
        {
          \childdoctrue
          \includeonly{\childdocname}
          \def\childdocjob{#1}
          \def\jobname{#1}
        }
      \fi
      \expandafter
    \endgroup
    \childdoctmp
  \fi
}
%    \end{macrocode}

% \macro{\childdocof}
% The command |\childdocof| redirects
% compilation to the main file |#1|.
%    \begin{macrocode}
\newcommand{\childdocof}[1]
{
  \childdocdisable
  \childdoctrue
  \includeonly{\childdocname}
  \def\jobname{#1}
  \def\childdocjob{#1}
  \input{#1}
}
%    \end{macrocode}

% \macro{\childdocby}
% The command |\childdocby| ....
%    \begin{macrocode}
\newcommand{\childdocby}[2][]
{
  \childdocdisable
  \childdoctrue
  \childdocmanualtrue
  \if?#1?\else
    \def\jobname{#2}
  \fi
  \def\childdocjob{#2}
  \input{#2}
  \endinput
}
%    \end{macrocode}

% \macro{\childdocforward}
% The command |\childdocforward| redirects
% compilation to the main file or
% (if the optional argument is given) a child file.
% Parameters are set as if the main file
% or a child file starting with |\childdocof| was compiled.
% Then compilation is handed over to the main file:
%    \begin{macrocode}
\newcommand{\childdocforward}[2][]
{
  \begingroup
    \if?#1?
      \def\childdoctmp
      {
        \def\childdocname{#2}
        \def\childdocjob{#2}
        \def\jobname{#2}
        \input{#2}
        \endinput
      }
    \else
      \def\childdoctmp
      {
        \childdocdisable
        \def\childdocname{#2}
        \childdoctrue
        \includeonly{#2}
        \def\childdocjob{#1}
        \def\jobname{#1}
        \input{#1}
        \endinput
      }
    \fi
    \expandafter
  \endgroup
  \childdoctmp
}
%    \end{macrocode}

% \macro{\childdocforwardprefix}
% The command |\childdocforwardprefix| redirects
% compilation to the main or a child file by means of a pattern.
% The prefix |#1| in the current filename is replaced by |#2|
% and the suffix of the current filename is kept
% (it is assumed that the filename does not contain the substring `|~~~|'
% which is used as a delimiter).
% Compilation is handed over to the new file by |\childdocforward|:
%    \begin{macrocode}
\newcommand{\childdocforwardprefix}[3][]
{
  \begingroup
    \def\childdocextract #2##1~~~{\def\childdoctmp{\childdocforward[#1]{#3##1}}}
    \expandafter\childdocextract\childdocname~~~
    \expandafter
  \endgroup
  \childdoctmp
}
%    \end{macrocode}

% \macro{\childdoc}
% The deprecated macro |\childdoc| is a legacy version of |\childdocmain|:
%    \begin{macrocode}
\newcommand{\childdoc}{\childdocmain}
%    \end{macrocode}

% \macro{\childdocredirect}
% The deprecated macro |\childdocredirect| is a legacy version
% of |\childdocforward| and |\childdocforwardprefix|:
%    \begin{macrocode}
\newcommand{\childdocredirect}[2][]
{
  \begingroup
    \if?#1?
      \def\childdoctmp{\childdocforward{#2}}
    \else
      \def\childdoctmp{\childdocforwardprefix{#1}{#2}}
    \fi
    \expandafter
  \endgroup
  \childdoctmp
}
%    \end{macrocode}

%\iffalse
%</package>
%\fi
%
\endinput
|\\
|\childdocby{|\textit{main}|}|\\
\end{tabular}
\end{center}
%
The directive |\childdocby| is similar to |\childdocof|
described in \secref{sec:include},
but the subsequent selection of content must be done manually.
To that end, both |\ifchilddoc| and |\ifchilddocmanual|
will be true upon processing of a part,
and the name of the part is stored in |\childdocname|.
Note that |\jobname| will be set to the filename of the current part
so that each part receives an individual |.aux| file
that does not interfere with the |.aux| file(s) of the main document.
This behaviour can be altered by the alternative form
|\childdocby[*]{|\textit{main}|}| (with a non-empty optional argument)
which uses the |.aux| file of the main document
by setting |\jobname| to \textit{main}.

%%%%%%%%%%%%%%%%%%%%%%%%%%%%%%%%%%%%%%%%%%%%%%%%%%%%%%%%%%%%%%%%%%%%%%%%%%%%%%%%
\subsection{Driver Development}
\label{sec:driver}

The \textsf{childdoc} mechanism can also be use for the development
of definition files such as \LaTeX{} styles or classes.
This case differs from the above setup with multiple parts
included by |\include| in that no |\includeonly| should be invoked.
This can be achieved by starting the include file
(before |\ProvidesPackage|) with:
%
\begin{center}
\begin{tabular}{l}
|% \iffalse
%
% childdoc.dtx Copyright (C) 2017-2018 Niklas Beisert
%
% This work may be distributed and/or modified under the
% conditions of the LaTeX Project Public License, either version 1.3
% of this license or (at your option) any later version.
% The latest version of this license is in
%   http://www.latex-project.org/lppl.txt
% and version 1.3 or later is part of all distributions of LaTeX
% version 2005/12/01 or later.
%
% This work has the LPPL maintenance status `maintained'.
%
% The Current Maintainer of this work is Niklas Beisert.
%
% This work consists of the files childdoc.dtx and childdoc.ins
% and the derived files childdoc.def and cdocsamp.tex with
% cdocsch1.tex, cdocsch2.tex, cdocsdrf.tex, cdocsfn1.tex, cdocsfn2.tex.
%
%<package>\ifdefined\childdocmain\endinput\fi
%<package>\ProvidesFile{childdoc.def}[2018/12/30 v2.0 child document driver]
%<samplemain>\ProvidesFile{cdocsamp.tex}[2018/12/30 v2.0 sample for childdoc]
%<*driver>
%\ProvidesFile{childdoc.drv}[2018/12/30 v2.0 childdoc reference manual file]
\PassOptionsToClass{10pt,a4paper}{article}
\documentclass{ltxdoc}

\usepackage[margin=35mm]{geometry}
\usepackage{hyperref}
\usepackage{hyperxmp}
\usepackage[usenames]{color}

\hypersetup{colorlinks=true}
\hypersetup{pdfstartview=FitH}
\hypersetup{pdfpagemode=UseNone}
\hypersetup{pdfsource={}}
\hypersetup{pdflang={en-UK}}
\hypersetup{pdfcopyright={Copyright 2017-2018 Niklas Beisert.
  This work may be distributed and/or modified under the
  conditions of the LaTeX Project Public License, either version 1.3
  of this license or (at your option) any later version.}}
\hypersetup{pdflicenseurl={http://www.latex-project.org/lppl.txt}}
\hypersetup{pdfcontactaddress={ETH Zurich, ITP, HIT K,
  Wolfgang-Pauli-Strasse 27}}
\hypersetup{pdfcontactpostcode={8093}}
\hypersetup{pdfcontactcity={Zurich}}
\hypersetup{pdfcontactcountry={Switzerland}}
\hypersetup{pdfcontactemail={nbeisert@itp.phys.ethz.ch}}
\hypersetup{pdfcontacturl={http://people.phys.ethz.ch/\xmptilde nbeisert/}}

\newcommand{\secref}[1]{\hyperref[#1]{section \ref*{#1}}}

\parskip1ex
\parindent0pt
\let\olditemize\itemize
\def\itemize{\olditemize\parskip0pt}

\begin{document}

\title{The \textsf{childdoc} Package}
\hypersetup{pdftitle={The childdoc Package}}
\author{Niklas Beisert\\[2ex]
  Institut f\"ur Theoretische Physik\\
  Eidgen\"ossische Technische Hochschule Z\"urich\\
  Wolfgang-Pauli-Strasse 27, 8093 Z\"urich, Switzerland\\[1ex]
  \href{mailto:nbeisert@itp.phys.ethz.ch}
  {\texttt{nbeisert@itp.phys.ethz.ch}}}
\hypersetup{pdfauthor={Niklas Beisert}}
\hypersetup{pdfsubject={Manual for the LaTeX2e Package childdoc}}
\date{30 December 2018, \textsf{v2.0}}
\maketitle

\begin{abstract}\noindent
\textsf{childdoc} is a \LaTeXe{} package
that enables the direct compilation
of document sections included by |\include|
to individual files.
\end{abstract}

\begingroup
\parskip0ex
\tableofcontents
\endgroup

%%%%%%%%%%%%%%%%%%%%%%%%%%%%%%%%%%%%%%%%%%%%%%%%%%%%%%%%%%%%%%%%%%%%%%%%%%%%%%%%
%%%%%%%%%%%%%%%%%%%%%%%%%%%%%%%%%%%%%%%%%%%%%%%%%%%%%%%%%%%%%%%%%%%%%%%%%%%%%%%%
\section{Introduction}

\LaTeX{} provides a mechanism to structure a large document (such as a book)
into a main file and several child files (containing the chapters)
using the |\include| command.
This mechanism is beneficial for documents
which span hundreds of pages in order to
make the source file(s) more manageable.
Moreover, compilation can be restricted to
selected child files by means of the |\includeonly| command.
The latter feature can be used to reduce the compilation time while editing
(this was significantly more useful in the earlier days of \LaTeX{})
or to generate a smaller document which is easier to navigate.
Another application of |\includeonly| is to generate
documents consisting of selected parts of the complete document.

However, there are a few drawbacks of the plain |\include| mechanism:
\begin{itemize}
\item
The child files cannot be compiled on their own,
they can only be compiled via the main file.
A naive editing environment
(such as a text editor with an option
to have the current file processed by \LaTeX)
may require one to switch to the main file before compiling;
attempting to compile the child file produces errors.
\item
The main file must be modified (each time)
to adjust the |\includeonly| command
to the present needs. This easily leaves the main file in a messy state.
\item
The generated document will always carry the filename
of the main document. This is inconvenient if
several child files are to be compiled and
to be kept for distribution.
\end{itemize}

The present package provides a simple interface
to make child files individually compilable by \LaTeX{}.
Compiling a child file then has the same effect as compiling
the main file with an |\includeonly| command
to select the appropriate child.
Moreover the generated document will carry the name of the child
rather than the main file.
This resolves all three above issues.

This feature is meant to make the editing of books,
thesis documents and lecture notes somewhat more convenient.
However, the package can also be used efficiently for
composing a series of documents (such as exercise sheets)
which are typically distributed individually.
It then assists the author in generating the individual documents
(potentially in different versions)
as well as a document containing the collected series.
Another application is in developing style files
or other kinds of included material
where compilation of the style file could redirect
to a sample or test file.

%%%%%%%%%%%%%%%%%%%%%%%%%%%%%%%%%%%%%%%%%%%%%%%%%%%%%%%%%%%%%%%%%%%%%%%%%%%%%%%%
%%%%%%%%%%%%%%%%%%%%%%%%%%%%%%%%%%%%%%%%%%%%%%%%%%%%%%%%%%%%%%%%%%%%%%%%%%%%%%%%
\section{Usage}

First of all, the package \textsf{childdoc} is \emph{not} a standard
\LaTeXe{} |.sty| style file! Therefore it needs to be invoked in
a non-standard way.

%%%%%%%%%%%%%%%%%%%%%%%%%%%%%%%%%%%%%%%%%%%%%%%%%%%%%%%%%%%%%%%%%%%%%%%%%%%%%%%%
\subsection{Included Files}
\label{sec:include}

%%%%%%%%%%%%%%%%%%%%%%%%%%%%%%%%%%%%%%%%
\DescribeMacro{\childdocmain}
To use the package, add the commands
\begin{center}
\begin{tabular}{l}
|\input{childdoc.def}|\\
|\childdocmain{}|\\
\end{tabular}
\end{center}
at the very top of the main \LaTeX{} file,
in particular \emph{before} the |\documentclass| statement!
The argument of |\childdocmain| should be left empty
(but it must be present).

%%%%%%%%%%%%%%%%%%%%%%%%%%%%%%%%%%%%%%%%
\DescribeMacro{\childdocof}
Furthermore, add the commands
\begin{center}
\begin{tabular}{l}
|\input{childdoc.def}|\\
|\childdocof{|\textit{main}|}|\\
\end{tabular}
\end{center}
at the top of every child file \textit{child}
which is included by |\include{|\textit{child}|}|
from within the main file
(or at least for those files to be compiled individually).
The argument \textit{main} must be the filename of the main file.

There are a couple of
considerations in setting up the main and child documents:

%%%%%%%%%%%%%%%%%%%%%%%%%%%%%%%%%%%%%%%%
\paragraph{Restrictions.}

Please note the following restrictions:
\begin{itemize}
\item
|\childdocmain| must be called with one argument \textit{main}
to ensure compatibility with earlier version of the package.
It must either be empty (|\childdocmain{}|)
or precisely match the filename of the main file in which it is specified.
See \secref{sec:detection} for further information.
\item
The filename \textit{main} must be specified without the |.tex| extension.
\item
The filename \textit{main} is case sensitive
(even in case-insensitive file systems)
due to internal string comparison.
\item
The argument \textit{main} should be fully expanded, it cannot be a macro.
\item
Subdirectories and special characters should be avoided in filenames.
\item
The command |\childdocmain{|\textit{main}|}| must be followed by a whitespace.
It should not be followed immediately by another command
or by a comment mark `|%|'.
This is because the \TeX{} parser reads the token immediately following
the argument of |\childdocmain| and puts it
at the beginning of every child section;
however, a white\-space is ignored.
\end{itemize}

%%%%%%%%%%%%%%%%%%%%%%%%%%%%%%%%%%%%%%%%
\paragraph{Content of Main File.}

It is advisable to place all content in the child files included by |\include|.
Any output contained in the main file will appear in all child documents
unless suppressed manually;
it cannot be suppressed automatically by the |\includeonly| directive
and thus should normally be avoided.
A method to include some content in the main file
by means of conditional processing is described in \secref{sec:conditional}.

%%%%%%%%%%%%%%%%%%%%%%%%%%%%%%%%%%%%%%%%
\paragraph{Page Numbering.}

When only a part of the document is compiled,
the appropriate numbering of pages
(as well as other status parameters)
is determined from the |.aux| files.
The latter contain information from previous passes.
However this information needs to propagate through
all intermediate child documents.
Therefore the page numbering in child documents may well
be inconsistent until the complete document is compiled at least once.

A useful (if unconventional) way to always ensure a consistent
page numbering is to restart the numbering in each child document
and denote the pages by `\textit{child}|.|\textit{page}'
where \textit{child} represents the chapter/section number of the child file.
This can be achieved by the command
|\numberwithin{page}{|\textit{child}|}|
of the \textsf{amsmath} package
where \textit{child} can be |chapter| or |section|
depending on the chosen structuring.
Alternatively, one can modify the macro |\thepage| appropriately
and reset the counter |page| at the start of each child file.

%%%%%%%%%%%%%%%%%%%%%%%%%%%%%%%%%%%%%%%%%%%%%%%%%%%%%%%%%%%%%%%%%%%%%%%%%%%%%%%%
\subsection{Conditional Processing}
\label{sec:conditional}

The package provides a mechanism to compile different versions
of a document. To customise the versions further some conditional processing
can come in handy to distinguish which version is being compiled.
The package provides two macros to describe the compilation context:

%%%%%%%%%%%%%%%%%%%%%%%%%%%%%%%%%%%%%%%%
\DescribeMacro{\ifchilddoc}
The conditional |\ifchilddoc| distinguishes between the compilation of
child documents and the main document:
%
\begin{center}
|\ifchilddoc |\textit{child-code}| |[|\||else |\textit{main-code}]| \||fi|
\end{center}

%%%%%%%%%%%%%%%%%%%%%%%%%%%%%%%%%%%%%%%%
\DescribeMacro{\childdocname}
\DescribeMacro{\childdocjob}
The macro |\childdocname| contains the filename (without extension)
of the main or child file being processed.
Note that |\childdocjob| will always contain the name of the main file.

%%%%%%%%%%%%%%%%%%%%%%%%%%%%%%%%%%%%%%%%
\paragraph{Title Page.}

Conditional processing can be used to include a title or banner page
in the main document when proper precautions are taken.
Importantly, the code in the main file should ensure that the page counter
(as well as other status parameters which are stored in the |.aux| files)
takes the same value after the conditional processing.
Otherwise the page numbers may take divergent values
depending on which part is compiled.

For example, a title page could be declared by:
%
\begin{center}
\begin{tabular}{l}
|\ifchilddoc\||else|\\
|\addtocounter{page}{-1}|\\
\textit{code for title page}\\
|\newpage|\\
|\||fi|
\end{tabular}
\end{center}
%
A banner page for the child documents can be generated by:
%
\begin{center}
\begin{tabular}{l}
|\ifchilddoc|\\
|\addtocounter{page}{-1}|\\
\textit{code for banner page}\\
|\newpage|\\
|\||fi|
\end{tabular}
\end{center}
%
Here one could write a message such as:
\begin{center}
|This is the part \childdocname{} of \childdocjob{}.|
\end{center}

%%%%%%%%%%%%%%%%%%%%%%%%%%%%%%%%%%%%%%%%%%%%%%%%%%%%%%%%%%%%%%%%%%%%%%%%%%%%%%%%
\subsection{Flags}
\label{sec:flags}

The package makes it easy to generate different versions
of the main or child documents.
To this end compilation flags can be defined
and assigned different default values.
They will be particularly useful in conjunction
with the forwarding mechanism described in \secref{sec:forward}.

For example, it may be useful to have a flag |\version|
which can be set to |draft| or |final|.
The document source will contain some conditional code
depending on the value of |\version|.
Suppose further, the flag should default to |final| for the main file
and to |draft| for child files
which is a natural assignment for editing the document.
This is achieved by placing the following code
in the preamble of the main document
(below the |\childdocmain| directive):
%
\begin{center}
\begin{tabular}{l}
|\ifchilddoc|\\
|\providecommand{\version}{draft}|\\
|\||else|\\
|\providecommand{\version}{final}|\\
|\||fi|
\end{tabular}
\end{center}
%
The definition by |\providecommand| makes sure
that previous definitions are not overwritten.
Further statements |\providecommand{\version}{...}|
can thus be added before the above code to override it.

For the main file, one might add a line
(between |\childdocmain| and the above block)
%
\begin{center}
|%\ifchilddoc\||else\providecommand{\version}{draft}\||fi|
\end{center}
%
which can be uncommented to produce a draft version.
Likewise one can add a line to the very top of a child file
(above the |\childdocof{|\textit{main}|}| directive)
%
\begin{center}
|%\providecommand{\version}{final}|
\end{center}
%
which can be uncommented to produce the final version of this child document.

%%%%%%%%%%%%%%%%%%%%%%%%%%%%%%%%%%%%%%%%%%%%%%%%%%%%%%%%%%%%%%%%%%%%%%%%%%%%%%%%
\subsection{Forwarding}
\label{sec:forward}

Different versions of the main or child documents
using compilation flags as described in \secref{sec:flags}
can be (permanently) stored in different files
for convenient compilation, viewing and distribution.
To this end, the package defines a command
to pass on compilation to a different file:

%%%%%%%%%%%%%%%%%%%%%%%%%%%%%%%%%%%%%%%%
\DescribeMacro{\childdocforward}
The command |\childdocforward| redirects processing to
another source file:
%
\begin{center}
\begin{tabular}{l}
|\input{childdoc.def}|\\
|\childdocforward[|\textit{main}|]{|\textit{dest}|}|\\
\end{tabular}
\end{center}
%
The argument \textit{dest} is the destination file
(without extension).
It should be the main file or one of the child files.
Note that further \textsf{childdoc} directives
such as |\childdocof| and |\childdocforward|
in the indicated file will be processed in this form.
The optional argument \textit{main}
passes on directly to the main file \textit{main}
while pretending to compile the child \textit{dest}.
This form behaves as if \textit{dest}
issues |\childdocof{|\textit{main}|}| right away,
and no further \textsf{childdoc} directives will be processed.

%%%%%%%%%%%%%%%%%%%%%%%%%%%%%%%%%%%%%%%%
\DescribeMacro{\...prefix}
In the alternative form |\childdocforwardprefix|,
%
\begin{center}
\begin{tabular}{l}
|\input{childdoc.def}|\\
|\childdocforwardprefix[|\textit{main}|]{|\textit{prefix}|}{|\textit{dest}|}|
\end{tabular}
\end{center}
%
the destination file is determined by a pattern
depending on the current file:
To make this work, the current file must be called
`{\textit{prefix}\hspace{0.2em}\textit{suffix}}'
with \textit{prefix} matching precisely the argument.
Processing is then passed on to the file
`{\textit{dest}\hspace{0.2em}\textit{suffix}}'.
Surely, the same effect is achieved by
directly specifying the
argument `{\textit{dest}\hspace{0.2em}\textit{suffix}}'
in the first form.
However, that requires to set up a different file
for each child. With the alternative form of the command
all these files can have exactly the same content
which simplifies setting them up and maintaining them.

For example, the following file |draft.tex|
with a compilation flag |\version| as described in \secref{sec:flags}
compiles the main document as a draft:
%
\begin{center}
\begin{tabular}{l}
|\def\version{draft}|\\
|\input{childdoc.def}|\\
|\childdocforward{|\textit{main}|}|
\end{tabular}
\end{center}
%
Likewise, the following files |final|\textit{nn}|.tex|
compile the final version of the child document
|child|\textit{nn}|.tex|:
%
\begin{center}
\begin{tabular}{l}
|\def\version{final}|\\
|\input{childdoc.def}|\\
|\childdocforwardprefix{final}{child}|
\end{tabular}
\end{center}
%

Note that when several versions of a main file and/or of each child file
are to be generated, it may be convenient to set up a |Makefile| or
shell script to automatise the process.

%%%%%%%%%%%%%%%%%%%%%%%%%%%%%%%%%%%%%%%%%%%%%%%%%%%%%%%%%%%%%%%%%%%%%%%%%%%%%%%%
\subsection{Command Line Processing}
\label{sec:commandline}

The effect of redirection files can also be achieved by invoking
the \LaTeX{} compiler with a more elaborate command line.
Most conveniently this should be done as part
of a shell script or a |Makefile|.

When using \textsf{childdoc} in the main file, the following
command lines effectively perform a redirection
(note that depending on the shell being used,
backslashes may have to be doubled: `|\|' $\to$ `|\\|'):
%
\begin{center}
|... -jobname "|\textit{target}|" |\\|"|[\textit{flags}]%
|\input{childdoc.def}\childdocforward[|\textit{main}|]{|\textit{dest}|}"|
\end{center}
%
Here \textit{target} is the name of the output file,
\textit{main} is the name of the main file
and \textit{dest} is the name of the main or child file to be processed
(all filenames without extensions).
The optional argument \textit{main} can be omitted
if \textit{main} matches \textit{dest}.
Optionally, compilation \textit{flags} can be defined via |\def| commands.
This command line makes the \TeX{} engine believe
it is compiling the file \textit{target}
whose content is specified as the latter parameter.
The provided code then forwards the processing to
\textit{main} or \textit{dest} as described in \secref{sec:forward}.

%%%%%%%%%%%%%%%%%%%%%%%%%%%%%%%%%%%%%%%%%%%%%%%%%%%%%%%%%%%%%%%%%%%%%%%%%%%%%%%%
\subsection{Include by Input}
\label{sec:input}

Including child documents by |\include| has some restrictions by design.
Most notably, the content of a child document always occupies
its own set of pages; pages cannot be shared between child documents.
Usually, this behaviour makes perfect sense
because each child document contain an essential part of the document.
However, in some situations it may be desirable to compose
a document from a collection of parts
without having mandatory page breaks between then.
For this case, the package
provides a mechanism to include parts
by |\input| which can also be processed individually.
However, by construction this mechanism
requires manual handling of the content to be output.

%%%%%%%%%%%%%%%%%%%%%%%%%%%%%%%%%%%%%%%%
\DescribeMacro{\ifchilddocmanual}
The main file should be prepared as usual, see \secref{sec:include}.
However, the document body must make a distinction
between processing of an individual part and of the main document, e.g.:
%
\begin{center}
\begin{tabular}{l}
|\ifchilddocmanual|\\
|\input{\childdocname}|\\
|\||else|\\
\textit{document body with }|\input{|\textit{part}|}|\\
|\||fi|
\end{tabular}
\end{center}
%
The conditional |\ifchilddocmanual| is true whenever
a part to be included by |\input| is being compiled,
and the name of the part is stored in |\childdocname|.

%%%%%%%%%%%%%%%%%%%%%%%%%%%%%%%%%%%%%%%%
\DescribeMacro{\childdocby}
Each part to be included by |\input| should start with:
%
\begin{center}
\begin{tabular}{l}
|\input{childdoc.def}|\\
|\childdocby{|\textit{main}|}|\\
\end{tabular}
\end{center}
%
The directive |\childdocby| is similar to |\childdocof|
described in \secref{sec:include},
but the subsequent selection of content must be done manually.
To that end, both |\ifchilddoc| and |\ifchilddocmanual|
will be true upon processing of a part,
and the name of the part is stored in |\childdocname|.
Note that |\jobname| will be set to the filename of the current part
so that each part receives an individual |.aux| file
that does not interfere with the |.aux| file(s) of the main document.
This behaviour can be altered by the alternative form
|\childdocby[*]{|\textit{main}|}| (with a non-empty optional argument)
which uses the |.aux| file of the main document
by setting |\jobname| to \textit{main}.

%%%%%%%%%%%%%%%%%%%%%%%%%%%%%%%%%%%%%%%%%%%%%%%%%%%%%%%%%%%%%%%%%%%%%%%%%%%%%%%%
\subsection{Driver Development}
\label{sec:driver}

The \textsf{childdoc} mechanism can also be use for the development
of definition files such as \LaTeX{} styles or classes.
This case differs from the above setup with multiple parts
included by |\include| in that no |\includeonly| should be invoked.
This can be achieved by starting the include file
(before |\ProvidesPackage|) with:
%
\begin{center}
\begin{tabular}{l}
|\input{childdoc.def}|\\
|\childdocforward{|\textit{main}|}|\\
\end{tabular}
\end{center}
%
or alternatively with:
%
\begin{center}
\begin{tabular}{l}
|\input{childdoc.def}|\\
|\childdocby{|\textit{main}|}|\\
\end{tabular}
\end{center}
%
Both forms have slightly different effects as described above.
The main file is prepared as usual, see \secref{sec:include}.

%%%%%%%%%%%%%%%%%%%%%%%%%%%%%%%%%%%%%%%%%%%%%%%%%%%%%%%%%%%%%%%%%%%%%%%%%%%%%%%%
\subsection{Legacy Detection}
\label{sec:detection}

The directive |\childdocmain| in the main file can detect
whether the complete document or merely a child is to be compiled
even without using the directive |\childdocof|.
This method is deprecated because it is less robust
and there is no compelling reason to use it;
it is merely provided for backward compatibility
and it may be removed in future versions.

If the detection mechanism is to be used,
it is mandatory to correctly specify
the filename of the main file as the argument of |\childdocmain|:
%
\begin{center}
\begin{tabular}{l}
|\input{childdoc.def}|\\
|\childdocmain{|\textit{main}|}|\\
\end{tabular}
\end{center}
%
If |\jobname| does not match the argument \textit{main} of |\childdocmain|,
it is assumed that |\jobname| points to the child file to be compiled.
When using |\childdocmain| with the main file specified as argument,
it suffices to start a child file
with just |\input{|\textit{main}|}|
without loading of the package and using |\childdocof|.
If instead all processing is done
with the appropriate \textsf{childdoc} directives,
the argument of \textit{main} of |\childdocmain| can be empty.

An alternative version of the command line processing described
in \secref{sec:commandline} using the detection mechanism reads:
%
\begin{center}
|... -jobname "|\textit{target}|" "|[\textit{flags}]%
[|\def\jobname{|\textit{dest}|}|]|\input{|\textit{main}|}"|
\end{center}

%%%%%%%%%%%%%%%%%%%%%%%%%%%%%%%%%%%%%%%%%%%%%%%%%%%%%%%%%%%%%%%%%%%%%%%%%%%%%%%%
\subsection{Manual Code}
\label{sec:manual}

In case one cannot be certain whether the definitions file |childdoc.def|
is installed on the target \TeX{} distribution
and one prefers not to ship it,
it is conceivable to paste a few relevant commands into the sources.

To that end, drop all statements |\input{childdoc.def}|
and perform the replacements as outlined below.
Instead of |\childdocmain{|\textit{main}|}| add the following code
to the top of the main file:
%
\begin{center}
\begin{tabular}{l}
|\||ifdefined\childdocname\endinput\||fi\newif\ifchilddoc|\\
|\edef\childdocname{\scantokens\expandafter{\jobname\noexpand}}|\\
|\def\childdocmain{|\textit{main}|}\||ifx\childdocmain\childdocname\||else|\\
|\childdoctrue\includeonly{\childdocname}\let\jobname\childdocmain\||fi|\\
\end{tabular}
\end{center}
%
Instead of |\childdocof{|\textit{main}|}| just include the main file
at the top of each child file:
%
\begin{center}
|\input{|\textit{main}|}|
\end{center}
%
A simple redirection |\childdocforward{|\textit{dest}|}| is achieved by:
%
\begin{center}
|\def\jobname{|\textit{dest}|}\input{\jobname}|
\end{center}
%
The redirection with prefix
|\childdocforwardprefix[|\textit{prefix}|]{|\textit{dest}|}|
is accomplished by:
%
\begin{center}
\begin{tabular}{l}
|{\edef\jobname{\scantokens\expandafter{\jobname\noexpand}}|\\
|\def\redirectjob |\textit{prefix}|#1~~~{\gdef\jobname{|\textit{dest}|#1}}|\\
|\expandafter\redirectjob\jobname~~~}\input{\jobname}|
\end{tabular}
\end{center}

In an alternative approach,
child documents can be compiled by a specific command line
without additional code or specific definitions:
%
\begin{center}
|... -jobname "|\textit{target}|" "|[\textit{flags}]%
|\includeonly{|\textit{dest}|}\input{|\textit{main}|}"|
\end{center}
%

%%%%%%%%%%%%%%%%%%%%%%%%%%%%%%%%%%%%%%%%%%%%%%%%%%%%%%%%%%%%%%%%%%%%%%%%%%%%%%%%
%%%%%%%%%%%%%%%%%%%%%%%%%%%%%%%%%%%%%%%%%%%%%%%%%%%%%%%%%%%%%%%%%%%%%%%%%%%%%%%%
\section{Information}

%%%%%%%%%%%%%%%%%%%%%%%%%%%%%%%%%%%%%%%%%%%%%%%%%%%%%%%%%%%%%%%%%%%%%%%%%%%%%%%%
\subsection{Copyright}

Copyright \copyright{} 2017--2018 Niklas Beisert

This work may be distributed and/or modified under the
conditions of the \LaTeX{} Project Public License, either version 1.3
of this license or (at your option) any later version.
The latest version of this license is in
  \url{http://www.latex-project.org/lppl.txt}
and version 1.3 or later is part of all distributions of \LaTeX{}
version 2005/12/01 or later.

This work has the LPPL maintenance status `maintained'.

The Current Maintainer of this work is Niklas Beisert.

This work consists of the files |README.txt|, |childdoc.ins| and |childdoc.dtx|
as well as the derived files |childdoc.def|, |cdocsamp.tex|
with |cdocsch1.tex|, |cdocsch2.tex|, |cdocspt3.tex|, |cdocspt4.tex|,
|cdocsdrf.tex|, |cdocsfn1.tex|, |cdocsfn2.tex|
as well as |childdoc.pdf|.

%%%%%%%%%%%%%%%%%%%%%%%%%%%%%%%%%%%%%%%%%%%%%%%%%%%%%%%%%%%%%%%%%%%%%%%%%%%%%%%%
\subsection{Files and Installation}

The package consists of the files:
%
\begin{center}
\begin{tabular}{ll}
    |README.txt|   & readme file \\
    |childdoc.ins| & installation file \\
    |childdoc.dtx| & source file \\
    |childdoc.def| & definition file \\
    |cdocsamp.tex| & sample main file \\
    |cdocsch1.tex| & sample include file \\
    |cdocsch2.tex| & sample include file \\
    |cdocspt3.tex| & sample part file \\
    |cdocspt4.tex| & sample part file \\
    |cdocsdrf.tex| & sample redirection file \\
    |cdocsfn1.tex| & sample redirection file \\
    |cdocsfn2.tex| & sample redirection file \\
    |childdoc.pdf| & manual
\end{tabular}
\end{center}
%
The distribution consists of the files
|README.txt|, |childdoc.ins| and |childdoc.dtx|.
%
\begin{itemize}
\item
Run (pdf)\LaTeX{} on |childdoc.dtx|
to compile the manual |childdoc.pdf| (this file).
\item
Run \LaTeX{} on |childdoc.ins| to create the definitions file |childdoc.def|
and the sample |cdocsamp.tex| with include files
|cdocsch1.tex|, |cdocsch2.tex|, |cdocspt3.tex|, |cdocspt4.tex|,
|cdocsdrf.tex|, |cdocsfn1.tex|, |cdocsfn2.tex|.
Then copy the file |childdoc.def| to an appropriate directory of your \LaTeX{}
distribution, e.g.\ \textit{texmf-root}|/tex/latex/childdoc|.
\end{itemize}

%%%%%%%%%%%%%%%%%%%%%%%%%%%%%%%%%%%%%%%%%%%%%%%%%%%%%%%%%%%%%%%%%%%%%%%%%%%%%%%%
\subsection{Related CTAN Packages}

There are several other packages which offer a similar functionality:
%
\begin{itemize}
\item
The packages
\href{http://ctan.org/pkg/docmute}{\textsf{docmute}},
\href{http://ctan.org/pkg/includex}{\textsf{includex}} and
\href{http://ctan.org/pkg/standalone}{\textsf{standalone}}
provide commands to include only the document body of
a child file thus allowing both files to be compiled individually.
\item
The packages \href{http://ctan.org/pkg/subdocs}{\textsf{subdocs}}
and \href{http://ctan.org/pkg/subfiles}{\textsf{subfiles}}
provide structures in which the main and child documents can be
encapsulated and allowing them to be compiled individually.
The inclusion mechanism is different from the conventional |\include|.
\item
The package \href{http://ctan.org/pkg/combine}{\textsf{combine}}
is an elaborate solution to combine several documents into one.
\end{itemize}
%
See also the CTAN topic \href{http://ctan.org/topic/subdocs}{\textsf{subdocs}}
for further related packages.
The present package differs from the above solutions in that
a document structure constructed with the conventional |\include| mechanism
just needs two extra commands at the top of every file
such that all constituent files can be compiled individually.

%%%%%%%%%%%%%%%%%%%%%%%%%%%%%%%%%%%%%%%%%%%%%%%%%%%%%%%%%%%%%%%%%%%%%%%%%%%%%%%%
%\subsection{Feature Suggestions}
%
%The following is a list of features which may be useful for future
%versions of this package:
%%
%\begin{itemize}
%\item
%\ldots
%\end{itemize}

%%%%%%%%%%%%%%%%%%%%%%%%%%%%%%%%%%%%%%%%%%%%%%%%%%%%%%%%%%%%%%%%%%%%%%%%%%%%%%%%
\subsection{Revision History}

%%%%%%%%%%%%%%%%%%%%%%%%%%%%%%%%%%%%%%%%
\paragraph{v2.0:} 2018/12/30

\begin{itemize}
\item
immediate forward processing
\item
added |\childdocby| mechanism
\item
manual restructured
\end{itemize}

%%%%%%%%%%%%%%%%%%%%%%%%%%%%%%%%%%%%%%%%
\paragraph{v1.6:} 2018/01/17

\begin{itemize}
\item
application for development of include files
\item
corrections to manual
\end{itemize}

%%%%%%%%%%%%%%%%%%%%%%%%%%%%%%%%%%%%%%%%
\paragraph{v1.5:} 2017/05/21

\begin{itemize}
\item
more complete structuring introduced
\item
|\childdocof| introduced
\item
|\childdoc| renamed to |\childdocmain|
\item
|\childredirect| renamed to |\childdocforward| and |\childdocforwardprefix|
and functionality expanded
\end{itemize}

%%%%%%%%%%%%%%%%%%%%%%%%%%%%%%%%%%%%%%%%
\paragraph{v1.0:} 2017/04/27

\begin{itemize}
\item
manual and install package
\item
first version published on CTAN
\end{itemize}

%%%%%%%%%%%%%%%%%%%%%%%%%%%%%%%%%%%%%%%%
\paragraph{v0.6:} 2017/04/26

\begin{itemize}
\item
redirection mechanism added
\end{itemize}

%%%%%%%%%%%%%%%%%%%%%%%%%%%%%%%%%%%%%%%%
\paragraph{v0.5:} 2017/04/26

\begin{itemize}
\item
functionality in definition file
\end{itemize}


%%%%%%%%%%%%%%%%%%%%%%%%%%%%%%%%%%%%%%%%%%%%%%%%%%%%%%%%%%%%%%%%%%%%%%%%%%%%%%%%
%%%%%%%%%%%%%%%%%%%%%%%%%%%%%%%%%%%%%%%%%%%%%%%%%%%%%%%%%%%%%%%%%%%%%%%%%%%%%%%%
%%%%%%%%%%%%%%%%%%%%%%%%%%%%%%%%%%%%%%%%%%%%%%%%%%%%%%%%%%%%%%%%%%%%%%%%%%%%%%%%
\appendix

\settowidth\MacroIndent{\rmfamily\scriptsize 000\ }

 \DocInput{childdoc.dtx}

\end{document}
%</driver>
% \fi
%
% %%%%%%%%%%%%%%%%%%%%%%%%%%%%%%%%%%%%%%%%%%%%%%%%%%%%%%%%%%%%%%%%%%%%%%%%%%%%%%
% %%%%%%%%%%%%%%%%%%%%%%%%%%%%%%%%%%%%%%%%%%%%%%%%%%%%%%%%%%%%%%%%%%%%%%%%%%%%%%
% \section{Sample}
%\iffalse
%<*samplemain>
%\fi
%
% The following presents a sample document
% with two chapters, two parts, a title page,
% a compile flag as well as three forwarding files to set the flag.
% It consists of eight |.tex| files:
% \begin{center}
% \begin{tabular}{ll}
% |cdocsamp.tex|&main file\\
% |cdocsch1.tex|&include file for chapter 1\\
% |cdocsch2.tex|&include file for chapter 2\\
% |cdocspt3.tex|&include file for part 3\\
% |cdocspt4.tex|&include file for part 4\\
% |cdocsdrf.tex|&forwarding file for main file in draft mode\\
% |cdocsfi1.tex|&forwarding file for final version of chapter 1\\
% |cdocsfi2.tex|&forwarding file for final version of chapter 2\\
% \end{tabular}
% \end{center}
% Each of the eight files can be compiled directly by the \LaTeX{} compiler.
%
% %%%%%%%%%%%%%%%%%%%%%%%%%%%%%%%%%%%%%%
% \paragraph{Main File.}
%
% The main file is called |cdocsamp.tex|.
%
% Load the \textsf{childdoc} definitions and
% declare the filename for the main document:
%    \begin{macrocode}
\input{childdoc.def}
\childdocmain{}
%    \end{macrocode}

% Optional override for |\version| flag:
%    \begin{macrocode}
%%\ifchilddoc\else\providecommand{\version}{draft}\fi
%    \end{macrocode}

% Define the default values for the |\version| flag
% (|final| for the main file and |draft| for childs):
%    \begin{macrocode}
\ifchilddoc
\providecommand{\version}{draft}
\else
\providecommand{\version}{final}
\fi
%    \end{macrocode}

% Load the standard document class:
%    \begin{macrocode}
\documentclass[12pt]{article}
%    \end{macrocode}

% Start the document body:
%    \begin{macrocode}
\begin{document}
%    \end{macrocode}

% Declare a title page.
% Print title, part of document being processed and version flag:
%    \begin{macrocode}
\addtocounter{page}{-1}
\begin{center}
{\LARGE\bfseries{}childdoc example\par}
\vspace{1cm}
\ifchilddoc
\ifchilddocmanual part\else chapter\fi:
`\childdocname' of `\childdocjob'\par
\else
main document: `\childdocjob'\par
\fi
version: \version\par
\end{center}
\newpage
%    \end{macrocode}

% Manually include selected file,
% otherwise process as usual:
%    \begin{macrocode}
\ifchilddocmanual
\section*{part `\childdocname'}
\input{\childdocname}
\else
%    \end{macrocode}

% Include the two chapters:
%    \begin{macrocode}
\include{cdocsch1}
\include{cdocsch2}
%    \end{macrocode}

% Include the two parts unless only chapters should be displayed:
%    \begin{macrocode}
\ifchilddoc\else
\section{part three}
\input{cdocspt3}
\section{part four}
\input{cdocspt4}
\fi
%    \end{macrocode}

% Process as usual until here:
%    \begin{macrocode}
\fi
%    \end{macrocode}

% End of document body:
%    \begin{macrocode}
\end{document}
%    \end{macrocode}
%\iffalse
%</samplemain>
%\fi
%
% %%%%%%%%%%%%%%%%%%%%%%%%%%%%%%%%%%%%%%
% \paragraph{Chapter Include Files.}
%
% The include files are called |cdocsch1.tex| and |cdocsch2.tex|.
%
%\iffalse
%<*samplechap1|samplechap2>
%\fi

% Optional override for |\version| flag:
%    \begin{macrocode}
%%\providecommand{\version}{final}
%    \end{macrocode}

% Include the main document:
%    \begin{macrocode}
\input{childdoc.def}
\childdocof{cdocsamp}
%    \end{macrocode}

%\iffalse
%</samplechap1|samplechap2>
%\fi
%
%\iffalse
%<*samplechap1>
%\fi
% Some text for chapter 1:
%    \begin{macrocode}
\section{one}
some text in chapter one
%    \end{macrocode}

%\iffalse
%</samplechap1>
%\fi
% Some text for chapter 2:
%\iffalse
%<*samplechap2>
%\fi
%    \begin{macrocode}
\section{two}
more text in chapter two
%    \end{macrocode}

%\iffalse
%</samplechap2>
%\fi
%
% %%%%%%%%%%%%%%%%%%%%%%%%%%%%%%%%%%%%%%
% \paragraph{Part Include Files.}
%
% The include files are called |cdocspt3.tex| and |cdocspt4.tex|.
%
%\iffalse
%<*samplepart3|samplepart4>
%\fi

% Optional override for |\version| flag:
%    \begin{macrocode}
%%\providecommand{\version}{final}
%    \end{macrocode}

% Include the main document:
%    \begin{macrocode}
\input{childdoc.def}
\childdocby{cdocsamp}
%    \end{macrocode}

%\iffalse
%</samplepart3|samplepart4>
%\fi
%
%\iffalse
%<*samplepart3>
%\fi
% Some text for part 3:
%    \begin{macrocode}
some text in part three
%    \end{macrocode}

%\iffalse
%</samplepart3>
%\fi
% Some text for part 4:
%\iffalse
%<*samplepart4>
%\fi
%    \begin{macrocode}
more text in part four
%    \end{macrocode}

%\iffalse
%</samplepart4>
%\fi
%
% %%%%%%%%%%%%%%%%%%%%%%%%%%%%%%%%%%%%%%
% \paragraph{Forwarding for a Complete Draft.}
%
% The following forwarding file |cdocsdrf.tex|
% compiles the main document in draft mode:
%\iffalse
%<*sampledraft>
%\fi
%    \begin{macrocode}
\def\version{draft}
\input{childdoc.def}
\childdocforward{cdocsamp}
%    \end{macrocode}

%\iffalse
%</sampledraft>
%\fi
%
% %%%%%%%%%%%%%%%%%%%%%%%%%%%%%%%%%%%%%%
% \paragraph{Forwarding for Final Version of the Chapters.}
%
% The following forwarding files |cdocsfn1.tex| and |cdocsfn2.tex|
% (with identical content)
% compile the final versions of the child documents
% |cdocsch1.tex| and |cdocsch2.tex|, respectively:
%\iffalse
%<*samplefinal>
%\fi
%    \begin{macrocode}
\def\version{final}
\input{childdoc.def}
\childdocforwardprefix[cdocsamp]{cdocsfn}{cdocsch}
%    \end{macrocode}

%\iffalse
%</samplefinal>
%\fi
%
% %%%%%%%%%%%%%%%%%%%%%%%%%%%%%%%%%%%%%%
% \paragraph{Command Line Processing.}
%
% The following three command lines generate the output files
% |cdocscld|, |cdocscl1| and |cdocscl2|
% which should be identical to
% |cdocsdrf|, |cdocsch1| and |cdocsfn2|, respectively:
% \begin{center}
% \begin{tabular}{l}
% |latex -jobname cdocscld \|\\
% |  "\def\version{draft}\input{childdoc.def}\childdocforward{cdocsamp}"|\\
% |latex -jobname cdocscl1 \|\\
% |  "\input{childdoc.def}\childdocforward[cdocsamp]{cdocsch1}"|\\
% |latex -jobname cdocscl2 \|\\
% |  "\def\version{final}\input{childdoc.def}\childdocforward{cdocsch2}"|
% \end{tabular}
% \end{center}
% Note that the trailing backslash on each first line
% merely continues the input to the second line
% (for convenient cut ant paste).
% Furthermore, the command |latex| can be replaced by any
% of its alternative versions such as |pdflatex|.
%
% %%%%%%%%%%%%%%%%%%%%%%%%%%%%%%%%%%%%%%%%%%%%%%%%%%%%%%%%%%%%%%%%%%%%%%%%%%%%%%
% %%%%%%%%%%%%%%%%%%%%%%%%%%%%%%%%%%%%%%%%%%%%%%%%%%%%%%%%%%%%%%%%%%%%%%%%%%%%%%
% \section{Implementation}
%\iffalse
%<*package>
%\fi
%
% This section describes the definitions file |childdoc.def|.

% The definitions cannot be loaded using |\usepackage| or |\RequirePackage|
% which has a mechanism to prevent loading a style file more than once.
% When loading the definitions by means of |\input|
% multiple instances have to be prevented manually:
%\iffalse
%This code needs to be before the `\ProvidesFile' directive
%which is defined at the beginning of this file.
%Therefore it is also placed there and commented out here.
%</package>
%<*discard>
%\fi
%    \begin{macrocode}
\ifdefined\childdocmain\endinput\fi
%    \end{macrocode}
%\iffalse
%</discard>
%<*package>
%\fi
%
% \macro{\ifchilddoc}
% \macro{\ifchilddocmanual}
% The conditional |\ifchilddoc| tells whether a
% child (true) or main (false) document is being compiled.
% The conditional |\ifchilddocmanual| tells whether
% the |\includeonly| mechanism is used (false) or
% the selection of child files must be performed manually (true).
% The definitions initialise to false:
%    \begin{macrocode}
\newif\ifchilddoc
\newif\ifchilddocmanual
%    \end{macrocode}

% \macro{\childdocname}
% \macro{\childdocjob}
% The macro |\childdocname| stores the name of the main document
% to be compiled. The macro |\childdocjob| stores the name of
% the document on which the \LaTeX{} compiler was originally invoked.
% The content of |\jobname| cannot be compared
% to filenames specified in the source due to different catcodes.
% The following code rescans |\jobname|, stores the result
% in |\childdocname| and saves a copy in |\childdocjob|:
%    \begin{macrocode}
\edef\childdocname{\scantokens\expandafter{\jobname\noexpand}}
\let\childdocjob\childdocname
%    \end{macrocode}

% \macro{\childdocdisable}
% The macro |\childdocdisable| prevents the main file
% from being processed more than once.
% At this stage, the main document command |\childdocmain|
% is assumed to be called once again where it should do nothing.
% Any subsequent call to it should prevent
% a secondary processing of the main document
% It overwrites the forwarding commands
% |\childdocof| and |\childdocforward|
% with empty macros to prevent further inclusions of the main document:
%    \begin{macrocode}
\newcommand{\childdocdisable}
{
  \renewcommand{\childdocmain}[1]{\renewcommand{\childdocmain}[1]{\endinput}}
  \renewcommand{\childdocof}[1]{}
  \renewcommand{\childdocby}[2][]{}
  \renewcommand{\childdocforward}[2][]{}
  \renewcommand{\childdocdisable}{}
}
%    \end{macrocode}

% \macro{\childdocmain}
% The macro |\childdocmain| is to be called at the top of the main file
% with nothing or the main filename (without extension) as argument.
% First, it breaks loops.
% If the argument is not empty and does not match |\childdocname|
% (which is set by the first inclusion of |childdoc.def|),
% |\ifchilddoc| is set to true, |\includeonly| is applied to the child file
% and |\jobname| is set to the main file
% (for proper handling of |.aux| files):
%    \begin{macrocode}
\newcommand{\childdocmain}[1]
{
  \childdocdisable\childdocmain{}
  \if?#1?\else
    \begingroup
      \def\childdoctmp{#1}
      \ifx\childdoctmp\childdocname
        \def\childdoctmp{}
      \else
        \def\childdoctmp
        {
          \childdoctrue
          \includeonly{\childdocname}
          \def\childdocjob{#1}
          \def\jobname{#1}
        }
      \fi
      \expandafter
    \endgroup
    \childdoctmp
  \fi
}
%    \end{macrocode}

% \macro{\childdocof}
% The command |\childdocof| redirects
% compilation to the main file |#1|.
%    \begin{macrocode}
\newcommand{\childdocof}[1]
{
  \childdocdisable
  \childdoctrue
  \includeonly{\childdocname}
  \def\jobname{#1}
  \def\childdocjob{#1}
  \input{#1}
}
%    \end{macrocode}

% \macro{\childdocby}
% The command |\childdocby| ....
%    \begin{macrocode}
\newcommand{\childdocby}[2][]
{
  \childdocdisable
  \childdoctrue
  \childdocmanualtrue
  \if?#1?\else
    \def\jobname{#2}
  \fi
  \def\childdocjob{#2}
  \input{#2}
  \endinput
}
%    \end{macrocode}

% \macro{\childdocforward}
% The command |\childdocforward| redirects
% compilation to the main file or
% (if the optional argument is given) a child file.
% Parameters are set as if the main file
% or a child file starting with |\childdocof| was compiled.
% Then compilation is handed over to the main file:
%    \begin{macrocode}
\newcommand{\childdocforward}[2][]
{
  \begingroup
    \if?#1?
      \def\childdoctmp
      {
        \def\childdocname{#2}
        \def\childdocjob{#2}
        \def\jobname{#2}
        \input{#2}
        \endinput
      }
    \else
      \def\childdoctmp
      {
        \childdocdisable
        \def\childdocname{#2}
        \childdoctrue
        \includeonly{#2}
        \def\childdocjob{#1}
        \def\jobname{#1}
        \input{#1}
        \endinput
      }
    \fi
    \expandafter
  \endgroup
  \childdoctmp
}
%    \end{macrocode}

% \macro{\childdocforwardprefix}
% The command |\childdocforwardprefix| redirects
% compilation to the main or a child file by means of a pattern.
% The prefix |#1| in the current filename is replaced by |#2|
% and the suffix of the current filename is kept
% (it is assumed that the filename does not contain the substring `|~~~|'
% which is used as a delimiter).
% Compilation is handed over to the new file by |\childdocforward|:
%    \begin{macrocode}
\newcommand{\childdocforwardprefix}[3][]
{
  \begingroup
    \def\childdocextract #2##1~~~{\def\childdoctmp{\childdocforward[#1]{#3##1}}}
    \expandafter\childdocextract\childdocname~~~
    \expandafter
  \endgroup
  \childdoctmp
}
%    \end{macrocode}

% \macro{\childdoc}
% The deprecated macro |\childdoc| is a legacy version of |\childdocmain|:
%    \begin{macrocode}
\newcommand{\childdoc}{\childdocmain}
%    \end{macrocode}

% \macro{\childdocredirect}
% The deprecated macro |\childdocredirect| is a legacy version
% of |\childdocforward| and |\childdocforwardprefix|:
%    \begin{macrocode}
\newcommand{\childdocredirect}[2][]
{
  \begingroup
    \if?#1?
      \def\childdoctmp{\childdocforward{#2}}
    \else
      \def\childdoctmp{\childdocforwardprefix{#1}{#2}}
    \fi
    \expandafter
  \endgroup
  \childdoctmp
}
%    \end{macrocode}

%\iffalse
%</package>
%\fi
%
\endinput
|\\
|\childdocforward{|\textit{main}|}|\\
\end{tabular}
\end{center}
%
or alternatively with:
%
\begin{center}
\begin{tabular}{l}
|% \iffalse
%
% childdoc.dtx Copyright (C) 2017-2018 Niklas Beisert
%
% This work may be distributed and/or modified under the
% conditions of the LaTeX Project Public License, either version 1.3
% of this license or (at your option) any later version.
% The latest version of this license is in
%   http://www.latex-project.org/lppl.txt
% and version 1.3 or later is part of all distributions of LaTeX
% version 2005/12/01 or later.
%
% This work has the LPPL maintenance status `maintained'.
%
% The Current Maintainer of this work is Niklas Beisert.
%
% This work consists of the files childdoc.dtx and childdoc.ins
% and the derived files childdoc.def and cdocsamp.tex with
% cdocsch1.tex, cdocsch2.tex, cdocsdrf.tex, cdocsfn1.tex, cdocsfn2.tex.
%
%<package>\ifdefined\childdocmain\endinput\fi
%<package>\ProvidesFile{childdoc.def}[2018/12/30 v2.0 child document driver]
%<samplemain>\ProvidesFile{cdocsamp.tex}[2018/12/30 v2.0 sample for childdoc]
%<*driver>
%\ProvidesFile{childdoc.drv}[2018/12/30 v2.0 childdoc reference manual file]
\PassOptionsToClass{10pt,a4paper}{article}
\documentclass{ltxdoc}

\usepackage[margin=35mm]{geometry}
\usepackage{hyperref}
\usepackage{hyperxmp}
\usepackage[usenames]{color}

\hypersetup{colorlinks=true}
\hypersetup{pdfstartview=FitH}
\hypersetup{pdfpagemode=UseNone}
\hypersetup{pdfsource={}}
\hypersetup{pdflang={en-UK}}
\hypersetup{pdfcopyright={Copyright 2017-2018 Niklas Beisert.
  This work may be distributed and/or modified under the
  conditions of the LaTeX Project Public License, either version 1.3
  of this license or (at your option) any later version.}}
\hypersetup{pdflicenseurl={http://www.latex-project.org/lppl.txt}}
\hypersetup{pdfcontactaddress={ETH Zurich, ITP, HIT K,
  Wolfgang-Pauli-Strasse 27}}
\hypersetup{pdfcontactpostcode={8093}}
\hypersetup{pdfcontactcity={Zurich}}
\hypersetup{pdfcontactcountry={Switzerland}}
\hypersetup{pdfcontactemail={nbeisert@itp.phys.ethz.ch}}
\hypersetup{pdfcontacturl={http://people.phys.ethz.ch/\xmptilde nbeisert/}}

\newcommand{\secref}[1]{\hyperref[#1]{section \ref*{#1}}}

\parskip1ex
\parindent0pt
\let\olditemize\itemize
\def\itemize{\olditemize\parskip0pt}

\begin{document}

\title{The \textsf{childdoc} Package}
\hypersetup{pdftitle={The childdoc Package}}
\author{Niklas Beisert\\[2ex]
  Institut f\"ur Theoretische Physik\\
  Eidgen\"ossische Technische Hochschule Z\"urich\\
  Wolfgang-Pauli-Strasse 27, 8093 Z\"urich, Switzerland\\[1ex]
  \href{mailto:nbeisert@itp.phys.ethz.ch}
  {\texttt{nbeisert@itp.phys.ethz.ch}}}
\hypersetup{pdfauthor={Niklas Beisert}}
\hypersetup{pdfsubject={Manual for the LaTeX2e Package childdoc}}
\date{30 December 2018, \textsf{v2.0}}
\maketitle

\begin{abstract}\noindent
\textsf{childdoc} is a \LaTeXe{} package
that enables the direct compilation
of document sections included by |\include|
to individual files.
\end{abstract}

\begingroup
\parskip0ex
\tableofcontents
\endgroup

%%%%%%%%%%%%%%%%%%%%%%%%%%%%%%%%%%%%%%%%%%%%%%%%%%%%%%%%%%%%%%%%%%%%%%%%%%%%%%%%
%%%%%%%%%%%%%%%%%%%%%%%%%%%%%%%%%%%%%%%%%%%%%%%%%%%%%%%%%%%%%%%%%%%%%%%%%%%%%%%%
\section{Introduction}

\LaTeX{} provides a mechanism to structure a large document (such as a book)
into a main file and several child files (containing the chapters)
using the |\include| command.
This mechanism is beneficial for documents
which span hundreds of pages in order to
make the source file(s) more manageable.
Moreover, compilation can be restricted to
selected child files by means of the |\includeonly| command.
The latter feature can be used to reduce the compilation time while editing
(this was significantly more useful in the earlier days of \LaTeX{})
or to generate a smaller document which is easier to navigate.
Another application of |\includeonly| is to generate
documents consisting of selected parts of the complete document.

However, there are a few drawbacks of the plain |\include| mechanism:
\begin{itemize}
\item
The child files cannot be compiled on their own,
they can only be compiled via the main file.
A naive editing environment
(such as a text editor with an option
to have the current file processed by \LaTeX)
may require one to switch to the main file before compiling;
attempting to compile the child file produces errors.
\item
The main file must be modified (each time)
to adjust the |\includeonly| command
to the present needs. This easily leaves the main file in a messy state.
\item
The generated document will always carry the filename
of the main document. This is inconvenient if
several child files are to be compiled and
to be kept for distribution.
\end{itemize}

The present package provides a simple interface
to make child files individually compilable by \LaTeX{}.
Compiling a child file then has the same effect as compiling
the main file with an |\includeonly| command
to select the appropriate child.
Moreover the generated document will carry the name of the child
rather than the main file.
This resolves all three above issues.

This feature is meant to make the editing of books,
thesis documents and lecture notes somewhat more convenient.
However, the package can also be used efficiently for
composing a series of documents (such as exercise sheets)
which are typically distributed individually.
It then assists the author in generating the individual documents
(potentially in different versions)
as well as a document containing the collected series.
Another application is in developing style files
or other kinds of included material
where compilation of the style file could redirect
to a sample or test file.

%%%%%%%%%%%%%%%%%%%%%%%%%%%%%%%%%%%%%%%%%%%%%%%%%%%%%%%%%%%%%%%%%%%%%%%%%%%%%%%%
%%%%%%%%%%%%%%%%%%%%%%%%%%%%%%%%%%%%%%%%%%%%%%%%%%%%%%%%%%%%%%%%%%%%%%%%%%%%%%%%
\section{Usage}

First of all, the package \textsf{childdoc} is \emph{not} a standard
\LaTeXe{} |.sty| style file! Therefore it needs to be invoked in
a non-standard way.

%%%%%%%%%%%%%%%%%%%%%%%%%%%%%%%%%%%%%%%%%%%%%%%%%%%%%%%%%%%%%%%%%%%%%%%%%%%%%%%%
\subsection{Included Files}
\label{sec:include}

%%%%%%%%%%%%%%%%%%%%%%%%%%%%%%%%%%%%%%%%
\DescribeMacro{\childdocmain}
To use the package, add the commands
\begin{center}
\begin{tabular}{l}
|\input{childdoc.def}|\\
|\childdocmain{}|\\
\end{tabular}
\end{center}
at the very top of the main \LaTeX{} file,
in particular \emph{before} the |\documentclass| statement!
The argument of |\childdocmain| should be left empty
(but it must be present).

%%%%%%%%%%%%%%%%%%%%%%%%%%%%%%%%%%%%%%%%
\DescribeMacro{\childdocof}
Furthermore, add the commands
\begin{center}
\begin{tabular}{l}
|\input{childdoc.def}|\\
|\childdocof{|\textit{main}|}|\\
\end{tabular}
\end{center}
at the top of every child file \textit{child}
which is included by |\include{|\textit{child}|}|
from within the main file
(or at least for those files to be compiled individually).
The argument \textit{main} must be the filename of the main file.

There are a couple of
considerations in setting up the main and child documents:

%%%%%%%%%%%%%%%%%%%%%%%%%%%%%%%%%%%%%%%%
\paragraph{Restrictions.}

Please note the following restrictions:
\begin{itemize}
\item
|\childdocmain| must be called with one argument \textit{main}
to ensure compatibility with earlier version of the package.
It must either be empty (|\childdocmain{}|)
or precisely match the filename of the main file in which it is specified.
See \secref{sec:detection} for further information.
\item
The filename \textit{main} must be specified without the |.tex| extension.
\item
The filename \textit{main} is case sensitive
(even in case-insensitive file systems)
due to internal string comparison.
\item
The argument \textit{main} should be fully expanded, it cannot be a macro.
\item
Subdirectories and special characters should be avoided in filenames.
\item
The command |\childdocmain{|\textit{main}|}| must be followed by a whitespace.
It should not be followed immediately by another command
or by a comment mark `|%|'.
This is because the \TeX{} parser reads the token immediately following
the argument of |\childdocmain| and puts it
at the beginning of every child section;
however, a white\-space is ignored.
\end{itemize}

%%%%%%%%%%%%%%%%%%%%%%%%%%%%%%%%%%%%%%%%
\paragraph{Content of Main File.}

It is advisable to place all content in the child files included by |\include|.
Any output contained in the main file will appear in all child documents
unless suppressed manually;
it cannot be suppressed automatically by the |\includeonly| directive
and thus should normally be avoided.
A method to include some content in the main file
by means of conditional processing is described in \secref{sec:conditional}.

%%%%%%%%%%%%%%%%%%%%%%%%%%%%%%%%%%%%%%%%
\paragraph{Page Numbering.}

When only a part of the document is compiled,
the appropriate numbering of pages
(as well as other status parameters)
is determined from the |.aux| files.
The latter contain information from previous passes.
However this information needs to propagate through
all intermediate child documents.
Therefore the page numbering in child documents may well
be inconsistent until the complete document is compiled at least once.

A useful (if unconventional) way to always ensure a consistent
page numbering is to restart the numbering in each child document
and denote the pages by `\textit{child}|.|\textit{page}'
where \textit{child} represents the chapter/section number of the child file.
This can be achieved by the command
|\numberwithin{page}{|\textit{child}|}|
of the \textsf{amsmath} package
where \textit{child} can be |chapter| or |section|
depending on the chosen structuring.
Alternatively, one can modify the macro |\thepage| appropriately
and reset the counter |page| at the start of each child file.

%%%%%%%%%%%%%%%%%%%%%%%%%%%%%%%%%%%%%%%%%%%%%%%%%%%%%%%%%%%%%%%%%%%%%%%%%%%%%%%%
\subsection{Conditional Processing}
\label{sec:conditional}

The package provides a mechanism to compile different versions
of a document. To customise the versions further some conditional processing
can come in handy to distinguish which version is being compiled.
The package provides two macros to describe the compilation context:

%%%%%%%%%%%%%%%%%%%%%%%%%%%%%%%%%%%%%%%%
\DescribeMacro{\ifchilddoc}
The conditional |\ifchilddoc| distinguishes between the compilation of
child documents and the main document:
%
\begin{center}
|\ifchilddoc |\textit{child-code}| |[|\||else |\textit{main-code}]| \||fi|
\end{center}

%%%%%%%%%%%%%%%%%%%%%%%%%%%%%%%%%%%%%%%%
\DescribeMacro{\childdocname}
\DescribeMacro{\childdocjob}
The macro |\childdocname| contains the filename (without extension)
of the main or child file being processed.
Note that |\childdocjob| will always contain the name of the main file.

%%%%%%%%%%%%%%%%%%%%%%%%%%%%%%%%%%%%%%%%
\paragraph{Title Page.}

Conditional processing can be used to include a title or banner page
in the main document when proper precautions are taken.
Importantly, the code in the main file should ensure that the page counter
(as well as other status parameters which are stored in the |.aux| files)
takes the same value after the conditional processing.
Otherwise the page numbers may take divergent values
depending on which part is compiled.

For example, a title page could be declared by:
%
\begin{center}
\begin{tabular}{l}
|\ifchilddoc\||else|\\
|\addtocounter{page}{-1}|\\
\textit{code for title page}\\
|\newpage|\\
|\||fi|
\end{tabular}
\end{center}
%
A banner page for the child documents can be generated by:
%
\begin{center}
\begin{tabular}{l}
|\ifchilddoc|\\
|\addtocounter{page}{-1}|\\
\textit{code for banner page}\\
|\newpage|\\
|\||fi|
\end{tabular}
\end{center}
%
Here one could write a message such as:
\begin{center}
|This is the part \childdocname{} of \childdocjob{}.|
\end{center}

%%%%%%%%%%%%%%%%%%%%%%%%%%%%%%%%%%%%%%%%%%%%%%%%%%%%%%%%%%%%%%%%%%%%%%%%%%%%%%%%
\subsection{Flags}
\label{sec:flags}

The package makes it easy to generate different versions
of the main or child documents.
To this end compilation flags can be defined
and assigned different default values.
They will be particularly useful in conjunction
with the forwarding mechanism described in \secref{sec:forward}.

For example, it may be useful to have a flag |\version|
which can be set to |draft| or |final|.
The document source will contain some conditional code
depending on the value of |\version|.
Suppose further, the flag should default to |final| for the main file
and to |draft| for child files
which is a natural assignment for editing the document.
This is achieved by placing the following code
in the preamble of the main document
(below the |\childdocmain| directive):
%
\begin{center}
\begin{tabular}{l}
|\ifchilddoc|\\
|\providecommand{\version}{draft}|\\
|\||else|\\
|\providecommand{\version}{final}|\\
|\||fi|
\end{tabular}
\end{center}
%
The definition by |\providecommand| makes sure
that previous definitions are not overwritten.
Further statements |\providecommand{\version}{...}|
can thus be added before the above code to override it.

For the main file, one might add a line
(between |\childdocmain| and the above block)
%
\begin{center}
|%\ifchilddoc\||else\providecommand{\version}{draft}\||fi|
\end{center}
%
which can be uncommented to produce a draft version.
Likewise one can add a line to the very top of a child file
(above the |\childdocof{|\textit{main}|}| directive)
%
\begin{center}
|%\providecommand{\version}{final}|
\end{center}
%
which can be uncommented to produce the final version of this child document.

%%%%%%%%%%%%%%%%%%%%%%%%%%%%%%%%%%%%%%%%%%%%%%%%%%%%%%%%%%%%%%%%%%%%%%%%%%%%%%%%
\subsection{Forwarding}
\label{sec:forward}

Different versions of the main or child documents
using compilation flags as described in \secref{sec:flags}
can be (permanently) stored in different files
for convenient compilation, viewing and distribution.
To this end, the package defines a command
to pass on compilation to a different file:

%%%%%%%%%%%%%%%%%%%%%%%%%%%%%%%%%%%%%%%%
\DescribeMacro{\childdocforward}
The command |\childdocforward| redirects processing to
another source file:
%
\begin{center}
\begin{tabular}{l}
|\input{childdoc.def}|\\
|\childdocforward[|\textit{main}|]{|\textit{dest}|}|\\
\end{tabular}
\end{center}
%
The argument \textit{dest} is the destination file
(without extension).
It should be the main file or one of the child files.
Note that further \textsf{childdoc} directives
such as |\childdocof| and |\childdocforward|
in the indicated file will be processed in this form.
The optional argument \textit{main}
passes on directly to the main file \textit{main}
while pretending to compile the child \textit{dest}.
This form behaves as if \textit{dest}
issues |\childdocof{|\textit{main}|}| right away,
and no further \textsf{childdoc} directives will be processed.

%%%%%%%%%%%%%%%%%%%%%%%%%%%%%%%%%%%%%%%%
\DescribeMacro{\...prefix}
In the alternative form |\childdocforwardprefix|,
%
\begin{center}
\begin{tabular}{l}
|\input{childdoc.def}|\\
|\childdocforwardprefix[|\textit{main}|]{|\textit{prefix}|}{|\textit{dest}|}|
\end{tabular}
\end{center}
%
the destination file is determined by a pattern
depending on the current file:
To make this work, the current file must be called
`{\textit{prefix}\hspace{0.2em}\textit{suffix}}'
with \textit{prefix} matching precisely the argument.
Processing is then passed on to the file
`{\textit{dest}\hspace{0.2em}\textit{suffix}}'.
Surely, the same effect is achieved by
directly specifying the
argument `{\textit{dest}\hspace{0.2em}\textit{suffix}}'
in the first form.
However, that requires to set up a different file
for each child. With the alternative form of the command
all these files can have exactly the same content
which simplifies setting them up and maintaining them.

For example, the following file |draft.tex|
with a compilation flag |\version| as described in \secref{sec:flags}
compiles the main document as a draft:
%
\begin{center}
\begin{tabular}{l}
|\def\version{draft}|\\
|\input{childdoc.def}|\\
|\childdocforward{|\textit{main}|}|
\end{tabular}
\end{center}
%
Likewise, the following files |final|\textit{nn}|.tex|
compile the final version of the child document
|child|\textit{nn}|.tex|:
%
\begin{center}
\begin{tabular}{l}
|\def\version{final}|\\
|\input{childdoc.def}|\\
|\childdocforwardprefix{final}{child}|
\end{tabular}
\end{center}
%

Note that when several versions of a main file and/or of each child file
are to be generated, it may be convenient to set up a |Makefile| or
shell script to automatise the process.

%%%%%%%%%%%%%%%%%%%%%%%%%%%%%%%%%%%%%%%%%%%%%%%%%%%%%%%%%%%%%%%%%%%%%%%%%%%%%%%%
\subsection{Command Line Processing}
\label{sec:commandline}

The effect of redirection files can also be achieved by invoking
the \LaTeX{} compiler with a more elaborate command line.
Most conveniently this should be done as part
of a shell script or a |Makefile|.

When using \textsf{childdoc} in the main file, the following
command lines effectively perform a redirection
(note that depending on the shell being used,
backslashes may have to be doubled: `|\|' $\to$ `|\\|'):
%
\begin{center}
|... -jobname "|\textit{target}|" |\\|"|[\textit{flags}]%
|\input{childdoc.def}\childdocforward[|\textit{main}|]{|\textit{dest}|}"|
\end{center}
%
Here \textit{target} is the name of the output file,
\textit{main} is the name of the main file
and \textit{dest} is the name of the main or child file to be processed
(all filenames without extensions).
The optional argument \textit{main} can be omitted
if \textit{main} matches \textit{dest}.
Optionally, compilation \textit{flags} can be defined via |\def| commands.
This command line makes the \TeX{} engine believe
it is compiling the file \textit{target}
whose content is specified as the latter parameter.
The provided code then forwards the processing to
\textit{main} or \textit{dest} as described in \secref{sec:forward}.

%%%%%%%%%%%%%%%%%%%%%%%%%%%%%%%%%%%%%%%%%%%%%%%%%%%%%%%%%%%%%%%%%%%%%%%%%%%%%%%%
\subsection{Include by Input}
\label{sec:input}

Including child documents by |\include| has some restrictions by design.
Most notably, the content of a child document always occupies
its own set of pages; pages cannot be shared between child documents.
Usually, this behaviour makes perfect sense
because each child document contain an essential part of the document.
However, in some situations it may be desirable to compose
a document from a collection of parts
without having mandatory page breaks between then.
For this case, the package
provides a mechanism to include parts
by |\input| which can also be processed individually.
However, by construction this mechanism
requires manual handling of the content to be output.

%%%%%%%%%%%%%%%%%%%%%%%%%%%%%%%%%%%%%%%%
\DescribeMacro{\ifchilddocmanual}
The main file should be prepared as usual, see \secref{sec:include}.
However, the document body must make a distinction
between processing of an individual part and of the main document, e.g.:
%
\begin{center}
\begin{tabular}{l}
|\ifchilddocmanual|\\
|\input{\childdocname}|\\
|\||else|\\
\textit{document body with }|\input{|\textit{part}|}|\\
|\||fi|
\end{tabular}
\end{center}
%
The conditional |\ifchilddocmanual| is true whenever
a part to be included by |\input| is being compiled,
and the name of the part is stored in |\childdocname|.

%%%%%%%%%%%%%%%%%%%%%%%%%%%%%%%%%%%%%%%%
\DescribeMacro{\childdocby}
Each part to be included by |\input| should start with:
%
\begin{center}
\begin{tabular}{l}
|\input{childdoc.def}|\\
|\childdocby{|\textit{main}|}|\\
\end{tabular}
\end{center}
%
The directive |\childdocby| is similar to |\childdocof|
described in \secref{sec:include},
but the subsequent selection of content must be done manually.
To that end, both |\ifchilddoc| and |\ifchilddocmanual|
will be true upon processing of a part,
and the name of the part is stored in |\childdocname|.
Note that |\jobname| will be set to the filename of the current part
so that each part receives an individual |.aux| file
that does not interfere with the |.aux| file(s) of the main document.
This behaviour can be altered by the alternative form
|\childdocby[*]{|\textit{main}|}| (with a non-empty optional argument)
which uses the |.aux| file of the main document
by setting |\jobname| to \textit{main}.

%%%%%%%%%%%%%%%%%%%%%%%%%%%%%%%%%%%%%%%%%%%%%%%%%%%%%%%%%%%%%%%%%%%%%%%%%%%%%%%%
\subsection{Driver Development}
\label{sec:driver}

The \textsf{childdoc} mechanism can also be use for the development
of definition files such as \LaTeX{} styles or classes.
This case differs from the above setup with multiple parts
included by |\include| in that no |\includeonly| should be invoked.
This can be achieved by starting the include file
(before |\ProvidesPackage|) with:
%
\begin{center}
\begin{tabular}{l}
|\input{childdoc.def}|\\
|\childdocforward{|\textit{main}|}|\\
\end{tabular}
\end{center}
%
or alternatively with:
%
\begin{center}
\begin{tabular}{l}
|\input{childdoc.def}|\\
|\childdocby{|\textit{main}|}|\\
\end{tabular}
\end{center}
%
Both forms have slightly different effects as described above.
The main file is prepared as usual, see \secref{sec:include}.

%%%%%%%%%%%%%%%%%%%%%%%%%%%%%%%%%%%%%%%%%%%%%%%%%%%%%%%%%%%%%%%%%%%%%%%%%%%%%%%%
\subsection{Legacy Detection}
\label{sec:detection}

The directive |\childdocmain| in the main file can detect
whether the complete document or merely a child is to be compiled
even without using the directive |\childdocof|.
This method is deprecated because it is less robust
and there is no compelling reason to use it;
it is merely provided for backward compatibility
and it may be removed in future versions.

If the detection mechanism is to be used,
it is mandatory to correctly specify
the filename of the main file as the argument of |\childdocmain|:
%
\begin{center}
\begin{tabular}{l}
|\input{childdoc.def}|\\
|\childdocmain{|\textit{main}|}|\\
\end{tabular}
\end{center}
%
If |\jobname| does not match the argument \textit{main} of |\childdocmain|,
it is assumed that |\jobname| points to the child file to be compiled.
When using |\childdocmain| with the main file specified as argument,
it suffices to start a child file
with just |\input{|\textit{main}|}|
without loading of the package and using |\childdocof|.
If instead all processing is done
with the appropriate \textsf{childdoc} directives,
the argument of \textit{main} of |\childdocmain| can be empty.

An alternative version of the command line processing described
in \secref{sec:commandline} using the detection mechanism reads:
%
\begin{center}
|... -jobname "|\textit{target}|" "|[\textit{flags}]%
[|\def\jobname{|\textit{dest}|}|]|\input{|\textit{main}|}"|
\end{center}

%%%%%%%%%%%%%%%%%%%%%%%%%%%%%%%%%%%%%%%%%%%%%%%%%%%%%%%%%%%%%%%%%%%%%%%%%%%%%%%%
\subsection{Manual Code}
\label{sec:manual}

In case one cannot be certain whether the definitions file |childdoc.def|
is installed on the target \TeX{} distribution
and one prefers not to ship it,
it is conceivable to paste a few relevant commands into the sources.

To that end, drop all statements |\input{childdoc.def}|
and perform the replacements as outlined below.
Instead of |\childdocmain{|\textit{main}|}| add the following code
to the top of the main file:
%
\begin{center}
\begin{tabular}{l}
|\||ifdefined\childdocname\endinput\||fi\newif\ifchilddoc|\\
|\edef\childdocname{\scantokens\expandafter{\jobname\noexpand}}|\\
|\def\childdocmain{|\textit{main}|}\||ifx\childdocmain\childdocname\||else|\\
|\childdoctrue\includeonly{\childdocname}\let\jobname\childdocmain\||fi|\\
\end{tabular}
\end{center}
%
Instead of |\childdocof{|\textit{main}|}| just include the main file
at the top of each child file:
%
\begin{center}
|\input{|\textit{main}|}|
\end{center}
%
A simple redirection |\childdocforward{|\textit{dest}|}| is achieved by:
%
\begin{center}
|\def\jobname{|\textit{dest}|}\input{\jobname}|
\end{center}
%
The redirection with prefix
|\childdocforwardprefix[|\textit{prefix}|]{|\textit{dest}|}|
is accomplished by:
%
\begin{center}
\begin{tabular}{l}
|{\edef\jobname{\scantokens\expandafter{\jobname\noexpand}}|\\
|\def\redirectjob |\textit{prefix}|#1~~~{\gdef\jobname{|\textit{dest}|#1}}|\\
|\expandafter\redirectjob\jobname~~~}\input{\jobname}|
\end{tabular}
\end{center}

In an alternative approach,
child documents can be compiled by a specific command line
without additional code or specific definitions:
%
\begin{center}
|... -jobname "|\textit{target}|" "|[\textit{flags}]%
|\includeonly{|\textit{dest}|}\input{|\textit{main}|}"|
\end{center}
%

%%%%%%%%%%%%%%%%%%%%%%%%%%%%%%%%%%%%%%%%%%%%%%%%%%%%%%%%%%%%%%%%%%%%%%%%%%%%%%%%
%%%%%%%%%%%%%%%%%%%%%%%%%%%%%%%%%%%%%%%%%%%%%%%%%%%%%%%%%%%%%%%%%%%%%%%%%%%%%%%%
\section{Information}

%%%%%%%%%%%%%%%%%%%%%%%%%%%%%%%%%%%%%%%%%%%%%%%%%%%%%%%%%%%%%%%%%%%%%%%%%%%%%%%%
\subsection{Copyright}

Copyright \copyright{} 2017--2018 Niklas Beisert

This work may be distributed and/or modified under the
conditions of the \LaTeX{} Project Public License, either version 1.3
of this license or (at your option) any later version.
The latest version of this license is in
  \url{http://www.latex-project.org/lppl.txt}
and version 1.3 or later is part of all distributions of \LaTeX{}
version 2005/12/01 or later.

This work has the LPPL maintenance status `maintained'.

The Current Maintainer of this work is Niklas Beisert.

This work consists of the files |README.txt|, |childdoc.ins| and |childdoc.dtx|
as well as the derived files |childdoc.def|, |cdocsamp.tex|
with |cdocsch1.tex|, |cdocsch2.tex|, |cdocspt3.tex|, |cdocspt4.tex|,
|cdocsdrf.tex|, |cdocsfn1.tex|, |cdocsfn2.tex|
as well as |childdoc.pdf|.

%%%%%%%%%%%%%%%%%%%%%%%%%%%%%%%%%%%%%%%%%%%%%%%%%%%%%%%%%%%%%%%%%%%%%%%%%%%%%%%%
\subsection{Files and Installation}

The package consists of the files:
%
\begin{center}
\begin{tabular}{ll}
    |README.txt|   & readme file \\
    |childdoc.ins| & installation file \\
    |childdoc.dtx| & source file \\
    |childdoc.def| & definition file \\
    |cdocsamp.tex| & sample main file \\
    |cdocsch1.tex| & sample include file \\
    |cdocsch2.tex| & sample include file \\
    |cdocspt3.tex| & sample part file \\
    |cdocspt4.tex| & sample part file \\
    |cdocsdrf.tex| & sample redirection file \\
    |cdocsfn1.tex| & sample redirection file \\
    |cdocsfn2.tex| & sample redirection file \\
    |childdoc.pdf| & manual
\end{tabular}
\end{center}
%
The distribution consists of the files
|README.txt|, |childdoc.ins| and |childdoc.dtx|.
%
\begin{itemize}
\item
Run (pdf)\LaTeX{} on |childdoc.dtx|
to compile the manual |childdoc.pdf| (this file).
\item
Run \LaTeX{} on |childdoc.ins| to create the definitions file |childdoc.def|
and the sample |cdocsamp.tex| with include files
|cdocsch1.tex|, |cdocsch2.tex|, |cdocspt3.tex|, |cdocspt4.tex|,
|cdocsdrf.tex|, |cdocsfn1.tex|, |cdocsfn2.tex|.
Then copy the file |childdoc.def| to an appropriate directory of your \LaTeX{}
distribution, e.g.\ \textit{texmf-root}|/tex/latex/childdoc|.
\end{itemize}

%%%%%%%%%%%%%%%%%%%%%%%%%%%%%%%%%%%%%%%%%%%%%%%%%%%%%%%%%%%%%%%%%%%%%%%%%%%%%%%%
\subsection{Related CTAN Packages}

There are several other packages which offer a similar functionality:
%
\begin{itemize}
\item
The packages
\href{http://ctan.org/pkg/docmute}{\textsf{docmute}},
\href{http://ctan.org/pkg/includex}{\textsf{includex}} and
\href{http://ctan.org/pkg/standalone}{\textsf{standalone}}
provide commands to include only the document body of
a child file thus allowing both files to be compiled individually.
\item
The packages \href{http://ctan.org/pkg/subdocs}{\textsf{subdocs}}
and \href{http://ctan.org/pkg/subfiles}{\textsf{subfiles}}
provide structures in which the main and child documents can be
encapsulated and allowing them to be compiled individually.
The inclusion mechanism is different from the conventional |\include|.
\item
The package \href{http://ctan.org/pkg/combine}{\textsf{combine}}
is an elaborate solution to combine several documents into one.
\end{itemize}
%
See also the CTAN topic \href{http://ctan.org/topic/subdocs}{\textsf{subdocs}}
for further related packages.
The present package differs from the above solutions in that
a document structure constructed with the conventional |\include| mechanism
just needs two extra commands at the top of every file
such that all constituent files can be compiled individually.

%%%%%%%%%%%%%%%%%%%%%%%%%%%%%%%%%%%%%%%%%%%%%%%%%%%%%%%%%%%%%%%%%%%%%%%%%%%%%%%%
%\subsection{Feature Suggestions}
%
%The following is a list of features which may be useful for future
%versions of this package:
%%
%\begin{itemize}
%\item
%\ldots
%\end{itemize}

%%%%%%%%%%%%%%%%%%%%%%%%%%%%%%%%%%%%%%%%%%%%%%%%%%%%%%%%%%%%%%%%%%%%%%%%%%%%%%%%
\subsection{Revision History}

%%%%%%%%%%%%%%%%%%%%%%%%%%%%%%%%%%%%%%%%
\paragraph{v2.0:} 2018/12/30

\begin{itemize}
\item
immediate forward processing
\item
added |\childdocby| mechanism
\item
manual restructured
\end{itemize}

%%%%%%%%%%%%%%%%%%%%%%%%%%%%%%%%%%%%%%%%
\paragraph{v1.6:} 2018/01/17

\begin{itemize}
\item
application for development of include files
\item
corrections to manual
\end{itemize}

%%%%%%%%%%%%%%%%%%%%%%%%%%%%%%%%%%%%%%%%
\paragraph{v1.5:} 2017/05/21

\begin{itemize}
\item
more complete structuring introduced
\item
|\childdocof| introduced
\item
|\childdoc| renamed to |\childdocmain|
\item
|\childredirect| renamed to |\childdocforward| and |\childdocforwardprefix|
and functionality expanded
\end{itemize}

%%%%%%%%%%%%%%%%%%%%%%%%%%%%%%%%%%%%%%%%
\paragraph{v1.0:} 2017/04/27

\begin{itemize}
\item
manual and install package
\item
first version published on CTAN
\end{itemize}

%%%%%%%%%%%%%%%%%%%%%%%%%%%%%%%%%%%%%%%%
\paragraph{v0.6:} 2017/04/26

\begin{itemize}
\item
redirection mechanism added
\end{itemize}

%%%%%%%%%%%%%%%%%%%%%%%%%%%%%%%%%%%%%%%%
\paragraph{v0.5:} 2017/04/26

\begin{itemize}
\item
functionality in definition file
\end{itemize}


%%%%%%%%%%%%%%%%%%%%%%%%%%%%%%%%%%%%%%%%%%%%%%%%%%%%%%%%%%%%%%%%%%%%%%%%%%%%%%%%
%%%%%%%%%%%%%%%%%%%%%%%%%%%%%%%%%%%%%%%%%%%%%%%%%%%%%%%%%%%%%%%%%%%%%%%%%%%%%%%%
%%%%%%%%%%%%%%%%%%%%%%%%%%%%%%%%%%%%%%%%%%%%%%%%%%%%%%%%%%%%%%%%%%%%%%%%%%%%%%%%
\appendix

\settowidth\MacroIndent{\rmfamily\scriptsize 000\ }

 \DocInput{childdoc.dtx}

\end{document}
%</driver>
% \fi
%
% %%%%%%%%%%%%%%%%%%%%%%%%%%%%%%%%%%%%%%%%%%%%%%%%%%%%%%%%%%%%%%%%%%%%%%%%%%%%%%
% %%%%%%%%%%%%%%%%%%%%%%%%%%%%%%%%%%%%%%%%%%%%%%%%%%%%%%%%%%%%%%%%%%%%%%%%%%%%%%
% \section{Sample}
%\iffalse
%<*samplemain>
%\fi
%
% The following presents a sample document
% with two chapters, two parts, a title page,
% a compile flag as well as three forwarding files to set the flag.
% It consists of eight |.tex| files:
% \begin{center}
% \begin{tabular}{ll}
% |cdocsamp.tex|&main file\\
% |cdocsch1.tex|&include file for chapter 1\\
% |cdocsch2.tex|&include file for chapter 2\\
% |cdocspt3.tex|&include file for part 3\\
% |cdocspt4.tex|&include file for part 4\\
% |cdocsdrf.tex|&forwarding file for main file in draft mode\\
% |cdocsfi1.tex|&forwarding file for final version of chapter 1\\
% |cdocsfi2.tex|&forwarding file for final version of chapter 2\\
% \end{tabular}
% \end{center}
% Each of the eight files can be compiled directly by the \LaTeX{} compiler.
%
% %%%%%%%%%%%%%%%%%%%%%%%%%%%%%%%%%%%%%%
% \paragraph{Main File.}
%
% The main file is called |cdocsamp.tex|.
%
% Load the \textsf{childdoc} definitions and
% declare the filename for the main document:
%    \begin{macrocode}
\input{childdoc.def}
\childdocmain{}
%    \end{macrocode}

% Optional override for |\version| flag:
%    \begin{macrocode}
%%\ifchilddoc\else\providecommand{\version}{draft}\fi
%    \end{macrocode}

% Define the default values for the |\version| flag
% (|final| for the main file and |draft| for childs):
%    \begin{macrocode}
\ifchilddoc
\providecommand{\version}{draft}
\else
\providecommand{\version}{final}
\fi
%    \end{macrocode}

% Load the standard document class:
%    \begin{macrocode}
\documentclass[12pt]{article}
%    \end{macrocode}

% Start the document body:
%    \begin{macrocode}
\begin{document}
%    \end{macrocode}

% Declare a title page.
% Print title, part of document being processed and version flag:
%    \begin{macrocode}
\addtocounter{page}{-1}
\begin{center}
{\LARGE\bfseries{}childdoc example\par}
\vspace{1cm}
\ifchilddoc
\ifchilddocmanual part\else chapter\fi:
`\childdocname' of `\childdocjob'\par
\else
main document: `\childdocjob'\par
\fi
version: \version\par
\end{center}
\newpage
%    \end{macrocode}

% Manually include selected file,
% otherwise process as usual:
%    \begin{macrocode}
\ifchilddocmanual
\section*{part `\childdocname'}
\input{\childdocname}
\else
%    \end{macrocode}

% Include the two chapters:
%    \begin{macrocode}
\include{cdocsch1}
\include{cdocsch2}
%    \end{macrocode}

% Include the two parts unless only chapters should be displayed:
%    \begin{macrocode}
\ifchilddoc\else
\section{part three}
\input{cdocspt3}
\section{part four}
\input{cdocspt4}
\fi
%    \end{macrocode}

% Process as usual until here:
%    \begin{macrocode}
\fi
%    \end{macrocode}

% End of document body:
%    \begin{macrocode}
\end{document}
%    \end{macrocode}
%\iffalse
%</samplemain>
%\fi
%
% %%%%%%%%%%%%%%%%%%%%%%%%%%%%%%%%%%%%%%
% \paragraph{Chapter Include Files.}
%
% The include files are called |cdocsch1.tex| and |cdocsch2.tex|.
%
%\iffalse
%<*samplechap1|samplechap2>
%\fi

% Optional override for |\version| flag:
%    \begin{macrocode}
%%\providecommand{\version}{final}
%    \end{macrocode}

% Include the main document:
%    \begin{macrocode}
\input{childdoc.def}
\childdocof{cdocsamp}
%    \end{macrocode}

%\iffalse
%</samplechap1|samplechap2>
%\fi
%
%\iffalse
%<*samplechap1>
%\fi
% Some text for chapter 1:
%    \begin{macrocode}
\section{one}
some text in chapter one
%    \end{macrocode}

%\iffalse
%</samplechap1>
%\fi
% Some text for chapter 2:
%\iffalse
%<*samplechap2>
%\fi
%    \begin{macrocode}
\section{two}
more text in chapter two
%    \end{macrocode}

%\iffalse
%</samplechap2>
%\fi
%
% %%%%%%%%%%%%%%%%%%%%%%%%%%%%%%%%%%%%%%
% \paragraph{Part Include Files.}
%
% The include files are called |cdocspt3.tex| and |cdocspt4.tex|.
%
%\iffalse
%<*samplepart3|samplepart4>
%\fi

% Optional override for |\version| flag:
%    \begin{macrocode}
%%\providecommand{\version}{final}
%    \end{macrocode}

% Include the main document:
%    \begin{macrocode}
\input{childdoc.def}
\childdocby{cdocsamp}
%    \end{macrocode}

%\iffalse
%</samplepart3|samplepart4>
%\fi
%
%\iffalse
%<*samplepart3>
%\fi
% Some text for part 3:
%    \begin{macrocode}
some text in part three
%    \end{macrocode}

%\iffalse
%</samplepart3>
%\fi
% Some text for part 4:
%\iffalse
%<*samplepart4>
%\fi
%    \begin{macrocode}
more text in part four
%    \end{macrocode}

%\iffalse
%</samplepart4>
%\fi
%
% %%%%%%%%%%%%%%%%%%%%%%%%%%%%%%%%%%%%%%
% \paragraph{Forwarding for a Complete Draft.}
%
% The following forwarding file |cdocsdrf.tex|
% compiles the main document in draft mode:
%\iffalse
%<*sampledraft>
%\fi
%    \begin{macrocode}
\def\version{draft}
\input{childdoc.def}
\childdocforward{cdocsamp}
%    \end{macrocode}

%\iffalse
%</sampledraft>
%\fi
%
% %%%%%%%%%%%%%%%%%%%%%%%%%%%%%%%%%%%%%%
% \paragraph{Forwarding for Final Version of the Chapters.}
%
% The following forwarding files |cdocsfn1.tex| and |cdocsfn2.tex|
% (with identical content)
% compile the final versions of the child documents
% |cdocsch1.tex| and |cdocsch2.tex|, respectively:
%\iffalse
%<*samplefinal>
%\fi
%    \begin{macrocode}
\def\version{final}
\input{childdoc.def}
\childdocforwardprefix[cdocsamp]{cdocsfn}{cdocsch}
%    \end{macrocode}

%\iffalse
%</samplefinal>
%\fi
%
% %%%%%%%%%%%%%%%%%%%%%%%%%%%%%%%%%%%%%%
% \paragraph{Command Line Processing.}
%
% The following three command lines generate the output files
% |cdocscld|, |cdocscl1| and |cdocscl2|
% which should be identical to
% |cdocsdrf|, |cdocsch1| and |cdocsfn2|, respectively:
% \begin{center}
% \begin{tabular}{l}
% |latex -jobname cdocscld \|\\
% |  "\def\version{draft}\input{childdoc.def}\childdocforward{cdocsamp}"|\\
% |latex -jobname cdocscl1 \|\\
% |  "\input{childdoc.def}\childdocforward[cdocsamp]{cdocsch1}"|\\
% |latex -jobname cdocscl2 \|\\
% |  "\def\version{final}\input{childdoc.def}\childdocforward{cdocsch2}"|
% \end{tabular}
% \end{center}
% Note that the trailing backslash on each first line
% merely continues the input to the second line
% (for convenient cut ant paste).
% Furthermore, the command |latex| can be replaced by any
% of its alternative versions such as |pdflatex|.
%
% %%%%%%%%%%%%%%%%%%%%%%%%%%%%%%%%%%%%%%%%%%%%%%%%%%%%%%%%%%%%%%%%%%%%%%%%%%%%%%
% %%%%%%%%%%%%%%%%%%%%%%%%%%%%%%%%%%%%%%%%%%%%%%%%%%%%%%%%%%%%%%%%%%%%%%%%%%%%%%
% \section{Implementation}
%\iffalse
%<*package>
%\fi
%
% This section describes the definitions file |childdoc.def|.

% The definitions cannot be loaded using |\usepackage| or |\RequirePackage|
% which has a mechanism to prevent loading a style file more than once.
% When loading the definitions by means of |\input|
% multiple instances have to be prevented manually:
%\iffalse
%This code needs to be before the `\ProvidesFile' directive
%which is defined at the beginning of this file.
%Therefore it is also placed there and commented out here.
%</package>
%<*discard>
%\fi
%    \begin{macrocode}
\ifdefined\childdocmain\endinput\fi
%    \end{macrocode}
%\iffalse
%</discard>
%<*package>
%\fi
%
% \macro{\ifchilddoc}
% \macro{\ifchilddocmanual}
% The conditional |\ifchilddoc| tells whether a
% child (true) or main (false) document is being compiled.
% The conditional |\ifchilddocmanual| tells whether
% the |\includeonly| mechanism is used (false) or
% the selection of child files must be performed manually (true).
% The definitions initialise to false:
%    \begin{macrocode}
\newif\ifchilddoc
\newif\ifchilddocmanual
%    \end{macrocode}

% \macro{\childdocname}
% \macro{\childdocjob}
% The macro |\childdocname| stores the name of the main document
% to be compiled. The macro |\childdocjob| stores the name of
% the document on which the \LaTeX{} compiler was originally invoked.
% The content of |\jobname| cannot be compared
% to filenames specified in the source due to different catcodes.
% The following code rescans |\jobname|, stores the result
% in |\childdocname| and saves a copy in |\childdocjob|:
%    \begin{macrocode}
\edef\childdocname{\scantokens\expandafter{\jobname\noexpand}}
\let\childdocjob\childdocname
%    \end{macrocode}

% \macro{\childdocdisable}
% The macro |\childdocdisable| prevents the main file
% from being processed more than once.
% At this stage, the main document command |\childdocmain|
% is assumed to be called once again where it should do nothing.
% Any subsequent call to it should prevent
% a secondary processing of the main document
% It overwrites the forwarding commands
% |\childdocof| and |\childdocforward|
% with empty macros to prevent further inclusions of the main document:
%    \begin{macrocode}
\newcommand{\childdocdisable}
{
  \renewcommand{\childdocmain}[1]{\renewcommand{\childdocmain}[1]{\endinput}}
  \renewcommand{\childdocof}[1]{}
  \renewcommand{\childdocby}[2][]{}
  \renewcommand{\childdocforward}[2][]{}
  \renewcommand{\childdocdisable}{}
}
%    \end{macrocode}

% \macro{\childdocmain}
% The macro |\childdocmain| is to be called at the top of the main file
% with nothing or the main filename (without extension) as argument.
% First, it breaks loops.
% If the argument is not empty and does not match |\childdocname|
% (which is set by the first inclusion of |childdoc.def|),
% |\ifchilddoc| is set to true, |\includeonly| is applied to the child file
% and |\jobname| is set to the main file
% (for proper handling of |.aux| files):
%    \begin{macrocode}
\newcommand{\childdocmain}[1]
{
  \childdocdisable\childdocmain{}
  \if?#1?\else
    \begingroup
      \def\childdoctmp{#1}
      \ifx\childdoctmp\childdocname
        \def\childdoctmp{}
      \else
        \def\childdoctmp
        {
          \childdoctrue
          \includeonly{\childdocname}
          \def\childdocjob{#1}
          \def\jobname{#1}
        }
      \fi
      \expandafter
    \endgroup
    \childdoctmp
  \fi
}
%    \end{macrocode}

% \macro{\childdocof}
% The command |\childdocof| redirects
% compilation to the main file |#1|.
%    \begin{macrocode}
\newcommand{\childdocof}[1]
{
  \childdocdisable
  \childdoctrue
  \includeonly{\childdocname}
  \def\jobname{#1}
  \def\childdocjob{#1}
  \input{#1}
}
%    \end{macrocode}

% \macro{\childdocby}
% The command |\childdocby| ....
%    \begin{macrocode}
\newcommand{\childdocby}[2][]
{
  \childdocdisable
  \childdoctrue
  \childdocmanualtrue
  \if?#1?\else
    \def\jobname{#2}
  \fi
  \def\childdocjob{#2}
  \input{#2}
  \endinput
}
%    \end{macrocode}

% \macro{\childdocforward}
% The command |\childdocforward| redirects
% compilation to the main file or
% (if the optional argument is given) a child file.
% Parameters are set as if the main file
% or a child file starting with |\childdocof| was compiled.
% Then compilation is handed over to the main file:
%    \begin{macrocode}
\newcommand{\childdocforward}[2][]
{
  \begingroup
    \if?#1?
      \def\childdoctmp
      {
        \def\childdocname{#2}
        \def\childdocjob{#2}
        \def\jobname{#2}
        \input{#2}
        \endinput
      }
    \else
      \def\childdoctmp
      {
        \childdocdisable
        \def\childdocname{#2}
        \childdoctrue
        \includeonly{#2}
        \def\childdocjob{#1}
        \def\jobname{#1}
        \input{#1}
        \endinput
      }
    \fi
    \expandafter
  \endgroup
  \childdoctmp
}
%    \end{macrocode}

% \macro{\childdocforwardprefix}
% The command |\childdocforwardprefix| redirects
% compilation to the main or a child file by means of a pattern.
% The prefix |#1| in the current filename is replaced by |#2|
% and the suffix of the current filename is kept
% (it is assumed that the filename does not contain the substring `|~~~|'
% which is used as a delimiter).
% Compilation is handed over to the new file by |\childdocforward|:
%    \begin{macrocode}
\newcommand{\childdocforwardprefix}[3][]
{
  \begingroup
    \def\childdocextract #2##1~~~{\def\childdoctmp{\childdocforward[#1]{#3##1}}}
    \expandafter\childdocextract\childdocname~~~
    \expandafter
  \endgroup
  \childdoctmp
}
%    \end{macrocode}

% \macro{\childdoc}
% The deprecated macro |\childdoc| is a legacy version of |\childdocmain|:
%    \begin{macrocode}
\newcommand{\childdoc}{\childdocmain}
%    \end{macrocode}

% \macro{\childdocredirect}
% The deprecated macro |\childdocredirect| is a legacy version
% of |\childdocforward| and |\childdocforwardprefix|:
%    \begin{macrocode}
\newcommand{\childdocredirect}[2][]
{
  \begingroup
    \if?#1?
      \def\childdoctmp{\childdocforward{#2}}
    \else
      \def\childdoctmp{\childdocforwardprefix{#1}{#2}}
    \fi
    \expandafter
  \endgroup
  \childdoctmp
}
%    \end{macrocode}

%\iffalse
%</package>
%\fi
%
\endinput
|\\
|\childdocby{|\textit{main}|}|\\
\end{tabular}
\end{center}
%
Both forms have slightly different effects as described above.
The main file is prepared as usual, see \secref{sec:include}.

%%%%%%%%%%%%%%%%%%%%%%%%%%%%%%%%%%%%%%%%%%%%%%%%%%%%%%%%%%%%%%%%%%%%%%%%%%%%%%%%
\subsection{Legacy Detection}
\label{sec:detection}

The directive |\childdocmain| in the main file can detect
whether the complete document or merely a child is to be compiled
even without using the directive |\childdocof|.
This method is deprecated because it is less robust
and there is no compelling reason to use it;
it is merely provided for backward compatibility
and it may be removed in future versions.

If the detection mechanism is to be used,
it is mandatory to correctly specify
the filename of the main file as the argument of |\childdocmain|:
%
\begin{center}
\begin{tabular}{l}
|% \iffalse
%
% childdoc.dtx Copyright (C) 2017-2018 Niklas Beisert
%
% This work may be distributed and/or modified under the
% conditions of the LaTeX Project Public License, either version 1.3
% of this license or (at your option) any later version.
% The latest version of this license is in
%   http://www.latex-project.org/lppl.txt
% and version 1.3 or later is part of all distributions of LaTeX
% version 2005/12/01 or later.
%
% This work has the LPPL maintenance status `maintained'.
%
% The Current Maintainer of this work is Niklas Beisert.
%
% This work consists of the files childdoc.dtx and childdoc.ins
% and the derived files childdoc.def and cdocsamp.tex with
% cdocsch1.tex, cdocsch2.tex, cdocsdrf.tex, cdocsfn1.tex, cdocsfn2.tex.
%
%<package>\ifdefined\childdocmain\endinput\fi
%<package>\ProvidesFile{childdoc.def}[2018/12/30 v2.0 child document driver]
%<samplemain>\ProvidesFile{cdocsamp.tex}[2018/12/30 v2.0 sample for childdoc]
%<*driver>
%\ProvidesFile{childdoc.drv}[2018/12/30 v2.0 childdoc reference manual file]
\PassOptionsToClass{10pt,a4paper}{article}
\documentclass{ltxdoc}

\usepackage[margin=35mm]{geometry}
\usepackage{hyperref}
\usepackage{hyperxmp}
\usepackage[usenames]{color}

\hypersetup{colorlinks=true}
\hypersetup{pdfstartview=FitH}
\hypersetup{pdfpagemode=UseNone}
\hypersetup{pdfsource={}}
\hypersetup{pdflang={en-UK}}
\hypersetup{pdfcopyright={Copyright 2017-2018 Niklas Beisert.
  This work may be distributed and/or modified under the
  conditions of the LaTeX Project Public License, either version 1.3
  of this license or (at your option) any later version.}}
\hypersetup{pdflicenseurl={http://www.latex-project.org/lppl.txt}}
\hypersetup{pdfcontactaddress={ETH Zurich, ITP, HIT K,
  Wolfgang-Pauli-Strasse 27}}
\hypersetup{pdfcontactpostcode={8093}}
\hypersetup{pdfcontactcity={Zurich}}
\hypersetup{pdfcontactcountry={Switzerland}}
\hypersetup{pdfcontactemail={nbeisert@itp.phys.ethz.ch}}
\hypersetup{pdfcontacturl={http://people.phys.ethz.ch/\xmptilde nbeisert/}}

\newcommand{\secref}[1]{\hyperref[#1]{section \ref*{#1}}}

\parskip1ex
\parindent0pt
\let\olditemize\itemize
\def\itemize{\olditemize\parskip0pt}

\begin{document}

\title{The \textsf{childdoc} Package}
\hypersetup{pdftitle={The childdoc Package}}
\author{Niklas Beisert\\[2ex]
  Institut f\"ur Theoretische Physik\\
  Eidgen\"ossische Technische Hochschule Z\"urich\\
  Wolfgang-Pauli-Strasse 27, 8093 Z\"urich, Switzerland\\[1ex]
  \href{mailto:nbeisert@itp.phys.ethz.ch}
  {\texttt{nbeisert@itp.phys.ethz.ch}}}
\hypersetup{pdfauthor={Niklas Beisert}}
\hypersetup{pdfsubject={Manual for the LaTeX2e Package childdoc}}
\date{30 December 2018, \textsf{v2.0}}
\maketitle

\begin{abstract}\noindent
\textsf{childdoc} is a \LaTeXe{} package
that enables the direct compilation
of document sections included by |\include|
to individual files.
\end{abstract}

\begingroup
\parskip0ex
\tableofcontents
\endgroup

%%%%%%%%%%%%%%%%%%%%%%%%%%%%%%%%%%%%%%%%%%%%%%%%%%%%%%%%%%%%%%%%%%%%%%%%%%%%%%%%
%%%%%%%%%%%%%%%%%%%%%%%%%%%%%%%%%%%%%%%%%%%%%%%%%%%%%%%%%%%%%%%%%%%%%%%%%%%%%%%%
\section{Introduction}

\LaTeX{} provides a mechanism to structure a large document (such as a book)
into a main file and several child files (containing the chapters)
using the |\include| command.
This mechanism is beneficial for documents
which span hundreds of pages in order to
make the source file(s) more manageable.
Moreover, compilation can be restricted to
selected child files by means of the |\includeonly| command.
The latter feature can be used to reduce the compilation time while editing
(this was significantly more useful in the earlier days of \LaTeX{})
or to generate a smaller document which is easier to navigate.
Another application of |\includeonly| is to generate
documents consisting of selected parts of the complete document.

However, there are a few drawbacks of the plain |\include| mechanism:
\begin{itemize}
\item
The child files cannot be compiled on their own,
they can only be compiled via the main file.
A naive editing environment
(such as a text editor with an option
to have the current file processed by \LaTeX)
may require one to switch to the main file before compiling;
attempting to compile the child file produces errors.
\item
The main file must be modified (each time)
to adjust the |\includeonly| command
to the present needs. This easily leaves the main file in a messy state.
\item
The generated document will always carry the filename
of the main document. This is inconvenient if
several child files are to be compiled and
to be kept for distribution.
\end{itemize}

The present package provides a simple interface
to make child files individually compilable by \LaTeX{}.
Compiling a child file then has the same effect as compiling
the main file with an |\includeonly| command
to select the appropriate child.
Moreover the generated document will carry the name of the child
rather than the main file.
This resolves all three above issues.

This feature is meant to make the editing of books,
thesis documents and lecture notes somewhat more convenient.
However, the package can also be used efficiently for
composing a series of documents (such as exercise sheets)
which are typically distributed individually.
It then assists the author in generating the individual documents
(potentially in different versions)
as well as a document containing the collected series.
Another application is in developing style files
or other kinds of included material
where compilation of the style file could redirect
to a sample or test file.

%%%%%%%%%%%%%%%%%%%%%%%%%%%%%%%%%%%%%%%%%%%%%%%%%%%%%%%%%%%%%%%%%%%%%%%%%%%%%%%%
%%%%%%%%%%%%%%%%%%%%%%%%%%%%%%%%%%%%%%%%%%%%%%%%%%%%%%%%%%%%%%%%%%%%%%%%%%%%%%%%
\section{Usage}

First of all, the package \textsf{childdoc} is \emph{not} a standard
\LaTeXe{} |.sty| style file! Therefore it needs to be invoked in
a non-standard way.

%%%%%%%%%%%%%%%%%%%%%%%%%%%%%%%%%%%%%%%%%%%%%%%%%%%%%%%%%%%%%%%%%%%%%%%%%%%%%%%%
\subsection{Included Files}
\label{sec:include}

%%%%%%%%%%%%%%%%%%%%%%%%%%%%%%%%%%%%%%%%
\DescribeMacro{\childdocmain}
To use the package, add the commands
\begin{center}
\begin{tabular}{l}
|\input{childdoc.def}|\\
|\childdocmain{}|\\
\end{tabular}
\end{center}
at the very top of the main \LaTeX{} file,
in particular \emph{before} the |\documentclass| statement!
The argument of |\childdocmain| should be left empty
(but it must be present).

%%%%%%%%%%%%%%%%%%%%%%%%%%%%%%%%%%%%%%%%
\DescribeMacro{\childdocof}
Furthermore, add the commands
\begin{center}
\begin{tabular}{l}
|\input{childdoc.def}|\\
|\childdocof{|\textit{main}|}|\\
\end{tabular}
\end{center}
at the top of every child file \textit{child}
which is included by |\include{|\textit{child}|}|
from within the main file
(or at least for those files to be compiled individually).
The argument \textit{main} must be the filename of the main file.

There are a couple of
considerations in setting up the main and child documents:

%%%%%%%%%%%%%%%%%%%%%%%%%%%%%%%%%%%%%%%%
\paragraph{Restrictions.}

Please note the following restrictions:
\begin{itemize}
\item
|\childdocmain| must be called with one argument \textit{main}
to ensure compatibility with earlier version of the package.
It must either be empty (|\childdocmain{}|)
or precisely match the filename of the main file in which it is specified.
See \secref{sec:detection} for further information.
\item
The filename \textit{main} must be specified without the |.tex| extension.
\item
The filename \textit{main} is case sensitive
(even in case-insensitive file systems)
due to internal string comparison.
\item
The argument \textit{main} should be fully expanded, it cannot be a macro.
\item
Subdirectories and special characters should be avoided in filenames.
\item
The command |\childdocmain{|\textit{main}|}| must be followed by a whitespace.
It should not be followed immediately by another command
or by a comment mark `|%|'.
This is because the \TeX{} parser reads the token immediately following
the argument of |\childdocmain| and puts it
at the beginning of every child section;
however, a white\-space is ignored.
\end{itemize}

%%%%%%%%%%%%%%%%%%%%%%%%%%%%%%%%%%%%%%%%
\paragraph{Content of Main File.}

It is advisable to place all content in the child files included by |\include|.
Any output contained in the main file will appear in all child documents
unless suppressed manually;
it cannot be suppressed automatically by the |\includeonly| directive
and thus should normally be avoided.
A method to include some content in the main file
by means of conditional processing is described in \secref{sec:conditional}.

%%%%%%%%%%%%%%%%%%%%%%%%%%%%%%%%%%%%%%%%
\paragraph{Page Numbering.}

When only a part of the document is compiled,
the appropriate numbering of pages
(as well as other status parameters)
is determined from the |.aux| files.
The latter contain information from previous passes.
However this information needs to propagate through
all intermediate child documents.
Therefore the page numbering in child documents may well
be inconsistent until the complete document is compiled at least once.

A useful (if unconventional) way to always ensure a consistent
page numbering is to restart the numbering in each child document
and denote the pages by `\textit{child}|.|\textit{page}'
where \textit{child} represents the chapter/section number of the child file.
This can be achieved by the command
|\numberwithin{page}{|\textit{child}|}|
of the \textsf{amsmath} package
where \textit{child} can be |chapter| or |section|
depending on the chosen structuring.
Alternatively, one can modify the macro |\thepage| appropriately
and reset the counter |page| at the start of each child file.

%%%%%%%%%%%%%%%%%%%%%%%%%%%%%%%%%%%%%%%%%%%%%%%%%%%%%%%%%%%%%%%%%%%%%%%%%%%%%%%%
\subsection{Conditional Processing}
\label{sec:conditional}

The package provides a mechanism to compile different versions
of a document. To customise the versions further some conditional processing
can come in handy to distinguish which version is being compiled.
The package provides two macros to describe the compilation context:

%%%%%%%%%%%%%%%%%%%%%%%%%%%%%%%%%%%%%%%%
\DescribeMacro{\ifchilddoc}
The conditional |\ifchilddoc| distinguishes between the compilation of
child documents and the main document:
%
\begin{center}
|\ifchilddoc |\textit{child-code}| |[|\||else |\textit{main-code}]| \||fi|
\end{center}

%%%%%%%%%%%%%%%%%%%%%%%%%%%%%%%%%%%%%%%%
\DescribeMacro{\childdocname}
\DescribeMacro{\childdocjob}
The macro |\childdocname| contains the filename (without extension)
of the main or child file being processed.
Note that |\childdocjob| will always contain the name of the main file.

%%%%%%%%%%%%%%%%%%%%%%%%%%%%%%%%%%%%%%%%
\paragraph{Title Page.}

Conditional processing can be used to include a title or banner page
in the main document when proper precautions are taken.
Importantly, the code in the main file should ensure that the page counter
(as well as other status parameters which are stored in the |.aux| files)
takes the same value after the conditional processing.
Otherwise the page numbers may take divergent values
depending on which part is compiled.

For example, a title page could be declared by:
%
\begin{center}
\begin{tabular}{l}
|\ifchilddoc\||else|\\
|\addtocounter{page}{-1}|\\
\textit{code for title page}\\
|\newpage|\\
|\||fi|
\end{tabular}
\end{center}
%
A banner page for the child documents can be generated by:
%
\begin{center}
\begin{tabular}{l}
|\ifchilddoc|\\
|\addtocounter{page}{-1}|\\
\textit{code for banner page}\\
|\newpage|\\
|\||fi|
\end{tabular}
\end{center}
%
Here one could write a message such as:
\begin{center}
|This is the part \childdocname{} of \childdocjob{}.|
\end{center}

%%%%%%%%%%%%%%%%%%%%%%%%%%%%%%%%%%%%%%%%%%%%%%%%%%%%%%%%%%%%%%%%%%%%%%%%%%%%%%%%
\subsection{Flags}
\label{sec:flags}

The package makes it easy to generate different versions
of the main or child documents.
To this end compilation flags can be defined
and assigned different default values.
They will be particularly useful in conjunction
with the forwarding mechanism described in \secref{sec:forward}.

For example, it may be useful to have a flag |\version|
which can be set to |draft| or |final|.
The document source will contain some conditional code
depending on the value of |\version|.
Suppose further, the flag should default to |final| for the main file
and to |draft| for child files
which is a natural assignment for editing the document.
This is achieved by placing the following code
in the preamble of the main document
(below the |\childdocmain| directive):
%
\begin{center}
\begin{tabular}{l}
|\ifchilddoc|\\
|\providecommand{\version}{draft}|\\
|\||else|\\
|\providecommand{\version}{final}|\\
|\||fi|
\end{tabular}
\end{center}
%
The definition by |\providecommand| makes sure
that previous definitions are not overwritten.
Further statements |\providecommand{\version}{...}|
can thus be added before the above code to override it.

For the main file, one might add a line
(between |\childdocmain| and the above block)
%
\begin{center}
|%\ifchilddoc\||else\providecommand{\version}{draft}\||fi|
\end{center}
%
which can be uncommented to produce a draft version.
Likewise one can add a line to the very top of a child file
(above the |\childdocof{|\textit{main}|}| directive)
%
\begin{center}
|%\providecommand{\version}{final}|
\end{center}
%
which can be uncommented to produce the final version of this child document.

%%%%%%%%%%%%%%%%%%%%%%%%%%%%%%%%%%%%%%%%%%%%%%%%%%%%%%%%%%%%%%%%%%%%%%%%%%%%%%%%
\subsection{Forwarding}
\label{sec:forward}

Different versions of the main or child documents
using compilation flags as described in \secref{sec:flags}
can be (permanently) stored in different files
for convenient compilation, viewing and distribution.
To this end, the package defines a command
to pass on compilation to a different file:

%%%%%%%%%%%%%%%%%%%%%%%%%%%%%%%%%%%%%%%%
\DescribeMacro{\childdocforward}
The command |\childdocforward| redirects processing to
another source file:
%
\begin{center}
\begin{tabular}{l}
|\input{childdoc.def}|\\
|\childdocforward[|\textit{main}|]{|\textit{dest}|}|\\
\end{tabular}
\end{center}
%
The argument \textit{dest} is the destination file
(without extension).
It should be the main file or one of the child files.
Note that further \textsf{childdoc} directives
such as |\childdocof| and |\childdocforward|
in the indicated file will be processed in this form.
The optional argument \textit{main}
passes on directly to the main file \textit{main}
while pretending to compile the child \textit{dest}.
This form behaves as if \textit{dest}
issues |\childdocof{|\textit{main}|}| right away,
and no further \textsf{childdoc} directives will be processed.

%%%%%%%%%%%%%%%%%%%%%%%%%%%%%%%%%%%%%%%%
\DescribeMacro{\...prefix}
In the alternative form |\childdocforwardprefix|,
%
\begin{center}
\begin{tabular}{l}
|\input{childdoc.def}|\\
|\childdocforwardprefix[|\textit{main}|]{|\textit{prefix}|}{|\textit{dest}|}|
\end{tabular}
\end{center}
%
the destination file is determined by a pattern
depending on the current file:
To make this work, the current file must be called
`{\textit{prefix}\hspace{0.2em}\textit{suffix}}'
with \textit{prefix} matching precisely the argument.
Processing is then passed on to the file
`{\textit{dest}\hspace{0.2em}\textit{suffix}}'.
Surely, the same effect is achieved by
directly specifying the
argument `{\textit{dest}\hspace{0.2em}\textit{suffix}}'
in the first form.
However, that requires to set up a different file
for each child. With the alternative form of the command
all these files can have exactly the same content
which simplifies setting them up and maintaining them.

For example, the following file |draft.tex|
with a compilation flag |\version| as described in \secref{sec:flags}
compiles the main document as a draft:
%
\begin{center}
\begin{tabular}{l}
|\def\version{draft}|\\
|\input{childdoc.def}|\\
|\childdocforward{|\textit{main}|}|
\end{tabular}
\end{center}
%
Likewise, the following files |final|\textit{nn}|.tex|
compile the final version of the child document
|child|\textit{nn}|.tex|:
%
\begin{center}
\begin{tabular}{l}
|\def\version{final}|\\
|\input{childdoc.def}|\\
|\childdocforwardprefix{final}{child}|
\end{tabular}
\end{center}
%

Note that when several versions of a main file and/or of each child file
are to be generated, it may be convenient to set up a |Makefile| or
shell script to automatise the process.

%%%%%%%%%%%%%%%%%%%%%%%%%%%%%%%%%%%%%%%%%%%%%%%%%%%%%%%%%%%%%%%%%%%%%%%%%%%%%%%%
\subsection{Command Line Processing}
\label{sec:commandline}

The effect of redirection files can also be achieved by invoking
the \LaTeX{} compiler with a more elaborate command line.
Most conveniently this should be done as part
of a shell script or a |Makefile|.

When using \textsf{childdoc} in the main file, the following
command lines effectively perform a redirection
(note that depending on the shell being used,
backslashes may have to be doubled: `|\|' $\to$ `|\\|'):
%
\begin{center}
|... -jobname "|\textit{target}|" |\\|"|[\textit{flags}]%
|\input{childdoc.def}\childdocforward[|\textit{main}|]{|\textit{dest}|}"|
\end{center}
%
Here \textit{target} is the name of the output file,
\textit{main} is the name of the main file
and \textit{dest} is the name of the main or child file to be processed
(all filenames without extensions).
The optional argument \textit{main} can be omitted
if \textit{main} matches \textit{dest}.
Optionally, compilation \textit{flags} can be defined via |\def| commands.
This command line makes the \TeX{} engine believe
it is compiling the file \textit{target}
whose content is specified as the latter parameter.
The provided code then forwards the processing to
\textit{main} or \textit{dest} as described in \secref{sec:forward}.

%%%%%%%%%%%%%%%%%%%%%%%%%%%%%%%%%%%%%%%%%%%%%%%%%%%%%%%%%%%%%%%%%%%%%%%%%%%%%%%%
\subsection{Include by Input}
\label{sec:input}

Including child documents by |\include| has some restrictions by design.
Most notably, the content of a child document always occupies
its own set of pages; pages cannot be shared between child documents.
Usually, this behaviour makes perfect sense
because each child document contain an essential part of the document.
However, in some situations it may be desirable to compose
a document from a collection of parts
without having mandatory page breaks between then.
For this case, the package
provides a mechanism to include parts
by |\input| which can also be processed individually.
However, by construction this mechanism
requires manual handling of the content to be output.

%%%%%%%%%%%%%%%%%%%%%%%%%%%%%%%%%%%%%%%%
\DescribeMacro{\ifchilddocmanual}
The main file should be prepared as usual, see \secref{sec:include}.
However, the document body must make a distinction
between processing of an individual part and of the main document, e.g.:
%
\begin{center}
\begin{tabular}{l}
|\ifchilddocmanual|\\
|\input{\childdocname}|\\
|\||else|\\
\textit{document body with }|\input{|\textit{part}|}|\\
|\||fi|
\end{tabular}
\end{center}
%
The conditional |\ifchilddocmanual| is true whenever
a part to be included by |\input| is being compiled,
and the name of the part is stored in |\childdocname|.

%%%%%%%%%%%%%%%%%%%%%%%%%%%%%%%%%%%%%%%%
\DescribeMacro{\childdocby}
Each part to be included by |\input| should start with:
%
\begin{center}
\begin{tabular}{l}
|\input{childdoc.def}|\\
|\childdocby{|\textit{main}|}|\\
\end{tabular}
\end{center}
%
The directive |\childdocby| is similar to |\childdocof|
described in \secref{sec:include},
but the subsequent selection of content must be done manually.
To that end, both |\ifchilddoc| and |\ifchilddocmanual|
will be true upon processing of a part,
and the name of the part is stored in |\childdocname|.
Note that |\jobname| will be set to the filename of the current part
so that each part receives an individual |.aux| file
that does not interfere with the |.aux| file(s) of the main document.
This behaviour can be altered by the alternative form
|\childdocby[*]{|\textit{main}|}| (with a non-empty optional argument)
which uses the |.aux| file of the main document
by setting |\jobname| to \textit{main}.

%%%%%%%%%%%%%%%%%%%%%%%%%%%%%%%%%%%%%%%%%%%%%%%%%%%%%%%%%%%%%%%%%%%%%%%%%%%%%%%%
\subsection{Driver Development}
\label{sec:driver}

The \textsf{childdoc} mechanism can also be use for the development
of definition files such as \LaTeX{} styles or classes.
This case differs from the above setup with multiple parts
included by |\include| in that no |\includeonly| should be invoked.
This can be achieved by starting the include file
(before |\ProvidesPackage|) with:
%
\begin{center}
\begin{tabular}{l}
|\input{childdoc.def}|\\
|\childdocforward{|\textit{main}|}|\\
\end{tabular}
\end{center}
%
or alternatively with:
%
\begin{center}
\begin{tabular}{l}
|\input{childdoc.def}|\\
|\childdocby{|\textit{main}|}|\\
\end{tabular}
\end{center}
%
Both forms have slightly different effects as described above.
The main file is prepared as usual, see \secref{sec:include}.

%%%%%%%%%%%%%%%%%%%%%%%%%%%%%%%%%%%%%%%%%%%%%%%%%%%%%%%%%%%%%%%%%%%%%%%%%%%%%%%%
\subsection{Legacy Detection}
\label{sec:detection}

The directive |\childdocmain| in the main file can detect
whether the complete document or merely a child is to be compiled
even without using the directive |\childdocof|.
This method is deprecated because it is less robust
and there is no compelling reason to use it;
it is merely provided for backward compatibility
and it may be removed in future versions.

If the detection mechanism is to be used,
it is mandatory to correctly specify
the filename of the main file as the argument of |\childdocmain|:
%
\begin{center}
\begin{tabular}{l}
|\input{childdoc.def}|\\
|\childdocmain{|\textit{main}|}|\\
\end{tabular}
\end{center}
%
If |\jobname| does not match the argument \textit{main} of |\childdocmain|,
it is assumed that |\jobname| points to the child file to be compiled.
When using |\childdocmain| with the main file specified as argument,
it suffices to start a child file
with just |\input{|\textit{main}|}|
without loading of the package and using |\childdocof|.
If instead all processing is done
with the appropriate \textsf{childdoc} directives,
the argument of \textit{main} of |\childdocmain| can be empty.

An alternative version of the command line processing described
in \secref{sec:commandline} using the detection mechanism reads:
%
\begin{center}
|... -jobname "|\textit{target}|" "|[\textit{flags}]%
[|\def\jobname{|\textit{dest}|}|]|\input{|\textit{main}|}"|
\end{center}

%%%%%%%%%%%%%%%%%%%%%%%%%%%%%%%%%%%%%%%%%%%%%%%%%%%%%%%%%%%%%%%%%%%%%%%%%%%%%%%%
\subsection{Manual Code}
\label{sec:manual}

In case one cannot be certain whether the definitions file |childdoc.def|
is installed on the target \TeX{} distribution
and one prefers not to ship it,
it is conceivable to paste a few relevant commands into the sources.

To that end, drop all statements |\input{childdoc.def}|
and perform the replacements as outlined below.
Instead of |\childdocmain{|\textit{main}|}| add the following code
to the top of the main file:
%
\begin{center}
\begin{tabular}{l}
|\||ifdefined\childdocname\endinput\||fi\newif\ifchilddoc|\\
|\edef\childdocname{\scantokens\expandafter{\jobname\noexpand}}|\\
|\def\childdocmain{|\textit{main}|}\||ifx\childdocmain\childdocname\||else|\\
|\childdoctrue\includeonly{\childdocname}\let\jobname\childdocmain\||fi|\\
\end{tabular}
\end{center}
%
Instead of |\childdocof{|\textit{main}|}| just include the main file
at the top of each child file:
%
\begin{center}
|\input{|\textit{main}|}|
\end{center}
%
A simple redirection |\childdocforward{|\textit{dest}|}| is achieved by:
%
\begin{center}
|\def\jobname{|\textit{dest}|}\input{\jobname}|
\end{center}
%
The redirection with prefix
|\childdocforwardprefix[|\textit{prefix}|]{|\textit{dest}|}|
is accomplished by:
%
\begin{center}
\begin{tabular}{l}
|{\edef\jobname{\scantokens\expandafter{\jobname\noexpand}}|\\
|\def\redirectjob |\textit{prefix}|#1~~~{\gdef\jobname{|\textit{dest}|#1}}|\\
|\expandafter\redirectjob\jobname~~~}\input{\jobname}|
\end{tabular}
\end{center}

In an alternative approach,
child documents can be compiled by a specific command line
without additional code or specific definitions:
%
\begin{center}
|... -jobname "|\textit{target}|" "|[\textit{flags}]%
|\includeonly{|\textit{dest}|}\input{|\textit{main}|}"|
\end{center}
%

%%%%%%%%%%%%%%%%%%%%%%%%%%%%%%%%%%%%%%%%%%%%%%%%%%%%%%%%%%%%%%%%%%%%%%%%%%%%%%%%
%%%%%%%%%%%%%%%%%%%%%%%%%%%%%%%%%%%%%%%%%%%%%%%%%%%%%%%%%%%%%%%%%%%%%%%%%%%%%%%%
\section{Information}

%%%%%%%%%%%%%%%%%%%%%%%%%%%%%%%%%%%%%%%%%%%%%%%%%%%%%%%%%%%%%%%%%%%%%%%%%%%%%%%%
\subsection{Copyright}

Copyright \copyright{} 2017--2018 Niklas Beisert

This work may be distributed and/or modified under the
conditions of the \LaTeX{} Project Public License, either version 1.3
of this license or (at your option) any later version.
The latest version of this license is in
  \url{http://www.latex-project.org/lppl.txt}
and version 1.3 or later is part of all distributions of \LaTeX{}
version 2005/12/01 or later.

This work has the LPPL maintenance status `maintained'.

The Current Maintainer of this work is Niklas Beisert.

This work consists of the files |README.txt|, |childdoc.ins| and |childdoc.dtx|
as well as the derived files |childdoc.def|, |cdocsamp.tex|
with |cdocsch1.tex|, |cdocsch2.tex|, |cdocspt3.tex|, |cdocspt4.tex|,
|cdocsdrf.tex|, |cdocsfn1.tex|, |cdocsfn2.tex|
as well as |childdoc.pdf|.

%%%%%%%%%%%%%%%%%%%%%%%%%%%%%%%%%%%%%%%%%%%%%%%%%%%%%%%%%%%%%%%%%%%%%%%%%%%%%%%%
\subsection{Files and Installation}

The package consists of the files:
%
\begin{center}
\begin{tabular}{ll}
    |README.txt|   & readme file \\
    |childdoc.ins| & installation file \\
    |childdoc.dtx| & source file \\
    |childdoc.def| & definition file \\
    |cdocsamp.tex| & sample main file \\
    |cdocsch1.tex| & sample include file \\
    |cdocsch2.tex| & sample include file \\
    |cdocspt3.tex| & sample part file \\
    |cdocspt4.tex| & sample part file \\
    |cdocsdrf.tex| & sample redirection file \\
    |cdocsfn1.tex| & sample redirection file \\
    |cdocsfn2.tex| & sample redirection file \\
    |childdoc.pdf| & manual
\end{tabular}
\end{center}
%
The distribution consists of the files
|README.txt|, |childdoc.ins| and |childdoc.dtx|.
%
\begin{itemize}
\item
Run (pdf)\LaTeX{} on |childdoc.dtx|
to compile the manual |childdoc.pdf| (this file).
\item
Run \LaTeX{} on |childdoc.ins| to create the definitions file |childdoc.def|
and the sample |cdocsamp.tex| with include files
|cdocsch1.tex|, |cdocsch2.tex|, |cdocspt3.tex|, |cdocspt4.tex|,
|cdocsdrf.tex|, |cdocsfn1.tex|, |cdocsfn2.tex|.
Then copy the file |childdoc.def| to an appropriate directory of your \LaTeX{}
distribution, e.g.\ \textit{texmf-root}|/tex/latex/childdoc|.
\end{itemize}

%%%%%%%%%%%%%%%%%%%%%%%%%%%%%%%%%%%%%%%%%%%%%%%%%%%%%%%%%%%%%%%%%%%%%%%%%%%%%%%%
\subsection{Related CTAN Packages}

There are several other packages which offer a similar functionality:
%
\begin{itemize}
\item
The packages
\href{http://ctan.org/pkg/docmute}{\textsf{docmute}},
\href{http://ctan.org/pkg/includex}{\textsf{includex}} and
\href{http://ctan.org/pkg/standalone}{\textsf{standalone}}
provide commands to include only the document body of
a child file thus allowing both files to be compiled individually.
\item
The packages \href{http://ctan.org/pkg/subdocs}{\textsf{subdocs}}
and \href{http://ctan.org/pkg/subfiles}{\textsf{subfiles}}
provide structures in which the main and child documents can be
encapsulated and allowing them to be compiled individually.
The inclusion mechanism is different from the conventional |\include|.
\item
The package \href{http://ctan.org/pkg/combine}{\textsf{combine}}
is an elaborate solution to combine several documents into one.
\end{itemize}
%
See also the CTAN topic \href{http://ctan.org/topic/subdocs}{\textsf{subdocs}}
for further related packages.
The present package differs from the above solutions in that
a document structure constructed with the conventional |\include| mechanism
just needs two extra commands at the top of every file
such that all constituent files can be compiled individually.

%%%%%%%%%%%%%%%%%%%%%%%%%%%%%%%%%%%%%%%%%%%%%%%%%%%%%%%%%%%%%%%%%%%%%%%%%%%%%%%%
%\subsection{Feature Suggestions}
%
%The following is a list of features which may be useful for future
%versions of this package:
%%
%\begin{itemize}
%\item
%\ldots
%\end{itemize}

%%%%%%%%%%%%%%%%%%%%%%%%%%%%%%%%%%%%%%%%%%%%%%%%%%%%%%%%%%%%%%%%%%%%%%%%%%%%%%%%
\subsection{Revision History}

%%%%%%%%%%%%%%%%%%%%%%%%%%%%%%%%%%%%%%%%
\paragraph{v2.0:} 2018/12/30

\begin{itemize}
\item
immediate forward processing
\item
added |\childdocby| mechanism
\item
manual restructured
\end{itemize}

%%%%%%%%%%%%%%%%%%%%%%%%%%%%%%%%%%%%%%%%
\paragraph{v1.6:} 2018/01/17

\begin{itemize}
\item
application for development of include files
\item
corrections to manual
\end{itemize}

%%%%%%%%%%%%%%%%%%%%%%%%%%%%%%%%%%%%%%%%
\paragraph{v1.5:} 2017/05/21

\begin{itemize}
\item
more complete structuring introduced
\item
|\childdocof| introduced
\item
|\childdoc| renamed to |\childdocmain|
\item
|\childredirect| renamed to |\childdocforward| and |\childdocforwardprefix|
and functionality expanded
\end{itemize}

%%%%%%%%%%%%%%%%%%%%%%%%%%%%%%%%%%%%%%%%
\paragraph{v1.0:} 2017/04/27

\begin{itemize}
\item
manual and install package
\item
first version published on CTAN
\end{itemize}

%%%%%%%%%%%%%%%%%%%%%%%%%%%%%%%%%%%%%%%%
\paragraph{v0.6:} 2017/04/26

\begin{itemize}
\item
redirection mechanism added
\end{itemize}

%%%%%%%%%%%%%%%%%%%%%%%%%%%%%%%%%%%%%%%%
\paragraph{v0.5:} 2017/04/26

\begin{itemize}
\item
functionality in definition file
\end{itemize}


%%%%%%%%%%%%%%%%%%%%%%%%%%%%%%%%%%%%%%%%%%%%%%%%%%%%%%%%%%%%%%%%%%%%%%%%%%%%%%%%
%%%%%%%%%%%%%%%%%%%%%%%%%%%%%%%%%%%%%%%%%%%%%%%%%%%%%%%%%%%%%%%%%%%%%%%%%%%%%%%%
%%%%%%%%%%%%%%%%%%%%%%%%%%%%%%%%%%%%%%%%%%%%%%%%%%%%%%%%%%%%%%%%%%%%%%%%%%%%%%%%
\appendix

\settowidth\MacroIndent{\rmfamily\scriptsize 000\ }

 \DocInput{childdoc.dtx}

\end{document}
%</driver>
% \fi
%
% %%%%%%%%%%%%%%%%%%%%%%%%%%%%%%%%%%%%%%%%%%%%%%%%%%%%%%%%%%%%%%%%%%%%%%%%%%%%%%
% %%%%%%%%%%%%%%%%%%%%%%%%%%%%%%%%%%%%%%%%%%%%%%%%%%%%%%%%%%%%%%%%%%%%%%%%%%%%%%
% \section{Sample}
%\iffalse
%<*samplemain>
%\fi
%
% The following presents a sample document
% with two chapters, two parts, a title page,
% a compile flag as well as three forwarding files to set the flag.
% It consists of eight |.tex| files:
% \begin{center}
% \begin{tabular}{ll}
% |cdocsamp.tex|&main file\\
% |cdocsch1.tex|&include file for chapter 1\\
% |cdocsch2.tex|&include file for chapter 2\\
% |cdocspt3.tex|&include file for part 3\\
% |cdocspt4.tex|&include file for part 4\\
% |cdocsdrf.tex|&forwarding file for main file in draft mode\\
% |cdocsfi1.tex|&forwarding file for final version of chapter 1\\
% |cdocsfi2.tex|&forwarding file for final version of chapter 2\\
% \end{tabular}
% \end{center}
% Each of the eight files can be compiled directly by the \LaTeX{} compiler.
%
% %%%%%%%%%%%%%%%%%%%%%%%%%%%%%%%%%%%%%%
% \paragraph{Main File.}
%
% The main file is called |cdocsamp.tex|.
%
% Load the \textsf{childdoc} definitions and
% declare the filename for the main document:
%    \begin{macrocode}
\input{childdoc.def}
\childdocmain{}
%    \end{macrocode}

% Optional override for |\version| flag:
%    \begin{macrocode}
%%\ifchilddoc\else\providecommand{\version}{draft}\fi
%    \end{macrocode}

% Define the default values for the |\version| flag
% (|final| for the main file and |draft| for childs):
%    \begin{macrocode}
\ifchilddoc
\providecommand{\version}{draft}
\else
\providecommand{\version}{final}
\fi
%    \end{macrocode}

% Load the standard document class:
%    \begin{macrocode}
\documentclass[12pt]{article}
%    \end{macrocode}

% Start the document body:
%    \begin{macrocode}
\begin{document}
%    \end{macrocode}

% Declare a title page.
% Print title, part of document being processed and version flag:
%    \begin{macrocode}
\addtocounter{page}{-1}
\begin{center}
{\LARGE\bfseries{}childdoc example\par}
\vspace{1cm}
\ifchilddoc
\ifchilddocmanual part\else chapter\fi:
`\childdocname' of `\childdocjob'\par
\else
main document: `\childdocjob'\par
\fi
version: \version\par
\end{center}
\newpage
%    \end{macrocode}

% Manually include selected file,
% otherwise process as usual:
%    \begin{macrocode}
\ifchilddocmanual
\section*{part `\childdocname'}
\input{\childdocname}
\else
%    \end{macrocode}

% Include the two chapters:
%    \begin{macrocode}
\include{cdocsch1}
\include{cdocsch2}
%    \end{macrocode}

% Include the two parts unless only chapters should be displayed:
%    \begin{macrocode}
\ifchilddoc\else
\section{part three}
\input{cdocspt3}
\section{part four}
\input{cdocspt4}
\fi
%    \end{macrocode}

% Process as usual until here:
%    \begin{macrocode}
\fi
%    \end{macrocode}

% End of document body:
%    \begin{macrocode}
\end{document}
%    \end{macrocode}
%\iffalse
%</samplemain>
%\fi
%
% %%%%%%%%%%%%%%%%%%%%%%%%%%%%%%%%%%%%%%
% \paragraph{Chapter Include Files.}
%
% The include files are called |cdocsch1.tex| and |cdocsch2.tex|.
%
%\iffalse
%<*samplechap1|samplechap2>
%\fi

% Optional override for |\version| flag:
%    \begin{macrocode}
%%\providecommand{\version}{final}
%    \end{macrocode}

% Include the main document:
%    \begin{macrocode}
\input{childdoc.def}
\childdocof{cdocsamp}
%    \end{macrocode}

%\iffalse
%</samplechap1|samplechap2>
%\fi
%
%\iffalse
%<*samplechap1>
%\fi
% Some text for chapter 1:
%    \begin{macrocode}
\section{one}
some text in chapter one
%    \end{macrocode}

%\iffalse
%</samplechap1>
%\fi
% Some text for chapter 2:
%\iffalse
%<*samplechap2>
%\fi
%    \begin{macrocode}
\section{two}
more text in chapter two
%    \end{macrocode}

%\iffalse
%</samplechap2>
%\fi
%
% %%%%%%%%%%%%%%%%%%%%%%%%%%%%%%%%%%%%%%
% \paragraph{Part Include Files.}
%
% The include files are called |cdocspt3.tex| and |cdocspt4.tex|.
%
%\iffalse
%<*samplepart3|samplepart4>
%\fi

% Optional override for |\version| flag:
%    \begin{macrocode}
%%\providecommand{\version}{final}
%    \end{macrocode}

% Include the main document:
%    \begin{macrocode}
\input{childdoc.def}
\childdocby{cdocsamp}
%    \end{macrocode}

%\iffalse
%</samplepart3|samplepart4>
%\fi
%
%\iffalse
%<*samplepart3>
%\fi
% Some text for part 3:
%    \begin{macrocode}
some text in part three
%    \end{macrocode}

%\iffalse
%</samplepart3>
%\fi
% Some text for part 4:
%\iffalse
%<*samplepart4>
%\fi
%    \begin{macrocode}
more text in part four
%    \end{macrocode}

%\iffalse
%</samplepart4>
%\fi
%
% %%%%%%%%%%%%%%%%%%%%%%%%%%%%%%%%%%%%%%
% \paragraph{Forwarding for a Complete Draft.}
%
% The following forwarding file |cdocsdrf.tex|
% compiles the main document in draft mode:
%\iffalse
%<*sampledraft>
%\fi
%    \begin{macrocode}
\def\version{draft}
\input{childdoc.def}
\childdocforward{cdocsamp}
%    \end{macrocode}

%\iffalse
%</sampledraft>
%\fi
%
% %%%%%%%%%%%%%%%%%%%%%%%%%%%%%%%%%%%%%%
% \paragraph{Forwarding for Final Version of the Chapters.}
%
% The following forwarding files |cdocsfn1.tex| and |cdocsfn2.tex|
% (with identical content)
% compile the final versions of the child documents
% |cdocsch1.tex| and |cdocsch2.tex|, respectively:
%\iffalse
%<*samplefinal>
%\fi
%    \begin{macrocode}
\def\version{final}
\input{childdoc.def}
\childdocforwardprefix[cdocsamp]{cdocsfn}{cdocsch}
%    \end{macrocode}

%\iffalse
%</samplefinal>
%\fi
%
% %%%%%%%%%%%%%%%%%%%%%%%%%%%%%%%%%%%%%%
% \paragraph{Command Line Processing.}
%
% The following three command lines generate the output files
% |cdocscld|, |cdocscl1| and |cdocscl2|
% which should be identical to
% |cdocsdrf|, |cdocsch1| and |cdocsfn2|, respectively:
% \begin{center}
% \begin{tabular}{l}
% |latex -jobname cdocscld \|\\
% |  "\def\version{draft}\input{childdoc.def}\childdocforward{cdocsamp}"|\\
% |latex -jobname cdocscl1 \|\\
% |  "\input{childdoc.def}\childdocforward[cdocsamp]{cdocsch1}"|\\
% |latex -jobname cdocscl2 \|\\
% |  "\def\version{final}\input{childdoc.def}\childdocforward{cdocsch2}"|
% \end{tabular}
% \end{center}
% Note that the trailing backslash on each first line
% merely continues the input to the second line
% (for convenient cut ant paste).
% Furthermore, the command |latex| can be replaced by any
% of its alternative versions such as |pdflatex|.
%
% %%%%%%%%%%%%%%%%%%%%%%%%%%%%%%%%%%%%%%%%%%%%%%%%%%%%%%%%%%%%%%%%%%%%%%%%%%%%%%
% %%%%%%%%%%%%%%%%%%%%%%%%%%%%%%%%%%%%%%%%%%%%%%%%%%%%%%%%%%%%%%%%%%%%%%%%%%%%%%
% \section{Implementation}
%\iffalse
%<*package>
%\fi
%
% This section describes the definitions file |childdoc.def|.

% The definitions cannot be loaded using |\usepackage| or |\RequirePackage|
% which has a mechanism to prevent loading a style file more than once.
% When loading the definitions by means of |\input|
% multiple instances have to be prevented manually:
%\iffalse
%This code needs to be before the `\ProvidesFile' directive
%which is defined at the beginning of this file.
%Therefore it is also placed there and commented out here.
%</package>
%<*discard>
%\fi
%    \begin{macrocode}
\ifdefined\childdocmain\endinput\fi
%    \end{macrocode}
%\iffalse
%</discard>
%<*package>
%\fi
%
% \macro{\ifchilddoc}
% \macro{\ifchilddocmanual}
% The conditional |\ifchilddoc| tells whether a
% child (true) or main (false) document is being compiled.
% The conditional |\ifchilddocmanual| tells whether
% the |\includeonly| mechanism is used (false) or
% the selection of child files must be performed manually (true).
% The definitions initialise to false:
%    \begin{macrocode}
\newif\ifchilddoc
\newif\ifchilddocmanual
%    \end{macrocode}

% \macro{\childdocname}
% \macro{\childdocjob}
% The macro |\childdocname| stores the name of the main document
% to be compiled. The macro |\childdocjob| stores the name of
% the document on which the \LaTeX{} compiler was originally invoked.
% The content of |\jobname| cannot be compared
% to filenames specified in the source due to different catcodes.
% The following code rescans |\jobname|, stores the result
% in |\childdocname| and saves a copy in |\childdocjob|:
%    \begin{macrocode}
\edef\childdocname{\scantokens\expandafter{\jobname\noexpand}}
\let\childdocjob\childdocname
%    \end{macrocode}

% \macro{\childdocdisable}
% The macro |\childdocdisable| prevents the main file
% from being processed more than once.
% At this stage, the main document command |\childdocmain|
% is assumed to be called once again where it should do nothing.
% Any subsequent call to it should prevent
% a secondary processing of the main document
% It overwrites the forwarding commands
% |\childdocof| and |\childdocforward|
% with empty macros to prevent further inclusions of the main document:
%    \begin{macrocode}
\newcommand{\childdocdisable}
{
  \renewcommand{\childdocmain}[1]{\renewcommand{\childdocmain}[1]{\endinput}}
  \renewcommand{\childdocof}[1]{}
  \renewcommand{\childdocby}[2][]{}
  \renewcommand{\childdocforward}[2][]{}
  \renewcommand{\childdocdisable}{}
}
%    \end{macrocode}

% \macro{\childdocmain}
% The macro |\childdocmain| is to be called at the top of the main file
% with nothing or the main filename (without extension) as argument.
% First, it breaks loops.
% If the argument is not empty and does not match |\childdocname|
% (which is set by the first inclusion of |childdoc.def|),
% |\ifchilddoc| is set to true, |\includeonly| is applied to the child file
% and |\jobname| is set to the main file
% (for proper handling of |.aux| files):
%    \begin{macrocode}
\newcommand{\childdocmain}[1]
{
  \childdocdisable\childdocmain{}
  \if?#1?\else
    \begingroup
      \def\childdoctmp{#1}
      \ifx\childdoctmp\childdocname
        \def\childdoctmp{}
      \else
        \def\childdoctmp
        {
          \childdoctrue
          \includeonly{\childdocname}
          \def\childdocjob{#1}
          \def\jobname{#1}
        }
      \fi
      \expandafter
    \endgroup
    \childdoctmp
  \fi
}
%    \end{macrocode}

% \macro{\childdocof}
% The command |\childdocof| redirects
% compilation to the main file |#1|.
%    \begin{macrocode}
\newcommand{\childdocof}[1]
{
  \childdocdisable
  \childdoctrue
  \includeonly{\childdocname}
  \def\jobname{#1}
  \def\childdocjob{#1}
  \input{#1}
}
%    \end{macrocode}

% \macro{\childdocby}
% The command |\childdocby| ....
%    \begin{macrocode}
\newcommand{\childdocby}[2][]
{
  \childdocdisable
  \childdoctrue
  \childdocmanualtrue
  \if?#1?\else
    \def\jobname{#2}
  \fi
  \def\childdocjob{#2}
  \input{#2}
  \endinput
}
%    \end{macrocode}

% \macro{\childdocforward}
% The command |\childdocforward| redirects
% compilation to the main file or
% (if the optional argument is given) a child file.
% Parameters are set as if the main file
% or a child file starting with |\childdocof| was compiled.
% Then compilation is handed over to the main file:
%    \begin{macrocode}
\newcommand{\childdocforward}[2][]
{
  \begingroup
    \if?#1?
      \def\childdoctmp
      {
        \def\childdocname{#2}
        \def\childdocjob{#2}
        \def\jobname{#2}
        \input{#2}
        \endinput
      }
    \else
      \def\childdoctmp
      {
        \childdocdisable
        \def\childdocname{#2}
        \childdoctrue
        \includeonly{#2}
        \def\childdocjob{#1}
        \def\jobname{#1}
        \input{#1}
        \endinput
      }
    \fi
    \expandafter
  \endgroup
  \childdoctmp
}
%    \end{macrocode}

% \macro{\childdocforwardprefix}
% The command |\childdocforwardprefix| redirects
% compilation to the main or a child file by means of a pattern.
% The prefix |#1| in the current filename is replaced by |#2|
% and the suffix of the current filename is kept
% (it is assumed that the filename does not contain the substring `|~~~|'
% which is used as a delimiter).
% Compilation is handed over to the new file by |\childdocforward|:
%    \begin{macrocode}
\newcommand{\childdocforwardprefix}[3][]
{
  \begingroup
    \def\childdocextract #2##1~~~{\def\childdoctmp{\childdocforward[#1]{#3##1}}}
    \expandafter\childdocextract\childdocname~~~
    \expandafter
  \endgroup
  \childdoctmp
}
%    \end{macrocode}

% \macro{\childdoc}
% The deprecated macro |\childdoc| is a legacy version of |\childdocmain|:
%    \begin{macrocode}
\newcommand{\childdoc}{\childdocmain}
%    \end{macrocode}

% \macro{\childdocredirect}
% The deprecated macro |\childdocredirect| is a legacy version
% of |\childdocforward| and |\childdocforwardprefix|:
%    \begin{macrocode}
\newcommand{\childdocredirect}[2][]
{
  \begingroup
    \if?#1?
      \def\childdoctmp{\childdocforward{#2}}
    \else
      \def\childdoctmp{\childdocforwardprefix{#1}{#2}}
    \fi
    \expandafter
  \endgroup
  \childdoctmp
}
%    \end{macrocode}

%\iffalse
%</package>
%\fi
%
\endinput
|\\
|\childdocmain{|\textit{main}|}|\\
\end{tabular}
\end{center}
%
If |\jobname| does not match the argument \textit{main} of |\childdocmain|,
it is assumed that |\jobname| points to the child file to be compiled.
When using |\childdocmain| with the main file specified as argument,
it suffices to start a child file
with just |\input{|\textit{main}|}|
without loading of the package and using |\childdocof|.
If instead all processing is done
with the appropriate \textsf{childdoc} directives,
the argument of \textit{main} of |\childdocmain| can be empty.

An alternative version of the command line processing described
in \secref{sec:commandline} using the detection mechanism reads:
%
\begin{center}
|... -jobname "|\textit{target}|" "|[\textit{flags}]%
[|\def\jobname{|\textit{dest}|}|]|\input{|\textit{main}|}"|
\end{center}

%%%%%%%%%%%%%%%%%%%%%%%%%%%%%%%%%%%%%%%%%%%%%%%%%%%%%%%%%%%%%%%%%%%%%%%%%%%%%%%%
\subsection{Manual Code}
\label{sec:manual}

In case one cannot be certain whether the definitions file |childdoc.def|
is installed on the target \TeX{} distribution
and one prefers not to ship it,
it is conceivable to paste a few relevant commands into the sources.

To that end, drop all statements |% \iffalse
%
% childdoc.dtx Copyright (C) 2017-2018 Niklas Beisert
%
% This work may be distributed and/or modified under the
% conditions of the LaTeX Project Public License, either version 1.3
% of this license or (at your option) any later version.
% The latest version of this license is in
%   http://www.latex-project.org/lppl.txt
% and version 1.3 or later is part of all distributions of LaTeX
% version 2005/12/01 or later.
%
% This work has the LPPL maintenance status `maintained'.
%
% The Current Maintainer of this work is Niklas Beisert.
%
% This work consists of the files childdoc.dtx and childdoc.ins
% and the derived files childdoc.def and cdocsamp.tex with
% cdocsch1.tex, cdocsch2.tex, cdocsdrf.tex, cdocsfn1.tex, cdocsfn2.tex.
%
%<package>\ifdefined\childdocmain\endinput\fi
%<package>\ProvidesFile{childdoc.def}[2018/12/30 v2.0 child document driver]
%<samplemain>\ProvidesFile{cdocsamp.tex}[2018/12/30 v2.0 sample for childdoc]
%<*driver>
%\ProvidesFile{childdoc.drv}[2018/12/30 v2.0 childdoc reference manual file]
\PassOptionsToClass{10pt,a4paper}{article}
\documentclass{ltxdoc}

\usepackage[margin=35mm]{geometry}
\usepackage{hyperref}
\usepackage{hyperxmp}
\usepackage[usenames]{color}

\hypersetup{colorlinks=true}
\hypersetup{pdfstartview=FitH}
\hypersetup{pdfpagemode=UseNone}
\hypersetup{pdfsource={}}
\hypersetup{pdflang={en-UK}}
\hypersetup{pdfcopyright={Copyright 2017-2018 Niklas Beisert.
  This work may be distributed and/or modified under the
  conditions of the LaTeX Project Public License, either version 1.3
  of this license or (at your option) any later version.}}
\hypersetup{pdflicenseurl={http://www.latex-project.org/lppl.txt}}
\hypersetup{pdfcontactaddress={ETH Zurich, ITP, HIT K,
  Wolfgang-Pauli-Strasse 27}}
\hypersetup{pdfcontactpostcode={8093}}
\hypersetup{pdfcontactcity={Zurich}}
\hypersetup{pdfcontactcountry={Switzerland}}
\hypersetup{pdfcontactemail={nbeisert@itp.phys.ethz.ch}}
\hypersetup{pdfcontacturl={http://people.phys.ethz.ch/\xmptilde nbeisert/}}

\newcommand{\secref}[1]{\hyperref[#1]{section \ref*{#1}}}

\parskip1ex
\parindent0pt
\let\olditemize\itemize
\def\itemize{\olditemize\parskip0pt}

\begin{document}

\title{The \textsf{childdoc} Package}
\hypersetup{pdftitle={The childdoc Package}}
\author{Niklas Beisert\\[2ex]
  Institut f\"ur Theoretische Physik\\
  Eidgen\"ossische Technische Hochschule Z\"urich\\
  Wolfgang-Pauli-Strasse 27, 8093 Z\"urich, Switzerland\\[1ex]
  \href{mailto:nbeisert@itp.phys.ethz.ch}
  {\texttt{nbeisert@itp.phys.ethz.ch}}}
\hypersetup{pdfauthor={Niklas Beisert}}
\hypersetup{pdfsubject={Manual for the LaTeX2e Package childdoc}}
\date{30 December 2018, \textsf{v2.0}}
\maketitle

\begin{abstract}\noindent
\textsf{childdoc} is a \LaTeXe{} package
that enables the direct compilation
of document sections included by |\include|
to individual files.
\end{abstract}

\begingroup
\parskip0ex
\tableofcontents
\endgroup

%%%%%%%%%%%%%%%%%%%%%%%%%%%%%%%%%%%%%%%%%%%%%%%%%%%%%%%%%%%%%%%%%%%%%%%%%%%%%%%%
%%%%%%%%%%%%%%%%%%%%%%%%%%%%%%%%%%%%%%%%%%%%%%%%%%%%%%%%%%%%%%%%%%%%%%%%%%%%%%%%
\section{Introduction}

\LaTeX{} provides a mechanism to structure a large document (such as a book)
into a main file and several child files (containing the chapters)
using the |\include| command.
This mechanism is beneficial for documents
which span hundreds of pages in order to
make the source file(s) more manageable.
Moreover, compilation can be restricted to
selected child files by means of the |\includeonly| command.
The latter feature can be used to reduce the compilation time while editing
(this was significantly more useful in the earlier days of \LaTeX{})
or to generate a smaller document which is easier to navigate.
Another application of |\includeonly| is to generate
documents consisting of selected parts of the complete document.

However, there are a few drawbacks of the plain |\include| mechanism:
\begin{itemize}
\item
The child files cannot be compiled on their own,
they can only be compiled via the main file.
A naive editing environment
(such as a text editor with an option
to have the current file processed by \LaTeX)
may require one to switch to the main file before compiling;
attempting to compile the child file produces errors.
\item
The main file must be modified (each time)
to adjust the |\includeonly| command
to the present needs. This easily leaves the main file in a messy state.
\item
The generated document will always carry the filename
of the main document. This is inconvenient if
several child files are to be compiled and
to be kept for distribution.
\end{itemize}

The present package provides a simple interface
to make child files individually compilable by \LaTeX{}.
Compiling a child file then has the same effect as compiling
the main file with an |\includeonly| command
to select the appropriate child.
Moreover the generated document will carry the name of the child
rather than the main file.
This resolves all three above issues.

This feature is meant to make the editing of books,
thesis documents and lecture notes somewhat more convenient.
However, the package can also be used efficiently for
composing a series of documents (such as exercise sheets)
which are typically distributed individually.
It then assists the author in generating the individual documents
(potentially in different versions)
as well as a document containing the collected series.
Another application is in developing style files
or other kinds of included material
where compilation of the style file could redirect
to a sample or test file.

%%%%%%%%%%%%%%%%%%%%%%%%%%%%%%%%%%%%%%%%%%%%%%%%%%%%%%%%%%%%%%%%%%%%%%%%%%%%%%%%
%%%%%%%%%%%%%%%%%%%%%%%%%%%%%%%%%%%%%%%%%%%%%%%%%%%%%%%%%%%%%%%%%%%%%%%%%%%%%%%%
\section{Usage}

First of all, the package \textsf{childdoc} is \emph{not} a standard
\LaTeXe{} |.sty| style file! Therefore it needs to be invoked in
a non-standard way.

%%%%%%%%%%%%%%%%%%%%%%%%%%%%%%%%%%%%%%%%%%%%%%%%%%%%%%%%%%%%%%%%%%%%%%%%%%%%%%%%
\subsection{Included Files}
\label{sec:include}

%%%%%%%%%%%%%%%%%%%%%%%%%%%%%%%%%%%%%%%%
\DescribeMacro{\childdocmain}
To use the package, add the commands
\begin{center}
\begin{tabular}{l}
|\input{childdoc.def}|\\
|\childdocmain{}|\\
\end{tabular}
\end{center}
at the very top of the main \LaTeX{} file,
in particular \emph{before} the |\documentclass| statement!
The argument of |\childdocmain| should be left empty
(but it must be present).

%%%%%%%%%%%%%%%%%%%%%%%%%%%%%%%%%%%%%%%%
\DescribeMacro{\childdocof}
Furthermore, add the commands
\begin{center}
\begin{tabular}{l}
|\input{childdoc.def}|\\
|\childdocof{|\textit{main}|}|\\
\end{tabular}
\end{center}
at the top of every child file \textit{child}
which is included by |\include{|\textit{child}|}|
from within the main file
(or at least for those files to be compiled individually).
The argument \textit{main} must be the filename of the main file.

There are a couple of
considerations in setting up the main and child documents:

%%%%%%%%%%%%%%%%%%%%%%%%%%%%%%%%%%%%%%%%
\paragraph{Restrictions.}

Please note the following restrictions:
\begin{itemize}
\item
|\childdocmain| must be called with one argument \textit{main}
to ensure compatibility with earlier version of the package.
It must either be empty (|\childdocmain{}|)
or precisely match the filename of the main file in which it is specified.
See \secref{sec:detection} for further information.
\item
The filename \textit{main} must be specified without the |.tex| extension.
\item
The filename \textit{main} is case sensitive
(even in case-insensitive file systems)
due to internal string comparison.
\item
The argument \textit{main} should be fully expanded, it cannot be a macro.
\item
Subdirectories and special characters should be avoided in filenames.
\item
The command |\childdocmain{|\textit{main}|}| must be followed by a whitespace.
It should not be followed immediately by another command
or by a comment mark `|%|'.
This is because the \TeX{} parser reads the token immediately following
the argument of |\childdocmain| and puts it
at the beginning of every child section;
however, a white\-space is ignored.
\end{itemize}

%%%%%%%%%%%%%%%%%%%%%%%%%%%%%%%%%%%%%%%%
\paragraph{Content of Main File.}

It is advisable to place all content in the child files included by |\include|.
Any output contained in the main file will appear in all child documents
unless suppressed manually;
it cannot be suppressed automatically by the |\includeonly| directive
and thus should normally be avoided.
A method to include some content in the main file
by means of conditional processing is described in \secref{sec:conditional}.

%%%%%%%%%%%%%%%%%%%%%%%%%%%%%%%%%%%%%%%%
\paragraph{Page Numbering.}

When only a part of the document is compiled,
the appropriate numbering of pages
(as well as other status parameters)
is determined from the |.aux| files.
The latter contain information from previous passes.
However this information needs to propagate through
all intermediate child documents.
Therefore the page numbering in child documents may well
be inconsistent until the complete document is compiled at least once.

A useful (if unconventional) way to always ensure a consistent
page numbering is to restart the numbering in each child document
and denote the pages by `\textit{child}|.|\textit{page}'
where \textit{child} represents the chapter/section number of the child file.
This can be achieved by the command
|\numberwithin{page}{|\textit{child}|}|
of the \textsf{amsmath} package
where \textit{child} can be |chapter| or |section|
depending on the chosen structuring.
Alternatively, one can modify the macro |\thepage| appropriately
and reset the counter |page| at the start of each child file.

%%%%%%%%%%%%%%%%%%%%%%%%%%%%%%%%%%%%%%%%%%%%%%%%%%%%%%%%%%%%%%%%%%%%%%%%%%%%%%%%
\subsection{Conditional Processing}
\label{sec:conditional}

The package provides a mechanism to compile different versions
of a document. To customise the versions further some conditional processing
can come in handy to distinguish which version is being compiled.
The package provides two macros to describe the compilation context:

%%%%%%%%%%%%%%%%%%%%%%%%%%%%%%%%%%%%%%%%
\DescribeMacro{\ifchilddoc}
The conditional |\ifchilddoc| distinguishes between the compilation of
child documents and the main document:
%
\begin{center}
|\ifchilddoc |\textit{child-code}| |[|\||else |\textit{main-code}]| \||fi|
\end{center}

%%%%%%%%%%%%%%%%%%%%%%%%%%%%%%%%%%%%%%%%
\DescribeMacro{\childdocname}
\DescribeMacro{\childdocjob}
The macro |\childdocname| contains the filename (without extension)
of the main or child file being processed.
Note that |\childdocjob| will always contain the name of the main file.

%%%%%%%%%%%%%%%%%%%%%%%%%%%%%%%%%%%%%%%%
\paragraph{Title Page.}

Conditional processing can be used to include a title or banner page
in the main document when proper precautions are taken.
Importantly, the code in the main file should ensure that the page counter
(as well as other status parameters which are stored in the |.aux| files)
takes the same value after the conditional processing.
Otherwise the page numbers may take divergent values
depending on which part is compiled.

For example, a title page could be declared by:
%
\begin{center}
\begin{tabular}{l}
|\ifchilddoc\||else|\\
|\addtocounter{page}{-1}|\\
\textit{code for title page}\\
|\newpage|\\
|\||fi|
\end{tabular}
\end{center}
%
A banner page for the child documents can be generated by:
%
\begin{center}
\begin{tabular}{l}
|\ifchilddoc|\\
|\addtocounter{page}{-1}|\\
\textit{code for banner page}\\
|\newpage|\\
|\||fi|
\end{tabular}
\end{center}
%
Here one could write a message such as:
\begin{center}
|This is the part \childdocname{} of \childdocjob{}.|
\end{center}

%%%%%%%%%%%%%%%%%%%%%%%%%%%%%%%%%%%%%%%%%%%%%%%%%%%%%%%%%%%%%%%%%%%%%%%%%%%%%%%%
\subsection{Flags}
\label{sec:flags}

The package makes it easy to generate different versions
of the main or child documents.
To this end compilation flags can be defined
and assigned different default values.
They will be particularly useful in conjunction
with the forwarding mechanism described in \secref{sec:forward}.

For example, it may be useful to have a flag |\version|
which can be set to |draft| or |final|.
The document source will contain some conditional code
depending on the value of |\version|.
Suppose further, the flag should default to |final| for the main file
and to |draft| for child files
which is a natural assignment for editing the document.
This is achieved by placing the following code
in the preamble of the main document
(below the |\childdocmain| directive):
%
\begin{center}
\begin{tabular}{l}
|\ifchilddoc|\\
|\providecommand{\version}{draft}|\\
|\||else|\\
|\providecommand{\version}{final}|\\
|\||fi|
\end{tabular}
\end{center}
%
The definition by |\providecommand| makes sure
that previous definitions are not overwritten.
Further statements |\providecommand{\version}{...}|
can thus be added before the above code to override it.

For the main file, one might add a line
(between |\childdocmain| and the above block)
%
\begin{center}
|%\ifchilddoc\||else\providecommand{\version}{draft}\||fi|
\end{center}
%
which can be uncommented to produce a draft version.
Likewise one can add a line to the very top of a child file
(above the |\childdocof{|\textit{main}|}| directive)
%
\begin{center}
|%\providecommand{\version}{final}|
\end{center}
%
which can be uncommented to produce the final version of this child document.

%%%%%%%%%%%%%%%%%%%%%%%%%%%%%%%%%%%%%%%%%%%%%%%%%%%%%%%%%%%%%%%%%%%%%%%%%%%%%%%%
\subsection{Forwarding}
\label{sec:forward}

Different versions of the main or child documents
using compilation flags as described in \secref{sec:flags}
can be (permanently) stored in different files
for convenient compilation, viewing and distribution.
To this end, the package defines a command
to pass on compilation to a different file:

%%%%%%%%%%%%%%%%%%%%%%%%%%%%%%%%%%%%%%%%
\DescribeMacro{\childdocforward}
The command |\childdocforward| redirects processing to
another source file:
%
\begin{center}
\begin{tabular}{l}
|\input{childdoc.def}|\\
|\childdocforward[|\textit{main}|]{|\textit{dest}|}|\\
\end{tabular}
\end{center}
%
The argument \textit{dest} is the destination file
(without extension).
It should be the main file or one of the child files.
Note that further \textsf{childdoc} directives
such as |\childdocof| and |\childdocforward|
in the indicated file will be processed in this form.
The optional argument \textit{main}
passes on directly to the main file \textit{main}
while pretending to compile the child \textit{dest}.
This form behaves as if \textit{dest}
issues |\childdocof{|\textit{main}|}| right away,
and no further \textsf{childdoc} directives will be processed.

%%%%%%%%%%%%%%%%%%%%%%%%%%%%%%%%%%%%%%%%
\DescribeMacro{\...prefix}
In the alternative form |\childdocforwardprefix|,
%
\begin{center}
\begin{tabular}{l}
|\input{childdoc.def}|\\
|\childdocforwardprefix[|\textit{main}|]{|\textit{prefix}|}{|\textit{dest}|}|
\end{tabular}
\end{center}
%
the destination file is determined by a pattern
depending on the current file:
To make this work, the current file must be called
`{\textit{prefix}\hspace{0.2em}\textit{suffix}}'
with \textit{prefix} matching precisely the argument.
Processing is then passed on to the file
`{\textit{dest}\hspace{0.2em}\textit{suffix}}'.
Surely, the same effect is achieved by
directly specifying the
argument `{\textit{dest}\hspace{0.2em}\textit{suffix}}'
in the first form.
However, that requires to set up a different file
for each child. With the alternative form of the command
all these files can have exactly the same content
which simplifies setting them up and maintaining them.

For example, the following file |draft.tex|
with a compilation flag |\version| as described in \secref{sec:flags}
compiles the main document as a draft:
%
\begin{center}
\begin{tabular}{l}
|\def\version{draft}|\\
|\input{childdoc.def}|\\
|\childdocforward{|\textit{main}|}|
\end{tabular}
\end{center}
%
Likewise, the following files |final|\textit{nn}|.tex|
compile the final version of the child document
|child|\textit{nn}|.tex|:
%
\begin{center}
\begin{tabular}{l}
|\def\version{final}|\\
|\input{childdoc.def}|\\
|\childdocforwardprefix{final}{child}|
\end{tabular}
\end{center}
%

Note that when several versions of a main file and/or of each child file
are to be generated, it may be convenient to set up a |Makefile| or
shell script to automatise the process.

%%%%%%%%%%%%%%%%%%%%%%%%%%%%%%%%%%%%%%%%%%%%%%%%%%%%%%%%%%%%%%%%%%%%%%%%%%%%%%%%
\subsection{Command Line Processing}
\label{sec:commandline}

The effect of redirection files can also be achieved by invoking
the \LaTeX{} compiler with a more elaborate command line.
Most conveniently this should be done as part
of a shell script or a |Makefile|.

When using \textsf{childdoc} in the main file, the following
command lines effectively perform a redirection
(note that depending on the shell being used,
backslashes may have to be doubled: `|\|' $\to$ `|\\|'):
%
\begin{center}
|... -jobname "|\textit{target}|" |\\|"|[\textit{flags}]%
|\input{childdoc.def}\childdocforward[|\textit{main}|]{|\textit{dest}|}"|
\end{center}
%
Here \textit{target} is the name of the output file,
\textit{main} is the name of the main file
and \textit{dest} is the name of the main or child file to be processed
(all filenames without extensions).
The optional argument \textit{main} can be omitted
if \textit{main} matches \textit{dest}.
Optionally, compilation \textit{flags} can be defined via |\def| commands.
This command line makes the \TeX{} engine believe
it is compiling the file \textit{target}
whose content is specified as the latter parameter.
The provided code then forwards the processing to
\textit{main} or \textit{dest} as described in \secref{sec:forward}.

%%%%%%%%%%%%%%%%%%%%%%%%%%%%%%%%%%%%%%%%%%%%%%%%%%%%%%%%%%%%%%%%%%%%%%%%%%%%%%%%
\subsection{Include by Input}
\label{sec:input}

Including child documents by |\include| has some restrictions by design.
Most notably, the content of a child document always occupies
its own set of pages; pages cannot be shared between child documents.
Usually, this behaviour makes perfect sense
because each child document contain an essential part of the document.
However, in some situations it may be desirable to compose
a document from a collection of parts
without having mandatory page breaks between then.
For this case, the package
provides a mechanism to include parts
by |\input| which can also be processed individually.
However, by construction this mechanism
requires manual handling of the content to be output.

%%%%%%%%%%%%%%%%%%%%%%%%%%%%%%%%%%%%%%%%
\DescribeMacro{\ifchilddocmanual}
The main file should be prepared as usual, see \secref{sec:include}.
However, the document body must make a distinction
between processing of an individual part and of the main document, e.g.:
%
\begin{center}
\begin{tabular}{l}
|\ifchilddocmanual|\\
|\input{\childdocname}|\\
|\||else|\\
\textit{document body with }|\input{|\textit{part}|}|\\
|\||fi|
\end{tabular}
\end{center}
%
The conditional |\ifchilddocmanual| is true whenever
a part to be included by |\input| is being compiled,
and the name of the part is stored in |\childdocname|.

%%%%%%%%%%%%%%%%%%%%%%%%%%%%%%%%%%%%%%%%
\DescribeMacro{\childdocby}
Each part to be included by |\input| should start with:
%
\begin{center}
\begin{tabular}{l}
|\input{childdoc.def}|\\
|\childdocby{|\textit{main}|}|\\
\end{tabular}
\end{center}
%
The directive |\childdocby| is similar to |\childdocof|
described in \secref{sec:include},
but the subsequent selection of content must be done manually.
To that end, both |\ifchilddoc| and |\ifchilddocmanual|
will be true upon processing of a part,
and the name of the part is stored in |\childdocname|.
Note that |\jobname| will be set to the filename of the current part
so that each part receives an individual |.aux| file
that does not interfere with the |.aux| file(s) of the main document.
This behaviour can be altered by the alternative form
|\childdocby[*]{|\textit{main}|}| (with a non-empty optional argument)
which uses the |.aux| file of the main document
by setting |\jobname| to \textit{main}.

%%%%%%%%%%%%%%%%%%%%%%%%%%%%%%%%%%%%%%%%%%%%%%%%%%%%%%%%%%%%%%%%%%%%%%%%%%%%%%%%
\subsection{Driver Development}
\label{sec:driver}

The \textsf{childdoc} mechanism can also be use for the development
of definition files such as \LaTeX{} styles or classes.
This case differs from the above setup with multiple parts
included by |\include| in that no |\includeonly| should be invoked.
This can be achieved by starting the include file
(before |\ProvidesPackage|) with:
%
\begin{center}
\begin{tabular}{l}
|\input{childdoc.def}|\\
|\childdocforward{|\textit{main}|}|\\
\end{tabular}
\end{center}
%
or alternatively with:
%
\begin{center}
\begin{tabular}{l}
|\input{childdoc.def}|\\
|\childdocby{|\textit{main}|}|\\
\end{tabular}
\end{center}
%
Both forms have slightly different effects as described above.
The main file is prepared as usual, see \secref{sec:include}.

%%%%%%%%%%%%%%%%%%%%%%%%%%%%%%%%%%%%%%%%%%%%%%%%%%%%%%%%%%%%%%%%%%%%%%%%%%%%%%%%
\subsection{Legacy Detection}
\label{sec:detection}

The directive |\childdocmain| in the main file can detect
whether the complete document or merely a child is to be compiled
even without using the directive |\childdocof|.
This method is deprecated because it is less robust
and there is no compelling reason to use it;
it is merely provided for backward compatibility
and it may be removed in future versions.

If the detection mechanism is to be used,
it is mandatory to correctly specify
the filename of the main file as the argument of |\childdocmain|:
%
\begin{center}
\begin{tabular}{l}
|\input{childdoc.def}|\\
|\childdocmain{|\textit{main}|}|\\
\end{tabular}
\end{center}
%
If |\jobname| does not match the argument \textit{main} of |\childdocmain|,
it is assumed that |\jobname| points to the child file to be compiled.
When using |\childdocmain| with the main file specified as argument,
it suffices to start a child file
with just |\input{|\textit{main}|}|
without loading of the package and using |\childdocof|.
If instead all processing is done
with the appropriate \textsf{childdoc} directives,
the argument of \textit{main} of |\childdocmain| can be empty.

An alternative version of the command line processing described
in \secref{sec:commandline} using the detection mechanism reads:
%
\begin{center}
|... -jobname "|\textit{target}|" "|[\textit{flags}]%
[|\def\jobname{|\textit{dest}|}|]|\input{|\textit{main}|}"|
\end{center}

%%%%%%%%%%%%%%%%%%%%%%%%%%%%%%%%%%%%%%%%%%%%%%%%%%%%%%%%%%%%%%%%%%%%%%%%%%%%%%%%
\subsection{Manual Code}
\label{sec:manual}

In case one cannot be certain whether the definitions file |childdoc.def|
is installed on the target \TeX{} distribution
and one prefers not to ship it,
it is conceivable to paste a few relevant commands into the sources.

To that end, drop all statements |\input{childdoc.def}|
and perform the replacements as outlined below.
Instead of |\childdocmain{|\textit{main}|}| add the following code
to the top of the main file:
%
\begin{center}
\begin{tabular}{l}
|\||ifdefined\childdocname\endinput\||fi\newif\ifchilddoc|\\
|\edef\childdocname{\scantokens\expandafter{\jobname\noexpand}}|\\
|\def\childdocmain{|\textit{main}|}\||ifx\childdocmain\childdocname\||else|\\
|\childdoctrue\includeonly{\childdocname}\let\jobname\childdocmain\||fi|\\
\end{tabular}
\end{center}
%
Instead of |\childdocof{|\textit{main}|}| just include the main file
at the top of each child file:
%
\begin{center}
|\input{|\textit{main}|}|
\end{center}
%
A simple redirection |\childdocforward{|\textit{dest}|}| is achieved by:
%
\begin{center}
|\def\jobname{|\textit{dest}|}\input{\jobname}|
\end{center}
%
The redirection with prefix
|\childdocforwardprefix[|\textit{prefix}|]{|\textit{dest}|}|
is accomplished by:
%
\begin{center}
\begin{tabular}{l}
|{\edef\jobname{\scantokens\expandafter{\jobname\noexpand}}|\\
|\def\redirectjob |\textit{prefix}|#1~~~{\gdef\jobname{|\textit{dest}|#1}}|\\
|\expandafter\redirectjob\jobname~~~}\input{\jobname}|
\end{tabular}
\end{center}

In an alternative approach,
child documents can be compiled by a specific command line
without additional code or specific definitions:
%
\begin{center}
|... -jobname "|\textit{target}|" "|[\textit{flags}]%
|\includeonly{|\textit{dest}|}\input{|\textit{main}|}"|
\end{center}
%

%%%%%%%%%%%%%%%%%%%%%%%%%%%%%%%%%%%%%%%%%%%%%%%%%%%%%%%%%%%%%%%%%%%%%%%%%%%%%%%%
%%%%%%%%%%%%%%%%%%%%%%%%%%%%%%%%%%%%%%%%%%%%%%%%%%%%%%%%%%%%%%%%%%%%%%%%%%%%%%%%
\section{Information}

%%%%%%%%%%%%%%%%%%%%%%%%%%%%%%%%%%%%%%%%%%%%%%%%%%%%%%%%%%%%%%%%%%%%%%%%%%%%%%%%
\subsection{Copyright}

Copyright \copyright{} 2017--2018 Niklas Beisert

This work may be distributed and/or modified under the
conditions of the \LaTeX{} Project Public License, either version 1.3
of this license or (at your option) any later version.
The latest version of this license is in
  \url{http://www.latex-project.org/lppl.txt}
and version 1.3 or later is part of all distributions of \LaTeX{}
version 2005/12/01 or later.

This work has the LPPL maintenance status `maintained'.

The Current Maintainer of this work is Niklas Beisert.

This work consists of the files |README.txt|, |childdoc.ins| and |childdoc.dtx|
as well as the derived files |childdoc.def|, |cdocsamp.tex|
with |cdocsch1.tex|, |cdocsch2.tex|, |cdocspt3.tex|, |cdocspt4.tex|,
|cdocsdrf.tex|, |cdocsfn1.tex|, |cdocsfn2.tex|
as well as |childdoc.pdf|.

%%%%%%%%%%%%%%%%%%%%%%%%%%%%%%%%%%%%%%%%%%%%%%%%%%%%%%%%%%%%%%%%%%%%%%%%%%%%%%%%
\subsection{Files and Installation}

The package consists of the files:
%
\begin{center}
\begin{tabular}{ll}
    |README.txt|   & readme file \\
    |childdoc.ins| & installation file \\
    |childdoc.dtx| & source file \\
    |childdoc.def| & definition file \\
    |cdocsamp.tex| & sample main file \\
    |cdocsch1.tex| & sample include file \\
    |cdocsch2.tex| & sample include file \\
    |cdocspt3.tex| & sample part file \\
    |cdocspt4.tex| & sample part file \\
    |cdocsdrf.tex| & sample redirection file \\
    |cdocsfn1.tex| & sample redirection file \\
    |cdocsfn2.tex| & sample redirection file \\
    |childdoc.pdf| & manual
\end{tabular}
\end{center}
%
The distribution consists of the files
|README.txt|, |childdoc.ins| and |childdoc.dtx|.
%
\begin{itemize}
\item
Run (pdf)\LaTeX{} on |childdoc.dtx|
to compile the manual |childdoc.pdf| (this file).
\item
Run \LaTeX{} on |childdoc.ins| to create the definitions file |childdoc.def|
and the sample |cdocsamp.tex| with include files
|cdocsch1.tex|, |cdocsch2.tex|, |cdocspt3.tex|, |cdocspt4.tex|,
|cdocsdrf.tex|, |cdocsfn1.tex|, |cdocsfn2.tex|.
Then copy the file |childdoc.def| to an appropriate directory of your \LaTeX{}
distribution, e.g.\ \textit{texmf-root}|/tex/latex/childdoc|.
\end{itemize}

%%%%%%%%%%%%%%%%%%%%%%%%%%%%%%%%%%%%%%%%%%%%%%%%%%%%%%%%%%%%%%%%%%%%%%%%%%%%%%%%
\subsection{Related CTAN Packages}

There are several other packages which offer a similar functionality:
%
\begin{itemize}
\item
The packages
\href{http://ctan.org/pkg/docmute}{\textsf{docmute}},
\href{http://ctan.org/pkg/includex}{\textsf{includex}} and
\href{http://ctan.org/pkg/standalone}{\textsf{standalone}}
provide commands to include only the document body of
a child file thus allowing both files to be compiled individually.
\item
The packages \href{http://ctan.org/pkg/subdocs}{\textsf{subdocs}}
and \href{http://ctan.org/pkg/subfiles}{\textsf{subfiles}}
provide structures in which the main and child documents can be
encapsulated and allowing them to be compiled individually.
The inclusion mechanism is different from the conventional |\include|.
\item
The package \href{http://ctan.org/pkg/combine}{\textsf{combine}}
is an elaborate solution to combine several documents into one.
\end{itemize}
%
See also the CTAN topic \href{http://ctan.org/topic/subdocs}{\textsf{subdocs}}
for further related packages.
The present package differs from the above solutions in that
a document structure constructed with the conventional |\include| mechanism
just needs two extra commands at the top of every file
such that all constituent files can be compiled individually.

%%%%%%%%%%%%%%%%%%%%%%%%%%%%%%%%%%%%%%%%%%%%%%%%%%%%%%%%%%%%%%%%%%%%%%%%%%%%%%%%
%\subsection{Feature Suggestions}
%
%The following is a list of features which may be useful for future
%versions of this package:
%%
%\begin{itemize}
%\item
%\ldots
%\end{itemize}

%%%%%%%%%%%%%%%%%%%%%%%%%%%%%%%%%%%%%%%%%%%%%%%%%%%%%%%%%%%%%%%%%%%%%%%%%%%%%%%%
\subsection{Revision History}

%%%%%%%%%%%%%%%%%%%%%%%%%%%%%%%%%%%%%%%%
\paragraph{v2.0:} 2018/12/30

\begin{itemize}
\item
immediate forward processing
\item
added |\childdocby| mechanism
\item
manual restructured
\end{itemize}

%%%%%%%%%%%%%%%%%%%%%%%%%%%%%%%%%%%%%%%%
\paragraph{v1.6:} 2018/01/17

\begin{itemize}
\item
application for development of include files
\item
corrections to manual
\end{itemize}

%%%%%%%%%%%%%%%%%%%%%%%%%%%%%%%%%%%%%%%%
\paragraph{v1.5:} 2017/05/21

\begin{itemize}
\item
more complete structuring introduced
\item
|\childdocof| introduced
\item
|\childdoc| renamed to |\childdocmain|
\item
|\childredirect| renamed to |\childdocforward| and |\childdocforwardprefix|
and functionality expanded
\end{itemize}

%%%%%%%%%%%%%%%%%%%%%%%%%%%%%%%%%%%%%%%%
\paragraph{v1.0:} 2017/04/27

\begin{itemize}
\item
manual and install package
\item
first version published on CTAN
\end{itemize}

%%%%%%%%%%%%%%%%%%%%%%%%%%%%%%%%%%%%%%%%
\paragraph{v0.6:} 2017/04/26

\begin{itemize}
\item
redirection mechanism added
\end{itemize}

%%%%%%%%%%%%%%%%%%%%%%%%%%%%%%%%%%%%%%%%
\paragraph{v0.5:} 2017/04/26

\begin{itemize}
\item
functionality in definition file
\end{itemize}


%%%%%%%%%%%%%%%%%%%%%%%%%%%%%%%%%%%%%%%%%%%%%%%%%%%%%%%%%%%%%%%%%%%%%%%%%%%%%%%%
%%%%%%%%%%%%%%%%%%%%%%%%%%%%%%%%%%%%%%%%%%%%%%%%%%%%%%%%%%%%%%%%%%%%%%%%%%%%%%%%
%%%%%%%%%%%%%%%%%%%%%%%%%%%%%%%%%%%%%%%%%%%%%%%%%%%%%%%%%%%%%%%%%%%%%%%%%%%%%%%%
\appendix

\settowidth\MacroIndent{\rmfamily\scriptsize 000\ }

 \DocInput{childdoc.dtx}

\end{document}
%</driver>
% \fi
%
% %%%%%%%%%%%%%%%%%%%%%%%%%%%%%%%%%%%%%%%%%%%%%%%%%%%%%%%%%%%%%%%%%%%%%%%%%%%%%%
% %%%%%%%%%%%%%%%%%%%%%%%%%%%%%%%%%%%%%%%%%%%%%%%%%%%%%%%%%%%%%%%%%%%%%%%%%%%%%%
% \section{Sample}
%\iffalse
%<*samplemain>
%\fi
%
% The following presents a sample document
% with two chapters, two parts, a title page,
% a compile flag as well as three forwarding files to set the flag.
% It consists of eight |.tex| files:
% \begin{center}
% \begin{tabular}{ll}
% |cdocsamp.tex|&main file\\
% |cdocsch1.tex|&include file for chapter 1\\
% |cdocsch2.tex|&include file for chapter 2\\
% |cdocspt3.tex|&include file for part 3\\
% |cdocspt4.tex|&include file for part 4\\
% |cdocsdrf.tex|&forwarding file for main file in draft mode\\
% |cdocsfi1.tex|&forwarding file for final version of chapter 1\\
% |cdocsfi2.tex|&forwarding file for final version of chapter 2\\
% \end{tabular}
% \end{center}
% Each of the eight files can be compiled directly by the \LaTeX{} compiler.
%
% %%%%%%%%%%%%%%%%%%%%%%%%%%%%%%%%%%%%%%
% \paragraph{Main File.}
%
% The main file is called |cdocsamp.tex|.
%
% Load the \textsf{childdoc} definitions and
% declare the filename for the main document:
%    \begin{macrocode}
\input{childdoc.def}
\childdocmain{}
%    \end{macrocode}

% Optional override for |\version| flag:
%    \begin{macrocode}
%%\ifchilddoc\else\providecommand{\version}{draft}\fi
%    \end{macrocode}

% Define the default values for the |\version| flag
% (|final| for the main file and |draft| for childs):
%    \begin{macrocode}
\ifchilddoc
\providecommand{\version}{draft}
\else
\providecommand{\version}{final}
\fi
%    \end{macrocode}

% Load the standard document class:
%    \begin{macrocode}
\documentclass[12pt]{article}
%    \end{macrocode}

% Start the document body:
%    \begin{macrocode}
\begin{document}
%    \end{macrocode}

% Declare a title page.
% Print title, part of document being processed and version flag:
%    \begin{macrocode}
\addtocounter{page}{-1}
\begin{center}
{\LARGE\bfseries{}childdoc example\par}
\vspace{1cm}
\ifchilddoc
\ifchilddocmanual part\else chapter\fi:
`\childdocname' of `\childdocjob'\par
\else
main document: `\childdocjob'\par
\fi
version: \version\par
\end{center}
\newpage
%    \end{macrocode}

% Manually include selected file,
% otherwise process as usual:
%    \begin{macrocode}
\ifchilddocmanual
\section*{part `\childdocname'}
\input{\childdocname}
\else
%    \end{macrocode}

% Include the two chapters:
%    \begin{macrocode}
\include{cdocsch1}
\include{cdocsch2}
%    \end{macrocode}

% Include the two parts unless only chapters should be displayed:
%    \begin{macrocode}
\ifchilddoc\else
\section{part three}
\input{cdocspt3}
\section{part four}
\input{cdocspt4}
\fi
%    \end{macrocode}

% Process as usual until here:
%    \begin{macrocode}
\fi
%    \end{macrocode}

% End of document body:
%    \begin{macrocode}
\end{document}
%    \end{macrocode}
%\iffalse
%</samplemain>
%\fi
%
% %%%%%%%%%%%%%%%%%%%%%%%%%%%%%%%%%%%%%%
% \paragraph{Chapter Include Files.}
%
% The include files are called |cdocsch1.tex| and |cdocsch2.tex|.
%
%\iffalse
%<*samplechap1|samplechap2>
%\fi

% Optional override for |\version| flag:
%    \begin{macrocode}
%%\providecommand{\version}{final}
%    \end{macrocode}

% Include the main document:
%    \begin{macrocode}
\input{childdoc.def}
\childdocof{cdocsamp}
%    \end{macrocode}

%\iffalse
%</samplechap1|samplechap2>
%\fi
%
%\iffalse
%<*samplechap1>
%\fi
% Some text for chapter 1:
%    \begin{macrocode}
\section{one}
some text in chapter one
%    \end{macrocode}

%\iffalse
%</samplechap1>
%\fi
% Some text for chapter 2:
%\iffalse
%<*samplechap2>
%\fi
%    \begin{macrocode}
\section{two}
more text in chapter two
%    \end{macrocode}

%\iffalse
%</samplechap2>
%\fi
%
% %%%%%%%%%%%%%%%%%%%%%%%%%%%%%%%%%%%%%%
% \paragraph{Part Include Files.}
%
% The include files are called |cdocspt3.tex| and |cdocspt4.tex|.
%
%\iffalse
%<*samplepart3|samplepart4>
%\fi

% Optional override for |\version| flag:
%    \begin{macrocode}
%%\providecommand{\version}{final}
%    \end{macrocode}

% Include the main document:
%    \begin{macrocode}
\input{childdoc.def}
\childdocby{cdocsamp}
%    \end{macrocode}

%\iffalse
%</samplepart3|samplepart4>
%\fi
%
%\iffalse
%<*samplepart3>
%\fi
% Some text for part 3:
%    \begin{macrocode}
some text in part three
%    \end{macrocode}

%\iffalse
%</samplepart3>
%\fi
% Some text for part 4:
%\iffalse
%<*samplepart4>
%\fi
%    \begin{macrocode}
more text in part four
%    \end{macrocode}

%\iffalse
%</samplepart4>
%\fi
%
% %%%%%%%%%%%%%%%%%%%%%%%%%%%%%%%%%%%%%%
% \paragraph{Forwarding for a Complete Draft.}
%
% The following forwarding file |cdocsdrf.tex|
% compiles the main document in draft mode:
%\iffalse
%<*sampledraft>
%\fi
%    \begin{macrocode}
\def\version{draft}
\input{childdoc.def}
\childdocforward{cdocsamp}
%    \end{macrocode}

%\iffalse
%</sampledraft>
%\fi
%
% %%%%%%%%%%%%%%%%%%%%%%%%%%%%%%%%%%%%%%
% \paragraph{Forwarding for Final Version of the Chapters.}
%
% The following forwarding files |cdocsfn1.tex| and |cdocsfn2.tex|
% (with identical content)
% compile the final versions of the child documents
% |cdocsch1.tex| and |cdocsch2.tex|, respectively:
%\iffalse
%<*samplefinal>
%\fi
%    \begin{macrocode}
\def\version{final}
\input{childdoc.def}
\childdocforwardprefix[cdocsamp]{cdocsfn}{cdocsch}
%    \end{macrocode}

%\iffalse
%</samplefinal>
%\fi
%
% %%%%%%%%%%%%%%%%%%%%%%%%%%%%%%%%%%%%%%
% \paragraph{Command Line Processing.}
%
% The following three command lines generate the output files
% |cdocscld|, |cdocscl1| and |cdocscl2|
% which should be identical to
% |cdocsdrf|, |cdocsch1| and |cdocsfn2|, respectively:
% \begin{center}
% \begin{tabular}{l}
% |latex -jobname cdocscld \|\\
% |  "\def\version{draft}\input{childdoc.def}\childdocforward{cdocsamp}"|\\
% |latex -jobname cdocscl1 \|\\
% |  "\input{childdoc.def}\childdocforward[cdocsamp]{cdocsch1}"|\\
% |latex -jobname cdocscl2 \|\\
% |  "\def\version{final}\input{childdoc.def}\childdocforward{cdocsch2}"|
% \end{tabular}
% \end{center}
% Note that the trailing backslash on each first line
% merely continues the input to the second line
% (for convenient cut ant paste).
% Furthermore, the command |latex| can be replaced by any
% of its alternative versions such as |pdflatex|.
%
% %%%%%%%%%%%%%%%%%%%%%%%%%%%%%%%%%%%%%%%%%%%%%%%%%%%%%%%%%%%%%%%%%%%%%%%%%%%%%%
% %%%%%%%%%%%%%%%%%%%%%%%%%%%%%%%%%%%%%%%%%%%%%%%%%%%%%%%%%%%%%%%%%%%%%%%%%%%%%%
% \section{Implementation}
%\iffalse
%<*package>
%\fi
%
% This section describes the definitions file |childdoc.def|.

% The definitions cannot be loaded using |\usepackage| or |\RequirePackage|
% which has a mechanism to prevent loading a style file more than once.
% When loading the definitions by means of |\input|
% multiple instances have to be prevented manually:
%\iffalse
%This code needs to be before the `\ProvidesFile' directive
%which is defined at the beginning of this file.
%Therefore it is also placed there and commented out here.
%</package>
%<*discard>
%\fi
%    \begin{macrocode}
\ifdefined\childdocmain\endinput\fi
%    \end{macrocode}
%\iffalse
%</discard>
%<*package>
%\fi
%
% \macro{\ifchilddoc}
% \macro{\ifchilddocmanual}
% The conditional |\ifchilddoc| tells whether a
% child (true) or main (false) document is being compiled.
% The conditional |\ifchilddocmanual| tells whether
% the |\includeonly| mechanism is used (false) or
% the selection of child files must be performed manually (true).
% The definitions initialise to false:
%    \begin{macrocode}
\newif\ifchilddoc
\newif\ifchilddocmanual
%    \end{macrocode}

% \macro{\childdocname}
% \macro{\childdocjob}
% The macro |\childdocname| stores the name of the main document
% to be compiled. The macro |\childdocjob| stores the name of
% the document on which the \LaTeX{} compiler was originally invoked.
% The content of |\jobname| cannot be compared
% to filenames specified in the source due to different catcodes.
% The following code rescans |\jobname|, stores the result
% in |\childdocname| and saves a copy in |\childdocjob|:
%    \begin{macrocode}
\edef\childdocname{\scantokens\expandafter{\jobname\noexpand}}
\let\childdocjob\childdocname
%    \end{macrocode}

% \macro{\childdocdisable}
% The macro |\childdocdisable| prevents the main file
% from being processed more than once.
% At this stage, the main document command |\childdocmain|
% is assumed to be called once again where it should do nothing.
% Any subsequent call to it should prevent
% a secondary processing of the main document
% It overwrites the forwarding commands
% |\childdocof| and |\childdocforward|
% with empty macros to prevent further inclusions of the main document:
%    \begin{macrocode}
\newcommand{\childdocdisable}
{
  \renewcommand{\childdocmain}[1]{\renewcommand{\childdocmain}[1]{\endinput}}
  \renewcommand{\childdocof}[1]{}
  \renewcommand{\childdocby}[2][]{}
  \renewcommand{\childdocforward}[2][]{}
  \renewcommand{\childdocdisable}{}
}
%    \end{macrocode}

% \macro{\childdocmain}
% The macro |\childdocmain| is to be called at the top of the main file
% with nothing or the main filename (without extension) as argument.
% First, it breaks loops.
% If the argument is not empty and does not match |\childdocname|
% (which is set by the first inclusion of |childdoc.def|),
% |\ifchilddoc| is set to true, |\includeonly| is applied to the child file
% and |\jobname| is set to the main file
% (for proper handling of |.aux| files):
%    \begin{macrocode}
\newcommand{\childdocmain}[1]
{
  \childdocdisable\childdocmain{}
  \if?#1?\else
    \begingroup
      \def\childdoctmp{#1}
      \ifx\childdoctmp\childdocname
        \def\childdoctmp{}
      \else
        \def\childdoctmp
        {
          \childdoctrue
          \includeonly{\childdocname}
          \def\childdocjob{#1}
          \def\jobname{#1}
        }
      \fi
      \expandafter
    \endgroup
    \childdoctmp
  \fi
}
%    \end{macrocode}

% \macro{\childdocof}
% The command |\childdocof| redirects
% compilation to the main file |#1|.
%    \begin{macrocode}
\newcommand{\childdocof}[1]
{
  \childdocdisable
  \childdoctrue
  \includeonly{\childdocname}
  \def\jobname{#1}
  \def\childdocjob{#1}
  \input{#1}
}
%    \end{macrocode}

% \macro{\childdocby}
% The command |\childdocby| ....
%    \begin{macrocode}
\newcommand{\childdocby}[2][]
{
  \childdocdisable
  \childdoctrue
  \childdocmanualtrue
  \if?#1?\else
    \def\jobname{#2}
  \fi
  \def\childdocjob{#2}
  \input{#2}
  \endinput
}
%    \end{macrocode}

% \macro{\childdocforward}
% The command |\childdocforward| redirects
% compilation to the main file or
% (if the optional argument is given) a child file.
% Parameters are set as if the main file
% or a child file starting with |\childdocof| was compiled.
% Then compilation is handed over to the main file:
%    \begin{macrocode}
\newcommand{\childdocforward}[2][]
{
  \begingroup
    \if?#1?
      \def\childdoctmp
      {
        \def\childdocname{#2}
        \def\childdocjob{#2}
        \def\jobname{#2}
        \input{#2}
        \endinput
      }
    \else
      \def\childdoctmp
      {
        \childdocdisable
        \def\childdocname{#2}
        \childdoctrue
        \includeonly{#2}
        \def\childdocjob{#1}
        \def\jobname{#1}
        \input{#1}
        \endinput
      }
    \fi
    \expandafter
  \endgroup
  \childdoctmp
}
%    \end{macrocode}

% \macro{\childdocforwardprefix}
% The command |\childdocforwardprefix| redirects
% compilation to the main or a child file by means of a pattern.
% The prefix |#1| in the current filename is replaced by |#2|
% and the suffix of the current filename is kept
% (it is assumed that the filename does not contain the substring `|~~~|'
% which is used as a delimiter).
% Compilation is handed over to the new file by |\childdocforward|:
%    \begin{macrocode}
\newcommand{\childdocforwardprefix}[3][]
{
  \begingroup
    \def\childdocextract #2##1~~~{\def\childdoctmp{\childdocforward[#1]{#3##1}}}
    \expandafter\childdocextract\childdocname~~~
    \expandafter
  \endgroup
  \childdoctmp
}
%    \end{macrocode}

% \macro{\childdoc}
% The deprecated macro |\childdoc| is a legacy version of |\childdocmain|:
%    \begin{macrocode}
\newcommand{\childdoc}{\childdocmain}
%    \end{macrocode}

% \macro{\childdocredirect}
% The deprecated macro |\childdocredirect| is a legacy version
% of |\childdocforward| and |\childdocforwardprefix|:
%    \begin{macrocode}
\newcommand{\childdocredirect}[2][]
{
  \begingroup
    \if?#1?
      \def\childdoctmp{\childdocforward{#2}}
    \else
      \def\childdoctmp{\childdocforwardprefix{#1}{#2}}
    \fi
    \expandafter
  \endgroup
  \childdoctmp
}
%    \end{macrocode}

%\iffalse
%</package>
%\fi
%
\endinput
|
and perform the replacements as outlined below.
Instead of |\childdocmain{|\textit{main}|}| add the following code
to the top of the main file:
%
\begin{center}
\begin{tabular}{l}
|\||ifdefined\childdocname\endinput\||fi\newif\ifchilddoc|\\
|\edef\childdocname{\scantokens\expandafter{\jobname\noexpand}}|\\
|\def\childdocmain{|\textit{main}|}\||ifx\childdocmain\childdocname\||else|\\
|\childdoctrue\includeonly{\childdocname}\let\jobname\childdocmain\||fi|\\
\end{tabular}
\end{center}
%
Instead of |\childdocof{|\textit{main}|}| just include the main file
at the top of each child file:
%
\begin{center}
|\input{|\textit{main}|}|
\end{center}
%
A simple redirection |\childdocforward{|\textit{dest}|}| is achieved by:
%
\begin{center}
|\def\jobname{|\textit{dest}|}\input{\jobname}|
\end{center}
%
The redirection with prefix
|\childdocforwardprefix[|\textit{prefix}|]{|\textit{dest}|}|
is accomplished by:
%
\begin{center}
\begin{tabular}{l}
|{\edef\jobname{\scantokens\expandafter{\jobname\noexpand}}|\\
|\def\redirectjob |\textit{prefix}|#1~~~{\gdef\jobname{|\textit{dest}|#1}}|\\
|\expandafter\redirectjob\jobname~~~}\input{\jobname}|
\end{tabular}
\end{center}

In an alternative approach,
child documents can be compiled by a specific command line
without additional code or specific definitions:
%
\begin{center}
|... -jobname "|\textit{target}|" "|[\textit{flags}]%
|\includeonly{|\textit{dest}|}\input{|\textit{main}|}"|
\end{center}
%

%%%%%%%%%%%%%%%%%%%%%%%%%%%%%%%%%%%%%%%%%%%%%%%%%%%%%%%%%%%%%%%%%%%%%%%%%%%%%%%%
%%%%%%%%%%%%%%%%%%%%%%%%%%%%%%%%%%%%%%%%%%%%%%%%%%%%%%%%%%%%%%%%%%%%%%%%%%%%%%%%
\section{Information}

%%%%%%%%%%%%%%%%%%%%%%%%%%%%%%%%%%%%%%%%%%%%%%%%%%%%%%%%%%%%%%%%%%%%%%%%%%%%%%%%
\subsection{Copyright}

Copyright \copyright{} 2017--2018 Niklas Beisert

This work may be distributed and/or modified under the
conditions of the \LaTeX{} Project Public License, either version 1.3
of this license or (at your option) any later version.
The latest version of this license is in
  \url{http://www.latex-project.org/lppl.txt}
and version 1.3 or later is part of all distributions of \LaTeX{}
version 2005/12/01 or later.

This work has the LPPL maintenance status `maintained'.

The Current Maintainer of this work is Niklas Beisert.

This work consists of the files |README.txt|, |childdoc.ins| and |childdoc.dtx|
as well as the derived files |childdoc.def|, |cdocsamp.tex|
with |cdocsch1.tex|, |cdocsch2.tex|, |cdocspt3.tex|, |cdocspt4.tex|,
|cdocsdrf.tex|, |cdocsfn1.tex|, |cdocsfn2.tex|
as well as |childdoc.pdf|.

%%%%%%%%%%%%%%%%%%%%%%%%%%%%%%%%%%%%%%%%%%%%%%%%%%%%%%%%%%%%%%%%%%%%%%%%%%%%%%%%
\subsection{Files and Installation}

The package consists of the files:
%
\begin{center}
\begin{tabular}{ll}
    |README.txt|   & readme file \\
    |childdoc.ins| & installation file \\
    |childdoc.dtx| & source file \\
    |childdoc.def| & definition file \\
    |cdocsamp.tex| & sample main file \\
    |cdocsch1.tex| & sample include file \\
    |cdocsch2.tex| & sample include file \\
    |cdocspt3.tex| & sample part file \\
    |cdocspt4.tex| & sample part file \\
    |cdocsdrf.tex| & sample redirection file \\
    |cdocsfn1.tex| & sample redirection file \\
    |cdocsfn2.tex| & sample redirection file \\
    |childdoc.pdf| & manual
\end{tabular}
\end{center}
%
The distribution consists of the files
|README.txt|, |childdoc.ins| and |childdoc.dtx|.
%
\begin{itemize}
\item
Run (pdf)\LaTeX{} on |childdoc.dtx|
to compile the manual |childdoc.pdf| (this file).
\item
Run \LaTeX{} on |childdoc.ins| to create the definitions file |childdoc.def|
and the sample |cdocsamp.tex| with include files
|cdocsch1.tex|, |cdocsch2.tex|, |cdocspt3.tex|, |cdocspt4.tex|,
|cdocsdrf.tex|, |cdocsfn1.tex|, |cdocsfn2.tex|.
Then copy the file |childdoc.def| to an appropriate directory of your \LaTeX{}
distribution, e.g.\ \textit{texmf-root}|/tex/latex/childdoc|.
\end{itemize}

%%%%%%%%%%%%%%%%%%%%%%%%%%%%%%%%%%%%%%%%%%%%%%%%%%%%%%%%%%%%%%%%%%%%%%%%%%%%%%%%
\subsection{Related CTAN Packages}

There are several other packages which offer a similar functionality:
%
\begin{itemize}
\item
The packages
\href{http://ctan.org/pkg/docmute}{\textsf{docmute}},
\href{http://ctan.org/pkg/includex}{\textsf{includex}} and
\href{http://ctan.org/pkg/standalone}{\textsf{standalone}}
provide commands to include only the document body of
a child file thus allowing both files to be compiled individually.
\item
The packages \href{http://ctan.org/pkg/subdocs}{\textsf{subdocs}}
and \href{http://ctan.org/pkg/subfiles}{\textsf{subfiles}}
provide structures in which the main and child documents can be
encapsulated and allowing them to be compiled individually.
The inclusion mechanism is different from the conventional |\include|.
\item
The package \href{http://ctan.org/pkg/combine}{\textsf{combine}}
is an elaborate solution to combine several documents into one.
\end{itemize}
%
See also the CTAN topic \href{http://ctan.org/topic/subdocs}{\textsf{subdocs}}
for further related packages.
The present package differs from the above solutions in that
a document structure constructed with the conventional |\include| mechanism
just needs two extra commands at the top of every file
such that all constituent files can be compiled individually.

%%%%%%%%%%%%%%%%%%%%%%%%%%%%%%%%%%%%%%%%%%%%%%%%%%%%%%%%%%%%%%%%%%%%%%%%%%%%%%%%
%\subsection{Feature Suggestions}
%
%The following is a list of features which may be useful for future
%versions of this package:
%%
%\begin{itemize}
%\item
%\ldots
%\end{itemize}

%%%%%%%%%%%%%%%%%%%%%%%%%%%%%%%%%%%%%%%%%%%%%%%%%%%%%%%%%%%%%%%%%%%%%%%%%%%%%%%%
\subsection{Revision History}

%%%%%%%%%%%%%%%%%%%%%%%%%%%%%%%%%%%%%%%%
\paragraph{v2.0:} 2018/12/30

\begin{itemize}
\item
immediate forward processing
\item
added |\childdocby| mechanism
\item
manual restructured
\end{itemize}

%%%%%%%%%%%%%%%%%%%%%%%%%%%%%%%%%%%%%%%%
\paragraph{v1.6:} 2018/01/17

\begin{itemize}
\item
application for development of include files
\item
corrections to manual
\end{itemize}

%%%%%%%%%%%%%%%%%%%%%%%%%%%%%%%%%%%%%%%%
\paragraph{v1.5:} 2017/05/21

\begin{itemize}
\item
more complete structuring introduced
\item
|\childdocof| introduced
\item
|\childdoc| renamed to |\childdocmain|
\item
|\childredirect| renamed to |\childdocforward| and |\childdocforwardprefix|
and functionality expanded
\end{itemize}

%%%%%%%%%%%%%%%%%%%%%%%%%%%%%%%%%%%%%%%%
\paragraph{v1.0:} 2017/04/27

\begin{itemize}
\item
manual and install package
\item
first version published on CTAN
\end{itemize}

%%%%%%%%%%%%%%%%%%%%%%%%%%%%%%%%%%%%%%%%
\paragraph{v0.6:} 2017/04/26

\begin{itemize}
\item
redirection mechanism added
\end{itemize}

%%%%%%%%%%%%%%%%%%%%%%%%%%%%%%%%%%%%%%%%
\paragraph{v0.5:} 2017/04/26

\begin{itemize}
\item
functionality in definition file
\end{itemize}


%%%%%%%%%%%%%%%%%%%%%%%%%%%%%%%%%%%%%%%%%%%%%%%%%%%%%%%%%%%%%%%%%%%%%%%%%%%%%%%%
%%%%%%%%%%%%%%%%%%%%%%%%%%%%%%%%%%%%%%%%%%%%%%%%%%%%%%%%%%%%%%%%%%%%%%%%%%%%%%%%
%%%%%%%%%%%%%%%%%%%%%%%%%%%%%%%%%%%%%%%%%%%%%%%%%%%%%%%%%%%%%%%%%%%%%%%%%%%%%%%%
\appendix

\settowidth\MacroIndent{\rmfamily\scriptsize 000\ }

 \DocInput{childdoc.dtx}

\end{document}
%</driver>
% \fi
%
% %%%%%%%%%%%%%%%%%%%%%%%%%%%%%%%%%%%%%%%%%%%%%%%%%%%%%%%%%%%%%%%%%%%%%%%%%%%%%%
% %%%%%%%%%%%%%%%%%%%%%%%%%%%%%%%%%%%%%%%%%%%%%%%%%%%%%%%%%%%%%%%%%%%%%%%%%%%%%%
% \section{Sample}
%\iffalse
%<*samplemain>
%\fi
%
% The following presents a sample document
% with two chapters, two parts, a title page,
% a compile flag as well as three forwarding files to set the flag.
% It consists of eight |.tex| files:
% \begin{center}
% \begin{tabular}{ll}
% |cdocsamp.tex|&main file\\
% |cdocsch1.tex|&include file for chapter 1\\
% |cdocsch2.tex|&include file for chapter 2\\
% |cdocspt3.tex|&include file for part 3\\
% |cdocspt4.tex|&include file for part 4\\
% |cdocsdrf.tex|&forwarding file for main file in draft mode\\
% |cdocsfi1.tex|&forwarding file for final version of chapter 1\\
% |cdocsfi2.tex|&forwarding file for final version of chapter 2\\
% \end{tabular}
% \end{center}
% Each of the eight files can be compiled directly by the \LaTeX{} compiler.
%
% %%%%%%%%%%%%%%%%%%%%%%%%%%%%%%%%%%%%%%
% \paragraph{Main File.}
%
% The main file is called |cdocsamp.tex|.
%
% Load the \textsf{childdoc} definitions and
% declare the filename for the main document:
%    \begin{macrocode}
% \iffalse
%
% childdoc.dtx Copyright (C) 2017-2018 Niklas Beisert
%
% This work may be distributed and/or modified under the
% conditions of the LaTeX Project Public License, either version 1.3
% of this license or (at your option) any later version.
% The latest version of this license is in
%   http://www.latex-project.org/lppl.txt
% and version 1.3 or later is part of all distributions of LaTeX
% version 2005/12/01 or later.
%
% This work has the LPPL maintenance status `maintained'.
%
% The Current Maintainer of this work is Niklas Beisert.
%
% This work consists of the files childdoc.dtx and childdoc.ins
% and the derived files childdoc.def and cdocsamp.tex with
% cdocsch1.tex, cdocsch2.tex, cdocsdrf.tex, cdocsfn1.tex, cdocsfn2.tex.
%
%<package>\ifdefined\childdocmain\endinput\fi
%<package>\ProvidesFile{childdoc.def}[2018/12/30 v2.0 child document driver]
%<samplemain>\ProvidesFile{cdocsamp.tex}[2018/12/30 v2.0 sample for childdoc]
%<*driver>
%\ProvidesFile{childdoc.drv}[2018/12/30 v2.0 childdoc reference manual file]
\PassOptionsToClass{10pt,a4paper}{article}
\documentclass{ltxdoc}

\usepackage[margin=35mm]{geometry}
\usepackage{hyperref}
\usepackage{hyperxmp}
\usepackage[usenames]{color}

\hypersetup{colorlinks=true}
\hypersetup{pdfstartview=FitH}
\hypersetup{pdfpagemode=UseNone}
\hypersetup{pdfsource={}}
\hypersetup{pdflang={en-UK}}
\hypersetup{pdfcopyright={Copyright 2017-2018 Niklas Beisert.
  This work may be distributed and/or modified under the
  conditions of the LaTeX Project Public License, either version 1.3
  of this license or (at your option) any later version.}}
\hypersetup{pdflicenseurl={http://www.latex-project.org/lppl.txt}}
\hypersetup{pdfcontactaddress={ETH Zurich, ITP, HIT K,
  Wolfgang-Pauli-Strasse 27}}
\hypersetup{pdfcontactpostcode={8093}}
\hypersetup{pdfcontactcity={Zurich}}
\hypersetup{pdfcontactcountry={Switzerland}}
\hypersetup{pdfcontactemail={nbeisert@itp.phys.ethz.ch}}
\hypersetup{pdfcontacturl={http://people.phys.ethz.ch/\xmptilde nbeisert/}}

\newcommand{\secref}[1]{\hyperref[#1]{section \ref*{#1}}}

\parskip1ex
\parindent0pt
\let\olditemize\itemize
\def\itemize{\olditemize\parskip0pt}

\begin{document}

\title{The \textsf{childdoc} Package}
\hypersetup{pdftitle={The childdoc Package}}
\author{Niklas Beisert\\[2ex]
  Institut f\"ur Theoretische Physik\\
  Eidgen\"ossische Technische Hochschule Z\"urich\\
  Wolfgang-Pauli-Strasse 27, 8093 Z\"urich, Switzerland\\[1ex]
  \href{mailto:nbeisert@itp.phys.ethz.ch}
  {\texttt{nbeisert@itp.phys.ethz.ch}}}
\hypersetup{pdfauthor={Niklas Beisert}}
\hypersetup{pdfsubject={Manual for the LaTeX2e Package childdoc}}
\date{30 December 2018, \textsf{v2.0}}
\maketitle

\begin{abstract}\noindent
\textsf{childdoc} is a \LaTeXe{} package
that enables the direct compilation
of document sections included by |\include|
to individual files.
\end{abstract}

\begingroup
\parskip0ex
\tableofcontents
\endgroup

%%%%%%%%%%%%%%%%%%%%%%%%%%%%%%%%%%%%%%%%%%%%%%%%%%%%%%%%%%%%%%%%%%%%%%%%%%%%%%%%
%%%%%%%%%%%%%%%%%%%%%%%%%%%%%%%%%%%%%%%%%%%%%%%%%%%%%%%%%%%%%%%%%%%%%%%%%%%%%%%%
\section{Introduction}

\LaTeX{} provides a mechanism to structure a large document (such as a book)
into a main file and several child files (containing the chapters)
using the |\include| command.
This mechanism is beneficial for documents
which span hundreds of pages in order to
make the source file(s) more manageable.
Moreover, compilation can be restricted to
selected child files by means of the |\includeonly| command.
The latter feature can be used to reduce the compilation time while editing
(this was significantly more useful in the earlier days of \LaTeX{})
or to generate a smaller document which is easier to navigate.
Another application of |\includeonly| is to generate
documents consisting of selected parts of the complete document.

However, there are a few drawbacks of the plain |\include| mechanism:
\begin{itemize}
\item
The child files cannot be compiled on their own,
they can only be compiled via the main file.
A naive editing environment
(such as a text editor with an option
to have the current file processed by \LaTeX)
may require one to switch to the main file before compiling;
attempting to compile the child file produces errors.
\item
The main file must be modified (each time)
to adjust the |\includeonly| command
to the present needs. This easily leaves the main file in a messy state.
\item
The generated document will always carry the filename
of the main document. This is inconvenient if
several child files are to be compiled and
to be kept for distribution.
\end{itemize}

The present package provides a simple interface
to make child files individually compilable by \LaTeX{}.
Compiling a child file then has the same effect as compiling
the main file with an |\includeonly| command
to select the appropriate child.
Moreover the generated document will carry the name of the child
rather than the main file.
This resolves all three above issues.

This feature is meant to make the editing of books,
thesis documents and lecture notes somewhat more convenient.
However, the package can also be used efficiently for
composing a series of documents (such as exercise sheets)
which are typically distributed individually.
It then assists the author in generating the individual documents
(potentially in different versions)
as well as a document containing the collected series.
Another application is in developing style files
or other kinds of included material
where compilation of the style file could redirect
to a sample or test file.

%%%%%%%%%%%%%%%%%%%%%%%%%%%%%%%%%%%%%%%%%%%%%%%%%%%%%%%%%%%%%%%%%%%%%%%%%%%%%%%%
%%%%%%%%%%%%%%%%%%%%%%%%%%%%%%%%%%%%%%%%%%%%%%%%%%%%%%%%%%%%%%%%%%%%%%%%%%%%%%%%
\section{Usage}

First of all, the package \textsf{childdoc} is \emph{not} a standard
\LaTeXe{} |.sty| style file! Therefore it needs to be invoked in
a non-standard way.

%%%%%%%%%%%%%%%%%%%%%%%%%%%%%%%%%%%%%%%%%%%%%%%%%%%%%%%%%%%%%%%%%%%%%%%%%%%%%%%%
\subsection{Included Files}
\label{sec:include}

%%%%%%%%%%%%%%%%%%%%%%%%%%%%%%%%%%%%%%%%
\DescribeMacro{\childdocmain}
To use the package, add the commands
\begin{center}
\begin{tabular}{l}
|\input{childdoc.def}|\\
|\childdocmain{}|\\
\end{tabular}
\end{center}
at the very top of the main \LaTeX{} file,
in particular \emph{before} the |\documentclass| statement!
The argument of |\childdocmain| should be left empty
(but it must be present).

%%%%%%%%%%%%%%%%%%%%%%%%%%%%%%%%%%%%%%%%
\DescribeMacro{\childdocof}
Furthermore, add the commands
\begin{center}
\begin{tabular}{l}
|\input{childdoc.def}|\\
|\childdocof{|\textit{main}|}|\\
\end{tabular}
\end{center}
at the top of every child file \textit{child}
which is included by |\include{|\textit{child}|}|
from within the main file
(or at least for those files to be compiled individually).
The argument \textit{main} must be the filename of the main file.

There are a couple of
considerations in setting up the main and child documents:

%%%%%%%%%%%%%%%%%%%%%%%%%%%%%%%%%%%%%%%%
\paragraph{Restrictions.}

Please note the following restrictions:
\begin{itemize}
\item
|\childdocmain| must be called with one argument \textit{main}
to ensure compatibility with earlier version of the package.
It must either be empty (|\childdocmain{}|)
or precisely match the filename of the main file in which it is specified.
See \secref{sec:detection} for further information.
\item
The filename \textit{main} must be specified without the |.tex| extension.
\item
The filename \textit{main} is case sensitive
(even in case-insensitive file systems)
due to internal string comparison.
\item
The argument \textit{main} should be fully expanded, it cannot be a macro.
\item
Subdirectories and special characters should be avoided in filenames.
\item
The command |\childdocmain{|\textit{main}|}| must be followed by a whitespace.
It should not be followed immediately by another command
or by a comment mark `|%|'.
This is because the \TeX{} parser reads the token immediately following
the argument of |\childdocmain| and puts it
at the beginning of every child section;
however, a white\-space is ignored.
\end{itemize}

%%%%%%%%%%%%%%%%%%%%%%%%%%%%%%%%%%%%%%%%
\paragraph{Content of Main File.}

It is advisable to place all content in the child files included by |\include|.
Any output contained in the main file will appear in all child documents
unless suppressed manually;
it cannot be suppressed automatically by the |\includeonly| directive
and thus should normally be avoided.
A method to include some content in the main file
by means of conditional processing is described in \secref{sec:conditional}.

%%%%%%%%%%%%%%%%%%%%%%%%%%%%%%%%%%%%%%%%
\paragraph{Page Numbering.}

When only a part of the document is compiled,
the appropriate numbering of pages
(as well as other status parameters)
is determined from the |.aux| files.
The latter contain information from previous passes.
However this information needs to propagate through
all intermediate child documents.
Therefore the page numbering in child documents may well
be inconsistent until the complete document is compiled at least once.

A useful (if unconventional) way to always ensure a consistent
page numbering is to restart the numbering in each child document
and denote the pages by `\textit{child}|.|\textit{page}'
where \textit{child} represents the chapter/section number of the child file.
This can be achieved by the command
|\numberwithin{page}{|\textit{child}|}|
of the \textsf{amsmath} package
where \textit{child} can be |chapter| or |section|
depending on the chosen structuring.
Alternatively, one can modify the macro |\thepage| appropriately
and reset the counter |page| at the start of each child file.

%%%%%%%%%%%%%%%%%%%%%%%%%%%%%%%%%%%%%%%%%%%%%%%%%%%%%%%%%%%%%%%%%%%%%%%%%%%%%%%%
\subsection{Conditional Processing}
\label{sec:conditional}

The package provides a mechanism to compile different versions
of a document. To customise the versions further some conditional processing
can come in handy to distinguish which version is being compiled.
The package provides two macros to describe the compilation context:

%%%%%%%%%%%%%%%%%%%%%%%%%%%%%%%%%%%%%%%%
\DescribeMacro{\ifchilddoc}
The conditional |\ifchilddoc| distinguishes between the compilation of
child documents and the main document:
%
\begin{center}
|\ifchilddoc |\textit{child-code}| |[|\||else |\textit{main-code}]| \||fi|
\end{center}

%%%%%%%%%%%%%%%%%%%%%%%%%%%%%%%%%%%%%%%%
\DescribeMacro{\childdocname}
\DescribeMacro{\childdocjob}
The macro |\childdocname| contains the filename (without extension)
of the main or child file being processed.
Note that |\childdocjob| will always contain the name of the main file.

%%%%%%%%%%%%%%%%%%%%%%%%%%%%%%%%%%%%%%%%
\paragraph{Title Page.}

Conditional processing can be used to include a title or banner page
in the main document when proper precautions are taken.
Importantly, the code in the main file should ensure that the page counter
(as well as other status parameters which are stored in the |.aux| files)
takes the same value after the conditional processing.
Otherwise the page numbers may take divergent values
depending on which part is compiled.

For example, a title page could be declared by:
%
\begin{center}
\begin{tabular}{l}
|\ifchilddoc\||else|\\
|\addtocounter{page}{-1}|\\
\textit{code for title page}\\
|\newpage|\\
|\||fi|
\end{tabular}
\end{center}
%
A banner page for the child documents can be generated by:
%
\begin{center}
\begin{tabular}{l}
|\ifchilddoc|\\
|\addtocounter{page}{-1}|\\
\textit{code for banner page}\\
|\newpage|\\
|\||fi|
\end{tabular}
\end{center}
%
Here one could write a message such as:
\begin{center}
|This is the part \childdocname{} of \childdocjob{}.|
\end{center}

%%%%%%%%%%%%%%%%%%%%%%%%%%%%%%%%%%%%%%%%%%%%%%%%%%%%%%%%%%%%%%%%%%%%%%%%%%%%%%%%
\subsection{Flags}
\label{sec:flags}

The package makes it easy to generate different versions
of the main or child documents.
To this end compilation flags can be defined
and assigned different default values.
They will be particularly useful in conjunction
with the forwarding mechanism described in \secref{sec:forward}.

For example, it may be useful to have a flag |\version|
which can be set to |draft| or |final|.
The document source will contain some conditional code
depending on the value of |\version|.
Suppose further, the flag should default to |final| for the main file
and to |draft| for child files
which is a natural assignment for editing the document.
This is achieved by placing the following code
in the preamble of the main document
(below the |\childdocmain| directive):
%
\begin{center}
\begin{tabular}{l}
|\ifchilddoc|\\
|\providecommand{\version}{draft}|\\
|\||else|\\
|\providecommand{\version}{final}|\\
|\||fi|
\end{tabular}
\end{center}
%
The definition by |\providecommand| makes sure
that previous definitions are not overwritten.
Further statements |\providecommand{\version}{...}|
can thus be added before the above code to override it.

For the main file, one might add a line
(between |\childdocmain| and the above block)
%
\begin{center}
|%\ifchilddoc\||else\providecommand{\version}{draft}\||fi|
\end{center}
%
which can be uncommented to produce a draft version.
Likewise one can add a line to the very top of a child file
(above the |\childdocof{|\textit{main}|}| directive)
%
\begin{center}
|%\providecommand{\version}{final}|
\end{center}
%
which can be uncommented to produce the final version of this child document.

%%%%%%%%%%%%%%%%%%%%%%%%%%%%%%%%%%%%%%%%%%%%%%%%%%%%%%%%%%%%%%%%%%%%%%%%%%%%%%%%
\subsection{Forwarding}
\label{sec:forward}

Different versions of the main or child documents
using compilation flags as described in \secref{sec:flags}
can be (permanently) stored in different files
for convenient compilation, viewing and distribution.
To this end, the package defines a command
to pass on compilation to a different file:

%%%%%%%%%%%%%%%%%%%%%%%%%%%%%%%%%%%%%%%%
\DescribeMacro{\childdocforward}
The command |\childdocforward| redirects processing to
another source file:
%
\begin{center}
\begin{tabular}{l}
|\input{childdoc.def}|\\
|\childdocforward[|\textit{main}|]{|\textit{dest}|}|\\
\end{tabular}
\end{center}
%
The argument \textit{dest} is the destination file
(without extension).
It should be the main file or one of the child files.
Note that further \textsf{childdoc} directives
such as |\childdocof| and |\childdocforward|
in the indicated file will be processed in this form.
The optional argument \textit{main}
passes on directly to the main file \textit{main}
while pretending to compile the child \textit{dest}.
This form behaves as if \textit{dest}
issues |\childdocof{|\textit{main}|}| right away,
and no further \textsf{childdoc} directives will be processed.

%%%%%%%%%%%%%%%%%%%%%%%%%%%%%%%%%%%%%%%%
\DescribeMacro{\...prefix}
In the alternative form |\childdocforwardprefix|,
%
\begin{center}
\begin{tabular}{l}
|\input{childdoc.def}|\\
|\childdocforwardprefix[|\textit{main}|]{|\textit{prefix}|}{|\textit{dest}|}|
\end{tabular}
\end{center}
%
the destination file is determined by a pattern
depending on the current file:
To make this work, the current file must be called
`{\textit{prefix}\hspace{0.2em}\textit{suffix}}'
with \textit{prefix} matching precisely the argument.
Processing is then passed on to the file
`{\textit{dest}\hspace{0.2em}\textit{suffix}}'.
Surely, the same effect is achieved by
directly specifying the
argument `{\textit{dest}\hspace{0.2em}\textit{suffix}}'
in the first form.
However, that requires to set up a different file
for each child. With the alternative form of the command
all these files can have exactly the same content
which simplifies setting them up and maintaining them.

For example, the following file |draft.tex|
with a compilation flag |\version| as described in \secref{sec:flags}
compiles the main document as a draft:
%
\begin{center}
\begin{tabular}{l}
|\def\version{draft}|\\
|\input{childdoc.def}|\\
|\childdocforward{|\textit{main}|}|
\end{tabular}
\end{center}
%
Likewise, the following files |final|\textit{nn}|.tex|
compile the final version of the child document
|child|\textit{nn}|.tex|:
%
\begin{center}
\begin{tabular}{l}
|\def\version{final}|\\
|\input{childdoc.def}|\\
|\childdocforwardprefix{final}{child}|
\end{tabular}
\end{center}
%

Note that when several versions of a main file and/or of each child file
are to be generated, it may be convenient to set up a |Makefile| or
shell script to automatise the process.

%%%%%%%%%%%%%%%%%%%%%%%%%%%%%%%%%%%%%%%%%%%%%%%%%%%%%%%%%%%%%%%%%%%%%%%%%%%%%%%%
\subsection{Command Line Processing}
\label{sec:commandline}

The effect of redirection files can also be achieved by invoking
the \LaTeX{} compiler with a more elaborate command line.
Most conveniently this should be done as part
of a shell script or a |Makefile|.

When using \textsf{childdoc} in the main file, the following
command lines effectively perform a redirection
(note that depending on the shell being used,
backslashes may have to be doubled: `|\|' $\to$ `|\\|'):
%
\begin{center}
|... -jobname "|\textit{target}|" |\\|"|[\textit{flags}]%
|\input{childdoc.def}\childdocforward[|\textit{main}|]{|\textit{dest}|}"|
\end{center}
%
Here \textit{target} is the name of the output file,
\textit{main} is the name of the main file
and \textit{dest} is the name of the main or child file to be processed
(all filenames without extensions).
The optional argument \textit{main} can be omitted
if \textit{main} matches \textit{dest}.
Optionally, compilation \textit{flags} can be defined via |\def| commands.
This command line makes the \TeX{} engine believe
it is compiling the file \textit{target}
whose content is specified as the latter parameter.
The provided code then forwards the processing to
\textit{main} or \textit{dest} as described in \secref{sec:forward}.

%%%%%%%%%%%%%%%%%%%%%%%%%%%%%%%%%%%%%%%%%%%%%%%%%%%%%%%%%%%%%%%%%%%%%%%%%%%%%%%%
\subsection{Include by Input}
\label{sec:input}

Including child documents by |\include| has some restrictions by design.
Most notably, the content of a child document always occupies
its own set of pages; pages cannot be shared between child documents.
Usually, this behaviour makes perfect sense
because each child document contain an essential part of the document.
However, in some situations it may be desirable to compose
a document from a collection of parts
without having mandatory page breaks between then.
For this case, the package
provides a mechanism to include parts
by |\input| which can also be processed individually.
However, by construction this mechanism
requires manual handling of the content to be output.

%%%%%%%%%%%%%%%%%%%%%%%%%%%%%%%%%%%%%%%%
\DescribeMacro{\ifchilddocmanual}
The main file should be prepared as usual, see \secref{sec:include}.
However, the document body must make a distinction
between processing of an individual part and of the main document, e.g.:
%
\begin{center}
\begin{tabular}{l}
|\ifchilddocmanual|\\
|\input{\childdocname}|\\
|\||else|\\
\textit{document body with }|\input{|\textit{part}|}|\\
|\||fi|
\end{tabular}
\end{center}
%
The conditional |\ifchilddocmanual| is true whenever
a part to be included by |\input| is being compiled,
and the name of the part is stored in |\childdocname|.

%%%%%%%%%%%%%%%%%%%%%%%%%%%%%%%%%%%%%%%%
\DescribeMacro{\childdocby}
Each part to be included by |\input| should start with:
%
\begin{center}
\begin{tabular}{l}
|\input{childdoc.def}|\\
|\childdocby{|\textit{main}|}|\\
\end{tabular}
\end{center}
%
The directive |\childdocby| is similar to |\childdocof|
described in \secref{sec:include},
but the subsequent selection of content must be done manually.
To that end, both |\ifchilddoc| and |\ifchilddocmanual|
will be true upon processing of a part,
and the name of the part is stored in |\childdocname|.
Note that |\jobname| will be set to the filename of the current part
so that each part receives an individual |.aux| file
that does not interfere with the |.aux| file(s) of the main document.
This behaviour can be altered by the alternative form
|\childdocby[*]{|\textit{main}|}| (with a non-empty optional argument)
which uses the |.aux| file of the main document
by setting |\jobname| to \textit{main}.

%%%%%%%%%%%%%%%%%%%%%%%%%%%%%%%%%%%%%%%%%%%%%%%%%%%%%%%%%%%%%%%%%%%%%%%%%%%%%%%%
\subsection{Driver Development}
\label{sec:driver}

The \textsf{childdoc} mechanism can also be use for the development
of definition files such as \LaTeX{} styles or classes.
This case differs from the above setup with multiple parts
included by |\include| in that no |\includeonly| should be invoked.
This can be achieved by starting the include file
(before |\ProvidesPackage|) with:
%
\begin{center}
\begin{tabular}{l}
|\input{childdoc.def}|\\
|\childdocforward{|\textit{main}|}|\\
\end{tabular}
\end{center}
%
or alternatively with:
%
\begin{center}
\begin{tabular}{l}
|\input{childdoc.def}|\\
|\childdocby{|\textit{main}|}|\\
\end{tabular}
\end{center}
%
Both forms have slightly different effects as described above.
The main file is prepared as usual, see \secref{sec:include}.

%%%%%%%%%%%%%%%%%%%%%%%%%%%%%%%%%%%%%%%%%%%%%%%%%%%%%%%%%%%%%%%%%%%%%%%%%%%%%%%%
\subsection{Legacy Detection}
\label{sec:detection}

The directive |\childdocmain| in the main file can detect
whether the complete document or merely a child is to be compiled
even without using the directive |\childdocof|.
This method is deprecated because it is less robust
and there is no compelling reason to use it;
it is merely provided for backward compatibility
and it may be removed in future versions.

If the detection mechanism is to be used,
it is mandatory to correctly specify
the filename of the main file as the argument of |\childdocmain|:
%
\begin{center}
\begin{tabular}{l}
|\input{childdoc.def}|\\
|\childdocmain{|\textit{main}|}|\\
\end{tabular}
\end{center}
%
If |\jobname| does not match the argument \textit{main} of |\childdocmain|,
it is assumed that |\jobname| points to the child file to be compiled.
When using |\childdocmain| with the main file specified as argument,
it suffices to start a child file
with just |\input{|\textit{main}|}|
without loading of the package and using |\childdocof|.
If instead all processing is done
with the appropriate \textsf{childdoc} directives,
the argument of \textit{main} of |\childdocmain| can be empty.

An alternative version of the command line processing described
in \secref{sec:commandline} using the detection mechanism reads:
%
\begin{center}
|... -jobname "|\textit{target}|" "|[\textit{flags}]%
[|\def\jobname{|\textit{dest}|}|]|\input{|\textit{main}|}"|
\end{center}

%%%%%%%%%%%%%%%%%%%%%%%%%%%%%%%%%%%%%%%%%%%%%%%%%%%%%%%%%%%%%%%%%%%%%%%%%%%%%%%%
\subsection{Manual Code}
\label{sec:manual}

In case one cannot be certain whether the definitions file |childdoc.def|
is installed on the target \TeX{} distribution
and one prefers not to ship it,
it is conceivable to paste a few relevant commands into the sources.

To that end, drop all statements |\input{childdoc.def}|
and perform the replacements as outlined below.
Instead of |\childdocmain{|\textit{main}|}| add the following code
to the top of the main file:
%
\begin{center}
\begin{tabular}{l}
|\||ifdefined\childdocname\endinput\||fi\newif\ifchilddoc|\\
|\edef\childdocname{\scantokens\expandafter{\jobname\noexpand}}|\\
|\def\childdocmain{|\textit{main}|}\||ifx\childdocmain\childdocname\||else|\\
|\childdoctrue\includeonly{\childdocname}\let\jobname\childdocmain\||fi|\\
\end{tabular}
\end{center}
%
Instead of |\childdocof{|\textit{main}|}| just include the main file
at the top of each child file:
%
\begin{center}
|\input{|\textit{main}|}|
\end{center}
%
A simple redirection |\childdocforward{|\textit{dest}|}| is achieved by:
%
\begin{center}
|\def\jobname{|\textit{dest}|}\input{\jobname}|
\end{center}
%
The redirection with prefix
|\childdocforwardprefix[|\textit{prefix}|]{|\textit{dest}|}|
is accomplished by:
%
\begin{center}
\begin{tabular}{l}
|{\edef\jobname{\scantokens\expandafter{\jobname\noexpand}}|\\
|\def\redirectjob |\textit{prefix}|#1~~~{\gdef\jobname{|\textit{dest}|#1}}|\\
|\expandafter\redirectjob\jobname~~~}\input{\jobname}|
\end{tabular}
\end{center}

In an alternative approach,
child documents can be compiled by a specific command line
without additional code or specific definitions:
%
\begin{center}
|... -jobname "|\textit{target}|" "|[\textit{flags}]%
|\includeonly{|\textit{dest}|}\input{|\textit{main}|}"|
\end{center}
%

%%%%%%%%%%%%%%%%%%%%%%%%%%%%%%%%%%%%%%%%%%%%%%%%%%%%%%%%%%%%%%%%%%%%%%%%%%%%%%%%
%%%%%%%%%%%%%%%%%%%%%%%%%%%%%%%%%%%%%%%%%%%%%%%%%%%%%%%%%%%%%%%%%%%%%%%%%%%%%%%%
\section{Information}

%%%%%%%%%%%%%%%%%%%%%%%%%%%%%%%%%%%%%%%%%%%%%%%%%%%%%%%%%%%%%%%%%%%%%%%%%%%%%%%%
\subsection{Copyright}

Copyright \copyright{} 2017--2018 Niklas Beisert

This work may be distributed and/or modified under the
conditions of the \LaTeX{} Project Public License, either version 1.3
of this license or (at your option) any later version.
The latest version of this license is in
  \url{http://www.latex-project.org/lppl.txt}
and version 1.3 or later is part of all distributions of \LaTeX{}
version 2005/12/01 or later.

This work has the LPPL maintenance status `maintained'.

The Current Maintainer of this work is Niklas Beisert.

This work consists of the files |README.txt|, |childdoc.ins| and |childdoc.dtx|
as well as the derived files |childdoc.def|, |cdocsamp.tex|
with |cdocsch1.tex|, |cdocsch2.tex|, |cdocspt3.tex|, |cdocspt4.tex|,
|cdocsdrf.tex|, |cdocsfn1.tex|, |cdocsfn2.tex|
as well as |childdoc.pdf|.

%%%%%%%%%%%%%%%%%%%%%%%%%%%%%%%%%%%%%%%%%%%%%%%%%%%%%%%%%%%%%%%%%%%%%%%%%%%%%%%%
\subsection{Files and Installation}

The package consists of the files:
%
\begin{center}
\begin{tabular}{ll}
    |README.txt|   & readme file \\
    |childdoc.ins| & installation file \\
    |childdoc.dtx| & source file \\
    |childdoc.def| & definition file \\
    |cdocsamp.tex| & sample main file \\
    |cdocsch1.tex| & sample include file \\
    |cdocsch2.tex| & sample include file \\
    |cdocspt3.tex| & sample part file \\
    |cdocspt4.tex| & sample part file \\
    |cdocsdrf.tex| & sample redirection file \\
    |cdocsfn1.tex| & sample redirection file \\
    |cdocsfn2.tex| & sample redirection file \\
    |childdoc.pdf| & manual
\end{tabular}
\end{center}
%
The distribution consists of the files
|README.txt|, |childdoc.ins| and |childdoc.dtx|.
%
\begin{itemize}
\item
Run (pdf)\LaTeX{} on |childdoc.dtx|
to compile the manual |childdoc.pdf| (this file).
\item
Run \LaTeX{} on |childdoc.ins| to create the definitions file |childdoc.def|
and the sample |cdocsamp.tex| with include files
|cdocsch1.tex|, |cdocsch2.tex|, |cdocspt3.tex|, |cdocspt4.tex|,
|cdocsdrf.tex|, |cdocsfn1.tex|, |cdocsfn2.tex|.
Then copy the file |childdoc.def| to an appropriate directory of your \LaTeX{}
distribution, e.g.\ \textit{texmf-root}|/tex/latex/childdoc|.
\end{itemize}

%%%%%%%%%%%%%%%%%%%%%%%%%%%%%%%%%%%%%%%%%%%%%%%%%%%%%%%%%%%%%%%%%%%%%%%%%%%%%%%%
\subsection{Related CTAN Packages}

There are several other packages which offer a similar functionality:
%
\begin{itemize}
\item
The packages
\href{http://ctan.org/pkg/docmute}{\textsf{docmute}},
\href{http://ctan.org/pkg/includex}{\textsf{includex}} and
\href{http://ctan.org/pkg/standalone}{\textsf{standalone}}
provide commands to include only the document body of
a child file thus allowing both files to be compiled individually.
\item
The packages \href{http://ctan.org/pkg/subdocs}{\textsf{subdocs}}
and \href{http://ctan.org/pkg/subfiles}{\textsf{subfiles}}
provide structures in which the main and child documents can be
encapsulated and allowing them to be compiled individually.
The inclusion mechanism is different from the conventional |\include|.
\item
The package \href{http://ctan.org/pkg/combine}{\textsf{combine}}
is an elaborate solution to combine several documents into one.
\end{itemize}
%
See also the CTAN topic \href{http://ctan.org/topic/subdocs}{\textsf{subdocs}}
for further related packages.
The present package differs from the above solutions in that
a document structure constructed with the conventional |\include| mechanism
just needs two extra commands at the top of every file
such that all constituent files can be compiled individually.

%%%%%%%%%%%%%%%%%%%%%%%%%%%%%%%%%%%%%%%%%%%%%%%%%%%%%%%%%%%%%%%%%%%%%%%%%%%%%%%%
%\subsection{Feature Suggestions}
%
%The following is a list of features which may be useful for future
%versions of this package:
%%
%\begin{itemize}
%\item
%\ldots
%\end{itemize}

%%%%%%%%%%%%%%%%%%%%%%%%%%%%%%%%%%%%%%%%%%%%%%%%%%%%%%%%%%%%%%%%%%%%%%%%%%%%%%%%
\subsection{Revision History}

%%%%%%%%%%%%%%%%%%%%%%%%%%%%%%%%%%%%%%%%
\paragraph{v2.0:} 2018/12/30

\begin{itemize}
\item
immediate forward processing
\item
added |\childdocby| mechanism
\item
manual restructured
\end{itemize}

%%%%%%%%%%%%%%%%%%%%%%%%%%%%%%%%%%%%%%%%
\paragraph{v1.6:} 2018/01/17

\begin{itemize}
\item
application for development of include files
\item
corrections to manual
\end{itemize}

%%%%%%%%%%%%%%%%%%%%%%%%%%%%%%%%%%%%%%%%
\paragraph{v1.5:} 2017/05/21

\begin{itemize}
\item
more complete structuring introduced
\item
|\childdocof| introduced
\item
|\childdoc| renamed to |\childdocmain|
\item
|\childredirect| renamed to |\childdocforward| and |\childdocforwardprefix|
and functionality expanded
\end{itemize}

%%%%%%%%%%%%%%%%%%%%%%%%%%%%%%%%%%%%%%%%
\paragraph{v1.0:} 2017/04/27

\begin{itemize}
\item
manual and install package
\item
first version published on CTAN
\end{itemize}

%%%%%%%%%%%%%%%%%%%%%%%%%%%%%%%%%%%%%%%%
\paragraph{v0.6:} 2017/04/26

\begin{itemize}
\item
redirection mechanism added
\end{itemize}

%%%%%%%%%%%%%%%%%%%%%%%%%%%%%%%%%%%%%%%%
\paragraph{v0.5:} 2017/04/26

\begin{itemize}
\item
functionality in definition file
\end{itemize}


%%%%%%%%%%%%%%%%%%%%%%%%%%%%%%%%%%%%%%%%%%%%%%%%%%%%%%%%%%%%%%%%%%%%%%%%%%%%%%%%
%%%%%%%%%%%%%%%%%%%%%%%%%%%%%%%%%%%%%%%%%%%%%%%%%%%%%%%%%%%%%%%%%%%%%%%%%%%%%%%%
%%%%%%%%%%%%%%%%%%%%%%%%%%%%%%%%%%%%%%%%%%%%%%%%%%%%%%%%%%%%%%%%%%%%%%%%%%%%%%%%
\appendix

\settowidth\MacroIndent{\rmfamily\scriptsize 000\ }

 \DocInput{childdoc.dtx}

\end{document}
%</driver>
% \fi
%
% %%%%%%%%%%%%%%%%%%%%%%%%%%%%%%%%%%%%%%%%%%%%%%%%%%%%%%%%%%%%%%%%%%%%%%%%%%%%%%
% %%%%%%%%%%%%%%%%%%%%%%%%%%%%%%%%%%%%%%%%%%%%%%%%%%%%%%%%%%%%%%%%%%%%%%%%%%%%%%
% \section{Sample}
%\iffalse
%<*samplemain>
%\fi
%
% The following presents a sample document
% with two chapters, two parts, a title page,
% a compile flag as well as three forwarding files to set the flag.
% It consists of eight |.tex| files:
% \begin{center}
% \begin{tabular}{ll}
% |cdocsamp.tex|&main file\\
% |cdocsch1.tex|&include file for chapter 1\\
% |cdocsch2.tex|&include file for chapter 2\\
% |cdocspt3.tex|&include file for part 3\\
% |cdocspt4.tex|&include file for part 4\\
% |cdocsdrf.tex|&forwarding file for main file in draft mode\\
% |cdocsfi1.tex|&forwarding file for final version of chapter 1\\
% |cdocsfi2.tex|&forwarding file for final version of chapter 2\\
% \end{tabular}
% \end{center}
% Each of the eight files can be compiled directly by the \LaTeX{} compiler.
%
% %%%%%%%%%%%%%%%%%%%%%%%%%%%%%%%%%%%%%%
% \paragraph{Main File.}
%
% The main file is called |cdocsamp.tex|.
%
% Load the \textsf{childdoc} definitions and
% declare the filename for the main document:
%    \begin{macrocode}
\input{childdoc.def}
\childdocmain{}
%    \end{macrocode}

% Optional override for |\version| flag:
%    \begin{macrocode}
%%\ifchilddoc\else\providecommand{\version}{draft}\fi
%    \end{macrocode}

% Define the default values for the |\version| flag
% (|final| for the main file and |draft| for childs):
%    \begin{macrocode}
\ifchilddoc
\providecommand{\version}{draft}
\else
\providecommand{\version}{final}
\fi
%    \end{macrocode}

% Load the standard document class:
%    \begin{macrocode}
\documentclass[12pt]{article}
%    \end{macrocode}

% Start the document body:
%    \begin{macrocode}
\begin{document}
%    \end{macrocode}

% Declare a title page.
% Print title, part of document being processed and version flag:
%    \begin{macrocode}
\addtocounter{page}{-1}
\begin{center}
{\LARGE\bfseries{}childdoc example\par}
\vspace{1cm}
\ifchilddoc
\ifchilddocmanual part\else chapter\fi:
`\childdocname' of `\childdocjob'\par
\else
main document: `\childdocjob'\par
\fi
version: \version\par
\end{center}
\newpage
%    \end{macrocode}

% Manually include selected file,
% otherwise process as usual:
%    \begin{macrocode}
\ifchilddocmanual
\section*{part `\childdocname'}
\input{\childdocname}
\else
%    \end{macrocode}

% Include the two chapters:
%    \begin{macrocode}
\include{cdocsch1}
\include{cdocsch2}
%    \end{macrocode}

% Include the two parts unless only chapters should be displayed:
%    \begin{macrocode}
\ifchilddoc\else
\section{part three}
\input{cdocspt3}
\section{part four}
\input{cdocspt4}
\fi
%    \end{macrocode}

% Process as usual until here:
%    \begin{macrocode}
\fi
%    \end{macrocode}

% End of document body:
%    \begin{macrocode}
\end{document}
%    \end{macrocode}
%\iffalse
%</samplemain>
%\fi
%
% %%%%%%%%%%%%%%%%%%%%%%%%%%%%%%%%%%%%%%
% \paragraph{Chapter Include Files.}
%
% The include files are called |cdocsch1.tex| and |cdocsch2.tex|.
%
%\iffalse
%<*samplechap1|samplechap2>
%\fi

% Optional override for |\version| flag:
%    \begin{macrocode}
%%\providecommand{\version}{final}
%    \end{macrocode}

% Include the main document:
%    \begin{macrocode}
\input{childdoc.def}
\childdocof{cdocsamp}
%    \end{macrocode}

%\iffalse
%</samplechap1|samplechap2>
%\fi
%
%\iffalse
%<*samplechap1>
%\fi
% Some text for chapter 1:
%    \begin{macrocode}
\section{one}
some text in chapter one
%    \end{macrocode}

%\iffalse
%</samplechap1>
%\fi
% Some text for chapter 2:
%\iffalse
%<*samplechap2>
%\fi
%    \begin{macrocode}
\section{two}
more text in chapter two
%    \end{macrocode}

%\iffalse
%</samplechap2>
%\fi
%
% %%%%%%%%%%%%%%%%%%%%%%%%%%%%%%%%%%%%%%
% \paragraph{Part Include Files.}
%
% The include files are called |cdocspt3.tex| and |cdocspt4.tex|.
%
%\iffalse
%<*samplepart3|samplepart4>
%\fi

% Optional override for |\version| flag:
%    \begin{macrocode}
%%\providecommand{\version}{final}
%    \end{macrocode}

% Include the main document:
%    \begin{macrocode}
\input{childdoc.def}
\childdocby{cdocsamp}
%    \end{macrocode}

%\iffalse
%</samplepart3|samplepart4>
%\fi
%
%\iffalse
%<*samplepart3>
%\fi
% Some text for part 3:
%    \begin{macrocode}
some text in part three
%    \end{macrocode}

%\iffalse
%</samplepart3>
%\fi
% Some text for part 4:
%\iffalse
%<*samplepart4>
%\fi
%    \begin{macrocode}
more text in part four
%    \end{macrocode}

%\iffalse
%</samplepart4>
%\fi
%
% %%%%%%%%%%%%%%%%%%%%%%%%%%%%%%%%%%%%%%
% \paragraph{Forwarding for a Complete Draft.}
%
% The following forwarding file |cdocsdrf.tex|
% compiles the main document in draft mode:
%\iffalse
%<*sampledraft>
%\fi
%    \begin{macrocode}
\def\version{draft}
\input{childdoc.def}
\childdocforward{cdocsamp}
%    \end{macrocode}

%\iffalse
%</sampledraft>
%\fi
%
% %%%%%%%%%%%%%%%%%%%%%%%%%%%%%%%%%%%%%%
% \paragraph{Forwarding for Final Version of the Chapters.}
%
% The following forwarding files |cdocsfn1.tex| and |cdocsfn2.tex|
% (with identical content)
% compile the final versions of the child documents
% |cdocsch1.tex| and |cdocsch2.tex|, respectively:
%\iffalse
%<*samplefinal>
%\fi
%    \begin{macrocode}
\def\version{final}
\input{childdoc.def}
\childdocforwardprefix[cdocsamp]{cdocsfn}{cdocsch}
%    \end{macrocode}

%\iffalse
%</samplefinal>
%\fi
%
% %%%%%%%%%%%%%%%%%%%%%%%%%%%%%%%%%%%%%%
% \paragraph{Command Line Processing.}
%
% The following three command lines generate the output files
% |cdocscld|, |cdocscl1| and |cdocscl2|
% which should be identical to
% |cdocsdrf|, |cdocsch1| and |cdocsfn2|, respectively:
% \begin{center}
% \begin{tabular}{l}
% |latex -jobname cdocscld \|\\
% |  "\def\version{draft}\input{childdoc.def}\childdocforward{cdocsamp}"|\\
% |latex -jobname cdocscl1 \|\\
% |  "\input{childdoc.def}\childdocforward[cdocsamp]{cdocsch1}"|\\
% |latex -jobname cdocscl2 \|\\
% |  "\def\version{final}\input{childdoc.def}\childdocforward{cdocsch2}"|
% \end{tabular}
% \end{center}
% Note that the trailing backslash on each first line
% merely continues the input to the second line
% (for convenient cut ant paste).
% Furthermore, the command |latex| can be replaced by any
% of its alternative versions such as |pdflatex|.
%
% %%%%%%%%%%%%%%%%%%%%%%%%%%%%%%%%%%%%%%%%%%%%%%%%%%%%%%%%%%%%%%%%%%%%%%%%%%%%%%
% %%%%%%%%%%%%%%%%%%%%%%%%%%%%%%%%%%%%%%%%%%%%%%%%%%%%%%%%%%%%%%%%%%%%%%%%%%%%%%
% \section{Implementation}
%\iffalse
%<*package>
%\fi
%
% This section describes the definitions file |childdoc.def|.

% The definitions cannot be loaded using |\usepackage| or |\RequirePackage|
% which has a mechanism to prevent loading a style file more than once.
% When loading the definitions by means of |\input|
% multiple instances have to be prevented manually:
%\iffalse
%This code needs to be before the `\ProvidesFile' directive
%which is defined at the beginning of this file.
%Therefore it is also placed there and commented out here.
%</package>
%<*discard>
%\fi
%    \begin{macrocode}
\ifdefined\childdocmain\endinput\fi
%    \end{macrocode}
%\iffalse
%</discard>
%<*package>
%\fi
%
% \macro{\ifchilddoc}
% \macro{\ifchilddocmanual}
% The conditional |\ifchilddoc| tells whether a
% child (true) or main (false) document is being compiled.
% The conditional |\ifchilddocmanual| tells whether
% the |\includeonly| mechanism is used (false) or
% the selection of child files must be performed manually (true).
% The definitions initialise to false:
%    \begin{macrocode}
\newif\ifchilddoc
\newif\ifchilddocmanual
%    \end{macrocode}

% \macro{\childdocname}
% \macro{\childdocjob}
% The macro |\childdocname| stores the name of the main document
% to be compiled. The macro |\childdocjob| stores the name of
% the document on which the \LaTeX{} compiler was originally invoked.
% The content of |\jobname| cannot be compared
% to filenames specified in the source due to different catcodes.
% The following code rescans |\jobname|, stores the result
% in |\childdocname| and saves a copy in |\childdocjob|:
%    \begin{macrocode}
\edef\childdocname{\scantokens\expandafter{\jobname\noexpand}}
\let\childdocjob\childdocname
%    \end{macrocode}

% \macro{\childdocdisable}
% The macro |\childdocdisable| prevents the main file
% from being processed more than once.
% At this stage, the main document command |\childdocmain|
% is assumed to be called once again where it should do nothing.
% Any subsequent call to it should prevent
% a secondary processing of the main document
% It overwrites the forwarding commands
% |\childdocof| and |\childdocforward|
% with empty macros to prevent further inclusions of the main document:
%    \begin{macrocode}
\newcommand{\childdocdisable}
{
  \renewcommand{\childdocmain}[1]{\renewcommand{\childdocmain}[1]{\endinput}}
  \renewcommand{\childdocof}[1]{}
  \renewcommand{\childdocby}[2][]{}
  \renewcommand{\childdocforward}[2][]{}
  \renewcommand{\childdocdisable}{}
}
%    \end{macrocode}

% \macro{\childdocmain}
% The macro |\childdocmain| is to be called at the top of the main file
% with nothing or the main filename (without extension) as argument.
% First, it breaks loops.
% If the argument is not empty and does not match |\childdocname|
% (which is set by the first inclusion of |childdoc.def|),
% |\ifchilddoc| is set to true, |\includeonly| is applied to the child file
% and |\jobname| is set to the main file
% (for proper handling of |.aux| files):
%    \begin{macrocode}
\newcommand{\childdocmain}[1]
{
  \childdocdisable\childdocmain{}
  \if?#1?\else
    \begingroup
      \def\childdoctmp{#1}
      \ifx\childdoctmp\childdocname
        \def\childdoctmp{}
      \else
        \def\childdoctmp
        {
          \childdoctrue
          \includeonly{\childdocname}
          \def\childdocjob{#1}
          \def\jobname{#1}
        }
      \fi
      \expandafter
    \endgroup
    \childdoctmp
  \fi
}
%    \end{macrocode}

% \macro{\childdocof}
% The command |\childdocof| redirects
% compilation to the main file |#1|.
%    \begin{macrocode}
\newcommand{\childdocof}[1]
{
  \childdocdisable
  \childdoctrue
  \includeonly{\childdocname}
  \def\jobname{#1}
  \def\childdocjob{#1}
  \input{#1}
}
%    \end{macrocode}

% \macro{\childdocby}
% The command |\childdocby| ....
%    \begin{macrocode}
\newcommand{\childdocby}[2][]
{
  \childdocdisable
  \childdoctrue
  \childdocmanualtrue
  \if?#1?\else
    \def\jobname{#2}
  \fi
  \def\childdocjob{#2}
  \input{#2}
  \endinput
}
%    \end{macrocode}

% \macro{\childdocforward}
% The command |\childdocforward| redirects
% compilation to the main file or
% (if the optional argument is given) a child file.
% Parameters are set as if the main file
% or a child file starting with |\childdocof| was compiled.
% Then compilation is handed over to the main file:
%    \begin{macrocode}
\newcommand{\childdocforward}[2][]
{
  \begingroup
    \if?#1?
      \def\childdoctmp
      {
        \def\childdocname{#2}
        \def\childdocjob{#2}
        \def\jobname{#2}
        \input{#2}
        \endinput
      }
    \else
      \def\childdoctmp
      {
        \childdocdisable
        \def\childdocname{#2}
        \childdoctrue
        \includeonly{#2}
        \def\childdocjob{#1}
        \def\jobname{#1}
        \input{#1}
        \endinput
      }
    \fi
    \expandafter
  \endgroup
  \childdoctmp
}
%    \end{macrocode}

% \macro{\childdocforwardprefix}
% The command |\childdocforwardprefix| redirects
% compilation to the main or a child file by means of a pattern.
% The prefix |#1| in the current filename is replaced by |#2|
% and the suffix of the current filename is kept
% (it is assumed that the filename does not contain the substring `|~~~|'
% which is used as a delimiter).
% Compilation is handed over to the new file by |\childdocforward|:
%    \begin{macrocode}
\newcommand{\childdocforwardprefix}[3][]
{
  \begingroup
    \def\childdocextract #2##1~~~{\def\childdoctmp{\childdocforward[#1]{#3##1}}}
    \expandafter\childdocextract\childdocname~~~
    \expandafter
  \endgroup
  \childdoctmp
}
%    \end{macrocode}

% \macro{\childdoc}
% The deprecated macro |\childdoc| is a legacy version of |\childdocmain|:
%    \begin{macrocode}
\newcommand{\childdoc}{\childdocmain}
%    \end{macrocode}

% \macro{\childdocredirect}
% The deprecated macro |\childdocredirect| is a legacy version
% of |\childdocforward| and |\childdocforwardprefix|:
%    \begin{macrocode}
\newcommand{\childdocredirect}[2][]
{
  \begingroup
    \if?#1?
      \def\childdoctmp{\childdocforward{#2}}
    \else
      \def\childdoctmp{\childdocforwardprefix{#1}{#2}}
    \fi
    \expandafter
  \endgroup
  \childdoctmp
}
%    \end{macrocode}

%\iffalse
%</package>
%\fi
%
\endinput

\childdocmain{}
%    \end{macrocode}

% Optional override for |\version| flag:
%    \begin{macrocode}
%%\ifchilddoc\else\providecommand{\version}{draft}\fi
%    \end{macrocode}

% Define the default values for the |\version| flag
% (|final| for the main file and |draft| for childs):
%    \begin{macrocode}
\ifchilddoc
\providecommand{\version}{draft}
\else
\providecommand{\version}{final}
\fi
%    \end{macrocode}

% Load the standard document class:
%    \begin{macrocode}
\documentclass[12pt]{article}
%    \end{macrocode}

% Start the document body:
%    \begin{macrocode}
\begin{document}
%    \end{macrocode}

% Declare a title page.
% Print title, part of document being processed and version flag:
%    \begin{macrocode}
\addtocounter{page}{-1}
\begin{center}
{\LARGE\bfseries{}childdoc example\par}
\vspace{1cm}
\ifchilddoc
\ifchilddocmanual part\else chapter\fi:
`\childdocname' of `\childdocjob'\par
\else
main document: `\childdocjob'\par
\fi
version: \version\par
\end{center}
\newpage
%    \end{macrocode}

% Manually include selected file,
% otherwise process as usual:
%    \begin{macrocode}
\ifchilddocmanual
\section*{part `\childdocname'}
\input{\childdocname}
\else
%    \end{macrocode}

% Include the two chapters:
%    \begin{macrocode}
\include{cdocsch1}
\include{cdocsch2}
%    \end{macrocode}

% Include the two parts unless only chapters should be displayed:
%    \begin{macrocode}
\ifchilddoc\else
\section{part three}
\input{cdocspt3}
\section{part four}
\input{cdocspt4}
\fi
%    \end{macrocode}

% Process as usual until here:
%    \begin{macrocode}
\fi
%    \end{macrocode}

% End of document body:
%    \begin{macrocode}
\end{document}
%    \end{macrocode}
%\iffalse
%</samplemain>
%\fi
%
% %%%%%%%%%%%%%%%%%%%%%%%%%%%%%%%%%%%%%%
% \paragraph{Chapter Include Files.}
%
% The include files are called |cdocsch1.tex| and |cdocsch2.tex|.
%
%\iffalse
%<*samplechap1|samplechap2>
%\fi

% Optional override for |\version| flag:
%    \begin{macrocode}
%%\providecommand{\version}{final}
%    \end{macrocode}

% Include the main document:
%    \begin{macrocode}
% \iffalse
%
% childdoc.dtx Copyright (C) 2017-2018 Niklas Beisert
%
% This work may be distributed and/or modified under the
% conditions of the LaTeX Project Public License, either version 1.3
% of this license or (at your option) any later version.
% The latest version of this license is in
%   http://www.latex-project.org/lppl.txt
% and version 1.3 or later is part of all distributions of LaTeX
% version 2005/12/01 or later.
%
% This work has the LPPL maintenance status `maintained'.
%
% The Current Maintainer of this work is Niklas Beisert.
%
% This work consists of the files childdoc.dtx and childdoc.ins
% and the derived files childdoc.def and cdocsamp.tex with
% cdocsch1.tex, cdocsch2.tex, cdocsdrf.tex, cdocsfn1.tex, cdocsfn2.tex.
%
%<package>\ifdefined\childdocmain\endinput\fi
%<package>\ProvidesFile{childdoc.def}[2018/12/30 v2.0 child document driver]
%<samplemain>\ProvidesFile{cdocsamp.tex}[2018/12/30 v2.0 sample for childdoc]
%<*driver>
%\ProvidesFile{childdoc.drv}[2018/12/30 v2.0 childdoc reference manual file]
\PassOptionsToClass{10pt,a4paper}{article}
\documentclass{ltxdoc}

\usepackage[margin=35mm]{geometry}
\usepackage{hyperref}
\usepackage{hyperxmp}
\usepackage[usenames]{color}

\hypersetup{colorlinks=true}
\hypersetup{pdfstartview=FitH}
\hypersetup{pdfpagemode=UseNone}
\hypersetup{pdfsource={}}
\hypersetup{pdflang={en-UK}}
\hypersetup{pdfcopyright={Copyright 2017-2018 Niklas Beisert.
  This work may be distributed and/or modified under the
  conditions of the LaTeX Project Public License, either version 1.3
  of this license or (at your option) any later version.}}
\hypersetup{pdflicenseurl={http://www.latex-project.org/lppl.txt}}
\hypersetup{pdfcontactaddress={ETH Zurich, ITP, HIT K,
  Wolfgang-Pauli-Strasse 27}}
\hypersetup{pdfcontactpostcode={8093}}
\hypersetup{pdfcontactcity={Zurich}}
\hypersetup{pdfcontactcountry={Switzerland}}
\hypersetup{pdfcontactemail={nbeisert@itp.phys.ethz.ch}}
\hypersetup{pdfcontacturl={http://people.phys.ethz.ch/\xmptilde nbeisert/}}

\newcommand{\secref}[1]{\hyperref[#1]{section \ref*{#1}}}

\parskip1ex
\parindent0pt
\let\olditemize\itemize
\def\itemize{\olditemize\parskip0pt}

\begin{document}

\title{The \textsf{childdoc} Package}
\hypersetup{pdftitle={The childdoc Package}}
\author{Niklas Beisert\\[2ex]
  Institut f\"ur Theoretische Physik\\
  Eidgen\"ossische Technische Hochschule Z\"urich\\
  Wolfgang-Pauli-Strasse 27, 8093 Z\"urich, Switzerland\\[1ex]
  \href{mailto:nbeisert@itp.phys.ethz.ch}
  {\texttt{nbeisert@itp.phys.ethz.ch}}}
\hypersetup{pdfauthor={Niklas Beisert}}
\hypersetup{pdfsubject={Manual for the LaTeX2e Package childdoc}}
\date{30 December 2018, \textsf{v2.0}}
\maketitle

\begin{abstract}\noindent
\textsf{childdoc} is a \LaTeXe{} package
that enables the direct compilation
of document sections included by |\include|
to individual files.
\end{abstract}

\begingroup
\parskip0ex
\tableofcontents
\endgroup

%%%%%%%%%%%%%%%%%%%%%%%%%%%%%%%%%%%%%%%%%%%%%%%%%%%%%%%%%%%%%%%%%%%%%%%%%%%%%%%%
%%%%%%%%%%%%%%%%%%%%%%%%%%%%%%%%%%%%%%%%%%%%%%%%%%%%%%%%%%%%%%%%%%%%%%%%%%%%%%%%
\section{Introduction}

\LaTeX{} provides a mechanism to structure a large document (such as a book)
into a main file and several child files (containing the chapters)
using the |\include| command.
This mechanism is beneficial for documents
which span hundreds of pages in order to
make the source file(s) more manageable.
Moreover, compilation can be restricted to
selected child files by means of the |\includeonly| command.
The latter feature can be used to reduce the compilation time while editing
(this was significantly more useful in the earlier days of \LaTeX{})
or to generate a smaller document which is easier to navigate.
Another application of |\includeonly| is to generate
documents consisting of selected parts of the complete document.

However, there are a few drawbacks of the plain |\include| mechanism:
\begin{itemize}
\item
The child files cannot be compiled on their own,
they can only be compiled via the main file.
A naive editing environment
(such as a text editor with an option
to have the current file processed by \LaTeX)
may require one to switch to the main file before compiling;
attempting to compile the child file produces errors.
\item
The main file must be modified (each time)
to adjust the |\includeonly| command
to the present needs. This easily leaves the main file in a messy state.
\item
The generated document will always carry the filename
of the main document. This is inconvenient if
several child files are to be compiled and
to be kept for distribution.
\end{itemize}

The present package provides a simple interface
to make child files individually compilable by \LaTeX{}.
Compiling a child file then has the same effect as compiling
the main file with an |\includeonly| command
to select the appropriate child.
Moreover the generated document will carry the name of the child
rather than the main file.
This resolves all three above issues.

This feature is meant to make the editing of books,
thesis documents and lecture notes somewhat more convenient.
However, the package can also be used efficiently for
composing a series of documents (such as exercise sheets)
which are typically distributed individually.
It then assists the author in generating the individual documents
(potentially in different versions)
as well as a document containing the collected series.
Another application is in developing style files
or other kinds of included material
where compilation of the style file could redirect
to a sample or test file.

%%%%%%%%%%%%%%%%%%%%%%%%%%%%%%%%%%%%%%%%%%%%%%%%%%%%%%%%%%%%%%%%%%%%%%%%%%%%%%%%
%%%%%%%%%%%%%%%%%%%%%%%%%%%%%%%%%%%%%%%%%%%%%%%%%%%%%%%%%%%%%%%%%%%%%%%%%%%%%%%%
\section{Usage}

First of all, the package \textsf{childdoc} is \emph{not} a standard
\LaTeXe{} |.sty| style file! Therefore it needs to be invoked in
a non-standard way.

%%%%%%%%%%%%%%%%%%%%%%%%%%%%%%%%%%%%%%%%%%%%%%%%%%%%%%%%%%%%%%%%%%%%%%%%%%%%%%%%
\subsection{Included Files}
\label{sec:include}

%%%%%%%%%%%%%%%%%%%%%%%%%%%%%%%%%%%%%%%%
\DescribeMacro{\childdocmain}
To use the package, add the commands
\begin{center}
\begin{tabular}{l}
|\input{childdoc.def}|\\
|\childdocmain{}|\\
\end{tabular}
\end{center}
at the very top of the main \LaTeX{} file,
in particular \emph{before} the |\documentclass| statement!
The argument of |\childdocmain| should be left empty
(but it must be present).

%%%%%%%%%%%%%%%%%%%%%%%%%%%%%%%%%%%%%%%%
\DescribeMacro{\childdocof}
Furthermore, add the commands
\begin{center}
\begin{tabular}{l}
|\input{childdoc.def}|\\
|\childdocof{|\textit{main}|}|\\
\end{tabular}
\end{center}
at the top of every child file \textit{child}
which is included by |\include{|\textit{child}|}|
from within the main file
(or at least for those files to be compiled individually).
The argument \textit{main} must be the filename of the main file.

There are a couple of
considerations in setting up the main and child documents:

%%%%%%%%%%%%%%%%%%%%%%%%%%%%%%%%%%%%%%%%
\paragraph{Restrictions.}

Please note the following restrictions:
\begin{itemize}
\item
|\childdocmain| must be called with one argument \textit{main}
to ensure compatibility with earlier version of the package.
It must either be empty (|\childdocmain{}|)
or precisely match the filename of the main file in which it is specified.
See \secref{sec:detection} for further information.
\item
The filename \textit{main} must be specified without the |.tex| extension.
\item
The filename \textit{main} is case sensitive
(even in case-insensitive file systems)
due to internal string comparison.
\item
The argument \textit{main} should be fully expanded, it cannot be a macro.
\item
Subdirectories and special characters should be avoided in filenames.
\item
The command |\childdocmain{|\textit{main}|}| must be followed by a whitespace.
It should not be followed immediately by another command
or by a comment mark `|%|'.
This is because the \TeX{} parser reads the token immediately following
the argument of |\childdocmain| and puts it
at the beginning of every child section;
however, a white\-space is ignored.
\end{itemize}

%%%%%%%%%%%%%%%%%%%%%%%%%%%%%%%%%%%%%%%%
\paragraph{Content of Main File.}

It is advisable to place all content in the child files included by |\include|.
Any output contained in the main file will appear in all child documents
unless suppressed manually;
it cannot be suppressed automatically by the |\includeonly| directive
and thus should normally be avoided.
A method to include some content in the main file
by means of conditional processing is described in \secref{sec:conditional}.

%%%%%%%%%%%%%%%%%%%%%%%%%%%%%%%%%%%%%%%%
\paragraph{Page Numbering.}

When only a part of the document is compiled,
the appropriate numbering of pages
(as well as other status parameters)
is determined from the |.aux| files.
The latter contain information from previous passes.
However this information needs to propagate through
all intermediate child documents.
Therefore the page numbering in child documents may well
be inconsistent until the complete document is compiled at least once.

A useful (if unconventional) way to always ensure a consistent
page numbering is to restart the numbering in each child document
and denote the pages by `\textit{child}|.|\textit{page}'
where \textit{child} represents the chapter/section number of the child file.
This can be achieved by the command
|\numberwithin{page}{|\textit{child}|}|
of the \textsf{amsmath} package
where \textit{child} can be |chapter| or |section|
depending on the chosen structuring.
Alternatively, one can modify the macro |\thepage| appropriately
and reset the counter |page| at the start of each child file.

%%%%%%%%%%%%%%%%%%%%%%%%%%%%%%%%%%%%%%%%%%%%%%%%%%%%%%%%%%%%%%%%%%%%%%%%%%%%%%%%
\subsection{Conditional Processing}
\label{sec:conditional}

The package provides a mechanism to compile different versions
of a document. To customise the versions further some conditional processing
can come in handy to distinguish which version is being compiled.
The package provides two macros to describe the compilation context:

%%%%%%%%%%%%%%%%%%%%%%%%%%%%%%%%%%%%%%%%
\DescribeMacro{\ifchilddoc}
The conditional |\ifchilddoc| distinguishes between the compilation of
child documents and the main document:
%
\begin{center}
|\ifchilddoc |\textit{child-code}| |[|\||else |\textit{main-code}]| \||fi|
\end{center}

%%%%%%%%%%%%%%%%%%%%%%%%%%%%%%%%%%%%%%%%
\DescribeMacro{\childdocname}
\DescribeMacro{\childdocjob}
The macro |\childdocname| contains the filename (without extension)
of the main or child file being processed.
Note that |\childdocjob| will always contain the name of the main file.

%%%%%%%%%%%%%%%%%%%%%%%%%%%%%%%%%%%%%%%%
\paragraph{Title Page.}

Conditional processing can be used to include a title or banner page
in the main document when proper precautions are taken.
Importantly, the code in the main file should ensure that the page counter
(as well as other status parameters which are stored in the |.aux| files)
takes the same value after the conditional processing.
Otherwise the page numbers may take divergent values
depending on which part is compiled.

For example, a title page could be declared by:
%
\begin{center}
\begin{tabular}{l}
|\ifchilddoc\||else|\\
|\addtocounter{page}{-1}|\\
\textit{code for title page}\\
|\newpage|\\
|\||fi|
\end{tabular}
\end{center}
%
A banner page for the child documents can be generated by:
%
\begin{center}
\begin{tabular}{l}
|\ifchilddoc|\\
|\addtocounter{page}{-1}|\\
\textit{code for banner page}\\
|\newpage|\\
|\||fi|
\end{tabular}
\end{center}
%
Here one could write a message such as:
\begin{center}
|This is the part \childdocname{} of \childdocjob{}.|
\end{center}

%%%%%%%%%%%%%%%%%%%%%%%%%%%%%%%%%%%%%%%%%%%%%%%%%%%%%%%%%%%%%%%%%%%%%%%%%%%%%%%%
\subsection{Flags}
\label{sec:flags}

The package makes it easy to generate different versions
of the main or child documents.
To this end compilation flags can be defined
and assigned different default values.
They will be particularly useful in conjunction
with the forwarding mechanism described in \secref{sec:forward}.

For example, it may be useful to have a flag |\version|
which can be set to |draft| or |final|.
The document source will contain some conditional code
depending on the value of |\version|.
Suppose further, the flag should default to |final| for the main file
and to |draft| for child files
which is a natural assignment for editing the document.
This is achieved by placing the following code
in the preamble of the main document
(below the |\childdocmain| directive):
%
\begin{center}
\begin{tabular}{l}
|\ifchilddoc|\\
|\providecommand{\version}{draft}|\\
|\||else|\\
|\providecommand{\version}{final}|\\
|\||fi|
\end{tabular}
\end{center}
%
The definition by |\providecommand| makes sure
that previous definitions are not overwritten.
Further statements |\providecommand{\version}{...}|
can thus be added before the above code to override it.

For the main file, one might add a line
(between |\childdocmain| and the above block)
%
\begin{center}
|%\ifchilddoc\||else\providecommand{\version}{draft}\||fi|
\end{center}
%
which can be uncommented to produce a draft version.
Likewise one can add a line to the very top of a child file
(above the |\childdocof{|\textit{main}|}| directive)
%
\begin{center}
|%\providecommand{\version}{final}|
\end{center}
%
which can be uncommented to produce the final version of this child document.

%%%%%%%%%%%%%%%%%%%%%%%%%%%%%%%%%%%%%%%%%%%%%%%%%%%%%%%%%%%%%%%%%%%%%%%%%%%%%%%%
\subsection{Forwarding}
\label{sec:forward}

Different versions of the main or child documents
using compilation flags as described in \secref{sec:flags}
can be (permanently) stored in different files
for convenient compilation, viewing and distribution.
To this end, the package defines a command
to pass on compilation to a different file:

%%%%%%%%%%%%%%%%%%%%%%%%%%%%%%%%%%%%%%%%
\DescribeMacro{\childdocforward}
The command |\childdocforward| redirects processing to
another source file:
%
\begin{center}
\begin{tabular}{l}
|\input{childdoc.def}|\\
|\childdocforward[|\textit{main}|]{|\textit{dest}|}|\\
\end{tabular}
\end{center}
%
The argument \textit{dest} is the destination file
(without extension).
It should be the main file or one of the child files.
Note that further \textsf{childdoc} directives
such as |\childdocof| and |\childdocforward|
in the indicated file will be processed in this form.
The optional argument \textit{main}
passes on directly to the main file \textit{main}
while pretending to compile the child \textit{dest}.
This form behaves as if \textit{dest}
issues |\childdocof{|\textit{main}|}| right away,
and no further \textsf{childdoc} directives will be processed.

%%%%%%%%%%%%%%%%%%%%%%%%%%%%%%%%%%%%%%%%
\DescribeMacro{\...prefix}
In the alternative form |\childdocforwardprefix|,
%
\begin{center}
\begin{tabular}{l}
|\input{childdoc.def}|\\
|\childdocforwardprefix[|\textit{main}|]{|\textit{prefix}|}{|\textit{dest}|}|
\end{tabular}
\end{center}
%
the destination file is determined by a pattern
depending on the current file:
To make this work, the current file must be called
`{\textit{prefix}\hspace{0.2em}\textit{suffix}}'
with \textit{prefix} matching precisely the argument.
Processing is then passed on to the file
`{\textit{dest}\hspace{0.2em}\textit{suffix}}'.
Surely, the same effect is achieved by
directly specifying the
argument `{\textit{dest}\hspace{0.2em}\textit{suffix}}'
in the first form.
However, that requires to set up a different file
for each child. With the alternative form of the command
all these files can have exactly the same content
which simplifies setting them up and maintaining them.

For example, the following file |draft.tex|
with a compilation flag |\version| as described in \secref{sec:flags}
compiles the main document as a draft:
%
\begin{center}
\begin{tabular}{l}
|\def\version{draft}|\\
|\input{childdoc.def}|\\
|\childdocforward{|\textit{main}|}|
\end{tabular}
\end{center}
%
Likewise, the following files |final|\textit{nn}|.tex|
compile the final version of the child document
|child|\textit{nn}|.tex|:
%
\begin{center}
\begin{tabular}{l}
|\def\version{final}|\\
|\input{childdoc.def}|\\
|\childdocforwardprefix{final}{child}|
\end{tabular}
\end{center}
%

Note that when several versions of a main file and/or of each child file
are to be generated, it may be convenient to set up a |Makefile| or
shell script to automatise the process.

%%%%%%%%%%%%%%%%%%%%%%%%%%%%%%%%%%%%%%%%%%%%%%%%%%%%%%%%%%%%%%%%%%%%%%%%%%%%%%%%
\subsection{Command Line Processing}
\label{sec:commandline}

The effect of redirection files can also be achieved by invoking
the \LaTeX{} compiler with a more elaborate command line.
Most conveniently this should be done as part
of a shell script or a |Makefile|.

When using \textsf{childdoc} in the main file, the following
command lines effectively perform a redirection
(note that depending on the shell being used,
backslashes may have to be doubled: `|\|' $\to$ `|\\|'):
%
\begin{center}
|... -jobname "|\textit{target}|" |\\|"|[\textit{flags}]%
|\input{childdoc.def}\childdocforward[|\textit{main}|]{|\textit{dest}|}"|
\end{center}
%
Here \textit{target} is the name of the output file,
\textit{main} is the name of the main file
and \textit{dest} is the name of the main or child file to be processed
(all filenames without extensions).
The optional argument \textit{main} can be omitted
if \textit{main} matches \textit{dest}.
Optionally, compilation \textit{flags} can be defined via |\def| commands.
This command line makes the \TeX{} engine believe
it is compiling the file \textit{target}
whose content is specified as the latter parameter.
The provided code then forwards the processing to
\textit{main} or \textit{dest} as described in \secref{sec:forward}.

%%%%%%%%%%%%%%%%%%%%%%%%%%%%%%%%%%%%%%%%%%%%%%%%%%%%%%%%%%%%%%%%%%%%%%%%%%%%%%%%
\subsection{Include by Input}
\label{sec:input}

Including child documents by |\include| has some restrictions by design.
Most notably, the content of a child document always occupies
its own set of pages; pages cannot be shared between child documents.
Usually, this behaviour makes perfect sense
because each child document contain an essential part of the document.
However, in some situations it may be desirable to compose
a document from a collection of parts
without having mandatory page breaks between then.
For this case, the package
provides a mechanism to include parts
by |\input| which can also be processed individually.
However, by construction this mechanism
requires manual handling of the content to be output.

%%%%%%%%%%%%%%%%%%%%%%%%%%%%%%%%%%%%%%%%
\DescribeMacro{\ifchilddocmanual}
The main file should be prepared as usual, see \secref{sec:include}.
However, the document body must make a distinction
between processing of an individual part and of the main document, e.g.:
%
\begin{center}
\begin{tabular}{l}
|\ifchilddocmanual|\\
|\input{\childdocname}|\\
|\||else|\\
\textit{document body with }|\input{|\textit{part}|}|\\
|\||fi|
\end{tabular}
\end{center}
%
The conditional |\ifchilddocmanual| is true whenever
a part to be included by |\input| is being compiled,
and the name of the part is stored in |\childdocname|.

%%%%%%%%%%%%%%%%%%%%%%%%%%%%%%%%%%%%%%%%
\DescribeMacro{\childdocby}
Each part to be included by |\input| should start with:
%
\begin{center}
\begin{tabular}{l}
|\input{childdoc.def}|\\
|\childdocby{|\textit{main}|}|\\
\end{tabular}
\end{center}
%
The directive |\childdocby| is similar to |\childdocof|
described in \secref{sec:include},
but the subsequent selection of content must be done manually.
To that end, both |\ifchilddoc| and |\ifchilddocmanual|
will be true upon processing of a part,
and the name of the part is stored in |\childdocname|.
Note that |\jobname| will be set to the filename of the current part
so that each part receives an individual |.aux| file
that does not interfere with the |.aux| file(s) of the main document.
This behaviour can be altered by the alternative form
|\childdocby[*]{|\textit{main}|}| (with a non-empty optional argument)
which uses the |.aux| file of the main document
by setting |\jobname| to \textit{main}.

%%%%%%%%%%%%%%%%%%%%%%%%%%%%%%%%%%%%%%%%%%%%%%%%%%%%%%%%%%%%%%%%%%%%%%%%%%%%%%%%
\subsection{Driver Development}
\label{sec:driver}

The \textsf{childdoc} mechanism can also be use for the development
of definition files such as \LaTeX{} styles or classes.
This case differs from the above setup with multiple parts
included by |\include| in that no |\includeonly| should be invoked.
This can be achieved by starting the include file
(before |\ProvidesPackage|) with:
%
\begin{center}
\begin{tabular}{l}
|\input{childdoc.def}|\\
|\childdocforward{|\textit{main}|}|\\
\end{tabular}
\end{center}
%
or alternatively with:
%
\begin{center}
\begin{tabular}{l}
|\input{childdoc.def}|\\
|\childdocby{|\textit{main}|}|\\
\end{tabular}
\end{center}
%
Both forms have slightly different effects as described above.
The main file is prepared as usual, see \secref{sec:include}.

%%%%%%%%%%%%%%%%%%%%%%%%%%%%%%%%%%%%%%%%%%%%%%%%%%%%%%%%%%%%%%%%%%%%%%%%%%%%%%%%
\subsection{Legacy Detection}
\label{sec:detection}

The directive |\childdocmain| in the main file can detect
whether the complete document or merely a child is to be compiled
even without using the directive |\childdocof|.
This method is deprecated because it is less robust
and there is no compelling reason to use it;
it is merely provided for backward compatibility
and it may be removed in future versions.

If the detection mechanism is to be used,
it is mandatory to correctly specify
the filename of the main file as the argument of |\childdocmain|:
%
\begin{center}
\begin{tabular}{l}
|\input{childdoc.def}|\\
|\childdocmain{|\textit{main}|}|\\
\end{tabular}
\end{center}
%
If |\jobname| does not match the argument \textit{main} of |\childdocmain|,
it is assumed that |\jobname| points to the child file to be compiled.
When using |\childdocmain| with the main file specified as argument,
it suffices to start a child file
with just |\input{|\textit{main}|}|
without loading of the package and using |\childdocof|.
If instead all processing is done
with the appropriate \textsf{childdoc} directives,
the argument of \textit{main} of |\childdocmain| can be empty.

An alternative version of the command line processing described
in \secref{sec:commandline} using the detection mechanism reads:
%
\begin{center}
|... -jobname "|\textit{target}|" "|[\textit{flags}]%
[|\def\jobname{|\textit{dest}|}|]|\input{|\textit{main}|}"|
\end{center}

%%%%%%%%%%%%%%%%%%%%%%%%%%%%%%%%%%%%%%%%%%%%%%%%%%%%%%%%%%%%%%%%%%%%%%%%%%%%%%%%
\subsection{Manual Code}
\label{sec:manual}

In case one cannot be certain whether the definitions file |childdoc.def|
is installed on the target \TeX{} distribution
and one prefers not to ship it,
it is conceivable to paste a few relevant commands into the sources.

To that end, drop all statements |\input{childdoc.def}|
and perform the replacements as outlined below.
Instead of |\childdocmain{|\textit{main}|}| add the following code
to the top of the main file:
%
\begin{center}
\begin{tabular}{l}
|\||ifdefined\childdocname\endinput\||fi\newif\ifchilddoc|\\
|\edef\childdocname{\scantokens\expandafter{\jobname\noexpand}}|\\
|\def\childdocmain{|\textit{main}|}\||ifx\childdocmain\childdocname\||else|\\
|\childdoctrue\includeonly{\childdocname}\let\jobname\childdocmain\||fi|\\
\end{tabular}
\end{center}
%
Instead of |\childdocof{|\textit{main}|}| just include the main file
at the top of each child file:
%
\begin{center}
|\input{|\textit{main}|}|
\end{center}
%
A simple redirection |\childdocforward{|\textit{dest}|}| is achieved by:
%
\begin{center}
|\def\jobname{|\textit{dest}|}\input{\jobname}|
\end{center}
%
The redirection with prefix
|\childdocforwardprefix[|\textit{prefix}|]{|\textit{dest}|}|
is accomplished by:
%
\begin{center}
\begin{tabular}{l}
|{\edef\jobname{\scantokens\expandafter{\jobname\noexpand}}|\\
|\def\redirectjob |\textit{prefix}|#1~~~{\gdef\jobname{|\textit{dest}|#1}}|\\
|\expandafter\redirectjob\jobname~~~}\input{\jobname}|
\end{tabular}
\end{center}

In an alternative approach,
child documents can be compiled by a specific command line
without additional code or specific definitions:
%
\begin{center}
|... -jobname "|\textit{target}|" "|[\textit{flags}]%
|\includeonly{|\textit{dest}|}\input{|\textit{main}|}"|
\end{center}
%

%%%%%%%%%%%%%%%%%%%%%%%%%%%%%%%%%%%%%%%%%%%%%%%%%%%%%%%%%%%%%%%%%%%%%%%%%%%%%%%%
%%%%%%%%%%%%%%%%%%%%%%%%%%%%%%%%%%%%%%%%%%%%%%%%%%%%%%%%%%%%%%%%%%%%%%%%%%%%%%%%
\section{Information}

%%%%%%%%%%%%%%%%%%%%%%%%%%%%%%%%%%%%%%%%%%%%%%%%%%%%%%%%%%%%%%%%%%%%%%%%%%%%%%%%
\subsection{Copyright}

Copyright \copyright{} 2017--2018 Niklas Beisert

This work may be distributed and/or modified under the
conditions of the \LaTeX{} Project Public License, either version 1.3
of this license or (at your option) any later version.
The latest version of this license is in
  \url{http://www.latex-project.org/lppl.txt}
and version 1.3 or later is part of all distributions of \LaTeX{}
version 2005/12/01 or later.

This work has the LPPL maintenance status `maintained'.

The Current Maintainer of this work is Niklas Beisert.

This work consists of the files |README.txt|, |childdoc.ins| and |childdoc.dtx|
as well as the derived files |childdoc.def|, |cdocsamp.tex|
with |cdocsch1.tex|, |cdocsch2.tex|, |cdocspt3.tex|, |cdocspt4.tex|,
|cdocsdrf.tex|, |cdocsfn1.tex|, |cdocsfn2.tex|
as well as |childdoc.pdf|.

%%%%%%%%%%%%%%%%%%%%%%%%%%%%%%%%%%%%%%%%%%%%%%%%%%%%%%%%%%%%%%%%%%%%%%%%%%%%%%%%
\subsection{Files and Installation}

The package consists of the files:
%
\begin{center}
\begin{tabular}{ll}
    |README.txt|   & readme file \\
    |childdoc.ins| & installation file \\
    |childdoc.dtx| & source file \\
    |childdoc.def| & definition file \\
    |cdocsamp.tex| & sample main file \\
    |cdocsch1.tex| & sample include file \\
    |cdocsch2.tex| & sample include file \\
    |cdocspt3.tex| & sample part file \\
    |cdocspt4.tex| & sample part file \\
    |cdocsdrf.tex| & sample redirection file \\
    |cdocsfn1.tex| & sample redirection file \\
    |cdocsfn2.tex| & sample redirection file \\
    |childdoc.pdf| & manual
\end{tabular}
\end{center}
%
The distribution consists of the files
|README.txt|, |childdoc.ins| and |childdoc.dtx|.
%
\begin{itemize}
\item
Run (pdf)\LaTeX{} on |childdoc.dtx|
to compile the manual |childdoc.pdf| (this file).
\item
Run \LaTeX{} on |childdoc.ins| to create the definitions file |childdoc.def|
and the sample |cdocsamp.tex| with include files
|cdocsch1.tex|, |cdocsch2.tex|, |cdocspt3.tex|, |cdocspt4.tex|,
|cdocsdrf.tex|, |cdocsfn1.tex|, |cdocsfn2.tex|.
Then copy the file |childdoc.def| to an appropriate directory of your \LaTeX{}
distribution, e.g.\ \textit{texmf-root}|/tex/latex/childdoc|.
\end{itemize}

%%%%%%%%%%%%%%%%%%%%%%%%%%%%%%%%%%%%%%%%%%%%%%%%%%%%%%%%%%%%%%%%%%%%%%%%%%%%%%%%
\subsection{Related CTAN Packages}

There are several other packages which offer a similar functionality:
%
\begin{itemize}
\item
The packages
\href{http://ctan.org/pkg/docmute}{\textsf{docmute}},
\href{http://ctan.org/pkg/includex}{\textsf{includex}} and
\href{http://ctan.org/pkg/standalone}{\textsf{standalone}}
provide commands to include only the document body of
a child file thus allowing both files to be compiled individually.
\item
The packages \href{http://ctan.org/pkg/subdocs}{\textsf{subdocs}}
and \href{http://ctan.org/pkg/subfiles}{\textsf{subfiles}}
provide structures in which the main and child documents can be
encapsulated and allowing them to be compiled individually.
The inclusion mechanism is different from the conventional |\include|.
\item
The package \href{http://ctan.org/pkg/combine}{\textsf{combine}}
is an elaborate solution to combine several documents into one.
\end{itemize}
%
See also the CTAN topic \href{http://ctan.org/topic/subdocs}{\textsf{subdocs}}
for further related packages.
The present package differs from the above solutions in that
a document structure constructed with the conventional |\include| mechanism
just needs two extra commands at the top of every file
such that all constituent files can be compiled individually.

%%%%%%%%%%%%%%%%%%%%%%%%%%%%%%%%%%%%%%%%%%%%%%%%%%%%%%%%%%%%%%%%%%%%%%%%%%%%%%%%
%\subsection{Feature Suggestions}
%
%The following is a list of features which may be useful for future
%versions of this package:
%%
%\begin{itemize}
%\item
%\ldots
%\end{itemize}

%%%%%%%%%%%%%%%%%%%%%%%%%%%%%%%%%%%%%%%%%%%%%%%%%%%%%%%%%%%%%%%%%%%%%%%%%%%%%%%%
\subsection{Revision History}

%%%%%%%%%%%%%%%%%%%%%%%%%%%%%%%%%%%%%%%%
\paragraph{v2.0:} 2018/12/30

\begin{itemize}
\item
immediate forward processing
\item
added |\childdocby| mechanism
\item
manual restructured
\end{itemize}

%%%%%%%%%%%%%%%%%%%%%%%%%%%%%%%%%%%%%%%%
\paragraph{v1.6:} 2018/01/17

\begin{itemize}
\item
application for development of include files
\item
corrections to manual
\end{itemize}

%%%%%%%%%%%%%%%%%%%%%%%%%%%%%%%%%%%%%%%%
\paragraph{v1.5:} 2017/05/21

\begin{itemize}
\item
more complete structuring introduced
\item
|\childdocof| introduced
\item
|\childdoc| renamed to |\childdocmain|
\item
|\childredirect| renamed to |\childdocforward| and |\childdocforwardprefix|
and functionality expanded
\end{itemize}

%%%%%%%%%%%%%%%%%%%%%%%%%%%%%%%%%%%%%%%%
\paragraph{v1.0:} 2017/04/27

\begin{itemize}
\item
manual and install package
\item
first version published on CTAN
\end{itemize}

%%%%%%%%%%%%%%%%%%%%%%%%%%%%%%%%%%%%%%%%
\paragraph{v0.6:} 2017/04/26

\begin{itemize}
\item
redirection mechanism added
\end{itemize}

%%%%%%%%%%%%%%%%%%%%%%%%%%%%%%%%%%%%%%%%
\paragraph{v0.5:} 2017/04/26

\begin{itemize}
\item
functionality in definition file
\end{itemize}


%%%%%%%%%%%%%%%%%%%%%%%%%%%%%%%%%%%%%%%%%%%%%%%%%%%%%%%%%%%%%%%%%%%%%%%%%%%%%%%%
%%%%%%%%%%%%%%%%%%%%%%%%%%%%%%%%%%%%%%%%%%%%%%%%%%%%%%%%%%%%%%%%%%%%%%%%%%%%%%%%
%%%%%%%%%%%%%%%%%%%%%%%%%%%%%%%%%%%%%%%%%%%%%%%%%%%%%%%%%%%%%%%%%%%%%%%%%%%%%%%%
\appendix

\settowidth\MacroIndent{\rmfamily\scriptsize 000\ }

 \DocInput{childdoc.dtx}

\end{document}
%</driver>
% \fi
%
% %%%%%%%%%%%%%%%%%%%%%%%%%%%%%%%%%%%%%%%%%%%%%%%%%%%%%%%%%%%%%%%%%%%%%%%%%%%%%%
% %%%%%%%%%%%%%%%%%%%%%%%%%%%%%%%%%%%%%%%%%%%%%%%%%%%%%%%%%%%%%%%%%%%%%%%%%%%%%%
% \section{Sample}
%\iffalse
%<*samplemain>
%\fi
%
% The following presents a sample document
% with two chapters, two parts, a title page,
% a compile flag as well as three forwarding files to set the flag.
% It consists of eight |.tex| files:
% \begin{center}
% \begin{tabular}{ll}
% |cdocsamp.tex|&main file\\
% |cdocsch1.tex|&include file for chapter 1\\
% |cdocsch2.tex|&include file for chapter 2\\
% |cdocspt3.tex|&include file for part 3\\
% |cdocspt4.tex|&include file for part 4\\
% |cdocsdrf.tex|&forwarding file for main file in draft mode\\
% |cdocsfi1.tex|&forwarding file for final version of chapter 1\\
% |cdocsfi2.tex|&forwarding file for final version of chapter 2\\
% \end{tabular}
% \end{center}
% Each of the eight files can be compiled directly by the \LaTeX{} compiler.
%
% %%%%%%%%%%%%%%%%%%%%%%%%%%%%%%%%%%%%%%
% \paragraph{Main File.}
%
% The main file is called |cdocsamp.tex|.
%
% Load the \textsf{childdoc} definitions and
% declare the filename for the main document:
%    \begin{macrocode}
\input{childdoc.def}
\childdocmain{}
%    \end{macrocode}

% Optional override for |\version| flag:
%    \begin{macrocode}
%%\ifchilddoc\else\providecommand{\version}{draft}\fi
%    \end{macrocode}

% Define the default values for the |\version| flag
% (|final| for the main file and |draft| for childs):
%    \begin{macrocode}
\ifchilddoc
\providecommand{\version}{draft}
\else
\providecommand{\version}{final}
\fi
%    \end{macrocode}

% Load the standard document class:
%    \begin{macrocode}
\documentclass[12pt]{article}
%    \end{macrocode}

% Start the document body:
%    \begin{macrocode}
\begin{document}
%    \end{macrocode}

% Declare a title page.
% Print title, part of document being processed and version flag:
%    \begin{macrocode}
\addtocounter{page}{-1}
\begin{center}
{\LARGE\bfseries{}childdoc example\par}
\vspace{1cm}
\ifchilddoc
\ifchilddocmanual part\else chapter\fi:
`\childdocname' of `\childdocjob'\par
\else
main document: `\childdocjob'\par
\fi
version: \version\par
\end{center}
\newpage
%    \end{macrocode}

% Manually include selected file,
% otherwise process as usual:
%    \begin{macrocode}
\ifchilddocmanual
\section*{part `\childdocname'}
\input{\childdocname}
\else
%    \end{macrocode}

% Include the two chapters:
%    \begin{macrocode}
\include{cdocsch1}
\include{cdocsch2}
%    \end{macrocode}

% Include the two parts unless only chapters should be displayed:
%    \begin{macrocode}
\ifchilddoc\else
\section{part three}
\input{cdocspt3}
\section{part four}
\input{cdocspt4}
\fi
%    \end{macrocode}

% Process as usual until here:
%    \begin{macrocode}
\fi
%    \end{macrocode}

% End of document body:
%    \begin{macrocode}
\end{document}
%    \end{macrocode}
%\iffalse
%</samplemain>
%\fi
%
% %%%%%%%%%%%%%%%%%%%%%%%%%%%%%%%%%%%%%%
% \paragraph{Chapter Include Files.}
%
% The include files are called |cdocsch1.tex| and |cdocsch2.tex|.
%
%\iffalse
%<*samplechap1|samplechap2>
%\fi

% Optional override for |\version| flag:
%    \begin{macrocode}
%%\providecommand{\version}{final}
%    \end{macrocode}

% Include the main document:
%    \begin{macrocode}
\input{childdoc.def}
\childdocof{cdocsamp}
%    \end{macrocode}

%\iffalse
%</samplechap1|samplechap2>
%\fi
%
%\iffalse
%<*samplechap1>
%\fi
% Some text for chapter 1:
%    \begin{macrocode}
\section{one}
some text in chapter one
%    \end{macrocode}

%\iffalse
%</samplechap1>
%\fi
% Some text for chapter 2:
%\iffalse
%<*samplechap2>
%\fi
%    \begin{macrocode}
\section{two}
more text in chapter two
%    \end{macrocode}

%\iffalse
%</samplechap2>
%\fi
%
% %%%%%%%%%%%%%%%%%%%%%%%%%%%%%%%%%%%%%%
% \paragraph{Part Include Files.}
%
% The include files are called |cdocspt3.tex| and |cdocspt4.tex|.
%
%\iffalse
%<*samplepart3|samplepart4>
%\fi

% Optional override for |\version| flag:
%    \begin{macrocode}
%%\providecommand{\version}{final}
%    \end{macrocode}

% Include the main document:
%    \begin{macrocode}
\input{childdoc.def}
\childdocby{cdocsamp}
%    \end{macrocode}

%\iffalse
%</samplepart3|samplepart4>
%\fi
%
%\iffalse
%<*samplepart3>
%\fi
% Some text for part 3:
%    \begin{macrocode}
some text in part three
%    \end{macrocode}

%\iffalse
%</samplepart3>
%\fi
% Some text for part 4:
%\iffalse
%<*samplepart4>
%\fi
%    \begin{macrocode}
more text in part four
%    \end{macrocode}

%\iffalse
%</samplepart4>
%\fi
%
% %%%%%%%%%%%%%%%%%%%%%%%%%%%%%%%%%%%%%%
% \paragraph{Forwarding for a Complete Draft.}
%
% The following forwarding file |cdocsdrf.tex|
% compiles the main document in draft mode:
%\iffalse
%<*sampledraft>
%\fi
%    \begin{macrocode}
\def\version{draft}
\input{childdoc.def}
\childdocforward{cdocsamp}
%    \end{macrocode}

%\iffalse
%</sampledraft>
%\fi
%
% %%%%%%%%%%%%%%%%%%%%%%%%%%%%%%%%%%%%%%
% \paragraph{Forwarding for Final Version of the Chapters.}
%
% The following forwarding files |cdocsfn1.tex| and |cdocsfn2.tex|
% (with identical content)
% compile the final versions of the child documents
% |cdocsch1.tex| and |cdocsch2.tex|, respectively:
%\iffalse
%<*samplefinal>
%\fi
%    \begin{macrocode}
\def\version{final}
\input{childdoc.def}
\childdocforwardprefix[cdocsamp]{cdocsfn}{cdocsch}
%    \end{macrocode}

%\iffalse
%</samplefinal>
%\fi
%
% %%%%%%%%%%%%%%%%%%%%%%%%%%%%%%%%%%%%%%
% \paragraph{Command Line Processing.}
%
% The following three command lines generate the output files
% |cdocscld|, |cdocscl1| and |cdocscl2|
% which should be identical to
% |cdocsdrf|, |cdocsch1| and |cdocsfn2|, respectively:
% \begin{center}
% \begin{tabular}{l}
% |latex -jobname cdocscld \|\\
% |  "\def\version{draft}\input{childdoc.def}\childdocforward{cdocsamp}"|\\
% |latex -jobname cdocscl1 \|\\
% |  "\input{childdoc.def}\childdocforward[cdocsamp]{cdocsch1}"|\\
% |latex -jobname cdocscl2 \|\\
% |  "\def\version{final}\input{childdoc.def}\childdocforward{cdocsch2}"|
% \end{tabular}
% \end{center}
% Note that the trailing backslash on each first line
% merely continues the input to the second line
% (for convenient cut ant paste).
% Furthermore, the command |latex| can be replaced by any
% of its alternative versions such as |pdflatex|.
%
% %%%%%%%%%%%%%%%%%%%%%%%%%%%%%%%%%%%%%%%%%%%%%%%%%%%%%%%%%%%%%%%%%%%%%%%%%%%%%%
% %%%%%%%%%%%%%%%%%%%%%%%%%%%%%%%%%%%%%%%%%%%%%%%%%%%%%%%%%%%%%%%%%%%%%%%%%%%%%%
% \section{Implementation}
%\iffalse
%<*package>
%\fi
%
% This section describes the definitions file |childdoc.def|.

% The definitions cannot be loaded using |\usepackage| or |\RequirePackage|
% which has a mechanism to prevent loading a style file more than once.
% When loading the definitions by means of |\input|
% multiple instances have to be prevented manually:
%\iffalse
%This code needs to be before the `\ProvidesFile' directive
%which is defined at the beginning of this file.
%Therefore it is also placed there and commented out here.
%</package>
%<*discard>
%\fi
%    \begin{macrocode}
\ifdefined\childdocmain\endinput\fi
%    \end{macrocode}
%\iffalse
%</discard>
%<*package>
%\fi
%
% \macro{\ifchilddoc}
% \macro{\ifchilddocmanual}
% The conditional |\ifchilddoc| tells whether a
% child (true) or main (false) document is being compiled.
% The conditional |\ifchilddocmanual| tells whether
% the |\includeonly| mechanism is used (false) or
% the selection of child files must be performed manually (true).
% The definitions initialise to false:
%    \begin{macrocode}
\newif\ifchilddoc
\newif\ifchilddocmanual
%    \end{macrocode}

% \macro{\childdocname}
% \macro{\childdocjob}
% The macro |\childdocname| stores the name of the main document
% to be compiled. The macro |\childdocjob| stores the name of
% the document on which the \LaTeX{} compiler was originally invoked.
% The content of |\jobname| cannot be compared
% to filenames specified in the source due to different catcodes.
% The following code rescans |\jobname|, stores the result
% in |\childdocname| and saves a copy in |\childdocjob|:
%    \begin{macrocode}
\edef\childdocname{\scantokens\expandafter{\jobname\noexpand}}
\let\childdocjob\childdocname
%    \end{macrocode}

% \macro{\childdocdisable}
% The macro |\childdocdisable| prevents the main file
% from being processed more than once.
% At this stage, the main document command |\childdocmain|
% is assumed to be called once again where it should do nothing.
% Any subsequent call to it should prevent
% a secondary processing of the main document
% It overwrites the forwarding commands
% |\childdocof| and |\childdocforward|
% with empty macros to prevent further inclusions of the main document:
%    \begin{macrocode}
\newcommand{\childdocdisable}
{
  \renewcommand{\childdocmain}[1]{\renewcommand{\childdocmain}[1]{\endinput}}
  \renewcommand{\childdocof}[1]{}
  \renewcommand{\childdocby}[2][]{}
  \renewcommand{\childdocforward}[2][]{}
  \renewcommand{\childdocdisable}{}
}
%    \end{macrocode}

% \macro{\childdocmain}
% The macro |\childdocmain| is to be called at the top of the main file
% with nothing or the main filename (without extension) as argument.
% First, it breaks loops.
% If the argument is not empty and does not match |\childdocname|
% (which is set by the first inclusion of |childdoc.def|),
% |\ifchilddoc| is set to true, |\includeonly| is applied to the child file
% and |\jobname| is set to the main file
% (for proper handling of |.aux| files):
%    \begin{macrocode}
\newcommand{\childdocmain}[1]
{
  \childdocdisable\childdocmain{}
  \if?#1?\else
    \begingroup
      \def\childdoctmp{#1}
      \ifx\childdoctmp\childdocname
        \def\childdoctmp{}
      \else
        \def\childdoctmp
        {
          \childdoctrue
          \includeonly{\childdocname}
          \def\childdocjob{#1}
          \def\jobname{#1}
        }
      \fi
      \expandafter
    \endgroup
    \childdoctmp
  \fi
}
%    \end{macrocode}

% \macro{\childdocof}
% The command |\childdocof| redirects
% compilation to the main file |#1|.
%    \begin{macrocode}
\newcommand{\childdocof}[1]
{
  \childdocdisable
  \childdoctrue
  \includeonly{\childdocname}
  \def\jobname{#1}
  \def\childdocjob{#1}
  \input{#1}
}
%    \end{macrocode}

% \macro{\childdocby}
% The command |\childdocby| ....
%    \begin{macrocode}
\newcommand{\childdocby}[2][]
{
  \childdocdisable
  \childdoctrue
  \childdocmanualtrue
  \if?#1?\else
    \def\jobname{#2}
  \fi
  \def\childdocjob{#2}
  \input{#2}
  \endinput
}
%    \end{macrocode}

% \macro{\childdocforward}
% The command |\childdocforward| redirects
% compilation to the main file or
% (if the optional argument is given) a child file.
% Parameters are set as if the main file
% or a child file starting with |\childdocof| was compiled.
% Then compilation is handed over to the main file:
%    \begin{macrocode}
\newcommand{\childdocforward}[2][]
{
  \begingroup
    \if?#1?
      \def\childdoctmp
      {
        \def\childdocname{#2}
        \def\childdocjob{#2}
        \def\jobname{#2}
        \input{#2}
        \endinput
      }
    \else
      \def\childdoctmp
      {
        \childdocdisable
        \def\childdocname{#2}
        \childdoctrue
        \includeonly{#2}
        \def\childdocjob{#1}
        \def\jobname{#1}
        \input{#1}
        \endinput
      }
    \fi
    \expandafter
  \endgroup
  \childdoctmp
}
%    \end{macrocode}

% \macro{\childdocforwardprefix}
% The command |\childdocforwardprefix| redirects
% compilation to the main or a child file by means of a pattern.
% The prefix |#1| in the current filename is replaced by |#2|
% and the suffix of the current filename is kept
% (it is assumed that the filename does not contain the substring `|~~~|'
% which is used as a delimiter).
% Compilation is handed over to the new file by |\childdocforward|:
%    \begin{macrocode}
\newcommand{\childdocforwardprefix}[3][]
{
  \begingroup
    \def\childdocextract #2##1~~~{\def\childdoctmp{\childdocforward[#1]{#3##1}}}
    \expandafter\childdocextract\childdocname~~~
    \expandafter
  \endgroup
  \childdoctmp
}
%    \end{macrocode}

% \macro{\childdoc}
% The deprecated macro |\childdoc| is a legacy version of |\childdocmain|:
%    \begin{macrocode}
\newcommand{\childdoc}{\childdocmain}
%    \end{macrocode}

% \macro{\childdocredirect}
% The deprecated macro |\childdocredirect| is a legacy version
% of |\childdocforward| and |\childdocforwardprefix|:
%    \begin{macrocode}
\newcommand{\childdocredirect}[2][]
{
  \begingroup
    \if?#1?
      \def\childdoctmp{\childdocforward{#2}}
    \else
      \def\childdoctmp{\childdocforwardprefix{#1}{#2}}
    \fi
    \expandafter
  \endgroup
  \childdoctmp
}
%    \end{macrocode}

%\iffalse
%</package>
%\fi
%
\endinput

\childdocof{cdocsamp}
%    \end{macrocode}

%\iffalse
%</samplechap1|samplechap2>
%\fi
%
%\iffalse
%<*samplechap1>
%\fi
% Some text for chapter 1:
%    \begin{macrocode}
\section{one}
some text in chapter one
%    \end{macrocode}

%\iffalse
%</samplechap1>
%\fi
% Some text for chapter 2:
%\iffalse
%<*samplechap2>
%\fi
%    \begin{macrocode}
\section{two}
more text in chapter two
%    \end{macrocode}

%\iffalse
%</samplechap2>
%\fi
%
% %%%%%%%%%%%%%%%%%%%%%%%%%%%%%%%%%%%%%%
% \paragraph{Part Include Files.}
%
% The include files are called |cdocspt3.tex| and |cdocspt4.tex|.
%
%\iffalse
%<*samplepart3|samplepart4>
%\fi

% Optional override for |\version| flag:
%    \begin{macrocode}
%%\providecommand{\version}{final}
%    \end{macrocode}

% Include the main document:
%    \begin{macrocode}
% \iffalse
%
% childdoc.dtx Copyright (C) 2017-2018 Niklas Beisert
%
% This work may be distributed and/or modified under the
% conditions of the LaTeX Project Public License, either version 1.3
% of this license or (at your option) any later version.
% The latest version of this license is in
%   http://www.latex-project.org/lppl.txt
% and version 1.3 or later is part of all distributions of LaTeX
% version 2005/12/01 or later.
%
% This work has the LPPL maintenance status `maintained'.
%
% The Current Maintainer of this work is Niklas Beisert.
%
% This work consists of the files childdoc.dtx and childdoc.ins
% and the derived files childdoc.def and cdocsamp.tex with
% cdocsch1.tex, cdocsch2.tex, cdocsdrf.tex, cdocsfn1.tex, cdocsfn2.tex.
%
%<package>\ifdefined\childdocmain\endinput\fi
%<package>\ProvidesFile{childdoc.def}[2018/12/30 v2.0 child document driver]
%<samplemain>\ProvidesFile{cdocsamp.tex}[2018/12/30 v2.0 sample for childdoc]
%<*driver>
%\ProvidesFile{childdoc.drv}[2018/12/30 v2.0 childdoc reference manual file]
\PassOptionsToClass{10pt,a4paper}{article}
\documentclass{ltxdoc}

\usepackage[margin=35mm]{geometry}
\usepackage{hyperref}
\usepackage{hyperxmp}
\usepackage[usenames]{color}

\hypersetup{colorlinks=true}
\hypersetup{pdfstartview=FitH}
\hypersetup{pdfpagemode=UseNone}
\hypersetup{pdfsource={}}
\hypersetup{pdflang={en-UK}}
\hypersetup{pdfcopyright={Copyright 2017-2018 Niklas Beisert.
  This work may be distributed and/or modified under the
  conditions of the LaTeX Project Public License, either version 1.3
  of this license or (at your option) any later version.}}
\hypersetup{pdflicenseurl={http://www.latex-project.org/lppl.txt}}
\hypersetup{pdfcontactaddress={ETH Zurich, ITP, HIT K,
  Wolfgang-Pauli-Strasse 27}}
\hypersetup{pdfcontactpostcode={8093}}
\hypersetup{pdfcontactcity={Zurich}}
\hypersetup{pdfcontactcountry={Switzerland}}
\hypersetup{pdfcontactemail={nbeisert@itp.phys.ethz.ch}}
\hypersetup{pdfcontacturl={http://people.phys.ethz.ch/\xmptilde nbeisert/}}

\newcommand{\secref}[1]{\hyperref[#1]{section \ref*{#1}}}

\parskip1ex
\parindent0pt
\let\olditemize\itemize
\def\itemize{\olditemize\parskip0pt}

\begin{document}

\title{The \textsf{childdoc} Package}
\hypersetup{pdftitle={The childdoc Package}}
\author{Niklas Beisert\\[2ex]
  Institut f\"ur Theoretische Physik\\
  Eidgen\"ossische Technische Hochschule Z\"urich\\
  Wolfgang-Pauli-Strasse 27, 8093 Z\"urich, Switzerland\\[1ex]
  \href{mailto:nbeisert@itp.phys.ethz.ch}
  {\texttt{nbeisert@itp.phys.ethz.ch}}}
\hypersetup{pdfauthor={Niklas Beisert}}
\hypersetup{pdfsubject={Manual for the LaTeX2e Package childdoc}}
\date{30 December 2018, \textsf{v2.0}}
\maketitle

\begin{abstract}\noindent
\textsf{childdoc} is a \LaTeXe{} package
that enables the direct compilation
of document sections included by |\include|
to individual files.
\end{abstract}

\begingroup
\parskip0ex
\tableofcontents
\endgroup

%%%%%%%%%%%%%%%%%%%%%%%%%%%%%%%%%%%%%%%%%%%%%%%%%%%%%%%%%%%%%%%%%%%%%%%%%%%%%%%%
%%%%%%%%%%%%%%%%%%%%%%%%%%%%%%%%%%%%%%%%%%%%%%%%%%%%%%%%%%%%%%%%%%%%%%%%%%%%%%%%
\section{Introduction}

\LaTeX{} provides a mechanism to structure a large document (such as a book)
into a main file and several child files (containing the chapters)
using the |\include| command.
This mechanism is beneficial for documents
which span hundreds of pages in order to
make the source file(s) more manageable.
Moreover, compilation can be restricted to
selected child files by means of the |\includeonly| command.
The latter feature can be used to reduce the compilation time while editing
(this was significantly more useful in the earlier days of \LaTeX{})
or to generate a smaller document which is easier to navigate.
Another application of |\includeonly| is to generate
documents consisting of selected parts of the complete document.

However, there are a few drawbacks of the plain |\include| mechanism:
\begin{itemize}
\item
The child files cannot be compiled on their own,
they can only be compiled via the main file.
A naive editing environment
(such as a text editor with an option
to have the current file processed by \LaTeX)
may require one to switch to the main file before compiling;
attempting to compile the child file produces errors.
\item
The main file must be modified (each time)
to adjust the |\includeonly| command
to the present needs. This easily leaves the main file in a messy state.
\item
The generated document will always carry the filename
of the main document. This is inconvenient if
several child files are to be compiled and
to be kept for distribution.
\end{itemize}

The present package provides a simple interface
to make child files individually compilable by \LaTeX{}.
Compiling a child file then has the same effect as compiling
the main file with an |\includeonly| command
to select the appropriate child.
Moreover the generated document will carry the name of the child
rather than the main file.
This resolves all three above issues.

This feature is meant to make the editing of books,
thesis documents and lecture notes somewhat more convenient.
However, the package can also be used efficiently for
composing a series of documents (such as exercise sheets)
which are typically distributed individually.
It then assists the author in generating the individual documents
(potentially in different versions)
as well as a document containing the collected series.
Another application is in developing style files
or other kinds of included material
where compilation of the style file could redirect
to a sample or test file.

%%%%%%%%%%%%%%%%%%%%%%%%%%%%%%%%%%%%%%%%%%%%%%%%%%%%%%%%%%%%%%%%%%%%%%%%%%%%%%%%
%%%%%%%%%%%%%%%%%%%%%%%%%%%%%%%%%%%%%%%%%%%%%%%%%%%%%%%%%%%%%%%%%%%%%%%%%%%%%%%%
\section{Usage}

First of all, the package \textsf{childdoc} is \emph{not} a standard
\LaTeXe{} |.sty| style file! Therefore it needs to be invoked in
a non-standard way.

%%%%%%%%%%%%%%%%%%%%%%%%%%%%%%%%%%%%%%%%%%%%%%%%%%%%%%%%%%%%%%%%%%%%%%%%%%%%%%%%
\subsection{Included Files}
\label{sec:include}

%%%%%%%%%%%%%%%%%%%%%%%%%%%%%%%%%%%%%%%%
\DescribeMacro{\childdocmain}
To use the package, add the commands
\begin{center}
\begin{tabular}{l}
|\input{childdoc.def}|\\
|\childdocmain{}|\\
\end{tabular}
\end{center}
at the very top of the main \LaTeX{} file,
in particular \emph{before} the |\documentclass| statement!
The argument of |\childdocmain| should be left empty
(but it must be present).

%%%%%%%%%%%%%%%%%%%%%%%%%%%%%%%%%%%%%%%%
\DescribeMacro{\childdocof}
Furthermore, add the commands
\begin{center}
\begin{tabular}{l}
|\input{childdoc.def}|\\
|\childdocof{|\textit{main}|}|\\
\end{tabular}
\end{center}
at the top of every child file \textit{child}
which is included by |\include{|\textit{child}|}|
from within the main file
(or at least for those files to be compiled individually).
The argument \textit{main} must be the filename of the main file.

There are a couple of
considerations in setting up the main and child documents:

%%%%%%%%%%%%%%%%%%%%%%%%%%%%%%%%%%%%%%%%
\paragraph{Restrictions.}

Please note the following restrictions:
\begin{itemize}
\item
|\childdocmain| must be called with one argument \textit{main}
to ensure compatibility with earlier version of the package.
It must either be empty (|\childdocmain{}|)
or precisely match the filename of the main file in which it is specified.
See \secref{sec:detection} for further information.
\item
The filename \textit{main} must be specified without the |.tex| extension.
\item
The filename \textit{main} is case sensitive
(even in case-insensitive file systems)
due to internal string comparison.
\item
The argument \textit{main} should be fully expanded, it cannot be a macro.
\item
Subdirectories and special characters should be avoided in filenames.
\item
The command |\childdocmain{|\textit{main}|}| must be followed by a whitespace.
It should not be followed immediately by another command
or by a comment mark `|%|'.
This is because the \TeX{} parser reads the token immediately following
the argument of |\childdocmain| and puts it
at the beginning of every child section;
however, a white\-space is ignored.
\end{itemize}

%%%%%%%%%%%%%%%%%%%%%%%%%%%%%%%%%%%%%%%%
\paragraph{Content of Main File.}

It is advisable to place all content in the child files included by |\include|.
Any output contained in the main file will appear in all child documents
unless suppressed manually;
it cannot be suppressed automatically by the |\includeonly| directive
and thus should normally be avoided.
A method to include some content in the main file
by means of conditional processing is described in \secref{sec:conditional}.

%%%%%%%%%%%%%%%%%%%%%%%%%%%%%%%%%%%%%%%%
\paragraph{Page Numbering.}

When only a part of the document is compiled,
the appropriate numbering of pages
(as well as other status parameters)
is determined from the |.aux| files.
The latter contain information from previous passes.
However this information needs to propagate through
all intermediate child documents.
Therefore the page numbering in child documents may well
be inconsistent until the complete document is compiled at least once.

A useful (if unconventional) way to always ensure a consistent
page numbering is to restart the numbering in each child document
and denote the pages by `\textit{child}|.|\textit{page}'
where \textit{child} represents the chapter/section number of the child file.
This can be achieved by the command
|\numberwithin{page}{|\textit{child}|}|
of the \textsf{amsmath} package
where \textit{child} can be |chapter| or |section|
depending on the chosen structuring.
Alternatively, one can modify the macro |\thepage| appropriately
and reset the counter |page| at the start of each child file.

%%%%%%%%%%%%%%%%%%%%%%%%%%%%%%%%%%%%%%%%%%%%%%%%%%%%%%%%%%%%%%%%%%%%%%%%%%%%%%%%
\subsection{Conditional Processing}
\label{sec:conditional}

The package provides a mechanism to compile different versions
of a document. To customise the versions further some conditional processing
can come in handy to distinguish which version is being compiled.
The package provides two macros to describe the compilation context:

%%%%%%%%%%%%%%%%%%%%%%%%%%%%%%%%%%%%%%%%
\DescribeMacro{\ifchilddoc}
The conditional |\ifchilddoc| distinguishes between the compilation of
child documents and the main document:
%
\begin{center}
|\ifchilddoc |\textit{child-code}| |[|\||else |\textit{main-code}]| \||fi|
\end{center}

%%%%%%%%%%%%%%%%%%%%%%%%%%%%%%%%%%%%%%%%
\DescribeMacro{\childdocname}
\DescribeMacro{\childdocjob}
The macro |\childdocname| contains the filename (without extension)
of the main or child file being processed.
Note that |\childdocjob| will always contain the name of the main file.

%%%%%%%%%%%%%%%%%%%%%%%%%%%%%%%%%%%%%%%%
\paragraph{Title Page.}

Conditional processing can be used to include a title or banner page
in the main document when proper precautions are taken.
Importantly, the code in the main file should ensure that the page counter
(as well as other status parameters which are stored in the |.aux| files)
takes the same value after the conditional processing.
Otherwise the page numbers may take divergent values
depending on which part is compiled.

For example, a title page could be declared by:
%
\begin{center}
\begin{tabular}{l}
|\ifchilddoc\||else|\\
|\addtocounter{page}{-1}|\\
\textit{code for title page}\\
|\newpage|\\
|\||fi|
\end{tabular}
\end{center}
%
A banner page for the child documents can be generated by:
%
\begin{center}
\begin{tabular}{l}
|\ifchilddoc|\\
|\addtocounter{page}{-1}|\\
\textit{code for banner page}\\
|\newpage|\\
|\||fi|
\end{tabular}
\end{center}
%
Here one could write a message such as:
\begin{center}
|This is the part \childdocname{} of \childdocjob{}.|
\end{center}

%%%%%%%%%%%%%%%%%%%%%%%%%%%%%%%%%%%%%%%%%%%%%%%%%%%%%%%%%%%%%%%%%%%%%%%%%%%%%%%%
\subsection{Flags}
\label{sec:flags}

The package makes it easy to generate different versions
of the main or child documents.
To this end compilation flags can be defined
and assigned different default values.
They will be particularly useful in conjunction
with the forwarding mechanism described in \secref{sec:forward}.

For example, it may be useful to have a flag |\version|
which can be set to |draft| or |final|.
The document source will contain some conditional code
depending on the value of |\version|.
Suppose further, the flag should default to |final| for the main file
and to |draft| for child files
which is a natural assignment for editing the document.
This is achieved by placing the following code
in the preamble of the main document
(below the |\childdocmain| directive):
%
\begin{center}
\begin{tabular}{l}
|\ifchilddoc|\\
|\providecommand{\version}{draft}|\\
|\||else|\\
|\providecommand{\version}{final}|\\
|\||fi|
\end{tabular}
\end{center}
%
The definition by |\providecommand| makes sure
that previous definitions are not overwritten.
Further statements |\providecommand{\version}{...}|
can thus be added before the above code to override it.

For the main file, one might add a line
(between |\childdocmain| and the above block)
%
\begin{center}
|%\ifchilddoc\||else\providecommand{\version}{draft}\||fi|
\end{center}
%
which can be uncommented to produce a draft version.
Likewise one can add a line to the very top of a child file
(above the |\childdocof{|\textit{main}|}| directive)
%
\begin{center}
|%\providecommand{\version}{final}|
\end{center}
%
which can be uncommented to produce the final version of this child document.

%%%%%%%%%%%%%%%%%%%%%%%%%%%%%%%%%%%%%%%%%%%%%%%%%%%%%%%%%%%%%%%%%%%%%%%%%%%%%%%%
\subsection{Forwarding}
\label{sec:forward}

Different versions of the main or child documents
using compilation flags as described in \secref{sec:flags}
can be (permanently) stored in different files
for convenient compilation, viewing and distribution.
To this end, the package defines a command
to pass on compilation to a different file:

%%%%%%%%%%%%%%%%%%%%%%%%%%%%%%%%%%%%%%%%
\DescribeMacro{\childdocforward}
The command |\childdocforward| redirects processing to
another source file:
%
\begin{center}
\begin{tabular}{l}
|\input{childdoc.def}|\\
|\childdocforward[|\textit{main}|]{|\textit{dest}|}|\\
\end{tabular}
\end{center}
%
The argument \textit{dest} is the destination file
(without extension).
It should be the main file or one of the child files.
Note that further \textsf{childdoc} directives
such as |\childdocof| and |\childdocforward|
in the indicated file will be processed in this form.
The optional argument \textit{main}
passes on directly to the main file \textit{main}
while pretending to compile the child \textit{dest}.
This form behaves as if \textit{dest}
issues |\childdocof{|\textit{main}|}| right away,
and no further \textsf{childdoc} directives will be processed.

%%%%%%%%%%%%%%%%%%%%%%%%%%%%%%%%%%%%%%%%
\DescribeMacro{\...prefix}
In the alternative form |\childdocforwardprefix|,
%
\begin{center}
\begin{tabular}{l}
|\input{childdoc.def}|\\
|\childdocforwardprefix[|\textit{main}|]{|\textit{prefix}|}{|\textit{dest}|}|
\end{tabular}
\end{center}
%
the destination file is determined by a pattern
depending on the current file:
To make this work, the current file must be called
`{\textit{prefix}\hspace{0.2em}\textit{suffix}}'
with \textit{prefix} matching precisely the argument.
Processing is then passed on to the file
`{\textit{dest}\hspace{0.2em}\textit{suffix}}'.
Surely, the same effect is achieved by
directly specifying the
argument `{\textit{dest}\hspace{0.2em}\textit{suffix}}'
in the first form.
However, that requires to set up a different file
for each child. With the alternative form of the command
all these files can have exactly the same content
which simplifies setting them up and maintaining them.

For example, the following file |draft.tex|
with a compilation flag |\version| as described in \secref{sec:flags}
compiles the main document as a draft:
%
\begin{center}
\begin{tabular}{l}
|\def\version{draft}|\\
|\input{childdoc.def}|\\
|\childdocforward{|\textit{main}|}|
\end{tabular}
\end{center}
%
Likewise, the following files |final|\textit{nn}|.tex|
compile the final version of the child document
|child|\textit{nn}|.tex|:
%
\begin{center}
\begin{tabular}{l}
|\def\version{final}|\\
|\input{childdoc.def}|\\
|\childdocforwardprefix{final}{child}|
\end{tabular}
\end{center}
%

Note that when several versions of a main file and/or of each child file
are to be generated, it may be convenient to set up a |Makefile| or
shell script to automatise the process.

%%%%%%%%%%%%%%%%%%%%%%%%%%%%%%%%%%%%%%%%%%%%%%%%%%%%%%%%%%%%%%%%%%%%%%%%%%%%%%%%
\subsection{Command Line Processing}
\label{sec:commandline}

The effect of redirection files can also be achieved by invoking
the \LaTeX{} compiler with a more elaborate command line.
Most conveniently this should be done as part
of a shell script or a |Makefile|.

When using \textsf{childdoc} in the main file, the following
command lines effectively perform a redirection
(note that depending on the shell being used,
backslashes may have to be doubled: `|\|' $\to$ `|\\|'):
%
\begin{center}
|... -jobname "|\textit{target}|" |\\|"|[\textit{flags}]%
|\input{childdoc.def}\childdocforward[|\textit{main}|]{|\textit{dest}|}"|
\end{center}
%
Here \textit{target} is the name of the output file,
\textit{main} is the name of the main file
and \textit{dest} is the name of the main or child file to be processed
(all filenames without extensions).
The optional argument \textit{main} can be omitted
if \textit{main} matches \textit{dest}.
Optionally, compilation \textit{flags} can be defined via |\def| commands.
This command line makes the \TeX{} engine believe
it is compiling the file \textit{target}
whose content is specified as the latter parameter.
The provided code then forwards the processing to
\textit{main} or \textit{dest} as described in \secref{sec:forward}.

%%%%%%%%%%%%%%%%%%%%%%%%%%%%%%%%%%%%%%%%%%%%%%%%%%%%%%%%%%%%%%%%%%%%%%%%%%%%%%%%
\subsection{Include by Input}
\label{sec:input}

Including child documents by |\include| has some restrictions by design.
Most notably, the content of a child document always occupies
its own set of pages; pages cannot be shared between child documents.
Usually, this behaviour makes perfect sense
because each child document contain an essential part of the document.
However, in some situations it may be desirable to compose
a document from a collection of parts
without having mandatory page breaks between then.
For this case, the package
provides a mechanism to include parts
by |\input| which can also be processed individually.
However, by construction this mechanism
requires manual handling of the content to be output.

%%%%%%%%%%%%%%%%%%%%%%%%%%%%%%%%%%%%%%%%
\DescribeMacro{\ifchilddocmanual}
The main file should be prepared as usual, see \secref{sec:include}.
However, the document body must make a distinction
between processing of an individual part and of the main document, e.g.:
%
\begin{center}
\begin{tabular}{l}
|\ifchilddocmanual|\\
|\input{\childdocname}|\\
|\||else|\\
\textit{document body with }|\input{|\textit{part}|}|\\
|\||fi|
\end{tabular}
\end{center}
%
The conditional |\ifchilddocmanual| is true whenever
a part to be included by |\input| is being compiled,
and the name of the part is stored in |\childdocname|.

%%%%%%%%%%%%%%%%%%%%%%%%%%%%%%%%%%%%%%%%
\DescribeMacro{\childdocby}
Each part to be included by |\input| should start with:
%
\begin{center}
\begin{tabular}{l}
|\input{childdoc.def}|\\
|\childdocby{|\textit{main}|}|\\
\end{tabular}
\end{center}
%
The directive |\childdocby| is similar to |\childdocof|
described in \secref{sec:include},
but the subsequent selection of content must be done manually.
To that end, both |\ifchilddoc| and |\ifchilddocmanual|
will be true upon processing of a part,
and the name of the part is stored in |\childdocname|.
Note that |\jobname| will be set to the filename of the current part
so that each part receives an individual |.aux| file
that does not interfere with the |.aux| file(s) of the main document.
This behaviour can be altered by the alternative form
|\childdocby[*]{|\textit{main}|}| (with a non-empty optional argument)
which uses the |.aux| file of the main document
by setting |\jobname| to \textit{main}.

%%%%%%%%%%%%%%%%%%%%%%%%%%%%%%%%%%%%%%%%%%%%%%%%%%%%%%%%%%%%%%%%%%%%%%%%%%%%%%%%
\subsection{Driver Development}
\label{sec:driver}

The \textsf{childdoc} mechanism can also be use for the development
of definition files such as \LaTeX{} styles or classes.
This case differs from the above setup with multiple parts
included by |\include| in that no |\includeonly| should be invoked.
This can be achieved by starting the include file
(before |\ProvidesPackage|) with:
%
\begin{center}
\begin{tabular}{l}
|\input{childdoc.def}|\\
|\childdocforward{|\textit{main}|}|\\
\end{tabular}
\end{center}
%
or alternatively with:
%
\begin{center}
\begin{tabular}{l}
|\input{childdoc.def}|\\
|\childdocby{|\textit{main}|}|\\
\end{tabular}
\end{center}
%
Both forms have slightly different effects as described above.
The main file is prepared as usual, see \secref{sec:include}.

%%%%%%%%%%%%%%%%%%%%%%%%%%%%%%%%%%%%%%%%%%%%%%%%%%%%%%%%%%%%%%%%%%%%%%%%%%%%%%%%
\subsection{Legacy Detection}
\label{sec:detection}

The directive |\childdocmain| in the main file can detect
whether the complete document or merely a child is to be compiled
even without using the directive |\childdocof|.
This method is deprecated because it is less robust
and there is no compelling reason to use it;
it is merely provided for backward compatibility
and it may be removed in future versions.

If the detection mechanism is to be used,
it is mandatory to correctly specify
the filename of the main file as the argument of |\childdocmain|:
%
\begin{center}
\begin{tabular}{l}
|\input{childdoc.def}|\\
|\childdocmain{|\textit{main}|}|\\
\end{tabular}
\end{center}
%
If |\jobname| does not match the argument \textit{main} of |\childdocmain|,
it is assumed that |\jobname| points to the child file to be compiled.
When using |\childdocmain| with the main file specified as argument,
it suffices to start a child file
with just |\input{|\textit{main}|}|
without loading of the package and using |\childdocof|.
If instead all processing is done
with the appropriate \textsf{childdoc} directives,
the argument of \textit{main} of |\childdocmain| can be empty.

An alternative version of the command line processing described
in \secref{sec:commandline} using the detection mechanism reads:
%
\begin{center}
|... -jobname "|\textit{target}|" "|[\textit{flags}]%
[|\def\jobname{|\textit{dest}|}|]|\input{|\textit{main}|}"|
\end{center}

%%%%%%%%%%%%%%%%%%%%%%%%%%%%%%%%%%%%%%%%%%%%%%%%%%%%%%%%%%%%%%%%%%%%%%%%%%%%%%%%
\subsection{Manual Code}
\label{sec:manual}

In case one cannot be certain whether the definitions file |childdoc.def|
is installed on the target \TeX{} distribution
and one prefers not to ship it,
it is conceivable to paste a few relevant commands into the sources.

To that end, drop all statements |\input{childdoc.def}|
and perform the replacements as outlined below.
Instead of |\childdocmain{|\textit{main}|}| add the following code
to the top of the main file:
%
\begin{center}
\begin{tabular}{l}
|\||ifdefined\childdocname\endinput\||fi\newif\ifchilddoc|\\
|\edef\childdocname{\scantokens\expandafter{\jobname\noexpand}}|\\
|\def\childdocmain{|\textit{main}|}\||ifx\childdocmain\childdocname\||else|\\
|\childdoctrue\includeonly{\childdocname}\let\jobname\childdocmain\||fi|\\
\end{tabular}
\end{center}
%
Instead of |\childdocof{|\textit{main}|}| just include the main file
at the top of each child file:
%
\begin{center}
|\input{|\textit{main}|}|
\end{center}
%
A simple redirection |\childdocforward{|\textit{dest}|}| is achieved by:
%
\begin{center}
|\def\jobname{|\textit{dest}|}\input{\jobname}|
\end{center}
%
The redirection with prefix
|\childdocforwardprefix[|\textit{prefix}|]{|\textit{dest}|}|
is accomplished by:
%
\begin{center}
\begin{tabular}{l}
|{\edef\jobname{\scantokens\expandafter{\jobname\noexpand}}|\\
|\def\redirectjob |\textit{prefix}|#1~~~{\gdef\jobname{|\textit{dest}|#1}}|\\
|\expandafter\redirectjob\jobname~~~}\input{\jobname}|
\end{tabular}
\end{center}

In an alternative approach,
child documents can be compiled by a specific command line
without additional code or specific definitions:
%
\begin{center}
|... -jobname "|\textit{target}|" "|[\textit{flags}]%
|\includeonly{|\textit{dest}|}\input{|\textit{main}|}"|
\end{center}
%

%%%%%%%%%%%%%%%%%%%%%%%%%%%%%%%%%%%%%%%%%%%%%%%%%%%%%%%%%%%%%%%%%%%%%%%%%%%%%%%%
%%%%%%%%%%%%%%%%%%%%%%%%%%%%%%%%%%%%%%%%%%%%%%%%%%%%%%%%%%%%%%%%%%%%%%%%%%%%%%%%
\section{Information}

%%%%%%%%%%%%%%%%%%%%%%%%%%%%%%%%%%%%%%%%%%%%%%%%%%%%%%%%%%%%%%%%%%%%%%%%%%%%%%%%
\subsection{Copyright}

Copyright \copyright{} 2017--2018 Niklas Beisert

This work may be distributed and/or modified under the
conditions of the \LaTeX{} Project Public License, either version 1.3
of this license or (at your option) any later version.
The latest version of this license is in
  \url{http://www.latex-project.org/lppl.txt}
and version 1.3 or later is part of all distributions of \LaTeX{}
version 2005/12/01 or later.

This work has the LPPL maintenance status `maintained'.

The Current Maintainer of this work is Niklas Beisert.

This work consists of the files |README.txt|, |childdoc.ins| and |childdoc.dtx|
as well as the derived files |childdoc.def|, |cdocsamp.tex|
with |cdocsch1.tex|, |cdocsch2.tex|, |cdocspt3.tex|, |cdocspt4.tex|,
|cdocsdrf.tex|, |cdocsfn1.tex|, |cdocsfn2.tex|
as well as |childdoc.pdf|.

%%%%%%%%%%%%%%%%%%%%%%%%%%%%%%%%%%%%%%%%%%%%%%%%%%%%%%%%%%%%%%%%%%%%%%%%%%%%%%%%
\subsection{Files and Installation}

The package consists of the files:
%
\begin{center}
\begin{tabular}{ll}
    |README.txt|   & readme file \\
    |childdoc.ins| & installation file \\
    |childdoc.dtx| & source file \\
    |childdoc.def| & definition file \\
    |cdocsamp.tex| & sample main file \\
    |cdocsch1.tex| & sample include file \\
    |cdocsch2.tex| & sample include file \\
    |cdocspt3.tex| & sample part file \\
    |cdocspt4.tex| & sample part file \\
    |cdocsdrf.tex| & sample redirection file \\
    |cdocsfn1.tex| & sample redirection file \\
    |cdocsfn2.tex| & sample redirection file \\
    |childdoc.pdf| & manual
\end{tabular}
\end{center}
%
The distribution consists of the files
|README.txt|, |childdoc.ins| and |childdoc.dtx|.
%
\begin{itemize}
\item
Run (pdf)\LaTeX{} on |childdoc.dtx|
to compile the manual |childdoc.pdf| (this file).
\item
Run \LaTeX{} on |childdoc.ins| to create the definitions file |childdoc.def|
and the sample |cdocsamp.tex| with include files
|cdocsch1.tex|, |cdocsch2.tex|, |cdocspt3.tex|, |cdocspt4.tex|,
|cdocsdrf.tex|, |cdocsfn1.tex|, |cdocsfn2.tex|.
Then copy the file |childdoc.def| to an appropriate directory of your \LaTeX{}
distribution, e.g.\ \textit{texmf-root}|/tex/latex/childdoc|.
\end{itemize}

%%%%%%%%%%%%%%%%%%%%%%%%%%%%%%%%%%%%%%%%%%%%%%%%%%%%%%%%%%%%%%%%%%%%%%%%%%%%%%%%
\subsection{Related CTAN Packages}

There are several other packages which offer a similar functionality:
%
\begin{itemize}
\item
The packages
\href{http://ctan.org/pkg/docmute}{\textsf{docmute}},
\href{http://ctan.org/pkg/includex}{\textsf{includex}} and
\href{http://ctan.org/pkg/standalone}{\textsf{standalone}}
provide commands to include only the document body of
a child file thus allowing both files to be compiled individually.
\item
The packages \href{http://ctan.org/pkg/subdocs}{\textsf{subdocs}}
and \href{http://ctan.org/pkg/subfiles}{\textsf{subfiles}}
provide structures in which the main and child documents can be
encapsulated and allowing them to be compiled individually.
The inclusion mechanism is different from the conventional |\include|.
\item
The package \href{http://ctan.org/pkg/combine}{\textsf{combine}}
is an elaborate solution to combine several documents into one.
\end{itemize}
%
See also the CTAN topic \href{http://ctan.org/topic/subdocs}{\textsf{subdocs}}
for further related packages.
The present package differs from the above solutions in that
a document structure constructed with the conventional |\include| mechanism
just needs two extra commands at the top of every file
such that all constituent files can be compiled individually.

%%%%%%%%%%%%%%%%%%%%%%%%%%%%%%%%%%%%%%%%%%%%%%%%%%%%%%%%%%%%%%%%%%%%%%%%%%%%%%%%
%\subsection{Feature Suggestions}
%
%The following is a list of features which may be useful for future
%versions of this package:
%%
%\begin{itemize}
%\item
%\ldots
%\end{itemize}

%%%%%%%%%%%%%%%%%%%%%%%%%%%%%%%%%%%%%%%%%%%%%%%%%%%%%%%%%%%%%%%%%%%%%%%%%%%%%%%%
\subsection{Revision History}

%%%%%%%%%%%%%%%%%%%%%%%%%%%%%%%%%%%%%%%%
\paragraph{v2.0:} 2018/12/30

\begin{itemize}
\item
immediate forward processing
\item
added |\childdocby| mechanism
\item
manual restructured
\end{itemize}

%%%%%%%%%%%%%%%%%%%%%%%%%%%%%%%%%%%%%%%%
\paragraph{v1.6:} 2018/01/17

\begin{itemize}
\item
application for development of include files
\item
corrections to manual
\end{itemize}

%%%%%%%%%%%%%%%%%%%%%%%%%%%%%%%%%%%%%%%%
\paragraph{v1.5:} 2017/05/21

\begin{itemize}
\item
more complete structuring introduced
\item
|\childdocof| introduced
\item
|\childdoc| renamed to |\childdocmain|
\item
|\childredirect| renamed to |\childdocforward| and |\childdocforwardprefix|
and functionality expanded
\end{itemize}

%%%%%%%%%%%%%%%%%%%%%%%%%%%%%%%%%%%%%%%%
\paragraph{v1.0:} 2017/04/27

\begin{itemize}
\item
manual and install package
\item
first version published on CTAN
\end{itemize}

%%%%%%%%%%%%%%%%%%%%%%%%%%%%%%%%%%%%%%%%
\paragraph{v0.6:} 2017/04/26

\begin{itemize}
\item
redirection mechanism added
\end{itemize}

%%%%%%%%%%%%%%%%%%%%%%%%%%%%%%%%%%%%%%%%
\paragraph{v0.5:} 2017/04/26

\begin{itemize}
\item
functionality in definition file
\end{itemize}


%%%%%%%%%%%%%%%%%%%%%%%%%%%%%%%%%%%%%%%%%%%%%%%%%%%%%%%%%%%%%%%%%%%%%%%%%%%%%%%%
%%%%%%%%%%%%%%%%%%%%%%%%%%%%%%%%%%%%%%%%%%%%%%%%%%%%%%%%%%%%%%%%%%%%%%%%%%%%%%%%
%%%%%%%%%%%%%%%%%%%%%%%%%%%%%%%%%%%%%%%%%%%%%%%%%%%%%%%%%%%%%%%%%%%%%%%%%%%%%%%%
\appendix

\settowidth\MacroIndent{\rmfamily\scriptsize 000\ }

 \DocInput{childdoc.dtx}

\end{document}
%</driver>
% \fi
%
% %%%%%%%%%%%%%%%%%%%%%%%%%%%%%%%%%%%%%%%%%%%%%%%%%%%%%%%%%%%%%%%%%%%%%%%%%%%%%%
% %%%%%%%%%%%%%%%%%%%%%%%%%%%%%%%%%%%%%%%%%%%%%%%%%%%%%%%%%%%%%%%%%%%%%%%%%%%%%%
% \section{Sample}
%\iffalse
%<*samplemain>
%\fi
%
% The following presents a sample document
% with two chapters, two parts, a title page,
% a compile flag as well as three forwarding files to set the flag.
% It consists of eight |.tex| files:
% \begin{center}
% \begin{tabular}{ll}
% |cdocsamp.tex|&main file\\
% |cdocsch1.tex|&include file for chapter 1\\
% |cdocsch2.tex|&include file for chapter 2\\
% |cdocspt3.tex|&include file for part 3\\
% |cdocspt4.tex|&include file for part 4\\
% |cdocsdrf.tex|&forwarding file for main file in draft mode\\
% |cdocsfi1.tex|&forwarding file for final version of chapter 1\\
% |cdocsfi2.tex|&forwarding file for final version of chapter 2\\
% \end{tabular}
% \end{center}
% Each of the eight files can be compiled directly by the \LaTeX{} compiler.
%
% %%%%%%%%%%%%%%%%%%%%%%%%%%%%%%%%%%%%%%
% \paragraph{Main File.}
%
% The main file is called |cdocsamp.tex|.
%
% Load the \textsf{childdoc} definitions and
% declare the filename for the main document:
%    \begin{macrocode}
\input{childdoc.def}
\childdocmain{}
%    \end{macrocode}

% Optional override for |\version| flag:
%    \begin{macrocode}
%%\ifchilddoc\else\providecommand{\version}{draft}\fi
%    \end{macrocode}

% Define the default values for the |\version| flag
% (|final| for the main file and |draft| for childs):
%    \begin{macrocode}
\ifchilddoc
\providecommand{\version}{draft}
\else
\providecommand{\version}{final}
\fi
%    \end{macrocode}

% Load the standard document class:
%    \begin{macrocode}
\documentclass[12pt]{article}
%    \end{macrocode}

% Start the document body:
%    \begin{macrocode}
\begin{document}
%    \end{macrocode}

% Declare a title page.
% Print title, part of document being processed and version flag:
%    \begin{macrocode}
\addtocounter{page}{-1}
\begin{center}
{\LARGE\bfseries{}childdoc example\par}
\vspace{1cm}
\ifchilddoc
\ifchilddocmanual part\else chapter\fi:
`\childdocname' of `\childdocjob'\par
\else
main document: `\childdocjob'\par
\fi
version: \version\par
\end{center}
\newpage
%    \end{macrocode}

% Manually include selected file,
% otherwise process as usual:
%    \begin{macrocode}
\ifchilddocmanual
\section*{part `\childdocname'}
\input{\childdocname}
\else
%    \end{macrocode}

% Include the two chapters:
%    \begin{macrocode}
\include{cdocsch1}
\include{cdocsch2}
%    \end{macrocode}

% Include the two parts unless only chapters should be displayed:
%    \begin{macrocode}
\ifchilddoc\else
\section{part three}
\input{cdocspt3}
\section{part four}
\input{cdocspt4}
\fi
%    \end{macrocode}

% Process as usual until here:
%    \begin{macrocode}
\fi
%    \end{macrocode}

% End of document body:
%    \begin{macrocode}
\end{document}
%    \end{macrocode}
%\iffalse
%</samplemain>
%\fi
%
% %%%%%%%%%%%%%%%%%%%%%%%%%%%%%%%%%%%%%%
% \paragraph{Chapter Include Files.}
%
% The include files are called |cdocsch1.tex| and |cdocsch2.tex|.
%
%\iffalse
%<*samplechap1|samplechap2>
%\fi

% Optional override for |\version| flag:
%    \begin{macrocode}
%%\providecommand{\version}{final}
%    \end{macrocode}

% Include the main document:
%    \begin{macrocode}
\input{childdoc.def}
\childdocof{cdocsamp}
%    \end{macrocode}

%\iffalse
%</samplechap1|samplechap2>
%\fi
%
%\iffalse
%<*samplechap1>
%\fi
% Some text for chapter 1:
%    \begin{macrocode}
\section{one}
some text in chapter one
%    \end{macrocode}

%\iffalse
%</samplechap1>
%\fi
% Some text for chapter 2:
%\iffalse
%<*samplechap2>
%\fi
%    \begin{macrocode}
\section{two}
more text in chapter two
%    \end{macrocode}

%\iffalse
%</samplechap2>
%\fi
%
% %%%%%%%%%%%%%%%%%%%%%%%%%%%%%%%%%%%%%%
% \paragraph{Part Include Files.}
%
% The include files are called |cdocspt3.tex| and |cdocspt4.tex|.
%
%\iffalse
%<*samplepart3|samplepart4>
%\fi

% Optional override for |\version| flag:
%    \begin{macrocode}
%%\providecommand{\version}{final}
%    \end{macrocode}

% Include the main document:
%    \begin{macrocode}
\input{childdoc.def}
\childdocby{cdocsamp}
%    \end{macrocode}

%\iffalse
%</samplepart3|samplepart4>
%\fi
%
%\iffalse
%<*samplepart3>
%\fi
% Some text for part 3:
%    \begin{macrocode}
some text in part three
%    \end{macrocode}

%\iffalse
%</samplepart3>
%\fi
% Some text for part 4:
%\iffalse
%<*samplepart4>
%\fi
%    \begin{macrocode}
more text in part four
%    \end{macrocode}

%\iffalse
%</samplepart4>
%\fi
%
% %%%%%%%%%%%%%%%%%%%%%%%%%%%%%%%%%%%%%%
% \paragraph{Forwarding for a Complete Draft.}
%
% The following forwarding file |cdocsdrf.tex|
% compiles the main document in draft mode:
%\iffalse
%<*sampledraft>
%\fi
%    \begin{macrocode}
\def\version{draft}
\input{childdoc.def}
\childdocforward{cdocsamp}
%    \end{macrocode}

%\iffalse
%</sampledraft>
%\fi
%
% %%%%%%%%%%%%%%%%%%%%%%%%%%%%%%%%%%%%%%
% \paragraph{Forwarding for Final Version of the Chapters.}
%
% The following forwarding files |cdocsfn1.tex| and |cdocsfn2.tex|
% (with identical content)
% compile the final versions of the child documents
% |cdocsch1.tex| and |cdocsch2.tex|, respectively:
%\iffalse
%<*samplefinal>
%\fi
%    \begin{macrocode}
\def\version{final}
\input{childdoc.def}
\childdocforwardprefix[cdocsamp]{cdocsfn}{cdocsch}
%    \end{macrocode}

%\iffalse
%</samplefinal>
%\fi
%
% %%%%%%%%%%%%%%%%%%%%%%%%%%%%%%%%%%%%%%
% \paragraph{Command Line Processing.}
%
% The following three command lines generate the output files
% |cdocscld|, |cdocscl1| and |cdocscl2|
% which should be identical to
% |cdocsdrf|, |cdocsch1| and |cdocsfn2|, respectively:
% \begin{center}
% \begin{tabular}{l}
% |latex -jobname cdocscld \|\\
% |  "\def\version{draft}\input{childdoc.def}\childdocforward{cdocsamp}"|\\
% |latex -jobname cdocscl1 \|\\
% |  "\input{childdoc.def}\childdocforward[cdocsamp]{cdocsch1}"|\\
% |latex -jobname cdocscl2 \|\\
% |  "\def\version{final}\input{childdoc.def}\childdocforward{cdocsch2}"|
% \end{tabular}
% \end{center}
% Note that the trailing backslash on each first line
% merely continues the input to the second line
% (for convenient cut ant paste).
% Furthermore, the command |latex| can be replaced by any
% of its alternative versions such as |pdflatex|.
%
% %%%%%%%%%%%%%%%%%%%%%%%%%%%%%%%%%%%%%%%%%%%%%%%%%%%%%%%%%%%%%%%%%%%%%%%%%%%%%%
% %%%%%%%%%%%%%%%%%%%%%%%%%%%%%%%%%%%%%%%%%%%%%%%%%%%%%%%%%%%%%%%%%%%%%%%%%%%%%%
% \section{Implementation}
%\iffalse
%<*package>
%\fi
%
% This section describes the definitions file |childdoc.def|.

% The definitions cannot be loaded using |\usepackage| or |\RequirePackage|
% which has a mechanism to prevent loading a style file more than once.
% When loading the definitions by means of |\input|
% multiple instances have to be prevented manually:
%\iffalse
%This code needs to be before the `\ProvidesFile' directive
%which is defined at the beginning of this file.
%Therefore it is also placed there and commented out here.
%</package>
%<*discard>
%\fi
%    \begin{macrocode}
\ifdefined\childdocmain\endinput\fi
%    \end{macrocode}
%\iffalse
%</discard>
%<*package>
%\fi
%
% \macro{\ifchilddoc}
% \macro{\ifchilddocmanual}
% The conditional |\ifchilddoc| tells whether a
% child (true) or main (false) document is being compiled.
% The conditional |\ifchilddocmanual| tells whether
% the |\includeonly| mechanism is used (false) or
% the selection of child files must be performed manually (true).
% The definitions initialise to false:
%    \begin{macrocode}
\newif\ifchilddoc
\newif\ifchilddocmanual
%    \end{macrocode}

% \macro{\childdocname}
% \macro{\childdocjob}
% The macro |\childdocname| stores the name of the main document
% to be compiled. The macro |\childdocjob| stores the name of
% the document on which the \LaTeX{} compiler was originally invoked.
% The content of |\jobname| cannot be compared
% to filenames specified in the source due to different catcodes.
% The following code rescans |\jobname|, stores the result
% in |\childdocname| and saves a copy in |\childdocjob|:
%    \begin{macrocode}
\edef\childdocname{\scantokens\expandafter{\jobname\noexpand}}
\let\childdocjob\childdocname
%    \end{macrocode}

% \macro{\childdocdisable}
% The macro |\childdocdisable| prevents the main file
% from being processed more than once.
% At this stage, the main document command |\childdocmain|
% is assumed to be called once again where it should do nothing.
% Any subsequent call to it should prevent
% a secondary processing of the main document
% It overwrites the forwarding commands
% |\childdocof| and |\childdocforward|
% with empty macros to prevent further inclusions of the main document:
%    \begin{macrocode}
\newcommand{\childdocdisable}
{
  \renewcommand{\childdocmain}[1]{\renewcommand{\childdocmain}[1]{\endinput}}
  \renewcommand{\childdocof}[1]{}
  \renewcommand{\childdocby}[2][]{}
  \renewcommand{\childdocforward}[2][]{}
  \renewcommand{\childdocdisable}{}
}
%    \end{macrocode}

% \macro{\childdocmain}
% The macro |\childdocmain| is to be called at the top of the main file
% with nothing or the main filename (without extension) as argument.
% First, it breaks loops.
% If the argument is not empty and does not match |\childdocname|
% (which is set by the first inclusion of |childdoc.def|),
% |\ifchilddoc| is set to true, |\includeonly| is applied to the child file
% and |\jobname| is set to the main file
% (for proper handling of |.aux| files):
%    \begin{macrocode}
\newcommand{\childdocmain}[1]
{
  \childdocdisable\childdocmain{}
  \if?#1?\else
    \begingroup
      \def\childdoctmp{#1}
      \ifx\childdoctmp\childdocname
        \def\childdoctmp{}
      \else
        \def\childdoctmp
        {
          \childdoctrue
          \includeonly{\childdocname}
          \def\childdocjob{#1}
          \def\jobname{#1}
        }
      \fi
      \expandafter
    \endgroup
    \childdoctmp
  \fi
}
%    \end{macrocode}

% \macro{\childdocof}
% The command |\childdocof| redirects
% compilation to the main file |#1|.
%    \begin{macrocode}
\newcommand{\childdocof}[1]
{
  \childdocdisable
  \childdoctrue
  \includeonly{\childdocname}
  \def\jobname{#1}
  \def\childdocjob{#1}
  \input{#1}
}
%    \end{macrocode}

% \macro{\childdocby}
% The command |\childdocby| ....
%    \begin{macrocode}
\newcommand{\childdocby}[2][]
{
  \childdocdisable
  \childdoctrue
  \childdocmanualtrue
  \if?#1?\else
    \def\jobname{#2}
  \fi
  \def\childdocjob{#2}
  \input{#2}
  \endinput
}
%    \end{macrocode}

% \macro{\childdocforward}
% The command |\childdocforward| redirects
% compilation to the main file or
% (if the optional argument is given) a child file.
% Parameters are set as if the main file
% or a child file starting with |\childdocof| was compiled.
% Then compilation is handed over to the main file:
%    \begin{macrocode}
\newcommand{\childdocforward}[2][]
{
  \begingroup
    \if?#1?
      \def\childdoctmp
      {
        \def\childdocname{#2}
        \def\childdocjob{#2}
        \def\jobname{#2}
        \input{#2}
        \endinput
      }
    \else
      \def\childdoctmp
      {
        \childdocdisable
        \def\childdocname{#2}
        \childdoctrue
        \includeonly{#2}
        \def\childdocjob{#1}
        \def\jobname{#1}
        \input{#1}
        \endinput
      }
    \fi
    \expandafter
  \endgroup
  \childdoctmp
}
%    \end{macrocode}

% \macro{\childdocforwardprefix}
% The command |\childdocforwardprefix| redirects
% compilation to the main or a child file by means of a pattern.
% The prefix |#1| in the current filename is replaced by |#2|
% and the suffix of the current filename is kept
% (it is assumed that the filename does not contain the substring `|~~~|'
% which is used as a delimiter).
% Compilation is handed over to the new file by |\childdocforward|:
%    \begin{macrocode}
\newcommand{\childdocforwardprefix}[3][]
{
  \begingroup
    \def\childdocextract #2##1~~~{\def\childdoctmp{\childdocforward[#1]{#3##1}}}
    \expandafter\childdocextract\childdocname~~~
    \expandafter
  \endgroup
  \childdoctmp
}
%    \end{macrocode}

% \macro{\childdoc}
% The deprecated macro |\childdoc| is a legacy version of |\childdocmain|:
%    \begin{macrocode}
\newcommand{\childdoc}{\childdocmain}
%    \end{macrocode}

% \macro{\childdocredirect}
% The deprecated macro |\childdocredirect| is a legacy version
% of |\childdocforward| and |\childdocforwardprefix|:
%    \begin{macrocode}
\newcommand{\childdocredirect}[2][]
{
  \begingroup
    \if?#1?
      \def\childdoctmp{\childdocforward{#2}}
    \else
      \def\childdoctmp{\childdocforwardprefix{#1}{#2}}
    \fi
    \expandafter
  \endgroup
  \childdoctmp
}
%    \end{macrocode}

%\iffalse
%</package>
%\fi
%
\endinput

\childdocby{cdocsamp}
%    \end{macrocode}

%\iffalse
%</samplepart3|samplepart4>
%\fi
%
%\iffalse
%<*samplepart3>
%\fi
% Some text for part 3:
%    \begin{macrocode}
some text in part three
%    \end{macrocode}

%\iffalse
%</samplepart3>
%\fi
% Some text for part 4:
%\iffalse
%<*samplepart4>
%\fi
%    \begin{macrocode}
more text in part four
%    \end{macrocode}

%\iffalse
%</samplepart4>
%\fi
%
% %%%%%%%%%%%%%%%%%%%%%%%%%%%%%%%%%%%%%%
% \paragraph{Forwarding for a Complete Draft.}
%
% The following forwarding file |cdocsdrf.tex|
% compiles the main document in draft mode:
%\iffalse
%<*sampledraft>
%\fi
%    \begin{macrocode}
\def\version{draft}
% \iffalse
%
% childdoc.dtx Copyright (C) 2017-2018 Niklas Beisert
%
% This work may be distributed and/or modified under the
% conditions of the LaTeX Project Public License, either version 1.3
% of this license or (at your option) any later version.
% The latest version of this license is in
%   http://www.latex-project.org/lppl.txt
% and version 1.3 or later is part of all distributions of LaTeX
% version 2005/12/01 or later.
%
% This work has the LPPL maintenance status `maintained'.
%
% The Current Maintainer of this work is Niklas Beisert.
%
% This work consists of the files childdoc.dtx and childdoc.ins
% and the derived files childdoc.def and cdocsamp.tex with
% cdocsch1.tex, cdocsch2.tex, cdocsdrf.tex, cdocsfn1.tex, cdocsfn2.tex.
%
%<package>\ifdefined\childdocmain\endinput\fi
%<package>\ProvidesFile{childdoc.def}[2018/12/30 v2.0 child document driver]
%<samplemain>\ProvidesFile{cdocsamp.tex}[2018/12/30 v2.0 sample for childdoc]
%<*driver>
%\ProvidesFile{childdoc.drv}[2018/12/30 v2.0 childdoc reference manual file]
\PassOptionsToClass{10pt,a4paper}{article}
\documentclass{ltxdoc}

\usepackage[margin=35mm]{geometry}
\usepackage{hyperref}
\usepackage{hyperxmp}
\usepackage[usenames]{color}

\hypersetup{colorlinks=true}
\hypersetup{pdfstartview=FitH}
\hypersetup{pdfpagemode=UseNone}
\hypersetup{pdfsource={}}
\hypersetup{pdflang={en-UK}}
\hypersetup{pdfcopyright={Copyright 2017-2018 Niklas Beisert.
  This work may be distributed and/or modified under the
  conditions of the LaTeX Project Public License, either version 1.3
  of this license or (at your option) any later version.}}
\hypersetup{pdflicenseurl={http://www.latex-project.org/lppl.txt}}
\hypersetup{pdfcontactaddress={ETH Zurich, ITP, HIT K,
  Wolfgang-Pauli-Strasse 27}}
\hypersetup{pdfcontactpostcode={8093}}
\hypersetup{pdfcontactcity={Zurich}}
\hypersetup{pdfcontactcountry={Switzerland}}
\hypersetup{pdfcontactemail={nbeisert@itp.phys.ethz.ch}}
\hypersetup{pdfcontacturl={http://people.phys.ethz.ch/\xmptilde nbeisert/}}

\newcommand{\secref}[1]{\hyperref[#1]{section \ref*{#1}}}

\parskip1ex
\parindent0pt
\let\olditemize\itemize
\def\itemize{\olditemize\parskip0pt}

\begin{document}

\title{The \textsf{childdoc} Package}
\hypersetup{pdftitle={The childdoc Package}}
\author{Niklas Beisert\\[2ex]
  Institut f\"ur Theoretische Physik\\
  Eidgen\"ossische Technische Hochschule Z\"urich\\
  Wolfgang-Pauli-Strasse 27, 8093 Z\"urich, Switzerland\\[1ex]
  \href{mailto:nbeisert@itp.phys.ethz.ch}
  {\texttt{nbeisert@itp.phys.ethz.ch}}}
\hypersetup{pdfauthor={Niklas Beisert}}
\hypersetup{pdfsubject={Manual for the LaTeX2e Package childdoc}}
\date{30 December 2018, \textsf{v2.0}}
\maketitle

\begin{abstract}\noindent
\textsf{childdoc} is a \LaTeXe{} package
that enables the direct compilation
of document sections included by |\include|
to individual files.
\end{abstract}

\begingroup
\parskip0ex
\tableofcontents
\endgroup

%%%%%%%%%%%%%%%%%%%%%%%%%%%%%%%%%%%%%%%%%%%%%%%%%%%%%%%%%%%%%%%%%%%%%%%%%%%%%%%%
%%%%%%%%%%%%%%%%%%%%%%%%%%%%%%%%%%%%%%%%%%%%%%%%%%%%%%%%%%%%%%%%%%%%%%%%%%%%%%%%
\section{Introduction}

\LaTeX{} provides a mechanism to structure a large document (such as a book)
into a main file and several child files (containing the chapters)
using the |\include| command.
This mechanism is beneficial for documents
which span hundreds of pages in order to
make the source file(s) more manageable.
Moreover, compilation can be restricted to
selected child files by means of the |\includeonly| command.
The latter feature can be used to reduce the compilation time while editing
(this was significantly more useful in the earlier days of \LaTeX{})
or to generate a smaller document which is easier to navigate.
Another application of |\includeonly| is to generate
documents consisting of selected parts of the complete document.

However, there are a few drawbacks of the plain |\include| mechanism:
\begin{itemize}
\item
The child files cannot be compiled on their own,
they can only be compiled via the main file.
A naive editing environment
(such as a text editor with an option
to have the current file processed by \LaTeX)
may require one to switch to the main file before compiling;
attempting to compile the child file produces errors.
\item
The main file must be modified (each time)
to adjust the |\includeonly| command
to the present needs. This easily leaves the main file in a messy state.
\item
The generated document will always carry the filename
of the main document. This is inconvenient if
several child files are to be compiled and
to be kept for distribution.
\end{itemize}

The present package provides a simple interface
to make child files individually compilable by \LaTeX{}.
Compiling a child file then has the same effect as compiling
the main file with an |\includeonly| command
to select the appropriate child.
Moreover the generated document will carry the name of the child
rather than the main file.
This resolves all three above issues.

This feature is meant to make the editing of books,
thesis documents and lecture notes somewhat more convenient.
However, the package can also be used efficiently for
composing a series of documents (such as exercise sheets)
which are typically distributed individually.
It then assists the author in generating the individual documents
(potentially in different versions)
as well as a document containing the collected series.
Another application is in developing style files
or other kinds of included material
where compilation of the style file could redirect
to a sample or test file.

%%%%%%%%%%%%%%%%%%%%%%%%%%%%%%%%%%%%%%%%%%%%%%%%%%%%%%%%%%%%%%%%%%%%%%%%%%%%%%%%
%%%%%%%%%%%%%%%%%%%%%%%%%%%%%%%%%%%%%%%%%%%%%%%%%%%%%%%%%%%%%%%%%%%%%%%%%%%%%%%%
\section{Usage}

First of all, the package \textsf{childdoc} is \emph{not} a standard
\LaTeXe{} |.sty| style file! Therefore it needs to be invoked in
a non-standard way.

%%%%%%%%%%%%%%%%%%%%%%%%%%%%%%%%%%%%%%%%%%%%%%%%%%%%%%%%%%%%%%%%%%%%%%%%%%%%%%%%
\subsection{Included Files}
\label{sec:include}

%%%%%%%%%%%%%%%%%%%%%%%%%%%%%%%%%%%%%%%%
\DescribeMacro{\childdocmain}
To use the package, add the commands
\begin{center}
\begin{tabular}{l}
|\input{childdoc.def}|\\
|\childdocmain{}|\\
\end{tabular}
\end{center}
at the very top of the main \LaTeX{} file,
in particular \emph{before} the |\documentclass| statement!
The argument of |\childdocmain| should be left empty
(but it must be present).

%%%%%%%%%%%%%%%%%%%%%%%%%%%%%%%%%%%%%%%%
\DescribeMacro{\childdocof}
Furthermore, add the commands
\begin{center}
\begin{tabular}{l}
|\input{childdoc.def}|\\
|\childdocof{|\textit{main}|}|\\
\end{tabular}
\end{center}
at the top of every child file \textit{child}
which is included by |\include{|\textit{child}|}|
from within the main file
(or at least for those files to be compiled individually).
The argument \textit{main} must be the filename of the main file.

There are a couple of
considerations in setting up the main and child documents:

%%%%%%%%%%%%%%%%%%%%%%%%%%%%%%%%%%%%%%%%
\paragraph{Restrictions.}

Please note the following restrictions:
\begin{itemize}
\item
|\childdocmain| must be called with one argument \textit{main}
to ensure compatibility with earlier version of the package.
It must either be empty (|\childdocmain{}|)
or precisely match the filename of the main file in which it is specified.
See \secref{sec:detection} for further information.
\item
The filename \textit{main} must be specified without the |.tex| extension.
\item
The filename \textit{main} is case sensitive
(even in case-insensitive file systems)
due to internal string comparison.
\item
The argument \textit{main} should be fully expanded, it cannot be a macro.
\item
Subdirectories and special characters should be avoided in filenames.
\item
The command |\childdocmain{|\textit{main}|}| must be followed by a whitespace.
It should not be followed immediately by another command
or by a comment mark `|%|'.
This is because the \TeX{} parser reads the token immediately following
the argument of |\childdocmain| and puts it
at the beginning of every child section;
however, a white\-space is ignored.
\end{itemize}

%%%%%%%%%%%%%%%%%%%%%%%%%%%%%%%%%%%%%%%%
\paragraph{Content of Main File.}

It is advisable to place all content in the child files included by |\include|.
Any output contained in the main file will appear in all child documents
unless suppressed manually;
it cannot be suppressed automatically by the |\includeonly| directive
and thus should normally be avoided.
A method to include some content in the main file
by means of conditional processing is described in \secref{sec:conditional}.

%%%%%%%%%%%%%%%%%%%%%%%%%%%%%%%%%%%%%%%%
\paragraph{Page Numbering.}

When only a part of the document is compiled,
the appropriate numbering of pages
(as well as other status parameters)
is determined from the |.aux| files.
The latter contain information from previous passes.
However this information needs to propagate through
all intermediate child documents.
Therefore the page numbering in child documents may well
be inconsistent until the complete document is compiled at least once.

A useful (if unconventional) way to always ensure a consistent
page numbering is to restart the numbering in each child document
and denote the pages by `\textit{child}|.|\textit{page}'
where \textit{child} represents the chapter/section number of the child file.
This can be achieved by the command
|\numberwithin{page}{|\textit{child}|}|
of the \textsf{amsmath} package
where \textit{child} can be |chapter| or |section|
depending on the chosen structuring.
Alternatively, one can modify the macro |\thepage| appropriately
and reset the counter |page| at the start of each child file.

%%%%%%%%%%%%%%%%%%%%%%%%%%%%%%%%%%%%%%%%%%%%%%%%%%%%%%%%%%%%%%%%%%%%%%%%%%%%%%%%
\subsection{Conditional Processing}
\label{sec:conditional}

The package provides a mechanism to compile different versions
of a document. To customise the versions further some conditional processing
can come in handy to distinguish which version is being compiled.
The package provides two macros to describe the compilation context:

%%%%%%%%%%%%%%%%%%%%%%%%%%%%%%%%%%%%%%%%
\DescribeMacro{\ifchilddoc}
The conditional |\ifchilddoc| distinguishes between the compilation of
child documents and the main document:
%
\begin{center}
|\ifchilddoc |\textit{child-code}| |[|\||else |\textit{main-code}]| \||fi|
\end{center}

%%%%%%%%%%%%%%%%%%%%%%%%%%%%%%%%%%%%%%%%
\DescribeMacro{\childdocname}
\DescribeMacro{\childdocjob}
The macro |\childdocname| contains the filename (without extension)
of the main or child file being processed.
Note that |\childdocjob| will always contain the name of the main file.

%%%%%%%%%%%%%%%%%%%%%%%%%%%%%%%%%%%%%%%%
\paragraph{Title Page.}

Conditional processing can be used to include a title or banner page
in the main document when proper precautions are taken.
Importantly, the code in the main file should ensure that the page counter
(as well as other status parameters which are stored in the |.aux| files)
takes the same value after the conditional processing.
Otherwise the page numbers may take divergent values
depending on which part is compiled.

For example, a title page could be declared by:
%
\begin{center}
\begin{tabular}{l}
|\ifchilddoc\||else|\\
|\addtocounter{page}{-1}|\\
\textit{code for title page}\\
|\newpage|\\
|\||fi|
\end{tabular}
\end{center}
%
A banner page for the child documents can be generated by:
%
\begin{center}
\begin{tabular}{l}
|\ifchilddoc|\\
|\addtocounter{page}{-1}|\\
\textit{code for banner page}\\
|\newpage|\\
|\||fi|
\end{tabular}
\end{center}
%
Here one could write a message such as:
\begin{center}
|This is the part \childdocname{} of \childdocjob{}.|
\end{center}

%%%%%%%%%%%%%%%%%%%%%%%%%%%%%%%%%%%%%%%%%%%%%%%%%%%%%%%%%%%%%%%%%%%%%%%%%%%%%%%%
\subsection{Flags}
\label{sec:flags}

The package makes it easy to generate different versions
of the main or child documents.
To this end compilation flags can be defined
and assigned different default values.
They will be particularly useful in conjunction
with the forwarding mechanism described in \secref{sec:forward}.

For example, it may be useful to have a flag |\version|
which can be set to |draft| or |final|.
The document source will contain some conditional code
depending on the value of |\version|.
Suppose further, the flag should default to |final| for the main file
and to |draft| for child files
which is a natural assignment for editing the document.
This is achieved by placing the following code
in the preamble of the main document
(below the |\childdocmain| directive):
%
\begin{center}
\begin{tabular}{l}
|\ifchilddoc|\\
|\providecommand{\version}{draft}|\\
|\||else|\\
|\providecommand{\version}{final}|\\
|\||fi|
\end{tabular}
\end{center}
%
The definition by |\providecommand| makes sure
that previous definitions are not overwritten.
Further statements |\providecommand{\version}{...}|
can thus be added before the above code to override it.

For the main file, one might add a line
(between |\childdocmain| and the above block)
%
\begin{center}
|%\ifchilddoc\||else\providecommand{\version}{draft}\||fi|
\end{center}
%
which can be uncommented to produce a draft version.
Likewise one can add a line to the very top of a child file
(above the |\childdocof{|\textit{main}|}| directive)
%
\begin{center}
|%\providecommand{\version}{final}|
\end{center}
%
which can be uncommented to produce the final version of this child document.

%%%%%%%%%%%%%%%%%%%%%%%%%%%%%%%%%%%%%%%%%%%%%%%%%%%%%%%%%%%%%%%%%%%%%%%%%%%%%%%%
\subsection{Forwarding}
\label{sec:forward}

Different versions of the main or child documents
using compilation flags as described in \secref{sec:flags}
can be (permanently) stored in different files
for convenient compilation, viewing and distribution.
To this end, the package defines a command
to pass on compilation to a different file:

%%%%%%%%%%%%%%%%%%%%%%%%%%%%%%%%%%%%%%%%
\DescribeMacro{\childdocforward}
The command |\childdocforward| redirects processing to
another source file:
%
\begin{center}
\begin{tabular}{l}
|\input{childdoc.def}|\\
|\childdocforward[|\textit{main}|]{|\textit{dest}|}|\\
\end{tabular}
\end{center}
%
The argument \textit{dest} is the destination file
(without extension).
It should be the main file or one of the child files.
Note that further \textsf{childdoc} directives
such as |\childdocof| and |\childdocforward|
in the indicated file will be processed in this form.
The optional argument \textit{main}
passes on directly to the main file \textit{main}
while pretending to compile the child \textit{dest}.
This form behaves as if \textit{dest}
issues |\childdocof{|\textit{main}|}| right away,
and no further \textsf{childdoc} directives will be processed.

%%%%%%%%%%%%%%%%%%%%%%%%%%%%%%%%%%%%%%%%
\DescribeMacro{\...prefix}
In the alternative form |\childdocforwardprefix|,
%
\begin{center}
\begin{tabular}{l}
|\input{childdoc.def}|\\
|\childdocforwardprefix[|\textit{main}|]{|\textit{prefix}|}{|\textit{dest}|}|
\end{tabular}
\end{center}
%
the destination file is determined by a pattern
depending on the current file:
To make this work, the current file must be called
`{\textit{prefix}\hspace{0.2em}\textit{suffix}}'
with \textit{prefix} matching precisely the argument.
Processing is then passed on to the file
`{\textit{dest}\hspace{0.2em}\textit{suffix}}'.
Surely, the same effect is achieved by
directly specifying the
argument `{\textit{dest}\hspace{0.2em}\textit{suffix}}'
in the first form.
However, that requires to set up a different file
for each child. With the alternative form of the command
all these files can have exactly the same content
which simplifies setting them up and maintaining them.

For example, the following file |draft.tex|
with a compilation flag |\version| as described in \secref{sec:flags}
compiles the main document as a draft:
%
\begin{center}
\begin{tabular}{l}
|\def\version{draft}|\\
|\input{childdoc.def}|\\
|\childdocforward{|\textit{main}|}|
\end{tabular}
\end{center}
%
Likewise, the following files |final|\textit{nn}|.tex|
compile the final version of the child document
|child|\textit{nn}|.tex|:
%
\begin{center}
\begin{tabular}{l}
|\def\version{final}|\\
|\input{childdoc.def}|\\
|\childdocforwardprefix{final}{child}|
\end{tabular}
\end{center}
%

Note that when several versions of a main file and/or of each child file
are to be generated, it may be convenient to set up a |Makefile| or
shell script to automatise the process.

%%%%%%%%%%%%%%%%%%%%%%%%%%%%%%%%%%%%%%%%%%%%%%%%%%%%%%%%%%%%%%%%%%%%%%%%%%%%%%%%
\subsection{Command Line Processing}
\label{sec:commandline}

The effect of redirection files can also be achieved by invoking
the \LaTeX{} compiler with a more elaborate command line.
Most conveniently this should be done as part
of a shell script or a |Makefile|.

When using \textsf{childdoc} in the main file, the following
command lines effectively perform a redirection
(note that depending on the shell being used,
backslashes may have to be doubled: `|\|' $\to$ `|\\|'):
%
\begin{center}
|... -jobname "|\textit{target}|" |\\|"|[\textit{flags}]%
|\input{childdoc.def}\childdocforward[|\textit{main}|]{|\textit{dest}|}"|
\end{center}
%
Here \textit{target} is the name of the output file,
\textit{main} is the name of the main file
and \textit{dest} is the name of the main or child file to be processed
(all filenames without extensions).
The optional argument \textit{main} can be omitted
if \textit{main} matches \textit{dest}.
Optionally, compilation \textit{flags} can be defined via |\def| commands.
This command line makes the \TeX{} engine believe
it is compiling the file \textit{target}
whose content is specified as the latter parameter.
The provided code then forwards the processing to
\textit{main} or \textit{dest} as described in \secref{sec:forward}.

%%%%%%%%%%%%%%%%%%%%%%%%%%%%%%%%%%%%%%%%%%%%%%%%%%%%%%%%%%%%%%%%%%%%%%%%%%%%%%%%
\subsection{Include by Input}
\label{sec:input}

Including child documents by |\include| has some restrictions by design.
Most notably, the content of a child document always occupies
its own set of pages; pages cannot be shared between child documents.
Usually, this behaviour makes perfect sense
because each child document contain an essential part of the document.
However, in some situations it may be desirable to compose
a document from a collection of parts
without having mandatory page breaks between then.
For this case, the package
provides a mechanism to include parts
by |\input| which can also be processed individually.
However, by construction this mechanism
requires manual handling of the content to be output.

%%%%%%%%%%%%%%%%%%%%%%%%%%%%%%%%%%%%%%%%
\DescribeMacro{\ifchilddocmanual}
The main file should be prepared as usual, see \secref{sec:include}.
However, the document body must make a distinction
between processing of an individual part and of the main document, e.g.:
%
\begin{center}
\begin{tabular}{l}
|\ifchilddocmanual|\\
|\input{\childdocname}|\\
|\||else|\\
\textit{document body with }|\input{|\textit{part}|}|\\
|\||fi|
\end{tabular}
\end{center}
%
The conditional |\ifchilddocmanual| is true whenever
a part to be included by |\input| is being compiled,
and the name of the part is stored in |\childdocname|.

%%%%%%%%%%%%%%%%%%%%%%%%%%%%%%%%%%%%%%%%
\DescribeMacro{\childdocby}
Each part to be included by |\input| should start with:
%
\begin{center}
\begin{tabular}{l}
|\input{childdoc.def}|\\
|\childdocby{|\textit{main}|}|\\
\end{tabular}
\end{center}
%
The directive |\childdocby| is similar to |\childdocof|
described in \secref{sec:include},
but the subsequent selection of content must be done manually.
To that end, both |\ifchilddoc| and |\ifchilddocmanual|
will be true upon processing of a part,
and the name of the part is stored in |\childdocname|.
Note that |\jobname| will be set to the filename of the current part
so that each part receives an individual |.aux| file
that does not interfere with the |.aux| file(s) of the main document.
This behaviour can be altered by the alternative form
|\childdocby[*]{|\textit{main}|}| (with a non-empty optional argument)
which uses the |.aux| file of the main document
by setting |\jobname| to \textit{main}.

%%%%%%%%%%%%%%%%%%%%%%%%%%%%%%%%%%%%%%%%%%%%%%%%%%%%%%%%%%%%%%%%%%%%%%%%%%%%%%%%
\subsection{Driver Development}
\label{sec:driver}

The \textsf{childdoc} mechanism can also be use for the development
of definition files such as \LaTeX{} styles or classes.
This case differs from the above setup with multiple parts
included by |\include| in that no |\includeonly| should be invoked.
This can be achieved by starting the include file
(before |\ProvidesPackage|) with:
%
\begin{center}
\begin{tabular}{l}
|\input{childdoc.def}|\\
|\childdocforward{|\textit{main}|}|\\
\end{tabular}
\end{center}
%
or alternatively with:
%
\begin{center}
\begin{tabular}{l}
|\input{childdoc.def}|\\
|\childdocby{|\textit{main}|}|\\
\end{tabular}
\end{center}
%
Both forms have slightly different effects as described above.
The main file is prepared as usual, see \secref{sec:include}.

%%%%%%%%%%%%%%%%%%%%%%%%%%%%%%%%%%%%%%%%%%%%%%%%%%%%%%%%%%%%%%%%%%%%%%%%%%%%%%%%
\subsection{Legacy Detection}
\label{sec:detection}

The directive |\childdocmain| in the main file can detect
whether the complete document or merely a child is to be compiled
even without using the directive |\childdocof|.
This method is deprecated because it is less robust
and there is no compelling reason to use it;
it is merely provided for backward compatibility
and it may be removed in future versions.

If the detection mechanism is to be used,
it is mandatory to correctly specify
the filename of the main file as the argument of |\childdocmain|:
%
\begin{center}
\begin{tabular}{l}
|\input{childdoc.def}|\\
|\childdocmain{|\textit{main}|}|\\
\end{tabular}
\end{center}
%
If |\jobname| does not match the argument \textit{main} of |\childdocmain|,
it is assumed that |\jobname| points to the child file to be compiled.
When using |\childdocmain| with the main file specified as argument,
it suffices to start a child file
with just |\input{|\textit{main}|}|
without loading of the package and using |\childdocof|.
If instead all processing is done
with the appropriate \textsf{childdoc} directives,
the argument of \textit{main} of |\childdocmain| can be empty.

An alternative version of the command line processing described
in \secref{sec:commandline} using the detection mechanism reads:
%
\begin{center}
|... -jobname "|\textit{target}|" "|[\textit{flags}]%
[|\def\jobname{|\textit{dest}|}|]|\input{|\textit{main}|}"|
\end{center}

%%%%%%%%%%%%%%%%%%%%%%%%%%%%%%%%%%%%%%%%%%%%%%%%%%%%%%%%%%%%%%%%%%%%%%%%%%%%%%%%
\subsection{Manual Code}
\label{sec:manual}

In case one cannot be certain whether the definitions file |childdoc.def|
is installed on the target \TeX{} distribution
and one prefers not to ship it,
it is conceivable to paste a few relevant commands into the sources.

To that end, drop all statements |\input{childdoc.def}|
and perform the replacements as outlined below.
Instead of |\childdocmain{|\textit{main}|}| add the following code
to the top of the main file:
%
\begin{center}
\begin{tabular}{l}
|\||ifdefined\childdocname\endinput\||fi\newif\ifchilddoc|\\
|\edef\childdocname{\scantokens\expandafter{\jobname\noexpand}}|\\
|\def\childdocmain{|\textit{main}|}\||ifx\childdocmain\childdocname\||else|\\
|\childdoctrue\includeonly{\childdocname}\let\jobname\childdocmain\||fi|\\
\end{tabular}
\end{center}
%
Instead of |\childdocof{|\textit{main}|}| just include the main file
at the top of each child file:
%
\begin{center}
|\input{|\textit{main}|}|
\end{center}
%
A simple redirection |\childdocforward{|\textit{dest}|}| is achieved by:
%
\begin{center}
|\def\jobname{|\textit{dest}|}\input{\jobname}|
\end{center}
%
The redirection with prefix
|\childdocforwardprefix[|\textit{prefix}|]{|\textit{dest}|}|
is accomplished by:
%
\begin{center}
\begin{tabular}{l}
|{\edef\jobname{\scantokens\expandafter{\jobname\noexpand}}|\\
|\def\redirectjob |\textit{prefix}|#1~~~{\gdef\jobname{|\textit{dest}|#1}}|\\
|\expandafter\redirectjob\jobname~~~}\input{\jobname}|
\end{tabular}
\end{center}

In an alternative approach,
child documents can be compiled by a specific command line
without additional code or specific definitions:
%
\begin{center}
|... -jobname "|\textit{target}|" "|[\textit{flags}]%
|\includeonly{|\textit{dest}|}\input{|\textit{main}|}"|
\end{center}
%

%%%%%%%%%%%%%%%%%%%%%%%%%%%%%%%%%%%%%%%%%%%%%%%%%%%%%%%%%%%%%%%%%%%%%%%%%%%%%%%%
%%%%%%%%%%%%%%%%%%%%%%%%%%%%%%%%%%%%%%%%%%%%%%%%%%%%%%%%%%%%%%%%%%%%%%%%%%%%%%%%
\section{Information}

%%%%%%%%%%%%%%%%%%%%%%%%%%%%%%%%%%%%%%%%%%%%%%%%%%%%%%%%%%%%%%%%%%%%%%%%%%%%%%%%
\subsection{Copyright}

Copyright \copyright{} 2017--2018 Niklas Beisert

This work may be distributed and/or modified under the
conditions of the \LaTeX{} Project Public License, either version 1.3
of this license or (at your option) any later version.
The latest version of this license is in
  \url{http://www.latex-project.org/lppl.txt}
and version 1.3 or later is part of all distributions of \LaTeX{}
version 2005/12/01 or later.

This work has the LPPL maintenance status `maintained'.

The Current Maintainer of this work is Niklas Beisert.

This work consists of the files |README.txt|, |childdoc.ins| and |childdoc.dtx|
as well as the derived files |childdoc.def|, |cdocsamp.tex|
with |cdocsch1.tex|, |cdocsch2.tex|, |cdocspt3.tex|, |cdocspt4.tex|,
|cdocsdrf.tex|, |cdocsfn1.tex|, |cdocsfn2.tex|
as well as |childdoc.pdf|.

%%%%%%%%%%%%%%%%%%%%%%%%%%%%%%%%%%%%%%%%%%%%%%%%%%%%%%%%%%%%%%%%%%%%%%%%%%%%%%%%
\subsection{Files and Installation}

The package consists of the files:
%
\begin{center}
\begin{tabular}{ll}
    |README.txt|   & readme file \\
    |childdoc.ins| & installation file \\
    |childdoc.dtx| & source file \\
    |childdoc.def| & definition file \\
    |cdocsamp.tex| & sample main file \\
    |cdocsch1.tex| & sample include file \\
    |cdocsch2.tex| & sample include file \\
    |cdocspt3.tex| & sample part file \\
    |cdocspt4.tex| & sample part file \\
    |cdocsdrf.tex| & sample redirection file \\
    |cdocsfn1.tex| & sample redirection file \\
    |cdocsfn2.tex| & sample redirection file \\
    |childdoc.pdf| & manual
\end{tabular}
\end{center}
%
The distribution consists of the files
|README.txt|, |childdoc.ins| and |childdoc.dtx|.
%
\begin{itemize}
\item
Run (pdf)\LaTeX{} on |childdoc.dtx|
to compile the manual |childdoc.pdf| (this file).
\item
Run \LaTeX{} on |childdoc.ins| to create the definitions file |childdoc.def|
and the sample |cdocsamp.tex| with include files
|cdocsch1.tex|, |cdocsch2.tex|, |cdocspt3.tex|, |cdocspt4.tex|,
|cdocsdrf.tex|, |cdocsfn1.tex|, |cdocsfn2.tex|.
Then copy the file |childdoc.def| to an appropriate directory of your \LaTeX{}
distribution, e.g.\ \textit{texmf-root}|/tex/latex/childdoc|.
\end{itemize}

%%%%%%%%%%%%%%%%%%%%%%%%%%%%%%%%%%%%%%%%%%%%%%%%%%%%%%%%%%%%%%%%%%%%%%%%%%%%%%%%
\subsection{Related CTAN Packages}

There are several other packages which offer a similar functionality:
%
\begin{itemize}
\item
The packages
\href{http://ctan.org/pkg/docmute}{\textsf{docmute}},
\href{http://ctan.org/pkg/includex}{\textsf{includex}} and
\href{http://ctan.org/pkg/standalone}{\textsf{standalone}}
provide commands to include only the document body of
a child file thus allowing both files to be compiled individually.
\item
The packages \href{http://ctan.org/pkg/subdocs}{\textsf{subdocs}}
and \href{http://ctan.org/pkg/subfiles}{\textsf{subfiles}}
provide structures in which the main and child documents can be
encapsulated and allowing them to be compiled individually.
The inclusion mechanism is different from the conventional |\include|.
\item
The package \href{http://ctan.org/pkg/combine}{\textsf{combine}}
is an elaborate solution to combine several documents into one.
\end{itemize}
%
See also the CTAN topic \href{http://ctan.org/topic/subdocs}{\textsf{subdocs}}
for further related packages.
The present package differs from the above solutions in that
a document structure constructed with the conventional |\include| mechanism
just needs two extra commands at the top of every file
such that all constituent files can be compiled individually.

%%%%%%%%%%%%%%%%%%%%%%%%%%%%%%%%%%%%%%%%%%%%%%%%%%%%%%%%%%%%%%%%%%%%%%%%%%%%%%%%
%\subsection{Feature Suggestions}
%
%The following is a list of features which may be useful for future
%versions of this package:
%%
%\begin{itemize}
%\item
%\ldots
%\end{itemize}

%%%%%%%%%%%%%%%%%%%%%%%%%%%%%%%%%%%%%%%%%%%%%%%%%%%%%%%%%%%%%%%%%%%%%%%%%%%%%%%%
\subsection{Revision History}

%%%%%%%%%%%%%%%%%%%%%%%%%%%%%%%%%%%%%%%%
\paragraph{v2.0:} 2018/12/30

\begin{itemize}
\item
immediate forward processing
\item
added |\childdocby| mechanism
\item
manual restructured
\end{itemize}

%%%%%%%%%%%%%%%%%%%%%%%%%%%%%%%%%%%%%%%%
\paragraph{v1.6:} 2018/01/17

\begin{itemize}
\item
application for development of include files
\item
corrections to manual
\end{itemize}

%%%%%%%%%%%%%%%%%%%%%%%%%%%%%%%%%%%%%%%%
\paragraph{v1.5:} 2017/05/21

\begin{itemize}
\item
more complete structuring introduced
\item
|\childdocof| introduced
\item
|\childdoc| renamed to |\childdocmain|
\item
|\childredirect| renamed to |\childdocforward| and |\childdocforwardprefix|
and functionality expanded
\end{itemize}

%%%%%%%%%%%%%%%%%%%%%%%%%%%%%%%%%%%%%%%%
\paragraph{v1.0:} 2017/04/27

\begin{itemize}
\item
manual and install package
\item
first version published on CTAN
\end{itemize}

%%%%%%%%%%%%%%%%%%%%%%%%%%%%%%%%%%%%%%%%
\paragraph{v0.6:} 2017/04/26

\begin{itemize}
\item
redirection mechanism added
\end{itemize}

%%%%%%%%%%%%%%%%%%%%%%%%%%%%%%%%%%%%%%%%
\paragraph{v0.5:} 2017/04/26

\begin{itemize}
\item
functionality in definition file
\end{itemize}


%%%%%%%%%%%%%%%%%%%%%%%%%%%%%%%%%%%%%%%%%%%%%%%%%%%%%%%%%%%%%%%%%%%%%%%%%%%%%%%%
%%%%%%%%%%%%%%%%%%%%%%%%%%%%%%%%%%%%%%%%%%%%%%%%%%%%%%%%%%%%%%%%%%%%%%%%%%%%%%%%
%%%%%%%%%%%%%%%%%%%%%%%%%%%%%%%%%%%%%%%%%%%%%%%%%%%%%%%%%%%%%%%%%%%%%%%%%%%%%%%%
\appendix

\settowidth\MacroIndent{\rmfamily\scriptsize 000\ }

 \DocInput{childdoc.dtx}

\end{document}
%</driver>
% \fi
%
% %%%%%%%%%%%%%%%%%%%%%%%%%%%%%%%%%%%%%%%%%%%%%%%%%%%%%%%%%%%%%%%%%%%%%%%%%%%%%%
% %%%%%%%%%%%%%%%%%%%%%%%%%%%%%%%%%%%%%%%%%%%%%%%%%%%%%%%%%%%%%%%%%%%%%%%%%%%%%%
% \section{Sample}
%\iffalse
%<*samplemain>
%\fi
%
% The following presents a sample document
% with two chapters, two parts, a title page,
% a compile flag as well as three forwarding files to set the flag.
% It consists of eight |.tex| files:
% \begin{center}
% \begin{tabular}{ll}
% |cdocsamp.tex|&main file\\
% |cdocsch1.tex|&include file for chapter 1\\
% |cdocsch2.tex|&include file for chapter 2\\
% |cdocspt3.tex|&include file for part 3\\
% |cdocspt4.tex|&include file for part 4\\
% |cdocsdrf.tex|&forwarding file for main file in draft mode\\
% |cdocsfi1.tex|&forwarding file for final version of chapter 1\\
% |cdocsfi2.tex|&forwarding file for final version of chapter 2\\
% \end{tabular}
% \end{center}
% Each of the eight files can be compiled directly by the \LaTeX{} compiler.
%
% %%%%%%%%%%%%%%%%%%%%%%%%%%%%%%%%%%%%%%
% \paragraph{Main File.}
%
% The main file is called |cdocsamp.tex|.
%
% Load the \textsf{childdoc} definitions and
% declare the filename for the main document:
%    \begin{macrocode}
\input{childdoc.def}
\childdocmain{}
%    \end{macrocode}

% Optional override for |\version| flag:
%    \begin{macrocode}
%%\ifchilddoc\else\providecommand{\version}{draft}\fi
%    \end{macrocode}

% Define the default values for the |\version| flag
% (|final| for the main file and |draft| for childs):
%    \begin{macrocode}
\ifchilddoc
\providecommand{\version}{draft}
\else
\providecommand{\version}{final}
\fi
%    \end{macrocode}

% Load the standard document class:
%    \begin{macrocode}
\documentclass[12pt]{article}
%    \end{macrocode}

% Start the document body:
%    \begin{macrocode}
\begin{document}
%    \end{macrocode}

% Declare a title page.
% Print title, part of document being processed and version flag:
%    \begin{macrocode}
\addtocounter{page}{-1}
\begin{center}
{\LARGE\bfseries{}childdoc example\par}
\vspace{1cm}
\ifchilddoc
\ifchilddocmanual part\else chapter\fi:
`\childdocname' of `\childdocjob'\par
\else
main document: `\childdocjob'\par
\fi
version: \version\par
\end{center}
\newpage
%    \end{macrocode}

% Manually include selected file,
% otherwise process as usual:
%    \begin{macrocode}
\ifchilddocmanual
\section*{part `\childdocname'}
\input{\childdocname}
\else
%    \end{macrocode}

% Include the two chapters:
%    \begin{macrocode}
\include{cdocsch1}
\include{cdocsch2}
%    \end{macrocode}

% Include the two parts unless only chapters should be displayed:
%    \begin{macrocode}
\ifchilddoc\else
\section{part three}
\input{cdocspt3}
\section{part four}
\input{cdocspt4}
\fi
%    \end{macrocode}

% Process as usual until here:
%    \begin{macrocode}
\fi
%    \end{macrocode}

% End of document body:
%    \begin{macrocode}
\end{document}
%    \end{macrocode}
%\iffalse
%</samplemain>
%\fi
%
% %%%%%%%%%%%%%%%%%%%%%%%%%%%%%%%%%%%%%%
% \paragraph{Chapter Include Files.}
%
% The include files are called |cdocsch1.tex| and |cdocsch2.tex|.
%
%\iffalse
%<*samplechap1|samplechap2>
%\fi

% Optional override for |\version| flag:
%    \begin{macrocode}
%%\providecommand{\version}{final}
%    \end{macrocode}

% Include the main document:
%    \begin{macrocode}
\input{childdoc.def}
\childdocof{cdocsamp}
%    \end{macrocode}

%\iffalse
%</samplechap1|samplechap2>
%\fi
%
%\iffalse
%<*samplechap1>
%\fi
% Some text for chapter 1:
%    \begin{macrocode}
\section{one}
some text in chapter one
%    \end{macrocode}

%\iffalse
%</samplechap1>
%\fi
% Some text for chapter 2:
%\iffalse
%<*samplechap2>
%\fi
%    \begin{macrocode}
\section{two}
more text in chapter two
%    \end{macrocode}

%\iffalse
%</samplechap2>
%\fi
%
% %%%%%%%%%%%%%%%%%%%%%%%%%%%%%%%%%%%%%%
% \paragraph{Part Include Files.}
%
% The include files are called |cdocspt3.tex| and |cdocspt4.tex|.
%
%\iffalse
%<*samplepart3|samplepart4>
%\fi

% Optional override for |\version| flag:
%    \begin{macrocode}
%%\providecommand{\version}{final}
%    \end{macrocode}

% Include the main document:
%    \begin{macrocode}
\input{childdoc.def}
\childdocby{cdocsamp}
%    \end{macrocode}

%\iffalse
%</samplepart3|samplepart4>
%\fi
%
%\iffalse
%<*samplepart3>
%\fi
% Some text for part 3:
%    \begin{macrocode}
some text in part three
%    \end{macrocode}

%\iffalse
%</samplepart3>
%\fi
% Some text for part 4:
%\iffalse
%<*samplepart4>
%\fi
%    \begin{macrocode}
more text in part four
%    \end{macrocode}

%\iffalse
%</samplepart4>
%\fi
%
% %%%%%%%%%%%%%%%%%%%%%%%%%%%%%%%%%%%%%%
% \paragraph{Forwarding for a Complete Draft.}
%
% The following forwarding file |cdocsdrf.tex|
% compiles the main document in draft mode:
%\iffalse
%<*sampledraft>
%\fi
%    \begin{macrocode}
\def\version{draft}
\input{childdoc.def}
\childdocforward{cdocsamp}
%    \end{macrocode}

%\iffalse
%</sampledraft>
%\fi
%
% %%%%%%%%%%%%%%%%%%%%%%%%%%%%%%%%%%%%%%
% \paragraph{Forwarding for Final Version of the Chapters.}
%
% The following forwarding files |cdocsfn1.tex| and |cdocsfn2.tex|
% (with identical content)
% compile the final versions of the child documents
% |cdocsch1.tex| and |cdocsch2.tex|, respectively:
%\iffalse
%<*samplefinal>
%\fi
%    \begin{macrocode}
\def\version{final}
\input{childdoc.def}
\childdocforwardprefix[cdocsamp]{cdocsfn}{cdocsch}
%    \end{macrocode}

%\iffalse
%</samplefinal>
%\fi
%
% %%%%%%%%%%%%%%%%%%%%%%%%%%%%%%%%%%%%%%
% \paragraph{Command Line Processing.}
%
% The following three command lines generate the output files
% |cdocscld|, |cdocscl1| and |cdocscl2|
% which should be identical to
% |cdocsdrf|, |cdocsch1| and |cdocsfn2|, respectively:
% \begin{center}
% \begin{tabular}{l}
% |latex -jobname cdocscld \|\\
% |  "\def\version{draft}\input{childdoc.def}\childdocforward{cdocsamp}"|\\
% |latex -jobname cdocscl1 \|\\
% |  "\input{childdoc.def}\childdocforward[cdocsamp]{cdocsch1}"|\\
% |latex -jobname cdocscl2 \|\\
% |  "\def\version{final}\input{childdoc.def}\childdocforward{cdocsch2}"|
% \end{tabular}
% \end{center}
% Note that the trailing backslash on each first line
% merely continues the input to the second line
% (for convenient cut ant paste).
% Furthermore, the command |latex| can be replaced by any
% of its alternative versions such as |pdflatex|.
%
% %%%%%%%%%%%%%%%%%%%%%%%%%%%%%%%%%%%%%%%%%%%%%%%%%%%%%%%%%%%%%%%%%%%%%%%%%%%%%%
% %%%%%%%%%%%%%%%%%%%%%%%%%%%%%%%%%%%%%%%%%%%%%%%%%%%%%%%%%%%%%%%%%%%%%%%%%%%%%%
% \section{Implementation}
%\iffalse
%<*package>
%\fi
%
% This section describes the definitions file |childdoc.def|.

% The definitions cannot be loaded using |\usepackage| or |\RequirePackage|
% which has a mechanism to prevent loading a style file more than once.
% When loading the definitions by means of |\input|
% multiple instances have to be prevented manually:
%\iffalse
%This code needs to be before the `\ProvidesFile' directive
%which is defined at the beginning of this file.
%Therefore it is also placed there and commented out here.
%</package>
%<*discard>
%\fi
%    \begin{macrocode}
\ifdefined\childdocmain\endinput\fi
%    \end{macrocode}
%\iffalse
%</discard>
%<*package>
%\fi
%
% \macro{\ifchilddoc}
% \macro{\ifchilddocmanual}
% The conditional |\ifchilddoc| tells whether a
% child (true) or main (false) document is being compiled.
% The conditional |\ifchilddocmanual| tells whether
% the |\includeonly| mechanism is used (false) or
% the selection of child files must be performed manually (true).
% The definitions initialise to false:
%    \begin{macrocode}
\newif\ifchilddoc
\newif\ifchilddocmanual
%    \end{macrocode}

% \macro{\childdocname}
% \macro{\childdocjob}
% The macro |\childdocname| stores the name of the main document
% to be compiled. The macro |\childdocjob| stores the name of
% the document on which the \LaTeX{} compiler was originally invoked.
% The content of |\jobname| cannot be compared
% to filenames specified in the source due to different catcodes.
% The following code rescans |\jobname|, stores the result
% in |\childdocname| and saves a copy in |\childdocjob|:
%    \begin{macrocode}
\edef\childdocname{\scantokens\expandafter{\jobname\noexpand}}
\let\childdocjob\childdocname
%    \end{macrocode}

% \macro{\childdocdisable}
% The macro |\childdocdisable| prevents the main file
% from being processed more than once.
% At this stage, the main document command |\childdocmain|
% is assumed to be called once again where it should do nothing.
% Any subsequent call to it should prevent
% a secondary processing of the main document
% It overwrites the forwarding commands
% |\childdocof| and |\childdocforward|
% with empty macros to prevent further inclusions of the main document:
%    \begin{macrocode}
\newcommand{\childdocdisable}
{
  \renewcommand{\childdocmain}[1]{\renewcommand{\childdocmain}[1]{\endinput}}
  \renewcommand{\childdocof}[1]{}
  \renewcommand{\childdocby}[2][]{}
  \renewcommand{\childdocforward}[2][]{}
  \renewcommand{\childdocdisable}{}
}
%    \end{macrocode}

% \macro{\childdocmain}
% The macro |\childdocmain| is to be called at the top of the main file
% with nothing or the main filename (without extension) as argument.
% First, it breaks loops.
% If the argument is not empty and does not match |\childdocname|
% (which is set by the first inclusion of |childdoc.def|),
% |\ifchilddoc| is set to true, |\includeonly| is applied to the child file
% and |\jobname| is set to the main file
% (for proper handling of |.aux| files):
%    \begin{macrocode}
\newcommand{\childdocmain}[1]
{
  \childdocdisable\childdocmain{}
  \if?#1?\else
    \begingroup
      \def\childdoctmp{#1}
      \ifx\childdoctmp\childdocname
        \def\childdoctmp{}
      \else
        \def\childdoctmp
        {
          \childdoctrue
          \includeonly{\childdocname}
          \def\childdocjob{#1}
          \def\jobname{#1}
        }
      \fi
      \expandafter
    \endgroup
    \childdoctmp
  \fi
}
%    \end{macrocode}

% \macro{\childdocof}
% The command |\childdocof| redirects
% compilation to the main file |#1|.
%    \begin{macrocode}
\newcommand{\childdocof}[1]
{
  \childdocdisable
  \childdoctrue
  \includeonly{\childdocname}
  \def\jobname{#1}
  \def\childdocjob{#1}
  \input{#1}
}
%    \end{macrocode}

% \macro{\childdocby}
% The command |\childdocby| ....
%    \begin{macrocode}
\newcommand{\childdocby}[2][]
{
  \childdocdisable
  \childdoctrue
  \childdocmanualtrue
  \if?#1?\else
    \def\jobname{#2}
  \fi
  \def\childdocjob{#2}
  \input{#2}
  \endinput
}
%    \end{macrocode}

% \macro{\childdocforward}
% The command |\childdocforward| redirects
% compilation to the main file or
% (if the optional argument is given) a child file.
% Parameters are set as if the main file
% or a child file starting with |\childdocof| was compiled.
% Then compilation is handed over to the main file:
%    \begin{macrocode}
\newcommand{\childdocforward}[2][]
{
  \begingroup
    \if?#1?
      \def\childdoctmp
      {
        \def\childdocname{#2}
        \def\childdocjob{#2}
        \def\jobname{#2}
        \input{#2}
        \endinput
      }
    \else
      \def\childdoctmp
      {
        \childdocdisable
        \def\childdocname{#2}
        \childdoctrue
        \includeonly{#2}
        \def\childdocjob{#1}
        \def\jobname{#1}
        \input{#1}
        \endinput
      }
    \fi
    \expandafter
  \endgroup
  \childdoctmp
}
%    \end{macrocode}

% \macro{\childdocforwardprefix}
% The command |\childdocforwardprefix| redirects
% compilation to the main or a child file by means of a pattern.
% The prefix |#1| in the current filename is replaced by |#2|
% and the suffix of the current filename is kept
% (it is assumed that the filename does not contain the substring `|~~~|'
% which is used as a delimiter).
% Compilation is handed over to the new file by |\childdocforward|:
%    \begin{macrocode}
\newcommand{\childdocforwardprefix}[3][]
{
  \begingroup
    \def\childdocextract #2##1~~~{\def\childdoctmp{\childdocforward[#1]{#3##1}}}
    \expandafter\childdocextract\childdocname~~~
    \expandafter
  \endgroup
  \childdoctmp
}
%    \end{macrocode}

% \macro{\childdoc}
% The deprecated macro |\childdoc| is a legacy version of |\childdocmain|:
%    \begin{macrocode}
\newcommand{\childdoc}{\childdocmain}
%    \end{macrocode}

% \macro{\childdocredirect}
% The deprecated macro |\childdocredirect| is a legacy version
% of |\childdocforward| and |\childdocforwardprefix|:
%    \begin{macrocode}
\newcommand{\childdocredirect}[2][]
{
  \begingroup
    \if?#1?
      \def\childdoctmp{\childdocforward{#2}}
    \else
      \def\childdoctmp{\childdocforwardprefix{#1}{#2}}
    \fi
    \expandafter
  \endgroup
  \childdoctmp
}
%    \end{macrocode}

%\iffalse
%</package>
%\fi
%
\endinput

\childdocforward{cdocsamp}
%    \end{macrocode}

%\iffalse
%</sampledraft>
%\fi
%
% %%%%%%%%%%%%%%%%%%%%%%%%%%%%%%%%%%%%%%
% \paragraph{Forwarding for Final Version of the Chapters.}
%
% The following forwarding files |cdocsfn1.tex| and |cdocsfn2.tex|
% (with identical content)
% compile the final versions of the child documents
% |cdocsch1.tex| and |cdocsch2.tex|, respectively:
%\iffalse
%<*samplefinal>
%\fi
%    \begin{macrocode}
\def\version{final}
% \iffalse
%
% childdoc.dtx Copyright (C) 2017-2018 Niklas Beisert
%
% This work may be distributed and/or modified under the
% conditions of the LaTeX Project Public License, either version 1.3
% of this license or (at your option) any later version.
% The latest version of this license is in
%   http://www.latex-project.org/lppl.txt
% and version 1.3 or later is part of all distributions of LaTeX
% version 2005/12/01 or later.
%
% This work has the LPPL maintenance status `maintained'.
%
% The Current Maintainer of this work is Niklas Beisert.
%
% This work consists of the files childdoc.dtx and childdoc.ins
% and the derived files childdoc.def and cdocsamp.tex with
% cdocsch1.tex, cdocsch2.tex, cdocsdrf.tex, cdocsfn1.tex, cdocsfn2.tex.
%
%<package>\ifdefined\childdocmain\endinput\fi
%<package>\ProvidesFile{childdoc.def}[2018/12/30 v2.0 child document driver]
%<samplemain>\ProvidesFile{cdocsamp.tex}[2018/12/30 v2.0 sample for childdoc]
%<*driver>
%\ProvidesFile{childdoc.drv}[2018/12/30 v2.0 childdoc reference manual file]
\PassOptionsToClass{10pt,a4paper}{article}
\documentclass{ltxdoc}

\usepackage[margin=35mm]{geometry}
\usepackage{hyperref}
\usepackage{hyperxmp}
\usepackage[usenames]{color}

\hypersetup{colorlinks=true}
\hypersetup{pdfstartview=FitH}
\hypersetup{pdfpagemode=UseNone}
\hypersetup{pdfsource={}}
\hypersetup{pdflang={en-UK}}
\hypersetup{pdfcopyright={Copyright 2017-2018 Niklas Beisert.
  This work may be distributed and/or modified under the
  conditions of the LaTeX Project Public License, either version 1.3
  of this license or (at your option) any later version.}}
\hypersetup{pdflicenseurl={http://www.latex-project.org/lppl.txt}}
\hypersetup{pdfcontactaddress={ETH Zurich, ITP, HIT K,
  Wolfgang-Pauli-Strasse 27}}
\hypersetup{pdfcontactpostcode={8093}}
\hypersetup{pdfcontactcity={Zurich}}
\hypersetup{pdfcontactcountry={Switzerland}}
\hypersetup{pdfcontactemail={nbeisert@itp.phys.ethz.ch}}
\hypersetup{pdfcontacturl={http://people.phys.ethz.ch/\xmptilde nbeisert/}}

\newcommand{\secref}[1]{\hyperref[#1]{section \ref*{#1}}}

\parskip1ex
\parindent0pt
\let\olditemize\itemize
\def\itemize{\olditemize\parskip0pt}

\begin{document}

\title{The \textsf{childdoc} Package}
\hypersetup{pdftitle={The childdoc Package}}
\author{Niklas Beisert\\[2ex]
  Institut f\"ur Theoretische Physik\\
  Eidgen\"ossische Technische Hochschule Z\"urich\\
  Wolfgang-Pauli-Strasse 27, 8093 Z\"urich, Switzerland\\[1ex]
  \href{mailto:nbeisert@itp.phys.ethz.ch}
  {\texttt{nbeisert@itp.phys.ethz.ch}}}
\hypersetup{pdfauthor={Niklas Beisert}}
\hypersetup{pdfsubject={Manual for the LaTeX2e Package childdoc}}
\date{30 December 2018, \textsf{v2.0}}
\maketitle

\begin{abstract}\noindent
\textsf{childdoc} is a \LaTeXe{} package
that enables the direct compilation
of document sections included by |\include|
to individual files.
\end{abstract}

\begingroup
\parskip0ex
\tableofcontents
\endgroup

%%%%%%%%%%%%%%%%%%%%%%%%%%%%%%%%%%%%%%%%%%%%%%%%%%%%%%%%%%%%%%%%%%%%%%%%%%%%%%%%
%%%%%%%%%%%%%%%%%%%%%%%%%%%%%%%%%%%%%%%%%%%%%%%%%%%%%%%%%%%%%%%%%%%%%%%%%%%%%%%%
\section{Introduction}

\LaTeX{} provides a mechanism to structure a large document (such as a book)
into a main file and several child files (containing the chapters)
using the |\include| command.
This mechanism is beneficial for documents
which span hundreds of pages in order to
make the source file(s) more manageable.
Moreover, compilation can be restricted to
selected child files by means of the |\includeonly| command.
The latter feature can be used to reduce the compilation time while editing
(this was significantly more useful in the earlier days of \LaTeX{})
or to generate a smaller document which is easier to navigate.
Another application of |\includeonly| is to generate
documents consisting of selected parts of the complete document.

However, there are a few drawbacks of the plain |\include| mechanism:
\begin{itemize}
\item
The child files cannot be compiled on their own,
they can only be compiled via the main file.
A naive editing environment
(such as a text editor with an option
to have the current file processed by \LaTeX)
may require one to switch to the main file before compiling;
attempting to compile the child file produces errors.
\item
The main file must be modified (each time)
to adjust the |\includeonly| command
to the present needs. This easily leaves the main file in a messy state.
\item
The generated document will always carry the filename
of the main document. This is inconvenient if
several child files are to be compiled and
to be kept for distribution.
\end{itemize}

The present package provides a simple interface
to make child files individually compilable by \LaTeX{}.
Compiling a child file then has the same effect as compiling
the main file with an |\includeonly| command
to select the appropriate child.
Moreover the generated document will carry the name of the child
rather than the main file.
This resolves all three above issues.

This feature is meant to make the editing of books,
thesis documents and lecture notes somewhat more convenient.
However, the package can also be used efficiently for
composing a series of documents (such as exercise sheets)
which are typically distributed individually.
It then assists the author in generating the individual documents
(potentially in different versions)
as well as a document containing the collected series.
Another application is in developing style files
or other kinds of included material
where compilation of the style file could redirect
to a sample or test file.

%%%%%%%%%%%%%%%%%%%%%%%%%%%%%%%%%%%%%%%%%%%%%%%%%%%%%%%%%%%%%%%%%%%%%%%%%%%%%%%%
%%%%%%%%%%%%%%%%%%%%%%%%%%%%%%%%%%%%%%%%%%%%%%%%%%%%%%%%%%%%%%%%%%%%%%%%%%%%%%%%
\section{Usage}

First of all, the package \textsf{childdoc} is \emph{not} a standard
\LaTeXe{} |.sty| style file! Therefore it needs to be invoked in
a non-standard way.

%%%%%%%%%%%%%%%%%%%%%%%%%%%%%%%%%%%%%%%%%%%%%%%%%%%%%%%%%%%%%%%%%%%%%%%%%%%%%%%%
\subsection{Included Files}
\label{sec:include}

%%%%%%%%%%%%%%%%%%%%%%%%%%%%%%%%%%%%%%%%
\DescribeMacro{\childdocmain}
To use the package, add the commands
\begin{center}
\begin{tabular}{l}
|\input{childdoc.def}|\\
|\childdocmain{}|\\
\end{tabular}
\end{center}
at the very top of the main \LaTeX{} file,
in particular \emph{before} the |\documentclass| statement!
The argument of |\childdocmain| should be left empty
(but it must be present).

%%%%%%%%%%%%%%%%%%%%%%%%%%%%%%%%%%%%%%%%
\DescribeMacro{\childdocof}
Furthermore, add the commands
\begin{center}
\begin{tabular}{l}
|\input{childdoc.def}|\\
|\childdocof{|\textit{main}|}|\\
\end{tabular}
\end{center}
at the top of every child file \textit{child}
which is included by |\include{|\textit{child}|}|
from within the main file
(or at least for those files to be compiled individually).
The argument \textit{main} must be the filename of the main file.

There are a couple of
considerations in setting up the main and child documents:

%%%%%%%%%%%%%%%%%%%%%%%%%%%%%%%%%%%%%%%%
\paragraph{Restrictions.}

Please note the following restrictions:
\begin{itemize}
\item
|\childdocmain| must be called with one argument \textit{main}
to ensure compatibility with earlier version of the package.
It must either be empty (|\childdocmain{}|)
or precisely match the filename of the main file in which it is specified.
See \secref{sec:detection} for further information.
\item
The filename \textit{main} must be specified without the |.tex| extension.
\item
The filename \textit{main} is case sensitive
(even in case-insensitive file systems)
due to internal string comparison.
\item
The argument \textit{main} should be fully expanded, it cannot be a macro.
\item
Subdirectories and special characters should be avoided in filenames.
\item
The command |\childdocmain{|\textit{main}|}| must be followed by a whitespace.
It should not be followed immediately by another command
or by a comment mark `|%|'.
This is because the \TeX{} parser reads the token immediately following
the argument of |\childdocmain| and puts it
at the beginning of every child section;
however, a white\-space is ignored.
\end{itemize}

%%%%%%%%%%%%%%%%%%%%%%%%%%%%%%%%%%%%%%%%
\paragraph{Content of Main File.}

It is advisable to place all content in the child files included by |\include|.
Any output contained in the main file will appear in all child documents
unless suppressed manually;
it cannot be suppressed automatically by the |\includeonly| directive
and thus should normally be avoided.
A method to include some content in the main file
by means of conditional processing is described in \secref{sec:conditional}.

%%%%%%%%%%%%%%%%%%%%%%%%%%%%%%%%%%%%%%%%
\paragraph{Page Numbering.}

When only a part of the document is compiled,
the appropriate numbering of pages
(as well as other status parameters)
is determined from the |.aux| files.
The latter contain information from previous passes.
However this information needs to propagate through
all intermediate child documents.
Therefore the page numbering in child documents may well
be inconsistent until the complete document is compiled at least once.

A useful (if unconventional) way to always ensure a consistent
page numbering is to restart the numbering in each child document
and denote the pages by `\textit{child}|.|\textit{page}'
where \textit{child} represents the chapter/section number of the child file.
This can be achieved by the command
|\numberwithin{page}{|\textit{child}|}|
of the \textsf{amsmath} package
where \textit{child} can be |chapter| or |section|
depending on the chosen structuring.
Alternatively, one can modify the macro |\thepage| appropriately
and reset the counter |page| at the start of each child file.

%%%%%%%%%%%%%%%%%%%%%%%%%%%%%%%%%%%%%%%%%%%%%%%%%%%%%%%%%%%%%%%%%%%%%%%%%%%%%%%%
\subsection{Conditional Processing}
\label{sec:conditional}

The package provides a mechanism to compile different versions
of a document. To customise the versions further some conditional processing
can come in handy to distinguish which version is being compiled.
The package provides two macros to describe the compilation context:

%%%%%%%%%%%%%%%%%%%%%%%%%%%%%%%%%%%%%%%%
\DescribeMacro{\ifchilddoc}
The conditional |\ifchilddoc| distinguishes between the compilation of
child documents and the main document:
%
\begin{center}
|\ifchilddoc |\textit{child-code}| |[|\||else |\textit{main-code}]| \||fi|
\end{center}

%%%%%%%%%%%%%%%%%%%%%%%%%%%%%%%%%%%%%%%%
\DescribeMacro{\childdocname}
\DescribeMacro{\childdocjob}
The macro |\childdocname| contains the filename (without extension)
of the main or child file being processed.
Note that |\childdocjob| will always contain the name of the main file.

%%%%%%%%%%%%%%%%%%%%%%%%%%%%%%%%%%%%%%%%
\paragraph{Title Page.}

Conditional processing can be used to include a title or banner page
in the main document when proper precautions are taken.
Importantly, the code in the main file should ensure that the page counter
(as well as other status parameters which are stored in the |.aux| files)
takes the same value after the conditional processing.
Otherwise the page numbers may take divergent values
depending on which part is compiled.

For example, a title page could be declared by:
%
\begin{center}
\begin{tabular}{l}
|\ifchilddoc\||else|\\
|\addtocounter{page}{-1}|\\
\textit{code for title page}\\
|\newpage|\\
|\||fi|
\end{tabular}
\end{center}
%
A banner page for the child documents can be generated by:
%
\begin{center}
\begin{tabular}{l}
|\ifchilddoc|\\
|\addtocounter{page}{-1}|\\
\textit{code for banner page}\\
|\newpage|\\
|\||fi|
\end{tabular}
\end{center}
%
Here one could write a message such as:
\begin{center}
|This is the part \childdocname{} of \childdocjob{}.|
\end{center}

%%%%%%%%%%%%%%%%%%%%%%%%%%%%%%%%%%%%%%%%%%%%%%%%%%%%%%%%%%%%%%%%%%%%%%%%%%%%%%%%
\subsection{Flags}
\label{sec:flags}

The package makes it easy to generate different versions
of the main or child documents.
To this end compilation flags can be defined
and assigned different default values.
They will be particularly useful in conjunction
with the forwarding mechanism described in \secref{sec:forward}.

For example, it may be useful to have a flag |\version|
which can be set to |draft| or |final|.
The document source will contain some conditional code
depending on the value of |\version|.
Suppose further, the flag should default to |final| for the main file
and to |draft| for child files
which is a natural assignment for editing the document.
This is achieved by placing the following code
in the preamble of the main document
(below the |\childdocmain| directive):
%
\begin{center}
\begin{tabular}{l}
|\ifchilddoc|\\
|\providecommand{\version}{draft}|\\
|\||else|\\
|\providecommand{\version}{final}|\\
|\||fi|
\end{tabular}
\end{center}
%
The definition by |\providecommand| makes sure
that previous definitions are not overwritten.
Further statements |\providecommand{\version}{...}|
can thus be added before the above code to override it.

For the main file, one might add a line
(between |\childdocmain| and the above block)
%
\begin{center}
|%\ifchilddoc\||else\providecommand{\version}{draft}\||fi|
\end{center}
%
which can be uncommented to produce a draft version.
Likewise one can add a line to the very top of a child file
(above the |\childdocof{|\textit{main}|}| directive)
%
\begin{center}
|%\providecommand{\version}{final}|
\end{center}
%
which can be uncommented to produce the final version of this child document.

%%%%%%%%%%%%%%%%%%%%%%%%%%%%%%%%%%%%%%%%%%%%%%%%%%%%%%%%%%%%%%%%%%%%%%%%%%%%%%%%
\subsection{Forwarding}
\label{sec:forward}

Different versions of the main or child documents
using compilation flags as described in \secref{sec:flags}
can be (permanently) stored in different files
for convenient compilation, viewing and distribution.
To this end, the package defines a command
to pass on compilation to a different file:

%%%%%%%%%%%%%%%%%%%%%%%%%%%%%%%%%%%%%%%%
\DescribeMacro{\childdocforward}
The command |\childdocforward| redirects processing to
another source file:
%
\begin{center}
\begin{tabular}{l}
|\input{childdoc.def}|\\
|\childdocforward[|\textit{main}|]{|\textit{dest}|}|\\
\end{tabular}
\end{center}
%
The argument \textit{dest} is the destination file
(without extension).
It should be the main file or one of the child files.
Note that further \textsf{childdoc} directives
such as |\childdocof| and |\childdocforward|
in the indicated file will be processed in this form.
The optional argument \textit{main}
passes on directly to the main file \textit{main}
while pretending to compile the child \textit{dest}.
This form behaves as if \textit{dest}
issues |\childdocof{|\textit{main}|}| right away,
and no further \textsf{childdoc} directives will be processed.

%%%%%%%%%%%%%%%%%%%%%%%%%%%%%%%%%%%%%%%%
\DescribeMacro{\...prefix}
In the alternative form |\childdocforwardprefix|,
%
\begin{center}
\begin{tabular}{l}
|\input{childdoc.def}|\\
|\childdocforwardprefix[|\textit{main}|]{|\textit{prefix}|}{|\textit{dest}|}|
\end{tabular}
\end{center}
%
the destination file is determined by a pattern
depending on the current file:
To make this work, the current file must be called
`{\textit{prefix}\hspace{0.2em}\textit{suffix}}'
with \textit{prefix} matching precisely the argument.
Processing is then passed on to the file
`{\textit{dest}\hspace{0.2em}\textit{suffix}}'.
Surely, the same effect is achieved by
directly specifying the
argument `{\textit{dest}\hspace{0.2em}\textit{suffix}}'
in the first form.
However, that requires to set up a different file
for each child. With the alternative form of the command
all these files can have exactly the same content
which simplifies setting them up and maintaining them.

For example, the following file |draft.tex|
with a compilation flag |\version| as described in \secref{sec:flags}
compiles the main document as a draft:
%
\begin{center}
\begin{tabular}{l}
|\def\version{draft}|\\
|\input{childdoc.def}|\\
|\childdocforward{|\textit{main}|}|
\end{tabular}
\end{center}
%
Likewise, the following files |final|\textit{nn}|.tex|
compile the final version of the child document
|child|\textit{nn}|.tex|:
%
\begin{center}
\begin{tabular}{l}
|\def\version{final}|\\
|\input{childdoc.def}|\\
|\childdocforwardprefix{final}{child}|
\end{tabular}
\end{center}
%

Note that when several versions of a main file and/or of each child file
are to be generated, it may be convenient to set up a |Makefile| or
shell script to automatise the process.

%%%%%%%%%%%%%%%%%%%%%%%%%%%%%%%%%%%%%%%%%%%%%%%%%%%%%%%%%%%%%%%%%%%%%%%%%%%%%%%%
\subsection{Command Line Processing}
\label{sec:commandline}

The effect of redirection files can also be achieved by invoking
the \LaTeX{} compiler with a more elaborate command line.
Most conveniently this should be done as part
of a shell script or a |Makefile|.

When using \textsf{childdoc} in the main file, the following
command lines effectively perform a redirection
(note that depending on the shell being used,
backslashes may have to be doubled: `|\|' $\to$ `|\\|'):
%
\begin{center}
|... -jobname "|\textit{target}|" |\\|"|[\textit{flags}]%
|\input{childdoc.def}\childdocforward[|\textit{main}|]{|\textit{dest}|}"|
\end{center}
%
Here \textit{target} is the name of the output file,
\textit{main} is the name of the main file
and \textit{dest} is the name of the main or child file to be processed
(all filenames without extensions).
The optional argument \textit{main} can be omitted
if \textit{main} matches \textit{dest}.
Optionally, compilation \textit{flags} can be defined via |\def| commands.
This command line makes the \TeX{} engine believe
it is compiling the file \textit{target}
whose content is specified as the latter parameter.
The provided code then forwards the processing to
\textit{main} or \textit{dest} as described in \secref{sec:forward}.

%%%%%%%%%%%%%%%%%%%%%%%%%%%%%%%%%%%%%%%%%%%%%%%%%%%%%%%%%%%%%%%%%%%%%%%%%%%%%%%%
\subsection{Include by Input}
\label{sec:input}

Including child documents by |\include| has some restrictions by design.
Most notably, the content of a child document always occupies
its own set of pages; pages cannot be shared between child documents.
Usually, this behaviour makes perfect sense
because each child document contain an essential part of the document.
However, in some situations it may be desirable to compose
a document from a collection of parts
without having mandatory page breaks between then.
For this case, the package
provides a mechanism to include parts
by |\input| which can also be processed individually.
However, by construction this mechanism
requires manual handling of the content to be output.

%%%%%%%%%%%%%%%%%%%%%%%%%%%%%%%%%%%%%%%%
\DescribeMacro{\ifchilddocmanual}
The main file should be prepared as usual, see \secref{sec:include}.
However, the document body must make a distinction
between processing of an individual part and of the main document, e.g.:
%
\begin{center}
\begin{tabular}{l}
|\ifchilddocmanual|\\
|\input{\childdocname}|\\
|\||else|\\
\textit{document body with }|\input{|\textit{part}|}|\\
|\||fi|
\end{tabular}
\end{center}
%
The conditional |\ifchilddocmanual| is true whenever
a part to be included by |\input| is being compiled,
and the name of the part is stored in |\childdocname|.

%%%%%%%%%%%%%%%%%%%%%%%%%%%%%%%%%%%%%%%%
\DescribeMacro{\childdocby}
Each part to be included by |\input| should start with:
%
\begin{center}
\begin{tabular}{l}
|\input{childdoc.def}|\\
|\childdocby{|\textit{main}|}|\\
\end{tabular}
\end{center}
%
The directive |\childdocby| is similar to |\childdocof|
described in \secref{sec:include},
but the subsequent selection of content must be done manually.
To that end, both |\ifchilddoc| and |\ifchilddocmanual|
will be true upon processing of a part,
and the name of the part is stored in |\childdocname|.
Note that |\jobname| will be set to the filename of the current part
so that each part receives an individual |.aux| file
that does not interfere with the |.aux| file(s) of the main document.
This behaviour can be altered by the alternative form
|\childdocby[*]{|\textit{main}|}| (with a non-empty optional argument)
which uses the |.aux| file of the main document
by setting |\jobname| to \textit{main}.

%%%%%%%%%%%%%%%%%%%%%%%%%%%%%%%%%%%%%%%%%%%%%%%%%%%%%%%%%%%%%%%%%%%%%%%%%%%%%%%%
\subsection{Driver Development}
\label{sec:driver}

The \textsf{childdoc} mechanism can also be use for the development
of definition files such as \LaTeX{} styles or classes.
This case differs from the above setup with multiple parts
included by |\include| in that no |\includeonly| should be invoked.
This can be achieved by starting the include file
(before |\ProvidesPackage|) with:
%
\begin{center}
\begin{tabular}{l}
|\input{childdoc.def}|\\
|\childdocforward{|\textit{main}|}|\\
\end{tabular}
\end{center}
%
or alternatively with:
%
\begin{center}
\begin{tabular}{l}
|\input{childdoc.def}|\\
|\childdocby{|\textit{main}|}|\\
\end{tabular}
\end{center}
%
Both forms have slightly different effects as described above.
The main file is prepared as usual, see \secref{sec:include}.

%%%%%%%%%%%%%%%%%%%%%%%%%%%%%%%%%%%%%%%%%%%%%%%%%%%%%%%%%%%%%%%%%%%%%%%%%%%%%%%%
\subsection{Legacy Detection}
\label{sec:detection}

The directive |\childdocmain| in the main file can detect
whether the complete document or merely a child is to be compiled
even without using the directive |\childdocof|.
This method is deprecated because it is less robust
and there is no compelling reason to use it;
it is merely provided for backward compatibility
and it may be removed in future versions.

If the detection mechanism is to be used,
it is mandatory to correctly specify
the filename of the main file as the argument of |\childdocmain|:
%
\begin{center}
\begin{tabular}{l}
|\input{childdoc.def}|\\
|\childdocmain{|\textit{main}|}|\\
\end{tabular}
\end{center}
%
If |\jobname| does not match the argument \textit{main} of |\childdocmain|,
it is assumed that |\jobname| points to the child file to be compiled.
When using |\childdocmain| with the main file specified as argument,
it suffices to start a child file
with just |\input{|\textit{main}|}|
without loading of the package and using |\childdocof|.
If instead all processing is done
with the appropriate \textsf{childdoc} directives,
the argument of \textit{main} of |\childdocmain| can be empty.

An alternative version of the command line processing described
in \secref{sec:commandline} using the detection mechanism reads:
%
\begin{center}
|... -jobname "|\textit{target}|" "|[\textit{flags}]%
[|\def\jobname{|\textit{dest}|}|]|\input{|\textit{main}|}"|
\end{center}

%%%%%%%%%%%%%%%%%%%%%%%%%%%%%%%%%%%%%%%%%%%%%%%%%%%%%%%%%%%%%%%%%%%%%%%%%%%%%%%%
\subsection{Manual Code}
\label{sec:manual}

In case one cannot be certain whether the definitions file |childdoc.def|
is installed on the target \TeX{} distribution
and one prefers not to ship it,
it is conceivable to paste a few relevant commands into the sources.

To that end, drop all statements |\input{childdoc.def}|
and perform the replacements as outlined below.
Instead of |\childdocmain{|\textit{main}|}| add the following code
to the top of the main file:
%
\begin{center}
\begin{tabular}{l}
|\||ifdefined\childdocname\endinput\||fi\newif\ifchilddoc|\\
|\edef\childdocname{\scantokens\expandafter{\jobname\noexpand}}|\\
|\def\childdocmain{|\textit{main}|}\||ifx\childdocmain\childdocname\||else|\\
|\childdoctrue\includeonly{\childdocname}\let\jobname\childdocmain\||fi|\\
\end{tabular}
\end{center}
%
Instead of |\childdocof{|\textit{main}|}| just include the main file
at the top of each child file:
%
\begin{center}
|\input{|\textit{main}|}|
\end{center}
%
A simple redirection |\childdocforward{|\textit{dest}|}| is achieved by:
%
\begin{center}
|\def\jobname{|\textit{dest}|}\input{\jobname}|
\end{center}
%
The redirection with prefix
|\childdocforwardprefix[|\textit{prefix}|]{|\textit{dest}|}|
is accomplished by:
%
\begin{center}
\begin{tabular}{l}
|{\edef\jobname{\scantokens\expandafter{\jobname\noexpand}}|\\
|\def\redirectjob |\textit{prefix}|#1~~~{\gdef\jobname{|\textit{dest}|#1}}|\\
|\expandafter\redirectjob\jobname~~~}\input{\jobname}|
\end{tabular}
\end{center}

In an alternative approach,
child documents can be compiled by a specific command line
without additional code or specific definitions:
%
\begin{center}
|... -jobname "|\textit{target}|" "|[\textit{flags}]%
|\includeonly{|\textit{dest}|}\input{|\textit{main}|}"|
\end{center}
%

%%%%%%%%%%%%%%%%%%%%%%%%%%%%%%%%%%%%%%%%%%%%%%%%%%%%%%%%%%%%%%%%%%%%%%%%%%%%%%%%
%%%%%%%%%%%%%%%%%%%%%%%%%%%%%%%%%%%%%%%%%%%%%%%%%%%%%%%%%%%%%%%%%%%%%%%%%%%%%%%%
\section{Information}

%%%%%%%%%%%%%%%%%%%%%%%%%%%%%%%%%%%%%%%%%%%%%%%%%%%%%%%%%%%%%%%%%%%%%%%%%%%%%%%%
\subsection{Copyright}

Copyright \copyright{} 2017--2018 Niklas Beisert

This work may be distributed and/or modified under the
conditions of the \LaTeX{} Project Public License, either version 1.3
of this license or (at your option) any later version.
The latest version of this license is in
  \url{http://www.latex-project.org/lppl.txt}
and version 1.3 or later is part of all distributions of \LaTeX{}
version 2005/12/01 or later.

This work has the LPPL maintenance status `maintained'.

The Current Maintainer of this work is Niklas Beisert.

This work consists of the files |README.txt|, |childdoc.ins| and |childdoc.dtx|
as well as the derived files |childdoc.def|, |cdocsamp.tex|
with |cdocsch1.tex|, |cdocsch2.tex|, |cdocspt3.tex|, |cdocspt4.tex|,
|cdocsdrf.tex|, |cdocsfn1.tex|, |cdocsfn2.tex|
as well as |childdoc.pdf|.

%%%%%%%%%%%%%%%%%%%%%%%%%%%%%%%%%%%%%%%%%%%%%%%%%%%%%%%%%%%%%%%%%%%%%%%%%%%%%%%%
\subsection{Files and Installation}

The package consists of the files:
%
\begin{center}
\begin{tabular}{ll}
    |README.txt|   & readme file \\
    |childdoc.ins| & installation file \\
    |childdoc.dtx| & source file \\
    |childdoc.def| & definition file \\
    |cdocsamp.tex| & sample main file \\
    |cdocsch1.tex| & sample include file \\
    |cdocsch2.tex| & sample include file \\
    |cdocspt3.tex| & sample part file \\
    |cdocspt4.tex| & sample part file \\
    |cdocsdrf.tex| & sample redirection file \\
    |cdocsfn1.tex| & sample redirection file \\
    |cdocsfn2.tex| & sample redirection file \\
    |childdoc.pdf| & manual
\end{tabular}
\end{center}
%
The distribution consists of the files
|README.txt|, |childdoc.ins| and |childdoc.dtx|.
%
\begin{itemize}
\item
Run (pdf)\LaTeX{} on |childdoc.dtx|
to compile the manual |childdoc.pdf| (this file).
\item
Run \LaTeX{} on |childdoc.ins| to create the definitions file |childdoc.def|
and the sample |cdocsamp.tex| with include files
|cdocsch1.tex|, |cdocsch2.tex|, |cdocspt3.tex|, |cdocspt4.tex|,
|cdocsdrf.tex|, |cdocsfn1.tex|, |cdocsfn2.tex|.
Then copy the file |childdoc.def| to an appropriate directory of your \LaTeX{}
distribution, e.g.\ \textit{texmf-root}|/tex/latex/childdoc|.
\end{itemize}

%%%%%%%%%%%%%%%%%%%%%%%%%%%%%%%%%%%%%%%%%%%%%%%%%%%%%%%%%%%%%%%%%%%%%%%%%%%%%%%%
\subsection{Related CTAN Packages}

There are several other packages which offer a similar functionality:
%
\begin{itemize}
\item
The packages
\href{http://ctan.org/pkg/docmute}{\textsf{docmute}},
\href{http://ctan.org/pkg/includex}{\textsf{includex}} and
\href{http://ctan.org/pkg/standalone}{\textsf{standalone}}
provide commands to include only the document body of
a child file thus allowing both files to be compiled individually.
\item
The packages \href{http://ctan.org/pkg/subdocs}{\textsf{subdocs}}
and \href{http://ctan.org/pkg/subfiles}{\textsf{subfiles}}
provide structures in which the main and child documents can be
encapsulated and allowing them to be compiled individually.
The inclusion mechanism is different from the conventional |\include|.
\item
The package \href{http://ctan.org/pkg/combine}{\textsf{combine}}
is an elaborate solution to combine several documents into one.
\end{itemize}
%
See also the CTAN topic \href{http://ctan.org/topic/subdocs}{\textsf{subdocs}}
for further related packages.
The present package differs from the above solutions in that
a document structure constructed with the conventional |\include| mechanism
just needs two extra commands at the top of every file
such that all constituent files can be compiled individually.

%%%%%%%%%%%%%%%%%%%%%%%%%%%%%%%%%%%%%%%%%%%%%%%%%%%%%%%%%%%%%%%%%%%%%%%%%%%%%%%%
%\subsection{Feature Suggestions}
%
%The following is a list of features which may be useful for future
%versions of this package:
%%
%\begin{itemize}
%\item
%\ldots
%\end{itemize}

%%%%%%%%%%%%%%%%%%%%%%%%%%%%%%%%%%%%%%%%%%%%%%%%%%%%%%%%%%%%%%%%%%%%%%%%%%%%%%%%
\subsection{Revision History}

%%%%%%%%%%%%%%%%%%%%%%%%%%%%%%%%%%%%%%%%
\paragraph{v2.0:} 2018/12/30

\begin{itemize}
\item
immediate forward processing
\item
added |\childdocby| mechanism
\item
manual restructured
\end{itemize}

%%%%%%%%%%%%%%%%%%%%%%%%%%%%%%%%%%%%%%%%
\paragraph{v1.6:} 2018/01/17

\begin{itemize}
\item
application for development of include files
\item
corrections to manual
\end{itemize}

%%%%%%%%%%%%%%%%%%%%%%%%%%%%%%%%%%%%%%%%
\paragraph{v1.5:} 2017/05/21

\begin{itemize}
\item
more complete structuring introduced
\item
|\childdocof| introduced
\item
|\childdoc| renamed to |\childdocmain|
\item
|\childredirect| renamed to |\childdocforward| and |\childdocforwardprefix|
and functionality expanded
\end{itemize}

%%%%%%%%%%%%%%%%%%%%%%%%%%%%%%%%%%%%%%%%
\paragraph{v1.0:} 2017/04/27

\begin{itemize}
\item
manual and install package
\item
first version published on CTAN
\end{itemize}

%%%%%%%%%%%%%%%%%%%%%%%%%%%%%%%%%%%%%%%%
\paragraph{v0.6:} 2017/04/26

\begin{itemize}
\item
redirection mechanism added
\end{itemize}

%%%%%%%%%%%%%%%%%%%%%%%%%%%%%%%%%%%%%%%%
\paragraph{v0.5:} 2017/04/26

\begin{itemize}
\item
functionality in definition file
\end{itemize}


%%%%%%%%%%%%%%%%%%%%%%%%%%%%%%%%%%%%%%%%%%%%%%%%%%%%%%%%%%%%%%%%%%%%%%%%%%%%%%%%
%%%%%%%%%%%%%%%%%%%%%%%%%%%%%%%%%%%%%%%%%%%%%%%%%%%%%%%%%%%%%%%%%%%%%%%%%%%%%%%%
%%%%%%%%%%%%%%%%%%%%%%%%%%%%%%%%%%%%%%%%%%%%%%%%%%%%%%%%%%%%%%%%%%%%%%%%%%%%%%%%
\appendix

\settowidth\MacroIndent{\rmfamily\scriptsize 000\ }

 \DocInput{childdoc.dtx}

\end{document}
%</driver>
% \fi
%
% %%%%%%%%%%%%%%%%%%%%%%%%%%%%%%%%%%%%%%%%%%%%%%%%%%%%%%%%%%%%%%%%%%%%%%%%%%%%%%
% %%%%%%%%%%%%%%%%%%%%%%%%%%%%%%%%%%%%%%%%%%%%%%%%%%%%%%%%%%%%%%%%%%%%%%%%%%%%%%
% \section{Sample}
%\iffalse
%<*samplemain>
%\fi
%
% The following presents a sample document
% with two chapters, two parts, a title page,
% a compile flag as well as three forwarding files to set the flag.
% It consists of eight |.tex| files:
% \begin{center}
% \begin{tabular}{ll}
% |cdocsamp.tex|&main file\\
% |cdocsch1.tex|&include file for chapter 1\\
% |cdocsch2.tex|&include file for chapter 2\\
% |cdocspt3.tex|&include file for part 3\\
% |cdocspt4.tex|&include file for part 4\\
% |cdocsdrf.tex|&forwarding file for main file in draft mode\\
% |cdocsfi1.tex|&forwarding file for final version of chapter 1\\
% |cdocsfi2.tex|&forwarding file for final version of chapter 2\\
% \end{tabular}
% \end{center}
% Each of the eight files can be compiled directly by the \LaTeX{} compiler.
%
% %%%%%%%%%%%%%%%%%%%%%%%%%%%%%%%%%%%%%%
% \paragraph{Main File.}
%
% The main file is called |cdocsamp.tex|.
%
% Load the \textsf{childdoc} definitions and
% declare the filename for the main document:
%    \begin{macrocode}
\input{childdoc.def}
\childdocmain{}
%    \end{macrocode}

% Optional override for |\version| flag:
%    \begin{macrocode}
%%\ifchilddoc\else\providecommand{\version}{draft}\fi
%    \end{macrocode}

% Define the default values for the |\version| flag
% (|final| for the main file and |draft| for childs):
%    \begin{macrocode}
\ifchilddoc
\providecommand{\version}{draft}
\else
\providecommand{\version}{final}
\fi
%    \end{macrocode}

% Load the standard document class:
%    \begin{macrocode}
\documentclass[12pt]{article}
%    \end{macrocode}

% Start the document body:
%    \begin{macrocode}
\begin{document}
%    \end{macrocode}

% Declare a title page.
% Print title, part of document being processed and version flag:
%    \begin{macrocode}
\addtocounter{page}{-1}
\begin{center}
{\LARGE\bfseries{}childdoc example\par}
\vspace{1cm}
\ifchilddoc
\ifchilddocmanual part\else chapter\fi:
`\childdocname' of `\childdocjob'\par
\else
main document: `\childdocjob'\par
\fi
version: \version\par
\end{center}
\newpage
%    \end{macrocode}

% Manually include selected file,
% otherwise process as usual:
%    \begin{macrocode}
\ifchilddocmanual
\section*{part `\childdocname'}
\input{\childdocname}
\else
%    \end{macrocode}

% Include the two chapters:
%    \begin{macrocode}
\include{cdocsch1}
\include{cdocsch2}
%    \end{macrocode}

% Include the two parts unless only chapters should be displayed:
%    \begin{macrocode}
\ifchilddoc\else
\section{part three}
\input{cdocspt3}
\section{part four}
\input{cdocspt4}
\fi
%    \end{macrocode}

% Process as usual until here:
%    \begin{macrocode}
\fi
%    \end{macrocode}

% End of document body:
%    \begin{macrocode}
\end{document}
%    \end{macrocode}
%\iffalse
%</samplemain>
%\fi
%
% %%%%%%%%%%%%%%%%%%%%%%%%%%%%%%%%%%%%%%
% \paragraph{Chapter Include Files.}
%
% The include files are called |cdocsch1.tex| and |cdocsch2.tex|.
%
%\iffalse
%<*samplechap1|samplechap2>
%\fi

% Optional override for |\version| flag:
%    \begin{macrocode}
%%\providecommand{\version}{final}
%    \end{macrocode}

% Include the main document:
%    \begin{macrocode}
\input{childdoc.def}
\childdocof{cdocsamp}
%    \end{macrocode}

%\iffalse
%</samplechap1|samplechap2>
%\fi
%
%\iffalse
%<*samplechap1>
%\fi
% Some text for chapter 1:
%    \begin{macrocode}
\section{one}
some text in chapter one
%    \end{macrocode}

%\iffalse
%</samplechap1>
%\fi
% Some text for chapter 2:
%\iffalse
%<*samplechap2>
%\fi
%    \begin{macrocode}
\section{two}
more text in chapter two
%    \end{macrocode}

%\iffalse
%</samplechap2>
%\fi
%
% %%%%%%%%%%%%%%%%%%%%%%%%%%%%%%%%%%%%%%
% \paragraph{Part Include Files.}
%
% The include files are called |cdocspt3.tex| and |cdocspt4.tex|.
%
%\iffalse
%<*samplepart3|samplepart4>
%\fi

% Optional override for |\version| flag:
%    \begin{macrocode}
%%\providecommand{\version}{final}
%    \end{macrocode}

% Include the main document:
%    \begin{macrocode}
\input{childdoc.def}
\childdocby{cdocsamp}
%    \end{macrocode}

%\iffalse
%</samplepart3|samplepart4>
%\fi
%
%\iffalse
%<*samplepart3>
%\fi
% Some text for part 3:
%    \begin{macrocode}
some text in part three
%    \end{macrocode}

%\iffalse
%</samplepart3>
%\fi
% Some text for part 4:
%\iffalse
%<*samplepart4>
%\fi
%    \begin{macrocode}
more text in part four
%    \end{macrocode}

%\iffalse
%</samplepart4>
%\fi
%
% %%%%%%%%%%%%%%%%%%%%%%%%%%%%%%%%%%%%%%
% \paragraph{Forwarding for a Complete Draft.}
%
% The following forwarding file |cdocsdrf.tex|
% compiles the main document in draft mode:
%\iffalse
%<*sampledraft>
%\fi
%    \begin{macrocode}
\def\version{draft}
\input{childdoc.def}
\childdocforward{cdocsamp}
%    \end{macrocode}

%\iffalse
%</sampledraft>
%\fi
%
% %%%%%%%%%%%%%%%%%%%%%%%%%%%%%%%%%%%%%%
% \paragraph{Forwarding for Final Version of the Chapters.}
%
% The following forwarding files |cdocsfn1.tex| and |cdocsfn2.tex|
% (with identical content)
% compile the final versions of the child documents
% |cdocsch1.tex| and |cdocsch2.tex|, respectively:
%\iffalse
%<*samplefinal>
%\fi
%    \begin{macrocode}
\def\version{final}
\input{childdoc.def}
\childdocforwardprefix[cdocsamp]{cdocsfn}{cdocsch}
%    \end{macrocode}

%\iffalse
%</samplefinal>
%\fi
%
% %%%%%%%%%%%%%%%%%%%%%%%%%%%%%%%%%%%%%%
% \paragraph{Command Line Processing.}
%
% The following three command lines generate the output files
% |cdocscld|, |cdocscl1| and |cdocscl2|
% which should be identical to
% |cdocsdrf|, |cdocsch1| and |cdocsfn2|, respectively:
% \begin{center}
% \begin{tabular}{l}
% |latex -jobname cdocscld \|\\
% |  "\def\version{draft}\input{childdoc.def}\childdocforward{cdocsamp}"|\\
% |latex -jobname cdocscl1 \|\\
% |  "\input{childdoc.def}\childdocforward[cdocsamp]{cdocsch1}"|\\
% |latex -jobname cdocscl2 \|\\
% |  "\def\version{final}\input{childdoc.def}\childdocforward{cdocsch2}"|
% \end{tabular}
% \end{center}
% Note that the trailing backslash on each first line
% merely continues the input to the second line
% (for convenient cut ant paste).
% Furthermore, the command |latex| can be replaced by any
% of its alternative versions such as |pdflatex|.
%
% %%%%%%%%%%%%%%%%%%%%%%%%%%%%%%%%%%%%%%%%%%%%%%%%%%%%%%%%%%%%%%%%%%%%%%%%%%%%%%
% %%%%%%%%%%%%%%%%%%%%%%%%%%%%%%%%%%%%%%%%%%%%%%%%%%%%%%%%%%%%%%%%%%%%%%%%%%%%%%
% \section{Implementation}
%\iffalse
%<*package>
%\fi
%
% This section describes the definitions file |childdoc.def|.

% The definitions cannot be loaded using |\usepackage| or |\RequirePackage|
% which has a mechanism to prevent loading a style file more than once.
% When loading the definitions by means of |\input|
% multiple instances have to be prevented manually:
%\iffalse
%This code needs to be before the `\ProvidesFile' directive
%which is defined at the beginning of this file.
%Therefore it is also placed there and commented out here.
%</package>
%<*discard>
%\fi
%    \begin{macrocode}
\ifdefined\childdocmain\endinput\fi
%    \end{macrocode}
%\iffalse
%</discard>
%<*package>
%\fi
%
% \macro{\ifchilddoc}
% \macro{\ifchilddocmanual}
% The conditional |\ifchilddoc| tells whether a
% child (true) or main (false) document is being compiled.
% The conditional |\ifchilddocmanual| tells whether
% the |\includeonly| mechanism is used (false) or
% the selection of child files must be performed manually (true).
% The definitions initialise to false:
%    \begin{macrocode}
\newif\ifchilddoc
\newif\ifchilddocmanual
%    \end{macrocode}

% \macro{\childdocname}
% \macro{\childdocjob}
% The macro |\childdocname| stores the name of the main document
% to be compiled. The macro |\childdocjob| stores the name of
% the document on which the \LaTeX{} compiler was originally invoked.
% The content of |\jobname| cannot be compared
% to filenames specified in the source due to different catcodes.
% The following code rescans |\jobname|, stores the result
% in |\childdocname| and saves a copy in |\childdocjob|:
%    \begin{macrocode}
\edef\childdocname{\scantokens\expandafter{\jobname\noexpand}}
\let\childdocjob\childdocname
%    \end{macrocode}

% \macro{\childdocdisable}
% The macro |\childdocdisable| prevents the main file
% from being processed more than once.
% At this stage, the main document command |\childdocmain|
% is assumed to be called once again where it should do nothing.
% Any subsequent call to it should prevent
% a secondary processing of the main document
% It overwrites the forwarding commands
% |\childdocof| and |\childdocforward|
% with empty macros to prevent further inclusions of the main document:
%    \begin{macrocode}
\newcommand{\childdocdisable}
{
  \renewcommand{\childdocmain}[1]{\renewcommand{\childdocmain}[1]{\endinput}}
  \renewcommand{\childdocof}[1]{}
  \renewcommand{\childdocby}[2][]{}
  \renewcommand{\childdocforward}[2][]{}
  \renewcommand{\childdocdisable}{}
}
%    \end{macrocode}

% \macro{\childdocmain}
% The macro |\childdocmain| is to be called at the top of the main file
% with nothing or the main filename (without extension) as argument.
% First, it breaks loops.
% If the argument is not empty and does not match |\childdocname|
% (which is set by the first inclusion of |childdoc.def|),
% |\ifchilddoc| is set to true, |\includeonly| is applied to the child file
% and |\jobname| is set to the main file
% (for proper handling of |.aux| files):
%    \begin{macrocode}
\newcommand{\childdocmain}[1]
{
  \childdocdisable\childdocmain{}
  \if?#1?\else
    \begingroup
      \def\childdoctmp{#1}
      \ifx\childdoctmp\childdocname
        \def\childdoctmp{}
      \else
        \def\childdoctmp
        {
          \childdoctrue
          \includeonly{\childdocname}
          \def\childdocjob{#1}
          \def\jobname{#1}
        }
      \fi
      \expandafter
    \endgroup
    \childdoctmp
  \fi
}
%    \end{macrocode}

% \macro{\childdocof}
% The command |\childdocof| redirects
% compilation to the main file |#1|.
%    \begin{macrocode}
\newcommand{\childdocof}[1]
{
  \childdocdisable
  \childdoctrue
  \includeonly{\childdocname}
  \def\jobname{#1}
  \def\childdocjob{#1}
  \input{#1}
}
%    \end{macrocode}

% \macro{\childdocby}
% The command |\childdocby| ....
%    \begin{macrocode}
\newcommand{\childdocby}[2][]
{
  \childdocdisable
  \childdoctrue
  \childdocmanualtrue
  \if?#1?\else
    \def\jobname{#2}
  \fi
  \def\childdocjob{#2}
  \input{#2}
  \endinput
}
%    \end{macrocode}

% \macro{\childdocforward}
% The command |\childdocforward| redirects
% compilation to the main file or
% (if the optional argument is given) a child file.
% Parameters are set as if the main file
% or a child file starting with |\childdocof| was compiled.
% Then compilation is handed over to the main file:
%    \begin{macrocode}
\newcommand{\childdocforward}[2][]
{
  \begingroup
    \if?#1?
      \def\childdoctmp
      {
        \def\childdocname{#2}
        \def\childdocjob{#2}
        \def\jobname{#2}
        \input{#2}
        \endinput
      }
    \else
      \def\childdoctmp
      {
        \childdocdisable
        \def\childdocname{#2}
        \childdoctrue
        \includeonly{#2}
        \def\childdocjob{#1}
        \def\jobname{#1}
        \input{#1}
        \endinput
      }
    \fi
    \expandafter
  \endgroup
  \childdoctmp
}
%    \end{macrocode}

% \macro{\childdocforwardprefix}
% The command |\childdocforwardprefix| redirects
% compilation to the main or a child file by means of a pattern.
% The prefix |#1| in the current filename is replaced by |#2|
% and the suffix of the current filename is kept
% (it is assumed that the filename does not contain the substring `|~~~|'
% which is used as a delimiter).
% Compilation is handed over to the new file by |\childdocforward|:
%    \begin{macrocode}
\newcommand{\childdocforwardprefix}[3][]
{
  \begingroup
    \def\childdocextract #2##1~~~{\def\childdoctmp{\childdocforward[#1]{#3##1}}}
    \expandafter\childdocextract\childdocname~~~
    \expandafter
  \endgroup
  \childdoctmp
}
%    \end{macrocode}

% \macro{\childdoc}
% The deprecated macro |\childdoc| is a legacy version of |\childdocmain|:
%    \begin{macrocode}
\newcommand{\childdoc}{\childdocmain}
%    \end{macrocode}

% \macro{\childdocredirect}
% The deprecated macro |\childdocredirect| is a legacy version
% of |\childdocforward| and |\childdocforwardprefix|:
%    \begin{macrocode}
\newcommand{\childdocredirect}[2][]
{
  \begingroup
    \if?#1?
      \def\childdoctmp{\childdocforward{#2}}
    \else
      \def\childdoctmp{\childdocforwardprefix{#1}{#2}}
    \fi
    \expandafter
  \endgroup
  \childdoctmp
}
%    \end{macrocode}

%\iffalse
%</package>
%\fi
%
\endinput

\childdocforwardprefix[cdocsamp]{cdocsfn}{cdocsch}
%    \end{macrocode}

%\iffalse
%</samplefinal>
%\fi
%
% %%%%%%%%%%%%%%%%%%%%%%%%%%%%%%%%%%%%%%
% \paragraph{Command Line Processing.}
%
% The following three command lines generate the output files
% |cdocscld|, |cdocscl1| and |cdocscl2|
% which should be identical to
% |cdocsdrf|, |cdocsch1| and |cdocsfn2|, respectively:
% \begin{center}
% \begin{tabular}{l}
% |latex -jobname cdocscld \|\\
% |  "\def\version{draft}% \iffalse
%
% childdoc.dtx Copyright (C) 2017-2018 Niklas Beisert
%
% This work may be distributed and/or modified under the
% conditions of the LaTeX Project Public License, either version 1.3
% of this license or (at your option) any later version.
% The latest version of this license is in
%   http://www.latex-project.org/lppl.txt
% and version 1.3 or later is part of all distributions of LaTeX
% version 2005/12/01 or later.
%
% This work has the LPPL maintenance status `maintained'.
%
% The Current Maintainer of this work is Niklas Beisert.
%
% This work consists of the files childdoc.dtx and childdoc.ins
% and the derived files childdoc.def and cdocsamp.tex with
% cdocsch1.tex, cdocsch2.tex, cdocsdrf.tex, cdocsfn1.tex, cdocsfn2.tex.
%
%<package>\ifdefined\childdocmain\endinput\fi
%<package>\ProvidesFile{childdoc.def}[2018/12/30 v2.0 child document driver]
%<samplemain>\ProvidesFile{cdocsamp.tex}[2018/12/30 v2.0 sample for childdoc]
%<*driver>
%\ProvidesFile{childdoc.drv}[2018/12/30 v2.0 childdoc reference manual file]
\PassOptionsToClass{10pt,a4paper}{article}
\documentclass{ltxdoc}

\usepackage[margin=35mm]{geometry}
\usepackage{hyperref}
\usepackage{hyperxmp}
\usepackage[usenames]{color}

\hypersetup{colorlinks=true}
\hypersetup{pdfstartview=FitH}
\hypersetup{pdfpagemode=UseNone}
\hypersetup{pdfsource={}}
\hypersetup{pdflang={en-UK}}
\hypersetup{pdfcopyright={Copyright 2017-2018 Niklas Beisert.
  This work may be distributed and/or modified under the
  conditions of the LaTeX Project Public License, either version 1.3
  of this license or (at your option) any later version.}}
\hypersetup{pdflicenseurl={http://www.latex-project.org/lppl.txt}}
\hypersetup{pdfcontactaddress={ETH Zurich, ITP, HIT K,
  Wolfgang-Pauli-Strasse 27}}
\hypersetup{pdfcontactpostcode={8093}}
\hypersetup{pdfcontactcity={Zurich}}
\hypersetup{pdfcontactcountry={Switzerland}}
\hypersetup{pdfcontactemail={nbeisert@itp.phys.ethz.ch}}
\hypersetup{pdfcontacturl={http://people.phys.ethz.ch/\xmptilde nbeisert/}}

\newcommand{\secref}[1]{\hyperref[#1]{section \ref*{#1}}}

\parskip1ex
\parindent0pt
\let\olditemize\itemize
\def\itemize{\olditemize\parskip0pt}

\begin{document}

\title{The \textsf{childdoc} Package}
\hypersetup{pdftitle={The childdoc Package}}
\author{Niklas Beisert\\[2ex]
  Institut f\"ur Theoretische Physik\\
  Eidgen\"ossische Technische Hochschule Z\"urich\\
  Wolfgang-Pauli-Strasse 27, 8093 Z\"urich, Switzerland\\[1ex]
  \href{mailto:nbeisert@itp.phys.ethz.ch}
  {\texttt{nbeisert@itp.phys.ethz.ch}}}
\hypersetup{pdfauthor={Niklas Beisert}}
\hypersetup{pdfsubject={Manual for the LaTeX2e Package childdoc}}
\date{30 December 2018, \textsf{v2.0}}
\maketitle

\begin{abstract}\noindent
\textsf{childdoc} is a \LaTeXe{} package
that enables the direct compilation
of document sections included by |\include|
to individual files.
\end{abstract}

\begingroup
\parskip0ex
\tableofcontents
\endgroup

%%%%%%%%%%%%%%%%%%%%%%%%%%%%%%%%%%%%%%%%%%%%%%%%%%%%%%%%%%%%%%%%%%%%%%%%%%%%%%%%
%%%%%%%%%%%%%%%%%%%%%%%%%%%%%%%%%%%%%%%%%%%%%%%%%%%%%%%%%%%%%%%%%%%%%%%%%%%%%%%%
\section{Introduction}

\LaTeX{} provides a mechanism to structure a large document (such as a book)
into a main file and several child files (containing the chapters)
using the |\include| command.
This mechanism is beneficial for documents
which span hundreds of pages in order to
make the source file(s) more manageable.
Moreover, compilation can be restricted to
selected child files by means of the |\includeonly| command.
The latter feature can be used to reduce the compilation time while editing
(this was significantly more useful in the earlier days of \LaTeX{})
or to generate a smaller document which is easier to navigate.
Another application of |\includeonly| is to generate
documents consisting of selected parts of the complete document.

However, there are a few drawbacks of the plain |\include| mechanism:
\begin{itemize}
\item
The child files cannot be compiled on their own,
they can only be compiled via the main file.
A naive editing environment
(such as a text editor with an option
to have the current file processed by \LaTeX)
may require one to switch to the main file before compiling;
attempting to compile the child file produces errors.
\item
The main file must be modified (each time)
to adjust the |\includeonly| command
to the present needs. This easily leaves the main file in a messy state.
\item
The generated document will always carry the filename
of the main document. This is inconvenient if
several child files are to be compiled and
to be kept for distribution.
\end{itemize}

The present package provides a simple interface
to make child files individually compilable by \LaTeX{}.
Compiling a child file then has the same effect as compiling
the main file with an |\includeonly| command
to select the appropriate child.
Moreover the generated document will carry the name of the child
rather than the main file.
This resolves all three above issues.

This feature is meant to make the editing of books,
thesis documents and lecture notes somewhat more convenient.
However, the package can also be used efficiently for
composing a series of documents (such as exercise sheets)
which are typically distributed individually.
It then assists the author in generating the individual documents
(potentially in different versions)
as well as a document containing the collected series.
Another application is in developing style files
or other kinds of included material
where compilation of the style file could redirect
to a sample or test file.

%%%%%%%%%%%%%%%%%%%%%%%%%%%%%%%%%%%%%%%%%%%%%%%%%%%%%%%%%%%%%%%%%%%%%%%%%%%%%%%%
%%%%%%%%%%%%%%%%%%%%%%%%%%%%%%%%%%%%%%%%%%%%%%%%%%%%%%%%%%%%%%%%%%%%%%%%%%%%%%%%
\section{Usage}

First of all, the package \textsf{childdoc} is \emph{not} a standard
\LaTeXe{} |.sty| style file! Therefore it needs to be invoked in
a non-standard way.

%%%%%%%%%%%%%%%%%%%%%%%%%%%%%%%%%%%%%%%%%%%%%%%%%%%%%%%%%%%%%%%%%%%%%%%%%%%%%%%%
\subsection{Included Files}
\label{sec:include}

%%%%%%%%%%%%%%%%%%%%%%%%%%%%%%%%%%%%%%%%
\DescribeMacro{\childdocmain}
To use the package, add the commands
\begin{center}
\begin{tabular}{l}
|\input{childdoc.def}|\\
|\childdocmain{}|\\
\end{tabular}
\end{center}
at the very top of the main \LaTeX{} file,
in particular \emph{before} the |\documentclass| statement!
The argument of |\childdocmain| should be left empty
(but it must be present).

%%%%%%%%%%%%%%%%%%%%%%%%%%%%%%%%%%%%%%%%
\DescribeMacro{\childdocof}
Furthermore, add the commands
\begin{center}
\begin{tabular}{l}
|\input{childdoc.def}|\\
|\childdocof{|\textit{main}|}|\\
\end{tabular}
\end{center}
at the top of every child file \textit{child}
which is included by |\include{|\textit{child}|}|
from within the main file
(or at least for those files to be compiled individually).
The argument \textit{main} must be the filename of the main file.

There are a couple of
considerations in setting up the main and child documents:

%%%%%%%%%%%%%%%%%%%%%%%%%%%%%%%%%%%%%%%%
\paragraph{Restrictions.}

Please note the following restrictions:
\begin{itemize}
\item
|\childdocmain| must be called with one argument \textit{main}
to ensure compatibility with earlier version of the package.
It must either be empty (|\childdocmain{}|)
or precisely match the filename of the main file in which it is specified.
See \secref{sec:detection} for further information.
\item
The filename \textit{main} must be specified without the |.tex| extension.
\item
The filename \textit{main} is case sensitive
(even in case-insensitive file systems)
due to internal string comparison.
\item
The argument \textit{main} should be fully expanded, it cannot be a macro.
\item
Subdirectories and special characters should be avoided in filenames.
\item
The command |\childdocmain{|\textit{main}|}| must be followed by a whitespace.
It should not be followed immediately by another command
or by a comment mark `|%|'.
This is because the \TeX{} parser reads the token immediately following
the argument of |\childdocmain| and puts it
at the beginning of every child section;
however, a white\-space is ignored.
\end{itemize}

%%%%%%%%%%%%%%%%%%%%%%%%%%%%%%%%%%%%%%%%
\paragraph{Content of Main File.}

It is advisable to place all content in the child files included by |\include|.
Any output contained in the main file will appear in all child documents
unless suppressed manually;
it cannot be suppressed automatically by the |\includeonly| directive
and thus should normally be avoided.
A method to include some content in the main file
by means of conditional processing is described in \secref{sec:conditional}.

%%%%%%%%%%%%%%%%%%%%%%%%%%%%%%%%%%%%%%%%
\paragraph{Page Numbering.}

When only a part of the document is compiled,
the appropriate numbering of pages
(as well as other status parameters)
is determined from the |.aux| files.
The latter contain information from previous passes.
However this information needs to propagate through
all intermediate child documents.
Therefore the page numbering in child documents may well
be inconsistent until the complete document is compiled at least once.

A useful (if unconventional) way to always ensure a consistent
page numbering is to restart the numbering in each child document
and denote the pages by `\textit{child}|.|\textit{page}'
where \textit{child} represents the chapter/section number of the child file.
This can be achieved by the command
|\numberwithin{page}{|\textit{child}|}|
of the \textsf{amsmath} package
where \textit{child} can be |chapter| or |section|
depending on the chosen structuring.
Alternatively, one can modify the macro |\thepage| appropriately
and reset the counter |page| at the start of each child file.

%%%%%%%%%%%%%%%%%%%%%%%%%%%%%%%%%%%%%%%%%%%%%%%%%%%%%%%%%%%%%%%%%%%%%%%%%%%%%%%%
\subsection{Conditional Processing}
\label{sec:conditional}

The package provides a mechanism to compile different versions
of a document. To customise the versions further some conditional processing
can come in handy to distinguish which version is being compiled.
The package provides two macros to describe the compilation context:

%%%%%%%%%%%%%%%%%%%%%%%%%%%%%%%%%%%%%%%%
\DescribeMacro{\ifchilddoc}
The conditional |\ifchilddoc| distinguishes between the compilation of
child documents and the main document:
%
\begin{center}
|\ifchilddoc |\textit{child-code}| |[|\||else |\textit{main-code}]| \||fi|
\end{center}

%%%%%%%%%%%%%%%%%%%%%%%%%%%%%%%%%%%%%%%%
\DescribeMacro{\childdocname}
\DescribeMacro{\childdocjob}
The macro |\childdocname| contains the filename (without extension)
of the main or child file being processed.
Note that |\childdocjob| will always contain the name of the main file.

%%%%%%%%%%%%%%%%%%%%%%%%%%%%%%%%%%%%%%%%
\paragraph{Title Page.}

Conditional processing can be used to include a title or banner page
in the main document when proper precautions are taken.
Importantly, the code in the main file should ensure that the page counter
(as well as other status parameters which are stored in the |.aux| files)
takes the same value after the conditional processing.
Otherwise the page numbers may take divergent values
depending on which part is compiled.

For example, a title page could be declared by:
%
\begin{center}
\begin{tabular}{l}
|\ifchilddoc\||else|\\
|\addtocounter{page}{-1}|\\
\textit{code for title page}\\
|\newpage|\\
|\||fi|
\end{tabular}
\end{center}
%
A banner page for the child documents can be generated by:
%
\begin{center}
\begin{tabular}{l}
|\ifchilddoc|\\
|\addtocounter{page}{-1}|\\
\textit{code for banner page}\\
|\newpage|\\
|\||fi|
\end{tabular}
\end{center}
%
Here one could write a message such as:
\begin{center}
|This is the part \childdocname{} of \childdocjob{}.|
\end{center}

%%%%%%%%%%%%%%%%%%%%%%%%%%%%%%%%%%%%%%%%%%%%%%%%%%%%%%%%%%%%%%%%%%%%%%%%%%%%%%%%
\subsection{Flags}
\label{sec:flags}

The package makes it easy to generate different versions
of the main or child documents.
To this end compilation flags can be defined
and assigned different default values.
They will be particularly useful in conjunction
with the forwarding mechanism described in \secref{sec:forward}.

For example, it may be useful to have a flag |\version|
which can be set to |draft| or |final|.
The document source will contain some conditional code
depending on the value of |\version|.
Suppose further, the flag should default to |final| for the main file
and to |draft| for child files
which is a natural assignment for editing the document.
This is achieved by placing the following code
in the preamble of the main document
(below the |\childdocmain| directive):
%
\begin{center}
\begin{tabular}{l}
|\ifchilddoc|\\
|\providecommand{\version}{draft}|\\
|\||else|\\
|\providecommand{\version}{final}|\\
|\||fi|
\end{tabular}
\end{center}
%
The definition by |\providecommand| makes sure
that previous definitions are not overwritten.
Further statements |\providecommand{\version}{...}|
can thus be added before the above code to override it.

For the main file, one might add a line
(between |\childdocmain| and the above block)
%
\begin{center}
|%\ifchilddoc\||else\providecommand{\version}{draft}\||fi|
\end{center}
%
which can be uncommented to produce a draft version.
Likewise one can add a line to the very top of a child file
(above the |\childdocof{|\textit{main}|}| directive)
%
\begin{center}
|%\providecommand{\version}{final}|
\end{center}
%
which can be uncommented to produce the final version of this child document.

%%%%%%%%%%%%%%%%%%%%%%%%%%%%%%%%%%%%%%%%%%%%%%%%%%%%%%%%%%%%%%%%%%%%%%%%%%%%%%%%
\subsection{Forwarding}
\label{sec:forward}

Different versions of the main or child documents
using compilation flags as described in \secref{sec:flags}
can be (permanently) stored in different files
for convenient compilation, viewing and distribution.
To this end, the package defines a command
to pass on compilation to a different file:

%%%%%%%%%%%%%%%%%%%%%%%%%%%%%%%%%%%%%%%%
\DescribeMacro{\childdocforward}
The command |\childdocforward| redirects processing to
another source file:
%
\begin{center}
\begin{tabular}{l}
|\input{childdoc.def}|\\
|\childdocforward[|\textit{main}|]{|\textit{dest}|}|\\
\end{tabular}
\end{center}
%
The argument \textit{dest} is the destination file
(without extension).
It should be the main file or one of the child files.
Note that further \textsf{childdoc} directives
such as |\childdocof| and |\childdocforward|
in the indicated file will be processed in this form.
The optional argument \textit{main}
passes on directly to the main file \textit{main}
while pretending to compile the child \textit{dest}.
This form behaves as if \textit{dest}
issues |\childdocof{|\textit{main}|}| right away,
and no further \textsf{childdoc} directives will be processed.

%%%%%%%%%%%%%%%%%%%%%%%%%%%%%%%%%%%%%%%%
\DescribeMacro{\...prefix}
In the alternative form |\childdocforwardprefix|,
%
\begin{center}
\begin{tabular}{l}
|\input{childdoc.def}|\\
|\childdocforwardprefix[|\textit{main}|]{|\textit{prefix}|}{|\textit{dest}|}|
\end{tabular}
\end{center}
%
the destination file is determined by a pattern
depending on the current file:
To make this work, the current file must be called
`{\textit{prefix}\hspace{0.2em}\textit{suffix}}'
with \textit{prefix} matching precisely the argument.
Processing is then passed on to the file
`{\textit{dest}\hspace{0.2em}\textit{suffix}}'.
Surely, the same effect is achieved by
directly specifying the
argument `{\textit{dest}\hspace{0.2em}\textit{suffix}}'
in the first form.
However, that requires to set up a different file
for each child. With the alternative form of the command
all these files can have exactly the same content
which simplifies setting them up and maintaining them.

For example, the following file |draft.tex|
with a compilation flag |\version| as described in \secref{sec:flags}
compiles the main document as a draft:
%
\begin{center}
\begin{tabular}{l}
|\def\version{draft}|\\
|\input{childdoc.def}|\\
|\childdocforward{|\textit{main}|}|
\end{tabular}
\end{center}
%
Likewise, the following files |final|\textit{nn}|.tex|
compile the final version of the child document
|child|\textit{nn}|.tex|:
%
\begin{center}
\begin{tabular}{l}
|\def\version{final}|\\
|\input{childdoc.def}|\\
|\childdocforwardprefix{final}{child}|
\end{tabular}
\end{center}
%

Note that when several versions of a main file and/or of each child file
are to be generated, it may be convenient to set up a |Makefile| or
shell script to automatise the process.

%%%%%%%%%%%%%%%%%%%%%%%%%%%%%%%%%%%%%%%%%%%%%%%%%%%%%%%%%%%%%%%%%%%%%%%%%%%%%%%%
\subsection{Command Line Processing}
\label{sec:commandline}

The effect of redirection files can also be achieved by invoking
the \LaTeX{} compiler with a more elaborate command line.
Most conveniently this should be done as part
of a shell script or a |Makefile|.

When using \textsf{childdoc} in the main file, the following
command lines effectively perform a redirection
(note that depending on the shell being used,
backslashes may have to be doubled: `|\|' $\to$ `|\\|'):
%
\begin{center}
|... -jobname "|\textit{target}|" |\\|"|[\textit{flags}]%
|\input{childdoc.def}\childdocforward[|\textit{main}|]{|\textit{dest}|}"|
\end{center}
%
Here \textit{target} is the name of the output file,
\textit{main} is the name of the main file
and \textit{dest} is the name of the main or child file to be processed
(all filenames without extensions).
The optional argument \textit{main} can be omitted
if \textit{main} matches \textit{dest}.
Optionally, compilation \textit{flags} can be defined via |\def| commands.
This command line makes the \TeX{} engine believe
it is compiling the file \textit{target}
whose content is specified as the latter parameter.
The provided code then forwards the processing to
\textit{main} or \textit{dest} as described in \secref{sec:forward}.

%%%%%%%%%%%%%%%%%%%%%%%%%%%%%%%%%%%%%%%%%%%%%%%%%%%%%%%%%%%%%%%%%%%%%%%%%%%%%%%%
\subsection{Include by Input}
\label{sec:input}

Including child documents by |\include| has some restrictions by design.
Most notably, the content of a child document always occupies
its own set of pages; pages cannot be shared between child documents.
Usually, this behaviour makes perfect sense
because each child document contain an essential part of the document.
However, in some situations it may be desirable to compose
a document from a collection of parts
without having mandatory page breaks between then.
For this case, the package
provides a mechanism to include parts
by |\input| which can also be processed individually.
However, by construction this mechanism
requires manual handling of the content to be output.

%%%%%%%%%%%%%%%%%%%%%%%%%%%%%%%%%%%%%%%%
\DescribeMacro{\ifchilddocmanual}
The main file should be prepared as usual, see \secref{sec:include}.
However, the document body must make a distinction
between processing of an individual part and of the main document, e.g.:
%
\begin{center}
\begin{tabular}{l}
|\ifchilddocmanual|\\
|\input{\childdocname}|\\
|\||else|\\
\textit{document body with }|\input{|\textit{part}|}|\\
|\||fi|
\end{tabular}
\end{center}
%
The conditional |\ifchilddocmanual| is true whenever
a part to be included by |\input| is being compiled,
and the name of the part is stored in |\childdocname|.

%%%%%%%%%%%%%%%%%%%%%%%%%%%%%%%%%%%%%%%%
\DescribeMacro{\childdocby}
Each part to be included by |\input| should start with:
%
\begin{center}
\begin{tabular}{l}
|\input{childdoc.def}|\\
|\childdocby{|\textit{main}|}|\\
\end{tabular}
\end{center}
%
The directive |\childdocby| is similar to |\childdocof|
described in \secref{sec:include},
but the subsequent selection of content must be done manually.
To that end, both |\ifchilddoc| and |\ifchilddocmanual|
will be true upon processing of a part,
and the name of the part is stored in |\childdocname|.
Note that |\jobname| will be set to the filename of the current part
so that each part receives an individual |.aux| file
that does not interfere with the |.aux| file(s) of the main document.
This behaviour can be altered by the alternative form
|\childdocby[*]{|\textit{main}|}| (with a non-empty optional argument)
which uses the |.aux| file of the main document
by setting |\jobname| to \textit{main}.

%%%%%%%%%%%%%%%%%%%%%%%%%%%%%%%%%%%%%%%%%%%%%%%%%%%%%%%%%%%%%%%%%%%%%%%%%%%%%%%%
\subsection{Driver Development}
\label{sec:driver}

The \textsf{childdoc} mechanism can also be use for the development
of definition files such as \LaTeX{} styles or classes.
This case differs from the above setup with multiple parts
included by |\include| in that no |\includeonly| should be invoked.
This can be achieved by starting the include file
(before |\ProvidesPackage|) with:
%
\begin{center}
\begin{tabular}{l}
|\input{childdoc.def}|\\
|\childdocforward{|\textit{main}|}|\\
\end{tabular}
\end{center}
%
or alternatively with:
%
\begin{center}
\begin{tabular}{l}
|\input{childdoc.def}|\\
|\childdocby{|\textit{main}|}|\\
\end{tabular}
\end{center}
%
Both forms have slightly different effects as described above.
The main file is prepared as usual, see \secref{sec:include}.

%%%%%%%%%%%%%%%%%%%%%%%%%%%%%%%%%%%%%%%%%%%%%%%%%%%%%%%%%%%%%%%%%%%%%%%%%%%%%%%%
\subsection{Legacy Detection}
\label{sec:detection}

The directive |\childdocmain| in the main file can detect
whether the complete document or merely a child is to be compiled
even without using the directive |\childdocof|.
This method is deprecated because it is less robust
and there is no compelling reason to use it;
it is merely provided for backward compatibility
and it may be removed in future versions.

If the detection mechanism is to be used,
it is mandatory to correctly specify
the filename of the main file as the argument of |\childdocmain|:
%
\begin{center}
\begin{tabular}{l}
|\input{childdoc.def}|\\
|\childdocmain{|\textit{main}|}|\\
\end{tabular}
\end{center}
%
If |\jobname| does not match the argument \textit{main} of |\childdocmain|,
it is assumed that |\jobname| points to the child file to be compiled.
When using |\childdocmain| with the main file specified as argument,
it suffices to start a child file
with just |\input{|\textit{main}|}|
without loading of the package and using |\childdocof|.
If instead all processing is done
with the appropriate \textsf{childdoc} directives,
the argument of \textit{main} of |\childdocmain| can be empty.

An alternative version of the command line processing described
in \secref{sec:commandline} using the detection mechanism reads:
%
\begin{center}
|... -jobname "|\textit{target}|" "|[\textit{flags}]%
[|\def\jobname{|\textit{dest}|}|]|\input{|\textit{main}|}"|
\end{center}

%%%%%%%%%%%%%%%%%%%%%%%%%%%%%%%%%%%%%%%%%%%%%%%%%%%%%%%%%%%%%%%%%%%%%%%%%%%%%%%%
\subsection{Manual Code}
\label{sec:manual}

In case one cannot be certain whether the definitions file |childdoc.def|
is installed on the target \TeX{} distribution
and one prefers not to ship it,
it is conceivable to paste a few relevant commands into the sources.

To that end, drop all statements |\input{childdoc.def}|
and perform the replacements as outlined below.
Instead of |\childdocmain{|\textit{main}|}| add the following code
to the top of the main file:
%
\begin{center}
\begin{tabular}{l}
|\||ifdefined\childdocname\endinput\||fi\newif\ifchilddoc|\\
|\edef\childdocname{\scantokens\expandafter{\jobname\noexpand}}|\\
|\def\childdocmain{|\textit{main}|}\||ifx\childdocmain\childdocname\||else|\\
|\childdoctrue\includeonly{\childdocname}\let\jobname\childdocmain\||fi|\\
\end{tabular}
\end{center}
%
Instead of |\childdocof{|\textit{main}|}| just include the main file
at the top of each child file:
%
\begin{center}
|\input{|\textit{main}|}|
\end{center}
%
A simple redirection |\childdocforward{|\textit{dest}|}| is achieved by:
%
\begin{center}
|\def\jobname{|\textit{dest}|}\input{\jobname}|
\end{center}
%
The redirection with prefix
|\childdocforwardprefix[|\textit{prefix}|]{|\textit{dest}|}|
is accomplished by:
%
\begin{center}
\begin{tabular}{l}
|{\edef\jobname{\scantokens\expandafter{\jobname\noexpand}}|\\
|\def\redirectjob |\textit{prefix}|#1~~~{\gdef\jobname{|\textit{dest}|#1}}|\\
|\expandafter\redirectjob\jobname~~~}\input{\jobname}|
\end{tabular}
\end{center}

In an alternative approach,
child documents can be compiled by a specific command line
without additional code or specific definitions:
%
\begin{center}
|... -jobname "|\textit{target}|" "|[\textit{flags}]%
|\includeonly{|\textit{dest}|}\input{|\textit{main}|}"|
\end{center}
%

%%%%%%%%%%%%%%%%%%%%%%%%%%%%%%%%%%%%%%%%%%%%%%%%%%%%%%%%%%%%%%%%%%%%%%%%%%%%%%%%
%%%%%%%%%%%%%%%%%%%%%%%%%%%%%%%%%%%%%%%%%%%%%%%%%%%%%%%%%%%%%%%%%%%%%%%%%%%%%%%%
\section{Information}

%%%%%%%%%%%%%%%%%%%%%%%%%%%%%%%%%%%%%%%%%%%%%%%%%%%%%%%%%%%%%%%%%%%%%%%%%%%%%%%%
\subsection{Copyright}

Copyright \copyright{} 2017--2018 Niklas Beisert

This work may be distributed and/or modified under the
conditions of the \LaTeX{} Project Public License, either version 1.3
of this license or (at your option) any later version.
The latest version of this license is in
  \url{http://www.latex-project.org/lppl.txt}
and version 1.3 or later is part of all distributions of \LaTeX{}
version 2005/12/01 or later.

This work has the LPPL maintenance status `maintained'.

The Current Maintainer of this work is Niklas Beisert.

This work consists of the files |README.txt|, |childdoc.ins| and |childdoc.dtx|
as well as the derived files |childdoc.def|, |cdocsamp.tex|
with |cdocsch1.tex|, |cdocsch2.tex|, |cdocspt3.tex|, |cdocspt4.tex|,
|cdocsdrf.tex|, |cdocsfn1.tex|, |cdocsfn2.tex|
as well as |childdoc.pdf|.

%%%%%%%%%%%%%%%%%%%%%%%%%%%%%%%%%%%%%%%%%%%%%%%%%%%%%%%%%%%%%%%%%%%%%%%%%%%%%%%%
\subsection{Files and Installation}

The package consists of the files:
%
\begin{center}
\begin{tabular}{ll}
    |README.txt|   & readme file \\
    |childdoc.ins| & installation file \\
    |childdoc.dtx| & source file \\
    |childdoc.def| & definition file \\
    |cdocsamp.tex| & sample main file \\
    |cdocsch1.tex| & sample include file \\
    |cdocsch2.tex| & sample include file \\
    |cdocspt3.tex| & sample part file \\
    |cdocspt4.tex| & sample part file \\
    |cdocsdrf.tex| & sample redirection file \\
    |cdocsfn1.tex| & sample redirection file \\
    |cdocsfn2.tex| & sample redirection file \\
    |childdoc.pdf| & manual
\end{tabular}
\end{center}
%
The distribution consists of the files
|README.txt|, |childdoc.ins| and |childdoc.dtx|.
%
\begin{itemize}
\item
Run (pdf)\LaTeX{} on |childdoc.dtx|
to compile the manual |childdoc.pdf| (this file).
\item
Run \LaTeX{} on |childdoc.ins| to create the definitions file |childdoc.def|
and the sample |cdocsamp.tex| with include files
|cdocsch1.tex|, |cdocsch2.tex|, |cdocspt3.tex|, |cdocspt4.tex|,
|cdocsdrf.tex|, |cdocsfn1.tex|, |cdocsfn2.tex|.
Then copy the file |childdoc.def| to an appropriate directory of your \LaTeX{}
distribution, e.g.\ \textit{texmf-root}|/tex/latex/childdoc|.
\end{itemize}

%%%%%%%%%%%%%%%%%%%%%%%%%%%%%%%%%%%%%%%%%%%%%%%%%%%%%%%%%%%%%%%%%%%%%%%%%%%%%%%%
\subsection{Related CTAN Packages}

There are several other packages which offer a similar functionality:
%
\begin{itemize}
\item
The packages
\href{http://ctan.org/pkg/docmute}{\textsf{docmute}},
\href{http://ctan.org/pkg/includex}{\textsf{includex}} and
\href{http://ctan.org/pkg/standalone}{\textsf{standalone}}
provide commands to include only the document body of
a child file thus allowing both files to be compiled individually.
\item
The packages \href{http://ctan.org/pkg/subdocs}{\textsf{subdocs}}
and \href{http://ctan.org/pkg/subfiles}{\textsf{subfiles}}
provide structures in which the main and child documents can be
encapsulated and allowing them to be compiled individually.
The inclusion mechanism is different from the conventional |\include|.
\item
The package \href{http://ctan.org/pkg/combine}{\textsf{combine}}
is an elaborate solution to combine several documents into one.
\end{itemize}
%
See also the CTAN topic \href{http://ctan.org/topic/subdocs}{\textsf{subdocs}}
for further related packages.
The present package differs from the above solutions in that
a document structure constructed with the conventional |\include| mechanism
just needs two extra commands at the top of every file
such that all constituent files can be compiled individually.

%%%%%%%%%%%%%%%%%%%%%%%%%%%%%%%%%%%%%%%%%%%%%%%%%%%%%%%%%%%%%%%%%%%%%%%%%%%%%%%%
%\subsection{Feature Suggestions}
%
%The following is a list of features which may be useful for future
%versions of this package:
%%
%\begin{itemize}
%\item
%\ldots
%\end{itemize}

%%%%%%%%%%%%%%%%%%%%%%%%%%%%%%%%%%%%%%%%%%%%%%%%%%%%%%%%%%%%%%%%%%%%%%%%%%%%%%%%
\subsection{Revision History}

%%%%%%%%%%%%%%%%%%%%%%%%%%%%%%%%%%%%%%%%
\paragraph{v2.0:} 2018/12/30

\begin{itemize}
\item
immediate forward processing
\item
added |\childdocby| mechanism
\item
manual restructured
\end{itemize}

%%%%%%%%%%%%%%%%%%%%%%%%%%%%%%%%%%%%%%%%
\paragraph{v1.6:} 2018/01/17

\begin{itemize}
\item
application for development of include files
\item
corrections to manual
\end{itemize}

%%%%%%%%%%%%%%%%%%%%%%%%%%%%%%%%%%%%%%%%
\paragraph{v1.5:} 2017/05/21

\begin{itemize}
\item
more complete structuring introduced
\item
|\childdocof| introduced
\item
|\childdoc| renamed to |\childdocmain|
\item
|\childredirect| renamed to |\childdocforward| and |\childdocforwardprefix|
and functionality expanded
\end{itemize}

%%%%%%%%%%%%%%%%%%%%%%%%%%%%%%%%%%%%%%%%
\paragraph{v1.0:} 2017/04/27

\begin{itemize}
\item
manual and install package
\item
first version published on CTAN
\end{itemize}

%%%%%%%%%%%%%%%%%%%%%%%%%%%%%%%%%%%%%%%%
\paragraph{v0.6:} 2017/04/26

\begin{itemize}
\item
redirection mechanism added
\end{itemize}

%%%%%%%%%%%%%%%%%%%%%%%%%%%%%%%%%%%%%%%%
\paragraph{v0.5:} 2017/04/26

\begin{itemize}
\item
functionality in definition file
\end{itemize}


%%%%%%%%%%%%%%%%%%%%%%%%%%%%%%%%%%%%%%%%%%%%%%%%%%%%%%%%%%%%%%%%%%%%%%%%%%%%%%%%
%%%%%%%%%%%%%%%%%%%%%%%%%%%%%%%%%%%%%%%%%%%%%%%%%%%%%%%%%%%%%%%%%%%%%%%%%%%%%%%%
%%%%%%%%%%%%%%%%%%%%%%%%%%%%%%%%%%%%%%%%%%%%%%%%%%%%%%%%%%%%%%%%%%%%%%%%%%%%%%%%
\appendix

\settowidth\MacroIndent{\rmfamily\scriptsize 000\ }

 \DocInput{childdoc.dtx}

\end{document}
%</driver>
% \fi
%
% %%%%%%%%%%%%%%%%%%%%%%%%%%%%%%%%%%%%%%%%%%%%%%%%%%%%%%%%%%%%%%%%%%%%%%%%%%%%%%
% %%%%%%%%%%%%%%%%%%%%%%%%%%%%%%%%%%%%%%%%%%%%%%%%%%%%%%%%%%%%%%%%%%%%%%%%%%%%%%
% \section{Sample}
%\iffalse
%<*samplemain>
%\fi
%
% The following presents a sample document
% with two chapters, two parts, a title page,
% a compile flag as well as three forwarding files to set the flag.
% It consists of eight |.tex| files:
% \begin{center}
% \begin{tabular}{ll}
% |cdocsamp.tex|&main file\\
% |cdocsch1.tex|&include file for chapter 1\\
% |cdocsch2.tex|&include file for chapter 2\\
% |cdocspt3.tex|&include file for part 3\\
% |cdocspt4.tex|&include file for part 4\\
% |cdocsdrf.tex|&forwarding file for main file in draft mode\\
% |cdocsfi1.tex|&forwarding file for final version of chapter 1\\
% |cdocsfi2.tex|&forwarding file for final version of chapter 2\\
% \end{tabular}
% \end{center}
% Each of the eight files can be compiled directly by the \LaTeX{} compiler.
%
% %%%%%%%%%%%%%%%%%%%%%%%%%%%%%%%%%%%%%%
% \paragraph{Main File.}
%
% The main file is called |cdocsamp.tex|.
%
% Load the \textsf{childdoc} definitions and
% declare the filename for the main document:
%    \begin{macrocode}
\input{childdoc.def}
\childdocmain{}
%    \end{macrocode}

% Optional override for |\version| flag:
%    \begin{macrocode}
%%\ifchilddoc\else\providecommand{\version}{draft}\fi
%    \end{macrocode}

% Define the default values for the |\version| flag
% (|final| for the main file and |draft| for childs):
%    \begin{macrocode}
\ifchilddoc
\providecommand{\version}{draft}
\else
\providecommand{\version}{final}
\fi
%    \end{macrocode}

% Load the standard document class:
%    \begin{macrocode}
\documentclass[12pt]{article}
%    \end{macrocode}

% Start the document body:
%    \begin{macrocode}
\begin{document}
%    \end{macrocode}

% Declare a title page.
% Print title, part of document being processed and version flag:
%    \begin{macrocode}
\addtocounter{page}{-1}
\begin{center}
{\LARGE\bfseries{}childdoc example\par}
\vspace{1cm}
\ifchilddoc
\ifchilddocmanual part\else chapter\fi:
`\childdocname' of `\childdocjob'\par
\else
main document: `\childdocjob'\par
\fi
version: \version\par
\end{center}
\newpage
%    \end{macrocode}

% Manually include selected file,
% otherwise process as usual:
%    \begin{macrocode}
\ifchilddocmanual
\section*{part `\childdocname'}
\input{\childdocname}
\else
%    \end{macrocode}

% Include the two chapters:
%    \begin{macrocode}
\include{cdocsch1}
\include{cdocsch2}
%    \end{macrocode}

% Include the two parts unless only chapters should be displayed:
%    \begin{macrocode}
\ifchilddoc\else
\section{part three}
\input{cdocspt3}
\section{part four}
\input{cdocspt4}
\fi
%    \end{macrocode}

% Process as usual until here:
%    \begin{macrocode}
\fi
%    \end{macrocode}

% End of document body:
%    \begin{macrocode}
\end{document}
%    \end{macrocode}
%\iffalse
%</samplemain>
%\fi
%
% %%%%%%%%%%%%%%%%%%%%%%%%%%%%%%%%%%%%%%
% \paragraph{Chapter Include Files.}
%
% The include files are called |cdocsch1.tex| and |cdocsch2.tex|.
%
%\iffalse
%<*samplechap1|samplechap2>
%\fi

% Optional override for |\version| flag:
%    \begin{macrocode}
%%\providecommand{\version}{final}
%    \end{macrocode}

% Include the main document:
%    \begin{macrocode}
\input{childdoc.def}
\childdocof{cdocsamp}
%    \end{macrocode}

%\iffalse
%</samplechap1|samplechap2>
%\fi
%
%\iffalse
%<*samplechap1>
%\fi
% Some text for chapter 1:
%    \begin{macrocode}
\section{one}
some text in chapter one
%    \end{macrocode}

%\iffalse
%</samplechap1>
%\fi
% Some text for chapter 2:
%\iffalse
%<*samplechap2>
%\fi
%    \begin{macrocode}
\section{two}
more text in chapter two
%    \end{macrocode}

%\iffalse
%</samplechap2>
%\fi
%
% %%%%%%%%%%%%%%%%%%%%%%%%%%%%%%%%%%%%%%
% \paragraph{Part Include Files.}
%
% The include files are called |cdocspt3.tex| and |cdocspt4.tex|.
%
%\iffalse
%<*samplepart3|samplepart4>
%\fi

% Optional override for |\version| flag:
%    \begin{macrocode}
%%\providecommand{\version}{final}
%    \end{macrocode}

% Include the main document:
%    \begin{macrocode}
\input{childdoc.def}
\childdocby{cdocsamp}
%    \end{macrocode}

%\iffalse
%</samplepart3|samplepart4>
%\fi
%
%\iffalse
%<*samplepart3>
%\fi
% Some text for part 3:
%    \begin{macrocode}
some text in part three
%    \end{macrocode}

%\iffalse
%</samplepart3>
%\fi
% Some text for part 4:
%\iffalse
%<*samplepart4>
%\fi
%    \begin{macrocode}
more text in part four
%    \end{macrocode}

%\iffalse
%</samplepart4>
%\fi
%
% %%%%%%%%%%%%%%%%%%%%%%%%%%%%%%%%%%%%%%
% \paragraph{Forwarding for a Complete Draft.}
%
% The following forwarding file |cdocsdrf.tex|
% compiles the main document in draft mode:
%\iffalse
%<*sampledraft>
%\fi
%    \begin{macrocode}
\def\version{draft}
\input{childdoc.def}
\childdocforward{cdocsamp}
%    \end{macrocode}

%\iffalse
%</sampledraft>
%\fi
%
% %%%%%%%%%%%%%%%%%%%%%%%%%%%%%%%%%%%%%%
% \paragraph{Forwarding for Final Version of the Chapters.}
%
% The following forwarding files |cdocsfn1.tex| and |cdocsfn2.tex|
% (with identical content)
% compile the final versions of the child documents
% |cdocsch1.tex| and |cdocsch2.tex|, respectively:
%\iffalse
%<*samplefinal>
%\fi
%    \begin{macrocode}
\def\version{final}
\input{childdoc.def}
\childdocforwardprefix[cdocsamp]{cdocsfn}{cdocsch}
%    \end{macrocode}

%\iffalse
%</samplefinal>
%\fi
%
% %%%%%%%%%%%%%%%%%%%%%%%%%%%%%%%%%%%%%%
% \paragraph{Command Line Processing.}
%
% The following three command lines generate the output files
% |cdocscld|, |cdocscl1| and |cdocscl2|
% which should be identical to
% |cdocsdrf|, |cdocsch1| and |cdocsfn2|, respectively:
% \begin{center}
% \begin{tabular}{l}
% |latex -jobname cdocscld \|\\
% |  "\def\version{draft}\input{childdoc.def}\childdocforward{cdocsamp}"|\\
% |latex -jobname cdocscl1 \|\\
% |  "\input{childdoc.def}\childdocforward[cdocsamp]{cdocsch1}"|\\
% |latex -jobname cdocscl2 \|\\
% |  "\def\version{final}\input{childdoc.def}\childdocforward{cdocsch2}"|
% \end{tabular}
% \end{center}
% Note that the trailing backslash on each first line
% merely continues the input to the second line
% (for convenient cut ant paste).
% Furthermore, the command |latex| can be replaced by any
% of its alternative versions such as |pdflatex|.
%
% %%%%%%%%%%%%%%%%%%%%%%%%%%%%%%%%%%%%%%%%%%%%%%%%%%%%%%%%%%%%%%%%%%%%%%%%%%%%%%
% %%%%%%%%%%%%%%%%%%%%%%%%%%%%%%%%%%%%%%%%%%%%%%%%%%%%%%%%%%%%%%%%%%%%%%%%%%%%%%
% \section{Implementation}
%\iffalse
%<*package>
%\fi
%
% This section describes the definitions file |childdoc.def|.

% The definitions cannot be loaded using |\usepackage| or |\RequirePackage|
% which has a mechanism to prevent loading a style file more than once.
% When loading the definitions by means of |\input|
% multiple instances have to be prevented manually:
%\iffalse
%This code needs to be before the `\ProvidesFile' directive
%which is defined at the beginning of this file.
%Therefore it is also placed there and commented out here.
%</package>
%<*discard>
%\fi
%    \begin{macrocode}
\ifdefined\childdocmain\endinput\fi
%    \end{macrocode}
%\iffalse
%</discard>
%<*package>
%\fi
%
% \macro{\ifchilddoc}
% \macro{\ifchilddocmanual}
% The conditional |\ifchilddoc| tells whether a
% child (true) or main (false) document is being compiled.
% The conditional |\ifchilddocmanual| tells whether
% the |\includeonly| mechanism is used (false) or
% the selection of child files must be performed manually (true).
% The definitions initialise to false:
%    \begin{macrocode}
\newif\ifchilddoc
\newif\ifchilddocmanual
%    \end{macrocode}

% \macro{\childdocname}
% \macro{\childdocjob}
% The macro |\childdocname| stores the name of the main document
% to be compiled. The macro |\childdocjob| stores the name of
% the document on which the \LaTeX{} compiler was originally invoked.
% The content of |\jobname| cannot be compared
% to filenames specified in the source due to different catcodes.
% The following code rescans |\jobname|, stores the result
% in |\childdocname| and saves a copy in |\childdocjob|:
%    \begin{macrocode}
\edef\childdocname{\scantokens\expandafter{\jobname\noexpand}}
\let\childdocjob\childdocname
%    \end{macrocode}

% \macro{\childdocdisable}
% The macro |\childdocdisable| prevents the main file
% from being processed more than once.
% At this stage, the main document command |\childdocmain|
% is assumed to be called once again where it should do nothing.
% Any subsequent call to it should prevent
% a secondary processing of the main document
% It overwrites the forwarding commands
% |\childdocof| and |\childdocforward|
% with empty macros to prevent further inclusions of the main document:
%    \begin{macrocode}
\newcommand{\childdocdisable}
{
  \renewcommand{\childdocmain}[1]{\renewcommand{\childdocmain}[1]{\endinput}}
  \renewcommand{\childdocof}[1]{}
  \renewcommand{\childdocby}[2][]{}
  \renewcommand{\childdocforward}[2][]{}
  \renewcommand{\childdocdisable}{}
}
%    \end{macrocode}

% \macro{\childdocmain}
% The macro |\childdocmain| is to be called at the top of the main file
% with nothing or the main filename (without extension) as argument.
% First, it breaks loops.
% If the argument is not empty and does not match |\childdocname|
% (which is set by the first inclusion of |childdoc.def|),
% |\ifchilddoc| is set to true, |\includeonly| is applied to the child file
% and |\jobname| is set to the main file
% (for proper handling of |.aux| files):
%    \begin{macrocode}
\newcommand{\childdocmain}[1]
{
  \childdocdisable\childdocmain{}
  \if?#1?\else
    \begingroup
      \def\childdoctmp{#1}
      \ifx\childdoctmp\childdocname
        \def\childdoctmp{}
      \else
        \def\childdoctmp
        {
          \childdoctrue
          \includeonly{\childdocname}
          \def\childdocjob{#1}
          \def\jobname{#1}
        }
      \fi
      \expandafter
    \endgroup
    \childdoctmp
  \fi
}
%    \end{macrocode}

% \macro{\childdocof}
% The command |\childdocof| redirects
% compilation to the main file |#1|.
%    \begin{macrocode}
\newcommand{\childdocof}[1]
{
  \childdocdisable
  \childdoctrue
  \includeonly{\childdocname}
  \def\jobname{#1}
  \def\childdocjob{#1}
  \input{#1}
}
%    \end{macrocode}

% \macro{\childdocby}
% The command |\childdocby| ....
%    \begin{macrocode}
\newcommand{\childdocby}[2][]
{
  \childdocdisable
  \childdoctrue
  \childdocmanualtrue
  \if?#1?\else
    \def\jobname{#2}
  \fi
  \def\childdocjob{#2}
  \input{#2}
  \endinput
}
%    \end{macrocode}

% \macro{\childdocforward}
% The command |\childdocforward| redirects
% compilation to the main file or
% (if the optional argument is given) a child file.
% Parameters are set as if the main file
% or a child file starting with |\childdocof| was compiled.
% Then compilation is handed over to the main file:
%    \begin{macrocode}
\newcommand{\childdocforward}[2][]
{
  \begingroup
    \if?#1?
      \def\childdoctmp
      {
        \def\childdocname{#2}
        \def\childdocjob{#2}
        \def\jobname{#2}
        \input{#2}
        \endinput
      }
    \else
      \def\childdoctmp
      {
        \childdocdisable
        \def\childdocname{#2}
        \childdoctrue
        \includeonly{#2}
        \def\childdocjob{#1}
        \def\jobname{#1}
        \input{#1}
        \endinput
      }
    \fi
    \expandafter
  \endgroup
  \childdoctmp
}
%    \end{macrocode}

% \macro{\childdocforwardprefix}
% The command |\childdocforwardprefix| redirects
% compilation to the main or a child file by means of a pattern.
% The prefix |#1| in the current filename is replaced by |#2|
% and the suffix of the current filename is kept
% (it is assumed that the filename does not contain the substring `|~~~|'
% which is used as a delimiter).
% Compilation is handed over to the new file by |\childdocforward|:
%    \begin{macrocode}
\newcommand{\childdocforwardprefix}[3][]
{
  \begingroup
    \def\childdocextract #2##1~~~{\def\childdoctmp{\childdocforward[#1]{#3##1}}}
    \expandafter\childdocextract\childdocname~~~
    \expandafter
  \endgroup
  \childdoctmp
}
%    \end{macrocode}

% \macro{\childdoc}
% The deprecated macro |\childdoc| is a legacy version of |\childdocmain|:
%    \begin{macrocode}
\newcommand{\childdoc}{\childdocmain}
%    \end{macrocode}

% \macro{\childdocredirect}
% The deprecated macro |\childdocredirect| is a legacy version
% of |\childdocforward| and |\childdocforwardprefix|:
%    \begin{macrocode}
\newcommand{\childdocredirect}[2][]
{
  \begingroup
    \if?#1?
      \def\childdoctmp{\childdocforward{#2}}
    \else
      \def\childdoctmp{\childdocforwardprefix{#1}{#2}}
    \fi
    \expandafter
  \endgroup
  \childdoctmp
}
%    \end{macrocode}

%\iffalse
%</package>
%\fi
%
\endinput
\childdocforward{cdocsamp}"|\\
% |latex -jobname cdocscl1 \|\\
% |  "% \iffalse
%
% childdoc.dtx Copyright (C) 2017-2018 Niklas Beisert
%
% This work may be distributed and/or modified under the
% conditions of the LaTeX Project Public License, either version 1.3
% of this license or (at your option) any later version.
% The latest version of this license is in
%   http://www.latex-project.org/lppl.txt
% and version 1.3 or later is part of all distributions of LaTeX
% version 2005/12/01 or later.
%
% This work has the LPPL maintenance status `maintained'.
%
% The Current Maintainer of this work is Niklas Beisert.
%
% This work consists of the files childdoc.dtx and childdoc.ins
% and the derived files childdoc.def and cdocsamp.tex with
% cdocsch1.tex, cdocsch2.tex, cdocsdrf.tex, cdocsfn1.tex, cdocsfn2.tex.
%
%<package>\ifdefined\childdocmain\endinput\fi
%<package>\ProvidesFile{childdoc.def}[2018/12/30 v2.0 child document driver]
%<samplemain>\ProvidesFile{cdocsamp.tex}[2018/12/30 v2.0 sample for childdoc]
%<*driver>
%\ProvidesFile{childdoc.drv}[2018/12/30 v2.0 childdoc reference manual file]
\PassOptionsToClass{10pt,a4paper}{article}
\documentclass{ltxdoc}

\usepackage[margin=35mm]{geometry}
\usepackage{hyperref}
\usepackage{hyperxmp}
\usepackage[usenames]{color}

\hypersetup{colorlinks=true}
\hypersetup{pdfstartview=FitH}
\hypersetup{pdfpagemode=UseNone}
\hypersetup{pdfsource={}}
\hypersetup{pdflang={en-UK}}
\hypersetup{pdfcopyright={Copyright 2017-2018 Niklas Beisert.
  This work may be distributed and/or modified under the
  conditions of the LaTeX Project Public License, either version 1.3
  of this license or (at your option) any later version.}}
\hypersetup{pdflicenseurl={http://www.latex-project.org/lppl.txt}}
\hypersetup{pdfcontactaddress={ETH Zurich, ITP, HIT K,
  Wolfgang-Pauli-Strasse 27}}
\hypersetup{pdfcontactpostcode={8093}}
\hypersetup{pdfcontactcity={Zurich}}
\hypersetup{pdfcontactcountry={Switzerland}}
\hypersetup{pdfcontactemail={nbeisert@itp.phys.ethz.ch}}
\hypersetup{pdfcontacturl={http://people.phys.ethz.ch/\xmptilde nbeisert/}}

\newcommand{\secref}[1]{\hyperref[#1]{section \ref*{#1}}}

\parskip1ex
\parindent0pt
\let\olditemize\itemize
\def\itemize{\olditemize\parskip0pt}

\begin{document}

\title{The \textsf{childdoc} Package}
\hypersetup{pdftitle={The childdoc Package}}
\author{Niklas Beisert\\[2ex]
  Institut f\"ur Theoretische Physik\\
  Eidgen\"ossische Technische Hochschule Z\"urich\\
  Wolfgang-Pauli-Strasse 27, 8093 Z\"urich, Switzerland\\[1ex]
  \href{mailto:nbeisert@itp.phys.ethz.ch}
  {\texttt{nbeisert@itp.phys.ethz.ch}}}
\hypersetup{pdfauthor={Niklas Beisert}}
\hypersetup{pdfsubject={Manual for the LaTeX2e Package childdoc}}
\date{30 December 2018, \textsf{v2.0}}
\maketitle

\begin{abstract}\noindent
\textsf{childdoc} is a \LaTeXe{} package
that enables the direct compilation
of document sections included by |\include|
to individual files.
\end{abstract}

\begingroup
\parskip0ex
\tableofcontents
\endgroup

%%%%%%%%%%%%%%%%%%%%%%%%%%%%%%%%%%%%%%%%%%%%%%%%%%%%%%%%%%%%%%%%%%%%%%%%%%%%%%%%
%%%%%%%%%%%%%%%%%%%%%%%%%%%%%%%%%%%%%%%%%%%%%%%%%%%%%%%%%%%%%%%%%%%%%%%%%%%%%%%%
\section{Introduction}

\LaTeX{} provides a mechanism to structure a large document (such as a book)
into a main file and several child files (containing the chapters)
using the |\include| command.
This mechanism is beneficial for documents
which span hundreds of pages in order to
make the source file(s) more manageable.
Moreover, compilation can be restricted to
selected child files by means of the |\includeonly| command.
The latter feature can be used to reduce the compilation time while editing
(this was significantly more useful in the earlier days of \LaTeX{})
or to generate a smaller document which is easier to navigate.
Another application of |\includeonly| is to generate
documents consisting of selected parts of the complete document.

However, there are a few drawbacks of the plain |\include| mechanism:
\begin{itemize}
\item
The child files cannot be compiled on their own,
they can only be compiled via the main file.
A naive editing environment
(such as a text editor with an option
to have the current file processed by \LaTeX)
may require one to switch to the main file before compiling;
attempting to compile the child file produces errors.
\item
The main file must be modified (each time)
to adjust the |\includeonly| command
to the present needs. This easily leaves the main file in a messy state.
\item
The generated document will always carry the filename
of the main document. This is inconvenient if
several child files are to be compiled and
to be kept for distribution.
\end{itemize}

The present package provides a simple interface
to make child files individually compilable by \LaTeX{}.
Compiling a child file then has the same effect as compiling
the main file with an |\includeonly| command
to select the appropriate child.
Moreover the generated document will carry the name of the child
rather than the main file.
This resolves all three above issues.

This feature is meant to make the editing of books,
thesis documents and lecture notes somewhat more convenient.
However, the package can also be used efficiently for
composing a series of documents (such as exercise sheets)
which are typically distributed individually.
It then assists the author in generating the individual documents
(potentially in different versions)
as well as a document containing the collected series.
Another application is in developing style files
or other kinds of included material
where compilation of the style file could redirect
to a sample or test file.

%%%%%%%%%%%%%%%%%%%%%%%%%%%%%%%%%%%%%%%%%%%%%%%%%%%%%%%%%%%%%%%%%%%%%%%%%%%%%%%%
%%%%%%%%%%%%%%%%%%%%%%%%%%%%%%%%%%%%%%%%%%%%%%%%%%%%%%%%%%%%%%%%%%%%%%%%%%%%%%%%
\section{Usage}

First of all, the package \textsf{childdoc} is \emph{not} a standard
\LaTeXe{} |.sty| style file! Therefore it needs to be invoked in
a non-standard way.

%%%%%%%%%%%%%%%%%%%%%%%%%%%%%%%%%%%%%%%%%%%%%%%%%%%%%%%%%%%%%%%%%%%%%%%%%%%%%%%%
\subsection{Included Files}
\label{sec:include}

%%%%%%%%%%%%%%%%%%%%%%%%%%%%%%%%%%%%%%%%
\DescribeMacro{\childdocmain}
To use the package, add the commands
\begin{center}
\begin{tabular}{l}
|\input{childdoc.def}|\\
|\childdocmain{}|\\
\end{tabular}
\end{center}
at the very top of the main \LaTeX{} file,
in particular \emph{before} the |\documentclass| statement!
The argument of |\childdocmain| should be left empty
(but it must be present).

%%%%%%%%%%%%%%%%%%%%%%%%%%%%%%%%%%%%%%%%
\DescribeMacro{\childdocof}
Furthermore, add the commands
\begin{center}
\begin{tabular}{l}
|\input{childdoc.def}|\\
|\childdocof{|\textit{main}|}|\\
\end{tabular}
\end{center}
at the top of every child file \textit{child}
which is included by |\include{|\textit{child}|}|
from within the main file
(or at least for those files to be compiled individually).
The argument \textit{main} must be the filename of the main file.

There are a couple of
considerations in setting up the main and child documents:

%%%%%%%%%%%%%%%%%%%%%%%%%%%%%%%%%%%%%%%%
\paragraph{Restrictions.}

Please note the following restrictions:
\begin{itemize}
\item
|\childdocmain| must be called with one argument \textit{main}
to ensure compatibility with earlier version of the package.
It must either be empty (|\childdocmain{}|)
or precisely match the filename of the main file in which it is specified.
See \secref{sec:detection} for further information.
\item
The filename \textit{main} must be specified without the |.tex| extension.
\item
The filename \textit{main} is case sensitive
(even in case-insensitive file systems)
due to internal string comparison.
\item
The argument \textit{main} should be fully expanded, it cannot be a macro.
\item
Subdirectories and special characters should be avoided in filenames.
\item
The command |\childdocmain{|\textit{main}|}| must be followed by a whitespace.
It should not be followed immediately by another command
or by a comment mark `|%|'.
This is because the \TeX{} parser reads the token immediately following
the argument of |\childdocmain| and puts it
at the beginning of every child section;
however, a white\-space is ignored.
\end{itemize}

%%%%%%%%%%%%%%%%%%%%%%%%%%%%%%%%%%%%%%%%
\paragraph{Content of Main File.}

It is advisable to place all content in the child files included by |\include|.
Any output contained in the main file will appear in all child documents
unless suppressed manually;
it cannot be suppressed automatically by the |\includeonly| directive
and thus should normally be avoided.
A method to include some content in the main file
by means of conditional processing is described in \secref{sec:conditional}.

%%%%%%%%%%%%%%%%%%%%%%%%%%%%%%%%%%%%%%%%
\paragraph{Page Numbering.}

When only a part of the document is compiled,
the appropriate numbering of pages
(as well as other status parameters)
is determined from the |.aux| files.
The latter contain information from previous passes.
However this information needs to propagate through
all intermediate child documents.
Therefore the page numbering in child documents may well
be inconsistent until the complete document is compiled at least once.

A useful (if unconventional) way to always ensure a consistent
page numbering is to restart the numbering in each child document
and denote the pages by `\textit{child}|.|\textit{page}'
where \textit{child} represents the chapter/section number of the child file.
This can be achieved by the command
|\numberwithin{page}{|\textit{child}|}|
of the \textsf{amsmath} package
where \textit{child} can be |chapter| or |section|
depending on the chosen structuring.
Alternatively, one can modify the macro |\thepage| appropriately
and reset the counter |page| at the start of each child file.

%%%%%%%%%%%%%%%%%%%%%%%%%%%%%%%%%%%%%%%%%%%%%%%%%%%%%%%%%%%%%%%%%%%%%%%%%%%%%%%%
\subsection{Conditional Processing}
\label{sec:conditional}

The package provides a mechanism to compile different versions
of a document. To customise the versions further some conditional processing
can come in handy to distinguish which version is being compiled.
The package provides two macros to describe the compilation context:

%%%%%%%%%%%%%%%%%%%%%%%%%%%%%%%%%%%%%%%%
\DescribeMacro{\ifchilddoc}
The conditional |\ifchilddoc| distinguishes between the compilation of
child documents and the main document:
%
\begin{center}
|\ifchilddoc |\textit{child-code}| |[|\||else |\textit{main-code}]| \||fi|
\end{center}

%%%%%%%%%%%%%%%%%%%%%%%%%%%%%%%%%%%%%%%%
\DescribeMacro{\childdocname}
\DescribeMacro{\childdocjob}
The macro |\childdocname| contains the filename (without extension)
of the main or child file being processed.
Note that |\childdocjob| will always contain the name of the main file.

%%%%%%%%%%%%%%%%%%%%%%%%%%%%%%%%%%%%%%%%
\paragraph{Title Page.}

Conditional processing can be used to include a title or banner page
in the main document when proper precautions are taken.
Importantly, the code in the main file should ensure that the page counter
(as well as other status parameters which are stored in the |.aux| files)
takes the same value after the conditional processing.
Otherwise the page numbers may take divergent values
depending on which part is compiled.

For example, a title page could be declared by:
%
\begin{center}
\begin{tabular}{l}
|\ifchilddoc\||else|\\
|\addtocounter{page}{-1}|\\
\textit{code for title page}\\
|\newpage|\\
|\||fi|
\end{tabular}
\end{center}
%
A banner page for the child documents can be generated by:
%
\begin{center}
\begin{tabular}{l}
|\ifchilddoc|\\
|\addtocounter{page}{-1}|\\
\textit{code for banner page}\\
|\newpage|\\
|\||fi|
\end{tabular}
\end{center}
%
Here one could write a message such as:
\begin{center}
|This is the part \childdocname{} of \childdocjob{}.|
\end{center}

%%%%%%%%%%%%%%%%%%%%%%%%%%%%%%%%%%%%%%%%%%%%%%%%%%%%%%%%%%%%%%%%%%%%%%%%%%%%%%%%
\subsection{Flags}
\label{sec:flags}

The package makes it easy to generate different versions
of the main or child documents.
To this end compilation flags can be defined
and assigned different default values.
They will be particularly useful in conjunction
with the forwarding mechanism described in \secref{sec:forward}.

For example, it may be useful to have a flag |\version|
which can be set to |draft| or |final|.
The document source will contain some conditional code
depending on the value of |\version|.
Suppose further, the flag should default to |final| for the main file
and to |draft| for child files
which is a natural assignment for editing the document.
This is achieved by placing the following code
in the preamble of the main document
(below the |\childdocmain| directive):
%
\begin{center}
\begin{tabular}{l}
|\ifchilddoc|\\
|\providecommand{\version}{draft}|\\
|\||else|\\
|\providecommand{\version}{final}|\\
|\||fi|
\end{tabular}
\end{center}
%
The definition by |\providecommand| makes sure
that previous definitions are not overwritten.
Further statements |\providecommand{\version}{...}|
can thus be added before the above code to override it.

For the main file, one might add a line
(between |\childdocmain| and the above block)
%
\begin{center}
|%\ifchilddoc\||else\providecommand{\version}{draft}\||fi|
\end{center}
%
which can be uncommented to produce a draft version.
Likewise one can add a line to the very top of a child file
(above the |\childdocof{|\textit{main}|}| directive)
%
\begin{center}
|%\providecommand{\version}{final}|
\end{center}
%
which can be uncommented to produce the final version of this child document.

%%%%%%%%%%%%%%%%%%%%%%%%%%%%%%%%%%%%%%%%%%%%%%%%%%%%%%%%%%%%%%%%%%%%%%%%%%%%%%%%
\subsection{Forwarding}
\label{sec:forward}

Different versions of the main or child documents
using compilation flags as described in \secref{sec:flags}
can be (permanently) stored in different files
for convenient compilation, viewing and distribution.
To this end, the package defines a command
to pass on compilation to a different file:

%%%%%%%%%%%%%%%%%%%%%%%%%%%%%%%%%%%%%%%%
\DescribeMacro{\childdocforward}
The command |\childdocforward| redirects processing to
another source file:
%
\begin{center}
\begin{tabular}{l}
|\input{childdoc.def}|\\
|\childdocforward[|\textit{main}|]{|\textit{dest}|}|\\
\end{tabular}
\end{center}
%
The argument \textit{dest} is the destination file
(without extension).
It should be the main file or one of the child files.
Note that further \textsf{childdoc} directives
such as |\childdocof| and |\childdocforward|
in the indicated file will be processed in this form.
The optional argument \textit{main}
passes on directly to the main file \textit{main}
while pretending to compile the child \textit{dest}.
This form behaves as if \textit{dest}
issues |\childdocof{|\textit{main}|}| right away,
and no further \textsf{childdoc} directives will be processed.

%%%%%%%%%%%%%%%%%%%%%%%%%%%%%%%%%%%%%%%%
\DescribeMacro{\...prefix}
In the alternative form |\childdocforwardprefix|,
%
\begin{center}
\begin{tabular}{l}
|\input{childdoc.def}|\\
|\childdocforwardprefix[|\textit{main}|]{|\textit{prefix}|}{|\textit{dest}|}|
\end{tabular}
\end{center}
%
the destination file is determined by a pattern
depending on the current file:
To make this work, the current file must be called
`{\textit{prefix}\hspace{0.2em}\textit{suffix}}'
with \textit{prefix} matching precisely the argument.
Processing is then passed on to the file
`{\textit{dest}\hspace{0.2em}\textit{suffix}}'.
Surely, the same effect is achieved by
directly specifying the
argument `{\textit{dest}\hspace{0.2em}\textit{suffix}}'
in the first form.
However, that requires to set up a different file
for each child. With the alternative form of the command
all these files can have exactly the same content
which simplifies setting them up and maintaining them.

For example, the following file |draft.tex|
with a compilation flag |\version| as described in \secref{sec:flags}
compiles the main document as a draft:
%
\begin{center}
\begin{tabular}{l}
|\def\version{draft}|\\
|\input{childdoc.def}|\\
|\childdocforward{|\textit{main}|}|
\end{tabular}
\end{center}
%
Likewise, the following files |final|\textit{nn}|.tex|
compile the final version of the child document
|child|\textit{nn}|.tex|:
%
\begin{center}
\begin{tabular}{l}
|\def\version{final}|\\
|\input{childdoc.def}|\\
|\childdocforwardprefix{final}{child}|
\end{tabular}
\end{center}
%

Note that when several versions of a main file and/or of each child file
are to be generated, it may be convenient to set up a |Makefile| or
shell script to automatise the process.

%%%%%%%%%%%%%%%%%%%%%%%%%%%%%%%%%%%%%%%%%%%%%%%%%%%%%%%%%%%%%%%%%%%%%%%%%%%%%%%%
\subsection{Command Line Processing}
\label{sec:commandline}

The effect of redirection files can also be achieved by invoking
the \LaTeX{} compiler with a more elaborate command line.
Most conveniently this should be done as part
of a shell script or a |Makefile|.

When using \textsf{childdoc} in the main file, the following
command lines effectively perform a redirection
(note that depending on the shell being used,
backslashes may have to be doubled: `|\|' $\to$ `|\\|'):
%
\begin{center}
|... -jobname "|\textit{target}|" |\\|"|[\textit{flags}]%
|\input{childdoc.def}\childdocforward[|\textit{main}|]{|\textit{dest}|}"|
\end{center}
%
Here \textit{target} is the name of the output file,
\textit{main} is the name of the main file
and \textit{dest} is the name of the main or child file to be processed
(all filenames without extensions).
The optional argument \textit{main} can be omitted
if \textit{main} matches \textit{dest}.
Optionally, compilation \textit{flags} can be defined via |\def| commands.
This command line makes the \TeX{} engine believe
it is compiling the file \textit{target}
whose content is specified as the latter parameter.
The provided code then forwards the processing to
\textit{main} or \textit{dest} as described in \secref{sec:forward}.

%%%%%%%%%%%%%%%%%%%%%%%%%%%%%%%%%%%%%%%%%%%%%%%%%%%%%%%%%%%%%%%%%%%%%%%%%%%%%%%%
\subsection{Include by Input}
\label{sec:input}

Including child documents by |\include| has some restrictions by design.
Most notably, the content of a child document always occupies
its own set of pages; pages cannot be shared between child documents.
Usually, this behaviour makes perfect sense
because each child document contain an essential part of the document.
However, in some situations it may be desirable to compose
a document from a collection of parts
without having mandatory page breaks between then.
For this case, the package
provides a mechanism to include parts
by |\input| which can also be processed individually.
However, by construction this mechanism
requires manual handling of the content to be output.

%%%%%%%%%%%%%%%%%%%%%%%%%%%%%%%%%%%%%%%%
\DescribeMacro{\ifchilddocmanual}
The main file should be prepared as usual, see \secref{sec:include}.
However, the document body must make a distinction
between processing of an individual part and of the main document, e.g.:
%
\begin{center}
\begin{tabular}{l}
|\ifchilddocmanual|\\
|\input{\childdocname}|\\
|\||else|\\
\textit{document body with }|\input{|\textit{part}|}|\\
|\||fi|
\end{tabular}
\end{center}
%
The conditional |\ifchilddocmanual| is true whenever
a part to be included by |\input| is being compiled,
and the name of the part is stored in |\childdocname|.

%%%%%%%%%%%%%%%%%%%%%%%%%%%%%%%%%%%%%%%%
\DescribeMacro{\childdocby}
Each part to be included by |\input| should start with:
%
\begin{center}
\begin{tabular}{l}
|\input{childdoc.def}|\\
|\childdocby{|\textit{main}|}|\\
\end{tabular}
\end{center}
%
The directive |\childdocby| is similar to |\childdocof|
described in \secref{sec:include},
but the subsequent selection of content must be done manually.
To that end, both |\ifchilddoc| and |\ifchilddocmanual|
will be true upon processing of a part,
and the name of the part is stored in |\childdocname|.
Note that |\jobname| will be set to the filename of the current part
so that each part receives an individual |.aux| file
that does not interfere with the |.aux| file(s) of the main document.
This behaviour can be altered by the alternative form
|\childdocby[*]{|\textit{main}|}| (with a non-empty optional argument)
which uses the |.aux| file of the main document
by setting |\jobname| to \textit{main}.

%%%%%%%%%%%%%%%%%%%%%%%%%%%%%%%%%%%%%%%%%%%%%%%%%%%%%%%%%%%%%%%%%%%%%%%%%%%%%%%%
\subsection{Driver Development}
\label{sec:driver}

The \textsf{childdoc} mechanism can also be use for the development
of definition files such as \LaTeX{} styles or classes.
This case differs from the above setup with multiple parts
included by |\include| in that no |\includeonly| should be invoked.
This can be achieved by starting the include file
(before |\ProvidesPackage|) with:
%
\begin{center}
\begin{tabular}{l}
|\input{childdoc.def}|\\
|\childdocforward{|\textit{main}|}|\\
\end{tabular}
\end{center}
%
or alternatively with:
%
\begin{center}
\begin{tabular}{l}
|\input{childdoc.def}|\\
|\childdocby{|\textit{main}|}|\\
\end{tabular}
\end{center}
%
Both forms have slightly different effects as described above.
The main file is prepared as usual, see \secref{sec:include}.

%%%%%%%%%%%%%%%%%%%%%%%%%%%%%%%%%%%%%%%%%%%%%%%%%%%%%%%%%%%%%%%%%%%%%%%%%%%%%%%%
\subsection{Legacy Detection}
\label{sec:detection}

The directive |\childdocmain| in the main file can detect
whether the complete document or merely a child is to be compiled
even without using the directive |\childdocof|.
This method is deprecated because it is less robust
and there is no compelling reason to use it;
it is merely provided for backward compatibility
and it may be removed in future versions.

If the detection mechanism is to be used,
it is mandatory to correctly specify
the filename of the main file as the argument of |\childdocmain|:
%
\begin{center}
\begin{tabular}{l}
|\input{childdoc.def}|\\
|\childdocmain{|\textit{main}|}|\\
\end{tabular}
\end{center}
%
If |\jobname| does not match the argument \textit{main} of |\childdocmain|,
it is assumed that |\jobname| points to the child file to be compiled.
When using |\childdocmain| with the main file specified as argument,
it suffices to start a child file
with just |\input{|\textit{main}|}|
without loading of the package and using |\childdocof|.
If instead all processing is done
with the appropriate \textsf{childdoc} directives,
the argument of \textit{main} of |\childdocmain| can be empty.

An alternative version of the command line processing described
in \secref{sec:commandline} using the detection mechanism reads:
%
\begin{center}
|... -jobname "|\textit{target}|" "|[\textit{flags}]%
[|\def\jobname{|\textit{dest}|}|]|\input{|\textit{main}|}"|
\end{center}

%%%%%%%%%%%%%%%%%%%%%%%%%%%%%%%%%%%%%%%%%%%%%%%%%%%%%%%%%%%%%%%%%%%%%%%%%%%%%%%%
\subsection{Manual Code}
\label{sec:manual}

In case one cannot be certain whether the definitions file |childdoc.def|
is installed on the target \TeX{} distribution
and one prefers not to ship it,
it is conceivable to paste a few relevant commands into the sources.

To that end, drop all statements |\input{childdoc.def}|
and perform the replacements as outlined below.
Instead of |\childdocmain{|\textit{main}|}| add the following code
to the top of the main file:
%
\begin{center}
\begin{tabular}{l}
|\||ifdefined\childdocname\endinput\||fi\newif\ifchilddoc|\\
|\edef\childdocname{\scantokens\expandafter{\jobname\noexpand}}|\\
|\def\childdocmain{|\textit{main}|}\||ifx\childdocmain\childdocname\||else|\\
|\childdoctrue\includeonly{\childdocname}\let\jobname\childdocmain\||fi|\\
\end{tabular}
\end{center}
%
Instead of |\childdocof{|\textit{main}|}| just include the main file
at the top of each child file:
%
\begin{center}
|\input{|\textit{main}|}|
\end{center}
%
A simple redirection |\childdocforward{|\textit{dest}|}| is achieved by:
%
\begin{center}
|\def\jobname{|\textit{dest}|}\input{\jobname}|
\end{center}
%
The redirection with prefix
|\childdocforwardprefix[|\textit{prefix}|]{|\textit{dest}|}|
is accomplished by:
%
\begin{center}
\begin{tabular}{l}
|{\edef\jobname{\scantokens\expandafter{\jobname\noexpand}}|\\
|\def\redirectjob |\textit{prefix}|#1~~~{\gdef\jobname{|\textit{dest}|#1}}|\\
|\expandafter\redirectjob\jobname~~~}\input{\jobname}|
\end{tabular}
\end{center}

In an alternative approach,
child documents can be compiled by a specific command line
without additional code or specific definitions:
%
\begin{center}
|... -jobname "|\textit{target}|" "|[\textit{flags}]%
|\includeonly{|\textit{dest}|}\input{|\textit{main}|}"|
\end{center}
%

%%%%%%%%%%%%%%%%%%%%%%%%%%%%%%%%%%%%%%%%%%%%%%%%%%%%%%%%%%%%%%%%%%%%%%%%%%%%%%%%
%%%%%%%%%%%%%%%%%%%%%%%%%%%%%%%%%%%%%%%%%%%%%%%%%%%%%%%%%%%%%%%%%%%%%%%%%%%%%%%%
\section{Information}

%%%%%%%%%%%%%%%%%%%%%%%%%%%%%%%%%%%%%%%%%%%%%%%%%%%%%%%%%%%%%%%%%%%%%%%%%%%%%%%%
\subsection{Copyright}

Copyright \copyright{} 2017--2018 Niklas Beisert

This work may be distributed and/or modified under the
conditions of the \LaTeX{} Project Public License, either version 1.3
of this license or (at your option) any later version.
The latest version of this license is in
  \url{http://www.latex-project.org/lppl.txt}
and version 1.3 or later is part of all distributions of \LaTeX{}
version 2005/12/01 or later.

This work has the LPPL maintenance status `maintained'.

The Current Maintainer of this work is Niklas Beisert.

This work consists of the files |README.txt|, |childdoc.ins| and |childdoc.dtx|
as well as the derived files |childdoc.def|, |cdocsamp.tex|
with |cdocsch1.tex|, |cdocsch2.tex|, |cdocspt3.tex|, |cdocspt4.tex|,
|cdocsdrf.tex|, |cdocsfn1.tex|, |cdocsfn2.tex|
as well as |childdoc.pdf|.

%%%%%%%%%%%%%%%%%%%%%%%%%%%%%%%%%%%%%%%%%%%%%%%%%%%%%%%%%%%%%%%%%%%%%%%%%%%%%%%%
\subsection{Files and Installation}

The package consists of the files:
%
\begin{center}
\begin{tabular}{ll}
    |README.txt|   & readme file \\
    |childdoc.ins| & installation file \\
    |childdoc.dtx| & source file \\
    |childdoc.def| & definition file \\
    |cdocsamp.tex| & sample main file \\
    |cdocsch1.tex| & sample include file \\
    |cdocsch2.tex| & sample include file \\
    |cdocspt3.tex| & sample part file \\
    |cdocspt4.tex| & sample part file \\
    |cdocsdrf.tex| & sample redirection file \\
    |cdocsfn1.tex| & sample redirection file \\
    |cdocsfn2.tex| & sample redirection file \\
    |childdoc.pdf| & manual
\end{tabular}
\end{center}
%
The distribution consists of the files
|README.txt|, |childdoc.ins| and |childdoc.dtx|.
%
\begin{itemize}
\item
Run (pdf)\LaTeX{} on |childdoc.dtx|
to compile the manual |childdoc.pdf| (this file).
\item
Run \LaTeX{} on |childdoc.ins| to create the definitions file |childdoc.def|
and the sample |cdocsamp.tex| with include files
|cdocsch1.tex|, |cdocsch2.tex|, |cdocspt3.tex|, |cdocspt4.tex|,
|cdocsdrf.tex|, |cdocsfn1.tex|, |cdocsfn2.tex|.
Then copy the file |childdoc.def| to an appropriate directory of your \LaTeX{}
distribution, e.g.\ \textit{texmf-root}|/tex/latex/childdoc|.
\end{itemize}

%%%%%%%%%%%%%%%%%%%%%%%%%%%%%%%%%%%%%%%%%%%%%%%%%%%%%%%%%%%%%%%%%%%%%%%%%%%%%%%%
\subsection{Related CTAN Packages}

There are several other packages which offer a similar functionality:
%
\begin{itemize}
\item
The packages
\href{http://ctan.org/pkg/docmute}{\textsf{docmute}},
\href{http://ctan.org/pkg/includex}{\textsf{includex}} and
\href{http://ctan.org/pkg/standalone}{\textsf{standalone}}
provide commands to include only the document body of
a child file thus allowing both files to be compiled individually.
\item
The packages \href{http://ctan.org/pkg/subdocs}{\textsf{subdocs}}
and \href{http://ctan.org/pkg/subfiles}{\textsf{subfiles}}
provide structures in which the main and child documents can be
encapsulated and allowing them to be compiled individually.
The inclusion mechanism is different from the conventional |\include|.
\item
The package \href{http://ctan.org/pkg/combine}{\textsf{combine}}
is an elaborate solution to combine several documents into one.
\end{itemize}
%
See also the CTAN topic \href{http://ctan.org/topic/subdocs}{\textsf{subdocs}}
for further related packages.
The present package differs from the above solutions in that
a document structure constructed with the conventional |\include| mechanism
just needs two extra commands at the top of every file
such that all constituent files can be compiled individually.

%%%%%%%%%%%%%%%%%%%%%%%%%%%%%%%%%%%%%%%%%%%%%%%%%%%%%%%%%%%%%%%%%%%%%%%%%%%%%%%%
%\subsection{Feature Suggestions}
%
%The following is a list of features which may be useful for future
%versions of this package:
%%
%\begin{itemize}
%\item
%\ldots
%\end{itemize}

%%%%%%%%%%%%%%%%%%%%%%%%%%%%%%%%%%%%%%%%%%%%%%%%%%%%%%%%%%%%%%%%%%%%%%%%%%%%%%%%
\subsection{Revision History}

%%%%%%%%%%%%%%%%%%%%%%%%%%%%%%%%%%%%%%%%
\paragraph{v2.0:} 2018/12/30

\begin{itemize}
\item
immediate forward processing
\item
added |\childdocby| mechanism
\item
manual restructured
\end{itemize}

%%%%%%%%%%%%%%%%%%%%%%%%%%%%%%%%%%%%%%%%
\paragraph{v1.6:} 2018/01/17

\begin{itemize}
\item
application for development of include files
\item
corrections to manual
\end{itemize}

%%%%%%%%%%%%%%%%%%%%%%%%%%%%%%%%%%%%%%%%
\paragraph{v1.5:} 2017/05/21

\begin{itemize}
\item
more complete structuring introduced
\item
|\childdocof| introduced
\item
|\childdoc| renamed to |\childdocmain|
\item
|\childredirect| renamed to |\childdocforward| and |\childdocforwardprefix|
and functionality expanded
\end{itemize}

%%%%%%%%%%%%%%%%%%%%%%%%%%%%%%%%%%%%%%%%
\paragraph{v1.0:} 2017/04/27

\begin{itemize}
\item
manual and install package
\item
first version published on CTAN
\end{itemize}

%%%%%%%%%%%%%%%%%%%%%%%%%%%%%%%%%%%%%%%%
\paragraph{v0.6:} 2017/04/26

\begin{itemize}
\item
redirection mechanism added
\end{itemize}

%%%%%%%%%%%%%%%%%%%%%%%%%%%%%%%%%%%%%%%%
\paragraph{v0.5:} 2017/04/26

\begin{itemize}
\item
functionality in definition file
\end{itemize}


%%%%%%%%%%%%%%%%%%%%%%%%%%%%%%%%%%%%%%%%%%%%%%%%%%%%%%%%%%%%%%%%%%%%%%%%%%%%%%%%
%%%%%%%%%%%%%%%%%%%%%%%%%%%%%%%%%%%%%%%%%%%%%%%%%%%%%%%%%%%%%%%%%%%%%%%%%%%%%%%%
%%%%%%%%%%%%%%%%%%%%%%%%%%%%%%%%%%%%%%%%%%%%%%%%%%%%%%%%%%%%%%%%%%%%%%%%%%%%%%%%
\appendix

\settowidth\MacroIndent{\rmfamily\scriptsize 000\ }

 \DocInput{childdoc.dtx}

\end{document}
%</driver>
% \fi
%
% %%%%%%%%%%%%%%%%%%%%%%%%%%%%%%%%%%%%%%%%%%%%%%%%%%%%%%%%%%%%%%%%%%%%%%%%%%%%%%
% %%%%%%%%%%%%%%%%%%%%%%%%%%%%%%%%%%%%%%%%%%%%%%%%%%%%%%%%%%%%%%%%%%%%%%%%%%%%%%
% \section{Sample}
%\iffalse
%<*samplemain>
%\fi
%
% The following presents a sample document
% with two chapters, two parts, a title page,
% a compile flag as well as three forwarding files to set the flag.
% It consists of eight |.tex| files:
% \begin{center}
% \begin{tabular}{ll}
% |cdocsamp.tex|&main file\\
% |cdocsch1.tex|&include file for chapter 1\\
% |cdocsch2.tex|&include file for chapter 2\\
% |cdocspt3.tex|&include file for part 3\\
% |cdocspt4.tex|&include file for part 4\\
% |cdocsdrf.tex|&forwarding file for main file in draft mode\\
% |cdocsfi1.tex|&forwarding file for final version of chapter 1\\
% |cdocsfi2.tex|&forwarding file for final version of chapter 2\\
% \end{tabular}
% \end{center}
% Each of the eight files can be compiled directly by the \LaTeX{} compiler.
%
% %%%%%%%%%%%%%%%%%%%%%%%%%%%%%%%%%%%%%%
% \paragraph{Main File.}
%
% The main file is called |cdocsamp.tex|.
%
% Load the \textsf{childdoc} definitions and
% declare the filename for the main document:
%    \begin{macrocode}
\input{childdoc.def}
\childdocmain{}
%    \end{macrocode}

% Optional override for |\version| flag:
%    \begin{macrocode}
%%\ifchilddoc\else\providecommand{\version}{draft}\fi
%    \end{macrocode}

% Define the default values for the |\version| flag
% (|final| for the main file and |draft| for childs):
%    \begin{macrocode}
\ifchilddoc
\providecommand{\version}{draft}
\else
\providecommand{\version}{final}
\fi
%    \end{macrocode}

% Load the standard document class:
%    \begin{macrocode}
\documentclass[12pt]{article}
%    \end{macrocode}

% Start the document body:
%    \begin{macrocode}
\begin{document}
%    \end{macrocode}

% Declare a title page.
% Print title, part of document being processed and version flag:
%    \begin{macrocode}
\addtocounter{page}{-1}
\begin{center}
{\LARGE\bfseries{}childdoc example\par}
\vspace{1cm}
\ifchilddoc
\ifchilddocmanual part\else chapter\fi:
`\childdocname' of `\childdocjob'\par
\else
main document: `\childdocjob'\par
\fi
version: \version\par
\end{center}
\newpage
%    \end{macrocode}

% Manually include selected file,
% otherwise process as usual:
%    \begin{macrocode}
\ifchilddocmanual
\section*{part `\childdocname'}
\input{\childdocname}
\else
%    \end{macrocode}

% Include the two chapters:
%    \begin{macrocode}
\include{cdocsch1}
\include{cdocsch2}
%    \end{macrocode}

% Include the two parts unless only chapters should be displayed:
%    \begin{macrocode}
\ifchilddoc\else
\section{part three}
\input{cdocspt3}
\section{part four}
\input{cdocspt4}
\fi
%    \end{macrocode}

% Process as usual until here:
%    \begin{macrocode}
\fi
%    \end{macrocode}

% End of document body:
%    \begin{macrocode}
\end{document}
%    \end{macrocode}
%\iffalse
%</samplemain>
%\fi
%
% %%%%%%%%%%%%%%%%%%%%%%%%%%%%%%%%%%%%%%
% \paragraph{Chapter Include Files.}
%
% The include files are called |cdocsch1.tex| and |cdocsch2.tex|.
%
%\iffalse
%<*samplechap1|samplechap2>
%\fi

% Optional override for |\version| flag:
%    \begin{macrocode}
%%\providecommand{\version}{final}
%    \end{macrocode}

% Include the main document:
%    \begin{macrocode}
\input{childdoc.def}
\childdocof{cdocsamp}
%    \end{macrocode}

%\iffalse
%</samplechap1|samplechap2>
%\fi
%
%\iffalse
%<*samplechap1>
%\fi
% Some text for chapter 1:
%    \begin{macrocode}
\section{one}
some text in chapter one
%    \end{macrocode}

%\iffalse
%</samplechap1>
%\fi
% Some text for chapter 2:
%\iffalse
%<*samplechap2>
%\fi
%    \begin{macrocode}
\section{two}
more text in chapter two
%    \end{macrocode}

%\iffalse
%</samplechap2>
%\fi
%
% %%%%%%%%%%%%%%%%%%%%%%%%%%%%%%%%%%%%%%
% \paragraph{Part Include Files.}
%
% The include files are called |cdocspt3.tex| and |cdocspt4.tex|.
%
%\iffalse
%<*samplepart3|samplepart4>
%\fi

% Optional override for |\version| flag:
%    \begin{macrocode}
%%\providecommand{\version}{final}
%    \end{macrocode}

% Include the main document:
%    \begin{macrocode}
\input{childdoc.def}
\childdocby{cdocsamp}
%    \end{macrocode}

%\iffalse
%</samplepart3|samplepart4>
%\fi
%
%\iffalse
%<*samplepart3>
%\fi
% Some text for part 3:
%    \begin{macrocode}
some text in part three
%    \end{macrocode}

%\iffalse
%</samplepart3>
%\fi
% Some text for part 4:
%\iffalse
%<*samplepart4>
%\fi
%    \begin{macrocode}
more text in part four
%    \end{macrocode}

%\iffalse
%</samplepart4>
%\fi
%
% %%%%%%%%%%%%%%%%%%%%%%%%%%%%%%%%%%%%%%
% \paragraph{Forwarding for a Complete Draft.}
%
% The following forwarding file |cdocsdrf.tex|
% compiles the main document in draft mode:
%\iffalse
%<*sampledraft>
%\fi
%    \begin{macrocode}
\def\version{draft}
\input{childdoc.def}
\childdocforward{cdocsamp}
%    \end{macrocode}

%\iffalse
%</sampledraft>
%\fi
%
% %%%%%%%%%%%%%%%%%%%%%%%%%%%%%%%%%%%%%%
% \paragraph{Forwarding for Final Version of the Chapters.}
%
% The following forwarding files |cdocsfn1.tex| and |cdocsfn2.tex|
% (with identical content)
% compile the final versions of the child documents
% |cdocsch1.tex| and |cdocsch2.tex|, respectively:
%\iffalse
%<*samplefinal>
%\fi
%    \begin{macrocode}
\def\version{final}
\input{childdoc.def}
\childdocforwardprefix[cdocsamp]{cdocsfn}{cdocsch}
%    \end{macrocode}

%\iffalse
%</samplefinal>
%\fi
%
% %%%%%%%%%%%%%%%%%%%%%%%%%%%%%%%%%%%%%%
% \paragraph{Command Line Processing.}
%
% The following three command lines generate the output files
% |cdocscld|, |cdocscl1| and |cdocscl2|
% which should be identical to
% |cdocsdrf|, |cdocsch1| and |cdocsfn2|, respectively:
% \begin{center}
% \begin{tabular}{l}
% |latex -jobname cdocscld \|\\
% |  "\def\version{draft}\input{childdoc.def}\childdocforward{cdocsamp}"|\\
% |latex -jobname cdocscl1 \|\\
% |  "\input{childdoc.def}\childdocforward[cdocsamp]{cdocsch1}"|\\
% |latex -jobname cdocscl2 \|\\
% |  "\def\version{final}\input{childdoc.def}\childdocforward{cdocsch2}"|
% \end{tabular}
% \end{center}
% Note that the trailing backslash on each first line
% merely continues the input to the second line
% (for convenient cut ant paste).
% Furthermore, the command |latex| can be replaced by any
% of its alternative versions such as |pdflatex|.
%
% %%%%%%%%%%%%%%%%%%%%%%%%%%%%%%%%%%%%%%%%%%%%%%%%%%%%%%%%%%%%%%%%%%%%%%%%%%%%%%
% %%%%%%%%%%%%%%%%%%%%%%%%%%%%%%%%%%%%%%%%%%%%%%%%%%%%%%%%%%%%%%%%%%%%%%%%%%%%%%
% \section{Implementation}
%\iffalse
%<*package>
%\fi
%
% This section describes the definitions file |childdoc.def|.

% The definitions cannot be loaded using |\usepackage| or |\RequirePackage|
% which has a mechanism to prevent loading a style file more than once.
% When loading the definitions by means of |\input|
% multiple instances have to be prevented manually:
%\iffalse
%This code needs to be before the `\ProvidesFile' directive
%which is defined at the beginning of this file.
%Therefore it is also placed there and commented out here.
%</package>
%<*discard>
%\fi
%    \begin{macrocode}
\ifdefined\childdocmain\endinput\fi
%    \end{macrocode}
%\iffalse
%</discard>
%<*package>
%\fi
%
% \macro{\ifchilddoc}
% \macro{\ifchilddocmanual}
% The conditional |\ifchilddoc| tells whether a
% child (true) or main (false) document is being compiled.
% The conditional |\ifchilddocmanual| tells whether
% the |\includeonly| mechanism is used (false) or
% the selection of child files must be performed manually (true).
% The definitions initialise to false:
%    \begin{macrocode}
\newif\ifchilddoc
\newif\ifchilddocmanual
%    \end{macrocode}

% \macro{\childdocname}
% \macro{\childdocjob}
% The macro |\childdocname| stores the name of the main document
% to be compiled. The macro |\childdocjob| stores the name of
% the document on which the \LaTeX{} compiler was originally invoked.
% The content of |\jobname| cannot be compared
% to filenames specified in the source due to different catcodes.
% The following code rescans |\jobname|, stores the result
% in |\childdocname| and saves a copy in |\childdocjob|:
%    \begin{macrocode}
\edef\childdocname{\scantokens\expandafter{\jobname\noexpand}}
\let\childdocjob\childdocname
%    \end{macrocode}

% \macro{\childdocdisable}
% The macro |\childdocdisable| prevents the main file
% from being processed more than once.
% At this stage, the main document command |\childdocmain|
% is assumed to be called once again where it should do nothing.
% Any subsequent call to it should prevent
% a secondary processing of the main document
% It overwrites the forwarding commands
% |\childdocof| and |\childdocforward|
% with empty macros to prevent further inclusions of the main document:
%    \begin{macrocode}
\newcommand{\childdocdisable}
{
  \renewcommand{\childdocmain}[1]{\renewcommand{\childdocmain}[1]{\endinput}}
  \renewcommand{\childdocof}[1]{}
  \renewcommand{\childdocby}[2][]{}
  \renewcommand{\childdocforward}[2][]{}
  \renewcommand{\childdocdisable}{}
}
%    \end{macrocode}

% \macro{\childdocmain}
% The macro |\childdocmain| is to be called at the top of the main file
% with nothing or the main filename (without extension) as argument.
% First, it breaks loops.
% If the argument is not empty and does not match |\childdocname|
% (which is set by the first inclusion of |childdoc.def|),
% |\ifchilddoc| is set to true, |\includeonly| is applied to the child file
% and |\jobname| is set to the main file
% (for proper handling of |.aux| files):
%    \begin{macrocode}
\newcommand{\childdocmain}[1]
{
  \childdocdisable\childdocmain{}
  \if?#1?\else
    \begingroup
      \def\childdoctmp{#1}
      \ifx\childdoctmp\childdocname
        \def\childdoctmp{}
      \else
        \def\childdoctmp
        {
          \childdoctrue
          \includeonly{\childdocname}
          \def\childdocjob{#1}
          \def\jobname{#1}
        }
      \fi
      \expandafter
    \endgroup
    \childdoctmp
  \fi
}
%    \end{macrocode}

% \macro{\childdocof}
% The command |\childdocof| redirects
% compilation to the main file |#1|.
%    \begin{macrocode}
\newcommand{\childdocof}[1]
{
  \childdocdisable
  \childdoctrue
  \includeonly{\childdocname}
  \def\jobname{#1}
  \def\childdocjob{#1}
  \input{#1}
}
%    \end{macrocode}

% \macro{\childdocby}
% The command |\childdocby| ....
%    \begin{macrocode}
\newcommand{\childdocby}[2][]
{
  \childdocdisable
  \childdoctrue
  \childdocmanualtrue
  \if?#1?\else
    \def\jobname{#2}
  \fi
  \def\childdocjob{#2}
  \input{#2}
  \endinput
}
%    \end{macrocode}

% \macro{\childdocforward}
% The command |\childdocforward| redirects
% compilation to the main file or
% (if the optional argument is given) a child file.
% Parameters are set as if the main file
% or a child file starting with |\childdocof| was compiled.
% Then compilation is handed over to the main file:
%    \begin{macrocode}
\newcommand{\childdocforward}[2][]
{
  \begingroup
    \if?#1?
      \def\childdoctmp
      {
        \def\childdocname{#2}
        \def\childdocjob{#2}
        \def\jobname{#2}
        \input{#2}
        \endinput
      }
    \else
      \def\childdoctmp
      {
        \childdocdisable
        \def\childdocname{#2}
        \childdoctrue
        \includeonly{#2}
        \def\childdocjob{#1}
        \def\jobname{#1}
        \input{#1}
        \endinput
      }
    \fi
    \expandafter
  \endgroup
  \childdoctmp
}
%    \end{macrocode}

% \macro{\childdocforwardprefix}
% The command |\childdocforwardprefix| redirects
% compilation to the main or a child file by means of a pattern.
% The prefix |#1| in the current filename is replaced by |#2|
% and the suffix of the current filename is kept
% (it is assumed that the filename does not contain the substring `|~~~|'
% which is used as a delimiter).
% Compilation is handed over to the new file by |\childdocforward|:
%    \begin{macrocode}
\newcommand{\childdocforwardprefix}[3][]
{
  \begingroup
    \def\childdocextract #2##1~~~{\def\childdoctmp{\childdocforward[#1]{#3##1}}}
    \expandafter\childdocextract\childdocname~~~
    \expandafter
  \endgroup
  \childdoctmp
}
%    \end{macrocode}

% \macro{\childdoc}
% The deprecated macro |\childdoc| is a legacy version of |\childdocmain|:
%    \begin{macrocode}
\newcommand{\childdoc}{\childdocmain}
%    \end{macrocode}

% \macro{\childdocredirect}
% The deprecated macro |\childdocredirect| is a legacy version
% of |\childdocforward| and |\childdocforwardprefix|:
%    \begin{macrocode}
\newcommand{\childdocredirect}[2][]
{
  \begingroup
    \if?#1?
      \def\childdoctmp{\childdocforward{#2}}
    \else
      \def\childdoctmp{\childdocforwardprefix{#1}{#2}}
    \fi
    \expandafter
  \endgroup
  \childdoctmp
}
%    \end{macrocode}

%\iffalse
%</package>
%\fi
%
\endinput
\childdocforward[cdocsamp]{cdocsch1}"|\\
% |latex -jobname cdocscl2 \|\\
% |  "\def\version{final}% \iffalse
%
% childdoc.dtx Copyright (C) 2017-2018 Niklas Beisert
%
% This work may be distributed and/or modified under the
% conditions of the LaTeX Project Public License, either version 1.3
% of this license or (at your option) any later version.
% The latest version of this license is in
%   http://www.latex-project.org/lppl.txt
% and version 1.3 or later is part of all distributions of LaTeX
% version 2005/12/01 or later.
%
% This work has the LPPL maintenance status `maintained'.
%
% The Current Maintainer of this work is Niklas Beisert.
%
% This work consists of the files childdoc.dtx and childdoc.ins
% and the derived files childdoc.def and cdocsamp.tex with
% cdocsch1.tex, cdocsch2.tex, cdocsdrf.tex, cdocsfn1.tex, cdocsfn2.tex.
%
%<package>\ifdefined\childdocmain\endinput\fi
%<package>\ProvidesFile{childdoc.def}[2018/12/30 v2.0 child document driver]
%<samplemain>\ProvidesFile{cdocsamp.tex}[2018/12/30 v2.0 sample for childdoc]
%<*driver>
%\ProvidesFile{childdoc.drv}[2018/12/30 v2.0 childdoc reference manual file]
\PassOptionsToClass{10pt,a4paper}{article}
\documentclass{ltxdoc}

\usepackage[margin=35mm]{geometry}
\usepackage{hyperref}
\usepackage{hyperxmp}
\usepackage[usenames]{color}

\hypersetup{colorlinks=true}
\hypersetup{pdfstartview=FitH}
\hypersetup{pdfpagemode=UseNone}
\hypersetup{pdfsource={}}
\hypersetup{pdflang={en-UK}}
\hypersetup{pdfcopyright={Copyright 2017-2018 Niklas Beisert.
  This work may be distributed and/or modified under the
  conditions of the LaTeX Project Public License, either version 1.3
  of this license or (at your option) any later version.}}
\hypersetup{pdflicenseurl={http://www.latex-project.org/lppl.txt}}
\hypersetup{pdfcontactaddress={ETH Zurich, ITP, HIT K,
  Wolfgang-Pauli-Strasse 27}}
\hypersetup{pdfcontactpostcode={8093}}
\hypersetup{pdfcontactcity={Zurich}}
\hypersetup{pdfcontactcountry={Switzerland}}
\hypersetup{pdfcontactemail={nbeisert@itp.phys.ethz.ch}}
\hypersetup{pdfcontacturl={http://people.phys.ethz.ch/\xmptilde nbeisert/}}

\newcommand{\secref}[1]{\hyperref[#1]{section \ref*{#1}}}

\parskip1ex
\parindent0pt
\let\olditemize\itemize
\def\itemize{\olditemize\parskip0pt}

\begin{document}

\title{The \textsf{childdoc} Package}
\hypersetup{pdftitle={The childdoc Package}}
\author{Niklas Beisert\\[2ex]
  Institut f\"ur Theoretische Physik\\
  Eidgen\"ossische Technische Hochschule Z\"urich\\
  Wolfgang-Pauli-Strasse 27, 8093 Z\"urich, Switzerland\\[1ex]
  \href{mailto:nbeisert@itp.phys.ethz.ch}
  {\texttt{nbeisert@itp.phys.ethz.ch}}}
\hypersetup{pdfauthor={Niklas Beisert}}
\hypersetup{pdfsubject={Manual for the LaTeX2e Package childdoc}}
\date{30 December 2018, \textsf{v2.0}}
\maketitle

\begin{abstract}\noindent
\textsf{childdoc} is a \LaTeXe{} package
that enables the direct compilation
of document sections included by |\include|
to individual files.
\end{abstract}

\begingroup
\parskip0ex
\tableofcontents
\endgroup

%%%%%%%%%%%%%%%%%%%%%%%%%%%%%%%%%%%%%%%%%%%%%%%%%%%%%%%%%%%%%%%%%%%%%%%%%%%%%%%%
%%%%%%%%%%%%%%%%%%%%%%%%%%%%%%%%%%%%%%%%%%%%%%%%%%%%%%%%%%%%%%%%%%%%%%%%%%%%%%%%
\section{Introduction}

\LaTeX{} provides a mechanism to structure a large document (such as a book)
into a main file and several child files (containing the chapters)
using the |\include| command.
This mechanism is beneficial for documents
which span hundreds of pages in order to
make the source file(s) more manageable.
Moreover, compilation can be restricted to
selected child files by means of the |\includeonly| command.
The latter feature can be used to reduce the compilation time while editing
(this was significantly more useful in the earlier days of \LaTeX{})
or to generate a smaller document which is easier to navigate.
Another application of |\includeonly| is to generate
documents consisting of selected parts of the complete document.

However, there are a few drawbacks of the plain |\include| mechanism:
\begin{itemize}
\item
The child files cannot be compiled on their own,
they can only be compiled via the main file.
A naive editing environment
(such as a text editor with an option
to have the current file processed by \LaTeX)
may require one to switch to the main file before compiling;
attempting to compile the child file produces errors.
\item
The main file must be modified (each time)
to adjust the |\includeonly| command
to the present needs. This easily leaves the main file in a messy state.
\item
The generated document will always carry the filename
of the main document. This is inconvenient if
several child files are to be compiled and
to be kept for distribution.
\end{itemize}

The present package provides a simple interface
to make child files individually compilable by \LaTeX{}.
Compiling a child file then has the same effect as compiling
the main file with an |\includeonly| command
to select the appropriate child.
Moreover the generated document will carry the name of the child
rather than the main file.
This resolves all three above issues.

This feature is meant to make the editing of books,
thesis documents and lecture notes somewhat more convenient.
However, the package can also be used efficiently for
composing a series of documents (such as exercise sheets)
which are typically distributed individually.
It then assists the author in generating the individual documents
(potentially in different versions)
as well as a document containing the collected series.
Another application is in developing style files
or other kinds of included material
where compilation of the style file could redirect
to a sample or test file.

%%%%%%%%%%%%%%%%%%%%%%%%%%%%%%%%%%%%%%%%%%%%%%%%%%%%%%%%%%%%%%%%%%%%%%%%%%%%%%%%
%%%%%%%%%%%%%%%%%%%%%%%%%%%%%%%%%%%%%%%%%%%%%%%%%%%%%%%%%%%%%%%%%%%%%%%%%%%%%%%%
\section{Usage}

First of all, the package \textsf{childdoc} is \emph{not} a standard
\LaTeXe{} |.sty| style file! Therefore it needs to be invoked in
a non-standard way.

%%%%%%%%%%%%%%%%%%%%%%%%%%%%%%%%%%%%%%%%%%%%%%%%%%%%%%%%%%%%%%%%%%%%%%%%%%%%%%%%
\subsection{Included Files}
\label{sec:include}

%%%%%%%%%%%%%%%%%%%%%%%%%%%%%%%%%%%%%%%%
\DescribeMacro{\childdocmain}
To use the package, add the commands
\begin{center}
\begin{tabular}{l}
|\input{childdoc.def}|\\
|\childdocmain{}|\\
\end{tabular}
\end{center}
at the very top of the main \LaTeX{} file,
in particular \emph{before} the |\documentclass| statement!
The argument of |\childdocmain| should be left empty
(but it must be present).

%%%%%%%%%%%%%%%%%%%%%%%%%%%%%%%%%%%%%%%%
\DescribeMacro{\childdocof}
Furthermore, add the commands
\begin{center}
\begin{tabular}{l}
|\input{childdoc.def}|\\
|\childdocof{|\textit{main}|}|\\
\end{tabular}
\end{center}
at the top of every child file \textit{child}
which is included by |\include{|\textit{child}|}|
from within the main file
(or at least for those files to be compiled individually).
The argument \textit{main} must be the filename of the main file.

There are a couple of
considerations in setting up the main and child documents:

%%%%%%%%%%%%%%%%%%%%%%%%%%%%%%%%%%%%%%%%
\paragraph{Restrictions.}

Please note the following restrictions:
\begin{itemize}
\item
|\childdocmain| must be called with one argument \textit{main}
to ensure compatibility with earlier version of the package.
It must either be empty (|\childdocmain{}|)
or precisely match the filename of the main file in which it is specified.
See \secref{sec:detection} for further information.
\item
The filename \textit{main} must be specified without the |.tex| extension.
\item
The filename \textit{main} is case sensitive
(even in case-insensitive file systems)
due to internal string comparison.
\item
The argument \textit{main} should be fully expanded, it cannot be a macro.
\item
Subdirectories and special characters should be avoided in filenames.
\item
The command |\childdocmain{|\textit{main}|}| must be followed by a whitespace.
It should not be followed immediately by another command
or by a comment mark `|%|'.
This is because the \TeX{} parser reads the token immediately following
the argument of |\childdocmain| and puts it
at the beginning of every child section;
however, a white\-space is ignored.
\end{itemize}

%%%%%%%%%%%%%%%%%%%%%%%%%%%%%%%%%%%%%%%%
\paragraph{Content of Main File.}

It is advisable to place all content in the child files included by |\include|.
Any output contained in the main file will appear in all child documents
unless suppressed manually;
it cannot be suppressed automatically by the |\includeonly| directive
and thus should normally be avoided.
A method to include some content in the main file
by means of conditional processing is described in \secref{sec:conditional}.

%%%%%%%%%%%%%%%%%%%%%%%%%%%%%%%%%%%%%%%%
\paragraph{Page Numbering.}

When only a part of the document is compiled,
the appropriate numbering of pages
(as well as other status parameters)
is determined from the |.aux| files.
The latter contain information from previous passes.
However this information needs to propagate through
all intermediate child documents.
Therefore the page numbering in child documents may well
be inconsistent until the complete document is compiled at least once.

A useful (if unconventional) way to always ensure a consistent
page numbering is to restart the numbering in each child document
and denote the pages by `\textit{child}|.|\textit{page}'
where \textit{child} represents the chapter/section number of the child file.
This can be achieved by the command
|\numberwithin{page}{|\textit{child}|}|
of the \textsf{amsmath} package
where \textit{child} can be |chapter| or |section|
depending on the chosen structuring.
Alternatively, one can modify the macro |\thepage| appropriately
and reset the counter |page| at the start of each child file.

%%%%%%%%%%%%%%%%%%%%%%%%%%%%%%%%%%%%%%%%%%%%%%%%%%%%%%%%%%%%%%%%%%%%%%%%%%%%%%%%
\subsection{Conditional Processing}
\label{sec:conditional}

The package provides a mechanism to compile different versions
of a document. To customise the versions further some conditional processing
can come in handy to distinguish which version is being compiled.
The package provides two macros to describe the compilation context:

%%%%%%%%%%%%%%%%%%%%%%%%%%%%%%%%%%%%%%%%
\DescribeMacro{\ifchilddoc}
The conditional |\ifchilddoc| distinguishes between the compilation of
child documents and the main document:
%
\begin{center}
|\ifchilddoc |\textit{child-code}| |[|\||else |\textit{main-code}]| \||fi|
\end{center}

%%%%%%%%%%%%%%%%%%%%%%%%%%%%%%%%%%%%%%%%
\DescribeMacro{\childdocname}
\DescribeMacro{\childdocjob}
The macro |\childdocname| contains the filename (without extension)
of the main or child file being processed.
Note that |\childdocjob| will always contain the name of the main file.

%%%%%%%%%%%%%%%%%%%%%%%%%%%%%%%%%%%%%%%%
\paragraph{Title Page.}

Conditional processing can be used to include a title or banner page
in the main document when proper precautions are taken.
Importantly, the code in the main file should ensure that the page counter
(as well as other status parameters which are stored in the |.aux| files)
takes the same value after the conditional processing.
Otherwise the page numbers may take divergent values
depending on which part is compiled.

For example, a title page could be declared by:
%
\begin{center}
\begin{tabular}{l}
|\ifchilddoc\||else|\\
|\addtocounter{page}{-1}|\\
\textit{code for title page}\\
|\newpage|\\
|\||fi|
\end{tabular}
\end{center}
%
A banner page for the child documents can be generated by:
%
\begin{center}
\begin{tabular}{l}
|\ifchilddoc|\\
|\addtocounter{page}{-1}|\\
\textit{code for banner page}\\
|\newpage|\\
|\||fi|
\end{tabular}
\end{center}
%
Here one could write a message such as:
\begin{center}
|This is the part \childdocname{} of \childdocjob{}.|
\end{center}

%%%%%%%%%%%%%%%%%%%%%%%%%%%%%%%%%%%%%%%%%%%%%%%%%%%%%%%%%%%%%%%%%%%%%%%%%%%%%%%%
\subsection{Flags}
\label{sec:flags}

The package makes it easy to generate different versions
of the main or child documents.
To this end compilation flags can be defined
and assigned different default values.
They will be particularly useful in conjunction
with the forwarding mechanism described in \secref{sec:forward}.

For example, it may be useful to have a flag |\version|
which can be set to |draft| or |final|.
The document source will contain some conditional code
depending on the value of |\version|.
Suppose further, the flag should default to |final| for the main file
and to |draft| for child files
which is a natural assignment for editing the document.
This is achieved by placing the following code
in the preamble of the main document
(below the |\childdocmain| directive):
%
\begin{center}
\begin{tabular}{l}
|\ifchilddoc|\\
|\providecommand{\version}{draft}|\\
|\||else|\\
|\providecommand{\version}{final}|\\
|\||fi|
\end{tabular}
\end{center}
%
The definition by |\providecommand| makes sure
that previous definitions are not overwritten.
Further statements |\providecommand{\version}{...}|
can thus be added before the above code to override it.

For the main file, one might add a line
(between |\childdocmain| and the above block)
%
\begin{center}
|%\ifchilddoc\||else\providecommand{\version}{draft}\||fi|
\end{center}
%
which can be uncommented to produce a draft version.
Likewise one can add a line to the very top of a child file
(above the |\childdocof{|\textit{main}|}| directive)
%
\begin{center}
|%\providecommand{\version}{final}|
\end{center}
%
which can be uncommented to produce the final version of this child document.

%%%%%%%%%%%%%%%%%%%%%%%%%%%%%%%%%%%%%%%%%%%%%%%%%%%%%%%%%%%%%%%%%%%%%%%%%%%%%%%%
\subsection{Forwarding}
\label{sec:forward}

Different versions of the main or child documents
using compilation flags as described in \secref{sec:flags}
can be (permanently) stored in different files
for convenient compilation, viewing and distribution.
To this end, the package defines a command
to pass on compilation to a different file:

%%%%%%%%%%%%%%%%%%%%%%%%%%%%%%%%%%%%%%%%
\DescribeMacro{\childdocforward}
The command |\childdocforward| redirects processing to
another source file:
%
\begin{center}
\begin{tabular}{l}
|\input{childdoc.def}|\\
|\childdocforward[|\textit{main}|]{|\textit{dest}|}|\\
\end{tabular}
\end{center}
%
The argument \textit{dest} is the destination file
(without extension).
It should be the main file or one of the child files.
Note that further \textsf{childdoc} directives
such as |\childdocof| and |\childdocforward|
in the indicated file will be processed in this form.
The optional argument \textit{main}
passes on directly to the main file \textit{main}
while pretending to compile the child \textit{dest}.
This form behaves as if \textit{dest}
issues |\childdocof{|\textit{main}|}| right away,
and no further \textsf{childdoc} directives will be processed.

%%%%%%%%%%%%%%%%%%%%%%%%%%%%%%%%%%%%%%%%
\DescribeMacro{\...prefix}
In the alternative form |\childdocforwardprefix|,
%
\begin{center}
\begin{tabular}{l}
|\input{childdoc.def}|\\
|\childdocforwardprefix[|\textit{main}|]{|\textit{prefix}|}{|\textit{dest}|}|
\end{tabular}
\end{center}
%
the destination file is determined by a pattern
depending on the current file:
To make this work, the current file must be called
`{\textit{prefix}\hspace{0.2em}\textit{suffix}}'
with \textit{prefix} matching precisely the argument.
Processing is then passed on to the file
`{\textit{dest}\hspace{0.2em}\textit{suffix}}'.
Surely, the same effect is achieved by
directly specifying the
argument `{\textit{dest}\hspace{0.2em}\textit{suffix}}'
in the first form.
However, that requires to set up a different file
for each child. With the alternative form of the command
all these files can have exactly the same content
which simplifies setting them up and maintaining them.

For example, the following file |draft.tex|
with a compilation flag |\version| as described in \secref{sec:flags}
compiles the main document as a draft:
%
\begin{center}
\begin{tabular}{l}
|\def\version{draft}|\\
|\input{childdoc.def}|\\
|\childdocforward{|\textit{main}|}|
\end{tabular}
\end{center}
%
Likewise, the following files |final|\textit{nn}|.tex|
compile the final version of the child document
|child|\textit{nn}|.tex|:
%
\begin{center}
\begin{tabular}{l}
|\def\version{final}|\\
|\input{childdoc.def}|\\
|\childdocforwardprefix{final}{child}|
\end{tabular}
\end{center}
%

Note that when several versions of a main file and/or of each child file
are to be generated, it may be convenient to set up a |Makefile| or
shell script to automatise the process.

%%%%%%%%%%%%%%%%%%%%%%%%%%%%%%%%%%%%%%%%%%%%%%%%%%%%%%%%%%%%%%%%%%%%%%%%%%%%%%%%
\subsection{Command Line Processing}
\label{sec:commandline}

The effect of redirection files can also be achieved by invoking
the \LaTeX{} compiler with a more elaborate command line.
Most conveniently this should be done as part
of a shell script or a |Makefile|.

When using \textsf{childdoc} in the main file, the following
command lines effectively perform a redirection
(note that depending on the shell being used,
backslashes may have to be doubled: `|\|' $\to$ `|\\|'):
%
\begin{center}
|... -jobname "|\textit{target}|" |\\|"|[\textit{flags}]%
|\input{childdoc.def}\childdocforward[|\textit{main}|]{|\textit{dest}|}"|
\end{center}
%
Here \textit{target} is the name of the output file,
\textit{main} is the name of the main file
and \textit{dest} is the name of the main or child file to be processed
(all filenames without extensions).
The optional argument \textit{main} can be omitted
if \textit{main} matches \textit{dest}.
Optionally, compilation \textit{flags} can be defined via |\def| commands.
This command line makes the \TeX{} engine believe
it is compiling the file \textit{target}
whose content is specified as the latter parameter.
The provided code then forwards the processing to
\textit{main} or \textit{dest} as described in \secref{sec:forward}.

%%%%%%%%%%%%%%%%%%%%%%%%%%%%%%%%%%%%%%%%%%%%%%%%%%%%%%%%%%%%%%%%%%%%%%%%%%%%%%%%
\subsection{Include by Input}
\label{sec:input}

Including child documents by |\include| has some restrictions by design.
Most notably, the content of a child document always occupies
its own set of pages; pages cannot be shared between child documents.
Usually, this behaviour makes perfect sense
because each child document contain an essential part of the document.
However, in some situations it may be desirable to compose
a document from a collection of parts
without having mandatory page breaks between then.
For this case, the package
provides a mechanism to include parts
by |\input| which can also be processed individually.
However, by construction this mechanism
requires manual handling of the content to be output.

%%%%%%%%%%%%%%%%%%%%%%%%%%%%%%%%%%%%%%%%
\DescribeMacro{\ifchilddocmanual}
The main file should be prepared as usual, see \secref{sec:include}.
However, the document body must make a distinction
between processing of an individual part and of the main document, e.g.:
%
\begin{center}
\begin{tabular}{l}
|\ifchilddocmanual|\\
|\input{\childdocname}|\\
|\||else|\\
\textit{document body with }|\input{|\textit{part}|}|\\
|\||fi|
\end{tabular}
\end{center}
%
The conditional |\ifchilddocmanual| is true whenever
a part to be included by |\input| is being compiled,
and the name of the part is stored in |\childdocname|.

%%%%%%%%%%%%%%%%%%%%%%%%%%%%%%%%%%%%%%%%
\DescribeMacro{\childdocby}
Each part to be included by |\input| should start with:
%
\begin{center}
\begin{tabular}{l}
|\input{childdoc.def}|\\
|\childdocby{|\textit{main}|}|\\
\end{tabular}
\end{center}
%
The directive |\childdocby| is similar to |\childdocof|
described in \secref{sec:include},
but the subsequent selection of content must be done manually.
To that end, both |\ifchilddoc| and |\ifchilddocmanual|
will be true upon processing of a part,
and the name of the part is stored in |\childdocname|.
Note that |\jobname| will be set to the filename of the current part
so that each part receives an individual |.aux| file
that does not interfere with the |.aux| file(s) of the main document.
This behaviour can be altered by the alternative form
|\childdocby[*]{|\textit{main}|}| (with a non-empty optional argument)
which uses the |.aux| file of the main document
by setting |\jobname| to \textit{main}.

%%%%%%%%%%%%%%%%%%%%%%%%%%%%%%%%%%%%%%%%%%%%%%%%%%%%%%%%%%%%%%%%%%%%%%%%%%%%%%%%
\subsection{Driver Development}
\label{sec:driver}

The \textsf{childdoc} mechanism can also be use for the development
of definition files such as \LaTeX{} styles or classes.
This case differs from the above setup with multiple parts
included by |\include| in that no |\includeonly| should be invoked.
This can be achieved by starting the include file
(before |\ProvidesPackage|) with:
%
\begin{center}
\begin{tabular}{l}
|\input{childdoc.def}|\\
|\childdocforward{|\textit{main}|}|\\
\end{tabular}
\end{center}
%
or alternatively with:
%
\begin{center}
\begin{tabular}{l}
|\input{childdoc.def}|\\
|\childdocby{|\textit{main}|}|\\
\end{tabular}
\end{center}
%
Both forms have slightly different effects as described above.
The main file is prepared as usual, see \secref{sec:include}.

%%%%%%%%%%%%%%%%%%%%%%%%%%%%%%%%%%%%%%%%%%%%%%%%%%%%%%%%%%%%%%%%%%%%%%%%%%%%%%%%
\subsection{Legacy Detection}
\label{sec:detection}

The directive |\childdocmain| in the main file can detect
whether the complete document or merely a child is to be compiled
even without using the directive |\childdocof|.
This method is deprecated because it is less robust
and there is no compelling reason to use it;
it is merely provided for backward compatibility
and it may be removed in future versions.

If the detection mechanism is to be used,
it is mandatory to correctly specify
the filename of the main file as the argument of |\childdocmain|:
%
\begin{center}
\begin{tabular}{l}
|\input{childdoc.def}|\\
|\childdocmain{|\textit{main}|}|\\
\end{tabular}
\end{center}
%
If |\jobname| does not match the argument \textit{main} of |\childdocmain|,
it is assumed that |\jobname| points to the child file to be compiled.
When using |\childdocmain| with the main file specified as argument,
it suffices to start a child file
with just |\input{|\textit{main}|}|
without loading of the package and using |\childdocof|.
If instead all processing is done
with the appropriate \textsf{childdoc} directives,
the argument of \textit{main} of |\childdocmain| can be empty.

An alternative version of the command line processing described
in \secref{sec:commandline} using the detection mechanism reads:
%
\begin{center}
|... -jobname "|\textit{target}|" "|[\textit{flags}]%
[|\def\jobname{|\textit{dest}|}|]|\input{|\textit{main}|}"|
\end{center}

%%%%%%%%%%%%%%%%%%%%%%%%%%%%%%%%%%%%%%%%%%%%%%%%%%%%%%%%%%%%%%%%%%%%%%%%%%%%%%%%
\subsection{Manual Code}
\label{sec:manual}

In case one cannot be certain whether the definitions file |childdoc.def|
is installed on the target \TeX{} distribution
and one prefers not to ship it,
it is conceivable to paste a few relevant commands into the sources.

To that end, drop all statements |\input{childdoc.def}|
and perform the replacements as outlined below.
Instead of |\childdocmain{|\textit{main}|}| add the following code
to the top of the main file:
%
\begin{center}
\begin{tabular}{l}
|\||ifdefined\childdocname\endinput\||fi\newif\ifchilddoc|\\
|\edef\childdocname{\scantokens\expandafter{\jobname\noexpand}}|\\
|\def\childdocmain{|\textit{main}|}\||ifx\childdocmain\childdocname\||else|\\
|\childdoctrue\includeonly{\childdocname}\let\jobname\childdocmain\||fi|\\
\end{tabular}
\end{center}
%
Instead of |\childdocof{|\textit{main}|}| just include the main file
at the top of each child file:
%
\begin{center}
|\input{|\textit{main}|}|
\end{center}
%
A simple redirection |\childdocforward{|\textit{dest}|}| is achieved by:
%
\begin{center}
|\def\jobname{|\textit{dest}|}\input{\jobname}|
\end{center}
%
The redirection with prefix
|\childdocforwardprefix[|\textit{prefix}|]{|\textit{dest}|}|
is accomplished by:
%
\begin{center}
\begin{tabular}{l}
|{\edef\jobname{\scantokens\expandafter{\jobname\noexpand}}|\\
|\def\redirectjob |\textit{prefix}|#1~~~{\gdef\jobname{|\textit{dest}|#1}}|\\
|\expandafter\redirectjob\jobname~~~}\input{\jobname}|
\end{tabular}
\end{center}

In an alternative approach,
child documents can be compiled by a specific command line
without additional code or specific definitions:
%
\begin{center}
|... -jobname "|\textit{target}|" "|[\textit{flags}]%
|\includeonly{|\textit{dest}|}\input{|\textit{main}|}"|
\end{center}
%

%%%%%%%%%%%%%%%%%%%%%%%%%%%%%%%%%%%%%%%%%%%%%%%%%%%%%%%%%%%%%%%%%%%%%%%%%%%%%%%%
%%%%%%%%%%%%%%%%%%%%%%%%%%%%%%%%%%%%%%%%%%%%%%%%%%%%%%%%%%%%%%%%%%%%%%%%%%%%%%%%
\section{Information}

%%%%%%%%%%%%%%%%%%%%%%%%%%%%%%%%%%%%%%%%%%%%%%%%%%%%%%%%%%%%%%%%%%%%%%%%%%%%%%%%
\subsection{Copyright}

Copyright \copyright{} 2017--2018 Niklas Beisert

This work may be distributed and/or modified under the
conditions of the \LaTeX{} Project Public License, either version 1.3
of this license or (at your option) any later version.
The latest version of this license is in
  \url{http://www.latex-project.org/lppl.txt}
and version 1.3 or later is part of all distributions of \LaTeX{}
version 2005/12/01 or later.

This work has the LPPL maintenance status `maintained'.

The Current Maintainer of this work is Niklas Beisert.

This work consists of the files |README.txt|, |childdoc.ins| and |childdoc.dtx|
as well as the derived files |childdoc.def|, |cdocsamp.tex|
with |cdocsch1.tex|, |cdocsch2.tex|, |cdocspt3.tex|, |cdocspt4.tex|,
|cdocsdrf.tex|, |cdocsfn1.tex|, |cdocsfn2.tex|
as well as |childdoc.pdf|.

%%%%%%%%%%%%%%%%%%%%%%%%%%%%%%%%%%%%%%%%%%%%%%%%%%%%%%%%%%%%%%%%%%%%%%%%%%%%%%%%
\subsection{Files and Installation}

The package consists of the files:
%
\begin{center}
\begin{tabular}{ll}
    |README.txt|   & readme file \\
    |childdoc.ins| & installation file \\
    |childdoc.dtx| & source file \\
    |childdoc.def| & definition file \\
    |cdocsamp.tex| & sample main file \\
    |cdocsch1.tex| & sample include file \\
    |cdocsch2.tex| & sample include file \\
    |cdocspt3.tex| & sample part file \\
    |cdocspt4.tex| & sample part file \\
    |cdocsdrf.tex| & sample redirection file \\
    |cdocsfn1.tex| & sample redirection file \\
    |cdocsfn2.tex| & sample redirection file \\
    |childdoc.pdf| & manual
\end{tabular}
\end{center}
%
The distribution consists of the files
|README.txt|, |childdoc.ins| and |childdoc.dtx|.
%
\begin{itemize}
\item
Run (pdf)\LaTeX{} on |childdoc.dtx|
to compile the manual |childdoc.pdf| (this file).
\item
Run \LaTeX{} on |childdoc.ins| to create the definitions file |childdoc.def|
and the sample |cdocsamp.tex| with include files
|cdocsch1.tex|, |cdocsch2.tex|, |cdocspt3.tex|, |cdocspt4.tex|,
|cdocsdrf.tex|, |cdocsfn1.tex|, |cdocsfn2.tex|.
Then copy the file |childdoc.def| to an appropriate directory of your \LaTeX{}
distribution, e.g.\ \textit{texmf-root}|/tex/latex/childdoc|.
\end{itemize}

%%%%%%%%%%%%%%%%%%%%%%%%%%%%%%%%%%%%%%%%%%%%%%%%%%%%%%%%%%%%%%%%%%%%%%%%%%%%%%%%
\subsection{Related CTAN Packages}

There are several other packages which offer a similar functionality:
%
\begin{itemize}
\item
The packages
\href{http://ctan.org/pkg/docmute}{\textsf{docmute}},
\href{http://ctan.org/pkg/includex}{\textsf{includex}} and
\href{http://ctan.org/pkg/standalone}{\textsf{standalone}}
provide commands to include only the document body of
a child file thus allowing both files to be compiled individually.
\item
The packages \href{http://ctan.org/pkg/subdocs}{\textsf{subdocs}}
and \href{http://ctan.org/pkg/subfiles}{\textsf{subfiles}}
provide structures in which the main and child documents can be
encapsulated and allowing them to be compiled individually.
The inclusion mechanism is different from the conventional |\include|.
\item
The package \href{http://ctan.org/pkg/combine}{\textsf{combine}}
is an elaborate solution to combine several documents into one.
\end{itemize}
%
See also the CTAN topic \href{http://ctan.org/topic/subdocs}{\textsf{subdocs}}
for further related packages.
The present package differs from the above solutions in that
a document structure constructed with the conventional |\include| mechanism
just needs two extra commands at the top of every file
such that all constituent files can be compiled individually.

%%%%%%%%%%%%%%%%%%%%%%%%%%%%%%%%%%%%%%%%%%%%%%%%%%%%%%%%%%%%%%%%%%%%%%%%%%%%%%%%
%\subsection{Feature Suggestions}
%
%The following is a list of features which may be useful for future
%versions of this package:
%%
%\begin{itemize}
%\item
%\ldots
%\end{itemize}

%%%%%%%%%%%%%%%%%%%%%%%%%%%%%%%%%%%%%%%%%%%%%%%%%%%%%%%%%%%%%%%%%%%%%%%%%%%%%%%%
\subsection{Revision History}

%%%%%%%%%%%%%%%%%%%%%%%%%%%%%%%%%%%%%%%%
\paragraph{v2.0:} 2018/12/30

\begin{itemize}
\item
immediate forward processing
\item
added |\childdocby| mechanism
\item
manual restructured
\end{itemize}

%%%%%%%%%%%%%%%%%%%%%%%%%%%%%%%%%%%%%%%%
\paragraph{v1.6:} 2018/01/17

\begin{itemize}
\item
application for development of include files
\item
corrections to manual
\end{itemize}

%%%%%%%%%%%%%%%%%%%%%%%%%%%%%%%%%%%%%%%%
\paragraph{v1.5:} 2017/05/21

\begin{itemize}
\item
more complete structuring introduced
\item
|\childdocof| introduced
\item
|\childdoc| renamed to |\childdocmain|
\item
|\childredirect| renamed to |\childdocforward| and |\childdocforwardprefix|
and functionality expanded
\end{itemize}

%%%%%%%%%%%%%%%%%%%%%%%%%%%%%%%%%%%%%%%%
\paragraph{v1.0:} 2017/04/27

\begin{itemize}
\item
manual and install package
\item
first version published on CTAN
\end{itemize}

%%%%%%%%%%%%%%%%%%%%%%%%%%%%%%%%%%%%%%%%
\paragraph{v0.6:} 2017/04/26

\begin{itemize}
\item
redirection mechanism added
\end{itemize}

%%%%%%%%%%%%%%%%%%%%%%%%%%%%%%%%%%%%%%%%
\paragraph{v0.5:} 2017/04/26

\begin{itemize}
\item
functionality in definition file
\end{itemize}


%%%%%%%%%%%%%%%%%%%%%%%%%%%%%%%%%%%%%%%%%%%%%%%%%%%%%%%%%%%%%%%%%%%%%%%%%%%%%%%%
%%%%%%%%%%%%%%%%%%%%%%%%%%%%%%%%%%%%%%%%%%%%%%%%%%%%%%%%%%%%%%%%%%%%%%%%%%%%%%%%
%%%%%%%%%%%%%%%%%%%%%%%%%%%%%%%%%%%%%%%%%%%%%%%%%%%%%%%%%%%%%%%%%%%%%%%%%%%%%%%%
\appendix

\settowidth\MacroIndent{\rmfamily\scriptsize 000\ }

 \DocInput{childdoc.dtx}

\end{document}
%</driver>
% \fi
%
% %%%%%%%%%%%%%%%%%%%%%%%%%%%%%%%%%%%%%%%%%%%%%%%%%%%%%%%%%%%%%%%%%%%%%%%%%%%%%%
% %%%%%%%%%%%%%%%%%%%%%%%%%%%%%%%%%%%%%%%%%%%%%%%%%%%%%%%%%%%%%%%%%%%%%%%%%%%%%%
% \section{Sample}
%\iffalse
%<*samplemain>
%\fi
%
% The following presents a sample document
% with two chapters, two parts, a title page,
% a compile flag as well as three forwarding files to set the flag.
% It consists of eight |.tex| files:
% \begin{center}
% \begin{tabular}{ll}
% |cdocsamp.tex|&main file\\
% |cdocsch1.tex|&include file for chapter 1\\
% |cdocsch2.tex|&include file for chapter 2\\
% |cdocspt3.tex|&include file for part 3\\
% |cdocspt4.tex|&include file for part 4\\
% |cdocsdrf.tex|&forwarding file for main file in draft mode\\
% |cdocsfi1.tex|&forwarding file for final version of chapter 1\\
% |cdocsfi2.tex|&forwarding file for final version of chapter 2\\
% \end{tabular}
% \end{center}
% Each of the eight files can be compiled directly by the \LaTeX{} compiler.
%
% %%%%%%%%%%%%%%%%%%%%%%%%%%%%%%%%%%%%%%
% \paragraph{Main File.}
%
% The main file is called |cdocsamp.tex|.
%
% Load the \textsf{childdoc} definitions and
% declare the filename for the main document:
%    \begin{macrocode}
\input{childdoc.def}
\childdocmain{}
%    \end{macrocode}

% Optional override for |\version| flag:
%    \begin{macrocode}
%%\ifchilddoc\else\providecommand{\version}{draft}\fi
%    \end{macrocode}

% Define the default values for the |\version| flag
% (|final| for the main file and |draft| for childs):
%    \begin{macrocode}
\ifchilddoc
\providecommand{\version}{draft}
\else
\providecommand{\version}{final}
\fi
%    \end{macrocode}

% Load the standard document class:
%    \begin{macrocode}
\documentclass[12pt]{article}
%    \end{macrocode}

% Start the document body:
%    \begin{macrocode}
\begin{document}
%    \end{macrocode}

% Declare a title page.
% Print title, part of document being processed and version flag:
%    \begin{macrocode}
\addtocounter{page}{-1}
\begin{center}
{\LARGE\bfseries{}childdoc example\par}
\vspace{1cm}
\ifchilddoc
\ifchilddocmanual part\else chapter\fi:
`\childdocname' of `\childdocjob'\par
\else
main document: `\childdocjob'\par
\fi
version: \version\par
\end{center}
\newpage
%    \end{macrocode}

% Manually include selected file,
% otherwise process as usual:
%    \begin{macrocode}
\ifchilddocmanual
\section*{part `\childdocname'}
\input{\childdocname}
\else
%    \end{macrocode}

% Include the two chapters:
%    \begin{macrocode}
\include{cdocsch1}
\include{cdocsch2}
%    \end{macrocode}

% Include the two parts unless only chapters should be displayed:
%    \begin{macrocode}
\ifchilddoc\else
\section{part three}
\input{cdocspt3}
\section{part four}
\input{cdocspt4}
\fi
%    \end{macrocode}

% Process as usual until here:
%    \begin{macrocode}
\fi
%    \end{macrocode}

% End of document body:
%    \begin{macrocode}
\end{document}
%    \end{macrocode}
%\iffalse
%</samplemain>
%\fi
%
% %%%%%%%%%%%%%%%%%%%%%%%%%%%%%%%%%%%%%%
% \paragraph{Chapter Include Files.}
%
% The include files are called |cdocsch1.tex| and |cdocsch2.tex|.
%
%\iffalse
%<*samplechap1|samplechap2>
%\fi

% Optional override for |\version| flag:
%    \begin{macrocode}
%%\providecommand{\version}{final}
%    \end{macrocode}

% Include the main document:
%    \begin{macrocode}
\input{childdoc.def}
\childdocof{cdocsamp}
%    \end{macrocode}

%\iffalse
%</samplechap1|samplechap2>
%\fi
%
%\iffalse
%<*samplechap1>
%\fi
% Some text for chapter 1:
%    \begin{macrocode}
\section{one}
some text in chapter one
%    \end{macrocode}

%\iffalse
%</samplechap1>
%\fi
% Some text for chapter 2:
%\iffalse
%<*samplechap2>
%\fi
%    \begin{macrocode}
\section{two}
more text in chapter two
%    \end{macrocode}

%\iffalse
%</samplechap2>
%\fi
%
% %%%%%%%%%%%%%%%%%%%%%%%%%%%%%%%%%%%%%%
% \paragraph{Part Include Files.}
%
% The include files are called |cdocspt3.tex| and |cdocspt4.tex|.
%
%\iffalse
%<*samplepart3|samplepart4>
%\fi

% Optional override for |\version| flag:
%    \begin{macrocode}
%%\providecommand{\version}{final}
%    \end{macrocode}

% Include the main document:
%    \begin{macrocode}
\input{childdoc.def}
\childdocby{cdocsamp}
%    \end{macrocode}

%\iffalse
%</samplepart3|samplepart4>
%\fi
%
%\iffalse
%<*samplepart3>
%\fi
% Some text for part 3:
%    \begin{macrocode}
some text in part three
%    \end{macrocode}

%\iffalse
%</samplepart3>
%\fi
% Some text for part 4:
%\iffalse
%<*samplepart4>
%\fi
%    \begin{macrocode}
more text in part four
%    \end{macrocode}

%\iffalse
%</samplepart4>
%\fi
%
% %%%%%%%%%%%%%%%%%%%%%%%%%%%%%%%%%%%%%%
% \paragraph{Forwarding for a Complete Draft.}
%
% The following forwarding file |cdocsdrf.tex|
% compiles the main document in draft mode:
%\iffalse
%<*sampledraft>
%\fi
%    \begin{macrocode}
\def\version{draft}
\input{childdoc.def}
\childdocforward{cdocsamp}
%    \end{macrocode}

%\iffalse
%</sampledraft>
%\fi
%
% %%%%%%%%%%%%%%%%%%%%%%%%%%%%%%%%%%%%%%
% \paragraph{Forwarding for Final Version of the Chapters.}
%
% The following forwarding files |cdocsfn1.tex| and |cdocsfn2.tex|
% (with identical content)
% compile the final versions of the child documents
% |cdocsch1.tex| and |cdocsch2.tex|, respectively:
%\iffalse
%<*samplefinal>
%\fi
%    \begin{macrocode}
\def\version{final}
\input{childdoc.def}
\childdocforwardprefix[cdocsamp]{cdocsfn}{cdocsch}
%    \end{macrocode}

%\iffalse
%</samplefinal>
%\fi
%
% %%%%%%%%%%%%%%%%%%%%%%%%%%%%%%%%%%%%%%
% \paragraph{Command Line Processing.}
%
% The following three command lines generate the output files
% |cdocscld|, |cdocscl1| and |cdocscl2|
% which should be identical to
% |cdocsdrf|, |cdocsch1| and |cdocsfn2|, respectively:
% \begin{center}
% \begin{tabular}{l}
% |latex -jobname cdocscld \|\\
% |  "\def\version{draft}\input{childdoc.def}\childdocforward{cdocsamp}"|\\
% |latex -jobname cdocscl1 \|\\
% |  "\input{childdoc.def}\childdocforward[cdocsamp]{cdocsch1}"|\\
% |latex -jobname cdocscl2 \|\\
% |  "\def\version{final}\input{childdoc.def}\childdocforward{cdocsch2}"|
% \end{tabular}
% \end{center}
% Note that the trailing backslash on each first line
% merely continues the input to the second line
% (for convenient cut ant paste).
% Furthermore, the command |latex| can be replaced by any
% of its alternative versions such as |pdflatex|.
%
% %%%%%%%%%%%%%%%%%%%%%%%%%%%%%%%%%%%%%%%%%%%%%%%%%%%%%%%%%%%%%%%%%%%%%%%%%%%%%%
% %%%%%%%%%%%%%%%%%%%%%%%%%%%%%%%%%%%%%%%%%%%%%%%%%%%%%%%%%%%%%%%%%%%%%%%%%%%%%%
% \section{Implementation}
%\iffalse
%<*package>
%\fi
%
% This section describes the definitions file |childdoc.def|.

% The definitions cannot be loaded using |\usepackage| or |\RequirePackage|
% which has a mechanism to prevent loading a style file more than once.
% When loading the definitions by means of |\input|
% multiple instances have to be prevented manually:
%\iffalse
%This code needs to be before the `\ProvidesFile' directive
%which is defined at the beginning of this file.
%Therefore it is also placed there and commented out here.
%</package>
%<*discard>
%\fi
%    \begin{macrocode}
\ifdefined\childdocmain\endinput\fi
%    \end{macrocode}
%\iffalse
%</discard>
%<*package>
%\fi
%
% \macro{\ifchilddoc}
% \macro{\ifchilddocmanual}
% The conditional |\ifchilddoc| tells whether a
% child (true) or main (false) document is being compiled.
% The conditional |\ifchilddocmanual| tells whether
% the |\includeonly| mechanism is used (false) or
% the selection of child files must be performed manually (true).
% The definitions initialise to false:
%    \begin{macrocode}
\newif\ifchilddoc
\newif\ifchilddocmanual
%    \end{macrocode}

% \macro{\childdocname}
% \macro{\childdocjob}
% The macro |\childdocname| stores the name of the main document
% to be compiled. The macro |\childdocjob| stores the name of
% the document on which the \LaTeX{} compiler was originally invoked.
% The content of |\jobname| cannot be compared
% to filenames specified in the source due to different catcodes.
% The following code rescans |\jobname|, stores the result
% in |\childdocname| and saves a copy in |\childdocjob|:
%    \begin{macrocode}
\edef\childdocname{\scantokens\expandafter{\jobname\noexpand}}
\let\childdocjob\childdocname
%    \end{macrocode}

% \macro{\childdocdisable}
% The macro |\childdocdisable| prevents the main file
% from being processed more than once.
% At this stage, the main document command |\childdocmain|
% is assumed to be called once again where it should do nothing.
% Any subsequent call to it should prevent
% a secondary processing of the main document
% It overwrites the forwarding commands
% |\childdocof| and |\childdocforward|
% with empty macros to prevent further inclusions of the main document:
%    \begin{macrocode}
\newcommand{\childdocdisable}
{
  \renewcommand{\childdocmain}[1]{\renewcommand{\childdocmain}[1]{\endinput}}
  \renewcommand{\childdocof}[1]{}
  \renewcommand{\childdocby}[2][]{}
  \renewcommand{\childdocforward}[2][]{}
  \renewcommand{\childdocdisable}{}
}
%    \end{macrocode}

% \macro{\childdocmain}
% The macro |\childdocmain| is to be called at the top of the main file
% with nothing or the main filename (without extension) as argument.
% First, it breaks loops.
% If the argument is not empty and does not match |\childdocname|
% (which is set by the first inclusion of |childdoc.def|),
% |\ifchilddoc| is set to true, |\includeonly| is applied to the child file
% and |\jobname| is set to the main file
% (for proper handling of |.aux| files):
%    \begin{macrocode}
\newcommand{\childdocmain}[1]
{
  \childdocdisable\childdocmain{}
  \if?#1?\else
    \begingroup
      \def\childdoctmp{#1}
      \ifx\childdoctmp\childdocname
        \def\childdoctmp{}
      \else
        \def\childdoctmp
        {
          \childdoctrue
          \includeonly{\childdocname}
          \def\childdocjob{#1}
          \def\jobname{#1}
        }
      \fi
      \expandafter
    \endgroup
    \childdoctmp
  \fi
}
%    \end{macrocode}

% \macro{\childdocof}
% The command |\childdocof| redirects
% compilation to the main file |#1|.
%    \begin{macrocode}
\newcommand{\childdocof}[1]
{
  \childdocdisable
  \childdoctrue
  \includeonly{\childdocname}
  \def\jobname{#1}
  \def\childdocjob{#1}
  \input{#1}
}
%    \end{macrocode}

% \macro{\childdocby}
% The command |\childdocby| ....
%    \begin{macrocode}
\newcommand{\childdocby}[2][]
{
  \childdocdisable
  \childdoctrue
  \childdocmanualtrue
  \if?#1?\else
    \def\jobname{#2}
  \fi
  \def\childdocjob{#2}
  \input{#2}
  \endinput
}
%    \end{macrocode}

% \macro{\childdocforward}
% The command |\childdocforward| redirects
% compilation to the main file or
% (if the optional argument is given) a child file.
% Parameters are set as if the main file
% or a child file starting with |\childdocof| was compiled.
% Then compilation is handed over to the main file:
%    \begin{macrocode}
\newcommand{\childdocforward}[2][]
{
  \begingroup
    \if?#1?
      \def\childdoctmp
      {
        \def\childdocname{#2}
        \def\childdocjob{#2}
        \def\jobname{#2}
        \input{#2}
        \endinput
      }
    \else
      \def\childdoctmp
      {
        \childdocdisable
        \def\childdocname{#2}
        \childdoctrue
        \includeonly{#2}
        \def\childdocjob{#1}
        \def\jobname{#1}
        \input{#1}
        \endinput
      }
    \fi
    \expandafter
  \endgroup
  \childdoctmp
}
%    \end{macrocode}

% \macro{\childdocforwardprefix}
% The command |\childdocforwardprefix| redirects
% compilation to the main or a child file by means of a pattern.
% The prefix |#1| in the current filename is replaced by |#2|
% and the suffix of the current filename is kept
% (it is assumed that the filename does not contain the substring `|~~~|'
% which is used as a delimiter).
% Compilation is handed over to the new file by |\childdocforward|:
%    \begin{macrocode}
\newcommand{\childdocforwardprefix}[3][]
{
  \begingroup
    \def\childdocextract #2##1~~~{\def\childdoctmp{\childdocforward[#1]{#3##1}}}
    \expandafter\childdocextract\childdocname~~~
    \expandafter
  \endgroup
  \childdoctmp
}
%    \end{macrocode}

% \macro{\childdoc}
% The deprecated macro |\childdoc| is a legacy version of |\childdocmain|:
%    \begin{macrocode}
\newcommand{\childdoc}{\childdocmain}
%    \end{macrocode}

% \macro{\childdocredirect}
% The deprecated macro |\childdocredirect| is a legacy version
% of |\childdocforward| and |\childdocforwardprefix|:
%    \begin{macrocode}
\newcommand{\childdocredirect}[2][]
{
  \begingroup
    \if?#1?
      \def\childdoctmp{\childdocforward{#2}}
    \else
      \def\childdoctmp{\childdocforwardprefix{#1}{#2}}
    \fi
    \expandafter
  \endgroup
  \childdoctmp
}
%    \end{macrocode}

%\iffalse
%</package>
%\fi
%
\endinput
\childdocforward{cdocsch2}"|
% \end{tabular}
% \end{center}
% Note that the trailing backslash on each first line
% merely continues the input to the second line
% (for convenient cut ant paste).
% Furthermore, the command |latex| can be replaced by any
% of its alternative versions such as |pdflatex|.
%
% %%%%%%%%%%%%%%%%%%%%%%%%%%%%%%%%%%%%%%%%%%%%%%%%%%%%%%%%%%%%%%%%%%%%%%%%%%%%%%
% %%%%%%%%%%%%%%%%%%%%%%%%%%%%%%%%%%%%%%%%%%%%%%%%%%%%%%%%%%%%%%%%%%%%%%%%%%%%%%
% \section{Implementation}
%\iffalse
%<*package>
%\fi
%
% This section describes the definitions file |childdoc.def|.

% The definitions cannot be loaded using |\usepackage| or |\RequirePackage|
% which has a mechanism to prevent loading a style file more than once.
% When loading the definitions by means of |\input|
% multiple instances have to be prevented manually:
%\iffalse
%This code needs to be before the `\ProvidesFile' directive
%which is defined at the beginning of this file.
%Therefore it is also placed there and commented out here.
%</package>
%<*discard>
%\fi
%    \begin{macrocode}
\ifdefined\childdocmain\endinput\fi
%    \end{macrocode}
%\iffalse
%</discard>
%<*package>
%\fi
%
% \macro{\ifchilddoc}
% \macro{\ifchilddocmanual}
% The conditional |\ifchilddoc| tells whether a
% child (true) or main (false) document is being compiled.
% The conditional |\ifchilddocmanual| tells whether
% the |\includeonly| mechanism is used (false) or
% the selection of child files must be performed manually (true).
% The definitions initialise to false:
%    \begin{macrocode}
\newif\ifchilddoc
\newif\ifchilddocmanual
%    \end{macrocode}

% \macro{\childdocname}
% \macro{\childdocjob}
% The macro |\childdocname| stores the name of the main document
% to be compiled. The macro |\childdocjob| stores the name of
% the document on which the \LaTeX{} compiler was originally invoked.
% The content of |\jobname| cannot be compared
% to filenames specified in the source due to different catcodes.
% The following code rescans |\jobname|, stores the result
% in |\childdocname| and saves a copy in |\childdocjob|:
%    \begin{macrocode}
\edef\childdocname{\scantokens\expandafter{\jobname\noexpand}}
\let\childdocjob\childdocname
%    \end{macrocode}

% \macro{\childdocdisable}
% The macro |\childdocdisable| prevents the main file
% from being processed more than once.
% At this stage, the main document command |\childdocmain|
% is assumed to be called once again where it should do nothing.
% Any subsequent call to it should prevent
% a secondary processing of the main document
% It overwrites the forwarding commands
% |\childdocof| and |\childdocforward|
% with empty macros to prevent further inclusions of the main document:
%    \begin{macrocode}
\newcommand{\childdocdisable}
{
  \renewcommand{\childdocmain}[1]{\renewcommand{\childdocmain}[1]{\endinput}}
  \renewcommand{\childdocof}[1]{}
  \renewcommand{\childdocby}[2][]{}
  \renewcommand{\childdocforward}[2][]{}
  \renewcommand{\childdocdisable}{}
}
%    \end{macrocode}

% \macro{\childdocmain}
% The macro |\childdocmain| is to be called at the top of the main file
% with nothing or the main filename (without extension) as argument.
% First, it breaks loops.
% If the argument is not empty and does not match |\childdocname|
% (which is set by the first inclusion of |childdoc.def|),
% |\ifchilddoc| is set to true, |\includeonly| is applied to the child file
% and |\jobname| is set to the main file
% (for proper handling of |.aux| files):
%    \begin{macrocode}
\newcommand{\childdocmain}[1]
{
  \childdocdisable\childdocmain{}
  \if?#1?\else
    \begingroup
      \def\childdoctmp{#1}
      \ifx\childdoctmp\childdocname
        \def\childdoctmp{}
      \else
        \def\childdoctmp
        {
          \childdoctrue
          \includeonly{\childdocname}
          \def\childdocjob{#1}
          \def\jobname{#1}
        }
      \fi
      \expandafter
    \endgroup
    \childdoctmp
  \fi
}
%    \end{macrocode}

% \macro{\childdocof}
% The command |\childdocof| redirects
% compilation to the main file |#1|.
%    \begin{macrocode}
\newcommand{\childdocof}[1]
{
  \childdocdisable
  \childdoctrue
  \includeonly{\childdocname}
  \def\jobname{#1}
  \def\childdocjob{#1}
  \input{#1}
}
%    \end{macrocode}

% \macro{\childdocby}
% The command |\childdocby| ....
%    \begin{macrocode}
\newcommand{\childdocby}[2][]
{
  \childdocdisable
  \childdoctrue
  \childdocmanualtrue
  \if?#1?\else
    \def\jobname{#2}
  \fi
  \def\childdocjob{#2}
  \input{#2}
  \endinput
}
%    \end{macrocode}

% \macro{\childdocforward}
% The command |\childdocforward| redirects
% compilation to the main file or
% (if the optional argument is given) a child file.
% Parameters are set as if the main file
% or a child file starting with |\childdocof| was compiled.
% Then compilation is handed over to the main file:
%    \begin{macrocode}
\newcommand{\childdocforward}[2][]
{
  \begingroup
    \if?#1?
      \def\childdoctmp
      {
        \def\childdocname{#2}
        \def\childdocjob{#2}
        \def\jobname{#2}
        \input{#2}
        \endinput
      }
    \else
      \def\childdoctmp
      {
        \childdocdisable
        \def\childdocname{#2}
        \childdoctrue
        \includeonly{#2}
        \def\childdocjob{#1}
        \def\jobname{#1}
        \input{#1}
        \endinput
      }
    \fi
    \expandafter
  \endgroup
  \childdoctmp
}
%    \end{macrocode}

% \macro{\childdocforwardprefix}
% The command |\childdocforwardprefix| redirects
% compilation to the main or a child file by means of a pattern.
% The prefix |#1| in the current filename is replaced by |#2|
% and the suffix of the current filename is kept
% (it is assumed that the filename does not contain the substring `|~~~|'
% which is used as a delimiter).
% Compilation is handed over to the new file by |\childdocforward|:
%    \begin{macrocode}
\newcommand{\childdocforwardprefix}[3][]
{
  \begingroup
    \def\childdocextract #2##1~~~{\def\childdoctmp{\childdocforward[#1]{#3##1}}}
    \expandafter\childdocextract\childdocname~~~
    \expandafter
  \endgroup
  \childdoctmp
}
%    \end{macrocode}

% \macro{\childdoc}
% The deprecated macro |\childdoc| is a legacy version of |\childdocmain|:
%    \begin{macrocode}
\newcommand{\childdoc}{\childdocmain}
%    \end{macrocode}

% \macro{\childdocredirect}
% The deprecated macro |\childdocredirect| is a legacy version
% of |\childdocforward| and |\childdocforwardprefix|:
%    \begin{macrocode}
\newcommand{\childdocredirect}[2][]
{
  \begingroup
    \if?#1?
      \def\childdoctmp{\childdocforward{#2}}
    \else
      \def\childdoctmp{\childdocforwardprefix{#1}{#2}}
    \fi
    \expandafter
  \endgroup
  \childdoctmp
}
%    \end{macrocode}

%\iffalse
%</package>
%\fi
%
\endinput
|\\
|\childdocof{|\textit{main}|}|\\
\end{tabular}
\end{center}
at the top of every child file \textit{child}
which is included by |\include{|\textit{child}|}|
from within the main file
(or at least for those files to be compiled individually).
The argument \textit{main} must be the filename of the main file.

There are a couple of
considerations in setting up the main and child documents:

%%%%%%%%%%%%%%%%%%%%%%%%%%%%%%%%%%%%%%%%
\paragraph{Restrictions.}

Please note the following restrictions:
\begin{itemize}
\item
|\childdocmain| must be called with one argument \textit{main}
to ensure compatibility with earlier version of the package.
It must either be empty (|\childdocmain{}|)
or precisely match the filename of the main file in which it is specified.
See \secref{sec:detection} for further information.
\item
The filename \textit{main} must be specified without the |.tex| extension.
\item
The filename \textit{main} is case sensitive
(even in case-insensitive file systems)
due to internal string comparison.
\item
The argument \textit{main} should be fully expanded, it cannot be a macro.
\item
Subdirectories and special characters should be avoided in filenames.
\item
The command |\childdocmain{|\textit{main}|}| must be followed by a whitespace.
It should not be followed immediately by another command
or by a comment mark `|%|'.
This is because the \TeX{} parser reads the token immediately following
the argument of |\childdocmain| and puts it
at the beginning of every child section;
however, a white\-space is ignored.
\end{itemize}

%%%%%%%%%%%%%%%%%%%%%%%%%%%%%%%%%%%%%%%%
\paragraph{Content of Main File.}

It is advisable to place all content in the child files included by |\include|.
Any output contained in the main file will appear in all child documents
unless suppressed manually;
it cannot be suppressed automatically by the |\includeonly| directive
and thus should normally be avoided.
A method to include some content in the main file
by means of conditional processing is described in \secref{sec:conditional}.

%%%%%%%%%%%%%%%%%%%%%%%%%%%%%%%%%%%%%%%%
\paragraph{Page Numbering.}

When only a part of the document is compiled,
the appropriate numbering of pages
(as well as other status parameters)
is determined from the |.aux| files.
The latter contain information from previous passes.
However this information needs to propagate through
all intermediate child documents.
Therefore the page numbering in child documents may well
be inconsistent until the complete document is compiled at least once.

A useful (if unconventional) way to always ensure a consistent
page numbering is to restart the numbering in each child document
and denote the pages by `\textit{child}|.|\textit{page}'
where \textit{child} represents the chapter/section number of the child file.
This can be achieved by the command
|\numberwithin{page}{|\textit{child}|}|
of the \textsf{amsmath} package
where \textit{child} can be |chapter| or |section|
depending on the chosen structuring.
Alternatively, one can modify the macro |\thepage| appropriately
and reset the counter |page| at the start of each child file.

%%%%%%%%%%%%%%%%%%%%%%%%%%%%%%%%%%%%%%%%%%%%%%%%%%%%%%%%%%%%%%%%%%%%%%%%%%%%%%%%
\subsection{Conditional Processing}
\label{sec:conditional}

The package provides a mechanism to compile different versions
of a document. To customise the versions further some conditional processing
can come in handy to distinguish which version is being compiled.
The package provides two macros to describe the compilation context:

%%%%%%%%%%%%%%%%%%%%%%%%%%%%%%%%%%%%%%%%
\DescribeMacro{\ifchilddoc}
The conditional |\ifchilddoc| distinguishes between the compilation of
child documents and the main document:
%
\begin{center}
|\ifchilddoc |\textit{child-code}| |[|\||else |\textit{main-code}]| \||fi|
\end{center}

%%%%%%%%%%%%%%%%%%%%%%%%%%%%%%%%%%%%%%%%
\DescribeMacro{\childdocname}
\DescribeMacro{\childdocjob}
The macro |\childdocname| contains the filename (without extension)
of the main or child file being processed.
Note that |\childdocjob| will always contain the name of the main file.

%%%%%%%%%%%%%%%%%%%%%%%%%%%%%%%%%%%%%%%%
\paragraph{Title Page.}

Conditional processing can be used to include a title or banner page
in the main document when proper precautions are taken.
Importantly, the code in the main file should ensure that the page counter
(as well as other status parameters which are stored in the |.aux| files)
takes the same value after the conditional processing.
Otherwise the page numbers may take divergent values
depending on which part is compiled.

For example, a title page could be declared by:
%
\begin{center}
\begin{tabular}{l}
|\ifchilddoc\||else|\\
|\addtocounter{page}{-1}|\\
\textit{code for title page}\\
|\newpage|\\
|\||fi|
\end{tabular}
\end{center}
%
A banner page for the child documents can be generated by:
%
\begin{center}
\begin{tabular}{l}
|\ifchilddoc|\\
|\addtocounter{page}{-1}|\\
\textit{code for banner page}\\
|\newpage|\\
|\||fi|
\end{tabular}
\end{center}
%
Here one could write a message such as:
\begin{center}
|This is the part \childdocname{} of \childdocjob{}.|
\end{center}

%%%%%%%%%%%%%%%%%%%%%%%%%%%%%%%%%%%%%%%%%%%%%%%%%%%%%%%%%%%%%%%%%%%%%%%%%%%%%%%%
\subsection{Flags}
\label{sec:flags}

The package makes it easy to generate different versions
of the main or child documents.
To this end compilation flags can be defined
and assigned different default values.
They will be particularly useful in conjunction
with the forwarding mechanism described in \secref{sec:forward}.

For example, it may be useful to have a flag |\version|
which can be set to |draft| or |final|.
The document source will contain some conditional code
depending on the value of |\version|.
Suppose further, the flag should default to |final| for the main file
and to |draft| for child files
which is a natural assignment for editing the document.
This is achieved by placing the following code
in the preamble of the main document
(below the |\childdocmain| directive):
%
\begin{center}
\begin{tabular}{l}
|\ifchilddoc|\\
|\providecommand{\version}{draft}|\\
|\||else|\\
|\providecommand{\version}{final}|\\
|\||fi|
\end{tabular}
\end{center}
%
The definition by |\providecommand| makes sure
that previous definitions are not overwritten.
Further statements |\providecommand{\version}{...}|
can thus be added before the above code to override it.

For the main file, one might add a line
(between |\childdocmain| and the above block)
%
\begin{center}
|%\ifchilddoc\||else\providecommand{\version}{draft}\||fi|
\end{center}
%
which can be uncommented to produce a draft version.
Likewise one can add a line to the very top of a child file
(above the |\childdocof{|\textit{main}|}| directive)
%
\begin{center}
|%\providecommand{\version}{final}|
\end{center}
%
which can be uncommented to produce the final version of this child document.

%%%%%%%%%%%%%%%%%%%%%%%%%%%%%%%%%%%%%%%%%%%%%%%%%%%%%%%%%%%%%%%%%%%%%%%%%%%%%%%%
\subsection{Forwarding}
\label{sec:forward}

Different versions of the main or child documents
using compilation flags as described in \secref{sec:flags}
can be (permanently) stored in different files
for convenient compilation, viewing and distribution.
To this end, the package defines a command
to pass on compilation to a different file:

%%%%%%%%%%%%%%%%%%%%%%%%%%%%%%%%%%%%%%%%
\DescribeMacro{\childdocforward}
The command |\childdocforward| redirects processing to
another source file:
%
\begin{center}
\begin{tabular}{l}
|% \iffalse
%
% childdoc.dtx Copyright (C) 2017-2018 Niklas Beisert
%
% This work may be distributed and/or modified under the
% conditions of the LaTeX Project Public License, either version 1.3
% of this license or (at your option) any later version.
% The latest version of this license is in
%   http://www.latex-project.org/lppl.txt
% and version 1.3 or later is part of all distributions of LaTeX
% version 2005/12/01 or later.
%
% This work has the LPPL maintenance status `maintained'.
%
% The Current Maintainer of this work is Niklas Beisert.
%
% This work consists of the files childdoc.dtx and childdoc.ins
% and the derived files childdoc.def and cdocsamp.tex with
% cdocsch1.tex, cdocsch2.tex, cdocsdrf.tex, cdocsfn1.tex, cdocsfn2.tex.
%
%<package>\ifdefined\childdocmain\endinput\fi
%<package>\ProvidesFile{childdoc.def}[2018/12/30 v2.0 child document driver]
%<samplemain>\ProvidesFile{cdocsamp.tex}[2018/12/30 v2.0 sample for childdoc]
%<*driver>
%\ProvidesFile{childdoc.drv}[2018/12/30 v2.0 childdoc reference manual file]
\PassOptionsToClass{10pt,a4paper}{article}
\documentclass{ltxdoc}

\usepackage[margin=35mm]{geometry}
\usepackage{hyperref}
\usepackage{hyperxmp}
\usepackage[usenames]{color}

\hypersetup{colorlinks=true}
\hypersetup{pdfstartview=FitH}
\hypersetup{pdfpagemode=UseNone}
\hypersetup{pdfsource={}}
\hypersetup{pdflang={en-UK}}
\hypersetup{pdfcopyright={Copyright 2017-2018 Niklas Beisert.
  This work may be distributed and/or modified under the
  conditions of the LaTeX Project Public License, either version 1.3
  of this license or (at your option) any later version.}}
\hypersetup{pdflicenseurl={http://www.latex-project.org/lppl.txt}}
\hypersetup{pdfcontactaddress={ETH Zurich, ITP, HIT K,
  Wolfgang-Pauli-Strasse 27}}
\hypersetup{pdfcontactpostcode={8093}}
\hypersetup{pdfcontactcity={Zurich}}
\hypersetup{pdfcontactcountry={Switzerland}}
\hypersetup{pdfcontactemail={nbeisert@itp.phys.ethz.ch}}
\hypersetup{pdfcontacturl={http://people.phys.ethz.ch/\xmptilde nbeisert/}}

\newcommand{\secref}[1]{\hyperref[#1]{section \ref*{#1}}}

\parskip1ex
\parindent0pt
\let\olditemize\itemize
\def\itemize{\olditemize\parskip0pt}

\begin{document}

\title{The \textsf{childdoc} Package}
\hypersetup{pdftitle={The childdoc Package}}
\author{Niklas Beisert\\[2ex]
  Institut f\"ur Theoretische Physik\\
  Eidgen\"ossische Technische Hochschule Z\"urich\\
  Wolfgang-Pauli-Strasse 27, 8093 Z\"urich, Switzerland\\[1ex]
  \href{mailto:nbeisert@itp.phys.ethz.ch}
  {\texttt{nbeisert@itp.phys.ethz.ch}}}
\hypersetup{pdfauthor={Niklas Beisert}}
\hypersetup{pdfsubject={Manual for the LaTeX2e Package childdoc}}
\date{30 December 2018, \textsf{v2.0}}
\maketitle

\begin{abstract}\noindent
\textsf{childdoc} is a \LaTeXe{} package
that enables the direct compilation
of document sections included by |\include|
to individual files.
\end{abstract}

\begingroup
\parskip0ex
\tableofcontents
\endgroup

%%%%%%%%%%%%%%%%%%%%%%%%%%%%%%%%%%%%%%%%%%%%%%%%%%%%%%%%%%%%%%%%%%%%%%%%%%%%%%%%
%%%%%%%%%%%%%%%%%%%%%%%%%%%%%%%%%%%%%%%%%%%%%%%%%%%%%%%%%%%%%%%%%%%%%%%%%%%%%%%%
\section{Introduction}

\LaTeX{} provides a mechanism to structure a large document (such as a book)
into a main file and several child files (containing the chapters)
using the |\include| command.
This mechanism is beneficial for documents
which span hundreds of pages in order to
make the source file(s) more manageable.
Moreover, compilation can be restricted to
selected child files by means of the |\includeonly| command.
The latter feature can be used to reduce the compilation time while editing
(this was significantly more useful in the earlier days of \LaTeX{})
or to generate a smaller document which is easier to navigate.
Another application of |\includeonly| is to generate
documents consisting of selected parts of the complete document.

However, there are a few drawbacks of the plain |\include| mechanism:
\begin{itemize}
\item
The child files cannot be compiled on their own,
they can only be compiled via the main file.
A naive editing environment
(such as a text editor with an option
to have the current file processed by \LaTeX)
may require one to switch to the main file before compiling;
attempting to compile the child file produces errors.
\item
The main file must be modified (each time)
to adjust the |\includeonly| command
to the present needs. This easily leaves the main file in a messy state.
\item
The generated document will always carry the filename
of the main document. This is inconvenient if
several child files are to be compiled and
to be kept for distribution.
\end{itemize}

The present package provides a simple interface
to make child files individually compilable by \LaTeX{}.
Compiling a child file then has the same effect as compiling
the main file with an |\includeonly| command
to select the appropriate child.
Moreover the generated document will carry the name of the child
rather than the main file.
This resolves all three above issues.

This feature is meant to make the editing of books,
thesis documents and lecture notes somewhat more convenient.
However, the package can also be used efficiently for
composing a series of documents (such as exercise sheets)
which are typically distributed individually.
It then assists the author in generating the individual documents
(potentially in different versions)
as well as a document containing the collected series.
Another application is in developing style files
or other kinds of included material
where compilation of the style file could redirect
to a sample or test file.

%%%%%%%%%%%%%%%%%%%%%%%%%%%%%%%%%%%%%%%%%%%%%%%%%%%%%%%%%%%%%%%%%%%%%%%%%%%%%%%%
%%%%%%%%%%%%%%%%%%%%%%%%%%%%%%%%%%%%%%%%%%%%%%%%%%%%%%%%%%%%%%%%%%%%%%%%%%%%%%%%
\section{Usage}

First of all, the package \textsf{childdoc} is \emph{not} a standard
\LaTeXe{} |.sty| style file! Therefore it needs to be invoked in
a non-standard way.

%%%%%%%%%%%%%%%%%%%%%%%%%%%%%%%%%%%%%%%%%%%%%%%%%%%%%%%%%%%%%%%%%%%%%%%%%%%%%%%%
\subsection{Included Files}
\label{sec:include}

%%%%%%%%%%%%%%%%%%%%%%%%%%%%%%%%%%%%%%%%
\DescribeMacro{\childdocmain}
To use the package, add the commands
\begin{center}
\begin{tabular}{l}
|% \iffalse
%
% childdoc.dtx Copyright (C) 2017-2018 Niklas Beisert
%
% This work may be distributed and/or modified under the
% conditions of the LaTeX Project Public License, either version 1.3
% of this license or (at your option) any later version.
% The latest version of this license is in
%   http://www.latex-project.org/lppl.txt
% and version 1.3 or later is part of all distributions of LaTeX
% version 2005/12/01 or later.
%
% This work has the LPPL maintenance status `maintained'.
%
% The Current Maintainer of this work is Niklas Beisert.
%
% This work consists of the files childdoc.dtx and childdoc.ins
% and the derived files childdoc.def and cdocsamp.tex with
% cdocsch1.tex, cdocsch2.tex, cdocsdrf.tex, cdocsfn1.tex, cdocsfn2.tex.
%
%<package>\ifdefined\childdocmain\endinput\fi
%<package>\ProvidesFile{childdoc.def}[2018/12/30 v2.0 child document driver]
%<samplemain>\ProvidesFile{cdocsamp.tex}[2018/12/30 v2.0 sample for childdoc]
%<*driver>
%\ProvidesFile{childdoc.drv}[2018/12/30 v2.0 childdoc reference manual file]
\PassOptionsToClass{10pt,a4paper}{article}
\documentclass{ltxdoc}

\usepackage[margin=35mm]{geometry}
\usepackage{hyperref}
\usepackage{hyperxmp}
\usepackage[usenames]{color}

\hypersetup{colorlinks=true}
\hypersetup{pdfstartview=FitH}
\hypersetup{pdfpagemode=UseNone}
\hypersetup{pdfsource={}}
\hypersetup{pdflang={en-UK}}
\hypersetup{pdfcopyright={Copyright 2017-2018 Niklas Beisert.
  This work may be distributed and/or modified under the
  conditions of the LaTeX Project Public License, either version 1.3
  of this license or (at your option) any later version.}}
\hypersetup{pdflicenseurl={http://www.latex-project.org/lppl.txt}}
\hypersetup{pdfcontactaddress={ETH Zurich, ITP, HIT K,
  Wolfgang-Pauli-Strasse 27}}
\hypersetup{pdfcontactpostcode={8093}}
\hypersetup{pdfcontactcity={Zurich}}
\hypersetup{pdfcontactcountry={Switzerland}}
\hypersetup{pdfcontactemail={nbeisert@itp.phys.ethz.ch}}
\hypersetup{pdfcontacturl={http://people.phys.ethz.ch/\xmptilde nbeisert/}}

\newcommand{\secref}[1]{\hyperref[#1]{section \ref*{#1}}}

\parskip1ex
\parindent0pt
\let\olditemize\itemize
\def\itemize{\olditemize\parskip0pt}

\begin{document}

\title{The \textsf{childdoc} Package}
\hypersetup{pdftitle={The childdoc Package}}
\author{Niklas Beisert\\[2ex]
  Institut f\"ur Theoretische Physik\\
  Eidgen\"ossische Technische Hochschule Z\"urich\\
  Wolfgang-Pauli-Strasse 27, 8093 Z\"urich, Switzerland\\[1ex]
  \href{mailto:nbeisert@itp.phys.ethz.ch}
  {\texttt{nbeisert@itp.phys.ethz.ch}}}
\hypersetup{pdfauthor={Niklas Beisert}}
\hypersetup{pdfsubject={Manual for the LaTeX2e Package childdoc}}
\date{30 December 2018, \textsf{v2.0}}
\maketitle

\begin{abstract}\noindent
\textsf{childdoc} is a \LaTeXe{} package
that enables the direct compilation
of document sections included by |\include|
to individual files.
\end{abstract}

\begingroup
\parskip0ex
\tableofcontents
\endgroup

%%%%%%%%%%%%%%%%%%%%%%%%%%%%%%%%%%%%%%%%%%%%%%%%%%%%%%%%%%%%%%%%%%%%%%%%%%%%%%%%
%%%%%%%%%%%%%%%%%%%%%%%%%%%%%%%%%%%%%%%%%%%%%%%%%%%%%%%%%%%%%%%%%%%%%%%%%%%%%%%%
\section{Introduction}

\LaTeX{} provides a mechanism to structure a large document (such as a book)
into a main file and several child files (containing the chapters)
using the |\include| command.
This mechanism is beneficial for documents
which span hundreds of pages in order to
make the source file(s) more manageable.
Moreover, compilation can be restricted to
selected child files by means of the |\includeonly| command.
The latter feature can be used to reduce the compilation time while editing
(this was significantly more useful in the earlier days of \LaTeX{})
or to generate a smaller document which is easier to navigate.
Another application of |\includeonly| is to generate
documents consisting of selected parts of the complete document.

However, there are a few drawbacks of the plain |\include| mechanism:
\begin{itemize}
\item
The child files cannot be compiled on their own,
they can only be compiled via the main file.
A naive editing environment
(such as a text editor with an option
to have the current file processed by \LaTeX)
may require one to switch to the main file before compiling;
attempting to compile the child file produces errors.
\item
The main file must be modified (each time)
to adjust the |\includeonly| command
to the present needs. This easily leaves the main file in a messy state.
\item
The generated document will always carry the filename
of the main document. This is inconvenient if
several child files are to be compiled and
to be kept for distribution.
\end{itemize}

The present package provides a simple interface
to make child files individually compilable by \LaTeX{}.
Compiling a child file then has the same effect as compiling
the main file with an |\includeonly| command
to select the appropriate child.
Moreover the generated document will carry the name of the child
rather than the main file.
This resolves all three above issues.

This feature is meant to make the editing of books,
thesis documents and lecture notes somewhat more convenient.
However, the package can also be used efficiently for
composing a series of documents (such as exercise sheets)
which are typically distributed individually.
It then assists the author in generating the individual documents
(potentially in different versions)
as well as a document containing the collected series.
Another application is in developing style files
or other kinds of included material
where compilation of the style file could redirect
to a sample or test file.

%%%%%%%%%%%%%%%%%%%%%%%%%%%%%%%%%%%%%%%%%%%%%%%%%%%%%%%%%%%%%%%%%%%%%%%%%%%%%%%%
%%%%%%%%%%%%%%%%%%%%%%%%%%%%%%%%%%%%%%%%%%%%%%%%%%%%%%%%%%%%%%%%%%%%%%%%%%%%%%%%
\section{Usage}

First of all, the package \textsf{childdoc} is \emph{not} a standard
\LaTeXe{} |.sty| style file! Therefore it needs to be invoked in
a non-standard way.

%%%%%%%%%%%%%%%%%%%%%%%%%%%%%%%%%%%%%%%%%%%%%%%%%%%%%%%%%%%%%%%%%%%%%%%%%%%%%%%%
\subsection{Included Files}
\label{sec:include}

%%%%%%%%%%%%%%%%%%%%%%%%%%%%%%%%%%%%%%%%
\DescribeMacro{\childdocmain}
To use the package, add the commands
\begin{center}
\begin{tabular}{l}
|\input{childdoc.def}|\\
|\childdocmain{}|\\
\end{tabular}
\end{center}
at the very top of the main \LaTeX{} file,
in particular \emph{before} the |\documentclass| statement!
The argument of |\childdocmain| should be left empty
(but it must be present).

%%%%%%%%%%%%%%%%%%%%%%%%%%%%%%%%%%%%%%%%
\DescribeMacro{\childdocof}
Furthermore, add the commands
\begin{center}
\begin{tabular}{l}
|\input{childdoc.def}|\\
|\childdocof{|\textit{main}|}|\\
\end{tabular}
\end{center}
at the top of every child file \textit{child}
which is included by |\include{|\textit{child}|}|
from within the main file
(or at least for those files to be compiled individually).
The argument \textit{main} must be the filename of the main file.

There are a couple of
considerations in setting up the main and child documents:

%%%%%%%%%%%%%%%%%%%%%%%%%%%%%%%%%%%%%%%%
\paragraph{Restrictions.}

Please note the following restrictions:
\begin{itemize}
\item
|\childdocmain| must be called with one argument \textit{main}
to ensure compatibility with earlier version of the package.
It must either be empty (|\childdocmain{}|)
or precisely match the filename of the main file in which it is specified.
See \secref{sec:detection} for further information.
\item
The filename \textit{main} must be specified without the |.tex| extension.
\item
The filename \textit{main} is case sensitive
(even in case-insensitive file systems)
due to internal string comparison.
\item
The argument \textit{main} should be fully expanded, it cannot be a macro.
\item
Subdirectories and special characters should be avoided in filenames.
\item
The command |\childdocmain{|\textit{main}|}| must be followed by a whitespace.
It should not be followed immediately by another command
or by a comment mark `|%|'.
This is because the \TeX{} parser reads the token immediately following
the argument of |\childdocmain| and puts it
at the beginning of every child section;
however, a white\-space is ignored.
\end{itemize}

%%%%%%%%%%%%%%%%%%%%%%%%%%%%%%%%%%%%%%%%
\paragraph{Content of Main File.}

It is advisable to place all content in the child files included by |\include|.
Any output contained in the main file will appear in all child documents
unless suppressed manually;
it cannot be suppressed automatically by the |\includeonly| directive
and thus should normally be avoided.
A method to include some content in the main file
by means of conditional processing is described in \secref{sec:conditional}.

%%%%%%%%%%%%%%%%%%%%%%%%%%%%%%%%%%%%%%%%
\paragraph{Page Numbering.}

When only a part of the document is compiled,
the appropriate numbering of pages
(as well as other status parameters)
is determined from the |.aux| files.
The latter contain information from previous passes.
However this information needs to propagate through
all intermediate child documents.
Therefore the page numbering in child documents may well
be inconsistent until the complete document is compiled at least once.

A useful (if unconventional) way to always ensure a consistent
page numbering is to restart the numbering in each child document
and denote the pages by `\textit{child}|.|\textit{page}'
where \textit{child} represents the chapter/section number of the child file.
This can be achieved by the command
|\numberwithin{page}{|\textit{child}|}|
of the \textsf{amsmath} package
where \textit{child} can be |chapter| or |section|
depending on the chosen structuring.
Alternatively, one can modify the macro |\thepage| appropriately
and reset the counter |page| at the start of each child file.

%%%%%%%%%%%%%%%%%%%%%%%%%%%%%%%%%%%%%%%%%%%%%%%%%%%%%%%%%%%%%%%%%%%%%%%%%%%%%%%%
\subsection{Conditional Processing}
\label{sec:conditional}

The package provides a mechanism to compile different versions
of a document. To customise the versions further some conditional processing
can come in handy to distinguish which version is being compiled.
The package provides two macros to describe the compilation context:

%%%%%%%%%%%%%%%%%%%%%%%%%%%%%%%%%%%%%%%%
\DescribeMacro{\ifchilddoc}
The conditional |\ifchilddoc| distinguishes between the compilation of
child documents and the main document:
%
\begin{center}
|\ifchilddoc |\textit{child-code}| |[|\||else |\textit{main-code}]| \||fi|
\end{center}

%%%%%%%%%%%%%%%%%%%%%%%%%%%%%%%%%%%%%%%%
\DescribeMacro{\childdocname}
\DescribeMacro{\childdocjob}
The macro |\childdocname| contains the filename (without extension)
of the main or child file being processed.
Note that |\childdocjob| will always contain the name of the main file.

%%%%%%%%%%%%%%%%%%%%%%%%%%%%%%%%%%%%%%%%
\paragraph{Title Page.}

Conditional processing can be used to include a title or banner page
in the main document when proper precautions are taken.
Importantly, the code in the main file should ensure that the page counter
(as well as other status parameters which are stored in the |.aux| files)
takes the same value after the conditional processing.
Otherwise the page numbers may take divergent values
depending on which part is compiled.

For example, a title page could be declared by:
%
\begin{center}
\begin{tabular}{l}
|\ifchilddoc\||else|\\
|\addtocounter{page}{-1}|\\
\textit{code for title page}\\
|\newpage|\\
|\||fi|
\end{tabular}
\end{center}
%
A banner page for the child documents can be generated by:
%
\begin{center}
\begin{tabular}{l}
|\ifchilddoc|\\
|\addtocounter{page}{-1}|\\
\textit{code for banner page}\\
|\newpage|\\
|\||fi|
\end{tabular}
\end{center}
%
Here one could write a message such as:
\begin{center}
|This is the part \childdocname{} of \childdocjob{}.|
\end{center}

%%%%%%%%%%%%%%%%%%%%%%%%%%%%%%%%%%%%%%%%%%%%%%%%%%%%%%%%%%%%%%%%%%%%%%%%%%%%%%%%
\subsection{Flags}
\label{sec:flags}

The package makes it easy to generate different versions
of the main or child documents.
To this end compilation flags can be defined
and assigned different default values.
They will be particularly useful in conjunction
with the forwarding mechanism described in \secref{sec:forward}.

For example, it may be useful to have a flag |\version|
which can be set to |draft| or |final|.
The document source will contain some conditional code
depending on the value of |\version|.
Suppose further, the flag should default to |final| for the main file
and to |draft| for child files
which is a natural assignment for editing the document.
This is achieved by placing the following code
in the preamble of the main document
(below the |\childdocmain| directive):
%
\begin{center}
\begin{tabular}{l}
|\ifchilddoc|\\
|\providecommand{\version}{draft}|\\
|\||else|\\
|\providecommand{\version}{final}|\\
|\||fi|
\end{tabular}
\end{center}
%
The definition by |\providecommand| makes sure
that previous definitions are not overwritten.
Further statements |\providecommand{\version}{...}|
can thus be added before the above code to override it.

For the main file, one might add a line
(between |\childdocmain| and the above block)
%
\begin{center}
|%\ifchilddoc\||else\providecommand{\version}{draft}\||fi|
\end{center}
%
which can be uncommented to produce a draft version.
Likewise one can add a line to the very top of a child file
(above the |\childdocof{|\textit{main}|}| directive)
%
\begin{center}
|%\providecommand{\version}{final}|
\end{center}
%
which can be uncommented to produce the final version of this child document.

%%%%%%%%%%%%%%%%%%%%%%%%%%%%%%%%%%%%%%%%%%%%%%%%%%%%%%%%%%%%%%%%%%%%%%%%%%%%%%%%
\subsection{Forwarding}
\label{sec:forward}

Different versions of the main or child documents
using compilation flags as described in \secref{sec:flags}
can be (permanently) stored in different files
for convenient compilation, viewing and distribution.
To this end, the package defines a command
to pass on compilation to a different file:

%%%%%%%%%%%%%%%%%%%%%%%%%%%%%%%%%%%%%%%%
\DescribeMacro{\childdocforward}
The command |\childdocforward| redirects processing to
another source file:
%
\begin{center}
\begin{tabular}{l}
|\input{childdoc.def}|\\
|\childdocforward[|\textit{main}|]{|\textit{dest}|}|\\
\end{tabular}
\end{center}
%
The argument \textit{dest} is the destination file
(without extension).
It should be the main file or one of the child files.
Note that further \textsf{childdoc} directives
such as |\childdocof| and |\childdocforward|
in the indicated file will be processed in this form.
The optional argument \textit{main}
passes on directly to the main file \textit{main}
while pretending to compile the child \textit{dest}.
This form behaves as if \textit{dest}
issues |\childdocof{|\textit{main}|}| right away,
and no further \textsf{childdoc} directives will be processed.

%%%%%%%%%%%%%%%%%%%%%%%%%%%%%%%%%%%%%%%%
\DescribeMacro{\...prefix}
In the alternative form |\childdocforwardprefix|,
%
\begin{center}
\begin{tabular}{l}
|\input{childdoc.def}|\\
|\childdocforwardprefix[|\textit{main}|]{|\textit{prefix}|}{|\textit{dest}|}|
\end{tabular}
\end{center}
%
the destination file is determined by a pattern
depending on the current file:
To make this work, the current file must be called
`{\textit{prefix}\hspace{0.2em}\textit{suffix}}'
with \textit{prefix} matching precisely the argument.
Processing is then passed on to the file
`{\textit{dest}\hspace{0.2em}\textit{suffix}}'.
Surely, the same effect is achieved by
directly specifying the
argument `{\textit{dest}\hspace{0.2em}\textit{suffix}}'
in the first form.
However, that requires to set up a different file
for each child. With the alternative form of the command
all these files can have exactly the same content
which simplifies setting them up and maintaining them.

For example, the following file |draft.tex|
with a compilation flag |\version| as described in \secref{sec:flags}
compiles the main document as a draft:
%
\begin{center}
\begin{tabular}{l}
|\def\version{draft}|\\
|\input{childdoc.def}|\\
|\childdocforward{|\textit{main}|}|
\end{tabular}
\end{center}
%
Likewise, the following files |final|\textit{nn}|.tex|
compile the final version of the child document
|child|\textit{nn}|.tex|:
%
\begin{center}
\begin{tabular}{l}
|\def\version{final}|\\
|\input{childdoc.def}|\\
|\childdocforwardprefix{final}{child}|
\end{tabular}
\end{center}
%

Note that when several versions of a main file and/or of each child file
are to be generated, it may be convenient to set up a |Makefile| or
shell script to automatise the process.

%%%%%%%%%%%%%%%%%%%%%%%%%%%%%%%%%%%%%%%%%%%%%%%%%%%%%%%%%%%%%%%%%%%%%%%%%%%%%%%%
\subsection{Command Line Processing}
\label{sec:commandline}

The effect of redirection files can also be achieved by invoking
the \LaTeX{} compiler with a more elaborate command line.
Most conveniently this should be done as part
of a shell script or a |Makefile|.

When using \textsf{childdoc} in the main file, the following
command lines effectively perform a redirection
(note that depending on the shell being used,
backslashes may have to be doubled: `|\|' $\to$ `|\\|'):
%
\begin{center}
|... -jobname "|\textit{target}|" |\\|"|[\textit{flags}]%
|\input{childdoc.def}\childdocforward[|\textit{main}|]{|\textit{dest}|}"|
\end{center}
%
Here \textit{target} is the name of the output file,
\textit{main} is the name of the main file
and \textit{dest} is the name of the main or child file to be processed
(all filenames without extensions).
The optional argument \textit{main} can be omitted
if \textit{main} matches \textit{dest}.
Optionally, compilation \textit{flags} can be defined via |\def| commands.
This command line makes the \TeX{} engine believe
it is compiling the file \textit{target}
whose content is specified as the latter parameter.
The provided code then forwards the processing to
\textit{main} or \textit{dest} as described in \secref{sec:forward}.

%%%%%%%%%%%%%%%%%%%%%%%%%%%%%%%%%%%%%%%%%%%%%%%%%%%%%%%%%%%%%%%%%%%%%%%%%%%%%%%%
\subsection{Include by Input}
\label{sec:input}

Including child documents by |\include| has some restrictions by design.
Most notably, the content of a child document always occupies
its own set of pages; pages cannot be shared between child documents.
Usually, this behaviour makes perfect sense
because each child document contain an essential part of the document.
However, in some situations it may be desirable to compose
a document from a collection of parts
without having mandatory page breaks between then.
For this case, the package
provides a mechanism to include parts
by |\input| which can also be processed individually.
However, by construction this mechanism
requires manual handling of the content to be output.

%%%%%%%%%%%%%%%%%%%%%%%%%%%%%%%%%%%%%%%%
\DescribeMacro{\ifchilddocmanual}
The main file should be prepared as usual, see \secref{sec:include}.
However, the document body must make a distinction
between processing of an individual part and of the main document, e.g.:
%
\begin{center}
\begin{tabular}{l}
|\ifchilddocmanual|\\
|\input{\childdocname}|\\
|\||else|\\
\textit{document body with }|\input{|\textit{part}|}|\\
|\||fi|
\end{tabular}
\end{center}
%
The conditional |\ifchilddocmanual| is true whenever
a part to be included by |\input| is being compiled,
and the name of the part is stored in |\childdocname|.

%%%%%%%%%%%%%%%%%%%%%%%%%%%%%%%%%%%%%%%%
\DescribeMacro{\childdocby}
Each part to be included by |\input| should start with:
%
\begin{center}
\begin{tabular}{l}
|\input{childdoc.def}|\\
|\childdocby{|\textit{main}|}|\\
\end{tabular}
\end{center}
%
The directive |\childdocby| is similar to |\childdocof|
described in \secref{sec:include},
but the subsequent selection of content must be done manually.
To that end, both |\ifchilddoc| and |\ifchilddocmanual|
will be true upon processing of a part,
and the name of the part is stored in |\childdocname|.
Note that |\jobname| will be set to the filename of the current part
so that each part receives an individual |.aux| file
that does not interfere with the |.aux| file(s) of the main document.
This behaviour can be altered by the alternative form
|\childdocby[*]{|\textit{main}|}| (with a non-empty optional argument)
which uses the |.aux| file of the main document
by setting |\jobname| to \textit{main}.

%%%%%%%%%%%%%%%%%%%%%%%%%%%%%%%%%%%%%%%%%%%%%%%%%%%%%%%%%%%%%%%%%%%%%%%%%%%%%%%%
\subsection{Driver Development}
\label{sec:driver}

The \textsf{childdoc} mechanism can also be use for the development
of definition files such as \LaTeX{} styles or classes.
This case differs from the above setup with multiple parts
included by |\include| in that no |\includeonly| should be invoked.
This can be achieved by starting the include file
(before |\ProvidesPackage|) with:
%
\begin{center}
\begin{tabular}{l}
|\input{childdoc.def}|\\
|\childdocforward{|\textit{main}|}|\\
\end{tabular}
\end{center}
%
or alternatively with:
%
\begin{center}
\begin{tabular}{l}
|\input{childdoc.def}|\\
|\childdocby{|\textit{main}|}|\\
\end{tabular}
\end{center}
%
Both forms have slightly different effects as described above.
The main file is prepared as usual, see \secref{sec:include}.

%%%%%%%%%%%%%%%%%%%%%%%%%%%%%%%%%%%%%%%%%%%%%%%%%%%%%%%%%%%%%%%%%%%%%%%%%%%%%%%%
\subsection{Legacy Detection}
\label{sec:detection}

The directive |\childdocmain| in the main file can detect
whether the complete document or merely a child is to be compiled
even without using the directive |\childdocof|.
This method is deprecated because it is less robust
and there is no compelling reason to use it;
it is merely provided for backward compatibility
and it may be removed in future versions.

If the detection mechanism is to be used,
it is mandatory to correctly specify
the filename of the main file as the argument of |\childdocmain|:
%
\begin{center}
\begin{tabular}{l}
|\input{childdoc.def}|\\
|\childdocmain{|\textit{main}|}|\\
\end{tabular}
\end{center}
%
If |\jobname| does not match the argument \textit{main} of |\childdocmain|,
it is assumed that |\jobname| points to the child file to be compiled.
When using |\childdocmain| with the main file specified as argument,
it suffices to start a child file
with just |\input{|\textit{main}|}|
without loading of the package and using |\childdocof|.
If instead all processing is done
with the appropriate \textsf{childdoc} directives,
the argument of \textit{main} of |\childdocmain| can be empty.

An alternative version of the command line processing described
in \secref{sec:commandline} using the detection mechanism reads:
%
\begin{center}
|... -jobname "|\textit{target}|" "|[\textit{flags}]%
[|\def\jobname{|\textit{dest}|}|]|\input{|\textit{main}|}"|
\end{center}

%%%%%%%%%%%%%%%%%%%%%%%%%%%%%%%%%%%%%%%%%%%%%%%%%%%%%%%%%%%%%%%%%%%%%%%%%%%%%%%%
\subsection{Manual Code}
\label{sec:manual}

In case one cannot be certain whether the definitions file |childdoc.def|
is installed on the target \TeX{} distribution
and one prefers not to ship it,
it is conceivable to paste a few relevant commands into the sources.

To that end, drop all statements |\input{childdoc.def}|
and perform the replacements as outlined below.
Instead of |\childdocmain{|\textit{main}|}| add the following code
to the top of the main file:
%
\begin{center}
\begin{tabular}{l}
|\||ifdefined\childdocname\endinput\||fi\newif\ifchilddoc|\\
|\edef\childdocname{\scantokens\expandafter{\jobname\noexpand}}|\\
|\def\childdocmain{|\textit{main}|}\||ifx\childdocmain\childdocname\||else|\\
|\childdoctrue\includeonly{\childdocname}\let\jobname\childdocmain\||fi|\\
\end{tabular}
\end{center}
%
Instead of |\childdocof{|\textit{main}|}| just include the main file
at the top of each child file:
%
\begin{center}
|\input{|\textit{main}|}|
\end{center}
%
A simple redirection |\childdocforward{|\textit{dest}|}| is achieved by:
%
\begin{center}
|\def\jobname{|\textit{dest}|}\input{\jobname}|
\end{center}
%
The redirection with prefix
|\childdocforwardprefix[|\textit{prefix}|]{|\textit{dest}|}|
is accomplished by:
%
\begin{center}
\begin{tabular}{l}
|{\edef\jobname{\scantokens\expandafter{\jobname\noexpand}}|\\
|\def\redirectjob |\textit{prefix}|#1~~~{\gdef\jobname{|\textit{dest}|#1}}|\\
|\expandafter\redirectjob\jobname~~~}\input{\jobname}|
\end{tabular}
\end{center}

In an alternative approach,
child documents can be compiled by a specific command line
without additional code or specific definitions:
%
\begin{center}
|... -jobname "|\textit{target}|" "|[\textit{flags}]%
|\includeonly{|\textit{dest}|}\input{|\textit{main}|}"|
\end{center}
%

%%%%%%%%%%%%%%%%%%%%%%%%%%%%%%%%%%%%%%%%%%%%%%%%%%%%%%%%%%%%%%%%%%%%%%%%%%%%%%%%
%%%%%%%%%%%%%%%%%%%%%%%%%%%%%%%%%%%%%%%%%%%%%%%%%%%%%%%%%%%%%%%%%%%%%%%%%%%%%%%%
\section{Information}

%%%%%%%%%%%%%%%%%%%%%%%%%%%%%%%%%%%%%%%%%%%%%%%%%%%%%%%%%%%%%%%%%%%%%%%%%%%%%%%%
\subsection{Copyright}

Copyright \copyright{} 2017--2018 Niklas Beisert

This work may be distributed and/or modified under the
conditions of the \LaTeX{} Project Public License, either version 1.3
of this license or (at your option) any later version.
The latest version of this license is in
  \url{http://www.latex-project.org/lppl.txt}
and version 1.3 or later is part of all distributions of \LaTeX{}
version 2005/12/01 or later.

This work has the LPPL maintenance status `maintained'.

The Current Maintainer of this work is Niklas Beisert.

This work consists of the files |README.txt|, |childdoc.ins| and |childdoc.dtx|
as well as the derived files |childdoc.def|, |cdocsamp.tex|
with |cdocsch1.tex|, |cdocsch2.tex|, |cdocspt3.tex|, |cdocspt4.tex|,
|cdocsdrf.tex|, |cdocsfn1.tex|, |cdocsfn2.tex|
as well as |childdoc.pdf|.

%%%%%%%%%%%%%%%%%%%%%%%%%%%%%%%%%%%%%%%%%%%%%%%%%%%%%%%%%%%%%%%%%%%%%%%%%%%%%%%%
\subsection{Files and Installation}

The package consists of the files:
%
\begin{center}
\begin{tabular}{ll}
    |README.txt|   & readme file \\
    |childdoc.ins| & installation file \\
    |childdoc.dtx| & source file \\
    |childdoc.def| & definition file \\
    |cdocsamp.tex| & sample main file \\
    |cdocsch1.tex| & sample include file \\
    |cdocsch2.tex| & sample include file \\
    |cdocspt3.tex| & sample part file \\
    |cdocspt4.tex| & sample part file \\
    |cdocsdrf.tex| & sample redirection file \\
    |cdocsfn1.tex| & sample redirection file \\
    |cdocsfn2.tex| & sample redirection file \\
    |childdoc.pdf| & manual
\end{tabular}
\end{center}
%
The distribution consists of the files
|README.txt|, |childdoc.ins| and |childdoc.dtx|.
%
\begin{itemize}
\item
Run (pdf)\LaTeX{} on |childdoc.dtx|
to compile the manual |childdoc.pdf| (this file).
\item
Run \LaTeX{} on |childdoc.ins| to create the definitions file |childdoc.def|
and the sample |cdocsamp.tex| with include files
|cdocsch1.tex|, |cdocsch2.tex|, |cdocspt3.tex|, |cdocspt4.tex|,
|cdocsdrf.tex|, |cdocsfn1.tex|, |cdocsfn2.tex|.
Then copy the file |childdoc.def| to an appropriate directory of your \LaTeX{}
distribution, e.g.\ \textit{texmf-root}|/tex/latex/childdoc|.
\end{itemize}

%%%%%%%%%%%%%%%%%%%%%%%%%%%%%%%%%%%%%%%%%%%%%%%%%%%%%%%%%%%%%%%%%%%%%%%%%%%%%%%%
\subsection{Related CTAN Packages}

There are several other packages which offer a similar functionality:
%
\begin{itemize}
\item
The packages
\href{http://ctan.org/pkg/docmute}{\textsf{docmute}},
\href{http://ctan.org/pkg/includex}{\textsf{includex}} and
\href{http://ctan.org/pkg/standalone}{\textsf{standalone}}
provide commands to include only the document body of
a child file thus allowing both files to be compiled individually.
\item
The packages \href{http://ctan.org/pkg/subdocs}{\textsf{subdocs}}
and \href{http://ctan.org/pkg/subfiles}{\textsf{subfiles}}
provide structures in which the main and child documents can be
encapsulated and allowing them to be compiled individually.
The inclusion mechanism is different from the conventional |\include|.
\item
The package \href{http://ctan.org/pkg/combine}{\textsf{combine}}
is an elaborate solution to combine several documents into one.
\end{itemize}
%
See also the CTAN topic \href{http://ctan.org/topic/subdocs}{\textsf{subdocs}}
for further related packages.
The present package differs from the above solutions in that
a document structure constructed with the conventional |\include| mechanism
just needs two extra commands at the top of every file
such that all constituent files can be compiled individually.

%%%%%%%%%%%%%%%%%%%%%%%%%%%%%%%%%%%%%%%%%%%%%%%%%%%%%%%%%%%%%%%%%%%%%%%%%%%%%%%%
%\subsection{Feature Suggestions}
%
%The following is a list of features which may be useful for future
%versions of this package:
%%
%\begin{itemize}
%\item
%\ldots
%\end{itemize}

%%%%%%%%%%%%%%%%%%%%%%%%%%%%%%%%%%%%%%%%%%%%%%%%%%%%%%%%%%%%%%%%%%%%%%%%%%%%%%%%
\subsection{Revision History}

%%%%%%%%%%%%%%%%%%%%%%%%%%%%%%%%%%%%%%%%
\paragraph{v2.0:} 2018/12/30

\begin{itemize}
\item
immediate forward processing
\item
added |\childdocby| mechanism
\item
manual restructured
\end{itemize}

%%%%%%%%%%%%%%%%%%%%%%%%%%%%%%%%%%%%%%%%
\paragraph{v1.6:} 2018/01/17

\begin{itemize}
\item
application for development of include files
\item
corrections to manual
\end{itemize}

%%%%%%%%%%%%%%%%%%%%%%%%%%%%%%%%%%%%%%%%
\paragraph{v1.5:} 2017/05/21

\begin{itemize}
\item
more complete structuring introduced
\item
|\childdocof| introduced
\item
|\childdoc| renamed to |\childdocmain|
\item
|\childredirect| renamed to |\childdocforward| and |\childdocforwardprefix|
and functionality expanded
\end{itemize}

%%%%%%%%%%%%%%%%%%%%%%%%%%%%%%%%%%%%%%%%
\paragraph{v1.0:} 2017/04/27

\begin{itemize}
\item
manual and install package
\item
first version published on CTAN
\end{itemize}

%%%%%%%%%%%%%%%%%%%%%%%%%%%%%%%%%%%%%%%%
\paragraph{v0.6:} 2017/04/26

\begin{itemize}
\item
redirection mechanism added
\end{itemize}

%%%%%%%%%%%%%%%%%%%%%%%%%%%%%%%%%%%%%%%%
\paragraph{v0.5:} 2017/04/26

\begin{itemize}
\item
functionality in definition file
\end{itemize}


%%%%%%%%%%%%%%%%%%%%%%%%%%%%%%%%%%%%%%%%%%%%%%%%%%%%%%%%%%%%%%%%%%%%%%%%%%%%%%%%
%%%%%%%%%%%%%%%%%%%%%%%%%%%%%%%%%%%%%%%%%%%%%%%%%%%%%%%%%%%%%%%%%%%%%%%%%%%%%%%%
%%%%%%%%%%%%%%%%%%%%%%%%%%%%%%%%%%%%%%%%%%%%%%%%%%%%%%%%%%%%%%%%%%%%%%%%%%%%%%%%
\appendix

\settowidth\MacroIndent{\rmfamily\scriptsize 000\ }

 \DocInput{childdoc.dtx}

\end{document}
%</driver>
% \fi
%
% %%%%%%%%%%%%%%%%%%%%%%%%%%%%%%%%%%%%%%%%%%%%%%%%%%%%%%%%%%%%%%%%%%%%%%%%%%%%%%
% %%%%%%%%%%%%%%%%%%%%%%%%%%%%%%%%%%%%%%%%%%%%%%%%%%%%%%%%%%%%%%%%%%%%%%%%%%%%%%
% \section{Sample}
%\iffalse
%<*samplemain>
%\fi
%
% The following presents a sample document
% with two chapters, two parts, a title page,
% a compile flag as well as three forwarding files to set the flag.
% It consists of eight |.tex| files:
% \begin{center}
% \begin{tabular}{ll}
% |cdocsamp.tex|&main file\\
% |cdocsch1.tex|&include file for chapter 1\\
% |cdocsch2.tex|&include file for chapter 2\\
% |cdocspt3.tex|&include file for part 3\\
% |cdocspt4.tex|&include file for part 4\\
% |cdocsdrf.tex|&forwarding file for main file in draft mode\\
% |cdocsfi1.tex|&forwarding file for final version of chapter 1\\
% |cdocsfi2.tex|&forwarding file for final version of chapter 2\\
% \end{tabular}
% \end{center}
% Each of the eight files can be compiled directly by the \LaTeX{} compiler.
%
% %%%%%%%%%%%%%%%%%%%%%%%%%%%%%%%%%%%%%%
% \paragraph{Main File.}
%
% The main file is called |cdocsamp.tex|.
%
% Load the \textsf{childdoc} definitions and
% declare the filename for the main document:
%    \begin{macrocode}
\input{childdoc.def}
\childdocmain{}
%    \end{macrocode}

% Optional override for |\version| flag:
%    \begin{macrocode}
%%\ifchilddoc\else\providecommand{\version}{draft}\fi
%    \end{macrocode}

% Define the default values for the |\version| flag
% (|final| for the main file and |draft| for childs):
%    \begin{macrocode}
\ifchilddoc
\providecommand{\version}{draft}
\else
\providecommand{\version}{final}
\fi
%    \end{macrocode}

% Load the standard document class:
%    \begin{macrocode}
\documentclass[12pt]{article}
%    \end{macrocode}

% Start the document body:
%    \begin{macrocode}
\begin{document}
%    \end{macrocode}

% Declare a title page.
% Print title, part of document being processed and version flag:
%    \begin{macrocode}
\addtocounter{page}{-1}
\begin{center}
{\LARGE\bfseries{}childdoc example\par}
\vspace{1cm}
\ifchilddoc
\ifchilddocmanual part\else chapter\fi:
`\childdocname' of `\childdocjob'\par
\else
main document: `\childdocjob'\par
\fi
version: \version\par
\end{center}
\newpage
%    \end{macrocode}

% Manually include selected file,
% otherwise process as usual:
%    \begin{macrocode}
\ifchilddocmanual
\section*{part `\childdocname'}
\input{\childdocname}
\else
%    \end{macrocode}

% Include the two chapters:
%    \begin{macrocode}
\include{cdocsch1}
\include{cdocsch2}
%    \end{macrocode}

% Include the two parts unless only chapters should be displayed:
%    \begin{macrocode}
\ifchilddoc\else
\section{part three}
\input{cdocspt3}
\section{part four}
\input{cdocspt4}
\fi
%    \end{macrocode}

% Process as usual until here:
%    \begin{macrocode}
\fi
%    \end{macrocode}

% End of document body:
%    \begin{macrocode}
\end{document}
%    \end{macrocode}
%\iffalse
%</samplemain>
%\fi
%
% %%%%%%%%%%%%%%%%%%%%%%%%%%%%%%%%%%%%%%
% \paragraph{Chapter Include Files.}
%
% The include files are called |cdocsch1.tex| and |cdocsch2.tex|.
%
%\iffalse
%<*samplechap1|samplechap2>
%\fi

% Optional override for |\version| flag:
%    \begin{macrocode}
%%\providecommand{\version}{final}
%    \end{macrocode}

% Include the main document:
%    \begin{macrocode}
\input{childdoc.def}
\childdocof{cdocsamp}
%    \end{macrocode}

%\iffalse
%</samplechap1|samplechap2>
%\fi
%
%\iffalse
%<*samplechap1>
%\fi
% Some text for chapter 1:
%    \begin{macrocode}
\section{one}
some text in chapter one
%    \end{macrocode}

%\iffalse
%</samplechap1>
%\fi
% Some text for chapter 2:
%\iffalse
%<*samplechap2>
%\fi
%    \begin{macrocode}
\section{two}
more text in chapter two
%    \end{macrocode}

%\iffalse
%</samplechap2>
%\fi
%
% %%%%%%%%%%%%%%%%%%%%%%%%%%%%%%%%%%%%%%
% \paragraph{Part Include Files.}
%
% The include files are called |cdocspt3.tex| and |cdocspt4.tex|.
%
%\iffalse
%<*samplepart3|samplepart4>
%\fi

% Optional override for |\version| flag:
%    \begin{macrocode}
%%\providecommand{\version}{final}
%    \end{macrocode}

% Include the main document:
%    \begin{macrocode}
\input{childdoc.def}
\childdocby{cdocsamp}
%    \end{macrocode}

%\iffalse
%</samplepart3|samplepart4>
%\fi
%
%\iffalse
%<*samplepart3>
%\fi
% Some text for part 3:
%    \begin{macrocode}
some text in part three
%    \end{macrocode}

%\iffalse
%</samplepart3>
%\fi
% Some text for part 4:
%\iffalse
%<*samplepart4>
%\fi
%    \begin{macrocode}
more text in part four
%    \end{macrocode}

%\iffalse
%</samplepart4>
%\fi
%
% %%%%%%%%%%%%%%%%%%%%%%%%%%%%%%%%%%%%%%
% \paragraph{Forwarding for a Complete Draft.}
%
% The following forwarding file |cdocsdrf.tex|
% compiles the main document in draft mode:
%\iffalse
%<*sampledraft>
%\fi
%    \begin{macrocode}
\def\version{draft}
\input{childdoc.def}
\childdocforward{cdocsamp}
%    \end{macrocode}

%\iffalse
%</sampledraft>
%\fi
%
% %%%%%%%%%%%%%%%%%%%%%%%%%%%%%%%%%%%%%%
% \paragraph{Forwarding for Final Version of the Chapters.}
%
% The following forwarding files |cdocsfn1.tex| and |cdocsfn2.tex|
% (with identical content)
% compile the final versions of the child documents
% |cdocsch1.tex| and |cdocsch2.tex|, respectively:
%\iffalse
%<*samplefinal>
%\fi
%    \begin{macrocode}
\def\version{final}
\input{childdoc.def}
\childdocforwardprefix[cdocsamp]{cdocsfn}{cdocsch}
%    \end{macrocode}

%\iffalse
%</samplefinal>
%\fi
%
% %%%%%%%%%%%%%%%%%%%%%%%%%%%%%%%%%%%%%%
% \paragraph{Command Line Processing.}
%
% The following three command lines generate the output files
% |cdocscld|, |cdocscl1| and |cdocscl2|
% which should be identical to
% |cdocsdrf|, |cdocsch1| and |cdocsfn2|, respectively:
% \begin{center}
% \begin{tabular}{l}
% |latex -jobname cdocscld \|\\
% |  "\def\version{draft}\input{childdoc.def}\childdocforward{cdocsamp}"|\\
% |latex -jobname cdocscl1 \|\\
% |  "\input{childdoc.def}\childdocforward[cdocsamp]{cdocsch1}"|\\
% |latex -jobname cdocscl2 \|\\
% |  "\def\version{final}\input{childdoc.def}\childdocforward{cdocsch2}"|
% \end{tabular}
% \end{center}
% Note that the trailing backslash on each first line
% merely continues the input to the second line
% (for convenient cut ant paste).
% Furthermore, the command |latex| can be replaced by any
% of its alternative versions such as |pdflatex|.
%
% %%%%%%%%%%%%%%%%%%%%%%%%%%%%%%%%%%%%%%%%%%%%%%%%%%%%%%%%%%%%%%%%%%%%%%%%%%%%%%
% %%%%%%%%%%%%%%%%%%%%%%%%%%%%%%%%%%%%%%%%%%%%%%%%%%%%%%%%%%%%%%%%%%%%%%%%%%%%%%
% \section{Implementation}
%\iffalse
%<*package>
%\fi
%
% This section describes the definitions file |childdoc.def|.

% The definitions cannot be loaded using |\usepackage| or |\RequirePackage|
% which has a mechanism to prevent loading a style file more than once.
% When loading the definitions by means of |\input|
% multiple instances have to be prevented manually:
%\iffalse
%This code needs to be before the `\ProvidesFile' directive
%which is defined at the beginning of this file.
%Therefore it is also placed there and commented out here.
%</package>
%<*discard>
%\fi
%    \begin{macrocode}
\ifdefined\childdocmain\endinput\fi
%    \end{macrocode}
%\iffalse
%</discard>
%<*package>
%\fi
%
% \macro{\ifchilddoc}
% \macro{\ifchilddocmanual}
% The conditional |\ifchilddoc| tells whether a
% child (true) or main (false) document is being compiled.
% The conditional |\ifchilddocmanual| tells whether
% the |\includeonly| mechanism is used (false) or
% the selection of child files must be performed manually (true).
% The definitions initialise to false:
%    \begin{macrocode}
\newif\ifchilddoc
\newif\ifchilddocmanual
%    \end{macrocode}

% \macro{\childdocname}
% \macro{\childdocjob}
% The macro |\childdocname| stores the name of the main document
% to be compiled. The macro |\childdocjob| stores the name of
% the document on which the \LaTeX{} compiler was originally invoked.
% The content of |\jobname| cannot be compared
% to filenames specified in the source due to different catcodes.
% The following code rescans |\jobname|, stores the result
% in |\childdocname| and saves a copy in |\childdocjob|:
%    \begin{macrocode}
\edef\childdocname{\scantokens\expandafter{\jobname\noexpand}}
\let\childdocjob\childdocname
%    \end{macrocode}

% \macro{\childdocdisable}
% The macro |\childdocdisable| prevents the main file
% from being processed more than once.
% At this stage, the main document command |\childdocmain|
% is assumed to be called once again where it should do nothing.
% Any subsequent call to it should prevent
% a secondary processing of the main document
% It overwrites the forwarding commands
% |\childdocof| and |\childdocforward|
% with empty macros to prevent further inclusions of the main document:
%    \begin{macrocode}
\newcommand{\childdocdisable}
{
  \renewcommand{\childdocmain}[1]{\renewcommand{\childdocmain}[1]{\endinput}}
  \renewcommand{\childdocof}[1]{}
  \renewcommand{\childdocby}[2][]{}
  \renewcommand{\childdocforward}[2][]{}
  \renewcommand{\childdocdisable}{}
}
%    \end{macrocode}

% \macro{\childdocmain}
% The macro |\childdocmain| is to be called at the top of the main file
% with nothing or the main filename (without extension) as argument.
% First, it breaks loops.
% If the argument is not empty and does not match |\childdocname|
% (which is set by the first inclusion of |childdoc.def|),
% |\ifchilddoc| is set to true, |\includeonly| is applied to the child file
% and |\jobname| is set to the main file
% (for proper handling of |.aux| files):
%    \begin{macrocode}
\newcommand{\childdocmain}[1]
{
  \childdocdisable\childdocmain{}
  \if?#1?\else
    \begingroup
      \def\childdoctmp{#1}
      \ifx\childdoctmp\childdocname
        \def\childdoctmp{}
      \else
        \def\childdoctmp
        {
          \childdoctrue
          \includeonly{\childdocname}
          \def\childdocjob{#1}
          \def\jobname{#1}
        }
      \fi
      \expandafter
    \endgroup
    \childdoctmp
  \fi
}
%    \end{macrocode}

% \macro{\childdocof}
% The command |\childdocof| redirects
% compilation to the main file |#1|.
%    \begin{macrocode}
\newcommand{\childdocof}[1]
{
  \childdocdisable
  \childdoctrue
  \includeonly{\childdocname}
  \def\jobname{#1}
  \def\childdocjob{#1}
  \input{#1}
}
%    \end{macrocode}

% \macro{\childdocby}
% The command |\childdocby| ....
%    \begin{macrocode}
\newcommand{\childdocby}[2][]
{
  \childdocdisable
  \childdoctrue
  \childdocmanualtrue
  \if?#1?\else
    \def\jobname{#2}
  \fi
  \def\childdocjob{#2}
  \input{#2}
  \endinput
}
%    \end{macrocode}

% \macro{\childdocforward}
% The command |\childdocforward| redirects
% compilation to the main file or
% (if the optional argument is given) a child file.
% Parameters are set as if the main file
% or a child file starting with |\childdocof| was compiled.
% Then compilation is handed over to the main file:
%    \begin{macrocode}
\newcommand{\childdocforward}[2][]
{
  \begingroup
    \if?#1?
      \def\childdoctmp
      {
        \def\childdocname{#2}
        \def\childdocjob{#2}
        \def\jobname{#2}
        \input{#2}
        \endinput
      }
    \else
      \def\childdoctmp
      {
        \childdocdisable
        \def\childdocname{#2}
        \childdoctrue
        \includeonly{#2}
        \def\childdocjob{#1}
        \def\jobname{#1}
        \input{#1}
        \endinput
      }
    \fi
    \expandafter
  \endgroup
  \childdoctmp
}
%    \end{macrocode}

% \macro{\childdocforwardprefix}
% The command |\childdocforwardprefix| redirects
% compilation to the main or a child file by means of a pattern.
% The prefix |#1| in the current filename is replaced by |#2|
% and the suffix of the current filename is kept
% (it is assumed that the filename does not contain the substring `|~~~|'
% which is used as a delimiter).
% Compilation is handed over to the new file by |\childdocforward|:
%    \begin{macrocode}
\newcommand{\childdocforwardprefix}[3][]
{
  \begingroup
    \def\childdocextract #2##1~~~{\def\childdoctmp{\childdocforward[#1]{#3##1}}}
    \expandafter\childdocextract\childdocname~~~
    \expandafter
  \endgroup
  \childdoctmp
}
%    \end{macrocode}

% \macro{\childdoc}
% The deprecated macro |\childdoc| is a legacy version of |\childdocmain|:
%    \begin{macrocode}
\newcommand{\childdoc}{\childdocmain}
%    \end{macrocode}

% \macro{\childdocredirect}
% The deprecated macro |\childdocredirect| is a legacy version
% of |\childdocforward| and |\childdocforwardprefix|:
%    \begin{macrocode}
\newcommand{\childdocredirect}[2][]
{
  \begingroup
    \if?#1?
      \def\childdoctmp{\childdocforward{#2}}
    \else
      \def\childdoctmp{\childdocforwardprefix{#1}{#2}}
    \fi
    \expandafter
  \endgroup
  \childdoctmp
}
%    \end{macrocode}

%\iffalse
%</package>
%\fi
%
\endinput
|\\
|\childdocmain{}|\\
\end{tabular}
\end{center}
at the very top of the main \LaTeX{} file,
in particular \emph{before} the |\documentclass| statement!
The argument of |\childdocmain| should be left empty
(but it must be present).

%%%%%%%%%%%%%%%%%%%%%%%%%%%%%%%%%%%%%%%%
\DescribeMacro{\childdocof}
Furthermore, add the commands
\begin{center}
\begin{tabular}{l}
|% \iffalse
%
% childdoc.dtx Copyright (C) 2017-2018 Niklas Beisert
%
% This work may be distributed and/or modified under the
% conditions of the LaTeX Project Public License, either version 1.3
% of this license or (at your option) any later version.
% The latest version of this license is in
%   http://www.latex-project.org/lppl.txt
% and version 1.3 or later is part of all distributions of LaTeX
% version 2005/12/01 or later.
%
% This work has the LPPL maintenance status `maintained'.
%
% The Current Maintainer of this work is Niklas Beisert.
%
% This work consists of the files childdoc.dtx and childdoc.ins
% and the derived files childdoc.def and cdocsamp.tex with
% cdocsch1.tex, cdocsch2.tex, cdocsdrf.tex, cdocsfn1.tex, cdocsfn2.tex.
%
%<package>\ifdefined\childdocmain\endinput\fi
%<package>\ProvidesFile{childdoc.def}[2018/12/30 v2.0 child document driver]
%<samplemain>\ProvidesFile{cdocsamp.tex}[2018/12/30 v2.0 sample for childdoc]
%<*driver>
%\ProvidesFile{childdoc.drv}[2018/12/30 v2.0 childdoc reference manual file]
\PassOptionsToClass{10pt,a4paper}{article}
\documentclass{ltxdoc}

\usepackage[margin=35mm]{geometry}
\usepackage{hyperref}
\usepackage{hyperxmp}
\usepackage[usenames]{color}

\hypersetup{colorlinks=true}
\hypersetup{pdfstartview=FitH}
\hypersetup{pdfpagemode=UseNone}
\hypersetup{pdfsource={}}
\hypersetup{pdflang={en-UK}}
\hypersetup{pdfcopyright={Copyright 2017-2018 Niklas Beisert.
  This work may be distributed and/or modified under the
  conditions of the LaTeX Project Public License, either version 1.3
  of this license or (at your option) any later version.}}
\hypersetup{pdflicenseurl={http://www.latex-project.org/lppl.txt}}
\hypersetup{pdfcontactaddress={ETH Zurich, ITP, HIT K,
  Wolfgang-Pauli-Strasse 27}}
\hypersetup{pdfcontactpostcode={8093}}
\hypersetup{pdfcontactcity={Zurich}}
\hypersetup{pdfcontactcountry={Switzerland}}
\hypersetup{pdfcontactemail={nbeisert@itp.phys.ethz.ch}}
\hypersetup{pdfcontacturl={http://people.phys.ethz.ch/\xmptilde nbeisert/}}

\newcommand{\secref}[1]{\hyperref[#1]{section \ref*{#1}}}

\parskip1ex
\parindent0pt
\let\olditemize\itemize
\def\itemize{\olditemize\parskip0pt}

\begin{document}

\title{The \textsf{childdoc} Package}
\hypersetup{pdftitle={The childdoc Package}}
\author{Niklas Beisert\\[2ex]
  Institut f\"ur Theoretische Physik\\
  Eidgen\"ossische Technische Hochschule Z\"urich\\
  Wolfgang-Pauli-Strasse 27, 8093 Z\"urich, Switzerland\\[1ex]
  \href{mailto:nbeisert@itp.phys.ethz.ch}
  {\texttt{nbeisert@itp.phys.ethz.ch}}}
\hypersetup{pdfauthor={Niklas Beisert}}
\hypersetup{pdfsubject={Manual for the LaTeX2e Package childdoc}}
\date{30 December 2018, \textsf{v2.0}}
\maketitle

\begin{abstract}\noindent
\textsf{childdoc} is a \LaTeXe{} package
that enables the direct compilation
of document sections included by |\include|
to individual files.
\end{abstract}

\begingroup
\parskip0ex
\tableofcontents
\endgroup

%%%%%%%%%%%%%%%%%%%%%%%%%%%%%%%%%%%%%%%%%%%%%%%%%%%%%%%%%%%%%%%%%%%%%%%%%%%%%%%%
%%%%%%%%%%%%%%%%%%%%%%%%%%%%%%%%%%%%%%%%%%%%%%%%%%%%%%%%%%%%%%%%%%%%%%%%%%%%%%%%
\section{Introduction}

\LaTeX{} provides a mechanism to structure a large document (such as a book)
into a main file and several child files (containing the chapters)
using the |\include| command.
This mechanism is beneficial for documents
which span hundreds of pages in order to
make the source file(s) more manageable.
Moreover, compilation can be restricted to
selected child files by means of the |\includeonly| command.
The latter feature can be used to reduce the compilation time while editing
(this was significantly more useful in the earlier days of \LaTeX{})
or to generate a smaller document which is easier to navigate.
Another application of |\includeonly| is to generate
documents consisting of selected parts of the complete document.

However, there are a few drawbacks of the plain |\include| mechanism:
\begin{itemize}
\item
The child files cannot be compiled on their own,
they can only be compiled via the main file.
A naive editing environment
(such as a text editor with an option
to have the current file processed by \LaTeX)
may require one to switch to the main file before compiling;
attempting to compile the child file produces errors.
\item
The main file must be modified (each time)
to adjust the |\includeonly| command
to the present needs. This easily leaves the main file in a messy state.
\item
The generated document will always carry the filename
of the main document. This is inconvenient if
several child files are to be compiled and
to be kept for distribution.
\end{itemize}

The present package provides a simple interface
to make child files individually compilable by \LaTeX{}.
Compiling a child file then has the same effect as compiling
the main file with an |\includeonly| command
to select the appropriate child.
Moreover the generated document will carry the name of the child
rather than the main file.
This resolves all three above issues.

This feature is meant to make the editing of books,
thesis documents and lecture notes somewhat more convenient.
However, the package can also be used efficiently for
composing a series of documents (such as exercise sheets)
which are typically distributed individually.
It then assists the author in generating the individual documents
(potentially in different versions)
as well as a document containing the collected series.
Another application is in developing style files
or other kinds of included material
where compilation of the style file could redirect
to a sample or test file.

%%%%%%%%%%%%%%%%%%%%%%%%%%%%%%%%%%%%%%%%%%%%%%%%%%%%%%%%%%%%%%%%%%%%%%%%%%%%%%%%
%%%%%%%%%%%%%%%%%%%%%%%%%%%%%%%%%%%%%%%%%%%%%%%%%%%%%%%%%%%%%%%%%%%%%%%%%%%%%%%%
\section{Usage}

First of all, the package \textsf{childdoc} is \emph{not} a standard
\LaTeXe{} |.sty| style file! Therefore it needs to be invoked in
a non-standard way.

%%%%%%%%%%%%%%%%%%%%%%%%%%%%%%%%%%%%%%%%%%%%%%%%%%%%%%%%%%%%%%%%%%%%%%%%%%%%%%%%
\subsection{Included Files}
\label{sec:include}

%%%%%%%%%%%%%%%%%%%%%%%%%%%%%%%%%%%%%%%%
\DescribeMacro{\childdocmain}
To use the package, add the commands
\begin{center}
\begin{tabular}{l}
|\input{childdoc.def}|\\
|\childdocmain{}|\\
\end{tabular}
\end{center}
at the very top of the main \LaTeX{} file,
in particular \emph{before} the |\documentclass| statement!
The argument of |\childdocmain| should be left empty
(but it must be present).

%%%%%%%%%%%%%%%%%%%%%%%%%%%%%%%%%%%%%%%%
\DescribeMacro{\childdocof}
Furthermore, add the commands
\begin{center}
\begin{tabular}{l}
|\input{childdoc.def}|\\
|\childdocof{|\textit{main}|}|\\
\end{tabular}
\end{center}
at the top of every child file \textit{child}
which is included by |\include{|\textit{child}|}|
from within the main file
(or at least for those files to be compiled individually).
The argument \textit{main} must be the filename of the main file.

There are a couple of
considerations in setting up the main and child documents:

%%%%%%%%%%%%%%%%%%%%%%%%%%%%%%%%%%%%%%%%
\paragraph{Restrictions.}

Please note the following restrictions:
\begin{itemize}
\item
|\childdocmain| must be called with one argument \textit{main}
to ensure compatibility with earlier version of the package.
It must either be empty (|\childdocmain{}|)
or precisely match the filename of the main file in which it is specified.
See \secref{sec:detection} for further information.
\item
The filename \textit{main} must be specified without the |.tex| extension.
\item
The filename \textit{main} is case sensitive
(even in case-insensitive file systems)
due to internal string comparison.
\item
The argument \textit{main} should be fully expanded, it cannot be a macro.
\item
Subdirectories and special characters should be avoided in filenames.
\item
The command |\childdocmain{|\textit{main}|}| must be followed by a whitespace.
It should not be followed immediately by another command
or by a comment mark `|%|'.
This is because the \TeX{} parser reads the token immediately following
the argument of |\childdocmain| and puts it
at the beginning of every child section;
however, a white\-space is ignored.
\end{itemize}

%%%%%%%%%%%%%%%%%%%%%%%%%%%%%%%%%%%%%%%%
\paragraph{Content of Main File.}

It is advisable to place all content in the child files included by |\include|.
Any output contained in the main file will appear in all child documents
unless suppressed manually;
it cannot be suppressed automatically by the |\includeonly| directive
and thus should normally be avoided.
A method to include some content in the main file
by means of conditional processing is described in \secref{sec:conditional}.

%%%%%%%%%%%%%%%%%%%%%%%%%%%%%%%%%%%%%%%%
\paragraph{Page Numbering.}

When only a part of the document is compiled,
the appropriate numbering of pages
(as well as other status parameters)
is determined from the |.aux| files.
The latter contain information from previous passes.
However this information needs to propagate through
all intermediate child documents.
Therefore the page numbering in child documents may well
be inconsistent until the complete document is compiled at least once.

A useful (if unconventional) way to always ensure a consistent
page numbering is to restart the numbering in each child document
and denote the pages by `\textit{child}|.|\textit{page}'
where \textit{child} represents the chapter/section number of the child file.
This can be achieved by the command
|\numberwithin{page}{|\textit{child}|}|
of the \textsf{amsmath} package
where \textit{child} can be |chapter| or |section|
depending on the chosen structuring.
Alternatively, one can modify the macro |\thepage| appropriately
and reset the counter |page| at the start of each child file.

%%%%%%%%%%%%%%%%%%%%%%%%%%%%%%%%%%%%%%%%%%%%%%%%%%%%%%%%%%%%%%%%%%%%%%%%%%%%%%%%
\subsection{Conditional Processing}
\label{sec:conditional}

The package provides a mechanism to compile different versions
of a document. To customise the versions further some conditional processing
can come in handy to distinguish which version is being compiled.
The package provides two macros to describe the compilation context:

%%%%%%%%%%%%%%%%%%%%%%%%%%%%%%%%%%%%%%%%
\DescribeMacro{\ifchilddoc}
The conditional |\ifchilddoc| distinguishes between the compilation of
child documents and the main document:
%
\begin{center}
|\ifchilddoc |\textit{child-code}| |[|\||else |\textit{main-code}]| \||fi|
\end{center}

%%%%%%%%%%%%%%%%%%%%%%%%%%%%%%%%%%%%%%%%
\DescribeMacro{\childdocname}
\DescribeMacro{\childdocjob}
The macro |\childdocname| contains the filename (without extension)
of the main or child file being processed.
Note that |\childdocjob| will always contain the name of the main file.

%%%%%%%%%%%%%%%%%%%%%%%%%%%%%%%%%%%%%%%%
\paragraph{Title Page.}

Conditional processing can be used to include a title or banner page
in the main document when proper precautions are taken.
Importantly, the code in the main file should ensure that the page counter
(as well as other status parameters which are stored in the |.aux| files)
takes the same value after the conditional processing.
Otherwise the page numbers may take divergent values
depending on which part is compiled.

For example, a title page could be declared by:
%
\begin{center}
\begin{tabular}{l}
|\ifchilddoc\||else|\\
|\addtocounter{page}{-1}|\\
\textit{code for title page}\\
|\newpage|\\
|\||fi|
\end{tabular}
\end{center}
%
A banner page for the child documents can be generated by:
%
\begin{center}
\begin{tabular}{l}
|\ifchilddoc|\\
|\addtocounter{page}{-1}|\\
\textit{code for banner page}\\
|\newpage|\\
|\||fi|
\end{tabular}
\end{center}
%
Here one could write a message such as:
\begin{center}
|This is the part \childdocname{} of \childdocjob{}.|
\end{center}

%%%%%%%%%%%%%%%%%%%%%%%%%%%%%%%%%%%%%%%%%%%%%%%%%%%%%%%%%%%%%%%%%%%%%%%%%%%%%%%%
\subsection{Flags}
\label{sec:flags}

The package makes it easy to generate different versions
of the main or child documents.
To this end compilation flags can be defined
and assigned different default values.
They will be particularly useful in conjunction
with the forwarding mechanism described in \secref{sec:forward}.

For example, it may be useful to have a flag |\version|
which can be set to |draft| or |final|.
The document source will contain some conditional code
depending on the value of |\version|.
Suppose further, the flag should default to |final| for the main file
and to |draft| for child files
which is a natural assignment for editing the document.
This is achieved by placing the following code
in the preamble of the main document
(below the |\childdocmain| directive):
%
\begin{center}
\begin{tabular}{l}
|\ifchilddoc|\\
|\providecommand{\version}{draft}|\\
|\||else|\\
|\providecommand{\version}{final}|\\
|\||fi|
\end{tabular}
\end{center}
%
The definition by |\providecommand| makes sure
that previous definitions are not overwritten.
Further statements |\providecommand{\version}{...}|
can thus be added before the above code to override it.

For the main file, one might add a line
(between |\childdocmain| and the above block)
%
\begin{center}
|%\ifchilddoc\||else\providecommand{\version}{draft}\||fi|
\end{center}
%
which can be uncommented to produce a draft version.
Likewise one can add a line to the very top of a child file
(above the |\childdocof{|\textit{main}|}| directive)
%
\begin{center}
|%\providecommand{\version}{final}|
\end{center}
%
which can be uncommented to produce the final version of this child document.

%%%%%%%%%%%%%%%%%%%%%%%%%%%%%%%%%%%%%%%%%%%%%%%%%%%%%%%%%%%%%%%%%%%%%%%%%%%%%%%%
\subsection{Forwarding}
\label{sec:forward}

Different versions of the main or child documents
using compilation flags as described in \secref{sec:flags}
can be (permanently) stored in different files
for convenient compilation, viewing and distribution.
To this end, the package defines a command
to pass on compilation to a different file:

%%%%%%%%%%%%%%%%%%%%%%%%%%%%%%%%%%%%%%%%
\DescribeMacro{\childdocforward}
The command |\childdocforward| redirects processing to
another source file:
%
\begin{center}
\begin{tabular}{l}
|\input{childdoc.def}|\\
|\childdocforward[|\textit{main}|]{|\textit{dest}|}|\\
\end{tabular}
\end{center}
%
The argument \textit{dest} is the destination file
(without extension).
It should be the main file or one of the child files.
Note that further \textsf{childdoc} directives
such as |\childdocof| and |\childdocforward|
in the indicated file will be processed in this form.
The optional argument \textit{main}
passes on directly to the main file \textit{main}
while pretending to compile the child \textit{dest}.
This form behaves as if \textit{dest}
issues |\childdocof{|\textit{main}|}| right away,
and no further \textsf{childdoc} directives will be processed.

%%%%%%%%%%%%%%%%%%%%%%%%%%%%%%%%%%%%%%%%
\DescribeMacro{\...prefix}
In the alternative form |\childdocforwardprefix|,
%
\begin{center}
\begin{tabular}{l}
|\input{childdoc.def}|\\
|\childdocforwardprefix[|\textit{main}|]{|\textit{prefix}|}{|\textit{dest}|}|
\end{tabular}
\end{center}
%
the destination file is determined by a pattern
depending on the current file:
To make this work, the current file must be called
`{\textit{prefix}\hspace{0.2em}\textit{suffix}}'
with \textit{prefix} matching precisely the argument.
Processing is then passed on to the file
`{\textit{dest}\hspace{0.2em}\textit{suffix}}'.
Surely, the same effect is achieved by
directly specifying the
argument `{\textit{dest}\hspace{0.2em}\textit{suffix}}'
in the first form.
However, that requires to set up a different file
for each child. With the alternative form of the command
all these files can have exactly the same content
which simplifies setting them up and maintaining them.

For example, the following file |draft.tex|
with a compilation flag |\version| as described in \secref{sec:flags}
compiles the main document as a draft:
%
\begin{center}
\begin{tabular}{l}
|\def\version{draft}|\\
|\input{childdoc.def}|\\
|\childdocforward{|\textit{main}|}|
\end{tabular}
\end{center}
%
Likewise, the following files |final|\textit{nn}|.tex|
compile the final version of the child document
|child|\textit{nn}|.tex|:
%
\begin{center}
\begin{tabular}{l}
|\def\version{final}|\\
|\input{childdoc.def}|\\
|\childdocforwardprefix{final}{child}|
\end{tabular}
\end{center}
%

Note that when several versions of a main file and/or of each child file
are to be generated, it may be convenient to set up a |Makefile| or
shell script to automatise the process.

%%%%%%%%%%%%%%%%%%%%%%%%%%%%%%%%%%%%%%%%%%%%%%%%%%%%%%%%%%%%%%%%%%%%%%%%%%%%%%%%
\subsection{Command Line Processing}
\label{sec:commandline}

The effect of redirection files can also be achieved by invoking
the \LaTeX{} compiler with a more elaborate command line.
Most conveniently this should be done as part
of a shell script or a |Makefile|.

When using \textsf{childdoc} in the main file, the following
command lines effectively perform a redirection
(note that depending on the shell being used,
backslashes may have to be doubled: `|\|' $\to$ `|\\|'):
%
\begin{center}
|... -jobname "|\textit{target}|" |\\|"|[\textit{flags}]%
|\input{childdoc.def}\childdocforward[|\textit{main}|]{|\textit{dest}|}"|
\end{center}
%
Here \textit{target} is the name of the output file,
\textit{main} is the name of the main file
and \textit{dest} is the name of the main or child file to be processed
(all filenames without extensions).
The optional argument \textit{main} can be omitted
if \textit{main} matches \textit{dest}.
Optionally, compilation \textit{flags} can be defined via |\def| commands.
This command line makes the \TeX{} engine believe
it is compiling the file \textit{target}
whose content is specified as the latter parameter.
The provided code then forwards the processing to
\textit{main} or \textit{dest} as described in \secref{sec:forward}.

%%%%%%%%%%%%%%%%%%%%%%%%%%%%%%%%%%%%%%%%%%%%%%%%%%%%%%%%%%%%%%%%%%%%%%%%%%%%%%%%
\subsection{Include by Input}
\label{sec:input}

Including child documents by |\include| has some restrictions by design.
Most notably, the content of a child document always occupies
its own set of pages; pages cannot be shared between child documents.
Usually, this behaviour makes perfect sense
because each child document contain an essential part of the document.
However, in some situations it may be desirable to compose
a document from a collection of parts
without having mandatory page breaks between then.
For this case, the package
provides a mechanism to include parts
by |\input| which can also be processed individually.
However, by construction this mechanism
requires manual handling of the content to be output.

%%%%%%%%%%%%%%%%%%%%%%%%%%%%%%%%%%%%%%%%
\DescribeMacro{\ifchilddocmanual}
The main file should be prepared as usual, see \secref{sec:include}.
However, the document body must make a distinction
between processing of an individual part and of the main document, e.g.:
%
\begin{center}
\begin{tabular}{l}
|\ifchilddocmanual|\\
|\input{\childdocname}|\\
|\||else|\\
\textit{document body with }|\input{|\textit{part}|}|\\
|\||fi|
\end{tabular}
\end{center}
%
The conditional |\ifchilddocmanual| is true whenever
a part to be included by |\input| is being compiled,
and the name of the part is stored in |\childdocname|.

%%%%%%%%%%%%%%%%%%%%%%%%%%%%%%%%%%%%%%%%
\DescribeMacro{\childdocby}
Each part to be included by |\input| should start with:
%
\begin{center}
\begin{tabular}{l}
|\input{childdoc.def}|\\
|\childdocby{|\textit{main}|}|\\
\end{tabular}
\end{center}
%
The directive |\childdocby| is similar to |\childdocof|
described in \secref{sec:include},
but the subsequent selection of content must be done manually.
To that end, both |\ifchilddoc| and |\ifchilddocmanual|
will be true upon processing of a part,
and the name of the part is stored in |\childdocname|.
Note that |\jobname| will be set to the filename of the current part
so that each part receives an individual |.aux| file
that does not interfere with the |.aux| file(s) of the main document.
This behaviour can be altered by the alternative form
|\childdocby[*]{|\textit{main}|}| (with a non-empty optional argument)
which uses the |.aux| file of the main document
by setting |\jobname| to \textit{main}.

%%%%%%%%%%%%%%%%%%%%%%%%%%%%%%%%%%%%%%%%%%%%%%%%%%%%%%%%%%%%%%%%%%%%%%%%%%%%%%%%
\subsection{Driver Development}
\label{sec:driver}

The \textsf{childdoc} mechanism can also be use for the development
of definition files such as \LaTeX{} styles or classes.
This case differs from the above setup with multiple parts
included by |\include| in that no |\includeonly| should be invoked.
This can be achieved by starting the include file
(before |\ProvidesPackage|) with:
%
\begin{center}
\begin{tabular}{l}
|\input{childdoc.def}|\\
|\childdocforward{|\textit{main}|}|\\
\end{tabular}
\end{center}
%
or alternatively with:
%
\begin{center}
\begin{tabular}{l}
|\input{childdoc.def}|\\
|\childdocby{|\textit{main}|}|\\
\end{tabular}
\end{center}
%
Both forms have slightly different effects as described above.
The main file is prepared as usual, see \secref{sec:include}.

%%%%%%%%%%%%%%%%%%%%%%%%%%%%%%%%%%%%%%%%%%%%%%%%%%%%%%%%%%%%%%%%%%%%%%%%%%%%%%%%
\subsection{Legacy Detection}
\label{sec:detection}

The directive |\childdocmain| in the main file can detect
whether the complete document or merely a child is to be compiled
even without using the directive |\childdocof|.
This method is deprecated because it is less robust
and there is no compelling reason to use it;
it is merely provided for backward compatibility
and it may be removed in future versions.

If the detection mechanism is to be used,
it is mandatory to correctly specify
the filename of the main file as the argument of |\childdocmain|:
%
\begin{center}
\begin{tabular}{l}
|\input{childdoc.def}|\\
|\childdocmain{|\textit{main}|}|\\
\end{tabular}
\end{center}
%
If |\jobname| does not match the argument \textit{main} of |\childdocmain|,
it is assumed that |\jobname| points to the child file to be compiled.
When using |\childdocmain| with the main file specified as argument,
it suffices to start a child file
with just |\input{|\textit{main}|}|
without loading of the package and using |\childdocof|.
If instead all processing is done
with the appropriate \textsf{childdoc} directives,
the argument of \textit{main} of |\childdocmain| can be empty.

An alternative version of the command line processing described
in \secref{sec:commandline} using the detection mechanism reads:
%
\begin{center}
|... -jobname "|\textit{target}|" "|[\textit{flags}]%
[|\def\jobname{|\textit{dest}|}|]|\input{|\textit{main}|}"|
\end{center}

%%%%%%%%%%%%%%%%%%%%%%%%%%%%%%%%%%%%%%%%%%%%%%%%%%%%%%%%%%%%%%%%%%%%%%%%%%%%%%%%
\subsection{Manual Code}
\label{sec:manual}

In case one cannot be certain whether the definitions file |childdoc.def|
is installed on the target \TeX{} distribution
and one prefers not to ship it,
it is conceivable to paste a few relevant commands into the sources.

To that end, drop all statements |\input{childdoc.def}|
and perform the replacements as outlined below.
Instead of |\childdocmain{|\textit{main}|}| add the following code
to the top of the main file:
%
\begin{center}
\begin{tabular}{l}
|\||ifdefined\childdocname\endinput\||fi\newif\ifchilddoc|\\
|\edef\childdocname{\scantokens\expandafter{\jobname\noexpand}}|\\
|\def\childdocmain{|\textit{main}|}\||ifx\childdocmain\childdocname\||else|\\
|\childdoctrue\includeonly{\childdocname}\let\jobname\childdocmain\||fi|\\
\end{tabular}
\end{center}
%
Instead of |\childdocof{|\textit{main}|}| just include the main file
at the top of each child file:
%
\begin{center}
|\input{|\textit{main}|}|
\end{center}
%
A simple redirection |\childdocforward{|\textit{dest}|}| is achieved by:
%
\begin{center}
|\def\jobname{|\textit{dest}|}\input{\jobname}|
\end{center}
%
The redirection with prefix
|\childdocforwardprefix[|\textit{prefix}|]{|\textit{dest}|}|
is accomplished by:
%
\begin{center}
\begin{tabular}{l}
|{\edef\jobname{\scantokens\expandafter{\jobname\noexpand}}|\\
|\def\redirectjob |\textit{prefix}|#1~~~{\gdef\jobname{|\textit{dest}|#1}}|\\
|\expandafter\redirectjob\jobname~~~}\input{\jobname}|
\end{tabular}
\end{center}

In an alternative approach,
child documents can be compiled by a specific command line
without additional code or specific definitions:
%
\begin{center}
|... -jobname "|\textit{target}|" "|[\textit{flags}]%
|\includeonly{|\textit{dest}|}\input{|\textit{main}|}"|
\end{center}
%

%%%%%%%%%%%%%%%%%%%%%%%%%%%%%%%%%%%%%%%%%%%%%%%%%%%%%%%%%%%%%%%%%%%%%%%%%%%%%%%%
%%%%%%%%%%%%%%%%%%%%%%%%%%%%%%%%%%%%%%%%%%%%%%%%%%%%%%%%%%%%%%%%%%%%%%%%%%%%%%%%
\section{Information}

%%%%%%%%%%%%%%%%%%%%%%%%%%%%%%%%%%%%%%%%%%%%%%%%%%%%%%%%%%%%%%%%%%%%%%%%%%%%%%%%
\subsection{Copyright}

Copyright \copyright{} 2017--2018 Niklas Beisert

This work may be distributed and/or modified under the
conditions of the \LaTeX{} Project Public License, either version 1.3
of this license or (at your option) any later version.
The latest version of this license is in
  \url{http://www.latex-project.org/lppl.txt}
and version 1.3 or later is part of all distributions of \LaTeX{}
version 2005/12/01 or later.

This work has the LPPL maintenance status `maintained'.

The Current Maintainer of this work is Niklas Beisert.

This work consists of the files |README.txt|, |childdoc.ins| and |childdoc.dtx|
as well as the derived files |childdoc.def|, |cdocsamp.tex|
with |cdocsch1.tex|, |cdocsch2.tex|, |cdocspt3.tex|, |cdocspt4.tex|,
|cdocsdrf.tex|, |cdocsfn1.tex|, |cdocsfn2.tex|
as well as |childdoc.pdf|.

%%%%%%%%%%%%%%%%%%%%%%%%%%%%%%%%%%%%%%%%%%%%%%%%%%%%%%%%%%%%%%%%%%%%%%%%%%%%%%%%
\subsection{Files and Installation}

The package consists of the files:
%
\begin{center}
\begin{tabular}{ll}
    |README.txt|   & readme file \\
    |childdoc.ins| & installation file \\
    |childdoc.dtx| & source file \\
    |childdoc.def| & definition file \\
    |cdocsamp.tex| & sample main file \\
    |cdocsch1.tex| & sample include file \\
    |cdocsch2.tex| & sample include file \\
    |cdocspt3.tex| & sample part file \\
    |cdocspt4.tex| & sample part file \\
    |cdocsdrf.tex| & sample redirection file \\
    |cdocsfn1.tex| & sample redirection file \\
    |cdocsfn2.tex| & sample redirection file \\
    |childdoc.pdf| & manual
\end{tabular}
\end{center}
%
The distribution consists of the files
|README.txt|, |childdoc.ins| and |childdoc.dtx|.
%
\begin{itemize}
\item
Run (pdf)\LaTeX{} on |childdoc.dtx|
to compile the manual |childdoc.pdf| (this file).
\item
Run \LaTeX{} on |childdoc.ins| to create the definitions file |childdoc.def|
and the sample |cdocsamp.tex| with include files
|cdocsch1.tex|, |cdocsch2.tex|, |cdocspt3.tex|, |cdocspt4.tex|,
|cdocsdrf.tex|, |cdocsfn1.tex|, |cdocsfn2.tex|.
Then copy the file |childdoc.def| to an appropriate directory of your \LaTeX{}
distribution, e.g.\ \textit{texmf-root}|/tex/latex/childdoc|.
\end{itemize}

%%%%%%%%%%%%%%%%%%%%%%%%%%%%%%%%%%%%%%%%%%%%%%%%%%%%%%%%%%%%%%%%%%%%%%%%%%%%%%%%
\subsection{Related CTAN Packages}

There are several other packages which offer a similar functionality:
%
\begin{itemize}
\item
The packages
\href{http://ctan.org/pkg/docmute}{\textsf{docmute}},
\href{http://ctan.org/pkg/includex}{\textsf{includex}} and
\href{http://ctan.org/pkg/standalone}{\textsf{standalone}}
provide commands to include only the document body of
a child file thus allowing both files to be compiled individually.
\item
The packages \href{http://ctan.org/pkg/subdocs}{\textsf{subdocs}}
and \href{http://ctan.org/pkg/subfiles}{\textsf{subfiles}}
provide structures in which the main and child documents can be
encapsulated and allowing them to be compiled individually.
The inclusion mechanism is different from the conventional |\include|.
\item
The package \href{http://ctan.org/pkg/combine}{\textsf{combine}}
is an elaborate solution to combine several documents into one.
\end{itemize}
%
See also the CTAN topic \href{http://ctan.org/topic/subdocs}{\textsf{subdocs}}
for further related packages.
The present package differs from the above solutions in that
a document structure constructed with the conventional |\include| mechanism
just needs two extra commands at the top of every file
such that all constituent files can be compiled individually.

%%%%%%%%%%%%%%%%%%%%%%%%%%%%%%%%%%%%%%%%%%%%%%%%%%%%%%%%%%%%%%%%%%%%%%%%%%%%%%%%
%\subsection{Feature Suggestions}
%
%The following is a list of features which may be useful for future
%versions of this package:
%%
%\begin{itemize}
%\item
%\ldots
%\end{itemize}

%%%%%%%%%%%%%%%%%%%%%%%%%%%%%%%%%%%%%%%%%%%%%%%%%%%%%%%%%%%%%%%%%%%%%%%%%%%%%%%%
\subsection{Revision History}

%%%%%%%%%%%%%%%%%%%%%%%%%%%%%%%%%%%%%%%%
\paragraph{v2.0:} 2018/12/30

\begin{itemize}
\item
immediate forward processing
\item
added |\childdocby| mechanism
\item
manual restructured
\end{itemize}

%%%%%%%%%%%%%%%%%%%%%%%%%%%%%%%%%%%%%%%%
\paragraph{v1.6:} 2018/01/17

\begin{itemize}
\item
application for development of include files
\item
corrections to manual
\end{itemize}

%%%%%%%%%%%%%%%%%%%%%%%%%%%%%%%%%%%%%%%%
\paragraph{v1.5:} 2017/05/21

\begin{itemize}
\item
more complete structuring introduced
\item
|\childdocof| introduced
\item
|\childdoc| renamed to |\childdocmain|
\item
|\childredirect| renamed to |\childdocforward| and |\childdocforwardprefix|
and functionality expanded
\end{itemize}

%%%%%%%%%%%%%%%%%%%%%%%%%%%%%%%%%%%%%%%%
\paragraph{v1.0:} 2017/04/27

\begin{itemize}
\item
manual and install package
\item
first version published on CTAN
\end{itemize}

%%%%%%%%%%%%%%%%%%%%%%%%%%%%%%%%%%%%%%%%
\paragraph{v0.6:} 2017/04/26

\begin{itemize}
\item
redirection mechanism added
\end{itemize}

%%%%%%%%%%%%%%%%%%%%%%%%%%%%%%%%%%%%%%%%
\paragraph{v0.5:} 2017/04/26

\begin{itemize}
\item
functionality in definition file
\end{itemize}


%%%%%%%%%%%%%%%%%%%%%%%%%%%%%%%%%%%%%%%%%%%%%%%%%%%%%%%%%%%%%%%%%%%%%%%%%%%%%%%%
%%%%%%%%%%%%%%%%%%%%%%%%%%%%%%%%%%%%%%%%%%%%%%%%%%%%%%%%%%%%%%%%%%%%%%%%%%%%%%%%
%%%%%%%%%%%%%%%%%%%%%%%%%%%%%%%%%%%%%%%%%%%%%%%%%%%%%%%%%%%%%%%%%%%%%%%%%%%%%%%%
\appendix

\settowidth\MacroIndent{\rmfamily\scriptsize 000\ }

 \DocInput{childdoc.dtx}

\end{document}
%</driver>
% \fi
%
% %%%%%%%%%%%%%%%%%%%%%%%%%%%%%%%%%%%%%%%%%%%%%%%%%%%%%%%%%%%%%%%%%%%%%%%%%%%%%%
% %%%%%%%%%%%%%%%%%%%%%%%%%%%%%%%%%%%%%%%%%%%%%%%%%%%%%%%%%%%%%%%%%%%%%%%%%%%%%%
% \section{Sample}
%\iffalse
%<*samplemain>
%\fi
%
% The following presents a sample document
% with two chapters, two parts, a title page,
% a compile flag as well as three forwarding files to set the flag.
% It consists of eight |.tex| files:
% \begin{center}
% \begin{tabular}{ll}
% |cdocsamp.tex|&main file\\
% |cdocsch1.tex|&include file for chapter 1\\
% |cdocsch2.tex|&include file for chapter 2\\
% |cdocspt3.tex|&include file for part 3\\
% |cdocspt4.tex|&include file for part 4\\
% |cdocsdrf.tex|&forwarding file for main file in draft mode\\
% |cdocsfi1.tex|&forwarding file for final version of chapter 1\\
% |cdocsfi2.tex|&forwarding file for final version of chapter 2\\
% \end{tabular}
% \end{center}
% Each of the eight files can be compiled directly by the \LaTeX{} compiler.
%
% %%%%%%%%%%%%%%%%%%%%%%%%%%%%%%%%%%%%%%
% \paragraph{Main File.}
%
% The main file is called |cdocsamp.tex|.
%
% Load the \textsf{childdoc} definitions and
% declare the filename for the main document:
%    \begin{macrocode}
\input{childdoc.def}
\childdocmain{}
%    \end{macrocode}

% Optional override for |\version| flag:
%    \begin{macrocode}
%%\ifchilddoc\else\providecommand{\version}{draft}\fi
%    \end{macrocode}

% Define the default values for the |\version| flag
% (|final| for the main file and |draft| for childs):
%    \begin{macrocode}
\ifchilddoc
\providecommand{\version}{draft}
\else
\providecommand{\version}{final}
\fi
%    \end{macrocode}

% Load the standard document class:
%    \begin{macrocode}
\documentclass[12pt]{article}
%    \end{macrocode}

% Start the document body:
%    \begin{macrocode}
\begin{document}
%    \end{macrocode}

% Declare a title page.
% Print title, part of document being processed and version flag:
%    \begin{macrocode}
\addtocounter{page}{-1}
\begin{center}
{\LARGE\bfseries{}childdoc example\par}
\vspace{1cm}
\ifchilddoc
\ifchilddocmanual part\else chapter\fi:
`\childdocname' of `\childdocjob'\par
\else
main document: `\childdocjob'\par
\fi
version: \version\par
\end{center}
\newpage
%    \end{macrocode}

% Manually include selected file,
% otherwise process as usual:
%    \begin{macrocode}
\ifchilddocmanual
\section*{part `\childdocname'}
\input{\childdocname}
\else
%    \end{macrocode}

% Include the two chapters:
%    \begin{macrocode}
\include{cdocsch1}
\include{cdocsch2}
%    \end{macrocode}

% Include the two parts unless only chapters should be displayed:
%    \begin{macrocode}
\ifchilddoc\else
\section{part three}
\input{cdocspt3}
\section{part four}
\input{cdocspt4}
\fi
%    \end{macrocode}

% Process as usual until here:
%    \begin{macrocode}
\fi
%    \end{macrocode}

% End of document body:
%    \begin{macrocode}
\end{document}
%    \end{macrocode}
%\iffalse
%</samplemain>
%\fi
%
% %%%%%%%%%%%%%%%%%%%%%%%%%%%%%%%%%%%%%%
% \paragraph{Chapter Include Files.}
%
% The include files are called |cdocsch1.tex| and |cdocsch2.tex|.
%
%\iffalse
%<*samplechap1|samplechap2>
%\fi

% Optional override for |\version| flag:
%    \begin{macrocode}
%%\providecommand{\version}{final}
%    \end{macrocode}

% Include the main document:
%    \begin{macrocode}
\input{childdoc.def}
\childdocof{cdocsamp}
%    \end{macrocode}

%\iffalse
%</samplechap1|samplechap2>
%\fi
%
%\iffalse
%<*samplechap1>
%\fi
% Some text for chapter 1:
%    \begin{macrocode}
\section{one}
some text in chapter one
%    \end{macrocode}

%\iffalse
%</samplechap1>
%\fi
% Some text for chapter 2:
%\iffalse
%<*samplechap2>
%\fi
%    \begin{macrocode}
\section{two}
more text in chapter two
%    \end{macrocode}

%\iffalse
%</samplechap2>
%\fi
%
% %%%%%%%%%%%%%%%%%%%%%%%%%%%%%%%%%%%%%%
% \paragraph{Part Include Files.}
%
% The include files are called |cdocspt3.tex| and |cdocspt4.tex|.
%
%\iffalse
%<*samplepart3|samplepart4>
%\fi

% Optional override for |\version| flag:
%    \begin{macrocode}
%%\providecommand{\version}{final}
%    \end{macrocode}

% Include the main document:
%    \begin{macrocode}
\input{childdoc.def}
\childdocby{cdocsamp}
%    \end{macrocode}

%\iffalse
%</samplepart3|samplepart4>
%\fi
%
%\iffalse
%<*samplepart3>
%\fi
% Some text for part 3:
%    \begin{macrocode}
some text in part three
%    \end{macrocode}

%\iffalse
%</samplepart3>
%\fi
% Some text for part 4:
%\iffalse
%<*samplepart4>
%\fi
%    \begin{macrocode}
more text in part four
%    \end{macrocode}

%\iffalse
%</samplepart4>
%\fi
%
% %%%%%%%%%%%%%%%%%%%%%%%%%%%%%%%%%%%%%%
% \paragraph{Forwarding for a Complete Draft.}
%
% The following forwarding file |cdocsdrf.tex|
% compiles the main document in draft mode:
%\iffalse
%<*sampledraft>
%\fi
%    \begin{macrocode}
\def\version{draft}
\input{childdoc.def}
\childdocforward{cdocsamp}
%    \end{macrocode}

%\iffalse
%</sampledraft>
%\fi
%
% %%%%%%%%%%%%%%%%%%%%%%%%%%%%%%%%%%%%%%
% \paragraph{Forwarding for Final Version of the Chapters.}
%
% The following forwarding files |cdocsfn1.tex| and |cdocsfn2.tex|
% (with identical content)
% compile the final versions of the child documents
% |cdocsch1.tex| and |cdocsch2.tex|, respectively:
%\iffalse
%<*samplefinal>
%\fi
%    \begin{macrocode}
\def\version{final}
\input{childdoc.def}
\childdocforwardprefix[cdocsamp]{cdocsfn}{cdocsch}
%    \end{macrocode}

%\iffalse
%</samplefinal>
%\fi
%
% %%%%%%%%%%%%%%%%%%%%%%%%%%%%%%%%%%%%%%
% \paragraph{Command Line Processing.}
%
% The following three command lines generate the output files
% |cdocscld|, |cdocscl1| and |cdocscl2|
% which should be identical to
% |cdocsdrf|, |cdocsch1| and |cdocsfn2|, respectively:
% \begin{center}
% \begin{tabular}{l}
% |latex -jobname cdocscld \|\\
% |  "\def\version{draft}\input{childdoc.def}\childdocforward{cdocsamp}"|\\
% |latex -jobname cdocscl1 \|\\
% |  "\input{childdoc.def}\childdocforward[cdocsamp]{cdocsch1}"|\\
% |latex -jobname cdocscl2 \|\\
% |  "\def\version{final}\input{childdoc.def}\childdocforward{cdocsch2}"|
% \end{tabular}
% \end{center}
% Note that the trailing backslash on each first line
% merely continues the input to the second line
% (for convenient cut ant paste).
% Furthermore, the command |latex| can be replaced by any
% of its alternative versions such as |pdflatex|.
%
% %%%%%%%%%%%%%%%%%%%%%%%%%%%%%%%%%%%%%%%%%%%%%%%%%%%%%%%%%%%%%%%%%%%%%%%%%%%%%%
% %%%%%%%%%%%%%%%%%%%%%%%%%%%%%%%%%%%%%%%%%%%%%%%%%%%%%%%%%%%%%%%%%%%%%%%%%%%%%%
% \section{Implementation}
%\iffalse
%<*package>
%\fi
%
% This section describes the definitions file |childdoc.def|.

% The definitions cannot be loaded using |\usepackage| or |\RequirePackage|
% which has a mechanism to prevent loading a style file more than once.
% When loading the definitions by means of |\input|
% multiple instances have to be prevented manually:
%\iffalse
%This code needs to be before the `\ProvidesFile' directive
%which is defined at the beginning of this file.
%Therefore it is also placed there and commented out here.
%</package>
%<*discard>
%\fi
%    \begin{macrocode}
\ifdefined\childdocmain\endinput\fi
%    \end{macrocode}
%\iffalse
%</discard>
%<*package>
%\fi
%
% \macro{\ifchilddoc}
% \macro{\ifchilddocmanual}
% The conditional |\ifchilddoc| tells whether a
% child (true) or main (false) document is being compiled.
% The conditional |\ifchilddocmanual| tells whether
% the |\includeonly| mechanism is used (false) or
% the selection of child files must be performed manually (true).
% The definitions initialise to false:
%    \begin{macrocode}
\newif\ifchilddoc
\newif\ifchilddocmanual
%    \end{macrocode}

% \macro{\childdocname}
% \macro{\childdocjob}
% The macro |\childdocname| stores the name of the main document
% to be compiled. The macro |\childdocjob| stores the name of
% the document on which the \LaTeX{} compiler was originally invoked.
% The content of |\jobname| cannot be compared
% to filenames specified in the source due to different catcodes.
% The following code rescans |\jobname|, stores the result
% in |\childdocname| and saves a copy in |\childdocjob|:
%    \begin{macrocode}
\edef\childdocname{\scantokens\expandafter{\jobname\noexpand}}
\let\childdocjob\childdocname
%    \end{macrocode}

% \macro{\childdocdisable}
% The macro |\childdocdisable| prevents the main file
% from being processed more than once.
% At this stage, the main document command |\childdocmain|
% is assumed to be called once again where it should do nothing.
% Any subsequent call to it should prevent
% a secondary processing of the main document
% It overwrites the forwarding commands
% |\childdocof| and |\childdocforward|
% with empty macros to prevent further inclusions of the main document:
%    \begin{macrocode}
\newcommand{\childdocdisable}
{
  \renewcommand{\childdocmain}[1]{\renewcommand{\childdocmain}[1]{\endinput}}
  \renewcommand{\childdocof}[1]{}
  \renewcommand{\childdocby}[2][]{}
  \renewcommand{\childdocforward}[2][]{}
  \renewcommand{\childdocdisable}{}
}
%    \end{macrocode}

% \macro{\childdocmain}
% The macro |\childdocmain| is to be called at the top of the main file
% with nothing or the main filename (without extension) as argument.
% First, it breaks loops.
% If the argument is not empty and does not match |\childdocname|
% (which is set by the first inclusion of |childdoc.def|),
% |\ifchilddoc| is set to true, |\includeonly| is applied to the child file
% and |\jobname| is set to the main file
% (for proper handling of |.aux| files):
%    \begin{macrocode}
\newcommand{\childdocmain}[1]
{
  \childdocdisable\childdocmain{}
  \if?#1?\else
    \begingroup
      \def\childdoctmp{#1}
      \ifx\childdoctmp\childdocname
        \def\childdoctmp{}
      \else
        \def\childdoctmp
        {
          \childdoctrue
          \includeonly{\childdocname}
          \def\childdocjob{#1}
          \def\jobname{#1}
        }
      \fi
      \expandafter
    \endgroup
    \childdoctmp
  \fi
}
%    \end{macrocode}

% \macro{\childdocof}
% The command |\childdocof| redirects
% compilation to the main file |#1|.
%    \begin{macrocode}
\newcommand{\childdocof}[1]
{
  \childdocdisable
  \childdoctrue
  \includeonly{\childdocname}
  \def\jobname{#1}
  \def\childdocjob{#1}
  \input{#1}
}
%    \end{macrocode}

% \macro{\childdocby}
% The command |\childdocby| ....
%    \begin{macrocode}
\newcommand{\childdocby}[2][]
{
  \childdocdisable
  \childdoctrue
  \childdocmanualtrue
  \if?#1?\else
    \def\jobname{#2}
  \fi
  \def\childdocjob{#2}
  \input{#2}
  \endinput
}
%    \end{macrocode}

% \macro{\childdocforward}
% The command |\childdocforward| redirects
% compilation to the main file or
% (if the optional argument is given) a child file.
% Parameters are set as if the main file
% or a child file starting with |\childdocof| was compiled.
% Then compilation is handed over to the main file:
%    \begin{macrocode}
\newcommand{\childdocforward}[2][]
{
  \begingroup
    \if?#1?
      \def\childdoctmp
      {
        \def\childdocname{#2}
        \def\childdocjob{#2}
        \def\jobname{#2}
        \input{#2}
        \endinput
      }
    \else
      \def\childdoctmp
      {
        \childdocdisable
        \def\childdocname{#2}
        \childdoctrue
        \includeonly{#2}
        \def\childdocjob{#1}
        \def\jobname{#1}
        \input{#1}
        \endinput
      }
    \fi
    \expandafter
  \endgroup
  \childdoctmp
}
%    \end{macrocode}

% \macro{\childdocforwardprefix}
% The command |\childdocforwardprefix| redirects
% compilation to the main or a child file by means of a pattern.
% The prefix |#1| in the current filename is replaced by |#2|
% and the suffix of the current filename is kept
% (it is assumed that the filename does not contain the substring `|~~~|'
% which is used as a delimiter).
% Compilation is handed over to the new file by |\childdocforward|:
%    \begin{macrocode}
\newcommand{\childdocforwardprefix}[3][]
{
  \begingroup
    \def\childdocextract #2##1~~~{\def\childdoctmp{\childdocforward[#1]{#3##1}}}
    \expandafter\childdocextract\childdocname~~~
    \expandafter
  \endgroup
  \childdoctmp
}
%    \end{macrocode}

% \macro{\childdoc}
% The deprecated macro |\childdoc| is a legacy version of |\childdocmain|:
%    \begin{macrocode}
\newcommand{\childdoc}{\childdocmain}
%    \end{macrocode}

% \macro{\childdocredirect}
% The deprecated macro |\childdocredirect| is a legacy version
% of |\childdocforward| and |\childdocforwardprefix|:
%    \begin{macrocode}
\newcommand{\childdocredirect}[2][]
{
  \begingroup
    \if?#1?
      \def\childdoctmp{\childdocforward{#2}}
    \else
      \def\childdoctmp{\childdocforwardprefix{#1}{#2}}
    \fi
    \expandafter
  \endgroup
  \childdoctmp
}
%    \end{macrocode}

%\iffalse
%</package>
%\fi
%
\endinput
|\\
|\childdocof{|\textit{main}|}|\\
\end{tabular}
\end{center}
at the top of every child file \textit{child}
which is included by |\include{|\textit{child}|}|
from within the main file
(or at least for those files to be compiled individually).
The argument \textit{main} must be the filename of the main file.

There are a couple of
considerations in setting up the main and child documents:

%%%%%%%%%%%%%%%%%%%%%%%%%%%%%%%%%%%%%%%%
\paragraph{Restrictions.}

Please note the following restrictions:
\begin{itemize}
\item
|\childdocmain| must be called with one argument \textit{main}
to ensure compatibility with earlier version of the package.
It must either be empty (|\childdocmain{}|)
or precisely match the filename of the main file in which it is specified.
See \secref{sec:detection} for further information.
\item
The filename \textit{main} must be specified without the |.tex| extension.
\item
The filename \textit{main} is case sensitive
(even in case-insensitive file systems)
due to internal string comparison.
\item
The argument \textit{main} should be fully expanded, it cannot be a macro.
\item
Subdirectories and special characters should be avoided in filenames.
\item
The command |\childdocmain{|\textit{main}|}| must be followed by a whitespace.
It should not be followed immediately by another command
or by a comment mark `|%|'.
This is because the \TeX{} parser reads the token immediately following
the argument of |\childdocmain| and puts it
at the beginning of every child section;
however, a white\-space is ignored.
\end{itemize}

%%%%%%%%%%%%%%%%%%%%%%%%%%%%%%%%%%%%%%%%
\paragraph{Content of Main File.}

It is advisable to place all content in the child files included by |\include|.
Any output contained in the main file will appear in all child documents
unless suppressed manually;
it cannot be suppressed automatically by the |\includeonly| directive
and thus should normally be avoided.
A method to include some content in the main file
by means of conditional processing is described in \secref{sec:conditional}.

%%%%%%%%%%%%%%%%%%%%%%%%%%%%%%%%%%%%%%%%
\paragraph{Page Numbering.}

When only a part of the document is compiled,
the appropriate numbering of pages
(as well as other status parameters)
is determined from the |.aux| files.
The latter contain information from previous passes.
However this information needs to propagate through
all intermediate child documents.
Therefore the page numbering in child documents may well
be inconsistent until the complete document is compiled at least once.

A useful (if unconventional) way to always ensure a consistent
page numbering is to restart the numbering in each child document
and denote the pages by `\textit{child}|.|\textit{page}'
where \textit{child} represents the chapter/section number of the child file.
This can be achieved by the command
|\numberwithin{page}{|\textit{child}|}|
of the \textsf{amsmath} package
where \textit{child} can be |chapter| or |section|
depending on the chosen structuring.
Alternatively, one can modify the macro |\thepage| appropriately
and reset the counter |page| at the start of each child file.

%%%%%%%%%%%%%%%%%%%%%%%%%%%%%%%%%%%%%%%%%%%%%%%%%%%%%%%%%%%%%%%%%%%%%%%%%%%%%%%%
\subsection{Conditional Processing}
\label{sec:conditional}

The package provides a mechanism to compile different versions
of a document. To customise the versions further some conditional processing
can come in handy to distinguish which version is being compiled.
The package provides two macros to describe the compilation context:

%%%%%%%%%%%%%%%%%%%%%%%%%%%%%%%%%%%%%%%%
\DescribeMacro{\ifchilddoc}
The conditional |\ifchilddoc| distinguishes between the compilation of
child documents and the main document:
%
\begin{center}
|\ifchilddoc |\textit{child-code}| |[|\||else |\textit{main-code}]| \||fi|
\end{center}

%%%%%%%%%%%%%%%%%%%%%%%%%%%%%%%%%%%%%%%%
\DescribeMacro{\childdocname}
\DescribeMacro{\childdocjob}
The macro |\childdocname| contains the filename (without extension)
of the main or child file being processed.
Note that |\childdocjob| will always contain the name of the main file.

%%%%%%%%%%%%%%%%%%%%%%%%%%%%%%%%%%%%%%%%
\paragraph{Title Page.}

Conditional processing can be used to include a title or banner page
in the main document when proper precautions are taken.
Importantly, the code in the main file should ensure that the page counter
(as well as other status parameters which are stored in the |.aux| files)
takes the same value after the conditional processing.
Otherwise the page numbers may take divergent values
depending on which part is compiled.

For example, a title page could be declared by:
%
\begin{center}
\begin{tabular}{l}
|\ifchilddoc\||else|\\
|\addtocounter{page}{-1}|\\
\textit{code for title page}\\
|\newpage|\\
|\||fi|
\end{tabular}
\end{center}
%
A banner page for the child documents can be generated by:
%
\begin{center}
\begin{tabular}{l}
|\ifchilddoc|\\
|\addtocounter{page}{-1}|\\
\textit{code for banner page}\\
|\newpage|\\
|\||fi|
\end{tabular}
\end{center}
%
Here one could write a message such as:
\begin{center}
|This is the part \childdocname{} of \childdocjob{}.|
\end{center}

%%%%%%%%%%%%%%%%%%%%%%%%%%%%%%%%%%%%%%%%%%%%%%%%%%%%%%%%%%%%%%%%%%%%%%%%%%%%%%%%
\subsection{Flags}
\label{sec:flags}

The package makes it easy to generate different versions
of the main or child documents.
To this end compilation flags can be defined
and assigned different default values.
They will be particularly useful in conjunction
with the forwarding mechanism described in \secref{sec:forward}.

For example, it may be useful to have a flag |\version|
which can be set to |draft| or |final|.
The document source will contain some conditional code
depending on the value of |\version|.
Suppose further, the flag should default to |final| for the main file
and to |draft| for child files
which is a natural assignment for editing the document.
This is achieved by placing the following code
in the preamble of the main document
(below the |\childdocmain| directive):
%
\begin{center}
\begin{tabular}{l}
|\ifchilddoc|\\
|\providecommand{\version}{draft}|\\
|\||else|\\
|\providecommand{\version}{final}|\\
|\||fi|
\end{tabular}
\end{center}
%
The definition by |\providecommand| makes sure
that previous definitions are not overwritten.
Further statements |\providecommand{\version}{...}|
can thus be added before the above code to override it.

For the main file, one might add a line
(between |\childdocmain| and the above block)
%
\begin{center}
|%\ifchilddoc\||else\providecommand{\version}{draft}\||fi|
\end{center}
%
which can be uncommented to produce a draft version.
Likewise one can add a line to the very top of a child file
(above the |\childdocof{|\textit{main}|}| directive)
%
\begin{center}
|%\providecommand{\version}{final}|
\end{center}
%
which can be uncommented to produce the final version of this child document.

%%%%%%%%%%%%%%%%%%%%%%%%%%%%%%%%%%%%%%%%%%%%%%%%%%%%%%%%%%%%%%%%%%%%%%%%%%%%%%%%
\subsection{Forwarding}
\label{sec:forward}

Different versions of the main or child documents
using compilation flags as described in \secref{sec:flags}
can be (permanently) stored in different files
for convenient compilation, viewing and distribution.
To this end, the package defines a command
to pass on compilation to a different file:

%%%%%%%%%%%%%%%%%%%%%%%%%%%%%%%%%%%%%%%%
\DescribeMacro{\childdocforward}
The command |\childdocforward| redirects processing to
another source file:
%
\begin{center}
\begin{tabular}{l}
|% \iffalse
%
% childdoc.dtx Copyright (C) 2017-2018 Niklas Beisert
%
% This work may be distributed and/or modified under the
% conditions of the LaTeX Project Public License, either version 1.3
% of this license or (at your option) any later version.
% The latest version of this license is in
%   http://www.latex-project.org/lppl.txt
% and version 1.3 or later is part of all distributions of LaTeX
% version 2005/12/01 or later.
%
% This work has the LPPL maintenance status `maintained'.
%
% The Current Maintainer of this work is Niklas Beisert.
%
% This work consists of the files childdoc.dtx and childdoc.ins
% and the derived files childdoc.def and cdocsamp.tex with
% cdocsch1.tex, cdocsch2.tex, cdocsdrf.tex, cdocsfn1.tex, cdocsfn2.tex.
%
%<package>\ifdefined\childdocmain\endinput\fi
%<package>\ProvidesFile{childdoc.def}[2018/12/30 v2.0 child document driver]
%<samplemain>\ProvidesFile{cdocsamp.tex}[2018/12/30 v2.0 sample for childdoc]
%<*driver>
%\ProvidesFile{childdoc.drv}[2018/12/30 v2.0 childdoc reference manual file]
\PassOptionsToClass{10pt,a4paper}{article}
\documentclass{ltxdoc}

\usepackage[margin=35mm]{geometry}
\usepackage{hyperref}
\usepackage{hyperxmp}
\usepackage[usenames]{color}

\hypersetup{colorlinks=true}
\hypersetup{pdfstartview=FitH}
\hypersetup{pdfpagemode=UseNone}
\hypersetup{pdfsource={}}
\hypersetup{pdflang={en-UK}}
\hypersetup{pdfcopyright={Copyright 2017-2018 Niklas Beisert.
  This work may be distributed and/or modified under the
  conditions of the LaTeX Project Public License, either version 1.3
  of this license or (at your option) any later version.}}
\hypersetup{pdflicenseurl={http://www.latex-project.org/lppl.txt}}
\hypersetup{pdfcontactaddress={ETH Zurich, ITP, HIT K,
  Wolfgang-Pauli-Strasse 27}}
\hypersetup{pdfcontactpostcode={8093}}
\hypersetup{pdfcontactcity={Zurich}}
\hypersetup{pdfcontactcountry={Switzerland}}
\hypersetup{pdfcontactemail={nbeisert@itp.phys.ethz.ch}}
\hypersetup{pdfcontacturl={http://people.phys.ethz.ch/\xmptilde nbeisert/}}

\newcommand{\secref}[1]{\hyperref[#1]{section \ref*{#1}}}

\parskip1ex
\parindent0pt
\let\olditemize\itemize
\def\itemize{\olditemize\parskip0pt}

\begin{document}

\title{The \textsf{childdoc} Package}
\hypersetup{pdftitle={The childdoc Package}}
\author{Niklas Beisert\\[2ex]
  Institut f\"ur Theoretische Physik\\
  Eidgen\"ossische Technische Hochschule Z\"urich\\
  Wolfgang-Pauli-Strasse 27, 8093 Z\"urich, Switzerland\\[1ex]
  \href{mailto:nbeisert@itp.phys.ethz.ch}
  {\texttt{nbeisert@itp.phys.ethz.ch}}}
\hypersetup{pdfauthor={Niklas Beisert}}
\hypersetup{pdfsubject={Manual for the LaTeX2e Package childdoc}}
\date{30 December 2018, \textsf{v2.0}}
\maketitle

\begin{abstract}\noindent
\textsf{childdoc} is a \LaTeXe{} package
that enables the direct compilation
of document sections included by |\include|
to individual files.
\end{abstract}

\begingroup
\parskip0ex
\tableofcontents
\endgroup

%%%%%%%%%%%%%%%%%%%%%%%%%%%%%%%%%%%%%%%%%%%%%%%%%%%%%%%%%%%%%%%%%%%%%%%%%%%%%%%%
%%%%%%%%%%%%%%%%%%%%%%%%%%%%%%%%%%%%%%%%%%%%%%%%%%%%%%%%%%%%%%%%%%%%%%%%%%%%%%%%
\section{Introduction}

\LaTeX{} provides a mechanism to structure a large document (such as a book)
into a main file and several child files (containing the chapters)
using the |\include| command.
This mechanism is beneficial for documents
which span hundreds of pages in order to
make the source file(s) more manageable.
Moreover, compilation can be restricted to
selected child files by means of the |\includeonly| command.
The latter feature can be used to reduce the compilation time while editing
(this was significantly more useful in the earlier days of \LaTeX{})
or to generate a smaller document which is easier to navigate.
Another application of |\includeonly| is to generate
documents consisting of selected parts of the complete document.

However, there are a few drawbacks of the plain |\include| mechanism:
\begin{itemize}
\item
The child files cannot be compiled on their own,
they can only be compiled via the main file.
A naive editing environment
(such as a text editor with an option
to have the current file processed by \LaTeX)
may require one to switch to the main file before compiling;
attempting to compile the child file produces errors.
\item
The main file must be modified (each time)
to adjust the |\includeonly| command
to the present needs. This easily leaves the main file in a messy state.
\item
The generated document will always carry the filename
of the main document. This is inconvenient if
several child files are to be compiled and
to be kept for distribution.
\end{itemize}

The present package provides a simple interface
to make child files individually compilable by \LaTeX{}.
Compiling a child file then has the same effect as compiling
the main file with an |\includeonly| command
to select the appropriate child.
Moreover the generated document will carry the name of the child
rather than the main file.
This resolves all three above issues.

This feature is meant to make the editing of books,
thesis documents and lecture notes somewhat more convenient.
However, the package can also be used efficiently for
composing a series of documents (such as exercise sheets)
which are typically distributed individually.
It then assists the author in generating the individual documents
(potentially in different versions)
as well as a document containing the collected series.
Another application is in developing style files
or other kinds of included material
where compilation of the style file could redirect
to a sample or test file.

%%%%%%%%%%%%%%%%%%%%%%%%%%%%%%%%%%%%%%%%%%%%%%%%%%%%%%%%%%%%%%%%%%%%%%%%%%%%%%%%
%%%%%%%%%%%%%%%%%%%%%%%%%%%%%%%%%%%%%%%%%%%%%%%%%%%%%%%%%%%%%%%%%%%%%%%%%%%%%%%%
\section{Usage}

First of all, the package \textsf{childdoc} is \emph{not} a standard
\LaTeXe{} |.sty| style file! Therefore it needs to be invoked in
a non-standard way.

%%%%%%%%%%%%%%%%%%%%%%%%%%%%%%%%%%%%%%%%%%%%%%%%%%%%%%%%%%%%%%%%%%%%%%%%%%%%%%%%
\subsection{Included Files}
\label{sec:include}

%%%%%%%%%%%%%%%%%%%%%%%%%%%%%%%%%%%%%%%%
\DescribeMacro{\childdocmain}
To use the package, add the commands
\begin{center}
\begin{tabular}{l}
|\input{childdoc.def}|\\
|\childdocmain{}|\\
\end{tabular}
\end{center}
at the very top of the main \LaTeX{} file,
in particular \emph{before} the |\documentclass| statement!
The argument of |\childdocmain| should be left empty
(but it must be present).

%%%%%%%%%%%%%%%%%%%%%%%%%%%%%%%%%%%%%%%%
\DescribeMacro{\childdocof}
Furthermore, add the commands
\begin{center}
\begin{tabular}{l}
|\input{childdoc.def}|\\
|\childdocof{|\textit{main}|}|\\
\end{tabular}
\end{center}
at the top of every child file \textit{child}
which is included by |\include{|\textit{child}|}|
from within the main file
(or at least for those files to be compiled individually).
The argument \textit{main} must be the filename of the main file.

There are a couple of
considerations in setting up the main and child documents:

%%%%%%%%%%%%%%%%%%%%%%%%%%%%%%%%%%%%%%%%
\paragraph{Restrictions.}

Please note the following restrictions:
\begin{itemize}
\item
|\childdocmain| must be called with one argument \textit{main}
to ensure compatibility with earlier version of the package.
It must either be empty (|\childdocmain{}|)
or precisely match the filename of the main file in which it is specified.
See \secref{sec:detection} for further information.
\item
The filename \textit{main} must be specified without the |.tex| extension.
\item
The filename \textit{main} is case sensitive
(even in case-insensitive file systems)
due to internal string comparison.
\item
The argument \textit{main} should be fully expanded, it cannot be a macro.
\item
Subdirectories and special characters should be avoided in filenames.
\item
The command |\childdocmain{|\textit{main}|}| must be followed by a whitespace.
It should not be followed immediately by another command
or by a comment mark `|%|'.
This is because the \TeX{} parser reads the token immediately following
the argument of |\childdocmain| and puts it
at the beginning of every child section;
however, a white\-space is ignored.
\end{itemize}

%%%%%%%%%%%%%%%%%%%%%%%%%%%%%%%%%%%%%%%%
\paragraph{Content of Main File.}

It is advisable to place all content in the child files included by |\include|.
Any output contained in the main file will appear in all child documents
unless suppressed manually;
it cannot be suppressed automatically by the |\includeonly| directive
and thus should normally be avoided.
A method to include some content in the main file
by means of conditional processing is described in \secref{sec:conditional}.

%%%%%%%%%%%%%%%%%%%%%%%%%%%%%%%%%%%%%%%%
\paragraph{Page Numbering.}

When only a part of the document is compiled,
the appropriate numbering of pages
(as well as other status parameters)
is determined from the |.aux| files.
The latter contain information from previous passes.
However this information needs to propagate through
all intermediate child documents.
Therefore the page numbering in child documents may well
be inconsistent until the complete document is compiled at least once.

A useful (if unconventional) way to always ensure a consistent
page numbering is to restart the numbering in each child document
and denote the pages by `\textit{child}|.|\textit{page}'
where \textit{child} represents the chapter/section number of the child file.
This can be achieved by the command
|\numberwithin{page}{|\textit{child}|}|
of the \textsf{amsmath} package
where \textit{child} can be |chapter| or |section|
depending on the chosen structuring.
Alternatively, one can modify the macro |\thepage| appropriately
and reset the counter |page| at the start of each child file.

%%%%%%%%%%%%%%%%%%%%%%%%%%%%%%%%%%%%%%%%%%%%%%%%%%%%%%%%%%%%%%%%%%%%%%%%%%%%%%%%
\subsection{Conditional Processing}
\label{sec:conditional}

The package provides a mechanism to compile different versions
of a document. To customise the versions further some conditional processing
can come in handy to distinguish which version is being compiled.
The package provides two macros to describe the compilation context:

%%%%%%%%%%%%%%%%%%%%%%%%%%%%%%%%%%%%%%%%
\DescribeMacro{\ifchilddoc}
The conditional |\ifchilddoc| distinguishes between the compilation of
child documents and the main document:
%
\begin{center}
|\ifchilddoc |\textit{child-code}| |[|\||else |\textit{main-code}]| \||fi|
\end{center}

%%%%%%%%%%%%%%%%%%%%%%%%%%%%%%%%%%%%%%%%
\DescribeMacro{\childdocname}
\DescribeMacro{\childdocjob}
The macro |\childdocname| contains the filename (without extension)
of the main or child file being processed.
Note that |\childdocjob| will always contain the name of the main file.

%%%%%%%%%%%%%%%%%%%%%%%%%%%%%%%%%%%%%%%%
\paragraph{Title Page.}

Conditional processing can be used to include a title or banner page
in the main document when proper precautions are taken.
Importantly, the code in the main file should ensure that the page counter
(as well as other status parameters which are stored in the |.aux| files)
takes the same value after the conditional processing.
Otherwise the page numbers may take divergent values
depending on which part is compiled.

For example, a title page could be declared by:
%
\begin{center}
\begin{tabular}{l}
|\ifchilddoc\||else|\\
|\addtocounter{page}{-1}|\\
\textit{code for title page}\\
|\newpage|\\
|\||fi|
\end{tabular}
\end{center}
%
A banner page for the child documents can be generated by:
%
\begin{center}
\begin{tabular}{l}
|\ifchilddoc|\\
|\addtocounter{page}{-1}|\\
\textit{code for banner page}\\
|\newpage|\\
|\||fi|
\end{tabular}
\end{center}
%
Here one could write a message such as:
\begin{center}
|This is the part \childdocname{} of \childdocjob{}.|
\end{center}

%%%%%%%%%%%%%%%%%%%%%%%%%%%%%%%%%%%%%%%%%%%%%%%%%%%%%%%%%%%%%%%%%%%%%%%%%%%%%%%%
\subsection{Flags}
\label{sec:flags}

The package makes it easy to generate different versions
of the main or child documents.
To this end compilation flags can be defined
and assigned different default values.
They will be particularly useful in conjunction
with the forwarding mechanism described in \secref{sec:forward}.

For example, it may be useful to have a flag |\version|
which can be set to |draft| or |final|.
The document source will contain some conditional code
depending on the value of |\version|.
Suppose further, the flag should default to |final| for the main file
and to |draft| for child files
which is a natural assignment for editing the document.
This is achieved by placing the following code
in the preamble of the main document
(below the |\childdocmain| directive):
%
\begin{center}
\begin{tabular}{l}
|\ifchilddoc|\\
|\providecommand{\version}{draft}|\\
|\||else|\\
|\providecommand{\version}{final}|\\
|\||fi|
\end{tabular}
\end{center}
%
The definition by |\providecommand| makes sure
that previous definitions are not overwritten.
Further statements |\providecommand{\version}{...}|
can thus be added before the above code to override it.

For the main file, one might add a line
(between |\childdocmain| and the above block)
%
\begin{center}
|%\ifchilddoc\||else\providecommand{\version}{draft}\||fi|
\end{center}
%
which can be uncommented to produce a draft version.
Likewise one can add a line to the very top of a child file
(above the |\childdocof{|\textit{main}|}| directive)
%
\begin{center}
|%\providecommand{\version}{final}|
\end{center}
%
which can be uncommented to produce the final version of this child document.

%%%%%%%%%%%%%%%%%%%%%%%%%%%%%%%%%%%%%%%%%%%%%%%%%%%%%%%%%%%%%%%%%%%%%%%%%%%%%%%%
\subsection{Forwarding}
\label{sec:forward}

Different versions of the main or child documents
using compilation flags as described in \secref{sec:flags}
can be (permanently) stored in different files
for convenient compilation, viewing and distribution.
To this end, the package defines a command
to pass on compilation to a different file:

%%%%%%%%%%%%%%%%%%%%%%%%%%%%%%%%%%%%%%%%
\DescribeMacro{\childdocforward}
The command |\childdocforward| redirects processing to
another source file:
%
\begin{center}
\begin{tabular}{l}
|\input{childdoc.def}|\\
|\childdocforward[|\textit{main}|]{|\textit{dest}|}|\\
\end{tabular}
\end{center}
%
The argument \textit{dest} is the destination file
(without extension).
It should be the main file or one of the child files.
Note that further \textsf{childdoc} directives
such as |\childdocof| and |\childdocforward|
in the indicated file will be processed in this form.
The optional argument \textit{main}
passes on directly to the main file \textit{main}
while pretending to compile the child \textit{dest}.
This form behaves as if \textit{dest}
issues |\childdocof{|\textit{main}|}| right away,
and no further \textsf{childdoc} directives will be processed.

%%%%%%%%%%%%%%%%%%%%%%%%%%%%%%%%%%%%%%%%
\DescribeMacro{\...prefix}
In the alternative form |\childdocforwardprefix|,
%
\begin{center}
\begin{tabular}{l}
|\input{childdoc.def}|\\
|\childdocforwardprefix[|\textit{main}|]{|\textit{prefix}|}{|\textit{dest}|}|
\end{tabular}
\end{center}
%
the destination file is determined by a pattern
depending on the current file:
To make this work, the current file must be called
`{\textit{prefix}\hspace{0.2em}\textit{suffix}}'
with \textit{prefix} matching precisely the argument.
Processing is then passed on to the file
`{\textit{dest}\hspace{0.2em}\textit{suffix}}'.
Surely, the same effect is achieved by
directly specifying the
argument `{\textit{dest}\hspace{0.2em}\textit{suffix}}'
in the first form.
However, that requires to set up a different file
for each child. With the alternative form of the command
all these files can have exactly the same content
which simplifies setting them up and maintaining them.

For example, the following file |draft.tex|
with a compilation flag |\version| as described in \secref{sec:flags}
compiles the main document as a draft:
%
\begin{center}
\begin{tabular}{l}
|\def\version{draft}|\\
|\input{childdoc.def}|\\
|\childdocforward{|\textit{main}|}|
\end{tabular}
\end{center}
%
Likewise, the following files |final|\textit{nn}|.tex|
compile the final version of the child document
|child|\textit{nn}|.tex|:
%
\begin{center}
\begin{tabular}{l}
|\def\version{final}|\\
|\input{childdoc.def}|\\
|\childdocforwardprefix{final}{child}|
\end{tabular}
\end{center}
%

Note that when several versions of a main file and/or of each child file
are to be generated, it may be convenient to set up a |Makefile| or
shell script to automatise the process.

%%%%%%%%%%%%%%%%%%%%%%%%%%%%%%%%%%%%%%%%%%%%%%%%%%%%%%%%%%%%%%%%%%%%%%%%%%%%%%%%
\subsection{Command Line Processing}
\label{sec:commandline}

The effect of redirection files can also be achieved by invoking
the \LaTeX{} compiler with a more elaborate command line.
Most conveniently this should be done as part
of a shell script or a |Makefile|.

When using \textsf{childdoc} in the main file, the following
command lines effectively perform a redirection
(note that depending on the shell being used,
backslashes may have to be doubled: `|\|' $\to$ `|\\|'):
%
\begin{center}
|... -jobname "|\textit{target}|" |\\|"|[\textit{flags}]%
|\input{childdoc.def}\childdocforward[|\textit{main}|]{|\textit{dest}|}"|
\end{center}
%
Here \textit{target} is the name of the output file,
\textit{main} is the name of the main file
and \textit{dest} is the name of the main or child file to be processed
(all filenames without extensions).
The optional argument \textit{main} can be omitted
if \textit{main} matches \textit{dest}.
Optionally, compilation \textit{flags} can be defined via |\def| commands.
This command line makes the \TeX{} engine believe
it is compiling the file \textit{target}
whose content is specified as the latter parameter.
The provided code then forwards the processing to
\textit{main} or \textit{dest} as described in \secref{sec:forward}.

%%%%%%%%%%%%%%%%%%%%%%%%%%%%%%%%%%%%%%%%%%%%%%%%%%%%%%%%%%%%%%%%%%%%%%%%%%%%%%%%
\subsection{Include by Input}
\label{sec:input}

Including child documents by |\include| has some restrictions by design.
Most notably, the content of a child document always occupies
its own set of pages; pages cannot be shared between child documents.
Usually, this behaviour makes perfect sense
because each child document contain an essential part of the document.
However, in some situations it may be desirable to compose
a document from a collection of parts
without having mandatory page breaks between then.
For this case, the package
provides a mechanism to include parts
by |\input| which can also be processed individually.
However, by construction this mechanism
requires manual handling of the content to be output.

%%%%%%%%%%%%%%%%%%%%%%%%%%%%%%%%%%%%%%%%
\DescribeMacro{\ifchilddocmanual}
The main file should be prepared as usual, see \secref{sec:include}.
However, the document body must make a distinction
between processing of an individual part and of the main document, e.g.:
%
\begin{center}
\begin{tabular}{l}
|\ifchilddocmanual|\\
|\input{\childdocname}|\\
|\||else|\\
\textit{document body with }|\input{|\textit{part}|}|\\
|\||fi|
\end{tabular}
\end{center}
%
The conditional |\ifchilddocmanual| is true whenever
a part to be included by |\input| is being compiled,
and the name of the part is stored in |\childdocname|.

%%%%%%%%%%%%%%%%%%%%%%%%%%%%%%%%%%%%%%%%
\DescribeMacro{\childdocby}
Each part to be included by |\input| should start with:
%
\begin{center}
\begin{tabular}{l}
|\input{childdoc.def}|\\
|\childdocby{|\textit{main}|}|\\
\end{tabular}
\end{center}
%
The directive |\childdocby| is similar to |\childdocof|
described in \secref{sec:include},
but the subsequent selection of content must be done manually.
To that end, both |\ifchilddoc| and |\ifchilddocmanual|
will be true upon processing of a part,
and the name of the part is stored in |\childdocname|.
Note that |\jobname| will be set to the filename of the current part
so that each part receives an individual |.aux| file
that does not interfere with the |.aux| file(s) of the main document.
This behaviour can be altered by the alternative form
|\childdocby[*]{|\textit{main}|}| (with a non-empty optional argument)
which uses the |.aux| file of the main document
by setting |\jobname| to \textit{main}.

%%%%%%%%%%%%%%%%%%%%%%%%%%%%%%%%%%%%%%%%%%%%%%%%%%%%%%%%%%%%%%%%%%%%%%%%%%%%%%%%
\subsection{Driver Development}
\label{sec:driver}

The \textsf{childdoc} mechanism can also be use for the development
of definition files such as \LaTeX{} styles or classes.
This case differs from the above setup with multiple parts
included by |\include| in that no |\includeonly| should be invoked.
This can be achieved by starting the include file
(before |\ProvidesPackage|) with:
%
\begin{center}
\begin{tabular}{l}
|\input{childdoc.def}|\\
|\childdocforward{|\textit{main}|}|\\
\end{tabular}
\end{center}
%
or alternatively with:
%
\begin{center}
\begin{tabular}{l}
|\input{childdoc.def}|\\
|\childdocby{|\textit{main}|}|\\
\end{tabular}
\end{center}
%
Both forms have slightly different effects as described above.
The main file is prepared as usual, see \secref{sec:include}.

%%%%%%%%%%%%%%%%%%%%%%%%%%%%%%%%%%%%%%%%%%%%%%%%%%%%%%%%%%%%%%%%%%%%%%%%%%%%%%%%
\subsection{Legacy Detection}
\label{sec:detection}

The directive |\childdocmain| in the main file can detect
whether the complete document or merely a child is to be compiled
even without using the directive |\childdocof|.
This method is deprecated because it is less robust
and there is no compelling reason to use it;
it is merely provided for backward compatibility
and it may be removed in future versions.

If the detection mechanism is to be used,
it is mandatory to correctly specify
the filename of the main file as the argument of |\childdocmain|:
%
\begin{center}
\begin{tabular}{l}
|\input{childdoc.def}|\\
|\childdocmain{|\textit{main}|}|\\
\end{tabular}
\end{center}
%
If |\jobname| does not match the argument \textit{main} of |\childdocmain|,
it is assumed that |\jobname| points to the child file to be compiled.
When using |\childdocmain| with the main file specified as argument,
it suffices to start a child file
with just |\input{|\textit{main}|}|
without loading of the package and using |\childdocof|.
If instead all processing is done
with the appropriate \textsf{childdoc} directives,
the argument of \textit{main} of |\childdocmain| can be empty.

An alternative version of the command line processing described
in \secref{sec:commandline} using the detection mechanism reads:
%
\begin{center}
|... -jobname "|\textit{target}|" "|[\textit{flags}]%
[|\def\jobname{|\textit{dest}|}|]|\input{|\textit{main}|}"|
\end{center}

%%%%%%%%%%%%%%%%%%%%%%%%%%%%%%%%%%%%%%%%%%%%%%%%%%%%%%%%%%%%%%%%%%%%%%%%%%%%%%%%
\subsection{Manual Code}
\label{sec:manual}

In case one cannot be certain whether the definitions file |childdoc.def|
is installed on the target \TeX{} distribution
and one prefers not to ship it,
it is conceivable to paste a few relevant commands into the sources.

To that end, drop all statements |\input{childdoc.def}|
and perform the replacements as outlined below.
Instead of |\childdocmain{|\textit{main}|}| add the following code
to the top of the main file:
%
\begin{center}
\begin{tabular}{l}
|\||ifdefined\childdocname\endinput\||fi\newif\ifchilddoc|\\
|\edef\childdocname{\scantokens\expandafter{\jobname\noexpand}}|\\
|\def\childdocmain{|\textit{main}|}\||ifx\childdocmain\childdocname\||else|\\
|\childdoctrue\includeonly{\childdocname}\let\jobname\childdocmain\||fi|\\
\end{tabular}
\end{center}
%
Instead of |\childdocof{|\textit{main}|}| just include the main file
at the top of each child file:
%
\begin{center}
|\input{|\textit{main}|}|
\end{center}
%
A simple redirection |\childdocforward{|\textit{dest}|}| is achieved by:
%
\begin{center}
|\def\jobname{|\textit{dest}|}\input{\jobname}|
\end{center}
%
The redirection with prefix
|\childdocforwardprefix[|\textit{prefix}|]{|\textit{dest}|}|
is accomplished by:
%
\begin{center}
\begin{tabular}{l}
|{\edef\jobname{\scantokens\expandafter{\jobname\noexpand}}|\\
|\def\redirectjob |\textit{prefix}|#1~~~{\gdef\jobname{|\textit{dest}|#1}}|\\
|\expandafter\redirectjob\jobname~~~}\input{\jobname}|
\end{tabular}
\end{center}

In an alternative approach,
child documents can be compiled by a specific command line
without additional code or specific definitions:
%
\begin{center}
|... -jobname "|\textit{target}|" "|[\textit{flags}]%
|\includeonly{|\textit{dest}|}\input{|\textit{main}|}"|
\end{center}
%

%%%%%%%%%%%%%%%%%%%%%%%%%%%%%%%%%%%%%%%%%%%%%%%%%%%%%%%%%%%%%%%%%%%%%%%%%%%%%%%%
%%%%%%%%%%%%%%%%%%%%%%%%%%%%%%%%%%%%%%%%%%%%%%%%%%%%%%%%%%%%%%%%%%%%%%%%%%%%%%%%
\section{Information}

%%%%%%%%%%%%%%%%%%%%%%%%%%%%%%%%%%%%%%%%%%%%%%%%%%%%%%%%%%%%%%%%%%%%%%%%%%%%%%%%
\subsection{Copyright}

Copyright \copyright{} 2017--2018 Niklas Beisert

This work may be distributed and/or modified under the
conditions of the \LaTeX{} Project Public License, either version 1.3
of this license or (at your option) any later version.
The latest version of this license is in
  \url{http://www.latex-project.org/lppl.txt}
and version 1.3 or later is part of all distributions of \LaTeX{}
version 2005/12/01 or later.

This work has the LPPL maintenance status `maintained'.

The Current Maintainer of this work is Niklas Beisert.

This work consists of the files |README.txt|, |childdoc.ins| and |childdoc.dtx|
as well as the derived files |childdoc.def|, |cdocsamp.tex|
with |cdocsch1.tex|, |cdocsch2.tex|, |cdocspt3.tex|, |cdocspt4.tex|,
|cdocsdrf.tex|, |cdocsfn1.tex|, |cdocsfn2.tex|
as well as |childdoc.pdf|.

%%%%%%%%%%%%%%%%%%%%%%%%%%%%%%%%%%%%%%%%%%%%%%%%%%%%%%%%%%%%%%%%%%%%%%%%%%%%%%%%
\subsection{Files and Installation}

The package consists of the files:
%
\begin{center}
\begin{tabular}{ll}
    |README.txt|   & readme file \\
    |childdoc.ins| & installation file \\
    |childdoc.dtx| & source file \\
    |childdoc.def| & definition file \\
    |cdocsamp.tex| & sample main file \\
    |cdocsch1.tex| & sample include file \\
    |cdocsch2.tex| & sample include file \\
    |cdocspt3.tex| & sample part file \\
    |cdocspt4.tex| & sample part file \\
    |cdocsdrf.tex| & sample redirection file \\
    |cdocsfn1.tex| & sample redirection file \\
    |cdocsfn2.tex| & sample redirection file \\
    |childdoc.pdf| & manual
\end{tabular}
\end{center}
%
The distribution consists of the files
|README.txt|, |childdoc.ins| and |childdoc.dtx|.
%
\begin{itemize}
\item
Run (pdf)\LaTeX{} on |childdoc.dtx|
to compile the manual |childdoc.pdf| (this file).
\item
Run \LaTeX{} on |childdoc.ins| to create the definitions file |childdoc.def|
and the sample |cdocsamp.tex| with include files
|cdocsch1.tex|, |cdocsch2.tex|, |cdocspt3.tex|, |cdocspt4.tex|,
|cdocsdrf.tex|, |cdocsfn1.tex|, |cdocsfn2.tex|.
Then copy the file |childdoc.def| to an appropriate directory of your \LaTeX{}
distribution, e.g.\ \textit{texmf-root}|/tex/latex/childdoc|.
\end{itemize}

%%%%%%%%%%%%%%%%%%%%%%%%%%%%%%%%%%%%%%%%%%%%%%%%%%%%%%%%%%%%%%%%%%%%%%%%%%%%%%%%
\subsection{Related CTAN Packages}

There are several other packages which offer a similar functionality:
%
\begin{itemize}
\item
The packages
\href{http://ctan.org/pkg/docmute}{\textsf{docmute}},
\href{http://ctan.org/pkg/includex}{\textsf{includex}} and
\href{http://ctan.org/pkg/standalone}{\textsf{standalone}}
provide commands to include only the document body of
a child file thus allowing both files to be compiled individually.
\item
The packages \href{http://ctan.org/pkg/subdocs}{\textsf{subdocs}}
and \href{http://ctan.org/pkg/subfiles}{\textsf{subfiles}}
provide structures in which the main and child documents can be
encapsulated and allowing them to be compiled individually.
The inclusion mechanism is different from the conventional |\include|.
\item
The package \href{http://ctan.org/pkg/combine}{\textsf{combine}}
is an elaborate solution to combine several documents into one.
\end{itemize}
%
See also the CTAN topic \href{http://ctan.org/topic/subdocs}{\textsf{subdocs}}
for further related packages.
The present package differs from the above solutions in that
a document structure constructed with the conventional |\include| mechanism
just needs two extra commands at the top of every file
such that all constituent files can be compiled individually.

%%%%%%%%%%%%%%%%%%%%%%%%%%%%%%%%%%%%%%%%%%%%%%%%%%%%%%%%%%%%%%%%%%%%%%%%%%%%%%%%
%\subsection{Feature Suggestions}
%
%The following is a list of features which may be useful for future
%versions of this package:
%%
%\begin{itemize}
%\item
%\ldots
%\end{itemize}

%%%%%%%%%%%%%%%%%%%%%%%%%%%%%%%%%%%%%%%%%%%%%%%%%%%%%%%%%%%%%%%%%%%%%%%%%%%%%%%%
\subsection{Revision History}

%%%%%%%%%%%%%%%%%%%%%%%%%%%%%%%%%%%%%%%%
\paragraph{v2.0:} 2018/12/30

\begin{itemize}
\item
immediate forward processing
\item
added |\childdocby| mechanism
\item
manual restructured
\end{itemize}

%%%%%%%%%%%%%%%%%%%%%%%%%%%%%%%%%%%%%%%%
\paragraph{v1.6:} 2018/01/17

\begin{itemize}
\item
application for development of include files
\item
corrections to manual
\end{itemize}

%%%%%%%%%%%%%%%%%%%%%%%%%%%%%%%%%%%%%%%%
\paragraph{v1.5:} 2017/05/21

\begin{itemize}
\item
more complete structuring introduced
\item
|\childdocof| introduced
\item
|\childdoc| renamed to |\childdocmain|
\item
|\childredirect| renamed to |\childdocforward| and |\childdocforwardprefix|
and functionality expanded
\end{itemize}

%%%%%%%%%%%%%%%%%%%%%%%%%%%%%%%%%%%%%%%%
\paragraph{v1.0:} 2017/04/27

\begin{itemize}
\item
manual and install package
\item
first version published on CTAN
\end{itemize}

%%%%%%%%%%%%%%%%%%%%%%%%%%%%%%%%%%%%%%%%
\paragraph{v0.6:} 2017/04/26

\begin{itemize}
\item
redirection mechanism added
\end{itemize}

%%%%%%%%%%%%%%%%%%%%%%%%%%%%%%%%%%%%%%%%
\paragraph{v0.5:} 2017/04/26

\begin{itemize}
\item
functionality in definition file
\end{itemize}


%%%%%%%%%%%%%%%%%%%%%%%%%%%%%%%%%%%%%%%%%%%%%%%%%%%%%%%%%%%%%%%%%%%%%%%%%%%%%%%%
%%%%%%%%%%%%%%%%%%%%%%%%%%%%%%%%%%%%%%%%%%%%%%%%%%%%%%%%%%%%%%%%%%%%%%%%%%%%%%%%
%%%%%%%%%%%%%%%%%%%%%%%%%%%%%%%%%%%%%%%%%%%%%%%%%%%%%%%%%%%%%%%%%%%%%%%%%%%%%%%%
\appendix

\settowidth\MacroIndent{\rmfamily\scriptsize 000\ }

 \DocInput{childdoc.dtx}

\end{document}
%</driver>
% \fi
%
% %%%%%%%%%%%%%%%%%%%%%%%%%%%%%%%%%%%%%%%%%%%%%%%%%%%%%%%%%%%%%%%%%%%%%%%%%%%%%%
% %%%%%%%%%%%%%%%%%%%%%%%%%%%%%%%%%%%%%%%%%%%%%%%%%%%%%%%%%%%%%%%%%%%%%%%%%%%%%%
% \section{Sample}
%\iffalse
%<*samplemain>
%\fi
%
% The following presents a sample document
% with two chapters, two parts, a title page,
% a compile flag as well as three forwarding files to set the flag.
% It consists of eight |.tex| files:
% \begin{center}
% \begin{tabular}{ll}
% |cdocsamp.tex|&main file\\
% |cdocsch1.tex|&include file for chapter 1\\
% |cdocsch2.tex|&include file for chapter 2\\
% |cdocspt3.tex|&include file for part 3\\
% |cdocspt4.tex|&include file for part 4\\
% |cdocsdrf.tex|&forwarding file for main file in draft mode\\
% |cdocsfi1.tex|&forwarding file for final version of chapter 1\\
% |cdocsfi2.tex|&forwarding file for final version of chapter 2\\
% \end{tabular}
% \end{center}
% Each of the eight files can be compiled directly by the \LaTeX{} compiler.
%
% %%%%%%%%%%%%%%%%%%%%%%%%%%%%%%%%%%%%%%
% \paragraph{Main File.}
%
% The main file is called |cdocsamp.tex|.
%
% Load the \textsf{childdoc} definitions and
% declare the filename for the main document:
%    \begin{macrocode}
\input{childdoc.def}
\childdocmain{}
%    \end{macrocode}

% Optional override for |\version| flag:
%    \begin{macrocode}
%%\ifchilddoc\else\providecommand{\version}{draft}\fi
%    \end{macrocode}

% Define the default values for the |\version| flag
% (|final| for the main file and |draft| for childs):
%    \begin{macrocode}
\ifchilddoc
\providecommand{\version}{draft}
\else
\providecommand{\version}{final}
\fi
%    \end{macrocode}

% Load the standard document class:
%    \begin{macrocode}
\documentclass[12pt]{article}
%    \end{macrocode}

% Start the document body:
%    \begin{macrocode}
\begin{document}
%    \end{macrocode}

% Declare a title page.
% Print title, part of document being processed and version flag:
%    \begin{macrocode}
\addtocounter{page}{-1}
\begin{center}
{\LARGE\bfseries{}childdoc example\par}
\vspace{1cm}
\ifchilddoc
\ifchilddocmanual part\else chapter\fi:
`\childdocname' of `\childdocjob'\par
\else
main document: `\childdocjob'\par
\fi
version: \version\par
\end{center}
\newpage
%    \end{macrocode}

% Manually include selected file,
% otherwise process as usual:
%    \begin{macrocode}
\ifchilddocmanual
\section*{part `\childdocname'}
\input{\childdocname}
\else
%    \end{macrocode}

% Include the two chapters:
%    \begin{macrocode}
\include{cdocsch1}
\include{cdocsch2}
%    \end{macrocode}

% Include the two parts unless only chapters should be displayed:
%    \begin{macrocode}
\ifchilddoc\else
\section{part three}
\input{cdocspt3}
\section{part four}
\input{cdocspt4}
\fi
%    \end{macrocode}

% Process as usual until here:
%    \begin{macrocode}
\fi
%    \end{macrocode}

% End of document body:
%    \begin{macrocode}
\end{document}
%    \end{macrocode}
%\iffalse
%</samplemain>
%\fi
%
% %%%%%%%%%%%%%%%%%%%%%%%%%%%%%%%%%%%%%%
% \paragraph{Chapter Include Files.}
%
% The include files are called |cdocsch1.tex| and |cdocsch2.tex|.
%
%\iffalse
%<*samplechap1|samplechap2>
%\fi

% Optional override for |\version| flag:
%    \begin{macrocode}
%%\providecommand{\version}{final}
%    \end{macrocode}

% Include the main document:
%    \begin{macrocode}
\input{childdoc.def}
\childdocof{cdocsamp}
%    \end{macrocode}

%\iffalse
%</samplechap1|samplechap2>
%\fi
%
%\iffalse
%<*samplechap1>
%\fi
% Some text for chapter 1:
%    \begin{macrocode}
\section{one}
some text in chapter one
%    \end{macrocode}

%\iffalse
%</samplechap1>
%\fi
% Some text for chapter 2:
%\iffalse
%<*samplechap2>
%\fi
%    \begin{macrocode}
\section{two}
more text in chapter two
%    \end{macrocode}

%\iffalse
%</samplechap2>
%\fi
%
% %%%%%%%%%%%%%%%%%%%%%%%%%%%%%%%%%%%%%%
% \paragraph{Part Include Files.}
%
% The include files are called |cdocspt3.tex| and |cdocspt4.tex|.
%
%\iffalse
%<*samplepart3|samplepart4>
%\fi

% Optional override for |\version| flag:
%    \begin{macrocode}
%%\providecommand{\version}{final}
%    \end{macrocode}

% Include the main document:
%    \begin{macrocode}
\input{childdoc.def}
\childdocby{cdocsamp}
%    \end{macrocode}

%\iffalse
%</samplepart3|samplepart4>
%\fi
%
%\iffalse
%<*samplepart3>
%\fi
% Some text for part 3:
%    \begin{macrocode}
some text in part three
%    \end{macrocode}

%\iffalse
%</samplepart3>
%\fi
% Some text for part 4:
%\iffalse
%<*samplepart4>
%\fi
%    \begin{macrocode}
more text in part four
%    \end{macrocode}

%\iffalse
%</samplepart4>
%\fi
%
% %%%%%%%%%%%%%%%%%%%%%%%%%%%%%%%%%%%%%%
% \paragraph{Forwarding for a Complete Draft.}
%
% The following forwarding file |cdocsdrf.tex|
% compiles the main document in draft mode:
%\iffalse
%<*sampledraft>
%\fi
%    \begin{macrocode}
\def\version{draft}
\input{childdoc.def}
\childdocforward{cdocsamp}
%    \end{macrocode}

%\iffalse
%</sampledraft>
%\fi
%
% %%%%%%%%%%%%%%%%%%%%%%%%%%%%%%%%%%%%%%
% \paragraph{Forwarding for Final Version of the Chapters.}
%
% The following forwarding files |cdocsfn1.tex| and |cdocsfn2.tex|
% (with identical content)
% compile the final versions of the child documents
% |cdocsch1.tex| and |cdocsch2.tex|, respectively:
%\iffalse
%<*samplefinal>
%\fi
%    \begin{macrocode}
\def\version{final}
\input{childdoc.def}
\childdocforwardprefix[cdocsamp]{cdocsfn}{cdocsch}
%    \end{macrocode}

%\iffalse
%</samplefinal>
%\fi
%
% %%%%%%%%%%%%%%%%%%%%%%%%%%%%%%%%%%%%%%
% \paragraph{Command Line Processing.}
%
% The following three command lines generate the output files
% |cdocscld|, |cdocscl1| and |cdocscl2|
% which should be identical to
% |cdocsdrf|, |cdocsch1| and |cdocsfn2|, respectively:
% \begin{center}
% \begin{tabular}{l}
% |latex -jobname cdocscld \|\\
% |  "\def\version{draft}\input{childdoc.def}\childdocforward{cdocsamp}"|\\
% |latex -jobname cdocscl1 \|\\
% |  "\input{childdoc.def}\childdocforward[cdocsamp]{cdocsch1}"|\\
% |latex -jobname cdocscl2 \|\\
% |  "\def\version{final}\input{childdoc.def}\childdocforward{cdocsch2}"|
% \end{tabular}
% \end{center}
% Note that the trailing backslash on each first line
% merely continues the input to the second line
% (for convenient cut ant paste).
% Furthermore, the command |latex| can be replaced by any
% of its alternative versions such as |pdflatex|.
%
% %%%%%%%%%%%%%%%%%%%%%%%%%%%%%%%%%%%%%%%%%%%%%%%%%%%%%%%%%%%%%%%%%%%%%%%%%%%%%%
% %%%%%%%%%%%%%%%%%%%%%%%%%%%%%%%%%%%%%%%%%%%%%%%%%%%%%%%%%%%%%%%%%%%%%%%%%%%%%%
% \section{Implementation}
%\iffalse
%<*package>
%\fi
%
% This section describes the definitions file |childdoc.def|.

% The definitions cannot be loaded using |\usepackage| or |\RequirePackage|
% which has a mechanism to prevent loading a style file more than once.
% When loading the definitions by means of |\input|
% multiple instances have to be prevented manually:
%\iffalse
%This code needs to be before the `\ProvidesFile' directive
%which is defined at the beginning of this file.
%Therefore it is also placed there and commented out here.
%</package>
%<*discard>
%\fi
%    \begin{macrocode}
\ifdefined\childdocmain\endinput\fi
%    \end{macrocode}
%\iffalse
%</discard>
%<*package>
%\fi
%
% \macro{\ifchilddoc}
% \macro{\ifchilddocmanual}
% The conditional |\ifchilddoc| tells whether a
% child (true) or main (false) document is being compiled.
% The conditional |\ifchilddocmanual| tells whether
% the |\includeonly| mechanism is used (false) or
% the selection of child files must be performed manually (true).
% The definitions initialise to false:
%    \begin{macrocode}
\newif\ifchilddoc
\newif\ifchilddocmanual
%    \end{macrocode}

% \macro{\childdocname}
% \macro{\childdocjob}
% The macro |\childdocname| stores the name of the main document
% to be compiled. The macro |\childdocjob| stores the name of
% the document on which the \LaTeX{} compiler was originally invoked.
% The content of |\jobname| cannot be compared
% to filenames specified in the source due to different catcodes.
% The following code rescans |\jobname|, stores the result
% in |\childdocname| and saves a copy in |\childdocjob|:
%    \begin{macrocode}
\edef\childdocname{\scantokens\expandafter{\jobname\noexpand}}
\let\childdocjob\childdocname
%    \end{macrocode}

% \macro{\childdocdisable}
% The macro |\childdocdisable| prevents the main file
% from being processed more than once.
% At this stage, the main document command |\childdocmain|
% is assumed to be called once again where it should do nothing.
% Any subsequent call to it should prevent
% a secondary processing of the main document
% It overwrites the forwarding commands
% |\childdocof| and |\childdocforward|
% with empty macros to prevent further inclusions of the main document:
%    \begin{macrocode}
\newcommand{\childdocdisable}
{
  \renewcommand{\childdocmain}[1]{\renewcommand{\childdocmain}[1]{\endinput}}
  \renewcommand{\childdocof}[1]{}
  \renewcommand{\childdocby}[2][]{}
  \renewcommand{\childdocforward}[2][]{}
  \renewcommand{\childdocdisable}{}
}
%    \end{macrocode}

% \macro{\childdocmain}
% The macro |\childdocmain| is to be called at the top of the main file
% with nothing or the main filename (without extension) as argument.
% First, it breaks loops.
% If the argument is not empty and does not match |\childdocname|
% (which is set by the first inclusion of |childdoc.def|),
% |\ifchilddoc| is set to true, |\includeonly| is applied to the child file
% and |\jobname| is set to the main file
% (for proper handling of |.aux| files):
%    \begin{macrocode}
\newcommand{\childdocmain}[1]
{
  \childdocdisable\childdocmain{}
  \if?#1?\else
    \begingroup
      \def\childdoctmp{#1}
      \ifx\childdoctmp\childdocname
        \def\childdoctmp{}
      \else
        \def\childdoctmp
        {
          \childdoctrue
          \includeonly{\childdocname}
          \def\childdocjob{#1}
          \def\jobname{#1}
        }
      \fi
      \expandafter
    \endgroup
    \childdoctmp
  \fi
}
%    \end{macrocode}

% \macro{\childdocof}
% The command |\childdocof| redirects
% compilation to the main file |#1|.
%    \begin{macrocode}
\newcommand{\childdocof}[1]
{
  \childdocdisable
  \childdoctrue
  \includeonly{\childdocname}
  \def\jobname{#1}
  \def\childdocjob{#1}
  \input{#1}
}
%    \end{macrocode}

% \macro{\childdocby}
% The command |\childdocby| ....
%    \begin{macrocode}
\newcommand{\childdocby}[2][]
{
  \childdocdisable
  \childdoctrue
  \childdocmanualtrue
  \if?#1?\else
    \def\jobname{#2}
  \fi
  \def\childdocjob{#2}
  \input{#2}
  \endinput
}
%    \end{macrocode}

% \macro{\childdocforward}
% The command |\childdocforward| redirects
% compilation to the main file or
% (if the optional argument is given) a child file.
% Parameters are set as if the main file
% or a child file starting with |\childdocof| was compiled.
% Then compilation is handed over to the main file:
%    \begin{macrocode}
\newcommand{\childdocforward}[2][]
{
  \begingroup
    \if?#1?
      \def\childdoctmp
      {
        \def\childdocname{#2}
        \def\childdocjob{#2}
        \def\jobname{#2}
        \input{#2}
        \endinput
      }
    \else
      \def\childdoctmp
      {
        \childdocdisable
        \def\childdocname{#2}
        \childdoctrue
        \includeonly{#2}
        \def\childdocjob{#1}
        \def\jobname{#1}
        \input{#1}
        \endinput
      }
    \fi
    \expandafter
  \endgroup
  \childdoctmp
}
%    \end{macrocode}

% \macro{\childdocforwardprefix}
% The command |\childdocforwardprefix| redirects
% compilation to the main or a child file by means of a pattern.
% The prefix |#1| in the current filename is replaced by |#2|
% and the suffix of the current filename is kept
% (it is assumed that the filename does not contain the substring `|~~~|'
% which is used as a delimiter).
% Compilation is handed over to the new file by |\childdocforward|:
%    \begin{macrocode}
\newcommand{\childdocforwardprefix}[3][]
{
  \begingroup
    \def\childdocextract #2##1~~~{\def\childdoctmp{\childdocforward[#1]{#3##1}}}
    \expandafter\childdocextract\childdocname~~~
    \expandafter
  \endgroup
  \childdoctmp
}
%    \end{macrocode}

% \macro{\childdoc}
% The deprecated macro |\childdoc| is a legacy version of |\childdocmain|:
%    \begin{macrocode}
\newcommand{\childdoc}{\childdocmain}
%    \end{macrocode}

% \macro{\childdocredirect}
% The deprecated macro |\childdocredirect| is a legacy version
% of |\childdocforward| and |\childdocforwardprefix|:
%    \begin{macrocode}
\newcommand{\childdocredirect}[2][]
{
  \begingroup
    \if?#1?
      \def\childdoctmp{\childdocforward{#2}}
    \else
      \def\childdoctmp{\childdocforwardprefix{#1}{#2}}
    \fi
    \expandafter
  \endgroup
  \childdoctmp
}
%    \end{macrocode}

%\iffalse
%</package>
%\fi
%
\endinput
|\\
|\childdocforward[|\textit{main}|]{|\textit{dest}|}|\\
\end{tabular}
\end{center}
%
The argument \textit{dest} is the destination file
(without extension).
It should be the main file or one of the child files.
Note that further \textsf{childdoc} directives
such as |\childdocof| and |\childdocforward|
in the indicated file will be processed in this form.
The optional argument \textit{main}
passes on directly to the main file \textit{main}
while pretending to compile the child \textit{dest}.
This form behaves as if \textit{dest}
issues |\childdocof{|\textit{main}|}| right away,
and no further \textsf{childdoc} directives will be processed.

%%%%%%%%%%%%%%%%%%%%%%%%%%%%%%%%%%%%%%%%
\DescribeMacro{\...prefix}
In the alternative form |\childdocforwardprefix|,
%
\begin{center}
\begin{tabular}{l}
|% \iffalse
%
% childdoc.dtx Copyright (C) 2017-2018 Niklas Beisert
%
% This work may be distributed and/or modified under the
% conditions of the LaTeX Project Public License, either version 1.3
% of this license or (at your option) any later version.
% The latest version of this license is in
%   http://www.latex-project.org/lppl.txt
% and version 1.3 or later is part of all distributions of LaTeX
% version 2005/12/01 or later.
%
% This work has the LPPL maintenance status `maintained'.
%
% The Current Maintainer of this work is Niklas Beisert.
%
% This work consists of the files childdoc.dtx and childdoc.ins
% and the derived files childdoc.def and cdocsamp.tex with
% cdocsch1.tex, cdocsch2.tex, cdocsdrf.tex, cdocsfn1.tex, cdocsfn2.tex.
%
%<package>\ifdefined\childdocmain\endinput\fi
%<package>\ProvidesFile{childdoc.def}[2018/12/30 v2.0 child document driver]
%<samplemain>\ProvidesFile{cdocsamp.tex}[2018/12/30 v2.0 sample for childdoc]
%<*driver>
%\ProvidesFile{childdoc.drv}[2018/12/30 v2.0 childdoc reference manual file]
\PassOptionsToClass{10pt,a4paper}{article}
\documentclass{ltxdoc}

\usepackage[margin=35mm]{geometry}
\usepackage{hyperref}
\usepackage{hyperxmp}
\usepackage[usenames]{color}

\hypersetup{colorlinks=true}
\hypersetup{pdfstartview=FitH}
\hypersetup{pdfpagemode=UseNone}
\hypersetup{pdfsource={}}
\hypersetup{pdflang={en-UK}}
\hypersetup{pdfcopyright={Copyright 2017-2018 Niklas Beisert.
  This work may be distributed and/or modified under the
  conditions of the LaTeX Project Public License, either version 1.3
  of this license or (at your option) any later version.}}
\hypersetup{pdflicenseurl={http://www.latex-project.org/lppl.txt}}
\hypersetup{pdfcontactaddress={ETH Zurich, ITP, HIT K,
  Wolfgang-Pauli-Strasse 27}}
\hypersetup{pdfcontactpostcode={8093}}
\hypersetup{pdfcontactcity={Zurich}}
\hypersetup{pdfcontactcountry={Switzerland}}
\hypersetup{pdfcontactemail={nbeisert@itp.phys.ethz.ch}}
\hypersetup{pdfcontacturl={http://people.phys.ethz.ch/\xmptilde nbeisert/}}

\newcommand{\secref}[1]{\hyperref[#1]{section \ref*{#1}}}

\parskip1ex
\parindent0pt
\let\olditemize\itemize
\def\itemize{\olditemize\parskip0pt}

\begin{document}

\title{The \textsf{childdoc} Package}
\hypersetup{pdftitle={The childdoc Package}}
\author{Niklas Beisert\\[2ex]
  Institut f\"ur Theoretische Physik\\
  Eidgen\"ossische Technische Hochschule Z\"urich\\
  Wolfgang-Pauli-Strasse 27, 8093 Z\"urich, Switzerland\\[1ex]
  \href{mailto:nbeisert@itp.phys.ethz.ch}
  {\texttt{nbeisert@itp.phys.ethz.ch}}}
\hypersetup{pdfauthor={Niklas Beisert}}
\hypersetup{pdfsubject={Manual for the LaTeX2e Package childdoc}}
\date{30 December 2018, \textsf{v2.0}}
\maketitle

\begin{abstract}\noindent
\textsf{childdoc} is a \LaTeXe{} package
that enables the direct compilation
of document sections included by |\include|
to individual files.
\end{abstract}

\begingroup
\parskip0ex
\tableofcontents
\endgroup

%%%%%%%%%%%%%%%%%%%%%%%%%%%%%%%%%%%%%%%%%%%%%%%%%%%%%%%%%%%%%%%%%%%%%%%%%%%%%%%%
%%%%%%%%%%%%%%%%%%%%%%%%%%%%%%%%%%%%%%%%%%%%%%%%%%%%%%%%%%%%%%%%%%%%%%%%%%%%%%%%
\section{Introduction}

\LaTeX{} provides a mechanism to structure a large document (such as a book)
into a main file and several child files (containing the chapters)
using the |\include| command.
This mechanism is beneficial for documents
which span hundreds of pages in order to
make the source file(s) more manageable.
Moreover, compilation can be restricted to
selected child files by means of the |\includeonly| command.
The latter feature can be used to reduce the compilation time while editing
(this was significantly more useful in the earlier days of \LaTeX{})
or to generate a smaller document which is easier to navigate.
Another application of |\includeonly| is to generate
documents consisting of selected parts of the complete document.

However, there are a few drawbacks of the plain |\include| mechanism:
\begin{itemize}
\item
The child files cannot be compiled on their own,
they can only be compiled via the main file.
A naive editing environment
(such as a text editor with an option
to have the current file processed by \LaTeX)
may require one to switch to the main file before compiling;
attempting to compile the child file produces errors.
\item
The main file must be modified (each time)
to adjust the |\includeonly| command
to the present needs. This easily leaves the main file in a messy state.
\item
The generated document will always carry the filename
of the main document. This is inconvenient if
several child files are to be compiled and
to be kept for distribution.
\end{itemize}

The present package provides a simple interface
to make child files individually compilable by \LaTeX{}.
Compiling a child file then has the same effect as compiling
the main file with an |\includeonly| command
to select the appropriate child.
Moreover the generated document will carry the name of the child
rather than the main file.
This resolves all three above issues.

This feature is meant to make the editing of books,
thesis documents and lecture notes somewhat more convenient.
However, the package can also be used efficiently for
composing a series of documents (such as exercise sheets)
which are typically distributed individually.
It then assists the author in generating the individual documents
(potentially in different versions)
as well as a document containing the collected series.
Another application is in developing style files
or other kinds of included material
where compilation of the style file could redirect
to a sample or test file.

%%%%%%%%%%%%%%%%%%%%%%%%%%%%%%%%%%%%%%%%%%%%%%%%%%%%%%%%%%%%%%%%%%%%%%%%%%%%%%%%
%%%%%%%%%%%%%%%%%%%%%%%%%%%%%%%%%%%%%%%%%%%%%%%%%%%%%%%%%%%%%%%%%%%%%%%%%%%%%%%%
\section{Usage}

First of all, the package \textsf{childdoc} is \emph{not} a standard
\LaTeXe{} |.sty| style file! Therefore it needs to be invoked in
a non-standard way.

%%%%%%%%%%%%%%%%%%%%%%%%%%%%%%%%%%%%%%%%%%%%%%%%%%%%%%%%%%%%%%%%%%%%%%%%%%%%%%%%
\subsection{Included Files}
\label{sec:include}

%%%%%%%%%%%%%%%%%%%%%%%%%%%%%%%%%%%%%%%%
\DescribeMacro{\childdocmain}
To use the package, add the commands
\begin{center}
\begin{tabular}{l}
|\input{childdoc.def}|\\
|\childdocmain{}|\\
\end{tabular}
\end{center}
at the very top of the main \LaTeX{} file,
in particular \emph{before} the |\documentclass| statement!
The argument of |\childdocmain| should be left empty
(but it must be present).

%%%%%%%%%%%%%%%%%%%%%%%%%%%%%%%%%%%%%%%%
\DescribeMacro{\childdocof}
Furthermore, add the commands
\begin{center}
\begin{tabular}{l}
|\input{childdoc.def}|\\
|\childdocof{|\textit{main}|}|\\
\end{tabular}
\end{center}
at the top of every child file \textit{child}
which is included by |\include{|\textit{child}|}|
from within the main file
(or at least for those files to be compiled individually).
The argument \textit{main} must be the filename of the main file.

There are a couple of
considerations in setting up the main and child documents:

%%%%%%%%%%%%%%%%%%%%%%%%%%%%%%%%%%%%%%%%
\paragraph{Restrictions.}

Please note the following restrictions:
\begin{itemize}
\item
|\childdocmain| must be called with one argument \textit{main}
to ensure compatibility with earlier version of the package.
It must either be empty (|\childdocmain{}|)
or precisely match the filename of the main file in which it is specified.
See \secref{sec:detection} for further information.
\item
The filename \textit{main} must be specified without the |.tex| extension.
\item
The filename \textit{main} is case sensitive
(even in case-insensitive file systems)
due to internal string comparison.
\item
The argument \textit{main} should be fully expanded, it cannot be a macro.
\item
Subdirectories and special characters should be avoided in filenames.
\item
The command |\childdocmain{|\textit{main}|}| must be followed by a whitespace.
It should not be followed immediately by another command
or by a comment mark `|%|'.
This is because the \TeX{} parser reads the token immediately following
the argument of |\childdocmain| and puts it
at the beginning of every child section;
however, a white\-space is ignored.
\end{itemize}

%%%%%%%%%%%%%%%%%%%%%%%%%%%%%%%%%%%%%%%%
\paragraph{Content of Main File.}

It is advisable to place all content in the child files included by |\include|.
Any output contained in the main file will appear in all child documents
unless suppressed manually;
it cannot be suppressed automatically by the |\includeonly| directive
and thus should normally be avoided.
A method to include some content in the main file
by means of conditional processing is described in \secref{sec:conditional}.

%%%%%%%%%%%%%%%%%%%%%%%%%%%%%%%%%%%%%%%%
\paragraph{Page Numbering.}

When only a part of the document is compiled,
the appropriate numbering of pages
(as well as other status parameters)
is determined from the |.aux| files.
The latter contain information from previous passes.
However this information needs to propagate through
all intermediate child documents.
Therefore the page numbering in child documents may well
be inconsistent until the complete document is compiled at least once.

A useful (if unconventional) way to always ensure a consistent
page numbering is to restart the numbering in each child document
and denote the pages by `\textit{child}|.|\textit{page}'
where \textit{child} represents the chapter/section number of the child file.
This can be achieved by the command
|\numberwithin{page}{|\textit{child}|}|
of the \textsf{amsmath} package
where \textit{child} can be |chapter| or |section|
depending on the chosen structuring.
Alternatively, one can modify the macro |\thepage| appropriately
and reset the counter |page| at the start of each child file.

%%%%%%%%%%%%%%%%%%%%%%%%%%%%%%%%%%%%%%%%%%%%%%%%%%%%%%%%%%%%%%%%%%%%%%%%%%%%%%%%
\subsection{Conditional Processing}
\label{sec:conditional}

The package provides a mechanism to compile different versions
of a document. To customise the versions further some conditional processing
can come in handy to distinguish which version is being compiled.
The package provides two macros to describe the compilation context:

%%%%%%%%%%%%%%%%%%%%%%%%%%%%%%%%%%%%%%%%
\DescribeMacro{\ifchilddoc}
The conditional |\ifchilddoc| distinguishes between the compilation of
child documents and the main document:
%
\begin{center}
|\ifchilddoc |\textit{child-code}| |[|\||else |\textit{main-code}]| \||fi|
\end{center}

%%%%%%%%%%%%%%%%%%%%%%%%%%%%%%%%%%%%%%%%
\DescribeMacro{\childdocname}
\DescribeMacro{\childdocjob}
The macro |\childdocname| contains the filename (without extension)
of the main or child file being processed.
Note that |\childdocjob| will always contain the name of the main file.

%%%%%%%%%%%%%%%%%%%%%%%%%%%%%%%%%%%%%%%%
\paragraph{Title Page.}

Conditional processing can be used to include a title or banner page
in the main document when proper precautions are taken.
Importantly, the code in the main file should ensure that the page counter
(as well as other status parameters which are stored in the |.aux| files)
takes the same value after the conditional processing.
Otherwise the page numbers may take divergent values
depending on which part is compiled.

For example, a title page could be declared by:
%
\begin{center}
\begin{tabular}{l}
|\ifchilddoc\||else|\\
|\addtocounter{page}{-1}|\\
\textit{code for title page}\\
|\newpage|\\
|\||fi|
\end{tabular}
\end{center}
%
A banner page for the child documents can be generated by:
%
\begin{center}
\begin{tabular}{l}
|\ifchilddoc|\\
|\addtocounter{page}{-1}|\\
\textit{code for banner page}\\
|\newpage|\\
|\||fi|
\end{tabular}
\end{center}
%
Here one could write a message such as:
\begin{center}
|This is the part \childdocname{} of \childdocjob{}.|
\end{center}

%%%%%%%%%%%%%%%%%%%%%%%%%%%%%%%%%%%%%%%%%%%%%%%%%%%%%%%%%%%%%%%%%%%%%%%%%%%%%%%%
\subsection{Flags}
\label{sec:flags}

The package makes it easy to generate different versions
of the main or child documents.
To this end compilation flags can be defined
and assigned different default values.
They will be particularly useful in conjunction
with the forwarding mechanism described in \secref{sec:forward}.

For example, it may be useful to have a flag |\version|
which can be set to |draft| or |final|.
The document source will contain some conditional code
depending on the value of |\version|.
Suppose further, the flag should default to |final| for the main file
and to |draft| for child files
which is a natural assignment for editing the document.
This is achieved by placing the following code
in the preamble of the main document
(below the |\childdocmain| directive):
%
\begin{center}
\begin{tabular}{l}
|\ifchilddoc|\\
|\providecommand{\version}{draft}|\\
|\||else|\\
|\providecommand{\version}{final}|\\
|\||fi|
\end{tabular}
\end{center}
%
The definition by |\providecommand| makes sure
that previous definitions are not overwritten.
Further statements |\providecommand{\version}{...}|
can thus be added before the above code to override it.

For the main file, one might add a line
(between |\childdocmain| and the above block)
%
\begin{center}
|%\ifchilddoc\||else\providecommand{\version}{draft}\||fi|
\end{center}
%
which can be uncommented to produce a draft version.
Likewise one can add a line to the very top of a child file
(above the |\childdocof{|\textit{main}|}| directive)
%
\begin{center}
|%\providecommand{\version}{final}|
\end{center}
%
which can be uncommented to produce the final version of this child document.

%%%%%%%%%%%%%%%%%%%%%%%%%%%%%%%%%%%%%%%%%%%%%%%%%%%%%%%%%%%%%%%%%%%%%%%%%%%%%%%%
\subsection{Forwarding}
\label{sec:forward}

Different versions of the main or child documents
using compilation flags as described in \secref{sec:flags}
can be (permanently) stored in different files
for convenient compilation, viewing and distribution.
To this end, the package defines a command
to pass on compilation to a different file:

%%%%%%%%%%%%%%%%%%%%%%%%%%%%%%%%%%%%%%%%
\DescribeMacro{\childdocforward}
The command |\childdocforward| redirects processing to
another source file:
%
\begin{center}
\begin{tabular}{l}
|\input{childdoc.def}|\\
|\childdocforward[|\textit{main}|]{|\textit{dest}|}|\\
\end{tabular}
\end{center}
%
The argument \textit{dest} is the destination file
(without extension).
It should be the main file or one of the child files.
Note that further \textsf{childdoc} directives
such as |\childdocof| and |\childdocforward|
in the indicated file will be processed in this form.
The optional argument \textit{main}
passes on directly to the main file \textit{main}
while pretending to compile the child \textit{dest}.
This form behaves as if \textit{dest}
issues |\childdocof{|\textit{main}|}| right away,
and no further \textsf{childdoc} directives will be processed.

%%%%%%%%%%%%%%%%%%%%%%%%%%%%%%%%%%%%%%%%
\DescribeMacro{\...prefix}
In the alternative form |\childdocforwardprefix|,
%
\begin{center}
\begin{tabular}{l}
|\input{childdoc.def}|\\
|\childdocforwardprefix[|\textit{main}|]{|\textit{prefix}|}{|\textit{dest}|}|
\end{tabular}
\end{center}
%
the destination file is determined by a pattern
depending on the current file:
To make this work, the current file must be called
`{\textit{prefix}\hspace{0.2em}\textit{suffix}}'
with \textit{prefix} matching precisely the argument.
Processing is then passed on to the file
`{\textit{dest}\hspace{0.2em}\textit{suffix}}'.
Surely, the same effect is achieved by
directly specifying the
argument `{\textit{dest}\hspace{0.2em}\textit{suffix}}'
in the first form.
However, that requires to set up a different file
for each child. With the alternative form of the command
all these files can have exactly the same content
which simplifies setting them up and maintaining them.

For example, the following file |draft.tex|
with a compilation flag |\version| as described in \secref{sec:flags}
compiles the main document as a draft:
%
\begin{center}
\begin{tabular}{l}
|\def\version{draft}|\\
|\input{childdoc.def}|\\
|\childdocforward{|\textit{main}|}|
\end{tabular}
\end{center}
%
Likewise, the following files |final|\textit{nn}|.tex|
compile the final version of the child document
|child|\textit{nn}|.tex|:
%
\begin{center}
\begin{tabular}{l}
|\def\version{final}|\\
|\input{childdoc.def}|\\
|\childdocforwardprefix{final}{child}|
\end{tabular}
\end{center}
%

Note that when several versions of a main file and/or of each child file
are to be generated, it may be convenient to set up a |Makefile| or
shell script to automatise the process.

%%%%%%%%%%%%%%%%%%%%%%%%%%%%%%%%%%%%%%%%%%%%%%%%%%%%%%%%%%%%%%%%%%%%%%%%%%%%%%%%
\subsection{Command Line Processing}
\label{sec:commandline}

The effect of redirection files can also be achieved by invoking
the \LaTeX{} compiler with a more elaborate command line.
Most conveniently this should be done as part
of a shell script or a |Makefile|.

When using \textsf{childdoc} in the main file, the following
command lines effectively perform a redirection
(note that depending on the shell being used,
backslashes may have to be doubled: `|\|' $\to$ `|\\|'):
%
\begin{center}
|... -jobname "|\textit{target}|" |\\|"|[\textit{flags}]%
|\input{childdoc.def}\childdocforward[|\textit{main}|]{|\textit{dest}|}"|
\end{center}
%
Here \textit{target} is the name of the output file,
\textit{main} is the name of the main file
and \textit{dest} is the name of the main or child file to be processed
(all filenames without extensions).
The optional argument \textit{main} can be omitted
if \textit{main} matches \textit{dest}.
Optionally, compilation \textit{flags} can be defined via |\def| commands.
This command line makes the \TeX{} engine believe
it is compiling the file \textit{target}
whose content is specified as the latter parameter.
The provided code then forwards the processing to
\textit{main} or \textit{dest} as described in \secref{sec:forward}.

%%%%%%%%%%%%%%%%%%%%%%%%%%%%%%%%%%%%%%%%%%%%%%%%%%%%%%%%%%%%%%%%%%%%%%%%%%%%%%%%
\subsection{Include by Input}
\label{sec:input}

Including child documents by |\include| has some restrictions by design.
Most notably, the content of a child document always occupies
its own set of pages; pages cannot be shared between child documents.
Usually, this behaviour makes perfect sense
because each child document contain an essential part of the document.
However, in some situations it may be desirable to compose
a document from a collection of parts
without having mandatory page breaks between then.
For this case, the package
provides a mechanism to include parts
by |\input| which can also be processed individually.
However, by construction this mechanism
requires manual handling of the content to be output.

%%%%%%%%%%%%%%%%%%%%%%%%%%%%%%%%%%%%%%%%
\DescribeMacro{\ifchilddocmanual}
The main file should be prepared as usual, see \secref{sec:include}.
However, the document body must make a distinction
between processing of an individual part and of the main document, e.g.:
%
\begin{center}
\begin{tabular}{l}
|\ifchilddocmanual|\\
|\input{\childdocname}|\\
|\||else|\\
\textit{document body with }|\input{|\textit{part}|}|\\
|\||fi|
\end{tabular}
\end{center}
%
The conditional |\ifchilddocmanual| is true whenever
a part to be included by |\input| is being compiled,
and the name of the part is stored in |\childdocname|.

%%%%%%%%%%%%%%%%%%%%%%%%%%%%%%%%%%%%%%%%
\DescribeMacro{\childdocby}
Each part to be included by |\input| should start with:
%
\begin{center}
\begin{tabular}{l}
|\input{childdoc.def}|\\
|\childdocby{|\textit{main}|}|\\
\end{tabular}
\end{center}
%
The directive |\childdocby| is similar to |\childdocof|
described in \secref{sec:include},
but the subsequent selection of content must be done manually.
To that end, both |\ifchilddoc| and |\ifchilddocmanual|
will be true upon processing of a part,
and the name of the part is stored in |\childdocname|.
Note that |\jobname| will be set to the filename of the current part
so that each part receives an individual |.aux| file
that does not interfere with the |.aux| file(s) of the main document.
This behaviour can be altered by the alternative form
|\childdocby[*]{|\textit{main}|}| (with a non-empty optional argument)
which uses the |.aux| file of the main document
by setting |\jobname| to \textit{main}.

%%%%%%%%%%%%%%%%%%%%%%%%%%%%%%%%%%%%%%%%%%%%%%%%%%%%%%%%%%%%%%%%%%%%%%%%%%%%%%%%
\subsection{Driver Development}
\label{sec:driver}

The \textsf{childdoc} mechanism can also be use for the development
of definition files such as \LaTeX{} styles or classes.
This case differs from the above setup with multiple parts
included by |\include| in that no |\includeonly| should be invoked.
This can be achieved by starting the include file
(before |\ProvidesPackage|) with:
%
\begin{center}
\begin{tabular}{l}
|\input{childdoc.def}|\\
|\childdocforward{|\textit{main}|}|\\
\end{tabular}
\end{center}
%
or alternatively with:
%
\begin{center}
\begin{tabular}{l}
|\input{childdoc.def}|\\
|\childdocby{|\textit{main}|}|\\
\end{tabular}
\end{center}
%
Both forms have slightly different effects as described above.
The main file is prepared as usual, see \secref{sec:include}.

%%%%%%%%%%%%%%%%%%%%%%%%%%%%%%%%%%%%%%%%%%%%%%%%%%%%%%%%%%%%%%%%%%%%%%%%%%%%%%%%
\subsection{Legacy Detection}
\label{sec:detection}

The directive |\childdocmain| in the main file can detect
whether the complete document or merely a child is to be compiled
even without using the directive |\childdocof|.
This method is deprecated because it is less robust
and there is no compelling reason to use it;
it is merely provided for backward compatibility
and it may be removed in future versions.

If the detection mechanism is to be used,
it is mandatory to correctly specify
the filename of the main file as the argument of |\childdocmain|:
%
\begin{center}
\begin{tabular}{l}
|\input{childdoc.def}|\\
|\childdocmain{|\textit{main}|}|\\
\end{tabular}
\end{center}
%
If |\jobname| does not match the argument \textit{main} of |\childdocmain|,
it is assumed that |\jobname| points to the child file to be compiled.
When using |\childdocmain| with the main file specified as argument,
it suffices to start a child file
with just |\input{|\textit{main}|}|
without loading of the package and using |\childdocof|.
If instead all processing is done
with the appropriate \textsf{childdoc} directives,
the argument of \textit{main} of |\childdocmain| can be empty.

An alternative version of the command line processing described
in \secref{sec:commandline} using the detection mechanism reads:
%
\begin{center}
|... -jobname "|\textit{target}|" "|[\textit{flags}]%
[|\def\jobname{|\textit{dest}|}|]|\input{|\textit{main}|}"|
\end{center}

%%%%%%%%%%%%%%%%%%%%%%%%%%%%%%%%%%%%%%%%%%%%%%%%%%%%%%%%%%%%%%%%%%%%%%%%%%%%%%%%
\subsection{Manual Code}
\label{sec:manual}

In case one cannot be certain whether the definitions file |childdoc.def|
is installed on the target \TeX{} distribution
and one prefers not to ship it,
it is conceivable to paste a few relevant commands into the sources.

To that end, drop all statements |\input{childdoc.def}|
and perform the replacements as outlined below.
Instead of |\childdocmain{|\textit{main}|}| add the following code
to the top of the main file:
%
\begin{center}
\begin{tabular}{l}
|\||ifdefined\childdocname\endinput\||fi\newif\ifchilddoc|\\
|\edef\childdocname{\scantokens\expandafter{\jobname\noexpand}}|\\
|\def\childdocmain{|\textit{main}|}\||ifx\childdocmain\childdocname\||else|\\
|\childdoctrue\includeonly{\childdocname}\let\jobname\childdocmain\||fi|\\
\end{tabular}
\end{center}
%
Instead of |\childdocof{|\textit{main}|}| just include the main file
at the top of each child file:
%
\begin{center}
|\input{|\textit{main}|}|
\end{center}
%
A simple redirection |\childdocforward{|\textit{dest}|}| is achieved by:
%
\begin{center}
|\def\jobname{|\textit{dest}|}\input{\jobname}|
\end{center}
%
The redirection with prefix
|\childdocforwardprefix[|\textit{prefix}|]{|\textit{dest}|}|
is accomplished by:
%
\begin{center}
\begin{tabular}{l}
|{\edef\jobname{\scantokens\expandafter{\jobname\noexpand}}|\\
|\def\redirectjob |\textit{prefix}|#1~~~{\gdef\jobname{|\textit{dest}|#1}}|\\
|\expandafter\redirectjob\jobname~~~}\input{\jobname}|
\end{tabular}
\end{center}

In an alternative approach,
child documents can be compiled by a specific command line
without additional code or specific definitions:
%
\begin{center}
|... -jobname "|\textit{target}|" "|[\textit{flags}]%
|\includeonly{|\textit{dest}|}\input{|\textit{main}|}"|
\end{center}
%

%%%%%%%%%%%%%%%%%%%%%%%%%%%%%%%%%%%%%%%%%%%%%%%%%%%%%%%%%%%%%%%%%%%%%%%%%%%%%%%%
%%%%%%%%%%%%%%%%%%%%%%%%%%%%%%%%%%%%%%%%%%%%%%%%%%%%%%%%%%%%%%%%%%%%%%%%%%%%%%%%
\section{Information}

%%%%%%%%%%%%%%%%%%%%%%%%%%%%%%%%%%%%%%%%%%%%%%%%%%%%%%%%%%%%%%%%%%%%%%%%%%%%%%%%
\subsection{Copyright}

Copyright \copyright{} 2017--2018 Niklas Beisert

This work may be distributed and/or modified under the
conditions of the \LaTeX{} Project Public License, either version 1.3
of this license or (at your option) any later version.
The latest version of this license is in
  \url{http://www.latex-project.org/lppl.txt}
and version 1.3 or later is part of all distributions of \LaTeX{}
version 2005/12/01 or later.

This work has the LPPL maintenance status `maintained'.

The Current Maintainer of this work is Niklas Beisert.

This work consists of the files |README.txt|, |childdoc.ins| and |childdoc.dtx|
as well as the derived files |childdoc.def|, |cdocsamp.tex|
with |cdocsch1.tex|, |cdocsch2.tex|, |cdocspt3.tex|, |cdocspt4.tex|,
|cdocsdrf.tex|, |cdocsfn1.tex|, |cdocsfn2.tex|
as well as |childdoc.pdf|.

%%%%%%%%%%%%%%%%%%%%%%%%%%%%%%%%%%%%%%%%%%%%%%%%%%%%%%%%%%%%%%%%%%%%%%%%%%%%%%%%
\subsection{Files and Installation}

The package consists of the files:
%
\begin{center}
\begin{tabular}{ll}
    |README.txt|   & readme file \\
    |childdoc.ins| & installation file \\
    |childdoc.dtx| & source file \\
    |childdoc.def| & definition file \\
    |cdocsamp.tex| & sample main file \\
    |cdocsch1.tex| & sample include file \\
    |cdocsch2.tex| & sample include file \\
    |cdocspt3.tex| & sample part file \\
    |cdocspt4.tex| & sample part file \\
    |cdocsdrf.tex| & sample redirection file \\
    |cdocsfn1.tex| & sample redirection file \\
    |cdocsfn2.tex| & sample redirection file \\
    |childdoc.pdf| & manual
\end{tabular}
\end{center}
%
The distribution consists of the files
|README.txt|, |childdoc.ins| and |childdoc.dtx|.
%
\begin{itemize}
\item
Run (pdf)\LaTeX{} on |childdoc.dtx|
to compile the manual |childdoc.pdf| (this file).
\item
Run \LaTeX{} on |childdoc.ins| to create the definitions file |childdoc.def|
and the sample |cdocsamp.tex| with include files
|cdocsch1.tex|, |cdocsch2.tex|, |cdocspt3.tex|, |cdocspt4.tex|,
|cdocsdrf.tex|, |cdocsfn1.tex|, |cdocsfn2.tex|.
Then copy the file |childdoc.def| to an appropriate directory of your \LaTeX{}
distribution, e.g.\ \textit{texmf-root}|/tex/latex/childdoc|.
\end{itemize}

%%%%%%%%%%%%%%%%%%%%%%%%%%%%%%%%%%%%%%%%%%%%%%%%%%%%%%%%%%%%%%%%%%%%%%%%%%%%%%%%
\subsection{Related CTAN Packages}

There are several other packages which offer a similar functionality:
%
\begin{itemize}
\item
The packages
\href{http://ctan.org/pkg/docmute}{\textsf{docmute}},
\href{http://ctan.org/pkg/includex}{\textsf{includex}} and
\href{http://ctan.org/pkg/standalone}{\textsf{standalone}}
provide commands to include only the document body of
a child file thus allowing both files to be compiled individually.
\item
The packages \href{http://ctan.org/pkg/subdocs}{\textsf{subdocs}}
and \href{http://ctan.org/pkg/subfiles}{\textsf{subfiles}}
provide structures in which the main and child documents can be
encapsulated and allowing them to be compiled individually.
The inclusion mechanism is different from the conventional |\include|.
\item
The package \href{http://ctan.org/pkg/combine}{\textsf{combine}}
is an elaborate solution to combine several documents into one.
\end{itemize}
%
See also the CTAN topic \href{http://ctan.org/topic/subdocs}{\textsf{subdocs}}
for further related packages.
The present package differs from the above solutions in that
a document structure constructed with the conventional |\include| mechanism
just needs two extra commands at the top of every file
such that all constituent files can be compiled individually.

%%%%%%%%%%%%%%%%%%%%%%%%%%%%%%%%%%%%%%%%%%%%%%%%%%%%%%%%%%%%%%%%%%%%%%%%%%%%%%%%
%\subsection{Feature Suggestions}
%
%The following is a list of features which may be useful for future
%versions of this package:
%%
%\begin{itemize}
%\item
%\ldots
%\end{itemize}

%%%%%%%%%%%%%%%%%%%%%%%%%%%%%%%%%%%%%%%%%%%%%%%%%%%%%%%%%%%%%%%%%%%%%%%%%%%%%%%%
\subsection{Revision History}

%%%%%%%%%%%%%%%%%%%%%%%%%%%%%%%%%%%%%%%%
\paragraph{v2.0:} 2018/12/30

\begin{itemize}
\item
immediate forward processing
\item
added |\childdocby| mechanism
\item
manual restructured
\end{itemize}

%%%%%%%%%%%%%%%%%%%%%%%%%%%%%%%%%%%%%%%%
\paragraph{v1.6:} 2018/01/17

\begin{itemize}
\item
application for development of include files
\item
corrections to manual
\end{itemize}

%%%%%%%%%%%%%%%%%%%%%%%%%%%%%%%%%%%%%%%%
\paragraph{v1.5:} 2017/05/21

\begin{itemize}
\item
more complete structuring introduced
\item
|\childdocof| introduced
\item
|\childdoc| renamed to |\childdocmain|
\item
|\childredirect| renamed to |\childdocforward| and |\childdocforwardprefix|
and functionality expanded
\end{itemize}

%%%%%%%%%%%%%%%%%%%%%%%%%%%%%%%%%%%%%%%%
\paragraph{v1.0:} 2017/04/27

\begin{itemize}
\item
manual and install package
\item
first version published on CTAN
\end{itemize}

%%%%%%%%%%%%%%%%%%%%%%%%%%%%%%%%%%%%%%%%
\paragraph{v0.6:} 2017/04/26

\begin{itemize}
\item
redirection mechanism added
\end{itemize}

%%%%%%%%%%%%%%%%%%%%%%%%%%%%%%%%%%%%%%%%
\paragraph{v0.5:} 2017/04/26

\begin{itemize}
\item
functionality in definition file
\end{itemize}


%%%%%%%%%%%%%%%%%%%%%%%%%%%%%%%%%%%%%%%%%%%%%%%%%%%%%%%%%%%%%%%%%%%%%%%%%%%%%%%%
%%%%%%%%%%%%%%%%%%%%%%%%%%%%%%%%%%%%%%%%%%%%%%%%%%%%%%%%%%%%%%%%%%%%%%%%%%%%%%%%
%%%%%%%%%%%%%%%%%%%%%%%%%%%%%%%%%%%%%%%%%%%%%%%%%%%%%%%%%%%%%%%%%%%%%%%%%%%%%%%%
\appendix

\settowidth\MacroIndent{\rmfamily\scriptsize 000\ }

 \DocInput{childdoc.dtx}

\end{document}
%</driver>
% \fi
%
% %%%%%%%%%%%%%%%%%%%%%%%%%%%%%%%%%%%%%%%%%%%%%%%%%%%%%%%%%%%%%%%%%%%%%%%%%%%%%%
% %%%%%%%%%%%%%%%%%%%%%%%%%%%%%%%%%%%%%%%%%%%%%%%%%%%%%%%%%%%%%%%%%%%%%%%%%%%%%%
% \section{Sample}
%\iffalse
%<*samplemain>
%\fi
%
% The following presents a sample document
% with two chapters, two parts, a title page,
% a compile flag as well as three forwarding files to set the flag.
% It consists of eight |.tex| files:
% \begin{center}
% \begin{tabular}{ll}
% |cdocsamp.tex|&main file\\
% |cdocsch1.tex|&include file for chapter 1\\
% |cdocsch2.tex|&include file for chapter 2\\
% |cdocspt3.tex|&include file for part 3\\
% |cdocspt4.tex|&include file for part 4\\
% |cdocsdrf.tex|&forwarding file for main file in draft mode\\
% |cdocsfi1.tex|&forwarding file for final version of chapter 1\\
% |cdocsfi2.tex|&forwarding file for final version of chapter 2\\
% \end{tabular}
% \end{center}
% Each of the eight files can be compiled directly by the \LaTeX{} compiler.
%
% %%%%%%%%%%%%%%%%%%%%%%%%%%%%%%%%%%%%%%
% \paragraph{Main File.}
%
% The main file is called |cdocsamp.tex|.
%
% Load the \textsf{childdoc} definitions and
% declare the filename for the main document:
%    \begin{macrocode}
\input{childdoc.def}
\childdocmain{}
%    \end{macrocode}

% Optional override for |\version| flag:
%    \begin{macrocode}
%%\ifchilddoc\else\providecommand{\version}{draft}\fi
%    \end{macrocode}

% Define the default values for the |\version| flag
% (|final| for the main file and |draft| for childs):
%    \begin{macrocode}
\ifchilddoc
\providecommand{\version}{draft}
\else
\providecommand{\version}{final}
\fi
%    \end{macrocode}

% Load the standard document class:
%    \begin{macrocode}
\documentclass[12pt]{article}
%    \end{macrocode}

% Start the document body:
%    \begin{macrocode}
\begin{document}
%    \end{macrocode}

% Declare a title page.
% Print title, part of document being processed and version flag:
%    \begin{macrocode}
\addtocounter{page}{-1}
\begin{center}
{\LARGE\bfseries{}childdoc example\par}
\vspace{1cm}
\ifchilddoc
\ifchilddocmanual part\else chapter\fi:
`\childdocname' of `\childdocjob'\par
\else
main document: `\childdocjob'\par
\fi
version: \version\par
\end{center}
\newpage
%    \end{macrocode}

% Manually include selected file,
% otherwise process as usual:
%    \begin{macrocode}
\ifchilddocmanual
\section*{part `\childdocname'}
\input{\childdocname}
\else
%    \end{macrocode}

% Include the two chapters:
%    \begin{macrocode}
\include{cdocsch1}
\include{cdocsch2}
%    \end{macrocode}

% Include the two parts unless only chapters should be displayed:
%    \begin{macrocode}
\ifchilddoc\else
\section{part three}
\input{cdocspt3}
\section{part four}
\input{cdocspt4}
\fi
%    \end{macrocode}

% Process as usual until here:
%    \begin{macrocode}
\fi
%    \end{macrocode}

% End of document body:
%    \begin{macrocode}
\end{document}
%    \end{macrocode}
%\iffalse
%</samplemain>
%\fi
%
% %%%%%%%%%%%%%%%%%%%%%%%%%%%%%%%%%%%%%%
% \paragraph{Chapter Include Files.}
%
% The include files are called |cdocsch1.tex| and |cdocsch2.tex|.
%
%\iffalse
%<*samplechap1|samplechap2>
%\fi

% Optional override for |\version| flag:
%    \begin{macrocode}
%%\providecommand{\version}{final}
%    \end{macrocode}

% Include the main document:
%    \begin{macrocode}
\input{childdoc.def}
\childdocof{cdocsamp}
%    \end{macrocode}

%\iffalse
%</samplechap1|samplechap2>
%\fi
%
%\iffalse
%<*samplechap1>
%\fi
% Some text for chapter 1:
%    \begin{macrocode}
\section{one}
some text in chapter one
%    \end{macrocode}

%\iffalse
%</samplechap1>
%\fi
% Some text for chapter 2:
%\iffalse
%<*samplechap2>
%\fi
%    \begin{macrocode}
\section{two}
more text in chapter two
%    \end{macrocode}

%\iffalse
%</samplechap2>
%\fi
%
% %%%%%%%%%%%%%%%%%%%%%%%%%%%%%%%%%%%%%%
% \paragraph{Part Include Files.}
%
% The include files are called |cdocspt3.tex| and |cdocspt4.tex|.
%
%\iffalse
%<*samplepart3|samplepart4>
%\fi

% Optional override for |\version| flag:
%    \begin{macrocode}
%%\providecommand{\version}{final}
%    \end{macrocode}

% Include the main document:
%    \begin{macrocode}
\input{childdoc.def}
\childdocby{cdocsamp}
%    \end{macrocode}

%\iffalse
%</samplepart3|samplepart4>
%\fi
%
%\iffalse
%<*samplepart3>
%\fi
% Some text for part 3:
%    \begin{macrocode}
some text in part three
%    \end{macrocode}

%\iffalse
%</samplepart3>
%\fi
% Some text for part 4:
%\iffalse
%<*samplepart4>
%\fi
%    \begin{macrocode}
more text in part four
%    \end{macrocode}

%\iffalse
%</samplepart4>
%\fi
%
% %%%%%%%%%%%%%%%%%%%%%%%%%%%%%%%%%%%%%%
% \paragraph{Forwarding for a Complete Draft.}
%
% The following forwarding file |cdocsdrf.tex|
% compiles the main document in draft mode:
%\iffalse
%<*sampledraft>
%\fi
%    \begin{macrocode}
\def\version{draft}
\input{childdoc.def}
\childdocforward{cdocsamp}
%    \end{macrocode}

%\iffalse
%</sampledraft>
%\fi
%
% %%%%%%%%%%%%%%%%%%%%%%%%%%%%%%%%%%%%%%
% \paragraph{Forwarding for Final Version of the Chapters.}
%
% The following forwarding files |cdocsfn1.tex| and |cdocsfn2.tex|
% (with identical content)
% compile the final versions of the child documents
% |cdocsch1.tex| and |cdocsch2.tex|, respectively:
%\iffalse
%<*samplefinal>
%\fi
%    \begin{macrocode}
\def\version{final}
\input{childdoc.def}
\childdocforwardprefix[cdocsamp]{cdocsfn}{cdocsch}
%    \end{macrocode}

%\iffalse
%</samplefinal>
%\fi
%
% %%%%%%%%%%%%%%%%%%%%%%%%%%%%%%%%%%%%%%
% \paragraph{Command Line Processing.}
%
% The following three command lines generate the output files
% |cdocscld|, |cdocscl1| and |cdocscl2|
% which should be identical to
% |cdocsdrf|, |cdocsch1| and |cdocsfn2|, respectively:
% \begin{center}
% \begin{tabular}{l}
% |latex -jobname cdocscld \|\\
% |  "\def\version{draft}\input{childdoc.def}\childdocforward{cdocsamp}"|\\
% |latex -jobname cdocscl1 \|\\
% |  "\input{childdoc.def}\childdocforward[cdocsamp]{cdocsch1}"|\\
% |latex -jobname cdocscl2 \|\\
% |  "\def\version{final}\input{childdoc.def}\childdocforward{cdocsch2}"|
% \end{tabular}
% \end{center}
% Note that the trailing backslash on each first line
% merely continues the input to the second line
% (for convenient cut ant paste).
% Furthermore, the command |latex| can be replaced by any
% of its alternative versions such as |pdflatex|.
%
% %%%%%%%%%%%%%%%%%%%%%%%%%%%%%%%%%%%%%%%%%%%%%%%%%%%%%%%%%%%%%%%%%%%%%%%%%%%%%%
% %%%%%%%%%%%%%%%%%%%%%%%%%%%%%%%%%%%%%%%%%%%%%%%%%%%%%%%%%%%%%%%%%%%%%%%%%%%%%%
% \section{Implementation}
%\iffalse
%<*package>
%\fi
%
% This section describes the definitions file |childdoc.def|.

% The definitions cannot be loaded using |\usepackage| or |\RequirePackage|
% which has a mechanism to prevent loading a style file more than once.
% When loading the definitions by means of |\input|
% multiple instances have to be prevented manually:
%\iffalse
%This code needs to be before the `\ProvidesFile' directive
%which is defined at the beginning of this file.
%Therefore it is also placed there and commented out here.
%</package>
%<*discard>
%\fi
%    \begin{macrocode}
\ifdefined\childdocmain\endinput\fi
%    \end{macrocode}
%\iffalse
%</discard>
%<*package>
%\fi
%
% \macro{\ifchilddoc}
% \macro{\ifchilddocmanual}
% The conditional |\ifchilddoc| tells whether a
% child (true) or main (false) document is being compiled.
% The conditional |\ifchilddocmanual| tells whether
% the |\includeonly| mechanism is used (false) or
% the selection of child files must be performed manually (true).
% The definitions initialise to false:
%    \begin{macrocode}
\newif\ifchilddoc
\newif\ifchilddocmanual
%    \end{macrocode}

% \macro{\childdocname}
% \macro{\childdocjob}
% The macro |\childdocname| stores the name of the main document
% to be compiled. The macro |\childdocjob| stores the name of
% the document on which the \LaTeX{} compiler was originally invoked.
% The content of |\jobname| cannot be compared
% to filenames specified in the source due to different catcodes.
% The following code rescans |\jobname|, stores the result
% in |\childdocname| and saves a copy in |\childdocjob|:
%    \begin{macrocode}
\edef\childdocname{\scantokens\expandafter{\jobname\noexpand}}
\let\childdocjob\childdocname
%    \end{macrocode}

% \macro{\childdocdisable}
% The macro |\childdocdisable| prevents the main file
% from being processed more than once.
% At this stage, the main document command |\childdocmain|
% is assumed to be called once again where it should do nothing.
% Any subsequent call to it should prevent
% a secondary processing of the main document
% It overwrites the forwarding commands
% |\childdocof| and |\childdocforward|
% with empty macros to prevent further inclusions of the main document:
%    \begin{macrocode}
\newcommand{\childdocdisable}
{
  \renewcommand{\childdocmain}[1]{\renewcommand{\childdocmain}[1]{\endinput}}
  \renewcommand{\childdocof}[1]{}
  \renewcommand{\childdocby}[2][]{}
  \renewcommand{\childdocforward}[2][]{}
  \renewcommand{\childdocdisable}{}
}
%    \end{macrocode}

% \macro{\childdocmain}
% The macro |\childdocmain| is to be called at the top of the main file
% with nothing or the main filename (without extension) as argument.
% First, it breaks loops.
% If the argument is not empty and does not match |\childdocname|
% (which is set by the first inclusion of |childdoc.def|),
% |\ifchilddoc| is set to true, |\includeonly| is applied to the child file
% and |\jobname| is set to the main file
% (for proper handling of |.aux| files):
%    \begin{macrocode}
\newcommand{\childdocmain}[1]
{
  \childdocdisable\childdocmain{}
  \if?#1?\else
    \begingroup
      \def\childdoctmp{#1}
      \ifx\childdoctmp\childdocname
        \def\childdoctmp{}
      \else
        \def\childdoctmp
        {
          \childdoctrue
          \includeonly{\childdocname}
          \def\childdocjob{#1}
          \def\jobname{#1}
        }
      \fi
      \expandafter
    \endgroup
    \childdoctmp
  \fi
}
%    \end{macrocode}

% \macro{\childdocof}
% The command |\childdocof| redirects
% compilation to the main file |#1|.
%    \begin{macrocode}
\newcommand{\childdocof}[1]
{
  \childdocdisable
  \childdoctrue
  \includeonly{\childdocname}
  \def\jobname{#1}
  \def\childdocjob{#1}
  \input{#1}
}
%    \end{macrocode}

% \macro{\childdocby}
% The command |\childdocby| ....
%    \begin{macrocode}
\newcommand{\childdocby}[2][]
{
  \childdocdisable
  \childdoctrue
  \childdocmanualtrue
  \if?#1?\else
    \def\jobname{#2}
  \fi
  \def\childdocjob{#2}
  \input{#2}
  \endinput
}
%    \end{macrocode}

% \macro{\childdocforward}
% The command |\childdocforward| redirects
% compilation to the main file or
% (if the optional argument is given) a child file.
% Parameters are set as if the main file
% or a child file starting with |\childdocof| was compiled.
% Then compilation is handed over to the main file:
%    \begin{macrocode}
\newcommand{\childdocforward}[2][]
{
  \begingroup
    \if?#1?
      \def\childdoctmp
      {
        \def\childdocname{#2}
        \def\childdocjob{#2}
        \def\jobname{#2}
        \input{#2}
        \endinput
      }
    \else
      \def\childdoctmp
      {
        \childdocdisable
        \def\childdocname{#2}
        \childdoctrue
        \includeonly{#2}
        \def\childdocjob{#1}
        \def\jobname{#1}
        \input{#1}
        \endinput
      }
    \fi
    \expandafter
  \endgroup
  \childdoctmp
}
%    \end{macrocode}

% \macro{\childdocforwardprefix}
% The command |\childdocforwardprefix| redirects
% compilation to the main or a child file by means of a pattern.
% The prefix |#1| in the current filename is replaced by |#2|
% and the suffix of the current filename is kept
% (it is assumed that the filename does not contain the substring `|~~~|'
% which is used as a delimiter).
% Compilation is handed over to the new file by |\childdocforward|:
%    \begin{macrocode}
\newcommand{\childdocforwardprefix}[3][]
{
  \begingroup
    \def\childdocextract #2##1~~~{\def\childdoctmp{\childdocforward[#1]{#3##1}}}
    \expandafter\childdocextract\childdocname~~~
    \expandafter
  \endgroup
  \childdoctmp
}
%    \end{macrocode}

% \macro{\childdoc}
% The deprecated macro |\childdoc| is a legacy version of |\childdocmain|:
%    \begin{macrocode}
\newcommand{\childdoc}{\childdocmain}
%    \end{macrocode}

% \macro{\childdocredirect}
% The deprecated macro |\childdocredirect| is a legacy version
% of |\childdocforward| and |\childdocforwardprefix|:
%    \begin{macrocode}
\newcommand{\childdocredirect}[2][]
{
  \begingroup
    \if?#1?
      \def\childdoctmp{\childdocforward{#2}}
    \else
      \def\childdoctmp{\childdocforwardprefix{#1}{#2}}
    \fi
    \expandafter
  \endgroup
  \childdoctmp
}
%    \end{macrocode}

%\iffalse
%</package>
%\fi
%
\endinput
|\\
|\childdocforwardprefix[|\textit{main}|]{|\textit{prefix}|}{|\textit{dest}|}|
\end{tabular}
\end{center}
%
the destination file is determined by a pattern
depending on the current file:
To make this work, the current file must be called
`{\textit{prefix}\hspace{0.2em}\textit{suffix}}'
with \textit{prefix} matching precisely the argument.
Processing is then passed on to the file
`{\textit{dest}\hspace{0.2em}\textit{suffix}}'.
Surely, the same effect is achieved by
directly specifying the
argument `{\textit{dest}\hspace{0.2em}\textit{suffix}}'
in the first form.
However, that requires to set up a different file
for each child. With the alternative form of the command
all these files can have exactly the same content
which simplifies setting them up and maintaining them.

For example, the following file |draft.tex|
with a compilation flag |\version| as described in \secref{sec:flags}
compiles the main document as a draft:
%
\begin{center}
\begin{tabular}{l}
|\def\version{draft}|\\
|% \iffalse
%
% childdoc.dtx Copyright (C) 2017-2018 Niklas Beisert
%
% This work may be distributed and/or modified under the
% conditions of the LaTeX Project Public License, either version 1.3
% of this license or (at your option) any later version.
% The latest version of this license is in
%   http://www.latex-project.org/lppl.txt
% and version 1.3 or later is part of all distributions of LaTeX
% version 2005/12/01 or later.
%
% This work has the LPPL maintenance status `maintained'.
%
% The Current Maintainer of this work is Niklas Beisert.
%
% This work consists of the files childdoc.dtx and childdoc.ins
% and the derived files childdoc.def and cdocsamp.tex with
% cdocsch1.tex, cdocsch2.tex, cdocsdrf.tex, cdocsfn1.tex, cdocsfn2.tex.
%
%<package>\ifdefined\childdocmain\endinput\fi
%<package>\ProvidesFile{childdoc.def}[2018/12/30 v2.0 child document driver]
%<samplemain>\ProvidesFile{cdocsamp.tex}[2018/12/30 v2.0 sample for childdoc]
%<*driver>
%\ProvidesFile{childdoc.drv}[2018/12/30 v2.0 childdoc reference manual file]
\PassOptionsToClass{10pt,a4paper}{article}
\documentclass{ltxdoc}

\usepackage[margin=35mm]{geometry}
\usepackage{hyperref}
\usepackage{hyperxmp}
\usepackage[usenames]{color}

\hypersetup{colorlinks=true}
\hypersetup{pdfstartview=FitH}
\hypersetup{pdfpagemode=UseNone}
\hypersetup{pdfsource={}}
\hypersetup{pdflang={en-UK}}
\hypersetup{pdfcopyright={Copyright 2017-2018 Niklas Beisert.
  This work may be distributed and/or modified under the
  conditions of the LaTeX Project Public License, either version 1.3
  of this license or (at your option) any later version.}}
\hypersetup{pdflicenseurl={http://www.latex-project.org/lppl.txt}}
\hypersetup{pdfcontactaddress={ETH Zurich, ITP, HIT K,
  Wolfgang-Pauli-Strasse 27}}
\hypersetup{pdfcontactpostcode={8093}}
\hypersetup{pdfcontactcity={Zurich}}
\hypersetup{pdfcontactcountry={Switzerland}}
\hypersetup{pdfcontactemail={nbeisert@itp.phys.ethz.ch}}
\hypersetup{pdfcontacturl={http://people.phys.ethz.ch/\xmptilde nbeisert/}}

\newcommand{\secref}[1]{\hyperref[#1]{section \ref*{#1}}}

\parskip1ex
\parindent0pt
\let\olditemize\itemize
\def\itemize{\olditemize\parskip0pt}

\begin{document}

\title{The \textsf{childdoc} Package}
\hypersetup{pdftitle={The childdoc Package}}
\author{Niklas Beisert\\[2ex]
  Institut f\"ur Theoretische Physik\\
  Eidgen\"ossische Technische Hochschule Z\"urich\\
  Wolfgang-Pauli-Strasse 27, 8093 Z\"urich, Switzerland\\[1ex]
  \href{mailto:nbeisert@itp.phys.ethz.ch}
  {\texttt{nbeisert@itp.phys.ethz.ch}}}
\hypersetup{pdfauthor={Niklas Beisert}}
\hypersetup{pdfsubject={Manual for the LaTeX2e Package childdoc}}
\date{30 December 2018, \textsf{v2.0}}
\maketitle

\begin{abstract}\noindent
\textsf{childdoc} is a \LaTeXe{} package
that enables the direct compilation
of document sections included by |\include|
to individual files.
\end{abstract}

\begingroup
\parskip0ex
\tableofcontents
\endgroup

%%%%%%%%%%%%%%%%%%%%%%%%%%%%%%%%%%%%%%%%%%%%%%%%%%%%%%%%%%%%%%%%%%%%%%%%%%%%%%%%
%%%%%%%%%%%%%%%%%%%%%%%%%%%%%%%%%%%%%%%%%%%%%%%%%%%%%%%%%%%%%%%%%%%%%%%%%%%%%%%%
\section{Introduction}

\LaTeX{} provides a mechanism to structure a large document (such as a book)
into a main file and several child files (containing the chapters)
using the |\include| command.
This mechanism is beneficial for documents
which span hundreds of pages in order to
make the source file(s) more manageable.
Moreover, compilation can be restricted to
selected child files by means of the |\includeonly| command.
The latter feature can be used to reduce the compilation time while editing
(this was significantly more useful in the earlier days of \LaTeX{})
or to generate a smaller document which is easier to navigate.
Another application of |\includeonly| is to generate
documents consisting of selected parts of the complete document.

However, there are a few drawbacks of the plain |\include| mechanism:
\begin{itemize}
\item
The child files cannot be compiled on their own,
they can only be compiled via the main file.
A naive editing environment
(such as a text editor with an option
to have the current file processed by \LaTeX)
may require one to switch to the main file before compiling;
attempting to compile the child file produces errors.
\item
The main file must be modified (each time)
to adjust the |\includeonly| command
to the present needs. This easily leaves the main file in a messy state.
\item
The generated document will always carry the filename
of the main document. This is inconvenient if
several child files are to be compiled and
to be kept for distribution.
\end{itemize}

The present package provides a simple interface
to make child files individually compilable by \LaTeX{}.
Compiling a child file then has the same effect as compiling
the main file with an |\includeonly| command
to select the appropriate child.
Moreover the generated document will carry the name of the child
rather than the main file.
This resolves all three above issues.

This feature is meant to make the editing of books,
thesis documents and lecture notes somewhat more convenient.
However, the package can also be used efficiently for
composing a series of documents (such as exercise sheets)
which are typically distributed individually.
It then assists the author in generating the individual documents
(potentially in different versions)
as well as a document containing the collected series.
Another application is in developing style files
or other kinds of included material
where compilation of the style file could redirect
to a sample or test file.

%%%%%%%%%%%%%%%%%%%%%%%%%%%%%%%%%%%%%%%%%%%%%%%%%%%%%%%%%%%%%%%%%%%%%%%%%%%%%%%%
%%%%%%%%%%%%%%%%%%%%%%%%%%%%%%%%%%%%%%%%%%%%%%%%%%%%%%%%%%%%%%%%%%%%%%%%%%%%%%%%
\section{Usage}

First of all, the package \textsf{childdoc} is \emph{not} a standard
\LaTeXe{} |.sty| style file! Therefore it needs to be invoked in
a non-standard way.

%%%%%%%%%%%%%%%%%%%%%%%%%%%%%%%%%%%%%%%%%%%%%%%%%%%%%%%%%%%%%%%%%%%%%%%%%%%%%%%%
\subsection{Included Files}
\label{sec:include}

%%%%%%%%%%%%%%%%%%%%%%%%%%%%%%%%%%%%%%%%
\DescribeMacro{\childdocmain}
To use the package, add the commands
\begin{center}
\begin{tabular}{l}
|\input{childdoc.def}|\\
|\childdocmain{}|\\
\end{tabular}
\end{center}
at the very top of the main \LaTeX{} file,
in particular \emph{before} the |\documentclass| statement!
The argument of |\childdocmain| should be left empty
(but it must be present).

%%%%%%%%%%%%%%%%%%%%%%%%%%%%%%%%%%%%%%%%
\DescribeMacro{\childdocof}
Furthermore, add the commands
\begin{center}
\begin{tabular}{l}
|\input{childdoc.def}|\\
|\childdocof{|\textit{main}|}|\\
\end{tabular}
\end{center}
at the top of every child file \textit{child}
which is included by |\include{|\textit{child}|}|
from within the main file
(or at least for those files to be compiled individually).
The argument \textit{main} must be the filename of the main file.

There are a couple of
considerations in setting up the main and child documents:

%%%%%%%%%%%%%%%%%%%%%%%%%%%%%%%%%%%%%%%%
\paragraph{Restrictions.}

Please note the following restrictions:
\begin{itemize}
\item
|\childdocmain| must be called with one argument \textit{main}
to ensure compatibility with earlier version of the package.
It must either be empty (|\childdocmain{}|)
or precisely match the filename of the main file in which it is specified.
See \secref{sec:detection} for further information.
\item
The filename \textit{main} must be specified without the |.tex| extension.
\item
The filename \textit{main} is case sensitive
(even in case-insensitive file systems)
due to internal string comparison.
\item
The argument \textit{main} should be fully expanded, it cannot be a macro.
\item
Subdirectories and special characters should be avoided in filenames.
\item
The command |\childdocmain{|\textit{main}|}| must be followed by a whitespace.
It should not be followed immediately by another command
or by a comment mark `|%|'.
This is because the \TeX{} parser reads the token immediately following
the argument of |\childdocmain| and puts it
at the beginning of every child section;
however, a white\-space is ignored.
\end{itemize}

%%%%%%%%%%%%%%%%%%%%%%%%%%%%%%%%%%%%%%%%
\paragraph{Content of Main File.}

It is advisable to place all content in the child files included by |\include|.
Any output contained in the main file will appear in all child documents
unless suppressed manually;
it cannot be suppressed automatically by the |\includeonly| directive
and thus should normally be avoided.
A method to include some content in the main file
by means of conditional processing is described in \secref{sec:conditional}.

%%%%%%%%%%%%%%%%%%%%%%%%%%%%%%%%%%%%%%%%
\paragraph{Page Numbering.}

When only a part of the document is compiled,
the appropriate numbering of pages
(as well as other status parameters)
is determined from the |.aux| files.
The latter contain information from previous passes.
However this information needs to propagate through
all intermediate child documents.
Therefore the page numbering in child documents may well
be inconsistent until the complete document is compiled at least once.

A useful (if unconventional) way to always ensure a consistent
page numbering is to restart the numbering in each child document
and denote the pages by `\textit{child}|.|\textit{page}'
where \textit{child} represents the chapter/section number of the child file.
This can be achieved by the command
|\numberwithin{page}{|\textit{child}|}|
of the \textsf{amsmath} package
where \textit{child} can be |chapter| or |section|
depending on the chosen structuring.
Alternatively, one can modify the macro |\thepage| appropriately
and reset the counter |page| at the start of each child file.

%%%%%%%%%%%%%%%%%%%%%%%%%%%%%%%%%%%%%%%%%%%%%%%%%%%%%%%%%%%%%%%%%%%%%%%%%%%%%%%%
\subsection{Conditional Processing}
\label{sec:conditional}

The package provides a mechanism to compile different versions
of a document. To customise the versions further some conditional processing
can come in handy to distinguish which version is being compiled.
The package provides two macros to describe the compilation context:

%%%%%%%%%%%%%%%%%%%%%%%%%%%%%%%%%%%%%%%%
\DescribeMacro{\ifchilddoc}
The conditional |\ifchilddoc| distinguishes between the compilation of
child documents and the main document:
%
\begin{center}
|\ifchilddoc |\textit{child-code}| |[|\||else |\textit{main-code}]| \||fi|
\end{center}

%%%%%%%%%%%%%%%%%%%%%%%%%%%%%%%%%%%%%%%%
\DescribeMacro{\childdocname}
\DescribeMacro{\childdocjob}
The macro |\childdocname| contains the filename (without extension)
of the main or child file being processed.
Note that |\childdocjob| will always contain the name of the main file.

%%%%%%%%%%%%%%%%%%%%%%%%%%%%%%%%%%%%%%%%
\paragraph{Title Page.}

Conditional processing can be used to include a title or banner page
in the main document when proper precautions are taken.
Importantly, the code in the main file should ensure that the page counter
(as well as other status parameters which are stored in the |.aux| files)
takes the same value after the conditional processing.
Otherwise the page numbers may take divergent values
depending on which part is compiled.

For example, a title page could be declared by:
%
\begin{center}
\begin{tabular}{l}
|\ifchilddoc\||else|\\
|\addtocounter{page}{-1}|\\
\textit{code for title page}\\
|\newpage|\\
|\||fi|
\end{tabular}
\end{center}
%
A banner page for the child documents can be generated by:
%
\begin{center}
\begin{tabular}{l}
|\ifchilddoc|\\
|\addtocounter{page}{-1}|\\
\textit{code for banner page}\\
|\newpage|\\
|\||fi|
\end{tabular}
\end{center}
%
Here one could write a message such as:
\begin{center}
|This is the part \childdocname{} of \childdocjob{}.|
\end{center}

%%%%%%%%%%%%%%%%%%%%%%%%%%%%%%%%%%%%%%%%%%%%%%%%%%%%%%%%%%%%%%%%%%%%%%%%%%%%%%%%
\subsection{Flags}
\label{sec:flags}

The package makes it easy to generate different versions
of the main or child documents.
To this end compilation flags can be defined
and assigned different default values.
They will be particularly useful in conjunction
with the forwarding mechanism described in \secref{sec:forward}.

For example, it may be useful to have a flag |\version|
which can be set to |draft| or |final|.
The document source will contain some conditional code
depending on the value of |\version|.
Suppose further, the flag should default to |final| for the main file
and to |draft| for child files
which is a natural assignment for editing the document.
This is achieved by placing the following code
in the preamble of the main document
(below the |\childdocmain| directive):
%
\begin{center}
\begin{tabular}{l}
|\ifchilddoc|\\
|\providecommand{\version}{draft}|\\
|\||else|\\
|\providecommand{\version}{final}|\\
|\||fi|
\end{tabular}
\end{center}
%
The definition by |\providecommand| makes sure
that previous definitions are not overwritten.
Further statements |\providecommand{\version}{...}|
can thus be added before the above code to override it.

For the main file, one might add a line
(between |\childdocmain| and the above block)
%
\begin{center}
|%\ifchilddoc\||else\providecommand{\version}{draft}\||fi|
\end{center}
%
which can be uncommented to produce a draft version.
Likewise one can add a line to the very top of a child file
(above the |\childdocof{|\textit{main}|}| directive)
%
\begin{center}
|%\providecommand{\version}{final}|
\end{center}
%
which can be uncommented to produce the final version of this child document.

%%%%%%%%%%%%%%%%%%%%%%%%%%%%%%%%%%%%%%%%%%%%%%%%%%%%%%%%%%%%%%%%%%%%%%%%%%%%%%%%
\subsection{Forwarding}
\label{sec:forward}

Different versions of the main or child documents
using compilation flags as described in \secref{sec:flags}
can be (permanently) stored in different files
for convenient compilation, viewing and distribution.
To this end, the package defines a command
to pass on compilation to a different file:

%%%%%%%%%%%%%%%%%%%%%%%%%%%%%%%%%%%%%%%%
\DescribeMacro{\childdocforward}
The command |\childdocforward| redirects processing to
another source file:
%
\begin{center}
\begin{tabular}{l}
|\input{childdoc.def}|\\
|\childdocforward[|\textit{main}|]{|\textit{dest}|}|\\
\end{tabular}
\end{center}
%
The argument \textit{dest} is the destination file
(without extension).
It should be the main file or one of the child files.
Note that further \textsf{childdoc} directives
such as |\childdocof| and |\childdocforward|
in the indicated file will be processed in this form.
The optional argument \textit{main}
passes on directly to the main file \textit{main}
while pretending to compile the child \textit{dest}.
This form behaves as if \textit{dest}
issues |\childdocof{|\textit{main}|}| right away,
and no further \textsf{childdoc} directives will be processed.

%%%%%%%%%%%%%%%%%%%%%%%%%%%%%%%%%%%%%%%%
\DescribeMacro{\...prefix}
In the alternative form |\childdocforwardprefix|,
%
\begin{center}
\begin{tabular}{l}
|\input{childdoc.def}|\\
|\childdocforwardprefix[|\textit{main}|]{|\textit{prefix}|}{|\textit{dest}|}|
\end{tabular}
\end{center}
%
the destination file is determined by a pattern
depending on the current file:
To make this work, the current file must be called
`{\textit{prefix}\hspace{0.2em}\textit{suffix}}'
with \textit{prefix} matching precisely the argument.
Processing is then passed on to the file
`{\textit{dest}\hspace{0.2em}\textit{suffix}}'.
Surely, the same effect is achieved by
directly specifying the
argument `{\textit{dest}\hspace{0.2em}\textit{suffix}}'
in the first form.
However, that requires to set up a different file
for each child. With the alternative form of the command
all these files can have exactly the same content
which simplifies setting them up and maintaining them.

For example, the following file |draft.tex|
with a compilation flag |\version| as described in \secref{sec:flags}
compiles the main document as a draft:
%
\begin{center}
\begin{tabular}{l}
|\def\version{draft}|\\
|\input{childdoc.def}|\\
|\childdocforward{|\textit{main}|}|
\end{tabular}
\end{center}
%
Likewise, the following files |final|\textit{nn}|.tex|
compile the final version of the child document
|child|\textit{nn}|.tex|:
%
\begin{center}
\begin{tabular}{l}
|\def\version{final}|\\
|\input{childdoc.def}|\\
|\childdocforwardprefix{final}{child}|
\end{tabular}
\end{center}
%

Note that when several versions of a main file and/or of each child file
are to be generated, it may be convenient to set up a |Makefile| or
shell script to automatise the process.

%%%%%%%%%%%%%%%%%%%%%%%%%%%%%%%%%%%%%%%%%%%%%%%%%%%%%%%%%%%%%%%%%%%%%%%%%%%%%%%%
\subsection{Command Line Processing}
\label{sec:commandline}

The effect of redirection files can also be achieved by invoking
the \LaTeX{} compiler with a more elaborate command line.
Most conveniently this should be done as part
of a shell script or a |Makefile|.

When using \textsf{childdoc} in the main file, the following
command lines effectively perform a redirection
(note that depending on the shell being used,
backslashes may have to be doubled: `|\|' $\to$ `|\\|'):
%
\begin{center}
|... -jobname "|\textit{target}|" |\\|"|[\textit{flags}]%
|\input{childdoc.def}\childdocforward[|\textit{main}|]{|\textit{dest}|}"|
\end{center}
%
Here \textit{target} is the name of the output file,
\textit{main} is the name of the main file
and \textit{dest} is the name of the main or child file to be processed
(all filenames without extensions).
The optional argument \textit{main} can be omitted
if \textit{main} matches \textit{dest}.
Optionally, compilation \textit{flags} can be defined via |\def| commands.
This command line makes the \TeX{} engine believe
it is compiling the file \textit{target}
whose content is specified as the latter parameter.
The provided code then forwards the processing to
\textit{main} or \textit{dest} as described in \secref{sec:forward}.

%%%%%%%%%%%%%%%%%%%%%%%%%%%%%%%%%%%%%%%%%%%%%%%%%%%%%%%%%%%%%%%%%%%%%%%%%%%%%%%%
\subsection{Include by Input}
\label{sec:input}

Including child documents by |\include| has some restrictions by design.
Most notably, the content of a child document always occupies
its own set of pages; pages cannot be shared between child documents.
Usually, this behaviour makes perfect sense
because each child document contain an essential part of the document.
However, in some situations it may be desirable to compose
a document from a collection of parts
without having mandatory page breaks between then.
For this case, the package
provides a mechanism to include parts
by |\input| which can also be processed individually.
However, by construction this mechanism
requires manual handling of the content to be output.

%%%%%%%%%%%%%%%%%%%%%%%%%%%%%%%%%%%%%%%%
\DescribeMacro{\ifchilddocmanual}
The main file should be prepared as usual, see \secref{sec:include}.
However, the document body must make a distinction
between processing of an individual part and of the main document, e.g.:
%
\begin{center}
\begin{tabular}{l}
|\ifchilddocmanual|\\
|\input{\childdocname}|\\
|\||else|\\
\textit{document body with }|\input{|\textit{part}|}|\\
|\||fi|
\end{tabular}
\end{center}
%
The conditional |\ifchilddocmanual| is true whenever
a part to be included by |\input| is being compiled,
and the name of the part is stored in |\childdocname|.

%%%%%%%%%%%%%%%%%%%%%%%%%%%%%%%%%%%%%%%%
\DescribeMacro{\childdocby}
Each part to be included by |\input| should start with:
%
\begin{center}
\begin{tabular}{l}
|\input{childdoc.def}|\\
|\childdocby{|\textit{main}|}|\\
\end{tabular}
\end{center}
%
The directive |\childdocby| is similar to |\childdocof|
described in \secref{sec:include},
but the subsequent selection of content must be done manually.
To that end, both |\ifchilddoc| and |\ifchilddocmanual|
will be true upon processing of a part,
and the name of the part is stored in |\childdocname|.
Note that |\jobname| will be set to the filename of the current part
so that each part receives an individual |.aux| file
that does not interfere with the |.aux| file(s) of the main document.
This behaviour can be altered by the alternative form
|\childdocby[*]{|\textit{main}|}| (with a non-empty optional argument)
which uses the |.aux| file of the main document
by setting |\jobname| to \textit{main}.

%%%%%%%%%%%%%%%%%%%%%%%%%%%%%%%%%%%%%%%%%%%%%%%%%%%%%%%%%%%%%%%%%%%%%%%%%%%%%%%%
\subsection{Driver Development}
\label{sec:driver}

The \textsf{childdoc} mechanism can also be use for the development
of definition files such as \LaTeX{} styles or classes.
This case differs from the above setup with multiple parts
included by |\include| in that no |\includeonly| should be invoked.
This can be achieved by starting the include file
(before |\ProvidesPackage|) with:
%
\begin{center}
\begin{tabular}{l}
|\input{childdoc.def}|\\
|\childdocforward{|\textit{main}|}|\\
\end{tabular}
\end{center}
%
or alternatively with:
%
\begin{center}
\begin{tabular}{l}
|\input{childdoc.def}|\\
|\childdocby{|\textit{main}|}|\\
\end{tabular}
\end{center}
%
Both forms have slightly different effects as described above.
The main file is prepared as usual, see \secref{sec:include}.

%%%%%%%%%%%%%%%%%%%%%%%%%%%%%%%%%%%%%%%%%%%%%%%%%%%%%%%%%%%%%%%%%%%%%%%%%%%%%%%%
\subsection{Legacy Detection}
\label{sec:detection}

The directive |\childdocmain| in the main file can detect
whether the complete document or merely a child is to be compiled
even without using the directive |\childdocof|.
This method is deprecated because it is less robust
and there is no compelling reason to use it;
it is merely provided for backward compatibility
and it may be removed in future versions.

If the detection mechanism is to be used,
it is mandatory to correctly specify
the filename of the main file as the argument of |\childdocmain|:
%
\begin{center}
\begin{tabular}{l}
|\input{childdoc.def}|\\
|\childdocmain{|\textit{main}|}|\\
\end{tabular}
\end{center}
%
If |\jobname| does not match the argument \textit{main} of |\childdocmain|,
it is assumed that |\jobname| points to the child file to be compiled.
When using |\childdocmain| with the main file specified as argument,
it suffices to start a child file
with just |\input{|\textit{main}|}|
without loading of the package and using |\childdocof|.
If instead all processing is done
with the appropriate \textsf{childdoc} directives,
the argument of \textit{main} of |\childdocmain| can be empty.

An alternative version of the command line processing described
in \secref{sec:commandline} using the detection mechanism reads:
%
\begin{center}
|... -jobname "|\textit{target}|" "|[\textit{flags}]%
[|\def\jobname{|\textit{dest}|}|]|\input{|\textit{main}|}"|
\end{center}

%%%%%%%%%%%%%%%%%%%%%%%%%%%%%%%%%%%%%%%%%%%%%%%%%%%%%%%%%%%%%%%%%%%%%%%%%%%%%%%%
\subsection{Manual Code}
\label{sec:manual}

In case one cannot be certain whether the definitions file |childdoc.def|
is installed on the target \TeX{} distribution
and one prefers not to ship it,
it is conceivable to paste a few relevant commands into the sources.

To that end, drop all statements |\input{childdoc.def}|
and perform the replacements as outlined below.
Instead of |\childdocmain{|\textit{main}|}| add the following code
to the top of the main file:
%
\begin{center}
\begin{tabular}{l}
|\||ifdefined\childdocname\endinput\||fi\newif\ifchilddoc|\\
|\edef\childdocname{\scantokens\expandafter{\jobname\noexpand}}|\\
|\def\childdocmain{|\textit{main}|}\||ifx\childdocmain\childdocname\||else|\\
|\childdoctrue\includeonly{\childdocname}\let\jobname\childdocmain\||fi|\\
\end{tabular}
\end{center}
%
Instead of |\childdocof{|\textit{main}|}| just include the main file
at the top of each child file:
%
\begin{center}
|\input{|\textit{main}|}|
\end{center}
%
A simple redirection |\childdocforward{|\textit{dest}|}| is achieved by:
%
\begin{center}
|\def\jobname{|\textit{dest}|}\input{\jobname}|
\end{center}
%
The redirection with prefix
|\childdocforwardprefix[|\textit{prefix}|]{|\textit{dest}|}|
is accomplished by:
%
\begin{center}
\begin{tabular}{l}
|{\edef\jobname{\scantokens\expandafter{\jobname\noexpand}}|\\
|\def\redirectjob |\textit{prefix}|#1~~~{\gdef\jobname{|\textit{dest}|#1}}|\\
|\expandafter\redirectjob\jobname~~~}\input{\jobname}|
\end{tabular}
\end{center}

In an alternative approach,
child documents can be compiled by a specific command line
without additional code or specific definitions:
%
\begin{center}
|... -jobname "|\textit{target}|" "|[\textit{flags}]%
|\includeonly{|\textit{dest}|}\input{|\textit{main}|}"|
\end{center}
%

%%%%%%%%%%%%%%%%%%%%%%%%%%%%%%%%%%%%%%%%%%%%%%%%%%%%%%%%%%%%%%%%%%%%%%%%%%%%%%%%
%%%%%%%%%%%%%%%%%%%%%%%%%%%%%%%%%%%%%%%%%%%%%%%%%%%%%%%%%%%%%%%%%%%%%%%%%%%%%%%%
\section{Information}

%%%%%%%%%%%%%%%%%%%%%%%%%%%%%%%%%%%%%%%%%%%%%%%%%%%%%%%%%%%%%%%%%%%%%%%%%%%%%%%%
\subsection{Copyright}

Copyright \copyright{} 2017--2018 Niklas Beisert

This work may be distributed and/or modified under the
conditions of the \LaTeX{} Project Public License, either version 1.3
of this license or (at your option) any later version.
The latest version of this license is in
  \url{http://www.latex-project.org/lppl.txt}
and version 1.3 or later is part of all distributions of \LaTeX{}
version 2005/12/01 or later.

This work has the LPPL maintenance status `maintained'.

The Current Maintainer of this work is Niklas Beisert.

This work consists of the files |README.txt|, |childdoc.ins| and |childdoc.dtx|
as well as the derived files |childdoc.def|, |cdocsamp.tex|
with |cdocsch1.tex|, |cdocsch2.tex|, |cdocspt3.tex|, |cdocspt4.tex|,
|cdocsdrf.tex|, |cdocsfn1.tex|, |cdocsfn2.tex|
as well as |childdoc.pdf|.

%%%%%%%%%%%%%%%%%%%%%%%%%%%%%%%%%%%%%%%%%%%%%%%%%%%%%%%%%%%%%%%%%%%%%%%%%%%%%%%%
\subsection{Files and Installation}

The package consists of the files:
%
\begin{center}
\begin{tabular}{ll}
    |README.txt|   & readme file \\
    |childdoc.ins| & installation file \\
    |childdoc.dtx| & source file \\
    |childdoc.def| & definition file \\
    |cdocsamp.tex| & sample main file \\
    |cdocsch1.tex| & sample include file \\
    |cdocsch2.tex| & sample include file \\
    |cdocspt3.tex| & sample part file \\
    |cdocspt4.tex| & sample part file \\
    |cdocsdrf.tex| & sample redirection file \\
    |cdocsfn1.tex| & sample redirection file \\
    |cdocsfn2.tex| & sample redirection file \\
    |childdoc.pdf| & manual
\end{tabular}
\end{center}
%
The distribution consists of the files
|README.txt|, |childdoc.ins| and |childdoc.dtx|.
%
\begin{itemize}
\item
Run (pdf)\LaTeX{} on |childdoc.dtx|
to compile the manual |childdoc.pdf| (this file).
\item
Run \LaTeX{} on |childdoc.ins| to create the definitions file |childdoc.def|
and the sample |cdocsamp.tex| with include files
|cdocsch1.tex|, |cdocsch2.tex|, |cdocspt3.tex|, |cdocspt4.tex|,
|cdocsdrf.tex|, |cdocsfn1.tex|, |cdocsfn2.tex|.
Then copy the file |childdoc.def| to an appropriate directory of your \LaTeX{}
distribution, e.g.\ \textit{texmf-root}|/tex/latex/childdoc|.
\end{itemize}

%%%%%%%%%%%%%%%%%%%%%%%%%%%%%%%%%%%%%%%%%%%%%%%%%%%%%%%%%%%%%%%%%%%%%%%%%%%%%%%%
\subsection{Related CTAN Packages}

There are several other packages which offer a similar functionality:
%
\begin{itemize}
\item
The packages
\href{http://ctan.org/pkg/docmute}{\textsf{docmute}},
\href{http://ctan.org/pkg/includex}{\textsf{includex}} and
\href{http://ctan.org/pkg/standalone}{\textsf{standalone}}
provide commands to include only the document body of
a child file thus allowing both files to be compiled individually.
\item
The packages \href{http://ctan.org/pkg/subdocs}{\textsf{subdocs}}
and \href{http://ctan.org/pkg/subfiles}{\textsf{subfiles}}
provide structures in which the main and child documents can be
encapsulated and allowing them to be compiled individually.
The inclusion mechanism is different from the conventional |\include|.
\item
The package \href{http://ctan.org/pkg/combine}{\textsf{combine}}
is an elaborate solution to combine several documents into one.
\end{itemize}
%
See also the CTAN topic \href{http://ctan.org/topic/subdocs}{\textsf{subdocs}}
for further related packages.
The present package differs from the above solutions in that
a document structure constructed with the conventional |\include| mechanism
just needs two extra commands at the top of every file
such that all constituent files can be compiled individually.

%%%%%%%%%%%%%%%%%%%%%%%%%%%%%%%%%%%%%%%%%%%%%%%%%%%%%%%%%%%%%%%%%%%%%%%%%%%%%%%%
%\subsection{Feature Suggestions}
%
%The following is a list of features which may be useful for future
%versions of this package:
%%
%\begin{itemize}
%\item
%\ldots
%\end{itemize}

%%%%%%%%%%%%%%%%%%%%%%%%%%%%%%%%%%%%%%%%%%%%%%%%%%%%%%%%%%%%%%%%%%%%%%%%%%%%%%%%
\subsection{Revision History}

%%%%%%%%%%%%%%%%%%%%%%%%%%%%%%%%%%%%%%%%
\paragraph{v2.0:} 2018/12/30

\begin{itemize}
\item
immediate forward processing
\item
added |\childdocby| mechanism
\item
manual restructured
\end{itemize}

%%%%%%%%%%%%%%%%%%%%%%%%%%%%%%%%%%%%%%%%
\paragraph{v1.6:} 2018/01/17

\begin{itemize}
\item
application for development of include files
\item
corrections to manual
\end{itemize}

%%%%%%%%%%%%%%%%%%%%%%%%%%%%%%%%%%%%%%%%
\paragraph{v1.5:} 2017/05/21

\begin{itemize}
\item
more complete structuring introduced
\item
|\childdocof| introduced
\item
|\childdoc| renamed to |\childdocmain|
\item
|\childredirect| renamed to |\childdocforward| and |\childdocforwardprefix|
and functionality expanded
\end{itemize}

%%%%%%%%%%%%%%%%%%%%%%%%%%%%%%%%%%%%%%%%
\paragraph{v1.0:} 2017/04/27

\begin{itemize}
\item
manual and install package
\item
first version published on CTAN
\end{itemize}

%%%%%%%%%%%%%%%%%%%%%%%%%%%%%%%%%%%%%%%%
\paragraph{v0.6:} 2017/04/26

\begin{itemize}
\item
redirection mechanism added
\end{itemize}

%%%%%%%%%%%%%%%%%%%%%%%%%%%%%%%%%%%%%%%%
\paragraph{v0.5:} 2017/04/26

\begin{itemize}
\item
functionality in definition file
\end{itemize}


%%%%%%%%%%%%%%%%%%%%%%%%%%%%%%%%%%%%%%%%%%%%%%%%%%%%%%%%%%%%%%%%%%%%%%%%%%%%%%%%
%%%%%%%%%%%%%%%%%%%%%%%%%%%%%%%%%%%%%%%%%%%%%%%%%%%%%%%%%%%%%%%%%%%%%%%%%%%%%%%%
%%%%%%%%%%%%%%%%%%%%%%%%%%%%%%%%%%%%%%%%%%%%%%%%%%%%%%%%%%%%%%%%%%%%%%%%%%%%%%%%
\appendix

\settowidth\MacroIndent{\rmfamily\scriptsize 000\ }

 \DocInput{childdoc.dtx}

\end{document}
%</driver>
% \fi
%
% %%%%%%%%%%%%%%%%%%%%%%%%%%%%%%%%%%%%%%%%%%%%%%%%%%%%%%%%%%%%%%%%%%%%%%%%%%%%%%
% %%%%%%%%%%%%%%%%%%%%%%%%%%%%%%%%%%%%%%%%%%%%%%%%%%%%%%%%%%%%%%%%%%%%%%%%%%%%%%
% \section{Sample}
%\iffalse
%<*samplemain>
%\fi
%
% The following presents a sample document
% with two chapters, two parts, a title page,
% a compile flag as well as three forwarding files to set the flag.
% It consists of eight |.tex| files:
% \begin{center}
% \begin{tabular}{ll}
% |cdocsamp.tex|&main file\\
% |cdocsch1.tex|&include file for chapter 1\\
% |cdocsch2.tex|&include file for chapter 2\\
% |cdocspt3.tex|&include file for part 3\\
% |cdocspt4.tex|&include file for part 4\\
% |cdocsdrf.tex|&forwarding file for main file in draft mode\\
% |cdocsfi1.tex|&forwarding file for final version of chapter 1\\
% |cdocsfi2.tex|&forwarding file for final version of chapter 2\\
% \end{tabular}
% \end{center}
% Each of the eight files can be compiled directly by the \LaTeX{} compiler.
%
% %%%%%%%%%%%%%%%%%%%%%%%%%%%%%%%%%%%%%%
% \paragraph{Main File.}
%
% The main file is called |cdocsamp.tex|.
%
% Load the \textsf{childdoc} definitions and
% declare the filename for the main document:
%    \begin{macrocode}
\input{childdoc.def}
\childdocmain{}
%    \end{macrocode}

% Optional override for |\version| flag:
%    \begin{macrocode}
%%\ifchilddoc\else\providecommand{\version}{draft}\fi
%    \end{macrocode}

% Define the default values for the |\version| flag
% (|final| for the main file and |draft| for childs):
%    \begin{macrocode}
\ifchilddoc
\providecommand{\version}{draft}
\else
\providecommand{\version}{final}
\fi
%    \end{macrocode}

% Load the standard document class:
%    \begin{macrocode}
\documentclass[12pt]{article}
%    \end{macrocode}

% Start the document body:
%    \begin{macrocode}
\begin{document}
%    \end{macrocode}

% Declare a title page.
% Print title, part of document being processed and version flag:
%    \begin{macrocode}
\addtocounter{page}{-1}
\begin{center}
{\LARGE\bfseries{}childdoc example\par}
\vspace{1cm}
\ifchilddoc
\ifchilddocmanual part\else chapter\fi:
`\childdocname' of `\childdocjob'\par
\else
main document: `\childdocjob'\par
\fi
version: \version\par
\end{center}
\newpage
%    \end{macrocode}

% Manually include selected file,
% otherwise process as usual:
%    \begin{macrocode}
\ifchilddocmanual
\section*{part `\childdocname'}
\input{\childdocname}
\else
%    \end{macrocode}

% Include the two chapters:
%    \begin{macrocode}
\include{cdocsch1}
\include{cdocsch2}
%    \end{macrocode}

% Include the two parts unless only chapters should be displayed:
%    \begin{macrocode}
\ifchilddoc\else
\section{part three}
\input{cdocspt3}
\section{part four}
\input{cdocspt4}
\fi
%    \end{macrocode}

% Process as usual until here:
%    \begin{macrocode}
\fi
%    \end{macrocode}

% End of document body:
%    \begin{macrocode}
\end{document}
%    \end{macrocode}
%\iffalse
%</samplemain>
%\fi
%
% %%%%%%%%%%%%%%%%%%%%%%%%%%%%%%%%%%%%%%
% \paragraph{Chapter Include Files.}
%
% The include files are called |cdocsch1.tex| and |cdocsch2.tex|.
%
%\iffalse
%<*samplechap1|samplechap2>
%\fi

% Optional override for |\version| flag:
%    \begin{macrocode}
%%\providecommand{\version}{final}
%    \end{macrocode}

% Include the main document:
%    \begin{macrocode}
\input{childdoc.def}
\childdocof{cdocsamp}
%    \end{macrocode}

%\iffalse
%</samplechap1|samplechap2>
%\fi
%
%\iffalse
%<*samplechap1>
%\fi
% Some text for chapter 1:
%    \begin{macrocode}
\section{one}
some text in chapter one
%    \end{macrocode}

%\iffalse
%</samplechap1>
%\fi
% Some text for chapter 2:
%\iffalse
%<*samplechap2>
%\fi
%    \begin{macrocode}
\section{two}
more text in chapter two
%    \end{macrocode}

%\iffalse
%</samplechap2>
%\fi
%
% %%%%%%%%%%%%%%%%%%%%%%%%%%%%%%%%%%%%%%
% \paragraph{Part Include Files.}
%
% The include files are called |cdocspt3.tex| and |cdocspt4.tex|.
%
%\iffalse
%<*samplepart3|samplepart4>
%\fi

% Optional override for |\version| flag:
%    \begin{macrocode}
%%\providecommand{\version}{final}
%    \end{macrocode}

% Include the main document:
%    \begin{macrocode}
\input{childdoc.def}
\childdocby{cdocsamp}
%    \end{macrocode}

%\iffalse
%</samplepart3|samplepart4>
%\fi
%
%\iffalse
%<*samplepart3>
%\fi
% Some text for part 3:
%    \begin{macrocode}
some text in part three
%    \end{macrocode}

%\iffalse
%</samplepart3>
%\fi
% Some text for part 4:
%\iffalse
%<*samplepart4>
%\fi
%    \begin{macrocode}
more text in part four
%    \end{macrocode}

%\iffalse
%</samplepart4>
%\fi
%
% %%%%%%%%%%%%%%%%%%%%%%%%%%%%%%%%%%%%%%
% \paragraph{Forwarding for a Complete Draft.}
%
% The following forwarding file |cdocsdrf.tex|
% compiles the main document in draft mode:
%\iffalse
%<*sampledraft>
%\fi
%    \begin{macrocode}
\def\version{draft}
\input{childdoc.def}
\childdocforward{cdocsamp}
%    \end{macrocode}

%\iffalse
%</sampledraft>
%\fi
%
% %%%%%%%%%%%%%%%%%%%%%%%%%%%%%%%%%%%%%%
% \paragraph{Forwarding for Final Version of the Chapters.}
%
% The following forwarding files |cdocsfn1.tex| and |cdocsfn2.tex|
% (with identical content)
% compile the final versions of the child documents
% |cdocsch1.tex| and |cdocsch2.tex|, respectively:
%\iffalse
%<*samplefinal>
%\fi
%    \begin{macrocode}
\def\version{final}
\input{childdoc.def}
\childdocforwardprefix[cdocsamp]{cdocsfn}{cdocsch}
%    \end{macrocode}

%\iffalse
%</samplefinal>
%\fi
%
% %%%%%%%%%%%%%%%%%%%%%%%%%%%%%%%%%%%%%%
% \paragraph{Command Line Processing.}
%
% The following three command lines generate the output files
% |cdocscld|, |cdocscl1| and |cdocscl2|
% which should be identical to
% |cdocsdrf|, |cdocsch1| and |cdocsfn2|, respectively:
% \begin{center}
% \begin{tabular}{l}
% |latex -jobname cdocscld \|\\
% |  "\def\version{draft}\input{childdoc.def}\childdocforward{cdocsamp}"|\\
% |latex -jobname cdocscl1 \|\\
% |  "\input{childdoc.def}\childdocforward[cdocsamp]{cdocsch1}"|\\
% |latex -jobname cdocscl2 \|\\
% |  "\def\version{final}\input{childdoc.def}\childdocforward{cdocsch2}"|
% \end{tabular}
% \end{center}
% Note that the trailing backslash on each first line
% merely continues the input to the second line
% (for convenient cut ant paste).
% Furthermore, the command |latex| can be replaced by any
% of its alternative versions such as |pdflatex|.
%
% %%%%%%%%%%%%%%%%%%%%%%%%%%%%%%%%%%%%%%%%%%%%%%%%%%%%%%%%%%%%%%%%%%%%%%%%%%%%%%
% %%%%%%%%%%%%%%%%%%%%%%%%%%%%%%%%%%%%%%%%%%%%%%%%%%%%%%%%%%%%%%%%%%%%%%%%%%%%%%
% \section{Implementation}
%\iffalse
%<*package>
%\fi
%
% This section describes the definitions file |childdoc.def|.

% The definitions cannot be loaded using |\usepackage| or |\RequirePackage|
% which has a mechanism to prevent loading a style file more than once.
% When loading the definitions by means of |\input|
% multiple instances have to be prevented manually:
%\iffalse
%This code needs to be before the `\ProvidesFile' directive
%which is defined at the beginning of this file.
%Therefore it is also placed there and commented out here.
%</package>
%<*discard>
%\fi
%    \begin{macrocode}
\ifdefined\childdocmain\endinput\fi
%    \end{macrocode}
%\iffalse
%</discard>
%<*package>
%\fi
%
% \macro{\ifchilddoc}
% \macro{\ifchilddocmanual}
% The conditional |\ifchilddoc| tells whether a
% child (true) or main (false) document is being compiled.
% The conditional |\ifchilddocmanual| tells whether
% the |\includeonly| mechanism is used (false) or
% the selection of child files must be performed manually (true).
% The definitions initialise to false:
%    \begin{macrocode}
\newif\ifchilddoc
\newif\ifchilddocmanual
%    \end{macrocode}

% \macro{\childdocname}
% \macro{\childdocjob}
% The macro |\childdocname| stores the name of the main document
% to be compiled. The macro |\childdocjob| stores the name of
% the document on which the \LaTeX{} compiler was originally invoked.
% The content of |\jobname| cannot be compared
% to filenames specified in the source due to different catcodes.
% The following code rescans |\jobname|, stores the result
% in |\childdocname| and saves a copy in |\childdocjob|:
%    \begin{macrocode}
\edef\childdocname{\scantokens\expandafter{\jobname\noexpand}}
\let\childdocjob\childdocname
%    \end{macrocode}

% \macro{\childdocdisable}
% The macro |\childdocdisable| prevents the main file
% from being processed more than once.
% At this stage, the main document command |\childdocmain|
% is assumed to be called once again where it should do nothing.
% Any subsequent call to it should prevent
% a secondary processing of the main document
% It overwrites the forwarding commands
% |\childdocof| and |\childdocforward|
% with empty macros to prevent further inclusions of the main document:
%    \begin{macrocode}
\newcommand{\childdocdisable}
{
  \renewcommand{\childdocmain}[1]{\renewcommand{\childdocmain}[1]{\endinput}}
  \renewcommand{\childdocof}[1]{}
  \renewcommand{\childdocby}[2][]{}
  \renewcommand{\childdocforward}[2][]{}
  \renewcommand{\childdocdisable}{}
}
%    \end{macrocode}

% \macro{\childdocmain}
% The macro |\childdocmain| is to be called at the top of the main file
% with nothing or the main filename (without extension) as argument.
% First, it breaks loops.
% If the argument is not empty and does not match |\childdocname|
% (which is set by the first inclusion of |childdoc.def|),
% |\ifchilddoc| is set to true, |\includeonly| is applied to the child file
% and |\jobname| is set to the main file
% (for proper handling of |.aux| files):
%    \begin{macrocode}
\newcommand{\childdocmain}[1]
{
  \childdocdisable\childdocmain{}
  \if?#1?\else
    \begingroup
      \def\childdoctmp{#1}
      \ifx\childdoctmp\childdocname
        \def\childdoctmp{}
      \else
        \def\childdoctmp
        {
          \childdoctrue
          \includeonly{\childdocname}
          \def\childdocjob{#1}
          \def\jobname{#1}
        }
      \fi
      \expandafter
    \endgroup
    \childdoctmp
  \fi
}
%    \end{macrocode}

% \macro{\childdocof}
% The command |\childdocof| redirects
% compilation to the main file |#1|.
%    \begin{macrocode}
\newcommand{\childdocof}[1]
{
  \childdocdisable
  \childdoctrue
  \includeonly{\childdocname}
  \def\jobname{#1}
  \def\childdocjob{#1}
  \input{#1}
}
%    \end{macrocode}

% \macro{\childdocby}
% The command |\childdocby| ....
%    \begin{macrocode}
\newcommand{\childdocby}[2][]
{
  \childdocdisable
  \childdoctrue
  \childdocmanualtrue
  \if?#1?\else
    \def\jobname{#2}
  \fi
  \def\childdocjob{#2}
  \input{#2}
  \endinput
}
%    \end{macrocode}

% \macro{\childdocforward}
% The command |\childdocforward| redirects
% compilation to the main file or
% (if the optional argument is given) a child file.
% Parameters are set as if the main file
% or a child file starting with |\childdocof| was compiled.
% Then compilation is handed over to the main file:
%    \begin{macrocode}
\newcommand{\childdocforward}[2][]
{
  \begingroup
    \if?#1?
      \def\childdoctmp
      {
        \def\childdocname{#2}
        \def\childdocjob{#2}
        \def\jobname{#2}
        \input{#2}
        \endinput
      }
    \else
      \def\childdoctmp
      {
        \childdocdisable
        \def\childdocname{#2}
        \childdoctrue
        \includeonly{#2}
        \def\childdocjob{#1}
        \def\jobname{#1}
        \input{#1}
        \endinput
      }
    \fi
    \expandafter
  \endgroup
  \childdoctmp
}
%    \end{macrocode}

% \macro{\childdocforwardprefix}
% The command |\childdocforwardprefix| redirects
% compilation to the main or a child file by means of a pattern.
% The prefix |#1| in the current filename is replaced by |#2|
% and the suffix of the current filename is kept
% (it is assumed that the filename does not contain the substring `|~~~|'
% which is used as a delimiter).
% Compilation is handed over to the new file by |\childdocforward|:
%    \begin{macrocode}
\newcommand{\childdocforwardprefix}[3][]
{
  \begingroup
    \def\childdocextract #2##1~~~{\def\childdoctmp{\childdocforward[#1]{#3##1}}}
    \expandafter\childdocextract\childdocname~~~
    \expandafter
  \endgroup
  \childdoctmp
}
%    \end{macrocode}

% \macro{\childdoc}
% The deprecated macro |\childdoc| is a legacy version of |\childdocmain|:
%    \begin{macrocode}
\newcommand{\childdoc}{\childdocmain}
%    \end{macrocode}

% \macro{\childdocredirect}
% The deprecated macro |\childdocredirect| is a legacy version
% of |\childdocforward| and |\childdocforwardprefix|:
%    \begin{macrocode}
\newcommand{\childdocredirect}[2][]
{
  \begingroup
    \if?#1?
      \def\childdoctmp{\childdocforward{#2}}
    \else
      \def\childdoctmp{\childdocforwardprefix{#1}{#2}}
    \fi
    \expandafter
  \endgroup
  \childdoctmp
}
%    \end{macrocode}

%\iffalse
%</package>
%\fi
%
\endinput
|\\
|\childdocforward{|\textit{main}|}|
\end{tabular}
\end{center}
%
Likewise, the following files |final|\textit{nn}|.tex|
compile the final version of the child document
|child|\textit{nn}|.tex|:
%
\begin{center}
\begin{tabular}{l}
|\def\version{final}|\\
|% \iffalse
%
% childdoc.dtx Copyright (C) 2017-2018 Niklas Beisert
%
% This work may be distributed and/or modified under the
% conditions of the LaTeX Project Public License, either version 1.3
% of this license or (at your option) any later version.
% The latest version of this license is in
%   http://www.latex-project.org/lppl.txt
% and version 1.3 or later is part of all distributions of LaTeX
% version 2005/12/01 or later.
%
% This work has the LPPL maintenance status `maintained'.
%
% The Current Maintainer of this work is Niklas Beisert.
%
% This work consists of the files childdoc.dtx and childdoc.ins
% and the derived files childdoc.def and cdocsamp.tex with
% cdocsch1.tex, cdocsch2.tex, cdocsdrf.tex, cdocsfn1.tex, cdocsfn2.tex.
%
%<package>\ifdefined\childdocmain\endinput\fi
%<package>\ProvidesFile{childdoc.def}[2018/12/30 v2.0 child document driver]
%<samplemain>\ProvidesFile{cdocsamp.tex}[2018/12/30 v2.0 sample for childdoc]
%<*driver>
%\ProvidesFile{childdoc.drv}[2018/12/30 v2.0 childdoc reference manual file]
\PassOptionsToClass{10pt,a4paper}{article}
\documentclass{ltxdoc}

\usepackage[margin=35mm]{geometry}
\usepackage{hyperref}
\usepackage{hyperxmp}
\usepackage[usenames]{color}

\hypersetup{colorlinks=true}
\hypersetup{pdfstartview=FitH}
\hypersetup{pdfpagemode=UseNone}
\hypersetup{pdfsource={}}
\hypersetup{pdflang={en-UK}}
\hypersetup{pdfcopyright={Copyright 2017-2018 Niklas Beisert.
  This work may be distributed and/or modified under the
  conditions of the LaTeX Project Public License, either version 1.3
  of this license or (at your option) any later version.}}
\hypersetup{pdflicenseurl={http://www.latex-project.org/lppl.txt}}
\hypersetup{pdfcontactaddress={ETH Zurich, ITP, HIT K,
  Wolfgang-Pauli-Strasse 27}}
\hypersetup{pdfcontactpostcode={8093}}
\hypersetup{pdfcontactcity={Zurich}}
\hypersetup{pdfcontactcountry={Switzerland}}
\hypersetup{pdfcontactemail={nbeisert@itp.phys.ethz.ch}}
\hypersetup{pdfcontacturl={http://people.phys.ethz.ch/\xmptilde nbeisert/}}

\newcommand{\secref}[1]{\hyperref[#1]{section \ref*{#1}}}

\parskip1ex
\parindent0pt
\let\olditemize\itemize
\def\itemize{\olditemize\parskip0pt}

\begin{document}

\title{The \textsf{childdoc} Package}
\hypersetup{pdftitle={The childdoc Package}}
\author{Niklas Beisert\\[2ex]
  Institut f\"ur Theoretische Physik\\
  Eidgen\"ossische Technische Hochschule Z\"urich\\
  Wolfgang-Pauli-Strasse 27, 8093 Z\"urich, Switzerland\\[1ex]
  \href{mailto:nbeisert@itp.phys.ethz.ch}
  {\texttt{nbeisert@itp.phys.ethz.ch}}}
\hypersetup{pdfauthor={Niklas Beisert}}
\hypersetup{pdfsubject={Manual for the LaTeX2e Package childdoc}}
\date{30 December 2018, \textsf{v2.0}}
\maketitle

\begin{abstract}\noindent
\textsf{childdoc} is a \LaTeXe{} package
that enables the direct compilation
of document sections included by |\include|
to individual files.
\end{abstract}

\begingroup
\parskip0ex
\tableofcontents
\endgroup

%%%%%%%%%%%%%%%%%%%%%%%%%%%%%%%%%%%%%%%%%%%%%%%%%%%%%%%%%%%%%%%%%%%%%%%%%%%%%%%%
%%%%%%%%%%%%%%%%%%%%%%%%%%%%%%%%%%%%%%%%%%%%%%%%%%%%%%%%%%%%%%%%%%%%%%%%%%%%%%%%
\section{Introduction}

\LaTeX{} provides a mechanism to structure a large document (such as a book)
into a main file and several child files (containing the chapters)
using the |\include| command.
This mechanism is beneficial for documents
which span hundreds of pages in order to
make the source file(s) more manageable.
Moreover, compilation can be restricted to
selected child files by means of the |\includeonly| command.
The latter feature can be used to reduce the compilation time while editing
(this was significantly more useful in the earlier days of \LaTeX{})
or to generate a smaller document which is easier to navigate.
Another application of |\includeonly| is to generate
documents consisting of selected parts of the complete document.

However, there are a few drawbacks of the plain |\include| mechanism:
\begin{itemize}
\item
The child files cannot be compiled on their own,
they can only be compiled via the main file.
A naive editing environment
(such as a text editor with an option
to have the current file processed by \LaTeX)
may require one to switch to the main file before compiling;
attempting to compile the child file produces errors.
\item
The main file must be modified (each time)
to adjust the |\includeonly| command
to the present needs. This easily leaves the main file in a messy state.
\item
The generated document will always carry the filename
of the main document. This is inconvenient if
several child files are to be compiled and
to be kept for distribution.
\end{itemize}

The present package provides a simple interface
to make child files individually compilable by \LaTeX{}.
Compiling a child file then has the same effect as compiling
the main file with an |\includeonly| command
to select the appropriate child.
Moreover the generated document will carry the name of the child
rather than the main file.
This resolves all three above issues.

This feature is meant to make the editing of books,
thesis documents and lecture notes somewhat more convenient.
However, the package can also be used efficiently for
composing a series of documents (such as exercise sheets)
which are typically distributed individually.
It then assists the author in generating the individual documents
(potentially in different versions)
as well as a document containing the collected series.
Another application is in developing style files
or other kinds of included material
where compilation of the style file could redirect
to a sample or test file.

%%%%%%%%%%%%%%%%%%%%%%%%%%%%%%%%%%%%%%%%%%%%%%%%%%%%%%%%%%%%%%%%%%%%%%%%%%%%%%%%
%%%%%%%%%%%%%%%%%%%%%%%%%%%%%%%%%%%%%%%%%%%%%%%%%%%%%%%%%%%%%%%%%%%%%%%%%%%%%%%%
\section{Usage}

First of all, the package \textsf{childdoc} is \emph{not} a standard
\LaTeXe{} |.sty| style file! Therefore it needs to be invoked in
a non-standard way.

%%%%%%%%%%%%%%%%%%%%%%%%%%%%%%%%%%%%%%%%%%%%%%%%%%%%%%%%%%%%%%%%%%%%%%%%%%%%%%%%
\subsection{Included Files}
\label{sec:include}

%%%%%%%%%%%%%%%%%%%%%%%%%%%%%%%%%%%%%%%%
\DescribeMacro{\childdocmain}
To use the package, add the commands
\begin{center}
\begin{tabular}{l}
|\input{childdoc.def}|\\
|\childdocmain{}|\\
\end{tabular}
\end{center}
at the very top of the main \LaTeX{} file,
in particular \emph{before} the |\documentclass| statement!
The argument of |\childdocmain| should be left empty
(but it must be present).

%%%%%%%%%%%%%%%%%%%%%%%%%%%%%%%%%%%%%%%%
\DescribeMacro{\childdocof}
Furthermore, add the commands
\begin{center}
\begin{tabular}{l}
|\input{childdoc.def}|\\
|\childdocof{|\textit{main}|}|\\
\end{tabular}
\end{center}
at the top of every child file \textit{child}
which is included by |\include{|\textit{child}|}|
from within the main file
(or at least for those files to be compiled individually).
The argument \textit{main} must be the filename of the main file.

There are a couple of
considerations in setting up the main and child documents:

%%%%%%%%%%%%%%%%%%%%%%%%%%%%%%%%%%%%%%%%
\paragraph{Restrictions.}

Please note the following restrictions:
\begin{itemize}
\item
|\childdocmain| must be called with one argument \textit{main}
to ensure compatibility with earlier version of the package.
It must either be empty (|\childdocmain{}|)
or precisely match the filename of the main file in which it is specified.
See \secref{sec:detection} for further information.
\item
The filename \textit{main} must be specified without the |.tex| extension.
\item
The filename \textit{main} is case sensitive
(even in case-insensitive file systems)
due to internal string comparison.
\item
The argument \textit{main} should be fully expanded, it cannot be a macro.
\item
Subdirectories and special characters should be avoided in filenames.
\item
The command |\childdocmain{|\textit{main}|}| must be followed by a whitespace.
It should not be followed immediately by another command
or by a comment mark `|%|'.
This is because the \TeX{} parser reads the token immediately following
the argument of |\childdocmain| and puts it
at the beginning of every child section;
however, a white\-space is ignored.
\end{itemize}

%%%%%%%%%%%%%%%%%%%%%%%%%%%%%%%%%%%%%%%%
\paragraph{Content of Main File.}

It is advisable to place all content in the child files included by |\include|.
Any output contained in the main file will appear in all child documents
unless suppressed manually;
it cannot be suppressed automatically by the |\includeonly| directive
and thus should normally be avoided.
A method to include some content in the main file
by means of conditional processing is described in \secref{sec:conditional}.

%%%%%%%%%%%%%%%%%%%%%%%%%%%%%%%%%%%%%%%%
\paragraph{Page Numbering.}

When only a part of the document is compiled,
the appropriate numbering of pages
(as well as other status parameters)
is determined from the |.aux| files.
The latter contain information from previous passes.
However this information needs to propagate through
all intermediate child documents.
Therefore the page numbering in child documents may well
be inconsistent until the complete document is compiled at least once.

A useful (if unconventional) way to always ensure a consistent
page numbering is to restart the numbering in each child document
and denote the pages by `\textit{child}|.|\textit{page}'
where \textit{child} represents the chapter/section number of the child file.
This can be achieved by the command
|\numberwithin{page}{|\textit{child}|}|
of the \textsf{amsmath} package
where \textit{child} can be |chapter| or |section|
depending on the chosen structuring.
Alternatively, one can modify the macro |\thepage| appropriately
and reset the counter |page| at the start of each child file.

%%%%%%%%%%%%%%%%%%%%%%%%%%%%%%%%%%%%%%%%%%%%%%%%%%%%%%%%%%%%%%%%%%%%%%%%%%%%%%%%
\subsection{Conditional Processing}
\label{sec:conditional}

The package provides a mechanism to compile different versions
of a document. To customise the versions further some conditional processing
can come in handy to distinguish which version is being compiled.
The package provides two macros to describe the compilation context:

%%%%%%%%%%%%%%%%%%%%%%%%%%%%%%%%%%%%%%%%
\DescribeMacro{\ifchilddoc}
The conditional |\ifchilddoc| distinguishes between the compilation of
child documents and the main document:
%
\begin{center}
|\ifchilddoc |\textit{child-code}| |[|\||else |\textit{main-code}]| \||fi|
\end{center}

%%%%%%%%%%%%%%%%%%%%%%%%%%%%%%%%%%%%%%%%
\DescribeMacro{\childdocname}
\DescribeMacro{\childdocjob}
The macro |\childdocname| contains the filename (without extension)
of the main or child file being processed.
Note that |\childdocjob| will always contain the name of the main file.

%%%%%%%%%%%%%%%%%%%%%%%%%%%%%%%%%%%%%%%%
\paragraph{Title Page.}

Conditional processing can be used to include a title or banner page
in the main document when proper precautions are taken.
Importantly, the code in the main file should ensure that the page counter
(as well as other status parameters which are stored in the |.aux| files)
takes the same value after the conditional processing.
Otherwise the page numbers may take divergent values
depending on which part is compiled.

For example, a title page could be declared by:
%
\begin{center}
\begin{tabular}{l}
|\ifchilddoc\||else|\\
|\addtocounter{page}{-1}|\\
\textit{code for title page}\\
|\newpage|\\
|\||fi|
\end{tabular}
\end{center}
%
A banner page for the child documents can be generated by:
%
\begin{center}
\begin{tabular}{l}
|\ifchilddoc|\\
|\addtocounter{page}{-1}|\\
\textit{code for banner page}\\
|\newpage|\\
|\||fi|
\end{tabular}
\end{center}
%
Here one could write a message such as:
\begin{center}
|This is the part \childdocname{} of \childdocjob{}.|
\end{center}

%%%%%%%%%%%%%%%%%%%%%%%%%%%%%%%%%%%%%%%%%%%%%%%%%%%%%%%%%%%%%%%%%%%%%%%%%%%%%%%%
\subsection{Flags}
\label{sec:flags}

The package makes it easy to generate different versions
of the main or child documents.
To this end compilation flags can be defined
and assigned different default values.
They will be particularly useful in conjunction
with the forwarding mechanism described in \secref{sec:forward}.

For example, it may be useful to have a flag |\version|
which can be set to |draft| or |final|.
The document source will contain some conditional code
depending on the value of |\version|.
Suppose further, the flag should default to |final| for the main file
and to |draft| for child files
which is a natural assignment for editing the document.
This is achieved by placing the following code
in the preamble of the main document
(below the |\childdocmain| directive):
%
\begin{center}
\begin{tabular}{l}
|\ifchilddoc|\\
|\providecommand{\version}{draft}|\\
|\||else|\\
|\providecommand{\version}{final}|\\
|\||fi|
\end{tabular}
\end{center}
%
The definition by |\providecommand| makes sure
that previous definitions are not overwritten.
Further statements |\providecommand{\version}{...}|
can thus be added before the above code to override it.

For the main file, one might add a line
(between |\childdocmain| and the above block)
%
\begin{center}
|%\ifchilddoc\||else\providecommand{\version}{draft}\||fi|
\end{center}
%
which can be uncommented to produce a draft version.
Likewise one can add a line to the very top of a child file
(above the |\childdocof{|\textit{main}|}| directive)
%
\begin{center}
|%\providecommand{\version}{final}|
\end{center}
%
which can be uncommented to produce the final version of this child document.

%%%%%%%%%%%%%%%%%%%%%%%%%%%%%%%%%%%%%%%%%%%%%%%%%%%%%%%%%%%%%%%%%%%%%%%%%%%%%%%%
\subsection{Forwarding}
\label{sec:forward}

Different versions of the main or child documents
using compilation flags as described in \secref{sec:flags}
can be (permanently) stored in different files
for convenient compilation, viewing and distribution.
To this end, the package defines a command
to pass on compilation to a different file:

%%%%%%%%%%%%%%%%%%%%%%%%%%%%%%%%%%%%%%%%
\DescribeMacro{\childdocforward}
The command |\childdocforward| redirects processing to
another source file:
%
\begin{center}
\begin{tabular}{l}
|\input{childdoc.def}|\\
|\childdocforward[|\textit{main}|]{|\textit{dest}|}|\\
\end{tabular}
\end{center}
%
The argument \textit{dest} is the destination file
(without extension).
It should be the main file or one of the child files.
Note that further \textsf{childdoc} directives
such as |\childdocof| and |\childdocforward|
in the indicated file will be processed in this form.
The optional argument \textit{main}
passes on directly to the main file \textit{main}
while pretending to compile the child \textit{dest}.
This form behaves as if \textit{dest}
issues |\childdocof{|\textit{main}|}| right away,
and no further \textsf{childdoc} directives will be processed.

%%%%%%%%%%%%%%%%%%%%%%%%%%%%%%%%%%%%%%%%
\DescribeMacro{\...prefix}
In the alternative form |\childdocforwardprefix|,
%
\begin{center}
\begin{tabular}{l}
|\input{childdoc.def}|\\
|\childdocforwardprefix[|\textit{main}|]{|\textit{prefix}|}{|\textit{dest}|}|
\end{tabular}
\end{center}
%
the destination file is determined by a pattern
depending on the current file:
To make this work, the current file must be called
`{\textit{prefix}\hspace{0.2em}\textit{suffix}}'
with \textit{prefix} matching precisely the argument.
Processing is then passed on to the file
`{\textit{dest}\hspace{0.2em}\textit{suffix}}'.
Surely, the same effect is achieved by
directly specifying the
argument `{\textit{dest}\hspace{0.2em}\textit{suffix}}'
in the first form.
However, that requires to set up a different file
for each child. With the alternative form of the command
all these files can have exactly the same content
which simplifies setting them up and maintaining them.

For example, the following file |draft.tex|
with a compilation flag |\version| as described in \secref{sec:flags}
compiles the main document as a draft:
%
\begin{center}
\begin{tabular}{l}
|\def\version{draft}|\\
|\input{childdoc.def}|\\
|\childdocforward{|\textit{main}|}|
\end{tabular}
\end{center}
%
Likewise, the following files |final|\textit{nn}|.tex|
compile the final version of the child document
|child|\textit{nn}|.tex|:
%
\begin{center}
\begin{tabular}{l}
|\def\version{final}|\\
|\input{childdoc.def}|\\
|\childdocforwardprefix{final}{child}|
\end{tabular}
\end{center}
%

Note that when several versions of a main file and/or of each child file
are to be generated, it may be convenient to set up a |Makefile| or
shell script to automatise the process.

%%%%%%%%%%%%%%%%%%%%%%%%%%%%%%%%%%%%%%%%%%%%%%%%%%%%%%%%%%%%%%%%%%%%%%%%%%%%%%%%
\subsection{Command Line Processing}
\label{sec:commandline}

The effect of redirection files can also be achieved by invoking
the \LaTeX{} compiler with a more elaborate command line.
Most conveniently this should be done as part
of a shell script or a |Makefile|.

When using \textsf{childdoc} in the main file, the following
command lines effectively perform a redirection
(note that depending on the shell being used,
backslashes may have to be doubled: `|\|' $\to$ `|\\|'):
%
\begin{center}
|... -jobname "|\textit{target}|" |\\|"|[\textit{flags}]%
|\input{childdoc.def}\childdocforward[|\textit{main}|]{|\textit{dest}|}"|
\end{center}
%
Here \textit{target} is the name of the output file,
\textit{main} is the name of the main file
and \textit{dest} is the name of the main or child file to be processed
(all filenames without extensions).
The optional argument \textit{main} can be omitted
if \textit{main} matches \textit{dest}.
Optionally, compilation \textit{flags} can be defined via |\def| commands.
This command line makes the \TeX{} engine believe
it is compiling the file \textit{target}
whose content is specified as the latter parameter.
The provided code then forwards the processing to
\textit{main} or \textit{dest} as described in \secref{sec:forward}.

%%%%%%%%%%%%%%%%%%%%%%%%%%%%%%%%%%%%%%%%%%%%%%%%%%%%%%%%%%%%%%%%%%%%%%%%%%%%%%%%
\subsection{Include by Input}
\label{sec:input}

Including child documents by |\include| has some restrictions by design.
Most notably, the content of a child document always occupies
its own set of pages; pages cannot be shared between child documents.
Usually, this behaviour makes perfect sense
because each child document contain an essential part of the document.
However, in some situations it may be desirable to compose
a document from a collection of parts
without having mandatory page breaks between then.
For this case, the package
provides a mechanism to include parts
by |\input| which can also be processed individually.
However, by construction this mechanism
requires manual handling of the content to be output.

%%%%%%%%%%%%%%%%%%%%%%%%%%%%%%%%%%%%%%%%
\DescribeMacro{\ifchilddocmanual}
The main file should be prepared as usual, see \secref{sec:include}.
However, the document body must make a distinction
between processing of an individual part and of the main document, e.g.:
%
\begin{center}
\begin{tabular}{l}
|\ifchilddocmanual|\\
|\input{\childdocname}|\\
|\||else|\\
\textit{document body with }|\input{|\textit{part}|}|\\
|\||fi|
\end{tabular}
\end{center}
%
The conditional |\ifchilddocmanual| is true whenever
a part to be included by |\input| is being compiled,
and the name of the part is stored in |\childdocname|.

%%%%%%%%%%%%%%%%%%%%%%%%%%%%%%%%%%%%%%%%
\DescribeMacro{\childdocby}
Each part to be included by |\input| should start with:
%
\begin{center}
\begin{tabular}{l}
|\input{childdoc.def}|\\
|\childdocby{|\textit{main}|}|\\
\end{tabular}
\end{center}
%
The directive |\childdocby| is similar to |\childdocof|
described in \secref{sec:include},
but the subsequent selection of content must be done manually.
To that end, both |\ifchilddoc| and |\ifchilddocmanual|
will be true upon processing of a part,
and the name of the part is stored in |\childdocname|.
Note that |\jobname| will be set to the filename of the current part
so that each part receives an individual |.aux| file
that does not interfere with the |.aux| file(s) of the main document.
This behaviour can be altered by the alternative form
|\childdocby[*]{|\textit{main}|}| (with a non-empty optional argument)
which uses the |.aux| file of the main document
by setting |\jobname| to \textit{main}.

%%%%%%%%%%%%%%%%%%%%%%%%%%%%%%%%%%%%%%%%%%%%%%%%%%%%%%%%%%%%%%%%%%%%%%%%%%%%%%%%
\subsection{Driver Development}
\label{sec:driver}

The \textsf{childdoc} mechanism can also be use for the development
of definition files such as \LaTeX{} styles or classes.
This case differs from the above setup with multiple parts
included by |\include| in that no |\includeonly| should be invoked.
This can be achieved by starting the include file
(before |\ProvidesPackage|) with:
%
\begin{center}
\begin{tabular}{l}
|\input{childdoc.def}|\\
|\childdocforward{|\textit{main}|}|\\
\end{tabular}
\end{center}
%
or alternatively with:
%
\begin{center}
\begin{tabular}{l}
|\input{childdoc.def}|\\
|\childdocby{|\textit{main}|}|\\
\end{tabular}
\end{center}
%
Both forms have slightly different effects as described above.
The main file is prepared as usual, see \secref{sec:include}.

%%%%%%%%%%%%%%%%%%%%%%%%%%%%%%%%%%%%%%%%%%%%%%%%%%%%%%%%%%%%%%%%%%%%%%%%%%%%%%%%
\subsection{Legacy Detection}
\label{sec:detection}

The directive |\childdocmain| in the main file can detect
whether the complete document or merely a child is to be compiled
even without using the directive |\childdocof|.
This method is deprecated because it is less robust
and there is no compelling reason to use it;
it is merely provided for backward compatibility
and it may be removed in future versions.

If the detection mechanism is to be used,
it is mandatory to correctly specify
the filename of the main file as the argument of |\childdocmain|:
%
\begin{center}
\begin{tabular}{l}
|\input{childdoc.def}|\\
|\childdocmain{|\textit{main}|}|\\
\end{tabular}
\end{center}
%
If |\jobname| does not match the argument \textit{main} of |\childdocmain|,
it is assumed that |\jobname| points to the child file to be compiled.
When using |\childdocmain| with the main file specified as argument,
it suffices to start a child file
with just |\input{|\textit{main}|}|
without loading of the package and using |\childdocof|.
If instead all processing is done
with the appropriate \textsf{childdoc} directives,
the argument of \textit{main} of |\childdocmain| can be empty.

An alternative version of the command line processing described
in \secref{sec:commandline} using the detection mechanism reads:
%
\begin{center}
|... -jobname "|\textit{target}|" "|[\textit{flags}]%
[|\def\jobname{|\textit{dest}|}|]|\input{|\textit{main}|}"|
\end{center}

%%%%%%%%%%%%%%%%%%%%%%%%%%%%%%%%%%%%%%%%%%%%%%%%%%%%%%%%%%%%%%%%%%%%%%%%%%%%%%%%
\subsection{Manual Code}
\label{sec:manual}

In case one cannot be certain whether the definitions file |childdoc.def|
is installed on the target \TeX{} distribution
and one prefers not to ship it,
it is conceivable to paste a few relevant commands into the sources.

To that end, drop all statements |\input{childdoc.def}|
and perform the replacements as outlined below.
Instead of |\childdocmain{|\textit{main}|}| add the following code
to the top of the main file:
%
\begin{center}
\begin{tabular}{l}
|\||ifdefined\childdocname\endinput\||fi\newif\ifchilddoc|\\
|\edef\childdocname{\scantokens\expandafter{\jobname\noexpand}}|\\
|\def\childdocmain{|\textit{main}|}\||ifx\childdocmain\childdocname\||else|\\
|\childdoctrue\includeonly{\childdocname}\let\jobname\childdocmain\||fi|\\
\end{tabular}
\end{center}
%
Instead of |\childdocof{|\textit{main}|}| just include the main file
at the top of each child file:
%
\begin{center}
|\input{|\textit{main}|}|
\end{center}
%
A simple redirection |\childdocforward{|\textit{dest}|}| is achieved by:
%
\begin{center}
|\def\jobname{|\textit{dest}|}\input{\jobname}|
\end{center}
%
The redirection with prefix
|\childdocforwardprefix[|\textit{prefix}|]{|\textit{dest}|}|
is accomplished by:
%
\begin{center}
\begin{tabular}{l}
|{\edef\jobname{\scantokens\expandafter{\jobname\noexpand}}|\\
|\def\redirectjob |\textit{prefix}|#1~~~{\gdef\jobname{|\textit{dest}|#1}}|\\
|\expandafter\redirectjob\jobname~~~}\input{\jobname}|
\end{tabular}
\end{center}

In an alternative approach,
child documents can be compiled by a specific command line
without additional code or specific definitions:
%
\begin{center}
|... -jobname "|\textit{target}|" "|[\textit{flags}]%
|\includeonly{|\textit{dest}|}\input{|\textit{main}|}"|
\end{center}
%

%%%%%%%%%%%%%%%%%%%%%%%%%%%%%%%%%%%%%%%%%%%%%%%%%%%%%%%%%%%%%%%%%%%%%%%%%%%%%%%%
%%%%%%%%%%%%%%%%%%%%%%%%%%%%%%%%%%%%%%%%%%%%%%%%%%%%%%%%%%%%%%%%%%%%%%%%%%%%%%%%
\section{Information}

%%%%%%%%%%%%%%%%%%%%%%%%%%%%%%%%%%%%%%%%%%%%%%%%%%%%%%%%%%%%%%%%%%%%%%%%%%%%%%%%
\subsection{Copyright}

Copyright \copyright{} 2017--2018 Niklas Beisert

This work may be distributed and/or modified under the
conditions of the \LaTeX{} Project Public License, either version 1.3
of this license or (at your option) any later version.
The latest version of this license is in
  \url{http://www.latex-project.org/lppl.txt}
and version 1.3 or later is part of all distributions of \LaTeX{}
version 2005/12/01 or later.

This work has the LPPL maintenance status `maintained'.

The Current Maintainer of this work is Niklas Beisert.

This work consists of the files |README.txt|, |childdoc.ins| and |childdoc.dtx|
as well as the derived files |childdoc.def|, |cdocsamp.tex|
with |cdocsch1.tex|, |cdocsch2.tex|, |cdocspt3.tex|, |cdocspt4.tex|,
|cdocsdrf.tex|, |cdocsfn1.tex|, |cdocsfn2.tex|
as well as |childdoc.pdf|.

%%%%%%%%%%%%%%%%%%%%%%%%%%%%%%%%%%%%%%%%%%%%%%%%%%%%%%%%%%%%%%%%%%%%%%%%%%%%%%%%
\subsection{Files and Installation}

The package consists of the files:
%
\begin{center}
\begin{tabular}{ll}
    |README.txt|   & readme file \\
    |childdoc.ins| & installation file \\
    |childdoc.dtx| & source file \\
    |childdoc.def| & definition file \\
    |cdocsamp.tex| & sample main file \\
    |cdocsch1.tex| & sample include file \\
    |cdocsch2.tex| & sample include file \\
    |cdocspt3.tex| & sample part file \\
    |cdocspt4.tex| & sample part file \\
    |cdocsdrf.tex| & sample redirection file \\
    |cdocsfn1.tex| & sample redirection file \\
    |cdocsfn2.tex| & sample redirection file \\
    |childdoc.pdf| & manual
\end{tabular}
\end{center}
%
The distribution consists of the files
|README.txt|, |childdoc.ins| and |childdoc.dtx|.
%
\begin{itemize}
\item
Run (pdf)\LaTeX{} on |childdoc.dtx|
to compile the manual |childdoc.pdf| (this file).
\item
Run \LaTeX{} on |childdoc.ins| to create the definitions file |childdoc.def|
and the sample |cdocsamp.tex| with include files
|cdocsch1.tex|, |cdocsch2.tex|, |cdocspt3.tex|, |cdocspt4.tex|,
|cdocsdrf.tex|, |cdocsfn1.tex|, |cdocsfn2.tex|.
Then copy the file |childdoc.def| to an appropriate directory of your \LaTeX{}
distribution, e.g.\ \textit{texmf-root}|/tex/latex/childdoc|.
\end{itemize}

%%%%%%%%%%%%%%%%%%%%%%%%%%%%%%%%%%%%%%%%%%%%%%%%%%%%%%%%%%%%%%%%%%%%%%%%%%%%%%%%
\subsection{Related CTAN Packages}

There are several other packages which offer a similar functionality:
%
\begin{itemize}
\item
The packages
\href{http://ctan.org/pkg/docmute}{\textsf{docmute}},
\href{http://ctan.org/pkg/includex}{\textsf{includex}} and
\href{http://ctan.org/pkg/standalone}{\textsf{standalone}}
provide commands to include only the document body of
a child file thus allowing both files to be compiled individually.
\item
The packages \href{http://ctan.org/pkg/subdocs}{\textsf{subdocs}}
and \href{http://ctan.org/pkg/subfiles}{\textsf{subfiles}}
provide structures in which the main and child documents can be
encapsulated and allowing them to be compiled individually.
The inclusion mechanism is different from the conventional |\include|.
\item
The package \href{http://ctan.org/pkg/combine}{\textsf{combine}}
is an elaborate solution to combine several documents into one.
\end{itemize}
%
See also the CTAN topic \href{http://ctan.org/topic/subdocs}{\textsf{subdocs}}
for further related packages.
The present package differs from the above solutions in that
a document structure constructed with the conventional |\include| mechanism
just needs two extra commands at the top of every file
such that all constituent files can be compiled individually.

%%%%%%%%%%%%%%%%%%%%%%%%%%%%%%%%%%%%%%%%%%%%%%%%%%%%%%%%%%%%%%%%%%%%%%%%%%%%%%%%
%\subsection{Feature Suggestions}
%
%The following is a list of features which may be useful for future
%versions of this package:
%%
%\begin{itemize}
%\item
%\ldots
%\end{itemize}

%%%%%%%%%%%%%%%%%%%%%%%%%%%%%%%%%%%%%%%%%%%%%%%%%%%%%%%%%%%%%%%%%%%%%%%%%%%%%%%%
\subsection{Revision History}

%%%%%%%%%%%%%%%%%%%%%%%%%%%%%%%%%%%%%%%%
\paragraph{v2.0:} 2018/12/30

\begin{itemize}
\item
immediate forward processing
\item
added |\childdocby| mechanism
\item
manual restructured
\end{itemize}

%%%%%%%%%%%%%%%%%%%%%%%%%%%%%%%%%%%%%%%%
\paragraph{v1.6:} 2018/01/17

\begin{itemize}
\item
application for development of include files
\item
corrections to manual
\end{itemize}

%%%%%%%%%%%%%%%%%%%%%%%%%%%%%%%%%%%%%%%%
\paragraph{v1.5:} 2017/05/21

\begin{itemize}
\item
more complete structuring introduced
\item
|\childdocof| introduced
\item
|\childdoc| renamed to |\childdocmain|
\item
|\childredirect| renamed to |\childdocforward| and |\childdocforwardprefix|
and functionality expanded
\end{itemize}

%%%%%%%%%%%%%%%%%%%%%%%%%%%%%%%%%%%%%%%%
\paragraph{v1.0:} 2017/04/27

\begin{itemize}
\item
manual and install package
\item
first version published on CTAN
\end{itemize}

%%%%%%%%%%%%%%%%%%%%%%%%%%%%%%%%%%%%%%%%
\paragraph{v0.6:} 2017/04/26

\begin{itemize}
\item
redirection mechanism added
\end{itemize}

%%%%%%%%%%%%%%%%%%%%%%%%%%%%%%%%%%%%%%%%
\paragraph{v0.5:} 2017/04/26

\begin{itemize}
\item
functionality in definition file
\end{itemize}


%%%%%%%%%%%%%%%%%%%%%%%%%%%%%%%%%%%%%%%%%%%%%%%%%%%%%%%%%%%%%%%%%%%%%%%%%%%%%%%%
%%%%%%%%%%%%%%%%%%%%%%%%%%%%%%%%%%%%%%%%%%%%%%%%%%%%%%%%%%%%%%%%%%%%%%%%%%%%%%%%
%%%%%%%%%%%%%%%%%%%%%%%%%%%%%%%%%%%%%%%%%%%%%%%%%%%%%%%%%%%%%%%%%%%%%%%%%%%%%%%%
\appendix

\settowidth\MacroIndent{\rmfamily\scriptsize 000\ }

 \DocInput{childdoc.dtx}

\end{document}
%</driver>
% \fi
%
% %%%%%%%%%%%%%%%%%%%%%%%%%%%%%%%%%%%%%%%%%%%%%%%%%%%%%%%%%%%%%%%%%%%%%%%%%%%%%%
% %%%%%%%%%%%%%%%%%%%%%%%%%%%%%%%%%%%%%%%%%%%%%%%%%%%%%%%%%%%%%%%%%%%%%%%%%%%%%%
% \section{Sample}
%\iffalse
%<*samplemain>
%\fi
%
% The following presents a sample document
% with two chapters, two parts, a title page,
% a compile flag as well as three forwarding files to set the flag.
% It consists of eight |.tex| files:
% \begin{center}
% \begin{tabular}{ll}
% |cdocsamp.tex|&main file\\
% |cdocsch1.tex|&include file for chapter 1\\
% |cdocsch2.tex|&include file for chapter 2\\
% |cdocspt3.tex|&include file for part 3\\
% |cdocspt4.tex|&include file for part 4\\
% |cdocsdrf.tex|&forwarding file for main file in draft mode\\
% |cdocsfi1.tex|&forwarding file for final version of chapter 1\\
% |cdocsfi2.tex|&forwarding file for final version of chapter 2\\
% \end{tabular}
% \end{center}
% Each of the eight files can be compiled directly by the \LaTeX{} compiler.
%
% %%%%%%%%%%%%%%%%%%%%%%%%%%%%%%%%%%%%%%
% \paragraph{Main File.}
%
% The main file is called |cdocsamp.tex|.
%
% Load the \textsf{childdoc} definitions and
% declare the filename for the main document:
%    \begin{macrocode}
\input{childdoc.def}
\childdocmain{}
%    \end{macrocode}

% Optional override for |\version| flag:
%    \begin{macrocode}
%%\ifchilddoc\else\providecommand{\version}{draft}\fi
%    \end{macrocode}

% Define the default values for the |\version| flag
% (|final| for the main file and |draft| for childs):
%    \begin{macrocode}
\ifchilddoc
\providecommand{\version}{draft}
\else
\providecommand{\version}{final}
\fi
%    \end{macrocode}

% Load the standard document class:
%    \begin{macrocode}
\documentclass[12pt]{article}
%    \end{macrocode}

% Start the document body:
%    \begin{macrocode}
\begin{document}
%    \end{macrocode}

% Declare a title page.
% Print title, part of document being processed and version flag:
%    \begin{macrocode}
\addtocounter{page}{-1}
\begin{center}
{\LARGE\bfseries{}childdoc example\par}
\vspace{1cm}
\ifchilddoc
\ifchilddocmanual part\else chapter\fi:
`\childdocname' of `\childdocjob'\par
\else
main document: `\childdocjob'\par
\fi
version: \version\par
\end{center}
\newpage
%    \end{macrocode}

% Manually include selected file,
% otherwise process as usual:
%    \begin{macrocode}
\ifchilddocmanual
\section*{part `\childdocname'}
\input{\childdocname}
\else
%    \end{macrocode}

% Include the two chapters:
%    \begin{macrocode}
\include{cdocsch1}
\include{cdocsch2}
%    \end{macrocode}

% Include the two parts unless only chapters should be displayed:
%    \begin{macrocode}
\ifchilddoc\else
\section{part three}
\input{cdocspt3}
\section{part four}
\input{cdocspt4}
\fi
%    \end{macrocode}

% Process as usual until here:
%    \begin{macrocode}
\fi
%    \end{macrocode}

% End of document body:
%    \begin{macrocode}
\end{document}
%    \end{macrocode}
%\iffalse
%</samplemain>
%\fi
%
% %%%%%%%%%%%%%%%%%%%%%%%%%%%%%%%%%%%%%%
% \paragraph{Chapter Include Files.}
%
% The include files are called |cdocsch1.tex| and |cdocsch2.tex|.
%
%\iffalse
%<*samplechap1|samplechap2>
%\fi

% Optional override for |\version| flag:
%    \begin{macrocode}
%%\providecommand{\version}{final}
%    \end{macrocode}

% Include the main document:
%    \begin{macrocode}
\input{childdoc.def}
\childdocof{cdocsamp}
%    \end{macrocode}

%\iffalse
%</samplechap1|samplechap2>
%\fi
%
%\iffalse
%<*samplechap1>
%\fi
% Some text for chapter 1:
%    \begin{macrocode}
\section{one}
some text in chapter one
%    \end{macrocode}

%\iffalse
%</samplechap1>
%\fi
% Some text for chapter 2:
%\iffalse
%<*samplechap2>
%\fi
%    \begin{macrocode}
\section{two}
more text in chapter two
%    \end{macrocode}

%\iffalse
%</samplechap2>
%\fi
%
% %%%%%%%%%%%%%%%%%%%%%%%%%%%%%%%%%%%%%%
% \paragraph{Part Include Files.}
%
% The include files are called |cdocspt3.tex| and |cdocspt4.tex|.
%
%\iffalse
%<*samplepart3|samplepart4>
%\fi

% Optional override for |\version| flag:
%    \begin{macrocode}
%%\providecommand{\version}{final}
%    \end{macrocode}

% Include the main document:
%    \begin{macrocode}
\input{childdoc.def}
\childdocby{cdocsamp}
%    \end{macrocode}

%\iffalse
%</samplepart3|samplepart4>
%\fi
%
%\iffalse
%<*samplepart3>
%\fi
% Some text for part 3:
%    \begin{macrocode}
some text in part three
%    \end{macrocode}

%\iffalse
%</samplepart3>
%\fi
% Some text for part 4:
%\iffalse
%<*samplepart4>
%\fi
%    \begin{macrocode}
more text in part four
%    \end{macrocode}

%\iffalse
%</samplepart4>
%\fi
%
% %%%%%%%%%%%%%%%%%%%%%%%%%%%%%%%%%%%%%%
% \paragraph{Forwarding for a Complete Draft.}
%
% The following forwarding file |cdocsdrf.tex|
% compiles the main document in draft mode:
%\iffalse
%<*sampledraft>
%\fi
%    \begin{macrocode}
\def\version{draft}
\input{childdoc.def}
\childdocforward{cdocsamp}
%    \end{macrocode}

%\iffalse
%</sampledraft>
%\fi
%
% %%%%%%%%%%%%%%%%%%%%%%%%%%%%%%%%%%%%%%
% \paragraph{Forwarding for Final Version of the Chapters.}
%
% The following forwarding files |cdocsfn1.tex| and |cdocsfn2.tex|
% (with identical content)
% compile the final versions of the child documents
% |cdocsch1.tex| and |cdocsch2.tex|, respectively:
%\iffalse
%<*samplefinal>
%\fi
%    \begin{macrocode}
\def\version{final}
\input{childdoc.def}
\childdocforwardprefix[cdocsamp]{cdocsfn}{cdocsch}
%    \end{macrocode}

%\iffalse
%</samplefinal>
%\fi
%
% %%%%%%%%%%%%%%%%%%%%%%%%%%%%%%%%%%%%%%
% \paragraph{Command Line Processing.}
%
% The following three command lines generate the output files
% |cdocscld|, |cdocscl1| and |cdocscl2|
% which should be identical to
% |cdocsdrf|, |cdocsch1| and |cdocsfn2|, respectively:
% \begin{center}
% \begin{tabular}{l}
% |latex -jobname cdocscld \|\\
% |  "\def\version{draft}\input{childdoc.def}\childdocforward{cdocsamp}"|\\
% |latex -jobname cdocscl1 \|\\
% |  "\input{childdoc.def}\childdocforward[cdocsamp]{cdocsch1}"|\\
% |latex -jobname cdocscl2 \|\\
% |  "\def\version{final}\input{childdoc.def}\childdocforward{cdocsch2}"|
% \end{tabular}
% \end{center}
% Note that the trailing backslash on each first line
% merely continues the input to the second line
% (for convenient cut ant paste).
% Furthermore, the command |latex| can be replaced by any
% of its alternative versions such as |pdflatex|.
%
% %%%%%%%%%%%%%%%%%%%%%%%%%%%%%%%%%%%%%%%%%%%%%%%%%%%%%%%%%%%%%%%%%%%%%%%%%%%%%%
% %%%%%%%%%%%%%%%%%%%%%%%%%%%%%%%%%%%%%%%%%%%%%%%%%%%%%%%%%%%%%%%%%%%%%%%%%%%%%%
% \section{Implementation}
%\iffalse
%<*package>
%\fi
%
% This section describes the definitions file |childdoc.def|.

% The definitions cannot be loaded using |\usepackage| or |\RequirePackage|
% which has a mechanism to prevent loading a style file more than once.
% When loading the definitions by means of |\input|
% multiple instances have to be prevented manually:
%\iffalse
%This code needs to be before the `\ProvidesFile' directive
%which is defined at the beginning of this file.
%Therefore it is also placed there and commented out here.
%</package>
%<*discard>
%\fi
%    \begin{macrocode}
\ifdefined\childdocmain\endinput\fi
%    \end{macrocode}
%\iffalse
%</discard>
%<*package>
%\fi
%
% \macro{\ifchilddoc}
% \macro{\ifchilddocmanual}
% The conditional |\ifchilddoc| tells whether a
% child (true) or main (false) document is being compiled.
% The conditional |\ifchilddocmanual| tells whether
% the |\includeonly| mechanism is used (false) or
% the selection of child files must be performed manually (true).
% The definitions initialise to false:
%    \begin{macrocode}
\newif\ifchilddoc
\newif\ifchilddocmanual
%    \end{macrocode}

% \macro{\childdocname}
% \macro{\childdocjob}
% The macro |\childdocname| stores the name of the main document
% to be compiled. The macro |\childdocjob| stores the name of
% the document on which the \LaTeX{} compiler was originally invoked.
% The content of |\jobname| cannot be compared
% to filenames specified in the source due to different catcodes.
% The following code rescans |\jobname|, stores the result
% in |\childdocname| and saves a copy in |\childdocjob|:
%    \begin{macrocode}
\edef\childdocname{\scantokens\expandafter{\jobname\noexpand}}
\let\childdocjob\childdocname
%    \end{macrocode}

% \macro{\childdocdisable}
% The macro |\childdocdisable| prevents the main file
% from being processed more than once.
% At this stage, the main document command |\childdocmain|
% is assumed to be called once again where it should do nothing.
% Any subsequent call to it should prevent
% a secondary processing of the main document
% It overwrites the forwarding commands
% |\childdocof| and |\childdocforward|
% with empty macros to prevent further inclusions of the main document:
%    \begin{macrocode}
\newcommand{\childdocdisable}
{
  \renewcommand{\childdocmain}[1]{\renewcommand{\childdocmain}[1]{\endinput}}
  \renewcommand{\childdocof}[1]{}
  \renewcommand{\childdocby}[2][]{}
  \renewcommand{\childdocforward}[2][]{}
  \renewcommand{\childdocdisable}{}
}
%    \end{macrocode}

% \macro{\childdocmain}
% The macro |\childdocmain| is to be called at the top of the main file
% with nothing or the main filename (without extension) as argument.
% First, it breaks loops.
% If the argument is not empty and does not match |\childdocname|
% (which is set by the first inclusion of |childdoc.def|),
% |\ifchilddoc| is set to true, |\includeonly| is applied to the child file
% and |\jobname| is set to the main file
% (for proper handling of |.aux| files):
%    \begin{macrocode}
\newcommand{\childdocmain}[1]
{
  \childdocdisable\childdocmain{}
  \if?#1?\else
    \begingroup
      \def\childdoctmp{#1}
      \ifx\childdoctmp\childdocname
        \def\childdoctmp{}
      \else
        \def\childdoctmp
        {
          \childdoctrue
          \includeonly{\childdocname}
          \def\childdocjob{#1}
          \def\jobname{#1}
        }
      \fi
      \expandafter
    \endgroup
    \childdoctmp
  \fi
}
%    \end{macrocode}

% \macro{\childdocof}
% The command |\childdocof| redirects
% compilation to the main file |#1|.
%    \begin{macrocode}
\newcommand{\childdocof}[1]
{
  \childdocdisable
  \childdoctrue
  \includeonly{\childdocname}
  \def\jobname{#1}
  \def\childdocjob{#1}
  \input{#1}
}
%    \end{macrocode}

% \macro{\childdocby}
% The command |\childdocby| ....
%    \begin{macrocode}
\newcommand{\childdocby}[2][]
{
  \childdocdisable
  \childdoctrue
  \childdocmanualtrue
  \if?#1?\else
    \def\jobname{#2}
  \fi
  \def\childdocjob{#2}
  \input{#2}
  \endinput
}
%    \end{macrocode}

% \macro{\childdocforward}
% The command |\childdocforward| redirects
% compilation to the main file or
% (if the optional argument is given) a child file.
% Parameters are set as if the main file
% or a child file starting with |\childdocof| was compiled.
% Then compilation is handed over to the main file:
%    \begin{macrocode}
\newcommand{\childdocforward}[2][]
{
  \begingroup
    \if?#1?
      \def\childdoctmp
      {
        \def\childdocname{#2}
        \def\childdocjob{#2}
        \def\jobname{#2}
        \input{#2}
        \endinput
      }
    \else
      \def\childdoctmp
      {
        \childdocdisable
        \def\childdocname{#2}
        \childdoctrue
        \includeonly{#2}
        \def\childdocjob{#1}
        \def\jobname{#1}
        \input{#1}
        \endinput
      }
    \fi
    \expandafter
  \endgroup
  \childdoctmp
}
%    \end{macrocode}

% \macro{\childdocforwardprefix}
% The command |\childdocforwardprefix| redirects
% compilation to the main or a child file by means of a pattern.
% The prefix |#1| in the current filename is replaced by |#2|
% and the suffix of the current filename is kept
% (it is assumed that the filename does not contain the substring `|~~~|'
% which is used as a delimiter).
% Compilation is handed over to the new file by |\childdocforward|:
%    \begin{macrocode}
\newcommand{\childdocforwardprefix}[3][]
{
  \begingroup
    \def\childdocextract #2##1~~~{\def\childdoctmp{\childdocforward[#1]{#3##1}}}
    \expandafter\childdocextract\childdocname~~~
    \expandafter
  \endgroup
  \childdoctmp
}
%    \end{macrocode}

% \macro{\childdoc}
% The deprecated macro |\childdoc| is a legacy version of |\childdocmain|:
%    \begin{macrocode}
\newcommand{\childdoc}{\childdocmain}
%    \end{macrocode}

% \macro{\childdocredirect}
% The deprecated macro |\childdocredirect| is a legacy version
% of |\childdocforward| and |\childdocforwardprefix|:
%    \begin{macrocode}
\newcommand{\childdocredirect}[2][]
{
  \begingroup
    \if?#1?
      \def\childdoctmp{\childdocforward{#2}}
    \else
      \def\childdoctmp{\childdocforwardprefix{#1}{#2}}
    \fi
    \expandafter
  \endgroup
  \childdoctmp
}
%    \end{macrocode}

%\iffalse
%</package>
%\fi
%
\endinput
|\\
|\childdocforwardprefix{final}{child}|
\end{tabular}
\end{center}
%

Note that when several versions of a main file and/or of each child file
are to be generated, it may be convenient to set up a |Makefile| or
shell script to automatise the process.

%%%%%%%%%%%%%%%%%%%%%%%%%%%%%%%%%%%%%%%%%%%%%%%%%%%%%%%%%%%%%%%%%%%%%%%%%%%%%%%%
\subsection{Command Line Processing}
\label{sec:commandline}

The effect of redirection files can also be achieved by invoking
the \LaTeX{} compiler with a more elaborate command line.
Most conveniently this should be done as part
of a shell script or a |Makefile|.

When using \textsf{childdoc} in the main file, the following
command lines effectively perform a redirection
(note that depending on the shell being used,
backslashes may have to be doubled: `|\|' $\to$ `|\\|'):
%
\begin{center}
|... -jobname "|\textit{target}|" |\\|"|[\textit{flags}]%
|% \iffalse
%
% childdoc.dtx Copyright (C) 2017-2018 Niklas Beisert
%
% This work may be distributed and/or modified under the
% conditions of the LaTeX Project Public License, either version 1.3
% of this license or (at your option) any later version.
% The latest version of this license is in
%   http://www.latex-project.org/lppl.txt
% and version 1.3 or later is part of all distributions of LaTeX
% version 2005/12/01 or later.
%
% This work has the LPPL maintenance status `maintained'.
%
% The Current Maintainer of this work is Niklas Beisert.
%
% This work consists of the files childdoc.dtx and childdoc.ins
% and the derived files childdoc.def and cdocsamp.tex with
% cdocsch1.tex, cdocsch2.tex, cdocsdrf.tex, cdocsfn1.tex, cdocsfn2.tex.
%
%<package>\ifdefined\childdocmain\endinput\fi
%<package>\ProvidesFile{childdoc.def}[2018/12/30 v2.0 child document driver]
%<samplemain>\ProvidesFile{cdocsamp.tex}[2018/12/30 v2.0 sample for childdoc]
%<*driver>
%\ProvidesFile{childdoc.drv}[2018/12/30 v2.0 childdoc reference manual file]
\PassOptionsToClass{10pt,a4paper}{article}
\documentclass{ltxdoc}

\usepackage[margin=35mm]{geometry}
\usepackage{hyperref}
\usepackage{hyperxmp}
\usepackage[usenames]{color}

\hypersetup{colorlinks=true}
\hypersetup{pdfstartview=FitH}
\hypersetup{pdfpagemode=UseNone}
\hypersetup{pdfsource={}}
\hypersetup{pdflang={en-UK}}
\hypersetup{pdfcopyright={Copyright 2017-2018 Niklas Beisert.
  This work may be distributed and/or modified under the
  conditions of the LaTeX Project Public License, either version 1.3
  of this license or (at your option) any later version.}}
\hypersetup{pdflicenseurl={http://www.latex-project.org/lppl.txt}}
\hypersetup{pdfcontactaddress={ETH Zurich, ITP, HIT K,
  Wolfgang-Pauli-Strasse 27}}
\hypersetup{pdfcontactpostcode={8093}}
\hypersetup{pdfcontactcity={Zurich}}
\hypersetup{pdfcontactcountry={Switzerland}}
\hypersetup{pdfcontactemail={nbeisert@itp.phys.ethz.ch}}
\hypersetup{pdfcontacturl={http://people.phys.ethz.ch/\xmptilde nbeisert/}}

\newcommand{\secref}[1]{\hyperref[#1]{section \ref*{#1}}}

\parskip1ex
\parindent0pt
\let\olditemize\itemize
\def\itemize{\olditemize\parskip0pt}

\begin{document}

\title{The \textsf{childdoc} Package}
\hypersetup{pdftitle={The childdoc Package}}
\author{Niklas Beisert\\[2ex]
  Institut f\"ur Theoretische Physik\\
  Eidgen\"ossische Technische Hochschule Z\"urich\\
  Wolfgang-Pauli-Strasse 27, 8093 Z\"urich, Switzerland\\[1ex]
  \href{mailto:nbeisert@itp.phys.ethz.ch}
  {\texttt{nbeisert@itp.phys.ethz.ch}}}
\hypersetup{pdfauthor={Niklas Beisert}}
\hypersetup{pdfsubject={Manual for the LaTeX2e Package childdoc}}
\date{30 December 2018, \textsf{v2.0}}
\maketitle

\begin{abstract}\noindent
\textsf{childdoc} is a \LaTeXe{} package
that enables the direct compilation
of document sections included by |\include|
to individual files.
\end{abstract}

\begingroup
\parskip0ex
\tableofcontents
\endgroup

%%%%%%%%%%%%%%%%%%%%%%%%%%%%%%%%%%%%%%%%%%%%%%%%%%%%%%%%%%%%%%%%%%%%%%%%%%%%%%%%
%%%%%%%%%%%%%%%%%%%%%%%%%%%%%%%%%%%%%%%%%%%%%%%%%%%%%%%%%%%%%%%%%%%%%%%%%%%%%%%%
\section{Introduction}

\LaTeX{} provides a mechanism to structure a large document (such as a book)
into a main file and several child files (containing the chapters)
using the |\include| command.
This mechanism is beneficial for documents
which span hundreds of pages in order to
make the source file(s) more manageable.
Moreover, compilation can be restricted to
selected child files by means of the |\includeonly| command.
The latter feature can be used to reduce the compilation time while editing
(this was significantly more useful in the earlier days of \LaTeX{})
or to generate a smaller document which is easier to navigate.
Another application of |\includeonly| is to generate
documents consisting of selected parts of the complete document.

However, there are a few drawbacks of the plain |\include| mechanism:
\begin{itemize}
\item
The child files cannot be compiled on their own,
they can only be compiled via the main file.
A naive editing environment
(such as a text editor with an option
to have the current file processed by \LaTeX)
may require one to switch to the main file before compiling;
attempting to compile the child file produces errors.
\item
The main file must be modified (each time)
to adjust the |\includeonly| command
to the present needs. This easily leaves the main file in a messy state.
\item
The generated document will always carry the filename
of the main document. This is inconvenient if
several child files are to be compiled and
to be kept for distribution.
\end{itemize}

The present package provides a simple interface
to make child files individually compilable by \LaTeX{}.
Compiling a child file then has the same effect as compiling
the main file with an |\includeonly| command
to select the appropriate child.
Moreover the generated document will carry the name of the child
rather than the main file.
This resolves all three above issues.

This feature is meant to make the editing of books,
thesis documents and lecture notes somewhat more convenient.
However, the package can also be used efficiently for
composing a series of documents (such as exercise sheets)
which are typically distributed individually.
It then assists the author in generating the individual documents
(potentially in different versions)
as well as a document containing the collected series.
Another application is in developing style files
or other kinds of included material
where compilation of the style file could redirect
to a sample or test file.

%%%%%%%%%%%%%%%%%%%%%%%%%%%%%%%%%%%%%%%%%%%%%%%%%%%%%%%%%%%%%%%%%%%%%%%%%%%%%%%%
%%%%%%%%%%%%%%%%%%%%%%%%%%%%%%%%%%%%%%%%%%%%%%%%%%%%%%%%%%%%%%%%%%%%%%%%%%%%%%%%
\section{Usage}

First of all, the package \textsf{childdoc} is \emph{not} a standard
\LaTeXe{} |.sty| style file! Therefore it needs to be invoked in
a non-standard way.

%%%%%%%%%%%%%%%%%%%%%%%%%%%%%%%%%%%%%%%%%%%%%%%%%%%%%%%%%%%%%%%%%%%%%%%%%%%%%%%%
\subsection{Included Files}
\label{sec:include}

%%%%%%%%%%%%%%%%%%%%%%%%%%%%%%%%%%%%%%%%
\DescribeMacro{\childdocmain}
To use the package, add the commands
\begin{center}
\begin{tabular}{l}
|\input{childdoc.def}|\\
|\childdocmain{}|\\
\end{tabular}
\end{center}
at the very top of the main \LaTeX{} file,
in particular \emph{before} the |\documentclass| statement!
The argument of |\childdocmain| should be left empty
(but it must be present).

%%%%%%%%%%%%%%%%%%%%%%%%%%%%%%%%%%%%%%%%
\DescribeMacro{\childdocof}
Furthermore, add the commands
\begin{center}
\begin{tabular}{l}
|\input{childdoc.def}|\\
|\childdocof{|\textit{main}|}|\\
\end{tabular}
\end{center}
at the top of every child file \textit{child}
which is included by |\include{|\textit{child}|}|
from within the main file
(or at least for those files to be compiled individually).
The argument \textit{main} must be the filename of the main file.

There are a couple of
considerations in setting up the main and child documents:

%%%%%%%%%%%%%%%%%%%%%%%%%%%%%%%%%%%%%%%%
\paragraph{Restrictions.}

Please note the following restrictions:
\begin{itemize}
\item
|\childdocmain| must be called with one argument \textit{main}
to ensure compatibility with earlier version of the package.
It must either be empty (|\childdocmain{}|)
or precisely match the filename of the main file in which it is specified.
See \secref{sec:detection} for further information.
\item
The filename \textit{main} must be specified without the |.tex| extension.
\item
The filename \textit{main} is case sensitive
(even in case-insensitive file systems)
due to internal string comparison.
\item
The argument \textit{main} should be fully expanded, it cannot be a macro.
\item
Subdirectories and special characters should be avoided in filenames.
\item
The command |\childdocmain{|\textit{main}|}| must be followed by a whitespace.
It should not be followed immediately by another command
or by a comment mark `|%|'.
This is because the \TeX{} parser reads the token immediately following
the argument of |\childdocmain| and puts it
at the beginning of every child section;
however, a white\-space is ignored.
\end{itemize}

%%%%%%%%%%%%%%%%%%%%%%%%%%%%%%%%%%%%%%%%
\paragraph{Content of Main File.}

It is advisable to place all content in the child files included by |\include|.
Any output contained in the main file will appear in all child documents
unless suppressed manually;
it cannot be suppressed automatically by the |\includeonly| directive
and thus should normally be avoided.
A method to include some content in the main file
by means of conditional processing is described in \secref{sec:conditional}.

%%%%%%%%%%%%%%%%%%%%%%%%%%%%%%%%%%%%%%%%
\paragraph{Page Numbering.}

When only a part of the document is compiled,
the appropriate numbering of pages
(as well as other status parameters)
is determined from the |.aux| files.
The latter contain information from previous passes.
However this information needs to propagate through
all intermediate child documents.
Therefore the page numbering in child documents may well
be inconsistent until the complete document is compiled at least once.

A useful (if unconventional) way to always ensure a consistent
page numbering is to restart the numbering in each child document
and denote the pages by `\textit{child}|.|\textit{page}'
where \textit{child} represents the chapter/section number of the child file.
This can be achieved by the command
|\numberwithin{page}{|\textit{child}|}|
of the \textsf{amsmath} package
where \textit{child} can be |chapter| or |section|
depending on the chosen structuring.
Alternatively, one can modify the macro |\thepage| appropriately
and reset the counter |page| at the start of each child file.

%%%%%%%%%%%%%%%%%%%%%%%%%%%%%%%%%%%%%%%%%%%%%%%%%%%%%%%%%%%%%%%%%%%%%%%%%%%%%%%%
\subsection{Conditional Processing}
\label{sec:conditional}

The package provides a mechanism to compile different versions
of a document. To customise the versions further some conditional processing
can come in handy to distinguish which version is being compiled.
The package provides two macros to describe the compilation context:

%%%%%%%%%%%%%%%%%%%%%%%%%%%%%%%%%%%%%%%%
\DescribeMacro{\ifchilddoc}
The conditional |\ifchilddoc| distinguishes between the compilation of
child documents and the main document:
%
\begin{center}
|\ifchilddoc |\textit{child-code}| |[|\||else |\textit{main-code}]| \||fi|
\end{center}

%%%%%%%%%%%%%%%%%%%%%%%%%%%%%%%%%%%%%%%%
\DescribeMacro{\childdocname}
\DescribeMacro{\childdocjob}
The macro |\childdocname| contains the filename (without extension)
of the main or child file being processed.
Note that |\childdocjob| will always contain the name of the main file.

%%%%%%%%%%%%%%%%%%%%%%%%%%%%%%%%%%%%%%%%
\paragraph{Title Page.}

Conditional processing can be used to include a title or banner page
in the main document when proper precautions are taken.
Importantly, the code in the main file should ensure that the page counter
(as well as other status parameters which are stored in the |.aux| files)
takes the same value after the conditional processing.
Otherwise the page numbers may take divergent values
depending on which part is compiled.

For example, a title page could be declared by:
%
\begin{center}
\begin{tabular}{l}
|\ifchilddoc\||else|\\
|\addtocounter{page}{-1}|\\
\textit{code for title page}\\
|\newpage|\\
|\||fi|
\end{tabular}
\end{center}
%
A banner page for the child documents can be generated by:
%
\begin{center}
\begin{tabular}{l}
|\ifchilddoc|\\
|\addtocounter{page}{-1}|\\
\textit{code for banner page}\\
|\newpage|\\
|\||fi|
\end{tabular}
\end{center}
%
Here one could write a message such as:
\begin{center}
|This is the part \childdocname{} of \childdocjob{}.|
\end{center}

%%%%%%%%%%%%%%%%%%%%%%%%%%%%%%%%%%%%%%%%%%%%%%%%%%%%%%%%%%%%%%%%%%%%%%%%%%%%%%%%
\subsection{Flags}
\label{sec:flags}

The package makes it easy to generate different versions
of the main or child documents.
To this end compilation flags can be defined
and assigned different default values.
They will be particularly useful in conjunction
with the forwarding mechanism described in \secref{sec:forward}.

For example, it may be useful to have a flag |\version|
which can be set to |draft| or |final|.
The document source will contain some conditional code
depending on the value of |\version|.
Suppose further, the flag should default to |final| for the main file
and to |draft| for child files
which is a natural assignment for editing the document.
This is achieved by placing the following code
in the preamble of the main document
(below the |\childdocmain| directive):
%
\begin{center}
\begin{tabular}{l}
|\ifchilddoc|\\
|\providecommand{\version}{draft}|\\
|\||else|\\
|\providecommand{\version}{final}|\\
|\||fi|
\end{tabular}
\end{center}
%
The definition by |\providecommand| makes sure
that previous definitions are not overwritten.
Further statements |\providecommand{\version}{...}|
can thus be added before the above code to override it.

For the main file, one might add a line
(between |\childdocmain| and the above block)
%
\begin{center}
|%\ifchilddoc\||else\providecommand{\version}{draft}\||fi|
\end{center}
%
which can be uncommented to produce a draft version.
Likewise one can add a line to the very top of a child file
(above the |\childdocof{|\textit{main}|}| directive)
%
\begin{center}
|%\providecommand{\version}{final}|
\end{center}
%
which can be uncommented to produce the final version of this child document.

%%%%%%%%%%%%%%%%%%%%%%%%%%%%%%%%%%%%%%%%%%%%%%%%%%%%%%%%%%%%%%%%%%%%%%%%%%%%%%%%
\subsection{Forwarding}
\label{sec:forward}

Different versions of the main or child documents
using compilation flags as described in \secref{sec:flags}
can be (permanently) stored in different files
for convenient compilation, viewing and distribution.
To this end, the package defines a command
to pass on compilation to a different file:

%%%%%%%%%%%%%%%%%%%%%%%%%%%%%%%%%%%%%%%%
\DescribeMacro{\childdocforward}
The command |\childdocforward| redirects processing to
another source file:
%
\begin{center}
\begin{tabular}{l}
|\input{childdoc.def}|\\
|\childdocforward[|\textit{main}|]{|\textit{dest}|}|\\
\end{tabular}
\end{center}
%
The argument \textit{dest} is the destination file
(without extension).
It should be the main file or one of the child files.
Note that further \textsf{childdoc} directives
such as |\childdocof| and |\childdocforward|
in the indicated file will be processed in this form.
The optional argument \textit{main}
passes on directly to the main file \textit{main}
while pretending to compile the child \textit{dest}.
This form behaves as if \textit{dest}
issues |\childdocof{|\textit{main}|}| right away,
and no further \textsf{childdoc} directives will be processed.

%%%%%%%%%%%%%%%%%%%%%%%%%%%%%%%%%%%%%%%%
\DescribeMacro{\...prefix}
In the alternative form |\childdocforwardprefix|,
%
\begin{center}
\begin{tabular}{l}
|\input{childdoc.def}|\\
|\childdocforwardprefix[|\textit{main}|]{|\textit{prefix}|}{|\textit{dest}|}|
\end{tabular}
\end{center}
%
the destination file is determined by a pattern
depending on the current file:
To make this work, the current file must be called
`{\textit{prefix}\hspace{0.2em}\textit{suffix}}'
with \textit{prefix} matching precisely the argument.
Processing is then passed on to the file
`{\textit{dest}\hspace{0.2em}\textit{suffix}}'.
Surely, the same effect is achieved by
directly specifying the
argument `{\textit{dest}\hspace{0.2em}\textit{suffix}}'
in the first form.
However, that requires to set up a different file
for each child. With the alternative form of the command
all these files can have exactly the same content
which simplifies setting them up and maintaining them.

For example, the following file |draft.tex|
with a compilation flag |\version| as described in \secref{sec:flags}
compiles the main document as a draft:
%
\begin{center}
\begin{tabular}{l}
|\def\version{draft}|\\
|\input{childdoc.def}|\\
|\childdocforward{|\textit{main}|}|
\end{tabular}
\end{center}
%
Likewise, the following files |final|\textit{nn}|.tex|
compile the final version of the child document
|child|\textit{nn}|.tex|:
%
\begin{center}
\begin{tabular}{l}
|\def\version{final}|\\
|\input{childdoc.def}|\\
|\childdocforwardprefix{final}{child}|
\end{tabular}
\end{center}
%

Note that when several versions of a main file and/or of each child file
are to be generated, it may be convenient to set up a |Makefile| or
shell script to automatise the process.

%%%%%%%%%%%%%%%%%%%%%%%%%%%%%%%%%%%%%%%%%%%%%%%%%%%%%%%%%%%%%%%%%%%%%%%%%%%%%%%%
\subsection{Command Line Processing}
\label{sec:commandline}

The effect of redirection files can also be achieved by invoking
the \LaTeX{} compiler with a more elaborate command line.
Most conveniently this should be done as part
of a shell script or a |Makefile|.

When using \textsf{childdoc} in the main file, the following
command lines effectively perform a redirection
(note that depending on the shell being used,
backslashes may have to be doubled: `|\|' $\to$ `|\\|'):
%
\begin{center}
|... -jobname "|\textit{target}|" |\\|"|[\textit{flags}]%
|\input{childdoc.def}\childdocforward[|\textit{main}|]{|\textit{dest}|}"|
\end{center}
%
Here \textit{target} is the name of the output file,
\textit{main} is the name of the main file
and \textit{dest} is the name of the main or child file to be processed
(all filenames without extensions).
The optional argument \textit{main} can be omitted
if \textit{main} matches \textit{dest}.
Optionally, compilation \textit{flags} can be defined via |\def| commands.
This command line makes the \TeX{} engine believe
it is compiling the file \textit{target}
whose content is specified as the latter parameter.
The provided code then forwards the processing to
\textit{main} or \textit{dest} as described in \secref{sec:forward}.

%%%%%%%%%%%%%%%%%%%%%%%%%%%%%%%%%%%%%%%%%%%%%%%%%%%%%%%%%%%%%%%%%%%%%%%%%%%%%%%%
\subsection{Include by Input}
\label{sec:input}

Including child documents by |\include| has some restrictions by design.
Most notably, the content of a child document always occupies
its own set of pages; pages cannot be shared between child documents.
Usually, this behaviour makes perfect sense
because each child document contain an essential part of the document.
However, in some situations it may be desirable to compose
a document from a collection of parts
without having mandatory page breaks between then.
For this case, the package
provides a mechanism to include parts
by |\input| which can also be processed individually.
However, by construction this mechanism
requires manual handling of the content to be output.

%%%%%%%%%%%%%%%%%%%%%%%%%%%%%%%%%%%%%%%%
\DescribeMacro{\ifchilddocmanual}
The main file should be prepared as usual, see \secref{sec:include}.
However, the document body must make a distinction
between processing of an individual part and of the main document, e.g.:
%
\begin{center}
\begin{tabular}{l}
|\ifchilddocmanual|\\
|\input{\childdocname}|\\
|\||else|\\
\textit{document body with }|\input{|\textit{part}|}|\\
|\||fi|
\end{tabular}
\end{center}
%
The conditional |\ifchilddocmanual| is true whenever
a part to be included by |\input| is being compiled,
and the name of the part is stored in |\childdocname|.

%%%%%%%%%%%%%%%%%%%%%%%%%%%%%%%%%%%%%%%%
\DescribeMacro{\childdocby}
Each part to be included by |\input| should start with:
%
\begin{center}
\begin{tabular}{l}
|\input{childdoc.def}|\\
|\childdocby{|\textit{main}|}|\\
\end{tabular}
\end{center}
%
The directive |\childdocby| is similar to |\childdocof|
described in \secref{sec:include},
but the subsequent selection of content must be done manually.
To that end, both |\ifchilddoc| and |\ifchilddocmanual|
will be true upon processing of a part,
and the name of the part is stored in |\childdocname|.
Note that |\jobname| will be set to the filename of the current part
so that each part receives an individual |.aux| file
that does not interfere with the |.aux| file(s) of the main document.
This behaviour can be altered by the alternative form
|\childdocby[*]{|\textit{main}|}| (with a non-empty optional argument)
which uses the |.aux| file of the main document
by setting |\jobname| to \textit{main}.

%%%%%%%%%%%%%%%%%%%%%%%%%%%%%%%%%%%%%%%%%%%%%%%%%%%%%%%%%%%%%%%%%%%%%%%%%%%%%%%%
\subsection{Driver Development}
\label{sec:driver}

The \textsf{childdoc} mechanism can also be use for the development
of definition files such as \LaTeX{} styles or classes.
This case differs from the above setup with multiple parts
included by |\include| in that no |\includeonly| should be invoked.
This can be achieved by starting the include file
(before |\ProvidesPackage|) with:
%
\begin{center}
\begin{tabular}{l}
|\input{childdoc.def}|\\
|\childdocforward{|\textit{main}|}|\\
\end{tabular}
\end{center}
%
or alternatively with:
%
\begin{center}
\begin{tabular}{l}
|\input{childdoc.def}|\\
|\childdocby{|\textit{main}|}|\\
\end{tabular}
\end{center}
%
Both forms have slightly different effects as described above.
The main file is prepared as usual, see \secref{sec:include}.

%%%%%%%%%%%%%%%%%%%%%%%%%%%%%%%%%%%%%%%%%%%%%%%%%%%%%%%%%%%%%%%%%%%%%%%%%%%%%%%%
\subsection{Legacy Detection}
\label{sec:detection}

The directive |\childdocmain| in the main file can detect
whether the complete document or merely a child is to be compiled
even without using the directive |\childdocof|.
This method is deprecated because it is less robust
and there is no compelling reason to use it;
it is merely provided for backward compatibility
and it may be removed in future versions.

If the detection mechanism is to be used,
it is mandatory to correctly specify
the filename of the main file as the argument of |\childdocmain|:
%
\begin{center}
\begin{tabular}{l}
|\input{childdoc.def}|\\
|\childdocmain{|\textit{main}|}|\\
\end{tabular}
\end{center}
%
If |\jobname| does not match the argument \textit{main} of |\childdocmain|,
it is assumed that |\jobname| points to the child file to be compiled.
When using |\childdocmain| with the main file specified as argument,
it suffices to start a child file
with just |\input{|\textit{main}|}|
without loading of the package and using |\childdocof|.
If instead all processing is done
with the appropriate \textsf{childdoc} directives,
the argument of \textit{main} of |\childdocmain| can be empty.

An alternative version of the command line processing described
in \secref{sec:commandline} using the detection mechanism reads:
%
\begin{center}
|... -jobname "|\textit{target}|" "|[\textit{flags}]%
[|\def\jobname{|\textit{dest}|}|]|\input{|\textit{main}|}"|
\end{center}

%%%%%%%%%%%%%%%%%%%%%%%%%%%%%%%%%%%%%%%%%%%%%%%%%%%%%%%%%%%%%%%%%%%%%%%%%%%%%%%%
\subsection{Manual Code}
\label{sec:manual}

In case one cannot be certain whether the definitions file |childdoc.def|
is installed on the target \TeX{} distribution
and one prefers not to ship it,
it is conceivable to paste a few relevant commands into the sources.

To that end, drop all statements |\input{childdoc.def}|
and perform the replacements as outlined below.
Instead of |\childdocmain{|\textit{main}|}| add the following code
to the top of the main file:
%
\begin{center}
\begin{tabular}{l}
|\||ifdefined\childdocname\endinput\||fi\newif\ifchilddoc|\\
|\edef\childdocname{\scantokens\expandafter{\jobname\noexpand}}|\\
|\def\childdocmain{|\textit{main}|}\||ifx\childdocmain\childdocname\||else|\\
|\childdoctrue\includeonly{\childdocname}\let\jobname\childdocmain\||fi|\\
\end{tabular}
\end{center}
%
Instead of |\childdocof{|\textit{main}|}| just include the main file
at the top of each child file:
%
\begin{center}
|\input{|\textit{main}|}|
\end{center}
%
A simple redirection |\childdocforward{|\textit{dest}|}| is achieved by:
%
\begin{center}
|\def\jobname{|\textit{dest}|}\input{\jobname}|
\end{center}
%
The redirection with prefix
|\childdocforwardprefix[|\textit{prefix}|]{|\textit{dest}|}|
is accomplished by:
%
\begin{center}
\begin{tabular}{l}
|{\edef\jobname{\scantokens\expandafter{\jobname\noexpand}}|\\
|\def\redirectjob |\textit{prefix}|#1~~~{\gdef\jobname{|\textit{dest}|#1}}|\\
|\expandafter\redirectjob\jobname~~~}\input{\jobname}|
\end{tabular}
\end{center}

In an alternative approach,
child documents can be compiled by a specific command line
without additional code or specific definitions:
%
\begin{center}
|... -jobname "|\textit{target}|" "|[\textit{flags}]%
|\includeonly{|\textit{dest}|}\input{|\textit{main}|}"|
\end{center}
%

%%%%%%%%%%%%%%%%%%%%%%%%%%%%%%%%%%%%%%%%%%%%%%%%%%%%%%%%%%%%%%%%%%%%%%%%%%%%%%%%
%%%%%%%%%%%%%%%%%%%%%%%%%%%%%%%%%%%%%%%%%%%%%%%%%%%%%%%%%%%%%%%%%%%%%%%%%%%%%%%%
\section{Information}

%%%%%%%%%%%%%%%%%%%%%%%%%%%%%%%%%%%%%%%%%%%%%%%%%%%%%%%%%%%%%%%%%%%%%%%%%%%%%%%%
\subsection{Copyright}

Copyright \copyright{} 2017--2018 Niklas Beisert

This work may be distributed and/or modified under the
conditions of the \LaTeX{} Project Public License, either version 1.3
of this license or (at your option) any later version.
The latest version of this license is in
  \url{http://www.latex-project.org/lppl.txt}
and version 1.3 or later is part of all distributions of \LaTeX{}
version 2005/12/01 or later.

This work has the LPPL maintenance status `maintained'.

The Current Maintainer of this work is Niklas Beisert.

This work consists of the files |README.txt|, |childdoc.ins| and |childdoc.dtx|
as well as the derived files |childdoc.def|, |cdocsamp.tex|
with |cdocsch1.tex|, |cdocsch2.tex|, |cdocspt3.tex|, |cdocspt4.tex|,
|cdocsdrf.tex|, |cdocsfn1.tex|, |cdocsfn2.tex|
as well as |childdoc.pdf|.

%%%%%%%%%%%%%%%%%%%%%%%%%%%%%%%%%%%%%%%%%%%%%%%%%%%%%%%%%%%%%%%%%%%%%%%%%%%%%%%%
\subsection{Files and Installation}

The package consists of the files:
%
\begin{center}
\begin{tabular}{ll}
    |README.txt|   & readme file \\
    |childdoc.ins| & installation file \\
    |childdoc.dtx| & source file \\
    |childdoc.def| & definition file \\
    |cdocsamp.tex| & sample main file \\
    |cdocsch1.tex| & sample include file \\
    |cdocsch2.tex| & sample include file \\
    |cdocspt3.tex| & sample part file \\
    |cdocspt4.tex| & sample part file \\
    |cdocsdrf.tex| & sample redirection file \\
    |cdocsfn1.tex| & sample redirection file \\
    |cdocsfn2.tex| & sample redirection file \\
    |childdoc.pdf| & manual
\end{tabular}
\end{center}
%
The distribution consists of the files
|README.txt|, |childdoc.ins| and |childdoc.dtx|.
%
\begin{itemize}
\item
Run (pdf)\LaTeX{} on |childdoc.dtx|
to compile the manual |childdoc.pdf| (this file).
\item
Run \LaTeX{} on |childdoc.ins| to create the definitions file |childdoc.def|
and the sample |cdocsamp.tex| with include files
|cdocsch1.tex|, |cdocsch2.tex|, |cdocspt3.tex|, |cdocspt4.tex|,
|cdocsdrf.tex|, |cdocsfn1.tex|, |cdocsfn2.tex|.
Then copy the file |childdoc.def| to an appropriate directory of your \LaTeX{}
distribution, e.g.\ \textit{texmf-root}|/tex/latex/childdoc|.
\end{itemize}

%%%%%%%%%%%%%%%%%%%%%%%%%%%%%%%%%%%%%%%%%%%%%%%%%%%%%%%%%%%%%%%%%%%%%%%%%%%%%%%%
\subsection{Related CTAN Packages}

There are several other packages which offer a similar functionality:
%
\begin{itemize}
\item
The packages
\href{http://ctan.org/pkg/docmute}{\textsf{docmute}},
\href{http://ctan.org/pkg/includex}{\textsf{includex}} and
\href{http://ctan.org/pkg/standalone}{\textsf{standalone}}
provide commands to include only the document body of
a child file thus allowing both files to be compiled individually.
\item
The packages \href{http://ctan.org/pkg/subdocs}{\textsf{subdocs}}
and \href{http://ctan.org/pkg/subfiles}{\textsf{subfiles}}
provide structures in which the main and child documents can be
encapsulated and allowing them to be compiled individually.
The inclusion mechanism is different from the conventional |\include|.
\item
The package \href{http://ctan.org/pkg/combine}{\textsf{combine}}
is an elaborate solution to combine several documents into one.
\end{itemize}
%
See also the CTAN topic \href{http://ctan.org/topic/subdocs}{\textsf{subdocs}}
for further related packages.
The present package differs from the above solutions in that
a document structure constructed with the conventional |\include| mechanism
just needs two extra commands at the top of every file
such that all constituent files can be compiled individually.

%%%%%%%%%%%%%%%%%%%%%%%%%%%%%%%%%%%%%%%%%%%%%%%%%%%%%%%%%%%%%%%%%%%%%%%%%%%%%%%%
%\subsection{Feature Suggestions}
%
%The following is a list of features which may be useful for future
%versions of this package:
%%
%\begin{itemize}
%\item
%\ldots
%\end{itemize}

%%%%%%%%%%%%%%%%%%%%%%%%%%%%%%%%%%%%%%%%%%%%%%%%%%%%%%%%%%%%%%%%%%%%%%%%%%%%%%%%
\subsection{Revision History}

%%%%%%%%%%%%%%%%%%%%%%%%%%%%%%%%%%%%%%%%
\paragraph{v2.0:} 2018/12/30

\begin{itemize}
\item
immediate forward processing
\item
added |\childdocby| mechanism
\item
manual restructured
\end{itemize}

%%%%%%%%%%%%%%%%%%%%%%%%%%%%%%%%%%%%%%%%
\paragraph{v1.6:} 2018/01/17

\begin{itemize}
\item
application for development of include files
\item
corrections to manual
\end{itemize}

%%%%%%%%%%%%%%%%%%%%%%%%%%%%%%%%%%%%%%%%
\paragraph{v1.5:} 2017/05/21

\begin{itemize}
\item
more complete structuring introduced
\item
|\childdocof| introduced
\item
|\childdoc| renamed to |\childdocmain|
\item
|\childredirect| renamed to |\childdocforward| and |\childdocforwardprefix|
and functionality expanded
\end{itemize}

%%%%%%%%%%%%%%%%%%%%%%%%%%%%%%%%%%%%%%%%
\paragraph{v1.0:} 2017/04/27

\begin{itemize}
\item
manual and install package
\item
first version published on CTAN
\end{itemize}

%%%%%%%%%%%%%%%%%%%%%%%%%%%%%%%%%%%%%%%%
\paragraph{v0.6:} 2017/04/26

\begin{itemize}
\item
redirection mechanism added
\end{itemize}

%%%%%%%%%%%%%%%%%%%%%%%%%%%%%%%%%%%%%%%%
\paragraph{v0.5:} 2017/04/26

\begin{itemize}
\item
functionality in definition file
\end{itemize}


%%%%%%%%%%%%%%%%%%%%%%%%%%%%%%%%%%%%%%%%%%%%%%%%%%%%%%%%%%%%%%%%%%%%%%%%%%%%%%%%
%%%%%%%%%%%%%%%%%%%%%%%%%%%%%%%%%%%%%%%%%%%%%%%%%%%%%%%%%%%%%%%%%%%%%%%%%%%%%%%%
%%%%%%%%%%%%%%%%%%%%%%%%%%%%%%%%%%%%%%%%%%%%%%%%%%%%%%%%%%%%%%%%%%%%%%%%%%%%%%%%
\appendix

\settowidth\MacroIndent{\rmfamily\scriptsize 000\ }

 \DocInput{childdoc.dtx}

\end{document}
%</driver>
% \fi
%
% %%%%%%%%%%%%%%%%%%%%%%%%%%%%%%%%%%%%%%%%%%%%%%%%%%%%%%%%%%%%%%%%%%%%%%%%%%%%%%
% %%%%%%%%%%%%%%%%%%%%%%%%%%%%%%%%%%%%%%%%%%%%%%%%%%%%%%%%%%%%%%%%%%%%%%%%%%%%%%
% \section{Sample}
%\iffalse
%<*samplemain>
%\fi
%
% The following presents a sample document
% with two chapters, two parts, a title page,
% a compile flag as well as three forwarding files to set the flag.
% It consists of eight |.tex| files:
% \begin{center}
% \begin{tabular}{ll}
% |cdocsamp.tex|&main file\\
% |cdocsch1.tex|&include file for chapter 1\\
% |cdocsch2.tex|&include file for chapter 2\\
% |cdocspt3.tex|&include file for part 3\\
% |cdocspt4.tex|&include file for part 4\\
% |cdocsdrf.tex|&forwarding file for main file in draft mode\\
% |cdocsfi1.tex|&forwarding file for final version of chapter 1\\
% |cdocsfi2.tex|&forwarding file for final version of chapter 2\\
% \end{tabular}
% \end{center}
% Each of the eight files can be compiled directly by the \LaTeX{} compiler.
%
% %%%%%%%%%%%%%%%%%%%%%%%%%%%%%%%%%%%%%%
% \paragraph{Main File.}
%
% The main file is called |cdocsamp.tex|.
%
% Load the \textsf{childdoc} definitions and
% declare the filename for the main document:
%    \begin{macrocode}
\input{childdoc.def}
\childdocmain{}
%    \end{macrocode}

% Optional override for |\version| flag:
%    \begin{macrocode}
%%\ifchilddoc\else\providecommand{\version}{draft}\fi
%    \end{macrocode}

% Define the default values for the |\version| flag
% (|final| for the main file and |draft| for childs):
%    \begin{macrocode}
\ifchilddoc
\providecommand{\version}{draft}
\else
\providecommand{\version}{final}
\fi
%    \end{macrocode}

% Load the standard document class:
%    \begin{macrocode}
\documentclass[12pt]{article}
%    \end{macrocode}

% Start the document body:
%    \begin{macrocode}
\begin{document}
%    \end{macrocode}

% Declare a title page.
% Print title, part of document being processed and version flag:
%    \begin{macrocode}
\addtocounter{page}{-1}
\begin{center}
{\LARGE\bfseries{}childdoc example\par}
\vspace{1cm}
\ifchilddoc
\ifchilddocmanual part\else chapter\fi:
`\childdocname' of `\childdocjob'\par
\else
main document: `\childdocjob'\par
\fi
version: \version\par
\end{center}
\newpage
%    \end{macrocode}

% Manually include selected file,
% otherwise process as usual:
%    \begin{macrocode}
\ifchilddocmanual
\section*{part `\childdocname'}
\input{\childdocname}
\else
%    \end{macrocode}

% Include the two chapters:
%    \begin{macrocode}
\include{cdocsch1}
\include{cdocsch2}
%    \end{macrocode}

% Include the two parts unless only chapters should be displayed:
%    \begin{macrocode}
\ifchilddoc\else
\section{part three}
\input{cdocspt3}
\section{part four}
\input{cdocspt4}
\fi
%    \end{macrocode}

% Process as usual until here:
%    \begin{macrocode}
\fi
%    \end{macrocode}

% End of document body:
%    \begin{macrocode}
\end{document}
%    \end{macrocode}
%\iffalse
%</samplemain>
%\fi
%
% %%%%%%%%%%%%%%%%%%%%%%%%%%%%%%%%%%%%%%
% \paragraph{Chapter Include Files.}
%
% The include files are called |cdocsch1.tex| and |cdocsch2.tex|.
%
%\iffalse
%<*samplechap1|samplechap2>
%\fi

% Optional override for |\version| flag:
%    \begin{macrocode}
%%\providecommand{\version}{final}
%    \end{macrocode}

% Include the main document:
%    \begin{macrocode}
\input{childdoc.def}
\childdocof{cdocsamp}
%    \end{macrocode}

%\iffalse
%</samplechap1|samplechap2>
%\fi
%
%\iffalse
%<*samplechap1>
%\fi
% Some text for chapter 1:
%    \begin{macrocode}
\section{one}
some text in chapter one
%    \end{macrocode}

%\iffalse
%</samplechap1>
%\fi
% Some text for chapter 2:
%\iffalse
%<*samplechap2>
%\fi
%    \begin{macrocode}
\section{two}
more text in chapter two
%    \end{macrocode}

%\iffalse
%</samplechap2>
%\fi
%
% %%%%%%%%%%%%%%%%%%%%%%%%%%%%%%%%%%%%%%
% \paragraph{Part Include Files.}
%
% The include files are called |cdocspt3.tex| and |cdocspt4.tex|.
%
%\iffalse
%<*samplepart3|samplepart4>
%\fi

% Optional override for |\version| flag:
%    \begin{macrocode}
%%\providecommand{\version}{final}
%    \end{macrocode}

% Include the main document:
%    \begin{macrocode}
\input{childdoc.def}
\childdocby{cdocsamp}
%    \end{macrocode}

%\iffalse
%</samplepart3|samplepart4>
%\fi
%
%\iffalse
%<*samplepart3>
%\fi
% Some text for part 3:
%    \begin{macrocode}
some text in part three
%    \end{macrocode}

%\iffalse
%</samplepart3>
%\fi
% Some text for part 4:
%\iffalse
%<*samplepart4>
%\fi
%    \begin{macrocode}
more text in part four
%    \end{macrocode}

%\iffalse
%</samplepart4>
%\fi
%
% %%%%%%%%%%%%%%%%%%%%%%%%%%%%%%%%%%%%%%
% \paragraph{Forwarding for a Complete Draft.}
%
% The following forwarding file |cdocsdrf.tex|
% compiles the main document in draft mode:
%\iffalse
%<*sampledraft>
%\fi
%    \begin{macrocode}
\def\version{draft}
\input{childdoc.def}
\childdocforward{cdocsamp}
%    \end{macrocode}

%\iffalse
%</sampledraft>
%\fi
%
% %%%%%%%%%%%%%%%%%%%%%%%%%%%%%%%%%%%%%%
% \paragraph{Forwarding for Final Version of the Chapters.}
%
% The following forwarding files |cdocsfn1.tex| and |cdocsfn2.tex|
% (with identical content)
% compile the final versions of the child documents
% |cdocsch1.tex| and |cdocsch2.tex|, respectively:
%\iffalse
%<*samplefinal>
%\fi
%    \begin{macrocode}
\def\version{final}
\input{childdoc.def}
\childdocforwardprefix[cdocsamp]{cdocsfn}{cdocsch}
%    \end{macrocode}

%\iffalse
%</samplefinal>
%\fi
%
% %%%%%%%%%%%%%%%%%%%%%%%%%%%%%%%%%%%%%%
% \paragraph{Command Line Processing.}
%
% The following three command lines generate the output files
% |cdocscld|, |cdocscl1| and |cdocscl2|
% which should be identical to
% |cdocsdrf|, |cdocsch1| and |cdocsfn2|, respectively:
% \begin{center}
% \begin{tabular}{l}
% |latex -jobname cdocscld \|\\
% |  "\def\version{draft}\input{childdoc.def}\childdocforward{cdocsamp}"|\\
% |latex -jobname cdocscl1 \|\\
% |  "\input{childdoc.def}\childdocforward[cdocsamp]{cdocsch1}"|\\
% |latex -jobname cdocscl2 \|\\
% |  "\def\version{final}\input{childdoc.def}\childdocforward{cdocsch2}"|
% \end{tabular}
% \end{center}
% Note that the trailing backslash on each first line
% merely continues the input to the second line
% (for convenient cut ant paste).
% Furthermore, the command |latex| can be replaced by any
% of its alternative versions such as |pdflatex|.
%
% %%%%%%%%%%%%%%%%%%%%%%%%%%%%%%%%%%%%%%%%%%%%%%%%%%%%%%%%%%%%%%%%%%%%%%%%%%%%%%
% %%%%%%%%%%%%%%%%%%%%%%%%%%%%%%%%%%%%%%%%%%%%%%%%%%%%%%%%%%%%%%%%%%%%%%%%%%%%%%
% \section{Implementation}
%\iffalse
%<*package>
%\fi
%
% This section describes the definitions file |childdoc.def|.

% The definitions cannot be loaded using |\usepackage| or |\RequirePackage|
% which has a mechanism to prevent loading a style file more than once.
% When loading the definitions by means of |\input|
% multiple instances have to be prevented manually:
%\iffalse
%This code needs to be before the `\ProvidesFile' directive
%which is defined at the beginning of this file.
%Therefore it is also placed there and commented out here.
%</package>
%<*discard>
%\fi
%    \begin{macrocode}
\ifdefined\childdocmain\endinput\fi
%    \end{macrocode}
%\iffalse
%</discard>
%<*package>
%\fi
%
% \macro{\ifchilddoc}
% \macro{\ifchilddocmanual}
% The conditional |\ifchilddoc| tells whether a
% child (true) or main (false) document is being compiled.
% The conditional |\ifchilddocmanual| tells whether
% the |\includeonly| mechanism is used (false) or
% the selection of child files must be performed manually (true).
% The definitions initialise to false:
%    \begin{macrocode}
\newif\ifchilddoc
\newif\ifchilddocmanual
%    \end{macrocode}

% \macro{\childdocname}
% \macro{\childdocjob}
% The macro |\childdocname| stores the name of the main document
% to be compiled. The macro |\childdocjob| stores the name of
% the document on which the \LaTeX{} compiler was originally invoked.
% The content of |\jobname| cannot be compared
% to filenames specified in the source due to different catcodes.
% The following code rescans |\jobname|, stores the result
% in |\childdocname| and saves a copy in |\childdocjob|:
%    \begin{macrocode}
\edef\childdocname{\scantokens\expandafter{\jobname\noexpand}}
\let\childdocjob\childdocname
%    \end{macrocode}

% \macro{\childdocdisable}
% The macro |\childdocdisable| prevents the main file
% from being processed more than once.
% At this stage, the main document command |\childdocmain|
% is assumed to be called once again where it should do nothing.
% Any subsequent call to it should prevent
% a secondary processing of the main document
% It overwrites the forwarding commands
% |\childdocof| and |\childdocforward|
% with empty macros to prevent further inclusions of the main document:
%    \begin{macrocode}
\newcommand{\childdocdisable}
{
  \renewcommand{\childdocmain}[1]{\renewcommand{\childdocmain}[1]{\endinput}}
  \renewcommand{\childdocof}[1]{}
  \renewcommand{\childdocby}[2][]{}
  \renewcommand{\childdocforward}[2][]{}
  \renewcommand{\childdocdisable}{}
}
%    \end{macrocode}

% \macro{\childdocmain}
% The macro |\childdocmain| is to be called at the top of the main file
% with nothing or the main filename (without extension) as argument.
% First, it breaks loops.
% If the argument is not empty and does not match |\childdocname|
% (which is set by the first inclusion of |childdoc.def|),
% |\ifchilddoc| is set to true, |\includeonly| is applied to the child file
% and |\jobname| is set to the main file
% (for proper handling of |.aux| files):
%    \begin{macrocode}
\newcommand{\childdocmain}[1]
{
  \childdocdisable\childdocmain{}
  \if?#1?\else
    \begingroup
      \def\childdoctmp{#1}
      \ifx\childdoctmp\childdocname
        \def\childdoctmp{}
      \else
        \def\childdoctmp
        {
          \childdoctrue
          \includeonly{\childdocname}
          \def\childdocjob{#1}
          \def\jobname{#1}
        }
      \fi
      \expandafter
    \endgroup
    \childdoctmp
  \fi
}
%    \end{macrocode}

% \macro{\childdocof}
% The command |\childdocof| redirects
% compilation to the main file |#1|.
%    \begin{macrocode}
\newcommand{\childdocof}[1]
{
  \childdocdisable
  \childdoctrue
  \includeonly{\childdocname}
  \def\jobname{#1}
  \def\childdocjob{#1}
  \input{#1}
}
%    \end{macrocode}

% \macro{\childdocby}
% The command |\childdocby| ....
%    \begin{macrocode}
\newcommand{\childdocby}[2][]
{
  \childdocdisable
  \childdoctrue
  \childdocmanualtrue
  \if?#1?\else
    \def\jobname{#2}
  \fi
  \def\childdocjob{#2}
  \input{#2}
  \endinput
}
%    \end{macrocode}

% \macro{\childdocforward}
% The command |\childdocforward| redirects
% compilation to the main file or
% (if the optional argument is given) a child file.
% Parameters are set as if the main file
% or a child file starting with |\childdocof| was compiled.
% Then compilation is handed over to the main file:
%    \begin{macrocode}
\newcommand{\childdocforward}[2][]
{
  \begingroup
    \if?#1?
      \def\childdoctmp
      {
        \def\childdocname{#2}
        \def\childdocjob{#2}
        \def\jobname{#2}
        \input{#2}
        \endinput
      }
    \else
      \def\childdoctmp
      {
        \childdocdisable
        \def\childdocname{#2}
        \childdoctrue
        \includeonly{#2}
        \def\childdocjob{#1}
        \def\jobname{#1}
        \input{#1}
        \endinput
      }
    \fi
    \expandafter
  \endgroup
  \childdoctmp
}
%    \end{macrocode}

% \macro{\childdocforwardprefix}
% The command |\childdocforwardprefix| redirects
% compilation to the main or a child file by means of a pattern.
% The prefix |#1| in the current filename is replaced by |#2|
% and the suffix of the current filename is kept
% (it is assumed that the filename does not contain the substring `|~~~|'
% which is used as a delimiter).
% Compilation is handed over to the new file by |\childdocforward|:
%    \begin{macrocode}
\newcommand{\childdocforwardprefix}[3][]
{
  \begingroup
    \def\childdocextract #2##1~~~{\def\childdoctmp{\childdocforward[#1]{#3##1}}}
    \expandafter\childdocextract\childdocname~~~
    \expandafter
  \endgroup
  \childdoctmp
}
%    \end{macrocode}

% \macro{\childdoc}
% The deprecated macro |\childdoc| is a legacy version of |\childdocmain|:
%    \begin{macrocode}
\newcommand{\childdoc}{\childdocmain}
%    \end{macrocode}

% \macro{\childdocredirect}
% The deprecated macro |\childdocredirect| is a legacy version
% of |\childdocforward| and |\childdocforwardprefix|:
%    \begin{macrocode}
\newcommand{\childdocredirect}[2][]
{
  \begingroup
    \if?#1?
      \def\childdoctmp{\childdocforward{#2}}
    \else
      \def\childdoctmp{\childdocforwardprefix{#1}{#2}}
    \fi
    \expandafter
  \endgroup
  \childdoctmp
}
%    \end{macrocode}

%\iffalse
%</package>
%\fi
%
\endinput
\childdocforward[|\textit{main}|]{|\textit{dest}|}"|
\end{center}
%
Here \textit{target} is the name of the output file,
\textit{main} is the name of the main file
and \textit{dest} is the name of the main or child file to be processed
(all filenames without extensions).
The optional argument \textit{main} can be omitted
if \textit{main} matches \textit{dest}.
Optionally, compilation \textit{flags} can be defined via |\def| commands.
This command line makes the \TeX{} engine believe
it is compiling the file \textit{target}
whose content is specified as the latter parameter.
The provided code then forwards the processing to
\textit{main} or \textit{dest} as described in \secref{sec:forward}.

%%%%%%%%%%%%%%%%%%%%%%%%%%%%%%%%%%%%%%%%%%%%%%%%%%%%%%%%%%%%%%%%%%%%%%%%%%%%%%%%
\subsection{Include by Input}
\label{sec:input}

Including child documents by |\include| has some restrictions by design.
Most notably, the content of a child document always occupies
its own set of pages; pages cannot be shared between child documents.
Usually, this behaviour makes perfect sense
because each child document contain an essential part of the document.
However, in some situations it may be desirable to compose
a document from a collection of parts
without having mandatory page breaks between then.
For this case, the package
provides a mechanism to include parts
by |\input| which can also be processed individually.
However, by construction this mechanism
requires manual handling of the content to be output.

%%%%%%%%%%%%%%%%%%%%%%%%%%%%%%%%%%%%%%%%
\DescribeMacro{\ifchilddocmanual}
The main file should be prepared as usual, see \secref{sec:include}.
However, the document body must make a distinction
between processing of an individual part and of the main document, e.g.:
%
\begin{center}
\begin{tabular}{l}
|\ifchilddocmanual|\\
|\input{\childdocname}|\\
|\||else|\\
\textit{document body with }|\input{|\textit{part}|}|\\
|\||fi|
\end{tabular}
\end{center}
%
The conditional |\ifchilddocmanual| is true whenever
a part to be included by |\input| is being compiled,
and the name of the part is stored in |\childdocname|.

%%%%%%%%%%%%%%%%%%%%%%%%%%%%%%%%%%%%%%%%
\DescribeMacro{\childdocby}
Each part to be included by |\input| should start with:
%
\begin{center}
\begin{tabular}{l}
|% \iffalse
%
% childdoc.dtx Copyright (C) 2017-2018 Niklas Beisert
%
% This work may be distributed and/or modified under the
% conditions of the LaTeX Project Public License, either version 1.3
% of this license or (at your option) any later version.
% The latest version of this license is in
%   http://www.latex-project.org/lppl.txt
% and version 1.3 or later is part of all distributions of LaTeX
% version 2005/12/01 or later.
%
% This work has the LPPL maintenance status `maintained'.
%
% The Current Maintainer of this work is Niklas Beisert.
%
% This work consists of the files childdoc.dtx and childdoc.ins
% and the derived files childdoc.def and cdocsamp.tex with
% cdocsch1.tex, cdocsch2.tex, cdocsdrf.tex, cdocsfn1.tex, cdocsfn2.tex.
%
%<package>\ifdefined\childdocmain\endinput\fi
%<package>\ProvidesFile{childdoc.def}[2018/12/30 v2.0 child document driver]
%<samplemain>\ProvidesFile{cdocsamp.tex}[2018/12/30 v2.0 sample for childdoc]
%<*driver>
%\ProvidesFile{childdoc.drv}[2018/12/30 v2.0 childdoc reference manual file]
\PassOptionsToClass{10pt,a4paper}{article}
\documentclass{ltxdoc}

\usepackage[margin=35mm]{geometry}
\usepackage{hyperref}
\usepackage{hyperxmp}
\usepackage[usenames]{color}

\hypersetup{colorlinks=true}
\hypersetup{pdfstartview=FitH}
\hypersetup{pdfpagemode=UseNone}
\hypersetup{pdfsource={}}
\hypersetup{pdflang={en-UK}}
\hypersetup{pdfcopyright={Copyright 2017-2018 Niklas Beisert.
  This work may be distributed and/or modified under the
  conditions of the LaTeX Project Public License, either version 1.3
  of this license or (at your option) any later version.}}
\hypersetup{pdflicenseurl={http://www.latex-project.org/lppl.txt}}
\hypersetup{pdfcontactaddress={ETH Zurich, ITP, HIT K,
  Wolfgang-Pauli-Strasse 27}}
\hypersetup{pdfcontactpostcode={8093}}
\hypersetup{pdfcontactcity={Zurich}}
\hypersetup{pdfcontactcountry={Switzerland}}
\hypersetup{pdfcontactemail={nbeisert@itp.phys.ethz.ch}}
\hypersetup{pdfcontacturl={http://people.phys.ethz.ch/\xmptilde nbeisert/}}

\newcommand{\secref}[1]{\hyperref[#1]{section \ref*{#1}}}

\parskip1ex
\parindent0pt
\let\olditemize\itemize
\def\itemize{\olditemize\parskip0pt}

\begin{document}

\title{The \textsf{childdoc} Package}
\hypersetup{pdftitle={The childdoc Package}}
\author{Niklas Beisert\\[2ex]
  Institut f\"ur Theoretische Physik\\
  Eidgen\"ossische Technische Hochschule Z\"urich\\
  Wolfgang-Pauli-Strasse 27, 8093 Z\"urich, Switzerland\\[1ex]
  \href{mailto:nbeisert@itp.phys.ethz.ch}
  {\texttt{nbeisert@itp.phys.ethz.ch}}}
\hypersetup{pdfauthor={Niklas Beisert}}
\hypersetup{pdfsubject={Manual for the LaTeX2e Package childdoc}}
\date{30 December 2018, \textsf{v2.0}}
\maketitle

\begin{abstract}\noindent
\textsf{childdoc} is a \LaTeXe{} package
that enables the direct compilation
of document sections included by |\include|
to individual files.
\end{abstract}

\begingroup
\parskip0ex
\tableofcontents
\endgroup

%%%%%%%%%%%%%%%%%%%%%%%%%%%%%%%%%%%%%%%%%%%%%%%%%%%%%%%%%%%%%%%%%%%%%%%%%%%%%%%%
%%%%%%%%%%%%%%%%%%%%%%%%%%%%%%%%%%%%%%%%%%%%%%%%%%%%%%%%%%%%%%%%%%%%%%%%%%%%%%%%
\section{Introduction}

\LaTeX{} provides a mechanism to structure a large document (such as a book)
into a main file and several child files (containing the chapters)
using the |\include| command.
This mechanism is beneficial for documents
which span hundreds of pages in order to
make the source file(s) more manageable.
Moreover, compilation can be restricted to
selected child files by means of the |\includeonly| command.
The latter feature can be used to reduce the compilation time while editing
(this was significantly more useful in the earlier days of \LaTeX{})
or to generate a smaller document which is easier to navigate.
Another application of |\includeonly| is to generate
documents consisting of selected parts of the complete document.

However, there are a few drawbacks of the plain |\include| mechanism:
\begin{itemize}
\item
The child files cannot be compiled on their own,
they can only be compiled via the main file.
A naive editing environment
(such as a text editor with an option
to have the current file processed by \LaTeX)
may require one to switch to the main file before compiling;
attempting to compile the child file produces errors.
\item
The main file must be modified (each time)
to adjust the |\includeonly| command
to the present needs. This easily leaves the main file in a messy state.
\item
The generated document will always carry the filename
of the main document. This is inconvenient if
several child files are to be compiled and
to be kept for distribution.
\end{itemize}

The present package provides a simple interface
to make child files individually compilable by \LaTeX{}.
Compiling a child file then has the same effect as compiling
the main file with an |\includeonly| command
to select the appropriate child.
Moreover the generated document will carry the name of the child
rather than the main file.
This resolves all three above issues.

This feature is meant to make the editing of books,
thesis documents and lecture notes somewhat more convenient.
However, the package can also be used efficiently for
composing a series of documents (such as exercise sheets)
which are typically distributed individually.
It then assists the author in generating the individual documents
(potentially in different versions)
as well as a document containing the collected series.
Another application is in developing style files
or other kinds of included material
where compilation of the style file could redirect
to a sample or test file.

%%%%%%%%%%%%%%%%%%%%%%%%%%%%%%%%%%%%%%%%%%%%%%%%%%%%%%%%%%%%%%%%%%%%%%%%%%%%%%%%
%%%%%%%%%%%%%%%%%%%%%%%%%%%%%%%%%%%%%%%%%%%%%%%%%%%%%%%%%%%%%%%%%%%%%%%%%%%%%%%%
\section{Usage}

First of all, the package \textsf{childdoc} is \emph{not} a standard
\LaTeXe{} |.sty| style file! Therefore it needs to be invoked in
a non-standard way.

%%%%%%%%%%%%%%%%%%%%%%%%%%%%%%%%%%%%%%%%%%%%%%%%%%%%%%%%%%%%%%%%%%%%%%%%%%%%%%%%
\subsection{Included Files}
\label{sec:include}

%%%%%%%%%%%%%%%%%%%%%%%%%%%%%%%%%%%%%%%%
\DescribeMacro{\childdocmain}
To use the package, add the commands
\begin{center}
\begin{tabular}{l}
|\input{childdoc.def}|\\
|\childdocmain{}|\\
\end{tabular}
\end{center}
at the very top of the main \LaTeX{} file,
in particular \emph{before} the |\documentclass| statement!
The argument of |\childdocmain| should be left empty
(but it must be present).

%%%%%%%%%%%%%%%%%%%%%%%%%%%%%%%%%%%%%%%%
\DescribeMacro{\childdocof}
Furthermore, add the commands
\begin{center}
\begin{tabular}{l}
|\input{childdoc.def}|\\
|\childdocof{|\textit{main}|}|\\
\end{tabular}
\end{center}
at the top of every child file \textit{child}
which is included by |\include{|\textit{child}|}|
from within the main file
(or at least for those files to be compiled individually).
The argument \textit{main} must be the filename of the main file.

There are a couple of
considerations in setting up the main and child documents:

%%%%%%%%%%%%%%%%%%%%%%%%%%%%%%%%%%%%%%%%
\paragraph{Restrictions.}

Please note the following restrictions:
\begin{itemize}
\item
|\childdocmain| must be called with one argument \textit{main}
to ensure compatibility with earlier version of the package.
It must either be empty (|\childdocmain{}|)
or precisely match the filename of the main file in which it is specified.
See \secref{sec:detection} for further information.
\item
The filename \textit{main} must be specified without the |.tex| extension.
\item
The filename \textit{main} is case sensitive
(even in case-insensitive file systems)
due to internal string comparison.
\item
The argument \textit{main} should be fully expanded, it cannot be a macro.
\item
Subdirectories and special characters should be avoided in filenames.
\item
The command |\childdocmain{|\textit{main}|}| must be followed by a whitespace.
It should not be followed immediately by another command
or by a comment mark `|%|'.
This is because the \TeX{} parser reads the token immediately following
the argument of |\childdocmain| and puts it
at the beginning of every child section;
however, a white\-space is ignored.
\end{itemize}

%%%%%%%%%%%%%%%%%%%%%%%%%%%%%%%%%%%%%%%%
\paragraph{Content of Main File.}

It is advisable to place all content in the child files included by |\include|.
Any output contained in the main file will appear in all child documents
unless suppressed manually;
it cannot be suppressed automatically by the |\includeonly| directive
and thus should normally be avoided.
A method to include some content in the main file
by means of conditional processing is described in \secref{sec:conditional}.

%%%%%%%%%%%%%%%%%%%%%%%%%%%%%%%%%%%%%%%%
\paragraph{Page Numbering.}

When only a part of the document is compiled,
the appropriate numbering of pages
(as well as other status parameters)
is determined from the |.aux| files.
The latter contain information from previous passes.
However this information needs to propagate through
all intermediate child documents.
Therefore the page numbering in child documents may well
be inconsistent until the complete document is compiled at least once.

A useful (if unconventional) way to always ensure a consistent
page numbering is to restart the numbering in each child document
and denote the pages by `\textit{child}|.|\textit{page}'
where \textit{child} represents the chapter/section number of the child file.
This can be achieved by the command
|\numberwithin{page}{|\textit{child}|}|
of the \textsf{amsmath} package
where \textit{child} can be |chapter| or |section|
depending on the chosen structuring.
Alternatively, one can modify the macro |\thepage| appropriately
and reset the counter |page| at the start of each child file.

%%%%%%%%%%%%%%%%%%%%%%%%%%%%%%%%%%%%%%%%%%%%%%%%%%%%%%%%%%%%%%%%%%%%%%%%%%%%%%%%
\subsection{Conditional Processing}
\label{sec:conditional}

The package provides a mechanism to compile different versions
of a document. To customise the versions further some conditional processing
can come in handy to distinguish which version is being compiled.
The package provides two macros to describe the compilation context:

%%%%%%%%%%%%%%%%%%%%%%%%%%%%%%%%%%%%%%%%
\DescribeMacro{\ifchilddoc}
The conditional |\ifchilddoc| distinguishes between the compilation of
child documents and the main document:
%
\begin{center}
|\ifchilddoc |\textit{child-code}| |[|\||else |\textit{main-code}]| \||fi|
\end{center}

%%%%%%%%%%%%%%%%%%%%%%%%%%%%%%%%%%%%%%%%
\DescribeMacro{\childdocname}
\DescribeMacro{\childdocjob}
The macro |\childdocname| contains the filename (without extension)
of the main or child file being processed.
Note that |\childdocjob| will always contain the name of the main file.

%%%%%%%%%%%%%%%%%%%%%%%%%%%%%%%%%%%%%%%%
\paragraph{Title Page.}

Conditional processing can be used to include a title or banner page
in the main document when proper precautions are taken.
Importantly, the code in the main file should ensure that the page counter
(as well as other status parameters which are stored in the |.aux| files)
takes the same value after the conditional processing.
Otherwise the page numbers may take divergent values
depending on which part is compiled.

For example, a title page could be declared by:
%
\begin{center}
\begin{tabular}{l}
|\ifchilddoc\||else|\\
|\addtocounter{page}{-1}|\\
\textit{code for title page}\\
|\newpage|\\
|\||fi|
\end{tabular}
\end{center}
%
A banner page for the child documents can be generated by:
%
\begin{center}
\begin{tabular}{l}
|\ifchilddoc|\\
|\addtocounter{page}{-1}|\\
\textit{code for banner page}\\
|\newpage|\\
|\||fi|
\end{tabular}
\end{center}
%
Here one could write a message such as:
\begin{center}
|This is the part \childdocname{} of \childdocjob{}.|
\end{center}

%%%%%%%%%%%%%%%%%%%%%%%%%%%%%%%%%%%%%%%%%%%%%%%%%%%%%%%%%%%%%%%%%%%%%%%%%%%%%%%%
\subsection{Flags}
\label{sec:flags}

The package makes it easy to generate different versions
of the main or child documents.
To this end compilation flags can be defined
and assigned different default values.
They will be particularly useful in conjunction
with the forwarding mechanism described in \secref{sec:forward}.

For example, it may be useful to have a flag |\version|
which can be set to |draft| or |final|.
The document source will contain some conditional code
depending on the value of |\version|.
Suppose further, the flag should default to |final| for the main file
and to |draft| for child files
which is a natural assignment for editing the document.
This is achieved by placing the following code
in the preamble of the main document
(below the |\childdocmain| directive):
%
\begin{center}
\begin{tabular}{l}
|\ifchilddoc|\\
|\providecommand{\version}{draft}|\\
|\||else|\\
|\providecommand{\version}{final}|\\
|\||fi|
\end{tabular}
\end{center}
%
The definition by |\providecommand| makes sure
that previous definitions are not overwritten.
Further statements |\providecommand{\version}{...}|
can thus be added before the above code to override it.

For the main file, one might add a line
(between |\childdocmain| and the above block)
%
\begin{center}
|%\ifchilddoc\||else\providecommand{\version}{draft}\||fi|
\end{center}
%
which can be uncommented to produce a draft version.
Likewise one can add a line to the very top of a child file
(above the |\childdocof{|\textit{main}|}| directive)
%
\begin{center}
|%\providecommand{\version}{final}|
\end{center}
%
which can be uncommented to produce the final version of this child document.

%%%%%%%%%%%%%%%%%%%%%%%%%%%%%%%%%%%%%%%%%%%%%%%%%%%%%%%%%%%%%%%%%%%%%%%%%%%%%%%%
\subsection{Forwarding}
\label{sec:forward}

Different versions of the main or child documents
using compilation flags as described in \secref{sec:flags}
can be (permanently) stored in different files
for convenient compilation, viewing and distribution.
To this end, the package defines a command
to pass on compilation to a different file:

%%%%%%%%%%%%%%%%%%%%%%%%%%%%%%%%%%%%%%%%
\DescribeMacro{\childdocforward}
The command |\childdocforward| redirects processing to
another source file:
%
\begin{center}
\begin{tabular}{l}
|\input{childdoc.def}|\\
|\childdocforward[|\textit{main}|]{|\textit{dest}|}|\\
\end{tabular}
\end{center}
%
The argument \textit{dest} is the destination file
(without extension).
It should be the main file or one of the child files.
Note that further \textsf{childdoc} directives
such as |\childdocof| and |\childdocforward|
in the indicated file will be processed in this form.
The optional argument \textit{main}
passes on directly to the main file \textit{main}
while pretending to compile the child \textit{dest}.
This form behaves as if \textit{dest}
issues |\childdocof{|\textit{main}|}| right away,
and no further \textsf{childdoc} directives will be processed.

%%%%%%%%%%%%%%%%%%%%%%%%%%%%%%%%%%%%%%%%
\DescribeMacro{\...prefix}
In the alternative form |\childdocforwardprefix|,
%
\begin{center}
\begin{tabular}{l}
|\input{childdoc.def}|\\
|\childdocforwardprefix[|\textit{main}|]{|\textit{prefix}|}{|\textit{dest}|}|
\end{tabular}
\end{center}
%
the destination file is determined by a pattern
depending on the current file:
To make this work, the current file must be called
`{\textit{prefix}\hspace{0.2em}\textit{suffix}}'
with \textit{prefix} matching precisely the argument.
Processing is then passed on to the file
`{\textit{dest}\hspace{0.2em}\textit{suffix}}'.
Surely, the same effect is achieved by
directly specifying the
argument `{\textit{dest}\hspace{0.2em}\textit{suffix}}'
in the first form.
However, that requires to set up a different file
for each child. With the alternative form of the command
all these files can have exactly the same content
which simplifies setting them up and maintaining them.

For example, the following file |draft.tex|
with a compilation flag |\version| as described in \secref{sec:flags}
compiles the main document as a draft:
%
\begin{center}
\begin{tabular}{l}
|\def\version{draft}|\\
|\input{childdoc.def}|\\
|\childdocforward{|\textit{main}|}|
\end{tabular}
\end{center}
%
Likewise, the following files |final|\textit{nn}|.tex|
compile the final version of the child document
|child|\textit{nn}|.tex|:
%
\begin{center}
\begin{tabular}{l}
|\def\version{final}|\\
|\input{childdoc.def}|\\
|\childdocforwardprefix{final}{child}|
\end{tabular}
\end{center}
%

Note that when several versions of a main file and/or of each child file
are to be generated, it may be convenient to set up a |Makefile| or
shell script to automatise the process.

%%%%%%%%%%%%%%%%%%%%%%%%%%%%%%%%%%%%%%%%%%%%%%%%%%%%%%%%%%%%%%%%%%%%%%%%%%%%%%%%
\subsection{Command Line Processing}
\label{sec:commandline}

The effect of redirection files can also be achieved by invoking
the \LaTeX{} compiler with a more elaborate command line.
Most conveniently this should be done as part
of a shell script or a |Makefile|.

When using \textsf{childdoc} in the main file, the following
command lines effectively perform a redirection
(note that depending on the shell being used,
backslashes may have to be doubled: `|\|' $\to$ `|\\|'):
%
\begin{center}
|... -jobname "|\textit{target}|" |\\|"|[\textit{flags}]%
|\input{childdoc.def}\childdocforward[|\textit{main}|]{|\textit{dest}|}"|
\end{center}
%
Here \textit{target} is the name of the output file,
\textit{main} is the name of the main file
and \textit{dest} is the name of the main or child file to be processed
(all filenames without extensions).
The optional argument \textit{main} can be omitted
if \textit{main} matches \textit{dest}.
Optionally, compilation \textit{flags} can be defined via |\def| commands.
This command line makes the \TeX{} engine believe
it is compiling the file \textit{target}
whose content is specified as the latter parameter.
The provided code then forwards the processing to
\textit{main} or \textit{dest} as described in \secref{sec:forward}.

%%%%%%%%%%%%%%%%%%%%%%%%%%%%%%%%%%%%%%%%%%%%%%%%%%%%%%%%%%%%%%%%%%%%%%%%%%%%%%%%
\subsection{Include by Input}
\label{sec:input}

Including child documents by |\include| has some restrictions by design.
Most notably, the content of a child document always occupies
its own set of pages; pages cannot be shared between child documents.
Usually, this behaviour makes perfect sense
because each child document contain an essential part of the document.
However, in some situations it may be desirable to compose
a document from a collection of parts
without having mandatory page breaks between then.
For this case, the package
provides a mechanism to include parts
by |\input| which can also be processed individually.
However, by construction this mechanism
requires manual handling of the content to be output.

%%%%%%%%%%%%%%%%%%%%%%%%%%%%%%%%%%%%%%%%
\DescribeMacro{\ifchilddocmanual}
The main file should be prepared as usual, see \secref{sec:include}.
However, the document body must make a distinction
between processing of an individual part and of the main document, e.g.:
%
\begin{center}
\begin{tabular}{l}
|\ifchilddocmanual|\\
|\input{\childdocname}|\\
|\||else|\\
\textit{document body with }|\input{|\textit{part}|}|\\
|\||fi|
\end{tabular}
\end{center}
%
The conditional |\ifchilddocmanual| is true whenever
a part to be included by |\input| is being compiled,
and the name of the part is stored in |\childdocname|.

%%%%%%%%%%%%%%%%%%%%%%%%%%%%%%%%%%%%%%%%
\DescribeMacro{\childdocby}
Each part to be included by |\input| should start with:
%
\begin{center}
\begin{tabular}{l}
|\input{childdoc.def}|\\
|\childdocby{|\textit{main}|}|\\
\end{tabular}
\end{center}
%
The directive |\childdocby| is similar to |\childdocof|
described in \secref{sec:include},
but the subsequent selection of content must be done manually.
To that end, both |\ifchilddoc| and |\ifchilddocmanual|
will be true upon processing of a part,
and the name of the part is stored in |\childdocname|.
Note that |\jobname| will be set to the filename of the current part
so that each part receives an individual |.aux| file
that does not interfere with the |.aux| file(s) of the main document.
This behaviour can be altered by the alternative form
|\childdocby[*]{|\textit{main}|}| (with a non-empty optional argument)
which uses the |.aux| file of the main document
by setting |\jobname| to \textit{main}.

%%%%%%%%%%%%%%%%%%%%%%%%%%%%%%%%%%%%%%%%%%%%%%%%%%%%%%%%%%%%%%%%%%%%%%%%%%%%%%%%
\subsection{Driver Development}
\label{sec:driver}

The \textsf{childdoc} mechanism can also be use for the development
of definition files such as \LaTeX{} styles or classes.
This case differs from the above setup with multiple parts
included by |\include| in that no |\includeonly| should be invoked.
This can be achieved by starting the include file
(before |\ProvidesPackage|) with:
%
\begin{center}
\begin{tabular}{l}
|\input{childdoc.def}|\\
|\childdocforward{|\textit{main}|}|\\
\end{tabular}
\end{center}
%
or alternatively with:
%
\begin{center}
\begin{tabular}{l}
|\input{childdoc.def}|\\
|\childdocby{|\textit{main}|}|\\
\end{tabular}
\end{center}
%
Both forms have slightly different effects as described above.
The main file is prepared as usual, see \secref{sec:include}.

%%%%%%%%%%%%%%%%%%%%%%%%%%%%%%%%%%%%%%%%%%%%%%%%%%%%%%%%%%%%%%%%%%%%%%%%%%%%%%%%
\subsection{Legacy Detection}
\label{sec:detection}

The directive |\childdocmain| in the main file can detect
whether the complete document or merely a child is to be compiled
even without using the directive |\childdocof|.
This method is deprecated because it is less robust
and there is no compelling reason to use it;
it is merely provided for backward compatibility
and it may be removed in future versions.

If the detection mechanism is to be used,
it is mandatory to correctly specify
the filename of the main file as the argument of |\childdocmain|:
%
\begin{center}
\begin{tabular}{l}
|\input{childdoc.def}|\\
|\childdocmain{|\textit{main}|}|\\
\end{tabular}
\end{center}
%
If |\jobname| does not match the argument \textit{main} of |\childdocmain|,
it is assumed that |\jobname| points to the child file to be compiled.
When using |\childdocmain| with the main file specified as argument,
it suffices to start a child file
with just |\input{|\textit{main}|}|
without loading of the package and using |\childdocof|.
If instead all processing is done
with the appropriate \textsf{childdoc} directives,
the argument of \textit{main} of |\childdocmain| can be empty.

An alternative version of the command line processing described
in \secref{sec:commandline} using the detection mechanism reads:
%
\begin{center}
|... -jobname "|\textit{target}|" "|[\textit{flags}]%
[|\def\jobname{|\textit{dest}|}|]|\input{|\textit{main}|}"|
\end{center}

%%%%%%%%%%%%%%%%%%%%%%%%%%%%%%%%%%%%%%%%%%%%%%%%%%%%%%%%%%%%%%%%%%%%%%%%%%%%%%%%
\subsection{Manual Code}
\label{sec:manual}

In case one cannot be certain whether the definitions file |childdoc.def|
is installed on the target \TeX{} distribution
and one prefers not to ship it,
it is conceivable to paste a few relevant commands into the sources.

To that end, drop all statements |\input{childdoc.def}|
and perform the replacements as outlined below.
Instead of |\childdocmain{|\textit{main}|}| add the following code
to the top of the main file:
%
\begin{center}
\begin{tabular}{l}
|\||ifdefined\childdocname\endinput\||fi\newif\ifchilddoc|\\
|\edef\childdocname{\scantokens\expandafter{\jobname\noexpand}}|\\
|\def\childdocmain{|\textit{main}|}\||ifx\childdocmain\childdocname\||else|\\
|\childdoctrue\includeonly{\childdocname}\let\jobname\childdocmain\||fi|\\
\end{tabular}
\end{center}
%
Instead of |\childdocof{|\textit{main}|}| just include the main file
at the top of each child file:
%
\begin{center}
|\input{|\textit{main}|}|
\end{center}
%
A simple redirection |\childdocforward{|\textit{dest}|}| is achieved by:
%
\begin{center}
|\def\jobname{|\textit{dest}|}\input{\jobname}|
\end{center}
%
The redirection with prefix
|\childdocforwardprefix[|\textit{prefix}|]{|\textit{dest}|}|
is accomplished by:
%
\begin{center}
\begin{tabular}{l}
|{\edef\jobname{\scantokens\expandafter{\jobname\noexpand}}|\\
|\def\redirectjob |\textit{prefix}|#1~~~{\gdef\jobname{|\textit{dest}|#1}}|\\
|\expandafter\redirectjob\jobname~~~}\input{\jobname}|
\end{tabular}
\end{center}

In an alternative approach,
child documents can be compiled by a specific command line
without additional code or specific definitions:
%
\begin{center}
|... -jobname "|\textit{target}|" "|[\textit{flags}]%
|\includeonly{|\textit{dest}|}\input{|\textit{main}|}"|
\end{center}
%

%%%%%%%%%%%%%%%%%%%%%%%%%%%%%%%%%%%%%%%%%%%%%%%%%%%%%%%%%%%%%%%%%%%%%%%%%%%%%%%%
%%%%%%%%%%%%%%%%%%%%%%%%%%%%%%%%%%%%%%%%%%%%%%%%%%%%%%%%%%%%%%%%%%%%%%%%%%%%%%%%
\section{Information}

%%%%%%%%%%%%%%%%%%%%%%%%%%%%%%%%%%%%%%%%%%%%%%%%%%%%%%%%%%%%%%%%%%%%%%%%%%%%%%%%
\subsection{Copyright}

Copyright \copyright{} 2017--2018 Niklas Beisert

This work may be distributed and/or modified under the
conditions of the \LaTeX{} Project Public License, either version 1.3
of this license or (at your option) any later version.
The latest version of this license is in
  \url{http://www.latex-project.org/lppl.txt}
and version 1.3 or later is part of all distributions of \LaTeX{}
version 2005/12/01 or later.

This work has the LPPL maintenance status `maintained'.

The Current Maintainer of this work is Niklas Beisert.

This work consists of the files |README.txt|, |childdoc.ins| and |childdoc.dtx|
as well as the derived files |childdoc.def|, |cdocsamp.tex|
with |cdocsch1.tex|, |cdocsch2.tex|, |cdocspt3.tex|, |cdocspt4.tex|,
|cdocsdrf.tex|, |cdocsfn1.tex|, |cdocsfn2.tex|
as well as |childdoc.pdf|.

%%%%%%%%%%%%%%%%%%%%%%%%%%%%%%%%%%%%%%%%%%%%%%%%%%%%%%%%%%%%%%%%%%%%%%%%%%%%%%%%
\subsection{Files and Installation}

The package consists of the files:
%
\begin{center}
\begin{tabular}{ll}
    |README.txt|   & readme file \\
    |childdoc.ins| & installation file \\
    |childdoc.dtx| & source file \\
    |childdoc.def| & definition file \\
    |cdocsamp.tex| & sample main file \\
    |cdocsch1.tex| & sample include file \\
    |cdocsch2.tex| & sample include file \\
    |cdocspt3.tex| & sample part file \\
    |cdocspt4.tex| & sample part file \\
    |cdocsdrf.tex| & sample redirection file \\
    |cdocsfn1.tex| & sample redirection file \\
    |cdocsfn2.tex| & sample redirection file \\
    |childdoc.pdf| & manual
\end{tabular}
\end{center}
%
The distribution consists of the files
|README.txt|, |childdoc.ins| and |childdoc.dtx|.
%
\begin{itemize}
\item
Run (pdf)\LaTeX{} on |childdoc.dtx|
to compile the manual |childdoc.pdf| (this file).
\item
Run \LaTeX{} on |childdoc.ins| to create the definitions file |childdoc.def|
and the sample |cdocsamp.tex| with include files
|cdocsch1.tex|, |cdocsch2.tex|, |cdocspt3.tex|, |cdocspt4.tex|,
|cdocsdrf.tex|, |cdocsfn1.tex|, |cdocsfn2.tex|.
Then copy the file |childdoc.def| to an appropriate directory of your \LaTeX{}
distribution, e.g.\ \textit{texmf-root}|/tex/latex/childdoc|.
\end{itemize}

%%%%%%%%%%%%%%%%%%%%%%%%%%%%%%%%%%%%%%%%%%%%%%%%%%%%%%%%%%%%%%%%%%%%%%%%%%%%%%%%
\subsection{Related CTAN Packages}

There are several other packages which offer a similar functionality:
%
\begin{itemize}
\item
The packages
\href{http://ctan.org/pkg/docmute}{\textsf{docmute}},
\href{http://ctan.org/pkg/includex}{\textsf{includex}} and
\href{http://ctan.org/pkg/standalone}{\textsf{standalone}}
provide commands to include only the document body of
a child file thus allowing both files to be compiled individually.
\item
The packages \href{http://ctan.org/pkg/subdocs}{\textsf{subdocs}}
and \href{http://ctan.org/pkg/subfiles}{\textsf{subfiles}}
provide structures in which the main and child documents can be
encapsulated and allowing them to be compiled individually.
The inclusion mechanism is different from the conventional |\include|.
\item
The package \href{http://ctan.org/pkg/combine}{\textsf{combine}}
is an elaborate solution to combine several documents into one.
\end{itemize}
%
See also the CTAN topic \href{http://ctan.org/topic/subdocs}{\textsf{subdocs}}
for further related packages.
The present package differs from the above solutions in that
a document structure constructed with the conventional |\include| mechanism
just needs two extra commands at the top of every file
such that all constituent files can be compiled individually.

%%%%%%%%%%%%%%%%%%%%%%%%%%%%%%%%%%%%%%%%%%%%%%%%%%%%%%%%%%%%%%%%%%%%%%%%%%%%%%%%
%\subsection{Feature Suggestions}
%
%The following is a list of features which may be useful for future
%versions of this package:
%%
%\begin{itemize}
%\item
%\ldots
%\end{itemize}

%%%%%%%%%%%%%%%%%%%%%%%%%%%%%%%%%%%%%%%%%%%%%%%%%%%%%%%%%%%%%%%%%%%%%%%%%%%%%%%%
\subsection{Revision History}

%%%%%%%%%%%%%%%%%%%%%%%%%%%%%%%%%%%%%%%%
\paragraph{v2.0:} 2018/12/30

\begin{itemize}
\item
immediate forward processing
\item
added |\childdocby| mechanism
\item
manual restructured
\end{itemize}

%%%%%%%%%%%%%%%%%%%%%%%%%%%%%%%%%%%%%%%%
\paragraph{v1.6:} 2018/01/17

\begin{itemize}
\item
application for development of include files
\item
corrections to manual
\end{itemize}

%%%%%%%%%%%%%%%%%%%%%%%%%%%%%%%%%%%%%%%%
\paragraph{v1.5:} 2017/05/21

\begin{itemize}
\item
more complete structuring introduced
\item
|\childdocof| introduced
\item
|\childdoc| renamed to |\childdocmain|
\item
|\childredirect| renamed to |\childdocforward| and |\childdocforwardprefix|
and functionality expanded
\end{itemize}

%%%%%%%%%%%%%%%%%%%%%%%%%%%%%%%%%%%%%%%%
\paragraph{v1.0:} 2017/04/27

\begin{itemize}
\item
manual and install package
\item
first version published on CTAN
\end{itemize}

%%%%%%%%%%%%%%%%%%%%%%%%%%%%%%%%%%%%%%%%
\paragraph{v0.6:} 2017/04/26

\begin{itemize}
\item
redirection mechanism added
\end{itemize}

%%%%%%%%%%%%%%%%%%%%%%%%%%%%%%%%%%%%%%%%
\paragraph{v0.5:} 2017/04/26

\begin{itemize}
\item
functionality in definition file
\end{itemize}


%%%%%%%%%%%%%%%%%%%%%%%%%%%%%%%%%%%%%%%%%%%%%%%%%%%%%%%%%%%%%%%%%%%%%%%%%%%%%%%%
%%%%%%%%%%%%%%%%%%%%%%%%%%%%%%%%%%%%%%%%%%%%%%%%%%%%%%%%%%%%%%%%%%%%%%%%%%%%%%%%
%%%%%%%%%%%%%%%%%%%%%%%%%%%%%%%%%%%%%%%%%%%%%%%%%%%%%%%%%%%%%%%%%%%%%%%%%%%%%%%%
\appendix

\settowidth\MacroIndent{\rmfamily\scriptsize 000\ }

 \DocInput{childdoc.dtx}

\end{document}
%</driver>
% \fi
%
% %%%%%%%%%%%%%%%%%%%%%%%%%%%%%%%%%%%%%%%%%%%%%%%%%%%%%%%%%%%%%%%%%%%%%%%%%%%%%%
% %%%%%%%%%%%%%%%%%%%%%%%%%%%%%%%%%%%%%%%%%%%%%%%%%%%%%%%%%%%%%%%%%%%%%%%%%%%%%%
% \section{Sample}
%\iffalse
%<*samplemain>
%\fi
%
% The following presents a sample document
% with two chapters, two parts, a title page,
% a compile flag as well as three forwarding files to set the flag.
% It consists of eight |.tex| files:
% \begin{center}
% \begin{tabular}{ll}
% |cdocsamp.tex|&main file\\
% |cdocsch1.tex|&include file for chapter 1\\
% |cdocsch2.tex|&include file for chapter 2\\
% |cdocspt3.tex|&include file for part 3\\
% |cdocspt4.tex|&include file for part 4\\
% |cdocsdrf.tex|&forwarding file for main file in draft mode\\
% |cdocsfi1.tex|&forwarding file for final version of chapter 1\\
% |cdocsfi2.tex|&forwarding file for final version of chapter 2\\
% \end{tabular}
% \end{center}
% Each of the eight files can be compiled directly by the \LaTeX{} compiler.
%
% %%%%%%%%%%%%%%%%%%%%%%%%%%%%%%%%%%%%%%
% \paragraph{Main File.}
%
% The main file is called |cdocsamp.tex|.
%
% Load the \textsf{childdoc} definitions and
% declare the filename for the main document:
%    \begin{macrocode}
\input{childdoc.def}
\childdocmain{}
%    \end{macrocode}

% Optional override for |\version| flag:
%    \begin{macrocode}
%%\ifchilddoc\else\providecommand{\version}{draft}\fi
%    \end{macrocode}

% Define the default values for the |\version| flag
% (|final| for the main file and |draft| for childs):
%    \begin{macrocode}
\ifchilddoc
\providecommand{\version}{draft}
\else
\providecommand{\version}{final}
\fi
%    \end{macrocode}

% Load the standard document class:
%    \begin{macrocode}
\documentclass[12pt]{article}
%    \end{macrocode}

% Start the document body:
%    \begin{macrocode}
\begin{document}
%    \end{macrocode}

% Declare a title page.
% Print title, part of document being processed and version flag:
%    \begin{macrocode}
\addtocounter{page}{-1}
\begin{center}
{\LARGE\bfseries{}childdoc example\par}
\vspace{1cm}
\ifchilddoc
\ifchilddocmanual part\else chapter\fi:
`\childdocname' of `\childdocjob'\par
\else
main document: `\childdocjob'\par
\fi
version: \version\par
\end{center}
\newpage
%    \end{macrocode}

% Manually include selected file,
% otherwise process as usual:
%    \begin{macrocode}
\ifchilddocmanual
\section*{part `\childdocname'}
\input{\childdocname}
\else
%    \end{macrocode}

% Include the two chapters:
%    \begin{macrocode}
\include{cdocsch1}
\include{cdocsch2}
%    \end{macrocode}

% Include the two parts unless only chapters should be displayed:
%    \begin{macrocode}
\ifchilddoc\else
\section{part three}
\input{cdocspt3}
\section{part four}
\input{cdocspt4}
\fi
%    \end{macrocode}

% Process as usual until here:
%    \begin{macrocode}
\fi
%    \end{macrocode}

% End of document body:
%    \begin{macrocode}
\end{document}
%    \end{macrocode}
%\iffalse
%</samplemain>
%\fi
%
% %%%%%%%%%%%%%%%%%%%%%%%%%%%%%%%%%%%%%%
% \paragraph{Chapter Include Files.}
%
% The include files are called |cdocsch1.tex| and |cdocsch2.tex|.
%
%\iffalse
%<*samplechap1|samplechap2>
%\fi

% Optional override for |\version| flag:
%    \begin{macrocode}
%%\providecommand{\version}{final}
%    \end{macrocode}

% Include the main document:
%    \begin{macrocode}
\input{childdoc.def}
\childdocof{cdocsamp}
%    \end{macrocode}

%\iffalse
%</samplechap1|samplechap2>
%\fi
%
%\iffalse
%<*samplechap1>
%\fi
% Some text for chapter 1:
%    \begin{macrocode}
\section{one}
some text in chapter one
%    \end{macrocode}

%\iffalse
%</samplechap1>
%\fi
% Some text for chapter 2:
%\iffalse
%<*samplechap2>
%\fi
%    \begin{macrocode}
\section{two}
more text in chapter two
%    \end{macrocode}

%\iffalse
%</samplechap2>
%\fi
%
% %%%%%%%%%%%%%%%%%%%%%%%%%%%%%%%%%%%%%%
% \paragraph{Part Include Files.}
%
% The include files are called |cdocspt3.tex| and |cdocspt4.tex|.
%
%\iffalse
%<*samplepart3|samplepart4>
%\fi

% Optional override for |\version| flag:
%    \begin{macrocode}
%%\providecommand{\version}{final}
%    \end{macrocode}

% Include the main document:
%    \begin{macrocode}
\input{childdoc.def}
\childdocby{cdocsamp}
%    \end{macrocode}

%\iffalse
%</samplepart3|samplepart4>
%\fi
%
%\iffalse
%<*samplepart3>
%\fi
% Some text for part 3:
%    \begin{macrocode}
some text in part three
%    \end{macrocode}

%\iffalse
%</samplepart3>
%\fi
% Some text for part 4:
%\iffalse
%<*samplepart4>
%\fi
%    \begin{macrocode}
more text in part four
%    \end{macrocode}

%\iffalse
%</samplepart4>
%\fi
%
% %%%%%%%%%%%%%%%%%%%%%%%%%%%%%%%%%%%%%%
% \paragraph{Forwarding for a Complete Draft.}
%
% The following forwarding file |cdocsdrf.tex|
% compiles the main document in draft mode:
%\iffalse
%<*sampledraft>
%\fi
%    \begin{macrocode}
\def\version{draft}
\input{childdoc.def}
\childdocforward{cdocsamp}
%    \end{macrocode}

%\iffalse
%</sampledraft>
%\fi
%
% %%%%%%%%%%%%%%%%%%%%%%%%%%%%%%%%%%%%%%
% \paragraph{Forwarding for Final Version of the Chapters.}
%
% The following forwarding files |cdocsfn1.tex| and |cdocsfn2.tex|
% (with identical content)
% compile the final versions of the child documents
% |cdocsch1.tex| and |cdocsch2.tex|, respectively:
%\iffalse
%<*samplefinal>
%\fi
%    \begin{macrocode}
\def\version{final}
\input{childdoc.def}
\childdocforwardprefix[cdocsamp]{cdocsfn}{cdocsch}
%    \end{macrocode}

%\iffalse
%</samplefinal>
%\fi
%
% %%%%%%%%%%%%%%%%%%%%%%%%%%%%%%%%%%%%%%
% \paragraph{Command Line Processing.}
%
% The following three command lines generate the output files
% |cdocscld|, |cdocscl1| and |cdocscl2|
% which should be identical to
% |cdocsdrf|, |cdocsch1| and |cdocsfn2|, respectively:
% \begin{center}
% \begin{tabular}{l}
% |latex -jobname cdocscld \|\\
% |  "\def\version{draft}\input{childdoc.def}\childdocforward{cdocsamp}"|\\
% |latex -jobname cdocscl1 \|\\
% |  "\input{childdoc.def}\childdocforward[cdocsamp]{cdocsch1}"|\\
% |latex -jobname cdocscl2 \|\\
% |  "\def\version{final}\input{childdoc.def}\childdocforward{cdocsch2}"|
% \end{tabular}
% \end{center}
% Note that the trailing backslash on each first line
% merely continues the input to the second line
% (for convenient cut ant paste).
% Furthermore, the command |latex| can be replaced by any
% of its alternative versions such as |pdflatex|.
%
% %%%%%%%%%%%%%%%%%%%%%%%%%%%%%%%%%%%%%%%%%%%%%%%%%%%%%%%%%%%%%%%%%%%%%%%%%%%%%%
% %%%%%%%%%%%%%%%%%%%%%%%%%%%%%%%%%%%%%%%%%%%%%%%%%%%%%%%%%%%%%%%%%%%%%%%%%%%%%%
% \section{Implementation}
%\iffalse
%<*package>
%\fi
%
% This section describes the definitions file |childdoc.def|.

% The definitions cannot be loaded using |\usepackage| or |\RequirePackage|
% which has a mechanism to prevent loading a style file more than once.
% When loading the definitions by means of |\input|
% multiple instances have to be prevented manually:
%\iffalse
%This code needs to be before the `\ProvidesFile' directive
%which is defined at the beginning of this file.
%Therefore it is also placed there and commented out here.
%</package>
%<*discard>
%\fi
%    \begin{macrocode}
\ifdefined\childdocmain\endinput\fi
%    \end{macrocode}
%\iffalse
%</discard>
%<*package>
%\fi
%
% \macro{\ifchilddoc}
% \macro{\ifchilddocmanual}
% The conditional |\ifchilddoc| tells whether a
% child (true) or main (false) document is being compiled.
% The conditional |\ifchilddocmanual| tells whether
% the |\includeonly| mechanism is used (false) or
% the selection of child files must be performed manually (true).
% The definitions initialise to false:
%    \begin{macrocode}
\newif\ifchilddoc
\newif\ifchilddocmanual
%    \end{macrocode}

% \macro{\childdocname}
% \macro{\childdocjob}
% The macro |\childdocname| stores the name of the main document
% to be compiled. The macro |\childdocjob| stores the name of
% the document on which the \LaTeX{} compiler was originally invoked.
% The content of |\jobname| cannot be compared
% to filenames specified in the source due to different catcodes.
% The following code rescans |\jobname|, stores the result
% in |\childdocname| and saves a copy in |\childdocjob|:
%    \begin{macrocode}
\edef\childdocname{\scantokens\expandafter{\jobname\noexpand}}
\let\childdocjob\childdocname
%    \end{macrocode}

% \macro{\childdocdisable}
% The macro |\childdocdisable| prevents the main file
% from being processed more than once.
% At this stage, the main document command |\childdocmain|
% is assumed to be called once again where it should do nothing.
% Any subsequent call to it should prevent
% a secondary processing of the main document
% It overwrites the forwarding commands
% |\childdocof| and |\childdocforward|
% with empty macros to prevent further inclusions of the main document:
%    \begin{macrocode}
\newcommand{\childdocdisable}
{
  \renewcommand{\childdocmain}[1]{\renewcommand{\childdocmain}[1]{\endinput}}
  \renewcommand{\childdocof}[1]{}
  \renewcommand{\childdocby}[2][]{}
  \renewcommand{\childdocforward}[2][]{}
  \renewcommand{\childdocdisable}{}
}
%    \end{macrocode}

% \macro{\childdocmain}
% The macro |\childdocmain| is to be called at the top of the main file
% with nothing or the main filename (without extension) as argument.
% First, it breaks loops.
% If the argument is not empty and does not match |\childdocname|
% (which is set by the first inclusion of |childdoc.def|),
% |\ifchilddoc| is set to true, |\includeonly| is applied to the child file
% and |\jobname| is set to the main file
% (for proper handling of |.aux| files):
%    \begin{macrocode}
\newcommand{\childdocmain}[1]
{
  \childdocdisable\childdocmain{}
  \if?#1?\else
    \begingroup
      \def\childdoctmp{#1}
      \ifx\childdoctmp\childdocname
        \def\childdoctmp{}
      \else
        \def\childdoctmp
        {
          \childdoctrue
          \includeonly{\childdocname}
          \def\childdocjob{#1}
          \def\jobname{#1}
        }
      \fi
      \expandafter
    \endgroup
    \childdoctmp
  \fi
}
%    \end{macrocode}

% \macro{\childdocof}
% The command |\childdocof| redirects
% compilation to the main file |#1|.
%    \begin{macrocode}
\newcommand{\childdocof}[1]
{
  \childdocdisable
  \childdoctrue
  \includeonly{\childdocname}
  \def\jobname{#1}
  \def\childdocjob{#1}
  \input{#1}
}
%    \end{macrocode}

% \macro{\childdocby}
% The command |\childdocby| ....
%    \begin{macrocode}
\newcommand{\childdocby}[2][]
{
  \childdocdisable
  \childdoctrue
  \childdocmanualtrue
  \if?#1?\else
    \def\jobname{#2}
  \fi
  \def\childdocjob{#2}
  \input{#2}
  \endinput
}
%    \end{macrocode}

% \macro{\childdocforward}
% The command |\childdocforward| redirects
% compilation to the main file or
% (if the optional argument is given) a child file.
% Parameters are set as if the main file
% or a child file starting with |\childdocof| was compiled.
% Then compilation is handed over to the main file:
%    \begin{macrocode}
\newcommand{\childdocforward}[2][]
{
  \begingroup
    \if?#1?
      \def\childdoctmp
      {
        \def\childdocname{#2}
        \def\childdocjob{#2}
        \def\jobname{#2}
        \input{#2}
        \endinput
      }
    \else
      \def\childdoctmp
      {
        \childdocdisable
        \def\childdocname{#2}
        \childdoctrue
        \includeonly{#2}
        \def\childdocjob{#1}
        \def\jobname{#1}
        \input{#1}
        \endinput
      }
    \fi
    \expandafter
  \endgroup
  \childdoctmp
}
%    \end{macrocode}

% \macro{\childdocforwardprefix}
% The command |\childdocforwardprefix| redirects
% compilation to the main or a child file by means of a pattern.
% The prefix |#1| in the current filename is replaced by |#2|
% and the suffix of the current filename is kept
% (it is assumed that the filename does not contain the substring `|~~~|'
% which is used as a delimiter).
% Compilation is handed over to the new file by |\childdocforward|:
%    \begin{macrocode}
\newcommand{\childdocforwardprefix}[3][]
{
  \begingroup
    \def\childdocextract #2##1~~~{\def\childdoctmp{\childdocforward[#1]{#3##1}}}
    \expandafter\childdocextract\childdocname~~~
    \expandafter
  \endgroup
  \childdoctmp
}
%    \end{macrocode}

% \macro{\childdoc}
% The deprecated macro |\childdoc| is a legacy version of |\childdocmain|:
%    \begin{macrocode}
\newcommand{\childdoc}{\childdocmain}
%    \end{macrocode}

% \macro{\childdocredirect}
% The deprecated macro |\childdocredirect| is a legacy version
% of |\childdocforward| and |\childdocforwardprefix|:
%    \begin{macrocode}
\newcommand{\childdocredirect}[2][]
{
  \begingroup
    \if?#1?
      \def\childdoctmp{\childdocforward{#2}}
    \else
      \def\childdoctmp{\childdocforwardprefix{#1}{#2}}
    \fi
    \expandafter
  \endgroup
  \childdoctmp
}
%    \end{macrocode}

%\iffalse
%</package>
%\fi
%
\endinput
|\\
|\childdocby{|\textit{main}|}|\\
\end{tabular}
\end{center}
%
The directive |\childdocby| is similar to |\childdocof|
described in \secref{sec:include},
but the subsequent selection of content must be done manually.
To that end, both |\ifchilddoc| and |\ifchilddocmanual|
will be true upon processing of a part,
and the name of the part is stored in |\childdocname|.
Note that |\jobname| will be set to the filename of the current part
so that each part receives an individual |.aux| file
that does not interfere with the |.aux| file(s) of the main document.
This behaviour can be altered by the alternative form
|\childdocby[*]{|\textit{main}|}| (with a non-empty optional argument)
which uses the |.aux| file of the main document
by setting |\jobname| to \textit{main}.

%%%%%%%%%%%%%%%%%%%%%%%%%%%%%%%%%%%%%%%%%%%%%%%%%%%%%%%%%%%%%%%%%%%%%%%%%%%%%%%%
\subsection{Driver Development}
\label{sec:driver}

The \textsf{childdoc} mechanism can also be use for the development
of definition files such as \LaTeX{} styles or classes.
This case differs from the above setup with multiple parts
included by |\include| in that no |\includeonly| should be invoked.
This can be achieved by starting the include file
(before |\ProvidesPackage|) with:
%
\begin{center}
\begin{tabular}{l}
|% \iffalse
%
% childdoc.dtx Copyright (C) 2017-2018 Niklas Beisert
%
% This work may be distributed and/or modified under the
% conditions of the LaTeX Project Public License, either version 1.3
% of this license or (at your option) any later version.
% The latest version of this license is in
%   http://www.latex-project.org/lppl.txt
% and version 1.3 or later is part of all distributions of LaTeX
% version 2005/12/01 or later.
%
% This work has the LPPL maintenance status `maintained'.
%
% The Current Maintainer of this work is Niklas Beisert.
%
% This work consists of the files childdoc.dtx and childdoc.ins
% and the derived files childdoc.def and cdocsamp.tex with
% cdocsch1.tex, cdocsch2.tex, cdocsdrf.tex, cdocsfn1.tex, cdocsfn2.tex.
%
%<package>\ifdefined\childdocmain\endinput\fi
%<package>\ProvidesFile{childdoc.def}[2018/12/30 v2.0 child document driver]
%<samplemain>\ProvidesFile{cdocsamp.tex}[2018/12/30 v2.0 sample for childdoc]
%<*driver>
%\ProvidesFile{childdoc.drv}[2018/12/30 v2.0 childdoc reference manual file]
\PassOptionsToClass{10pt,a4paper}{article}
\documentclass{ltxdoc}

\usepackage[margin=35mm]{geometry}
\usepackage{hyperref}
\usepackage{hyperxmp}
\usepackage[usenames]{color}

\hypersetup{colorlinks=true}
\hypersetup{pdfstartview=FitH}
\hypersetup{pdfpagemode=UseNone}
\hypersetup{pdfsource={}}
\hypersetup{pdflang={en-UK}}
\hypersetup{pdfcopyright={Copyright 2017-2018 Niklas Beisert.
  This work may be distributed and/or modified under the
  conditions of the LaTeX Project Public License, either version 1.3
  of this license or (at your option) any later version.}}
\hypersetup{pdflicenseurl={http://www.latex-project.org/lppl.txt}}
\hypersetup{pdfcontactaddress={ETH Zurich, ITP, HIT K,
  Wolfgang-Pauli-Strasse 27}}
\hypersetup{pdfcontactpostcode={8093}}
\hypersetup{pdfcontactcity={Zurich}}
\hypersetup{pdfcontactcountry={Switzerland}}
\hypersetup{pdfcontactemail={nbeisert@itp.phys.ethz.ch}}
\hypersetup{pdfcontacturl={http://people.phys.ethz.ch/\xmptilde nbeisert/}}

\newcommand{\secref}[1]{\hyperref[#1]{section \ref*{#1}}}

\parskip1ex
\parindent0pt
\let\olditemize\itemize
\def\itemize{\olditemize\parskip0pt}

\begin{document}

\title{The \textsf{childdoc} Package}
\hypersetup{pdftitle={The childdoc Package}}
\author{Niklas Beisert\\[2ex]
  Institut f\"ur Theoretische Physik\\
  Eidgen\"ossische Technische Hochschule Z\"urich\\
  Wolfgang-Pauli-Strasse 27, 8093 Z\"urich, Switzerland\\[1ex]
  \href{mailto:nbeisert@itp.phys.ethz.ch}
  {\texttt{nbeisert@itp.phys.ethz.ch}}}
\hypersetup{pdfauthor={Niklas Beisert}}
\hypersetup{pdfsubject={Manual for the LaTeX2e Package childdoc}}
\date{30 December 2018, \textsf{v2.0}}
\maketitle

\begin{abstract}\noindent
\textsf{childdoc} is a \LaTeXe{} package
that enables the direct compilation
of document sections included by |\include|
to individual files.
\end{abstract}

\begingroup
\parskip0ex
\tableofcontents
\endgroup

%%%%%%%%%%%%%%%%%%%%%%%%%%%%%%%%%%%%%%%%%%%%%%%%%%%%%%%%%%%%%%%%%%%%%%%%%%%%%%%%
%%%%%%%%%%%%%%%%%%%%%%%%%%%%%%%%%%%%%%%%%%%%%%%%%%%%%%%%%%%%%%%%%%%%%%%%%%%%%%%%
\section{Introduction}

\LaTeX{} provides a mechanism to structure a large document (such as a book)
into a main file and several child files (containing the chapters)
using the |\include| command.
This mechanism is beneficial for documents
which span hundreds of pages in order to
make the source file(s) more manageable.
Moreover, compilation can be restricted to
selected child files by means of the |\includeonly| command.
The latter feature can be used to reduce the compilation time while editing
(this was significantly more useful in the earlier days of \LaTeX{})
or to generate a smaller document which is easier to navigate.
Another application of |\includeonly| is to generate
documents consisting of selected parts of the complete document.

However, there are a few drawbacks of the plain |\include| mechanism:
\begin{itemize}
\item
The child files cannot be compiled on their own,
they can only be compiled via the main file.
A naive editing environment
(such as a text editor with an option
to have the current file processed by \LaTeX)
may require one to switch to the main file before compiling;
attempting to compile the child file produces errors.
\item
The main file must be modified (each time)
to adjust the |\includeonly| command
to the present needs. This easily leaves the main file in a messy state.
\item
The generated document will always carry the filename
of the main document. This is inconvenient if
several child files are to be compiled and
to be kept for distribution.
\end{itemize}

The present package provides a simple interface
to make child files individually compilable by \LaTeX{}.
Compiling a child file then has the same effect as compiling
the main file with an |\includeonly| command
to select the appropriate child.
Moreover the generated document will carry the name of the child
rather than the main file.
This resolves all three above issues.

This feature is meant to make the editing of books,
thesis documents and lecture notes somewhat more convenient.
However, the package can also be used efficiently for
composing a series of documents (such as exercise sheets)
which are typically distributed individually.
It then assists the author in generating the individual documents
(potentially in different versions)
as well as a document containing the collected series.
Another application is in developing style files
or other kinds of included material
where compilation of the style file could redirect
to a sample or test file.

%%%%%%%%%%%%%%%%%%%%%%%%%%%%%%%%%%%%%%%%%%%%%%%%%%%%%%%%%%%%%%%%%%%%%%%%%%%%%%%%
%%%%%%%%%%%%%%%%%%%%%%%%%%%%%%%%%%%%%%%%%%%%%%%%%%%%%%%%%%%%%%%%%%%%%%%%%%%%%%%%
\section{Usage}

First of all, the package \textsf{childdoc} is \emph{not} a standard
\LaTeXe{} |.sty| style file! Therefore it needs to be invoked in
a non-standard way.

%%%%%%%%%%%%%%%%%%%%%%%%%%%%%%%%%%%%%%%%%%%%%%%%%%%%%%%%%%%%%%%%%%%%%%%%%%%%%%%%
\subsection{Included Files}
\label{sec:include}

%%%%%%%%%%%%%%%%%%%%%%%%%%%%%%%%%%%%%%%%
\DescribeMacro{\childdocmain}
To use the package, add the commands
\begin{center}
\begin{tabular}{l}
|\input{childdoc.def}|\\
|\childdocmain{}|\\
\end{tabular}
\end{center}
at the very top of the main \LaTeX{} file,
in particular \emph{before} the |\documentclass| statement!
The argument of |\childdocmain| should be left empty
(but it must be present).

%%%%%%%%%%%%%%%%%%%%%%%%%%%%%%%%%%%%%%%%
\DescribeMacro{\childdocof}
Furthermore, add the commands
\begin{center}
\begin{tabular}{l}
|\input{childdoc.def}|\\
|\childdocof{|\textit{main}|}|\\
\end{tabular}
\end{center}
at the top of every child file \textit{child}
which is included by |\include{|\textit{child}|}|
from within the main file
(or at least for those files to be compiled individually).
The argument \textit{main} must be the filename of the main file.

There are a couple of
considerations in setting up the main and child documents:

%%%%%%%%%%%%%%%%%%%%%%%%%%%%%%%%%%%%%%%%
\paragraph{Restrictions.}

Please note the following restrictions:
\begin{itemize}
\item
|\childdocmain| must be called with one argument \textit{main}
to ensure compatibility with earlier version of the package.
It must either be empty (|\childdocmain{}|)
or precisely match the filename of the main file in which it is specified.
See \secref{sec:detection} for further information.
\item
The filename \textit{main} must be specified without the |.tex| extension.
\item
The filename \textit{main} is case sensitive
(even in case-insensitive file systems)
due to internal string comparison.
\item
The argument \textit{main} should be fully expanded, it cannot be a macro.
\item
Subdirectories and special characters should be avoided in filenames.
\item
The command |\childdocmain{|\textit{main}|}| must be followed by a whitespace.
It should not be followed immediately by another command
or by a comment mark `|%|'.
This is because the \TeX{} parser reads the token immediately following
the argument of |\childdocmain| and puts it
at the beginning of every child section;
however, a white\-space is ignored.
\end{itemize}

%%%%%%%%%%%%%%%%%%%%%%%%%%%%%%%%%%%%%%%%
\paragraph{Content of Main File.}

It is advisable to place all content in the child files included by |\include|.
Any output contained in the main file will appear in all child documents
unless suppressed manually;
it cannot be suppressed automatically by the |\includeonly| directive
and thus should normally be avoided.
A method to include some content in the main file
by means of conditional processing is described in \secref{sec:conditional}.

%%%%%%%%%%%%%%%%%%%%%%%%%%%%%%%%%%%%%%%%
\paragraph{Page Numbering.}

When only a part of the document is compiled,
the appropriate numbering of pages
(as well as other status parameters)
is determined from the |.aux| files.
The latter contain information from previous passes.
However this information needs to propagate through
all intermediate child documents.
Therefore the page numbering in child documents may well
be inconsistent until the complete document is compiled at least once.

A useful (if unconventional) way to always ensure a consistent
page numbering is to restart the numbering in each child document
and denote the pages by `\textit{child}|.|\textit{page}'
where \textit{child} represents the chapter/section number of the child file.
This can be achieved by the command
|\numberwithin{page}{|\textit{child}|}|
of the \textsf{amsmath} package
where \textit{child} can be |chapter| or |section|
depending on the chosen structuring.
Alternatively, one can modify the macro |\thepage| appropriately
and reset the counter |page| at the start of each child file.

%%%%%%%%%%%%%%%%%%%%%%%%%%%%%%%%%%%%%%%%%%%%%%%%%%%%%%%%%%%%%%%%%%%%%%%%%%%%%%%%
\subsection{Conditional Processing}
\label{sec:conditional}

The package provides a mechanism to compile different versions
of a document. To customise the versions further some conditional processing
can come in handy to distinguish which version is being compiled.
The package provides two macros to describe the compilation context:

%%%%%%%%%%%%%%%%%%%%%%%%%%%%%%%%%%%%%%%%
\DescribeMacro{\ifchilddoc}
The conditional |\ifchilddoc| distinguishes between the compilation of
child documents and the main document:
%
\begin{center}
|\ifchilddoc |\textit{child-code}| |[|\||else |\textit{main-code}]| \||fi|
\end{center}

%%%%%%%%%%%%%%%%%%%%%%%%%%%%%%%%%%%%%%%%
\DescribeMacro{\childdocname}
\DescribeMacro{\childdocjob}
The macro |\childdocname| contains the filename (without extension)
of the main or child file being processed.
Note that |\childdocjob| will always contain the name of the main file.

%%%%%%%%%%%%%%%%%%%%%%%%%%%%%%%%%%%%%%%%
\paragraph{Title Page.}

Conditional processing can be used to include a title or banner page
in the main document when proper precautions are taken.
Importantly, the code in the main file should ensure that the page counter
(as well as other status parameters which are stored in the |.aux| files)
takes the same value after the conditional processing.
Otherwise the page numbers may take divergent values
depending on which part is compiled.

For example, a title page could be declared by:
%
\begin{center}
\begin{tabular}{l}
|\ifchilddoc\||else|\\
|\addtocounter{page}{-1}|\\
\textit{code for title page}\\
|\newpage|\\
|\||fi|
\end{tabular}
\end{center}
%
A banner page for the child documents can be generated by:
%
\begin{center}
\begin{tabular}{l}
|\ifchilddoc|\\
|\addtocounter{page}{-1}|\\
\textit{code for banner page}\\
|\newpage|\\
|\||fi|
\end{tabular}
\end{center}
%
Here one could write a message such as:
\begin{center}
|This is the part \childdocname{} of \childdocjob{}.|
\end{center}

%%%%%%%%%%%%%%%%%%%%%%%%%%%%%%%%%%%%%%%%%%%%%%%%%%%%%%%%%%%%%%%%%%%%%%%%%%%%%%%%
\subsection{Flags}
\label{sec:flags}

The package makes it easy to generate different versions
of the main or child documents.
To this end compilation flags can be defined
and assigned different default values.
They will be particularly useful in conjunction
with the forwarding mechanism described in \secref{sec:forward}.

For example, it may be useful to have a flag |\version|
which can be set to |draft| or |final|.
The document source will contain some conditional code
depending on the value of |\version|.
Suppose further, the flag should default to |final| for the main file
and to |draft| for child files
which is a natural assignment for editing the document.
This is achieved by placing the following code
in the preamble of the main document
(below the |\childdocmain| directive):
%
\begin{center}
\begin{tabular}{l}
|\ifchilddoc|\\
|\providecommand{\version}{draft}|\\
|\||else|\\
|\providecommand{\version}{final}|\\
|\||fi|
\end{tabular}
\end{center}
%
The definition by |\providecommand| makes sure
that previous definitions are not overwritten.
Further statements |\providecommand{\version}{...}|
can thus be added before the above code to override it.

For the main file, one might add a line
(between |\childdocmain| and the above block)
%
\begin{center}
|%\ifchilddoc\||else\providecommand{\version}{draft}\||fi|
\end{center}
%
which can be uncommented to produce a draft version.
Likewise one can add a line to the very top of a child file
(above the |\childdocof{|\textit{main}|}| directive)
%
\begin{center}
|%\providecommand{\version}{final}|
\end{center}
%
which can be uncommented to produce the final version of this child document.

%%%%%%%%%%%%%%%%%%%%%%%%%%%%%%%%%%%%%%%%%%%%%%%%%%%%%%%%%%%%%%%%%%%%%%%%%%%%%%%%
\subsection{Forwarding}
\label{sec:forward}

Different versions of the main or child documents
using compilation flags as described in \secref{sec:flags}
can be (permanently) stored in different files
for convenient compilation, viewing and distribution.
To this end, the package defines a command
to pass on compilation to a different file:

%%%%%%%%%%%%%%%%%%%%%%%%%%%%%%%%%%%%%%%%
\DescribeMacro{\childdocforward}
The command |\childdocforward| redirects processing to
another source file:
%
\begin{center}
\begin{tabular}{l}
|\input{childdoc.def}|\\
|\childdocforward[|\textit{main}|]{|\textit{dest}|}|\\
\end{tabular}
\end{center}
%
The argument \textit{dest} is the destination file
(without extension).
It should be the main file or one of the child files.
Note that further \textsf{childdoc} directives
such as |\childdocof| and |\childdocforward|
in the indicated file will be processed in this form.
The optional argument \textit{main}
passes on directly to the main file \textit{main}
while pretending to compile the child \textit{dest}.
This form behaves as if \textit{dest}
issues |\childdocof{|\textit{main}|}| right away,
and no further \textsf{childdoc} directives will be processed.

%%%%%%%%%%%%%%%%%%%%%%%%%%%%%%%%%%%%%%%%
\DescribeMacro{\...prefix}
In the alternative form |\childdocforwardprefix|,
%
\begin{center}
\begin{tabular}{l}
|\input{childdoc.def}|\\
|\childdocforwardprefix[|\textit{main}|]{|\textit{prefix}|}{|\textit{dest}|}|
\end{tabular}
\end{center}
%
the destination file is determined by a pattern
depending on the current file:
To make this work, the current file must be called
`{\textit{prefix}\hspace{0.2em}\textit{suffix}}'
with \textit{prefix} matching precisely the argument.
Processing is then passed on to the file
`{\textit{dest}\hspace{0.2em}\textit{suffix}}'.
Surely, the same effect is achieved by
directly specifying the
argument `{\textit{dest}\hspace{0.2em}\textit{suffix}}'
in the first form.
However, that requires to set up a different file
for each child. With the alternative form of the command
all these files can have exactly the same content
which simplifies setting them up and maintaining them.

For example, the following file |draft.tex|
with a compilation flag |\version| as described in \secref{sec:flags}
compiles the main document as a draft:
%
\begin{center}
\begin{tabular}{l}
|\def\version{draft}|\\
|\input{childdoc.def}|\\
|\childdocforward{|\textit{main}|}|
\end{tabular}
\end{center}
%
Likewise, the following files |final|\textit{nn}|.tex|
compile the final version of the child document
|child|\textit{nn}|.tex|:
%
\begin{center}
\begin{tabular}{l}
|\def\version{final}|\\
|\input{childdoc.def}|\\
|\childdocforwardprefix{final}{child}|
\end{tabular}
\end{center}
%

Note that when several versions of a main file and/or of each child file
are to be generated, it may be convenient to set up a |Makefile| or
shell script to automatise the process.

%%%%%%%%%%%%%%%%%%%%%%%%%%%%%%%%%%%%%%%%%%%%%%%%%%%%%%%%%%%%%%%%%%%%%%%%%%%%%%%%
\subsection{Command Line Processing}
\label{sec:commandline}

The effect of redirection files can also be achieved by invoking
the \LaTeX{} compiler with a more elaborate command line.
Most conveniently this should be done as part
of a shell script or a |Makefile|.

When using \textsf{childdoc} in the main file, the following
command lines effectively perform a redirection
(note that depending on the shell being used,
backslashes may have to be doubled: `|\|' $\to$ `|\\|'):
%
\begin{center}
|... -jobname "|\textit{target}|" |\\|"|[\textit{flags}]%
|\input{childdoc.def}\childdocforward[|\textit{main}|]{|\textit{dest}|}"|
\end{center}
%
Here \textit{target} is the name of the output file,
\textit{main} is the name of the main file
and \textit{dest} is the name of the main or child file to be processed
(all filenames without extensions).
The optional argument \textit{main} can be omitted
if \textit{main} matches \textit{dest}.
Optionally, compilation \textit{flags} can be defined via |\def| commands.
This command line makes the \TeX{} engine believe
it is compiling the file \textit{target}
whose content is specified as the latter parameter.
The provided code then forwards the processing to
\textit{main} or \textit{dest} as described in \secref{sec:forward}.

%%%%%%%%%%%%%%%%%%%%%%%%%%%%%%%%%%%%%%%%%%%%%%%%%%%%%%%%%%%%%%%%%%%%%%%%%%%%%%%%
\subsection{Include by Input}
\label{sec:input}

Including child documents by |\include| has some restrictions by design.
Most notably, the content of a child document always occupies
its own set of pages; pages cannot be shared between child documents.
Usually, this behaviour makes perfect sense
because each child document contain an essential part of the document.
However, in some situations it may be desirable to compose
a document from a collection of parts
without having mandatory page breaks between then.
For this case, the package
provides a mechanism to include parts
by |\input| which can also be processed individually.
However, by construction this mechanism
requires manual handling of the content to be output.

%%%%%%%%%%%%%%%%%%%%%%%%%%%%%%%%%%%%%%%%
\DescribeMacro{\ifchilddocmanual}
The main file should be prepared as usual, see \secref{sec:include}.
However, the document body must make a distinction
between processing of an individual part and of the main document, e.g.:
%
\begin{center}
\begin{tabular}{l}
|\ifchilddocmanual|\\
|\input{\childdocname}|\\
|\||else|\\
\textit{document body with }|\input{|\textit{part}|}|\\
|\||fi|
\end{tabular}
\end{center}
%
The conditional |\ifchilddocmanual| is true whenever
a part to be included by |\input| is being compiled,
and the name of the part is stored in |\childdocname|.

%%%%%%%%%%%%%%%%%%%%%%%%%%%%%%%%%%%%%%%%
\DescribeMacro{\childdocby}
Each part to be included by |\input| should start with:
%
\begin{center}
\begin{tabular}{l}
|\input{childdoc.def}|\\
|\childdocby{|\textit{main}|}|\\
\end{tabular}
\end{center}
%
The directive |\childdocby| is similar to |\childdocof|
described in \secref{sec:include},
but the subsequent selection of content must be done manually.
To that end, both |\ifchilddoc| and |\ifchilddocmanual|
will be true upon processing of a part,
and the name of the part is stored in |\childdocname|.
Note that |\jobname| will be set to the filename of the current part
so that each part receives an individual |.aux| file
that does not interfere with the |.aux| file(s) of the main document.
This behaviour can be altered by the alternative form
|\childdocby[*]{|\textit{main}|}| (with a non-empty optional argument)
which uses the |.aux| file of the main document
by setting |\jobname| to \textit{main}.

%%%%%%%%%%%%%%%%%%%%%%%%%%%%%%%%%%%%%%%%%%%%%%%%%%%%%%%%%%%%%%%%%%%%%%%%%%%%%%%%
\subsection{Driver Development}
\label{sec:driver}

The \textsf{childdoc} mechanism can also be use for the development
of definition files such as \LaTeX{} styles or classes.
This case differs from the above setup with multiple parts
included by |\include| in that no |\includeonly| should be invoked.
This can be achieved by starting the include file
(before |\ProvidesPackage|) with:
%
\begin{center}
\begin{tabular}{l}
|\input{childdoc.def}|\\
|\childdocforward{|\textit{main}|}|\\
\end{tabular}
\end{center}
%
or alternatively with:
%
\begin{center}
\begin{tabular}{l}
|\input{childdoc.def}|\\
|\childdocby{|\textit{main}|}|\\
\end{tabular}
\end{center}
%
Both forms have slightly different effects as described above.
The main file is prepared as usual, see \secref{sec:include}.

%%%%%%%%%%%%%%%%%%%%%%%%%%%%%%%%%%%%%%%%%%%%%%%%%%%%%%%%%%%%%%%%%%%%%%%%%%%%%%%%
\subsection{Legacy Detection}
\label{sec:detection}

The directive |\childdocmain| in the main file can detect
whether the complete document or merely a child is to be compiled
even without using the directive |\childdocof|.
This method is deprecated because it is less robust
and there is no compelling reason to use it;
it is merely provided for backward compatibility
and it may be removed in future versions.

If the detection mechanism is to be used,
it is mandatory to correctly specify
the filename of the main file as the argument of |\childdocmain|:
%
\begin{center}
\begin{tabular}{l}
|\input{childdoc.def}|\\
|\childdocmain{|\textit{main}|}|\\
\end{tabular}
\end{center}
%
If |\jobname| does not match the argument \textit{main} of |\childdocmain|,
it is assumed that |\jobname| points to the child file to be compiled.
When using |\childdocmain| with the main file specified as argument,
it suffices to start a child file
with just |\input{|\textit{main}|}|
without loading of the package and using |\childdocof|.
If instead all processing is done
with the appropriate \textsf{childdoc} directives,
the argument of \textit{main} of |\childdocmain| can be empty.

An alternative version of the command line processing described
in \secref{sec:commandline} using the detection mechanism reads:
%
\begin{center}
|... -jobname "|\textit{target}|" "|[\textit{flags}]%
[|\def\jobname{|\textit{dest}|}|]|\input{|\textit{main}|}"|
\end{center}

%%%%%%%%%%%%%%%%%%%%%%%%%%%%%%%%%%%%%%%%%%%%%%%%%%%%%%%%%%%%%%%%%%%%%%%%%%%%%%%%
\subsection{Manual Code}
\label{sec:manual}

In case one cannot be certain whether the definitions file |childdoc.def|
is installed on the target \TeX{} distribution
and one prefers not to ship it,
it is conceivable to paste a few relevant commands into the sources.

To that end, drop all statements |\input{childdoc.def}|
and perform the replacements as outlined below.
Instead of |\childdocmain{|\textit{main}|}| add the following code
to the top of the main file:
%
\begin{center}
\begin{tabular}{l}
|\||ifdefined\childdocname\endinput\||fi\newif\ifchilddoc|\\
|\edef\childdocname{\scantokens\expandafter{\jobname\noexpand}}|\\
|\def\childdocmain{|\textit{main}|}\||ifx\childdocmain\childdocname\||else|\\
|\childdoctrue\includeonly{\childdocname}\let\jobname\childdocmain\||fi|\\
\end{tabular}
\end{center}
%
Instead of |\childdocof{|\textit{main}|}| just include the main file
at the top of each child file:
%
\begin{center}
|\input{|\textit{main}|}|
\end{center}
%
A simple redirection |\childdocforward{|\textit{dest}|}| is achieved by:
%
\begin{center}
|\def\jobname{|\textit{dest}|}\input{\jobname}|
\end{center}
%
The redirection with prefix
|\childdocforwardprefix[|\textit{prefix}|]{|\textit{dest}|}|
is accomplished by:
%
\begin{center}
\begin{tabular}{l}
|{\edef\jobname{\scantokens\expandafter{\jobname\noexpand}}|\\
|\def\redirectjob |\textit{prefix}|#1~~~{\gdef\jobname{|\textit{dest}|#1}}|\\
|\expandafter\redirectjob\jobname~~~}\input{\jobname}|
\end{tabular}
\end{center}

In an alternative approach,
child documents can be compiled by a specific command line
without additional code or specific definitions:
%
\begin{center}
|... -jobname "|\textit{target}|" "|[\textit{flags}]%
|\includeonly{|\textit{dest}|}\input{|\textit{main}|}"|
\end{center}
%

%%%%%%%%%%%%%%%%%%%%%%%%%%%%%%%%%%%%%%%%%%%%%%%%%%%%%%%%%%%%%%%%%%%%%%%%%%%%%%%%
%%%%%%%%%%%%%%%%%%%%%%%%%%%%%%%%%%%%%%%%%%%%%%%%%%%%%%%%%%%%%%%%%%%%%%%%%%%%%%%%
\section{Information}

%%%%%%%%%%%%%%%%%%%%%%%%%%%%%%%%%%%%%%%%%%%%%%%%%%%%%%%%%%%%%%%%%%%%%%%%%%%%%%%%
\subsection{Copyright}

Copyright \copyright{} 2017--2018 Niklas Beisert

This work may be distributed and/or modified under the
conditions of the \LaTeX{} Project Public License, either version 1.3
of this license or (at your option) any later version.
The latest version of this license is in
  \url{http://www.latex-project.org/lppl.txt}
and version 1.3 or later is part of all distributions of \LaTeX{}
version 2005/12/01 or later.

This work has the LPPL maintenance status `maintained'.

The Current Maintainer of this work is Niklas Beisert.

This work consists of the files |README.txt|, |childdoc.ins| and |childdoc.dtx|
as well as the derived files |childdoc.def|, |cdocsamp.tex|
with |cdocsch1.tex|, |cdocsch2.tex|, |cdocspt3.tex|, |cdocspt4.tex|,
|cdocsdrf.tex|, |cdocsfn1.tex|, |cdocsfn2.tex|
as well as |childdoc.pdf|.

%%%%%%%%%%%%%%%%%%%%%%%%%%%%%%%%%%%%%%%%%%%%%%%%%%%%%%%%%%%%%%%%%%%%%%%%%%%%%%%%
\subsection{Files and Installation}

The package consists of the files:
%
\begin{center}
\begin{tabular}{ll}
    |README.txt|   & readme file \\
    |childdoc.ins| & installation file \\
    |childdoc.dtx| & source file \\
    |childdoc.def| & definition file \\
    |cdocsamp.tex| & sample main file \\
    |cdocsch1.tex| & sample include file \\
    |cdocsch2.tex| & sample include file \\
    |cdocspt3.tex| & sample part file \\
    |cdocspt4.tex| & sample part file \\
    |cdocsdrf.tex| & sample redirection file \\
    |cdocsfn1.tex| & sample redirection file \\
    |cdocsfn2.tex| & sample redirection file \\
    |childdoc.pdf| & manual
\end{tabular}
\end{center}
%
The distribution consists of the files
|README.txt|, |childdoc.ins| and |childdoc.dtx|.
%
\begin{itemize}
\item
Run (pdf)\LaTeX{} on |childdoc.dtx|
to compile the manual |childdoc.pdf| (this file).
\item
Run \LaTeX{} on |childdoc.ins| to create the definitions file |childdoc.def|
and the sample |cdocsamp.tex| with include files
|cdocsch1.tex|, |cdocsch2.tex|, |cdocspt3.tex|, |cdocspt4.tex|,
|cdocsdrf.tex|, |cdocsfn1.tex|, |cdocsfn2.tex|.
Then copy the file |childdoc.def| to an appropriate directory of your \LaTeX{}
distribution, e.g.\ \textit{texmf-root}|/tex/latex/childdoc|.
\end{itemize}

%%%%%%%%%%%%%%%%%%%%%%%%%%%%%%%%%%%%%%%%%%%%%%%%%%%%%%%%%%%%%%%%%%%%%%%%%%%%%%%%
\subsection{Related CTAN Packages}

There are several other packages which offer a similar functionality:
%
\begin{itemize}
\item
The packages
\href{http://ctan.org/pkg/docmute}{\textsf{docmute}},
\href{http://ctan.org/pkg/includex}{\textsf{includex}} and
\href{http://ctan.org/pkg/standalone}{\textsf{standalone}}
provide commands to include only the document body of
a child file thus allowing both files to be compiled individually.
\item
The packages \href{http://ctan.org/pkg/subdocs}{\textsf{subdocs}}
and \href{http://ctan.org/pkg/subfiles}{\textsf{subfiles}}
provide structures in which the main and child documents can be
encapsulated and allowing them to be compiled individually.
The inclusion mechanism is different from the conventional |\include|.
\item
The package \href{http://ctan.org/pkg/combine}{\textsf{combine}}
is an elaborate solution to combine several documents into one.
\end{itemize}
%
See also the CTAN topic \href{http://ctan.org/topic/subdocs}{\textsf{subdocs}}
for further related packages.
The present package differs from the above solutions in that
a document structure constructed with the conventional |\include| mechanism
just needs two extra commands at the top of every file
such that all constituent files can be compiled individually.

%%%%%%%%%%%%%%%%%%%%%%%%%%%%%%%%%%%%%%%%%%%%%%%%%%%%%%%%%%%%%%%%%%%%%%%%%%%%%%%%
%\subsection{Feature Suggestions}
%
%The following is a list of features which may be useful for future
%versions of this package:
%%
%\begin{itemize}
%\item
%\ldots
%\end{itemize}

%%%%%%%%%%%%%%%%%%%%%%%%%%%%%%%%%%%%%%%%%%%%%%%%%%%%%%%%%%%%%%%%%%%%%%%%%%%%%%%%
\subsection{Revision History}

%%%%%%%%%%%%%%%%%%%%%%%%%%%%%%%%%%%%%%%%
\paragraph{v2.0:} 2018/12/30

\begin{itemize}
\item
immediate forward processing
\item
added |\childdocby| mechanism
\item
manual restructured
\end{itemize}

%%%%%%%%%%%%%%%%%%%%%%%%%%%%%%%%%%%%%%%%
\paragraph{v1.6:} 2018/01/17

\begin{itemize}
\item
application for development of include files
\item
corrections to manual
\end{itemize}

%%%%%%%%%%%%%%%%%%%%%%%%%%%%%%%%%%%%%%%%
\paragraph{v1.5:} 2017/05/21

\begin{itemize}
\item
more complete structuring introduced
\item
|\childdocof| introduced
\item
|\childdoc| renamed to |\childdocmain|
\item
|\childredirect| renamed to |\childdocforward| and |\childdocforwardprefix|
and functionality expanded
\end{itemize}

%%%%%%%%%%%%%%%%%%%%%%%%%%%%%%%%%%%%%%%%
\paragraph{v1.0:} 2017/04/27

\begin{itemize}
\item
manual and install package
\item
first version published on CTAN
\end{itemize}

%%%%%%%%%%%%%%%%%%%%%%%%%%%%%%%%%%%%%%%%
\paragraph{v0.6:} 2017/04/26

\begin{itemize}
\item
redirection mechanism added
\end{itemize}

%%%%%%%%%%%%%%%%%%%%%%%%%%%%%%%%%%%%%%%%
\paragraph{v0.5:} 2017/04/26

\begin{itemize}
\item
functionality in definition file
\end{itemize}


%%%%%%%%%%%%%%%%%%%%%%%%%%%%%%%%%%%%%%%%%%%%%%%%%%%%%%%%%%%%%%%%%%%%%%%%%%%%%%%%
%%%%%%%%%%%%%%%%%%%%%%%%%%%%%%%%%%%%%%%%%%%%%%%%%%%%%%%%%%%%%%%%%%%%%%%%%%%%%%%%
%%%%%%%%%%%%%%%%%%%%%%%%%%%%%%%%%%%%%%%%%%%%%%%%%%%%%%%%%%%%%%%%%%%%%%%%%%%%%%%%
\appendix

\settowidth\MacroIndent{\rmfamily\scriptsize 000\ }

 \DocInput{childdoc.dtx}

\end{document}
%</driver>
% \fi
%
% %%%%%%%%%%%%%%%%%%%%%%%%%%%%%%%%%%%%%%%%%%%%%%%%%%%%%%%%%%%%%%%%%%%%%%%%%%%%%%
% %%%%%%%%%%%%%%%%%%%%%%%%%%%%%%%%%%%%%%%%%%%%%%%%%%%%%%%%%%%%%%%%%%%%%%%%%%%%%%
% \section{Sample}
%\iffalse
%<*samplemain>
%\fi
%
% The following presents a sample document
% with two chapters, two parts, a title page,
% a compile flag as well as three forwarding files to set the flag.
% It consists of eight |.tex| files:
% \begin{center}
% \begin{tabular}{ll}
% |cdocsamp.tex|&main file\\
% |cdocsch1.tex|&include file for chapter 1\\
% |cdocsch2.tex|&include file for chapter 2\\
% |cdocspt3.tex|&include file for part 3\\
% |cdocspt4.tex|&include file for part 4\\
% |cdocsdrf.tex|&forwarding file for main file in draft mode\\
% |cdocsfi1.tex|&forwarding file for final version of chapter 1\\
% |cdocsfi2.tex|&forwarding file for final version of chapter 2\\
% \end{tabular}
% \end{center}
% Each of the eight files can be compiled directly by the \LaTeX{} compiler.
%
% %%%%%%%%%%%%%%%%%%%%%%%%%%%%%%%%%%%%%%
% \paragraph{Main File.}
%
% The main file is called |cdocsamp.tex|.
%
% Load the \textsf{childdoc} definitions and
% declare the filename for the main document:
%    \begin{macrocode}
\input{childdoc.def}
\childdocmain{}
%    \end{macrocode}

% Optional override for |\version| flag:
%    \begin{macrocode}
%%\ifchilddoc\else\providecommand{\version}{draft}\fi
%    \end{macrocode}

% Define the default values for the |\version| flag
% (|final| for the main file and |draft| for childs):
%    \begin{macrocode}
\ifchilddoc
\providecommand{\version}{draft}
\else
\providecommand{\version}{final}
\fi
%    \end{macrocode}

% Load the standard document class:
%    \begin{macrocode}
\documentclass[12pt]{article}
%    \end{macrocode}

% Start the document body:
%    \begin{macrocode}
\begin{document}
%    \end{macrocode}

% Declare a title page.
% Print title, part of document being processed and version flag:
%    \begin{macrocode}
\addtocounter{page}{-1}
\begin{center}
{\LARGE\bfseries{}childdoc example\par}
\vspace{1cm}
\ifchilddoc
\ifchilddocmanual part\else chapter\fi:
`\childdocname' of `\childdocjob'\par
\else
main document: `\childdocjob'\par
\fi
version: \version\par
\end{center}
\newpage
%    \end{macrocode}

% Manually include selected file,
% otherwise process as usual:
%    \begin{macrocode}
\ifchilddocmanual
\section*{part `\childdocname'}
\input{\childdocname}
\else
%    \end{macrocode}

% Include the two chapters:
%    \begin{macrocode}
\include{cdocsch1}
\include{cdocsch2}
%    \end{macrocode}

% Include the two parts unless only chapters should be displayed:
%    \begin{macrocode}
\ifchilddoc\else
\section{part three}
\input{cdocspt3}
\section{part four}
\input{cdocspt4}
\fi
%    \end{macrocode}

% Process as usual until here:
%    \begin{macrocode}
\fi
%    \end{macrocode}

% End of document body:
%    \begin{macrocode}
\end{document}
%    \end{macrocode}
%\iffalse
%</samplemain>
%\fi
%
% %%%%%%%%%%%%%%%%%%%%%%%%%%%%%%%%%%%%%%
% \paragraph{Chapter Include Files.}
%
% The include files are called |cdocsch1.tex| and |cdocsch2.tex|.
%
%\iffalse
%<*samplechap1|samplechap2>
%\fi

% Optional override for |\version| flag:
%    \begin{macrocode}
%%\providecommand{\version}{final}
%    \end{macrocode}

% Include the main document:
%    \begin{macrocode}
\input{childdoc.def}
\childdocof{cdocsamp}
%    \end{macrocode}

%\iffalse
%</samplechap1|samplechap2>
%\fi
%
%\iffalse
%<*samplechap1>
%\fi
% Some text for chapter 1:
%    \begin{macrocode}
\section{one}
some text in chapter one
%    \end{macrocode}

%\iffalse
%</samplechap1>
%\fi
% Some text for chapter 2:
%\iffalse
%<*samplechap2>
%\fi
%    \begin{macrocode}
\section{two}
more text in chapter two
%    \end{macrocode}

%\iffalse
%</samplechap2>
%\fi
%
% %%%%%%%%%%%%%%%%%%%%%%%%%%%%%%%%%%%%%%
% \paragraph{Part Include Files.}
%
% The include files are called |cdocspt3.tex| and |cdocspt4.tex|.
%
%\iffalse
%<*samplepart3|samplepart4>
%\fi

% Optional override for |\version| flag:
%    \begin{macrocode}
%%\providecommand{\version}{final}
%    \end{macrocode}

% Include the main document:
%    \begin{macrocode}
\input{childdoc.def}
\childdocby{cdocsamp}
%    \end{macrocode}

%\iffalse
%</samplepart3|samplepart4>
%\fi
%
%\iffalse
%<*samplepart3>
%\fi
% Some text for part 3:
%    \begin{macrocode}
some text in part three
%    \end{macrocode}

%\iffalse
%</samplepart3>
%\fi
% Some text for part 4:
%\iffalse
%<*samplepart4>
%\fi
%    \begin{macrocode}
more text in part four
%    \end{macrocode}

%\iffalse
%</samplepart4>
%\fi
%
% %%%%%%%%%%%%%%%%%%%%%%%%%%%%%%%%%%%%%%
% \paragraph{Forwarding for a Complete Draft.}
%
% The following forwarding file |cdocsdrf.tex|
% compiles the main document in draft mode:
%\iffalse
%<*sampledraft>
%\fi
%    \begin{macrocode}
\def\version{draft}
\input{childdoc.def}
\childdocforward{cdocsamp}
%    \end{macrocode}

%\iffalse
%</sampledraft>
%\fi
%
% %%%%%%%%%%%%%%%%%%%%%%%%%%%%%%%%%%%%%%
% \paragraph{Forwarding for Final Version of the Chapters.}
%
% The following forwarding files |cdocsfn1.tex| and |cdocsfn2.tex|
% (with identical content)
% compile the final versions of the child documents
% |cdocsch1.tex| and |cdocsch2.tex|, respectively:
%\iffalse
%<*samplefinal>
%\fi
%    \begin{macrocode}
\def\version{final}
\input{childdoc.def}
\childdocforwardprefix[cdocsamp]{cdocsfn}{cdocsch}
%    \end{macrocode}

%\iffalse
%</samplefinal>
%\fi
%
% %%%%%%%%%%%%%%%%%%%%%%%%%%%%%%%%%%%%%%
% \paragraph{Command Line Processing.}
%
% The following three command lines generate the output files
% |cdocscld|, |cdocscl1| and |cdocscl2|
% which should be identical to
% |cdocsdrf|, |cdocsch1| and |cdocsfn2|, respectively:
% \begin{center}
% \begin{tabular}{l}
% |latex -jobname cdocscld \|\\
% |  "\def\version{draft}\input{childdoc.def}\childdocforward{cdocsamp}"|\\
% |latex -jobname cdocscl1 \|\\
% |  "\input{childdoc.def}\childdocforward[cdocsamp]{cdocsch1}"|\\
% |latex -jobname cdocscl2 \|\\
% |  "\def\version{final}\input{childdoc.def}\childdocforward{cdocsch2}"|
% \end{tabular}
% \end{center}
% Note that the trailing backslash on each first line
% merely continues the input to the second line
% (for convenient cut ant paste).
% Furthermore, the command |latex| can be replaced by any
% of its alternative versions such as |pdflatex|.
%
% %%%%%%%%%%%%%%%%%%%%%%%%%%%%%%%%%%%%%%%%%%%%%%%%%%%%%%%%%%%%%%%%%%%%%%%%%%%%%%
% %%%%%%%%%%%%%%%%%%%%%%%%%%%%%%%%%%%%%%%%%%%%%%%%%%%%%%%%%%%%%%%%%%%%%%%%%%%%%%
% \section{Implementation}
%\iffalse
%<*package>
%\fi
%
% This section describes the definitions file |childdoc.def|.

% The definitions cannot be loaded using |\usepackage| or |\RequirePackage|
% which has a mechanism to prevent loading a style file more than once.
% When loading the definitions by means of |\input|
% multiple instances have to be prevented manually:
%\iffalse
%This code needs to be before the `\ProvidesFile' directive
%which is defined at the beginning of this file.
%Therefore it is also placed there and commented out here.
%</package>
%<*discard>
%\fi
%    \begin{macrocode}
\ifdefined\childdocmain\endinput\fi
%    \end{macrocode}
%\iffalse
%</discard>
%<*package>
%\fi
%
% \macro{\ifchilddoc}
% \macro{\ifchilddocmanual}
% The conditional |\ifchilddoc| tells whether a
% child (true) or main (false) document is being compiled.
% The conditional |\ifchilddocmanual| tells whether
% the |\includeonly| mechanism is used (false) or
% the selection of child files must be performed manually (true).
% The definitions initialise to false:
%    \begin{macrocode}
\newif\ifchilddoc
\newif\ifchilddocmanual
%    \end{macrocode}

% \macro{\childdocname}
% \macro{\childdocjob}
% The macro |\childdocname| stores the name of the main document
% to be compiled. The macro |\childdocjob| stores the name of
% the document on which the \LaTeX{} compiler was originally invoked.
% The content of |\jobname| cannot be compared
% to filenames specified in the source due to different catcodes.
% The following code rescans |\jobname|, stores the result
% in |\childdocname| and saves a copy in |\childdocjob|:
%    \begin{macrocode}
\edef\childdocname{\scantokens\expandafter{\jobname\noexpand}}
\let\childdocjob\childdocname
%    \end{macrocode}

% \macro{\childdocdisable}
% The macro |\childdocdisable| prevents the main file
% from being processed more than once.
% At this stage, the main document command |\childdocmain|
% is assumed to be called once again where it should do nothing.
% Any subsequent call to it should prevent
% a secondary processing of the main document
% It overwrites the forwarding commands
% |\childdocof| and |\childdocforward|
% with empty macros to prevent further inclusions of the main document:
%    \begin{macrocode}
\newcommand{\childdocdisable}
{
  \renewcommand{\childdocmain}[1]{\renewcommand{\childdocmain}[1]{\endinput}}
  \renewcommand{\childdocof}[1]{}
  \renewcommand{\childdocby}[2][]{}
  \renewcommand{\childdocforward}[2][]{}
  \renewcommand{\childdocdisable}{}
}
%    \end{macrocode}

% \macro{\childdocmain}
% The macro |\childdocmain| is to be called at the top of the main file
% with nothing or the main filename (without extension) as argument.
% First, it breaks loops.
% If the argument is not empty and does not match |\childdocname|
% (which is set by the first inclusion of |childdoc.def|),
% |\ifchilddoc| is set to true, |\includeonly| is applied to the child file
% and |\jobname| is set to the main file
% (for proper handling of |.aux| files):
%    \begin{macrocode}
\newcommand{\childdocmain}[1]
{
  \childdocdisable\childdocmain{}
  \if?#1?\else
    \begingroup
      \def\childdoctmp{#1}
      \ifx\childdoctmp\childdocname
        \def\childdoctmp{}
      \else
        \def\childdoctmp
        {
          \childdoctrue
          \includeonly{\childdocname}
          \def\childdocjob{#1}
          \def\jobname{#1}
        }
      \fi
      \expandafter
    \endgroup
    \childdoctmp
  \fi
}
%    \end{macrocode}

% \macro{\childdocof}
% The command |\childdocof| redirects
% compilation to the main file |#1|.
%    \begin{macrocode}
\newcommand{\childdocof}[1]
{
  \childdocdisable
  \childdoctrue
  \includeonly{\childdocname}
  \def\jobname{#1}
  \def\childdocjob{#1}
  \input{#1}
}
%    \end{macrocode}

% \macro{\childdocby}
% The command |\childdocby| ....
%    \begin{macrocode}
\newcommand{\childdocby}[2][]
{
  \childdocdisable
  \childdoctrue
  \childdocmanualtrue
  \if?#1?\else
    \def\jobname{#2}
  \fi
  \def\childdocjob{#2}
  \input{#2}
  \endinput
}
%    \end{macrocode}

% \macro{\childdocforward}
% The command |\childdocforward| redirects
% compilation to the main file or
% (if the optional argument is given) a child file.
% Parameters are set as if the main file
% or a child file starting with |\childdocof| was compiled.
% Then compilation is handed over to the main file:
%    \begin{macrocode}
\newcommand{\childdocforward}[2][]
{
  \begingroup
    \if?#1?
      \def\childdoctmp
      {
        \def\childdocname{#2}
        \def\childdocjob{#2}
        \def\jobname{#2}
        \input{#2}
        \endinput
      }
    \else
      \def\childdoctmp
      {
        \childdocdisable
        \def\childdocname{#2}
        \childdoctrue
        \includeonly{#2}
        \def\childdocjob{#1}
        \def\jobname{#1}
        \input{#1}
        \endinput
      }
    \fi
    \expandafter
  \endgroup
  \childdoctmp
}
%    \end{macrocode}

% \macro{\childdocforwardprefix}
% The command |\childdocforwardprefix| redirects
% compilation to the main or a child file by means of a pattern.
% The prefix |#1| in the current filename is replaced by |#2|
% and the suffix of the current filename is kept
% (it is assumed that the filename does not contain the substring `|~~~|'
% which is used as a delimiter).
% Compilation is handed over to the new file by |\childdocforward|:
%    \begin{macrocode}
\newcommand{\childdocforwardprefix}[3][]
{
  \begingroup
    \def\childdocextract #2##1~~~{\def\childdoctmp{\childdocforward[#1]{#3##1}}}
    \expandafter\childdocextract\childdocname~~~
    \expandafter
  \endgroup
  \childdoctmp
}
%    \end{macrocode}

% \macro{\childdoc}
% The deprecated macro |\childdoc| is a legacy version of |\childdocmain|:
%    \begin{macrocode}
\newcommand{\childdoc}{\childdocmain}
%    \end{macrocode}

% \macro{\childdocredirect}
% The deprecated macro |\childdocredirect| is a legacy version
% of |\childdocforward| and |\childdocforwardprefix|:
%    \begin{macrocode}
\newcommand{\childdocredirect}[2][]
{
  \begingroup
    \if?#1?
      \def\childdoctmp{\childdocforward{#2}}
    \else
      \def\childdoctmp{\childdocforwardprefix{#1}{#2}}
    \fi
    \expandafter
  \endgroup
  \childdoctmp
}
%    \end{macrocode}

%\iffalse
%</package>
%\fi
%
\endinput
|\\
|\childdocforward{|\textit{main}|}|\\
\end{tabular}
\end{center}
%
or alternatively with:
%
\begin{center}
\begin{tabular}{l}
|% \iffalse
%
% childdoc.dtx Copyright (C) 2017-2018 Niklas Beisert
%
% This work may be distributed and/or modified under the
% conditions of the LaTeX Project Public License, either version 1.3
% of this license or (at your option) any later version.
% The latest version of this license is in
%   http://www.latex-project.org/lppl.txt
% and version 1.3 or later is part of all distributions of LaTeX
% version 2005/12/01 or later.
%
% This work has the LPPL maintenance status `maintained'.
%
% The Current Maintainer of this work is Niklas Beisert.
%
% This work consists of the files childdoc.dtx and childdoc.ins
% and the derived files childdoc.def and cdocsamp.tex with
% cdocsch1.tex, cdocsch2.tex, cdocsdrf.tex, cdocsfn1.tex, cdocsfn2.tex.
%
%<package>\ifdefined\childdocmain\endinput\fi
%<package>\ProvidesFile{childdoc.def}[2018/12/30 v2.0 child document driver]
%<samplemain>\ProvidesFile{cdocsamp.tex}[2018/12/30 v2.0 sample for childdoc]
%<*driver>
%\ProvidesFile{childdoc.drv}[2018/12/30 v2.0 childdoc reference manual file]
\PassOptionsToClass{10pt,a4paper}{article}
\documentclass{ltxdoc}

\usepackage[margin=35mm]{geometry}
\usepackage{hyperref}
\usepackage{hyperxmp}
\usepackage[usenames]{color}

\hypersetup{colorlinks=true}
\hypersetup{pdfstartview=FitH}
\hypersetup{pdfpagemode=UseNone}
\hypersetup{pdfsource={}}
\hypersetup{pdflang={en-UK}}
\hypersetup{pdfcopyright={Copyright 2017-2018 Niklas Beisert.
  This work may be distributed and/or modified under the
  conditions of the LaTeX Project Public License, either version 1.3
  of this license or (at your option) any later version.}}
\hypersetup{pdflicenseurl={http://www.latex-project.org/lppl.txt}}
\hypersetup{pdfcontactaddress={ETH Zurich, ITP, HIT K,
  Wolfgang-Pauli-Strasse 27}}
\hypersetup{pdfcontactpostcode={8093}}
\hypersetup{pdfcontactcity={Zurich}}
\hypersetup{pdfcontactcountry={Switzerland}}
\hypersetup{pdfcontactemail={nbeisert@itp.phys.ethz.ch}}
\hypersetup{pdfcontacturl={http://people.phys.ethz.ch/\xmptilde nbeisert/}}

\newcommand{\secref}[1]{\hyperref[#1]{section \ref*{#1}}}

\parskip1ex
\parindent0pt
\let\olditemize\itemize
\def\itemize{\olditemize\parskip0pt}

\begin{document}

\title{The \textsf{childdoc} Package}
\hypersetup{pdftitle={The childdoc Package}}
\author{Niklas Beisert\\[2ex]
  Institut f\"ur Theoretische Physik\\
  Eidgen\"ossische Technische Hochschule Z\"urich\\
  Wolfgang-Pauli-Strasse 27, 8093 Z\"urich, Switzerland\\[1ex]
  \href{mailto:nbeisert@itp.phys.ethz.ch}
  {\texttt{nbeisert@itp.phys.ethz.ch}}}
\hypersetup{pdfauthor={Niklas Beisert}}
\hypersetup{pdfsubject={Manual for the LaTeX2e Package childdoc}}
\date{30 December 2018, \textsf{v2.0}}
\maketitle

\begin{abstract}\noindent
\textsf{childdoc} is a \LaTeXe{} package
that enables the direct compilation
of document sections included by |\include|
to individual files.
\end{abstract}

\begingroup
\parskip0ex
\tableofcontents
\endgroup

%%%%%%%%%%%%%%%%%%%%%%%%%%%%%%%%%%%%%%%%%%%%%%%%%%%%%%%%%%%%%%%%%%%%%%%%%%%%%%%%
%%%%%%%%%%%%%%%%%%%%%%%%%%%%%%%%%%%%%%%%%%%%%%%%%%%%%%%%%%%%%%%%%%%%%%%%%%%%%%%%
\section{Introduction}

\LaTeX{} provides a mechanism to structure a large document (such as a book)
into a main file and several child files (containing the chapters)
using the |\include| command.
This mechanism is beneficial for documents
which span hundreds of pages in order to
make the source file(s) more manageable.
Moreover, compilation can be restricted to
selected child files by means of the |\includeonly| command.
The latter feature can be used to reduce the compilation time while editing
(this was significantly more useful in the earlier days of \LaTeX{})
or to generate a smaller document which is easier to navigate.
Another application of |\includeonly| is to generate
documents consisting of selected parts of the complete document.

However, there are a few drawbacks of the plain |\include| mechanism:
\begin{itemize}
\item
The child files cannot be compiled on their own,
they can only be compiled via the main file.
A naive editing environment
(such as a text editor with an option
to have the current file processed by \LaTeX)
may require one to switch to the main file before compiling;
attempting to compile the child file produces errors.
\item
The main file must be modified (each time)
to adjust the |\includeonly| command
to the present needs. This easily leaves the main file in a messy state.
\item
The generated document will always carry the filename
of the main document. This is inconvenient if
several child files are to be compiled and
to be kept for distribution.
\end{itemize}

The present package provides a simple interface
to make child files individually compilable by \LaTeX{}.
Compiling a child file then has the same effect as compiling
the main file with an |\includeonly| command
to select the appropriate child.
Moreover the generated document will carry the name of the child
rather than the main file.
This resolves all three above issues.

This feature is meant to make the editing of books,
thesis documents and lecture notes somewhat more convenient.
However, the package can also be used efficiently for
composing a series of documents (such as exercise sheets)
which are typically distributed individually.
It then assists the author in generating the individual documents
(potentially in different versions)
as well as a document containing the collected series.
Another application is in developing style files
or other kinds of included material
where compilation of the style file could redirect
to a sample or test file.

%%%%%%%%%%%%%%%%%%%%%%%%%%%%%%%%%%%%%%%%%%%%%%%%%%%%%%%%%%%%%%%%%%%%%%%%%%%%%%%%
%%%%%%%%%%%%%%%%%%%%%%%%%%%%%%%%%%%%%%%%%%%%%%%%%%%%%%%%%%%%%%%%%%%%%%%%%%%%%%%%
\section{Usage}

First of all, the package \textsf{childdoc} is \emph{not} a standard
\LaTeXe{} |.sty| style file! Therefore it needs to be invoked in
a non-standard way.

%%%%%%%%%%%%%%%%%%%%%%%%%%%%%%%%%%%%%%%%%%%%%%%%%%%%%%%%%%%%%%%%%%%%%%%%%%%%%%%%
\subsection{Included Files}
\label{sec:include}

%%%%%%%%%%%%%%%%%%%%%%%%%%%%%%%%%%%%%%%%
\DescribeMacro{\childdocmain}
To use the package, add the commands
\begin{center}
\begin{tabular}{l}
|\input{childdoc.def}|\\
|\childdocmain{}|\\
\end{tabular}
\end{center}
at the very top of the main \LaTeX{} file,
in particular \emph{before} the |\documentclass| statement!
The argument of |\childdocmain| should be left empty
(but it must be present).

%%%%%%%%%%%%%%%%%%%%%%%%%%%%%%%%%%%%%%%%
\DescribeMacro{\childdocof}
Furthermore, add the commands
\begin{center}
\begin{tabular}{l}
|\input{childdoc.def}|\\
|\childdocof{|\textit{main}|}|\\
\end{tabular}
\end{center}
at the top of every child file \textit{child}
which is included by |\include{|\textit{child}|}|
from within the main file
(or at least for those files to be compiled individually).
The argument \textit{main} must be the filename of the main file.

There are a couple of
considerations in setting up the main and child documents:

%%%%%%%%%%%%%%%%%%%%%%%%%%%%%%%%%%%%%%%%
\paragraph{Restrictions.}

Please note the following restrictions:
\begin{itemize}
\item
|\childdocmain| must be called with one argument \textit{main}
to ensure compatibility with earlier version of the package.
It must either be empty (|\childdocmain{}|)
or precisely match the filename of the main file in which it is specified.
See \secref{sec:detection} for further information.
\item
The filename \textit{main} must be specified without the |.tex| extension.
\item
The filename \textit{main} is case sensitive
(even in case-insensitive file systems)
due to internal string comparison.
\item
The argument \textit{main} should be fully expanded, it cannot be a macro.
\item
Subdirectories and special characters should be avoided in filenames.
\item
The command |\childdocmain{|\textit{main}|}| must be followed by a whitespace.
It should not be followed immediately by another command
or by a comment mark `|%|'.
This is because the \TeX{} parser reads the token immediately following
the argument of |\childdocmain| and puts it
at the beginning of every child section;
however, a white\-space is ignored.
\end{itemize}

%%%%%%%%%%%%%%%%%%%%%%%%%%%%%%%%%%%%%%%%
\paragraph{Content of Main File.}

It is advisable to place all content in the child files included by |\include|.
Any output contained in the main file will appear in all child documents
unless suppressed manually;
it cannot be suppressed automatically by the |\includeonly| directive
and thus should normally be avoided.
A method to include some content in the main file
by means of conditional processing is described in \secref{sec:conditional}.

%%%%%%%%%%%%%%%%%%%%%%%%%%%%%%%%%%%%%%%%
\paragraph{Page Numbering.}

When only a part of the document is compiled,
the appropriate numbering of pages
(as well as other status parameters)
is determined from the |.aux| files.
The latter contain information from previous passes.
However this information needs to propagate through
all intermediate child documents.
Therefore the page numbering in child documents may well
be inconsistent until the complete document is compiled at least once.

A useful (if unconventional) way to always ensure a consistent
page numbering is to restart the numbering in each child document
and denote the pages by `\textit{child}|.|\textit{page}'
where \textit{child} represents the chapter/section number of the child file.
This can be achieved by the command
|\numberwithin{page}{|\textit{child}|}|
of the \textsf{amsmath} package
where \textit{child} can be |chapter| or |section|
depending on the chosen structuring.
Alternatively, one can modify the macro |\thepage| appropriately
and reset the counter |page| at the start of each child file.

%%%%%%%%%%%%%%%%%%%%%%%%%%%%%%%%%%%%%%%%%%%%%%%%%%%%%%%%%%%%%%%%%%%%%%%%%%%%%%%%
\subsection{Conditional Processing}
\label{sec:conditional}

The package provides a mechanism to compile different versions
of a document. To customise the versions further some conditional processing
can come in handy to distinguish which version is being compiled.
The package provides two macros to describe the compilation context:

%%%%%%%%%%%%%%%%%%%%%%%%%%%%%%%%%%%%%%%%
\DescribeMacro{\ifchilddoc}
The conditional |\ifchilddoc| distinguishes between the compilation of
child documents and the main document:
%
\begin{center}
|\ifchilddoc |\textit{child-code}| |[|\||else |\textit{main-code}]| \||fi|
\end{center}

%%%%%%%%%%%%%%%%%%%%%%%%%%%%%%%%%%%%%%%%
\DescribeMacro{\childdocname}
\DescribeMacro{\childdocjob}
The macro |\childdocname| contains the filename (without extension)
of the main or child file being processed.
Note that |\childdocjob| will always contain the name of the main file.

%%%%%%%%%%%%%%%%%%%%%%%%%%%%%%%%%%%%%%%%
\paragraph{Title Page.}

Conditional processing can be used to include a title or banner page
in the main document when proper precautions are taken.
Importantly, the code in the main file should ensure that the page counter
(as well as other status parameters which are stored in the |.aux| files)
takes the same value after the conditional processing.
Otherwise the page numbers may take divergent values
depending on which part is compiled.

For example, a title page could be declared by:
%
\begin{center}
\begin{tabular}{l}
|\ifchilddoc\||else|\\
|\addtocounter{page}{-1}|\\
\textit{code for title page}\\
|\newpage|\\
|\||fi|
\end{tabular}
\end{center}
%
A banner page for the child documents can be generated by:
%
\begin{center}
\begin{tabular}{l}
|\ifchilddoc|\\
|\addtocounter{page}{-1}|\\
\textit{code for banner page}\\
|\newpage|\\
|\||fi|
\end{tabular}
\end{center}
%
Here one could write a message such as:
\begin{center}
|This is the part \childdocname{} of \childdocjob{}.|
\end{center}

%%%%%%%%%%%%%%%%%%%%%%%%%%%%%%%%%%%%%%%%%%%%%%%%%%%%%%%%%%%%%%%%%%%%%%%%%%%%%%%%
\subsection{Flags}
\label{sec:flags}

The package makes it easy to generate different versions
of the main or child documents.
To this end compilation flags can be defined
and assigned different default values.
They will be particularly useful in conjunction
with the forwarding mechanism described in \secref{sec:forward}.

For example, it may be useful to have a flag |\version|
which can be set to |draft| or |final|.
The document source will contain some conditional code
depending on the value of |\version|.
Suppose further, the flag should default to |final| for the main file
and to |draft| for child files
which is a natural assignment for editing the document.
This is achieved by placing the following code
in the preamble of the main document
(below the |\childdocmain| directive):
%
\begin{center}
\begin{tabular}{l}
|\ifchilddoc|\\
|\providecommand{\version}{draft}|\\
|\||else|\\
|\providecommand{\version}{final}|\\
|\||fi|
\end{tabular}
\end{center}
%
The definition by |\providecommand| makes sure
that previous definitions are not overwritten.
Further statements |\providecommand{\version}{...}|
can thus be added before the above code to override it.

For the main file, one might add a line
(between |\childdocmain| and the above block)
%
\begin{center}
|%\ifchilddoc\||else\providecommand{\version}{draft}\||fi|
\end{center}
%
which can be uncommented to produce a draft version.
Likewise one can add a line to the very top of a child file
(above the |\childdocof{|\textit{main}|}| directive)
%
\begin{center}
|%\providecommand{\version}{final}|
\end{center}
%
which can be uncommented to produce the final version of this child document.

%%%%%%%%%%%%%%%%%%%%%%%%%%%%%%%%%%%%%%%%%%%%%%%%%%%%%%%%%%%%%%%%%%%%%%%%%%%%%%%%
\subsection{Forwarding}
\label{sec:forward}

Different versions of the main or child documents
using compilation flags as described in \secref{sec:flags}
can be (permanently) stored in different files
for convenient compilation, viewing and distribution.
To this end, the package defines a command
to pass on compilation to a different file:

%%%%%%%%%%%%%%%%%%%%%%%%%%%%%%%%%%%%%%%%
\DescribeMacro{\childdocforward}
The command |\childdocforward| redirects processing to
another source file:
%
\begin{center}
\begin{tabular}{l}
|\input{childdoc.def}|\\
|\childdocforward[|\textit{main}|]{|\textit{dest}|}|\\
\end{tabular}
\end{center}
%
The argument \textit{dest} is the destination file
(without extension).
It should be the main file or one of the child files.
Note that further \textsf{childdoc} directives
such as |\childdocof| and |\childdocforward|
in the indicated file will be processed in this form.
The optional argument \textit{main}
passes on directly to the main file \textit{main}
while pretending to compile the child \textit{dest}.
This form behaves as if \textit{dest}
issues |\childdocof{|\textit{main}|}| right away,
and no further \textsf{childdoc} directives will be processed.

%%%%%%%%%%%%%%%%%%%%%%%%%%%%%%%%%%%%%%%%
\DescribeMacro{\...prefix}
In the alternative form |\childdocforwardprefix|,
%
\begin{center}
\begin{tabular}{l}
|\input{childdoc.def}|\\
|\childdocforwardprefix[|\textit{main}|]{|\textit{prefix}|}{|\textit{dest}|}|
\end{tabular}
\end{center}
%
the destination file is determined by a pattern
depending on the current file:
To make this work, the current file must be called
`{\textit{prefix}\hspace{0.2em}\textit{suffix}}'
with \textit{prefix} matching precisely the argument.
Processing is then passed on to the file
`{\textit{dest}\hspace{0.2em}\textit{suffix}}'.
Surely, the same effect is achieved by
directly specifying the
argument `{\textit{dest}\hspace{0.2em}\textit{suffix}}'
in the first form.
However, that requires to set up a different file
for each child. With the alternative form of the command
all these files can have exactly the same content
which simplifies setting them up and maintaining them.

For example, the following file |draft.tex|
with a compilation flag |\version| as described in \secref{sec:flags}
compiles the main document as a draft:
%
\begin{center}
\begin{tabular}{l}
|\def\version{draft}|\\
|\input{childdoc.def}|\\
|\childdocforward{|\textit{main}|}|
\end{tabular}
\end{center}
%
Likewise, the following files |final|\textit{nn}|.tex|
compile the final version of the child document
|child|\textit{nn}|.tex|:
%
\begin{center}
\begin{tabular}{l}
|\def\version{final}|\\
|\input{childdoc.def}|\\
|\childdocforwardprefix{final}{child}|
\end{tabular}
\end{center}
%

Note that when several versions of a main file and/or of each child file
are to be generated, it may be convenient to set up a |Makefile| or
shell script to automatise the process.

%%%%%%%%%%%%%%%%%%%%%%%%%%%%%%%%%%%%%%%%%%%%%%%%%%%%%%%%%%%%%%%%%%%%%%%%%%%%%%%%
\subsection{Command Line Processing}
\label{sec:commandline}

The effect of redirection files can also be achieved by invoking
the \LaTeX{} compiler with a more elaborate command line.
Most conveniently this should be done as part
of a shell script or a |Makefile|.

When using \textsf{childdoc} in the main file, the following
command lines effectively perform a redirection
(note that depending on the shell being used,
backslashes may have to be doubled: `|\|' $\to$ `|\\|'):
%
\begin{center}
|... -jobname "|\textit{target}|" |\\|"|[\textit{flags}]%
|\input{childdoc.def}\childdocforward[|\textit{main}|]{|\textit{dest}|}"|
\end{center}
%
Here \textit{target} is the name of the output file,
\textit{main} is the name of the main file
and \textit{dest} is the name of the main or child file to be processed
(all filenames without extensions).
The optional argument \textit{main} can be omitted
if \textit{main} matches \textit{dest}.
Optionally, compilation \textit{flags} can be defined via |\def| commands.
This command line makes the \TeX{} engine believe
it is compiling the file \textit{target}
whose content is specified as the latter parameter.
The provided code then forwards the processing to
\textit{main} or \textit{dest} as described in \secref{sec:forward}.

%%%%%%%%%%%%%%%%%%%%%%%%%%%%%%%%%%%%%%%%%%%%%%%%%%%%%%%%%%%%%%%%%%%%%%%%%%%%%%%%
\subsection{Include by Input}
\label{sec:input}

Including child documents by |\include| has some restrictions by design.
Most notably, the content of a child document always occupies
its own set of pages; pages cannot be shared between child documents.
Usually, this behaviour makes perfect sense
because each child document contain an essential part of the document.
However, in some situations it may be desirable to compose
a document from a collection of parts
without having mandatory page breaks between then.
For this case, the package
provides a mechanism to include parts
by |\input| which can also be processed individually.
However, by construction this mechanism
requires manual handling of the content to be output.

%%%%%%%%%%%%%%%%%%%%%%%%%%%%%%%%%%%%%%%%
\DescribeMacro{\ifchilddocmanual}
The main file should be prepared as usual, see \secref{sec:include}.
However, the document body must make a distinction
between processing of an individual part and of the main document, e.g.:
%
\begin{center}
\begin{tabular}{l}
|\ifchilddocmanual|\\
|\input{\childdocname}|\\
|\||else|\\
\textit{document body with }|\input{|\textit{part}|}|\\
|\||fi|
\end{tabular}
\end{center}
%
The conditional |\ifchilddocmanual| is true whenever
a part to be included by |\input| is being compiled,
and the name of the part is stored in |\childdocname|.

%%%%%%%%%%%%%%%%%%%%%%%%%%%%%%%%%%%%%%%%
\DescribeMacro{\childdocby}
Each part to be included by |\input| should start with:
%
\begin{center}
\begin{tabular}{l}
|\input{childdoc.def}|\\
|\childdocby{|\textit{main}|}|\\
\end{tabular}
\end{center}
%
The directive |\childdocby| is similar to |\childdocof|
described in \secref{sec:include},
but the subsequent selection of content must be done manually.
To that end, both |\ifchilddoc| and |\ifchilddocmanual|
will be true upon processing of a part,
and the name of the part is stored in |\childdocname|.
Note that |\jobname| will be set to the filename of the current part
so that each part receives an individual |.aux| file
that does not interfere with the |.aux| file(s) of the main document.
This behaviour can be altered by the alternative form
|\childdocby[*]{|\textit{main}|}| (with a non-empty optional argument)
which uses the |.aux| file of the main document
by setting |\jobname| to \textit{main}.

%%%%%%%%%%%%%%%%%%%%%%%%%%%%%%%%%%%%%%%%%%%%%%%%%%%%%%%%%%%%%%%%%%%%%%%%%%%%%%%%
\subsection{Driver Development}
\label{sec:driver}

The \textsf{childdoc} mechanism can also be use for the development
of definition files such as \LaTeX{} styles or classes.
This case differs from the above setup with multiple parts
included by |\include| in that no |\includeonly| should be invoked.
This can be achieved by starting the include file
(before |\ProvidesPackage|) with:
%
\begin{center}
\begin{tabular}{l}
|\input{childdoc.def}|\\
|\childdocforward{|\textit{main}|}|\\
\end{tabular}
\end{center}
%
or alternatively with:
%
\begin{center}
\begin{tabular}{l}
|\input{childdoc.def}|\\
|\childdocby{|\textit{main}|}|\\
\end{tabular}
\end{center}
%
Both forms have slightly different effects as described above.
The main file is prepared as usual, see \secref{sec:include}.

%%%%%%%%%%%%%%%%%%%%%%%%%%%%%%%%%%%%%%%%%%%%%%%%%%%%%%%%%%%%%%%%%%%%%%%%%%%%%%%%
\subsection{Legacy Detection}
\label{sec:detection}

The directive |\childdocmain| in the main file can detect
whether the complete document or merely a child is to be compiled
even without using the directive |\childdocof|.
This method is deprecated because it is less robust
and there is no compelling reason to use it;
it is merely provided for backward compatibility
and it may be removed in future versions.

If the detection mechanism is to be used,
it is mandatory to correctly specify
the filename of the main file as the argument of |\childdocmain|:
%
\begin{center}
\begin{tabular}{l}
|\input{childdoc.def}|\\
|\childdocmain{|\textit{main}|}|\\
\end{tabular}
\end{center}
%
If |\jobname| does not match the argument \textit{main} of |\childdocmain|,
it is assumed that |\jobname| points to the child file to be compiled.
When using |\childdocmain| with the main file specified as argument,
it suffices to start a child file
with just |\input{|\textit{main}|}|
without loading of the package and using |\childdocof|.
If instead all processing is done
with the appropriate \textsf{childdoc} directives,
the argument of \textit{main} of |\childdocmain| can be empty.

An alternative version of the command line processing described
in \secref{sec:commandline} using the detection mechanism reads:
%
\begin{center}
|... -jobname "|\textit{target}|" "|[\textit{flags}]%
[|\def\jobname{|\textit{dest}|}|]|\input{|\textit{main}|}"|
\end{center}

%%%%%%%%%%%%%%%%%%%%%%%%%%%%%%%%%%%%%%%%%%%%%%%%%%%%%%%%%%%%%%%%%%%%%%%%%%%%%%%%
\subsection{Manual Code}
\label{sec:manual}

In case one cannot be certain whether the definitions file |childdoc.def|
is installed on the target \TeX{} distribution
and one prefers not to ship it,
it is conceivable to paste a few relevant commands into the sources.

To that end, drop all statements |\input{childdoc.def}|
and perform the replacements as outlined below.
Instead of |\childdocmain{|\textit{main}|}| add the following code
to the top of the main file:
%
\begin{center}
\begin{tabular}{l}
|\||ifdefined\childdocname\endinput\||fi\newif\ifchilddoc|\\
|\edef\childdocname{\scantokens\expandafter{\jobname\noexpand}}|\\
|\def\childdocmain{|\textit{main}|}\||ifx\childdocmain\childdocname\||else|\\
|\childdoctrue\includeonly{\childdocname}\let\jobname\childdocmain\||fi|\\
\end{tabular}
\end{center}
%
Instead of |\childdocof{|\textit{main}|}| just include the main file
at the top of each child file:
%
\begin{center}
|\input{|\textit{main}|}|
\end{center}
%
A simple redirection |\childdocforward{|\textit{dest}|}| is achieved by:
%
\begin{center}
|\def\jobname{|\textit{dest}|}\input{\jobname}|
\end{center}
%
The redirection with prefix
|\childdocforwardprefix[|\textit{prefix}|]{|\textit{dest}|}|
is accomplished by:
%
\begin{center}
\begin{tabular}{l}
|{\edef\jobname{\scantokens\expandafter{\jobname\noexpand}}|\\
|\def\redirectjob |\textit{prefix}|#1~~~{\gdef\jobname{|\textit{dest}|#1}}|\\
|\expandafter\redirectjob\jobname~~~}\input{\jobname}|
\end{tabular}
\end{center}

In an alternative approach,
child documents can be compiled by a specific command line
without additional code or specific definitions:
%
\begin{center}
|... -jobname "|\textit{target}|" "|[\textit{flags}]%
|\includeonly{|\textit{dest}|}\input{|\textit{main}|}"|
\end{center}
%

%%%%%%%%%%%%%%%%%%%%%%%%%%%%%%%%%%%%%%%%%%%%%%%%%%%%%%%%%%%%%%%%%%%%%%%%%%%%%%%%
%%%%%%%%%%%%%%%%%%%%%%%%%%%%%%%%%%%%%%%%%%%%%%%%%%%%%%%%%%%%%%%%%%%%%%%%%%%%%%%%
\section{Information}

%%%%%%%%%%%%%%%%%%%%%%%%%%%%%%%%%%%%%%%%%%%%%%%%%%%%%%%%%%%%%%%%%%%%%%%%%%%%%%%%
\subsection{Copyright}

Copyright \copyright{} 2017--2018 Niklas Beisert

This work may be distributed and/or modified under the
conditions of the \LaTeX{} Project Public License, either version 1.3
of this license or (at your option) any later version.
The latest version of this license is in
  \url{http://www.latex-project.org/lppl.txt}
and version 1.3 or later is part of all distributions of \LaTeX{}
version 2005/12/01 or later.

This work has the LPPL maintenance status `maintained'.

The Current Maintainer of this work is Niklas Beisert.

This work consists of the files |README.txt|, |childdoc.ins| and |childdoc.dtx|
as well as the derived files |childdoc.def|, |cdocsamp.tex|
with |cdocsch1.tex|, |cdocsch2.tex|, |cdocspt3.tex|, |cdocspt4.tex|,
|cdocsdrf.tex|, |cdocsfn1.tex|, |cdocsfn2.tex|
as well as |childdoc.pdf|.

%%%%%%%%%%%%%%%%%%%%%%%%%%%%%%%%%%%%%%%%%%%%%%%%%%%%%%%%%%%%%%%%%%%%%%%%%%%%%%%%
\subsection{Files and Installation}

The package consists of the files:
%
\begin{center}
\begin{tabular}{ll}
    |README.txt|   & readme file \\
    |childdoc.ins| & installation file \\
    |childdoc.dtx| & source file \\
    |childdoc.def| & definition file \\
    |cdocsamp.tex| & sample main file \\
    |cdocsch1.tex| & sample include file \\
    |cdocsch2.tex| & sample include file \\
    |cdocspt3.tex| & sample part file \\
    |cdocspt4.tex| & sample part file \\
    |cdocsdrf.tex| & sample redirection file \\
    |cdocsfn1.tex| & sample redirection file \\
    |cdocsfn2.tex| & sample redirection file \\
    |childdoc.pdf| & manual
\end{tabular}
\end{center}
%
The distribution consists of the files
|README.txt|, |childdoc.ins| and |childdoc.dtx|.
%
\begin{itemize}
\item
Run (pdf)\LaTeX{} on |childdoc.dtx|
to compile the manual |childdoc.pdf| (this file).
\item
Run \LaTeX{} on |childdoc.ins| to create the definitions file |childdoc.def|
and the sample |cdocsamp.tex| with include files
|cdocsch1.tex|, |cdocsch2.tex|, |cdocspt3.tex|, |cdocspt4.tex|,
|cdocsdrf.tex|, |cdocsfn1.tex|, |cdocsfn2.tex|.
Then copy the file |childdoc.def| to an appropriate directory of your \LaTeX{}
distribution, e.g.\ \textit{texmf-root}|/tex/latex/childdoc|.
\end{itemize}

%%%%%%%%%%%%%%%%%%%%%%%%%%%%%%%%%%%%%%%%%%%%%%%%%%%%%%%%%%%%%%%%%%%%%%%%%%%%%%%%
\subsection{Related CTAN Packages}

There are several other packages which offer a similar functionality:
%
\begin{itemize}
\item
The packages
\href{http://ctan.org/pkg/docmute}{\textsf{docmute}},
\href{http://ctan.org/pkg/includex}{\textsf{includex}} and
\href{http://ctan.org/pkg/standalone}{\textsf{standalone}}
provide commands to include only the document body of
a child file thus allowing both files to be compiled individually.
\item
The packages \href{http://ctan.org/pkg/subdocs}{\textsf{subdocs}}
and \href{http://ctan.org/pkg/subfiles}{\textsf{subfiles}}
provide structures in which the main and child documents can be
encapsulated and allowing them to be compiled individually.
The inclusion mechanism is different from the conventional |\include|.
\item
The package \href{http://ctan.org/pkg/combine}{\textsf{combine}}
is an elaborate solution to combine several documents into one.
\end{itemize}
%
See also the CTAN topic \href{http://ctan.org/topic/subdocs}{\textsf{subdocs}}
for further related packages.
The present package differs from the above solutions in that
a document structure constructed with the conventional |\include| mechanism
just needs two extra commands at the top of every file
such that all constituent files can be compiled individually.

%%%%%%%%%%%%%%%%%%%%%%%%%%%%%%%%%%%%%%%%%%%%%%%%%%%%%%%%%%%%%%%%%%%%%%%%%%%%%%%%
%\subsection{Feature Suggestions}
%
%The following is a list of features which may be useful for future
%versions of this package:
%%
%\begin{itemize}
%\item
%\ldots
%\end{itemize}

%%%%%%%%%%%%%%%%%%%%%%%%%%%%%%%%%%%%%%%%%%%%%%%%%%%%%%%%%%%%%%%%%%%%%%%%%%%%%%%%
\subsection{Revision History}

%%%%%%%%%%%%%%%%%%%%%%%%%%%%%%%%%%%%%%%%
\paragraph{v2.0:} 2018/12/30

\begin{itemize}
\item
immediate forward processing
\item
added |\childdocby| mechanism
\item
manual restructured
\end{itemize}

%%%%%%%%%%%%%%%%%%%%%%%%%%%%%%%%%%%%%%%%
\paragraph{v1.6:} 2018/01/17

\begin{itemize}
\item
application for development of include files
\item
corrections to manual
\end{itemize}

%%%%%%%%%%%%%%%%%%%%%%%%%%%%%%%%%%%%%%%%
\paragraph{v1.5:} 2017/05/21

\begin{itemize}
\item
more complete structuring introduced
\item
|\childdocof| introduced
\item
|\childdoc| renamed to |\childdocmain|
\item
|\childredirect| renamed to |\childdocforward| and |\childdocforwardprefix|
and functionality expanded
\end{itemize}

%%%%%%%%%%%%%%%%%%%%%%%%%%%%%%%%%%%%%%%%
\paragraph{v1.0:} 2017/04/27

\begin{itemize}
\item
manual and install package
\item
first version published on CTAN
\end{itemize}

%%%%%%%%%%%%%%%%%%%%%%%%%%%%%%%%%%%%%%%%
\paragraph{v0.6:} 2017/04/26

\begin{itemize}
\item
redirection mechanism added
\end{itemize}

%%%%%%%%%%%%%%%%%%%%%%%%%%%%%%%%%%%%%%%%
\paragraph{v0.5:} 2017/04/26

\begin{itemize}
\item
functionality in definition file
\end{itemize}


%%%%%%%%%%%%%%%%%%%%%%%%%%%%%%%%%%%%%%%%%%%%%%%%%%%%%%%%%%%%%%%%%%%%%%%%%%%%%%%%
%%%%%%%%%%%%%%%%%%%%%%%%%%%%%%%%%%%%%%%%%%%%%%%%%%%%%%%%%%%%%%%%%%%%%%%%%%%%%%%%
%%%%%%%%%%%%%%%%%%%%%%%%%%%%%%%%%%%%%%%%%%%%%%%%%%%%%%%%%%%%%%%%%%%%%%%%%%%%%%%%
\appendix

\settowidth\MacroIndent{\rmfamily\scriptsize 000\ }

 \DocInput{childdoc.dtx}

\end{document}
%</driver>
% \fi
%
% %%%%%%%%%%%%%%%%%%%%%%%%%%%%%%%%%%%%%%%%%%%%%%%%%%%%%%%%%%%%%%%%%%%%%%%%%%%%%%
% %%%%%%%%%%%%%%%%%%%%%%%%%%%%%%%%%%%%%%%%%%%%%%%%%%%%%%%%%%%%%%%%%%%%%%%%%%%%%%
% \section{Sample}
%\iffalse
%<*samplemain>
%\fi
%
% The following presents a sample document
% with two chapters, two parts, a title page,
% a compile flag as well as three forwarding files to set the flag.
% It consists of eight |.tex| files:
% \begin{center}
% \begin{tabular}{ll}
% |cdocsamp.tex|&main file\\
% |cdocsch1.tex|&include file for chapter 1\\
% |cdocsch2.tex|&include file for chapter 2\\
% |cdocspt3.tex|&include file for part 3\\
% |cdocspt4.tex|&include file for part 4\\
% |cdocsdrf.tex|&forwarding file for main file in draft mode\\
% |cdocsfi1.tex|&forwarding file for final version of chapter 1\\
% |cdocsfi2.tex|&forwarding file for final version of chapter 2\\
% \end{tabular}
% \end{center}
% Each of the eight files can be compiled directly by the \LaTeX{} compiler.
%
% %%%%%%%%%%%%%%%%%%%%%%%%%%%%%%%%%%%%%%
% \paragraph{Main File.}
%
% The main file is called |cdocsamp.tex|.
%
% Load the \textsf{childdoc} definitions and
% declare the filename for the main document:
%    \begin{macrocode}
\input{childdoc.def}
\childdocmain{}
%    \end{macrocode}

% Optional override for |\version| flag:
%    \begin{macrocode}
%%\ifchilddoc\else\providecommand{\version}{draft}\fi
%    \end{macrocode}

% Define the default values for the |\version| flag
% (|final| for the main file and |draft| for childs):
%    \begin{macrocode}
\ifchilddoc
\providecommand{\version}{draft}
\else
\providecommand{\version}{final}
\fi
%    \end{macrocode}

% Load the standard document class:
%    \begin{macrocode}
\documentclass[12pt]{article}
%    \end{macrocode}

% Start the document body:
%    \begin{macrocode}
\begin{document}
%    \end{macrocode}

% Declare a title page.
% Print title, part of document being processed and version flag:
%    \begin{macrocode}
\addtocounter{page}{-1}
\begin{center}
{\LARGE\bfseries{}childdoc example\par}
\vspace{1cm}
\ifchilddoc
\ifchilddocmanual part\else chapter\fi:
`\childdocname' of `\childdocjob'\par
\else
main document: `\childdocjob'\par
\fi
version: \version\par
\end{center}
\newpage
%    \end{macrocode}

% Manually include selected file,
% otherwise process as usual:
%    \begin{macrocode}
\ifchilddocmanual
\section*{part `\childdocname'}
\input{\childdocname}
\else
%    \end{macrocode}

% Include the two chapters:
%    \begin{macrocode}
\include{cdocsch1}
\include{cdocsch2}
%    \end{macrocode}

% Include the two parts unless only chapters should be displayed:
%    \begin{macrocode}
\ifchilddoc\else
\section{part three}
\input{cdocspt3}
\section{part four}
\input{cdocspt4}
\fi
%    \end{macrocode}

% Process as usual until here:
%    \begin{macrocode}
\fi
%    \end{macrocode}

% End of document body:
%    \begin{macrocode}
\end{document}
%    \end{macrocode}
%\iffalse
%</samplemain>
%\fi
%
% %%%%%%%%%%%%%%%%%%%%%%%%%%%%%%%%%%%%%%
% \paragraph{Chapter Include Files.}
%
% The include files are called |cdocsch1.tex| and |cdocsch2.tex|.
%
%\iffalse
%<*samplechap1|samplechap2>
%\fi

% Optional override for |\version| flag:
%    \begin{macrocode}
%%\providecommand{\version}{final}
%    \end{macrocode}

% Include the main document:
%    \begin{macrocode}
\input{childdoc.def}
\childdocof{cdocsamp}
%    \end{macrocode}

%\iffalse
%</samplechap1|samplechap2>
%\fi
%
%\iffalse
%<*samplechap1>
%\fi
% Some text for chapter 1:
%    \begin{macrocode}
\section{one}
some text in chapter one
%    \end{macrocode}

%\iffalse
%</samplechap1>
%\fi
% Some text for chapter 2:
%\iffalse
%<*samplechap2>
%\fi
%    \begin{macrocode}
\section{two}
more text in chapter two
%    \end{macrocode}

%\iffalse
%</samplechap2>
%\fi
%
% %%%%%%%%%%%%%%%%%%%%%%%%%%%%%%%%%%%%%%
% \paragraph{Part Include Files.}
%
% The include files are called |cdocspt3.tex| and |cdocspt4.tex|.
%
%\iffalse
%<*samplepart3|samplepart4>
%\fi

% Optional override for |\version| flag:
%    \begin{macrocode}
%%\providecommand{\version}{final}
%    \end{macrocode}

% Include the main document:
%    \begin{macrocode}
\input{childdoc.def}
\childdocby{cdocsamp}
%    \end{macrocode}

%\iffalse
%</samplepart3|samplepart4>
%\fi
%
%\iffalse
%<*samplepart3>
%\fi
% Some text for part 3:
%    \begin{macrocode}
some text in part three
%    \end{macrocode}

%\iffalse
%</samplepart3>
%\fi
% Some text for part 4:
%\iffalse
%<*samplepart4>
%\fi
%    \begin{macrocode}
more text in part four
%    \end{macrocode}

%\iffalse
%</samplepart4>
%\fi
%
% %%%%%%%%%%%%%%%%%%%%%%%%%%%%%%%%%%%%%%
% \paragraph{Forwarding for a Complete Draft.}
%
% The following forwarding file |cdocsdrf.tex|
% compiles the main document in draft mode:
%\iffalse
%<*sampledraft>
%\fi
%    \begin{macrocode}
\def\version{draft}
\input{childdoc.def}
\childdocforward{cdocsamp}
%    \end{macrocode}

%\iffalse
%</sampledraft>
%\fi
%
% %%%%%%%%%%%%%%%%%%%%%%%%%%%%%%%%%%%%%%
% \paragraph{Forwarding for Final Version of the Chapters.}
%
% The following forwarding files |cdocsfn1.tex| and |cdocsfn2.tex|
% (with identical content)
% compile the final versions of the child documents
% |cdocsch1.tex| and |cdocsch2.tex|, respectively:
%\iffalse
%<*samplefinal>
%\fi
%    \begin{macrocode}
\def\version{final}
\input{childdoc.def}
\childdocforwardprefix[cdocsamp]{cdocsfn}{cdocsch}
%    \end{macrocode}

%\iffalse
%</samplefinal>
%\fi
%
% %%%%%%%%%%%%%%%%%%%%%%%%%%%%%%%%%%%%%%
% \paragraph{Command Line Processing.}
%
% The following three command lines generate the output files
% |cdocscld|, |cdocscl1| and |cdocscl2|
% which should be identical to
% |cdocsdrf|, |cdocsch1| and |cdocsfn2|, respectively:
% \begin{center}
% \begin{tabular}{l}
% |latex -jobname cdocscld \|\\
% |  "\def\version{draft}\input{childdoc.def}\childdocforward{cdocsamp}"|\\
% |latex -jobname cdocscl1 \|\\
% |  "\input{childdoc.def}\childdocforward[cdocsamp]{cdocsch1}"|\\
% |latex -jobname cdocscl2 \|\\
% |  "\def\version{final}\input{childdoc.def}\childdocforward{cdocsch2}"|
% \end{tabular}
% \end{center}
% Note that the trailing backslash on each first line
% merely continues the input to the second line
% (for convenient cut ant paste).
% Furthermore, the command |latex| can be replaced by any
% of its alternative versions such as |pdflatex|.
%
% %%%%%%%%%%%%%%%%%%%%%%%%%%%%%%%%%%%%%%%%%%%%%%%%%%%%%%%%%%%%%%%%%%%%%%%%%%%%%%
% %%%%%%%%%%%%%%%%%%%%%%%%%%%%%%%%%%%%%%%%%%%%%%%%%%%%%%%%%%%%%%%%%%%%%%%%%%%%%%
% \section{Implementation}
%\iffalse
%<*package>
%\fi
%
% This section describes the definitions file |childdoc.def|.

% The definitions cannot be loaded using |\usepackage| or |\RequirePackage|
% which has a mechanism to prevent loading a style file more than once.
% When loading the definitions by means of |\input|
% multiple instances have to be prevented manually:
%\iffalse
%This code needs to be before the `\ProvidesFile' directive
%which is defined at the beginning of this file.
%Therefore it is also placed there and commented out here.
%</package>
%<*discard>
%\fi
%    \begin{macrocode}
\ifdefined\childdocmain\endinput\fi
%    \end{macrocode}
%\iffalse
%</discard>
%<*package>
%\fi
%
% \macro{\ifchilddoc}
% \macro{\ifchilddocmanual}
% The conditional |\ifchilddoc| tells whether a
% child (true) or main (false) document is being compiled.
% The conditional |\ifchilddocmanual| tells whether
% the |\includeonly| mechanism is used (false) or
% the selection of child files must be performed manually (true).
% The definitions initialise to false:
%    \begin{macrocode}
\newif\ifchilddoc
\newif\ifchilddocmanual
%    \end{macrocode}

% \macro{\childdocname}
% \macro{\childdocjob}
% The macro |\childdocname| stores the name of the main document
% to be compiled. The macro |\childdocjob| stores the name of
% the document on which the \LaTeX{} compiler was originally invoked.
% The content of |\jobname| cannot be compared
% to filenames specified in the source due to different catcodes.
% The following code rescans |\jobname|, stores the result
% in |\childdocname| and saves a copy in |\childdocjob|:
%    \begin{macrocode}
\edef\childdocname{\scantokens\expandafter{\jobname\noexpand}}
\let\childdocjob\childdocname
%    \end{macrocode}

% \macro{\childdocdisable}
% The macro |\childdocdisable| prevents the main file
% from being processed more than once.
% At this stage, the main document command |\childdocmain|
% is assumed to be called once again where it should do nothing.
% Any subsequent call to it should prevent
% a secondary processing of the main document
% It overwrites the forwarding commands
% |\childdocof| and |\childdocforward|
% with empty macros to prevent further inclusions of the main document:
%    \begin{macrocode}
\newcommand{\childdocdisable}
{
  \renewcommand{\childdocmain}[1]{\renewcommand{\childdocmain}[1]{\endinput}}
  \renewcommand{\childdocof}[1]{}
  \renewcommand{\childdocby}[2][]{}
  \renewcommand{\childdocforward}[2][]{}
  \renewcommand{\childdocdisable}{}
}
%    \end{macrocode}

% \macro{\childdocmain}
% The macro |\childdocmain| is to be called at the top of the main file
% with nothing or the main filename (without extension) as argument.
% First, it breaks loops.
% If the argument is not empty and does not match |\childdocname|
% (which is set by the first inclusion of |childdoc.def|),
% |\ifchilddoc| is set to true, |\includeonly| is applied to the child file
% and |\jobname| is set to the main file
% (for proper handling of |.aux| files):
%    \begin{macrocode}
\newcommand{\childdocmain}[1]
{
  \childdocdisable\childdocmain{}
  \if?#1?\else
    \begingroup
      \def\childdoctmp{#1}
      \ifx\childdoctmp\childdocname
        \def\childdoctmp{}
      \else
        \def\childdoctmp
        {
          \childdoctrue
          \includeonly{\childdocname}
          \def\childdocjob{#1}
          \def\jobname{#1}
        }
      \fi
      \expandafter
    \endgroup
    \childdoctmp
  \fi
}
%    \end{macrocode}

% \macro{\childdocof}
% The command |\childdocof| redirects
% compilation to the main file |#1|.
%    \begin{macrocode}
\newcommand{\childdocof}[1]
{
  \childdocdisable
  \childdoctrue
  \includeonly{\childdocname}
  \def\jobname{#1}
  \def\childdocjob{#1}
  \input{#1}
}
%    \end{macrocode}

% \macro{\childdocby}
% The command |\childdocby| ....
%    \begin{macrocode}
\newcommand{\childdocby}[2][]
{
  \childdocdisable
  \childdoctrue
  \childdocmanualtrue
  \if?#1?\else
    \def\jobname{#2}
  \fi
  \def\childdocjob{#2}
  \input{#2}
  \endinput
}
%    \end{macrocode}

% \macro{\childdocforward}
% The command |\childdocforward| redirects
% compilation to the main file or
% (if the optional argument is given) a child file.
% Parameters are set as if the main file
% or a child file starting with |\childdocof| was compiled.
% Then compilation is handed over to the main file:
%    \begin{macrocode}
\newcommand{\childdocforward}[2][]
{
  \begingroup
    \if?#1?
      \def\childdoctmp
      {
        \def\childdocname{#2}
        \def\childdocjob{#2}
        \def\jobname{#2}
        \input{#2}
        \endinput
      }
    \else
      \def\childdoctmp
      {
        \childdocdisable
        \def\childdocname{#2}
        \childdoctrue
        \includeonly{#2}
        \def\childdocjob{#1}
        \def\jobname{#1}
        \input{#1}
        \endinput
      }
    \fi
    \expandafter
  \endgroup
  \childdoctmp
}
%    \end{macrocode}

% \macro{\childdocforwardprefix}
% The command |\childdocforwardprefix| redirects
% compilation to the main or a child file by means of a pattern.
% The prefix |#1| in the current filename is replaced by |#2|
% and the suffix of the current filename is kept
% (it is assumed that the filename does not contain the substring `|~~~|'
% which is used as a delimiter).
% Compilation is handed over to the new file by |\childdocforward|:
%    \begin{macrocode}
\newcommand{\childdocforwardprefix}[3][]
{
  \begingroup
    \def\childdocextract #2##1~~~{\def\childdoctmp{\childdocforward[#1]{#3##1}}}
    \expandafter\childdocextract\childdocname~~~
    \expandafter
  \endgroup
  \childdoctmp
}
%    \end{macrocode}

% \macro{\childdoc}
% The deprecated macro |\childdoc| is a legacy version of |\childdocmain|:
%    \begin{macrocode}
\newcommand{\childdoc}{\childdocmain}
%    \end{macrocode}

% \macro{\childdocredirect}
% The deprecated macro |\childdocredirect| is a legacy version
% of |\childdocforward| and |\childdocforwardprefix|:
%    \begin{macrocode}
\newcommand{\childdocredirect}[2][]
{
  \begingroup
    \if?#1?
      \def\childdoctmp{\childdocforward{#2}}
    \else
      \def\childdoctmp{\childdocforwardprefix{#1}{#2}}
    \fi
    \expandafter
  \endgroup
  \childdoctmp
}
%    \end{macrocode}

%\iffalse
%</package>
%\fi
%
\endinput
|\\
|\childdocby{|\textit{main}|}|\\
\end{tabular}
\end{center}
%
Both forms have slightly different effects as described above.
The main file is prepared as usual, see \secref{sec:include}.

%%%%%%%%%%%%%%%%%%%%%%%%%%%%%%%%%%%%%%%%%%%%%%%%%%%%%%%%%%%%%%%%%%%%%%%%%%%%%%%%
\subsection{Legacy Detection}
\label{sec:detection}

The directive |\childdocmain| in the main file can detect
whether the complete document or merely a child is to be compiled
even without using the directive |\childdocof|.
This method is deprecated because it is less robust
and there is no compelling reason to use it;
it is merely provided for backward compatibility
and it may be removed in future versions.

If the detection mechanism is to be used,
it is mandatory to correctly specify
the filename of the main file as the argument of |\childdocmain|:
%
\begin{center}
\begin{tabular}{l}
|% \iffalse
%
% childdoc.dtx Copyright (C) 2017-2018 Niklas Beisert
%
% This work may be distributed and/or modified under the
% conditions of the LaTeX Project Public License, either version 1.3
% of this license or (at your option) any later version.
% The latest version of this license is in
%   http://www.latex-project.org/lppl.txt
% and version 1.3 or later is part of all distributions of LaTeX
% version 2005/12/01 or later.
%
% This work has the LPPL maintenance status `maintained'.
%
% The Current Maintainer of this work is Niklas Beisert.
%
% This work consists of the files childdoc.dtx and childdoc.ins
% and the derived files childdoc.def and cdocsamp.tex with
% cdocsch1.tex, cdocsch2.tex, cdocsdrf.tex, cdocsfn1.tex, cdocsfn2.tex.
%
%<package>\ifdefined\childdocmain\endinput\fi
%<package>\ProvidesFile{childdoc.def}[2018/12/30 v2.0 child document driver]
%<samplemain>\ProvidesFile{cdocsamp.tex}[2018/12/30 v2.0 sample for childdoc]
%<*driver>
%\ProvidesFile{childdoc.drv}[2018/12/30 v2.0 childdoc reference manual file]
\PassOptionsToClass{10pt,a4paper}{article}
\documentclass{ltxdoc}

\usepackage[margin=35mm]{geometry}
\usepackage{hyperref}
\usepackage{hyperxmp}
\usepackage[usenames]{color}

\hypersetup{colorlinks=true}
\hypersetup{pdfstartview=FitH}
\hypersetup{pdfpagemode=UseNone}
\hypersetup{pdfsource={}}
\hypersetup{pdflang={en-UK}}
\hypersetup{pdfcopyright={Copyright 2017-2018 Niklas Beisert.
  This work may be distributed and/or modified under the
  conditions of the LaTeX Project Public License, either version 1.3
  of this license or (at your option) any later version.}}
\hypersetup{pdflicenseurl={http://www.latex-project.org/lppl.txt}}
\hypersetup{pdfcontactaddress={ETH Zurich, ITP, HIT K,
  Wolfgang-Pauli-Strasse 27}}
\hypersetup{pdfcontactpostcode={8093}}
\hypersetup{pdfcontactcity={Zurich}}
\hypersetup{pdfcontactcountry={Switzerland}}
\hypersetup{pdfcontactemail={nbeisert@itp.phys.ethz.ch}}
\hypersetup{pdfcontacturl={http://people.phys.ethz.ch/\xmptilde nbeisert/}}

\newcommand{\secref}[1]{\hyperref[#1]{section \ref*{#1}}}

\parskip1ex
\parindent0pt
\let\olditemize\itemize
\def\itemize{\olditemize\parskip0pt}

\begin{document}

\title{The \textsf{childdoc} Package}
\hypersetup{pdftitle={The childdoc Package}}
\author{Niklas Beisert\\[2ex]
  Institut f\"ur Theoretische Physik\\
  Eidgen\"ossische Technische Hochschule Z\"urich\\
  Wolfgang-Pauli-Strasse 27, 8093 Z\"urich, Switzerland\\[1ex]
  \href{mailto:nbeisert@itp.phys.ethz.ch}
  {\texttt{nbeisert@itp.phys.ethz.ch}}}
\hypersetup{pdfauthor={Niklas Beisert}}
\hypersetup{pdfsubject={Manual for the LaTeX2e Package childdoc}}
\date{30 December 2018, \textsf{v2.0}}
\maketitle

\begin{abstract}\noindent
\textsf{childdoc} is a \LaTeXe{} package
that enables the direct compilation
of document sections included by |\include|
to individual files.
\end{abstract}

\begingroup
\parskip0ex
\tableofcontents
\endgroup

%%%%%%%%%%%%%%%%%%%%%%%%%%%%%%%%%%%%%%%%%%%%%%%%%%%%%%%%%%%%%%%%%%%%%%%%%%%%%%%%
%%%%%%%%%%%%%%%%%%%%%%%%%%%%%%%%%%%%%%%%%%%%%%%%%%%%%%%%%%%%%%%%%%%%%%%%%%%%%%%%
\section{Introduction}

\LaTeX{} provides a mechanism to structure a large document (such as a book)
into a main file and several child files (containing the chapters)
using the |\include| command.
This mechanism is beneficial for documents
which span hundreds of pages in order to
make the source file(s) more manageable.
Moreover, compilation can be restricted to
selected child files by means of the |\includeonly| command.
The latter feature can be used to reduce the compilation time while editing
(this was significantly more useful in the earlier days of \LaTeX{})
or to generate a smaller document which is easier to navigate.
Another application of |\includeonly| is to generate
documents consisting of selected parts of the complete document.

However, there are a few drawbacks of the plain |\include| mechanism:
\begin{itemize}
\item
The child files cannot be compiled on their own,
they can only be compiled via the main file.
A naive editing environment
(such as a text editor with an option
to have the current file processed by \LaTeX)
may require one to switch to the main file before compiling;
attempting to compile the child file produces errors.
\item
The main file must be modified (each time)
to adjust the |\includeonly| command
to the present needs. This easily leaves the main file in a messy state.
\item
The generated document will always carry the filename
of the main document. This is inconvenient if
several child files are to be compiled and
to be kept for distribution.
\end{itemize}

The present package provides a simple interface
to make child files individually compilable by \LaTeX{}.
Compiling a child file then has the same effect as compiling
the main file with an |\includeonly| command
to select the appropriate child.
Moreover the generated document will carry the name of the child
rather than the main file.
This resolves all three above issues.

This feature is meant to make the editing of books,
thesis documents and lecture notes somewhat more convenient.
However, the package can also be used efficiently for
composing a series of documents (such as exercise sheets)
which are typically distributed individually.
It then assists the author in generating the individual documents
(potentially in different versions)
as well as a document containing the collected series.
Another application is in developing style files
or other kinds of included material
where compilation of the style file could redirect
to a sample or test file.

%%%%%%%%%%%%%%%%%%%%%%%%%%%%%%%%%%%%%%%%%%%%%%%%%%%%%%%%%%%%%%%%%%%%%%%%%%%%%%%%
%%%%%%%%%%%%%%%%%%%%%%%%%%%%%%%%%%%%%%%%%%%%%%%%%%%%%%%%%%%%%%%%%%%%%%%%%%%%%%%%
\section{Usage}

First of all, the package \textsf{childdoc} is \emph{not} a standard
\LaTeXe{} |.sty| style file! Therefore it needs to be invoked in
a non-standard way.

%%%%%%%%%%%%%%%%%%%%%%%%%%%%%%%%%%%%%%%%%%%%%%%%%%%%%%%%%%%%%%%%%%%%%%%%%%%%%%%%
\subsection{Included Files}
\label{sec:include}

%%%%%%%%%%%%%%%%%%%%%%%%%%%%%%%%%%%%%%%%
\DescribeMacro{\childdocmain}
To use the package, add the commands
\begin{center}
\begin{tabular}{l}
|\input{childdoc.def}|\\
|\childdocmain{}|\\
\end{tabular}
\end{center}
at the very top of the main \LaTeX{} file,
in particular \emph{before} the |\documentclass| statement!
The argument of |\childdocmain| should be left empty
(but it must be present).

%%%%%%%%%%%%%%%%%%%%%%%%%%%%%%%%%%%%%%%%
\DescribeMacro{\childdocof}
Furthermore, add the commands
\begin{center}
\begin{tabular}{l}
|\input{childdoc.def}|\\
|\childdocof{|\textit{main}|}|\\
\end{tabular}
\end{center}
at the top of every child file \textit{child}
which is included by |\include{|\textit{child}|}|
from within the main file
(or at least for those files to be compiled individually).
The argument \textit{main} must be the filename of the main file.

There are a couple of
considerations in setting up the main and child documents:

%%%%%%%%%%%%%%%%%%%%%%%%%%%%%%%%%%%%%%%%
\paragraph{Restrictions.}

Please note the following restrictions:
\begin{itemize}
\item
|\childdocmain| must be called with one argument \textit{main}
to ensure compatibility with earlier version of the package.
It must either be empty (|\childdocmain{}|)
or precisely match the filename of the main file in which it is specified.
See \secref{sec:detection} for further information.
\item
The filename \textit{main} must be specified without the |.tex| extension.
\item
The filename \textit{main} is case sensitive
(even in case-insensitive file systems)
due to internal string comparison.
\item
The argument \textit{main} should be fully expanded, it cannot be a macro.
\item
Subdirectories and special characters should be avoided in filenames.
\item
The command |\childdocmain{|\textit{main}|}| must be followed by a whitespace.
It should not be followed immediately by another command
or by a comment mark `|%|'.
This is because the \TeX{} parser reads the token immediately following
the argument of |\childdocmain| and puts it
at the beginning of every child section;
however, a white\-space is ignored.
\end{itemize}

%%%%%%%%%%%%%%%%%%%%%%%%%%%%%%%%%%%%%%%%
\paragraph{Content of Main File.}

It is advisable to place all content in the child files included by |\include|.
Any output contained in the main file will appear in all child documents
unless suppressed manually;
it cannot be suppressed automatically by the |\includeonly| directive
and thus should normally be avoided.
A method to include some content in the main file
by means of conditional processing is described in \secref{sec:conditional}.

%%%%%%%%%%%%%%%%%%%%%%%%%%%%%%%%%%%%%%%%
\paragraph{Page Numbering.}

When only a part of the document is compiled,
the appropriate numbering of pages
(as well as other status parameters)
is determined from the |.aux| files.
The latter contain information from previous passes.
However this information needs to propagate through
all intermediate child documents.
Therefore the page numbering in child documents may well
be inconsistent until the complete document is compiled at least once.

A useful (if unconventional) way to always ensure a consistent
page numbering is to restart the numbering in each child document
and denote the pages by `\textit{child}|.|\textit{page}'
where \textit{child} represents the chapter/section number of the child file.
This can be achieved by the command
|\numberwithin{page}{|\textit{child}|}|
of the \textsf{amsmath} package
where \textit{child} can be |chapter| or |section|
depending on the chosen structuring.
Alternatively, one can modify the macro |\thepage| appropriately
and reset the counter |page| at the start of each child file.

%%%%%%%%%%%%%%%%%%%%%%%%%%%%%%%%%%%%%%%%%%%%%%%%%%%%%%%%%%%%%%%%%%%%%%%%%%%%%%%%
\subsection{Conditional Processing}
\label{sec:conditional}

The package provides a mechanism to compile different versions
of a document. To customise the versions further some conditional processing
can come in handy to distinguish which version is being compiled.
The package provides two macros to describe the compilation context:

%%%%%%%%%%%%%%%%%%%%%%%%%%%%%%%%%%%%%%%%
\DescribeMacro{\ifchilddoc}
The conditional |\ifchilddoc| distinguishes between the compilation of
child documents and the main document:
%
\begin{center}
|\ifchilddoc |\textit{child-code}| |[|\||else |\textit{main-code}]| \||fi|
\end{center}

%%%%%%%%%%%%%%%%%%%%%%%%%%%%%%%%%%%%%%%%
\DescribeMacro{\childdocname}
\DescribeMacro{\childdocjob}
The macro |\childdocname| contains the filename (without extension)
of the main or child file being processed.
Note that |\childdocjob| will always contain the name of the main file.

%%%%%%%%%%%%%%%%%%%%%%%%%%%%%%%%%%%%%%%%
\paragraph{Title Page.}

Conditional processing can be used to include a title or banner page
in the main document when proper precautions are taken.
Importantly, the code in the main file should ensure that the page counter
(as well as other status parameters which are stored in the |.aux| files)
takes the same value after the conditional processing.
Otherwise the page numbers may take divergent values
depending on which part is compiled.

For example, a title page could be declared by:
%
\begin{center}
\begin{tabular}{l}
|\ifchilddoc\||else|\\
|\addtocounter{page}{-1}|\\
\textit{code for title page}\\
|\newpage|\\
|\||fi|
\end{tabular}
\end{center}
%
A banner page for the child documents can be generated by:
%
\begin{center}
\begin{tabular}{l}
|\ifchilddoc|\\
|\addtocounter{page}{-1}|\\
\textit{code for banner page}\\
|\newpage|\\
|\||fi|
\end{tabular}
\end{center}
%
Here one could write a message such as:
\begin{center}
|This is the part \childdocname{} of \childdocjob{}.|
\end{center}

%%%%%%%%%%%%%%%%%%%%%%%%%%%%%%%%%%%%%%%%%%%%%%%%%%%%%%%%%%%%%%%%%%%%%%%%%%%%%%%%
\subsection{Flags}
\label{sec:flags}

The package makes it easy to generate different versions
of the main or child documents.
To this end compilation flags can be defined
and assigned different default values.
They will be particularly useful in conjunction
with the forwarding mechanism described in \secref{sec:forward}.

For example, it may be useful to have a flag |\version|
which can be set to |draft| or |final|.
The document source will contain some conditional code
depending on the value of |\version|.
Suppose further, the flag should default to |final| for the main file
and to |draft| for child files
which is a natural assignment for editing the document.
This is achieved by placing the following code
in the preamble of the main document
(below the |\childdocmain| directive):
%
\begin{center}
\begin{tabular}{l}
|\ifchilddoc|\\
|\providecommand{\version}{draft}|\\
|\||else|\\
|\providecommand{\version}{final}|\\
|\||fi|
\end{tabular}
\end{center}
%
The definition by |\providecommand| makes sure
that previous definitions are not overwritten.
Further statements |\providecommand{\version}{...}|
can thus be added before the above code to override it.

For the main file, one might add a line
(between |\childdocmain| and the above block)
%
\begin{center}
|%\ifchilddoc\||else\providecommand{\version}{draft}\||fi|
\end{center}
%
which can be uncommented to produce a draft version.
Likewise one can add a line to the very top of a child file
(above the |\childdocof{|\textit{main}|}| directive)
%
\begin{center}
|%\providecommand{\version}{final}|
\end{center}
%
which can be uncommented to produce the final version of this child document.

%%%%%%%%%%%%%%%%%%%%%%%%%%%%%%%%%%%%%%%%%%%%%%%%%%%%%%%%%%%%%%%%%%%%%%%%%%%%%%%%
\subsection{Forwarding}
\label{sec:forward}

Different versions of the main or child documents
using compilation flags as described in \secref{sec:flags}
can be (permanently) stored in different files
for convenient compilation, viewing and distribution.
To this end, the package defines a command
to pass on compilation to a different file:

%%%%%%%%%%%%%%%%%%%%%%%%%%%%%%%%%%%%%%%%
\DescribeMacro{\childdocforward}
The command |\childdocforward| redirects processing to
another source file:
%
\begin{center}
\begin{tabular}{l}
|\input{childdoc.def}|\\
|\childdocforward[|\textit{main}|]{|\textit{dest}|}|\\
\end{tabular}
\end{center}
%
The argument \textit{dest} is the destination file
(without extension).
It should be the main file or one of the child files.
Note that further \textsf{childdoc} directives
such as |\childdocof| and |\childdocforward|
in the indicated file will be processed in this form.
The optional argument \textit{main}
passes on directly to the main file \textit{main}
while pretending to compile the child \textit{dest}.
This form behaves as if \textit{dest}
issues |\childdocof{|\textit{main}|}| right away,
and no further \textsf{childdoc} directives will be processed.

%%%%%%%%%%%%%%%%%%%%%%%%%%%%%%%%%%%%%%%%
\DescribeMacro{\...prefix}
In the alternative form |\childdocforwardprefix|,
%
\begin{center}
\begin{tabular}{l}
|\input{childdoc.def}|\\
|\childdocforwardprefix[|\textit{main}|]{|\textit{prefix}|}{|\textit{dest}|}|
\end{tabular}
\end{center}
%
the destination file is determined by a pattern
depending on the current file:
To make this work, the current file must be called
`{\textit{prefix}\hspace{0.2em}\textit{suffix}}'
with \textit{prefix} matching precisely the argument.
Processing is then passed on to the file
`{\textit{dest}\hspace{0.2em}\textit{suffix}}'.
Surely, the same effect is achieved by
directly specifying the
argument `{\textit{dest}\hspace{0.2em}\textit{suffix}}'
in the first form.
However, that requires to set up a different file
for each child. With the alternative form of the command
all these files can have exactly the same content
which simplifies setting them up and maintaining them.

For example, the following file |draft.tex|
with a compilation flag |\version| as described in \secref{sec:flags}
compiles the main document as a draft:
%
\begin{center}
\begin{tabular}{l}
|\def\version{draft}|\\
|\input{childdoc.def}|\\
|\childdocforward{|\textit{main}|}|
\end{tabular}
\end{center}
%
Likewise, the following files |final|\textit{nn}|.tex|
compile the final version of the child document
|child|\textit{nn}|.tex|:
%
\begin{center}
\begin{tabular}{l}
|\def\version{final}|\\
|\input{childdoc.def}|\\
|\childdocforwardprefix{final}{child}|
\end{tabular}
\end{center}
%

Note that when several versions of a main file and/or of each child file
are to be generated, it may be convenient to set up a |Makefile| or
shell script to automatise the process.

%%%%%%%%%%%%%%%%%%%%%%%%%%%%%%%%%%%%%%%%%%%%%%%%%%%%%%%%%%%%%%%%%%%%%%%%%%%%%%%%
\subsection{Command Line Processing}
\label{sec:commandline}

The effect of redirection files can also be achieved by invoking
the \LaTeX{} compiler with a more elaborate command line.
Most conveniently this should be done as part
of a shell script or a |Makefile|.

When using \textsf{childdoc} in the main file, the following
command lines effectively perform a redirection
(note that depending on the shell being used,
backslashes may have to be doubled: `|\|' $\to$ `|\\|'):
%
\begin{center}
|... -jobname "|\textit{target}|" |\\|"|[\textit{flags}]%
|\input{childdoc.def}\childdocforward[|\textit{main}|]{|\textit{dest}|}"|
\end{center}
%
Here \textit{target} is the name of the output file,
\textit{main} is the name of the main file
and \textit{dest} is the name of the main or child file to be processed
(all filenames without extensions).
The optional argument \textit{main} can be omitted
if \textit{main} matches \textit{dest}.
Optionally, compilation \textit{flags} can be defined via |\def| commands.
This command line makes the \TeX{} engine believe
it is compiling the file \textit{target}
whose content is specified as the latter parameter.
The provided code then forwards the processing to
\textit{main} or \textit{dest} as described in \secref{sec:forward}.

%%%%%%%%%%%%%%%%%%%%%%%%%%%%%%%%%%%%%%%%%%%%%%%%%%%%%%%%%%%%%%%%%%%%%%%%%%%%%%%%
\subsection{Include by Input}
\label{sec:input}

Including child documents by |\include| has some restrictions by design.
Most notably, the content of a child document always occupies
its own set of pages; pages cannot be shared between child documents.
Usually, this behaviour makes perfect sense
because each child document contain an essential part of the document.
However, in some situations it may be desirable to compose
a document from a collection of parts
without having mandatory page breaks between then.
For this case, the package
provides a mechanism to include parts
by |\input| which can also be processed individually.
However, by construction this mechanism
requires manual handling of the content to be output.

%%%%%%%%%%%%%%%%%%%%%%%%%%%%%%%%%%%%%%%%
\DescribeMacro{\ifchilddocmanual}
The main file should be prepared as usual, see \secref{sec:include}.
However, the document body must make a distinction
between processing of an individual part and of the main document, e.g.:
%
\begin{center}
\begin{tabular}{l}
|\ifchilddocmanual|\\
|\input{\childdocname}|\\
|\||else|\\
\textit{document body with }|\input{|\textit{part}|}|\\
|\||fi|
\end{tabular}
\end{center}
%
The conditional |\ifchilddocmanual| is true whenever
a part to be included by |\input| is being compiled,
and the name of the part is stored in |\childdocname|.

%%%%%%%%%%%%%%%%%%%%%%%%%%%%%%%%%%%%%%%%
\DescribeMacro{\childdocby}
Each part to be included by |\input| should start with:
%
\begin{center}
\begin{tabular}{l}
|\input{childdoc.def}|\\
|\childdocby{|\textit{main}|}|\\
\end{tabular}
\end{center}
%
The directive |\childdocby| is similar to |\childdocof|
described in \secref{sec:include},
but the subsequent selection of content must be done manually.
To that end, both |\ifchilddoc| and |\ifchilddocmanual|
will be true upon processing of a part,
and the name of the part is stored in |\childdocname|.
Note that |\jobname| will be set to the filename of the current part
so that each part receives an individual |.aux| file
that does not interfere with the |.aux| file(s) of the main document.
This behaviour can be altered by the alternative form
|\childdocby[*]{|\textit{main}|}| (with a non-empty optional argument)
which uses the |.aux| file of the main document
by setting |\jobname| to \textit{main}.

%%%%%%%%%%%%%%%%%%%%%%%%%%%%%%%%%%%%%%%%%%%%%%%%%%%%%%%%%%%%%%%%%%%%%%%%%%%%%%%%
\subsection{Driver Development}
\label{sec:driver}

The \textsf{childdoc} mechanism can also be use for the development
of definition files such as \LaTeX{} styles or classes.
This case differs from the above setup with multiple parts
included by |\include| in that no |\includeonly| should be invoked.
This can be achieved by starting the include file
(before |\ProvidesPackage|) with:
%
\begin{center}
\begin{tabular}{l}
|\input{childdoc.def}|\\
|\childdocforward{|\textit{main}|}|\\
\end{tabular}
\end{center}
%
or alternatively with:
%
\begin{center}
\begin{tabular}{l}
|\input{childdoc.def}|\\
|\childdocby{|\textit{main}|}|\\
\end{tabular}
\end{center}
%
Both forms have slightly different effects as described above.
The main file is prepared as usual, see \secref{sec:include}.

%%%%%%%%%%%%%%%%%%%%%%%%%%%%%%%%%%%%%%%%%%%%%%%%%%%%%%%%%%%%%%%%%%%%%%%%%%%%%%%%
\subsection{Legacy Detection}
\label{sec:detection}

The directive |\childdocmain| in the main file can detect
whether the complete document or merely a child is to be compiled
even without using the directive |\childdocof|.
This method is deprecated because it is less robust
and there is no compelling reason to use it;
it is merely provided for backward compatibility
and it may be removed in future versions.

If the detection mechanism is to be used,
it is mandatory to correctly specify
the filename of the main file as the argument of |\childdocmain|:
%
\begin{center}
\begin{tabular}{l}
|\input{childdoc.def}|\\
|\childdocmain{|\textit{main}|}|\\
\end{tabular}
\end{center}
%
If |\jobname| does not match the argument \textit{main} of |\childdocmain|,
it is assumed that |\jobname| points to the child file to be compiled.
When using |\childdocmain| with the main file specified as argument,
it suffices to start a child file
with just |\input{|\textit{main}|}|
without loading of the package and using |\childdocof|.
If instead all processing is done
with the appropriate \textsf{childdoc} directives,
the argument of \textit{main} of |\childdocmain| can be empty.

An alternative version of the command line processing described
in \secref{sec:commandline} using the detection mechanism reads:
%
\begin{center}
|... -jobname "|\textit{target}|" "|[\textit{flags}]%
[|\def\jobname{|\textit{dest}|}|]|\input{|\textit{main}|}"|
\end{center}

%%%%%%%%%%%%%%%%%%%%%%%%%%%%%%%%%%%%%%%%%%%%%%%%%%%%%%%%%%%%%%%%%%%%%%%%%%%%%%%%
\subsection{Manual Code}
\label{sec:manual}

In case one cannot be certain whether the definitions file |childdoc.def|
is installed on the target \TeX{} distribution
and one prefers not to ship it,
it is conceivable to paste a few relevant commands into the sources.

To that end, drop all statements |\input{childdoc.def}|
and perform the replacements as outlined below.
Instead of |\childdocmain{|\textit{main}|}| add the following code
to the top of the main file:
%
\begin{center}
\begin{tabular}{l}
|\||ifdefined\childdocname\endinput\||fi\newif\ifchilddoc|\\
|\edef\childdocname{\scantokens\expandafter{\jobname\noexpand}}|\\
|\def\childdocmain{|\textit{main}|}\||ifx\childdocmain\childdocname\||else|\\
|\childdoctrue\includeonly{\childdocname}\let\jobname\childdocmain\||fi|\\
\end{tabular}
\end{center}
%
Instead of |\childdocof{|\textit{main}|}| just include the main file
at the top of each child file:
%
\begin{center}
|\input{|\textit{main}|}|
\end{center}
%
A simple redirection |\childdocforward{|\textit{dest}|}| is achieved by:
%
\begin{center}
|\def\jobname{|\textit{dest}|}\input{\jobname}|
\end{center}
%
The redirection with prefix
|\childdocforwardprefix[|\textit{prefix}|]{|\textit{dest}|}|
is accomplished by:
%
\begin{center}
\begin{tabular}{l}
|{\edef\jobname{\scantokens\expandafter{\jobname\noexpand}}|\\
|\def\redirectjob |\textit{prefix}|#1~~~{\gdef\jobname{|\textit{dest}|#1}}|\\
|\expandafter\redirectjob\jobname~~~}\input{\jobname}|
\end{tabular}
\end{center}

In an alternative approach,
child documents can be compiled by a specific command line
without additional code or specific definitions:
%
\begin{center}
|... -jobname "|\textit{target}|" "|[\textit{flags}]%
|\includeonly{|\textit{dest}|}\input{|\textit{main}|}"|
\end{center}
%

%%%%%%%%%%%%%%%%%%%%%%%%%%%%%%%%%%%%%%%%%%%%%%%%%%%%%%%%%%%%%%%%%%%%%%%%%%%%%%%%
%%%%%%%%%%%%%%%%%%%%%%%%%%%%%%%%%%%%%%%%%%%%%%%%%%%%%%%%%%%%%%%%%%%%%%%%%%%%%%%%
\section{Information}

%%%%%%%%%%%%%%%%%%%%%%%%%%%%%%%%%%%%%%%%%%%%%%%%%%%%%%%%%%%%%%%%%%%%%%%%%%%%%%%%
\subsection{Copyright}

Copyright \copyright{} 2017--2018 Niklas Beisert

This work may be distributed and/or modified under the
conditions of the \LaTeX{} Project Public License, either version 1.3
of this license or (at your option) any later version.
The latest version of this license is in
  \url{http://www.latex-project.org/lppl.txt}
and version 1.3 or later is part of all distributions of \LaTeX{}
version 2005/12/01 or later.

This work has the LPPL maintenance status `maintained'.

The Current Maintainer of this work is Niklas Beisert.

This work consists of the files |README.txt|, |childdoc.ins| and |childdoc.dtx|
as well as the derived files |childdoc.def|, |cdocsamp.tex|
with |cdocsch1.tex|, |cdocsch2.tex|, |cdocspt3.tex|, |cdocspt4.tex|,
|cdocsdrf.tex|, |cdocsfn1.tex|, |cdocsfn2.tex|
as well as |childdoc.pdf|.

%%%%%%%%%%%%%%%%%%%%%%%%%%%%%%%%%%%%%%%%%%%%%%%%%%%%%%%%%%%%%%%%%%%%%%%%%%%%%%%%
\subsection{Files and Installation}

The package consists of the files:
%
\begin{center}
\begin{tabular}{ll}
    |README.txt|   & readme file \\
    |childdoc.ins| & installation file \\
    |childdoc.dtx| & source file \\
    |childdoc.def| & definition file \\
    |cdocsamp.tex| & sample main file \\
    |cdocsch1.tex| & sample include file \\
    |cdocsch2.tex| & sample include file \\
    |cdocspt3.tex| & sample part file \\
    |cdocspt4.tex| & sample part file \\
    |cdocsdrf.tex| & sample redirection file \\
    |cdocsfn1.tex| & sample redirection file \\
    |cdocsfn2.tex| & sample redirection file \\
    |childdoc.pdf| & manual
\end{tabular}
\end{center}
%
The distribution consists of the files
|README.txt|, |childdoc.ins| and |childdoc.dtx|.
%
\begin{itemize}
\item
Run (pdf)\LaTeX{} on |childdoc.dtx|
to compile the manual |childdoc.pdf| (this file).
\item
Run \LaTeX{} on |childdoc.ins| to create the definitions file |childdoc.def|
and the sample |cdocsamp.tex| with include files
|cdocsch1.tex|, |cdocsch2.tex|, |cdocspt3.tex|, |cdocspt4.tex|,
|cdocsdrf.tex|, |cdocsfn1.tex|, |cdocsfn2.tex|.
Then copy the file |childdoc.def| to an appropriate directory of your \LaTeX{}
distribution, e.g.\ \textit{texmf-root}|/tex/latex/childdoc|.
\end{itemize}

%%%%%%%%%%%%%%%%%%%%%%%%%%%%%%%%%%%%%%%%%%%%%%%%%%%%%%%%%%%%%%%%%%%%%%%%%%%%%%%%
\subsection{Related CTAN Packages}

There are several other packages which offer a similar functionality:
%
\begin{itemize}
\item
The packages
\href{http://ctan.org/pkg/docmute}{\textsf{docmute}},
\href{http://ctan.org/pkg/includex}{\textsf{includex}} and
\href{http://ctan.org/pkg/standalone}{\textsf{standalone}}
provide commands to include only the document body of
a child file thus allowing both files to be compiled individually.
\item
The packages \href{http://ctan.org/pkg/subdocs}{\textsf{subdocs}}
and \href{http://ctan.org/pkg/subfiles}{\textsf{subfiles}}
provide structures in which the main and child documents can be
encapsulated and allowing them to be compiled individually.
The inclusion mechanism is different from the conventional |\include|.
\item
The package \href{http://ctan.org/pkg/combine}{\textsf{combine}}
is an elaborate solution to combine several documents into one.
\end{itemize}
%
See also the CTAN topic \href{http://ctan.org/topic/subdocs}{\textsf{subdocs}}
for further related packages.
The present package differs from the above solutions in that
a document structure constructed with the conventional |\include| mechanism
just needs two extra commands at the top of every file
such that all constituent files can be compiled individually.

%%%%%%%%%%%%%%%%%%%%%%%%%%%%%%%%%%%%%%%%%%%%%%%%%%%%%%%%%%%%%%%%%%%%%%%%%%%%%%%%
%\subsection{Feature Suggestions}
%
%The following is a list of features which may be useful for future
%versions of this package:
%%
%\begin{itemize}
%\item
%\ldots
%\end{itemize}

%%%%%%%%%%%%%%%%%%%%%%%%%%%%%%%%%%%%%%%%%%%%%%%%%%%%%%%%%%%%%%%%%%%%%%%%%%%%%%%%
\subsection{Revision History}

%%%%%%%%%%%%%%%%%%%%%%%%%%%%%%%%%%%%%%%%
\paragraph{v2.0:} 2018/12/30

\begin{itemize}
\item
immediate forward processing
\item
added |\childdocby| mechanism
\item
manual restructured
\end{itemize}

%%%%%%%%%%%%%%%%%%%%%%%%%%%%%%%%%%%%%%%%
\paragraph{v1.6:} 2018/01/17

\begin{itemize}
\item
application for development of include files
\item
corrections to manual
\end{itemize}

%%%%%%%%%%%%%%%%%%%%%%%%%%%%%%%%%%%%%%%%
\paragraph{v1.5:} 2017/05/21

\begin{itemize}
\item
more complete structuring introduced
\item
|\childdocof| introduced
\item
|\childdoc| renamed to |\childdocmain|
\item
|\childredirect| renamed to |\childdocforward| and |\childdocforwardprefix|
and functionality expanded
\end{itemize}

%%%%%%%%%%%%%%%%%%%%%%%%%%%%%%%%%%%%%%%%
\paragraph{v1.0:} 2017/04/27

\begin{itemize}
\item
manual and install package
\item
first version published on CTAN
\end{itemize}

%%%%%%%%%%%%%%%%%%%%%%%%%%%%%%%%%%%%%%%%
\paragraph{v0.6:} 2017/04/26

\begin{itemize}
\item
redirection mechanism added
\end{itemize}

%%%%%%%%%%%%%%%%%%%%%%%%%%%%%%%%%%%%%%%%
\paragraph{v0.5:} 2017/04/26

\begin{itemize}
\item
functionality in definition file
\end{itemize}


%%%%%%%%%%%%%%%%%%%%%%%%%%%%%%%%%%%%%%%%%%%%%%%%%%%%%%%%%%%%%%%%%%%%%%%%%%%%%%%%
%%%%%%%%%%%%%%%%%%%%%%%%%%%%%%%%%%%%%%%%%%%%%%%%%%%%%%%%%%%%%%%%%%%%%%%%%%%%%%%%
%%%%%%%%%%%%%%%%%%%%%%%%%%%%%%%%%%%%%%%%%%%%%%%%%%%%%%%%%%%%%%%%%%%%%%%%%%%%%%%%
\appendix

\settowidth\MacroIndent{\rmfamily\scriptsize 000\ }

 \DocInput{childdoc.dtx}

\end{document}
%</driver>
% \fi
%
% %%%%%%%%%%%%%%%%%%%%%%%%%%%%%%%%%%%%%%%%%%%%%%%%%%%%%%%%%%%%%%%%%%%%%%%%%%%%%%
% %%%%%%%%%%%%%%%%%%%%%%%%%%%%%%%%%%%%%%%%%%%%%%%%%%%%%%%%%%%%%%%%%%%%%%%%%%%%%%
% \section{Sample}
%\iffalse
%<*samplemain>
%\fi
%
% The following presents a sample document
% with two chapters, two parts, a title page,
% a compile flag as well as three forwarding files to set the flag.
% It consists of eight |.tex| files:
% \begin{center}
% \begin{tabular}{ll}
% |cdocsamp.tex|&main file\\
% |cdocsch1.tex|&include file for chapter 1\\
% |cdocsch2.tex|&include file for chapter 2\\
% |cdocspt3.tex|&include file for part 3\\
% |cdocspt4.tex|&include file for part 4\\
% |cdocsdrf.tex|&forwarding file for main file in draft mode\\
% |cdocsfi1.tex|&forwarding file for final version of chapter 1\\
% |cdocsfi2.tex|&forwarding file for final version of chapter 2\\
% \end{tabular}
% \end{center}
% Each of the eight files can be compiled directly by the \LaTeX{} compiler.
%
% %%%%%%%%%%%%%%%%%%%%%%%%%%%%%%%%%%%%%%
% \paragraph{Main File.}
%
% The main file is called |cdocsamp.tex|.
%
% Load the \textsf{childdoc} definitions and
% declare the filename for the main document:
%    \begin{macrocode}
\input{childdoc.def}
\childdocmain{}
%    \end{macrocode}

% Optional override for |\version| flag:
%    \begin{macrocode}
%%\ifchilddoc\else\providecommand{\version}{draft}\fi
%    \end{macrocode}

% Define the default values for the |\version| flag
% (|final| for the main file and |draft| for childs):
%    \begin{macrocode}
\ifchilddoc
\providecommand{\version}{draft}
\else
\providecommand{\version}{final}
\fi
%    \end{macrocode}

% Load the standard document class:
%    \begin{macrocode}
\documentclass[12pt]{article}
%    \end{macrocode}

% Start the document body:
%    \begin{macrocode}
\begin{document}
%    \end{macrocode}

% Declare a title page.
% Print title, part of document being processed and version flag:
%    \begin{macrocode}
\addtocounter{page}{-1}
\begin{center}
{\LARGE\bfseries{}childdoc example\par}
\vspace{1cm}
\ifchilddoc
\ifchilddocmanual part\else chapter\fi:
`\childdocname' of `\childdocjob'\par
\else
main document: `\childdocjob'\par
\fi
version: \version\par
\end{center}
\newpage
%    \end{macrocode}

% Manually include selected file,
% otherwise process as usual:
%    \begin{macrocode}
\ifchilddocmanual
\section*{part `\childdocname'}
\input{\childdocname}
\else
%    \end{macrocode}

% Include the two chapters:
%    \begin{macrocode}
\include{cdocsch1}
\include{cdocsch2}
%    \end{macrocode}

% Include the two parts unless only chapters should be displayed:
%    \begin{macrocode}
\ifchilddoc\else
\section{part three}
\input{cdocspt3}
\section{part four}
\input{cdocspt4}
\fi
%    \end{macrocode}

% Process as usual until here:
%    \begin{macrocode}
\fi
%    \end{macrocode}

% End of document body:
%    \begin{macrocode}
\end{document}
%    \end{macrocode}
%\iffalse
%</samplemain>
%\fi
%
% %%%%%%%%%%%%%%%%%%%%%%%%%%%%%%%%%%%%%%
% \paragraph{Chapter Include Files.}
%
% The include files are called |cdocsch1.tex| and |cdocsch2.tex|.
%
%\iffalse
%<*samplechap1|samplechap2>
%\fi

% Optional override for |\version| flag:
%    \begin{macrocode}
%%\providecommand{\version}{final}
%    \end{macrocode}

% Include the main document:
%    \begin{macrocode}
\input{childdoc.def}
\childdocof{cdocsamp}
%    \end{macrocode}

%\iffalse
%</samplechap1|samplechap2>
%\fi
%
%\iffalse
%<*samplechap1>
%\fi
% Some text for chapter 1:
%    \begin{macrocode}
\section{one}
some text in chapter one
%    \end{macrocode}

%\iffalse
%</samplechap1>
%\fi
% Some text for chapter 2:
%\iffalse
%<*samplechap2>
%\fi
%    \begin{macrocode}
\section{two}
more text in chapter two
%    \end{macrocode}

%\iffalse
%</samplechap2>
%\fi
%
% %%%%%%%%%%%%%%%%%%%%%%%%%%%%%%%%%%%%%%
% \paragraph{Part Include Files.}
%
% The include files are called |cdocspt3.tex| and |cdocspt4.tex|.
%
%\iffalse
%<*samplepart3|samplepart4>
%\fi

% Optional override for |\version| flag:
%    \begin{macrocode}
%%\providecommand{\version}{final}
%    \end{macrocode}

% Include the main document:
%    \begin{macrocode}
\input{childdoc.def}
\childdocby{cdocsamp}
%    \end{macrocode}

%\iffalse
%</samplepart3|samplepart4>
%\fi
%
%\iffalse
%<*samplepart3>
%\fi
% Some text for part 3:
%    \begin{macrocode}
some text in part three
%    \end{macrocode}

%\iffalse
%</samplepart3>
%\fi
% Some text for part 4:
%\iffalse
%<*samplepart4>
%\fi
%    \begin{macrocode}
more text in part four
%    \end{macrocode}

%\iffalse
%</samplepart4>
%\fi
%
% %%%%%%%%%%%%%%%%%%%%%%%%%%%%%%%%%%%%%%
% \paragraph{Forwarding for a Complete Draft.}
%
% The following forwarding file |cdocsdrf.tex|
% compiles the main document in draft mode:
%\iffalse
%<*sampledraft>
%\fi
%    \begin{macrocode}
\def\version{draft}
\input{childdoc.def}
\childdocforward{cdocsamp}
%    \end{macrocode}

%\iffalse
%</sampledraft>
%\fi
%
% %%%%%%%%%%%%%%%%%%%%%%%%%%%%%%%%%%%%%%
% \paragraph{Forwarding for Final Version of the Chapters.}
%
% The following forwarding files |cdocsfn1.tex| and |cdocsfn2.tex|
% (with identical content)
% compile the final versions of the child documents
% |cdocsch1.tex| and |cdocsch2.tex|, respectively:
%\iffalse
%<*samplefinal>
%\fi
%    \begin{macrocode}
\def\version{final}
\input{childdoc.def}
\childdocforwardprefix[cdocsamp]{cdocsfn}{cdocsch}
%    \end{macrocode}

%\iffalse
%</samplefinal>
%\fi
%
% %%%%%%%%%%%%%%%%%%%%%%%%%%%%%%%%%%%%%%
% \paragraph{Command Line Processing.}
%
% The following three command lines generate the output files
% |cdocscld|, |cdocscl1| and |cdocscl2|
% which should be identical to
% |cdocsdrf|, |cdocsch1| and |cdocsfn2|, respectively:
% \begin{center}
% \begin{tabular}{l}
% |latex -jobname cdocscld \|\\
% |  "\def\version{draft}\input{childdoc.def}\childdocforward{cdocsamp}"|\\
% |latex -jobname cdocscl1 \|\\
% |  "\input{childdoc.def}\childdocforward[cdocsamp]{cdocsch1}"|\\
% |latex -jobname cdocscl2 \|\\
% |  "\def\version{final}\input{childdoc.def}\childdocforward{cdocsch2}"|
% \end{tabular}
% \end{center}
% Note that the trailing backslash on each first line
% merely continues the input to the second line
% (for convenient cut ant paste).
% Furthermore, the command |latex| can be replaced by any
% of its alternative versions such as |pdflatex|.
%
% %%%%%%%%%%%%%%%%%%%%%%%%%%%%%%%%%%%%%%%%%%%%%%%%%%%%%%%%%%%%%%%%%%%%%%%%%%%%%%
% %%%%%%%%%%%%%%%%%%%%%%%%%%%%%%%%%%%%%%%%%%%%%%%%%%%%%%%%%%%%%%%%%%%%%%%%%%%%%%
% \section{Implementation}
%\iffalse
%<*package>
%\fi
%
% This section describes the definitions file |childdoc.def|.

% The definitions cannot be loaded using |\usepackage| or |\RequirePackage|
% which has a mechanism to prevent loading a style file more than once.
% When loading the definitions by means of |\input|
% multiple instances have to be prevented manually:
%\iffalse
%This code needs to be before the `\ProvidesFile' directive
%which is defined at the beginning of this file.
%Therefore it is also placed there and commented out here.
%</package>
%<*discard>
%\fi
%    \begin{macrocode}
\ifdefined\childdocmain\endinput\fi
%    \end{macrocode}
%\iffalse
%</discard>
%<*package>
%\fi
%
% \macro{\ifchilddoc}
% \macro{\ifchilddocmanual}
% The conditional |\ifchilddoc| tells whether a
% child (true) or main (false) document is being compiled.
% The conditional |\ifchilddocmanual| tells whether
% the |\includeonly| mechanism is used (false) or
% the selection of child files must be performed manually (true).
% The definitions initialise to false:
%    \begin{macrocode}
\newif\ifchilddoc
\newif\ifchilddocmanual
%    \end{macrocode}

% \macro{\childdocname}
% \macro{\childdocjob}
% The macro |\childdocname| stores the name of the main document
% to be compiled. The macro |\childdocjob| stores the name of
% the document on which the \LaTeX{} compiler was originally invoked.
% The content of |\jobname| cannot be compared
% to filenames specified in the source due to different catcodes.
% The following code rescans |\jobname|, stores the result
% in |\childdocname| and saves a copy in |\childdocjob|:
%    \begin{macrocode}
\edef\childdocname{\scantokens\expandafter{\jobname\noexpand}}
\let\childdocjob\childdocname
%    \end{macrocode}

% \macro{\childdocdisable}
% The macro |\childdocdisable| prevents the main file
% from being processed more than once.
% At this stage, the main document command |\childdocmain|
% is assumed to be called once again where it should do nothing.
% Any subsequent call to it should prevent
% a secondary processing of the main document
% It overwrites the forwarding commands
% |\childdocof| and |\childdocforward|
% with empty macros to prevent further inclusions of the main document:
%    \begin{macrocode}
\newcommand{\childdocdisable}
{
  \renewcommand{\childdocmain}[1]{\renewcommand{\childdocmain}[1]{\endinput}}
  \renewcommand{\childdocof}[1]{}
  \renewcommand{\childdocby}[2][]{}
  \renewcommand{\childdocforward}[2][]{}
  \renewcommand{\childdocdisable}{}
}
%    \end{macrocode}

% \macro{\childdocmain}
% The macro |\childdocmain| is to be called at the top of the main file
% with nothing or the main filename (without extension) as argument.
% First, it breaks loops.
% If the argument is not empty and does not match |\childdocname|
% (which is set by the first inclusion of |childdoc.def|),
% |\ifchilddoc| is set to true, |\includeonly| is applied to the child file
% and |\jobname| is set to the main file
% (for proper handling of |.aux| files):
%    \begin{macrocode}
\newcommand{\childdocmain}[1]
{
  \childdocdisable\childdocmain{}
  \if?#1?\else
    \begingroup
      \def\childdoctmp{#1}
      \ifx\childdoctmp\childdocname
        \def\childdoctmp{}
      \else
        \def\childdoctmp
        {
          \childdoctrue
          \includeonly{\childdocname}
          \def\childdocjob{#1}
          \def\jobname{#1}
        }
      \fi
      \expandafter
    \endgroup
    \childdoctmp
  \fi
}
%    \end{macrocode}

% \macro{\childdocof}
% The command |\childdocof| redirects
% compilation to the main file |#1|.
%    \begin{macrocode}
\newcommand{\childdocof}[1]
{
  \childdocdisable
  \childdoctrue
  \includeonly{\childdocname}
  \def\jobname{#1}
  \def\childdocjob{#1}
  \input{#1}
}
%    \end{macrocode}

% \macro{\childdocby}
% The command |\childdocby| ....
%    \begin{macrocode}
\newcommand{\childdocby}[2][]
{
  \childdocdisable
  \childdoctrue
  \childdocmanualtrue
  \if?#1?\else
    \def\jobname{#2}
  \fi
  \def\childdocjob{#2}
  \input{#2}
  \endinput
}
%    \end{macrocode}

% \macro{\childdocforward}
% The command |\childdocforward| redirects
% compilation to the main file or
% (if the optional argument is given) a child file.
% Parameters are set as if the main file
% or a child file starting with |\childdocof| was compiled.
% Then compilation is handed over to the main file:
%    \begin{macrocode}
\newcommand{\childdocforward}[2][]
{
  \begingroup
    \if?#1?
      \def\childdoctmp
      {
        \def\childdocname{#2}
        \def\childdocjob{#2}
        \def\jobname{#2}
        \input{#2}
        \endinput
      }
    \else
      \def\childdoctmp
      {
        \childdocdisable
        \def\childdocname{#2}
        \childdoctrue
        \includeonly{#2}
        \def\childdocjob{#1}
        \def\jobname{#1}
        \input{#1}
        \endinput
      }
    \fi
    \expandafter
  \endgroup
  \childdoctmp
}
%    \end{macrocode}

% \macro{\childdocforwardprefix}
% The command |\childdocforwardprefix| redirects
% compilation to the main or a child file by means of a pattern.
% The prefix |#1| in the current filename is replaced by |#2|
% and the suffix of the current filename is kept
% (it is assumed that the filename does not contain the substring `|~~~|'
% which is used as a delimiter).
% Compilation is handed over to the new file by |\childdocforward|:
%    \begin{macrocode}
\newcommand{\childdocforwardprefix}[3][]
{
  \begingroup
    \def\childdocextract #2##1~~~{\def\childdoctmp{\childdocforward[#1]{#3##1}}}
    \expandafter\childdocextract\childdocname~~~
    \expandafter
  \endgroup
  \childdoctmp
}
%    \end{macrocode}

% \macro{\childdoc}
% The deprecated macro |\childdoc| is a legacy version of |\childdocmain|:
%    \begin{macrocode}
\newcommand{\childdoc}{\childdocmain}
%    \end{macrocode}

% \macro{\childdocredirect}
% The deprecated macro |\childdocredirect| is a legacy version
% of |\childdocforward| and |\childdocforwardprefix|:
%    \begin{macrocode}
\newcommand{\childdocredirect}[2][]
{
  \begingroup
    \if?#1?
      \def\childdoctmp{\childdocforward{#2}}
    \else
      \def\childdoctmp{\childdocforwardprefix{#1}{#2}}
    \fi
    \expandafter
  \endgroup
  \childdoctmp
}
%    \end{macrocode}

%\iffalse
%</package>
%\fi
%
\endinput
|\\
|\childdocmain{|\textit{main}|}|\\
\end{tabular}
\end{center}
%
If |\jobname| does not match the argument \textit{main} of |\childdocmain|,
it is assumed that |\jobname| points to the child file to be compiled.
When using |\childdocmain| with the main file specified as argument,
it suffices to start a child file
with just |\input{|\textit{main}|}|
without loading of the package and using |\childdocof|.
If instead all processing is done
with the appropriate \textsf{childdoc} directives,
the argument of \textit{main} of |\childdocmain| can be empty.

An alternative version of the command line processing described
in \secref{sec:commandline} using the detection mechanism reads:
%
\begin{center}
|... -jobname "|\textit{target}|" "|[\textit{flags}]%
[|\def\jobname{|\textit{dest}|}|]|\input{|\textit{main}|}"|
\end{center}

%%%%%%%%%%%%%%%%%%%%%%%%%%%%%%%%%%%%%%%%%%%%%%%%%%%%%%%%%%%%%%%%%%%%%%%%%%%%%%%%
\subsection{Manual Code}
\label{sec:manual}

In case one cannot be certain whether the definitions file |childdoc.def|
is installed on the target \TeX{} distribution
and one prefers not to ship it,
it is conceivable to paste a few relevant commands into the sources.

To that end, drop all statements |% \iffalse
%
% childdoc.dtx Copyright (C) 2017-2018 Niklas Beisert
%
% This work may be distributed and/or modified under the
% conditions of the LaTeX Project Public License, either version 1.3
% of this license or (at your option) any later version.
% The latest version of this license is in
%   http://www.latex-project.org/lppl.txt
% and version 1.3 or later is part of all distributions of LaTeX
% version 2005/12/01 or later.
%
% This work has the LPPL maintenance status `maintained'.
%
% The Current Maintainer of this work is Niklas Beisert.
%
% This work consists of the files childdoc.dtx and childdoc.ins
% and the derived files childdoc.def and cdocsamp.tex with
% cdocsch1.tex, cdocsch2.tex, cdocsdrf.tex, cdocsfn1.tex, cdocsfn2.tex.
%
%<package>\ifdefined\childdocmain\endinput\fi
%<package>\ProvidesFile{childdoc.def}[2018/12/30 v2.0 child document driver]
%<samplemain>\ProvidesFile{cdocsamp.tex}[2018/12/30 v2.0 sample for childdoc]
%<*driver>
%\ProvidesFile{childdoc.drv}[2018/12/30 v2.0 childdoc reference manual file]
\PassOptionsToClass{10pt,a4paper}{article}
\documentclass{ltxdoc}

\usepackage[margin=35mm]{geometry}
\usepackage{hyperref}
\usepackage{hyperxmp}
\usepackage[usenames]{color}

\hypersetup{colorlinks=true}
\hypersetup{pdfstartview=FitH}
\hypersetup{pdfpagemode=UseNone}
\hypersetup{pdfsource={}}
\hypersetup{pdflang={en-UK}}
\hypersetup{pdfcopyright={Copyright 2017-2018 Niklas Beisert.
  This work may be distributed and/or modified under the
  conditions of the LaTeX Project Public License, either version 1.3
  of this license or (at your option) any later version.}}
\hypersetup{pdflicenseurl={http://www.latex-project.org/lppl.txt}}
\hypersetup{pdfcontactaddress={ETH Zurich, ITP, HIT K,
  Wolfgang-Pauli-Strasse 27}}
\hypersetup{pdfcontactpostcode={8093}}
\hypersetup{pdfcontactcity={Zurich}}
\hypersetup{pdfcontactcountry={Switzerland}}
\hypersetup{pdfcontactemail={nbeisert@itp.phys.ethz.ch}}
\hypersetup{pdfcontacturl={http://people.phys.ethz.ch/\xmptilde nbeisert/}}

\newcommand{\secref}[1]{\hyperref[#1]{section \ref*{#1}}}

\parskip1ex
\parindent0pt
\let\olditemize\itemize
\def\itemize{\olditemize\parskip0pt}

\begin{document}

\title{The \textsf{childdoc} Package}
\hypersetup{pdftitle={The childdoc Package}}
\author{Niklas Beisert\\[2ex]
  Institut f\"ur Theoretische Physik\\
  Eidgen\"ossische Technische Hochschule Z\"urich\\
  Wolfgang-Pauli-Strasse 27, 8093 Z\"urich, Switzerland\\[1ex]
  \href{mailto:nbeisert@itp.phys.ethz.ch}
  {\texttt{nbeisert@itp.phys.ethz.ch}}}
\hypersetup{pdfauthor={Niklas Beisert}}
\hypersetup{pdfsubject={Manual for the LaTeX2e Package childdoc}}
\date{30 December 2018, \textsf{v2.0}}
\maketitle

\begin{abstract}\noindent
\textsf{childdoc} is a \LaTeXe{} package
that enables the direct compilation
of document sections included by |\include|
to individual files.
\end{abstract}

\begingroup
\parskip0ex
\tableofcontents
\endgroup

%%%%%%%%%%%%%%%%%%%%%%%%%%%%%%%%%%%%%%%%%%%%%%%%%%%%%%%%%%%%%%%%%%%%%%%%%%%%%%%%
%%%%%%%%%%%%%%%%%%%%%%%%%%%%%%%%%%%%%%%%%%%%%%%%%%%%%%%%%%%%%%%%%%%%%%%%%%%%%%%%
\section{Introduction}

\LaTeX{} provides a mechanism to structure a large document (such as a book)
into a main file and several child files (containing the chapters)
using the |\include| command.
This mechanism is beneficial for documents
which span hundreds of pages in order to
make the source file(s) more manageable.
Moreover, compilation can be restricted to
selected child files by means of the |\includeonly| command.
The latter feature can be used to reduce the compilation time while editing
(this was significantly more useful in the earlier days of \LaTeX{})
or to generate a smaller document which is easier to navigate.
Another application of |\includeonly| is to generate
documents consisting of selected parts of the complete document.

However, there are a few drawbacks of the plain |\include| mechanism:
\begin{itemize}
\item
The child files cannot be compiled on their own,
they can only be compiled via the main file.
A naive editing environment
(such as a text editor with an option
to have the current file processed by \LaTeX)
may require one to switch to the main file before compiling;
attempting to compile the child file produces errors.
\item
The main file must be modified (each time)
to adjust the |\includeonly| command
to the present needs. This easily leaves the main file in a messy state.
\item
The generated document will always carry the filename
of the main document. This is inconvenient if
several child files are to be compiled and
to be kept for distribution.
\end{itemize}

The present package provides a simple interface
to make child files individually compilable by \LaTeX{}.
Compiling a child file then has the same effect as compiling
the main file with an |\includeonly| command
to select the appropriate child.
Moreover the generated document will carry the name of the child
rather than the main file.
This resolves all three above issues.

This feature is meant to make the editing of books,
thesis documents and lecture notes somewhat more convenient.
However, the package can also be used efficiently for
composing a series of documents (such as exercise sheets)
which are typically distributed individually.
It then assists the author in generating the individual documents
(potentially in different versions)
as well as a document containing the collected series.
Another application is in developing style files
or other kinds of included material
where compilation of the style file could redirect
to a sample or test file.

%%%%%%%%%%%%%%%%%%%%%%%%%%%%%%%%%%%%%%%%%%%%%%%%%%%%%%%%%%%%%%%%%%%%%%%%%%%%%%%%
%%%%%%%%%%%%%%%%%%%%%%%%%%%%%%%%%%%%%%%%%%%%%%%%%%%%%%%%%%%%%%%%%%%%%%%%%%%%%%%%
\section{Usage}

First of all, the package \textsf{childdoc} is \emph{not} a standard
\LaTeXe{} |.sty| style file! Therefore it needs to be invoked in
a non-standard way.

%%%%%%%%%%%%%%%%%%%%%%%%%%%%%%%%%%%%%%%%%%%%%%%%%%%%%%%%%%%%%%%%%%%%%%%%%%%%%%%%
\subsection{Included Files}
\label{sec:include}

%%%%%%%%%%%%%%%%%%%%%%%%%%%%%%%%%%%%%%%%
\DescribeMacro{\childdocmain}
To use the package, add the commands
\begin{center}
\begin{tabular}{l}
|\input{childdoc.def}|\\
|\childdocmain{}|\\
\end{tabular}
\end{center}
at the very top of the main \LaTeX{} file,
in particular \emph{before} the |\documentclass| statement!
The argument of |\childdocmain| should be left empty
(but it must be present).

%%%%%%%%%%%%%%%%%%%%%%%%%%%%%%%%%%%%%%%%
\DescribeMacro{\childdocof}
Furthermore, add the commands
\begin{center}
\begin{tabular}{l}
|\input{childdoc.def}|\\
|\childdocof{|\textit{main}|}|\\
\end{tabular}
\end{center}
at the top of every child file \textit{child}
which is included by |\include{|\textit{child}|}|
from within the main file
(or at least for those files to be compiled individually).
The argument \textit{main} must be the filename of the main file.

There are a couple of
considerations in setting up the main and child documents:

%%%%%%%%%%%%%%%%%%%%%%%%%%%%%%%%%%%%%%%%
\paragraph{Restrictions.}

Please note the following restrictions:
\begin{itemize}
\item
|\childdocmain| must be called with one argument \textit{main}
to ensure compatibility with earlier version of the package.
It must either be empty (|\childdocmain{}|)
or precisely match the filename of the main file in which it is specified.
See \secref{sec:detection} for further information.
\item
The filename \textit{main} must be specified without the |.tex| extension.
\item
The filename \textit{main} is case sensitive
(even in case-insensitive file systems)
due to internal string comparison.
\item
The argument \textit{main} should be fully expanded, it cannot be a macro.
\item
Subdirectories and special characters should be avoided in filenames.
\item
The command |\childdocmain{|\textit{main}|}| must be followed by a whitespace.
It should not be followed immediately by another command
or by a comment mark `|%|'.
This is because the \TeX{} parser reads the token immediately following
the argument of |\childdocmain| and puts it
at the beginning of every child section;
however, a white\-space is ignored.
\end{itemize}

%%%%%%%%%%%%%%%%%%%%%%%%%%%%%%%%%%%%%%%%
\paragraph{Content of Main File.}

It is advisable to place all content in the child files included by |\include|.
Any output contained in the main file will appear in all child documents
unless suppressed manually;
it cannot be suppressed automatically by the |\includeonly| directive
and thus should normally be avoided.
A method to include some content in the main file
by means of conditional processing is described in \secref{sec:conditional}.

%%%%%%%%%%%%%%%%%%%%%%%%%%%%%%%%%%%%%%%%
\paragraph{Page Numbering.}

When only a part of the document is compiled,
the appropriate numbering of pages
(as well as other status parameters)
is determined from the |.aux| files.
The latter contain information from previous passes.
However this information needs to propagate through
all intermediate child documents.
Therefore the page numbering in child documents may well
be inconsistent until the complete document is compiled at least once.

A useful (if unconventional) way to always ensure a consistent
page numbering is to restart the numbering in each child document
and denote the pages by `\textit{child}|.|\textit{page}'
where \textit{child} represents the chapter/section number of the child file.
This can be achieved by the command
|\numberwithin{page}{|\textit{child}|}|
of the \textsf{amsmath} package
where \textit{child} can be |chapter| or |section|
depending on the chosen structuring.
Alternatively, one can modify the macro |\thepage| appropriately
and reset the counter |page| at the start of each child file.

%%%%%%%%%%%%%%%%%%%%%%%%%%%%%%%%%%%%%%%%%%%%%%%%%%%%%%%%%%%%%%%%%%%%%%%%%%%%%%%%
\subsection{Conditional Processing}
\label{sec:conditional}

The package provides a mechanism to compile different versions
of a document. To customise the versions further some conditional processing
can come in handy to distinguish which version is being compiled.
The package provides two macros to describe the compilation context:

%%%%%%%%%%%%%%%%%%%%%%%%%%%%%%%%%%%%%%%%
\DescribeMacro{\ifchilddoc}
The conditional |\ifchilddoc| distinguishes between the compilation of
child documents and the main document:
%
\begin{center}
|\ifchilddoc |\textit{child-code}| |[|\||else |\textit{main-code}]| \||fi|
\end{center}

%%%%%%%%%%%%%%%%%%%%%%%%%%%%%%%%%%%%%%%%
\DescribeMacro{\childdocname}
\DescribeMacro{\childdocjob}
The macro |\childdocname| contains the filename (without extension)
of the main or child file being processed.
Note that |\childdocjob| will always contain the name of the main file.

%%%%%%%%%%%%%%%%%%%%%%%%%%%%%%%%%%%%%%%%
\paragraph{Title Page.}

Conditional processing can be used to include a title or banner page
in the main document when proper precautions are taken.
Importantly, the code in the main file should ensure that the page counter
(as well as other status parameters which are stored in the |.aux| files)
takes the same value after the conditional processing.
Otherwise the page numbers may take divergent values
depending on which part is compiled.

For example, a title page could be declared by:
%
\begin{center}
\begin{tabular}{l}
|\ifchilddoc\||else|\\
|\addtocounter{page}{-1}|\\
\textit{code for title page}\\
|\newpage|\\
|\||fi|
\end{tabular}
\end{center}
%
A banner page for the child documents can be generated by:
%
\begin{center}
\begin{tabular}{l}
|\ifchilddoc|\\
|\addtocounter{page}{-1}|\\
\textit{code for banner page}\\
|\newpage|\\
|\||fi|
\end{tabular}
\end{center}
%
Here one could write a message such as:
\begin{center}
|This is the part \childdocname{} of \childdocjob{}.|
\end{center}

%%%%%%%%%%%%%%%%%%%%%%%%%%%%%%%%%%%%%%%%%%%%%%%%%%%%%%%%%%%%%%%%%%%%%%%%%%%%%%%%
\subsection{Flags}
\label{sec:flags}

The package makes it easy to generate different versions
of the main or child documents.
To this end compilation flags can be defined
and assigned different default values.
They will be particularly useful in conjunction
with the forwarding mechanism described in \secref{sec:forward}.

For example, it may be useful to have a flag |\version|
which can be set to |draft| or |final|.
The document source will contain some conditional code
depending on the value of |\version|.
Suppose further, the flag should default to |final| for the main file
and to |draft| for child files
which is a natural assignment for editing the document.
This is achieved by placing the following code
in the preamble of the main document
(below the |\childdocmain| directive):
%
\begin{center}
\begin{tabular}{l}
|\ifchilddoc|\\
|\providecommand{\version}{draft}|\\
|\||else|\\
|\providecommand{\version}{final}|\\
|\||fi|
\end{tabular}
\end{center}
%
The definition by |\providecommand| makes sure
that previous definitions are not overwritten.
Further statements |\providecommand{\version}{...}|
can thus be added before the above code to override it.

For the main file, one might add a line
(between |\childdocmain| and the above block)
%
\begin{center}
|%\ifchilddoc\||else\providecommand{\version}{draft}\||fi|
\end{center}
%
which can be uncommented to produce a draft version.
Likewise one can add a line to the very top of a child file
(above the |\childdocof{|\textit{main}|}| directive)
%
\begin{center}
|%\providecommand{\version}{final}|
\end{center}
%
which can be uncommented to produce the final version of this child document.

%%%%%%%%%%%%%%%%%%%%%%%%%%%%%%%%%%%%%%%%%%%%%%%%%%%%%%%%%%%%%%%%%%%%%%%%%%%%%%%%
\subsection{Forwarding}
\label{sec:forward}

Different versions of the main or child documents
using compilation flags as described in \secref{sec:flags}
can be (permanently) stored in different files
for convenient compilation, viewing and distribution.
To this end, the package defines a command
to pass on compilation to a different file:

%%%%%%%%%%%%%%%%%%%%%%%%%%%%%%%%%%%%%%%%
\DescribeMacro{\childdocforward}
The command |\childdocforward| redirects processing to
another source file:
%
\begin{center}
\begin{tabular}{l}
|\input{childdoc.def}|\\
|\childdocforward[|\textit{main}|]{|\textit{dest}|}|\\
\end{tabular}
\end{center}
%
The argument \textit{dest} is the destination file
(without extension).
It should be the main file or one of the child files.
Note that further \textsf{childdoc} directives
such as |\childdocof| and |\childdocforward|
in the indicated file will be processed in this form.
The optional argument \textit{main}
passes on directly to the main file \textit{main}
while pretending to compile the child \textit{dest}.
This form behaves as if \textit{dest}
issues |\childdocof{|\textit{main}|}| right away,
and no further \textsf{childdoc} directives will be processed.

%%%%%%%%%%%%%%%%%%%%%%%%%%%%%%%%%%%%%%%%
\DescribeMacro{\...prefix}
In the alternative form |\childdocforwardprefix|,
%
\begin{center}
\begin{tabular}{l}
|\input{childdoc.def}|\\
|\childdocforwardprefix[|\textit{main}|]{|\textit{prefix}|}{|\textit{dest}|}|
\end{tabular}
\end{center}
%
the destination file is determined by a pattern
depending on the current file:
To make this work, the current file must be called
`{\textit{prefix}\hspace{0.2em}\textit{suffix}}'
with \textit{prefix} matching precisely the argument.
Processing is then passed on to the file
`{\textit{dest}\hspace{0.2em}\textit{suffix}}'.
Surely, the same effect is achieved by
directly specifying the
argument `{\textit{dest}\hspace{0.2em}\textit{suffix}}'
in the first form.
However, that requires to set up a different file
for each child. With the alternative form of the command
all these files can have exactly the same content
which simplifies setting them up and maintaining them.

For example, the following file |draft.tex|
with a compilation flag |\version| as described in \secref{sec:flags}
compiles the main document as a draft:
%
\begin{center}
\begin{tabular}{l}
|\def\version{draft}|\\
|\input{childdoc.def}|\\
|\childdocforward{|\textit{main}|}|
\end{tabular}
\end{center}
%
Likewise, the following files |final|\textit{nn}|.tex|
compile the final version of the child document
|child|\textit{nn}|.tex|:
%
\begin{center}
\begin{tabular}{l}
|\def\version{final}|\\
|\input{childdoc.def}|\\
|\childdocforwardprefix{final}{child}|
\end{tabular}
\end{center}
%

Note that when several versions of a main file and/or of each child file
are to be generated, it may be convenient to set up a |Makefile| or
shell script to automatise the process.

%%%%%%%%%%%%%%%%%%%%%%%%%%%%%%%%%%%%%%%%%%%%%%%%%%%%%%%%%%%%%%%%%%%%%%%%%%%%%%%%
\subsection{Command Line Processing}
\label{sec:commandline}

The effect of redirection files can also be achieved by invoking
the \LaTeX{} compiler with a more elaborate command line.
Most conveniently this should be done as part
of a shell script or a |Makefile|.

When using \textsf{childdoc} in the main file, the following
command lines effectively perform a redirection
(note that depending on the shell being used,
backslashes may have to be doubled: `|\|' $\to$ `|\\|'):
%
\begin{center}
|... -jobname "|\textit{target}|" |\\|"|[\textit{flags}]%
|\input{childdoc.def}\childdocforward[|\textit{main}|]{|\textit{dest}|}"|
\end{center}
%
Here \textit{target} is the name of the output file,
\textit{main} is the name of the main file
and \textit{dest} is the name of the main or child file to be processed
(all filenames without extensions).
The optional argument \textit{main} can be omitted
if \textit{main} matches \textit{dest}.
Optionally, compilation \textit{flags} can be defined via |\def| commands.
This command line makes the \TeX{} engine believe
it is compiling the file \textit{target}
whose content is specified as the latter parameter.
The provided code then forwards the processing to
\textit{main} or \textit{dest} as described in \secref{sec:forward}.

%%%%%%%%%%%%%%%%%%%%%%%%%%%%%%%%%%%%%%%%%%%%%%%%%%%%%%%%%%%%%%%%%%%%%%%%%%%%%%%%
\subsection{Include by Input}
\label{sec:input}

Including child documents by |\include| has some restrictions by design.
Most notably, the content of a child document always occupies
its own set of pages; pages cannot be shared between child documents.
Usually, this behaviour makes perfect sense
because each child document contain an essential part of the document.
However, in some situations it may be desirable to compose
a document from a collection of parts
without having mandatory page breaks between then.
For this case, the package
provides a mechanism to include parts
by |\input| which can also be processed individually.
However, by construction this mechanism
requires manual handling of the content to be output.

%%%%%%%%%%%%%%%%%%%%%%%%%%%%%%%%%%%%%%%%
\DescribeMacro{\ifchilddocmanual}
The main file should be prepared as usual, see \secref{sec:include}.
However, the document body must make a distinction
between processing of an individual part and of the main document, e.g.:
%
\begin{center}
\begin{tabular}{l}
|\ifchilddocmanual|\\
|\input{\childdocname}|\\
|\||else|\\
\textit{document body with }|\input{|\textit{part}|}|\\
|\||fi|
\end{tabular}
\end{center}
%
The conditional |\ifchilddocmanual| is true whenever
a part to be included by |\input| is being compiled,
and the name of the part is stored in |\childdocname|.

%%%%%%%%%%%%%%%%%%%%%%%%%%%%%%%%%%%%%%%%
\DescribeMacro{\childdocby}
Each part to be included by |\input| should start with:
%
\begin{center}
\begin{tabular}{l}
|\input{childdoc.def}|\\
|\childdocby{|\textit{main}|}|\\
\end{tabular}
\end{center}
%
The directive |\childdocby| is similar to |\childdocof|
described in \secref{sec:include},
but the subsequent selection of content must be done manually.
To that end, both |\ifchilddoc| and |\ifchilddocmanual|
will be true upon processing of a part,
and the name of the part is stored in |\childdocname|.
Note that |\jobname| will be set to the filename of the current part
so that each part receives an individual |.aux| file
that does not interfere with the |.aux| file(s) of the main document.
This behaviour can be altered by the alternative form
|\childdocby[*]{|\textit{main}|}| (with a non-empty optional argument)
which uses the |.aux| file of the main document
by setting |\jobname| to \textit{main}.

%%%%%%%%%%%%%%%%%%%%%%%%%%%%%%%%%%%%%%%%%%%%%%%%%%%%%%%%%%%%%%%%%%%%%%%%%%%%%%%%
\subsection{Driver Development}
\label{sec:driver}

The \textsf{childdoc} mechanism can also be use for the development
of definition files such as \LaTeX{} styles or classes.
This case differs from the above setup with multiple parts
included by |\include| in that no |\includeonly| should be invoked.
This can be achieved by starting the include file
(before |\ProvidesPackage|) with:
%
\begin{center}
\begin{tabular}{l}
|\input{childdoc.def}|\\
|\childdocforward{|\textit{main}|}|\\
\end{tabular}
\end{center}
%
or alternatively with:
%
\begin{center}
\begin{tabular}{l}
|\input{childdoc.def}|\\
|\childdocby{|\textit{main}|}|\\
\end{tabular}
\end{center}
%
Both forms have slightly different effects as described above.
The main file is prepared as usual, see \secref{sec:include}.

%%%%%%%%%%%%%%%%%%%%%%%%%%%%%%%%%%%%%%%%%%%%%%%%%%%%%%%%%%%%%%%%%%%%%%%%%%%%%%%%
\subsection{Legacy Detection}
\label{sec:detection}

The directive |\childdocmain| in the main file can detect
whether the complete document or merely a child is to be compiled
even without using the directive |\childdocof|.
This method is deprecated because it is less robust
and there is no compelling reason to use it;
it is merely provided for backward compatibility
and it may be removed in future versions.

If the detection mechanism is to be used,
it is mandatory to correctly specify
the filename of the main file as the argument of |\childdocmain|:
%
\begin{center}
\begin{tabular}{l}
|\input{childdoc.def}|\\
|\childdocmain{|\textit{main}|}|\\
\end{tabular}
\end{center}
%
If |\jobname| does not match the argument \textit{main} of |\childdocmain|,
it is assumed that |\jobname| points to the child file to be compiled.
When using |\childdocmain| with the main file specified as argument,
it suffices to start a child file
with just |\input{|\textit{main}|}|
without loading of the package and using |\childdocof|.
If instead all processing is done
with the appropriate \textsf{childdoc} directives,
the argument of \textit{main} of |\childdocmain| can be empty.

An alternative version of the command line processing described
in \secref{sec:commandline} using the detection mechanism reads:
%
\begin{center}
|... -jobname "|\textit{target}|" "|[\textit{flags}]%
[|\def\jobname{|\textit{dest}|}|]|\input{|\textit{main}|}"|
\end{center}

%%%%%%%%%%%%%%%%%%%%%%%%%%%%%%%%%%%%%%%%%%%%%%%%%%%%%%%%%%%%%%%%%%%%%%%%%%%%%%%%
\subsection{Manual Code}
\label{sec:manual}

In case one cannot be certain whether the definitions file |childdoc.def|
is installed on the target \TeX{} distribution
and one prefers not to ship it,
it is conceivable to paste a few relevant commands into the sources.

To that end, drop all statements |\input{childdoc.def}|
and perform the replacements as outlined below.
Instead of |\childdocmain{|\textit{main}|}| add the following code
to the top of the main file:
%
\begin{center}
\begin{tabular}{l}
|\||ifdefined\childdocname\endinput\||fi\newif\ifchilddoc|\\
|\edef\childdocname{\scantokens\expandafter{\jobname\noexpand}}|\\
|\def\childdocmain{|\textit{main}|}\||ifx\childdocmain\childdocname\||else|\\
|\childdoctrue\includeonly{\childdocname}\let\jobname\childdocmain\||fi|\\
\end{tabular}
\end{center}
%
Instead of |\childdocof{|\textit{main}|}| just include the main file
at the top of each child file:
%
\begin{center}
|\input{|\textit{main}|}|
\end{center}
%
A simple redirection |\childdocforward{|\textit{dest}|}| is achieved by:
%
\begin{center}
|\def\jobname{|\textit{dest}|}\input{\jobname}|
\end{center}
%
The redirection with prefix
|\childdocforwardprefix[|\textit{prefix}|]{|\textit{dest}|}|
is accomplished by:
%
\begin{center}
\begin{tabular}{l}
|{\edef\jobname{\scantokens\expandafter{\jobname\noexpand}}|\\
|\def\redirectjob |\textit{prefix}|#1~~~{\gdef\jobname{|\textit{dest}|#1}}|\\
|\expandafter\redirectjob\jobname~~~}\input{\jobname}|
\end{tabular}
\end{center}

In an alternative approach,
child documents can be compiled by a specific command line
without additional code or specific definitions:
%
\begin{center}
|... -jobname "|\textit{target}|" "|[\textit{flags}]%
|\includeonly{|\textit{dest}|}\input{|\textit{main}|}"|
\end{center}
%

%%%%%%%%%%%%%%%%%%%%%%%%%%%%%%%%%%%%%%%%%%%%%%%%%%%%%%%%%%%%%%%%%%%%%%%%%%%%%%%%
%%%%%%%%%%%%%%%%%%%%%%%%%%%%%%%%%%%%%%%%%%%%%%%%%%%%%%%%%%%%%%%%%%%%%%%%%%%%%%%%
\section{Information}

%%%%%%%%%%%%%%%%%%%%%%%%%%%%%%%%%%%%%%%%%%%%%%%%%%%%%%%%%%%%%%%%%%%%%%%%%%%%%%%%
\subsection{Copyright}

Copyright \copyright{} 2017--2018 Niklas Beisert

This work may be distributed and/or modified under the
conditions of the \LaTeX{} Project Public License, either version 1.3
of this license or (at your option) any later version.
The latest version of this license is in
  \url{http://www.latex-project.org/lppl.txt}
and version 1.3 or later is part of all distributions of \LaTeX{}
version 2005/12/01 or later.

This work has the LPPL maintenance status `maintained'.

The Current Maintainer of this work is Niklas Beisert.

This work consists of the files |README.txt|, |childdoc.ins| and |childdoc.dtx|
as well as the derived files |childdoc.def|, |cdocsamp.tex|
with |cdocsch1.tex|, |cdocsch2.tex|, |cdocspt3.tex|, |cdocspt4.tex|,
|cdocsdrf.tex|, |cdocsfn1.tex|, |cdocsfn2.tex|
as well as |childdoc.pdf|.

%%%%%%%%%%%%%%%%%%%%%%%%%%%%%%%%%%%%%%%%%%%%%%%%%%%%%%%%%%%%%%%%%%%%%%%%%%%%%%%%
\subsection{Files and Installation}

The package consists of the files:
%
\begin{center}
\begin{tabular}{ll}
    |README.txt|   & readme file \\
    |childdoc.ins| & installation file \\
    |childdoc.dtx| & source file \\
    |childdoc.def| & definition file \\
    |cdocsamp.tex| & sample main file \\
    |cdocsch1.tex| & sample include file \\
    |cdocsch2.tex| & sample include file \\
    |cdocspt3.tex| & sample part file \\
    |cdocspt4.tex| & sample part file \\
    |cdocsdrf.tex| & sample redirection file \\
    |cdocsfn1.tex| & sample redirection file \\
    |cdocsfn2.tex| & sample redirection file \\
    |childdoc.pdf| & manual
\end{tabular}
\end{center}
%
The distribution consists of the files
|README.txt|, |childdoc.ins| and |childdoc.dtx|.
%
\begin{itemize}
\item
Run (pdf)\LaTeX{} on |childdoc.dtx|
to compile the manual |childdoc.pdf| (this file).
\item
Run \LaTeX{} on |childdoc.ins| to create the definitions file |childdoc.def|
and the sample |cdocsamp.tex| with include files
|cdocsch1.tex|, |cdocsch2.tex|, |cdocspt3.tex|, |cdocspt4.tex|,
|cdocsdrf.tex|, |cdocsfn1.tex|, |cdocsfn2.tex|.
Then copy the file |childdoc.def| to an appropriate directory of your \LaTeX{}
distribution, e.g.\ \textit{texmf-root}|/tex/latex/childdoc|.
\end{itemize}

%%%%%%%%%%%%%%%%%%%%%%%%%%%%%%%%%%%%%%%%%%%%%%%%%%%%%%%%%%%%%%%%%%%%%%%%%%%%%%%%
\subsection{Related CTAN Packages}

There are several other packages which offer a similar functionality:
%
\begin{itemize}
\item
The packages
\href{http://ctan.org/pkg/docmute}{\textsf{docmute}},
\href{http://ctan.org/pkg/includex}{\textsf{includex}} and
\href{http://ctan.org/pkg/standalone}{\textsf{standalone}}
provide commands to include only the document body of
a child file thus allowing both files to be compiled individually.
\item
The packages \href{http://ctan.org/pkg/subdocs}{\textsf{subdocs}}
and \href{http://ctan.org/pkg/subfiles}{\textsf{subfiles}}
provide structures in which the main and child documents can be
encapsulated and allowing them to be compiled individually.
The inclusion mechanism is different from the conventional |\include|.
\item
The package \href{http://ctan.org/pkg/combine}{\textsf{combine}}
is an elaborate solution to combine several documents into one.
\end{itemize}
%
See also the CTAN topic \href{http://ctan.org/topic/subdocs}{\textsf{subdocs}}
for further related packages.
The present package differs from the above solutions in that
a document structure constructed with the conventional |\include| mechanism
just needs two extra commands at the top of every file
such that all constituent files can be compiled individually.

%%%%%%%%%%%%%%%%%%%%%%%%%%%%%%%%%%%%%%%%%%%%%%%%%%%%%%%%%%%%%%%%%%%%%%%%%%%%%%%%
%\subsection{Feature Suggestions}
%
%The following is a list of features which may be useful for future
%versions of this package:
%%
%\begin{itemize}
%\item
%\ldots
%\end{itemize}

%%%%%%%%%%%%%%%%%%%%%%%%%%%%%%%%%%%%%%%%%%%%%%%%%%%%%%%%%%%%%%%%%%%%%%%%%%%%%%%%
\subsection{Revision History}

%%%%%%%%%%%%%%%%%%%%%%%%%%%%%%%%%%%%%%%%
\paragraph{v2.0:} 2018/12/30

\begin{itemize}
\item
immediate forward processing
\item
added |\childdocby| mechanism
\item
manual restructured
\end{itemize}

%%%%%%%%%%%%%%%%%%%%%%%%%%%%%%%%%%%%%%%%
\paragraph{v1.6:} 2018/01/17

\begin{itemize}
\item
application for development of include files
\item
corrections to manual
\end{itemize}

%%%%%%%%%%%%%%%%%%%%%%%%%%%%%%%%%%%%%%%%
\paragraph{v1.5:} 2017/05/21

\begin{itemize}
\item
more complete structuring introduced
\item
|\childdocof| introduced
\item
|\childdoc| renamed to |\childdocmain|
\item
|\childredirect| renamed to |\childdocforward| and |\childdocforwardprefix|
and functionality expanded
\end{itemize}

%%%%%%%%%%%%%%%%%%%%%%%%%%%%%%%%%%%%%%%%
\paragraph{v1.0:} 2017/04/27

\begin{itemize}
\item
manual and install package
\item
first version published on CTAN
\end{itemize}

%%%%%%%%%%%%%%%%%%%%%%%%%%%%%%%%%%%%%%%%
\paragraph{v0.6:} 2017/04/26

\begin{itemize}
\item
redirection mechanism added
\end{itemize}

%%%%%%%%%%%%%%%%%%%%%%%%%%%%%%%%%%%%%%%%
\paragraph{v0.5:} 2017/04/26

\begin{itemize}
\item
functionality in definition file
\end{itemize}


%%%%%%%%%%%%%%%%%%%%%%%%%%%%%%%%%%%%%%%%%%%%%%%%%%%%%%%%%%%%%%%%%%%%%%%%%%%%%%%%
%%%%%%%%%%%%%%%%%%%%%%%%%%%%%%%%%%%%%%%%%%%%%%%%%%%%%%%%%%%%%%%%%%%%%%%%%%%%%%%%
%%%%%%%%%%%%%%%%%%%%%%%%%%%%%%%%%%%%%%%%%%%%%%%%%%%%%%%%%%%%%%%%%%%%%%%%%%%%%%%%
\appendix

\settowidth\MacroIndent{\rmfamily\scriptsize 000\ }

 \DocInput{childdoc.dtx}

\end{document}
%</driver>
% \fi
%
% %%%%%%%%%%%%%%%%%%%%%%%%%%%%%%%%%%%%%%%%%%%%%%%%%%%%%%%%%%%%%%%%%%%%%%%%%%%%%%
% %%%%%%%%%%%%%%%%%%%%%%%%%%%%%%%%%%%%%%%%%%%%%%%%%%%%%%%%%%%%%%%%%%%%%%%%%%%%%%
% \section{Sample}
%\iffalse
%<*samplemain>
%\fi
%
% The following presents a sample document
% with two chapters, two parts, a title page,
% a compile flag as well as three forwarding files to set the flag.
% It consists of eight |.tex| files:
% \begin{center}
% \begin{tabular}{ll}
% |cdocsamp.tex|&main file\\
% |cdocsch1.tex|&include file for chapter 1\\
% |cdocsch2.tex|&include file for chapter 2\\
% |cdocspt3.tex|&include file for part 3\\
% |cdocspt4.tex|&include file for part 4\\
% |cdocsdrf.tex|&forwarding file for main file in draft mode\\
% |cdocsfi1.tex|&forwarding file for final version of chapter 1\\
% |cdocsfi2.tex|&forwarding file for final version of chapter 2\\
% \end{tabular}
% \end{center}
% Each of the eight files can be compiled directly by the \LaTeX{} compiler.
%
% %%%%%%%%%%%%%%%%%%%%%%%%%%%%%%%%%%%%%%
% \paragraph{Main File.}
%
% The main file is called |cdocsamp.tex|.
%
% Load the \textsf{childdoc} definitions and
% declare the filename for the main document:
%    \begin{macrocode}
\input{childdoc.def}
\childdocmain{}
%    \end{macrocode}

% Optional override for |\version| flag:
%    \begin{macrocode}
%%\ifchilddoc\else\providecommand{\version}{draft}\fi
%    \end{macrocode}

% Define the default values for the |\version| flag
% (|final| for the main file and |draft| for childs):
%    \begin{macrocode}
\ifchilddoc
\providecommand{\version}{draft}
\else
\providecommand{\version}{final}
\fi
%    \end{macrocode}

% Load the standard document class:
%    \begin{macrocode}
\documentclass[12pt]{article}
%    \end{macrocode}

% Start the document body:
%    \begin{macrocode}
\begin{document}
%    \end{macrocode}

% Declare a title page.
% Print title, part of document being processed and version flag:
%    \begin{macrocode}
\addtocounter{page}{-1}
\begin{center}
{\LARGE\bfseries{}childdoc example\par}
\vspace{1cm}
\ifchilddoc
\ifchilddocmanual part\else chapter\fi:
`\childdocname' of `\childdocjob'\par
\else
main document: `\childdocjob'\par
\fi
version: \version\par
\end{center}
\newpage
%    \end{macrocode}

% Manually include selected file,
% otherwise process as usual:
%    \begin{macrocode}
\ifchilddocmanual
\section*{part `\childdocname'}
\input{\childdocname}
\else
%    \end{macrocode}

% Include the two chapters:
%    \begin{macrocode}
\include{cdocsch1}
\include{cdocsch2}
%    \end{macrocode}

% Include the two parts unless only chapters should be displayed:
%    \begin{macrocode}
\ifchilddoc\else
\section{part three}
\input{cdocspt3}
\section{part four}
\input{cdocspt4}
\fi
%    \end{macrocode}

% Process as usual until here:
%    \begin{macrocode}
\fi
%    \end{macrocode}

% End of document body:
%    \begin{macrocode}
\end{document}
%    \end{macrocode}
%\iffalse
%</samplemain>
%\fi
%
% %%%%%%%%%%%%%%%%%%%%%%%%%%%%%%%%%%%%%%
% \paragraph{Chapter Include Files.}
%
% The include files are called |cdocsch1.tex| and |cdocsch2.tex|.
%
%\iffalse
%<*samplechap1|samplechap2>
%\fi

% Optional override for |\version| flag:
%    \begin{macrocode}
%%\providecommand{\version}{final}
%    \end{macrocode}

% Include the main document:
%    \begin{macrocode}
\input{childdoc.def}
\childdocof{cdocsamp}
%    \end{macrocode}

%\iffalse
%</samplechap1|samplechap2>
%\fi
%
%\iffalse
%<*samplechap1>
%\fi
% Some text for chapter 1:
%    \begin{macrocode}
\section{one}
some text in chapter one
%    \end{macrocode}

%\iffalse
%</samplechap1>
%\fi
% Some text for chapter 2:
%\iffalse
%<*samplechap2>
%\fi
%    \begin{macrocode}
\section{two}
more text in chapter two
%    \end{macrocode}

%\iffalse
%</samplechap2>
%\fi
%
% %%%%%%%%%%%%%%%%%%%%%%%%%%%%%%%%%%%%%%
% \paragraph{Part Include Files.}
%
% The include files are called |cdocspt3.tex| and |cdocspt4.tex|.
%
%\iffalse
%<*samplepart3|samplepart4>
%\fi

% Optional override for |\version| flag:
%    \begin{macrocode}
%%\providecommand{\version}{final}
%    \end{macrocode}

% Include the main document:
%    \begin{macrocode}
\input{childdoc.def}
\childdocby{cdocsamp}
%    \end{macrocode}

%\iffalse
%</samplepart3|samplepart4>
%\fi
%
%\iffalse
%<*samplepart3>
%\fi
% Some text for part 3:
%    \begin{macrocode}
some text in part three
%    \end{macrocode}

%\iffalse
%</samplepart3>
%\fi
% Some text for part 4:
%\iffalse
%<*samplepart4>
%\fi
%    \begin{macrocode}
more text in part four
%    \end{macrocode}

%\iffalse
%</samplepart4>
%\fi
%
% %%%%%%%%%%%%%%%%%%%%%%%%%%%%%%%%%%%%%%
% \paragraph{Forwarding for a Complete Draft.}
%
% The following forwarding file |cdocsdrf.tex|
% compiles the main document in draft mode:
%\iffalse
%<*sampledraft>
%\fi
%    \begin{macrocode}
\def\version{draft}
\input{childdoc.def}
\childdocforward{cdocsamp}
%    \end{macrocode}

%\iffalse
%</sampledraft>
%\fi
%
% %%%%%%%%%%%%%%%%%%%%%%%%%%%%%%%%%%%%%%
% \paragraph{Forwarding for Final Version of the Chapters.}
%
% The following forwarding files |cdocsfn1.tex| and |cdocsfn2.tex|
% (with identical content)
% compile the final versions of the child documents
% |cdocsch1.tex| and |cdocsch2.tex|, respectively:
%\iffalse
%<*samplefinal>
%\fi
%    \begin{macrocode}
\def\version{final}
\input{childdoc.def}
\childdocforwardprefix[cdocsamp]{cdocsfn}{cdocsch}
%    \end{macrocode}

%\iffalse
%</samplefinal>
%\fi
%
% %%%%%%%%%%%%%%%%%%%%%%%%%%%%%%%%%%%%%%
% \paragraph{Command Line Processing.}
%
% The following three command lines generate the output files
% |cdocscld|, |cdocscl1| and |cdocscl2|
% which should be identical to
% |cdocsdrf|, |cdocsch1| and |cdocsfn2|, respectively:
% \begin{center}
% \begin{tabular}{l}
% |latex -jobname cdocscld \|\\
% |  "\def\version{draft}\input{childdoc.def}\childdocforward{cdocsamp}"|\\
% |latex -jobname cdocscl1 \|\\
% |  "\input{childdoc.def}\childdocforward[cdocsamp]{cdocsch1}"|\\
% |latex -jobname cdocscl2 \|\\
% |  "\def\version{final}\input{childdoc.def}\childdocforward{cdocsch2}"|
% \end{tabular}
% \end{center}
% Note that the trailing backslash on each first line
% merely continues the input to the second line
% (for convenient cut ant paste).
% Furthermore, the command |latex| can be replaced by any
% of its alternative versions such as |pdflatex|.
%
% %%%%%%%%%%%%%%%%%%%%%%%%%%%%%%%%%%%%%%%%%%%%%%%%%%%%%%%%%%%%%%%%%%%%%%%%%%%%%%
% %%%%%%%%%%%%%%%%%%%%%%%%%%%%%%%%%%%%%%%%%%%%%%%%%%%%%%%%%%%%%%%%%%%%%%%%%%%%%%
% \section{Implementation}
%\iffalse
%<*package>
%\fi
%
% This section describes the definitions file |childdoc.def|.

% The definitions cannot be loaded using |\usepackage| or |\RequirePackage|
% which has a mechanism to prevent loading a style file more than once.
% When loading the definitions by means of |\input|
% multiple instances have to be prevented manually:
%\iffalse
%This code needs to be before the `\ProvidesFile' directive
%which is defined at the beginning of this file.
%Therefore it is also placed there and commented out here.
%</package>
%<*discard>
%\fi
%    \begin{macrocode}
\ifdefined\childdocmain\endinput\fi
%    \end{macrocode}
%\iffalse
%</discard>
%<*package>
%\fi
%
% \macro{\ifchilddoc}
% \macro{\ifchilddocmanual}
% The conditional |\ifchilddoc| tells whether a
% child (true) or main (false) document is being compiled.
% The conditional |\ifchilddocmanual| tells whether
% the |\includeonly| mechanism is used (false) or
% the selection of child files must be performed manually (true).
% The definitions initialise to false:
%    \begin{macrocode}
\newif\ifchilddoc
\newif\ifchilddocmanual
%    \end{macrocode}

% \macro{\childdocname}
% \macro{\childdocjob}
% The macro |\childdocname| stores the name of the main document
% to be compiled. The macro |\childdocjob| stores the name of
% the document on which the \LaTeX{} compiler was originally invoked.
% The content of |\jobname| cannot be compared
% to filenames specified in the source due to different catcodes.
% The following code rescans |\jobname|, stores the result
% in |\childdocname| and saves a copy in |\childdocjob|:
%    \begin{macrocode}
\edef\childdocname{\scantokens\expandafter{\jobname\noexpand}}
\let\childdocjob\childdocname
%    \end{macrocode}

% \macro{\childdocdisable}
% The macro |\childdocdisable| prevents the main file
% from being processed more than once.
% At this stage, the main document command |\childdocmain|
% is assumed to be called once again where it should do nothing.
% Any subsequent call to it should prevent
% a secondary processing of the main document
% It overwrites the forwarding commands
% |\childdocof| and |\childdocforward|
% with empty macros to prevent further inclusions of the main document:
%    \begin{macrocode}
\newcommand{\childdocdisable}
{
  \renewcommand{\childdocmain}[1]{\renewcommand{\childdocmain}[1]{\endinput}}
  \renewcommand{\childdocof}[1]{}
  \renewcommand{\childdocby}[2][]{}
  \renewcommand{\childdocforward}[2][]{}
  \renewcommand{\childdocdisable}{}
}
%    \end{macrocode}

% \macro{\childdocmain}
% The macro |\childdocmain| is to be called at the top of the main file
% with nothing or the main filename (without extension) as argument.
% First, it breaks loops.
% If the argument is not empty and does not match |\childdocname|
% (which is set by the first inclusion of |childdoc.def|),
% |\ifchilddoc| is set to true, |\includeonly| is applied to the child file
% and |\jobname| is set to the main file
% (for proper handling of |.aux| files):
%    \begin{macrocode}
\newcommand{\childdocmain}[1]
{
  \childdocdisable\childdocmain{}
  \if?#1?\else
    \begingroup
      \def\childdoctmp{#1}
      \ifx\childdoctmp\childdocname
        \def\childdoctmp{}
      \else
        \def\childdoctmp
        {
          \childdoctrue
          \includeonly{\childdocname}
          \def\childdocjob{#1}
          \def\jobname{#1}
        }
      \fi
      \expandafter
    \endgroup
    \childdoctmp
  \fi
}
%    \end{macrocode}

% \macro{\childdocof}
% The command |\childdocof| redirects
% compilation to the main file |#1|.
%    \begin{macrocode}
\newcommand{\childdocof}[1]
{
  \childdocdisable
  \childdoctrue
  \includeonly{\childdocname}
  \def\jobname{#1}
  \def\childdocjob{#1}
  \input{#1}
}
%    \end{macrocode}

% \macro{\childdocby}
% The command |\childdocby| ....
%    \begin{macrocode}
\newcommand{\childdocby}[2][]
{
  \childdocdisable
  \childdoctrue
  \childdocmanualtrue
  \if?#1?\else
    \def\jobname{#2}
  \fi
  \def\childdocjob{#2}
  \input{#2}
  \endinput
}
%    \end{macrocode}

% \macro{\childdocforward}
% The command |\childdocforward| redirects
% compilation to the main file or
% (if the optional argument is given) a child file.
% Parameters are set as if the main file
% or a child file starting with |\childdocof| was compiled.
% Then compilation is handed over to the main file:
%    \begin{macrocode}
\newcommand{\childdocforward}[2][]
{
  \begingroup
    \if?#1?
      \def\childdoctmp
      {
        \def\childdocname{#2}
        \def\childdocjob{#2}
        \def\jobname{#2}
        \input{#2}
        \endinput
      }
    \else
      \def\childdoctmp
      {
        \childdocdisable
        \def\childdocname{#2}
        \childdoctrue
        \includeonly{#2}
        \def\childdocjob{#1}
        \def\jobname{#1}
        \input{#1}
        \endinput
      }
    \fi
    \expandafter
  \endgroup
  \childdoctmp
}
%    \end{macrocode}

% \macro{\childdocforwardprefix}
% The command |\childdocforwardprefix| redirects
% compilation to the main or a child file by means of a pattern.
% The prefix |#1| in the current filename is replaced by |#2|
% and the suffix of the current filename is kept
% (it is assumed that the filename does not contain the substring `|~~~|'
% which is used as a delimiter).
% Compilation is handed over to the new file by |\childdocforward|:
%    \begin{macrocode}
\newcommand{\childdocforwardprefix}[3][]
{
  \begingroup
    \def\childdocextract #2##1~~~{\def\childdoctmp{\childdocforward[#1]{#3##1}}}
    \expandafter\childdocextract\childdocname~~~
    \expandafter
  \endgroup
  \childdoctmp
}
%    \end{macrocode}

% \macro{\childdoc}
% The deprecated macro |\childdoc| is a legacy version of |\childdocmain|:
%    \begin{macrocode}
\newcommand{\childdoc}{\childdocmain}
%    \end{macrocode}

% \macro{\childdocredirect}
% The deprecated macro |\childdocredirect| is a legacy version
% of |\childdocforward| and |\childdocforwardprefix|:
%    \begin{macrocode}
\newcommand{\childdocredirect}[2][]
{
  \begingroup
    \if?#1?
      \def\childdoctmp{\childdocforward{#2}}
    \else
      \def\childdoctmp{\childdocforwardprefix{#1}{#2}}
    \fi
    \expandafter
  \endgroup
  \childdoctmp
}
%    \end{macrocode}

%\iffalse
%</package>
%\fi
%
\endinput
|
and perform the replacements as outlined below.
Instead of |\childdocmain{|\textit{main}|}| add the following code
to the top of the main file:
%
\begin{center}
\begin{tabular}{l}
|\||ifdefined\childdocname\endinput\||fi\newif\ifchilddoc|\\
|\edef\childdocname{\scantokens\expandafter{\jobname\noexpand}}|\\
|\def\childdocmain{|\textit{main}|}\||ifx\childdocmain\childdocname\||else|\\
|\childdoctrue\includeonly{\childdocname}\let\jobname\childdocmain\||fi|\\
\end{tabular}
\end{center}
%
Instead of |\childdocof{|\textit{main}|}| just include the main file
at the top of each child file:
%
\begin{center}
|\input{|\textit{main}|}|
\end{center}
%
A simple redirection |\childdocforward{|\textit{dest}|}| is achieved by:
%
\begin{center}
|\def\jobname{|\textit{dest}|}\input{\jobname}|
\end{center}
%
The redirection with prefix
|\childdocforwardprefix[|\textit{prefix}|]{|\textit{dest}|}|
is accomplished by:
%
\begin{center}
\begin{tabular}{l}
|{\edef\jobname{\scantokens\expandafter{\jobname\noexpand}}|\\
|\def\redirectjob |\textit{prefix}|#1~~~{\gdef\jobname{|\textit{dest}|#1}}|\\
|\expandafter\redirectjob\jobname~~~}\input{\jobname}|
\end{tabular}
\end{center}

In an alternative approach,
child documents can be compiled by a specific command line
without additional code or specific definitions:
%
\begin{center}
|... -jobname "|\textit{target}|" "|[\textit{flags}]%
|\includeonly{|\textit{dest}|}\input{|\textit{main}|}"|
\end{center}
%

%%%%%%%%%%%%%%%%%%%%%%%%%%%%%%%%%%%%%%%%%%%%%%%%%%%%%%%%%%%%%%%%%%%%%%%%%%%%%%%%
%%%%%%%%%%%%%%%%%%%%%%%%%%%%%%%%%%%%%%%%%%%%%%%%%%%%%%%%%%%%%%%%%%%%%%%%%%%%%%%%
\section{Information}

%%%%%%%%%%%%%%%%%%%%%%%%%%%%%%%%%%%%%%%%%%%%%%%%%%%%%%%%%%%%%%%%%%%%%%%%%%%%%%%%
\subsection{Copyright}

Copyright \copyright{} 2017--2018 Niklas Beisert

This work may be distributed and/or modified under the
conditions of the \LaTeX{} Project Public License, either version 1.3
of this license or (at your option) any later version.
The latest version of this license is in
  \url{http://www.latex-project.org/lppl.txt}
and version 1.3 or later is part of all distributions of \LaTeX{}
version 2005/12/01 or later.

This work has the LPPL maintenance status `maintained'.

The Current Maintainer of this work is Niklas Beisert.

This work consists of the files |README.txt|, |childdoc.ins| and |childdoc.dtx|
as well as the derived files |childdoc.def|, |cdocsamp.tex|
with |cdocsch1.tex|, |cdocsch2.tex|, |cdocspt3.tex|, |cdocspt4.tex|,
|cdocsdrf.tex|, |cdocsfn1.tex|, |cdocsfn2.tex|
as well as |childdoc.pdf|.

%%%%%%%%%%%%%%%%%%%%%%%%%%%%%%%%%%%%%%%%%%%%%%%%%%%%%%%%%%%%%%%%%%%%%%%%%%%%%%%%
\subsection{Files and Installation}

The package consists of the files:
%
\begin{center}
\begin{tabular}{ll}
    |README.txt|   & readme file \\
    |childdoc.ins| & installation file \\
    |childdoc.dtx| & source file \\
    |childdoc.def| & definition file \\
    |cdocsamp.tex| & sample main file \\
    |cdocsch1.tex| & sample include file \\
    |cdocsch2.tex| & sample include file \\
    |cdocspt3.tex| & sample part file \\
    |cdocspt4.tex| & sample part file \\
    |cdocsdrf.tex| & sample redirection file \\
    |cdocsfn1.tex| & sample redirection file \\
    |cdocsfn2.tex| & sample redirection file \\
    |childdoc.pdf| & manual
\end{tabular}
\end{center}
%
The distribution consists of the files
|README.txt|, |childdoc.ins| and |childdoc.dtx|.
%
\begin{itemize}
\item
Run (pdf)\LaTeX{} on |childdoc.dtx|
to compile the manual |childdoc.pdf| (this file).
\item
Run \LaTeX{} on |childdoc.ins| to create the definitions file |childdoc.def|
and the sample |cdocsamp.tex| with include files
|cdocsch1.tex|, |cdocsch2.tex|, |cdocspt3.tex|, |cdocspt4.tex|,
|cdocsdrf.tex|, |cdocsfn1.tex|, |cdocsfn2.tex|.
Then copy the file |childdoc.def| to an appropriate directory of your \LaTeX{}
distribution, e.g.\ \textit{texmf-root}|/tex/latex/childdoc|.
\end{itemize}

%%%%%%%%%%%%%%%%%%%%%%%%%%%%%%%%%%%%%%%%%%%%%%%%%%%%%%%%%%%%%%%%%%%%%%%%%%%%%%%%
\subsection{Related CTAN Packages}

There are several other packages which offer a similar functionality:
%
\begin{itemize}
\item
The packages
\href{http://ctan.org/pkg/docmute}{\textsf{docmute}},
\href{http://ctan.org/pkg/includex}{\textsf{includex}} and
\href{http://ctan.org/pkg/standalone}{\textsf{standalone}}
provide commands to include only the document body of
a child file thus allowing both files to be compiled individually.
\item
The packages \href{http://ctan.org/pkg/subdocs}{\textsf{subdocs}}
and \href{http://ctan.org/pkg/subfiles}{\textsf{subfiles}}
provide structures in which the main and child documents can be
encapsulated and allowing them to be compiled individually.
The inclusion mechanism is different from the conventional |\include|.
\item
The package \href{http://ctan.org/pkg/combine}{\textsf{combine}}
is an elaborate solution to combine several documents into one.
\end{itemize}
%
See also the CTAN topic \href{http://ctan.org/topic/subdocs}{\textsf{subdocs}}
for further related packages.
The present package differs from the above solutions in that
a document structure constructed with the conventional |\include| mechanism
just needs two extra commands at the top of every file
such that all constituent files can be compiled individually.

%%%%%%%%%%%%%%%%%%%%%%%%%%%%%%%%%%%%%%%%%%%%%%%%%%%%%%%%%%%%%%%%%%%%%%%%%%%%%%%%
%\subsection{Feature Suggestions}
%
%The following is a list of features which may be useful for future
%versions of this package:
%%
%\begin{itemize}
%\item
%\ldots
%\end{itemize}

%%%%%%%%%%%%%%%%%%%%%%%%%%%%%%%%%%%%%%%%%%%%%%%%%%%%%%%%%%%%%%%%%%%%%%%%%%%%%%%%
\subsection{Revision History}

%%%%%%%%%%%%%%%%%%%%%%%%%%%%%%%%%%%%%%%%
\paragraph{v2.0:} 2018/12/30

\begin{itemize}
\item
immediate forward processing
\item
added |\childdocby| mechanism
\item
manual restructured
\end{itemize}

%%%%%%%%%%%%%%%%%%%%%%%%%%%%%%%%%%%%%%%%
\paragraph{v1.6:} 2018/01/17

\begin{itemize}
\item
application for development of include files
\item
corrections to manual
\end{itemize}

%%%%%%%%%%%%%%%%%%%%%%%%%%%%%%%%%%%%%%%%
\paragraph{v1.5:} 2017/05/21

\begin{itemize}
\item
more complete structuring introduced
\item
|\childdocof| introduced
\item
|\childdoc| renamed to |\childdocmain|
\item
|\childredirect| renamed to |\childdocforward| and |\childdocforwardprefix|
and functionality expanded
\end{itemize}

%%%%%%%%%%%%%%%%%%%%%%%%%%%%%%%%%%%%%%%%
\paragraph{v1.0:} 2017/04/27

\begin{itemize}
\item
manual and install package
\item
first version published on CTAN
\end{itemize}

%%%%%%%%%%%%%%%%%%%%%%%%%%%%%%%%%%%%%%%%
\paragraph{v0.6:} 2017/04/26

\begin{itemize}
\item
redirection mechanism added
\end{itemize}

%%%%%%%%%%%%%%%%%%%%%%%%%%%%%%%%%%%%%%%%
\paragraph{v0.5:} 2017/04/26

\begin{itemize}
\item
functionality in definition file
\end{itemize}


%%%%%%%%%%%%%%%%%%%%%%%%%%%%%%%%%%%%%%%%%%%%%%%%%%%%%%%%%%%%%%%%%%%%%%%%%%%%%%%%
%%%%%%%%%%%%%%%%%%%%%%%%%%%%%%%%%%%%%%%%%%%%%%%%%%%%%%%%%%%%%%%%%%%%%%%%%%%%%%%%
%%%%%%%%%%%%%%%%%%%%%%%%%%%%%%%%%%%%%%%%%%%%%%%%%%%%%%%%%%%%%%%%%%%%%%%%%%%%%%%%
\appendix

\settowidth\MacroIndent{\rmfamily\scriptsize 000\ }

 \DocInput{childdoc.dtx}

\end{document}
%</driver>
% \fi
%
% %%%%%%%%%%%%%%%%%%%%%%%%%%%%%%%%%%%%%%%%%%%%%%%%%%%%%%%%%%%%%%%%%%%%%%%%%%%%%%
% %%%%%%%%%%%%%%%%%%%%%%%%%%%%%%%%%%%%%%%%%%%%%%%%%%%%%%%%%%%%%%%%%%%%%%%%%%%%%%
% \section{Sample}
%\iffalse
%<*samplemain>
%\fi
%
% The following presents a sample document
% with two chapters, two parts, a title page,
% a compile flag as well as three forwarding files to set the flag.
% It consists of eight |.tex| files:
% \begin{center}
% \begin{tabular}{ll}
% |cdocsamp.tex|&main file\\
% |cdocsch1.tex|&include file for chapter 1\\
% |cdocsch2.tex|&include file for chapter 2\\
% |cdocspt3.tex|&include file for part 3\\
% |cdocspt4.tex|&include file for part 4\\
% |cdocsdrf.tex|&forwarding file for main file in draft mode\\
% |cdocsfi1.tex|&forwarding file for final version of chapter 1\\
% |cdocsfi2.tex|&forwarding file for final version of chapter 2\\
% \end{tabular}
% \end{center}
% Each of the eight files can be compiled directly by the \LaTeX{} compiler.
%
% %%%%%%%%%%%%%%%%%%%%%%%%%%%%%%%%%%%%%%
% \paragraph{Main File.}
%
% The main file is called |cdocsamp.tex|.
%
% Load the \textsf{childdoc} definitions and
% declare the filename for the main document:
%    \begin{macrocode}
% \iffalse
%
% childdoc.dtx Copyright (C) 2017-2018 Niklas Beisert
%
% This work may be distributed and/or modified under the
% conditions of the LaTeX Project Public License, either version 1.3
% of this license or (at your option) any later version.
% The latest version of this license is in
%   http://www.latex-project.org/lppl.txt
% and version 1.3 or later is part of all distributions of LaTeX
% version 2005/12/01 or later.
%
% This work has the LPPL maintenance status `maintained'.
%
% The Current Maintainer of this work is Niklas Beisert.
%
% This work consists of the files childdoc.dtx and childdoc.ins
% and the derived files childdoc.def and cdocsamp.tex with
% cdocsch1.tex, cdocsch2.tex, cdocsdrf.tex, cdocsfn1.tex, cdocsfn2.tex.
%
%<package>\ifdefined\childdocmain\endinput\fi
%<package>\ProvidesFile{childdoc.def}[2018/12/30 v2.0 child document driver]
%<samplemain>\ProvidesFile{cdocsamp.tex}[2018/12/30 v2.0 sample for childdoc]
%<*driver>
%\ProvidesFile{childdoc.drv}[2018/12/30 v2.0 childdoc reference manual file]
\PassOptionsToClass{10pt,a4paper}{article}
\documentclass{ltxdoc}

\usepackage[margin=35mm]{geometry}
\usepackage{hyperref}
\usepackage{hyperxmp}
\usepackage[usenames]{color}

\hypersetup{colorlinks=true}
\hypersetup{pdfstartview=FitH}
\hypersetup{pdfpagemode=UseNone}
\hypersetup{pdfsource={}}
\hypersetup{pdflang={en-UK}}
\hypersetup{pdfcopyright={Copyright 2017-2018 Niklas Beisert.
  This work may be distributed and/or modified under the
  conditions of the LaTeX Project Public License, either version 1.3
  of this license or (at your option) any later version.}}
\hypersetup{pdflicenseurl={http://www.latex-project.org/lppl.txt}}
\hypersetup{pdfcontactaddress={ETH Zurich, ITP, HIT K,
  Wolfgang-Pauli-Strasse 27}}
\hypersetup{pdfcontactpostcode={8093}}
\hypersetup{pdfcontactcity={Zurich}}
\hypersetup{pdfcontactcountry={Switzerland}}
\hypersetup{pdfcontactemail={nbeisert@itp.phys.ethz.ch}}
\hypersetup{pdfcontacturl={http://people.phys.ethz.ch/\xmptilde nbeisert/}}

\newcommand{\secref}[1]{\hyperref[#1]{section \ref*{#1}}}

\parskip1ex
\parindent0pt
\let\olditemize\itemize
\def\itemize{\olditemize\parskip0pt}

\begin{document}

\title{The \textsf{childdoc} Package}
\hypersetup{pdftitle={The childdoc Package}}
\author{Niklas Beisert\\[2ex]
  Institut f\"ur Theoretische Physik\\
  Eidgen\"ossische Technische Hochschule Z\"urich\\
  Wolfgang-Pauli-Strasse 27, 8093 Z\"urich, Switzerland\\[1ex]
  \href{mailto:nbeisert@itp.phys.ethz.ch}
  {\texttt{nbeisert@itp.phys.ethz.ch}}}
\hypersetup{pdfauthor={Niklas Beisert}}
\hypersetup{pdfsubject={Manual for the LaTeX2e Package childdoc}}
\date{30 December 2018, \textsf{v2.0}}
\maketitle

\begin{abstract}\noindent
\textsf{childdoc} is a \LaTeXe{} package
that enables the direct compilation
of document sections included by |\include|
to individual files.
\end{abstract}

\begingroup
\parskip0ex
\tableofcontents
\endgroup

%%%%%%%%%%%%%%%%%%%%%%%%%%%%%%%%%%%%%%%%%%%%%%%%%%%%%%%%%%%%%%%%%%%%%%%%%%%%%%%%
%%%%%%%%%%%%%%%%%%%%%%%%%%%%%%%%%%%%%%%%%%%%%%%%%%%%%%%%%%%%%%%%%%%%%%%%%%%%%%%%
\section{Introduction}

\LaTeX{} provides a mechanism to structure a large document (such as a book)
into a main file and several child files (containing the chapters)
using the |\include| command.
This mechanism is beneficial for documents
which span hundreds of pages in order to
make the source file(s) more manageable.
Moreover, compilation can be restricted to
selected child files by means of the |\includeonly| command.
The latter feature can be used to reduce the compilation time while editing
(this was significantly more useful in the earlier days of \LaTeX{})
or to generate a smaller document which is easier to navigate.
Another application of |\includeonly| is to generate
documents consisting of selected parts of the complete document.

However, there are a few drawbacks of the plain |\include| mechanism:
\begin{itemize}
\item
The child files cannot be compiled on their own,
they can only be compiled via the main file.
A naive editing environment
(such as a text editor with an option
to have the current file processed by \LaTeX)
may require one to switch to the main file before compiling;
attempting to compile the child file produces errors.
\item
The main file must be modified (each time)
to adjust the |\includeonly| command
to the present needs. This easily leaves the main file in a messy state.
\item
The generated document will always carry the filename
of the main document. This is inconvenient if
several child files are to be compiled and
to be kept for distribution.
\end{itemize}

The present package provides a simple interface
to make child files individually compilable by \LaTeX{}.
Compiling a child file then has the same effect as compiling
the main file with an |\includeonly| command
to select the appropriate child.
Moreover the generated document will carry the name of the child
rather than the main file.
This resolves all three above issues.

This feature is meant to make the editing of books,
thesis documents and lecture notes somewhat more convenient.
However, the package can also be used efficiently for
composing a series of documents (such as exercise sheets)
which are typically distributed individually.
It then assists the author in generating the individual documents
(potentially in different versions)
as well as a document containing the collected series.
Another application is in developing style files
or other kinds of included material
where compilation of the style file could redirect
to a sample or test file.

%%%%%%%%%%%%%%%%%%%%%%%%%%%%%%%%%%%%%%%%%%%%%%%%%%%%%%%%%%%%%%%%%%%%%%%%%%%%%%%%
%%%%%%%%%%%%%%%%%%%%%%%%%%%%%%%%%%%%%%%%%%%%%%%%%%%%%%%%%%%%%%%%%%%%%%%%%%%%%%%%
\section{Usage}

First of all, the package \textsf{childdoc} is \emph{not} a standard
\LaTeXe{} |.sty| style file! Therefore it needs to be invoked in
a non-standard way.

%%%%%%%%%%%%%%%%%%%%%%%%%%%%%%%%%%%%%%%%%%%%%%%%%%%%%%%%%%%%%%%%%%%%%%%%%%%%%%%%
\subsection{Included Files}
\label{sec:include}

%%%%%%%%%%%%%%%%%%%%%%%%%%%%%%%%%%%%%%%%
\DescribeMacro{\childdocmain}
To use the package, add the commands
\begin{center}
\begin{tabular}{l}
|\input{childdoc.def}|\\
|\childdocmain{}|\\
\end{tabular}
\end{center}
at the very top of the main \LaTeX{} file,
in particular \emph{before} the |\documentclass| statement!
The argument of |\childdocmain| should be left empty
(but it must be present).

%%%%%%%%%%%%%%%%%%%%%%%%%%%%%%%%%%%%%%%%
\DescribeMacro{\childdocof}
Furthermore, add the commands
\begin{center}
\begin{tabular}{l}
|\input{childdoc.def}|\\
|\childdocof{|\textit{main}|}|\\
\end{tabular}
\end{center}
at the top of every child file \textit{child}
which is included by |\include{|\textit{child}|}|
from within the main file
(or at least for those files to be compiled individually).
The argument \textit{main} must be the filename of the main file.

There are a couple of
considerations in setting up the main and child documents:

%%%%%%%%%%%%%%%%%%%%%%%%%%%%%%%%%%%%%%%%
\paragraph{Restrictions.}

Please note the following restrictions:
\begin{itemize}
\item
|\childdocmain| must be called with one argument \textit{main}
to ensure compatibility with earlier version of the package.
It must either be empty (|\childdocmain{}|)
or precisely match the filename of the main file in which it is specified.
See \secref{sec:detection} for further information.
\item
The filename \textit{main} must be specified without the |.tex| extension.
\item
The filename \textit{main} is case sensitive
(even in case-insensitive file systems)
due to internal string comparison.
\item
The argument \textit{main} should be fully expanded, it cannot be a macro.
\item
Subdirectories and special characters should be avoided in filenames.
\item
The command |\childdocmain{|\textit{main}|}| must be followed by a whitespace.
It should not be followed immediately by another command
or by a comment mark `|%|'.
This is because the \TeX{} parser reads the token immediately following
the argument of |\childdocmain| and puts it
at the beginning of every child section;
however, a white\-space is ignored.
\end{itemize}

%%%%%%%%%%%%%%%%%%%%%%%%%%%%%%%%%%%%%%%%
\paragraph{Content of Main File.}

It is advisable to place all content in the child files included by |\include|.
Any output contained in the main file will appear in all child documents
unless suppressed manually;
it cannot be suppressed automatically by the |\includeonly| directive
and thus should normally be avoided.
A method to include some content in the main file
by means of conditional processing is described in \secref{sec:conditional}.

%%%%%%%%%%%%%%%%%%%%%%%%%%%%%%%%%%%%%%%%
\paragraph{Page Numbering.}

When only a part of the document is compiled,
the appropriate numbering of pages
(as well as other status parameters)
is determined from the |.aux| files.
The latter contain information from previous passes.
However this information needs to propagate through
all intermediate child documents.
Therefore the page numbering in child documents may well
be inconsistent until the complete document is compiled at least once.

A useful (if unconventional) way to always ensure a consistent
page numbering is to restart the numbering in each child document
and denote the pages by `\textit{child}|.|\textit{page}'
where \textit{child} represents the chapter/section number of the child file.
This can be achieved by the command
|\numberwithin{page}{|\textit{child}|}|
of the \textsf{amsmath} package
where \textit{child} can be |chapter| or |section|
depending on the chosen structuring.
Alternatively, one can modify the macro |\thepage| appropriately
and reset the counter |page| at the start of each child file.

%%%%%%%%%%%%%%%%%%%%%%%%%%%%%%%%%%%%%%%%%%%%%%%%%%%%%%%%%%%%%%%%%%%%%%%%%%%%%%%%
\subsection{Conditional Processing}
\label{sec:conditional}

The package provides a mechanism to compile different versions
of a document. To customise the versions further some conditional processing
can come in handy to distinguish which version is being compiled.
The package provides two macros to describe the compilation context:

%%%%%%%%%%%%%%%%%%%%%%%%%%%%%%%%%%%%%%%%
\DescribeMacro{\ifchilddoc}
The conditional |\ifchilddoc| distinguishes between the compilation of
child documents and the main document:
%
\begin{center}
|\ifchilddoc |\textit{child-code}| |[|\||else |\textit{main-code}]| \||fi|
\end{center}

%%%%%%%%%%%%%%%%%%%%%%%%%%%%%%%%%%%%%%%%
\DescribeMacro{\childdocname}
\DescribeMacro{\childdocjob}
The macro |\childdocname| contains the filename (without extension)
of the main or child file being processed.
Note that |\childdocjob| will always contain the name of the main file.

%%%%%%%%%%%%%%%%%%%%%%%%%%%%%%%%%%%%%%%%
\paragraph{Title Page.}

Conditional processing can be used to include a title or banner page
in the main document when proper precautions are taken.
Importantly, the code in the main file should ensure that the page counter
(as well as other status parameters which are stored in the |.aux| files)
takes the same value after the conditional processing.
Otherwise the page numbers may take divergent values
depending on which part is compiled.

For example, a title page could be declared by:
%
\begin{center}
\begin{tabular}{l}
|\ifchilddoc\||else|\\
|\addtocounter{page}{-1}|\\
\textit{code for title page}\\
|\newpage|\\
|\||fi|
\end{tabular}
\end{center}
%
A banner page for the child documents can be generated by:
%
\begin{center}
\begin{tabular}{l}
|\ifchilddoc|\\
|\addtocounter{page}{-1}|\\
\textit{code for banner page}\\
|\newpage|\\
|\||fi|
\end{tabular}
\end{center}
%
Here one could write a message such as:
\begin{center}
|This is the part \childdocname{} of \childdocjob{}.|
\end{center}

%%%%%%%%%%%%%%%%%%%%%%%%%%%%%%%%%%%%%%%%%%%%%%%%%%%%%%%%%%%%%%%%%%%%%%%%%%%%%%%%
\subsection{Flags}
\label{sec:flags}

The package makes it easy to generate different versions
of the main or child documents.
To this end compilation flags can be defined
and assigned different default values.
They will be particularly useful in conjunction
with the forwarding mechanism described in \secref{sec:forward}.

For example, it may be useful to have a flag |\version|
which can be set to |draft| or |final|.
The document source will contain some conditional code
depending on the value of |\version|.
Suppose further, the flag should default to |final| for the main file
and to |draft| for child files
which is a natural assignment for editing the document.
This is achieved by placing the following code
in the preamble of the main document
(below the |\childdocmain| directive):
%
\begin{center}
\begin{tabular}{l}
|\ifchilddoc|\\
|\providecommand{\version}{draft}|\\
|\||else|\\
|\providecommand{\version}{final}|\\
|\||fi|
\end{tabular}
\end{center}
%
The definition by |\providecommand| makes sure
that previous definitions are not overwritten.
Further statements |\providecommand{\version}{...}|
can thus be added before the above code to override it.

For the main file, one might add a line
(between |\childdocmain| and the above block)
%
\begin{center}
|%\ifchilddoc\||else\providecommand{\version}{draft}\||fi|
\end{center}
%
which can be uncommented to produce a draft version.
Likewise one can add a line to the very top of a child file
(above the |\childdocof{|\textit{main}|}| directive)
%
\begin{center}
|%\providecommand{\version}{final}|
\end{center}
%
which can be uncommented to produce the final version of this child document.

%%%%%%%%%%%%%%%%%%%%%%%%%%%%%%%%%%%%%%%%%%%%%%%%%%%%%%%%%%%%%%%%%%%%%%%%%%%%%%%%
\subsection{Forwarding}
\label{sec:forward}

Different versions of the main or child documents
using compilation flags as described in \secref{sec:flags}
can be (permanently) stored in different files
for convenient compilation, viewing and distribution.
To this end, the package defines a command
to pass on compilation to a different file:

%%%%%%%%%%%%%%%%%%%%%%%%%%%%%%%%%%%%%%%%
\DescribeMacro{\childdocforward}
The command |\childdocforward| redirects processing to
another source file:
%
\begin{center}
\begin{tabular}{l}
|\input{childdoc.def}|\\
|\childdocforward[|\textit{main}|]{|\textit{dest}|}|\\
\end{tabular}
\end{center}
%
The argument \textit{dest} is the destination file
(without extension).
It should be the main file or one of the child files.
Note that further \textsf{childdoc} directives
such as |\childdocof| and |\childdocforward|
in the indicated file will be processed in this form.
The optional argument \textit{main}
passes on directly to the main file \textit{main}
while pretending to compile the child \textit{dest}.
This form behaves as if \textit{dest}
issues |\childdocof{|\textit{main}|}| right away,
and no further \textsf{childdoc} directives will be processed.

%%%%%%%%%%%%%%%%%%%%%%%%%%%%%%%%%%%%%%%%
\DescribeMacro{\...prefix}
In the alternative form |\childdocforwardprefix|,
%
\begin{center}
\begin{tabular}{l}
|\input{childdoc.def}|\\
|\childdocforwardprefix[|\textit{main}|]{|\textit{prefix}|}{|\textit{dest}|}|
\end{tabular}
\end{center}
%
the destination file is determined by a pattern
depending on the current file:
To make this work, the current file must be called
`{\textit{prefix}\hspace{0.2em}\textit{suffix}}'
with \textit{prefix} matching precisely the argument.
Processing is then passed on to the file
`{\textit{dest}\hspace{0.2em}\textit{suffix}}'.
Surely, the same effect is achieved by
directly specifying the
argument `{\textit{dest}\hspace{0.2em}\textit{suffix}}'
in the first form.
However, that requires to set up a different file
for each child. With the alternative form of the command
all these files can have exactly the same content
which simplifies setting them up and maintaining them.

For example, the following file |draft.tex|
with a compilation flag |\version| as described in \secref{sec:flags}
compiles the main document as a draft:
%
\begin{center}
\begin{tabular}{l}
|\def\version{draft}|\\
|\input{childdoc.def}|\\
|\childdocforward{|\textit{main}|}|
\end{tabular}
\end{center}
%
Likewise, the following files |final|\textit{nn}|.tex|
compile the final version of the child document
|child|\textit{nn}|.tex|:
%
\begin{center}
\begin{tabular}{l}
|\def\version{final}|\\
|\input{childdoc.def}|\\
|\childdocforwardprefix{final}{child}|
\end{tabular}
\end{center}
%

Note that when several versions of a main file and/or of each child file
are to be generated, it may be convenient to set up a |Makefile| or
shell script to automatise the process.

%%%%%%%%%%%%%%%%%%%%%%%%%%%%%%%%%%%%%%%%%%%%%%%%%%%%%%%%%%%%%%%%%%%%%%%%%%%%%%%%
\subsection{Command Line Processing}
\label{sec:commandline}

The effect of redirection files can also be achieved by invoking
the \LaTeX{} compiler with a more elaborate command line.
Most conveniently this should be done as part
of a shell script or a |Makefile|.

When using \textsf{childdoc} in the main file, the following
command lines effectively perform a redirection
(note that depending on the shell being used,
backslashes may have to be doubled: `|\|' $\to$ `|\\|'):
%
\begin{center}
|... -jobname "|\textit{target}|" |\\|"|[\textit{flags}]%
|\input{childdoc.def}\childdocforward[|\textit{main}|]{|\textit{dest}|}"|
\end{center}
%
Here \textit{target} is the name of the output file,
\textit{main} is the name of the main file
and \textit{dest} is the name of the main or child file to be processed
(all filenames without extensions).
The optional argument \textit{main} can be omitted
if \textit{main} matches \textit{dest}.
Optionally, compilation \textit{flags} can be defined via |\def| commands.
This command line makes the \TeX{} engine believe
it is compiling the file \textit{target}
whose content is specified as the latter parameter.
The provided code then forwards the processing to
\textit{main} or \textit{dest} as described in \secref{sec:forward}.

%%%%%%%%%%%%%%%%%%%%%%%%%%%%%%%%%%%%%%%%%%%%%%%%%%%%%%%%%%%%%%%%%%%%%%%%%%%%%%%%
\subsection{Include by Input}
\label{sec:input}

Including child documents by |\include| has some restrictions by design.
Most notably, the content of a child document always occupies
its own set of pages; pages cannot be shared between child documents.
Usually, this behaviour makes perfect sense
because each child document contain an essential part of the document.
However, in some situations it may be desirable to compose
a document from a collection of parts
without having mandatory page breaks between then.
For this case, the package
provides a mechanism to include parts
by |\input| which can also be processed individually.
However, by construction this mechanism
requires manual handling of the content to be output.

%%%%%%%%%%%%%%%%%%%%%%%%%%%%%%%%%%%%%%%%
\DescribeMacro{\ifchilddocmanual}
The main file should be prepared as usual, see \secref{sec:include}.
However, the document body must make a distinction
between processing of an individual part and of the main document, e.g.:
%
\begin{center}
\begin{tabular}{l}
|\ifchilddocmanual|\\
|\input{\childdocname}|\\
|\||else|\\
\textit{document body with }|\input{|\textit{part}|}|\\
|\||fi|
\end{tabular}
\end{center}
%
The conditional |\ifchilddocmanual| is true whenever
a part to be included by |\input| is being compiled,
and the name of the part is stored in |\childdocname|.

%%%%%%%%%%%%%%%%%%%%%%%%%%%%%%%%%%%%%%%%
\DescribeMacro{\childdocby}
Each part to be included by |\input| should start with:
%
\begin{center}
\begin{tabular}{l}
|\input{childdoc.def}|\\
|\childdocby{|\textit{main}|}|\\
\end{tabular}
\end{center}
%
The directive |\childdocby| is similar to |\childdocof|
described in \secref{sec:include},
but the subsequent selection of content must be done manually.
To that end, both |\ifchilddoc| and |\ifchilddocmanual|
will be true upon processing of a part,
and the name of the part is stored in |\childdocname|.
Note that |\jobname| will be set to the filename of the current part
so that each part receives an individual |.aux| file
that does not interfere with the |.aux| file(s) of the main document.
This behaviour can be altered by the alternative form
|\childdocby[*]{|\textit{main}|}| (with a non-empty optional argument)
which uses the |.aux| file of the main document
by setting |\jobname| to \textit{main}.

%%%%%%%%%%%%%%%%%%%%%%%%%%%%%%%%%%%%%%%%%%%%%%%%%%%%%%%%%%%%%%%%%%%%%%%%%%%%%%%%
\subsection{Driver Development}
\label{sec:driver}

The \textsf{childdoc} mechanism can also be use for the development
of definition files such as \LaTeX{} styles or classes.
This case differs from the above setup with multiple parts
included by |\include| in that no |\includeonly| should be invoked.
This can be achieved by starting the include file
(before |\ProvidesPackage|) with:
%
\begin{center}
\begin{tabular}{l}
|\input{childdoc.def}|\\
|\childdocforward{|\textit{main}|}|\\
\end{tabular}
\end{center}
%
or alternatively with:
%
\begin{center}
\begin{tabular}{l}
|\input{childdoc.def}|\\
|\childdocby{|\textit{main}|}|\\
\end{tabular}
\end{center}
%
Both forms have slightly different effects as described above.
The main file is prepared as usual, see \secref{sec:include}.

%%%%%%%%%%%%%%%%%%%%%%%%%%%%%%%%%%%%%%%%%%%%%%%%%%%%%%%%%%%%%%%%%%%%%%%%%%%%%%%%
\subsection{Legacy Detection}
\label{sec:detection}

The directive |\childdocmain| in the main file can detect
whether the complete document or merely a child is to be compiled
even without using the directive |\childdocof|.
This method is deprecated because it is less robust
and there is no compelling reason to use it;
it is merely provided for backward compatibility
and it may be removed in future versions.

If the detection mechanism is to be used,
it is mandatory to correctly specify
the filename of the main file as the argument of |\childdocmain|:
%
\begin{center}
\begin{tabular}{l}
|\input{childdoc.def}|\\
|\childdocmain{|\textit{main}|}|\\
\end{tabular}
\end{center}
%
If |\jobname| does not match the argument \textit{main} of |\childdocmain|,
it is assumed that |\jobname| points to the child file to be compiled.
When using |\childdocmain| with the main file specified as argument,
it suffices to start a child file
with just |\input{|\textit{main}|}|
without loading of the package and using |\childdocof|.
If instead all processing is done
with the appropriate \textsf{childdoc} directives,
the argument of \textit{main} of |\childdocmain| can be empty.

An alternative version of the command line processing described
in \secref{sec:commandline} using the detection mechanism reads:
%
\begin{center}
|... -jobname "|\textit{target}|" "|[\textit{flags}]%
[|\def\jobname{|\textit{dest}|}|]|\input{|\textit{main}|}"|
\end{center}

%%%%%%%%%%%%%%%%%%%%%%%%%%%%%%%%%%%%%%%%%%%%%%%%%%%%%%%%%%%%%%%%%%%%%%%%%%%%%%%%
\subsection{Manual Code}
\label{sec:manual}

In case one cannot be certain whether the definitions file |childdoc.def|
is installed on the target \TeX{} distribution
and one prefers not to ship it,
it is conceivable to paste a few relevant commands into the sources.

To that end, drop all statements |\input{childdoc.def}|
and perform the replacements as outlined below.
Instead of |\childdocmain{|\textit{main}|}| add the following code
to the top of the main file:
%
\begin{center}
\begin{tabular}{l}
|\||ifdefined\childdocname\endinput\||fi\newif\ifchilddoc|\\
|\edef\childdocname{\scantokens\expandafter{\jobname\noexpand}}|\\
|\def\childdocmain{|\textit{main}|}\||ifx\childdocmain\childdocname\||else|\\
|\childdoctrue\includeonly{\childdocname}\let\jobname\childdocmain\||fi|\\
\end{tabular}
\end{center}
%
Instead of |\childdocof{|\textit{main}|}| just include the main file
at the top of each child file:
%
\begin{center}
|\input{|\textit{main}|}|
\end{center}
%
A simple redirection |\childdocforward{|\textit{dest}|}| is achieved by:
%
\begin{center}
|\def\jobname{|\textit{dest}|}\input{\jobname}|
\end{center}
%
The redirection with prefix
|\childdocforwardprefix[|\textit{prefix}|]{|\textit{dest}|}|
is accomplished by:
%
\begin{center}
\begin{tabular}{l}
|{\edef\jobname{\scantokens\expandafter{\jobname\noexpand}}|\\
|\def\redirectjob |\textit{prefix}|#1~~~{\gdef\jobname{|\textit{dest}|#1}}|\\
|\expandafter\redirectjob\jobname~~~}\input{\jobname}|
\end{tabular}
\end{center}

In an alternative approach,
child documents can be compiled by a specific command line
without additional code or specific definitions:
%
\begin{center}
|... -jobname "|\textit{target}|" "|[\textit{flags}]%
|\includeonly{|\textit{dest}|}\input{|\textit{main}|}"|
\end{center}
%

%%%%%%%%%%%%%%%%%%%%%%%%%%%%%%%%%%%%%%%%%%%%%%%%%%%%%%%%%%%%%%%%%%%%%%%%%%%%%%%%
%%%%%%%%%%%%%%%%%%%%%%%%%%%%%%%%%%%%%%%%%%%%%%%%%%%%%%%%%%%%%%%%%%%%%%%%%%%%%%%%
\section{Information}

%%%%%%%%%%%%%%%%%%%%%%%%%%%%%%%%%%%%%%%%%%%%%%%%%%%%%%%%%%%%%%%%%%%%%%%%%%%%%%%%
\subsection{Copyright}

Copyright \copyright{} 2017--2018 Niklas Beisert

This work may be distributed and/or modified under the
conditions of the \LaTeX{} Project Public License, either version 1.3
of this license or (at your option) any later version.
The latest version of this license is in
  \url{http://www.latex-project.org/lppl.txt}
and version 1.3 or later is part of all distributions of \LaTeX{}
version 2005/12/01 or later.

This work has the LPPL maintenance status `maintained'.

The Current Maintainer of this work is Niklas Beisert.

This work consists of the files |README.txt|, |childdoc.ins| and |childdoc.dtx|
as well as the derived files |childdoc.def|, |cdocsamp.tex|
with |cdocsch1.tex|, |cdocsch2.tex|, |cdocspt3.tex|, |cdocspt4.tex|,
|cdocsdrf.tex|, |cdocsfn1.tex|, |cdocsfn2.tex|
as well as |childdoc.pdf|.

%%%%%%%%%%%%%%%%%%%%%%%%%%%%%%%%%%%%%%%%%%%%%%%%%%%%%%%%%%%%%%%%%%%%%%%%%%%%%%%%
\subsection{Files and Installation}

The package consists of the files:
%
\begin{center}
\begin{tabular}{ll}
    |README.txt|   & readme file \\
    |childdoc.ins| & installation file \\
    |childdoc.dtx| & source file \\
    |childdoc.def| & definition file \\
    |cdocsamp.tex| & sample main file \\
    |cdocsch1.tex| & sample include file \\
    |cdocsch2.tex| & sample include file \\
    |cdocspt3.tex| & sample part file \\
    |cdocspt4.tex| & sample part file \\
    |cdocsdrf.tex| & sample redirection file \\
    |cdocsfn1.tex| & sample redirection file \\
    |cdocsfn2.tex| & sample redirection file \\
    |childdoc.pdf| & manual
\end{tabular}
\end{center}
%
The distribution consists of the files
|README.txt|, |childdoc.ins| and |childdoc.dtx|.
%
\begin{itemize}
\item
Run (pdf)\LaTeX{} on |childdoc.dtx|
to compile the manual |childdoc.pdf| (this file).
\item
Run \LaTeX{} on |childdoc.ins| to create the definitions file |childdoc.def|
and the sample |cdocsamp.tex| with include files
|cdocsch1.tex|, |cdocsch2.tex|, |cdocspt3.tex|, |cdocspt4.tex|,
|cdocsdrf.tex|, |cdocsfn1.tex|, |cdocsfn2.tex|.
Then copy the file |childdoc.def| to an appropriate directory of your \LaTeX{}
distribution, e.g.\ \textit{texmf-root}|/tex/latex/childdoc|.
\end{itemize}

%%%%%%%%%%%%%%%%%%%%%%%%%%%%%%%%%%%%%%%%%%%%%%%%%%%%%%%%%%%%%%%%%%%%%%%%%%%%%%%%
\subsection{Related CTAN Packages}

There are several other packages which offer a similar functionality:
%
\begin{itemize}
\item
The packages
\href{http://ctan.org/pkg/docmute}{\textsf{docmute}},
\href{http://ctan.org/pkg/includex}{\textsf{includex}} and
\href{http://ctan.org/pkg/standalone}{\textsf{standalone}}
provide commands to include only the document body of
a child file thus allowing both files to be compiled individually.
\item
The packages \href{http://ctan.org/pkg/subdocs}{\textsf{subdocs}}
and \href{http://ctan.org/pkg/subfiles}{\textsf{subfiles}}
provide structures in which the main and child documents can be
encapsulated and allowing them to be compiled individually.
The inclusion mechanism is different from the conventional |\include|.
\item
The package \href{http://ctan.org/pkg/combine}{\textsf{combine}}
is an elaborate solution to combine several documents into one.
\end{itemize}
%
See also the CTAN topic \href{http://ctan.org/topic/subdocs}{\textsf{subdocs}}
for further related packages.
The present package differs from the above solutions in that
a document structure constructed with the conventional |\include| mechanism
just needs two extra commands at the top of every file
such that all constituent files can be compiled individually.

%%%%%%%%%%%%%%%%%%%%%%%%%%%%%%%%%%%%%%%%%%%%%%%%%%%%%%%%%%%%%%%%%%%%%%%%%%%%%%%%
%\subsection{Feature Suggestions}
%
%The following is a list of features which may be useful for future
%versions of this package:
%%
%\begin{itemize}
%\item
%\ldots
%\end{itemize}

%%%%%%%%%%%%%%%%%%%%%%%%%%%%%%%%%%%%%%%%%%%%%%%%%%%%%%%%%%%%%%%%%%%%%%%%%%%%%%%%
\subsection{Revision History}

%%%%%%%%%%%%%%%%%%%%%%%%%%%%%%%%%%%%%%%%
\paragraph{v2.0:} 2018/12/30

\begin{itemize}
\item
immediate forward processing
\item
added |\childdocby| mechanism
\item
manual restructured
\end{itemize}

%%%%%%%%%%%%%%%%%%%%%%%%%%%%%%%%%%%%%%%%
\paragraph{v1.6:} 2018/01/17

\begin{itemize}
\item
application for development of include files
\item
corrections to manual
\end{itemize}

%%%%%%%%%%%%%%%%%%%%%%%%%%%%%%%%%%%%%%%%
\paragraph{v1.5:} 2017/05/21

\begin{itemize}
\item
more complete structuring introduced
\item
|\childdocof| introduced
\item
|\childdoc| renamed to |\childdocmain|
\item
|\childredirect| renamed to |\childdocforward| and |\childdocforwardprefix|
and functionality expanded
\end{itemize}

%%%%%%%%%%%%%%%%%%%%%%%%%%%%%%%%%%%%%%%%
\paragraph{v1.0:} 2017/04/27

\begin{itemize}
\item
manual and install package
\item
first version published on CTAN
\end{itemize}

%%%%%%%%%%%%%%%%%%%%%%%%%%%%%%%%%%%%%%%%
\paragraph{v0.6:} 2017/04/26

\begin{itemize}
\item
redirection mechanism added
\end{itemize}

%%%%%%%%%%%%%%%%%%%%%%%%%%%%%%%%%%%%%%%%
\paragraph{v0.5:} 2017/04/26

\begin{itemize}
\item
functionality in definition file
\end{itemize}


%%%%%%%%%%%%%%%%%%%%%%%%%%%%%%%%%%%%%%%%%%%%%%%%%%%%%%%%%%%%%%%%%%%%%%%%%%%%%%%%
%%%%%%%%%%%%%%%%%%%%%%%%%%%%%%%%%%%%%%%%%%%%%%%%%%%%%%%%%%%%%%%%%%%%%%%%%%%%%%%%
%%%%%%%%%%%%%%%%%%%%%%%%%%%%%%%%%%%%%%%%%%%%%%%%%%%%%%%%%%%%%%%%%%%%%%%%%%%%%%%%
\appendix

\settowidth\MacroIndent{\rmfamily\scriptsize 000\ }

 \DocInput{childdoc.dtx}

\end{document}
%</driver>
% \fi
%
% %%%%%%%%%%%%%%%%%%%%%%%%%%%%%%%%%%%%%%%%%%%%%%%%%%%%%%%%%%%%%%%%%%%%%%%%%%%%%%
% %%%%%%%%%%%%%%%%%%%%%%%%%%%%%%%%%%%%%%%%%%%%%%%%%%%%%%%%%%%%%%%%%%%%%%%%%%%%%%
% \section{Sample}
%\iffalse
%<*samplemain>
%\fi
%
% The following presents a sample document
% with two chapters, two parts, a title page,
% a compile flag as well as three forwarding files to set the flag.
% It consists of eight |.tex| files:
% \begin{center}
% \begin{tabular}{ll}
% |cdocsamp.tex|&main file\\
% |cdocsch1.tex|&include file for chapter 1\\
% |cdocsch2.tex|&include file for chapter 2\\
% |cdocspt3.tex|&include file for part 3\\
% |cdocspt4.tex|&include file for part 4\\
% |cdocsdrf.tex|&forwarding file for main file in draft mode\\
% |cdocsfi1.tex|&forwarding file for final version of chapter 1\\
% |cdocsfi2.tex|&forwarding file for final version of chapter 2\\
% \end{tabular}
% \end{center}
% Each of the eight files can be compiled directly by the \LaTeX{} compiler.
%
% %%%%%%%%%%%%%%%%%%%%%%%%%%%%%%%%%%%%%%
% \paragraph{Main File.}
%
% The main file is called |cdocsamp.tex|.
%
% Load the \textsf{childdoc} definitions and
% declare the filename for the main document:
%    \begin{macrocode}
\input{childdoc.def}
\childdocmain{}
%    \end{macrocode}

% Optional override for |\version| flag:
%    \begin{macrocode}
%%\ifchilddoc\else\providecommand{\version}{draft}\fi
%    \end{macrocode}

% Define the default values for the |\version| flag
% (|final| for the main file and |draft| for childs):
%    \begin{macrocode}
\ifchilddoc
\providecommand{\version}{draft}
\else
\providecommand{\version}{final}
\fi
%    \end{macrocode}

% Load the standard document class:
%    \begin{macrocode}
\documentclass[12pt]{article}
%    \end{macrocode}

% Start the document body:
%    \begin{macrocode}
\begin{document}
%    \end{macrocode}

% Declare a title page.
% Print title, part of document being processed and version flag:
%    \begin{macrocode}
\addtocounter{page}{-1}
\begin{center}
{\LARGE\bfseries{}childdoc example\par}
\vspace{1cm}
\ifchilddoc
\ifchilddocmanual part\else chapter\fi:
`\childdocname' of `\childdocjob'\par
\else
main document: `\childdocjob'\par
\fi
version: \version\par
\end{center}
\newpage
%    \end{macrocode}

% Manually include selected file,
% otherwise process as usual:
%    \begin{macrocode}
\ifchilddocmanual
\section*{part `\childdocname'}
\input{\childdocname}
\else
%    \end{macrocode}

% Include the two chapters:
%    \begin{macrocode}
\include{cdocsch1}
\include{cdocsch2}
%    \end{macrocode}

% Include the two parts unless only chapters should be displayed:
%    \begin{macrocode}
\ifchilddoc\else
\section{part three}
\input{cdocspt3}
\section{part four}
\input{cdocspt4}
\fi
%    \end{macrocode}

% Process as usual until here:
%    \begin{macrocode}
\fi
%    \end{macrocode}

% End of document body:
%    \begin{macrocode}
\end{document}
%    \end{macrocode}
%\iffalse
%</samplemain>
%\fi
%
% %%%%%%%%%%%%%%%%%%%%%%%%%%%%%%%%%%%%%%
% \paragraph{Chapter Include Files.}
%
% The include files are called |cdocsch1.tex| and |cdocsch2.tex|.
%
%\iffalse
%<*samplechap1|samplechap2>
%\fi

% Optional override for |\version| flag:
%    \begin{macrocode}
%%\providecommand{\version}{final}
%    \end{macrocode}

% Include the main document:
%    \begin{macrocode}
\input{childdoc.def}
\childdocof{cdocsamp}
%    \end{macrocode}

%\iffalse
%</samplechap1|samplechap2>
%\fi
%
%\iffalse
%<*samplechap1>
%\fi
% Some text for chapter 1:
%    \begin{macrocode}
\section{one}
some text in chapter one
%    \end{macrocode}

%\iffalse
%</samplechap1>
%\fi
% Some text for chapter 2:
%\iffalse
%<*samplechap2>
%\fi
%    \begin{macrocode}
\section{two}
more text in chapter two
%    \end{macrocode}

%\iffalse
%</samplechap2>
%\fi
%
% %%%%%%%%%%%%%%%%%%%%%%%%%%%%%%%%%%%%%%
% \paragraph{Part Include Files.}
%
% The include files are called |cdocspt3.tex| and |cdocspt4.tex|.
%
%\iffalse
%<*samplepart3|samplepart4>
%\fi

% Optional override for |\version| flag:
%    \begin{macrocode}
%%\providecommand{\version}{final}
%    \end{macrocode}

% Include the main document:
%    \begin{macrocode}
\input{childdoc.def}
\childdocby{cdocsamp}
%    \end{macrocode}

%\iffalse
%</samplepart3|samplepart4>
%\fi
%
%\iffalse
%<*samplepart3>
%\fi
% Some text for part 3:
%    \begin{macrocode}
some text in part three
%    \end{macrocode}

%\iffalse
%</samplepart3>
%\fi
% Some text for part 4:
%\iffalse
%<*samplepart4>
%\fi
%    \begin{macrocode}
more text in part four
%    \end{macrocode}

%\iffalse
%</samplepart4>
%\fi
%
% %%%%%%%%%%%%%%%%%%%%%%%%%%%%%%%%%%%%%%
% \paragraph{Forwarding for a Complete Draft.}
%
% The following forwarding file |cdocsdrf.tex|
% compiles the main document in draft mode:
%\iffalse
%<*sampledraft>
%\fi
%    \begin{macrocode}
\def\version{draft}
\input{childdoc.def}
\childdocforward{cdocsamp}
%    \end{macrocode}

%\iffalse
%</sampledraft>
%\fi
%
% %%%%%%%%%%%%%%%%%%%%%%%%%%%%%%%%%%%%%%
% \paragraph{Forwarding for Final Version of the Chapters.}
%
% The following forwarding files |cdocsfn1.tex| and |cdocsfn2.tex|
% (with identical content)
% compile the final versions of the child documents
% |cdocsch1.tex| and |cdocsch2.tex|, respectively:
%\iffalse
%<*samplefinal>
%\fi
%    \begin{macrocode}
\def\version{final}
\input{childdoc.def}
\childdocforwardprefix[cdocsamp]{cdocsfn}{cdocsch}
%    \end{macrocode}

%\iffalse
%</samplefinal>
%\fi
%
% %%%%%%%%%%%%%%%%%%%%%%%%%%%%%%%%%%%%%%
% \paragraph{Command Line Processing.}
%
% The following three command lines generate the output files
% |cdocscld|, |cdocscl1| and |cdocscl2|
% which should be identical to
% |cdocsdrf|, |cdocsch1| and |cdocsfn2|, respectively:
% \begin{center}
% \begin{tabular}{l}
% |latex -jobname cdocscld \|\\
% |  "\def\version{draft}\input{childdoc.def}\childdocforward{cdocsamp}"|\\
% |latex -jobname cdocscl1 \|\\
% |  "\input{childdoc.def}\childdocforward[cdocsamp]{cdocsch1}"|\\
% |latex -jobname cdocscl2 \|\\
% |  "\def\version{final}\input{childdoc.def}\childdocforward{cdocsch2}"|
% \end{tabular}
% \end{center}
% Note that the trailing backslash on each first line
% merely continues the input to the second line
% (for convenient cut ant paste).
% Furthermore, the command |latex| can be replaced by any
% of its alternative versions such as |pdflatex|.
%
% %%%%%%%%%%%%%%%%%%%%%%%%%%%%%%%%%%%%%%%%%%%%%%%%%%%%%%%%%%%%%%%%%%%%%%%%%%%%%%
% %%%%%%%%%%%%%%%%%%%%%%%%%%%%%%%%%%%%%%%%%%%%%%%%%%%%%%%%%%%%%%%%%%%%%%%%%%%%%%
% \section{Implementation}
%\iffalse
%<*package>
%\fi
%
% This section describes the definitions file |childdoc.def|.

% The definitions cannot be loaded using |\usepackage| or |\RequirePackage|
% which has a mechanism to prevent loading a style file more than once.
% When loading the definitions by means of |\input|
% multiple instances have to be prevented manually:
%\iffalse
%This code needs to be before the `\ProvidesFile' directive
%which is defined at the beginning of this file.
%Therefore it is also placed there and commented out here.
%</package>
%<*discard>
%\fi
%    \begin{macrocode}
\ifdefined\childdocmain\endinput\fi
%    \end{macrocode}
%\iffalse
%</discard>
%<*package>
%\fi
%
% \macro{\ifchilddoc}
% \macro{\ifchilddocmanual}
% The conditional |\ifchilddoc| tells whether a
% child (true) or main (false) document is being compiled.
% The conditional |\ifchilddocmanual| tells whether
% the |\includeonly| mechanism is used (false) or
% the selection of child files must be performed manually (true).
% The definitions initialise to false:
%    \begin{macrocode}
\newif\ifchilddoc
\newif\ifchilddocmanual
%    \end{macrocode}

% \macro{\childdocname}
% \macro{\childdocjob}
% The macro |\childdocname| stores the name of the main document
% to be compiled. The macro |\childdocjob| stores the name of
% the document on which the \LaTeX{} compiler was originally invoked.
% The content of |\jobname| cannot be compared
% to filenames specified in the source due to different catcodes.
% The following code rescans |\jobname|, stores the result
% in |\childdocname| and saves a copy in |\childdocjob|:
%    \begin{macrocode}
\edef\childdocname{\scantokens\expandafter{\jobname\noexpand}}
\let\childdocjob\childdocname
%    \end{macrocode}

% \macro{\childdocdisable}
% The macro |\childdocdisable| prevents the main file
% from being processed more than once.
% At this stage, the main document command |\childdocmain|
% is assumed to be called once again where it should do nothing.
% Any subsequent call to it should prevent
% a secondary processing of the main document
% It overwrites the forwarding commands
% |\childdocof| and |\childdocforward|
% with empty macros to prevent further inclusions of the main document:
%    \begin{macrocode}
\newcommand{\childdocdisable}
{
  \renewcommand{\childdocmain}[1]{\renewcommand{\childdocmain}[1]{\endinput}}
  \renewcommand{\childdocof}[1]{}
  \renewcommand{\childdocby}[2][]{}
  \renewcommand{\childdocforward}[2][]{}
  \renewcommand{\childdocdisable}{}
}
%    \end{macrocode}

% \macro{\childdocmain}
% The macro |\childdocmain| is to be called at the top of the main file
% with nothing or the main filename (without extension) as argument.
% First, it breaks loops.
% If the argument is not empty and does not match |\childdocname|
% (which is set by the first inclusion of |childdoc.def|),
% |\ifchilddoc| is set to true, |\includeonly| is applied to the child file
% and |\jobname| is set to the main file
% (for proper handling of |.aux| files):
%    \begin{macrocode}
\newcommand{\childdocmain}[1]
{
  \childdocdisable\childdocmain{}
  \if?#1?\else
    \begingroup
      \def\childdoctmp{#1}
      \ifx\childdoctmp\childdocname
        \def\childdoctmp{}
      \else
        \def\childdoctmp
        {
          \childdoctrue
          \includeonly{\childdocname}
          \def\childdocjob{#1}
          \def\jobname{#1}
        }
      \fi
      \expandafter
    \endgroup
    \childdoctmp
  \fi
}
%    \end{macrocode}

% \macro{\childdocof}
% The command |\childdocof| redirects
% compilation to the main file |#1|.
%    \begin{macrocode}
\newcommand{\childdocof}[1]
{
  \childdocdisable
  \childdoctrue
  \includeonly{\childdocname}
  \def\jobname{#1}
  \def\childdocjob{#1}
  \input{#1}
}
%    \end{macrocode}

% \macro{\childdocby}
% The command |\childdocby| ....
%    \begin{macrocode}
\newcommand{\childdocby}[2][]
{
  \childdocdisable
  \childdoctrue
  \childdocmanualtrue
  \if?#1?\else
    \def\jobname{#2}
  \fi
  \def\childdocjob{#2}
  \input{#2}
  \endinput
}
%    \end{macrocode}

% \macro{\childdocforward}
% The command |\childdocforward| redirects
% compilation to the main file or
% (if the optional argument is given) a child file.
% Parameters are set as if the main file
% or a child file starting with |\childdocof| was compiled.
% Then compilation is handed over to the main file:
%    \begin{macrocode}
\newcommand{\childdocforward}[2][]
{
  \begingroup
    \if?#1?
      \def\childdoctmp
      {
        \def\childdocname{#2}
        \def\childdocjob{#2}
        \def\jobname{#2}
        \input{#2}
        \endinput
      }
    \else
      \def\childdoctmp
      {
        \childdocdisable
        \def\childdocname{#2}
        \childdoctrue
        \includeonly{#2}
        \def\childdocjob{#1}
        \def\jobname{#1}
        \input{#1}
        \endinput
      }
    \fi
    \expandafter
  \endgroup
  \childdoctmp
}
%    \end{macrocode}

% \macro{\childdocforwardprefix}
% The command |\childdocforwardprefix| redirects
% compilation to the main or a child file by means of a pattern.
% The prefix |#1| in the current filename is replaced by |#2|
% and the suffix of the current filename is kept
% (it is assumed that the filename does not contain the substring `|~~~|'
% which is used as a delimiter).
% Compilation is handed over to the new file by |\childdocforward|:
%    \begin{macrocode}
\newcommand{\childdocforwardprefix}[3][]
{
  \begingroup
    \def\childdocextract #2##1~~~{\def\childdoctmp{\childdocforward[#1]{#3##1}}}
    \expandafter\childdocextract\childdocname~~~
    \expandafter
  \endgroup
  \childdoctmp
}
%    \end{macrocode}

% \macro{\childdoc}
% The deprecated macro |\childdoc| is a legacy version of |\childdocmain|:
%    \begin{macrocode}
\newcommand{\childdoc}{\childdocmain}
%    \end{macrocode}

% \macro{\childdocredirect}
% The deprecated macro |\childdocredirect| is a legacy version
% of |\childdocforward| and |\childdocforwardprefix|:
%    \begin{macrocode}
\newcommand{\childdocredirect}[2][]
{
  \begingroup
    \if?#1?
      \def\childdoctmp{\childdocforward{#2}}
    \else
      \def\childdoctmp{\childdocforwardprefix{#1}{#2}}
    \fi
    \expandafter
  \endgroup
  \childdoctmp
}
%    \end{macrocode}

%\iffalse
%</package>
%\fi
%
\endinput

\childdocmain{}
%    \end{macrocode}

% Optional override for |\version| flag:
%    \begin{macrocode}
%%\ifchilddoc\else\providecommand{\version}{draft}\fi
%    \end{macrocode}

% Define the default values for the |\version| flag
% (|final| for the main file and |draft| for childs):
%    \begin{macrocode}
\ifchilddoc
\providecommand{\version}{draft}
\else
\providecommand{\version}{final}
\fi
%    \end{macrocode}

% Load the standard document class:
%    \begin{macrocode}
\documentclass[12pt]{article}
%    \end{macrocode}

% Start the document body:
%    \begin{macrocode}
\begin{document}
%    \end{macrocode}

% Declare a title page.
% Print title, part of document being processed and version flag:
%    \begin{macrocode}
\addtocounter{page}{-1}
\begin{center}
{\LARGE\bfseries{}childdoc example\par}
\vspace{1cm}
\ifchilddoc
\ifchilddocmanual part\else chapter\fi:
`\childdocname' of `\childdocjob'\par
\else
main document: `\childdocjob'\par
\fi
version: \version\par
\end{center}
\newpage
%    \end{macrocode}

% Manually include selected file,
% otherwise process as usual:
%    \begin{macrocode}
\ifchilddocmanual
\section*{part `\childdocname'}
\input{\childdocname}
\else
%    \end{macrocode}

% Include the two chapters:
%    \begin{macrocode}
\include{cdocsch1}
\include{cdocsch2}
%    \end{macrocode}

% Include the two parts unless only chapters should be displayed:
%    \begin{macrocode}
\ifchilddoc\else
\section{part three}
\input{cdocspt3}
\section{part four}
\input{cdocspt4}
\fi
%    \end{macrocode}

% Process as usual until here:
%    \begin{macrocode}
\fi
%    \end{macrocode}

% End of document body:
%    \begin{macrocode}
\end{document}
%    \end{macrocode}
%\iffalse
%</samplemain>
%\fi
%
% %%%%%%%%%%%%%%%%%%%%%%%%%%%%%%%%%%%%%%
% \paragraph{Chapter Include Files.}
%
% The include files are called |cdocsch1.tex| and |cdocsch2.tex|.
%
%\iffalse
%<*samplechap1|samplechap2>
%\fi

% Optional override for |\version| flag:
%    \begin{macrocode}
%%\providecommand{\version}{final}
%    \end{macrocode}

% Include the main document:
%    \begin{macrocode}
% \iffalse
%
% childdoc.dtx Copyright (C) 2017-2018 Niklas Beisert
%
% This work may be distributed and/or modified under the
% conditions of the LaTeX Project Public License, either version 1.3
% of this license or (at your option) any later version.
% The latest version of this license is in
%   http://www.latex-project.org/lppl.txt
% and version 1.3 or later is part of all distributions of LaTeX
% version 2005/12/01 or later.
%
% This work has the LPPL maintenance status `maintained'.
%
% The Current Maintainer of this work is Niklas Beisert.
%
% This work consists of the files childdoc.dtx and childdoc.ins
% and the derived files childdoc.def and cdocsamp.tex with
% cdocsch1.tex, cdocsch2.tex, cdocsdrf.tex, cdocsfn1.tex, cdocsfn2.tex.
%
%<package>\ifdefined\childdocmain\endinput\fi
%<package>\ProvidesFile{childdoc.def}[2018/12/30 v2.0 child document driver]
%<samplemain>\ProvidesFile{cdocsamp.tex}[2018/12/30 v2.0 sample for childdoc]
%<*driver>
%\ProvidesFile{childdoc.drv}[2018/12/30 v2.0 childdoc reference manual file]
\PassOptionsToClass{10pt,a4paper}{article}
\documentclass{ltxdoc}

\usepackage[margin=35mm]{geometry}
\usepackage{hyperref}
\usepackage{hyperxmp}
\usepackage[usenames]{color}

\hypersetup{colorlinks=true}
\hypersetup{pdfstartview=FitH}
\hypersetup{pdfpagemode=UseNone}
\hypersetup{pdfsource={}}
\hypersetup{pdflang={en-UK}}
\hypersetup{pdfcopyright={Copyright 2017-2018 Niklas Beisert.
  This work may be distributed and/or modified under the
  conditions of the LaTeX Project Public License, either version 1.3
  of this license or (at your option) any later version.}}
\hypersetup{pdflicenseurl={http://www.latex-project.org/lppl.txt}}
\hypersetup{pdfcontactaddress={ETH Zurich, ITP, HIT K,
  Wolfgang-Pauli-Strasse 27}}
\hypersetup{pdfcontactpostcode={8093}}
\hypersetup{pdfcontactcity={Zurich}}
\hypersetup{pdfcontactcountry={Switzerland}}
\hypersetup{pdfcontactemail={nbeisert@itp.phys.ethz.ch}}
\hypersetup{pdfcontacturl={http://people.phys.ethz.ch/\xmptilde nbeisert/}}

\newcommand{\secref}[1]{\hyperref[#1]{section \ref*{#1}}}

\parskip1ex
\parindent0pt
\let\olditemize\itemize
\def\itemize{\olditemize\parskip0pt}

\begin{document}

\title{The \textsf{childdoc} Package}
\hypersetup{pdftitle={The childdoc Package}}
\author{Niklas Beisert\\[2ex]
  Institut f\"ur Theoretische Physik\\
  Eidgen\"ossische Technische Hochschule Z\"urich\\
  Wolfgang-Pauli-Strasse 27, 8093 Z\"urich, Switzerland\\[1ex]
  \href{mailto:nbeisert@itp.phys.ethz.ch}
  {\texttt{nbeisert@itp.phys.ethz.ch}}}
\hypersetup{pdfauthor={Niklas Beisert}}
\hypersetup{pdfsubject={Manual for the LaTeX2e Package childdoc}}
\date{30 December 2018, \textsf{v2.0}}
\maketitle

\begin{abstract}\noindent
\textsf{childdoc} is a \LaTeXe{} package
that enables the direct compilation
of document sections included by |\include|
to individual files.
\end{abstract}

\begingroup
\parskip0ex
\tableofcontents
\endgroup

%%%%%%%%%%%%%%%%%%%%%%%%%%%%%%%%%%%%%%%%%%%%%%%%%%%%%%%%%%%%%%%%%%%%%%%%%%%%%%%%
%%%%%%%%%%%%%%%%%%%%%%%%%%%%%%%%%%%%%%%%%%%%%%%%%%%%%%%%%%%%%%%%%%%%%%%%%%%%%%%%
\section{Introduction}

\LaTeX{} provides a mechanism to structure a large document (such as a book)
into a main file and several child files (containing the chapters)
using the |\include| command.
This mechanism is beneficial for documents
which span hundreds of pages in order to
make the source file(s) more manageable.
Moreover, compilation can be restricted to
selected child files by means of the |\includeonly| command.
The latter feature can be used to reduce the compilation time while editing
(this was significantly more useful in the earlier days of \LaTeX{})
or to generate a smaller document which is easier to navigate.
Another application of |\includeonly| is to generate
documents consisting of selected parts of the complete document.

However, there are a few drawbacks of the plain |\include| mechanism:
\begin{itemize}
\item
The child files cannot be compiled on their own,
they can only be compiled via the main file.
A naive editing environment
(such as a text editor with an option
to have the current file processed by \LaTeX)
may require one to switch to the main file before compiling;
attempting to compile the child file produces errors.
\item
The main file must be modified (each time)
to adjust the |\includeonly| command
to the present needs. This easily leaves the main file in a messy state.
\item
The generated document will always carry the filename
of the main document. This is inconvenient if
several child files are to be compiled and
to be kept for distribution.
\end{itemize}

The present package provides a simple interface
to make child files individually compilable by \LaTeX{}.
Compiling a child file then has the same effect as compiling
the main file with an |\includeonly| command
to select the appropriate child.
Moreover the generated document will carry the name of the child
rather than the main file.
This resolves all three above issues.

This feature is meant to make the editing of books,
thesis documents and lecture notes somewhat more convenient.
However, the package can also be used efficiently for
composing a series of documents (such as exercise sheets)
which are typically distributed individually.
It then assists the author in generating the individual documents
(potentially in different versions)
as well as a document containing the collected series.
Another application is in developing style files
or other kinds of included material
where compilation of the style file could redirect
to a sample or test file.

%%%%%%%%%%%%%%%%%%%%%%%%%%%%%%%%%%%%%%%%%%%%%%%%%%%%%%%%%%%%%%%%%%%%%%%%%%%%%%%%
%%%%%%%%%%%%%%%%%%%%%%%%%%%%%%%%%%%%%%%%%%%%%%%%%%%%%%%%%%%%%%%%%%%%%%%%%%%%%%%%
\section{Usage}

First of all, the package \textsf{childdoc} is \emph{not} a standard
\LaTeXe{} |.sty| style file! Therefore it needs to be invoked in
a non-standard way.

%%%%%%%%%%%%%%%%%%%%%%%%%%%%%%%%%%%%%%%%%%%%%%%%%%%%%%%%%%%%%%%%%%%%%%%%%%%%%%%%
\subsection{Included Files}
\label{sec:include}

%%%%%%%%%%%%%%%%%%%%%%%%%%%%%%%%%%%%%%%%
\DescribeMacro{\childdocmain}
To use the package, add the commands
\begin{center}
\begin{tabular}{l}
|\input{childdoc.def}|\\
|\childdocmain{}|\\
\end{tabular}
\end{center}
at the very top of the main \LaTeX{} file,
in particular \emph{before} the |\documentclass| statement!
The argument of |\childdocmain| should be left empty
(but it must be present).

%%%%%%%%%%%%%%%%%%%%%%%%%%%%%%%%%%%%%%%%
\DescribeMacro{\childdocof}
Furthermore, add the commands
\begin{center}
\begin{tabular}{l}
|\input{childdoc.def}|\\
|\childdocof{|\textit{main}|}|\\
\end{tabular}
\end{center}
at the top of every child file \textit{child}
which is included by |\include{|\textit{child}|}|
from within the main file
(or at least for those files to be compiled individually).
The argument \textit{main} must be the filename of the main file.

There are a couple of
considerations in setting up the main and child documents:

%%%%%%%%%%%%%%%%%%%%%%%%%%%%%%%%%%%%%%%%
\paragraph{Restrictions.}

Please note the following restrictions:
\begin{itemize}
\item
|\childdocmain| must be called with one argument \textit{main}
to ensure compatibility with earlier version of the package.
It must either be empty (|\childdocmain{}|)
or precisely match the filename of the main file in which it is specified.
See \secref{sec:detection} for further information.
\item
The filename \textit{main} must be specified without the |.tex| extension.
\item
The filename \textit{main} is case sensitive
(even in case-insensitive file systems)
due to internal string comparison.
\item
The argument \textit{main} should be fully expanded, it cannot be a macro.
\item
Subdirectories and special characters should be avoided in filenames.
\item
The command |\childdocmain{|\textit{main}|}| must be followed by a whitespace.
It should not be followed immediately by another command
or by a comment mark `|%|'.
This is because the \TeX{} parser reads the token immediately following
the argument of |\childdocmain| and puts it
at the beginning of every child section;
however, a white\-space is ignored.
\end{itemize}

%%%%%%%%%%%%%%%%%%%%%%%%%%%%%%%%%%%%%%%%
\paragraph{Content of Main File.}

It is advisable to place all content in the child files included by |\include|.
Any output contained in the main file will appear in all child documents
unless suppressed manually;
it cannot be suppressed automatically by the |\includeonly| directive
and thus should normally be avoided.
A method to include some content in the main file
by means of conditional processing is described in \secref{sec:conditional}.

%%%%%%%%%%%%%%%%%%%%%%%%%%%%%%%%%%%%%%%%
\paragraph{Page Numbering.}

When only a part of the document is compiled,
the appropriate numbering of pages
(as well as other status parameters)
is determined from the |.aux| files.
The latter contain information from previous passes.
However this information needs to propagate through
all intermediate child documents.
Therefore the page numbering in child documents may well
be inconsistent until the complete document is compiled at least once.

A useful (if unconventional) way to always ensure a consistent
page numbering is to restart the numbering in each child document
and denote the pages by `\textit{child}|.|\textit{page}'
where \textit{child} represents the chapter/section number of the child file.
This can be achieved by the command
|\numberwithin{page}{|\textit{child}|}|
of the \textsf{amsmath} package
where \textit{child} can be |chapter| or |section|
depending on the chosen structuring.
Alternatively, one can modify the macro |\thepage| appropriately
and reset the counter |page| at the start of each child file.

%%%%%%%%%%%%%%%%%%%%%%%%%%%%%%%%%%%%%%%%%%%%%%%%%%%%%%%%%%%%%%%%%%%%%%%%%%%%%%%%
\subsection{Conditional Processing}
\label{sec:conditional}

The package provides a mechanism to compile different versions
of a document. To customise the versions further some conditional processing
can come in handy to distinguish which version is being compiled.
The package provides two macros to describe the compilation context:

%%%%%%%%%%%%%%%%%%%%%%%%%%%%%%%%%%%%%%%%
\DescribeMacro{\ifchilddoc}
The conditional |\ifchilddoc| distinguishes between the compilation of
child documents and the main document:
%
\begin{center}
|\ifchilddoc |\textit{child-code}| |[|\||else |\textit{main-code}]| \||fi|
\end{center}

%%%%%%%%%%%%%%%%%%%%%%%%%%%%%%%%%%%%%%%%
\DescribeMacro{\childdocname}
\DescribeMacro{\childdocjob}
The macro |\childdocname| contains the filename (without extension)
of the main or child file being processed.
Note that |\childdocjob| will always contain the name of the main file.

%%%%%%%%%%%%%%%%%%%%%%%%%%%%%%%%%%%%%%%%
\paragraph{Title Page.}

Conditional processing can be used to include a title or banner page
in the main document when proper precautions are taken.
Importantly, the code in the main file should ensure that the page counter
(as well as other status parameters which are stored in the |.aux| files)
takes the same value after the conditional processing.
Otherwise the page numbers may take divergent values
depending on which part is compiled.

For example, a title page could be declared by:
%
\begin{center}
\begin{tabular}{l}
|\ifchilddoc\||else|\\
|\addtocounter{page}{-1}|\\
\textit{code for title page}\\
|\newpage|\\
|\||fi|
\end{tabular}
\end{center}
%
A banner page for the child documents can be generated by:
%
\begin{center}
\begin{tabular}{l}
|\ifchilddoc|\\
|\addtocounter{page}{-1}|\\
\textit{code for banner page}\\
|\newpage|\\
|\||fi|
\end{tabular}
\end{center}
%
Here one could write a message such as:
\begin{center}
|This is the part \childdocname{} of \childdocjob{}.|
\end{center}

%%%%%%%%%%%%%%%%%%%%%%%%%%%%%%%%%%%%%%%%%%%%%%%%%%%%%%%%%%%%%%%%%%%%%%%%%%%%%%%%
\subsection{Flags}
\label{sec:flags}

The package makes it easy to generate different versions
of the main or child documents.
To this end compilation flags can be defined
and assigned different default values.
They will be particularly useful in conjunction
with the forwarding mechanism described in \secref{sec:forward}.

For example, it may be useful to have a flag |\version|
which can be set to |draft| or |final|.
The document source will contain some conditional code
depending on the value of |\version|.
Suppose further, the flag should default to |final| for the main file
and to |draft| for child files
which is a natural assignment for editing the document.
This is achieved by placing the following code
in the preamble of the main document
(below the |\childdocmain| directive):
%
\begin{center}
\begin{tabular}{l}
|\ifchilddoc|\\
|\providecommand{\version}{draft}|\\
|\||else|\\
|\providecommand{\version}{final}|\\
|\||fi|
\end{tabular}
\end{center}
%
The definition by |\providecommand| makes sure
that previous definitions are not overwritten.
Further statements |\providecommand{\version}{...}|
can thus be added before the above code to override it.

For the main file, one might add a line
(between |\childdocmain| and the above block)
%
\begin{center}
|%\ifchilddoc\||else\providecommand{\version}{draft}\||fi|
\end{center}
%
which can be uncommented to produce a draft version.
Likewise one can add a line to the very top of a child file
(above the |\childdocof{|\textit{main}|}| directive)
%
\begin{center}
|%\providecommand{\version}{final}|
\end{center}
%
which can be uncommented to produce the final version of this child document.

%%%%%%%%%%%%%%%%%%%%%%%%%%%%%%%%%%%%%%%%%%%%%%%%%%%%%%%%%%%%%%%%%%%%%%%%%%%%%%%%
\subsection{Forwarding}
\label{sec:forward}

Different versions of the main or child documents
using compilation flags as described in \secref{sec:flags}
can be (permanently) stored in different files
for convenient compilation, viewing and distribution.
To this end, the package defines a command
to pass on compilation to a different file:

%%%%%%%%%%%%%%%%%%%%%%%%%%%%%%%%%%%%%%%%
\DescribeMacro{\childdocforward}
The command |\childdocforward| redirects processing to
another source file:
%
\begin{center}
\begin{tabular}{l}
|\input{childdoc.def}|\\
|\childdocforward[|\textit{main}|]{|\textit{dest}|}|\\
\end{tabular}
\end{center}
%
The argument \textit{dest} is the destination file
(without extension).
It should be the main file or one of the child files.
Note that further \textsf{childdoc} directives
such as |\childdocof| and |\childdocforward|
in the indicated file will be processed in this form.
The optional argument \textit{main}
passes on directly to the main file \textit{main}
while pretending to compile the child \textit{dest}.
This form behaves as if \textit{dest}
issues |\childdocof{|\textit{main}|}| right away,
and no further \textsf{childdoc} directives will be processed.

%%%%%%%%%%%%%%%%%%%%%%%%%%%%%%%%%%%%%%%%
\DescribeMacro{\...prefix}
In the alternative form |\childdocforwardprefix|,
%
\begin{center}
\begin{tabular}{l}
|\input{childdoc.def}|\\
|\childdocforwardprefix[|\textit{main}|]{|\textit{prefix}|}{|\textit{dest}|}|
\end{tabular}
\end{center}
%
the destination file is determined by a pattern
depending on the current file:
To make this work, the current file must be called
`{\textit{prefix}\hspace{0.2em}\textit{suffix}}'
with \textit{prefix} matching precisely the argument.
Processing is then passed on to the file
`{\textit{dest}\hspace{0.2em}\textit{suffix}}'.
Surely, the same effect is achieved by
directly specifying the
argument `{\textit{dest}\hspace{0.2em}\textit{suffix}}'
in the first form.
However, that requires to set up a different file
for each child. With the alternative form of the command
all these files can have exactly the same content
which simplifies setting them up and maintaining them.

For example, the following file |draft.tex|
with a compilation flag |\version| as described in \secref{sec:flags}
compiles the main document as a draft:
%
\begin{center}
\begin{tabular}{l}
|\def\version{draft}|\\
|\input{childdoc.def}|\\
|\childdocforward{|\textit{main}|}|
\end{tabular}
\end{center}
%
Likewise, the following files |final|\textit{nn}|.tex|
compile the final version of the child document
|child|\textit{nn}|.tex|:
%
\begin{center}
\begin{tabular}{l}
|\def\version{final}|\\
|\input{childdoc.def}|\\
|\childdocforwardprefix{final}{child}|
\end{tabular}
\end{center}
%

Note that when several versions of a main file and/or of each child file
are to be generated, it may be convenient to set up a |Makefile| or
shell script to automatise the process.

%%%%%%%%%%%%%%%%%%%%%%%%%%%%%%%%%%%%%%%%%%%%%%%%%%%%%%%%%%%%%%%%%%%%%%%%%%%%%%%%
\subsection{Command Line Processing}
\label{sec:commandline}

The effect of redirection files can also be achieved by invoking
the \LaTeX{} compiler with a more elaborate command line.
Most conveniently this should be done as part
of a shell script or a |Makefile|.

When using \textsf{childdoc} in the main file, the following
command lines effectively perform a redirection
(note that depending on the shell being used,
backslashes may have to be doubled: `|\|' $\to$ `|\\|'):
%
\begin{center}
|... -jobname "|\textit{target}|" |\\|"|[\textit{flags}]%
|\input{childdoc.def}\childdocforward[|\textit{main}|]{|\textit{dest}|}"|
\end{center}
%
Here \textit{target} is the name of the output file,
\textit{main} is the name of the main file
and \textit{dest} is the name of the main or child file to be processed
(all filenames without extensions).
The optional argument \textit{main} can be omitted
if \textit{main} matches \textit{dest}.
Optionally, compilation \textit{flags} can be defined via |\def| commands.
This command line makes the \TeX{} engine believe
it is compiling the file \textit{target}
whose content is specified as the latter parameter.
The provided code then forwards the processing to
\textit{main} or \textit{dest} as described in \secref{sec:forward}.

%%%%%%%%%%%%%%%%%%%%%%%%%%%%%%%%%%%%%%%%%%%%%%%%%%%%%%%%%%%%%%%%%%%%%%%%%%%%%%%%
\subsection{Include by Input}
\label{sec:input}

Including child documents by |\include| has some restrictions by design.
Most notably, the content of a child document always occupies
its own set of pages; pages cannot be shared between child documents.
Usually, this behaviour makes perfect sense
because each child document contain an essential part of the document.
However, in some situations it may be desirable to compose
a document from a collection of parts
without having mandatory page breaks between then.
For this case, the package
provides a mechanism to include parts
by |\input| which can also be processed individually.
However, by construction this mechanism
requires manual handling of the content to be output.

%%%%%%%%%%%%%%%%%%%%%%%%%%%%%%%%%%%%%%%%
\DescribeMacro{\ifchilddocmanual}
The main file should be prepared as usual, see \secref{sec:include}.
However, the document body must make a distinction
between processing of an individual part and of the main document, e.g.:
%
\begin{center}
\begin{tabular}{l}
|\ifchilddocmanual|\\
|\input{\childdocname}|\\
|\||else|\\
\textit{document body with }|\input{|\textit{part}|}|\\
|\||fi|
\end{tabular}
\end{center}
%
The conditional |\ifchilddocmanual| is true whenever
a part to be included by |\input| is being compiled,
and the name of the part is stored in |\childdocname|.

%%%%%%%%%%%%%%%%%%%%%%%%%%%%%%%%%%%%%%%%
\DescribeMacro{\childdocby}
Each part to be included by |\input| should start with:
%
\begin{center}
\begin{tabular}{l}
|\input{childdoc.def}|\\
|\childdocby{|\textit{main}|}|\\
\end{tabular}
\end{center}
%
The directive |\childdocby| is similar to |\childdocof|
described in \secref{sec:include},
but the subsequent selection of content must be done manually.
To that end, both |\ifchilddoc| and |\ifchilddocmanual|
will be true upon processing of a part,
and the name of the part is stored in |\childdocname|.
Note that |\jobname| will be set to the filename of the current part
so that each part receives an individual |.aux| file
that does not interfere with the |.aux| file(s) of the main document.
This behaviour can be altered by the alternative form
|\childdocby[*]{|\textit{main}|}| (with a non-empty optional argument)
which uses the |.aux| file of the main document
by setting |\jobname| to \textit{main}.

%%%%%%%%%%%%%%%%%%%%%%%%%%%%%%%%%%%%%%%%%%%%%%%%%%%%%%%%%%%%%%%%%%%%%%%%%%%%%%%%
\subsection{Driver Development}
\label{sec:driver}

The \textsf{childdoc} mechanism can also be use for the development
of definition files such as \LaTeX{} styles or classes.
This case differs from the above setup with multiple parts
included by |\include| in that no |\includeonly| should be invoked.
This can be achieved by starting the include file
(before |\ProvidesPackage|) with:
%
\begin{center}
\begin{tabular}{l}
|\input{childdoc.def}|\\
|\childdocforward{|\textit{main}|}|\\
\end{tabular}
\end{center}
%
or alternatively with:
%
\begin{center}
\begin{tabular}{l}
|\input{childdoc.def}|\\
|\childdocby{|\textit{main}|}|\\
\end{tabular}
\end{center}
%
Both forms have slightly different effects as described above.
The main file is prepared as usual, see \secref{sec:include}.

%%%%%%%%%%%%%%%%%%%%%%%%%%%%%%%%%%%%%%%%%%%%%%%%%%%%%%%%%%%%%%%%%%%%%%%%%%%%%%%%
\subsection{Legacy Detection}
\label{sec:detection}

The directive |\childdocmain| in the main file can detect
whether the complete document or merely a child is to be compiled
even without using the directive |\childdocof|.
This method is deprecated because it is less robust
and there is no compelling reason to use it;
it is merely provided for backward compatibility
and it may be removed in future versions.

If the detection mechanism is to be used,
it is mandatory to correctly specify
the filename of the main file as the argument of |\childdocmain|:
%
\begin{center}
\begin{tabular}{l}
|\input{childdoc.def}|\\
|\childdocmain{|\textit{main}|}|\\
\end{tabular}
\end{center}
%
If |\jobname| does not match the argument \textit{main} of |\childdocmain|,
it is assumed that |\jobname| points to the child file to be compiled.
When using |\childdocmain| with the main file specified as argument,
it suffices to start a child file
with just |\input{|\textit{main}|}|
without loading of the package and using |\childdocof|.
If instead all processing is done
with the appropriate \textsf{childdoc} directives,
the argument of \textit{main} of |\childdocmain| can be empty.

An alternative version of the command line processing described
in \secref{sec:commandline} using the detection mechanism reads:
%
\begin{center}
|... -jobname "|\textit{target}|" "|[\textit{flags}]%
[|\def\jobname{|\textit{dest}|}|]|\input{|\textit{main}|}"|
\end{center}

%%%%%%%%%%%%%%%%%%%%%%%%%%%%%%%%%%%%%%%%%%%%%%%%%%%%%%%%%%%%%%%%%%%%%%%%%%%%%%%%
\subsection{Manual Code}
\label{sec:manual}

In case one cannot be certain whether the definitions file |childdoc.def|
is installed on the target \TeX{} distribution
and one prefers not to ship it,
it is conceivable to paste a few relevant commands into the sources.

To that end, drop all statements |\input{childdoc.def}|
and perform the replacements as outlined below.
Instead of |\childdocmain{|\textit{main}|}| add the following code
to the top of the main file:
%
\begin{center}
\begin{tabular}{l}
|\||ifdefined\childdocname\endinput\||fi\newif\ifchilddoc|\\
|\edef\childdocname{\scantokens\expandafter{\jobname\noexpand}}|\\
|\def\childdocmain{|\textit{main}|}\||ifx\childdocmain\childdocname\||else|\\
|\childdoctrue\includeonly{\childdocname}\let\jobname\childdocmain\||fi|\\
\end{tabular}
\end{center}
%
Instead of |\childdocof{|\textit{main}|}| just include the main file
at the top of each child file:
%
\begin{center}
|\input{|\textit{main}|}|
\end{center}
%
A simple redirection |\childdocforward{|\textit{dest}|}| is achieved by:
%
\begin{center}
|\def\jobname{|\textit{dest}|}\input{\jobname}|
\end{center}
%
The redirection with prefix
|\childdocforwardprefix[|\textit{prefix}|]{|\textit{dest}|}|
is accomplished by:
%
\begin{center}
\begin{tabular}{l}
|{\edef\jobname{\scantokens\expandafter{\jobname\noexpand}}|\\
|\def\redirectjob |\textit{prefix}|#1~~~{\gdef\jobname{|\textit{dest}|#1}}|\\
|\expandafter\redirectjob\jobname~~~}\input{\jobname}|
\end{tabular}
\end{center}

In an alternative approach,
child documents can be compiled by a specific command line
without additional code or specific definitions:
%
\begin{center}
|... -jobname "|\textit{target}|" "|[\textit{flags}]%
|\includeonly{|\textit{dest}|}\input{|\textit{main}|}"|
\end{center}
%

%%%%%%%%%%%%%%%%%%%%%%%%%%%%%%%%%%%%%%%%%%%%%%%%%%%%%%%%%%%%%%%%%%%%%%%%%%%%%%%%
%%%%%%%%%%%%%%%%%%%%%%%%%%%%%%%%%%%%%%%%%%%%%%%%%%%%%%%%%%%%%%%%%%%%%%%%%%%%%%%%
\section{Information}

%%%%%%%%%%%%%%%%%%%%%%%%%%%%%%%%%%%%%%%%%%%%%%%%%%%%%%%%%%%%%%%%%%%%%%%%%%%%%%%%
\subsection{Copyright}

Copyright \copyright{} 2017--2018 Niklas Beisert

This work may be distributed and/or modified under the
conditions of the \LaTeX{} Project Public License, either version 1.3
of this license or (at your option) any later version.
The latest version of this license is in
  \url{http://www.latex-project.org/lppl.txt}
and version 1.3 or later is part of all distributions of \LaTeX{}
version 2005/12/01 or later.

This work has the LPPL maintenance status `maintained'.

The Current Maintainer of this work is Niklas Beisert.

This work consists of the files |README.txt|, |childdoc.ins| and |childdoc.dtx|
as well as the derived files |childdoc.def|, |cdocsamp.tex|
with |cdocsch1.tex|, |cdocsch2.tex|, |cdocspt3.tex|, |cdocspt4.tex|,
|cdocsdrf.tex|, |cdocsfn1.tex|, |cdocsfn2.tex|
as well as |childdoc.pdf|.

%%%%%%%%%%%%%%%%%%%%%%%%%%%%%%%%%%%%%%%%%%%%%%%%%%%%%%%%%%%%%%%%%%%%%%%%%%%%%%%%
\subsection{Files and Installation}

The package consists of the files:
%
\begin{center}
\begin{tabular}{ll}
    |README.txt|   & readme file \\
    |childdoc.ins| & installation file \\
    |childdoc.dtx| & source file \\
    |childdoc.def| & definition file \\
    |cdocsamp.tex| & sample main file \\
    |cdocsch1.tex| & sample include file \\
    |cdocsch2.tex| & sample include file \\
    |cdocspt3.tex| & sample part file \\
    |cdocspt4.tex| & sample part file \\
    |cdocsdrf.tex| & sample redirection file \\
    |cdocsfn1.tex| & sample redirection file \\
    |cdocsfn2.tex| & sample redirection file \\
    |childdoc.pdf| & manual
\end{tabular}
\end{center}
%
The distribution consists of the files
|README.txt|, |childdoc.ins| and |childdoc.dtx|.
%
\begin{itemize}
\item
Run (pdf)\LaTeX{} on |childdoc.dtx|
to compile the manual |childdoc.pdf| (this file).
\item
Run \LaTeX{} on |childdoc.ins| to create the definitions file |childdoc.def|
and the sample |cdocsamp.tex| with include files
|cdocsch1.tex|, |cdocsch2.tex|, |cdocspt3.tex|, |cdocspt4.tex|,
|cdocsdrf.tex|, |cdocsfn1.tex|, |cdocsfn2.tex|.
Then copy the file |childdoc.def| to an appropriate directory of your \LaTeX{}
distribution, e.g.\ \textit{texmf-root}|/tex/latex/childdoc|.
\end{itemize}

%%%%%%%%%%%%%%%%%%%%%%%%%%%%%%%%%%%%%%%%%%%%%%%%%%%%%%%%%%%%%%%%%%%%%%%%%%%%%%%%
\subsection{Related CTAN Packages}

There are several other packages which offer a similar functionality:
%
\begin{itemize}
\item
The packages
\href{http://ctan.org/pkg/docmute}{\textsf{docmute}},
\href{http://ctan.org/pkg/includex}{\textsf{includex}} and
\href{http://ctan.org/pkg/standalone}{\textsf{standalone}}
provide commands to include only the document body of
a child file thus allowing both files to be compiled individually.
\item
The packages \href{http://ctan.org/pkg/subdocs}{\textsf{subdocs}}
and \href{http://ctan.org/pkg/subfiles}{\textsf{subfiles}}
provide structures in which the main and child documents can be
encapsulated and allowing them to be compiled individually.
The inclusion mechanism is different from the conventional |\include|.
\item
The package \href{http://ctan.org/pkg/combine}{\textsf{combine}}
is an elaborate solution to combine several documents into one.
\end{itemize}
%
See also the CTAN topic \href{http://ctan.org/topic/subdocs}{\textsf{subdocs}}
for further related packages.
The present package differs from the above solutions in that
a document structure constructed with the conventional |\include| mechanism
just needs two extra commands at the top of every file
such that all constituent files can be compiled individually.

%%%%%%%%%%%%%%%%%%%%%%%%%%%%%%%%%%%%%%%%%%%%%%%%%%%%%%%%%%%%%%%%%%%%%%%%%%%%%%%%
%\subsection{Feature Suggestions}
%
%The following is a list of features which may be useful for future
%versions of this package:
%%
%\begin{itemize}
%\item
%\ldots
%\end{itemize}

%%%%%%%%%%%%%%%%%%%%%%%%%%%%%%%%%%%%%%%%%%%%%%%%%%%%%%%%%%%%%%%%%%%%%%%%%%%%%%%%
\subsection{Revision History}

%%%%%%%%%%%%%%%%%%%%%%%%%%%%%%%%%%%%%%%%
\paragraph{v2.0:} 2018/12/30

\begin{itemize}
\item
immediate forward processing
\item
added |\childdocby| mechanism
\item
manual restructured
\end{itemize}

%%%%%%%%%%%%%%%%%%%%%%%%%%%%%%%%%%%%%%%%
\paragraph{v1.6:} 2018/01/17

\begin{itemize}
\item
application for development of include files
\item
corrections to manual
\end{itemize}

%%%%%%%%%%%%%%%%%%%%%%%%%%%%%%%%%%%%%%%%
\paragraph{v1.5:} 2017/05/21

\begin{itemize}
\item
more complete structuring introduced
\item
|\childdocof| introduced
\item
|\childdoc| renamed to |\childdocmain|
\item
|\childredirect| renamed to |\childdocforward| and |\childdocforwardprefix|
and functionality expanded
\end{itemize}

%%%%%%%%%%%%%%%%%%%%%%%%%%%%%%%%%%%%%%%%
\paragraph{v1.0:} 2017/04/27

\begin{itemize}
\item
manual and install package
\item
first version published on CTAN
\end{itemize}

%%%%%%%%%%%%%%%%%%%%%%%%%%%%%%%%%%%%%%%%
\paragraph{v0.6:} 2017/04/26

\begin{itemize}
\item
redirection mechanism added
\end{itemize}

%%%%%%%%%%%%%%%%%%%%%%%%%%%%%%%%%%%%%%%%
\paragraph{v0.5:} 2017/04/26

\begin{itemize}
\item
functionality in definition file
\end{itemize}


%%%%%%%%%%%%%%%%%%%%%%%%%%%%%%%%%%%%%%%%%%%%%%%%%%%%%%%%%%%%%%%%%%%%%%%%%%%%%%%%
%%%%%%%%%%%%%%%%%%%%%%%%%%%%%%%%%%%%%%%%%%%%%%%%%%%%%%%%%%%%%%%%%%%%%%%%%%%%%%%%
%%%%%%%%%%%%%%%%%%%%%%%%%%%%%%%%%%%%%%%%%%%%%%%%%%%%%%%%%%%%%%%%%%%%%%%%%%%%%%%%
\appendix

\settowidth\MacroIndent{\rmfamily\scriptsize 000\ }

 \DocInput{childdoc.dtx}

\end{document}
%</driver>
% \fi
%
% %%%%%%%%%%%%%%%%%%%%%%%%%%%%%%%%%%%%%%%%%%%%%%%%%%%%%%%%%%%%%%%%%%%%%%%%%%%%%%
% %%%%%%%%%%%%%%%%%%%%%%%%%%%%%%%%%%%%%%%%%%%%%%%%%%%%%%%%%%%%%%%%%%%%%%%%%%%%%%
% \section{Sample}
%\iffalse
%<*samplemain>
%\fi
%
% The following presents a sample document
% with two chapters, two parts, a title page,
% a compile flag as well as three forwarding files to set the flag.
% It consists of eight |.tex| files:
% \begin{center}
% \begin{tabular}{ll}
% |cdocsamp.tex|&main file\\
% |cdocsch1.tex|&include file for chapter 1\\
% |cdocsch2.tex|&include file for chapter 2\\
% |cdocspt3.tex|&include file for part 3\\
% |cdocspt4.tex|&include file for part 4\\
% |cdocsdrf.tex|&forwarding file for main file in draft mode\\
% |cdocsfi1.tex|&forwarding file for final version of chapter 1\\
% |cdocsfi2.tex|&forwarding file for final version of chapter 2\\
% \end{tabular}
% \end{center}
% Each of the eight files can be compiled directly by the \LaTeX{} compiler.
%
% %%%%%%%%%%%%%%%%%%%%%%%%%%%%%%%%%%%%%%
% \paragraph{Main File.}
%
% The main file is called |cdocsamp.tex|.
%
% Load the \textsf{childdoc} definitions and
% declare the filename for the main document:
%    \begin{macrocode}
\input{childdoc.def}
\childdocmain{}
%    \end{macrocode}

% Optional override for |\version| flag:
%    \begin{macrocode}
%%\ifchilddoc\else\providecommand{\version}{draft}\fi
%    \end{macrocode}

% Define the default values for the |\version| flag
% (|final| for the main file and |draft| for childs):
%    \begin{macrocode}
\ifchilddoc
\providecommand{\version}{draft}
\else
\providecommand{\version}{final}
\fi
%    \end{macrocode}

% Load the standard document class:
%    \begin{macrocode}
\documentclass[12pt]{article}
%    \end{macrocode}

% Start the document body:
%    \begin{macrocode}
\begin{document}
%    \end{macrocode}

% Declare a title page.
% Print title, part of document being processed and version flag:
%    \begin{macrocode}
\addtocounter{page}{-1}
\begin{center}
{\LARGE\bfseries{}childdoc example\par}
\vspace{1cm}
\ifchilddoc
\ifchilddocmanual part\else chapter\fi:
`\childdocname' of `\childdocjob'\par
\else
main document: `\childdocjob'\par
\fi
version: \version\par
\end{center}
\newpage
%    \end{macrocode}

% Manually include selected file,
% otherwise process as usual:
%    \begin{macrocode}
\ifchilddocmanual
\section*{part `\childdocname'}
\input{\childdocname}
\else
%    \end{macrocode}

% Include the two chapters:
%    \begin{macrocode}
\include{cdocsch1}
\include{cdocsch2}
%    \end{macrocode}

% Include the two parts unless only chapters should be displayed:
%    \begin{macrocode}
\ifchilddoc\else
\section{part three}
\input{cdocspt3}
\section{part four}
\input{cdocspt4}
\fi
%    \end{macrocode}

% Process as usual until here:
%    \begin{macrocode}
\fi
%    \end{macrocode}

% End of document body:
%    \begin{macrocode}
\end{document}
%    \end{macrocode}
%\iffalse
%</samplemain>
%\fi
%
% %%%%%%%%%%%%%%%%%%%%%%%%%%%%%%%%%%%%%%
% \paragraph{Chapter Include Files.}
%
% The include files are called |cdocsch1.tex| and |cdocsch2.tex|.
%
%\iffalse
%<*samplechap1|samplechap2>
%\fi

% Optional override for |\version| flag:
%    \begin{macrocode}
%%\providecommand{\version}{final}
%    \end{macrocode}

% Include the main document:
%    \begin{macrocode}
\input{childdoc.def}
\childdocof{cdocsamp}
%    \end{macrocode}

%\iffalse
%</samplechap1|samplechap2>
%\fi
%
%\iffalse
%<*samplechap1>
%\fi
% Some text for chapter 1:
%    \begin{macrocode}
\section{one}
some text in chapter one
%    \end{macrocode}

%\iffalse
%</samplechap1>
%\fi
% Some text for chapter 2:
%\iffalse
%<*samplechap2>
%\fi
%    \begin{macrocode}
\section{two}
more text in chapter two
%    \end{macrocode}

%\iffalse
%</samplechap2>
%\fi
%
% %%%%%%%%%%%%%%%%%%%%%%%%%%%%%%%%%%%%%%
% \paragraph{Part Include Files.}
%
% The include files are called |cdocspt3.tex| and |cdocspt4.tex|.
%
%\iffalse
%<*samplepart3|samplepart4>
%\fi

% Optional override for |\version| flag:
%    \begin{macrocode}
%%\providecommand{\version}{final}
%    \end{macrocode}

% Include the main document:
%    \begin{macrocode}
\input{childdoc.def}
\childdocby{cdocsamp}
%    \end{macrocode}

%\iffalse
%</samplepart3|samplepart4>
%\fi
%
%\iffalse
%<*samplepart3>
%\fi
% Some text for part 3:
%    \begin{macrocode}
some text in part three
%    \end{macrocode}

%\iffalse
%</samplepart3>
%\fi
% Some text for part 4:
%\iffalse
%<*samplepart4>
%\fi
%    \begin{macrocode}
more text in part four
%    \end{macrocode}

%\iffalse
%</samplepart4>
%\fi
%
% %%%%%%%%%%%%%%%%%%%%%%%%%%%%%%%%%%%%%%
% \paragraph{Forwarding for a Complete Draft.}
%
% The following forwarding file |cdocsdrf.tex|
% compiles the main document in draft mode:
%\iffalse
%<*sampledraft>
%\fi
%    \begin{macrocode}
\def\version{draft}
\input{childdoc.def}
\childdocforward{cdocsamp}
%    \end{macrocode}

%\iffalse
%</sampledraft>
%\fi
%
% %%%%%%%%%%%%%%%%%%%%%%%%%%%%%%%%%%%%%%
% \paragraph{Forwarding for Final Version of the Chapters.}
%
% The following forwarding files |cdocsfn1.tex| and |cdocsfn2.tex|
% (with identical content)
% compile the final versions of the child documents
% |cdocsch1.tex| and |cdocsch2.tex|, respectively:
%\iffalse
%<*samplefinal>
%\fi
%    \begin{macrocode}
\def\version{final}
\input{childdoc.def}
\childdocforwardprefix[cdocsamp]{cdocsfn}{cdocsch}
%    \end{macrocode}

%\iffalse
%</samplefinal>
%\fi
%
% %%%%%%%%%%%%%%%%%%%%%%%%%%%%%%%%%%%%%%
% \paragraph{Command Line Processing.}
%
% The following three command lines generate the output files
% |cdocscld|, |cdocscl1| and |cdocscl2|
% which should be identical to
% |cdocsdrf|, |cdocsch1| and |cdocsfn2|, respectively:
% \begin{center}
% \begin{tabular}{l}
% |latex -jobname cdocscld \|\\
% |  "\def\version{draft}\input{childdoc.def}\childdocforward{cdocsamp}"|\\
% |latex -jobname cdocscl1 \|\\
% |  "\input{childdoc.def}\childdocforward[cdocsamp]{cdocsch1}"|\\
% |latex -jobname cdocscl2 \|\\
% |  "\def\version{final}\input{childdoc.def}\childdocforward{cdocsch2}"|
% \end{tabular}
% \end{center}
% Note that the trailing backslash on each first line
% merely continues the input to the second line
% (for convenient cut ant paste).
% Furthermore, the command |latex| can be replaced by any
% of its alternative versions such as |pdflatex|.
%
% %%%%%%%%%%%%%%%%%%%%%%%%%%%%%%%%%%%%%%%%%%%%%%%%%%%%%%%%%%%%%%%%%%%%%%%%%%%%%%
% %%%%%%%%%%%%%%%%%%%%%%%%%%%%%%%%%%%%%%%%%%%%%%%%%%%%%%%%%%%%%%%%%%%%%%%%%%%%%%
% \section{Implementation}
%\iffalse
%<*package>
%\fi
%
% This section describes the definitions file |childdoc.def|.

% The definitions cannot be loaded using |\usepackage| or |\RequirePackage|
% which has a mechanism to prevent loading a style file more than once.
% When loading the definitions by means of |\input|
% multiple instances have to be prevented manually:
%\iffalse
%This code needs to be before the `\ProvidesFile' directive
%which is defined at the beginning of this file.
%Therefore it is also placed there and commented out here.
%</package>
%<*discard>
%\fi
%    \begin{macrocode}
\ifdefined\childdocmain\endinput\fi
%    \end{macrocode}
%\iffalse
%</discard>
%<*package>
%\fi
%
% \macro{\ifchilddoc}
% \macro{\ifchilddocmanual}
% The conditional |\ifchilddoc| tells whether a
% child (true) or main (false) document is being compiled.
% The conditional |\ifchilddocmanual| tells whether
% the |\includeonly| mechanism is used (false) or
% the selection of child files must be performed manually (true).
% The definitions initialise to false:
%    \begin{macrocode}
\newif\ifchilddoc
\newif\ifchilddocmanual
%    \end{macrocode}

% \macro{\childdocname}
% \macro{\childdocjob}
% The macro |\childdocname| stores the name of the main document
% to be compiled. The macro |\childdocjob| stores the name of
% the document on which the \LaTeX{} compiler was originally invoked.
% The content of |\jobname| cannot be compared
% to filenames specified in the source due to different catcodes.
% The following code rescans |\jobname|, stores the result
% in |\childdocname| and saves a copy in |\childdocjob|:
%    \begin{macrocode}
\edef\childdocname{\scantokens\expandafter{\jobname\noexpand}}
\let\childdocjob\childdocname
%    \end{macrocode}

% \macro{\childdocdisable}
% The macro |\childdocdisable| prevents the main file
% from being processed more than once.
% At this stage, the main document command |\childdocmain|
% is assumed to be called once again where it should do nothing.
% Any subsequent call to it should prevent
% a secondary processing of the main document
% It overwrites the forwarding commands
% |\childdocof| and |\childdocforward|
% with empty macros to prevent further inclusions of the main document:
%    \begin{macrocode}
\newcommand{\childdocdisable}
{
  \renewcommand{\childdocmain}[1]{\renewcommand{\childdocmain}[1]{\endinput}}
  \renewcommand{\childdocof}[1]{}
  \renewcommand{\childdocby}[2][]{}
  \renewcommand{\childdocforward}[2][]{}
  \renewcommand{\childdocdisable}{}
}
%    \end{macrocode}

% \macro{\childdocmain}
% The macro |\childdocmain| is to be called at the top of the main file
% with nothing or the main filename (without extension) as argument.
% First, it breaks loops.
% If the argument is not empty and does not match |\childdocname|
% (which is set by the first inclusion of |childdoc.def|),
% |\ifchilddoc| is set to true, |\includeonly| is applied to the child file
% and |\jobname| is set to the main file
% (for proper handling of |.aux| files):
%    \begin{macrocode}
\newcommand{\childdocmain}[1]
{
  \childdocdisable\childdocmain{}
  \if?#1?\else
    \begingroup
      \def\childdoctmp{#1}
      \ifx\childdoctmp\childdocname
        \def\childdoctmp{}
      \else
        \def\childdoctmp
        {
          \childdoctrue
          \includeonly{\childdocname}
          \def\childdocjob{#1}
          \def\jobname{#1}
        }
      \fi
      \expandafter
    \endgroup
    \childdoctmp
  \fi
}
%    \end{macrocode}

% \macro{\childdocof}
% The command |\childdocof| redirects
% compilation to the main file |#1|.
%    \begin{macrocode}
\newcommand{\childdocof}[1]
{
  \childdocdisable
  \childdoctrue
  \includeonly{\childdocname}
  \def\jobname{#1}
  \def\childdocjob{#1}
  \input{#1}
}
%    \end{macrocode}

% \macro{\childdocby}
% The command |\childdocby| ....
%    \begin{macrocode}
\newcommand{\childdocby}[2][]
{
  \childdocdisable
  \childdoctrue
  \childdocmanualtrue
  \if?#1?\else
    \def\jobname{#2}
  \fi
  \def\childdocjob{#2}
  \input{#2}
  \endinput
}
%    \end{macrocode}

% \macro{\childdocforward}
% The command |\childdocforward| redirects
% compilation to the main file or
% (if the optional argument is given) a child file.
% Parameters are set as if the main file
% or a child file starting with |\childdocof| was compiled.
% Then compilation is handed over to the main file:
%    \begin{macrocode}
\newcommand{\childdocforward}[2][]
{
  \begingroup
    \if?#1?
      \def\childdoctmp
      {
        \def\childdocname{#2}
        \def\childdocjob{#2}
        \def\jobname{#2}
        \input{#2}
        \endinput
      }
    \else
      \def\childdoctmp
      {
        \childdocdisable
        \def\childdocname{#2}
        \childdoctrue
        \includeonly{#2}
        \def\childdocjob{#1}
        \def\jobname{#1}
        \input{#1}
        \endinput
      }
    \fi
    \expandafter
  \endgroup
  \childdoctmp
}
%    \end{macrocode}

% \macro{\childdocforwardprefix}
% The command |\childdocforwardprefix| redirects
% compilation to the main or a child file by means of a pattern.
% The prefix |#1| in the current filename is replaced by |#2|
% and the suffix of the current filename is kept
% (it is assumed that the filename does not contain the substring `|~~~|'
% which is used as a delimiter).
% Compilation is handed over to the new file by |\childdocforward|:
%    \begin{macrocode}
\newcommand{\childdocforwardprefix}[3][]
{
  \begingroup
    \def\childdocextract #2##1~~~{\def\childdoctmp{\childdocforward[#1]{#3##1}}}
    \expandafter\childdocextract\childdocname~~~
    \expandafter
  \endgroup
  \childdoctmp
}
%    \end{macrocode}

% \macro{\childdoc}
% The deprecated macro |\childdoc| is a legacy version of |\childdocmain|:
%    \begin{macrocode}
\newcommand{\childdoc}{\childdocmain}
%    \end{macrocode}

% \macro{\childdocredirect}
% The deprecated macro |\childdocredirect| is a legacy version
% of |\childdocforward| and |\childdocforwardprefix|:
%    \begin{macrocode}
\newcommand{\childdocredirect}[2][]
{
  \begingroup
    \if?#1?
      \def\childdoctmp{\childdocforward{#2}}
    \else
      \def\childdoctmp{\childdocforwardprefix{#1}{#2}}
    \fi
    \expandafter
  \endgroup
  \childdoctmp
}
%    \end{macrocode}

%\iffalse
%</package>
%\fi
%
\endinput

\childdocof{cdocsamp}
%    \end{macrocode}

%\iffalse
%</samplechap1|samplechap2>
%\fi
%
%\iffalse
%<*samplechap1>
%\fi
% Some text for chapter 1:
%    \begin{macrocode}
\section{one}
some text in chapter one
%    \end{macrocode}

%\iffalse
%</samplechap1>
%\fi
% Some text for chapter 2:
%\iffalse
%<*samplechap2>
%\fi
%    \begin{macrocode}
\section{two}
more text in chapter two
%    \end{macrocode}

%\iffalse
%</samplechap2>
%\fi
%
% %%%%%%%%%%%%%%%%%%%%%%%%%%%%%%%%%%%%%%
% \paragraph{Part Include Files.}
%
% The include files are called |cdocspt3.tex| and |cdocspt4.tex|.
%
%\iffalse
%<*samplepart3|samplepart4>
%\fi

% Optional override for |\version| flag:
%    \begin{macrocode}
%%\providecommand{\version}{final}
%    \end{macrocode}

% Include the main document:
%    \begin{macrocode}
% \iffalse
%
% childdoc.dtx Copyright (C) 2017-2018 Niklas Beisert
%
% This work may be distributed and/or modified under the
% conditions of the LaTeX Project Public License, either version 1.3
% of this license or (at your option) any later version.
% The latest version of this license is in
%   http://www.latex-project.org/lppl.txt
% and version 1.3 or later is part of all distributions of LaTeX
% version 2005/12/01 or later.
%
% This work has the LPPL maintenance status `maintained'.
%
% The Current Maintainer of this work is Niklas Beisert.
%
% This work consists of the files childdoc.dtx and childdoc.ins
% and the derived files childdoc.def and cdocsamp.tex with
% cdocsch1.tex, cdocsch2.tex, cdocsdrf.tex, cdocsfn1.tex, cdocsfn2.tex.
%
%<package>\ifdefined\childdocmain\endinput\fi
%<package>\ProvidesFile{childdoc.def}[2018/12/30 v2.0 child document driver]
%<samplemain>\ProvidesFile{cdocsamp.tex}[2018/12/30 v2.0 sample for childdoc]
%<*driver>
%\ProvidesFile{childdoc.drv}[2018/12/30 v2.0 childdoc reference manual file]
\PassOptionsToClass{10pt,a4paper}{article}
\documentclass{ltxdoc}

\usepackage[margin=35mm]{geometry}
\usepackage{hyperref}
\usepackage{hyperxmp}
\usepackage[usenames]{color}

\hypersetup{colorlinks=true}
\hypersetup{pdfstartview=FitH}
\hypersetup{pdfpagemode=UseNone}
\hypersetup{pdfsource={}}
\hypersetup{pdflang={en-UK}}
\hypersetup{pdfcopyright={Copyright 2017-2018 Niklas Beisert.
  This work may be distributed and/or modified under the
  conditions of the LaTeX Project Public License, either version 1.3
  of this license or (at your option) any later version.}}
\hypersetup{pdflicenseurl={http://www.latex-project.org/lppl.txt}}
\hypersetup{pdfcontactaddress={ETH Zurich, ITP, HIT K,
  Wolfgang-Pauli-Strasse 27}}
\hypersetup{pdfcontactpostcode={8093}}
\hypersetup{pdfcontactcity={Zurich}}
\hypersetup{pdfcontactcountry={Switzerland}}
\hypersetup{pdfcontactemail={nbeisert@itp.phys.ethz.ch}}
\hypersetup{pdfcontacturl={http://people.phys.ethz.ch/\xmptilde nbeisert/}}

\newcommand{\secref}[1]{\hyperref[#1]{section \ref*{#1}}}

\parskip1ex
\parindent0pt
\let\olditemize\itemize
\def\itemize{\olditemize\parskip0pt}

\begin{document}

\title{The \textsf{childdoc} Package}
\hypersetup{pdftitle={The childdoc Package}}
\author{Niklas Beisert\\[2ex]
  Institut f\"ur Theoretische Physik\\
  Eidgen\"ossische Technische Hochschule Z\"urich\\
  Wolfgang-Pauli-Strasse 27, 8093 Z\"urich, Switzerland\\[1ex]
  \href{mailto:nbeisert@itp.phys.ethz.ch}
  {\texttt{nbeisert@itp.phys.ethz.ch}}}
\hypersetup{pdfauthor={Niklas Beisert}}
\hypersetup{pdfsubject={Manual for the LaTeX2e Package childdoc}}
\date{30 December 2018, \textsf{v2.0}}
\maketitle

\begin{abstract}\noindent
\textsf{childdoc} is a \LaTeXe{} package
that enables the direct compilation
of document sections included by |\include|
to individual files.
\end{abstract}

\begingroup
\parskip0ex
\tableofcontents
\endgroup

%%%%%%%%%%%%%%%%%%%%%%%%%%%%%%%%%%%%%%%%%%%%%%%%%%%%%%%%%%%%%%%%%%%%%%%%%%%%%%%%
%%%%%%%%%%%%%%%%%%%%%%%%%%%%%%%%%%%%%%%%%%%%%%%%%%%%%%%%%%%%%%%%%%%%%%%%%%%%%%%%
\section{Introduction}

\LaTeX{} provides a mechanism to structure a large document (such as a book)
into a main file and several child files (containing the chapters)
using the |\include| command.
This mechanism is beneficial for documents
which span hundreds of pages in order to
make the source file(s) more manageable.
Moreover, compilation can be restricted to
selected child files by means of the |\includeonly| command.
The latter feature can be used to reduce the compilation time while editing
(this was significantly more useful in the earlier days of \LaTeX{})
or to generate a smaller document which is easier to navigate.
Another application of |\includeonly| is to generate
documents consisting of selected parts of the complete document.

However, there are a few drawbacks of the plain |\include| mechanism:
\begin{itemize}
\item
The child files cannot be compiled on their own,
they can only be compiled via the main file.
A naive editing environment
(such as a text editor with an option
to have the current file processed by \LaTeX)
may require one to switch to the main file before compiling;
attempting to compile the child file produces errors.
\item
The main file must be modified (each time)
to adjust the |\includeonly| command
to the present needs. This easily leaves the main file in a messy state.
\item
The generated document will always carry the filename
of the main document. This is inconvenient if
several child files are to be compiled and
to be kept for distribution.
\end{itemize}

The present package provides a simple interface
to make child files individually compilable by \LaTeX{}.
Compiling a child file then has the same effect as compiling
the main file with an |\includeonly| command
to select the appropriate child.
Moreover the generated document will carry the name of the child
rather than the main file.
This resolves all three above issues.

This feature is meant to make the editing of books,
thesis documents and lecture notes somewhat more convenient.
However, the package can also be used efficiently for
composing a series of documents (such as exercise sheets)
which are typically distributed individually.
It then assists the author in generating the individual documents
(potentially in different versions)
as well as a document containing the collected series.
Another application is in developing style files
or other kinds of included material
where compilation of the style file could redirect
to a sample or test file.

%%%%%%%%%%%%%%%%%%%%%%%%%%%%%%%%%%%%%%%%%%%%%%%%%%%%%%%%%%%%%%%%%%%%%%%%%%%%%%%%
%%%%%%%%%%%%%%%%%%%%%%%%%%%%%%%%%%%%%%%%%%%%%%%%%%%%%%%%%%%%%%%%%%%%%%%%%%%%%%%%
\section{Usage}

First of all, the package \textsf{childdoc} is \emph{not} a standard
\LaTeXe{} |.sty| style file! Therefore it needs to be invoked in
a non-standard way.

%%%%%%%%%%%%%%%%%%%%%%%%%%%%%%%%%%%%%%%%%%%%%%%%%%%%%%%%%%%%%%%%%%%%%%%%%%%%%%%%
\subsection{Included Files}
\label{sec:include}

%%%%%%%%%%%%%%%%%%%%%%%%%%%%%%%%%%%%%%%%
\DescribeMacro{\childdocmain}
To use the package, add the commands
\begin{center}
\begin{tabular}{l}
|\input{childdoc.def}|\\
|\childdocmain{}|\\
\end{tabular}
\end{center}
at the very top of the main \LaTeX{} file,
in particular \emph{before} the |\documentclass| statement!
The argument of |\childdocmain| should be left empty
(but it must be present).

%%%%%%%%%%%%%%%%%%%%%%%%%%%%%%%%%%%%%%%%
\DescribeMacro{\childdocof}
Furthermore, add the commands
\begin{center}
\begin{tabular}{l}
|\input{childdoc.def}|\\
|\childdocof{|\textit{main}|}|\\
\end{tabular}
\end{center}
at the top of every child file \textit{child}
which is included by |\include{|\textit{child}|}|
from within the main file
(or at least for those files to be compiled individually).
The argument \textit{main} must be the filename of the main file.

There are a couple of
considerations in setting up the main and child documents:

%%%%%%%%%%%%%%%%%%%%%%%%%%%%%%%%%%%%%%%%
\paragraph{Restrictions.}

Please note the following restrictions:
\begin{itemize}
\item
|\childdocmain| must be called with one argument \textit{main}
to ensure compatibility with earlier version of the package.
It must either be empty (|\childdocmain{}|)
or precisely match the filename of the main file in which it is specified.
See \secref{sec:detection} for further information.
\item
The filename \textit{main} must be specified without the |.tex| extension.
\item
The filename \textit{main} is case sensitive
(even in case-insensitive file systems)
due to internal string comparison.
\item
The argument \textit{main} should be fully expanded, it cannot be a macro.
\item
Subdirectories and special characters should be avoided in filenames.
\item
The command |\childdocmain{|\textit{main}|}| must be followed by a whitespace.
It should not be followed immediately by another command
or by a comment mark `|%|'.
This is because the \TeX{} parser reads the token immediately following
the argument of |\childdocmain| and puts it
at the beginning of every child section;
however, a white\-space is ignored.
\end{itemize}

%%%%%%%%%%%%%%%%%%%%%%%%%%%%%%%%%%%%%%%%
\paragraph{Content of Main File.}

It is advisable to place all content in the child files included by |\include|.
Any output contained in the main file will appear in all child documents
unless suppressed manually;
it cannot be suppressed automatically by the |\includeonly| directive
and thus should normally be avoided.
A method to include some content in the main file
by means of conditional processing is described in \secref{sec:conditional}.

%%%%%%%%%%%%%%%%%%%%%%%%%%%%%%%%%%%%%%%%
\paragraph{Page Numbering.}

When only a part of the document is compiled,
the appropriate numbering of pages
(as well as other status parameters)
is determined from the |.aux| files.
The latter contain information from previous passes.
However this information needs to propagate through
all intermediate child documents.
Therefore the page numbering in child documents may well
be inconsistent until the complete document is compiled at least once.

A useful (if unconventional) way to always ensure a consistent
page numbering is to restart the numbering in each child document
and denote the pages by `\textit{child}|.|\textit{page}'
where \textit{child} represents the chapter/section number of the child file.
This can be achieved by the command
|\numberwithin{page}{|\textit{child}|}|
of the \textsf{amsmath} package
where \textit{child} can be |chapter| or |section|
depending on the chosen structuring.
Alternatively, one can modify the macro |\thepage| appropriately
and reset the counter |page| at the start of each child file.

%%%%%%%%%%%%%%%%%%%%%%%%%%%%%%%%%%%%%%%%%%%%%%%%%%%%%%%%%%%%%%%%%%%%%%%%%%%%%%%%
\subsection{Conditional Processing}
\label{sec:conditional}

The package provides a mechanism to compile different versions
of a document. To customise the versions further some conditional processing
can come in handy to distinguish which version is being compiled.
The package provides two macros to describe the compilation context:

%%%%%%%%%%%%%%%%%%%%%%%%%%%%%%%%%%%%%%%%
\DescribeMacro{\ifchilddoc}
The conditional |\ifchilddoc| distinguishes between the compilation of
child documents and the main document:
%
\begin{center}
|\ifchilddoc |\textit{child-code}| |[|\||else |\textit{main-code}]| \||fi|
\end{center}

%%%%%%%%%%%%%%%%%%%%%%%%%%%%%%%%%%%%%%%%
\DescribeMacro{\childdocname}
\DescribeMacro{\childdocjob}
The macro |\childdocname| contains the filename (without extension)
of the main or child file being processed.
Note that |\childdocjob| will always contain the name of the main file.

%%%%%%%%%%%%%%%%%%%%%%%%%%%%%%%%%%%%%%%%
\paragraph{Title Page.}

Conditional processing can be used to include a title or banner page
in the main document when proper precautions are taken.
Importantly, the code in the main file should ensure that the page counter
(as well as other status parameters which are stored in the |.aux| files)
takes the same value after the conditional processing.
Otherwise the page numbers may take divergent values
depending on which part is compiled.

For example, a title page could be declared by:
%
\begin{center}
\begin{tabular}{l}
|\ifchilddoc\||else|\\
|\addtocounter{page}{-1}|\\
\textit{code for title page}\\
|\newpage|\\
|\||fi|
\end{tabular}
\end{center}
%
A banner page for the child documents can be generated by:
%
\begin{center}
\begin{tabular}{l}
|\ifchilddoc|\\
|\addtocounter{page}{-1}|\\
\textit{code for banner page}\\
|\newpage|\\
|\||fi|
\end{tabular}
\end{center}
%
Here one could write a message such as:
\begin{center}
|This is the part \childdocname{} of \childdocjob{}.|
\end{center}

%%%%%%%%%%%%%%%%%%%%%%%%%%%%%%%%%%%%%%%%%%%%%%%%%%%%%%%%%%%%%%%%%%%%%%%%%%%%%%%%
\subsection{Flags}
\label{sec:flags}

The package makes it easy to generate different versions
of the main or child documents.
To this end compilation flags can be defined
and assigned different default values.
They will be particularly useful in conjunction
with the forwarding mechanism described in \secref{sec:forward}.

For example, it may be useful to have a flag |\version|
which can be set to |draft| or |final|.
The document source will contain some conditional code
depending on the value of |\version|.
Suppose further, the flag should default to |final| for the main file
and to |draft| for child files
which is a natural assignment for editing the document.
This is achieved by placing the following code
in the preamble of the main document
(below the |\childdocmain| directive):
%
\begin{center}
\begin{tabular}{l}
|\ifchilddoc|\\
|\providecommand{\version}{draft}|\\
|\||else|\\
|\providecommand{\version}{final}|\\
|\||fi|
\end{tabular}
\end{center}
%
The definition by |\providecommand| makes sure
that previous definitions are not overwritten.
Further statements |\providecommand{\version}{...}|
can thus be added before the above code to override it.

For the main file, one might add a line
(between |\childdocmain| and the above block)
%
\begin{center}
|%\ifchilddoc\||else\providecommand{\version}{draft}\||fi|
\end{center}
%
which can be uncommented to produce a draft version.
Likewise one can add a line to the very top of a child file
(above the |\childdocof{|\textit{main}|}| directive)
%
\begin{center}
|%\providecommand{\version}{final}|
\end{center}
%
which can be uncommented to produce the final version of this child document.

%%%%%%%%%%%%%%%%%%%%%%%%%%%%%%%%%%%%%%%%%%%%%%%%%%%%%%%%%%%%%%%%%%%%%%%%%%%%%%%%
\subsection{Forwarding}
\label{sec:forward}

Different versions of the main or child documents
using compilation flags as described in \secref{sec:flags}
can be (permanently) stored in different files
for convenient compilation, viewing and distribution.
To this end, the package defines a command
to pass on compilation to a different file:

%%%%%%%%%%%%%%%%%%%%%%%%%%%%%%%%%%%%%%%%
\DescribeMacro{\childdocforward}
The command |\childdocforward| redirects processing to
another source file:
%
\begin{center}
\begin{tabular}{l}
|\input{childdoc.def}|\\
|\childdocforward[|\textit{main}|]{|\textit{dest}|}|\\
\end{tabular}
\end{center}
%
The argument \textit{dest} is the destination file
(without extension).
It should be the main file or one of the child files.
Note that further \textsf{childdoc} directives
such as |\childdocof| and |\childdocforward|
in the indicated file will be processed in this form.
The optional argument \textit{main}
passes on directly to the main file \textit{main}
while pretending to compile the child \textit{dest}.
This form behaves as if \textit{dest}
issues |\childdocof{|\textit{main}|}| right away,
and no further \textsf{childdoc} directives will be processed.

%%%%%%%%%%%%%%%%%%%%%%%%%%%%%%%%%%%%%%%%
\DescribeMacro{\...prefix}
In the alternative form |\childdocforwardprefix|,
%
\begin{center}
\begin{tabular}{l}
|\input{childdoc.def}|\\
|\childdocforwardprefix[|\textit{main}|]{|\textit{prefix}|}{|\textit{dest}|}|
\end{tabular}
\end{center}
%
the destination file is determined by a pattern
depending on the current file:
To make this work, the current file must be called
`{\textit{prefix}\hspace{0.2em}\textit{suffix}}'
with \textit{prefix} matching precisely the argument.
Processing is then passed on to the file
`{\textit{dest}\hspace{0.2em}\textit{suffix}}'.
Surely, the same effect is achieved by
directly specifying the
argument `{\textit{dest}\hspace{0.2em}\textit{suffix}}'
in the first form.
However, that requires to set up a different file
for each child. With the alternative form of the command
all these files can have exactly the same content
which simplifies setting them up and maintaining them.

For example, the following file |draft.tex|
with a compilation flag |\version| as described in \secref{sec:flags}
compiles the main document as a draft:
%
\begin{center}
\begin{tabular}{l}
|\def\version{draft}|\\
|\input{childdoc.def}|\\
|\childdocforward{|\textit{main}|}|
\end{tabular}
\end{center}
%
Likewise, the following files |final|\textit{nn}|.tex|
compile the final version of the child document
|child|\textit{nn}|.tex|:
%
\begin{center}
\begin{tabular}{l}
|\def\version{final}|\\
|\input{childdoc.def}|\\
|\childdocforwardprefix{final}{child}|
\end{tabular}
\end{center}
%

Note that when several versions of a main file and/or of each child file
are to be generated, it may be convenient to set up a |Makefile| or
shell script to automatise the process.

%%%%%%%%%%%%%%%%%%%%%%%%%%%%%%%%%%%%%%%%%%%%%%%%%%%%%%%%%%%%%%%%%%%%%%%%%%%%%%%%
\subsection{Command Line Processing}
\label{sec:commandline}

The effect of redirection files can also be achieved by invoking
the \LaTeX{} compiler with a more elaborate command line.
Most conveniently this should be done as part
of a shell script or a |Makefile|.

When using \textsf{childdoc} in the main file, the following
command lines effectively perform a redirection
(note that depending on the shell being used,
backslashes may have to be doubled: `|\|' $\to$ `|\\|'):
%
\begin{center}
|... -jobname "|\textit{target}|" |\\|"|[\textit{flags}]%
|\input{childdoc.def}\childdocforward[|\textit{main}|]{|\textit{dest}|}"|
\end{center}
%
Here \textit{target} is the name of the output file,
\textit{main} is the name of the main file
and \textit{dest} is the name of the main or child file to be processed
(all filenames without extensions).
The optional argument \textit{main} can be omitted
if \textit{main} matches \textit{dest}.
Optionally, compilation \textit{flags} can be defined via |\def| commands.
This command line makes the \TeX{} engine believe
it is compiling the file \textit{target}
whose content is specified as the latter parameter.
The provided code then forwards the processing to
\textit{main} or \textit{dest} as described in \secref{sec:forward}.

%%%%%%%%%%%%%%%%%%%%%%%%%%%%%%%%%%%%%%%%%%%%%%%%%%%%%%%%%%%%%%%%%%%%%%%%%%%%%%%%
\subsection{Include by Input}
\label{sec:input}

Including child documents by |\include| has some restrictions by design.
Most notably, the content of a child document always occupies
its own set of pages; pages cannot be shared between child documents.
Usually, this behaviour makes perfect sense
because each child document contain an essential part of the document.
However, in some situations it may be desirable to compose
a document from a collection of parts
without having mandatory page breaks between then.
For this case, the package
provides a mechanism to include parts
by |\input| which can also be processed individually.
However, by construction this mechanism
requires manual handling of the content to be output.

%%%%%%%%%%%%%%%%%%%%%%%%%%%%%%%%%%%%%%%%
\DescribeMacro{\ifchilddocmanual}
The main file should be prepared as usual, see \secref{sec:include}.
However, the document body must make a distinction
between processing of an individual part and of the main document, e.g.:
%
\begin{center}
\begin{tabular}{l}
|\ifchilddocmanual|\\
|\input{\childdocname}|\\
|\||else|\\
\textit{document body with }|\input{|\textit{part}|}|\\
|\||fi|
\end{tabular}
\end{center}
%
The conditional |\ifchilddocmanual| is true whenever
a part to be included by |\input| is being compiled,
and the name of the part is stored in |\childdocname|.

%%%%%%%%%%%%%%%%%%%%%%%%%%%%%%%%%%%%%%%%
\DescribeMacro{\childdocby}
Each part to be included by |\input| should start with:
%
\begin{center}
\begin{tabular}{l}
|\input{childdoc.def}|\\
|\childdocby{|\textit{main}|}|\\
\end{tabular}
\end{center}
%
The directive |\childdocby| is similar to |\childdocof|
described in \secref{sec:include},
but the subsequent selection of content must be done manually.
To that end, both |\ifchilddoc| and |\ifchilddocmanual|
will be true upon processing of a part,
and the name of the part is stored in |\childdocname|.
Note that |\jobname| will be set to the filename of the current part
so that each part receives an individual |.aux| file
that does not interfere with the |.aux| file(s) of the main document.
This behaviour can be altered by the alternative form
|\childdocby[*]{|\textit{main}|}| (with a non-empty optional argument)
which uses the |.aux| file of the main document
by setting |\jobname| to \textit{main}.

%%%%%%%%%%%%%%%%%%%%%%%%%%%%%%%%%%%%%%%%%%%%%%%%%%%%%%%%%%%%%%%%%%%%%%%%%%%%%%%%
\subsection{Driver Development}
\label{sec:driver}

The \textsf{childdoc} mechanism can also be use for the development
of definition files such as \LaTeX{} styles or classes.
This case differs from the above setup with multiple parts
included by |\include| in that no |\includeonly| should be invoked.
This can be achieved by starting the include file
(before |\ProvidesPackage|) with:
%
\begin{center}
\begin{tabular}{l}
|\input{childdoc.def}|\\
|\childdocforward{|\textit{main}|}|\\
\end{tabular}
\end{center}
%
or alternatively with:
%
\begin{center}
\begin{tabular}{l}
|\input{childdoc.def}|\\
|\childdocby{|\textit{main}|}|\\
\end{tabular}
\end{center}
%
Both forms have slightly different effects as described above.
The main file is prepared as usual, see \secref{sec:include}.

%%%%%%%%%%%%%%%%%%%%%%%%%%%%%%%%%%%%%%%%%%%%%%%%%%%%%%%%%%%%%%%%%%%%%%%%%%%%%%%%
\subsection{Legacy Detection}
\label{sec:detection}

The directive |\childdocmain| in the main file can detect
whether the complete document or merely a child is to be compiled
even without using the directive |\childdocof|.
This method is deprecated because it is less robust
and there is no compelling reason to use it;
it is merely provided for backward compatibility
and it may be removed in future versions.

If the detection mechanism is to be used,
it is mandatory to correctly specify
the filename of the main file as the argument of |\childdocmain|:
%
\begin{center}
\begin{tabular}{l}
|\input{childdoc.def}|\\
|\childdocmain{|\textit{main}|}|\\
\end{tabular}
\end{center}
%
If |\jobname| does not match the argument \textit{main} of |\childdocmain|,
it is assumed that |\jobname| points to the child file to be compiled.
When using |\childdocmain| with the main file specified as argument,
it suffices to start a child file
with just |\input{|\textit{main}|}|
without loading of the package and using |\childdocof|.
If instead all processing is done
with the appropriate \textsf{childdoc} directives,
the argument of \textit{main} of |\childdocmain| can be empty.

An alternative version of the command line processing described
in \secref{sec:commandline} using the detection mechanism reads:
%
\begin{center}
|... -jobname "|\textit{target}|" "|[\textit{flags}]%
[|\def\jobname{|\textit{dest}|}|]|\input{|\textit{main}|}"|
\end{center}

%%%%%%%%%%%%%%%%%%%%%%%%%%%%%%%%%%%%%%%%%%%%%%%%%%%%%%%%%%%%%%%%%%%%%%%%%%%%%%%%
\subsection{Manual Code}
\label{sec:manual}

In case one cannot be certain whether the definitions file |childdoc.def|
is installed on the target \TeX{} distribution
and one prefers not to ship it,
it is conceivable to paste a few relevant commands into the sources.

To that end, drop all statements |\input{childdoc.def}|
and perform the replacements as outlined below.
Instead of |\childdocmain{|\textit{main}|}| add the following code
to the top of the main file:
%
\begin{center}
\begin{tabular}{l}
|\||ifdefined\childdocname\endinput\||fi\newif\ifchilddoc|\\
|\edef\childdocname{\scantokens\expandafter{\jobname\noexpand}}|\\
|\def\childdocmain{|\textit{main}|}\||ifx\childdocmain\childdocname\||else|\\
|\childdoctrue\includeonly{\childdocname}\let\jobname\childdocmain\||fi|\\
\end{tabular}
\end{center}
%
Instead of |\childdocof{|\textit{main}|}| just include the main file
at the top of each child file:
%
\begin{center}
|\input{|\textit{main}|}|
\end{center}
%
A simple redirection |\childdocforward{|\textit{dest}|}| is achieved by:
%
\begin{center}
|\def\jobname{|\textit{dest}|}\input{\jobname}|
\end{center}
%
The redirection with prefix
|\childdocforwardprefix[|\textit{prefix}|]{|\textit{dest}|}|
is accomplished by:
%
\begin{center}
\begin{tabular}{l}
|{\edef\jobname{\scantokens\expandafter{\jobname\noexpand}}|\\
|\def\redirectjob |\textit{prefix}|#1~~~{\gdef\jobname{|\textit{dest}|#1}}|\\
|\expandafter\redirectjob\jobname~~~}\input{\jobname}|
\end{tabular}
\end{center}

In an alternative approach,
child documents can be compiled by a specific command line
without additional code or specific definitions:
%
\begin{center}
|... -jobname "|\textit{target}|" "|[\textit{flags}]%
|\includeonly{|\textit{dest}|}\input{|\textit{main}|}"|
\end{center}
%

%%%%%%%%%%%%%%%%%%%%%%%%%%%%%%%%%%%%%%%%%%%%%%%%%%%%%%%%%%%%%%%%%%%%%%%%%%%%%%%%
%%%%%%%%%%%%%%%%%%%%%%%%%%%%%%%%%%%%%%%%%%%%%%%%%%%%%%%%%%%%%%%%%%%%%%%%%%%%%%%%
\section{Information}

%%%%%%%%%%%%%%%%%%%%%%%%%%%%%%%%%%%%%%%%%%%%%%%%%%%%%%%%%%%%%%%%%%%%%%%%%%%%%%%%
\subsection{Copyright}

Copyright \copyright{} 2017--2018 Niklas Beisert

This work may be distributed and/or modified under the
conditions of the \LaTeX{} Project Public License, either version 1.3
of this license or (at your option) any later version.
The latest version of this license is in
  \url{http://www.latex-project.org/lppl.txt}
and version 1.3 or later is part of all distributions of \LaTeX{}
version 2005/12/01 or later.

This work has the LPPL maintenance status `maintained'.

The Current Maintainer of this work is Niklas Beisert.

This work consists of the files |README.txt|, |childdoc.ins| and |childdoc.dtx|
as well as the derived files |childdoc.def|, |cdocsamp.tex|
with |cdocsch1.tex|, |cdocsch2.tex|, |cdocspt3.tex|, |cdocspt4.tex|,
|cdocsdrf.tex|, |cdocsfn1.tex|, |cdocsfn2.tex|
as well as |childdoc.pdf|.

%%%%%%%%%%%%%%%%%%%%%%%%%%%%%%%%%%%%%%%%%%%%%%%%%%%%%%%%%%%%%%%%%%%%%%%%%%%%%%%%
\subsection{Files and Installation}

The package consists of the files:
%
\begin{center}
\begin{tabular}{ll}
    |README.txt|   & readme file \\
    |childdoc.ins| & installation file \\
    |childdoc.dtx| & source file \\
    |childdoc.def| & definition file \\
    |cdocsamp.tex| & sample main file \\
    |cdocsch1.tex| & sample include file \\
    |cdocsch2.tex| & sample include file \\
    |cdocspt3.tex| & sample part file \\
    |cdocspt4.tex| & sample part file \\
    |cdocsdrf.tex| & sample redirection file \\
    |cdocsfn1.tex| & sample redirection file \\
    |cdocsfn2.tex| & sample redirection file \\
    |childdoc.pdf| & manual
\end{tabular}
\end{center}
%
The distribution consists of the files
|README.txt|, |childdoc.ins| and |childdoc.dtx|.
%
\begin{itemize}
\item
Run (pdf)\LaTeX{} on |childdoc.dtx|
to compile the manual |childdoc.pdf| (this file).
\item
Run \LaTeX{} on |childdoc.ins| to create the definitions file |childdoc.def|
and the sample |cdocsamp.tex| with include files
|cdocsch1.tex|, |cdocsch2.tex|, |cdocspt3.tex|, |cdocspt4.tex|,
|cdocsdrf.tex|, |cdocsfn1.tex|, |cdocsfn2.tex|.
Then copy the file |childdoc.def| to an appropriate directory of your \LaTeX{}
distribution, e.g.\ \textit{texmf-root}|/tex/latex/childdoc|.
\end{itemize}

%%%%%%%%%%%%%%%%%%%%%%%%%%%%%%%%%%%%%%%%%%%%%%%%%%%%%%%%%%%%%%%%%%%%%%%%%%%%%%%%
\subsection{Related CTAN Packages}

There are several other packages which offer a similar functionality:
%
\begin{itemize}
\item
The packages
\href{http://ctan.org/pkg/docmute}{\textsf{docmute}},
\href{http://ctan.org/pkg/includex}{\textsf{includex}} and
\href{http://ctan.org/pkg/standalone}{\textsf{standalone}}
provide commands to include only the document body of
a child file thus allowing both files to be compiled individually.
\item
The packages \href{http://ctan.org/pkg/subdocs}{\textsf{subdocs}}
and \href{http://ctan.org/pkg/subfiles}{\textsf{subfiles}}
provide structures in which the main and child documents can be
encapsulated and allowing them to be compiled individually.
The inclusion mechanism is different from the conventional |\include|.
\item
The package \href{http://ctan.org/pkg/combine}{\textsf{combine}}
is an elaborate solution to combine several documents into one.
\end{itemize}
%
See also the CTAN topic \href{http://ctan.org/topic/subdocs}{\textsf{subdocs}}
for further related packages.
The present package differs from the above solutions in that
a document structure constructed with the conventional |\include| mechanism
just needs two extra commands at the top of every file
such that all constituent files can be compiled individually.

%%%%%%%%%%%%%%%%%%%%%%%%%%%%%%%%%%%%%%%%%%%%%%%%%%%%%%%%%%%%%%%%%%%%%%%%%%%%%%%%
%\subsection{Feature Suggestions}
%
%The following is a list of features which may be useful for future
%versions of this package:
%%
%\begin{itemize}
%\item
%\ldots
%\end{itemize}

%%%%%%%%%%%%%%%%%%%%%%%%%%%%%%%%%%%%%%%%%%%%%%%%%%%%%%%%%%%%%%%%%%%%%%%%%%%%%%%%
\subsection{Revision History}

%%%%%%%%%%%%%%%%%%%%%%%%%%%%%%%%%%%%%%%%
\paragraph{v2.0:} 2018/12/30

\begin{itemize}
\item
immediate forward processing
\item
added |\childdocby| mechanism
\item
manual restructured
\end{itemize}

%%%%%%%%%%%%%%%%%%%%%%%%%%%%%%%%%%%%%%%%
\paragraph{v1.6:} 2018/01/17

\begin{itemize}
\item
application for development of include files
\item
corrections to manual
\end{itemize}

%%%%%%%%%%%%%%%%%%%%%%%%%%%%%%%%%%%%%%%%
\paragraph{v1.5:} 2017/05/21

\begin{itemize}
\item
more complete structuring introduced
\item
|\childdocof| introduced
\item
|\childdoc| renamed to |\childdocmain|
\item
|\childredirect| renamed to |\childdocforward| and |\childdocforwardprefix|
and functionality expanded
\end{itemize}

%%%%%%%%%%%%%%%%%%%%%%%%%%%%%%%%%%%%%%%%
\paragraph{v1.0:} 2017/04/27

\begin{itemize}
\item
manual and install package
\item
first version published on CTAN
\end{itemize}

%%%%%%%%%%%%%%%%%%%%%%%%%%%%%%%%%%%%%%%%
\paragraph{v0.6:} 2017/04/26

\begin{itemize}
\item
redirection mechanism added
\end{itemize}

%%%%%%%%%%%%%%%%%%%%%%%%%%%%%%%%%%%%%%%%
\paragraph{v0.5:} 2017/04/26

\begin{itemize}
\item
functionality in definition file
\end{itemize}


%%%%%%%%%%%%%%%%%%%%%%%%%%%%%%%%%%%%%%%%%%%%%%%%%%%%%%%%%%%%%%%%%%%%%%%%%%%%%%%%
%%%%%%%%%%%%%%%%%%%%%%%%%%%%%%%%%%%%%%%%%%%%%%%%%%%%%%%%%%%%%%%%%%%%%%%%%%%%%%%%
%%%%%%%%%%%%%%%%%%%%%%%%%%%%%%%%%%%%%%%%%%%%%%%%%%%%%%%%%%%%%%%%%%%%%%%%%%%%%%%%
\appendix

\settowidth\MacroIndent{\rmfamily\scriptsize 000\ }

 \DocInput{childdoc.dtx}

\end{document}
%</driver>
% \fi
%
% %%%%%%%%%%%%%%%%%%%%%%%%%%%%%%%%%%%%%%%%%%%%%%%%%%%%%%%%%%%%%%%%%%%%%%%%%%%%%%
% %%%%%%%%%%%%%%%%%%%%%%%%%%%%%%%%%%%%%%%%%%%%%%%%%%%%%%%%%%%%%%%%%%%%%%%%%%%%%%
% \section{Sample}
%\iffalse
%<*samplemain>
%\fi
%
% The following presents a sample document
% with two chapters, two parts, a title page,
% a compile flag as well as three forwarding files to set the flag.
% It consists of eight |.tex| files:
% \begin{center}
% \begin{tabular}{ll}
% |cdocsamp.tex|&main file\\
% |cdocsch1.tex|&include file for chapter 1\\
% |cdocsch2.tex|&include file for chapter 2\\
% |cdocspt3.tex|&include file for part 3\\
% |cdocspt4.tex|&include file for part 4\\
% |cdocsdrf.tex|&forwarding file for main file in draft mode\\
% |cdocsfi1.tex|&forwarding file for final version of chapter 1\\
% |cdocsfi2.tex|&forwarding file for final version of chapter 2\\
% \end{tabular}
% \end{center}
% Each of the eight files can be compiled directly by the \LaTeX{} compiler.
%
% %%%%%%%%%%%%%%%%%%%%%%%%%%%%%%%%%%%%%%
% \paragraph{Main File.}
%
% The main file is called |cdocsamp.tex|.
%
% Load the \textsf{childdoc} definitions and
% declare the filename for the main document:
%    \begin{macrocode}
\input{childdoc.def}
\childdocmain{}
%    \end{macrocode}

% Optional override for |\version| flag:
%    \begin{macrocode}
%%\ifchilddoc\else\providecommand{\version}{draft}\fi
%    \end{macrocode}

% Define the default values for the |\version| flag
% (|final| for the main file and |draft| for childs):
%    \begin{macrocode}
\ifchilddoc
\providecommand{\version}{draft}
\else
\providecommand{\version}{final}
\fi
%    \end{macrocode}

% Load the standard document class:
%    \begin{macrocode}
\documentclass[12pt]{article}
%    \end{macrocode}

% Start the document body:
%    \begin{macrocode}
\begin{document}
%    \end{macrocode}

% Declare a title page.
% Print title, part of document being processed and version flag:
%    \begin{macrocode}
\addtocounter{page}{-1}
\begin{center}
{\LARGE\bfseries{}childdoc example\par}
\vspace{1cm}
\ifchilddoc
\ifchilddocmanual part\else chapter\fi:
`\childdocname' of `\childdocjob'\par
\else
main document: `\childdocjob'\par
\fi
version: \version\par
\end{center}
\newpage
%    \end{macrocode}

% Manually include selected file,
% otherwise process as usual:
%    \begin{macrocode}
\ifchilddocmanual
\section*{part `\childdocname'}
\input{\childdocname}
\else
%    \end{macrocode}

% Include the two chapters:
%    \begin{macrocode}
\include{cdocsch1}
\include{cdocsch2}
%    \end{macrocode}

% Include the two parts unless only chapters should be displayed:
%    \begin{macrocode}
\ifchilddoc\else
\section{part three}
\input{cdocspt3}
\section{part four}
\input{cdocspt4}
\fi
%    \end{macrocode}

% Process as usual until here:
%    \begin{macrocode}
\fi
%    \end{macrocode}

% End of document body:
%    \begin{macrocode}
\end{document}
%    \end{macrocode}
%\iffalse
%</samplemain>
%\fi
%
% %%%%%%%%%%%%%%%%%%%%%%%%%%%%%%%%%%%%%%
% \paragraph{Chapter Include Files.}
%
% The include files are called |cdocsch1.tex| and |cdocsch2.tex|.
%
%\iffalse
%<*samplechap1|samplechap2>
%\fi

% Optional override for |\version| flag:
%    \begin{macrocode}
%%\providecommand{\version}{final}
%    \end{macrocode}

% Include the main document:
%    \begin{macrocode}
\input{childdoc.def}
\childdocof{cdocsamp}
%    \end{macrocode}

%\iffalse
%</samplechap1|samplechap2>
%\fi
%
%\iffalse
%<*samplechap1>
%\fi
% Some text for chapter 1:
%    \begin{macrocode}
\section{one}
some text in chapter one
%    \end{macrocode}

%\iffalse
%</samplechap1>
%\fi
% Some text for chapter 2:
%\iffalse
%<*samplechap2>
%\fi
%    \begin{macrocode}
\section{two}
more text in chapter two
%    \end{macrocode}

%\iffalse
%</samplechap2>
%\fi
%
% %%%%%%%%%%%%%%%%%%%%%%%%%%%%%%%%%%%%%%
% \paragraph{Part Include Files.}
%
% The include files are called |cdocspt3.tex| and |cdocspt4.tex|.
%
%\iffalse
%<*samplepart3|samplepart4>
%\fi

% Optional override for |\version| flag:
%    \begin{macrocode}
%%\providecommand{\version}{final}
%    \end{macrocode}

% Include the main document:
%    \begin{macrocode}
\input{childdoc.def}
\childdocby{cdocsamp}
%    \end{macrocode}

%\iffalse
%</samplepart3|samplepart4>
%\fi
%
%\iffalse
%<*samplepart3>
%\fi
% Some text for part 3:
%    \begin{macrocode}
some text in part three
%    \end{macrocode}

%\iffalse
%</samplepart3>
%\fi
% Some text for part 4:
%\iffalse
%<*samplepart4>
%\fi
%    \begin{macrocode}
more text in part four
%    \end{macrocode}

%\iffalse
%</samplepart4>
%\fi
%
% %%%%%%%%%%%%%%%%%%%%%%%%%%%%%%%%%%%%%%
% \paragraph{Forwarding for a Complete Draft.}
%
% The following forwarding file |cdocsdrf.tex|
% compiles the main document in draft mode:
%\iffalse
%<*sampledraft>
%\fi
%    \begin{macrocode}
\def\version{draft}
\input{childdoc.def}
\childdocforward{cdocsamp}
%    \end{macrocode}

%\iffalse
%</sampledraft>
%\fi
%
% %%%%%%%%%%%%%%%%%%%%%%%%%%%%%%%%%%%%%%
% \paragraph{Forwarding for Final Version of the Chapters.}
%
% The following forwarding files |cdocsfn1.tex| and |cdocsfn2.tex|
% (with identical content)
% compile the final versions of the child documents
% |cdocsch1.tex| and |cdocsch2.tex|, respectively:
%\iffalse
%<*samplefinal>
%\fi
%    \begin{macrocode}
\def\version{final}
\input{childdoc.def}
\childdocforwardprefix[cdocsamp]{cdocsfn}{cdocsch}
%    \end{macrocode}

%\iffalse
%</samplefinal>
%\fi
%
% %%%%%%%%%%%%%%%%%%%%%%%%%%%%%%%%%%%%%%
% \paragraph{Command Line Processing.}
%
% The following three command lines generate the output files
% |cdocscld|, |cdocscl1| and |cdocscl2|
% which should be identical to
% |cdocsdrf|, |cdocsch1| and |cdocsfn2|, respectively:
% \begin{center}
% \begin{tabular}{l}
% |latex -jobname cdocscld \|\\
% |  "\def\version{draft}\input{childdoc.def}\childdocforward{cdocsamp}"|\\
% |latex -jobname cdocscl1 \|\\
% |  "\input{childdoc.def}\childdocforward[cdocsamp]{cdocsch1}"|\\
% |latex -jobname cdocscl2 \|\\
% |  "\def\version{final}\input{childdoc.def}\childdocforward{cdocsch2}"|
% \end{tabular}
% \end{center}
% Note that the trailing backslash on each first line
% merely continues the input to the second line
% (for convenient cut ant paste).
% Furthermore, the command |latex| can be replaced by any
% of its alternative versions such as |pdflatex|.
%
% %%%%%%%%%%%%%%%%%%%%%%%%%%%%%%%%%%%%%%%%%%%%%%%%%%%%%%%%%%%%%%%%%%%%%%%%%%%%%%
% %%%%%%%%%%%%%%%%%%%%%%%%%%%%%%%%%%%%%%%%%%%%%%%%%%%%%%%%%%%%%%%%%%%%%%%%%%%%%%
% \section{Implementation}
%\iffalse
%<*package>
%\fi
%
% This section describes the definitions file |childdoc.def|.

% The definitions cannot be loaded using |\usepackage| or |\RequirePackage|
% which has a mechanism to prevent loading a style file more than once.
% When loading the definitions by means of |\input|
% multiple instances have to be prevented manually:
%\iffalse
%This code needs to be before the `\ProvidesFile' directive
%which is defined at the beginning of this file.
%Therefore it is also placed there and commented out here.
%</package>
%<*discard>
%\fi
%    \begin{macrocode}
\ifdefined\childdocmain\endinput\fi
%    \end{macrocode}
%\iffalse
%</discard>
%<*package>
%\fi
%
% \macro{\ifchilddoc}
% \macro{\ifchilddocmanual}
% The conditional |\ifchilddoc| tells whether a
% child (true) or main (false) document is being compiled.
% The conditional |\ifchilddocmanual| tells whether
% the |\includeonly| mechanism is used (false) or
% the selection of child files must be performed manually (true).
% The definitions initialise to false:
%    \begin{macrocode}
\newif\ifchilddoc
\newif\ifchilddocmanual
%    \end{macrocode}

% \macro{\childdocname}
% \macro{\childdocjob}
% The macro |\childdocname| stores the name of the main document
% to be compiled. The macro |\childdocjob| stores the name of
% the document on which the \LaTeX{} compiler was originally invoked.
% The content of |\jobname| cannot be compared
% to filenames specified in the source due to different catcodes.
% The following code rescans |\jobname|, stores the result
% in |\childdocname| and saves a copy in |\childdocjob|:
%    \begin{macrocode}
\edef\childdocname{\scantokens\expandafter{\jobname\noexpand}}
\let\childdocjob\childdocname
%    \end{macrocode}

% \macro{\childdocdisable}
% The macro |\childdocdisable| prevents the main file
% from being processed more than once.
% At this stage, the main document command |\childdocmain|
% is assumed to be called once again where it should do nothing.
% Any subsequent call to it should prevent
% a secondary processing of the main document
% It overwrites the forwarding commands
% |\childdocof| and |\childdocforward|
% with empty macros to prevent further inclusions of the main document:
%    \begin{macrocode}
\newcommand{\childdocdisable}
{
  \renewcommand{\childdocmain}[1]{\renewcommand{\childdocmain}[1]{\endinput}}
  \renewcommand{\childdocof}[1]{}
  \renewcommand{\childdocby}[2][]{}
  \renewcommand{\childdocforward}[2][]{}
  \renewcommand{\childdocdisable}{}
}
%    \end{macrocode}

% \macro{\childdocmain}
% The macro |\childdocmain| is to be called at the top of the main file
% with nothing or the main filename (without extension) as argument.
% First, it breaks loops.
% If the argument is not empty and does not match |\childdocname|
% (which is set by the first inclusion of |childdoc.def|),
% |\ifchilddoc| is set to true, |\includeonly| is applied to the child file
% and |\jobname| is set to the main file
% (for proper handling of |.aux| files):
%    \begin{macrocode}
\newcommand{\childdocmain}[1]
{
  \childdocdisable\childdocmain{}
  \if?#1?\else
    \begingroup
      \def\childdoctmp{#1}
      \ifx\childdoctmp\childdocname
        \def\childdoctmp{}
      \else
        \def\childdoctmp
        {
          \childdoctrue
          \includeonly{\childdocname}
          \def\childdocjob{#1}
          \def\jobname{#1}
        }
      \fi
      \expandafter
    \endgroup
    \childdoctmp
  \fi
}
%    \end{macrocode}

% \macro{\childdocof}
% The command |\childdocof| redirects
% compilation to the main file |#1|.
%    \begin{macrocode}
\newcommand{\childdocof}[1]
{
  \childdocdisable
  \childdoctrue
  \includeonly{\childdocname}
  \def\jobname{#1}
  \def\childdocjob{#1}
  \input{#1}
}
%    \end{macrocode}

% \macro{\childdocby}
% The command |\childdocby| ....
%    \begin{macrocode}
\newcommand{\childdocby}[2][]
{
  \childdocdisable
  \childdoctrue
  \childdocmanualtrue
  \if?#1?\else
    \def\jobname{#2}
  \fi
  \def\childdocjob{#2}
  \input{#2}
  \endinput
}
%    \end{macrocode}

% \macro{\childdocforward}
% The command |\childdocforward| redirects
% compilation to the main file or
% (if the optional argument is given) a child file.
% Parameters are set as if the main file
% or a child file starting with |\childdocof| was compiled.
% Then compilation is handed over to the main file:
%    \begin{macrocode}
\newcommand{\childdocforward}[2][]
{
  \begingroup
    \if?#1?
      \def\childdoctmp
      {
        \def\childdocname{#2}
        \def\childdocjob{#2}
        \def\jobname{#2}
        \input{#2}
        \endinput
      }
    \else
      \def\childdoctmp
      {
        \childdocdisable
        \def\childdocname{#2}
        \childdoctrue
        \includeonly{#2}
        \def\childdocjob{#1}
        \def\jobname{#1}
        \input{#1}
        \endinput
      }
    \fi
    \expandafter
  \endgroup
  \childdoctmp
}
%    \end{macrocode}

% \macro{\childdocforwardprefix}
% The command |\childdocforwardprefix| redirects
% compilation to the main or a child file by means of a pattern.
% The prefix |#1| in the current filename is replaced by |#2|
% and the suffix of the current filename is kept
% (it is assumed that the filename does not contain the substring `|~~~|'
% which is used as a delimiter).
% Compilation is handed over to the new file by |\childdocforward|:
%    \begin{macrocode}
\newcommand{\childdocforwardprefix}[3][]
{
  \begingroup
    \def\childdocextract #2##1~~~{\def\childdoctmp{\childdocforward[#1]{#3##1}}}
    \expandafter\childdocextract\childdocname~~~
    \expandafter
  \endgroup
  \childdoctmp
}
%    \end{macrocode}

% \macro{\childdoc}
% The deprecated macro |\childdoc| is a legacy version of |\childdocmain|:
%    \begin{macrocode}
\newcommand{\childdoc}{\childdocmain}
%    \end{macrocode}

% \macro{\childdocredirect}
% The deprecated macro |\childdocredirect| is a legacy version
% of |\childdocforward| and |\childdocforwardprefix|:
%    \begin{macrocode}
\newcommand{\childdocredirect}[2][]
{
  \begingroup
    \if?#1?
      \def\childdoctmp{\childdocforward{#2}}
    \else
      \def\childdoctmp{\childdocforwardprefix{#1}{#2}}
    \fi
    \expandafter
  \endgroup
  \childdoctmp
}
%    \end{macrocode}

%\iffalse
%</package>
%\fi
%
\endinput

\childdocby{cdocsamp}
%    \end{macrocode}

%\iffalse
%</samplepart3|samplepart4>
%\fi
%
%\iffalse
%<*samplepart3>
%\fi
% Some text for part 3:
%    \begin{macrocode}
some text in part three
%    \end{macrocode}

%\iffalse
%</samplepart3>
%\fi
% Some text for part 4:
%\iffalse
%<*samplepart4>
%\fi
%    \begin{macrocode}
more text in part four
%    \end{macrocode}

%\iffalse
%</samplepart4>
%\fi
%
% %%%%%%%%%%%%%%%%%%%%%%%%%%%%%%%%%%%%%%
% \paragraph{Forwarding for a Complete Draft.}
%
% The following forwarding file |cdocsdrf.tex|
% compiles the main document in draft mode:
%\iffalse
%<*sampledraft>
%\fi
%    \begin{macrocode}
\def\version{draft}
% \iffalse
%
% childdoc.dtx Copyright (C) 2017-2018 Niklas Beisert
%
% This work may be distributed and/or modified under the
% conditions of the LaTeX Project Public License, either version 1.3
% of this license or (at your option) any later version.
% The latest version of this license is in
%   http://www.latex-project.org/lppl.txt
% and version 1.3 or later is part of all distributions of LaTeX
% version 2005/12/01 or later.
%
% This work has the LPPL maintenance status `maintained'.
%
% The Current Maintainer of this work is Niklas Beisert.
%
% This work consists of the files childdoc.dtx and childdoc.ins
% and the derived files childdoc.def and cdocsamp.tex with
% cdocsch1.tex, cdocsch2.tex, cdocsdrf.tex, cdocsfn1.tex, cdocsfn2.tex.
%
%<package>\ifdefined\childdocmain\endinput\fi
%<package>\ProvidesFile{childdoc.def}[2018/12/30 v2.0 child document driver]
%<samplemain>\ProvidesFile{cdocsamp.tex}[2018/12/30 v2.0 sample for childdoc]
%<*driver>
%\ProvidesFile{childdoc.drv}[2018/12/30 v2.0 childdoc reference manual file]
\PassOptionsToClass{10pt,a4paper}{article}
\documentclass{ltxdoc}

\usepackage[margin=35mm]{geometry}
\usepackage{hyperref}
\usepackage{hyperxmp}
\usepackage[usenames]{color}

\hypersetup{colorlinks=true}
\hypersetup{pdfstartview=FitH}
\hypersetup{pdfpagemode=UseNone}
\hypersetup{pdfsource={}}
\hypersetup{pdflang={en-UK}}
\hypersetup{pdfcopyright={Copyright 2017-2018 Niklas Beisert.
  This work may be distributed and/or modified under the
  conditions of the LaTeX Project Public License, either version 1.3
  of this license or (at your option) any later version.}}
\hypersetup{pdflicenseurl={http://www.latex-project.org/lppl.txt}}
\hypersetup{pdfcontactaddress={ETH Zurich, ITP, HIT K,
  Wolfgang-Pauli-Strasse 27}}
\hypersetup{pdfcontactpostcode={8093}}
\hypersetup{pdfcontactcity={Zurich}}
\hypersetup{pdfcontactcountry={Switzerland}}
\hypersetup{pdfcontactemail={nbeisert@itp.phys.ethz.ch}}
\hypersetup{pdfcontacturl={http://people.phys.ethz.ch/\xmptilde nbeisert/}}

\newcommand{\secref}[1]{\hyperref[#1]{section \ref*{#1}}}

\parskip1ex
\parindent0pt
\let\olditemize\itemize
\def\itemize{\olditemize\parskip0pt}

\begin{document}

\title{The \textsf{childdoc} Package}
\hypersetup{pdftitle={The childdoc Package}}
\author{Niklas Beisert\\[2ex]
  Institut f\"ur Theoretische Physik\\
  Eidgen\"ossische Technische Hochschule Z\"urich\\
  Wolfgang-Pauli-Strasse 27, 8093 Z\"urich, Switzerland\\[1ex]
  \href{mailto:nbeisert@itp.phys.ethz.ch}
  {\texttt{nbeisert@itp.phys.ethz.ch}}}
\hypersetup{pdfauthor={Niklas Beisert}}
\hypersetup{pdfsubject={Manual for the LaTeX2e Package childdoc}}
\date{30 December 2018, \textsf{v2.0}}
\maketitle

\begin{abstract}\noindent
\textsf{childdoc} is a \LaTeXe{} package
that enables the direct compilation
of document sections included by |\include|
to individual files.
\end{abstract}

\begingroup
\parskip0ex
\tableofcontents
\endgroup

%%%%%%%%%%%%%%%%%%%%%%%%%%%%%%%%%%%%%%%%%%%%%%%%%%%%%%%%%%%%%%%%%%%%%%%%%%%%%%%%
%%%%%%%%%%%%%%%%%%%%%%%%%%%%%%%%%%%%%%%%%%%%%%%%%%%%%%%%%%%%%%%%%%%%%%%%%%%%%%%%
\section{Introduction}

\LaTeX{} provides a mechanism to structure a large document (such as a book)
into a main file and several child files (containing the chapters)
using the |\include| command.
This mechanism is beneficial for documents
which span hundreds of pages in order to
make the source file(s) more manageable.
Moreover, compilation can be restricted to
selected child files by means of the |\includeonly| command.
The latter feature can be used to reduce the compilation time while editing
(this was significantly more useful in the earlier days of \LaTeX{})
or to generate a smaller document which is easier to navigate.
Another application of |\includeonly| is to generate
documents consisting of selected parts of the complete document.

However, there are a few drawbacks of the plain |\include| mechanism:
\begin{itemize}
\item
The child files cannot be compiled on their own,
they can only be compiled via the main file.
A naive editing environment
(such as a text editor with an option
to have the current file processed by \LaTeX)
may require one to switch to the main file before compiling;
attempting to compile the child file produces errors.
\item
The main file must be modified (each time)
to adjust the |\includeonly| command
to the present needs. This easily leaves the main file in a messy state.
\item
The generated document will always carry the filename
of the main document. This is inconvenient if
several child files are to be compiled and
to be kept for distribution.
\end{itemize}

The present package provides a simple interface
to make child files individually compilable by \LaTeX{}.
Compiling a child file then has the same effect as compiling
the main file with an |\includeonly| command
to select the appropriate child.
Moreover the generated document will carry the name of the child
rather than the main file.
This resolves all three above issues.

This feature is meant to make the editing of books,
thesis documents and lecture notes somewhat more convenient.
However, the package can also be used efficiently for
composing a series of documents (such as exercise sheets)
which are typically distributed individually.
It then assists the author in generating the individual documents
(potentially in different versions)
as well as a document containing the collected series.
Another application is in developing style files
or other kinds of included material
where compilation of the style file could redirect
to a sample or test file.

%%%%%%%%%%%%%%%%%%%%%%%%%%%%%%%%%%%%%%%%%%%%%%%%%%%%%%%%%%%%%%%%%%%%%%%%%%%%%%%%
%%%%%%%%%%%%%%%%%%%%%%%%%%%%%%%%%%%%%%%%%%%%%%%%%%%%%%%%%%%%%%%%%%%%%%%%%%%%%%%%
\section{Usage}

First of all, the package \textsf{childdoc} is \emph{not} a standard
\LaTeXe{} |.sty| style file! Therefore it needs to be invoked in
a non-standard way.

%%%%%%%%%%%%%%%%%%%%%%%%%%%%%%%%%%%%%%%%%%%%%%%%%%%%%%%%%%%%%%%%%%%%%%%%%%%%%%%%
\subsection{Included Files}
\label{sec:include}

%%%%%%%%%%%%%%%%%%%%%%%%%%%%%%%%%%%%%%%%
\DescribeMacro{\childdocmain}
To use the package, add the commands
\begin{center}
\begin{tabular}{l}
|\input{childdoc.def}|\\
|\childdocmain{}|\\
\end{tabular}
\end{center}
at the very top of the main \LaTeX{} file,
in particular \emph{before} the |\documentclass| statement!
The argument of |\childdocmain| should be left empty
(but it must be present).

%%%%%%%%%%%%%%%%%%%%%%%%%%%%%%%%%%%%%%%%
\DescribeMacro{\childdocof}
Furthermore, add the commands
\begin{center}
\begin{tabular}{l}
|\input{childdoc.def}|\\
|\childdocof{|\textit{main}|}|\\
\end{tabular}
\end{center}
at the top of every child file \textit{child}
which is included by |\include{|\textit{child}|}|
from within the main file
(or at least for those files to be compiled individually).
The argument \textit{main} must be the filename of the main file.

There are a couple of
considerations in setting up the main and child documents:

%%%%%%%%%%%%%%%%%%%%%%%%%%%%%%%%%%%%%%%%
\paragraph{Restrictions.}

Please note the following restrictions:
\begin{itemize}
\item
|\childdocmain| must be called with one argument \textit{main}
to ensure compatibility with earlier version of the package.
It must either be empty (|\childdocmain{}|)
or precisely match the filename of the main file in which it is specified.
See \secref{sec:detection} for further information.
\item
The filename \textit{main} must be specified without the |.tex| extension.
\item
The filename \textit{main} is case sensitive
(even in case-insensitive file systems)
due to internal string comparison.
\item
The argument \textit{main} should be fully expanded, it cannot be a macro.
\item
Subdirectories and special characters should be avoided in filenames.
\item
The command |\childdocmain{|\textit{main}|}| must be followed by a whitespace.
It should not be followed immediately by another command
or by a comment mark `|%|'.
This is because the \TeX{} parser reads the token immediately following
the argument of |\childdocmain| and puts it
at the beginning of every child section;
however, a white\-space is ignored.
\end{itemize}

%%%%%%%%%%%%%%%%%%%%%%%%%%%%%%%%%%%%%%%%
\paragraph{Content of Main File.}

It is advisable to place all content in the child files included by |\include|.
Any output contained in the main file will appear in all child documents
unless suppressed manually;
it cannot be suppressed automatically by the |\includeonly| directive
and thus should normally be avoided.
A method to include some content in the main file
by means of conditional processing is described in \secref{sec:conditional}.

%%%%%%%%%%%%%%%%%%%%%%%%%%%%%%%%%%%%%%%%
\paragraph{Page Numbering.}

When only a part of the document is compiled,
the appropriate numbering of pages
(as well as other status parameters)
is determined from the |.aux| files.
The latter contain information from previous passes.
However this information needs to propagate through
all intermediate child documents.
Therefore the page numbering in child documents may well
be inconsistent until the complete document is compiled at least once.

A useful (if unconventional) way to always ensure a consistent
page numbering is to restart the numbering in each child document
and denote the pages by `\textit{child}|.|\textit{page}'
where \textit{child} represents the chapter/section number of the child file.
This can be achieved by the command
|\numberwithin{page}{|\textit{child}|}|
of the \textsf{amsmath} package
where \textit{child} can be |chapter| or |section|
depending on the chosen structuring.
Alternatively, one can modify the macro |\thepage| appropriately
and reset the counter |page| at the start of each child file.

%%%%%%%%%%%%%%%%%%%%%%%%%%%%%%%%%%%%%%%%%%%%%%%%%%%%%%%%%%%%%%%%%%%%%%%%%%%%%%%%
\subsection{Conditional Processing}
\label{sec:conditional}

The package provides a mechanism to compile different versions
of a document. To customise the versions further some conditional processing
can come in handy to distinguish which version is being compiled.
The package provides two macros to describe the compilation context:

%%%%%%%%%%%%%%%%%%%%%%%%%%%%%%%%%%%%%%%%
\DescribeMacro{\ifchilddoc}
The conditional |\ifchilddoc| distinguishes between the compilation of
child documents and the main document:
%
\begin{center}
|\ifchilddoc |\textit{child-code}| |[|\||else |\textit{main-code}]| \||fi|
\end{center}

%%%%%%%%%%%%%%%%%%%%%%%%%%%%%%%%%%%%%%%%
\DescribeMacro{\childdocname}
\DescribeMacro{\childdocjob}
The macro |\childdocname| contains the filename (without extension)
of the main or child file being processed.
Note that |\childdocjob| will always contain the name of the main file.

%%%%%%%%%%%%%%%%%%%%%%%%%%%%%%%%%%%%%%%%
\paragraph{Title Page.}

Conditional processing can be used to include a title or banner page
in the main document when proper precautions are taken.
Importantly, the code in the main file should ensure that the page counter
(as well as other status parameters which are stored in the |.aux| files)
takes the same value after the conditional processing.
Otherwise the page numbers may take divergent values
depending on which part is compiled.

For example, a title page could be declared by:
%
\begin{center}
\begin{tabular}{l}
|\ifchilddoc\||else|\\
|\addtocounter{page}{-1}|\\
\textit{code for title page}\\
|\newpage|\\
|\||fi|
\end{tabular}
\end{center}
%
A banner page for the child documents can be generated by:
%
\begin{center}
\begin{tabular}{l}
|\ifchilddoc|\\
|\addtocounter{page}{-1}|\\
\textit{code for banner page}\\
|\newpage|\\
|\||fi|
\end{tabular}
\end{center}
%
Here one could write a message such as:
\begin{center}
|This is the part \childdocname{} of \childdocjob{}.|
\end{center}

%%%%%%%%%%%%%%%%%%%%%%%%%%%%%%%%%%%%%%%%%%%%%%%%%%%%%%%%%%%%%%%%%%%%%%%%%%%%%%%%
\subsection{Flags}
\label{sec:flags}

The package makes it easy to generate different versions
of the main or child documents.
To this end compilation flags can be defined
and assigned different default values.
They will be particularly useful in conjunction
with the forwarding mechanism described in \secref{sec:forward}.

For example, it may be useful to have a flag |\version|
which can be set to |draft| or |final|.
The document source will contain some conditional code
depending on the value of |\version|.
Suppose further, the flag should default to |final| for the main file
and to |draft| for child files
which is a natural assignment for editing the document.
This is achieved by placing the following code
in the preamble of the main document
(below the |\childdocmain| directive):
%
\begin{center}
\begin{tabular}{l}
|\ifchilddoc|\\
|\providecommand{\version}{draft}|\\
|\||else|\\
|\providecommand{\version}{final}|\\
|\||fi|
\end{tabular}
\end{center}
%
The definition by |\providecommand| makes sure
that previous definitions are not overwritten.
Further statements |\providecommand{\version}{...}|
can thus be added before the above code to override it.

For the main file, one might add a line
(between |\childdocmain| and the above block)
%
\begin{center}
|%\ifchilddoc\||else\providecommand{\version}{draft}\||fi|
\end{center}
%
which can be uncommented to produce a draft version.
Likewise one can add a line to the very top of a child file
(above the |\childdocof{|\textit{main}|}| directive)
%
\begin{center}
|%\providecommand{\version}{final}|
\end{center}
%
which can be uncommented to produce the final version of this child document.

%%%%%%%%%%%%%%%%%%%%%%%%%%%%%%%%%%%%%%%%%%%%%%%%%%%%%%%%%%%%%%%%%%%%%%%%%%%%%%%%
\subsection{Forwarding}
\label{sec:forward}

Different versions of the main or child documents
using compilation flags as described in \secref{sec:flags}
can be (permanently) stored in different files
for convenient compilation, viewing and distribution.
To this end, the package defines a command
to pass on compilation to a different file:

%%%%%%%%%%%%%%%%%%%%%%%%%%%%%%%%%%%%%%%%
\DescribeMacro{\childdocforward}
The command |\childdocforward| redirects processing to
another source file:
%
\begin{center}
\begin{tabular}{l}
|\input{childdoc.def}|\\
|\childdocforward[|\textit{main}|]{|\textit{dest}|}|\\
\end{tabular}
\end{center}
%
The argument \textit{dest} is the destination file
(without extension).
It should be the main file or one of the child files.
Note that further \textsf{childdoc} directives
such as |\childdocof| and |\childdocforward|
in the indicated file will be processed in this form.
The optional argument \textit{main}
passes on directly to the main file \textit{main}
while pretending to compile the child \textit{dest}.
This form behaves as if \textit{dest}
issues |\childdocof{|\textit{main}|}| right away,
and no further \textsf{childdoc} directives will be processed.

%%%%%%%%%%%%%%%%%%%%%%%%%%%%%%%%%%%%%%%%
\DescribeMacro{\...prefix}
In the alternative form |\childdocforwardprefix|,
%
\begin{center}
\begin{tabular}{l}
|\input{childdoc.def}|\\
|\childdocforwardprefix[|\textit{main}|]{|\textit{prefix}|}{|\textit{dest}|}|
\end{tabular}
\end{center}
%
the destination file is determined by a pattern
depending on the current file:
To make this work, the current file must be called
`{\textit{prefix}\hspace{0.2em}\textit{suffix}}'
with \textit{prefix} matching precisely the argument.
Processing is then passed on to the file
`{\textit{dest}\hspace{0.2em}\textit{suffix}}'.
Surely, the same effect is achieved by
directly specifying the
argument `{\textit{dest}\hspace{0.2em}\textit{suffix}}'
in the first form.
However, that requires to set up a different file
for each child. With the alternative form of the command
all these files can have exactly the same content
which simplifies setting them up and maintaining them.

For example, the following file |draft.tex|
with a compilation flag |\version| as described in \secref{sec:flags}
compiles the main document as a draft:
%
\begin{center}
\begin{tabular}{l}
|\def\version{draft}|\\
|\input{childdoc.def}|\\
|\childdocforward{|\textit{main}|}|
\end{tabular}
\end{center}
%
Likewise, the following files |final|\textit{nn}|.tex|
compile the final version of the child document
|child|\textit{nn}|.tex|:
%
\begin{center}
\begin{tabular}{l}
|\def\version{final}|\\
|\input{childdoc.def}|\\
|\childdocforwardprefix{final}{child}|
\end{tabular}
\end{center}
%

Note that when several versions of a main file and/or of each child file
are to be generated, it may be convenient to set up a |Makefile| or
shell script to automatise the process.

%%%%%%%%%%%%%%%%%%%%%%%%%%%%%%%%%%%%%%%%%%%%%%%%%%%%%%%%%%%%%%%%%%%%%%%%%%%%%%%%
\subsection{Command Line Processing}
\label{sec:commandline}

The effect of redirection files can also be achieved by invoking
the \LaTeX{} compiler with a more elaborate command line.
Most conveniently this should be done as part
of a shell script or a |Makefile|.

When using \textsf{childdoc} in the main file, the following
command lines effectively perform a redirection
(note that depending on the shell being used,
backslashes may have to be doubled: `|\|' $\to$ `|\\|'):
%
\begin{center}
|... -jobname "|\textit{target}|" |\\|"|[\textit{flags}]%
|\input{childdoc.def}\childdocforward[|\textit{main}|]{|\textit{dest}|}"|
\end{center}
%
Here \textit{target} is the name of the output file,
\textit{main} is the name of the main file
and \textit{dest} is the name of the main or child file to be processed
(all filenames without extensions).
The optional argument \textit{main} can be omitted
if \textit{main} matches \textit{dest}.
Optionally, compilation \textit{flags} can be defined via |\def| commands.
This command line makes the \TeX{} engine believe
it is compiling the file \textit{target}
whose content is specified as the latter parameter.
The provided code then forwards the processing to
\textit{main} or \textit{dest} as described in \secref{sec:forward}.

%%%%%%%%%%%%%%%%%%%%%%%%%%%%%%%%%%%%%%%%%%%%%%%%%%%%%%%%%%%%%%%%%%%%%%%%%%%%%%%%
\subsection{Include by Input}
\label{sec:input}

Including child documents by |\include| has some restrictions by design.
Most notably, the content of a child document always occupies
its own set of pages; pages cannot be shared between child documents.
Usually, this behaviour makes perfect sense
because each child document contain an essential part of the document.
However, in some situations it may be desirable to compose
a document from a collection of parts
without having mandatory page breaks between then.
For this case, the package
provides a mechanism to include parts
by |\input| which can also be processed individually.
However, by construction this mechanism
requires manual handling of the content to be output.

%%%%%%%%%%%%%%%%%%%%%%%%%%%%%%%%%%%%%%%%
\DescribeMacro{\ifchilddocmanual}
The main file should be prepared as usual, see \secref{sec:include}.
However, the document body must make a distinction
between processing of an individual part and of the main document, e.g.:
%
\begin{center}
\begin{tabular}{l}
|\ifchilddocmanual|\\
|\input{\childdocname}|\\
|\||else|\\
\textit{document body with }|\input{|\textit{part}|}|\\
|\||fi|
\end{tabular}
\end{center}
%
The conditional |\ifchilddocmanual| is true whenever
a part to be included by |\input| is being compiled,
and the name of the part is stored in |\childdocname|.

%%%%%%%%%%%%%%%%%%%%%%%%%%%%%%%%%%%%%%%%
\DescribeMacro{\childdocby}
Each part to be included by |\input| should start with:
%
\begin{center}
\begin{tabular}{l}
|\input{childdoc.def}|\\
|\childdocby{|\textit{main}|}|\\
\end{tabular}
\end{center}
%
The directive |\childdocby| is similar to |\childdocof|
described in \secref{sec:include},
but the subsequent selection of content must be done manually.
To that end, both |\ifchilddoc| and |\ifchilddocmanual|
will be true upon processing of a part,
and the name of the part is stored in |\childdocname|.
Note that |\jobname| will be set to the filename of the current part
so that each part receives an individual |.aux| file
that does not interfere with the |.aux| file(s) of the main document.
This behaviour can be altered by the alternative form
|\childdocby[*]{|\textit{main}|}| (with a non-empty optional argument)
which uses the |.aux| file of the main document
by setting |\jobname| to \textit{main}.

%%%%%%%%%%%%%%%%%%%%%%%%%%%%%%%%%%%%%%%%%%%%%%%%%%%%%%%%%%%%%%%%%%%%%%%%%%%%%%%%
\subsection{Driver Development}
\label{sec:driver}

The \textsf{childdoc} mechanism can also be use for the development
of definition files such as \LaTeX{} styles or classes.
This case differs from the above setup with multiple parts
included by |\include| in that no |\includeonly| should be invoked.
This can be achieved by starting the include file
(before |\ProvidesPackage|) with:
%
\begin{center}
\begin{tabular}{l}
|\input{childdoc.def}|\\
|\childdocforward{|\textit{main}|}|\\
\end{tabular}
\end{center}
%
or alternatively with:
%
\begin{center}
\begin{tabular}{l}
|\input{childdoc.def}|\\
|\childdocby{|\textit{main}|}|\\
\end{tabular}
\end{center}
%
Both forms have slightly different effects as described above.
The main file is prepared as usual, see \secref{sec:include}.

%%%%%%%%%%%%%%%%%%%%%%%%%%%%%%%%%%%%%%%%%%%%%%%%%%%%%%%%%%%%%%%%%%%%%%%%%%%%%%%%
\subsection{Legacy Detection}
\label{sec:detection}

The directive |\childdocmain| in the main file can detect
whether the complete document or merely a child is to be compiled
even without using the directive |\childdocof|.
This method is deprecated because it is less robust
and there is no compelling reason to use it;
it is merely provided for backward compatibility
and it may be removed in future versions.

If the detection mechanism is to be used,
it is mandatory to correctly specify
the filename of the main file as the argument of |\childdocmain|:
%
\begin{center}
\begin{tabular}{l}
|\input{childdoc.def}|\\
|\childdocmain{|\textit{main}|}|\\
\end{tabular}
\end{center}
%
If |\jobname| does not match the argument \textit{main} of |\childdocmain|,
it is assumed that |\jobname| points to the child file to be compiled.
When using |\childdocmain| with the main file specified as argument,
it suffices to start a child file
with just |\input{|\textit{main}|}|
without loading of the package and using |\childdocof|.
If instead all processing is done
with the appropriate \textsf{childdoc} directives,
the argument of \textit{main} of |\childdocmain| can be empty.

An alternative version of the command line processing described
in \secref{sec:commandline} using the detection mechanism reads:
%
\begin{center}
|... -jobname "|\textit{target}|" "|[\textit{flags}]%
[|\def\jobname{|\textit{dest}|}|]|\input{|\textit{main}|}"|
\end{center}

%%%%%%%%%%%%%%%%%%%%%%%%%%%%%%%%%%%%%%%%%%%%%%%%%%%%%%%%%%%%%%%%%%%%%%%%%%%%%%%%
\subsection{Manual Code}
\label{sec:manual}

In case one cannot be certain whether the definitions file |childdoc.def|
is installed on the target \TeX{} distribution
and one prefers not to ship it,
it is conceivable to paste a few relevant commands into the sources.

To that end, drop all statements |\input{childdoc.def}|
and perform the replacements as outlined below.
Instead of |\childdocmain{|\textit{main}|}| add the following code
to the top of the main file:
%
\begin{center}
\begin{tabular}{l}
|\||ifdefined\childdocname\endinput\||fi\newif\ifchilddoc|\\
|\edef\childdocname{\scantokens\expandafter{\jobname\noexpand}}|\\
|\def\childdocmain{|\textit{main}|}\||ifx\childdocmain\childdocname\||else|\\
|\childdoctrue\includeonly{\childdocname}\let\jobname\childdocmain\||fi|\\
\end{tabular}
\end{center}
%
Instead of |\childdocof{|\textit{main}|}| just include the main file
at the top of each child file:
%
\begin{center}
|\input{|\textit{main}|}|
\end{center}
%
A simple redirection |\childdocforward{|\textit{dest}|}| is achieved by:
%
\begin{center}
|\def\jobname{|\textit{dest}|}\input{\jobname}|
\end{center}
%
The redirection with prefix
|\childdocforwardprefix[|\textit{prefix}|]{|\textit{dest}|}|
is accomplished by:
%
\begin{center}
\begin{tabular}{l}
|{\edef\jobname{\scantokens\expandafter{\jobname\noexpand}}|\\
|\def\redirectjob |\textit{prefix}|#1~~~{\gdef\jobname{|\textit{dest}|#1}}|\\
|\expandafter\redirectjob\jobname~~~}\input{\jobname}|
\end{tabular}
\end{center}

In an alternative approach,
child documents can be compiled by a specific command line
without additional code or specific definitions:
%
\begin{center}
|... -jobname "|\textit{target}|" "|[\textit{flags}]%
|\includeonly{|\textit{dest}|}\input{|\textit{main}|}"|
\end{center}
%

%%%%%%%%%%%%%%%%%%%%%%%%%%%%%%%%%%%%%%%%%%%%%%%%%%%%%%%%%%%%%%%%%%%%%%%%%%%%%%%%
%%%%%%%%%%%%%%%%%%%%%%%%%%%%%%%%%%%%%%%%%%%%%%%%%%%%%%%%%%%%%%%%%%%%%%%%%%%%%%%%
\section{Information}

%%%%%%%%%%%%%%%%%%%%%%%%%%%%%%%%%%%%%%%%%%%%%%%%%%%%%%%%%%%%%%%%%%%%%%%%%%%%%%%%
\subsection{Copyright}

Copyright \copyright{} 2017--2018 Niklas Beisert

This work may be distributed and/or modified under the
conditions of the \LaTeX{} Project Public License, either version 1.3
of this license or (at your option) any later version.
The latest version of this license is in
  \url{http://www.latex-project.org/lppl.txt}
and version 1.3 or later is part of all distributions of \LaTeX{}
version 2005/12/01 or later.

This work has the LPPL maintenance status `maintained'.

The Current Maintainer of this work is Niklas Beisert.

This work consists of the files |README.txt|, |childdoc.ins| and |childdoc.dtx|
as well as the derived files |childdoc.def|, |cdocsamp.tex|
with |cdocsch1.tex|, |cdocsch2.tex|, |cdocspt3.tex|, |cdocspt4.tex|,
|cdocsdrf.tex|, |cdocsfn1.tex|, |cdocsfn2.tex|
as well as |childdoc.pdf|.

%%%%%%%%%%%%%%%%%%%%%%%%%%%%%%%%%%%%%%%%%%%%%%%%%%%%%%%%%%%%%%%%%%%%%%%%%%%%%%%%
\subsection{Files and Installation}

The package consists of the files:
%
\begin{center}
\begin{tabular}{ll}
    |README.txt|   & readme file \\
    |childdoc.ins| & installation file \\
    |childdoc.dtx| & source file \\
    |childdoc.def| & definition file \\
    |cdocsamp.tex| & sample main file \\
    |cdocsch1.tex| & sample include file \\
    |cdocsch2.tex| & sample include file \\
    |cdocspt3.tex| & sample part file \\
    |cdocspt4.tex| & sample part file \\
    |cdocsdrf.tex| & sample redirection file \\
    |cdocsfn1.tex| & sample redirection file \\
    |cdocsfn2.tex| & sample redirection file \\
    |childdoc.pdf| & manual
\end{tabular}
\end{center}
%
The distribution consists of the files
|README.txt|, |childdoc.ins| and |childdoc.dtx|.
%
\begin{itemize}
\item
Run (pdf)\LaTeX{} on |childdoc.dtx|
to compile the manual |childdoc.pdf| (this file).
\item
Run \LaTeX{} on |childdoc.ins| to create the definitions file |childdoc.def|
and the sample |cdocsamp.tex| with include files
|cdocsch1.tex|, |cdocsch2.tex|, |cdocspt3.tex|, |cdocspt4.tex|,
|cdocsdrf.tex|, |cdocsfn1.tex|, |cdocsfn2.tex|.
Then copy the file |childdoc.def| to an appropriate directory of your \LaTeX{}
distribution, e.g.\ \textit{texmf-root}|/tex/latex/childdoc|.
\end{itemize}

%%%%%%%%%%%%%%%%%%%%%%%%%%%%%%%%%%%%%%%%%%%%%%%%%%%%%%%%%%%%%%%%%%%%%%%%%%%%%%%%
\subsection{Related CTAN Packages}

There are several other packages which offer a similar functionality:
%
\begin{itemize}
\item
The packages
\href{http://ctan.org/pkg/docmute}{\textsf{docmute}},
\href{http://ctan.org/pkg/includex}{\textsf{includex}} and
\href{http://ctan.org/pkg/standalone}{\textsf{standalone}}
provide commands to include only the document body of
a child file thus allowing both files to be compiled individually.
\item
The packages \href{http://ctan.org/pkg/subdocs}{\textsf{subdocs}}
and \href{http://ctan.org/pkg/subfiles}{\textsf{subfiles}}
provide structures in which the main and child documents can be
encapsulated and allowing them to be compiled individually.
The inclusion mechanism is different from the conventional |\include|.
\item
The package \href{http://ctan.org/pkg/combine}{\textsf{combine}}
is an elaborate solution to combine several documents into one.
\end{itemize}
%
See also the CTAN topic \href{http://ctan.org/topic/subdocs}{\textsf{subdocs}}
for further related packages.
The present package differs from the above solutions in that
a document structure constructed with the conventional |\include| mechanism
just needs two extra commands at the top of every file
such that all constituent files can be compiled individually.

%%%%%%%%%%%%%%%%%%%%%%%%%%%%%%%%%%%%%%%%%%%%%%%%%%%%%%%%%%%%%%%%%%%%%%%%%%%%%%%%
%\subsection{Feature Suggestions}
%
%The following is a list of features which may be useful for future
%versions of this package:
%%
%\begin{itemize}
%\item
%\ldots
%\end{itemize}

%%%%%%%%%%%%%%%%%%%%%%%%%%%%%%%%%%%%%%%%%%%%%%%%%%%%%%%%%%%%%%%%%%%%%%%%%%%%%%%%
\subsection{Revision History}

%%%%%%%%%%%%%%%%%%%%%%%%%%%%%%%%%%%%%%%%
\paragraph{v2.0:} 2018/12/30

\begin{itemize}
\item
immediate forward processing
\item
added |\childdocby| mechanism
\item
manual restructured
\end{itemize}

%%%%%%%%%%%%%%%%%%%%%%%%%%%%%%%%%%%%%%%%
\paragraph{v1.6:} 2018/01/17

\begin{itemize}
\item
application for development of include files
\item
corrections to manual
\end{itemize}

%%%%%%%%%%%%%%%%%%%%%%%%%%%%%%%%%%%%%%%%
\paragraph{v1.5:} 2017/05/21

\begin{itemize}
\item
more complete structuring introduced
\item
|\childdocof| introduced
\item
|\childdoc| renamed to |\childdocmain|
\item
|\childredirect| renamed to |\childdocforward| and |\childdocforwardprefix|
and functionality expanded
\end{itemize}

%%%%%%%%%%%%%%%%%%%%%%%%%%%%%%%%%%%%%%%%
\paragraph{v1.0:} 2017/04/27

\begin{itemize}
\item
manual and install package
\item
first version published on CTAN
\end{itemize}

%%%%%%%%%%%%%%%%%%%%%%%%%%%%%%%%%%%%%%%%
\paragraph{v0.6:} 2017/04/26

\begin{itemize}
\item
redirection mechanism added
\end{itemize}

%%%%%%%%%%%%%%%%%%%%%%%%%%%%%%%%%%%%%%%%
\paragraph{v0.5:} 2017/04/26

\begin{itemize}
\item
functionality in definition file
\end{itemize}


%%%%%%%%%%%%%%%%%%%%%%%%%%%%%%%%%%%%%%%%%%%%%%%%%%%%%%%%%%%%%%%%%%%%%%%%%%%%%%%%
%%%%%%%%%%%%%%%%%%%%%%%%%%%%%%%%%%%%%%%%%%%%%%%%%%%%%%%%%%%%%%%%%%%%%%%%%%%%%%%%
%%%%%%%%%%%%%%%%%%%%%%%%%%%%%%%%%%%%%%%%%%%%%%%%%%%%%%%%%%%%%%%%%%%%%%%%%%%%%%%%
\appendix

\settowidth\MacroIndent{\rmfamily\scriptsize 000\ }

 \DocInput{childdoc.dtx}

\end{document}
%</driver>
% \fi
%
% %%%%%%%%%%%%%%%%%%%%%%%%%%%%%%%%%%%%%%%%%%%%%%%%%%%%%%%%%%%%%%%%%%%%%%%%%%%%%%
% %%%%%%%%%%%%%%%%%%%%%%%%%%%%%%%%%%%%%%%%%%%%%%%%%%%%%%%%%%%%%%%%%%%%%%%%%%%%%%
% \section{Sample}
%\iffalse
%<*samplemain>
%\fi
%
% The following presents a sample document
% with two chapters, two parts, a title page,
% a compile flag as well as three forwarding files to set the flag.
% It consists of eight |.tex| files:
% \begin{center}
% \begin{tabular}{ll}
% |cdocsamp.tex|&main file\\
% |cdocsch1.tex|&include file for chapter 1\\
% |cdocsch2.tex|&include file for chapter 2\\
% |cdocspt3.tex|&include file for part 3\\
% |cdocspt4.tex|&include file for part 4\\
% |cdocsdrf.tex|&forwarding file for main file in draft mode\\
% |cdocsfi1.tex|&forwarding file for final version of chapter 1\\
% |cdocsfi2.tex|&forwarding file for final version of chapter 2\\
% \end{tabular}
% \end{center}
% Each of the eight files can be compiled directly by the \LaTeX{} compiler.
%
% %%%%%%%%%%%%%%%%%%%%%%%%%%%%%%%%%%%%%%
% \paragraph{Main File.}
%
% The main file is called |cdocsamp.tex|.
%
% Load the \textsf{childdoc} definitions and
% declare the filename for the main document:
%    \begin{macrocode}
\input{childdoc.def}
\childdocmain{}
%    \end{macrocode}

% Optional override for |\version| flag:
%    \begin{macrocode}
%%\ifchilddoc\else\providecommand{\version}{draft}\fi
%    \end{macrocode}

% Define the default values for the |\version| flag
% (|final| for the main file and |draft| for childs):
%    \begin{macrocode}
\ifchilddoc
\providecommand{\version}{draft}
\else
\providecommand{\version}{final}
\fi
%    \end{macrocode}

% Load the standard document class:
%    \begin{macrocode}
\documentclass[12pt]{article}
%    \end{macrocode}

% Start the document body:
%    \begin{macrocode}
\begin{document}
%    \end{macrocode}

% Declare a title page.
% Print title, part of document being processed and version flag:
%    \begin{macrocode}
\addtocounter{page}{-1}
\begin{center}
{\LARGE\bfseries{}childdoc example\par}
\vspace{1cm}
\ifchilddoc
\ifchilddocmanual part\else chapter\fi:
`\childdocname' of `\childdocjob'\par
\else
main document: `\childdocjob'\par
\fi
version: \version\par
\end{center}
\newpage
%    \end{macrocode}

% Manually include selected file,
% otherwise process as usual:
%    \begin{macrocode}
\ifchilddocmanual
\section*{part `\childdocname'}
\input{\childdocname}
\else
%    \end{macrocode}

% Include the two chapters:
%    \begin{macrocode}
\include{cdocsch1}
\include{cdocsch2}
%    \end{macrocode}

% Include the two parts unless only chapters should be displayed:
%    \begin{macrocode}
\ifchilddoc\else
\section{part three}
\input{cdocspt3}
\section{part four}
\input{cdocspt4}
\fi
%    \end{macrocode}

% Process as usual until here:
%    \begin{macrocode}
\fi
%    \end{macrocode}

% End of document body:
%    \begin{macrocode}
\end{document}
%    \end{macrocode}
%\iffalse
%</samplemain>
%\fi
%
% %%%%%%%%%%%%%%%%%%%%%%%%%%%%%%%%%%%%%%
% \paragraph{Chapter Include Files.}
%
% The include files are called |cdocsch1.tex| and |cdocsch2.tex|.
%
%\iffalse
%<*samplechap1|samplechap2>
%\fi

% Optional override for |\version| flag:
%    \begin{macrocode}
%%\providecommand{\version}{final}
%    \end{macrocode}

% Include the main document:
%    \begin{macrocode}
\input{childdoc.def}
\childdocof{cdocsamp}
%    \end{macrocode}

%\iffalse
%</samplechap1|samplechap2>
%\fi
%
%\iffalse
%<*samplechap1>
%\fi
% Some text for chapter 1:
%    \begin{macrocode}
\section{one}
some text in chapter one
%    \end{macrocode}

%\iffalse
%</samplechap1>
%\fi
% Some text for chapter 2:
%\iffalse
%<*samplechap2>
%\fi
%    \begin{macrocode}
\section{two}
more text in chapter two
%    \end{macrocode}

%\iffalse
%</samplechap2>
%\fi
%
% %%%%%%%%%%%%%%%%%%%%%%%%%%%%%%%%%%%%%%
% \paragraph{Part Include Files.}
%
% The include files are called |cdocspt3.tex| and |cdocspt4.tex|.
%
%\iffalse
%<*samplepart3|samplepart4>
%\fi

% Optional override for |\version| flag:
%    \begin{macrocode}
%%\providecommand{\version}{final}
%    \end{macrocode}

% Include the main document:
%    \begin{macrocode}
\input{childdoc.def}
\childdocby{cdocsamp}
%    \end{macrocode}

%\iffalse
%</samplepart3|samplepart4>
%\fi
%
%\iffalse
%<*samplepart3>
%\fi
% Some text for part 3:
%    \begin{macrocode}
some text in part three
%    \end{macrocode}

%\iffalse
%</samplepart3>
%\fi
% Some text for part 4:
%\iffalse
%<*samplepart4>
%\fi
%    \begin{macrocode}
more text in part four
%    \end{macrocode}

%\iffalse
%</samplepart4>
%\fi
%
% %%%%%%%%%%%%%%%%%%%%%%%%%%%%%%%%%%%%%%
% \paragraph{Forwarding for a Complete Draft.}
%
% The following forwarding file |cdocsdrf.tex|
% compiles the main document in draft mode:
%\iffalse
%<*sampledraft>
%\fi
%    \begin{macrocode}
\def\version{draft}
\input{childdoc.def}
\childdocforward{cdocsamp}
%    \end{macrocode}

%\iffalse
%</sampledraft>
%\fi
%
% %%%%%%%%%%%%%%%%%%%%%%%%%%%%%%%%%%%%%%
% \paragraph{Forwarding for Final Version of the Chapters.}
%
% The following forwarding files |cdocsfn1.tex| and |cdocsfn2.tex|
% (with identical content)
% compile the final versions of the child documents
% |cdocsch1.tex| and |cdocsch2.tex|, respectively:
%\iffalse
%<*samplefinal>
%\fi
%    \begin{macrocode}
\def\version{final}
\input{childdoc.def}
\childdocforwardprefix[cdocsamp]{cdocsfn}{cdocsch}
%    \end{macrocode}

%\iffalse
%</samplefinal>
%\fi
%
% %%%%%%%%%%%%%%%%%%%%%%%%%%%%%%%%%%%%%%
% \paragraph{Command Line Processing.}
%
% The following three command lines generate the output files
% |cdocscld|, |cdocscl1| and |cdocscl2|
% which should be identical to
% |cdocsdrf|, |cdocsch1| and |cdocsfn2|, respectively:
% \begin{center}
% \begin{tabular}{l}
% |latex -jobname cdocscld \|\\
% |  "\def\version{draft}\input{childdoc.def}\childdocforward{cdocsamp}"|\\
% |latex -jobname cdocscl1 \|\\
% |  "\input{childdoc.def}\childdocforward[cdocsamp]{cdocsch1}"|\\
% |latex -jobname cdocscl2 \|\\
% |  "\def\version{final}\input{childdoc.def}\childdocforward{cdocsch2}"|
% \end{tabular}
% \end{center}
% Note that the trailing backslash on each first line
% merely continues the input to the second line
% (for convenient cut ant paste).
% Furthermore, the command |latex| can be replaced by any
% of its alternative versions such as |pdflatex|.
%
% %%%%%%%%%%%%%%%%%%%%%%%%%%%%%%%%%%%%%%%%%%%%%%%%%%%%%%%%%%%%%%%%%%%%%%%%%%%%%%
% %%%%%%%%%%%%%%%%%%%%%%%%%%%%%%%%%%%%%%%%%%%%%%%%%%%%%%%%%%%%%%%%%%%%%%%%%%%%%%
% \section{Implementation}
%\iffalse
%<*package>
%\fi
%
% This section describes the definitions file |childdoc.def|.

% The definitions cannot be loaded using |\usepackage| or |\RequirePackage|
% which has a mechanism to prevent loading a style file more than once.
% When loading the definitions by means of |\input|
% multiple instances have to be prevented manually:
%\iffalse
%This code needs to be before the `\ProvidesFile' directive
%which is defined at the beginning of this file.
%Therefore it is also placed there and commented out here.
%</package>
%<*discard>
%\fi
%    \begin{macrocode}
\ifdefined\childdocmain\endinput\fi
%    \end{macrocode}
%\iffalse
%</discard>
%<*package>
%\fi
%
% \macro{\ifchilddoc}
% \macro{\ifchilddocmanual}
% The conditional |\ifchilddoc| tells whether a
% child (true) or main (false) document is being compiled.
% The conditional |\ifchilddocmanual| tells whether
% the |\includeonly| mechanism is used (false) or
% the selection of child files must be performed manually (true).
% The definitions initialise to false:
%    \begin{macrocode}
\newif\ifchilddoc
\newif\ifchilddocmanual
%    \end{macrocode}

% \macro{\childdocname}
% \macro{\childdocjob}
% The macro |\childdocname| stores the name of the main document
% to be compiled. The macro |\childdocjob| stores the name of
% the document on which the \LaTeX{} compiler was originally invoked.
% The content of |\jobname| cannot be compared
% to filenames specified in the source due to different catcodes.
% The following code rescans |\jobname|, stores the result
% in |\childdocname| and saves a copy in |\childdocjob|:
%    \begin{macrocode}
\edef\childdocname{\scantokens\expandafter{\jobname\noexpand}}
\let\childdocjob\childdocname
%    \end{macrocode}

% \macro{\childdocdisable}
% The macro |\childdocdisable| prevents the main file
% from being processed more than once.
% At this stage, the main document command |\childdocmain|
% is assumed to be called once again where it should do nothing.
% Any subsequent call to it should prevent
% a secondary processing of the main document
% It overwrites the forwarding commands
% |\childdocof| and |\childdocforward|
% with empty macros to prevent further inclusions of the main document:
%    \begin{macrocode}
\newcommand{\childdocdisable}
{
  \renewcommand{\childdocmain}[1]{\renewcommand{\childdocmain}[1]{\endinput}}
  \renewcommand{\childdocof}[1]{}
  \renewcommand{\childdocby}[2][]{}
  \renewcommand{\childdocforward}[2][]{}
  \renewcommand{\childdocdisable}{}
}
%    \end{macrocode}

% \macro{\childdocmain}
% The macro |\childdocmain| is to be called at the top of the main file
% with nothing or the main filename (without extension) as argument.
% First, it breaks loops.
% If the argument is not empty and does not match |\childdocname|
% (which is set by the first inclusion of |childdoc.def|),
% |\ifchilddoc| is set to true, |\includeonly| is applied to the child file
% and |\jobname| is set to the main file
% (for proper handling of |.aux| files):
%    \begin{macrocode}
\newcommand{\childdocmain}[1]
{
  \childdocdisable\childdocmain{}
  \if?#1?\else
    \begingroup
      \def\childdoctmp{#1}
      \ifx\childdoctmp\childdocname
        \def\childdoctmp{}
      \else
        \def\childdoctmp
        {
          \childdoctrue
          \includeonly{\childdocname}
          \def\childdocjob{#1}
          \def\jobname{#1}
        }
      \fi
      \expandafter
    \endgroup
    \childdoctmp
  \fi
}
%    \end{macrocode}

% \macro{\childdocof}
% The command |\childdocof| redirects
% compilation to the main file |#1|.
%    \begin{macrocode}
\newcommand{\childdocof}[1]
{
  \childdocdisable
  \childdoctrue
  \includeonly{\childdocname}
  \def\jobname{#1}
  \def\childdocjob{#1}
  \input{#1}
}
%    \end{macrocode}

% \macro{\childdocby}
% The command |\childdocby| ....
%    \begin{macrocode}
\newcommand{\childdocby}[2][]
{
  \childdocdisable
  \childdoctrue
  \childdocmanualtrue
  \if?#1?\else
    \def\jobname{#2}
  \fi
  \def\childdocjob{#2}
  \input{#2}
  \endinput
}
%    \end{macrocode}

% \macro{\childdocforward}
% The command |\childdocforward| redirects
% compilation to the main file or
% (if the optional argument is given) a child file.
% Parameters are set as if the main file
% or a child file starting with |\childdocof| was compiled.
% Then compilation is handed over to the main file:
%    \begin{macrocode}
\newcommand{\childdocforward}[2][]
{
  \begingroup
    \if?#1?
      \def\childdoctmp
      {
        \def\childdocname{#2}
        \def\childdocjob{#2}
        \def\jobname{#2}
        \input{#2}
        \endinput
      }
    \else
      \def\childdoctmp
      {
        \childdocdisable
        \def\childdocname{#2}
        \childdoctrue
        \includeonly{#2}
        \def\childdocjob{#1}
        \def\jobname{#1}
        \input{#1}
        \endinput
      }
    \fi
    \expandafter
  \endgroup
  \childdoctmp
}
%    \end{macrocode}

% \macro{\childdocforwardprefix}
% The command |\childdocforwardprefix| redirects
% compilation to the main or a child file by means of a pattern.
% The prefix |#1| in the current filename is replaced by |#2|
% and the suffix of the current filename is kept
% (it is assumed that the filename does not contain the substring `|~~~|'
% which is used as a delimiter).
% Compilation is handed over to the new file by |\childdocforward|:
%    \begin{macrocode}
\newcommand{\childdocforwardprefix}[3][]
{
  \begingroup
    \def\childdocextract #2##1~~~{\def\childdoctmp{\childdocforward[#1]{#3##1}}}
    \expandafter\childdocextract\childdocname~~~
    \expandafter
  \endgroup
  \childdoctmp
}
%    \end{macrocode}

% \macro{\childdoc}
% The deprecated macro |\childdoc| is a legacy version of |\childdocmain|:
%    \begin{macrocode}
\newcommand{\childdoc}{\childdocmain}
%    \end{macrocode}

% \macro{\childdocredirect}
% The deprecated macro |\childdocredirect| is a legacy version
% of |\childdocforward| and |\childdocforwardprefix|:
%    \begin{macrocode}
\newcommand{\childdocredirect}[2][]
{
  \begingroup
    \if?#1?
      \def\childdoctmp{\childdocforward{#2}}
    \else
      \def\childdoctmp{\childdocforwardprefix{#1}{#2}}
    \fi
    \expandafter
  \endgroup
  \childdoctmp
}
%    \end{macrocode}

%\iffalse
%</package>
%\fi
%
\endinput

\childdocforward{cdocsamp}
%    \end{macrocode}

%\iffalse
%</sampledraft>
%\fi
%
% %%%%%%%%%%%%%%%%%%%%%%%%%%%%%%%%%%%%%%
% \paragraph{Forwarding for Final Version of the Chapters.}
%
% The following forwarding files |cdocsfn1.tex| and |cdocsfn2.tex|
% (with identical content)
% compile the final versions of the child documents
% |cdocsch1.tex| and |cdocsch2.tex|, respectively:
%\iffalse
%<*samplefinal>
%\fi
%    \begin{macrocode}
\def\version{final}
% \iffalse
%
% childdoc.dtx Copyright (C) 2017-2018 Niklas Beisert
%
% This work may be distributed and/or modified under the
% conditions of the LaTeX Project Public License, either version 1.3
% of this license or (at your option) any later version.
% The latest version of this license is in
%   http://www.latex-project.org/lppl.txt
% and version 1.3 or later is part of all distributions of LaTeX
% version 2005/12/01 or later.
%
% This work has the LPPL maintenance status `maintained'.
%
% The Current Maintainer of this work is Niklas Beisert.
%
% This work consists of the files childdoc.dtx and childdoc.ins
% and the derived files childdoc.def and cdocsamp.tex with
% cdocsch1.tex, cdocsch2.tex, cdocsdrf.tex, cdocsfn1.tex, cdocsfn2.tex.
%
%<package>\ifdefined\childdocmain\endinput\fi
%<package>\ProvidesFile{childdoc.def}[2018/12/30 v2.0 child document driver]
%<samplemain>\ProvidesFile{cdocsamp.tex}[2018/12/30 v2.0 sample for childdoc]
%<*driver>
%\ProvidesFile{childdoc.drv}[2018/12/30 v2.0 childdoc reference manual file]
\PassOptionsToClass{10pt,a4paper}{article}
\documentclass{ltxdoc}

\usepackage[margin=35mm]{geometry}
\usepackage{hyperref}
\usepackage{hyperxmp}
\usepackage[usenames]{color}

\hypersetup{colorlinks=true}
\hypersetup{pdfstartview=FitH}
\hypersetup{pdfpagemode=UseNone}
\hypersetup{pdfsource={}}
\hypersetup{pdflang={en-UK}}
\hypersetup{pdfcopyright={Copyright 2017-2018 Niklas Beisert.
  This work may be distributed and/or modified under the
  conditions of the LaTeX Project Public License, either version 1.3
  of this license or (at your option) any later version.}}
\hypersetup{pdflicenseurl={http://www.latex-project.org/lppl.txt}}
\hypersetup{pdfcontactaddress={ETH Zurich, ITP, HIT K,
  Wolfgang-Pauli-Strasse 27}}
\hypersetup{pdfcontactpostcode={8093}}
\hypersetup{pdfcontactcity={Zurich}}
\hypersetup{pdfcontactcountry={Switzerland}}
\hypersetup{pdfcontactemail={nbeisert@itp.phys.ethz.ch}}
\hypersetup{pdfcontacturl={http://people.phys.ethz.ch/\xmptilde nbeisert/}}

\newcommand{\secref}[1]{\hyperref[#1]{section \ref*{#1}}}

\parskip1ex
\parindent0pt
\let\olditemize\itemize
\def\itemize{\olditemize\parskip0pt}

\begin{document}

\title{The \textsf{childdoc} Package}
\hypersetup{pdftitle={The childdoc Package}}
\author{Niklas Beisert\\[2ex]
  Institut f\"ur Theoretische Physik\\
  Eidgen\"ossische Technische Hochschule Z\"urich\\
  Wolfgang-Pauli-Strasse 27, 8093 Z\"urich, Switzerland\\[1ex]
  \href{mailto:nbeisert@itp.phys.ethz.ch}
  {\texttt{nbeisert@itp.phys.ethz.ch}}}
\hypersetup{pdfauthor={Niklas Beisert}}
\hypersetup{pdfsubject={Manual for the LaTeX2e Package childdoc}}
\date{30 December 2018, \textsf{v2.0}}
\maketitle

\begin{abstract}\noindent
\textsf{childdoc} is a \LaTeXe{} package
that enables the direct compilation
of document sections included by |\include|
to individual files.
\end{abstract}

\begingroup
\parskip0ex
\tableofcontents
\endgroup

%%%%%%%%%%%%%%%%%%%%%%%%%%%%%%%%%%%%%%%%%%%%%%%%%%%%%%%%%%%%%%%%%%%%%%%%%%%%%%%%
%%%%%%%%%%%%%%%%%%%%%%%%%%%%%%%%%%%%%%%%%%%%%%%%%%%%%%%%%%%%%%%%%%%%%%%%%%%%%%%%
\section{Introduction}

\LaTeX{} provides a mechanism to structure a large document (such as a book)
into a main file and several child files (containing the chapters)
using the |\include| command.
This mechanism is beneficial for documents
which span hundreds of pages in order to
make the source file(s) more manageable.
Moreover, compilation can be restricted to
selected child files by means of the |\includeonly| command.
The latter feature can be used to reduce the compilation time while editing
(this was significantly more useful in the earlier days of \LaTeX{})
or to generate a smaller document which is easier to navigate.
Another application of |\includeonly| is to generate
documents consisting of selected parts of the complete document.

However, there are a few drawbacks of the plain |\include| mechanism:
\begin{itemize}
\item
The child files cannot be compiled on their own,
they can only be compiled via the main file.
A naive editing environment
(such as a text editor with an option
to have the current file processed by \LaTeX)
may require one to switch to the main file before compiling;
attempting to compile the child file produces errors.
\item
The main file must be modified (each time)
to adjust the |\includeonly| command
to the present needs. This easily leaves the main file in a messy state.
\item
The generated document will always carry the filename
of the main document. This is inconvenient if
several child files are to be compiled and
to be kept for distribution.
\end{itemize}

The present package provides a simple interface
to make child files individually compilable by \LaTeX{}.
Compiling a child file then has the same effect as compiling
the main file with an |\includeonly| command
to select the appropriate child.
Moreover the generated document will carry the name of the child
rather than the main file.
This resolves all three above issues.

This feature is meant to make the editing of books,
thesis documents and lecture notes somewhat more convenient.
However, the package can also be used efficiently for
composing a series of documents (such as exercise sheets)
which are typically distributed individually.
It then assists the author in generating the individual documents
(potentially in different versions)
as well as a document containing the collected series.
Another application is in developing style files
or other kinds of included material
where compilation of the style file could redirect
to a sample or test file.

%%%%%%%%%%%%%%%%%%%%%%%%%%%%%%%%%%%%%%%%%%%%%%%%%%%%%%%%%%%%%%%%%%%%%%%%%%%%%%%%
%%%%%%%%%%%%%%%%%%%%%%%%%%%%%%%%%%%%%%%%%%%%%%%%%%%%%%%%%%%%%%%%%%%%%%%%%%%%%%%%
\section{Usage}

First of all, the package \textsf{childdoc} is \emph{not} a standard
\LaTeXe{} |.sty| style file! Therefore it needs to be invoked in
a non-standard way.

%%%%%%%%%%%%%%%%%%%%%%%%%%%%%%%%%%%%%%%%%%%%%%%%%%%%%%%%%%%%%%%%%%%%%%%%%%%%%%%%
\subsection{Included Files}
\label{sec:include}

%%%%%%%%%%%%%%%%%%%%%%%%%%%%%%%%%%%%%%%%
\DescribeMacro{\childdocmain}
To use the package, add the commands
\begin{center}
\begin{tabular}{l}
|\input{childdoc.def}|\\
|\childdocmain{}|\\
\end{tabular}
\end{center}
at the very top of the main \LaTeX{} file,
in particular \emph{before} the |\documentclass| statement!
The argument of |\childdocmain| should be left empty
(but it must be present).

%%%%%%%%%%%%%%%%%%%%%%%%%%%%%%%%%%%%%%%%
\DescribeMacro{\childdocof}
Furthermore, add the commands
\begin{center}
\begin{tabular}{l}
|\input{childdoc.def}|\\
|\childdocof{|\textit{main}|}|\\
\end{tabular}
\end{center}
at the top of every child file \textit{child}
which is included by |\include{|\textit{child}|}|
from within the main file
(or at least for those files to be compiled individually).
The argument \textit{main} must be the filename of the main file.

There are a couple of
considerations in setting up the main and child documents:

%%%%%%%%%%%%%%%%%%%%%%%%%%%%%%%%%%%%%%%%
\paragraph{Restrictions.}

Please note the following restrictions:
\begin{itemize}
\item
|\childdocmain| must be called with one argument \textit{main}
to ensure compatibility with earlier version of the package.
It must either be empty (|\childdocmain{}|)
or precisely match the filename of the main file in which it is specified.
See \secref{sec:detection} for further information.
\item
The filename \textit{main} must be specified without the |.tex| extension.
\item
The filename \textit{main} is case sensitive
(even in case-insensitive file systems)
due to internal string comparison.
\item
The argument \textit{main} should be fully expanded, it cannot be a macro.
\item
Subdirectories and special characters should be avoided in filenames.
\item
The command |\childdocmain{|\textit{main}|}| must be followed by a whitespace.
It should not be followed immediately by another command
or by a comment mark `|%|'.
This is because the \TeX{} parser reads the token immediately following
the argument of |\childdocmain| and puts it
at the beginning of every child section;
however, a white\-space is ignored.
\end{itemize}

%%%%%%%%%%%%%%%%%%%%%%%%%%%%%%%%%%%%%%%%
\paragraph{Content of Main File.}

It is advisable to place all content in the child files included by |\include|.
Any output contained in the main file will appear in all child documents
unless suppressed manually;
it cannot be suppressed automatically by the |\includeonly| directive
and thus should normally be avoided.
A method to include some content in the main file
by means of conditional processing is described in \secref{sec:conditional}.

%%%%%%%%%%%%%%%%%%%%%%%%%%%%%%%%%%%%%%%%
\paragraph{Page Numbering.}

When only a part of the document is compiled,
the appropriate numbering of pages
(as well as other status parameters)
is determined from the |.aux| files.
The latter contain information from previous passes.
However this information needs to propagate through
all intermediate child documents.
Therefore the page numbering in child documents may well
be inconsistent until the complete document is compiled at least once.

A useful (if unconventional) way to always ensure a consistent
page numbering is to restart the numbering in each child document
and denote the pages by `\textit{child}|.|\textit{page}'
where \textit{child} represents the chapter/section number of the child file.
This can be achieved by the command
|\numberwithin{page}{|\textit{child}|}|
of the \textsf{amsmath} package
where \textit{child} can be |chapter| or |section|
depending on the chosen structuring.
Alternatively, one can modify the macro |\thepage| appropriately
and reset the counter |page| at the start of each child file.

%%%%%%%%%%%%%%%%%%%%%%%%%%%%%%%%%%%%%%%%%%%%%%%%%%%%%%%%%%%%%%%%%%%%%%%%%%%%%%%%
\subsection{Conditional Processing}
\label{sec:conditional}

The package provides a mechanism to compile different versions
of a document. To customise the versions further some conditional processing
can come in handy to distinguish which version is being compiled.
The package provides two macros to describe the compilation context:

%%%%%%%%%%%%%%%%%%%%%%%%%%%%%%%%%%%%%%%%
\DescribeMacro{\ifchilddoc}
The conditional |\ifchilddoc| distinguishes between the compilation of
child documents and the main document:
%
\begin{center}
|\ifchilddoc |\textit{child-code}| |[|\||else |\textit{main-code}]| \||fi|
\end{center}

%%%%%%%%%%%%%%%%%%%%%%%%%%%%%%%%%%%%%%%%
\DescribeMacro{\childdocname}
\DescribeMacro{\childdocjob}
The macro |\childdocname| contains the filename (without extension)
of the main or child file being processed.
Note that |\childdocjob| will always contain the name of the main file.

%%%%%%%%%%%%%%%%%%%%%%%%%%%%%%%%%%%%%%%%
\paragraph{Title Page.}

Conditional processing can be used to include a title or banner page
in the main document when proper precautions are taken.
Importantly, the code in the main file should ensure that the page counter
(as well as other status parameters which are stored in the |.aux| files)
takes the same value after the conditional processing.
Otherwise the page numbers may take divergent values
depending on which part is compiled.

For example, a title page could be declared by:
%
\begin{center}
\begin{tabular}{l}
|\ifchilddoc\||else|\\
|\addtocounter{page}{-1}|\\
\textit{code for title page}\\
|\newpage|\\
|\||fi|
\end{tabular}
\end{center}
%
A banner page for the child documents can be generated by:
%
\begin{center}
\begin{tabular}{l}
|\ifchilddoc|\\
|\addtocounter{page}{-1}|\\
\textit{code for banner page}\\
|\newpage|\\
|\||fi|
\end{tabular}
\end{center}
%
Here one could write a message such as:
\begin{center}
|This is the part \childdocname{} of \childdocjob{}.|
\end{center}

%%%%%%%%%%%%%%%%%%%%%%%%%%%%%%%%%%%%%%%%%%%%%%%%%%%%%%%%%%%%%%%%%%%%%%%%%%%%%%%%
\subsection{Flags}
\label{sec:flags}

The package makes it easy to generate different versions
of the main or child documents.
To this end compilation flags can be defined
and assigned different default values.
They will be particularly useful in conjunction
with the forwarding mechanism described in \secref{sec:forward}.

For example, it may be useful to have a flag |\version|
which can be set to |draft| or |final|.
The document source will contain some conditional code
depending on the value of |\version|.
Suppose further, the flag should default to |final| for the main file
and to |draft| for child files
which is a natural assignment for editing the document.
This is achieved by placing the following code
in the preamble of the main document
(below the |\childdocmain| directive):
%
\begin{center}
\begin{tabular}{l}
|\ifchilddoc|\\
|\providecommand{\version}{draft}|\\
|\||else|\\
|\providecommand{\version}{final}|\\
|\||fi|
\end{tabular}
\end{center}
%
The definition by |\providecommand| makes sure
that previous definitions are not overwritten.
Further statements |\providecommand{\version}{...}|
can thus be added before the above code to override it.

For the main file, one might add a line
(between |\childdocmain| and the above block)
%
\begin{center}
|%\ifchilddoc\||else\providecommand{\version}{draft}\||fi|
\end{center}
%
which can be uncommented to produce a draft version.
Likewise one can add a line to the very top of a child file
(above the |\childdocof{|\textit{main}|}| directive)
%
\begin{center}
|%\providecommand{\version}{final}|
\end{center}
%
which can be uncommented to produce the final version of this child document.

%%%%%%%%%%%%%%%%%%%%%%%%%%%%%%%%%%%%%%%%%%%%%%%%%%%%%%%%%%%%%%%%%%%%%%%%%%%%%%%%
\subsection{Forwarding}
\label{sec:forward}

Different versions of the main or child documents
using compilation flags as described in \secref{sec:flags}
can be (permanently) stored in different files
for convenient compilation, viewing and distribution.
To this end, the package defines a command
to pass on compilation to a different file:

%%%%%%%%%%%%%%%%%%%%%%%%%%%%%%%%%%%%%%%%
\DescribeMacro{\childdocforward}
The command |\childdocforward| redirects processing to
another source file:
%
\begin{center}
\begin{tabular}{l}
|\input{childdoc.def}|\\
|\childdocforward[|\textit{main}|]{|\textit{dest}|}|\\
\end{tabular}
\end{center}
%
The argument \textit{dest} is the destination file
(without extension).
It should be the main file or one of the child files.
Note that further \textsf{childdoc} directives
such as |\childdocof| and |\childdocforward|
in the indicated file will be processed in this form.
The optional argument \textit{main}
passes on directly to the main file \textit{main}
while pretending to compile the child \textit{dest}.
This form behaves as if \textit{dest}
issues |\childdocof{|\textit{main}|}| right away,
and no further \textsf{childdoc} directives will be processed.

%%%%%%%%%%%%%%%%%%%%%%%%%%%%%%%%%%%%%%%%
\DescribeMacro{\...prefix}
In the alternative form |\childdocforwardprefix|,
%
\begin{center}
\begin{tabular}{l}
|\input{childdoc.def}|\\
|\childdocforwardprefix[|\textit{main}|]{|\textit{prefix}|}{|\textit{dest}|}|
\end{tabular}
\end{center}
%
the destination file is determined by a pattern
depending on the current file:
To make this work, the current file must be called
`{\textit{prefix}\hspace{0.2em}\textit{suffix}}'
with \textit{prefix} matching precisely the argument.
Processing is then passed on to the file
`{\textit{dest}\hspace{0.2em}\textit{suffix}}'.
Surely, the same effect is achieved by
directly specifying the
argument `{\textit{dest}\hspace{0.2em}\textit{suffix}}'
in the first form.
However, that requires to set up a different file
for each child. With the alternative form of the command
all these files can have exactly the same content
which simplifies setting them up and maintaining them.

For example, the following file |draft.tex|
with a compilation flag |\version| as described in \secref{sec:flags}
compiles the main document as a draft:
%
\begin{center}
\begin{tabular}{l}
|\def\version{draft}|\\
|\input{childdoc.def}|\\
|\childdocforward{|\textit{main}|}|
\end{tabular}
\end{center}
%
Likewise, the following files |final|\textit{nn}|.tex|
compile the final version of the child document
|child|\textit{nn}|.tex|:
%
\begin{center}
\begin{tabular}{l}
|\def\version{final}|\\
|\input{childdoc.def}|\\
|\childdocforwardprefix{final}{child}|
\end{tabular}
\end{center}
%

Note that when several versions of a main file and/or of each child file
are to be generated, it may be convenient to set up a |Makefile| or
shell script to automatise the process.

%%%%%%%%%%%%%%%%%%%%%%%%%%%%%%%%%%%%%%%%%%%%%%%%%%%%%%%%%%%%%%%%%%%%%%%%%%%%%%%%
\subsection{Command Line Processing}
\label{sec:commandline}

The effect of redirection files can also be achieved by invoking
the \LaTeX{} compiler with a more elaborate command line.
Most conveniently this should be done as part
of a shell script or a |Makefile|.

When using \textsf{childdoc} in the main file, the following
command lines effectively perform a redirection
(note that depending on the shell being used,
backslashes may have to be doubled: `|\|' $\to$ `|\\|'):
%
\begin{center}
|... -jobname "|\textit{target}|" |\\|"|[\textit{flags}]%
|\input{childdoc.def}\childdocforward[|\textit{main}|]{|\textit{dest}|}"|
\end{center}
%
Here \textit{target} is the name of the output file,
\textit{main} is the name of the main file
and \textit{dest} is the name of the main or child file to be processed
(all filenames without extensions).
The optional argument \textit{main} can be omitted
if \textit{main} matches \textit{dest}.
Optionally, compilation \textit{flags} can be defined via |\def| commands.
This command line makes the \TeX{} engine believe
it is compiling the file \textit{target}
whose content is specified as the latter parameter.
The provided code then forwards the processing to
\textit{main} or \textit{dest} as described in \secref{sec:forward}.

%%%%%%%%%%%%%%%%%%%%%%%%%%%%%%%%%%%%%%%%%%%%%%%%%%%%%%%%%%%%%%%%%%%%%%%%%%%%%%%%
\subsection{Include by Input}
\label{sec:input}

Including child documents by |\include| has some restrictions by design.
Most notably, the content of a child document always occupies
its own set of pages; pages cannot be shared between child documents.
Usually, this behaviour makes perfect sense
because each child document contain an essential part of the document.
However, in some situations it may be desirable to compose
a document from a collection of parts
without having mandatory page breaks between then.
For this case, the package
provides a mechanism to include parts
by |\input| which can also be processed individually.
However, by construction this mechanism
requires manual handling of the content to be output.

%%%%%%%%%%%%%%%%%%%%%%%%%%%%%%%%%%%%%%%%
\DescribeMacro{\ifchilddocmanual}
The main file should be prepared as usual, see \secref{sec:include}.
However, the document body must make a distinction
between processing of an individual part and of the main document, e.g.:
%
\begin{center}
\begin{tabular}{l}
|\ifchilddocmanual|\\
|\input{\childdocname}|\\
|\||else|\\
\textit{document body with }|\input{|\textit{part}|}|\\
|\||fi|
\end{tabular}
\end{center}
%
The conditional |\ifchilddocmanual| is true whenever
a part to be included by |\input| is being compiled,
and the name of the part is stored in |\childdocname|.

%%%%%%%%%%%%%%%%%%%%%%%%%%%%%%%%%%%%%%%%
\DescribeMacro{\childdocby}
Each part to be included by |\input| should start with:
%
\begin{center}
\begin{tabular}{l}
|\input{childdoc.def}|\\
|\childdocby{|\textit{main}|}|\\
\end{tabular}
\end{center}
%
The directive |\childdocby| is similar to |\childdocof|
described in \secref{sec:include},
but the subsequent selection of content must be done manually.
To that end, both |\ifchilddoc| and |\ifchilddocmanual|
will be true upon processing of a part,
and the name of the part is stored in |\childdocname|.
Note that |\jobname| will be set to the filename of the current part
so that each part receives an individual |.aux| file
that does not interfere with the |.aux| file(s) of the main document.
This behaviour can be altered by the alternative form
|\childdocby[*]{|\textit{main}|}| (with a non-empty optional argument)
which uses the |.aux| file of the main document
by setting |\jobname| to \textit{main}.

%%%%%%%%%%%%%%%%%%%%%%%%%%%%%%%%%%%%%%%%%%%%%%%%%%%%%%%%%%%%%%%%%%%%%%%%%%%%%%%%
\subsection{Driver Development}
\label{sec:driver}

The \textsf{childdoc} mechanism can also be use for the development
of definition files such as \LaTeX{} styles or classes.
This case differs from the above setup with multiple parts
included by |\include| in that no |\includeonly| should be invoked.
This can be achieved by starting the include file
(before |\ProvidesPackage|) with:
%
\begin{center}
\begin{tabular}{l}
|\input{childdoc.def}|\\
|\childdocforward{|\textit{main}|}|\\
\end{tabular}
\end{center}
%
or alternatively with:
%
\begin{center}
\begin{tabular}{l}
|\input{childdoc.def}|\\
|\childdocby{|\textit{main}|}|\\
\end{tabular}
\end{center}
%
Both forms have slightly different effects as described above.
The main file is prepared as usual, see \secref{sec:include}.

%%%%%%%%%%%%%%%%%%%%%%%%%%%%%%%%%%%%%%%%%%%%%%%%%%%%%%%%%%%%%%%%%%%%%%%%%%%%%%%%
\subsection{Legacy Detection}
\label{sec:detection}

The directive |\childdocmain| in the main file can detect
whether the complete document or merely a child is to be compiled
even without using the directive |\childdocof|.
This method is deprecated because it is less robust
and there is no compelling reason to use it;
it is merely provided for backward compatibility
and it may be removed in future versions.

If the detection mechanism is to be used,
it is mandatory to correctly specify
the filename of the main file as the argument of |\childdocmain|:
%
\begin{center}
\begin{tabular}{l}
|\input{childdoc.def}|\\
|\childdocmain{|\textit{main}|}|\\
\end{tabular}
\end{center}
%
If |\jobname| does not match the argument \textit{main} of |\childdocmain|,
it is assumed that |\jobname| points to the child file to be compiled.
When using |\childdocmain| with the main file specified as argument,
it suffices to start a child file
with just |\input{|\textit{main}|}|
without loading of the package and using |\childdocof|.
If instead all processing is done
with the appropriate \textsf{childdoc} directives,
the argument of \textit{main} of |\childdocmain| can be empty.

An alternative version of the command line processing described
in \secref{sec:commandline} using the detection mechanism reads:
%
\begin{center}
|... -jobname "|\textit{target}|" "|[\textit{flags}]%
[|\def\jobname{|\textit{dest}|}|]|\input{|\textit{main}|}"|
\end{center}

%%%%%%%%%%%%%%%%%%%%%%%%%%%%%%%%%%%%%%%%%%%%%%%%%%%%%%%%%%%%%%%%%%%%%%%%%%%%%%%%
\subsection{Manual Code}
\label{sec:manual}

In case one cannot be certain whether the definitions file |childdoc.def|
is installed on the target \TeX{} distribution
and one prefers not to ship it,
it is conceivable to paste a few relevant commands into the sources.

To that end, drop all statements |\input{childdoc.def}|
and perform the replacements as outlined below.
Instead of |\childdocmain{|\textit{main}|}| add the following code
to the top of the main file:
%
\begin{center}
\begin{tabular}{l}
|\||ifdefined\childdocname\endinput\||fi\newif\ifchilddoc|\\
|\edef\childdocname{\scantokens\expandafter{\jobname\noexpand}}|\\
|\def\childdocmain{|\textit{main}|}\||ifx\childdocmain\childdocname\||else|\\
|\childdoctrue\includeonly{\childdocname}\let\jobname\childdocmain\||fi|\\
\end{tabular}
\end{center}
%
Instead of |\childdocof{|\textit{main}|}| just include the main file
at the top of each child file:
%
\begin{center}
|\input{|\textit{main}|}|
\end{center}
%
A simple redirection |\childdocforward{|\textit{dest}|}| is achieved by:
%
\begin{center}
|\def\jobname{|\textit{dest}|}\input{\jobname}|
\end{center}
%
The redirection with prefix
|\childdocforwardprefix[|\textit{prefix}|]{|\textit{dest}|}|
is accomplished by:
%
\begin{center}
\begin{tabular}{l}
|{\edef\jobname{\scantokens\expandafter{\jobname\noexpand}}|\\
|\def\redirectjob |\textit{prefix}|#1~~~{\gdef\jobname{|\textit{dest}|#1}}|\\
|\expandafter\redirectjob\jobname~~~}\input{\jobname}|
\end{tabular}
\end{center}

In an alternative approach,
child documents can be compiled by a specific command line
without additional code or specific definitions:
%
\begin{center}
|... -jobname "|\textit{target}|" "|[\textit{flags}]%
|\includeonly{|\textit{dest}|}\input{|\textit{main}|}"|
\end{center}
%

%%%%%%%%%%%%%%%%%%%%%%%%%%%%%%%%%%%%%%%%%%%%%%%%%%%%%%%%%%%%%%%%%%%%%%%%%%%%%%%%
%%%%%%%%%%%%%%%%%%%%%%%%%%%%%%%%%%%%%%%%%%%%%%%%%%%%%%%%%%%%%%%%%%%%%%%%%%%%%%%%
\section{Information}

%%%%%%%%%%%%%%%%%%%%%%%%%%%%%%%%%%%%%%%%%%%%%%%%%%%%%%%%%%%%%%%%%%%%%%%%%%%%%%%%
\subsection{Copyright}

Copyright \copyright{} 2017--2018 Niklas Beisert

This work may be distributed and/or modified under the
conditions of the \LaTeX{} Project Public License, either version 1.3
of this license or (at your option) any later version.
The latest version of this license is in
  \url{http://www.latex-project.org/lppl.txt}
and version 1.3 or later is part of all distributions of \LaTeX{}
version 2005/12/01 or later.

This work has the LPPL maintenance status `maintained'.

The Current Maintainer of this work is Niklas Beisert.

This work consists of the files |README.txt|, |childdoc.ins| and |childdoc.dtx|
as well as the derived files |childdoc.def|, |cdocsamp.tex|
with |cdocsch1.tex|, |cdocsch2.tex|, |cdocspt3.tex|, |cdocspt4.tex|,
|cdocsdrf.tex|, |cdocsfn1.tex|, |cdocsfn2.tex|
as well as |childdoc.pdf|.

%%%%%%%%%%%%%%%%%%%%%%%%%%%%%%%%%%%%%%%%%%%%%%%%%%%%%%%%%%%%%%%%%%%%%%%%%%%%%%%%
\subsection{Files and Installation}

The package consists of the files:
%
\begin{center}
\begin{tabular}{ll}
    |README.txt|   & readme file \\
    |childdoc.ins| & installation file \\
    |childdoc.dtx| & source file \\
    |childdoc.def| & definition file \\
    |cdocsamp.tex| & sample main file \\
    |cdocsch1.tex| & sample include file \\
    |cdocsch2.tex| & sample include file \\
    |cdocspt3.tex| & sample part file \\
    |cdocspt4.tex| & sample part file \\
    |cdocsdrf.tex| & sample redirection file \\
    |cdocsfn1.tex| & sample redirection file \\
    |cdocsfn2.tex| & sample redirection file \\
    |childdoc.pdf| & manual
\end{tabular}
\end{center}
%
The distribution consists of the files
|README.txt|, |childdoc.ins| and |childdoc.dtx|.
%
\begin{itemize}
\item
Run (pdf)\LaTeX{} on |childdoc.dtx|
to compile the manual |childdoc.pdf| (this file).
\item
Run \LaTeX{} on |childdoc.ins| to create the definitions file |childdoc.def|
and the sample |cdocsamp.tex| with include files
|cdocsch1.tex|, |cdocsch2.tex|, |cdocspt3.tex|, |cdocspt4.tex|,
|cdocsdrf.tex|, |cdocsfn1.tex|, |cdocsfn2.tex|.
Then copy the file |childdoc.def| to an appropriate directory of your \LaTeX{}
distribution, e.g.\ \textit{texmf-root}|/tex/latex/childdoc|.
\end{itemize}

%%%%%%%%%%%%%%%%%%%%%%%%%%%%%%%%%%%%%%%%%%%%%%%%%%%%%%%%%%%%%%%%%%%%%%%%%%%%%%%%
\subsection{Related CTAN Packages}

There are several other packages which offer a similar functionality:
%
\begin{itemize}
\item
The packages
\href{http://ctan.org/pkg/docmute}{\textsf{docmute}},
\href{http://ctan.org/pkg/includex}{\textsf{includex}} and
\href{http://ctan.org/pkg/standalone}{\textsf{standalone}}
provide commands to include only the document body of
a child file thus allowing both files to be compiled individually.
\item
The packages \href{http://ctan.org/pkg/subdocs}{\textsf{subdocs}}
and \href{http://ctan.org/pkg/subfiles}{\textsf{subfiles}}
provide structures in which the main and child documents can be
encapsulated and allowing them to be compiled individually.
The inclusion mechanism is different from the conventional |\include|.
\item
The package \href{http://ctan.org/pkg/combine}{\textsf{combine}}
is an elaborate solution to combine several documents into one.
\end{itemize}
%
See also the CTAN topic \href{http://ctan.org/topic/subdocs}{\textsf{subdocs}}
for further related packages.
The present package differs from the above solutions in that
a document structure constructed with the conventional |\include| mechanism
just needs two extra commands at the top of every file
such that all constituent files can be compiled individually.

%%%%%%%%%%%%%%%%%%%%%%%%%%%%%%%%%%%%%%%%%%%%%%%%%%%%%%%%%%%%%%%%%%%%%%%%%%%%%%%%
%\subsection{Feature Suggestions}
%
%The following is a list of features which may be useful for future
%versions of this package:
%%
%\begin{itemize}
%\item
%\ldots
%\end{itemize}

%%%%%%%%%%%%%%%%%%%%%%%%%%%%%%%%%%%%%%%%%%%%%%%%%%%%%%%%%%%%%%%%%%%%%%%%%%%%%%%%
\subsection{Revision History}

%%%%%%%%%%%%%%%%%%%%%%%%%%%%%%%%%%%%%%%%
\paragraph{v2.0:} 2018/12/30

\begin{itemize}
\item
immediate forward processing
\item
added |\childdocby| mechanism
\item
manual restructured
\end{itemize}

%%%%%%%%%%%%%%%%%%%%%%%%%%%%%%%%%%%%%%%%
\paragraph{v1.6:} 2018/01/17

\begin{itemize}
\item
application for development of include files
\item
corrections to manual
\end{itemize}

%%%%%%%%%%%%%%%%%%%%%%%%%%%%%%%%%%%%%%%%
\paragraph{v1.5:} 2017/05/21

\begin{itemize}
\item
more complete structuring introduced
\item
|\childdocof| introduced
\item
|\childdoc| renamed to |\childdocmain|
\item
|\childredirect| renamed to |\childdocforward| and |\childdocforwardprefix|
and functionality expanded
\end{itemize}

%%%%%%%%%%%%%%%%%%%%%%%%%%%%%%%%%%%%%%%%
\paragraph{v1.0:} 2017/04/27

\begin{itemize}
\item
manual and install package
\item
first version published on CTAN
\end{itemize}

%%%%%%%%%%%%%%%%%%%%%%%%%%%%%%%%%%%%%%%%
\paragraph{v0.6:} 2017/04/26

\begin{itemize}
\item
redirection mechanism added
\end{itemize}

%%%%%%%%%%%%%%%%%%%%%%%%%%%%%%%%%%%%%%%%
\paragraph{v0.5:} 2017/04/26

\begin{itemize}
\item
functionality in definition file
\end{itemize}


%%%%%%%%%%%%%%%%%%%%%%%%%%%%%%%%%%%%%%%%%%%%%%%%%%%%%%%%%%%%%%%%%%%%%%%%%%%%%%%%
%%%%%%%%%%%%%%%%%%%%%%%%%%%%%%%%%%%%%%%%%%%%%%%%%%%%%%%%%%%%%%%%%%%%%%%%%%%%%%%%
%%%%%%%%%%%%%%%%%%%%%%%%%%%%%%%%%%%%%%%%%%%%%%%%%%%%%%%%%%%%%%%%%%%%%%%%%%%%%%%%
\appendix

\settowidth\MacroIndent{\rmfamily\scriptsize 000\ }

 \DocInput{childdoc.dtx}

\end{document}
%</driver>
% \fi
%
% %%%%%%%%%%%%%%%%%%%%%%%%%%%%%%%%%%%%%%%%%%%%%%%%%%%%%%%%%%%%%%%%%%%%%%%%%%%%%%
% %%%%%%%%%%%%%%%%%%%%%%%%%%%%%%%%%%%%%%%%%%%%%%%%%%%%%%%%%%%%%%%%%%%%%%%%%%%%%%
% \section{Sample}
%\iffalse
%<*samplemain>
%\fi
%
% The following presents a sample document
% with two chapters, two parts, a title page,
% a compile flag as well as three forwarding files to set the flag.
% It consists of eight |.tex| files:
% \begin{center}
% \begin{tabular}{ll}
% |cdocsamp.tex|&main file\\
% |cdocsch1.tex|&include file for chapter 1\\
% |cdocsch2.tex|&include file for chapter 2\\
% |cdocspt3.tex|&include file for part 3\\
% |cdocspt4.tex|&include file for part 4\\
% |cdocsdrf.tex|&forwarding file for main file in draft mode\\
% |cdocsfi1.tex|&forwarding file for final version of chapter 1\\
% |cdocsfi2.tex|&forwarding file for final version of chapter 2\\
% \end{tabular}
% \end{center}
% Each of the eight files can be compiled directly by the \LaTeX{} compiler.
%
% %%%%%%%%%%%%%%%%%%%%%%%%%%%%%%%%%%%%%%
% \paragraph{Main File.}
%
% The main file is called |cdocsamp.tex|.
%
% Load the \textsf{childdoc} definitions and
% declare the filename for the main document:
%    \begin{macrocode}
\input{childdoc.def}
\childdocmain{}
%    \end{macrocode}

% Optional override for |\version| flag:
%    \begin{macrocode}
%%\ifchilddoc\else\providecommand{\version}{draft}\fi
%    \end{macrocode}

% Define the default values for the |\version| flag
% (|final| for the main file and |draft| for childs):
%    \begin{macrocode}
\ifchilddoc
\providecommand{\version}{draft}
\else
\providecommand{\version}{final}
\fi
%    \end{macrocode}

% Load the standard document class:
%    \begin{macrocode}
\documentclass[12pt]{article}
%    \end{macrocode}

% Start the document body:
%    \begin{macrocode}
\begin{document}
%    \end{macrocode}

% Declare a title page.
% Print title, part of document being processed and version flag:
%    \begin{macrocode}
\addtocounter{page}{-1}
\begin{center}
{\LARGE\bfseries{}childdoc example\par}
\vspace{1cm}
\ifchilddoc
\ifchilddocmanual part\else chapter\fi:
`\childdocname' of `\childdocjob'\par
\else
main document: `\childdocjob'\par
\fi
version: \version\par
\end{center}
\newpage
%    \end{macrocode}

% Manually include selected file,
% otherwise process as usual:
%    \begin{macrocode}
\ifchilddocmanual
\section*{part `\childdocname'}
\input{\childdocname}
\else
%    \end{macrocode}

% Include the two chapters:
%    \begin{macrocode}
\include{cdocsch1}
\include{cdocsch2}
%    \end{macrocode}

% Include the two parts unless only chapters should be displayed:
%    \begin{macrocode}
\ifchilddoc\else
\section{part three}
\input{cdocspt3}
\section{part four}
\input{cdocspt4}
\fi
%    \end{macrocode}

% Process as usual until here:
%    \begin{macrocode}
\fi
%    \end{macrocode}

% End of document body:
%    \begin{macrocode}
\end{document}
%    \end{macrocode}
%\iffalse
%</samplemain>
%\fi
%
% %%%%%%%%%%%%%%%%%%%%%%%%%%%%%%%%%%%%%%
% \paragraph{Chapter Include Files.}
%
% The include files are called |cdocsch1.tex| and |cdocsch2.tex|.
%
%\iffalse
%<*samplechap1|samplechap2>
%\fi

% Optional override for |\version| flag:
%    \begin{macrocode}
%%\providecommand{\version}{final}
%    \end{macrocode}

% Include the main document:
%    \begin{macrocode}
\input{childdoc.def}
\childdocof{cdocsamp}
%    \end{macrocode}

%\iffalse
%</samplechap1|samplechap2>
%\fi
%
%\iffalse
%<*samplechap1>
%\fi
% Some text for chapter 1:
%    \begin{macrocode}
\section{one}
some text in chapter one
%    \end{macrocode}

%\iffalse
%</samplechap1>
%\fi
% Some text for chapter 2:
%\iffalse
%<*samplechap2>
%\fi
%    \begin{macrocode}
\section{two}
more text in chapter two
%    \end{macrocode}

%\iffalse
%</samplechap2>
%\fi
%
% %%%%%%%%%%%%%%%%%%%%%%%%%%%%%%%%%%%%%%
% \paragraph{Part Include Files.}
%
% The include files are called |cdocspt3.tex| and |cdocspt4.tex|.
%
%\iffalse
%<*samplepart3|samplepart4>
%\fi

% Optional override for |\version| flag:
%    \begin{macrocode}
%%\providecommand{\version}{final}
%    \end{macrocode}

% Include the main document:
%    \begin{macrocode}
\input{childdoc.def}
\childdocby{cdocsamp}
%    \end{macrocode}

%\iffalse
%</samplepart3|samplepart4>
%\fi
%
%\iffalse
%<*samplepart3>
%\fi
% Some text for part 3:
%    \begin{macrocode}
some text in part three
%    \end{macrocode}

%\iffalse
%</samplepart3>
%\fi
% Some text for part 4:
%\iffalse
%<*samplepart4>
%\fi
%    \begin{macrocode}
more text in part four
%    \end{macrocode}

%\iffalse
%</samplepart4>
%\fi
%
% %%%%%%%%%%%%%%%%%%%%%%%%%%%%%%%%%%%%%%
% \paragraph{Forwarding for a Complete Draft.}
%
% The following forwarding file |cdocsdrf.tex|
% compiles the main document in draft mode:
%\iffalse
%<*sampledraft>
%\fi
%    \begin{macrocode}
\def\version{draft}
\input{childdoc.def}
\childdocforward{cdocsamp}
%    \end{macrocode}

%\iffalse
%</sampledraft>
%\fi
%
% %%%%%%%%%%%%%%%%%%%%%%%%%%%%%%%%%%%%%%
% \paragraph{Forwarding for Final Version of the Chapters.}
%
% The following forwarding files |cdocsfn1.tex| and |cdocsfn2.tex|
% (with identical content)
% compile the final versions of the child documents
% |cdocsch1.tex| and |cdocsch2.tex|, respectively:
%\iffalse
%<*samplefinal>
%\fi
%    \begin{macrocode}
\def\version{final}
\input{childdoc.def}
\childdocforwardprefix[cdocsamp]{cdocsfn}{cdocsch}
%    \end{macrocode}

%\iffalse
%</samplefinal>
%\fi
%
% %%%%%%%%%%%%%%%%%%%%%%%%%%%%%%%%%%%%%%
% \paragraph{Command Line Processing.}
%
% The following three command lines generate the output files
% |cdocscld|, |cdocscl1| and |cdocscl2|
% which should be identical to
% |cdocsdrf|, |cdocsch1| and |cdocsfn2|, respectively:
% \begin{center}
% \begin{tabular}{l}
% |latex -jobname cdocscld \|\\
% |  "\def\version{draft}\input{childdoc.def}\childdocforward{cdocsamp}"|\\
% |latex -jobname cdocscl1 \|\\
% |  "\input{childdoc.def}\childdocforward[cdocsamp]{cdocsch1}"|\\
% |latex -jobname cdocscl2 \|\\
% |  "\def\version{final}\input{childdoc.def}\childdocforward{cdocsch2}"|
% \end{tabular}
% \end{center}
% Note that the trailing backslash on each first line
% merely continues the input to the second line
% (for convenient cut ant paste).
% Furthermore, the command |latex| can be replaced by any
% of its alternative versions such as |pdflatex|.
%
% %%%%%%%%%%%%%%%%%%%%%%%%%%%%%%%%%%%%%%%%%%%%%%%%%%%%%%%%%%%%%%%%%%%%%%%%%%%%%%
% %%%%%%%%%%%%%%%%%%%%%%%%%%%%%%%%%%%%%%%%%%%%%%%%%%%%%%%%%%%%%%%%%%%%%%%%%%%%%%
% \section{Implementation}
%\iffalse
%<*package>
%\fi
%
% This section describes the definitions file |childdoc.def|.

% The definitions cannot be loaded using |\usepackage| or |\RequirePackage|
% which has a mechanism to prevent loading a style file more than once.
% When loading the definitions by means of |\input|
% multiple instances have to be prevented manually:
%\iffalse
%This code needs to be before the `\ProvidesFile' directive
%which is defined at the beginning of this file.
%Therefore it is also placed there and commented out here.
%</package>
%<*discard>
%\fi
%    \begin{macrocode}
\ifdefined\childdocmain\endinput\fi
%    \end{macrocode}
%\iffalse
%</discard>
%<*package>
%\fi
%
% \macro{\ifchilddoc}
% \macro{\ifchilddocmanual}
% The conditional |\ifchilddoc| tells whether a
% child (true) or main (false) document is being compiled.
% The conditional |\ifchilddocmanual| tells whether
% the |\includeonly| mechanism is used (false) or
% the selection of child files must be performed manually (true).
% The definitions initialise to false:
%    \begin{macrocode}
\newif\ifchilddoc
\newif\ifchilddocmanual
%    \end{macrocode}

% \macro{\childdocname}
% \macro{\childdocjob}
% The macro |\childdocname| stores the name of the main document
% to be compiled. The macro |\childdocjob| stores the name of
% the document on which the \LaTeX{} compiler was originally invoked.
% The content of |\jobname| cannot be compared
% to filenames specified in the source due to different catcodes.
% The following code rescans |\jobname|, stores the result
% in |\childdocname| and saves a copy in |\childdocjob|:
%    \begin{macrocode}
\edef\childdocname{\scantokens\expandafter{\jobname\noexpand}}
\let\childdocjob\childdocname
%    \end{macrocode}

% \macro{\childdocdisable}
% The macro |\childdocdisable| prevents the main file
% from being processed more than once.
% At this stage, the main document command |\childdocmain|
% is assumed to be called once again where it should do nothing.
% Any subsequent call to it should prevent
% a secondary processing of the main document
% It overwrites the forwarding commands
% |\childdocof| and |\childdocforward|
% with empty macros to prevent further inclusions of the main document:
%    \begin{macrocode}
\newcommand{\childdocdisable}
{
  \renewcommand{\childdocmain}[1]{\renewcommand{\childdocmain}[1]{\endinput}}
  \renewcommand{\childdocof}[1]{}
  \renewcommand{\childdocby}[2][]{}
  \renewcommand{\childdocforward}[2][]{}
  \renewcommand{\childdocdisable}{}
}
%    \end{macrocode}

% \macro{\childdocmain}
% The macro |\childdocmain| is to be called at the top of the main file
% with nothing or the main filename (without extension) as argument.
% First, it breaks loops.
% If the argument is not empty and does not match |\childdocname|
% (which is set by the first inclusion of |childdoc.def|),
% |\ifchilddoc| is set to true, |\includeonly| is applied to the child file
% and |\jobname| is set to the main file
% (for proper handling of |.aux| files):
%    \begin{macrocode}
\newcommand{\childdocmain}[1]
{
  \childdocdisable\childdocmain{}
  \if?#1?\else
    \begingroup
      \def\childdoctmp{#1}
      \ifx\childdoctmp\childdocname
        \def\childdoctmp{}
      \else
        \def\childdoctmp
        {
          \childdoctrue
          \includeonly{\childdocname}
          \def\childdocjob{#1}
          \def\jobname{#1}
        }
      \fi
      \expandafter
    \endgroup
    \childdoctmp
  \fi
}
%    \end{macrocode}

% \macro{\childdocof}
% The command |\childdocof| redirects
% compilation to the main file |#1|.
%    \begin{macrocode}
\newcommand{\childdocof}[1]
{
  \childdocdisable
  \childdoctrue
  \includeonly{\childdocname}
  \def\jobname{#1}
  \def\childdocjob{#1}
  \input{#1}
}
%    \end{macrocode}

% \macro{\childdocby}
% The command |\childdocby| ....
%    \begin{macrocode}
\newcommand{\childdocby}[2][]
{
  \childdocdisable
  \childdoctrue
  \childdocmanualtrue
  \if?#1?\else
    \def\jobname{#2}
  \fi
  \def\childdocjob{#2}
  \input{#2}
  \endinput
}
%    \end{macrocode}

% \macro{\childdocforward}
% The command |\childdocforward| redirects
% compilation to the main file or
% (if the optional argument is given) a child file.
% Parameters are set as if the main file
% or a child file starting with |\childdocof| was compiled.
% Then compilation is handed over to the main file:
%    \begin{macrocode}
\newcommand{\childdocforward}[2][]
{
  \begingroup
    \if?#1?
      \def\childdoctmp
      {
        \def\childdocname{#2}
        \def\childdocjob{#2}
        \def\jobname{#2}
        \input{#2}
        \endinput
      }
    \else
      \def\childdoctmp
      {
        \childdocdisable
        \def\childdocname{#2}
        \childdoctrue
        \includeonly{#2}
        \def\childdocjob{#1}
        \def\jobname{#1}
        \input{#1}
        \endinput
      }
    \fi
    \expandafter
  \endgroup
  \childdoctmp
}
%    \end{macrocode}

% \macro{\childdocforwardprefix}
% The command |\childdocforwardprefix| redirects
% compilation to the main or a child file by means of a pattern.
% The prefix |#1| in the current filename is replaced by |#2|
% and the suffix of the current filename is kept
% (it is assumed that the filename does not contain the substring `|~~~|'
% which is used as a delimiter).
% Compilation is handed over to the new file by |\childdocforward|:
%    \begin{macrocode}
\newcommand{\childdocforwardprefix}[3][]
{
  \begingroup
    \def\childdocextract #2##1~~~{\def\childdoctmp{\childdocforward[#1]{#3##1}}}
    \expandafter\childdocextract\childdocname~~~
    \expandafter
  \endgroup
  \childdoctmp
}
%    \end{macrocode}

% \macro{\childdoc}
% The deprecated macro |\childdoc| is a legacy version of |\childdocmain|:
%    \begin{macrocode}
\newcommand{\childdoc}{\childdocmain}
%    \end{macrocode}

% \macro{\childdocredirect}
% The deprecated macro |\childdocredirect| is a legacy version
% of |\childdocforward| and |\childdocforwardprefix|:
%    \begin{macrocode}
\newcommand{\childdocredirect}[2][]
{
  \begingroup
    \if?#1?
      \def\childdoctmp{\childdocforward{#2}}
    \else
      \def\childdoctmp{\childdocforwardprefix{#1}{#2}}
    \fi
    \expandafter
  \endgroup
  \childdoctmp
}
%    \end{macrocode}

%\iffalse
%</package>
%\fi
%
\endinput

\childdocforwardprefix[cdocsamp]{cdocsfn}{cdocsch}
%    \end{macrocode}

%\iffalse
%</samplefinal>
%\fi
%
% %%%%%%%%%%%%%%%%%%%%%%%%%%%%%%%%%%%%%%
% \paragraph{Command Line Processing.}
%
% The following three command lines generate the output files
% |cdocscld|, |cdocscl1| and |cdocscl2|
% which should be identical to
% |cdocsdrf|, |cdocsch1| and |cdocsfn2|, respectively:
% \begin{center}
% \begin{tabular}{l}
% |latex -jobname cdocscld \|\\
% |  "\def\version{draft}% \iffalse
%
% childdoc.dtx Copyright (C) 2017-2018 Niklas Beisert
%
% This work may be distributed and/or modified under the
% conditions of the LaTeX Project Public License, either version 1.3
% of this license or (at your option) any later version.
% The latest version of this license is in
%   http://www.latex-project.org/lppl.txt
% and version 1.3 or later is part of all distributions of LaTeX
% version 2005/12/01 or later.
%
% This work has the LPPL maintenance status `maintained'.
%
% The Current Maintainer of this work is Niklas Beisert.
%
% This work consists of the files childdoc.dtx and childdoc.ins
% and the derived files childdoc.def and cdocsamp.tex with
% cdocsch1.tex, cdocsch2.tex, cdocsdrf.tex, cdocsfn1.tex, cdocsfn2.tex.
%
%<package>\ifdefined\childdocmain\endinput\fi
%<package>\ProvidesFile{childdoc.def}[2018/12/30 v2.0 child document driver]
%<samplemain>\ProvidesFile{cdocsamp.tex}[2018/12/30 v2.0 sample for childdoc]
%<*driver>
%\ProvidesFile{childdoc.drv}[2018/12/30 v2.0 childdoc reference manual file]
\PassOptionsToClass{10pt,a4paper}{article}
\documentclass{ltxdoc}

\usepackage[margin=35mm]{geometry}
\usepackage{hyperref}
\usepackage{hyperxmp}
\usepackage[usenames]{color}

\hypersetup{colorlinks=true}
\hypersetup{pdfstartview=FitH}
\hypersetup{pdfpagemode=UseNone}
\hypersetup{pdfsource={}}
\hypersetup{pdflang={en-UK}}
\hypersetup{pdfcopyright={Copyright 2017-2018 Niklas Beisert.
  This work may be distributed and/or modified under the
  conditions of the LaTeX Project Public License, either version 1.3
  of this license or (at your option) any later version.}}
\hypersetup{pdflicenseurl={http://www.latex-project.org/lppl.txt}}
\hypersetup{pdfcontactaddress={ETH Zurich, ITP, HIT K,
  Wolfgang-Pauli-Strasse 27}}
\hypersetup{pdfcontactpostcode={8093}}
\hypersetup{pdfcontactcity={Zurich}}
\hypersetup{pdfcontactcountry={Switzerland}}
\hypersetup{pdfcontactemail={nbeisert@itp.phys.ethz.ch}}
\hypersetup{pdfcontacturl={http://people.phys.ethz.ch/\xmptilde nbeisert/}}

\newcommand{\secref}[1]{\hyperref[#1]{section \ref*{#1}}}

\parskip1ex
\parindent0pt
\let\olditemize\itemize
\def\itemize{\olditemize\parskip0pt}

\begin{document}

\title{The \textsf{childdoc} Package}
\hypersetup{pdftitle={The childdoc Package}}
\author{Niklas Beisert\\[2ex]
  Institut f\"ur Theoretische Physik\\
  Eidgen\"ossische Technische Hochschule Z\"urich\\
  Wolfgang-Pauli-Strasse 27, 8093 Z\"urich, Switzerland\\[1ex]
  \href{mailto:nbeisert@itp.phys.ethz.ch}
  {\texttt{nbeisert@itp.phys.ethz.ch}}}
\hypersetup{pdfauthor={Niklas Beisert}}
\hypersetup{pdfsubject={Manual for the LaTeX2e Package childdoc}}
\date{30 December 2018, \textsf{v2.0}}
\maketitle

\begin{abstract}\noindent
\textsf{childdoc} is a \LaTeXe{} package
that enables the direct compilation
of document sections included by |\include|
to individual files.
\end{abstract}

\begingroup
\parskip0ex
\tableofcontents
\endgroup

%%%%%%%%%%%%%%%%%%%%%%%%%%%%%%%%%%%%%%%%%%%%%%%%%%%%%%%%%%%%%%%%%%%%%%%%%%%%%%%%
%%%%%%%%%%%%%%%%%%%%%%%%%%%%%%%%%%%%%%%%%%%%%%%%%%%%%%%%%%%%%%%%%%%%%%%%%%%%%%%%
\section{Introduction}

\LaTeX{} provides a mechanism to structure a large document (such as a book)
into a main file and several child files (containing the chapters)
using the |\include| command.
This mechanism is beneficial for documents
which span hundreds of pages in order to
make the source file(s) more manageable.
Moreover, compilation can be restricted to
selected child files by means of the |\includeonly| command.
The latter feature can be used to reduce the compilation time while editing
(this was significantly more useful in the earlier days of \LaTeX{})
or to generate a smaller document which is easier to navigate.
Another application of |\includeonly| is to generate
documents consisting of selected parts of the complete document.

However, there are a few drawbacks of the plain |\include| mechanism:
\begin{itemize}
\item
The child files cannot be compiled on their own,
they can only be compiled via the main file.
A naive editing environment
(such as a text editor with an option
to have the current file processed by \LaTeX)
may require one to switch to the main file before compiling;
attempting to compile the child file produces errors.
\item
The main file must be modified (each time)
to adjust the |\includeonly| command
to the present needs. This easily leaves the main file in a messy state.
\item
The generated document will always carry the filename
of the main document. This is inconvenient if
several child files are to be compiled and
to be kept for distribution.
\end{itemize}

The present package provides a simple interface
to make child files individually compilable by \LaTeX{}.
Compiling a child file then has the same effect as compiling
the main file with an |\includeonly| command
to select the appropriate child.
Moreover the generated document will carry the name of the child
rather than the main file.
This resolves all three above issues.

This feature is meant to make the editing of books,
thesis documents and lecture notes somewhat more convenient.
However, the package can also be used efficiently for
composing a series of documents (such as exercise sheets)
which are typically distributed individually.
It then assists the author in generating the individual documents
(potentially in different versions)
as well as a document containing the collected series.
Another application is in developing style files
or other kinds of included material
where compilation of the style file could redirect
to a sample or test file.

%%%%%%%%%%%%%%%%%%%%%%%%%%%%%%%%%%%%%%%%%%%%%%%%%%%%%%%%%%%%%%%%%%%%%%%%%%%%%%%%
%%%%%%%%%%%%%%%%%%%%%%%%%%%%%%%%%%%%%%%%%%%%%%%%%%%%%%%%%%%%%%%%%%%%%%%%%%%%%%%%
\section{Usage}

First of all, the package \textsf{childdoc} is \emph{not} a standard
\LaTeXe{} |.sty| style file! Therefore it needs to be invoked in
a non-standard way.

%%%%%%%%%%%%%%%%%%%%%%%%%%%%%%%%%%%%%%%%%%%%%%%%%%%%%%%%%%%%%%%%%%%%%%%%%%%%%%%%
\subsection{Included Files}
\label{sec:include}

%%%%%%%%%%%%%%%%%%%%%%%%%%%%%%%%%%%%%%%%
\DescribeMacro{\childdocmain}
To use the package, add the commands
\begin{center}
\begin{tabular}{l}
|\input{childdoc.def}|\\
|\childdocmain{}|\\
\end{tabular}
\end{center}
at the very top of the main \LaTeX{} file,
in particular \emph{before} the |\documentclass| statement!
The argument of |\childdocmain| should be left empty
(but it must be present).

%%%%%%%%%%%%%%%%%%%%%%%%%%%%%%%%%%%%%%%%
\DescribeMacro{\childdocof}
Furthermore, add the commands
\begin{center}
\begin{tabular}{l}
|\input{childdoc.def}|\\
|\childdocof{|\textit{main}|}|\\
\end{tabular}
\end{center}
at the top of every child file \textit{child}
which is included by |\include{|\textit{child}|}|
from within the main file
(or at least for those files to be compiled individually).
The argument \textit{main} must be the filename of the main file.

There are a couple of
considerations in setting up the main and child documents:

%%%%%%%%%%%%%%%%%%%%%%%%%%%%%%%%%%%%%%%%
\paragraph{Restrictions.}

Please note the following restrictions:
\begin{itemize}
\item
|\childdocmain| must be called with one argument \textit{main}
to ensure compatibility with earlier version of the package.
It must either be empty (|\childdocmain{}|)
or precisely match the filename of the main file in which it is specified.
See \secref{sec:detection} for further information.
\item
The filename \textit{main} must be specified without the |.tex| extension.
\item
The filename \textit{main} is case sensitive
(even in case-insensitive file systems)
due to internal string comparison.
\item
The argument \textit{main} should be fully expanded, it cannot be a macro.
\item
Subdirectories and special characters should be avoided in filenames.
\item
The command |\childdocmain{|\textit{main}|}| must be followed by a whitespace.
It should not be followed immediately by another command
or by a comment mark `|%|'.
This is because the \TeX{} parser reads the token immediately following
the argument of |\childdocmain| and puts it
at the beginning of every child section;
however, a white\-space is ignored.
\end{itemize}

%%%%%%%%%%%%%%%%%%%%%%%%%%%%%%%%%%%%%%%%
\paragraph{Content of Main File.}

It is advisable to place all content in the child files included by |\include|.
Any output contained in the main file will appear in all child documents
unless suppressed manually;
it cannot be suppressed automatically by the |\includeonly| directive
and thus should normally be avoided.
A method to include some content in the main file
by means of conditional processing is described in \secref{sec:conditional}.

%%%%%%%%%%%%%%%%%%%%%%%%%%%%%%%%%%%%%%%%
\paragraph{Page Numbering.}

When only a part of the document is compiled,
the appropriate numbering of pages
(as well as other status parameters)
is determined from the |.aux| files.
The latter contain information from previous passes.
However this information needs to propagate through
all intermediate child documents.
Therefore the page numbering in child documents may well
be inconsistent until the complete document is compiled at least once.

A useful (if unconventional) way to always ensure a consistent
page numbering is to restart the numbering in each child document
and denote the pages by `\textit{child}|.|\textit{page}'
where \textit{child} represents the chapter/section number of the child file.
This can be achieved by the command
|\numberwithin{page}{|\textit{child}|}|
of the \textsf{amsmath} package
where \textit{child} can be |chapter| or |section|
depending on the chosen structuring.
Alternatively, one can modify the macro |\thepage| appropriately
and reset the counter |page| at the start of each child file.

%%%%%%%%%%%%%%%%%%%%%%%%%%%%%%%%%%%%%%%%%%%%%%%%%%%%%%%%%%%%%%%%%%%%%%%%%%%%%%%%
\subsection{Conditional Processing}
\label{sec:conditional}

The package provides a mechanism to compile different versions
of a document. To customise the versions further some conditional processing
can come in handy to distinguish which version is being compiled.
The package provides two macros to describe the compilation context:

%%%%%%%%%%%%%%%%%%%%%%%%%%%%%%%%%%%%%%%%
\DescribeMacro{\ifchilddoc}
The conditional |\ifchilddoc| distinguishes between the compilation of
child documents and the main document:
%
\begin{center}
|\ifchilddoc |\textit{child-code}| |[|\||else |\textit{main-code}]| \||fi|
\end{center}

%%%%%%%%%%%%%%%%%%%%%%%%%%%%%%%%%%%%%%%%
\DescribeMacro{\childdocname}
\DescribeMacro{\childdocjob}
The macro |\childdocname| contains the filename (without extension)
of the main or child file being processed.
Note that |\childdocjob| will always contain the name of the main file.

%%%%%%%%%%%%%%%%%%%%%%%%%%%%%%%%%%%%%%%%
\paragraph{Title Page.}

Conditional processing can be used to include a title or banner page
in the main document when proper precautions are taken.
Importantly, the code in the main file should ensure that the page counter
(as well as other status parameters which are stored in the |.aux| files)
takes the same value after the conditional processing.
Otherwise the page numbers may take divergent values
depending on which part is compiled.

For example, a title page could be declared by:
%
\begin{center}
\begin{tabular}{l}
|\ifchilddoc\||else|\\
|\addtocounter{page}{-1}|\\
\textit{code for title page}\\
|\newpage|\\
|\||fi|
\end{tabular}
\end{center}
%
A banner page for the child documents can be generated by:
%
\begin{center}
\begin{tabular}{l}
|\ifchilddoc|\\
|\addtocounter{page}{-1}|\\
\textit{code for banner page}\\
|\newpage|\\
|\||fi|
\end{tabular}
\end{center}
%
Here one could write a message such as:
\begin{center}
|This is the part \childdocname{} of \childdocjob{}.|
\end{center}

%%%%%%%%%%%%%%%%%%%%%%%%%%%%%%%%%%%%%%%%%%%%%%%%%%%%%%%%%%%%%%%%%%%%%%%%%%%%%%%%
\subsection{Flags}
\label{sec:flags}

The package makes it easy to generate different versions
of the main or child documents.
To this end compilation flags can be defined
and assigned different default values.
They will be particularly useful in conjunction
with the forwarding mechanism described in \secref{sec:forward}.

For example, it may be useful to have a flag |\version|
which can be set to |draft| or |final|.
The document source will contain some conditional code
depending on the value of |\version|.
Suppose further, the flag should default to |final| for the main file
and to |draft| for child files
which is a natural assignment for editing the document.
This is achieved by placing the following code
in the preamble of the main document
(below the |\childdocmain| directive):
%
\begin{center}
\begin{tabular}{l}
|\ifchilddoc|\\
|\providecommand{\version}{draft}|\\
|\||else|\\
|\providecommand{\version}{final}|\\
|\||fi|
\end{tabular}
\end{center}
%
The definition by |\providecommand| makes sure
that previous definitions are not overwritten.
Further statements |\providecommand{\version}{...}|
can thus be added before the above code to override it.

For the main file, one might add a line
(between |\childdocmain| and the above block)
%
\begin{center}
|%\ifchilddoc\||else\providecommand{\version}{draft}\||fi|
\end{center}
%
which can be uncommented to produce a draft version.
Likewise one can add a line to the very top of a child file
(above the |\childdocof{|\textit{main}|}| directive)
%
\begin{center}
|%\providecommand{\version}{final}|
\end{center}
%
which can be uncommented to produce the final version of this child document.

%%%%%%%%%%%%%%%%%%%%%%%%%%%%%%%%%%%%%%%%%%%%%%%%%%%%%%%%%%%%%%%%%%%%%%%%%%%%%%%%
\subsection{Forwarding}
\label{sec:forward}

Different versions of the main or child documents
using compilation flags as described in \secref{sec:flags}
can be (permanently) stored in different files
for convenient compilation, viewing and distribution.
To this end, the package defines a command
to pass on compilation to a different file:

%%%%%%%%%%%%%%%%%%%%%%%%%%%%%%%%%%%%%%%%
\DescribeMacro{\childdocforward}
The command |\childdocforward| redirects processing to
another source file:
%
\begin{center}
\begin{tabular}{l}
|\input{childdoc.def}|\\
|\childdocforward[|\textit{main}|]{|\textit{dest}|}|\\
\end{tabular}
\end{center}
%
The argument \textit{dest} is the destination file
(without extension).
It should be the main file or one of the child files.
Note that further \textsf{childdoc} directives
such as |\childdocof| and |\childdocforward|
in the indicated file will be processed in this form.
The optional argument \textit{main}
passes on directly to the main file \textit{main}
while pretending to compile the child \textit{dest}.
This form behaves as if \textit{dest}
issues |\childdocof{|\textit{main}|}| right away,
and no further \textsf{childdoc} directives will be processed.

%%%%%%%%%%%%%%%%%%%%%%%%%%%%%%%%%%%%%%%%
\DescribeMacro{\...prefix}
In the alternative form |\childdocforwardprefix|,
%
\begin{center}
\begin{tabular}{l}
|\input{childdoc.def}|\\
|\childdocforwardprefix[|\textit{main}|]{|\textit{prefix}|}{|\textit{dest}|}|
\end{tabular}
\end{center}
%
the destination file is determined by a pattern
depending on the current file:
To make this work, the current file must be called
`{\textit{prefix}\hspace{0.2em}\textit{suffix}}'
with \textit{prefix} matching precisely the argument.
Processing is then passed on to the file
`{\textit{dest}\hspace{0.2em}\textit{suffix}}'.
Surely, the same effect is achieved by
directly specifying the
argument `{\textit{dest}\hspace{0.2em}\textit{suffix}}'
in the first form.
However, that requires to set up a different file
for each child. With the alternative form of the command
all these files can have exactly the same content
which simplifies setting them up and maintaining them.

For example, the following file |draft.tex|
with a compilation flag |\version| as described in \secref{sec:flags}
compiles the main document as a draft:
%
\begin{center}
\begin{tabular}{l}
|\def\version{draft}|\\
|\input{childdoc.def}|\\
|\childdocforward{|\textit{main}|}|
\end{tabular}
\end{center}
%
Likewise, the following files |final|\textit{nn}|.tex|
compile the final version of the child document
|child|\textit{nn}|.tex|:
%
\begin{center}
\begin{tabular}{l}
|\def\version{final}|\\
|\input{childdoc.def}|\\
|\childdocforwardprefix{final}{child}|
\end{tabular}
\end{center}
%

Note that when several versions of a main file and/or of each child file
are to be generated, it may be convenient to set up a |Makefile| or
shell script to automatise the process.

%%%%%%%%%%%%%%%%%%%%%%%%%%%%%%%%%%%%%%%%%%%%%%%%%%%%%%%%%%%%%%%%%%%%%%%%%%%%%%%%
\subsection{Command Line Processing}
\label{sec:commandline}

The effect of redirection files can also be achieved by invoking
the \LaTeX{} compiler with a more elaborate command line.
Most conveniently this should be done as part
of a shell script or a |Makefile|.

When using \textsf{childdoc} in the main file, the following
command lines effectively perform a redirection
(note that depending on the shell being used,
backslashes may have to be doubled: `|\|' $\to$ `|\\|'):
%
\begin{center}
|... -jobname "|\textit{target}|" |\\|"|[\textit{flags}]%
|\input{childdoc.def}\childdocforward[|\textit{main}|]{|\textit{dest}|}"|
\end{center}
%
Here \textit{target} is the name of the output file,
\textit{main} is the name of the main file
and \textit{dest} is the name of the main or child file to be processed
(all filenames without extensions).
The optional argument \textit{main} can be omitted
if \textit{main} matches \textit{dest}.
Optionally, compilation \textit{flags} can be defined via |\def| commands.
This command line makes the \TeX{} engine believe
it is compiling the file \textit{target}
whose content is specified as the latter parameter.
The provided code then forwards the processing to
\textit{main} or \textit{dest} as described in \secref{sec:forward}.

%%%%%%%%%%%%%%%%%%%%%%%%%%%%%%%%%%%%%%%%%%%%%%%%%%%%%%%%%%%%%%%%%%%%%%%%%%%%%%%%
\subsection{Include by Input}
\label{sec:input}

Including child documents by |\include| has some restrictions by design.
Most notably, the content of a child document always occupies
its own set of pages; pages cannot be shared between child documents.
Usually, this behaviour makes perfect sense
because each child document contain an essential part of the document.
However, in some situations it may be desirable to compose
a document from a collection of parts
without having mandatory page breaks between then.
For this case, the package
provides a mechanism to include parts
by |\input| which can also be processed individually.
However, by construction this mechanism
requires manual handling of the content to be output.

%%%%%%%%%%%%%%%%%%%%%%%%%%%%%%%%%%%%%%%%
\DescribeMacro{\ifchilddocmanual}
The main file should be prepared as usual, see \secref{sec:include}.
However, the document body must make a distinction
between processing of an individual part and of the main document, e.g.:
%
\begin{center}
\begin{tabular}{l}
|\ifchilddocmanual|\\
|\input{\childdocname}|\\
|\||else|\\
\textit{document body with }|\input{|\textit{part}|}|\\
|\||fi|
\end{tabular}
\end{center}
%
The conditional |\ifchilddocmanual| is true whenever
a part to be included by |\input| is being compiled,
and the name of the part is stored in |\childdocname|.

%%%%%%%%%%%%%%%%%%%%%%%%%%%%%%%%%%%%%%%%
\DescribeMacro{\childdocby}
Each part to be included by |\input| should start with:
%
\begin{center}
\begin{tabular}{l}
|\input{childdoc.def}|\\
|\childdocby{|\textit{main}|}|\\
\end{tabular}
\end{center}
%
The directive |\childdocby| is similar to |\childdocof|
described in \secref{sec:include},
but the subsequent selection of content must be done manually.
To that end, both |\ifchilddoc| and |\ifchilddocmanual|
will be true upon processing of a part,
and the name of the part is stored in |\childdocname|.
Note that |\jobname| will be set to the filename of the current part
so that each part receives an individual |.aux| file
that does not interfere with the |.aux| file(s) of the main document.
This behaviour can be altered by the alternative form
|\childdocby[*]{|\textit{main}|}| (with a non-empty optional argument)
which uses the |.aux| file of the main document
by setting |\jobname| to \textit{main}.

%%%%%%%%%%%%%%%%%%%%%%%%%%%%%%%%%%%%%%%%%%%%%%%%%%%%%%%%%%%%%%%%%%%%%%%%%%%%%%%%
\subsection{Driver Development}
\label{sec:driver}

The \textsf{childdoc} mechanism can also be use for the development
of definition files such as \LaTeX{} styles or classes.
This case differs from the above setup with multiple parts
included by |\include| in that no |\includeonly| should be invoked.
This can be achieved by starting the include file
(before |\ProvidesPackage|) with:
%
\begin{center}
\begin{tabular}{l}
|\input{childdoc.def}|\\
|\childdocforward{|\textit{main}|}|\\
\end{tabular}
\end{center}
%
or alternatively with:
%
\begin{center}
\begin{tabular}{l}
|\input{childdoc.def}|\\
|\childdocby{|\textit{main}|}|\\
\end{tabular}
\end{center}
%
Both forms have slightly different effects as described above.
The main file is prepared as usual, see \secref{sec:include}.

%%%%%%%%%%%%%%%%%%%%%%%%%%%%%%%%%%%%%%%%%%%%%%%%%%%%%%%%%%%%%%%%%%%%%%%%%%%%%%%%
\subsection{Legacy Detection}
\label{sec:detection}

The directive |\childdocmain| in the main file can detect
whether the complete document or merely a child is to be compiled
even without using the directive |\childdocof|.
This method is deprecated because it is less robust
and there is no compelling reason to use it;
it is merely provided for backward compatibility
and it may be removed in future versions.

If the detection mechanism is to be used,
it is mandatory to correctly specify
the filename of the main file as the argument of |\childdocmain|:
%
\begin{center}
\begin{tabular}{l}
|\input{childdoc.def}|\\
|\childdocmain{|\textit{main}|}|\\
\end{tabular}
\end{center}
%
If |\jobname| does not match the argument \textit{main} of |\childdocmain|,
it is assumed that |\jobname| points to the child file to be compiled.
When using |\childdocmain| with the main file specified as argument,
it suffices to start a child file
with just |\input{|\textit{main}|}|
without loading of the package and using |\childdocof|.
If instead all processing is done
with the appropriate \textsf{childdoc} directives,
the argument of \textit{main} of |\childdocmain| can be empty.

An alternative version of the command line processing described
in \secref{sec:commandline} using the detection mechanism reads:
%
\begin{center}
|... -jobname "|\textit{target}|" "|[\textit{flags}]%
[|\def\jobname{|\textit{dest}|}|]|\input{|\textit{main}|}"|
\end{center}

%%%%%%%%%%%%%%%%%%%%%%%%%%%%%%%%%%%%%%%%%%%%%%%%%%%%%%%%%%%%%%%%%%%%%%%%%%%%%%%%
\subsection{Manual Code}
\label{sec:manual}

In case one cannot be certain whether the definitions file |childdoc.def|
is installed on the target \TeX{} distribution
and one prefers not to ship it,
it is conceivable to paste a few relevant commands into the sources.

To that end, drop all statements |\input{childdoc.def}|
and perform the replacements as outlined below.
Instead of |\childdocmain{|\textit{main}|}| add the following code
to the top of the main file:
%
\begin{center}
\begin{tabular}{l}
|\||ifdefined\childdocname\endinput\||fi\newif\ifchilddoc|\\
|\edef\childdocname{\scantokens\expandafter{\jobname\noexpand}}|\\
|\def\childdocmain{|\textit{main}|}\||ifx\childdocmain\childdocname\||else|\\
|\childdoctrue\includeonly{\childdocname}\let\jobname\childdocmain\||fi|\\
\end{tabular}
\end{center}
%
Instead of |\childdocof{|\textit{main}|}| just include the main file
at the top of each child file:
%
\begin{center}
|\input{|\textit{main}|}|
\end{center}
%
A simple redirection |\childdocforward{|\textit{dest}|}| is achieved by:
%
\begin{center}
|\def\jobname{|\textit{dest}|}\input{\jobname}|
\end{center}
%
The redirection with prefix
|\childdocforwardprefix[|\textit{prefix}|]{|\textit{dest}|}|
is accomplished by:
%
\begin{center}
\begin{tabular}{l}
|{\edef\jobname{\scantokens\expandafter{\jobname\noexpand}}|\\
|\def\redirectjob |\textit{prefix}|#1~~~{\gdef\jobname{|\textit{dest}|#1}}|\\
|\expandafter\redirectjob\jobname~~~}\input{\jobname}|
\end{tabular}
\end{center}

In an alternative approach,
child documents can be compiled by a specific command line
without additional code or specific definitions:
%
\begin{center}
|... -jobname "|\textit{target}|" "|[\textit{flags}]%
|\includeonly{|\textit{dest}|}\input{|\textit{main}|}"|
\end{center}
%

%%%%%%%%%%%%%%%%%%%%%%%%%%%%%%%%%%%%%%%%%%%%%%%%%%%%%%%%%%%%%%%%%%%%%%%%%%%%%%%%
%%%%%%%%%%%%%%%%%%%%%%%%%%%%%%%%%%%%%%%%%%%%%%%%%%%%%%%%%%%%%%%%%%%%%%%%%%%%%%%%
\section{Information}

%%%%%%%%%%%%%%%%%%%%%%%%%%%%%%%%%%%%%%%%%%%%%%%%%%%%%%%%%%%%%%%%%%%%%%%%%%%%%%%%
\subsection{Copyright}

Copyright \copyright{} 2017--2018 Niklas Beisert

This work may be distributed and/or modified under the
conditions of the \LaTeX{} Project Public License, either version 1.3
of this license or (at your option) any later version.
The latest version of this license is in
  \url{http://www.latex-project.org/lppl.txt}
and version 1.3 or later is part of all distributions of \LaTeX{}
version 2005/12/01 or later.

This work has the LPPL maintenance status `maintained'.

The Current Maintainer of this work is Niklas Beisert.

This work consists of the files |README.txt|, |childdoc.ins| and |childdoc.dtx|
as well as the derived files |childdoc.def|, |cdocsamp.tex|
with |cdocsch1.tex|, |cdocsch2.tex|, |cdocspt3.tex|, |cdocspt4.tex|,
|cdocsdrf.tex|, |cdocsfn1.tex|, |cdocsfn2.tex|
as well as |childdoc.pdf|.

%%%%%%%%%%%%%%%%%%%%%%%%%%%%%%%%%%%%%%%%%%%%%%%%%%%%%%%%%%%%%%%%%%%%%%%%%%%%%%%%
\subsection{Files and Installation}

The package consists of the files:
%
\begin{center}
\begin{tabular}{ll}
    |README.txt|   & readme file \\
    |childdoc.ins| & installation file \\
    |childdoc.dtx| & source file \\
    |childdoc.def| & definition file \\
    |cdocsamp.tex| & sample main file \\
    |cdocsch1.tex| & sample include file \\
    |cdocsch2.tex| & sample include file \\
    |cdocspt3.tex| & sample part file \\
    |cdocspt4.tex| & sample part file \\
    |cdocsdrf.tex| & sample redirection file \\
    |cdocsfn1.tex| & sample redirection file \\
    |cdocsfn2.tex| & sample redirection file \\
    |childdoc.pdf| & manual
\end{tabular}
\end{center}
%
The distribution consists of the files
|README.txt|, |childdoc.ins| and |childdoc.dtx|.
%
\begin{itemize}
\item
Run (pdf)\LaTeX{} on |childdoc.dtx|
to compile the manual |childdoc.pdf| (this file).
\item
Run \LaTeX{} on |childdoc.ins| to create the definitions file |childdoc.def|
and the sample |cdocsamp.tex| with include files
|cdocsch1.tex|, |cdocsch2.tex|, |cdocspt3.tex|, |cdocspt4.tex|,
|cdocsdrf.tex|, |cdocsfn1.tex|, |cdocsfn2.tex|.
Then copy the file |childdoc.def| to an appropriate directory of your \LaTeX{}
distribution, e.g.\ \textit{texmf-root}|/tex/latex/childdoc|.
\end{itemize}

%%%%%%%%%%%%%%%%%%%%%%%%%%%%%%%%%%%%%%%%%%%%%%%%%%%%%%%%%%%%%%%%%%%%%%%%%%%%%%%%
\subsection{Related CTAN Packages}

There are several other packages which offer a similar functionality:
%
\begin{itemize}
\item
The packages
\href{http://ctan.org/pkg/docmute}{\textsf{docmute}},
\href{http://ctan.org/pkg/includex}{\textsf{includex}} and
\href{http://ctan.org/pkg/standalone}{\textsf{standalone}}
provide commands to include only the document body of
a child file thus allowing both files to be compiled individually.
\item
The packages \href{http://ctan.org/pkg/subdocs}{\textsf{subdocs}}
and \href{http://ctan.org/pkg/subfiles}{\textsf{subfiles}}
provide structures in which the main and child documents can be
encapsulated and allowing them to be compiled individually.
The inclusion mechanism is different from the conventional |\include|.
\item
The package \href{http://ctan.org/pkg/combine}{\textsf{combine}}
is an elaborate solution to combine several documents into one.
\end{itemize}
%
See also the CTAN topic \href{http://ctan.org/topic/subdocs}{\textsf{subdocs}}
for further related packages.
The present package differs from the above solutions in that
a document structure constructed with the conventional |\include| mechanism
just needs two extra commands at the top of every file
such that all constituent files can be compiled individually.

%%%%%%%%%%%%%%%%%%%%%%%%%%%%%%%%%%%%%%%%%%%%%%%%%%%%%%%%%%%%%%%%%%%%%%%%%%%%%%%%
%\subsection{Feature Suggestions}
%
%The following is a list of features which may be useful for future
%versions of this package:
%%
%\begin{itemize}
%\item
%\ldots
%\end{itemize}

%%%%%%%%%%%%%%%%%%%%%%%%%%%%%%%%%%%%%%%%%%%%%%%%%%%%%%%%%%%%%%%%%%%%%%%%%%%%%%%%
\subsection{Revision History}

%%%%%%%%%%%%%%%%%%%%%%%%%%%%%%%%%%%%%%%%
\paragraph{v2.0:} 2018/12/30

\begin{itemize}
\item
immediate forward processing
\item
added |\childdocby| mechanism
\item
manual restructured
\end{itemize}

%%%%%%%%%%%%%%%%%%%%%%%%%%%%%%%%%%%%%%%%
\paragraph{v1.6:} 2018/01/17

\begin{itemize}
\item
application for development of include files
\item
corrections to manual
\end{itemize}

%%%%%%%%%%%%%%%%%%%%%%%%%%%%%%%%%%%%%%%%
\paragraph{v1.5:} 2017/05/21

\begin{itemize}
\item
more complete structuring introduced
\item
|\childdocof| introduced
\item
|\childdoc| renamed to |\childdocmain|
\item
|\childredirect| renamed to |\childdocforward| and |\childdocforwardprefix|
and functionality expanded
\end{itemize}

%%%%%%%%%%%%%%%%%%%%%%%%%%%%%%%%%%%%%%%%
\paragraph{v1.0:} 2017/04/27

\begin{itemize}
\item
manual and install package
\item
first version published on CTAN
\end{itemize}

%%%%%%%%%%%%%%%%%%%%%%%%%%%%%%%%%%%%%%%%
\paragraph{v0.6:} 2017/04/26

\begin{itemize}
\item
redirection mechanism added
\end{itemize}

%%%%%%%%%%%%%%%%%%%%%%%%%%%%%%%%%%%%%%%%
\paragraph{v0.5:} 2017/04/26

\begin{itemize}
\item
functionality in definition file
\end{itemize}


%%%%%%%%%%%%%%%%%%%%%%%%%%%%%%%%%%%%%%%%%%%%%%%%%%%%%%%%%%%%%%%%%%%%%%%%%%%%%%%%
%%%%%%%%%%%%%%%%%%%%%%%%%%%%%%%%%%%%%%%%%%%%%%%%%%%%%%%%%%%%%%%%%%%%%%%%%%%%%%%%
%%%%%%%%%%%%%%%%%%%%%%%%%%%%%%%%%%%%%%%%%%%%%%%%%%%%%%%%%%%%%%%%%%%%%%%%%%%%%%%%
\appendix

\settowidth\MacroIndent{\rmfamily\scriptsize 000\ }

 \DocInput{childdoc.dtx}

\end{document}
%</driver>
% \fi
%
% %%%%%%%%%%%%%%%%%%%%%%%%%%%%%%%%%%%%%%%%%%%%%%%%%%%%%%%%%%%%%%%%%%%%%%%%%%%%%%
% %%%%%%%%%%%%%%%%%%%%%%%%%%%%%%%%%%%%%%%%%%%%%%%%%%%%%%%%%%%%%%%%%%%%%%%%%%%%%%
% \section{Sample}
%\iffalse
%<*samplemain>
%\fi
%
% The following presents a sample document
% with two chapters, two parts, a title page,
% a compile flag as well as three forwarding files to set the flag.
% It consists of eight |.tex| files:
% \begin{center}
% \begin{tabular}{ll}
% |cdocsamp.tex|&main file\\
% |cdocsch1.tex|&include file for chapter 1\\
% |cdocsch2.tex|&include file for chapter 2\\
% |cdocspt3.tex|&include file for part 3\\
% |cdocspt4.tex|&include file for part 4\\
% |cdocsdrf.tex|&forwarding file for main file in draft mode\\
% |cdocsfi1.tex|&forwarding file for final version of chapter 1\\
% |cdocsfi2.tex|&forwarding file for final version of chapter 2\\
% \end{tabular}
% \end{center}
% Each of the eight files can be compiled directly by the \LaTeX{} compiler.
%
% %%%%%%%%%%%%%%%%%%%%%%%%%%%%%%%%%%%%%%
% \paragraph{Main File.}
%
% The main file is called |cdocsamp.tex|.
%
% Load the \textsf{childdoc} definitions and
% declare the filename for the main document:
%    \begin{macrocode}
\input{childdoc.def}
\childdocmain{}
%    \end{macrocode}

% Optional override for |\version| flag:
%    \begin{macrocode}
%%\ifchilddoc\else\providecommand{\version}{draft}\fi
%    \end{macrocode}

% Define the default values for the |\version| flag
% (|final| for the main file and |draft| for childs):
%    \begin{macrocode}
\ifchilddoc
\providecommand{\version}{draft}
\else
\providecommand{\version}{final}
\fi
%    \end{macrocode}

% Load the standard document class:
%    \begin{macrocode}
\documentclass[12pt]{article}
%    \end{macrocode}

% Start the document body:
%    \begin{macrocode}
\begin{document}
%    \end{macrocode}

% Declare a title page.
% Print title, part of document being processed and version flag:
%    \begin{macrocode}
\addtocounter{page}{-1}
\begin{center}
{\LARGE\bfseries{}childdoc example\par}
\vspace{1cm}
\ifchilddoc
\ifchilddocmanual part\else chapter\fi:
`\childdocname' of `\childdocjob'\par
\else
main document: `\childdocjob'\par
\fi
version: \version\par
\end{center}
\newpage
%    \end{macrocode}

% Manually include selected file,
% otherwise process as usual:
%    \begin{macrocode}
\ifchilddocmanual
\section*{part `\childdocname'}
\input{\childdocname}
\else
%    \end{macrocode}

% Include the two chapters:
%    \begin{macrocode}
\include{cdocsch1}
\include{cdocsch2}
%    \end{macrocode}

% Include the two parts unless only chapters should be displayed:
%    \begin{macrocode}
\ifchilddoc\else
\section{part three}
\input{cdocspt3}
\section{part four}
\input{cdocspt4}
\fi
%    \end{macrocode}

% Process as usual until here:
%    \begin{macrocode}
\fi
%    \end{macrocode}

% End of document body:
%    \begin{macrocode}
\end{document}
%    \end{macrocode}
%\iffalse
%</samplemain>
%\fi
%
% %%%%%%%%%%%%%%%%%%%%%%%%%%%%%%%%%%%%%%
% \paragraph{Chapter Include Files.}
%
% The include files are called |cdocsch1.tex| and |cdocsch2.tex|.
%
%\iffalse
%<*samplechap1|samplechap2>
%\fi

% Optional override for |\version| flag:
%    \begin{macrocode}
%%\providecommand{\version}{final}
%    \end{macrocode}

% Include the main document:
%    \begin{macrocode}
\input{childdoc.def}
\childdocof{cdocsamp}
%    \end{macrocode}

%\iffalse
%</samplechap1|samplechap2>
%\fi
%
%\iffalse
%<*samplechap1>
%\fi
% Some text for chapter 1:
%    \begin{macrocode}
\section{one}
some text in chapter one
%    \end{macrocode}

%\iffalse
%</samplechap1>
%\fi
% Some text for chapter 2:
%\iffalse
%<*samplechap2>
%\fi
%    \begin{macrocode}
\section{two}
more text in chapter two
%    \end{macrocode}

%\iffalse
%</samplechap2>
%\fi
%
% %%%%%%%%%%%%%%%%%%%%%%%%%%%%%%%%%%%%%%
% \paragraph{Part Include Files.}
%
% The include files are called |cdocspt3.tex| and |cdocspt4.tex|.
%
%\iffalse
%<*samplepart3|samplepart4>
%\fi

% Optional override for |\version| flag:
%    \begin{macrocode}
%%\providecommand{\version}{final}
%    \end{macrocode}

% Include the main document:
%    \begin{macrocode}
\input{childdoc.def}
\childdocby{cdocsamp}
%    \end{macrocode}

%\iffalse
%</samplepart3|samplepart4>
%\fi
%
%\iffalse
%<*samplepart3>
%\fi
% Some text for part 3:
%    \begin{macrocode}
some text in part three
%    \end{macrocode}

%\iffalse
%</samplepart3>
%\fi
% Some text for part 4:
%\iffalse
%<*samplepart4>
%\fi
%    \begin{macrocode}
more text in part four
%    \end{macrocode}

%\iffalse
%</samplepart4>
%\fi
%
% %%%%%%%%%%%%%%%%%%%%%%%%%%%%%%%%%%%%%%
% \paragraph{Forwarding for a Complete Draft.}
%
% The following forwarding file |cdocsdrf.tex|
% compiles the main document in draft mode:
%\iffalse
%<*sampledraft>
%\fi
%    \begin{macrocode}
\def\version{draft}
\input{childdoc.def}
\childdocforward{cdocsamp}
%    \end{macrocode}

%\iffalse
%</sampledraft>
%\fi
%
% %%%%%%%%%%%%%%%%%%%%%%%%%%%%%%%%%%%%%%
% \paragraph{Forwarding for Final Version of the Chapters.}
%
% The following forwarding files |cdocsfn1.tex| and |cdocsfn2.tex|
% (with identical content)
% compile the final versions of the child documents
% |cdocsch1.tex| and |cdocsch2.tex|, respectively:
%\iffalse
%<*samplefinal>
%\fi
%    \begin{macrocode}
\def\version{final}
\input{childdoc.def}
\childdocforwardprefix[cdocsamp]{cdocsfn}{cdocsch}
%    \end{macrocode}

%\iffalse
%</samplefinal>
%\fi
%
% %%%%%%%%%%%%%%%%%%%%%%%%%%%%%%%%%%%%%%
% \paragraph{Command Line Processing.}
%
% The following three command lines generate the output files
% |cdocscld|, |cdocscl1| and |cdocscl2|
% which should be identical to
% |cdocsdrf|, |cdocsch1| and |cdocsfn2|, respectively:
% \begin{center}
% \begin{tabular}{l}
% |latex -jobname cdocscld \|\\
% |  "\def\version{draft}\input{childdoc.def}\childdocforward{cdocsamp}"|\\
% |latex -jobname cdocscl1 \|\\
% |  "\input{childdoc.def}\childdocforward[cdocsamp]{cdocsch1}"|\\
% |latex -jobname cdocscl2 \|\\
% |  "\def\version{final}\input{childdoc.def}\childdocforward{cdocsch2}"|
% \end{tabular}
% \end{center}
% Note that the trailing backslash on each first line
% merely continues the input to the second line
% (for convenient cut ant paste).
% Furthermore, the command |latex| can be replaced by any
% of its alternative versions such as |pdflatex|.
%
% %%%%%%%%%%%%%%%%%%%%%%%%%%%%%%%%%%%%%%%%%%%%%%%%%%%%%%%%%%%%%%%%%%%%%%%%%%%%%%
% %%%%%%%%%%%%%%%%%%%%%%%%%%%%%%%%%%%%%%%%%%%%%%%%%%%%%%%%%%%%%%%%%%%%%%%%%%%%%%
% \section{Implementation}
%\iffalse
%<*package>
%\fi
%
% This section describes the definitions file |childdoc.def|.

% The definitions cannot be loaded using |\usepackage| or |\RequirePackage|
% which has a mechanism to prevent loading a style file more than once.
% When loading the definitions by means of |\input|
% multiple instances have to be prevented manually:
%\iffalse
%This code needs to be before the `\ProvidesFile' directive
%which is defined at the beginning of this file.
%Therefore it is also placed there and commented out here.
%</package>
%<*discard>
%\fi
%    \begin{macrocode}
\ifdefined\childdocmain\endinput\fi
%    \end{macrocode}
%\iffalse
%</discard>
%<*package>
%\fi
%
% \macro{\ifchilddoc}
% \macro{\ifchilddocmanual}
% The conditional |\ifchilddoc| tells whether a
% child (true) or main (false) document is being compiled.
% The conditional |\ifchilddocmanual| tells whether
% the |\includeonly| mechanism is used (false) or
% the selection of child files must be performed manually (true).
% The definitions initialise to false:
%    \begin{macrocode}
\newif\ifchilddoc
\newif\ifchilddocmanual
%    \end{macrocode}

% \macro{\childdocname}
% \macro{\childdocjob}
% The macro |\childdocname| stores the name of the main document
% to be compiled. The macro |\childdocjob| stores the name of
% the document on which the \LaTeX{} compiler was originally invoked.
% The content of |\jobname| cannot be compared
% to filenames specified in the source due to different catcodes.
% The following code rescans |\jobname|, stores the result
% in |\childdocname| and saves a copy in |\childdocjob|:
%    \begin{macrocode}
\edef\childdocname{\scantokens\expandafter{\jobname\noexpand}}
\let\childdocjob\childdocname
%    \end{macrocode}

% \macro{\childdocdisable}
% The macro |\childdocdisable| prevents the main file
% from being processed more than once.
% At this stage, the main document command |\childdocmain|
% is assumed to be called once again where it should do nothing.
% Any subsequent call to it should prevent
% a secondary processing of the main document
% It overwrites the forwarding commands
% |\childdocof| and |\childdocforward|
% with empty macros to prevent further inclusions of the main document:
%    \begin{macrocode}
\newcommand{\childdocdisable}
{
  \renewcommand{\childdocmain}[1]{\renewcommand{\childdocmain}[1]{\endinput}}
  \renewcommand{\childdocof}[1]{}
  \renewcommand{\childdocby}[2][]{}
  \renewcommand{\childdocforward}[2][]{}
  \renewcommand{\childdocdisable}{}
}
%    \end{macrocode}

% \macro{\childdocmain}
% The macro |\childdocmain| is to be called at the top of the main file
% with nothing or the main filename (without extension) as argument.
% First, it breaks loops.
% If the argument is not empty and does not match |\childdocname|
% (which is set by the first inclusion of |childdoc.def|),
% |\ifchilddoc| is set to true, |\includeonly| is applied to the child file
% and |\jobname| is set to the main file
% (for proper handling of |.aux| files):
%    \begin{macrocode}
\newcommand{\childdocmain}[1]
{
  \childdocdisable\childdocmain{}
  \if?#1?\else
    \begingroup
      \def\childdoctmp{#1}
      \ifx\childdoctmp\childdocname
        \def\childdoctmp{}
      \else
        \def\childdoctmp
        {
          \childdoctrue
          \includeonly{\childdocname}
          \def\childdocjob{#1}
          \def\jobname{#1}
        }
      \fi
      \expandafter
    \endgroup
    \childdoctmp
  \fi
}
%    \end{macrocode}

% \macro{\childdocof}
% The command |\childdocof| redirects
% compilation to the main file |#1|.
%    \begin{macrocode}
\newcommand{\childdocof}[1]
{
  \childdocdisable
  \childdoctrue
  \includeonly{\childdocname}
  \def\jobname{#1}
  \def\childdocjob{#1}
  \input{#1}
}
%    \end{macrocode}

% \macro{\childdocby}
% The command |\childdocby| ....
%    \begin{macrocode}
\newcommand{\childdocby}[2][]
{
  \childdocdisable
  \childdoctrue
  \childdocmanualtrue
  \if?#1?\else
    \def\jobname{#2}
  \fi
  \def\childdocjob{#2}
  \input{#2}
  \endinput
}
%    \end{macrocode}

% \macro{\childdocforward}
% The command |\childdocforward| redirects
% compilation to the main file or
% (if the optional argument is given) a child file.
% Parameters are set as if the main file
% or a child file starting with |\childdocof| was compiled.
% Then compilation is handed over to the main file:
%    \begin{macrocode}
\newcommand{\childdocforward}[2][]
{
  \begingroup
    \if?#1?
      \def\childdoctmp
      {
        \def\childdocname{#2}
        \def\childdocjob{#2}
        \def\jobname{#2}
        \input{#2}
        \endinput
      }
    \else
      \def\childdoctmp
      {
        \childdocdisable
        \def\childdocname{#2}
        \childdoctrue
        \includeonly{#2}
        \def\childdocjob{#1}
        \def\jobname{#1}
        \input{#1}
        \endinput
      }
    \fi
    \expandafter
  \endgroup
  \childdoctmp
}
%    \end{macrocode}

% \macro{\childdocforwardprefix}
% The command |\childdocforwardprefix| redirects
% compilation to the main or a child file by means of a pattern.
% The prefix |#1| in the current filename is replaced by |#2|
% and the suffix of the current filename is kept
% (it is assumed that the filename does not contain the substring `|~~~|'
% which is used as a delimiter).
% Compilation is handed over to the new file by |\childdocforward|:
%    \begin{macrocode}
\newcommand{\childdocforwardprefix}[3][]
{
  \begingroup
    \def\childdocextract #2##1~~~{\def\childdoctmp{\childdocforward[#1]{#3##1}}}
    \expandafter\childdocextract\childdocname~~~
    \expandafter
  \endgroup
  \childdoctmp
}
%    \end{macrocode}

% \macro{\childdoc}
% The deprecated macro |\childdoc| is a legacy version of |\childdocmain|:
%    \begin{macrocode}
\newcommand{\childdoc}{\childdocmain}
%    \end{macrocode}

% \macro{\childdocredirect}
% The deprecated macro |\childdocredirect| is a legacy version
% of |\childdocforward| and |\childdocforwardprefix|:
%    \begin{macrocode}
\newcommand{\childdocredirect}[2][]
{
  \begingroup
    \if?#1?
      \def\childdoctmp{\childdocforward{#2}}
    \else
      \def\childdoctmp{\childdocforwardprefix{#1}{#2}}
    \fi
    \expandafter
  \endgroup
  \childdoctmp
}
%    \end{macrocode}

%\iffalse
%</package>
%\fi
%
\endinput
\childdocforward{cdocsamp}"|\\
% |latex -jobname cdocscl1 \|\\
% |  "% \iffalse
%
% childdoc.dtx Copyright (C) 2017-2018 Niklas Beisert
%
% This work may be distributed and/or modified under the
% conditions of the LaTeX Project Public License, either version 1.3
% of this license or (at your option) any later version.
% The latest version of this license is in
%   http://www.latex-project.org/lppl.txt
% and version 1.3 or later is part of all distributions of LaTeX
% version 2005/12/01 or later.
%
% This work has the LPPL maintenance status `maintained'.
%
% The Current Maintainer of this work is Niklas Beisert.
%
% This work consists of the files childdoc.dtx and childdoc.ins
% and the derived files childdoc.def and cdocsamp.tex with
% cdocsch1.tex, cdocsch2.tex, cdocsdrf.tex, cdocsfn1.tex, cdocsfn2.tex.
%
%<package>\ifdefined\childdocmain\endinput\fi
%<package>\ProvidesFile{childdoc.def}[2018/12/30 v2.0 child document driver]
%<samplemain>\ProvidesFile{cdocsamp.tex}[2018/12/30 v2.0 sample for childdoc]
%<*driver>
%\ProvidesFile{childdoc.drv}[2018/12/30 v2.0 childdoc reference manual file]
\PassOptionsToClass{10pt,a4paper}{article}
\documentclass{ltxdoc}

\usepackage[margin=35mm]{geometry}
\usepackage{hyperref}
\usepackage{hyperxmp}
\usepackage[usenames]{color}

\hypersetup{colorlinks=true}
\hypersetup{pdfstartview=FitH}
\hypersetup{pdfpagemode=UseNone}
\hypersetup{pdfsource={}}
\hypersetup{pdflang={en-UK}}
\hypersetup{pdfcopyright={Copyright 2017-2018 Niklas Beisert.
  This work may be distributed and/or modified under the
  conditions of the LaTeX Project Public License, either version 1.3
  of this license or (at your option) any later version.}}
\hypersetup{pdflicenseurl={http://www.latex-project.org/lppl.txt}}
\hypersetup{pdfcontactaddress={ETH Zurich, ITP, HIT K,
  Wolfgang-Pauli-Strasse 27}}
\hypersetup{pdfcontactpostcode={8093}}
\hypersetup{pdfcontactcity={Zurich}}
\hypersetup{pdfcontactcountry={Switzerland}}
\hypersetup{pdfcontactemail={nbeisert@itp.phys.ethz.ch}}
\hypersetup{pdfcontacturl={http://people.phys.ethz.ch/\xmptilde nbeisert/}}

\newcommand{\secref}[1]{\hyperref[#1]{section \ref*{#1}}}

\parskip1ex
\parindent0pt
\let\olditemize\itemize
\def\itemize{\olditemize\parskip0pt}

\begin{document}

\title{The \textsf{childdoc} Package}
\hypersetup{pdftitle={The childdoc Package}}
\author{Niklas Beisert\\[2ex]
  Institut f\"ur Theoretische Physik\\
  Eidgen\"ossische Technische Hochschule Z\"urich\\
  Wolfgang-Pauli-Strasse 27, 8093 Z\"urich, Switzerland\\[1ex]
  \href{mailto:nbeisert@itp.phys.ethz.ch}
  {\texttt{nbeisert@itp.phys.ethz.ch}}}
\hypersetup{pdfauthor={Niklas Beisert}}
\hypersetup{pdfsubject={Manual for the LaTeX2e Package childdoc}}
\date{30 December 2018, \textsf{v2.0}}
\maketitle

\begin{abstract}\noindent
\textsf{childdoc} is a \LaTeXe{} package
that enables the direct compilation
of document sections included by |\include|
to individual files.
\end{abstract}

\begingroup
\parskip0ex
\tableofcontents
\endgroup

%%%%%%%%%%%%%%%%%%%%%%%%%%%%%%%%%%%%%%%%%%%%%%%%%%%%%%%%%%%%%%%%%%%%%%%%%%%%%%%%
%%%%%%%%%%%%%%%%%%%%%%%%%%%%%%%%%%%%%%%%%%%%%%%%%%%%%%%%%%%%%%%%%%%%%%%%%%%%%%%%
\section{Introduction}

\LaTeX{} provides a mechanism to structure a large document (such as a book)
into a main file and several child files (containing the chapters)
using the |\include| command.
This mechanism is beneficial for documents
which span hundreds of pages in order to
make the source file(s) more manageable.
Moreover, compilation can be restricted to
selected child files by means of the |\includeonly| command.
The latter feature can be used to reduce the compilation time while editing
(this was significantly more useful in the earlier days of \LaTeX{})
or to generate a smaller document which is easier to navigate.
Another application of |\includeonly| is to generate
documents consisting of selected parts of the complete document.

However, there are a few drawbacks of the plain |\include| mechanism:
\begin{itemize}
\item
The child files cannot be compiled on their own,
they can only be compiled via the main file.
A naive editing environment
(such as a text editor with an option
to have the current file processed by \LaTeX)
may require one to switch to the main file before compiling;
attempting to compile the child file produces errors.
\item
The main file must be modified (each time)
to adjust the |\includeonly| command
to the present needs. This easily leaves the main file in a messy state.
\item
The generated document will always carry the filename
of the main document. This is inconvenient if
several child files are to be compiled and
to be kept for distribution.
\end{itemize}

The present package provides a simple interface
to make child files individually compilable by \LaTeX{}.
Compiling a child file then has the same effect as compiling
the main file with an |\includeonly| command
to select the appropriate child.
Moreover the generated document will carry the name of the child
rather than the main file.
This resolves all three above issues.

This feature is meant to make the editing of books,
thesis documents and lecture notes somewhat more convenient.
However, the package can also be used efficiently for
composing a series of documents (such as exercise sheets)
which are typically distributed individually.
It then assists the author in generating the individual documents
(potentially in different versions)
as well as a document containing the collected series.
Another application is in developing style files
or other kinds of included material
where compilation of the style file could redirect
to a sample or test file.

%%%%%%%%%%%%%%%%%%%%%%%%%%%%%%%%%%%%%%%%%%%%%%%%%%%%%%%%%%%%%%%%%%%%%%%%%%%%%%%%
%%%%%%%%%%%%%%%%%%%%%%%%%%%%%%%%%%%%%%%%%%%%%%%%%%%%%%%%%%%%%%%%%%%%%%%%%%%%%%%%
\section{Usage}

First of all, the package \textsf{childdoc} is \emph{not} a standard
\LaTeXe{} |.sty| style file! Therefore it needs to be invoked in
a non-standard way.

%%%%%%%%%%%%%%%%%%%%%%%%%%%%%%%%%%%%%%%%%%%%%%%%%%%%%%%%%%%%%%%%%%%%%%%%%%%%%%%%
\subsection{Included Files}
\label{sec:include}

%%%%%%%%%%%%%%%%%%%%%%%%%%%%%%%%%%%%%%%%
\DescribeMacro{\childdocmain}
To use the package, add the commands
\begin{center}
\begin{tabular}{l}
|\input{childdoc.def}|\\
|\childdocmain{}|\\
\end{tabular}
\end{center}
at the very top of the main \LaTeX{} file,
in particular \emph{before} the |\documentclass| statement!
The argument of |\childdocmain| should be left empty
(but it must be present).

%%%%%%%%%%%%%%%%%%%%%%%%%%%%%%%%%%%%%%%%
\DescribeMacro{\childdocof}
Furthermore, add the commands
\begin{center}
\begin{tabular}{l}
|\input{childdoc.def}|\\
|\childdocof{|\textit{main}|}|\\
\end{tabular}
\end{center}
at the top of every child file \textit{child}
which is included by |\include{|\textit{child}|}|
from within the main file
(or at least for those files to be compiled individually).
The argument \textit{main} must be the filename of the main file.

There are a couple of
considerations in setting up the main and child documents:

%%%%%%%%%%%%%%%%%%%%%%%%%%%%%%%%%%%%%%%%
\paragraph{Restrictions.}

Please note the following restrictions:
\begin{itemize}
\item
|\childdocmain| must be called with one argument \textit{main}
to ensure compatibility with earlier version of the package.
It must either be empty (|\childdocmain{}|)
or precisely match the filename of the main file in which it is specified.
See \secref{sec:detection} for further information.
\item
The filename \textit{main} must be specified without the |.tex| extension.
\item
The filename \textit{main} is case sensitive
(even in case-insensitive file systems)
due to internal string comparison.
\item
The argument \textit{main} should be fully expanded, it cannot be a macro.
\item
Subdirectories and special characters should be avoided in filenames.
\item
The command |\childdocmain{|\textit{main}|}| must be followed by a whitespace.
It should not be followed immediately by another command
or by a comment mark `|%|'.
This is because the \TeX{} parser reads the token immediately following
the argument of |\childdocmain| and puts it
at the beginning of every child section;
however, a white\-space is ignored.
\end{itemize}

%%%%%%%%%%%%%%%%%%%%%%%%%%%%%%%%%%%%%%%%
\paragraph{Content of Main File.}

It is advisable to place all content in the child files included by |\include|.
Any output contained in the main file will appear in all child documents
unless suppressed manually;
it cannot be suppressed automatically by the |\includeonly| directive
and thus should normally be avoided.
A method to include some content in the main file
by means of conditional processing is described in \secref{sec:conditional}.

%%%%%%%%%%%%%%%%%%%%%%%%%%%%%%%%%%%%%%%%
\paragraph{Page Numbering.}

When only a part of the document is compiled,
the appropriate numbering of pages
(as well as other status parameters)
is determined from the |.aux| files.
The latter contain information from previous passes.
However this information needs to propagate through
all intermediate child documents.
Therefore the page numbering in child documents may well
be inconsistent until the complete document is compiled at least once.

A useful (if unconventional) way to always ensure a consistent
page numbering is to restart the numbering in each child document
and denote the pages by `\textit{child}|.|\textit{page}'
where \textit{child} represents the chapter/section number of the child file.
This can be achieved by the command
|\numberwithin{page}{|\textit{child}|}|
of the \textsf{amsmath} package
where \textit{child} can be |chapter| or |section|
depending on the chosen structuring.
Alternatively, one can modify the macro |\thepage| appropriately
and reset the counter |page| at the start of each child file.

%%%%%%%%%%%%%%%%%%%%%%%%%%%%%%%%%%%%%%%%%%%%%%%%%%%%%%%%%%%%%%%%%%%%%%%%%%%%%%%%
\subsection{Conditional Processing}
\label{sec:conditional}

The package provides a mechanism to compile different versions
of a document. To customise the versions further some conditional processing
can come in handy to distinguish which version is being compiled.
The package provides two macros to describe the compilation context:

%%%%%%%%%%%%%%%%%%%%%%%%%%%%%%%%%%%%%%%%
\DescribeMacro{\ifchilddoc}
The conditional |\ifchilddoc| distinguishes between the compilation of
child documents and the main document:
%
\begin{center}
|\ifchilddoc |\textit{child-code}| |[|\||else |\textit{main-code}]| \||fi|
\end{center}

%%%%%%%%%%%%%%%%%%%%%%%%%%%%%%%%%%%%%%%%
\DescribeMacro{\childdocname}
\DescribeMacro{\childdocjob}
The macro |\childdocname| contains the filename (without extension)
of the main or child file being processed.
Note that |\childdocjob| will always contain the name of the main file.

%%%%%%%%%%%%%%%%%%%%%%%%%%%%%%%%%%%%%%%%
\paragraph{Title Page.}

Conditional processing can be used to include a title or banner page
in the main document when proper precautions are taken.
Importantly, the code in the main file should ensure that the page counter
(as well as other status parameters which are stored in the |.aux| files)
takes the same value after the conditional processing.
Otherwise the page numbers may take divergent values
depending on which part is compiled.

For example, a title page could be declared by:
%
\begin{center}
\begin{tabular}{l}
|\ifchilddoc\||else|\\
|\addtocounter{page}{-1}|\\
\textit{code for title page}\\
|\newpage|\\
|\||fi|
\end{tabular}
\end{center}
%
A banner page for the child documents can be generated by:
%
\begin{center}
\begin{tabular}{l}
|\ifchilddoc|\\
|\addtocounter{page}{-1}|\\
\textit{code for banner page}\\
|\newpage|\\
|\||fi|
\end{tabular}
\end{center}
%
Here one could write a message such as:
\begin{center}
|This is the part \childdocname{} of \childdocjob{}.|
\end{center}

%%%%%%%%%%%%%%%%%%%%%%%%%%%%%%%%%%%%%%%%%%%%%%%%%%%%%%%%%%%%%%%%%%%%%%%%%%%%%%%%
\subsection{Flags}
\label{sec:flags}

The package makes it easy to generate different versions
of the main or child documents.
To this end compilation flags can be defined
and assigned different default values.
They will be particularly useful in conjunction
with the forwarding mechanism described in \secref{sec:forward}.

For example, it may be useful to have a flag |\version|
which can be set to |draft| or |final|.
The document source will contain some conditional code
depending on the value of |\version|.
Suppose further, the flag should default to |final| for the main file
and to |draft| for child files
which is a natural assignment for editing the document.
This is achieved by placing the following code
in the preamble of the main document
(below the |\childdocmain| directive):
%
\begin{center}
\begin{tabular}{l}
|\ifchilddoc|\\
|\providecommand{\version}{draft}|\\
|\||else|\\
|\providecommand{\version}{final}|\\
|\||fi|
\end{tabular}
\end{center}
%
The definition by |\providecommand| makes sure
that previous definitions are not overwritten.
Further statements |\providecommand{\version}{...}|
can thus be added before the above code to override it.

For the main file, one might add a line
(between |\childdocmain| and the above block)
%
\begin{center}
|%\ifchilddoc\||else\providecommand{\version}{draft}\||fi|
\end{center}
%
which can be uncommented to produce a draft version.
Likewise one can add a line to the very top of a child file
(above the |\childdocof{|\textit{main}|}| directive)
%
\begin{center}
|%\providecommand{\version}{final}|
\end{center}
%
which can be uncommented to produce the final version of this child document.

%%%%%%%%%%%%%%%%%%%%%%%%%%%%%%%%%%%%%%%%%%%%%%%%%%%%%%%%%%%%%%%%%%%%%%%%%%%%%%%%
\subsection{Forwarding}
\label{sec:forward}

Different versions of the main or child documents
using compilation flags as described in \secref{sec:flags}
can be (permanently) stored in different files
for convenient compilation, viewing and distribution.
To this end, the package defines a command
to pass on compilation to a different file:

%%%%%%%%%%%%%%%%%%%%%%%%%%%%%%%%%%%%%%%%
\DescribeMacro{\childdocforward}
The command |\childdocforward| redirects processing to
another source file:
%
\begin{center}
\begin{tabular}{l}
|\input{childdoc.def}|\\
|\childdocforward[|\textit{main}|]{|\textit{dest}|}|\\
\end{tabular}
\end{center}
%
The argument \textit{dest} is the destination file
(without extension).
It should be the main file or one of the child files.
Note that further \textsf{childdoc} directives
such as |\childdocof| and |\childdocforward|
in the indicated file will be processed in this form.
The optional argument \textit{main}
passes on directly to the main file \textit{main}
while pretending to compile the child \textit{dest}.
This form behaves as if \textit{dest}
issues |\childdocof{|\textit{main}|}| right away,
and no further \textsf{childdoc} directives will be processed.

%%%%%%%%%%%%%%%%%%%%%%%%%%%%%%%%%%%%%%%%
\DescribeMacro{\...prefix}
In the alternative form |\childdocforwardprefix|,
%
\begin{center}
\begin{tabular}{l}
|\input{childdoc.def}|\\
|\childdocforwardprefix[|\textit{main}|]{|\textit{prefix}|}{|\textit{dest}|}|
\end{tabular}
\end{center}
%
the destination file is determined by a pattern
depending on the current file:
To make this work, the current file must be called
`{\textit{prefix}\hspace{0.2em}\textit{suffix}}'
with \textit{prefix} matching precisely the argument.
Processing is then passed on to the file
`{\textit{dest}\hspace{0.2em}\textit{suffix}}'.
Surely, the same effect is achieved by
directly specifying the
argument `{\textit{dest}\hspace{0.2em}\textit{suffix}}'
in the first form.
However, that requires to set up a different file
for each child. With the alternative form of the command
all these files can have exactly the same content
which simplifies setting them up and maintaining them.

For example, the following file |draft.tex|
with a compilation flag |\version| as described in \secref{sec:flags}
compiles the main document as a draft:
%
\begin{center}
\begin{tabular}{l}
|\def\version{draft}|\\
|\input{childdoc.def}|\\
|\childdocforward{|\textit{main}|}|
\end{tabular}
\end{center}
%
Likewise, the following files |final|\textit{nn}|.tex|
compile the final version of the child document
|child|\textit{nn}|.tex|:
%
\begin{center}
\begin{tabular}{l}
|\def\version{final}|\\
|\input{childdoc.def}|\\
|\childdocforwardprefix{final}{child}|
\end{tabular}
\end{center}
%

Note that when several versions of a main file and/or of each child file
are to be generated, it may be convenient to set up a |Makefile| or
shell script to automatise the process.

%%%%%%%%%%%%%%%%%%%%%%%%%%%%%%%%%%%%%%%%%%%%%%%%%%%%%%%%%%%%%%%%%%%%%%%%%%%%%%%%
\subsection{Command Line Processing}
\label{sec:commandline}

The effect of redirection files can also be achieved by invoking
the \LaTeX{} compiler with a more elaborate command line.
Most conveniently this should be done as part
of a shell script or a |Makefile|.

When using \textsf{childdoc} in the main file, the following
command lines effectively perform a redirection
(note that depending on the shell being used,
backslashes may have to be doubled: `|\|' $\to$ `|\\|'):
%
\begin{center}
|... -jobname "|\textit{target}|" |\\|"|[\textit{flags}]%
|\input{childdoc.def}\childdocforward[|\textit{main}|]{|\textit{dest}|}"|
\end{center}
%
Here \textit{target} is the name of the output file,
\textit{main} is the name of the main file
and \textit{dest} is the name of the main or child file to be processed
(all filenames without extensions).
The optional argument \textit{main} can be omitted
if \textit{main} matches \textit{dest}.
Optionally, compilation \textit{flags} can be defined via |\def| commands.
This command line makes the \TeX{} engine believe
it is compiling the file \textit{target}
whose content is specified as the latter parameter.
The provided code then forwards the processing to
\textit{main} or \textit{dest} as described in \secref{sec:forward}.

%%%%%%%%%%%%%%%%%%%%%%%%%%%%%%%%%%%%%%%%%%%%%%%%%%%%%%%%%%%%%%%%%%%%%%%%%%%%%%%%
\subsection{Include by Input}
\label{sec:input}

Including child documents by |\include| has some restrictions by design.
Most notably, the content of a child document always occupies
its own set of pages; pages cannot be shared between child documents.
Usually, this behaviour makes perfect sense
because each child document contain an essential part of the document.
However, in some situations it may be desirable to compose
a document from a collection of parts
without having mandatory page breaks between then.
For this case, the package
provides a mechanism to include parts
by |\input| which can also be processed individually.
However, by construction this mechanism
requires manual handling of the content to be output.

%%%%%%%%%%%%%%%%%%%%%%%%%%%%%%%%%%%%%%%%
\DescribeMacro{\ifchilddocmanual}
The main file should be prepared as usual, see \secref{sec:include}.
However, the document body must make a distinction
between processing of an individual part and of the main document, e.g.:
%
\begin{center}
\begin{tabular}{l}
|\ifchilddocmanual|\\
|\input{\childdocname}|\\
|\||else|\\
\textit{document body with }|\input{|\textit{part}|}|\\
|\||fi|
\end{tabular}
\end{center}
%
The conditional |\ifchilddocmanual| is true whenever
a part to be included by |\input| is being compiled,
and the name of the part is stored in |\childdocname|.

%%%%%%%%%%%%%%%%%%%%%%%%%%%%%%%%%%%%%%%%
\DescribeMacro{\childdocby}
Each part to be included by |\input| should start with:
%
\begin{center}
\begin{tabular}{l}
|\input{childdoc.def}|\\
|\childdocby{|\textit{main}|}|\\
\end{tabular}
\end{center}
%
The directive |\childdocby| is similar to |\childdocof|
described in \secref{sec:include},
but the subsequent selection of content must be done manually.
To that end, both |\ifchilddoc| and |\ifchilddocmanual|
will be true upon processing of a part,
and the name of the part is stored in |\childdocname|.
Note that |\jobname| will be set to the filename of the current part
so that each part receives an individual |.aux| file
that does not interfere with the |.aux| file(s) of the main document.
This behaviour can be altered by the alternative form
|\childdocby[*]{|\textit{main}|}| (with a non-empty optional argument)
which uses the |.aux| file of the main document
by setting |\jobname| to \textit{main}.

%%%%%%%%%%%%%%%%%%%%%%%%%%%%%%%%%%%%%%%%%%%%%%%%%%%%%%%%%%%%%%%%%%%%%%%%%%%%%%%%
\subsection{Driver Development}
\label{sec:driver}

The \textsf{childdoc} mechanism can also be use for the development
of definition files such as \LaTeX{} styles or classes.
This case differs from the above setup with multiple parts
included by |\include| in that no |\includeonly| should be invoked.
This can be achieved by starting the include file
(before |\ProvidesPackage|) with:
%
\begin{center}
\begin{tabular}{l}
|\input{childdoc.def}|\\
|\childdocforward{|\textit{main}|}|\\
\end{tabular}
\end{center}
%
or alternatively with:
%
\begin{center}
\begin{tabular}{l}
|\input{childdoc.def}|\\
|\childdocby{|\textit{main}|}|\\
\end{tabular}
\end{center}
%
Both forms have slightly different effects as described above.
The main file is prepared as usual, see \secref{sec:include}.

%%%%%%%%%%%%%%%%%%%%%%%%%%%%%%%%%%%%%%%%%%%%%%%%%%%%%%%%%%%%%%%%%%%%%%%%%%%%%%%%
\subsection{Legacy Detection}
\label{sec:detection}

The directive |\childdocmain| in the main file can detect
whether the complete document or merely a child is to be compiled
even without using the directive |\childdocof|.
This method is deprecated because it is less robust
and there is no compelling reason to use it;
it is merely provided for backward compatibility
and it may be removed in future versions.

If the detection mechanism is to be used,
it is mandatory to correctly specify
the filename of the main file as the argument of |\childdocmain|:
%
\begin{center}
\begin{tabular}{l}
|\input{childdoc.def}|\\
|\childdocmain{|\textit{main}|}|\\
\end{tabular}
\end{center}
%
If |\jobname| does not match the argument \textit{main} of |\childdocmain|,
it is assumed that |\jobname| points to the child file to be compiled.
When using |\childdocmain| with the main file specified as argument,
it suffices to start a child file
with just |\input{|\textit{main}|}|
without loading of the package and using |\childdocof|.
If instead all processing is done
with the appropriate \textsf{childdoc} directives,
the argument of \textit{main} of |\childdocmain| can be empty.

An alternative version of the command line processing described
in \secref{sec:commandline} using the detection mechanism reads:
%
\begin{center}
|... -jobname "|\textit{target}|" "|[\textit{flags}]%
[|\def\jobname{|\textit{dest}|}|]|\input{|\textit{main}|}"|
\end{center}

%%%%%%%%%%%%%%%%%%%%%%%%%%%%%%%%%%%%%%%%%%%%%%%%%%%%%%%%%%%%%%%%%%%%%%%%%%%%%%%%
\subsection{Manual Code}
\label{sec:manual}

In case one cannot be certain whether the definitions file |childdoc.def|
is installed on the target \TeX{} distribution
and one prefers not to ship it,
it is conceivable to paste a few relevant commands into the sources.

To that end, drop all statements |\input{childdoc.def}|
and perform the replacements as outlined below.
Instead of |\childdocmain{|\textit{main}|}| add the following code
to the top of the main file:
%
\begin{center}
\begin{tabular}{l}
|\||ifdefined\childdocname\endinput\||fi\newif\ifchilddoc|\\
|\edef\childdocname{\scantokens\expandafter{\jobname\noexpand}}|\\
|\def\childdocmain{|\textit{main}|}\||ifx\childdocmain\childdocname\||else|\\
|\childdoctrue\includeonly{\childdocname}\let\jobname\childdocmain\||fi|\\
\end{tabular}
\end{center}
%
Instead of |\childdocof{|\textit{main}|}| just include the main file
at the top of each child file:
%
\begin{center}
|\input{|\textit{main}|}|
\end{center}
%
A simple redirection |\childdocforward{|\textit{dest}|}| is achieved by:
%
\begin{center}
|\def\jobname{|\textit{dest}|}\input{\jobname}|
\end{center}
%
The redirection with prefix
|\childdocforwardprefix[|\textit{prefix}|]{|\textit{dest}|}|
is accomplished by:
%
\begin{center}
\begin{tabular}{l}
|{\edef\jobname{\scantokens\expandafter{\jobname\noexpand}}|\\
|\def\redirectjob |\textit{prefix}|#1~~~{\gdef\jobname{|\textit{dest}|#1}}|\\
|\expandafter\redirectjob\jobname~~~}\input{\jobname}|
\end{tabular}
\end{center}

In an alternative approach,
child documents can be compiled by a specific command line
without additional code or specific definitions:
%
\begin{center}
|... -jobname "|\textit{target}|" "|[\textit{flags}]%
|\includeonly{|\textit{dest}|}\input{|\textit{main}|}"|
\end{center}
%

%%%%%%%%%%%%%%%%%%%%%%%%%%%%%%%%%%%%%%%%%%%%%%%%%%%%%%%%%%%%%%%%%%%%%%%%%%%%%%%%
%%%%%%%%%%%%%%%%%%%%%%%%%%%%%%%%%%%%%%%%%%%%%%%%%%%%%%%%%%%%%%%%%%%%%%%%%%%%%%%%
\section{Information}

%%%%%%%%%%%%%%%%%%%%%%%%%%%%%%%%%%%%%%%%%%%%%%%%%%%%%%%%%%%%%%%%%%%%%%%%%%%%%%%%
\subsection{Copyright}

Copyright \copyright{} 2017--2018 Niklas Beisert

This work may be distributed and/or modified under the
conditions of the \LaTeX{} Project Public License, either version 1.3
of this license or (at your option) any later version.
The latest version of this license is in
  \url{http://www.latex-project.org/lppl.txt}
and version 1.3 or later is part of all distributions of \LaTeX{}
version 2005/12/01 or later.

This work has the LPPL maintenance status `maintained'.

The Current Maintainer of this work is Niklas Beisert.

This work consists of the files |README.txt|, |childdoc.ins| and |childdoc.dtx|
as well as the derived files |childdoc.def|, |cdocsamp.tex|
with |cdocsch1.tex|, |cdocsch2.tex|, |cdocspt3.tex|, |cdocspt4.tex|,
|cdocsdrf.tex|, |cdocsfn1.tex|, |cdocsfn2.tex|
as well as |childdoc.pdf|.

%%%%%%%%%%%%%%%%%%%%%%%%%%%%%%%%%%%%%%%%%%%%%%%%%%%%%%%%%%%%%%%%%%%%%%%%%%%%%%%%
\subsection{Files and Installation}

The package consists of the files:
%
\begin{center}
\begin{tabular}{ll}
    |README.txt|   & readme file \\
    |childdoc.ins| & installation file \\
    |childdoc.dtx| & source file \\
    |childdoc.def| & definition file \\
    |cdocsamp.tex| & sample main file \\
    |cdocsch1.tex| & sample include file \\
    |cdocsch2.tex| & sample include file \\
    |cdocspt3.tex| & sample part file \\
    |cdocspt4.tex| & sample part file \\
    |cdocsdrf.tex| & sample redirection file \\
    |cdocsfn1.tex| & sample redirection file \\
    |cdocsfn2.tex| & sample redirection file \\
    |childdoc.pdf| & manual
\end{tabular}
\end{center}
%
The distribution consists of the files
|README.txt|, |childdoc.ins| and |childdoc.dtx|.
%
\begin{itemize}
\item
Run (pdf)\LaTeX{} on |childdoc.dtx|
to compile the manual |childdoc.pdf| (this file).
\item
Run \LaTeX{} on |childdoc.ins| to create the definitions file |childdoc.def|
and the sample |cdocsamp.tex| with include files
|cdocsch1.tex|, |cdocsch2.tex|, |cdocspt3.tex|, |cdocspt4.tex|,
|cdocsdrf.tex|, |cdocsfn1.tex|, |cdocsfn2.tex|.
Then copy the file |childdoc.def| to an appropriate directory of your \LaTeX{}
distribution, e.g.\ \textit{texmf-root}|/tex/latex/childdoc|.
\end{itemize}

%%%%%%%%%%%%%%%%%%%%%%%%%%%%%%%%%%%%%%%%%%%%%%%%%%%%%%%%%%%%%%%%%%%%%%%%%%%%%%%%
\subsection{Related CTAN Packages}

There are several other packages which offer a similar functionality:
%
\begin{itemize}
\item
The packages
\href{http://ctan.org/pkg/docmute}{\textsf{docmute}},
\href{http://ctan.org/pkg/includex}{\textsf{includex}} and
\href{http://ctan.org/pkg/standalone}{\textsf{standalone}}
provide commands to include only the document body of
a child file thus allowing both files to be compiled individually.
\item
The packages \href{http://ctan.org/pkg/subdocs}{\textsf{subdocs}}
and \href{http://ctan.org/pkg/subfiles}{\textsf{subfiles}}
provide structures in which the main and child documents can be
encapsulated and allowing them to be compiled individually.
The inclusion mechanism is different from the conventional |\include|.
\item
The package \href{http://ctan.org/pkg/combine}{\textsf{combine}}
is an elaborate solution to combine several documents into one.
\end{itemize}
%
See also the CTAN topic \href{http://ctan.org/topic/subdocs}{\textsf{subdocs}}
for further related packages.
The present package differs from the above solutions in that
a document structure constructed with the conventional |\include| mechanism
just needs two extra commands at the top of every file
such that all constituent files can be compiled individually.

%%%%%%%%%%%%%%%%%%%%%%%%%%%%%%%%%%%%%%%%%%%%%%%%%%%%%%%%%%%%%%%%%%%%%%%%%%%%%%%%
%\subsection{Feature Suggestions}
%
%The following is a list of features which may be useful for future
%versions of this package:
%%
%\begin{itemize}
%\item
%\ldots
%\end{itemize}

%%%%%%%%%%%%%%%%%%%%%%%%%%%%%%%%%%%%%%%%%%%%%%%%%%%%%%%%%%%%%%%%%%%%%%%%%%%%%%%%
\subsection{Revision History}

%%%%%%%%%%%%%%%%%%%%%%%%%%%%%%%%%%%%%%%%
\paragraph{v2.0:} 2018/12/30

\begin{itemize}
\item
immediate forward processing
\item
added |\childdocby| mechanism
\item
manual restructured
\end{itemize}

%%%%%%%%%%%%%%%%%%%%%%%%%%%%%%%%%%%%%%%%
\paragraph{v1.6:} 2018/01/17

\begin{itemize}
\item
application for development of include files
\item
corrections to manual
\end{itemize}

%%%%%%%%%%%%%%%%%%%%%%%%%%%%%%%%%%%%%%%%
\paragraph{v1.5:} 2017/05/21

\begin{itemize}
\item
more complete structuring introduced
\item
|\childdocof| introduced
\item
|\childdoc| renamed to |\childdocmain|
\item
|\childredirect| renamed to |\childdocforward| and |\childdocforwardprefix|
and functionality expanded
\end{itemize}

%%%%%%%%%%%%%%%%%%%%%%%%%%%%%%%%%%%%%%%%
\paragraph{v1.0:} 2017/04/27

\begin{itemize}
\item
manual and install package
\item
first version published on CTAN
\end{itemize}

%%%%%%%%%%%%%%%%%%%%%%%%%%%%%%%%%%%%%%%%
\paragraph{v0.6:} 2017/04/26

\begin{itemize}
\item
redirection mechanism added
\end{itemize}

%%%%%%%%%%%%%%%%%%%%%%%%%%%%%%%%%%%%%%%%
\paragraph{v0.5:} 2017/04/26

\begin{itemize}
\item
functionality in definition file
\end{itemize}


%%%%%%%%%%%%%%%%%%%%%%%%%%%%%%%%%%%%%%%%%%%%%%%%%%%%%%%%%%%%%%%%%%%%%%%%%%%%%%%%
%%%%%%%%%%%%%%%%%%%%%%%%%%%%%%%%%%%%%%%%%%%%%%%%%%%%%%%%%%%%%%%%%%%%%%%%%%%%%%%%
%%%%%%%%%%%%%%%%%%%%%%%%%%%%%%%%%%%%%%%%%%%%%%%%%%%%%%%%%%%%%%%%%%%%%%%%%%%%%%%%
\appendix

\settowidth\MacroIndent{\rmfamily\scriptsize 000\ }

 \DocInput{childdoc.dtx}

\end{document}
%</driver>
% \fi
%
% %%%%%%%%%%%%%%%%%%%%%%%%%%%%%%%%%%%%%%%%%%%%%%%%%%%%%%%%%%%%%%%%%%%%%%%%%%%%%%
% %%%%%%%%%%%%%%%%%%%%%%%%%%%%%%%%%%%%%%%%%%%%%%%%%%%%%%%%%%%%%%%%%%%%%%%%%%%%%%
% \section{Sample}
%\iffalse
%<*samplemain>
%\fi
%
% The following presents a sample document
% with two chapters, two parts, a title page,
% a compile flag as well as three forwarding files to set the flag.
% It consists of eight |.tex| files:
% \begin{center}
% \begin{tabular}{ll}
% |cdocsamp.tex|&main file\\
% |cdocsch1.tex|&include file for chapter 1\\
% |cdocsch2.tex|&include file for chapter 2\\
% |cdocspt3.tex|&include file for part 3\\
% |cdocspt4.tex|&include file for part 4\\
% |cdocsdrf.tex|&forwarding file for main file in draft mode\\
% |cdocsfi1.tex|&forwarding file for final version of chapter 1\\
% |cdocsfi2.tex|&forwarding file for final version of chapter 2\\
% \end{tabular}
% \end{center}
% Each of the eight files can be compiled directly by the \LaTeX{} compiler.
%
% %%%%%%%%%%%%%%%%%%%%%%%%%%%%%%%%%%%%%%
% \paragraph{Main File.}
%
% The main file is called |cdocsamp.tex|.
%
% Load the \textsf{childdoc} definitions and
% declare the filename for the main document:
%    \begin{macrocode}
\input{childdoc.def}
\childdocmain{}
%    \end{macrocode}

% Optional override for |\version| flag:
%    \begin{macrocode}
%%\ifchilddoc\else\providecommand{\version}{draft}\fi
%    \end{macrocode}

% Define the default values for the |\version| flag
% (|final| for the main file and |draft| for childs):
%    \begin{macrocode}
\ifchilddoc
\providecommand{\version}{draft}
\else
\providecommand{\version}{final}
\fi
%    \end{macrocode}

% Load the standard document class:
%    \begin{macrocode}
\documentclass[12pt]{article}
%    \end{macrocode}

% Start the document body:
%    \begin{macrocode}
\begin{document}
%    \end{macrocode}

% Declare a title page.
% Print title, part of document being processed and version flag:
%    \begin{macrocode}
\addtocounter{page}{-1}
\begin{center}
{\LARGE\bfseries{}childdoc example\par}
\vspace{1cm}
\ifchilddoc
\ifchilddocmanual part\else chapter\fi:
`\childdocname' of `\childdocjob'\par
\else
main document: `\childdocjob'\par
\fi
version: \version\par
\end{center}
\newpage
%    \end{macrocode}

% Manually include selected file,
% otherwise process as usual:
%    \begin{macrocode}
\ifchilddocmanual
\section*{part `\childdocname'}
\input{\childdocname}
\else
%    \end{macrocode}

% Include the two chapters:
%    \begin{macrocode}
\include{cdocsch1}
\include{cdocsch2}
%    \end{macrocode}

% Include the two parts unless only chapters should be displayed:
%    \begin{macrocode}
\ifchilddoc\else
\section{part three}
\input{cdocspt3}
\section{part four}
\input{cdocspt4}
\fi
%    \end{macrocode}

% Process as usual until here:
%    \begin{macrocode}
\fi
%    \end{macrocode}

% End of document body:
%    \begin{macrocode}
\end{document}
%    \end{macrocode}
%\iffalse
%</samplemain>
%\fi
%
% %%%%%%%%%%%%%%%%%%%%%%%%%%%%%%%%%%%%%%
% \paragraph{Chapter Include Files.}
%
% The include files are called |cdocsch1.tex| and |cdocsch2.tex|.
%
%\iffalse
%<*samplechap1|samplechap2>
%\fi

% Optional override for |\version| flag:
%    \begin{macrocode}
%%\providecommand{\version}{final}
%    \end{macrocode}

% Include the main document:
%    \begin{macrocode}
\input{childdoc.def}
\childdocof{cdocsamp}
%    \end{macrocode}

%\iffalse
%</samplechap1|samplechap2>
%\fi
%
%\iffalse
%<*samplechap1>
%\fi
% Some text for chapter 1:
%    \begin{macrocode}
\section{one}
some text in chapter one
%    \end{macrocode}

%\iffalse
%</samplechap1>
%\fi
% Some text for chapter 2:
%\iffalse
%<*samplechap2>
%\fi
%    \begin{macrocode}
\section{two}
more text in chapter two
%    \end{macrocode}

%\iffalse
%</samplechap2>
%\fi
%
% %%%%%%%%%%%%%%%%%%%%%%%%%%%%%%%%%%%%%%
% \paragraph{Part Include Files.}
%
% The include files are called |cdocspt3.tex| and |cdocspt4.tex|.
%
%\iffalse
%<*samplepart3|samplepart4>
%\fi

% Optional override for |\version| flag:
%    \begin{macrocode}
%%\providecommand{\version}{final}
%    \end{macrocode}

% Include the main document:
%    \begin{macrocode}
\input{childdoc.def}
\childdocby{cdocsamp}
%    \end{macrocode}

%\iffalse
%</samplepart3|samplepart4>
%\fi
%
%\iffalse
%<*samplepart3>
%\fi
% Some text for part 3:
%    \begin{macrocode}
some text in part three
%    \end{macrocode}

%\iffalse
%</samplepart3>
%\fi
% Some text for part 4:
%\iffalse
%<*samplepart4>
%\fi
%    \begin{macrocode}
more text in part four
%    \end{macrocode}

%\iffalse
%</samplepart4>
%\fi
%
% %%%%%%%%%%%%%%%%%%%%%%%%%%%%%%%%%%%%%%
% \paragraph{Forwarding for a Complete Draft.}
%
% The following forwarding file |cdocsdrf.tex|
% compiles the main document in draft mode:
%\iffalse
%<*sampledraft>
%\fi
%    \begin{macrocode}
\def\version{draft}
\input{childdoc.def}
\childdocforward{cdocsamp}
%    \end{macrocode}

%\iffalse
%</sampledraft>
%\fi
%
% %%%%%%%%%%%%%%%%%%%%%%%%%%%%%%%%%%%%%%
% \paragraph{Forwarding for Final Version of the Chapters.}
%
% The following forwarding files |cdocsfn1.tex| and |cdocsfn2.tex|
% (with identical content)
% compile the final versions of the child documents
% |cdocsch1.tex| and |cdocsch2.tex|, respectively:
%\iffalse
%<*samplefinal>
%\fi
%    \begin{macrocode}
\def\version{final}
\input{childdoc.def}
\childdocforwardprefix[cdocsamp]{cdocsfn}{cdocsch}
%    \end{macrocode}

%\iffalse
%</samplefinal>
%\fi
%
% %%%%%%%%%%%%%%%%%%%%%%%%%%%%%%%%%%%%%%
% \paragraph{Command Line Processing.}
%
% The following three command lines generate the output files
% |cdocscld|, |cdocscl1| and |cdocscl2|
% which should be identical to
% |cdocsdrf|, |cdocsch1| and |cdocsfn2|, respectively:
% \begin{center}
% \begin{tabular}{l}
% |latex -jobname cdocscld \|\\
% |  "\def\version{draft}\input{childdoc.def}\childdocforward{cdocsamp}"|\\
% |latex -jobname cdocscl1 \|\\
% |  "\input{childdoc.def}\childdocforward[cdocsamp]{cdocsch1}"|\\
% |latex -jobname cdocscl2 \|\\
% |  "\def\version{final}\input{childdoc.def}\childdocforward{cdocsch2}"|
% \end{tabular}
% \end{center}
% Note that the trailing backslash on each first line
% merely continues the input to the second line
% (for convenient cut ant paste).
% Furthermore, the command |latex| can be replaced by any
% of its alternative versions such as |pdflatex|.
%
% %%%%%%%%%%%%%%%%%%%%%%%%%%%%%%%%%%%%%%%%%%%%%%%%%%%%%%%%%%%%%%%%%%%%%%%%%%%%%%
% %%%%%%%%%%%%%%%%%%%%%%%%%%%%%%%%%%%%%%%%%%%%%%%%%%%%%%%%%%%%%%%%%%%%%%%%%%%%%%
% \section{Implementation}
%\iffalse
%<*package>
%\fi
%
% This section describes the definitions file |childdoc.def|.

% The definitions cannot be loaded using |\usepackage| or |\RequirePackage|
% which has a mechanism to prevent loading a style file more than once.
% When loading the definitions by means of |\input|
% multiple instances have to be prevented manually:
%\iffalse
%This code needs to be before the `\ProvidesFile' directive
%which is defined at the beginning of this file.
%Therefore it is also placed there and commented out here.
%</package>
%<*discard>
%\fi
%    \begin{macrocode}
\ifdefined\childdocmain\endinput\fi
%    \end{macrocode}
%\iffalse
%</discard>
%<*package>
%\fi
%
% \macro{\ifchilddoc}
% \macro{\ifchilddocmanual}
% The conditional |\ifchilddoc| tells whether a
% child (true) or main (false) document is being compiled.
% The conditional |\ifchilddocmanual| tells whether
% the |\includeonly| mechanism is used (false) or
% the selection of child files must be performed manually (true).
% The definitions initialise to false:
%    \begin{macrocode}
\newif\ifchilddoc
\newif\ifchilddocmanual
%    \end{macrocode}

% \macro{\childdocname}
% \macro{\childdocjob}
% The macro |\childdocname| stores the name of the main document
% to be compiled. The macro |\childdocjob| stores the name of
% the document on which the \LaTeX{} compiler was originally invoked.
% The content of |\jobname| cannot be compared
% to filenames specified in the source due to different catcodes.
% The following code rescans |\jobname|, stores the result
% in |\childdocname| and saves a copy in |\childdocjob|:
%    \begin{macrocode}
\edef\childdocname{\scantokens\expandafter{\jobname\noexpand}}
\let\childdocjob\childdocname
%    \end{macrocode}

% \macro{\childdocdisable}
% The macro |\childdocdisable| prevents the main file
% from being processed more than once.
% At this stage, the main document command |\childdocmain|
% is assumed to be called once again where it should do nothing.
% Any subsequent call to it should prevent
% a secondary processing of the main document
% It overwrites the forwarding commands
% |\childdocof| and |\childdocforward|
% with empty macros to prevent further inclusions of the main document:
%    \begin{macrocode}
\newcommand{\childdocdisable}
{
  \renewcommand{\childdocmain}[1]{\renewcommand{\childdocmain}[1]{\endinput}}
  \renewcommand{\childdocof}[1]{}
  \renewcommand{\childdocby}[2][]{}
  \renewcommand{\childdocforward}[2][]{}
  \renewcommand{\childdocdisable}{}
}
%    \end{macrocode}

% \macro{\childdocmain}
% The macro |\childdocmain| is to be called at the top of the main file
% with nothing or the main filename (without extension) as argument.
% First, it breaks loops.
% If the argument is not empty and does not match |\childdocname|
% (which is set by the first inclusion of |childdoc.def|),
% |\ifchilddoc| is set to true, |\includeonly| is applied to the child file
% and |\jobname| is set to the main file
% (for proper handling of |.aux| files):
%    \begin{macrocode}
\newcommand{\childdocmain}[1]
{
  \childdocdisable\childdocmain{}
  \if?#1?\else
    \begingroup
      \def\childdoctmp{#1}
      \ifx\childdoctmp\childdocname
        \def\childdoctmp{}
      \else
        \def\childdoctmp
        {
          \childdoctrue
          \includeonly{\childdocname}
          \def\childdocjob{#1}
          \def\jobname{#1}
        }
      \fi
      \expandafter
    \endgroup
    \childdoctmp
  \fi
}
%    \end{macrocode}

% \macro{\childdocof}
% The command |\childdocof| redirects
% compilation to the main file |#1|.
%    \begin{macrocode}
\newcommand{\childdocof}[1]
{
  \childdocdisable
  \childdoctrue
  \includeonly{\childdocname}
  \def\jobname{#1}
  \def\childdocjob{#1}
  \input{#1}
}
%    \end{macrocode}

% \macro{\childdocby}
% The command |\childdocby| ....
%    \begin{macrocode}
\newcommand{\childdocby}[2][]
{
  \childdocdisable
  \childdoctrue
  \childdocmanualtrue
  \if?#1?\else
    \def\jobname{#2}
  \fi
  \def\childdocjob{#2}
  \input{#2}
  \endinput
}
%    \end{macrocode}

% \macro{\childdocforward}
% The command |\childdocforward| redirects
% compilation to the main file or
% (if the optional argument is given) a child file.
% Parameters are set as if the main file
% or a child file starting with |\childdocof| was compiled.
% Then compilation is handed over to the main file:
%    \begin{macrocode}
\newcommand{\childdocforward}[2][]
{
  \begingroup
    \if?#1?
      \def\childdoctmp
      {
        \def\childdocname{#2}
        \def\childdocjob{#2}
        \def\jobname{#2}
        \input{#2}
        \endinput
      }
    \else
      \def\childdoctmp
      {
        \childdocdisable
        \def\childdocname{#2}
        \childdoctrue
        \includeonly{#2}
        \def\childdocjob{#1}
        \def\jobname{#1}
        \input{#1}
        \endinput
      }
    \fi
    \expandafter
  \endgroup
  \childdoctmp
}
%    \end{macrocode}

% \macro{\childdocforwardprefix}
% The command |\childdocforwardprefix| redirects
% compilation to the main or a child file by means of a pattern.
% The prefix |#1| in the current filename is replaced by |#2|
% and the suffix of the current filename is kept
% (it is assumed that the filename does not contain the substring `|~~~|'
% which is used as a delimiter).
% Compilation is handed over to the new file by |\childdocforward|:
%    \begin{macrocode}
\newcommand{\childdocforwardprefix}[3][]
{
  \begingroup
    \def\childdocextract #2##1~~~{\def\childdoctmp{\childdocforward[#1]{#3##1}}}
    \expandafter\childdocextract\childdocname~~~
    \expandafter
  \endgroup
  \childdoctmp
}
%    \end{macrocode}

% \macro{\childdoc}
% The deprecated macro |\childdoc| is a legacy version of |\childdocmain|:
%    \begin{macrocode}
\newcommand{\childdoc}{\childdocmain}
%    \end{macrocode}

% \macro{\childdocredirect}
% The deprecated macro |\childdocredirect| is a legacy version
% of |\childdocforward| and |\childdocforwardprefix|:
%    \begin{macrocode}
\newcommand{\childdocredirect}[2][]
{
  \begingroup
    \if?#1?
      \def\childdoctmp{\childdocforward{#2}}
    \else
      \def\childdoctmp{\childdocforwardprefix{#1}{#2}}
    \fi
    \expandafter
  \endgroup
  \childdoctmp
}
%    \end{macrocode}

%\iffalse
%</package>
%\fi
%
\endinput
\childdocforward[cdocsamp]{cdocsch1}"|\\
% |latex -jobname cdocscl2 \|\\
% |  "\def\version{final}% \iffalse
%
% childdoc.dtx Copyright (C) 2017-2018 Niklas Beisert
%
% This work may be distributed and/or modified under the
% conditions of the LaTeX Project Public License, either version 1.3
% of this license or (at your option) any later version.
% The latest version of this license is in
%   http://www.latex-project.org/lppl.txt
% and version 1.3 or later is part of all distributions of LaTeX
% version 2005/12/01 or later.
%
% This work has the LPPL maintenance status `maintained'.
%
% The Current Maintainer of this work is Niklas Beisert.
%
% This work consists of the files childdoc.dtx and childdoc.ins
% and the derived files childdoc.def and cdocsamp.tex with
% cdocsch1.tex, cdocsch2.tex, cdocsdrf.tex, cdocsfn1.tex, cdocsfn2.tex.
%
%<package>\ifdefined\childdocmain\endinput\fi
%<package>\ProvidesFile{childdoc.def}[2018/12/30 v2.0 child document driver]
%<samplemain>\ProvidesFile{cdocsamp.tex}[2018/12/30 v2.0 sample for childdoc]
%<*driver>
%\ProvidesFile{childdoc.drv}[2018/12/30 v2.0 childdoc reference manual file]
\PassOptionsToClass{10pt,a4paper}{article}
\documentclass{ltxdoc}

\usepackage[margin=35mm]{geometry}
\usepackage{hyperref}
\usepackage{hyperxmp}
\usepackage[usenames]{color}

\hypersetup{colorlinks=true}
\hypersetup{pdfstartview=FitH}
\hypersetup{pdfpagemode=UseNone}
\hypersetup{pdfsource={}}
\hypersetup{pdflang={en-UK}}
\hypersetup{pdfcopyright={Copyright 2017-2018 Niklas Beisert.
  This work may be distributed and/or modified under the
  conditions of the LaTeX Project Public License, either version 1.3
  of this license or (at your option) any later version.}}
\hypersetup{pdflicenseurl={http://www.latex-project.org/lppl.txt}}
\hypersetup{pdfcontactaddress={ETH Zurich, ITP, HIT K,
  Wolfgang-Pauli-Strasse 27}}
\hypersetup{pdfcontactpostcode={8093}}
\hypersetup{pdfcontactcity={Zurich}}
\hypersetup{pdfcontactcountry={Switzerland}}
\hypersetup{pdfcontactemail={nbeisert@itp.phys.ethz.ch}}
\hypersetup{pdfcontacturl={http://people.phys.ethz.ch/\xmptilde nbeisert/}}

\newcommand{\secref}[1]{\hyperref[#1]{section \ref*{#1}}}

\parskip1ex
\parindent0pt
\let\olditemize\itemize
\def\itemize{\olditemize\parskip0pt}

\begin{document}

\title{The \textsf{childdoc} Package}
\hypersetup{pdftitle={The childdoc Package}}
\author{Niklas Beisert\\[2ex]
  Institut f\"ur Theoretische Physik\\
  Eidgen\"ossische Technische Hochschule Z\"urich\\
  Wolfgang-Pauli-Strasse 27, 8093 Z\"urich, Switzerland\\[1ex]
  \href{mailto:nbeisert@itp.phys.ethz.ch}
  {\texttt{nbeisert@itp.phys.ethz.ch}}}
\hypersetup{pdfauthor={Niklas Beisert}}
\hypersetup{pdfsubject={Manual for the LaTeX2e Package childdoc}}
\date{30 December 2018, \textsf{v2.0}}
\maketitle

\begin{abstract}\noindent
\textsf{childdoc} is a \LaTeXe{} package
that enables the direct compilation
of document sections included by |\include|
to individual files.
\end{abstract}

\begingroup
\parskip0ex
\tableofcontents
\endgroup

%%%%%%%%%%%%%%%%%%%%%%%%%%%%%%%%%%%%%%%%%%%%%%%%%%%%%%%%%%%%%%%%%%%%%%%%%%%%%%%%
%%%%%%%%%%%%%%%%%%%%%%%%%%%%%%%%%%%%%%%%%%%%%%%%%%%%%%%%%%%%%%%%%%%%%%%%%%%%%%%%
\section{Introduction}

\LaTeX{} provides a mechanism to structure a large document (such as a book)
into a main file and several child files (containing the chapters)
using the |\include| command.
This mechanism is beneficial for documents
which span hundreds of pages in order to
make the source file(s) more manageable.
Moreover, compilation can be restricted to
selected child files by means of the |\includeonly| command.
The latter feature can be used to reduce the compilation time while editing
(this was significantly more useful in the earlier days of \LaTeX{})
or to generate a smaller document which is easier to navigate.
Another application of |\includeonly| is to generate
documents consisting of selected parts of the complete document.

However, there are a few drawbacks of the plain |\include| mechanism:
\begin{itemize}
\item
The child files cannot be compiled on their own,
they can only be compiled via the main file.
A naive editing environment
(such as a text editor with an option
to have the current file processed by \LaTeX)
may require one to switch to the main file before compiling;
attempting to compile the child file produces errors.
\item
The main file must be modified (each time)
to adjust the |\includeonly| command
to the present needs. This easily leaves the main file in a messy state.
\item
The generated document will always carry the filename
of the main document. This is inconvenient if
several child files are to be compiled and
to be kept for distribution.
\end{itemize}

The present package provides a simple interface
to make child files individually compilable by \LaTeX{}.
Compiling a child file then has the same effect as compiling
the main file with an |\includeonly| command
to select the appropriate child.
Moreover the generated document will carry the name of the child
rather than the main file.
This resolves all three above issues.

This feature is meant to make the editing of books,
thesis documents and lecture notes somewhat more convenient.
However, the package can also be used efficiently for
composing a series of documents (such as exercise sheets)
which are typically distributed individually.
It then assists the author in generating the individual documents
(potentially in different versions)
as well as a document containing the collected series.
Another application is in developing style files
or other kinds of included material
where compilation of the style file could redirect
to a sample or test file.

%%%%%%%%%%%%%%%%%%%%%%%%%%%%%%%%%%%%%%%%%%%%%%%%%%%%%%%%%%%%%%%%%%%%%%%%%%%%%%%%
%%%%%%%%%%%%%%%%%%%%%%%%%%%%%%%%%%%%%%%%%%%%%%%%%%%%%%%%%%%%%%%%%%%%%%%%%%%%%%%%
\section{Usage}

First of all, the package \textsf{childdoc} is \emph{not} a standard
\LaTeXe{} |.sty| style file! Therefore it needs to be invoked in
a non-standard way.

%%%%%%%%%%%%%%%%%%%%%%%%%%%%%%%%%%%%%%%%%%%%%%%%%%%%%%%%%%%%%%%%%%%%%%%%%%%%%%%%
\subsection{Included Files}
\label{sec:include}

%%%%%%%%%%%%%%%%%%%%%%%%%%%%%%%%%%%%%%%%
\DescribeMacro{\childdocmain}
To use the package, add the commands
\begin{center}
\begin{tabular}{l}
|\input{childdoc.def}|\\
|\childdocmain{}|\\
\end{tabular}
\end{center}
at the very top of the main \LaTeX{} file,
in particular \emph{before} the |\documentclass| statement!
The argument of |\childdocmain| should be left empty
(but it must be present).

%%%%%%%%%%%%%%%%%%%%%%%%%%%%%%%%%%%%%%%%
\DescribeMacro{\childdocof}
Furthermore, add the commands
\begin{center}
\begin{tabular}{l}
|\input{childdoc.def}|\\
|\childdocof{|\textit{main}|}|\\
\end{tabular}
\end{center}
at the top of every child file \textit{child}
which is included by |\include{|\textit{child}|}|
from within the main file
(or at least for those files to be compiled individually).
The argument \textit{main} must be the filename of the main file.

There are a couple of
considerations in setting up the main and child documents:

%%%%%%%%%%%%%%%%%%%%%%%%%%%%%%%%%%%%%%%%
\paragraph{Restrictions.}

Please note the following restrictions:
\begin{itemize}
\item
|\childdocmain| must be called with one argument \textit{main}
to ensure compatibility with earlier version of the package.
It must either be empty (|\childdocmain{}|)
or precisely match the filename of the main file in which it is specified.
See \secref{sec:detection} for further information.
\item
The filename \textit{main} must be specified without the |.tex| extension.
\item
The filename \textit{main} is case sensitive
(even in case-insensitive file systems)
due to internal string comparison.
\item
The argument \textit{main} should be fully expanded, it cannot be a macro.
\item
Subdirectories and special characters should be avoided in filenames.
\item
The command |\childdocmain{|\textit{main}|}| must be followed by a whitespace.
It should not be followed immediately by another command
or by a comment mark `|%|'.
This is because the \TeX{} parser reads the token immediately following
the argument of |\childdocmain| and puts it
at the beginning of every child section;
however, a white\-space is ignored.
\end{itemize}

%%%%%%%%%%%%%%%%%%%%%%%%%%%%%%%%%%%%%%%%
\paragraph{Content of Main File.}

It is advisable to place all content in the child files included by |\include|.
Any output contained in the main file will appear in all child documents
unless suppressed manually;
it cannot be suppressed automatically by the |\includeonly| directive
and thus should normally be avoided.
A method to include some content in the main file
by means of conditional processing is described in \secref{sec:conditional}.

%%%%%%%%%%%%%%%%%%%%%%%%%%%%%%%%%%%%%%%%
\paragraph{Page Numbering.}

When only a part of the document is compiled,
the appropriate numbering of pages
(as well as other status parameters)
is determined from the |.aux| files.
The latter contain information from previous passes.
However this information needs to propagate through
all intermediate child documents.
Therefore the page numbering in child documents may well
be inconsistent until the complete document is compiled at least once.

A useful (if unconventional) way to always ensure a consistent
page numbering is to restart the numbering in each child document
and denote the pages by `\textit{child}|.|\textit{page}'
where \textit{child} represents the chapter/section number of the child file.
This can be achieved by the command
|\numberwithin{page}{|\textit{child}|}|
of the \textsf{amsmath} package
where \textit{child} can be |chapter| or |section|
depending on the chosen structuring.
Alternatively, one can modify the macro |\thepage| appropriately
and reset the counter |page| at the start of each child file.

%%%%%%%%%%%%%%%%%%%%%%%%%%%%%%%%%%%%%%%%%%%%%%%%%%%%%%%%%%%%%%%%%%%%%%%%%%%%%%%%
\subsection{Conditional Processing}
\label{sec:conditional}

The package provides a mechanism to compile different versions
of a document. To customise the versions further some conditional processing
can come in handy to distinguish which version is being compiled.
The package provides two macros to describe the compilation context:

%%%%%%%%%%%%%%%%%%%%%%%%%%%%%%%%%%%%%%%%
\DescribeMacro{\ifchilddoc}
The conditional |\ifchilddoc| distinguishes between the compilation of
child documents and the main document:
%
\begin{center}
|\ifchilddoc |\textit{child-code}| |[|\||else |\textit{main-code}]| \||fi|
\end{center}

%%%%%%%%%%%%%%%%%%%%%%%%%%%%%%%%%%%%%%%%
\DescribeMacro{\childdocname}
\DescribeMacro{\childdocjob}
The macro |\childdocname| contains the filename (without extension)
of the main or child file being processed.
Note that |\childdocjob| will always contain the name of the main file.

%%%%%%%%%%%%%%%%%%%%%%%%%%%%%%%%%%%%%%%%
\paragraph{Title Page.}

Conditional processing can be used to include a title or banner page
in the main document when proper precautions are taken.
Importantly, the code in the main file should ensure that the page counter
(as well as other status parameters which are stored in the |.aux| files)
takes the same value after the conditional processing.
Otherwise the page numbers may take divergent values
depending on which part is compiled.

For example, a title page could be declared by:
%
\begin{center}
\begin{tabular}{l}
|\ifchilddoc\||else|\\
|\addtocounter{page}{-1}|\\
\textit{code for title page}\\
|\newpage|\\
|\||fi|
\end{tabular}
\end{center}
%
A banner page for the child documents can be generated by:
%
\begin{center}
\begin{tabular}{l}
|\ifchilddoc|\\
|\addtocounter{page}{-1}|\\
\textit{code for banner page}\\
|\newpage|\\
|\||fi|
\end{tabular}
\end{center}
%
Here one could write a message such as:
\begin{center}
|This is the part \childdocname{} of \childdocjob{}.|
\end{center}

%%%%%%%%%%%%%%%%%%%%%%%%%%%%%%%%%%%%%%%%%%%%%%%%%%%%%%%%%%%%%%%%%%%%%%%%%%%%%%%%
\subsection{Flags}
\label{sec:flags}

The package makes it easy to generate different versions
of the main or child documents.
To this end compilation flags can be defined
and assigned different default values.
They will be particularly useful in conjunction
with the forwarding mechanism described in \secref{sec:forward}.

For example, it may be useful to have a flag |\version|
which can be set to |draft| or |final|.
The document source will contain some conditional code
depending on the value of |\version|.
Suppose further, the flag should default to |final| for the main file
and to |draft| for child files
which is a natural assignment for editing the document.
This is achieved by placing the following code
in the preamble of the main document
(below the |\childdocmain| directive):
%
\begin{center}
\begin{tabular}{l}
|\ifchilddoc|\\
|\providecommand{\version}{draft}|\\
|\||else|\\
|\providecommand{\version}{final}|\\
|\||fi|
\end{tabular}
\end{center}
%
The definition by |\providecommand| makes sure
that previous definitions are not overwritten.
Further statements |\providecommand{\version}{...}|
can thus be added before the above code to override it.

For the main file, one might add a line
(between |\childdocmain| and the above block)
%
\begin{center}
|%\ifchilddoc\||else\providecommand{\version}{draft}\||fi|
\end{center}
%
which can be uncommented to produce a draft version.
Likewise one can add a line to the very top of a child file
(above the |\childdocof{|\textit{main}|}| directive)
%
\begin{center}
|%\providecommand{\version}{final}|
\end{center}
%
which can be uncommented to produce the final version of this child document.

%%%%%%%%%%%%%%%%%%%%%%%%%%%%%%%%%%%%%%%%%%%%%%%%%%%%%%%%%%%%%%%%%%%%%%%%%%%%%%%%
\subsection{Forwarding}
\label{sec:forward}

Different versions of the main or child documents
using compilation flags as described in \secref{sec:flags}
can be (permanently) stored in different files
for convenient compilation, viewing and distribution.
To this end, the package defines a command
to pass on compilation to a different file:

%%%%%%%%%%%%%%%%%%%%%%%%%%%%%%%%%%%%%%%%
\DescribeMacro{\childdocforward}
The command |\childdocforward| redirects processing to
another source file:
%
\begin{center}
\begin{tabular}{l}
|\input{childdoc.def}|\\
|\childdocforward[|\textit{main}|]{|\textit{dest}|}|\\
\end{tabular}
\end{center}
%
The argument \textit{dest} is the destination file
(without extension).
It should be the main file or one of the child files.
Note that further \textsf{childdoc} directives
such as |\childdocof| and |\childdocforward|
in the indicated file will be processed in this form.
The optional argument \textit{main}
passes on directly to the main file \textit{main}
while pretending to compile the child \textit{dest}.
This form behaves as if \textit{dest}
issues |\childdocof{|\textit{main}|}| right away,
and no further \textsf{childdoc} directives will be processed.

%%%%%%%%%%%%%%%%%%%%%%%%%%%%%%%%%%%%%%%%
\DescribeMacro{\...prefix}
In the alternative form |\childdocforwardprefix|,
%
\begin{center}
\begin{tabular}{l}
|\input{childdoc.def}|\\
|\childdocforwardprefix[|\textit{main}|]{|\textit{prefix}|}{|\textit{dest}|}|
\end{tabular}
\end{center}
%
the destination file is determined by a pattern
depending on the current file:
To make this work, the current file must be called
`{\textit{prefix}\hspace{0.2em}\textit{suffix}}'
with \textit{prefix} matching precisely the argument.
Processing is then passed on to the file
`{\textit{dest}\hspace{0.2em}\textit{suffix}}'.
Surely, the same effect is achieved by
directly specifying the
argument `{\textit{dest}\hspace{0.2em}\textit{suffix}}'
in the first form.
However, that requires to set up a different file
for each child. With the alternative form of the command
all these files can have exactly the same content
which simplifies setting them up and maintaining them.

For example, the following file |draft.tex|
with a compilation flag |\version| as described in \secref{sec:flags}
compiles the main document as a draft:
%
\begin{center}
\begin{tabular}{l}
|\def\version{draft}|\\
|\input{childdoc.def}|\\
|\childdocforward{|\textit{main}|}|
\end{tabular}
\end{center}
%
Likewise, the following files |final|\textit{nn}|.tex|
compile the final version of the child document
|child|\textit{nn}|.tex|:
%
\begin{center}
\begin{tabular}{l}
|\def\version{final}|\\
|\input{childdoc.def}|\\
|\childdocforwardprefix{final}{child}|
\end{tabular}
\end{center}
%

Note that when several versions of a main file and/or of each child file
are to be generated, it may be convenient to set up a |Makefile| or
shell script to automatise the process.

%%%%%%%%%%%%%%%%%%%%%%%%%%%%%%%%%%%%%%%%%%%%%%%%%%%%%%%%%%%%%%%%%%%%%%%%%%%%%%%%
\subsection{Command Line Processing}
\label{sec:commandline}

The effect of redirection files can also be achieved by invoking
the \LaTeX{} compiler with a more elaborate command line.
Most conveniently this should be done as part
of a shell script or a |Makefile|.

When using \textsf{childdoc} in the main file, the following
command lines effectively perform a redirection
(note that depending on the shell being used,
backslashes may have to be doubled: `|\|' $\to$ `|\\|'):
%
\begin{center}
|... -jobname "|\textit{target}|" |\\|"|[\textit{flags}]%
|\input{childdoc.def}\childdocforward[|\textit{main}|]{|\textit{dest}|}"|
\end{center}
%
Here \textit{target} is the name of the output file,
\textit{main} is the name of the main file
and \textit{dest} is the name of the main or child file to be processed
(all filenames without extensions).
The optional argument \textit{main} can be omitted
if \textit{main} matches \textit{dest}.
Optionally, compilation \textit{flags} can be defined via |\def| commands.
This command line makes the \TeX{} engine believe
it is compiling the file \textit{target}
whose content is specified as the latter parameter.
The provided code then forwards the processing to
\textit{main} or \textit{dest} as described in \secref{sec:forward}.

%%%%%%%%%%%%%%%%%%%%%%%%%%%%%%%%%%%%%%%%%%%%%%%%%%%%%%%%%%%%%%%%%%%%%%%%%%%%%%%%
\subsection{Include by Input}
\label{sec:input}

Including child documents by |\include| has some restrictions by design.
Most notably, the content of a child document always occupies
its own set of pages; pages cannot be shared between child documents.
Usually, this behaviour makes perfect sense
because each child document contain an essential part of the document.
However, in some situations it may be desirable to compose
a document from a collection of parts
without having mandatory page breaks between then.
For this case, the package
provides a mechanism to include parts
by |\input| which can also be processed individually.
However, by construction this mechanism
requires manual handling of the content to be output.

%%%%%%%%%%%%%%%%%%%%%%%%%%%%%%%%%%%%%%%%
\DescribeMacro{\ifchilddocmanual}
The main file should be prepared as usual, see \secref{sec:include}.
However, the document body must make a distinction
between processing of an individual part and of the main document, e.g.:
%
\begin{center}
\begin{tabular}{l}
|\ifchilddocmanual|\\
|\input{\childdocname}|\\
|\||else|\\
\textit{document body with }|\input{|\textit{part}|}|\\
|\||fi|
\end{tabular}
\end{center}
%
The conditional |\ifchilddocmanual| is true whenever
a part to be included by |\input| is being compiled,
and the name of the part is stored in |\childdocname|.

%%%%%%%%%%%%%%%%%%%%%%%%%%%%%%%%%%%%%%%%
\DescribeMacro{\childdocby}
Each part to be included by |\input| should start with:
%
\begin{center}
\begin{tabular}{l}
|\input{childdoc.def}|\\
|\childdocby{|\textit{main}|}|\\
\end{tabular}
\end{center}
%
The directive |\childdocby| is similar to |\childdocof|
described in \secref{sec:include},
but the subsequent selection of content must be done manually.
To that end, both |\ifchilddoc| and |\ifchilddocmanual|
will be true upon processing of a part,
and the name of the part is stored in |\childdocname|.
Note that |\jobname| will be set to the filename of the current part
so that each part receives an individual |.aux| file
that does not interfere with the |.aux| file(s) of the main document.
This behaviour can be altered by the alternative form
|\childdocby[*]{|\textit{main}|}| (with a non-empty optional argument)
which uses the |.aux| file of the main document
by setting |\jobname| to \textit{main}.

%%%%%%%%%%%%%%%%%%%%%%%%%%%%%%%%%%%%%%%%%%%%%%%%%%%%%%%%%%%%%%%%%%%%%%%%%%%%%%%%
\subsection{Driver Development}
\label{sec:driver}

The \textsf{childdoc} mechanism can also be use for the development
of definition files such as \LaTeX{} styles or classes.
This case differs from the above setup with multiple parts
included by |\include| in that no |\includeonly| should be invoked.
This can be achieved by starting the include file
(before |\ProvidesPackage|) with:
%
\begin{center}
\begin{tabular}{l}
|\input{childdoc.def}|\\
|\childdocforward{|\textit{main}|}|\\
\end{tabular}
\end{center}
%
or alternatively with:
%
\begin{center}
\begin{tabular}{l}
|\input{childdoc.def}|\\
|\childdocby{|\textit{main}|}|\\
\end{tabular}
\end{center}
%
Both forms have slightly different effects as described above.
The main file is prepared as usual, see \secref{sec:include}.

%%%%%%%%%%%%%%%%%%%%%%%%%%%%%%%%%%%%%%%%%%%%%%%%%%%%%%%%%%%%%%%%%%%%%%%%%%%%%%%%
\subsection{Legacy Detection}
\label{sec:detection}

The directive |\childdocmain| in the main file can detect
whether the complete document or merely a child is to be compiled
even without using the directive |\childdocof|.
This method is deprecated because it is less robust
and there is no compelling reason to use it;
it is merely provided for backward compatibility
and it may be removed in future versions.

If the detection mechanism is to be used,
it is mandatory to correctly specify
the filename of the main file as the argument of |\childdocmain|:
%
\begin{center}
\begin{tabular}{l}
|\input{childdoc.def}|\\
|\childdocmain{|\textit{main}|}|\\
\end{tabular}
\end{center}
%
If |\jobname| does not match the argument \textit{main} of |\childdocmain|,
it is assumed that |\jobname| points to the child file to be compiled.
When using |\childdocmain| with the main file specified as argument,
it suffices to start a child file
with just |\input{|\textit{main}|}|
without loading of the package and using |\childdocof|.
If instead all processing is done
with the appropriate \textsf{childdoc} directives,
the argument of \textit{main} of |\childdocmain| can be empty.

An alternative version of the command line processing described
in \secref{sec:commandline} using the detection mechanism reads:
%
\begin{center}
|... -jobname "|\textit{target}|" "|[\textit{flags}]%
[|\def\jobname{|\textit{dest}|}|]|\input{|\textit{main}|}"|
\end{center}

%%%%%%%%%%%%%%%%%%%%%%%%%%%%%%%%%%%%%%%%%%%%%%%%%%%%%%%%%%%%%%%%%%%%%%%%%%%%%%%%
\subsection{Manual Code}
\label{sec:manual}

In case one cannot be certain whether the definitions file |childdoc.def|
is installed on the target \TeX{} distribution
and one prefers not to ship it,
it is conceivable to paste a few relevant commands into the sources.

To that end, drop all statements |\input{childdoc.def}|
and perform the replacements as outlined below.
Instead of |\childdocmain{|\textit{main}|}| add the following code
to the top of the main file:
%
\begin{center}
\begin{tabular}{l}
|\||ifdefined\childdocname\endinput\||fi\newif\ifchilddoc|\\
|\edef\childdocname{\scantokens\expandafter{\jobname\noexpand}}|\\
|\def\childdocmain{|\textit{main}|}\||ifx\childdocmain\childdocname\||else|\\
|\childdoctrue\includeonly{\childdocname}\let\jobname\childdocmain\||fi|\\
\end{tabular}
\end{center}
%
Instead of |\childdocof{|\textit{main}|}| just include the main file
at the top of each child file:
%
\begin{center}
|\input{|\textit{main}|}|
\end{center}
%
A simple redirection |\childdocforward{|\textit{dest}|}| is achieved by:
%
\begin{center}
|\def\jobname{|\textit{dest}|}\input{\jobname}|
\end{center}
%
The redirection with prefix
|\childdocforwardprefix[|\textit{prefix}|]{|\textit{dest}|}|
is accomplished by:
%
\begin{center}
\begin{tabular}{l}
|{\edef\jobname{\scantokens\expandafter{\jobname\noexpand}}|\\
|\def\redirectjob |\textit{prefix}|#1~~~{\gdef\jobname{|\textit{dest}|#1}}|\\
|\expandafter\redirectjob\jobname~~~}\input{\jobname}|
\end{tabular}
\end{center}

In an alternative approach,
child documents can be compiled by a specific command line
without additional code or specific definitions:
%
\begin{center}
|... -jobname "|\textit{target}|" "|[\textit{flags}]%
|\includeonly{|\textit{dest}|}\input{|\textit{main}|}"|
\end{center}
%

%%%%%%%%%%%%%%%%%%%%%%%%%%%%%%%%%%%%%%%%%%%%%%%%%%%%%%%%%%%%%%%%%%%%%%%%%%%%%%%%
%%%%%%%%%%%%%%%%%%%%%%%%%%%%%%%%%%%%%%%%%%%%%%%%%%%%%%%%%%%%%%%%%%%%%%%%%%%%%%%%
\section{Information}

%%%%%%%%%%%%%%%%%%%%%%%%%%%%%%%%%%%%%%%%%%%%%%%%%%%%%%%%%%%%%%%%%%%%%%%%%%%%%%%%
\subsection{Copyright}

Copyright \copyright{} 2017--2018 Niklas Beisert

This work may be distributed and/or modified under the
conditions of the \LaTeX{} Project Public License, either version 1.3
of this license or (at your option) any later version.
The latest version of this license is in
  \url{http://www.latex-project.org/lppl.txt}
and version 1.3 or later is part of all distributions of \LaTeX{}
version 2005/12/01 or later.

This work has the LPPL maintenance status `maintained'.

The Current Maintainer of this work is Niklas Beisert.

This work consists of the files |README.txt|, |childdoc.ins| and |childdoc.dtx|
as well as the derived files |childdoc.def|, |cdocsamp.tex|
with |cdocsch1.tex|, |cdocsch2.tex|, |cdocspt3.tex|, |cdocspt4.tex|,
|cdocsdrf.tex|, |cdocsfn1.tex|, |cdocsfn2.tex|
as well as |childdoc.pdf|.

%%%%%%%%%%%%%%%%%%%%%%%%%%%%%%%%%%%%%%%%%%%%%%%%%%%%%%%%%%%%%%%%%%%%%%%%%%%%%%%%
\subsection{Files and Installation}

The package consists of the files:
%
\begin{center}
\begin{tabular}{ll}
    |README.txt|   & readme file \\
    |childdoc.ins| & installation file \\
    |childdoc.dtx| & source file \\
    |childdoc.def| & definition file \\
    |cdocsamp.tex| & sample main file \\
    |cdocsch1.tex| & sample include file \\
    |cdocsch2.tex| & sample include file \\
    |cdocspt3.tex| & sample part file \\
    |cdocspt4.tex| & sample part file \\
    |cdocsdrf.tex| & sample redirection file \\
    |cdocsfn1.tex| & sample redirection file \\
    |cdocsfn2.tex| & sample redirection file \\
    |childdoc.pdf| & manual
\end{tabular}
\end{center}
%
The distribution consists of the files
|README.txt|, |childdoc.ins| and |childdoc.dtx|.
%
\begin{itemize}
\item
Run (pdf)\LaTeX{} on |childdoc.dtx|
to compile the manual |childdoc.pdf| (this file).
\item
Run \LaTeX{} on |childdoc.ins| to create the definitions file |childdoc.def|
and the sample |cdocsamp.tex| with include files
|cdocsch1.tex|, |cdocsch2.tex|, |cdocspt3.tex|, |cdocspt4.tex|,
|cdocsdrf.tex|, |cdocsfn1.tex|, |cdocsfn2.tex|.
Then copy the file |childdoc.def| to an appropriate directory of your \LaTeX{}
distribution, e.g.\ \textit{texmf-root}|/tex/latex/childdoc|.
\end{itemize}

%%%%%%%%%%%%%%%%%%%%%%%%%%%%%%%%%%%%%%%%%%%%%%%%%%%%%%%%%%%%%%%%%%%%%%%%%%%%%%%%
\subsection{Related CTAN Packages}

There are several other packages which offer a similar functionality:
%
\begin{itemize}
\item
The packages
\href{http://ctan.org/pkg/docmute}{\textsf{docmute}},
\href{http://ctan.org/pkg/includex}{\textsf{includex}} and
\href{http://ctan.org/pkg/standalone}{\textsf{standalone}}
provide commands to include only the document body of
a child file thus allowing both files to be compiled individually.
\item
The packages \href{http://ctan.org/pkg/subdocs}{\textsf{subdocs}}
and \href{http://ctan.org/pkg/subfiles}{\textsf{subfiles}}
provide structures in which the main and child documents can be
encapsulated and allowing them to be compiled individually.
The inclusion mechanism is different from the conventional |\include|.
\item
The package \href{http://ctan.org/pkg/combine}{\textsf{combine}}
is an elaborate solution to combine several documents into one.
\end{itemize}
%
See also the CTAN topic \href{http://ctan.org/topic/subdocs}{\textsf{subdocs}}
for further related packages.
The present package differs from the above solutions in that
a document structure constructed with the conventional |\include| mechanism
just needs two extra commands at the top of every file
such that all constituent files can be compiled individually.

%%%%%%%%%%%%%%%%%%%%%%%%%%%%%%%%%%%%%%%%%%%%%%%%%%%%%%%%%%%%%%%%%%%%%%%%%%%%%%%%
%\subsection{Feature Suggestions}
%
%The following is a list of features which may be useful for future
%versions of this package:
%%
%\begin{itemize}
%\item
%\ldots
%\end{itemize}

%%%%%%%%%%%%%%%%%%%%%%%%%%%%%%%%%%%%%%%%%%%%%%%%%%%%%%%%%%%%%%%%%%%%%%%%%%%%%%%%
\subsection{Revision History}

%%%%%%%%%%%%%%%%%%%%%%%%%%%%%%%%%%%%%%%%
\paragraph{v2.0:} 2018/12/30

\begin{itemize}
\item
immediate forward processing
\item
added |\childdocby| mechanism
\item
manual restructured
\end{itemize}

%%%%%%%%%%%%%%%%%%%%%%%%%%%%%%%%%%%%%%%%
\paragraph{v1.6:} 2018/01/17

\begin{itemize}
\item
application for development of include files
\item
corrections to manual
\end{itemize}

%%%%%%%%%%%%%%%%%%%%%%%%%%%%%%%%%%%%%%%%
\paragraph{v1.5:} 2017/05/21

\begin{itemize}
\item
more complete structuring introduced
\item
|\childdocof| introduced
\item
|\childdoc| renamed to |\childdocmain|
\item
|\childredirect| renamed to |\childdocforward| and |\childdocforwardprefix|
and functionality expanded
\end{itemize}

%%%%%%%%%%%%%%%%%%%%%%%%%%%%%%%%%%%%%%%%
\paragraph{v1.0:} 2017/04/27

\begin{itemize}
\item
manual and install package
\item
first version published on CTAN
\end{itemize}

%%%%%%%%%%%%%%%%%%%%%%%%%%%%%%%%%%%%%%%%
\paragraph{v0.6:} 2017/04/26

\begin{itemize}
\item
redirection mechanism added
\end{itemize}

%%%%%%%%%%%%%%%%%%%%%%%%%%%%%%%%%%%%%%%%
\paragraph{v0.5:} 2017/04/26

\begin{itemize}
\item
functionality in definition file
\end{itemize}


%%%%%%%%%%%%%%%%%%%%%%%%%%%%%%%%%%%%%%%%%%%%%%%%%%%%%%%%%%%%%%%%%%%%%%%%%%%%%%%%
%%%%%%%%%%%%%%%%%%%%%%%%%%%%%%%%%%%%%%%%%%%%%%%%%%%%%%%%%%%%%%%%%%%%%%%%%%%%%%%%
%%%%%%%%%%%%%%%%%%%%%%%%%%%%%%%%%%%%%%%%%%%%%%%%%%%%%%%%%%%%%%%%%%%%%%%%%%%%%%%%
\appendix

\settowidth\MacroIndent{\rmfamily\scriptsize 000\ }

 \DocInput{childdoc.dtx}

\end{document}
%</driver>
% \fi
%
% %%%%%%%%%%%%%%%%%%%%%%%%%%%%%%%%%%%%%%%%%%%%%%%%%%%%%%%%%%%%%%%%%%%%%%%%%%%%%%
% %%%%%%%%%%%%%%%%%%%%%%%%%%%%%%%%%%%%%%%%%%%%%%%%%%%%%%%%%%%%%%%%%%%%%%%%%%%%%%
% \section{Sample}
%\iffalse
%<*samplemain>
%\fi
%
% The following presents a sample document
% with two chapters, two parts, a title page,
% a compile flag as well as three forwarding files to set the flag.
% It consists of eight |.tex| files:
% \begin{center}
% \begin{tabular}{ll}
% |cdocsamp.tex|&main file\\
% |cdocsch1.tex|&include file for chapter 1\\
% |cdocsch2.tex|&include file for chapter 2\\
% |cdocspt3.tex|&include file for part 3\\
% |cdocspt4.tex|&include file for part 4\\
% |cdocsdrf.tex|&forwarding file for main file in draft mode\\
% |cdocsfi1.tex|&forwarding file for final version of chapter 1\\
% |cdocsfi2.tex|&forwarding file for final version of chapter 2\\
% \end{tabular}
% \end{center}
% Each of the eight files can be compiled directly by the \LaTeX{} compiler.
%
% %%%%%%%%%%%%%%%%%%%%%%%%%%%%%%%%%%%%%%
% \paragraph{Main File.}
%
% The main file is called |cdocsamp.tex|.
%
% Load the \textsf{childdoc} definitions and
% declare the filename for the main document:
%    \begin{macrocode}
\input{childdoc.def}
\childdocmain{}
%    \end{macrocode}

% Optional override for |\version| flag:
%    \begin{macrocode}
%%\ifchilddoc\else\providecommand{\version}{draft}\fi
%    \end{macrocode}

% Define the default values for the |\version| flag
% (|final| for the main file and |draft| for childs):
%    \begin{macrocode}
\ifchilddoc
\providecommand{\version}{draft}
\else
\providecommand{\version}{final}
\fi
%    \end{macrocode}

% Load the standard document class:
%    \begin{macrocode}
\documentclass[12pt]{article}
%    \end{macrocode}

% Start the document body:
%    \begin{macrocode}
\begin{document}
%    \end{macrocode}

% Declare a title page.
% Print title, part of document being processed and version flag:
%    \begin{macrocode}
\addtocounter{page}{-1}
\begin{center}
{\LARGE\bfseries{}childdoc example\par}
\vspace{1cm}
\ifchilddoc
\ifchilddocmanual part\else chapter\fi:
`\childdocname' of `\childdocjob'\par
\else
main document: `\childdocjob'\par
\fi
version: \version\par
\end{center}
\newpage
%    \end{macrocode}

% Manually include selected file,
% otherwise process as usual:
%    \begin{macrocode}
\ifchilddocmanual
\section*{part `\childdocname'}
\input{\childdocname}
\else
%    \end{macrocode}

% Include the two chapters:
%    \begin{macrocode}
\include{cdocsch1}
\include{cdocsch2}
%    \end{macrocode}

% Include the two parts unless only chapters should be displayed:
%    \begin{macrocode}
\ifchilddoc\else
\section{part three}
\input{cdocspt3}
\section{part four}
\input{cdocspt4}
\fi
%    \end{macrocode}

% Process as usual until here:
%    \begin{macrocode}
\fi
%    \end{macrocode}

% End of document body:
%    \begin{macrocode}
\end{document}
%    \end{macrocode}
%\iffalse
%</samplemain>
%\fi
%
% %%%%%%%%%%%%%%%%%%%%%%%%%%%%%%%%%%%%%%
% \paragraph{Chapter Include Files.}
%
% The include files are called |cdocsch1.tex| and |cdocsch2.tex|.
%
%\iffalse
%<*samplechap1|samplechap2>
%\fi

% Optional override for |\version| flag:
%    \begin{macrocode}
%%\providecommand{\version}{final}
%    \end{macrocode}

% Include the main document:
%    \begin{macrocode}
\input{childdoc.def}
\childdocof{cdocsamp}
%    \end{macrocode}

%\iffalse
%</samplechap1|samplechap2>
%\fi
%
%\iffalse
%<*samplechap1>
%\fi
% Some text for chapter 1:
%    \begin{macrocode}
\section{one}
some text in chapter one
%    \end{macrocode}

%\iffalse
%</samplechap1>
%\fi
% Some text for chapter 2:
%\iffalse
%<*samplechap2>
%\fi
%    \begin{macrocode}
\section{two}
more text in chapter two
%    \end{macrocode}

%\iffalse
%</samplechap2>
%\fi
%
% %%%%%%%%%%%%%%%%%%%%%%%%%%%%%%%%%%%%%%
% \paragraph{Part Include Files.}
%
% The include files are called |cdocspt3.tex| and |cdocspt4.tex|.
%
%\iffalse
%<*samplepart3|samplepart4>
%\fi

% Optional override for |\version| flag:
%    \begin{macrocode}
%%\providecommand{\version}{final}
%    \end{macrocode}

% Include the main document:
%    \begin{macrocode}
\input{childdoc.def}
\childdocby{cdocsamp}
%    \end{macrocode}

%\iffalse
%</samplepart3|samplepart4>
%\fi
%
%\iffalse
%<*samplepart3>
%\fi
% Some text for part 3:
%    \begin{macrocode}
some text in part three
%    \end{macrocode}

%\iffalse
%</samplepart3>
%\fi
% Some text for part 4:
%\iffalse
%<*samplepart4>
%\fi
%    \begin{macrocode}
more text in part four
%    \end{macrocode}

%\iffalse
%</samplepart4>
%\fi
%
% %%%%%%%%%%%%%%%%%%%%%%%%%%%%%%%%%%%%%%
% \paragraph{Forwarding for a Complete Draft.}
%
% The following forwarding file |cdocsdrf.tex|
% compiles the main document in draft mode:
%\iffalse
%<*sampledraft>
%\fi
%    \begin{macrocode}
\def\version{draft}
\input{childdoc.def}
\childdocforward{cdocsamp}
%    \end{macrocode}

%\iffalse
%</sampledraft>
%\fi
%
% %%%%%%%%%%%%%%%%%%%%%%%%%%%%%%%%%%%%%%
% \paragraph{Forwarding for Final Version of the Chapters.}
%
% The following forwarding files |cdocsfn1.tex| and |cdocsfn2.tex|
% (with identical content)
% compile the final versions of the child documents
% |cdocsch1.tex| and |cdocsch2.tex|, respectively:
%\iffalse
%<*samplefinal>
%\fi
%    \begin{macrocode}
\def\version{final}
\input{childdoc.def}
\childdocforwardprefix[cdocsamp]{cdocsfn}{cdocsch}
%    \end{macrocode}

%\iffalse
%</samplefinal>
%\fi
%
% %%%%%%%%%%%%%%%%%%%%%%%%%%%%%%%%%%%%%%
% \paragraph{Command Line Processing.}
%
% The following three command lines generate the output files
% |cdocscld|, |cdocscl1| and |cdocscl2|
% which should be identical to
% |cdocsdrf|, |cdocsch1| and |cdocsfn2|, respectively:
% \begin{center}
% \begin{tabular}{l}
% |latex -jobname cdocscld \|\\
% |  "\def\version{draft}\input{childdoc.def}\childdocforward{cdocsamp}"|\\
% |latex -jobname cdocscl1 \|\\
% |  "\input{childdoc.def}\childdocforward[cdocsamp]{cdocsch1}"|\\
% |latex -jobname cdocscl2 \|\\
% |  "\def\version{final}\input{childdoc.def}\childdocforward{cdocsch2}"|
% \end{tabular}
% \end{center}
% Note that the trailing backslash on each first line
% merely continues the input to the second line
% (for convenient cut ant paste).
% Furthermore, the command |latex| can be replaced by any
% of its alternative versions such as |pdflatex|.
%
% %%%%%%%%%%%%%%%%%%%%%%%%%%%%%%%%%%%%%%%%%%%%%%%%%%%%%%%%%%%%%%%%%%%%%%%%%%%%%%
% %%%%%%%%%%%%%%%%%%%%%%%%%%%%%%%%%%%%%%%%%%%%%%%%%%%%%%%%%%%%%%%%%%%%%%%%%%%%%%
% \section{Implementation}
%\iffalse
%<*package>
%\fi
%
% This section describes the definitions file |childdoc.def|.

% The definitions cannot be loaded using |\usepackage| or |\RequirePackage|
% which has a mechanism to prevent loading a style file more than once.
% When loading the definitions by means of |\input|
% multiple instances have to be prevented manually:
%\iffalse
%This code needs to be before the `\ProvidesFile' directive
%which is defined at the beginning of this file.
%Therefore it is also placed there and commented out here.
%</package>
%<*discard>
%\fi
%    \begin{macrocode}
\ifdefined\childdocmain\endinput\fi
%    \end{macrocode}
%\iffalse
%</discard>
%<*package>
%\fi
%
% \macro{\ifchilddoc}
% \macro{\ifchilddocmanual}
% The conditional |\ifchilddoc| tells whether a
% child (true) or main (false) document is being compiled.
% The conditional |\ifchilddocmanual| tells whether
% the |\includeonly| mechanism is used (false) or
% the selection of child files must be performed manually (true).
% The definitions initialise to false:
%    \begin{macrocode}
\newif\ifchilddoc
\newif\ifchilddocmanual
%    \end{macrocode}

% \macro{\childdocname}
% \macro{\childdocjob}
% The macro |\childdocname| stores the name of the main document
% to be compiled. The macro |\childdocjob| stores the name of
% the document on which the \LaTeX{} compiler was originally invoked.
% The content of |\jobname| cannot be compared
% to filenames specified in the source due to different catcodes.
% The following code rescans |\jobname|, stores the result
% in |\childdocname| and saves a copy in |\childdocjob|:
%    \begin{macrocode}
\edef\childdocname{\scantokens\expandafter{\jobname\noexpand}}
\let\childdocjob\childdocname
%    \end{macrocode}

% \macro{\childdocdisable}
% The macro |\childdocdisable| prevents the main file
% from being processed more than once.
% At this stage, the main document command |\childdocmain|
% is assumed to be called once again where it should do nothing.
% Any subsequent call to it should prevent
% a secondary processing of the main document
% It overwrites the forwarding commands
% |\childdocof| and |\childdocforward|
% with empty macros to prevent further inclusions of the main document:
%    \begin{macrocode}
\newcommand{\childdocdisable}
{
  \renewcommand{\childdocmain}[1]{\renewcommand{\childdocmain}[1]{\endinput}}
  \renewcommand{\childdocof}[1]{}
  \renewcommand{\childdocby}[2][]{}
  \renewcommand{\childdocforward}[2][]{}
  \renewcommand{\childdocdisable}{}
}
%    \end{macrocode}

% \macro{\childdocmain}
% The macro |\childdocmain| is to be called at the top of the main file
% with nothing or the main filename (without extension) as argument.
% First, it breaks loops.
% If the argument is not empty and does not match |\childdocname|
% (which is set by the first inclusion of |childdoc.def|),
% |\ifchilddoc| is set to true, |\includeonly| is applied to the child file
% and |\jobname| is set to the main file
% (for proper handling of |.aux| files):
%    \begin{macrocode}
\newcommand{\childdocmain}[1]
{
  \childdocdisable\childdocmain{}
  \if?#1?\else
    \begingroup
      \def\childdoctmp{#1}
      \ifx\childdoctmp\childdocname
        \def\childdoctmp{}
      \else
        \def\childdoctmp
        {
          \childdoctrue
          \includeonly{\childdocname}
          \def\childdocjob{#1}
          \def\jobname{#1}
        }
      \fi
      \expandafter
    \endgroup
    \childdoctmp
  \fi
}
%    \end{macrocode}

% \macro{\childdocof}
% The command |\childdocof| redirects
% compilation to the main file |#1|.
%    \begin{macrocode}
\newcommand{\childdocof}[1]
{
  \childdocdisable
  \childdoctrue
  \includeonly{\childdocname}
  \def\jobname{#1}
  \def\childdocjob{#1}
  \input{#1}
}
%    \end{macrocode}

% \macro{\childdocby}
% The command |\childdocby| ....
%    \begin{macrocode}
\newcommand{\childdocby}[2][]
{
  \childdocdisable
  \childdoctrue
  \childdocmanualtrue
  \if?#1?\else
    \def\jobname{#2}
  \fi
  \def\childdocjob{#2}
  \input{#2}
  \endinput
}
%    \end{macrocode}

% \macro{\childdocforward}
% The command |\childdocforward| redirects
% compilation to the main file or
% (if the optional argument is given) a child file.
% Parameters are set as if the main file
% or a child file starting with |\childdocof| was compiled.
% Then compilation is handed over to the main file:
%    \begin{macrocode}
\newcommand{\childdocforward}[2][]
{
  \begingroup
    \if?#1?
      \def\childdoctmp
      {
        \def\childdocname{#2}
        \def\childdocjob{#2}
        \def\jobname{#2}
        \input{#2}
        \endinput
      }
    \else
      \def\childdoctmp
      {
        \childdocdisable
        \def\childdocname{#2}
        \childdoctrue
        \includeonly{#2}
        \def\childdocjob{#1}
        \def\jobname{#1}
        \input{#1}
        \endinput
      }
    \fi
    \expandafter
  \endgroup
  \childdoctmp
}
%    \end{macrocode}

% \macro{\childdocforwardprefix}
% The command |\childdocforwardprefix| redirects
% compilation to the main or a child file by means of a pattern.
% The prefix |#1| in the current filename is replaced by |#2|
% and the suffix of the current filename is kept
% (it is assumed that the filename does not contain the substring `|~~~|'
% which is used as a delimiter).
% Compilation is handed over to the new file by |\childdocforward|:
%    \begin{macrocode}
\newcommand{\childdocforwardprefix}[3][]
{
  \begingroup
    \def\childdocextract #2##1~~~{\def\childdoctmp{\childdocforward[#1]{#3##1}}}
    \expandafter\childdocextract\childdocname~~~
    \expandafter
  \endgroup
  \childdoctmp
}
%    \end{macrocode}

% \macro{\childdoc}
% The deprecated macro |\childdoc| is a legacy version of |\childdocmain|:
%    \begin{macrocode}
\newcommand{\childdoc}{\childdocmain}
%    \end{macrocode}

% \macro{\childdocredirect}
% The deprecated macro |\childdocredirect| is a legacy version
% of |\childdocforward| and |\childdocforwardprefix|:
%    \begin{macrocode}
\newcommand{\childdocredirect}[2][]
{
  \begingroup
    \if?#1?
      \def\childdoctmp{\childdocforward{#2}}
    \else
      \def\childdoctmp{\childdocforwardprefix{#1}{#2}}
    \fi
    \expandafter
  \endgroup
  \childdoctmp
}
%    \end{macrocode}

%\iffalse
%</package>
%\fi
%
\endinput
\childdocforward{cdocsch2}"|
% \end{tabular}
% \end{center}
% Note that the trailing backslash on each first line
% merely continues the input to the second line
% (for convenient cut ant paste).
% Furthermore, the command |latex| can be replaced by any
% of its alternative versions such as |pdflatex|.
%
% %%%%%%%%%%%%%%%%%%%%%%%%%%%%%%%%%%%%%%%%%%%%%%%%%%%%%%%%%%%%%%%%%%%%%%%%%%%%%%
% %%%%%%%%%%%%%%%%%%%%%%%%%%%%%%%%%%%%%%%%%%%%%%%%%%%%%%%%%%%%%%%%%%%%%%%%%%%%%%
% \section{Implementation}
%\iffalse
%<*package>
%\fi
%
% This section describes the definitions file |childdoc.def|.

% The definitions cannot be loaded using |\usepackage| or |\RequirePackage|
% which has a mechanism to prevent loading a style file more than once.
% When loading the definitions by means of |\input|
% multiple instances have to be prevented manually:
%\iffalse
%This code needs to be before the `\ProvidesFile' directive
%which is defined at the beginning of this file.
%Therefore it is also placed there and commented out here.
%</package>
%<*discard>
%\fi
%    \begin{macrocode}
\ifdefined\childdocmain\endinput\fi
%    \end{macrocode}
%\iffalse
%</discard>
%<*package>
%\fi
%
% \macro{\ifchilddoc}
% \macro{\ifchilddocmanual}
% The conditional |\ifchilddoc| tells whether a
% child (true) or main (false) document is being compiled.
% The conditional |\ifchilddocmanual| tells whether
% the |\includeonly| mechanism is used (false) or
% the selection of child files must be performed manually (true).
% The definitions initialise to false:
%    \begin{macrocode}
\newif\ifchilddoc
\newif\ifchilddocmanual
%    \end{macrocode}

% \macro{\childdocname}
% \macro{\childdocjob}
% The macro |\childdocname| stores the name of the main document
% to be compiled. The macro |\childdocjob| stores the name of
% the document on which the \LaTeX{} compiler was originally invoked.
% The content of |\jobname| cannot be compared
% to filenames specified in the source due to different catcodes.
% The following code rescans |\jobname|, stores the result
% in |\childdocname| and saves a copy in |\childdocjob|:
%    \begin{macrocode}
\edef\childdocname{\scantokens\expandafter{\jobname\noexpand}}
\let\childdocjob\childdocname
%    \end{macrocode}

% \macro{\childdocdisable}
% The macro |\childdocdisable| prevents the main file
% from being processed more than once.
% At this stage, the main document command |\childdocmain|
% is assumed to be called once again where it should do nothing.
% Any subsequent call to it should prevent
% a secondary processing of the main document
% It overwrites the forwarding commands
% |\childdocof| and |\childdocforward|
% with empty macros to prevent further inclusions of the main document:
%    \begin{macrocode}
\newcommand{\childdocdisable}
{
  \renewcommand{\childdocmain}[1]{\renewcommand{\childdocmain}[1]{\endinput}}
  \renewcommand{\childdocof}[1]{}
  \renewcommand{\childdocby}[2][]{}
  \renewcommand{\childdocforward}[2][]{}
  \renewcommand{\childdocdisable}{}
}
%    \end{macrocode}

% \macro{\childdocmain}
% The macro |\childdocmain| is to be called at the top of the main file
% with nothing or the main filename (without extension) as argument.
% First, it breaks loops.
% If the argument is not empty and does not match |\childdocname|
% (which is set by the first inclusion of |childdoc.def|),
% |\ifchilddoc| is set to true, |\includeonly| is applied to the child file
% and |\jobname| is set to the main file
% (for proper handling of |.aux| files):
%    \begin{macrocode}
\newcommand{\childdocmain}[1]
{
  \childdocdisable\childdocmain{}
  \if?#1?\else
    \begingroup
      \def\childdoctmp{#1}
      \ifx\childdoctmp\childdocname
        \def\childdoctmp{}
      \else
        \def\childdoctmp
        {
          \childdoctrue
          \includeonly{\childdocname}
          \def\childdocjob{#1}
          \def\jobname{#1}
        }
      \fi
      \expandafter
    \endgroup
    \childdoctmp
  \fi
}
%    \end{macrocode}

% \macro{\childdocof}
% The command |\childdocof| redirects
% compilation to the main file |#1|.
%    \begin{macrocode}
\newcommand{\childdocof}[1]
{
  \childdocdisable
  \childdoctrue
  \includeonly{\childdocname}
  \def\jobname{#1}
  \def\childdocjob{#1}
  \input{#1}
}
%    \end{macrocode}

% \macro{\childdocby}
% The command |\childdocby| ....
%    \begin{macrocode}
\newcommand{\childdocby}[2][]
{
  \childdocdisable
  \childdoctrue
  \childdocmanualtrue
  \if?#1?\else
    \def\jobname{#2}
  \fi
  \def\childdocjob{#2}
  \input{#2}
  \endinput
}
%    \end{macrocode}

% \macro{\childdocforward}
% The command |\childdocforward| redirects
% compilation to the main file or
% (if the optional argument is given) a child file.
% Parameters are set as if the main file
% or a child file starting with |\childdocof| was compiled.
% Then compilation is handed over to the main file:
%    \begin{macrocode}
\newcommand{\childdocforward}[2][]
{
  \begingroup
    \if?#1?
      \def\childdoctmp
      {
        \def\childdocname{#2}
        \def\childdocjob{#2}
        \def\jobname{#2}
        \input{#2}
        \endinput
      }
    \else
      \def\childdoctmp
      {
        \childdocdisable
        \def\childdocname{#2}
        \childdoctrue
        \includeonly{#2}
        \def\childdocjob{#1}
        \def\jobname{#1}
        \input{#1}
        \endinput
      }
    \fi
    \expandafter
  \endgroup
  \childdoctmp
}
%    \end{macrocode}

% \macro{\childdocforwardprefix}
% The command |\childdocforwardprefix| redirects
% compilation to the main or a child file by means of a pattern.
% The prefix |#1| in the current filename is replaced by |#2|
% and the suffix of the current filename is kept
% (it is assumed that the filename does not contain the substring `|~~~|'
% which is used as a delimiter).
% Compilation is handed over to the new file by |\childdocforward|:
%    \begin{macrocode}
\newcommand{\childdocforwardprefix}[3][]
{
  \begingroup
    \def\childdocextract #2##1~~~{\def\childdoctmp{\childdocforward[#1]{#3##1}}}
    \expandafter\childdocextract\childdocname~~~
    \expandafter
  \endgroup
  \childdoctmp
}
%    \end{macrocode}

% \macro{\childdoc}
% The deprecated macro |\childdoc| is a legacy version of |\childdocmain|:
%    \begin{macrocode}
\newcommand{\childdoc}{\childdocmain}
%    \end{macrocode}

% \macro{\childdocredirect}
% The deprecated macro |\childdocredirect| is a legacy version
% of |\childdocforward| and |\childdocforwardprefix|:
%    \begin{macrocode}
\newcommand{\childdocredirect}[2][]
{
  \begingroup
    \if?#1?
      \def\childdoctmp{\childdocforward{#2}}
    \else
      \def\childdoctmp{\childdocforwardprefix{#1}{#2}}
    \fi
    \expandafter
  \endgroup
  \childdoctmp
}
%    \end{macrocode}

%\iffalse
%</package>
%\fi
%
\endinput
|\\
|\childdocforward[|\textit{main}|]{|\textit{dest}|}|\\
\end{tabular}
\end{center}
%
The argument \textit{dest} is the destination file
(without extension).
It should be the main file or one of the child files.
Note that further \textsf{childdoc} directives
such as |\childdocof| and |\childdocforward|
in the indicated file will be processed in this form.
The optional argument \textit{main}
passes on directly to the main file \textit{main}
while pretending to compile the child \textit{dest}.
This form behaves as if \textit{dest}
issues |\childdocof{|\textit{main}|}| right away,
and no further \textsf{childdoc} directives will be processed.

%%%%%%%%%%%%%%%%%%%%%%%%%%%%%%%%%%%%%%%%
\DescribeMacro{\...prefix}
In the alternative form |\childdocforwardprefix|,
%
\begin{center}
\begin{tabular}{l}
|% \iffalse
%
% childdoc.dtx Copyright (C) 2017-2018 Niklas Beisert
%
% This work may be distributed and/or modified under the
% conditions of the LaTeX Project Public License, either version 1.3
% of this license or (at your option) any later version.
% The latest version of this license is in
%   http://www.latex-project.org/lppl.txt
% and version 1.3 or later is part of all distributions of LaTeX
% version 2005/12/01 or later.
%
% This work has the LPPL maintenance status `maintained'.
%
% The Current Maintainer of this work is Niklas Beisert.
%
% This work consists of the files childdoc.dtx and childdoc.ins
% and the derived files childdoc.def and cdocsamp.tex with
% cdocsch1.tex, cdocsch2.tex, cdocsdrf.tex, cdocsfn1.tex, cdocsfn2.tex.
%
%<package>\ifdefined\childdocmain\endinput\fi
%<package>\ProvidesFile{childdoc.def}[2018/12/30 v2.0 child document driver]
%<samplemain>\ProvidesFile{cdocsamp.tex}[2018/12/30 v2.0 sample for childdoc]
%<*driver>
%\ProvidesFile{childdoc.drv}[2018/12/30 v2.0 childdoc reference manual file]
\PassOptionsToClass{10pt,a4paper}{article}
\documentclass{ltxdoc}

\usepackage[margin=35mm]{geometry}
\usepackage{hyperref}
\usepackage{hyperxmp}
\usepackage[usenames]{color}

\hypersetup{colorlinks=true}
\hypersetup{pdfstartview=FitH}
\hypersetup{pdfpagemode=UseNone}
\hypersetup{pdfsource={}}
\hypersetup{pdflang={en-UK}}
\hypersetup{pdfcopyright={Copyright 2017-2018 Niklas Beisert.
  This work may be distributed and/or modified under the
  conditions of the LaTeX Project Public License, either version 1.3
  of this license or (at your option) any later version.}}
\hypersetup{pdflicenseurl={http://www.latex-project.org/lppl.txt}}
\hypersetup{pdfcontactaddress={ETH Zurich, ITP, HIT K,
  Wolfgang-Pauli-Strasse 27}}
\hypersetup{pdfcontactpostcode={8093}}
\hypersetup{pdfcontactcity={Zurich}}
\hypersetup{pdfcontactcountry={Switzerland}}
\hypersetup{pdfcontactemail={nbeisert@itp.phys.ethz.ch}}
\hypersetup{pdfcontacturl={http://people.phys.ethz.ch/\xmptilde nbeisert/}}

\newcommand{\secref}[1]{\hyperref[#1]{section \ref*{#1}}}

\parskip1ex
\parindent0pt
\let\olditemize\itemize
\def\itemize{\olditemize\parskip0pt}

\begin{document}

\title{The \textsf{childdoc} Package}
\hypersetup{pdftitle={The childdoc Package}}
\author{Niklas Beisert\\[2ex]
  Institut f\"ur Theoretische Physik\\
  Eidgen\"ossische Technische Hochschule Z\"urich\\
  Wolfgang-Pauli-Strasse 27, 8093 Z\"urich, Switzerland\\[1ex]
  \href{mailto:nbeisert@itp.phys.ethz.ch}
  {\texttt{nbeisert@itp.phys.ethz.ch}}}
\hypersetup{pdfauthor={Niklas Beisert}}
\hypersetup{pdfsubject={Manual for the LaTeX2e Package childdoc}}
\date{30 December 2018, \textsf{v2.0}}
\maketitle

\begin{abstract}\noindent
\textsf{childdoc} is a \LaTeXe{} package
that enables the direct compilation
of document sections included by |\include|
to individual files.
\end{abstract}

\begingroup
\parskip0ex
\tableofcontents
\endgroup

%%%%%%%%%%%%%%%%%%%%%%%%%%%%%%%%%%%%%%%%%%%%%%%%%%%%%%%%%%%%%%%%%%%%%%%%%%%%%%%%
%%%%%%%%%%%%%%%%%%%%%%%%%%%%%%%%%%%%%%%%%%%%%%%%%%%%%%%%%%%%%%%%%%%%%%%%%%%%%%%%
\section{Introduction}

\LaTeX{} provides a mechanism to structure a large document (such as a book)
into a main file and several child files (containing the chapters)
using the |\include| command.
This mechanism is beneficial for documents
which span hundreds of pages in order to
make the source file(s) more manageable.
Moreover, compilation can be restricted to
selected child files by means of the |\includeonly| command.
The latter feature can be used to reduce the compilation time while editing
(this was significantly more useful in the earlier days of \LaTeX{})
or to generate a smaller document which is easier to navigate.
Another application of |\includeonly| is to generate
documents consisting of selected parts of the complete document.

However, there are a few drawbacks of the plain |\include| mechanism:
\begin{itemize}
\item
The child files cannot be compiled on their own,
they can only be compiled via the main file.
A naive editing environment
(such as a text editor with an option
to have the current file processed by \LaTeX)
may require one to switch to the main file before compiling;
attempting to compile the child file produces errors.
\item
The main file must be modified (each time)
to adjust the |\includeonly| command
to the present needs. This easily leaves the main file in a messy state.
\item
The generated document will always carry the filename
of the main document. This is inconvenient if
several child files are to be compiled and
to be kept for distribution.
\end{itemize}

The present package provides a simple interface
to make child files individually compilable by \LaTeX{}.
Compiling a child file then has the same effect as compiling
the main file with an |\includeonly| command
to select the appropriate child.
Moreover the generated document will carry the name of the child
rather than the main file.
This resolves all three above issues.

This feature is meant to make the editing of books,
thesis documents and lecture notes somewhat more convenient.
However, the package can also be used efficiently for
composing a series of documents (such as exercise sheets)
which are typically distributed individually.
It then assists the author in generating the individual documents
(potentially in different versions)
as well as a document containing the collected series.
Another application is in developing style files
or other kinds of included material
where compilation of the style file could redirect
to a sample or test file.

%%%%%%%%%%%%%%%%%%%%%%%%%%%%%%%%%%%%%%%%%%%%%%%%%%%%%%%%%%%%%%%%%%%%%%%%%%%%%%%%
%%%%%%%%%%%%%%%%%%%%%%%%%%%%%%%%%%%%%%%%%%%%%%%%%%%%%%%%%%%%%%%%%%%%%%%%%%%%%%%%
\section{Usage}

First of all, the package \textsf{childdoc} is \emph{not} a standard
\LaTeXe{} |.sty| style file! Therefore it needs to be invoked in
a non-standard way.

%%%%%%%%%%%%%%%%%%%%%%%%%%%%%%%%%%%%%%%%%%%%%%%%%%%%%%%%%%%%%%%%%%%%%%%%%%%%%%%%
\subsection{Included Files}
\label{sec:include}

%%%%%%%%%%%%%%%%%%%%%%%%%%%%%%%%%%%%%%%%
\DescribeMacro{\childdocmain}
To use the package, add the commands
\begin{center}
\begin{tabular}{l}
|% \iffalse
%
% childdoc.dtx Copyright (C) 2017-2018 Niklas Beisert
%
% This work may be distributed and/or modified under the
% conditions of the LaTeX Project Public License, either version 1.3
% of this license or (at your option) any later version.
% The latest version of this license is in
%   http://www.latex-project.org/lppl.txt
% and version 1.3 or later is part of all distributions of LaTeX
% version 2005/12/01 or later.
%
% This work has the LPPL maintenance status `maintained'.
%
% The Current Maintainer of this work is Niklas Beisert.
%
% This work consists of the files childdoc.dtx and childdoc.ins
% and the derived files childdoc.def and cdocsamp.tex with
% cdocsch1.tex, cdocsch2.tex, cdocsdrf.tex, cdocsfn1.tex, cdocsfn2.tex.
%
%<package>\ifdefined\childdocmain\endinput\fi
%<package>\ProvidesFile{childdoc.def}[2018/12/30 v2.0 child document driver]
%<samplemain>\ProvidesFile{cdocsamp.tex}[2018/12/30 v2.0 sample for childdoc]
%<*driver>
%\ProvidesFile{childdoc.drv}[2018/12/30 v2.0 childdoc reference manual file]
\PassOptionsToClass{10pt,a4paper}{article}
\documentclass{ltxdoc}

\usepackage[margin=35mm]{geometry}
\usepackage{hyperref}
\usepackage{hyperxmp}
\usepackage[usenames]{color}

\hypersetup{colorlinks=true}
\hypersetup{pdfstartview=FitH}
\hypersetup{pdfpagemode=UseNone}
\hypersetup{pdfsource={}}
\hypersetup{pdflang={en-UK}}
\hypersetup{pdfcopyright={Copyright 2017-2018 Niklas Beisert.
  This work may be distributed and/or modified under the
  conditions of the LaTeX Project Public License, either version 1.3
  of this license or (at your option) any later version.}}
\hypersetup{pdflicenseurl={http://www.latex-project.org/lppl.txt}}
\hypersetup{pdfcontactaddress={ETH Zurich, ITP, HIT K,
  Wolfgang-Pauli-Strasse 27}}
\hypersetup{pdfcontactpostcode={8093}}
\hypersetup{pdfcontactcity={Zurich}}
\hypersetup{pdfcontactcountry={Switzerland}}
\hypersetup{pdfcontactemail={nbeisert@itp.phys.ethz.ch}}
\hypersetup{pdfcontacturl={http://people.phys.ethz.ch/\xmptilde nbeisert/}}

\newcommand{\secref}[1]{\hyperref[#1]{section \ref*{#1}}}

\parskip1ex
\parindent0pt
\let\olditemize\itemize
\def\itemize{\olditemize\parskip0pt}

\begin{document}

\title{The \textsf{childdoc} Package}
\hypersetup{pdftitle={The childdoc Package}}
\author{Niklas Beisert\\[2ex]
  Institut f\"ur Theoretische Physik\\
  Eidgen\"ossische Technische Hochschule Z\"urich\\
  Wolfgang-Pauli-Strasse 27, 8093 Z\"urich, Switzerland\\[1ex]
  \href{mailto:nbeisert@itp.phys.ethz.ch}
  {\texttt{nbeisert@itp.phys.ethz.ch}}}
\hypersetup{pdfauthor={Niklas Beisert}}
\hypersetup{pdfsubject={Manual for the LaTeX2e Package childdoc}}
\date{30 December 2018, \textsf{v2.0}}
\maketitle

\begin{abstract}\noindent
\textsf{childdoc} is a \LaTeXe{} package
that enables the direct compilation
of document sections included by |\include|
to individual files.
\end{abstract}

\begingroup
\parskip0ex
\tableofcontents
\endgroup

%%%%%%%%%%%%%%%%%%%%%%%%%%%%%%%%%%%%%%%%%%%%%%%%%%%%%%%%%%%%%%%%%%%%%%%%%%%%%%%%
%%%%%%%%%%%%%%%%%%%%%%%%%%%%%%%%%%%%%%%%%%%%%%%%%%%%%%%%%%%%%%%%%%%%%%%%%%%%%%%%
\section{Introduction}

\LaTeX{} provides a mechanism to structure a large document (such as a book)
into a main file and several child files (containing the chapters)
using the |\include| command.
This mechanism is beneficial for documents
which span hundreds of pages in order to
make the source file(s) more manageable.
Moreover, compilation can be restricted to
selected child files by means of the |\includeonly| command.
The latter feature can be used to reduce the compilation time while editing
(this was significantly more useful in the earlier days of \LaTeX{})
or to generate a smaller document which is easier to navigate.
Another application of |\includeonly| is to generate
documents consisting of selected parts of the complete document.

However, there are a few drawbacks of the plain |\include| mechanism:
\begin{itemize}
\item
The child files cannot be compiled on their own,
they can only be compiled via the main file.
A naive editing environment
(such as a text editor with an option
to have the current file processed by \LaTeX)
may require one to switch to the main file before compiling;
attempting to compile the child file produces errors.
\item
The main file must be modified (each time)
to adjust the |\includeonly| command
to the present needs. This easily leaves the main file in a messy state.
\item
The generated document will always carry the filename
of the main document. This is inconvenient if
several child files are to be compiled and
to be kept for distribution.
\end{itemize}

The present package provides a simple interface
to make child files individually compilable by \LaTeX{}.
Compiling a child file then has the same effect as compiling
the main file with an |\includeonly| command
to select the appropriate child.
Moreover the generated document will carry the name of the child
rather than the main file.
This resolves all three above issues.

This feature is meant to make the editing of books,
thesis documents and lecture notes somewhat more convenient.
However, the package can also be used efficiently for
composing a series of documents (such as exercise sheets)
which are typically distributed individually.
It then assists the author in generating the individual documents
(potentially in different versions)
as well as a document containing the collected series.
Another application is in developing style files
or other kinds of included material
where compilation of the style file could redirect
to a sample or test file.

%%%%%%%%%%%%%%%%%%%%%%%%%%%%%%%%%%%%%%%%%%%%%%%%%%%%%%%%%%%%%%%%%%%%%%%%%%%%%%%%
%%%%%%%%%%%%%%%%%%%%%%%%%%%%%%%%%%%%%%%%%%%%%%%%%%%%%%%%%%%%%%%%%%%%%%%%%%%%%%%%
\section{Usage}

First of all, the package \textsf{childdoc} is \emph{not} a standard
\LaTeXe{} |.sty| style file! Therefore it needs to be invoked in
a non-standard way.

%%%%%%%%%%%%%%%%%%%%%%%%%%%%%%%%%%%%%%%%%%%%%%%%%%%%%%%%%%%%%%%%%%%%%%%%%%%%%%%%
\subsection{Included Files}
\label{sec:include}

%%%%%%%%%%%%%%%%%%%%%%%%%%%%%%%%%%%%%%%%
\DescribeMacro{\childdocmain}
To use the package, add the commands
\begin{center}
\begin{tabular}{l}
|\input{childdoc.def}|\\
|\childdocmain{}|\\
\end{tabular}
\end{center}
at the very top of the main \LaTeX{} file,
in particular \emph{before} the |\documentclass| statement!
The argument of |\childdocmain| should be left empty
(but it must be present).

%%%%%%%%%%%%%%%%%%%%%%%%%%%%%%%%%%%%%%%%
\DescribeMacro{\childdocof}
Furthermore, add the commands
\begin{center}
\begin{tabular}{l}
|\input{childdoc.def}|\\
|\childdocof{|\textit{main}|}|\\
\end{tabular}
\end{center}
at the top of every child file \textit{child}
which is included by |\include{|\textit{child}|}|
from within the main file
(or at least for those files to be compiled individually).
The argument \textit{main} must be the filename of the main file.

There are a couple of
considerations in setting up the main and child documents:

%%%%%%%%%%%%%%%%%%%%%%%%%%%%%%%%%%%%%%%%
\paragraph{Restrictions.}

Please note the following restrictions:
\begin{itemize}
\item
|\childdocmain| must be called with one argument \textit{main}
to ensure compatibility with earlier version of the package.
It must either be empty (|\childdocmain{}|)
or precisely match the filename of the main file in which it is specified.
See \secref{sec:detection} for further information.
\item
The filename \textit{main} must be specified without the |.tex| extension.
\item
The filename \textit{main} is case sensitive
(even in case-insensitive file systems)
due to internal string comparison.
\item
The argument \textit{main} should be fully expanded, it cannot be a macro.
\item
Subdirectories and special characters should be avoided in filenames.
\item
The command |\childdocmain{|\textit{main}|}| must be followed by a whitespace.
It should not be followed immediately by another command
or by a comment mark `|%|'.
This is because the \TeX{} parser reads the token immediately following
the argument of |\childdocmain| and puts it
at the beginning of every child section;
however, a white\-space is ignored.
\end{itemize}

%%%%%%%%%%%%%%%%%%%%%%%%%%%%%%%%%%%%%%%%
\paragraph{Content of Main File.}

It is advisable to place all content in the child files included by |\include|.
Any output contained in the main file will appear in all child documents
unless suppressed manually;
it cannot be suppressed automatically by the |\includeonly| directive
and thus should normally be avoided.
A method to include some content in the main file
by means of conditional processing is described in \secref{sec:conditional}.

%%%%%%%%%%%%%%%%%%%%%%%%%%%%%%%%%%%%%%%%
\paragraph{Page Numbering.}

When only a part of the document is compiled,
the appropriate numbering of pages
(as well as other status parameters)
is determined from the |.aux| files.
The latter contain information from previous passes.
However this information needs to propagate through
all intermediate child documents.
Therefore the page numbering in child documents may well
be inconsistent until the complete document is compiled at least once.

A useful (if unconventional) way to always ensure a consistent
page numbering is to restart the numbering in each child document
and denote the pages by `\textit{child}|.|\textit{page}'
where \textit{child} represents the chapter/section number of the child file.
This can be achieved by the command
|\numberwithin{page}{|\textit{child}|}|
of the \textsf{amsmath} package
where \textit{child} can be |chapter| or |section|
depending on the chosen structuring.
Alternatively, one can modify the macro |\thepage| appropriately
and reset the counter |page| at the start of each child file.

%%%%%%%%%%%%%%%%%%%%%%%%%%%%%%%%%%%%%%%%%%%%%%%%%%%%%%%%%%%%%%%%%%%%%%%%%%%%%%%%
\subsection{Conditional Processing}
\label{sec:conditional}

The package provides a mechanism to compile different versions
of a document. To customise the versions further some conditional processing
can come in handy to distinguish which version is being compiled.
The package provides two macros to describe the compilation context:

%%%%%%%%%%%%%%%%%%%%%%%%%%%%%%%%%%%%%%%%
\DescribeMacro{\ifchilddoc}
The conditional |\ifchilddoc| distinguishes between the compilation of
child documents and the main document:
%
\begin{center}
|\ifchilddoc |\textit{child-code}| |[|\||else |\textit{main-code}]| \||fi|
\end{center}

%%%%%%%%%%%%%%%%%%%%%%%%%%%%%%%%%%%%%%%%
\DescribeMacro{\childdocname}
\DescribeMacro{\childdocjob}
The macro |\childdocname| contains the filename (without extension)
of the main or child file being processed.
Note that |\childdocjob| will always contain the name of the main file.

%%%%%%%%%%%%%%%%%%%%%%%%%%%%%%%%%%%%%%%%
\paragraph{Title Page.}

Conditional processing can be used to include a title or banner page
in the main document when proper precautions are taken.
Importantly, the code in the main file should ensure that the page counter
(as well as other status parameters which are stored in the |.aux| files)
takes the same value after the conditional processing.
Otherwise the page numbers may take divergent values
depending on which part is compiled.

For example, a title page could be declared by:
%
\begin{center}
\begin{tabular}{l}
|\ifchilddoc\||else|\\
|\addtocounter{page}{-1}|\\
\textit{code for title page}\\
|\newpage|\\
|\||fi|
\end{tabular}
\end{center}
%
A banner page for the child documents can be generated by:
%
\begin{center}
\begin{tabular}{l}
|\ifchilddoc|\\
|\addtocounter{page}{-1}|\\
\textit{code for banner page}\\
|\newpage|\\
|\||fi|
\end{tabular}
\end{center}
%
Here one could write a message such as:
\begin{center}
|This is the part \childdocname{} of \childdocjob{}.|
\end{center}

%%%%%%%%%%%%%%%%%%%%%%%%%%%%%%%%%%%%%%%%%%%%%%%%%%%%%%%%%%%%%%%%%%%%%%%%%%%%%%%%
\subsection{Flags}
\label{sec:flags}

The package makes it easy to generate different versions
of the main or child documents.
To this end compilation flags can be defined
and assigned different default values.
They will be particularly useful in conjunction
with the forwarding mechanism described in \secref{sec:forward}.

For example, it may be useful to have a flag |\version|
which can be set to |draft| or |final|.
The document source will contain some conditional code
depending on the value of |\version|.
Suppose further, the flag should default to |final| for the main file
and to |draft| for child files
which is a natural assignment for editing the document.
This is achieved by placing the following code
in the preamble of the main document
(below the |\childdocmain| directive):
%
\begin{center}
\begin{tabular}{l}
|\ifchilddoc|\\
|\providecommand{\version}{draft}|\\
|\||else|\\
|\providecommand{\version}{final}|\\
|\||fi|
\end{tabular}
\end{center}
%
The definition by |\providecommand| makes sure
that previous definitions are not overwritten.
Further statements |\providecommand{\version}{...}|
can thus be added before the above code to override it.

For the main file, one might add a line
(between |\childdocmain| and the above block)
%
\begin{center}
|%\ifchilddoc\||else\providecommand{\version}{draft}\||fi|
\end{center}
%
which can be uncommented to produce a draft version.
Likewise one can add a line to the very top of a child file
(above the |\childdocof{|\textit{main}|}| directive)
%
\begin{center}
|%\providecommand{\version}{final}|
\end{center}
%
which can be uncommented to produce the final version of this child document.

%%%%%%%%%%%%%%%%%%%%%%%%%%%%%%%%%%%%%%%%%%%%%%%%%%%%%%%%%%%%%%%%%%%%%%%%%%%%%%%%
\subsection{Forwarding}
\label{sec:forward}

Different versions of the main or child documents
using compilation flags as described in \secref{sec:flags}
can be (permanently) stored in different files
for convenient compilation, viewing and distribution.
To this end, the package defines a command
to pass on compilation to a different file:

%%%%%%%%%%%%%%%%%%%%%%%%%%%%%%%%%%%%%%%%
\DescribeMacro{\childdocforward}
The command |\childdocforward| redirects processing to
another source file:
%
\begin{center}
\begin{tabular}{l}
|\input{childdoc.def}|\\
|\childdocforward[|\textit{main}|]{|\textit{dest}|}|\\
\end{tabular}
\end{center}
%
The argument \textit{dest} is the destination file
(without extension).
It should be the main file or one of the child files.
Note that further \textsf{childdoc} directives
such as |\childdocof| and |\childdocforward|
in the indicated file will be processed in this form.
The optional argument \textit{main}
passes on directly to the main file \textit{main}
while pretending to compile the child \textit{dest}.
This form behaves as if \textit{dest}
issues |\childdocof{|\textit{main}|}| right away,
and no further \textsf{childdoc} directives will be processed.

%%%%%%%%%%%%%%%%%%%%%%%%%%%%%%%%%%%%%%%%
\DescribeMacro{\...prefix}
In the alternative form |\childdocforwardprefix|,
%
\begin{center}
\begin{tabular}{l}
|\input{childdoc.def}|\\
|\childdocforwardprefix[|\textit{main}|]{|\textit{prefix}|}{|\textit{dest}|}|
\end{tabular}
\end{center}
%
the destination file is determined by a pattern
depending on the current file:
To make this work, the current file must be called
`{\textit{prefix}\hspace{0.2em}\textit{suffix}}'
with \textit{prefix} matching precisely the argument.
Processing is then passed on to the file
`{\textit{dest}\hspace{0.2em}\textit{suffix}}'.
Surely, the same effect is achieved by
directly specifying the
argument `{\textit{dest}\hspace{0.2em}\textit{suffix}}'
in the first form.
However, that requires to set up a different file
for each child. With the alternative form of the command
all these files can have exactly the same content
which simplifies setting them up and maintaining them.

For example, the following file |draft.tex|
with a compilation flag |\version| as described in \secref{sec:flags}
compiles the main document as a draft:
%
\begin{center}
\begin{tabular}{l}
|\def\version{draft}|\\
|\input{childdoc.def}|\\
|\childdocforward{|\textit{main}|}|
\end{tabular}
\end{center}
%
Likewise, the following files |final|\textit{nn}|.tex|
compile the final version of the child document
|child|\textit{nn}|.tex|:
%
\begin{center}
\begin{tabular}{l}
|\def\version{final}|\\
|\input{childdoc.def}|\\
|\childdocforwardprefix{final}{child}|
\end{tabular}
\end{center}
%

Note that when several versions of a main file and/or of each child file
are to be generated, it may be convenient to set up a |Makefile| or
shell script to automatise the process.

%%%%%%%%%%%%%%%%%%%%%%%%%%%%%%%%%%%%%%%%%%%%%%%%%%%%%%%%%%%%%%%%%%%%%%%%%%%%%%%%
\subsection{Command Line Processing}
\label{sec:commandline}

The effect of redirection files can also be achieved by invoking
the \LaTeX{} compiler with a more elaborate command line.
Most conveniently this should be done as part
of a shell script or a |Makefile|.

When using \textsf{childdoc} in the main file, the following
command lines effectively perform a redirection
(note that depending on the shell being used,
backslashes may have to be doubled: `|\|' $\to$ `|\\|'):
%
\begin{center}
|... -jobname "|\textit{target}|" |\\|"|[\textit{flags}]%
|\input{childdoc.def}\childdocforward[|\textit{main}|]{|\textit{dest}|}"|
\end{center}
%
Here \textit{target} is the name of the output file,
\textit{main} is the name of the main file
and \textit{dest} is the name of the main or child file to be processed
(all filenames without extensions).
The optional argument \textit{main} can be omitted
if \textit{main} matches \textit{dest}.
Optionally, compilation \textit{flags} can be defined via |\def| commands.
This command line makes the \TeX{} engine believe
it is compiling the file \textit{target}
whose content is specified as the latter parameter.
The provided code then forwards the processing to
\textit{main} or \textit{dest} as described in \secref{sec:forward}.

%%%%%%%%%%%%%%%%%%%%%%%%%%%%%%%%%%%%%%%%%%%%%%%%%%%%%%%%%%%%%%%%%%%%%%%%%%%%%%%%
\subsection{Include by Input}
\label{sec:input}

Including child documents by |\include| has some restrictions by design.
Most notably, the content of a child document always occupies
its own set of pages; pages cannot be shared between child documents.
Usually, this behaviour makes perfect sense
because each child document contain an essential part of the document.
However, in some situations it may be desirable to compose
a document from a collection of parts
without having mandatory page breaks between then.
For this case, the package
provides a mechanism to include parts
by |\input| which can also be processed individually.
However, by construction this mechanism
requires manual handling of the content to be output.

%%%%%%%%%%%%%%%%%%%%%%%%%%%%%%%%%%%%%%%%
\DescribeMacro{\ifchilddocmanual}
The main file should be prepared as usual, see \secref{sec:include}.
However, the document body must make a distinction
between processing of an individual part and of the main document, e.g.:
%
\begin{center}
\begin{tabular}{l}
|\ifchilddocmanual|\\
|\input{\childdocname}|\\
|\||else|\\
\textit{document body with }|\input{|\textit{part}|}|\\
|\||fi|
\end{tabular}
\end{center}
%
The conditional |\ifchilddocmanual| is true whenever
a part to be included by |\input| is being compiled,
and the name of the part is stored in |\childdocname|.

%%%%%%%%%%%%%%%%%%%%%%%%%%%%%%%%%%%%%%%%
\DescribeMacro{\childdocby}
Each part to be included by |\input| should start with:
%
\begin{center}
\begin{tabular}{l}
|\input{childdoc.def}|\\
|\childdocby{|\textit{main}|}|\\
\end{tabular}
\end{center}
%
The directive |\childdocby| is similar to |\childdocof|
described in \secref{sec:include},
but the subsequent selection of content must be done manually.
To that end, both |\ifchilddoc| and |\ifchilddocmanual|
will be true upon processing of a part,
and the name of the part is stored in |\childdocname|.
Note that |\jobname| will be set to the filename of the current part
so that each part receives an individual |.aux| file
that does not interfere with the |.aux| file(s) of the main document.
This behaviour can be altered by the alternative form
|\childdocby[*]{|\textit{main}|}| (with a non-empty optional argument)
which uses the |.aux| file of the main document
by setting |\jobname| to \textit{main}.

%%%%%%%%%%%%%%%%%%%%%%%%%%%%%%%%%%%%%%%%%%%%%%%%%%%%%%%%%%%%%%%%%%%%%%%%%%%%%%%%
\subsection{Driver Development}
\label{sec:driver}

The \textsf{childdoc} mechanism can also be use for the development
of definition files such as \LaTeX{} styles or classes.
This case differs from the above setup with multiple parts
included by |\include| in that no |\includeonly| should be invoked.
This can be achieved by starting the include file
(before |\ProvidesPackage|) with:
%
\begin{center}
\begin{tabular}{l}
|\input{childdoc.def}|\\
|\childdocforward{|\textit{main}|}|\\
\end{tabular}
\end{center}
%
or alternatively with:
%
\begin{center}
\begin{tabular}{l}
|\input{childdoc.def}|\\
|\childdocby{|\textit{main}|}|\\
\end{tabular}
\end{center}
%
Both forms have slightly different effects as described above.
The main file is prepared as usual, see \secref{sec:include}.

%%%%%%%%%%%%%%%%%%%%%%%%%%%%%%%%%%%%%%%%%%%%%%%%%%%%%%%%%%%%%%%%%%%%%%%%%%%%%%%%
\subsection{Legacy Detection}
\label{sec:detection}

The directive |\childdocmain| in the main file can detect
whether the complete document or merely a child is to be compiled
even without using the directive |\childdocof|.
This method is deprecated because it is less robust
and there is no compelling reason to use it;
it is merely provided for backward compatibility
and it may be removed in future versions.

If the detection mechanism is to be used,
it is mandatory to correctly specify
the filename of the main file as the argument of |\childdocmain|:
%
\begin{center}
\begin{tabular}{l}
|\input{childdoc.def}|\\
|\childdocmain{|\textit{main}|}|\\
\end{tabular}
\end{center}
%
If |\jobname| does not match the argument \textit{main} of |\childdocmain|,
it is assumed that |\jobname| points to the child file to be compiled.
When using |\childdocmain| with the main file specified as argument,
it suffices to start a child file
with just |\input{|\textit{main}|}|
without loading of the package and using |\childdocof|.
If instead all processing is done
with the appropriate \textsf{childdoc} directives,
the argument of \textit{main} of |\childdocmain| can be empty.

An alternative version of the command line processing described
in \secref{sec:commandline} using the detection mechanism reads:
%
\begin{center}
|... -jobname "|\textit{target}|" "|[\textit{flags}]%
[|\def\jobname{|\textit{dest}|}|]|\input{|\textit{main}|}"|
\end{center}

%%%%%%%%%%%%%%%%%%%%%%%%%%%%%%%%%%%%%%%%%%%%%%%%%%%%%%%%%%%%%%%%%%%%%%%%%%%%%%%%
\subsection{Manual Code}
\label{sec:manual}

In case one cannot be certain whether the definitions file |childdoc.def|
is installed on the target \TeX{} distribution
and one prefers not to ship it,
it is conceivable to paste a few relevant commands into the sources.

To that end, drop all statements |\input{childdoc.def}|
and perform the replacements as outlined below.
Instead of |\childdocmain{|\textit{main}|}| add the following code
to the top of the main file:
%
\begin{center}
\begin{tabular}{l}
|\||ifdefined\childdocname\endinput\||fi\newif\ifchilddoc|\\
|\edef\childdocname{\scantokens\expandafter{\jobname\noexpand}}|\\
|\def\childdocmain{|\textit{main}|}\||ifx\childdocmain\childdocname\||else|\\
|\childdoctrue\includeonly{\childdocname}\let\jobname\childdocmain\||fi|\\
\end{tabular}
\end{center}
%
Instead of |\childdocof{|\textit{main}|}| just include the main file
at the top of each child file:
%
\begin{center}
|\input{|\textit{main}|}|
\end{center}
%
A simple redirection |\childdocforward{|\textit{dest}|}| is achieved by:
%
\begin{center}
|\def\jobname{|\textit{dest}|}\input{\jobname}|
\end{center}
%
The redirection with prefix
|\childdocforwardprefix[|\textit{prefix}|]{|\textit{dest}|}|
is accomplished by:
%
\begin{center}
\begin{tabular}{l}
|{\edef\jobname{\scantokens\expandafter{\jobname\noexpand}}|\\
|\def\redirectjob |\textit{prefix}|#1~~~{\gdef\jobname{|\textit{dest}|#1}}|\\
|\expandafter\redirectjob\jobname~~~}\input{\jobname}|
\end{tabular}
\end{center}

In an alternative approach,
child documents can be compiled by a specific command line
without additional code or specific definitions:
%
\begin{center}
|... -jobname "|\textit{target}|" "|[\textit{flags}]%
|\includeonly{|\textit{dest}|}\input{|\textit{main}|}"|
\end{center}
%

%%%%%%%%%%%%%%%%%%%%%%%%%%%%%%%%%%%%%%%%%%%%%%%%%%%%%%%%%%%%%%%%%%%%%%%%%%%%%%%%
%%%%%%%%%%%%%%%%%%%%%%%%%%%%%%%%%%%%%%%%%%%%%%%%%%%%%%%%%%%%%%%%%%%%%%%%%%%%%%%%
\section{Information}

%%%%%%%%%%%%%%%%%%%%%%%%%%%%%%%%%%%%%%%%%%%%%%%%%%%%%%%%%%%%%%%%%%%%%%%%%%%%%%%%
\subsection{Copyright}

Copyright \copyright{} 2017--2018 Niklas Beisert

This work may be distributed and/or modified under the
conditions of the \LaTeX{} Project Public License, either version 1.3
of this license or (at your option) any later version.
The latest version of this license is in
  \url{http://www.latex-project.org/lppl.txt}
and version 1.3 or later is part of all distributions of \LaTeX{}
version 2005/12/01 or later.

This work has the LPPL maintenance status `maintained'.

The Current Maintainer of this work is Niklas Beisert.

This work consists of the files |README.txt|, |childdoc.ins| and |childdoc.dtx|
as well as the derived files |childdoc.def|, |cdocsamp.tex|
with |cdocsch1.tex|, |cdocsch2.tex|, |cdocspt3.tex|, |cdocspt4.tex|,
|cdocsdrf.tex|, |cdocsfn1.tex|, |cdocsfn2.tex|
as well as |childdoc.pdf|.

%%%%%%%%%%%%%%%%%%%%%%%%%%%%%%%%%%%%%%%%%%%%%%%%%%%%%%%%%%%%%%%%%%%%%%%%%%%%%%%%
\subsection{Files and Installation}

The package consists of the files:
%
\begin{center}
\begin{tabular}{ll}
    |README.txt|   & readme file \\
    |childdoc.ins| & installation file \\
    |childdoc.dtx| & source file \\
    |childdoc.def| & definition file \\
    |cdocsamp.tex| & sample main file \\
    |cdocsch1.tex| & sample include file \\
    |cdocsch2.tex| & sample include file \\
    |cdocspt3.tex| & sample part file \\
    |cdocspt4.tex| & sample part file \\
    |cdocsdrf.tex| & sample redirection file \\
    |cdocsfn1.tex| & sample redirection file \\
    |cdocsfn2.tex| & sample redirection file \\
    |childdoc.pdf| & manual
\end{tabular}
\end{center}
%
The distribution consists of the files
|README.txt|, |childdoc.ins| and |childdoc.dtx|.
%
\begin{itemize}
\item
Run (pdf)\LaTeX{} on |childdoc.dtx|
to compile the manual |childdoc.pdf| (this file).
\item
Run \LaTeX{} on |childdoc.ins| to create the definitions file |childdoc.def|
and the sample |cdocsamp.tex| with include files
|cdocsch1.tex|, |cdocsch2.tex|, |cdocspt3.tex|, |cdocspt4.tex|,
|cdocsdrf.tex|, |cdocsfn1.tex|, |cdocsfn2.tex|.
Then copy the file |childdoc.def| to an appropriate directory of your \LaTeX{}
distribution, e.g.\ \textit{texmf-root}|/tex/latex/childdoc|.
\end{itemize}

%%%%%%%%%%%%%%%%%%%%%%%%%%%%%%%%%%%%%%%%%%%%%%%%%%%%%%%%%%%%%%%%%%%%%%%%%%%%%%%%
\subsection{Related CTAN Packages}

There are several other packages which offer a similar functionality:
%
\begin{itemize}
\item
The packages
\href{http://ctan.org/pkg/docmute}{\textsf{docmute}},
\href{http://ctan.org/pkg/includex}{\textsf{includex}} and
\href{http://ctan.org/pkg/standalone}{\textsf{standalone}}
provide commands to include only the document body of
a child file thus allowing both files to be compiled individually.
\item
The packages \href{http://ctan.org/pkg/subdocs}{\textsf{subdocs}}
and \href{http://ctan.org/pkg/subfiles}{\textsf{subfiles}}
provide structures in which the main and child documents can be
encapsulated and allowing them to be compiled individually.
The inclusion mechanism is different from the conventional |\include|.
\item
The package \href{http://ctan.org/pkg/combine}{\textsf{combine}}
is an elaborate solution to combine several documents into one.
\end{itemize}
%
See also the CTAN topic \href{http://ctan.org/topic/subdocs}{\textsf{subdocs}}
for further related packages.
The present package differs from the above solutions in that
a document structure constructed with the conventional |\include| mechanism
just needs two extra commands at the top of every file
such that all constituent files can be compiled individually.

%%%%%%%%%%%%%%%%%%%%%%%%%%%%%%%%%%%%%%%%%%%%%%%%%%%%%%%%%%%%%%%%%%%%%%%%%%%%%%%%
%\subsection{Feature Suggestions}
%
%The following is a list of features which may be useful for future
%versions of this package:
%%
%\begin{itemize}
%\item
%\ldots
%\end{itemize}

%%%%%%%%%%%%%%%%%%%%%%%%%%%%%%%%%%%%%%%%%%%%%%%%%%%%%%%%%%%%%%%%%%%%%%%%%%%%%%%%
\subsection{Revision History}

%%%%%%%%%%%%%%%%%%%%%%%%%%%%%%%%%%%%%%%%
\paragraph{v2.0:} 2018/12/30

\begin{itemize}
\item
immediate forward processing
\item
added |\childdocby| mechanism
\item
manual restructured
\end{itemize}

%%%%%%%%%%%%%%%%%%%%%%%%%%%%%%%%%%%%%%%%
\paragraph{v1.6:} 2018/01/17

\begin{itemize}
\item
application for development of include files
\item
corrections to manual
\end{itemize}

%%%%%%%%%%%%%%%%%%%%%%%%%%%%%%%%%%%%%%%%
\paragraph{v1.5:} 2017/05/21

\begin{itemize}
\item
more complete structuring introduced
\item
|\childdocof| introduced
\item
|\childdoc| renamed to |\childdocmain|
\item
|\childredirect| renamed to |\childdocforward| and |\childdocforwardprefix|
and functionality expanded
\end{itemize}

%%%%%%%%%%%%%%%%%%%%%%%%%%%%%%%%%%%%%%%%
\paragraph{v1.0:} 2017/04/27

\begin{itemize}
\item
manual and install package
\item
first version published on CTAN
\end{itemize}

%%%%%%%%%%%%%%%%%%%%%%%%%%%%%%%%%%%%%%%%
\paragraph{v0.6:} 2017/04/26

\begin{itemize}
\item
redirection mechanism added
\end{itemize}

%%%%%%%%%%%%%%%%%%%%%%%%%%%%%%%%%%%%%%%%
\paragraph{v0.5:} 2017/04/26

\begin{itemize}
\item
functionality in definition file
\end{itemize}


%%%%%%%%%%%%%%%%%%%%%%%%%%%%%%%%%%%%%%%%%%%%%%%%%%%%%%%%%%%%%%%%%%%%%%%%%%%%%%%%
%%%%%%%%%%%%%%%%%%%%%%%%%%%%%%%%%%%%%%%%%%%%%%%%%%%%%%%%%%%%%%%%%%%%%%%%%%%%%%%%
%%%%%%%%%%%%%%%%%%%%%%%%%%%%%%%%%%%%%%%%%%%%%%%%%%%%%%%%%%%%%%%%%%%%%%%%%%%%%%%%
\appendix

\settowidth\MacroIndent{\rmfamily\scriptsize 000\ }

 \DocInput{childdoc.dtx}

\end{document}
%</driver>
% \fi
%
% %%%%%%%%%%%%%%%%%%%%%%%%%%%%%%%%%%%%%%%%%%%%%%%%%%%%%%%%%%%%%%%%%%%%%%%%%%%%%%
% %%%%%%%%%%%%%%%%%%%%%%%%%%%%%%%%%%%%%%%%%%%%%%%%%%%%%%%%%%%%%%%%%%%%%%%%%%%%%%
% \section{Sample}
%\iffalse
%<*samplemain>
%\fi
%
% The following presents a sample document
% with two chapters, two parts, a title page,
% a compile flag as well as three forwarding files to set the flag.
% It consists of eight |.tex| files:
% \begin{center}
% \begin{tabular}{ll}
% |cdocsamp.tex|&main file\\
% |cdocsch1.tex|&include file for chapter 1\\
% |cdocsch2.tex|&include file for chapter 2\\
% |cdocspt3.tex|&include file for part 3\\
% |cdocspt4.tex|&include file for part 4\\
% |cdocsdrf.tex|&forwarding file for main file in draft mode\\
% |cdocsfi1.tex|&forwarding file for final version of chapter 1\\
% |cdocsfi2.tex|&forwarding file for final version of chapter 2\\
% \end{tabular}
% \end{center}
% Each of the eight files can be compiled directly by the \LaTeX{} compiler.
%
% %%%%%%%%%%%%%%%%%%%%%%%%%%%%%%%%%%%%%%
% \paragraph{Main File.}
%
% The main file is called |cdocsamp.tex|.
%
% Load the \textsf{childdoc} definitions and
% declare the filename for the main document:
%    \begin{macrocode}
\input{childdoc.def}
\childdocmain{}
%    \end{macrocode}

% Optional override for |\version| flag:
%    \begin{macrocode}
%%\ifchilddoc\else\providecommand{\version}{draft}\fi
%    \end{macrocode}

% Define the default values for the |\version| flag
% (|final| for the main file and |draft| for childs):
%    \begin{macrocode}
\ifchilddoc
\providecommand{\version}{draft}
\else
\providecommand{\version}{final}
\fi
%    \end{macrocode}

% Load the standard document class:
%    \begin{macrocode}
\documentclass[12pt]{article}
%    \end{macrocode}

% Start the document body:
%    \begin{macrocode}
\begin{document}
%    \end{macrocode}

% Declare a title page.
% Print title, part of document being processed and version flag:
%    \begin{macrocode}
\addtocounter{page}{-1}
\begin{center}
{\LARGE\bfseries{}childdoc example\par}
\vspace{1cm}
\ifchilddoc
\ifchilddocmanual part\else chapter\fi:
`\childdocname' of `\childdocjob'\par
\else
main document: `\childdocjob'\par
\fi
version: \version\par
\end{center}
\newpage
%    \end{macrocode}

% Manually include selected file,
% otherwise process as usual:
%    \begin{macrocode}
\ifchilddocmanual
\section*{part `\childdocname'}
\input{\childdocname}
\else
%    \end{macrocode}

% Include the two chapters:
%    \begin{macrocode}
\include{cdocsch1}
\include{cdocsch2}
%    \end{macrocode}

% Include the two parts unless only chapters should be displayed:
%    \begin{macrocode}
\ifchilddoc\else
\section{part three}
\input{cdocspt3}
\section{part four}
\input{cdocspt4}
\fi
%    \end{macrocode}

% Process as usual until here:
%    \begin{macrocode}
\fi
%    \end{macrocode}

% End of document body:
%    \begin{macrocode}
\end{document}
%    \end{macrocode}
%\iffalse
%</samplemain>
%\fi
%
% %%%%%%%%%%%%%%%%%%%%%%%%%%%%%%%%%%%%%%
% \paragraph{Chapter Include Files.}
%
% The include files are called |cdocsch1.tex| and |cdocsch2.tex|.
%
%\iffalse
%<*samplechap1|samplechap2>
%\fi

% Optional override for |\version| flag:
%    \begin{macrocode}
%%\providecommand{\version}{final}
%    \end{macrocode}

% Include the main document:
%    \begin{macrocode}
\input{childdoc.def}
\childdocof{cdocsamp}
%    \end{macrocode}

%\iffalse
%</samplechap1|samplechap2>
%\fi
%
%\iffalse
%<*samplechap1>
%\fi
% Some text for chapter 1:
%    \begin{macrocode}
\section{one}
some text in chapter one
%    \end{macrocode}

%\iffalse
%</samplechap1>
%\fi
% Some text for chapter 2:
%\iffalse
%<*samplechap2>
%\fi
%    \begin{macrocode}
\section{two}
more text in chapter two
%    \end{macrocode}

%\iffalse
%</samplechap2>
%\fi
%
% %%%%%%%%%%%%%%%%%%%%%%%%%%%%%%%%%%%%%%
% \paragraph{Part Include Files.}
%
% The include files are called |cdocspt3.tex| and |cdocspt4.tex|.
%
%\iffalse
%<*samplepart3|samplepart4>
%\fi

% Optional override for |\version| flag:
%    \begin{macrocode}
%%\providecommand{\version}{final}
%    \end{macrocode}

% Include the main document:
%    \begin{macrocode}
\input{childdoc.def}
\childdocby{cdocsamp}
%    \end{macrocode}

%\iffalse
%</samplepart3|samplepart4>
%\fi
%
%\iffalse
%<*samplepart3>
%\fi
% Some text for part 3:
%    \begin{macrocode}
some text in part three
%    \end{macrocode}

%\iffalse
%</samplepart3>
%\fi
% Some text for part 4:
%\iffalse
%<*samplepart4>
%\fi
%    \begin{macrocode}
more text in part four
%    \end{macrocode}

%\iffalse
%</samplepart4>
%\fi
%
% %%%%%%%%%%%%%%%%%%%%%%%%%%%%%%%%%%%%%%
% \paragraph{Forwarding for a Complete Draft.}
%
% The following forwarding file |cdocsdrf.tex|
% compiles the main document in draft mode:
%\iffalse
%<*sampledraft>
%\fi
%    \begin{macrocode}
\def\version{draft}
\input{childdoc.def}
\childdocforward{cdocsamp}
%    \end{macrocode}

%\iffalse
%</sampledraft>
%\fi
%
% %%%%%%%%%%%%%%%%%%%%%%%%%%%%%%%%%%%%%%
% \paragraph{Forwarding for Final Version of the Chapters.}
%
% The following forwarding files |cdocsfn1.tex| and |cdocsfn2.tex|
% (with identical content)
% compile the final versions of the child documents
% |cdocsch1.tex| and |cdocsch2.tex|, respectively:
%\iffalse
%<*samplefinal>
%\fi
%    \begin{macrocode}
\def\version{final}
\input{childdoc.def}
\childdocforwardprefix[cdocsamp]{cdocsfn}{cdocsch}
%    \end{macrocode}

%\iffalse
%</samplefinal>
%\fi
%
% %%%%%%%%%%%%%%%%%%%%%%%%%%%%%%%%%%%%%%
% \paragraph{Command Line Processing.}
%
% The following three command lines generate the output files
% |cdocscld|, |cdocscl1| and |cdocscl2|
% which should be identical to
% |cdocsdrf|, |cdocsch1| and |cdocsfn2|, respectively:
% \begin{center}
% \begin{tabular}{l}
% |latex -jobname cdocscld \|\\
% |  "\def\version{draft}\input{childdoc.def}\childdocforward{cdocsamp}"|\\
% |latex -jobname cdocscl1 \|\\
% |  "\input{childdoc.def}\childdocforward[cdocsamp]{cdocsch1}"|\\
% |latex -jobname cdocscl2 \|\\
% |  "\def\version{final}\input{childdoc.def}\childdocforward{cdocsch2}"|
% \end{tabular}
% \end{center}
% Note that the trailing backslash on each first line
% merely continues the input to the second line
% (for convenient cut ant paste).
% Furthermore, the command |latex| can be replaced by any
% of its alternative versions such as |pdflatex|.
%
% %%%%%%%%%%%%%%%%%%%%%%%%%%%%%%%%%%%%%%%%%%%%%%%%%%%%%%%%%%%%%%%%%%%%%%%%%%%%%%
% %%%%%%%%%%%%%%%%%%%%%%%%%%%%%%%%%%%%%%%%%%%%%%%%%%%%%%%%%%%%%%%%%%%%%%%%%%%%%%
% \section{Implementation}
%\iffalse
%<*package>
%\fi
%
% This section describes the definitions file |childdoc.def|.

% The definitions cannot be loaded using |\usepackage| or |\RequirePackage|
% which has a mechanism to prevent loading a style file more than once.
% When loading the definitions by means of |\input|
% multiple instances have to be prevented manually:
%\iffalse
%This code needs to be before the `\ProvidesFile' directive
%which is defined at the beginning of this file.
%Therefore it is also placed there and commented out here.
%</package>
%<*discard>
%\fi
%    \begin{macrocode}
\ifdefined\childdocmain\endinput\fi
%    \end{macrocode}
%\iffalse
%</discard>
%<*package>
%\fi
%
% \macro{\ifchilddoc}
% \macro{\ifchilddocmanual}
% The conditional |\ifchilddoc| tells whether a
% child (true) or main (false) document is being compiled.
% The conditional |\ifchilddocmanual| tells whether
% the |\includeonly| mechanism is used (false) or
% the selection of child files must be performed manually (true).
% The definitions initialise to false:
%    \begin{macrocode}
\newif\ifchilddoc
\newif\ifchilddocmanual
%    \end{macrocode}

% \macro{\childdocname}
% \macro{\childdocjob}
% The macro |\childdocname| stores the name of the main document
% to be compiled. The macro |\childdocjob| stores the name of
% the document on which the \LaTeX{} compiler was originally invoked.
% The content of |\jobname| cannot be compared
% to filenames specified in the source due to different catcodes.
% The following code rescans |\jobname|, stores the result
% in |\childdocname| and saves a copy in |\childdocjob|:
%    \begin{macrocode}
\edef\childdocname{\scantokens\expandafter{\jobname\noexpand}}
\let\childdocjob\childdocname
%    \end{macrocode}

% \macro{\childdocdisable}
% The macro |\childdocdisable| prevents the main file
% from being processed more than once.
% At this stage, the main document command |\childdocmain|
% is assumed to be called once again where it should do nothing.
% Any subsequent call to it should prevent
% a secondary processing of the main document
% It overwrites the forwarding commands
% |\childdocof| and |\childdocforward|
% with empty macros to prevent further inclusions of the main document:
%    \begin{macrocode}
\newcommand{\childdocdisable}
{
  \renewcommand{\childdocmain}[1]{\renewcommand{\childdocmain}[1]{\endinput}}
  \renewcommand{\childdocof}[1]{}
  \renewcommand{\childdocby}[2][]{}
  \renewcommand{\childdocforward}[2][]{}
  \renewcommand{\childdocdisable}{}
}
%    \end{macrocode}

% \macro{\childdocmain}
% The macro |\childdocmain| is to be called at the top of the main file
% with nothing or the main filename (without extension) as argument.
% First, it breaks loops.
% If the argument is not empty and does not match |\childdocname|
% (which is set by the first inclusion of |childdoc.def|),
% |\ifchilddoc| is set to true, |\includeonly| is applied to the child file
% and |\jobname| is set to the main file
% (for proper handling of |.aux| files):
%    \begin{macrocode}
\newcommand{\childdocmain}[1]
{
  \childdocdisable\childdocmain{}
  \if?#1?\else
    \begingroup
      \def\childdoctmp{#1}
      \ifx\childdoctmp\childdocname
        \def\childdoctmp{}
      \else
        \def\childdoctmp
        {
          \childdoctrue
          \includeonly{\childdocname}
          \def\childdocjob{#1}
          \def\jobname{#1}
        }
      \fi
      \expandafter
    \endgroup
    \childdoctmp
  \fi
}
%    \end{macrocode}

% \macro{\childdocof}
% The command |\childdocof| redirects
% compilation to the main file |#1|.
%    \begin{macrocode}
\newcommand{\childdocof}[1]
{
  \childdocdisable
  \childdoctrue
  \includeonly{\childdocname}
  \def\jobname{#1}
  \def\childdocjob{#1}
  \input{#1}
}
%    \end{macrocode}

% \macro{\childdocby}
% The command |\childdocby| ....
%    \begin{macrocode}
\newcommand{\childdocby}[2][]
{
  \childdocdisable
  \childdoctrue
  \childdocmanualtrue
  \if?#1?\else
    \def\jobname{#2}
  \fi
  \def\childdocjob{#2}
  \input{#2}
  \endinput
}
%    \end{macrocode}

% \macro{\childdocforward}
% The command |\childdocforward| redirects
% compilation to the main file or
% (if the optional argument is given) a child file.
% Parameters are set as if the main file
% or a child file starting with |\childdocof| was compiled.
% Then compilation is handed over to the main file:
%    \begin{macrocode}
\newcommand{\childdocforward}[2][]
{
  \begingroup
    \if?#1?
      \def\childdoctmp
      {
        \def\childdocname{#2}
        \def\childdocjob{#2}
        \def\jobname{#2}
        \input{#2}
        \endinput
      }
    \else
      \def\childdoctmp
      {
        \childdocdisable
        \def\childdocname{#2}
        \childdoctrue
        \includeonly{#2}
        \def\childdocjob{#1}
        \def\jobname{#1}
        \input{#1}
        \endinput
      }
    \fi
    \expandafter
  \endgroup
  \childdoctmp
}
%    \end{macrocode}

% \macro{\childdocforwardprefix}
% The command |\childdocforwardprefix| redirects
% compilation to the main or a child file by means of a pattern.
% The prefix |#1| in the current filename is replaced by |#2|
% and the suffix of the current filename is kept
% (it is assumed that the filename does not contain the substring `|~~~|'
% which is used as a delimiter).
% Compilation is handed over to the new file by |\childdocforward|:
%    \begin{macrocode}
\newcommand{\childdocforwardprefix}[3][]
{
  \begingroup
    \def\childdocextract #2##1~~~{\def\childdoctmp{\childdocforward[#1]{#3##1}}}
    \expandafter\childdocextract\childdocname~~~
    \expandafter
  \endgroup
  \childdoctmp
}
%    \end{macrocode}

% \macro{\childdoc}
% The deprecated macro |\childdoc| is a legacy version of |\childdocmain|:
%    \begin{macrocode}
\newcommand{\childdoc}{\childdocmain}
%    \end{macrocode}

% \macro{\childdocredirect}
% The deprecated macro |\childdocredirect| is a legacy version
% of |\childdocforward| and |\childdocforwardprefix|:
%    \begin{macrocode}
\newcommand{\childdocredirect}[2][]
{
  \begingroup
    \if?#1?
      \def\childdoctmp{\childdocforward{#2}}
    \else
      \def\childdoctmp{\childdocforwardprefix{#1}{#2}}
    \fi
    \expandafter
  \endgroup
  \childdoctmp
}
%    \end{macrocode}

%\iffalse
%</package>
%\fi
%
\endinput
|\\
|\childdocmain{}|\\
\end{tabular}
\end{center}
at the very top of the main \LaTeX{} file,
in particular \emph{before} the |\documentclass| statement!
The argument of |\childdocmain| should be left empty
(but it must be present).

%%%%%%%%%%%%%%%%%%%%%%%%%%%%%%%%%%%%%%%%
\DescribeMacro{\childdocof}
Furthermore, add the commands
\begin{center}
\begin{tabular}{l}
|% \iffalse
%
% childdoc.dtx Copyright (C) 2017-2018 Niklas Beisert
%
% This work may be distributed and/or modified under the
% conditions of the LaTeX Project Public License, either version 1.3
% of this license or (at your option) any later version.
% The latest version of this license is in
%   http://www.latex-project.org/lppl.txt
% and version 1.3 or later is part of all distributions of LaTeX
% version 2005/12/01 or later.
%
% This work has the LPPL maintenance status `maintained'.
%
% The Current Maintainer of this work is Niklas Beisert.
%
% This work consists of the files childdoc.dtx and childdoc.ins
% and the derived files childdoc.def and cdocsamp.tex with
% cdocsch1.tex, cdocsch2.tex, cdocsdrf.tex, cdocsfn1.tex, cdocsfn2.tex.
%
%<package>\ifdefined\childdocmain\endinput\fi
%<package>\ProvidesFile{childdoc.def}[2018/12/30 v2.0 child document driver]
%<samplemain>\ProvidesFile{cdocsamp.tex}[2018/12/30 v2.0 sample for childdoc]
%<*driver>
%\ProvidesFile{childdoc.drv}[2018/12/30 v2.0 childdoc reference manual file]
\PassOptionsToClass{10pt,a4paper}{article}
\documentclass{ltxdoc}

\usepackage[margin=35mm]{geometry}
\usepackage{hyperref}
\usepackage{hyperxmp}
\usepackage[usenames]{color}

\hypersetup{colorlinks=true}
\hypersetup{pdfstartview=FitH}
\hypersetup{pdfpagemode=UseNone}
\hypersetup{pdfsource={}}
\hypersetup{pdflang={en-UK}}
\hypersetup{pdfcopyright={Copyright 2017-2018 Niklas Beisert.
  This work may be distributed and/or modified under the
  conditions of the LaTeX Project Public License, either version 1.3
  of this license or (at your option) any later version.}}
\hypersetup{pdflicenseurl={http://www.latex-project.org/lppl.txt}}
\hypersetup{pdfcontactaddress={ETH Zurich, ITP, HIT K,
  Wolfgang-Pauli-Strasse 27}}
\hypersetup{pdfcontactpostcode={8093}}
\hypersetup{pdfcontactcity={Zurich}}
\hypersetup{pdfcontactcountry={Switzerland}}
\hypersetup{pdfcontactemail={nbeisert@itp.phys.ethz.ch}}
\hypersetup{pdfcontacturl={http://people.phys.ethz.ch/\xmptilde nbeisert/}}

\newcommand{\secref}[1]{\hyperref[#1]{section \ref*{#1}}}

\parskip1ex
\parindent0pt
\let\olditemize\itemize
\def\itemize{\olditemize\parskip0pt}

\begin{document}

\title{The \textsf{childdoc} Package}
\hypersetup{pdftitle={The childdoc Package}}
\author{Niklas Beisert\\[2ex]
  Institut f\"ur Theoretische Physik\\
  Eidgen\"ossische Technische Hochschule Z\"urich\\
  Wolfgang-Pauli-Strasse 27, 8093 Z\"urich, Switzerland\\[1ex]
  \href{mailto:nbeisert@itp.phys.ethz.ch}
  {\texttt{nbeisert@itp.phys.ethz.ch}}}
\hypersetup{pdfauthor={Niklas Beisert}}
\hypersetup{pdfsubject={Manual for the LaTeX2e Package childdoc}}
\date{30 December 2018, \textsf{v2.0}}
\maketitle

\begin{abstract}\noindent
\textsf{childdoc} is a \LaTeXe{} package
that enables the direct compilation
of document sections included by |\include|
to individual files.
\end{abstract}

\begingroup
\parskip0ex
\tableofcontents
\endgroup

%%%%%%%%%%%%%%%%%%%%%%%%%%%%%%%%%%%%%%%%%%%%%%%%%%%%%%%%%%%%%%%%%%%%%%%%%%%%%%%%
%%%%%%%%%%%%%%%%%%%%%%%%%%%%%%%%%%%%%%%%%%%%%%%%%%%%%%%%%%%%%%%%%%%%%%%%%%%%%%%%
\section{Introduction}

\LaTeX{} provides a mechanism to structure a large document (such as a book)
into a main file and several child files (containing the chapters)
using the |\include| command.
This mechanism is beneficial for documents
which span hundreds of pages in order to
make the source file(s) more manageable.
Moreover, compilation can be restricted to
selected child files by means of the |\includeonly| command.
The latter feature can be used to reduce the compilation time while editing
(this was significantly more useful in the earlier days of \LaTeX{})
or to generate a smaller document which is easier to navigate.
Another application of |\includeonly| is to generate
documents consisting of selected parts of the complete document.

However, there are a few drawbacks of the plain |\include| mechanism:
\begin{itemize}
\item
The child files cannot be compiled on their own,
they can only be compiled via the main file.
A naive editing environment
(such as a text editor with an option
to have the current file processed by \LaTeX)
may require one to switch to the main file before compiling;
attempting to compile the child file produces errors.
\item
The main file must be modified (each time)
to adjust the |\includeonly| command
to the present needs. This easily leaves the main file in a messy state.
\item
The generated document will always carry the filename
of the main document. This is inconvenient if
several child files are to be compiled and
to be kept for distribution.
\end{itemize}

The present package provides a simple interface
to make child files individually compilable by \LaTeX{}.
Compiling a child file then has the same effect as compiling
the main file with an |\includeonly| command
to select the appropriate child.
Moreover the generated document will carry the name of the child
rather than the main file.
This resolves all three above issues.

This feature is meant to make the editing of books,
thesis documents and lecture notes somewhat more convenient.
However, the package can also be used efficiently for
composing a series of documents (such as exercise sheets)
which are typically distributed individually.
It then assists the author in generating the individual documents
(potentially in different versions)
as well as a document containing the collected series.
Another application is in developing style files
or other kinds of included material
where compilation of the style file could redirect
to a sample or test file.

%%%%%%%%%%%%%%%%%%%%%%%%%%%%%%%%%%%%%%%%%%%%%%%%%%%%%%%%%%%%%%%%%%%%%%%%%%%%%%%%
%%%%%%%%%%%%%%%%%%%%%%%%%%%%%%%%%%%%%%%%%%%%%%%%%%%%%%%%%%%%%%%%%%%%%%%%%%%%%%%%
\section{Usage}

First of all, the package \textsf{childdoc} is \emph{not} a standard
\LaTeXe{} |.sty| style file! Therefore it needs to be invoked in
a non-standard way.

%%%%%%%%%%%%%%%%%%%%%%%%%%%%%%%%%%%%%%%%%%%%%%%%%%%%%%%%%%%%%%%%%%%%%%%%%%%%%%%%
\subsection{Included Files}
\label{sec:include}

%%%%%%%%%%%%%%%%%%%%%%%%%%%%%%%%%%%%%%%%
\DescribeMacro{\childdocmain}
To use the package, add the commands
\begin{center}
\begin{tabular}{l}
|\input{childdoc.def}|\\
|\childdocmain{}|\\
\end{tabular}
\end{center}
at the very top of the main \LaTeX{} file,
in particular \emph{before} the |\documentclass| statement!
The argument of |\childdocmain| should be left empty
(but it must be present).

%%%%%%%%%%%%%%%%%%%%%%%%%%%%%%%%%%%%%%%%
\DescribeMacro{\childdocof}
Furthermore, add the commands
\begin{center}
\begin{tabular}{l}
|\input{childdoc.def}|\\
|\childdocof{|\textit{main}|}|\\
\end{tabular}
\end{center}
at the top of every child file \textit{child}
which is included by |\include{|\textit{child}|}|
from within the main file
(or at least for those files to be compiled individually).
The argument \textit{main} must be the filename of the main file.

There are a couple of
considerations in setting up the main and child documents:

%%%%%%%%%%%%%%%%%%%%%%%%%%%%%%%%%%%%%%%%
\paragraph{Restrictions.}

Please note the following restrictions:
\begin{itemize}
\item
|\childdocmain| must be called with one argument \textit{main}
to ensure compatibility with earlier version of the package.
It must either be empty (|\childdocmain{}|)
or precisely match the filename of the main file in which it is specified.
See \secref{sec:detection} for further information.
\item
The filename \textit{main} must be specified without the |.tex| extension.
\item
The filename \textit{main} is case sensitive
(even in case-insensitive file systems)
due to internal string comparison.
\item
The argument \textit{main} should be fully expanded, it cannot be a macro.
\item
Subdirectories and special characters should be avoided in filenames.
\item
The command |\childdocmain{|\textit{main}|}| must be followed by a whitespace.
It should not be followed immediately by another command
or by a comment mark `|%|'.
This is because the \TeX{} parser reads the token immediately following
the argument of |\childdocmain| and puts it
at the beginning of every child section;
however, a white\-space is ignored.
\end{itemize}

%%%%%%%%%%%%%%%%%%%%%%%%%%%%%%%%%%%%%%%%
\paragraph{Content of Main File.}

It is advisable to place all content in the child files included by |\include|.
Any output contained in the main file will appear in all child documents
unless suppressed manually;
it cannot be suppressed automatically by the |\includeonly| directive
and thus should normally be avoided.
A method to include some content in the main file
by means of conditional processing is described in \secref{sec:conditional}.

%%%%%%%%%%%%%%%%%%%%%%%%%%%%%%%%%%%%%%%%
\paragraph{Page Numbering.}

When only a part of the document is compiled,
the appropriate numbering of pages
(as well as other status parameters)
is determined from the |.aux| files.
The latter contain information from previous passes.
However this information needs to propagate through
all intermediate child documents.
Therefore the page numbering in child documents may well
be inconsistent until the complete document is compiled at least once.

A useful (if unconventional) way to always ensure a consistent
page numbering is to restart the numbering in each child document
and denote the pages by `\textit{child}|.|\textit{page}'
where \textit{child} represents the chapter/section number of the child file.
This can be achieved by the command
|\numberwithin{page}{|\textit{child}|}|
of the \textsf{amsmath} package
where \textit{child} can be |chapter| or |section|
depending on the chosen structuring.
Alternatively, one can modify the macro |\thepage| appropriately
and reset the counter |page| at the start of each child file.

%%%%%%%%%%%%%%%%%%%%%%%%%%%%%%%%%%%%%%%%%%%%%%%%%%%%%%%%%%%%%%%%%%%%%%%%%%%%%%%%
\subsection{Conditional Processing}
\label{sec:conditional}

The package provides a mechanism to compile different versions
of a document. To customise the versions further some conditional processing
can come in handy to distinguish which version is being compiled.
The package provides two macros to describe the compilation context:

%%%%%%%%%%%%%%%%%%%%%%%%%%%%%%%%%%%%%%%%
\DescribeMacro{\ifchilddoc}
The conditional |\ifchilddoc| distinguishes between the compilation of
child documents and the main document:
%
\begin{center}
|\ifchilddoc |\textit{child-code}| |[|\||else |\textit{main-code}]| \||fi|
\end{center}

%%%%%%%%%%%%%%%%%%%%%%%%%%%%%%%%%%%%%%%%
\DescribeMacro{\childdocname}
\DescribeMacro{\childdocjob}
The macro |\childdocname| contains the filename (without extension)
of the main or child file being processed.
Note that |\childdocjob| will always contain the name of the main file.

%%%%%%%%%%%%%%%%%%%%%%%%%%%%%%%%%%%%%%%%
\paragraph{Title Page.}

Conditional processing can be used to include a title or banner page
in the main document when proper precautions are taken.
Importantly, the code in the main file should ensure that the page counter
(as well as other status parameters which are stored in the |.aux| files)
takes the same value after the conditional processing.
Otherwise the page numbers may take divergent values
depending on which part is compiled.

For example, a title page could be declared by:
%
\begin{center}
\begin{tabular}{l}
|\ifchilddoc\||else|\\
|\addtocounter{page}{-1}|\\
\textit{code for title page}\\
|\newpage|\\
|\||fi|
\end{tabular}
\end{center}
%
A banner page for the child documents can be generated by:
%
\begin{center}
\begin{tabular}{l}
|\ifchilddoc|\\
|\addtocounter{page}{-1}|\\
\textit{code for banner page}\\
|\newpage|\\
|\||fi|
\end{tabular}
\end{center}
%
Here one could write a message such as:
\begin{center}
|This is the part \childdocname{} of \childdocjob{}.|
\end{center}

%%%%%%%%%%%%%%%%%%%%%%%%%%%%%%%%%%%%%%%%%%%%%%%%%%%%%%%%%%%%%%%%%%%%%%%%%%%%%%%%
\subsection{Flags}
\label{sec:flags}

The package makes it easy to generate different versions
of the main or child documents.
To this end compilation flags can be defined
and assigned different default values.
They will be particularly useful in conjunction
with the forwarding mechanism described in \secref{sec:forward}.

For example, it may be useful to have a flag |\version|
which can be set to |draft| or |final|.
The document source will contain some conditional code
depending on the value of |\version|.
Suppose further, the flag should default to |final| for the main file
and to |draft| for child files
which is a natural assignment for editing the document.
This is achieved by placing the following code
in the preamble of the main document
(below the |\childdocmain| directive):
%
\begin{center}
\begin{tabular}{l}
|\ifchilddoc|\\
|\providecommand{\version}{draft}|\\
|\||else|\\
|\providecommand{\version}{final}|\\
|\||fi|
\end{tabular}
\end{center}
%
The definition by |\providecommand| makes sure
that previous definitions are not overwritten.
Further statements |\providecommand{\version}{...}|
can thus be added before the above code to override it.

For the main file, one might add a line
(between |\childdocmain| and the above block)
%
\begin{center}
|%\ifchilddoc\||else\providecommand{\version}{draft}\||fi|
\end{center}
%
which can be uncommented to produce a draft version.
Likewise one can add a line to the very top of a child file
(above the |\childdocof{|\textit{main}|}| directive)
%
\begin{center}
|%\providecommand{\version}{final}|
\end{center}
%
which can be uncommented to produce the final version of this child document.

%%%%%%%%%%%%%%%%%%%%%%%%%%%%%%%%%%%%%%%%%%%%%%%%%%%%%%%%%%%%%%%%%%%%%%%%%%%%%%%%
\subsection{Forwarding}
\label{sec:forward}

Different versions of the main or child documents
using compilation flags as described in \secref{sec:flags}
can be (permanently) stored in different files
for convenient compilation, viewing and distribution.
To this end, the package defines a command
to pass on compilation to a different file:

%%%%%%%%%%%%%%%%%%%%%%%%%%%%%%%%%%%%%%%%
\DescribeMacro{\childdocforward}
The command |\childdocforward| redirects processing to
another source file:
%
\begin{center}
\begin{tabular}{l}
|\input{childdoc.def}|\\
|\childdocforward[|\textit{main}|]{|\textit{dest}|}|\\
\end{tabular}
\end{center}
%
The argument \textit{dest} is the destination file
(without extension).
It should be the main file or one of the child files.
Note that further \textsf{childdoc} directives
such as |\childdocof| and |\childdocforward|
in the indicated file will be processed in this form.
The optional argument \textit{main}
passes on directly to the main file \textit{main}
while pretending to compile the child \textit{dest}.
This form behaves as if \textit{dest}
issues |\childdocof{|\textit{main}|}| right away,
and no further \textsf{childdoc} directives will be processed.

%%%%%%%%%%%%%%%%%%%%%%%%%%%%%%%%%%%%%%%%
\DescribeMacro{\...prefix}
In the alternative form |\childdocforwardprefix|,
%
\begin{center}
\begin{tabular}{l}
|\input{childdoc.def}|\\
|\childdocforwardprefix[|\textit{main}|]{|\textit{prefix}|}{|\textit{dest}|}|
\end{tabular}
\end{center}
%
the destination file is determined by a pattern
depending on the current file:
To make this work, the current file must be called
`{\textit{prefix}\hspace{0.2em}\textit{suffix}}'
with \textit{prefix} matching precisely the argument.
Processing is then passed on to the file
`{\textit{dest}\hspace{0.2em}\textit{suffix}}'.
Surely, the same effect is achieved by
directly specifying the
argument `{\textit{dest}\hspace{0.2em}\textit{suffix}}'
in the first form.
However, that requires to set up a different file
for each child. With the alternative form of the command
all these files can have exactly the same content
which simplifies setting them up and maintaining them.

For example, the following file |draft.tex|
with a compilation flag |\version| as described in \secref{sec:flags}
compiles the main document as a draft:
%
\begin{center}
\begin{tabular}{l}
|\def\version{draft}|\\
|\input{childdoc.def}|\\
|\childdocforward{|\textit{main}|}|
\end{tabular}
\end{center}
%
Likewise, the following files |final|\textit{nn}|.tex|
compile the final version of the child document
|child|\textit{nn}|.tex|:
%
\begin{center}
\begin{tabular}{l}
|\def\version{final}|\\
|\input{childdoc.def}|\\
|\childdocforwardprefix{final}{child}|
\end{tabular}
\end{center}
%

Note that when several versions of a main file and/or of each child file
are to be generated, it may be convenient to set up a |Makefile| or
shell script to automatise the process.

%%%%%%%%%%%%%%%%%%%%%%%%%%%%%%%%%%%%%%%%%%%%%%%%%%%%%%%%%%%%%%%%%%%%%%%%%%%%%%%%
\subsection{Command Line Processing}
\label{sec:commandline}

The effect of redirection files can also be achieved by invoking
the \LaTeX{} compiler with a more elaborate command line.
Most conveniently this should be done as part
of a shell script or a |Makefile|.

When using \textsf{childdoc} in the main file, the following
command lines effectively perform a redirection
(note that depending on the shell being used,
backslashes may have to be doubled: `|\|' $\to$ `|\\|'):
%
\begin{center}
|... -jobname "|\textit{target}|" |\\|"|[\textit{flags}]%
|\input{childdoc.def}\childdocforward[|\textit{main}|]{|\textit{dest}|}"|
\end{center}
%
Here \textit{target} is the name of the output file,
\textit{main} is the name of the main file
and \textit{dest} is the name of the main or child file to be processed
(all filenames without extensions).
The optional argument \textit{main} can be omitted
if \textit{main} matches \textit{dest}.
Optionally, compilation \textit{flags} can be defined via |\def| commands.
This command line makes the \TeX{} engine believe
it is compiling the file \textit{target}
whose content is specified as the latter parameter.
The provided code then forwards the processing to
\textit{main} or \textit{dest} as described in \secref{sec:forward}.

%%%%%%%%%%%%%%%%%%%%%%%%%%%%%%%%%%%%%%%%%%%%%%%%%%%%%%%%%%%%%%%%%%%%%%%%%%%%%%%%
\subsection{Include by Input}
\label{sec:input}

Including child documents by |\include| has some restrictions by design.
Most notably, the content of a child document always occupies
its own set of pages; pages cannot be shared between child documents.
Usually, this behaviour makes perfect sense
because each child document contain an essential part of the document.
However, in some situations it may be desirable to compose
a document from a collection of parts
without having mandatory page breaks between then.
For this case, the package
provides a mechanism to include parts
by |\input| which can also be processed individually.
However, by construction this mechanism
requires manual handling of the content to be output.

%%%%%%%%%%%%%%%%%%%%%%%%%%%%%%%%%%%%%%%%
\DescribeMacro{\ifchilddocmanual}
The main file should be prepared as usual, see \secref{sec:include}.
However, the document body must make a distinction
between processing of an individual part and of the main document, e.g.:
%
\begin{center}
\begin{tabular}{l}
|\ifchilddocmanual|\\
|\input{\childdocname}|\\
|\||else|\\
\textit{document body with }|\input{|\textit{part}|}|\\
|\||fi|
\end{tabular}
\end{center}
%
The conditional |\ifchilddocmanual| is true whenever
a part to be included by |\input| is being compiled,
and the name of the part is stored in |\childdocname|.

%%%%%%%%%%%%%%%%%%%%%%%%%%%%%%%%%%%%%%%%
\DescribeMacro{\childdocby}
Each part to be included by |\input| should start with:
%
\begin{center}
\begin{tabular}{l}
|\input{childdoc.def}|\\
|\childdocby{|\textit{main}|}|\\
\end{tabular}
\end{center}
%
The directive |\childdocby| is similar to |\childdocof|
described in \secref{sec:include},
but the subsequent selection of content must be done manually.
To that end, both |\ifchilddoc| and |\ifchilddocmanual|
will be true upon processing of a part,
and the name of the part is stored in |\childdocname|.
Note that |\jobname| will be set to the filename of the current part
so that each part receives an individual |.aux| file
that does not interfere with the |.aux| file(s) of the main document.
This behaviour can be altered by the alternative form
|\childdocby[*]{|\textit{main}|}| (with a non-empty optional argument)
which uses the |.aux| file of the main document
by setting |\jobname| to \textit{main}.

%%%%%%%%%%%%%%%%%%%%%%%%%%%%%%%%%%%%%%%%%%%%%%%%%%%%%%%%%%%%%%%%%%%%%%%%%%%%%%%%
\subsection{Driver Development}
\label{sec:driver}

The \textsf{childdoc} mechanism can also be use for the development
of definition files such as \LaTeX{} styles or classes.
This case differs from the above setup with multiple parts
included by |\include| in that no |\includeonly| should be invoked.
This can be achieved by starting the include file
(before |\ProvidesPackage|) with:
%
\begin{center}
\begin{tabular}{l}
|\input{childdoc.def}|\\
|\childdocforward{|\textit{main}|}|\\
\end{tabular}
\end{center}
%
or alternatively with:
%
\begin{center}
\begin{tabular}{l}
|\input{childdoc.def}|\\
|\childdocby{|\textit{main}|}|\\
\end{tabular}
\end{center}
%
Both forms have slightly different effects as described above.
The main file is prepared as usual, see \secref{sec:include}.

%%%%%%%%%%%%%%%%%%%%%%%%%%%%%%%%%%%%%%%%%%%%%%%%%%%%%%%%%%%%%%%%%%%%%%%%%%%%%%%%
\subsection{Legacy Detection}
\label{sec:detection}

The directive |\childdocmain| in the main file can detect
whether the complete document or merely a child is to be compiled
even without using the directive |\childdocof|.
This method is deprecated because it is less robust
and there is no compelling reason to use it;
it is merely provided for backward compatibility
and it may be removed in future versions.

If the detection mechanism is to be used,
it is mandatory to correctly specify
the filename of the main file as the argument of |\childdocmain|:
%
\begin{center}
\begin{tabular}{l}
|\input{childdoc.def}|\\
|\childdocmain{|\textit{main}|}|\\
\end{tabular}
\end{center}
%
If |\jobname| does not match the argument \textit{main} of |\childdocmain|,
it is assumed that |\jobname| points to the child file to be compiled.
When using |\childdocmain| with the main file specified as argument,
it suffices to start a child file
with just |\input{|\textit{main}|}|
without loading of the package and using |\childdocof|.
If instead all processing is done
with the appropriate \textsf{childdoc} directives,
the argument of \textit{main} of |\childdocmain| can be empty.

An alternative version of the command line processing described
in \secref{sec:commandline} using the detection mechanism reads:
%
\begin{center}
|... -jobname "|\textit{target}|" "|[\textit{flags}]%
[|\def\jobname{|\textit{dest}|}|]|\input{|\textit{main}|}"|
\end{center}

%%%%%%%%%%%%%%%%%%%%%%%%%%%%%%%%%%%%%%%%%%%%%%%%%%%%%%%%%%%%%%%%%%%%%%%%%%%%%%%%
\subsection{Manual Code}
\label{sec:manual}

In case one cannot be certain whether the definitions file |childdoc.def|
is installed on the target \TeX{} distribution
and one prefers not to ship it,
it is conceivable to paste a few relevant commands into the sources.

To that end, drop all statements |\input{childdoc.def}|
and perform the replacements as outlined below.
Instead of |\childdocmain{|\textit{main}|}| add the following code
to the top of the main file:
%
\begin{center}
\begin{tabular}{l}
|\||ifdefined\childdocname\endinput\||fi\newif\ifchilddoc|\\
|\edef\childdocname{\scantokens\expandafter{\jobname\noexpand}}|\\
|\def\childdocmain{|\textit{main}|}\||ifx\childdocmain\childdocname\||else|\\
|\childdoctrue\includeonly{\childdocname}\let\jobname\childdocmain\||fi|\\
\end{tabular}
\end{center}
%
Instead of |\childdocof{|\textit{main}|}| just include the main file
at the top of each child file:
%
\begin{center}
|\input{|\textit{main}|}|
\end{center}
%
A simple redirection |\childdocforward{|\textit{dest}|}| is achieved by:
%
\begin{center}
|\def\jobname{|\textit{dest}|}\input{\jobname}|
\end{center}
%
The redirection with prefix
|\childdocforwardprefix[|\textit{prefix}|]{|\textit{dest}|}|
is accomplished by:
%
\begin{center}
\begin{tabular}{l}
|{\edef\jobname{\scantokens\expandafter{\jobname\noexpand}}|\\
|\def\redirectjob |\textit{prefix}|#1~~~{\gdef\jobname{|\textit{dest}|#1}}|\\
|\expandafter\redirectjob\jobname~~~}\input{\jobname}|
\end{tabular}
\end{center}

In an alternative approach,
child documents can be compiled by a specific command line
without additional code or specific definitions:
%
\begin{center}
|... -jobname "|\textit{target}|" "|[\textit{flags}]%
|\includeonly{|\textit{dest}|}\input{|\textit{main}|}"|
\end{center}
%

%%%%%%%%%%%%%%%%%%%%%%%%%%%%%%%%%%%%%%%%%%%%%%%%%%%%%%%%%%%%%%%%%%%%%%%%%%%%%%%%
%%%%%%%%%%%%%%%%%%%%%%%%%%%%%%%%%%%%%%%%%%%%%%%%%%%%%%%%%%%%%%%%%%%%%%%%%%%%%%%%
\section{Information}

%%%%%%%%%%%%%%%%%%%%%%%%%%%%%%%%%%%%%%%%%%%%%%%%%%%%%%%%%%%%%%%%%%%%%%%%%%%%%%%%
\subsection{Copyright}

Copyright \copyright{} 2017--2018 Niklas Beisert

This work may be distributed and/or modified under the
conditions of the \LaTeX{} Project Public License, either version 1.3
of this license or (at your option) any later version.
The latest version of this license is in
  \url{http://www.latex-project.org/lppl.txt}
and version 1.3 or later is part of all distributions of \LaTeX{}
version 2005/12/01 or later.

This work has the LPPL maintenance status `maintained'.

The Current Maintainer of this work is Niklas Beisert.

This work consists of the files |README.txt|, |childdoc.ins| and |childdoc.dtx|
as well as the derived files |childdoc.def|, |cdocsamp.tex|
with |cdocsch1.tex|, |cdocsch2.tex|, |cdocspt3.tex|, |cdocspt4.tex|,
|cdocsdrf.tex|, |cdocsfn1.tex|, |cdocsfn2.tex|
as well as |childdoc.pdf|.

%%%%%%%%%%%%%%%%%%%%%%%%%%%%%%%%%%%%%%%%%%%%%%%%%%%%%%%%%%%%%%%%%%%%%%%%%%%%%%%%
\subsection{Files and Installation}

The package consists of the files:
%
\begin{center}
\begin{tabular}{ll}
    |README.txt|   & readme file \\
    |childdoc.ins| & installation file \\
    |childdoc.dtx| & source file \\
    |childdoc.def| & definition file \\
    |cdocsamp.tex| & sample main file \\
    |cdocsch1.tex| & sample include file \\
    |cdocsch2.tex| & sample include file \\
    |cdocspt3.tex| & sample part file \\
    |cdocspt4.tex| & sample part file \\
    |cdocsdrf.tex| & sample redirection file \\
    |cdocsfn1.tex| & sample redirection file \\
    |cdocsfn2.tex| & sample redirection file \\
    |childdoc.pdf| & manual
\end{tabular}
\end{center}
%
The distribution consists of the files
|README.txt|, |childdoc.ins| and |childdoc.dtx|.
%
\begin{itemize}
\item
Run (pdf)\LaTeX{} on |childdoc.dtx|
to compile the manual |childdoc.pdf| (this file).
\item
Run \LaTeX{} on |childdoc.ins| to create the definitions file |childdoc.def|
and the sample |cdocsamp.tex| with include files
|cdocsch1.tex|, |cdocsch2.tex|, |cdocspt3.tex|, |cdocspt4.tex|,
|cdocsdrf.tex|, |cdocsfn1.tex|, |cdocsfn2.tex|.
Then copy the file |childdoc.def| to an appropriate directory of your \LaTeX{}
distribution, e.g.\ \textit{texmf-root}|/tex/latex/childdoc|.
\end{itemize}

%%%%%%%%%%%%%%%%%%%%%%%%%%%%%%%%%%%%%%%%%%%%%%%%%%%%%%%%%%%%%%%%%%%%%%%%%%%%%%%%
\subsection{Related CTAN Packages}

There are several other packages which offer a similar functionality:
%
\begin{itemize}
\item
The packages
\href{http://ctan.org/pkg/docmute}{\textsf{docmute}},
\href{http://ctan.org/pkg/includex}{\textsf{includex}} and
\href{http://ctan.org/pkg/standalone}{\textsf{standalone}}
provide commands to include only the document body of
a child file thus allowing both files to be compiled individually.
\item
The packages \href{http://ctan.org/pkg/subdocs}{\textsf{subdocs}}
and \href{http://ctan.org/pkg/subfiles}{\textsf{subfiles}}
provide structures in which the main and child documents can be
encapsulated and allowing them to be compiled individually.
The inclusion mechanism is different from the conventional |\include|.
\item
The package \href{http://ctan.org/pkg/combine}{\textsf{combine}}
is an elaborate solution to combine several documents into one.
\end{itemize}
%
See also the CTAN topic \href{http://ctan.org/topic/subdocs}{\textsf{subdocs}}
for further related packages.
The present package differs from the above solutions in that
a document structure constructed with the conventional |\include| mechanism
just needs two extra commands at the top of every file
such that all constituent files can be compiled individually.

%%%%%%%%%%%%%%%%%%%%%%%%%%%%%%%%%%%%%%%%%%%%%%%%%%%%%%%%%%%%%%%%%%%%%%%%%%%%%%%%
%\subsection{Feature Suggestions}
%
%The following is a list of features which may be useful for future
%versions of this package:
%%
%\begin{itemize}
%\item
%\ldots
%\end{itemize}

%%%%%%%%%%%%%%%%%%%%%%%%%%%%%%%%%%%%%%%%%%%%%%%%%%%%%%%%%%%%%%%%%%%%%%%%%%%%%%%%
\subsection{Revision History}

%%%%%%%%%%%%%%%%%%%%%%%%%%%%%%%%%%%%%%%%
\paragraph{v2.0:} 2018/12/30

\begin{itemize}
\item
immediate forward processing
\item
added |\childdocby| mechanism
\item
manual restructured
\end{itemize}

%%%%%%%%%%%%%%%%%%%%%%%%%%%%%%%%%%%%%%%%
\paragraph{v1.6:} 2018/01/17

\begin{itemize}
\item
application for development of include files
\item
corrections to manual
\end{itemize}

%%%%%%%%%%%%%%%%%%%%%%%%%%%%%%%%%%%%%%%%
\paragraph{v1.5:} 2017/05/21

\begin{itemize}
\item
more complete structuring introduced
\item
|\childdocof| introduced
\item
|\childdoc| renamed to |\childdocmain|
\item
|\childredirect| renamed to |\childdocforward| and |\childdocforwardprefix|
and functionality expanded
\end{itemize}

%%%%%%%%%%%%%%%%%%%%%%%%%%%%%%%%%%%%%%%%
\paragraph{v1.0:} 2017/04/27

\begin{itemize}
\item
manual and install package
\item
first version published on CTAN
\end{itemize}

%%%%%%%%%%%%%%%%%%%%%%%%%%%%%%%%%%%%%%%%
\paragraph{v0.6:} 2017/04/26

\begin{itemize}
\item
redirection mechanism added
\end{itemize}

%%%%%%%%%%%%%%%%%%%%%%%%%%%%%%%%%%%%%%%%
\paragraph{v0.5:} 2017/04/26

\begin{itemize}
\item
functionality in definition file
\end{itemize}


%%%%%%%%%%%%%%%%%%%%%%%%%%%%%%%%%%%%%%%%%%%%%%%%%%%%%%%%%%%%%%%%%%%%%%%%%%%%%%%%
%%%%%%%%%%%%%%%%%%%%%%%%%%%%%%%%%%%%%%%%%%%%%%%%%%%%%%%%%%%%%%%%%%%%%%%%%%%%%%%%
%%%%%%%%%%%%%%%%%%%%%%%%%%%%%%%%%%%%%%%%%%%%%%%%%%%%%%%%%%%%%%%%%%%%%%%%%%%%%%%%
\appendix

\settowidth\MacroIndent{\rmfamily\scriptsize 000\ }

 \DocInput{childdoc.dtx}

\end{document}
%</driver>
% \fi
%
% %%%%%%%%%%%%%%%%%%%%%%%%%%%%%%%%%%%%%%%%%%%%%%%%%%%%%%%%%%%%%%%%%%%%%%%%%%%%%%
% %%%%%%%%%%%%%%%%%%%%%%%%%%%%%%%%%%%%%%%%%%%%%%%%%%%%%%%%%%%%%%%%%%%%%%%%%%%%%%
% \section{Sample}
%\iffalse
%<*samplemain>
%\fi
%
% The following presents a sample document
% with two chapters, two parts, a title page,
% a compile flag as well as three forwarding files to set the flag.
% It consists of eight |.tex| files:
% \begin{center}
% \begin{tabular}{ll}
% |cdocsamp.tex|&main file\\
% |cdocsch1.tex|&include file for chapter 1\\
% |cdocsch2.tex|&include file for chapter 2\\
% |cdocspt3.tex|&include file for part 3\\
% |cdocspt4.tex|&include file for part 4\\
% |cdocsdrf.tex|&forwarding file for main file in draft mode\\
% |cdocsfi1.tex|&forwarding file for final version of chapter 1\\
% |cdocsfi2.tex|&forwarding file for final version of chapter 2\\
% \end{tabular}
% \end{center}
% Each of the eight files can be compiled directly by the \LaTeX{} compiler.
%
% %%%%%%%%%%%%%%%%%%%%%%%%%%%%%%%%%%%%%%
% \paragraph{Main File.}
%
% The main file is called |cdocsamp.tex|.
%
% Load the \textsf{childdoc} definitions and
% declare the filename for the main document:
%    \begin{macrocode}
\input{childdoc.def}
\childdocmain{}
%    \end{macrocode}

% Optional override for |\version| flag:
%    \begin{macrocode}
%%\ifchilddoc\else\providecommand{\version}{draft}\fi
%    \end{macrocode}

% Define the default values for the |\version| flag
% (|final| for the main file and |draft| for childs):
%    \begin{macrocode}
\ifchilddoc
\providecommand{\version}{draft}
\else
\providecommand{\version}{final}
\fi
%    \end{macrocode}

% Load the standard document class:
%    \begin{macrocode}
\documentclass[12pt]{article}
%    \end{macrocode}

% Start the document body:
%    \begin{macrocode}
\begin{document}
%    \end{macrocode}

% Declare a title page.
% Print title, part of document being processed and version flag:
%    \begin{macrocode}
\addtocounter{page}{-1}
\begin{center}
{\LARGE\bfseries{}childdoc example\par}
\vspace{1cm}
\ifchilddoc
\ifchilddocmanual part\else chapter\fi:
`\childdocname' of `\childdocjob'\par
\else
main document: `\childdocjob'\par
\fi
version: \version\par
\end{center}
\newpage
%    \end{macrocode}

% Manually include selected file,
% otherwise process as usual:
%    \begin{macrocode}
\ifchilddocmanual
\section*{part `\childdocname'}
\input{\childdocname}
\else
%    \end{macrocode}

% Include the two chapters:
%    \begin{macrocode}
\include{cdocsch1}
\include{cdocsch2}
%    \end{macrocode}

% Include the two parts unless only chapters should be displayed:
%    \begin{macrocode}
\ifchilddoc\else
\section{part three}
\input{cdocspt3}
\section{part four}
\input{cdocspt4}
\fi
%    \end{macrocode}

% Process as usual until here:
%    \begin{macrocode}
\fi
%    \end{macrocode}

% End of document body:
%    \begin{macrocode}
\end{document}
%    \end{macrocode}
%\iffalse
%</samplemain>
%\fi
%
% %%%%%%%%%%%%%%%%%%%%%%%%%%%%%%%%%%%%%%
% \paragraph{Chapter Include Files.}
%
% The include files are called |cdocsch1.tex| and |cdocsch2.tex|.
%
%\iffalse
%<*samplechap1|samplechap2>
%\fi

% Optional override for |\version| flag:
%    \begin{macrocode}
%%\providecommand{\version}{final}
%    \end{macrocode}

% Include the main document:
%    \begin{macrocode}
\input{childdoc.def}
\childdocof{cdocsamp}
%    \end{macrocode}

%\iffalse
%</samplechap1|samplechap2>
%\fi
%
%\iffalse
%<*samplechap1>
%\fi
% Some text for chapter 1:
%    \begin{macrocode}
\section{one}
some text in chapter one
%    \end{macrocode}

%\iffalse
%</samplechap1>
%\fi
% Some text for chapter 2:
%\iffalse
%<*samplechap2>
%\fi
%    \begin{macrocode}
\section{two}
more text in chapter two
%    \end{macrocode}

%\iffalse
%</samplechap2>
%\fi
%
% %%%%%%%%%%%%%%%%%%%%%%%%%%%%%%%%%%%%%%
% \paragraph{Part Include Files.}
%
% The include files are called |cdocspt3.tex| and |cdocspt4.tex|.
%
%\iffalse
%<*samplepart3|samplepart4>
%\fi

% Optional override for |\version| flag:
%    \begin{macrocode}
%%\providecommand{\version}{final}
%    \end{macrocode}

% Include the main document:
%    \begin{macrocode}
\input{childdoc.def}
\childdocby{cdocsamp}
%    \end{macrocode}

%\iffalse
%</samplepart3|samplepart4>
%\fi
%
%\iffalse
%<*samplepart3>
%\fi
% Some text for part 3:
%    \begin{macrocode}
some text in part three
%    \end{macrocode}

%\iffalse
%</samplepart3>
%\fi
% Some text for part 4:
%\iffalse
%<*samplepart4>
%\fi
%    \begin{macrocode}
more text in part four
%    \end{macrocode}

%\iffalse
%</samplepart4>
%\fi
%
% %%%%%%%%%%%%%%%%%%%%%%%%%%%%%%%%%%%%%%
% \paragraph{Forwarding for a Complete Draft.}
%
% The following forwarding file |cdocsdrf.tex|
% compiles the main document in draft mode:
%\iffalse
%<*sampledraft>
%\fi
%    \begin{macrocode}
\def\version{draft}
\input{childdoc.def}
\childdocforward{cdocsamp}
%    \end{macrocode}

%\iffalse
%</sampledraft>
%\fi
%
% %%%%%%%%%%%%%%%%%%%%%%%%%%%%%%%%%%%%%%
% \paragraph{Forwarding for Final Version of the Chapters.}
%
% The following forwarding files |cdocsfn1.tex| and |cdocsfn2.tex|
% (with identical content)
% compile the final versions of the child documents
% |cdocsch1.tex| and |cdocsch2.tex|, respectively:
%\iffalse
%<*samplefinal>
%\fi
%    \begin{macrocode}
\def\version{final}
\input{childdoc.def}
\childdocforwardprefix[cdocsamp]{cdocsfn}{cdocsch}
%    \end{macrocode}

%\iffalse
%</samplefinal>
%\fi
%
% %%%%%%%%%%%%%%%%%%%%%%%%%%%%%%%%%%%%%%
% \paragraph{Command Line Processing.}
%
% The following three command lines generate the output files
% |cdocscld|, |cdocscl1| and |cdocscl2|
% which should be identical to
% |cdocsdrf|, |cdocsch1| and |cdocsfn2|, respectively:
% \begin{center}
% \begin{tabular}{l}
% |latex -jobname cdocscld \|\\
% |  "\def\version{draft}\input{childdoc.def}\childdocforward{cdocsamp}"|\\
% |latex -jobname cdocscl1 \|\\
% |  "\input{childdoc.def}\childdocforward[cdocsamp]{cdocsch1}"|\\
% |latex -jobname cdocscl2 \|\\
% |  "\def\version{final}\input{childdoc.def}\childdocforward{cdocsch2}"|
% \end{tabular}
% \end{center}
% Note that the trailing backslash on each first line
% merely continues the input to the second line
% (for convenient cut ant paste).
% Furthermore, the command |latex| can be replaced by any
% of its alternative versions such as |pdflatex|.
%
% %%%%%%%%%%%%%%%%%%%%%%%%%%%%%%%%%%%%%%%%%%%%%%%%%%%%%%%%%%%%%%%%%%%%%%%%%%%%%%
% %%%%%%%%%%%%%%%%%%%%%%%%%%%%%%%%%%%%%%%%%%%%%%%%%%%%%%%%%%%%%%%%%%%%%%%%%%%%%%
% \section{Implementation}
%\iffalse
%<*package>
%\fi
%
% This section describes the definitions file |childdoc.def|.

% The definitions cannot be loaded using |\usepackage| or |\RequirePackage|
% which has a mechanism to prevent loading a style file more than once.
% When loading the definitions by means of |\input|
% multiple instances have to be prevented manually:
%\iffalse
%This code needs to be before the `\ProvidesFile' directive
%which is defined at the beginning of this file.
%Therefore it is also placed there and commented out here.
%</package>
%<*discard>
%\fi
%    \begin{macrocode}
\ifdefined\childdocmain\endinput\fi
%    \end{macrocode}
%\iffalse
%</discard>
%<*package>
%\fi
%
% \macro{\ifchilddoc}
% \macro{\ifchilddocmanual}
% The conditional |\ifchilddoc| tells whether a
% child (true) or main (false) document is being compiled.
% The conditional |\ifchilddocmanual| tells whether
% the |\includeonly| mechanism is used (false) or
% the selection of child files must be performed manually (true).
% The definitions initialise to false:
%    \begin{macrocode}
\newif\ifchilddoc
\newif\ifchilddocmanual
%    \end{macrocode}

% \macro{\childdocname}
% \macro{\childdocjob}
% The macro |\childdocname| stores the name of the main document
% to be compiled. The macro |\childdocjob| stores the name of
% the document on which the \LaTeX{} compiler was originally invoked.
% The content of |\jobname| cannot be compared
% to filenames specified in the source due to different catcodes.
% The following code rescans |\jobname|, stores the result
% in |\childdocname| and saves a copy in |\childdocjob|:
%    \begin{macrocode}
\edef\childdocname{\scantokens\expandafter{\jobname\noexpand}}
\let\childdocjob\childdocname
%    \end{macrocode}

% \macro{\childdocdisable}
% The macro |\childdocdisable| prevents the main file
% from being processed more than once.
% At this stage, the main document command |\childdocmain|
% is assumed to be called once again where it should do nothing.
% Any subsequent call to it should prevent
% a secondary processing of the main document
% It overwrites the forwarding commands
% |\childdocof| and |\childdocforward|
% with empty macros to prevent further inclusions of the main document:
%    \begin{macrocode}
\newcommand{\childdocdisable}
{
  \renewcommand{\childdocmain}[1]{\renewcommand{\childdocmain}[1]{\endinput}}
  \renewcommand{\childdocof}[1]{}
  \renewcommand{\childdocby}[2][]{}
  \renewcommand{\childdocforward}[2][]{}
  \renewcommand{\childdocdisable}{}
}
%    \end{macrocode}

% \macro{\childdocmain}
% The macro |\childdocmain| is to be called at the top of the main file
% with nothing or the main filename (without extension) as argument.
% First, it breaks loops.
% If the argument is not empty and does not match |\childdocname|
% (which is set by the first inclusion of |childdoc.def|),
% |\ifchilddoc| is set to true, |\includeonly| is applied to the child file
% and |\jobname| is set to the main file
% (for proper handling of |.aux| files):
%    \begin{macrocode}
\newcommand{\childdocmain}[1]
{
  \childdocdisable\childdocmain{}
  \if?#1?\else
    \begingroup
      \def\childdoctmp{#1}
      \ifx\childdoctmp\childdocname
        \def\childdoctmp{}
      \else
        \def\childdoctmp
        {
          \childdoctrue
          \includeonly{\childdocname}
          \def\childdocjob{#1}
          \def\jobname{#1}
        }
      \fi
      \expandafter
    \endgroup
    \childdoctmp
  \fi
}
%    \end{macrocode}

% \macro{\childdocof}
% The command |\childdocof| redirects
% compilation to the main file |#1|.
%    \begin{macrocode}
\newcommand{\childdocof}[1]
{
  \childdocdisable
  \childdoctrue
  \includeonly{\childdocname}
  \def\jobname{#1}
  \def\childdocjob{#1}
  \input{#1}
}
%    \end{macrocode}

% \macro{\childdocby}
% The command |\childdocby| ....
%    \begin{macrocode}
\newcommand{\childdocby}[2][]
{
  \childdocdisable
  \childdoctrue
  \childdocmanualtrue
  \if?#1?\else
    \def\jobname{#2}
  \fi
  \def\childdocjob{#2}
  \input{#2}
  \endinput
}
%    \end{macrocode}

% \macro{\childdocforward}
% The command |\childdocforward| redirects
% compilation to the main file or
% (if the optional argument is given) a child file.
% Parameters are set as if the main file
% or a child file starting with |\childdocof| was compiled.
% Then compilation is handed over to the main file:
%    \begin{macrocode}
\newcommand{\childdocforward}[2][]
{
  \begingroup
    \if?#1?
      \def\childdoctmp
      {
        \def\childdocname{#2}
        \def\childdocjob{#2}
        \def\jobname{#2}
        \input{#2}
        \endinput
      }
    \else
      \def\childdoctmp
      {
        \childdocdisable
        \def\childdocname{#2}
        \childdoctrue
        \includeonly{#2}
        \def\childdocjob{#1}
        \def\jobname{#1}
        \input{#1}
        \endinput
      }
    \fi
    \expandafter
  \endgroup
  \childdoctmp
}
%    \end{macrocode}

% \macro{\childdocforwardprefix}
% The command |\childdocforwardprefix| redirects
% compilation to the main or a child file by means of a pattern.
% The prefix |#1| in the current filename is replaced by |#2|
% and the suffix of the current filename is kept
% (it is assumed that the filename does not contain the substring `|~~~|'
% which is used as a delimiter).
% Compilation is handed over to the new file by |\childdocforward|:
%    \begin{macrocode}
\newcommand{\childdocforwardprefix}[3][]
{
  \begingroup
    \def\childdocextract #2##1~~~{\def\childdoctmp{\childdocforward[#1]{#3##1}}}
    \expandafter\childdocextract\childdocname~~~
    \expandafter
  \endgroup
  \childdoctmp
}
%    \end{macrocode}

% \macro{\childdoc}
% The deprecated macro |\childdoc| is a legacy version of |\childdocmain|:
%    \begin{macrocode}
\newcommand{\childdoc}{\childdocmain}
%    \end{macrocode}

% \macro{\childdocredirect}
% The deprecated macro |\childdocredirect| is a legacy version
% of |\childdocforward| and |\childdocforwardprefix|:
%    \begin{macrocode}
\newcommand{\childdocredirect}[2][]
{
  \begingroup
    \if?#1?
      \def\childdoctmp{\childdocforward{#2}}
    \else
      \def\childdoctmp{\childdocforwardprefix{#1}{#2}}
    \fi
    \expandafter
  \endgroup
  \childdoctmp
}
%    \end{macrocode}

%\iffalse
%</package>
%\fi
%
\endinput
|\\
|\childdocof{|\textit{main}|}|\\
\end{tabular}
\end{center}
at the top of every child file \textit{child}
which is included by |\include{|\textit{child}|}|
from within the main file
(or at least for those files to be compiled individually).
The argument \textit{main} must be the filename of the main file.

There are a couple of
considerations in setting up the main and child documents:

%%%%%%%%%%%%%%%%%%%%%%%%%%%%%%%%%%%%%%%%
\paragraph{Restrictions.}

Please note the following restrictions:
\begin{itemize}
\item
|\childdocmain| must be called with one argument \textit{main}
to ensure compatibility with earlier version of the package.
It must either be empty (|\childdocmain{}|)
or precisely match the filename of the main file in which it is specified.
See \secref{sec:detection} for further information.
\item
The filename \textit{main} must be specified without the |.tex| extension.
\item
The filename \textit{main} is case sensitive
(even in case-insensitive file systems)
due to internal string comparison.
\item
The argument \textit{main} should be fully expanded, it cannot be a macro.
\item
Subdirectories and special characters should be avoided in filenames.
\item
The command |\childdocmain{|\textit{main}|}| must be followed by a whitespace.
It should not be followed immediately by another command
or by a comment mark `|%|'.
This is because the \TeX{} parser reads the token immediately following
the argument of |\childdocmain| and puts it
at the beginning of every child section;
however, a white\-space is ignored.
\end{itemize}

%%%%%%%%%%%%%%%%%%%%%%%%%%%%%%%%%%%%%%%%
\paragraph{Content of Main File.}

It is advisable to place all content in the child files included by |\include|.
Any output contained in the main file will appear in all child documents
unless suppressed manually;
it cannot be suppressed automatically by the |\includeonly| directive
and thus should normally be avoided.
A method to include some content in the main file
by means of conditional processing is described in \secref{sec:conditional}.

%%%%%%%%%%%%%%%%%%%%%%%%%%%%%%%%%%%%%%%%
\paragraph{Page Numbering.}

When only a part of the document is compiled,
the appropriate numbering of pages
(as well as other status parameters)
is determined from the |.aux| files.
The latter contain information from previous passes.
However this information needs to propagate through
all intermediate child documents.
Therefore the page numbering in child documents may well
be inconsistent until the complete document is compiled at least once.

A useful (if unconventional) way to always ensure a consistent
page numbering is to restart the numbering in each child document
and denote the pages by `\textit{child}|.|\textit{page}'
where \textit{child} represents the chapter/section number of the child file.
This can be achieved by the command
|\numberwithin{page}{|\textit{child}|}|
of the \textsf{amsmath} package
where \textit{child} can be |chapter| or |section|
depending on the chosen structuring.
Alternatively, one can modify the macro |\thepage| appropriately
and reset the counter |page| at the start of each child file.

%%%%%%%%%%%%%%%%%%%%%%%%%%%%%%%%%%%%%%%%%%%%%%%%%%%%%%%%%%%%%%%%%%%%%%%%%%%%%%%%
\subsection{Conditional Processing}
\label{sec:conditional}

The package provides a mechanism to compile different versions
of a document. To customise the versions further some conditional processing
can come in handy to distinguish which version is being compiled.
The package provides two macros to describe the compilation context:

%%%%%%%%%%%%%%%%%%%%%%%%%%%%%%%%%%%%%%%%
\DescribeMacro{\ifchilddoc}
The conditional |\ifchilddoc| distinguishes between the compilation of
child documents and the main document:
%
\begin{center}
|\ifchilddoc |\textit{child-code}| |[|\||else |\textit{main-code}]| \||fi|
\end{center}

%%%%%%%%%%%%%%%%%%%%%%%%%%%%%%%%%%%%%%%%
\DescribeMacro{\childdocname}
\DescribeMacro{\childdocjob}
The macro |\childdocname| contains the filename (without extension)
of the main or child file being processed.
Note that |\childdocjob| will always contain the name of the main file.

%%%%%%%%%%%%%%%%%%%%%%%%%%%%%%%%%%%%%%%%
\paragraph{Title Page.}

Conditional processing can be used to include a title or banner page
in the main document when proper precautions are taken.
Importantly, the code in the main file should ensure that the page counter
(as well as other status parameters which are stored in the |.aux| files)
takes the same value after the conditional processing.
Otherwise the page numbers may take divergent values
depending on which part is compiled.

For example, a title page could be declared by:
%
\begin{center}
\begin{tabular}{l}
|\ifchilddoc\||else|\\
|\addtocounter{page}{-1}|\\
\textit{code for title page}\\
|\newpage|\\
|\||fi|
\end{tabular}
\end{center}
%
A banner page for the child documents can be generated by:
%
\begin{center}
\begin{tabular}{l}
|\ifchilddoc|\\
|\addtocounter{page}{-1}|\\
\textit{code for banner page}\\
|\newpage|\\
|\||fi|
\end{tabular}
\end{center}
%
Here one could write a message such as:
\begin{center}
|This is the part \childdocname{} of \childdocjob{}.|
\end{center}

%%%%%%%%%%%%%%%%%%%%%%%%%%%%%%%%%%%%%%%%%%%%%%%%%%%%%%%%%%%%%%%%%%%%%%%%%%%%%%%%
\subsection{Flags}
\label{sec:flags}

The package makes it easy to generate different versions
of the main or child documents.
To this end compilation flags can be defined
and assigned different default values.
They will be particularly useful in conjunction
with the forwarding mechanism described in \secref{sec:forward}.

For example, it may be useful to have a flag |\version|
which can be set to |draft| or |final|.
The document source will contain some conditional code
depending on the value of |\version|.
Suppose further, the flag should default to |final| for the main file
and to |draft| for child files
which is a natural assignment for editing the document.
This is achieved by placing the following code
in the preamble of the main document
(below the |\childdocmain| directive):
%
\begin{center}
\begin{tabular}{l}
|\ifchilddoc|\\
|\providecommand{\version}{draft}|\\
|\||else|\\
|\providecommand{\version}{final}|\\
|\||fi|
\end{tabular}
\end{center}
%
The definition by |\providecommand| makes sure
that previous definitions are not overwritten.
Further statements |\providecommand{\version}{...}|
can thus be added before the above code to override it.

For the main file, one might add a line
(between |\childdocmain| and the above block)
%
\begin{center}
|%\ifchilddoc\||else\providecommand{\version}{draft}\||fi|
\end{center}
%
which can be uncommented to produce a draft version.
Likewise one can add a line to the very top of a child file
(above the |\childdocof{|\textit{main}|}| directive)
%
\begin{center}
|%\providecommand{\version}{final}|
\end{center}
%
which can be uncommented to produce the final version of this child document.

%%%%%%%%%%%%%%%%%%%%%%%%%%%%%%%%%%%%%%%%%%%%%%%%%%%%%%%%%%%%%%%%%%%%%%%%%%%%%%%%
\subsection{Forwarding}
\label{sec:forward}

Different versions of the main or child documents
using compilation flags as described in \secref{sec:flags}
can be (permanently) stored in different files
for convenient compilation, viewing and distribution.
To this end, the package defines a command
to pass on compilation to a different file:

%%%%%%%%%%%%%%%%%%%%%%%%%%%%%%%%%%%%%%%%
\DescribeMacro{\childdocforward}
The command |\childdocforward| redirects processing to
another source file:
%
\begin{center}
\begin{tabular}{l}
|% \iffalse
%
% childdoc.dtx Copyright (C) 2017-2018 Niklas Beisert
%
% This work may be distributed and/or modified under the
% conditions of the LaTeX Project Public License, either version 1.3
% of this license or (at your option) any later version.
% The latest version of this license is in
%   http://www.latex-project.org/lppl.txt
% and version 1.3 or later is part of all distributions of LaTeX
% version 2005/12/01 or later.
%
% This work has the LPPL maintenance status `maintained'.
%
% The Current Maintainer of this work is Niklas Beisert.
%
% This work consists of the files childdoc.dtx and childdoc.ins
% and the derived files childdoc.def and cdocsamp.tex with
% cdocsch1.tex, cdocsch2.tex, cdocsdrf.tex, cdocsfn1.tex, cdocsfn2.tex.
%
%<package>\ifdefined\childdocmain\endinput\fi
%<package>\ProvidesFile{childdoc.def}[2018/12/30 v2.0 child document driver]
%<samplemain>\ProvidesFile{cdocsamp.tex}[2018/12/30 v2.0 sample for childdoc]
%<*driver>
%\ProvidesFile{childdoc.drv}[2018/12/30 v2.0 childdoc reference manual file]
\PassOptionsToClass{10pt,a4paper}{article}
\documentclass{ltxdoc}

\usepackage[margin=35mm]{geometry}
\usepackage{hyperref}
\usepackage{hyperxmp}
\usepackage[usenames]{color}

\hypersetup{colorlinks=true}
\hypersetup{pdfstartview=FitH}
\hypersetup{pdfpagemode=UseNone}
\hypersetup{pdfsource={}}
\hypersetup{pdflang={en-UK}}
\hypersetup{pdfcopyright={Copyright 2017-2018 Niklas Beisert.
  This work may be distributed and/or modified under the
  conditions of the LaTeX Project Public License, either version 1.3
  of this license or (at your option) any later version.}}
\hypersetup{pdflicenseurl={http://www.latex-project.org/lppl.txt}}
\hypersetup{pdfcontactaddress={ETH Zurich, ITP, HIT K,
  Wolfgang-Pauli-Strasse 27}}
\hypersetup{pdfcontactpostcode={8093}}
\hypersetup{pdfcontactcity={Zurich}}
\hypersetup{pdfcontactcountry={Switzerland}}
\hypersetup{pdfcontactemail={nbeisert@itp.phys.ethz.ch}}
\hypersetup{pdfcontacturl={http://people.phys.ethz.ch/\xmptilde nbeisert/}}

\newcommand{\secref}[1]{\hyperref[#1]{section \ref*{#1}}}

\parskip1ex
\parindent0pt
\let\olditemize\itemize
\def\itemize{\olditemize\parskip0pt}

\begin{document}

\title{The \textsf{childdoc} Package}
\hypersetup{pdftitle={The childdoc Package}}
\author{Niklas Beisert\\[2ex]
  Institut f\"ur Theoretische Physik\\
  Eidgen\"ossische Technische Hochschule Z\"urich\\
  Wolfgang-Pauli-Strasse 27, 8093 Z\"urich, Switzerland\\[1ex]
  \href{mailto:nbeisert@itp.phys.ethz.ch}
  {\texttt{nbeisert@itp.phys.ethz.ch}}}
\hypersetup{pdfauthor={Niklas Beisert}}
\hypersetup{pdfsubject={Manual for the LaTeX2e Package childdoc}}
\date{30 December 2018, \textsf{v2.0}}
\maketitle

\begin{abstract}\noindent
\textsf{childdoc} is a \LaTeXe{} package
that enables the direct compilation
of document sections included by |\include|
to individual files.
\end{abstract}

\begingroup
\parskip0ex
\tableofcontents
\endgroup

%%%%%%%%%%%%%%%%%%%%%%%%%%%%%%%%%%%%%%%%%%%%%%%%%%%%%%%%%%%%%%%%%%%%%%%%%%%%%%%%
%%%%%%%%%%%%%%%%%%%%%%%%%%%%%%%%%%%%%%%%%%%%%%%%%%%%%%%%%%%%%%%%%%%%%%%%%%%%%%%%
\section{Introduction}

\LaTeX{} provides a mechanism to structure a large document (such as a book)
into a main file and several child files (containing the chapters)
using the |\include| command.
This mechanism is beneficial for documents
which span hundreds of pages in order to
make the source file(s) more manageable.
Moreover, compilation can be restricted to
selected child files by means of the |\includeonly| command.
The latter feature can be used to reduce the compilation time while editing
(this was significantly more useful in the earlier days of \LaTeX{})
or to generate a smaller document which is easier to navigate.
Another application of |\includeonly| is to generate
documents consisting of selected parts of the complete document.

However, there are a few drawbacks of the plain |\include| mechanism:
\begin{itemize}
\item
The child files cannot be compiled on their own,
they can only be compiled via the main file.
A naive editing environment
(such as a text editor with an option
to have the current file processed by \LaTeX)
may require one to switch to the main file before compiling;
attempting to compile the child file produces errors.
\item
The main file must be modified (each time)
to adjust the |\includeonly| command
to the present needs. This easily leaves the main file in a messy state.
\item
The generated document will always carry the filename
of the main document. This is inconvenient if
several child files are to be compiled and
to be kept for distribution.
\end{itemize}

The present package provides a simple interface
to make child files individually compilable by \LaTeX{}.
Compiling a child file then has the same effect as compiling
the main file with an |\includeonly| command
to select the appropriate child.
Moreover the generated document will carry the name of the child
rather than the main file.
This resolves all three above issues.

This feature is meant to make the editing of books,
thesis documents and lecture notes somewhat more convenient.
However, the package can also be used efficiently for
composing a series of documents (such as exercise sheets)
which are typically distributed individually.
It then assists the author in generating the individual documents
(potentially in different versions)
as well as a document containing the collected series.
Another application is in developing style files
or other kinds of included material
where compilation of the style file could redirect
to a sample or test file.

%%%%%%%%%%%%%%%%%%%%%%%%%%%%%%%%%%%%%%%%%%%%%%%%%%%%%%%%%%%%%%%%%%%%%%%%%%%%%%%%
%%%%%%%%%%%%%%%%%%%%%%%%%%%%%%%%%%%%%%%%%%%%%%%%%%%%%%%%%%%%%%%%%%%%%%%%%%%%%%%%
\section{Usage}

First of all, the package \textsf{childdoc} is \emph{not} a standard
\LaTeXe{} |.sty| style file! Therefore it needs to be invoked in
a non-standard way.

%%%%%%%%%%%%%%%%%%%%%%%%%%%%%%%%%%%%%%%%%%%%%%%%%%%%%%%%%%%%%%%%%%%%%%%%%%%%%%%%
\subsection{Included Files}
\label{sec:include}

%%%%%%%%%%%%%%%%%%%%%%%%%%%%%%%%%%%%%%%%
\DescribeMacro{\childdocmain}
To use the package, add the commands
\begin{center}
\begin{tabular}{l}
|\input{childdoc.def}|\\
|\childdocmain{}|\\
\end{tabular}
\end{center}
at the very top of the main \LaTeX{} file,
in particular \emph{before} the |\documentclass| statement!
The argument of |\childdocmain| should be left empty
(but it must be present).

%%%%%%%%%%%%%%%%%%%%%%%%%%%%%%%%%%%%%%%%
\DescribeMacro{\childdocof}
Furthermore, add the commands
\begin{center}
\begin{tabular}{l}
|\input{childdoc.def}|\\
|\childdocof{|\textit{main}|}|\\
\end{tabular}
\end{center}
at the top of every child file \textit{child}
which is included by |\include{|\textit{child}|}|
from within the main file
(or at least for those files to be compiled individually).
The argument \textit{main} must be the filename of the main file.

There are a couple of
considerations in setting up the main and child documents:

%%%%%%%%%%%%%%%%%%%%%%%%%%%%%%%%%%%%%%%%
\paragraph{Restrictions.}

Please note the following restrictions:
\begin{itemize}
\item
|\childdocmain| must be called with one argument \textit{main}
to ensure compatibility with earlier version of the package.
It must either be empty (|\childdocmain{}|)
or precisely match the filename of the main file in which it is specified.
See \secref{sec:detection} for further information.
\item
The filename \textit{main} must be specified without the |.tex| extension.
\item
The filename \textit{main} is case sensitive
(even in case-insensitive file systems)
due to internal string comparison.
\item
The argument \textit{main} should be fully expanded, it cannot be a macro.
\item
Subdirectories and special characters should be avoided in filenames.
\item
The command |\childdocmain{|\textit{main}|}| must be followed by a whitespace.
It should not be followed immediately by another command
or by a comment mark `|%|'.
This is because the \TeX{} parser reads the token immediately following
the argument of |\childdocmain| and puts it
at the beginning of every child section;
however, a white\-space is ignored.
\end{itemize}

%%%%%%%%%%%%%%%%%%%%%%%%%%%%%%%%%%%%%%%%
\paragraph{Content of Main File.}

It is advisable to place all content in the child files included by |\include|.
Any output contained in the main file will appear in all child documents
unless suppressed manually;
it cannot be suppressed automatically by the |\includeonly| directive
and thus should normally be avoided.
A method to include some content in the main file
by means of conditional processing is described in \secref{sec:conditional}.

%%%%%%%%%%%%%%%%%%%%%%%%%%%%%%%%%%%%%%%%
\paragraph{Page Numbering.}

When only a part of the document is compiled,
the appropriate numbering of pages
(as well as other status parameters)
is determined from the |.aux| files.
The latter contain information from previous passes.
However this information needs to propagate through
all intermediate child documents.
Therefore the page numbering in child documents may well
be inconsistent until the complete document is compiled at least once.

A useful (if unconventional) way to always ensure a consistent
page numbering is to restart the numbering in each child document
and denote the pages by `\textit{child}|.|\textit{page}'
where \textit{child} represents the chapter/section number of the child file.
This can be achieved by the command
|\numberwithin{page}{|\textit{child}|}|
of the \textsf{amsmath} package
where \textit{child} can be |chapter| or |section|
depending on the chosen structuring.
Alternatively, one can modify the macro |\thepage| appropriately
and reset the counter |page| at the start of each child file.

%%%%%%%%%%%%%%%%%%%%%%%%%%%%%%%%%%%%%%%%%%%%%%%%%%%%%%%%%%%%%%%%%%%%%%%%%%%%%%%%
\subsection{Conditional Processing}
\label{sec:conditional}

The package provides a mechanism to compile different versions
of a document. To customise the versions further some conditional processing
can come in handy to distinguish which version is being compiled.
The package provides two macros to describe the compilation context:

%%%%%%%%%%%%%%%%%%%%%%%%%%%%%%%%%%%%%%%%
\DescribeMacro{\ifchilddoc}
The conditional |\ifchilddoc| distinguishes between the compilation of
child documents and the main document:
%
\begin{center}
|\ifchilddoc |\textit{child-code}| |[|\||else |\textit{main-code}]| \||fi|
\end{center}

%%%%%%%%%%%%%%%%%%%%%%%%%%%%%%%%%%%%%%%%
\DescribeMacro{\childdocname}
\DescribeMacro{\childdocjob}
The macro |\childdocname| contains the filename (without extension)
of the main or child file being processed.
Note that |\childdocjob| will always contain the name of the main file.

%%%%%%%%%%%%%%%%%%%%%%%%%%%%%%%%%%%%%%%%
\paragraph{Title Page.}

Conditional processing can be used to include a title or banner page
in the main document when proper precautions are taken.
Importantly, the code in the main file should ensure that the page counter
(as well as other status parameters which are stored in the |.aux| files)
takes the same value after the conditional processing.
Otherwise the page numbers may take divergent values
depending on which part is compiled.

For example, a title page could be declared by:
%
\begin{center}
\begin{tabular}{l}
|\ifchilddoc\||else|\\
|\addtocounter{page}{-1}|\\
\textit{code for title page}\\
|\newpage|\\
|\||fi|
\end{tabular}
\end{center}
%
A banner page for the child documents can be generated by:
%
\begin{center}
\begin{tabular}{l}
|\ifchilddoc|\\
|\addtocounter{page}{-1}|\\
\textit{code for banner page}\\
|\newpage|\\
|\||fi|
\end{tabular}
\end{center}
%
Here one could write a message such as:
\begin{center}
|This is the part \childdocname{} of \childdocjob{}.|
\end{center}

%%%%%%%%%%%%%%%%%%%%%%%%%%%%%%%%%%%%%%%%%%%%%%%%%%%%%%%%%%%%%%%%%%%%%%%%%%%%%%%%
\subsection{Flags}
\label{sec:flags}

The package makes it easy to generate different versions
of the main or child documents.
To this end compilation flags can be defined
and assigned different default values.
They will be particularly useful in conjunction
with the forwarding mechanism described in \secref{sec:forward}.

For example, it may be useful to have a flag |\version|
which can be set to |draft| or |final|.
The document source will contain some conditional code
depending on the value of |\version|.
Suppose further, the flag should default to |final| for the main file
and to |draft| for child files
which is a natural assignment for editing the document.
This is achieved by placing the following code
in the preamble of the main document
(below the |\childdocmain| directive):
%
\begin{center}
\begin{tabular}{l}
|\ifchilddoc|\\
|\providecommand{\version}{draft}|\\
|\||else|\\
|\providecommand{\version}{final}|\\
|\||fi|
\end{tabular}
\end{center}
%
The definition by |\providecommand| makes sure
that previous definitions are not overwritten.
Further statements |\providecommand{\version}{...}|
can thus be added before the above code to override it.

For the main file, one might add a line
(between |\childdocmain| and the above block)
%
\begin{center}
|%\ifchilddoc\||else\providecommand{\version}{draft}\||fi|
\end{center}
%
which can be uncommented to produce a draft version.
Likewise one can add a line to the very top of a child file
(above the |\childdocof{|\textit{main}|}| directive)
%
\begin{center}
|%\providecommand{\version}{final}|
\end{center}
%
which can be uncommented to produce the final version of this child document.

%%%%%%%%%%%%%%%%%%%%%%%%%%%%%%%%%%%%%%%%%%%%%%%%%%%%%%%%%%%%%%%%%%%%%%%%%%%%%%%%
\subsection{Forwarding}
\label{sec:forward}

Different versions of the main or child documents
using compilation flags as described in \secref{sec:flags}
can be (permanently) stored in different files
for convenient compilation, viewing and distribution.
To this end, the package defines a command
to pass on compilation to a different file:

%%%%%%%%%%%%%%%%%%%%%%%%%%%%%%%%%%%%%%%%
\DescribeMacro{\childdocforward}
The command |\childdocforward| redirects processing to
another source file:
%
\begin{center}
\begin{tabular}{l}
|\input{childdoc.def}|\\
|\childdocforward[|\textit{main}|]{|\textit{dest}|}|\\
\end{tabular}
\end{center}
%
The argument \textit{dest} is the destination file
(without extension).
It should be the main file or one of the child files.
Note that further \textsf{childdoc} directives
such as |\childdocof| and |\childdocforward|
in the indicated file will be processed in this form.
The optional argument \textit{main}
passes on directly to the main file \textit{main}
while pretending to compile the child \textit{dest}.
This form behaves as if \textit{dest}
issues |\childdocof{|\textit{main}|}| right away,
and no further \textsf{childdoc} directives will be processed.

%%%%%%%%%%%%%%%%%%%%%%%%%%%%%%%%%%%%%%%%
\DescribeMacro{\...prefix}
In the alternative form |\childdocforwardprefix|,
%
\begin{center}
\begin{tabular}{l}
|\input{childdoc.def}|\\
|\childdocforwardprefix[|\textit{main}|]{|\textit{prefix}|}{|\textit{dest}|}|
\end{tabular}
\end{center}
%
the destination file is determined by a pattern
depending on the current file:
To make this work, the current file must be called
`{\textit{prefix}\hspace{0.2em}\textit{suffix}}'
with \textit{prefix} matching precisely the argument.
Processing is then passed on to the file
`{\textit{dest}\hspace{0.2em}\textit{suffix}}'.
Surely, the same effect is achieved by
directly specifying the
argument `{\textit{dest}\hspace{0.2em}\textit{suffix}}'
in the first form.
However, that requires to set up a different file
for each child. With the alternative form of the command
all these files can have exactly the same content
which simplifies setting them up and maintaining them.

For example, the following file |draft.tex|
with a compilation flag |\version| as described in \secref{sec:flags}
compiles the main document as a draft:
%
\begin{center}
\begin{tabular}{l}
|\def\version{draft}|\\
|\input{childdoc.def}|\\
|\childdocforward{|\textit{main}|}|
\end{tabular}
\end{center}
%
Likewise, the following files |final|\textit{nn}|.tex|
compile the final version of the child document
|child|\textit{nn}|.tex|:
%
\begin{center}
\begin{tabular}{l}
|\def\version{final}|\\
|\input{childdoc.def}|\\
|\childdocforwardprefix{final}{child}|
\end{tabular}
\end{center}
%

Note that when several versions of a main file and/or of each child file
are to be generated, it may be convenient to set up a |Makefile| or
shell script to automatise the process.

%%%%%%%%%%%%%%%%%%%%%%%%%%%%%%%%%%%%%%%%%%%%%%%%%%%%%%%%%%%%%%%%%%%%%%%%%%%%%%%%
\subsection{Command Line Processing}
\label{sec:commandline}

The effect of redirection files can also be achieved by invoking
the \LaTeX{} compiler with a more elaborate command line.
Most conveniently this should be done as part
of a shell script or a |Makefile|.

When using \textsf{childdoc} in the main file, the following
command lines effectively perform a redirection
(note that depending on the shell being used,
backslashes may have to be doubled: `|\|' $\to$ `|\\|'):
%
\begin{center}
|... -jobname "|\textit{target}|" |\\|"|[\textit{flags}]%
|\input{childdoc.def}\childdocforward[|\textit{main}|]{|\textit{dest}|}"|
\end{center}
%
Here \textit{target} is the name of the output file,
\textit{main} is the name of the main file
and \textit{dest} is the name of the main or child file to be processed
(all filenames without extensions).
The optional argument \textit{main} can be omitted
if \textit{main} matches \textit{dest}.
Optionally, compilation \textit{flags} can be defined via |\def| commands.
This command line makes the \TeX{} engine believe
it is compiling the file \textit{target}
whose content is specified as the latter parameter.
The provided code then forwards the processing to
\textit{main} or \textit{dest} as described in \secref{sec:forward}.

%%%%%%%%%%%%%%%%%%%%%%%%%%%%%%%%%%%%%%%%%%%%%%%%%%%%%%%%%%%%%%%%%%%%%%%%%%%%%%%%
\subsection{Include by Input}
\label{sec:input}

Including child documents by |\include| has some restrictions by design.
Most notably, the content of a child document always occupies
its own set of pages; pages cannot be shared between child documents.
Usually, this behaviour makes perfect sense
because each child document contain an essential part of the document.
However, in some situations it may be desirable to compose
a document from a collection of parts
without having mandatory page breaks between then.
For this case, the package
provides a mechanism to include parts
by |\input| which can also be processed individually.
However, by construction this mechanism
requires manual handling of the content to be output.

%%%%%%%%%%%%%%%%%%%%%%%%%%%%%%%%%%%%%%%%
\DescribeMacro{\ifchilddocmanual}
The main file should be prepared as usual, see \secref{sec:include}.
However, the document body must make a distinction
between processing of an individual part and of the main document, e.g.:
%
\begin{center}
\begin{tabular}{l}
|\ifchilddocmanual|\\
|\input{\childdocname}|\\
|\||else|\\
\textit{document body with }|\input{|\textit{part}|}|\\
|\||fi|
\end{tabular}
\end{center}
%
The conditional |\ifchilddocmanual| is true whenever
a part to be included by |\input| is being compiled,
and the name of the part is stored in |\childdocname|.

%%%%%%%%%%%%%%%%%%%%%%%%%%%%%%%%%%%%%%%%
\DescribeMacro{\childdocby}
Each part to be included by |\input| should start with:
%
\begin{center}
\begin{tabular}{l}
|\input{childdoc.def}|\\
|\childdocby{|\textit{main}|}|\\
\end{tabular}
\end{center}
%
The directive |\childdocby| is similar to |\childdocof|
described in \secref{sec:include},
but the subsequent selection of content must be done manually.
To that end, both |\ifchilddoc| and |\ifchilddocmanual|
will be true upon processing of a part,
and the name of the part is stored in |\childdocname|.
Note that |\jobname| will be set to the filename of the current part
so that each part receives an individual |.aux| file
that does not interfere with the |.aux| file(s) of the main document.
This behaviour can be altered by the alternative form
|\childdocby[*]{|\textit{main}|}| (with a non-empty optional argument)
which uses the |.aux| file of the main document
by setting |\jobname| to \textit{main}.

%%%%%%%%%%%%%%%%%%%%%%%%%%%%%%%%%%%%%%%%%%%%%%%%%%%%%%%%%%%%%%%%%%%%%%%%%%%%%%%%
\subsection{Driver Development}
\label{sec:driver}

The \textsf{childdoc} mechanism can also be use for the development
of definition files such as \LaTeX{} styles or classes.
This case differs from the above setup with multiple parts
included by |\include| in that no |\includeonly| should be invoked.
This can be achieved by starting the include file
(before |\ProvidesPackage|) with:
%
\begin{center}
\begin{tabular}{l}
|\input{childdoc.def}|\\
|\childdocforward{|\textit{main}|}|\\
\end{tabular}
\end{center}
%
or alternatively with:
%
\begin{center}
\begin{tabular}{l}
|\input{childdoc.def}|\\
|\childdocby{|\textit{main}|}|\\
\end{tabular}
\end{center}
%
Both forms have slightly different effects as described above.
The main file is prepared as usual, see \secref{sec:include}.

%%%%%%%%%%%%%%%%%%%%%%%%%%%%%%%%%%%%%%%%%%%%%%%%%%%%%%%%%%%%%%%%%%%%%%%%%%%%%%%%
\subsection{Legacy Detection}
\label{sec:detection}

The directive |\childdocmain| in the main file can detect
whether the complete document or merely a child is to be compiled
even without using the directive |\childdocof|.
This method is deprecated because it is less robust
and there is no compelling reason to use it;
it is merely provided for backward compatibility
and it may be removed in future versions.

If the detection mechanism is to be used,
it is mandatory to correctly specify
the filename of the main file as the argument of |\childdocmain|:
%
\begin{center}
\begin{tabular}{l}
|\input{childdoc.def}|\\
|\childdocmain{|\textit{main}|}|\\
\end{tabular}
\end{center}
%
If |\jobname| does not match the argument \textit{main} of |\childdocmain|,
it is assumed that |\jobname| points to the child file to be compiled.
When using |\childdocmain| with the main file specified as argument,
it suffices to start a child file
with just |\input{|\textit{main}|}|
without loading of the package and using |\childdocof|.
If instead all processing is done
with the appropriate \textsf{childdoc} directives,
the argument of \textit{main} of |\childdocmain| can be empty.

An alternative version of the command line processing described
in \secref{sec:commandline} using the detection mechanism reads:
%
\begin{center}
|... -jobname "|\textit{target}|" "|[\textit{flags}]%
[|\def\jobname{|\textit{dest}|}|]|\input{|\textit{main}|}"|
\end{center}

%%%%%%%%%%%%%%%%%%%%%%%%%%%%%%%%%%%%%%%%%%%%%%%%%%%%%%%%%%%%%%%%%%%%%%%%%%%%%%%%
\subsection{Manual Code}
\label{sec:manual}

In case one cannot be certain whether the definitions file |childdoc.def|
is installed on the target \TeX{} distribution
and one prefers not to ship it,
it is conceivable to paste a few relevant commands into the sources.

To that end, drop all statements |\input{childdoc.def}|
and perform the replacements as outlined below.
Instead of |\childdocmain{|\textit{main}|}| add the following code
to the top of the main file:
%
\begin{center}
\begin{tabular}{l}
|\||ifdefined\childdocname\endinput\||fi\newif\ifchilddoc|\\
|\edef\childdocname{\scantokens\expandafter{\jobname\noexpand}}|\\
|\def\childdocmain{|\textit{main}|}\||ifx\childdocmain\childdocname\||else|\\
|\childdoctrue\includeonly{\childdocname}\let\jobname\childdocmain\||fi|\\
\end{tabular}
\end{center}
%
Instead of |\childdocof{|\textit{main}|}| just include the main file
at the top of each child file:
%
\begin{center}
|\input{|\textit{main}|}|
\end{center}
%
A simple redirection |\childdocforward{|\textit{dest}|}| is achieved by:
%
\begin{center}
|\def\jobname{|\textit{dest}|}\input{\jobname}|
\end{center}
%
The redirection with prefix
|\childdocforwardprefix[|\textit{prefix}|]{|\textit{dest}|}|
is accomplished by:
%
\begin{center}
\begin{tabular}{l}
|{\edef\jobname{\scantokens\expandafter{\jobname\noexpand}}|\\
|\def\redirectjob |\textit{prefix}|#1~~~{\gdef\jobname{|\textit{dest}|#1}}|\\
|\expandafter\redirectjob\jobname~~~}\input{\jobname}|
\end{tabular}
\end{center}

In an alternative approach,
child documents can be compiled by a specific command line
without additional code or specific definitions:
%
\begin{center}
|... -jobname "|\textit{target}|" "|[\textit{flags}]%
|\includeonly{|\textit{dest}|}\input{|\textit{main}|}"|
\end{center}
%

%%%%%%%%%%%%%%%%%%%%%%%%%%%%%%%%%%%%%%%%%%%%%%%%%%%%%%%%%%%%%%%%%%%%%%%%%%%%%%%%
%%%%%%%%%%%%%%%%%%%%%%%%%%%%%%%%%%%%%%%%%%%%%%%%%%%%%%%%%%%%%%%%%%%%%%%%%%%%%%%%
\section{Information}

%%%%%%%%%%%%%%%%%%%%%%%%%%%%%%%%%%%%%%%%%%%%%%%%%%%%%%%%%%%%%%%%%%%%%%%%%%%%%%%%
\subsection{Copyright}

Copyright \copyright{} 2017--2018 Niklas Beisert

This work may be distributed and/or modified under the
conditions of the \LaTeX{} Project Public License, either version 1.3
of this license or (at your option) any later version.
The latest version of this license is in
  \url{http://www.latex-project.org/lppl.txt}
and version 1.3 or later is part of all distributions of \LaTeX{}
version 2005/12/01 or later.

This work has the LPPL maintenance status `maintained'.

The Current Maintainer of this work is Niklas Beisert.

This work consists of the files |README.txt|, |childdoc.ins| and |childdoc.dtx|
as well as the derived files |childdoc.def|, |cdocsamp.tex|
with |cdocsch1.tex|, |cdocsch2.tex|, |cdocspt3.tex|, |cdocspt4.tex|,
|cdocsdrf.tex|, |cdocsfn1.tex|, |cdocsfn2.tex|
as well as |childdoc.pdf|.

%%%%%%%%%%%%%%%%%%%%%%%%%%%%%%%%%%%%%%%%%%%%%%%%%%%%%%%%%%%%%%%%%%%%%%%%%%%%%%%%
\subsection{Files and Installation}

The package consists of the files:
%
\begin{center}
\begin{tabular}{ll}
    |README.txt|   & readme file \\
    |childdoc.ins| & installation file \\
    |childdoc.dtx| & source file \\
    |childdoc.def| & definition file \\
    |cdocsamp.tex| & sample main file \\
    |cdocsch1.tex| & sample include file \\
    |cdocsch2.tex| & sample include file \\
    |cdocspt3.tex| & sample part file \\
    |cdocspt4.tex| & sample part file \\
    |cdocsdrf.tex| & sample redirection file \\
    |cdocsfn1.tex| & sample redirection file \\
    |cdocsfn2.tex| & sample redirection file \\
    |childdoc.pdf| & manual
\end{tabular}
\end{center}
%
The distribution consists of the files
|README.txt|, |childdoc.ins| and |childdoc.dtx|.
%
\begin{itemize}
\item
Run (pdf)\LaTeX{} on |childdoc.dtx|
to compile the manual |childdoc.pdf| (this file).
\item
Run \LaTeX{} on |childdoc.ins| to create the definitions file |childdoc.def|
and the sample |cdocsamp.tex| with include files
|cdocsch1.tex|, |cdocsch2.tex|, |cdocspt3.tex|, |cdocspt4.tex|,
|cdocsdrf.tex|, |cdocsfn1.tex|, |cdocsfn2.tex|.
Then copy the file |childdoc.def| to an appropriate directory of your \LaTeX{}
distribution, e.g.\ \textit{texmf-root}|/tex/latex/childdoc|.
\end{itemize}

%%%%%%%%%%%%%%%%%%%%%%%%%%%%%%%%%%%%%%%%%%%%%%%%%%%%%%%%%%%%%%%%%%%%%%%%%%%%%%%%
\subsection{Related CTAN Packages}

There are several other packages which offer a similar functionality:
%
\begin{itemize}
\item
The packages
\href{http://ctan.org/pkg/docmute}{\textsf{docmute}},
\href{http://ctan.org/pkg/includex}{\textsf{includex}} and
\href{http://ctan.org/pkg/standalone}{\textsf{standalone}}
provide commands to include only the document body of
a child file thus allowing both files to be compiled individually.
\item
The packages \href{http://ctan.org/pkg/subdocs}{\textsf{subdocs}}
and \href{http://ctan.org/pkg/subfiles}{\textsf{subfiles}}
provide structures in which the main and child documents can be
encapsulated and allowing them to be compiled individually.
The inclusion mechanism is different from the conventional |\include|.
\item
The package \href{http://ctan.org/pkg/combine}{\textsf{combine}}
is an elaborate solution to combine several documents into one.
\end{itemize}
%
See also the CTAN topic \href{http://ctan.org/topic/subdocs}{\textsf{subdocs}}
for further related packages.
The present package differs from the above solutions in that
a document structure constructed with the conventional |\include| mechanism
just needs two extra commands at the top of every file
such that all constituent files can be compiled individually.

%%%%%%%%%%%%%%%%%%%%%%%%%%%%%%%%%%%%%%%%%%%%%%%%%%%%%%%%%%%%%%%%%%%%%%%%%%%%%%%%
%\subsection{Feature Suggestions}
%
%The following is a list of features which may be useful for future
%versions of this package:
%%
%\begin{itemize}
%\item
%\ldots
%\end{itemize}

%%%%%%%%%%%%%%%%%%%%%%%%%%%%%%%%%%%%%%%%%%%%%%%%%%%%%%%%%%%%%%%%%%%%%%%%%%%%%%%%
\subsection{Revision History}

%%%%%%%%%%%%%%%%%%%%%%%%%%%%%%%%%%%%%%%%
\paragraph{v2.0:} 2018/12/30

\begin{itemize}
\item
immediate forward processing
\item
added |\childdocby| mechanism
\item
manual restructured
\end{itemize}

%%%%%%%%%%%%%%%%%%%%%%%%%%%%%%%%%%%%%%%%
\paragraph{v1.6:} 2018/01/17

\begin{itemize}
\item
application for development of include files
\item
corrections to manual
\end{itemize}

%%%%%%%%%%%%%%%%%%%%%%%%%%%%%%%%%%%%%%%%
\paragraph{v1.5:} 2017/05/21

\begin{itemize}
\item
more complete structuring introduced
\item
|\childdocof| introduced
\item
|\childdoc| renamed to |\childdocmain|
\item
|\childredirect| renamed to |\childdocforward| and |\childdocforwardprefix|
and functionality expanded
\end{itemize}

%%%%%%%%%%%%%%%%%%%%%%%%%%%%%%%%%%%%%%%%
\paragraph{v1.0:} 2017/04/27

\begin{itemize}
\item
manual and install package
\item
first version published on CTAN
\end{itemize}

%%%%%%%%%%%%%%%%%%%%%%%%%%%%%%%%%%%%%%%%
\paragraph{v0.6:} 2017/04/26

\begin{itemize}
\item
redirection mechanism added
\end{itemize}

%%%%%%%%%%%%%%%%%%%%%%%%%%%%%%%%%%%%%%%%
\paragraph{v0.5:} 2017/04/26

\begin{itemize}
\item
functionality in definition file
\end{itemize}


%%%%%%%%%%%%%%%%%%%%%%%%%%%%%%%%%%%%%%%%%%%%%%%%%%%%%%%%%%%%%%%%%%%%%%%%%%%%%%%%
%%%%%%%%%%%%%%%%%%%%%%%%%%%%%%%%%%%%%%%%%%%%%%%%%%%%%%%%%%%%%%%%%%%%%%%%%%%%%%%%
%%%%%%%%%%%%%%%%%%%%%%%%%%%%%%%%%%%%%%%%%%%%%%%%%%%%%%%%%%%%%%%%%%%%%%%%%%%%%%%%
\appendix

\settowidth\MacroIndent{\rmfamily\scriptsize 000\ }

 \DocInput{childdoc.dtx}

\end{document}
%</driver>
% \fi
%
% %%%%%%%%%%%%%%%%%%%%%%%%%%%%%%%%%%%%%%%%%%%%%%%%%%%%%%%%%%%%%%%%%%%%%%%%%%%%%%
% %%%%%%%%%%%%%%%%%%%%%%%%%%%%%%%%%%%%%%%%%%%%%%%%%%%%%%%%%%%%%%%%%%%%%%%%%%%%%%
% \section{Sample}
%\iffalse
%<*samplemain>
%\fi
%
% The following presents a sample document
% with two chapters, two parts, a title page,
% a compile flag as well as three forwarding files to set the flag.
% It consists of eight |.tex| files:
% \begin{center}
% \begin{tabular}{ll}
% |cdocsamp.tex|&main file\\
% |cdocsch1.tex|&include file for chapter 1\\
% |cdocsch2.tex|&include file for chapter 2\\
% |cdocspt3.tex|&include file for part 3\\
% |cdocspt4.tex|&include file for part 4\\
% |cdocsdrf.tex|&forwarding file for main file in draft mode\\
% |cdocsfi1.tex|&forwarding file for final version of chapter 1\\
% |cdocsfi2.tex|&forwarding file for final version of chapter 2\\
% \end{tabular}
% \end{center}
% Each of the eight files can be compiled directly by the \LaTeX{} compiler.
%
% %%%%%%%%%%%%%%%%%%%%%%%%%%%%%%%%%%%%%%
% \paragraph{Main File.}
%
% The main file is called |cdocsamp.tex|.
%
% Load the \textsf{childdoc} definitions and
% declare the filename for the main document:
%    \begin{macrocode}
\input{childdoc.def}
\childdocmain{}
%    \end{macrocode}

% Optional override for |\version| flag:
%    \begin{macrocode}
%%\ifchilddoc\else\providecommand{\version}{draft}\fi
%    \end{macrocode}

% Define the default values for the |\version| flag
% (|final| for the main file and |draft| for childs):
%    \begin{macrocode}
\ifchilddoc
\providecommand{\version}{draft}
\else
\providecommand{\version}{final}
\fi
%    \end{macrocode}

% Load the standard document class:
%    \begin{macrocode}
\documentclass[12pt]{article}
%    \end{macrocode}

% Start the document body:
%    \begin{macrocode}
\begin{document}
%    \end{macrocode}

% Declare a title page.
% Print title, part of document being processed and version flag:
%    \begin{macrocode}
\addtocounter{page}{-1}
\begin{center}
{\LARGE\bfseries{}childdoc example\par}
\vspace{1cm}
\ifchilddoc
\ifchilddocmanual part\else chapter\fi:
`\childdocname' of `\childdocjob'\par
\else
main document: `\childdocjob'\par
\fi
version: \version\par
\end{center}
\newpage
%    \end{macrocode}

% Manually include selected file,
% otherwise process as usual:
%    \begin{macrocode}
\ifchilddocmanual
\section*{part `\childdocname'}
\input{\childdocname}
\else
%    \end{macrocode}

% Include the two chapters:
%    \begin{macrocode}
\include{cdocsch1}
\include{cdocsch2}
%    \end{macrocode}

% Include the two parts unless only chapters should be displayed:
%    \begin{macrocode}
\ifchilddoc\else
\section{part three}
\input{cdocspt3}
\section{part four}
\input{cdocspt4}
\fi
%    \end{macrocode}

% Process as usual until here:
%    \begin{macrocode}
\fi
%    \end{macrocode}

% End of document body:
%    \begin{macrocode}
\end{document}
%    \end{macrocode}
%\iffalse
%</samplemain>
%\fi
%
% %%%%%%%%%%%%%%%%%%%%%%%%%%%%%%%%%%%%%%
% \paragraph{Chapter Include Files.}
%
% The include files are called |cdocsch1.tex| and |cdocsch2.tex|.
%
%\iffalse
%<*samplechap1|samplechap2>
%\fi

% Optional override for |\version| flag:
%    \begin{macrocode}
%%\providecommand{\version}{final}
%    \end{macrocode}

% Include the main document:
%    \begin{macrocode}
\input{childdoc.def}
\childdocof{cdocsamp}
%    \end{macrocode}

%\iffalse
%</samplechap1|samplechap2>
%\fi
%
%\iffalse
%<*samplechap1>
%\fi
% Some text for chapter 1:
%    \begin{macrocode}
\section{one}
some text in chapter one
%    \end{macrocode}

%\iffalse
%</samplechap1>
%\fi
% Some text for chapter 2:
%\iffalse
%<*samplechap2>
%\fi
%    \begin{macrocode}
\section{two}
more text in chapter two
%    \end{macrocode}

%\iffalse
%</samplechap2>
%\fi
%
% %%%%%%%%%%%%%%%%%%%%%%%%%%%%%%%%%%%%%%
% \paragraph{Part Include Files.}
%
% The include files are called |cdocspt3.tex| and |cdocspt4.tex|.
%
%\iffalse
%<*samplepart3|samplepart4>
%\fi

% Optional override for |\version| flag:
%    \begin{macrocode}
%%\providecommand{\version}{final}
%    \end{macrocode}

% Include the main document:
%    \begin{macrocode}
\input{childdoc.def}
\childdocby{cdocsamp}
%    \end{macrocode}

%\iffalse
%</samplepart3|samplepart4>
%\fi
%
%\iffalse
%<*samplepart3>
%\fi
% Some text for part 3:
%    \begin{macrocode}
some text in part three
%    \end{macrocode}

%\iffalse
%</samplepart3>
%\fi
% Some text for part 4:
%\iffalse
%<*samplepart4>
%\fi
%    \begin{macrocode}
more text in part four
%    \end{macrocode}

%\iffalse
%</samplepart4>
%\fi
%
% %%%%%%%%%%%%%%%%%%%%%%%%%%%%%%%%%%%%%%
% \paragraph{Forwarding for a Complete Draft.}
%
% The following forwarding file |cdocsdrf.tex|
% compiles the main document in draft mode:
%\iffalse
%<*sampledraft>
%\fi
%    \begin{macrocode}
\def\version{draft}
\input{childdoc.def}
\childdocforward{cdocsamp}
%    \end{macrocode}

%\iffalse
%</sampledraft>
%\fi
%
% %%%%%%%%%%%%%%%%%%%%%%%%%%%%%%%%%%%%%%
% \paragraph{Forwarding for Final Version of the Chapters.}
%
% The following forwarding files |cdocsfn1.tex| and |cdocsfn2.tex|
% (with identical content)
% compile the final versions of the child documents
% |cdocsch1.tex| and |cdocsch2.tex|, respectively:
%\iffalse
%<*samplefinal>
%\fi
%    \begin{macrocode}
\def\version{final}
\input{childdoc.def}
\childdocforwardprefix[cdocsamp]{cdocsfn}{cdocsch}
%    \end{macrocode}

%\iffalse
%</samplefinal>
%\fi
%
% %%%%%%%%%%%%%%%%%%%%%%%%%%%%%%%%%%%%%%
% \paragraph{Command Line Processing.}
%
% The following three command lines generate the output files
% |cdocscld|, |cdocscl1| and |cdocscl2|
% which should be identical to
% |cdocsdrf|, |cdocsch1| and |cdocsfn2|, respectively:
% \begin{center}
% \begin{tabular}{l}
% |latex -jobname cdocscld \|\\
% |  "\def\version{draft}\input{childdoc.def}\childdocforward{cdocsamp}"|\\
% |latex -jobname cdocscl1 \|\\
% |  "\input{childdoc.def}\childdocforward[cdocsamp]{cdocsch1}"|\\
% |latex -jobname cdocscl2 \|\\
% |  "\def\version{final}\input{childdoc.def}\childdocforward{cdocsch2}"|
% \end{tabular}
% \end{center}
% Note that the trailing backslash on each first line
% merely continues the input to the second line
% (for convenient cut ant paste).
% Furthermore, the command |latex| can be replaced by any
% of its alternative versions such as |pdflatex|.
%
% %%%%%%%%%%%%%%%%%%%%%%%%%%%%%%%%%%%%%%%%%%%%%%%%%%%%%%%%%%%%%%%%%%%%%%%%%%%%%%
% %%%%%%%%%%%%%%%%%%%%%%%%%%%%%%%%%%%%%%%%%%%%%%%%%%%%%%%%%%%%%%%%%%%%%%%%%%%%%%
% \section{Implementation}
%\iffalse
%<*package>
%\fi
%
% This section describes the definitions file |childdoc.def|.

% The definitions cannot be loaded using |\usepackage| or |\RequirePackage|
% which has a mechanism to prevent loading a style file more than once.
% When loading the definitions by means of |\input|
% multiple instances have to be prevented manually:
%\iffalse
%This code needs to be before the `\ProvidesFile' directive
%which is defined at the beginning of this file.
%Therefore it is also placed there and commented out here.
%</package>
%<*discard>
%\fi
%    \begin{macrocode}
\ifdefined\childdocmain\endinput\fi
%    \end{macrocode}
%\iffalse
%</discard>
%<*package>
%\fi
%
% \macro{\ifchilddoc}
% \macro{\ifchilddocmanual}
% The conditional |\ifchilddoc| tells whether a
% child (true) or main (false) document is being compiled.
% The conditional |\ifchilddocmanual| tells whether
% the |\includeonly| mechanism is used (false) or
% the selection of child files must be performed manually (true).
% The definitions initialise to false:
%    \begin{macrocode}
\newif\ifchilddoc
\newif\ifchilddocmanual
%    \end{macrocode}

% \macro{\childdocname}
% \macro{\childdocjob}
% The macro |\childdocname| stores the name of the main document
% to be compiled. The macro |\childdocjob| stores the name of
% the document on which the \LaTeX{} compiler was originally invoked.
% The content of |\jobname| cannot be compared
% to filenames specified in the source due to different catcodes.
% The following code rescans |\jobname|, stores the result
% in |\childdocname| and saves a copy in |\childdocjob|:
%    \begin{macrocode}
\edef\childdocname{\scantokens\expandafter{\jobname\noexpand}}
\let\childdocjob\childdocname
%    \end{macrocode}

% \macro{\childdocdisable}
% The macro |\childdocdisable| prevents the main file
% from being processed more than once.
% At this stage, the main document command |\childdocmain|
% is assumed to be called once again where it should do nothing.
% Any subsequent call to it should prevent
% a secondary processing of the main document
% It overwrites the forwarding commands
% |\childdocof| and |\childdocforward|
% with empty macros to prevent further inclusions of the main document:
%    \begin{macrocode}
\newcommand{\childdocdisable}
{
  \renewcommand{\childdocmain}[1]{\renewcommand{\childdocmain}[1]{\endinput}}
  \renewcommand{\childdocof}[1]{}
  \renewcommand{\childdocby}[2][]{}
  \renewcommand{\childdocforward}[2][]{}
  \renewcommand{\childdocdisable}{}
}
%    \end{macrocode}

% \macro{\childdocmain}
% The macro |\childdocmain| is to be called at the top of the main file
% with nothing or the main filename (without extension) as argument.
% First, it breaks loops.
% If the argument is not empty and does not match |\childdocname|
% (which is set by the first inclusion of |childdoc.def|),
% |\ifchilddoc| is set to true, |\includeonly| is applied to the child file
% and |\jobname| is set to the main file
% (for proper handling of |.aux| files):
%    \begin{macrocode}
\newcommand{\childdocmain}[1]
{
  \childdocdisable\childdocmain{}
  \if?#1?\else
    \begingroup
      \def\childdoctmp{#1}
      \ifx\childdoctmp\childdocname
        \def\childdoctmp{}
      \else
        \def\childdoctmp
        {
          \childdoctrue
          \includeonly{\childdocname}
          \def\childdocjob{#1}
          \def\jobname{#1}
        }
      \fi
      \expandafter
    \endgroup
    \childdoctmp
  \fi
}
%    \end{macrocode}

% \macro{\childdocof}
% The command |\childdocof| redirects
% compilation to the main file |#1|.
%    \begin{macrocode}
\newcommand{\childdocof}[1]
{
  \childdocdisable
  \childdoctrue
  \includeonly{\childdocname}
  \def\jobname{#1}
  \def\childdocjob{#1}
  \input{#1}
}
%    \end{macrocode}

% \macro{\childdocby}
% The command |\childdocby| ....
%    \begin{macrocode}
\newcommand{\childdocby}[2][]
{
  \childdocdisable
  \childdoctrue
  \childdocmanualtrue
  \if?#1?\else
    \def\jobname{#2}
  \fi
  \def\childdocjob{#2}
  \input{#2}
  \endinput
}
%    \end{macrocode}

% \macro{\childdocforward}
% The command |\childdocforward| redirects
% compilation to the main file or
% (if the optional argument is given) a child file.
% Parameters are set as if the main file
% or a child file starting with |\childdocof| was compiled.
% Then compilation is handed over to the main file:
%    \begin{macrocode}
\newcommand{\childdocforward}[2][]
{
  \begingroup
    \if?#1?
      \def\childdoctmp
      {
        \def\childdocname{#2}
        \def\childdocjob{#2}
        \def\jobname{#2}
        \input{#2}
        \endinput
      }
    \else
      \def\childdoctmp
      {
        \childdocdisable
        \def\childdocname{#2}
        \childdoctrue
        \includeonly{#2}
        \def\childdocjob{#1}
        \def\jobname{#1}
        \input{#1}
        \endinput
      }
    \fi
    \expandafter
  \endgroup
  \childdoctmp
}
%    \end{macrocode}

% \macro{\childdocforwardprefix}
% The command |\childdocforwardprefix| redirects
% compilation to the main or a child file by means of a pattern.
% The prefix |#1| in the current filename is replaced by |#2|
% and the suffix of the current filename is kept
% (it is assumed that the filename does not contain the substring `|~~~|'
% which is used as a delimiter).
% Compilation is handed over to the new file by |\childdocforward|:
%    \begin{macrocode}
\newcommand{\childdocforwardprefix}[3][]
{
  \begingroup
    \def\childdocextract #2##1~~~{\def\childdoctmp{\childdocforward[#1]{#3##1}}}
    \expandafter\childdocextract\childdocname~~~
    \expandafter
  \endgroup
  \childdoctmp
}
%    \end{macrocode}

% \macro{\childdoc}
% The deprecated macro |\childdoc| is a legacy version of |\childdocmain|:
%    \begin{macrocode}
\newcommand{\childdoc}{\childdocmain}
%    \end{macrocode}

% \macro{\childdocredirect}
% The deprecated macro |\childdocredirect| is a legacy version
% of |\childdocforward| and |\childdocforwardprefix|:
%    \begin{macrocode}
\newcommand{\childdocredirect}[2][]
{
  \begingroup
    \if?#1?
      \def\childdoctmp{\childdocforward{#2}}
    \else
      \def\childdoctmp{\childdocforwardprefix{#1}{#2}}
    \fi
    \expandafter
  \endgroup
  \childdoctmp
}
%    \end{macrocode}

%\iffalse
%</package>
%\fi
%
\endinput
|\\
|\childdocforward[|\textit{main}|]{|\textit{dest}|}|\\
\end{tabular}
\end{center}
%
The argument \textit{dest} is the destination file
(without extension).
It should be the main file or one of the child files.
Note that further \textsf{childdoc} directives
such as |\childdocof| and |\childdocforward|
in the indicated file will be processed in this form.
The optional argument \textit{main}
passes on directly to the main file \textit{main}
while pretending to compile the child \textit{dest}.
This form behaves as if \textit{dest}
issues |\childdocof{|\textit{main}|}| right away,
and no further \textsf{childdoc} directives will be processed.

%%%%%%%%%%%%%%%%%%%%%%%%%%%%%%%%%%%%%%%%
\DescribeMacro{\...prefix}
In the alternative form |\childdocforwardprefix|,
%
\begin{center}
\begin{tabular}{l}
|% \iffalse
%
% childdoc.dtx Copyright (C) 2017-2018 Niklas Beisert
%
% This work may be distributed and/or modified under the
% conditions of the LaTeX Project Public License, either version 1.3
% of this license or (at your option) any later version.
% The latest version of this license is in
%   http://www.latex-project.org/lppl.txt
% and version 1.3 or later is part of all distributions of LaTeX
% version 2005/12/01 or later.
%
% This work has the LPPL maintenance status `maintained'.
%
% The Current Maintainer of this work is Niklas Beisert.
%
% This work consists of the files childdoc.dtx and childdoc.ins
% and the derived files childdoc.def and cdocsamp.tex with
% cdocsch1.tex, cdocsch2.tex, cdocsdrf.tex, cdocsfn1.tex, cdocsfn2.tex.
%
%<package>\ifdefined\childdocmain\endinput\fi
%<package>\ProvidesFile{childdoc.def}[2018/12/30 v2.0 child document driver]
%<samplemain>\ProvidesFile{cdocsamp.tex}[2018/12/30 v2.0 sample for childdoc]
%<*driver>
%\ProvidesFile{childdoc.drv}[2018/12/30 v2.0 childdoc reference manual file]
\PassOptionsToClass{10pt,a4paper}{article}
\documentclass{ltxdoc}

\usepackage[margin=35mm]{geometry}
\usepackage{hyperref}
\usepackage{hyperxmp}
\usepackage[usenames]{color}

\hypersetup{colorlinks=true}
\hypersetup{pdfstartview=FitH}
\hypersetup{pdfpagemode=UseNone}
\hypersetup{pdfsource={}}
\hypersetup{pdflang={en-UK}}
\hypersetup{pdfcopyright={Copyright 2017-2018 Niklas Beisert.
  This work may be distributed and/or modified under the
  conditions of the LaTeX Project Public License, either version 1.3
  of this license or (at your option) any later version.}}
\hypersetup{pdflicenseurl={http://www.latex-project.org/lppl.txt}}
\hypersetup{pdfcontactaddress={ETH Zurich, ITP, HIT K,
  Wolfgang-Pauli-Strasse 27}}
\hypersetup{pdfcontactpostcode={8093}}
\hypersetup{pdfcontactcity={Zurich}}
\hypersetup{pdfcontactcountry={Switzerland}}
\hypersetup{pdfcontactemail={nbeisert@itp.phys.ethz.ch}}
\hypersetup{pdfcontacturl={http://people.phys.ethz.ch/\xmptilde nbeisert/}}

\newcommand{\secref}[1]{\hyperref[#1]{section \ref*{#1}}}

\parskip1ex
\parindent0pt
\let\olditemize\itemize
\def\itemize{\olditemize\parskip0pt}

\begin{document}

\title{The \textsf{childdoc} Package}
\hypersetup{pdftitle={The childdoc Package}}
\author{Niklas Beisert\\[2ex]
  Institut f\"ur Theoretische Physik\\
  Eidgen\"ossische Technische Hochschule Z\"urich\\
  Wolfgang-Pauli-Strasse 27, 8093 Z\"urich, Switzerland\\[1ex]
  \href{mailto:nbeisert@itp.phys.ethz.ch}
  {\texttt{nbeisert@itp.phys.ethz.ch}}}
\hypersetup{pdfauthor={Niklas Beisert}}
\hypersetup{pdfsubject={Manual for the LaTeX2e Package childdoc}}
\date{30 December 2018, \textsf{v2.0}}
\maketitle

\begin{abstract}\noindent
\textsf{childdoc} is a \LaTeXe{} package
that enables the direct compilation
of document sections included by |\include|
to individual files.
\end{abstract}

\begingroup
\parskip0ex
\tableofcontents
\endgroup

%%%%%%%%%%%%%%%%%%%%%%%%%%%%%%%%%%%%%%%%%%%%%%%%%%%%%%%%%%%%%%%%%%%%%%%%%%%%%%%%
%%%%%%%%%%%%%%%%%%%%%%%%%%%%%%%%%%%%%%%%%%%%%%%%%%%%%%%%%%%%%%%%%%%%%%%%%%%%%%%%
\section{Introduction}

\LaTeX{} provides a mechanism to structure a large document (such as a book)
into a main file and several child files (containing the chapters)
using the |\include| command.
This mechanism is beneficial for documents
which span hundreds of pages in order to
make the source file(s) more manageable.
Moreover, compilation can be restricted to
selected child files by means of the |\includeonly| command.
The latter feature can be used to reduce the compilation time while editing
(this was significantly more useful in the earlier days of \LaTeX{})
or to generate a smaller document which is easier to navigate.
Another application of |\includeonly| is to generate
documents consisting of selected parts of the complete document.

However, there are a few drawbacks of the plain |\include| mechanism:
\begin{itemize}
\item
The child files cannot be compiled on their own,
they can only be compiled via the main file.
A naive editing environment
(such as a text editor with an option
to have the current file processed by \LaTeX)
may require one to switch to the main file before compiling;
attempting to compile the child file produces errors.
\item
The main file must be modified (each time)
to adjust the |\includeonly| command
to the present needs. This easily leaves the main file in a messy state.
\item
The generated document will always carry the filename
of the main document. This is inconvenient if
several child files are to be compiled and
to be kept for distribution.
\end{itemize}

The present package provides a simple interface
to make child files individually compilable by \LaTeX{}.
Compiling a child file then has the same effect as compiling
the main file with an |\includeonly| command
to select the appropriate child.
Moreover the generated document will carry the name of the child
rather than the main file.
This resolves all three above issues.

This feature is meant to make the editing of books,
thesis documents and lecture notes somewhat more convenient.
However, the package can also be used efficiently for
composing a series of documents (such as exercise sheets)
which are typically distributed individually.
It then assists the author in generating the individual documents
(potentially in different versions)
as well as a document containing the collected series.
Another application is in developing style files
or other kinds of included material
where compilation of the style file could redirect
to a sample or test file.

%%%%%%%%%%%%%%%%%%%%%%%%%%%%%%%%%%%%%%%%%%%%%%%%%%%%%%%%%%%%%%%%%%%%%%%%%%%%%%%%
%%%%%%%%%%%%%%%%%%%%%%%%%%%%%%%%%%%%%%%%%%%%%%%%%%%%%%%%%%%%%%%%%%%%%%%%%%%%%%%%
\section{Usage}

First of all, the package \textsf{childdoc} is \emph{not} a standard
\LaTeXe{} |.sty| style file! Therefore it needs to be invoked in
a non-standard way.

%%%%%%%%%%%%%%%%%%%%%%%%%%%%%%%%%%%%%%%%%%%%%%%%%%%%%%%%%%%%%%%%%%%%%%%%%%%%%%%%
\subsection{Included Files}
\label{sec:include}

%%%%%%%%%%%%%%%%%%%%%%%%%%%%%%%%%%%%%%%%
\DescribeMacro{\childdocmain}
To use the package, add the commands
\begin{center}
\begin{tabular}{l}
|\input{childdoc.def}|\\
|\childdocmain{}|\\
\end{tabular}
\end{center}
at the very top of the main \LaTeX{} file,
in particular \emph{before} the |\documentclass| statement!
The argument of |\childdocmain| should be left empty
(but it must be present).

%%%%%%%%%%%%%%%%%%%%%%%%%%%%%%%%%%%%%%%%
\DescribeMacro{\childdocof}
Furthermore, add the commands
\begin{center}
\begin{tabular}{l}
|\input{childdoc.def}|\\
|\childdocof{|\textit{main}|}|\\
\end{tabular}
\end{center}
at the top of every child file \textit{child}
which is included by |\include{|\textit{child}|}|
from within the main file
(or at least for those files to be compiled individually).
The argument \textit{main} must be the filename of the main file.

There are a couple of
considerations in setting up the main and child documents:

%%%%%%%%%%%%%%%%%%%%%%%%%%%%%%%%%%%%%%%%
\paragraph{Restrictions.}

Please note the following restrictions:
\begin{itemize}
\item
|\childdocmain| must be called with one argument \textit{main}
to ensure compatibility with earlier version of the package.
It must either be empty (|\childdocmain{}|)
or precisely match the filename of the main file in which it is specified.
See \secref{sec:detection} for further information.
\item
The filename \textit{main} must be specified without the |.tex| extension.
\item
The filename \textit{main} is case sensitive
(even in case-insensitive file systems)
due to internal string comparison.
\item
The argument \textit{main} should be fully expanded, it cannot be a macro.
\item
Subdirectories and special characters should be avoided in filenames.
\item
The command |\childdocmain{|\textit{main}|}| must be followed by a whitespace.
It should not be followed immediately by another command
or by a comment mark `|%|'.
This is because the \TeX{} parser reads the token immediately following
the argument of |\childdocmain| and puts it
at the beginning of every child section;
however, a white\-space is ignored.
\end{itemize}

%%%%%%%%%%%%%%%%%%%%%%%%%%%%%%%%%%%%%%%%
\paragraph{Content of Main File.}

It is advisable to place all content in the child files included by |\include|.
Any output contained in the main file will appear in all child documents
unless suppressed manually;
it cannot be suppressed automatically by the |\includeonly| directive
and thus should normally be avoided.
A method to include some content in the main file
by means of conditional processing is described in \secref{sec:conditional}.

%%%%%%%%%%%%%%%%%%%%%%%%%%%%%%%%%%%%%%%%
\paragraph{Page Numbering.}

When only a part of the document is compiled,
the appropriate numbering of pages
(as well as other status parameters)
is determined from the |.aux| files.
The latter contain information from previous passes.
However this information needs to propagate through
all intermediate child documents.
Therefore the page numbering in child documents may well
be inconsistent until the complete document is compiled at least once.

A useful (if unconventional) way to always ensure a consistent
page numbering is to restart the numbering in each child document
and denote the pages by `\textit{child}|.|\textit{page}'
where \textit{child} represents the chapter/section number of the child file.
This can be achieved by the command
|\numberwithin{page}{|\textit{child}|}|
of the \textsf{amsmath} package
where \textit{child} can be |chapter| or |section|
depending on the chosen structuring.
Alternatively, one can modify the macro |\thepage| appropriately
and reset the counter |page| at the start of each child file.

%%%%%%%%%%%%%%%%%%%%%%%%%%%%%%%%%%%%%%%%%%%%%%%%%%%%%%%%%%%%%%%%%%%%%%%%%%%%%%%%
\subsection{Conditional Processing}
\label{sec:conditional}

The package provides a mechanism to compile different versions
of a document. To customise the versions further some conditional processing
can come in handy to distinguish which version is being compiled.
The package provides two macros to describe the compilation context:

%%%%%%%%%%%%%%%%%%%%%%%%%%%%%%%%%%%%%%%%
\DescribeMacro{\ifchilddoc}
The conditional |\ifchilddoc| distinguishes between the compilation of
child documents and the main document:
%
\begin{center}
|\ifchilddoc |\textit{child-code}| |[|\||else |\textit{main-code}]| \||fi|
\end{center}

%%%%%%%%%%%%%%%%%%%%%%%%%%%%%%%%%%%%%%%%
\DescribeMacro{\childdocname}
\DescribeMacro{\childdocjob}
The macro |\childdocname| contains the filename (without extension)
of the main or child file being processed.
Note that |\childdocjob| will always contain the name of the main file.

%%%%%%%%%%%%%%%%%%%%%%%%%%%%%%%%%%%%%%%%
\paragraph{Title Page.}

Conditional processing can be used to include a title or banner page
in the main document when proper precautions are taken.
Importantly, the code in the main file should ensure that the page counter
(as well as other status parameters which are stored in the |.aux| files)
takes the same value after the conditional processing.
Otherwise the page numbers may take divergent values
depending on which part is compiled.

For example, a title page could be declared by:
%
\begin{center}
\begin{tabular}{l}
|\ifchilddoc\||else|\\
|\addtocounter{page}{-1}|\\
\textit{code for title page}\\
|\newpage|\\
|\||fi|
\end{tabular}
\end{center}
%
A banner page for the child documents can be generated by:
%
\begin{center}
\begin{tabular}{l}
|\ifchilddoc|\\
|\addtocounter{page}{-1}|\\
\textit{code for banner page}\\
|\newpage|\\
|\||fi|
\end{tabular}
\end{center}
%
Here one could write a message such as:
\begin{center}
|This is the part \childdocname{} of \childdocjob{}.|
\end{center}

%%%%%%%%%%%%%%%%%%%%%%%%%%%%%%%%%%%%%%%%%%%%%%%%%%%%%%%%%%%%%%%%%%%%%%%%%%%%%%%%
\subsection{Flags}
\label{sec:flags}

The package makes it easy to generate different versions
of the main or child documents.
To this end compilation flags can be defined
and assigned different default values.
They will be particularly useful in conjunction
with the forwarding mechanism described in \secref{sec:forward}.

For example, it may be useful to have a flag |\version|
which can be set to |draft| or |final|.
The document source will contain some conditional code
depending on the value of |\version|.
Suppose further, the flag should default to |final| for the main file
and to |draft| for child files
which is a natural assignment for editing the document.
This is achieved by placing the following code
in the preamble of the main document
(below the |\childdocmain| directive):
%
\begin{center}
\begin{tabular}{l}
|\ifchilddoc|\\
|\providecommand{\version}{draft}|\\
|\||else|\\
|\providecommand{\version}{final}|\\
|\||fi|
\end{tabular}
\end{center}
%
The definition by |\providecommand| makes sure
that previous definitions are not overwritten.
Further statements |\providecommand{\version}{...}|
can thus be added before the above code to override it.

For the main file, one might add a line
(between |\childdocmain| and the above block)
%
\begin{center}
|%\ifchilddoc\||else\providecommand{\version}{draft}\||fi|
\end{center}
%
which can be uncommented to produce a draft version.
Likewise one can add a line to the very top of a child file
(above the |\childdocof{|\textit{main}|}| directive)
%
\begin{center}
|%\providecommand{\version}{final}|
\end{center}
%
which can be uncommented to produce the final version of this child document.

%%%%%%%%%%%%%%%%%%%%%%%%%%%%%%%%%%%%%%%%%%%%%%%%%%%%%%%%%%%%%%%%%%%%%%%%%%%%%%%%
\subsection{Forwarding}
\label{sec:forward}

Different versions of the main or child documents
using compilation flags as described in \secref{sec:flags}
can be (permanently) stored in different files
for convenient compilation, viewing and distribution.
To this end, the package defines a command
to pass on compilation to a different file:

%%%%%%%%%%%%%%%%%%%%%%%%%%%%%%%%%%%%%%%%
\DescribeMacro{\childdocforward}
The command |\childdocforward| redirects processing to
another source file:
%
\begin{center}
\begin{tabular}{l}
|\input{childdoc.def}|\\
|\childdocforward[|\textit{main}|]{|\textit{dest}|}|\\
\end{tabular}
\end{center}
%
The argument \textit{dest} is the destination file
(without extension).
It should be the main file or one of the child files.
Note that further \textsf{childdoc} directives
such as |\childdocof| and |\childdocforward|
in the indicated file will be processed in this form.
The optional argument \textit{main}
passes on directly to the main file \textit{main}
while pretending to compile the child \textit{dest}.
This form behaves as if \textit{dest}
issues |\childdocof{|\textit{main}|}| right away,
and no further \textsf{childdoc} directives will be processed.

%%%%%%%%%%%%%%%%%%%%%%%%%%%%%%%%%%%%%%%%
\DescribeMacro{\...prefix}
In the alternative form |\childdocforwardprefix|,
%
\begin{center}
\begin{tabular}{l}
|\input{childdoc.def}|\\
|\childdocforwardprefix[|\textit{main}|]{|\textit{prefix}|}{|\textit{dest}|}|
\end{tabular}
\end{center}
%
the destination file is determined by a pattern
depending on the current file:
To make this work, the current file must be called
`{\textit{prefix}\hspace{0.2em}\textit{suffix}}'
with \textit{prefix} matching precisely the argument.
Processing is then passed on to the file
`{\textit{dest}\hspace{0.2em}\textit{suffix}}'.
Surely, the same effect is achieved by
directly specifying the
argument `{\textit{dest}\hspace{0.2em}\textit{suffix}}'
in the first form.
However, that requires to set up a different file
for each child. With the alternative form of the command
all these files can have exactly the same content
which simplifies setting them up and maintaining them.

For example, the following file |draft.tex|
with a compilation flag |\version| as described in \secref{sec:flags}
compiles the main document as a draft:
%
\begin{center}
\begin{tabular}{l}
|\def\version{draft}|\\
|\input{childdoc.def}|\\
|\childdocforward{|\textit{main}|}|
\end{tabular}
\end{center}
%
Likewise, the following files |final|\textit{nn}|.tex|
compile the final version of the child document
|child|\textit{nn}|.tex|:
%
\begin{center}
\begin{tabular}{l}
|\def\version{final}|\\
|\input{childdoc.def}|\\
|\childdocforwardprefix{final}{child}|
\end{tabular}
\end{center}
%

Note that when several versions of a main file and/or of each child file
are to be generated, it may be convenient to set up a |Makefile| or
shell script to automatise the process.

%%%%%%%%%%%%%%%%%%%%%%%%%%%%%%%%%%%%%%%%%%%%%%%%%%%%%%%%%%%%%%%%%%%%%%%%%%%%%%%%
\subsection{Command Line Processing}
\label{sec:commandline}

The effect of redirection files can also be achieved by invoking
the \LaTeX{} compiler with a more elaborate command line.
Most conveniently this should be done as part
of a shell script or a |Makefile|.

When using \textsf{childdoc} in the main file, the following
command lines effectively perform a redirection
(note that depending on the shell being used,
backslashes may have to be doubled: `|\|' $\to$ `|\\|'):
%
\begin{center}
|... -jobname "|\textit{target}|" |\\|"|[\textit{flags}]%
|\input{childdoc.def}\childdocforward[|\textit{main}|]{|\textit{dest}|}"|
\end{center}
%
Here \textit{target} is the name of the output file,
\textit{main} is the name of the main file
and \textit{dest} is the name of the main or child file to be processed
(all filenames without extensions).
The optional argument \textit{main} can be omitted
if \textit{main} matches \textit{dest}.
Optionally, compilation \textit{flags} can be defined via |\def| commands.
This command line makes the \TeX{} engine believe
it is compiling the file \textit{target}
whose content is specified as the latter parameter.
The provided code then forwards the processing to
\textit{main} or \textit{dest} as described in \secref{sec:forward}.

%%%%%%%%%%%%%%%%%%%%%%%%%%%%%%%%%%%%%%%%%%%%%%%%%%%%%%%%%%%%%%%%%%%%%%%%%%%%%%%%
\subsection{Include by Input}
\label{sec:input}

Including child documents by |\include| has some restrictions by design.
Most notably, the content of a child document always occupies
its own set of pages; pages cannot be shared between child documents.
Usually, this behaviour makes perfect sense
because each child document contain an essential part of the document.
However, in some situations it may be desirable to compose
a document from a collection of parts
without having mandatory page breaks between then.
For this case, the package
provides a mechanism to include parts
by |\input| which can also be processed individually.
However, by construction this mechanism
requires manual handling of the content to be output.

%%%%%%%%%%%%%%%%%%%%%%%%%%%%%%%%%%%%%%%%
\DescribeMacro{\ifchilddocmanual}
The main file should be prepared as usual, see \secref{sec:include}.
However, the document body must make a distinction
between processing of an individual part and of the main document, e.g.:
%
\begin{center}
\begin{tabular}{l}
|\ifchilddocmanual|\\
|\input{\childdocname}|\\
|\||else|\\
\textit{document body with }|\input{|\textit{part}|}|\\
|\||fi|
\end{tabular}
\end{center}
%
The conditional |\ifchilddocmanual| is true whenever
a part to be included by |\input| is being compiled,
and the name of the part is stored in |\childdocname|.

%%%%%%%%%%%%%%%%%%%%%%%%%%%%%%%%%%%%%%%%
\DescribeMacro{\childdocby}
Each part to be included by |\input| should start with:
%
\begin{center}
\begin{tabular}{l}
|\input{childdoc.def}|\\
|\childdocby{|\textit{main}|}|\\
\end{tabular}
\end{center}
%
The directive |\childdocby| is similar to |\childdocof|
described in \secref{sec:include},
but the subsequent selection of content must be done manually.
To that end, both |\ifchilddoc| and |\ifchilddocmanual|
will be true upon processing of a part,
and the name of the part is stored in |\childdocname|.
Note that |\jobname| will be set to the filename of the current part
so that each part receives an individual |.aux| file
that does not interfere with the |.aux| file(s) of the main document.
This behaviour can be altered by the alternative form
|\childdocby[*]{|\textit{main}|}| (with a non-empty optional argument)
which uses the |.aux| file of the main document
by setting |\jobname| to \textit{main}.

%%%%%%%%%%%%%%%%%%%%%%%%%%%%%%%%%%%%%%%%%%%%%%%%%%%%%%%%%%%%%%%%%%%%%%%%%%%%%%%%
\subsection{Driver Development}
\label{sec:driver}

The \textsf{childdoc} mechanism can also be use for the development
of definition files such as \LaTeX{} styles or classes.
This case differs from the above setup with multiple parts
included by |\include| in that no |\includeonly| should be invoked.
This can be achieved by starting the include file
(before |\ProvidesPackage|) with:
%
\begin{center}
\begin{tabular}{l}
|\input{childdoc.def}|\\
|\childdocforward{|\textit{main}|}|\\
\end{tabular}
\end{center}
%
or alternatively with:
%
\begin{center}
\begin{tabular}{l}
|\input{childdoc.def}|\\
|\childdocby{|\textit{main}|}|\\
\end{tabular}
\end{center}
%
Both forms have slightly different effects as described above.
The main file is prepared as usual, see \secref{sec:include}.

%%%%%%%%%%%%%%%%%%%%%%%%%%%%%%%%%%%%%%%%%%%%%%%%%%%%%%%%%%%%%%%%%%%%%%%%%%%%%%%%
\subsection{Legacy Detection}
\label{sec:detection}

The directive |\childdocmain| in the main file can detect
whether the complete document or merely a child is to be compiled
even without using the directive |\childdocof|.
This method is deprecated because it is less robust
and there is no compelling reason to use it;
it is merely provided for backward compatibility
and it may be removed in future versions.

If the detection mechanism is to be used,
it is mandatory to correctly specify
the filename of the main file as the argument of |\childdocmain|:
%
\begin{center}
\begin{tabular}{l}
|\input{childdoc.def}|\\
|\childdocmain{|\textit{main}|}|\\
\end{tabular}
\end{center}
%
If |\jobname| does not match the argument \textit{main} of |\childdocmain|,
it is assumed that |\jobname| points to the child file to be compiled.
When using |\childdocmain| with the main file specified as argument,
it suffices to start a child file
with just |\input{|\textit{main}|}|
without loading of the package and using |\childdocof|.
If instead all processing is done
with the appropriate \textsf{childdoc} directives,
the argument of \textit{main} of |\childdocmain| can be empty.

An alternative version of the command line processing described
in \secref{sec:commandline} using the detection mechanism reads:
%
\begin{center}
|... -jobname "|\textit{target}|" "|[\textit{flags}]%
[|\def\jobname{|\textit{dest}|}|]|\input{|\textit{main}|}"|
\end{center}

%%%%%%%%%%%%%%%%%%%%%%%%%%%%%%%%%%%%%%%%%%%%%%%%%%%%%%%%%%%%%%%%%%%%%%%%%%%%%%%%
\subsection{Manual Code}
\label{sec:manual}

In case one cannot be certain whether the definitions file |childdoc.def|
is installed on the target \TeX{} distribution
and one prefers not to ship it,
it is conceivable to paste a few relevant commands into the sources.

To that end, drop all statements |\input{childdoc.def}|
and perform the replacements as outlined below.
Instead of |\childdocmain{|\textit{main}|}| add the following code
to the top of the main file:
%
\begin{center}
\begin{tabular}{l}
|\||ifdefined\childdocname\endinput\||fi\newif\ifchilddoc|\\
|\edef\childdocname{\scantokens\expandafter{\jobname\noexpand}}|\\
|\def\childdocmain{|\textit{main}|}\||ifx\childdocmain\childdocname\||else|\\
|\childdoctrue\includeonly{\childdocname}\let\jobname\childdocmain\||fi|\\
\end{tabular}
\end{center}
%
Instead of |\childdocof{|\textit{main}|}| just include the main file
at the top of each child file:
%
\begin{center}
|\input{|\textit{main}|}|
\end{center}
%
A simple redirection |\childdocforward{|\textit{dest}|}| is achieved by:
%
\begin{center}
|\def\jobname{|\textit{dest}|}\input{\jobname}|
\end{center}
%
The redirection with prefix
|\childdocforwardprefix[|\textit{prefix}|]{|\textit{dest}|}|
is accomplished by:
%
\begin{center}
\begin{tabular}{l}
|{\edef\jobname{\scantokens\expandafter{\jobname\noexpand}}|\\
|\def\redirectjob |\textit{prefix}|#1~~~{\gdef\jobname{|\textit{dest}|#1}}|\\
|\expandafter\redirectjob\jobname~~~}\input{\jobname}|
\end{tabular}
\end{center}

In an alternative approach,
child documents can be compiled by a specific command line
without additional code or specific definitions:
%
\begin{center}
|... -jobname "|\textit{target}|" "|[\textit{flags}]%
|\includeonly{|\textit{dest}|}\input{|\textit{main}|}"|
\end{center}
%

%%%%%%%%%%%%%%%%%%%%%%%%%%%%%%%%%%%%%%%%%%%%%%%%%%%%%%%%%%%%%%%%%%%%%%%%%%%%%%%%
%%%%%%%%%%%%%%%%%%%%%%%%%%%%%%%%%%%%%%%%%%%%%%%%%%%%%%%%%%%%%%%%%%%%%%%%%%%%%%%%
\section{Information}

%%%%%%%%%%%%%%%%%%%%%%%%%%%%%%%%%%%%%%%%%%%%%%%%%%%%%%%%%%%%%%%%%%%%%%%%%%%%%%%%
\subsection{Copyright}

Copyright \copyright{} 2017--2018 Niklas Beisert

This work may be distributed and/or modified under the
conditions of the \LaTeX{} Project Public License, either version 1.3
of this license or (at your option) any later version.
The latest version of this license is in
  \url{http://www.latex-project.org/lppl.txt}
and version 1.3 or later is part of all distributions of \LaTeX{}
version 2005/12/01 or later.

This work has the LPPL maintenance status `maintained'.

The Current Maintainer of this work is Niklas Beisert.

This work consists of the files |README.txt|, |childdoc.ins| and |childdoc.dtx|
as well as the derived files |childdoc.def|, |cdocsamp.tex|
with |cdocsch1.tex|, |cdocsch2.tex|, |cdocspt3.tex|, |cdocspt4.tex|,
|cdocsdrf.tex|, |cdocsfn1.tex|, |cdocsfn2.tex|
as well as |childdoc.pdf|.

%%%%%%%%%%%%%%%%%%%%%%%%%%%%%%%%%%%%%%%%%%%%%%%%%%%%%%%%%%%%%%%%%%%%%%%%%%%%%%%%
\subsection{Files and Installation}

The package consists of the files:
%
\begin{center}
\begin{tabular}{ll}
    |README.txt|   & readme file \\
    |childdoc.ins| & installation file \\
    |childdoc.dtx| & source file \\
    |childdoc.def| & definition file \\
    |cdocsamp.tex| & sample main file \\
    |cdocsch1.tex| & sample include file \\
    |cdocsch2.tex| & sample include file \\
    |cdocspt3.tex| & sample part file \\
    |cdocspt4.tex| & sample part file \\
    |cdocsdrf.tex| & sample redirection file \\
    |cdocsfn1.tex| & sample redirection file \\
    |cdocsfn2.tex| & sample redirection file \\
    |childdoc.pdf| & manual
\end{tabular}
\end{center}
%
The distribution consists of the files
|README.txt|, |childdoc.ins| and |childdoc.dtx|.
%
\begin{itemize}
\item
Run (pdf)\LaTeX{} on |childdoc.dtx|
to compile the manual |childdoc.pdf| (this file).
\item
Run \LaTeX{} on |childdoc.ins| to create the definitions file |childdoc.def|
and the sample |cdocsamp.tex| with include files
|cdocsch1.tex|, |cdocsch2.tex|, |cdocspt3.tex|, |cdocspt4.tex|,
|cdocsdrf.tex|, |cdocsfn1.tex|, |cdocsfn2.tex|.
Then copy the file |childdoc.def| to an appropriate directory of your \LaTeX{}
distribution, e.g.\ \textit{texmf-root}|/tex/latex/childdoc|.
\end{itemize}

%%%%%%%%%%%%%%%%%%%%%%%%%%%%%%%%%%%%%%%%%%%%%%%%%%%%%%%%%%%%%%%%%%%%%%%%%%%%%%%%
\subsection{Related CTAN Packages}

There are several other packages which offer a similar functionality:
%
\begin{itemize}
\item
The packages
\href{http://ctan.org/pkg/docmute}{\textsf{docmute}},
\href{http://ctan.org/pkg/includex}{\textsf{includex}} and
\href{http://ctan.org/pkg/standalone}{\textsf{standalone}}
provide commands to include only the document body of
a child file thus allowing both files to be compiled individually.
\item
The packages \href{http://ctan.org/pkg/subdocs}{\textsf{subdocs}}
and \href{http://ctan.org/pkg/subfiles}{\textsf{subfiles}}
provide structures in which the main and child documents can be
encapsulated and allowing them to be compiled individually.
The inclusion mechanism is different from the conventional |\include|.
\item
The package \href{http://ctan.org/pkg/combine}{\textsf{combine}}
is an elaborate solution to combine several documents into one.
\end{itemize}
%
See also the CTAN topic \href{http://ctan.org/topic/subdocs}{\textsf{subdocs}}
for further related packages.
The present package differs from the above solutions in that
a document structure constructed with the conventional |\include| mechanism
just needs two extra commands at the top of every file
such that all constituent files can be compiled individually.

%%%%%%%%%%%%%%%%%%%%%%%%%%%%%%%%%%%%%%%%%%%%%%%%%%%%%%%%%%%%%%%%%%%%%%%%%%%%%%%%
%\subsection{Feature Suggestions}
%
%The following is a list of features which may be useful for future
%versions of this package:
%%
%\begin{itemize}
%\item
%\ldots
%\end{itemize}

%%%%%%%%%%%%%%%%%%%%%%%%%%%%%%%%%%%%%%%%%%%%%%%%%%%%%%%%%%%%%%%%%%%%%%%%%%%%%%%%
\subsection{Revision History}

%%%%%%%%%%%%%%%%%%%%%%%%%%%%%%%%%%%%%%%%
\paragraph{v2.0:} 2018/12/30

\begin{itemize}
\item
immediate forward processing
\item
added |\childdocby| mechanism
\item
manual restructured
\end{itemize}

%%%%%%%%%%%%%%%%%%%%%%%%%%%%%%%%%%%%%%%%
\paragraph{v1.6:} 2018/01/17

\begin{itemize}
\item
application for development of include files
\item
corrections to manual
\end{itemize}

%%%%%%%%%%%%%%%%%%%%%%%%%%%%%%%%%%%%%%%%
\paragraph{v1.5:} 2017/05/21

\begin{itemize}
\item
more complete structuring introduced
\item
|\childdocof| introduced
\item
|\childdoc| renamed to |\childdocmain|
\item
|\childredirect| renamed to |\childdocforward| and |\childdocforwardprefix|
and functionality expanded
\end{itemize}

%%%%%%%%%%%%%%%%%%%%%%%%%%%%%%%%%%%%%%%%
\paragraph{v1.0:} 2017/04/27

\begin{itemize}
\item
manual and install package
\item
first version published on CTAN
\end{itemize}

%%%%%%%%%%%%%%%%%%%%%%%%%%%%%%%%%%%%%%%%
\paragraph{v0.6:} 2017/04/26

\begin{itemize}
\item
redirection mechanism added
\end{itemize}

%%%%%%%%%%%%%%%%%%%%%%%%%%%%%%%%%%%%%%%%
\paragraph{v0.5:} 2017/04/26

\begin{itemize}
\item
functionality in definition file
\end{itemize}


%%%%%%%%%%%%%%%%%%%%%%%%%%%%%%%%%%%%%%%%%%%%%%%%%%%%%%%%%%%%%%%%%%%%%%%%%%%%%%%%
%%%%%%%%%%%%%%%%%%%%%%%%%%%%%%%%%%%%%%%%%%%%%%%%%%%%%%%%%%%%%%%%%%%%%%%%%%%%%%%%
%%%%%%%%%%%%%%%%%%%%%%%%%%%%%%%%%%%%%%%%%%%%%%%%%%%%%%%%%%%%%%%%%%%%%%%%%%%%%%%%
\appendix

\settowidth\MacroIndent{\rmfamily\scriptsize 000\ }

 \DocInput{childdoc.dtx}

\end{document}
%</driver>
% \fi
%
% %%%%%%%%%%%%%%%%%%%%%%%%%%%%%%%%%%%%%%%%%%%%%%%%%%%%%%%%%%%%%%%%%%%%%%%%%%%%%%
% %%%%%%%%%%%%%%%%%%%%%%%%%%%%%%%%%%%%%%%%%%%%%%%%%%%%%%%%%%%%%%%%%%%%%%%%%%%%%%
% \section{Sample}
%\iffalse
%<*samplemain>
%\fi
%
% The following presents a sample document
% with two chapters, two parts, a title page,
% a compile flag as well as three forwarding files to set the flag.
% It consists of eight |.tex| files:
% \begin{center}
% \begin{tabular}{ll}
% |cdocsamp.tex|&main file\\
% |cdocsch1.tex|&include file for chapter 1\\
% |cdocsch2.tex|&include file for chapter 2\\
% |cdocspt3.tex|&include file for part 3\\
% |cdocspt4.tex|&include file for part 4\\
% |cdocsdrf.tex|&forwarding file for main file in draft mode\\
% |cdocsfi1.tex|&forwarding file for final version of chapter 1\\
% |cdocsfi2.tex|&forwarding file for final version of chapter 2\\
% \end{tabular}
% \end{center}
% Each of the eight files can be compiled directly by the \LaTeX{} compiler.
%
% %%%%%%%%%%%%%%%%%%%%%%%%%%%%%%%%%%%%%%
% \paragraph{Main File.}
%
% The main file is called |cdocsamp.tex|.
%
% Load the \textsf{childdoc} definitions and
% declare the filename for the main document:
%    \begin{macrocode}
\input{childdoc.def}
\childdocmain{}
%    \end{macrocode}

% Optional override for |\version| flag:
%    \begin{macrocode}
%%\ifchilddoc\else\providecommand{\version}{draft}\fi
%    \end{macrocode}

% Define the default values for the |\version| flag
% (|final| for the main file and |draft| for childs):
%    \begin{macrocode}
\ifchilddoc
\providecommand{\version}{draft}
\else
\providecommand{\version}{final}
\fi
%    \end{macrocode}

% Load the standard document class:
%    \begin{macrocode}
\documentclass[12pt]{article}
%    \end{macrocode}

% Start the document body:
%    \begin{macrocode}
\begin{document}
%    \end{macrocode}

% Declare a title page.
% Print title, part of document being processed and version flag:
%    \begin{macrocode}
\addtocounter{page}{-1}
\begin{center}
{\LARGE\bfseries{}childdoc example\par}
\vspace{1cm}
\ifchilddoc
\ifchilddocmanual part\else chapter\fi:
`\childdocname' of `\childdocjob'\par
\else
main document: `\childdocjob'\par
\fi
version: \version\par
\end{center}
\newpage
%    \end{macrocode}

% Manually include selected file,
% otherwise process as usual:
%    \begin{macrocode}
\ifchilddocmanual
\section*{part `\childdocname'}
\input{\childdocname}
\else
%    \end{macrocode}

% Include the two chapters:
%    \begin{macrocode}
\include{cdocsch1}
\include{cdocsch2}
%    \end{macrocode}

% Include the two parts unless only chapters should be displayed:
%    \begin{macrocode}
\ifchilddoc\else
\section{part three}
\input{cdocspt3}
\section{part four}
\input{cdocspt4}
\fi
%    \end{macrocode}

% Process as usual until here:
%    \begin{macrocode}
\fi
%    \end{macrocode}

% End of document body:
%    \begin{macrocode}
\end{document}
%    \end{macrocode}
%\iffalse
%</samplemain>
%\fi
%
% %%%%%%%%%%%%%%%%%%%%%%%%%%%%%%%%%%%%%%
% \paragraph{Chapter Include Files.}
%
% The include files are called |cdocsch1.tex| and |cdocsch2.tex|.
%
%\iffalse
%<*samplechap1|samplechap2>
%\fi

% Optional override for |\version| flag:
%    \begin{macrocode}
%%\providecommand{\version}{final}
%    \end{macrocode}

% Include the main document:
%    \begin{macrocode}
\input{childdoc.def}
\childdocof{cdocsamp}
%    \end{macrocode}

%\iffalse
%</samplechap1|samplechap2>
%\fi
%
%\iffalse
%<*samplechap1>
%\fi
% Some text for chapter 1:
%    \begin{macrocode}
\section{one}
some text in chapter one
%    \end{macrocode}

%\iffalse
%</samplechap1>
%\fi
% Some text for chapter 2:
%\iffalse
%<*samplechap2>
%\fi
%    \begin{macrocode}
\section{two}
more text in chapter two
%    \end{macrocode}

%\iffalse
%</samplechap2>
%\fi
%
% %%%%%%%%%%%%%%%%%%%%%%%%%%%%%%%%%%%%%%
% \paragraph{Part Include Files.}
%
% The include files are called |cdocspt3.tex| and |cdocspt4.tex|.
%
%\iffalse
%<*samplepart3|samplepart4>
%\fi

% Optional override for |\version| flag:
%    \begin{macrocode}
%%\providecommand{\version}{final}
%    \end{macrocode}

% Include the main document:
%    \begin{macrocode}
\input{childdoc.def}
\childdocby{cdocsamp}
%    \end{macrocode}

%\iffalse
%</samplepart3|samplepart4>
%\fi
%
%\iffalse
%<*samplepart3>
%\fi
% Some text for part 3:
%    \begin{macrocode}
some text in part three
%    \end{macrocode}

%\iffalse
%</samplepart3>
%\fi
% Some text for part 4:
%\iffalse
%<*samplepart4>
%\fi
%    \begin{macrocode}
more text in part four
%    \end{macrocode}

%\iffalse
%</samplepart4>
%\fi
%
% %%%%%%%%%%%%%%%%%%%%%%%%%%%%%%%%%%%%%%
% \paragraph{Forwarding for a Complete Draft.}
%
% The following forwarding file |cdocsdrf.tex|
% compiles the main document in draft mode:
%\iffalse
%<*sampledraft>
%\fi
%    \begin{macrocode}
\def\version{draft}
\input{childdoc.def}
\childdocforward{cdocsamp}
%    \end{macrocode}

%\iffalse
%</sampledraft>
%\fi
%
% %%%%%%%%%%%%%%%%%%%%%%%%%%%%%%%%%%%%%%
% \paragraph{Forwarding for Final Version of the Chapters.}
%
% The following forwarding files |cdocsfn1.tex| and |cdocsfn2.tex|
% (with identical content)
% compile the final versions of the child documents
% |cdocsch1.tex| and |cdocsch2.tex|, respectively:
%\iffalse
%<*samplefinal>
%\fi
%    \begin{macrocode}
\def\version{final}
\input{childdoc.def}
\childdocforwardprefix[cdocsamp]{cdocsfn}{cdocsch}
%    \end{macrocode}

%\iffalse
%</samplefinal>
%\fi
%
% %%%%%%%%%%%%%%%%%%%%%%%%%%%%%%%%%%%%%%
% \paragraph{Command Line Processing.}
%
% The following three command lines generate the output files
% |cdocscld|, |cdocscl1| and |cdocscl2|
% which should be identical to
% |cdocsdrf|, |cdocsch1| and |cdocsfn2|, respectively:
% \begin{center}
% \begin{tabular}{l}
% |latex -jobname cdocscld \|\\
% |  "\def\version{draft}\input{childdoc.def}\childdocforward{cdocsamp}"|\\
% |latex -jobname cdocscl1 \|\\
% |  "\input{childdoc.def}\childdocforward[cdocsamp]{cdocsch1}"|\\
% |latex -jobname cdocscl2 \|\\
% |  "\def\version{final}\input{childdoc.def}\childdocforward{cdocsch2}"|
% \end{tabular}
% \end{center}
% Note that the trailing backslash on each first line
% merely continues the input to the second line
% (for convenient cut ant paste).
% Furthermore, the command |latex| can be replaced by any
% of its alternative versions such as |pdflatex|.
%
% %%%%%%%%%%%%%%%%%%%%%%%%%%%%%%%%%%%%%%%%%%%%%%%%%%%%%%%%%%%%%%%%%%%%%%%%%%%%%%
% %%%%%%%%%%%%%%%%%%%%%%%%%%%%%%%%%%%%%%%%%%%%%%%%%%%%%%%%%%%%%%%%%%%%%%%%%%%%%%
% \section{Implementation}
%\iffalse
%<*package>
%\fi
%
% This section describes the definitions file |childdoc.def|.

% The definitions cannot be loaded using |\usepackage| or |\RequirePackage|
% which has a mechanism to prevent loading a style file more than once.
% When loading the definitions by means of |\input|
% multiple instances have to be prevented manually:
%\iffalse
%This code needs to be before the `\ProvidesFile' directive
%which is defined at the beginning of this file.
%Therefore it is also placed there and commented out here.
%</package>
%<*discard>
%\fi
%    \begin{macrocode}
\ifdefined\childdocmain\endinput\fi
%    \end{macrocode}
%\iffalse
%</discard>
%<*package>
%\fi
%
% \macro{\ifchilddoc}
% \macro{\ifchilddocmanual}
% The conditional |\ifchilddoc| tells whether a
% child (true) or main (false) document is being compiled.
% The conditional |\ifchilddocmanual| tells whether
% the |\includeonly| mechanism is used (false) or
% the selection of child files must be performed manually (true).
% The definitions initialise to false:
%    \begin{macrocode}
\newif\ifchilddoc
\newif\ifchilddocmanual
%    \end{macrocode}

% \macro{\childdocname}
% \macro{\childdocjob}
% The macro |\childdocname| stores the name of the main document
% to be compiled. The macro |\childdocjob| stores the name of
% the document on which the \LaTeX{} compiler was originally invoked.
% The content of |\jobname| cannot be compared
% to filenames specified in the source due to different catcodes.
% The following code rescans |\jobname|, stores the result
% in |\childdocname| and saves a copy in |\childdocjob|:
%    \begin{macrocode}
\edef\childdocname{\scantokens\expandafter{\jobname\noexpand}}
\let\childdocjob\childdocname
%    \end{macrocode}

% \macro{\childdocdisable}
% The macro |\childdocdisable| prevents the main file
% from being processed more than once.
% At this stage, the main document command |\childdocmain|
% is assumed to be called once again where it should do nothing.
% Any subsequent call to it should prevent
% a secondary processing of the main document
% It overwrites the forwarding commands
% |\childdocof| and |\childdocforward|
% with empty macros to prevent further inclusions of the main document:
%    \begin{macrocode}
\newcommand{\childdocdisable}
{
  \renewcommand{\childdocmain}[1]{\renewcommand{\childdocmain}[1]{\endinput}}
  \renewcommand{\childdocof}[1]{}
  \renewcommand{\childdocby}[2][]{}
  \renewcommand{\childdocforward}[2][]{}
  \renewcommand{\childdocdisable}{}
}
%    \end{macrocode}

% \macro{\childdocmain}
% The macro |\childdocmain| is to be called at the top of the main file
% with nothing or the main filename (without extension) as argument.
% First, it breaks loops.
% If the argument is not empty and does not match |\childdocname|
% (which is set by the first inclusion of |childdoc.def|),
% |\ifchilddoc| is set to true, |\includeonly| is applied to the child file
% and |\jobname| is set to the main file
% (for proper handling of |.aux| files):
%    \begin{macrocode}
\newcommand{\childdocmain}[1]
{
  \childdocdisable\childdocmain{}
  \if?#1?\else
    \begingroup
      \def\childdoctmp{#1}
      \ifx\childdoctmp\childdocname
        \def\childdoctmp{}
      \else
        \def\childdoctmp
        {
          \childdoctrue
          \includeonly{\childdocname}
          \def\childdocjob{#1}
          \def\jobname{#1}
        }
      \fi
      \expandafter
    \endgroup
    \childdoctmp
  \fi
}
%    \end{macrocode}

% \macro{\childdocof}
% The command |\childdocof| redirects
% compilation to the main file |#1|.
%    \begin{macrocode}
\newcommand{\childdocof}[1]
{
  \childdocdisable
  \childdoctrue
  \includeonly{\childdocname}
  \def\jobname{#1}
  \def\childdocjob{#1}
  \input{#1}
}
%    \end{macrocode}

% \macro{\childdocby}
% The command |\childdocby| ....
%    \begin{macrocode}
\newcommand{\childdocby}[2][]
{
  \childdocdisable
  \childdoctrue
  \childdocmanualtrue
  \if?#1?\else
    \def\jobname{#2}
  \fi
  \def\childdocjob{#2}
  \input{#2}
  \endinput
}
%    \end{macrocode}

% \macro{\childdocforward}
% The command |\childdocforward| redirects
% compilation to the main file or
% (if the optional argument is given) a child file.
% Parameters are set as if the main file
% or a child file starting with |\childdocof| was compiled.
% Then compilation is handed over to the main file:
%    \begin{macrocode}
\newcommand{\childdocforward}[2][]
{
  \begingroup
    \if?#1?
      \def\childdoctmp
      {
        \def\childdocname{#2}
        \def\childdocjob{#2}
        \def\jobname{#2}
        \input{#2}
        \endinput
      }
    \else
      \def\childdoctmp
      {
        \childdocdisable
        \def\childdocname{#2}
        \childdoctrue
        \includeonly{#2}
        \def\childdocjob{#1}
        \def\jobname{#1}
        \input{#1}
        \endinput
      }
    \fi
    \expandafter
  \endgroup
  \childdoctmp
}
%    \end{macrocode}

% \macro{\childdocforwardprefix}
% The command |\childdocforwardprefix| redirects
% compilation to the main or a child file by means of a pattern.
% The prefix |#1| in the current filename is replaced by |#2|
% and the suffix of the current filename is kept
% (it is assumed that the filename does not contain the substring `|~~~|'
% which is used as a delimiter).
% Compilation is handed over to the new file by |\childdocforward|:
%    \begin{macrocode}
\newcommand{\childdocforwardprefix}[3][]
{
  \begingroup
    \def\childdocextract #2##1~~~{\def\childdoctmp{\childdocforward[#1]{#3##1}}}
    \expandafter\childdocextract\childdocname~~~
    \expandafter
  \endgroup
  \childdoctmp
}
%    \end{macrocode}

% \macro{\childdoc}
% The deprecated macro |\childdoc| is a legacy version of |\childdocmain|:
%    \begin{macrocode}
\newcommand{\childdoc}{\childdocmain}
%    \end{macrocode}

% \macro{\childdocredirect}
% The deprecated macro |\childdocredirect| is a legacy version
% of |\childdocforward| and |\childdocforwardprefix|:
%    \begin{macrocode}
\newcommand{\childdocredirect}[2][]
{
  \begingroup
    \if?#1?
      \def\childdoctmp{\childdocforward{#2}}
    \else
      \def\childdoctmp{\childdocforwardprefix{#1}{#2}}
    \fi
    \expandafter
  \endgroup
  \childdoctmp
}
%    \end{macrocode}

%\iffalse
%</package>
%\fi
%
\endinput
|\\
|\childdocforwardprefix[|\textit{main}|]{|\textit{prefix}|}{|\textit{dest}|}|
\end{tabular}
\end{center}
%
the destination file is determined by a pattern
depending on the current file:
To make this work, the current file must be called
`{\textit{prefix}\hspace{0.2em}\textit{suffix}}'
with \textit{prefix} matching precisely the argument.
Processing is then passed on to the file
`{\textit{dest}\hspace{0.2em}\textit{suffix}}'.
Surely, the same effect is achieved by
directly specifying the
argument `{\textit{dest}\hspace{0.2em}\textit{suffix}}'
in the first form.
However, that requires to set up a different file
for each child. With the alternative form of the command
all these files can have exactly the same content
which simplifies setting them up and maintaining them.

For example, the following file |draft.tex|
with a compilation flag |\version| as described in \secref{sec:flags}
compiles the main document as a draft:
%
\begin{center}
\begin{tabular}{l}
|\def\version{draft}|\\
|% \iffalse
%
% childdoc.dtx Copyright (C) 2017-2018 Niklas Beisert
%
% This work may be distributed and/or modified under the
% conditions of the LaTeX Project Public License, either version 1.3
% of this license or (at your option) any later version.
% The latest version of this license is in
%   http://www.latex-project.org/lppl.txt
% and version 1.3 or later is part of all distributions of LaTeX
% version 2005/12/01 or later.
%
% This work has the LPPL maintenance status `maintained'.
%
% The Current Maintainer of this work is Niklas Beisert.
%
% This work consists of the files childdoc.dtx and childdoc.ins
% and the derived files childdoc.def and cdocsamp.tex with
% cdocsch1.tex, cdocsch2.tex, cdocsdrf.tex, cdocsfn1.tex, cdocsfn2.tex.
%
%<package>\ifdefined\childdocmain\endinput\fi
%<package>\ProvidesFile{childdoc.def}[2018/12/30 v2.0 child document driver]
%<samplemain>\ProvidesFile{cdocsamp.tex}[2018/12/30 v2.0 sample for childdoc]
%<*driver>
%\ProvidesFile{childdoc.drv}[2018/12/30 v2.0 childdoc reference manual file]
\PassOptionsToClass{10pt,a4paper}{article}
\documentclass{ltxdoc}

\usepackage[margin=35mm]{geometry}
\usepackage{hyperref}
\usepackage{hyperxmp}
\usepackage[usenames]{color}

\hypersetup{colorlinks=true}
\hypersetup{pdfstartview=FitH}
\hypersetup{pdfpagemode=UseNone}
\hypersetup{pdfsource={}}
\hypersetup{pdflang={en-UK}}
\hypersetup{pdfcopyright={Copyright 2017-2018 Niklas Beisert.
  This work may be distributed and/or modified under the
  conditions of the LaTeX Project Public License, either version 1.3
  of this license or (at your option) any later version.}}
\hypersetup{pdflicenseurl={http://www.latex-project.org/lppl.txt}}
\hypersetup{pdfcontactaddress={ETH Zurich, ITP, HIT K,
  Wolfgang-Pauli-Strasse 27}}
\hypersetup{pdfcontactpostcode={8093}}
\hypersetup{pdfcontactcity={Zurich}}
\hypersetup{pdfcontactcountry={Switzerland}}
\hypersetup{pdfcontactemail={nbeisert@itp.phys.ethz.ch}}
\hypersetup{pdfcontacturl={http://people.phys.ethz.ch/\xmptilde nbeisert/}}

\newcommand{\secref}[1]{\hyperref[#1]{section \ref*{#1}}}

\parskip1ex
\parindent0pt
\let\olditemize\itemize
\def\itemize{\olditemize\parskip0pt}

\begin{document}

\title{The \textsf{childdoc} Package}
\hypersetup{pdftitle={The childdoc Package}}
\author{Niklas Beisert\\[2ex]
  Institut f\"ur Theoretische Physik\\
  Eidgen\"ossische Technische Hochschule Z\"urich\\
  Wolfgang-Pauli-Strasse 27, 8093 Z\"urich, Switzerland\\[1ex]
  \href{mailto:nbeisert@itp.phys.ethz.ch}
  {\texttt{nbeisert@itp.phys.ethz.ch}}}
\hypersetup{pdfauthor={Niklas Beisert}}
\hypersetup{pdfsubject={Manual for the LaTeX2e Package childdoc}}
\date{30 December 2018, \textsf{v2.0}}
\maketitle

\begin{abstract}\noindent
\textsf{childdoc} is a \LaTeXe{} package
that enables the direct compilation
of document sections included by |\include|
to individual files.
\end{abstract}

\begingroup
\parskip0ex
\tableofcontents
\endgroup

%%%%%%%%%%%%%%%%%%%%%%%%%%%%%%%%%%%%%%%%%%%%%%%%%%%%%%%%%%%%%%%%%%%%%%%%%%%%%%%%
%%%%%%%%%%%%%%%%%%%%%%%%%%%%%%%%%%%%%%%%%%%%%%%%%%%%%%%%%%%%%%%%%%%%%%%%%%%%%%%%
\section{Introduction}

\LaTeX{} provides a mechanism to structure a large document (such as a book)
into a main file and several child files (containing the chapters)
using the |\include| command.
This mechanism is beneficial for documents
which span hundreds of pages in order to
make the source file(s) more manageable.
Moreover, compilation can be restricted to
selected child files by means of the |\includeonly| command.
The latter feature can be used to reduce the compilation time while editing
(this was significantly more useful in the earlier days of \LaTeX{})
or to generate a smaller document which is easier to navigate.
Another application of |\includeonly| is to generate
documents consisting of selected parts of the complete document.

However, there are a few drawbacks of the plain |\include| mechanism:
\begin{itemize}
\item
The child files cannot be compiled on their own,
they can only be compiled via the main file.
A naive editing environment
(such as a text editor with an option
to have the current file processed by \LaTeX)
may require one to switch to the main file before compiling;
attempting to compile the child file produces errors.
\item
The main file must be modified (each time)
to adjust the |\includeonly| command
to the present needs. This easily leaves the main file in a messy state.
\item
The generated document will always carry the filename
of the main document. This is inconvenient if
several child files are to be compiled and
to be kept for distribution.
\end{itemize}

The present package provides a simple interface
to make child files individually compilable by \LaTeX{}.
Compiling a child file then has the same effect as compiling
the main file with an |\includeonly| command
to select the appropriate child.
Moreover the generated document will carry the name of the child
rather than the main file.
This resolves all three above issues.

This feature is meant to make the editing of books,
thesis documents and lecture notes somewhat more convenient.
However, the package can also be used efficiently for
composing a series of documents (such as exercise sheets)
which are typically distributed individually.
It then assists the author in generating the individual documents
(potentially in different versions)
as well as a document containing the collected series.
Another application is in developing style files
or other kinds of included material
where compilation of the style file could redirect
to a sample or test file.

%%%%%%%%%%%%%%%%%%%%%%%%%%%%%%%%%%%%%%%%%%%%%%%%%%%%%%%%%%%%%%%%%%%%%%%%%%%%%%%%
%%%%%%%%%%%%%%%%%%%%%%%%%%%%%%%%%%%%%%%%%%%%%%%%%%%%%%%%%%%%%%%%%%%%%%%%%%%%%%%%
\section{Usage}

First of all, the package \textsf{childdoc} is \emph{not} a standard
\LaTeXe{} |.sty| style file! Therefore it needs to be invoked in
a non-standard way.

%%%%%%%%%%%%%%%%%%%%%%%%%%%%%%%%%%%%%%%%%%%%%%%%%%%%%%%%%%%%%%%%%%%%%%%%%%%%%%%%
\subsection{Included Files}
\label{sec:include}

%%%%%%%%%%%%%%%%%%%%%%%%%%%%%%%%%%%%%%%%
\DescribeMacro{\childdocmain}
To use the package, add the commands
\begin{center}
\begin{tabular}{l}
|\input{childdoc.def}|\\
|\childdocmain{}|\\
\end{tabular}
\end{center}
at the very top of the main \LaTeX{} file,
in particular \emph{before} the |\documentclass| statement!
The argument of |\childdocmain| should be left empty
(but it must be present).

%%%%%%%%%%%%%%%%%%%%%%%%%%%%%%%%%%%%%%%%
\DescribeMacro{\childdocof}
Furthermore, add the commands
\begin{center}
\begin{tabular}{l}
|\input{childdoc.def}|\\
|\childdocof{|\textit{main}|}|\\
\end{tabular}
\end{center}
at the top of every child file \textit{child}
which is included by |\include{|\textit{child}|}|
from within the main file
(or at least for those files to be compiled individually).
The argument \textit{main} must be the filename of the main file.

There are a couple of
considerations in setting up the main and child documents:

%%%%%%%%%%%%%%%%%%%%%%%%%%%%%%%%%%%%%%%%
\paragraph{Restrictions.}

Please note the following restrictions:
\begin{itemize}
\item
|\childdocmain| must be called with one argument \textit{main}
to ensure compatibility with earlier version of the package.
It must either be empty (|\childdocmain{}|)
or precisely match the filename of the main file in which it is specified.
See \secref{sec:detection} for further information.
\item
The filename \textit{main} must be specified without the |.tex| extension.
\item
The filename \textit{main} is case sensitive
(even in case-insensitive file systems)
due to internal string comparison.
\item
The argument \textit{main} should be fully expanded, it cannot be a macro.
\item
Subdirectories and special characters should be avoided in filenames.
\item
The command |\childdocmain{|\textit{main}|}| must be followed by a whitespace.
It should not be followed immediately by another command
or by a comment mark `|%|'.
This is because the \TeX{} parser reads the token immediately following
the argument of |\childdocmain| and puts it
at the beginning of every child section;
however, a white\-space is ignored.
\end{itemize}

%%%%%%%%%%%%%%%%%%%%%%%%%%%%%%%%%%%%%%%%
\paragraph{Content of Main File.}

It is advisable to place all content in the child files included by |\include|.
Any output contained in the main file will appear in all child documents
unless suppressed manually;
it cannot be suppressed automatically by the |\includeonly| directive
and thus should normally be avoided.
A method to include some content in the main file
by means of conditional processing is described in \secref{sec:conditional}.

%%%%%%%%%%%%%%%%%%%%%%%%%%%%%%%%%%%%%%%%
\paragraph{Page Numbering.}

When only a part of the document is compiled,
the appropriate numbering of pages
(as well as other status parameters)
is determined from the |.aux| files.
The latter contain information from previous passes.
However this information needs to propagate through
all intermediate child documents.
Therefore the page numbering in child documents may well
be inconsistent until the complete document is compiled at least once.

A useful (if unconventional) way to always ensure a consistent
page numbering is to restart the numbering in each child document
and denote the pages by `\textit{child}|.|\textit{page}'
where \textit{child} represents the chapter/section number of the child file.
This can be achieved by the command
|\numberwithin{page}{|\textit{child}|}|
of the \textsf{amsmath} package
where \textit{child} can be |chapter| or |section|
depending on the chosen structuring.
Alternatively, one can modify the macro |\thepage| appropriately
and reset the counter |page| at the start of each child file.

%%%%%%%%%%%%%%%%%%%%%%%%%%%%%%%%%%%%%%%%%%%%%%%%%%%%%%%%%%%%%%%%%%%%%%%%%%%%%%%%
\subsection{Conditional Processing}
\label{sec:conditional}

The package provides a mechanism to compile different versions
of a document. To customise the versions further some conditional processing
can come in handy to distinguish which version is being compiled.
The package provides two macros to describe the compilation context:

%%%%%%%%%%%%%%%%%%%%%%%%%%%%%%%%%%%%%%%%
\DescribeMacro{\ifchilddoc}
The conditional |\ifchilddoc| distinguishes between the compilation of
child documents and the main document:
%
\begin{center}
|\ifchilddoc |\textit{child-code}| |[|\||else |\textit{main-code}]| \||fi|
\end{center}

%%%%%%%%%%%%%%%%%%%%%%%%%%%%%%%%%%%%%%%%
\DescribeMacro{\childdocname}
\DescribeMacro{\childdocjob}
The macro |\childdocname| contains the filename (without extension)
of the main or child file being processed.
Note that |\childdocjob| will always contain the name of the main file.

%%%%%%%%%%%%%%%%%%%%%%%%%%%%%%%%%%%%%%%%
\paragraph{Title Page.}

Conditional processing can be used to include a title or banner page
in the main document when proper precautions are taken.
Importantly, the code in the main file should ensure that the page counter
(as well as other status parameters which are stored in the |.aux| files)
takes the same value after the conditional processing.
Otherwise the page numbers may take divergent values
depending on which part is compiled.

For example, a title page could be declared by:
%
\begin{center}
\begin{tabular}{l}
|\ifchilddoc\||else|\\
|\addtocounter{page}{-1}|\\
\textit{code for title page}\\
|\newpage|\\
|\||fi|
\end{tabular}
\end{center}
%
A banner page for the child documents can be generated by:
%
\begin{center}
\begin{tabular}{l}
|\ifchilddoc|\\
|\addtocounter{page}{-1}|\\
\textit{code for banner page}\\
|\newpage|\\
|\||fi|
\end{tabular}
\end{center}
%
Here one could write a message such as:
\begin{center}
|This is the part \childdocname{} of \childdocjob{}.|
\end{center}

%%%%%%%%%%%%%%%%%%%%%%%%%%%%%%%%%%%%%%%%%%%%%%%%%%%%%%%%%%%%%%%%%%%%%%%%%%%%%%%%
\subsection{Flags}
\label{sec:flags}

The package makes it easy to generate different versions
of the main or child documents.
To this end compilation flags can be defined
and assigned different default values.
They will be particularly useful in conjunction
with the forwarding mechanism described in \secref{sec:forward}.

For example, it may be useful to have a flag |\version|
which can be set to |draft| or |final|.
The document source will contain some conditional code
depending on the value of |\version|.
Suppose further, the flag should default to |final| for the main file
and to |draft| for child files
which is a natural assignment for editing the document.
This is achieved by placing the following code
in the preamble of the main document
(below the |\childdocmain| directive):
%
\begin{center}
\begin{tabular}{l}
|\ifchilddoc|\\
|\providecommand{\version}{draft}|\\
|\||else|\\
|\providecommand{\version}{final}|\\
|\||fi|
\end{tabular}
\end{center}
%
The definition by |\providecommand| makes sure
that previous definitions are not overwritten.
Further statements |\providecommand{\version}{...}|
can thus be added before the above code to override it.

For the main file, one might add a line
(between |\childdocmain| and the above block)
%
\begin{center}
|%\ifchilddoc\||else\providecommand{\version}{draft}\||fi|
\end{center}
%
which can be uncommented to produce a draft version.
Likewise one can add a line to the very top of a child file
(above the |\childdocof{|\textit{main}|}| directive)
%
\begin{center}
|%\providecommand{\version}{final}|
\end{center}
%
which can be uncommented to produce the final version of this child document.

%%%%%%%%%%%%%%%%%%%%%%%%%%%%%%%%%%%%%%%%%%%%%%%%%%%%%%%%%%%%%%%%%%%%%%%%%%%%%%%%
\subsection{Forwarding}
\label{sec:forward}

Different versions of the main or child documents
using compilation flags as described in \secref{sec:flags}
can be (permanently) stored in different files
for convenient compilation, viewing and distribution.
To this end, the package defines a command
to pass on compilation to a different file:

%%%%%%%%%%%%%%%%%%%%%%%%%%%%%%%%%%%%%%%%
\DescribeMacro{\childdocforward}
The command |\childdocforward| redirects processing to
another source file:
%
\begin{center}
\begin{tabular}{l}
|\input{childdoc.def}|\\
|\childdocforward[|\textit{main}|]{|\textit{dest}|}|\\
\end{tabular}
\end{center}
%
The argument \textit{dest} is the destination file
(without extension).
It should be the main file or one of the child files.
Note that further \textsf{childdoc} directives
such as |\childdocof| and |\childdocforward|
in the indicated file will be processed in this form.
The optional argument \textit{main}
passes on directly to the main file \textit{main}
while pretending to compile the child \textit{dest}.
This form behaves as if \textit{dest}
issues |\childdocof{|\textit{main}|}| right away,
and no further \textsf{childdoc} directives will be processed.

%%%%%%%%%%%%%%%%%%%%%%%%%%%%%%%%%%%%%%%%
\DescribeMacro{\...prefix}
In the alternative form |\childdocforwardprefix|,
%
\begin{center}
\begin{tabular}{l}
|\input{childdoc.def}|\\
|\childdocforwardprefix[|\textit{main}|]{|\textit{prefix}|}{|\textit{dest}|}|
\end{tabular}
\end{center}
%
the destination file is determined by a pattern
depending on the current file:
To make this work, the current file must be called
`{\textit{prefix}\hspace{0.2em}\textit{suffix}}'
with \textit{prefix} matching precisely the argument.
Processing is then passed on to the file
`{\textit{dest}\hspace{0.2em}\textit{suffix}}'.
Surely, the same effect is achieved by
directly specifying the
argument `{\textit{dest}\hspace{0.2em}\textit{suffix}}'
in the first form.
However, that requires to set up a different file
for each child. With the alternative form of the command
all these files can have exactly the same content
which simplifies setting them up and maintaining them.

For example, the following file |draft.tex|
with a compilation flag |\version| as described in \secref{sec:flags}
compiles the main document as a draft:
%
\begin{center}
\begin{tabular}{l}
|\def\version{draft}|\\
|\input{childdoc.def}|\\
|\childdocforward{|\textit{main}|}|
\end{tabular}
\end{center}
%
Likewise, the following files |final|\textit{nn}|.tex|
compile the final version of the child document
|child|\textit{nn}|.tex|:
%
\begin{center}
\begin{tabular}{l}
|\def\version{final}|\\
|\input{childdoc.def}|\\
|\childdocforwardprefix{final}{child}|
\end{tabular}
\end{center}
%

Note that when several versions of a main file and/or of each child file
are to be generated, it may be convenient to set up a |Makefile| or
shell script to automatise the process.

%%%%%%%%%%%%%%%%%%%%%%%%%%%%%%%%%%%%%%%%%%%%%%%%%%%%%%%%%%%%%%%%%%%%%%%%%%%%%%%%
\subsection{Command Line Processing}
\label{sec:commandline}

The effect of redirection files can also be achieved by invoking
the \LaTeX{} compiler with a more elaborate command line.
Most conveniently this should be done as part
of a shell script or a |Makefile|.

When using \textsf{childdoc} in the main file, the following
command lines effectively perform a redirection
(note that depending on the shell being used,
backslashes may have to be doubled: `|\|' $\to$ `|\\|'):
%
\begin{center}
|... -jobname "|\textit{target}|" |\\|"|[\textit{flags}]%
|\input{childdoc.def}\childdocforward[|\textit{main}|]{|\textit{dest}|}"|
\end{center}
%
Here \textit{target} is the name of the output file,
\textit{main} is the name of the main file
and \textit{dest} is the name of the main or child file to be processed
(all filenames without extensions).
The optional argument \textit{main} can be omitted
if \textit{main} matches \textit{dest}.
Optionally, compilation \textit{flags} can be defined via |\def| commands.
This command line makes the \TeX{} engine believe
it is compiling the file \textit{target}
whose content is specified as the latter parameter.
The provided code then forwards the processing to
\textit{main} or \textit{dest} as described in \secref{sec:forward}.

%%%%%%%%%%%%%%%%%%%%%%%%%%%%%%%%%%%%%%%%%%%%%%%%%%%%%%%%%%%%%%%%%%%%%%%%%%%%%%%%
\subsection{Include by Input}
\label{sec:input}

Including child documents by |\include| has some restrictions by design.
Most notably, the content of a child document always occupies
its own set of pages; pages cannot be shared between child documents.
Usually, this behaviour makes perfect sense
because each child document contain an essential part of the document.
However, in some situations it may be desirable to compose
a document from a collection of parts
without having mandatory page breaks between then.
For this case, the package
provides a mechanism to include parts
by |\input| which can also be processed individually.
However, by construction this mechanism
requires manual handling of the content to be output.

%%%%%%%%%%%%%%%%%%%%%%%%%%%%%%%%%%%%%%%%
\DescribeMacro{\ifchilddocmanual}
The main file should be prepared as usual, see \secref{sec:include}.
However, the document body must make a distinction
between processing of an individual part and of the main document, e.g.:
%
\begin{center}
\begin{tabular}{l}
|\ifchilddocmanual|\\
|\input{\childdocname}|\\
|\||else|\\
\textit{document body with }|\input{|\textit{part}|}|\\
|\||fi|
\end{tabular}
\end{center}
%
The conditional |\ifchilddocmanual| is true whenever
a part to be included by |\input| is being compiled,
and the name of the part is stored in |\childdocname|.

%%%%%%%%%%%%%%%%%%%%%%%%%%%%%%%%%%%%%%%%
\DescribeMacro{\childdocby}
Each part to be included by |\input| should start with:
%
\begin{center}
\begin{tabular}{l}
|\input{childdoc.def}|\\
|\childdocby{|\textit{main}|}|\\
\end{tabular}
\end{center}
%
The directive |\childdocby| is similar to |\childdocof|
described in \secref{sec:include},
but the subsequent selection of content must be done manually.
To that end, both |\ifchilddoc| and |\ifchilddocmanual|
will be true upon processing of a part,
and the name of the part is stored in |\childdocname|.
Note that |\jobname| will be set to the filename of the current part
so that each part receives an individual |.aux| file
that does not interfere with the |.aux| file(s) of the main document.
This behaviour can be altered by the alternative form
|\childdocby[*]{|\textit{main}|}| (with a non-empty optional argument)
which uses the |.aux| file of the main document
by setting |\jobname| to \textit{main}.

%%%%%%%%%%%%%%%%%%%%%%%%%%%%%%%%%%%%%%%%%%%%%%%%%%%%%%%%%%%%%%%%%%%%%%%%%%%%%%%%
\subsection{Driver Development}
\label{sec:driver}

The \textsf{childdoc} mechanism can also be use for the development
of definition files such as \LaTeX{} styles or classes.
This case differs from the above setup with multiple parts
included by |\include| in that no |\includeonly| should be invoked.
This can be achieved by starting the include file
(before |\ProvidesPackage|) with:
%
\begin{center}
\begin{tabular}{l}
|\input{childdoc.def}|\\
|\childdocforward{|\textit{main}|}|\\
\end{tabular}
\end{center}
%
or alternatively with:
%
\begin{center}
\begin{tabular}{l}
|\input{childdoc.def}|\\
|\childdocby{|\textit{main}|}|\\
\end{tabular}
\end{center}
%
Both forms have slightly different effects as described above.
The main file is prepared as usual, see \secref{sec:include}.

%%%%%%%%%%%%%%%%%%%%%%%%%%%%%%%%%%%%%%%%%%%%%%%%%%%%%%%%%%%%%%%%%%%%%%%%%%%%%%%%
\subsection{Legacy Detection}
\label{sec:detection}

The directive |\childdocmain| in the main file can detect
whether the complete document or merely a child is to be compiled
even without using the directive |\childdocof|.
This method is deprecated because it is less robust
and there is no compelling reason to use it;
it is merely provided for backward compatibility
and it may be removed in future versions.

If the detection mechanism is to be used,
it is mandatory to correctly specify
the filename of the main file as the argument of |\childdocmain|:
%
\begin{center}
\begin{tabular}{l}
|\input{childdoc.def}|\\
|\childdocmain{|\textit{main}|}|\\
\end{tabular}
\end{center}
%
If |\jobname| does not match the argument \textit{main} of |\childdocmain|,
it is assumed that |\jobname| points to the child file to be compiled.
When using |\childdocmain| with the main file specified as argument,
it suffices to start a child file
with just |\input{|\textit{main}|}|
without loading of the package and using |\childdocof|.
If instead all processing is done
with the appropriate \textsf{childdoc} directives,
the argument of \textit{main} of |\childdocmain| can be empty.

An alternative version of the command line processing described
in \secref{sec:commandline} using the detection mechanism reads:
%
\begin{center}
|... -jobname "|\textit{target}|" "|[\textit{flags}]%
[|\def\jobname{|\textit{dest}|}|]|\input{|\textit{main}|}"|
\end{center}

%%%%%%%%%%%%%%%%%%%%%%%%%%%%%%%%%%%%%%%%%%%%%%%%%%%%%%%%%%%%%%%%%%%%%%%%%%%%%%%%
\subsection{Manual Code}
\label{sec:manual}

In case one cannot be certain whether the definitions file |childdoc.def|
is installed on the target \TeX{} distribution
and one prefers not to ship it,
it is conceivable to paste a few relevant commands into the sources.

To that end, drop all statements |\input{childdoc.def}|
and perform the replacements as outlined below.
Instead of |\childdocmain{|\textit{main}|}| add the following code
to the top of the main file:
%
\begin{center}
\begin{tabular}{l}
|\||ifdefined\childdocname\endinput\||fi\newif\ifchilddoc|\\
|\edef\childdocname{\scantokens\expandafter{\jobname\noexpand}}|\\
|\def\childdocmain{|\textit{main}|}\||ifx\childdocmain\childdocname\||else|\\
|\childdoctrue\includeonly{\childdocname}\let\jobname\childdocmain\||fi|\\
\end{tabular}
\end{center}
%
Instead of |\childdocof{|\textit{main}|}| just include the main file
at the top of each child file:
%
\begin{center}
|\input{|\textit{main}|}|
\end{center}
%
A simple redirection |\childdocforward{|\textit{dest}|}| is achieved by:
%
\begin{center}
|\def\jobname{|\textit{dest}|}\input{\jobname}|
\end{center}
%
The redirection with prefix
|\childdocforwardprefix[|\textit{prefix}|]{|\textit{dest}|}|
is accomplished by:
%
\begin{center}
\begin{tabular}{l}
|{\edef\jobname{\scantokens\expandafter{\jobname\noexpand}}|\\
|\def\redirectjob |\textit{prefix}|#1~~~{\gdef\jobname{|\textit{dest}|#1}}|\\
|\expandafter\redirectjob\jobname~~~}\input{\jobname}|
\end{tabular}
\end{center}

In an alternative approach,
child documents can be compiled by a specific command line
without additional code or specific definitions:
%
\begin{center}
|... -jobname "|\textit{target}|" "|[\textit{flags}]%
|\includeonly{|\textit{dest}|}\input{|\textit{main}|}"|
\end{center}
%

%%%%%%%%%%%%%%%%%%%%%%%%%%%%%%%%%%%%%%%%%%%%%%%%%%%%%%%%%%%%%%%%%%%%%%%%%%%%%%%%
%%%%%%%%%%%%%%%%%%%%%%%%%%%%%%%%%%%%%%%%%%%%%%%%%%%%%%%%%%%%%%%%%%%%%%%%%%%%%%%%
\section{Information}

%%%%%%%%%%%%%%%%%%%%%%%%%%%%%%%%%%%%%%%%%%%%%%%%%%%%%%%%%%%%%%%%%%%%%%%%%%%%%%%%
\subsection{Copyright}

Copyright \copyright{} 2017--2018 Niklas Beisert

This work may be distributed and/or modified under the
conditions of the \LaTeX{} Project Public License, either version 1.3
of this license or (at your option) any later version.
The latest version of this license is in
  \url{http://www.latex-project.org/lppl.txt}
and version 1.3 or later is part of all distributions of \LaTeX{}
version 2005/12/01 or later.

This work has the LPPL maintenance status `maintained'.

The Current Maintainer of this work is Niklas Beisert.

This work consists of the files |README.txt|, |childdoc.ins| and |childdoc.dtx|
as well as the derived files |childdoc.def|, |cdocsamp.tex|
with |cdocsch1.tex|, |cdocsch2.tex|, |cdocspt3.tex|, |cdocspt4.tex|,
|cdocsdrf.tex|, |cdocsfn1.tex|, |cdocsfn2.tex|
as well as |childdoc.pdf|.

%%%%%%%%%%%%%%%%%%%%%%%%%%%%%%%%%%%%%%%%%%%%%%%%%%%%%%%%%%%%%%%%%%%%%%%%%%%%%%%%
\subsection{Files and Installation}

The package consists of the files:
%
\begin{center}
\begin{tabular}{ll}
    |README.txt|   & readme file \\
    |childdoc.ins| & installation file \\
    |childdoc.dtx| & source file \\
    |childdoc.def| & definition file \\
    |cdocsamp.tex| & sample main file \\
    |cdocsch1.tex| & sample include file \\
    |cdocsch2.tex| & sample include file \\
    |cdocspt3.tex| & sample part file \\
    |cdocspt4.tex| & sample part file \\
    |cdocsdrf.tex| & sample redirection file \\
    |cdocsfn1.tex| & sample redirection file \\
    |cdocsfn2.tex| & sample redirection file \\
    |childdoc.pdf| & manual
\end{tabular}
\end{center}
%
The distribution consists of the files
|README.txt|, |childdoc.ins| and |childdoc.dtx|.
%
\begin{itemize}
\item
Run (pdf)\LaTeX{} on |childdoc.dtx|
to compile the manual |childdoc.pdf| (this file).
\item
Run \LaTeX{} on |childdoc.ins| to create the definitions file |childdoc.def|
and the sample |cdocsamp.tex| with include files
|cdocsch1.tex|, |cdocsch2.tex|, |cdocspt3.tex|, |cdocspt4.tex|,
|cdocsdrf.tex|, |cdocsfn1.tex|, |cdocsfn2.tex|.
Then copy the file |childdoc.def| to an appropriate directory of your \LaTeX{}
distribution, e.g.\ \textit{texmf-root}|/tex/latex/childdoc|.
\end{itemize}

%%%%%%%%%%%%%%%%%%%%%%%%%%%%%%%%%%%%%%%%%%%%%%%%%%%%%%%%%%%%%%%%%%%%%%%%%%%%%%%%
\subsection{Related CTAN Packages}

There are several other packages which offer a similar functionality:
%
\begin{itemize}
\item
The packages
\href{http://ctan.org/pkg/docmute}{\textsf{docmute}},
\href{http://ctan.org/pkg/includex}{\textsf{includex}} and
\href{http://ctan.org/pkg/standalone}{\textsf{standalone}}
provide commands to include only the document body of
a child file thus allowing both files to be compiled individually.
\item
The packages \href{http://ctan.org/pkg/subdocs}{\textsf{subdocs}}
and \href{http://ctan.org/pkg/subfiles}{\textsf{subfiles}}
provide structures in which the main and child documents can be
encapsulated and allowing them to be compiled individually.
The inclusion mechanism is different from the conventional |\include|.
\item
The package \href{http://ctan.org/pkg/combine}{\textsf{combine}}
is an elaborate solution to combine several documents into one.
\end{itemize}
%
See also the CTAN topic \href{http://ctan.org/topic/subdocs}{\textsf{subdocs}}
for further related packages.
The present package differs from the above solutions in that
a document structure constructed with the conventional |\include| mechanism
just needs two extra commands at the top of every file
such that all constituent files can be compiled individually.

%%%%%%%%%%%%%%%%%%%%%%%%%%%%%%%%%%%%%%%%%%%%%%%%%%%%%%%%%%%%%%%%%%%%%%%%%%%%%%%%
%\subsection{Feature Suggestions}
%
%The following is a list of features which may be useful for future
%versions of this package:
%%
%\begin{itemize}
%\item
%\ldots
%\end{itemize}

%%%%%%%%%%%%%%%%%%%%%%%%%%%%%%%%%%%%%%%%%%%%%%%%%%%%%%%%%%%%%%%%%%%%%%%%%%%%%%%%
\subsection{Revision History}

%%%%%%%%%%%%%%%%%%%%%%%%%%%%%%%%%%%%%%%%
\paragraph{v2.0:} 2018/12/30

\begin{itemize}
\item
immediate forward processing
\item
added |\childdocby| mechanism
\item
manual restructured
\end{itemize}

%%%%%%%%%%%%%%%%%%%%%%%%%%%%%%%%%%%%%%%%
\paragraph{v1.6:} 2018/01/17

\begin{itemize}
\item
application for development of include files
\item
corrections to manual
\end{itemize}

%%%%%%%%%%%%%%%%%%%%%%%%%%%%%%%%%%%%%%%%
\paragraph{v1.5:} 2017/05/21

\begin{itemize}
\item
more complete structuring introduced
\item
|\childdocof| introduced
\item
|\childdoc| renamed to |\childdocmain|
\item
|\childredirect| renamed to |\childdocforward| and |\childdocforwardprefix|
and functionality expanded
\end{itemize}

%%%%%%%%%%%%%%%%%%%%%%%%%%%%%%%%%%%%%%%%
\paragraph{v1.0:} 2017/04/27

\begin{itemize}
\item
manual and install package
\item
first version published on CTAN
\end{itemize}

%%%%%%%%%%%%%%%%%%%%%%%%%%%%%%%%%%%%%%%%
\paragraph{v0.6:} 2017/04/26

\begin{itemize}
\item
redirection mechanism added
\end{itemize}

%%%%%%%%%%%%%%%%%%%%%%%%%%%%%%%%%%%%%%%%
\paragraph{v0.5:} 2017/04/26

\begin{itemize}
\item
functionality in definition file
\end{itemize}


%%%%%%%%%%%%%%%%%%%%%%%%%%%%%%%%%%%%%%%%%%%%%%%%%%%%%%%%%%%%%%%%%%%%%%%%%%%%%%%%
%%%%%%%%%%%%%%%%%%%%%%%%%%%%%%%%%%%%%%%%%%%%%%%%%%%%%%%%%%%%%%%%%%%%%%%%%%%%%%%%
%%%%%%%%%%%%%%%%%%%%%%%%%%%%%%%%%%%%%%%%%%%%%%%%%%%%%%%%%%%%%%%%%%%%%%%%%%%%%%%%
\appendix

\settowidth\MacroIndent{\rmfamily\scriptsize 000\ }

 \DocInput{childdoc.dtx}

\end{document}
%</driver>
% \fi
%
% %%%%%%%%%%%%%%%%%%%%%%%%%%%%%%%%%%%%%%%%%%%%%%%%%%%%%%%%%%%%%%%%%%%%%%%%%%%%%%
% %%%%%%%%%%%%%%%%%%%%%%%%%%%%%%%%%%%%%%%%%%%%%%%%%%%%%%%%%%%%%%%%%%%%%%%%%%%%%%
% \section{Sample}
%\iffalse
%<*samplemain>
%\fi
%
% The following presents a sample document
% with two chapters, two parts, a title page,
% a compile flag as well as three forwarding files to set the flag.
% It consists of eight |.tex| files:
% \begin{center}
% \begin{tabular}{ll}
% |cdocsamp.tex|&main file\\
% |cdocsch1.tex|&include file for chapter 1\\
% |cdocsch2.tex|&include file for chapter 2\\
% |cdocspt3.tex|&include file for part 3\\
% |cdocspt4.tex|&include file for part 4\\
% |cdocsdrf.tex|&forwarding file for main file in draft mode\\
% |cdocsfi1.tex|&forwarding file for final version of chapter 1\\
% |cdocsfi2.tex|&forwarding file for final version of chapter 2\\
% \end{tabular}
% \end{center}
% Each of the eight files can be compiled directly by the \LaTeX{} compiler.
%
% %%%%%%%%%%%%%%%%%%%%%%%%%%%%%%%%%%%%%%
% \paragraph{Main File.}
%
% The main file is called |cdocsamp.tex|.
%
% Load the \textsf{childdoc} definitions and
% declare the filename for the main document:
%    \begin{macrocode}
\input{childdoc.def}
\childdocmain{}
%    \end{macrocode}

% Optional override for |\version| flag:
%    \begin{macrocode}
%%\ifchilddoc\else\providecommand{\version}{draft}\fi
%    \end{macrocode}

% Define the default values for the |\version| flag
% (|final| for the main file and |draft| for childs):
%    \begin{macrocode}
\ifchilddoc
\providecommand{\version}{draft}
\else
\providecommand{\version}{final}
\fi
%    \end{macrocode}

% Load the standard document class:
%    \begin{macrocode}
\documentclass[12pt]{article}
%    \end{macrocode}

% Start the document body:
%    \begin{macrocode}
\begin{document}
%    \end{macrocode}

% Declare a title page.
% Print title, part of document being processed and version flag:
%    \begin{macrocode}
\addtocounter{page}{-1}
\begin{center}
{\LARGE\bfseries{}childdoc example\par}
\vspace{1cm}
\ifchilddoc
\ifchilddocmanual part\else chapter\fi:
`\childdocname' of `\childdocjob'\par
\else
main document: `\childdocjob'\par
\fi
version: \version\par
\end{center}
\newpage
%    \end{macrocode}

% Manually include selected file,
% otherwise process as usual:
%    \begin{macrocode}
\ifchilddocmanual
\section*{part `\childdocname'}
\input{\childdocname}
\else
%    \end{macrocode}

% Include the two chapters:
%    \begin{macrocode}
\include{cdocsch1}
\include{cdocsch2}
%    \end{macrocode}

% Include the two parts unless only chapters should be displayed:
%    \begin{macrocode}
\ifchilddoc\else
\section{part three}
\input{cdocspt3}
\section{part four}
\input{cdocspt4}
\fi
%    \end{macrocode}

% Process as usual until here:
%    \begin{macrocode}
\fi
%    \end{macrocode}

% End of document body:
%    \begin{macrocode}
\end{document}
%    \end{macrocode}
%\iffalse
%</samplemain>
%\fi
%
% %%%%%%%%%%%%%%%%%%%%%%%%%%%%%%%%%%%%%%
% \paragraph{Chapter Include Files.}
%
% The include files are called |cdocsch1.tex| and |cdocsch2.tex|.
%
%\iffalse
%<*samplechap1|samplechap2>
%\fi

% Optional override for |\version| flag:
%    \begin{macrocode}
%%\providecommand{\version}{final}
%    \end{macrocode}

% Include the main document:
%    \begin{macrocode}
\input{childdoc.def}
\childdocof{cdocsamp}
%    \end{macrocode}

%\iffalse
%</samplechap1|samplechap2>
%\fi
%
%\iffalse
%<*samplechap1>
%\fi
% Some text for chapter 1:
%    \begin{macrocode}
\section{one}
some text in chapter one
%    \end{macrocode}

%\iffalse
%</samplechap1>
%\fi
% Some text for chapter 2:
%\iffalse
%<*samplechap2>
%\fi
%    \begin{macrocode}
\section{two}
more text in chapter two
%    \end{macrocode}

%\iffalse
%</samplechap2>
%\fi
%
% %%%%%%%%%%%%%%%%%%%%%%%%%%%%%%%%%%%%%%
% \paragraph{Part Include Files.}
%
% The include files are called |cdocspt3.tex| and |cdocspt4.tex|.
%
%\iffalse
%<*samplepart3|samplepart4>
%\fi

% Optional override for |\version| flag:
%    \begin{macrocode}
%%\providecommand{\version}{final}
%    \end{macrocode}

% Include the main document:
%    \begin{macrocode}
\input{childdoc.def}
\childdocby{cdocsamp}
%    \end{macrocode}

%\iffalse
%</samplepart3|samplepart4>
%\fi
%
%\iffalse
%<*samplepart3>
%\fi
% Some text for part 3:
%    \begin{macrocode}
some text in part three
%    \end{macrocode}

%\iffalse
%</samplepart3>
%\fi
% Some text for part 4:
%\iffalse
%<*samplepart4>
%\fi
%    \begin{macrocode}
more text in part four
%    \end{macrocode}

%\iffalse
%</samplepart4>
%\fi
%
% %%%%%%%%%%%%%%%%%%%%%%%%%%%%%%%%%%%%%%
% \paragraph{Forwarding for a Complete Draft.}
%
% The following forwarding file |cdocsdrf.tex|
% compiles the main document in draft mode:
%\iffalse
%<*sampledraft>
%\fi
%    \begin{macrocode}
\def\version{draft}
\input{childdoc.def}
\childdocforward{cdocsamp}
%    \end{macrocode}

%\iffalse
%</sampledraft>
%\fi
%
% %%%%%%%%%%%%%%%%%%%%%%%%%%%%%%%%%%%%%%
% \paragraph{Forwarding for Final Version of the Chapters.}
%
% The following forwarding files |cdocsfn1.tex| and |cdocsfn2.tex|
% (with identical content)
% compile the final versions of the child documents
% |cdocsch1.tex| and |cdocsch2.tex|, respectively:
%\iffalse
%<*samplefinal>
%\fi
%    \begin{macrocode}
\def\version{final}
\input{childdoc.def}
\childdocforwardprefix[cdocsamp]{cdocsfn}{cdocsch}
%    \end{macrocode}

%\iffalse
%</samplefinal>
%\fi
%
% %%%%%%%%%%%%%%%%%%%%%%%%%%%%%%%%%%%%%%
% \paragraph{Command Line Processing.}
%
% The following three command lines generate the output files
% |cdocscld|, |cdocscl1| and |cdocscl2|
% which should be identical to
% |cdocsdrf|, |cdocsch1| and |cdocsfn2|, respectively:
% \begin{center}
% \begin{tabular}{l}
% |latex -jobname cdocscld \|\\
% |  "\def\version{draft}\input{childdoc.def}\childdocforward{cdocsamp}"|\\
% |latex -jobname cdocscl1 \|\\
% |  "\input{childdoc.def}\childdocforward[cdocsamp]{cdocsch1}"|\\
% |latex -jobname cdocscl2 \|\\
% |  "\def\version{final}\input{childdoc.def}\childdocforward{cdocsch2}"|
% \end{tabular}
% \end{center}
% Note that the trailing backslash on each first line
% merely continues the input to the second line
% (for convenient cut ant paste).
% Furthermore, the command |latex| can be replaced by any
% of its alternative versions such as |pdflatex|.
%
% %%%%%%%%%%%%%%%%%%%%%%%%%%%%%%%%%%%%%%%%%%%%%%%%%%%%%%%%%%%%%%%%%%%%%%%%%%%%%%
% %%%%%%%%%%%%%%%%%%%%%%%%%%%%%%%%%%%%%%%%%%%%%%%%%%%%%%%%%%%%%%%%%%%%%%%%%%%%%%
% \section{Implementation}
%\iffalse
%<*package>
%\fi
%
% This section describes the definitions file |childdoc.def|.

% The definitions cannot be loaded using |\usepackage| or |\RequirePackage|
% which has a mechanism to prevent loading a style file more than once.
% When loading the definitions by means of |\input|
% multiple instances have to be prevented manually:
%\iffalse
%This code needs to be before the `\ProvidesFile' directive
%which is defined at the beginning of this file.
%Therefore it is also placed there and commented out here.
%</package>
%<*discard>
%\fi
%    \begin{macrocode}
\ifdefined\childdocmain\endinput\fi
%    \end{macrocode}
%\iffalse
%</discard>
%<*package>
%\fi
%
% \macro{\ifchilddoc}
% \macro{\ifchilddocmanual}
% The conditional |\ifchilddoc| tells whether a
% child (true) or main (false) document is being compiled.
% The conditional |\ifchilddocmanual| tells whether
% the |\includeonly| mechanism is used (false) or
% the selection of child files must be performed manually (true).
% The definitions initialise to false:
%    \begin{macrocode}
\newif\ifchilddoc
\newif\ifchilddocmanual
%    \end{macrocode}

% \macro{\childdocname}
% \macro{\childdocjob}
% The macro |\childdocname| stores the name of the main document
% to be compiled. The macro |\childdocjob| stores the name of
% the document on which the \LaTeX{} compiler was originally invoked.
% The content of |\jobname| cannot be compared
% to filenames specified in the source due to different catcodes.
% The following code rescans |\jobname|, stores the result
% in |\childdocname| and saves a copy in |\childdocjob|:
%    \begin{macrocode}
\edef\childdocname{\scantokens\expandafter{\jobname\noexpand}}
\let\childdocjob\childdocname
%    \end{macrocode}

% \macro{\childdocdisable}
% The macro |\childdocdisable| prevents the main file
% from being processed more than once.
% At this stage, the main document command |\childdocmain|
% is assumed to be called once again where it should do nothing.
% Any subsequent call to it should prevent
% a secondary processing of the main document
% It overwrites the forwarding commands
% |\childdocof| and |\childdocforward|
% with empty macros to prevent further inclusions of the main document:
%    \begin{macrocode}
\newcommand{\childdocdisable}
{
  \renewcommand{\childdocmain}[1]{\renewcommand{\childdocmain}[1]{\endinput}}
  \renewcommand{\childdocof}[1]{}
  \renewcommand{\childdocby}[2][]{}
  \renewcommand{\childdocforward}[2][]{}
  \renewcommand{\childdocdisable}{}
}
%    \end{macrocode}

% \macro{\childdocmain}
% The macro |\childdocmain| is to be called at the top of the main file
% with nothing or the main filename (without extension) as argument.
% First, it breaks loops.
% If the argument is not empty and does not match |\childdocname|
% (which is set by the first inclusion of |childdoc.def|),
% |\ifchilddoc| is set to true, |\includeonly| is applied to the child file
% and |\jobname| is set to the main file
% (for proper handling of |.aux| files):
%    \begin{macrocode}
\newcommand{\childdocmain}[1]
{
  \childdocdisable\childdocmain{}
  \if?#1?\else
    \begingroup
      \def\childdoctmp{#1}
      \ifx\childdoctmp\childdocname
        \def\childdoctmp{}
      \else
        \def\childdoctmp
        {
          \childdoctrue
          \includeonly{\childdocname}
          \def\childdocjob{#1}
          \def\jobname{#1}
        }
      \fi
      \expandafter
    \endgroup
    \childdoctmp
  \fi
}
%    \end{macrocode}

% \macro{\childdocof}
% The command |\childdocof| redirects
% compilation to the main file |#1|.
%    \begin{macrocode}
\newcommand{\childdocof}[1]
{
  \childdocdisable
  \childdoctrue
  \includeonly{\childdocname}
  \def\jobname{#1}
  \def\childdocjob{#1}
  \input{#1}
}
%    \end{macrocode}

% \macro{\childdocby}
% The command |\childdocby| ....
%    \begin{macrocode}
\newcommand{\childdocby}[2][]
{
  \childdocdisable
  \childdoctrue
  \childdocmanualtrue
  \if?#1?\else
    \def\jobname{#2}
  \fi
  \def\childdocjob{#2}
  \input{#2}
  \endinput
}
%    \end{macrocode}

% \macro{\childdocforward}
% The command |\childdocforward| redirects
% compilation to the main file or
% (if the optional argument is given) a child file.
% Parameters are set as if the main file
% or a child file starting with |\childdocof| was compiled.
% Then compilation is handed over to the main file:
%    \begin{macrocode}
\newcommand{\childdocforward}[2][]
{
  \begingroup
    \if?#1?
      \def\childdoctmp
      {
        \def\childdocname{#2}
        \def\childdocjob{#2}
        \def\jobname{#2}
        \input{#2}
        \endinput
      }
    \else
      \def\childdoctmp
      {
        \childdocdisable
        \def\childdocname{#2}
        \childdoctrue
        \includeonly{#2}
        \def\childdocjob{#1}
        \def\jobname{#1}
        \input{#1}
        \endinput
      }
    \fi
    \expandafter
  \endgroup
  \childdoctmp
}
%    \end{macrocode}

% \macro{\childdocforwardprefix}
% The command |\childdocforwardprefix| redirects
% compilation to the main or a child file by means of a pattern.
% The prefix |#1| in the current filename is replaced by |#2|
% and the suffix of the current filename is kept
% (it is assumed that the filename does not contain the substring `|~~~|'
% which is used as a delimiter).
% Compilation is handed over to the new file by |\childdocforward|:
%    \begin{macrocode}
\newcommand{\childdocforwardprefix}[3][]
{
  \begingroup
    \def\childdocextract #2##1~~~{\def\childdoctmp{\childdocforward[#1]{#3##1}}}
    \expandafter\childdocextract\childdocname~~~
    \expandafter
  \endgroup
  \childdoctmp
}
%    \end{macrocode}

% \macro{\childdoc}
% The deprecated macro |\childdoc| is a legacy version of |\childdocmain|:
%    \begin{macrocode}
\newcommand{\childdoc}{\childdocmain}
%    \end{macrocode}

% \macro{\childdocredirect}
% The deprecated macro |\childdocredirect| is a legacy version
% of |\childdocforward| and |\childdocforwardprefix|:
%    \begin{macrocode}
\newcommand{\childdocredirect}[2][]
{
  \begingroup
    \if?#1?
      \def\childdoctmp{\childdocforward{#2}}
    \else
      \def\childdoctmp{\childdocforwardprefix{#1}{#2}}
    \fi
    \expandafter
  \endgroup
  \childdoctmp
}
%    \end{macrocode}

%\iffalse
%</package>
%\fi
%
\endinput
|\\
|\childdocforward{|\textit{main}|}|
\end{tabular}
\end{center}
%
Likewise, the following files |final|\textit{nn}|.tex|
compile the final version of the child document
|child|\textit{nn}|.tex|:
%
\begin{center}
\begin{tabular}{l}
|\def\version{final}|\\
|% \iffalse
%
% childdoc.dtx Copyright (C) 2017-2018 Niklas Beisert
%
% This work may be distributed and/or modified under the
% conditions of the LaTeX Project Public License, either version 1.3
% of this license or (at your option) any later version.
% The latest version of this license is in
%   http://www.latex-project.org/lppl.txt
% and version 1.3 or later is part of all distributions of LaTeX
% version 2005/12/01 or later.
%
% This work has the LPPL maintenance status `maintained'.
%
% The Current Maintainer of this work is Niklas Beisert.
%
% This work consists of the files childdoc.dtx and childdoc.ins
% and the derived files childdoc.def and cdocsamp.tex with
% cdocsch1.tex, cdocsch2.tex, cdocsdrf.tex, cdocsfn1.tex, cdocsfn2.tex.
%
%<package>\ifdefined\childdocmain\endinput\fi
%<package>\ProvidesFile{childdoc.def}[2018/12/30 v2.0 child document driver]
%<samplemain>\ProvidesFile{cdocsamp.tex}[2018/12/30 v2.0 sample for childdoc]
%<*driver>
%\ProvidesFile{childdoc.drv}[2018/12/30 v2.0 childdoc reference manual file]
\PassOptionsToClass{10pt,a4paper}{article}
\documentclass{ltxdoc}

\usepackage[margin=35mm]{geometry}
\usepackage{hyperref}
\usepackage{hyperxmp}
\usepackage[usenames]{color}

\hypersetup{colorlinks=true}
\hypersetup{pdfstartview=FitH}
\hypersetup{pdfpagemode=UseNone}
\hypersetup{pdfsource={}}
\hypersetup{pdflang={en-UK}}
\hypersetup{pdfcopyright={Copyright 2017-2018 Niklas Beisert.
  This work may be distributed and/or modified under the
  conditions of the LaTeX Project Public License, either version 1.3
  of this license or (at your option) any later version.}}
\hypersetup{pdflicenseurl={http://www.latex-project.org/lppl.txt}}
\hypersetup{pdfcontactaddress={ETH Zurich, ITP, HIT K,
  Wolfgang-Pauli-Strasse 27}}
\hypersetup{pdfcontactpostcode={8093}}
\hypersetup{pdfcontactcity={Zurich}}
\hypersetup{pdfcontactcountry={Switzerland}}
\hypersetup{pdfcontactemail={nbeisert@itp.phys.ethz.ch}}
\hypersetup{pdfcontacturl={http://people.phys.ethz.ch/\xmptilde nbeisert/}}

\newcommand{\secref}[1]{\hyperref[#1]{section \ref*{#1}}}

\parskip1ex
\parindent0pt
\let\olditemize\itemize
\def\itemize{\olditemize\parskip0pt}

\begin{document}

\title{The \textsf{childdoc} Package}
\hypersetup{pdftitle={The childdoc Package}}
\author{Niklas Beisert\\[2ex]
  Institut f\"ur Theoretische Physik\\
  Eidgen\"ossische Technische Hochschule Z\"urich\\
  Wolfgang-Pauli-Strasse 27, 8093 Z\"urich, Switzerland\\[1ex]
  \href{mailto:nbeisert@itp.phys.ethz.ch}
  {\texttt{nbeisert@itp.phys.ethz.ch}}}
\hypersetup{pdfauthor={Niklas Beisert}}
\hypersetup{pdfsubject={Manual for the LaTeX2e Package childdoc}}
\date{30 December 2018, \textsf{v2.0}}
\maketitle

\begin{abstract}\noindent
\textsf{childdoc} is a \LaTeXe{} package
that enables the direct compilation
of document sections included by |\include|
to individual files.
\end{abstract}

\begingroup
\parskip0ex
\tableofcontents
\endgroup

%%%%%%%%%%%%%%%%%%%%%%%%%%%%%%%%%%%%%%%%%%%%%%%%%%%%%%%%%%%%%%%%%%%%%%%%%%%%%%%%
%%%%%%%%%%%%%%%%%%%%%%%%%%%%%%%%%%%%%%%%%%%%%%%%%%%%%%%%%%%%%%%%%%%%%%%%%%%%%%%%
\section{Introduction}

\LaTeX{} provides a mechanism to structure a large document (such as a book)
into a main file and several child files (containing the chapters)
using the |\include| command.
This mechanism is beneficial for documents
which span hundreds of pages in order to
make the source file(s) more manageable.
Moreover, compilation can be restricted to
selected child files by means of the |\includeonly| command.
The latter feature can be used to reduce the compilation time while editing
(this was significantly more useful in the earlier days of \LaTeX{})
or to generate a smaller document which is easier to navigate.
Another application of |\includeonly| is to generate
documents consisting of selected parts of the complete document.

However, there are a few drawbacks of the plain |\include| mechanism:
\begin{itemize}
\item
The child files cannot be compiled on their own,
they can only be compiled via the main file.
A naive editing environment
(such as a text editor with an option
to have the current file processed by \LaTeX)
may require one to switch to the main file before compiling;
attempting to compile the child file produces errors.
\item
The main file must be modified (each time)
to adjust the |\includeonly| command
to the present needs. This easily leaves the main file in a messy state.
\item
The generated document will always carry the filename
of the main document. This is inconvenient if
several child files are to be compiled and
to be kept for distribution.
\end{itemize}

The present package provides a simple interface
to make child files individually compilable by \LaTeX{}.
Compiling a child file then has the same effect as compiling
the main file with an |\includeonly| command
to select the appropriate child.
Moreover the generated document will carry the name of the child
rather than the main file.
This resolves all three above issues.

This feature is meant to make the editing of books,
thesis documents and lecture notes somewhat more convenient.
However, the package can also be used efficiently for
composing a series of documents (such as exercise sheets)
which are typically distributed individually.
It then assists the author in generating the individual documents
(potentially in different versions)
as well as a document containing the collected series.
Another application is in developing style files
or other kinds of included material
where compilation of the style file could redirect
to a sample or test file.

%%%%%%%%%%%%%%%%%%%%%%%%%%%%%%%%%%%%%%%%%%%%%%%%%%%%%%%%%%%%%%%%%%%%%%%%%%%%%%%%
%%%%%%%%%%%%%%%%%%%%%%%%%%%%%%%%%%%%%%%%%%%%%%%%%%%%%%%%%%%%%%%%%%%%%%%%%%%%%%%%
\section{Usage}

First of all, the package \textsf{childdoc} is \emph{not} a standard
\LaTeXe{} |.sty| style file! Therefore it needs to be invoked in
a non-standard way.

%%%%%%%%%%%%%%%%%%%%%%%%%%%%%%%%%%%%%%%%%%%%%%%%%%%%%%%%%%%%%%%%%%%%%%%%%%%%%%%%
\subsection{Included Files}
\label{sec:include}

%%%%%%%%%%%%%%%%%%%%%%%%%%%%%%%%%%%%%%%%
\DescribeMacro{\childdocmain}
To use the package, add the commands
\begin{center}
\begin{tabular}{l}
|\input{childdoc.def}|\\
|\childdocmain{}|\\
\end{tabular}
\end{center}
at the very top of the main \LaTeX{} file,
in particular \emph{before} the |\documentclass| statement!
The argument of |\childdocmain| should be left empty
(but it must be present).

%%%%%%%%%%%%%%%%%%%%%%%%%%%%%%%%%%%%%%%%
\DescribeMacro{\childdocof}
Furthermore, add the commands
\begin{center}
\begin{tabular}{l}
|\input{childdoc.def}|\\
|\childdocof{|\textit{main}|}|\\
\end{tabular}
\end{center}
at the top of every child file \textit{child}
which is included by |\include{|\textit{child}|}|
from within the main file
(or at least for those files to be compiled individually).
The argument \textit{main} must be the filename of the main file.

There are a couple of
considerations in setting up the main and child documents:

%%%%%%%%%%%%%%%%%%%%%%%%%%%%%%%%%%%%%%%%
\paragraph{Restrictions.}

Please note the following restrictions:
\begin{itemize}
\item
|\childdocmain| must be called with one argument \textit{main}
to ensure compatibility with earlier version of the package.
It must either be empty (|\childdocmain{}|)
or precisely match the filename of the main file in which it is specified.
See \secref{sec:detection} for further information.
\item
The filename \textit{main} must be specified without the |.tex| extension.
\item
The filename \textit{main} is case sensitive
(even in case-insensitive file systems)
due to internal string comparison.
\item
The argument \textit{main} should be fully expanded, it cannot be a macro.
\item
Subdirectories and special characters should be avoided in filenames.
\item
The command |\childdocmain{|\textit{main}|}| must be followed by a whitespace.
It should not be followed immediately by another command
or by a comment mark `|%|'.
This is because the \TeX{} parser reads the token immediately following
the argument of |\childdocmain| and puts it
at the beginning of every child section;
however, a white\-space is ignored.
\end{itemize}

%%%%%%%%%%%%%%%%%%%%%%%%%%%%%%%%%%%%%%%%
\paragraph{Content of Main File.}

It is advisable to place all content in the child files included by |\include|.
Any output contained in the main file will appear in all child documents
unless suppressed manually;
it cannot be suppressed automatically by the |\includeonly| directive
and thus should normally be avoided.
A method to include some content in the main file
by means of conditional processing is described in \secref{sec:conditional}.

%%%%%%%%%%%%%%%%%%%%%%%%%%%%%%%%%%%%%%%%
\paragraph{Page Numbering.}

When only a part of the document is compiled,
the appropriate numbering of pages
(as well as other status parameters)
is determined from the |.aux| files.
The latter contain information from previous passes.
However this information needs to propagate through
all intermediate child documents.
Therefore the page numbering in child documents may well
be inconsistent until the complete document is compiled at least once.

A useful (if unconventional) way to always ensure a consistent
page numbering is to restart the numbering in each child document
and denote the pages by `\textit{child}|.|\textit{page}'
where \textit{child} represents the chapter/section number of the child file.
This can be achieved by the command
|\numberwithin{page}{|\textit{child}|}|
of the \textsf{amsmath} package
where \textit{child} can be |chapter| or |section|
depending on the chosen structuring.
Alternatively, one can modify the macro |\thepage| appropriately
and reset the counter |page| at the start of each child file.

%%%%%%%%%%%%%%%%%%%%%%%%%%%%%%%%%%%%%%%%%%%%%%%%%%%%%%%%%%%%%%%%%%%%%%%%%%%%%%%%
\subsection{Conditional Processing}
\label{sec:conditional}

The package provides a mechanism to compile different versions
of a document. To customise the versions further some conditional processing
can come in handy to distinguish which version is being compiled.
The package provides two macros to describe the compilation context:

%%%%%%%%%%%%%%%%%%%%%%%%%%%%%%%%%%%%%%%%
\DescribeMacro{\ifchilddoc}
The conditional |\ifchilddoc| distinguishes between the compilation of
child documents and the main document:
%
\begin{center}
|\ifchilddoc |\textit{child-code}| |[|\||else |\textit{main-code}]| \||fi|
\end{center}

%%%%%%%%%%%%%%%%%%%%%%%%%%%%%%%%%%%%%%%%
\DescribeMacro{\childdocname}
\DescribeMacro{\childdocjob}
The macro |\childdocname| contains the filename (without extension)
of the main or child file being processed.
Note that |\childdocjob| will always contain the name of the main file.

%%%%%%%%%%%%%%%%%%%%%%%%%%%%%%%%%%%%%%%%
\paragraph{Title Page.}

Conditional processing can be used to include a title or banner page
in the main document when proper precautions are taken.
Importantly, the code in the main file should ensure that the page counter
(as well as other status parameters which are stored in the |.aux| files)
takes the same value after the conditional processing.
Otherwise the page numbers may take divergent values
depending on which part is compiled.

For example, a title page could be declared by:
%
\begin{center}
\begin{tabular}{l}
|\ifchilddoc\||else|\\
|\addtocounter{page}{-1}|\\
\textit{code for title page}\\
|\newpage|\\
|\||fi|
\end{tabular}
\end{center}
%
A banner page for the child documents can be generated by:
%
\begin{center}
\begin{tabular}{l}
|\ifchilddoc|\\
|\addtocounter{page}{-1}|\\
\textit{code for banner page}\\
|\newpage|\\
|\||fi|
\end{tabular}
\end{center}
%
Here one could write a message such as:
\begin{center}
|This is the part \childdocname{} of \childdocjob{}.|
\end{center}

%%%%%%%%%%%%%%%%%%%%%%%%%%%%%%%%%%%%%%%%%%%%%%%%%%%%%%%%%%%%%%%%%%%%%%%%%%%%%%%%
\subsection{Flags}
\label{sec:flags}

The package makes it easy to generate different versions
of the main or child documents.
To this end compilation flags can be defined
and assigned different default values.
They will be particularly useful in conjunction
with the forwarding mechanism described in \secref{sec:forward}.

For example, it may be useful to have a flag |\version|
which can be set to |draft| or |final|.
The document source will contain some conditional code
depending on the value of |\version|.
Suppose further, the flag should default to |final| for the main file
and to |draft| for child files
which is a natural assignment for editing the document.
This is achieved by placing the following code
in the preamble of the main document
(below the |\childdocmain| directive):
%
\begin{center}
\begin{tabular}{l}
|\ifchilddoc|\\
|\providecommand{\version}{draft}|\\
|\||else|\\
|\providecommand{\version}{final}|\\
|\||fi|
\end{tabular}
\end{center}
%
The definition by |\providecommand| makes sure
that previous definitions are not overwritten.
Further statements |\providecommand{\version}{...}|
can thus be added before the above code to override it.

For the main file, one might add a line
(between |\childdocmain| and the above block)
%
\begin{center}
|%\ifchilddoc\||else\providecommand{\version}{draft}\||fi|
\end{center}
%
which can be uncommented to produce a draft version.
Likewise one can add a line to the very top of a child file
(above the |\childdocof{|\textit{main}|}| directive)
%
\begin{center}
|%\providecommand{\version}{final}|
\end{center}
%
which can be uncommented to produce the final version of this child document.

%%%%%%%%%%%%%%%%%%%%%%%%%%%%%%%%%%%%%%%%%%%%%%%%%%%%%%%%%%%%%%%%%%%%%%%%%%%%%%%%
\subsection{Forwarding}
\label{sec:forward}

Different versions of the main or child documents
using compilation flags as described in \secref{sec:flags}
can be (permanently) stored in different files
for convenient compilation, viewing and distribution.
To this end, the package defines a command
to pass on compilation to a different file:

%%%%%%%%%%%%%%%%%%%%%%%%%%%%%%%%%%%%%%%%
\DescribeMacro{\childdocforward}
The command |\childdocforward| redirects processing to
another source file:
%
\begin{center}
\begin{tabular}{l}
|\input{childdoc.def}|\\
|\childdocforward[|\textit{main}|]{|\textit{dest}|}|\\
\end{tabular}
\end{center}
%
The argument \textit{dest} is the destination file
(without extension).
It should be the main file or one of the child files.
Note that further \textsf{childdoc} directives
such as |\childdocof| and |\childdocforward|
in the indicated file will be processed in this form.
The optional argument \textit{main}
passes on directly to the main file \textit{main}
while pretending to compile the child \textit{dest}.
This form behaves as if \textit{dest}
issues |\childdocof{|\textit{main}|}| right away,
and no further \textsf{childdoc} directives will be processed.

%%%%%%%%%%%%%%%%%%%%%%%%%%%%%%%%%%%%%%%%
\DescribeMacro{\...prefix}
In the alternative form |\childdocforwardprefix|,
%
\begin{center}
\begin{tabular}{l}
|\input{childdoc.def}|\\
|\childdocforwardprefix[|\textit{main}|]{|\textit{prefix}|}{|\textit{dest}|}|
\end{tabular}
\end{center}
%
the destination file is determined by a pattern
depending on the current file:
To make this work, the current file must be called
`{\textit{prefix}\hspace{0.2em}\textit{suffix}}'
with \textit{prefix} matching precisely the argument.
Processing is then passed on to the file
`{\textit{dest}\hspace{0.2em}\textit{suffix}}'.
Surely, the same effect is achieved by
directly specifying the
argument `{\textit{dest}\hspace{0.2em}\textit{suffix}}'
in the first form.
However, that requires to set up a different file
for each child. With the alternative form of the command
all these files can have exactly the same content
which simplifies setting them up and maintaining them.

For example, the following file |draft.tex|
with a compilation flag |\version| as described in \secref{sec:flags}
compiles the main document as a draft:
%
\begin{center}
\begin{tabular}{l}
|\def\version{draft}|\\
|\input{childdoc.def}|\\
|\childdocforward{|\textit{main}|}|
\end{tabular}
\end{center}
%
Likewise, the following files |final|\textit{nn}|.tex|
compile the final version of the child document
|child|\textit{nn}|.tex|:
%
\begin{center}
\begin{tabular}{l}
|\def\version{final}|\\
|\input{childdoc.def}|\\
|\childdocforwardprefix{final}{child}|
\end{tabular}
\end{center}
%

Note that when several versions of a main file and/or of each child file
are to be generated, it may be convenient to set up a |Makefile| or
shell script to automatise the process.

%%%%%%%%%%%%%%%%%%%%%%%%%%%%%%%%%%%%%%%%%%%%%%%%%%%%%%%%%%%%%%%%%%%%%%%%%%%%%%%%
\subsection{Command Line Processing}
\label{sec:commandline}

The effect of redirection files can also be achieved by invoking
the \LaTeX{} compiler with a more elaborate command line.
Most conveniently this should be done as part
of a shell script or a |Makefile|.

When using \textsf{childdoc} in the main file, the following
command lines effectively perform a redirection
(note that depending on the shell being used,
backslashes may have to be doubled: `|\|' $\to$ `|\\|'):
%
\begin{center}
|... -jobname "|\textit{target}|" |\\|"|[\textit{flags}]%
|\input{childdoc.def}\childdocforward[|\textit{main}|]{|\textit{dest}|}"|
\end{center}
%
Here \textit{target} is the name of the output file,
\textit{main} is the name of the main file
and \textit{dest} is the name of the main or child file to be processed
(all filenames without extensions).
The optional argument \textit{main} can be omitted
if \textit{main} matches \textit{dest}.
Optionally, compilation \textit{flags} can be defined via |\def| commands.
This command line makes the \TeX{} engine believe
it is compiling the file \textit{target}
whose content is specified as the latter parameter.
The provided code then forwards the processing to
\textit{main} or \textit{dest} as described in \secref{sec:forward}.

%%%%%%%%%%%%%%%%%%%%%%%%%%%%%%%%%%%%%%%%%%%%%%%%%%%%%%%%%%%%%%%%%%%%%%%%%%%%%%%%
\subsection{Include by Input}
\label{sec:input}

Including child documents by |\include| has some restrictions by design.
Most notably, the content of a child document always occupies
its own set of pages; pages cannot be shared between child documents.
Usually, this behaviour makes perfect sense
because each child document contain an essential part of the document.
However, in some situations it may be desirable to compose
a document from a collection of parts
without having mandatory page breaks between then.
For this case, the package
provides a mechanism to include parts
by |\input| which can also be processed individually.
However, by construction this mechanism
requires manual handling of the content to be output.

%%%%%%%%%%%%%%%%%%%%%%%%%%%%%%%%%%%%%%%%
\DescribeMacro{\ifchilddocmanual}
The main file should be prepared as usual, see \secref{sec:include}.
However, the document body must make a distinction
between processing of an individual part and of the main document, e.g.:
%
\begin{center}
\begin{tabular}{l}
|\ifchilddocmanual|\\
|\input{\childdocname}|\\
|\||else|\\
\textit{document body with }|\input{|\textit{part}|}|\\
|\||fi|
\end{tabular}
\end{center}
%
The conditional |\ifchilddocmanual| is true whenever
a part to be included by |\input| is being compiled,
and the name of the part is stored in |\childdocname|.

%%%%%%%%%%%%%%%%%%%%%%%%%%%%%%%%%%%%%%%%
\DescribeMacro{\childdocby}
Each part to be included by |\input| should start with:
%
\begin{center}
\begin{tabular}{l}
|\input{childdoc.def}|\\
|\childdocby{|\textit{main}|}|\\
\end{tabular}
\end{center}
%
The directive |\childdocby| is similar to |\childdocof|
described in \secref{sec:include},
but the subsequent selection of content must be done manually.
To that end, both |\ifchilddoc| and |\ifchilddocmanual|
will be true upon processing of a part,
and the name of the part is stored in |\childdocname|.
Note that |\jobname| will be set to the filename of the current part
so that each part receives an individual |.aux| file
that does not interfere with the |.aux| file(s) of the main document.
This behaviour can be altered by the alternative form
|\childdocby[*]{|\textit{main}|}| (with a non-empty optional argument)
which uses the |.aux| file of the main document
by setting |\jobname| to \textit{main}.

%%%%%%%%%%%%%%%%%%%%%%%%%%%%%%%%%%%%%%%%%%%%%%%%%%%%%%%%%%%%%%%%%%%%%%%%%%%%%%%%
\subsection{Driver Development}
\label{sec:driver}

The \textsf{childdoc} mechanism can also be use for the development
of definition files such as \LaTeX{} styles or classes.
This case differs from the above setup with multiple parts
included by |\include| in that no |\includeonly| should be invoked.
This can be achieved by starting the include file
(before |\ProvidesPackage|) with:
%
\begin{center}
\begin{tabular}{l}
|\input{childdoc.def}|\\
|\childdocforward{|\textit{main}|}|\\
\end{tabular}
\end{center}
%
or alternatively with:
%
\begin{center}
\begin{tabular}{l}
|\input{childdoc.def}|\\
|\childdocby{|\textit{main}|}|\\
\end{tabular}
\end{center}
%
Both forms have slightly different effects as described above.
The main file is prepared as usual, see \secref{sec:include}.

%%%%%%%%%%%%%%%%%%%%%%%%%%%%%%%%%%%%%%%%%%%%%%%%%%%%%%%%%%%%%%%%%%%%%%%%%%%%%%%%
\subsection{Legacy Detection}
\label{sec:detection}

The directive |\childdocmain| in the main file can detect
whether the complete document or merely a child is to be compiled
even without using the directive |\childdocof|.
This method is deprecated because it is less robust
and there is no compelling reason to use it;
it is merely provided for backward compatibility
and it may be removed in future versions.

If the detection mechanism is to be used,
it is mandatory to correctly specify
the filename of the main file as the argument of |\childdocmain|:
%
\begin{center}
\begin{tabular}{l}
|\input{childdoc.def}|\\
|\childdocmain{|\textit{main}|}|\\
\end{tabular}
\end{center}
%
If |\jobname| does not match the argument \textit{main} of |\childdocmain|,
it is assumed that |\jobname| points to the child file to be compiled.
When using |\childdocmain| with the main file specified as argument,
it suffices to start a child file
with just |\input{|\textit{main}|}|
without loading of the package and using |\childdocof|.
If instead all processing is done
with the appropriate \textsf{childdoc} directives,
the argument of \textit{main} of |\childdocmain| can be empty.

An alternative version of the command line processing described
in \secref{sec:commandline} using the detection mechanism reads:
%
\begin{center}
|... -jobname "|\textit{target}|" "|[\textit{flags}]%
[|\def\jobname{|\textit{dest}|}|]|\input{|\textit{main}|}"|
\end{center}

%%%%%%%%%%%%%%%%%%%%%%%%%%%%%%%%%%%%%%%%%%%%%%%%%%%%%%%%%%%%%%%%%%%%%%%%%%%%%%%%
\subsection{Manual Code}
\label{sec:manual}

In case one cannot be certain whether the definitions file |childdoc.def|
is installed on the target \TeX{} distribution
and one prefers not to ship it,
it is conceivable to paste a few relevant commands into the sources.

To that end, drop all statements |\input{childdoc.def}|
and perform the replacements as outlined below.
Instead of |\childdocmain{|\textit{main}|}| add the following code
to the top of the main file:
%
\begin{center}
\begin{tabular}{l}
|\||ifdefined\childdocname\endinput\||fi\newif\ifchilddoc|\\
|\edef\childdocname{\scantokens\expandafter{\jobname\noexpand}}|\\
|\def\childdocmain{|\textit{main}|}\||ifx\childdocmain\childdocname\||else|\\
|\childdoctrue\includeonly{\childdocname}\let\jobname\childdocmain\||fi|\\
\end{tabular}
\end{center}
%
Instead of |\childdocof{|\textit{main}|}| just include the main file
at the top of each child file:
%
\begin{center}
|\input{|\textit{main}|}|
\end{center}
%
A simple redirection |\childdocforward{|\textit{dest}|}| is achieved by:
%
\begin{center}
|\def\jobname{|\textit{dest}|}\input{\jobname}|
\end{center}
%
The redirection with prefix
|\childdocforwardprefix[|\textit{prefix}|]{|\textit{dest}|}|
is accomplished by:
%
\begin{center}
\begin{tabular}{l}
|{\edef\jobname{\scantokens\expandafter{\jobname\noexpand}}|\\
|\def\redirectjob |\textit{prefix}|#1~~~{\gdef\jobname{|\textit{dest}|#1}}|\\
|\expandafter\redirectjob\jobname~~~}\input{\jobname}|
\end{tabular}
\end{center}

In an alternative approach,
child documents can be compiled by a specific command line
without additional code or specific definitions:
%
\begin{center}
|... -jobname "|\textit{target}|" "|[\textit{flags}]%
|\includeonly{|\textit{dest}|}\input{|\textit{main}|}"|
\end{center}
%

%%%%%%%%%%%%%%%%%%%%%%%%%%%%%%%%%%%%%%%%%%%%%%%%%%%%%%%%%%%%%%%%%%%%%%%%%%%%%%%%
%%%%%%%%%%%%%%%%%%%%%%%%%%%%%%%%%%%%%%%%%%%%%%%%%%%%%%%%%%%%%%%%%%%%%%%%%%%%%%%%
\section{Information}

%%%%%%%%%%%%%%%%%%%%%%%%%%%%%%%%%%%%%%%%%%%%%%%%%%%%%%%%%%%%%%%%%%%%%%%%%%%%%%%%
\subsection{Copyright}

Copyright \copyright{} 2017--2018 Niklas Beisert

This work may be distributed and/or modified under the
conditions of the \LaTeX{} Project Public License, either version 1.3
of this license or (at your option) any later version.
The latest version of this license is in
  \url{http://www.latex-project.org/lppl.txt}
and version 1.3 or later is part of all distributions of \LaTeX{}
version 2005/12/01 or later.

This work has the LPPL maintenance status `maintained'.

The Current Maintainer of this work is Niklas Beisert.

This work consists of the files |README.txt|, |childdoc.ins| and |childdoc.dtx|
as well as the derived files |childdoc.def|, |cdocsamp.tex|
with |cdocsch1.tex|, |cdocsch2.tex|, |cdocspt3.tex|, |cdocspt4.tex|,
|cdocsdrf.tex|, |cdocsfn1.tex|, |cdocsfn2.tex|
as well as |childdoc.pdf|.

%%%%%%%%%%%%%%%%%%%%%%%%%%%%%%%%%%%%%%%%%%%%%%%%%%%%%%%%%%%%%%%%%%%%%%%%%%%%%%%%
\subsection{Files and Installation}

The package consists of the files:
%
\begin{center}
\begin{tabular}{ll}
    |README.txt|   & readme file \\
    |childdoc.ins| & installation file \\
    |childdoc.dtx| & source file \\
    |childdoc.def| & definition file \\
    |cdocsamp.tex| & sample main file \\
    |cdocsch1.tex| & sample include file \\
    |cdocsch2.tex| & sample include file \\
    |cdocspt3.tex| & sample part file \\
    |cdocspt4.tex| & sample part file \\
    |cdocsdrf.tex| & sample redirection file \\
    |cdocsfn1.tex| & sample redirection file \\
    |cdocsfn2.tex| & sample redirection file \\
    |childdoc.pdf| & manual
\end{tabular}
\end{center}
%
The distribution consists of the files
|README.txt|, |childdoc.ins| and |childdoc.dtx|.
%
\begin{itemize}
\item
Run (pdf)\LaTeX{} on |childdoc.dtx|
to compile the manual |childdoc.pdf| (this file).
\item
Run \LaTeX{} on |childdoc.ins| to create the definitions file |childdoc.def|
and the sample |cdocsamp.tex| with include files
|cdocsch1.tex|, |cdocsch2.tex|, |cdocspt3.tex|, |cdocspt4.tex|,
|cdocsdrf.tex|, |cdocsfn1.tex|, |cdocsfn2.tex|.
Then copy the file |childdoc.def| to an appropriate directory of your \LaTeX{}
distribution, e.g.\ \textit{texmf-root}|/tex/latex/childdoc|.
\end{itemize}

%%%%%%%%%%%%%%%%%%%%%%%%%%%%%%%%%%%%%%%%%%%%%%%%%%%%%%%%%%%%%%%%%%%%%%%%%%%%%%%%
\subsection{Related CTAN Packages}

There are several other packages which offer a similar functionality:
%
\begin{itemize}
\item
The packages
\href{http://ctan.org/pkg/docmute}{\textsf{docmute}},
\href{http://ctan.org/pkg/includex}{\textsf{includex}} and
\href{http://ctan.org/pkg/standalone}{\textsf{standalone}}
provide commands to include only the document body of
a child file thus allowing both files to be compiled individually.
\item
The packages \href{http://ctan.org/pkg/subdocs}{\textsf{subdocs}}
and \href{http://ctan.org/pkg/subfiles}{\textsf{subfiles}}
provide structures in which the main and child documents can be
encapsulated and allowing them to be compiled individually.
The inclusion mechanism is different from the conventional |\include|.
\item
The package \href{http://ctan.org/pkg/combine}{\textsf{combine}}
is an elaborate solution to combine several documents into one.
\end{itemize}
%
See also the CTAN topic \href{http://ctan.org/topic/subdocs}{\textsf{subdocs}}
for further related packages.
The present package differs from the above solutions in that
a document structure constructed with the conventional |\include| mechanism
just needs two extra commands at the top of every file
such that all constituent files can be compiled individually.

%%%%%%%%%%%%%%%%%%%%%%%%%%%%%%%%%%%%%%%%%%%%%%%%%%%%%%%%%%%%%%%%%%%%%%%%%%%%%%%%
%\subsection{Feature Suggestions}
%
%The following is a list of features which may be useful for future
%versions of this package:
%%
%\begin{itemize}
%\item
%\ldots
%\end{itemize}

%%%%%%%%%%%%%%%%%%%%%%%%%%%%%%%%%%%%%%%%%%%%%%%%%%%%%%%%%%%%%%%%%%%%%%%%%%%%%%%%
\subsection{Revision History}

%%%%%%%%%%%%%%%%%%%%%%%%%%%%%%%%%%%%%%%%
\paragraph{v2.0:} 2018/12/30

\begin{itemize}
\item
immediate forward processing
\item
added |\childdocby| mechanism
\item
manual restructured
\end{itemize}

%%%%%%%%%%%%%%%%%%%%%%%%%%%%%%%%%%%%%%%%
\paragraph{v1.6:} 2018/01/17

\begin{itemize}
\item
application for development of include files
\item
corrections to manual
\end{itemize}

%%%%%%%%%%%%%%%%%%%%%%%%%%%%%%%%%%%%%%%%
\paragraph{v1.5:} 2017/05/21

\begin{itemize}
\item
more complete structuring introduced
\item
|\childdocof| introduced
\item
|\childdoc| renamed to |\childdocmain|
\item
|\childredirect| renamed to |\childdocforward| and |\childdocforwardprefix|
and functionality expanded
\end{itemize}

%%%%%%%%%%%%%%%%%%%%%%%%%%%%%%%%%%%%%%%%
\paragraph{v1.0:} 2017/04/27

\begin{itemize}
\item
manual and install package
\item
first version published on CTAN
\end{itemize}

%%%%%%%%%%%%%%%%%%%%%%%%%%%%%%%%%%%%%%%%
\paragraph{v0.6:} 2017/04/26

\begin{itemize}
\item
redirection mechanism added
\end{itemize}

%%%%%%%%%%%%%%%%%%%%%%%%%%%%%%%%%%%%%%%%
\paragraph{v0.5:} 2017/04/26

\begin{itemize}
\item
functionality in definition file
\end{itemize}


%%%%%%%%%%%%%%%%%%%%%%%%%%%%%%%%%%%%%%%%%%%%%%%%%%%%%%%%%%%%%%%%%%%%%%%%%%%%%%%%
%%%%%%%%%%%%%%%%%%%%%%%%%%%%%%%%%%%%%%%%%%%%%%%%%%%%%%%%%%%%%%%%%%%%%%%%%%%%%%%%
%%%%%%%%%%%%%%%%%%%%%%%%%%%%%%%%%%%%%%%%%%%%%%%%%%%%%%%%%%%%%%%%%%%%%%%%%%%%%%%%
\appendix

\settowidth\MacroIndent{\rmfamily\scriptsize 000\ }

 \DocInput{childdoc.dtx}

\end{document}
%</driver>
% \fi
%
% %%%%%%%%%%%%%%%%%%%%%%%%%%%%%%%%%%%%%%%%%%%%%%%%%%%%%%%%%%%%%%%%%%%%%%%%%%%%%%
% %%%%%%%%%%%%%%%%%%%%%%%%%%%%%%%%%%%%%%%%%%%%%%%%%%%%%%%%%%%%%%%%%%%%%%%%%%%%%%
% \section{Sample}
%\iffalse
%<*samplemain>
%\fi
%
% The following presents a sample document
% with two chapters, two parts, a title page,
% a compile flag as well as three forwarding files to set the flag.
% It consists of eight |.tex| files:
% \begin{center}
% \begin{tabular}{ll}
% |cdocsamp.tex|&main file\\
% |cdocsch1.tex|&include file for chapter 1\\
% |cdocsch2.tex|&include file for chapter 2\\
% |cdocspt3.tex|&include file for part 3\\
% |cdocspt4.tex|&include file for part 4\\
% |cdocsdrf.tex|&forwarding file for main file in draft mode\\
% |cdocsfi1.tex|&forwarding file for final version of chapter 1\\
% |cdocsfi2.tex|&forwarding file for final version of chapter 2\\
% \end{tabular}
% \end{center}
% Each of the eight files can be compiled directly by the \LaTeX{} compiler.
%
% %%%%%%%%%%%%%%%%%%%%%%%%%%%%%%%%%%%%%%
% \paragraph{Main File.}
%
% The main file is called |cdocsamp.tex|.
%
% Load the \textsf{childdoc} definitions and
% declare the filename for the main document:
%    \begin{macrocode}
\input{childdoc.def}
\childdocmain{}
%    \end{macrocode}

% Optional override for |\version| flag:
%    \begin{macrocode}
%%\ifchilddoc\else\providecommand{\version}{draft}\fi
%    \end{macrocode}

% Define the default values for the |\version| flag
% (|final| for the main file and |draft| for childs):
%    \begin{macrocode}
\ifchilddoc
\providecommand{\version}{draft}
\else
\providecommand{\version}{final}
\fi
%    \end{macrocode}

% Load the standard document class:
%    \begin{macrocode}
\documentclass[12pt]{article}
%    \end{macrocode}

% Start the document body:
%    \begin{macrocode}
\begin{document}
%    \end{macrocode}

% Declare a title page.
% Print title, part of document being processed and version flag:
%    \begin{macrocode}
\addtocounter{page}{-1}
\begin{center}
{\LARGE\bfseries{}childdoc example\par}
\vspace{1cm}
\ifchilddoc
\ifchilddocmanual part\else chapter\fi:
`\childdocname' of `\childdocjob'\par
\else
main document: `\childdocjob'\par
\fi
version: \version\par
\end{center}
\newpage
%    \end{macrocode}

% Manually include selected file,
% otherwise process as usual:
%    \begin{macrocode}
\ifchilddocmanual
\section*{part `\childdocname'}
\input{\childdocname}
\else
%    \end{macrocode}

% Include the two chapters:
%    \begin{macrocode}
\include{cdocsch1}
\include{cdocsch2}
%    \end{macrocode}

% Include the two parts unless only chapters should be displayed:
%    \begin{macrocode}
\ifchilddoc\else
\section{part three}
\input{cdocspt3}
\section{part four}
\input{cdocspt4}
\fi
%    \end{macrocode}

% Process as usual until here:
%    \begin{macrocode}
\fi
%    \end{macrocode}

% End of document body:
%    \begin{macrocode}
\end{document}
%    \end{macrocode}
%\iffalse
%</samplemain>
%\fi
%
% %%%%%%%%%%%%%%%%%%%%%%%%%%%%%%%%%%%%%%
% \paragraph{Chapter Include Files.}
%
% The include files are called |cdocsch1.tex| and |cdocsch2.tex|.
%
%\iffalse
%<*samplechap1|samplechap2>
%\fi

% Optional override for |\version| flag:
%    \begin{macrocode}
%%\providecommand{\version}{final}
%    \end{macrocode}

% Include the main document:
%    \begin{macrocode}
\input{childdoc.def}
\childdocof{cdocsamp}
%    \end{macrocode}

%\iffalse
%</samplechap1|samplechap2>
%\fi
%
%\iffalse
%<*samplechap1>
%\fi
% Some text for chapter 1:
%    \begin{macrocode}
\section{one}
some text in chapter one
%    \end{macrocode}

%\iffalse
%</samplechap1>
%\fi
% Some text for chapter 2:
%\iffalse
%<*samplechap2>
%\fi
%    \begin{macrocode}
\section{two}
more text in chapter two
%    \end{macrocode}

%\iffalse
%</samplechap2>
%\fi
%
% %%%%%%%%%%%%%%%%%%%%%%%%%%%%%%%%%%%%%%
% \paragraph{Part Include Files.}
%
% The include files are called |cdocspt3.tex| and |cdocspt4.tex|.
%
%\iffalse
%<*samplepart3|samplepart4>
%\fi

% Optional override for |\version| flag:
%    \begin{macrocode}
%%\providecommand{\version}{final}
%    \end{macrocode}

% Include the main document:
%    \begin{macrocode}
\input{childdoc.def}
\childdocby{cdocsamp}
%    \end{macrocode}

%\iffalse
%</samplepart3|samplepart4>
%\fi
%
%\iffalse
%<*samplepart3>
%\fi
% Some text for part 3:
%    \begin{macrocode}
some text in part three
%    \end{macrocode}

%\iffalse
%</samplepart3>
%\fi
% Some text for part 4:
%\iffalse
%<*samplepart4>
%\fi
%    \begin{macrocode}
more text in part four
%    \end{macrocode}

%\iffalse
%</samplepart4>
%\fi
%
% %%%%%%%%%%%%%%%%%%%%%%%%%%%%%%%%%%%%%%
% \paragraph{Forwarding for a Complete Draft.}
%
% The following forwarding file |cdocsdrf.tex|
% compiles the main document in draft mode:
%\iffalse
%<*sampledraft>
%\fi
%    \begin{macrocode}
\def\version{draft}
\input{childdoc.def}
\childdocforward{cdocsamp}
%    \end{macrocode}

%\iffalse
%</sampledraft>
%\fi
%
% %%%%%%%%%%%%%%%%%%%%%%%%%%%%%%%%%%%%%%
% \paragraph{Forwarding for Final Version of the Chapters.}
%
% The following forwarding files |cdocsfn1.tex| and |cdocsfn2.tex|
% (with identical content)
% compile the final versions of the child documents
% |cdocsch1.tex| and |cdocsch2.tex|, respectively:
%\iffalse
%<*samplefinal>
%\fi
%    \begin{macrocode}
\def\version{final}
\input{childdoc.def}
\childdocforwardprefix[cdocsamp]{cdocsfn}{cdocsch}
%    \end{macrocode}

%\iffalse
%</samplefinal>
%\fi
%
% %%%%%%%%%%%%%%%%%%%%%%%%%%%%%%%%%%%%%%
% \paragraph{Command Line Processing.}
%
% The following three command lines generate the output files
% |cdocscld|, |cdocscl1| and |cdocscl2|
% which should be identical to
% |cdocsdrf|, |cdocsch1| and |cdocsfn2|, respectively:
% \begin{center}
% \begin{tabular}{l}
% |latex -jobname cdocscld \|\\
% |  "\def\version{draft}\input{childdoc.def}\childdocforward{cdocsamp}"|\\
% |latex -jobname cdocscl1 \|\\
% |  "\input{childdoc.def}\childdocforward[cdocsamp]{cdocsch1}"|\\
% |latex -jobname cdocscl2 \|\\
% |  "\def\version{final}\input{childdoc.def}\childdocforward{cdocsch2}"|
% \end{tabular}
% \end{center}
% Note that the trailing backslash on each first line
% merely continues the input to the second line
% (for convenient cut ant paste).
% Furthermore, the command |latex| can be replaced by any
% of its alternative versions such as |pdflatex|.
%
% %%%%%%%%%%%%%%%%%%%%%%%%%%%%%%%%%%%%%%%%%%%%%%%%%%%%%%%%%%%%%%%%%%%%%%%%%%%%%%
% %%%%%%%%%%%%%%%%%%%%%%%%%%%%%%%%%%%%%%%%%%%%%%%%%%%%%%%%%%%%%%%%%%%%%%%%%%%%%%
% \section{Implementation}
%\iffalse
%<*package>
%\fi
%
% This section describes the definitions file |childdoc.def|.

% The definitions cannot be loaded using |\usepackage| or |\RequirePackage|
% which has a mechanism to prevent loading a style file more than once.
% When loading the definitions by means of |\input|
% multiple instances have to be prevented manually:
%\iffalse
%This code needs to be before the `\ProvidesFile' directive
%which is defined at the beginning of this file.
%Therefore it is also placed there and commented out here.
%</package>
%<*discard>
%\fi
%    \begin{macrocode}
\ifdefined\childdocmain\endinput\fi
%    \end{macrocode}
%\iffalse
%</discard>
%<*package>
%\fi
%
% \macro{\ifchilddoc}
% \macro{\ifchilddocmanual}
% The conditional |\ifchilddoc| tells whether a
% child (true) or main (false) document is being compiled.
% The conditional |\ifchilddocmanual| tells whether
% the |\includeonly| mechanism is used (false) or
% the selection of child files must be performed manually (true).
% The definitions initialise to false:
%    \begin{macrocode}
\newif\ifchilddoc
\newif\ifchilddocmanual
%    \end{macrocode}

% \macro{\childdocname}
% \macro{\childdocjob}
% The macro |\childdocname| stores the name of the main document
% to be compiled. The macro |\childdocjob| stores the name of
% the document on which the \LaTeX{} compiler was originally invoked.
% The content of |\jobname| cannot be compared
% to filenames specified in the source due to different catcodes.
% The following code rescans |\jobname|, stores the result
% in |\childdocname| and saves a copy in |\childdocjob|:
%    \begin{macrocode}
\edef\childdocname{\scantokens\expandafter{\jobname\noexpand}}
\let\childdocjob\childdocname
%    \end{macrocode}

% \macro{\childdocdisable}
% The macro |\childdocdisable| prevents the main file
% from being processed more than once.
% At this stage, the main document command |\childdocmain|
% is assumed to be called once again where it should do nothing.
% Any subsequent call to it should prevent
% a secondary processing of the main document
% It overwrites the forwarding commands
% |\childdocof| and |\childdocforward|
% with empty macros to prevent further inclusions of the main document:
%    \begin{macrocode}
\newcommand{\childdocdisable}
{
  \renewcommand{\childdocmain}[1]{\renewcommand{\childdocmain}[1]{\endinput}}
  \renewcommand{\childdocof}[1]{}
  \renewcommand{\childdocby}[2][]{}
  \renewcommand{\childdocforward}[2][]{}
  \renewcommand{\childdocdisable}{}
}
%    \end{macrocode}

% \macro{\childdocmain}
% The macro |\childdocmain| is to be called at the top of the main file
% with nothing or the main filename (without extension) as argument.
% First, it breaks loops.
% If the argument is not empty and does not match |\childdocname|
% (which is set by the first inclusion of |childdoc.def|),
% |\ifchilddoc| is set to true, |\includeonly| is applied to the child file
% and |\jobname| is set to the main file
% (for proper handling of |.aux| files):
%    \begin{macrocode}
\newcommand{\childdocmain}[1]
{
  \childdocdisable\childdocmain{}
  \if?#1?\else
    \begingroup
      \def\childdoctmp{#1}
      \ifx\childdoctmp\childdocname
        \def\childdoctmp{}
      \else
        \def\childdoctmp
        {
          \childdoctrue
          \includeonly{\childdocname}
          \def\childdocjob{#1}
          \def\jobname{#1}
        }
      \fi
      \expandafter
    \endgroup
    \childdoctmp
  \fi
}
%    \end{macrocode}

% \macro{\childdocof}
% The command |\childdocof| redirects
% compilation to the main file |#1|.
%    \begin{macrocode}
\newcommand{\childdocof}[1]
{
  \childdocdisable
  \childdoctrue
  \includeonly{\childdocname}
  \def\jobname{#1}
  \def\childdocjob{#1}
  \input{#1}
}
%    \end{macrocode}

% \macro{\childdocby}
% The command |\childdocby| ....
%    \begin{macrocode}
\newcommand{\childdocby}[2][]
{
  \childdocdisable
  \childdoctrue
  \childdocmanualtrue
  \if?#1?\else
    \def\jobname{#2}
  \fi
  \def\childdocjob{#2}
  \input{#2}
  \endinput
}
%    \end{macrocode}

% \macro{\childdocforward}
% The command |\childdocforward| redirects
% compilation to the main file or
% (if the optional argument is given) a child file.
% Parameters are set as if the main file
% or a child file starting with |\childdocof| was compiled.
% Then compilation is handed over to the main file:
%    \begin{macrocode}
\newcommand{\childdocforward}[2][]
{
  \begingroup
    \if?#1?
      \def\childdoctmp
      {
        \def\childdocname{#2}
        \def\childdocjob{#2}
        \def\jobname{#2}
        \input{#2}
        \endinput
      }
    \else
      \def\childdoctmp
      {
        \childdocdisable
        \def\childdocname{#2}
        \childdoctrue
        \includeonly{#2}
        \def\childdocjob{#1}
        \def\jobname{#1}
        \input{#1}
        \endinput
      }
    \fi
    \expandafter
  \endgroup
  \childdoctmp
}
%    \end{macrocode}

% \macro{\childdocforwardprefix}
% The command |\childdocforwardprefix| redirects
% compilation to the main or a child file by means of a pattern.
% The prefix |#1| in the current filename is replaced by |#2|
% and the suffix of the current filename is kept
% (it is assumed that the filename does not contain the substring `|~~~|'
% which is used as a delimiter).
% Compilation is handed over to the new file by |\childdocforward|:
%    \begin{macrocode}
\newcommand{\childdocforwardprefix}[3][]
{
  \begingroup
    \def\childdocextract #2##1~~~{\def\childdoctmp{\childdocforward[#1]{#3##1}}}
    \expandafter\childdocextract\childdocname~~~
    \expandafter
  \endgroup
  \childdoctmp
}
%    \end{macrocode}

% \macro{\childdoc}
% The deprecated macro |\childdoc| is a legacy version of |\childdocmain|:
%    \begin{macrocode}
\newcommand{\childdoc}{\childdocmain}
%    \end{macrocode}

% \macro{\childdocredirect}
% The deprecated macro |\childdocredirect| is a legacy version
% of |\childdocforward| and |\childdocforwardprefix|:
%    \begin{macrocode}
\newcommand{\childdocredirect}[2][]
{
  \begingroup
    \if?#1?
      \def\childdoctmp{\childdocforward{#2}}
    \else
      \def\childdoctmp{\childdocforwardprefix{#1}{#2}}
    \fi
    \expandafter
  \endgroup
  \childdoctmp
}
%    \end{macrocode}

%\iffalse
%</package>
%\fi
%
\endinput
|\\
|\childdocforwardprefix{final}{child}|
\end{tabular}
\end{center}
%

Note that when several versions of a main file and/or of each child file
are to be generated, it may be convenient to set up a |Makefile| or
shell script to automatise the process.

%%%%%%%%%%%%%%%%%%%%%%%%%%%%%%%%%%%%%%%%%%%%%%%%%%%%%%%%%%%%%%%%%%%%%%%%%%%%%%%%
\subsection{Command Line Processing}
\label{sec:commandline}

The effect of redirection files can also be achieved by invoking
the \LaTeX{} compiler with a more elaborate command line.
Most conveniently this should be done as part
of a shell script or a |Makefile|.

When using \textsf{childdoc} in the main file, the following
command lines effectively perform a redirection
(note that depending on the shell being used,
backslashes may have to be doubled: `|\|' $\to$ `|\\|'):
%
\begin{center}
|... -jobname "|\textit{target}|" |\\|"|[\textit{flags}]%
|% \iffalse
%
% childdoc.dtx Copyright (C) 2017-2018 Niklas Beisert
%
% This work may be distributed and/or modified under the
% conditions of the LaTeX Project Public License, either version 1.3
% of this license or (at your option) any later version.
% The latest version of this license is in
%   http://www.latex-project.org/lppl.txt
% and version 1.3 or later is part of all distributions of LaTeX
% version 2005/12/01 or later.
%
% This work has the LPPL maintenance status `maintained'.
%
% The Current Maintainer of this work is Niklas Beisert.
%
% This work consists of the files childdoc.dtx and childdoc.ins
% and the derived files childdoc.def and cdocsamp.tex with
% cdocsch1.tex, cdocsch2.tex, cdocsdrf.tex, cdocsfn1.tex, cdocsfn2.tex.
%
%<package>\ifdefined\childdocmain\endinput\fi
%<package>\ProvidesFile{childdoc.def}[2018/12/30 v2.0 child document driver]
%<samplemain>\ProvidesFile{cdocsamp.tex}[2018/12/30 v2.0 sample for childdoc]
%<*driver>
%\ProvidesFile{childdoc.drv}[2018/12/30 v2.0 childdoc reference manual file]
\PassOptionsToClass{10pt,a4paper}{article}
\documentclass{ltxdoc}

\usepackage[margin=35mm]{geometry}
\usepackage{hyperref}
\usepackage{hyperxmp}
\usepackage[usenames]{color}

\hypersetup{colorlinks=true}
\hypersetup{pdfstartview=FitH}
\hypersetup{pdfpagemode=UseNone}
\hypersetup{pdfsource={}}
\hypersetup{pdflang={en-UK}}
\hypersetup{pdfcopyright={Copyright 2017-2018 Niklas Beisert.
  This work may be distributed and/or modified under the
  conditions of the LaTeX Project Public License, either version 1.3
  of this license or (at your option) any later version.}}
\hypersetup{pdflicenseurl={http://www.latex-project.org/lppl.txt}}
\hypersetup{pdfcontactaddress={ETH Zurich, ITP, HIT K,
  Wolfgang-Pauli-Strasse 27}}
\hypersetup{pdfcontactpostcode={8093}}
\hypersetup{pdfcontactcity={Zurich}}
\hypersetup{pdfcontactcountry={Switzerland}}
\hypersetup{pdfcontactemail={nbeisert@itp.phys.ethz.ch}}
\hypersetup{pdfcontacturl={http://people.phys.ethz.ch/\xmptilde nbeisert/}}

\newcommand{\secref}[1]{\hyperref[#1]{section \ref*{#1}}}

\parskip1ex
\parindent0pt
\let\olditemize\itemize
\def\itemize{\olditemize\parskip0pt}

\begin{document}

\title{The \textsf{childdoc} Package}
\hypersetup{pdftitle={The childdoc Package}}
\author{Niklas Beisert\\[2ex]
  Institut f\"ur Theoretische Physik\\
  Eidgen\"ossische Technische Hochschule Z\"urich\\
  Wolfgang-Pauli-Strasse 27, 8093 Z\"urich, Switzerland\\[1ex]
  \href{mailto:nbeisert@itp.phys.ethz.ch}
  {\texttt{nbeisert@itp.phys.ethz.ch}}}
\hypersetup{pdfauthor={Niklas Beisert}}
\hypersetup{pdfsubject={Manual for the LaTeX2e Package childdoc}}
\date{30 December 2018, \textsf{v2.0}}
\maketitle

\begin{abstract}\noindent
\textsf{childdoc} is a \LaTeXe{} package
that enables the direct compilation
of document sections included by |\include|
to individual files.
\end{abstract}

\begingroup
\parskip0ex
\tableofcontents
\endgroup

%%%%%%%%%%%%%%%%%%%%%%%%%%%%%%%%%%%%%%%%%%%%%%%%%%%%%%%%%%%%%%%%%%%%%%%%%%%%%%%%
%%%%%%%%%%%%%%%%%%%%%%%%%%%%%%%%%%%%%%%%%%%%%%%%%%%%%%%%%%%%%%%%%%%%%%%%%%%%%%%%
\section{Introduction}

\LaTeX{} provides a mechanism to structure a large document (such as a book)
into a main file and several child files (containing the chapters)
using the |\include| command.
This mechanism is beneficial for documents
which span hundreds of pages in order to
make the source file(s) more manageable.
Moreover, compilation can be restricted to
selected child files by means of the |\includeonly| command.
The latter feature can be used to reduce the compilation time while editing
(this was significantly more useful in the earlier days of \LaTeX{})
or to generate a smaller document which is easier to navigate.
Another application of |\includeonly| is to generate
documents consisting of selected parts of the complete document.

However, there are a few drawbacks of the plain |\include| mechanism:
\begin{itemize}
\item
The child files cannot be compiled on their own,
they can only be compiled via the main file.
A naive editing environment
(such as a text editor with an option
to have the current file processed by \LaTeX)
may require one to switch to the main file before compiling;
attempting to compile the child file produces errors.
\item
The main file must be modified (each time)
to adjust the |\includeonly| command
to the present needs. This easily leaves the main file in a messy state.
\item
The generated document will always carry the filename
of the main document. This is inconvenient if
several child files are to be compiled and
to be kept for distribution.
\end{itemize}

The present package provides a simple interface
to make child files individually compilable by \LaTeX{}.
Compiling a child file then has the same effect as compiling
the main file with an |\includeonly| command
to select the appropriate child.
Moreover the generated document will carry the name of the child
rather than the main file.
This resolves all three above issues.

This feature is meant to make the editing of books,
thesis documents and lecture notes somewhat more convenient.
However, the package can also be used efficiently for
composing a series of documents (such as exercise sheets)
which are typically distributed individually.
It then assists the author in generating the individual documents
(potentially in different versions)
as well as a document containing the collected series.
Another application is in developing style files
or other kinds of included material
where compilation of the style file could redirect
to a sample or test file.

%%%%%%%%%%%%%%%%%%%%%%%%%%%%%%%%%%%%%%%%%%%%%%%%%%%%%%%%%%%%%%%%%%%%%%%%%%%%%%%%
%%%%%%%%%%%%%%%%%%%%%%%%%%%%%%%%%%%%%%%%%%%%%%%%%%%%%%%%%%%%%%%%%%%%%%%%%%%%%%%%
\section{Usage}

First of all, the package \textsf{childdoc} is \emph{not} a standard
\LaTeXe{} |.sty| style file! Therefore it needs to be invoked in
a non-standard way.

%%%%%%%%%%%%%%%%%%%%%%%%%%%%%%%%%%%%%%%%%%%%%%%%%%%%%%%%%%%%%%%%%%%%%%%%%%%%%%%%
\subsection{Included Files}
\label{sec:include}

%%%%%%%%%%%%%%%%%%%%%%%%%%%%%%%%%%%%%%%%
\DescribeMacro{\childdocmain}
To use the package, add the commands
\begin{center}
\begin{tabular}{l}
|\input{childdoc.def}|\\
|\childdocmain{}|\\
\end{tabular}
\end{center}
at the very top of the main \LaTeX{} file,
in particular \emph{before} the |\documentclass| statement!
The argument of |\childdocmain| should be left empty
(but it must be present).

%%%%%%%%%%%%%%%%%%%%%%%%%%%%%%%%%%%%%%%%
\DescribeMacro{\childdocof}
Furthermore, add the commands
\begin{center}
\begin{tabular}{l}
|\input{childdoc.def}|\\
|\childdocof{|\textit{main}|}|\\
\end{tabular}
\end{center}
at the top of every child file \textit{child}
which is included by |\include{|\textit{child}|}|
from within the main file
(or at least for those files to be compiled individually).
The argument \textit{main} must be the filename of the main file.

There are a couple of
considerations in setting up the main and child documents:

%%%%%%%%%%%%%%%%%%%%%%%%%%%%%%%%%%%%%%%%
\paragraph{Restrictions.}

Please note the following restrictions:
\begin{itemize}
\item
|\childdocmain| must be called with one argument \textit{main}
to ensure compatibility with earlier version of the package.
It must either be empty (|\childdocmain{}|)
or precisely match the filename of the main file in which it is specified.
See \secref{sec:detection} for further information.
\item
The filename \textit{main} must be specified without the |.tex| extension.
\item
The filename \textit{main} is case sensitive
(even in case-insensitive file systems)
due to internal string comparison.
\item
The argument \textit{main} should be fully expanded, it cannot be a macro.
\item
Subdirectories and special characters should be avoided in filenames.
\item
The command |\childdocmain{|\textit{main}|}| must be followed by a whitespace.
It should not be followed immediately by another command
or by a comment mark `|%|'.
This is because the \TeX{} parser reads the token immediately following
the argument of |\childdocmain| and puts it
at the beginning of every child section;
however, a white\-space is ignored.
\end{itemize}

%%%%%%%%%%%%%%%%%%%%%%%%%%%%%%%%%%%%%%%%
\paragraph{Content of Main File.}

It is advisable to place all content in the child files included by |\include|.
Any output contained in the main file will appear in all child documents
unless suppressed manually;
it cannot be suppressed automatically by the |\includeonly| directive
and thus should normally be avoided.
A method to include some content in the main file
by means of conditional processing is described in \secref{sec:conditional}.

%%%%%%%%%%%%%%%%%%%%%%%%%%%%%%%%%%%%%%%%
\paragraph{Page Numbering.}

When only a part of the document is compiled,
the appropriate numbering of pages
(as well as other status parameters)
is determined from the |.aux| files.
The latter contain information from previous passes.
However this information needs to propagate through
all intermediate child documents.
Therefore the page numbering in child documents may well
be inconsistent until the complete document is compiled at least once.

A useful (if unconventional) way to always ensure a consistent
page numbering is to restart the numbering in each child document
and denote the pages by `\textit{child}|.|\textit{page}'
where \textit{child} represents the chapter/section number of the child file.
This can be achieved by the command
|\numberwithin{page}{|\textit{child}|}|
of the \textsf{amsmath} package
where \textit{child} can be |chapter| or |section|
depending on the chosen structuring.
Alternatively, one can modify the macro |\thepage| appropriately
and reset the counter |page| at the start of each child file.

%%%%%%%%%%%%%%%%%%%%%%%%%%%%%%%%%%%%%%%%%%%%%%%%%%%%%%%%%%%%%%%%%%%%%%%%%%%%%%%%
\subsection{Conditional Processing}
\label{sec:conditional}

The package provides a mechanism to compile different versions
of a document. To customise the versions further some conditional processing
can come in handy to distinguish which version is being compiled.
The package provides two macros to describe the compilation context:

%%%%%%%%%%%%%%%%%%%%%%%%%%%%%%%%%%%%%%%%
\DescribeMacro{\ifchilddoc}
The conditional |\ifchilddoc| distinguishes between the compilation of
child documents and the main document:
%
\begin{center}
|\ifchilddoc |\textit{child-code}| |[|\||else |\textit{main-code}]| \||fi|
\end{center}

%%%%%%%%%%%%%%%%%%%%%%%%%%%%%%%%%%%%%%%%
\DescribeMacro{\childdocname}
\DescribeMacro{\childdocjob}
The macro |\childdocname| contains the filename (without extension)
of the main or child file being processed.
Note that |\childdocjob| will always contain the name of the main file.

%%%%%%%%%%%%%%%%%%%%%%%%%%%%%%%%%%%%%%%%
\paragraph{Title Page.}

Conditional processing can be used to include a title or banner page
in the main document when proper precautions are taken.
Importantly, the code in the main file should ensure that the page counter
(as well as other status parameters which are stored in the |.aux| files)
takes the same value after the conditional processing.
Otherwise the page numbers may take divergent values
depending on which part is compiled.

For example, a title page could be declared by:
%
\begin{center}
\begin{tabular}{l}
|\ifchilddoc\||else|\\
|\addtocounter{page}{-1}|\\
\textit{code for title page}\\
|\newpage|\\
|\||fi|
\end{tabular}
\end{center}
%
A banner page for the child documents can be generated by:
%
\begin{center}
\begin{tabular}{l}
|\ifchilddoc|\\
|\addtocounter{page}{-1}|\\
\textit{code for banner page}\\
|\newpage|\\
|\||fi|
\end{tabular}
\end{center}
%
Here one could write a message such as:
\begin{center}
|This is the part \childdocname{} of \childdocjob{}.|
\end{center}

%%%%%%%%%%%%%%%%%%%%%%%%%%%%%%%%%%%%%%%%%%%%%%%%%%%%%%%%%%%%%%%%%%%%%%%%%%%%%%%%
\subsection{Flags}
\label{sec:flags}

The package makes it easy to generate different versions
of the main or child documents.
To this end compilation flags can be defined
and assigned different default values.
They will be particularly useful in conjunction
with the forwarding mechanism described in \secref{sec:forward}.

For example, it may be useful to have a flag |\version|
which can be set to |draft| or |final|.
The document source will contain some conditional code
depending on the value of |\version|.
Suppose further, the flag should default to |final| for the main file
and to |draft| for child files
which is a natural assignment for editing the document.
This is achieved by placing the following code
in the preamble of the main document
(below the |\childdocmain| directive):
%
\begin{center}
\begin{tabular}{l}
|\ifchilddoc|\\
|\providecommand{\version}{draft}|\\
|\||else|\\
|\providecommand{\version}{final}|\\
|\||fi|
\end{tabular}
\end{center}
%
The definition by |\providecommand| makes sure
that previous definitions are not overwritten.
Further statements |\providecommand{\version}{...}|
can thus be added before the above code to override it.

For the main file, one might add a line
(between |\childdocmain| and the above block)
%
\begin{center}
|%\ifchilddoc\||else\providecommand{\version}{draft}\||fi|
\end{center}
%
which can be uncommented to produce a draft version.
Likewise one can add a line to the very top of a child file
(above the |\childdocof{|\textit{main}|}| directive)
%
\begin{center}
|%\providecommand{\version}{final}|
\end{center}
%
which can be uncommented to produce the final version of this child document.

%%%%%%%%%%%%%%%%%%%%%%%%%%%%%%%%%%%%%%%%%%%%%%%%%%%%%%%%%%%%%%%%%%%%%%%%%%%%%%%%
\subsection{Forwarding}
\label{sec:forward}

Different versions of the main or child documents
using compilation flags as described in \secref{sec:flags}
can be (permanently) stored in different files
for convenient compilation, viewing and distribution.
To this end, the package defines a command
to pass on compilation to a different file:

%%%%%%%%%%%%%%%%%%%%%%%%%%%%%%%%%%%%%%%%
\DescribeMacro{\childdocforward}
The command |\childdocforward| redirects processing to
another source file:
%
\begin{center}
\begin{tabular}{l}
|\input{childdoc.def}|\\
|\childdocforward[|\textit{main}|]{|\textit{dest}|}|\\
\end{tabular}
\end{center}
%
The argument \textit{dest} is the destination file
(without extension).
It should be the main file or one of the child files.
Note that further \textsf{childdoc} directives
such as |\childdocof| and |\childdocforward|
in the indicated file will be processed in this form.
The optional argument \textit{main}
passes on directly to the main file \textit{main}
while pretending to compile the child \textit{dest}.
This form behaves as if \textit{dest}
issues |\childdocof{|\textit{main}|}| right away,
and no further \textsf{childdoc} directives will be processed.

%%%%%%%%%%%%%%%%%%%%%%%%%%%%%%%%%%%%%%%%
\DescribeMacro{\...prefix}
In the alternative form |\childdocforwardprefix|,
%
\begin{center}
\begin{tabular}{l}
|\input{childdoc.def}|\\
|\childdocforwardprefix[|\textit{main}|]{|\textit{prefix}|}{|\textit{dest}|}|
\end{tabular}
\end{center}
%
the destination file is determined by a pattern
depending on the current file:
To make this work, the current file must be called
`{\textit{prefix}\hspace{0.2em}\textit{suffix}}'
with \textit{prefix} matching precisely the argument.
Processing is then passed on to the file
`{\textit{dest}\hspace{0.2em}\textit{suffix}}'.
Surely, the same effect is achieved by
directly specifying the
argument `{\textit{dest}\hspace{0.2em}\textit{suffix}}'
in the first form.
However, that requires to set up a different file
for each child. With the alternative form of the command
all these files can have exactly the same content
which simplifies setting them up and maintaining them.

For example, the following file |draft.tex|
with a compilation flag |\version| as described in \secref{sec:flags}
compiles the main document as a draft:
%
\begin{center}
\begin{tabular}{l}
|\def\version{draft}|\\
|\input{childdoc.def}|\\
|\childdocforward{|\textit{main}|}|
\end{tabular}
\end{center}
%
Likewise, the following files |final|\textit{nn}|.tex|
compile the final version of the child document
|child|\textit{nn}|.tex|:
%
\begin{center}
\begin{tabular}{l}
|\def\version{final}|\\
|\input{childdoc.def}|\\
|\childdocforwardprefix{final}{child}|
\end{tabular}
\end{center}
%

Note that when several versions of a main file and/or of each child file
are to be generated, it may be convenient to set up a |Makefile| or
shell script to automatise the process.

%%%%%%%%%%%%%%%%%%%%%%%%%%%%%%%%%%%%%%%%%%%%%%%%%%%%%%%%%%%%%%%%%%%%%%%%%%%%%%%%
\subsection{Command Line Processing}
\label{sec:commandline}

The effect of redirection files can also be achieved by invoking
the \LaTeX{} compiler with a more elaborate command line.
Most conveniently this should be done as part
of a shell script or a |Makefile|.

When using \textsf{childdoc} in the main file, the following
command lines effectively perform a redirection
(note that depending on the shell being used,
backslashes may have to be doubled: `|\|' $\to$ `|\\|'):
%
\begin{center}
|... -jobname "|\textit{target}|" |\\|"|[\textit{flags}]%
|\input{childdoc.def}\childdocforward[|\textit{main}|]{|\textit{dest}|}"|
\end{center}
%
Here \textit{target} is the name of the output file,
\textit{main} is the name of the main file
and \textit{dest} is the name of the main or child file to be processed
(all filenames without extensions).
The optional argument \textit{main} can be omitted
if \textit{main} matches \textit{dest}.
Optionally, compilation \textit{flags} can be defined via |\def| commands.
This command line makes the \TeX{} engine believe
it is compiling the file \textit{target}
whose content is specified as the latter parameter.
The provided code then forwards the processing to
\textit{main} or \textit{dest} as described in \secref{sec:forward}.

%%%%%%%%%%%%%%%%%%%%%%%%%%%%%%%%%%%%%%%%%%%%%%%%%%%%%%%%%%%%%%%%%%%%%%%%%%%%%%%%
\subsection{Include by Input}
\label{sec:input}

Including child documents by |\include| has some restrictions by design.
Most notably, the content of a child document always occupies
its own set of pages; pages cannot be shared between child documents.
Usually, this behaviour makes perfect sense
because each child document contain an essential part of the document.
However, in some situations it may be desirable to compose
a document from a collection of parts
without having mandatory page breaks between then.
For this case, the package
provides a mechanism to include parts
by |\input| which can also be processed individually.
However, by construction this mechanism
requires manual handling of the content to be output.

%%%%%%%%%%%%%%%%%%%%%%%%%%%%%%%%%%%%%%%%
\DescribeMacro{\ifchilddocmanual}
The main file should be prepared as usual, see \secref{sec:include}.
However, the document body must make a distinction
between processing of an individual part and of the main document, e.g.:
%
\begin{center}
\begin{tabular}{l}
|\ifchilddocmanual|\\
|\input{\childdocname}|\\
|\||else|\\
\textit{document body with }|\input{|\textit{part}|}|\\
|\||fi|
\end{tabular}
\end{center}
%
The conditional |\ifchilddocmanual| is true whenever
a part to be included by |\input| is being compiled,
and the name of the part is stored in |\childdocname|.

%%%%%%%%%%%%%%%%%%%%%%%%%%%%%%%%%%%%%%%%
\DescribeMacro{\childdocby}
Each part to be included by |\input| should start with:
%
\begin{center}
\begin{tabular}{l}
|\input{childdoc.def}|\\
|\childdocby{|\textit{main}|}|\\
\end{tabular}
\end{center}
%
The directive |\childdocby| is similar to |\childdocof|
described in \secref{sec:include},
but the subsequent selection of content must be done manually.
To that end, both |\ifchilddoc| and |\ifchilddocmanual|
will be true upon processing of a part,
and the name of the part is stored in |\childdocname|.
Note that |\jobname| will be set to the filename of the current part
so that each part receives an individual |.aux| file
that does not interfere with the |.aux| file(s) of the main document.
This behaviour can be altered by the alternative form
|\childdocby[*]{|\textit{main}|}| (with a non-empty optional argument)
which uses the |.aux| file of the main document
by setting |\jobname| to \textit{main}.

%%%%%%%%%%%%%%%%%%%%%%%%%%%%%%%%%%%%%%%%%%%%%%%%%%%%%%%%%%%%%%%%%%%%%%%%%%%%%%%%
\subsection{Driver Development}
\label{sec:driver}

The \textsf{childdoc} mechanism can also be use for the development
of definition files such as \LaTeX{} styles or classes.
This case differs from the above setup with multiple parts
included by |\include| in that no |\includeonly| should be invoked.
This can be achieved by starting the include file
(before |\ProvidesPackage|) with:
%
\begin{center}
\begin{tabular}{l}
|\input{childdoc.def}|\\
|\childdocforward{|\textit{main}|}|\\
\end{tabular}
\end{center}
%
or alternatively with:
%
\begin{center}
\begin{tabular}{l}
|\input{childdoc.def}|\\
|\childdocby{|\textit{main}|}|\\
\end{tabular}
\end{center}
%
Both forms have slightly different effects as described above.
The main file is prepared as usual, see \secref{sec:include}.

%%%%%%%%%%%%%%%%%%%%%%%%%%%%%%%%%%%%%%%%%%%%%%%%%%%%%%%%%%%%%%%%%%%%%%%%%%%%%%%%
\subsection{Legacy Detection}
\label{sec:detection}

The directive |\childdocmain| in the main file can detect
whether the complete document or merely a child is to be compiled
even without using the directive |\childdocof|.
This method is deprecated because it is less robust
and there is no compelling reason to use it;
it is merely provided for backward compatibility
and it may be removed in future versions.

If the detection mechanism is to be used,
it is mandatory to correctly specify
the filename of the main file as the argument of |\childdocmain|:
%
\begin{center}
\begin{tabular}{l}
|\input{childdoc.def}|\\
|\childdocmain{|\textit{main}|}|\\
\end{tabular}
\end{center}
%
If |\jobname| does not match the argument \textit{main} of |\childdocmain|,
it is assumed that |\jobname| points to the child file to be compiled.
When using |\childdocmain| with the main file specified as argument,
it suffices to start a child file
with just |\input{|\textit{main}|}|
without loading of the package and using |\childdocof|.
If instead all processing is done
with the appropriate \textsf{childdoc} directives,
the argument of \textit{main} of |\childdocmain| can be empty.

An alternative version of the command line processing described
in \secref{sec:commandline} using the detection mechanism reads:
%
\begin{center}
|... -jobname "|\textit{target}|" "|[\textit{flags}]%
[|\def\jobname{|\textit{dest}|}|]|\input{|\textit{main}|}"|
\end{center}

%%%%%%%%%%%%%%%%%%%%%%%%%%%%%%%%%%%%%%%%%%%%%%%%%%%%%%%%%%%%%%%%%%%%%%%%%%%%%%%%
\subsection{Manual Code}
\label{sec:manual}

In case one cannot be certain whether the definitions file |childdoc.def|
is installed on the target \TeX{} distribution
and one prefers not to ship it,
it is conceivable to paste a few relevant commands into the sources.

To that end, drop all statements |\input{childdoc.def}|
and perform the replacements as outlined below.
Instead of |\childdocmain{|\textit{main}|}| add the following code
to the top of the main file:
%
\begin{center}
\begin{tabular}{l}
|\||ifdefined\childdocname\endinput\||fi\newif\ifchilddoc|\\
|\edef\childdocname{\scantokens\expandafter{\jobname\noexpand}}|\\
|\def\childdocmain{|\textit{main}|}\||ifx\childdocmain\childdocname\||else|\\
|\childdoctrue\includeonly{\childdocname}\let\jobname\childdocmain\||fi|\\
\end{tabular}
\end{center}
%
Instead of |\childdocof{|\textit{main}|}| just include the main file
at the top of each child file:
%
\begin{center}
|\input{|\textit{main}|}|
\end{center}
%
A simple redirection |\childdocforward{|\textit{dest}|}| is achieved by:
%
\begin{center}
|\def\jobname{|\textit{dest}|}\input{\jobname}|
\end{center}
%
The redirection with prefix
|\childdocforwardprefix[|\textit{prefix}|]{|\textit{dest}|}|
is accomplished by:
%
\begin{center}
\begin{tabular}{l}
|{\edef\jobname{\scantokens\expandafter{\jobname\noexpand}}|\\
|\def\redirectjob |\textit{prefix}|#1~~~{\gdef\jobname{|\textit{dest}|#1}}|\\
|\expandafter\redirectjob\jobname~~~}\input{\jobname}|
\end{tabular}
\end{center}

In an alternative approach,
child documents can be compiled by a specific command line
without additional code or specific definitions:
%
\begin{center}
|... -jobname "|\textit{target}|" "|[\textit{flags}]%
|\includeonly{|\textit{dest}|}\input{|\textit{main}|}"|
\end{center}
%

%%%%%%%%%%%%%%%%%%%%%%%%%%%%%%%%%%%%%%%%%%%%%%%%%%%%%%%%%%%%%%%%%%%%%%%%%%%%%%%%
%%%%%%%%%%%%%%%%%%%%%%%%%%%%%%%%%%%%%%%%%%%%%%%%%%%%%%%%%%%%%%%%%%%%%%%%%%%%%%%%
\section{Information}

%%%%%%%%%%%%%%%%%%%%%%%%%%%%%%%%%%%%%%%%%%%%%%%%%%%%%%%%%%%%%%%%%%%%%%%%%%%%%%%%
\subsection{Copyright}

Copyright \copyright{} 2017--2018 Niklas Beisert

This work may be distributed and/or modified under the
conditions of the \LaTeX{} Project Public License, either version 1.3
of this license or (at your option) any later version.
The latest version of this license is in
  \url{http://www.latex-project.org/lppl.txt}
and version 1.3 or later is part of all distributions of \LaTeX{}
version 2005/12/01 or later.

This work has the LPPL maintenance status `maintained'.

The Current Maintainer of this work is Niklas Beisert.

This work consists of the files |README.txt|, |childdoc.ins| and |childdoc.dtx|
as well as the derived files |childdoc.def|, |cdocsamp.tex|
with |cdocsch1.tex|, |cdocsch2.tex|, |cdocspt3.tex|, |cdocspt4.tex|,
|cdocsdrf.tex|, |cdocsfn1.tex|, |cdocsfn2.tex|
as well as |childdoc.pdf|.

%%%%%%%%%%%%%%%%%%%%%%%%%%%%%%%%%%%%%%%%%%%%%%%%%%%%%%%%%%%%%%%%%%%%%%%%%%%%%%%%
\subsection{Files and Installation}

The package consists of the files:
%
\begin{center}
\begin{tabular}{ll}
    |README.txt|   & readme file \\
    |childdoc.ins| & installation file \\
    |childdoc.dtx| & source file \\
    |childdoc.def| & definition file \\
    |cdocsamp.tex| & sample main file \\
    |cdocsch1.tex| & sample include file \\
    |cdocsch2.tex| & sample include file \\
    |cdocspt3.tex| & sample part file \\
    |cdocspt4.tex| & sample part file \\
    |cdocsdrf.tex| & sample redirection file \\
    |cdocsfn1.tex| & sample redirection file \\
    |cdocsfn2.tex| & sample redirection file \\
    |childdoc.pdf| & manual
\end{tabular}
\end{center}
%
The distribution consists of the files
|README.txt|, |childdoc.ins| and |childdoc.dtx|.
%
\begin{itemize}
\item
Run (pdf)\LaTeX{} on |childdoc.dtx|
to compile the manual |childdoc.pdf| (this file).
\item
Run \LaTeX{} on |childdoc.ins| to create the definitions file |childdoc.def|
and the sample |cdocsamp.tex| with include files
|cdocsch1.tex|, |cdocsch2.tex|, |cdocspt3.tex|, |cdocspt4.tex|,
|cdocsdrf.tex|, |cdocsfn1.tex|, |cdocsfn2.tex|.
Then copy the file |childdoc.def| to an appropriate directory of your \LaTeX{}
distribution, e.g.\ \textit{texmf-root}|/tex/latex/childdoc|.
\end{itemize}

%%%%%%%%%%%%%%%%%%%%%%%%%%%%%%%%%%%%%%%%%%%%%%%%%%%%%%%%%%%%%%%%%%%%%%%%%%%%%%%%
\subsection{Related CTAN Packages}

There are several other packages which offer a similar functionality:
%
\begin{itemize}
\item
The packages
\href{http://ctan.org/pkg/docmute}{\textsf{docmute}},
\href{http://ctan.org/pkg/includex}{\textsf{includex}} and
\href{http://ctan.org/pkg/standalone}{\textsf{standalone}}
provide commands to include only the document body of
a child file thus allowing both files to be compiled individually.
\item
The packages \href{http://ctan.org/pkg/subdocs}{\textsf{subdocs}}
and \href{http://ctan.org/pkg/subfiles}{\textsf{subfiles}}
provide structures in which the main and child documents can be
encapsulated and allowing them to be compiled individually.
The inclusion mechanism is different from the conventional |\include|.
\item
The package \href{http://ctan.org/pkg/combine}{\textsf{combine}}
is an elaborate solution to combine several documents into one.
\end{itemize}
%
See also the CTAN topic \href{http://ctan.org/topic/subdocs}{\textsf{subdocs}}
for further related packages.
The present package differs from the above solutions in that
a document structure constructed with the conventional |\include| mechanism
just needs two extra commands at the top of every file
such that all constituent files can be compiled individually.

%%%%%%%%%%%%%%%%%%%%%%%%%%%%%%%%%%%%%%%%%%%%%%%%%%%%%%%%%%%%%%%%%%%%%%%%%%%%%%%%
%\subsection{Feature Suggestions}
%
%The following is a list of features which may be useful for future
%versions of this package:
%%
%\begin{itemize}
%\item
%\ldots
%\end{itemize}

%%%%%%%%%%%%%%%%%%%%%%%%%%%%%%%%%%%%%%%%%%%%%%%%%%%%%%%%%%%%%%%%%%%%%%%%%%%%%%%%
\subsection{Revision History}

%%%%%%%%%%%%%%%%%%%%%%%%%%%%%%%%%%%%%%%%
\paragraph{v2.0:} 2018/12/30

\begin{itemize}
\item
immediate forward processing
\item
added |\childdocby| mechanism
\item
manual restructured
\end{itemize}

%%%%%%%%%%%%%%%%%%%%%%%%%%%%%%%%%%%%%%%%
\paragraph{v1.6:} 2018/01/17

\begin{itemize}
\item
application for development of include files
\item
corrections to manual
\end{itemize}

%%%%%%%%%%%%%%%%%%%%%%%%%%%%%%%%%%%%%%%%
\paragraph{v1.5:} 2017/05/21

\begin{itemize}
\item
more complete structuring introduced
\item
|\childdocof| introduced
\item
|\childdoc| renamed to |\childdocmain|
\item
|\childredirect| renamed to |\childdocforward| and |\childdocforwardprefix|
and functionality expanded
\end{itemize}

%%%%%%%%%%%%%%%%%%%%%%%%%%%%%%%%%%%%%%%%
\paragraph{v1.0:} 2017/04/27

\begin{itemize}
\item
manual and install package
\item
first version published on CTAN
\end{itemize}

%%%%%%%%%%%%%%%%%%%%%%%%%%%%%%%%%%%%%%%%
\paragraph{v0.6:} 2017/04/26

\begin{itemize}
\item
redirection mechanism added
\end{itemize}

%%%%%%%%%%%%%%%%%%%%%%%%%%%%%%%%%%%%%%%%
\paragraph{v0.5:} 2017/04/26

\begin{itemize}
\item
functionality in definition file
\end{itemize}


%%%%%%%%%%%%%%%%%%%%%%%%%%%%%%%%%%%%%%%%%%%%%%%%%%%%%%%%%%%%%%%%%%%%%%%%%%%%%%%%
%%%%%%%%%%%%%%%%%%%%%%%%%%%%%%%%%%%%%%%%%%%%%%%%%%%%%%%%%%%%%%%%%%%%%%%%%%%%%%%%
%%%%%%%%%%%%%%%%%%%%%%%%%%%%%%%%%%%%%%%%%%%%%%%%%%%%%%%%%%%%%%%%%%%%%%%%%%%%%%%%
\appendix

\settowidth\MacroIndent{\rmfamily\scriptsize 000\ }

 \DocInput{childdoc.dtx}

\end{document}
%</driver>
% \fi
%
% %%%%%%%%%%%%%%%%%%%%%%%%%%%%%%%%%%%%%%%%%%%%%%%%%%%%%%%%%%%%%%%%%%%%%%%%%%%%%%
% %%%%%%%%%%%%%%%%%%%%%%%%%%%%%%%%%%%%%%%%%%%%%%%%%%%%%%%%%%%%%%%%%%%%%%%%%%%%%%
% \section{Sample}
%\iffalse
%<*samplemain>
%\fi
%
% The following presents a sample document
% with two chapters, two parts, a title page,
% a compile flag as well as three forwarding files to set the flag.
% It consists of eight |.tex| files:
% \begin{center}
% \begin{tabular}{ll}
% |cdocsamp.tex|&main file\\
% |cdocsch1.tex|&include file for chapter 1\\
% |cdocsch2.tex|&include file for chapter 2\\
% |cdocspt3.tex|&include file for part 3\\
% |cdocspt4.tex|&include file for part 4\\
% |cdocsdrf.tex|&forwarding file for main file in draft mode\\
% |cdocsfi1.tex|&forwarding file for final version of chapter 1\\
% |cdocsfi2.tex|&forwarding file for final version of chapter 2\\
% \end{tabular}
% \end{center}
% Each of the eight files can be compiled directly by the \LaTeX{} compiler.
%
% %%%%%%%%%%%%%%%%%%%%%%%%%%%%%%%%%%%%%%
% \paragraph{Main File.}
%
% The main file is called |cdocsamp.tex|.
%
% Load the \textsf{childdoc} definitions and
% declare the filename for the main document:
%    \begin{macrocode}
\input{childdoc.def}
\childdocmain{}
%    \end{macrocode}

% Optional override for |\version| flag:
%    \begin{macrocode}
%%\ifchilddoc\else\providecommand{\version}{draft}\fi
%    \end{macrocode}

% Define the default values for the |\version| flag
% (|final| for the main file and |draft| for childs):
%    \begin{macrocode}
\ifchilddoc
\providecommand{\version}{draft}
\else
\providecommand{\version}{final}
\fi
%    \end{macrocode}

% Load the standard document class:
%    \begin{macrocode}
\documentclass[12pt]{article}
%    \end{macrocode}

% Start the document body:
%    \begin{macrocode}
\begin{document}
%    \end{macrocode}

% Declare a title page.
% Print title, part of document being processed and version flag:
%    \begin{macrocode}
\addtocounter{page}{-1}
\begin{center}
{\LARGE\bfseries{}childdoc example\par}
\vspace{1cm}
\ifchilddoc
\ifchilddocmanual part\else chapter\fi:
`\childdocname' of `\childdocjob'\par
\else
main document: `\childdocjob'\par
\fi
version: \version\par
\end{center}
\newpage
%    \end{macrocode}

% Manually include selected file,
% otherwise process as usual:
%    \begin{macrocode}
\ifchilddocmanual
\section*{part `\childdocname'}
\input{\childdocname}
\else
%    \end{macrocode}

% Include the two chapters:
%    \begin{macrocode}
\include{cdocsch1}
\include{cdocsch2}
%    \end{macrocode}

% Include the two parts unless only chapters should be displayed:
%    \begin{macrocode}
\ifchilddoc\else
\section{part three}
\input{cdocspt3}
\section{part four}
\input{cdocspt4}
\fi
%    \end{macrocode}

% Process as usual until here:
%    \begin{macrocode}
\fi
%    \end{macrocode}

% End of document body:
%    \begin{macrocode}
\end{document}
%    \end{macrocode}
%\iffalse
%</samplemain>
%\fi
%
% %%%%%%%%%%%%%%%%%%%%%%%%%%%%%%%%%%%%%%
% \paragraph{Chapter Include Files.}
%
% The include files are called |cdocsch1.tex| and |cdocsch2.tex|.
%
%\iffalse
%<*samplechap1|samplechap2>
%\fi

% Optional override for |\version| flag:
%    \begin{macrocode}
%%\providecommand{\version}{final}
%    \end{macrocode}

% Include the main document:
%    \begin{macrocode}
\input{childdoc.def}
\childdocof{cdocsamp}
%    \end{macrocode}

%\iffalse
%</samplechap1|samplechap2>
%\fi
%
%\iffalse
%<*samplechap1>
%\fi
% Some text for chapter 1:
%    \begin{macrocode}
\section{one}
some text in chapter one
%    \end{macrocode}

%\iffalse
%</samplechap1>
%\fi
% Some text for chapter 2:
%\iffalse
%<*samplechap2>
%\fi
%    \begin{macrocode}
\section{two}
more text in chapter two
%    \end{macrocode}

%\iffalse
%</samplechap2>
%\fi
%
% %%%%%%%%%%%%%%%%%%%%%%%%%%%%%%%%%%%%%%
% \paragraph{Part Include Files.}
%
% The include files are called |cdocspt3.tex| and |cdocspt4.tex|.
%
%\iffalse
%<*samplepart3|samplepart4>
%\fi

% Optional override for |\version| flag:
%    \begin{macrocode}
%%\providecommand{\version}{final}
%    \end{macrocode}

% Include the main document:
%    \begin{macrocode}
\input{childdoc.def}
\childdocby{cdocsamp}
%    \end{macrocode}

%\iffalse
%</samplepart3|samplepart4>
%\fi
%
%\iffalse
%<*samplepart3>
%\fi
% Some text for part 3:
%    \begin{macrocode}
some text in part three
%    \end{macrocode}

%\iffalse
%</samplepart3>
%\fi
% Some text for part 4:
%\iffalse
%<*samplepart4>
%\fi
%    \begin{macrocode}
more text in part four
%    \end{macrocode}

%\iffalse
%</samplepart4>
%\fi
%
% %%%%%%%%%%%%%%%%%%%%%%%%%%%%%%%%%%%%%%
% \paragraph{Forwarding for a Complete Draft.}
%
% The following forwarding file |cdocsdrf.tex|
% compiles the main document in draft mode:
%\iffalse
%<*sampledraft>
%\fi
%    \begin{macrocode}
\def\version{draft}
\input{childdoc.def}
\childdocforward{cdocsamp}
%    \end{macrocode}

%\iffalse
%</sampledraft>
%\fi
%
% %%%%%%%%%%%%%%%%%%%%%%%%%%%%%%%%%%%%%%
% \paragraph{Forwarding for Final Version of the Chapters.}
%
% The following forwarding files |cdocsfn1.tex| and |cdocsfn2.tex|
% (with identical content)
% compile the final versions of the child documents
% |cdocsch1.tex| and |cdocsch2.tex|, respectively:
%\iffalse
%<*samplefinal>
%\fi
%    \begin{macrocode}
\def\version{final}
\input{childdoc.def}
\childdocforwardprefix[cdocsamp]{cdocsfn}{cdocsch}
%    \end{macrocode}

%\iffalse
%</samplefinal>
%\fi
%
% %%%%%%%%%%%%%%%%%%%%%%%%%%%%%%%%%%%%%%
% \paragraph{Command Line Processing.}
%
% The following three command lines generate the output files
% |cdocscld|, |cdocscl1| and |cdocscl2|
% which should be identical to
% |cdocsdrf|, |cdocsch1| and |cdocsfn2|, respectively:
% \begin{center}
% \begin{tabular}{l}
% |latex -jobname cdocscld \|\\
% |  "\def\version{draft}\input{childdoc.def}\childdocforward{cdocsamp}"|\\
% |latex -jobname cdocscl1 \|\\
% |  "\input{childdoc.def}\childdocforward[cdocsamp]{cdocsch1}"|\\
% |latex -jobname cdocscl2 \|\\
% |  "\def\version{final}\input{childdoc.def}\childdocforward{cdocsch2}"|
% \end{tabular}
% \end{center}
% Note that the trailing backslash on each first line
% merely continues the input to the second line
% (for convenient cut ant paste).
% Furthermore, the command |latex| can be replaced by any
% of its alternative versions such as |pdflatex|.
%
% %%%%%%%%%%%%%%%%%%%%%%%%%%%%%%%%%%%%%%%%%%%%%%%%%%%%%%%%%%%%%%%%%%%%%%%%%%%%%%
% %%%%%%%%%%%%%%%%%%%%%%%%%%%%%%%%%%%%%%%%%%%%%%%%%%%%%%%%%%%%%%%%%%%%%%%%%%%%%%
% \section{Implementation}
%\iffalse
%<*package>
%\fi
%
% This section describes the definitions file |childdoc.def|.

% The definitions cannot be loaded using |\usepackage| or |\RequirePackage|
% which has a mechanism to prevent loading a style file more than once.
% When loading the definitions by means of |\input|
% multiple instances have to be prevented manually:
%\iffalse
%This code needs to be before the `\ProvidesFile' directive
%which is defined at the beginning of this file.
%Therefore it is also placed there and commented out here.
%</package>
%<*discard>
%\fi
%    \begin{macrocode}
\ifdefined\childdocmain\endinput\fi
%    \end{macrocode}
%\iffalse
%</discard>
%<*package>
%\fi
%
% \macro{\ifchilddoc}
% \macro{\ifchilddocmanual}
% The conditional |\ifchilddoc| tells whether a
% child (true) or main (false) document is being compiled.
% The conditional |\ifchilddocmanual| tells whether
% the |\includeonly| mechanism is used (false) or
% the selection of child files must be performed manually (true).
% The definitions initialise to false:
%    \begin{macrocode}
\newif\ifchilddoc
\newif\ifchilddocmanual
%    \end{macrocode}

% \macro{\childdocname}
% \macro{\childdocjob}
% The macro |\childdocname| stores the name of the main document
% to be compiled. The macro |\childdocjob| stores the name of
% the document on which the \LaTeX{} compiler was originally invoked.
% The content of |\jobname| cannot be compared
% to filenames specified in the source due to different catcodes.
% The following code rescans |\jobname|, stores the result
% in |\childdocname| and saves a copy in |\childdocjob|:
%    \begin{macrocode}
\edef\childdocname{\scantokens\expandafter{\jobname\noexpand}}
\let\childdocjob\childdocname
%    \end{macrocode}

% \macro{\childdocdisable}
% The macro |\childdocdisable| prevents the main file
% from being processed more than once.
% At this stage, the main document command |\childdocmain|
% is assumed to be called once again where it should do nothing.
% Any subsequent call to it should prevent
% a secondary processing of the main document
% It overwrites the forwarding commands
% |\childdocof| and |\childdocforward|
% with empty macros to prevent further inclusions of the main document:
%    \begin{macrocode}
\newcommand{\childdocdisable}
{
  \renewcommand{\childdocmain}[1]{\renewcommand{\childdocmain}[1]{\endinput}}
  \renewcommand{\childdocof}[1]{}
  \renewcommand{\childdocby}[2][]{}
  \renewcommand{\childdocforward}[2][]{}
  \renewcommand{\childdocdisable}{}
}
%    \end{macrocode}

% \macro{\childdocmain}
% The macro |\childdocmain| is to be called at the top of the main file
% with nothing or the main filename (without extension) as argument.
% First, it breaks loops.
% If the argument is not empty and does not match |\childdocname|
% (which is set by the first inclusion of |childdoc.def|),
% |\ifchilddoc| is set to true, |\includeonly| is applied to the child file
% and |\jobname| is set to the main file
% (for proper handling of |.aux| files):
%    \begin{macrocode}
\newcommand{\childdocmain}[1]
{
  \childdocdisable\childdocmain{}
  \if?#1?\else
    \begingroup
      \def\childdoctmp{#1}
      \ifx\childdoctmp\childdocname
        \def\childdoctmp{}
      \else
        \def\childdoctmp
        {
          \childdoctrue
          \includeonly{\childdocname}
          \def\childdocjob{#1}
          \def\jobname{#1}
        }
      \fi
      \expandafter
    \endgroup
    \childdoctmp
  \fi
}
%    \end{macrocode}

% \macro{\childdocof}
% The command |\childdocof| redirects
% compilation to the main file |#1|.
%    \begin{macrocode}
\newcommand{\childdocof}[1]
{
  \childdocdisable
  \childdoctrue
  \includeonly{\childdocname}
  \def\jobname{#1}
  \def\childdocjob{#1}
  \input{#1}
}
%    \end{macrocode}

% \macro{\childdocby}
% The command |\childdocby| ....
%    \begin{macrocode}
\newcommand{\childdocby}[2][]
{
  \childdocdisable
  \childdoctrue
  \childdocmanualtrue
  \if?#1?\else
    \def\jobname{#2}
  \fi
  \def\childdocjob{#2}
  \input{#2}
  \endinput
}
%    \end{macrocode}

% \macro{\childdocforward}
% The command |\childdocforward| redirects
% compilation to the main file or
% (if the optional argument is given) a child file.
% Parameters are set as if the main file
% or a child file starting with |\childdocof| was compiled.
% Then compilation is handed over to the main file:
%    \begin{macrocode}
\newcommand{\childdocforward}[2][]
{
  \begingroup
    \if?#1?
      \def\childdoctmp
      {
        \def\childdocname{#2}
        \def\childdocjob{#2}
        \def\jobname{#2}
        \input{#2}
        \endinput
      }
    \else
      \def\childdoctmp
      {
        \childdocdisable
        \def\childdocname{#2}
        \childdoctrue
        \includeonly{#2}
        \def\childdocjob{#1}
        \def\jobname{#1}
        \input{#1}
        \endinput
      }
    \fi
    \expandafter
  \endgroup
  \childdoctmp
}
%    \end{macrocode}

% \macro{\childdocforwardprefix}
% The command |\childdocforwardprefix| redirects
% compilation to the main or a child file by means of a pattern.
% The prefix |#1| in the current filename is replaced by |#2|
% and the suffix of the current filename is kept
% (it is assumed that the filename does not contain the substring `|~~~|'
% which is used as a delimiter).
% Compilation is handed over to the new file by |\childdocforward|:
%    \begin{macrocode}
\newcommand{\childdocforwardprefix}[3][]
{
  \begingroup
    \def\childdocextract #2##1~~~{\def\childdoctmp{\childdocforward[#1]{#3##1}}}
    \expandafter\childdocextract\childdocname~~~
    \expandafter
  \endgroup
  \childdoctmp
}
%    \end{macrocode}

% \macro{\childdoc}
% The deprecated macro |\childdoc| is a legacy version of |\childdocmain|:
%    \begin{macrocode}
\newcommand{\childdoc}{\childdocmain}
%    \end{macrocode}

% \macro{\childdocredirect}
% The deprecated macro |\childdocredirect| is a legacy version
% of |\childdocforward| and |\childdocforwardprefix|:
%    \begin{macrocode}
\newcommand{\childdocredirect}[2][]
{
  \begingroup
    \if?#1?
      \def\childdoctmp{\childdocforward{#2}}
    \else
      \def\childdoctmp{\childdocforwardprefix{#1}{#2}}
    \fi
    \expandafter
  \endgroup
  \childdoctmp
}
%    \end{macrocode}

%\iffalse
%</package>
%\fi
%
\endinput
\childdocforward[|\textit{main}|]{|\textit{dest}|}"|
\end{center}
%
Here \textit{target} is the name of the output file,
\textit{main} is the name of the main file
and \textit{dest} is the name of the main or child file to be processed
(all filenames without extensions).
The optional argument \textit{main} can be omitted
if \textit{main} matches \textit{dest}.
Optionally, compilation \textit{flags} can be defined via |\def| commands.
This command line makes the \TeX{} engine believe
it is compiling the file \textit{target}
whose content is specified as the latter parameter.
The provided code then forwards the processing to
\textit{main} or \textit{dest} as described in \secref{sec:forward}.

%%%%%%%%%%%%%%%%%%%%%%%%%%%%%%%%%%%%%%%%%%%%%%%%%%%%%%%%%%%%%%%%%%%%%%%%%%%%%%%%
\subsection{Include by Input}
\label{sec:input}

Including child documents by |\include| has some restrictions by design.
Most notably, the content of a child document always occupies
its own set of pages; pages cannot be shared between child documents.
Usually, this behaviour makes perfect sense
because each child document contain an essential part of the document.
However, in some situations it may be desirable to compose
a document from a collection of parts
without having mandatory page breaks between then.
For this case, the package
provides a mechanism to include parts
by |\input| which can also be processed individually.
However, by construction this mechanism
requires manual handling of the content to be output.

%%%%%%%%%%%%%%%%%%%%%%%%%%%%%%%%%%%%%%%%
\DescribeMacro{\ifchilddocmanual}
The main file should be prepared as usual, see \secref{sec:include}.
However, the document body must make a distinction
between processing of an individual part and of the main document, e.g.:
%
\begin{center}
\begin{tabular}{l}
|\ifchilddocmanual|\\
|\input{\childdocname}|\\
|\||else|\\
\textit{document body with }|\input{|\textit{part}|}|\\
|\||fi|
\end{tabular}
\end{center}
%
The conditional |\ifchilddocmanual| is true whenever
a part to be included by |\input| is being compiled,
and the name of the part is stored in |\childdocname|.

%%%%%%%%%%%%%%%%%%%%%%%%%%%%%%%%%%%%%%%%
\DescribeMacro{\childdocby}
Each part to be included by |\input| should start with:
%
\begin{center}
\begin{tabular}{l}
|% \iffalse
%
% childdoc.dtx Copyright (C) 2017-2018 Niklas Beisert
%
% This work may be distributed and/or modified under the
% conditions of the LaTeX Project Public License, either version 1.3
% of this license or (at your option) any later version.
% The latest version of this license is in
%   http://www.latex-project.org/lppl.txt
% and version 1.3 or later is part of all distributions of LaTeX
% version 2005/12/01 or later.
%
% This work has the LPPL maintenance status `maintained'.
%
% The Current Maintainer of this work is Niklas Beisert.
%
% This work consists of the files childdoc.dtx and childdoc.ins
% and the derived files childdoc.def and cdocsamp.tex with
% cdocsch1.tex, cdocsch2.tex, cdocsdrf.tex, cdocsfn1.tex, cdocsfn2.tex.
%
%<package>\ifdefined\childdocmain\endinput\fi
%<package>\ProvidesFile{childdoc.def}[2018/12/30 v2.0 child document driver]
%<samplemain>\ProvidesFile{cdocsamp.tex}[2018/12/30 v2.0 sample for childdoc]
%<*driver>
%\ProvidesFile{childdoc.drv}[2018/12/30 v2.0 childdoc reference manual file]
\PassOptionsToClass{10pt,a4paper}{article}
\documentclass{ltxdoc}

\usepackage[margin=35mm]{geometry}
\usepackage{hyperref}
\usepackage{hyperxmp}
\usepackage[usenames]{color}

\hypersetup{colorlinks=true}
\hypersetup{pdfstartview=FitH}
\hypersetup{pdfpagemode=UseNone}
\hypersetup{pdfsource={}}
\hypersetup{pdflang={en-UK}}
\hypersetup{pdfcopyright={Copyright 2017-2018 Niklas Beisert.
  This work may be distributed and/or modified under the
  conditions of the LaTeX Project Public License, either version 1.3
  of this license or (at your option) any later version.}}
\hypersetup{pdflicenseurl={http://www.latex-project.org/lppl.txt}}
\hypersetup{pdfcontactaddress={ETH Zurich, ITP, HIT K,
  Wolfgang-Pauli-Strasse 27}}
\hypersetup{pdfcontactpostcode={8093}}
\hypersetup{pdfcontactcity={Zurich}}
\hypersetup{pdfcontactcountry={Switzerland}}
\hypersetup{pdfcontactemail={nbeisert@itp.phys.ethz.ch}}
\hypersetup{pdfcontacturl={http://people.phys.ethz.ch/\xmptilde nbeisert/}}

\newcommand{\secref}[1]{\hyperref[#1]{section \ref*{#1}}}

\parskip1ex
\parindent0pt
\let\olditemize\itemize
\def\itemize{\olditemize\parskip0pt}

\begin{document}

\title{The \textsf{childdoc} Package}
\hypersetup{pdftitle={The childdoc Package}}
\author{Niklas Beisert\\[2ex]
  Institut f\"ur Theoretische Physik\\
  Eidgen\"ossische Technische Hochschule Z\"urich\\
  Wolfgang-Pauli-Strasse 27, 8093 Z\"urich, Switzerland\\[1ex]
  \href{mailto:nbeisert@itp.phys.ethz.ch}
  {\texttt{nbeisert@itp.phys.ethz.ch}}}
\hypersetup{pdfauthor={Niklas Beisert}}
\hypersetup{pdfsubject={Manual for the LaTeX2e Package childdoc}}
\date{30 December 2018, \textsf{v2.0}}
\maketitle

\begin{abstract}\noindent
\textsf{childdoc} is a \LaTeXe{} package
that enables the direct compilation
of document sections included by |\include|
to individual files.
\end{abstract}

\begingroup
\parskip0ex
\tableofcontents
\endgroup

%%%%%%%%%%%%%%%%%%%%%%%%%%%%%%%%%%%%%%%%%%%%%%%%%%%%%%%%%%%%%%%%%%%%%%%%%%%%%%%%
%%%%%%%%%%%%%%%%%%%%%%%%%%%%%%%%%%%%%%%%%%%%%%%%%%%%%%%%%%%%%%%%%%%%%%%%%%%%%%%%
\section{Introduction}

\LaTeX{} provides a mechanism to structure a large document (such as a book)
into a main file and several child files (containing the chapters)
using the |\include| command.
This mechanism is beneficial for documents
which span hundreds of pages in order to
make the source file(s) more manageable.
Moreover, compilation can be restricted to
selected child files by means of the |\includeonly| command.
The latter feature can be used to reduce the compilation time while editing
(this was significantly more useful in the earlier days of \LaTeX{})
or to generate a smaller document which is easier to navigate.
Another application of |\includeonly| is to generate
documents consisting of selected parts of the complete document.

However, there are a few drawbacks of the plain |\include| mechanism:
\begin{itemize}
\item
The child files cannot be compiled on their own,
they can only be compiled via the main file.
A naive editing environment
(such as a text editor with an option
to have the current file processed by \LaTeX)
may require one to switch to the main file before compiling;
attempting to compile the child file produces errors.
\item
The main file must be modified (each time)
to adjust the |\includeonly| command
to the present needs. This easily leaves the main file in a messy state.
\item
The generated document will always carry the filename
of the main document. This is inconvenient if
several child files are to be compiled and
to be kept for distribution.
\end{itemize}

The present package provides a simple interface
to make child files individually compilable by \LaTeX{}.
Compiling a child file then has the same effect as compiling
the main file with an |\includeonly| command
to select the appropriate child.
Moreover the generated document will carry the name of the child
rather than the main file.
This resolves all three above issues.

This feature is meant to make the editing of books,
thesis documents and lecture notes somewhat more convenient.
However, the package can also be used efficiently for
composing a series of documents (such as exercise sheets)
which are typically distributed individually.
It then assists the author in generating the individual documents
(potentially in different versions)
as well as a document containing the collected series.
Another application is in developing style files
or other kinds of included material
where compilation of the style file could redirect
to a sample or test file.

%%%%%%%%%%%%%%%%%%%%%%%%%%%%%%%%%%%%%%%%%%%%%%%%%%%%%%%%%%%%%%%%%%%%%%%%%%%%%%%%
%%%%%%%%%%%%%%%%%%%%%%%%%%%%%%%%%%%%%%%%%%%%%%%%%%%%%%%%%%%%%%%%%%%%%%%%%%%%%%%%
\section{Usage}

First of all, the package \textsf{childdoc} is \emph{not} a standard
\LaTeXe{} |.sty| style file! Therefore it needs to be invoked in
a non-standard way.

%%%%%%%%%%%%%%%%%%%%%%%%%%%%%%%%%%%%%%%%%%%%%%%%%%%%%%%%%%%%%%%%%%%%%%%%%%%%%%%%
\subsection{Included Files}
\label{sec:include}

%%%%%%%%%%%%%%%%%%%%%%%%%%%%%%%%%%%%%%%%
\DescribeMacro{\childdocmain}
To use the package, add the commands
\begin{center}
\begin{tabular}{l}
|\input{childdoc.def}|\\
|\childdocmain{}|\\
\end{tabular}
\end{center}
at the very top of the main \LaTeX{} file,
in particular \emph{before} the |\documentclass| statement!
The argument of |\childdocmain| should be left empty
(but it must be present).

%%%%%%%%%%%%%%%%%%%%%%%%%%%%%%%%%%%%%%%%
\DescribeMacro{\childdocof}
Furthermore, add the commands
\begin{center}
\begin{tabular}{l}
|\input{childdoc.def}|\\
|\childdocof{|\textit{main}|}|\\
\end{tabular}
\end{center}
at the top of every child file \textit{child}
which is included by |\include{|\textit{child}|}|
from within the main file
(or at least for those files to be compiled individually).
The argument \textit{main} must be the filename of the main file.

There are a couple of
considerations in setting up the main and child documents:

%%%%%%%%%%%%%%%%%%%%%%%%%%%%%%%%%%%%%%%%
\paragraph{Restrictions.}

Please note the following restrictions:
\begin{itemize}
\item
|\childdocmain| must be called with one argument \textit{main}
to ensure compatibility with earlier version of the package.
It must either be empty (|\childdocmain{}|)
or precisely match the filename of the main file in which it is specified.
See \secref{sec:detection} for further information.
\item
The filename \textit{main} must be specified without the |.tex| extension.
\item
The filename \textit{main} is case sensitive
(even in case-insensitive file systems)
due to internal string comparison.
\item
The argument \textit{main} should be fully expanded, it cannot be a macro.
\item
Subdirectories and special characters should be avoided in filenames.
\item
The command |\childdocmain{|\textit{main}|}| must be followed by a whitespace.
It should not be followed immediately by another command
or by a comment mark `|%|'.
This is because the \TeX{} parser reads the token immediately following
the argument of |\childdocmain| and puts it
at the beginning of every child section;
however, a white\-space is ignored.
\end{itemize}

%%%%%%%%%%%%%%%%%%%%%%%%%%%%%%%%%%%%%%%%
\paragraph{Content of Main File.}

It is advisable to place all content in the child files included by |\include|.
Any output contained in the main file will appear in all child documents
unless suppressed manually;
it cannot be suppressed automatically by the |\includeonly| directive
and thus should normally be avoided.
A method to include some content in the main file
by means of conditional processing is described in \secref{sec:conditional}.

%%%%%%%%%%%%%%%%%%%%%%%%%%%%%%%%%%%%%%%%
\paragraph{Page Numbering.}

When only a part of the document is compiled,
the appropriate numbering of pages
(as well as other status parameters)
is determined from the |.aux| files.
The latter contain information from previous passes.
However this information needs to propagate through
all intermediate child documents.
Therefore the page numbering in child documents may well
be inconsistent until the complete document is compiled at least once.

A useful (if unconventional) way to always ensure a consistent
page numbering is to restart the numbering in each child document
and denote the pages by `\textit{child}|.|\textit{page}'
where \textit{child} represents the chapter/section number of the child file.
This can be achieved by the command
|\numberwithin{page}{|\textit{child}|}|
of the \textsf{amsmath} package
where \textit{child} can be |chapter| or |section|
depending on the chosen structuring.
Alternatively, one can modify the macro |\thepage| appropriately
and reset the counter |page| at the start of each child file.

%%%%%%%%%%%%%%%%%%%%%%%%%%%%%%%%%%%%%%%%%%%%%%%%%%%%%%%%%%%%%%%%%%%%%%%%%%%%%%%%
\subsection{Conditional Processing}
\label{sec:conditional}

The package provides a mechanism to compile different versions
of a document. To customise the versions further some conditional processing
can come in handy to distinguish which version is being compiled.
The package provides two macros to describe the compilation context:

%%%%%%%%%%%%%%%%%%%%%%%%%%%%%%%%%%%%%%%%
\DescribeMacro{\ifchilddoc}
The conditional |\ifchilddoc| distinguishes between the compilation of
child documents and the main document:
%
\begin{center}
|\ifchilddoc |\textit{child-code}| |[|\||else |\textit{main-code}]| \||fi|
\end{center}

%%%%%%%%%%%%%%%%%%%%%%%%%%%%%%%%%%%%%%%%
\DescribeMacro{\childdocname}
\DescribeMacro{\childdocjob}
The macro |\childdocname| contains the filename (without extension)
of the main or child file being processed.
Note that |\childdocjob| will always contain the name of the main file.

%%%%%%%%%%%%%%%%%%%%%%%%%%%%%%%%%%%%%%%%
\paragraph{Title Page.}

Conditional processing can be used to include a title or banner page
in the main document when proper precautions are taken.
Importantly, the code in the main file should ensure that the page counter
(as well as other status parameters which are stored in the |.aux| files)
takes the same value after the conditional processing.
Otherwise the page numbers may take divergent values
depending on which part is compiled.

For example, a title page could be declared by:
%
\begin{center}
\begin{tabular}{l}
|\ifchilddoc\||else|\\
|\addtocounter{page}{-1}|\\
\textit{code for title page}\\
|\newpage|\\
|\||fi|
\end{tabular}
\end{center}
%
A banner page for the child documents can be generated by:
%
\begin{center}
\begin{tabular}{l}
|\ifchilddoc|\\
|\addtocounter{page}{-1}|\\
\textit{code for banner page}\\
|\newpage|\\
|\||fi|
\end{tabular}
\end{center}
%
Here one could write a message such as:
\begin{center}
|This is the part \childdocname{} of \childdocjob{}.|
\end{center}

%%%%%%%%%%%%%%%%%%%%%%%%%%%%%%%%%%%%%%%%%%%%%%%%%%%%%%%%%%%%%%%%%%%%%%%%%%%%%%%%
\subsection{Flags}
\label{sec:flags}

The package makes it easy to generate different versions
of the main or child documents.
To this end compilation flags can be defined
and assigned different default values.
They will be particularly useful in conjunction
with the forwarding mechanism described in \secref{sec:forward}.

For example, it may be useful to have a flag |\version|
which can be set to |draft| or |final|.
The document source will contain some conditional code
depending on the value of |\version|.
Suppose further, the flag should default to |final| for the main file
and to |draft| for child files
which is a natural assignment for editing the document.
This is achieved by placing the following code
in the preamble of the main document
(below the |\childdocmain| directive):
%
\begin{center}
\begin{tabular}{l}
|\ifchilddoc|\\
|\providecommand{\version}{draft}|\\
|\||else|\\
|\providecommand{\version}{final}|\\
|\||fi|
\end{tabular}
\end{center}
%
The definition by |\providecommand| makes sure
that previous definitions are not overwritten.
Further statements |\providecommand{\version}{...}|
can thus be added before the above code to override it.

For the main file, one might add a line
(between |\childdocmain| and the above block)
%
\begin{center}
|%\ifchilddoc\||else\providecommand{\version}{draft}\||fi|
\end{center}
%
which can be uncommented to produce a draft version.
Likewise one can add a line to the very top of a child file
(above the |\childdocof{|\textit{main}|}| directive)
%
\begin{center}
|%\providecommand{\version}{final}|
\end{center}
%
which can be uncommented to produce the final version of this child document.

%%%%%%%%%%%%%%%%%%%%%%%%%%%%%%%%%%%%%%%%%%%%%%%%%%%%%%%%%%%%%%%%%%%%%%%%%%%%%%%%
\subsection{Forwarding}
\label{sec:forward}

Different versions of the main or child documents
using compilation flags as described in \secref{sec:flags}
can be (permanently) stored in different files
for convenient compilation, viewing and distribution.
To this end, the package defines a command
to pass on compilation to a different file:

%%%%%%%%%%%%%%%%%%%%%%%%%%%%%%%%%%%%%%%%
\DescribeMacro{\childdocforward}
The command |\childdocforward| redirects processing to
another source file:
%
\begin{center}
\begin{tabular}{l}
|\input{childdoc.def}|\\
|\childdocforward[|\textit{main}|]{|\textit{dest}|}|\\
\end{tabular}
\end{center}
%
The argument \textit{dest} is the destination file
(without extension).
It should be the main file or one of the child files.
Note that further \textsf{childdoc} directives
such as |\childdocof| and |\childdocforward|
in the indicated file will be processed in this form.
The optional argument \textit{main}
passes on directly to the main file \textit{main}
while pretending to compile the child \textit{dest}.
This form behaves as if \textit{dest}
issues |\childdocof{|\textit{main}|}| right away,
and no further \textsf{childdoc} directives will be processed.

%%%%%%%%%%%%%%%%%%%%%%%%%%%%%%%%%%%%%%%%
\DescribeMacro{\...prefix}
In the alternative form |\childdocforwardprefix|,
%
\begin{center}
\begin{tabular}{l}
|\input{childdoc.def}|\\
|\childdocforwardprefix[|\textit{main}|]{|\textit{prefix}|}{|\textit{dest}|}|
\end{tabular}
\end{center}
%
the destination file is determined by a pattern
depending on the current file:
To make this work, the current file must be called
`{\textit{prefix}\hspace{0.2em}\textit{suffix}}'
with \textit{prefix} matching precisely the argument.
Processing is then passed on to the file
`{\textit{dest}\hspace{0.2em}\textit{suffix}}'.
Surely, the same effect is achieved by
directly specifying the
argument `{\textit{dest}\hspace{0.2em}\textit{suffix}}'
in the first form.
However, that requires to set up a different file
for each child. With the alternative form of the command
all these files can have exactly the same content
which simplifies setting them up and maintaining them.

For example, the following file |draft.tex|
with a compilation flag |\version| as described in \secref{sec:flags}
compiles the main document as a draft:
%
\begin{center}
\begin{tabular}{l}
|\def\version{draft}|\\
|\input{childdoc.def}|\\
|\childdocforward{|\textit{main}|}|
\end{tabular}
\end{center}
%
Likewise, the following files |final|\textit{nn}|.tex|
compile the final version of the child document
|child|\textit{nn}|.tex|:
%
\begin{center}
\begin{tabular}{l}
|\def\version{final}|\\
|\input{childdoc.def}|\\
|\childdocforwardprefix{final}{child}|
\end{tabular}
\end{center}
%

Note that when several versions of a main file and/or of each child file
are to be generated, it may be convenient to set up a |Makefile| or
shell script to automatise the process.

%%%%%%%%%%%%%%%%%%%%%%%%%%%%%%%%%%%%%%%%%%%%%%%%%%%%%%%%%%%%%%%%%%%%%%%%%%%%%%%%
\subsection{Command Line Processing}
\label{sec:commandline}

The effect of redirection files can also be achieved by invoking
the \LaTeX{} compiler with a more elaborate command line.
Most conveniently this should be done as part
of a shell script or a |Makefile|.

When using \textsf{childdoc} in the main file, the following
command lines effectively perform a redirection
(note that depending on the shell being used,
backslashes may have to be doubled: `|\|' $\to$ `|\\|'):
%
\begin{center}
|... -jobname "|\textit{target}|" |\\|"|[\textit{flags}]%
|\input{childdoc.def}\childdocforward[|\textit{main}|]{|\textit{dest}|}"|
\end{center}
%
Here \textit{target} is the name of the output file,
\textit{main} is the name of the main file
and \textit{dest} is the name of the main or child file to be processed
(all filenames without extensions).
The optional argument \textit{main} can be omitted
if \textit{main} matches \textit{dest}.
Optionally, compilation \textit{flags} can be defined via |\def| commands.
This command line makes the \TeX{} engine believe
it is compiling the file \textit{target}
whose content is specified as the latter parameter.
The provided code then forwards the processing to
\textit{main} or \textit{dest} as described in \secref{sec:forward}.

%%%%%%%%%%%%%%%%%%%%%%%%%%%%%%%%%%%%%%%%%%%%%%%%%%%%%%%%%%%%%%%%%%%%%%%%%%%%%%%%
\subsection{Include by Input}
\label{sec:input}

Including child documents by |\include| has some restrictions by design.
Most notably, the content of a child document always occupies
its own set of pages; pages cannot be shared between child documents.
Usually, this behaviour makes perfect sense
because each child document contain an essential part of the document.
However, in some situations it may be desirable to compose
a document from a collection of parts
without having mandatory page breaks between then.
For this case, the package
provides a mechanism to include parts
by |\input| which can also be processed individually.
However, by construction this mechanism
requires manual handling of the content to be output.

%%%%%%%%%%%%%%%%%%%%%%%%%%%%%%%%%%%%%%%%
\DescribeMacro{\ifchilddocmanual}
The main file should be prepared as usual, see \secref{sec:include}.
However, the document body must make a distinction
between processing of an individual part and of the main document, e.g.:
%
\begin{center}
\begin{tabular}{l}
|\ifchilddocmanual|\\
|\input{\childdocname}|\\
|\||else|\\
\textit{document body with }|\input{|\textit{part}|}|\\
|\||fi|
\end{tabular}
\end{center}
%
The conditional |\ifchilddocmanual| is true whenever
a part to be included by |\input| is being compiled,
and the name of the part is stored in |\childdocname|.

%%%%%%%%%%%%%%%%%%%%%%%%%%%%%%%%%%%%%%%%
\DescribeMacro{\childdocby}
Each part to be included by |\input| should start with:
%
\begin{center}
\begin{tabular}{l}
|\input{childdoc.def}|\\
|\childdocby{|\textit{main}|}|\\
\end{tabular}
\end{center}
%
The directive |\childdocby| is similar to |\childdocof|
described in \secref{sec:include},
but the subsequent selection of content must be done manually.
To that end, both |\ifchilddoc| and |\ifchilddocmanual|
will be true upon processing of a part,
and the name of the part is stored in |\childdocname|.
Note that |\jobname| will be set to the filename of the current part
so that each part receives an individual |.aux| file
that does not interfere with the |.aux| file(s) of the main document.
This behaviour can be altered by the alternative form
|\childdocby[*]{|\textit{main}|}| (with a non-empty optional argument)
which uses the |.aux| file of the main document
by setting |\jobname| to \textit{main}.

%%%%%%%%%%%%%%%%%%%%%%%%%%%%%%%%%%%%%%%%%%%%%%%%%%%%%%%%%%%%%%%%%%%%%%%%%%%%%%%%
\subsection{Driver Development}
\label{sec:driver}

The \textsf{childdoc} mechanism can also be use for the development
of definition files such as \LaTeX{} styles or classes.
This case differs from the above setup with multiple parts
included by |\include| in that no |\includeonly| should be invoked.
This can be achieved by starting the include file
(before |\ProvidesPackage|) with:
%
\begin{center}
\begin{tabular}{l}
|\input{childdoc.def}|\\
|\childdocforward{|\textit{main}|}|\\
\end{tabular}
\end{center}
%
or alternatively with:
%
\begin{center}
\begin{tabular}{l}
|\input{childdoc.def}|\\
|\childdocby{|\textit{main}|}|\\
\end{tabular}
\end{center}
%
Both forms have slightly different effects as described above.
The main file is prepared as usual, see \secref{sec:include}.

%%%%%%%%%%%%%%%%%%%%%%%%%%%%%%%%%%%%%%%%%%%%%%%%%%%%%%%%%%%%%%%%%%%%%%%%%%%%%%%%
\subsection{Legacy Detection}
\label{sec:detection}

The directive |\childdocmain| in the main file can detect
whether the complete document or merely a child is to be compiled
even without using the directive |\childdocof|.
This method is deprecated because it is less robust
and there is no compelling reason to use it;
it is merely provided for backward compatibility
and it may be removed in future versions.

If the detection mechanism is to be used,
it is mandatory to correctly specify
the filename of the main file as the argument of |\childdocmain|:
%
\begin{center}
\begin{tabular}{l}
|\input{childdoc.def}|\\
|\childdocmain{|\textit{main}|}|\\
\end{tabular}
\end{center}
%
If |\jobname| does not match the argument \textit{main} of |\childdocmain|,
it is assumed that |\jobname| points to the child file to be compiled.
When using |\childdocmain| with the main file specified as argument,
it suffices to start a child file
with just |\input{|\textit{main}|}|
without loading of the package and using |\childdocof|.
If instead all processing is done
with the appropriate \textsf{childdoc} directives,
the argument of \textit{main} of |\childdocmain| can be empty.

An alternative version of the command line processing described
in \secref{sec:commandline} using the detection mechanism reads:
%
\begin{center}
|... -jobname "|\textit{target}|" "|[\textit{flags}]%
[|\def\jobname{|\textit{dest}|}|]|\input{|\textit{main}|}"|
\end{center}

%%%%%%%%%%%%%%%%%%%%%%%%%%%%%%%%%%%%%%%%%%%%%%%%%%%%%%%%%%%%%%%%%%%%%%%%%%%%%%%%
\subsection{Manual Code}
\label{sec:manual}

In case one cannot be certain whether the definitions file |childdoc.def|
is installed on the target \TeX{} distribution
and one prefers not to ship it,
it is conceivable to paste a few relevant commands into the sources.

To that end, drop all statements |\input{childdoc.def}|
and perform the replacements as outlined below.
Instead of |\childdocmain{|\textit{main}|}| add the following code
to the top of the main file:
%
\begin{center}
\begin{tabular}{l}
|\||ifdefined\childdocname\endinput\||fi\newif\ifchilddoc|\\
|\edef\childdocname{\scantokens\expandafter{\jobname\noexpand}}|\\
|\def\childdocmain{|\textit{main}|}\||ifx\childdocmain\childdocname\||else|\\
|\childdoctrue\includeonly{\childdocname}\let\jobname\childdocmain\||fi|\\
\end{tabular}
\end{center}
%
Instead of |\childdocof{|\textit{main}|}| just include the main file
at the top of each child file:
%
\begin{center}
|\input{|\textit{main}|}|
\end{center}
%
A simple redirection |\childdocforward{|\textit{dest}|}| is achieved by:
%
\begin{center}
|\def\jobname{|\textit{dest}|}\input{\jobname}|
\end{center}
%
The redirection with prefix
|\childdocforwardprefix[|\textit{prefix}|]{|\textit{dest}|}|
is accomplished by:
%
\begin{center}
\begin{tabular}{l}
|{\edef\jobname{\scantokens\expandafter{\jobname\noexpand}}|\\
|\def\redirectjob |\textit{prefix}|#1~~~{\gdef\jobname{|\textit{dest}|#1}}|\\
|\expandafter\redirectjob\jobname~~~}\input{\jobname}|
\end{tabular}
\end{center}

In an alternative approach,
child documents can be compiled by a specific command line
without additional code or specific definitions:
%
\begin{center}
|... -jobname "|\textit{target}|" "|[\textit{flags}]%
|\includeonly{|\textit{dest}|}\input{|\textit{main}|}"|
\end{center}
%

%%%%%%%%%%%%%%%%%%%%%%%%%%%%%%%%%%%%%%%%%%%%%%%%%%%%%%%%%%%%%%%%%%%%%%%%%%%%%%%%
%%%%%%%%%%%%%%%%%%%%%%%%%%%%%%%%%%%%%%%%%%%%%%%%%%%%%%%%%%%%%%%%%%%%%%%%%%%%%%%%
\section{Information}

%%%%%%%%%%%%%%%%%%%%%%%%%%%%%%%%%%%%%%%%%%%%%%%%%%%%%%%%%%%%%%%%%%%%%%%%%%%%%%%%
\subsection{Copyright}

Copyright \copyright{} 2017--2018 Niklas Beisert

This work may be distributed and/or modified under the
conditions of the \LaTeX{} Project Public License, either version 1.3
of this license or (at your option) any later version.
The latest version of this license is in
  \url{http://www.latex-project.org/lppl.txt}
and version 1.3 or later is part of all distributions of \LaTeX{}
version 2005/12/01 or later.

This work has the LPPL maintenance status `maintained'.

The Current Maintainer of this work is Niklas Beisert.

This work consists of the files |README.txt|, |childdoc.ins| and |childdoc.dtx|
as well as the derived files |childdoc.def|, |cdocsamp.tex|
with |cdocsch1.tex|, |cdocsch2.tex|, |cdocspt3.tex|, |cdocspt4.tex|,
|cdocsdrf.tex|, |cdocsfn1.tex|, |cdocsfn2.tex|
as well as |childdoc.pdf|.

%%%%%%%%%%%%%%%%%%%%%%%%%%%%%%%%%%%%%%%%%%%%%%%%%%%%%%%%%%%%%%%%%%%%%%%%%%%%%%%%
\subsection{Files and Installation}

The package consists of the files:
%
\begin{center}
\begin{tabular}{ll}
    |README.txt|   & readme file \\
    |childdoc.ins| & installation file \\
    |childdoc.dtx| & source file \\
    |childdoc.def| & definition file \\
    |cdocsamp.tex| & sample main file \\
    |cdocsch1.tex| & sample include file \\
    |cdocsch2.tex| & sample include file \\
    |cdocspt3.tex| & sample part file \\
    |cdocspt4.tex| & sample part file \\
    |cdocsdrf.tex| & sample redirection file \\
    |cdocsfn1.tex| & sample redirection file \\
    |cdocsfn2.tex| & sample redirection file \\
    |childdoc.pdf| & manual
\end{tabular}
\end{center}
%
The distribution consists of the files
|README.txt|, |childdoc.ins| and |childdoc.dtx|.
%
\begin{itemize}
\item
Run (pdf)\LaTeX{} on |childdoc.dtx|
to compile the manual |childdoc.pdf| (this file).
\item
Run \LaTeX{} on |childdoc.ins| to create the definitions file |childdoc.def|
and the sample |cdocsamp.tex| with include files
|cdocsch1.tex|, |cdocsch2.tex|, |cdocspt3.tex|, |cdocspt4.tex|,
|cdocsdrf.tex|, |cdocsfn1.tex|, |cdocsfn2.tex|.
Then copy the file |childdoc.def| to an appropriate directory of your \LaTeX{}
distribution, e.g.\ \textit{texmf-root}|/tex/latex/childdoc|.
\end{itemize}

%%%%%%%%%%%%%%%%%%%%%%%%%%%%%%%%%%%%%%%%%%%%%%%%%%%%%%%%%%%%%%%%%%%%%%%%%%%%%%%%
\subsection{Related CTAN Packages}

There are several other packages which offer a similar functionality:
%
\begin{itemize}
\item
The packages
\href{http://ctan.org/pkg/docmute}{\textsf{docmute}},
\href{http://ctan.org/pkg/includex}{\textsf{includex}} and
\href{http://ctan.org/pkg/standalone}{\textsf{standalone}}
provide commands to include only the document body of
a child file thus allowing both files to be compiled individually.
\item
The packages \href{http://ctan.org/pkg/subdocs}{\textsf{subdocs}}
and \href{http://ctan.org/pkg/subfiles}{\textsf{subfiles}}
provide structures in which the main and child documents can be
encapsulated and allowing them to be compiled individually.
The inclusion mechanism is different from the conventional |\include|.
\item
The package \href{http://ctan.org/pkg/combine}{\textsf{combine}}
is an elaborate solution to combine several documents into one.
\end{itemize}
%
See also the CTAN topic \href{http://ctan.org/topic/subdocs}{\textsf{subdocs}}
for further related packages.
The present package differs from the above solutions in that
a document structure constructed with the conventional |\include| mechanism
just needs two extra commands at the top of every file
such that all constituent files can be compiled individually.

%%%%%%%%%%%%%%%%%%%%%%%%%%%%%%%%%%%%%%%%%%%%%%%%%%%%%%%%%%%%%%%%%%%%%%%%%%%%%%%%
%\subsection{Feature Suggestions}
%
%The following is a list of features which may be useful for future
%versions of this package:
%%
%\begin{itemize}
%\item
%\ldots
%\end{itemize}

%%%%%%%%%%%%%%%%%%%%%%%%%%%%%%%%%%%%%%%%%%%%%%%%%%%%%%%%%%%%%%%%%%%%%%%%%%%%%%%%
\subsection{Revision History}

%%%%%%%%%%%%%%%%%%%%%%%%%%%%%%%%%%%%%%%%
\paragraph{v2.0:} 2018/12/30

\begin{itemize}
\item
immediate forward processing
\item
added |\childdocby| mechanism
\item
manual restructured
\end{itemize}

%%%%%%%%%%%%%%%%%%%%%%%%%%%%%%%%%%%%%%%%
\paragraph{v1.6:} 2018/01/17

\begin{itemize}
\item
application for development of include files
\item
corrections to manual
\end{itemize}

%%%%%%%%%%%%%%%%%%%%%%%%%%%%%%%%%%%%%%%%
\paragraph{v1.5:} 2017/05/21

\begin{itemize}
\item
more complete structuring introduced
\item
|\childdocof| introduced
\item
|\childdoc| renamed to |\childdocmain|
\item
|\childredirect| renamed to |\childdocforward| and |\childdocforwardprefix|
and functionality expanded
\end{itemize}

%%%%%%%%%%%%%%%%%%%%%%%%%%%%%%%%%%%%%%%%
\paragraph{v1.0:} 2017/04/27

\begin{itemize}
\item
manual and install package
\item
first version published on CTAN
\end{itemize}

%%%%%%%%%%%%%%%%%%%%%%%%%%%%%%%%%%%%%%%%
\paragraph{v0.6:} 2017/04/26

\begin{itemize}
\item
redirection mechanism added
\end{itemize}

%%%%%%%%%%%%%%%%%%%%%%%%%%%%%%%%%%%%%%%%
\paragraph{v0.5:} 2017/04/26

\begin{itemize}
\item
functionality in definition file
\end{itemize}


%%%%%%%%%%%%%%%%%%%%%%%%%%%%%%%%%%%%%%%%%%%%%%%%%%%%%%%%%%%%%%%%%%%%%%%%%%%%%%%%
%%%%%%%%%%%%%%%%%%%%%%%%%%%%%%%%%%%%%%%%%%%%%%%%%%%%%%%%%%%%%%%%%%%%%%%%%%%%%%%%
%%%%%%%%%%%%%%%%%%%%%%%%%%%%%%%%%%%%%%%%%%%%%%%%%%%%%%%%%%%%%%%%%%%%%%%%%%%%%%%%
\appendix

\settowidth\MacroIndent{\rmfamily\scriptsize 000\ }

 \DocInput{childdoc.dtx}

\end{document}
%</driver>
% \fi
%
% %%%%%%%%%%%%%%%%%%%%%%%%%%%%%%%%%%%%%%%%%%%%%%%%%%%%%%%%%%%%%%%%%%%%%%%%%%%%%%
% %%%%%%%%%%%%%%%%%%%%%%%%%%%%%%%%%%%%%%%%%%%%%%%%%%%%%%%%%%%%%%%%%%%%%%%%%%%%%%
% \section{Sample}
%\iffalse
%<*samplemain>
%\fi
%
% The following presents a sample document
% with two chapters, two parts, a title page,
% a compile flag as well as three forwarding files to set the flag.
% It consists of eight |.tex| files:
% \begin{center}
% \begin{tabular}{ll}
% |cdocsamp.tex|&main file\\
% |cdocsch1.tex|&include file for chapter 1\\
% |cdocsch2.tex|&include file for chapter 2\\
% |cdocspt3.tex|&include file for part 3\\
% |cdocspt4.tex|&include file for part 4\\
% |cdocsdrf.tex|&forwarding file for main file in draft mode\\
% |cdocsfi1.tex|&forwarding file for final version of chapter 1\\
% |cdocsfi2.tex|&forwarding file for final version of chapter 2\\
% \end{tabular}
% \end{center}
% Each of the eight files can be compiled directly by the \LaTeX{} compiler.
%
% %%%%%%%%%%%%%%%%%%%%%%%%%%%%%%%%%%%%%%
% \paragraph{Main File.}
%
% The main file is called |cdocsamp.tex|.
%
% Load the \textsf{childdoc} definitions and
% declare the filename for the main document:
%    \begin{macrocode}
\input{childdoc.def}
\childdocmain{}
%    \end{macrocode}

% Optional override for |\version| flag:
%    \begin{macrocode}
%%\ifchilddoc\else\providecommand{\version}{draft}\fi
%    \end{macrocode}

% Define the default values for the |\version| flag
% (|final| for the main file and |draft| for childs):
%    \begin{macrocode}
\ifchilddoc
\providecommand{\version}{draft}
\else
\providecommand{\version}{final}
\fi
%    \end{macrocode}

% Load the standard document class:
%    \begin{macrocode}
\documentclass[12pt]{article}
%    \end{macrocode}

% Start the document body:
%    \begin{macrocode}
\begin{document}
%    \end{macrocode}

% Declare a title page.
% Print title, part of document being processed and version flag:
%    \begin{macrocode}
\addtocounter{page}{-1}
\begin{center}
{\LARGE\bfseries{}childdoc example\par}
\vspace{1cm}
\ifchilddoc
\ifchilddocmanual part\else chapter\fi:
`\childdocname' of `\childdocjob'\par
\else
main document: `\childdocjob'\par
\fi
version: \version\par
\end{center}
\newpage
%    \end{macrocode}

% Manually include selected file,
% otherwise process as usual:
%    \begin{macrocode}
\ifchilddocmanual
\section*{part `\childdocname'}
\input{\childdocname}
\else
%    \end{macrocode}

% Include the two chapters:
%    \begin{macrocode}
\include{cdocsch1}
\include{cdocsch2}
%    \end{macrocode}

% Include the two parts unless only chapters should be displayed:
%    \begin{macrocode}
\ifchilddoc\else
\section{part three}
\input{cdocspt3}
\section{part four}
\input{cdocspt4}
\fi
%    \end{macrocode}

% Process as usual until here:
%    \begin{macrocode}
\fi
%    \end{macrocode}

% End of document body:
%    \begin{macrocode}
\end{document}
%    \end{macrocode}
%\iffalse
%</samplemain>
%\fi
%
% %%%%%%%%%%%%%%%%%%%%%%%%%%%%%%%%%%%%%%
% \paragraph{Chapter Include Files.}
%
% The include files are called |cdocsch1.tex| and |cdocsch2.tex|.
%
%\iffalse
%<*samplechap1|samplechap2>
%\fi

% Optional override for |\version| flag:
%    \begin{macrocode}
%%\providecommand{\version}{final}
%    \end{macrocode}

% Include the main document:
%    \begin{macrocode}
\input{childdoc.def}
\childdocof{cdocsamp}
%    \end{macrocode}

%\iffalse
%</samplechap1|samplechap2>
%\fi
%
%\iffalse
%<*samplechap1>
%\fi
% Some text for chapter 1:
%    \begin{macrocode}
\section{one}
some text in chapter one
%    \end{macrocode}

%\iffalse
%</samplechap1>
%\fi
% Some text for chapter 2:
%\iffalse
%<*samplechap2>
%\fi
%    \begin{macrocode}
\section{two}
more text in chapter two
%    \end{macrocode}

%\iffalse
%</samplechap2>
%\fi
%
% %%%%%%%%%%%%%%%%%%%%%%%%%%%%%%%%%%%%%%
% \paragraph{Part Include Files.}
%
% The include files are called |cdocspt3.tex| and |cdocspt4.tex|.
%
%\iffalse
%<*samplepart3|samplepart4>
%\fi

% Optional override for |\version| flag:
%    \begin{macrocode}
%%\providecommand{\version}{final}
%    \end{macrocode}

% Include the main document:
%    \begin{macrocode}
\input{childdoc.def}
\childdocby{cdocsamp}
%    \end{macrocode}

%\iffalse
%</samplepart3|samplepart4>
%\fi
%
%\iffalse
%<*samplepart3>
%\fi
% Some text for part 3:
%    \begin{macrocode}
some text in part three
%    \end{macrocode}

%\iffalse
%</samplepart3>
%\fi
% Some text for part 4:
%\iffalse
%<*samplepart4>
%\fi
%    \begin{macrocode}
more text in part four
%    \end{macrocode}

%\iffalse
%</samplepart4>
%\fi
%
% %%%%%%%%%%%%%%%%%%%%%%%%%%%%%%%%%%%%%%
% \paragraph{Forwarding for a Complete Draft.}
%
% The following forwarding file |cdocsdrf.tex|
% compiles the main document in draft mode:
%\iffalse
%<*sampledraft>
%\fi
%    \begin{macrocode}
\def\version{draft}
\input{childdoc.def}
\childdocforward{cdocsamp}
%    \end{macrocode}

%\iffalse
%</sampledraft>
%\fi
%
% %%%%%%%%%%%%%%%%%%%%%%%%%%%%%%%%%%%%%%
% \paragraph{Forwarding for Final Version of the Chapters.}
%
% The following forwarding files |cdocsfn1.tex| and |cdocsfn2.tex|
% (with identical content)
% compile the final versions of the child documents
% |cdocsch1.tex| and |cdocsch2.tex|, respectively:
%\iffalse
%<*samplefinal>
%\fi
%    \begin{macrocode}
\def\version{final}
\input{childdoc.def}
\childdocforwardprefix[cdocsamp]{cdocsfn}{cdocsch}
%    \end{macrocode}

%\iffalse
%</samplefinal>
%\fi
%
% %%%%%%%%%%%%%%%%%%%%%%%%%%%%%%%%%%%%%%
% \paragraph{Command Line Processing.}
%
% The following three command lines generate the output files
% |cdocscld|, |cdocscl1| and |cdocscl2|
% which should be identical to
% |cdocsdrf|, |cdocsch1| and |cdocsfn2|, respectively:
% \begin{center}
% \begin{tabular}{l}
% |latex -jobname cdocscld \|\\
% |  "\def\version{draft}\input{childdoc.def}\childdocforward{cdocsamp}"|\\
% |latex -jobname cdocscl1 \|\\
% |  "\input{childdoc.def}\childdocforward[cdocsamp]{cdocsch1}"|\\
% |latex -jobname cdocscl2 \|\\
% |  "\def\version{final}\input{childdoc.def}\childdocforward{cdocsch2}"|
% \end{tabular}
% \end{center}
% Note that the trailing backslash on each first line
% merely continues the input to the second line
% (for convenient cut ant paste).
% Furthermore, the command |latex| can be replaced by any
% of its alternative versions such as |pdflatex|.
%
% %%%%%%%%%%%%%%%%%%%%%%%%%%%%%%%%%%%%%%%%%%%%%%%%%%%%%%%%%%%%%%%%%%%%%%%%%%%%%%
% %%%%%%%%%%%%%%%%%%%%%%%%%%%%%%%%%%%%%%%%%%%%%%%%%%%%%%%%%%%%%%%%%%%%%%%%%%%%%%
% \section{Implementation}
%\iffalse
%<*package>
%\fi
%
% This section describes the definitions file |childdoc.def|.

% The definitions cannot be loaded using |\usepackage| or |\RequirePackage|
% which has a mechanism to prevent loading a style file more than once.
% When loading the definitions by means of |\input|
% multiple instances have to be prevented manually:
%\iffalse
%This code needs to be before the `\ProvidesFile' directive
%which is defined at the beginning of this file.
%Therefore it is also placed there and commented out here.
%</package>
%<*discard>
%\fi
%    \begin{macrocode}
\ifdefined\childdocmain\endinput\fi
%    \end{macrocode}
%\iffalse
%</discard>
%<*package>
%\fi
%
% \macro{\ifchilddoc}
% \macro{\ifchilddocmanual}
% The conditional |\ifchilddoc| tells whether a
% child (true) or main (false) document is being compiled.
% The conditional |\ifchilddocmanual| tells whether
% the |\includeonly| mechanism is used (false) or
% the selection of child files must be performed manually (true).
% The definitions initialise to false:
%    \begin{macrocode}
\newif\ifchilddoc
\newif\ifchilddocmanual
%    \end{macrocode}

% \macro{\childdocname}
% \macro{\childdocjob}
% The macro |\childdocname| stores the name of the main document
% to be compiled. The macro |\childdocjob| stores the name of
% the document on which the \LaTeX{} compiler was originally invoked.
% The content of |\jobname| cannot be compared
% to filenames specified in the source due to different catcodes.
% The following code rescans |\jobname|, stores the result
% in |\childdocname| and saves a copy in |\childdocjob|:
%    \begin{macrocode}
\edef\childdocname{\scantokens\expandafter{\jobname\noexpand}}
\let\childdocjob\childdocname
%    \end{macrocode}

% \macro{\childdocdisable}
% The macro |\childdocdisable| prevents the main file
% from being processed more than once.
% At this stage, the main document command |\childdocmain|
% is assumed to be called once again where it should do nothing.
% Any subsequent call to it should prevent
% a secondary processing of the main document
% It overwrites the forwarding commands
% |\childdocof| and |\childdocforward|
% with empty macros to prevent further inclusions of the main document:
%    \begin{macrocode}
\newcommand{\childdocdisable}
{
  \renewcommand{\childdocmain}[1]{\renewcommand{\childdocmain}[1]{\endinput}}
  \renewcommand{\childdocof}[1]{}
  \renewcommand{\childdocby}[2][]{}
  \renewcommand{\childdocforward}[2][]{}
  \renewcommand{\childdocdisable}{}
}
%    \end{macrocode}

% \macro{\childdocmain}
% The macro |\childdocmain| is to be called at the top of the main file
% with nothing or the main filename (without extension) as argument.
% First, it breaks loops.
% If the argument is not empty and does not match |\childdocname|
% (which is set by the first inclusion of |childdoc.def|),
% |\ifchilddoc| is set to true, |\includeonly| is applied to the child file
% and |\jobname| is set to the main file
% (for proper handling of |.aux| files):
%    \begin{macrocode}
\newcommand{\childdocmain}[1]
{
  \childdocdisable\childdocmain{}
  \if?#1?\else
    \begingroup
      \def\childdoctmp{#1}
      \ifx\childdoctmp\childdocname
        \def\childdoctmp{}
      \else
        \def\childdoctmp
        {
          \childdoctrue
          \includeonly{\childdocname}
          \def\childdocjob{#1}
          \def\jobname{#1}
        }
      \fi
      \expandafter
    \endgroup
    \childdoctmp
  \fi
}
%    \end{macrocode}

% \macro{\childdocof}
% The command |\childdocof| redirects
% compilation to the main file |#1|.
%    \begin{macrocode}
\newcommand{\childdocof}[1]
{
  \childdocdisable
  \childdoctrue
  \includeonly{\childdocname}
  \def\jobname{#1}
  \def\childdocjob{#1}
  \input{#1}
}
%    \end{macrocode}

% \macro{\childdocby}
% The command |\childdocby| ....
%    \begin{macrocode}
\newcommand{\childdocby}[2][]
{
  \childdocdisable
  \childdoctrue
  \childdocmanualtrue
  \if?#1?\else
    \def\jobname{#2}
  \fi
  \def\childdocjob{#2}
  \input{#2}
  \endinput
}
%    \end{macrocode}

% \macro{\childdocforward}
% The command |\childdocforward| redirects
% compilation to the main file or
% (if the optional argument is given) a child file.
% Parameters are set as if the main file
% or a child file starting with |\childdocof| was compiled.
% Then compilation is handed over to the main file:
%    \begin{macrocode}
\newcommand{\childdocforward}[2][]
{
  \begingroup
    \if?#1?
      \def\childdoctmp
      {
        \def\childdocname{#2}
        \def\childdocjob{#2}
        \def\jobname{#2}
        \input{#2}
        \endinput
      }
    \else
      \def\childdoctmp
      {
        \childdocdisable
        \def\childdocname{#2}
        \childdoctrue
        \includeonly{#2}
        \def\childdocjob{#1}
        \def\jobname{#1}
        \input{#1}
        \endinput
      }
    \fi
    \expandafter
  \endgroup
  \childdoctmp
}
%    \end{macrocode}

% \macro{\childdocforwardprefix}
% The command |\childdocforwardprefix| redirects
% compilation to the main or a child file by means of a pattern.
% The prefix |#1| in the current filename is replaced by |#2|
% and the suffix of the current filename is kept
% (it is assumed that the filename does not contain the substring `|~~~|'
% which is used as a delimiter).
% Compilation is handed over to the new file by |\childdocforward|:
%    \begin{macrocode}
\newcommand{\childdocforwardprefix}[3][]
{
  \begingroup
    \def\childdocextract #2##1~~~{\def\childdoctmp{\childdocforward[#1]{#3##1}}}
    \expandafter\childdocextract\childdocname~~~
    \expandafter
  \endgroup
  \childdoctmp
}
%    \end{macrocode}

% \macro{\childdoc}
% The deprecated macro |\childdoc| is a legacy version of |\childdocmain|:
%    \begin{macrocode}
\newcommand{\childdoc}{\childdocmain}
%    \end{macrocode}

% \macro{\childdocredirect}
% The deprecated macro |\childdocredirect| is a legacy version
% of |\childdocforward| and |\childdocforwardprefix|:
%    \begin{macrocode}
\newcommand{\childdocredirect}[2][]
{
  \begingroup
    \if?#1?
      \def\childdoctmp{\childdocforward{#2}}
    \else
      \def\childdoctmp{\childdocforwardprefix{#1}{#2}}
    \fi
    \expandafter
  \endgroup
  \childdoctmp
}
%    \end{macrocode}

%\iffalse
%</package>
%\fi
%
\endinput
|\\
|\childdocby{|\textit{main}|}|\\
\end{tabular}
\end{center}
%
The directive |\childdocby| is similar to |\childdocof|
described in \secref{sec:include},
but the subsequent selection of content must be done manually.
To that end, both |\ifchilddoc| and |\ifchilddocmanual|
will be true upon processing of a part,
and the name of the part is stored in |\childdocname|.
Note that |\jobname| will be set to the filename of the current part
so that each part receives an individual |.aux| file
that does not interfere with the |.aux| file(s) of the main document.
This behaviour can be altered by the alternative form
|\childdocby[*]{|\textit{main}|}| (with a non-empty optional argument)
which uses the |.aux| file of the main document
by setting |\jobname| to \textit{main}.

%%%%%%%%%%%%%%%%%%%%%%%%%%%%%%%%%%%%%%%%%%%%%%%%%%%%%%%%%%%%%%%%%%%%%%%%%%%%%%%%
\subsection{Driver Development}
\label{sec:driver}

The \textsf{childdoc} mechanism can also be use for the development
of definition files such as \LaTeX{} styles or classes.
This case differs from the above setup with multiple parts
included by |\include| in that no |\includeonly| should be invoked.
This can be achieved by starting the include file
(before |\ProvidesPackage|) with:
%
\begin{center}
\begin{tabular}{l}
|% \iffalse
%
% childdoc.dtx Copyright (C) 2017-2018 Niklas Beisert
%
% This work may be distributed and/or modified under the
% conditions of the LaTeX Project Public License, either version 1.3
% of this license or (at your option) any later version.
% The latest version of this license is in
%   http://www.latex-project.org/lppl.txt
% and version 1.3 or later is part of all distributions of LaTeX
% version 2005/12/01 or later.
%
% This work has the LPPL maintenance status `maintained'.
%
% The Current Maintainer of this work is Niklas Beisert.
%
% This work consists of the files childdoc.dtx and childdoc.ins
% and the derived files childdoc.def and cdocsamp.tex with
% cdocsch1.tex, cdocsch2.tex, cdocsdrf.tex, cdocsfn1.tex, cdocsfn2.tex.
%
%<package>\ifdefined\childdocmain\endinput\fi
%<package>\ProvidesFile{childdoc.def}[2018/12/30 v2.0 child document driver]
%<samplemain>\ProvidesFile{cdocsamp.tex}[2018/12/30 v2.0 sample for childdoc]
%<*driver>
%\ProvidesFile{childdoc.drv}[2018/12/30 v2.0 childdoc reference manual file]
\PassOptionsToClass{10pt,a4paper}{article}
\documentclass{ltxdoc}

\usepackage[margin=35mm]{geometry}
\usepackage{hyperref}
\usepackage{hyperxmp}
\usepackage[usenames]{color}

\hypersetup{colorlinks=true}
\hypersetup{pdfstartview=FitH}
\hypersetup{pdfpagemode=UseNone}
\hypersetup{pdfsource={}}
\hypersetup{pdflang={en-UK}}
\hypersetup{pdfcopyright={Copyright 2017-2018 Niklas Beisert.
  This work may be distributed and/or modified under the
  conditions of the LaTeX Project Public License, either version 1.3
  of this license or (at your option) any later version.}}
\hypersetup{pdflicenseurl={http://www.latex-project.org/lppl.txt}}
\hypersetup{pdfcontactaddress={ETH Zurich, ITP, HIT K,
  Wolfgang-Pauli-Strasse 27}}
\hypersetup{pdfcontactpostcode={8093}}
\hypersetup{pdfcontactcity={Zurich}}
\hypersetup{pdfcontactcountry={Switzerland}}
\hypersetup{pdfcontactemail={nbeisert@itp.phys.ethz.ch}}
\hypersetup{pdfcontacturl={http://people.phys.ethz.ch/\xmptilde nbeisert/}}

\newcommand{\secref}[1]{\hyperref[#1]{section \ref*{#1}}}

\parskip1ex
\parindent0pt
\let\olditemize\itemize
\def\itemize{\olditemize\parskip0pt}

\begin{document}

\title{The \textsf{childdoc} Package}
\hypersetup{pdftitle={The childdoc Package}}
\author{Niklas Beisert\\[2ex]
  Institut f\"ur Theoretische Physik\\
  Eidgen\"ossische Technische Hochschule Z\"urich\\
  Wolfgang-Pauli-Strasse 27, 8093 Z\"urich, Switzerland\\[1ex]
  \href{mailto:nbeisert@itp.phys.ethz.ch}
  {\texttt{nbeisert@itp.phys.ethz.ch}}}
\hypersetup{pdfauthor={Niklas Beisert}}
\hypersetup{pdfsubject={Manual for the LaTeX2e Package childdoc}}
\date{30 December 2018, \textsf{v2.0}}
\maketitle

\begin{abstract}\noindent
\textsf{childdoc} is a \LaTeXe{} package
that enables the direct compilation
of document sections included by |\include|
to individual files.
\end{abstract}

\begingroup
\parskip0ex
\tableofcontents
\endgroup

%%%%%%%%%%%%%%%%%%%%%%%%%%%%%%%%%%%%%%%%%%%%%%%%%%%%%%%%%%%%%%%%%%%%%%%%%%%%%%%%
%%%%%%%%%%%%%%%%%%%%%%%%%%%%%%%%%%%%%%%%%%%%%%%%%%%%%%%%%%%%%%%%%%%%%%%%%%%%%%%%
\section{Introduction}

\LaTeX{} provides a mechanism to structure a large document (such as a book)
into a main file and several child files (containing the chapters)
using the |\include| command.
This mechanism is beneficial for documents
which span hundreds of pages in order to
make the source file(s) more manageable.
Moreover, compilation can be restricted to
selected child files by means of the |\includeonly| command.
The latter feature can be used to reduce the compilation time while editing
(this was significantly more useful in the earlier days of \LaTeX{})
or to generate a smaller document which is easier to navigate.
Another application of |\includeonly| is to generate
documents consisting of selected parts of the complete document.

However, there are a few drawbacks of the plain |\include| mechanism:
\begin{itemize}
\item
The child files cannot be compiled on their own,
they can only be compiled via the main file.
A naive editing environment
(such as a text editor with an option
to have the current file processed by \LaTeX)
may require one to switch to the main file before compiling;
attempting to compile the child file produces errors.
\item
The main file must be modified (each time)
to adjust the |\includeonly| command
to the present needs. This easily leaves the main file in a messy state.
\item
The generated document will always carry the filename
of the main document. This is inconvenient if
several child files are to be compiled and
to be kept for distribution.
\end{itemize}

The present package provides a simple interface
to make child files individually compilable by \LaTeX{}.
Compiling a child file then has the same effect as compiling
the main file with an |\includeonly| command
to select the appropriate child.
Moreover the generated document will carry the name of the child
rather than the main file.
This resolves all three above issues.

This feature is meant to make the editing of books,
thesis documents and lecture notes somewhat more convenient.
However, the package can also be used efficiently for
composing a series of documents (such as exercise sheets)
which are typically distributed individually.
It then assists the author in generating the individual documents
(potentially in different versions)
as well as a document containing the collected series.
Another application is in developing style files
or other kinds of included material
where compilation of the style file could redirect
to a sample or test file.

%%%%%%%%%%%%%%%%%%%%%%%%%%%%%%%%%%%%%%%%%%%%%%%%%%%%%%%%%%%%%%%%%%%%%%%%%%%%%%%%
%%%%%%%%%%%%%%%%%%%%%%%%%%%%%%%%%%%%%%%%%%%%%%%%%%%%%%%%%%%%%%%%%%%%%%%%%%%%%%%%
\section{Usage}

First of all, the package \textsf{childdoc} is \emph{not} a standard
\LaTeXe{} |.sty| style file! Therefore it needs to be invoked in
a non-standard way.

%%%%%%%%%%%%%%%%%%%%%%%%%%%%%%%%%%%%%%%%%%%%%%%%%%%%%%%%%%%%%%%%%%%%%%%%%%%%%%%%
\subsection{Included Files}
\label{sec:include}

%%%%%%%%%%%%%%%%%%%%%%%%%%%%%%%%%%%%%%%%
\DescribeMacro{\childdocmain}
To use the package, add the commands
\begin{center}
\begin{tabular}{l}
|\input{childdoc.def}|\\
|\childdocmain{}|\\
\end{tabular}
\end{center}
at the very top of the main \LaTeX{} file,
in particular \emph{before} the |\documentclass| statement!
The argument of |\childdocmain| should be left empty
(but it must be present).

%%%%%%%%%%%%%%%%%%%%%%%%%%%%%%%%%%%%%%%%
\DescribeMacro{\childdocof}
Furthermore, add the commands
\begin{center}
\begin{tabular}{l}
|\input{childdoc.def}|\\
|\childdocof{|\textit{main}|}|\\
\end{tabular}
\end{center}
at the top of every child file \textit{child}
which is included by |\include{|\textit{child}|}|
from within the main file
(or at least for those files to be compiled individually).
The argument \textit{main} must be the filename of the main file.

There are a couple of
considerations in setting up the main and child documents:

%%%%%%%%%%%%%%%%%%%%%%%%%%%%%%%%%%%%%%%%
\paragraph{Restrictions.}

Please note the following restrictions:
\begin{itemize}
\item
|\childdocmain| must be called with one argument \textit{main}
to ensure compatibility with earlier version of the package.
It must either be empty (|\childdocmain{}|)
or precisely match the filename of the main file in which it is specified.
See \secref{sec:detection} for further information.
\item
The filename \textit{main} must be specified without the |.tex| extension.
\item
The filename \textit{main} is case sensitive
(even in case-insensitive file systems)
due to internal string comparison.
\item
The argument \textit{main} should be fully expanded, it cannot be a macro.
\item
Subdirectories and special characters should be avoided in filenames.
\item
The command |\childdocmain{|\textit{main}|}| must be followed by a whitespace.
It should not be followed immediately by another command
or by a comment mark `|%|'.
This is because the \TeX{} parser reads the token immediately following
the argument of |\childdocmain| and puts it
at the beginning of every child section;
however, a white\-space is ignored.
\end{itemize}

%%%%%%%%%%%%%%%%%%%%%%%%%%%%%%%%%%%%%%%%
\paragraph{Content of Main File.}

It is advisable to place all content in the child files included by |\include|.
Any output contained in the main file will appear in all child documents
unless suppressed manually;
it cannot be suppressed automatically by the |\includeonly| directive
and thus should normally be avoided.
A method to include some content in the main file
by means of conditional processing is described in \secref{sec:conditional}.

%%%%%%%%%%%%%%%%%%%%%%%%%%%%%%%%%%%%%%%%
\paragraph{Page Numbering.}

When only a part of the document is compiled,
the appropriate numbering of pages
(as well as other status parameters)
is determined from the |.aux| files.
The latter contain information from previous passes.
However this information needs to propagate through
all intermediate child documents.
Therefore the page numbering in child documents may well
be inconsistent until the complete document is compiled at least once.

A useful (if unconventional) way to always ensure a consistent
page numbering is to restart the numbering in each child document
and denote the pages by `\textit{child}|.|\textit{page}'
where \textit{child} represents the chapter/section number of the child file.
This can be achieved by the command
|\numberwithin{page}{|\textit{child}|}|
of the \textsf{amsmath} package
where \textit{child} can be |chapter| or |section|
depending on the chosen structuring.
Alternatively, one can modify the macro |\thepage| appropriately
and reset the counter |page| at the start of each child file.

%%%%%%%%%%%%%%%%%%%%%%%%%%%%%%%%%%%%%%%%%%%%%%%%%%%%%%%%%%%%%%%%%%%%%%%%%%%%%%%%
\subsection{Conditional Processing}
\label{sec:conditional}

The package provides a mechanism to compile different versions
of a document. To customise the versions further some conditional processing
can come in handy to distinguish which version is being compiled.
The package provides two macros to describe the compilation context:

%%%%%%%%%%%%%%%%%%%%%%%%%%%%%%%%%%%%%%%%
\DescribeMacro{\ifchilddoc}
The conditional |\ifchilddoc| distinguishes between the compilation of
child documents and the main document:
%
\begin{center}
|\ifchilddoc |\textit{child-code}| |[|\||else |\textit{main-code}]| \||fi|
\end{center}

%%%%%%%%%%%%%%%%%%%%%%%%%%%%%%%%%%%%%%%%
\DescribeMacro{\childdocname}
\DescribeMacro{\childdocjob}
The macro |\childdocname| contains the filename (without extension)
of the main or child file being processed.
Note that |\childdocjob| will always contain the name of the main file.

%%%%%%%%%%%%%%%%%%%%%%%%%%%%%%%%%%%%%%%%
\paragraph{Title Page.}

Conditional processing can be used to include a title or banner page
in the main document when proper precautions are taken.
Importantly, the code in the main file should ensure that the page counter
(as well as other status parameters which are stored in the |.aux| files)
takes the same value after the conditional processing.
Otherwise the page numbers may take divergent values
depending on which part is compiled.

For example, a title page could be declared by:
%
\begin{center}
\begin{tabular}{l}
|\ifchilddoc\||else|\\
|\addtocounter{page}{-1}|\\
\textit{code for title page}\\
|\newpage|\\
|\||fi|
\end{tabular}
\end{center}
%
A banner page for the child documents can be generated by:
%
\begin{center}
\begin{tabular}{l}
|\ifchilddoc|\\
|\addtocounter{page}{-1}|\\
\textit{code for banner page}\\
|\newpage|\\
|\||fi|
\end{tabular}
\end{center}
%
Here one could write a message such as:
\begin{center}
|This is the part \childdocname{} of \childdocjob{}.|
\end{center}

%%%%%%%%%%%%%%%%%%%%%%%%%%%%%%%%%%%%%%%%%%%%%%%%%%%%%%%%%%%%%%%%%%%%%%%%%%%%%%%%
\subsection{Flags}
\label{sec:flags}

The package makes it easy to generate different versions
of the main or child documents.
To this end compilation flags can be defined
and assigned different default values.
They will be particularly useful in conjunction
with the forwarding mechanism described in \secref{sec:forward}.

For example, it may be useful to have a flag |\version|
which can be set to |draft| or |final|.
The document source will contain some conditional code
depending on the value of |\version|.
Suppose further, the flag should default to |final| for the main file
and to |draft| for child files
which is a natural assignment for editing the document.
This is achieved by placing the following code
in the preamble of the main document
(below the |\childdocmain| directive):
%
\begin{center}
\begin{tabular}{l}
|\ifchilddoc|\\
|\providecommand{\version}{draft}|\\
|\||else|\\
|\providecommand{\version}{final}|\\
|\||fi|
\end{tabular}
\end{center}
%
The definition by |\providecommand| makes sure
that previous definitions are not overwritten.
Further statements |\providecommand{\version}{...}|
can thus be added before the above code to override it.

For the main file, one might add a line
(between |\childdocmain| and the above block)
%
\begin{center}
|%\ifchilddoc\||else\providecommand{\version}{draft}\||fi|
\end{center}
%
which can be uncommented to produce a draft version.
Likewise one can add a line to the very top of a child file
(above the |\childdocof{|\textit{main}|}| directive)
%
\begin{center}
|%\providecommand{\version}{final}|
\end{center}
%
which can be uncommented to produce the final version of this child document.

%%%%%%%%%%%%%%%%%%%%%%%%%%%%%%%%%%%%%%%%%%%%%%%%%%%%%%%%%%%%%%%%%%%%%%%%%%%%%%%%
\subsection{Forwarding}
\label{sec:forward}

Different versions of the main or child documents
using compilation flags as described in \secref{sec:flags}
can be (permanently) stored in different files
for convenient compilation, viewing and distribution.
To this end, the package defines a command
to pass on compilation to a different file:

%%%%%%%%%%%%%%%%%%%%%%%%%%%%%%%%%%%%%%%%
\DescribeMacro{\childdocforward}
The command |\childdocforward| redirects processing to
another source file:
%
\begin{center}
\begin{tabular}{l}
|\input{childdoc.def}|\\
|\childdocforward[|\textit{main}|]{|\textit{dest}|}|\\
\end{tabular}
\end{center}
%
The argument \textit{dest} is the destination file
(without extension).
It should be the main file or one of the child files.
Note that further \textsf{childdoc} directives
such as |\childdocof| and |\childdocforward|
in the indicated file will be processed in this form.
The optional argument \textit{main}
passes on directly to the main file \textit{main}
while pretending to compile the child \textit{dest}.
This form behaves as if \textit{dest}
issues |\childdocof{|\textit{main}|}| right away,
and no further \textsf{childdoc} directives will be processed.

%%%%%%%%%%%%%%%%%%%%%%%%%%%%%%%%%%%%%%%%
\DescribeMacro{\...prefix}
In the alternative form |\childdocforwardprefix|,
%
\begin{center}
\begin{tabular}{l}
|\input{childdoc.def}|\\
|\childdocforwardprefix[|\textit{main}|]{|\textit{prefix}|}{|\textit{dest}|}|
\end{tabular}
\end{center}
%
the destination file is determined by a pattern
depending on the current file:
To make this work, the current file must be called
`{\textit{prefix}\hspace{0.2em}\textit{suffix}}'
with \textit{prefix} matching precisely the argument.
Processing is then passed on to the file
`{\textit{dest}\hspace{0.2em}\textit{suffix}}'.
Surely, the same effect is achieved by
directly specifying the
argument `{\textit{dest}\hspace{0.2em}\textit{suffix}}'
in the first form.
However, that requires to set up a different file
for each child. With the alternative form of the command
all these files can have exactly the same content
which simplifies setting them up and maintaining them.

For example, the following file |draft.tex|
with a compilation flag |\version| as described in \secref{sec:flags}
compiles the main document as a draft:
%
\begin{center}
\begin{tabular}{l}
|\def\version{draft}|\\
|\input{childdoc.def}|\\
|\childdocforward{|\textit{main}|}|
\end{tabular}
\end{center}
%
Likewise, the following files |final|\textit{nn}|.tex|
compile the final version of the child document
|child|\textit{nn}|.tex|:
%
\begin{center}
\begin{tabular}{l}
|\def\version{final}|\\
|\input{childdoc.def}|\\
|\childdocforwardprefix{final}{child}|
\end{tabular}
\end{center}
%

Note that when several versions of a main file and/or of each child file
are to be generated, it may be convenient to set up a |Makefile| or
shell script to automatise the process.

%%%%%%%%%%%%%%%%%%%%%%%%%%%%%%%%%%%%%%%%%%%%%%%%%%%%%%%%%%%%%%%%%%%%%%%%%%%%%%%%
\subsection{Command Line Processing}
\label{sec:commandline}

The effect of redirection files can also be achieved by invoking
the \LaTeX{} compiler with a more elaborate command line.
Most conveniently this should be done as part
of a shell script or a |Makefile|.

When using \textsf{childdoc} in the main file, the following
command lines effectively perform a redirection
(note that depending on the shell being used,
backslashes may have to be doubled: `|\|' $\to$ `|\\|'):
%
\begin{center}
|... -jobname "|\textit{target}|" |\\|"|[\textit{flags}]%
|\input{childdoc.def}\childdocforward[|\textit{main}|]{|\textit{dest}|}"|
\end{center}
%
Here \textit{target} is the name of the output file,
\textit{main} is the name of the main file
and \textit{dest} is the name of the main or child file to be processed
(all filenames without extensions).
The optional argument \textit{main} can be omitted
if \textit{main} matches \textit{dest}.
Optionally, compilation \textit{flags} can be defined via |\def| commands.
This command line makes the \TeX{} engine believe
it is compiling the file \textit{target}
whose content is specified as the latter parameter.
The provided code then forwards the processing to
\textit{main} or \textit{dest} as described in \secref{sec:forward}.

%%%%%%%%%%%%%%%%%%%%%%%%%%%%%%%%%%%%%%%%%%%%%%%%%%%%%%%%%%%%%%%%%%%%%%%%%%%%%%%%
\subsection{Include by Input}
\label{sec:input}

Including child documents by |\include| has some restrictions by design.
Most notably, the content of a child document always occupies
its own set of pages; pages cannot be shared between child documents.
Usually, this behaviour makes perfect sense
because each child document contain an essential part of the document.
However, in some situations it may be desirable to compose
a document from a collection of parts
without having mandatory page breaks between then.
For this case, the package
provides a mechanism to include parts
by |\input| which can also be processed individually.
However, by construction this mechanism
requires manual handling of the content to be output.

%%%%%%%%%%%%%%%%%%%%%%%%%%%%%%%%%%%%%%%%
\DescribeMacro{\ifchilddocmanual}
The main file should be prepared as usual, see \secref{sec:include}.
However, the document body must make a distinction
between processing of an individual part and of the main document, e.g.:
%
\begin{center}
\begin{tabular}{l}
|\ifchilddocmanual|\\
|\input{\childdocname}|\\
|\||else|\\
\textit{document body with }|\input{|\textit{part}|}|\\
|\||fi|
\end{tabular}
\end{center}
%
The conditional |\ifchilddocmanual| is true whenever
a part to be included by |\input| is being compiled,
and the name of the part is stored in |\childdocname|.

%%%%%%%%%%%%%%%%%%%%%%%%%%%%%%%%%%%%%%%%
\DescribeMacro{\childdocby}
Each part to be included by |\input| should start with:
%
\begin{center}
\begin{tabular}{l}
|\input{childdoc.def}|\\
|\childdocby{|\textit{main}|}|\\
\end{tabular}
\end{center}
%
The directive |\childdocby| is similar to |\childdocof|
described in \secref{sec:include},
but the subsequent selection of content must be done manually.
To that end, both |\ifchilddoc| and |\ifchilddocmanual|
will be true upon processing of a part,
and the name of the part is stored in |\childdocname|.
Note that |\jobname| will be set to the filename of the current part
so that each part receives an individual |.aux| file
that does not interfere with the |.aux| file(s) of the main document.
This behaviour can be altered by the alternative form
|\childdocby[*]{|\textit{main}|}| (with a non-empty optional argument)
which uses the |.aux| file of the main document
by setting |\jobname| to \textit{main}.

%%%%%%%%%%%%%%%%%%%%%%%%%%%%%%%%%%%%%%%%%%%%%%%%%%%%%%%%%%%%%%%%%%%%%%%%%%%%%%%%
\subsection{Driver Development}
\label{sec:driver}

The \textsf{childdoc} mechanism can also be use for the development
of definition files such as \LaTeX{} styles or classes.
This case differs from the above setup with multiple parts
included by |\include| in that no |\includeonly| should be invoked.
This can be achieved by starting the include file
(before |\ProvidesPackage|) with:
%
\begin{center}
\begin{tabular}{l}
|\input{childdoc.def}|\\
|\childdocforward{|\textit{main}|}|\\
\end{tabular}
\end{center}
%
or alternatively with:
%
\begin{center}
\begin{tabular}{l}
|\input{childdoc.def}|\\
|\childdocby{|\textit{main}|}|\\
\end{tabular}
\end{center}
%
Both forms have slightly different effects as described above.
The main file is prepared as usual, see \secref{sec:include}.

%%%%%%%%%%%%%%%%%%%%%%%%%%%%%%%%%%%%%%%%%%%%%%%%%%%%%%%%%%%%%%%%%%%%%%%%%%%%%%%%
\subsection{Legacy Detection}
\label{sec:detection}

The directive |\childdocmain| in the main file can detect
whether the complete document or merely a child is to be compiled
even without using the directive |\childdocof|.
This method is deprecated because it is less robust
and there is no compelling reason to use it;
it is merely provided for backward compatibility
and it may be removed in future versions.

If the detection mechanism is to be used,
it is mandatory to correctly specify
the filename of the main file as the argument of |\childdocmain|:
%
\begin{center}
\begin{tabular}{l}
|\input{childdoc.def}|\\
|\childdocmain{|\textit{main}|}|\\
\end{tabular}
\end{center}
%
If |\jobname| does not match the argument \textit{main} of |\childdocmain|,
it is assumed that |\jobname| points to the child file to be compiled.
When using |\childdocmain| with the main file specified as argument,
it suffices to start a child file
with just |\input{|\textit{main}|}|
without loading of the package and using |\childdocof|.
If instead all processing is done
with the appropriate \textsf{childdoc} directives,
the argument of \textit{main} of |\childdocmain| can be empty.

An alternative version of the command line processing described
in \secref{sec:commandline} using the detection mechanism reads:
%
\begin{center}
|... -jobname "|\textit{target}|" "|[\textit{flags}]%
[|\def\jobname{|\textit{dest}|}|]|\input{|\textit{main}|}"|
\end{center}

%%%%%%%%%%%%%%%%%%%%%%%%%%%%%%%%%%%%%%%%%%%%%%%%%%%%%%%%%%%%%%%%%%%%%%%%%%%%%%%%
\subsection{Manual Code}
\label{sec:manual}

In case one cannot be certain whether the definitions file |childdoc.def|
is installed on the target \TeX{} distribution
and one prefers not to ship it,
it is conceivable to paste a few relevant commands into the sources.

To that end, drop all statements |\input{childdoc.def}|
and perform the replacements as outlined below.
Instead of |\childdocmain{|\textit{main}|}| add the following code
to the top of the main file:
%
\begin{center}
\begin{tabular}{l}
|\||ifdefined\childdocname\endinput\||fi\newif\ifchilddoc|\\
|\edef\childdocname{\scantokens\expandafter{\jobname\noexpand}}|\\
|\def\childdocmain{|\textit{main}|}\||ifx\childdocmain\childdocname\||else|\\
|\childdoctrue\includeonly{\childdocname}\let\jobname\childdocmain\||fi|\\
\end{tabular}
\end{center}
%
Instead of |\childdocof{|\textit{main}|}| just include the main file
at the top of each child file:
%
\begin{center}
|\input{|\textit{main}|}|
\end{center}
%
A simple redirection |\childdocforward{|\textit{dest}|}| is achieved by:
%
\begin{center}
|\def\jobname{|\textit{dest}|}\input{\jobname}|
\end{center}
%
The redirection with prefix
|\childdocforwardprefix[|\textit{prefix}|]{|\textit{dest}|}|
is accomplished by:
%
\begin{center}
\begin{tabular}{l}
|{\edef\jobname{\scantokens\expandafter{\jobname\noexpand}}|\\
|\def\redirectjob |\textit{prefix}|#1~~~{\gdef\jobname{|\textit{dest}|#1}}|\\
|\expandafter\redirectjob\jobname~~~}\input{\jobname}|
\end{tabular}
\end{center}

In an alternative approach,
child documents can be compiled by a specific command line
without additional code or specific definitions:
%
\begin{center}
|... -jobname "|\textit{target}|" "|[\textit{flags}]%
|\includeonly{|\textit{dest}|}\input{|\textit{main}|}"|
\end{center}
%

%%%%%%%%%%%%%%%%%%%%%%%%%%%%%%%%%%%%%%%%%%%%%%%%%%%%%%%%%%%%%%%%%%%%%%%%%%%%%%%%
%%%%%%%%%%%%%%%%%%%%%%%%%%%%%%%%%%%%%%%%%%%%%%%%%%%%%%%%%%%%%%%%%%%%%%%%%%%%%%%%
\section{Information}

%%%%%%%%%%%%%%%%%%%%%%%%%%%%%%%%%%%%%%%%%%%%%%%%%%%%%%%%%%%%%%%%%%%%%%%%%%%%%%%%
\subsection{Copyright}

Copyright \copyright{} 2017--2018 Niklas Beisert

This work may be distributed and/or modified under the
conditions of the \LaTeX{} Project Public License, either version 1.3
of this license or (at your option) any later version.
The latest version of this license is in
  \url{http://www.latex-project.org/lppl.txt}
and version 1.3 or later is part of all distributions of \LaTeX{}
version 2005/12/01 or later.

This work has the LPPL maintenance status `maintained'.

The Current Maintainer of this work is Niklas Beisert.

This work consists of the files |README.txt|, |childdoc.ins| and |childdoc.dtx|
as well as the derived files |childdoc.def|, |cdocsamp.tex|
with |cdocsch1.tex|, |cdocsch2.tex|, |cdocspt3.tex|, |cdocspt4.tex|,
|cdocsdrf.tex|, |cdocsfn1.tex|, |cdocsfn2.tex|
as well as |childdoc.pdf|.

%%%%%%%%%%%%%%%%%%%%%%%%%%%%%%%%%%%%%%%%%%%%%%%%%%%%%%%%%%%%%%%%%%%%%%%%%%%%%%%%
\subsection{Files and Installation}

The package consists of the files:
%
\begin{center}
\begin{tabular}{ll}
    |README.txt|   & readme file \\
    |childdoc.ins| & installation file \\
    |childdoc.dtx| & source file \\
    |childdoc.def| & definition file \\
    |cdocsamp.tex| & sample main file \\
    |cdocsch1.tex| & sample include file \\
    |cdocsch2.tex| & sample include file \\
    |cdocspt3.tex| & sample part file \\
    |cdocspt4.tex| & sample part file \\
    |cdocsdrf.tex| & sample redirection file \\
    |cdocsfn1.tex| & sample redirection file \\
    |cdocsfn2.tex| & sample redirection file \\
    |childdoc.pdf| & manual
\end{tabular}
\end{center}
%
The distribution consists of the files
|README.txt|, |childdoc.ins| and |childdoc.dtx|.
%
\begin{itemize}
\item
Run (pdf)\LaTeX{} on |childdoc.dtx|
to compile the manual |childdoc.pdf| (this file).
\item
Run \LaTeX{} on |childdoc.ins| to create the definitions file |childdoc.def|
and the sample |cdocsamp.tex| with include files
|cdocsch1.tex|, |cdocsch2.tex|, |cdocspt3.tex|, |cdocspt4.tex|,
|cdocsdrf.tex|, |cdocsfn1.tex|, |cdocsfn2.tex|.
Then copy the file |childdoc.def| to an appropriate directory of your \LaTeX{}
distribution, e.g.\ \textit{texmf-root}|/tex/latex/childdoc|.
\end{itemize}

%%%%%%%%%%%%%%%%%%%%%%%%%%%%%%%%%%%%%%%%%%%%%%%%%%%%%%%%%%%%%%%%%%%%%%%%%%%%%%%%
\subsection{Related CTAN Packages}

There are several other packages which offer a similar functionality:
%
\begin{itemize}
\item
The packages
\href{http://ctan.org/pkg/docmute}{\textsf{docmute}},
\href{http://ctan.org/pkg/includex}{\textsf{includex}} and
\href{http://ctan.org/pkg/standalone}{\textsf{standalone}}
provide commands to include only the document body of
a child file thus allowing both files to be compiled individually.
\item
The packages \href{http://ctan.org/pkg/subdocs}{\textsf{subdocs}}
and \href{http://ctan.org/pkg/subfiles}{\textsf{subfiles}}
provide structures in which the main and child documents can be
encapsulated and allowing them to be compiled individually.
The inclusion mechanism is different from the conventional |\include|.
\item
The package \href{http://ctan.org/pkg/combine}{\textsf{combine}}
is an elaborate solution to combine several documents into one.
\end{itemize}
%
See also the CTAN topic \href{http://ctan.org/topic/subdocs}{\textsf{subdocs}}
for further related packages.
The present package differs from the above solutions in that
a document structure constructed with the conventional |\include| mechanism
just needs two extra commands at the top of every file
such that all constituent files can be compiled individually.

%%%%%%%%%%%%%%%%%%%%%%%%%%%%%%%%%%%%%%%%%%%%%%%%%%%%%%%%%%%%%%%%%%%%%%%%%%%%%%%%
%\subsection{Feature Suggestions}
%
%The following is a list of features which may be useful for future
%versions of this package:
%%
%\begin{itemize}
%\item
%\ldots
%\end{itemize}

%%%%%%%%%%%%%%%%%%%%%%%%%%%%%%%%%%%%%%%%%%%%%%%%%%%%%%%%%%%%%%%%%%%%%%%%%%%%%%%%
\subsection{Revision History}

%%%%%%%%%%%%%%%%%%%%%%%%%%%%%%%%%%%%%%%%
\paragraph{v2.0:} 2018/12/30

\begin{itemize}
\item
immediate forward processing
\item
added |\childdocby| mechanism
\item
manual restructured
\end{itemize}

%%%%%%%%%%%%%%%%%%%%%%%%%%%%%%%%%%%%%%%%
\paragraph{v1.6:} 2018/01/17

\begin{itemize}
\item
application for development of include files
\item
corrections to manual
\end{itemize}

%%%%%%%%%%%%%%%%%%%%%%%%%%%%%%%%%%%%%%%%
\paragraph{v1.5:} 2017/05/21

\begin{itemize}
\item
more complete structuring introduced
\item
|\childdocof| introduced
\item
|\childdoc| renamed to |\childdocmain|
\item
|\childredirect| renamed to |\childdocforward| and |\childdocforwardprefix|
and functionality expanded
\end{itemize}

%%%%%%%%%%%%%%%%%%%%%%%%%%%%%%%%%%%%%%%%
\paragraph{v1.0:} 2017/04/27

\begin{itemize}
\item
manual and install package
\item
first version published on CTAN
\end{itemize}

%%%%%%%%%%%%%%%%%%%%%%%%%%%%%%%%%%%%%%%%
\paragraph{v0.6:} 2017/04/26

\begin{itemize}
\item
redirection mechanism added
\end{itemize}

%%%%%%%%%%%%%%%%%%%%%%%%%%%%%%%%%%%%%%%%
\paragraph{v0.5:} 2017/04/26

\begin{itemize}
\item
functionality in definition file
\end{itemize}


%%%%%%%%%%%%%%%%%%%%%%%%%%%%%%%%%%%%%%%%%%%%%%%%%%%%%%%%%%%%%%%%%%%%%%%%%%%%%%%%
%%%%%%%%%%%%%%%%%%%%%%%%%%%%%%%%%%%%%%%%%%%%%%%%%%%%%%%%%%%%%%%%%%%%%%%%%%%%%%%%
%%%%%%%%%%%%%%%%%%%%%%%%%%%%%%%%%%%%%%%%%%%%%%%%%%%%%%%%%%%%%%%%%%%%%%%%%%%%%%%%
\appendix

\settowidth\MacroIndent{\rmfamily\scriptsize 000\ }

 \DocInput{childdoc.dtx}

\end{document}
%</driver>
% \fi
%
% %%%%%%%%%%%%%%%%%%%%%%%%%%%%%%%%%%%%%%%%%%%%%%%%%%%%%%%%%%%%%%%%%%%%%%%%%%%%%%
% %%%%%%%%%%%%%%%%%%%%%%%%%%%%%%%%%%%%%%%%%%%%%%%%%%%%%%%%%%%%%%%%%%%%%%%%%%%%%%
% \section{Sample}
%\iffalse
%<*samplemain>
%\fi
%
% The following presents a sample document
% with two chapters, two parts, a title page,
% a compile flag as well as three forwarding files to set the flag.
% It consists of eight |.tex| files:
% \begin{center}
% \begin{tabular}{ll}
% |cdocsamp.tex|&main file\\
% |cdocsch1.tex|&include file for chapter 1\\
% |cdocsch2.tex|&include file for chapter 2\\
% |cdocspt3.tex|&include file for part 3\\
% |cdocspt4.tex|&include file for part 4\\
% |cdocsdrf.tex|&forwarding file for main file in draft mode\\
% |cdocsfi1.tex|&forwarding file for final version of chapter 1\\
% |cdocsfi2.tex|&forwarding file for final version of chapter 2\\
% \end{tabular}
% \end{center}
% Each of the eight files can be compiled directly by the \LaTeX{} compiler.
%
% %%%%%%%%%%%%%%%%%%%%%%%%%%%%%%%%%%%%%%
% \paragraph{Main File.}
%
% The main file is called |cdocsamp.tex|.
%
% Load the \textsf{childdoc} definitions and
% declare the filename for the main document:
%    \begin{macrocode}
\input{childdoc.def}
\childdocmain{}
%    \end{macrocode}

% Optional override for |\version| flag:
%    \begin{macrocode}
%%\ifchilddoc\else\providecommand{\version}{draft}\fi
%    \end{macrocode}

% Define the default values for the |\version| flag
% (|final| for the main file and |draft| for childs):
%    \begin{macrocode}
\ifchilddoc
\providecommand{\version}{draft}
\else
\providecommand{\version}{final}
\fi
%    \end{macrocode}

% Load the standard document class:
%    \begin{macrocode}
\documentclass[12pt]{article}
%    \end{macrocode}

% Start the document body:
%    \begin{macrocode}
\begin{document}
%    \end{macrocode}

% Declare a title page.
% Print title, part of document being processed and version flag:
%    \begin{macrocode}
\addtocounter{page}{-1}
\begin{center}
{\LARGE\bfseries{}childdoc example\par}
\vspace{1cm}
\ifchilddoc
\ifchilddocmanual part\else chapter\fi:
`\childdocname' of `\childdocjob'\par
\else
main document: `\childdocjob'\par
\fi
version: \version\par
\end{center}
\newpage
%    \end{macrocode}

% Manually include selected file,
% otherwise process as usual:
%    \begin{macrocode}
\ifchilddocmanual
\section*{part `\childdocname'}
\input{\childdocname}
\else
%    \end{macrocode}

% Include the two chapters:
%    \begin{macrocode}
\include{cdocsch1}
\include{cdocsch2}
%    \end{macrocode}

% Include the two parts unless only chapters should be displayed:
%    \begin{macrocode}
\ifchilddoc\else
\section{part three}
\input{cdocspt3}
\section{part four}
\input{cdocspt4}
\fi
%    \end{macrocode}

% Process as usual until here:
%    \begin{macrocode}
\fi
%    \end{macrocode}

% End of document body:
%    \begin{macrocode}
\end{document}
%    \end{macrocode}
%\iffalse
%</samplemain>
%\fi
%
% %%%%%%%%%%%%%%%%%%%%%%%%%%%%%%%%%%%%%%
% \paragraph{Chapter Include Files.}
%
% The include files are called |cdocsch1.tex| and |cdocsch2.tex|.
%
%\iffalse
%<*samplechap1|samplechap2>
%\fi

% Optional override for |\version| flag:
%    \begin{macrocode}
%%\providecommand{\version}{final}
%    \end{macrocode}

% Include the main document:
%    \begin{macrocode}
\input{childdoc.def}
\childdocof{cdocsamp}
%    \end{macrocode}

%\iffalse
%</samplechap1|samplechap2>
%\fi
%
%\iffalse
%<*samplechap1>
%\fi
% Some text for chapter 1:
%    \begin{macrocode}
\section{one}
some text in chapter one
%    \end{macrocode}

%\iffalse
%</samplechap1>
%\fi
% Some text for chapter 2:
%\iffalse
%<*samplechap2>
%\fi
%    \begin{macrocode}
\section{two}
more text in chapter two
%    \end{macrocode}

%\iffalse
%</samplechap2>
%\fi
%
% %%%%%%%%%%%%%%%%%%%%%%%%%%%%%%%%%%%%%%
% \paragraph{Part Include Files.}
%
% The include files are called |cdocspt3.tex| and |cdocspt4.tex|.
%
%\iffalse
%<*samplepart3|samplepart4>
%\fi

% Optional override for |\version| flag:
%    \begin{macrocode}
%%\providecommand{\version}{final}
%    \end{macrocode}

% Include the main document:
%    \begin{macrocode}
\input{childdoc.def}
\childdocby{cdocsamp}
%    \end{macrocode}

%\iffalse
%</samplepart3|samplepart4>
%\fi
%
%\iffalse
%<*samplepart3>
%\fi
% Some text for part 3:
%    \begin{macrocode}
some text in part three
%    \end{macrocode}

%\iffalse
%</samplepart3>
%\fi
% Some text for part 4:
%\iffalse
%<*samplepart4>
%\fi
%    \begin{macrocode}
more text in part four
%    \end{macrocode}

%\iffalse
%</samplepart4>
%\fi
%
% %%%%%%%%%%%%%%%%%%%%%%%%%%%%%%%%%%%%%%
% \paragraph{Forwarding for a Complete Draft.}
%
% The following forwarding file |cdocsdrf.tex|
% compiles the main document in draft mode:
%\iffalse
%<*sampledraft>
%\fi
%    \begin{macrocode}
\def\version{draft}
\input{childdoc.def}
\childdocforward{cdocsamp}
%    \end{macrocode}

%\iffalse
%</sampledraft>
%\fi
%
% %%%%%%%%%%%%%%%%%%%%%%%%%%%%%%%%%%%%%%
% \paragraph{Forwarding for Final Version of the Chapters.}
%
% The following forwarding files |cdocsfn1.tex| and |cdocsfn2.tex|
% (with identical content)
% compile the final versions of the child documents
% |cdocsch1.tex| and |cdocsch2.tex|, respectively:
%\iffalse
%<*samplefinal>
%\fi
%    \begin{macrocode}
\def\version{final}
\input{childdoc.def}
\childdocforwardprefix[cdocsamp]{cdocsfn}{cdocsch}
%    \end{macrocode}

%\iffalse
%</samplefinal>
%\fi
%
% %%%%%%%%%%%%%%%%%%%%%%%%%%%%%%%%%%%%%%
% \paragraph{Command Line Processing.}
%
% The following three command lines generate the output files
% |cdocscld|, |cdocscl1| and |cdocscl2|
% which should be identical to
% |cdocsdrf|, |cdocsch1| and |cdocsfn2|, respectively:
% \begin{center}
% \begin{tabular}{l}
% |latex -jobname cdocscld \|\\
% |  "\def\version{draft}\input{childdoc.def}\childdocforward{cdocsamp}"|\\
% |latex -jobname cdocscl1 \|\\
% |  "\input{childdoc.def}\childdocforward[cdocsamp]{cdocsch1}"|\\
% |latex -jobname cdocscl2 \|\\
% |  "\def\version{final}\input{childdoc.def}\childdocforward{cdocsch2}"|
% \end{tabular}
% \end{center}
% Note that the trailing backslash on each first line
% merely continues the input to the second line
% (for convenient cut ant paste).
% Furthermore, the command |latex| can be replaced by any
% of its alternative versions such as |pdflatex|.
%
% %%%%%%%%%%%%%%%%%%%%%%%%%%%%%%%%%%%%%%%%%%%%%%%%%%%%%%%%%%%%%%%%%%%%%%%%%%%%%%
% %%%%%%%%%%%%%%%%%%%%%%%%%%%%%%%%%%%%%%%%%%%%%%%%%%%%%%%%%%%%%%%%%%%%%%%%%%%%%%
% \section{Implementation}
%\iffalse
%<*package>
%\fi
%
% This section describes the definitions file |childdoc.def|.

% The definitions cannot be loaded using |\usepackage| or |\RequirePackage|
% which has a mechanism to prevent loading a style file more than once.
% When loading the definitions by means of |\input|
% multiple instances have to be prevented manually:
%\iffalse
%This code needs to be before the `\ProvidesFile' directive
%which is defined at the beginning of this file.
%Therefore it is also placed there and commented out here.
%</package>
%<*discard>
%\fi
%    \begin{macrocode}
\ifdefined\childdocmain\endinput\fi
%    \end{macrocode}
%\iffalse
%</discard>
%<*package>
%\fi
%
% \macro{\ifchilddoc}
% \macro{\ifchilddocmanual}
% The conditional |\ifchilddoc| tells whether a
% child (true) or main (false) document is being compiled.
% The conditional |\ifchilddocmanual| tells whether
% the |\includeonly| mechanism is used (false) or
% the selection of child files must be performed manually (true).
% The definitions initialise to false:
%    \begin{macrocode}
\newif\ifchilddoc
\newif\ifchilddocmanual
%    \end{macrocode}

% \macro{\childdocname}
% \macro{\childdocjob}
% The macro |\childdocname| stores the name of the main document
% to be compiled. The macro |\childdocjob| stores the name of
% the document on which the \LaTeX{} compiler was originally invoked.
% The content of |\jobname| cannot be compared
% to filenames specified in the source due to different catcodes.
% The following code rescans |\jobname|, stores the result
% in |\childdocname| and saves a copy in |\childdocjob|:
%    \begin{macrocode}
\edef\childdocname{\scantokens\expandafter{\jobname\noexpand}}
\let\childdocjob\childdocname
%    \end{macrocode}

% \macro{\childdocdisable}
% The macro |\childdocdisable| prevents the main file
% from being processed more than once.
% At this stage, the main document command |\childdocmain|
% is assumed to be called once again where it should do nothing.
% Any subsequent call to it should prevent
% a secondary processing of the main document
% It overwrites the forwarding commands
% |\childdocof| and |\childdocforward|
% with empty macros to prevent further inclusions of the main document:
%    \begin{macrocode}
\newcommand{\childdocdisable}
{
  \renewcommand{\childdocmain}[1]{\renewcommand{\childdocmain}[1]{\endinput}}
  \renewcommand{\childdocof}[1]{}
  \renewcommand{\childdocby}[2][]{}
  \renewcommand{\childdocforward}[2][]{}
  \renewcommand{\childdocdisable}{}
}
%    \end{macrocode}

% \macro{\childdocmain}
% The macro |\childdocmain| is to be called at the top of the main file
% with nothing or the main filename (without extension) as argument.
% First, it breaks loops.
% If the argument is not empty and does not match |\childdocname|
% (which is set by the first inclusion of |childdoc.def|),
% |\ifchilddoc| is set to true, |\includeonly| is applied to the child file
% and |\jobname| is set to the main file
% (for proper handling of |.aux| files):
%    \begin{macrocode}
\newcommand{\childdocmain}[1]
{
  \childdocdisable\childdocmain{}
  \if?#1?\else
    \begingroup
      \def\childdoctmp{#1}
      \ifx\childdoctmp\childdocname
        \def\childdoctmp{}
      \else
        \def\childdoctmp
        {
          \childdoctrue
          \includeonly{\childdocname}
          \def\childdocjob{#1}
          \def\jobname{#1}
        }
      \fi
      \expandafter
    \endgroup
    \childdoctmp
  \fi
}
%    \end{macrocode}

% \macro{\childdocof}
% The command |\childdocof| redirects
% compilation to the main file |#1|.
%    \begin{macrocode}
\newcommand{\childdocof}[1]
{
  \childdocdisable
  \childdoctrue
  \includeonly{\childdocname}
  \def\jobname{#1}
  \def\childdocjob{#1}
  \input{#1}
}
%    \end{macrocode}

% \macro{\childdocby}
% The command |\childdocby| ....
%    \begin{macrocode}
\newcommand{\childdocby}[2][]
{
  \childdocdisable
  \childdoctrue
  \childdocmanualtrue
  \if?#1?\else
    \def\jobname{#2}
  \fi
  \def\childdocjob{#2}
  \input{#2}
  \endinput
}
%    \end{macrocode}

% \macro{\childdocforward}
% The command |\childdocforward| redirects
% compilation to the main file or
% (if the optional argument is given) a child file.
% Parameters are set as if the main file
% or a child file starting with |\childdocof| was compiled.
% Then compilation is handed over to the main file:
%    \begin{macrocode}
\newcommand{\childdocforward}[2][]
{
  \begingroup
    \if?#1?
      \def\childdoctmp
      {
        \def\childdocname{#2}
        \def\childdocjob{#2}
        \def\jobname{#2}
        \input{#2}
        \endinput
      }
    \else
      \def\childdoctmp
      {
        \childdocdisable
        \def\childdocname{#2}
        \childdoctrue
        \includeonly{#2}
        \def\childdocjob{#1}
        \def\jobname{#1}
        \input{#1}
        \endinput
      }
    \fi
    \expandafter
  \endgroup
  \childdoctmp
}
%    \end{macrocode}

% \macro{\childdocforwardprefix}
% The command |\childdocforwardprefix| redirects
% compilation to the main or a child file by means of a pattern.
% The prefix |#1| in the current filename is replaced by |#2|
% and the suffix of the current filename is kept
% (it is assumed that the filename does not contain the substring `|~~~|'
% which is used as a delimiter).
% Compilation is handed over to the new file by |\childdocforward|:
%    \begin{macrocode}
\newcommand{\childdocforwardprefix}[3][]
{
  \begingroup
    \def\childdocextract #2##1~~~{\def\childdoctmp{\childdocforward[#1]{#3##1}}}
    \expandafter\childdocextract\childdocname~~~
    \expandafter
  \endgroup
  \childdoctmp
}
%    \end{macrocode}

% \macro{\childdoc}
% The deprecated macro |\childdoc| is a legacy version of |\childdocmain|:
%    \begin{macrocode}
\newcommand{\childdoc}{\childdocmain}
%    \end{macrocode}

% \macro{\childdocredirect}
% The deprecated macro |\childdocredirect| is a legacy version
% of |\childdocforward| and |\childdocforwardprefix|:
%    \begin{macrocode}
\newcommand{\childdocredirect}[2][]
{
  \begingroup
    \if?#1?
      \def\childdoctmp{\childdocforward{#2}}
    \else
      \def\childdoctmp{\childdocforwardprefix{#1}{#2}}
    \fi
    \expandafter
  \endgroup
  \childdoctmp
}
%    \end{macrocode}

%\iffalse
%</package>
%\fi
%
\endinput
|\\
|\childdocforward{|\textit{main}|}|\\
\end{tabular}
\end{center}
%
or alternatively with:
%
\begin{center}
\begin{tabular}{l}
|% \iffalse
%
% childdoc.dtx Copyright (C) 2017-2018 Niklas Beisert
%
% This work may be distributed and/or modified under the
% conditions of the LaTeX Project Public License, either version 1.3
% of this license or (at your option) any later version.
% The latest version of this license is in
%   http://www.latex-project.org/lppl.txt
% and version 1.3 or later is part of all distributions of LaTeX
% version 2005/12/01 or later.
%
% This work has the LPPL maintenance status `maintained'.
%
% The Current Maintainer of this work is Niklas Beisert.
%
% This work consists of the files childdoc.dtx and childdoc.ins
% and the derived files childdoc.def and cdocsamp.tex with
% cdocsch1.tex, cdocsch2.tex, cdocsdrf.tex, cdocsfn1.tex, cdocsfn2.tex.
%
%<package>\ifdefined\childdocmain\endinput\fi
%<package>\ProvidesFile{childdoc.def}[2018/12/30 v2.0 child document driver]
%<samplemain>\ProvidesFile{cdocsamp.tex}[2018/12/30 v2.0 sample for childdoc]
%<*driver>
%\ProvidesFile{childdoc.drv}[2018/12/30 v2.0 childdoc reference manual file]
\PassOptionsToClass{10pt,a4paper}{article}
\documentclass{ltxdoc}

\usepackage[margin=35mm]{geometry}
\usepackage{hyperref}
\usepackage{hyperxmp}
\usepackage[usenames]{color}

\hypersetup{colorlinks=true}
\hypersetup{pdfstartview=FitH}
\hypersetup{pdfpagemode=UseNone}
\hypersetup{pdfsource={}}
\hypersetup{pdflang={en-UK}}
\hypersetup{pdfcopyright={Copyright 2017-2018 Niklas Beisert.
  This work may be distributed and/or modified under the
  conditions of the LaTeX Project Public License, either version 1.3
  of this license or (at your option) any later version.}}
\hypersetup{pdflicenseurl={http://www.latex-project.org/lppl.txt}}
\hypersetup{pdfcontactaddress={ETH Zurich, ITP, HIT K,
  Wolfgang-Pauli-Strasse 27}}
\hypersetup{pdfcontactpostcode={8093}}
\hypersetup{pdfcontactcity={Zurich}}
\hypersetup{pdfcontactcountry={Switzerland}}
\hypersetup{pdfcontactemail={nbeisert@itp.phys.ethz.ch}}
\hypersetup{pdfcontacturl={http://people.phys.ethz.ch/\xmptilde nbeisert/}}

\newcommand{\secref}[1]{\hyperref[#1]{section \ref*{#1}}}

\parskip1ex
\parindent0pt
\let\olditemize\itemize
\def\itemize{\olditemize\parskip0pt}

\begin{document}

\title{The \textsf{childdoc} Package}
\hypersetup{pdftitle={The childdoc Package}}
\author{Niklas Beisert\\[2ex]
  Institut f\"ur Theoretische Physik\\
  Eidgen\"ossische Technische Hochschule Z\"urich\\
  Wolfgang-Pauli-Strasse 27, 8093 Z\"urich, Switzerland\\[1ex]
  \href{mailto:nbeisert@itp.phys.ethz.ch}
  {\texttt{nbeisert@itp.phys.ethz.ch}}}
\hypersetup{pdfauthor={Niklas Beisert}}
\hypersetup{pdfsubject={Manual for the LaTeX2e Package childdoc}}
\date{30 December 2018, \textsf{v2.0}}
\maketitle

\begin{abstract}\noindent
\textsf{childdoc} is a \LaTeXe{} package
that enables the direct compilation
of document sections included by |\include|
to individual files.
\end{abstract}

\begingroup
\parskip0ex
\tableofcontents
\endgroup

%%%%%%%%%%%%%%%%%%%%%%%%%%%%%%%%%%%%%%%%%%%%%%%%%%%%%%%%%%%%%%%%%%%%%%%%%%%%%%%%
%%%%%%%%%%%%%%%%%%%%%%%%%%%%%%%%%%%%%%%%%%%%%%%%%%%%%%%%%%%%%%%%%%%%%%%%%%%%%%%%
\section{Introduction}

\LaTeX{} provides a mechanism to structure a large document (such as a book)
into a main file and several child files (containing the chapters)
using the |\include| command.
This mechanism is beneficial for documents
which span hundreds of pages in order to
make the source file(s) more manageable.
Moreover, compilation can be restricted to
selected child files by means of the |\includeonly| command.
The latter feature can be used to reduce the compilation time while editing
(this was significantly more useful in the earlier days of \LaTeX{})
or to generate a smaller document which is easier to navigate.
Another application of |\includeonly| is to generate
documents consisting of selected parts of the complete document.

However, there are a few drawbacks of the plain |\include| mechanism:
\begin{itemize}
\item
The child files cannot be compiled on their own,
they can only be compiled via the main file.
A naive editing environment
(such as a text editor with an option
to have the current file processed by \LaTeX)
may require one to switch to the main file before compiling;
attempting to compile the child file produces errors.
\item
The main file must be modified (each time)
to adjust the |\includeonly| command
to the present needs. This easily leaves the main file in a messy state.
\item
The generated document will always carry the filename
of the main document. This is inconvenient if
several child files are to be compiled and
to be kept for distribution.
\end{itemize}

The present package provides a simple interface
to make child files individually compilable by \LaTeX{}.
Compiling a child file then has the same effect as compiling
the main file with an |\includeonly| command
to select the appropriate child.
Moreover the generated document will carry the name of the child
rather than the main file.
This resolves all three above issues.

This feature is meant to make the editing of books,
thesis documents and lecture notes somewhat more convenient.
However, the package can also be used efficiently for
composing a series of documents (such as exercise sheets)
which are typically distributed individually.
It then assists the author in generating the individual documents
(potentially in different versions)
as well as a document containing the collected series.
Another application is in developing style files
or other kinds of included material
where compilation of the style file could redirect
to a sample or test file.

%%%%%%%%%%%%%%%%%%%%%%%%%%%%%%%%%%%%%%%%%%%%%%%%%%%%%%%%%%%%%%%%%%%%%%%%%%%%%%%%
%%%%%%%%%%%%%%%%%%%%%%%%%%%%%%%%%%%%%%%%%%%%%%%%%%%%%%%%%%%%%%%%%%%%%%%%%%%%%%%%
\section{Usage}

First of all, the package \textsf{childdoc} is \emph{not} a standard
\LaTeXe{} |.sty| style file! Therefore it needs to be invoked in
a non-standard way.

%%%%%%%%%%%%%%%%%%%%%%%%%%%%%%%%%%%%%%%%%%%%%%%%%%%%%%%%%%%%%%%%%%%%%%%%%%%%%%%%
\subsection{Included Files}
\label{sec:include}

%%%%%%%%%%%%%%%%%%%%%%%%%%%%%%%%%%%%%%%%
\DescribeMacro{\childdocmain}
To use the package, add the commands
\begin{center}
\begin{tabular}{l}
|\input{childdoc.def}|\\
|\childdocmain{}|\\
\end{tabular}
\end{center}
at the very top of the main \LaTeX{} file,
in particular \emph{before} the |\documentclass| statement!
The argument of |\childdocmain| should be left empty
(but it must be present).

%%%%%%%%%%%%%%%%%%%%%%%%%%%%%%%%%%%%%%%%
\DescribeMacro{\childdocof}
Furthermore, add the commands
\begin{center}
\begin{tabular}{l}
|\input{childdoc.def}|\\
|\childdocof{|\textit{main}|}|\\
\end{tabular}
\end{center}
at the top of every child file \textit{child}
which is included by |\include{|\textit{child}|}|
from within the main file
(or at least for those files to be compiled individually).
The argument \textit{main} must be the filename of the main file.

There are a couple of
considerations in setting up the main and child documents:

%%%%%%%%%%%%%%%%%%%%%%%%%%%%%%%%%%%%%%%%
\paragraph{Restrictions.}

Please note the following restrictions:
\begin{itemize}
\item
|\childdocmain| must be called with one argument \textit{main}
to ensure compatibility with earlier version of the package.
It must either be empty (|\childdocmain{}|)
or precisely match the filename of the main file in which it is specified.
See \secref{sec:detection} for further information.
\item
The filename \textit{main} must be specified without the |.tex| extension.
\item
The filename \textit{main} is case sensitive
(even in case-insensitive file systems)
due to internal string comparison.
\item
The argument \textit{main} should be fully expanded, it cannot be a macro.
\item
Subdirectories and special characters should be avoided in filenames.
\item
The command |\childdocmain{|\textit{main}|}| must be followed by a whitespace.
It should not be followed immediately by another command
or by a comment mark `|%|'.
This is because the \TeX{} parser reads the token immediately following
the argument of |\childdocmain| and puts it
at the beginning of every child section;
however, a white\-space is ignored.
\end{itemize}

%%%%%%%%%%%%%%%%%%%%%%%%%%%%%%%%%%%%%%%%
\paragraph{Content of Main File.}

It is advisable to place all content in the child files included by |\include|.
Any output contained in the main file will appear in all child documents
unless suppressed manually;
it cannot be suppressed automatically by the |\includeonly| directive
and thus should normally be avoided.
A method to include some content in the main file
by means of conditional processing is described in \secref{sec:conditional}.

%%%%%%%%%%%%%%%%%%%%%%%%%%%%%%%%%%%%%%%%
\paragraph{Page Numbering.}

When only a part of the document is compiled,
the appropriate numbering of pages
(as well as other status parameters)
is determined from the |.aux| files.
The latter contain information from previous passes.
However this information needs to propagate through
all intermediate child documents.
Therefore the page numbering in child documents may well
be inconsistent until the complete document is compiled at least once.

A useful (if unconventional) way to always ensure a consistent
page numbering is to restart the numbering in each child document
and denote the pages by `\textit{child}|.|\textit{page}'
where \textit{child} represents the chapter/section number of the child file.
This can be achieved by the command
|\numberwithin{page}{|\textit{child}|}|
of the \textsf{amsmath} package
where \textit{child} can be |chapter| or |section|
depending on the chosen structuring.
Alternatively, one can modify the macro |\thepage| appropriately
and reset the counter |page| at the start of each child file.

%%%%%%%%%%%%%%%%%%%%%%%%%%%%%%%%%%%%%%%%%%%%%%%%%%%%%%%%%%%%%%%%%%%%%%%%%%%%%%%%
\subsection{Conditional Processing}
\label{sec:conditional}

The package provides a mechanism to compile different versions
of a document. To customise the versions further some conditional processing
can come in handy to distinguish which version is being compiled.
The package provides two macros to describe the compilation context:

%%%%%%%%%%%%%%%%%%%%%%%%%%%%%%%%%%%%%%%%
\DescribeMacro{\ifchilddoc}
The conditional |\ifchilddoc| distinguishes between the compilation of
child documents and the main document:
%
\begin{center}
|\ifchilddoc |\textit{child-code}| |[|\||else |\textit{main-code}]| \||fi|
\end{center}

%%%%%%%%%%%%%%%%%%%%%%%%%%%%%%%%%%%%%%%%
\DescribeMacro{\childdocname}
\DescribeMacro{\childdocjob}
The macro |\childdocname| contains the filename (without extension)
of the main or child file being processed.
Note that |\childdocjob| will always contain the name of the main file.

%%%%%%%%%%%%%%%%%%%%%%%%%%%%%%%%%%%%%%%%
\paragraph{Title Page.}

Conditional processing can be used to include a title or banner page
in the main document when proper precautions are taken.
Importantly, the code in the main file should ensure that the page counter
(as well as other status parameters which are stored in the |.aux| files)
takes the same value after the conditional processing.
Otherwise the page numbers may take divergent values
depending on which part is compiled.

For example, a title page could be declared by:
%
\begin{center}
\begin{tabular}{l}
|\ifchilddoc\||else|\\
|\addtocounter{page}{-1}|\\
\textit{code for title page}\\
|\newpage|\\
|\||fi|
\end{tabular}
\end{center}
%
A banner page for the child documents can be generated by:
%
\begin{center}
\begin{tabular}{l}
|\ifchilddoc|\\
|\addtocounter{page}{-1}|\\
\textit{code for banner page}\\
|\newpage|\\
|\||fi|
\end{tabular}
\end{center}
%
Here one could write a message such as:
\begin{center}
|This is the part \childdocname{} of \childdocjob{}.|
\end{center}

%%%%%%%%%%%%%%%%%%%%%%%%%%%%%%%%%%%%%%%%%%%%%%%%%%%%%%%%%%%%%%%%%%%%%%%%%%%%%%%%
\subsection{Flags}
\label{sec:flags}

The package makes it easy to generate different versions
of the main or child documents.
To this end compilation flags can be defined
and assigned different default values.
They will be particularly useful in conjunction
with the forwarding mechanism described in \secref{sec:forward}.

For example, it may be useful to have a flag |\version|
which can be set to |draft| or |final|.
The document source will contain some conditional code
depending on the value of |\version|.
Suppose further, the flag should default to |final| for the main file
and to |draft| for child files
which is a natural assignment for editing the document.
This is achieved by placing the following code
in the preamble of the main document
(below the |\childdocmain| directive):
%
\begin{center}
\begin{tabular}{l}
|\ifchilddoc|\\
|\providecommand{\version}{draft}|\\
|\||else|\\
|\providecommand{\version}{final}|\\
|\||fi|
\end{tabular}
\end{center}
%
The definition by |\providecommand| makes sure
that previous definitions are not overwritten.
Further statements |\providecommand{\version}{...}|
can thus be added before the above code to override it.

For the main file, one might add a line
(between |\childdocmain| and the above block)
%
\begin{center}
|%\ifchilddoc\||else\providecommand{\version}{draft}\||fi|
\end{center}
%
which can be uncommented to produce a draft version.
Likewise one can add a line to the very top of a child file
(above the |\childdocof{|\textit{main}|}| directive)
%
\begin{center}
|%\providecommand{\version}{final}|
\end{center}
%
which can be uncommented to produce the final version of this child document.

%%%%%%%%%%%%%%%%%%%%%%%%%%%%%%%%%%%%%%%%%%%%%%%%%%%%%%%%%%%%%%%%%%%%%%%%%%%%%%%%
\subsection{Forwarding}
\label{sec:forward}

Different versions of the main or child documents
using compilation flags as described in \secref{sec:flags}
can be (permanently) stored in different files
for convenient compilation, viewing and distribution.
To this end, the package defines a command
to pass on compilation to a different file:

%%%%%%%%%%%%%%%%%%%%%%%%%%%%%%%%%%%%%%%%
\DescribeMacro{\childdocforward}
The command |\childdocforward| redirects processing to
another source file:
%
\begin{center}
\begin{tabular}{l}
|\input{childdoc.def}|\\
|\childdocforward[|\textit{main}|]{|\textit{dest}|}|\\
\end{tabular}
\end{center}
%
The argument \textit{dest} is the destination file
(without extension).
It should be the main file or one of the child files.
Note that further \textsf{childdoc} directives
such as |\childdocof| and |\childdocforward|
in the indicated file will be processed in this form.
The optional argument \textit{main}
passes on directly to the main file \textit{main}
while pretending to compile the child \textit{dest}.
This form behaves as if \textit{dest}
issues |\childdocof{|\textit{main}|}| right away,
and no further \textsf{childdoc} directives will be processed.

%%%%%%%%%%%%%%%%%%%%%%%%%%%%%%%%%%%%%%%%
\DescribeMacro{\...prefix}
In the alternative form |\childdocforwardprefix|,
%
\begin{center}
\begin{tabular}{l}
|\input{childdoc.def}|\\
|\childdocforwardprefix[|\textit{main}|]{|\textit{prefix}|}{|\textit{dest}|}|
\end{tabular}
\end{center}
%
the destination file is determined by a pattern
depending on the current file:
To make this work, the current file must be called
`{\textit{prefix}\hspace{0.2em}\textit{suffix}}'
with \textit{prefix} matching precisely the argument.
Processing is then passed on to the file
`{\textit{dest}\hspace{0.2em}\textit{suffix}}'.
Surely, the same effect is achieved by
directly specifying the
argument `{\textit{dest}\hspace{0.2em}\textit{suffix}}'
in the first form.
However, that requires to set up a different file
for each child. With the alternative form of the command
all these files can have exactly the same content
which simplifies setting them up and maintaining them.

For example, the following file |draft.tex|
with a compilation flag |\version| as described in \secref{sec:flags}
compiles the main document as a draft:
%
\begin{center}
\begin{tabular}{l}
|\def\version{draft}|\\
|\input{childdoc.def}|\\
|\childdocforward{|\textit{main}|}|
\end{tabular}
\end{center}
%
Likewise, the following files |final|\textit{nn}|.tex|
compile the final version of the child document
|child|\textit{nn}|.tex|:
%
\begin{center}
\begin{tabular}{l}
|\def\version{final}|\\
|\input{childdoc.def}|\\
|\childdocforwardprefix{final}{child}|
\end{tabular}
\end{center}
%

Note that when several versions of a main file and/or of each child file
are to be generated, it may be convenient to set up a |Makefile| or
shell script to automatise the process.

%%%%%%%%%%%%%%%%%%%%%%%%%%%%%%%%%%%%%%%%%%%%%%%%%%%%%%%%%%%%%%%%%%%%%%%%%%%%%%%%
\subsection{Command Line Processing}
\label{sec:commandline}

The effect of redirection files can also be achieved by invoking
the \LaTeX{} compiler with a more elaborate command line.
Most conveniently this should be done as part
of a shell script or a |Makefile|.

When using \textsf{childdoc} in the main file, the following
command lines effectively perform a redirection
(note that depending on the shell being used,
backslashes may have to be doubled: `|\|' $\to$ `|\\|'):
%
\begin{center}
|... -jobname "|\textit{target}|" |\\|"|[\textit{flags}]%
|\input{childdoc.def}\childdocforward[|\textit{main}|]{|\textit{dest}|}"|
\end{center}
%
Here \textit{target} is the name of the output file,
\textit{main} is the name of the main file
and \textit{dest} is the name of the main or child file to be processed
(all filenames without extensions).
The optional argument \textit{main} can be omitted
if \textit{main} matches \textit{dest}.
Optionally, compilation \textit{flags} can be defined via |\def| commands.
This command line makes the \TeX{} engine believe
it is compiling the file \textit{target}
whose content is specified as the latter parameter.
The provided code then forwards the processing to
\textit{main} or \textit{dest} as described in \secref{sec:forward}.

%%%%%%%%%%%%%%%%%%%%%%%%%%%%%%%%%%%%%%%%%%%%%%%%%%%%%%%%%%%%%%%%%%%%%%%%%%%%%%%%
\subsection{Include by Input}
\label{sec:input}

Including child documents by |\include| has some restrictions by design.
Most notably, the content of a child document always occupies
its own set of pages; pages cannot be shared between child documents.
Usually, this behaviour makes perfect sense
because each child document contain an essential part of the document.
However, in some situations it may be desirable to compose
a document from a collection of parts
without having mandatory page breaks between then.
For this case, the package
provides a mechanism to include parts
by |\input| which can also be processed individually.
However, by construction this mechanism
requires manual handling of the content to be output.

%%%%%%%%%%%%%%%%%%%%%%%%%%%%%%%%%%%%%%%%
\DescribeMacro{\ifchilddocmanual}
The main file should be prepared as usual, see \secref{sec:include}.
However, the document body must make a distinction
between processing of an individual part and of the main document, e.g.:
%
\begin{center}
\begin{tabular}{l}
|\ifchilddocmanual|\\
|\input{\childdocname}|\\
|\||else|\\
\textit{document body with }|\input{|\textit{part}|}|\\
|\||fi|
\end{tabular}
\end{center}
%
The conditional |\ifchilddocmanual| is true whenever
a part to be included by |\input| is being compiled,
and the name of the part is stored in |\childdocname|.

%%%%%%%%%%%%%%%%%%%%%%%%%%%%%%%%%%%%%%%%
\DescribeMacro{\childdocby}
Each part to be included by |\input| should start with:
%
\begin{center}
\begin{tabular}{l}
|\input{childdoc.def}|\\
|\childdocby{|\textit{main}|}|\\
\end{tabular}
\end{center}
%
The directive |\childdocby| is similar to |\childdocof|
described in \secref{sec:include},
but the subsequent selection of content must be done manually.
To that end, both |\ifchilddoc| and |\ifchilddocmanual|
will be true upon processing of a part,
and the name of the part is stored in |\childdocname|.
Note that |\jobname| will be set to the filename of the current part
so that each part receives an individual |.aux| file
that does not interfere with the |.aux| file(s) of the main document.
This behaviour can be altered by the alternative form
|\childdocby[*]{|\textit{main}|}| (with a non-empty optional argument)
which uses the |.aux| file of the main document
by setting |\jobname| to \textit{main}.

%%%%%%%%%%%%%%%%%%%%%%%%%%%%%%%%%%%%%%%%%%%%%%%%%%%%%%%%%%%%%%%%%%%%%%%%%%%%%%%%
\subsection{Driver Development}
\label{sec:driver}

The \textsf{childdoc} mechanism can also be use for the development
of definition files such as \LaTeX{} styles or classes.
This case differs from the above setup with multiple parts
included by |\include| in that no |\includeonly| should be invoked.
This can be achieved by starting the include file
(before |\ProvidesPackage|) with:
%
\begin{center}
\begin{tabular}{l}
|\input{childdoc.def}|\\
|\childdocforward{|\textit{main}|}|\\
\end{tabular}
\end{center}
%
or alternatively with:
%
\begin{center}
\begin{tabular}{l}
|\input{childdoc.def}|\\
|\childdocby{|\textit{main}|}|\\
\end{tabular}
\end{center}
%
Both forms have slightly different effects as described above.
The main file is prepared as usual, see \secref{sec:include}.

%%%%%%%%%%%%%%%%%%%%%%%%%%%%%%%%%%%%%%%%%%%%%%%%%%%%%%%%%%%%%%%%%%%%%%%%%%%%%%%%
\subsection{Legacy Detection}
\label{sec:detection}

The directive |\childdocmain| in the main file can detect
whether the complete document or merely a child is to be compiled
even without using the directive |\childdocof|.
This method is deprecated because it is less robust
and there is no compelling reason to use it;
it is merely provided for backward compatibility
and it may be removed in future versions.

If the detection mechanism is to be used,
it is mandatory to correctly specify
the filename of the main file as the argument of |\childdocmain|:
%
\begin{center}
\begin{tabular}{l}
|\input{childdoc.def}|\\
|\childdocmain{|\textit{main}|}|\\
\end{tabular}
\end{center}
%
If |\jobname| does not match the argument \textit{main} of |\childdocmain|,
it is assumed that |\jobname| points to the child file to be compiled.
When using |\childdocmain| with the main file specified as argument,
it suffices to start a child file
with just |\input{|\textit{main}|}|
without loading of the package and using |\childdocof|.
If instead all processing is done
with the appropriate \textsf{childdoc} directives,
the argument of \textit{main} of |\childdocmain| can be empty.

An alternative version of the command line processing described
in \secref{sec:commandline} using the detection mechanism reads:
%
\begin{center}
|... -jobname "|\textit{target}|" "|[\textit{flags}]%
[|\def\jobname{|\textit{dest}|}|]|\input{|\textit{main}|}"|
\end{center}

%%%%%%%%%%%%%%%%%%%%%%%%%%%%%%%%%%%%%%%%%%%%%%%%%%%%%%%%%%%%%%%%%%%%%%%%%%%%%%%%
\subsection{Manual Code}
\label{sec:manual}

In case one cannot be certain whether the definitions file |childdoc.def|
is installed on the target \TeX{} distribution
and one prefers not to ship it,
it is conceivable to paste a few relevant commands into the sources.

To that end, drop all statements |\input{childdoc.def}|
and perform the replacements as outlined below.
Instead of |\childdocmain{|\textit{main}|}| add the following code
to the top of the main file:
%
\begin{center}
\begin{tabular}{l}
|\||ifdefined\childdocname\endinput\||fi\newif\ifchilddoc|\\
|\edef\childdocname{\scantokens\expandafter{\jobname\noexpand}}|\\
|\def\childdocmain{|\textit{main}|}\||ifx\childdocmain\childdocname\||else|\\
|\childdoctrue\includeonly{\childdocname}\let\jobname\childdocmain\||fi|\\
\end{tabular}
\end{center}
%
Instead of |\childdocof{|\textit{main}|}| just include the main file
at the top of each child file:
%
\begin{center}
|\input{|\textit{main}|}|
\end{center}
%
A simple redirection |\childdocforward{|\textit{dest}|}| is achieved by:
%
\begin{center}
|\def\jobname{|\textit{dest}|}\input{\jobname}|
\end{center}
%
The redirection with prefix
|\childdocforwardprefix[|\textit{prefix}|]{|\textit{dest}|}|
is accomplished by:
%
\begin{center}
\begin{tabular}{l}
|{\edef\jobname{\scantokens\expandafter{\jobname\noexpand}}|\\
|\def\redirectjob |\textit{prefix}|#1~~~{\gdef\jobname{|\textit{dest}|#1}}|\\
|\expandafter\redirectjob\jobname~~~}\input{\jobname}|
\end{tabular}
\end{center}

In an alternative approach,
child documents can be compiled by a specific command line
without additional code or specific definitions:
%
\begin{center}
|... -jobname "|\textit{target}|" "|[\textit{flags}]%
|\includeonly{|\textit{dest}|}\input{|\textit{main}|}"|
\end{center}
%

%%%%%%%%%%%%%%%%%%%%%%%%%%%%%%%%%%%%%%%%%%%%%%%%%%%%%%%%%%%%%%%%%%%%%%%%%%%%%%%%
%%%%%%%%%%%%%%%%%%%%%%%%%%%%%%%%%%%%%%%%%%%%%%%%%%%%%%%%%%%%%%%%%%%%%%%%%%%%%%%%
\section{Information}

%%%%%%%%%%%%%%%%%%%%%%%%%%%%%%%%%%%%%%%%%%%%%%%%%%%%%%%%%%%%%%%%%%%%%%%%%%%%%%%%
\subsection{Copyright}

Copyright \copyright{} 2017--2018 Niklas Beisert

This work may be distributed and/or modified under the
conditions of the \LaTeX{} Project Public License, either version 1.3
of this license or (at your option) any later version.
The latest version of this license is in
  \url{http://www.latex-project.org/lppl.txt}
and version 1.3 or later is part of all distributions of \LaTeX{}
version 2005/12/01 or later.

This work has the LPPL maintenance status `maintained'.

The Current Maintainer of this work is Niklas Beisert.

This work consists of the files |README.txt|, |childdoc.ins| and |childdoc.dtx|
as well as the derived files |childdoc.def|, |cdocsamp.tex|
with |cdocsch1.tex|, |cdocsch2.tex|, |cdocspt3.tex|, |cdocspt4.tex|,
|cdocsdrf.tex|, |cdocsfn1.tex|, |cdocsfn2.tex|
as well as |childdoc.pdf|.

%%%%%%%%%%%%%%%%%%%%%%%%%%%%%%%%%%%%%%%%%%%%%%%%%%%%%%%%%%%%%%%%%%%%%%%%%%%%%%%%
\subsection{Files and Installation}

The package consists of the files:
%
\begin{center}
\begin{tabular}{ll}
    |README.txt|   & readme file \\
    |childdoc.ins| & installation file \\
    |childdoc.dtx| & source file \\
    |childdoc.def| & definition file \\
    |cdocsamp.tex| & sample main file \\
    |cdocsch1.tex| & sample include file \\
    |cdocsch2.tex| & sample include file \\
    |cdocspt3.tex| & sample part file \\
    |cdocspt4.tex| & sample part file \\
    |cdocsdrf.tex| & sample redirection file \\
    |cdocsfn1.tex| & sample redirection file \\
    |cdocsfn2.tex| & sample redirection file \\
    |childdoc.pdf| & manual
\end{tabular}
\end{center}
%
The distribution consists of the files
|README.txt|, |childdoc.ins| and |childdoc.dtx|.
%
\begin{itemize}
\item
Run (pdf)\LaTeX{} on |childdoc.dtx|
to compile the manual |childdoc.pdf| (this file).
\item
Run \LaTeX{} on |childdoc.ins| to create the definitions file |childdoc.def|
and the sample |cdocsamp.tex| with include files
|cdocsch1.tex|, |cdocsch2.tex|, |cdocspt3.tex|, |cdocspt4.tex|,
|cdocsdrf.tex|, |cdocsfn1.tex|, |cdocsfn2.tex|.
Then copy the file |childdoc.def| to an appropriate directory of your \LaTeX{}
distribution, e.g.\ \textit{texmf-root}|/tex/latex/childdoc|.
\end{itemize}

%%%%%%%%%%%%%%%%%%%%%%%%%%%%%%%%%%%%%%%%%%%%%%%%%%%%%%%%%%%%%%%%%%%%%%%%%%%%%%%%
\subsection{Related CTAN Packages}

There are several other packages which offer a similar functionality:
%
\begin{itemize}
\item
The packages
\href{http://ctan.org/pkg/docmute}{\textsf{docmute}},
\href{http://ctan.org/pkg/includex}{\textsf{includex}} and
\href{http://ctan.org/pkg/standalone}{\textsf{standalone}}
provide commands to include only the document body of
a child file thus allowing both files to be compiled individually.
\item
The packages \href{http://ctan.org/pkg/subdocs}{\textsf{subdocs}}
and \href{http://ctan.org/pkg/subfiles}{\textsf{subfiles}}
provide structures in which the main and child documents can be
encapsulated and allowing them to be compiled individually.
The inclusion mechanism is different from the conventional |\include|.
\item
The package \href{http://ctan.org/pkg/combine}{\textsf{combine}}
is an elaborate solution to combine several documents into one.
\end{itemize}
%
See also the CTAN topic \href{http://ctan.org/topic/subdocs}{\textsf{subdocs}}
for further related packages.
The present package differs from the above solutions in that
a document structure constructed with the conventional |\include| mechanism
just needs two extra commands at the top of every file
such that all constituent files can be compiled individually.

%%%%%%%%%%%%%%%%%%%%%%%%%%%%%%%%%%%%%%%%%%%%%%%%%%%%%%%%%%%%%%%%%%%%%%%%%%%%%%%%
%\subsection{Feature Suggestions}
%
%The following is a list of features which may be useful for future
%versions of this package:
%%
%\begin{itemize}
%\item
%\ldots
%\end{itemize}

%%%%%%%%%%%%%%%%%%%%%%%%%%%%%%%%%%%%%%%%%%%%%%%%%%%%%%%%%%%%%%%%%%%%%%%%%%%%%%%%
\subsection{Revision History}

%%%%%%%%%%%%%%%%%%%%%%%%%%%%%%%%%%%%%%%%
\paragraph{v2.0:} 2018/12/30

\begin{itemize}
\item
immediate forward processing
\item
added |\childdocby| mechanism
\item
manual restructured
\end{itemize}

%%%%%%%%%%%%%%%%%%%%%%%%%%%%%%%%%%%%%%%%
\paragraph{v1.6:} 2018/01/17

\begin{itemize}
\item
application for development of include files
\item
corrections to manual
\end{itemize}

%%%%%%%%%%%%%%%%%%%%%%%%%%%%%%%%%%%%%%%%
\paragraph{v1.5:} 2017/05/21

\begin{itemize}
\item
more complete structuring introduced
\item
|\childdocof| introduced
\item
|\childdoc| renamed to |\childdocmain|
\item
|\childredirect| renamed to |\childdocforward| and |\childdocforwardprefix|
and functionality expanded
\end{itemize}

%%%%%%%%%%%%%%%%%%%%%%%%%%%%%%%%%%%%%%%%
\paragraph{v1.0:} 2017/04/27

\begin{itemize}
\item
manual and install package
\item
first version published on CTAN
\end{itemize}

%%%%%%%%%%%%%%%%%%%%%%%%%%%%%%%%%%%%%%%%
\paragraph{v0.6:} 2017/04/26

\begin{itemize}
\item
redirection mechanism added
\end{itemize}

%%%%%%%%%%%%%%%%%%%%%%%%%%%%%%%%%%%%%%%%
\paragraph{v0.5:} 2017/04/26

\begin{itemize}
\item
functionality in definition file
\end{itemize}


%%%%%%%%%%%%%%%%%%%%%%%%%%%%%%%%%%%%%%%%%%%%%%%%%%%%%%%%%%%%%%%%%%%%%%%%%%%%%%%%
%%%%%%%%%%%%%%%%%%%%%%%%%%%%%%%%%%%%%%%%%%%%%%%%%%%%%%%%%%%%%%%%%%%%%%%%%%%%%%%%
%%%%%%%%%%%%%%%%%%%%%%%%%%%%%%%%%%%%%%%%%%%%%%%%%%%%%%%%%%%%%%%%%%%%%%%%%%%%%%%%
\appendix

\settowidth\MacroIndent{\rmfamily\scriptsize 000\ }

 \DocInput{childdoc.dtx}

\end{document}
%</driver>
% \fi
%
% %%%%%%%%%%%%%%%%%%%%%%%%%%%%%%%%%%%%%%%%%%%%%%%%%%%%%%%%%%%%%%%%%%%%%%%%%%%%%%
% %%%%%%%%%%%%%%%%%%%%%%%%%%%%%%%%%%%%%%%%%%%%%%%%%%%%%%%%%%%%%%%%%%%%%%%%%%%%%%
% \section{Sample}
%\iffalse
%<*samplemain>
%\fi
%
% The following presents a sample document
% with two chapters, two parts, a title page,
% a compile flag as well as three forwarding files to set the flag.
% It consists of eight |.tex| files:
% \begin{center}
% \begin{tabular}{ll}
% |cdocsamp.tex|&main file\\
% |cdocsch1.tex|&include file for chapter 1\\
% |cdocsch2.tex|&include file for chapter 2\\
% |cdocspt3.tex|&include file for part 3\\
% |cdocspt4.tex|&include file for part 4\\
% |cdocsdrf.tex|&forwarding file for main file in draft mode\\
% |cdocsfi1.tex|&forwarding file for final version of chapter 1\\
% |cdocsfi2.tex|&forwarding file for final version of chapter 2\\
% \end{tabular}
% \end{center}
% Each of the eight files can be compiled directly by the \LaTeX{} compiler.
%
% %%%%%%%%%%%%%%%%%%%%%%%%%%%%%%%%%%%%%%
% \paragraph{Main File.}
%
% The main file is called |cdocsamp.tex|.
%
% Load the \textsf{childdoc} definitions and
% declare the filename for the main document:
%    \begin{macrocode}
\input{childdoc.def}
\childdocmain{}
%    \end{macrocode}

% Optional override for |\version| flag:
%    \begin{macrocode}
%%\ifchilddoc\else\providecommand{\version}{draft}\fi
%    \end{macrocode}

% Define the default values for the |\version| flag
% (|final| for the main file and |draft| for childs):
%    \begin{macrocode}
\ifchilddoc
\providecommand{\version}{draft}
\else
\providecommand{\version}{final}
\fi
%    \end{macrocode}

% Load the standard document class:
%    \begin{macrocode}
\documentclass[12pt]{article}
%    \end{macrocode}

% Start the document body:
%    \begin{macrocode}
\begin{document}
%    \end{macrocode}

% Declare a title page.
% Print title, part of document being processed and version flag:
%    \begin{macrocode}
\addtocounter{page}{-1}
\begin{center}
{\LARGE\bfseries{}childdoc example\par}
\vspace{1cm}
\ifchilddoc
\ifchilddocmanual part\else chapter\fi:
`\childdocname' of `\childdocjob'\par
\else
main document: `\childdocjob'\par
\fi
version: \version\par
\end{center}
\newpage
%    \end{macrocode}

% Manually include selected file,
% otherwise process as usual:
%    \begin{macrocode}
\ifchilddocmanual
\section*{part `\childdocname'}
\input{\childdocname}
\else
%    \end{macrocode}

% Include the two chapters:
%    \begin{macrocode}
\include{cdocsch1}
\include{cdocsch2}
%    \end{macrocode}

% Include the two parts unless only chapters should be displayed:
%    \begin{macrocode}
\ifchilddoc\else
\section{part three}
\input{cdocspt3}
\section{part four}
\input{cdocspt4}
\fi
%    \end{macrocode}

% Process as usual until here:
%    \begin{macrocode}
\fi
%    \end{macrocode}

% End of document body:
%    \begin{macrocode}
\end{document}
%    \end{macrocode}
%\iffalse
%</samplemain>
%\fi
%
% %%%%%%%%%%%%%%%%%%%%%%%%%%%%%%%%%%%%%%
% \paragraph{Chapter Include Files.}
%
% The include files are called |cdocsch1.tex| and |cdocsch2.tex|.
%
%\iffalse
%<*samplechap1|samplechap2>
%\fi

% Optional override for |\version| flag:
%    \begin{macrocode}
%%\providecommand{\version}{final}
%    \end{macrocode}

% Include the main document:
%    \begin{macrocode}
\input{childdoc.def}
\childdocof{cdocsamp}
%    \end{macrocode}

%\iffalse
%</samplechap1|samplechap2>
%\fi
%
%\iffalse
%<*samplechap1>
%\fi
% Some text for chapter 1:
%    \begin{macrocode}
\section{one}
some text in chapter one
%    \end{macrocode}

%\iffalse
%</samplechap1>
%\fi
% Some text for chapter 2:
%\iffalse
%<*samplechap2>
%\fi
%    \begin{macrocode}
\section{two}
more text in chapter two
%    \end{macrocode}

%\iffalse
%</samplechap2>
%\fi
%
% %%%%%%%%%%%%%%%%%%%%%%%%%%%%%%%%%%%%%%
% \paragraph{Part Include Files.}
%
% The include files are called |cdocspt3.tex| and |cdocspt4.tex|.
%
%\iffalse
%<*samplepart3|samplepart4>
%\fi

% Optional override for |\version| flag:
%    \begin{macrocode}
%%\providecommand{\version}{final}
%    \end{macrocode}

% Include the main document:
%    \begin{macrocode}
\input{childdoc.def}
\childdocby{cdocsamp}
%    \end{macrocode}

%\iffalse
%</samplepart3|samplepart4>
%\fi
%
%\iffalse
%<*samplepart3>
%\fi
% Some text for part 3:
%    \begin{macrocode}
some text in part three
%    \end{macrocode}

%\iffalse
%</samplepart3>
%\fi
% Some text for part 4:
%\iffalse
%<*samplepart4>
%\fi
%    \begin{macrocode}
more text in part four
%    \end{macrocode}

%\iffalse
%</samplepart4>
%\fi
%
% %%%%%%%%%%%%%%%%%%%%%%%%%%%%%%%%%%%%%%
% \paragraph{Forwarding for a Complete Draft.}
%
% The following forwarding file |cdocsdrf.tex|
% compiles the main document in draft mode:
%\iffalse
%<*sampledraft>
%\fi
%    \begin{macrocode}
\def\version{draft}
\input{childdoc.def}
\childdocforward{cdocsamp}
%    \end{macrocode}

%\iffalse
%</sampledraft>
%\fi
%
% %%%%%%%%%%%%%%%%%%%%%%%%%%%%%%%%%%%%%%
% \paragraph{Forwarding for Final Version of the Chapters.}
%
% The following forwarding files |cdocsfn1.tex| and |cdocsfn2.tex|
% (with identical content)
% compile the final versions of the child documents
% |cdocsch1.tex| and |cdocsch2.tex|, respectively:
%\iffalse
%<*samplefinal>
%\fi
%    \begin{macrocode}
\def\version{final}
\input{childdoc.def}
\childdocforwardprefix[cdocsamp]{cdocsfn}{cdocsch}
%    \end{macrocode}

%\iffalse
%</samplefinal>
%\fi
%
% %%%%%%%%%%%%%%%%%%%%%%%%%%%%%%%%%%%%%%
% \paragraph{Command Line Processing.}
%
% The following three command lines generate the output files
% |cdocscld|, |cdocscl1| and |cdocscl2|
% which should be identical to
% |cdocsdrf|, |cdocsch1| and |cdocsfn2|, respectively:
% \begin{center}
% \begin{tabular}{l}
% |latex -jobname cdocscld \|\\
% |  "\def\version{draft}\input{childdoc.def}\childdocforward{cdocsamp}"|\\
% |latex -jobname cdocscl1 \|\\
% |  "\input{childdoc.def}\childdocforward[cdocsamp]{cdocsch1}"|\\
% |latex -jobname cdocscl2 \|\\
% |  "\def\version{final}\input{childdoc.def}\childdocforward{cdocsch2}"|
% \end{tabular}
% \end{center}
% Note that the trailing backslash on each first line
% merely continues the input to the second line
% (for convenient cut ant paste).
% Furthermore, the command |latex| can be replaced by any
% of its alternative versions such as |pdflatex|.
%
% %%%%%%%%%%%%%%%%%%%%%%%%%%%%%%%%%%%%%%%%%%%%%%%%%%%%%%%%%%%%%%%%%%%%%%%%%%%%%%
% %%%%%%%%%%%%%%%%%%%%%%%%%%%%%%%%%%%%%%%%%%%%%%%%%%%%%%%%%%%%%%%%%%%%%%%%%%%%%%
% \section{Implementation}
%\iffalse
%<*package>
%\fi
%
% This section describes the definitions file |childdoc.def|.

% The definitions cannot be loaded using |\usepackage| or |\RequirePackage|
% which has a mechanism to prevent loading a style file more than once.
% When loading the definitions by means of |\input|
% multiple instances have to be prevented manually:
%\iffalse
%This code needs to be before the `\ProvidesFile' directive
%which is defined at the beginning of this file.
%Therefore it is also placed there and commented out here.
%</package>
%<*discard>
%\fi
%    \begin{macrocode}
\ifdefined\childdocmain\endinput\fi
%    \end{macrocode}
%\iffalse
%</discard>
%<*package>
%\fi
%
% \macro{\ifchilddoc}
% \macro{\ifchilddocmanual}
% The conditional |\ifchilddoc| tells whether a
% child (true) or main (false) document is being compiled.
% The conditional |\ifchilddocmanual| tells whether
% the |\includeonly| mechanism is used (false) or
% the selection of child files must be performed manually (true).
% The definitions initialise to false:
%    \begin{macrocode}
\newif\ifchilddoc
\newif\ifchilddocmanual
%    \end{macrocode}

% \macro{\childdocname}
% \macro{\childdocjob}
% The macro |\childdocname| stores the name of the main document
% to be compiled. The macro |\childdocjob| stores the name of
% the document on which the \LaTeX{} compiler was originally invoked.
% The content of |\jobname| cannot be compared
% to filenames specified in the source due to different catcodes.
% The following code rescans |\jobname|, stores the result
% in |\childdocname| and saves a copy in |\childdocjob|:
%    \begin{macrocode}
\edef\childdocname{\scantokens\expandafter{\jobname\noexpand}}
\let\childdocjob\childdocname
%    \end{macrocode}

% \macro{\childdocdisable}
% The macro |\childdocdisable| prevents the main file
% from being processed more than once.
% At this stage, the main document command |\childdocmain|
% is assumed to be called once again where it should do nothing.
% Any subsequent call to it should prevent
% a secondary processing of the main document
% It overwrites the forwarding commands
% |\childdocof| and |\childdocforward|
% with empty macros to prevent further inclusions of the main document:
%    \begin{macrocode}
\newcommand{\childdocdisable}
{
  \renewcommand{\childdocmain}[1]{\renewcommand{\childdocmain}[1]{\endinput}}
  \renewcommand{\childdocof}[1]{}
  \renewcommand{\childdocby}[2][]{}
  \renewcommand{\childdocforward}[2][]{}
  \renewcommand{\childdocdisable}{}
}
%    \end{macrocode}

% \macro{\childdocmain}
% The macro |\childdocmain| is to be called at the top of the main file
% with nothing or the main filename (without extension) as argument.
% First, it breaks loops.
% If the argument is not empty and does not match |\childdocname|
% (which is set by the first inclusion of |childdoc.def|),
% |\ifchilddoc| is set to true, |\includeonly| is applied to the child file
% and |\jobname| is set to the main file
% (for proper handling of |.aux| files):
%    \begin{macrocode}
\newcommand{\childdocmain}[1]
{
  \childdocdisable\childdocmain{}
  \if?#1?\else
    \begingroup
      \def\childdoctmp{#1}
      \ifx\childdoctmp\childdocname
        \def\childdoctmp{}
      \else
        \def\childdoctmp
        {
          \childdoctrue
          \includeonly{\childdocname}
          \def\childdocjob{#1}
          \def\jobname{#1}
        }
      \fi
      \expandafter
    \endgroup
    \childdoctmp
  \fi
}
%    \end{macrocode}

% \macro{\childdocof}
% The command |\childdocof| redirects
% compilation to the main file |#1|.
%    \begin{macrocode}
\newcommand{\childdocof}[1]
{
  \childdocdisable
  \childdoctrue
  \includeonly{\childdocname}
  \def\jobname{#1}
  \def\childdocjob{#1}
  \input{#1}
}
%    \end{macrocode}

% \macro{\childdocby}
% The command |\childdocby| ....
%    \begin{macrocode}
\newcommand{\childdocby}[2][]
{
  \childdocdisable
  \childdoctrue
  \childdocmanualtrue
  \if?#1?\else
    \def\jobname{#2}
  \fi
  \def\childdocjob{#2}
  \input{#2}
  \endinput
}
%    \end{macrocode}

% \macro{\childdocforward}
% The command |\childdocforward| redirects
% compilation to the main file or
% (if the optional argument is given) a child file.
% Parameters are set as if the main file
% or a child file starting with |\childdocof| was compiled.
% Then compilation is handed over to the main file:
%    \begin{macrocode}
\newcommand{\childdocforward}[2][]
{
  \begingroup
    \if?#1?
      \def\childdoctmp
      {
        \def\childdocname{#2}
        \def\childdocjob{#2}
        \def\jobname{#2}
        \input{#2}
        \endinput
      }
    \else
      \def\childdoctmp
      {
        \childdocdisable
        \def\childdocname{#2}
        \childdoctrue
        \includeonly{#2}
        \def\childdocjob{#1}
        \def\jobname{#1}
        \input{#1}
        \endinput
      }
    \fi
    \expandafter
  \endgroup
  \childdoctmp
}
%    \end{macrocode}

% \macro{\childdocforwardprefix}
% The command |\childdocforwardprefix| redirects
% compilation to the main or a child file by means of a pattern.
% The prefix |#1| in the current filename is replaced by |#2|
% and the suffix of the current filename is kept
% (it is assumed that the filename does not contain the substring `|~~~|'
% which is used as a delimiter).
% Compilation is handed over to the new file by |\childdocforward|:
%    \begin{macrocode}
\newcommand{\childdocforwardprefix}[3][]
{
  \begingroup
    \def\childdocextract #2##1~~~{\def\childdoctmp{\childdocforward[#1]{#3##1}}}
    \expandafter\childdocextract\childdocname~~~
    \expandafter
  \endgroup
  \childdoctmp
}
%    \end{macrocode}

% \macro{\childdoc}
% The deprecated macro |\childdoc| is a legacy version of |\childdocmain|:
%    \begin{macrocode}
\newcommand{\childdoc}{\childdocmain}
%    \end{macrocode}

% \macro{\childdocredirect}
% The deprecated macro |\childdocredirect| is a legacy version
% of |\childdocforward| and |\childdocforwardprefix|:
%    \begin{macrocode}
\newcommand{\childdocredirect}[2][]
{
  \begingroup
    \if?#1?
      \def\childdoctmp{\childdocforward{#2}}
    \else
      \def\childdoctmp{\childdocforwardprefix{#1}{#2}}
    \fi
    \expandafter
  \endgroup
  \childdoctmp
}
%    \end{macrocode}

%\iffalse
%</package>
%\fi
%
\endinput
|\\
|\childdocby{|\textit{main}|}|\\
\end{tabular}
\end{center}
%
Both forms have slightly different effects as described above.
The main file is prepared as usual, see \secref{sec:include}.

%%%%%%%%%%%%%%%%%%%%%%%%%%%%%%%%%%%%%%%%%%%%%%%%%%%%%%%%%%%%%%%%%%%%%%%%%%%%%%%%
\subsection{Legacy Detection}
\label{sec:detection}

The directive |\childdocmain| in the main file can detect
whether the complete document or merely a child is to be compiled
even without using the directive |\childdocof|.
This method is deprecated because it is less robust
and there is no compelling reason to use it;
it is merely provided for backward compatibility
and it may be removed in future versions.

If the detection mechanism is to be used,
it is mandatory to correctly specify
the filename of the main file as the argument of |\childdocmain|:
%
\begin{center}
\begin{tabular}{l}
|% \iffalse
%
% childdoc.dtx Copyright (C) 2017-2018 Niklas Beisert
%
% This work may be distributed and/or modified under the
% conditions of the LaTeX Project Public License, either version 1.3
% of this license or (at your option) any later version.
% The latest version of this license is in
%   http://www.latex-project.org/lppl.txt
% and version 1.3 or later is part of all distributions of LaTeX
% version 2005/12/01 or later.
%
% This work has the LPPL maintenance status `maintained'.
%
% The Current Maintainer of this work is Niklas Beisert.
%
% This work consists of the files childdoc.dtx and childdoc.ins
% and the derived files childdoc.def and cdocsamp.tex with
% cdocsch1.tex, cdocsch2.tex, cdocsdrf.tex, cdocsfn1.tex, cdocsfn2.tex.
%
%<package>\ifdefined\childdocmain\endinput\fi
%<package>\ProvidesFile{childdoc.def}[2018/12/30 v2.0 child document driver]
%<samplemain>\ProvidesFile{cdocsamp.tex}[2018/12/30 v2.0 sample for childdoc]
%<*driver>
%\ProvidesFile{childdoc.drv}[2018/12/30 v2.0 childdoc reference manual file]
\PassOptionsToClass{10pt,a4paper}{article}
\documentclass{ltxdoc}

\usepackage[margin=35mm]{geometry}
\usepackage{hyperref}
\usepackage{hyperxmp}
\usepackage[usenames]{color}

\hypersetup{colorlinks=true}
\hypersetup{pdfstartview=FitH}
\hypersetup{pdfpagemode=UseNone}
\hypersetup{pdfsource={}}
\hypersetup{pdflang={en-UK}}
\hypersetup{pdfcopyright={Copyright 2017-2018 Niklas Beisert.
  This work may be distributed and/or modified under the
  conditions of the LaTeX Project Public License, either version 1.3
  of this license or (at your option) any later version.}}
\hypersetup{pdflicenseurl={http://www.latex-project.org/lppl.txt}}
\hypersetup{pdfcontactaddress={ETH Zurich, ITP, HIT K,
  Wolfgang-Pauli-Strasse 27}}
\hypersetup{pdfcontactpostcode={8093}}
\hypersetup{pdfcontactcity={Zurich}}
\hypersetup{pdfcontactcountry={Switzerland}}
\hypersetup{pdfcontactemail={nbeisert@itp.phys.ethz.ch}}
\hypersetup{pdfcontacturl={http://people.phys.ethz.ch/\xmptilde nbeisert/}}

\newcommand{\secref}[1]{\hyperref[#1]{section \ref*{#1}}}

\parskip1ex
\parindent0pt
\let\olditemize\itemize
\def\itemize{\olditemize\parskip0pt}

\begin{document}

\title{The \textsf{childdoc} Package}
\hypersetup{pdftitle={The childdoc Package}}
\author{Niklas Beisert\\[2ex]
  Institut f\"ur Theoretische Physik\\
  Eidgen\"ossische Technische Hochschule Z\"urich\\
  Wolfgang-Pauli-Strasse 27, 8093 Z\"urich, Switzerland\\[1ex]
  \href{mailto:nbeisert@itp.phys.ethz.ch}
  {\texttt{nbeisert@itp.phys.ethz.ch}}}
\hypersetup{pdfauthor={Niklas Beisert}}
\hypersetup{pdfsubject={Manual for the LaTeX2e Package childdoc}}
\date{30 December 2018, \textsf{v2.0}}
\maketitle

\begin{abstract}\noindent
\textsf{childdoc} is a \LaTeXe{} package
that enables the direct compilation
of document sections included by |\include|
to individual files.
\end{abstract}

\begingroup
\parskip0ex
\tableofcontents
\endgroup

%%%%%%%%%%%%%%%%%%%%%%%%%%%%%%%%%%%%%%%%%%%%%%%%%%%%%%%%%%%%%%%%%%%%%%%%%%%%%%%%
%%%%%%%%%%%%%%%%%%%%%%%%%%%%%%%%%%%%%%%%%%%%%%%%%%%%%%%%%%%%%%%%%%%%%%%%%%%%%%%%
\section{Introduction}

\LaTeX{} provides a mechanism to structure a large document (such as a book)
into a main file and several child files (containing the chapters)
using the |\include| command.
This mechanism is beneficial for documents
which span hundreds of pages in order to
make the source file(s) more manageable.
Moreover, compilation can be restricted to
selected child files by means of the |\includeonly| command.
The latter feature can be used to reduce the compilation time while editing
(this was significantly more useful in the earlier days of \LaTeX{})
or to generate a smaller document which is easier to navigate.
Another application of |\includeonly| is to generate
documents consisting of selected parts of the complete document.

However, there are a few drawbacks of the plain |\include| mechanism:
\begin{itemize}
\item
The child files cannot be compiled on their own,
they can only be compiled via the main file.
A naive editing environment
(such as a text editor with an option
to have the current file processed by \LaTeX)
may require one to switch to the main file before compiling;
attempting to compile the child file produces errors.
\item
The main file must be modified (each time)
to adjust the |\includeonly| command
to the present needs. This easily leaves the main file in a messy state.
\item
The generated document will always carry the filename
of the main document. This is inconvenient if
several child files are to be compiled and
to be kept for distribution.
\end{itemize}

The present package provides a simple interface
to make child files individually compilable by \LaTeX{}.
Compiling a child file then has the same effect as compiling
the main file with an |\includeonly| command
to select the appropriate child.
Moreover the generated document will carry the name of the child
rather than the main file.
This resolves all three above issues.

This feature is meant to make the editing of books,
thesis documents and lecture notes somewhat more convenient.
However, the package can also be used efficiently for
composing a series of documents (such as exercise sheets)
which are typically distributed individually.
It then assists the author in generating the individual documents
(potentially in different versions)
as well as a document containing the collected series.
Another application is in developing style files
or other kinds of included material
where compilation of the style file could redirect
to a sample or test file.

%%%%%%%%%%%%%%%%%%%%%%%%%%%%%%%%%%%%%%%%%%%%%%%%%%%%%%%%%%%%%%%%%%%%%%%%%%%%%%%%
%%%%%%%%%%%%%%%%%%%%%%%%%%%%%%%%%%%%%%%%%%%%%%%%%%%%%%%%%%%%%%%%%%%%%%%%%%%%%%%%
\section{Usage}

First of all, the package \textsf{childdoc} is \emph{not} a standard
\LaTeXe{} |.sty| style file! Therefore it needs to be invoked in
a non-standard way.

%%%%%%%%%%%%%%%%%%%%%%%%%%%%%%%%%%%%%%%%%%%%%%%%%%%%%%%%%%%%%%%%%%%%%%%%%%%%%%%%
\subsection{Included Files}
\label{sec:include}

%%%%%%%%%%%%%%%%%%%%%%%%%%%%%%%%%%%%%%%%
\DescribeMacro{\childdocmain}
To use the package, add the commands
\begin{center}
\begin{tabular}{l}
|\input{childdoc.def}|\\
|\childdocmain{}|\\
\end{tabular}
\end{center}
at the very top of the main \LaTeX{} file,
in particular \emph{before} the |\documentclass| statement!
The argument of |\childdocmain| should be left empty
(but it must be present).

%%%%%%%%%%%%%%%%%%%%%%%%%%%%%%%%%%%%%%%%
\DescribeMacro{\childdocof}
Furthermore, add the commands
\begin{center}
\begin{tabular}{l}
|\input{childdoc.def}|\\
|\childdocof{|\textit{main}|}|\\
\end{tabular}
\end{center}
at the top of every child file \textit{child}
which is included by |\include{|\textit{child}|}|
from within the main file
(or at least for those files to be compiled individually).
The argument \textit{main} must be the filename of the main file.

There are a couple of
considerations in setting up the main and child documents:

%%%%%%%%%%%%%%%%%%%%%%%%%%%%%%%%%%%%%%%%
\paragraph{Restrictions.}

Please note the following restrictions:
\begin{itemize}
\item
|\childdocmain| must be called with one argument \textit{main}
to ensure compatibility with earlier version of the package.
It must either be empty (|\childdocmain{}|)
or precisely match the filename of the main file in which it is specified.
See \secref{sec:detection} for further information.
\item
The filename \textit{main} must be specified without the |.tex| extension.
\item
The filename \textit{main} is case sensitive
(even in case-insensitive file systems)
due to internal string comparison.
\item
The argument \textit{main} should be fully expanded, it cannot be a macro.
\item
Subdirectories and special characters should be avoided in filenames.
\item
The command |\childdocmain{|\textit{main}|}| must be followed by a whitespace.
It should not be followed immediately by another command
or by a comment mark `|%|'.
This is because the \TeX{} parser reads the token immediately following
the argument of |\childdocmain| and puts it
at the beginning of every child section;
however, a white\-space is ignored.
\end{itemize}

%%%%%%%%%%%%%%%%%%%%%%%%%%%%%%%%%%%%%%%%
\paragraph{Content of Main File.}

It is advisable to place all content in the child files included by |\include|.
Any output contained in the main file will appear in all child documents
unless suppressed manually;
it cannot be suppressed automatically by the |\includeonly| directive
and thus should normally be avoided.
A method to include some content in the main file
by means of conditional processing is described in \secref{sec:conditional}.

%%%%%%%%%%%%%%%%%%%%%%%%%%%%%%%%%%%%%%%%
\paragraph{Page Numbering.}

When only a part of the document is compiled,
the appropriate numbering of pages
(as well as other status parameters)
is determined from the |.aux| files.
The latter contain information from previous passes.
However this information needs to propagate through
all intermediate child documents.
Therefore the page numbering in child documents may well
be inconsistent until the complete document is compiled at least once.

A useful (if unconventional) way to always ensure a consistent
page numbering is to restart the numbering in each child document
and denote the pages by `\textit{child}|.|\textit{page}'
where \textit{child} represents the chapter/section number of the child file.
This can be achieved by the command
|\numberwithin{page}{|\textit{child}|}|
of the \textsf{amsmath} package
where \textit{child} can be |chapter| or |section|
depending on the chosen structuring.
Alternatively, one can modify the macro |\thepage| appropriately
and reset the counter |page| at the start of each child file.

%%%%%%%%%%%%%%%%%%%%%%%%%%%%%%%%%%%%%%%%%%%%%%%%%%%%%%%%%%%%%%%%%%%%%%%%%%%%%%%%
\subsection{Conditional Processing}
\label{sec:conditional}

The package provides a mechanism to compile different versions
of a document. To customise the versions further some conditional processing
can come in handy to distinguish which version is being compiled.
The package provides two macros to describe the compilation context:

%%%%%%%%%%%%%%%%%%%%%%%%%%%%%%%%%%%%%%%%
\DescribeMacro{\ifchilddoc}
The conditional |\ifchilddoc| distinguishes between the compilation of
child documents and the main document:
%
\begin{center}
|\ifchilddoc |\textit{child-code}| |[|\||else |\textit{main-code}]| \||fi|
\end{center}

%%%%%%%%%%%%%%%%%%%%%%%%%%%%%%%%%%%%%%%%
\DescribeMacro{\childdocname}
\DescribeMacro{\childdocjob}
The macro |\childdocname| contains the filename (without extension)
of the main or child file being processed.
Note that |\childdocjob| will always contain the name of the main file.

%%%%%%%%%%%%%%%%%%%%%%%%%%%%%%%%%%%%%%%%
\paragraph{Title Page.}

Conditional processing can be used to include a title or banner page
in the main document when proper precautions are taken.
Importantly, the code in the main file should ensure that the page counter
(as well as other status parameters which are stored in the |.aux| files)
takes the same value after the conditional processing.
Otherwise the page numbers may take divergent values
depending on which part is compiled.

For example, a title page could be declared by:
%
\begin{center}
\begin{tabular}{l}
|\ifchilddoc\||else|\\
|\addtocounter{page}{-1}|\\
\textit{code for title page}\\
|\newpage|\\
|\||fi|
\end{tabular}
\end{center}
%
A banner page for the child documents can be generated by:
%
\begin{center}
\begin{tabular}{l}
|\ifchilddoc|\\
|\addtocounter{page}{-1}|\\
\textit{code for banner page}\\
|\newpage|\\
|\||fi|
\end{tabular}
\end{center}
%
Here one could write a message such as:
\begin{center}
|This is the part \childdocname{} of \childdocjob{}.|
\end{center}

%%%%%%%%%%%%%%%%%%%%%%%%%%%%%%%%%%%%%%%%%%%%%%%%%%%%%%%%%%%%%%%%%%%%%%%%%%%%%%%%
\subsection{Flags}
\label{sec:flags}

The package makes it easy to generate different versions
of the main or child documents.
To this end compilation flags can be defined
and assigned different default values.
They will be particularly useful in conjunction
with the forwarding mechanism described in \secref{sec:forward}.

For example, it may be useful to have a flag |\version|
which can be set to |draft| or |final|.
The document source will contain some conditional code
depending on the value of |\version|.
Suppose further, the flag should default to |final| for the main file
and to |draft| for child files
which is a natural assignment for editing the document.
This is achieved by placing the following code
in the preamble of the main document
(below the |\childdocmain| directive):
%
\begin{center}
\begin{tabular}{l}
|\ifchilddoc|\\
|\providecommand{\version}{draft}|\\
|\||else|\\
|\providecommand{\version}{final}|\\
|\||fi|
\end{tabular}
\end{center}
%
The definition by |\providecommand| makes sure
that previous definitions are not overwritten.
Further statements |\providecommand{\version}{...}|
can thus be added before the above code to override it.

For the main file, one might add a line
(between |\childdocmain| and the above block)
%
\begin{center}
|%\ifchilddoc\||else\providecommand{\version}{draft}\||fi|
\end{center}
%
which can be uncommented to produce a draft version.
Likewise one can add a line to the very top of a child file
(above the |\childdocof{|\textit{main}|}| directive)
%
\begin{center}
|%\providecommand{\version}{final}|
\end{center}
%
which can be uncommented to produce the final version of this child document.

%%%%%%%%%%%%%%%%%%%%%%%%%%%%%%%%%%%%%%%%%%%%%%%%%%%%%%%%%%%%%%%%%%%%%%%%%%%%%%%%
\subsection{Forwarding}
\label{sec:forward}

Different versions of the main or child documents
using compilation flags as described in \secref{sec:flags}
can be (permanently) stored in different files
for convenient compilation, viewing and distribution.
To this end, the package defines a command
to pass on compilation to a different file:

%%%%%%%%%%%%%%%%%%%%%%%%%%%%%%%%%%%%%%%%
\DescribeMacro{\childdocforward}
The command |\childdocforward| redirects processing to
another source file:
%
\begin{center}
\begin{tabular}{l}
|\input{childdoc.def}|\\
|\childdocforward[|\textit{main}|]{|\textit{dest}|}|\\
\end{tabular}
\end{center}
%
The argument \textit{dest} is the destination file
(without extension).
It should be the main file or one of the child files.
Note that further \textsf{childdoc} directives
such as |\childdocof| and |\childdocforward|
in the indicated file will be processed in this form.
The optional argument \textit{main}
passes on directly to the main file \textit{main}
while pretending to compile the child \textit{dest}.
This form behaves as if \textit{dest}
issues |\childdocof{|\textit{main}|}| right away,
and no further \textsf{childdoc} directives will be processed.

%%%%%%%%%%%%%%%%%%%%%%%%%%%%%%%%%%%%%%%%
\DescribeMacro{\...prefix}
In the alternative form |\childdocforwardprefix|,
%
\begin{center}
\begin{tabular}{l}
|\input{childdoc.def}|\\
|\childdocforwardprefix[|\textit{main}|]{|\textit{prefix}|}{|\textit{dest}|}|
\end{tabular}
\end{center}
%
the destination file is determined by a pattern
depending on the current file:
To make this work, the current file must be called
`{\textit{prefix}\hspace{0.2em}\textit{suffix}}'
with \textit{prefix} matching precisely the argument.
Processing is then passed on to the file
`{\textit{dest}\hspace{0.2em}\textit{suffix}}'.
Surely, the same effect is achieved by
directly specifying the
argument `{\textit{dest}\hspace{0.2em}\textit{suffix}}'
in the first form.
However, that requires to set up a different file
for each child. With the alternative form of the command
all these files can have exactly the same content
which simplifies setting them up and maintaining them.

For example, the following file |draft.tex|
with a compilation flag |\version| as described in \secref{sec:flags}
compiles the main document as a draft:
%
\begin{center}
\begin{tabular}{l}
|\def\version{draft}|\\
|\input{childdoc.def}|\\
|\childdocforward{|\textit{main}|}|
\end{tabular}
\end{center}
%
Likewise, the following files |final|\textit{nn}|.tex|
compile the final version of the child document
|child|\textit{nn}|.tex|:
%
\begin{center}
\begin{tabular}{l}
|\def\version{final}|\\
|\input{childdoc.def}|\\
|\childdocforwardprefix{final}{child}|
\end{tabular}
\end{center}
%

Note that when several versions of a main file and/or of each child file
are to be generated, it may be convenient to set up a |Makefile| or
shell script to automatise the process.

%%%%%%%%%%%%%%%%%%%%%%%%%%%%%%%%%%%%%%%%%%%%%%%%%%%%%%%%%%%%%%%%%%%%%%%%%%%%%%%%
\subsection{Command Line Processing}
\label{sec:commandline}

The effect of redirection files can also be achieved by invoking
the \LaTeX{} compiler with a more elaborate command line.
Most conveniently this should be done as part
of a shell script or a |Makefile|.

When using \textsf{childdoc} in the main file, the following
command lines effectively perform a redirection
(note that depending on the shell being used,
backslashes may have to be doubled: `|\|' $\to$ `|\\|'):
%
\begin{center}
|... -jobname "|\textit{target}|" |\\|"|[\textit{flags}]%
|\input{childdoc.def}\childdocforward[|\textit{main}|]{|\textit{dest}|}"|
\end{center}
%
Here \textit{target} is the name of the output file,
\textit{main} is the name of the main file
and \textit{dest} is the name of the main or child file to be processed
(all filenames without extensions).
The optional argument \textit{main} can be omitted
if \textit{main} matches \textit{dest}.
Optionally, compilation \textit{flags} can be defined via |\def| commands.
This command line makes the \TeX{} engine believe
it is compiling the file \textit{target}
whose content is specified as the latter parameter.
The provided code then forwards the processing to
\textit{main} or \textit{dest} as described in \secref{sec:forward}.

%%%%%%%%%%%%%%%%%%%%%%%%%%%%%%%%%%%%%%%%%%%%%%%%%%%%%%%%%%%%%%%%%%%%%%%%%%%%%%%%
\subsection{Include by Input}
\label{sec:input}

Including child documents by |\include| has some restrictions by design.
Most notably, the content of a child document always occupies
its own set of pages; pages cannot be shared between child documents.
Usually, this behaviour makes perfect sense
because each child document contain an essential part of the document.
However, in some situations it may be desirable to compose
a document from a collection of parts
without having mandatory page breaks between then.
For this case, the package
provides a mechanism to include parts
by |\input| which can also be processed individually.
However, by construction this mechanism
requires manual handling of the content to be output.

%%%%%%%%%%%%%%%%%%%%%%%%%%%%%%%%%%%%%%%%
\DescribeMacro{\ifchilddocmanual}
The main file should be prepared as usual, see \secref{sec:include}.
However, the document body must make a distinction
between processing of an individual part and of the main document, e.g.:
%
\begin{center}
\begin{tabular}{l}
|\ifchilddocmanual|\\
|\input{\childdocname}|\\
|\||else|\\
\textit{document body with }|\input{|\textit{part}|}|\\
|\||fi|
\end{tabular}
\end{center}
%
The conditional |\ifchilddocmanual| is true whenever
a part to be included by |\input| is being compiled,
and the name of the part is stored in |\childdocname|.

%%%%%%%%%%%%%%%%%%%%%%%%%%%%%%%%%%%%%%%%
\DescribeMacro{\childdocby}
Each part to be included by |\input| should start with:
%
\begin{center}
\begin{tabular}{l}
|\input{childdoc.def}|\\
|\childdocby{|\textit{main}|}|\\
\end{tabular}
\end{center}
%
The directive |\childdocby| is similar to |\childdocof|
described in \secref{sec:include},
but the subsequent selection of content must be done manually.
To that end, both |\ifchilddoc| and |\ifchilddocmanual|
will be true upon processing of a part,
and the name of the part is stored in |\childdocname|.
Note that |\jobname| will be set to the filename of the current part
so that each part receives an individual |.aux| file
that does not interfere with the |.aux| file(s) of the main document.
This behaviour can be altered by the alternative form
|\childdocby[*]{|\textit{main}|}| (with a non-empty optional argument)
which uses the |.aux| file of the main document
by setting |\jobname| to \textit{main}.

%%%%%%%%%%%%%%%%%%%%%%%%%%%%%%%%%%%%%%%%%%%%%%%%%%%%%%%%%%%%%%%%%%%%%%%%%%%%%%%%
\subsection{Driver Development}
\label{sec:driver}

The \textsf{childdoc} mechanism can also be use for the development
of definition files such as \LaTeX{} styles or classes.
This case differs from the above setup with multiple parts
included by |\include| in that no |\includeonly| should be invoked.
This can be achieved by starting the include file
(before |\ProvidesPackage|) with:
%
\begin{center}
\begin{tabular}{l}
|\input{childdoc.def}|\\
|\childdocforward{|\textit{main}|}|\\
\end{tabular}
\end{center}
%
or alternatively with:
%
\begin{center}
\begin{tabular}{l}
|\input{childdoc.def}|\\
|\childdocby{|\textit{main}|}|\\
\end{tabular}
\end{center}
%
Both forms have slightly different effects as described above.
The main file is prepared as usual, see \secref{sec:include}.

%%%%%%%%%%%%%%%%%%%%%%%%%%%%%%%%%%%%%%%%%%%%%%%%%%%%%%%%%%%%%%%%%%%%%%%%%%%%%%%%
\subsection{Legacy Detection}
\label{sec:detection}

The directive |\childdocmain| in the main file can detect
whether the complete document or merely a child is to be compiled
even without using the directive |\childdocof|.
This method is deprecated because it is less robust
and there is no compelling reason to use it;
it is merely provided for backward compatibility
and it may be removed in future versions.

If the detection mechanism is to be used,
it is mandatory to correctly specify
the filename of the main file as the argument of |\childdocmain|:
%
\begin{center}
\begin{tabular}{l}
|\input{childdoc.def}|\\
|\childdocmain{|\textit{main}|}|\\
\end{tabular}
\end{center}
%
If |\jobname| does not match the argument \textit{main} of |\childdocmain|,
it is assumed that |\jobname| points to the child file to be compiled.
When using |\childdocmain| with the main file specified as argument,
it suffices to start a child file
with just |\input{|\textit{main}|}|
without loading of the package and using |\childdocof|.
If instead all processing is done
with the appropriate \textsf{childdoc} directives,
the argument of \textit{main} of |\childdocmain| can be empty.

An alternative version of the command line processing described
in \secref{sec:commandline} using the detection mechanism reads:
%
\begin{center}
|... -jobname "|\textit{target}|" "|[\textit{flags}]%
[|\def\jobname{|\textit{dest}|}|]|\input{|\textit{main}|}"|
\end{center}

%%%%%%%%%%%%%%%%%%%%%%%%%%%%%%%%%%%%%%%%%%%%%%%%%%%%%%%%%%%%%%%%%%%%%%%%%%%%%%%%
\subsection{Manual Code}
\label{sec:manual}

In case one cannot be certain whether the definitions file |childdoc.def|
is installed on the target \TeX{} distribution
and one prefers not to ship it,
it is conceivable to paste a few relevant commands into the sources.

To that end, drop all statements |\input{childdoc.def}|
and perform the replacements as outlined below.
Instead of |\childdocmain{|\textit{main}|}| add the following code
to the top of the main file:
%
\begin{center}
\begin{tabular}{l}
|\||ifdefined\childdocname\endinput\||fi\newif\ifchilddoc|\\
|\edef\childdocname{\scantokens\expandafter{\jobname\noexpand}}|\\
|\def\childdocmain{|\textit{main}|}\||ifx\childdocmain\childdocname\||else|\\
|\childdoctrue\includeonly{\childdocname}\let\jobname\childdocmain\||fi|\\
\end{tabular}
\end{center}
%
Instead of |\childdocof{|\textit{main}|}| just include the main file
at the top of each child file:
%
\begin{center}
|\input{|\textit{main}|}|
\end{center}
%
A simple redirection |\childdocforward{|\textit{dest}|}| is achieved by:
%
\begin{center}
|\def\jobname{|\textit{dest}|}\input{\jobname}|
\end{center}
%
The redirection with prefix
|\childdocforwardprefix[|\textit{prefix}|]{|\textit{dest}|}|
is accomplished by:
%
\begin{center}
\begin{tabular}{l}
|{\edef\jobname{\scantokens\expandafter{\jobname\noexpand}}|\\
|\def\redirectjob |\textit{prefix}|#1~~~{\gdef\jobname{|\textit{dest}|#1}}|\\
|\expandafter\redirectjob\jobname~~~}\input{\jobname}|
\end{tabular}
\end{center}

In an alternative approach,
child documents can be compiled by a specific command line
without additional code or specific definitions:
%
\begin{center}
|... -jobname "|\textit{target}|" "|[\textit{flags}]%
|\includeonly{|\textit{dest}|}\input{|\textit{main}|}"|
\end{center}
%

%%%%%%%%%%%%%%%%%%%%%%%%%%%%%%%%%%%%%%%%%%%%%%%%%%%%%%%%%%%%%%%%%%%%%%%%%%%%%%%%
%%%%%%%%%%%%%%%%%%%%%%%%%%%%%%%%%%%%%%%%%%%%%%%%%%%%%%%%%%%%%%%%%%%%%%%%%%%%%%%%
\section{Information}

%%%%%%%%%%%%%%%%%%%%%%%%%%%%%%%%%%%%%%%%%%%%%%%%%%%%%%%%%%%%%%%%%%%%%%%%%%%%%%%%
\subsection{Copyright}

Copyright \copyright{} 2017--2018 Niklas Beisert

This work may be distributed and/or modified under the
conditions of the \LaTeX{} Project Public License, either version 1.3
of this license or (at your option) any later version.
The latest version of this license is in
  \url{http://www.latex-project.org/lppl.txt}
and version 1.3 or later is part of all distributions of \LaTeX{}
version 2005/12/01 or later.

This work has the LPPL maintenance status `maintained'.

The Current Maintainer of this work is Niklas Beisert.

This work consists of the files |README.txt|, |childdoc.ins| and |childdoc.dtx|
as well as the derived files |childdoc.def|, |cdocsamp.tex|
with |cdocsch1.tex|, |cdocsch2.tex|, |cdocspt3.tex|, |cdocspt4.tex|,
|cdocsdrf.tex|, |cdocsfn1.tex|, |cdocsfn2.tex|
as well as |childdoc.pdf|.

%%%%%%%%%%%%%%%%%%%%%%%%%%%%%%%%%%%%%%%%%%%%%%%%%%%%%%%%%%%%%%%%%%%%%%%%%%%%%%%%
\subsection{Files and Installation}

The package consists of the files:
%
\begin{center}
\begin{tabular}{ll}
    |README.txt|   & readme file \\
    |childdoc.ins| & installation file \\
    |childdoc.dtx| & source file \\
    |childdoc.def| & definition file \\
    |cdocsamp.tex| & sample main file \\
    |cdocsch1.tex| & sample include file \\
    |cdocsch2.tex| & sample include file \\
    |cdocspt3.tex| & sample part file \\
    |cdocspt4.tex| & sample part file \\
    |cdocsdrf.tex| & sample redirection file \\
    |cdocsfn1.tex| & sample redirection file \\
    |cdocsfn2.tex| & sample redirection file \\
    |childdoc.pdf| & manual
\end{tabular}
\end{center}
%
The distribution consists of the files
|README.txt|, |childdoc.ins| and |childdoc.dtx|.
%
\begin{itemize}
\item
Run (pdf)\LaTeX{} on |childdoc.dtx|
to compile the manual |childdoc.pdf| (this file).
\item
Run \LaTeX{} on |childdoc.ins| to create the definitions file |childdoc.def|
and the sample |cdocsamp.tex| with include files
|cdocsch1.tex|, |cdocsch2.tex|, |cdocspt3.tex|, |cdocspt4.tex|,
|cdocsdrf.tex|, |cdocsfn1.tex|, |cdocsfn2.tex|.
Then copy the file |childdoc.def| to an appropriate directory of your \LaTeX{}
distribution, e.g.\ \textit{texmf-root}|/tex/latex/childdoc|.
\end{itemize}

%%%%%%%%%%%%%%%%%%%%%%%%%%%%%%%%%%%%%%%%%%%%%%%%%%%%%%%%%%%%%%%%%%%%%%%%%%%%%%%%
\subsection{Related CTAN Packages}

There are several other packages which offer a similar functionality:
%
\begin{itemize}
\item
The packages
\href{http://ctan.org/pkg/docmute}{\textsf{docmute}},
\href{http://ctan.org/pkg/includex}{\textsf{includex}} and
\href{http://ctan.org/pkg/standalone}{\textsf{standalone}}
provide commands to include only the document body of
a child file thus allowing both files to be compiled individually.
\item
The packages \href{http://ctan.org/pkg/subdocs}{\textsf{subdocs}}
and \href{http://ctan.org/pkg/subfiles}{\textsf{subfiles}}
provide structures in which the main and child documents can be
encapsulated and allowing them to be compiled individually.
The inclusion mechanism is different from the conventional |\include|.
\item
The package \href{http://ctan.org/pkg/combine}{\textsf{combine}}
is an elaborate solution to combine several documents into one.
\end{itemize}
%
See also the CTAN topic \href{http://ctan.org/topic/subdocs}{\textsf{subdocs}}
for further related packages.
The present package differs from the above solutions in that
a document structure constructed with the conventional |\include| mechanism
just needs two extra commands at the top of every file
such that all constituent files can be compiled individually.

%%%%%%%%%%%%%%%%%%%%%%%%%%%%%%%%%%%%%%%%%%%%%%%%%%%%%%%%%%%%%%%%%%%%%%%%%%%%%%%%
%\subsection{Feature Suggestions}
%
%The following is a list of features which may be useful for future
%versions of this package:
%%
%\begin{itemize}
%\item
%\ldots
%\end{itemize}

%%%%%%%%%%%%%%%%%%%%%%%%%%%%%%%%%%%%%%%%%%%%%%%%%%%%%%%%%%%%%%%%%%%%%%%%%%%%%%%%
\subsection{Revision History}

%%%%%%%%%%%%%%%%%%%%%%%%%%%%%%%%%%%%%%%%
\paragraph{v2.0:} 2018/12/30

\begin{itemize}
\item
immediate forward processing
\item
added |\childdocby| mechanism
\item
manual restructured
\end{itemize}

%%%%%%%%%%%%%%%%%%%%%%%%%%%%%%%%%%%%%%%%
\paragraph{v1.6:} 2018/01/17

\begin{itemize}
\item
application for development of include files
\item
corrections to manual
\end{itemize}

%%%%%%%%%%%%%%%%%%%%%%%%%%%%%%%%%%%%%%%%
\paragraph{v1.5:} 2017/05/21

\begin{itemize}
\item
more complete structuring introduced
\item
|\childdocof| introduced
\item
|\childdoc| renamed to |\childdocmain|
\item
|\childredirect| renamed to |\childdocforward| and |\childdocforwardprefix|
and functionality expanded
\end{itemize}

%%%%%%%%%%%%%%%%%%%%%%%%%%%%%%%%%%%%%%%%
\paragraph{v1.0:} 2017/04/27

\begin{itemize}
\item
manual and install package
\item
first version published on CTAN
\end{itemize}

%%%%%%%%%%%%%%%%%%%%%%%%%%%%%%%%%%%%%%%%
\paragraph{v0.6:} 2017/04/26

\begin{itemize}
\item
redirection mechanism added
\end{itemize}

%%%%%%%%%%%%%%%%%%%%%%%%%%%%%%%%%%%%%%%%
\paragraph{v0.5:} 2017/04/26

\begin{itemize}
\item
functionality in definition file
\end{itemize}


%%%%%%%%%%%%%%%%%%%%%%%%%%%%%%%%%%%%%%%%%%%%%%%%%%%%%%%%%%%%%%%%%%%%%%%%%%%%%%%%
%%%%%%%%%%%%%%%%%%%%%%%%%%%%%%%%%%%%%%%%%%%%%%%%%%%%%%%%%%%%%%%%%%%%%%%%%%%%%%%%
%%%%%%%%%%%%%%%%%%%%%%%%%%%%%%%%%%%%%%%%%%%%%%%%%%%%%%%%%%%%%%%%%%%%%%%%%%%%%%%%
\appendix

\settowidth\MacroIndent{\rmfamily\scriptsize 000\ }

 \DocInput{childdoc.dtx}

\end{document}
%</driver>
% \fi
%
% %%%%%%%%%%%%%%%%%%%%%%%%%%%%%%%%%%%%%%%%%%%%%%%%%%%%%%%%%%%%%%%%%%%%%%%%%%%%%%
% %%%%%%%%%%%%%%%%%%%%%%%%%%%%%%%%%%%%%%%%%%%%%%%%%%%%%%%%%%%%%%%%%%%%%%%%%%%%%%
% \section{Sample}
%\iffalse
%<*samplemain>
%\fi
%
% The following presents a sample document
% with two chapters, two parts, a title page,
% a compile flag as well as three forwarding files to set the flag.
% It consists of eight |.tex| files:
% \begin{center}
% \begin{tabular}{ll}
% |cdocsamp.tex|&main file\\
% |cdocsch1.tex|&include file for chapter 1\\
% |cdocsch2.tex|&include file for chapter 2\\
% |cdocspt3.tex|&include file for part 3\\
% |cdocspt4.tex|&include file for part 4\\
% |cdocsdrf.tex|&forwarding file for main file in draft mode\\
% |cdocsfi1.tex|&forwarding file for final version of chapter 1\\
% |cdocsfi2.tex|&forwarding file for final version of chapter 2\\
% \end{tabular}
% \end{center}
% Each of the eight files can be compiled directly by the \LaTeX{} compiler.
%
% %%%%%%%%%%%%%%%%%%%%%%%%%%%%%%%%%%%%%%
% \paragraph{Main File.}
%
% The main file is called |cdocsamp.tex|.
%
% Load the \textsf{childdoc} definitions and
% declare the filename for the main document:
%    \begin{macrocode}
\input{childdoc.def}
\childdocmain{}
%    \end{macrocode}

% Optional override for |\version| flag:
%    \begin{macrocode}
%%\ifchilddoc\else\providecommand{\version}{draft}\fi
%    \end{macrocode}

% Define the default values for the |\version| flag
% (|final| for the main file and |draft| for childs):
%    \begin{macrocode}
\ifchilddoc
\providecommand{\version}{draft}
\else
\providecommand{\version}{final}
\fi
%    \end{macrocode}

% Load the standard document class:
%    \begin{macrocode}
\documentclass[12pt]{article}
%    \end{macrocode}

% Start the document body:
%    \begin{macrocode}
\begin{document}
%    \end{macrocode}

% Declare a title page.
% Print title, part of document being processed and version flag:
%    \begin{macrocode}
\addtocounter{page}{-1}
\begin{center}
{\LARGE\bfseries{}childdoc example\par}
\vspace{1cm}
\ifchilddoc
\ifchilddocmanual part\else chapter\fi:
`\childdocname' of `\childdocjob'\par
\else
main document: `\childdocjob'\par
\fi
version: \version\par
\end{center}
\newpage
%    \end{macrocode}

% Manually include selected file,
% otherwise process as usual:
%    \begin{macrocode}
\ifchilddocmanual
\section*{part `\childdocname'}
\input{\childdocname}
\else
%    \end{macrocode}

% Include the two chapters:
%    \begin{macrocode}
\include{cdocsch1}
\include{cdocsch2}
%    \end{macrocode}

% Include the two parts unless only chapters should be displayed:
%    \begin{macrocode}
\ifchilddoc\else
\section{part three}
\input{cdocspt3}
\section{part four}
\input{cdocspt4}
\fi
%    \end{macrocode}

% Process as usual until here:
%    \begin{macrocode}
\fi
%    \end{macrocode}

% End of document body:
%    \begin{macrocode}
\end{document}
%    \end{macrocode}
%\iffalse
%</samplemain>
%\fi
%
% %%%%%%%%%%%%%%%%%%%%%%%%%%%%%%%%%%%%%%
% \paragraph{Chapter Include Files.}
%
% The include files are called |cdocsch1.tex| and |cdocsch2.tex|.
%
%\iffalse
%<*samplechap1|samplechap2>
%\fi

% Optional override for |\version| flag:
%    \begin{macrocode}
%%\providecommand{\version}{final}
%    \end{macrocode}

% Include the main document:
%    \begin{macrocode}
\input{childdoc.def}
\childdocof{cdocsamp}
%    \end{macrocode}

%\iffalse
%</samplechap1|samplechap2>
%\fi
%
%\iffalse
%<*samplechap1>
%\fi
% Some text for chapter 1:
%    \begin{macrocode}
\section{one}
some text in chapter one
%    \end{macrocode}

%\iffalse
%</samplechap1>
%\fi
% Some text for chapter 2:
%\iffalse
%<*samplechap2>
%\fi
%    \begin{macrocode}
\section{two}
more text in chapter two
%    \end{macrocode}

%\iffalse
%</samplechap2>
%\fi
%
% %%%%%%%%%%%%%%%%%%%%%%%%%%%%%%%%%%%%%%
% \paragraph{Part Include Files.}
%
% The include files are called |cdocspt3.tex| and |cdocspt4.tex|.
%
%\iffalse
%<*samplepart3|samplepart4>
%\fi

% Optional override for |\version| flag:
%    \begin{macrocode}
%%\providecommand{\version}{final}
%    \end{macrocode}

% Include the main document:
%    \begin{macrocode}
\input{childdoc.def}
\childdocby{cdocsamp}
%    \end{macrocode}

%\iffalse
%</samplepart3|samplepart4>
%\fi
%
%\iffalse
%<*samplepart3>
%\fi
% Some text for part 3:
%    \begin{macrocode}
some text in part three
%    \end{macrocode}

%\iffalse
%</samplepart3>
%\fi
% Some text for part 4:
%\iffalse
%<*samplepart4>
%\fi
%    \begin{macrocode}
more text in part four
%    \end{macrocode}

%\iffalse
%</samplepart4>
%\fi
%
% %%%%%%%%%%%%%%%%%%%%%%%%%%%%%%%%%%%%%%
% \paragraph{Forwarding for a Complete Draft.}
%
% The following forwarding file |cdocsdrf.tex|
% compiles the main document in draft mode:
%\iffalse
%<*sampledraft>
%\fi
%    \begin{macrocode}
\def\version{draft}
\input{childdoc.def}
\childdocforward{cdocsamp}
%    \end{macrocode}

%\iffalse
%</sampledraft>
%\fi
%
% %%%%%%%%%%%%%%%%%%%%%%%%%%%%%%%%%%%%%%
% \paragraph{Forwarding for Final Version of the Chapters.}
%
% The following forwarding files |cdocsfn1.tex| and |cdocsfn2.tex|
% (with identical content)
% compile the final versions of the child documents
% |cdocsch1.tex| and |cdocsch2.tex|, respectively:
%\iffalse
%<*samplefinal>
%\fi
%    \begin{macrocode}
\def\version{final}
\input{childdoc.def}
\childdocforwardprefix[cdocsamp]{cdocsfn}{cdocsch}
%    \end{macrocode}

%\iffalse
%</samplefinal>
%\fi
%
% %%%%%%%%%%%%%%%%%%%%%%%%%%%%%%%%%%%%%%
% \paragraph{Command Line Processing.}
%
% The following three command lines generate the output files
% |cdocscld|, |cdocscl1| and |cdocscl2|
% which should be identical to
% |cdocsdrf|, |cdocsch1| and |cdocsfn2|, respectively:
% \begin{center}
% \begin{tabular}{l}
% |latex -jobname cdocscld \|\\
% |  "\def\version{draft}\input{childdoc.def}\childdocforward{cdocsamp}"|\\
% |latex -jobname cdocscl1 \|\\
% |  "\input{childdoc.def}\childdocforward[cdocsamp]{cdocsch1}"|\\
% |latex -jobname cdocscl2 \|\\
% |  "\def\version{final}\input{childdoc.def}\childdocforward{cdocsch2}"|
% \end{tabular}
% \end{center}
% Note that the trailing backslash on each first line
% merely continues the input to the second line
% (for convenient cut ant paste).
% Furthermore, the command |latex| can be replaced by any
% of its alternative versions such as |pdflatex|.
%
% %%%%%%%%%%%%%%%%%%%%%%%%%%%%%%%%%%%%%%%%%%%%%%%%%%%%%%%%%%%%%%%%%%%%%%%%%%%%%%
% %%%%%%%%%%%%%%%%%%%%%%%%%%%%%%%%%%%%%%%%%%%%%%%%%%%%%%%%%%%%%%%%%%%%%%%%%%%%%%
% \section{Implementation}
%\iffalse
%<*package>
%\fi
%
% This section describes the definitions file |childdoc.def|.

% The definitions cannot be loaded using |\usepackage| or |\RequirePackage|
% which has a mechanism to prevent loading a style file more than once.
% When loading the definitions by means of |\input|
% multiple instances have to be prevented manually:
%\iffalse
%This code needs to be before the `\ProvidesFile' directive
%which is defined at the beginning of this file.
%Therefore it is also placed there and commented out here.
%</package>
%<*discard>
%\fi
%    \begin{macrocode}
\ifdefined\childdocmain\endinput\fi
%    \end{macrocode}
%\iffalse
%</discard>
%<*package>
%\fi
%
% \macro{\ifchilddoc}
% \macro{\ifchilddocmanual}
% The conditional |\ifchilddoc| tells whether a
% child (true) or main (false) document is being compiled.
% The conditional |\ifchilddocmanual| tells whether
% the |\includeonly| mechanism is used (false) or
% the selection of child files must be performed manually (true).
% The definitions initialise to false:
%    \begin{macrocode}
\newif\ifchilddoc
\newif\ifchilddocmanual
%    \end{macrocode}

% \macro{\childdocname}
% \macro{\childdocjob}
% The macro |\childdocname| stores the name of the main document
% to be compiled. The macro |\childdocjob| stores the name of
% the document on which the \LaTeX{} compiler was originally invoked.
% The content of |\jobname| cannot be compared
% to filenames specified in the source due to different catcodes.
% The following code rescans |\jobname|, stores the result
% in |\childdocname| and saves a copy in |\childdocjob|:
%    \begin{macrocode}
\edef\childdocname{\scantokens\expandafter{\jobname\noexpand}}
\let\childdocjob\childdocname
%    \end{macrocode}

% \macro{\childdocdisable}
% The macro |\childdocdisable| prevents the main file
% from being processed more than once.
% At this stage, the main document command |\childdocmain|
% is assumed to be called once again where it should do nothing.
% Any subsequent call to it should prevent
% a secondary processing of the main document
% It overwrites the forwarding commands
% |\childdocof| and |\childdocforward|
% with empty macros to prevent further inclusions of the main document:
%    \begin{macrocode}
\newcommand{\childdocdisable}
{
  \renewcommand{\childdocmain}[1]{\renewcommand{\childdocmain}[1]{\endinput}}
  \renewcommand{\childdocof}[1]{}
  \renewcommand{\childdocby}[2][]{}
  \renewcommand{\childdocforward}[2][]{}
  \renewcommand{\childdocdisable}{}
}
%    \end{macrocode}

% \macro{\childdocmain}
% The macro |\childdocmain| is to be called at the top of the main file
% with nothing or the main filename (without extension) as argument.
% First, it breaks loops.
% If the argument is not empty and does not match |\childdocname|
% (which is set by the first inclusion of |childdoc.def|),
% |\ifchilddoc| is set to true, |\includeonly| is applied to the child file
% and |\jobname| is set to the main file
% (for proper handling of |.aux| files):
%    \begin{macrocode}
\newcommand{\childdocmain}[1]
{
  \childdocdisable\childdocmain{}
  \if?#1?\else
    \begingroup
      \def\childdoctmp{#1}
      \ifx\childdoctmp\childdocname
        \def\childdoctmp{}
      \else
        \def\childdoctmp
        {
          \childdoctrue
          \includeonly{\childdocname}
          \def\childdocjob{#1}
          \def\jobname{#1}
        }
      \fi
      \expandafter
    \endgroup
    \childdoctmp
  \fi
}
%    \end{macrocode}

% \macro{\childdocof}
% The command |\childdocof| redirects
% compilation to the main file |#1|.
%    \begin{macrocode}
\newcommand{\childdocof}[1]
{
  \childdocdisable
  \childdoctrue
  \includeonly{\childdocname}
  \def\jobname{#1}
  \def\childdocjob{#1}
  \input{#1}
}
%    \end{macrocode}

% \macro{\childdocby}
% The command |\childdocby| ....
%    \begin{macrocode}
\newcommand{\childdocby}[2][]
{
  \childdocdisable
  \childdoctrue
  \childdocmanualtrue
  \if?#1?\else
    \def\jobname{#2}
  \fi
  \def\childdocjob{#2}
  \input{#2}
  \endinput
}
%    \end{macrocode}

% \macro{\childdocforward}
% The command |\childdocforward| redirects
% compilation to the main file or
% (if the optional argument is given) a child file.
% Parameters are set as if the main file
% or a child file starting with |\childdocof| was compiled.
% Then compilation is handed over to the main file:
%    \begin{macrocode}
\newcommand{\childdocforward}[2][]
{
  \begingroup
    \if?#1?
      \def\childdoctmp
      {
        \def\childdocname{#2}
        \def\childdocjob{#2}
        \def\jobname{#2}
        \input{#2}
        \endinput
      }
    \else
      \def\childdoctmp
      {
        \childdocdisable
        \def\childdocname{#2}
        \childdoctrue
        \includeonly{#2}
        \def\childdocjob{#1}
        \def\jobname{#1}
        \input{#1}
        \endinput
      }
    \fi
    \expandafter
  \endgroup
  \childdoctmp
}
%    \end{macrocode}

% \macro{\childdocforwardprefix}
% The command |\childdocforwardprefix| redirects
% compilation to the main or a child file by means of a pattern.
% The prefix |#1| in the current filename is replaced by |#2|
% and the suffix of the current filename is kept
% (it is assumed that the filename does not contain the substring `|~~~|'
% which is used as a delimiter).
% Compilation is handed over to the new file by |\childdocforward|:
%    \begin{macrocode}
\newcommand{\childdocforwardprefix}[3][]
{
  \begingroup
    \def\childdocextract #2##1~~~{\def\childdoctmp{\childdocforward[#1]{#3##1}}}
    \expandafter\childdocextract\childdocname~~~
    \expandafter
  \endgroup
  \childdoctmp
}
%    \end{macrocode}

% \macro{\childdoc}
% The deprecated macro |\childdoc| is a legacy version of |\childdocmain|:
%    \begin{macrocode}
\newcommand{\childdoc}{\childdocmain}
%    \end{macrocode}

% \macro{\childdocredirect}
% The deprecated macro |\childdocredirect| is a legacy version
% of |\childdocforward| and |\childdocforwardprefix|:
%    \begin{macrocode}
\newcommand{\childdocredirect}[2][]
{
  \begingroup
    \if?#1?
      \def\childdoctmp{\childdocforward{#2}}
    \else
      \def\childdoctmp{\childdocforwardprefix{#1}{#2}}
    \fi
    \expandafter
  \endgroup
  \childdoctmp
}
%    \end{macrocode}

%\iffalse
%</package>
%\fi
%
\endinput
|\\
|\childdocmain{|\textit{main}|}|\\
\end{tabular}
\end{center}
%
If |\jobname| does not match the argument \textit{main} of |\childdocmain|,
it is assumed that |\jobname| points to the child file to be compiled.
When using |\childdocmain| with the main file specified as argument,
it suffices to start a child file
with just |\input{|\textit{main}|}|
without loading of the package and using |\childdocof|.
If instead all processing is done
with the appropriate \textsf{childdoc} directives,
the argument of \textit{main} of |\childdocmain| can be empty.

An alternative version of the command line processing described
in \secref{sec:commandline} using the detection mechanism reads:
%
\begin{center}
|... -jobname "|\textit{target}|" "|[\textit{flags}]%
[|\def\jobname{|\textit{dest}|}|]|\input{|\textit{main}|}"|
\end{center}

%%%%%%%%%%%%%%%%%%%%%%%%%%%%%%%%%%%%%%%%%%%%%%%%%%%%%%%%%%%%%%%%%%%%%%%%%%%%%%%%
\subsection{Manual Code}
\label{sec:manual}

In case one cannot be certain whether the definitions file |childdoc.def|
is installed on the target \TeX{} distribution
and one prefers not to ship it,
it is conceivable to paste a few relevant commands into the sources.

To that end, drop all statements |% \iffalse
%
% childdoc.dtx Copyright (C) 2017-2018 Niklas Beisert
%
% This work may be distributed and/or modified under the
% conditions of the LaTeX Project Public License, either version 1.3
% of this license or (at your option) any later version.
% The latest version of this license is in
%   http://www.latex-project.org/lppl.txt
% and version 1.3 or later is part of all distributions of LaTeX
% version 2005/12/01 or later.
%
% This work has the LPPL maintenance status `maintained'.
%
% The Current Maintainer of this work is Niklas Beisert.
%
% This work consists of the files childdoc.dtx and childdoc.ins
% and the derived files childdoc.def and cdocsamp.tex with
% cdocsch1.tex, cdocsch2.tex, cdocsdrf.tex, cdocsfn1.tex, cdocsfn2.tex.
%
%<package>\ifdefined\childdocmain\endinput\fi
%<package>\ProvidesFile{childdoc.def}[2018/12/30 v2.0 child document driver]
%<samplemain>\ProvidesFile{cdocsamp.tex}[2018/12/30 v2.0 sample for childdoc]
%<*driver>
%\ProvidesFile{childdoc.drv}[2018/12/30 v2.0 childdoc reference manual file]
\PassOptionsToClass{10pt,a4paper}{article}
\documentclass{ltxdoc}

\usepackage[margin=35mm]{geometry}
\usepackage{hyperref}
\usepackage{hyperxmp}
\usepackage[usenames]{color}

\hypersetup{colorlinks=true}
\hypersetup{pdfstartview=FitH}
\hypersetup{pdfpagemode=UseNone}
\hypersetup{pdfsource={}}
\hypersetup{pdflang={en-UK}}
\hypersetup{pdfcopyright={Copyright 2017-2018 Niklas Beisert.
  This work may be distributed and/or modified under the
  conditions of the LaTeX Project Public License, either version 1.3
  of this license or (at your option) any later version.}}
\hypersetup{pdflicenseurl={http://www.latex-project.org/lppl.txt}}
\hypersetup{pdfcontactaddress={ETH Zurich, ITP, HIT K,
  Wolfgang-Pauli-Strasse 27}}
\hypersetup{pdfcontactpostcode={8093}}
\hypersetup{pdfcontactcity={Zurich}}
\hypersetup{pdfcontactcountry={Switzerland}}
\hypersetup{pdfcontactemail={nbeisert@itp.phys.ethz.ch}}
\hypersetup{pdfcontacturl={http://people.phys.ethz.ch/\xmptilde nbeisert/}}

\newcommand{\secref}[1]{\hyperref[#1]{section \ref*{#1}}}

\parskip1ex
\parindent0pt
\let\olditemize\itemize
\def\itemize{\olditemize\parskip0pt}

\begin{document}

\title{The \textsf{childdoc} Package}
\hypersetup{pdftitle={The childdoc Package}}
\author{Niklas Beisert\\[2ex]
  Institut f\"ur Theoretische Physik\\
  Eidgen\"ossische Technische Hochschule Z\"urich\\
  Wolfgang-Pauli-Strasse 27, 8093 Z\"urich, Switzerland\\[1ex]
  \href{mailto:nbeisert@itp.phys.ethz.ch}
  {\texttt{nbeisert@itp.phys.ethz.ch}}}
\hypersetup{pdfauthor={Niklas Beisert}}
\hypersetup{pdfsubject={Manual for the LaTeX2e Package childdoc}}
\date{30 December 2018, \textsf{v2.0}}
\maketitle

\begin{abstract}\noindent
\textsf{childdoc} is a \LaTeXe{} package
that enables the direct compilation
of document sections included by |\include|
to individual files.
\end{abstract}

\begingroup
\parskip0ex
\tableofcontents
\endgroup

%%%%%%%%%%%%%%%%%%%%%%%%%%%%%%%%%%%%%%%%%%%%%%%%%%%%%%%%%%%%%%%%%%%%%%%%%%%%%%%%
%%%%%%%%%%%%%%%%%%%%%%%%%%%%%%%%%%%%%%%%%%%%%%%%%%%%%%%%%%%%%%%%%%%%%%%%%%%%%%%%
\section{Introduction}

\LaTeX{} provides a mechanism to structure a large document (such as a book)
into a main file and several child files (containing the chapters)
using the |\include| command.
This mechanism is beneficial for documents
which span hundreds of pages in order to
make the source file(s) more manageable.
Moreover, compilation can be restricted to
selected child files by means of the |\includeonly| command.
The latter feature can be used to reduce the compilation time while editing
(this was significantly more useful in the earlier days of \LaTeX{})
or to generate a smaller document which is easier to navigate.
Another application of |\includeonly| is to generate
documents consisting of selected parts of the complete document.

However, there are a few drawbacks of the plain |\include| mechanism:
\begin{itemize}
\item
The child files cannot be compiled on their own,
they can only be compiled via the main file.
A naive editing environment
(such as a text editor with an option
to have the current file processed by \LaTeX)
may require one to switch to the main file before compiling;
attempting to compile the child file produces errors.
\item
The main file must be modified (each time)
to adjust the |\includeonly| command
to the present needs. This easily leaves the main file in a messy state.
\item
The generated document will always carry the filename
of the main document. This is inconvenient if
several child files are to be compiled and
to be kept for distribution.
\end{itemize}

The present package provides a simple interface
to make child files individually compilable by \LaTeX{}.
Compiling a child file then has the same effect as compiling
the main file with an |\includeonly| command
to select the appropriate child.
Moreover the generated document will carry the name of the child
rather than the main file.
This resolves all three above issues.

This feature is meant to make the editing of books,
thesis documents and lecture notes somewhat more convenient.
However, the package can also be used efficiently for
composing a series of documents (such as exercise sheets)
which are typically distributed individually.
It then assists the author in generating the individual documents
(potentially in different versions)
as well as a document containing the collected series.
Another application is in developing style files
or other kinds of included material
where compilation of the style file could redirect
to a sample or test file.

%%%%%%%%%%%%%%%%%%%%%%%%%%%%%%%%%%%%%%%%%%%%%%%%%%%%%%%%%%%%%%%%%%%%%%%%%%%%%%%%
%%%%%%%%%%%%%%%%%%%%%%%%%%%%%%%%%%%%%%%%%%%%%%%%%%%%%%%%%%%%%%%%%%%%%%%%%%%%%%%%
\section{Usage}

First of all, the package \textsf{childdoc} is \emph{not} a standard
\LaTeXe{} |.sty| style file! Therefore it needs to be invoked in
a non-standard way.

%%%%%%%%%%%%%%%%%%%%%%%%%%%%%%%%%%%%%%%%%%%%%%%%%%%%%%%%%%%%%%%%%%%%%%%%%%%%%%%%
\subsection{Included Files}
\label{sec:include}

%%%%%%%%%%%%%%%%%%%%%%%%%%%%%%%%%%%%%%%%
\DescribeMacro{\childdocmain}
To use the package, add the commands
\begin{center}
\begin{tabular}{l}
|\input{childdoc.def}|\\
|\childdocmain{}|\\
\end{tabular}
\end{center}
at the very top of the main \LaTeX{} file,
in particular \emph{before} the |\documentclass| statement!
The argument of |\childdocmain| should be left empty
(but it must be present).

%%%%%%%%%%%%%%%%%%%%%%%%%%%%%%%%%%%%%%%%
\DescribeMacro{\childdocof}
Furthermore, add the commands
\begin{center}
\begin{tabular}{l}
|\input{childdoc.def}|\\
|\childdocof{|\textit{main}|}|\\
\end{tabular}
\end{center}
at the top of every child file \textit{child}
which is included by |\include{|\textit{child}|}|
from within the main file
(or at least for those files to be compiled individually).
The argument \textit{main} must be the filename of the main file.

There are a couple of
considerations in setting up the main and child documents:

%%%%%%%%%%%%%%%%%%%%%%%%%%%%%%%%%%%%%%%%
\paragraph{Restrictions.}

Please note the following restrictions:
\begin{itemize}
\item
|\childdocmain| must be called with one argument \textit{main}
to ensure compatibility with earlier version of the package.
It must either be empty (|\childdocmain{}|)
or precisely match the filename of the main file in which it is specified.
See \secref{sec:detection} for further information.
\item
The filename \textit{main} must be specified without the |.tex| extension.
\item
The filename \textit{main} is case sensitive
(even in case-insensitive file systems)
due to internal string comparison.
\item
The argument \textit{main} should be fully expanded, it cannot be a macro.
\item
Subdirectories and special characters should be avoided in filenames.
\item
The command |\childdocmain{|\textit{main}|}| must be followed by a whitespace.
It should not be followed immediately by another command
or by a comment mark `|%|'.
This is because the \TeX{} parser reads the token immediately following
the argument of |\childdocmain| and puts it
at the beginning of every child section;
however, a white\-space is ignored.
\end{itemize}

%%%%%%%%%%%%%%%%%%%%%%%%%%%%%%%%%%%%%%%%
\paragraph{Content of Main File.}

It is advisable to place all content in the child files included by |\include|.
Any output contained in the main file will appear in all child documents
unless suppressed manually;
it cannot be suppressed automatically by the |\includeonly| directive
and thus should normally be avoided.
A method to include some content in the main file
by means of conditional processing is described in \secref{sec:conditional}.

%%%%%%%%%%%%%%%%%%%%%%%%%%%%%%%%%%%%%%%%
\paragraph{Page Numbering.}

When only a part of the document is compiled,
the appropriate numbering of pages
(as well as other status parameters)
is determined from the |.aux| files.
The latter contain information from previous passes.
However this information needs to propagate through
all intermediate child documents.
Therefore the page numbering in child documents may well
be inconsistent until the complete document is compiled at least once.

A useful (if unconventional) way to always ensure a consistent
page numbering is to restart the numbering in each child document
and denote the pages by `\textit{child}|.|\textit{page}'
where \textit{child} represents the chapter/section number of the child file.
This can be achieved by the command
|\numberwithin{page}{|\textit{child}|}|
of the \textsf{amsmath} package
where \textit{child} can be |chapter| or |section|
depending on the chosen structuring.
Alternatively, one can modify the macro |\thepage| appropriately
and reset the counter |page| at the start of each child file.

%%%%%%%%%%%%%%%%%%%%%%%%%%%%%%%%%%%%%%%%%%%%%%%%%%%%%%%%%%%%%%%%%%%%%%%%%%%%%%%%
\subsection{Conditional Processing}
\label{sec:conditional}

The package provides a mechanism to compile different versions
of a document. To customise the versions further some conditional processing
can come in handy to distinguish which version is being compiled.
The package provides two macros to describe the compilation context:

%%%%%%%%%%%%%%%%%%%%%%%%%%%%%%%%%%%%%%%%
\DescribeMacro{\ifchilddoc}
The conditional |\ifchilddoc| distinguishes between the compilation of
child documents and the main document:
%
\begin{center}
|\ifchilddoc |\textit{child-code}| |[|\||else |\textit{main-code}]| \||fi|
\end{center}

%%%%%%%%%%%%%%%%%%%%%%%%%%%%%%%%%%%%%%%%
\DescribeMacro{\childdocname}
\DescribeMacro{\childdocjob}
The macro |\childdocname| contains the filename (without extension)
of the main or child file being processed.
Note that |\childdocjob| will always contain the name of the main file.

%%%%%%%%%%%%%%%%%%%%%%%%%%%%%%%%%%%%%%%%
\paragraph{Title Page.}

Conditional processing can be used to include a title or banner page
in the main document when proper precautions are taken.
Importantly, the code in the main file should ensure that the page counter
(as well as other status parameters which are stored in the |.aux| files)
takes the same value after the conditional processing.
Otherwise the page numbers may take divergent values
depending on which part is compiled.

For example, a title page could be declared by:
%
\begin{center}
\begin{tabular}{l}
|\ifchilddoc\||else|\\
|\addtocounter{page}{-1}|\\
\textit{code for title page}\\
|\newpage|\\
|\||fi|
\end{tabular}
\end{center}
%
A banner page for the child documents can be generated by:
%
\begin{center}
\begin{tabular}{l}
|\ifchilddoc|\\
|\addtocounter{page}{-1}|\\
\textit{code for banner page}\\
|\newpage|\\
|\||fi|
\end{tabular}
\end{center}
%
Here one could write a message such as:
\begin{center}
|This is the part \childdocname{} of \childdocjob{}.|
\end{center}

%%%%%%%%%%%%%%%%%%%%%%%%%%%%%%%%%%%%%%%%%%%%%%%%%%%%%%%%%%%%%%%%%%%%%%%%%%%%%%%%
\subsection{Flags}
\label{sec:flags}

The package makes it easy to generate different versions
of the main or child documents.
To this end compilation flags can be defined
and assigned different default values.
They will be particularly useful in conjunction
with the forwarding mechanism described in \secref{sec:forward}.

For example, it may be useful to have a flag |\version|
which can be set to |draft| or |final|.
The document source will contain some conditional code
depending on the value of |\version|.
Suppose further, the flag should default to |final| for the main file
and to |draft| for child files
which is a natural assignment for editing the document.
This is achieved by placing the following code
in the preamble of the main document
(below the |\childdocmain| directive):
%
\begin{center}
\begin{tabular}{l}
|\ifchilddoc|\\
|\providecommand{\version}{draft}|\\
|\||else|\\
|\providecommand{\version}{final}|\\
|\||fi|
\end{tabular}
\end{center}
%
The definition by |\providecommand| makes sure
that previous definitions are not overwritten.
Further statements |\providecommand{\version}{...}|
can thus be added before the above code to override it.

For the main file, one might add a line
(between |\childdocmain| and the above block)
%
\begin{center}
|%\ifchilddoc\||else\providecommand{\version}{draft}\||fi|
\end{center}
%
which can be uncommented to produce a draft version.
Likewise one can add a line to the very top of a child file
(above the |\childdocof{|\textit{main}|}| directive)
%
\begin{center}
|%\providecommand{\version}{final}|
\end{center}
%
which can be uncommented to produce the final version of this child document.

%%%%%%%%%%%%%%%%%%%%%%%%%%%%%%%%%%%%%%%%%%%%%%%%%%%%%%%%%%%%%%%%%%%%%%%%%%%%%%%%
\subsection{Forwarding}
\label{sec:forward}

Different versions of the main or child documents
using compilation flags as described in \secref{sec:flags}
can be (permanently) stored in different files
for convenient compilation, viewing and distribution.
To this end, the package defines a command
to pass on compilation to a different file:

%%%%%%%%%%%%%%%%%%%%%%%%%%%%%%%%%%%%%%%%
\DescribeMacro{\childdocforward}
The command |\childdocforward| redirects processing to
another source file:
%
\begin{center}
\begin{tabular}{l}
|\input{childdoc.def}|\\
|\childdocforward[|\textit{main}|]{|\textit{dest}|}|\\
\end{tabular}
\end{center}
%
The argument \textit{dest} is the destination file
(without extension).
It should be the main file or one of the child files.
Note that further \textsf{childdoc} directives
such as |\childdocof| and |\childdocforward|
in the indicated file will be processed in this form.
The optional argument \textit{main}
passes on directly to the main file \textit{main}
while pretending to compile the child \textit{dest}.
This form behaves as if \textit{dest}
issues |\childdocof{|\textit{main}|}| right away,
and no further \textsf{childdoc} directives will be processed.

%%%%%%%%%%%%%%%%%%%%%%%%%%%%%%%%%%%%%%%%
\DescribeMacro{\...prefix}
In the alternative form |\childdocforwardprefix|,
%
\begin{center}
\begin{tabular}{l}
|\input{childdoc.def}|\\
|\childdocforwardprefix[|\textit{main}|]{|\textit{prefix}|}{|\textit{dest}|}|
\end{tabular}
\end{center}
%
the destination file is determined by a pattern
depending on the current file:
To make this work, the current file must be called
`{\textit{prefix}\hspace{0.2em}\textit{suffix}}'
with \textit{prefix} matching precisely the argument.
Processing is then passed on to the file
`{\textit{dest}\hspace{0.2em}\textit{suffix}}'.
Surely, the same effect is achieved by
directly specifying the
argument `{\textit{dest}\hspace{0.2em}\textit{suffix}}'
in the first form.
However, that requires to set up a different file
for each child. With the alternative form of the command
all these files can have exactly the same content
which simplifies setting them up and maintaining them.

For example, the following file |draft.tex|
with a compilation flag |\version| as described in \secref{sec:flags}
compiles the main document as a draft:
%
\begin{center}
\begin{tabular}{l}
|\def\version{draft}|\\
|\input{childdoc.def}|\\
|\childdocforward{|\textit{main}|}|
\end{tabular}
\end{center}
%
Likewise, the following files |final|\textit{nn}|.tex|
compile the final version of the child document
|child|\textit{nn}|.tex|:
%
\begin{center}
\begin{tabular}{l}
|\def\version{final}|\\
|\input{childdoc.def}|\\
|\childdocforwardprefix{final}{child}|
\end{tabular}
\end{center}
%

Note that when several versions of a main file and/or of each child file
are to be generated, it may be convenient to set up a |Makefile| or
shell script to automatise the process.

%%%%%%%%%%%%%%%%%%%%%%%%%%%%%%%%%%%%%%%%%%%%%%%%%%%%%%%%%%%%%%%%%%%%%%%%%%%%%%%%
\subsection{Command Line Processing}
\label{sec:commandline}

The effect of redirection files can also be achieved by invoking
the \LaTeX{} compiler with a more elaborate command line.
Most conveniently this should be done as part
of a shell script or a |Makefile|.

When using \textsf{childdoc} in the main file, the following
command lines effectively perform a redirection
(note that depending on the shell being used,
backslashes may have to be doubled: `|\|' $\to$ `|\\|'):
%
\begin{center}
|... -jobname "|\textit{target}|" |\\|"|[\textit{flags}]%
|\input{childdoc.def}\childdocforward[|\textit{main}|]{|\textit{dest}|}"|
\end{center}
%
Here \textit{target} is the name of the output file,
\textit{main} is the name of the main file
and \textit{dest} is the name of the main or child file to be processed
(all filenames without extensions).
The optional argument \textit{main} can be omitted
if \textit{main} matches \textit{dest}.
Optionally, compilation \textit{flags} can be defined via |\def| commands.
This command line makes the \TeX{} engine believe
it is compiling the file \textit{target}
whose content is specified as the latter parameter.
The provided code then forwards the processing to
\textit{main} or \textit{dest} as described in \secref{sec:forward}.

%%%%%%%%%%%%%%%%%%%%%%%%%%%%%%%%%%%%%%%%%%%%%%%%%%%%%%%%%%%%%%%%%%%%%%%%%%%%%%%%
\subsection{Include by Input}
\label{sec:input}

Including child documents by |\include| has some restrictions by design.
Most notably, the content of a child document always occupies
its own set of pages; pages cannot be shared between child documents.
Usually, this behaviour makes perfect sense
because each child document contain an essential part of the document.
However, in some situations it may be desirable to compose
a document from a collection of parts
without having mandatory page breaks between then.
For this case, the package
provides a mechanism to include parts
by |\input| which can also be processed individually.
However, by construction this mechanism
requires manual handling of the content to be output.

%%%%%%%%%%%%%%%%%%%%%%%%%%%%%%%%%%%%%%%%
\DescribeMacro{\ifchilddocmanual}
The main file should be prepared as usual, see \secref{sec:include}.
However, the document body must make a distinction
between processing of an individual part and of the main document, e.g.:
%
\begin{center}
\begin{tabular}{l}
|\ifchilddocmanual|\\
|\input{\childdocname}|\\
|\||else|\\
\textit{document body with }|\input{|\textit{part}|}|\\
|\||fi|
\end{tabular}
\end{center}
%
The conditional |\ifchilddocmanual| is true whenever
a part to be included by |\input| is being compiled,
and the name of the part is stored in |\childdocname|.

%%%%%%%%%%%%%%%%%%%%%%%%%%%%%%%%%%%%%%%%
\DescribeMacro{\childdocby}
Each part to be included by |\input| should start with:
%
\begin{center}
\begin{tabular}{l}
|\input{childdoc.def}|\\
|\childdocby{|\textit{main}|}|\\
\end{tabular}
\end{center}
%
The directive |\childdocby| is similar to |\childdocof|
described in \secref{sec:include},
but the subsequent selection of content must be done manually.
To that end, both |\ifchilddoc| and |\ifchilddocmanual|
will be true upon processing of a part,
and the name of the part is stored in |\childdocname|.
Note that |\jobname| will be set to the filename of the current part
so that each part receives an individual |.aux| file
that does not interfere with the |.aux| file(s) of the main document.
This behaviour can be altered by the alternative form
|\childdocby[*]{|\textit{main}|}| (with a non-empty optional argument)
which uses the |.aux| file of the main document
by setting |\jobname| to \textit{main}.

%%%%%%%%%%%%%%%%%%%%%%%%%%%%%%%%%%%%%%%%%%%%%%%%%%%%%%%%%%%%%%%%%%%%%%%%%%%%%%%%
\subsection{Driver Development}
\label{sec:driver}

The \textsf{childdoc} mechanism can also be use for the development
of definition files such as \LaTeX{} styles or classes.
This case differs from the above setup with multiple parts
included by |\include| in that no |\includeonly| should be invoked.
This can be achieved by starting the include file
(before |\ProvidesPackage|) with:
%
\begin{center}
\begin{tabular}{l}
|\input{childdoc.def}|\\
|\childdocforward{|\textit{main}|}|\\
\end{tabular}
\end{center}
%
or alternatively with:
%
\begin{center}
\begin{tabular}{l}
|\input{childdoc.def}|\\
|\childdocby{|\textit{main}|}|\\
\end{tabular}
\end{center}
%
Both forms have slightly different effects as described above.
The main file is prepared as usual, see \secref{sec:include}.

%%%%%%%%%%%%%%%%%%%%%%%%%%%%%%%%%%%%%%%%%%%%%%%%%%%%%%%%%%%%%%%%%%%%%%%%%%%%%%%%
\subsection{Legacy Detection}
\label{sec:detection}

The directive |\childdocmain| in the main file can detect
whether the complete document or merely a child is to be compiled
even without using the directive |\childdocof|.
This method is deprecated because it is less robust
and there is no compelling reason to use it;
it is merely provided for backward compatibility
and it may be removed in future versions.

If the detection mechanism is to be used,
it is mandatory to correctly specify
the filename of the main file as the argument of |\childdocmain|:
%
\begin{center}
\begin{tabular}{l}
|\input{childdoc.def}|\\
|\childdocmain{|\textit{main}|}|\\
\end{tabular}
\end{center}
%
If |\jobname| does not match the argument \textit{main} of |\childdocmain|,
it is assumed that |\jobname| points to the child file to be compiled.
When using |\childdocmain| with the main file specified as argument,
it suffices to start a child file
with just |\input{|\textit{main}|}|
without loading of the package and using |\childdocof|.
If instead all processing is done
with the appropriate \textsf{childdoc} directives,
the argument of \textit{main} of |\childdocmain| can be empty.

An alternative version of the command line processing described
in \secref{sec:commandline} using the detection mechanism reads:
%
\begin{center}
|... -jobname "|\textit{target}|" "|[\textit{flags}]%
[|\def\jobname{|\textit{dest}|}|]|\input{|\textit{main}|}"|
\end{center}

%%%%%%%%%%%%%%%%%%%%%%%%%%%%%%%%%%%%%%%%%%%%%%%%%%%%%%%%%%%%%%%%%%%%%%%%%%%%%%%%
\subsection{Manual Code}
\label{sec:manual}

In case one cannot be certain whether the definitions file |childdoc.def|
is installed on the target \TeX{} distribution
and one prefers not to ship it,
it is conceivable to paste a few relevant commands into the sources.

To that end, drop all statements |\input{childdoc.def}|
and perform the replacements as outlined below.
Instead of |\childdocmain{|\textit{main}|}| add the following code
to the top of the main file:
%
\begin{center}
\begin{tabular}{l}
|\||ifdefined\childdocname\endinput\||fi\newif\ifchilddoc|\\
|\edef\childdocname{\scantokens\expandafter{\jobname\noexpand}}|\\
|\def\childdocmain{|\textit{main}|}\||ifx\childdocmain\childdocname\||else|\\
|\childdoctrue\includeonly{\childdocname}\let\jobname\childdocmain\||fi|\\
\end{tabular}
\end{center}
%
Instead of |\childdocof{|\textit{main}|}| just include the main file
at the top of each child file:
%
\begin{center}
|\input{|\textit{main}|}|
\end{center}
%
A simple redirection |\childdocforward{|\textit{dest}|}| is achieved by:
%
\begin{center}
|\def\jobname{|\textit{dest}|}\input{\jobname}|
\end{center}
%
The redirection with prefix
|\childdocforwardprefix[|\textit{prefix}|]{|\textit{dest}|}|
is accomplished by:
%
\begin{center}
\begin{tabular}{l}
|{\edef\jobname{\scantokens\expandafter{\jobname\noexpand}}|\\
|\def\redirectjob |\textit{prefix}|#1~~~{\gdef\jobname{|\textit{dest}|#1}}|\\
|\expandafter\redirectjob\jobname~~~}\input{\jobname}|
\end{tabular}
\end{center}

In an alternative approach,
child documents can be compiled by a specific command line
without additional code or specific definitions:
%
\begin{center}
|... -jobname "|\textit{target}|" "|[\textit{flags}]%
|\includeonly{|\textit{dest}|}\input{|\textit{main}|}"|
\end{center}
%

%%%%%%%%%%%%%%%%%%%%%%%%%%%%%%%%%%%%%%%%%%%%%%%%%%%%%%%%%%%%%%%%%%%%%%%%%%%%%%%%
%%%%%%%%%%%%%%%%%%%%%%%%%%%%%%%%%%%%%%%%%%%%%%%%%%%%%%%%%%%%%%%%%%%%%%%%%%%%%%%%
\section{Information}

%%%%%%%%%%%%%%%%%%%%%%%%%%%%%%%%%%%%%%%%%%%%%%%%%%%%%%%%%%%%%%%%%%%%%%%%%%%%%%%%
\subsection{Copyright}

Copyright \copyright{} 2017--2018 Niklas Beisert

This work may be distributed and/or modified under the
conditions of the \LaTeX{} Project Public License, either version 1.3
of this license or (at your option) any later version.
The latest version of this license is in
  \url{http://www.latex-project.org/lppl.txt}
and version 1.3 or later is part of all distributions of \LaTeX{}
version 2005/12/01 or later.

This work has the LPPL maintenance status `maintained'.

The Current Maintainer of this work is Niklas Beisert.

This work consists of the files |README.txt|, |childdoc.ins| and |childdoc.dtx|
as well as the derived files |childdoc.def|, |cdocsamp.tex|
with |cdocsch1.tex|, |cdocsch2.tex|, |cdocspt3.tex|, |cdocspt4.tex|,
|cdocsdrf.tex|, |cdocsfn1.tex|, |cdocsfn2.tex|
as well as |childdoc.pdf|.

%%%%%%%%%%%%%%%%%%%%%%%%%%%%%%%%%%%%%%%%%%%%%%%%%%%%%%%%%%%%%%%%%%%%%%%%%%%%%%%%
\subsection{Files and Installation}

The package consists of the files:
%
\begin{center}
\begin{tabular}{ll}
    |README.txt|   & readme file \\
    |childdoc.ins| & installation file \\
    |childdoc.dtx| & source file \\
    |childdoc.def| & definition file \\
    |cdocsamp.tex| & sample main file \\
    |cdocsch1.tex| & sample include file \\
    |cdocsch2.tex| & sample include file \\
    |cdocspt3.tex| & sample part file \\
    |cdocspt4.tex| & sample part file \\
    |cdocsdrf.tex| & sample redirection file \\
    |cdocsfn1.tex| & sample redirection file \\
    |cdocsfn2.tex| & sample redirection file \\
    |childdoc.pdf| & manual
\end{tabular}
\end{center}
%
The distribution consists of the files
|README.txt|, |childdoc.ins| and |childdoc.dtx|.
%
\begin{itemize}
\item
Run (pdf)\LaTeX{} on |childdoc.dtx|
to compile the manual |childdoc.pdf| (this file).
\item
Run \LaTeX{} on |childdoc.ins| to create the definitions file |childdoc.def|
and the sample |cdocsamp.tex| with include files
|cdocsch1.tex|, |cdocsch2.tex|, |cdocspt3.tex|, |cdocspt4.tex|,
|cdocsdrf.tex|, |cdocsfn1.tex|, |cdocsfn2.tex|.
Then copy the file |childdoc.def| to an appropriate directory of your \LaTeX{}
distribution, e.g.\ \textit{texmf-root}|/tex/latex/childdoc|.
\end{itemize}

%%%%%%%%%%%%%%%%%%%%%%%%%%%%%%%%%%%%%%%%%%%%%%%%%%%%%%%%%%%%%%%%%%%%%%%%%%%%%%%%
\subsection{Related CTAN Packages}

There are several other packages which offer a similar functionality:
%
\begin{itemize}
\item
The packages
\href{http://ctan.org/pkg/docmute}{\textsf{docmute}},
\href{http://ctan.org/pkg/includex}{\textsf{includex}} and
\href{http://ctan.org/pkg/standalone}{\textsf{standalone}}
provide commands to include only the document body of
a child file thus allowing both files to be compiled individually.
\item
The packages \href{http://ctan.org/pkg/subdocs}{\textsf{subdocs}}
and \href{http://ctan.org/pkg/subfiles}{\textsf{subfiles}}
provide structures in which the main and child documents can be
encapsulated and allowing them to be compiled individually.
The inclusion mechanism is different from the conventional |\include|.
\item
The package \href{http://ctan.org/pkg/combine}{\textsf{combine}}
is an elaborate solution to combine several documents into one.
\end{itemize}
%
See also the CTAN topic \href{http://ctan.org/topic/subdocs}{\textsf{subdocs}}
for further related packages.
The present package differs from the above solutions in that
a document structure constructed with the conventional |\include| mechanism
just needs two extra commands at the top of every file
such that all constituent files can be compiled individually.

%%%%%%%%%%%%%%%%%%%%%%%%%%%%%%%%%%%%%%%%%%%%%%%%%%%%%%%%%%%%%%%%%%%%%%%%%%%%%%%%
%\subsection{Feature Suggestions}
%
%The following is a list of features which may be useful for future
%versions of this package:
%%
%\begin{itemize}
%\item
%\ldots
%\end{itemize}

%%%%%%%%%%%%%%%%%%%%%%%%%%%%%%%%%%%%%%%%%%%%%%%%%%%%%%%%%%%%%%%%%%%%%%%%%%%%%%%%
\subsection{Revision History}

%%%%%%%%%%%%%%%%%%%%%%%%%%%%%%%%%%%%%%%%
\paragraph{v2.0:} 2018/12/30

\begin{itemize}
\item
immediate forward processing
\item
added |\childdocby| mechanism
\item
manual restructured
\end{itemize}

%%%%%%%%%%%%%%%%%%%%%%%%%%%%%%%%%%%%%%%%
\paragraph{v1.6:} 2018/01/17

\begin{itemize}
\item
application for development of include files
\item
corrections to manual
\end{itemize}

%%%%%%%%%%%%%%%%%%%%%%%%%%%%%%%%%%%%%%%%
\paragraph{v1.5:} 2017/05/21

\begin{itemize}
\item
more complete structuring introduced
\item
|\childdocof| introduced
\item
|\childdoc| renamed to |\childdocmain|
\item
|\childredirect| renamed to |\childdocforward| and |\childdocforwardprefix|
and functionality expanded
\end{itemize}

%%%%%%%%%%%%%%%%%%%%%%%%%%%%%%%%%%%%%%%%
\paragraph{v1.0:} 2017/04/27

\begin{itemize}
\item
manual and install package
\item
first version published on CTAN
\end{itemize}

%%%%%%%%%%%%%%%%%%%%%%%%%%%%%%%%%%%%%%%%
\paragraph{v0.6:} 2017/04/26

\begin{itemize}
\item
redirection mechanism added
\end{itemize}

%%%%%%%%%%%%%%%%%%%%%%%%%%%%%%%%%%%%%%%%
\paragraph{v0.5:} 2017/04/26

\begin{itemize}
\item
functionality in definition file
\end{itemize}


%%%%%%%%%%%%%%%%%%%%%%%%%%%%%%%%%%%%%%%%%%%%%%%%%%%%%%%%%%%%%%%%%%%%%%%%%%%%%%%%
%%%%%%%%%%%%%%%%%%%%%%%%%%%%%%%%%%%%%%%%%%%%%%%%%%%%%%%%%%%%%%%%%%%%%%%%%%%%%%%%
%%%%%%%%%%%%%%%%%%%%%%%%%%%%%%%%%%%%%%%%%%%%%%%%%%%%%%%%%%%%%%%%%%%%%%%%%%%%%%%%
\appendix

\settowidth\MacroIndent{\rmfamily\scriptsize 000\ }

 \DocInput{childdoc.dtx}

\end{document}
%</driver>
% \fi
%
% %%%%%%%%%%%%%%%%%%%%%%%%%%%%%%%%%%%%%%%%%%%%%%%%%%%%%%%%%%%%%%%%%%%%%%%%%%%%%%
% %%%%%%%%%%%%%%%%%%%%%%%%%%%%%%%%%%%%%%%%%%%%%%%%%%%%%%%%%%%%%%%%%%%%%%%%%%%%%%
% \section{Sample}
%\iffalse
%<*samplemain>
%\fi
%
% The following presents a sample document
% with two chapters, two parts, a title page,
% a compile flag as well as three forwarding files to set the flag.
% It consists of eight |.tex| files:
% \begin{center}
% \begin{tabular}{ll}
% |cdocsamp.tex|&main file\\
% |cdocsch1.tex|&include file for chapter 1\\
% |cdocsch2.tex|&include file for chapter 2\\
% |cdocspt3.tex|&include file for part 3\\
% |cdocspt4.tex|&include file for part 4\\
% |cdocsdrf.tex|&forwarding file for main file in draft mode\\
% |cdocsfi1.tex|&forwarding file for final version of chapter 1\\
% |cdocsfi2.tex|&forwarding file for final version of chapter 2\\
% \end{tabular}
% \end{center}
% Each of the eight files can be compiled directly by the \LaTeX{} compiler.
%
% %%%%%%%%%%%%%%%%%%%%%%%%%%%%%%%%%%%%%%
% \paragraph{Main File.}
%
% The main file is called |cdocsamp.tex|.
%
% Load the \textsf{childdoc} definitions and
% declare the filename for the main document:
%    \begin{macrocode}
\input{childdoc.def}
\childdocmain{}
%    \end{macrocode}

% Optional override for |\version| flag:
%    \begin{macrocode}
%%\ifchilddoc\else\providecommand{\version}{draft}\fi
%    \end{macrocode}

% Define the default values for the |\version| flag
% (|final| for the main file and |draft| for childs):
%    \begin{macrocode}
\ifchilddoc
\providecommand{\version}{draft}
\else
\providecommand{\version}{final}
\fi
%    \end{macrocode}

% Load the standard document class:
%    \begin{macrocode}
\documentclass[12pt]{article}
%    \end{macrocode}

% Start the document body:
%    \begin{macrocode}
\begin{document}
%    \end{macrocode}

% Declare a title page.
% Print title, part of document being processed and version flag:
%    \begin{macrocode}
\addtocounter{page}{-1}
\begin{center}
{\LARGE\bfseries{}childdoc example\par}
\vspace{1cm}
\ifchilddoc
\ifchilddocmanual part\else chapter\fi:
`\childdocname' of `\childdocjob'\par
\else
main document: `\childdocjob'\par
\fi
version: \version\par
\end{center}
\newpage
%    \end{macrocode}

% Manually include selected file,
% otherwise process as usual:
%    \begin{macrocode}
\ifchilddocmanual
\section*{part `\childdocname'}
\input{\childdocname}
\else
%    \end{macrocode}

% Include the two chapters:
%    \begin{macrocode}
\include{cdocsch1}
\include{cdocsch2}
%    \end{macrocode}

% Include the two parts unless only chapters should be displayed:
%    \begin{macrocode}
\ifchilddoc\else
\section{part three}
\input{cdocspt3}
\section{part four}
\input{cdocspt4}
\fi
%    \end{macrocode}

% Process as usual until here:
%    \begin{macrocode}
\fi
%    \end{macrocode}

% End of document body:
%    \begin{macrocode}
\end{document}
%    \end{macrocode}
%\iffalse
%</samplemain>
%\fi
%
% %%%%%%%%%%%%%%%%%%%%%%%%%%%%%%%%%%%%%%
% \paragraph{Chapter Include Files.}
%
% The include files are called |cdocsch1.tex| and |cdocsch2.tex|.
%
%\iffalse
%<*samplechap1|samplechap2>
%\fi

% Optional override for |\version| flag:
%    \begin{macrocode}
%%\providecommand{\version}{final}
%    \end{macrocode}

% Include the main document:
%    \begin{macrocode}
\input{childdoc.def}
\childdocof{cdocsamp}
%    \end{macrocode}

%\iffalse
%</samplechap1|samplechap2>
%\fi
%
%\iffalse
%<*samplechap1>
%\fi
% Some text for chapter 1:
%    \begin{macrocode}
\section{one}
some text in chapter one
%    \end{macrocode}

%\iffalse
%</samplechap1>
%\fi
% Some text for chapter 2:
%\iffalse
%<*samplechap2>
%\fi
%    \begin{macrocode}
\section{two}
more text in chapter two
%    \end{macrocode}

%\iffalse
%</samplechap2>
%\fi
%
% %%%%%%%%%%%%%%%%%%%%%%%%%%%%%%%%%%%%%%
% \paragraph{Part Include Files.}
%
% The include files are called |cdocspt3.tex| and |cdocspt4.tex|.
%
%\iffalse
%<*samplepart3|samplepart4>
%\fi

% Optional override for |\version| flag:
%    \begin{macrocode}
%%\providecommand{\version}{final}
%    \end{macrocode}

% Include the main document:
%    \begin{macrocode}
\input{childdoc.def}
\childdocby{cdocsamp}
%    \end{macrocode}

%\iffalse
%</samplepart3|samplepart4>
%\fi
%
%\iffalse
%<*samplepart3>
%\fi
% Some text for part 3:
%    \begin{macrocode}
some text in part three
%    \end{macrocode}

%\iffalse
%</samplepart3>
%\fi
% Some text for part 4:
%\iffalse
%<*samplepart4>
%\fi
%    \begin{macrocode}
more text in part four
%    \end{macrocode}

%\iffalse
%</samplepart4>
%\fi
%
% %%%%%%%%%%%%%%%%%%%%%%%%%%%%%%%%%%%%%%
% \paragraph{Forwarding for a Complete Draft.}
%
% The following forwarding file |cdocsdrf.tex|
% compiles the main document in draft mode:
%\iffalse
%<*sampledraft>
%\fi
%    \begin{macrocode}
\def\version{draft}
\input{childdoc.def}
\childdocforward{cdocsamp}
%    \end{macrocode}

%\iffalse
%</sampledraft>
%\fi
%
% %%%%%%%%%%%%%%%%%%%%%%%%%%%%%%%%%%%%%%
% \paragraph{Forwarding for Final Version of the Chapters.}
%
% The following forwarding files |cdocsfn1.tex| and |cdocsfn2.tex|
% (with identical content)
% compile the final versions of the child documents
% |cdocsch1.tex| and |cdocsch2.tex|, respectively:
%\iffalse
%<*samplefinal>
%\fi
%    \begin{macrocode}
\def\version{final}
\input{childdoc.def}
\childdocforwardprefix[cdocsamp]{cdocsfn}{cdocsch}
%    \end{macrocode}

%\iffalse
%</samplefinal>
%\fi
%
% %%%%%%%%%%%%%%%%%%%%%%%%%%%%%%%%%%%%%%
% \paragraph{Command Line Processing.}
%
% The following three command lines generate the output files
% |cdocscld|, |cdocscl1| and |cdocscl2|
% which should be identical to
% |cdocsdrf|, |cdocsch1| and |cdocsfn2|, respectively:
% \begin{center}
% \begin{tabular}{l}
% |latex -jobname cdocscld \|\\
% |  "\def\version{draft}\input{childdoc.def}\childdocforward{cdocsamp}"|\\
% |latex -jobname cdocscl1 \|\\
% |  "\input{childdoc.def}\childdocforward[cdocsamp]{cdocsch1}"|\\
% |latex -jobname cdocscl2 \|\\
% |  "\def\version{final}\input{childdoc.def}\childdocforward{cdocsch2}"|
% \end{tabular}
% \end{center}
% Note that the trailing backslash on each first line
% merely continues the input to the second line
% (for convenient cut ant paste).
% Furthermore, the command |latex| can be replaced by any
% of its alternative versions such as |pdflatex|.
%
% %%%%%%%%%%%%%%%%%%%%%%%%%%%%%%%%%%%%%%%%%%%%%%%%%%%%%%%%%%%%%%%%%%%%%%%%%%%%%%
% %%%%%%%%%%%%%%%%%%%%%%%%%%%%%%%%%%%%%%%%%%%%%%%%%%%%%%%%%%%%%%%%%%%%%%%%%%%%%%
% \section{Implementation}
%\iffalse
%<*package>
%\fi
%
% This section describes the definitions file |childdoc.def|.

% The definitions cannot be loaded using |\usepackage| or |\RequirePackage|
% which has a mechanism to prevent loading a style file more than once.
% When loading the definitions by means of |\input|
% multiple instances have to be prevented manually:
%\iffalse
%This code needs to be before the `\ProvidesFile' directive
%which is defined at the beginning of this file.
%Therefore it is also placed there and commented out here.
%</package>
%<*discard>
%\fi
%    \begin{macrocode}
\ifdefined\childdocmain\endinput\fi
%    \end{macrocode}
%\iffalse
%</discard>
%<*package>
%\fi
%
% \macro{\ifchilddoc}
% \macro{\ifchilddocmanual}
% The conditional |\ifchilddoc| tells whether a
% child (true) or main (false) document is being compiled.
% The conditional |\ifchilddocmanual| tells whether
% the |\includeonly| mechanism is used (false) or
% the selection of child files must be performed manually (true).
% The definitions initialise to false:
%    \begin{macrocode}
\newif\ifchilddoc
\newif\ifchilddocmanual
%    \end{macrocode}

% \macro{\childdocname}
% \macro{\childdocjob}
% The macro |\childdocname| stores the name of the main document
% to be compiled. The macro |\childdocjob| stores the name of
% the document on which the \LaTeX{} compiler was originally invoked.
% The content of |\jobname| cannot be compared
% to filenames specified in the source due to different catcodes.
% The following code rescans |\jobname|, stores the result
% in |\childdocname| and saves a copy in |\childdocjob|:
%    \begin{macrocode}
\edef\childdocname{\scantokens\expandafter{\jobname\noexpand}}
\let\childdocjob\childdocname
%    \end{macrocode}

% \macro{\childdocdisable}
% The macro |\childdocdisable| prevents the main file
% from being processed more than once.
% At this stage, the main document command |\childdocmain|
% is assumed to be called once again where it should do nothing.
% Any subsequent call to it should prevent
% a secondary processing of the main document
% It overwrites the forwarding commands
% |\childdocof| and |\childdocforward|
% with empty macros to prevent further inclusions of the main document:
%    \begin{macrocode}
\newcommand{\childdocdisable}
{
  \renewcommand{\childdocmain}[1]{\renewcommand{\childdocmain}[1]{\endinput}}
  \renewcommand{\childdocof}[1]{}
  \renewcommand{\childdocby}[2][]{}
  \renewcommand{\childdocforward}[2][]{}
  \renewcommand{\childdocdisable}{}
}
%    \end{macrocode}

% \macro{\childdocmain}
% The macro |\childdocmain| is to be called at the top of the main file
% with nothing or the main filename (without extension) as argument.
% First, it breaks loops.
% If the argument is not empty and does not match |\childdocname|
% (which is set by the first inclusion of |childdoc.def|),
% |\ifchilddoc| is set to true, |\includeonly| is applied to the child file
% and |\jobname| is set to the main file
% (for proper handling of |.aux| files):
%    \begin{macrocode}
\newcommand{\childdocmain}[1]
{
  \childdocdisable\childdocmain{}
  \if?#1?\else
    \begingroup
      \def\childdoctmp{#1}
      \ifx\childdoctmp\childdocname
        \def\childdoctmp{}
      \else
        \def\childdoctmp
        {
          \childdoctrue
          \includeonly{\childdocname}
          \def\childdocjob{#1}
          \def\jobname{#1}
        }
      \fi
      \expandafter
    \endgroup
    \childdoctmp
  \fi
}
%    \end{macrocode}

% \macro{\childdocof}
% The command |\childdocof| redirects
% compilation to the main file |#1|.
%    \begin{macrocode}
\newcommand{\childdocof}[1]
{
  \childdocdisable
  \childdoctrue
  \includeonly{\childdocname}
  \def\jobname{#1}
  \def\childdocjob{#1}
  \input{#1}
}
%    \end{macrocode}

% \macro{\childdocby}
% The command |\childdocby| ....
%    \begin{macrocode}
\newcommand{\childdocby}[2][]
{
  \childdocdisable
  \childdoctrue
  \childdocmanualtrue
  \if?#1?\else
    \def\jobname{#2}
  \fi
  \def\childdocjob{#2}
  \input{#2}
  \endinput
}
%    \end{macrocode}

% \macro{\childdocforward}
% The command |\childdocforward| redirects
% compilation to the main file or
% (if the optional argument is given) a child file.
% Parameters are set as if the main file
% or a child file starting with |\childdocof| was compiled.
% Then compilation is handed over to the main file:
%    \begin{macrocode}
\newcommand{\childdocforward}[2][]
{
  \begingroup
    \if?#1?
      \def\childdoctmp
      {
        \def\childdocname{#2}
        \def\childdocjob{#2}
        \def\jobname{#2}
        \input{#2}
        \endinput
      }
    \else
      \def\childdoctmp
      {
        \childdocdisable
        \def\childdocname{#2}
        \childdoctrue
        \includeonly{#2}
        \def\childdocjob{#1}
        \def\jobname{#1}
        \input{#1}
        \endinput
      }
    \fi
    \expandafter
  \endgroup
  \childdoctmp
}
%    \end{macrocode}

% \macro{\childdocforwardprefix}
% The command |\childdocforwardprefix| redirects
% compilation to the main or a child file by means of a pattern.
% The prefix |#1| in the current filename is replaced by |#2|
% and the suffix of the current filename is kept
% (it is assumed that the filename does not contain the substring `|~~~|'
% which is used as a delimiter).
% Compilation is handed over to the new file by |\childdocforward|:
%    \begin{macrocode}
\newcommand{\childdocforwardprefix}[3][]
{
  \begingroup
    \def\childdocextract #2##1~~~{\def\childdoctmp{\childdocforward[#1]{#3##1}}}
    \expandafter\childdocextract\childdocname~~~
    \expandafter
  \endgroup
  \childdoctmp
}
%    \end{macrocode}

% \macro{\childdoc}
% The deprecated macro |\childdoc| is a legacy version of |\childdocmain|:
%    \begin{macrocode}
\newcommand{\childdoc}{\childdocmain}
%    \end{macrocode}

% \macro{\childdocredirect}
% The deprecated macro |\childdocredirect| is a legacy version
% of |\childdocforward| and |\childdocforwardprefix|:
%    \begin{macrocode}
\newcommand{\childdocredirect}[2][]
{
  \begingroup
    \if?#1?
      \def\childdoctmp{\childdocforward{#2}}
    \else
      \def\childdoctmp{\childdocforwardprefix{#1}{#2}}
    \fi
    \expandafter
  \endgroup
  \childdoctmp
}
%    \end{macrocode}

%\iffalse
%</package>
%\fi
%
\endinput
|
and perform the replacements as outlined below.
Instead of |\childdocmain{|\textit{main}|}| add the following code
to the top of the main file:
%
\begin{center}
\begin{tabular}{l}
|\||ifdefined\childdocname\endinput\||fi\newif\ifchilddoc|\\
|\edef\childdocname{\scantokens\expandafter{\jobname\noexpand}}|\\
|\def\childdocmain{|\textit{main}|}\||ifx\childdocmain\childdocname\||else|\\
|\childdoctrue\includeonly{\childdocname}\let\jobname\childdocmain\||fi|\\
\end{tabular}
\end{center}
%
Instead of |\childdocof{|\textit{main}|}| just include the main file
at the top of each child file:
%
\begin{center}
|\input{|\textit{main}|}|
\end{center}
%
A simple redirection |\childdocforward{|\textit{dest}|}| is achieved by:
%
\begin{center}
|\def\jobname{|\textit{dest}|}\input{\jobname}|
\end{center}
%
The redirection with prefix
|\childdocforwardprefix[|\textit{prefix}|]{|\textit{dest}|}|
is accomplished by:
%
\begin{center}
\begin{tabular}{l}
|{\edef\jobname{\scantokens\expandafter{\jobname\noexpand}}|\\
|\def\redirectjob |\textit{prefix}|#1~~~{\gdef\jobname{|\textit{dest}|#1}}|\\
|\expandafter\redirectjob\jobname~~~}\input{\jobname}|
\end{tabular}
\end{center}

In an alternative approach,
child documents can be compiled by a specific command line
without additional code or specific definitions:
%
\begin{center}
|... -jobname "|\textit{target}|" "|[\textit{flags}]%
|\includeonly{|\textit{dest}|}\input{|\textit{main}|}"|
\end{center}
%

%%%%%%%%%%%%%%%%%%%%%%%%%%%%%%%%%%%%%%%%%%%%%%%%%%%%%%%%%%%%%%%%%%%%%%%%%%%%%%%%
%%%%%%%%%%%%%%%%%%%%%%%%%%%%%%%%%%%%%%%%%%%%%%%%%%%%%%%%%%%%%%%%%%%%%%%%%%%%%%%%
\section{Information}

%%%%%%%%%%%%%%%%%%%%%%%%%%%%%%%%%%%%%%%%%%%%%%%%%%%%%%%%%%%%%%%%%%%%%%%%%%%%%%%%
\subsection{Copyright}

Copyright \copyright{} 2017--2018 Niklas Beisert

This work may be distributed and/or modified under the
conditions of the \LaTeX{} Project Public License, either version 1.3
of this license or (at your option) any later version.
The latest version of this license is in
  \url{http://www.latex-project.org/lppl.txt}
and version 1.3 or later is part of all distributions of \LaTeX{}
version 2005/12/01 or later.

This work has the LPPL maintenance status `maintained'.

The Current Maintainer of this work is Niklas Beisert.

This work consists of the files |README.txt|, |childdoc.ins| and |childdoc.dtx|
as well as the derived files |childdoc.def|, |cdocsamp.tex|
with |cdocsch1.tex|, |cdocsch2.tex|, |cdocspt3.tex|, |cdocspt4.tex|,
|cdocsdrf.tex|, |cdocsfn1.tex|, |cdocsfn2.tex|
as well as |childdoc.pdf|.

%%%%%%%%%%%%%%%%%%%%%%%%%%%%%%%%%%%%%%%%%%%%%%%%%%%%%%%%%%%%%%%%%%%%%%%%%%%%%%%%
\subsection{Files and Installation}

The package consists of the files:
%
\begin{center}
\begin{tabular}{ll}
    |README.txt|   & readme file \\
    |childdoc.ins| & installation file \\
    |childdoc.dtx| & source file \\
    |childdoc.def| & definition file \\
    |cdocsamp.tex| & sample main file \\
    |cdocsch1.tex| & sample include file \\
    |cdocsch2.tex| & sample include file \\
    |cdocspt3.tex| & sample part file \\
    |cdocspt4.tex| & sample part file \\
    |cdocsdrf.tex| & sample redirection file \\
    |cdocsfn1.tex| & sample redirection file \\
    |cdocsfn2.tex| & sample redirection file \\
    |childdoc.pdf| & manual
\end{tabular}
\end{center}
%
The distribution consists of the files
|README.txt|, |childdoc.ins| and |childdoc.dtx|.
%
\begin{itemize}
\item
Run (pdf)\LaTeX{} on |childdoc.dtx|
to compile the manual |childdoc.pdf| (this file).
\item
Run \LaTeX{} on |childdoc.ins| to create the definitions file |childdoc.def|
and the sample |cdocsamp.tex| with include files
|cdocsch1.tex|, |cdocsch2.tex|, |cdocspt3.tex|, |cdocspt4.tex|,
|cdocsdrf.tex|, |cdocsfn1.tex|, |cdocsfn2.tex|.
Then copy the file |childdoc.def| to an appropriate directory of your \LaTeX{}
distribution, e.g.\ \textit{texmf-root}|/tex/latex/childdoc|.
\end{itemize}

%%%%%%%%%%%%%%%%%%%%%%%%%%%%%%%%%%%%%%%%%%%%%%%%%%%%%%%%%%%%%%%%%%%%%%%%%%%%%%%%
\subsection{Related CTAN Packages}

There are several other packages which offer a similar functionality:
%
\begin{itemize}
\item
The packages
\href{http://ctan.org/pkg/docmute}{\textsf{docmute}},
\href{http://ctan.org/pkg/includex}{\textsf{includex}} and
\href{http://ctan.org/pkg/standalone}{\textsf{standalone}}
provide commands to include only the document body of
a child file thus allowing both files to be compiled individually.
\item
The packages \href{http://ctan.org/pkg/subdocs}{\textsf{subdocs}}
and \href{http://ctan.org/pkg/subfiles}{\textsf{subfiles}}
provide structures in which the main and child documents can be
encapsulated and allowing them to be compiled individually.
The inclusion mechanism is different from the conventional |\include|.
\item
The package \href{http://ctan.org/pkg/combine}{\textsf{combine}}
is an elaborate solution to combine several documents into one.
\end{itemize}
%
See also the CTAN topic \href{http://ctan.org/topic/subdocs}{\textsf{subdocs}}
for further related packages.
The present package differs from the above solutions in that
a document structure constructed with the conventional |\include| mechanism
just needs two extra commands at the top of every file
such that all constituent files can be compiled individually.

%%%%%%%%%%%%%%%%%%%%%%%%%%%%%%%%%%%%%%%%%%%%%%%%%%%%%%%%%%%%%%%%%%%%%%%%%%%%%%%%
%\subsection{Feature Suggestions}
%
%The following is a list of features which may be useful for future
%versions of this package:
%%
%\begin{itemize}
%\item
%\ldots
%\end{itemize}

%%%%%%%%%%%%%%%%%%%%%%%%%%%%%%%%%%%%%%%%%%%%%%%%%%%%%%%%%%%%%%%%%%%%%%%%%%%%%%%%
\subsection{Revision History}

%%%%%%%%%%%%%%%%%%%%%%%%%%%%%%%%%%%%%%%%
\paragraph{v2.0:} 2018/12/30

\begin{itemize}
\item
immediate forward processing
\item
added |\childdocby| mechanism
\item
manual restructured
\end{itemize}

%%%%%%%%%%%%%%%%%%%%%%%%%%%%%%%%%%%%%%%%
\paragraph{v1.6:} 2018/01/17

\begin{itemize}
\item
application for development of include files
\item
corrections to manual
\end{itemize}

%%%%%%%%%%%%%%%%%%%%%%%%%%%%%%%%%%%%%%%%
\paragraph{v1.5:} 2017/05/21

\begin{itemize}
\item
more complete structuring introduced
\item
|\childdocof| introduced
\item
|\childdoc| renamed to |\childdocmain|
\item
|\childredirect| renamed to |\childdocforward| and |\childdocforwardprefix|
and functionality expanded
\end{itemize}

%%%%%%%%%%%%%%%%%%%%%%%%%%%%%%%%%%%%%%%%
\paragraph{v1.0:} 2017/04/27

\begin{itemize}
\item
manual and install package
\item
first version published on CTAN
\end{itemize}

%%%%%%%%%%%%%%%%%%%%%%%%%%%%%%%%%%%%%%%%
\paragraph{v0.6:} 2017/04/26

\begin{itemize}
\item
redirection mechanism added
\end{itemize}

%%%%%%%%%%%%%%%%%%%%%%%%%%%%%%%%%%%%%%%%
\paragraph{v0.5:} 2017/04/26

\begin{itemize}
\item
functionality in definition file
\end{itemize}


%%%%%%%%%%%%%%%%%%%%%%%%%%%%%%%%%%%%%%%%%%%%%%%%%%%%%%%%%%%%%%%%%%%%%%%%%%%%%%%%
%%%%%%%%%%%%%%%%%%%%%%%%%%%%%%%%%%%%%%%%%%%%%%%%%%%%%%%%%%%%%%%%%%%%%%%%%%%%%%%%
%%%%%%%%%%%%%%%%%%%%%%%%%%%%%%%%%%%%%%%%%%%%%%%%%%%%%%%%%%%%%%%%%%%%%%%%%%%%%%%%
\appendix

\settowidth\MacroIndent{\rmfamily\scriptsize 000\ }

 \DocInput{childdoc.dtx}

\end{document}
%</driver>
% \fi
%
% %%%%%%%%%%%%%%%%%%%%%%%%%%%%%%%%%%%%%%%%%%%%%%%%%%%%%%%%%%%%%%%%%%%%%%%%%%%%%%
% %%%%%%%%%%%%%%%%%%%%%%%%%%%%%%%%%%%%%%%%%%%%%%%%%%%%%%%%%%%%%%%%%%%%%%%%%%%%%%
% \section{Sample}
%\iffalse
%<*samplemain>
%\fi
%
% The following presents a sample document
% with two chapters, two parts, a title page,
% a compile flag as well as three forwarding files to set the flag.
% It consists of eight |.tex| files:
% \begin{center}
% \begin{tabular}{ll}
% |cdocsamp.tex|&main file\\
% |cdocsch1.tex|&include file for chapter 1\\
% |cdocsch2.tex|&include file for chapter 2\\
% |cdocspt3.tex|&include file for part 3\\
% |cdocspt4.tex|&include file for part 4\\
% |cdocsdrf.tex|&forwarding file for main file in draft mode\\
% |cdocsfi1.tex|&forwarding file for final version of chapter 1\\
% |cdocsfi2.tex|&forwarding file for final version of chapter 2\\
% \end{tabular}
% \end{center}
% Each of the eight files can be compiled directly by the \LaTeX{} compiler.
%
% %%%%%%%%%%%%%%%%%%%%%%%%%%%%%%%%%%%%%%
% \paragraph{Main File.}
%
% The main file is called |cdocsamp.tex|.
%
% Load the \textsf{childdoc} definitions and
% declare the filename for the main document:
%    \begin{macrocode}
% \iffalse
%
% childdoc.dtx Copyright (C) 2017-2018 Niklas Beisert
%
% This work may be distributed and/or modified under the
% conditions of the LaTeX Project Public License, either version 1.3
% of this license or (at your option) any later version.
% The latest version of this license is in
%   http://www.latex-project.org/lppl.txt
% and version 1.3 or later is part of all distributions of LaTeX
% version 2005/12/01 or later.
%
% This work has the LPPL maintenance status `maintained'.
%
% The Current Maintainer of this work is Niklas Beisert.
%
% This work consists of the files childdoc.dtx and childdoc.ins
% and the derived files childdoc.def and cdocsamp.tex with
% cdocsch1.tex, cdocsch2.tex, cdocsdrf.tex, cdocsfn1.tex, cdocsfn2.tex.
%
%<package>\ifdefined\childdocmain\endinput\fi
%<package>\ProvidesFile{childdoc.def}[2018/12/30 v2.0 child document driver]
%<samplemain>\ProvidesFile{cdocsamp.tex}[2018/12/30 v2.0 sample for childdoc]
%<*driver>
%\ProvidesFile{childdoc.drv}[2018/12/30 v2.0 childdoc reference manual file]
\PassOptionsToClass{10pt,a4paper}{article}
\documentclass{ltxdoc}

\usepackage[margin=35mm]{geometry}
\usepackage{hyperref}
\usepackage{hyperxmp}
\usepackage[usenames]{color}

\hypersetup{colorlinks=true}
\hypersetup{pdfstartview=FitH}
\hypersetup{pdfpagemode=UseNone}
\hypersetup{pdfsource={}}
\hypersetup{pdflang={en-UK}}
\hypersetup{pdfcopyright={Copyright 2017-2018 Niklas Beisert.
  This work may be distributed and/or modified under the
  conditions of the LaTeX Project Public License, either version 1.3
  of this license or (at your option) any later version.}}
\hypersetup{pdflicenseurl={http://www.latex-project.org/lppl.txt}}
\hypersetup{pdfcontactaddress={ETH Zurich, ITP, HIT K,
  Wolfgang-Pauli-Strasse 27}}
\hypersetup{pdfcontactpostcode={8093}}
\hypersetup{pdfcontactcity={Zurich}}
\hypersetup{pdfcontactcountry={Switzerland}}
\hypersetup{pdfcontactemail={nbeisert@itp.phys.ethz.ch}}
\hypersetup{pdfcontacturl={http://people.phys.ethz.ch/\xmptilde nbeisert/}}

\newcommand{\secref}[1]{\hyperref[#1]{section \ref*{#1}}}

\parskip1ex
\parindent0pt
\let\olditemize\itemize
\def\itemize{\olditemize\parskip0pt}

\begin{document}

\title{The \textsf{childdoc} Package}
\hypersetup{pdftitle={The childdoc Package}}
\author{Niklas Beisert\\[2ex]
  Institut f\"ur Theoretische Physik\\
  Eidgen\"ossische Technische Hochschule Z\"urich\\
  Wolfgang-Pauli-Strasse 27, 8093 Z\"urich, Switzerland\\[1ex]
  \href{mailto:nbeisert@itp.phys.ethz.ch}
  {\texttt{nbeisert@itp.phys.ethz.ch}}}
\hypersetup{pdfauthor={Niklas Beisert}}
\hypersetup{pdfsubject={Manual for the LaTeX2e Package childdoc}}
\date{30 December 2018, \textsf{v2.0}}
\maketitle

\begin{abstract}\noindent
\textsf{childdoc} is a \LaTeXe{} package
that enables the direct compilation
of document sections included by |\include|
to individual files.
\end{abstract}

\begingroup
\parskip0ex
\tableofcontents
\endgroup

%%%%%%%%%%%%%%%%%%%%%%%%%%%%%%%%%%%%%%%%%%%%%%%%%%%%%%%%%%%%%%%%%%%%%%%%%%%%%%%%
%%%%%%%%%%%%%%%%%%%%%%%%%%%%%%%%%%%%%%%%%%%%%%%%%%%%%%%%%%%%%%%%%%%%%%%%%%%%%%%%
\section{Introduction}

\LaTeX{} provides a mechanism to structure a large document (such as a book)
into a main file and several child files (containing the chapters)
using the |\include| command.
This mechanism is beneficial for documents
which span hundreds of pages in order to
make the source file(s) more manageable.
Moreover, compilation can be restricted to
selected child files by means of the |\includeonly| command.
The latter feature can be used to reduce the compilation time while editing
(this was significantly more useful in the earlier days of \LaTeX{})
or to generate a smaller document which is easier to navigate.
Another application of |\includeonly| is to generate
documents consisting of selected parts of the complete document.

However, there are a few drawbacks of the plain |\include| mechanism:
\begin{itemize}
\item
The child files cannot be compiled on their own,
they can only be compiled via the main file.
A naive editing environment
(such as a text editor with an option
to have the current file processed by \LaTeX)
may require one to switch to the main file before compiling;
attempting to compile the child file produces errors.
\item
The main file must be modified (each time)
to adjust the |\includeonly| command
to the present needs. This easily leaves the main file in a messy state.
\item
The generated document will always carry the filename
of the main document. This is inconvenient if
several child files are to be compiled and
to be kept for distribution.
\end{itemize}

The present package provides a simple interface
to make child files individually compilable by \LaTeX{}.
Compiling a child file then has the same effect as compiling
the main file with an |\includeonly| command
to select the appropriate child.
Moreover the generated document will carry the name of the child
rather than the main file.
This resolves all three above issues.

This feature is meant to make the editing of books,
thesis documents and lecture notes somewhat more convenient.
However, the package can also be used efficiently for
composing a series of documents (such as exercise sheets)
which are typically distributed individually.
It then assists the author in generating the individual documents
(potentially in different versions)
as well as a document containing the collected series.
Another application is in developing style files
or other kinds of included material
where compilation of the style file could redirect
to a sample or test file.

%%%%%%%%%%%%%%%%%%%%%%%%%%%%%%%%%%%%%%%%%%%%%%%%%%%%%%%%%%%%%%%%%%%%%%%%%%%%%%%%
%%%%%%%%%%%%%%%%%%%%%%%%%%%%%%%%%%%%%%%%%%%%%%%%%%%%%%%%%%%%%%%%%%%%%%%%%%%%%%%%
\section{Usage}

First of all, the package \textsf{childdoc} is \emph{not} a standard
\LaTeXe{} |.sty| style file! Therefore it needs to be invoked in
a non-standard way.

%%%%%%%%%%%%%%%%%%%%%%%%%%%%%%%%%%%%%%%%%%%%%%%%%%%%%%%%%%%%%%%%%%%%%%%%%%%%%%%%
\subsection{Included Files}
\label{sec:include}

%%%%%%%%%%%%%%%%%%%%%%%%%%%%%%%%%%%%%%%%
\DescribeMacro{\childdocmain}
To use the package, add the commands
\begin{center}
\begin{tabular}{l}
|\input{childdoc.def}|\\
|\childdocmain{}|\\
\end{tabular}
\end{center}
at the very top of the main \LaTeX{} file,
in particular \emph{before} the |\documentclass| statement!
The argument of |\childdocmain| should be left empty
(but it must be present).

%%%%%%%%%%%%%%%%%%%%%%%%%%%%%%%%%%%%%%%%
\DescribeMacro{\childdocof}
Furthermore, add the commands
\begin{center}
\begin{tabular}{l}
|\input{childdoc.def}|\\
|\childdocof{|\textit{main}|}|\\
\end{tabular}
\end{center}
at the top of every child file \textit{child}
which is included by |\include{|\textit{child}|}|
from within the main file
(or at least for those files to be compiled individually).
The argument \textit{main} must be the filename of the main file.

There are a couple of
considerations in setting up the main and child documents:

%%%%%%%%%%%%%%%%%%%%%%%%%%%%%%%%%%%%%%%%
\paragraph{Restrictions.}

Please note the following restrictions:
\begin{itemize}
\item
|\childdocmain| must be called with one argument \textit{main}
to ensure compatibility with earlier version of the package.
It must either be empty (|\childdocmain{}|)
or precisely match the filename of the main file in which it is specified.
See \secref{sec:detection} for further information.
\item
The filename \textit{main} must be specified without the |.tex| extension.
\item
The filename \textit{main} is case sensitive
(even in case-insensitive file systems)
due to internal string comparison.
\item
The argument \textit{main} should be fully expanded, it cannot be a macro.
\item
Subdirectories and special characters should be avoided in filenames.
\item
The command |\childdocmain{|\textit{main}|}| must be followed by a whitespace.
It should not be followed immediately by another command
or by a comment mark `|%|'.
This is because the \TeX{} parser reads the token immediately following
the argument of |\childdocmain| and puts it
at the beginning of every child section;
however, a white\-space is ignored.
\end{itemize}

%%%%%%%%%%%%%%%%%%%%%%%%%%%%%%%%%%%%%%%%
\paragraph{Content of Main File.}

It is advisable to place all content in the child files included by |\include|.
Any output contained in the main file will appear in all child documents
unless suppressed manually;
it cannot be suppressed automatically by the |\includeonly| directive
and thus should normally be avoided.
A method to include some content in the main file
by means of conditional processing is described in \secref{sec:conditional}.

%%%%%%%%%%%%%%%%%%%%%%%%%%%%%%%%%%%%%%%%
\paragraph{Page Numbering.}

When only a part of the document is compiled,
the appropriate numbering of pages
(as well as other status parameters)
is determined from the |.aux| files.
The latter contain information from previous passes.
However this information needs to propagate through
all intermediate child documents.
Therefore the page numbering in child documents may well
be inconsistent until the complete document is compiled at least once.

A useful (if unconventional) way to always ensure a consistent
page numbering is to restart the numbering in each child document
and denote the pages by `\textit{child}|.|\textit{page}'
where \textit{child} represents the chapter/section number of the child file.
This can be achieved by the command
|\numberwithin{page}{|\textit{child}|}|
of the \textsf{amsmath} package
where \textit{child} can be |chapter| or |section|
depending on the chosen structuring.
Alternatively, one can modify the macro |\thepage| appropriately
and reset the counter |page| at the start of each child file.

%%%%%%%%%%%%%%%%%%%%%%%%%%%%%%%%%%%%%%%%%%%%%%%%%%%%%%%%%%%%%%%%%%%%%%%%%%%%%%%%
\subsection{Conditional Processing}
\label{sec:conditional}

The package provides a mechanism to compile different versions
of a document. To customise the versions further some conditional processing
can come in handy to distinguish which version is being compiled.
The package provides two macros to describe the compilation context:

%%%%%%%%%%%%%%%%%%%%%%%%%%%%%%%%%%%%%%%%
\DescribeMacro{\ifchilddoc}
The conditional |\ifchilddoc| distinguishes between the compilation of
child documents and the main document:
%
\begin{center}
|\ifchilddoc |\textit{child-code}| |[|\||else |\textit{main-code}]| \||fi|
\end{center}

%%%%%%%%%%%%%%%%%%%%%%%%%%%%%%%%%%%%%%%%
\DescribeMacro{\childdocname}
\DescribeMacro{\childdocjob}
The macro |\childdocname| contains the filename (without extension)
of the main or child file being processed.
Note that |\childdocjob| will always contain the name of the main file.

%%%%%%%%%%%%%%%%%%%%%%%%%%%%%%%%%%%%%%%%
\paragraph{Title Page.}

Conditional processing can be used to include a title or banner page
in the main document when proper precautions are taken.
Importantly, the code in the main file should ensure that the page counter
(as well as other status parameters which are stored in the |.aux| files)
takes the same value after the conditional processing.
Otherwise the page numbers may take divergent values
depending on which part is compiled.

For example, a title page could be declared by:
%
\begin{center}
\begin{tabular}{l}
|\ifchilddoc\||else|\\
|\addtocounter{page}{-1}|\\
\textit{code for title page}\\
|\newpage|\\
|\||fi|
\end{tabular}
\end{center}
%
A banner page for the child documents can be generated by:
%
\begin{center}
\begin{tabular}{l}
|\ifchilddoc|\\
|\addtocounter{page}{-1}|\\
\textit{code for banner page}\\
|\newpage|\\
|\||fi|
\end{tabular}
\end{center}
%
Here one could write a message such as:
\begin{center}
|This is the part \childdocname{} of \childdocjob{}.|
\end{center}

%%%%%%%%%%%%%%%%%%%%%%%%%%%%%%%%%%%%%%%%%%%%%%%%%%%%%%%%%%%%%%%%%%%%%%%%%%%%%%%%
\subsection{Flags}
\label{sec:flags}

The package makes it easy to generate different versions
of the main or child documents.
To this end compilation flags can be defined
and assigned different default values.
They will be particularly useful in conjunction
with the forwarding mechanism described in \secref{sec:forward}.

For example, it may be useful to have a flag |\version|
which can be set to |draft| or |final|.
The document source will contain some conditional code
depending on the value of |\version|.
Suppose further, the flag should default to |final| for the main file
and to |draft| for child files
which is a natural assignment for editing the document.
This is achieved by placing the following code
in the preamble of the main document
(below the |\childdocmain| directive):
%
\begin{center}
\begin{tabular}{l}
|\ifchilddoc|\\
|\providecommand{\version}{draft}|\\
|\||else|\\
|\providecommand{\version}{final}|\\
|\||fi|
\end{tabular}
\end{center}
%
The definition by |\providecommand| makes sure
that previous definitions are not overwritten.
Further statements |\providecommand{\version}{...}|
can thus be added before the above code to override it.

For the main file, one might add a line
(between |\childdocmain| and the above block)
%
\begin{center}
|%\ifchilddoc\||else\providecommand{\version}{draft}\||fi|
\end{center}
%
which can be uncommented to produce a draft version.
Likewise one can add a line to the very top of a child file
(above the |\childdocof{|\textit{main}|}| directive)
%
\begin{center}
|%\providecommand{\version}{final}|
\end{center}
%
which can be uncommented to produce the final version of this child document.

%%%%%%%%%%%%%%%%%%%%%%%%%%%%%%%%%%%%%%%%%%%%%%%%%%%%%%%%%%%%%%%%%%%%%%%%%%%%%%%%
\subsection{Forwarding}
\label{sec:forward}

Different versions of the main or child documents
using compilation flags as described in \secref{sec:flags}
can be (permanently) stored in different files
for convenient compilation, viewing and distribution.
To this end, the package defines a command
to pass on compilation to a different file:

%%%%%%%%%%%%%%%%%%%%%%%%%%%%%%%%%%%%%%%%
\DescribeMacro{\childdocforward}
The command |\childdocforward| redirects processing to
another source file:
%
\begin{center}
\begin{tabular}{l}
|\input{childdoc.def}|\\
|\childdocforward[|\textit{main}|]{|\textit{dest}|}|\\
\end{tabular}
\end{center}
%
The argument \textit{dest} is the destination file
(without extension).
It should be the main file or one of the child files.
Note that further \textsf{childdoc} directives
such as |\childdocof| and |\childdocforward|
in the indicated file will be processed in this form.
The optional argument \textit{main}
passes on directly to the main file \textit{main}
while pretending to compile the child \textit{dest}.
This form behaves as if \textit{dest}
issues |\childdocof{|\textit{main}|}| right away,
and no further \textsf{childdoc} directives will be processed.

%%%%%%%%%%%%%%%%%%%%%%%%%%%%%%%%%%%%%%%%
\DescribeMacro{\...prefix}
In the alternative form |\childdocforwardprefix|,
%
\begin{center}
\begin{tabular}{l}
|\input{childdoc.def}|\\
|\childdocforwardprefix[|\textit{main}|]{|\textit{prefix}|}{|\textit{dest}|}|
\end{tabular}
\end{center}
%
the destination file is determined by a pattern
depending on the current file:
To make this work, the current file must be called
`{\textit{prefix}\hspace{0.2em}\textit{suffix}}'
with \textit{prefix} matching precisely the argument.
Processing is then passed on to the file
`{\textit{dest}\hspace{0.2em}\textit{suffix}}'.
Surely, the same effect is achieved by
directly specifying the
argument `{\textit{dest}\hspace{0.2em}\textit{suffix}}'
in the first form.
However, that requires to set up a different file
for each child. With the alternative form of the command
all these files can have exactly the same content
which simplifies setting them up and maintaining them.

For example, the following file |draft.tex|
with a compilation flag |\version| as described in \secref{sec:flags}
compiles the main document as a draft:
%
\begin{center}
\begin{tabular}{l}
|\def\version{draft}|\\
|\input{childdoc.def}|\\
|\childdocforward{|\textit{main}|}|
\end{tabular}
\end{center}
%
Likewise, the following files |final|\textit{nn}|.tex|
compile the final version of the child document
|child|\textit{nn}|.tex|:
%
\begin{center}
\begin{tabular}{l}
|\def\version{final}|\\
|\input{childdoc.def}|\\
|\childdocforwardprefix{final}{child}|
\end{tabular}
\end{center}
%

Note that when several versions of a main file and/or of each child file
are to be generated, it may be convenient to set up a |Makefile| or
shell script to automatise the process.

%%%%%%%%%%%%%%%%%%%%%%%%%%%%%%%%%%%%%%%%%%%%%%%%%%%%%%%%%%%%%%%%%%%%%%%%%%%%%%%%
\subsection{Command Line Processing}
\label{sec:commandline}

The effect of redirection files can also be achieved by invoking
the \LaTeX{} compiler with a more elaborate command line.
Most conveniently this should be done as part
of a shell script or a |Makefile|.

When using \textsf{childdoc} in the main file, the following
command lines effectively perform a redirection
(note that depending on the shell being used,
backslashes may have to be doubled: `|\|' $\to$ `|\\|'):
%
\begin{center}
|... -jobname "|\textit{target}|" |\\|"|[\textit{flags}]%
|\input{childdoc.def}\childdocforward[|\textit{main}|]{|\textit{dest}|}"|
\end{center}
%
Here \textit{target} is the name of the output file,
\textit{main} is the name of the main file
and \textit{dest} is the name of the main or child file to be processed
(all filenames without extensions).
The optional argument \textit{main} can be omitted
if \textit{main} matches \textit{dest}.
Optionally, compilation \textit{flags} can be defined via |\def| commands.
This command line makes the \TeX{} engine believe
it is compiling the file \textit{target}
whose content is specified as the latter parameter.
The provided code then forwards the processing to
\textit{main} or \textit{dest} as described in \secref{sec:forward}.

%%%%%%%%%%%%%%%%%%%%%%%%%%%%%%%%%%%%%%%%%%%%%%%%%%%%%%%%%%%%%%%%%%%%%%%%%%%%%%%%
\subsection{Include by Input}
\label{sec:input}

Including child documents by |\include| has some restrictions by design.
Most notably, the content of a child document always occupies
its own set of pages; pages cannot be shared between child documents.
Usually, this behaviour makes perfect sense
because each child document contain an essential part of the document.
However, in some situations it may be desirable to compose
a document from a collection of parts
without having mandatory page breaks between then.
For this case, the package
provides a mechanism to include parts
by |\input| which can also be processed individually.
However, by construction this mechanism
requires manual handling of the content to be output.

%%%%%%%%%%%%%%%%%%%%%%%%%%%%%%%%%%%%%%%%
\DescribeMacro{\ifchilddocmanual}
The main file should be prepared as usual, see \secref{sec:include}.
However, the document body must make a distinction
between processing of an individual part and of the main document, e.g.:
%
\begin{center}
\begin{tabular}{l}
|\ifchilddocmanual|\\
|\input{\childdocname}|\\
|\||else|\\
\textit{document body with }|\input{|\textit{part}|}|\\
|\||fi|
\end{tabular}
\end{center}
%
The conditional |\ifchilddocmanual| is true whenever
a part to be included by |\input| is being compiled,
and the name of the part is stored in |\childdocname|.

%%%%%%%%%%%%%%%%%%%%%%%%%%%%%%%%%%%%%%%%
\DescribeMacro{\childdocby}
Each part to be included by |\input| should start with:
%
\begin{center}
\begin{tabular}{l}
|\input{childdoc.def}|\\
|\childdocby{|\textit{main}|}|\\
\end{tabular}
\end{center}
%
The directive |\childdocby| is similar to |\childdocof|
described in \secref{sec:include},
but the subsequent selection of content must be done manually.
To that end, both |\ifchilddoc| and |\ifchilddocmanual|
will be true upon processing of a part,
and the name of the part is stored in |\childdocname|.
Note that |\jobname| will be set to the filename of the current part
so that each part receives an individual |.aux| file
that does not interfere with the |.aux| file(s) of the main document.
This behaviour can be altered by the alternative form
|\childdocby[*]{|\textit{main}|}| (with a non-empty optional argument)
which uses the |.aux| file of the main document
by setting |\jobname| to \textit{main}.

%%%%%%%%%%%%%%%%%%%%%%%%%%%%%%%%%%%%%%%%%%%%%%%%%%%%%%%%%%%%%%%%%%%%%%%%%%%%%%%%
\subsection{Driver Development}
\label{sec:driver}

The \textsf{childdoc} mechanism can also be use for the development
of definition files such as \LaTeX{} styles or classes.
This case differs from the above setup with multiple parts
included by |\include| in that no |\includeonly| should be invoked.
This can be achieved by starting the include file
(before |\ProvidesPackage|) with:
%
\begin{center}
\begin{tabular}{l}
|\input{childdoc.def}|\\
|\childdocforward{|\textit{main}|}|\\
\end{tabular}
\end{center}
%
or alternatively with:
%
\begin{center}
\begin{tabular}{l}
|\input{childdoc.def}|\\
|\childdocby{|\textit{main}|}|\\
\end{tabular}
\end{center}
%
Both forms have slightly different effects as described above.
The main file is prepared as usual, see \secref{sec:include}.

%%%%%%%%%%%%%%%%%%%%%%%%%%%%%%%%%%%%%%%%%%%%%%%%%%%%%%%%%%%%%%%%%%%%%%%%%%%%%%%%
\subsection{Legacy Detection}
\label{sec:detection}

The directive |\childdocmain| in the main file can detect
whether the complete document or merely a child is to be compiled
even without using the directive |\childdocof|.
This method is deprecated because it is less robust
and there is no compelling reason to use it;
it is merely provided for backward compatibility
and it may be removed in future versions.

If the detection mechanism is to be used,
it is mandatory to correctly specify
the filename of the main file as the argument of |\childdocmain|:
%
\begin{center}
\begin{tabular}{l}
|\input{childdoc.def}|\\
|\childdocmain{|\textit{main}|}|\\
\end{tabular}
\end{center}
%
If |\jobname| does not match the argument \textit{main} of |\childdocmain|,
it is assumed that |\jobname| points to the child file to be compiled.
When using |\childdocmain| with the main file specified as argument,
it suffices to start a child file
with just |\input{|\textit{main}|}|
without loading of the package and using |\childdocof|.
If instead all processing is done
with the appropriate \textsf{childdoc} directives,
the argument of \textit{main} of |\childdocmain| can be empty.

An alternative version of the command line processing described
in \secref{sec:commandline} using the detection mechanism reads:
%
\begin{center}
|... -jobname "|\textit{target}|" "|[\textit{flags}]%
[|\def\jobname{|\textit{dest}|}|]|\input{|\textit{main}|}"|
\end{center}

%%%%%%%%%%%%%%%%%%%%%%%%%%%%%%%%%%%%%%%%%%%%%%%%%%%%%%%%%%%%%%%%%%%%%%%%%%%%%%%%
\subsection{Manual Code}
\label{sec:manual}

In case one cannot be certain whether the definitions file |childdoc.def|
is installed on the target \TeX{} distribution
and one prefers not to ship it,
it is conceivable to paste a few relevant commands into the sources.

To that end, drop all statements |\input{childdoc.def}|
and perform the replacements as outlined below.
Instead of |\childdocmain{|\textit{main}|}| add the following code
to the top of the main file:
%
\begin{center}
\begin{tabular}{l}
|\||ifdefined\childdocname\endinput\||fi\newif\ifchilddoc|\\
|\edef\childdocname{\scantokens\expandafter{\jobname\noexpand}}|\\
|\def\childdocmain{|\textit{main}|}\||ifx\childdocmain\childdocname\||else|\\
|\childdoctrue\includeonly{\childdocname}\let\jobname\childdocmain\||fi|\\
\end{tabular}
\end{center}
%
Instead of |\childdocof{|\textit{main}|}| just include the main file
at the top of each child file:
%
\begin{center}
|\input{|\textit{main}|}|
\end{center}
%
A simple redirection |\childdocforward{|\textit{dest}|}| is achieved by:
%
\begin{center}
|\def\jobname{|\textit{dest}|}\input{\jobname}|
\end{center}
%
The redirection with prefix
|\childdocforwardprefix[|\textit{prefix}|]{|\textit{dest}|}|
is accomplished by:
%
\begin{center}
\begin{tabular}{l}
|{\edef\jobname{\scantokens\expandafter{\jobname\noexpand}}|\\
|\def\redirectjob |\textit{prefix}|#1~~~{\gdef\jobname{|\textit{dest}|#1}}|\\
|\expandafter\redirectjob\jobname~~~}\input{\jobname}|
\end{tabular}
\end{center}

In an alternative approach,
child documents can be compiled by a specific command line
without additional code or specific definitions:
%
\begin{center}
|... -jobname "|\textit{target}|" "|[\textit{flags}]%
|\includeonly{|\textit{dest}|}\input{|\textit{main}|}"|
\end{center}
%

%%%%%%%%%%%%%%%%%%%%%%%%%%%%%%%%%%%%%%%%%%%%%%%%%%%%%%%%%%%%%%%%%%%%%%%%%%%%%%%%
%%%%%%%%%%%%%%%%%%%%%%%%%%%%%%%%%%%%%%%%%%%%%%%%%%%%%%%%%%%%%%%%%%%%%%%%%%%%%%%%
\section{Information}

%%%%%%%%%%%%%%%%%%%%%%%%%%%%%%%%%%%%%%%%%%%%%%%%%%%%%%%%%%%%%%%%%%%%%%%%%%%%%%%%
\subsection{Copyright}

Copyright \copyright{} 2017--2018 Niklas Beisert

This work may be distributed and/or modified under the
conditions of the \LaTeX{} Project Public License, either version 1.3
of this license or (at your option) any later version.
The latest version of this license is in
  \url{http://www.latex-project.org/lppl.txt}
and version 1.3 or later is part of all distributions of \LaTeX{}
version 2005/12/01 or later.

This work has the LPPL maintenance status `maintained'.

The Current Maintainer of this work is Niklas Beisert.

This work consists of the files |README.txt|, |childdoc.ins| and |childdoc.dtx|
as well as the derived files |childdoc.def|, |cdocsamp.tex|
with |cdocsch1.tex|, |cdocsch2.tex|, |cdocspt3.tex|, |cdocspt4.tex|,
|cdocsdrf.tex|, |cdocsfn1.tex|, |cdocsfn2.tex|
as well as |childdoc.pdf|.

%%%%%%%%%%%%%%%%%%%%%%%%%%%%%%%%%%%%%%%%%%%%%%%%%%%%%%%%%%%%%%%%%%%%%%%%%%%%%%%%
\subsection{Files and Installation}

The package consists of the files:
%
\begin{center}
\begin{tabular}{ll}
    |README.txt|   & readme file \\
    |childdoc.ins| & installation file \\
    |childdoc.dtx| & source file \\
    |childdoc.def| & definition file \\
    |cdocsamp.tex| & sample main file \\
    |cdocsch1.tex| & sample include file \\
    |cdocsch2.tex| & sample include file \\
    |cdocspt3.tex| & sample part file \\
    |cdocspt4.tex| & sample part file \\
    |cdocsdrf.tex| & sample redirection file \\
    |cdocsfn1.tex| & sample redirection file \\
    |cdocsfn2.tex| & sample redirection file \\
    |childdoc.pdf| & manual
\end{tabular}
\end{center}
%
The distribution consists of the files
|README.txt|, |childdoc.ins| and |childdoc.dtx|.
%
\begin{itemize}
\item
Run (pdf)\LaTeX{} on |childdoc.dtx|
to compile the manual |childdoc.pdf| (this file).
\item
Run \LaTeX{} on |childdoc.ins| to create the definitions file |childdoc.def|
and the sample |cdocsamp.tex| with include files
|cdocsch1.tex|, |cdocsch2.tex|, |cdocspt3.tex|, |cdocspt4.tex|,
|cdocsdrf.tex|, |cdocsfn1.tex|, |cdocsfn2.tex|.
Then copy the file |childdoc.def| to an appropriate directory of your \LaTeX{}
distribution, e.g.\ \textit{texmf-root}|/tex/latex/childdoc|.
\end{itemize}

%%%%%%%%%%%%%%%%%%%%%%%%%%%%%%%%%%%%%%%%%%%%%%%%%%%%%%%%%%%%%%%%%%%%%%%%%%%%%%%%
\subsection{Related CTAN Packages}

There are several other packages which offer a similar functionality:
%
\begin{itemize}
\item
The packages
\href{http://ctan.org/pkg/docmute}{\textsf{docmute}},
\href{http://ctan.org/pkg/includex}{\textsf{includex}} and
\href{http://ctan.org/pkg/standalone}{\textsf{standalone}}
provide commands to include only the document body of
a child file thus allowing both files to be compiled individually.
\item
The packages \href{http://ctan.org/pkg/subdocs}{\textsf{subdocs}}
and \href{http://ctan.org/pkg/subfiles}{\textsf{subfiles}}
provide structures in which the main and child documents can be
encapsulated and allowing them to be compiled individually.
The inclusion mechanism is different from the conventional |\include|.
\item
The package \href{http://ctan.org/pkg/combine}{\textsf{combine}}
is an elaborate solution to combine several documents into one.
\end{itemize}
%
See also the CTAN topic \href{http://ctan.org/topic/subdocs}{\textsf{subdocs}}
for further related packages.
The present package differs from the above solutions in that
a document structure constructed with the conventional |\include| mechanism
just needs two extra commands at the top of every file
such that all constituent files can be compiled individually.

%%%%%%%%%%%%%%%%%%%%%%%%%%%%%%%%%%%%%%%%%%%%%%%%%%%%%%%%%%%%%%%%%%%%%%%%%%%%%%%%
%\subsection{Feature Suggestions}
%
%The following is a list of features which may be useful for future
%versions of this package:
%%
%\begin{itemize}
%\item
%\ldots
%\end{itemize}

%%%%%%%%%%%%%%%%%%%%%%%%%%%%%%%%%%%%%%%%%%%%%%%%%%%%%%%%%%%%%%%%%%%%%%%%%%%%%%%%
\subsection{Revision History}

%%%%%%%%%%%%%%%%%%%%%%%%%%%%%%%%%%%%%%%%
\paragraph{v2.0:} 2018/12/30

\begin{itemize}
\item
immediate forward processing
\item
added |\childdocby| mechanism
\item
manual restructured
\end{itemize}

%%%%%%%%%%%%%%%%%%%%%%%%%%%%%%%%%%%%%%%%
\paragraph{v1.6:} 2018/01/17

\begin{itemize}
\item
application for development of include files
\item
corrections to manual
\end{itemize}

%%%%%%%%%%%%%%%%%%%%%%%%%%%%%%%%%%%%%%%%
\paragraph{v1.5:} 2017/05/21

\begin{itemize}
\item
more complete structuring introduced
\item
|\childdocof| introduced
\item
|\childdoc| renamed to |\childdocmain|
\item
|\childredirect| renamed to |\childdocforward| and |\childdocforwardprefix|
and functionality expanded
\end{itemize}

%%%%%%%%%%%%%%%%%%%%%%%%%%%%%%%%%%%%%%%%
\paragraph{v1.0:} 2017/04/27

\begin{itemize}
\item
manual and install package
\item
first version published on CTAN
\end{itemize}

%%%%%%%%%%%%%%%%%%%%%%%%%%%%%%%%%%%%%%%%
\paragraph{v0.6:} 2017/04/26

\begin{itemize}
\item
redirection mechanism added
\end{itemize}

%%%%%%%%%%%%%%%%%%%%%%%%%%%%%%%%%%%%%%%%
\paragraph{v0.5:} 2017/04/26

\begin{itemize}
\item
functionality in definition file
\end{itemize}


%%%%%%%%%%%%%%%%%%%%%%%%%%%%%%%%%%%%%%%%%%%%%%%%%%%%%%%%%%%%%%%%%%%%%%%%%%%%%%%%
%%%%%%%%%%%%%%%%%%%%%%%%%%%%%%%%%%%%%%%%%%%%%%%%%%%%%%%%%%%%%%%%%%%%%%%%%%%%%%%%
%%%%%%%%%%%%%%%%%%%%%%%%%%%%%%%%%%%%%%%%%%%%%%%%%%%%%%%%%%%%%%%%%%%%%%%%%%%%%%%%
\appendix

\settowidth\MacroIndent{\rmfamily\scriptsize 000\ }

 \DocInput{childdoc.dtx}

\end{document}
%</driver>
% \fi
%
% %%%%%%%%%%%%%%%%%%%%%%%%%%%%%%%%%%%%%%%%%%%%%%%%%%%%%%%%%%%%%%%%%%%%%%%%%%%%%%
% %%%%%%%%%%%%%%%%%%%%%%%%%%%%%%%%%%%%%%%%%%%%%%%%%%%%%%%%%%%%%%%%%%%%%%%%%%%%%%
% \section{Sample}
%\iffalse
%<*samplemain>
%\fi
%
% The following presents a sample document
% with two chapters, two parts, a title page,
% a compile flag as well as three forwarding files to set the flag.
% It consists of eight |.tex| files:
% \begin{center}
% \begin{tabular}{ll}
% |cdocsamp.tex|&main file\\
% |cdocsch1.tex|&include file for chapter 1\\
% |cdocsch2.tex|&include file for chapter 2\\
% |cdocspt3.tex|&include file for part 3\\
% |cdocspt4.tex|&include file for part 4\\
% |cdocsdrf.tex|&forwarding file for main file in draft mode\\
% |cdocsfi1.tex|&forwarding file for final version of chapter 1\\
% |cdocsfi2.tex|&forwarding file for final version of chapter 2\\
% \end{tabular}
% \end{center}
% Each of the eight files can be compiled directly by the \LaTeX{} compiler.
%
% %%%%%%%%%%%%%%%%%%%%%%%%%%%%%%%%%%%%%%
% \paragraph{Main File.}
%
% The main file is called |cdocsamp.tex|.
%
% Load the \textsf{childdoc} definitions and
% declare the filename for the main document:
%    \begin{macrocode}
\input{childdoc.def}
\childdocmain{}
%    \end{macrocode}

% Optional override for |\version| flag:
%    \begin{macrocode}
%%\ifchilddoc\else\providecommand{\version}{draft}\fi
%    \end{macrocode}

% Define the default values for the |\version| flag
% (|final| for the main file and |draft| for childs):
%    \begin{macrocode}
\ifchilddoc
\providecommand{\version}{draft}
\else
\providecommand{\version}{final}
\fi
%    \end{macrocode}

% Load the standard document class:
%    \begin{macrocode}
\documentclass[12pt]{article}
%    \end{macrocode}

% Start the document body:
%    \begin{macrocode}
\begin{document}
%    \end{macrocode}

% Declare a title page.
% Print title, part of document being processed and version flag:
%    \begin{macrocode}
\addtocounter{page}{-1}
\begin{center}
{\LARGE\bfseries{}childdoc example\par}
\vspace{1cm}
\ifchilddoc
\ifchilddocmanual part\else chapter\fi:
`\childdocname' of `\childdocjob'\par
\else
main document: `\childdocjob'\par
\fi
version: \version\par
\end{center}
\newpage
%    \end{macrocode}

% Manually include selected file,
% otherwise process as usual:
%    \begin{macrocode}
\ifchilddocmanual
\section*{part `\childdocname'}
\input{\childdocname}
\else
%    \end{macrocode}

% Include the two chapters:
%    \begin{macrocode}
\include{cdocsch1}
\include{cdocsch2}
%    \end{macrocode}

% Include the two parts unless only chapters should be displayed:
%    \begin{macrocode}
\ifchilddoc\else
\section{part three}
\input{cdocspt3}
\section{part four}
\input{cdocspt4}
\fi
%    \end{macrocode}

% Process as usual until here:
%    \begin{macrocode}
\fi
%    \end{macrocode}

% End of document body:
%    \begin{macrocode}
\end{document}
%    \end{macrocode}
%\iffalse
%</samplemain>
%\fi
%
% %%%%%%%%%%%%%%%%%%%%%%%%%%%%%%%%%%%%%%
% \paragraph{Chapter Include Files.}
%
% The include files are called |cdocsch1.tex| and |cdocsch2.tex|.
%
%\iffalse
%<*samplechap1|samplechap2>
%\fi

% Optional override for |\version| flag:
%    \begin{macrocode}
%%\providecommand{\version}{final}
%    \end{macrocode}

% Include the main document:
%    \begin{macrocode}
\input{childdoc.def}
\childdocof{cdocsamp}
%    \end{macrocode}

%\iffalse
%</samplechap1|samplechap2>
%\fi
%
%\iffalse
%<*samplechap1>
%\fi
% Some text for chapter 1:
%    \begin{macrocode}
\section{one}
some text in chapter one
%    \end{macrocode}

%\iffalse
%</samplechap1>
%\fi
% Some text for chapter 2:
%\iffalse
%<*samplechap2>
%\fi
%    \begin{macrocode}
\section{two}
more text in chapter two
%    \end{macrocode}

%\iffalse
%</samplechap2>
%\fi
%
% %%%%%%%%%%%%%%%%%%%%%%%%%%%%%%%%%%%%%%
% \paragraph{Part Include Files.}
%
% The include files are called |cdocspt3.tex| and |cdocspt4.tex|.
%
%\iffalse
%<*samplepart3|samplepart4>
%\fi

% Optional override for |\version| flag:
%    \begin{macrocode}
%%\providecommand{\version}{final}
%    \end{macrocode}

% Include the main document:
%    \begin{macrocode}
\input{childdoc.def}
\childdocby{cdocsamp}
%    \end{macrocode}

%\iffalse
%</samplepart3|samplepart4>
%\fi
%
%\iffalse
%<*samplepart3>
%\fi
% Some text for part 3:
%    \begin{macrocode}
some text in part three
%    \end{macrocode}

%\iffalse
%</samplepart3>
%\fi
% Some text for part 4:
%\iffalse
%<*samplepart4>
%\fi
%    \begin{macrocode}
more text in part four
%    \end{macrocode}

%\iffalse
%</samplepart4>
%\fi
%
% %%%%%%%%%%%%%%%%%%%%%%%%%%%%%%%%%%%%%%
% \paragraph{Forwarding for a Complete Draft.}
%
% The following forwarding file |cdocsdrf.tex|
% compiles the main document in draft mode:
%\iffalse
%<*sampledraft>
%\fi
%    \begin{macrocode}
\def\version{draft}
\input{childdoc.def}
\childdocforward{cdocsamp}
%    \end{macrocode}

%\iffalse
%</sampledraft>
%\fi
%
% %%%%%%%%%%%%%%%%%%%%%%%%%%%%%%%%%%%%%%
% \paragraph{Forwarding for Final Version of the Chapters.}
%
% The following forwarding files |cdocsfn1.tex| and |cdocsfn2.tex|
% (with identical content)
% compile the final versions of the child documents
% |cdocsch1.tex| and |cdocsch2.tex|, respectively:
%\iffalse
%<*samplefinal>
%\fi
%    \begin{macrocode}
\def\version{final}
\input{childdoc.def}
\childdocforwardprefix[cdocsamp]{cdocsfn}{cdocsch}
%    \end{macrocode}

%\iffalse
%</samplefinal>
%\fi
%
% %%%%%%%%%%%%%%%%%%%%%%%%%%%%%%%%%%%%%%
% \paragraph{Command Line Processing.}
%
% The following three command lines generate the output files
% |cdocscld|, |cdocscl1| and |cdocscl2|
% which should be identical to
% |cdocsdrf|, |cdocsch1| and |cdocsfn2|, respectively:
% \begin{center}
% \begin{tabular}{l}
% |latex -jobname cdocscld \|\\
% |  "\def\version{draft}\input{childdoc.def}\childdocforward{cdocsamp}"|\\
% |latex -jobname cdocscl1 \|\\
% |  "\input{childdoc.def}\childdocforward[cdocsamp]{cdocsch1}"|\\
% |latex -jobname cdocscl2 \|\\
% |  "\def\version{final}\input{childdoc.def}\childdocforward{cdocsch2}"|
% \end{tabular}
% \end{center}
% Note that the trailing backslash on each first line
% merely continues the input to the second line
% (for convenient cut ant paste).
% Furthermore, the command |latex| can be replaced by any
% of its alternative versions such as |pdflatex|.
%
% %%%%%%%%%%%%%%%%%%%%%%%%%%%%%%%%%%%%%%%%%%%%%%%%%%%%%%%%%%%%%%%%%%%%%%%%%%%%%%
% %%%%%%%%%%%%%%%%%%%%%%%%%%%%%%%%%%%%%%%%%%%%%%%%%%%%%%%%%%%%%%%%%%%%%%%%%%%%%%
% \section{Implementation}
%\iffalse
%<*package>
%\fi
%
% This section describes the definitions file |childdoc.def|.

% The definitions cannot be loaded using |\usepackage| or |\RequirePackage|
% which has a mechanism to prevent loading a style file more than once.
% When loading the definitions by means of |\input|
% multiple instances have to be prevented manually:
%\iffalse
%This code needs to be before the `\ProvidesFile' directive
%which is defined at the beginning of this file.
%Therefore it is also placed there and commented out here.
%</package>
%<*discard>
%\fi
%    \begin{macrocode}
\ifdefined\childdocmain\endinput\fi
%    \end{macrocode}
%\iffalse
%</discard>
%<*package>
%\fi
%
% \macro{\ifchilddoc}
% \macro{\ifchilddocmanual}
% The conditional |\ifchilddoc| tells whether a
% child (true) or main (false) document is being compiled.
% The conditional |\ifchilddocmanual| tells whether
% the |\includeonly| mechanism is used (false) or
% the selection of child files must be performed manually (true).
% The definitions initialise to false:
%    \begin{macrocode}
\newif\ifchilddoc
\newif\ifchilddocmanual
%    \end{macrocode}

% \macro{\childdocname}
% \macro{\childdocjob}
% The macro |\childdocname| stores the name of the main document
% to be compiled. The macro |\childdocjob| stores the name of
% the document on which the \LaTeX{} compiler was originally invoked.
% The content of |\jobname| cannot be compared
% to filenames specified in the source due to different catcodes.
% The following code rescans |\jobname|, stores the result
% in |\childdocname| and saves a copy in |\childdocjob|:
%    \begin{macrocode}
\edef\childdocname{\scantokens\expandafter{\jobname\noexpand}}
\let\childdocjob\childdocname
%    \end{macrocode}

% \macro{\childdocdisable}
% The macro |\childdocdisable| prevents the main file
% from being processed more than once.
% At this stage, the main document command |\childdocmain|
% is assumed to be called once again where it should do nothing.
% Any subsequent call to it should prevent
% a secondary processing of the main document
% It overwrites the forwarding commands
% |\childdocof| and |\childdocforward|
% with empty macros to prevent further inclusions of the main document:
%    \begin{macrocode}
\newcommand{\childdocdisable}
{
  \renewcommand{\childdocmain}[1]{\renewcommand{\childdocmain}[1]{\endinput}}
  \renewcommand{\childdocof}[1]{}
  \renewcommand{\childdocby}[2][]{}
  \renewcommand{\childdocforward}[2][]{}
  \renewcommand{\childdocdisable}{}
}
%    \end{macrocode}

% \macro{\childdocmain}
% The macro |\childdocmain| is to be called at the top of the main file
% with nothing or the main filename (without extension) as argument.
% First, it breaks loops.
% If the argument is not empty and does not match |\childdocname|
% (which is set by the first inclusion of |childdoc.def|),
% |\ifchilddoc| is set to true, |\includeonly| is applied to the child file
% and |\jobname| is set to the main file
% (for proper handling of |.aux| files):
%    \begin{macrocode}
\newcommand{\childdocmain}[1]
{
  \childdocdisable\childdocmain{}
  \if?#1?\else
    \begingroup
      \def\childdoctmp{#1}
      \ifx\childdoctmp\childdocname
        \def\childdoctmp{}
      \else
        \def\childdoctmp
        {
          \childdoctrue
          \includeonly{\childdocname}
          \def\childdocjob{#1}
          \def\jobname{#1}
        }
      \fi
      \expandafter
    \endgroup
    \childdoctmp
  \fi
}
%    \end{macrocode}

% \macro{\childdocof}
% The command |\childdocof| redirects
% compilation to the main file |#1|.
%    \begin{macrocode}
\newcommand{\childdocof}[1]
{
  \childdocdisable
  \childdoctrue
  \includeonly{\childdocname}
  \def\jobname{#1}
  \def\childdocjob{#1}
  \input{#1}
}
%    \end{macrocode}

% \macro{\childdocby}
% The command |\childdocby| ....
%    \begin{macrocode}
\newcommand{\childdocby}[2][]
{
  \childdocdisable
  \childdoctrue
  \childdocmanualtrue
  \if?#1?\else
    \def\jobname{#2}
  \fi
  \def\childdocjob{#2}
  \input{#2}
  \endinput
}
%    \end{macrocode}

% \macro{\childdocforward}
% The command |\childdocforward| redirects
% compilation to the main file or
% (if the optional argument is given) a child file.
% Parameters are set as if the main file
% or a child file starting with |\childdocof| was compiled.
% Then compilation is handed over to the main file:
%    \begin{macrocode}
\newcommand{\childdocforward}[2][]
{
  \begingroup
    \if?#1?
      \def\childdoctmp
      {
        \def\childdocname{#2}
        \def\childdocjob{#2}
        \def\jobname{#2}
        \input{#2}
        \endinput
      }
    \else
      \def\childdoctmp
      {
        \childdocdisable
        \def\childdocname{#2}
        \childdoctrue
        \includeonly{#2}
        \def\childdocjob{#1}
        \def\jobname{#1}
        \input{#1}
        \endinput
      }
    \fi
    \expandafter
  \endgroup
  \childdoctmp
}
%    \end{macrocode}

% \macro{\childdocforwardprefix}
% The command |\childdocforwardprefix| redirects
% compilation to the main or a child file by means of a pattern.
% The prefix |#1| in the current filename is replaced by |#2|
% and the suffix of the current filename is kept
% (it is assumed that the filename does not contain the substring `|~~~|'
% which is used as a delimiter).
% Compilation is handed over to the new file by |\childdocforward|:
%    \begin{macrocode}
\newcommand{\childdocforwardprefix}[3][]
{
  \begingroup
    \def\childdocextract #2##1~~~{\def\childdoctmp{\childdocforward[#1]{#3##1}}}
    \expandafter\childdocextract\childdocname~~~
    \expandafter
  \endgroup
  \childdoctmp
}
%    \end{macrocode}

% \macro{\childdoc}
% The deprecated macro |\childdoc| is a legacy version of |\childdocmain|:
%    \begin{macrocode}
\newcommand{\childdoc}{\childdocmain}
%    \end{macrocode}

% \macro{\childdocredirect}
% The deprecated macro |\childdocredirect| is a legacy version
% of |\childdocforward| and |\childdocforwardprefix|:
%    \begin{macrocode}
\newcommand{\childdocredirect}[2][]
{
  \begingroup
    \if?#1?
      \def\childdoctmp{\childdocforward{#2}}
    \else
      \def\childdoctmp{\childdocforwardprefix{#1}{#2}}
    \fi
    \expandafter
  \endgroup
  \childdoctmp
}
%    \end{macrocode}

%\iffalse
%</package>
%\fi
%
\endinput

\childdocmain{}
%    \end{macrocode}

% Optional override for |\version| flag:
%    \begin{macrocode}
%%\ifchilddoc\else\providecommand{\version}{draft}\fi
%    \end{macrocode}

% Define the default values for the |\version| flag
% (|final| for the main file and |draft| for childs):
%    \begin{macrocode}
\ifchilddoc
\providecommand{\version}{draft}
\else
\providecommand{\version}{final}
\fi
%    \end{macrocode}

% Load the standard document class:
%    \begin{macrocode}
\documentclass[12pt]{article}
%    \end{macrocode}

% Start the document body:
%    \begin{macrocode}
\begin{document}
%    \end{macrocode}

% Declare a title page.
% Print title, part of document being processed and version flag:
%    \begin{macrocode}
\addtocounter{page}{-1}
\begin{center}
{\LARGE\bfseries{}childdoc example\par}
\vspace{1cm}
\ifchilddoc
\ifchilddocmanual part\else chapter\fi:
`\childdocname' of `\childdocjob'\par
\else
main document: `\childdocjob'\par
\fi
version: \version\par
\end{center}
\newpage
%    \end{macrocode}

% Manually include selected file,
% otherwise process as usual:
%    \begin{macrocode}
\ifchilddocmanual
\section*{part `\childdocname'}
\input{\childdocname}
\else
%    \end{macrocode}

% Include the two chapters:
%    \begin{macrocode}
\include{cdocsch1}
\include{cdocsch2}
%    \end{macrocode}

% Include the two parts unless only chapters should be displayed:
%    \begin{macrocode}
\ifchilddoc\else
\section{part three}
\input{cdocspt3}
\section{part four}
\input{cdocspt4}
\fi
%    \end{macrocode}

% Process as usual until here:
%    \begin{macrocode}
\fi
%    \end{macrocode}

% End of document body:
%    \begin{macrocode}
\end{document}
%    \end{macrocode}
%\iffalse
%</samplemain>
%\fi
%
% %%%%%%%%%%%%%%%%%%%%%%%%%%%%%%%%%%%%%%
% \paragraph{Chapter Include Files.}
%
% The include files are called |cdocsch1.tex| and |cdocsch2.tex|.
%
%\iffalse
%<*samplechap1|samplechap2>
%\fi

% Optional override for |\version| flag:
%    \begin{macrocode}
%%\providecommand{\version}{final}
%    \end{macrocode}

% Include the main document:
%    \begin{macrocode}
% \iffalse
%
% childdoc.dtx Copyright (C) 2017-2018 Niklas Beisert
%
% This work may be distributed and/or modified under the
% conditions of the LaTeX Project Public License, either version 1.3
% of this license or (at your option) any later version.
% The latest version of this license is in
%   http://www.latex-project.org/lppl.txt
% and version 1.3 or later is part of all distributions of LaTeX
% version 2005/12/01 or later.
%
% This work has the LPPL maintenance status `maintained'.
%
% The Current Maintainer of this work is Niklas Beisert.
%
% This work consists of the files childdoc.dtx and childdoc.ins
% and the derived files childdoc.def and cdocsamp.tex with
% cdocsch1.tex, cdocsch2.tex, cdocsdrf.tex, cdocsfn1.tex, cdocsfn2.tex.
%
%<package>\ifdefined\childdocmain\endinput\fi
%<package>\ProvidesFile{childdoc.def}[2018/12/30 v2.0 child document driver]
%<samplemain>\ProvidesFile{cdocsamp.tex}[2018/12/30 v2.0 sample for childdoc]
%<*driver>
%\ProvidesFile{childdoc.drv}[2018/12/30 v2.0 childdoc reference manual file]
\PassOptionsToClass{10pt,a4paper}{article}
\documentclass{ltxdoc}

\usepackage[margin=35mm]{geometry}
\usepackage{hyperref}
\usepackage{hyperxmp}
\usepackage[usenames]{color}

\hypersetup{colorlinks=true}
\hypersetup{pdfstartview=FitH}
\hypersetup{pdfpagemode=UseNone}
\hypersetup{pdfsource={}}
\hypersetup{pdflang={en-UK}}
\hypersetup{pdfcopyright={Copyright 2017-2018 Niklas Beisert.
  This work may be distributed and/or modified under the
  conditions of the LaTeX Project Public License, either version 1.3
  of this license or (at your option) any later version.}}
\hypersetup{pdflicenseurl={http://www.latex-project.org/lppl.txt}}
\hypersetup{pdfcontactaddress={ETH Zurich, ITP, HIT K,
  Wolfgang-Pauli-Strasse 27}}
\hypersetup{pdfcontactpostcode={8093}}
\hypersetup{pdfcontactcity={Zurich}}
\hypersetup{pdfcontactcountry={Switzerland}}
\hypersetup{pdfcontactemail={nbeisert@itp.phys.ethz.ch}}
\hypersetup{pdfcontacturl={http://people.phys.ethz.ch/\xmptilde nbeisert/}}

\newcommand{\secref}[1]{\hyperref[#1]{section \ref*{#1}}}

\parskip1ex
\parindent0pt
\let\olditemize\itemize
\def\itemize{\olditemize\parskip0pt}

\begin{document}

\title{The \textsf{childdoc} Package}
\hypersetup{pdftitle={The childdoc Package}}
\author{Niklas Beisert\\[2ex]
  Institut f\"ur Theoretische Physik\\
  Eidgen\"ossische Technische Hochschule Z\"urich\\
  Wolfgang-Pauli-Strasse 27, 8093 Z\"urich, Switzerland\\[1ex]
  \href{mailto:nbeisert@itp.phys.ethz.ch}
  {\texttt{nbeisert@itp.phys.ethz.ch}}}
\hypersetup{pdfauthor={Niklas Beisert}}
\hypersetup{pdfsubject={Manual for the LaTeX2e Package childdoc}}
\date{30 December 2018, \textsf{v2.0}}
\maketitle

\begin{abstract}\noindent
\textsf{childdoc} is a \LaTeXe{} package
that enables the direct compilation
of document sections included by |\include|
to individual files.
\end{abstract}

\begingroup
\parskip0ex
\tableofcontents
\endgroup

%%%%%%%%%%%%%%%%%%%%%%%%%%%%%%%%%%%%%%%%%%%%%%%%%%%%%%%%%%%%%%%%%%%%%%%%%%%%%%%%
%%%%%%%%%%%%%%%%%%%%%%%%%%%%%%%%%%%%%%%%%%%%%%%%%%%%%%%%%%%%%%%%%%%%%%%%%%%%%%%%
\section{Introduction}

\LaTeX{} provides a mechanism to structure a large document (such as a book)
into a main file and several child files (containing the chapters)
using the |\include| command.
This mechanism is beneficial for documents
which span hundreds of pages in order to
make the source file(s) more manageable.
Moreover, compilation can be restricted to
selected child files by means of the |\includeonly| command.
The latter feature can be used to reduce the compilation time while editing
(this was significantly more useful in the earlier days of \LaTeX{})
or to generate a smaller document which is easier to navigate.
Another application of |\includeonly| is to generate
documents consisting of selected parts of the complete document.

However, there are a few drawbacks of the plain |\include| mechanism:
\begin{itemize}
\item
The child files cannot be compiled on their own,
they can only be compiled via the main file.
A naive editing environment
(such as a text editor with an option
to have the current file processed by \LaTeX)
may require one to switch to the main file before compiling;
attempting to compile the child file produces errors.
\item
The main file must be modified (each time)
to adjust the |\includeonly| command
to the present needs. This easily leaves the main file in a messy state.
\item
The generated document will always carry the filename
of the main document. This is inconvenient if
several child files are to be compiled and
to be kept for distribution.
\end{itemize}

The present package provides a simple interface
to make child files individually compilable by \LaTeX{}.
Compiling a child file then has the same effect as compiling
the main file with an |\includeonly| command
to select the appropriate child.
Moreover the generated document will carry the name of the child
rather than the main file.
This resolves all three above issues.

This feature is meant to make the editing of books,
thesis documents and lecture notes somewhat more convenient.
However, the package can also be used efficiently for
composing a series of documents (such as exercise sheets)
which are typically distributed individually.
It then assists the author in generating the individual documents
(potentially in different versions)
as well as a document containing the collected series.
Another application is in developing style files
or other kinds of included material
where compilation of the style file could redirect
to a sample or test file.

%%%%%%%%%%%%%%%%%%%%%%%%%%%%%%%%%%%%%%%%%%%%%%%%%%%%%%%%%%%%%%%%%%%%%%%%%%%%%%%%
%%%%%%%%%%%%%%%%%%%%%%%%%%%%%%%%%%%%%%%%%%%%%%%%%%%%%%%%%%%%%%%%%%%%%%%%%%%%%%%%
\section{Usage}

First of all, the package \textsf{childdoc} is \emph{not} a standard
\LaTeXe{} |.sty| style file! Therefore it needs to be invoked in
a non-standard way.

%%%%%%%%%%%%%%%%%%%%%%%%%%%%%%%%%%%%%%%%%%%%%%%%%%%%%%%%%%%%%%%%%%%%%%%%%%%%%%%%
\subsection{Included Files}
\label{sec:include}

%%%%%%%%%%%%%%%%%%%%%%%%%%%%%%%%%%%%%%%%
\DescribeMacro{\childdocmain}
To use the package, add the commands
\begin{center}
\begin{tabular}{l}
|\input{childdoc.def}|\\
|\childdocmain{}|\\
\end{tabular}
\end{center}
at the very top of the main \LaTeX{} file,
in particular \emph{before} the |\documentclass| statement!
The argument of |\childdocmain| should be left empty
(but it must be present).

%%%%%%%%%%%%%%%%%%%%%%%%%%%%%%%%%%%%%%%%
\DescribeMacro{\childdocof}
Furthermore, add the commands
\begin{center}
\begin{tabular}{l}
|\input{childdoc.def}|\\
|\childdocof{|\textit{main}|}|\\
\end{tabular}
\end{center}
at the top of every child file \textit{child}
which is included by |\include{|\textit{child}|}|
from within the main file
(or at least for those files to be compiled individually).
The argument \textit{main} must be the filename of the main file.

There are a couple of
considerations in setting up the main and child documents:

%%%%%%%%%%%%%%%%%%%%%%%%%%%%%%%%%%%%%%%%
\paragraph{Restrictions.}

Please note the following restrictions:
\begin{itemize}
\item
|\childdocmain| must be called with one argument \textit{main}
to ensure compatibility with earlier version of the package.
It must either be empty (|\childdocmain{}|)
or precisely match the filename of the main file in which it is specified.
See \secref{sec:detection} for further information.
\item
The filename \textit{main} must be specified without the |.tex| extension.
\item
The filename \textit{main} is case sensitive
(even in case-insensitive file systems)
due to internal string comparison.
\item
The argument \textit{main} should be fully expanded, it cannot be a macro.
\item
Subdirectories and special characters should be avoided in filenames.
\item
The command |\childdocmain{|\textit{main}|}| must be followed by a whitespace.
It should not be followed immediately by another command
or by a comment mark `|%|'.
This is because the \TeX{} parser reads the token immediately following
the argument of |\childdocmain| and puts it
at the beginning of every child section;
however, a white\-space is ignored.
\end{itemize}

%%%%%%%%%%%%%%%%%%%%%%%%%%%%%%%%%%%%%%%%
\paragraph{Content of Main File.}

It is advisable to place all content in the child files included by |\include|.
Any output contained in the main file will appear in all child documents
unless suppressed manually;
it cannot be suppressed automatically by the |\includeonly| directive
and thus should normally be avoided.
A method to include some content in the main file
by means of conditional processing is described in \secref{sec:conditional}.

%%%%%%%%%%%%%%%%%%%%%%%%%%%%%%%%%%%%%%%%
\paragraph{Page Numbering.}

When only a part of the document is compiled,
the appropriate numbering of pages
(as well as other status parameters)
is determined from the |.aux| files.
The latter contain information from previous passes.
However this information needs to propagate through
all intermediate child documents.
Therefore the page numbering in child documents may well
be inconsistent until the complete document is compiled at least once.

A useful (if unconventional) way to always ensure a consistent
page numbering is to restart the numbering in each child document
and denote the pages by `\textit{child}|.|\textit{page}'
where \textit{child} represents the chapter/section number of the child file.
This can be achieved by the command
|\numberwithin{page}{|\textit{child}|}|
of the \textsf{amsmath} package
where \textit{child} can be |chapter| or |section|
depending on the chosen structuring.
Alternatively, one can modify the macro |\thepage| appropriately
and reset the counter |page| at the start of each child file.

%%%%%%%%%%%%%%%%%%%%%%%%%%%%%%%%%%%%%%%%%%%%%%%%%%%%%%%%%%%%%%%%%%%%%%%%%%%%%%%%
\subsection{Conditional Processing}
\label{sec:conditional}

The package provides a mechanism to compile different versions
of a document. To customise the versions further some conditional processing
can come in handy to distinguish which version is being compiled.
The package provides two macros to describe the compilation context:

%%%%%%%%%%%%%%%%%%%%%%%%%%%%%%%%%%%%%%%%
\DescribeMacro{\ifchilddoc}
The conditional |\ifchilddoc| distinguishes between the compilation of
child documents and the main document:
%
\begin{center}
|\ifchilddoc |\textit{child-code}| |[|\||else |\textit{main-code}]| \||fi|
\end{center}

%%%%%%%%%%%%%%%%%%%%%%%%%%%%%%%%%%%%%%%%
\DescribeMacro{\childdocname}
\DescribeMacro{\childdocjob}
The macro |\childdocname| contains the filename (without extension)
of the main or child file being processed.
Note that |\childdocjob| will always contain the name of the main file.

%%%%%%%%%%%%%%%%%%%%%%%%%%%%%%%%%%%%%%%%
\paragraph{Title Page.}

Conditional processing can be used to include a title or banner page
in the main document when proper precautions are taken.
Importantly, the code in the main file should ensure that the page counter
(as well as other status parameters which are stored in the |.aux| files)
takes the same value after the conditional processing.
Otherwise the page numbers may take divergent values
depending on which part is compiled.

For example, a title page could be declared by:
%
\begin{center}
\begin{tabular}{l}
|\ifchilddoc\||else|\\
|\addtocounter{page}{-1}|\\
\textit{code for title page}\\
|\newpage|\\
|\||fi|
\end{tabular}
\end{center}
%
A banner page for the child documents can be generated by:
%
\begin{center}
\begin{tabular}{l}
|\ifchilddoc|\\
|\addtocounter{page}{-1}|\\
\textit{code for banner page}\\
|\newpage|\\
|\||fi|
\end{tabular}
\end{center}
%
Here one could write a message such as:
\begin{center}
|This is the part \childdocname{} of \childdocjob{}.|
\end{center}

%%%%%%%%%%%%%%%%%%%%%%%%%%%%%%%%%%%%%%%%%%%%%%%%%%%%%%%%%%%%%%%%%%%%%%%%%%%%%%%%
\subsection{Flags}
\label{sec:flags}

The package makes it easy to generate different versions
of the main or child documents.
To this end compilation flags can be defined
and assigned different default values.
They will be particularly useful in conjunction
with the forwarding mechanism described in \secref{sec:forward}.

For example, it may be useful to have a flag |\version|
which can be set to |draft| or |final|.
The document source will contain some conditional code
depending on the value of |\version|.
Suppose further, the flag should default to |final| for the main file
and to |draft| for child files
which is a natural assignment for editing the document.
This is achieved by placing the following code
in the preamble of the main document
(below the |\childdocmain| directive):
%
\begin{center}
\begin{tabular}{l}
|\ifchilddoc|\\
|\providecommand{\version}{draft}|\\
|\||else|\\
|\providecommand{\version}{final}|\\
|\||fi|
\end{tabular}
\end{center}
%
The definition by |\providecommand| makes sure
that previous definitions are not overwritten.
Further statements |\providecommand{\version}{...}|
can thus be added before the above code to override it.

For the main file, one might add a line
(between |\childdocmain| and the above block)
%
\begin{center}
|%\ifchilddoc\||else\providecommand{\version}{draft}\||fi|
\end{center}
%
which can be uncommented to produce a draft version.
Likewise one can add a line to the very top of a child file
(above the |\childdocof{|\textit{main}|}| directive)
%
\begin{center}
|%\providecommand{\version}{final}|
\end{center}
%
which can be uncommented to produce the final version of this child document.

%%%%%%%%%%%%%%%%%%%%%%%%%%%%%%%%%%%%%%%%%%%%%%%%%%%%%%%%%%%%%%%%%%%%%%%%%%%%%%%%
\subsection{Forwarding}
\label{sec:forward}

Different versions of the main or child documents
using compilation flags as described in \secref{sec:flags}
can be (permanently) stored in different files
for convenient compilation, viewing and distribution.
To this end, the package defines a command
to pass on compilation to a different file:

%%%%%%%%%%%%%%%%%%%%%%%%%%%%%%%%%%%%%%%%
\DescribeMacro{\childdocforward}
The command |\childdocforward| redirects processing to
another source file:
%
\begin{center}
\begin{tabular}{l}
|\input{childdoc.def}|\\
|\childdocforward[|\textit{main}|]{|\textit{dest}|}|\\
\end{tabular}
\end{center}
%
The argument \textit{dest} is the destination file
(without extension).
It should be the main file or one of the child files.
Note that further \textsf{childdoc} directives
such as |\childdocof| and |\childdocforward|
in the indicated file will be processed in this form.
The optional argument \textit{main}
passes on directly to the main file \textit{main}
while pretending to compile the child \textit{dest}.
This form behaves as if \textit{dest}
issues |\childdocof{|\textit{main}|}| right away,
and no further \textsf{childdoc} directives will be processed.

%%%%%%%%%%%%%%%%%%%%%%%%%%%%%%%%%%%%%%%%
\DescribeMacro{\...prefix}
In the alternative form |\childdocforwardprefix|,
%
\begin{center}
\begin{tabular}{l}
|\input{childdoc.def}|\\
|\childdocforwardprefix[|\textit{main}|]{|\textit{prefix}|}{|\textit{dest}|}|
\end{tabular}
\end{center}
%
the destination file is determined by a pattern
depending on the current file:
To make this work, the current file must be called
`{\textit{prefix}\hspace{0.2em}\textit{suffix}}'
with \textit{prefix} matching precisely the argument.
Processing is then passed on to the file
`{\textit{dest}\hspace{0.2em}\textit{suffix}}'.
Surely, the same effect is achieved by
directly specifying the
argument `{\textit{dest}\hspace{0.2em}\textit{suffix}}'
in the first form.
However, that requires to set up a different file
for each child. With the alternative form of the command
all these files can have exactly the same content
which simplifies setting them up and maintaining them.

For example, the following file |draft.tex|
with a compilation flag |\version| as described in \secref{sec:flags}
compiles the main document as a draft:
%
\begin{center}
\begin{tabular}{l}
|\def\version{draft}|\\
|\input{childdoc.def}|\\
|\childdocforward{|\textit{main}|}|
\end{tabular}
\end{center}
%
Likewise, the following files |final|\textit{nn}|.tex|
compile the final version of the child document
|child|\textit{nn}|.tex|:
%
\begin{center}
\begin{tabular}{l}
|\def\version{final}|\\
|\input{childdoc.def}|\\
|\childdocforwardprefix{final}{child}|
\end{tabular}
\end{center}
%

Note that when several versions of a main file and/or of each child file
are to be generated, it may be convenient to set up a |Makefile| or
shell script to automatise the process.

%%%%%%%%%%%%%%%%%%%%%%%%%%%%%%%%%%%%%%%%%%%%%%%%%%%%%%%%%%%%%%%%%%%%%%%%%%%%%%%%
\subsection{Command Line Processing}
\label{sec:commandline}

The effect of redirection files can also be achieved by invoking
the \LaTeX{} compiler with a more elaborate command line.
Most conveniently this should be done as part
of a shell script or a |Makefile|.

When using \textsf{childdoc} in the main file, the following
command lines effectively perform a redirection
(note that depending on the shell being used,
backslashes may have to be doubled: `|\|' $\to$ `|\\|'):
%
\begin{center}
|... -jobname "|\textit{target}|" |\\|"|[\textit{flags}]%
|\input{childdoc.def}\childdocforward[|\textit{main}|]{|\textit{dest}|}"|
\end{center}
%
Here \textit{target} is the name of the output file,
\textit{main} is the name of the main file
and \textit{dest} is the name of the main or child file to be processed
(all filenames without extensions).
The optional argument \textit{main} can be omitted
if \textit{main} matches \textit{dest}.
Optionally, compilation \textit{flags} can be defined via |\def| commands.
This command line makes the \TeX{} engine believe
it is compiling the file \textit{target}
whose content is specified as the latter parameter.
The provided code then forwards the processing to
\textit{main} or \textit{dest} as described in \secref{sec:forward}.

%%%%%%%%%%%%%%%%%%%%%%%%%%%%%%%%%%%%%%%%%%%%%%%%%%%%%%%%%%%%%%%%%%%%%%%%%%%%%%%%
\subsection{Include by Input}
\label{sec:input}

Including child documents by |\include| has some restrictions by design.
Most notably, the content of a child document always occupies
its own set of pages; pages cannot be shared between child documents.
Usually, this behaviour makes perfect sense
because each child document contain an essential part of the document.
However, in some situations it may be desirable to compose
a document from a collection of parts
without having mandatory page breaks between then.
For this case, the package
provides a mechanism to include parts
by |\input| which can also be processed individually.
However, by construction this mechanism
requires manual handling of the content to be output.

%%%%%%%%%%%%%%%%%%%%%%%%%%%%%%%%%%%%%%%%
\DescribeMacro{\ifchilddocmanual}
The main file should be prepared as usual, see \secref{sec:include}.
However, the document body must make a distinction
between processing of an individual part and of the main document, e.g.:
%
\begin{center}
\begin{tabular}{l}
|\ifchilddocmanual|\\
|\input{\childdocname}|\\
|\||else|\\
\textit{document body with }|\input{|\textit{part}|}|\\
|\||fi|
\end{tabular}
\end{center}
%
The conditional |\ifchilddocmanual| is true whenever
a part to be included by |\input| is being compiled,
and the name of the part is stored in |\childdocname|.

%%%%%%%%%%%%%%%%%%%%%%%%%%%%%%%%%%%%%%%%
\DescribeMacro{\childdocby}
Each part to be included by |\input| should start with:
%
\begin{center}
\begin{tabular}{l}
|\input{childdoc.def}|\\
|\childdocby{|\textit{main}|}|\\
\end{tabular}
\end{center}
%
The directive |\childdocby| is similar to |\childdocof|
described in \secref{sec:include},
but the subsequent selection of content must be done manually.
To that end, both |\ifchilddoc| and |\ifchilddocmanual|
will be true upon processing of a part,
and the name of the part is stored in |\childdocname|.
Note that |\jobname| will be set to the filename of the current part
so that each part receives an individual |.aux| file
that does not interfere with the |.aux| file(s) of the main document.
This behaviour can be altered by the alternative form
|\childdocby[*]{|\textit{main}|}| (with a non-empty optional argument)
which uses the |.aux| file of the main document
by setting |\jobname| to \textit{main}.

%%%%%%%%%%%%%%%%%%%%%%%%%%%%%%%%%%%%%%%%%%%%%%%%%%%%%%%%%%%%%%%%%%%%%%%%%%%%%%%%
\subsection{Driver Development}
\label{sec:driver}

The \textsf{childdoc} mechanism can also be use for the development
of definition files such as \LaTeX{} styles or classes.
This case differs from the above setup with multiple parts
included by |\include| in that no |\includeonly| should be invoked.
This can be achieved by starting the include file
(before |\ProvidesPackage|) with:
%
\begin{center}
\begin{tabular}{l}
|\input{childdoc.def}|\\
|\childdocforward{|\textit{main}|}|\\
\end{tabular}
\end{center}
%
or alternatively with:
%
\begin{center}
\begin{tabular}{l}
|\input{childdoc.def}|\\
|\childdocby{|\textit{main}|}|\\
\end{tabular}
\end{center}
%
Both forms have slightly different effects as described above.
The main file is prepared as usual, see \secref{sec:include}.

%%%%%%%%%%%%%%%%%%%%%%%%%%%%%%%%%%%%%%%%%%%%%%%%%%%%%%%%%%%%%%%%%%%%%%%%%%%%%%%%
\subsection{Legacy Detection}
\label{sec:detection}

The directive |\childdocmain| in the main file can detect
whether the complete document or merely a child is to be compiled
even without using the directive |\childdocof|.
This method is deprecated because it is less robust
and there is no compelling reason to use it;
it is merely provided for backward compatibility
and it may be removed in future versions.

If the detection mechanism is to be used,
it is mandatory to correctly specify
the filename of the main file as the argument of |\childdocmain|:
%
\begin{center}
\begin{tabular}{l}
|\input{childdoc.def}|\\
|\childdocmain{|\textit{main}|}|\\
\end{tabular}
\end{center}
%
If |\jobname| does not match the argument \textit{main} of |\childdocmain|,
it is assumed that |\jobname| points to the child file to be compiled.
When using |\childdocmain| with the main file specified as argument,
it suffices to start a child file
with just |\input{|\textit{main}|}|
without loading of the package and using |\childdocof|.
If instead all processing is done
with the appropriate \textsf{childdoc} directives,
the argument of \textit{main} of |\childdocmain| can be empty.

An alternative version of the command line processing described
in \secref{sec:commandline} using the detection mechanism reads:
%
\begin{center}
|... -jobname "|\textit{target}|" "|[\textit{flags}]%
[|\def\jobname{|\textit{dest}|}|]|\input{|\textit{main}|}"|
\end{center}

%%%%%%%%%%%%%%%%%%%%%%%%%%%%%%%%%%%%%%%%%%%%%%%%%%%%%%%%%%%%%%%%%%%%%%%%%%%%%%%%
\subsection{Manual Code}
\label{sec:manual}

In case one cannot be certain whether the definitions file |childdoc.def|
is installed on the target \TeX{} distribution
and one prefers not to ship it,
it is conceivable to paste a few relevant commands into the sources.

To that end, drop all statements |\input{childdoc.def}|
and perform the replacements as outlined below.
Instead of |\childdocmain{|\textit{main}|}| add the following code
to the top of the main file:
%
\begin{center}
\begin{tabular}{l}
|\||ifdefined\childdocname\endinput\||fi\newif\ifchilddoc|\\
|\edef\childdocname{\scantokens\expandafter{\jobname\noexpand}}|\\
|\def\childdocmain{|\textit{main}|}\||ifx\childdocmain\childdocname\||else|\\
|\childdoctrue\includeonly{\childdocname}\let\jobname\childdocmain\||fi|\\
\end{tabular}
\end{center}
%
Instead of |\childdocof{|\textit{main}|}| just include the main file
at the top of each child file:
%
\begin{center}
|\input{|\textit{main}|}|
\end{center}
%
A simple redirection |\childdocforward{|\textit{dest}|}| is achieved by:
%
\begin{center}
|\def\jobname{|\textit{dest}|}\input{\jobname}|
\end{center}
%
The redirection with prefix
|\childdocforwardprefix[|\textit{prefix}|]{|\textit{dest}|}|
is accomplished by:
%
\begin{center}
\begin{tabular}{l}
|{\edef\jobname{\scantokens\expandafter{\jobname\noexpand}}|\\
|\def\redirectjob |\textit{prefix}|#1~~~{\gdef\jobname{|\textit{dest}|#1}}|\\
|\expandafter\redirectjob\jobname~~~}\input{\jobname}|
\end{tabular}
\end{center}

In an alternative approach,
child documents can be compiled by a specific command line
without additional code or specific definitions:
%
\begin{center}
|... -jobname "|\textit{target}|" "|[\textit{flags}]%
|\includeonly{|\textit{dest}|}\input{|\textit{main}|}"|
\end{center}
%

%%%%%%%%%%%%%%%%%%%%%%%%%%%%%%%%%%%%%%%%%%%%%%%%%%%%%%%%%%%%%%%%%%%%%%%%%%%%%%%%
%%%%%%%%%%%%%%%%%%%%%%%%%%%%%%%%%%%%%%%%%%%%%%%%%%%%%%%%%%%%%%%%%%%%%%%%%%%%%%%%
\section{Information}

%%%%%%%%%%%%%%%%%%%%%%%%%%%%%%%%%%%%%%%%%%%%%%%%%%%%%%%%%%%%%%%%%%%%%%%%%%%%%%%%
\subsection{Copyright}

Copyright \copyright{} 2017--2018 Niklas Beisert

This work may be distributed and/or modified under the
conditions of the \LaTeX{} Project Public License, either version 1.3
of this license or (at your option) any later version.
The latest version of this license is in
  \url{http://www.latex-project.org/lppl.txt}
and version 1.3 or later is part of all distributions of \LaTeX{}
version 2005/12/01 or later.

This work has the LPPL maintenance status `maintained'.

The Current Maintainer of this work is Niklas Beisert.

This work consists of the files |README.txt|, |childdoc.ins| and |childdoc.dtx|
as well as the derived files |childdoc.def|, |cdocsamp.tex|
with |cdocsch1.tex|, |cdocsch2.tex|, |cdocspt3.tex|, |cdocspt4.tex|,
|cdocsdrf.tex|, |cdocsfn1.tex|, |cdocsfn2.tex|
as well as |childdoc.pdf|.

%%%%%%%%%%%%%%%%%%%%%%%%%%%%%%%%%%%%%%%%%%%%%%%%%%%%%%%%%%%%%%%%%%%%%%%%%%%%%%%%
\subsection{Files and Installation}

The package consists of the files:
%
\begin{center}
\begin{tabular}{ll}
    |README.txt|   & readme file \\
    |childdoc.ins| & installation file \\
    |childdoc.dtx| & source file \\
    |childdoc.def| & definition file \\
    |cdocsamp.tex| & sample main file \\
    |cdocsch1.tex| & sample include file \\
    |cdocsch2.tex| & sample include file \\
    |cdocspt3.tex| & sample part file \\
    |cdocspt4.tex| & sample part file \\
    |cdocsdrf.tex| & sample redirection file \\
    |cdocsfn1.tex| & sample redirection file \\
    |cdocsfn2.tex| & sample redirection file \\
    |childdoc.pdf| & manual
\end{tabular}
\end{center}
%
The distribution consists of the files
|README.txt|, |childdoc.ins| and |childdoc.dtx|.
%
\begin{itemize}
\item
Run (pdf)\LaTeX{} on |childdoc.dtx|
to compile the manual |childdoc.pdf| (this file).
\item
Run \LaTeX{} on |childdoc.ins| to create the definitions file |childdoc.def|
and the sample |cdocsamp.tex| with include files
|cdocsch1.tex|, |cdocsch2.tex|, |cdocspt3.tex|, |cdocspt4.tex|,
|cdocsdrf.tex|, |cdocsfn1.tex|, |cdocsfn2.tex|.
Then copy the file |childdoc.def| to an appropriate directory of your \LaTeX{}
distribution, e.g.\ \textit{texmf-root}|/tex/latex/childdoc|.
\end{itemize}

%%%%%%%%%%%%%%%%%%%%%%%%%%%%%%%%%%%%%%%%%%%%%%%%%%%%%%%%%%%%%%%%%%%%%%%%%%%%%%%%
\subsection{Related CTAN Packages}

There are several other packages which offer a similar functionality:
%
\begin{itemize}
\item
The packages
\href{http://ctan.org/pkg/docmute}{\textsf{docmute}},
\href{http://ctan.org/pkg/includex}{\textsf{includex}} and
\href{http://ctan.org/pkg/standalone}{\textsf{standalone}}
provide commands to include only the document body of
a child file thus allowing both files to be compiled individually.
\item
The packages \href{http://ctan.org/pkg/subdocs}{\textsf{subdocs}}
and \href{http://ctan.org/pkg/subfiles}{\textsf{subfiles}}
provide structures in which the main and child documents can be
encapsulated and allowing them to be compiled individually.
The inclusion mechanism is different from the conventional |\include|.
\item
The package \href{http://ctan.org/pkg/combine}{\textsf{combine}}
is an elaborate solution to combine several documents into one.
\end{itemize}
%
See also the CTAN topic \href{http://ctan.org/topic/subdocs}{\textsf{subdocs}}
for further related packages.
The present package differs from the above solutions in that
a document structure constructed with the conventional |\include| mechanism
just needs two extra commands at the top of every file
such that all constituent files can be compiled individually.

%%%%%%%%%%%%%%%%%%%%%%%%%%%%%%%%%%%%%%%%%%%%%%%%%%%%%%%%%%%%%%%%%%%%%%%%%%%%%%%%
%\subsection{Feature Suggestions}
%
%The following is a list of features which may be useful for future
%versions of this package:
%%
%\begin{itemize}
%\item
%\ldots
%\end{itemize}

%%%%%%%%%%%%%%%%%%%%%%%%%%%%%%%%%%%%%%%%%%%%%%%%%%%%%%%%%%%%%%%%%%%%%%%%%%%%%%%%
\subsection{Revision History}

%%%%%%%%%%%%%%%%%%%%%%%%%%%%%%%%%%%%%%%%
\paragraph{v2.0:} 2018/12/30

\begin{itemize}
\item
immediate forward processing
\item
added |\childdocby| mechanism
\item
manual restructured
\end{itemize}

%%%%%%%%%%%%%%%%%%%%%%%%%%%%%%%%%%%%%%%%
\paragraph{v1.6:} 2018/01/17

\begin{itemize}
\item
application for development of include files
\item
corrections to manual
\end{itemize}

%%%%%%%%%%%%%%%%%%%%%%%%%%%%%%%%%%%%%%%%
\paragraph{v1.5:} 2017/05/21

\begin{itemize}
\item
more complete structuring introduced
\item
|\childdocof| introduced
\item
|\childdoc| renamed to |\childdocmain|
\item
|\childredirect| renamed to |\childdocforward| and |\childdocforwardprefix|
and functionality expanded
\end{itemize}

%%%%%%%%%%%%%%%%%%%%%%%%%%%%%%%%%%%%%%%%
\paragraph{v1.0:} 2017/04/27

\begin{itemize}
\item
manual and install package
\item
first version published on CTAN
\end{itemize}

%%%%%%%%%%%%%%%%%%%%%%%%%%%%%%%%%%%%%%%%
\paragraph{v0.6:} 2017/04/26

\begin{itemize}
\item
redirection mechanism added
\end{itemize}

%%%%%%%%%%%%%%%%%%%%%%%%%%%%%%%%%%%%%%%%
\paragraph{v0.5:} 2017/04/26

\begin{itemize}
\item
functionality in definition file
\end{itemize}


%%%%%%%%%%%%%%%%%%%%%%%%%%%%%%%%%%%%%%%%%%%%%%%%%%%%%%%%%%%%%%%%%%%%%%%%%%%%%%%%
%%%%%%%%%%%%%%%%%%%%%%%%%%%%%%%%%%%%%%%%%%%%%%%%%%%%%%%%%%%%%%%%%%%%%%%%%%%%%%%%
%%%%%%%%%%%%%%%%%%%%%%%%%%%%%%%%%%%%%%%%%%%%%%%%%%%%%%%%%%%%%%%%%%%%%%%%%%%%%%%%
\appendix

\settowidth\MacroIndent{\rmfamily\scriptsize 000\ }

 \DocInput{childdoc.dtx}

\end{document}
%</driver>
% \fi
%
% %%%%%%%%%%%%%%%%%%%%%%%%%%%%%%%%%%%%%%%%%%%%%%%%%%%%%%%%%%%%%%%%%%%%%%%%%%%%%%
% %%%%%%%%%%%%%%%%%%%%%%%%%%%%%%%%%%%%%%%%%%%%%%%%%%%%%%%%%%%%%%%%%%%%%%%%%%%%%%
% \section{Sample}
%\iffalse
%<*samplemain>
%\fi
%
% The following presents a sample document
% with two chapters, two parts, a title page,
% a compile flag as well as three forwarding files to set the flag.
% It consists of eight |.tex| files:
% \begin{center}
% \begin{tabular}{ll}
% |cdocsamp.tex|&main file\\
% |cdocsch1.tex|&include file for chapter 1\\
% |cdocsch2.tex|&include file for chapter 2\\
% |cdocspt3.tex|&include file for part 3\\
% |cdocspt4.tex|&include file for part 4\\
% |cdocsdrf.tex|&forwarding file for main file in draft mode\\
% |cdocsfi1.tex|&forwarding file for final version of chapter 1\\
% |cdocsfi2.tex|&forwarding file for final version of chapter 2\\
% \end{tabular}
% \end{center}
% Each of the eight files can be compiled directly by the \LaTeX{} compiler.
%
% %%%%%%%%%%%%%%%%%%%%%%%%%%%%%%%%%%%%%%
% \paragraph{Main File.}
%
% The main file is called |cdocsamp.tex|.
%
% Load the \textsf{childdoc} definitions and
% declare the filename for the main document:
%    \begin{macrocode}
\input{childdoc.def}
\childdocmain{}
%    \end{macrocode}

% Optional override for |\version| flag:
%    \begin{macrocode}
%%\ifchilddoc\else\providecommand{\version}{draft}\fi
%    \end{macrocode}

% Define the default values for the |\version| flag
% (|final| for the main file and |draft| for childs):
%    \begin{macrocode}
\ifchilddoc
\providecommand{\version}{draft}
\else
\providecommand{\version}{final}
\fi
%    \end{macrocode}

% Load the standard document class:
%    \begin{macrocode}
\documentclass[12pt]{article}
%    \end{macrocode}

% Start the document body:
%    \begin{macrocode}
\begin{document}
%    \end{macrocode}

% Declare a title page.
% Print title, part of document being processed and version flag:
%    \begin{macrocode}
\addtocounter{page}{-1}
\begin{center}
{\LARGE\bfseries{}childdoc example\par}
\vspace{1cm}
\ifchilddoc
\ifchilddocmanual part\else chapter\fi:
`\childdocname' of `\childdocjob'\par
\else
main document: `\childdocjob'\par
\fi
version: \version\par
\end{center}
\newpage
%    \end{macrocode}

% Manually include selected file,
% otherwise process as usual:
%    \begin{macrocode}
\ifchilddocmanual
\section*{part `\childdocname'}
\input{\childdocname}
\else
%    \end{macrocode}

% Include the two chapters:
%    \begin{macrocode}
\include{cdocsch1}
\include{cdocsch2}
%    \end{macrocode}

% Include the two parts unless only chapters should be displayed:
%    \begin{macrocode}
\ifchilddoc\else
\section{part three}
\input{cdocspt3}
\section{part four}
\input{cdocspt4}
\fi
%    \end{macrocode}

% Process as usual until here:
%    \begin{macrocode}
\fi
%    \end{macrocode}

% End of document body:
%    \begin{macrocode}
\end{document}
%    \end{macrocode}
%\iffalse
%</samplemain>
%\fi
%
% %%%%%%%%%%%%%%%%%%%%%%%%%%%%%%%%%%%%%%
% \paragraph{Chapter Include Files.}
%
% The include files are called |cdocsch1.tex| and |cdocsch2.tex|.
%
%\iffalse
%<*samplechap1|samplechap2>
%\fi

% Optional override for |\version| flag:
%    \begin{macrocode}
%%\providecommand{\version}{final}
%    \end{macrocode}

% Include the main document:
%    \begin{macrocode}
\input{childdoc.def}
\childdocof{cdocsamp}
%    \end{macrocode}

%\iffalse
%</samplechap1|samplechap2>
%\fi
%
%\iffalse
%<*samplechap1>
%\fi
% Some text for chapter 1:
%    \begin{macrocode}
\section{one}
some text in chapter one
%    \end{macrocode}

%\iffalse
%</samplechap1>
%\fi
% Some text for chapter 2:
%\iffalse
%<*samplechap2>
%\fi
%    \begin{macrocode}
\section{two}
more text in chapter two
%    \end{macrocode}

%\iffalse
%</samplechap2>
%\fi
%
% %%%%%%%%%%%%%%%%%%%%%%%%%%%%%%%%%%%%%%
% \paragraph{Part Include Files.}
%
% The include files are called |cdocspt3.tex| and |cdocspt4.tex|.
%
%\iffalse
%<*samplepart3|samplepart4>
%\fi

% Optional override for |\version| flag:
%    \begin{macrocode}
%%\providecommand{\version}{final}
%    \end{macrocode}

% Include the main document:
%    \begin{macrocode}
\input{childdoc.def}
\childdocby{cdocsamp}
%    \end{macrocode}

%\iffalse
%</samplepart3|samplepart4>
%\fi
%
%\iffalse
%<*samplepart3>
%\fi
% Some text for part 3:
%    \begin{macrocode}
some text in part three
%    \end{macrocode}

%\iffalse
%</samplepart3>
%\fi
% Some text for part 4:
%\iffalse
%<*samplepart4>
%\fi
%    \begin{macrocode}
more text in part four
%    \end{macrocode}

%\iffalse
%</samplepart4>
%\fi
%
% %%%%%%%%%%%%%%%%%%%%%%%%%%%%%%%%%%%%%%
% \paragraph{Forwarding for a Complete Draft.}
%
% The following forwarding file |cdocsdrf.tex|
% compiles the main document in draft mode:
%\iffalse
%<*sampledraft>
%\fi
%    \begin{macrocode}
\def\version{draft}
\input{childdoc.def}
\childdocforward{cdocsamp}
%    \end{macrocode}

%\iffalse
%</sampledraft>
%\fi
%
% %%%%%%%%%%%%%%%%%%%%%%%%%%%%%%%%%%%%%%
% \paragraph{Forwarding for Final Version of the Chapters.}
%
% The following forwarding files |cdocsfn1.tex| and |cdocsfn2.tex|
% (with identical content)
% compile the final versions of the child documents
% |cdocsch1.tex| and |cdocsch2.tex|, respectively:
%\iffalse
%<*samplefinal>
%\fi
%    \begin{macrocode}
\def\version{final}
\input{childdoc.def}
\childdocforwardprefix[cdocsamp]{cdocsfn}{cdocsch}
%    \end{macrocode}

%\iffalse
%</samplefinal>
%\fi
%
% %%%%%%%%%%%%%%%%%%%%%%%%%%%%%%%%%%%%%%
% \paragraph{Command Line Processing.}
%
% The following three command lines generate the output files
% |cdocscld|, |cdocscl1| and |cdocscl2|
% which should be identical to
% |cdocsdrf|, |cdocsch1| and |cdocsfn2|, respectively:
% \begin{center}
% \begin{tabular}{l}
% |latex -jobname cdocscld \|\\
% |  "\def\version{draft}\input{childdoc.def}\childdocforward{cdocsamp}"|\\
% |latex -jobname cdocscl1 \|\\
% |  "\input{childdoc.def}\childdocforward[cdocsamp]{cdocsch1}"|\\
% |latex -jobname cdocscl2 \|\\
% |  "\def\version{final}\input{childdoc.def}\childdocforward{cdocsch2}"|
% \end{tabular}
% \end{center}
% Note that the trailing backslash on each first line
% merely continues the input to the second line
% (for convenient cut ant paste).
% Furthermore, the command |latex| can be replaced by any
% of its alternative versions such as |pdflatex|.
%
% %%%%%%%%%%%%%%%%%%%%%%%%%%%%%%%%%%%%%%%%%%%%%%%%%%%%%%%%%%%%%%%%%%%%%%%%%%%%%%
% %%%%%%%%%%%%%%%%%%%%%%%%%%%%%%%%%%%%%%%%%%%%%%%%%%%%%%%%%%%%%%%%%%%%%%%%%%%%%%
% \section{Implementation}
%\iffalse
%<*package>
%\fi
%
% This section describes the definitions file |childdoc.def|.

% The definitions cannot be loaded using |\usepackage| or |\RequirePackage|
% which has a mechanism to prevent loading a style file more than once.
% When loading the definitions by means of |\input|
% multiple instances have to be prevented manually:
%\iffalse
%This code needs to be before the `\ProvidesFile' directive
%which is defined at the beginning of this file.
%Therefore it is also placed there and commented out here.
%</package>
%<*discard>
%\fi
%    \begin{macrocode}
\ifdefined\childdocmain\endinput\fi
%    \end{macrocode}
%\iffalse
%</discard>
%<*package>
%\fi
%
% \macro{\ifchilddoc}
% \macro{\ifchilddocmanual}
% The conditional |\ifchilddoc| tells whether a
% child (true) or main (false) document is being compiled.
% The conditional |\ifchilddocmanual| tells whether
% the |\includeonly| mechanism is used (false) or
% the selection of child files must be performed manually (true).
% The definitions initialise to false:
%    \begin{macrocode}
\newif\ifchilddoc
\newif\ifchilddocmanual
%    \end{macrocode}

% \macro{\childdocname}
% \macro{\childdocjob}
% The macro |\childdocname| stores the name of the main document
% to be compiled. The macro |\childdocjob| stores the name of
% the document on which the \LaTeX{} compiler was originally invoked.
% The content of |\jobname| cannot be compared
% to filenames specified in the source due to different catcodes.
% The following code rescans |\jobname|, stores the result
% in |\childdocname| and saves a copy in |\childdocjob|:
%    \begin{macrocode}
\edef\childdocname{\scantokens\expandafter{\jobname\noexpand}}
\let\childdocjob\childdocname
%    \end{macrocode}

% \macro{\childdocdisable}
% The macro |\childdocdisable| prevents the main file
% from being processed more than once.
% At this stage, the main document command |\childdocmain|
% is assumed to be called once again where it should do nothing.
% Any subsequent call to it should prevent
% a secondary processing of the main document
% It overwrites the forwarding commands
% |\childdocof| and |\childdocforward|
% with empty macros to prevent further inclusions of the main document:
%    \begin{macrocode}
\newcommand{\childdocdisable}
{
  \renewcommand{\childdocmain}[1]{\renewcommand{\childdocmain}[1]{\endinput}}
  \renewcommand{\childdocof}[1]{}
  \renewcommand{\childdocby}[2][]{}
  \renewcommand{\childdocforward}[2][]{}
  \renewcommand{\childdocdisable}{}
}
%    \end{macrocode}

% \macro{\childdocmain}
% The macro |\childdocmain| is to be called at the top of the main file
% with nothing or the main filename (without extension) as argument.
% First, it breaks loops.
% If the argument is not empty and does not match |\childdocname|
% (which is set by the first inclusion of |childdoc.def|),
% |\ifchilddoc| is set to true, |\includeonly| is applied to the child file
% and |\jobname| is set to the main file
% (for proper handling of |.aux| files):
%    \begin{macrocode}
\newcommand{\childdocmain}[1]
{
  \childdocdisable\childdocmain{}
  \if?#1?\else
    \begingroup
      \def\childdoctmp{#1}
      \ifx\childdoctmp\childdocname
        \def\childdoctmp{}
      \else
        \def\childdoctmp
        {
          \childdoctrue
          \includeonly{\childdocname}
          \def\childdocjob{#1}
          \def\jobname{#1}
        }
      \fi
      \expandafter
    \endgroup
    \childdoctmp
  \fi
}
%    \end{macrocode}

% \macro{\childdocof}
% The command |\childdocof| redirects
% compilation to the main file |#1|.
%    \begin{macrocode}
\newcommand{\childdocof}[1]
{
  \childdocdisable
  \childdoctrue
  \includeonly{\childdocname}
  \def\jobname{#1}
  \def\childdocjob{#1}
  \input{#1}
}
%    \end{macrocode}

% \macro{\childdocby}
% The command |\childdocby| ....
%    \begin{macrocode}
\newcommand{\childdocby}[2][]
{
  \childdocdisable
  \childdoctrue
  \childdocmanualtrue
  \if?#1?\else
    \def\jobname{#2}
  \fi
  \def\childdocjob{#2}
  \input{#2}
  \endinput
}
%    \end{macrocode}

% \macro{\childdocforward}
% The command |\childdocforward| redirects
% compilation to the main file or
% (if the optional argument is given) a child file.
% Parameters are set as if the main file
% or a child file starting with |\childdocof| was compiled.
% Then compilation is handed over to the main file:
%    \begin{macrocode}
\newcommand{\childdocforward}[2][]
{
  \begingroup
    \if?#1?
      \def\childdoctmp
      {
        \def\childdocname{#2}
        \def\childdocjob{#2}
        \def\jobname{#2}
        \input{#2}
        \endinput
      }
    \else
      \def\childdoctmp
      {
        \childdocdisable
        \def\childdocname{#2}
        \childdoctrue
        \includeonly{#2}
        \def\childdocjob{#1}
        \def\jobname{#1}
        \input{#1}
        \endinput
      }
    \fi
    \expandafter
  \endgroup
  \childdoctmp
}
%    \end{macrocode}

% \macro{\childdocforwardprefix}
% The command |\childdocforwardprefix| redirects
% compilation to the main or a child file by means of a pattern.
% The prefix |#1| in the current filename is replaced by |#2|
% and the suffix of the current filename is kept
% (it is assumed that the filename does not contain the substring `|~~~|'
% which is used as a delimiter).
% Compilation is handed over to the new file by |\childdocforward|:
%    \begin{macrocode}
\newcommand{\childdocforwardprefix}[3][]
{
  \begingroup
    \def\childdocextract #2##1~~~{\def\childdoctmp{\childdocforward[#1]{#3##1}}}
    \expandafter\childdocextract\childdocname~~~
    \expandafter
  \endgroup
  \childdoctmp
}
%    \end{macrocode}

% \macro{\childdoc}
% The deprecated macro |\childdoc| is a legacy version of |\childdocmain|:
%    \begin{macrocode}
\newcommand{\childdoc}{\childdocmain}
%    \end{macrocode}

% \macro{\childdocredirect}
% The deprecated macro |\childdocredirect| is a legacy version
% of |\childdocforward| and |\childdocforwardprefix|:
%    \begin{macrocode}
\newcommand{\childdocredirect}[2][]
{
  \begingroup
    \if?#1?
      \def\childdoctmp{\childdocforward{#2}}
    \else
      \def\childdoctmp{\childdocforwardprefix{#1}{#2}}
    \fi
    \expandafter
  \endgroup
  \childdoctmp
}
%    \end{macrocode}

%\iffalse
%</package>
%\fi
%
\endinput

\childdocof{cdocsamp}
%    \end{macrocode}

%\iffalse
%</samplechap1|samplechap2>
%\fi
%
%\iffalse
%<*samplechap1>
%\fi
% Some text for chapter 1:
%    \begin{macrocode}
\section{one}
some text in chapter one
%    \end{macrocode}

%\iffalse
%</samplechap1>
%\fi
% Some text for chapter 2:
%\iffalse
%<*samplechap2>
%\fi
%    \begin{macrocode}
\section{two}
more text in chapter two
%    \end{macrocode}

%\iffalse
%</samplechap2>
%\fi
%
% %%%%%%%%%%%%%%%%%%%%%%%%%%%%%%%%%%%%%%
% \paragraph{Part Include Files.}
%
% The include files are called |cdocspt3.tex| and |cdocspt4.tex|.
%
%\iffalse
%<*samplepart3|samplepart4>
%\fi

% Optional override for |\version| flag:
%    \begin{macrocode}
%%\providecommand{\version}{final}
%    \end{macrocode}

% Include the main document:
%    \begin{macrocode}
% \iffalse
%
% childdoc.dtx Copyright (C) 2017-2018 Niklas Beisert
%
% This work may be distributed and/or modified under the
% conditions of the LaTeX Project Public License, either version 1.3
% of this license or (at your option) any later version.
% The latest version of this license is in
%   http://www.latex-project.org/lppl.txt
% and version 1.3 or later is part of all distributions of LaTeX
% version 2005/12/01 or later.
%
% This work has the LPPL maintenance status `maintained'.
%
% The Current Maintainer of this work is Niklas Beisert.
%
% This work consists of the files childdoc.dtx and childdoc.ins
% and the derived files childdoc.def and cdocsamp.tex with
% cdocsch1.tex, cdocsch2.tex, cdocsdrf.tex, cdocsfn1.tex, cdocsfn2.tex.
%
%<package>\ifdefined\childdocmain\endinput\fi
%<package>\ProvidesFile{childdoc.def}[2018/12/30 v2.0 child document driver]
%<samplemain>\ProvidesFile{cdocsamp.tex}[2018/12/30 v2.0 sample for childdoc]
%<*driver>
%\ProvidesFile{childdoc.drv}[2018/12/30 v2.0 childdoc reference manual file]
\PassOptionsToClass{10pt,a4paper}{article}
\documentclass{ltxdoc}

\usepackage[margin=35mm]{geometry}
\usepackage{hyperref}
\usepackage{hyperxmp}
\usepackage[usenames]{color}

\hypersetup{colorlinks=true}
\hypersetup{pdfstartview=FitH}
\hypersetup{pdfpagemode=UseNone}
\hypersetup{pdfsource={}}
\hypersetup{pdflang={en-UK}}
\hypersetup{pdfcopyright={Copyright 2017-2018 Niklas Beisert.
  This work may be distributed and/or modified under the
  conditions of the LaTeX Project Public License, either version 1.3
  of this license or (at your option) any later version.}}
\hypersetup{pdflicenseurl={http://www.latex-project.org/lppl.txt}}
\hypersetup{pdfcontactaddress={ETH Zurich, ITP, HIT K,
  Wolfgang-Pauli-Strasse 27}}
\hypersetup{pdfcontactpostcode={8093}}
\hypersetup{pdfcontactcity={Zurich}}
\hypersetup{pdfcontactcountry={Switzerland}}
\hypersetup{pdfcontactemail={nbeisert@itp.phys.ethz.ch}}
\hypersetup{pdfcontacturl={http://people.phys.ethz.ch/\xmptilde nbeisert/}}

\newcommand{\secref}[1]{\hyperref[#1]{section \ref*{#1}}}

\parskip1ex
\parindent0pt
\let\olditemize\itemize
\def\itemize{\olditemize\parskip0pt}

\begin{document}

\title{The \textsf{childdoc} Package}
\hypersetup{pdftitle={The childdoc Package}}
\author{Niklas Beisert\\[2ex]
  Institut f\"ur Theoretische Physik\\
  Eidgen\"ossische Technische Hochschule Z\"urich\\
  Wolfgang-Pauli-Strasse 27, 8093 Z\"urich, Switzerland\\[1ex]
  \href{mailto:nbeisert@itp.phys.ethz.ch}
  {\texttt{nbeisert@itp.phys.ethz.ch}}}
\hypersetup{pdfauthor={Niklas Beisert}}
\hypersetup{pdfsubject={Manual for the LaTeX2e Package childdoc}}
\date{30 December 2018, \textsf{v2.0}}
\maketitle

\begin{abstract}\noindent
\textsf{childdoc} is a \LaTeXe{} package
that enables the direct compilation
of document sections included by |\include|
to individual files.
\end{abstract}

\begingroup
\parskip0ex
\tableofcontents
\endgroup

%%%%%%%%%%%%%%%%%%%%%%%%%%%%%%%%%%%%%%%%%%%%%%%%%%%%%%%%%%%%%%%%%%%%%%%%%%%%%%%%
%%%%%%%%%%%%%%%%%%%%%%%%%%%%%%%%%%%%%%%%%%%%%%%%%%%%%%%%%%%%%%%%%%%%%%%%%%%%%%%%
\section{Introduction}

\LaTeX{} provides a mechanism to structure a large document (such as a book)
into a main file and several child files (containing the chapters)
using the |\include| command.
This mechanism is beneficial for documents
which span hundreds of pages in order to
make the source file(s) more manageable.
Moreover, compilation can be restricted to
selected child files by means of the |\includeonly| command.
The latter feature can be used to reduce the compilation time while editing
(this was significantly more useful in the earlier days of \LaTeX{})
or to generate a smaller document which is easier to navigate.
Another application of |\includeonly| is to generate
documents consisting of selected parts of the complete document.

However, there are a few drawbacks of the plain |\include| mechanism:
\begin{itemize}
\item
The child files cannot be compiled on their own,
they can only be compiled via the main file.
A naive editing environment
(such as a text editor with an option
to have the current file processed by \LaTeX)
may require one to switch to the main file before compiling;
attempting to compile the child file produces errors.
\item
The main file must be modified (each time)
to adjust the |\includeonly| command
to the present needs. This easily leaves the main file in a messy state.
\item
The generated document will always carry the filename
of the main document. This is inconvenient if
several child files are to be compiled and
to be kept for distribution.
\end{itemize}

The present package provides a simple interface
to make child files individually compilable by \LaTeX{}.
Compiling a child file then has the same effect as compiling
the main file with an |\includeonly| command
to select the appropriate child.
Moreover the generated document will carry the name of the child
rather than the main file.
This resolves all three above issues.

This feature is meant to make the editing of books,
thesis documents and lecture notes somewhat more convenient.
However, the package can also be used efficiently for
composing a series of documents (such as exercise sheets)
which are typically distributed individually.
It then assists the author in generating the individual documents
(potentially in different versions)
as well as a document containing the collected series.
Another application is in developing style files
or other kinds of included material
where compilation of the style file could redirect
to a sample or test file.

%%%%%%%%%%%%%%%%%%%%%%%%%%%%%%%%%%%%%%%%%%%%%%%%%%%%%%%%%%%%%%%%%%%%%%%%%%%%%%%%
%%%%%%%%%%%%%%%%%%%%%%%%%%%%%%%%%%%%%%%%%%%%%%%%%%%%%%%%%%%%%%%%%%%%%%%%%%%%%%%%
\section{Usage}

First of all, the package \textsf{childdoc} is \emph{not} a standard
\LaTeXe{} |.sty| style file! Therefore it needs to be invoked in
a non-standard way.

%%%%%%%%%%%%%%%%%%%%%%%%%%%%%%%%%%%%%%%%%%%%%%%%%%%%%%%%%%%%%%%%%%%%%%%%%%%%%%%%
\subsection{Included Files}
\label{sec:include}

%%%%%%%%%%%%%%%%%%%%%%%%%%%%%%%%%%%%%%%%
\DescribeMacro{\childdocmain}
To use the package, add the commands
\begin{center}
\begin{tabular}{l}
|\input{childdoc.def}|\\
|\childdocmain{}|\\
\end{tabular}
\end{center}
at the very top of the main \LaTeX{} file,
in particular \emph{before} the |\documentclass| statement!
The argument of |\childdocmain| should be left empty
(but it must be present).

%%%%%%%%%%%%%%%%%%%%%%%%%%%%%%%%%%%%%%%%
\DescribeMacro{\childdocof}
Furthermore, add the commands
\begin{center}
\begin{tabular}{l}
|\input{childdoc.def}|\\
|\childdocof{|\textit{main}|}|\\
\end{tabular}
\end{center}
at the top of every child file \textit{child}
which is included by |\include{|\textit{child}|}|
from within the main file
(or at least for those files to be compiled individually).
The argument \textit{main} must be the filename of the main file.

There are a couple of
considerations in setting up the main and child documents:

%%%%%%%%%%%%%%%%%%%%%%%%%%%%%%%%%%%%%%%%
\paragraph{Restrictions.}

Please note the following restrictions:
\begin{itemize}
\item
|\childdocmain| must be called with one argument \textit{main}
to ensure compatibility with earlier version of the package.
It must either be empty (|\childdocmain{}|)
or precisely match the filename of the main file in which it is specified.
See \secref{sec:detection} for further information.
\item
The filename \textit{main} must be specified without the |.tex| extension.
\item
The filename \textit{main} is case sensitive
(even in case-insensitive file systems)
due to internal string comparison.
\item
The argument \textit{main} should be fully expanded, it cannot be a macro.
\item
Subdirectories and special characters should be avoided in filenames.
\item
The command |\childdocmain{|\textit{main}|}| must be followed by a whitespace.
It should not be followed immediately by another command
or by a comment mark `|%|'.
This is because the \TeX{} parser reads the token immediately following
the argument of |\childdocmain| and puts it
at the beginning of every child section;
however, a white\-space is ignored.
\end{itemize}

%%%%%%%%%%%%%%%%%%%%%%%%%%%%%%%%%%%%%%%%
\paragraph{Content of Main File.}

It is advisable to place all content in the child files included by |\include|.
Any output contained in the main file will appear in all child documents
unless suppressed manually;
it cannot be suppressed automatically by the |\includeonly| directive
and thus should normally be avoided.
A method to include some content in the main file
by means of conditional processing is described in \secref{sec:conditional}.

%%%%%%%%%%%%%%%%%%%%%%%%%%%%%%%%%%%%%%%%
\paragraph{Page Numbering.}

When only a part of the document is compiled,
the appropriate numbering of pages
(as well as other status parameters)
is determined from the |.aux| files.
The latter contain information from previous passes.
However this information needs to propagate through
all intermediate child documents.
Therefore the page numbering in child documents may well
be inconsistent until the complete document is compiled at least once.

A useful (if unconventional) way to always ensure a consistent
page numbering is to restart the numbering in each child document
and denote the pages by `\textit{child}|.|\textit{page}'
where \textit{child} represents the chapter/section number of the child file.
This can be achieved by the command
|\numberwithin{page}{|\textit{child}|}|
of the \textsf{amsmath} package
where \textit{child} can be |chapter| or |section|
depending on the chosen structuring.
Alternatively, one can modify the macro |\thepage| appropriately
and reset the counter |page| at the start of each child file.

%%%%%%%%%%%%%%%%%%%%%%%%%%%%%%%%%%%%%%%%%%%%%%%%%%%%%%%%%%%%%%%%%%%%%%%%%%%%%%%%
\subsection{Conditional Processing}
\label{sec:conditional}

The package provides a mechanism to compile different versions
of a document. To customise the versions further some conditional processing
can come in handy to distinguish which version is being compiled.
The package provides two macros to describe the compilation context:

%%%%%%%%%%%%%%%%%%%%%%%%%%%%%%%%%%%%%%%%
\DescribeMacro{\ifchilddoc}
The conditional |\ifchilddoc| distinguishes between the compilation of
child documents and the main document:
%
\begin{center}
|\ifchilddoc |\textit{child-code}| |[|\||else |\textit{main-code}]| \||fi|
\end{center}

%%%%%%%%%%%%%%%%%%%%%%%%%%%%%%%%%%%%%%%%
\DescribeMacro{\childdocname}
\DescribeMacro{\childdocjob}
The macro |\childdocname| contains the filename (without extension)
of the main or child file being processed.
Note that |\childdocjob| will always contain the name of the main file.

%%%%%%%%%%%%%%%%%%%%%%%%%%%%%%%%%%%%%%%%
\paragraph{Title Page.}

Conditional processing can be used to include a title or banner page
in the main document when proper precautions are taken.
Importantly, the code in the main file should ensure that the page counter
(as well as other status parameters which are stored in the |.aux| files)
takes the same value after the conditional processing.
Otherwise the page numbers may take divergent values
depending on which part is compiled.

For example, a title page could be declared by:
%
\begin{center}
\begin{tabular}{l}
|\ifchilddoc\||else|\\
|\addtocounter{page}{-1}|\\
\textit{code for title page}\\
|\newpage|\\
|\||fi|
\end{tabular}
\end{center}
%
A banner page for the child documents can be generated by:
%
\begin{center}
\begin{tabular}{l}
|\ifchilddoc|\\
|\addtocounter{page}{-1}|\\
\textit{code for banner page}\\
|\newpage|\\
|\||fi|
\end{tabular}
\end{center}
%
Here one could write a message such as:
\begin{center}
|This is the part \childdocname{} of \childdocjob{}.|
\end{center}

%%%%%%%%%%%%%%%%%%%%%%%%%%%%%%%%%%%%%%%%%%%%%%%%%%%%%%%%%%%%%%%%%%%%%%%%%%%%%%%%
\subsection{Flags}
\label{sec:flags}

The package makes it easy to generate different versions
of the main or child documents.
To this end compilation flags can be defined
and assigned different default values.
They will be particularly useful in conjunction
with the forwarding mechanism described in \secref{sec:forward}.

For example, it may be useful to have a flag |\version|
which can be set to |draft| or |final|.
The document source will contain some conditional code
depending on the value of |\version|.
Suppose further, the flag should default to |final| for the main file
and to |draft| for child files
which is a natural assignment for editing the document.
This is achieved by placing the following code
in the preamble of the main document
(below the |\childdocmain| directive):
%
\begin{center}
\begin{tabular}{l}
|\ifchilddoc|\\
|\providecommand{\version}{draft}|\\
|\||else|\\
|\providecommand{\version}{final}|\\
|\||fi|
\end{tabular}
\end{center}
%
The definition by |\providecommand| makes sure
that previous definitions are not overwritten.
Further statements |\providecommand{\version}{...}|
can thus be added before the above code to override it.

For the main file, one might add a line
(between |\childdocmain| and the above block)
%
\begin{center}
|%\ifchilddoc\||else\providecommand{\version}{draft}\||fi|
\end{center}
%
which can be uncommented to produce a draft version.
Likewise one can add a line to the very top of a child file
(above the |\childdocof{|\textit{main}|}| directive)
%
\begin{center}
|%\providecommand{\version}{final}|
\end{center}
%
which can be uncommented to produce the final version of this child document.

%%%%%%%%%%%%%%%%%%%%%%%%%%%%%%%%%%%%%%%%%%%%%%%%%%%%%%%%%%%%%%%%%%%%%%%%%%%%%%%%
\subsection{Forwarding}
\label{sec:forward}

Different versions of the main or child documents
using compilation flags as described in \secref{sec:flags}
can be (permanently) stored in different files
for convenient compilation, viewing and distribution.
To this end, the package defines a command
to pass on compilation to a different file:

%%%%%%%%%%%%%%%%%%%%%%%%%%%%%%%%%%%%%%%%
\DescribeMacro{\childdocforward}
The command |\childdocforward| redirects processing to
another source file:
%
\begin{center}
\begin{tabular}{l}
|\input{childdoc.def}|\\
|\childdocforward[|\textit{main}|]{|\textit{dest}|}|\\
\end{tabular}
\end{center}
%
The argument \textit{dest} is the destination file
(without extension).
It should be the main file or one of the child files.
Note that further \textsf{childdoc} directives
such as |\childdocof| and |\childdocforward|
in the indicated file will be processed in this form.
The optional argument \textit{main}
passes on directly to the main file \textit{main}
while pretending to compile the child \textit{dest}.
This form behaves as if \textit{dest}
issues |\childdocof{|\textit{main}|}| right away,
and no further \textsf{childdoc} directives will be processed.

%%%%%%%%%%%%%%%%%%%%%%%%%%%%%%%%%%%%%%%%
\DescribeMacro{\...prefix}
In the alternative form |\childdocforwardprefix|,
%
\begin{center}
\begin{tabular}{l}
|\input{childdoc.def}|\\
|\childdocforwardprefix[|\textit{main}|]{|\textit{prefix}|}{|\textit{dest}|}|
\end{tabular}
\end{center}
%
the destination file is determined by a pattern
depending on the current file:
To make this work, the current file must be called
`{\textit{prefix}\hspace{0.2em}\textit{suffix}}'
with \textit{prefix} matching precisely the argument.
Processing is then passed on to the file
`{\textit{dest}\hspace{0.2em}\textit{suffix}}'.
Surely, the same effect is achieved by
directly specifying the
argument `{\textit{dest}\hspace{0.2em}\textit{suffix}}'
in the first form.
However, that requires to set up a different file
for each child. With the alternative form of the command
all these files can have exactly the same content
which simplifies setting them up and maintaining them.

For example, the following file |draft.tex|
with a compilation flag |\version| as described in \secref{sec:flags}
compiles the main document as a draft:
%
\begin{center}
\begin{tabular}{l}
|\def\version{draft}|\\
|\input{childdoc.def}|\\
|\childdocforward{|\textit{main}|}|
\end{tabular}
\end{center}
%
Likewise, the following files |final|\textit{nn}|.tex|
compile the final version of the child document
|child|\textit{nn}|.tex|:
%
\begin{center}
\begin{tabular}{l}
|\def\version{final}|\\
|\input{childdoc.def}|\\
|\childdocforwardprefix{final}{child}|
\end{tabular}
\end{center}
%

Note that when several versions of a main file and/or of each child file
are to be generated, it may be convenient to set up a |Makefile| or
shell script to automatise the process.

%%%%%%%%%%%%%%%%%%%%%%%%%%%%%%%%%%%%%%%%%%%%%%%%%%%%%%%%%%%%%%%%%%%%%%%%%%%%%%%%
\subsection{Command Line Processing}
\label{sec:commandline}

The effect of redirection files can also be achieved by invoking
the \LaTeX{} compiler with a more elaborate command line.
Most conveniently this should be done as part
of a shell script or a |Makefile|.

When using \textsf{childdoc} in the main file, the following
command lines effectively perform a redirection
(note that depending on the shell being used,
backslashes may have to be doubled: `|\|' $\to$ `|\\|'):
%
\begin{center}
|... -jobname "|\textit{target}|" |\\|"|[\textit{flags}]%
|\input{childdoc.def}\childdocforward[|\textit{main}|]{|\textit{dest}|}"|
\end{center}
%
Here \textit{target} is the name of the output file,
\textit{main} is the name of the main file
and \textit{dest} is the name of the main or child file to be processed
(all filenames without extensions).
The optional argument \textit{main} can be omitted
if \textit{main} matches \textit{dest}.
Optionally, compilation \textit{flags} can be defined via |\def| commands.
This command line makes the \TeX{} engine believe
it is compiling the file \textit{target}
whose content is specified as the latter parameter.
The provided code then forwards the processing to
\textit{main} or \textit{dest} as described in \secref{sec:forward}.

%%%%%%%%%%%%%%%%%%%%%%%%%%%%%%%%%%%%%%%%%%%%%%%%%%%%%%%%%%%%%%%%%%%%%%%%%%%%%%%%
\subsection{Include by Input}
\label{sec:input}

Including child documents by |\include| has some restrictions by design.
Most notably, the content of a child document always occupies
its own set of pages; pages cannot be shared between child documents.
Usually, this behaviour makes perfect sense
because each child document contain an essential part of the document.
However, in some situations it may be desirable to compose
a document from a collection of parts
without having mandatory page breaks between then.
For this case, the package
provides a mechanism to include parts
by |\input| which can also be processed individually.
However, by construction this mechanism
requires manual handling of the content to be output.

%%%%%%%%%%%%%%%%%%%%%%%%%%%%%%%%%%%%%%%%
\DescribeMacro{\ifchilddocmanual}
The main file should be prepared as usual, see \secref{sec:include}.
However, the document body must make a distinction
between processing of an individual part and of the main document, e.g.:
%
\begin{center}
\begin{tabular}{l}
|\ifchilddocmanual|\\
|\input{\childdocname}|\\
|\||else|\\
\textit{document body with }|\input{|\textit{part}|}|\\
|\||fi|
\end{tabular}
\end{center}
%
The conditional |\ifchilddocmanual| is true whenever
a part to be included by |\input| is being compiled,
and the name of the part is stored in |\childdocname|.

%%%%%%%%%%%%%%%%%%%%%%%%%%%%%%%%%%%%%%%%
\DescribeMacro{\childdocby}
Each part to be included by |\input| should start with:
%
\begin{center}
\begin{tabular}{l}
|\input{childdoc.def}|\\
|\childdocby{|\textit{main}|}|\\
\end{tabular}
\end{center}
%
The directive |\childdocby| is similar to |\childdocof|
described in \secref{sec:include},
but the subsequent selection of content must be done manually.
To that end, both |\ifchilddoc| and |\ifchilddocmanual|
will be true upon processing of a part,
and the name of the part is stored in |\childdocname|.
Note that |\jobname| will be set to the filename of the current part
so that each part receives an individual |.aux| file
that does not interfere with the |.aux| file(s) of the main document.
This behaviour can be altered by the alternative form
|\childdocby[*]{|\textit{main}|}| (with a non-empty optional argument)
which uses the |.aux| file of the main document
by setting |\jobname| to \textit{main}.

%%%%%%%%%%%%%%%%%%%%%%%%%%%%%%%%%%%%%%%%%%%%%%%%%%%%%%%%%%%%%%%%%%%%%%%%%%%%%%%%
\subsection{Driver Development}
\label{sec:driver}

The \textsf{childdoc} mechanism can also be use for the development
of definition files such as \LaTeX{} styles or classes.
This case differs from the above setup with multiple parts
included by |\include| in that no |\includeonly| should be invoked.
This can be achieved by starting the include file
(before |\ProvidesPackage|) with:
%
\begin{center}
\begin{tabular}{l}
|\input{childdoc.def}|\\
|\childdocforward{|\textit{main}|}|\\
\end{tabular}
\end{center}
%
or alternatively with:
%
\begin{center}
\begin{tabular}{l}
|\input{childdoc.def}|\\
|\childdocby{|\textit{main}|}|\\
\end{tabular}
\end{center}
%
Both forms have slightly different effects as described above.
The main file is prepared as usual, see \secref{sec:include}.

%%%%%%%%%%%%%%%%%%%%%%%%%%%%%%%%%%%%%%%%%%%%%%%%%%%%%%%%%%%%%%%%%%%%%%%%%%%%%%%%
\subsection{Legacy Detection}
\label{sec:detection}

The directive |\childdocmain| in the main file can detect
whether the complete document or merely a child is to be compiled
even without using the directive |\childdocof|.
This method is deprecated because it is less robust
and there is no compelling reason to use it;
it is merely provided for backward compatibility
and it may be removed in future versions.

If the detection mechanism is to be used,
it is mandatory to correctly specify
the filename of the main file as the argument of |\childdocmain|:
%
\begin{center}
\begin{tabular}{l}
|\input{childdoc.def}|\\
|\childdocmain{|\textit{main}|}|\\
\end{tabular}
\end{center}
%
If |\jobname| does not match the argument \textit{main} of |\childdocmain|,
it is assumed that |\jobname| points to the child file to be compiled.
When using |\childdocmain| with the main file specified as argument,
it suffices to start a child file
with just |\input{|\textit{main}|}|
without loading of the package and using |\childdocof|.
If instead all processing is done
with the appropriate \textsf{childdoc} directives,
the argument of \textit{main} of |\childdocmain| can be empty.

An alternative version of the command line processing described
in \secref{sec:commandline} using the detection mechanism reads:
%
\begin{center}
|... -jobname "|\textit{target}|" "|[\textit{flags}]%
[|\def\jobname{|\textit{dest}|}|]|\input{|\textit{main}|}"|
\end{center}

%%%%%%%%%%%%%%%%%%%%%%%%%%%%%%%%%%%%%%%%%%%%%%%%%%%%%%%%%%%%%%%%%%%%%%%%%%%%%%%%
\subsection{Manual Code}
\label{sec:manual}

In case one cannot be certain whether the definitions file |childdoc.def|
is installed on the target \TeX{} distribution
and one prefers not to ship it,
it is conceivable to paste a few relevant commands into the sources.

To that end, drop all statements |\input{childdoc.def}|
and perform the replacements as outlined below.
Instead of |\childdocmain{|\textit{main}|}| add the following code
to the top of the main file:
%
\begin{center}
\begin{tabular}{l}
|\||ifdefined\childdocname\endinput\||fi\newif\ifchilddoc|\\
|\edef\childdocname{\scantokens\expandafter{\jobname\noexpand}}|\\
|\def\childdocmain{|\textit{main}|}\||ifx\childdocmain\childdocname\||else|\\
|\childdoctrue\includeonly{\childdocname}\let\jobname\childdocmain\||fi|\\
\end{tabular}
\end{center}
%
Instead of |\childdocof{|\textit{main}|}| just include the main file
at the top of each child file:
%
\begin{center}
|\input{|\textit{main}|}|
\end{center}
%
A simple redirection |\childdocforward{|\textit{dest}|}| is achieved by:
%
\begin{center}
|\def\jobname{|\textit{dest}|}\input{\jobname}|
\end{center}
%
The redirection with prefix
|\childdocforwardprefix[|\textit{prefix}|]{|\textit{dest}|}|
is accomplished by:
%
\begin{center}
\begin{tabular}{l}
|{\edef\jobname{\scantokens\expandafter{\jobname\noexpand}}|\\
|\def\redirectjob |\textit{prefix}|#1~~~{\gdef\jobname{|\textit{dest}|#1}}|\\
|\expandafter\redirectjob\jobname~~~}\input{\jobname}|
\end{tabular}
\end{center}

In an alternative approach,
child documents can be compiled by a specific command line
without additional code or specific definitions:
%
\begin{center}
|... -jobname "|\textit{target}|" "|[\textit{flags}]%
|\includeonly{|\textit{dest}|}\input{|\textit{main}|}"|
\end{center}
%

%%%%%%%%%%%%%%%%%%%%%%%%%%%%%%%%%%%%%%%%%%%%%%%%%%%%%%%%%%%%%%%%%%%%%%%%%%%%%%%%
%%%%%%%%%%%%%%%%%%%%%%%%%%%%%%%%%%%%%%%%%%%%%%%%%%%%%%%%%%%%%%%%%%%%%%%%%%%%%%%%
\section{Information}

%%%%%%%%%%%%%%%%%%%%%%%%%%%%%%%%%%%%%%%%%%%%%%%%%%%%%%%%%%%%%%%%%%%%%%%%%%%%%%%%
\subsection{Copyright}

Copyright \copyright{} 2017--2018 Niklas Beisert

This work may be distributed and/or modified under the
conditions of the \LaTeX{} Project Public License, either version 1.3
of this license or (at your option) any later version.
The latest version of this license is in
  \url{http://www.latex-project.org/lppl.txt}
and version 1.3 or later is part of all distributions of \LaTeX{}
version 2005/12/01 or later.

This work has the LPPL maintenance status `maintained'.

The Current Maintainer of this work is Niklas Beisert.

This work consists of the files |README.txt|, |childdoc.ins| and |childdoc.dtx|
as well as the derived files |childdoc.def|, |cdocsamp.tex|
with |cdocsch1.tex|, |cdocsch2.tex|, |cdocspt3.tex|, |cdocspt4.tex|,
|cdocsdrf.tex|, |cdocsfn1.tex|, |cdocsfn2.tex|
as well as |childdoc.pdf|.

%%%%%%%%%%%%%%%%%%%%%%%%%%%%%%%%%%%%%%%%%%%%%%%%%%%%%%%%%%%%%%%%%%%%%%%%%%%%%%%%
\subsection{Files and Installation}

The package consists of the files:
%
\begin{center}
\begin{tabular}{ll}
    |README.txt|   & readme file \\
    |childdoc.ins| & installation file \\
    |childdoc.dtx| & source file \\
    |childdoc.def| & definition file \\
    |cdocsamp.tex| & sample main file \\
    |cdocsch1.tex| & sample include file \\
    |cdocsch2.tex| & sample include file \\
    |cdocspt3.tex| & sample part file \\
    |cdocspt4.tex| & sample part file \\
    |cdocsdrf.tex| & sample redirection file \\
    |cdocsfn1.tex| & sample redirection file \\
    |cdocsfn2.tex| & sample redirection file \\
    |childdoc.pdf| & manual
\end{tabular}
\end{center}
%
The distribution consists of the files
|README.txt|, |childdoc.ins| and |childdoc.dtx|.
%
\begin{itemize}
\item
Run (pdf)\LaTeX{} on |childdoc.dtx|
to compile the manual |childdoc.pdf| (this file).
\item
Run \LaTeX{} on |childdoc.ins| to create the definitions file |childdoc.def|
and the sample |cdocsamp.tex| with include files
|cdocsch1.tex|, |cdocsch2.tex|, |cdocspt3.tex|, |cdocspt4.tex|,
|cdocsdrf.tex|, |cdocsfn1.tex|, |cdocsfn2.tex|.
Then copy the file |childdoc.def| to an appropriate directory of your \LaTeX{}
distribution, e.g.\ \textit{texmf-root}|/tex/latex/childdoc|.
\end{itemize}

%%%%%%%%%%%%%%%%%%%%%%%%%%%%%%%%%%%%%%%%%%%%%%%%%%%%%%%%%%%%%%%%%%%%%%%%%%%%%%%%
\subsection{Related CTAN Packages}

There are several other packages which offer a similar functionality:
%
\begin{itemize}
\item
The packages
\href{http://ctan.org/pkg/docmute}{\textsf{docmute}},
\href{http://ctan.org/pkg/includex}{\textsf{includex}} and
\href{http://ctan.org/pkg/standalone}{\textsf{standalone}}
provide commands to include only the document body of
a child file thus allowing both files to be compiled individually.
\item
The packages \href{http://ctan.org/pkg/subdocs}{\textsf{subdocs}}
and \href{http://ctan.org/pkg/subfiles}{\textsf{subfiles}}
provide structures in which the main and child documents can be
encapsulated and allowing them to be compiled individually.
The inclusion mechanism is different from the conventional |\include|.
\item
The package \href{http://ctan.org/pkg/combine}{\textsf{combine}}
is an elaborate solution to combine several documents into one.
\end{itemize}
%
See also the CTAN topic \href{http://ctan.org/topic/subdocs}{\textsf{subdocs}}
for further related packages.
The present package differs from the above solutions in that
a document structure constructed with the conventional |\include| mechanism
just needs two extra commands at the top of every file
such that all constituent files can be compiled individually.

%%%%%%%%%%%%%%%%%%%%%%%%%%%%%%%%%%%%%%%%%%%%%%%%%%%%%%%%%%%%%%%%%%%%%%%%%%%%%%%%
%\subsection{Feature Suggestions}
%
%The following is a list of features which may be useful for future
%versions of this package:
%%
%\begin{itemize}
%\item
%\ldots
%\end{itemize}

%%%%%%%%%%%%%%%%%%%%%%%%%%%%%%%%%%%%%%%%%%%%%%%%%%%%%%%%%%%%%%%%%%%%%%%%%%%%%%%%
\subsection{Revision History}

%%%%%%%%%%%%%%%%%%%%%%%%%%%%%%%%%%%%%%%%
\paragraph{v2.0:} 2018/12/30

\begin{itemize}
\item
immediate forward processing
\item
added |\childdocby| mechanism
\item
manual restructured
\end{itemize}

%%%%%%%%%%%%%%%%%%%%%%%%%%%%%%%%%%%%%%%%
\paragraph{v1.6:} 2018/01/17

\begin{itemize}
\item
application for development of include files
\item
corrections to manual
\end{itemize}

%%%%%%%%%%%%%%%%%%%%%%%%%%%%%%%%%%%%%%%%
\paragraph{v1.5:} 2017/05/21

\begin{itemize}
\item
more complete structuring introduced
\item
|\childdocof| introduced
\item
|\childdoc| renamed to |\childdocmain|
\item
|\childredirect| renamed to |\childdocforward| and |\childdocforwardprefix|
and functionality expanded
\end{itemize}

%%%%%%%%%%%%%%%%%%%%%%%%%%%%%%%%%%%%%%%%
\paragraph{v1.0:} 2017/04/27

\begin{itemize}
\item
manual and install package
\item
first version published on CTAN
\end{itemize}

%%%%%%%%%%%%%%%%%%%%%%%%%%%%%%%%%%%%%%%%
\paragraph{v0.6:} 2017/04/26

\begin{itemize}
\item
redirection mechanism added
\end{itemize}

%%%%%%%%%%%%%%%%%%%%%%%%%%%%%%%%%%%%%%%%
\paragraph{v0.5:} 2017/04/26

\begin{itemize}
\item
functionality in definition file
\end{itemize}


%%%%%%%%%%%%%%%%%%%%%%%%%%%%%%%%%%%%%%%%%%%%%%%%%%%%%%%%%%%%%%%%%%%%%%%%%%%%%%%%
%%%%%%%%%%%%%%%%%%%%%%%%%%%%%%%%%%%%%%%%%%%%%%%%%%%%%%%%%%%%%%%%%%%%%%%%%%%%%%%%
%%%%%%%%%%%%%%%%%%%%%%%%%%%%%%%%%%%%%%%%%%%%%%%%%%%%%%%%%%%%%%%%%%%%%%%%%%%%%%%%
\appendix

\settowidth\MacroIndent{\rmfamily\scriptsize 000\ }

 \DocInput{childdoc.dtx}

\end{document}
%</driver>
% \fi
%
% %%%%%%%%%%%%%%%%%%%%%%%%%%%%%%%%%%%%%%%%%%%%%%%%%%%%%%%%%%%%%%%%%%%%%%%%%%%%%%
% %%%%%%%%%%%%%%%%%%%%%%%%%%%%%%%%%%%%%%%%%%%%%%%%%%%%%%%%%%%%%%%%%%%%%%%%%%%%%%
% \section{Sample}
%\iffalse
%<*samplemain>
%\fi
%
% The following presents a sample document
% with two chapters, two parts, a title page,
% a compile flag as well as three forwarding files to set the flag.
% It consists of eight |.tex| files:
% \begin{center}
% \begin{tabular}{ll}
% |cdocsamp.tex|&main file\\
% |cdocsch1.tex|&include file for chapter 1\\
% |cdocsch2.tex|&include file for chapter 2\\
% |cdocspt3.tex|&include file for part 3\\
% |cdocspt4.tex|&include file for part 4\\
% |cdocsdrf.tex|&forwarding file for main file in draft mode\\
% |cdocsfi1.tex|&forwarding file for final version of chapter 1\\
% |cdocsfi2.tex|&forwarding file for final version of chapter 2\\
% \end{tabular}
% \end{center}
% Each of the eight files can be compiled directly by the \LaTeX{} compiler.
%
% %%%%%%%%%%%%%%%%%%%%%%%%%%%%%%%%%%%%%%
% \paragraph{Main File.}
%
% The main file is called |cdocsamp.tex|.
%
% Load the \textsf{childdoc} definitions and
% declare the filename for the main document:
%    \begin{macrocode}
\input{childdoc.def}
\childdocmain{}
%    \end{macrocode}

% Optional override for |\version| flag:
%    \begin{macrocode}
%%\ifchilddoc\else\providecommand{\version}{draft}\fi
%    \end{macrocode}

% Define the default values for the |\version| flag
% (|final| for the main file and |draft| for childs):
%    \begin{macrocode}
\ifchilddoc
\providecommand{\version}{draft}
\else
\providecommand{\version}{final}
\fi
%    \end{macrocode}

% Load the standard document class:
%    \begin{macrocode}
\documentclass[12pt]{article}
%    \end{macrocode}

% Start the document body:
%    \begin{macrocode}
\begin{document}
%    \end{macrocode}

% Declare a title page.
% Print title, part of document being processed and version flag:
%    \begin{macrocode}
\addtocounter{page}{-1}
\begin{center}
{\LARGE\bfseries{}childdoc example\par}
\vspace{1cm}
\ifchilddoc
\ifchilddocmanual part\else chapter\fi:
`\childdocname' of `\childdocjob'\par
\else
main document: `\childdocjob'\par
\fi
version: \version\par
\end{center}
\newpage
%    \end{macrocode}

% Manually include selected file,
% otherwise process as usual:
%    \begin{macrocode}
\ifchilddocmanual
\section*{part `\childdocname'}
\input{\childdocname}
\else
%    \end{macrocode}

% Include the two chapters:
%    \begin{macrocode}
\include{cdocsch1}
\include{cdocsch2}
%    \end{macrocode}

% Include the two parts unless only chapters should be displayed:
%    \begin{macrocode}
\ifchilddoc\else
\section{part three}
\input{cdocspt3}
\section{part four}
\input{cdocspt4}
\fi
%    \end{macrocode}

% Process as usual until here:
%    \begin{macrocode}
\fi
%    \end{macrocode}

% End of document body:
%    \begin{macrocode}
\end{document}
%    \end{macrocode}
%\iffalse
%</samplemain>
%\fi
%
% %%%%%%%%%%%%%%%%%%%%%%%%%%%%%%%%%%%%%%
% \paragraph{Chapter Include Files.}
%
% The include files are called |cdocsch1.tex| and |cdocsch2.tex|.
%
%\iffalse
%<*samplechap1|samplechap2>
%\fi

% Optional override for |\version| flag:
%    \begin{macrocode}
%%\providecommand{\version}{final}
%    \end{macrocode}

% Include the main document:
%    \begin{macrocode}
\input{childdoc.def}
\childdocof{cdocsamp}
%    \end{macrocode}

%\iffalse
%</samplechap1|samplechap2>
%\fi
%
%\iffalse
%<*samplechap1>
%\fi
% Some text for chapter 1:
%    \begin{macrocode}
\section{one}
some text in chapter one
%    \end{macrocode}

%\iffalse
%</samplechap1>
%\fi
% Some text for chapter 2:
%\iffalse
%<*samplechap2>
%\fi
%    \begin{macrocode}
\section{two}
more text in chapter two
%    \end{macrocode}

%\iffalse
%</samplechap2>
%\fi
%
% %%%%%%%%%%%%%%%%%%%%%%%%%%%%%%%%%%%%%%
% \paragraph{Part Include Files.}
%
% The include files are called |cdocspt3.tex| and |cdocspt4.tex|.
%
%\iffalse
%<*samplepart3|samplepart4>
%\fi

% Optional override for |\version| flag:
%    \begin{macrocode}
%%\providecommand{\version}{final}
%    \end{macrocode}

% Include the main document:
%    \begin{macrocode}
\input{childdoc.def}
\childdocby{cdocsamp}
%    \end{macrocode}

%\iffalse
%</samplepart3|samplepart4>
%\fi
%
%\iffalse
%<*samplepart3>
%\fi
% Some text for part 3:
%    \begin{macrocode}
some text in part three
%    \end{macrocode}

%\iffalse
%</samplepart3>
%\fi
% Some text for part 4:
%\iffalse
%<*samplepart4>
%\fi
%    \begin{macrocode}
more text in part four
%    \end{macrocode}

%\iffalse
%</samplepart4>
%\fi
%
% %%%%%%%%%%%%%%%%%%%%%%%%%%%%%%%%%%%%%%
% \paragraph{Forwarding for a Complete Draft.}
%
% The following forwarding file |cdocsdrf.tex|
% compiles the main document in draft mode:
%\iffalse
%<*sampledraft>
%\fi
%    \begin{macrocode}
\def\version{draft}
\input{childdoc.def}
\childdocforward{cdocsamp}
%    \end{macrocode}

%\iffalse
%</sampledraft>
%\fi
%
% %%%%%%%%%%%%%%%%%%%%%%%%%%%%%%%%%%%%%%
% \paragraph{Forwarding for Final Version of the Chapters.}
%
% The following forwarding files |cdocsfn1.tex| and |cdocsfn2.tex|
% (with identical content)
% compile the final versions of the child documents
% |cdocsch1.tex| and |cdocsch2.tex|, respectively:
%\iffalse
%<*samplefinal>
%\fi
%    \begin{macrocode}
\def\version{final}
\input{childdoc.def}
\childdocforwardprefix[cdocsamp]{cdocsfn}{cdocsch}
%    \end{macrocode}

%\iffalse
%</samplefinal>
%\fi
%
% %%%%%%%%%%%%%%%%%%%%%%%%%%%%%%%%%%%%%%
% \paragraph{Command Line Processing.}
%
% The following three command lines generate the output files
% |cdocscld|, |cdocscl1| and |cdocscl2|
% which should be identical to
% |cdocsdrf|, |cdocsch1| and |cdocsfn2|, respectively:
% \begin{center}
% \begin{tabular}{l}
% |latex -jobname cdocscld \|\\
% |  "\def\version{draft}\input{childdoc.def}\childdocforward{cdocsamp}"|\\
% |latex -jobname cdocscl1 \|\\
% |  "\input{childdoc.def}\childdocforward[cdocsamp]{cdocsch1}"|\\
% |latex -jobname cdocscl2 \|\\
% |  "\def\version{final}\input{childdoc.def}\childdocforward{cdocsch2}"|
% \end{tabular}
% \end{center}
% Note that the trailing backslash on each first line
% merely continues the input to the second line
% (for convenient cut ant paste).
% Furthermore, the command |latex| can be replaced by any
% of its alternative versions such as |pdflatex|.
%
% %%%%%%%%%%%%%%%%%%%%%%%%%%%%%%%%%%%%%%%%%%%%%%%%%%%%%%%%%%%%%%%%%%%%%%%%%%%%%%
% %%%%%%%%%%%%%%%%%%%%%%%%%%%%%%%%%%%%%%%%%%%%%%%%%%%%%%%%%%%%%%%%%%%%%%%%%%%%%%
% \section{Implementation}
%\iffalse
%<*package>
%\fi
%
% This section describes the definitions file |childdoc.def|.

% The definitions cannot be loaded using |\usepackage| or |\RequirePackage|
% which has a mechanism to prevent loading a style file more than once.
% When loading the definitions by means of |\input|
% multiple instances have to be prevented manually:
%\iffalse
%This code needs to be before the `\ProvidesFile' directive
%which is defined at the beginning of this file.
%Therefore it is also placed there and commented out here.
%</package>
%<*discard>
%\fi
%    \begin{macrocode}
\ifdefined\childdocmain\endinput\fi
%    \end{macrocode}
%\iffalse
%</discard>
%<*package>
%\fi
%
% \macro{\ifchilddoc}
% \macro{\ifchilddocmanual}
% The conditional |\ifchilddoc| tells whether a
% child (true) or main (false) document is being compiled.
% The conditional |\ifchilddocmanual| tells whether
% the |\includeonly| mechanism is used (false) or
% the selection of child files must be performed manually (true).
% The definitions initialise to false:
%    \begin{macrocode}
\newif\ifchilddoc
\newif\ifchilddocmanual
%    \end{macrocode}

% \macro{\childdocname}
% \macro{\childdocjob}
% The macro |\childdocname| stores the name of the main document
% to be compiled. The macro |\childdocjob| stores the name of
% the document on which the \LaTeX{} compiler was originally invoked.
% The content of |\jobname| cannot be compared
% to filenames specified in the source due to different catcodes.
% The following code rescans |\jobname|, stores the result
% in |\childdocname| and saves a copy in |\childdocjob|:
%    \begin{macrocode}
\edef\childdocname{\scantokens\expandafter{\jobname\noexpand}}
\let\childdocjob\childdocname
%    \end{macrocode}

% \macro{\childdocdisable}
% The macro |\childdocdisable| prevents the main file
% from being processed more than once.
% At this stage, the main document command |\childdocmain|
% is assumed to be called once again where it should do nothing.
% Any subsequent call to it should prevent
% a secondary processing of the main document
% It overwrites the forwarding commands
% |\childdocof| and |\childdocforward|
% with empty macros to prevent further inclusions of the main document:
%    \begin{macrocode}
\newcommand{\childdocdisable}
{
  \renewcommand{\childdocmain}[1]{\renewcommand{\childdocmain}[1]{\endinput}}
  \renewcommand{\childdocof}[1]{}
  \renewcommand{\childdocby}[2][]{}
  \renewcommand{\childdocforward}[2][]{}
  \renewcommand{\childdocdisable}{}
}
%    \end{macrocode}

% \macro{\childdocmain}
% The macro |\childdocmain| is to be called at the top of the main file
% with nothing or the main filename (without extension) as argument.
% First, it breaks loops.
% If the argument is not empty and does not match |\childdocname|
% (which is set by the first inclusion of |childdoc.def|),
% |\ifchilddoc| is set to true, |\includeonly| is applied to the child file
% and |\jobname| is set to the main file
% (for proper handling of |.aux| files):
%    \begin{macrocode}
\newcommand{\childdocmain}[1]
{
  \childdocdisable\childdocmain{}
  \if?#1?\else
    \begingroup
      \def\childdoctmp{#1}
      \ifx\childdoctmp\childdocname
        \def\childdoctmp{}
      \else
        \def\childdoctmp
        {
          \childdoctrue
          \includeonly{\childdocname}
          \def\childdocjob{#1}
          \def\jobname{#1}
        }
      \fi
      \expandafter
    \endgroup
    \childdoctmp
  \fi
}
%    \end{macrocode}

% \macro{\childdocof}
% The command |\childdocof| redirects
% compilation to the main file |#1|.
%    \begin{macrocode}
\newcommand{\childdocof}[1]
{
  \childdocdisable
  \childdoctrue
  \includeonly{\childdocname}
  \def\jobname{#1}
  \def\childdocjob{#1}
  \input{#1}
}
%    \end{macrocode}

% \macro{\childdocby}
% The command |\childdocby| ....
%    \begin{macrocode}
\newcommand{\childdocby}[2][]
{
  \childdocdisable
  \childdoctrue
  \childdocmanualtrue
  \if?#1?\else
    \def\jobname{#2}
  \fi
  \def\childdocjob{#2}
  \input{#2}
  \endinput
}
%    \end{macrocode}

% \macro{\childdocforward}
% The command |\childdocforward| redirects
% compilation to the main file or
% (if the optional argument is given) a child file.
% Parameters are set as if the main file
% or a child file starting with |\childdocof| was compiled.
% Then compilation is handed over to the main file:
%    \begin{macrocode}
\newcommand{\childdocforward}[2][]
{
  \begingroup
    \if?#1?
      \def\childdoctmp
      {
        \def\childdocname{#2}
        \def\childdocjob{#2}
        \def\jobname{#2}
        \input{#2}
        \endinput
      }
    \else
      \def\childdoctmp
      {
        \childdocdisable
        \def\childdocname{#2}
        \childdoctrue
        \includeonly{#2}
        \def\childdocjob{#1}
        \def\jobname{#1}
        \input{#1}
        \endinput
      }
    \fi
    \expandafter
  \endgroup
  \childdoctmp
}
%    \end{macrocode}

% \macro{\childdocforwardprefix}
% The command |\childdocforwardprefix| redirects
% compilation to the main or a child file by means of a pattern.
% The prefix |#1| in the current filename is replaced by |#2|
% and the suffix of the current filename is kept
% (it is assumed that the filename does not contain the substring `|~~~|'
% which is used as a delimiter).
% Compilation is handed over to the new file by |\childdocforward|:
%    \begin{macrocode}
\newcommand{\childdocforwardprefix}[3][]
{
  \begingroup
    \def\childdocextract #2##1~~~{\def\childdoctmp{\childdocforward[#1]{#3##1}}}
    \expandafter\childdocextract\childdocname~~~
    \expandafter
  \endgroup
  \childdoctmp
}
%    \end{macrocode}

% \macro{\childdoc}
% The deprecated macro |\childdoc| is a legacy version of |\childdocmain|:
%    \begin{macrocode}
\newcommand{\childdoc}{\childdocmain}
%    \end{macrocode}

% \macro{\childdocredirect}
% The deprecated macro |\childdocredirect| is a legacy version
% of |\childdocforward| and |\childdocforwardprefix|:
%    \begin{macrocode}
\newcommand{\childdocredirect}[2][]
{
  \begingroup
    \if?#1?
      \def\childdoctmp{\childdocforward{#2}}
    \else
      \def\childdoctmp{\childdocforwardprefix{#1}{#2}}
    \fi
    \expandafter
  \endgroup
  \childdoctmp
}
%    \end{macrocode}

%\iffalse
%</package>
%\fi
%
\endinput

\childdocby{cdocsamp}
%    \end{macrocode}

%\iffalse
%</samplepart3|samplepart4>
%\fi
%
%\iffalse
%<*samplepart3>
%\fi
% Some text for part 3:
%    \begin{macrocode}
some text in part three
%    \end{macrocode}

%\iffalse
%</samplepart3>
%\fi
% Some text for part 4:
%\iffalse
%<*samplepart4>
%\fi
%    \begin{macrocode}
more text in part four
%    \end{macrocode}

%\iffalse
%</samplepart4>
%\fi
%
% %%%%%%%%%%%%%%%%%%%%%%%%%%%%%%%%%%%%%%
% \paragraph{Forwarding for a Complete Draft.}
%
% The following forwarding file |cdocsdrf.tex|
% compiles the main document in draft mode:
%\iffalse
%<*sampledraft>
%\fi
%    \begin{macrocode}
\def\version{draft}
% \iffalse
%
% childdoc.dtx Copyright (C) 2017-2018 Niklas Beisert
%
% This work may be distributed and/or modified under the
% conditions of the LaTeX Project Public License, either version 1.3
% of this license or (at your option) any later version.
% The latest version of this license is in
%   http://www.latex-project.org/lppl.txt
% and version 1.3 or later is part of all distributions of LaTeX
% version 2005/12/01 or later.
%
% This work has the LPPL maintenance status `maintained'.
%
% The Current Maintainer of this work is Niklas Beisert.
%
% This work consists of the files childdoc.dtx and childdoc.ins
% and the derived files childdoc.def and cdocsamp.tex with
% cdocsch1.tex, cdocsch2.tex, cdocsdrf.tex, cdocsfn1.tex, cdocsfn2.tex.
%
%<package>\ifdefined\childdocmain\endinput\fi
%<package>\ProvidesFile{childdoc.def}[2018/12/30 v2.0 child document driver]
%<samplemain>\ProvidesFile{cdocsamp.tex}[2018/12/30 v2.0 sample for childdoc]
%<*driver>
%\ProvidesFile{childdoc.drv}[2018/12/30 v2.0 childdoc reference manual file]
\PassOptionsToClass{10pt,a4paper}{article}
\documentclass{ltxdoc}

\usepackage[margin=35mm]{geometry}
\usepackage{hyperref}
\usepackage{hyperxmp}
\usepackage[usenames]{color}

\hypersetup{colorlinks=true}
\hypersetup{pdfstartview=FitH}
\hypersetup{pdfpagemode=UseNone}
\hypersetup{pdfsource={}}
\hypersetup{pdflang={en-UK}}
\hypersetup{pdfcopyright={Copyright 2017-2018 Niklas Beisert.
  This work may be distributed and/or modified under the
  conditions of the LaTeX Project Public License, either version 1.3
  of this license or (at your option) any later version.}}
\hypersetup{pdflicenseurl={http://www.latex-project.org/lppl.txt}}
\hypersetup{pdfcontactaddress={ETH Zurich, ITP, HIT K,
  Wolfgang-Pauli-Strasse 27}}
\hypersetup{pdfcontactpostcode={8093}}
\hypersetup{pdfcontactcity={Zurich}}
\hypersetup{pdfcontactcountry={Switzerland}}
\hypersetup{pdfcontactemail={nbeisert@itp.phys.ethz.ch}}
\hypersetup{pdfcontacturl={http://people.phys.ethz.ch/\xmptilde nbeisert/}}

\newcommand{\secref}[1]{\hyperref[#1]{section \ref*{#1}}}

\parskip1ex
\parindent0pt
\let\olditemize\itemize
\def\itemize{\olditemize\parskip0pt}

\begin{document}

\title{The \textsf{childdoc} Package}
\hypersetup{pdftitle={The childdoc Package}}
\author{Niklas Beisert\\[2ex]
  Institut f\"ur Theoretische Physik\\
  Eidgen\"ossische Technische Hochschule Z\"urich\\
  Wolfgang-Pauli-Strasse 27, 8093 Z\"urich, Switzerland\\[1ex]
  \href{mailto:nbeisert@itp.phys.ethz.ch}
  {\texttt{nbeisert@itp.phys.ethz.ch}}}
\hypersetup{pdfauthor={Niklas Beisert}}
\hypersetup{pdfsubject={Manual for the LaTeX2e Package childdoc}}
\date{30 December 2018, \textsf{v2.0}}
\maketitle

\begin{abstract}\noindent
\textsf{childdoc} is a \LaTeXe{} package
that enables the direct compilation
of document sections included by |\include|
to individual files.
\end{abstract}

\begingroup
\parskip0ex
\tableofcontents
\endgroup

%%%%%%%%%%%%%%%%%%%%%%%%%%%%%%%%%%%%%%%%%%%%%%%%%%%%%%%%%%%%%%%%%%%%%%%%%%%%%%%%
%%%%%%%%%%%%%%%%%%%%%%%%%%%%%%%%%%%%%%%%%%%%%%%%%%%%%%%%%%%%%%%%%%%%%%%%%%%%%%%%
\section{Introduction}

\LaTeX{} provides a mechanism to structure a large document (such as a book)
into a main file and several child files (containing the chapters)
using the |\include| command.
This mechanism is beneficial for documents
which span hundreds of pages in order to
make the source file(s) more manageable.
Moreover, compilation can be restricted to
selected child files by means of the |\includeonly| command.
The latter feature can be used to reduce the compilation time while editing
(this was significantly more useful in the earlier days of \LaTeX{})
or to generate a smaller document which is easier to navigate.
Another application of |\includeonly| is to generate
documents consisting of selected parts of the complete document.

However, there are a few drawbacks of the plain |\include| mechanism:
\begin{itemize}
\item
The child files cannot be compiled on their own,
they can only be compiled via the main file.
A naive editing environment
(such as a text editor with an option
to have the current file processed by \LaTeX)
may require one to switch to the main file before compiling;
attempting to compile the child file produces errors.
\item
The main file must be modified (each time)
to adjust the |\includeonly| command
to the present needs. This easily leaves the main file in a messy state.
\item
The generated document will always carry the filename
of the main document. This is inconvenient if
several child files are to be compiled and
to be kept for distribution.
\end{itemize}

The present package provides a simple interface
to make child files individually compilable by \LaTeX{}.
Compiling a child file then has the same effect as compiling
the main file with an |\includeonly| command
to select the appropriate child.
Moreover the generated document will carry the name of the child
rather than the main file.
This resolves all three above issues.

This feature is meant to make the editing of books,
thesis documents and lecture notes somewhat more convenient.
However, the package can also be used efficiently for
composing a series of documents (such as exercise sheets)
which are typically distributed individually.
It then assists the author in generating the individual documents
(potentially in different versions)
as well as a document containing the collected series.
Another application is in developing style files
or other kinds of included material
where compilation of the style file could redirect
to a sample or test file.

%%%%%%%%%%%%%%%%%%%%%%%%%%%%%%%%%%%%%%%%%%%%%%%%%%%%%%%%%%%%%%%%%%%%%%%%%%%%%%%%
%%%%%%%%%%%%%%%%%%%%%%%%%%%%%%%%%%%%%%%%%%%%%%%%%%%%%%%%%%%%%%%%%%%%%%%%%%%%%%%%
\section{Usage}

First of all, the package \textsf{childdoc} is \emph{not} a standard
\LaTeXe{} |.sty| style file! Therefore it needs to be invoked in
a non-standard way.

%%%%%%%%%%%%%%%%%%%%%%%%%%%%%%%%%%%%%%%%%%%%%%%%%%%%%%%%%%%%%%%%%%%%%%%%%%%%%%%%
\subsection{Included Files}
\label{sec:include}

%%%%%%%%%%%%%%%%%%%%%%%%%%%%%%%%%%%%%%%%
\DescribeMacro{\childdocmain}
To use the package, add the commands
\begin{center}
\begin{tabular}{l}
|\input{childdoc.def}|\\
|\childdocmain{}|\\
\end{tabular}
\end{center}
at the very top of the main \LaTeX{} file,
in particular \emph{before} the |\documentclass| statement!
The argument of |\childdocmain| should be left empty
(but it must be present).

%%%%%%%%%%%%%%%%%%%%%%%%%%%%%%%%%%%%%%%%
\DescribeMacro{\childdocof}
Furthermore, add the commands
\begin{center}
\begin{tabular}{l}
|\input{childdoc.def}|\\
|\childdocof{|\textit{main}|}|\\
\end{tabular}
\end{center}
at the top of every child file \textit{child}
which is included by |\include{|\textit{child}|}|
from within the main file
(or at least for those files to be compiled individually).
The argument \textit{main} must be the filename of the main file.

There are a couple of
considerations in setting up the main and child documents:

%%%%%%%%%%%%%%%%%%%%%%%%%%%%%%%%%%%%%%%%
\paragraph{Restrictions.}

Please note the following restrictions:
\begin{itemize}
\item
|\childdocmain| must be called with one argument \textit{main}
to ensure compatibility with earlier version of the package.
It must either be empty (|\childdocmain{}|)
or precisely match the filename of the main file in which it is specified.
See \secref{sec:detection} for further information.
\item
The filename \textit{main} must be specified without the |.tex| extension.
\item
The filename \textit{main} is case sensitive
(even in case-insensitive file systems)
due to internal string comparison.
\item
The argument \textit{main} should be fully expanded, it cannot be a macro.
\item
Subdirectories and special characters should be avoided in filenames.
\item
The command |\childdocmain{|\textit{main}|}| must be followed by a whitespace.
It should not be followed immediately by another command
or by a comment mark `|%|'.
This is because the \TeX{} parser reads the token immediately following
the argument of |\childdocmain| and puts it
at the beginning of every child section;
however, a white\-space is ignored.
\end{itemize}

%%%%%%%%%%%%%%%%%%%%%%%%%%%%%%%%%%%%%%%%
\paragraph{Content of Main File.}

It is advisable to place all content in the child files included by |\include|.
Any output contained in the main file will appear in all child documents
unless suppressed manually;
it cannot be suppressed automatically by the |\includeonly| directive
and thus should normally be avoided.
A method to include some content in the main file
by means of conditional processing is described in \secref{sec:conditional}.

%%%%%%%%%%%%%%%%%%%%%%%%%%%%%%%%%%%%%%%%
\paragraph{Page Numbering.}

When only a part of the document is compiled,
the appropriate numbering of pages
(as well as other status parameters)
is determined from the |.aux| files.
The latter contain information from previous passes.
However this information needs to propagate through
all intermediate child documents.
Therefore the page numbering in child documents may well
be inconsistent until the complete document is compiled at least once.

A useful (if unconventional) way to always ensure a consistent
page numbering is to restart the numbering in each child document
and denote the pages by `\textit{child}|.|\textit{page}'
where \textit{child} represents the chapter/section number of the child file.
This can be achieved by the command
|\numberwithin{page}{|\textit{child}|}|
of the \textsf{amsmath} package
where \textit{child} can be |chapter| or |section|
depending on the chosen structuring.
Alternatively, one can modify the macro |\thepage| appropriately
and reset the counter |page| at the start of each child file.

%%%%%%%%%%%%%%%%%%%%%%%%%%%%%%%%%%%%%%%%%%%%%%%%%%%%%%%%%%%%%%%%%%%%%%%%%%%%%%%%
\subsection{Conditional Processing}
\label{sec:conditional}

The package provides a mechanism to compile different versions
of a document. To customise the versions further some conditional processing
can come in handy to distinguish which version is being compiled.
The package provides two macros to describe the compilation context:

%%%%%%%%%%%%%%%%%%%%%%%%%%%%%%%%%%%%%%%%
\DescribeMacro{\ifchilddoc}
The conditional |\ifchilddoc| distinguishes between the compilation of
child documents and the main document:
%
\begin{center}
|\ifchilddoc |\textit{child-code}| |[|\||else |\textit{main-code}]| \||fi|
\end{center}

%%%%%%%%%%%%%%%%%%%%%%%%%%%%%%%%%%%%%%%%
\DescribeMacro{\childdocname}
\DescribeMacro{\childdocjob}
The macro |\childdocname| contains the filename (without extension)
of the main or child file being processed.
Note that |\childdocjob| will always contain the name of the main file.

%%%%%%%%%%%%%%%%%%%%%%%%%%%%%%%%%%%%%%%%
\paragraph{Title Page.}

Conditional processing can be used to include a title or banner page
in the main document when proper precautions are taken.
Importantly, the code in the main file should ensure that the page counter
(as well as other status parameters which are stored in the |.aux| files)
takes the same value after the conditional processing.
Otherwise the page numbers may take divergent values
depending on which part is compiled.

For example, a title page could be declared by:
%
\begin{center}
\begin{tabular}{l}
|\ifchilddoc\||else|\\
|\addtocounter{page}{-1}|\\
\textit{code for title page}\\
|\newpage|\\
|\||fi|
\end{tabular}
\end{center}
%
A banner page for the child documents can be generated by:
%
\begin{center}
\begin{tabular}{l}
|\ifchilddoc|\\
|\addtocounter{page}{-1}|\\
\textit{code for banner page}\\
|\newpage|\\
|\||fi|
\end{tabular}
\end{center}
%
Here one could write a message such as:
\begin{center}
|This is the part \childdocname{} of \childdocjob{}.|
\end{center}

%%%%%%%%%%%%%%%%%%%%%%%%%%%%%%%%%%%%%%%%%%%%%%%%%%%%%%%%%%%%%%%%%%%%%%%%%%%%%%%%
\subsection{Flags}
\label{sec:flags}

The package makes it easy to generate different versions
of the main or child documents.
To this end compilation flags can be defined
and assigned different default values.
They will be particularly useful in conjunction
with the forwarding mechanism described in \secref{sec:forward}.

For example, it may be useful to have a flag |\version|
which can be set to |draft| or |final|.
The document source will contain some conditional code
depending on the value of |\version|.
Suppose further, the flag should default to |final| for the main file
and to |draft| for child files
which is a natural assignment for editing the document.
This is achieved by placing the following code
in the preamble of the main document
(below the |\childdocmain| directive):
%
\begin{center}
\begin{tabular}{l}
|\ifchilddoc|\\
|\providecommand{\version}{draft}|\\
|\||else|\\
|\providecommand{\version}{final}|\\
|\||fi|
\end{tabular}
\end{center}
%
The definition by |\providecommand| makes sure
that previous definitions are not overwritten.
Further statements |\providecommand{\version}{...}|
can thus be added before the above code to override it.

For the main file, one might add a line
(between |\childdocmain| and the above block)
%
\begin{center}
|%\ifchilddoc\||else\providecommand{\version}{draft}\||fi|
\end{center}
%
which can be uncommented to produce a draft version.
Likewise one can add a line to the very top of a child file
(above the |\childdocof{|\textit{main}|}| directive)
%
\begin{center}
|%\providecommand{\version}{final}|
\end{center}
%
which can be uncommented to produce the final version of this child document.

%%%%%%%%%%%%%%%%%%%%%%%%%%%%%%%%%%%%%%%%%%%%%%%%%%%%%%%%%%%%%%%%%%%%%%%%%%%%%%%%
\subsection{Forwarding}
\label{sec:forward}

Different versions of the main or child documents
using compilation flags as described in \secref{sec:flags}
can be (permanently) stored in different files
for convenient compilation, viewing and distribution.
To this end, the package defines a command
to pass on compilation to a different file:

%%%%%%%%%%%%%%%%%%%%%%%%%%%%%%%%%%%%%%%%
\DescribeMacro{\childdocforward}
The command |\childdocforward| redirects processing to
another source file:
%
\begin{center}
\begin{tabular}{l}
|\input{childdoc.def}|\\
|\childdocforward[|\textit{main}|]{|\textit{dest}|}|\\
\end{tabular}
\end{center}
%
The argument \textit{dest} is the destination file
(without extension).
It should be the main file or one of the child files.
Note that further \textsf{childdoc} directives
such as |\childdocof| and |\childdocforward|
in the indicated file will be processed in this form.
The optional argument \textit{main}
passes on directly to the main file \textit{main}
while pretending to compile the child \textit{dest}.
This form behaves as if \textit{dest}
issues |\childdocof{|\textit{main}|}| right away,
and no further \textsf{childdoc} directives will be processed.

%%%%%%%%%%%%%%%%%%%%%%%%%%%%%%%%%%%%%%%%
\DescribeMacro{\...prefix}
In the alternative form |\childdocforwardprefix|,
%
\begin{center}
\begin{tabular}{l}
|\input{childdoc.def}|\\
|\childdocforwardprefix[|\textit{main}|]{|\textit{prefix}|}{|\textit{dest}|}|
\end{tabular}
\end{center}
%
the destination file is determined by a pattern
depending on the current file:
To make this work, the current file must be called
`{\textit{prefix}\hspace{0.2em}\textit{suffix}}'
with \textit{prefix} matching precisely the argument.
Processing is then passed on to the file
`{\textit{dest}\hspace{0.2em}\textit{suffix}}'.
Surely, the same effect is achieved by
directly specifying the
argument `{\textit{dest}\hspace{0.2em}\textit{suffix}}'
in the first form.
However, that requires to set up a different file
for each child. With the alternative form of the command
all these files can have exactly the same content
which simplifies setting them up and maintaining them.

For example, the following file |draft.tex|
with a compilation flag |\version| as described in \secref{sec:flags}
compiles the main document as a draft:
%
\begin{center}
\begin{tabular}{l}
|\def\version{draft}|\\
|\input{childdoc.def}|\\
|\childdocforward{|\textit{main}|}|
\end{tabular}
\end{center}
%
Likewise, the following files |final|\textit{nn}|.tex|
compile the final version of the child document
|child|\textit{nn}|.tex|:
%
\begin{center}
\begin{tabular}{l}
|\def\version{final}|\\
|\input{childdoc.def}|\\
|\childdocforwardprefix{final}{child}|
\end{tabular}
\end{center}
%

Note that when several versions of a main file and/or of each child file
are to be generated, it may be convenient to set up a |Makefile| or
shell script to automatise the process.

%%%%%%%%%%%%%%%%%%%%%%%%%%%%%%%%%%%%%%%%%%%%%%%%%%%%%%%%%%%%%%%%%%%%%%%%%%%%%%%%
\subsection{Command Line Processing}
\label{sec:commandline}

The effect of redirection files can also be achieved by invoking
the \LaTeX{} compiler with a more elaborate command line.
Most conveniently this should be done as part
of a shell script or a |Makefile|.

When using \textsf{childdoc} in the main file, the following
command lines effectively perform a redirection
(note that depending on the shell being used,
backslashes may have to be doubled: `|\|' $\to$ `|\\|'):
%
\begin{center}
|... -jobname "|\textit{target}|" |\\|"|[\textit{flags}]%
|\input{childdoc.def}\childdocforward[|\textit{main}|]{|\textit{dest}|}"|
\end{center}
%
Here \textit{target} is the name of the output file,
\textit{main} is the name of the main file
and \textit{dest} is the name of the main or child file to be processed
(all filenames without extensions).
The optional argument \textit{main} can be omitted
if \textit{main} matches \textit{dest}.
Optionally, compilation \textit{flags} can be defined via |\def| commands.
This command line makes the \TeX{} engine believe
it is compiling the file \textit{target}
whose content is specified as the latter parameter.
The provided code then forwards the processing to
\textit{main} or \textit{dest} as described in \secref{sec:forward}.

%%%%%%%%%%%%%%%%%%%%%%%%%%%%%%%%%%%%%%%%%%%%%%%%%%%%%%%%%%%%%%%%%%%%%%%%%%%%%%%%
\subsection{Include by Input}
\label{sec:input}

Including child documents by |\include| has some restrictions by design.
Most notably, the content of a child document always occupies
its own set of pages; pages cannot be shared between child documents.
Usually, this behaviour makes perfect sense
because each child document contain an essential part of the document.
However, in some situations it may be desirable to compose
a document from a collection of parts
without having mandatory page breaks between then.
For this case, the package
provides a mechanism to include parts
by |\input| which can also be processed individually.
However, by construction this mechanism
requires manual handling of the content to be output.

%%%%%%%%%%%%%%%%%%%%%%%%%%%%%%%%%%%%%%%%
\DescribeMacro{\ifchilddocmanual}
The main file should be prepared as usual, see \secref{sec:include}.
However, the document body must make a distinction
between processing of an individual part and of the main document, e.g.:
%
\begin{center}
\begin{tabular}{l}
|\ifchilddocmanual|\\
|\input{\childdocname}|\\
|\||else|\\
\textit{document body with }|\input{|\textit{part}|}|\\
|\||fi|
\end{tabular}
\end{center}
%
The conditional |\ifchilddocmanual| is true whenever
a part to be included by |\input| is being compiled,
and the name of the part is stored in |\childdocname|.

%%%%%%%%%%%%%%%%%%%%%%%%%%%%%%%%%%%%%%%%
\DescribeMacro{\childdocby}
Each part to be included by |\input| should start with:
%
\begin{center}
\begin{tabular}{l}
|\input{childdoc.def}|\\
|\childdocby{|\textit{main}|}|\\
\end{tabular}
\end{center}
%
The directive |\childdocby| is similar to |\childdocof|
described in \secref{sec:include},
but the subsequent selection of content must be done manually.
To that end, both |\ifchilddoc| and |\ifchilddocmanual|
will be true upon processing of a part,
and the name of the part is stored in |\childdocname|.
Note that |\jobname| will be set to the filename of the current part
so that each part receives an individual |.aux| file
that does not interfere with the |.aux| file(s) of the main document.
This behaviour can be altered by the alternative form
|\childdocby[*]{|\textit{main}|}| (with a non-empty optional argument)
which uses the |.aux| file of the main document
by setting |\jobname| to \textit{main}.

%%%%%%%%%%%%%%%%%%%%%%%%%%%%%%%%%%%%%%%%%%%%%%%%%%%%%%%%%%%%%%%%%%%%%%%%%%%%%%%%
\subsection{Driver Development}
\label{sec:driver}

The \textsf{childdoc} mechanism can also be use for the development
of definition files such as \LaTeX{} styles or classes.
This case differs from the above setup with multiple parts
included by |\include| in that no |\includeonly| should be invoked.
This can be achieved by starting the include file
(before |\ProvidesPackage|) with:
%
\begin{center}
\begin{tabular}{l}
|\input{childdoc.def}|\\
|\childdocforward{|\textit{main}|}|\\
\end{tabular}
\end{center}
%
or alternatively with:
%
\begin{center}
\begin{tabular}{l}
|\input{childdoc.def}|\\
|\childdocby{|\textit{main}|}|\\
\end{tabular}
\end{center}
%
Both forms have slightly different effects as described above.
The main file is prepared as usual, see \secref{sec:include}.

%%%%%%%%%%%%%%%%%%%%%%%%%%%%%%%%%%%%%%%%%%%%%%%%%%%%%%%%%%%%%%%%%%%%%%%%%%%%%%%%
\subsection{Legacy Detection}
\label{sec:detection}

The directive |\childdocmain| in the main file can detect
whether the complete document or merely a child is to be compiled
even without using the directive |\childdocof|.
This method is deprecated because it is less robust
and there is no compelling reason to use it;
it is merely provided for backward compatibility
and it may be removed in future versions.

If the detection mechanism is to be used,
it is mandatory to correctly specify
the filename of the main file as the argument of |\childdocmain|:
%
\begin{center}
\begin{tabular}{l}
|\input{childdoc.def}|\\
|\childdocmain{|\textit{main}|}|\\
\end{tabular}
\end{center}
%
If |\jobname| does not match the argument \textit{main} of |\childdocmain|,
it is assumed that |\jobname| points to the child file to be compiled.
When using |\childdocmain| with the main file specified as argument,
it suffices to start a child file
with just |\input{|\textit{main}|}|
without loading of the package and using |\childdocof|.
If instead all processing is done
with the appropriate \textsf{childdoc} directives,
the argument of \textit{main} of |\childdocmain| can be empty.

An alternative version of the command line processing described
in \secref{sec:commandline} using the detection mechanism reads:
%
\begin{center}
|... -jobname "|\textit{target}|" "|[\textit{flags}]%
[|\def\jobname{|\textit{dest}|}|]|\input{|\textit{main}|}"|
\end{center}

%%%%%%%%%%%%%%%%%%%%%%%%%%%%%%%%%%%%%%%%%%%%%%%%%%%%%%%%%%%%%%%%%%%%%%%%%%%%%%%%
\subsection{Manual Code}
\label{sec:manual}

In case one cannot be certain whether the definitions file |childdoc.def|
is installed on the target \TeX{} distribution
and one prefers not to ship it,
it is conceivable to paste a few relevant commands into the sources.

To that end, drop all statements |\input{childdoc.def}|
and perform the replacements as outlined below.
Instead of |\childdocmain{|\textit{main}|}| add the following code
to the top of the main file:
%
\begin{center}
\begin{tabular}{l}
|\||ifdefined\childdocname\endinput\||fi\newif\ifchilddoc|\\
|\edef\childdocname{\scantokens\expandafter{\jobname\noexpand}}|\\
|\def\childdocmain{|\textit{main}|}\||ifx\childdocmain\childdocname\||else|\\
|\childdoctrue\includeonly{\childdocname}\let\jobname\childdocmain\||fi|\\
\end{tabular}
\end{center}
%
Instead of |\childdocof{|\textit{main}|}| just include the main file
at the top of each child file:
%
\begin{center}
|\input{|\textit{main}|}|
\end{center}
%
A simple redirection |\childdocforward{|\textit{dest}|}| is achieved by:
%
\begin{center}
|\def\jobname{|\textit{dest}|}\input{\jobname}|
\end{center}
%
The redirection with prefix
|\childdocforwardprefix[|\textit{prefix}|]{|\textit{dest}|}|
is accomplished by:
%
\begin{center}
\begin{tabular}{l}
|{\edef\jobname{\scantokens\expandafter{\jobname\noexpand}}|\\
|\def\redirectjob |\textit{prefix}|#1~~~{\gdef\jobname{|\textit{dest}|#1}}|\\
|\expandafter\redirectjob\jobname~~~}\input{\jobname}|
\end{tabular}
\end{center}

In an alternative approach,
child documents can be compiled by a specific command line
without additional code or specific definitions:
%
\begin{center}
|... -jobname "|\textit{target}|" "|[\textit{flags}]%
|\includeonly{|\textit{dest}|}\input{|\textit{main}|}"|
\end{center}
%

%%%%%%%%%%%%%%%%%%%%%%%%%%%%%%%%%%%%%%%%%%%%%%%%%%%%%%%%%%%%%%%%%%%%%%%%%%%%%%%%
%%%%%%%%%%%%%%%%%%%%%%%%%%%%%%%%%%%%%%%%%%%%%%%%%%%%%%%%%%%%%%%%%%%%%%%%%%%%%%%%
\section{Information}

%%%%%%%%%%%%%%%%%%%%%%%%%%%%%%%%%%%%%%%%%%%%%%%%%%%%%%%%%%%%%%%%%%%%%%%%%%%%%%%%
\subsection{Copyright}

Copyright \copyright{} 2017--2018 Niklas Beisert

This work may be distributed and/or modified under the
conditions of the \LaTeX{} Project Public License, either version 1.3
of this license or (at your option) any later version.
The latest version of this license is in
  \url{http://www.latex-project.org/lppl.txt}
and version 1.3 or later is part of all distributions of \LaTeX{}
version 2005/12/01 or later.

This work has the LPPL maintenance status `maintained'.

The Current Maintainer of this work is Niklas Beisert.

This work consists of the files |README.txt|, |childdoc.ins| and |childdoc.dtx|
as well as the derived files |childdoc.def|, |cdocsamp.tex|
with |cdocsch1.tex|, |cdocsch2.tex|, |cdocspt3.tex|, |cdocspt4.tex|,
|cdocsdrf.tex|, |cdocsfn1.tex|, |cdocsfn2.tex|
as well as |childdoc.pdf|.

%%%%%%%%%%%%%%%%%%%%%%%%%%%%%%%%%%%%%%%%%%%%%%%%%%%%%%%%%%%%%%%%%%%%%%%%%%%%%%%%
\subsection{Files and Installation}

The package consists of the files:
%
\begin{center}
\begin{tabular}{ll}
    |README.txt|   & readme file \\
    |childdoc.ins| & installation file \\
    |childdoc.dtx| & source file \\
    |childdoc.def| & definition file \\
    |cdocsamp.tex| & sample main file \\
    |cdocsch1.tex| & sample include file \\
    |cdocsch2.tex| & sample include file \\
    |cdocspt3.tex| & sample part file \\
    |cdocspt4.tex| & sample part file \\
    |cdocsdrf.tex| & sample redirection file \\
    |cdocsfn1.tex| & sample redirection file \\
    |cdocsfn2.tex| & sample redirection file \\
    |childdoc.pdf| & manual
\end{tabular}
\end{center}
%
The distribution consists of the files
|README.txt|, |childdoc.ins| and |childdoc.dtx|.
%
\begin{itemize}
\item
Run (pdf)\LaTeX{} on |childdoc.dtx|
to compile the manual |childdoc.pdf| (this file).
\item
Run \LaTeX{} on |childdoc.ins| to create the definitions file |childdoc.def|
and the sample |cdocsamp.tex| with include files
|cdocsch1.tex|, |cdocsch2.tex|, |cdocspt3.tex|, |cdocspt4.tex|,
|cdocsdrf.tex|, |cdocsfn1.tex|, |cdocsfn2.tex|.
Then copy the file |childdoc.def| to an appropriate directory of your \LaTeX{}
distribution, e.g.\ \textit{texmf-root}|/tex/latex/childdoc|.
\end{itemize}

%%%%%%%%%%%%%%%%%%%%%%%%%%%%%%%%%%%%%%%%%%%%%%%%%%%%%%%%%%%%%%%%%%%%%%%%%%%%%%%%
\subsection{Related CTAN Packages}

There are several other packages which offer a similar functionality:
%
\begin{itemize}
\item
The packages
\href{http://ctan.org/pkg/docmute}{\textsf{docmute}},
\href{http://ctan.org/pkg/includex}{\textsf{includex}} and
\href{http://ctan.org/pkg/standalone}{\textsf{standalone}}
provide commands to include only the document body of
a child file thus allowing both files to be compiled individually.
\item
The packages \href{http://ctan.org/pkg/subdocs}{\textsf{subdocs}}
and \href{http://ctan.org/pkg/subfiles}{\textsf{subfiles}}
provide structures in which the main and child documents can be
encapsulated and allowing them to be compiled individually.
The inclusion mechanism is different from the conventional |\include|.
\item
The package \href{http://ctan.org/pkg/combine}{\textsf{combine}}
is an elaborate solution to combine several documents into one.
\end{itemize}
%
See also the CTAN topic \href{http://ctan.org/topic/subdocs}{\textsf{subdocs}}
for further related packages.
The present package differs from the above solutions in that
a document structure constructed with the conventional |\include| mechanism
just needs two extra commands at the top of every file
such that all constituent files can be compiled individually.

%%%%%%%%%%%%%%%%%%%%%%%%%%%%%%%%%%%%%%%%%%%%%%%%%%%%%%%%%%%%%%%%%%%%%%%%%%%%%%%%
%\subsection{Feature Suggestions}
%
%The following is a list of features which may be useful for future
%versions of this package:
%%
%\begin{itemize}
%\item
%\ldots
%\end{itemize}

%%%%%%%%%%%%%%%%%%%%%%%%%%%%%%%%%%%%%%%%%%%%%%%%%%%%%%%%%%%%%%%%%%%%%%%%%%%%%%%%
\subsection{Revision History}

%%%%%%%%%%%%%%%%%%%%%%%%%%%%%%%%%%%%%%%%
\paragraph{v2.0:} 2018/12/30

\begin{itemize}
\item
immediate forward processing
\item
added |\childdocby| mechanism
\item
manual restructured
\end{itemize}

%%%%%%%%%%%%%%%%%%%%%%%%%%%%%%%%%%%%%%%%
\paragraph{v1.6:} 2018/01/17

\begin{itemize}
\item
application for development of include files
\item
corrections to manual
\end{itemize}

%%%%%%%%%%%%%%%%%%%%%%%%%%%%%%%%%%%%%%%%
\paragraph{v1.5:} 2017/05/21

\begin{itemize}
\item
more complete structuring introduced
\item
|\childdocof| introduced
\item
|\childdoc| renamed to |\childdocmain|
\item
|\childredirect| renamed to |\childdocforward| and |\childdocforwardprefix|
and functionality expanded
\end{itemize}

%%%%%%%%%%%%%%%%%%%%%%%%%%%%%%%%%%%%%%%%
\paragraph{v1.0:} 2017/04/27

\begin{itemize}
\item
manual and install package
\item
first version published on CTAN
\end{itemize}

%%%%%%%%%%%%%%%%%%%%%%%%%%%%%%%%%%%%%%%%
\paragraph{v0.6:} 2017/04/26

\begin{itemize}
\item
redirection mechanism added
\end{itemize}

%%%%%%%%%%%%%%%%%%%%%%%%%%%%%%%%%%%%%%%%
\paragraph{v0.5:} 2017/04/26

\begin{itemize}
\item
functionality in definition file
\end{itemize}


%%%%%%%%%%%%%%%%%%%%%%%%%%%%%%%%%%%%%%%%%%%%%%%%%%%%%%%%%%%%%%%%%%%%%%%%%%%%%%%%
%%%%%%%%%%%%%%%%%%%%%%%%%%%%%%%%%%%%%%%%%%%%%%%%%%%%%%%%%%%%%%%%%%%%%%%%%%%%%%%%
%%%%%%%%%%%%%%%%%%%%%%%%%%%%%%%%%%%%%%%%%%%%%%%%%%%%%%%%%%%%%%%%%%%%%%%%%%%%%%%%
\appendix

\settowidth\MacroIndent{\rmfamily\scriptsize 000\ }

 \DocInput{childdoc.dtx}

\end{document}
%</driver>
% \fi
%
% %%%%%%%%%%%%%%%%%%%%%%%%%%%%%%%%%%%%%%%%%%%%%%%%%%%%%%%%%%%%%%%%%%%%%%%%%%%%%%
% %%%%%%%%%%%%%%%%%%%%%%%%%%%%%%%%%%%%%%%%%%%%%%%%%%%%%%%%%%%%%%%%%%%%%%%%%%%%%%
% \section{Sample}
%\iffalse
%<*samplemain>
%\fi
%
% The following presents a sample document
% with two chapters, two parts, a title page,
% a compile flag as well as three forwarding files to set the flag.
% It consists of eight |.tex| files:
% \begin{center}
% \begin{tabular}{ll}
% |cdocsamp.tex|&main file\\
% |cdocsch1.tex|&include file for chapter 1\\
% |cdocsch2.tex|&include file for chapter 2\\
% |cdocspt3.tex|&include file for part 3\\
% |cdocspt4.tex|&include file for part 4\\
% |cdocsdrf.tex|&forwarding file for main file in draft mode\\
% |cdocsfi1.tex|&forwarding file for final version of chapter 1\\
% |cdocsfi2.tex|&forwarding file for final version of chapter 2\\
% \end{tabular}
% \end{center}
% Each of the eight files can be compiled directly by the \LaTeX{} compiler.
%
% %%%%%%%%%%%%%%%%%%%%%%%%%%%%%%%%%%%%%%
% \paragraph{Main File.}
%
% The main file is called |cdocsamp.tex|.
%
% Load the \textsf{childdoc} definitions and
% declare the filename for the main document:
%    \begin{macrocode}
\input{childdoc.def}
\childdocmain{}
%    \end{macrocode}

% Optional override for |\version| flag:
%    \begin{macrocode}
%%\ifchilddoc\else\providecommand{\version}{draft}\fi
%    \end{macrocode}

% Define the default values for the |\version| flag
% (|final| for the main file and |draft| for childs):
%    \begin{macrocode}
\ifchilddoc
\providecommand{\version}{draft}
\else
\providecommand{\version}{final}
\fi
%    \end{macrocode}

% Load the standard document class:
%    \begin{macrocode}
\documentclass[12pt]{article}
%    \end{macrocode}

% Start the document body:
%    \begin{macrocode}
\begin{document}
%    \end{macrocode}

% Declare a title page.
% Print title, part of document being processed and version flag:
%    \begin{macrocode}
\addtocounter{page}{-1}
\begin{center}
{\LARGE\bfseries{}childdoc example\par}
\vspace{1cm}
\ifchilddoc
\ifchilddocmanual part\else chapter\fi:
`\childdocname' of `\childdocjob'\par
\else
main document: `\childdocjob'\par
\fi
version: \version\par
\end{center}
\newpage
%    \end{macrocode}

% Manually include selected file,
% otherwise process as usual:
%    \begin{macrocode}
\ifchilddocmanual
\section*{part `\childdocname'}
\input{\childdocname}
\else
%    \end{macrocode}

% Include the two chapters:
%    \begin{macrocode}
\include{cdocsch1}
\include{cdocsch2}
%    \end{macrocode}

% Include the two parts unless only chapters should be displayed:
%    \begin{macrocode}
\ifchilddoc\else
\section{part three}
\input{cdocspt3}
\section{part four}
\input{cdocspt4}
\fi
%    \end{macrocode}

% Process as usual until here:
%    \begin{macrocode}
\fi
%    \end{macrocode}

% End of document body:
%    \begin{macrocode}
\end{document}
%    \end{macrocode}
%\iffalse
%</samplemain>
%\fi
%
% %%%%%%%%%%%%%%%%%%%%%%%%%%%%%%%%%%%%%%
% \paragraph{Chapter Include Files.}
%
% The include files are called |cdocsch1.tex| and |cdocsch2.tex|.
%
%\iffalse
%<*samplechap1|samplechap2>
%\fi

% Optional override for |\version| flag:
%    \begin{macrocode}
%%\providecommand{\version}{final}
%    \end{macrocode}

% Include the main document:
%    \begin{macrocode}
\input{childdoc.def}
\childdocof{cdocsamp}
%    \end{macrocode}

%\iffalse
%</samplechap1|samplechap2>
%\fi
%
%\iffalse
%<*samplechap1>
%\fi
% Some text for chapter 1:
%    \begin{macrocode}
\section{one}
some text in chapter one
%    \end{macrocode}

%\iffalse
%</samplechap1>
%\fi
% Some text for chapter 2:
%\iffalse
%<*samplechap2>
%\fi
%    \begin{macrocode}
\section{two}
more text in chapter two
%    \end{macrocode}

%\iffalse
%</samplechap2>
%\fi
%
% %%%%%%%%%%%%%%%%%%%%%%%%%%%%%%%%%%%%%%
% \paragraph{Part Include Files.}
%
% The include files are called |cdocspt3.tex| and |cdocspt4.tex|.
%
%\iffalse
%<*samplepart3|samplepart4>
%\fi

% Optional override for |\version| flag:
%    \begin{macrocode}
%%\providecommand{\version}{final}
%    \end{macrocode}

% Include the main document:
%    \begin{macrocode}
\input{childdoc.def}
\childdocby{cdocsamp}
%    \end{macrocode}

%\iffalse
%</samplepart3|samplepart4>
%\fi
%
%\iffalse
%<*samplepart3>
%\fi
% Some text for part 3:
%    \begin{macrocode}
some text in part three
%    \end{macrocode}

%\iffalse
%</samplepart3>
%\fi
% Some text for part 4:
%\iffalse
%<*samplepart4>
%\fi
%    \begin{macrocode}
more text in part four
%    \end{macrocode}

%\iffalse
%</samplepart4>
%\fi
%
% %%%%%%%%%%%%%%%%%%%%%%%%%%%%%%%%%%%%%%
% \paragraph{Forwarding for a Complete Draft.}
%
% The following forwarding file |cdocsdrf.tex|
% compiles the main document in draft mode:
%\iffalse
%<*sampledraft>
%\fi
%    \begin{macrocode}
\def\version{draft}
\input{childdoc.def}
\childdocforward{cdocsamp}
%    \end{macrocode}

%\iffalse
%</sampledraft>
%\fi
%
% %%%%%%%%%%%%%%%%%%%%%%%%%%%%%%%%%%%%%%
% \paragraph{Forwarding for Final Version of the Chapters.}
%
% The following forwarding files |cdocsfn1.tex| and |cdocsfn2.tex|
% (with identical content)
% compile the final versions of the child documents
% |cdocsch1.tex| and |cdocsch2.tex|, respectively:
%\iffalse
%<*samplefinal>
%\fi
%    \begin{macrocode}
\def\version{final}
\input{childdoc.def}
\childdocforwardprefix[cdocsamp]{cdocsfn}{cdocsch}
%    \end{macrocode}

%\iffalse
%</samplefinal>
%\fi
%
% %%%%%%%%%%%%%%%%%%%%%%%%%%%%%%%%%%%%%%
% \paragraph{Command Line Processing.}
%
% The following three command lines generate the output files
% |cdocscld|, |cdocscl1| and |cdocscl2|
% which should be identical to
% |cdocsdrf|, |cdocsch1| and |cdocsfn2|, respectively:
% \begin{center}
% \begin{tabular}{l}
% |latex -jobname cdocscld \|\\
% |  "\def\version{draft}\input{childdoc.def}\childdocforward{cdocsamp}"|\\
% |latex -jobname cdocscl1 \|\\
% |  "\input{childdoc.def}\childdocforward[cdocsamp]{cdocsch1}"|\\
% |latex -jobname cdocscl2 \|\\
% |  "\def\version{final}\input{childdoc.def}\childdocforward{cdocsch2}"|
% \end{tabular}
% \end{center}
% Note that the trailing backslash on each first line
% merely continues the input to the second line
% (for convenient cut ant paste).
% Furthermore, the command |latex| can be replaced by any
% of its alternative versions such as |pdflatex|.
%
% %%%%%%%%%%%%%%%%%%%%%%%%%%%%%%%%%%%%%%%%%%%%%%%%%%%%%%%%%%%%%%%%%%%%%%%%%%%%%%
% %%%%%%%%%%%%%%%%%%%%%%%%%%%%%%%%%%%%%%%%%%%%%%%%%%%%%%%%%%%%%%%%%%%%%%%%%%%%%%
% \section{Implementation}
%\iffalse
%<*package>
%\fi
%
% This section describes the definitions file |childdoc.def|.

% The definitions cannot be loaded using |\usepackage| or |\RequirePackage|
% which has a mechanism to prevent loading a style file more than once.
% When loading the definitions by means of |\input|
% multiple instances have to be prevented manually:
%\iffalse
%This code needs to be before the `\ProvidesFile' directive
%which is defined at the beginning of this file.
%Therefore it is also placed there and commented out here.
%</package>
%<*discard>
%\fi
%    \begin{macrocode}
\ifdefined\childdocmain\endinput\fi
%    \end{macrocode}
%\iffalse
%</discard>
%<*package>
%\fi
%
% \macro{\ifchilddoc}
% \macro{\ifchilddocmanual}
% The conditional |\ifchilddoc| tells whether a
% child (true) or main (false) document is being compiled.
% The conditional |\ifchilddocmanual| tells whether
% the |\includeonly| mechanism is used (false) or
% the selection of child files must be performed manually (true).
% The definitions initialise to false:
%    \begin{macrocode}
\newif\ifchilddoc
\newif\ifchilddocmanual
%    \end{macrocode}

% \macro{\childdocname}
% \macro{\childdocjob}
% The macro |\childdocname| stores the name of the main document
% to be compiled. The macro |\childdocjob| stores the name of
% the document on which the \LaTeX{} compiler was originally invoked.
% The content of |\jobname| cannot be compared
% to filenames specified in the source due to different catcodes.
% The following code rescans |\jobname|, stores the result
% in |\childdocname| and saves a copy in |\childdocjob|:
%    \begin{macrocode}
\edef\childdocname{\scantokens\expandafter{\jobname\noexpand}}
\let\childdocjob\childdocname
%    \end{macrocode}

% \macro{\childdocdisable}
% The macro |\childdocdisable| prevents the main file
% from being processed more than once.
% At this stage, the main document command |\childdocmain|
% is assumed to be called once again where it should do nothing.
% Any subsequent call to it should prevent
% a secondary processing of the main document
% It overwrites the forwarding commands
% |\childdocof| and |\childdocforward|
% with empty macros to prevent further inclusions of the main document:
%    \begin{macrocode}
\newcommand{\childdocdisable}
{
  \renewcommand{\childdocmain}[1]{\renewcommand{\childdocmain}[1]{\endinput}}
  \renewcommand{\childdocof}[1]{}
  \renewcommand{\childdocby}[2][]{}
  \renewcommand{\childdocforward}[2][]{}
  \renewcommand{\childdocdisable}{}
}
%    \end{macrocode}

% \macro{\childdocmain}
% The macro |\childdocmain| is to be called at the top of the main file
% with nothing or the main filename (without extension) as argument.
% First, it breaks loops.
% If the argument is not empty and does not match |\childdocname|
% (which is set by the first inclusion of |childdoc.def|),
% |\ifchilddoc| is set to true, |\includeonly| is applied to the child file
% and |\jobname| is set to the main file
% (for proper handling of |.aux| files):
%    \begin{macrocode}
\newcommand{\childdocmain}[1]
{
  \childdocdisable\childdocmain{}
  \if?#1?\else
    \begingroup
      \def\childdoctmp{#1}
      \ifx\childdoctmp\childdocname
        \def\childdoctmp{}
      \else
        \def\childdoctmp
        {
          \childdoctrue
          \includeonly{\childdocname}
          \def\childdocjob{#1}
          \def\jobname{#1}
        }
      \fi
      \expandafter
    \endgroup
    \childdoctmp
  \fi
}
%    \end{macrocode}

% \macro{\childdocof}
% The command |\childdocof| redirects
% compilation to the main file |#1|.
%    \begin{macrocode}
\newcommand{\childdocof}[1]
{
  \childdocdisable
  \childdoctrue
  \includeonly{\childdocname}
  \def\jobname{#1}
  \def\childdocjob{#1}
  \input{#1}
}
%    \end{macrocode}

% \macro{\childdocby}
% The command |\childdocby| ....
%    \begin{macrocode}
\newcommand{\childdocby}[2][]
{
  \childdocdisable
  \childdoctrue
  \childdocmanualtrue
  \if?#1?\else
    \def\jobname{#2}
  \fi
  \def\childdocjob{#2}
  \input{#2}
  \endinput
}
%    \end{macrocode}

% \macro{\childdocforward}
% The command |\childdocforward| redirects
% compilation to the main file or
% (if the optional argument is given) a child file.
% Parameters are set as if the main file
% or a child file starting with |\childdocof| was compiled.
% Then compilation is handed over to the main file:
%    \begin{macrocode}
\newcommand{\childdocforward}[2][]
{
  \begingroup
    \if?#1?
      \def\childdoctmp
      {
        \def\childdocname{#2}
        \def\childdocjob{#2}
        \def\jobname{#2}
        \input{#2}
        \endinput
      }
    \else
      \def\childdoctmp
      {
        \childdocdisable
        \def\childdocname{#2}
        \childdoctrue
        \includeonly{#2}
        \def\childdocjob{#1}
        \def\jobname{#1}
        \input{#1}
        \endinput
      }
    \fi
    \expandafter
  \endgroup
  \childdoctmp
}
%    \end{macrocode}

% \macro{\childdocforwardprefix}
% The command |\childdocforwardprefix| redirects
% compilation to the main or a child file by means of a pattern.
% The prefix |#1| in the current filename is replaced by |#2|
% and the suffix of the current filename is kept
% (it is assumed that the filename does not contain the substring `|~~~|'
% which is used as a delimiter).
% Compilation is handed over to the new file by |\childdocforward|:
%    \begin{macrocode}
\newcommand{\childdocforwardprefix}[3][]
{
  \begingroup
    \def\childdocextract #2##1~~~{\def\childdoctmp{\childdocforward[#1]{#3##1}}}
    \expandafter\childdocextract\childdocname~~~
    \expandafter
  \endgroup
  \childdoctmp
}
%    \end{macrocode}

% \macro{\childdoc}
% The deprecated macro |\childdoc| is a legacy version of |\childdocmain|:
%    \begin{macrocode}
\newcommand{\childdoc}{\childdocmain}
%    \end{macrocode}

% \macro{\childdocredirect}
% The deprecated macro |\childdocredirect| is a legacy version
% of |\childdocforward| and |\childdocforwardprefix|:
%    \begin{macrocode}
\newcommand{\childdocredirect}[2][]
{
  \begingroup
    \if?#1?
      \def\childdoctmp{\childdocforward{#2}}
    \else
      \def\childdoctmp{\childdocforwardprefix{#1}{#2}}
    \fi
    \expandafter
  \endgroup
  \childdoctmp
}
%    \end{macrocode}

%\iffalse
%</package>
%\fi
%
\endinput

\childdocforward{cdocsamp}
%    \end{macrocode}

%\iffalse
%</sampledraft>
%\fi
%
% %%%%%%%%%%%%%%%%%%%%%%%%%%%%%%%%%%%%%%
% \paragraph{Forwarding for Final Version of the Chapters.}
%
% The following forwarding files |cdocsfn1.tex| and |cdocsfn2.tex|
% (with identical content)
% compile the final versions of the child documents
% |cdocsch1.tex| and |cdocsch2.tex|, respectively:
%\iffalse
%<*samplefinal>
%\fi
%    \begin{macrocode}
\def\version{final}
% \iffalse
%
% childdoc.dtx Copyright (C) 2017-2018 Niklas Beisert
%
% This work may be distributed and/or modified under the
% conditions of the LaTeX Project Public License, either version 1.3
% of this license or (at your option) any later version.
% The latest version of this license is in
%   http://www.latex-project.org/lppl.txt
% and version 1.3 or later is part of all distributions of LaTeX
% version 2005/12/01 or later.
%
% This work has the LPPL maintenance status `maintained'.
%
% The Current Maintainer of this work is Niklas Beisert.
%
% This work consists of the files childdoc.dtx and childdoc.ins
% and the derived files childdoc.def and cdocsamp.tex with
% cdocsch1.tex, cdocsch2.tex, cdocsdrf.tex, cdocsfn1.tex, cdocsfn2.tex.
%
%<package>\ifdefined\childdocmain\endinput\fi
%<package>\ProvidesFile{childdoc.def}[2018/12/30 v2.0 child document driver]
%<samplemain>\ProvidesFile{cdocsamp.tex}[2018/12/30 v2.0 sample for childdoc]
%<*driver>
%\ProvidesFile{childdoc.drv}[2018/12/30 v2.0 childdoc reference manual file]
\PassOptionsToClass{10pt,a4paper}{article}
\documentclass{ltxdoc}

\usepackage[margin=35mm]{geometry}
\usepackage{hyperref}
\usepackage{hyperxmp}
\usepackage[usenames]{color}

\hypersetup{colorlinks=true}
\hypersetup{pdfstartview=FitH}
\hypersetup{pdfpagemode=UseNone}
\hypersetup{pdfsource={}}
\hypersetup{pdflang={en-UK}}
\hypersetup{pdfcopyright={Copyright 2017-2018 Niklas Beisert.
  This work may be distributed and/or modified under the
  conditions of the LaTeX Project Public License, either version 1.3
  of this license or (at your option) any later version.}}
\hypersetup{pdflicenseurl={http://www.latex-project.org/lppl.txt}}
\hypersetup{pdfcontactaddress={ETH Zurich, ITP, HIT K,
  Wolfgang-Pauli-Strasse 27}}
\hypersetup{pdfcontactpostcode={8093}}
\hypersetup{pdfcontactcity={Zurich}}
\hypersetup{pdfcontactcountry={Switzerland}}
\hypersetup{pdfcontactemail={nbeisert@itp.phys.ethz.ch}}
\hypersetup{pdfcontacturl={http://people.phys.ethz.ch/\xmptilde nbeisert/}}

\newcommand{\secref}[1]{\hyperref[#1]{section \ref*{#1}}}

\parskip1ex
\parindent0pt
\let\olditemize\itemize
\def\itemize{\olditemize\parskip0pt}

\begin{document}

\title{The \textsf{childdoc} Package}
\hypersetup{pdftitle={The childdoc Package}}
\author{Niklas Beisert\\[2ex]
  Institut f\"ur Theoretische Physik\\
  Eidgen\"ossische Technische Hochschule Z\"urich\\
  Wolfgang-Pauli-Strasse 27, 8093 Z\"urich, Switzerland\\[1ex]
  \href{mailto:nbeisert@itp.phys.ethz.ch}
  {\texttt{nbeisert@itp.phys.ethz.ch}}}
\hypersetup{pdfauthor={Niklas Beisert}}
\hypersetup{pdfsubject={Manual for the LaTeX2e Package childdoc}}
\date{30 December 2018, \textsf{v2.0}}
\maketitle

\begin{abstract}\noindent
\textsf{childdoc} is a \LaTeXe{} package
that enables the direct compilation
of document sections included by |\include|
to individual files.
\end{abstract}

\begingroup
\parskip0ex
\tableofcontents
\endgroup

%%%%%%%%%%%%%%%%%%%%%%%%%%%%%%%%%%%%%%%%%%%%%%%%%%%%%%%%%%%%%%%%%%%%%%%%%%%%%%%%
%%%%%%%%%%%%%%%%%%%%%%%%%%%%%%%%%%%%%%%%%%%%%%%%%%%%%%%%%%%%%%%%%%%%%%%%%%%%%%%%
\section{Introduction}

\LaTeX{} provides a mechanism to structure a large document (such as a book)
into a main file and several child files (containing the chapters)
using the |\include| command.
This mechanism is beneficial for documents
which span hundreds of pages in order to
make the source file(s) more manageable.
Moreover, compilation can be restricted to
selected child files by means of the |\includeonly| command.
The latter feature can be used to reduce the compilation time while editing
(this was significantly more useful in the earlier days of \LaTeX{})
or to generate a smaller document which is easier to navigate.
Another application of |\includeonly| is to generate
documents consisting of selected parts of the complete document.

However, there are a few drawbacks of the plain |\include| mechanism:
\begin{itemize}
\item
The child files cannot be compiled on their own,
they can only be compiled via the main file.
A naive editing environment
(such as a text editor with an option
to have the current file processed by \LaTeX)
may require one to switch to the main file before compiling;
attempting to compile the child file produces errors.
\item
The main file must be modified (each time)
to adjust the |\includeonly| command
to the present needs. This easily leaves the main file in a messy state.
\item
The generated document will always carry the filename
of the main document. This is inconvenient if
several child files are to be compiled and
to be kept for distribution.
\end{itemize}

The present package provides a simple interface
to make child files individually compilable by \LaTeX{}.
Compiling a child file then has the same effect as compiling
the main file with an |\includeonly| command
to select the appropriate child.
Moreover the generated document will carry the name of the child
rather than the main file.
This resolves all three above issues.

This feature is meant to make the editing of books,
thesis documents and lecture notes somewhat more convenient.
However, the package can also be used efficiently for
composing a series of documents (such as exercise sheets)
which are typically distributed individually.
It then assists the author in generating the individual documents
(potentially in different versions)
as well as a document containing the collected series.
Another application is in developing style files
or other kinds of included material
where compilation of the style file could redirect
to a sample or test file.

%%%%%%%%%%%%%%%%%%%%%%%%%%%%%%%%%%%%%%%%%%%%%%%%%%%%%%%%%%%%%%%%%%%%%%%%%%%%%%%%
%%%%%%%%%%%%%%%%%%%%%%%%%%%%%%%%%%%%%%%%%%%%%%%%%%%%%%%%%%%%%%%%%%%%%%%%%%%%%%%%
\section{Usage}

First of all, the package \textsf{childdoc} is \emph{not} a standard
\LaTeXe{} |.sty| style file! Therefore it needs to be invoked in
a non-standard way.

%%%%%%%%%%%%%%%%%%%%%%%%%%%%%%%%%%%%%%%%%%%%%%%%%%%%%%%%%%%%%%%%%%%%%%%%%%%%%%%%
\subsection{Included Files}
\label{sec:include}

%%%%%%%%%%%%%%%%%%%%%%%%%%%%%%%%%%%%%%%%
\DescribeMacro{\childdocmain}
To use the package, add the commands
\begin{center}
\begin{tabular}{l}
|\input{childdoc.def}|\\
|\childdocmain{}|\\
\end{tabular}
\end{center}
at the very top of the main \LaTeX{} file,
in particular \emph{before} the |\documentclass| statement!
The argument of |\childdocmain| should be left empty
(but it must be present).

%%%%%%%%%%%%%%%%%%%%%%%%%%%%%%%%%%%%%%%%
\DescribeMacro{\childdocof}
Furthermore, add the commands
\begin{center}
\begin{tabular}{l}
|\input{childdoc.def}|\\
|\childdocof{|\textit{main}|}|\\
\end{tabular}
\end{center}
at the top of every child file \textit{child}
which is included by |\include{|\textit{child}|}|
from within the main file
(or at least for those files to be compiled individually).
The argument \textit{main} must be the filename of the main file.

There are a couple of
considerations in setting up the main and child documents:

%%%%%%%%%%%%%%%%%%%%%%%%%%%%%%%%%%%%%%%%
\paragraph{Restrictions.}

Please note the following restrictions:
\begin{itemize}
\item
|\childdocmain| must be called with one argument \textit{main}
to ensure compatibility with earlier version of the package.
It must either be empty (|\childdocmain{}|)
or precisely match the filename of the main file in which it is specified.
See \secref{sec:detection} for further information.
\item
The filename \textit{main} must be specified without the |.tex| extension.
\item
The filename \textit{main} is case sensitive
(even in case-insensitive file systems)
due to internal string comparison.
\item
The argument \textit{main} should be fully expanded, it cannot be a macro.
\item
Subdirectories and special characters should be avoided in filenames.
\item
The command |\childdocmain{|\textit{main}|}| must be followed by a whitespace.
It should not be followed immediately by another command
or by a comment mark `|%|'.
This is because the \TeX{} parser reads the token immediately following
the argument of |\childdocmain| and puts it
at the beginning of every child section;
however, a white\-space is ignored.
\end{itemize}

%%%%%%%%%%%%%%%%%%%%%%%%%%%%%%%%%%%%%%%%
\paragraph{Content of Main File.}

It is advisable to place all content in the child files included by |\include|.
Any output contained in the main file will appear in all child documents
unless suppressed manually;
it cannot be suppressed automatically by the |\includeonly| directive
and thus should normally be avoided.
A method to include some content in the main file
by means of conditional processing is described in \secref{sec:conditional}.

%%%%%%%%%%%%%%%%%%%%%%%%%%%%%%%%%%%%%%%%
\paragraph{Page Numbering.}

When only a part of the document is compiled,
the appropriate numbering of pages
(as well as other status parameters)
is determined from the |.aux| files.
The latter contain information from previous passes.
However this information needs to propagate through
all intermediate child documents.
Therefore the page numbering in child documents may well
be inconsistent until the complete document is compiled at least once.

A useful (if unconventional) way to always ensure a consistent
page numbering is to restart the numbering in each child document
and denote the pages by `\textit{child}|.|\textit{page}'
where \textit{child} represents the chapter/section number of the child file.
This can be achieved by the command
|\numberwithin{page}{|\textit{child}|}|
of the \textsf{amsmath} package
where \textit{child} can be |chapter| or |section|
depending on the chosen structuring.
Alternatively, one can modify the macro |\thepage| appropriately
and reset the counter |page| at the start of each child file.

%%%%%%%%%%%%%%%%%%%%%%%%%%%%%%%%%%%%%%%%%%%%%%%%%%%%%%%%%%%%%%%%%%%%%%%%%%%%%%%%
\subsection{Conditional Processing}
\label{sec:conditional}

The package provides a mechanism to compile different versions
of a document. To customise the versions further some conditional processing
can come in handy to distinguish which version is being compiled.
The package provides two macros to describe the compilation context:

%%%%%%%%%%%%%%%%%%%%%%%%%%%%%%%%%%%%%%%%
\DescribeMacro{\ifchilddoc}
The conditional |\ifchilddoc| distinguishes between the compilation of
child documents and the main document:
%
\begin{center}
|\ifchilddoc |\textit{child-code}| |[|\||else |\textit{main-code}]| \||fi|
\end{center}

%%%%%%%%%%%%%%%%%%%%%%%%%%%%%%%%%%%%%%%%
\DescribeMacro{\childdocname}
\DescribeMacro{\childdocjob}
The macro |\childdocname| contains the filename (without extension)
of the main or child file being processed.
Note that |\childdocjob| will always contain the name of the main file.

%%%%%%%%%%%%%%%%%%%%%%%%%%%%%%%%%%%%%%%%
\paragraph{Title Page.}

Conditional processing can be used to include a title or banner page
in the main document when proper precautions are taken.
Importantly, the code in the main file should ensure that the page counter
(as well as other status parameters which are stored in the |.aux| files)
takes the same value after the conditional processing.
Otherwise the page numbers may take divergent values
depending on which part is compiled.

For example, a title page could be declared by:
%
\begin{center}
\begin{tabular}{l}
|\ifchilddoc\||else|\\
|\addtocounter{page}{-1}|\\
\textit{code for title page}\\
|\newpage|\\
|\||fi|
\end{tabular}
\end{center}
%
A banner page for the child documents can be generated by:
%
\begin{center}
\begin{tabular}{l}
|\ifchilddoc|\\
|\addtocounter{page}{-1}|\\
\textit{code for banner page}\\
|\newpage|\\
|\||fi|
\end{tabular}
\end{center}
%
Here one could write a message such as:
\begin{center}
|This is the part \childdocname{} of \childdocjob{}.|
\end{center}

%%%%%%%%%%%%%%%%%%%%%%%%%%%%%%%%%%%%%%%%%%%%%%%%%%%%%%%%%%%%%%%%%%%%%%%%%%%%%%%%
\subsection{Flags}
\label{sec:flags}

The package makes it easy to generate different versions
of the main or child documents.
To this end compilation flags can be defined
and assigned different default values.
They will be particularly useful in conjunction
with the forwarding mechanism described in \secref{sec:forward}.

For example, it may be useful to have a flag |\version|
which can be set to |draft| or |final|.
The document source will contain some conditional code
depending on the value of |\version|.
Suppose further, the flag should default to |final| for the main file
and to |draft| for child files
which is a natural assignment for editing the document.
This is achieved by placing the following code
in the preamble of the main document
(below the |\childdocmain| directive):
%
\begin{center}
\begin{tabular}{l}
|\ifchilddoc|\\
|\providecommand{\version}{draft}|\\
|\||else|\\
|\providecommand{\version}{final}|\\
|\||fi|
\end{tabular}
\end{center}
%
The definition by |\providecommand| makes sure
that previous definitions are not overwritten.
Further statements |\providecommand{\version}{...}|
can thus be added before the above code to override it.

For the main file, one might add a line
(between |\childdocmain| and the above block)
%
\begin{center}
|%\ifchilddoc\||else\providecommand{\version}{draft}\||fi|
\end{center}
%
which can be uncommented to produce a draft version.
Likewise one can add a line to the very top of a child file
(above the |\childdocof{|\textit{main}|}| directive)
%
\begin{center}
|%\providecommand{\version}{final}|
\end{center}
%
which can be uncommented to produce the final version of this child document.

%%%%%%%%%%%%%%%%%%%%%%%%%%%%%%%%%%%%%%%%%%%%%%%%%%%%%%%%%%%%%%%%%%%%%%%%%%%%%%%%
\subsection{Forwarding}
\label{sec:forward}

Different versions of the main or child documents
using compilation flags as described in \secref{sec:flags}
can be (permanently) stored in different files
for convenient compilation, viewing and distribution.
To this end, the package defines a command
to pass on compilation to a different file:

%%%%%%%%%%%%%%%%%%%%%%%%%%%%%%%%%%%%%%%%
\DescribeMacro{\childdocforward}
The command |\childdocforward| redirects processing to
another source file:
%
\begin{center}
\begin{tabular}{l}
|\input{childdoc.def}|\\
|\childdocforward[|\textit{main}|]{|\textit{dest}|}|\\
\end{tabular}
\end{center}
%
The argument \textit{dest} is the destination file
(without extension).
It should be the main file or one of the child files.
Note that further \textsf{childdoc} directives
such as |\childdocof| and |\childdocforward|
in the indicated file will be processed in this form.
The optional argument \textit{main}
passes on directly to the main file \textit{main}
while pretending to compile the child \textit{dest}.
This form behaves as if \textit{dest}
issues |\childdocof{|\textit{main}|}| right away,
and no further \textsf{childdoc} directives will be processed.

%%%%%%%%%%%%%%%%%%%%%%%%%%%%%%%%%%%%%%%%
\DescribeMacro{\...prefix}
In the alternative form |\childdocforwardprefix|,
%
\begin{center}
\begin{tabular}{l}
|\input{childdoc.def}|\\
|\childdocforwardprefix[|\textit{main}|]{|\textit{prefix}|}{|\textit{dest}|}|
\end{tabular}
\end{center}
%
the destination file is determined by a pattern
depending on the current file:
To make this work, the current file must be called
`{\textit{prefix}\hspace{0.2em}\textit{suffix}}'
with \textit{prefix} matching precisely the argument.
Processing is then passed on to the file
`{\textit{dest}\hspace{0.2em}\textit{suffix}}'.
Surely, the same effect is achieved by
directly specifying the
argument `{\textit{dest}\hspace{0.2em}\textit{suffix}}'
in the first form.
However, that requires to set up a different file
for each child. With the alternative form of the command
all these files can have exactly the same content
which simplifies setting them up and maintaining them.

For example, the following file |draft.tex|
with a compilation flag |\version| as described in \secref{sec:flags}
compiles the main document as a draft:
%
\begin{center}
\begin{tabular}{l}
|\def\version{draft}|\\
|\input{childdoc.def}|\\
|\childdocforward{|\textit{main}|}|
\end{tabular}
\end{center}
%
Likewise, the following files |final|\textit{nn}|.tex|
compile the final version of the child document
|child|\textit{nn}|.tex|:
%
\begin{center}
\begin{tabular}{l}
|\def\version{final}|\\
|\input{childdoc.def}|\\
|\childdocforwardprefix{final}{child}|
\end{tabular}
\end{center}
%

Note that when several versions of a main file and/or of each child file
are to be generated, it may be convenient to set up a |Makefile| or
shell script to automatise the process.

%%%%%%%%%%%%%%%%%%%%%%%%%%%%%%%%%%%%%%%%%%%%%%%%%%%%%%%%%%%%%%%%%%%%%%%%%%%%%%%%
\subsection{Command Line Processing}
\label{sec:commandline}

The effect of redirection files can also be achieved by invoking
the \LaTeX{} compiler with a more elaborate command line.
Most conveniently this should be done as part
of a shell script or a |Makefile|.

When using \textsf{childdoc} in the main file, the following
command lines effectively perform a redirection
(note that depending on the shell being used,
backslashes may have to be doubled: `|\|' $\to$ `|\\|'):
%
\begin{center}
|... -jobname "|\textit{target}|" |\\|"|[\textit{flags}]%
|\input{childdoc.def}\childdocforward[|\textit{main}|]{|\textit{dest}|}"|
\end{center}
%
Here \textit{target} is the name of the output file,
\textit{main} is the name of the main file
and \textit{dest} is the name of the main or child file to be processed
(all filenames without extensions).
The optional argument \textit{main} can be omitted
if \textit{main} matches \textit{dest}.
Optionally, compilation \textit{flags} can be defined via |\def| commands.
This command line makes the \TeX{} engine believe
it is compiling the file \textit{target}
whose content is specified as the latter parameter.
The provided code then forwards the processing to
\textit{main} or \textit{dest} as described in \secref{sec:forward}.

%%%%%%%%%%%%%%%%%%%%%%%%%%%%%%%%%%%%%%%%%%%%%%%%%%%%%%%%%%%%%%%%%%%%%%%%%%%%%%%%
\subsection{Include by Input}
\label{sec:input}

Including child documents by |\include| has some restrictions by design.
Most notably, the content of a child document always occupies
its own set of pages; pages cannot be shared between child documents.
Usually, this behaviour makes perfect sense
because each child document contain an essential part of the document.
However, in some situations it may be desirable to compose
a document from a collection of parts
without having mandatory page breaks between then.
For this case, the package
provides a mechanism to include parts
by |\input| which can also be processed individually.
However, by construction this mechanism
requires manual handling of the content to be output.

%%%%%%%%%%%%%%%%%%%%%%%%%%%%%%%%%%%%%%%%
\DescribeMacro{\ifchilddocmanual}
The main file should be prepared as usual, see \secref{sec:include}.
However, the document body must make a distinction
between processing of an individual part and of the main document, e.g.:
%
\begin{center}
\begin{tabular}{l}
|\ifchilddocmanual|\\
|\input{\childdocname}|\\
|\||else|\\
\textit{document body with }|\input{|\textit{part}|}|\\
|\||fi|
\end{tabular}
\end{center}
%
The conditional |\ifchilddocmanual| is true whenever
a part to be included by |\input| is being compiled,
and the name of the part is stored in |\childdocname|.

%%%%%%%%%%%%%%%%%%%%%%%%%%%%%%%%%%%%%%%%
\DescribeMacro{\childdocby}
Each part to be included by |\input| should start with:
%
\begin{center}
\begin{tabular}{l}
|\input{childdoc.def}|\\
|\childdocby{|\textit{main}|}|\\
\end{tabular}
\end{center}
%
The directive |\childdocby| is similar to |\childdocof|
described in \secref{sec:include},
but the subsequent selection of content must be done manually.
To that end, both |\ifchilddoc| and |\ifchilddocmanual|
will be true upon processing of a part,
and the name of the part is stored in |\childdocname|.
Note that |\jobname| will be set to the filename of the current part
so that each part receives an individual |.aux| file
that does not interfere with the |.aux| file(s) of the main document.
This behaviour can be altered by the alternative form
|\childdocby[*]{|\textit{main}|}| (with a non-empty optional argument)
which uses the |.aux| file of the main document
by setting |\jobname| to \textit{main}.

%%%%%%%%%%%%%%%%%%%%%%%%%%%%%%%%%%%%%%%%%%%%%%%%%%%%%%%%%%%%%%%%%%%%%%%%%%%%%%%%
\subsection{Driver Development}
\label{sec:driver}

The \textsf{childdoc} mechanism can also be use for the development
of definition files such as \LaTeX{} styles or classes.
This case differs from the above setup with multiple parts
included by |\include| in that no |\includeonly| should be invoked.
This can be achieved by starting the include file
(before |\ProvidesPackage|) with:
%
\begin{center}
\begin{tabular}{l}
|\input{childdoc.def}|\\
|\childdocforward{|\textit{main}|}|\\
\end{tabular}
\end{center}
%
or alternatively with:
%
\begin{center}
\begin{tabular}{l}
|\input{childdoc.def}|\\
|\childdocby{|\textit{main}|}|\\
\end{tabular}
\end{center}
%
Both forms have slightly different effects as described above.
The main file is prepared as usual, see \secref{sec:include}.

%%%%%%%%%%%%%%%%%%%%%%%%%%%%%%%%%%%%%%%%%%%%%%%%%%%%%%%%%%%%%%%%%%%%%%%%%%%%%%%%
\subsection{Legacy Detection}
\label{sec:detection}

The directive |\childdocmain| in the main file can detect
whether the complete document or merely a child is to be compiled
even without using the directive |\childdocof|.
This method is deprecated because it is less robust
and there is no compelling reason to use it;
it is merely provided for backward compatibility
and it may be removed in future versions.

If the detection mechanism is to be used,
it is mandatory to correctly specify
the filename of the main file as the argument of |\childdocmain|:
%
\begin{center}
\begin{tabular}{l}
|\input{childdoc.def}|\\
|\childdocmain{|\textit{main}|}|\\
\end{tabular}
\end{center}
%
If |\jobname| does not match the argument \textit{main} of |\childdocmain|,
it is assumed that |\jobname| points to the child file to be compiled.
When using |\childdocmain| with the main file specified as argument,
it suffices to start a child file
with just |\input{|\textit{main}|}|
without loading of the package and using |\childdocof|.
If instead all processing is done
with the appropriate \textsf{childdoc} directives,
the argument of \textit{main} of |\childdocmain| can be empty.

An alternative version of the command line processing described
in \secref{sec:commandline} using the detection mechanism reads:
%
\begin{center}
|... -jobname "|\textit{target}|" "|[\textit{flags}]%
[|\def\jobname{|\textit{dest}|}|]|\input{|\textit{main}|}"|
\end{center}

%%%%%%%%%%%%%%%%%%%%%%%%%%%%%%%%%%%%%%%%%%%%%%%%%%%%%%%%%%%%%%%%%%%%%%%%%%%%%%%%
\subsection{Manual Code}
\label{sec:manual}

In case one cannot be certain whether the definitions file |childdoc.def|
is installed on the target \TeX{} distribution
and one prefers not to ship it,
it is conceivable to paste a few relevant commands into the sources.

To that end, drop all statements |\input{childdoc.def}|
and perform the replacements as outlined below.
Instead of |\childdocmain{|\textit{main}|}| add the following code
to the top of the main file:
%
\begin{center}
\begin{tabular}{l}
|\||ifdefined\childdocname\endinput\||fi\newif\ifchilddoc|\\
|\edef\childdocname{\scantokens\expandafter{\jobname\noexpand}}|\\
|\def\childdocmain{|\textit{main}|}\||ifx\childdocmain\childdocname\||else|\\
|\childdoctrue\includeonly{\childdocname}\let\jobname\childdocmain\||fi|\\
\end{tabular}
\end{center}
%
Instead of |\childdocof{|\textit{main}|}| just include the main file
at the top of each child file:
%
\begin{center}
|\input{|\textit{main}|}|
\end{center}
%
A simple redirection |\childdocforward{|\textit{dest}|}| is achieved by:
%
\begin{center}
|\def\jobname{|\textit{dest}|}\input{\jobname}|
\end{center}
%
The redirection with prefix
|\childdocforwardprefix[|\textit{prefix}|]{|\textit{dest}|}|
is accomplished by:
%
\begin{center}
\begin{tabular}{l}
|{\edef\jobname{\scantokens\expandafter{\jobname\noexpand}}|\\
|\def\redirectjob |\textit{prefix}|#1~~~{\gdef\jobname{|\textit{dest}|#1}}|\\
|\expandafter\redirectjob\jobname~~~}\input{\jobname}|
\end{tabular}
\end{center}

In an alternative approach,
child documents can be compiled by a specific command line
without additional code or specific definitions:
%
\begin{center}
|... -jobname "|\textit{target}|" "|[\textit{flags}]%
|\includeonly{|\textit{dest}|}\input{|\textit{main}|}"|
\end{center}
%

%%%%%%%%%%%%%%%%%%%%%%%%%%%%%%%%%%%%%%%%%%%%%%%%%%%%%%%%%%%%%%%%%%%%%%%%%%%%%%%%
%%%%%%%%%%%%%%%%%%%%%%%%%%%%%%%%%%%%%%%%%%%%%%%%%%%%%%%%%%%%%%%%%%%%%%%%%%%%%%%%
\section{Information}

%%%%%%%%%%%%%%%%%%%%%%%%%%%%%%%%%%%%%%%%%%%%%%%%%%%%%%%%%%%%%%%%%%%%%%%%%%%%%%%%
\subsection{Copyright}

Copyright \copyright{} 2017--2018 Niklas Beisert

This work may be distributed and/or modified under the
conditions of the \LaTeX{} Project Public License, either version 1.3
of this license or (at your option) any later version.
The latest version of this license is in
  \url{http://www.latex-project.org/lppl.txt}
and version 1.3 or later is part of all distributions of \LaTeX{}
version 2005/12/01 or later.

This work has the LPPL maintenance status `maintained'.

The Current Maintainer of this work is Niklas Beisert.

This work consists of the files |README.txt|, |childdoc.ins| and |childdoc.dtx|
as well as the derived files |childdoc.def|, |cdocsamp.tex|
with |cdocsch1.tex|, |cdocsch2.tex|, |cdocspt3.tex|, |cdocspt4.tex|,
|cdocsdrf.tex|, |cdocsfn1.tex|, |cdocsfn2.tex|
as well as |childdoc.pdf|.

%%%%%%%%%%%%%%%%%%%%%%%%%%%%%%%%%%%%%%%%%%%%%%%%%%%%%%%%%%%%%%%%%%%%%%%%%%%%%%%%
\subsection{Files and Installation}

The package consists of the files:
%
\begin{center}
\begin{tabular}{ll}
    |README.txt|   & readme file \\
    |childdoc.ins| & installation file \\
    |childdoc.dtx| & source file \\
    |childdoc.def| & definition file \\
    |cdocsamp.tex| & sample main file \\
    |cdocsch1.tex| & sample include file \\
    |cdocsch2.tex| & sample include file \\
    |cdocspt3.tex| & sample part file \\
    |cdocspt4.tex| & sample part file \\
    |cdocsdrf.tex| & sample redirection file \\
    |cdocsfn1.tex| & sample redirection file \\
    |cdocsfn2.tex| & sample redirection file \\
    |childdoc.pdf| & manual
\end{tabular}
\end{center}
%
The distribution consists of the files
|README.txt|, |childdoc.ins| and |childdoc.dtx|.
%
\begin{itemize}
\item
Run (pdf)\LaTeX{} on |childdoc.dtx|
to compile the manual |childdoc.pdf| (this file).
\item
Run \LaTeX{} on |childdoc.ins| to create the definitions file |childdoc.def|
and the sample |cdocsamp.tex| with include files
|cdocsch1.tex|, |cdocsch2.tex|, |cdocspt3.tex|, |cdocspt4.tex|,
|cdocsdrf.tex|, |cdocsfn1.tex|, |cdocsfn2.tex|.
Then copy the file |childdoc.def| to an appropriate directory of your \LaTeX{}
distribution, e.g.\ \textit{texmf-root}|/tex/latex/childdoc|.
\end{itemize}

%%%%%%%%%%%%%%%%%%%%%%%%%%%%%%%%%%%%%%%%%%%%%%%%%%%%%%%%%%%%%%%%%%%%%%%%%%%%%%%%
\subsection{Related CTAN Packages}

There are several other packages which offer a similar functionality:
%
\begin{itemize}
\item
The packages
\href{http://ctan.org/pkg/docmute}{\textsf{docmute}},
\href{http://ctan.org/pkg/includex}{\textsf{includex}} and
\href{http://ctan.org/pkg/standalone}{\textsf{standalone}}
provide commands to include only the document body of
a child file thus allowing both files to be compiled individually.
\item
The packages \href{http://ctan.org/pkg/subdocs}{\textsf{subdocs}}
and \href{http://ctan.org/pkg/subfiles}{\textsf{subfiles}}
provide structures in which the main and child documents can be
encapsulated and allowing them to be compiled individually.
The inclusion mechanism is different from the conventional |\include|.
\item
The package \href{http://ctan.org/pkg/combine}{\textsf{combine}}
is an elaborate solution to combine several documents into one.
\end{itemize}
%
See also the CTAN topic \href{http://ctan.org/topic/subdocs}{\textsf{subdocs}}
for further related packages.
The present package differs from the above solutions in that
a document structure constructed with the conventional |\include| mechanism
just needs two extra commands at the top of every file
such that all constituent files can be compiled individually.

%%%%%%%%%%%%%%%%%%%%%%%%%%%%%%%%%%%%%%%%%%%%%%%%%%%%%%%%%%%%%%%%%%%%%%%%%%%%%%%%
%\subsection{Feature Suggestions}
%
%The following is a list of features which may be useful for future
%versions of this package:
%%
%\begin{itemize}
%\item
%\ldots
%\end{itemize}

%%%%%%%%%%%%%%%%%%%%%%%%%%%%%%%%%%%%%%%%%%%%%%%%%%%%%%%%%%%%%%%%%%%%%%%%%%%%%%%%
\subsection{Revision History}

%%%%%%%%%%%%%%%%%%%%%%%%%%%%%%%%%%%%%%%%
\paragraph{v2.0:} 2018/12/30

\begin{itemize}
\item
immediate forward processing
\item
added |\childdocby| mechanism
\item
manual restructured
\end{itemize}

%%%%%%%%%%%%%%%%%%%%%%%%%%%%%%%%%%%%%%%%
\paragraph{v1.6:} 2018/01/17

\begin{itemize}
\item
application for development of include files
\item
corrections to manual
\end{itemize}

%%%%%%%%%%%%%%%%%%%%%%%%%%%%%%%%%%%%%%%%
\paragraph{v1.5:} 2017/05/21

\begin{itemize}
\item
more complete structuring introduced
\item
|\childdocof| introduced
\item
|\childdoc| renamed to |\childdocmain|
\item
|\childredirect| renamed to |\childdocforward| and |\childdocforwardprefix|
and functionality expanded
\end{itemize}

%%%%%%%%%%%%%%%%%%%%%%%%%%%%%%%%%%%%%%%%
\paragraph{v1.0:} 2017/04/27

\begin{itemize}
\item
manual and install package
\item
first version published on CTAN
\end{itemize}

%%%%%%%%%%%%%%%%%%%%%%%%%%%%%%%%%%%%%%%%
\paragraph{v0.6:} 2017/04/26

\begin{itemize}
\item
redirection mechanism added
\end{itemize}

%%%%%%%%%%%%%%%%%%%%%%%%%%%%%%%%%%%%%%%%
\paragraph{v0.5:} 2017/04/26

\begin{itemize}
\item
functionality in definition file
\end{itemize}


%%%%%%%%%%%%%%%%%%%%%%%%%%%%%%%%%%%%%%%%%%%%%%%%%%%%%%%%%%%%%%%%%%%%%%%%%%%%%%%%
%%%%%%%%%%%%%%%%%%%%%%%%%%%%%%%%%%%%%%%%%%%%%%%%%%%%%%%%%%%%%%%%%%%%%%%%%%%%%%%%
%%%%%%%%%%%%%%%%%%%%%%%%%%%%%%%%%%%%%%%%%%%%%%%%%%%%%%%%%%%%%%%%%%%%%%%%%%%%%%%%
\appendix

\settowidth\MacroIndent{\rmfamily\scriptsize 000\ }

 \DocInput{childdoc.dtx}

\end{document}
%</driver>
% \fi
%
% %%%%%%%%%%%%%%%%%%%%%%%%%%%%%%%%%%%%%%%%%%%%%%%%%%%%%%%%%%%%%%%%%%%%%%%%%%%%%%
% %%%%%%%%%%%%%%%%%%%%%%%%%%%%%%%%%%%%%%%%%%%%%%%%%%%%%%%%%%%%%%%%%%%%%%%%%%%%%%
% \section{Sample}
%\iffalse
%<*samplemain>
%\fi
%
% The following presents a sample document
% with two chapters, two parts, a title page,
% a compile flag as well as three forwarding files to set the flag.
% It consists of eight |.tex| files:
% \begin{center}
% \begin{tabular}{ll}
% |cdocsamp.tex|&main file\\
% |cdocsch1.tex|&include file for chapter 1\\
% |cdocsch2.tex|&include file for chapter 2\\
% |cdocspt3.tex|&include file for part 3\\
% |cdocspt4.tex|&include file for part 4\\
% |cdocsdrf.tex|&forwarding file for main file in draft mode\\
% |cdocsfi1.tex|&forwarding file for final version of chapter 1\\
% |cdocsfi2.tex|&forwarding file for final version of chapter 2\\
% \end{tabular}
% \end{center}
% Each of the eight files can be compiled directly by the \LaTeX{} compiler.
%
% %%%%%%%%%%%%%%%%%%%%%%%%%%%%%%%%%%%%%%
% \paragraph{Main File.}
%
% The main file is called |cdocsamp.tex|.
%
% Load the \textsf{childdoc} definitions and
% declare the filename for the main document:
%    \begin{macrocode}
\input{childdoc.def}
\childdocmain{}
%    \end{macrocode}

% Optional override for |\version| flag:
%    \begin{macrocode}
%%\ifchilddoc\else\providecommand{\version}{draft}\fi
%    \end{macrocode}

% Define the default values for the |\version| flag
% (|final| for the main file and |draft| for childs):
%    \begin{macrocode}
\ifchilddoc
\providecommand{\version}{draft}
\else
\providecommand{\version}{final}
\fi
%    \end{macrocode}

% Load the standard document class:
%    \begin{macrocode}
\documentclass[12pt]{article}
%    \end{macrocode}

% Start the document body:
%    \begin{macrocode}
\begin{document}
%    \end{macrocode}

% Declare a title page.
% Print title, part of document being processed and version flag:
%    \begin{macrocode}
\addtocounter{page}{-1}
\begin{center}
{\LARGE\bfseries{}childdoc example\par}
\vspace{1cm}
\ifchilddoc
\ifchilddocmanual part\else chapter\fi:
`\childdocname' of `\childdocjob'\par
\else
main document: `\childdocjob'\par
\fi
version: \version\par
\end{center}
\newpage
%    \end{macrocode}

% Manually include selected file,
% otherwise process as usual:
%    \begin{macrocode}
\ifchilddocmanual
\section*{part `\childdocname'}
\input{\childdocname}
\else
%    \end{macrocode}

% Include the two chapters:
%    \begin{macrocode}
\include{cdocsch1}
\include{cdocsch2}
%    \end{macrocode}

% Include the two parts unless only chapters should be displayed:
%    \begin{macrocode}
\ifchilddoc\else
\section{part three}
\input{cdocspt3}
\section{part four}
\input{cdocspt4}
\fi
%    \end{macrocode}

% Process as usual until here:
%    \begin{macrocode}
\fi
%    \end{macrocode}

% End of document body:
%    \begin{macrocode}
\end{document}
%    \end{macrocode}
%\iffalse
%</samplemain>
%\fi
%
% %%%%%%%%%%%%%%%%%%%%%%%%%%%%%%%%%%%%%%
% \paragraph{Chapter Include Files.}
%
% The include files are called |cdocsch1.tex| and |cdocsch2.tex|.
%
%\iffalse
%<*samplechap1|samplechap2>
%\fi

% Optional override for |\version| flag:
%    \begin{macrocode}
%%\providecommand{\version}{final}
%    \end{macrocode}

% Include the main document:
%    \begin{macrocode}
\input{childdoc.def}
\childdocof{cdocsamp}
%    \end{macrocode}

%\iffalse
%</samplechap1|samplechap2>
%\fi
%
%\iffalse
%<*samplechap1>
%\fi
% Some text for chapter 1:
%    \begin{macrocode}
\section{one}
some text in chapter one
%    \end{macrocode}

%\iffalse
%</samplechap1>
%\fi
% Some text for chapter 2:
%\iffalse
%<*samplechap2>
%\fi
%    \begin{macrocode}
\section{two}
more text in chapter two
%    \end{macrocode}

%\iffalse
%</samplechap2>
%\fi
%
% %%%%%%%%%%%%%%%%%%%%%%%%%%%%%%%%%%%%%%
% \paragraph{Part Include Files.}
%
% The include files are called |cdocspt3.tex| and |cdocspt4.tex|.
%
%\iffalse
%<*samplepart3|samplepart4>
%\fi

% Optional override for |\version| flag:
%    \begin{macrocode}
%%\providecommand{\version}{final}
%    \end{macrocode}

% Include the main document:
%    \begin{macrocode}
\input{childdoc.def}
\childdocby{cdocsamp}
%    \end{macrocode}

%\iffalse
%</samplepart3|samplepart4>
%\fi
%
%\iffalse
%<*samplepart3>
%\fi
% Some text for part 3:
%    \begin{macrocode}
some text in part three
%    \end{macrocode}

%\iffalse
%</samplepart3>
%\fi
% Some text for part 4:
%\iffalse
%<*samplepart4>
%\fi
%    \begin{macrocode}
more text in part four
%    \end{macrocode}

%\iffalse
%</samplepart4>
%\fi
%
% %%%%%%%%%%%%%%%%%%%%%%%%%%%%%%%%%%%%%%
% \paragraph{Forwarding for a Complete Draft.}
%
% The following forwarding file |cdocsdrf.tex|
% compiles the main document in draft mode:
%\iffalse
%<*sampledraft>
%\fi
%    \begin{macrocode}
\def\version{draft}
\input{childdoc.def}
\childdocforward{cdocsamp}
%    \end{macrocode}

%\iffalse
%</sampledraft>
%\fi
%
% %%%%%%%%%%%%%%%%%%%%%%%%%%%%%%%%%%%%%%
% \paragraph{Forwarding for Final Version of the Chapters.}
%
% The following forwarding files |cdocsfn1.tex| and |cdocsfn2.tex|
% (with identical content)
% compile the final versions of the child documents
% |cdocsch1.tex| and |cdocsch2.tex|, respectively:
%\iffalse
%<*samplefinal>
%\fi
%    \begin{macrocode}
\def\version{final}
\input{childdoc.def}
\childdocforwardprefix[cdocsamp]{cdocsfn}{cdocsch}
%    \end{macrocode}

%\iffalse
%</samplefinal>
%\fi
%
% %%%%%%%%%%%%%%%%%%%%%%%%%%%%%%%%%%%%%%
% \paragraph{Command Line Processing.}
%
% The following three command lines generate the output files
% |cdocscld|, |cdocscl1| and |cdocscl2|
% which should be identical to
% |cdocsdrf|, |cdocsch1| and |cdocsfn2|, respectively:
% \begin{center}
% \begin{tabular}{l}
% |latex -jobname cdocscld \|\\
% |  "\def\version{draft}\input{childdoc.def}\childdocforward{cdocsamp}"|\\
% |latex -jobname cdocscl1 \|\\
% |  "\input{childdoc.def}\childdocforward[cdocsamp]{cdocsch1}"|\\
% |latex -jobname cdocscl2 \|\\
% |  "\def\version{final}\input{childdoc.def}\childdocforward{cdocsch2}"|
% \end{tabular}
% \end{center}
% Note that the trailing backslash on each first line
% merely continues the input to the second line
% (for convenient cut ant paste).
% Furthermore, the command |latex| can be replaced by any
% of its alternative versions such as |pdflatex|.
%
% %%%%%%%%%%%%%%%%%%%%%%%%%%%%%%%%%%%%%%%%%%%%%%%%%%%%%%%%%%%%%%%%%%%%%%%%%%%%%%
% %%%%%%%%%%%%%%%%%%%%%%%%%%%%%%%%%%%%%%%%%%%%%%%%%%%%%%%%%%%%%%%%%%%%%%%%%%%%%%
% \section{Implementation}
%\iffalse
%<*package>
%\fi
%
% This section describes the definitions file |childdoc.def|.

% The definitions cannot be loaded using |\usepackage| or |\RequirePackage|
% which has a mechanism to prevent loading a style file more than once.
% When loading the definitions by means of |\input|
% multiple instances have to be prevented manually:
%\iffalse
%This code needs to be before the `\ProvidesFile' directive
%which is defined at the beginning of this file.
%Therefore it is also placed there and commented out here.
%</package>
%<*discard>
%\fi
%    \begin{macrocode}
\ifdefined\childdocmain\endinput\fi
%    \end{macrocode}
%\iffalse
%</discard>
%<*package>
%\fi
%
% \macro{\ifchilddoc}
% \macro{\ifchilddocmanual}
% The conditional |\ifchilddoc| tells whether a
% child (true) or main (false) document is being compiled.
% The conditional |\ifchilddocmanual| tells whether
% the |\includeonly| mechanism is used (false) or
% the selection of child files must be performed manually (true).
% The definitions initialise to false:
%    \begin{macrocode}
\newif\ifchilddoc
\newif\ifchilddocmanual
%    \end{macrocode}

% \macro{\childdocname}
% \macro{\childdocjob}
% The macro |\childdocname| stores the name of the main document
% to be compiled. The macro |\childdocjob| stores the name of
% the document on which the \LaTeX{} compiler was originally invoked.
% The content of |\jobname| cannot be compared
% to filenames specified in the source due to different catcodes.
% The following code rescans |\jobname|, stores the result
% in |\childdocname| and saves a copy in |\childdocjob|:
%    \begin{macrocode}
\edef\childdocname{\scantokens\expandafter{\jobname\noexpand}}
\let\childdocjob\childdocname
%    \end{macrocode}

% \macro{\childdocdisable}
% The macro |\childdocdisable| prevents the main file
% from being processed more than once.
% At this stage, the main document command |\childdocmain|
% is assumed to be called once again where it should do nothing.
% Any subsequent call to it should prevent
% a secondary processing of the main document
% It overwrites the forwarding commands
% |\childdocof| and |\childdocforward|
% with empty macros to prevent further inclusions of the main document:
%    \begin{macrocode}
\newcommand{\childdocdisable}
{
  \renewcommand{\childdocmain}[1]{\renewcommand{\childdocmain}[1]{\endinput}}
  \renewcommand{\childdocof}[1]{}
  \renewcommand{\childdocby}[2][]{}
  \renewcommand{\childdocforward}[2][]{}
  \renewcommand{\childdocdisable}{}
}
%    \end{macrocode}

% \macro{\childdocmain}
% The macro |\childdocmain| is to be called at the top of the main file
% with nothing or the main filename (without extension) as argument.
% First, it breaks loops.
% If the argument is not empty and does not match |\childdocname|
% (which is set by the first inclusion of |childdoc.def|),
% |\ifchilddoc| is set to true, |\includeonly| is applied to the child file
% and |\jobname| is set to the main file
% (for proper handling of |.aux| files):
%    \begin{macrocode}
\newcommand{\childdocmain}[1]
{
  \childdocdisable\childdocmain{}
  \if?#1?\else
    \begingroup
      \def\childdoctmp{#1}
      \ifx\childdoctmp\childdocname
        \def\childdoctmp{}
      \else
        \def\childdoctmp
        {
          \childdoctrue
          \includeonly{\childdocname}
          \def\childdocjob{#1}
          \def\jobname{#1}
        }
      \fi
      \expandafter
    \endgroup
    \childdoctmp
  \fi
}
%    \end{macrocode}

% \macro{\childdocof}
% The command |\childdocof| redirects
% compilation to the main file |#1|.
%    \begin{macrocode}
\newcommand{\childdocof}[1]
{
  \childdocdisable
  \childdoctrue
  \includeonly{\childdocname}
  \def\jobname{#1}
  \def\childdocjob{#1}
  \input{#1}
}
%    \end{macrocode}

% \macro{\childdocby}
% The command |\childdocby| ....
%    \begin{macrocode}
\newcommand{\childdocby}[2][]
{
  \childdocdisable
  \childdoctrue
  \childdocmanualtrue
  \if?#1?\else
    \def\jobname{#2}
  \fi
  \def\childdocjob{#2}
  \input{#2}
  \endinput
}
%    \end{macrocode}

% \macro{\childdocforward}
% The command |\childdocforward| redirects
% compilation to the main file or
% (if the optional argument is given) a child file.
% Parameters are set as if the main file
% or a child file starting with |\childdocof| was compiled.
% Then compilation is handed over to the main file:
%    \begin{macrocode}
\newcommand{\childdocforward}[2][]
{
  \begingroup
    \if?#1?
      \def\childdoctmp
      {
        \def\childdocname{#2}
        \def\childdocjob{#2}
        \def\jobname{#2}
        \input{#2}
        \endinput
      }
    \else
      \def\childdoctmp
      {
        \childdocdisable
        \def\childdocname{#2}
        \childdoctrue
        \includeonly{#2}
        \def\childdocjob{#1}
        \def\jobname{#1}
        \input{#1}
        \endinput
      }
    \fi
    \expandafter
  \endgroup
  \childdoctmp
}
%    \end{macrocode}

% \macro{\childdocforwardprefix}
% The command |\childdocforwardprefix| redirects
% compilation to the main or a child file by means of a pattern.
% The prefix |#1| in the current filename is replaced by |#2|
% and the suffix of the current filename is kept
% (it is assumed that the filename does not contain the substring `|~~~|'
% which is used as a delimiter).
% Compilation is handed over to the new file by |\childdocforward|:
%    \begin{macrocode}
\newcommand{\childdocforwardprefix}[3][]
{
  \begingroup
    \def\childdocextract #2##1~~~{\def\childdoctmp{\childdocforward[#1]{#3##1}}}
    \expandafter\childdocextract\childdocname~~~
    \expandafter
  \endgroup
  \childdoctmp
}
%    \end{macrocode}

% \macro{\childdoc}
% The deprecated macro |\childdoc| is a legacy version of |\childdocmain|:
%    \begin{macrocode}
\newcommand{\childdoc}{\childdocmain}
%    \end{macrocode}

% \macro{\childdocredirect}
% The deprecated macro |\childdocredirect| is a legacy version
% of |\childdocforward| and |\childdocforwardprefix|:
%    \begin{macrocode}
\newcommand{\childdocredirect}[2][]
{
  \begingroup
    \if?#1?
      \def\childdoctmp{\childdocforward{#2}}
    \else
      \def\childdoctmp{\childdocforwardprefix{#1}{#2}}
    \fi
    \expandafter
  \endgroup
  \childdoctmp
}
%    \end{macrocode}

%\iffalse
%</package>
%\fi
%
\endinput

\childdocforwardprefix[cdocsamp]{cdocsfn}{cdocsch}
%    \end{macrocode}

%\iffalse
%</samplefinal>
%\fi
%
% %%%%%%%%%%%%%%%%%%%%%%%%%%%%%%%%%%%%%%
% \paragraph{Command Line Processing.}
%
% The following three command lines generate the output files
% |cdocscld|, |cdocscl1| and |cdocscl2|
% which should be identical to
% |cdocsdrf|, |cdocsch1| and |cdocsfn2|, respectively:
% \begin{center}
% \begin{tabular}{l}
% |latex -jobname cdocscld \|\\
% |  "\def\version{draft}% \iffalse
%
% childdoc.dtx Copyright (C) 2017-2018 Niklas Beisert
%
% This work may be distributed and/or modified under the
% conditions of the LaTeX Project Public License, either version 1.3
% of this license or (at your option) any later version.
% The latest version of this license is in
%   http://www.latex-project.org/lppl.txt
% and version 1.3 or later is part of all distributions of LaTeX
% version 2005/12/01 or later.
%
% This work has the LPPL maintenance status `maintained'.
%
% The Current Maintainer of this work is Niklas Beisert.
%
% This work consists of the files childdoc.dtx and childdoc.ins
% and the derived files childdoc.def and cdocsamp.tex with
% cdocsch1.tex, cdocsch2.tex, cdocsdrf.tex, cdocsfn1.tex, cdocsfn2.tex.
%
%<package>\ifdefined\childdocmain\endinput\fi
%<package>\ProvidesFile{childdoc.def}[2018/12/30 v2.0 child document driver]
%<samplemain>\ProvidesFile{cdocsamp.tex}[2018/12/30 v2.0 sample for childdoc]
%<*driver>
%\ProvidesFile{childdoc.drv}[2018/12/30 v2.0 childdoc reference manual file]
\PassOptionsToClass{10pt,a4paper}{article}
\documentclass{ltxdoc}

\usepackage[margin=35mm]{geometry}
\usepackage{hyperref}
\usepackage{hyperxmp}
\usepackage[usenames]{color}

\hypersetup{colorlinks=true}
\hypersetup{pdfstartview=FitH}
\hypersetup{pdfpagemode=UseNone}
\hypersetup{pdfsource={}}
\hypersetup{pdflang={en-UK}}
\hypersetup{pdfcopyright={Copyright 2017-2018 Niklas Beisert.
  This work may be distributed and/or modified under the
  conditions of the LaTeX Project Public License, either version 1.3
  of this license or (at your option) any later version.}}
\hypersetup{pdflicenseurl={http://www.latex-project.org/lppl.txt}}
\hypersetup{pdfcontactaddress={ETH Zurich, ITP, HIT K,
  Wolfgang-Pauli-Strasse 27}}
\hypersetup{pdfcontactpostcode={8093}}
\hypersetup{pdfcontactcity={Zurich}}
\hypersetup{pdfcontactcountry={Switzerland}}
\hypersetup{pdfcontactemail={nbeisert@itp.phys.ethz.ch}}
\hypersetup{pdfcontacturl={http://people.phys.ethz.ch/\xmptilde nbeisert/}}

\newcommand{\secref}[1]{\hyperref[#1]{section \ref*{#1}}}

\parskip1ex
\parindent0pt
\let\olditemize\itemize
\def\itemize{\olditemize\parskip0pt}

\begin{document}

\title{The \textsf{childdoc} Package}
\hypersetup{pdftitle={The childdoc Package}}
\author{Niklas Beisert\\[2ex]
  Institut f\"ur Theoretische Physik\\
  Eidgen\"ossische Technische Hochschule Z\"urich\\
  Wolfgang-Pauli-Strasse 27, 8093 Z\"urich, Switzerland\\[1ex]
  \href{mailto:nbeisert@itp.phys.ethz.ch}
  {\texttt{nbeisert@itp.phys.ethz.ch}}}
\hypersetup{pdfauthor={Niklas Beisert}}
\hypersetup{pdfsubject={Manual for the LaTeX2e Package childdoc}}
\date{30 December 2018, \textsf{v2.0}}
\maketitle

\begin{abstract}\noindent
\textsf{childdoc} is a \LaTeXe{} package
that enables the direct compilation
of document sections included by |\include|
to individual files.
\end{abstract}

\begingroup
\parskip0ex
\tableofcontents
\endgroup

%%%%%%%%%%%%%%%%%%%%%%%%%%%%%%%%%%%%%%%%%%%%%%%%%%%%%%%%%%%%%%%%%%%%%%%%%%%%%%%%
%%%%%%%%%%%%%%%%%%%%%%%%%%%%%%%%%%%%%%%%%%%%%%%%%%%%%%%%%%%%%%%%%%%%%%%%%%%%%%%%
\section{Introduction}

\LaTeX{} provides a mechanism to structure a large document (such as a book)
into a main file and several child files (containing the chapters)
using the |\include| command.
This mechanism is beneficial for documents
which span hundreds of pages in order to
make the source file(s) more manageable.
Moreover, compilation can be restricted to
selected child files by means of the |\includeonly| command.
The latter feature can be used to reduce the compilation time while editing
(this was significantly more useful in the earlier days of \LaTeX{})
or to generate a smaller document which is easier to navigate.
Another application of |\includeonly| is to generate
documents consisting of selected parts of the complete document.

However, there are a few drawbacks of the plain |\include| mechanism:
\begin{itemize}
\item
The child files cannot be compiled on their own,
they can only be compiled via the main file.
A naive editing environment
(such as a text editor with an option
to have the current file processed by \LaTeX)
may require one to switch to the main file before compiling;
attempting to compile the child file produces errors.
\item
The main file must be modified (each time)
to adjust the |\includeonly| command
to the present needs. This easily leaves the main file in a messy state.
\item
The generated document will always carry the filename
of the main document. This is inconvenient if
several child files are to be compiled and
to be kept for distribution.
\end{itemize}

The present package provides a simple interface
to make child files individually compilable by \LaTeX{}.
Compiling a child file then has the same effect as compiling
the main file with an |\includeonly| command
to select the appropriate child.
Moreover the generated document will carry the name of the child
rather than the main file.
This resolves all three above issues.

This feature is meant to make the editing of books,
thesis documents and lecture notes somewhat more convenient.
However, the package can also be used efficiently for
composing a series of documents (such as exercise sheets)
which are typically distributed individually.
It then assists the author in generating the individual documents
(potentially in different versions)
as well as a document containing the collected series.
Another application is in developing style files
or other kinds of included material
where compilation of the style file could redirect
to a sample or test file.

%%%%%%%%%%%%%%%%%%%%%%%%%%%%%%%%%%%%%%%%%%%%%%%%%%%%%%%%%%%%%%%%%%%%%%%%%%%%%%%%
%%%%%%%%%%%%%%%%%%%%%%%%%%%%%%%%%%%%%%%%%%%%%%%%%%%%%%%%%%%%%%%%%%%%%%%%%%%%%%%%
\section{Usage}

First of all, the package \textsf{childdoc} is \emph{not} a standard
\LaTeXe{} |.sty| style file! Therefore it needs to be invoked in
a non-standard way.

%%%%%%%%%%%%%%%%%%%%%%%%%%%%%%%%%%%%%%%%%%%%%%%%%%%%%%%%%%%%%%%%%%%%%%%%%%%%%%%%
\subsection{Included Files}
\label{sec:include}

%%%%%%%%%%%%%%%%%%%%%%%%%%%%%%%%%%%%%%%%
\DescribeMacro{\childdocmain}
To use the package, add the commands
\begin{center}
\begin{tabular}{l}
|\input{childdoc.def}|\\
|\childdocmain{}|\\
\end{tabular}
\end{center}
at the very top of the main \LaTeX{} file,
in particular \emph{before} the |\documentclass| statement!
The argument of |\childdocmain| should be left empty
(but it must be present).

%%%%%%%%%%%%%%%%%%%%%%%%%%%%%%%%%%%%%%%%
\DescribeMacro{\childdocof}
Furthermore, add the commands
\begin{center}
\begin{tabular}{l}
|\input{childdoc.def}|\\
|\childdocof{|\textit{main}|}|\\
\end{tabular}
\end{center}
at the top of every child file \textit{child}
which is included by |\include{|\textit{child}|}|
from within the main file
(or at least for those files to be compiled individually).
The argument \textit{main} must be the filename of the main file.

There are a couple of
considerations in setting up the main and child documents:

%%%%%%%%%%%%%%%%%%%%%%%%%%%%%%%%%%%%%%%%
\paragraph{Restrictions.}

Please note the following restrictions:
\begin{itemize}
\item
|\childdocmain| must be called with one argument \textit{main}
to ensure compatibility with earlier version of the package.
It must either be empty (|\childdocmain{}|)
or precisely match the filename of the main file in which it is specified.
See \secref{sec:detection} for further information.
\item
The filename \textit{main} must be specified without the |.tex| extension.
\item
The filename \textit{main} is case sensitive
(even in case-insensitive file systems)
due to internal string comparison.
\item
The argument \textit{main} should be fully expanded, it cannot be a macro.
\item
Subdirectories and special characters should be avoided in filenames.
\item
The command |\childdocmain{|\textit{main}|}| must be followed by a whitespace.
It should not be followed immediately by another command
or by a comment mark `|%|'.
This is because the \TeX{} parser reads the token immediately following
the argument of |\childdocmain| and puts it
at the beginning of every child section;
however, a white\-space is ignored.
\end{itemize}

%%%%%%%%%%%%%%%%%%%%%%%%%%%%%%%%%%%%%%%%
\paragraph{Content of Main File.}

It is advisable to place all content in the child files included by |\include|.
Any output contained in the main file will appear in all child documents
unless suppressed manually;
it cannot be suppressed automatically by the |\includeonly| directive
and thus should normally be avoided.
A method to include some content in the main file
by means of conditional processing is described in \secref{sec:conditional}.

%%%%%%%%%%%%%%%%%%%%%%%%%%%%%%%%%%%%%%%%
\paragraph{Page Numbering.}

When only a part of the document is compiled,
the appropriate numbering of pages
(as well as other status parameters)
is determined from the |.aux| files.
The latter contain information from previous passes.
However this information needs to propagate through
all intermediate child documents.
Therefore the page numbering in child documents may well
be inconsistent until the complete document is compiled at least once.

A useful (if unconventional) way to always ensure a consistent
page numbering is to restart the numbering in each child document
and denote the pages by `\textit{child}|.|\textit{page}'
where \textit{child} represents the chapter/section number of the child file.
This can be achieved by the command
|\numberwithin{page}{|\textit{child}|}|
of the \textsf{amsmath} package
where \textit{child} can be |chapter| or |section|
depending on the chosen structuring.
Alternatively, one can modify the macro |\thepage| appropriately
and reset the counter |page| at the start of each child file.

%%%%%%%%%%%%%%%%%%%%%%%%%%%%%%%%%%%%%%%%%%%%%%%%%%%%%%%%%%%%%%%%%%%%%%%%%%%%%%%%
\subsection{Conditional Processing}
\label{sec:conditional}

The package provides a mechanism to compile different versions
of a document. To customise the versions further some conditional processing
can come in handy to distinguish which version is being compiled.
The package provides two macros to describe the compilation context:

%%%%%%%%%%%%%%%%%%%%%%%%%%%%%%%%%%%%%%%%
\DescribeMacro{\ifchilddoc}
The conditional |\ifchilddoc| distinguishes between the compilation of
child documents and the main document:
%
\begin{center}
|\ifchilddoc |\textit{child-code}| |[|\||else |\textit{main-code}]| \||fi|
\end{center}

%%%%%%%%%%%%%%%%%%%%%%%%%%%%%%%%%%%%%%%%
\DescribeMacro{\childdocname}
\DescribeMacro{\childdocjob}
The macro |\childdocname| contains the filename (without extension)
of the main or child file being processed.
Note that |\childdocjob| will always contain the name of the main file.

%%%%%%%%%%%%%%%%%%%%%%%%%%%%%%%%%%%%%%%%
\paragraph{Title Page.}

Conditional processing can be used to include a title or banner page
in the main document when proper precautions are taken.
Importantly, the code in the main file should ensure that the page counter
(as well as other status parameters which are stored in the |.aux| files)
takes the same value after the conditional processing.
Otherwise the page numbers may take divergent values
depending on which part is compiled.

For example, a title page could be declared by:
%
\begin{center}
\begin{tabular}{l}
|\ifchilddoc\||else|\\
|\addtocounter{page}{-1}|\\
\textit{code for title page}\\
|\newpage|\\
|\||fi|
\end{tabular}
\end{center}
%
A banner page for the child documents can be generated by:
%
\begin{center}
\begin{tabular}{l}
|\ifchilddoc|\\
|\addtocounter{page}{-1}|\\
\textit{code for banner page}\\
|\newpage|\\
|\||fi|
\end{tabular}
\end{center}
%
Here one could write a message such as:
\begin{center}
|This is the part \childdocname{} of \childdocjob{}.|
\end{center}

%%%%%%%%%%%%%%%%%%%%%%%%%%%%%%%%%%%%%%%%%%%%%%%%%%%%%%%%%%%%%%%%%%%%%%%%%%%%%%%%
\subsection{Flags}
\label{sec:flags}

The package makes it easy to generate different versions
of the main or child documents.
To this end compilation flags can be defined
and assigned different default values.
They will be particularly useful in conjunction
with the forwarding mechanism described in \secref{sec:forward}.

For example, it may be useful to have a flag |\version|
which can be set to |draft| or |final|.
The document source will contain some conditional code
depending on the value of |\version|.
Suppose further, the flag should default to |final| for the main file
and to |draft| for child files
which is a natural assignment for editing the document.
This is achieved by placing the following code
in the preamble of the main document
(below the |\childdocmain| directive):
%
\begin{center}
\begin{tabular}{l}
|\ifchilddoc|\\
|\providecommand{\version}{draft}|\\
|\||else|\\
|\providecommand{\version}{final}|\\
|\||fi|
\end{tabular}
\end{center}
%
The definition by |\providecommand| makes sure
that previous definitions are not overwritten.
Further statements |\providecommand{\version}{...}|
can thus be added before the above code to override it.

For the main file, one might add a line
(between |\childdocmain| and the above block)
%
\begin{center}
|%\ifchilddoc\||else\providecommand{\version}{draft}\||fi|
\end{center}
%
which can be uncommented to produce a draft version.
Likewise one can add a line to the very top of a child file
(above the |\childdocof{|\textit{main}|}| directive)
%
\begin{center}
|%\providecommand{\version}{final}|
\end{center}
%
which can be uncommented to produce the final version of this child document.

%%%%%%%%%%%%%%%%%%%%%%%%%%%%%%%%%%%%%%%%%%%%%%%%%%%%%%%%%%%%%%%%%%%%%%%%%%%%%%%%
\subsection{Forwarding}
\label{sec:forward}

Different versions of the main or child documents
using compilation flags as described in \secref{sec:flags}
can be (permanently) stored in different files
for convenient compilation, viewing and distribution.
To this end, the package defines a command
to pass on compilation to a different file:

%%%%%%%%%%%%%%%%%%%%%%%%%%%%%%%%%%%%%%%%
\DescribeMacro{\childdocforward}
The command |\childdocforward| redirects processing to
another source file:
%
\begin{center}
\begin{tabular}{l}
|\input{childdoc.def}|\\
|\childdocforward[|\textit{main}|]{|\textit{dest}|}|\\
\end{tabular}
\end{center}
%
The argument \textit{dest} is the destination file
(without extension).
It should be the main file or one of the child files.
Note that further \textsf{childdoc} directives
such as |\childdocof| and |\childdocforward|
in the indicated file will be processed in this form.
The optional argument \textit{main}
passes on directly to the main file \textit{main}
while pretending to compile the child \textit{dest}.
This form behaves as if \textit{dest}
issues |\childdocof{|\textit{main}|}| right away,
and no further \textsf{childdoc} directives will be processed.

%%%%%%%%%%%%%%%%%%%%%%%%%%%%%%%%%%%%%%%%
\DescribeMacro{\...prefix}
In the alternative form |\childdocforwardprefix|,
%
\begin{center}
\begin{tabular}{l}
|\input{childdoc.def}|\\
|\childdocforwardprefix[|\textit{main}|]{|\textit{prefix}|}{|\textit{dest}|}|
\end{tabular}
\end{center}
%
the destination file is determined by a pattern
depending on the current file:
To make this work, the current file must be called
`{\textit{prefix}\hspace{0.2em}\textit{suffix}}'
with \textit{prefix} matching precisely the argument.
Processing is then passed on to the file
`{\textit{dest}\hspace{0.2em}\textit{suffix}}'.
Surely, the same effect is achieved by
directly specifying the
argument `{\textit{dest}\hspace{0.2em}\textit{suffix}}'
in the first form.
However, that requires to set up a different file
for each child. With the alternative form of the command
all these files can have exactly the same content
which simplifies setting them up and maintaining them.

For example, the following file |draft.tex|
with a compilation flag |\version| as described in \secref{sec:flags}
compiles the main document as a draft:
%
\begin{center}
\begin{tabular}{l}
|\def\version{draft}|\\
|\input{childdoc.def}|\\
|\childdocforward{|\textit{main}|}|
\end{tabular}
\end{center}
%
Likewise, the following files |final|\textit{nn}|.tex|
compile the final version of the child document
|child|\textit{nn}|.tex|:
%
\begin{center}
\begin{tabular}{l}
|\def\version{final}|\\
|\input{childdoc.def}|\\
|\childdocforwardprefix{final}{child}|
\end{tabular}
\end{center}
%

Note that when several versions of a main file and/or of each child file
are to be generated, it may be convenient to set up a |Makefile| or
shell script to automatise the process.

%%%%%%%%%%%%%%%%%%%%%%%%%%%%%%%%%%%%%%%%%%%%%%%%%%%%%%%%%%%%%%%%%%%%%%%%%%%%%%%%
\subsection{Command Line Processing}
\label{sec:commandline}

The effect of redirection files can also be achieved by invoking
the \LaTeX{} compiler with a more elaborate command line.
Most conveniently this should be done as part
of a shell script or a |Makefile|.

When using \textsf{childdoc} in the main file, the following
command lines effectively perform a redirection
(note that depending on the shell being used,
backslashes may have to be doubled: `|\|' $\to$ `|\\|'):
%
\begin{center}
|... -jobname "|\textit{target}|" |\\|"|[\textit{flags}]%
|\input{childdoc.def}\childdocforward[|\textit{main}|]{|\textit{dest}|}"|
\end{center}
%
Here \textit{target} is the name of the output file,
\textit{main} is the name of the main file
and \textit{dest} is the name of the main or child file to be processed
(all filenames without extensions).
The optional argument \textit{main} can be omitted
if \textit{main} matches \textit{dest}.
Optionally, compilation \textit{flags} can be defined via |\def| commands.
This command line makes the \TeX{} engine believe
it is compiling the file \textit{target}
whose content is specified as the latter parameter.
The provided code then forwards the processing to
\textit{main} or \textit{dest} as described in \secref{sec:forward}.

%%%%%%%%%%%%%%%%%%%%%%%%%%%%%%%%%%%%%%%%%%%%%%%%%%%%%%%%%%%%%%%%%%%%%%%%%%%%%%%%
\subsection{Include by Input}
\label{sec:input}

Including child documents by |\include| has some restrictions by design.
Most notably, the content of a child document always occupies
its own set of pages; pages cannot be shared between child documents.
Usually, this behaviour makes perfect sense
because each child document contain an essential part of the document.
However, in some situations it may be desirable to compose
a document from a collection of parts
without having mandatory page breaks between then.
For this case, the package
provides a mechanism to include parts
by |\input| which can also be processed individually.
However, by construction this mechanism
requires manual handling of the content to be output.

%%%%%%%%%%%%%%%%%%%%%%%%%%%%%%%%%%%%%%%%
\DescribeMacro{\ifchilddocmanual}
The main file should be prepared as usual, see \secref{sec:include}.
However, the document body must make a distinction
between processing of an individual part and of the main document, e.g.:
%
\begin{center}
\begin{tabular}{l}
|\ifchilddocmanual|\\
|\input{\childdocname}|\\
|\||else|\\
\textit{document body with }|\input{|\textit{part}|}|\\
|\||fi|
\end{tabular}
\end{center}
%
The conditional |\ifchilddocmanual| is true whenever
a part to be included by |\input| is being compiled,
and the name of the part is stored in |\childdocname|.

%%%%%%%%%%%%%%%%%%%%%%%%%%%%%%%%%%%%%%%%
\DescribeMacro{\childdocby}
Each part to be included by |\input| should start with:
%
\begin{center}
\begin{tabular}{l}
|\input{childdoc.def}|\\
|\childdocby{|\textit{main}|}|\\
\end{tabular}
\end{center}
%
The directive |\childdocby| is similar to |\childdocof|
described in \secref{sec:include},
but the subsequent selection of content must be done manually.
To that end, both |\ifchilddoc| and |\ifchilddocmanual|
will be true upon processing of a part,
and the name of the part is stored in |\childdocname|.
Note that |\jobname| will be set to the filename of the current part
so that each part receives an individual |.aux| file
that does not interfere with the |.aux| file(s) of the main document.
This behaviour can be altered by the alternative form
|\childdocby[*]{|\textit{main}|}| (with a non-empty optional argument)
which uses the |.aux| file of the main document
by setting |\jobname| to \textit{main}.

%%%%%%%%%%%%%%%%%%%%%%%%%%%%%%%%%%%%%%%%%%%%%%%%%%%%%%%%%%%%%%%%%%%%%%%%%%%%%%%%
\subsection{Driver Development}
\label{sec:driver}

The \textsf{childdoc} mechanism can also be use for the development
of definition files such as \LaTeX{} styles or classes.
This case differs from the above setup with multiple parts
included by |\include| in that no |\includeonly| should be invoked.
This can be achieved by starting the include file
(before |\ProvidesPackage|) with:
%
\begin{center}
\begin{tabular}{l}
|\input{childdoc.def}|\\
|\childdocforward{|\textit{main}|}|\\
\end{tabular}
\end{center}
%
or alternatively with:
%
\begin{center}
\begin{tabular}{l}
|\input{childdoc.def}|\\
|\childdocby{|\textit{main}|}|\\
\end{tabular}
\end{center}
%
Both forms have slightly different effects as described above.
The main file is prepared as usual, see \secref{sec:include}.

%%%%%%%%%%%%%%%%%%%%%%%%%%%%%%%%%%%%%%%%%%%%%%%%%%%%%%%%%%%%%%%%%%%%%%%%%%%%%%%%
\subsection{Legacy Detection}
\label{sec:detection}

The directive |\childdocmain| in the main file can detect
whether the complete document or merely a child is to be compiled
even without using the directive |\childdocof|.
This method is deprecated because it is less robust
and there is no compelling reason to use it;
it is merely provided for backward compatibility
and it may be removed in future versions.

If the detection mechanism is to be used,
it is mandatory to correctly specify
the filename of the main file as the argument of |\childdocmain|:
%
\begin{center}
\begin{tabular}{l}
|\input{childdoc.def}|\\
|\childdocmain{|\textit{main}|}|\\
\end{tabular}
\end{center}
%
If |\jobname| does not match the argument \textit{main} of |\childdocmain|,
it is assumed that |\jobname| points to the child file to be compiled.
When using |\childdocmain| with the main file specified as argument,
it suffices to start a child file
with just |\input{|\textit{main}|}|
without loading of the package and using |\childdocof|.
If instead all processing is done
with the appropriate \textsf{childdoc} directives,
the argument of \textit{main} of |\childdocmain| can be empty.

An alternative version of the command line processing described
in \secref{sec:commandline} using the detection mechanism reads:
%
\begin{center}
|... -jobname "|\textit{target}|" "|[\textit{flags}]%
[|\def\jobname{|\textit{dest}|}|]|\input{|\textit{main}|}"|
\end{center}

%%%%%%%%%%%%%%%%%%%%%%%%%%%%%%%%%%%%%%%%%%%%%%%%%%%%%%%%%%%%%%%%%%%%%%%%%%%%%%%%
\subsection{Manual Code}
\label{sec:manual}

In case one cannot be certain whether the definitions file |childdoc.def|
is installed on the target \TeX{} distribution
and one prefers not to ship it,
it is conceivable to paste a few relevant commands into the sources.

To that end, drop all statements |\input{childdoc.def}|
and perform the replacements as outlined below.
Instead of |\childdocmain{|\textit{main}|}| add the following code
to the top of the main file:
%
\begin{center}
\begin{tabular}{l}
|\||ifdefined\childdocname\endinput\||fi\newif\ifchilddoc|\\
|\edef\childdocname{\scantokens\expandafter{\jobname\noexpand}}|\\
|\def\childdocmain{|\textit{main}|}\||ifx\childdocmain\childdocname\||else|\\
|\childdoctrue\includeonly{\childdocname}\let\jobname\childdocmain\||fi|\\
\end{tabular}
\end{center}
%
Instead of |\childdocof{|\textit{main}|}| just include the main file
at the top of each child file:
%
\begin{center}
|\input{|\textit{main}|}|
\end{center}
%
A simple redirection |\childdocforward{|\textit{dest}|}| is achieved by:
%
\begin{center}
|\def\jobname{|\textit{dest}|}\input{\jobname}|
\end{center}
%
The redirection with prefix
|\childdocforwardprefix[|\textit{prefix}|]{|\textit{dest}|}|
is accomplished by:
%
\begin{center}
\begin{tabular}{l}
|{\edef\jobname{\scantokens\expandafter{\jobname\noexpand}}|\\
|\def\redirectjob |\textit{prefix}|#1~~~{\gdef\jobname{|\textit{dest}|#1}}|\\
|\expandafter\redirectjob\jobname~~~}\input{\jobname}|
\end{tabular}
\end{center}

In an alternative approach,
child documents can be compiled by a specific command line
without additional code or specific definitions:
%
\begin{center}
|... -jobname "|\textit{target}|" "|[\textit{flags}]%
|\includeonly{|\textit{dest}|}\input{|\textit{main}|}"|
\end{center}
%

%%%%%%%%%%%%%%%%%%%%%%%%%%%%%%%%%%%%%%%%%%%%%%%%%%%%%%%%%%%%%%%%%%%%%%%%%%%%%%%%
%%%%%%%%%%%%%%%%%%%%%%%%%%%%%%%%%%%%%%%%%%%%%%%%%%%%%%%%%%%%%%%%%%%%%%%%%%%%%%%%
\section{Information}

%%%%%%%%%%%%%%%%%%%%%%%%%%%%%%%%%%%%%%%%%%%%%%%%%%%%%%%%%%%%%%%%%%%%%%%%%%%%%%%%
\subsection{Copyright}

Copyright \copyright{} 2017--2018 Niklas Beisert

This work may be distributed and/or modified under the
conditions of the \LaTeX{} Project Public License, either version 1.3
of this license or (at your option) any later version.
The latest version of this license is in
  \url{http://www.latex-project.org/lppl.txt}
and version 1.3 or later is part of all distributions of \LaTeX{}
version 2005/12/01 or later.

This work has the LPPL maintenance status `maintained'.

The Current Maintainer of this work is Niklas Beisert.

This work consists of the files |README.txt|, |childdoc.ins| and |childdoc.dtx|
as well as the derived files |childdoc.def|, |cdocsamp.tex|
with |cdocsch1.tex|, |cdocsch2.tex|, |cdocspt3.tex|, |cdocspt4.tex|,
|cdocsdrf.tex|, |cdocsfn1.tex|, |cdocsfn2.tex|
as well as |childdoc.pdf|.

%%%%%%%%%%%%%%%%%%%%%%%%%%%%%%%%%%%%%%%%%%%%%%%%%%%%%%%%%%%%%%%%%%%%%%%%%%%%%%%%
\subsection{Files and Installation}

The package consists of the files:
%
\begin{center}
\begin{tabular}{ll}
    |README.txt|   & readme file \\
    |childdoc.ins| & installation file \\
    |childdoc.dtx| & source file \\
    |childdoc.def| & definition file \\
    |cdocsamp.tex| & sample main file \\
    |cdocsch1.tex| & sample include file \\
    |cdocsch2.tex| & sample include file \\
    |cdocspt3.tex| & sample part file \\
    |cdocspt4.tex| & sample part file \\
    |cdocsdrf.tex| & sample redirection file \\
    |cdocsfn1.tex| & sample redirection file \\
    |cdocsfn2.tex| & sample redirection file \\
    |childdoc.pdf| & manual
\end{tabular}
\end{center}
%
The distribution consists of the files
|README.txt|, |childdoc.ins| and |childdoc.dtx|.
%
\begin{itemize}
\item
Run (pdf)\LaTeX{} on |childdoc.dtx|
to compile the manual |childdoc.pdf| (this file).
\item
Run \LaTeX{} on |childdoc.ins| to create the definitions file |childdoc.def|
and the sample |cdocsamp.tex| with include files
|cdocsch1.tex|, |cdocsch2.tex|, |cdocspt3.tex|, |cdocspt4.tex|,
|cdocsdrf.tex|, |cdocsfn1.tex|, |cdocsfn2.tex|.
Then copy the file |childdoc.def| to an appropriate directory of your \LaTeX{}
distribution, e.g.\ \textit{texmf-root}|/tex/latex/childdoc|.
\end{itemize}

%%%%%%%%%%%%%%%%%%%%%%%%%%%%%%%%%%%%%%%%%%%%%%%%%%%%%%%%%%%%%%%%%%%%%%%%%%%%%%%%
\subsection{Related CTAN Packages}

There are several other packages which offer a similar functionality:
%
\begin{itemize}
\item
The packages
\href{http://ctan.org/pkg/docmute}{\textsf{docmute}},
\href{http://ctan.org/pkg/includex}{\textsf{includex}} and
\href{http://ctan.org/pkg/standalone}{\textsf{standalone}}
provide commands to include only the document body of
a child file thus allowing both files to be compiled individually.
\item
The packages \href{http://ctan.org/pkg/subdocs}{\textsf{subdocs}}
and \href{http://ctan.org/pkg/subfiles}{\textsf{subfiles}}
provide structures in which the main and child documents can be
encapsulated and allowing them to be compiled individually.
The inclusion mechanism is different from the conventional |\include|.
\item
The package \href{http://ctan.org/pkg/combine}{\textsf{combine}}
is an elaborate solution to combine several documents into one.
\end{itemize}
%
See also the CTAN topic \href{http://ctan.org/topic/subdocs}{\textsf{subdocs}}
for further related packages.
The present package differs from the above solutions in that
a document structure constructed with the conventional |\include| mechanism
just needs two extra commands at the top of every file
such that all constituent files can be compiled individually.

%%%%%%%%%%%%%%%%%%%%%%%%%%%%%%%%%%%%%%%%%%%%%%%%%%%%%%%%%%%%%%%%%%%%%%%%%%%%%%%%
%\subsection{Feature Suggestions}
%
%The following is a list of features which may be useful for future
%versions of this package:
%%
%\begin{itemize}
%\item
%\ldots
%\end{itemize}

%%%%%%%%%%%%%%%%%%%%%%%%%%%%%%%%%%%%%%%%%%%%%%%%%%%%%%%%%%%%%%%%%%%%%%%%%%%%%%%%
\subsection{Revision History}

%%%%%%%%%%%%%%%%%%%%%%%%%%%%%%%%%%%%%%%%
\paragraph{v2.0:} 2018/12/30

\begin{itemize}
\item
immediate forward processing
\item
added |\childdocby| mechanism
\item
manual restructured
\end{itemize}

%%%%%%%%%%%%%%%%%%%%%%%%%%%%%%%%%%%%%%%%
\paragraph{v1.6:} 2018/01/17

\begin{itemize}
\item
application for development of include files
\item
corrections to manual
\end{itemize}

%%%%%%%%%%%%%%%%%%%%%%%%%%%%%%%%%%%%%%%%
\paragraph{v1.5:} 2017/05/21

\begin{itemize}
\item
more complete structuring introduced
\item
|\childdocof| introduced
\item
|\childdoc| renamed to |\childdocmain|
\item
|\childredirect| renamed to |\childdocforward| and |\childdocforwardprefix|
and functionality expanded
\end{itemize}

%%%%%%%%%%%%%%%%%%%%%%%%%%%%%%%%%%%%%%%%
\paragraph{v1.0:} 2017/04/27

\begin{itemize}
\item
manual and install package
\item
first version published on CTAN
\end{itemize}

%%%%%%%%%%%%%%%%%%%%%%%%%%%%%%%%%%%%%%%%
\paragraph{v0.6:} 2017/04/26

\begin{itemize}
\item
redirection mechanism added
\end{itemize}

%%%%%%%%%%%%%%%%%%%%%%%%%%%%%%%%%%%%%%%%
\paragraph{v0.5:} 2017/04/26

\begin{itemize}
\item
functionality in definition file
\end{itemize}


%%%%%%%%%%%%%%%%%%%%%%%%%%%%%%%%%%%%%%%%%%%%%%%%%%%%%%%%%%%%%%%%%%%%%%%%%%%%%%%%
%%%%%%%%%%%%%%%%%%%%%%%%%%%%%%%%%%%%%%%%%%%%%%%%%%%%%%%%%%%%%%%%%%%%%%%%%%%%%%%%
%%%%%%%%%%%%%%%%%%%%%%%%%%%%%%%%%%%%%%%%%%%%%%%%%%%%%%%%%%%%%%%%%%%%%%%%%%%%%%%%
\appendix

\settowidth\MacroIndent{\rmfamily\scriptsize 000\ }

 \DocInput{childdoc.dtx}

\end{document}
%</driver>
% \fi
%
% %%%%%%%%%%%%%%%%%%%%%%%%%%%%%%%%%%%%%%%%%%%%%%%%%%%%%%%%%%%%%%%%%%%%%%%%%%%%%%
% %%%%%%%%%%%%%%%%%%%%%%%%%%%%%%%%%%%%%%%%%%%%%%%%%%%%%%%%%%%%%%%%%%%%%%%%%%%%%%
% \section{Sample}
%\iffalse
%<*samplemain>
%\fi
%
% The following presents a sample document
% with two chapters, two parts, a title page,
% a compile flag as well as three forwarding files to set the flag.
% It consists of eight |.tex| files:
% \begin{center}
% \begin{tabular}{ll}
% |cdocsamp.tex|&main file\\
% |cdocsch1.tex|&include file for chapter 1\\
% |cdocsch2.tex|&include file for chapter 2\\
% |cdocspt3.tex|&include file for part 3\\
% |cdocspt4.tex|&include file for part 4\\
% |cdocsdrf.tex|&forwarding file for main file in draft mode\\
% |cdocsfi1.tex|&forwarding file for final version of chapter 1\\
% |cdocsfi2.tex|&forwarding file for final version of chapter 2\\
% \end{tabular}
% \end{center}
% Each of the eight files can be compiled directly by the \LaTeX{} compiler.
%
% %%%%%%%%%%%%%%%%%%%%%%%%%%%%%%%%%%%%%%
% \paragraph{Main File.}
%
% The main file is called |cdocsamp.tex|.
%
% Load the \textsf{childdoc} definitions and
% declare the filename for the main document:
%    \begin{macrocode}
\input{childdoc.def}
\childdocmain{}
%    \end{macrocode}

% Optional override for |\version| flag:
%    \begin{macrocode}
%%\ifchilddoc\else\providecommand{\version}{draft}\fi
%    \end{macrocode}

% Define the default values for the |\version| flag
% (|final| for the main file and |draft| for childs):
%    \begin{macrocode}
\ifchilddoc
\providecommand{\version}{draft}
\else
\providecommand{\version}{final}
\fi
%    \end{macrocode}

% Load the standard document class:
%    \begin{macrocode}
\documentclass[12pt]{article}
%    \end{macrocode}

% Start the document body:
%    \begin{macrocode}
\begin{document}
%    \end{macrocode}

% Declare a title page.
% Print title, part of document being processed and version flag:
%    \begin{macrocode}
\addtocounter{page}{-1}
\begin{center}
{\LARGE\bfseries{}childdoc example\par}
\vspace{1cm}
\ifchilddoc
\ifchilddocmanual part\else chapter\fi:
`\childdocname' of `\childdocjob'\par
\else
main document: `\childdocjob'\par
\fi
version: \version\par
\end{center}
\newpage
%    \end{macrocode}

% Manually include selected file,
% otherwise process as usual:
%    \begin{macrocode}
\ifchilddocmanual
\section*{part `\childdocname'}
\input{\childdocname}
\else
%    \end{macrocode}

% Include the two chapters:
%    \begin{macrocode}
\include{cdocsch1}
\include{cdocsch2}
%    \end{macrocode}

% Include the two parts unless only chapters should be displayed:
%    \begin{macrocode}
\ifchilddoc\else
\section{part three}
\input{cdocspt3}
\section{part four}
\input{cdocspt4}
\fi
%    \end{macrocode}

% Process as usual until here:
%    \begin{macrocode}
\fi
%    \end{macrocode}

% End of document body:
%    \begin{macrocode}
\end{document}
%    \end{macrocode}
%\iffalse
%</samplemain>
%\fi
%
% %%%%%%%%%%%%%%%%%%%%%%%%%%%%%%%%%%%%%%
% \paragraph{Chapter Include Files.}
%
% The include files are called |cdocsch1.tex| and |cdocsch2.tex|.
%
%\iffalse
%<*samplechap1|samplechap2>
%\fi

% Optional override for |\version| flag:
%    \begin{macrocode}
%%\providecommand{\version}{final}
%    \end{macrocode}

% Include the main document:
%    \begin{macrocode}
\input{childdoc.def}
\childdocof{cdocsamp}
%    \end{macrocode}

%\iffalse
%</samplechap1|samplechap2>
%\fi
%
%\iffalse
%<*samplechap1>
%\fi
% Some text for chapter 1:
%    \begin{macrocode}
\section{one}
some text in chapter one
%    \end{macrocode}

%\iffalse
%</samplechap1>
%\fi
% Some text for chapter 2:
%\iffalse
%<*samplechap2>
%\fi
%    \begin{macrocode}
\section{two}
more text in chapter two
%    \end{macrocode}

%\iffalse
%</samplechap2>
%\fi
%
% %%%%%%%%%%%%%%%%%%%%%%%%%%%%%%%%%%%%%%
% \paragraph{Part Include Files.}
%
% The include files are called |cdocspt3.tex| and |cdocspt4.tex|.
%
%\iffalse
%<*samplepart3|samplepart4>
%\fi

% Optional override for |\version| flag:
%    \begin{macrocode}
%%\providecommand{\version}{final}
%    \end{macrocode}

% Include the main document:
%    \begin{macrocode}
\input{childdoc.def}
\childdocby{cdocsamp}
%    \end{macrocode}

%\iffalse
%</samplepart3|samplepart4>
%\fi
%
%\iffalse
%<*samplepart3>
%\fi
% Some text for part 3:
%    \begin{macrocode}
some text in part three
%    \end{macrocode}

%\iffalse
%</samplepart3>
%\fi
% Some text for part 4:
%\iffalse
%<*samplepart4>
%\fi
%    \begin{macrocode}
more text in part four
%    \end{macrocode}

%\iffalse
%</samplepart4>
%\fi
%
% %%%%%%%%%%%%%%%%%%%%%%%%%%%%%%%%%%%%%%
% \paragraph{Forwarding for a Complete Draft.}
%
% The following forwarding file |cdocsdrf.tex|
% compiles the main document in draft mode:
%\iffalse
%<*sampledraft>
%\fi
%    \begin{macrocode}
\def\version{draft}
\input{childdoc.def}
\childdocforward{cdocsamp}
%    \end{macrocode}

%\iffalse
%</sampledraft>
%\fi
%
% %%%%%%%%%%%%%%%%%%%%%%%%%%%%%%%%%%%%%%
% \paragraph{Forwarding for Final Version of the Chapters.}
%
% The following forwarding files |cdocsfn1.tex| and |cdocsfn2.tex|
% (with identical content)
% compile the final versions of the child documents
% |cdocsch1.tex| and |cdocsch2.tex|, respectively:
%\iffalse
%<*samplefinal>
%\fi
%    \begin{macrocode}
\def\version{final}
\input{childdoc.def}
\childdocforwardprefix[cdocsamp]{cdocsfn}{cdocsch}
%    \end{macrocode}

%\iffalse
%</samplefinal>
%\fi
%
% %%%%%%%%%%%%%%%%%%%%%%%%%%%%%%%%%%%%%%
% \paragraph{Command Line Processing.}
%
% The following three command lines generate the output files
% |cdocscld|, |cdocscl1| and |cdocscl2|
% which should be identical to
% |cdocsdrf|, |cdocsch1| and |cdocsfn2|, respectively:
% \begin{center}
% \begin{tabular}{l}
% |latex -jobname cdocscld \|\\
% |  "\def\version{draft}\input{childdoc.def}\childdocforward{cdocsamp}"|\\
% |latex -jobname cdocscl1 \|\\
% |  "\input{childdoc.def}\childdocforward[cdocsamp]{cdocsch1}"|\\
% |latex -jobname cdocscl2 \|\\
% |  "\def\version{final}\input{childdoc.def}\childdocforward{cdocsch2}"|
% \end{tabular}
% \end{center}
% Note that the trailing backslash on each first line
% merely continues the input to the second line
% (for convenient cut ant paste).
% Furthermore, the command |latex| can be replaced by any
% of its alternative versions such as |pdflatex|.
%
% %%%%%%%%%%%%%%%%%%%%%%%%%%%%%%%%%%%%%%%%%%%%%%%%%%%%%%%%%%%%%%%%%%%%%%%%%%%%%%
% %%%%%%%%%%%%%%%%%%%%%%%%%%%%%%%%%%%%%%%%%%%%%%%%%%%%%%%%%%%%%%%%%%%%%%%%%%%%%%
% \section{Implementation}
%\iffalse
%<*package>
%\fi
%
% This section describes the definitions file |childdoc.def|.

% The definitions cannot be loaded using |\usepackage| or |\RequirePackage|
% which has a mechanism to prevent loading a style file more than once.
% When loading the definitions by means of |\input|
% multiple instances have to be prevented manually:
%\iffalse
%This code needs to be before the `\ProvidesFile' directive
%which is defined at the beginning of this file.
%Therefore it is also placed there and commented out here.
%</package>
%<*discard>
%\fi
%    \begin{macrocode}
\ifdefined\childdocmain\endinput\fi
%    \end{macrocode}
%\iffalse
%</discard>
%<*package>
%\fi
%
% \macro{\ifchilddoc}
% \macro{\ifchilddocmanual}
% The conditional |\ifchilddoc| tells whether a
% child (true) or main (false) document is being compiled.
% The conditional |\ifchilddocmanual| tells whether
% the |\includeonly| mechanism is used (false) or
% the selection of child files must be performed manually (true).
% The definitions initialise to false:
%    \begin{macrocode}
\newif\ifchilddoc
\newif\ifchilddocmanual
%    \end{macrocode}

% \macro{\childdocname}
% \macro{\childdocjob}
% The macro |\childdocname| stores the name of the main document
% to be compiled. The macro |\childdocjob| stores the name of
% the document on which the \LaTeX{} compiler was originally invoked.
% The content of |\jobname| cannot be compared
% to filenames specified in the source due to different catcodes.
% The following code rescans |\jobname|, stores the result
% in |\childdocname| and saves a copy in |\childdocjob|:
%    \begin{macrocode}
\edef\childdocname{\scantokens\expandafter{\jobname\noexpand}}
\let\childdocjob\childdocname
%    \end{macrocode}

% \macro{\childdocdisable}
% The macro |\childdocdisable| prevents the main file
% from being processed more than once.
% At this stage, the main document command |\childdocmain|
% is assumed to be called once again where it should do nothing.
% Any subsequent call to it should prevent
% a secondary processing of the main document
% It overwrites the forwarding commands
% |\childdocof| and |\childdocforward|
% with empty macros to prevent further inclusions of the main document:
%    \begin{macrocode}
\newcommand{\childdocdisable}
{
  \renewcommand{\childdocmain}[1]{\renewcommand{\childdocmain}[1]{\endinput}}
  \renewcommand{\childdocof}[1]{}
  \renewcommand{\childdocby}[2][]{}
  \renewcommand{\childdocforward}[2][]{}
  \renewcommand{\childdocdisable}{}
}
%    \end{macrocode}

% \macro{\childdocmain}
% The macro |\childdocmain| is to be called at the top of the main file
% with nothing or the main filename (without extension) as argument.
% First, it breaks loops.
% If the argument is not empty and does not match |\childdocname|
% (which is set by the first inclusion of |childdoc.def|),
% |\ifchilddoc| is set to true, |\includeonly| is applied to the child file
% and |\jobname| is set to the main file
% (for proper handling of |.aux| files):
%    \begin{macrocode}
\newcommand{\childdocmain}[1]
{
  \childdocdisable\childdocmain{}
  \if?#1?\else
    \begingroup
      \def\childdoctmp{#1}
      \ifx\childdoctmp\childdocname
        \def\childdoctmp{}
      \else
        \def\childdoctmp
        {
          \childdoctrue
          \includeonly{\childdocname}
          \def\childdocjob{#1}
          \def\jobname{#1}
        }
      \fi
      \expandafter
    \endgroup
    \childdoctmp
  \fi
}
%    \end{macrocode}

% \macro{\childdocof}
% The command |\childdocof| redirects
% compilation to the main file |#1|.
%    \begin{macrocode}
\newcommand{\childdocof}[1]
{
  \childdocdisable
  \childdoctrue
  \includeonly{\childdocname}
  \def\jobname{#1}
  \def\childdocjob{#1}
  \input{#1}
}
%    \end{macrocode}

% \macro{\childdocby}
% The command |\childdocby| ....
%    \begin{macrocode}
\newcommand{\childdocby}[2][]
{
  \childdocdisable
  \childdoctrue
  \childdocmanualtrue
  \if?#1?\else
    \def\jobname{#2}
  \fi
  \def\childdocjob{#2}
  \input{#2}
  \endinput
}
%    \end{macrocode}

% \macro{\childdocforward}
% The command |\childdocforward| redirects
% compilation to the main file or
% (if the optional argument is given) a child file.
% Parameters are set as if the main file
% or a child file starting with |\childdocof| was compiled.
% Then compilation is handed over to the main file:
%    \begin{macrocode}
\newcommand{\childdocforward}[2][]
{
  \begingroup
    \if?#1?
      \def\childdoctmp
      {
        \def\childdocname{#2}
        \def\childdocjob{#2}
        \def\jobname{#2}
        \input{#2}
        \endinput
      }
    \else
      \def\childdoctmp
      {
        \childdocdisable
        \def\childdocname{#2}
        \childdoctrue
        \includeonly{#2}
        \def\childdocjob{#1}
        \def\jobname{#1}
        \input{#1}
        \endinput
      }
    \fi
    \expandafter
  \endgroup
  \childdoctmp
}
%    \end{macrocode}

% \macro{\childdocforwardprefix}
% The command |\childdocforwardprefix| redirects
% compilation to the main or a child file by means of a pattern.
% The prefix |#1| in the current filename is replaced by |#2|
% and the suffix of the current filename is kept
% (it is assumed that the filename does not contain the substring `|~~~|'
% which is used as a delimiter).
% Compilation is handed over to the new file by |\childdocforward|:
%    \begin{macrocode}
\newcommand{\childdocforwardprefix}[3][]
{
  \begingroup
    \def\childdocextract #2##1~~~{\def\childdoctmp{\childdocforward[#1]{#3##1}}}
    \expandafter\childdocextract\childdocname~~~
    \expandafter
  \endgroup
  \childdoctmp
}
%    \end{macrocode}

% \macro{\childdoc}
% The deprecated macro |\childdoc| is a legacy version of |\childdocmain|:
%    \begin{macrocode}
\newcommand{\childdoc}{\childdocmain}
%    \end{macrocode}

% \macro{\childdocredirect}
% The deprecated macro |\childdocredirect| is a legacy version
% of |\childdocforward| and |\childdocforwardprefix|:
%    \begin{macrocode}
\newcommand{\childdocredirect}[2][]
{
  \begingroup
    \if?#1?
      \def\childdoctmp{\childdocforward{#2}}
    \else
      \def\childdoctmp{\childdocforwardprefix{#1}{#2}}
    \fi
    \expandafter
  \endgroup
  \childdoctmp
}
%    \end{macrocode}

%\iffalse
%</package>
%\fi
%
\endinput
\childdocforward{cdocsamp}"|\\
% |latex -jobname cdocscl1 \|\\
% |  "% \iffalse
%
% childdoc.dtx Copyright (C) 2017-2018 Niklas Beisert
%
% This work may be distributed and/or modified under the
% conditions of the LaTeX Project Public License, either version 1.3
% of this license or (at your option) any later version.
% The latest version of this license is in
%   http://www.latex-project.org/lppl.txt
% and version 1.3 or later is part of all distributions of LaTeX
% version 2005/12/01 or later.
%
% This work has the LPPL maintenance status `maintained'.
%
% The Current Maintainer of this work is Niklas Beisert.
%
% This work consists of the files childdoc.dtx and childdoc.ins
% and the derived files childdoc.def and cdocsamp.tex with
% cdocsch1.tex, cdocsch2.tex, cdocsdrf.tex, cdocsfn1.tex, cdocsfn2.tex.
%
%<package>\ifdefined\childdocmain\endinput\fi
%<package>\ProvidesFile{childdoc.def}[2018/12/30 v2.0 child document driver]
%<samplemain>\ProvidesFile{cdocsamp.tex}[2018/12/30 v2.0 sample for childdoc]
%<*driver>
%\ProvidesFile{childdoc.drv}[2018/12/30 v2.0 childdoc reference manual file]
\PassOptionsToClass{10pt,a4paper}{article}
\documentclass{ltxdoc}

\usepackage[margin=35mm]{geometry}
\usepackage{hyperref}
\usepackage{hyperxmp}
\usepackage[usenames]{color}

\hypersetup{colorlinks=true}
\hypersetup{pdfstartview=FitH}
\hypersetup{pdfpagemode=UseNone}
\hypersetup{pdfsource={}}
\hypersetup{pdflang={en-UK}}
\hypersetup{pdfcopyright={Copyright 2017-2018 Niklas Beisert.
  This work may be distributed and/or modified under the
  conditions of the LaTeX Project Public License, either version 1.3
  of this license or (at your option) any later version.}}
\hypersetup{pdflicenseurl={http://www.latex-project.org/lppl.txt}}
\hypersetup{pdfcontactaddress={ETH Zurich, ITP, HIT K,
  Wolfgang-Pauli-Strasse 27}}
\hypersetup{pdfcontactpostcode={8093}}
\hypersetup{pdfcontactcity={Zurich}}
\hypersetup{pdfcontactcountry={Switzerland}}
\hypersetup{pdfcontactemail={nbeisert@itp.phys.ethz.ch}}
\hypersetup{pdfcontacturl={http://people.phys.ethz.ch/\xmptilde nbeisert/}}

\newcommand{\secref}[1]{\hyperref[#1]{section \ref*{#1}}}

\parskip1ex
\parindent0pt
\let\olditemize\itemize
\def\itemize{\olditemize\parskip0pt}

\begin{document}

\title{The \textsf{childdoc} Package}
\hypersetup{pdftitle={The childdoc Package}}
\author{Niklas Beisert\\[2ex]
  Institut f\"ur Theoretische Physik\\
  Eidgen\"ossische Technische Hochschule Z\"urich\\
  Wolfgang-Pauli-Strasse 27, 8093 Z\"urich, Switzerland\\[1ex]
  \href{mailto:nbeisert@itp.phys.ethz.ch}
  {\texttt{nbeisert@itp.phys.ethz.ch}}}
\hypersetup{pdfauthor={Niklas Beisert}}
\hypersetup{pdfsubject={Manual for the LaTeX2e Package childdoc}}
\date{30 December 2018, \textsf{v2.0}}
\maketitle

\begin{abstract}\noindent
\textsf{childdoc} is a \LaTeXe{} package
that enables the direct compilation
of document sections included by |\include|
to individual files.
\end{abstract}

\begingroup
\parskip0ex
\tableofcontents
\endgroup

%%%%%%%%%%%%%%%%%%%%%%%%%%%%%%%%%%%%%%%%%%%%%%%%%%%%%%%%%%%%%%%%%%%%%%%%%%%%%%%%
%%%%%%%%%%%%%%%%%%%%%%%%%%%%%%%%%%%%%%%%%%%%%%%%%%%%%%%%%%%%%%%%%%%%%%%%%%%%%%%%
\section{Introduction}

\LaTeX{} provides a mechanism to structure a large document (such as a book)
into a main file and several child files (containing the chapters)
using the |\include| command.
This mechanism is beneficial for documents
which span hundreds of pages in order to
make the source file(s) more manageable.
Moreover, compilation can be restricted to
selected child files by means of the |\includeonly| command.
The latter feature can be used to reduce the compilation time while editing
(this was significantly more useful in the earlier days of \LaTeX{})
or to generate a smaller document which is easier to navigate.
Another application of |\includeonly| is to generate
documents consisting of selected parts of the complete document.

However, there are a few drawbacks of the plain |\include| mechanism:
\begin{itemize}
\item
The child files cannot be compiled on their own,
they can only be compiled via the main file.
A naive editing environment
(such as a text editor with an option
to have the current file processed by \LaTeX)
may require one to switch to the main file before compiling;
attempting to compile the child file produces errors.
\item
The main file must be modified (each time)
to adjust the |\includeonly| command
to the present needs. This easily leaves the main file in a messy state.
\item
The generated document will always carry the filename
of the main document. This is inconvenient if
several child files are to be compiled and
to be kept for distribution.
\end{itemize}

The present package provides a simple interface
to make child files individually compilable by \LaTeX{}.
Compiling a child file then has the same effect as compiling
the main file with an |\includeonly| command
to select the appropriate child.
Moreover the generated document will carry the name of the child
rather than the main file.
This resolves all three above issues.

This feature is meant to make the editing of books,
thesis documents and lecture notes somewhat more convenient.
However, the package can also be used efficiently for
composing a series of documents (such as exercise sheets)
which are typically distributed individually.
It then assists the author in generating the individual documents
(potentially in different versions)
as well as a document containing the collected series.
Another application is in developing style files
or other kinds of included material
where compilation of the style file could redirect
to a sample or test file.

%%%%%%%%%%%%%%%%%%%%%%%%%%%%%%%%%%%%%%%%%%%%%%%%%%%%%%%%%%%%%%%%%%%%%%%%%%%%%%%%
%%%%%%%%%%%%%%%%%%%%%%%%%%%%%%%%%%%%%%%%%%%%%%%%%%%%%%%%%%%%%%%%%%%%%%%%%%%%%%%%
\section{Usage}

First of all, the package \textsf{childdoc} is \emph{not} a standard
\LaTeXe{} |.sty| style file! Therefore it needs to be invoked in
a non-standard way.

%%%%%%%%%%%%%%%%%%%%%%%%%%%%%%%%%%%%%%%%%%%%%%%%%%%%%%%%%%%%%%%%%%%%%%%%%%%%%%%%
\subsection{Included Files}
\label{sec:include}

%%%%%%%%%%%%%%%%%%%%%%%%%%%%%%%%%%%%%%%%
\DescribeMacro{\childdocmain}
To use the package, add the commands
\begin{center}
\begin{tabular}{l}
|\input{childdoc.def}|\\
|\childdocmain{}|\\
\end{tabular}
\end{center}
at the very top of the main \LaTeX{} file,
in particular \emph{before} the |\documentclass| statement!
The argument of |\childdocmain| should be left empty
(but it must be present).

%%%%%%%%%%%%%%%%%%%%%%%%%%%%%%%%%%%%%%%%
\DescribeMacro{\childdocof}
Furthermore, add the commands
\begin{center}
\begin{tabular}{l}
|\input{childdoc.def}|\\
|\childdocof{|\textit{main}|}|\\
\end{tabular}
\end{center}
at the top of every child file \textit{child}
which is included by |\include{|\textit{child}|}|
from within the main file
(or at least for those files to be compiled individually).
The argument \textit{main} must be the filename of the main file.

There are a couple of
considerations in setting up the main and child documents:

%%%%%%%%%%%%%%%%%%%%%%%%%%%%%%%%%%%%%%%%
\paragraph{Restrictions.}

Please note the following restrictions:
\begin{itemize}
\item
|\childdocmain| must be called with one argument \textit{main}
to ensure compatibility with earlier version of the package.
It must either be empty (|\childdocmain{}|)
or precisely match the filename of the main file in which it is specified.
See \secref{sec:detection} for further information.
\item
The filename \textit{main} must be specified without the |.tex| extension.
\item
The filename \textit{main} is case sensitive
(even in case-insensitive file systems)
due to internal string comparison.
\item
The argument \textit{main} should be fully expanded, it cannot be a macro.
\item
Subdirectories and special characters should be avoided in filenames.
\item
The command |\childdocmain{|\textit{main}|}| must be followed by a whitespace.
It should not be followed immediately by another command
or by a comment mark `|%|'.
This is because the \TeX{} parser reads the token immediately following
the argument of |\childdocmain| and puts it
at the beginning of every child section;
however, a white\-space is ignored.
\end{itemize}

%%%%%%%%%%%%%%%%%%%%%%%%%%%%%%%%%%%%%%%%
\paragraph{Content of Main File.}

It is advisable to place all content in the child files included by |\include|.
Any output contained in the main file will appear in all child documents
unless suppressed manually;
it cannot be suppressed automatically by the |\includeonly| directive
and thus should normally be avoided.
A method to include some content in the main file
by means of conditional processing is described in \secref{sec:conditional}.

%%%%%%%%%%%%%%%%%%%%%%%%%%%%%%%%%%%%%%%%
\paragraph{Page Numbering.}

When only a part of the document is compiled,
the appropriate numbering of pages
(as well as other status parameters)
is determined from the |.aux| files.
The latter contain information from previous passes.
However this information needs to propagate through
all intermediate child documents.
Therefore the page numbering in child documents may well
be inconsistent until the complete document is compiled at least once.

A useful (if unconventional) way to always ensure a consistent
page numbering is to restart the numbering in each child document
and denote the pages by `\textit{child}|.|\textit{page}'
where \textit{child} represents the chapter/section number of the child file.
This can be achieved by the command
|\numberwithin{page}{|\textit{child}|}|
of the \textsf{amsmath} package
where \textit{child} can be |chapter| or |section|
depending on the chosen structuring.
Alternatively, one can modify the macro |\thepage| appropriately
and reset the counter |page| at the start of each child file.

%%%%%%%%%%%%%%%%%%%%%%%%%%%%%%%%%%%%%%%%%%%%%%%%%%%%%%%%%%%%%%%%%%%%%%%%%%%%%%%%
\subsection{Conditional Processing}
\label{sec:conditional}

The package provides a mechanism to compile different versions
of a document. To customise the versions further some conditional processing
can come in handy to distinguish which version is being compiled.
The package provides two macros to describe the compilation context:

%%%%%%%%%%%%%%%%%%%%%%%%%%%%%%%%%%%%%%%%
\DescribeMacro{\ifchilddoc}
The conditional |\ifchilddoc| distinguishes between the compilation of
child documents and the main document:
%
\begin{center}
|\ifchilddoc |\textit{child-code}| |[|\||else |\textit{main-code}]| \||fi|
\end{center}

%%%%%%%%%%%%%%%%%%%%%%%%%%%%%%%%%%%%%%%%
\DescribeMacro{\childdocname}
\DescribeMacro{\childdocjob}
The macro |\childdocname| contains the filename (without extension)
of the main or child file being processed.
Note that |\childdocjob| will always contain the name of the main file.

%%%%%%%%%%%%%%%%%%%%%%%%%%%%%%%%%%%%%%%%
\paragraph{Title Page.}

Conditional processing can be used to include a title or banner page
in the main document when proper precautions are taken.
Importantly, the code in the main file should ensure that the page counter
(as well as other status parameters which are stored in the |.aux| files)
takes the same value after the conditional processing.
Otherwise the page numbers may take divergent values
depending on which part is compiled.

For example, a title page could be declared by:
%
\begin{center}
\begin{tabular}{l}
|\ifchilddoc\||else|\\
|\addtocounter{page}{-1}|\\
\textit{code for title page}\\
|\newpage|\\
|\||fi|
\end{tabular}
\end{center}
%
A banner page for the child documents can be generated by:
%
\begin{center}
\begin{tabular}{l}
|\ifchilddoc|\\
|\addtocounter{page}{-1}|\\
\textit{code for banner page}\\
|\newpage|\\
|\||fi|
\end{tabular}
\end{center}
%
Here one could write a message such as:
\begin{center}
|This is the part \childdocname{} of \childdocjob{}.|
\end{center}

%%%%%%%%%%%%%%%%%%%%%%%%%%%%%%%%%%%%%%%%%%%%%%%%%%%%%%%%%%%%%%%%%%%%%%%%%%%%%%%%
\subsection{Flags}
\label{sec:flags}

The package makes it easy to generate different versions
of the main or child documents.
To this end compilation flags can be defined
and assigned different default values.
They will be particularly useful in conjunction
with the forwarding mechanism described in \secref{sec:forward}.

For example, it may be useful to have a flag |\version|
which can be set to |draft| or |final|.
The document source will contain some conditional code
depending on the value of |\version|.
Suppose further, the flag should default to |final| for the main file
and to |draft| for child files
which is a natural assignment for editing the document.
This is achieved by placing the following code
in the preamble of the main document
(below the |\childdocmain| directive):
%
\begin{center}
\begin{tabular}{l}
|\ifchilddoc|\\
|\providecommand{\version}{draft}|\\
|\||else|\\
|\providecommand{\version}{final}|\\
|\||fi|
\end{tabular}
\end{center}
%
The definition by |\providecommand| makes sure
that previous definitions are not overwritten.
Further statements |\providecommand{\version}{...}|
can thus be added before the above code to override it.

For the main file, one might add a line
(between |\childdocmain| and the above block)
%
\begin{center}
|%\ifchilddoc\||else\providecommand{\version}{draft}\||fi|
\end{center}
%
which can be uncommented to produce a draft version.
Likewise one can add a line to the very top of a child file
(above the |\childdocof{|\textit{main}|}| directive)
%
\begin{center}
|%\providecommand{\version}{final}|
\end{center}
%
which can be uncommented to produce the final version of this child document.

%%%%%%%%%%%%%%%%%%%%%%%%%%%%%%%%%%%%%%%%%%%%%%%%%%%%%%%%%%%%%%%%%%%%%%%%%%%%%%%%
\subsection{Forwarding}
\label{sec:forward}

Different versions of the main or child documents
using compilation flags as described in \secref{sec:flags}
can be (permanently) stored in different files
for convenient compilation, viewing and distribution.
To this end, the package defines a command
to pass on compilation to a different file:

%%%%%%%%%%%%%%%%%%%%%%%%%%%%%%%%%%%%%%%%
\DescribeMacro{\childdocforward}
The command |\childdocforward| redirects processing to
another source file:
%
\begin{center}
\begin{tabular}{l}
|\input{childdoc.def}|\\
|\childdocforward[|\textit{main}|]{|\textit{dest}|}|\\
\end{tabular}
\end{center}
%
The argument \textit{dest} is the destination file
(without extension).
It should be the main file or one of the child files.
Note that further \textsf{childdoc} directives
such as |\childdocof| and |\childdocforward|
in the indicated file will be processed in this form.
The optional argument \textit{main}
passes on directly to the main file \textit{main}
while pretending to compile the child \textit{dest}.
This form behaves as if \textit{dest}
issues |\childdocof{|\textit{main}|}| right away,
and no further \textsf{childdoc} directives will be processed.

%%%%%%%%%%%%%%%%%%%%%%%%%%%%%%%%%%%%%%%%
\DescribeMacro{\...prefix}
In the alternative form |\childdocforwardprefix|,
%
\begin{center}
\begin{tabular}{l}
|\input{childdoc.def}|\\
|\childdocforwardprefix[|\textit{main}|]{|\textit{prefix}|}{|\textit{dest}|}|
\end{tabular}
\end{center}
%
the destination file is determined by a pattern
depending on the current file:
To make this work, the current file must be called
`{\textit{prefix}\hspace{0.2em}\textit{suffix}}'
with \textit{prefix} matching precisely the argument.
Processing is then passed on to the file
`{\textit{dest}\hspace{0.2em}\textit{suffix}}'.
Surely, the same effect is achieved by
directly specifying the
argument `{\textit{dest}\hspace{0.2em}\textit{suffix}}'
in the first form.
However, that requires to set up a different file
for each child. With the alternative form of the command
all these files can have exactly the same content
which simplifies setting them up and maintaining them.

For example, the following file |draft.tex|
with a compilation flag |\version| as described in \secref{sec:flags}
compiles the main document as a draft:
%
\begin{center}
\begin{tabular}{l}
|\def\version{draft}|\\
|\input{childdoc.def}|\\
|\childdocforward{|\textit{main}|}|
\end{tabular}
\end{center}
%
Likewise, the following files |final|\textit{nn}|.tex|
compile the final version of the child document
|child|\textit{nn}|.tex|:
%
\begin{center}
\begin{tabular}{l}
|\def\version{final}|\\
|\input{childdoc.def}|\\
|\childdocforwardprefix{final}{child}|
\end{tabular}
\end{center}
%

Note that when several versions of a main file and/or of each child file
are to be generated, it may be convenient to set up a |Makefile| or
shell script to automatise the process.

%%%%%%%%%%%%%%%%%%%%%%%%%%%%%%%%%%%%%%%%%%%%%%%%%%%%%%%%%%%%%%%%%%%%%%%%%%%%%%%%
\subsection{Command Line Processing}
\label{sec:commandline}

The effect of redirection files can also be achieved by invoking
the \LaTeX{} compiler with a more elaborate command line.
Most conveniently this should be done as part
of a shell script or a |Makefile|.

When using \textsf{childdoc} in the main file, the following
command lines effectively perform a redirection
(note that depending on the shell being used,
backslashes may have to be doubled: `|\|' $\to$ `|\\|'):
%
\begin{center}
|... -jobname "|\textit{target}|" |\\|"|[\textit{flags}]%
|\input{childdoc.def}\childdocforward[|\textit{main}|]{|\textit{dest}|}"|
\end{center}
%
Here \textit{target} is the name of the output file,
\textit{main} is the name of the main file
and \textit{dest} is the name of the main or child file to be processed
(all filenames without extensions).
The optional argument \textit{main} can be omitted
if \textit{main} matches \textit{dest}.
Optionally, compilation \textit{flags} can be defined via |\def| commands.
This command line makes the \TeX{} engine believe
it is compiling the file \textit{target}
whose content is specified as the latter parameter.
The provided code then forwards the processing to
\textit{main} or \textit{dest} as described in \secref{sec:forward}.

%%%%%%%%%%%%%%%%%%%%%%%%%%%%%%%%%%%%%%%%%%%%%%%%%%%%%%%%%%%%%%%%%%%%%%%%%%%%%%%%
\subsection{Include by Input}
\label{sec:input}

Including child documents by |\include| has some restrictions by design.
Most notably, the content of a child document always occupies
its own set of pages; pages cannot be shared between child documents.
Usually, this behaviour makes perfect sense
because each child document contain an essential part of the document.
However, in some situations it may be desirable to compose
a document from a collection of parts
without having mandatory page breaks between then.
For this case, the package
provides a mechanism to include parts
by |\input| which can also be processed individually.
However, by construction this mechanism
requires manual handling of the content to be output.

%%%%%%%%%%%%%%%%%%%%%%%%%%%%%%%%%%%%%%%%
\DescribeMacro{\ifchilddocmanual}
The main file should be prepared as usual, see \secref{sec:include}.
However, the document body must make a distinction
between processing of an individual part and of the main document, e.g.:
%
\begin{center}
\begin{tabular}{l}
|\ifchilddocmanual|\\
|\input{\childdocname}|\\
|\||else|\\
\textit{document body with }|\input{|\textit{part}|}|\\
|\||fi|
\end{tabular}
\end{center}
%
The conditional |\ifchilddocmanual| is true whenever
a part to be included by |\input| is being compiled,
and the name of the part is stored in |\childdocname|.

%%%%%%%%%%%%%%%%%%%%%%%%%%%%%%%%%%%%%%%%
\DescribeMacro{\childdocby}
Each part to be included by |\input| should start with:
%
\begin{center}
\begin{tabular}{l}
|\input{childdoc.def}|\\
|\childdocby{|\textit{main}|}|\\
\end{tabular}
\end{center}
%
The directive |\childdocby| is similar to |\childdocof|
described in \secref{sec:include},
but the subsequent selection of content must be done manually.
To that end, both |\ifchilddoc| and |\ifchilddocmanual|
will be true upon processing of a part,
and the name of the part is stored in |\childdocname|.
Note that |\jobname| will be set to the filename of the current part
so that each part receives an individual |.aux| file
that does not interfere with the |.aux| file(s) of the main document.
This behaviour can be altered by the alternative form
|\childdocby[*]{|\textit{main}|}| (with a non-empty optional argument)
which uses the |.aux| file of the main document
by setting |\jobname| to \textit{main}.

%%%%%%%%%%%%%%%%%%%%%%%%%%%%%%%%%%%%%%%%%%%%%%%%%%%%%%%%%%%%%%%%%%%%%%%%%%%%%%%%
\subsection{Driver Development}
\label{sec:driver}

The \textsf{childdoc} mechanism can also be use for the development
of definition files such as \LaTeX{} styles or classes.
This case differs from the above setup with multiple parts
included by |\include| in that no |\includeonly| should be invoked.
This can be achieved by starting the include file
(before |\ProvidesPackage|) with:
%
\begin{center}
\begin{tabular}{l}
|\input{childdoc.def}|\\
|\childdocforward{|\textit{main}|}|\\
\end{tabular}
\end{center}
%
or alternatively with:
%
\begin{center}
\begin{tabular}{l}
|\input{childdoc.def}|\\
|\childdocby{|\textit{main}|}|\\
\end{tabular}
\end{center}
%
Both forms have slightly different effects as described above.
The main file is prepared as usual, see \secref{sec:include}.

%%%%%%%%%%%%%%%%%%%%%%%%%%%%%%%%%%%%%%%%%%%%%%%%%%%%%%%%%%%%%%%%%%%%%%%%%%%%%%%%
\subsection{Legacy Detection}
\label{sec:detection}

The directive |\childdocmain| in the main file can detect
whether the complete document or merely a child is to be compiled
even without using the directive |\childdocof|.
This method is deprecated because it is less robust
and there is no compelling reason to use it;
it is merely provided for backward compatibility
and it may be removed in future versions.

If the detection mechanism is to be used,
it is mandatory to correctly specify
the filename of the main file as the argument of |\childdocmain|:
%
\begin{center}
\begin{tabular}{l}
|\input{childdoc.def}|\\
|\childdocmain{|\textit{main}|}|\\
\end{tabular}
\end{center}
%
If |\jobname| does not match the argument \textit{main} of |\childdocmain|,
it is assumed that |\jobname| points to the child file to be compiled.
When using |\childdocmain| with the main file specified as argument,
it suffices to start a child file
with just |\input{|\textit{main}|}|
without loading of the package and using |\childdocof|.
If instead all processing is done
with the appropriate \textsf{childdoc} directives,
the argument of \textit{main} of |\childdocmain| can be empty.

An alternative version of the command line processing described
in \secref{sec:commandline} using the detection mechanism reads:
%
\begin{center}
|... -jobname "|\textit{target}|" "|[\textit{flags}]%
[|\def\jobname{|\textit{dest}|}|]|\input{|\textit{main}|}"|
\end{center}

%%%%%%%%%%%%%%%%%%%%%%%%%%%%%%%%%%%%%%%%%%%%%%%%%%%%%%%%%%%%%%%%%%%%%%%%%%%%%%%%
\subsection{Manual Code}
\label{sec:manual}

In case one cannot be certain whether the definitions file |childdoc.def|
is installed on the target \TeX{} distribution
and one prefers not to ship it,
it is conceivable to paste a few relevant commands into the sources.

To that end, drop all statements |\input{childdoc.def}|
and perform the replacements as outlined below.
Instead of |\childdocmain{|\textit{main}|}| add the following code
to the top of the main file:
%
\begin{center}
\begin{tabular}{l}
|\||ifdefined\childdocname\endinput\||fi\newif\ifchilddoc|\\
|\edef\childdocname{\scantokens\expandafter{\jobname\noexpand}}|\\
|\def\childdocmain{|\textit{main}|}\||ifx\childdocmain\childdocname\||else|\\
|\childdoctrue\includeonly{\childdocname}\let\jobname\childdocmain\||fi|\\
\end{tabular}
\end{center}
%
Instead of |\childdocof{|\textit{main}|}| just include the main file
at the top of each child file:
%
\begin{center}
|\input{|\textit{main}|}|
\end{center}
%
A simple redirection |\childdocforward{|\textit{dest}|}| is achieved by:
%
\begin{center}
|\def\jobname{|\textit{dest}|}\input{\jobname}|
\end{center}
%
The redirection with prefix
|\childdocforwardprefix[|\textit{prefix}|]{|\textit{dest}|}|
is accomplished by:
%
\begin{center}
\begin{tabular}{l}
|{\edef\jobname{\scantokens\expandafter{\jobname\noexpand}}|\\
|\def\redirectjob |\textit{prefix}|#1~~~{\gdef\jobname{|\textit{dest}|#1}}|\\
|\expandafter\redirectjob\jobname~~~}\input{\jobname}|
\end{tabular}
\end{center}

In an alternative approach,
child documents can be compiled by a specific command line
without additional code or specific definitions:
%
\begin{center}
|... -jobname "|\textit{target}|" "|[\textit{flags}]%
|\includeonly{|\textit{dest}|}\input{|\textit{main}|}"|
\end{center}
%

%%%%%%%%%%%%%%%%%%%%%%%%%%%%%%%%%%%%%%%%%%%%%%%%%%%%%%%%%%%%%%%%%%%%%%%%%%%%%%%%
%%%%%%%%%%%%%%%%%%%%%%%%%%%%%%%%%%%%%%%%%%%%%%%%%%%%%%%%%%%%%%%%%%%%%%%%%%%%%%%%
\section{Information}

%%%%%%%%%%%%%%%%%%%%%%%%%%%%%%%%%%%%%%%%%%%%%%%%%%%%%%%%%%%%%%%%%%%%%%%%%%%%%%%%
\subsection{Copyright}

Copyright \copyright{} 2017--2018 Niklas Beisert

This work may be distributed and/or modified under the
conditions of the \LaTeX{} Project Public License, either version 1.3
of this license or (at your option) any later version.
The latest version of this license is in
  \url{http://www.latex-project.org/lppl.txt}
and version 1.3 or later is part of all distributions of \LaTeX{}
version 2005/12/01 or later.

This work has the LPPL maintenance status `maintained'.

The Current Maintainer of this work is Niklas Beisert.

This work consists of the files |README.txt|, |childdoc.ins| and |childdoc.dtx|
as well as the derived files |childdoc.def|, |cdocsamp.tex|
with |cdocsch1.tex|, |cdocsch2.tex|, |cdocspt3.tex|, |cdocspt4.tex|,
|cdocsdrf.tex|, |cdocsfn1.tex|, |cdocsfn2.tex|
as well as |childdoc.pdf|.

%%%%%%%%%%%%%%%%%%%%%%%%%%%%%%%%%%%%%%%%%%%%%%%%%%%%%%%%%%%%%%%%%%%%%%%%%%%%%%%%
\subsection{Files and Installation}

The package consists of the files:
%
\begin{center}
\begin{tabular}{ll}
    |README.txt|   & readme file \\
    |childdoc.ins| & installation file \\
    |childdoc.dtx| & source file \\
    |childdoc.def| & definition file \\
    |cdocsamp.tex| & sample main file \\
    |cdocsch1.tex| & sample include file \\
    |cdocsch2.tex| & sample include file \\
    |cdocspt3.tex| & sample part file \\
    |cdocspt4.tex| & sample part file \\
    |cdocsdrf.tex| & sample redirection file \\
    |cdocsfn1.tex| & sample redirection file \\
    |cdocsfn2.tex| & sample redirection file \\
    |childdoc.pdf| & manual
\end{tabular}
\end{center}
%
The distribution consists of the files
|README.txt|, |childdoc.ins| and |childdoc.dtx|.
%
\begin{itemize}
\item
Run (pdf)\LaTeX{} on |childdoc.dtx|
to compile the manual |childdoc.pdf| (this file).
\item
Run \LaTeX{} on |childdoc.ins| to create the definitions file |childdoc.def|
and the sample |cdocsamp.tex| with include files
|cdocsch1.tex|, |cdocsch2.tex|, |cdocspt3.tex|, |cdocspt4.tex|,
|cdocsdrf.tex|, |cdocsfn1.tex|, |cdocsfn2.tex|.
Then copy the file |childdoc.def| to an appropriate directory of your \LaTeX{}
distribution, e.g.\ \textit{texmf-root}|/tex/latex/childdoc|.
\end{itemize}

%%%%%%%%%%%%%%%%%%%%%%%%%%%%%%%%%%%%%%%%%%%%%%%%%%%%%%%%%%%%%%%%%%%%%%%%%%%%%%%%
\subsection{Related CTAN Packages}

There are several other packages which offer a similar functionality:
%
\begin{itemize}
\item
The packages
\href{http://ctan.org/pkg/docmute}{\textsf{docmute}},
\href{http://ctan.org/pkg/includex}{\textsf{includex}} and
\href{http://ctan.org/pkg/standalone}{\textsf{standalone}}
provide commands to include only the document body of
a child file thus allowing both files to be compiled individually.
\item
The packages \href{http://ctan.org/pkg/subdocs}{\textsf{subdocs}}
and \href{http://ctan.org/pkg/subfiles}{\textsf{subfiles}}
provide structures in which the main and child documents can be
encapsulated and allowing them to be compiled individually.
The inclusion mechanism is different from the conventional |\include|.
\item
The package \href{http://ctan.org/pkg/combine}{\textsf{combine}}
is an elaborate solution to combine several documents into one.
\end{itemize}
%
See also the CTAN topic \href{http://ctan.org/topic/subdocs}{\textsf{subdocs}}
for further related packages.
The present package differs from the above solutions in that
a document structure constructed with the conventional |\include| mechanism
just needs two extra commands at the top of every file
such that all constituent files can be compiled individually.

%%%%%%%%%%%%%%%%%%%%%%%%%%%%%%%%%%%%%%%%%%%%%%%%%%%%%%%%%%%%%%%%%%%%%%%%%%%%%%%%
%\subsection{Feature Suggestions}
%
%The following is a list of features which may be useful for future
%versions of this package:
%%
%\begin{itemize}
%\item
%\ldots
%\end{itemize}

%%%%%%%%%%%%%%%%%%%%%%%%%%%%%%%%%%%%%%%%%%%%%%%%%%%%%%%%%%%%%%%%%%%%%%%%%%%%%%%%
\subsection{Revision History}

%%%%%%%%%%%%%%%%%%%%%%%%%%%%%%%%%%%%%%%%
\paragraph{v2.0:} 2018/12/30

\begin{itemize}
\item
immediate forward processing
\item
added |\childdocby| mechanism
\item
manual restructured
\end{itemize}

%%%%%%%%%%%%%%%%%%%%%%%%%%%%%%%%%%%%%%%%
\paragraph{v1.6:} 2018/01/17

\begin{itemize}
\item
application for development of include files
\item
corrections to manual
\end{itemize}

%%%%%%%%%%%%%%%%%%%%%%%%%%%%%%%%%%%%%%%%
\paragraph{v1.5:} 2017/05/21

\begin{itemize}
\item
more complete structuring introduced
\item
|\childdocof| introduced
\item
|\childdoc| renamed to |\childdocmain|
\item
|\childredirect| renamed to |\childdocforward| and |\childdocforwardprefix|
and functionality expanded
\end{itemize}

%%%%%%%%%%%%%%%%%%%%%%%%%%%%%%%%%%%%%%%%
\paragraph{v1.0:} 2017/04/27

\begin{itemize}
\item
manual and install package
\item
first version published on CTAN
\end{itemize}

%%%%%%%%%%%%%%%%%%%%%%%%%%%%%%%%%%%%%%%%
\paragraph{v0.6:} 2017/04/26

\begin{itemize}
\item
redirection mechanism added
\end{itemize}

%%%%%%%%%%%%%%%%%%%%%%%%%%%%%%%%%%%%%%%%
\paragraph{v0.5:} 2017/04/26

\begin{itemize}
\item
functionality in definition file
\end{itemize}


%%%%%%%%%%%%%%%%%%%%%%%%%%%%%%%%%%%%%%%%%%%%%%%%%%%%%%%%%%%%%%%%%%%%%%%%%%%%%%%%
%%%%%%%%%%%%%%%%%%%%%%%%%%%%%%%%%%%%%%%%%%%%%%%%%%%%%%%%%%%%%%%%%%%%%%%%%%%%%%%%
%%%%%%%%%%%%%%%%%%%%%%%%%%%%%%%%%%%%%%%%%%%%%%%%%%%%%%%%%%%%%%%%%%%%%%%%%%%%%%%%
\appendix

\settowidth\MacroIndent{\rmfamily\scriptsize 000\ }

 \DocInput{childdoc.dtx}

\end{document}
%</driver>
% \fi
%
% %%%%%%%%%%%%%%%%%%%%%%%%%%%%%%%%%%%%%%%%%%%%%%%%%%%%%%%%%%%%%%%%%%%%%%%%%%%%%%
% %%%%%%%%%%%%%%%%%%%%%%%%%%%%%%%%%%%%%%%%%%%%%%%%%%%%%%%%%%%%%%%%%%%%%%%%%%%%%%
% \section{Sample}
%\iffalse
%<*samplemain>
%\fi
%
% The following presents a sample document
% with two chapters, two parts, a title page,
% a compile flag as well as three forwarding files to set the flag.
% It consists of eight |.tex| files:
% \begin{center}
% \begin{tabular}{ll}
% |cdocsamp.tex|&main file\\
% |cdocsch1.tex|&include file for chapter 1\\
% |cdocsch2.tex|&include file for chapter 2\\
% |cdocspt3.tex|&include file for part 3\\
% |cdocspt4.tex|&include file for part 4\\
% |cdocsdrf.tex|&forwarding file for main file in draft mode\\
% |cdocsfi1.tex|&forwarding file for final version of chapter 1\\
% |cdocsfi2.tex|&forwarding file for final version of chapter 2\\
% \end{tabular}
% \end{center}
% Each of the eight files can be compiled directly by the \LaTeX{} compiler.
%
% %%%%%%%%%%%%%%%%%%%%%%%%%%%%%%%%%%%%%%
% \paragraph{Main File.}
%
% The main file is called |cdocsamp.tex|.
%
% Load the \textsf{childdoc} definitions and
% declare the filename for the main document:
%    \begin{macrocode}
\input{childdoc.def}
\childdocmain{}
%    \end{macrocode}

% Optional override for |\version| flag:
%    \begin{macrocode}
%%\ifchilddoc\else\providecommand{\version}{draft}\fi
%    \end{macrocode}

% Define the default values for the |\version| flag
% (|final| for the main file and |draft| for childs):
%    \begin{macrocode}
\ifchilddoc
\providecommand{\version}{draft}
\else
\providecommand{\version}{final}
\fi
%    \end{macrocode}

% Load the standard document class:
%    \begin{macrocode}
\documentclass[12pt]{article}
%    \end{macrocode}

% Start the document body:
%    \begin{macrocode}
\begin{document}
%    \end{macrocode}

% Declare a title page.
% Print title, part of document being processed and version flag:
%    \begin{macrocode}
\addtocounter{page}{-1}
\begin{center}
{\LARGE\bfseries{}childdoc example\par}
\vspace{1cm}
\ifchilddoc
\ifchilddocmanual part\else chapter\fi:
`\childdocname' of `\childdocjob'\par
\else
main document: `\childdocjob'\par
\fi
version: \version\par
\end{center}
\newpage
%    \end{macrocode}

% Manually include selected file,
% otherwise process as usual:
%    \begin{macrocode}
\ifchilddocmanual
\section*{part `\childdocname'}
\input{\childdocname}
\else
%    \end{macrocode}

% Include the two chapters:
%    \begin{macrocode}
\include{cdocsch1}
\include{cdocsch2}
%    \end{macrocode}

% Include the two parts unless only chapters should be displayed:
%    \begin{macrocode}
\ifchilddoc\else
\section{part three}
\input{cdocspt3}
\section{part four}
\input{cdocspt4}
\fi
%    \end{macrocode}

% Process as usual until here:
%    \begin{macrocode}
\fi
%    \end{macrocode}

% End of document body:
%    \begin{macrocode}
\end{document}
%    \end{macrocode}
%\iffalse
%</samplemain>
%\fi
%
% %%%%%%%%%%%%%%%%%%%%%%%%%%%%%%%%%%%%%%
% \paragraph{Chapter Include Files.}
%
% The include files are called |cdocsch1.tex| and |cdocsch2.tex|.
%
%\iffalse
%<*samplechap1|samplechap2>
%\fi

% Optional override for |\version| flag:
%    \begin{macrocode}
%%\providecommand{\version}{final}
%    \end{macrocode}

% Include the main document:
%    \begin{macrocode}
\input{childdoc.def}
\childdocof{cdocsamp}
%    \end{macrocode}

%\iffalse
%</samplechap1|samplechap2>
%\fi
%
%\iffalse
%<*samplechap1>
%\fi
% Some text for chapter 1:
%    \begin{macrocode}
\section{one}
some text in chapter one
%    \end{macrocode}

%\iffalse
%</samplechap1>
%\fi
% Some text for chapter 2:
%\iffalse
%<*samplechap2>
%\fi
%    \begin{macrocode}
\section{two}
more text in chapter two
%    \end{macrocode}

%\iffalse
%</samplechap2>
%\fi
%
% %%%%%%%%%%%%%%%%%%%%%%%%%%%%%%%%%%%%%%
% \paragraph{Part Include Files.}
%
% The include files are called |cdocspt3.tex| and |cdocspt4.tex|.
%
%\iffalse
%<*samplepart3|samplepart4>
%\fi

% Optional override for |\version| flag:
%    \begin{macrocode}
%%\providecommand{\version}{final}
%    \end{macrocode}

% Include the main document:
%    \begin{macrocode}
\input{childdoc.def}
\childdocby{cdocsamp}
%    \end{macrocode}

%\iffalse
%</samplepart3|samplepart4>
%\fi
%
%\iffalse
%<*samplepart3>
%\fi
% Some text for part 3:
%    \begin{macrocode}
some text in part three
%    \end{macrocode}

%\iffalse
%</samplepart3>
%\fi
% Some text for part 4:
%\iffalse
%<*samplepart4>
%\fi
%    \begin{macrocode}
more text in part four
%    \end{macrocode}

%\iffalse
%</samplepart4>
%\fi
%
% %%%%%%%%%%%%%%%%%%%%%%%%%%%%%%%%%%%%%%
% \paragraph{Forwarding for a Complete Draft.}
%
% The following forwarding file |cdocsdrf.tex|
% compiles the main document in draft mode:
%\iffalse
%<*sampledraft>
%\fi
%    \begin{macrocode}
\def\version{draft}
\input{childdoc.def}
\childdocforward{cdocsamp}
%    \end{macrocode}

%\iffalse
%</sampledraft>
%\fi
%
% %%%%%%%%%%%%%%%%%%%%%%%%%%%%%%%%%%%%%%
% \paragraph{Forwarding for Final Version of the Chapters.}
%
% The following forwarding files |cdocsfn1.tex| and |cdocsfn2.tex|
% (with identical content)
% compile the final versions of the child documents
% |cdocsch1.tex| and |cdocsch2.tex|, respectively:
%\iffalse
%<*samplefinal>
%\fi
%    \begin{macrocode}
\def\version{final}
\input{childdoc.def}
\childdocforwardprefix[cdocsamp]{cdocsfn}{cdocsch}
%    \end{macrocode}

%\iffalse
%</samplefinal>
%\fi
%
% %%%%%%%%%%%%%%%%%%%%%%%%%%%%%%%%%%%%%%
% \paragraph{Command Line Processing.}
%
% The following three command lines generate the output files
% |cdocscld|, |cdocscl1| and |cdocscl2|
% which should be identical to
% |cdocsdrf|, |cdocsch1| and |cdocsfn2|, respectively:
% \begin{center}
% \begin{tabular}{l}
% |latex -jobname cdocscld \|\\
% |  "\def\version{draft}\input{childdoc.def}\childdocforward{cdocsamp}"|\\
% |latex -jobname cdocscl1 \|\\
% |  "\input{childdoc.def}\childdocforward[cdocsamp]{cdocsch1}"|\\
% |latex -jobname cdocscl2 \|\\
% |  "\def\version{final}\input{childdoc.def}\childdocforward{cdocsch2}"|
% \end{tabular}
% \end{center}
% Note that the trailing backslash on each first line
% merely continues the input to the second line
% (for convenient cut ant paste).
% Furthermore, the command |latex| can be replaced by any
% of its alternative versions such as |pdflatex|.
%
% %%%%%%%%%%%%%%%%%%%%%%%%%%%%%%%%%%%%%%%%%%%%%%%%%%%%%%%%%%%%%%%%%%%%%%%%%%%%%%
% %%%%%%%%%%%%%%%%%%%%%%%%%%%%%%%%%%%%%%%%%%%%%%%%%%%%%%%%%%%%%%%%%%%%%%%%%%%%%%
% \section{Implementation}
%\iffalse
%<*package>
%\fi
%
% This section describes the definitions file |childdoc.def|.

% The definitions cannot be loaded using |\usepackage| or |\RequirePackage|
% which has a mechanism to prevent loading a style file more than once.
% When loading the definitions by means of |\input|
% multiple instances have to be prevented manually:
%\iffalse
%This code needs to be before the `\ProvidesFile' directive
%which is defined at the beginning of this file.
%Therefore it is also placed there and commented out here.
%</package>
%<*discard>
%\fi
%    \begin{macrocode}
\ifdefined\childdocmain\endinput\fi
%    \end{macrocode}
%\iffalse
%</discard>
%<*package>
%\fi
%
% \macro{\ifchilddoc}
% \macro{\ifchilddocmanual}
% The conditional |\ifchilddoc| tells whether a
% child (true) or main (false) document is being compiled.
% The conditional |\ifchilddocmanual| tells whether
% the |\includeonly| mechanism is used (false) or
% the selection of child files must be performed manually (true).
% The definitions initialise to false:
%    \begin{macrocode}
\newif\ifchilddoc
\newif\ifchilddocmanual
%    \end{macrocode}

% \macro{\childdocname}
% \macro{\childdocjob}
% The macro |\childdocname| stores the name of the main document
% to be compiled. The macro |\childdocjob| stores the name of
% the document on which the \LaTeX{} compiler was originally invoked.
% The content of |\jobname| cannot be compared
% to filenames specified in the source due to different catcodes.
% The following code rescans |\jobname|, stores the result
% in |\childdocname| and saves a copy in |\childdocjob|:
%    \begin{macrocode}
\edef\childdocname{\scantokens\expandafter{\jobname\noexpand}}
\let\childdocjob\childdocname
%    \end{macrocode}

% \macro{\childdocdisable}
% The macro |\childdocdisable| prevents the main file
% from being processed more than once.
% At this stage, the main document command |\childdocmain|
% is assumed to be called once again where it should do nothing.
% Any subsequent call to it should prevent
% a secondary processing of the main document
% It overwrites the forwarding commands
% |\childdocof| and |\childdocforward|
% with empty macros to prevent further inclusions of the main document:
%    \begin{macrocode}
\newcommand{\childdocdisable}
{
  \renewcommand{\childdocmain}[1]{\renewcommand{\childdocmain}[1]{\endinput}}
  \renewcommand{\childdocof}[1]{}
  \renewcommand{\childdocby}[2][]{}
  \renewcommand{\childdocforward}[2][]{}
  \renewcommand{\childdocdisable}{}
}
%    \end{macrocode}

% \macro{\childdocmain}
% The macro |\childdocmain| is to be called at the top of the main file
% with nothing or the main filename (without extension) as argument.
% First, it breaks loops.
% If the argument is not empty and does not match |\childdocname|
% (which is set by the first inclusion of |childdoc.def|),
% |\ifchilddoc| is set to true, |\includeonly| is applied to the child file
% and |\jobname| is set to the main file
% (for proper handling of |.aux| files):
%    \begin{macrocode}
\newcommand{\childdocmain}[1]
{
  \childdocdisable\childdocmain{}
  \if?#1?\else
    \begingroup
      \def\childdoctmp{#1}
      \ifx\childdoctmp\childdocname
        \def\childdoctmp{}
      \else
        \def\childdoctmp
        {
          \childdoctrue
          \includeonly{\childdocname}
          \def\childdocjob{#1}
          \def\jobname{#1}
        }
      \fi
      \expandafter
    \endgroup
    \childdoctmp
  \fi
}
%    \end{macrocode}

% \macro{\childdocof}
% The command |\childdocof| redirects
% compilation to the main file |#1|.
%    \begin{macrocode}
\newcommand{\childdocof}[1]
{
  \childdocdisable
  \childdoctrue
  \includeonly{\childdocname}
  \def\jobname{#1}
  \def\childdocjob{#1}
  \input{#1}
}
%    \end{macrocode}

% \macro{\childdocby}
% The command |\childdocby| ....
%    \begin{macrocode}
\newcommand{\childdocby}[2][]
{
  \childdocdisable
  \childdoctrue
  \childdocmanualtrue
  \if?#1?\else
    \def\jobname{#2}
  \fi
  \def\childdocjob{#2}
  \input{#2}
  \endinput
}
%    \end{macrocode}

% \macro{\childdocforward}
% The command |\childdocforward| redirects
% compilation to the main file or
% (if the optional argument is given) a child file.
% Parameters are set as if the main file
% or a child file starting with |\childdocof| was compiled.
% Then compilation is handed over to the main file:
%    \begin{macrocode}
\newcommand{\childdocforward}[2][]
{
  \begingroup
    \if?#1?
      \def\childdoctmp
      {
        \def\childdocname{#2}
        \def\childdocjob{#2}
        \def\jobname{#2}
        \input{#2}
        \endinput
      }
    \else
      \def\childdoctmp
      {
        \childdocdisable
        \def\childdocname{#2}
        \childdoctrue
        \includeonly{#2}
        \def\childdocjob{#1}
        \def\jobname{#1}
        \input{#1}
        \endinput
      }
    \fi
    \expandafter
  \endgroup
  \childdoctmp
}
%    \end{macrocode}

% \macro{\childdocforwardprefix}
% The command |\childdocforwardprefix| redirects
% compilation to the main or a child file by means of a pattern.
% The prefix |#1| in the current filename is replaced by |#2|
% and the suffix of the current filename is kept
% (it is assumed that the filename does not contain the substring `|~~~|'
% which is used as a delimiter).
% Compilation is handed over to the new file by |\childdocforward|:
%    \begin{macrocode}
\newcommand{\childdocforwardprefix}[3][]
{
  \begingroup
    \def\childdocextract #2##1~~~{\def\childdoctmp{\childdocforward[#1]{#3##1}}}
    \expandafter\childdocextract\childdocname~~~
    \expandafter
  \endgroup
  \childdoctmp
}
%    \end{macrocode}

% \macro{\childdoc}
% The deprecated macro |\childdoc| is a legacy version of |\childdocmain|:
%    \begin{macrocode}
\newcommand{\childdoc}{\childdocmain}
%    \end{macrocode}

% \macro{\childdocredirect}
% The deprecated macro |\childdocredirect| is a legacy version
% of |\childdocforward| and |\childdocforwardprefix|:
%    \begin{macrocode}
\newcommand{\childdocredirect}[2][]
{
  \begingroup
    \if?#1?
      \def\childdoctmp{\childdocforward{#2}}
    \else
      \def\childdoctmp{\childdocforwardprefix{#1}{#2}}
    \fi
    \expandafter
  \endgroup
  \childdoctmp
}
%    \end{macrocode}

%\iffalse
%</package>
%\fi
%
\endinput
\childdocforward[cdocsamp]{cdocsch1}"|\\
% |latex -jobname cdocscl2 \|\\
% |  "\def\version{final}% \iffalse
%
% childdoc.dtx Copyright (C) 2017-2018 Niklas Beisert
%
% This work may be distributed and/or modified under the
% conditions of the LaTeX Project Public License, either version 1.3
% of this license or (at your option) any later version.
% The latest version of this license is in
%   http://www.latex-project.org/lppl.txt
% and version 1.3 or later is part of all distributions of LaTeX
% version 2005/12/01 or later.
%
% This work has the LPPL maintenance status `maintained'.
%
% The Current Maintainer of this work is Niklas Beisert.
%
% This work consists of the files childdoc.dtx and childdoc.ins
% and the derived files childdoc.def and cdocsamp.tex with
% cdocsch1.tex, cdocsch2.tex, cdocsdrf.tex, cdocsfn1.tex, cdocsfn2.tex.
%
%<package>\ifdefined\childdocmain\endinput\fi
%<package>\ProvidesFile{childdoc.def}[2018/12/30 v2.0 child document driver]
%<samplemain>\ProvidesFile{cdocsamp.tex}[2018/12/30 v2.0 sample for childdoc]
%<*driver>
%\ProvidesFile{childdoc.drv}[2018/12/30 v2.0 childdoc reference manual file]
\PassOptionsToClass{10pt,a4paper}{article}
\documentclass{ltxdoc}

\usepackage[margin=35mm]{geometry}
\usepackage{hyperref}
\usepackage{hyperxmp}
\usepackage[usenames]{color}

\hypersetup{colorlinks=true}
\hypersetup{pdfstartview=FitH}
\hypersetup{pdfpagemode=UseNone}
\hypersetup{pdfsource={}}
\hypersetup{pdflang={en-UK}}
\hypersetup{pdfcopyright={Copyright 2017-2018 Niklas Beisert.
  This work may be distributed and/or modified under the
  conditions of the LaTeX Project Public License, either version 1.3
  of this license or (at your option) any later version.}}
\hypersetup{pdflicenseurl={http://www.latex-project.org/lppl.txt}}
\hypersetup{pdfcontactaddress={ETH Zurich, ITP, HIT K,
  Wolfgang-Pauli-Strasse 27}}
\hypersetup{pdfcontactpostcode={8093}}
\hypersetup{pdfcontactcity={Zurich}}
\hypersetup{pdfcontactcountry={Switzerland}}
\hypersetup{pdfcontactemail={nbeisert@itp.phys.ethz.ch}}
\hypersetup{pdfcontacturl={http://people.phys.ethz.ch/\xmptilde nbeisert/}}

\newcommand{\secref}[1]{\hyperref[#1]{section \ref*{#1}}}

\parskip1ex
\parindent0pt
\let\olditemize\itemize
\def\itemize{\olditemize\parskip0pt}

\begin{document}

\title{The \textsf{childdoc} Package}
\hypersetup{pdftitle={The childdoc Package}}
\author{Niklas Beisert\\[2ex]
  Institut f\"ur Theoretische Physik\\
  Eidgen\"ossische Technische Hochschule Z\"urich\\
  Wolfgang-Pauli-Strasse 27, 8093 Z\"urich, Switzerland\\[1ex]
  \href{mailto:nbeisert@itp.phys.ethz.ch}
  {\texttt{nbeisert@itp.phys.ethz.ch}}}
\hypersetup{pdfauthor={Niklas Beisert}}
\hypersetup{pdfsubject={Manual for the LaTeX2e Package childdoc}}
\date{30 December 2018, \textsf{v2.0}}
\maketitle

\begin{abstract}\noindent
\textsf{childdoc} is a \LaTeXe{} package
that enables the direct compilation
of document sections included by |\include|
to individual files.
\end{abstract}

\begingroup
\parskip0ex
\tableofcontents
\endgroup

%%%%%%%%%%%%%%%%%%%%%%%%%%%%%%%%%%%%%%%%%%%%%%%%%%%%%%%%%%%%%%%%%%%%%%%%%%%%%%%%
%%%%%%%%%%%%%%%%%%%%%%%%%%%%%%%%%%%%%%%%%%%%%%%%%%%%%%%%%%%%%%%%%%%%%%%%%%%%%%%%
\section{Introduction}

\LaTeX{} provides a mechanism to structure a large document (such as a book)
into a main file and several child files (containing the chapters)
using the |\include| command.
This mechanism is beneficial for documents
which span hundreds of pages in order to
make the source file(s) more manageable.
Moreover, compilation can be restricted to
selected child files by means of the |\includeonly| command.
The latter feature can be used to reduce the compilation time while editing
(this was significantly more useful in the earlier days of \LaTeX{})
or to generate a smaller document which is easier to navigate.
Another application of |\includeonly| is to generate
documents consisting of selected parts of the complete document.

However, there are a few drawbacks of the plain |\include| mechanism:
\begin{itemize}
\item
The child files cannot be compiled on their own,
they can only be compiled via the main file.
A naive editing environment
(such as a text editor with an option
to have the current file processed by \LaTeX)
may require one to switch to the main file before compiling;
attempting to compile the child file produces errors.
\item
The main file must be modified (each time)
to adjust the |\includeonly| command
to the present needs. This easily leaves the main file in a messy state.
\item
The generated document will always carry the filename
of the main document. This is inconvenient if
several child files are to be compiled and
to be kept for distribution.
\end{itemize}

The present package provides a simple interface
to make child files individually compilable by \LaTeX{}.
Compiling a child file then has the same effect as compiling
the main file with an |\includeonly| command
to select the appropriate child.
Moreover the generated document will carry the name of the child
rather than the main file.
This resolves all three above issues.

This feature is meant to make the editing of books,
thesis documents and lecture notes somewhat more convenient.
However, the package can also be used efficiently for
composing a series of documents (such as exercise sheets)
which are typically distributed individually.
It then assists the author in generating the individual documents
(potentially in different versions)
as well as a document containing the collected series.
Another application is in developing style files
or other kinds of included material
where compilation of the style file could redirect
to a sample or test file.

%%%%%%%%%%%%%%%%%%%%%%%%%%%%%%%%%%%%%%%%%%%%%%%%%%%%%%%%%%%%%%%%%%%%%%%%%%%%%%%%
%%%%%%%%%%%%%%%%%%%%%%%%%%%%%%%%%%%%%%%%%%%%%%%%%%%%%%%%%%%%%%%%%%%%%%%%%%%%%%%%
\section{Usage}

First of all, the package \textsf{childdoc} is \emph{not} a standard
\LaTeXe{} |.sty| style file! Therefore it needs to be invoked in
a non-standard way.

%%%%%%%%%%%%%%%%%%%%%%%%%%%%%%%%%%%%%%%%%%%%%%%%%%%%%%%%%%%%%%%%%%%%%%%%%%%%%%%%
\subsection{Included Files}
\label{sec:include}

%%%%%%%%%%%%%%%%%%%%%%%%%%%%%%%%%%%%%%%%
\DescribeMacro{\childdocmain}
To use the package, add the commands
\begin{center}
\begin{tabular}{l}
|\input{childdoc.def}|\\
|\childdocmain{}|\\
\end{tabular}
\end{center}
at the very top of the main \LaTeX{} file,
in particular \emph{before} the |\documentclass| statement!
The argument of |\childdocmain| should be left empty
(but it must be present).

%%%%%%%%%%%%%%%%%%%%%%%%%%%%%%%%%%%%%%%%
\DescribeMacro{\childdocof}
Furthermore, add the commands
\begin{center}
\begin{tabular}{l}
|\input{childdoc.def}|\\
|\childdocof{|\textit{main}|}|\\
\end{tabular}
\end{center}
at the top of every child file \textit{child}
which is included by |\include{|\textit{child}|}|
from within the main file
(or at least for those files to be compiled individually).
The argument \textit{main} must be the filename of the main file.

There are a couple of
considerations in setting up the main and child documents:

%%%%%%%%%%%%%%%%%%%%%%%%%%%%%%%%%%%%%%%%
\paragraph{Restrictions.}

Please note the following restrictions:
\begin{itemize}
\item
|\childdocmain| must be called with one argument \textit{main}
to ensure compatibility with earlier version of the package.
It must either be empty (|\childdocmain{}|)
or precisely match the filename of the main file in which it is specified.
See \secref{sec:detection} for further information.
\item
The filename \textit{main} must be specified without the |.tex| extension.
\item
The filename \textit{main} is case sensitive
(even in case-insensitive file systems)
due to internal string comparison.
\item
The argument \textit{main} should be fully expanded, it cannot be a macro.
\item
Subdirectories and special characters should be avoided in filenames.
\item
The command |\childdocmain{|\textit{main}|}| must be followed by a whitespace.
It should not be followed immediately by another command
or by a comment mark `|%|'.
This is because the \TeX{} parser reads the token immediately following
the argument of |\childdocmain| and puts it
at the beginning of every child section;
however, a white\-space is ignored.
\end{itemize}

%%%%%%%%%%%%%%%%%%%%%%%%%%%%%%%%%%%%%%%%
\paragraph{Content of Main File.}

It is advisable to place all content in the child files included by |\include|.
Any output contained in the main file will appear in all child documents
unless suppressed manually;
it cannot be suppressed automatically by the |\includeonly| directive
and thus should normally be avoided.
A method to include some content in the main file
by means of conditional processing is described in \secref{sec:conditional}.

%%%%%%%%%%%%%%%%%%%%%%%%%%%%%%%%%%%%%%%%
\paragraph{Page Numbering.}

When only a part of the document is compiled,
the appropriate numbering of pages
(as well as other status parameters)
is determined from the |.aux| files.
The latter contain information from previous passes.
However this information needs to propagate through
all intermediate child documents.
Therefore the page numbering in child documents may well
be inconsistent until the complete document is compiled at least once.

A useful (if unconventional) way to always ensure a consistent
page numbering is to restart the numbering in each child document
and denote the pages by `\textit{child}|.|\textit{page}'
where \textit{child} represents the chapter/section number of the child file.
This can be achieved by the command
|\numberwithin{page}{|\textit{child}|}|
of the \textsf{amsmath} package
where \textit{child} can be |chapter| or |section|
depending on the chosen structuring.
Alternatively, one can modify the macro |\thepage| appropriately
and reset the counter |page| at the start of each child file.

%%%%%%%%%%%%%%%%%%%%%%%%%%%%%%%%%%%%%%%%%%%%%%%%%%%%%%%%%%%%%%%%%%%%%%%%%%%%%%%%
\subsection{Conditional Processing}
\label{sec:conditional}

The package provides a mechanism to compile different versions
of a document. To customise the versions further some conditional processing
can come in handy to distinguish which version is being compiled.
The package provides two macros to describe the compilation context:

%%%%%%%%%%%%%%%%%%%%%%%%%%%%%%%%%%%%%%%%
\DescribeMacro{\ifchilddoc}
The conditional |\ifchilddoc| distinguishes between the compilation of
child documents and the main document:
%
\begin{center}
|\ifchilddoc |\textit{child-code}| |[|\||else |\textit{main-code}]| \||fi|
\end{center}

%%%%%%%%%%%%%%%%%%%%%%%%%%%%%%%%%%%%%%%%
\DescribeMacro{\childdocname}
\DescribeMacro{\childdocjob}
The macro |\childdocname| contains the filename (without extension)
of the main or child file being processed.
Note that |\childdocjob| will always contain the name of the main file.

%%%%%%%%%%%%%%%%%%%%%%%%%%%%%%%%%%%%%%%%
\paragraph{Title Page.}

Conditional processing can be used to include a title or banner page
in the main document when proper precautions are taken.
Importantly, the code in the main file should ensure that the page counter
(as well as other status parameters which are stored in the |.aux| files)
takes the same value after the conditional processing.
Otherwise the page numbers may take divergent values
depending on which part is compiled.

For example, a title page could be declared by:
%
\begin{center}
\begin{tabular}{l}
|\ifchilddoc\||else|\\
|\addtocounter{page}{-1}|\\
\textit{code for title page}\\
|\newpage|\\
|\||fi|
\end{tabular}
\end{center}
%
A banner page for the child documents can be generated by:
%
\begin{center}
\begin{tabular}{l}
|\ifchilddoc|\\
|\addtocounter{page}{-1}|\\
\textit{code for banner page}\\
|\newpage|\\
|\||fi|
\end{tabular}
\end{center}
%
Here one could write a message such as:
\begin{center}
|This is the part \childdocname{} of \childdocjob{}.|
\end{center}

%%%%%%%%%%%%%%%%%%%%%%%%%%%%%%%%%%%%%%%%%%%%%%%%%%%%%%%%%%%%%%%%%%%%%%%%%%%%%%%%
\subsection{Flags}
\label{sec:flags}

The package makes it easy to generate different versions
of the main or child documents.
To this end compilation flags can be defined
and assigned different default values.
They will be particularly useful in conjunction
with the forwarding mechanism described in \secref{sec:forward}.

For example, it may be useful to have a flag |\version|
which can be set to |draft| or |final|.
The document source will contain some conditional code
depending on the value of |\version|.
Suppose further, the flag should default to |final| for the main file
and to |draft| for child files
which is a natural assignment for editing the document.
This is achieved by placing the following code
in the preamble of the main document
(below the |\childdocmain| directive):
%
\begin{center}
\begin{tabular}{l}
|\ifchilddoc|\\
|\providecommand{\version}{draft}|\\
|\||else|\\
|\providecommand{\version}{final}|\\
|\||fi|
\end{tabular}
\end{center}
%
The definition by |\providecommand| makes sure
that previous definitions are not overwritten.
Further statements |\providecommand{\version}{...}|
can thus be added before the above code to override it.

For the main file, one might add a line
(between |\childdocmain| and the above block)
%
\begin{center}
|%\ifchilddoc\||else\providecommand{\version}{draft}\||fi|
\end{center}
%
which can be uncommented to produce a draft version.
Likewise one can add a line to the very top of a child file
(above the |\childdocof{|\textit{main}|}| directive)
%
\begin{center}
|%\providecommand{\version}{final}|
\end{center}
%
which can be uncommented to produce the final version of this child document.

%%%%%%%%%%%%%%%%%%%%%%%%%%%%%%%%%%%%%%%%%%%%%%%%%%%%%%%%%%%%%%%%%%%%%%%%%%%%%%%%
\subsection{Forwarding}
\label{sec:forward}

Different versions of the main or child documents
using compilation flags as described in \secref{sec:flags}
can be (permanently) stored in different files
for convenient compilation, viewing and distribution.
To this end, the package defines a command
to pass on compilation to a different file:

%%%%%%%%%%%%%%%%%%%%%%%%%%%%%%%%%%%%%%%%
\DescribeMacro{\childdocforward}
The command |\childdocforward| redirects processing to
another source file:
%
\begin{center}
\begin{tabular}{l}
|\input{childdoc.def}|\\
|\childdocforward[|\textit{main}|]{|\textit{dest}|}|\\
\end{tabular}
\end{center}
%
The argument \textit{dest} is the destination file
(without extension).
It should be the main file or one of the child files.
Note that further \textsf{childdoc} directives
such as |\childdocof| and |\childdocforward|
in the indicated file will be processed in this form.
The optional argument \textit{main}
passes on directly to the main file \textit{main}
while pretending to compile the child \textit{dest}.
This form behaves as if \textit{dest}
issues |\childdocof{|\textit{main}|}| right away,
and no further \textsf{childdoc} directives will be processed.

%%%%%%%%%%%%%%%%%%%%%%%%%%%%%%%%%%%%%%%%
\DescribeMacro{\...prefix}
In the alternative form |\childdocforwardprefix|,
%
\begin{center}
\begin{tabular}{l}
|\input{childdoc.def}|\\
|\childdocforwardprefix[|\textit{main}|]{|\textit{prefix}|}{|\textit{dest}|}|
\end{tabular}
\end{center}
%
the destination file is determined by a pattern
depending on the current file:
To make this work, the current file must be called
`{\textit{prefix}\hspace{0.2em}\textit{suffix}}'
with \textit{prefix} matching precisely the argument.
Processing is then passed on to the file
`{\textit{dest}\hspace{0.2em}\textit{suffix}}'.
Surely, the same effect is achieved by
directly specifying the
argument `{\textit{dest}\hspace{0.2em}\textit{suffix}}'
in the first form.
However, that requires to set up a different file
for each child. With the alternative form of the command
all these files can have exactly the same content
which simplifies setting them up and maintaining them.

For example, the following file |draft.tex|
with a compilation flag |\version| as described in \secref{sec:flags}
compiles the main document as a draft:
%
\begin{center}
\begin{tabular}{l}
|\def\version{draft}|\\
|\input{childdoc.def}|\\
|\childdocforward{|\textit{main}|}|
\end{tabular}
\end{center}
%
Likewise, the following files |final|\textit{nn}|.tex|
compile the final version of the child document
|child|\textit{nn}|.tex|:
%
\begin{center}
\begin{tabular}{l}
|\def\version{final}|\\
|\input{childdoc.def}|\\
|\childdocforwardprefix{final}{child}|
\end{tabular}
\end{center}
%

Note that when several versions of a main file and/or of each child file
are to be generated, it may be convenient to set up a |Makefile| or
shell script to automatise the process.

%%%%%%%%%%%%%%%%%%%%%%%%%%%%%%%%%%%%%%%%%%%%%%%%%%%%%%%%%%%%%%%%%%%%%%%%%%%%%%%%
\subsection{Command Line Processing}
\label{sec:commandline}

The effect of redirection files can also be achieved by invoking
the \LaTeX{} compiler with a more elaborate command line.
Most conveniently this should be done as part
of a shell script or a |Makefile|.

When using \textsf{childdoc} in the main file, the following
command lines effectively perform a redirection
(note that depending on the shell being used,
backslashes may have to be doubled: `|\|' $\to$ `|\\|'):
%
\begin{center}
|... -jobname "|\textit{target}|" |\\|"|[\textit{flags}]%
|\input{childdoc.def}\childdocforward[|\textit{main}|]{|\textit{dest}|}"|
\end{center}
%
Here \textit{target} is the name of the output file,
\textit{main} is the name of the main file
and \textit{dest} is the name of the main or child file to be processed
(all filenames without extensions).
The optional argument \textit{main} can be omitted
if \textit{main} matches \textit{dest}.
Optionally, compilation \textit{flags} can be defined via |\def| commands.
This command line makes the \TeX{} engine believe
it is compiling the file \textit{target}
whose content is specified as the latter parameter.
The provided code then forwards the processing to
\textit{main} or \textit{dest} as described in \secref{sec:forward}.

%%%%%%%%%%%%%%%%%%%%%%%%%%%%%%%%%%%%%%%%%%%%%%%%%%%%%%%%%%%%%%%%%%%%%%%%%%%%%%%%
\subsection{Include by Input}
\label{sec:input}

Including child documents by |\include| has some restrictions by design.
Most notably, the content of a child document always occupies
its own set of pages; pages cannot be shared between child documents.
Usually, this behaviour makes perfect sense
because each child document contain an essential part of the document.
However, in some situations it may be desirable to compose
a document from a collection of parts
without having mandatory page breaks between then.
For this case, the package
provides a mechanism to include parts
by |\input| which can also be processed individually.
However, by construction this mechanism
requires manual handling of the content to be output.

%%%%%%%%%%%%%%%%%%%%%%%%%%%%%%%%%%%%%%%%
\DescribeMacro{\ifchilddocmanual}
The main file should be prepared as usual, see \secref{sec:include}.
However, the document body must make a distinction
between processing of an individual part and of the main document, e.g.:
%
\begin{center}
\begin{tabular}{l}
|\ifchilddocmanual|\\
|\input{\childdocname}|\\
|\||else|\\
\textit{document body with }|\input{|\textit{part}|}|\\
|\||fi|
\end{tabular}
\end{center}
%
The conditional |\ifchilddocmanual| is true whenever
a part to be included by |\input| is being compiled,
and the name of the part is stored in |\childdocname|.

%%%%%%%%%%%%%%%%%%%%%%%%%%%%%%%%%%%%%%%%
\DescribeMacro{\childdocby}
Each part to be included by |\input| should start with:
%
\begin{center}
\begin{tabular}{l}
|\input{childdoc.def}|\\
|\childdocby{|\textit{main}|}|\\
\end{tabular}
\end{center}
%
The directive |\childdocby| is similar to |\childdocof|
described in \secref{sec:include},
but the subsequent selection of content must be done manually.
To that end, both |\ifchilddoc| and |\ifchilddocmanual|
will be true upon processing of a part,
and the name of the part is stored in |\childdocname|.
Note that |\jobname| will be set to the filename of the current part
so that each part receives an individual |.aux| file
that does not interfere with the |.aux| file(s) of the main document.
This behaviour can be altered by the alternative form
|\childdocby[*]{|\textit{main}|}| (with a non-empty optional argument)
which uses the |.aux| file of the main document
by setting |\jobname| to \textit{main}.

%%%%%%%%%%%%%%%%%%%%%%%%%%%%%%%%%%%%%%%%%%%%%%%%%%%%%%%%%%%%%%%%%%%%%%%%%%%%%%%%
\subsection{Driver Development}
\label{sec:driver}

The \textsf{childdoc} mechanism can also be use for the development
of definition files such as \LaTeX{} styles or classes.
This case differs from the above setup with multiple parts
included by |\include| in that no |\includeonly| should be invoked.
This can be achieved by starting the include file
(before |\ProvidesPackage|) with:
%
\begin{center}
\begin{tabular}{l}
|\input{childdoc.def}|\\
|\childdocforward{|\textit{main}|}|\\
\end{tabular}
\end{center}
%
or alternatively with:
%
\begin{center}
\begin{tabular}{l}
|\input{childdoc.def}|\\
|\childdocby{|\textit{main}|}|\\
\end{tabular}
\end{center}
%
Both forms have slightly different effects as described above.
The main file is prepared as usual, see \secref{sec:include}.

%%%%%%%%%%%%%%%%%%%%%%%%%%%%%%%%%%%%%%%%%%%%%%%%%%%%%%%%%%%%%%%%%%%%%%%%%%%%%%%%
\subsection{Legacy Detection}
\label{sec:detection}

The directive |\childdocmain| in the main file can detect
whether the complete document or merely a child is to be compiled
even without using the directive |\childdocof|.
This method is deprecated because it is less robust
and there is no compelling reason to use it;
it is merely provided for backward compatibility
and it may be removed in future versions.

If the detection mechanism is to be used,
it is mandatory to correctly specify
the filename of the main file as the argument of |\childdocmain|:
%
\begin{center}
\begin{tabular}{l}
|\input{childdoc.def}|\\
|\childdocmain{|\textit{main}|}|\\
\end{tabular}
\end{center}
%
If |\jobname| does not match the argument \textit{main} of |\childdocmain|,
it is assumed that |\jobname| points to the child file to be compiled.
When using |\childdocmain| with the main file specified as argument,
it suffices to start a child file
with just |\input{|\textit{main}|}|
without loading of the package and using |\childdocof|.
If instead all processing is done
with the appropriate \textsf{childdoc} directives,
the argument of \textit{main} of |\childdocmain| can be empty.

An alternative version of the command line processing described
in \secref{sec:commandline} using the detection mechanism reads:
%
\begin{center}
|... -jobname "|\textit{target}|" "|[\textit{flags}]%
[|\def\jobname{|\textit{dest}|}|]|\input{|\textit{main}|}"|
\end{center}

%%%%%%%%%%%%%%%%%%%%%%%%%%%%%%%%%%%%%%%%%%%%%%%%%%%%%%%%%%%%%%%%%%%%%%%%%%%%%%%%
\subsection{Manual Code}
\label{sec:manual}

In case one cannot be certain whether the definitions file |childdoc.def|
is installed on the target \TeX{} distribution
and one prefers not to ship it,
it is conceivable to paste a few relevant commands into the sources.

To that end, drop all statements |\input{childdoc.def}|
and perform the replacements as outlined below.
Instead of |\childdocmain{|\textit{main}|}| add the following code
to the top of the main file:
%
\begin{center}
\begin{tabular}{l}
|\||ifdefined\childdocname\endinput\||fi\newif\ifchilddoc|\\
|\edef\childdocname{\scantokens\expandafter{\jobname\noexpand}}|\\
|\def\childdocmain{|\textit{main}|}\||ifx\childdocmain\childdocname\||else|\\
|\childdoctrue\includeonly{\childdocname}\let\jobname\childdocmain\||fi|\\
\end{tabular}
\end{center}
%
Instead of |\childdocof{|\textit{main}|}| just include the main file
at the top of each child file:
%
\begin{center}
|\input{|\textit{main}|}|
\end{center}
%
A simple redirection |\childdocforward{|\textit{dest}|}| is achieved by:
%
\begin{center}
|\def\jobname{|\textit{dest}|}\input{\jobname}|
\end{center}
%
The redirection with prefix
|\childdocforwardprefix[|\textit{prefix}|]{|\textit{dest}|}|
is accomplished by:
%
\begin{center}
\begin{tabular}{l}
|{\edef\jobname{\scantokens\expandafter{\jobname\noexpand}}|\\
|\def\redirectjob |\textit{prefix}|#1~~~{\gdef\jobname{|\textit{dest}|#1}}|\\
|\expandafter\redirectjob\jobname~~~}\input{\jobname}|
\end{tabular}
\end{center}

In an alternative approach,
child documents can be compiled by a specific command line
without additional code or specific definitions:
%
\begin{center}
|... -jobname "|\textit{target}|" "|[\textit{flags}]%
|\includeonly{|\textit{dest}|}\input{|\textit{main}|}"|
\end{center}
%

%%%%%%%%%%%%%%%%%%%%%%%%%%%%%%%%%%%%%%%%%%%%%%%%%%%%%%%%%%%%%%%%%%%%%%%%%%%%%%%%
%%%%%%%%%%%%%%%%%%%%%%%%%%%%%%%%%%%%%%%%%%%%%%%%%%%%%%%%%%%%%%%%%%%%%%%%%%%%%%%%
\section{Information}

%%%%%%%%%%%%%%%%%%%%%%%%%%%%%%%%%%%%%%%%%%%%%%%%%%%%%%%%%%%%%%%%%%%%%%%%%%%%%%%%
\subsection{Copyright}

Copyright \copyright{} 2017--2018 Niklas Beisert

This work may be distributed and/or modified under the
conditions of the \LaTeX{} Project Public License, either version 1.3
of this license or (at your option) any later version.
The latest version of this license is in
  \url{http://www.latex-project.org/lppl.txt}
and version 1.3 or later is part of all distributions of \LaTeX{}
version 2005/12/01 or later.

This work has the LPPL maintenance status `maintained'.

The Current Maintainer of this work is Niklas Beisert.

This work consists of the files |README.txt|, |childdoc.ins| and |childdoc.dtx|
as well as the derived files |childdoc.def|, |cdocsamp.tex|
with |cdocsch1.tex|, |cdocsch2.tex|, |cdocspt3.tex|, |cdocspt4.tex|,
|cdocsdrf.tex|, |cdocsfn1.tex|, |cdocsfn2.tex|
as well as |childdoc.pdf|.

%%%%%%%%%%%%%%%%%%%%%%%%%%%%%%%%%%%%%%%%%%%%%%%%%%%%%%%%%%%%%%%%%%%%%%%%%%%%%%%%
\subsection{Files and Installation}

The package consists of the files:
%
\begin{center}
\begin{tabular}{ll}
    |README.txt|   & readme file \\
    |childdoc.ins| & installation file \\
    |childdoc.dtx| & source file \\
    |childdoc.def| & definition file \\
    |cdocsamp.tex| & sample main file \\
    |cdocsch1.tex| & sample include file \\
    |cdocsch2.tex| & sample include file \\
    |cdocspt3.tex| & sample part file \\
    |cdocspt4.tex| & sample part file \\
    |cdocsdrf.tex| & sample redirection file \\
    |cdocsfn1.tex| & sample redirection file \\
    |cdocsfn2.tex| & sample redirection file \\
    |childdoc.pdf| & manual
\end{tabular}
\end{center}
%
The distribution consists of the files
|README.txt|, |childdoc.ins| and |childdoc.dtx|.
%
\begin{itemize}
\item
Run (pdf)\LaTeX{} on |childdoc.dtx|
to compile the manual |childdoc.pdf| (this file).
\item
Run \LaTeX{} on |childdoc.ins| to create the definitions file |childdoc.def|
and the sample |cdocsamp.tex| with include files
|cdocsch1.tex|, |cdocsch2.tex|, |cdocspt3.tex|, |cdocspt4.tex|,
|cdocsdrf.tex|, |cdocsfn1.tex|, |cdocsfn2.tex|.
Then copy the file |childdoc.def| to an appropriate directory of your \LaTeX{}
distribution, e.g.\ \textit{texmf-root}|/tex/latex/childdoc|.
\end{itemize}

%%%%%%%%%%%%%%%%%%%%%%%%%%%%%%%%%%%%%%%%%%%%%%%%%%%%%%%%%%%%%%%%%%%%%%%%%%%%%%%%
\subsection{Related CTAN Packages}

There are several other packages which offer a similar functionality:
%
\begin{itemize}
\item
The packages
\href{http://ctan.org/pkg/docmute}{\textsf{docmute}},
\href{http://ctan.org/pkg/includex}{\textsf{includex}} and
\href{http://ctan.org/pkg/standalone}{\textsf{standalone}}
provide commands to include only the document body of
a child file thus allowing both files to be compiled individually.
\item
The packages \href{http://ctan.org/pkg/subdocs}{\textsf{subdocs}}
and \href{http://ctan.org/pkg/subfiles}{\textsf{subfiles}}
provide structures in which the main and child documents can be
encapsulated and allowing them to be compiled individually.
The inclusion mechanism is different from the conventional |\include|.
\item
The package \href{http://ctan.org/pkg/combine}{\textsf{combine}}
is an elaborate solution to combine several documents into one.
\end{itemize}
%
See also the CTAN topic \href{http://ctan.org/topic/subdocs}{\textsf{subdocs}}
for further related packages.
The present package differs from the above solutions in that
a document structure constructed with the conventional |\include| mechanism
just needs two extra commands at the top of every file
such that all constituent files can be compiled individually.

%%%%%%%%%%%%%%%%%%%%%%%%%%%%%%%%%%%%%%%%%%%%%%%%%%%%%%%%%%%%%%%%%%%%%%%%%%%%%%%%
%\subsection{Feature Suggestions}
%
%The following is a list of features which may be useful for future
%versions of this package:
%%
%\begin{itemize}
%\item
%\ldots
%\end{itemize}

%%%%%%%%%%%%%%%%%%%%%%%%%%%%%%%%%%%%%%%%%%%%%%%%%%%%%%%%%%%%%%%%%%%%%%%%%%%%%%%%
\subsection{Revision History}

%%%%%%%%%%%%%%%%%%%%%%%%%%%%%%%%%%%%%%%%
\paragraph{v2.0:} 2018/12/30

\begin{itemize}
\item
immediate forward processing
\item
added |\childdocby| mechanism
\item
manual restructured
\end{itemize}

%%%%%%%%%%%%%%%%%%%%%%%%%%%%%%%%%%%%%%%%
\paragraph{v1.6:} 2018/01/17

\begin{itemize}
\item
application for development of include files
\item
corrections to manual
\end{itemize}

%%%%%%%%%%%%%%%%%%%%%%%%%%%%%%%%%%%%%%%%
\paragraph{v1.5:} 2017/05/21

\begin{itemize}
\item
more complete structuring introduced
\item
|\childdocof| introduced
\item
|\childdoc| renamed to |\childdocmain|
\item
|\childredirect| renamed to |\childdocforward| and |\childdocforwardprefix|
and functionality expanded
\end{itemize}

%%%%%%%%%%%%%%%%%%%%%%%%%%%%%%%%%%%%%%%%
\paragraph{v1.0:} 2017/04/27

\begin{itemize}
\item
manual and install package
\item
first version published on CTAN
\end{itemize}

%%%%%%%%%%%%%%%%%%%%%%%%%%%%%%%%%%%%%%%%
\paragraph{v0.6:} 2017/04/26

\begin{itemize}
\item
redirection mechanism added
\end{itemize}

%%%%%%%%%%%%%%%%%%%%%%%%%%%%%%%%%%%%%%%%
\paragraph{v0.5:} 2017/04/26

\begin{itemize}
\item
functionality in definition file
\end{itemize}


%%%%%%%%%%%%%%%%%%%%%%%%%%%%%%%%%%%%%%%%%%%%%%%%%%%%%%%%%%%%%%%%%%%%%%%%%%%%%%%%
%%%%%%%%%%%%%%%%%%%%%%%%%%%%%%%%%%%%%%%%%%%%%%%%%%%%%%%%%%%%%%%%%%%%%%%%%%%%%%%%
%%%%%%%%%%%%%%%%%%%%%%%%%%%%%%%%%%%%%%%%%%%%%%%%%%%%%%%%%%%%%%%%%%%%%%%%%%%%%%%%
\appendix

\settowidth\MacroIndent{\rmfamily\scriptsize 000\ }

 \DocInput{childdoc.dtx}

\end{document}
%</driver>
% \fi
%
% %%%%%%%%%%%%%%%%%%%%%%%%%%%%%%%%%%%%%%%%%%%%%%%%%%%%%%%%%%%%%%%%%%%%%%%%%%%%%%
% %%%%%%%%%%%%%%%%%%%%%%%%%%%%%%%%%%%%%%%%%%%%%%%%%%%%%%%%%%%%%%%%%%%%%%%%%%%%%%
% \section{Sample}
%\iffalse
%<*samplemain>
%\fi
%
% The following presents a sample document
% with two chapters, two parts, a title page,
% a compile flag as well as three forwarding files to set the flag.
% It consists of eight |.tex| files:
% \begin{center}
% \begin{tabular}{ll}
% |cdocsamp.tex|&main file\\
% |cdocsch1.tex|&include file for chapter 1\\
% |cdocsch2.tex|&include file for chapter 2\\
% |cdocspt3.tex|&include file for part 3\\
% |cdocspt4.tex|&include file for part 4\\
% |cdocsdrf.tex|&forwarding file for main file in draft mode\\
% |cdocsfi1.tex|&forwarding file for final version of chapter 1\\
% |cdocsfi2.tex|&forwarding file for final version of chapter 2\\
% \end{tabular}
% \end{center}
% Each of the eight files can be compiled directly by the \LaTeX{} compiler.
%
% %%%%%%%%%%%%%%%%%%%%%%%%%%%%%%%%%%%%%%
% \paragraph{Main File.}
%
% The main file is called |cdocsamp.tex|.
%
% Load the \textsf{childdoc} definitions and
% declare the filename for the main document:
%    \begin{macrocode}
\input{childdoc.def}
\childdocmain{}
%    \end{macrocode}

% Optional override for |\version| flag:
%    \begin{macrocode}
%%\ifchilddoc\else\providecommand{\version}{draft}\fi
%    \end{macrocode}

% Define the default values for the |\version| flag
% (|final| for the main file and |draft| for childs):
%    \begin{macrocode}
\ifchilddoc
\providecommand{\version}{draft}
\else
\providecommand{\version}{final}
\fi
%    \end{macrocode}

% Load the standard document class:
%    \begin{macrocode}
\documentclass[12pt]{article}
%    \end{macrocode}

% Start the document body:
%    \begin{macrocode}
\begin{document}
%    \end{macrocode}

% Declare a title page.
% Print title, part of document being processed and version flag:
%    \begin{macrocode}
\addtocounter{page}{-1}
\begin{center}
{\LARGE\bfseries{}childdoc example\par}
\vspace{1cm}
\ifchilddoc
\ifchilddocmanual part\else chapter\fi:
`\childdocname' of `\childdocjob'\par
\else
main document: `\childdocjob'\par
\fi
version: \version\par
\end{center}
\newpage
%    \end{macrocode}

% Manually include selected file,
% otherwise process as usual:
%    \begin{macrocode}
\ifchilddocmanual
\section*{part `\childdocname'}
\input{\childdocname}
\else
%    \end{macrocode}

% Include the two chapters:
%    \begin{macrocode}
\include{cdocsch1}
\include{cdocsch2}
%    \end{macrocode}

% Include the two parts unless only chapters should be displayed:
%    \begin{macrocode}
\ifchilddoc\else
\section{part three}
\input{cdocspt3}
\section{part four}
\input{cdocspt4}
\fi
%    \end{macrocode}

% Process as usual until here:
%    \begin{macrocode}
\fi
%    \end{macrocode}

% End of document body:
%    \begin{macrocode}
\end{document}
%    \end{macrocode}
%\iffalse
%</samplemain>
%\fi
%
% %%%%%%%%%%%%%%%%%%%%%%%%%%%%%%%%%%%%%%
% \paragraph{Chapter Include Files.}
%
% The include files are called |cdocsch1.tex| and |cdocsch2.tex|.
%
%\iffalse
%<*samplechap1|samplechap2>
%\fi

% Optional override for |\version| flag:
%    \begin{macrocode}
%%\providecommand{\version}{final}
%    \end{macrocode}

% Include the main document:
%    \begin{macrocode}
\input{childdoc.def}
\childdocof{cdocsamp}
%    \end{macrocode}

%\iffalse
%</samplechap1|samplechap2>
%\fi
%
%\iffalse
%<*samplechap1>
%\fi
% Some text for chapter 1:
%    \begin{macrocode}
\section{one}
some text in chapter one
%    \end{macrocode}

%\iffalse
%</samplechap1>
%\fi
% Some text for chapter 2:
%\iffalse
%<*samplechap2>
%\fi
%    \begin{macrocode}
\section{two}
more text in chapter two
%    \end{macrocode}

%\iffalse
%</samplechap2>
%\fi
%
% %%%%%%%%%%%%%%%%%%%%%%%%%%%%%%%%%%%%%%
% \paragraph{Part Include Files.}
%
% The include files are called |cdocspt3.tex| and |cdocspt4.tex|.
%
%\iffalse
%<*samplepart3|samplepart4>
%\fi

% Optional override for |\version| flag:
%    \begin{macrocode}
%%\providecommand{\version}{final}
%    \end{macrocode}

% Include the main document:
%    \begin{macrocode}
\input{childdoc.def}
\childdocby{cdocsamp}
%    \end{macrocode}

%\iffalse
%</samplepart3|samplepart4>
%\fi
%
%\iffalse
%<*samplepart3>
%\fi
% Some text for part 3:
%    \begin{macrocode}
some text in part three
%    \end{macrocode}

%\iffalse
%</samplepart3>
%\fi
% Some text for part 4:
%\iffalse
%<*samplepart4>
%\fi
%    \begin{macrocode}
more text in part four
%    \end{macrocode}

%\iffalse
%</samplepart4>
%\fi
%
% %%%%%%%%%%%%%%%%%%%%%%%%%%%%%%%%%%%%%%
% \paragraph{Forwarding for a Complete Draft.}
%
% The following forwarding file |cdocsdrf.tex|
% compiles the main document in draft mode:
%\iffalse
%<*sampledraft>
%\fi
%    \begin{macrocode}
\def\version{draft}
\input{childdoc.def}
\childdocforward{cdocsamp}
%    \end{macrocode}

%\iffalse
%</sampledraft>
%\fi
%
% %%%%%%%%%%%%%%%%%%%%%%%%%%%%%%%%%%%%%%
% \paragraph{Forwarding for Final Version of the Chapters.}
%
% The following forwarding files |cdocsfn1.tex| and |cdocsfn2.tex|
% (with identical content)
% compile the final versions of the child documents
% |cdocsch1.tex| and |cdocsch2.tex|, respectively:
%\iffalse
%<*samplefinal>
%\fi
%    \begin{macrocode}
\def\version{final}
\input{childdoc.def}
\childdocforwardprefix[cdocsamp]{cdocsfn}{cdocsch}
%    \end{macrocode}

%\iffalse
%</samplefinal>
%\fi
%
% %%%%%%%%%%%%%%%%%%%%%%%%%%%%%%%%%%%%%%
% \paragraph{Command Line Processing.}
%
% The following three command lines generate the output files
% |cdocscld|, |cdocscl1| and |cdocscl2|
% which should be identical to
% |cdocsdrf|, |cdocsch1| and |cdocsfn2|, respectively:
% \begin{center}
% \begin{tabular}{l}
% |latex -jobname cdocscld \|\\
% |  "\def\version{draft}\input{childdoc.def}\childdocforward{cdocsamp}"|\\
% |latex -jobname cdocscl1 \|\\
% |  "\input{childdoc.def}\childdocforward[cdocsamp]{cdocsch1}"|\\
% |latex -jobname cdocscl2 \|\\
% |  "\def\version{final}\input{childdoc.def}\childdocforward{cdocsch2}"|
% \end{tabular}
% \end{center}
% Note that the trailing backslash on each first line
% merely continues the input to the second line
% (for convenient cut ant paste).
% Furthermore, the command |latex| can be replaced by any
% of its alternative versions such as |pdflatex|.
%
% %%%%%%%%%%%%%%%%%%%%%%%%%%%%%%%%%%%%%%%%%%%%%%%%%%%%%%%%%%%%%%%%%%%%%%%%%%%%%%
% %%%%%%%%%%%%%%%%%%%%%%%%%%%%%%%%%%%%%%%%%%%%%%%%%%%%%%%%%%%%%%%%%%%%%%%%%%%%%%
% \section{Implementation}
%\iffalse
%<*package>
%\fi
%
% This section describes the definitions file |childdoc.def|.

% The definitions cannot be loaded using |\usepackage| or |\RequirePackage|
% which has a mechanism to prevent loading a style file more than once.
% When loading the definitions by means of |\input|
% multiple instances have to be prevented manually:
%\iffalse
%This code needs to be before the `\ProvidesFile' directive
%which is defined at the beginning of this file.
%Therefore it is also placed there and commented out here.
%</package>
%<*discard>
%\fi
%    \begin{macrocode}
\ifdefined\childdocmain\endinput\fi
%    \end{macrocode}
%\iffalse
%</discard>
%<*package>
%\fi
%
% \macro{\ifchilddoc}
% \macro{\ifchilddocmanual}
% The conditional |\ifchilddoc| tells whether a
% child (true) or main (false) document is being compiled.
% The conditional |\ifchilddocmanual| tells whether
% the |\includeonly| mechanism is used (false) or
% the selection of child files must be performed manually (true).
% The definitions initialise to false:
%    \begin{macrocode}
\newif\ifchilddoc
\newif\ifchilddocmanual
%    \end{macrocode}

% \macro{\childdocname}
% \macro{\childdocjob}
% The macro |\childdocname| stores the name of the main document
% to be compiled. The macro |\childdocjob| stores the name of
% the document on which the \LaTeX{} compiler was originally invoked.
% The content of |\jobname| cannot be compared
% to filenames specified in the source due to different catcodes.
% The following code rescans |\jobname|, stores the result
% in |\childdocname| and saves a copy in |\childdocjob|:
%    \begin{macrocode}
\edef\childdocname{\scantokens\expandafter{\jobname\noexpand}}
\let\childdocjob\childdocname
%    \end{macrocode}

% \macro{\childdocdisable}
% The macro |\childdocdisable| prevents the main file
% from being processed more than once.
% At this stage, the main document command |\childdocmain|
% is assumed to be called once again where it should do nothing.
% Any subsequent call to it should prevent
% a secondary processing of the main document
% It overwrites the forwarding commands
% |\childdocof| and |\childdocforward|
% with empty macros to prevent further inclusions of the main document:
%    \begin{macrocode}
\newcommand{\childdocdisable}
{
  \renewcommand{\childdocmain}[1]{\renewcommand{\childdocmain}[1]{\endinput}}
  \renewcommand{\childdocof}[1]{}
  \renewcommand{\childdocby}[2][]{}
  \renewcommand{\childdocforward}[2][]{}
  \renewcommand{\childdocdisable}{}
}
%    \end{macrocode}

% \macro{\childdocmain}
% The macro |\childdocmain| is to be called at the top of the main file
% with nothing or the main filename (without extension) as argument.
% First, it breaks loops.
% If the argument is not empty and does not match |\childdocname|
% (which is set by the first inclusion of |childdoc.def|),
% |\ifchilddoc| is set to true, |\includeonly| is applied to the child file
% and |\jobname| is set to the main file
% (for proper handling of |.aux| files):
%    \begin{macrocode}
\newcommand{\childdocmain}[1]
{
  \childdocdisable\childdocmain{}
  \if?#1?\else
    \begingroup
      \def\childdoctmp{#1}
      \ifx\childdoctmp\childdocname
        \def\childdoctmp{}
      \else
        \def\childdoctmp
        {
          \childdoctrue
          \includeonly{\childdocname}
          \def\childdocjob{#1}
          \def\jobname{#1}
        }
      \fi
      \expandafter
    \endgroup
    \childdoctmp
  \fi
}
%    \end{macrocode}

% \macro{\childdocof}
% The command |\childdocof| redirects
% compilation to the main file |#1|.
%    \begin{macrocode}
\newcommand{\childdocof}[1]
{
  \childdocdisable
  \childdoctrue
  \includeonly{\childdocname}
  \def\jobname{#1}
  \def\childdocjob{#1}
  \input{#1}
}
%    \end{macrocode}

% \macro{\childdocby}
% The command |\childdocby| ....
%    \begin{macrocode}
\newcommand{\childdocby}[2][]
{
  \childdocdisable
  \childdoctrue
  \childdocmanualtrue
  \if?#1?\else
    \def\jobname{#2}
  \fi
  \def\childdocjob{#2}
  \input{#2}
  \endinput
}
%    \end{macrocode}

% \macro{\childdocforward}
% The command |\childdocforward| redirects
% compilation to the main file or
% (if the optional argument is given) a child file.
% Parameters are set as if the main file
% or a child file starting with |\childdocof| was compiled.
% Then compilation is handed over to the main file:
%    \begin{macrocode}
\newcommand{\childdocforward}[2][]
{
  \begingroup
    \if?#1?
      \def\childdoctmp
      {
        \def\childdocname{#2}
        \def\childdocjob{#2}
        \def\jobname{#2}
        \input{#2}
        \endinput
      }
    \else
      \def\childdoctmp
      {
        \childdocdisable
        \def\childdocname{#2}
        \childdoctrue
        \includeonly{#2}
        \def\childdocjob{#1}
        \def\jobname{#1}
        \input{#1}
        \endinput
      }
    \fi
    \expandafter
  \endgroup
  \childdoctmp
}
%    \end{macrocode}

% \macro{\childdocforwardprefix}
% The command |\childdocforwardprefix| redirects
% compilation to the main or a child file by means of a pattern.
% The prefix |#1| in the current filename is replaced by |#2|
% and the suffix of the current filename is kept
% (it is assumed that the filename does not contain the substring `|~~~|'
% which is used as a delimiter).
% Compilation is handed over to the new file by |\childdocforward|:
%    \begin{macrocode}
\newcommand{\childdocforwardprefix}[3][]
{
  \begingroup
    \def\childdocextract #2##1~~~{\def\childdoctmp{\childdocforward[#1]{#3##1}}}
    \expandafter\childdocextract\childdocname~~~
    \expandafter
  \endgroup
  \childdoctmp
}
%    \end{macrocode}

% \macro{\childdoc}
% The deprecated macro |\childdoc| is a legacy version of |\childdocmain|:
%    \begin{macrocode}
\newcommand{\childdoc}{\childdocmain}
%    \end{macrocode}

% \macro{\childdocredirect}
% The deprecated macro |\childdocredirect| is a legacy version
% of |\childdocforward| and |\childdocforwardprefix|:
%    \begin{macrocode}
\newcommand{\childdocredirect}[2][]
{
  \begingroup
    \if?#1?
      \def\childdoctmp{\childdocforward{#2}}
    \else
      \def\childdoctmp{\childdocforwardprefix{#1}{#2}}
    \fi
    \expandafter
  \endgroup
  \childdoctmp
}
%    \end{macrocode}

%\iffalse
%</package>
%\fi
%
\endinput
\childdocforward{cdocsch2}"|
% \end{tabular}
% \end{center}
% Note that the trailing backslash on each first line
% merely continues the input to the second line
% (for convenient cut ant paste).
% Furthermore, the command |latex| can be replaced by any
% of its alternative versions such as |pdflatex|.
%
% %%%%%%%%%%%%%%%%%%%%%%%%%%%%%%%%%%%%%%%%%%%%%%%%%%%%%%%%%%%%%%%%%%%%%%%%%%%%%%
% %%%%%%%%%%%%%%%%%%%%%%%%%%%%%%%%%%%%%%%%%%%%%%%%%%%%%%%%%%%%%%%%%%%%%%%%%%%%%%
% \section{Implementation}
%\iffalse
%<*package>
%\fi
%
% This section describes the definitions file |childdoc.def|.

% The definitions cannot be loaded using |\usepackage| or |\RequirePackage|
% which has a mechanism to prevent loading a style file more than once.
% When loading the definitions by means of |\input|
% multiple instances have to be prevented manually:
%\iffalse
%This code needs to be before the `\ProvidesFile' directive
%which is defined at the beginning of this file.
%Therefore it is also placed there and commented out here.
%</package>
%<*discard>
%\fi
%    \begin{macrocode}
\ifdefined\childdocmain\endinput\fi
%    \end{macrocode}
%\iffalse
%</discard>
%<*package>
%\fi
%
% \macro{\ifchilddoc}
% \macro{\ifchilddocmanual}
% The conditional |\ifchilddoc| tells whether a
% child (true) or main (false) document is being compiled.
% The conditional |\ifchilddocmanual| tells whether
% the |\includeonly| mechanism is used (false) or
% the selection of child files must be performed manually (true).
% The definitions initialise to false:
%    \begin{macrocode}
\newif\ifchilddoc
\newif\ifchilddocmanual
%    \end{macrocode}

% \macro{\childdocname}
% \macro{\childdocjob}
% The macro |\childdocname| stores the name of the main document
% to be compiled. The macro |\childdocjob| stores the name of
% the document on which the \LaTeX{} compiler was originally invoked.
% The content of |\jobname| cannot be compared
% to filenames specified in the source due to different catcodes.
% The following code rescans |\jobname|, stores the result
% in |\childdocname| and saves a copy in |\childdocjob|:
%    \begin{macrocode}
\edef\childdocname{\scantokens\expandafter{\jobname\noexpand}}
\let\childdocjob\childdocname
%    \end{macrocode}

% \macro{\childdocdisable}
% The macro |\childdocdisable| prevents the main file
% from being processed more than once.
% At this stage, the main document command |\childdocmain|
% is assumed to be called once again where it should do nothing.
% Any subsequent call to it should prevent
% a secondary processing of the main document
% It overwrites the forwarding commands
% |\childdocof| and |\childdocforward|
% with empty macros to prevent further inclusions of the main document:
%    \begin{macrocode}
\newcommand{\childdocdisable}
{
  \renewcommand{\childdocmain}[1]{\renewcommand{\childdocmain}[1]{\endinput}}
  \renewcommand{\childdocof}[1]{}
  \renewcommand{\childdocby}[2][]{}
  \renewcommand{\childdocforward}[2][]{}
  \renewcommand{\childdocdisable}{}
}
%    \end{macrocode}

% \macro{\childdocmain}
% The macro |\childdocmain| is to be called at the top of the main file
% with nothing or the main filename (without extension) as argument.
% First, it breaks loops.
% If the argument is not empty and does not match |\childdocname|
% (which is set by the first inclusion of |childdoc.def|),
% |\ifchilddoc| is set to true, |\includeonly| is applied to the child file
% and |\jobname| is set to the main file
% (for proper handling of |.aux| files):
%    \begin{macrocode}
\newcommand{\childdocmain}[1]
{
  \childdocdisable\childdocmain{}
  \if?#1?\else
    \begingroup
      \def\childdoctmp{#1}
      \ifx\childdoctmp\childdocname
        \def\childdoctmp{}
      \else
        \def\childdoctmp
        {
          \childdoctrue
          \includeonly{\childdocname}
          \def\childdocjob{#1}
          \def\jobname{#1}
        }
      \fi
      \expandafter
    \endgroup
    \childdoctmp
  \fi
}
%    \end{macrocode}

% \macro{\childdocof}
% The command |\childdocof| redirects
% compilation to the main file |#1|.
%    \begin{macrocode}
\newcommand{\childdocof}[1]
{
  \childdocdisable
  \childdoctrue
  \includeonly{\childdocname}
  \def\jobname{#1}
  \def\childdocjob{#1}
  \input{#1}
}
%    \end{macrocode}

% \macro{\childdocby}
% The command |\childdocby| ....
%    \begin{macrocode}
\newcommand{\childdocby}[2][]
{
  \childdocdisable
  \childdoctrue
  \childdocmanualtrue
  \if?#1?\else
    \def\jobname{#2}
  \fi
  \def\childdocjob{#2}
  \input{#2}
  \endinput
}
%    \end{macrocode}

% \macro{\childdocforward}
% The command |\childdocforward| redirects
% compilation to the main file or
% (if the optional argument is given) a child file.
% Parameters are set as if the main file
% or a child file starting with |\childdocof| was compiled.
% Then compilation is handed over to the main file:
%    \begin{macrocode}
\newcommand{\childdocforward}[2][]
{
  \begingroup
    \if?#1?
      \def\childdoctmp
      {
        \def\childdocname{#2}
        \def\childdocjob{#2}
        \def\jobname{#2}
        \input{#2}
        \endinput
      }
    \else
      \def\childdoctmp
      {
        \childdocdisable
        \def\childdocname{#2}
        \childdoctrue
        \includeonly{#2}
        \def\childdocjob{#1}
        \def\jobname{#1}
        \input{#1}
        \endinput
      }
    \fi
    \expandafter
  \endgroup
  \childdoctmp
}
%    \end{macrocode}

% \macro{\childdocforwardprefix}
% The command |\childdocforwardprefix| redirects
% compilation to the main or a child file by means of a pattern.
% The prefix |#1| in the current filename is replaced by |#2|
% and the suffix of the current filename is kept
% (it is assumed that the filename does not contain the substring `|~~~|'
% which is used as a delimiter).
% Compilation is handed over to the new file by |\childdocforward|:
%    \begin{macrocode}
\newcommand{\childdocforwardprefix}[3][]
{
  \begingroup
    \def\childdocextract #2##1~~~{\def\childdoctmp{\childdocforward[#1]{#3##1}}}
    \expandafter\childdocextract\childdocname~~~
    \expandafter
  \endgroup
  \childdoctmp
}
%    \end{macrocode}

% \macro{\childdoc}
% The deprecated macro |\childdoc| is a legacy version of |\childdocmain|:
%    \begin{macrocode}
\newcommand{\childdoc}{\childdocmain}
%    \end{macrocode}

% \macro{\childdocredirect}
% The deprecated macro |\childdocredirect| is a legacy version
% of |\childdocforward| and |\childdocforwardprefix|:
%    \begin{macrocode}
\newcommand{\childdocredirect}[2][]
{
  \begingroup
    \if?#1?
      \def\childdoctmp{\childdocforward{#2}}
    \else
      \def\childdoctmp{\childdocforwardprefix{#1}{#2}}
    \fi
    \expandafter
  \endgroup
  \childdoctmp
}
%    \end{macrocode}

%\iffalse
%</package>
%\fi
%
\endinput
|\\
|\childdocforwardprefix[|\textit{main}|]{|\textit{prefix}|}{|\textit{dest}|}|
\end{tabular}
\end{center}
%
the destination file is determined by a pattern
depending on the current file:
To make this work, the current file must be called
`{\textit{prefix}\hspace{0.2em}\textit{suffix}}'
with \textit{prefix} matching precisely the argument.
Processing is then passed on to the file
`{\textit{dest}\hspace{0.2em}\textit{suffix}}'.
Surely, the same effect is achieved by
directly specifying the
argument `{\textit{dest}\hspace{0.2em}\textit{suffix}}'
in the first form.
However, that requires to set up a different file
for each child. With the alternative form of the command
all these files can have exactly the same content
which simplifies setting them up and maintaining them.

For example, the following file |draft.tex|
with a compilation flag |\version| as described in \secref{sec:flags}
compiles the main document as a draft:
%
\begin{center}
\begin{tabular}{l}
|\def\version{draft}|\\
|% \iffalse
%
% childdoc.dtx Copyright (C) 2017-2018 Niklas Beisert
%
% This work may be distributed and/or modified under the
% conditions of the LaTeX Project Public License, either version 1.3
% of this license or (at your option) any later version.
% The latest version of this license is in
%   http://www.latex-project.org/lppl.txt
% and version 1.3 or later is part of all distributions of LaTeX
% version 2005/12/01 or later.
%
% This work has the LPPL maintenance status `maintained'.
%
% The Current Maintainer of this work is Niklas Beisert.
%
% This work consists of the files childdoc.dtx and childdoc.ins
% and the derived files childdoc.def and cdocsamp.tex with
% cdocsch1.tex, cdocsch2.tex, cdocsdrf.tex, cdocsfn1.tex, cdocsfn2.tex.
%
%<package>\ifdefined\childdocmain\endinput\fi
%<package>\ProvidesFile{childdoc.def}[2018/12/30 v2.0 child document driver]
%<samplemain>\ProvidesFile{cdocsamp.tex}[2018/12/30 v2.0 sample for childdoc]
%<*driver>
%\ProvidesFile{childdoc.drv}[2018/12/30 v2.0 childdoc reference manual file]
\PassOptionsToClass{10pt,a4paper}{article}
\documentclass{ltxdoc}

\usepackage[margin=35mm]{geometry}
\usepackage{hyperref}
\usepackage{hyperxmp}
\usepackage[usenames]{color}

\hypersetup{colorlinks=true}
\hypersetup{pdfstartview=FitH}
\hypersetup{pdfpagemode=UseNone}
\hypersetup{pdfsource={}}
\hypersetup{pdflang={en-UK}}
\hypersetup{pdfcopyright={Copyright 2017-2018 Niklas Beisert.
  This work may be distributed and/or modified under the
  conditions of the LaTeX Project Public License, either version 1.3
  of this license or (at your option) any later version.}}
\hypersetup{pdflicenseurl={http://www.latex-project.org/lppl.txt}}
\hypersetup{pdfcontactaddress={ETH Zurich, ITP, HIT K,
  Wolfgang-Pauli-Strasse 27}}
\hypersetup{pdfcontactpostcode={8093}}
\hypersetup{pdfcontactcity={Zurich}}
\hypersetup{pdfcontactcountry={Switzerland}}
\hypersetup{pdfcontactemail={nbeisert@itp.phys.ethz.ch}}
\hypersetup{pdfcontacturl={http://people.phys.ethz.ch/\xmptilde nbeisert/}}

\newcommand{\secref}[1]{\hyperref[#1]{section \ref*{#1}}}

\parskip1ex
\parindent0pt
\let\olditemize\itemize
\def\itemize{\olditemize\parskip0pt}

\begin{document}

\title{The \textsf{childdoc} Package}
\hypersetup{pdftitle={The childdoc Package}}
\author{Niklas Beisert\\[2ex]
  Institut f\"ur Theoretische Physik\\
  Eidgen\"ossische Technische Hochschule Z\"urich\\
  Wolfgang-Pauli-Strasse 27, 8093 Z\"urich, Switzerland\\[1ex]
  \href{mailto:nbeisert@itp.phys.ethz.ch}
  {\texttt{nbeisert@itp.phys.ethz.ch}}}
\hypersetup{pdfauthor={Niklas Beisert}}
\hypersetup{pdfsubject={Manual for the LaTeX2e Package childdoc}}
\date{30 December 2018, \textsf{v2.0}}
\maketitle

\begin{abstract}\noindent
\textsf{childdoc} is a \LaTeXe{} package
that enables the direct compilation
of document sections included by |\include|
to individual files.
\end{abstract}

\begingroup
\parskip0ex
\tableofcontents
\endgroup

%%%%%%%%%%%%%%%%%%%%%%%%%%%%%%%%%%%%%%%%%%%%%%%%%%%%%%%%%%%%%%%%%%%%%%%%%%%%%%%%
%%%%%%%%%%%%%%%%%%%%%%%%%%%%%%%%%%%%%%%%%%%%%%%%%%%%%%%%%%%%%%%%%%%%%%%%%%%%%%%%
\section{Introduction}

\LaTeX{} provides a mechanism to structure a large document (such as a book)
into a main file and several child files (containing the chapters)
using the |\include| command.
This mechanism is beneficial for documents
which span hundreds of pages in order to
make the source file(s) more manageable.
Moreover, compilation can be restricted to
selected child files by means of the |\includeonly| command.
The latter feature can be used to reduce the compilation time while editing
(this was significantly more useful in the earlier days of \LaTeX{})
or to generate a smaller document which is easier to navigate.
Another application of |\includeonly| is to generate
documents consisting of selected parts of the complete document.

However, there are a few drawbacks of the plain |\include| mechanism:
\begin{itemize}
\item
The child files cannot be compiled on their own,
they can only be compiled via the main file.
A naive editing environment
(such as a text editor with an option
to have the current file processed by \LaTeX)
may require one to switch to the main file before compiling;
attempting to compile the child file produces errors.
\item
The main file must be modified (each time)
to adjust the |\includeonly| command
to the present needs. This easily leaves the main file in a messy state.
\item
The generated document will always carry the filename
of the main document. This is inconvenient if
several child files are to be compiled and
to be kept for distribution.
\end{itemize}

The present package provides a simple interface
to make child files individually compilable by \LaTeX{}.
Compiling a child file then has the same effect as compiling
the main file with an |\includeonly| command
to select the appropriate child.
Moreover the generated document will carry the name of the child
rather than the main file.
This resolves all three above issues.

This feature is meant to make the editing of books,
thesis documents and lecture notes somewhat more convenient.
However, the package can also be used efficiently for
composing a series of documents (such as exercise sheets)
which are typically distributed individually.
It then assists the author in generating the individual documents
(potentially in different versions)
as well as a document containing the collected series.
Another application is in developing style files
or other kinds of included material
where compilation of the style file could redirect
to a sample or test file.

%%%%%%%%%%%%%%%%%%%%%%%%%%%%%%%%%%%%%%%%%%%%%%%%%%%%%%%%%%%%%%%%%%%%%%%%%%%%%%%%
%%%%%%%%%%%%%%%%%%%%%%%%%%%%%%%%%%%%%%%%%%%%%%%%%%%%%%%%%%%%%%%%%%%%%%%%%%%%%%%%
\section{Usage}

First of all, the package \textsf{childdoc} is \emph{not} a standard
\LaTeXe{} |.sty| style file! Therefore it needs to be invoked in
a non-standard way.

%%%%%%%%%%%%%%%%%%%%%%%%%%%%%%%%%%%%%%%%%%%%%%%%%%%%%%%%%%%%%%%%%%%%%%%%%%%%%%%%
\subsection{Included Files}
\label{sec:include}

%%%%%%%%%%%%%%%%%%%%%%%%%%%%%%%%%%%%%%%%
\DescribeMacro{\childdocmain}
To use the package, add the commands
\begin{center}
\begin{tabular}{l}
|% \iffalse
%
% childdoc.dtx Copyright (C) 2017-2018 Niklas Beisert
%
% This work may be distributed and/or modified under the
% conditions of the LaTeX Project Public License, either version 1.3
% of this license or (at your option) any later version.
% The latest version of this license is in
%   http://www.latex-project.org/lppl.txt
% and version 1.3 or later is part of all distributions of LaTeX
% version 2005/12/01 or later.
%
% This work has the LPPL maintenance status `maintained'.
%
% The Current Maintainer of this work is Niklas Beisert.
%
% This work consists of the files childdoc.dtx and childdoc.ins
% and the derived files childdoc.def and cdocsamp.tex with
% cdocsch1.tex, cdocsch2.tex, cdocsdrf.tex, cdocsfn1.tex, cdocsfn2.tex.
%
%<package>\ifdefined\childdocmain\endinput\fi
%<package>\ProvidesFile{childdoc.def}[2018/12/30 v2.0 child document driver]
%<samplemain>\ProvidesFile{cdocsamp.tex}[2018/12/30 v2.0 sample for childdoc]
%<*driver>
%\ProvidesFile{childdoc.drv}[2018/12/30 v2.0 childdoc reference manual file]
\PassOptionsToClass{10pt,a4paper}{article}
\documentclass{ltxdoc}

\usepackage[margin=35mm]{geometry}
\usepackage{hyperref}
\usepackage{hyperxmp}
\usepackage[usenames]{color}

\hypersetup{colorlinks=true}
\hypersetup{pdfstartview=FitH}
\hypersetup{pdfpagemode=UseNone}
\hypersetup{pdfsource={}}
\hypersetup{pdflang={en-UK}}
\hypersetup{pdfcopyright={Copyright 2017-2018 Niklas Beisert.
  This work may be distributed and/or modified under the
  conditions of the LaTeX Project Public License, either version 1.3
  of this license or (at your option) any later version.}}
\hypersetup{pdflicenseurl={http://www.latex-project.org/lppl.txt}}
\hypersetup{pdfcontactaddress={ETH Zurich, ITP, HIT K,
  Wolfgang-Pauli-Strasse 27}}
\hypersetup{pdfcontactpostcode={8093}}
\hypersetup{pdfcontactcity={Zurich}}
\hypersetup{pdfcontactcountry={Switzerland}}
\hypersetup{pdfcontactemail={nbeisert@itp.phys.ethz.ch}}
\hypersetup{pdfcontacturl={http://people.phys.ethz.ch/\xmptilde nbeisert/}}

\newcommand{\secref}[1]{\hyperref[#1]{section \ref*{#1}}}

\parskip1ex
\parindent0pt
\let\olditemize\itemize
\def\itemize{\olditemize\parskip0pt}

\begin{document}

\title{The \textsf{childdoc} Package}
\hypersetup{pdftitle={The childdoc Package}}
\author{Niklas Beisert\\[2ex]
  Institut f\"ur Theoretische Physik\\
  Eidgen\"ossische Technische Hochschule Z\"urich\\
  Wolfgang-Pauli-Strasse 27, 8093 Z\"urich, Switzerland\\[1ex]
  \href{mailto:nbeisert@itp.phys.ethz.ch}
  {\texttt{nbeisert@itp.phys.ethz.ch}}}
\hypersetup{pdfauthor={Niklas Beisert}}
\hypersetup{pdfsubject={Manual for the LaTeX2e Package childdoc}}
\date{30 December 2018, \textsf{v2.0}}
\maketitle

\begin{abstract}\noindent
\textsf{childdoc} is a \LaTeXe{} package
that enables the direct compilation
of document sections included by |\include|
to individual files.
\end{abstract}

\begingroup
\parskip0ex
\tableofcontents
\endgroup

%%%%%%%%%%%%%%%%%%%%%%%%%%%%%%%%%%%%%%%%%%%%%%%%%%%%%%%%%%%%%%%%%%%%%%%%%%%%%%%%
%%%%%%%%%%%%%%%%%%%%%%%%%%%%%%%%%%%%%%%%%%%%%%%%%%%%%%%%%%%%%%%%%%%%%%%%%%%%%%%%
\section{Introduction}

\LaTeX{} provides a mechanism to structure a large document (such as a book)
into a main file and several child files (containing the chapters)
using the |\include| command.
This mechanism is beneficial for documents
which span hundreds of pages in order to
make the source file(s) more manageable.
Moreover, compilation can be restricted to
selected child files by means of the |\includeonly| command.
The latter feature can be used to reduce the compilation time while editing
(this was significantly more useful in the earlier days of \LaTeX{})
or to generate a smaller document which is easier to navigate.
Another application of |\includeonly| is to generate
documents consisting of selected parts of the complete document.

However, there are a few drawbacks of the plain |\include| mechanism:
\begin{itemize}
\item
The child files cannot be compiled on their own,
they can only be compiled via the main file.
A naive editing environment
(such as a text editor with an option
to have the current file processed by \LaTeX)
may require one to switch to the main file before compiling;
attempting to compile the child file produces errors.
\item
The main file must be modified (each time)
to adjust the |\includeonly| command
to the present needs. This easily leaves the main file in a messy state.
\item
The generated document will always carry the filename
of the main document. This is inconvenient if
several child files are to be compiled and
to be kept for distribution.
\end{itemize}

The present package provides a simple interface
to make child files individually compilable by \LaTeX{}.
Compiling a child file then has the same effect as compiling
the main file with an |\includeonly| command
to select the appropriate child.
Moreover the generated document will carry the name of the child
rather than the main file.
This resolves all three above issues.

This feature is meant to make the editing of books,
thesis documents and lecture notes somewhat more convenient.
However, the package can also be used efficiently for
composing a series of documents (such as exercise sheets)
which are typically distributed individually.
It then assists the author in generating the individual documents
(potentially in different versions)
as well as a document containing the collected series.
Another application is in developing style files
or other kinds of included material
where compilation of the style file could redirect
to a sample or test file.

%%%%%%%%%%%%%%%%%%%%%%%%%%%%%%%%%%%%%%%%%%%%%%%%%%%%%%%%%%%%%%%%%%%%%%%%%%%%%%%%
%%%%%%%%%%%%%%%%%%%%%%%%%%%%%%%%%%%%%%%%%%%%%%%%%%%%%%%%%%%%%%%%%%%%%%%%%%%%%%%%
\section{Usage}

First of all, the package \textsf{childdoc} is \emph{not} a standard
\LaTeXe{} |.sty| style file! Therefore it needs to be invoked in
a non-standard way.

%%%%%%%%%%%%%%%%%%%%%%%%%%%%%%%%%%%%%%%%%%%%%%%%%%%%%%%%%%%%%%%%%%%%%%%%%%%%%%%%
\subsection{Included Files}
\label{sec:include}

%%%%%%%%%%%%%%%%%%%%%%%%%%%%%%%%%%%%%%%%
\DescribeMacro{\childdocmain}
To use the package, add the commands
\begin{center}
\begin{tabular}{l}
|\input{childdoc.def}|\\
|\childdocmain{}|\\
\end{tabular}
\end{center}
at the very top of the main \LaTeX{} file,
in particular \emph{before} the |\documentclass| statement!
The argument of |\childdocmain| should be left empty
(but it must be present).

%%%%%%%%%%%%%%%%%%%%%%%%%%%%%%%%%%%%%%%%
\DescribeMacro{\childdocof}
Furthermore, add the commands
\begin{center}
\begin{tabular}{l}
|\input{childdoc.def}|\\
|\childdocof{|\textit{main}|}|\\
\end{tabular}
\end{center}
at the top of every child file \textit{child}
which is included by |\include{|\textit{child}|}|
from within the main file
(or at least for those files to be compiled individually).
The argument \textit{main} must be the filename of the main file.

There are a couple of
considerations in setting up the main and child documents:

%%%%%%%%%%%%%%%%%%%%%%%%%%%%%%%%%%%%%%%%
\paragraph{Restrictions.}

Please note the following restrictions:
\begin{itemize}
\item
|\childdocmain| must be called with one argument \textit{main}
to ensure compatibility with earlier version of the package.
It must either be empty (|\childdocmain{}|)
or precisely match the filename of the main file in which it is specified.
See \secref{sec:detection} for further information.
\item
The filename \textit{main} must be specified without the |.tex| extension.
\item
The filename \textit{main} is case sensitive
(even in case-insensitive file systems)
due to internal string comparison.
\item
The argument \textit{main} should be fully expanded, it cannot be a macro.
\item
Subdirectories and special characters should be avoided in filenames.
\item
The command |\childdocmain{|\textit{main}|}| must be followed by a whitespace.
It should not be followed immediately by another command
or by a comment mark `|%|'.
This is because the \TeX{} parser reads the token immediately following
the argument of |\childdocmain| and puts it
at the beginning of every child section;
however, a white\-space is ignored.
\end{itemize}

%%%%%%%%%%%%%%%%%%%%%%%%%%%%%%%%%%%%%%%%
\paragraph{Content of Main File.}

It is advisable to place all content in the child files included by |\include|.
Any output contained in the main file will appear in all child documents
unless suppressed manually;
it cannot be suppressed automatically by the |\includeonly| directive
and thus should normally be avoided.
A method to include some content in the main file
by means of conditional processing is described in \secref{sec:conditional}.

%%%%%%%%%%%%%%%%%%%%%%%%%%%%%%%%%%%%%%%%
\paragraph{Page Numbering.}

When only a part of the document is compiled,
the appropriate numbering of pages
(as well as other status parameters)
is determined from the |.aux| files.
The latter contain information from previous passes.
However this information needs to propagate through
all intermediate child documents.
Therefore the page numbering in child documents may well
be inconsistent until the complete document is compiled at least once.

A useful (if unconventional) way to always ensure a consistent
page numbering is to restart the numbering in each child document
and denote the pages by `\textit{child}|.|\textit{page}'
where \textit{child} represents the chapter/section number of the child file.
This can be achieved by the command
|\numberwithin{page}{|\textit{child}|}|
of the \textsf{amsmath} package
where \textit{child} can be |chapter| or |section|
depending on the chosen structuring.
Alternatively, one can modify the macro |\thepage| appropriately
and reset the counter |page| at the start of each child file.

%%%%%%%%%%%%%%%%%%%%%%%%%%%%%%%%%%%%%%%%%%%%%%%%%%%%%%%%%%%%%%%%%%%%%%%%%%%%%%%%
\subsection{Conditional Processing}
\label{sec:conditional}

The package provides a mechanism to compile different versions
of a document. To customise the versions further some conditional processing
can come in handy to distinguish which version is being compiled.
The package provides two macros to describe the compilation context:

%%%%%%%%%%%%%%%%%%%%%%%%%%%%%%%%%%%%%%%%
\DescribeMacro{\ifchilddoc}
The conditional |\ifchilddoc| distinguishes between the compilation of
child documents and the main document:
%
\begin{center}
|\ifchilddoc |\textit{child-code}| |[|\||else |\textit{main-code}]| \||fi|
\end{center}

%%%%%%%%%%%%%%%%%%%%%%%%%%%%%%%%%%%%%%%%
\DescribeMacro{\childdocname}
\DescribeMacro{\childdocjob}
The macro |\childdocname| contains the filename (without extension)
of the main or child file being processed.
Note that |\childdocjob| will always contain the name of the main file.

%%%%%%%%%%%%%%%%%%%%%%%%%%%%%%%%%%%%%%%%
\paragraph{Title Page.}

Conditional processing can be used to include a title or banner page
in the main document when proper precautions are taken.
Importantly, the code in the main file should ensure that the page counter
(as well as other status parameters which are stored in the |.aux| files)
takes the same value after the conditional processing.
Otherwise the page numbers may take divergent values
depending on which part is compiled.

For example, a title page could be declared by:
%
\begin{center}
\begin{tabular}{l}
|\ifchilddoc\||else|\\
|\addtocounter{page}{-1}|\\
\textit{code for title page}\\
|\newpage|\\
|\||fi|
\end{tabular}
\end{center}
%
A banner page for the child documents can be generated by:
%
\begin{center}
\begin{tabular}{l}
|\ifchilddoc|\\
|\addtocounter{page}{-1}|\\
\textit{code for banner page}\\
|\newpage|\\
|\||fi|
\end{tabular}
\end{center}
%
Here one could write a message such as:
\begin{center}
|This is the part \childdocname{} of \childdocjob{}.|
\end{center}

%%%%%%%%%%%%%%%%%%%%%%%%%%%%%%%%%%%%%%%%%%%%%%%%%%%%%%%%%%%%%%%%%%%%%%%%%%%%%%%%
\subsection{Flags}
\label{sec:flags}

The package makes it easy to generate different versions
of the main or child documents.
To this end compilation flags can be defined
and assigned different default values.
They will be particularly useful in conjunction
with the forwarding mechanism described in \secref{sec:forward}.

For example, it may be useful to have a flag |\version|
which can be set to |draft| or |final|.
The document source will contain some conditional code
depending on the value of |\version|.
Suppose further, the flag should default to |final| for the main file
and to |draft| for child files
which is a natural assignment for editing the document.
This is achieved by placing the following code
in the preamble of the main document
(below the |\childdocmain| directive):
%
\begin{center}
\begin{tabular}{l}
|\ifchilddoc|\\
|\providecommand{\version}{draft}|\\
|\||else|\\
|\providecommand{\version}{final}|\\
|\||fi|
\end{tabular}
\end{center}
%
The definition by |\providecommand| makes sure
that previous definitions are not overwritten.
Further statements |\providecommand{\version}{...}|
can thus be added before the above code to override it.

For the main file, one might add a line
(between |\childdocmain| and the above block)
%
\begin{center}
|%\ifchilddoc\||else\providecommand{\version}{draft}\||fi|
\end{center}
%
which can be uncommented to produce a draft version.
Likewise one can add a line to the very top of a child file
(above the |\childdocof{|\textit{main}|}| directive)
%
\begin{center}
|%\providecommand{\version}{final}|
\end{center}
%
which can be uncommented to produce the final version of this child document.

%%%%%%%%%%%%%%%%%%%%%%%%%%%%%%%%%%%%%%%%%%%%%%%%%%%%%%%%%%%%%%%%%%%%%%%%%%%%%%%%
\subsection{Forwarding}
\label{sec:forward}

Different versions of the main or child documents
using compilation flags as described in \secref{sec:flags}
can be (permanently) stored in different files
for convenient compilation, viewing and distribution.
To this end, the package defines a command
to pass on compilation to a different file:

%%%%%%%%%%%%%%%%%%%%%%%%%%%%%%%%%%%%%%%%
\DescribeMacro{\childdocforward}
The command |\childdocforward| redirects processing to
another source file:
%
\begin{center}
\begin{tabular}{l}
|\input{childdoc.def}|\\
|\childdocforward[|\textit{main}|]{|\textit{dest}|}|\\
\end{tabular}
\end{center}
%
The argument \textit{dest} is the destination file
(without extension).
It should be the main file or one of the child files.
Note that further \textsf{childdoc} directives
such as |\childdocof| and |\childdocforward|
in the indicated file will be processed in this form.
The optional argument \textit{main}
passes on directly to the main file \textit{main}
while pretending to compile the child \textit{dest}.
This form behaves as if \textit{dest}
issues |\childdocof{|\textit{main}|}| right away,
and no further \textsf{childdoc} directives will be processed.

%%%%%%%%%%%%%%%%%%%%%%%%%%%%%%%%%%%%%%%%
\DescribeMacro{\...prefix}
In the alternative form |\childdocforwardprefix|,
%
\begin{center}
\begin{tabular}{l}
|\input{childdoc.def}|\\
|\childdocforwardprefix[|\textit{main}|]{|\textit{prefix}|}{|\textit{dest}|}|
\end{tabular}
\end{center}
%
the destination file is determined by a pattern
depending on the current file:
To make this work, the current file must be called
`{\textit{prefix}\hspace{0.2em}\textit{suffix}}'
with \textit{prefix} matching precisely the argument.
Processing is then passed on to the file
`{\textit{dest}\hspace{0.2em}\textit{suffix}}'.
Surely, the same effect is achieved by
directly specifying the
argument `{\textit{dest}\hspace{0.2em}\textit{suffix}}'
in the first form.
However, that requires to set up a different file
for each child. With the alternative form of the command
all these files can have exactly the same content
which simplifies setting them up and maintaining them.

For example, the following file |draft.tex|
with a compilation flag |\version| as described in \secref{sec:flags}
compiles the main document as a draft:
%
\begin{center}
\begin{tabular}{l}
|\def\version{draft}|\\
|\input{childdoc.def}|\\
|\childdocforward{|\textit{main}|}|
\end{tabular}
\end{center}
%
Likewise, the following files |final|\textit{nn}|.tex|
compile the final version of the child document
|child|\textit{nn}|.tex|:
%
\begin{center}
\begin{tabular}{l}
|\def\version{final}|\\
|\input{childdoc.def}|\\
|\childdocforwardprefix{final}{child}|
\end{tabular}
\end{center}
%

Note that when several versions of a main file and/or of each child file
are to be generated, it may be convenient to set up a |Makefile| or
shell script to automatise the process.

%%%%%%%%%%%%%%%%%%%%%%%%%%%%%%%%%%%%%%%%%%%%%%%%%%%%%%%%%%%%%%%%%%%%%%%%%%%%%%%%
\subsection{Command Line Processing}
\label{sec:commandline}

The effect of redirection files can also be achieved by invoking
the \LaTeX{} compiler with a more elaborate command line.
Most conveniently this should be done as part
of a shell script or a |Makefile|.

When using \textsf{childdoc} in the main file, the following
command lines effectively perform a redirection
(note that depending on the shell being used,
backslashes may have to be doubled: `|\|' $\to$ `|\\|'):
%
\begin{center}
|... -jobname "|\textit{target}|" |\\|"|[\textit{flags}]%
|\input{childdoc.def}\childdocforward[|\textit{main}|]{|\textit{dest}|}"|
\end{center}
%
Here \textit{target} is the name of the output file,
\textit{main} is the name of the main file
and \textit{dest} is the name of the main or child file to be processed
(all filenames without extensions).
The optional argument \textit{main} can be omitted
if \textit{main} matches \textit{dest}.
Optionally, compilation \textit{flags} can be defined via |\def| commands.
This command line makes the \TeX{} engine believe
it is compiling the file \textit{target}
whose content is specified as the latter parameter.
The provided code then forwards the processing to
\textit{main} or \textit{dest} as described in \secref{sec:forward}.

%%%%%%%%%%%%%%%%%%%%%%%%%%%%%%%%%%%%%%%%%%%%%%%%%%%%%%%%%%%%%%%%%%%%%%%%%%%%%%%%
\subsection{Include by Input}
\label{sec:input}

Including child documents by |\include| has some restrictions by design.
Most notably, the content of a child document always occupies
its own set of pages; pages cannot be shared between child documents.
Usually, this behaviour makes perfect sense
because each child document contain an essential part of the document.
However, in some situations it may be desirable to compose
a document from a collection of parts
without having mandatory page breaks between then.
For this case, the package
provides a mechanism to include parts
by |\input| which can also be processed individually.
However, by construction this mechanism
requires manual handling of the content to be output.

%%%%%%%%%%%%%%%%%%%%%%%%%%%%%%%%%%%%%%%%
\DescribeMacro{\ifchilddocmanual}
The main file should be prepared as usual, see \secref{sec:include}.
However, the document body must make a distinction
between processing of an individual part and of the main document, e.g.:
%
\begin{center}
\begin{tabular}{l}
|\ifchilddocmanual|\\
|\input{\childdocname}|\\
|\||else|\\
\textit{document body with }|\input{|\textit{part}|}|\\
|\||fi|
\end{tabular}
\end{center}
%
The conditional |\ifchilddocmanual| is true whenever
a part to be included by |\input| is being compiled,
and the name of the part is stored in |\childdocname|.

%%%%%%%%%%%%%%%%%%%%%%%%%%%%%%%%%%%%%%%%
\DescribeMacro{\childdocby}
Each part to be included by |\input| should start with:
%
\begin{center}
\begin{tabular}{l}
|\input{childdoc.def}|\\
|\childdocby{|\textit{main}|}|\\
\end{tabular}
\end{center}
%
The directive |\childdocby| is similar to |\childdocof|
described in \secref{sec:include},
but the subsequent selection of content must be done manually.
To that end, both |\ifchilddoc| and |\ifchilddocmanual|
will be true upon processing of a part,
and the name of the part is stored in |\childdocname|.
Note that |\jobname| will be set to the filename of the current part
so that each part receives an individual |.aux| file
that does not interfere with the |.aux| file(s) of the main document.
This behaviour can be altered by the alternative form
|\childdocby[*]{|\textit{main}|}| (with a non-empty optional argument)
which uses the |.aux| file of the main document
by setting |\jobname| to \textit{main}.

%%%%%%%%%%%%%%%%%%%%%%%%%%%%%%%%%%%%%%%%%%%%%%%%%%%%%%%%%%%%%%%%%%%%%%%%%%%%%%%%
\subsection{Driver Development}
\label{sec:driver}

The \textsf{childdoc} mechanism can also be use for the development
of definition files such as \LaTeX{} styles or classes.
This case differs from the above setup with multiple parts
included by |\include| in that no |\includeonly| should be invoked.
This can be achieved by starting the include file
(before |\ProvidesPackage|) with:
%
\begin{center}
\begin{tabular}{l}
|\input{childdoc.def}|\\
|\childdocforward{|\textit{main}|}|\\
\end{tabular}
\end{center}
%
or alternatively with:
%
\begin{center}
\begin{tabular}{l}
|\input{childdoc.def}|\\
|\childdocby{|\textit{main}|}|\\
\end{tabular}
\end{center}
%
Both forms have slightly different effects as described above.
The main file is prepared as usual, see \secref{sec:include}.

%%%%%%%%%%%%%%%%%%%%%%%%%%%%%%%%%%%%%%%%%%%%%%%%%%%%%%%%%%%%%%%%%%%%%%%%%%%%%%%%
\subsection{Legacy Detection}
\label{sec:detection}

The directive |\childdocmain| in the main file can detect
whether the complete document or merely a child is to be compiled
even without using the directive |\childdocof|.
This method is deprecated because it is less robust
and there is no compelling reason to use it;
it is merely provided for backward compatibility
and it may be removed in future versions.

If the detection mechanism is to be used,
it is mandatory to correctly specify
the filename of the main file as the argument of |\childdocmain|:
%
\begin{center}
\begin{tabular}{l}
|\input{childdoc.def}|\\
|\childdocmain{|\textit{main}|}|\\
\end{tabular}
\end{center}
%
If |\jobname| does not match the argument \textit{main} of |\childdocmain|,
it is assumed that |\jobname| points to the child file to be compiled.
When using |\childdocmain| with the main file specified as argument,
it suffices to start a child file
with just |\input{|\textit{main}|}|
without loading of the package and using |\childdocof|.
If instead all processing is done
with the appropriate \textsf{childdoc} directives,
the argument of \textit{main} of |\childdocmain| can be empty.

An alternative version of the command line processing described
in \secref{sec:commandline} using the detection mechanism reads:
%
\begin{center}
|... -jobname "|\textit{target}|" "|[\textit{flags}]%
[|\def\jobname{|\textit{dest}|}|]|\input{|\textit{main}|}"|
\end{center}

%%%%%%%%%%%%%%%%%%%%%%%%%%%%%%%%%%%%%%%%%%%%%%%%%%%%%%%%%%%%%%%%%%%%%%%%%%%%%%%%
\subsection{Manual Code}
\label{sec:manual}

In case one cannot be certain whether the definitions file |childdoc.def|
is installed on the target \TeX{} distribution
and one prefers not to ship it,
it is conceivable to paste a few relevant commands into the sources.

To that end, drop all statements |\input{childdoc.def}|
and perform the replacements as outlined below.
Instead of |\childdocmain{|\textit{main}|}| add the following code
to the top of the main file:
%
\begin{center}
\begin{tabular}{l}
|\||ifdefined\childdocname\endinput\||fi\newif\ifchilddoc|\\
|\edef\childdocname{\scantokens\expandafter{\jobname\noexpand}}|\\
|\def\childdocmain{|\textit{main}|}\||ifx\childdocmain\childdocname\||else|\\
|\childdoctrue\includeonly{\childdocname}\let\jobname\childdocmain\||fi|\\
\end{tabular}
\end{center}
%
Instead of |\childdocof{|\textit{main}|}| just include the main file
at the top of each child file:
%
\begin{center}
|\input{|\textit{main}|}|
\end{center}
%
A simple redirection |\childdocforward{|\textit{dest}|}| is achieved by:
%
\begin{center}
|\def\jobname{|\textit{dest}|}\input{\jobname}|
\end{center}
%
The redirection with prefix
|\childdocforwardprefix[|\textit{prefix}|]{|\textit{dest}|}|
is accomplished by:
%
\begin{center}
\begin{tabular}{l}
|{\edef\jobname{\scantokens\expandafter{\jobname\noexpand}}|\\
|\def\redirectjob |\textit{prefix}|#1~~~{\gdef\jobname{|\textit{dest}|#1}}|\\
|\expandafter\redirectjob\jobname~~~}\input{\jobname}|
\end{tabular}
\end{center}

In an alternative approach,
child documents can be compiled by a specific command line
without additional code or specific definitions:
%
\begin{center}
|... -jobname "|\textit{target}|" "|[\textit{flags}]%
|\includeonly{|\textit{dest}|}\input{|\textit{main}|}"|
\end{center}
%

%%%%%%%%%%%%%%%%%%%%%%%%%%%%%%%%%%%%%%%%%%%%%%%%%%%%%%%%%%%%%%%%%%%%%%%%%%%%%%%%
%%%%%%%%%%%%%%%%%%%%%%%%%%%%%%%%%%%%%%%%%%%%%%%%%%%%%%%%%%%%%%%%%%%%%%%%%%%%%%%%
\section{Information}

%%%%%%%%%%%%%%%%%%%%%%%%%%%%%%%%%%%%%%%%%%%%%%%%%%%%%%%%%%%%%%%%%%%%%%%%%%%%%%%%
\subsection{Copyright}

Copyright \copyright{} 2017--2018 Niklas Beisert

This work may be distributed and/or modified under the
conditions of the \LaTeX{} Project Public License, either version 1.3
of this license or (at your option) any later version.
The latest version of this license is in
  \url{http://www.latex-project.org/lppl.txt}
and version 1.3 or later is part of all distributions of \LaTeX{}
version 2005/12/01 or later.

This work has the LPPL maintenance status `maintained'.

The Current Maintainer of this work is Niklas Beisert.

This work consists of the files |README.txt|, |childdoc.ins| and |childdoc.dtx|
as well as the derived files |childdoc.def|, |cdocsamp.tex|
with |cdocsch1.tex|, |cdocsch2.tex|, |cdocspt3.tex|, |cdocspt4.tex|,
|cdocsdrf.tex|, |cdocsfn1.tex|, |cdocsfn2.tex|
as well as |childdoc.pdf|.

%%%%%%%%%%%%%%%%%%%%%%%%%%%%%%%%%%%%%%%%%%%%%%%%%%%%%%%%%%%%%%%%%%%%%%%%%%%%%%%%
\subsection{Files and Installation}

The package consists of the files:
%
\begin{center}
\begin{tabular}{ll}
    |README.txt|   & readme file \\
    |childdoc.ins| & installation file \\
    |childdoc.dtx| & source file \\
    |childdoc.def| & definition file \\
    |cdocsamp.tex| & sample main file \\
    |cdocsch1.tex| & sample include file \\
    |cdocsch2.tex| & sample include file \\
    |cdocspt3.tex| & sample part file \\
    |cdocspt4.tex| & sample part file \\
    |cdocsdrf.tex| & sample redirection file \\
    |cdocsfn1.tex| & sample redirection file \\
    |cdocsfn2.tex| & sample redirection file \\
    |childdoc.pdf| & manual
\end{tabular}
\end{center}
%
The distribution consists of the files
|README.txt|, |childdoc.ins| and |childdoc.dtx|.
%
\begin{itemize}
\item
Run (pdf)\LaTeX{} on |childdoc.dtx|
to compile the manual |childdoc.pdf| (this file).
\item
Run \LaTeX{} on |childdoc.ins| to create the definitions file |childdoc.def|
and the sample |cdocsamp.tex| with include files
|cdocsch1.tex|, |cdocsch2.tex|, |cdocspt3.tex|, |cdocspt4.tex|,
|cdocsdrf.tex|, |cdocsfn1.tex|, |cdocsfn2.tex|.
Then copy the file |childdoc.def| to an appropriate directory of your \LaTeX{}
distribution, e.g.\ \textit{texmf-root}|/tex/latex/childdoc|.
\end{itemize}

%%%%%%%%%%%%%%%%%%%%%%%%%%%%%%%%%%%%%%%%%%%%%%%%%%%%%%%%%%%%%%%%%%%%%%%%%%%%%%%%
\subsection{Related CTAN Packages}

There are several other packages which offer a similar functionality:
%
\begin{itemize}
\item
The packages
\href{http://ctan.org/pkg/docmute}{\textsf{docmute}},
\href{http://ctan.org/pkg/includex}{\textsf{includex}} and
\href{http://ctan.org/pkg/standalone}{\textsf{standalone}}
provide commands to include only the document body of
a child file thus allowing both files to be compiled individually.
\item
The packages \href{http://ctan.org/pkg/subdocs}{\textsf{subdocs}}
and \href{http://ctan.org/pkg/subfiles}{\textsf{subfiles}}
provide structures in which the main and child documents can be
encapsulated and allowing them to be compiled individually.
The inclusion mechanism is different from the conventional |\include|.
\item
The package \href{http://ctan.org/pkg/combine}{\textsf{combine}}
is an elaborate solution to combine several documents into one.
\end{itemize}
%
See also the CTAN topic \href{http://ctan.org/topic/subdocs}{\textsf{subdocs}}
for further related packages.
The present package differs from the above solutions in that
a document structure constructed with the conventional |\include| mechanism
just needs two extra commands at the top of every file
such that all constituent files can be compiled individually.

%%%%%%%%%%%%%%%%%%%%%%%%%%%%%%%%%%%%%%%%%%%%%%%%%%%%%%%%%%%%%%%%%%%%%%%%%%%%%%%%
%\subsection{Feature Suggestions}
%
%The following is a list of features which may be useful for future
%versions of this package:
%%
%\begin{itemize}
%\item
%\ldots
%\end{itemize}

%%%%%%%%%%%%%%%%%%%%%%%%%%%%%%%%%%%%%%%%%%%%%%%%%%%%%%%%%%%%%%%%%%%%%%%%%%%%%%%%
\subsection{Revision History}

%%%%%%%%%%%%%%%%%%%%%%%%%%%%%%%%%%%%%%%%
\paragraph{v2.0:} 2018/12/30

\begin{itemize}
\item
immediate forward processing
\item
added |\childdocby| mechanism
\item
manual restructured
\end{itemize}

%%%%%%%%%%%%%%%%%%%%%%%%%%%%%%%%%%%%%%%%
\paragraph{v1.6:} 2018/01/17

\begin{itemize}
\item
application for development of include files
\item
corrections to manual
\end{itemize}

%%%%%%%%%%%%%%%%%%%%%%%%%%%%%%%%%%%%%%%%
\paragraph{v1.5:} 2017/05/21

\begin{itemize}
\item
more complete structuring introduced
\item
|\childdocof| introduced
\item
|\childdoc| renamed to |\childdocmain|
\item
|\childredirect| renamed to |\childdocforward| and |\childdocforwardprefix|
and functionality expanded
\end{itemize}

%%%%%%%%%%%%%%%%%%%%%%%%%%%%%%%%%%%%%%%%
\paragraph{v1.0:} 2017/04/27

\begin{itemize}
\item
manual and install package
\item
first version published on CTAN
\end{itemize}

%%%%%%%%%%%%%%%%%%%%%%%%%%%%%%%%%%%%%%%%
\paragraph{v0.6:} 2017/04/26

\begin{itemize}
\item
redirection mechanism added
\end{itemize}

%%%%%%%%%%%%%%%%%%%%%%%%%%%%%%%%%%%%%%%%
\paragraph{v0.5:} 2017/04/26

\begin{itemize}
\item
functionality in definition file
\end{itemize}


%%%%%%%%%%%%%%%%%%%%%%%%%%%%%%%%%%%%%%%%%%%%%%%%%%%%%%%%%%%%%%%%%%%%%%%%%%%%%%%%
%%%%%%%%%%%%%%%%%%%%%%%%%%%%%%%%%%%%%%%%%%%%%%%%%%%%%%%%%%%%%%%%%%%%%%%%%%%%%%%%
%%%%%%%%%%%%%%%%%%%%%%%%%%%%%%%%%%%%%%%%%%%%%%%%%%%%%%%%%%%%%%%%%%%%%%%%%%%%%%%%
\appendix

\settowidth\MacroIndent{\rmfamily\scriptsize 000\ }

 \DocInput{childdoc.dtx}

\end{document}
%</driver>
% \fi
%
% %%%%%%%%%%%%%%%%%%%%%%%%%%%%%%%%%%%%%%%%%%%%%%%%%%%%%%%%%%%%%%%%%%%%%%%%%%%%%%
% %%%%%%%%%%%%%%%%%%%%%%%%%%%%%%%%%%%%%%%%%%%%%%%%%%%%%%%%%%%%%%%%%%%%%%%%%%%%%%
% \section{Sample}
%\iffalse
%<*samplemain>
%\fi
%
% The following presents a sample document
% with two chapters, two parts, a title page,
% a compile flag as well as three forwarding files to set the flag.
% It consists of eight |.tex| files:
% \begin{center}
% \begin{tabular}{ll}
% |cdocsamp.tex|&main file\\
% |cdocsch1.tex|&include file for chapter 1\\
% |cdocsch2.tex|&include file for chapter 2\\
% |cdocspt3.tex|&include file for part 3\\
% |cdocspt4.tex|&include file for part 4\\
% |cdocsdrf.tex|&forwarding file for main file in draft mode\\
% |cdocsfi1.tex|&forwarding file for final version of chapter 1\\
% |cdocsfi2.tex|&forwarding file for final version of chapter 2\\
% \end{tabular}
% \end{center}
% Each of the eight files can be compiled directly by the \LaTeX{} compiler.
%
% %%%%%%%%%%%%%%%%%%%%%%%%%%%%%%%%%%%%%%
% \paragraph{Main File.}
%
% The main file is called |cdocsamp.tex|.
%
% Load the \textsf{childdoc} definitions and
% declare the filename for the main document:
%    \begin{macrocode}
\input{childdoc.def}
\childdocmain{}
%    \end{macrocode}

% Optional override for |\version| flag:
%    \begin{macrocode}
%%\ifchilddoc\else\providecommand{\version}{draft}\fi
%    \end{macrocode}

% Define the default values for the |\version| flag
% (|final| for the main file and |draft| for childs):
%    \begin{macrocode}
\ifchilddoc
\providecommand{\version}{draft}
\else
\providecommand{\version}{final}
\fi
%    \end{macrocode}

% Load the standard document class:
%    \begin{macrocode}
\documentclass[12pt]{article}
%    \end{macrocode}

% Start the document body:
%    \begin{macrocode}
\begin{document}
%    \end{macrocode}

% Declare a title page.
% Print title, part of document being processed and version flag:
%    \begin{macrocode}
\addtocounter{page}{-1}
\begin{center}
{\LARGE\bfseries{}childdoc example\par}
\vspace{1cm}
\ifchilddoc
\ifchilddocmanual part\else chapter\fi:
`\childdocname' of `\childdocjob'\par
\else
main document: `\childdocjob'\par
\fi
version: \version\par
\end{center}
\newpage
%    \end{macrocode}

% Manually include selected file,
% otherwise process as usual:
%    \begin{macrocode}
\ifchilddocmanual
\section*{part `\childdocname'}
\input{\childdocname}
\else
%    \end{macrocode}

% Include the two chapters:
%    \begin{macrocode}
\include{cdocsch1}
\include{cdocsch2}
%    \end{macrocode}

% Include the two parts unless only chapters should be displayed:
%    \begin{macrocode}
\ifchilddoc\else
\section{part three}
\input{cdocspt3}
\section{part four}
\input{cdocspt4}
\fi
%    \end{macrocode}

% Process as usual until here:
%    \begin{macrocode}
\fi
%    \end{macrocode}

% End of document body:
%    \begin{macrocode}
\end{document}
%    \end{macrocode}
%\iffalse
%</samplemain>
%\fi
%
% %%%%%%%%%%%%%%%%%%%%%%%%%%%%%%%%%%%%%%
% \paragraph{Chapter Include Files.}
%
% The include files are called |cdocsch1.tex| and |cdocsch2.tex|.
%
%\iffalse
%<*samplechap1|samplechap2>
%\fi

% Optional override for |\version| flag:
%    \begin{macrocode}
%%\providecommand{\version}{final}
%    \end{macrocode}

% Include the main document:
%    \begin{macrocode}
\input{childdoc.def}
\childdocof{cdocsamp}
%    \end{macrocode}

%\iffalse
%</samplechap1|samplechap2>
%\fi
%
%\iffalse
%<*samplechap1>
%\fi
% Some text for chapter 1:
%    \begin{macrocode}
\section{one}
some text in chapter one
%    \end{macrocode}

%\iffalse
%</samplechap1>
%\fi
% Some text for chapter 2:
%\iffalse
%<*samplechap2>
%\fi
%    \begin{macrocode}
\section{two}
more text in chapter two
%    \end{macrocode}

%\iffalse
%</samplechap2>
%\fi
%
% %%%%%%%%%%%%%%%%%%%%%%%%%%%%%%%%%%%%%%
% \paragraph{Part Include Files.}
%
% The include files are called |cdocspt3.tex| and |cdocspt4.tex|.
%
%\iffalse
%<*samplepart3|samplepart4>
%\fi

% Optional override for |\version| flag:
%    \begin{macrocode}
%%\providecommand{\version}{final}
%    \end{macrocode}

% Include the main document:
%    \begin{macrocode}
\input{childdoc.def}
\childdocby{cdocsamp}
%    \end{macrocode}

%\iffalse
%</samplepart3|samplepart4>
%\fi
%
%\iffalse
%<*samplepart3>
%\fi
% Some text for part 3:
%    \begin{macrocode}
some text in part three
%    \end{macrocode}

%\iffalse
%</samplepart3>
%\fi
% Some text for part 4:
%\iffalse
%<*samplepart4>
%\fi
%    \begin{macrocode}
more text in part four
%    \end{macrocode}

%\iffalse
%</samplepart4>
%\fi
%
% %%%%%%%%%%%%%%%%%%%%%%%%%%%%%%%%%%%%%%
% \paragraph{Forwarding for a Complete Draft.}
%
% The following forwarding file |cdocsdrf.tex|
% compiles the main document in draft mode:
%\iffalse
%<*sampledraft>
%\fi
%    \begin{macrocode}
\def\version{draft}
\input{childdoc.def}
\childdocforward{cdocsamp}
%    \end{macrocode}

%\iffalse
%</sampledraft>
%\fi
%
% %%%%%%%%%%%%%%%%%%%%%%%%%%%%%%%%%%%%%%
% \paragraph{Forwarding for Final Version of the Chapters.}
%
% The following forwarding files |cdocsfn1.tex| and |cdocsfn2.tex|
% (with identical content)
% compile the final versions of the child documents
% |cdocsch1.tex| and |cdocsch2.tex|, respectively:
%\iffalse
%<*samplefinal>
%\fi
%    \begin{macrocode}
\def\version{final}
\input{childdoc.def}
\childdocforwardprefix[cdocsamp]{cdocsfn}{cdocsch}
%    \end{macrocode}

%\iffalse
%</samplefinal>
%\fi
%
% %%%%%%%%%%%%%%%%%%%%%%%%%%%%%%%%%%%%%%
% \paragraph{Command Line Processing.}
%
% The following three command lines generate the output files
% |cdocscld|, |cdocscl1| and |cdocscl2|
% which should be identical to
% |cdocsdrf|, |cdocsch1| and |cdocsfn2|, respectively:
% \begin{center}
% \begin{tabular}{l}
% |latex -jobname cdocscld \|\\
% |  "\def\version{draft}\input{childdoc.def}\childdocforward{cdocsamp}"|\\
% |latex -jobname cdocscl1 \|\\
% |  "\input{childdoc.def}\childdocforward[cdocsamp]{cdocsch1}"|\\
% |latex -jobname cdocscl2 \|\\
% |  "\def\version{final}\input{childdoc.def}\childdocforward{cdocsch2}"|
% \end{tabular}
% \end{center}
% Note that the trailing backslash on each first line
% merely continues the input to the second line
% (for convenient cut ant paste).
% Furthermore, the command |latex| can be replaced by any
% of its alternative versions such as |pdflatex|.
%
% %%%%%%%%%%%%%%%%%%%%%%%%%%%%%%%%%%%%%%%%%%%%%%%%%%%%%%%%%%%%%%%%%%%%%%%%%%%%%%
% %%%%%%%%%%%%%%%%%%%%%%%%%%%%%%%%%%%%%%%%%%%%%%%%%%%%%%%%%%%%%%%%%%%%%%%%%%%%%%
% \section{Implementation}
%\iffalse
%<*package>
%\fi
%
% This section describes the definitions file |childdoc.def|.

% The definitions cannot be loaded using |\usepackage| or |\RequirePackage|
% which has a mechanism to prevent loading a style file more than once.
% When loading the definitions by means of |\input|
% multiple instances have to be prevented manually:
%\iffalse
%This code needs to be before the `\ProvidesFile' directive
%which is defined at the beginning of this file.
%Therefore it is also placed there and commented out here.
%</package>
%<*discard>
%\fi
%    \begin{macrocode}
\ifdefined\childdocmain\endinput\fi
%    \end{macrocode}
%\iffalse
%</discard>
%<*package>
%\fi
%
% \macro{\ifchilddoc}
% \macro{\ifchilddocmanual}
% The conditional |\ifchilddoc| tells whether a
% child (true) or main (false) document is being compiled.
% The conditional |\ifchilddocmanual| tells whether
% the |\includeonly| mechanism is used (false) or
% the selection of child files must be performed manually (true).
% The definitions initialise to false:
%    \begin{macrocode}
\newif\ifchilddoc
\newif\ifchilddocmanual
%    \end{macrocode}

% \macro{\childdocname}
% \macro{\childdocjob}
% The macro |\childdocname| stores the name of the main document
% to be compiled. The macro |\childdocjob| stores the name of
% the document on which the \LaTeX{} compiler was originally invoked.
% The content of |\jobname| cannot be compared
% to filenames specified in the source due to different catcodes.
% The following code rescans |\jobname|, stores the result
% in |\childdocname| and saves a copy in |\childdocjob|:
%    \begin{macrocode}
\edef\childdocname{\scantokens\expandafter{\jobname\noexpand}}
\let\childdocjob\childdocname
%    \end{macrocode}

% \macro{\childdocdisable}
% The macro |\childdocdisable| prevents the main file
% from being processed more than once.
% At this stage, the main document command |\childdocmain|
% is assumed to be called once again where it should do nothing.
% Any subsequent call to it should prevent
% a secondary processing of the main document
% It overwrites the forwarding commands
% |\childdocof| and |\childdocforward|
% with empty macros to prevent further inclusions of the main document:
%    \begin{macrocode}
\newcommand{\childdocdisable}
{
  \renewcommand{\childdocmain}[1]{\renewcommand{\childdocmain}[1]{\endinput}}
  \renewcommand{\childdocof}[1]{}
  \renewcommand{\childdocby}[2][]{}
  \renewcommand{\childdocforward}[2][]{}
  \renewcommand{\childdocdisable}{}
}
%    \end{macrocode}

% \macro{\childdocmain}
% The macro |\childdocmain| is to be called at the top of the main file
% with nothing or the main filename (without extension) as argument.
% First, it breaks loops.
% If the argument is not empty and does not match |\childdocname|
% (which is set by the first inclusion of |childdoc.def|),
% |\ifchilddoc| is set to true, |\includeonly| is applied to the child file
% and |\jobname| is set to the main file
% (for proper handling of |.aux| files):
%    \begin{macrocode}
\newcommand{\childdocmain}[1]
{
  \childdocdisable\childdocmain{}
  \if?#1?\else
    \begingroup
      \def\childdoctmp{#1}
      \ifx\childdoctmp\childdocname
        \def\childdoctmp{}
      \else
        \def\childdoctmp
        {
          \childdoctrue
          \includeonly{\childdocname}
          \def\childdocjob{#1}
          \def\jobname{#1}
        }
      \fi
      \expandafter
    \endgroup
    \childdoctmp
  \fi
}
%    \end{macrocode}

% \macro{\childdocof}
% The command |\childdocof| redirects
% compilation to the main file |#1|.
%    \begin{macrocode}
\newcommand{\childdocof}[1]
{
  \childdocdisable
  \childdoctrue
  \includeonly{\childdocname}
  \def\jobname{#1}
  \def\childdocjob{#1}
  \input{#1}
}
%    \end{macrocode}

% \macro{\childdocby}
% The command |\childdocby| ....
%    \begin{macrocode}
\newcommand{\childdocby}[2][]
{
  \childdocdisable
  \childdoctrue
  \childdocmanualtrue
  \if?#1?\else
    \def\jobname{#2}
  \fi
  \def\childdocjob{#2}
  \input{#2}
  \endinput
}
%    \end{macrocode}

% \macro{\childdocforward}
% The command |\childdocforward| redirects
% compilation to the main file or
% (if the optional argument is given) a child file.
% Parameters are set as if the main file
% or a child file starting with |\childdocof| was compiled.
% Then compilation is handed over to the main file:
%    \begin{macrocode}
\newcommand{\childdocforward}[2][]
{
  \begingroup
    \if?#1?
      \def\childdoctmp
      {
        \def\childdocname{#2}
        \def\childdocjob{#2}
        \def\jobname{#2}
        \input{#2}
        \endinput
      }
    \else
      \def\childdoctmp
      {
        \childdocdisable
        \def\childdocname{#2}
        \childdoctrue
        \includeonly{#2}
        \def\childdocjob{#1}
        \def\jobname{#1}
        \input{#1}
        \endinput
      }
    \fi
    \expandafter
  \endgroup
  \childdoctmp
}
%    \end{macrocode}

% \macro{\childdocforwardprefix}
% The command |\childdocforwardprefix| redirects
% compilation to the main or a child file by means of a pattern.
% The prefix |#1| in the current filename is replaced by |#2|
% and the suffix of the current filename is kept
% (it is assumed that the filename does not contain the substring `|~~~|'
% which is used as a delimiter).
% Compilation is handed over to the new file by |\childdocforward|:
%    \begin{macrocode}
\newcommand{\childdocforwardprefix}[3][]
{
  \begingroup
    \def\childdocextract #2##1~~~{\def\childdoctmp{\childdocforward[#1]{#3##1}}}
    \expandafter\childdocextract\childdocname~~~
    \expandafter
  \endgroup
  \childdoctmp
}
%    \end{macrocode}

% \macro{\childdoc}
% The deprecated macro |\childdoc| is a legacy version of |\childdocmain|:
%    \begin{macrocode}
\newcommand{\childdoc}{\childdocmain}
%    \end{macrocode}

% \macro{\childdocredirect}
% The deprecated macro |\childdocredirect| is a legacy version
% of |\childdocforward| and |\childdocforwardprefix|:
%    \begin{macrocode}
\newcommand{\childdocredirect}[2][]
{
  \begingroup
    \if?#1?
      \def\childdoctmp{\childdocforward{#2}}
    \else
      \def\childdoctmp{\childdocforwardprefix{#1}{#2}}
    \fi
    \expandafter
  \endgroup
  \childdoctmp
}
%    \end{macrocode}

%\iffalse
%</package>
%\fi
%
\endinput
|\\
|\childdocmain{}|\\
\end{tabular}
\end{center}
at the very top of the main \LaTeX{} file,
in particular \emph{before} the |\documentclass| statement!
The argument of |\childdocmain| should be left empty
(but it must be present).

%%%%%%%%%%%%%%%%%%%%%%%%%%%%%%%%%%%%%%%%
\DescribeMacro{\childdocof}
Furthermore, add the commands
\begin{center}
\begin{tabular}{l}
|% \iffalse
%
% childdoc.dtx Copyright (C) 2017-2018 Niklas Beisert
%
% This work may be distributed and/or modified under the
% conditions of the LaTeX Project Public License, either version 1.3
% of this license or (at your option) any later version.
% The latest version of this license is in
%   http://www.latex-project.org/lppl.txt
% and version 1.3 or later is part of all distributions of LaTeX
% version 2005/12/01 or later.
%
% This work has the LPPL maintenance status `maintained'.
%
% The Current Maintainer of this work is Niklas Beisert.
%
% This work consists of the files childdoc.dtx and childdoc.ins
% and the derived files childdoc.def and cdocsamp.tex with
% cdocsch1.tex, cdocsch2.tex, cdocsdrf.tex, cdocsfn1.tex, cdocsfn2.tex.
%
%<package>\ifdefined\childdocmain\endinput\fi
%<package>\ProvidesFile{childdoc.def}[2018/12/30 v2.0 child document driver]
%<samplemain>\ProvidesFile{cdocsamp.tex}[2018/12/30 v2.0 sample for childdoc]
%<*driver>
%\ProvidesFile{childdoc.drv}[2018/12/30 v2.0 childdoc reference manual file]
\PassOptionsToClass{10pt,a4paper}{article}
\documentclass{ltxdoc}

\usepackage[margin=35mm]{geometry}
\usepackage{hyperref}
\usepackage{hyperxmp}
\usepackage[usenames]{color}

\hypersetup{colorlinks=true}
\hypersetup{pdfstartview=FitH}
\hypersetup{pdfpagemode=UseNone}
\hypersetup{pdfsource={}}
\hypersetup{pdflang={en-UK}}
\hypersetup{pdfcopyright={Copyright 2017-2018 Niklas Beisert.
  This work may be distributed and/or modified under the
  conditions of the LaTeX Project Public License, either version 1.3
  of this license or (at your option) any later version.}}
\hypersetup{pdflicenseurl={http://www.latex-project.org/lppl.txt}}
\hypersetup{pdfcontactaddress={ETH Zurich, ITP, HIT K,
  Wolfgang-Pauli-Strasse 27}}
\hypersetup{pdfcontactpostcode={8093}}
\hypersetup{pdfcontactcity={Zurich}}
\hypersetup{pdfcontactcountry={Switzerland}}
\hypersetup{pdfcontactemail={nbeisert@itp.phys.ethz.ch}}
\hypersetup{pdfcontacturl={http://people.phys.ethz.ch/\xmptilde nbeisert/}}

\newcommand{\secref}[1]{\hyperref[#1]{section \ref*{#1}}}

\parskip1ex
\parindent0pt
\let\olditemize\itemize
\def\itemize{\olditemize\parskip0pt}

\begin{document}

\title{The \textsf{childdoc} Package}
\hypersetup{pdftitle={The childdoc Package}}
\author{Niklas Beisert\\[2ex]
  Institut f\"ur Theoretische Physik\\
  Eidgen\"ossische Technische Hochschule Z\"urich\\
  Wolfgang-Pauli-Strasse 27, 8093 Z\"urich, Switzerland\\[1ex]
  \href{mailto:nbeisert@itp.phys.ethz.ch}
  {\texttt{nbeisert@itp.phys.ethz.ch}}}
\hypersetup{pdfauthor={Niklas Beisert}}
\hypersetup{pdfsubject={Manual for the LaTeX2e Package childdoc}}
\date{30 December 2018, \textsf{v2.0}}
\maketitle

\begin{abstract}\noindent
\textsf{childdoc} is a \LaTeXe{} package
that enables the direct compilation
of document sections included by |\include|
to individual files.
\end{abstract}

\begingroup
\parskip0ex
\tableofcontents
\endgroup

%%%%%%%%%%%%%%%%%%%%%%%%%%%%%%%%%%%%%%%%%%%%%%%%%%%%%%%%%%%%%%%%%%%%%%%%%%%%%%%%
%%%%%%%%%%%%%%%%%%%%%%%%%%%%%%%%%%%%%%%%%%%%%%%%%%%%%%%%%%%%%%%%%%%%%%%%%%%%%%%%
\section{Introduction}

\LaTeX{} provides a mechanism to structure a large document (such as a book)
into a main file and several child files (containing the chapters)
using the |\include| command.
This mechanism is beneficial for documents
which span hundreds of pages in order to
make the source file(s) more manageable.
Moreover, compilation can be restricted to
selected child files by means of the |\includeonly| command.
The latter feature can be used to reduce the compilation time while editing
(this was significantly more useful in the earlier days of \LaTeX{})
or to generate a smaller document which is easier to navigate.
Another application of |\includeonly| is to generate
documents consisting of selected parts of the complete document.

However, there are a few drawbacks of the plain |\include| mechanism:
\begin{itemize}
\item
The child files cannot be compiled on their own,
they can only be compiled via the main file.
A naive editing environment
(such as a text editor with an option
to have the current file processed by \LaTeX)
may require one to switch to the main file before compiling;
attempting to compile the child file produces errors.
\item
The main file must be modified (each time)
to adjust the |\includeonly| command
to the present needs. This easily leaves the main file in a messy state.
\item
The generated document will always carry the filename
of the main document. This is inconvenient if
several child files are to be compiled and
to be kept for distribution.
\end{itemize}

The present package provides a simple interface
to make child files individually compilable by \LaTeX{}.
Compiling a child file then has the same effect as compiling
the main file with an |\includeonly| command
to select the appropriate child.
Moreover the generated document will carry the name of the child
rather than the main file.
This resolves all three above issues.

This feature is meant to make the editing of books,
thesis documents and lecture notes somewhat more convenient.
However, the package can also be used efficiently for
composing a series of documents (such as exercise sheets)
which are typically distributed individually.
It then assists the author in generating the individual documents
(potentially in different versions)
as well as a document containing the collected series.
Another application is in developing style files
or other kinds of included material
where compilation of the style file could redirect
to a sample or test file.

%%%%%%%%%%%%%%%%%%%%%%%%%%%%%%%%%%%%%%%%%%%%%%%%%%%%%%%%%%%%%%%%%%%%%%%%%%%%%%%%
%%%%%%%%%%%%%%%%%%%%%%%%%%%%%%%%%%%%%%%%%%%%%%%%%%%%%%%%%%%%%%%%%%%%%%%%%%%%%%%%
\section{Usage}

First of all, the package \textsf{childdoc} is \emph{not} a standard
\LaTeXe{} |.sty| style file! Therefore it needs to be invoked in
a non-standard way.

%%%%%%%%%%%%%%%%%%%%%%%%%%%%%%%%%%%%%%%%%%%%%%%%%%%%%%%%%%%%%%%%%%%%%%%%%%%%%%%%
\subsection{Included Files}
\label{sec:include}

%%%%%%%%%%%%%%%%%%%%%%%%%%%%%%%%%%%%%%%%
\DescribeMacro{\childdocmain}
To use the package, add the commands
\begin{center}
\begin{tabular}{l}
|\input{childdoc.def}|\\
|\childdocmain{}|\\
\end{tabular}
\end{center}
at the very top of the main \LaTeX{} file,
in particular \emph{before} the |\documentclass| statement!
The argument of |\childdocmain| should be left empty
(but it must be present).

%%%%%%%%%%%%%%%%%%%%%%%%%%%%%%%%%%%%%%%%
\DescribeMacro{\childdocof}
Furthermore, add the commands
\begin{center}
\begin{tabular}{l}
|\input{childdoc.def}|\\
|\childdocof{|\textit{main}|}|\\
\end{tabular}
\end{center}
at the top of every child file \textit{child}
which is included by |\include{|\textit{child}|}|
from within the main file
(or at least for those files to be compiled individually).
The argument \textit{main} must be the filename of the main file.

There are a couple of
considerations in setting up the main and child documents:

%%%%%%%%%%%%%%%%%%%%%%%%%%%%%%%%%%%%%%%%
\paragraph{Restrictions.}

Please note the following restrictions:
\begin{itemize}
\item
|\childdocmain| must be called with one argument \textit{main}
to ensure compatibility with earlier version of the package.
It must either be empty (|\childdocmain{}|)
or precisely match the filename of the main file in which it is specified.
See \secref{sec:detection} for further information.
\item
The filename \textit{main} must be specified without the |.tex| extension.
\item
The filename \textit{main} is case sensitive
(even in case-insensitive file systems)
due to internal string comparison.
\item
The argument \textit{main} should be fully expanded, it cannot be a macro.
\item
Subdirectories and special characters should be avoided in filenames.
\item
The command |\childdocmain{|\textit{main}|}| must be followed by a whitespace.
It should not be followed immediately by another command
or by a comment mark `|%|'.
This is because the \TeX{} parser reads the token immediately following
the argument of |\childdocmain| and puts it
at the beginning of every child section;
however, a white\-space is ignored.
\end{itemize}

%%%%%%%%%%%%%%%%%%%%%%%%%%%%%%%%%%%%%%%%
\paragraph{Content of Main File.}

It is advisable to place all content in the child files included by |\include|.
Any output contained in the main file will appear in all child documents
unless suppressed manually;
it cannot be suppressed automatically by the |\includeonly| directive
and thus should normally be avoided.
A method to include some content in the main file
by means of conditional processing is described in \secref{sec:conditional}.

%%%%%%%%%%%%%%%%%%%%%%%%%%%%%%%%%%%%%%%%
\paragraph{Page Numbering.}

When only a part of the document is compiled,
the appropriate numbering of pages
(as well as other status parameters)
is determined from the |.aux| files.
The latter contain information from previous passes.
However this information needs to propagate through
all intermediate child documents.
Therefore the page numbering in child documents may well
be inconsistent until the complete document is compiled at least once.

A useful (if unconventional) way to always ensure a consistent
page numbering is to restart the numbering in each child document
and denote the pages by `\textit{child}|.|\textit{page}'
where \textit{child} represents the chapter/section number of the child file.
This can be achieved by the command
|\numberwithin{page}{|\textit{child}|}|
of the \textsf{amsmath} package
where \textit{child} can be |chapter| or |section|
depending on the chosen structuring.
Alternatively, one can modify the macro |\thepage| appropriately
and reset the counter |page| at the start of each child file.

%%%%%%%%%%%%%%%%%%%%%%%%%%%%%%%%%%%%%%%%%%%%%%%%%%%%%%%%%%%%%%%%%%%%%%%%%%%%%%%%
\subsection{Conditional Processing}
\label{sec:conditional}

The package provides a mechanism to compile different versions
of a document. To customise the versions further some conditional processing
can come in handy to distinguish which version is being compiled.
The package provides two macros to describe the compilation context:

%%%%%%%%%%%%%%%%%%%%%%%%%%%%%%%%%%%%%%%%
\DescribeMacro{\ifchilddoc}
The conditional |\ifchilddoc| distinguishes between the compilation of
child documents and the main document:
%
\begin{center}
|\ifchilddoc |\textit{child-code}| |[|\||else |\textit{main-code}]| \||fi|
\end{center}

%%%%%%%%%%%%%%%%%%%%%%%%%%%%%%%%%%%%%%%%
\DescribeMacro{\childdocname}
\DescribeMacro{\childdocjob}
The macro |\childdocname| contains the filename (without extension)
of the main or child file being processed.
Note that |\childdocjob| will always contain the name of the main file.

%%%%%%%%%%%%%%%%%%%%%%%%%%%%%%%%%%%%%%%%
\paragraph{Title Page.}

Conditional processing can be used to include a title or banner page
in the main document when proper precautions are taken.
Importantly, the code in the main file should ensure that the page counter
(as well as other status parameters which are stored in the |.aux| files)
takes the same value after the conditional processing.
Otherwise the page numbers may take divergent values
depending on which part is compiled.

For example, a title page could be declared by:
%
\begin{center}
\begin{tabular}{l}
|\ifchilddoc\||else|\\
|\addtocounter{page}{-1}|\\
\textit{code for title page}\\
|\newpage|\\
|\||fi|
\end{tabular}
\end{center}
%
A banner page for the child documents can be generated by:
%
\begin{center}
\begin{tabular}{l}
|\ifchilddoc|\\
|\addtocounter{page}{-1}|\\
\textit{code for banner page}\\
|\newpage|\\
|\||fi|
\end{tabular}
\end{center}
%
Here one could write a message such as:
\begin{center}
|This is the part \childdocname{} of \childdocjob{}.|
\end{center}

%%%%%%%%%%%%%%%%%%%%%%%%%%%%%%%%%%%%%%%%%%%%%%%%%%%%%%%%%%%%%%%%%%%%%%%%%%%%%%%%
\subsection{Flags}
\label{sec:flags}

The package makes it easy to generate different versions
of the main or child documents.
To this end compilation flags can be defined
and assigned different default values.
They will be particularly useful in conjunction
with the forwarding mechanism described in \secref{sec:forward}.

For example, it may be useful to have a flag |\version|
which can be set to |draft| or |final|.
The document source will contain some conditional code
depending on the value of |\version|.
Suppose further, the flag should default to |final| for the main file
and to |draft| for child files
which is a natural assignment for editing the document.
This is achieved by placing the following code
in the preamble of the main document
(below the |\childdocmain| directive):
%
\begin{center}
\begin{tabular}{l}
|\ifchilddoc|\\
|\providecommand{\version}{draft}|\\
|\||else|\\
|\providecommand{\version}{final}|\\
|\||fi|
\end{tabular}
\end{center}
%
The definition by |\providecommand| makes sure
that previous definitions are not overwritten.
Further statements |\providecommand{\version}{...}|
can thus be added before the above code to override it.

For the main file, one might add a line
(between |\childdocmain| and the above block)
%
\begin{center}
|%\ifchilddoc\||else\providecommand{\version}{draft}\||fi|
\end{center}
%
which can be uncommented to produce a draft version.
Likewise one can add a line to the very top of a child file
(above the |\childdocof{|\textit{main}|}| directive)
%
\begin{center}
|%\providecommand{\version}{final}|
\end{center}
%
which can be uncommented to produce the final version of this child document.

%%%%%%%%%%%%%%%%%%%%%%%%%%%%%%%%%%%%%%%%%%%%%%%%%%%%%%%%%%%%%%%%%%%%%%%%%%%%%%%%
\subsection{Forwarding}
\label{sec:forward}

Different versions of the main or child documents
using compilation flags as described in \secref{sec:flags}
can be (permanently) stored in different files
for convenient compilation, viewing and distribution.
To this end, the package defines a command
to pass on compilation to a different file:

%%%%%%%%%%%%%%%%%%%%%%%%%%%%%%%%%%%%%%%%
\DescribeMacro{\childdocforward}
The command |\childdocforward| redirects processing to
another source file:
%
\begin{center}
\begin{tabular}{l}
|\input{childdoc.def}|\\
|\childdocforward[|\textit{main}|]{|\textit{dest}|}|\\
\end{tabular}
\end{center}
%
The argument \textit{dest} is the destination file
(without extension).
It should be the main file or one of the child files.
Note that further \textsf{childdoc} directives
such as |\childdocof| and |\childdocforward|
in the indicated file will be processed in this form.
The optional argument \textit{main}
passes on directly to the main file \textit{main}
while pretending to compile the child \textit{dest}.
This form behaves as if \textit{dest}
issues |\childdocof{|\textit{main}|}| right away,
and no further \textsf{childdoc} directives will be processed.

%%%%%%%%%%%%%%%%%%%%%%%%%%%%%%%%%%%%%%%%
\DescribeMacro{\...prefix}
In the alternative form |\childdocforwardprefix|,
%
\begin{center}
\begin{tabular}{l}
|\input{childdoc.def}|\\
|\childdocforwardprefix[|\textit{main}|]{|\textit{prefix}|}{|\textit{dest}|}|
\end{tabular}
\end{center}
%
the destination file is determined by a pattern
depending on the current file:
To make this work, the current file must be called
`{\textit{prefix}\hspace{0.2em}\textit{suffix}}'
with \textit{prefix} matching precisely the argument.
Processing is then passed on to the file
`{\textit{dest}\hspace{0.2em}\textit{suffix}}'.
Surely, the same effect is achieved by
directly specifying the
argument `{\textit{dest}\hspace{0.2em}\textit{suffix}}'
in the first form.
However, that requires to set up a different file
for each child. With the alternative form of the command
all these files can have exactly the same content
which simplifies setting them up and maintaining them.

For example, the following file |draft.tex|
with a compilation flag |\version| as described in \secref{sec:flags}
compiles the main document as a draft:
%
\begin{center}
\begin{tabular}{l}
|\def\version{draft}|\\
|\input{childdoc.def}|\\
|\childdocforward{|\textit{main}|}|
\end{tabular}
\end{center}
%
Likewise, the following files |final|\textit{nn}|.tex|
compile the final version of the child document
|child|\textit{nn}|.tex|:
%
\begin{center}
\begin{tabular}{l}
|\def\version{final}|\\
|\input{childdoc.def}|\\
|\childdocforwardprefix{final}{child}|
\end{tabular}
\end{center}
%

Note that when several versions of a main file and/or of each child file
are to be generated, it may be convenient to set up a |Makefile| or
shell script to automatise the process.

%%%%%%%%%%%%%%%%%%%%%%%%%%%%%%%%%%%%%%%%%%%%%%%%%%%%%%%%%%%%%%%%%%%%%%%%%%%%%%%%
\subsection{Command Line Processing}
\label{sec:commandline}

The effect of redirection files can also be achieved by invoking
the \LaTeX{} compiler with a more elaborate command line.
Most conveniently this should be done as part
of a shell script or a |Makefile|.

When using \textsf{childdoc} in the main file, the following
command lines effectively perform a redirection
(note that depending on the shell being used,
backslashes may have to be doubled: `|\|' $\to$ `|\\|'):
%
\begin{center}
|... -jobname "|\textit{target}|" |\\|"|[\textit{flags}]%
|\input{childdoc.def}\childdocforward[|\textit{main}|]{|\textit{dest}|}"|
\end{center}
%
Here \textit{target} is the name of the output file,
\textit{main} is the name of the main file
and \textit{dest} is the name of the main or child file to be processed
(all filenames without extensions).
The optional argument \textit{main} can be omitted
if \textit{main} matches \textit{dest}.
Optionally, compilation \textit{flags} can be defined via |\def| commands.
This command line makes the \TeX{} engine believe
it is compiling the file \textit{target}
whose content is specified as the latter parameter.
The provided code then forwards the processing to
\textit{main} or \textit{dest} as described in \secref{sec:forward}.

%%%%%%%%%%%%%%%%%%%%%%%%%%%%%%%%%%%%%%%%%%%%%%%%%%%%%%%%%%%%%%%%%%%%%%%%%%%%%%%%
\subsection{Include by Input}
\label{sec:input}

Including child documents by |\include| has some restrictions by design.
Most notably, the content of a child document always occupies
its own set of pages; pages cannot be shared between child documents.
Usually, this behaviour makes perfect sense
because each child document contain an essential part of the document.
However, in some situations it may be desirable to compose
a document from a collection of parts
without having mandatory page breaks between then.
For this case, the package
provides a mechanism to include parts
by |\input| which can also be processed individually.
However, by construction this mechanism
requires manual handling of the content to be output.

%%%%%%%%%%%%%%%%%%%%%%%%%%%%%%%%%%%%%%%%
\DescribeMacro{\ifchilddocmanual}
The main file should be prepared as usual, see \secref{sec:include}.
However, the document body must make a distinction
between processing of an individual part and of the main document, e.g.:
%
\begin{center}
\begin{tabular}{l}
|\ifchilddocmanual|\\
|\input{\childdocname}|\\
|\||else|\\
\textit{document body with }|\input{|\textit{part}|}|\\
|\||fi|
\end{tabular}
\end{center}
%
The conditional |\ifchilddocmanual| is true whenever
a part to be included by |\input| is being compiled,
and the name of the part is stored in |\childdocname|.

%%%%%%%%%%%%%%%%%%%%%%%%%%%%%%%%%%%%%%%%
\DescribeMacro{\childdocby}
Each part to be included by |\input| should start with:
%
\begin{center}
\begin{tabular}{l}
|\input{childdoc.def}|\\
|\childdocby{|\textit{main}|}|\\
\end{tabular}
\end{center}
%
The directive |\childdocby| is similar to |\childdocof|
described in \secref{sec:include},
but the subsequent selection of content must be done manually.
To that end, both |\ifchilddoc| and |\ifchilddocmanual|
will be true upon processing of a part,
and the name of the part is stored in |\childdocname|.
Note that |\jobname| will be set to the filename of the current part
so that each part receives an individual |.aux| file
that does not interfere with the |.aux| file(s) of the main document.
This behaviour can be altered by the alternative form
|\childdocby[*]{|\textit{main}|}| (with a non-empty optional argument)
which uses the |.aux| file of the main document
by setting |\jobname| to \textit{main}.

%%%%%%%%%%%%%%%%%%%%%%%%%%%%%%%%%%%%%%%%%%%%%%%%%%%%%%%%%%%%%%%%%%%%%%%%%%%%%%%%
\subsection{Driver Development}
\label{sec:driver}

The \textsf{childdoc} mechanism can also be use for the development
of definition files such as \LaTeX{} styles or classes.
This case differs from the above setup with multiple parts
included by |\include| in that no |\includeonly| should be invoked.
This can be achieved by starting the include file
(before |\ProvidesPackage|) with:
%
\begin{center}
\begin{tabular}{l}
|\input{childdoc.def}|\\
|\childdocforward{|\textit{main}|}|\\
\end{tabular}
\end{center}
%
or alternatively with:
%
\begin{center}
\begin{tabular}{l}
|\input{childdoc.def}|\\
|\childdocby{|\textit{main}|}|\\
\end{tabular}
\end{center}
%
Both forms have slightly different effects as described above.
The main file is prepared as usual, see \secref{sec:include}.

%%%%%%%%%%%%%%%%%%%%%%%%%%%%%%%%%%%%%%%%%%%%%%%%%%%%%%%%%%%%%%%%%%%%%%%%%%%%%%%%
\subsection{Legacy Detection}
\label{sec:detection}

The directive |\childdocmain| in the main file can detect
whether the complete document or merely a child is to be compiled
even without using the directive |\childdocof|.
This method is deprecated because it is less robust
and there is no compelling reason to use it;
it is merely provided for backward compatibility
and it may be removed in future versions.

If the detection mechanism is to be used,
it is mandatory to correctly specify
the filename of the main file as the argument of |\childdocmain|:
%
\begin{center}
\begin{tabular}{l}
|\input{childdoc.def}|\\
|\childdocmain{|\textit{main}|}|\\
\end{tabular}
\end{center}
%
If |\jobname| does not match the argument \textit{main} of |\childdocmain|,
it is assumed that |\jobname| points to the child file to be compiled.
When using |\childdocmain| with the main file specified as argument,
it suffices to start a child file
with just |\input{|\textit{main}|}|
without loading of the package and using |\childdocof|.
If instead all processing is done
with the appropriate \textsf{childdoc} directives,
the argument of \textit{main} of |\childdocmain| can be empty.

An alternative version of the command line processing described
in \secref{sec:commandline} using the detection mechanism reads:
%
\begin{center}
|... -jobname "|\textit{target}|" "|[\textit{flags}]%
[|\def\jobname{|\textit{dest}|}|]|\input{|\textit{main}|}"|
\end{center}

%%%%%%%%%%%%%%%%%%%%%%%%%%%%%%%%%%%%%%%%%%%%%%%%%%%%%%%%%%%%%%%%%%%%%%%%%%%%%%%%
\subsection{Manual Code}
\label{sec:manual}

In case one cannot be certain whether the definitions file |childdoc.def|
is installed on the target \TeX{} distribution
and one prefers not to ship it,
it is conceivable to paste a few relevant commands into the sources.

To that end, drop all statements |\input{childdoc.def}|
and perform the replacements as outlined below.
Instead of |\childdocmain{|\textit{main}|}| add the following code
to the top of the main file:
%
\begin{center}
\begin{tabular}{l}
|\||ifdefined\childdocname\endinput\||fi\newif\ifchilddoc|\\
|\edef\childdocname{\scantokens\expandafter{\jobname\noexpand}}|\\
|\def\childdocmain{|\textit{main}|}\||ifx\childdocmain\childdocname\||else|\\
|\childdoctrue\includeonly{\childdocname}\let\jobname\childdocmain\||fi|\\
\end{tabular}
\end{center}
%
Instead of |\childdocof{|\textit{main}|}| just include the main file
at the top of each child file:
%
\begin{center}
|\input{|\textit{main}|}|
\end{center}
%
A simple redirection |\childdocforward{|\textit{dest}|}| is achieved by:
%
\begin{center}
|\def\jobname{|\textit{dest}|}\input{\jobname}|
\end{center}
%
The redirection with prefix
|\childdocforwardprefix[|\textit{prefix}|]{|\textit{dest}|}|
is accomplished by:
%
\begin{center}
\begin{tabular}{l}
|{\edef\jobname{\scantokens\expandafter{\jobname\noexpand}}|\\
|\def\redirectjob |\textit{prefix}|#1~~~{\gdef\jobname{|\textit{dest}|#1}}|\\
|\expandafter\redirectjob\jobname~~~}\input{\jobname}|
\end{tabular}
\end{center}

In an alternative approach,
child documents can be compiled by a specific command line
without additional code or specific definitions:
%
\begin{center}
|... -jobname "|\textit{target}|" "|[\textit{flags}]%
|\includeonly{|\textit{dest}|}\input{|\textit{main}|}"|
\end{center}
%

%%%%%%%%%%%%%%%%%%%%%%%%%%%%%%%%%%%%%%%%%%%%%%%%%%%%%%%%%%%%%%%%%%%%%%%%%%%%%%%%
%%%%%%%%%%%%%%%%%%%%%%%%%%%%%%%%%%%%%%%%%%%%%%%%%%%%%%%%%%%%%%%%%%%%%%%%%%%%%%%%
\section{Information}

%%%%%%%%%%%%%%%%%%%%%%%%%%%%%%%%%%%%%%%%%%%%%%%%%%%%%%%%%%%%%%%%%%%%%%%%%%%%%%%%
\subsection{Copyright}

Copyright \copyright{} 2017--2018 Niklas Beisert

This work may be distributed and/or modified under the
conditions of the \LaTeX{} Project Public License, either version 1.3
of this license or (at your option) any later version.
The latest version of this license is in
  \url{http://www.latex-project.org/lppl.txt}
and version 1.3 or later is part of all distributions of \LaTeX{}
version 2005/12/01 or later.

This work has the LPPL maintenance status `maintained'.

The Current Maintainer of this work is Niklas Beisert.

This work consists of the files |README.txt|, |childdoc.ins| and |childdoc.dtx|
as well as the derived files |childdoc.def|, |cdocsamp.tex|
with |cdocsch1.tex|, |cdocsch2.tex|, |cdocspt3.tex|, |cdocspt4.tex|,
|cdocsdrf.tex|, |cdocsfn1.tex|, |cdocsfn2.tex|
as well as |childdoc.pdf|.

%%%%%%%%%%%%%%%%%%%%%%%%%%%%%%%%%%%%%%%%%%%%%%%%%%%%%%%%%%%%%%%%%%%%%%%%%%%%%%%%
\subsection{Files and Installation}

The package consists of the files:
%
\begin{center}
\begin{tabular}{ll}
    |README.txt|   & readme file \\
    |childdoc.ins| & installation file \\
    |childdoc.dtx| & source file \\
    |childdoc.def| & definition file \\
    |cdocsamp.tex| & sample main file \\
    |cdocsch1.tex| & sample include file \\
    |cdocsch2.tex| & sample include file \\
    |cdocspt3.tex| & sample part file \\
    |cdocspt4.tex| & sample part file \\
    |cdocsdrf.tex| & sample redirection file \\
    |cdocsfn1.tex| & sample redirection file \\
    |cdocsfn2.tex| & sample redirection file \\
    |childdoc.pdf| & manual
\end{tabular}
\end{center}
%
The distribution consists of the files
|README.txt|, |childdoc.ins| and |childdoc.dtx|.
%
\begin{itemize}
\item
Run (pdf)\LaTeX{} on |childdoc.dtx|
to compile the manual |childdoc.pdf| (this file).
\item
Run \LaTeX{} on |childdoc.ins| to create the definitions file |childdoc.def|
and the sample |cdocsamp.tex| with include files
|cdocsch1.tex|, |cdocsch2.tex|, |cdocspt3.tex|, |cdocspt4.tex|,
|cdocsdrf.tex|, |cdocsfn1.tex|, |cdocsfn2.tex|.
Then copy the file |childdoc.def| to an appropriate directory of your \LaTeX{}
distribution, e.g.\ \textit{texmf-root}|/tex/latex/childdoc|.
\end{itemize}

%%%%%%%%%%%%%%%%%%%%%%%%%%%%%%%%%%%%%%%%%%%%%%%%%%%%%%%%%%%%%%%%%%%%%%%%%%%%%%%%
\subsection{Related CTAN Packages}

There are several other packages which offer a similar functionality:
%
\begin{itemize}
\item
The packages
\href{http://ctan.org/pkg/docmute}{\textsf{docmute}},
\href{http://ctan.org/pkg/includex}{\textsf{includex}} and
\href{http://ctan.org/pkg/standalone}{\textsf{standalone}}
provide commands to include only the document body of
a child file thus allowing both files to be compiled individually.
\item
The packages \href{http://ctan.org/pkg/subdocs}{\textsf{subdocs}}
and \href{http://ctan.org/pkg/subfiles}{\textsf{subfiles}}
provide structures in which the main and child documents can be
encapsulated and allowing them to be compiled individually.
The inclusion mechanism is different from the conventional |\include|.
\item
The package \href{http://ctan.org/pkg/combine}{\textsf{combine}}
is an elaborate solution to combine several documents into one.
\end{itemize}
%
See also the CTAN topic \href{http://ctan.org/topic/subdocs}{\textsf{subdocs}}
for further related packages.
The present package differs from the above solutions in that
a document structure constructed with the conventional |\include| mechanism
just needs two extra commands at the top of every file
such that all constituent files can be compiled individually.

%%%%%%%%%%%%%%%%%%%%%%%%%%%%%%%%%%%%%%%%%%%%%%%%%%%%%%%%%%%%%%%%%%%%%%%%%%%%%%%%
%\subsection{Feature Suggestions}
%
%The following is a list of features which may be useful for future
%versions of this package:
%%
%\begin{itemize}
%\item
%\ldots
%\end{itemize}

%%%%%%%%%%%%%%%%%%%%%%%%%%%%%%%%%%%%%%%%%%%%%%%%%%%%%%%%%%%%%%%%%%%%%%%%%%%%%%%%
\subsection{Revision History}

%%%%%%%%%%%%%%%%%%%%%%%%%%%%%%%%%%%%%%%%
\paragraph{v2.0:} 2018/12/30

\begin{itemize}
\item
immediate forward processing
\item
added |\childdocby| mechanism
\item
manual restructured
\end{itemize}

%%%%%%%%%%%%%%%%%%%%%%%%%%%%%%%%%%%%%%%%
\paragraph{v1.6:} 2018/01/17

\begin{itemize}
\item
application for development of include files
\item
corrections to manual
\end{itemize}

%%%%%%%%%%%%%%%%%%%%%%%%%%%%%%%%%%%%%%%%
\paragraph{v1.5:} 2017/05/21

\begin{itemize}
\item
more complete structuring introduced
\item
|\childdocof| introduced
\item
|\childdoc| renamed to |\childdocmain|
\item
|\childredirect| renamed to |\childdocforward| and |\childdocforwardprefix|
and functionality expanded
\end{itemize}

%%%%%%%%%%%%%%%%%%%%%%%%%%%%%%%%%%%%%%%%
\paragraph{v1.0:} 2017/04/27

\begin{itemize}
\item
manual and install package
\item
first version published on CTAN
\end{itemize}

%%%%%%%%%%%%%%%%%%%%%%%%%%%%%%%%%%%%%%%%
\paragraph{v0.6:} 2017/04/26

\begin{itemize}
\item
redirection mechanism added
\end{itemize}

%%%%%%%%%%%%%%%%%%%%%%%%%%%%%%%%%%%%%%%%
\paragraph{v0.5:} 2017/04/26

\begin{itemize}
\item
functionality in definition file
\end{itemize}


%%%%%%%%%%%%%%%%%%%%%%%%%%%%%%%%%%%%%%%%%%%%%%%%%%%%%%%%%%%%%%%%%%%%%%%%%%%%%%%%
%%%%%%%%%%%%%%%%%%%%%%%%%%%%%%%%%%%%%%%%%%%%%%%%%%%%%%%%%%%%%%%%%%%%%%%%%%%%%%%%
%%%%%%%%%%%%%%%%%%%%%%%%%%%%%%%%%%%%%%%%%%%%%%%%%%%%%%%%%%%%%%%%%%%%%%%%%%%%%%%%
\appendix

\settowidth\MacroIndent{\rmfamily\scriptsize 000\ }

 \DocInput{childdoc.dtx}

\end{document}
%</driver>
% \fi
%
% %%%%%%%%%%%%%%%%%%%%%%%%%%%%%%%%%%%%%%%%%%%%%%%%%%%%%%%%%%%%%%%%%%%%%%%%%%%%%%
% %%%%%%%%%%%%%%%%%%%%%%%%%%%%%%%%%%%%%%%%%%%%%%%%%%%%%%%%%%%%%%%%%%%%%%%%%%%%%%
% \section{Sample}
%\iffalse
%<*samplemain>
%\fi
%
% The following presents a sample document
% with two chapters, two parts, a title page,
% a compile flag as well as three forwarding files to set the flag.
% It consists of eight |.tex| files:
% \begin{center}
% \begin{tabular}{ll}
% |cdocsamp.tex|&main file\\
% |cdocsch1.tex|&include file for chapter 1\\
% |cdocsch2.tex|&include file for chapter 2\\
% |cdocspt3.tex|&include file for part 3\\
% |cdocspt4.tex|&include file for part 4\\
% |cdocsdrf.tex|&forwarding file for main file in draft mode\\
% |cdocsfi1.tex|&forwarding file for final version of chapter 1\\
% |cdocsfi2.tex|&forwarding file for final version of chapter 2\\
% \end{tabular}
% \end{center}
% Each of the eight files can be compiled directly by the \LaTeX{} compiler.
%
% %%%%%%%%%%%%%%%%%%%%%%%%%%%%%%%%%%%%%%
% \paragraph{Main File.}
%
% The main file is called |cdocsamp.tex|.
%
% Load the \textsf{childdoc} definitions and
% declare the filename for the main document:
%    \begin{macrocode}
\input{childdoc.def}
\childdocmain{}
%    \end{macrocode}

% Optional override for |\version| flag:
%    \begin{macrocode}
%%\ifchilddoc\else\providecommand{\version}{draft}\fi
%    \end{macrocode}

% Define the default values for the |\version| flag
% (|final| for the main file and |draft| for childs):
%    \begin{macrocode}
\ifchilddoc
\providecommand{\version}{draft}
\else
\providecommand{\version}{final}
\fi
%    \end{macrocode}

% Load the standard document class:
%    \begin{macrocode}
\documentclass[12pt]{article}
%    \end{macrocode}

% Start the document body:
%    \begin{macrocode}
\begin{document}
%    \end{macrocode}

% Declare a title page.
% Print title, part of document being processed and version flag:
%    \begin{macrocode}
\addtocounter{page}{-1}
\begin{center}
{\LARGE\bfseries{}childdoc example\par}
\vspace{1cm}
\ifchilddoc
\ifchilddocmanual part\else chapter\fi:
`\childdocname' of `\childdocjob'\par
\else
main document: `\childdocjob'\par
\fi
version: \version\par
\end{center}
\newpage
%    \end{macrocode}

% Manually include selected file,
% otherwise process as usual:
%    \begin{macrocode}
\ifchilddocmanual
\section*{part `\childdocname'}
\input{\childdocname}
\else
%    \end{macrocode}

% Include the two chapters:
%    \begin{macrocode}
\include{cdocsch1}
\include{cdocsch2}
%    \end{macrocode}

% Include the two parts unless only chapters should be displayed:
%    \begin{macrocode}
\ifchilddoc\else
\section{part three}
\input{cdocspt3}
\section{part four}
\input{cdocspt4}
\fi
%    \end{macrocode}

% Process as usual until here:
%    \begin{macrocode}
\fi
%    \end{macrocode}

% End of document body:
%    \begin{macrocode}
\end{document}
%    \end{macrocode}
%\iffalse
%</samplemain>
%\fi
%
% %%%%%%%%%%%%%%%%%%%%%%%%%%%%%%%%%%%%%%
% \paragraph{Chapter Include Files.}
%
% The include files are called |cdocsch1.tex| and |cdocsch2.tex|.
%
%\iffalse
%<*samplechap1|samplechap2>
%\fi

% Optional override for |\version| flag:
%    \begin{macrocode}
%%\providecommand{\version}{final}
%    \end{macrocode}

% Include the main document:
%    \begin{macrocode}
\input{childdoc.def}
\childdocof{cdocsamp}
%    \end{macrocode}

%\iffalse
%</samplechap1|samplechap2>
%\fi
%
%\iffalse
%<*samplechap1>
%\fi
% Some text for chapter 1:
%    \begin{macrocode}
\section{one}
some text in chapter one
%    \end{macrocode}

%\iffalse
%</samplechap1>
%\fi
% Some text for chapter 2:
%\iffalse
%<*samplechap2>
%\fi
%    \begin{macrocode}
\section{two}
more text in chapter two
%    \end{macrocode}

%\iffalse
%</samplechap2>
%\fi
%
% %%%%%%%%%%%%%%%%%%%%%%%%%%%%%%%%%%%%%%
% \paragraph{Part Include Files.}
%
% The include files are called |cdocspt3.tex| and |cdocspt4.tex|.
%
%\iffalse
%<*samplepart3|samplepart4>
%\fi

% Optional override for |\version| flag:
%    \begin{macrocode}
%%\providecommand{\version}{final}
%    \end{macrocode}

% Include the main document:
%    \begin{macrocode}
\input{childdoc.def}
\childdocby{cdocsamp}
%    \end{macrocode}

%\iffalse
%</samplepart3|samplepart4>
%\fi
%
%\iffalse
%<*samplepart3>
%\fi
% Some text for part 3:
%    \begin{macrocode}
some text in part three
%    \end{macrocode}

%\iffalse
%</samplepart3>
%\fi
% Some text for part 4:
%\iffalse
%<*samplepart4>
%\fi
%    \begin{macrocode}
more text in part four
%    \end{macrocode}

%\iffalse
%</samplepart4>
%\fi
%
% %%%%%%%%%%%%%%%%%%%%%%%%%%%%%%%%%%%%%%
% \paragraph{Forwarding for a Complete Draft.}
%
% The following forwarding file |cdocsdrf.tex|
% compiles the main document in draft mode:
%\iffalse
%<*sampledraft>
%\fi
%    \begin{macrocode}
\def\version{draft}
\input{childdoc.def}
\childdocforward{cdocsamp}
%    \end{macrocode}

%\iffalse
%</sampledraft>
%\fi
%
% %%%%%%%%%%%%%%%%%%%%%%%%%%%%%%%%%%%%%%
% \paragraph{Forwarding for Final Version of the Chapters.}
%
% The following forwarding files |cdocsfn1.tex| and |cdocsfn2.tex|
% (with identical content)
% compile the final versions of the child documents
% |cdocsch1.tex| and |cdocsch2.tex|, respectively:
%\iffalse
%<*samplefinal>
%\fi
%    \begin{macrocode}
\def\version{final}
\input{childdoc.def}
\childdocforwardprefix[cdocsamp]{cdocsfn}{cdocsch}
%    \end{macrocode}

%\iffalse
%</samplefinal>
%\fi
%
% %%%%%%%%%%%%%%%%%%%%%%%%%%%%%%%%%%%%%%
% \paragraph{Command Line Processing.}
%
% The following three command lines generate the output files
% |cdocscld|, |cdocscl1| and |cdocscl2|
% which should be identical to
% |cdocsdrf|, |cdocsch1| and |cdocsfn2|, respectively:
% \begin{center}
% \begin{tabular}{l}
% |latex -jobname cdocscld \|\\
% |  "\def\version{draft}\input{childdoc.def}\childdocforward{cdocsamp}"|\\
% |latex -jobname cdocscl1 \|\\
% |  "\input{childdoc.def}\childdocforward[cdocsamp]{cdocsch1}"|\\
% |latex -jobname cdocscl2 \|\\
% |  "\def\version{final}\input{childdoc.def}\childdocforward{cdocsch2}"|
% \end{tabular}
% \end{center}
% Note that the trailing backslash on each first line
% merely continues the input to the second line
% (for convenient cut ant paste).
% Furthermore, the command |latex| can be replaced by any
% of its alternative versions such as |pdflatex|.
%
% %%%%%%%%%%%%%%%%%%%%%%%%%%%%%%%%%%%%%%%%%%%%%%%%%%%%%%%%%%%%%%%%%%%%%%%%%%%%%%
% %%%%%%%%%%%%%%%%%%%%%%%%%%%%%%%%%%%%%%%%%%%%%%%%%%%%%%%%%%%%%%%%%%%%%%%%%%%%%%
% \section{Implementation}
%\iffalse
%<*package>
%\fi
%
% This section describes the definitions file |childdoc.def|.

% The definitions cannot be loaded using |\usepackage| or |\RequirePackage|
% which has a mechanism to prevent loading a style file more than once.
% When loading the definitions by means of |\input|
% multiple instances have to be prevented manually:
%\iffalse
%This code needs to be before the `\ProvidesFile' directive
%which is defined at the beginning of this file.
%Therefore it is also placed there and commented out here.
%</package>
%<*discard>
%\fi
%    \begin{macrocode}
\ifdefined\childdocmain\endinput\fi
%    \end{macrocode}
%\iffalse
%</discard>
%<*package>
%\fi
%
% \macro{\ifchilddoc}
% \macro{\ifchilddocmanual}
% The conditional |\ifchilddoc| tells whether a
% child (true) or main (false) document is being compiled.
% The conditional |\ifchilddocmanual| tells whether
% the |\includeonly| mechanism is used (false) or
% the selection of child files must be performed manually (true).
% The definitions initialise to false:
%    \begin{macrocode}
\newif\ifchilddoc
\newif\ifchilddocmanual
%    \end{macrocode}

% \macro{\childdocname}
% \macro{\childdocjob}
% The macro |\childdocname| stores the name of the main document
% to be compiled. The macro |\childdocjob| stores the name of
% the document on which the \LaTeX{} compiler was originally invoked.
% The content of |\jobname| cannot be compared
% to filenames specified in the source due to different catcodes.
% The following code rescans |\jobname|, stores the result
% in |\childdocname| and saves a copy in |\childdocjob|:
%    \begin{macrocode}
\edef\childdocname{\scantokens\expandafter{\jobname\noexpand}}
\let\childdocjob\childdocname
%    \end{macrocode}

% \macro{\childdocdisable}
% The macro |\childdocdisable| prevents the main file
% from being processed more than once.
% At this stage, the main document command |\childdocmain|
% is assumed to be called once again where it should do nothing.
% Any subsequent call to it should prevent
% a secondary processing of the main document
% It overwrites the forwarding commands
% |\childdocof| and |\childdocforward|
% with empty macros to prevent further inclusions of the main document:
%    \begin{macrocode}
\newcommand{\childdocdisable}
{
  \renewcommand{\childdocmain}[1]{\renewcommand{\childdocmain}[1]{\endinput}}
  \renewcommand{\childdocof}[1]{}
  \renewcommand{\childdocby}[2][]{}
  \renewcommand{\childdocforward}[2][]{}
  \renewcommand{\childdocdisable}{}
}
%    \end{macrocode}

% \macro{\childdocmain}
% The macro |\childdocmain| is to be called at the top of the main file
% with nothing or the main filename (without extension) as argument.
% First, it breaks loops.
% If the argument is not empty and does not match |\childdocname|
% (which is set by the first inclusion of |childdoc.def|),
% |\ifchilddoc| is set to true, |\includeonly| is applied to the child file
% and |\jobname| is set to the main file
% (for proper handling of |.aux| files):
%    \begin{macrocode}
\newcommand{\childdocmain}[1]
{
  \childdocdisable\childdocmain{}
  \if?#1?\else
    \begingroup
      \def\childdoctmp{#1}
      \ifx\childdoctmp\childdocname
        \def\childdoctmp{}
      \else
        \def\childdoctmp
        {
          \childdoctrue
          \includeonly{\childdocname}
          \def\childdocjob{#1}
          \def\jobname{#1}
        }
      \fi
      \expandafter
    \endgroup
    \childdoctmp
  \fi
}
%    \end{macrocode}

% \macro{\childdocof}
% The command |\childdocof| redirects
% compilation to the main file |#1|.
%    \begin{macrocode}
\newcommand{\childdocof}[1]
{
  \childdocdisable
  \childdoctrue
  \includeonly{\childdocname}
  \def\jobname{#1}
  \def\childdocjob{#1}
  \input{#1}
}
%    \end{macrocode}

% \macro{\childdocby}
% The command |\childdocby| ....
%    \begin{macrocode}
\newcommand{\childdocby}[2][]
{
  \childdocdisable
  \childdoctrue
  \childdocmanualtrue
  \if?#1?\else
    \def\jobname{#2}
  \fi
  \def\childdocjob{#2}
  \input{#2}
  \endinput
}
%    \end{macrocode}

% \macro{\childdocforward}
% The command |\childdocforward| redirects
% compilation to the main file or
% (if the optional argument is given) a child file.
% Parameters are set as if the main file
% or a child file starting with |\childdocof| was compiled.
% Then compilation is handed over to the main file:
%    \begin{macrocode}
\newcommand{\childdocforward}[2][]
{
  \begingroup
    \if?#1?
      \def\childdoctmp
      {
        \def\childdocname{#2}
        \def\childdocjob{#2}
        \def\jobname{#2}
        \input{#2}
        \endinput
      }
    \else
      \def\childdoctmp
      {
        \childdocdisable
        \def\childdocname{#2}
        \childdoctrue
        \includeonly{#2}
        \def\childdocjob{#1}
        \def\jobname{#1}
        \input{#1}
        \endinput
      }
    \fi
    \expandafter
  \endgroup
  \childdoctmp
}
%    \end{macrocode}

% \macro{\childdocforwardprefix}
% The command |\childdocforwardprefix| redirects
% compilation to the main or a child file by means of a pattern.
% The prefix |#1| in the current filename is replaced by |#2|
% and the suffix of the current filename is kept
% (it is assumed that the filename does not contain the substring `|~~~|'
% which is used as a delimiter).
% Compilation is handed over to the new file by |\childdocforward|:
%    \begin{macrocode}
\newcommand{\childdocforwardprefix}[3][]
{
  \begingroup
    \def\childdocextract #2##1~~~{\def\childdoctmp{\childdocforward[#1]{#3##1}}}
    \expandafter\childdocextract\childdocname~~~
    \expandafter
  \endgroup
  \childdoctmp
}
%    \end{macrocode}

% \macro{\childdoc}
% The deprecated macro |\childdoc| is a legacy version of |\childdocmain|:
%    \begin{macrocode}
\newcommand{\childdoc}{\childdocmain}
%    \end{macrocode}

% \macro{\childdocredirect}
% The deprecated macro |\childdocredirect| is a legacy version
% of |\childdocforward| and |\childdocforwardprefix|:
%    \begin{macrocode}
\newcommand{\childdocredirect}[2][]
{
  \begingroup
    \if?#1?
      \def\childdoctmp{\childdocforward{#2}}
    \else
      \def\childdoctmp{\childdocforwardprefix{#1}{#2}}
    \fi
    \expandafter
  \endgroup
  \childdoctmp
}
%    \end{macrocode}

%\iffalse
%</package>
%\fi
%
\endinput
|\\
|\childdocof{|\textit{main}|}|\\
\end{tabular}
\end{center}
at the top of every child file \textit{child}
which is included by |\include{|\textit{child}|}|
from within the main file
(or at least for those files to be compiled individually).
The argument \textit{main} must be the filename of the main file.

There are a couple of
considerations in setting up the main and child documents:

%%%%%%%%%%%%%%%%%%%%%%%%%%%%%%%%%%%%%%%%
\paragraph{Restrictions.}

Please note the following restrictions:
\begin{itemize}
\item
|\childdocmain| must be called with one argument \textit{main}
to ensure compatibility with earlier version of the package.
It must either be empty (|\childdocmain{}|)
or precisely match the filename of the main file in which it is specified.
See \secref{sec:detection} for further information.
\item
The filename \textit{main} must be specified without the |.tex| extension.
\item
The filename \textit{main} is case sensitive
(even in case-insensitive file systems)
due to internal string comparison.
\item
The argument \textit{main} should be fully expanded, it cannot be a macro.
\item
Subdirectories and special characters should be avoided in filenames.
\item
The command |\childdocmain{|\textit{main}|}| must be followed by a whitespace.
It should not be followed immediately by another command
or by a comment mark `|%|'.
This is because the \TeX{} parser reads the token immediately following
the argument of |\childdocmain| and puts it
at the beginning of every child section;
however, a white\-space is ignored.
\end{itemize}

%%%%%%%%%%%%%%%%%%%%%%%%%%%%%%%%%%%%%%%%
\paragraph{Content of Main File.}

It is advisable to place all content in the child files included by |\include|.
Any output contained in the main file will appear in all child documents
unless suppressed manually;
it cannot be suppressed automatically by the |\includeonly| directive
and thus should normally be avoided.
A method to include some content in the main file
by means of conditional processing is described in \secref{sec:conditional}.

%%%%%%%%%%%%%%%%%%%%%%%%%%%%%%%%%%%%%%%%
\paragraph{Page Numbering.}

When only a part of the document is compiled,
the appropriate numbering of pages
(as well as other status parameters)
is determined from the |.aux| files.
The latter contain information from previous passes.
However this information needs to propagate through
all intermediate child documents.
Therefore the page numbering in child documents may well
be inconsistent until the complete document is compiled at least once.

A useful (if unconventional) way to always ensure a consistent
page numbering is to restart the numbering in each child document
and denote the pages by `\textit{child}|.|\textit{page}'
where \textit{child} represents the chapter/section number of the child file.
This can be achieved by the command
|\numberwithin{page}{|\textit{child}|}|
of the \textsf{amsmath} package
where \textit{child} can be |chapter| or |section|
depending on the chosen structuring.
Alternatively, one can modify the macro |\thepage| appropriately
and reset the counter |page| at the start of each child file.

%%%%%%%%%%%%%%%%%%%%%%%%%%%%%%%%%%%%%%%%%%%%%%%%%%%%%%%%%%%%%%%%%%%%%%%%%%%%%%%%
\subsection{Conditional Processing}
\label{sec:conditional}

The package provides a mechanism to compile different versions
of a document. To customise the versions further some conditional processing
can come in handy to distinguish which version is being compiled.
The package provides two macros to describe the compilation context:

%%%%%%%%%%%%%%%%%%%%%%%%%%%%%%%%%%%%%%%%
\DescribeMacro{\ifchilddoc}
The conditional |\ifchilddoc| distinguishes between the compilation of
child documents and the main document:
%
\begin{center}
|\ifchilddoc |\textit{child-code}| |[|\||else |\textit{main-code}]| \||fi|
\end{center}

%%%%%%%%%%%%%%%%%%%%%%%%%%%%%%%%%%%%%%%%
\DescribeMacro{\childdocname}
\DescribeMacro{\childdocjob}
The macro |\childdocname| contains the filename (without extension)
of the main or child file being processed.
Note that |\childdocjob| will always contain the name of the main file.

%%%%%%%%%%%%%%%%%%%%%%%%%%%%%%%%%%%%%%%%
\paragraph{Title Page.}

Conditional processing can be used to include a title or banner page
in the main document when proper precautions are taken.
Importantly, the code in the main file should ensure that the page counter
(as well as other status parameters which are stored in the |.aux| files)
takes the same value after the conditional processing.
Otherwise the page numbers may take divergent values
depending on which part is compiled.

For example, a title page could be declared by:
%
\begin{center}
\begin{tabular}{l}
|\ifchilddoc\||else|\\
|\addtocounter{page}{-1}|\\
\textit{code for title page}\\
|\newpage|\\
|\||fi|
\end{tabular}
\end{center}
%
A banner page for the child documents can be generated by:
%
\begin{center}
\begin{tabular}{l}
|\ifchilddoc|\\
|\addtocounter{page}{-1}|\\
\textit{code for banner page}\\
|\newpage|\\
|\||fi|
\end{tabular}
\end{center}
%
Here one could write a message such as:
\begin{center}
|This is the part \childdocname{} of \childdocjob{}.|
\end{center}

%%%%%%%%%%%%%%%%%%%%%%%%%%%%%%%%%%%%%%%%%%%%%%%%%%%%%%%%%%%%%%%%%%%%%%%%%%%%%%%%
\subsection{Flags}
\label{sec:flags}

The package makes it easy to generate different versions
of the main or child documents.
To this end compilation flags can be defined
and assigned different default values.
They will be particularly useful in conjunction
with the forwarding mechanism described in \secref{sec:forward}.

For example, it may be useful to have a flag |\version|
which can be set to |draft| or |final|.
The document source will contain some conditional code
depending on the value of |\version|.
Suppose further, the flag should default to |final| for the main file
and to |draft| for child files
which is a natural assignment for editing the document.
This is achieved by placing the following code
in the preamble of the main document
(below the |\childdocmain| directive):
%
\begin{center}
\begin{tabular}{l}
|\ifchilddoc|\\
|\providecommand{\version}{draft}|\\
|\||else|\\
|\providecommand{\version}{final}|\\
|\||fi|
\end{tabular}
\end{center}
%
The definition by |\providecommand| makes sure
that previous definitions are not overwritten.
Further statements |\providecommand{\version}{...}|
can thus be added before the above code to override it.

For the main file, one might add a line
(between |\childdocmain| and the above block)
%
\begin{center}
|%\ifchilddoc\||else\providecommand{\version}{draft}\||fi|
\end{center}
%
which can be uncommented to produce a draft version.
Likewise one can add a line to the very top of a child file
(above the |\childdocof{|\textit{main}|}| directive)
%
\begin{center}
|%\providecommand{\version}{final}|
\end{center}
%
which can be uncommented to produce the final version of this child document.

%%%%%%%%%%%%%%%%%%%%%%%%%%%%%%%%%%%%%%%%%%%%%%%%%%%%%%%%%%%%%%%%%%%%%%%%%%%%%%%%
\subsection{Forwarding}
\label{sec:forward}

Different versions of the main or child documents
using compilation flags as described in \secref{sec:flags}
can be (permanently) stored in different files
for convenient compilation, viewing and distribution.
To this end, the package defines a command
to pass on compilation to a different file:

%%%%%%%%%%%%%%%%%%%%%%%%%%%%%%%%%%%%%%%%
\DescribeMacro{\childdocforward}
The command |\childdocforward| redirects processing to
another source file:
%
\begin{center}
\begin{tabular}{l}
|% \iffalse
%
% childdoc.dtx Copyright (C) 2017-2018 Niklas Beisert
%
% This work may be distributed and/or modified under the
% conditions of the LaTeX Project Public License, either version 1.3
% of this license or (at your option) any later version.
% The latest version of this license is in
%   http://www.latex-project.org/lppl.txt
% and version 1.3 or later is part of all distributions of LaTeX
% version 2005/12/01 or later.
%
% This work has the LPPL maintenance status `maintained'.
%
% The Current Maintainer of this work is Niklas Beisert.
%
% This work consists of the files childdoc.dtx and childdoc.ins
% and the derived files childdoc.def and cdocsamp.tex with
% cdocsch1.tex, cdocsch2.tex, cdocsdrf.tex, cdocsfn1.tex, cdocsfn2.tex.
%
%<package>\ifdefined\childdocmain\endinput\fi
%<package>\ProvidesFile{childdoc.def}[2018/12/30 v2.0 child document driver]
%<samplemain>\ProvidesFile{cdocsamp.tex}[2018/12/30 v2.0 sample for childdoc]
%<*driver>
%\ProvidesFile{childdoc.drv}[2018/12/30 v2.0 childdoc reference manual file]
\PassOptionsToClass{10pt,a4paper}{article}
\documentclass{ltxdoc}

\usepackage[margin=35mm]{geometry}
\usepackage{hyperref}
\usepackage{hyperxmp}
\usepackage[usenames]{color}

\hypersetup{colorlinks=true}
\hypersetup{pdfstartview=FitH}
\hypersetup{pdfpagemode=UseNone}
\hypersetup{pdfsource={}}
\hypersetup{pdflang={en-UK}}
\hypersetup{pdfcopyright={Copyright 2017-2018 Niklas Beisert.
  This work may be distributed and/or modified under the
  conditions of the LaTeX Project Public License, either version 1.3
  of this license or (at your option) any later version.}}
\hypersetup{pdflicenseurl={http://www.latex-project.org/lppl.txt}}
\hypersetup{pdfcontactaddress={ETH Zurich, ITP, HIT K,
  Wolfgang-Pauli-Strasse 27}}
\hypersetup{pdfcontactpostcode={8093}}
\hypersetup{pdfcontactcity={Zurich}}
\hypersetup{pdfcontactcountry={Switzerland}}
\hypersetup{pdfcontactemail={nbeisert@itp.phys.ethz.ch}}
\hypersetup{pdfcontacturl={http://people.phys.ethz.ch/\xmptilde nbeisert/}}

\newcommand{\secref}[1]{\hyperref[#1]{section \ref*{#1}}}

\parskip1ex
\parindent0pt
\let\olditemize\itemize
\def\itemize{\olditemize\parskip0pt}

\begin{document}

\title{The \textsf{childdoc} Package}
\hypersetup{pdftitle={The childdoc Package}}
\author{Niklas Beisert\\[2ex]
  Institut f\"ur Theoretische Physik\\
  Eidgen\"ossische Technische Hochschule Z\"urich\\
  Wolfgang-Pauli-Strasse 27, 8093 Z\"urich, Switzerland\\[1ex]
  \href{mailto:nbeisert@itp.phys.ethz.ch}
  {\texttt{nbeisert@itp.phys.ethz.ch}}}
\hypersetup{pdfauthor={Niklas Beisert}}
\hypersetup{pdfsubject={Manual for the LaTeX2e Package childdoc}}
\date{30 December 2018, \textsf{v2.0}}
\maketitle

\begin{abstract}\noindent
\textsf{childdoc} is a \LaTeXe{} package
that enables the direct compilation
of document sections included by |\include|
to individual files.
\end{abstract}

\begingroup
\parskip0ex
\tableofcontents
\endgroup

%%%%%%%%%%%%%%%%%%%%%%%%%%%%%%%%%%%%%%%%%%%%%%%%%%%%%%%%%%%%%%%%%%%%%%%%%%%%%%%%
%%%%%%%%%%%%%%%%%%%%%%%%%%%%%%%%%%%%%%%%%%%%%%%%%%%%%%%%%%%%%%%%%%%%%%%%%%%%%%%%
\section{Introduction}

\LaTeX{} provides a mechanism to structure a large document (such as a book)
into a main file and several child files (containing the chapters)
using the |\include| command.
This mechanism is beneficial for documents
which span hundreds of pages in order to
make the source file(s) more manageable.
Moreover, compilation can be restricted to
selected child files by means of the |\includeonly| command.
The latter feature can be used to reduce the compilation time while editing
(this was significantly more useful in the earlier days of \LaTeX{})
or to generate a smaller document which is easier to navigate.
Another application of |\includeonly| is to generate
documents consisting of selected parts of the complete document.

However, there are a few drawbacks of the plain |\include| mechanism:
\begin{itemize}
\item
The child files cannot be compiled on their own,
they can only be compiled via the main file.
A naive editing environment
(such as a text editor with an option
to have the current file processed by \LaTeX)
may require one to switch to the main file before compiling;
attempting to compile the child file produces errors.
\item
The main file must be modified (each time)
to adjust the |\includeonly| command
to the present needs. This easily leaves the main file in a messy state.
\item
The generated document will always carry the filename
of the main document. This is inconvenient if
several child files are to be compiled and
to be kept for distribution.
\end{itemize}

The present package provides a simple interface
to make child files individually compilable by \LaTeX{}.
Compiling a child file then has the same effect as compiling
the main file with an |\includeonly| command
to select the appropriate child.
Moreover the generated document will carry the name of the child
rather than the main file.
This resolves all three above issues.

This feature is meant to make the editing of books,
thesis documents and lecture notes somewhat more convenient.
However, the package can also be used efficiently for
composing a series of documents (such as exercise sheets)
which are typically distributed individually.
It then assists the author in generating the individual documents
(potentially in different versions)
as well as a document containing the collected series.
Another application is in developing style files
or other kinds of included material
where compilation of the style file could redirect
to a sample or test file.

%%%%%%%%%%%%%%%%%%%%%%%%%%%%%%%%%%%%%%%%%%%%%%%%%%%%%%%%%%%%%%%%%%%%%%%%%%%%%%%%
%%%%%%%%%%%%%%%%%%%%%%%%%%%%%%%%%%%%%%%%%%%%%%%%%%%%%%%%%%%%%%%%%%%%%%%%%%%%%%%%
\section{Usage}

First of all, the package \textsf{childdoc} is \emph{not} a standard
\LaTeXe{} |.sty| style file! Therefore it needs to be invoked in
a non-standard way.

%%%%%%%%%%%%%%%%%%%%%%%%%%%%%%%%%%%%%%%%%%%%%%%%%%%%%%%%%%%%%%%%%%%%%%%%%%%%%%%%
\subsection{Included Files}
\label{sec:include}

%%%%%%%%%%%%%%%%%%%%%%%%%%%%%%%%%%%%%%%%
\DescribeMacro{\childdocmain}
To use the package, add the commands
\begin{center}
\begin{tabular}{l}
|\input{childdoc.def}|\\
|\childdocmain{}|\\
\end{tabular}
\end{center}
at the very top of the main \LaTeX{} file,
in particular \emph{before} the |\documentclass| statement!
The argument of |\childdocmain| should be left empty
(but it must be present).

%%%%%%%%%%%%%%%%%%%%%%%%%%%%%%%%%%%%%%%%
\DescribeMacro{\childdocof}
Furthermore, add the commands
\begin{center}
\begin{tabular}{l}
|\input{childdoc.def}|\\
|\childdocof{|\textit{main}|}|\\
\end{tabular}
\end{center}
at the top of every child file \textit{child}
which is included by |\include{|\textit{child}|}|
from within the main file
(or at least for those files to be compiled individually).
The argument \textit{main} must be the filename of the main file.

There are a couple of
considerations in setting up the main and child documents:

%%%%%%%%%%%%%%%%%%%%%%%%%%%%%%%%%%%%%%%%
\paragraph{Restrictions.}

Please note the following restrictions:
\begin{itemize}
\item
|\childdocmain| must be called with one argument \textit{main}
to ensure compatibility with earlier version of the package.
It must either be empty (|\childdocmain{}|)
or precisely match the filename of the main file in which it is specified.
See \secref{sec:detection} for further information.
\item
The filename \textit{main} must be specified without the |.tex| extension.
\item
The filename \textit{main} is case sensitive
(even in case-insensitive file systems)
due to internal string comparison.
\item
The argument \textit{main} should be fully expanded, it cannot be a macro.
\item
Subdirectories and special characters should be avoided in filenames.
\item
The command |\childdocmain{|\textit{main}|}| must be followed by a whitespace.
It should not be followed immediately by another command
or by a comment mark `|%|'.
This is because the \TeX{} parser reads the token immediately following
the argument of |\childdocmain| and puts it
at the beginning of every child section;
however, a white\-space is ignored.
\end{itemize}

%%%%%%%%%%%%%%%%%%%%%%%%%%%%%%%%%%%%%%%%
\paragraph{Content of Main File.}

It is advisable to place all content in the child files included by |\include|.
Any output contained in the main file will appear in all child documents
unless suppressed manually;
it cannot be suppressed automatically by the |\includeonly| directive
and thus should normally be avoided.
A method to include some content in the main file
by means of conditional processing is described in \secref{sec:conditional}.

%%%%%%%%%%%%%%%%%%%%%%%%%%%%%%%%%%%%%%%%
\paragraph{Page Numbering.}

When only a part of the document is compiled,
the appropriate numbering of pages
(as well as other status parameters)
is determined from the |.aux| files.
The latter contain information from previous passes.
However this information needs to propagate through
all intermediate child documents.
Therefore the page numbering in child documents may well
be inconsistent until the complete document is compiled at least once.

A useful (if unconventional) way to always ensure a consistent
page numbering is to restart the numbering in each child document
and denote the pages by `\textit{child}|.|\textit{page}'
where \textit{child} represents the chapter/section number of the child file.
This can be achieved by the command
|\numberwithin{page}{|\textit{child}|}|
of the \textsf{amsmath} package
where \textit{child} can be |chapter| or |section|
depending on the chosen structuring.
Alternatively, one can modify the macro |\thepage| appropriately
and reset the counter |page| at the start of each child file.

%%%%%%%%%%%%%%%%%%%%%%%%%%%%%%%%%%%%%%%%%%%%%%%%%%%%%%%%%%%%%%%%%%%%%%%%%%%%%%%%
\subsection{Conditional Processing}
\label{sec:conditional}

The package provides a mechanism to compile different versions
of a document. To customise the versions further some conditional processing
can come in handy to distinguish which version is being compiled.
The package provides two macros to describe the compilation context:

%%%%%%%%%%%%%%%%%%%%%%%%%%%%%%%%%%%%%%%%
\DescribeMacro{\ifchilddoc}
The conditional |\ifchilddoc| distinguishes between the compilation of
child documents and the main document:
%
\begin{center}
|\ifchilddoc |\textit{child-code}| |[|\||else |\textit{main-code}]| \||fi|
\end{center}

%%%%%%%%%%%%%%%%%%%%%%%%%%%%%%%%%%%%%%%%
\DescribeMacro{\childdocname}
\DescribeMacro{\childdocjob}
The macro |\childdocname| contains the filename (without extension)
of the main or child file being processed.
Note that |\childdocjob| will always contain the name of the main file.

%%%%%%%%%%%%%%%%%%%%%%%%%%%%%%%%%%%%%%%%
\paragraph{Title Page.}

Conditional processing can be used to include a title or banner page
in the main document when proper precautions are taken.
Importantly, the code in the main file should ensure that the page counter
(as well as other status parameters which are stored in the |.aux| files)
takes the same value after the conditional processing.
Otherwise the page numbers may take divergent values
depending on which part is compiled.

For example, a title page could be declared by:
%
\begin{center}
\begin{tabular}{l}
|\ifchilddoc\||else|\\
|\addtocounter{page}{-1}|\\
\textit{code for title page}\\
|\newpage|\\
|\||fi|
\end{tabular}
\end{center}
%
A banner page for the child documents can be generated by:
%
\begin{center}
\begin{tabular}{l}
|\ifchilddoc|\\
|\addtocounter{page}{-1}|\\
\textit{code for banner page}\\
|\newpage|\\
|\||fi|
\end{tabular}
\end{center}
%
Here one could write a message such as:
\begin{center}
|This is the part \childdocname{} of \childdocjob{}.|
\end{center}

%%%%%%%%%%%%%%%%%%%%%%%%%%%%%%%%%%%%%%%%%%%%%%%%%%%%%%%%%%%%%%%%%%%%%%%%%%%%%%%%
\subsection{Flags}
\label{sec:flags}

The package makes it easy to generate different versions
of the main or child documents.
To this end compilation flags can be defined
and assigned different default values.
They will be particularly useful in conjunction
with the forwarding mechanism described in \secref{sec:forward}.

For example, it may be useful to have a flag |\version|
which can be set to |draft| or |final|.
The document source will contain some conditional code
depending on the value of |\version|.
Suppose further, the flag should default to |final| for the main file
and to |draft| for child files
which is a natural assignment for editing the document.
This is achieved by placing the following code
in the preamble of the main document
(below the |\childdocmain| directive):
%
\begin{center}
\begin{tabular}{l}
|\ifchilddoc|\\
|\providecommand{\version}{draft}|\\
|\||else|\\
|\providecommand{\version}{final}|\\
|\||fi|
\end{tabular}
\end{center}
%
The definition by |\providecommand| makes sure
that previous definitions are not overwritten.
Further statements |\providecommand{\version}{...}|
can thus be added before the above code to override it.

For the main file, one might add a line
(between |\childdocmain| and the above block)
%
\begin{center}
|%\ifchilddoc\||else\providecommand{\version}{draft}\||fi|
\end{center}
%
which can be uncommented to produce a draft version.
Likewise one can add a line to the very top of a child file
(above the |\childdocof{|\textit{main}|}| directive)
%
\begin{center}
|%\providecommand{\version}{final}|
\end{center}
%
which can be uncommented to produce the final version of this child document.

%%%%%%%%%%%%%%%%%%%%%%%%%%%%%%%%%%%%%%%%%%%%%%%%%%%%%%%%%%%%%%%%%%%%%%%%%%%%%%%%
\subsection{Forwarding}
\label{sec:forward}

Different versions of the main or child documents
using compilation flags as described in \secref{sec:flags}
can be (permanently) stored in different files
for convenient compilation, viewing and distribution.
To this end, the package defines a command
to pass on compilation to a different file:

%%%%%%%%%%%%%%%%%%%%%%%%%%%%%%%%%%%%%%%%
\DescribeMacro{\childdocforward}
The command |\childdocforward| redirects processing to
another source file:
%
\begin{center}
\begin{tabular}{l}
|\input{childdoc.def}|\\
|\childdocforward[|\textit{main}|]{|\textit{dest}|}|\\
\end{tabular}
\end{center}
%
The argument \textit{dest} is the destination file
(without extension).
It should be the main file or one of the child files.
Note that further \textsf{childdoc} directives
such as |\childdocof| and |\childdocforward|
in the indicated file will be processed in this form.
The optional argument \textit{main}
passes on directly to the main file \textit{main}
while pretending to compile the child \textit{dest}.
This form behaves as if \textit{dest}
issues |\childdocof{|\textit{main}|}| right away,
and no further \textsf{childdoc} directives will be processed.

%%%%%%%%%%%%%%%%%%%%%%%%%%%%%%%%%%%%%%%%
\DescribeMacro{\...prefix}
In the alternative form |\childdocforwardprefix|,
%
\begin{center}
\begin{tabular}{l}
|\input{childdoc.def}|\\
|\childdocforwardprefix[|\textit{main}|]{|\textit{prefix}|}{|\textit{dest}|}|
\end{tabular}
\end{center}
%
the destination file is determined by a pattern
depending on the current file:
To make this work, the current file must be called
`{\textit{prefix}\hspace{0.2em}\textit{suffix}}'
with \textit{prefix} matching precisely the argument.
Processing is then passed on to the file
`{\textit{dest}\hspace{0.2em}\textit{suffix}}'.
Surely, the same effect is achieved by
directly specifying the
argument `{\textit{dest}\hspace{0.2em}\textit{suffix}}'
in the first form.
However, that requires to set up a different file
for each child. With the alternative form of the command
all these files can have exactly the same content
which simplifies setting them up and maintaining them.

For example, the following file |draft.tex|
with a compilation flag |\version| as described in \secref{sec:flags}
compiles the main document as a draft:
%
\begin{center}
\begin{tabular}{l}
|\def\version{draft}|\\
|\input{childdoc.def}|\\
|\childdocforward{|\textit{main}|}|
\end{tabular}
\end{center}
%
Likewise, the following files |final|\textit{nn}|.tex|
compile the final version of the child document
|child|\textit{nn}|.tex|:
%
\begin{center}
\begin{tabular}{l}
|\def\version{final}|\\
|\input{childdoc.def}|\\
|\childdocforwardprefix{final}{child}|
\end{tabular}
\end{center}
%

Note that when several versions of a main file and/or of each child file
are to be generated, it may be convenient to set up a |Makefile| or
shell script to automatise the process.

%%%%%%%%%%%%%%%%%%%%%%%%%%%%%%%%%%%%%%%%%%%%%%%%%%%%%%%%%%%%%%%%%%%%%%%%%%%%%%%%
\subsection{Command Line Processing}
\label{sec:commandline}

The effect of redirection files can also be achieved by invoking
the \LaTeX{} compiler with a more elaborate command line.
Most conveniently this should be done as part
of a shell script or a |Makefile|.

When using \textsf{childdoc} in the main file, the following
command lines effectively perform a redirection
(note that depending on the shell being used,
backslashes may have to be doubled: `|\|' $\to$ `|\\|'):
%
\begin{center}
|... -jobname "|\textit{target}|" |\\|"|[\textit{flags}]%
|\input{childdoc.def}\childdocforward[|\textit{main}|]{|\textit{dest}|}"|
\end{center}
%
Here \textit{target} is the name of the output file,
\textit{main} is the name of the main file
and \textit{dest} is the name of the main or child file to be processed
(all filenames without extensions).
The optional argument \textit{main} can be omitted
if \textit{main} matches \textit{dest}.
Optionally, compilation \textit{flags} can be defined via |\def| commands.
This command line makes the \TeX{} engine believe
it is compiling the file \textit{target}
whose content is specified as the latter parameter.
The provided code then forwards the processing to
\textit{main} or \textit{dest} as described in \secref{sec:forward}.

%%%%%%%%%%%%%%%%%%%%%%%%%%%%%%%%%%%%%%%%%%%%%%%%%%%%%%%%%%%%%%%%%%%%%%%%%%%%%%%%
\subsection{Include by Input}
\label{sec:input}

Including child documents by |\include| has some restrictions by design.
Most notably, the content of a child document always occupies
its own set of pages; pages cannot be shared between child documents.
Usually, this behaviour makes perfect sense
because each child document contain an essential part of the document.
However, in some situations it may be desirable to compose
a document from a collection of parts
without having mandatory page breaks between then.
For this case, the package
provides a mechanism to include parts
by |\input| which can also be processed individually.
However, by construction this mechanism
requires manual handling of the content to be output.

%%%%%%%%%%%%%%%%%%%%%%%%%%%%%%%%%%%%%%%%
\DescribeMacro{\ifchilddocmanual}
The main file should be prepared as usual, see \secref{sec:include}.
However, the document body must make a distinction
between processing of an individual part and of the main document, e.g.:
%
\begin{center}
\begin{tabular}{l}
|\ifchilddocmanual|\\
|\input{\childdocname}|\\
|\||else|\\
\textit{document body with }|\input{|\textit{part}|}|\\
|\||fi|
\end{tabular}
\end{center}
%
The conditional |\ifchilddocmanual| is true whenever
a part to be included by |\input| is being compiled,
and the name of the part is stored in |\childdocname|.

%%%%%%%%%%%%%%%%%%%%%%%%%%%%%%%%%%%%%%%%
\DescribeMacro{\childdocby}
Each part to be included by |\input| should start with:
%
\begin{center}
\begin{tabular}{l}
|\input{childdoc.def}|\\
|\childdocby{|\textit{main}|}|\\
\end{tabular}
\end{center}
%
The directive |\childdocby| is similar to |\childdocof|
described in \secref{sec:include},
but the subsequent selection of content must be done manually.
To that end, both |\ifchilddoc| and |\ifchilddocmanual|
will be true upon processing of a part,
and the name of the part is stored in |\childdocname|.
Note that |\jobname| will be set to the filename of the current part
so that each part receives an individual |.aux| file
that does not interfere with the |.aux| file(s) of the main document.
This behaviour can be altered by the alternative form
|\childdocby[*]{|\textit{main}|}| (with a non-empty optional argument)
which uses the |.aux| file of the main document
by setting |\jobname| to \textit{main}.

%%%%%%%%%%%%%%%%%%%%%%%%%%%%%%%%%%%%%%%%%%%%%%%%%%%%%%%%%%%%%%%%%%%%%%%%%%%%%%%%
\subsection{Driver Development}
\label{sec:driver}

The \textsf{childdoc} mechanism can also be use for the development
of definition files such as \LaTeX{} styles or classes.
This case differs from the above setup with multiple parts
included by |\include| in that no |\includeonly| should be invoked.
This can be achieved by starting the include file
(before |\ProvidesPackage|) with:
%
\begin{center}
\begin{tabular}{l}
|\input{childdoc.def}|\\
|\childdocforward{|\textit{main}|}|\\
\end{tabular}
\end{center}
%
or alternatively with:
%
\begin{center}
\begin{tabular}{l}
|\input{childdoc.def}|\\
|\childdocby{|\textit{main}|}|\\
\end{tabular}
\end{center}
%
Both forms have slightly different effects as described above.
The main file is prepared as usual, see \secref{sec:include}.

%%%%%%%%%%%%%%%%%%%%%%%%%%%%%%%%%%%%%%%%%%%%%%%%%%%%%%%%%%%%%%%%%%%%%%%%%%%%%%%%
\subsection{Legacy Detection}
\label{sec:detection}

The directive |\childdocmain| in the main file can detect
whether the complete document or merely a child is to be compiled
even without using the directive |\childdocof|.
This method is deprecated because it is less robust
and there is no compelling reason to use it;
it is merely provided for backward compatibility
and it may be removed in future versions.

If the detection mechanism is to be used,
it is mandatory to correctly specify
the filename of the main file as the argument of |\childdocmain|:
%
\begin{center}
\begin{tabular}{l}
|\input{childdoc.def}|\\
|\childdocmain{|\textit{main}|}|\\
\end{tabular}
\end{center}
%
If |\jobname| does not match the argument \textit{main} of |\childdocmain|,
it is assumed that |\jobname| points to the child file to be compiled.
When using |\childdocmain| with the main file specified as argument,
it suffices to start a child file
with just |\input{|\textit{main}|}|
without loading of the package and using |\childdocof|.
If instead all processing is done
with the appropriate \textsf{childdoc} directives,
the argument of \textit{main} of |\childdocmain| can be empty.

An alternative version of the command line processing described
in \secref{sec:commandline} using the detection mechanism reads:
%
\begin{center}
|... -jobname "|\textit{target}|" "|[\textit{flags}]%
[|\def\jobname{|\textit{dest}|}|]|\input{|\textit{main}|}"|
\end{center}

%%%%%%%%%%%%%%%%%%%%%%%%%%%%%%%%%%%%%%%%%%%%%%%%%%%%%%%%%%%%%%%%%%%%%%%%%%%%%%%%
\subsection{Manual Code}
\label{sec:manual}

In case one cannot be certain whether the definitions file |childdoc.def|
is installed on the target \TeX{} distribution
and one prefers not to ship it,
it is conceivable to paste a few relevant commands into the sources.

To that end, drop all statements |\input{childdoc.def}|
and perform the replacements as outlined below.
Instead of |\childdocmain{|\textit{main}|}| add the following code
to the top of the main file:
%
\begin{center}
\begin{tabular}{l}
|\||ifdefined\childdocname\endinput\||fi\newif\ifchilddoc|\\
|\edef\childdocname{\scantokens\expandafter{\jobname\noexpand}}|\\
|\def\childdocmain{|\textit{main}|}\||ifx\childdocmain\childdocname\||else|\\
|\childdoctrue\includeonly{\childdocname}\let\jobname\childdocmain\||fi|\\
\end{tabular}
\end{center}
%
Instead of |\childdocof{|\textit{main}|}| just include the main file
at the top of each child file:
%
\begin{center}
|\input{|\textit{main}|}|
\end{center}
%
A simple redirection |\childdocforward{|\textit{dest}|}| is achieved by:
%
\begin{center}
|\def\jobname{|\textit{dest}|}\input{\jobname}|
\end{center}
%
The redirection with prefix
|\childdocforwardprefix[|\textit{prefix}|]{|\textit{dest}|}|
is accomplished by:
%
\begin{center}
\begin{tabular}{l}
|{\edef\jobname{\scantokens\expandafter{\jobname\noexpand}}|\\
|\def\redirectjob |\textit{prefix}|#1~~~{\gdef\jobname{|\textit{dest}|#1}}|\\
|\expandafter\redirectjob\jobname~~~}\input{\jobname}|
\end{tabular}
\end{center}

In an alternative approach,
child documents can be compiled by a specific command line
without additional code or specific definitions:
%
\begin{center}
|... -jobname "|\textit{target}|" "|[\textit{flags}]%
|\includeonly{|\textit{dest}|}\input{|\textit{main}|}"|
\end{center}
%

%%%%%%%%%%%%%%%%%%%%%%%%%%%%%%%%%%%%%%%%%%%%%%%%%%%%%%%%%%%%%%%%%%%%%%%%%%%%%%%%
%%%%%%%%%%%%%%%%%%%%%%%%%%%%%%%%%%%%%%%%%%%%%%%%%%%%%%%%%%%%%%%%%%%%%%%%%%%%%%%%
\section{Information}

%%%%%%%%%%%%%%%%%%%%%%%%%%%%%%%%%%%%%%%%%%%%%%%%%%%%%%%%%%%%%%%%%%%%%%%%%%%%%%%%
\subsection{Copyright}

Copyright \copyright{} 2017--2018 Niklas Beisert

This work may be distributed and/or modified under the
conditions of the \LaTeX{} Project Public License, either version 1.3
of this license or (at your option) any later version.
The latest version of this license is in
  \url{http://www.latex-project.org/lppl.txt}
and version 1.3 or later is part of all distributions of \LaTeX{}
version 2005/12/01 or later.

This work has the LPPL maintenance status `maintained'.

The Current Maintainer of this work is Niklas Beisert.

This work consists of the files |README.txt|, |childdoc.ins| and |childdoc.dtx|
as well as the derived files |childdoc.def|, |cdocsamp.tex|
with |cdocsch1.tex|, |cdocsch2.tex|, |cdocspt3.tex|, |cdocspt4.tex|,
|cdocsdrf.tex|, |cdocsfn1.tex|, |cdocsfn2.tex|
as well as |childdoc.pdf|.

%%%%%%%%%%%%%%%%%%%%%%%%%%%%%%%%%%%%%%%%%%%%%%%%%%%%%%%%%%%%%%%%%%%%%%%%%%%%%%%%
\subsection{Files and Installation}

The package consists of the files:
%
\begin{center}
\begin{tabular}{ll}
    |README.txt|   & readme file \\
    |childdoc.ins| & installation file \\
    |childdoc.dtx| & source file \\
    |childdoc.def| & definition file \\
    |cdocsamp.tex| & sample main file \\
    |cdocsch1.tex| & sample include file \\
    |cdocsch2.tex| & sample include file \\
    |cdocspt3.tex| & sample part file \\
    |cdocspt4.tex| & sample part file \\
    |cdocsdrf.tex| & sample redirection file \\
    |cdocsfn1.tex| & sample redirection file \\
    |cdocsfn2.tex| & sample redirection file \\
    |childdoc.pdf| & manual
\end{tabular}
\end{center}
%
The distribution consists of the files
|README.txt|, |childdoc.ins| and |childdoc.dtx|.
%
\begin{itemize}
\item
Run (pdf)\LaTeX{} on |childdoc.dtx|
to compile the manual |childdoc.pdf| (this file).
\item
Run \LaTeX{} on |childdoc.ins| to create the definitions file |childdoc.def|
and the sample |cdocsamp.tex| with include files
|cdocsch1.tex|, |cdocsch2.tex|, |cdocspt3.tex|, |cdocspt4.tex|,
|cdocsdrf.tex|, |cdocsfn1.tex|, |cdocsfn2.tex|.
Then copy the file |childdoc.def| to an appropriate directory of your \LaTeX{}
distribution, e.g.\ \textit{texmf-root}|/tex/latex/childdoc|.
\end{itemize}

%%%%%%%%%%%%%%%%%%%%%%%%%%%%%%%%%%%%%%%%%%%%%%%%%%%%%%%%%%%%%%%%%%%%%%%%%%%%%%%%
\subsection{Related CTAN Packages}

There are several other packages which offer a similar functionality:
%
\begin{itemize}
\item
The packages
\href{http://ctan.org/pkg/docmute}{\textsf{docmute}},
\href{http://ctan.org/pkg/includex}{\textsf{includex}} and
\href{http://ctan.org/pkg/standalone}{\textsf{standalone}}
provide commands to include only the document body of
a child file thus allowing both files to be compiled individually.
\item
The packages \href{http://ctan.org/pkg/subdocs}{\textsf{subdocs}}
and \href{http://ctan.org/pkg/subfiles}{\textsf{subfiles}}
provide structures in which the main and child documents can be
encapsulated and allowing them to be compiled individually.
The inclusion mechanism is different from the conventional |\include|.
\item
The package \href{http://ctan.org/pkg/combine}{\textsf{combine}}
is an elaborate solution to combine several documents into one.
\end{itemize}
%
See also the CTAN topic \href{http://ctan.org/topic/subdocs}{\textsf{subdocs}}
for further related packages.
The present package differs from the above solutions in that
a document structure constructed with the conventional |\include| mechanism
just needs two extra commands at the top of every file
such that all constituent files can be compiled individually.

%%%%%%%%%%%%%%%%%%%%%%%%%%%%%%%%%%%%%%%%%%%%%%%%%%%%%%%%%%%%%%%%%%%%%%%%%%%%%%%%
%\subsection{Feature Suggestions}
%
%The following is a list of features which may be useful for future
%versions of this package:
%%
%\begin{itemize}
%\item
%\ldots
%\end{itemize}

%%%%%%%%%%%%%%%%%%%%%%%%%%%%%%%%%%%%%%%%%%%%%%%%%%%%%%%%%%%%%%%%%%%%%%%%%%%%%%%%
\subsection{Revision History}

%%%%%%%%%%%%%%%%%%%%%%%%%%%%%%%%%%%%%%%%
\paragraph{v2.0:} 2018/12/30

\begin{itemize}
\item
immediate forward processing
\item
added |\childdocby| mechanism
\item
manual restructured
\end{itemize}

%%%%%%%%%%%%%%%%%%%%%%%%%%%%%%%%%%%%%%%%
\paragraph{v1.6:} 2018/01/17

\begin{itemize}
\item
application for development of include files
\item
corrections to manual
\end{itemize}

%%%%%%%%%%%%%%%%%%%%%%%%%%%%%%%%%%%%%%%%
\paragraph{v1.5:} 2017/05/21

\begin{itemize}
\item
more complete structuring introduced
\item
|\childdocof| introduced
\item
|\childdoc| renamed to |\childdocmain|
\item
|\childredirect| renamed to |\childdocforward| and |\childdocforwardprefix|
and functionality expanded
\end{itemize}

%%%%%%%%%%%%%%%%%%%%%%%%%%%%%%%%%%%%%%%%
\paragraph{v1.0:} 2017/04/27

\begin{itemize}
\item
manual and install package
\item
first version published on CTAN
\end{itemize}

%%%%%%%%%%%%%%%%%%%%%%%%%%%%%%%%%%%%%%%%
\paragraph{v0.6:} 2017/04/26

\begin{itemize}
\item
redirection mechanism added
\end{itemize}

%%%%%%%%%%%%%%%%%%%%%%%%%%%%%%%%%%%%%%%%
\paragraph{v0.5:} 2017/04/26

\begin{itemize}
\item
functionality in definition file
\end{itemize}


%%%%%%%%%%%%%%%%%%%%%%%%%%%%%%%%%%%%%%%%%%%%%%%%%%%%%%%%%%%%%%%%%%%%%%%%%%%%%%%%
%%%%%%%%%%%%%%%%%%%%%%%%%%%%%%%%%%%%%%%%%%%%%%%%%%%%%%%%%%%%%%%%%%%%%%%%%%%%%%%%
%%%%%%%%%%%%%%%%%%%%%%%%%%%%%%%%%%%%%%%%%%%%%%%%%%%%%%%%%%%%%%%%%%%%%%%%%%%%%%%%
\appendix

\settowidth\MacroIndent{\rmfamily\scriptsize 000\ }

 \DocInput{childdoc.dtx}

\end{document}
%</driver>
% \fi
%
% %%%%%%%%%%%%%%%%%%%%%%%%%%%%%%%%%%%%%%%%%%%%%%%%%%%%%%%%%%%%%%%%%%%%%%%%%%%%%%
% %%%%%%%%%%%%%%%%%%%%%%%%%%%%%%%%%%%%%%%%%%%%%%%%%%%%%%%%%%%%%%%%%%%%%%%%%%%%%%
% \section{Sample}
%\iffalse
%<*samplemain>
%\fi
%
% The following presents a sample document
% with two chapters, two parts, a title page,
% a compile flag as well as three forwarding files to set the flag.
% It consists of eight |.tex| files:
% \begin{center}
% \begin{tabular}{ll}
% |cdocsamp.tex|&main file\\
% |cdocsch1.tex|&include file for chapter 1\\
% |cdocsch2.tex|&include file for chapter 2\\
% |cdocspt3.tex|&include file for part 3\\
% |cdocspt4.tex|&include file for part 4\\
% |cdocsdrf.tex|&forwarding file for main file in draft mode\\
% |cdocsfi1.tex|&forwarding file for final version of chapter 1\\
% |cdocsfi2.tex|&forwarding file for final version of chapter 2\\
% \end{tabular}
% \end{center}
% Each of the eight files can be compiled directly by the \LaTeX{} compiler.
%
% %%%%%%%%%%%%%%%%%%%%%%%%%%%%%%%%%%%%%%
% \paragraph{Main File.}
%
% The main file is called |cdocsamp.tex|.
%
% Load the \textsf{childdoc} definitions and
% declare the filename for the main document:
%    \begin{macrocode}
\input{childdoc.def}
\childdocmain{}
%    \end{macrocode}

% Optional override for |\version| flag:
%    \begin{macrocode}
%%\ifchilddoc\else\providecommand{\version}{draft}\fi
%    \end{macrocode}

% Define the default values for the |\version| flag
% (|final| for the main file and |draft| for childs):
%    \begin{macrocode}
\ifchilddoc
\providecommand{\version}{draft}
\else
\providecommand{\version}{final}
\fi
%    \end{macrocode}

% Load the standard document class:
%    \begin{macrocode}
\documentclass[12pt]{article}
%    \end{macrocode}

% Start the document body:
%    \begin{macrocode}
\begin{document}
%    \end{macrocode}

% Declare a title page.
% Print title, part of document being processed and version flag:
%    \begin{macrocode}
\addtocounter{page}{-1}
\begin{center}
{\LARGE\bfseries{}childdoc example\par}
\vspace{1cm}
\ifchilddoc
\ifchilddocmanual part\else chapter\fi:
`\childdocname' of `\childdocjob'\par
\else
main document: `\childdocjob'\par
\fi
version: \version\par
\end{center}
\newpage
%    \end{macrocode}

% Manually include selected file,
% otherwise process as usual:
%    \begin{macrocode}
\ifchilddocmanual
\section*{part `\childdocname'}
\input{\childdocname}
\else
%    \end{macrocode}

% Include the two chapters:
%    \begin{macrocode}
\include{cdocsch1}
\include{cdocsch2}
%    \end{macrocode}

% Include the two parts unless only chapters should be displayed:
%    \begin{macrocode}
\ifchilddoc\else
\section{part three}
\input{cdocspt3}
\section{part four}
\input{cdocspt4}
\fi
%    \end{macrocode}

% Process as usual until here:
%    \begin{macrocode}
\fi
%    \end{macrocode}

% End of document body:
%    \begin{macrocode}
\end{document}
%    \end{macrocode}
%\iffalse
%</samplemain>
%\fi
%
% %%%%%%%%%%%%%%%%%%%%%%%%%%%%%%%%%%%%%%
% \paragraph{Chapter Include Files.}
%
% The include files are called |cdocsch1.tex| and |cdocsch2.tex|.
%
%\iffalse
%<*samplechap1|samplechap2>
%\fi

% Optional override for |\version| flag:
%    \begin{macrocode}
%%\providecommand{\version}{final}
%    \end{macrocode}

% Include the main document:
%    \begin{macrocode}
\input{childdoc.def}
\childdocof{cdocsamp}
%    \end{macrocode}

%\iffalse
%</samplechap1|samplechap2>
%\fi
%
%\iffalse
%<*samplechap1>
%\fi
% Some text for chapter 1:
%    \begin{macrocode}
\section{one}
some text in chapter one
%    \end{macrocode}

%\iffalse
%</samplechap1>
%\fi
% Some text for chapter 2:
%\iffalse
%<*samplechap2>
%\fi
%    \begin{macrocode}
\section{two}
more text in chapter two
%    \end{macrocode}

%\iffalse
%</samplechap2>
%\fi
%
% %%%%%%%%%%%%%%%%%%%%%%%%%%%%%%%%%%%%%%
% \paragraph{Part Include Files.}
%
% The include files are called |cdocspt3.tex| and |cdocspt4.tex|.
%
%\iffalse
%<*samplepart3|samplepart4>
%\fi

% Optional override for |\version| flag:
%    \begin{macrocode}
%%\providecommand{\version}{final}
%    \end{macrocode}

% Include the main document:
%    \begin{macrocode}
\input{childdoc.def}
\childdocby{cdocsamp}
%    \end{macrocode}

%\iffalse
%</samplepart3|samplepart4>
%\fi
%
%\iffalse
%<*samplepart3>
%\fi
% Some text for part 3:
%    \begin{macrocode}
some text in part three
%    \end{macrocode}

%\iffalse
%</samplepart3>
%\fi
% Some text for part 4:
%\iffalse
%<*samplepart4>
%\fi
%    \begin{macrocode}
more text in part four
%    \end{macrocode}

%\iffalse
%</samplepart4>
%\fi
%
% %%%%%%%%%%%%%%%%%%%%%%%%%%%%%%%%%%%%%%
% \paragraph{Forwarding for a Complete Draft.}
%
% The following forwarding file |cdocsdrf.tex|
% compiles the main document in draft mode:
%\iffalse
%<*sampledraft>
%\fi
%    \begin{macrocode}
\def\version{draft}
\input{childdoc.def}
\childdocforward{cdocsamp}
%    \end{macrocode}

%\iffalse
%</sampledraft>
%\fi
%
% %%%%%%%%%%%%%%%%%%%%%%%%%%%%%%%%%%%%%%
% \paragraph{Forwarding for Final Version of the Chapters.}
%
% The following forwarding files |cdocsfn1.tex| and |cdocsfn2.tex|
% (with identical content)
% compile the final versions of the child documents
% |cdocsch1.tex| and |cdocsch2.tex|, respectively:
%\iffalse
%<*samplefinal>
%\fi
%    \begin{macrocode}
\def\version{final}
\input{childdoc.def}
\childdocforwardprefix[cdocsamp]{cdocsfn}{cdocsch}
%    \end{macrocode}

%\iffalse
%</samplefinal>
%\fi
%
% %%%%%%%%%%%%%%%%%%%%%%%%%%%%%%%%%%%%%%
% \paragraph{Command Line Processing.}
%
% The following three command lines generate the output files
% |cdocscld|, |cdocscl1| and |cdocscl2|
% which should be identical to
% |cdocsdrf|, |cdocsch1| and |cdocsfn2|, respectively:
% \begin{center}
% \begin{tabular}{l}
% |latex -jobname cdocscld \|\\
% |  "\def\version{draft}\input{childdoc.def}\childdocforward{cdocsamp}"|\\
% |latex -jobname cdocscl1 \|\\
% |  "\input{childdoc.def}\childdocforward[cdocsamp]{cdocsch1}"|\\
% |latex -jobname cdocscl2 \|\\
% |  "\def\version{final}\input{childdoc.def}\childdocforward{cdocsch2}"|
% \end{tabular}
% \end{center}
% Note that the trailing backslash on each first line
% merely continues the input to the second line
% (for convenient cut ant paste).
% Furthermore, the command |latex| can be replaced by any
% of its alternative versions such as |pdflatex|.
%
% %%%%%%%%%%%%%%%%%%%%%%%%%%%%%%%%%%%%%%%%%%%%%%%%%%%%%%%%%%%%%%%%%%%%%%%%%%%%%%
% %%%%%%%%%%%%%%%%%%%%%%%%%%%%%%%%%%%%%%%%%%%%%%%%%%%%%%%%%%%%%%%%%%%%%%%%%%%%%%
% \section{Implementation}
%\iffalse
%<*package>
%\fi
%
% This section describes the definitions file |childdoc.def|.

% The definitions cannot be loaded using |\usepackage| or |\RequirePackage|
% which has a mechanism to prevent loading a style file more than once.
% When loading the definitions by means of |\input|
% multiple instances have to be prevented manually:
%\iffalse
%This code needs to be before the `\ProvidesFile' directive
%which is defined at the beginning of this file.
%Therefore it is also placed there and commented out here.
%</package>
%<*discard>
%\fi
%    \begin{macrocode}
\ifdefined\childdocmain\endinput\fi
%    \end{macrocode}
%\iffalse
%</discard>
%<*package>
%\fi
%
% \macro{\ifchilddoc}
% \macro{\ifchilddocmanual}
% The conditional |\ifchilddoc| tells whether a
% child (true) or main (false) document is being compiled.
% The conditional |\ifchilddocmanual| tells whether
% the |\includeonly| mechanism is used (false) or
% the selection of child files must be performed manually (true).
% The definitions initialise to false:
%    \begin{macrocode}
\newif\ifchilddoc
\newif\ifchilddocmanual
%    \end{macrocode}

% \macro{\childdocname}
% \macro{\childdocjob}
% The macro |\childdocname| stores the name of the main document
% to be compiled. The macro |\childdocjob| stores the name of
% the document on which the \LaTeX{} compiler was originally invoked.
% The content of |\jobname| cannot be compared
% to filenames specified in the source due to different catcodes.
% The following code rescans |\jobname|, stores the result
% in |\childdocname| and saves a copy in |\childdocjob|:
%    \begin{macrocode}
\edef\childdocname{\scantokens\expandafter{\jobname\noexpand}}
\let\childdocjob\childdocname
%    \end{macrocode}

% \macro{\childdocdisable}
% The macro |\childdocdisable| prevents the main file
% from being processed more than once.
% At this stage, the main document command |\childdocmain|
% is assumed to be called once again where it should do nothing.
% Any subsequent call to it should prevent
% a secondary processing of the main document
% It overwrites the forwarding commands
% |\childdocof| and |\childdocforward|
% with empty macros to prevent further inclusions of the main document:
%    \begin{macrocode}
\newcommand{\childdocdisable}
{
  \renewcommand{\childdocmain}[1]{\renewcommand{\childdocmain}[1]{\endinput}}
  \renewcommand{\childdocof}[1]{}
  \renewcommand{\childdocby}[2][]{}
  \renewcommand{\childdocforward}[2][]{}
  \renewcommand{\childdocdisable}{}
}
%    \end{macrocode}

% \macro{\childdocmain}
% The macro |\childdocmain| is to be called at the top of the main file
% with nothing or the main filename (without extension) as argument.
% First, it breaks loops.
% If the argument is not empty and does not match |\childdocname|
% (which is set by the first inclusion of |childdoc.def|),
% |\ifchilddoc| is set to true, |\includeonly| is applied to the child file
% and |\jobname| is set to the main file
% (for proper handling of |.aux| files):
%    \begin{macrocode}
\newcommand{\childdocmain}[1]
{
  \childdocdisable\childdocmain{}
  \if?#1?\else
    \begingroup
      \def\childdoctmp{#1}
      \ifx\childdoctmp\childdocname
        \def\childdoctmp{}
      \else
        \def\childdoctmp
        {
          \childdoctrue
          \includeonly{\childdocname}
          \def\childdocjob{#1}
          \def\jobname{#1}
        }
      \fi
      \expandafter
    \endgroup
    \childdoctmp
  \fi
}
%    \end{macrocode}

% \macro{\childdocof}
% The command |\childdocof| redirects
% compilation to the main file |#1|.
%    \begin{macrocode}
\newcommand{\childdocof}[1]
{
  \childdocdisable
  \childdoctrue
  \includeonly{\childdocname}
  \def\jobname{#1}
  \def\childdocjob{#1}
  \input{#1}
}
%    \end{macrocode}

% \macro{\childdocby}
% The command |\childdocby| ....
%    \begin{macrocode}
\newcommand{\childdocby}[2][]
{
  \childdocdisable
  \childdoctrue
  \childdocmanualtrue
  \if?#1?\else
    \def\jobname{#2}
  \fi
  \def\childdocjob{#2}
  \input{#2}
  \endinput
}
%    \end{macrocode}

% \macro{\childdocforward}
% The command |\childdocforward| redirects
% compilation to the main file or
% (if the optional argument is given) a child file.
% Parameters are set as if the main file
% or a child file starting with |\childdocof| was compiled.
% Then compilation is handed over to the main file:
%    \begin{macrocode}
\newcommand{\childdocforward}[2][]
{
  \begingroup
    \if?#1?
      \def\childdoctmp
      {
        \def\childdocname{#2}
        \def\childdocjob{#2}
        \def\jobname{#2}
        \input{#2}
        \endinput
      }
    \else
      \def\childdoctmp
      {
        \childdocdisable
        \def\childdocname{#2}
        \childdoctrue
        \includeonly{#2}
        \def\childdocjob{#1}
        \def\jobname{#1}
        \input{#1}
        \endinput
      }
    \fi
    \expandafter
  \endgroup
  \childdoctmp
}
%    \end{macrocode}

% \macro{\childdocforwardprefix}
% The command |\childdocforwardprefix| redirects
% compilation to the main or a child file by means of a pattern.
% The prefix |#1| in the current filename is replaced by |#2|
% and the suffix of the current filename is kept
% (it is assumed that the filename does not contain the substring `|~~~|'
% which is used as a delimiter).
% Compilation is handed over to the new file by |\childdocforward|:
%    \begin{macrocode}
\newcommand{\childdocforwardprefix}[3][]
{
  \begingroup
    \def\childdocextract #2##1~~~{\def\childdoctmp{\childdocforward[#1]{#3##1}}}
    \expandafter\childdocextract\childdocname~~~
    \expandafter
  \endgroup
  \childdoctmp
}
%    \end{macrocode}

% \macro{\childdoc}
% The deprecated macro |\childdoc| is a legacy version of |\childdocmain|:
%    \begin{macrocode}
\newcommand{\childdoc}{\childdocmain}
%    \end{macrocode}

% \macro{\childdocredirect}
% The deprecated macro |\childdocredirect| is a legacy version
% of |\childdocforward| and |\childdocforwardprefix|:
%    \begin{macrocode}
\newcommand{\childdocredirect}[2][]
{
  \begingroup
    \if?#1?
      \def\childdoctmp{\childdocforward{#2}}
    \else
      \def\childdoctmp{\childdocforwardprefix{#1}{#2}}
    \fi
    \expandafter
  \endgroup
  \childdoctmp
}
%    \end{macrocode}

%\iffalse
%</package>
%\fi
%
\endinput
|\\
|\childdocforward[|\textit{main}|]{|\textit{dest}|}|\\
\end{tabular}
\end{center}
%
The argument \textit{dest} is the destination file
(without extension).
It should be the main file or one of the child files.
Note that further \textsf{childdoc} directives
such as |\childdocof| and |\childdocforward|
in the indicated file will be processed in this form.
The optional argument \textit{main}
passes on directly to the main file \textit{main}
while pretending to compile the child \textit{dest}.
This form behaves as if \textit{dest}
issues |\childdocof{|\textit{main}|}| right away,
and no further \textsf{childdoc} directives will be processed.

%%%%%%%%%%%%%%%%%%%%%%%%%%%%%%%%%%%%%%%%
\DescribeMacro{\...prefix}
In the alternative form |\childdocforwardprefix|,
%
\begin{center}
\begin{tabular}{l}
|% \iffalse
%
% childdoc.dtx Copyright (C) 2017-2018 Niklas Beisert
%
% This work may be distributed and/or modified under the
% conditions of the LaTeX Project Public License, either version 1.3
% of this license or (at your option) any later version.
% The latest version of this license is in
%   http://www.latex-project.org/lppl.txt
% and version 1.3 or later is part of all distributions of LaTeX
% version 2005/12/01 or later.
%
% This work has the LPPL maintenance status `maintained'.
%
% The Current Maintainer of this work is Niklas Beisert.
%
% This work consists of the files childdoc.dtx and childdoc.ins
% and the derived files childdoc.def and cdocsamp.tex with
% cdocsch1.tex, cdocsch2.tex, cdocsdrf.tex, cdocsfn1.tex, cdocsfn2.tex.
%
%<package>\ifdefined\childdocmain\endinput\fi
%<package>\ProvidesFile{childdoc.def}[2018/12/30 v2.0 child document driver]
%<samplemain>\ProvidesFile{cdocsamp.tex}[2018/12/30 v2.0 sample for childdoc]
%<*driver>
%\ProvidesFile{childdoc.drv}[2018/12/30 v2.0 childdoc reference manual file]
\PassOptionsToClass{10pt,a4paper}{article}
\documentclass{ltxdoc}

\usepackage[margin=35mm]{geometry}
\usepackage{hyperref}
\usepackage{hyperxmp}
\usepackage[usenames]{color}

\hypersetup{colorlinks=true}
\hypersetup{pdfstartview=FitH}
\hypersetup{pdfpagemode=UseNone}
\hypersetup{pdfsource={}}
\hypersetup{pdflang={en-UK}}
\hypersetup{pdfcopyright={Copyright 2017-2018 Niklas Beisert.
  This work may be distributed and/or modified under the
  conditions of the LaTeX Project Public License, either version 1.3
  of this license or (at your option) any later version.}}
\hypersetup{pdflicenseurl={http://www.latex-project.org/lppl.txt}}
\hypersetup{pdfcontactaddress={ETH Zurich, ITP, HIT K,
  Wolfgang-Pauli-Strasse 27}}
\hypersetup{pdfcontactpostcode={8093}}
\hypersetup{pdfcontactcity={Zurich}}
\hypersetup{pdfcontactcountry={Switzerland}}
\hypersetup{pdfcontactemail={nbeisert@itp.phys.ethz.ch}}
\hypersetup{pdfcontacturl={http://people.phys.ethz.ch/\xmptilde nbeisert/}}

\newcommand{\secref}[1]{\hyperref[#1]{section \ref*{#1}}}

\parskip1ex
\parindent0pt
\let\olditemize\itemize
\def\itemize{\olditemize\parskip0pt}

\begin{document}

\title{The \textsf{childdoc} Package}
\hypersetup{pdftitle={The childdoc Package}}
\author{Niklas Beisert\\[2ex]
  Institut f\"ur Theoretische Physik\\
  Eidgen\"ossische Technische Hochschule Z\"urich\\
  Wolfgang-Pauli-Strasse 27, 8093 Z\"urich, Switzerland\\[1ex]
  \href{mailto:nbeisert@itp.phys.ethz.ch}
  {\texttt{nbeisert@itp.phys.ethz.ch}}}
\hypersetup{pdfauthor={Niklas Beisert}}
\hypersetup{pdfsubject={Manual for the LaTeX2e Package childdoc}}
\date{30 December 2018, \textsf{v2.0}}
\maketitle

\begin{abstract}\noindent
\textsf{childdoc} is a \LaTeXe{} package
that enables the direct compilation
of document sections included by |\include|
to individual files.
\end{abstract}

\begingroup
\parskip0ex
\tableofcontents
\endgroup

%%%%%%%%%%%%%%%%%%%%%%%%%%%%%%%%%%%%%%%%%%%%%%%%%%%%%%%%%%%%%%%%%%%%%%%%%%%%%%%%
%%%%%%%%%%%%%%%%%%%%%%%%%%%%%%%%%%%%%%%%%%%%%%%%%%%%%%%%%%%%%%%%%%%%%%%%%%%%%%%%
\section{Introduction}

\LaTeX{} provides a mechanism to structure a large document (such as a book)
into a main file and several child files (containing the chapters)
using the |\include| command.
This mechanism is beneficial for documents
which span hundreds of pages in order to
make the source file(s) more manageable.
Moreover, compilation can be restricted to
selected child files by means of the |\includeonly| command.
The latter feature can be used to reduce the compilation time while editing
(this was significantly more useful in the earlier days of \LaTeX{})
or to generate a smaller document which is easier to navigate.
Another application of |\includeonly| is to generate
documents consisting of selected parts of the complete document.

However, there are a few drawbacks of the plain |\include| mechanism:
\begin{itemize}
\item
The child files cannot be compiled on their own,
they can only be compiled via the main file.
A naive editing environment
(such as a text editor with an option
to have the current file processed by \LaTeX)
may require one to switch to the main file before compiling;
attempting to compile the child file produces errors.
\item
The main file must be modified (each time)
to adjust the |\includeonly| command
to the present needs. This easily leaves the main file in a messy state.
\item
The generated document will always carry the filename
of the main document. This is inconvenient if
several child files are to be compiled and
to be kept for distribution.
\end{itemize}

The present package provides a simple interface
to make child files individually compilable by \LaTeX{}.
Compiling a child file then has the same effect as compiling
the main file with an |\includeonly| command
to select the appropriate child.
Moreover the generated document will carry the name of the child
rather than the main file.
This resolves all three above issues.

This feature is meant to make the editing of books,
thesis documents and lecture notes somewhat more convenient.
However, the package can also be used efficiently for
composing a series of documents (such as exercise sheets)
which are typically distributed individually.
It then assists the author in generating the individual documents
(potentially in different versions)
as well as a document containing the collected series.
Another application is in developing style files
or other kinds of included material
where compilation of the style file could redirect
to a sample or test file.

%%%%%%%%%%%%%%%%%%%%%%%%%%%%%%%%%%%%%%%%%%%%%%%%%%%%%%%%%%%%%%%%%%%%%%%%%%%%%%%%
%%%%%%%%%%%%%%%%%%%%%%%%%%%%%%%%%%%%%%%%%%%%%%%%%%%%%%%%%%%%%%%%%%%%%%%%%%%%%%%%
\section{Usage}

First of all, the package \textsf{childdoc} is \emph{not} a standard
\LaTeXe{} |.sty| style file! Therefore it needs to be invoked in
a non-standard way.

%%%%%%%%%%%%%%%%%%%%%%%%%%%%%%%%%%%%%%%%%%%%%%%%%%%%%%%%%%%%%%%%%%%%%%%%%%%%%%%%
\subsection{Included Files}
\label{sec:include}

%%%%%%%%%%%%%%%%%%%%%%%%%%%%%%%%%%%%%%%%
\DescribeMacro{\childdocmain}
To use the package, add the commands
\begin{center}
\begin{tabular}{l}
|\input{childdoc.def}|\\
|\childdocmain{}|\\
\end{tabular}
\end{center}
at the very top of the main \LaTeX{} file,
in particular \emph{before} the |\documentclass| statement!
The argument of |\childdocmain| should be left empty
(but it must be present).

%%%%%%%%%%%%%%%%%%%%%%%%%%%%%%%%%%%%%%%%
\DescribeMacro{\childdocof}
Furthermore, add the commands
\begin{center}
\begin{tabular}{l}
|\input{childdoc.def}|\\
|\childdocof{|\textit{main}|}|\\
\end{tabular}
\end{center}
at the top of every child file \textit{child}
which is included by |\include{|\textit{child}|}|
from within the main file
(or at least for those files to be compiled individually).
The argument \textit{main} must be the filename of the main file.

There are a couple of
considerations in setting up the main and child documents:

%%%%%%%%%%%%%%%%%%%%%%%%%%%%%%%%%%%%%%%%
\paragraph{Restrictions.}

Please note the following restrictions:
\begin{itemize}
\item
|\childdocmain| must be called with one argument \textit{main}
to ensure compatibility with earlier version of the package.
It must either be empty (|\childdocmain{}|)
or precisely match the filename of the main file in which it is specified.
See \secref{sec:detection} for further information.
\item
The filename \textit{main} must be specified without the |.tex| extension.
\item
The filename \textit{main} is case sensitive
(even in case-insensitive file systems)
due to internal string comparison.
\item
The argument \textit{main} should be fully expanded, it cannot be a macro.
\item
Subdirectories and special characters should be avoided in filenames.
\item
The command |\childdocmain{|\textit{main}|}| must be followed by a whitespace.
It should not be followed immediately by another command
or by a comment mark `|%|'.
This is because the \TeX{} parser reads the token immediately following
the argument of |\childdocmain| and puts it
at the beginning of every child section;
however, a white\-space is ignored.
\end{itemize}

%%%%%%%%%%%%%%%%%%%%%%%%%%%%%%%%%%%%%%%%
\paragraph{Content of Main File.}

It is advisable to place all content in the child files included by |\include|.
Any output contained in the main file will appear in all child documents
unless suppressed manually;
it cannot be suppressed automatically by the |\includeonly| directive
and thus should normally be avoided.
A method to include some content in the main file
by means of conditional processing is described in \secref{sec:conditional}.

%%%%%%%%%%%%%%%%%%%%%%%%%%%%%%%%%%%%%%%%
\paragraph{Page Numbering.}

When only a part of the document is compiled,
the appropriate numbering of pages
(as well as other status parameters)
is determined from the |.aux| files.
The latter contain information from previous passes.
However this information needs to propagate through
all intermediate child documents.
Therefore the page numbering in child documents may well
be inconsistent until the complete document is compiled at least once.

A useful (if unconventional) way to always ensure a consistent
page numbering is to restart the numbering in each child document
and denote the pages by `\textit{child}|.|\textit{page}'
where \textit{child} represents the chapter/section number of the child file.
This can be achieved by the command
|\numberwithin{page}{|\textit{child}|}|
of the \textsf{amsmath} package
where \textit{child} can be |chapter| or |section|
depending on the chosen structuring.
Alternatively, one can modify the macro |\thepage| appropriately
and reset the counter |page| at the start of each child file.

%%%%%%%%%%%%%%%%%%%%%%%%%%%%%%%%%%%%%%%%%%%%%%%%%%%%%%%%%%%%%%%%%%%%%%%%%%%%%%%%
\subsection{Conditional Processing}
\label{sec:conditional}

The package provides a mechanism to compile different versions
of a document. To customise the versions further some conditional processing
can come in handy to distinguish which version is being compiled.
The package provides two macros to describe the compilation context:

%%%%%%%%%%%%%%%%%%%%%%%%%%%%%%%%%%%%%%%%
\DescribeMacro{\ifchilddoc}
The conditional |\ifchilddoc| distinguishes between the compilation of
child documents and the main document:
%
\begin{center}
|\ifchilddoc |\textit{child-code}| |[|\||else |\textit{main-code}]| \||fi|
\end{center}

%%%%%%%%%%%%%%%%%%%%%%%%%%%%%%%%%%%%%%%%
\DescribeMacro{\childdocname}
\DescribeMacro{\childdocjob}
The macro |\childdocname| contains the filename (without extension)
of the main or child file being processed.
Note that |\childdocjob| will always contain the name of the main file.

%%%%%%%%%%%%%%%%%%%%%%%%%%%%%%%%%%%%%%%%
\paragraph{Title Page.}

Conditional processing can be used to include a title or banner page
in the main document when proper precautions are taken.
Importantly, the code in the main file should ensure that the page counter
(as well as other status parameters which are stored in the |.aux| files)
takes the same value after the conditional processing.
Otherwise the page numbers may take divergent values
depending on which part is compiled.

For example, a title page could be declared by:
%
\begin{center}
\begin{tabular}{l}
|\ifchilddoc\||else|\\
|\addtocounter{page}{-1}|\\
\textit{code for title page}\\
|\newpage|\\
|\||fi|
\end{tabular}
\end{center}
%
A banner page for the child documents can be generated by:
%
\begin{center}
\begin{tabular}{l}
|\ifchilddoc|\\
|\addtocounter{page}{-1}|\\
\textit{code for banner page}\\
|\newpage|\\
|\||fi|
\end{tabular}
\end{center}
%
Here one could write a message such as:
\begin{center}
|This is the part \childdocname{} of \childdocjob{}.|
\end{center}

%%%%%%%%%%%%%%%%%%%%%%%%%%%%%%%%%%%%%%%%%%%%%%%%%%%%%%%%%%%%%%%%%%%%%%%%%%%%%%%%
\subsection{Flags}
\label{sec:flags}

The package makes it easy to generate different versions
of the main or child documents.
To this end compilation flags can be defined
and assigned different default values.
They will be particularly useful in conjunction
with the forwarding mechanism described in \secref{sec:forward}.

For example, it may be useful to have a flag |\version|
which can be set to |draft| or |final|.
The document source will contain some conditional code
depending on the value of |\version|.
Suppose further, the flag should default to |final| for the main file
and to |draft| for child files
which is a natural assignment for editing the document.
This is achieved by placing the following code
in the preamble of the main document
(below the |\childdocmain| directive):
%
\begin{center}
\begin{tabular}{l}
|\ifchilddoc|\\
|\providecommand{\version}{draft}|\\
|\||else|\\
|\providecommand{\version}{final}|\\
|\||fi|
\end{tabular}
\end{center}
%
The definition by |\providecommand| makes sure
that previous definitions are not overwritten.
Further statements |\providecommand{\version}{...}|
can thus be added before the above code to override it.

For the main file, one might add a line
(between |\childdocmain| and the above block)
%
\begin{center}
|%\ifchilddoc\||else\providecommand{\version}{draft}\||fi|
\end{center}
%
which can be uncommented to produce a draft version.
Likewise one can add a line to the very top of a child file
(above the |\childdocof{|\textit{main}|}| directive)
%
\begin{center}
|%\providecommand{\version}{final}|
\end{center}
%
which can be uncommented to produce the final version of this child document.

%%%%%%%%%%%%%%%%%%%%%%%%%%%%%%%%%%%%%%%%%%%%%%%%%%%%%%%%%%%%%%%%%%%%%%%%%%%%%%%%
\subsection{Forwarding}
\label{sec:forward}

Different versions of the main or child documents
using compilation flags as described in \secref{sec:flags}
can be (permanently) stored in different files
for convenient compilation, viewing and distribution.
To this end, the package defines a command
to pass on compilation to a different file:

%%%%%%%%%%%%%%%%%%%%%%%%%%%%%%%%%%%%%%%%
\DescribeMacro{\childdocforward}
The command |\childdocforward| redirects processing to
another source file:
%
\begin{center}
\begin{tabular}{l}
|\input{childdoc.def}|\\
|\childdocforward[|\textit{main}|]{|\textit{dest}|}|\\
\end{tabular}
\end{center}
%
The argument \textit{dest} is the destination file
(without extension).
It should be the main file or one of the child files.
Note that further \textsf{childdoc} directives
such as |\childdocof| and |\childdocforward|
in the indicated file will be processed in this form.
The optional argument \textit{main}
passes on directly to the main file \textit{main}
while pretending to compile the child \textit{dest}.
This form behaves as if \textit{dest}
issues |\childdocof{|\textit{main}|}| right away,
and no further \textsf{childdoc} directives will be processed.

%%%%%%%%%%%%%%%%%%%%%%%%%%%%%%%%%%%%%%%%
\DescribeMacro{\...prefix}
In the alternative form |\childdocforwardprefix|,
%
\begin{center}
\begin{tabular}{l}
|\input{childdoc.def}|\\
|\childdocforwardprefix[|\textit{main}|]{|\textit{prefix}|}{|\textit{dest}|}|
\end{tabular}
\end{center}
%
the destination file is determined by a pattern
depending on the current file:
To make this work, the current file must be called
`{\textit{prefix}\hspace{0.2em}\textit{suffix}}'
with \textit{prefix} matching precisely the argument.
Processing is then passed on to the file
`{\textit{dest}\hspace{0.2em}\textit{suffix}}'.
Surely, the same effect is achieved by
directly specifying the
argument `{\textit{dest}\hspace{0.2em}\textit{suffix}}'
in the first form.
However, that requires to set up a different file
for each child. With the alternative form of the command
all these files can have exactly the same content
which simplifies setting them up and maintaining them.

For example, the following file |draft.tex|
with a compilation flag |\version| as described in \secref{sec:flags}
compiles the main document as a draft:
%
\begin{center}
\begin{tabular}{l}
|\def\version{draft}|\\
|\input{childdoc.def}|\\
|\childdocforward{|\textit{main}|}|
\end{tabular}
\end{center}
%
Likewise, the following files |final|\textit{nn}|.tex|
compile the final version of the child document
|child|\textit{nn}|.tex|:
%
\begin{center}
\begin{tabular}{l}
|\def\version{final}|\\
|\input{childdoc.def}|\\
|\childdocforwardprefix{final}{child}|
\end{tabular}
\end{center}
%

Note that when several versions of a main file and/or of each child file
are to be generated, it may be convenient to set up a |Makefile| or
shell script to automatise the process.

%%%%%%%%%%%%%%%%%%%%%%%%%%%%%%%%%%%%%%%%%%%%%%%%%%%%%%%%%%%%%%%%%%%%%%%%%%%%%%%%
\subsection{Command Line Processing}
\label{sec:commandline}

The effect of redirection files can also be achieved by invoking
the \LaTeX{} compiler with a more elaborate command line.
Most conveniently this should be done as part
of a shell script or a |Makefile|.

When using \textsf{childdoc} in the main file, the following
command lines effectively perform a redirection
(note that depending on the shell being used,
backslashes may have to be doubled: `|\|' $\to$ `|\\|'):
%
\begin{center}
|... -jobname "|\textit{target}|" |\\|"|[\textit{flags}]%
|\input{childdoc.def}\childdocforward[|\textit{main}|]{|\textit{dest}|}"|
\end{center}
%
Here \textit{target} is the name of the output file,
\textit{main} is the name of the main file
and \textit{dest} is the name of the main or child file to be processed
(all filenames without extensions).
The optional argument \textit{main} can be omitted
if \textit{main} matches \textit{dest}.
Optionally, compilation \textit{flags} can be defined via |\def| commands.
This command line makes the \TeX{} engine believe
it is compiling the file \textit{target}
whose content is specified as the latter parameter.
The provided code then forwards the processing to
\textit{main} or \textit{dest} as described in \secref{sec:forward}.

%%%%%%%%%%%%%%%%%%%%%%%%%%%%%%%%%%%%%%%%%%%%%%%%%%%%%%%%%%%%%%%%%%%%%%%%%%%%%%%%
\subsection{Include by Input}
\label{sec:input}

Including child documents by |\include| has some restrictions by design.
Most notably, the content of a child document always occupies
its own set of pages; pages cannot be shared between child documents.
Usually, this behaviour makes perfect sense
because each child document contain an essential part of the document.
However, in some situations it may be desirable to compose
a document from a collection of parts
without having mandatory page breaks between then.
For this case, the package
provides a mechanism to include parts
by |\input| which can also be processed individually.
However, by construction this mechanism
requires manual handling of the content to be output.

%%%%%%%%%%%%%%%%%%%%%%%%%%%%%%%%%%%%%%%%
\DescribeMacro{\ifchilddocmanual}
The main file should be prepared as usual, see \secref{sec:include}.
However, the document body must make a distinction
between processing of an individual part and of the main document, e.g.:
%
\begin{center}
\begin{tabular}{l}
|\ifchilddocmanual|\\
|\input{\childdocname}|\\
|\||else|\\
\textit{document body with }|\input{|\textit{part}|}|\\
|\||fi|
\end{tabular}
\end{center}
%
The conditional |\ifchilddocmanual| is true whenever
a part to be included by |\input| is being compiled,
and the name of the part is stored in |\childdocname|.

%%%%%%%%%%%%%%%%%%%%%%%%%%%%%%%%%%%%%%%%
\DescribeMacro{\childdocby}
Each part to be included by |\input| should start with:
%
\begin{center}
\begin{tabular}{l}
|\input{childdoc.def}|\\
|\childdocby{|\textit{main}|}|\\
\end{tabular}
\end{center}
%
The directive |\childdocby| is similar to |\childdocof|
described in \secref{sec:include},
but the subsequent selection of content must be done manually.
To that end, both |\ifchilddoc| and |\ifchilddocmanual|
will be true upon processing of a part,
and the name of the part is stored in |\childdocname|.
Note that |\jobname| will be set to the filename of the current part
so that each part receives an individual |.aux| file
that does not interfere with the |.aux| file(s) of the main document.
This behaviour can be altered by the alternative form
|\childdocby[*]{|\textit{main}|}| (with a non-empty optional argument)
which uses the |.aux| file of the main document
by setting |\jobname| to \textit{main}.

%%%%%%%%%%%%%%%%%%%%%%%%%%%%%%%%%%%%%%%%%%%%%%%%%%%%%%%%%%%%%%%%%%%%%%%%%%%%%%%%
\subsection{Driver Development}
\label{sec:driver}

The \textsf{childdoc} mechanism can also be use for the development
of definition files such as \LaTeX{} styles or classes.
This case differs from the above setup with multiple parts
included by |\include| in that no |\includeonly| should be invoked.
This can be achieved by starting the include file
(before |\ProvidesPackage|) with:
%
\begin{center}
\begin{tabular}{l}
|\input{childdoc.def}|\\
|\childdocforward{|\textit{main}|}|\\
\end{tabular}
\end{center}
%
or alternatively with:
%
\begin{center}
\begin{tabular}{l}
|\input{childdoc.def}|\\
|\childdocby{|\textit{main}|}|\\
\end{tabular}
\end{center}
%
Both forms have slightly different effects as described above.
The main file is prepared as usual, see \secref{sec:include}.

%%%%%%%%%%%%%%%%%%%%%%%%%%%%%%%%%%%%%%%%%%%%%%%%%%%%%%%%%%%%%%%%%%%%%%%%%%%%%%%%
\subsection{Legacy Detection}
\label{sec:detection}

The directive |\childdocmain| in the main file can detect
whether the complete document or merely a child is to be compiled
even without using the directive |\childdocof|.
This method is deprecated because it is less robust
and there is no compelling reason to use it;
it is merely provided for backward compatibility
and it may be removed in future versions.

If the detection mechanism is to be used,
it is mandatory to correctly specify
the filename of the main file as the argument of |\childdocmain|:
%
\begin{center}
\begin{tabular}{l}
|\input{childdoc.def}|\\
|\childdocmain{|\textit{main}|}|\\
\end{tabular}
\end{center}
%
If |\jobname| does not match the argument \textit{main} of |\childdocmain|,
it is assumed that |\jobname| points to the child file to be compiled.
When using |\childdocmain| with the main file specified as argument,
it suffices to start a child file
with just |\input{|\textit{main}|}|
without loading of the package and using |\childdocof|.
If instead all processing is done
with the appropriate \textsf{childdoc} directives,
the argument of \textit{main} of |\childdocmain| can be empty.

An alternative version of the command line processing described
in \secref{sec:commandline} using the detection mechanism reads:
%
\begin{center}
|... -jobname "|\textit{target}|" "|[\textit{flags}]%
[|\def\jobname{|\textit{dest}|}|]|\input{|\textit{main}|}"|
\end{center}

%%%%%%%%%%%%%%%%%%%%%%%%%%%%%%%%%%%%%%%%%%%%%%%%%%%%%%%%%%%%%%%%%%%%%%%%%%%%%%%%
\subsection{Manual Code}
\label{sec:manual}

In case one cannot be certain whether the definitions file |childdoc.def|
is installed on the target \TeX{} distribution
and one prefers not to ship it,
it is conceivable to paste a few relevant commands into the sources.

To that end, drop all statements |\input{childdoc.def}|
and perform the replacements as outlined below.
Instead of |\childdocmain{|\textit{main}|}| add the following code
to the top of the main file:
%
\begin{center}
\begin{tabular}{l}
|\||ifdefined\childdocname\endinput\||fi\newif\ifchilddoc|\\
|\edef\childdocname{\scantokens\expandafter{\jobname\noexpand}}|\\
|\def\childdocmain{|\textit{main}|}\||ifx\childdocmain\childdocname\||else|\\
|\childdoctrue\includeonly{\childdocname}\let\jobname\childdocmain\||fi|\\
\end{tabular}
\end{center}
%
Instead of |\childdocof{|\textit{main}|}| just include the main file
at the top of each child file:
%
\begin{center}
|\input{|\textit{main}|}|
\end{center}
%
A simple redirection |\childdocforward{|\textit{dest}|}| is achieved by:
%
\begin{center}
|\def\jobname{|\textit{dest}|}\input{\jobname}|
\end{center}
%
The redirection with prefix
|\childdocforwardprefix[|\textit{prefix}|]{|\textit{dest}|}|
is accomplished by:
%
\begin{center}
\begin{tabular}{l}
|{\edef\jobname{\scantokens\expandafter{\jobname\noexpand}}|\\
|\def\redirectjob |\textit{prefix}|#1~~~{\gdef\jobname{|\textit{dest}|#1}}|\\
|\expandafter\redirectjob\jobname~~~}\input{\jobname}|
\end{tabular}
\end{center}

In an alternative approach,
child documents can be compiled by a specific command line
without additional code or specific definitions:
%
\begin{center}
|... -jobname "|\textit{target}|" "|[\textit{flags}]%
|\includeonly{|\textit{dest}|}\input{|\textit{main}|}"|
\end{center}
%

%%%%%%%%%%%%%%%%%%%%%%%%%%%%%%%%%%%%%%%%%%%%%%%%%%%%%%%%%%%%%%%%%%%%%%%%%%%%%%%%
%%%%%%%%%%%%%%%%%%%%%%%%%%%%%%%%%%%%%%%%%%%%%%%%%%%%%%%%%%%%%%%%%%%%%%%%%%%%%%%%
\section{Information}

%%%%%%%%%%%%%%%%%%%%%%%%%%%%%%%%%%%%%%%%%%%%%%%%%%%%%%%%%%%%%%%%%%%%%%%%%%%%%%%%
\subsection{Copyright}

Copyright \copyright{} 2017--2018 Niklas Beisert

This work may be distributed and/or modified under the
conditions of the \LaTeX{} Project Public License, either version 1.3
of this license or (at your option) any later version.
The latest version of this license is in
  \url{http://www.latex-project.org/lppl.txt}
and version 1.3 or later is part of all distributions of \LaTeX{}
version 2005/12/01 or later.

This work has the LPPL maintenance status `maintained'.

The Current Maintainer of this work is Niklas Beisert.

This work consists of the files |README.txt|, |childdoc.ins| and |childdoc.dtx|
as well as the derived files |childdoc.def|, |cdocsamp.tex|
with |cdocsch1.tex|, |cdocsch2.tex|, |cdocspt3.tex|, |cdocspt4.tex|,
|cdocsdrf.tex|, |cdocsfn1.tex|, |cdocsfn2.tex|
as well as |childdoc.pdf|.

%%%%%%%%%%%%%%%%%%%%%%%%%%%%%%%%%%%%%%%%%%%%%%%%%%%%%%%%%%%%%%%%%%%%%%%%%%%%%%%%
\subsection{Files and Installation}

The package consists of the files:
%
\begin{center}
\begin{tabular}{ll}
    |README.txt|   & readme file \\
    |childdoc.ins| & installation file \\
    |childdoc.dtx| & source file \\
    |childdoc.def| & definition file \\
    |cdocsamp.tex| & sample main file \\
    |cdocsch1.tex| & sample include file \\
    |cdocsch2.tex| & sample include file \\
    |cdocspt3.tex| & sample part file \\
    |cdocspt4.tex| & sample part file \\
    |cdocsdrf.tex| & sample redirection file \\
    |cdocsfn1.tex| & sample redirection file \\
    |cdocsfn2.tex| & sample redirection file \\
    |childdoc.pdf| & manual
\end{tabular}
\end{center}
%
The distribution consists of the files
|README.txt|, |childdoc.ins| and |childdoc.dtx|.
%
\begin{itemize}
\item
Run (pdf)\LaTeX{} on |childdoc.dtx|
to compile the manual |childdoc.pdf| (this file).
\item
Run \LaTeX{} on |childdoc.ins| to create the definitions file |childdoc.def|
and the sample |cdocsamp.tex| with include files
|cdocsch1.tex|, |cdocsch2.tex|, |cdocspt3.tex|, |cdocspt4.tex|,
|cdocsdrf.tex|, |cdocsfn1.tex|, |cdocsfn2.tex|.
Then copy the file |childdoc.def| to an appropriate directory of your \LaTeX{}
distribution, e.g.\ \textit{texmf-root}|/tex/latex/childdoc|.
\end{itemize}

%%%%%%%%%%%%%%%%%%%%%%%%%%%%%%%%%%%%%%%%%%%%%%%%%%%%%%%%%%%%%%%%%%%%%%%%%%%%%%%%
\subsection{Related CTAN Packages}

There are several other packages which offer a similar functionality:
%
\begin{itemize}
\item
The packages
\href{http://ctan.org/pkg/docmute}{\textsf{docmute}},
\href{http://ctan.org/pkg/includex}{\textsf{includex}} and
\href{http://ctan.org/pkg/standalone}{\textsf{standalone}}
provide commands to include only the document body of
a child file thus allowing both files to be compiled individually.
\item
The packages \href{http://ctan.org/pkg/subdocs}{\textsf{subdocs}}
and \href{http://ctan.org/pkg/subfiles}{\textsf{subfiles}}
provide structures in which the main and child documents can be
encapsulated and allowing them to be compiled individually.
The inclusion mechanism is different from the conventional |\include|.
\item
The package \href{http://ctan.org/pkg/combine}{\textsf{combine}}
is an elaborate solution to combine several documents into one.
\end{itemize}
%
See also the CTAN topic \href{http://ctan.org/topic/subdocs}{\textsf{subdocs}}
for further related packages.
The present package differs from the above solutions in that
a document structure constructed with the conventional |\include| mechanism
just needs two extra commands at the top of every file
such that all constituent files can be compiled individually.

%%%%%%%%%%%%%%%%%%%%%%%%%%%%%%%%%%%%%%%%%%%%%%%%%%%%%%%%%%%%%%%%%%%%%%%%%%%%%%%%
%\subsection{Feature Suggestions}
%
%The following is a list of features which may be useful for future
%versions of this package:
%%
%\begin{itemize}
%\item
%\ldots
%\end{itemize}

%%%%%%%%%%%%%%%%%%%%%%%%%%%%%%%%%%%%%%%%%%%%%%%%%%%%%%%%%%%%%%%%%%%%%%%%%%%%%%%%
\subsection{Revision History}

%%%%%%%%%%%%%%%%%%%%%%%%%%%%%%%%%%%%%%%%
\paragraph{v2.0:} 2018/12/30

\begin{itemize}
\item
immediate forward processing
\item
added |\childdocby| mechanism
\item
manual restructured
\end{itemize}

%%%%%%%%%%%%%%%%%%%%%%%%%%%%%%%%%%%%%%%%
\paragraph{v1.6:} 2018/01/17

\begin{itemize}
\item
application for development of include files
\item
corrections to manual
\end{itemize}

%%%%%%%%%%%%%%%%%%%%%%%%%%%%%%%%%%%%%%%%
\paragraph{v1.5:} 2017/05/21

\begin{itemize}
\item
more complete structuring introduced
\item
|\childdocof| introduced
\item
|\childdoc| renamed to |\childdocmain|
\item
|\childredirect| renamed to |\childdocforward| and |\childdocforwardprefix|
and functionality expanded
\end{itemize}

%%%%%%%%%%%%%%%%%%%%%%%%%%%%%%%%%%%%%%%%
\paragraph{v1.0:} 2017/04/27

\begin{itemize}
\item
manual and install package
\item
first version published on CTAN
\end{itemize}

%%%%%%%%%%%%%%%%%%%%%%%%%%%%%%%%%%%%%%%%
\paragraph{v0.6:} 2017/04/26

\begin{itemize}
\item
redirection mechanism added
\end{itemize}

%%%%%%%%%%%%%%%%%%%%%%%%%%%%%%%%%%%%%%%%
\paragraph{v0.5:} 2017/04/26

\begin{itemize}
\item
functionality in definition file
\end{itemize}


%%%%%%%%%%%%%%%%%%%%%%%%%%%%%%%%%%%%%%%%%%%%%%%%%%%%%%%%%%%%%%%%%%%%%%%%%%%%%%%%
%%%%%%%%%%%%%%%%%%%%%%%%%%%%%%%%%%%%%%%%%%%%%%%%%%%%%%%%%%%%%%%%%%%%%%%%%%%%%%%%
%%%%%%%%%%%%%%%%%%%%%%%%%%%%%%%%%%%%%%%%%%%%%%%%%%%%%%%%%%%%%%%%%%%%%%%%%%%%%%%%
\appendix

\settowidth\MacroIndent{\rmfamily\scriptsize 000\ }

 \DocInput{childdoc.dtx}

\end{document}
%</driver>
% \fi
%
% %%%%%%%%%%%%%%%%%%%%%%%%%%%%%%%%%%%%%%%%%%%%%%%%%%%%%%%%%%%%%%%%%%%%%%%%%%%%%%
% %%%%%%%%%%%%%%%%%%%%%%%%%%%%%%%%%%%%%%%%%%%%%%%%%%%%%%%%%%%%%%%%%%%%%%%%%%%%%%
% \section{Sample}
%\iffalse
%<*samplemain>
%\fi
%
% The following presents a sample document
% with two chapters, two parts, a title page,
% a compile flag as well as three forwarding files to set the flag.
% It consists of eight |.tex| files:
% \begin{center}
% \begin{tabular}{ll}
% |cdocsamp.tex|&main file\\
% |cdocsch1.tex|&include file for chapter 1\\
% |cdocsch2.tex|&include file for chapter 2\\
% |cdocspt3.tex|&include file for part 3\\
% |cdocspt4.tex|&include file for part 4\\
% |cdocsdrf.tex|&forwarding file for main file in draft mode\\
% |cdocsfi1.tex|&forwarding file for final version of chapter 1\\
% |cdocsfi2.tex|&forwarding file for final version of chapter 2\\
% \end{tabular}
% \end{center}
% Each of the eight files can be compiled directly by the \LaTeX{} compiler.
%
% %%%%%%%%%%%%%%%%%%%%%%%%%%%%%%%%%%%%%%
% \paragraph{Main File.}
%
% The main file is called |cdocsamp.tex|.
%
% Load the \textsf{childdoc} definitions and
% declare the filename for the main document:
%    \begin{macrocode}
\input{childdoc.def}
\childdocmain{}
%    \end{macrocode}

% Optional override for |\version| flag:
%    \begin{macrocode}
%%\ifchilddoc\else\providecommand{\version}{draft}\fi
%    \end{macrocode}

% Define the default values for the |\version| flag
% (|final| for the main file and |draft| for childs):
%    \begin{macrocode}
\ifchilddoc
\providecommand{\version}{draft}
\else
\providecommand{\version}{final}
\fi
%    \end{macrocode}

% Load the standard document class:
%    \begin{macrocode}
\documentclass[12pt]{article}
%    \end{macrocode}

% Start the document body:
%    \begin{macrocode}
\begin{document}
%    \end{macrocode}

% Declare a title page.
% Print title, part of document being processed and version flag:
%    \begin{macrocode}
\addtocounter{page}{-1}
\begin{center}
{\LARGE\bfseries{}childdoc example\par}
\vspace{1cm}
\ifchilddoc
\ifchilddocmanual part\else chapter\fi:
`\childdocname' of `\childdocjob'\par
\else
main document: `\childdocjob'\par
\fi
version: \version\par
\end{center}
\newpage
%    \end{macrocode}

% Manually include selected file,
% otherwise process as usual:
%    \begin{macrocode}
\ifchilddocmanual
\section*{part `\childdocname'}
\input{\childdocname}
\else
%    \end{macrocode}

% Include the two chapters:
%    \begin{macrocode}
\include{cdocsch1}
\include{cdocsch2}
%    \end{macrocode}

% Include the two parts unless only chapters should be displayed:
%    \begin{macrocode}
\ifchilddoc\else
\section{part three}
\input{cdocspt3}
\section{part four}
\input{cdocspt4}
\fi
%    \end{macrocode}

% Process as usual until here:
%    \begin{macrocode}
\fi
%    \end{macrocode}

% End of document body:
%    \begin{macrocode}
\end{document}
%    \end{macrocode}
%\iffalse
%</samplemain>
%\fi
%
% %%%%%%%%%%%%%%%%%%%%%%%%%%%%%%%%%%%%%%
% \paragraph{Chapter Include Files.}
%
% The include files are called |cdocsch1.tex| and |cdocsch2.tex|.
%
%\iffalse
%<*samplechap1|samplechap2>
%\fi

% Optional override for |\version| flag:
%    \begin{macrocode}
%%\providecommand{\version}{final}
%    \end{macrocode}

% Include the main document:
%    \begin{macrocode}
\input{childdoc.def}
\childdocof{cdocsamp}
%    \end{macrocode}

%\iffalse
%</samplechap1|samplechap2>
%\fi
%
%\iffalse
%<*samplechap1>
%\fi
% Some text for chapter 1:
%    \begin{macrocode}
\section{one}
some text in chapter one
%    \end{macrocode}

%\iffalse
%</samplechap1>
%\fi
% Some text for chapter 2:
%\iffalse
%<*samplechap2>
%\fi
%    \begin{macrocode}
\section{two}
more text in chapter two
%    \end{macrocode}

%\iffalse
%</samplechap2>
%\fi
%
% %%%%%%%%%%%%%%%%%%%%%%%%%%%%%%%%%%%%%%
% \paragraph{Part Include Files.}
%
% The include files are called |cdocspt3.tex| and |cdocspt4.tex|.
%
%\iffalse
%<*samplepart3|samplepart4>
%\fi

% Optional override for |\version| flag:
%    \begin{macrocode}
%%\providecommand{\version}{final}
%    \end{macrocode}

% Include the main document:
%    \begin{macrocode}
\input{childdoc.def}
\childdocby{cdocsamp}
%    \end{macrocode}

%\iffalse
%</samplepart3|samplepart4>
%\fi
%
%\iffalse
%<*samplepart3>
%\fi
% Some text for part 3:
%    \begin{macrocode}
some text in part three
%    \end{macrocode}

%\iffalse
%</samplepart3>
%\fi
% Some text for part 4:
%\iffalse
%<*samplepart4>
%\fi
%    \begin{macrocode}
more text in part four
%    \end{macrocode}

%\iffalse
%</samplepart4>
%\fi
%
% %%%%%%%%%%%%%%%%%%%%%%%%%%%%%%%%%%%%%%
% \paragraph{Forwarding for a Complete Draft.}
%
% The following forwarding file |cdocsdrf.tex|
% compiles the main document in draft mode:
%\iffalse
%<*sampledraft>
%\fi
%    \begin{macrocode}
\def\version{draft}
\input{childdoc.def}
\childdocforward{cdocsamp}
%    \end{macrocode}

%\iffalse
%</sampledraft>
%\fi
%
% %%%%%%%%%%%%%%%%%%%%%%%%%%%%%%%%%%%%%%
% \paragraph{Forwarding for Final Version of the Chapters.}
%
% The following forwarding files |cdocsfn1.tex| and |cdocsfn2.tex|
% (with identical content)
% compile the final versions of the child documents
% |cdocsch1.tex| and |cdocsch2.tex|, respectively:
%\iffalse
%<*samplefinal>
%\fi
%    \begin{macrocode}
\def\version{final}
\input{childdoc.def}
\childdocforwardprefix[cdocsamp]{cdocsfn}{cdocsch}
%    \end{macrocode}

%\iffalse
%</samplefinal>
%\fi
%
% %%%%%%%%%%%%%%%%%%%%%%%%%%%%%%%%%%%%%%
% \paragraph{Command Line Processing.}
%
% The following three command lines generate the output files
% |cdocscld|, |cdocscl1| and |cdocscl2|
% which should be identical to
% |cdocsdrf|, |cdocsch1| and |cdocsfn2|, respectively:
% \begin{center}
% \begin{tabular}{l}
% |latex -jobname cdocscld \|\\
% |  "\def\version{draft}\input{childdoc.def}\childdocforward{cdocsamp}"|\\
% |latex -jobname cdocscl1 \|\\
% |  "\input{childdoc.def}\childdocforward[cdocsamp]{cdocsch1}"|\\
% |latex -jobname cdocscl2 \|\\
% |  "\def\version{final}\input{childdoc.def}\childdocforward{cdocsch2}"|
% \end{tabular}
% \end{center}
% Note that the trailing backslash on each first line
% merely continues the input to the second line
% (for convenient cut ant paste).
% Furthermore, the command |latex| can be replaced by any
% of its alternative versions such as |pdflatex|.
%
% %%%%%%%%%%%%%%%%%%%%%%%%%%%%%%%%%%%%%%%%%%%%%%%%%%%%%%%%%%%%%%%%%%%%%%%%%%%%%%
% %%%%%%%%%%%%%%%%%%%%%%%%%%%%%%%%%%%%%%%%%%%%%%%%%%%%%%%%%%%%%%%%%%%%%%%%%%%%%%
% \section{Implementation}
%\iffalse
%<*package>
%\fi
%
% This section describes the definitions file |childdoc.def|.

% The definitions cannot be loaded using |\usepackage| or |\RequirePackage|
% which has a mechanism to prevent loading a style file more than once.
% When loading the definitions by means of |\input|
% multiple instances have to be prevented manually:
%\iffalse
%This code needs to be before the `\ProvidesFile' directive
%which is defined at the beginning of this file.
%Therefore it is also placed there and commented out here.
%</package>
%<*discard>
%\fi
%    \begin{macrocode}
\ifdefined\childdocmain\endinput\fi
%    \end{macrocode}
%\iffalse
%</discard>
%<*package>
%\fi
%
% \macro{\ifchilddoc}
% \macro{\ifchilddocmanual}
% The conditional |\ifchilddoc| tells whether a
% child (true) or main (false) document is being compiled.
% The conditional |\ifchilddocmanual| tells whether
% the |\includeonly| mechanism is used (false) or
% the selection of child files must be performed manually (true).
% The definitions initialise to false:
%    \begin{macrocode}
\newif\ifchilddoc
\newif\ifchilddocmanual
%    \end{macrocode}

% \macro{\childdocname}
% \macro{\childdocjob}
% The macro |\childdocname| stores the name of the main document
% to be compiled. The macro |\childdocjob| stores the name of
% the document on which the \LaTeX{} compiler was originally invoked.
% The content of |\jobname| cannot be compared
% to filenames specified in the source due to different catcodes.
% The following code rescans |\jobname|, stores the result
% in |\childdocname| and saves a copy in |\childdocjob|:
%    \begin{macrocode}
\edef\childdocname{\scantokens\expandafter{\jobname\noexpand}}
\let\childdocjob\childdocname
%    \end{macrocode}

% \macro{\childdocdisable}
% The macro |\childdocdisable| prevents the main file
% from being processed more than once.
% At this stage, the main document command |\childdocmain|
% is assumed to be called once again where it should do nothing.
% Any subsequent call to it should prevent
% a secondary processing of the main document
% It overwrites the forwarding commands
% |\childdocof| and |\childdocforward|
% with empty macros to prevent further inclusions of the main document:
%    \begin{macrocode}
\newcommand{\childdocdisable}
{
  \renewcommand{\childdocmain}[1]{\renewcommand{\childdocmain}[1]{\endinput}}
  \renewcommand{\childdocof}[1]{}
  \renewcommand{\childdocby}[2][]{}
  \renewcommand{\childdocforward}[2][]{}
  \renewcommand{\childdocdisable}{}
}
%    \end{macrocode}

% \macro{\childdocmain}
% The macro |\childdocmain| is to be called at the top of the main file
% with nothing or the main filename (without extension) as argument.
% First, it breaks loops.
% If the argument is not empty and does not match |\childdocname|
% (which is set by the first inclusion of |childdoc.def|),
% |\ifchilddoc| is set to true, |\includeonly| is applied to the child file
% and |\jobname| is set to the main file
% (for proper handling of |.aux| files):
%    \begin{macrocode}
\newcommand{\childdocmain}[1]
{
  \childdocdisable\childdocmain{}
  \if?#1?\else
    \begingroup
      \def\childdoctmp{#1}
      \ifx\childdoctmp\childdocname
        \def\childdoctmp{}
      \else
        \def\childdoctmp
        {
          \childdoctrue
          \includeonly{\childdocname}
          \def\childdocjob{#1}
          \def\jobname{#1}
        }
      \fi
      \expandafter
    \endgroup
    \childdoctmp
  \fi
}
%    \end{macrocode}

% \macro{\childdocof}
% The command |\childdocof| redirects
% compilation to the main file |#1|.
%    \begin{macrocode}
\newcommand{\childdocof}[1]
{
  \childdocdisable
  \childdoctrue
  \includeonly{\childdocname}
  \def\jobname{#1}
  \def\childdocjob{#1}
  \input{#1}
}
%    \end{macrocode}

% \macro{\childdocby}
% The command |\childdocby| ....
%    \begin{macrocode}
\newcommand{\childdocby}[2][]
{
  \childdocdisable
  \childdoctrue
  \childdocmanualtrue
  \if?#1?\else
    \def\jobname{#2}
  \fi
  \def\childdocjob{#2}
  \input{#2}
  \endinput
}
%    \end{macrocode}

% \macro{\childdocforward}
% The command |\childdocforward| redirects
% compilation to the main file or
% (if the optional argument is given) a child file.
% Parameters are set as if the main file
% or a child file starting with |\childdocof| was compiled.
% Then compilation is handed over to the main file:
%    \begin{macrocode}
\newcommand{\childdocforward}[2][]
{
  \begingroup
    \if?#1?
      \def\childdoctmp
      {
        \def\childdocname{#2}
        \def\childdocjob{#2}
        \def\jobname{#2}
        \input{#2}
        \endinput
      }
    \else
      \def\childdoctmp
      {
        \childdocdisable
        \def\childdocname{#2}
        \childdoctrue
        \includeonly{#2}
        \def\childdocjob{#1}
        \def\jobname{#1}
        \input{#1}
        \endinput
      }
    \fi
    \expandafter
  \endgroup
  \childdoctmp
}
%    \end{macrocode}

% \macro{\childdocforwardprefix}
% The command |\childdocforwardprefix| redirects
% compilation to the main or a child file by means of a pattern.
% The prefix |#1| in the current filename is replaced by |#2|
% and the suffix of the current filename is kept
% (it is assumed that the filename does not contain the substring `|~~~|'
% which is used as a delimiter).
% Compilation is handed over to the new file by |\childdocforward|:
%    \begin{macrocode}
\newcommand{\childdocforwardprefix}[3][]
{
  \begingroup
    \def\childdocextract #2##1~~~{\def\childdoctmp{\childdocforward[#1]{#3##1}}}
    \expandafter\childdocextract\childdocname~~~
    \expandafter
  \endgroup
  \childdoctmp
}
%    \end{macrocode}

% \macro{\childdoc}
% The deprecated macro |\childdoc| is a legacy version of |\childdocmain|:
%    \begin{macrocode}
\newcommand{\childdoc}{\childdocmain}
%    \end{macrocode}

% \macro{\childdocredirect}
% The deprecated macro |\childdocredirect| is a legacy version
% of |\childdocforward| and |\childdocforwardprefix|:
%    \begin{macrocode}
\newcommand{\childdocredirect}[2][]
{
  \begingroup
    \if?#1?
      \def\childdoctmp{\childdocforward{#2}}
    \else
      \def\childdoctmp{\childdocforwardprefix{#1}{#2}}
    \fi
    \expandafter
  \endgroup
  \childdoctmp
}
%    \end{macrocode}

%\iffalse
%</package>
%\fi
%
\endinput
|\\
|\childdocforwardprefix[|\textit{main}|]{|\textit{prefix}|}{|\textit{dest}|}|
\end{tabular}
\end{center}
%
the destination file is determined by a pattern
depending on the current file:
To make this work, the current file must be called
`{\textit{prefix}\hspace{0.2em}\textit{suffix}}'
with \textit{prefix} matching precisely the argument.
Processing is then passed on to the file
`{\textit{dest}\hspace{0.2em}\textit{suffix}}'.
Surely, the same effect is achieved by
directly specifying the
argument `{\textit{dest}\hspace{0.2em}\textit{suffix}}'
in the first form.
However, that requires to set up a different file
for each child. With the alternative form of the command
all these files can have exactly the same content
which simplifies setting them up and maintaining them.

For example, the following file |draft.tex|
with a compilation flag |\version| as described in \secref{sec:flags}
compiles the main document as a draft:
%
\begin{center}
\begin{tabular}{l}
|\def\version{draft}|\\
|% \iffalse
%
% childdoc.dtx Copyright (C) 2017-2018 Niklas Beisert
%
% This work may be distributed and/or modified under the
% conditions of the LaTeX Project Public License, either version 1.3
% of this license or (at your option) any later version.
% The latest version of this license is in
%   http://www.latex-project.org/lppl.txt
% and version 1.3 or later is part of all distributions of LaTeX
% version 2005/12/01 or later.
%
% This work has the LPPL maintenance status `maintained'.
%
% The Current Maintainer of this work is Niklas Beisert.
%
% This work consists of the files childdoc.dtx and childdoc.ins
% and the derived files childdoc.def and cdocsamp.tex with
% cdocsch1.tex, cdocsch2.tex, cdocsdrf.tex, cdocsfn1.tex, cdocsfn2.tex.
%
%<package>\ifdefined\childdocmain\endinput\fi
%<package>\ProvidesFile{childdoc.def}[2018/12/30 v2.0 child document driver]
%<samplemain>\ProvidesFile{cdocsamp.tex}[2018/12/30 v2.0 sample for childdoc]
%<*driver>
%\ProvidesFile{childdoc.drv}[2018/12/30 v2.0 childdoc reference manual file]
\PassOptionsToClass{10pt,a4paper}{article}
\documentclass{ltxdoc}

\usepackage[margin=35mm]{geometry}
\usepackage{hyperref}
\usepackage{hyperxmp}
\usepackage[usenames]{color}

\hypersetup{colorlinks=true}
\hypersetup{pdfstartview=FitH}
\hypersetup{pdfpagemode=UseNone}
\hypersetup{pdfsource={}}
\hypersetup{pdflang={en-UK}}
\hypersetup{pdfcopyright={Copyright 2017-2018 Niklas Beisert.
  This work may be distributed and/or modified under the
  conditions of the LaTeX Project Public License, either version 1.3
  of this license or (at your option) any later version.}}
\hypersetup{pdflicenseurl={http://www.latex-project.org/lppl.txt}}
\hypersetup{pdfcontactaddress={ETH Zurich, ITP, HIT K,
  Wolfgang-Pauli-Strasse 27}}
\hypersetup{pdfcontactpostcode={8093}}
\hypersetup{pdfcontactcity={Zurich}}
\hypersetup{pdfcontactcountry={Switzerland}}
\hypersetup{pdfcontactemail={nbeisert@itp.phys.ethz.ch}}
\hypersetup{pdfcontacturl={http://people.phys.ethz.ch/\xmptilde nbeisert/}}

\newcommand{\secref}[1]{\hyperref[#1]{section \ref*{#1}}}

\parskip1ex
\parindent0pt
\let\olditemize\itemize
\def\itemize{\olditemize\parskip0pt}

\begin{document}

\title{The \textsf{childdoc} Package}
\hypersetup{pdftitle={The childdoc Package}}
\author{Niklas Beisert\\[2ex]
  Institut f\"ur Theoretische Physik\\
  Eidgen\"ossische Technische Hochschule Z\"urich\\
  Wolfgang-Pauli-Strasse 27, 8093 Z\"urich, Switzerland\\[1ex]
  \href{mailto:nbeisert@itp.phys.ethz.ch}
  {\texttt{nbeisert@itp.phys.ethz.ch}}}
\hypersetup{pdfauthor={Niklas Beisert}}
\hypersetup{pdfsubject={Manual for the LaTeX2e Package childdoc}}
\date{30 December 2018, \textsf{v2.0}}
\maketitle

\begin{abstract}\noindent
\textsf{childdoc} is a \LaTeXe{} package
that enables the direct compilation
of document sections included by |\include|
to individual files.
\end{abstract}

\begingroup
\parskip0ex
\tableofcontents
\endgroup

%%%%%%%%%%%%%%%%%%%%%%%%%%%%%%%%%%%%%%%%%%%%%%%%%%%%%%%%%%%%%%%%%%%%%%%%%%%%%%%%
%%%%%%%%%%%%%%%%%%%%%%%%%%%%%%%%%%%%%%%%%%%%%%%%%%%%%%%%%%%%%%%%%%%%%%%%%%%%%%%%
\section{Introduction}

\LaTeX{} provides a mechanism to structure a large document (such as a book)
into a main file and several child files (containing the chapters)
using the |\include| command.
This mechanism is beneficial for documents
which span hundreds of pages in order to
make the source file(s) more manageable.
Moreover, compilation can be restricted to
selected child files by means of the |\includeonly| command.
The latter feature can be used to reduce the compilation time while editing
(this was significantly more useful in the earlier days of \LaTeX{})
or to generate a smaller document which is easier to navigate.
Another application of |\includeonly| is to generate
documents consisting of selected parts of the complete document.

However, there are a few drawbacks of the plain |\include| mechanism:
\begin{itemize}
\item
The child files cannot be compiled on their own,
they can only be compiled via the main file.
A naive editing environment
(such as a text editor with an option
to have the current file processed by \LaTeX)
may require one to switch to the main file before compiling;
attempting to compile the child file produces errors.
\item
The main file must be modified (each time)
to adjust the |\includeonly| command
to the present needs. This easily leaves the main file in a messy state.
\item
The generated document will always carry the filename
of the main document. This is inconvenient if
several child files are to be compiled and
to be kept for distribution.
\end{itemize}

The present package provides a simple interface
to make child files individually compilable by \LaTeX{}.
Compiling a child file then has the same effect as compiling
the main file with an |\includeonly| command
to select the appropriate child.
Moreover the generated document will carry the name of the child
rather than the main file.
This resolves all three above issues.

This feature is meant to make the editing of books,
thesis documents and lecture notes somewhat more convenient.
However, the package can also be used efficiently for
composing a series of documents (such as exercise sheets)
which are typically distributed individually.
It then assists the author in generating the individual documents
(potentially in different versions)
as well as a document containing the collected series.
Another application is in developing style files
or other kinds of included material
where compilation of the style file could redirect
to a sample or test file.

%%%%%%%%%%%%%%%%%%%%%%%%%%%%%%%%%%%%%%%%%%%%%%%%%%%%%%%%%%%%%%%%%%%%%%%%%%%%%%%%
%%%%%%%%%%%%%%%%%%%%%%%%%%%%%%%%%%%%%%%%%%%%%%%%%%%%%%%%%%%%%%%%%%%%%%%%%%%%%%%%
\section{Usage}

First of all, the package \textsf{childdoc} is \emph{not} a standard
\LaTeXe{} |.sty| style file! Therefore it needs to be invoked in
a non-standard way.

%%%%%%%%%%%%%%%%%%%%%%%%%%%%%%%%%%%%%%%%%%%%%%%%%%%%%%%%%%%%%%%%%%%%%%%%%%%%%%%%
\subsection{Included Files}
\label{sec:include}

%%%%%%%%%%%%%%%%%%%%%%%%%%%%%%%%%%%%%%%%
\DescribeMacro{\childdocmain}
To use the package, add the commands
\begin{center}
\begin{tabular}{l}
|\input{childdoc.def}|\\
|\childdocmain{}|\\
\end{tabular}
\end{center}
at the very top of the main \LaTeX{} file,
in particular \emph{before} the |\documentclass| statement!
The argument of |\childdocmain| should be left empty
(but it must be present).

%%%%%%%%%%%%%%%%%%%%%%%%%%%%%%%%%%%%%%%%
\DescribeMacro{\childdocof}
Furthermore, add the commands
\begin{center}
\begin{tabular}{l}
|\input{childdoc.def}|\\
|\childdocof{|\textit{main}|}|\\
\end{tabular}
\end{center}
at the top of every child file \textit{child}
which is included by |\include{|\textit{child}|}|
from within the main file
(or at least for those files to be compiled individually).
The argument \textit{main} must be the filename of the main file.

There are a couple of
considerations in setting up the main and child documents:

%%%%%%%%%%%%%%%%%%%%%%%%%%%%%%%%%%%%%%%%
\paragraph{Restrictions.}

Please note the following restrictions:
\begin{itemize}
\item
|\childdocmain| must be called with one argument \textit{main}
to ensure compatibility with earlier version of the package.
It must either be empty (|\childdocmain{}|)
or precisely match the filename of the main file in which it is specified.
See \secref{sec:detection} for further information.
\item
The filename \textit{main} must be specified without the |.tex| extension.
\item
The filename \textit{main} is case sensitive
(even in case-insensitive file systems)
due to internal string comparison.
\item
The argument \textit{main} should be fully expanded, it cannot be a macro.
\item
Subdirectories and special characters should be avoided in filenames.
\item
The command |\childdocmain{|\textit{main}|}| must be followed by a whitespace.
It should not be followed immediately by another command
or by a comment mark `|%|'.
This is because the \TeX{} parser reads the token immediately following
the argument of |\childdocmain| and puts it
at the beginning of every child section;
however, a white\-space is ignored.
\end{itemize}

%%%%%%%%%%%%%%%%%%%%%%%%%%%%%%%%%%%%%%%%
\paragraph{Content of Main File.}

It is advisable to place all content in the child files included by |\include|.
Any output contained in the main file will appear in all child documents
unless suppressed manually;
it cannot be suppressed automatically by the |\includeonly| directive
and thus should normally be avoided.
A method to include some content in the main file
by means of conditional processing is described in \secref{sec:conditional}.

%%%%%%%%%%%%%%%%%%%%%%%%%%%%%%%%%%%%%%%%
\paragraph{Page Numbering.}

When only a part of the document is compiled,
the appropriate numbering of pages
(as well as other status parameters)
is determined from the |.aux| files.
The latter contain information from previous passes.
However this information needs to propagate through
all intermediate child documents.
Therefore the page numbering in child documents may well
be inconsistent until the complete document is compiled at least once.

A useful (if unconventional) way to always ensure a consistent
page numbering is to restart the numbering in each child document
and denote the pages by `\textit{child}|.|\textit{page}'
where \textit{child} represents the chapter/section number of the child file.
This can be achieved by the command
|\numberwithin{page}{|\textit{child}|}|
of the \textsf{amsmath} package
where \textit{child} can be |chapter| or |section|
depending on the chosen structuring.
Alternatively, one can modify the macro |\thepage| appropriately
and reset the counter |page| at the start of each child file.

%%%%%%%%%%%%%%%%%%%%%%%%%%%%%%%%%%%%%%%%%%%%%%%%%%%%%%%%%%%%%%%%%%%%%%%%%%%%%%%%
\subsection{Conditional Processing}
\label{sec:conditional}

The package provides a mechanism to compile different versions
of a document. To customise the versions further some conditional processing
can come in handy to distinguish which version is being compiled.
The package provides two macros to describe the compilation context:

%%%%%%%%%%%%%%%%%%%%%%%%%%%%%%%%%%%%%%%%
\DescribeMacro{\ifchilddoc}
The conditional |\ifchilddoc| distinguishes between the compilation of
child documents and the main document:
%
\begin{center}
|\ifchilddoc |\textit{child-code}| |[|\||else |\textit{main-code}]| \||fi|
\end{center}

%%%%%%%%%%%%%%%%%%%%%%%%%%%%%%%%%%%%%%%%
\DescribeMacro{\childdocname}
\DescribeMacro{\childdocjob}
The macro |\childdocname| contains the filename (without extension)
of the main or child file being processed.
Note that |\childdocjob| will always contain the name of the main file.

%%%%%%%%%%%%%%%%%%%%%%%%%%%%%%%%%%%%%%%%
\paragraph{Title Page.}

Conditional processing can be used to include a title or banner page
in the main document when proper precautions are taken.
Importantly, the code in the main file should ensure that the page counter
(as well as other status parameters which are stored in the |.aux| files)
takes the same value after the conditional processing.
Otherwise the page numbers may take divergent values
depending on which part is compiled.

For example, a title page could be declared by:
%
\begin{center}
\begin{tabular}{l}
|\ifchilddoc\||else|\\
|\addtocounter{page}{-1}|\\
\textit{code for title page}\\
|\newpage|\\
|\||fi|
\end{tabular}
\end{center}
%
A banner page for the child documents can be generated by:
%
\begin{center}
\begin{tabular}{l}
|\ifchilddoc|\\
|\addtocounter{page}{-1}|\\
\textit{code for banner page}\\
|\newpage|\\
|\||fi|
\end{tabular}
\end{center}
%
Here one could write a message such as:
\begin{center}
|This is the part \childdocname{} of \childdocjob{}.|
\end{center}

%%%%%%%%%%%%%%%%%%%%%%%%%%%%%%%%%%%%%%%%%%%%%%%%%%%%%%%%%%%%%%%%%%%%%%%%%%%%%%%%
\subsection{Flags}
\label{sec:flags}

The package makes it easy to generate different versions
of the main or child documents.
To this end compilation flags can be defined
and assigned different default values.
They will be particularly useful in conjunction
with the forwarding mechanism described in \secref{sec:forward}.

For example, it may be useful to have a flag |\version|
which can be set to |draft| or |final|.
The document source will contain some conditional code
depending on the value of |\version|.
Suppose further, the flag should default to |final| for the main file
and to |draft| for child files
which is a natural assignment for editing the document.
This is achieved by placing the following code
in the preamble of the main document
(below the |\childdocmain| directive):
%
\begin{center}
\begin{tabular}{l}
|\ifchilddoc|\\
|\providecommand{\version}{draft}|\\
|\||else|\\
|\providecommand{\version}{final}|\\
|\||fi|
\end{tabular}
\end{center}
%
The definition by |\providecommand| makes sure
that previous definitions are not overwritten.
Further statements |\providecommand{\version}{...}|
can thus be added before the above code to override it.

For the main file, one might add a line
(between |\childdocmain| and the above block)
%
\begin{center}
|%\ifchilddoc\||else\providecommand{\version}{draft}\||fi|
\end{center}
%
which can be uncommented to produce a draft version.
Likewise one can add a line to the very top of a child file
(above the |\childdocof{|\textit{main}|}| directive)
%
\begin{center}
|%\providecommand{\version}{final}|
\end{center}
%
which can be uncommented to produce the final version of this child document.

%%%%%%%%%%%%%%%%%%%%%%%%%%%%%%%%%%%%%%%%%%%%%%%%%%%%%%%%%%%%%%%%%%%%%%%%%%%%%%%%
\subsection{Forwarding}
\label{sec:forward}

Different versions of the main or child documents
using compilation flags as described in \secref{sec:flags}
can be (permanently) stored in different files
for convenient compilation, viewing and distribution.
To this end, the package defines a command
to pass on compilation to a different file:

%%%%%%%%%%%%%%%%%%%%%%%%%%%%%%%%%%%%%%%%
\DescribeMacro{\childdocforward}
The command |\childdocforward| redirects processing to
another source file:
%
\begin{center}
\begin{tabular}{l}
|\input{childdoc.def}|\\
|\childdocforward[|\textit{main}|]{|\textit{dest}|}|\\
\end{tabular}
\end{center}
%
The argument \textit{dest} is the destination file
(without extension).
It should be the main file or one of the child files.
Note that further \textsf{childdoc} directives
such as |\childdocof| and |\childdocforward|
in the indicated file will be processed in this form.
The optional argument \textit{main}
passes on directly to the main file \textit{main}
while pretending to compile the child \textit{dest}.
This form behaves as if \textit{dest}
issues |\childdocof{|\textit{main}|}| right away,
and no further \textsf{childdoc} directives will be processed.

%%%%%%%%%%%%%%%%%%%%%%%%%%%%%%%%%%%%%%%%
\DescribeMacro{\...prefix}
In the alternative form |\childdocforwardprefix|,
%
\begin{center}
\begin{tabular}{l}
|\input{childdoc.def}|\\
|\childdocforwardprefix[|\textit{main}|]{|\textit{prefix}|}{|\textit{dest}|}|
\end{tabular}
\end{center}
%
the destination file is determined by a pattern
depending on the current file:
To make this work, the current file must be called
`{\textit{prefix}\hspace{0.2em}\textit{suffix}}'
with \textit{prefix} matching precisely the argument.
Processing is then passed on to the file
`{\textit{dest}\hspace{0.2em}\textit{suffix}}'.
Surely, the same effect is achieved by
directly specifying the
argument `{\textit{dest}\hspace{0.2em}\textit{suffix}}'
in the first form.
However, that requires to set up a different file
for each child. With the alternative form of the command
all these files can have exactly the same content
which simplifies setting them up and maintaining them.

For example, the following file |draft.tex|
with a compilation flag |\version| as described in \secref{sec:flags}
compiles the main document as a draft:
%
\begin{center}
\begin{tabular}{l}
|\def\version{draft}|\\
|\input{childdoc.def}|\\
|\childdocforward{|\textit{main}|}|
\end{tabular}
\end{center}
%
Likewise, the following files |final|\textit{nn}|.tex|
compile the final version of the child document
|child|\textit{nn}|.tex|:
%
\begin{center}
\begin{tabular}{l}
|\def\version{final}|\\
|\input{childdoc.def}|\\
|\childdocforwardprefix{final}{child}|
\end{tabular}
\end{center}
%

Note that when several versions of a main file and/or of each child file
are to be generated, it may be convenient to set up a |Makefile| or
shell script to automatise the process.

%%%%%%%%%%%%%%%%%%%%%%%%%%%%%%%%%%%%%%%%%%%%%%%%%%%%%%%%%%%%%%%%%%%%%%%%%%%%%%%%
\subsection{Command Line Processing}
\label{sec:commandline}

The effect of redirection files can also be achieved by invoking
the \LaTeX{} compiler with a more elaborate command line.
Most conveniently this should be done as part
of a shell script or a |Makefile|.

When using \textsf{childdoc} in the main file, the following
command lines effectively perform a redirection
(note that depending on the shell being used,
backslashes may have to be doubled: `|\|' $\to$ `|\\|'):
%
\begin{center}
|... -jobname "|\textit{target}|" |\\|"|[\textit{flags}]%
|\input{childdoc.def}\childdocforward[|\textit{main}|]{|\textit{dest}|}"|
\end{center}
%
Here \textit{target} is the name of the output file,
\textit{main} is the name of the main file
and \textit{dest} is the name of the main or child file to be processed
(all filenames without extensions).
The optional argument \textit{main} can be omitted
if \textit{main} matches \textit{dest}.
Optionally, compilation \textit{flags} can be defined via |\def| commands.
This command line makes the \TeX{} engine believe
it is compiling the file \textit{target}
whose content is specified as the latter parameter.
The provided code then forwards the processing to
\textit{main} or \textit{dest} as described in \secref{sec:forward}.

%%%%%%%%%%%%%%%%%%%%%%%%%%%%%%%%%%%%%%%%%%%%%%%%%%%%%%%%%%%%%%%%%%%%%%%%%%%%%%%%
\subsection{Include by Input}
\label{sec:input}

Including child documents by |\include| has some restrictions by design.
Most notably, the content of a child document always occupies
its own set of pages; pages cannot be shared between child documents.
Usually, this behaviour makes perfect sense
because each child document contain an essential part of the document.
However, in some situations it may be desirable to compose
a document from a collection of parts
without having mandatory page breaks between then.
For this case, the package
provides a mechanism to include parts
by |\input| which can also be processed individually.
However, by construction this mechanism
requires manual handling of the content to be output.

%%%%%%%%%%%%%%%%%%%%%%%%%%%%%%%%%%%%%%%%
\DescribeMacro{\ifchilddocmanual}
The main file should be prepared as usual, see \secref{sec:include}.
However, the document body must make a distinction
between processing of an individual part and of the main document, e.g.:
%
\begin{center}
\begin{tabular}{l}
|\ifchilddocmanual|\\
|\input{\childdocname}|\\
|\||else|\\
\textit{document body with }|\input{|\textit{part}|}|\\
|\||fi|
\end{tabular}
\end{center}
%
The conditional |\ifchilddocmanual| is true whenever
a part to be included by |\input| is being compiled,
and the name of the part is stored in |\childdocname|.

%%%%%%%%%%%%%%%%%%%%%%%%%%%%%%%%%%%%%%%%
\DescribeMacro{\childdocby}
Each part to be included by |\input| should start with:
%
\begin{center}
\begin{tabular}{l}
|\input{childdoc.def}|\\
|\childdocby{|\textit{main}|}|\\
\end{tabular}
\end{center}
%
The directive |\childdocby| is similar to |\childdocof|
described in \secref{sec:include},
but the subsequent selection of content must be done manually.
To that end, both |\ifchilddoc| and |\ifchilddocmanual|
will be true upon processing of a part,
and the name of the part is stored in |\childdocname|.
Note that |\jobname| will be set to the filename of the current part
so that each part receives an individual |.aux| file
that does not interfere with the |.aux| file(s) of the main document.
This behaviour can be altered by the alternative form
|\childdocby[*]{|\textit{main}|}| (with a non-empty optional argument)
which uses the |.aux| file of the main document
by setting |\jobname| to \textit{main}.

%%%%%%%%%%%%%%%%%%%%%%%%%%%%%%%%%%%%%%%%%%%%%%%%%%%%%%%%%%%%%%%%%%%%%%%%%%%%%%%%
\subsection{Driver Development}
\label{sec:driver}

The \textsf{childdoc} mechanism can also be use for the development
of definition files such as \LaTeX{} styles or classes.
This case differs from the above setup with multiple parts
included by |\include| in that no |\includeonly| should be invoked.
This can be achieved by starting the include file
(before |\ProvidesPackage|) with:
%
\begin{center}
\begin{tabular}{l}
|\input{childdoc.def}|\\
|\childdocforward{|\textit{main}|}|\\
\end{tabular}
\end{center}
%
or alternatively with:
%
\begin{center}
\begin{tabular}{l}
|\input{childdoc.def}|\\
|\childdocby{|\textit{main}|}|\\
\end{tabular}
\end{center}
%
Both forms have slightly different effects as described above.
The main file is prepared as usual, see \secref{sec:include}.

%%%%%%%%%%%%%%%%%%%%%%%%%%%%%%%%%%%%%%%%%%%%%%%%%%%%%%%%%%%%%%%%%%%%%%%%%%%%%%%%
\subsection{Legacy Detection}
\label{sec:detection}

The directive |\childdocmain| in the main file can detect
whether the complete document or merely a child is to be compiled
even without using the directive |\childdocof|.
This method is deprecated because it is less robust
and there is no compelling reason to use it;
it is merely provided for backward compatibility
and it may be removed in future versions.

If the detection mechanism is to be used,
it is mandatory to correctly specify
the filename of the main file as the argument of |\childdocmain|:
%
\begin{center}
\begin{tabular}{l}
|\input{childdoc.def}|\\
|\childdocmain{|\textit{main}|}|\\
\end{tabular}
\end{center}
%
If |\jobname| does not match the argument \textit{main} of |\childdocmain|,
it is assumed that |\jobname| points to the child file to be compiled.
When using |\childdocmain| with the main file specified as argument,
it suffices to start a child file
with just |\input{|\textit{main}|}|
without loading of the package and using |\childdocof|.
If instead all processing is done
with the appropriate \textsf{childdoc} directives,
the argument of \textit{main} of |\childdocmain| can be empty.

An alternative version of the command line processing described
in \secref{sec:commandline} using the detection mechanism reads:
%
\begin{center}
|... -jobname "|\textit{target}|" "|[\textit{flags}]%
[|\def\jobname{|\textit{dest}|}|]|\input{|\textit{main}|}"|
\end{center}

%%%%%%%%%%%%%%%%%%%%%%%%%%%%%%%%%%%%%%%%%%%%%%%%%%%%%%%%%%%%%%%%%%%%%%%%%%%%%%%%
\subsection{Manual Code}
\label{sec:manual}

In case one cannot be certain whether the definitions file |childdoc.def|
is installed on the target \TeX{} distribution
and one prefers not to ship it,
it is conceivable to paste a few relevant commands into the sources.

To that end, drop all statements |\input{childdoc.def}|
and perform the replacements as outlined below.
Instead of |\childdocmain{|\textit{main}|}| add the following code
to the top of the main file:
%
\begin{center}
\begin{tabular}{l}
|\||ifdefined\childdocname\endinput\||fi\newif\ifchilddoc|\\
|\edef\childdocname{\scantokens\expandafter{\jobname\noexpand}}|\\
|\def\childdocmain{|\textit{main}|}\||ifx\childdocmain\childdocname\||else|\\
|\childdoctrue\includeonly{\childdocname}\let\jobname\childdocmain\||fi|\\
\end{tabular}
\end{center}
%
Instead of |\childdocof{|\textit{main}|}| just include the main file
at the top of each child file:
%
\begin{center}
|\input{|\textit{main}|}|
\end{center}
%
A simple redirection |\childdocforward{|\textit{dest}|}| is achieved by:
%
\begin{center}
|\def\jobname{|\textit{dest}|}\input{\jobname}|
\end{center}
%
The redirection with prefix
|\childdocforwardprefix[|\textit{prefix}|]{|\textit{dest}|}|
is accomplished by:
%
\begin{center}
\begin{tabular}{l}
|{\edef\jobname{\scantokens\expandafter{\jobname\noexpand}}|\\
|\def\redirectjob |\textit{prefix}|#1~~~{\gdef\jobname{|\textit{dest}|#1}}|\\
|\expandafter\redirectjob\jobname~~~}\input{\jobname}|
\end{tabular}
\end{center}

In an alternative approach,
child documents can be compiled by a specific command line
without additional code or specific definitions:
%
\begin{center}
|... -jobname "|\textit{target}|" "|[\textit{flags}]%
|\includeonly{|\textit{dest}|}\input{|\textit{main}|}"|
\end{center}
%

%%%%%%%%%%%%%%%%%%%%%%%%%%%%%%%%%%%%%%%%%%%%%%%%%%%%%%%%%%%%%%%%%%%%%%%%%%%%%%%%
%%%%%%%%%%%%%%%%%%%%%%%%%%%%%%%%%%%%%%%%%%%%%%%%%%%%%%%%%%%%%%%%%%%%%%%%%%%%%%%%
\section{Information}

%%%%%%%%%%%%%%%%%%%%%%%%%%%%%%%%%%%%%%%%%%%%%%%%%%%%%%%%%%%%%%%%%%%%%%%%%%%%%%%%
\subsection{Copyright}

Copyright \copyright{} 2017--2018 Niklas Beisert

This work may be distributed and/or modified under the
conditions of the \LaTeX{} Project Public License, either version 1.3
of this license or (at your option) any later version.
The latest version of this license is in
  \url{http://www.latex-project.org/lppl.txt}
and version 1.3 or later is part of all distributions of \LaTeX{}
version 2005/12/01 or later.

This work has the LPPL maintenance status `maintained'.

The Current Maintainer of this work is Niklas Beisert.

This work consists of the files |README.txt|, |childdoc.ins| and |childdoc.dtx|
as well as the derived files |childdoc.def|, |cdocsamp.tex|
with |cdocsch1.tex|, |cdocsch2.tex|, |cdocspt3.tex|, |cdocspt4.tex|,
|cdocsdrf.tex|, |cdocsfn1.tex|, |cdocsfn2.tex|
as well as |childdoc.pdf|.

%%%%%%%%%%%%%%%%%%%%%%%%%%%%%%%%%%%%%%%%%%%%%%%%%%%%%%%%%%%%%%%%%%%%%%%%%%%%%%%%
\subsection{Files and Installation}

The package consists of the files:
%
\begin{center}
\begin{tabular}{ll}
    |README.txt|   & readme file \\
    |childdoc.ins| & installation file \\
    |childdoc.dtx| & source file \\
    |childdoc.def| & definition file \\
    |cdocsamp.tex| & sample main file \\
    |cdocsch1.tex| & sample include file \\
    |cdocsch2.tex| & sample include file \\
    |cdocspt3.tex| & sample part file \\
    |cdocspt4.tex| & sample part file \\
    |cdocsdrf.tex| & sample redirection file \\
    |cdocsfn1.tex| & sample redirection file \\
    |cdocsfn2.tex| & sample redirection file \\
    |childdoc.pdf| & manual
\end{tabular}
\end{center}
%
The distribution consists of the files
|README.txt|, |childdoc.ins| and |childdoc.dtx|.
%
\begin{itemize}
\item
Run (pdf)\LaTeX{} on |childdoc.dtx|
to compile the manual |childdoc.pdf| (this file).
\item
Run \LaTeX{} on |childdoc.ins| to create the definitions file |childdoc.def|
and the sample |cdocsamp.tex| with include files
|cdocsch1.tex|, |cdocsch2.tex|, |cdocspt3.tex|, |cdocspt4.tex|,
|cdocsdrf.tex|, |cdocsfn1.tex|, |cdocsfn2.tex|.
Then copy the file |childdoc.def| to an appropriate directory of your \LaTeX{}
distribution, e.g.\ \textit{texmf-root}|/tex/latex/childdoc|.
\end{itemize}

%%%%%%%%%%%%%%%%%%%%%%%%%%%%%%%%%%%%%%%%%%%%%%%%%%%%%%%%%%%%%%%%%%%%%%%%%%%%%%%%
\subsection{Related CTAN Packages}

There are several other packages which offer a similar functionality:
%
\begin{itemize}
\item
The packages
\href{http://ctan.org/pkg/docmute}{\textsf{docmute}},
\href{http://ctan.org/pkg/includex}{\textsf{includex}} and
\href{http://ctan.org/pkg/standalone}{\textsf{standalone}}
provide commands to include only the document body of
a child file thus allowing both files to be compiled individually.
\item
The packages \href{http://ctan.org/pkg/subdocs}{\textsf{subdocs}}
and \href{http://ctan.org/pkg/subfiles}{\textsf{subfiles}}
provide structures in which the main and child documents can be
encapsulated and allowing them to be compiled individually.
The inclusion mechanism is different from the conventional |\include|.
\item
The package \href{http://ctan.org/pkg/combine}{\textsf{combine}}
is an elaborate solution to combine several documents into one.
\end{itemize}
%
See also the CTAN topic \href{http://ctan.org/topic/subdocs}{\textsf{subdocs}}
for further related packages.
The present package differs from the above solutions in that
a document structure constructed with the conventional |\include| mechanism
just needs two extra commands at the top of every file
such that all constituent files can be compiled individually.

%%%%%%%%%%%%%%%%%%%%%%%%%%%%%%%%%%%%%%%%%%%%%%%%%%%%%%%%%%%%%%%%%%%%%%%%%%%%%%%%
%\subsection{Feature Suggestions}
%
%The following is a list of features which may be useful for future
%versions of this package:
%%
%\begin{itemize}
%\item
%\ldots
%\end{itemize}

%%%%%%%%%%%%%%%%%%%%%%%%%%%%%%%%%%%%%%%%%%%%%%%%%%%%%%%%%%%%%%%%%%%%%%%%%%%%%%%%
\subsection{Revision History}

%%%%%%%%%%%%%%%%%%%%%%%%%%%%%%%%%%%%%%%%
\paragraph{v2.0:} 2018/12/30

\begin{itemize}
\item
immediate forward processing
\item
added |\childdocby| mechanism
\item
manual restructured
\end{itemize}

%%%%%%%%%%%%%%%%%%%%%%%%%%%%%%%%%%%%%%%%
\paragraph{v1.6:} 2018/01/17

\begin{itemize}
\item
application for development of include files
\item
corrections to manual
\end{itemize}

%%%%%%%%%%%%%%%%%%%%%%%%%%%%%%%%%%%%%%%%
\paragraph{v1.5:} 2017/05/21

\begin{itemize}
\item
more complete structuring introduced
\item
|\childdocof| introduced
\item
|\childdoc| renamed to |\childdocmain|
\item
|\childredirect| renamed to |\childdocforward| and |\childdocforwardprefix|
and functionality expanded
\end{itemize}

%%%%%%%%%%%%%%%%%%%%%%%%%%%%%%%%%%%%%%%%
\paragraph{v1.0:} 2017/04/27

\begin{itemize}
\item
manual and install package
\item
first version published on CTAN
\end{itemize}

%%%%%%%%%%%%%%%%%%%%%%%%%%%%%%%%%%%%%%%%
\paragraph{v0.6:} 2017/04/26

\begin{itemize}
\item
redirection mechanism added
\end{itemize}

%%%%%%%%%%%%%%%%%%%%%%%%%%%%%%%%%%%%%%%%
\paragraph{v0.5:} 2017/04/26

\begin{itemize}
\item
functionality in definition file
\end{itemize}


%%%%%%%%%%%%%%%%%%%%%%%%%%%%%%%%%%%%%%%%%%%%%%%%%%%%%%%%%%%%%%%%%%%%%%%%%%%%%%%%
%%%%%%%%%%%%%%%%%%%%%%%%%%%%%%%%%%%%%%%%%%%%%%%%%%%%%%%%%%%%%%%%%%%%%%%%%%%%%%%%
%%%%%%%%%%%%%%%%%%%%%%%%%%%%%%%%%%%%%%%%%%%%%%%%%%%%%%%%%%%%%%%%%%%%%%%%%%%%%%%%
\appendix

\settowidth\MacroIndent{\rmfamily\scriptsize 000\ }

 \DocInput{childdoc.dtx}

\end{document}
%</driver>
% \fi
%
% %%%%%%%%%%%%%%%%%%%%%%%%%%%%%%%%%%%%%%%%%%%%%%%%%%%%%%%%%%%%%%%%%%%%%%%%%%%%%%
% %%%%%%%%%%%%%%%%%%%%%%%%%%%%%%%%%%%%%%%%%%%%%%%%%%%%%%%%%%%%%%%%%%%%%%%%%%%%%%
% \section{Sample}
%\iffalse
%<*samplemain>
%\fi
%
% The following presents a sample document
% with two chapters, two parts, a title page,
% a compile flag as well as three forwarding files to set the flag.
% It consists of eight |.tex| files:
% \begin{center}
% \begin{tabular}{ll}
% |cdocsamp.tex|&main file\\
% |cdocsch1.tex|&include file for chapter 1\\
% |cdocsch2.tex|&include file for chapter 2\\
% |cdocspt3.tex|&include file for part 3\\
% |cdocspt4.tex|&include file for part 4\\
% |cdocsdrf.tex|&forwarding file for main file in draft mode\\
% |cdocsfi1.tex|&forwarding file for final version of chapter 1\\
% |cdocsfi2.tex|&forwarding file for final version of chapter 2\\
% \end{tabular}
% \end{center}
% Each of the eight files can be compiled directly by the \LaTeX{} compiler.
%
% %%%%%%%%%%%%%%%%%%%%%%%%%%%%%%%%%%%%%%
% \paragraph{Main File.}
%
% The main file is called |cdocsamp.tex|.
%
% Load the \textsf{childdoc} definitions and
% declare the filename for the main document:
%    \begin{macrocode}
\input{childdoc.def}
\childdocmain{}
%    \end{macrocode}

% Optional override for |\version| flag:
%    \begin{macrocode}
%%\ifchilddoc\else\providecommand{\version}{draft}\fi
%    \end{macrocode}

% Define the default values for the |\version| flag
% (|final| for the main file and |draft| for childs):
%    \begin{macrocode}
\ifchilddoc
\providecommand{\version}{draft}
\else
\providecommand{\version}{final}
\fi
%    \end{macrocode}

% Load the standard document class:
%    \begin{macrocode}
\documentclass[12pt]{article}
%    \end{macrocode}

% Start the document body:
%    \begin{macrocode}
\begin{document}
%    \end{macrocode}

% Declare a title page.
% Print title, part of document being processed and version flag:
%    \begin{macrocode}
\addtocounter{page}{-1}
\begin{center}
{\LARGE\bfseries{}childdoc example\par}
\vspace{1cm}
\ifchilddoc
\ifchilddocmanual part\else chapter\fi:
`\childdocname' of `\childdocjob'\par
\else
main document: `\childdocjob'\par
\fi
version: \version\par
\end{center}
\newpage
%    \end{macrocode}

% Manually include selected file,
% otherwise process as usual:
%    \begin{macrocode}
\ifchilddocmanual
\section*{part `\childdocname'}
\input{\childdocname}
\else
%    \end{macrocode}

% Include the two chapters:
%    \begin{macrocode}
\include{cdocsch1}
\include{cdocsch2}
%    \end{macrocode}

% Include the two parts unless only chapters should be displayed:
%    \begin{macrocode}
\ifchilddoc\else
\section{part three}
\input{cdocspt3}
\section{part four}
\input{cdocspt4}
\fi
%    \end{macrocode}

% Process as usual until here:
%    \begin{macrocode}
\fi
%    \end{macrocode}

% End of document body:
%    \begin{macrocode}
\end{document}
%    \end{macrocode}
%\iffalse
%</samplemain>
%\fi
%
% %%%%%%%%%%%%%%%%%%%%%%%%%%%%%%%%%%%%%%
% \paragraph{Chapter Include Files.}
%
% The include files are called |cdocsch1.tex| and |cdocsch2.tex|.
%
%\iffalse
%<*samplechap1|samplechap2>
%\fi

% Optional override for |\version| flag:
%    \begin{macrocode}
%%\providecommand{\version}{final}
%    \end{macrocode}

% Include the main document:
%    \begin{macrocode}
\input{childdoc.def}
\childdocof{cdocsamp}
%    \end{macrocode}

%\iffalse
%</samplechap1|samplechap2>
%\fi
%
%\iffalse
%<*samplechap1>
%\fi
% Some text for chapter 1:
%    \begin{macrocode}
\section{one}
some text in chapter one
%    \end{macrocode}

%\iffalse
%</samplechap1>
%\fi
% Some text for chapter 2:
%\iffalse
%<*samplechap2>
%\fi
%    \begin{macrocode}
\section{two}
more text in chapter two
%    \end{macrocode}

%\iffalse
%</samplechap2>
%\fi
%
% %%%%%%%%%%%%%%%%%%%%%%%%%%%%%%%%%%%%%%
% \paragraph{Part Include Files.}
%
% The include files are called |cdocspt3.tex| and |cdocspt4.tex|.
%
%\iffalse
%<*samplepart3|samplepart4>
%\fi

% Optional override for |\version| flag:
%    \begin{macrocode}
%%\providecommand{\version}{final}
%    \end{macrocode}

% Include the main document:
%    \begin{macrocode}
\input{childdoc.def}
\childdocby{cdocsamp}
%    \end{macrocode}

%\iffalse
%</samplepart3|samplepart4>
%\fi
%
%\iffalse
%<*samplepart3>
%\fi
% Some text for part 3:
%    \begin{macrocode}
some text in part three
%    \end{macrocode}

%\iffalse
%</samplepart3>
%\fi
% Some text for part 4:
%\iffalse
%<*samplepart4>
%\fi
%    \begin{macrocode}
more text in part four
%    \end{macrocode}

%\iffalse
%</samplepart4>
%\fi
%
% %%%%%%%%%%%%%%%%%%%%%%%%%%%%%%%%%%%%%%
% \paragraph{Forwarding for a Complete Draft.}
%
% The following forwarding file |cdocsdrf.tex|
% compiles the main document in draft mode:
%\iffalse
%<*sampledraft>
%\fi
%    \begin{macrocode}
\def\version{draft}
\input{childdoc.def}
\childdocforward{cdocsamp}
%    \end{macrocode}

%\iffalse
%</sampledraft>
%\fi
%
% %%%%%%%%%%%%%%%%%%%%%%%%%%%%%%%%%%%%%%
% \paragraph{Forwarding for Final Version of the Chapters.}
%
% The following forwarding files |cdocsfn1.tex| and |cdocsfn2.tex|
% (with identical content)
% compile the final versions of the child documents
% |cdocsch1.tex| and |cdocsch2.tex|, respectively:
%\iffalse
%<*samplefinal>
%\fi
%    \begin{macrocode}
\def\version{final}
\input{childdoc.def}
\childdocforwardprefix[cdocsamp]{cdocsfn}{cdocsch}
%    \end{macrocode}

%\iffalse
%</samplefinal>
%\fi
%
% %%%%%%%%%%%%%%%%%%%%%%%%%%%%%%%%%%%%%%
% \paragraph{Command Line Processing.}
%
% The following three command lines generate the output files
% |cdocscld|, |cdocscl1| and |cdocscl2|
% which should be identical to
% |cdocsdrf|, |cdocsch1| and |cdocsfn2|, respectively:
% \begin{center}
% \begin{tabular}{l}
% |latex -jobname cdocscld \|\\
% |  "\def\version{draft}\input{childdoc.def}\childdocforward{cdocsamp}"|\\
% |latex -jobname cdocscl1 \|\\
% |  "\input{childdoc.def}\childdocforward[cdocsamp]{cdocsch1}"|\\
% |latex -jobname cdocscl2 \|\\
% |  "\def\version{final}\input{childdoc.def}\childdocforward{cdocsch2}"|
% \end{tabular}
% \end{center}
% Note that the trailing backslash on each first line
% merely continues the input to the second line
% (for convenient cut ant paste).
% Furthermore, the command |latex| can be replaced by any
% of its alternative versions such as |pdflatex|.
%
% %%%%%%%%%%%%%%%%%%%%%%%%%%%%%%%%%%%%%%%%%%%%%%%%%%%%%%%%%%%%%%%%%%%%%%%%%%%%%%
% %%%%%%%%%%%%%%%%%%%%%%%%%%%%%%%%%%%%%%%%%%%%%%%%%%%%%%%%%%%%%%%%%%%%%%%%%%%%%%
% \section{Implementation}
%\iffalse
%<*package>
%\fi
%
% This section describes the definitions file |childdoc.def|.

% The definitions cannot be loaded using |\usepackage| or |\RequirePackage|
% which has a mechanism to prevent loading a style file more than once.
% When loading the definitions by means of |\input|
% multiple instances have to be prevented manually:
%\iffalse
%This code needs to be before the `\ProvidesFile' directive
%which is defined at the beginning of this file.
%Therefore it is also placed there and commented out here.
%</package>
%<*discard>
%\fi
%    \begin{macrocode}
\ifdefined\childdocmain\endinput\fi
%    \end{macrocode}
%\iffalse
%</discard>
%<*package>
%\fi
%
% \macro{\ifchilddoc}
% \macro{\ifchilddocmanual}
% The conditional |\ifchilddoc| tells whether a
% child (true) or main (false) document is being compiled.
% The conditional |\ifchilddocmanual| tells whether
% the |\includeonly| mechanism is used (false) or
% the selection of child files must be performed manually (true).
% The definitions initialise to false:
%    \begin{macrocode}
\newif\ifchilddoc
\newif\ifchilddocmanual
%    \end{macrocode}

% \macro{\childdocname}
% \macro{\childdocjob}
% The macro |\childdocname| stores the name of the main document
% to be compiled. The macro |\childdocjob| stores the name of
% the document on which the \LaTeX{} compiler was originally invoked.
% The content of |\jobname| cannot be compared
% to filenames specified in the source due to different catcodes.
% The following code rescans |\jobname|, stores the result
% in |\childdocname| and saves a copy in |\childdocjob|:
%    \begin{macrocode}
\edef\childdocname{\scantokens\expandafter{\jobname\noexpand}}
\let\childdocjob\childdocname
%    \end{macrocode}

% \macro{\childdocdisable}
% The macro |\childdocdisable| prevents the main file
% from being processed more than once.
% At this stage, the main document command |\childdocmain|
% is assumed to be called once again where it should do nothing.
% Any subsequent call to it should prevent
% a secondary processing of the main document
% It overwrites the forwarding commands
% |\childdocof| and |\childdocforward|
% with empty macros to prevent further inclusions of the main document:
%    \begin{macrocode}
\newcommand{\childdocdisable}
{
  \renewcommand{\childdocmain}[1]{\renewcommand{\childdocmain}[1]{\endinput}}
  \renewcommand{\childdocof}[1]{}
  \renewcommand{\childdocby}[2][]{}
  \renewcommand{\childdocforward}[2][]{}
  \renewcommand{\childdocdisable}{}
}
%    \end{macrocode}

% \macro{\childdocmain}
% The macro |\childdocmain| is to be called at the top of the main file
% with nothing or the main filename (without extension) as argument.
% First, it breaks loops.
% If the argument is not empty and does not match |\childdocname|
% (which is set by the first inclusion of |childdoc.def|),
% |\ifchilddoc| is set to true, |\includeonly| is applied to the child file
% and |\jobname| is set to the main file
% (for proper handling of |.aux| files):
%    \begin{macrocode}
\newcommand{\childdocmain}[1]
{
  \childdocdisable\childdocmain{}
  \if?#1?\else
    \begingroup
      \def\childdoctmp{#1}
      \ifx\childdoctmp\childdocname
        \def\childdoctmp{}
      \else
        \def\childdoctmp
        {
          \childdoctrue
          \includeonly{\childdocname}
          \def\childdocjob{#1}
          \def\jobname{#1}
        }
      \fi
      \expandafter
    \endgroup
    \childdoctmp
  \fi
}
%    \end{macrocode}

% \macro{\childdocof}
% The command |\childdocof| redirects
% compilation to the main file |#1|.
%    \begin{macrocode}
\newcommand{\childdocof}[1]
{
  \childdocdisable
  \childdoctrue
  \includeonly{\childdocname}
  \def\jobname{#1}
  \def\childdocjob{#1}
  \input{#1}
}
%    \end{macrocode}

% \macro{\childdocby}
% The command |\childdocby| ....
%    \begin{macrocode}
\newcommand{\childdocby}[2][]
{
  \childdocdisable
  \childdoctrue
  \childdocmanualtrue
  \if?#1?\else
    \def\jobname{#2}
  \fi
  \def\childdocjob{#2}
  \input{#2}
  \endinput
}
%    \end{macrocode}

% \macro{\childdocforward}
% The command |\childdocforward| redirects
% compilation to the main file or
% (if the optional argument is given) a child file.
% Parameters are set as if the main file
% or a child file starting with |\childdocof| was compiled.
% Then compilation is handed over to the main file:
%    \begin{macrocode}
\newcommand{\childdocforward}[2][]
{
  \begingroup
    \if?#1?
      \def\childdoctmp
      {
        \def\childdocname{#2}
        \def\childdocjob{#2}
        \def\jobname{#2}
        \input{#2}
        \endinput
      }
    \else
      \def\childdoctmp
      {
        \childdocdisable
        \def\childdocname{#2}
        \childdoctrue
        \includeonly{#2}
        \def\childdocjob{#1}
        \def\jobname{#1}
        \input{#1}
        \endinput
      }
    \fi
    \expandafter
  \endgroup
  \childdoctmp
}
%    \end{macrocode}

% \macro{\childdocforwardprefix}
% The command |\childdocforwardprefix| redirects
% compilation to the main or a child file by means of a pattern.
% The prefix |#1| in the current filename is replaced by |#2|
% and the suffix of the current filename is kept
% (it is assumed that the filename does not contain the substring `|~~~|'
% which is used as a delimiter).
% Compilation is handed over to the new file by |\childdocforward|:
%    \begin{macrocode}
\newcommand{\childdocforwardprefix}[3][]
{
  \begingroup
    \def\childdocextract #2##1~~~{\def\childdoctmp{\childdocforward[#1]{#3##1}}}
    \expandafter\childdocextract\childdocname~~~
    \expandafter
  \endgroup
  \childdoctmp
}
%    \end{macrocode}

% \macro{\childdoc}
% The deprecated macro |\childdoc| is a legacy version of |\childdocmain|:
%    \begin{macrocode}
\newcommand{\childdoc}{\childdocmain}
%    \end{macrocode}

% \macro{\childdocredirect}
% The deprecated macro |\childdocredirect| is a legacy version
% of |\childdocforward| and |\childdocforwardprefix|:
%    \begin{macrocode}
\newcommand{\childdocredirect}[2][]
{
  \begingroup
    \if?#1?
      \def\childdoctmp{\childdocforward{#2}}
    \else
      \def\childdoctmp{\childdocforwardprefix{#1}{#2}}
    \fi
    \expandafter
  \endgroup
  \childdoctmp
}
%    \end{macrocode}

%\iffalse
%</package>
%\fi
%
\endinput
|\\
|\childdocforward{|\textit{main}|}|
\end{tabular}
\end{center}
%
Likewise, the following files |final|\textit{nn}|.tex|
compile the final version of the child document
|child|\textit{nn}|.tex|:
%
\begin{center}
\begin{tabular}{l}
|\def\version{final}|\\
|% \iffalse
%
% childdoc.dtx Copyright (C) 2017-2018 Niklas Beisert
%
% This work may be distributed and/or modified under the
% conditions of the LaTeX Project Public License, either version 1.3
% of this license or (at your option) any later version.
% The latest version of this license is in
%   http://www.latex-project.org/lppl.txt
% and version 1.3 or later is part of all distributions of LaTeX
% version 2005/12/01 or later.
%
% This work has the LPPL maintenance status `maintained'.
%
% The Current Maintainer of this work is Niklas Beisert.
%
% This work consists of the files childdoc.dtx and childdoc.ins
% and the derived files childdoc.def and cdocsamp.tex with
% cdocsch1.tex, cdocsch2.tex, cdocsdrf.tex, cdocsfn1.tex, cdocsfn2.tex.
%
%<package>\ifdefined\childdocmain\endinput\fi
%<package>\ProvidesFile{childdoc.def}[2018/12/30 v2.0 child document driver]
%<samplemain>\ProvidesFile{cdocsamp.tex}[2018/12/30 v2.0 sample for childdoc]
%<*driver>
%\ProvidesFile{childdoc.drv}[2018/12/30 v2.0 childdoc reference manual file]
\PassOptionsToClass{10pt,a4paper}{article}
\documentclass{ltxdoc}

\usepackage[margin=35mm]{geometry}
\usepackage{hyperref}
\usepackage{hyperxmp}
\usepackage[usenames]{color}

\hypersetup{colorlinks=true}
\hypersetup{pdfstartview=FitH}
\hypersetup{pdfpagemode=UseNone}
\hypersetup{pdfsource={}}
\hypersetup{pdflang={en-UK}}
\hypersetup{pdfcopyright={Copyright 2017-2018 Niklas Beisert.
  This work may be distributed and/or modified under the
  conditions of the LaTeX Project Public License, either version 1.3
  of this license or (at your option) any later version.}}
\hypersetup{pdflicenseurl={http://www.latex-project.org/lppl.txt}}
\hypersetup{pdfcontactaddress={ETH Zurich, ITP, HIT K,
  Wolfgang-Pauli-Strasse 27}}
\hypersetup{pdfcontactpostcode={8093}}
\hypersetup{pdfcontactcity={Zurich}}
\hypersetup{pdfcontactcountry={Switzerland}}
\hypersetup{pdfcontactemail={nbeisert@itp.phys.ethz.ch}}
\hypersetup{pdfcontacturl={http://people.phys.ethz.ch/\xmptilde nbeisert/}}

\newcommand{\secref}[1]{\hyperref[#1]{section \ref*{#1}}}

\parskip1ex
\parindent0pt
\let\olditemize\itemize
\def\itemize{\olditemize\parskip0pt}

\begin{document}

\title{The \textsf{childdoc} Package}
\hypersetup{pdftitle={The childdoc Package}}
\author{Niklas Beisert\\[2ex]
  Institut f\"ur Theoretische Physik\\
  Eidgen\"ossische Technische Hochschule Z\"urich\\
  Wolfgang-Pauli-Strasse 27, 8093 Z\"urich, Switzerland\\[1ex]
  \href{mailto:nbeisert@itp.phys.ethz.ch}
  {\texttt{nbeisert@itp.phys.ethz.ch}}}
\hypersetup{pdfauthor={Niklas Beisert}}
\hypersetup{pdfsubject={Manual for the LaTeX2e Package childdoc}}
\date{30 December 2018, \textsf{v2.0}}
\maketitle

\begin{abstract}\noindent
\textsf{childdoc} is a \LaTeXe{} package
that enables the direct compilation
of document sections included by |\include|
to individual files.
\end{abstract}

\begingroup
\parskip0ex
\tableofcontents
\endgroup

%%%%%%%%%%%%%%%%%%%%%%%%%%%%%%%%%%%%%%%%%%%%%%%%%%%%%%%%%%%%%%%%%%%%%%%%%%%%%%%%
%%%%%%%%%%%%%%%%%%%%%%%%%%%%%%%%%%%%%%%%%%%%%%%%%%%%%%%%%%%%%%%%%%%%%%%%%%%%%%%%
\section{Introduction}

\LaTeX{} provides a mechanism to structure a large document (such as a book)
into a main file and several child files (containing the chapters)
using the |\include| command.
This mechanism is beneficial for documents
which span hundreds of pages in order to
make the source file(s) more manageable.
Moreover, compilation can be restricted to
selected child files by means of the |\includeonly| command.
The latter feature can be used to reduce the compilation time while editing
(this was significantly more useful in the earlier days of \LaTeX{})
or to generate a smaller document which is easier to navigate.
Another application of |\includeonly| is to generate
documents consisting of selected parts of the complete document.

However, there are a few drawbacks of the plain |\include| mechanism:
\begin{itemize}
\item
The child files cannot be compiled on their own,
they can only be compiled via the main file.
A naive editing environment
(such as a text editor with an option
to have the current file processed by \LaTeX)
may require one to switch to the main file before compiling;
attempting to compile the child file produces errors.
\item
The main file must be modified (each time)
to adjust the |\includeonly| command
to the present needs. This easily leaves the main file in a messy state.
\item
The generated document will always carry the filename
of the main document. This is inconvenient if
several child files are to be compiled and
to be kept for distribution.
\end{itemize}

The present package provides a simple interface
to make child files individually compilable by \LaTeX{}.
Compiling a child file then has the same effect as compiling
the main file with an |\includeonly| command
to select the appropriate child.
Moreover the generated document will carry the name of the child
rather than the main file.
This resolves all three above issues.

This feature is meant to make the editing of books,
thesis documents and lecture notes somewhat more convenient.
However, the package can also be used efficiently for
composing a series of documents (such as exercise sheets)
which are typically distributed individually.
It then assists the author in generating the individual documents
(potentially in different versions)
as well as a document containing the collected series.
Another application is in developing style files
or other kinds of included material
where compilation of the style file could redirect
to a sample or test file.

%%%%%%%%%%%%%%%%%%%%%%%%%%%%%%%%%%%%%%%%%%%%%%%%%%%%%%%%%%%%%%%%%%%%%%%%%%%%%%%%
%%%%%%%%%%%%%%%%%%%%%%%%%%%%%%%%%%%%%%%%%%%%%%%%%%%%%%%%%%%%%%%%%%%%%%%%%%%%%%%%
\section{Usage}

First of all, the package \textsf{childdoc} is \emph{not} a standard
\LaTeXe{} |.sty| style file! Therefore it needs to be invoked in
a non-standard way.

%%%%%%%%%%%%%%%%%%%%%%%%%%%%%%%%%%%%%%%%%%%%%%%%%%%%%%%%%%%%%%%%%%%%%%%%%%%%%%%%
\subsection{Included Files}
\label{sec:include}

%%%%%%%%%%%%%%%%%%%%%%%%%%%%%%%%%%%%%%%%
\DescribeMacro{\childdocmain}
To use the package, add the commands
\begin{center}
\begin{tabular}{l}
|\input{childdoc.def}|\\
|\childdocmain{}|\\
\end{tabular}
\end{center}
at the very top of the main \LaTeX{} file,
in particular \emph{before} the |\documentclass| statement!
The argument of |\childdocmain| should be left empty
(but it must be present).

%%%%%%%%%%%%%%%%%%%%%%%%%%%%%%%%%%%%%%%%
\DescribeMacro{\childdocof}
Furthermore, add the commands
\begin{center}
\begin{tabular}{l}
|\input{childdoc.def}|\\
|\childdocof{|\textit{main}|}|\\
\end{tabular}
\end{center}
at the top of every child file \textit{child}
which is included by |\include{|\textit{child}|}|
from within the main file
(or at least for those files to be compiled individually).
The argument \textit{main} must be the filename of the main file.

There are a couple of
considerations in setting up the main and child documents:

%%%%%%%%%%%%%%%%%%%%%%%%%%%%%%%%%%%%%%%%
\paragraph{Restrictions.}

Please note the following restrictions:
\begin{itemize}
\item
|\childdocmain| must be called with one argument \textit{main}
to ensure compatibility with earlier version of the package.
It must either be empty (|\childdocmain{}|)
or precisely match the filename of the main file in which it is specified.
See \secref{sec:detection} for further information.
\item
The filename \textit{main} must be specified without the |.tex| extension.
\item
The filename \textit{main} is case sensitive
(even in case-insensitive file systems)
due to internal string comparison.
\item
The argument \textit{main} should be fully expanded, it cannot be a macro.
\item
Subdirectories and special characters should be avoided in filenames.
\item
The command |\childdocmain{|\textit{main}|}| must be followed by a whitespace.
It should not be followed immediately by another command
or by a comment mark `|%|'.
This is because the \TeX{} parser reads the token immediately following
the argument of |\childdocmain| and puts it
at the beginning of every child section;
however, a white\-space is ignored.
\end{itemize}

%%%%%%%%%%%%%%%%%%%%%%%%%%%%%%%%%%%%%%%%
\paragraph{Content of Main File.}

It is advisable to place all content in the child files included by |\include|.
Any output contained in the main file will appear in all child documents
unless suppressed manually;
it cannot be suppressed automatically by the |\includeonly| directive
and thus should normally be avoided.
A method to include some content in the main file
by means of conditional processing is described in \secref{sec:conditional}.

%%%%%%%%%%%%%%%%%%%%%%%%%%%%%%%%%%%%%%%%
\paragraph{Page Numbering.}

When only a part of the document is compiled,
the appropriate numbering of pages
(as well as other status parameters)
is determined from the |.aux| files.
The latter contain information from previous passes.
However this information needs to propagate through
all intermediate child documents.
Therefore the page numbering in child documents may well
be inconsistent until the complete document is compiled at least once.

A useful (if unconventional) way to always ensure a consistent
page numbering is to restart the numbering in each child document
and denote the pages by `\textit{child}|.|\textit{page}'
where \textit{child} represents the chapter/section number of the child file.
This can be achieved by the command
|\numberwithin{page}{|\textit{child}|}|
of the \textsf{amsmath} package
where \textit{child} can be |chapter| or |section|
depending on the chosen structuring.
Alternatively, one can modify the macro |\thepage| appropriately
and reset the counter |page| at the start of each child file.

%%%%%%%%%%%%%%%%%%%%%%%%%%%%%%%%%%%%%%%%%%%%%%%%%%%%%%%%%%%%%%%%%%%%%%%%%%%%%%%%
\subsection{Conditional Processing}
\label{sec:conditional}

The package provides a mechanism to compile different versions
of a document. To customise the versions further some conditional processing
can come in handy to distinguish which version is being compiled.
The package provides two macros to describe the compilation context:

%%%%%%%%%%%%%%%%%%%%%%%%%%%%%%%%%%%%%%%%
\DescribeMacro{\ifchilddoc}
The conditional |\ifchilddoc| distinguishes between the compilation of
child documents and the main document:
%
\begin{center}
|\ifchilddoc |\textit{child-code}| |[|\||else |\textit{main-code}]| \||fi|
\end{center}

%%%%%%%%%%%%%%%%%%%%%%%%%%%%%%%%%%%%%%%%
\DescribeMacro{\childdocname}
\DescribeMacro{\childdocjob}
The macro |\childdocname| contains the filename (without extension)
of the main or child file being processed.
Note that |\childdocjob| will always contain the name of the main file.

%%%%%%%%%%%%%%%%%%%%%%%%%%%%%%%%%%%%%%%%
\paragraph{Title Page.}

Conditional processing can be used to include a title or banner page
in the main document when proper precautions are taken.
Importantly, the code in the main file should ensure that the page counter
(as well as other status parameters which are stored in the |.aux| files)
takes the same value after the conditional processing.
Otherwise the page numbers may take divergent values
depending on which part is compiled.

For example, a title page could be declared by:
%
\begin{center}
\begin{tabular}{l}
|\ifchilddoc\||else|\\
|\addtocounter{page}{-1}|\\
\textit{code for title page}\\
|\newpage|\\
|\||fi|
\end{tabular}
\end{center}
%
A banner page for the child documents can be generated by:
%
\begin{center}
\begin{tabular}{l}
|\ifchilddoc|\\
|\addtocounter{page}{-1}|\\
\textit{code for banner page}\\
|\newpage|\\
|\||fi|
\end{tabular}
\end{center}
%
Here one could write a message such as:
\begin{center}
|This is the part \childdocname{} of \childdocjob{}.|
\end{center}

%%%%%%%%%%%%%%%%%%%%%%%%%%%%%%%%%%%%%%%%%%%%%%%%%%%%%%%%%%%%%%%%%%%%%%%%%%%%%%%%
\subsection{Flags}
\label{sec:flags}

The package makes it easy to generate different versions
of the main or child documents.
To this end compilation flags can be defined
and assigned different default values.
They will be particularly useful in conjunction
with the forwarding mechanism described in \secref{sec:forward}.

For example, it may be useful to have a flag |\version|
which can be set to |draft| or |final|.
The document source will contain some conditional code
depending on the value of |\version|.
Suppose further, the flag should default to |final| for the main file
and to |draft| for child files
which is a natural assignment for editing the document.
This is achieved by placing the following code
in the preamble of the main document
(below the |\childdocmain| directive):
%
\begin{center}
\begin{tabular}{l}
|\ifchilddoc|\\
|\providecommand{\version}{draft}|\\
|\||else|\\
|\providecommand{\version}{final}|\\
|\||fi|
\end{tabular}
\end{center}
%
The definition by |\providecommand| makes sure
that previous definitions are not overwritten.
Further statements |\providecommand{\version}{...}|
can thus be added before the above code to override it.

For the main file, one might add a line
(between |\childdocmain| and the above block)
%
\begin{center}
|%\ifchilddoc\||else\providecommand{\version}{draft}\||fi|
\end{center}
%
which can be uncommented to produce a draft version.
Likewise one can add a line to the very top of a child file
(above the |\childdocof{|\textit{main}|}| directive)
%
\begin{center}
|%\providecommand{\version}{final}|
\end{center}
%
which can be uncommented to produce the final version of this child document.

%%%%%%%%%%%%%%%%%%%%%%%%%%%%%%%%%%%%%%%%%%%%%%%%%%%%%%%%%%%%%%%%%%%%%%%%%%%%%%%%
\subsection{Forwarding}
\label{sec:forward}

Different versions of the main or child documents
using compilation flags as described in \secref{sec:flags}
can be (permanently) stored in different files
for convenient compilation, viewing and distribution.
To this end, the package defines a command
to pass on compilation to a different file:

%%%%%%%%%%%%%%%%%%%%%%%%%%%%%%%%%%%%%%%%
\DescribeMacro{\childdocforward}
The command |\childdocforward| redirects processing to
another source file:
%
\begin{center}
\begin{tabular}{l}
|\input{childdoc.def}|\\
|\childdocforward[|\textit{main}|]{|\textit{dest}|}|\\
\end{tabular}
\end{center}
%
The argument \textit{dest} is the destination file
(without extension).
It should be the main file or one of the child files.
Note that further \textsf{childdoc} directives
such as |\childdocof| and |\childdocforward|
in the indicated file will be processed in this form.
The optional argument \textit{main}
passes on directly to the main file \textit{main}
while pretending to compile the child \textit{dest}.
This form behaves as if \textit{dest}
issues |\childdocof{|\textit{main}|}| right away,
and no further \textsf{childdoc} directives will be processed.

%%%%%%%%%%%%%%%%%%%%%%%%%%%%%%%%%%%%%%%%
\DescribeMacro{\...prefix}
In the alternative form |\childdocforwardprefix|,
%
\begin{center}
\begin{tabular}{l}
|\input{childdoc.def}|\\
|\childdocforwardprefix[|\textit{main}|]{|\textit{prefix}|}{|\textit{dest}|}|
\end{tabular}
\end{center}
%
the destination file is determined by a pattern
depending on the current file:
To make this work, the current file must be called
`{\textit{prefix}\hspace{0.2em}\textit{suffix}}'
with \textit{prefix} matching precisely the argument.
Processing is then passed on to the file
`{\textit{dest}\hspace{0.2em}\textit{suffix}}'.
Surely, the same effect is achieved by
directly specifying the
argument `{\textit{dest}\hspace{0.2em}\textit{suffix}}'
in the first form.
However, that requires to set up a different file
for each child. With the alternative form of the command
all these files can have exactly the same content
which simplifies setting them up and maintaining them.

For example, the following file |draft.tex|
with a compilation flag |\version| as described in \secref{sec:flags}
compiles the main document as a draft:
%
\begin{center}
\begin{tabular}{l}
|\def\version{draft}|\\
|\input{childdoc.def}|\\
|\childdocforward{|\textit{main}|}|
\end{tabular}
\end{center}
%
Likewise, the following files |final|\textit{nn}|.tex|
compile the final version of the child document
|child|\textit{nn}|.tex|:
%
\begin{center}
\begin{tabular}{l}
|\def\version{final}|\\
|\input{childdoc.def}|\\
|\childdocforwardprefix{final}{child}|
\end{tabular}
\end{center}
%

Note that when several versions of a main file and/or of each child file
are to be generated, it may be convenient to set up a |Makefile| or
shell script to automatise the process.

%%%%%%%%%%%%%%%%%%%%%%%%%%%%%%%%%%%%%%%%%%%%%%%%%%%%%%%%%%%%%%%%%%%%%%%%%%%%%%%%
\subsection{Command Line Processing}
\label{sec:commandline}

The effect of redirection files can also be achieved by invoking
the \LaTeX{} compiler with a more elaborate command line.
Most conveniently this should be done as part
of a shell script or a |Makefile|.

When using \textsf{childdoc} in the main file, the following
command lines effectively perform a redirection
(note that depending on the shell being used,
backslashes may have to be doubled: `|\|' $\to$ `|\\|'):
%
\begin{center}
|... -jobname "|\textit{target}|" |\\|"|[\textit{flags}]%
|\input{childdoc.def}\childdocforward[|\textit{main}|]{|\textit{dest}|}"|
\end{center}
%
Here \textit{target} is the name of the output file,
\textit{main} is the name of the main file
and \textit{dest} is the name of the main or child file to be processed
(all filenames without extensions).
The optional argument \textit{main} can be omitted
if \textit{main} matches \textit{dest}.
Optionally, compilation \textit{flags} can be defined via |\def| commands.
This command line makes the \TeX{} engine believe
it is compiling the file \textit{target}
whose content is specified as the latter parameter.
The provided code then forwards the processing to
\textit{main} or \textit{dest} as described in \secref{sec:forward}.

%%%%%%%%%%%%%%%%%%%%%%%%%%%%%%%%%%%%%%%%%%%%%%%%%%%%%%%%%%%%%%%%%%%%%%%%%%%%%%%%
\subsection{Include by Input}
\label{sec:input}

Including child documents by |\include| has some restrictions by design.
Most notably, the content of a child document always occupies
its own set of pages; pages cannot be shared between child documents.
Usually, this behaviour makes perfect sense
because each child document contain an essential part of the document.
However, in some situations it may be desirable to compose
a document from a collection of parts
without having mandatory page breaks between then.
For this case, the package
provides a mechanism to include parts
by |\input| which can also be processed individually.
However, by construction this mechanism
requires manual handling of the content to be output.

%%%%%%%%%%%%%%%%%%%%%%%%%%%%%%%%%%%%%%%%
\DescribeMacro{\ifchilddocmanual}
The main file should be prepared as usual, see \secref{sec:include}.
However, the document body must make a distinction
between processing of an individual part and of the main document, e.g.:
%
\begin{center}
\begin{tabular}{l}
|\ifchilddocmanual|\\
|\input{\childdocname}|\\
|\||else|\\
\textit{document body with }|\input{|\textit{part}|}|\\
|\||fi|
\end{tabular}
\end{center}
%
The conditional |\ifchilddocmanual| is true whenever
a part to be included by |\input| is being compiled,
and the name of the part is stored in |\childdocname|.

%%%%%%%%%%%%%%%%%%%%%%%%%%%%%%%%%%%%%%%%
\DescribeMacro{\childdocby}
Each part to be included by |\input| should start with:
%
\begin{center}
\begin{tabular}{l}
|\input{childdoc.def}|\\
|\childdocby{|\textit{main}|}|\\
\end{tabular}
\end{center}
%
The directive |\childdocby| is similar to |\childdocof|
described in \secref{sec:include},
but the subsequent selection of content must be done manually.
To that end, both |\ifchilddoc| and |\ifchilddocmanual|
will be true upon processing of a part,
and the name of the part is stored in |\childdocname|.
Note that |\jobname| will be set to the filename of the current part
so that each part receives an individual |.aux| file
that does not interfere with the |.aux| file(s) of the main document.
This behaviour can be altered by the alternative form
|\childdocby[*]{|\textit{main}|}| (with a non-empty optional argument)
which uses the |.aux| file of the main document
by setting |\jobname| to \textit{main}.

%%%%%%%%%%%%%%%%%%%%%%%%%%%%%%%%%%%%%%%%%%%%%%%%%%%%%%%%%%%%%%%%%%%%%%%%%%%%%%%%
\subsection{Driver Development}
\label{sec:driver}

The \textsf{childdoc} mechanism can also be use for the development
of definition files such as \LaTeX{} styles or classes.
This case differs from the above setup with multiple parts
included by |\include| in that no |\includeonly| should be invoked.
This can be achieved by starting the include file
(before |\ProvidesPackage|) with:
%
\begin{center}
\begin{tabular}{l}
|\input{childdoc.def}|\\
|\childdocforward{|\textit{main}|}|\\
\end{tabular}
\end{center}
%
or alternatively with:
%
\begin{center}
\begin{tabular}{l}
|\input{childdoc.def}|\\
|\childdocby{|\textit{main}|}|\\
\end{tabular}
\end{center}
%
Both forms have slightly different effects as described above.
The main file is prepared as usual, see \secref{sec:include}.

%%%%%%%%%%%%%%%%%%%%%%%%%%%%%%%%%%%%%%%%%%%%%%%%%%%%%%%%%%%%%%%%%%%%%%%%%%%%%%%%
\subsection{Legacy Detection}
\label{sec:detection}

The directive |\childdocmain| in the main file can detect
whether the complete document or merely a child is to be compiled
even without using the directive |\childdocof|.
This method is deprecated because it is less robust
and there is no compelling reason to use it;
it is merely provided for backward compatibility
and it may be removed in future versions.

If the detection mechanism is to be used,
it is mandatory to correctly specify
the filename of the main file as the argument of |\childdocmain|:
%
\begin{center}
\begin{tabular}{l}
|\input{childdoc.def}|\\
|\childdocmain{|\textit{main}|}|\\
\end{tabular}
\end{center}
%
If |\jobname| does not match the argument \textit{main} of |\childdocmain|,
it is assumed that |\jobname| points to the child file to be compiled.
When using |\childdocmain| with the main file specified as argument,
it suffices to start a child file
with just |\input{|\textit{main}|}|
without loading of the package and using |\childdocof|.
If instead all processing is done
with the appropriate \textsf{childdoc} directives,
the argument of \textit{main} of |\childdocmain| can be empty.

An alternative version of the command line processing described
in \secref{sec:commandline} using the detection mechanism reads:
%
\begin{center}
|... -jobname "|\textit{target}|" "|[\textit{flags}]%
[|\def\jobname{|\textit{dest}|}|]|\input{|\textit{main}|}"|
\end{center}

%%%%%%%%%%%%%%%%%%%%%%%%%%%%%%%%%%%%%%%%%%%%%%%%%%%%%%%%%%%%%%%%%%%%%%%%%%%%%%%%
\subsection{Manual Code}
\label{sec:manual}

In case one cannot be certain whether the definitions file |childdoc.def|
is installed on the target \TeX{} distribution
and one prefers not to ship it,
it is conceivable to paste a few relevant commands into the sources.

To that end, drop all statements |\input{childdoc.def}|
and perform the replacements as outlined below.
Instead of |\childdocmain{|\textit{main}|}| add the following code
to the top of the main file:
%
\begin{center}
\begin{tabular}{l}
|\||ifdefined\childdocname\endinput\||fi\newif\ifchilddoc|\\
|\edef\childdocname{\scantokens\expandafter{\jobname\noexpand}}|\\
|\def\childdocmain{|\textit{main}|}\||ifx\childdocmain\childdocname\||else|\\
|\childdoctrue\includeonly{\childdocname}\let\jobname\childdocmain\||fi|\\
\end{tabular}
\end{center}
%
Instead of |\childdocof{|\textit{main}|}| just include the main file
at the top of each child file:
%
\begin{center}
|\input{|\textit{main}|}|
\end{center}
%
A simple redirection |\childdocforward{|\textit{dest}|}| is achieved by:
%
\begin{center}
|\def\jobname{|\textit{dest}|}\input{\jobname}|
\end{center}
%
The redirection with prefix
|\childdocforwardprefix[|\textit{prefix}|]{|\textit{dest}|}|
is accomplished by:
%
\begin{center}
\begin{tabular}{l}
|{\edef\jobname{\scantokens\expandafter{\jobname\noexpand}}|\\
|\def\redirectjob |\textit{prefix}|#1~~~{\gdef\jobname{|\textit{dest}|#1}}|\\
|\expandafter\redirectjob\jobname~~~}\input{\jobname}|
\end{tabular}
\end{center}

In an alternative approach,
child documents can be compiled by a specific command line
without additional code or specific definitions:
%
\begin{center}
|... -jobname "|\textit{target}|" "|[\textit{flags}]%
|\includeonly{|\textit{dest}|}\input{|\textit{main}|}"|
\end{center}
%

%%%%%%%%%%%%%%%%%%%%%%%%%%%%%%%%%%%%%%%%%%%%%%%%%%%%%%%%%%%%%%%%%%%%%%%%%%%%%%%%
%%%%%%%%%%%%%%%%%%%%%%%%%%%%%%%%%%%%%%%%%%%%%%%%%%%%%%%%%%%%%%%%%%%%%%%%%%%%%%%%
\section{Information}

%%%%%%%%%%%%%%%%%%%%%%%%%%%%%%%%%%%%%%%%%%%%%%%%%%%%%%%%%%%%%%%%%%%%%%%%%%%%%%%%
\subsection{Copyright}

Copyright \copyright{} 2017--2018 Niklas Beisert

This work may be distributed and/or modified under the
conditions of the \LaTeX{} Project Public License, either version 1.3
of this license or (at your option) any later version.
The latest version of this license is in
  \url{http://www.latex-project.org/lppl.txt}
and version 1.3 or later is part of all distributions of \LaTeX{}
version 2005/12/01 or later.

This work has the LPPL maintenance status `maintained'.

The Current Maintainer of this work is Niklas Beisert.

This work consists of the files |README.txt|, |childdoc.ins| and |childdoc.dtx|
as well as the derived files |childdoc.def|, |cdocsamp.tex|
with |cdocsch1.tex|, |cdocsch2.tex|, |cdocspt3.tex|, |cdocspt4.tex|,
|cdocsdrf.tex|, |cdocsfn1.tex|, |cdocsfn2.tex|
as well as |childdoc.pdf|.

%%%%%%%%%%%%%%%%%%%%%%%%%%%%%%%%%%%%%%%%%%%%%%%%%%%%%%%%%%%%%%%%%%%%%%%%%%%%%%%%
\subsection{Files and Installation}

The package consists of the files:
%
\begin{center}
\begin{tabular}{ll}
    |README.txt|   & readme file \\
    |childdoc.ins| & installation file \\
    |childdoc.dtx| & source file \\
    |childdoc.def| & definition file \\
    |cdocsamp.tex| & sample main file \\
    |cdocsch1.tex| & sample include file \\
    |cdocsch2.tex| & sample include file \\
    |cdocspt3.tex| & sample part file \\
    |cdocspt4.tex| & sample part file \\
    |cdocsdrf.tex| & sample redirection file \\
    |cdocsfn1.tex| & sample redirection file \\
    |cdocsfn2.tex| & sample redirection file \\
    |childdoc.pdf| & manual
\end{tabular}
\end{center}
%
The distribution consists of the files
|README.txt|, |childdoc.ins| and |childdoc.dtx|.
%
\begin{itemize}
\item
Run (pdf)\LaTeX{} on |childdoc.dtx|
to compile the manual |childdoc.pdf| (this file).
\item
Run \LaTeX{} on |childdoc.ins| to create the definitions file |childdoc.def|
and the sample |cdocsamp.tex| with include files
|cdocsch1.tex|, |cdocsch2.tex|, |cdocspt3.tex|, |cdocspt4.tex|,
|cdocsdrf.tex|, |cdocsfn1.tex|, |cdocsfn2.tex|.
Then copy the file |childdoc.def| to an appropriate directory of your \LaTeX{}
distribution, e.g.\ \textit{texmf-root}|/tex/latex/childdoc|.
\end{itemize}

%%%%%%%%%%%%%%%%%%%%%%%%%%%%%%%%%%%%%%%%%%%%%%%%%%%%%%%%%%%%%%%%%%%%%%%%%%%%%%%%
\subsection{Related CTAN Packages}

There are several other packages which offer a similar functionality:
%
\begin{itemize}
\item
The packages
\href{http://ctan.org/pkg/docmute}{\textsf{docmute}},
\href{http://ctan.org/pkg/includex}{\textsf{includex}} and
\href{http://ctan.org/pkg/standalone}{\textsf{standalone}}
provide commands to include only the document body of
a child file thus allowing both files to be compiled individually.
\item
The packages \href{http://ctan.org/pkg/subdocs}{\textsf{subdocs}}
and \href{http://ctan.org/pkg/subfiles}{\textsf{subfiles}}
provide structures in which the main and child documents can be
encapsulated and allowing them to be compiled individually.
The inclusion mechanism is different from the conventional |\include|.
\item
The package \href{http://ctan.org/pkg/combine}{\textsf{combine}}
is an elaborate solution to combine several documents into one.
\end{itemize}
%
See also the CTAN topic \href{http://ctan.org/topic/subdocs}{\textsf{subdocs}}
for further related packages.
The present package differs from the above solutions in that
a document structure constructed with the conventional |\include| mechanism
just needs two extra commands at the top of every file
such that all constituent files can be compiled individually.

%%%%%%%%%%%%%%%%%%%%%%%%%%%%%%%%%%%%%%%%%%%%%%%%%%%%%%%%%%%%%%%%%%%%%%%%%%%%%%%%
%\subsection{Feature Suggestions}
%
%The following is a list of features which may be useful for future
%versions of this package:
%%
%\begin{itemize}
%\item
%\ldots
%\end{itemize}

%%%%%%%%%%%%%%%%%%%%%%%%%%%%%%%%%%%%%%%%%%%%%%%%%%%%%%%%%%%%%%%%%%%%%%%%%%%%%%%%
\subsection{Revision History}

%%%%%%%%%%%%%%%%%%%%%%%%%%%%%%%%%%%%%%%%
\paragraph{v2.0:} 2018/12/30

\begin{itemize}
\item
immediate forward processing
\item
added |\childdocby| mechanism
\item
manual restructured
\end{itemize}

%%%%%%%%%%%%%%%%%%%%%%%%%%%%%%%%%%%%%%%%
\paragraph{v1.6:} 2018/01/17

\begin{itemize}
\item
application for development of include files
\item
corrections to manual
\end{itemize}

%%%%%%%%%%%%%%%%%%%%%%%%%%%%%%%%%%%%%%%%
\paragraph{v1.5:} 2017/05/21

\begin{itemize}
\item
more complete structuring introduced
\item
|\childdocof| introduced
\item
|\childdoc| renamed to |\childdocmain|
\item
|\childredirect| renamed to |\childdocforward| and |\childdocforwardprefix|
and functionality expanded
\end{itemize}

%%%%%%%%%%%%%%%%%%%%%%%%%%%%%%%%%%%%%%%%
\paragraph{v1.0:} 2017/04/27

\begin{itemize}
\item
manual and install package
\item
first version published on CTAN
\end{itemize}

%%%%%%%%%%%%%%%%%%%%%%%%%%%%%%%%%%%%%%%%
\paragraph{v0.6:} 2017/04/26

\begin{itemize}
\item
redirection mechanism added
\end{itemize}

%%%%%%%%%%%%%%%%%%%%%%%%%%%%%%%%%%%%%%%%
\paragraph{v0.5:} 2017/04/26

\begin{itemize}
\item
functionality in definition file
\end{itemize}


%%%%%%%%%%%%%%%%%%%%%%%%%%%%%%%%%%%%%%%%%%%%%%%%%%%%%%%%%%%%%%%%%%%%%%%%%%%%%%%%
%%%%%%%%%%%%%%%%%%%%%%%%%%%%%%%%%%%%%%%%%%%%%%%%%%%%%%%%%%%%%%%%%%%%%%%%%%%%%%%%
%%%%%%%%%%%%%%%%%%%%%%%%%%%%%%%%%%%%%%%%%%%%%%%%%%%%%%%%%%%%%%%%%%%%%%%%%%%%%%%%
\appendix

\settowidth\MacroIndent{\rmfamily\scriptsize 000\ }

 \DocInput{childdoc.dtx}

\end{document}
%</driver>
% \fi
%
% %%%%%%%%%%%%%%%%%%%%%%%%%%%%%%%%%%%%%%%%%%%%%%%%%%%%%%%%%%%%%%%%%%%%%%%%%%%%%%
% %%%%%%%%%%%%%%%%%%%%%%%%%%%%%%%%%%%%%%%%%%%%%%%%%%%%%%%%%%%%%%%%%%%%%%%%%%%%%%
% \section{Sample}
%\iffalse
%<*samplemain>
%\fi
%
% The following presents a sample document
% with two chapters, two parts, a title page,
% a compile flag as well as three forwarding files to set the flag.
% It consists of eight |.tex| files:
% \begin{center}
% \begin{tabular}{ll}
% |cdocsamp.tex|&main file\\
% |cdocsch1.tex|&include file for chapter 1\\
% |cdocsch2.tex|&include file for chapter 2\\
% |cdocspt3.tex|&include file for part 3\\
% |cdocspt4.tex|&include file for part 4\\
% |cdocsdrf.tex|&forwarding file for main file in draft mode\\
% |cdocsfi1.tex|&forwarding file for final version of chapter 1\\
% |cdocsfi2.tex|&forwarding file for final version of chapter 2\\
% \end{tabular}
% \end{center}
% Each of the eight files can be compiled directly by the \LaTeX{} compiler.
%
% %%%%%%%%%%%%%%%%%%%%%%%%%%%%%%%%%%%%%%
% \paragraph{Main File.}
%
% The main file is called |cdocsamp.tex|.
%
% Load the \textsf{childdoc} definitions and
% declare the filename for the main document:
%    \begin{macrocode}
\input{childdoc.def}
\childdocmain{}
%    \end{macrocode}

% Optional override for |\version| flag:
%    \begin{macrocode}
%%\ifchilddoc\else\providecommand{\version}{draft}\fi
%    \end{macrocode}

% Define the default values for the |\version| flag
% (|final| for the main file and |draft| for childs):
%    \begin{macrocode}
\ifchilddoc
\providecommand{\version}{draft}
\else
\providecommand{\version}{final}
\fi
%    \end{macrocode}

% Load the standard document class:
%    \begin{macrocode}
\documentclass[12pt]{article}
%    \end{macrocode}

% Start the document body:
%    \begin{macrocode}
\begin{document}
%    \end{macrocode}

% Declare a title page.
% Print title, part of document being processed and version flag:
%    \begin{macrocode}
\addtocounter{page}{-1}
\begin{center}
{\LARGE\bfseries{}childdoc example\par}
\vspace{1cm}
\ifchilddoc
\ifchilddocmanual part\else chapter\fi:
`\childdocname' of `\childdocjob'\par
\else
main document: `\childdocjob'\par
\fi
version: \version\par
\end{center}
\newpage
%    \end{macrocode}

% Manually include selected file,
% otherwise process as usual:
%    \begin{macrocode}
\ifchilddocmanual
\section*{part `\childdocname'}
\input{\childdocname}
\else
%    \end{macrocode}

% Include the two chapters:
%    \begin{macrocode}
\include{cdocsch1}
\include{cdocsch2}
%    \end{macrocode}

% Include the two parts unless only chapters should be displayed:
%    \begin{macrocode}
\ifchilddoc\else
\section{part three}
\input{cdocspt3}
\section{part four}
\input{cdocspt4}
\fi
%    \end{macrocode}

% Process as usual until here:
%    \begin{macrocode}
\fi
%    \end{macrocode}

% End of document body:
%    \begin{macrocode}
\end{document}
%    \end{macrocode}
%\iffalse
%</samplemain>
%\fi
%
% %%%%%%%%%%%%%%%%%%%%%%%%%%%%%%%%%%%%%%
% \paragraph{Chapter Include Files.}
%
% The include files are called |cdocsch1.tex| and |cdocsch2.tex|.
%
%\iffalse
%<*samplechap1|samplechap2>
%\fi

% Optional override for |\version| flag:
%    \begin{macrocode}
%%\providecommand{\version}{final}
%    \end{macrocode}

% Include the main document:
%    \begin{macrocode}
\input{childdoc.def}
\childdocof{cdocsamp}
%    \end{macrocode}

%\iffalse
%</samplechap1|samplechap2>
%\fi
%
%\iffalse
%<*samplechap1>
%\fi
% Some text for chapter 1:
%    \begin{macrocode}
\section{one}
some text in chapter one
%    \end{macrocode}

%\iffalse
%</samplechap1>
%\fi
% Some text for chapter 2:
%\iffalse
%<*samplechap2>
%\fi
%    \begin{macrocode}
\section{two}
more text in chapter two
%    \end{macrocode}

%\iffalse
%</samplechap2>
%\fi
%
% %%%%%%%%%%%%%%%%%%%%%%%%%%%%%%%%%%%%%%
% \paragraph{Part Include Files.}
%
% The include files are called |cdocspt3.tex| and |cdocspt4.tex|.
%
%\iffalse
%<*samplepart3|samplepart4>
%\fi

% Optional override for |\version| flag:
%    \begin{macrocode}
%%\providecommand{\version}{final}
%    \end{macrocode}

% Include the main document:
%    \begin{macrocode}
\input{childdoc.def}
\childdocby{cdocsamp}
%    \end{macrocode}

%\iffalse
%</samplepart3|samplepart4>
%\fi
%
%\iffalse
%<*samplepart3>
%\fi
% Some text for part 3:
%    \begin{macrocode}
some text in part three
%    \end{macrocode}

%\iffalse
%</samplepart3>
%\fi
% Some text for part 4:
%\iffalse
%<*samplepart4>
%\fi
%    \begin{macrocode}
more text in part four
%    \end{macrocode}

%\iffalse
%</samplepart4>
%\fi
%
% %%%%%%%%%%%%%%%%%%%%%%%%%%%%%%%%%%%%%%
% \paragraph{Forwarding for a Complete Draft.}
%
% The following forwarding file |cdocsdrf.tex|
% compiles the main document in draft mode:
%\iffalse
%<*sampledraft>
%\fi
%    \begin{macrocode}
\def\version{draft}
\input{childdoc.def}
\childdocforward{cdocsamp}
%    \end{macrocode}

%\iffalse
%</sampledraft>
%\fi
%
% %%%%%%%%%%%%%%%%%%%%%%%%%%%%%%%%%%%%%%
% \paragraph{Forwarding for Final Version of the Chapters.}
%
% The following forwarding files |cdocsfn1.tex| and |cdocsfn2.tex|
% (with identical content)
% compile the final versions of the child documents
% |cdocsch1.tex| and |cdocsch2.tex|, respectively:
%\iffalse
%<*samplefinal>
%\fi
%    \begin{macrocode}
\def\version{final}
\input{childdoc.def}
\childdocforwardprefix[cdocsamp]{cdocsfn}{cdocsch}
%    \end{macrocode}

%\iffalse
%</samplefinal>
%\fi
%
% %%%%%%%%%%%%%%%%%%%%%%%%%%%%%%%%%%%%%%
% \paragraph{Command Line Processing.}
%
% The following three command lines generate the output files
% |cdocscld|, |cdocscl1| and |cdocscl2|
% which should be identical to
% |cdocsdrf|, |cdocsch1| and |cdocsfn2|, respectively:
% \begin{center}
% \begin{tabular}{l}
% |latex -jobname cdocscld \|\\
% |  "\def\version{draft}\input{childdoc.def}\childdocforward{cdocsamp}"|\\
% |latex -jobname cdocscl1 \|\\
% |  "\input{childdoc.def}\childdocforward[cdocsamp]{cdocsch1}"|\\
% |latex -jobname cdocscl2 \|\\
% |  "\def\version{final}\input{childdoc.def}\childdocforward{cdocsch2}"|
% \end{tabular}
% \end{center}
% Note that the trailing backslash on each first line
% merely continues the input to the second line
% (for convenient cut ant paste).
% Furthermore, the command |latex| can be replaced by any
% of its alternative versions such as |pdflatex|.
%
% %%%%%%%%%%%%%%%%%%%%%%%%%%%%%%%%%%%%%%%%%%%%%%%%%%%%%%%%%%%%%%%%%%%%%%%%%%%%%%
% %%%%%%%%%%%%%%%%%%%%%%%%%%%%%%%%%%%%%%%%%%%%%%%%%%%%%%%%%%%%%%%%%%%%%%%%%%%%%%
% \section{Implementation}
%\iffalse
%<*package>
%\fi
%
% This section describes the definitions file |childdoc.def|.

% The definitions cannot be loaded using |\usepackage| or |\RequirePackage|
% which has a mechanism to prevent loading a style file more than once.
% When loading the definitions by means of |\input|
% multiple instances have to be prevented manually:
%\iffalse
%This code needs to be before the `\ProvidesFile' directive
%which is defined at the beginning of this file.
%Therefore it is also placed there and commented out here.
%</package>
%<*discard>
%\fi
%    \begin{macrocode}
\ifdefined\childdocmain\endinput\fi
%    \end{macrocode}
%\iffalse
%</discard>
%<*package>
%\fi
%
% \macro{\ifchilddoc}
% \macro{\ifchilddocmanual}
% The conditional |\ifchilddoc| tells whether a
% child (true) or main (false) document is being compiled.
% The conditional |\ifchilddocmanual| tells whether
% the |\includeonly| mechanism is used (false) or
% the selection of child files must be performed manually (true).
% The definitions initialise to false:
%    \begin{macrocode}
\newif\ifchilddoc
\newif\ifchilddocmanual
%    \end{macrocode}

% \macro{\childdocname}
% \macro{\childdocjob}
% The macro |\childdocname| stores the name of the main document
% to be compiled. The macro |\childdocjob| stores the name of
% the document on which the \LaTeX{} compiler was originally invoked.
% The content of |\jobname| cannot be compared
% to filenames specified in the source due to different catcodes.
% The following code rescans |\jobname|, stores the result
% in |\childdocname| and saves a copy in |\childdocjob|:
%    \begin{macrocode}
\edef\childdocname{\scantokens\expandafter{\jobname\noexpand}}
\let\childdocjob\childdocname
%    \end{macrocode}

% \macro{\childdocdisable}
% The macro |\childdocdisable| prevents the main file
% from being processed more than once.
% At this stage, the main document command |\childdocmain|
% is assumed to be called once again where it should do nothing.
% Any subsequent call to it should prevent
% a secondary processing of the main document
% It overwrites the forwarding commands
% |\childdocof| and |\childdocforward|
% with empty macros to prevent further inclusions of the main document:
%    \begin{macrocode}
\newcommand{\childdocdisable}
{
  \renewcommand{\childdocmain}[1]{\renewcommand{\childdocmain}[1]{\endinput}}
  \renewcommand{\childdocof}[1]{}
  \renewcommand{\childdocby}[2][]{}
  \renewcommand{\childdocforward}[2][]{}
  \renewcommand{\childdocdisable}{}
}
%    \end{macrocode}

% \macro{\childdocmain}
% The macro |\childdocmain| is to be called at the top of the main file
% with nothing or the main filename (without extension) as argument.
% First, it breaks loops.
% If the argument is not empty and does not match |\childdocname|
% (which is set by the first inclusion of |childdoc.def|),
% |\ifchilddoc| is set to true, |\includeonly| is applied to the child file
% and |\jobname| is set to the main file
% (for proper handling of |.aux| files):
%    \begin{macrocode}
\newcommand{\childdocmain}[1]
{
  \childdocdisable\childdocmain{}
  \if?#1?\else
    \begingroup
      \def\childdoctmp{#1}
      \ifx\childdoctmp\childdocname
        \def\childdoctmp{}
      \else
        \def\childdoctmp
        {
          \childdoctrue
          \includeonly{\childdocname}
          \def\childdocjob{#1}
          \def\jobname{#1}
        }
      \fi
      \expandafter
    \endgroup
    \childdoctmp
  \fi
}
%    \end{macrocode}

% \macro{\childdocof}
% The command |\childdocof| redirects
% compilation to the main file |#1|.
%    \begin{macrocode}
\newcommand{\childdocof}[1]
{
  \childdocdisable
  \childdoctrue
  \includeonly{\childdocname}
  \def\jobname{#1}
  \def\childdocjob{#1}
  \input{#1}
}
%    \end{macrocode}

% \macro{\childdocby}
% The command |\childdocby| ....
%    \begin{macrocode}
\newcommand{\childdocby}[2][]
{
  \childdocdisable
  \childdoctrue
  \childdocmanualtrue
  \if?#1?\else
    \def\jobname{#2}
  \fi
  \def\childdocjob{#2}
  \input{#2}
  \endinput
}
%    \end{macrocode}

% \macro{\childdocforward}
% The command |\childdocforward| redirects
% compilation to the main file or
% (if the optional argument is given) a child file.
% Parameters are set as if the main file
% or a child file starting with |\childdocof| was compiled.
% Then compilation is handed over to the main file:
%    \begin{macrocode}
\newcommand{\childdocforward}[2][]
{
  \begingroup
    \if?#1?
      \def\childdoctmp
      {
        \def\childdocname{#2}
        \def\childdocjob{#2}
        \def\jobname{#2}
        \input{#2}
        \endinput
      }
    \else
      \def\childdoctmp
      {
        \childdocdisable
        \def\childdocname{#2}
        \childdoctrue
        \includeonly{#2}
        \def\childdocjob{#1}
        \def\jobname{#1}
        \input{#1}
        \endinput
      }
    \fi
    \expandafter
  \endgroup
  \childdoctmp
}
%    \end{macrocode}

% \macro{\childdocforwardprefix}
% The command |\childdocforwardprefix| redirects
% compilation to the main or a child file by means of a pattern.
% The prefix |#1| in the current filename is replaced by |#2|
% and the suffix of the current filename is kept
% (it is assumed that the filename does not contain the substring `|~~~|'
% which is used as a delimiter).
% Compilation is handed over to the new file by |\childdocforward|:
%    \begin{macrocode}
\newcommand{\childdocforwardprefix}[3][]
{
  \begingroup
    \def\childdocextract #2##1~~~{\def\childdoctmp{\childdocforward[#1]{#3##1}}}
    \expandafter\childdocextract\childdocname~~~
    \expandafter
  \endgroup
  \childdoctmp
}
%    \end{macrocode}

% \macro{\childdoc}
% The deprecated macro |\childdoc| is a legacy version of |\childdocmain|:
%    \begin{macrocode}
\newcommand{\childdoc}{\childdocmain}
%    \end{macrocode}

% \macro{\childdocredirect}
% The deprecated macro |\childdocredirect| is a legacy version
% of |\childdocforward| and |\childdocforwardprefix|:
%    \begin{macrocode}
\newcommand{\childdocredirect}[2][]
{
  \begingroup
    \if?#1?
      \def\childdoctmp{\childdocforward{#2}}
    \else
      \def\childdoctmp{\childdocforwardprefix{#1}{#2}}
    \fi
    \expandafter
  \endgroup
  \childdoctmp
}
%    \end{macrocode}

%\iffalse
%</package>
%\fi
%
\endinput
|\\
|\childdocforwardprefix{final}{child}|
\end{tabular}
\end{center}
%

Note that when several versions of a main file and/or of each child file
are to be generated, it may be convenient to set up a |Makefile| or
shell script to automatise the process.

%%%%%%%%%%%%%%%%%%%%%%%%%%%%%%%%%%%%%%%%%%%%%%%%%%%%%%%%%%%%%%%%%%%%%%%%%%%%%%%%
\subsection{Command Line Processing}
\label{sec:commandline}

The effect of redirection files can also be achieved by invoking
the \LaTeX{} compiler with a more elaborate command line.
Most conveniently this should be done as part
of a shell script or a |Makefile|.

When using \textsf{childdoc} in the main file, the following
command lines effectively perform a redirection
(note that depending on the shell being used,
backslashes may have to be doubled: `|\|' $\to$ `|\\|'):
%
\begin{center}
|... -jobname "|\textit{target}|" |\\|"|[\textit{flags}]%
|% \iffalse
%
% childdoc.dtx Copyright (C) 2017-2018 Niklas Beisert
%
% This work may be distributed and/or modified under the
% conditions of the LaTeX Project Public License, either version 1.3
% of this license or (at your option) any later version.
% The latest version of this license is in
%   http://www.latex-project.org/lppl.txt
% and version 1.3 or later is part of all distributions of LaTeX
% version 2005/12/01 or later.
%
% This work has the LPPL maintenance status `maintained'.
%
% The Current Maintainer of this work is Niklas Beisert.
%
% This work consists of the files childdoc.dtx and childdoc.ins
% and the derived files childdoc.def and cdocsamp.tex with
% cdocsch1.tex, cdocsch2.tex, cdocsdrf.tex, cdocsfn1.tex, cdocsfn2.tex.
%
%<package>\ifdefined\childdocmain\endinput\fi
%<package>\ProvidesFile{childdoc.def}[2018/12/30 v2.0 child document driver]
%<samplemain>\ProvidesFile{cdocsamp.tex}[2018/12/30 v2.0 sample for childdoc]
%<*driver>
%\ProvidesFile{childdoc.drv}[2018/12/30 v2.0 childdoc reference manual file]
\PassOptionsToClass{10pt,a4paper}{article}
\documentclass{ltxdoc}

\usepackage[margin=35mm]{geometry}
\usepackage{hyperref}
\usepackage{hyperxmp}
\usepackage[usenames]{color}

\hypersetup{colorlinks=true}
\hypersetup{pdfstartview=FitH}
\hypersetup{pdfpagemode=UseNone}
\hypersetup{pdfsource={}}
\hypersetup{pdflang={en-UK}}
\hypersetup{pdfcopyright={Copyright 2017-2018 Niklas Beisert.
  This work may be distributed and/or modified under the
  conditions of the LaTeX Project Public License, either version 1.3
  of this license or (at your option) any later version.}}
\hypersetup{pdflicenseurl={http://www.latex-project.org/lppl.txt}}
\hypersetup{pdfcontactaddress={ETH Zurich, ITP, HIT K,
  Wolfgang-Pauli-Strasse 27}}
\hypersetup{pdfcontactpostcode={8093}}
\hypersetup{pdfcontactcity={Zurich}}
\hypersetup{pdfcontactcountry={Switzerland}}
\hypersetup{pdfcontactemail={nbeisert@itp.phys.ethz.ch}}
\hypersetup{pdfcontacturl={http://people.phys.ethz.ch/\xmptilde nbeisert/}}

\newcommand{\secref}[1]{\hyperref[#1]{section \ref*{#1}}}

\parskip1ex
\parindent0pt
\let\olditemize\itemize
\def\itemize{\olditemize\parskip0pt}

\begin{document}

\title{The \textsf{childdoc} Package}
\hypersetup{pdftitle={The childdoc Package}}
\author{Niklas Beisert\\[2ex]
  Institut f\"ur Theoretische Physik\\
  Eidgen\"ossische Technische Hochschule Z\"urich\\
  Wolfgang-Pauli-Strasse 27, 8093 Z\"urich, Switzerland\\[1ex]
  \href{mailto:nbeisert@itp.phys.ethz.ch}
  {\texttt{nbeisert@itp.phys.ethz.ch}}}
\hypersetup{pdfauthor={Niklas Beisert}}
\hypersetup{pdfsubject={Manual for the LaTeX2e Package childdoc}}
\date{30 December 2018, \textsf{v2.0}}
\maketitle

\begin{abstract}\noindent
\textsf{childdoc} is a \LaTeXe{} package
that enables the direct compilation
of document sections included by |\include|
to individual files.
\end{abstract}

\begingroup
\parskip0ex
\tableofcontents
\endgroup

%%%%%%%%%%%%%%%%%%%%%%%%%%%%%%%%%%%%%%%%%%%%%%%%%%%%%%%%%%%%%%%%%%%%%%%%%%%%%%%%
%%%%%%%%%%%%%%%%%%%%%%%%%%%%%%%%%%%%%%%%%%%%%%%%%%%%%%%%%%%%%%%%%%%%%%%%%%%%%%%%
\section{Introduction}

\LaTeX{} provides a mechanism to structure a large document (such as a book)
into a main file and several child files (containing the chapters)
using the |\include| command.
This mechanism is beneficial for documents
which span hundreds of pages in order to
make the source file(s) more manageable.
Moreover, compilation can be restricted to
selected child files by means of the |\includeonly| command.
The latter feature can be used to reduce the compilation time while editing
(this was significantly more useful in the earlier days of \LaTeX{})
or to generate a smaller document which is easier to navigate.
Another application of |\includeonly| is to generate
documents consisting of selected parts of the complete document.

However, there are a few drawbacks of the plain |\include| mechanism:
\begin{itemize}
\item
The child files cannot be compiled on their own,
they can only be compiled via the main file.
A naive editing environment
(such as a text editor with an option
to have the current file processed by \LaTeX)
may require one to switch to the main file before compiling;
attempting to compile the child file produces errors.
\item
The main file must be modified (each time)
to adjust the |\includeonly| command
to the present needs. This easily leaves the main file in a messy state.
\item
The generated document will always carry the filename
of the main document. This is inconvenient if
several child files are to be compiled and
to be kept for distribution.
\end{itemize}

The present package provides a simple interface
to make child files individually compilable by \LaTeX{}.
Compiling a child file then has the same effect as compiling
the main file with an |\includeonly| command
to select the appropriate child.
Moreover the generated document will carry the name of the child
rather than the main file.
This resolves all three above issues.

This feature is meant to make the editing of books,
thesis documents and lecture notes somewhat more convenient.
However, the package can also be used efficiently for
composing a series of documents (such as exercise sheets)
which are typically distributed individually.
It then assists the author in generating the individual documents
(potentially in different versions)
as well as a document containing the collected series.
Another application is in developing style files
or other kinds of included material
where compilation of the style file could redirect
to a sample or test file.

%%%%%%%%%%%%%%%%%%%%%%%%%%%%%%%%%%%%%%%%%%%%%%%%%%%%%%%%%%%%%%%%%%%%%%%%%%%%%%%%
%%%%%%%%%%%%%%%%%%%%%%%%%%%%%%%%%%%%%%%%%%%%%%%%%%%%%%%%%%%%%%%%%%%%%%%%%%%%%%%%
\section{Usage}

First of all, the package \textsf{childdoc} is \emph{not} a standard
\LaTeXe{} |.sty| style file! Therefore it needs to be invoked in
a non-standard way.

%%%%%%%%%%%%%%%%%%%%%%%%%%%%%%%%%%%%%%%%%%%%%%%%%%%%%%%%%%%%%%%%%%%%%%%%%%%%%%%%
\subsection{Included Files}
\label{sec:include}

%%%%%%%%%%%%%%%%%%%%%%%%%%%%%%%%%%%%%%%%
\DescribeMacro{\childdocmain}
To use the package, add the commands
\begin{center}
\begin{tabular}{l}
|\input{childdoc.def}|\\
|\childdocmain{}|\\
\end{tabular}
\end{center}
at the very top of the main \LaTeX{} file,
in particular \emph{before} the |\documentclass| statement!
The argument of |\childdocmain| should be left empty
(but it must be present).

%%%%%%%%%%%%%%%%%%%%%%%%%%%%%%%%%%%%%%%%
\DescribeMacro{\childdocof}
Furthermore, add the commands
\begin{center}
\begin{tabular}{l}
|\input{childdoc.def}|\\
|\childdocof{|\textit{main}|}|\\
\end{tabular}
\end{center}
at the top of every child file \textit{child}
which is included by |\include{|\textit{child}|}|
from within the main file
(or at least for those files to be compiled individually).
The argument \textit{main} must be the filename of the main file.

There are a couple of
considerations in setting up the main and child documents:

%%%%%%%%%%%%%%%%%%%%%%%%%%%%%%%%%%%%%%%%
\paragraph{Restrictions.}

Please note the following restrictions:
\begin{itemize}
\item
|\childdocmain| must be called with one argument \textit{main}
to ensure compatibility with earlier version of the package.
It must either be empty (|\childdocmain{}|)
or precisely match the filename of the main file in which it is specified.
See \secref{sec:detection} for further information.
\item
The filename \textit{main} must be specified without the |.tex| extension.
\item
The filename \textit{main} is case sensitive
(even in case-insensitive file systems)
due to internal string comparison.
\item
The argument \textit{main} should be fully expanded, it cannot be a macro.
\item
Subdirectories and special characters should be avoided in filenames.
\item
The command |\childdocmain{|\textit{main}|}| must be followed by a whitespace.
It should not be followed immediately by another command
or by a comment mark `|%|'.
This is because the \TeX{} parser reads the token immediately following
the argument of |\childdocmain| and puts it
at the beginning of every child section;
however, a white\-space is ignored.
\end{itemize}

%%%%%%%%%%%%%%%%%%%%%%%%%%%%%%%%%%%%%%%%
\paragraph{Content of Main File.}

It is advisable to place all content in the child files included by |\include|.
Any output contained in the main file will appear in all child documents
unless suppressed manually;
it cannot be suppressed automatically by the |\includeonly| directive
and thus should normally be avoided.
A method to include some content in the main file
by means of conditional processing is described in \secref{sec:conditional}.

%%%%%%%%%%%%%%%%%%%%%%%%%%%%%%%%%%%%%%%%
\paragraph{Page Numbering.}

When only a part of the document is compiled,
the appropriate numbering of pages
(as well as other status parameters)
is determined from the |.aux| files.
The latter contain information from previous passes.
However this information needs to propagate through
all intermediate child documents.
Therefore the page numbering in child documents may well
be inconsistent until the complete document is compiled at least once.

A useful (if unconventional) way to always ensure a consistent
page numbering is to restart the numbering in each child document
and denote the pages by `\textit{child}|.|\textit{page}'
where \textit{child} represents the chapter/section number of the child file.
This can be achieved by the command
|\numberwithin{page}{|\textit{child}|}|
of the \textsf{amsmath} package
where \textit{child} can be |chapter| or |section|
depending on the chosen structuring.
Alternatively, one can modify the macro |\thepage| appropriately
and reset the counter |page| at the start of each child file.

%%%%%%%%%%%%%%%%%%%%%%%%%%%%%%%%%%%%%%%%%%%%%%%%%%%%%%%%%%%%%%%%%%%%%%%%%%%%%%%%
\subsection{Conditional Processing}
\label{sec:conditional}

The package provides a mechanism to compile different versions
of a document. To customise the versions further some conditional processing
can come in handy to distinguish which version is being compiled.
The package provides two macros to describe the compilation context:

%%%%%%%%%%%%%%%%%%%%%%%%%%%%%%%%%%%%%%%%
\DescribeMacro{\ifchilddoc}
The conditional |\ifchilddoc| distinguishes between the compilation of
child documents and the main document:
%
\begin{center}
|\ifchilddoc |\textit{child-code}| |[|\||else |\textit{main-code}]| \||fi|
\end{center}

%%%%%%%%%%%%%%%%%%%%%%%%%%%%%%%%%%%%%%%%
\DescribeMacro{\childdocname}
\DescribeMacro{\childdocjob}
The macro |\childdocname| contains the filename (without extension)
of the main or child file being processed.
Note that |\childdocjob| will always contain the name of the main file.

%%%%%%%%%%%%%%%%%%%%%%%%%%%%%%%%%%%%%%%%
\paragraph{Title Page.}

Conditional processing can be used to include a title or banner page
in the main document when proper precautions are taken.
Importantly, the code in the main file should ensure that the page counter
(as well as other status parameters which are stored in the |.aux| files)
takes the same value after the conditional processing.
Otherwise the page numbers may take divergent values
depending on which part is compiled.

For example, a title page could be declared by:
%
\begin{center}
\begin{tabular}{l}
|\ifchilddoc\||else|\\
|\addtocounter{page}{-1}|\\
\textit{code for title page}\\
|\newpage|\\
|\||fi|
\end{tabular}
\end{center}
%
A banner page for the child documents can be generated by:
%
\begin{center}
\begin{tabular}{l}
|\ifchilddoc|\\
|\addtocounter{page}{-1}|\\
\textit{code for banner page}\\
|\newpage|\\
|\||fi|
\end{tabular}
\end{center}
%
Here one could write a message such as:
\begin{center}
|This is the part \childdocname{} of \childdocjob{}.|
\end{center}

%%%%%%%%%%%%%%%%%%%%%%%%%%%%%%%%%%%%%%%%%%%%%%%%%%%%%%%%%%%%%%%%%%%%%%%%%%%%%%%%
\subsection{Flags}
\label{sec:flags}

The package makes it easy to generate different versions
of the main or child documents.
To this end compilation flags can be defined
and assigned different default values.
They will be particularly useful in conjunction
with the forwarding mechanism described in \secref{sec:forward}.

For example, it may be useful to have a flag |\version|
which can be set to |draft| or |final|.
The document source will contain some conditional code
depending on the value of |\version|.
Suppose further, the flag should default to |final| for the main file
and to |draft| for child files
which is a natural assignment for editing the document.
This is achieved by placing the following code
in the preamble of the main document
(below the |\childdocmain| directive):
%
\begin{center}
\begin{tabular}{l}
|\ifchilddoc|\\
|\providecommand{\version}{draft}|\\
|\||else|\\
|\providecommand{\version}{final}|\\
|\||fi|
\end{tabular}
\end{center}
%
The definition by |\providecommand| makes sure
that previous definitions are not overwritten.
Further statements |\providecommand{\version}{...}|
can thus be added before the above code to override it.

For the main file, one might add a line
(between |\childdocmain| and the above block)
%
\begin{center}
|%\ifchilddoc\||else\providecommand{\version}{draft}\||fi|
\end{center}
%
which can be uncommented to produce a draft version.
Likewise one can add a line to the very top of a child file
(above the |\childdocof{|\textit{main}|}| directive)
%
\begin{center}
|%\providecommand{\version}{final}|
\end{center}
%
which can be uncommented to produce the final version of this child document.

%%%%%%%%%%%%%%%%%%%%%%%%%%%%%%%%%%%%%%%%%%%%%%%%%%%%%%%%%%%%%%%%%%%%%%%%%%%%%%%%
\subsection{Forwarding}
\label{sec:forward}

Different versions of the main or child documents
using compilation flags as described in \secref{sec:flags}
can be (permanently) stored in different files
for convenient compilation, viewing and distribution.
To this end, the package defines a command
to pass on compilation to a different file:

%%%%%%%%%%%%%%%%%%%%%%%%%%%%%%%%%%%%%%%%
\DescribeMacro{\childdocforward}
The command |\childdocforward| redirects processing to
another source file:
%
\begin{center}
\begin{tabular}{l}
|\input{childdoc.def}|\\
|\childdocforward[|\textit{main}|]{|\textit{dest}|}|\\
\end{tabular}
\end{center}
%
The argument \textit{dest} is the destination file
(without extension).
It should be the main file or one of the child files.
Note that further \textsf{childdoc} directives
such as |\childdocof| and |\childdocforward|
in the indicated file will be processed in this form.
The optional argument \textit{main}
passes on directly to the main file \textit{main}
while pretending to compile the child \textit{dest}.
This form behaves as if \textit{dest}
issues |\childdocof{|\textit{main}|}| right away,
and no further \textsf{childdoc} directives will be processed.

%%%%%%%%%%%%%%%%%%%%%%%%%%%%%%%%%%%%%%%%
\DescribeMacro{\...prefix}
In the alternative form |\childdocforwardprefix|,
%
\begin{center}
\begin{tabular}{l}
|\input{childdoc.def}|\\
|\childdocforwardprefix[|\textit{main}|]{|\textit{prefix}|}{|\textit{dest}|}|
\end{tabular}
\end{center}
%
the destination file is determined by a pattern
depending on the current file:
To make this work, the current file must be called
`{\textit{prefix}\hspace{0.2em}\textit{suffix}}'
with \textit{prefix} matching precisely the argument.
Processing is then passed on to the file
`{\textit{dest}\hspace{0.2em}\textit{suffix}}'.
Surely, the same effect is achieved by
directly specifying the
argument `{\textit{dest}\hspace{0.2em}\textit{suffix}}'
in the first form.
However, that requires to set up a different file
for each child. With the alternative form of the command
all these files can have exactly the same content
which simplifies setting them up and maintaining them.

For example, the following file |draft.tex|
with a compilation flag |\version| as described in \secref{sec:flags}
compiles the main document as a draft:
%
\begin{center}
\begin{tabular}{l}
|\def\version{draft}|\\
|\input{childdoc.def}|\\
|\childdocforward{|\textit{main}|}|
\end{tabular}
\end{center}
%
Likewise, the following files |final|\textit{nn}|.tex|
compile the final version of the child document
|child|\textit{nn}|.tex|:
%
\begin{center}
\begin{tabular}{l}
|\def\version{final}|\\
|\input{childdoc.def}|\\
|\childdocforwardprefix{final}{child}|
\end{tabular}
\end{center}
%

Note that when several versions of a main file and/or of each child file
are to be generated, it may be convenient to set up a |Makefile| or
shell script to automatise the process.

%%%%%%%%%%%%%%%%%%%%%%%%%%%%%%%%%%%%%%%%%%%%%%%%%%%%%%%%%%%%%%%%%%%%%%%%%%%%%%%%
\subsection{Command Line Processing}
\label{sec:commandline}

The effect of redirection files can also be achieved by invoking
the \LaTeX{} compiler with a more elaborate command line.
Most conveniently this should be done as part
of a shell script or a |Makefile|.

When using \textsf{childdoc} in the main file, the following
command lines effectively perform a redirection
(note that depending on the shell being used,
backslashes may have to be doubled: `|\|' $\to$ `|\\|'):
%
\begin{center}
|... -jobname "|\textit{target}|" |\\|"|[\textit{flags}]%
|\input{childdoc.def}\childdocforward[|\textit{main}|]{|\textit{dest}|}"|
\end{center}
%
Here \textit{target} is the name of the output file,
\textit{main} is the name of the main file
and \textit{dest} is the name of the main or child file to be processed
(all filenames without extensions).
The optional argument \textit{main} can be omitted
if \textit{main} matches \textit{dest}.
Optionally, compilation \textit{flags} can be defined via |\def| commands.
This command line makes the \TeX{} engine believe
it is compiling the file \textit{target}
whose content is specified as the latter parameter.
The provided code then forwards the processing to
\textit{main} or \textit{dest} as described in \secref{sec:forward}.

%%%%%%%%%%%%%%%%%%%%%%%%%%%%%%%%%%%%%%%%%%%%%%%%%%%%%%%%%%%%%%%%%%%%%%%%%%%%%%%%
\subsection{Include by Input}
\label{sec:input}

Including child documents by |\include| has some restrictions by design.
Most notably, the content of a child document always occupies
its own set of pages; pages cannot be shared between child documents.
Usually, this behaviour makes perfect sense
because each child document contain an essential part of the document.
However, in some situations it may be desirable to compose
a document from a collection of parts
without having mandatory page breaks between then.
For this case, the package
provides a mechanism to include parts
by |\input| which can also be processed individually.
However, by construction this mechanism
requires manual handling of the content to be output.

%%%%%%%%%%%%%%%%%%%%%%%%%%%%%%%%%%%%%%%%
\DescribeMacro{\ifchilddocmanual}
The main file should be prepared as usual, see \secref{sec:include}.
However, the document body must make a distinction
between processing of an individual part and of the main document, e.g.:
%
\begin{center}
\begin{tabular}{l}
|\ifchilddocmanual|\\
|\input{\childdocname}|\\
|\||else|\\
\textit{document body with }|\input{|\textit{part}|}|\\
|\||fi|
\end{tabular}
\end{center}
%
The conditional |\ifchilddocmanual| is true whenever
a part to be included by |\input| is being compiled,
and the name of the part is stored in |\childdocname|.

%%%%%%%%%%%%%%%%%%%%%%%%%%%%%%%%%%%%%%%%
\DescribeMacro{\childdocby}
Each part to be included by |\input| should start with:
%
\begin{center}
\begin{tabular}{l}
|\input{childdoc.def}|\\
|\childdocby{|\textit{main}|}|\\
\end{tabular}
\end{center}
%
The directive |\childdocby| is similar to |\childdocof|
described in \secref{sec:include},
but the subsequent selection of content must be done manually.
To that end, both |\ifchilddoc| and |\ifchilddocmanual|
will be true upon processing of a part,
and the name of the part is stored in |\childdocname|.
Note that |\jobname| will be set to the filename of the current part
so that each part receives an individual |.aux| file
that does not interfere with the |.aux| file(s) of the main document.
This behaviour can be altered by the alternative form
|\childdocby[*]{|\textit{main}|}| (with a non-empty optional argument)
which uses the |.aux| file of the main document
by setting |\jobname| to \textit{main}.

%%%%%%%%%%%%%%%%%%%%%%%%%%%%%%%%%%%%%%%%%%%%%%%%%%%%%%%%%%%%%%%%%%%%%%%%%%%%%%%%
\subsection{Driver Development}
\label{sec:driver}

The \textsf{childdoc} mechanism can also be use for the development
of definition files such as \LaTeX{} styles or classes.
This case differs from the above setup with multiple parts
included by |\include| in that no |\includeonly| should be invoked.
This can be achieved by starting the include file
(before |\ProvidesPackage|) with:
%
\begin{center}
\begin{tabular}{l}
|\input{childdoc.def}|\\
|\childdocforward{|\textit{main}|}|\\
\end{tabular}
\end{center}
%
or alternatively with:
%
\begin{center}
\begin{tabular}{l}
|\input{childdoc.def}|\\
|\childdocby{|\textit{main}|}|\\
\end{tabular}
\end{center}
%
Both forms have slightly different effects as described above.
The main file is prepared as usual, see \secref{sec:include}.

%%%%%%%%%%%%%%%%%%%%%%%%%%%%%%%%%%%%%%%%%%%%%%%%%%%%%%%%%%%%%%%%%%%%%%%%%%%%%%%%
\subsection{Legacy Detection}
\label{sec:detection}

The directive |\childdocmain| in the main file can detect
whether the complete document or merely a child is to be compiled
even without using the directive |\childdocof|.
This method is deprecated because it is less robust
and there is no compelling reason to use it;
it is merely provided for backward compatibility
and it may be removed in future versions.

If the detection mechanism is to be used,
it is mandatory to correctly specify
the filename of the main file as the argument of |\childdocmain|:
%
\begin{center}
\begin{tabular}{l}
|\input{childdoc.def}|\\
|\childdocmain{|\textit{main}|}|\\
\end{tabular}
\end{center}
%
If |\jobname| does not match the argument \textit{main} of |\childdocmain|,
it is assumed that |\jobname| points to the child file to be compiled.
When using |\childdocmain| with the main file specified as argument,
it suffices to start a child file
with just |\input{|\textit{main}|}|
without loading of the package and using |\childdocof|.
If instead all processing is done
with the appropriate \textsf{childdoc} directives,
the argument of \textit{main} of |\childdocmain| can be empty.

An alternative version of the command line processing described
in \secref{sec:commandline} using the detection mechanism reads:
%
\begin{center}
|... -jobname "|\textit{target}|" "|[\textit{flags}]%
[|\def\jobname{|\textit{dest}|}|]|\input{|\textit{main}|}"|
\end{center}

%%%%%%%%%%%%%%%%%%%%%%%%%%%%%%%%%%%%%%%%%%%%%%%%%%%%%%%%%%%%%%%%%%%%%%%%%%%%%%%%
\subsection{Manual Code}
\label{sec:manual}

In case one cannot be certain whether the definitions file |childdoc.def|
is installed on the target \TeX{} distribution
and one prefers not to ship it,
it is conceivable to paste a few relevant commands into the sources.

To that end, drop all statements |\input{childdoc.def}|
and perform the replacements as outlined below.
Instead of |\childdocmain{|\textit{main}|}| add the following code
to the top of the main file:
%
\begin{center}
\begin{tabular}{l}
|\||ifdefined\childdocname\endinput\||fi\newif\ifchilddoc|\\
|\edef\childdocname{\scantokens\expandafter{\jobname\noexpand}}|\\
|\def\childdocmain{|\textit{main}|}\||ifx\childdocmain\childdocname\||else|\\
|\childdoctrue\includeonly{\childdocname}\let\jobname\childdocmain\||fi|\\
\end{tabular}
\end{center}
%
Instead of |\childdocof{|\textit{main}|}| just include the main file
at the top of each child file:
%
\begin{center}
|\input{|\textit{main}|}|
\end{center}
%
A simple redirection |\childdocforward{|\textit{dest}|}| is achieved by:
%
\begin{center}
|\def\jobname{|\textit{dest}|}\input{\jobname}|
\end{center}
%
The redirection with prefix
|\childdocforwardprefix[|\textit{prefix}|]{|\textit{dest}|}|
is accomplished by:
%
\begin{center}
\begin{tabular}{l}
|{\edef\jobname{\scantokens\expandafter{\jobname\noexpand}}|\\
|\def\redirectjob |\textit{prefix}|#1~~~{\gdef\jobname{|\textit{dest}|#1}}|\\
|\expandafter\redirectjob\jobname~~~}\input{\jobname}|
\end{tabular}
\end{center}

In an alternative approach,
child documents can be compiled by a specific command line
without additional code or specific definitions:
%
\begin{center}
|... -jobname "|\textit{target}|" "|[\textit{flags}]%
|\includeonly{|\textit{dest}|}\input{|\textit{main}|}"|
\end{center}
%

%%%%%%%%%%%%%%%%%%%%%%%%%%%%%%%%%%%%%%%%%%%%%%%%%%%%%%%%%%%%%%%%%%%%%%%%%%%%%%%%
%%%%%%%%%%%%%%%%%%%%%%%%%%%%%%%%%%%%%%%%%%%%%%%%%%%%%%%%%%%%%%%%%%%%%%%%%%%%%%%%
\section{Information}

%%%%%%%%%%%%%%%%%%%%%%%%%%%%%%%%%%%%%%%%%%%%%%%%%%%%%%%%%%%%%%%%%%%%%%%%%%%%%%%%
\subsection{Copyright}

Copyright \copyright{} 2017--2018 Niklas Beisert

This work may be distributed and/or modified under the
conditions of the \LaTeX{} Project Public License, either version 1.3
of this license or (at your option) any later version.
The latest version of this license is in
  \url{http://www.latex-project.org/lppl.txt}
and version 1.3 or later is part of all distributions of \LaTeX{}
version 2005/12/01 or later.

This work has the LPPL maintenance status `maintained'.

The Current Maintainer of this work is Niklas Beisert.

This work consists of the files |README.txt|, |childdoc.ins| and |childdoc.dtx|
as well as the derived files |childdoc.def|, |cdocsamp.tex|
with |cdocsch1.tex|, |cdocsch2.tex|, |cdocspt3.tex|, |cdocspt4.tex|,
|cdocsdrf.tex|, |cdocsfn1.tex|, |cdocsfn2.tex|
as well as |childdoc.pdf|.

%%%%%%%%%%%%%%%%%%%%%%%%%%%%%%%%%%%%%%%%%%%%%%%%%%%%%%%%%%%%%%%%%%%%%%%%%%%%%%%%
\subsection{Files and Installation}

The package consists of the files:
%
\begin{center}
\begin{tabular}{ll}
    |README.txt|   & readme file \\
    |childdoc.ins| & installation file \\
    |childdoc.dtx| & source file \\
    |childdoc.def| & definition file \\
    |cdocsamp.tex| & sample main file \\
    |cdocsch1.tex| & sample include file \\
    |cdocsch2.tex| & sample include file \\
    |cdocspt3.tex| & sample part file \\
    |cdocspt4.tex| & sample part file \\
    |cdocsdrf.tex| & sample redirection file \\
    |cdocsfn1.tex| & sample redirection file \\
    |cdocsfn2.tex| & sample redirection file \\
    |childdoc.pdf| & manual
\end{tabular}
\end{center}
%
The distribution consists of the files
|README.txt|, |childdoc.ins| and |childdoc.dtx|.
%
\begin{itemize}
\item
Run (pdf)\LaTeX{} on |childdoc.dtx|
to compile the manual |childdoc.pdf| (this file).
\item
Run \LaTeX{} on |childdoc.ins| to create the definitions file |childdoc.def|
and the sample |cdocsamp.tex| with include files
|cdocsch1.tex|, |cdocsch2.tex|, |cdocspt3.tex|, |cdocspt4.tex|,
|cdocsdrf.tex|, |cdocsfn1.tex|, |cdocsfn2.tex|.
Then copy the file |childdoc.def| to an appropriate directory of your \LaTeX{}
distribution, e.g.\ \textit{texmf-root}|/tex/latex/childdoc|.
\end{itemize}

%%%%%%%%%%%%%%%%%%%%%%%%%%%%%%%%%%%%%%%%%%%%%%%%%%%%%%%%%%%%%%%%%%%%%%%%%%%%%%%%
\subsection{Related CTAN Packages}

There are several other packages which offer a similar functionality:
%
\begin{itemize}
\item
The packages
\href{http://ctan.org/pkg/docmute}{\textsf{docmute}},
\href{http://ctan.org/pkg/includex}{\textsf{includex}} and
\href{http://ctan.org/pkg/standalone}{\textsf{standalone}}
provide commands to include only the document body of
a child file thus allowing both files to be compiled individually.
\item
The packages \href{http://ctan.org/pkg/subdocs}{\textsf{subdocs}}
and \href{http://ctan.org/pkg/subfiles}{\textsf{subfiles}}
provide structures in which the main and child documents can be
encapsulated and allowing them to be compiled individually.
The inclusion mechanism is different from the conventional |\include|.
\item
The package \href{http://ctan.org/pkg/combine}{\textsf{combine}}
is an elaborate solution to combine several documents into one.
\end{itemize}
%
See also the CTAN topic \href{http://ctan.org/topic/subdocs}{\textsf{subdocs}}
for further related packages.
The present package differs from the above solutions in that
a document structure constructed with the conventional |\include| mechanism
just needs two extra commands at the top of every file
such that all constituent files can be compiled individually.

%%%%%%%%%%%%%%%%%%%%%%%%%%%%%%%%%%%%%%%%%%%%%%%%%%%%%%%%%%%%%%%%%%%%%%%%%%%%%%%%
%\subsection{Feature Suggestions}
%
%The following is a list of features which may be useful for future
%versions of this package:
%%
%\begin{itemize}
%\item
%\ldots
%\end{itemize}

%%%%%%%%%%%%%%%%%%%%%%%%%%%%%%%%%%%%%%%%%%%%%%%%%%%%%%%%%%%%%%%%%%%%%%%%%%%%%%%%
\subsection{Revision History}

%%%%%%%%%%%%%%%%%%%%%%%%%%%%%%%%%%%%%%%%
\paragraph{v2.0:} 2018/12/30

\begin{itemize}
\item
immediate forward processing
\item
added |\childdocby| mechanism
\item
manual restructured
\end{itemize}

%%%%%%%%%%%%%%%%%%%%%%%%%%%%%%%%%%%%%%%%
\paragraph{v1.6:} 2018/01/17

\begin{itemize}
\item
application for development of include files
\item
corrections to manual
\end{itemize}

%%%%%%%%%%%%%%%%%%%%%%%%%%%%%%%%%%%%%%%%
\paragraph{v1.5:} 2017/05/21

\begin{itemize}
\item
more complete structuring introduced
\item
|\childdocof| introduced
\item
|\childdoc| renamed to |\childdocmain|
\item
|\childredirect| renamed to |\childdocforward| and |\childdocforwardprefix|
and functionality expanded
\end{itemize}

%%%%%%%%%%%%%%%%%%%%%%%%%%%%%%%%%%%%%%%%
\paragraph{v1.0:} 2017/04/27

\begin{itemize}
\item
manual and install package
\item
first version published on CTAN
\end{itemize}

%%%%%%%%%%%%%%%%%%%%%%%%%%%%%%%%%%%%%%%%
\paragraph{v0.6:} 2017/04/26

\begin{itemize}
\item
redirection mechanism added
\end{itemize}

%%%%%%%%%%%%%%%%%%%%%%%%%%%%%%%%%%%%%%%%
\paragraph{v0.5:} 2017/04/26

\begin{itemize}
\item
functionality in definition file
\end{itemize}


%%%%%%%%%%%%%%%%%%%%%%%%%%%%%%%%%%%%%%%%%%%%%%%%%%%%%%%%%%%%%%%%%%%%%%%%%%%%%%%%
%%%%%%%%%%%%%%%%%%%%%%%%%%%%%%%%%%%%%%%%%%%%%%%%%%%%%%%%%%%%%%%%%%%%%%%%%%%%%%%%
%%%%%%%%%%%%%%%%%%%%%%%%%%%%%%%%%%%%%%%%%%%%%%%%%%%%%%%%%%%%%%%%%%%%%%%%%%%%%%%%
\appendix

\settowidth\MacroIndent{\rmfamily\scriptsize 000\ }

 \DocInput{childdoc.dtx}

\end{document}
%</driver>
% \fi
%
% %%%%%%%%%%%%%%%%%%%%%%%%%%%%%%%%%%%%%%%%%%%%%%%%%%%%%%%%%%%%%%%%%%%%%%%%%%%%%%
% %%%%%%%%%%%%%%%%%%%%%%%%%%%%%%%%%%%%%%%%%%%%%%%%%%%%%%%%%%%%%%%%%%%%%%%%%%%%%%
% \section{Sample}
%\iffalse
%<*samplemain>
%\fi
%
% The following presents a sample document
% with two chapters, two parts, a title page,
% a compile flag as well as three forwarding files to set the flag.
% It consists of eight |.tex| files:
% \begin{center}
% \begin{tabular}{ll}
% |cdocsamp.tex|&main file\\
% |cdocsch1.tex|&include file for chapter 1\\
% |cdocsch2.tex|&include file for chapter 2\\
% |cdocspt3.tex|&include file for part 3\\
% |cdocspt4.tex|&include file for part 4\\
% |cdocsdrf.tex|&forwarding file for main file in draft mode\\
% |cdocsfi1.tex|&forwarding file for final version of chapter 1\\
% |cdocsfi2.tex|&forwarding file for final version of chapter 2\\
% \end{tabular}
% \end{center}
% Each of the eight files can be compiled directly by the \LaTeX{} compiler.
%
% %%%%%%%%%%%%%%%%%%%%%%%%%%%%%%%%%%%%%%
% \paragraph{Main File.}
%
% The main file is called |cdocsamp.tex|.
%
% Load the \textsf{childdoc} definitions and
% declare the filename for the main document:
%    \begin{macrocode}
\input{childdoc.def}
\childdocmain{}
%    \end{macrocode}

% Optional override for |\version| flag:
%    \begin{macrocode}
%%\ifchilddoc\else\providecommand{\version}{draft}\fi
%    \end{macrocode}

% Define the default values for the |\version| flag
% (|final| for the main file and |draft| for childs):
%    \begin{macrocode}
\ifchilddoc
\providecommand{\version}{draft}
\else
\providecommand{\version}{final}
\fi
%    \end{macrocode}

% Load the standard document class:
%    \begin{macrocode}
\documentclass[12pt]{article}
%    \end{macrocode}

% Start the document body:
%    \begin{macrocode}
\begin{document}
%    \end{macrocode}

% Declare a title page.
% Print title, part of document being processed and version flag:
%    \begin{macrocode}
\addtocounter{page}{-1}
\begin{center}
{\LARGE\bfseries{}childdoc example\par}
\vspace{1cm}
\ifchilddoc
\ifchilddocmanual part\else chapter\fi:
`\childdocname' of `\childdocjob'\par
\else
main document: `\childdocjob'\par
\fi
version: \version\par
\end{center}
\newpage
%    \end{macrocode}

% Manually include selected file,
% otherwise process as usual:
%    \begin{macrocode}
\ifchilddocmanual
\section*{part `\childdocname'}
\input{\childdocname}
\else
%    \end{macrocode}

% Include the two chapters:
%    \begin{macrocode}
\include{cdocsch1}
\include{cdocsch2}
%    \end{macrocode}

% Include the two parts unless only chapters should be displayed:
%    \begin{macrocode}
\ifchilddoc\else
\section{part three}
\input{cdocspt3}
\section{part four}
\input{cdocspt4}
\fi
%    \end{macrocode}

% Process as usual until here:
%    \begin{macrocode}
\fi
%    \end{macrocode}

% End of document body:
%    \begin{macrocode}
\end{document}
%    \end{macrocode}
%\iffalse
%</samplemain>
%\fi
%
% %%%%%%%%%%%%%%%%%%%%%%%%%%%%%%%%%%%%%%
% \paragraph{Chapter Include Files.}
%
% The include files are called |cdocsch1.tex| and |cdocsch2.tex|.
%
%\iffalse
%<*samplechap1|samplechap2>
%\fi

% Optional override for |\version| flag:
%    \begin{macrocode}
%%\providecommand{\version}{final}
%    \end{macrocode}

% Include the main document:
%    \begin{macrocode}
\input{childdoc.def}
\childdocof{cdocsamp}
%    \end{macrocode}

%\iffalse
%</samplechap1|samplechap2>
%\fi
%
%\iffalse
%<*samplechap1>
%\fi
% Some text for chapter 1:
%    \begin{macrocode}
\section{one}
some text in chapter one
%    \end{macrocode}

%\iffalse
%</samplechap1>
%\fi
% Some text for chapter 2:
%\iffalse
%<*samplechap2>
%\fi
%    \begin{macrocode}
\section{two}
more text in chapter two
%    \end{macrocode}

%\iffalse
%</samplechap2>
%\fi
%
% %%%%%%%%%%%%%%%%%%%%%%%%%%%%%%%%%%%%%%
% \paragraph{Part Include Files.}
%
% The include files are called |cdocspt3.tex| and |cdocspt4.tex|.
%
%\iffalse
%<*samplepart3|samplepart4>
%\fi

% Optional override for |\version| flag:
%    \begin{macrocode}
%%\providecommand{\version}{final}
%    \end{macrocode}

% Include the main document:
%    \begin{macrocode}
\input{childdoc.def}
\childdocby{cdocsamp}
%    \end{macrocode}

%\iffalse
%</samplepart3|samplepart4>
%\fi
%
%\iffalse
%<*samplepart3>
%\fi
% Some text for part 3:
%    \begin{macrocode}
some text in part three
%    \end{macrocode}

%\iffalse
%</samplepart3>
%\fi
% Some text for part 4:
%\iffalse
%<*samplepart4>
%\fi
%    \begin{macrocode}
more text in part four
%    \end{macrocode}

%\iffalse
%</samplepart4>
%\fi
%
% %%%%%%%%%%%%%%%%%%%%%%%%%%%%%%%%%%%%%%
% \paragraph{Forwarding for a Complete Draft.}
%
% The following forwarding file |cdocsdrf.tex|
% compiles the main document in draft mode:
%\iffalse
%<*sampledraft>
%\fi
%    \begin{macrocode}
\def\version{draft}
\input{childdoc.def}
\childdocforward{cdocsamp}
%    \end{macrocode}

%\iffalse
%</sampledraft>
%\fi
%
% %%%%%%%%%%%%%%%%%%%%%%%%%%%%%%%%%%%%%%
% \paragraph{Forwarding for Final Version of the Chapters.}
%
% The following forwarding files |cdocsfn1.tex| and |cdocsfn2.tex|
% (with identical content)
% compile the final versions of the child documents
% |cdocsch1.tex| and |cdocsch2.tex|, respectively:
%\iffalse
%<*samplefinal>
%\fi
%    \begin{macrocode}
\def\version{final}
\input{childdoc.def}
\childdocforwardprefix[cdocsamp]{cdocsfn}{cdocsch}
%    \end{macrocode}

%\iffalse
%</samplefinal>
%\fi
%
% %%%%%%%%%%%%%%%%%%%%%%%%%%%%%%%%%%%%%%
% \paragraph{Command Line Processing.}
%
% The following three command lines generate the output files
% |cdocscld|, |cdocscl1| and |cdocscl2|
% which should be identical to
% |cdocsdrf|, |cdocsch1| and |cdocsfn2|, respectively:
% \begin{center}
% \begin{tabular}{l}
% |latex -jobname cdocscld \|\\
% |  "\def\version{draft}\input{childdoc.def}\childdocforward{cdocsamp}"|\\
% |latex -jobname cdocscl1 \|\\
% |  "\input{childdoc.def}\childdocforward[cdocsamp]{cdocsch1}"|\\
% |latex -jobname cdocscl2 \|\\
% |  "\def\version{final}\input{childdoc.def}\childdocforward{cdocsch2}"|
% \end{tabular}
% \end{center}
% Note that the trailing backslash on each first line
% merely continues the input to the second line
% (for convenient cut ant paste).
% Furthermore, the command |latex| can be replaced by any
% of its alternative versions such as |pdflatex|.
%
% %%%%%%%%%%%%%%%%%%%%%%%%%%%%%%%%%%%%%%%%%%%%%%%%%%%%%%%%%%%%%%%%%%%%%%%%%%%%%%
% %%%%%%%%%%%%%%%%%%%%%%%%%%%%%%%%%%%%%%%%%%%%%%%%%%%%%%%%%%%%%%%%%%%%%%%%%%%%%%
% \section{Implementation}
%\iffalse
%<*package>
%\fi
%
% This section describes the definitions file |childdoc.def|.

% The definitions cannot be loaded using |\usepackage| or |\RequirePackage|
% which has a mechanism to prevent loading a style file more than once.
% When loading the definitions by means of |\input|
% multiple instances have to be prevented manually:
%\iffalse
%This code needs to be before the `\ProvidesFile' directive
%which is defined at the beginning of this file.
%Therefore it is also placed there and commented out here.
%</package>
%<*discard>
%\fi
%    \begin{macrocode}
\ifdefined\childdocmain\endinput\fi
%    \end{macrocode}
%\iffalse
%</discard>
%<*package>
%\fi
%
% \macro{\ifchilddoc}
% \macro{\ifchilddocmanual}
% The conditional |\ifchilddoc| tells whether a
% child (true) or main (false) document is being compiled.
% The conditional |\ifchilddocmanual| tells whether
% the |\includeonly| mechanism is used (false) or
% the selection of child files must be performed manually (true).
% The definitions initialise to false:
%    \begin{macrocode}
\newif\ifchilddoc
\newif\ifchilddocmanual
%    \end{macrocode}

% \macro{\childdocname}
% \macro{\childdocjob}
% The macro |\childdocname| stores the name of the main document
% to be compiled. The macro |\childdocjob| stores the name of
% the document on which the \LaTeX{} compiler was originally invoked.
% The content of |\jobname| cannot be compared
% to filenames specified in the source due to different catcodes.
% The following code rescans |\jobname|, stores the result
% in |\childdocname| and saves a copy in |\childdocjob|:
%    \begin{macrocode}
\edef\childdocname{\scantokens\expandafter{\jobname\noexpand}}
\let\childdocjob\childdocname
%    \end{macrocode}

% \macro{\childdocdisable}
% The macro |\childdocdisable| prevents the main file
% from being processed more than once.
% At this stage, the main document command |\childdocmain|
% is assumed to be called once again where it should do nothing.
% Any subsequent call to it should prevent
% a secondary processing of the main document
% It overwrites the forwarding commands
% |\childdocof| and |\childdocforward|
% with empty macros to prevent further inclusions of the main document:
%    \begin{macrocode}
\newcommand{\childdocdisable}
{
  \renewcommand{\childdocmain}[1]{\renewcommand{\childdocmain}[1]{\endinput}}
  \renewcommand{\childdocof}[1]{}
  \renewcommand{\childdocby}[2][]{}
  \renewcommand{\childdocforward}[2][]{}
  \renewcommand{\childdocdisable}{}
}
%    \end{macrocode}

% \macro{\childdocmain}
% The macro |\childdocmain| is to be called at the top of the main file
% with nothing or the main filename (without extension) as argument.
% First, it breaks loops.
% If the argument is not empty and does not match |\childdocname|
% (which is set by the first inclusion of |childdoc.def|),
% |\ifchilddoc| is set to true, |\includeonly| is applied to the child file
% and |\jobname| is set to the main file
% (for proper handling of |.aux| files):
%    \begin{macrocode}
\newcommand{\childdocmain}[1]
{
  \childdocdisable\childdocmain{}
  \if?#1?\else
    \begingroup
      \def\childdoctmp{#1}
      \ifx\childdoctmp\childdocname
        \def\childdoctmp{}
      \else
        \def\childdoctmp
        {
          \childdoctrue
          \includeonly{\childdocname}
          \def\childdocjob{#1}
          \def\jobname{#1}
        }
      \fi
      \expandafter
    \endgroup
    \childdoctmp
  \fi
}
%    \end{macrocode}

% \macro{\childdocof}
% The command |\childdocof| redirects
% compilation to the main file |#1|.
%    \begin{macrocode}
\newcommand{\childdocof}[1]
{
  \childdocdisable
  \childdoctrue
  \includeonly{\childdocname}
  \def\jobname{#1}
  \def\childdocjob{#1}
  \input{#1}
}
%    \end{macrocode}

% \macro{\childdocby}
% The command |\childdocby| ....
%    \begin{macrocode}
\newcommand{\childdocby}[2][]
{
  \childdocdisable
  \childdoctrue
  \childdocmanualtrue
  \if?#1?\else
    \def\jobname{#2}
  \fi
  \def\childdocjob{#2}
  \input{#2}
  \endinput
}
%    \end{macrocode}

% \macro{\childdocforward}
% The command |\childdocforward| redirects
% compilation to the main file or
% (if the optional argument is given) a child file.
% Parameters are set as if the main file
% or a child file starting with |\childdocof| was compiled.
% Then compilation is handed over to the main file:
%    \begin{macrocode}
\newcommand{\childdocforward}[2][]
{
  \begingroup
    \if?#1?
      \def\childdoctmp
      {
        \def\childdocname{#2}
        \def\childdocjob{#2}
        \def\jobname{#2}
        \input{#2}
        \endinput
      }
    \else
      \def\childdoctmp
      {
        \childdocdisable
        \def\childdocname{#2}
        \childdoctrue
        \includeonly{#2}
        \def\childdocjob{#1}
        \def\jobname{#1}
        \input{#1}
        \endinput
      }
    \fi
    \expandafter
  \endgroup
  \childdoctmp
}
%    \end{macrocode}

% \macro{\childdocforwardprefix}
% The command |\childdocforwardprefix| redirects
% compilation to the main or a child file by means of a pattern.
% The prefix |#1| in the current filename is replaced by |#2|
% and the suffix of the current filename is kept
% (it is assumed that the filename does not contain the substring `|~~~|'
% which is used as a delimiter).
% Compilation is handed over to the new file by |\childdocforward|:
%    \begin{macrocode}
\newcommand{\childdocforwardprefix}[3][]
{
  \begingroup
    \def\childdocextract #2##1~~~{\def\childdoctmp{\childdocforward[#1]{#3##1}}}
    \expandafter\childdocextract\childdocname~~~
    \expandafter
  \endgroup
  \childdoctmp
}
%    \end{macrocode}

% \macro{\childdoc}
% The deprecated macro |\childdoc| is a legacy version of |\childdocmain|:
%    \begin{macrocode}
\newcommand{\childdoc}{\childdocmain}
%    \end{macrocode}

% \macro{\childdocredirect}
% The deprecated macro |\childdocredirect| is a legacy version
% of |\childdocforward| and |\childdocforwardprefix|:
%    \begin{macrocode}
\newcommand{\childdocredirect}[2][]
{
  \begingroup
    \if?#1?
      \def\childdoctmp{\childdocforward{#2}}
    \else
      \def\childdoctmp{\childdocforwardprefix{#1}{#2}}
    \fi
    \expandafter
  \endgroup
  \childdoctmp
}
%    \end{macrocode}

%\iffalse
%</package>
%\fi
%
\endinput
\childdocforward[|\textit{main}|]{|\textit{dest}|}"|
\end{center}
%
Here \textit{target} is the name of the output file,
\textit{main} is the name of the main file
and \textit{dest} is the name of the main or child file to be processed
(all filenames without extensions).
The optional argument \textit{main} can be omitted
if \textit{main} matches \textit{dest}.
Optionally, compilation \textit{flags} can be defined via |\def| commands.
This command line makes the \TeX{} engine believe
it is compiling the file \textit{target}
whose content is specified as the latter parameter.
The provided code then forwards the processing to
\textit{main} or \textit{dest} as described in \secref{sec:forward}.

%%%%%%%%%%%%%%%%%%%%%%%%%%%%%%%%%%%%%%%%%%%%%%%%%%%%%%%%%%%%%%%%%%%%%%%%%%%%%%%%
\subsection{Include by Input}
\label{sec:input}

Including child documents by |\include| has some restrictions by design.
Most notably, the content of a child document always occupies
its own set of pages; pages cannot be shared between child documents.
Usually, this behaviour makes perfect sense
because each child document contain an essential part of the document.
However, in some situations it may be desirable to compose
a document from a collection of parts
without having mandatory page breaks between then.
For this case, the package
provides a mechanism to include parts
by |\input| which can also be processed individually.
However, by construction this mechanism
requires manual handling of the content to be output.

%%%%%%%%%%%%%%%%%%%%%%%%%%%%%%%%%%%%%%%%
\DescribeMacro{\ifchilddocmanual}
The main file should be prepared as usual, see \secref{sec:include}.
However, the document body must make a distinction
between processing of an individual part and of the main document, e.g.:
%
\begin{center}
\begin{tabular}{l}
|\ifchilddocmanual|\\
|\input{\childdocname}|\\
|\||else|\\
\textit{document body with }|\input{|\textit{part}|}|\\
|\||fi|
\end{tabular}
\end{center}
%
The conditional |\ifchilddocmanual| is true whenever
a part to be included by |\input| is being compiled,
and the name of the part is stored in |\childdocname|.

%%%%%%%%%%%%%%%%%%%%%%%%%%%%%%%%%%%%%%%%
\DescribeMacro{\childdocby}
Each part to be included by |\input| should start with:
%
\begin{center}
\begin{tabular}{l}
|% \iffalse
%
% childdoc.dtx Copyright (C) 2017-2018 Niklas Beisert
%
% This work may be distributed and/or modified under the
% conditions of the LaTeX Project Public License, either version 1.3
% of this license or (at your option) any later version.
% The latest version of this license is in
%   http://www.latex-project.org/lppl.txt
% and version 1.3 or later is part of all distributions of LaTeX
% version 2005/12/01 or later.
%
% This work has the LPPL maintenance status `maintained'.
%
% The Current Maintainer of this work is Niklas Beisert.
%
% This work consists of the files childdoc.dtx and childdoc.ins
% and the derived files childdoc.def and cdocsamp.tex with
% cdocsch1.tex, cdocsch2.tex, cdocsdrf.tex, cdocsfn1.tex, cdocsfn2.tex.
%
%<package>\ifdefined\childdocmain\endinput\fi
%<package>\ProvidesFile{childdoc.def}[2018/12/30 v2.0 child document driver]
%<samplemain>\ProvidesFile{cdocsamp.tex}[2018/12/30 v2.0 sample for childdoc]
%<*driver>
%\ProvidesFile{childdoc.drv}[2018/12/30 v2.0 childdoc reference manual file]
\PassOptionsToClass{10pt,a4paper}{article}
\documentclass{ltxdoc}

\usepackage[margin=35mm]{geometry}
\usepackage{hyperref}
\usepackage{hyperxmp}
\usepackage[usenames]{color}

\hypersetup{colorlinks=true}
\hypersetup{pdfstartview=FitH}
\hypersetup{pdfpagemode=UseNone}
\hypersetup{pdfsource={}}
\hypersetup{pdflang={en-UK}}
\hypersetup{pdfcopyright={Copyright 2017-2018 Niklas Beisert.
  This work may be distributed and/or modified under the
  conditions of the LaTeX Project Public License, either version 1.3
  of this license or (at your option) any later version.}}
\hypersetup{pdflicenseurl={http://www.latex-project.org/lppl.txt}}
\hypersetup{pdfcontactaddress={ETH Zurich, ITP, HIT K,
  Wolfgang-Pauli-Strasse 27}}
\hypersetup{pdfcontactpostcode={8093}}
\hypersetup{pdfcontactcity={Zurich}}
\hypersetup{pdfcontactcountry={Switzerland}}
\hypersetup{pdfcontactemail={nbeisert@itp.phys.ethz.ch}}
\hypersetup{pdfcontacturl={http://people.phys.ethz.ch/\xmptilde nbeisert/}}

\newcommand{\secref}[1]{\hyperref[#1]{section \ref*{#1}}}

\parskip1ex
\parindent0pt
\let\olditemize\itemize
\def\itemize{\olditemize\parskip0pt}

\begin{document}

\title{The \textsf{childdoc} Package}
\hypersetup{pdftitle={The childdoc Package}}
\author{Niklas Beisert\\[2ex]
  Institut f\"ur Theoretische Physik\\
  Eidgen\"ossische Technische Hochschule Z\"urich\\
  Wolfgang-Pauli-Strasse 27, 8093 Z\"urich, Switzerland\\[1ex]
  \href{mailto:nbeisert@itp.phys.ethz.ch}
  {\texttt{nbeisert@itp.phys.ethz.ch}}}
\hypersetup{pdfauthor={Niklas Beisert}}
\hypersetup{pdfsubject={Manual for the LaTeX2e Package childdoc}}
\date{30 December 2018, \textsf{v2.0}}
\maketitle

\begin{abstract}\noindent
\textsf{childdoc} is a \LaTeXe{} package
that enables the direct compilation
of document sections included by |\include|
to individual files.
\end{abstract}

\begingroup
\parskip0ex
\tableofcontents
\endgroup

%%%%%%%%%%%%%%%%%%%%%%%%%%%%%%%%%%%%%%%%%%%%%%%%%%%%%%%%%%%%%%%%%%%%%%%%%%%%%%%%
%%%%%%%%%%%%%%%%%%%%%%%%%%%%%%%%%%%%%%%%%%%%%%%%%%%%%%%%%%%%%%%%%%%%%%%%%%%%%%%%
\section{Introduction}

\LaTeX{} provides a mechanism to structure a large document (such as a book)
into a main file and several child files (containing the chapters)
using the |\include| command.
This mechanism is beneficial for documents
which span hundreds of pages in order to
make the source file(s) more manageable.
Moreover, compilation can be restricted to
selected child files by means of the |\includeonly| command.
The latter feature can be used to reduce the compilation time while editing
(this was significantly more useful in the earlier days of \LaTeX{})
or to generate a smaller document which is easier to navigate.
Another application of |\includeonly| is to generate
documents consisting of selected parts of the complete document.

However, there are a few drawbacks of the plain |\include| mechanism:
\begin{itemize}
\item
The child files cannot be compiled on their own,
they can only be compiled via the main file.
A naive editing environment
(such as a text editor with an option
to have the current file processed by \LaTeX)
may require one to switch to the main file before compiling;
attempting to compile the child file produces errors.
\item
The main file must be modified (each time)
to adjust the |\includeonly| command
to the present needs. This easily leaves the main file in a messy state.
\item
The generated document will always carry the filename
of the main document. This is inconvenient if
several child files are to be compiled and
to be kept for distribution.
\end{itemize}

The present package provides a simple interface
to make child files individually compilable by \LaTeX{}.
Compiling a child file then has the same effect as compiling
the main file with an |\includeonly| command
to select the appropriate child.
Moreover the generated document will carry the name of the child
rather than the main file.
This resolves all three above issues.

This feature is meant to make the editing of books,
thesis documents and lecture notes somewhat more convenient.
However, the package can also be used efficiently for
composing a series of documents (such as exercise sheets)
which are typically distributed individually.
It then assists the author in generating the individual documents
(potentially in different versions)
as well as a document containing the collected series.
Another application is in developing style files
or other kinds of included material
where compilation of the style file could redirect
to a sample or test file.

%%%%%%%%%%%%%%%%%%%%%%%%%%%%%%%%%%%%%%%%%%%%%%%%%%%%%%%%%%%%%%%%%%%%%%%%%%%%%%%%
%%%%%%%%%%%%%%%%%%%%%%%%%%%%%%%%%%%%%%%%%%%%%%%%%%%%%%%%%%%%%%%%%%%%%%%%%%%%%%%%
\section{Usage}

First of all, the package \textsf{childdoc} is \emph{not} a standard
\LaTeXe{} |.sty| style file! Therefore it needs to be invoked in
a non-standard way.

%%%%%%%%%%%%%%%%%%%%%%%%%%%%%%%%%%%%%%%%%%%%%%%%%%%%%%%%%%%%%%%%%%%%%%%%%%%%%%%%
\subsection{Included Files}
\label{sec:include}

%%%%%%%%%%%%%%%%%%%%%%%%%%%%%%%%%%%%%%%%
\DescribeMacro{\childdocmain}
To use the package, add the commands
\begin{center}
\begin{tabular}{l}
|\input{childdoc.def}|\\
|\childdocmain{}|\\
\end{tabular}
\end{center}
at the very top of the main \LaTeX{} file,
in particular \emph{before} the |\documentclass| statement!
The argument of |\childdocmain| should be left empty
(but it must be present).

%%%%%%%%%%%%%%%%%%%%%%%%%%%%%%%%%%%%%%%%
\DescribeMacro{\childdocof}
Furthermore, add the commands
\begin{center}
\begin{tabular}{l}
|\input{childdoc.def}|\\
|\childdocof{|\textit{main}|}|\\
\end{tabular}
\end{center}
at the top of every child file \textit{child}
which is included by |\include{|\textit{child}|}|
from within the main file
(or at least for those files to be compiled individually).
The argument \textit{main} must be the filename of the main file.

There are a couple of
considerations in setting up the main and child documents:

%%%%%%%%%%%%%%%%%%%%%%%%%%%%%%%%%%%%%%%%
\paragraph{Restrictions.}

Please note the following restrictions:
\begin{itemize}
\item
|\childdocmain| must be called with one argument \textit{main}
to ensure compatibility with earlier version of the package.
It must either be empty (|\childdocmain{}|)
or precisely match the filename of the main file in which it is specified.
See \secref{sec:detection} for further information.
\item
The filename \textit{main} must be specified without the |.tex| extension.
\item
The filename \textit{main} is case sensitive
(even in case-insensitive file systems)
due to internal string comparison.
\item
The argument \textit{main} should be fully expanded, it cannot be a macro.
\item
Subdirectories and special characters should be avoided in filenames.
\item
The command |\childdocmain{|\textit{main}|}| must be followed by a whitespace.
It should not be followed immediately by another command
or by a comment mark `|%|'.
This is because the \TeX{} parser reads the token immediately following
the argument of |\childdocmain| and puts it
at the beginning of every child section;
however, a white\-space is ignored.
\end{itemize}

%%%%%%%%%%%%%%%%%%%%%%%%%%%%%%%%%%%%%%%%
\paragraph{Content of Main File.}

It is advisable to place all content in the child files included by |\include|.
Any output contained in the main file will appear in all child documents
unless suppressed manually;
it cannot be suppressed automatically by the |\includeonly| directive
and thus should normally be avoided.
A method to include some content in the main file
by means of conditional processing is described in \secref{sec:conditional}.

%%%%%%%%%%%%%%%%%%%%%%%%%%%%%%%%%%%%%%%%
\paragraph{Page Numbering.}

When only a part of the document is compiled,
the appropriate numbering of pages
(as well as other status parameters)
is determined from the |.aux| files.
The latter contain information from previous passes.
However this information needs to propagate through
all intermediate child documents.
Therefore the page numbering in child documents may well
be inconsistent until the complete document is compiled at least once.

A useful (if unconventional) way to always ensure a consistent
page numbering is to restart the numbering in each child document
and denote the pages by `\textit{child}|.|\textit{page}'
where \textit{child} represents the chapter/section number of the child file.
This can be achieved by the command
|\numberwithin{page}{|\textit{child}|}|
of the \textsf{amsmath} package
where \textit{child} can be |chapter| or |section|
depending on the chosen structuring.
Alternatively, one can modify the macro |\thepage| appropriately
and reset the counter |page| at the start of each child file.

%%%%%%%%%%%%%%%%%%%%%%%%%%%%%%%%%%%%%%%%%%%%%%%%%%%%%%%%%%%%%%%%%%%%%%%%%%%%%%%%
\subsection{Conditional Processing}
\label{sec:conditional}

The package provides a mechanism to compile different versions
of a document. To customise the versions further some conditional processing
can come in handy to distinguish which version is being compiled.
The package provides two macros to describe the compilation context:

%%%%%%%%%%%%%%%%%%%%%%%%%%%%%%%%%%%%%%%%
\DescribeMacro{\ifchilddoc}
The conditional |\ifchilddoc| distinguishes between the compilation of
child documents and the main document:
%
\begin{center}
|\ifchilddoc |\textit{child-code}| |[|\||else |\textit{main-code}]| \||fi|
\end{center}

%%%%%%%%%%%%%%%%%%%%%%%%%%%%%%%%%%%%%%%%
\DescribeMacro{\childdocname}
\DescribeMacro{\childdocjob}
The macro |\childdocname| contains the filename (without extension)
of the main or child file being processed.
Note that |\childdocjob| will always contain the name of the main file.

%%%%%%%%%%%%%%%%%%%%%%%%%%%%%%%%%%%%%%%%
\paragraph{Title Page.}

Conditional processing can be used to include a title or banner page
in the main document when proper precautions are taken.
Importantly, the code in the main file should ensure that the page counter
(as well as other status parameters which are stored in the |.aux| files)
takes the same value after the conditional processing.
Otherwise the page numbers may take divergent values
depending on which part is compiled.

For example, a title page could be declared by:
%
\begin{center}
\begin{tabular}{l}
|\ifchilddoc\||else|\\
|\addtocounter{page}{-1}|\\
\textit{code for title page}\\
|\newpage|\\
|\||fi|
\end{tabular}
\end{center}
%
A banner page for the child documents can be generated by:
%
\begin{center}
\begin{tabular}{l}
|\ifchilddoc|\\
|\addtocounter{page}{-1}|\\
\textit{code for banner page}\\
|\newpage|\\
|\||fi|
\end{tabular}
\end{center}
%
Here one could write a message such as:
\begin{center}
|This is the part \childdocname{} of \childdocjob{}.|
\end{center}

%%%%%%%%%%%%%%%%%%%%%%%%%%%%%%%%%%%%%%%%%%%%%%%%%%%%%%%%%%%%%%%%%%%%%%%%%%%%%%%%
\subsection{Flags}
\label{sec:flags}

The package makes it easy to generate different versions
of the main or child documents.
To this end compilation flags can be defined
and assigned different default values.
They will be particularly useful in conjunction
with the forwarding mechanism described in \secref{sec:forward}.

For example, it may be useful to have a flag |\version|
which can be set to |draft| or |final|.
The document source will contain some conditional code
depending on the value of |\version|.
Suppose further, the flag should default to |final| for the main file
and to |draft| for child files
which is a natural assignment for editing the document.
This is achieved by placing the following code
in the preamble of the main document
(below the |\childdocmain| directive):
%
\begin{center}
\begin{tabular}{l}
|\ifchilddoc|\\
|\providecommand{\version}{draft}|\\
|\||else|\\
|\providecommand{\version}{final}|\\
|\||fi|
\end{tabular}
\end{center}
%
The definition by |\providecommand| makes sure
that previous definitions are not overwritten.
Further statements |\providecommand{\version}{...}|
can thus be added before the above code to override it.

For the main file, one might add a line
(between |\childdocmain| and the above block)
%
\begin{center}
|%\ifchilddoc\||else\providecommand{\version}{draft}\||fi|
\end{center}
%
which can be uncommented to produce a draft version.
Likewise one can add a line to the very top of a child file
(above the |\childdocof{|\textit{main}|}| directive)
%
\begin{center}
|%\providecommand{\version}{final}|
\end{center}
%
which can be uncommented to produce the final version of this child document.

%%%%%%%%%%%%%%%%%%%%%%%%%%%%%%%%%%%%%%%%%%%%%%%%%%%%%%%%%%%%%%%%%%%%%%%%%%%%%%%%
\subsection{Forwarding}
\label{sec:forward}

Different versions of the main or child documents
using compilation flags as described in \secref{sec:flags}
can be (permanently) stored in different files
for convenient compilation, viewing and distribution.
To this end, the package defines a command
to pass on compilation to a different file:

%%%%%%%%%%%%%%%%%%%%%%%%%%%%%%%%%%%%%%%%
\DescribeMacro{\childdocforward}
The command |\childdocforward| redirects processing to
another source file:
%
\begin{center}
\begin{tabular}{l}
|\input{childdoc.def}|\\
|\childdocforward[|\textit{main}|]{|\textit{dest}|}|\\
\end{tabular}
\end{center}
%
The argument \textit{dest} is the destination file
(without extension).
It should be the main file or one of the child files.
Note that further \textsf{childdoc} directives
such as |\childdocof| and |\childdocforward|
in the indicated file will be processed in this form.
The optional argument \textit{main}
passes on directly to the main file \textit{main}
while pretending to compile the child \textit{dest}.
This form behaves as if \textit{dest}
issues |\childdocof{|\textit{main}|}| right away,
and no further \textsf{childdoc} directives will be processed.

%%%%%%%%%%%%%%%%%%%%%%%%%%%%%%%%%%%%%%%%
\DescribeMacro{\...prefix}
In the alternative form |\childdocforwardprefix|,
%
\begin{center}
\begin{tabular}{l}
|\input{childdoc.def}|\\
|\childdocforwardprefix[|\textit{main}|]{|\textit{prefix}|}{|\textit{dest}|}|
\end{tabular}
\end{center}
%
the destination file is determined by a pattern
depending on the current file:
To make this work, the current file must be called
`{\textit{prefix}\hspace{0.2em}\textit{suffix}}'
with \textit{prefix} matching precisely the argument.
Processing is then passed on to the file
`{\textit{dest}\hspace{0.2em}\textit{suffix}}'.
Surely, the same effect is achieved by
directly specifying the
argument `{\textit{dest}\hspace{0.2em}\textit{suffix}}'
in the first form.
However, that requires to set up a different file
for each child. With the alternative form of the command
all these files can have exactly the same content
which simplifies setting them up and maintaining them.

For example, the following file |draft.tex|
with a compilation flag |\version| as described in \secref{sec:flags}
compiles the main document as a draft:
%
\begin{center}
\begin{tabular}{l}
|\def\version{draft}|\\
|\input{childdoc.def}|\\
|\childdocforward{|\textit{main}|}|
\end{tabular}
\end{center}
%
Likewise, the following files |final|\textit{nn}|.tex|
compile the final version of the child document
|child|\textit{nn}|.tex|:
%
\begin{center}
\begin{tabular}{l}
|\def\version{final}|\\
|\input{childdoc.def}|\\
|\childdocforwardprefix{final}{child}|
\end{tabular}
\end{center}
%

Note that when several versions of a main file and/or of each child file
are to be generated, it may be convenient to set up a |Makefile| or
shell script to automatise the process.

%%%%%%%%%%%%%%%%%%%%%%%%%%%%%%%%%%%%%%%%%%%%%%%%%%%%%%%%%%%%%%%%%%%%%%%%%%%%%%%%
\subsection{Command Line Processing}
\label{sec:commandline}

The effect of redirection files can also be achieved by invoking
the \LaTeX{} compiler with a more elaborate command line.
Most conveniently this should be done as part
of a shell script or a |Makefile|.

When using \textsf{childdoc} in the main file, the following
command lines effectively perform a redirection
(note that depending on the shell being used,
backslashes may have to be doubled: `|\|' $\to$ `|\\|'):
%
\begin{center}
|... -jobname "|\textit{target}|" |\\|"|[\textit{flags}]%
|\input{childdoc.def}\childdocforward[|\textit{main}|]{|\textit{dest}|}"|
\end{center}
%
Here \textit{target} is the name of the output file,
\textit{main} is the name of the main file
and \textit{dest} is the name of the main or child file to be processed
(all filenames without extensions).
The optional argument \textit{main} can be omitted
if \textit{main} matches \textit{dest}.
Optionally, compilation \textit{flags} can be defined via |\def| commands.
This command line makes the \TeX{} engine believe
it is compiling the file \textit{target}
whose content is specified as the latter parameter.
The provided code then forwards the processing to
\textit{main} or \textit{dest} as described in \secref{sec:forward}.

%%%%%%%%%%%%%%%%%%%%%%%%%%%%%%%%%%%%%%%%%%%%%%%%%%%%%%%%%%%%%%%%%%%%%%%%%%%%%%%%
\subsection{Include by Input}
\label{sec:input}

Including child documents by |\include| has some restrictions by design.
Most notably, the content of a child document always occupies
its own set of pages; pages cannot be shared between child documents.
Usually, this behaviour makes perfect sense
because each child document contain an essential part of the document.
However, in some situations it may be desirable to compose
a document from a collection of parts
without having mandatory page breaks between then.
For this case, the package
provides a mechanism to include parts
by |\input| which can also be processed individually.
However, by construction this mechanism
requires manual handling of the content to be output.

%%%%%%%%%%%%%%%%%%%%%%%%%%%%%%%%%%%%%%%%
\DescribeMacro{\ifchilddocmanual}
The main file should be prepared as usual, see \secref{sec:include}.
However, the document body must make a distinction
between processing of an individual part and of the main document, e.g.:
%
\begin{center}
\begin{tabular}{l}
|\ifchilddocmanual|\\
|\input{\childdocname}|\\
|\||else|\\
\textit{document body with }|\input{|\textit{part}|}|\\
|\||fi|
\end{tabular}
\end{center}
%
The conditional |\ifchilddocmanual| is true whenever
a part to be included by |\input| is being compiled,
and the name of the part is stored in |\childdocname|.

%%%%%%%%%%%%%%%%%%%%%%%%%%%%%%%%%%%%%%%%
\DescribeMacro{\childdocby}
Each part to be included by |\input| should start with:
%
\begin{center}
\begin{tabular}{l}
|\input{childdoc.def}|\\
|\childdocby{|\textit{main}|}|\\
\end{tabular}
\end{center}
%
The directive |\childdocby| is similar to |\childdocof|
described in \secref{sec:include},
but the subsequent selection of content must be done manually.
To that end, both |\ifchilddoc| and |\ifchilddocmanual|
will be true upon processing of a part,
and the name of the part is stored in |\childdocname|.
Note that |\jobname| will be set to the filename of the current part
so that each part receives an individual |.aux| file
that does not interfere with the |.aux| file(s) of the main document.
This behaviour can be altered by the alternative form
|\childdocby[*]{|\textit{main}|}| (with a non-empty optional argument)
which uses the |.aux| file of the main document
by setting |\jobname| to \textit{main}.

%%%%%%%%%%%%%%%%%%%%%%%%%%%%%%%%%%%%%%%%%%%%%%%%%%%%%%%%%%%%%%%%%%%%%%%%%%%%%%%%
\subsection{Driver Development}
\label{sec:driver}

The \textsf{childdoc} mechanism can also be use for the development
of definition files such as \LaTeX{} styles or classes.
This case differs from the above setup with multiple parts
included by |\include| in that no |\includeonly| should be invoked.
This can be achieved by starting the include file
(before |\ProvidesPackage|) with:
%
\begin{center}
\begin{tabular}{l}
|\input{childdoc.def}|\\
|\childdocforward{|\textit{main}|}|\\
\end{tabular}
\end{center}
%
or alternatively with:
%
\begin{center}
\begin{tabular}{l}
|\input{childdoc.def}|\\
|\childdocby{|\textit{main}|}|\\
\end{tabular}
\end{center}
%
Both forms have slightly different effects as described above.
The main file is prepared as usual, see \secref{sec:include}.

%%%%%%%%%%%%%%%%%%%%%%%%%%%%%%%%%%%%%%%%%%%%%%%%%%%%%%%%%%%%%%%%%%%%%%%%%%%%%%%%
\subsection{Legacy Detection}
\label{sec:detection}

The directive |\childdocmain| in the main file can detect
whether the complete document or merely a child is to be compiled
even without using the directive |\childdocof|.
This method is deprecated because it is less robust
and there is no compelling reason to use it;
it is merely provided for backward compatibility
and it may be removed in future versions.

If the detection mechanism is to be used,
it is mandatory to correctly specify
the filename of the main file as the argument of |\childdocmain|:
%
\begin{center}
\begin{tabular}{l}
|\input{childdoc.def}|\\
|\childdocmain{|\textit{main}|}|\\
\end{tabular}
\end{center}
%
If |\jobname| does not match the argument \textit{main} of |\childdocmain|,
it is assumed that |\jobname| points to the child file to be compiled.
When using |\childdocmain| with the main file specified as argument,
it suffices to start a child file
with just |\input{|\textit{main}|}|
without loading of the package and using |\childdocof|.
If instead all processing is done
with the appropriate \textsf{childdoc} directives,
the argument of \textit{main} of |\childdocmain| can be empty.

An alternative version of the command line processing described
in \secref{sec:commandline} using the detection mechanism reads:
%
\begin{center}
|... -jobname "|\textit{target}|" "|[\textit{flags}]%
[|\def\jobname{|\textit{dest}|}|]|\input{|\textit{main}|}"|
\end{center}

%%%%%%%%%%%%%%%%%%%%%%%%%%%%%%%%%%%%%%%%%%%%%%%%%%%%%%%%%%%%%%%%%%%%%%%%%%%%%%%%
\subsection{Manual Code}
\label{sec:manual}

In case one cannot be certain whether the definitions file |childdoc.def|
is installed on the target \TeX{} distribution
and one prefers not to ship it,
it is conceivable to paste a few relevant commands into the sources.

To that end, drop all statements |\input{childdoc.def}|
and perform the replacements as outlined below.
Instead of |\childdocmain{|\textit{main}|}| add the following code
to the top of the main file:
%
\begin{center}
\begin{tabular}{l}
|\||ifdefined\childdocname\endinput\||fi\newif\ifchilddoc|\\
|\edef\childdocname{\scantokens\expandafter{\jobname\noexpand}}|\\
|\def\childdocmain{|\textit{main}|}\||ifx\childdocmain\childdocname\||else|\\
|\childdoctrue\includeonly{\childdocname}\let\jobname\childdocmain\||fi|\\
\end{tabular}
\end{center}
%
Instead of |\childdocof{|\textit{main}|}| just include the main file
at the top of each child file:
%
\begin{center}
|\input{|\textit{main}|}|
\end{center}
%
A simple redirection |\childdocforward{|\textit{dest}|}| is achieved by:
%
\begin{center}
|\def\jobname{|\textit{dest}|}\input{\jobname}|
\end{center}
%
The redirection with prefix
|\childdocforwardprefix[|\textit{prefix}|]{|\textit{dest}|}|
is accomplished by:
%
\begin{center}
\begin{tabular}{l}
|{\edef\jobname{\scantokens\expandafter{\jobname\noexpand}}|\\
|\def\redirectjob |\textit{prefix}|#1~~~{\gdef\jobname{|\textit{dest}|#1}}|\\
|\expandafter\redirectjob\jobname~~~}\input{\jobname}|
\end{tabular}
\end{center}

In an alternative approach,
child documents can be compiled by a specific command line
without additional code or specific definitions:
%
\begin{center}
|... -jobname "|\textit{target}|" "|[\textit{flags}]%
|\includeonly{|\textit{dest}|}\input{|\textit{main}|}"|
\end{center}
%

%%%%%%%%%%%%%%%%%%%%%%%%%%%%%%%%%%%%%%%%%%%%%%%%%%%%%%%%%%%%%%%%%%%%%%%%%%%%%%%%
%%%%%%%%%%%%%%%%%%%%%%%%%%%%%%%%%%%%%%%%%%%%%%%%%%%%%%%%%%%%%%%%%%%%%%%%%%%%%%%%
\section{Information}

%%%%%%%%%%%%%%%%%%%%%%%%%%%%%%%%%%%%%%%%%%%%%%%%%%%%%%%%%%%%%%%%%%%%%%%%%%%%%%%%
\subsection{Copyright}

Copyright \copyright{} 2017--2018 Niklas Beisert

This work may be distributed and/or modified under the
conditions of the \LaTeX{} Project Public License, either version 1.3
of this license or (at your option) any later version.
The latest version of this license is in
  \url{http://www.latex-project.org/lppl.txt}
and version 1.3 or later is part of all distributions of \LaTeX{}
version 2005/12/01 or later.

This work has the LPPL maintenance status `maintained'.

The Current Maintainer of this work is Niklas Beisert.

This work consists of the files |README.txt|, |childdoc.ins| and |childdoc.dtx|
as well as the derived files |childdoc.def|, |cdocsamp.tex|
with |cdocsch1.tex|, |cdocsch2.tex|, |cdocspt3.tex|, |cdocspt4.tex|,
|cdocsdrf.tex|, |cdocsfn1.tex|, |cdocsfn2.tex|
as well as |childdoc.pdf|.

%%%%%%%%%%%%%%%%%%%%%%%%%%%%%%%%%%%%%%%%%%%%%%%%%%%%%%%%%%%%%%%%%%%%%%%%%%%%%%%%
\subsection{Files and Installation}

The package consists of the files:
%
\begin{center}
\begin{tabular}{ll}
    |README.txt|   & readme file \\
    |childdoc.ins| & installation file \\
    |childdoc.dtx| & source file \\
    |childdoc.def| & definition file \\
    |cdocsamp.tex| & sample main file \\
    |cdocsch1.tex| & sample include file \\
    |cdocsch2.tex| & sample include file \\
    |cdocspt3.tex| & sample part file \\
    |cdocspt4.tex| & sample part file \\
    |cdocsdrf.tex| & sample redirection file \\
    |cdocsfn1.tex| & sample redirection file \\
    |cdocsfn2.tex| & sample redirection file \\
    |childdoc.pdf| & manual
\end{tabular}
\end{center}
%
The distribution consists of the files
|README.txt|, |childdoc.ins| and |childdoc.dtx|.
%
\begin{itemize}
\item
Run (pdf)\LaTeX{} on |childdoc.dtx|
to compile the manual |childdoc.pdf| (this file).
\item
Run \LaTeX{} on |childdoc.ins| to create the definitions file |childdoc.def|
and the sample |cdocsamp.tex| with include files
|cdocsch1.tex|, |cdocsch2.tex|, |cdocspt3.tex|, |cdocspt4.tex|,
|cdocsdrf.tex|, |cdocsfn1.tex|, |cdocsfn2.tex|.
Then copy the file |childdoc.def| to an appropriate directory of your \LaTeX{}
distribution, e.g.\ \textit{texmf-root}|/tex/latex/childdoc|.
\end{itemize}

%%%%%%%%%%%%%%%%%%%%%%%%%%%%%%%%%%%%%%%%%%%%%%%%%%%%%%%%%%%%%%%%%%%%%%%%%%%%%%%%
\subsection{Related CTAN Packages}

There are several other packages which offer a similar functionality:
%
\begin{itemize}
\item
The packages
\href{http://ctan.org/pkg/docmute}{\textsf{docmute}},
\href{http://ctan.org/pkg/includex}{\textsf{includex}} and
\href{http://ctan.org/pkg/standalone}{\textsf{standalone}}
provide commands to include only the document body of
a child file thus allowing both files to be compiled individually.
\item
The packages \href{http://ctan.org/pkg/subdocs}{\textsf{subdocs}}
and \href{http://ctan.org/pkg/subfiles}{\textsf{subfiles}}
provide structures in which the main and child documents can be
encapsulated and allowing them to be compiled individually.
The inclusion mechanism is different from the conventional |\include|.
\item
The package \href{http://ctan.org/pkg/combine}{\textsf{combine}}
is an elaborate solution to combine several documents into one.
\end{itemize}
%
See also the CTAN topic \href{http://ctan.org/topic/subdocs}{\textsf{subdocs}}
for further related packages.
The present package differs from the above solutions in that
a document structure constructed with the conventional |\include| mechanism
just needs two extra commands at the top of every file
such that all constituent files can be compiled individually.

%%%%%%%%%%%%%%%%%%%%%%%%%%%%%%%%%%%%%%%%%%%%%%%%%%%%%%%%%%%%%%%%%%%%%%%%%%%%%%%%
%\subsection{Feature Suggestions}
%
%The following is a list of features which may be useful for future
%versions of this package:
%%
%\begin{itemize}
%\item
%\ldots
%\end{itemize}

%%%%%%%%%%%%%%%%%%%%%%%%%%%%%%%%%%%%%%%%%%%%%%%%%%%%%%%%%%%%%%%%%%%%%%%%%%%%%%%%
\subsection{Revision History}

%%%%%%%%%%%%%%%%%%%%%%%%%%%%%%%%%%%%%%%%
\paragraph{v2.0:} 2018/12/30

\begin{itemize}
\item
immediate forward processing
\item
added |\childdocby| mechanism
\item
manual restructured
\end{itemize}

%%%%%%%%%%%%%%%%%%%%%%%%%%%%%%%%%%%%%%%%
\paragraph{v1.6:} 2018/01/17

\begin{itemize}
\item
application for development of include files
\item
corrections to manual
\end{itemize}

%%%%%%%%%%%%%%%%%%%%%%%%%%%%%%%%%%%%%%%%
\paragraph{v1.5:} 2017/05/21

\begin{itemize}
\item
more complete structuring introduced
\item
|\childdocof| introduced
\item
|\childdoc| renamed to |\childdocmain|
\item
|\childredirect| renamed to |\childdocforward| and |\childdocforwardprefix|
and functionality expanded
\end{itemize}

%%%%%%%%%%%%%%%%%%%%%%%%%%%%%%%%%%%%%%%%
\paragraph{v1.0:} 2017/04/27

\begin{itemize}
\item
manual and install package
\item
first version published on CTAN
\end{itemize}

%%%%%%%%%%%%%%%%%%%%%%%%%%%%%%%%%%%%%%%%
\paragraph{v0.6:} 2017/04/26

\begin{itemize}
\item
redirection mechanism added
\end{itemize}

%%%%%%%%%%%%%%%%%%%%%%%%%%%%%%%%%%%%%%%%
\paragraph{v0.5:} 2017/04/26

\begin{itemize}
\item
functionality in definition file
\end{itemize}


%%%%%%%%%%%%%%%%%%%%%%%%%%%%%%%%%%%%%%%%%%%%%%%%%%%%%%%%%%%%%%%%%%%%%%%%%%%%%%%%
%%%%%%%%%%%%%%%%%%%%%%%%%%%%%%%%%%%%%%%%%%%%%%%%%%%%%%%%%%%%%%%%%%%%%%%%%%%%%%%%
%%%%%%%%%%%%%%%%%%%%%%%%%%%%%%%%%%%%%%%%%%%%%%%%%%%%%%%%%%%%%%%%%%%%%%%%%%%%%%%%
\appendix

\settowidth\MacroIndent{\rmfamily\scriptsize 000\ }

 \DocInput{childdoc.dtx}

\end{document}
%</driver>
% \fi
%
% %%%%%%%%%%%%%%%%%%%%%%%%%%%%%%%%%%%%%%%%%%%%%%%%%%%%%%%%%%%%%%%%%%%%%%%%%%%%%%
% %%%%%%%%%%%%%%%%%%%%%%%%%%%%%%%%%%%%%%%%%%%%%%%%%%%%%%%%%%%%%%%%%%%%%%%%%%%%%%
% \section{Sample}
%\iffalse
%<*samplemain>
%\fi
%
% The following presents a sample document
% with two chapters, two parts, a title page,
% a compile flag as well as three forwarding files to set the flag.
% It consists of eight |.tex| files:
% \begin{center}
% \begin{tabular}{ll}
% |cdocsamp.tex|&main file\\
% |cdocsch1.tex|&include file for chapter 1\\
% |cdocsch2.tex|&include file for chapter 2\\
% |cdocspt3.tex|&include file for part 3\\
% |cdocspt4.tex|&include file for part 4\\
% |cdocsdrf.tex|&forwarding file for main file in draft mode\\
% |cdocsfi1.tex|&forwarding file for final version of chapter 1\\
% |cdocsfi2.tex|&forwarding file for final version of chapter 2\\
% \end{tabular}
% \end{center}
% Each of the eight files can be compiled directly by the \LaTeX{} compiler.
%
% %%%%%%%%%%%%%%%%%%%%%%%%%%%%%%%%%%%%%%
% \paragraph{Main File.}
%
% The main file is called |cdocsamp.tex|.
%
% Load the \textsf{childdoc} definitions and
% declare the filename for the main document:
%    \begin{macrocode}
\input{childdoc.def}
\childdocmain{}
%    \end{macrocode}

% Optional override for |\version| flag:
%    \begin{macrocode}
%%\ifchilddoc\else\providecommand{\version}{draft}\fi
%    \end{macrocode}

% Define the default values for the |\version| flag
% (|final| for the main file and |draft| for childs):
%    \begin{macrocode}
\ifchilddoc
\providecommand{\version}{draft}
\else
\providecommand{\version}{final}
\fi
%    \end{macrocode}

% Load the standard document class:
%    \begin{macrocode}
\documentclass[12pt]{article}
%    \end{macrocode}

% Start the document body:
%    \begin{macrocode}
\begin{document}
%    \end{macrocode}

% Declare a title page.
% Print title, part of document being processed and version flag:
%    \begin{macrocode}
\addtocounter{page}{-1}
\begin{center}
{\LARGE\bfseries{}childdoc example\par}
\vspace{1cm}
\ifchilddoc
\ifchilddocmanual part\else chapter\fi:
`\childdocname' of `\childdocjob'\par
\else
main document: `\childdocjob'\par
\fi
version: \version\par
\end{center}
\newpage
%    \end{macrocode}

% Manually include selected file,
% otherwise process as usual:
%    \begin{macrocode}
\ifchilddocmanual
\section*{part `\childdocname'}
\input{\childdocname}
\else
%    \end{macrocode}

% Include the two chapters:
%    \begin{macrocode}
\include{cdocsch1}
\include{cdocsch2}
%    \end{macrocode}

% Include the two parts unless only chapters should be displayed:
%    \begin{macrocode}
\ifchilddoc\else
\section{part three}
\input{cdocspt3}
\section{part four}
\input{cdocspt4}
\fi
%    \end{macrocode}

% Process as usual until here:
%    \begin{macrocode}
\fi
%    \end{macrocode}

% End of document body:
%    \begin{macrocode}
\end{document}
%    \end{macrocode}
%\iffalse
%</samplemain>
%\fi
%
% %%%%%%%%%%%%%%%%%%%%%%%%%%%%%%%%%%%%%%
% \paragraph{Chapter Include Files.}
%
% The include files are called |cdocsch1.tex| and |cdocsch2.tex|.
%
%\iffalse
%<*samplechap1|samplechap2>
%\fi

% Optional override for |\version| flag:
%    \begin{macrocode}
%%\providecommand{\version}{final}
%    \end{macrocode}

% Include the main document:
%    \begin{macrocode}
\input{childdoc.def}
\childdocof{cdocsamp}
%    \end{macrocode}

%\iffalse
%</samplechap1|samplechap2>
%\fi
%
%\iffalse
%<*samplechap1>
%\fi
% Some text for chapter 1:
%    \begin{macrocode}
\section{one}
some text in chapter one
%    \end{macrocode}

%\iffalse
%</samplechap1>
%\fi
% Some text for chapter 2:
%\iffalse
%<*samplechap2>
%\fi
%    \begin{macrocode}
\section{two}
more text in chapter two
%    \end{macrocode}

%\iffalse
%</samplechap2>
%\fi
%
% %%%%%%%%%%%%%%%%%%%%%%%%%%%%%%%%%%%%%%
% \paragraph{Part Include Files.}
%
% The include files are called |cdocspt3.tex| and |cdocspt4.tex|.
%
%\iffalse
%<*samplepart3|samplepart4>
%\fi

% Optional override for |\version| flag:
%    \begin{macrocode}
%%\providecommand{\version}{final}
%    \end{macrocode}

% Include the main document:
%    \begin{macrocode}
\input{childdoc.def}
\childdocby{cdocsamp}
%    \end{macrocode}

%\iffalse
%</samplepart3|samplepart4>
%\fi
%
%\iffalse
%<*samplepart3>
%\fi
% Some text for part 3:
%    \begin{macrocode}
some text in part three
%    \end{macrocode}

%\iffalse
%</samplepart3>
%\fi
% Some text for part 4:
%\iffalse
%<*samplepart4>
%\fi
%    \begin{macrocode}
more text in part four
%    \end{macrocode}

%\iffalse
%</samplepart4>
%\fi
%
% %%%%%%%%%%%%%%%%%%%%%%%%%%%%%%%%%%%%%%
% \paragraph{Forwarding for a Complete Draft.}
%
% The following forwarding file |cdocsdrf.tex|
% compiles the main document in draft mode:
%\iffalse
%<*sampledraft>
%\fi
%    \begin{macrocode}
\def\version{draft}
\input{childdoc.def}
\childdocforward{cdocsamp}
%    \end{macrocode}

%\iffalse
%</sampledraft>
%\fi
%
% %%%%%%%%%%%%%%%%%%%%%%%%%%%%%%%%%%%%%%
% \paragraph{Forwarding for Final Version of the Chapters.}
%
% The following forwarding files |cdocsfn1.tex| and |cdocsfn2.tex|
% (with identical content)
% compile the final versions of the child documents
% |cdocsch1.tex| and |cdocsch2.tex|, respectively:
%\iffalse
%<*samplefinal>
%\fi
%    \begin{macrocode}
\def\version{final}
\input{childdoc.def}
\childdocforwardprefix[cdocsamp]{cdocsfn}{cdocsch}
%    \end{macrocode}

%\iffalse
%</samplefinal>
%\fi
%
% %%%%%%%%%%%%%%%%%%%%%%%%%%%%%%%%%%%%%%
% \paragraph{Command Line Processing.}
%
% The following three command lines generate the output files
% |cdocscld|, |cdocscl1| and |cdocscl2|
% which should be identical to
% |cdocsdrf|, |cdocsch1| and |cdocsfn2|, respectively:
% \begin{center}
% \begin{tabular}{l}
% |latex -jobname cdocscld \|\\
% |  "\def\version{draft}\input{childdoc.def}\childdocforward{cdocsamp}"|\\
% |latex -jobname cdocscl1 \|\\
% |  "\input{childdoc.def}\childdocforward[cdocsamp]{cdocsch1}"|\\
% |latex -jobname cdocscl2 \|\\
% |  "\def\version{final}\input{childdoc.def}\childdocforward{cdocsch2}"|
% \end{tabular}
% \end{center}
% Note that the trailing backslash on each first line
% merely continues the input to the second line
% (for convenient cut ant paste).
% Furthermore, the command |latex| can be replaced by any
% of its alternative versions such as |pdflatex|.
%
% %%%%%%%%%%%%%%%%%%%%%%%%%%%%%%%%%%%%%%%%%%%%%%%%%%%%%%%%%%%%%%%%%%%%%%%%%%%%%%
% %%%%%%%%%%%%%%%%%%%%%%%%%%%%%%%%%%%%%%%%%%%%%%%%%%%%%%%%%%%%%%%%%%%%%%%%%%%%%%
% \section{Implementation}
%\iffalse
%<*package>
%\fi
%
% This section describes the definitions file |childdoc.def|.

% The definitions cannot be loaded using |\usepackage| or |\RequirePackage|
% which has a mechanism to prevent loading a style file more than once.
% When loading the definitions by means of |\input|
% multiple instances have to be prevented manually:
%\iffalse
%This code needs to be before the `\ProvidesFile' directive
%which is defined at the beginning of this file.
%Therefore it is also placed there and commented out here.
%</package>
%<*discard>
%\fi
%    \begin{macrocode}
\ifdefined\childdocmain\endinput\fi
%    \end{macrocode}
%\iffalse
%</discard>
%<*package>
%\fi
%
% \macro{\ifchilddoc}
% \macro{\ifchilddocmanual}
% The conditional |\ifchilddoc| tells whether a
% child (true) or main (false) document is being compiled.
% The conditional |\ifchilddocmanual| tells whether
% the |\includeonly| mechanism is used (false) or
% the selection of child files must be performed manually (true).
% The definitions initialise to false:
%    \begin{macrocode}
\newif\ifchilddoc
\newif\ifchilddocmanual
%    \end{macrocode}

% \macro{\childdocname}
% \macro{\childdocjob}
% The macro |\childdocname| stores the name of the main document
% to be compiled. The macro |\childdocjob| stores the name of
% the document on which the \LaTeX{} compiler was originally invoked.
% The content of |\jobname| cannot be compared
% to filenames specified in the source due to different catcodes.
% The following code rescans |\jobname|, stores the result
% in |\childdocname| and saves a copy in |\childdocjob|:
%    \begin{macrocode}
\edef\childdocname{\scantokens\expandafter{\jobname\noexpand}}
\let\childdocjob\childdocname
%    \end{macrocode}

% \macro{\childdocdisable}
% The macro |\childdocdisable| prevents the main file
% from being processed more than once.
% At this stage, the main document command |\childdocmain|
% is assumed to be called once again where it should do nothing.
% Any subsequent call to it should prevent
% a secondary processing of the main document
% It overwrites the forwarding commands
% |\childdocof| and |\childdocforward|
% with empty macros to prevent further inclusions of the main document:
%    \begin{macrocode}
\newcommand{\childdocdisable}
{
  \renewcommand{\childdocmain}[1]{\renewcommand{\childdocmain}[1]{\endinput}}
  \renewcommand{\childdocof}[1]{}
  \renewcommand{\childdocby}[2][]{}
  \renewcommand{\childdocforward}[2][]{}
  \renewcommand{\childdocdisable}{}
}
%    \end{macrocode}

% \macro{\childdocmain}
% The macro |\childdocmain| is to be called at the top of the main file
% with nothing or the main filename (without extension) as argument.
% First, it breaks loops.
% If the argument is not empty and does not match |\childdocname|
% (which is set by the first inclusion of |childdoc.def|),
% |\ifchilddoc| is set to true, |\includeonly| is applied to the child file
% and |\jobname| is set to the main file
% (for proper handling of |.aux| files):
%    \begin{macrocode}
\newcommand{\childdocmain}[1]
{
  \childdocdisable\childdocmain{}
  \if?#1?\else
    \begingroup
      \def\childdoctmp{#1}
      \ifx\childdoctmp\childdocname
        \def\childdoctmp{}
      \else
        \def\childdoctmp
        {
          \childdoctrue
          \includeonly{\childdocname}
          \def\childdocjob{#1}
          \def\jobname{#1}
        }
      \fi
      \expandafter
    \endgroup
    \childdoctmp
  \fi
}
%    \end{macrocode}

% \macro{\childdocof}
% The command |\childdocof| redirects
% compilation to the main file |#1|.
%    \begin{macrocode}
\newcommand{\childdocof}[1]
{
  \childdocdisable
  \childdoctrue
  \includeonly{\childdocname}
  \def\jobname{#1}
  \def\childdocjob{#1}
  \input{#1}
}
%    \end{macrocode}

% \macro{\childdocby}
% The command |\childdocby| ....
%    \begin{macrocode}
\newcommand{\childdocby}[2][]
{
  \childdocdisable
  \childdoctrue
  \childdocmanualtrue
  \if?#1?\else
    \def\jobname{#2}
  \fi
  \def\childdocjob{#2}
  \input{#2}
  \endinput
}
%    \end{macrocode}

% \macro{\childdocforward}
% The command |\childdocforward| redirects
% compilation to the main file or
% (if the optional argument is given) a child file.
% Parameters are set as if the main file
% or a child file starting with |\childdocof| was compiled.
% Then compilation is handed over to the main file:
%    \begin{macrocode}
\newcommand{\childdocforward}[2][]
{
  \begingroup
    \if?#1?
      \def\childdoctmp
      {
        \def\childdocname{#2}
        \def\childdocjob{#2}
        \def\jobname{#2}
        \input{#2}
        \endinput
      }
    \else
      \def\childdoctmp
      {
        \childdocdisable
        \def\childdocname{#2}
        \childdoctrue
        \includeonly{#2}
        \def\childdocjob{#1}
        \def\jobname{#1}
        \input{#1}
        \endinput
      }
    \fi
    \expandafter
  \endgroup
  \childdoctmp
}
%    \end{macrocode}

% \macro{\childdocforwardprefix}
% The command |\childdocforwardprefix| redirects
% compilation to the main or a child file by means of a pattern.
% The prefix |#1| in the current filename is replaced by |#2|
% and the suffix of the current filename is kept
% (it is assumed that the filename does not contain the substring `|~~~|'
% which is used as a delimiter).
% Compilation is handed over to the new file by |\childdocforward|:
%    \begin{macrocode}
\newcommand{\childdocforwardprefix}[3][]
{
  \begingroup
    \def\childdocextract #2##1~~~{\def\childdoctmp{\childdocforward[#1]{#3##1}}}
    \expandafter\childdocextract\childdocname~~~
    \expandafter
  \endgroup
  \childdoctmp
}
%    \end{macrocode}

% \macro{\childdoc}
% The deprecated macro |\childdoc| is a legacy version of |\childdocmain|:
%    \begin{macrocode}
\newcommand{\childdoc}{\childdocmain}
%    \end{macrocode}

% \macro{\childdocredirect}
% The deprecated macro |\childdocredirect| is a legacy version
% of |\childdocforward| and |\childdocforwardprefix|:
%    \begin{macrocode}
\newcommand{\childdocredirect}[2][]
{
  \begingroup
    \if?#1?
      \def\childdoctmp{\childdocforward{#2}}
    \else
      \def\childdoctmp{\childdocforwardprefix{#1}{#2}}
    \fi
    \expandafter
  \endgroup
  \childdoctmp
}
%    \end{macrocode}

%\iffalse
%</package>
%\fi
%
\endinput
|\\
|\childdocby{|\textit{main}|}|\\
\end{tabular}
\end{center}
%
The directive |\childdocby| is similar to |\childdocof|
described in \secref{sec:include},
but the subsequent selection of content must be done manually.
To that end, both |\ifchilddoc| and |\ifchilddocmanual|
will be true upon processing of a part,
and the name of the part is stored in |\childdocname|.
Note that |\jobname| will be set to the filename of the current part
so that each part receives an individual |.aux| file
that does not interfere with the |.aux| file(s) of the main document.
This behaviour can be altered by the alternative form
|\childdocby[*]{|\textit{main}|}| (with a non-empty optional argument)
which uses the |.aux| file of the main document
by setting |\jobname| to \textit{main}.

%%%%%%%%%%%%%%%%%%%%%%%%%%%%%%%%%%%%%%%%%%%%%%%%%%%%%%%%%%%%%%%%%%%%%%%%%%%%%%%%
\subsection{Driver Development}
\label{sec:driver}

The \textsf{childdoc} mechanism can also be use for the development
of definition files such as \LaTeX{} styles or classes.
This case differs from the above setup with multiple parts
included by |\include| in that no |\includeonly| should be invoked.
This can be achieved by starting the include file
(before |\ProvidesPackage|) with:
%
\begin{center}
\begin{tabular}{l}
|% \iffalse
%
% childdoc.dtx Copyright (C) 2017-2018 Niklas Beisert
%
% This work may be distributed and/or modified under the
% conditions of the LaTeX Project Public License, either version 1.3
% of this license or (at your option) any later version.
% The latest version of this license is in
%   http://www.latex-project.org/lppl.txt
% and version 1.3 or later is part of all distributions of LaTeX
% version 2005/12/01 or later.
%
% This work has the LPPL maintenance status `maintained'.
%
% The Current Maintainer of this work is Niklas Beisert.
%
% This work consists of the files childdoc.dtx and childdoc.ins
% and the derived files childdoc.def and cdocsamp.tex with
% cdocsch1.tex, cdocsch2.tex, cdocsdrf.tex, cdocsfn1.tex, cdocsfn2.tex.
%
%<package>\ifdefined\childdocmain\endinput\fi
%<package>\ProvidesFile{childdoc.def}[2018/12/30 v2.0 child document driver]
%<samplemain>\ProvidesFile{cdocsamp.tex}[2018/12/30 v2.0 sample for childdoc]
%<*driver>
%\ProvidesFile{childdoc.drv}[2018/12/30 v2.0 childdoc reference manual file]
\PassOptionsToClass{10pt,a4paper}{article}
\documentclass{ltxdoc}

\usepackage[margin=35mm]{geometry}
\usepackage{hyperref}
\usepackage{hyperxmp}
\usepackage[usenames]{color}

\hypersetup{colorlinks=true}
\hypersetup{pdfstartview=FitH}
\hypersetup{pdfpagemode=UseNone}
\hypersetup{pdfsource={}}
\hypersetup{pdflang={en-UK}}
\hypersetup{pdfcopyright={Copyright 2017-2018 Niklas Beisert.
  This work may be distributed and/or modified under the
  conditions of the LaTeX Project Public License, either version 1.3
  of this license or (at your option) any later version.}}
\hypersetup{pdflicenseurl={http://www.latex-project.org/lppl.txt}}
\hypersetup{pdfcontactaddress={ETH Zurich, ITP, HIT K,
  Wolfgang-Pauli-Strasse 27}}
\hypersetup{pdfcontactpostcode={8093}}
\hypersetup{pdfcontactcity={Zurich}}
\hypersetup{pdfcontactcountry={Switzerland}}
\hypersetup{pdfcontactemail={nbeisert@itp.phys.ethz.ch}}
\hypersetup{pdfcontacturl={http://people.phys.ethz.ch/\xmptilde nbeisert/}}

\newcommand{\secref}[1]{\hyperref[#1]{section \ref*{#1}}}

\parskip1ex
\parindent0pt
\let\olditemize\itemize
\def\itemize{\olditemize\parskip0pt}

\begin{document}

\title{The \textsf{childdoc} Package}
\hypersetup{pdftitle={The childdoc Package}}
\author{Niklas Beisert\\[2ex]
  Institut f\"ur Theoretische Physik\\
  Eidgen\"ossische Technische Hochschule Z\"urich\\
  Wolfgang-Pauli-Strasse 27, 8093 Z\"urich, Switzerland\\[1ex]
  \href{mailto:nbeisert@itp.phys.ethz.ch}
  {\texttt{nbeisert@itp.phys.ethz.ch}}}
\hypersetup{pdfauthor={Niklas Beisert}}
\hypersetup{pdfsubject={Manual for the LaTeX2e Package childdoc}}
\date{30 December 2018, \textsf{v2.0}}
\maketitle

\begin{abstract}\noindent
\textsf{childdoc} is a \LaTeXe{} package
that enables the direct compilation
of document sections included by |\include|
to individual files.
\end{abstract}

\begingroup
\parskip0ex
\tableofcontents
\endgroup

%%%%%%%%%%%%%%%%%%%%%%%%%%%%%%%%%%%%%%%%%%%%%%%%%%%%%%%%%%%%%%%%%%%%%%%%%%%%%%%%
%%%%%%%%%%%%%%%%%%%%%%%%%%%%%%%%%%%%%%%%%%%%%%%%%%%%%%%%%%%%%%%%%%%%%%%%%%%%%%%%
\section{Introduction}

\LaTeX{} provides a mechanism to structure a large document (such as a book)
into a main file and several child files (containing the chapters)
using the |\include| command.
This mechanism is beneficial for documents
which span hundreds of pages in order to
make the source file(s) more manageable.
Moreover, compilation can be restricted to
selected child files by means of the |\includeonly| command.
The latter feature can be used to reduce the compilation time while editing
(this was significantly more useful in the earlier days of \LaTeX{})
or to generate a smaller document which is easier to navigate.
Another application of |\includeonly| is to generate
documents consisting of selected parts of the complete document.

However, there are a few drawbacks of the plain |\include| mechanism:
\begin{itemize}
\item
The child files cannot be compiled on their own,
they can only be compiled via the main file.
A naive editing environment
(such as a text editor with an option
to have the current file processed by \LaTeX)
may require one to switch to the main file before compiling;
attempting to compile the child file produces errors.
\item
The main file must be modified (each time)
to adjust the |\includeonly| command
to the present needs. This easily leaves the main file in a messy state.
\item
The generated document will always carry the filename
of the main document. This is inconvenient if
several child files are to be compiled and
to be kept for distribution.
\end{itemize}

The present package provides a simple interface
to make child files individually compilable by \LaTeX{}.
Compiling a child file then has the same effect as compiling
the main file with an |\includeonly| command
to select the appropriate child.
Moreover the generated document will carry the name of the child
rather than the main file.
This resolves all three above issues.

This feature is meant to make the editing of books,
thesis documents and lecture notes somewhat more convenient.
However, the package can also be used efficiently for
composing a series of documents (such as exercise sheets)
which are typically distributed individually.
It then assists the author in generating the individual documents
(potentially in different versions)
as well as a document containing the collected series.
Another application is in developing style files
or other kinds of included material
where compilation of the style file could redirect
to a sample or test file.

%%%%%%%%%%%%%%%%%%%%%%%%%%%%%%%%%%%%%%%%%%%%%%%%%%%%%%%%%%%%%%%%%%%%%%%%%%%%%%%%
%%%%%%%%%%%%%%%%%%%%%%%%%%%%%%%%%%%%%%%%%%%%%%%%%%%%%%%%%%%%%%%%%%%%%%%%%%%%%%%%
\section{Usage}

First of all, the package \textsf{childdoc} is \emph{not} a standard
\LaTeXe{} |.sty| style file! Therefore it needs to be invoked in
a non-standard way.

%%%%%%%%%%%%%%%%%%%%%%%%%%%%%%%%%%%%%%%%%%%%%%%%%%%%%%%%%%%%%%%%%%%%%%%%%%%%%%%%
\subsection{Included Files}
\label{sec:include}

%%%%%%%%%%%%%%%%%%%%%%%%%%%%%%%%%%%%%%%%
\DescribeMacro{\childdocmain}
To use the package, add the commands
\begin{center}
\begin{tabular}{l}
|\input{childdoc.def}|\\
|\childdocmain{}|\\
\end{tabular}
\end{center}
at the very top of the main \LaTeX{} file,
in particular \emph{before} the |\documentclass| statement!
The argument of |\childdocmain| should be left empty
(but it must be present).

%%%%%%%%%%%%%%%%%%%%%%%%%%%%%%%%%%%%%%%%
\DescribeMacro{\childdocof}
Furthermore, add the commands
\begin{center}
\begin{tabular}{l}
|\input{childdoc.def}|\\
|\childdocof{|\textit{main}|}|\\
\end{tabular}
\end{center}
at the top of every child file \textit{child}
which is included by |\include{|\textit{child}|}|
from within the main file
(or at least for those files to be compiled individually).
The argument \textit{main} must be the filename of the main file.

There are a couple of
considerations in setting up the main and child documents:

%%%%%%%%%%%%%%%%%%%%%%%%%%%%%%%%%%%%%%%%
\paragraph{Restrictions.}

Please note the following restrictions:
\begin{itemize}
\item
|\childdocmain| must be called with one argument \textit{main}
to ensure compatibility with earlier version of the package.
It must either be empty (|\childdocmain{}|)
or precisely match the filename of the main file in which it is specified.
See \secref{sec:detection} for further information.
\item
The filename \textit{main} must be specified without the |.tex| extension.
\item
The filename \textit{main} is case sensitive
(even in case-insensitive file systems)
due to internal string comparison.
\item
The argument \textit{main} should be fully expanded, it cannot be a macro.
\item
Subdirectories and special characters should be avoided in filenames.
\item
The command |\childdocmain{|\textit{main}|}| must be followed by a whitespace.
It should not be followed immediately by another command
or by a comment mark `|%|'.
This is because the \TeX{} parser reads the token immediately following
the argument of |\childdocmain| and puts it
at the beginning of every child section;
however, a white\-space is ignored.
\end{itemize}

%%%%%%%%%%%%%%%%%%%%%%%%%%%%%%%%%%%%%%%%
\paragraph{Content of Main File.}

It is advisable to place all content in the child files included by |\include|.
Any output contained in the main file will appear in all child documents
unless suppressed manually;
it cannot be suppressed automatically by the |\includeonly| directive
and thus should normally be avoided.
A method to include some content in the main file
by means of conditional processing is described in \secref{sec:conditional}.

%%%%%%%%%%%%%%%%%%%%%%%%%%%%%%%%%%%%%%%%
\paragraph{Page Numbering.}

When only a part of the document is compiled,
the appropriate numbering of pages
(as well as other status parameters)
is determined from the |.aux| files.
The latter contain information from previous passes.
However this information needs to propagate through
all intermediate child documents.
Therefore the page numbering in child documents may well
be inconsistent until the complete document is compiled at least once.

A useful (if unconventional) way to always ensure a consistent
page numbering is to restart the numbering in each child document
and denote the pages by `\textit{child}|.|\textit{page}'
where \textit{child} represents the chapter/section number of the child file.
This can be achieved by the command
|\numberwithin{page}{|\textit{child}|}|
of the \textsf{amsmath} package
where \textit{child} can be |chapter| or |section|
depending on the chosen structuring.
Alternatively, one can modify the macro |\thepage| appropriately
and reset the counter |page| at the start of each child file.

%%%%%%%%%%%%%%%%%%%%%%%%%%%%%%%%%%%%%%%%%%%%%%%%%%%%%%%%%%%%%%%%%%%%%%%%%%%%%%%%
\subsection{Conditional Processing}
\label{sec:conditional}

The package provides a mechanism to compile different versions
of a document. To customise the versions further some conditional processing
can come in handy to distinguish which version is being compiled.
The package provides two macros to describe the compilation context:

%%%%%%%%%%%%%%%%%%%%%%%%%%%%%%%%%%%%%%%%
\DescribeMacro{\ifchilddoc}
The conditional |\ifchilddoc| distinguishes between the compilation of
child documents and the main document:
%
\begin{center}
|\ifchilddoc |\textit{child-code}| |[|\||else |\textit{main-code}]| \||fi|
\end{center}

%%%%%%%%%%%%%%%%%%%%%%%%%%%%%%%%%%%%%%%%
\DescribeMacro{\childdocname}
\DescribeMacro{\childdocjob}
The macro |\childdocname| contains the filename (without extension)
of the main or child file being processed.
Note that |\childdocjob| will always contain the name of the main file.

%%%%%%%%%%%%%%%%%%%%%%%%%%%%%%%%%%%%%%%%
\paragraph{Title Page.}

Conditional processing can be used to include a title or banner page
in the main document when proper precautions are taken.
Importantly, the code in the main file should ensure that the page counter
(as well as other status parameters which are stored in the |.aux| files)
takes the same value after the conditional processing.
Otherwise the page numbers may take divergent values
depending on which part is compiled.

For example, a title page could be declared by:
%
\begin{center}
\begin{tabular}{l}
|\ifchilddoc\||else|\\
|\addtocounter{page}{-1}|\\
\textit{code for title page}\\
|\newpage|\\
|\||fi|
\end{tabular}
\end{center}
%
A banner page for the child documents can be generated by:
%
\begin{center}
\begin{tabular}{l}
|\ifchilddoc|\\
|\addtocounter{page}{-1}|\\
\textit{code for banner page}\\
|\newpage|\\
|\||fi|
\end{tabular}
\end{center}
%
Here one could write a message such as:
\begin{center}
|This is the part \childdocname{} of \childdocjob{}.|
\end{center}

%%%%%%%%%%%%%%%%%%%%%%%%%%%%%%%%%%%%%%%%%%%%%%%%%%%%%%%%%%%%%%%%%%%%%%%%%%%%%%%%
\subsection{Flags}
\label{sec:flags}

The package makes it easy to generate different versions
of the main or child documents.
To this end compilation flags can be defined
and assigned different default values.
They will be particularly useful in conjunction
with the forwarding mechanism described in \secref{sec:forward}.

For example, it may be useful to have a flag |\version|
which can be set to |draft| or |final|.
The document source will contain some conditional code
depending on the value of |\version|.
Suppose further, the flag should default to |final| for the main file
and to |draft| for child files
which is a natural assignment for editing the document.
This is achieved by placing the following code
in the preamble of the main document
(below the |\childdocmain| directive):
%
\begin{center}
\begin{tabular}{l}
|\ifchilddoc|\\
|\providecommand{\version}{draft}|\\
|\||else|\\
|\providecommand{\version}{final}|\\
|\||fi|
\end{tabular}
\end{center}
%
The definition by |\providecommand| makes sure
that previous definitions are not overwritten.
Further statements |\providecommand{\version}{...}|
can thus be added before the above code to override it.

For the main file, one might add a line
(between |\childdocmain| and the above block)
%
\begin{center}
|%\ifchilddoc\||else\providecommand{\version}{draft}\||fi|
\end{center}
%
which can be uncommented to produce a draft version.
Likewise one can add a line to the very top of a child file
(above the |\childdocof{|\textit{main}|}| directive)
%
\begin{center}
|%\providecommand{\version}{final}|
\end{center}
%
which can be uncommented to produce the final version of this child document.

%%%%%%%%%%%%%%%%%%%%%%%%%%%%%%%%%%%%%%%%%%%%%%%%%%%%%%%%%%%%%%%%%%%%%%%%%%%%%%%%
\subsection{Forwarding}
\label{sec:forward}

Different versions of the main or child documents
using compilation flags as described in \secref{sec:flags}
can be (permanently) stored in different files
for convenient compilation, viewing and distribution.
To this end, the package defines a command
to pass on compilation to a different file:

%%%%%%%%%%%%%%%%%%%%%%%%%%%%%%%%%%%%%%%%
\DescribeMacro{\childdocforward}
The command |\childdocforward| redirects processing to
another source file:
%
\begin{center}
\begin{tabular}{l}
|\input{childdoc.def}|\\
|\childdocforward[|\textit{main}|]{|\textit{dest}|}|\\
\end{tabular}
\end{center}
%
The argument \textit{dest} is the destination file
(without extension).
It should be the main file or one of the child files.
Note that further \textsf{childdoc} directives
such as |\childdocof| and |\childdocforward|
in the indicated file will be processed in this form.
The optional argument \textit{main}
passes on directly to the main file \textit{main}
while pretending to compile the child \textit{dest}.
This form behaves as if \textit{dest}
issues |\childdocof{|\textit{main}|}| right away,
and no further \textsf{childdoc} directives will be processed.

%%%%%%%%%%%%%%%%%%%%%%%%%%%%%%%%%%%%%%%%
\DescribeMacro{\...prefix}
In the alternative form |\childdocforwardprefix|,
%
\begin{center}
\begin{tabular}{l}
|\input{childdoc.def}|\\
|\childdocforwardprefix[|\textit{main}|]{|\textit{prefix}|}{|\textit{dest}|}|
\end{tabular}
\end{center}
%
the destination file is determined by a pattern
depending on the current file:
To make this work, the current file must be called
`{\textit{prefix}\hspace{0.2em}\textit{suffix}}'
with \textit{prefix} matching precisely the argument.
Processing is then passed on to the file
`{\textit{dest}\hspace{0.2em}\textit{suffix}}'.
Surely, the same effect is achieved by
directly specifying the
argument `{\textit{dest}\hspace{0.2em}\textit{suffix}}'
in the first form.
However, that requires to set up a different file
for each child. With the alternative form of the command
all these files can have exactly the same content
which simplifies setting them up and maintaining them.

For example, the following file |draft.tex|
with a compilation flag |\version| as described in \secref{sec:flags}
compiles the main document as a draft:
%
\begin{center}
\begin{tabular}{l}
|\def\version{draft}|\\
|\input{childdoc.def}|\\
|\childdocforward{|\textit{main}|}|
\end{tabular}
\end{center}
%
Likewise, the following files |final|\textit{nn}|.tex|
compile the final version of the child document
|child|\textit{nn}|.tex|:
%
\begin{center}
\begin{tabular}{l}
|\def\version{final}|\\
|\input{childdoc.def}|\\
|\childdocforwardprefix{final}{child}|
\end{tabular}
\end{center}
%

Note that when several versions of a main file and/or of each child file
are to be generated, it may be convenient to set up a |Makefile| or
shell script to automatise the process.

%%%%%%%%%%%%%%%%%%%%%%%%%%%%%%%%%%%%%%%%%%%%%%%%%%%%%%%%%%%%%%%%%%%%%%%%%%%%%%%%
\subsection{Command Line Processing}
\label{sec:commandline}

The effect of redirection files can also be achieved by invoking
the \LaTeX{} compiler with a more elaborate command line.
Most conveniently this should be done as part
of a shell script or a |Makefile|.

When using \textsf{childdoc} in the main file, the following
command lines effectively perform a redirection
(note that depending on the shell being used,
backslashes may have to be doubled: `|\|' $\to$ `|\\|'):
%
\begin{center}
|... -jobname "|\textit{target}|" |\\|"|[\textit{flags}]%
|\input{childdoc.def}\childdocforward[|\textit{main}|]{|\textit{dest}|}"|
\end{center}
%
Here \textit{target} is the name of the output file,
\textit{main} is the name of the main file
and \textit{dest} is the name of the main or child file to be processed
(all filenames without extensions).
The optional argument \textit{main} can be omitted
if \textit{main} matches \textit{dest}.
Optionally, compilation \textit{flags} can be defined via |\def| commands.
This command line makes the \TeX{} engine believe
it is compiling the file \textit{target}
whose content is specified as the latter parameter.
The provided code then forwards the processing to
\textit{main} or \textit{dest} as described in \secref{sec:forward}.

%%%%%%%%%%%%%%%%%%%%%%%%%%%%%%%%%%%%%%%%%%%%%%%%%%%%%%%%%%%%%%%%%%%%%%%%%%%%%%%%
\subsection{Include by Input}
\label{sec:input}

Including child documents by |\include| has some restrictions by design.
Most notably, the content of a child document always occupies
its own set of pages; pages cannot be shared between child documents.
Usually, this behaviour makes perfect sense
because each child document contain an essential part of the document.
However, in some situations it may be desirable to compose
a document from a collection of parts
without having mandatory page breaks between then.
For this case, the package
provides a mechanism to include parts
by |\input| which can also be processed individually.
However, by construction this mechanism
requires manual handling of the content to be output.

%%%%%%%%%%%%%%%%%%%%%%%%%%%%%%%%%%%%%%%%
\DescribeMacro{\ifchilddocmanual}
The main file should be prepared as usual, see \secref{sec:include}.
However, the document body must make a distinction
between processing of an individual part and of the main document, e.g.:
%
\begin{center}
\begin{tabular}{l}
|\ifchilddocmanual|\\
|\input{\childdocname}|\\
|\||else|\\
\textit{document body with }|\input{|\textit{part}|}|\\
|\||fi|
\end{tabular}
\end{center}
%
The conditional |\ifchilddocmanual| is true whenever
a part to be included by |\input| is being compiled,
and the name of the part is stored in |\childdocname|.

%%%%%%%%%%%%%%%%%%%%%%%%%%%%%%%%%%%%%%%%
\DescribeMacro{\childdocby}
Each part to be included by |\input| should start with:
%
\begin{center}
\begin{tabular}{l}
|\input{childdoc.def}|\\
|\childdocby{|\textit{main}|}|\\
\end{tabular}
\end{center}
%
The directive |\childdocby| is similar to |\childdocof|
described in \secref{sec:include},
but the subsequent selection of content must be done manually.
To that end, both |\ifchilddoc| and |\ifchilddocmanual|
will be true upon processing of a part,
and the name of the part is stored in |\childdocname|.
Note that |\jobname| will be set to the filename of the current part
so that each part receives an individual |.aux| file
that does not interfere with the |.aux| file(s) of the main document.
This behaviour can be altered by the alternative form
|\childdocby[*]{|\textit{main}|}| (with a non-empty optional argument)
which uses the |.aux| file of the main document
by setting |\jobname| to \textit{main}.

%%%%%%%%%%%%%%%%%%%%%%%%%%%%%%%%%%%%%%%%%%%%%%%%%%%%%%%%%%%%%%%%%%%%%%%%%%%%%%%%
\subsection{Driver Development}
\label{sec:driver}

The \textsf{childdoc} mechanism can also be use for the development
of definition files such as \LaTeX{} styles or classes.
This case differs from the above setup with multiple parts
included by |\include| in that no |\includeonly| should be invoked.
This can be achieved by starting the include file
(before |\ProvidesPackage|) with:
%
\begin{center}
\begin{tabular}{l}
|\input{childdoc.def}|\\
|\childdocforward{|\textit{main}|}|\\
\end{tabular}
\end{center}
%
or alternatively with:
%
\begin{center}
\begin{tabular}{l}
|\input{childdoc.def}|\\
|\childdocby{|\textit{main}|}|\\
\end{tabular}
\end{center}
%
Both forms have slightly different effects as described above.
The main file is prepared as usual, see \secref{sec:include}.

%%%%%%%%%%%%%%%%%%%%%%%%%%%%%%%%%%%%%%%%%%%%%%%%%%%%%%%%%%%%%%%%%%%%%%%%%%%%%%%%
\subsection{Legacy Detection}
\label{sec:detection}

The directive |\childdocmain| in the main file can detect
whether the complete document or merely a child is to be compiled
even without using the directive |\childdocof|.
This method is deprecated because it is less robust
and there is no compelling reason to use it;
it is merely provided for backward compatibility
and it may be removed in future versions.

If the detection mechanism is to be used,
it is mandatory to correctly specify
the filename of the main file as the argument of |\childdocmain|:
%
\begin{center}
\begin{tabular}{l}
|\input{childdoc.def}|\\
|\childdocmain{|\textit{main}|}|\\
\end{tabular}
\end{center}
%
If |\jobname| does not match the argument \textit{main} of |\childdocmain|,
it is assumed that |\jobname| points to the child file to be compiled.
When using |\childdocmain| with the main file specified as argument,
it suffices to start a child file
with just |\input{|\textit{main}|}|
without loading of the package and using |\childdocof|.
If instead all processing is done
with the appropriate \textsf{childdoc} directives,
the argument of \textit{main} of |\childdocmain| can be empty.

An alternative version of the command line processing described
in \secref{sec:commandline} using the detection mechanism reads:
%
\begin{center}
|... -jobname "|\textit{target}|" "|[\textit{flags}]%
[|\def\jobname{|\textit{dest}|}|]|\input{|\textit{main}|}"|
\end{center}

%%%%%%%%%%%%%%%%%%%%%%%%%%%%%%%%%%%%%%%%%%%%%%%%%%%%%%%%%%%%%%%%%%%%%%%%%%%%%%%%
\subsection{Manual Code}
\label{sec:manual}

In case one cannot be certain whether the definitions file |childdoc.def|
is installed on the target \TeX{} distribution
and one prefers not to ship it,
it is conceivable to paste a few relevant commands into the sources.

To that end, drop all statements |\input{childdoc.def}|
and perform the replacements as outlined below.
Instead of |\childdocmain{|\textit{main}|}| add the following code
to the top of the main file:
%
\begin{center}
\begin{tabular}{l}
|\||ifdefined\childdocname\endinput\||fi\newif\ifchilddoc|\\
|\edef\childdocname{\scantokens\expandafter{\jobname\noexpand}}|\\
|\def\childdocmain{|\textit{main}|}\||ifx\childdocmain\childdocname\||else|\\
|\childdoctrue\includeonly{\childdocname}\let\jobname\childdocmain\||fi|\\
\end{tabular}
\end{center}
%
Instead of |\childdocof{|\textit{main}|}| just include the main file
at the top of each child file:
%
\begin{center}
|\input{|\textit{main}|}|
\end{center}
%
A simple redirection |\childdocforward{|\textit{dest}|}| is achieved by:
%
\begin{center}
|\def\jobname{|\textit{dest}|}\input{\jobname}|
\end{center}
%
The redirection with prefix
|\childdocforwardprefix[|\textit{prefix}|]{|\textit{dest}|}|
is accomplished by:
%
\begin{center}
\begin{tabular}{l}
|{\edef\jobname{\scantokens\expandafter{\jobname\noexpand}}|\\
|\def\redirectjob |\textit{prefix}|#1~~~{\gdef\jobname{|\textit{dest}|#1}}|\\
|\expandafter\redirectjob\jobname~~~}\input{\jobname}|
\end{tabular}
\end{center}

In an alternative approach,
child documents can be compiled by a specific command line
without additional code or specific definitions:
%
\begin{center}
|... -jobname "|\textit{target}|" "|[\textit{flags}]%
|\includeonly{|\textit{dest}|}\input{|\textit{main}|}"|
\end{center}
%

%%%%%%%%%%%%%%%%%%%%%%%%%%%%%%%%%%%%%%%%%%%%%%%%%%%%%%%%%%%%%%%%%%%%%%%%%%%%%%%%
%%%%%%%%%%%%%%%%%%%%%%%%%%%%%%%%%%%%%%%%%%%%%%%%%%%%%%%%%%%%%%%%%%%%%%%%%%%%%%%%
\section{Information}

%%%%%%%%%%%%%%%%%%%%%%%%%%%%%%%%%%%%%%%%%%%%%%%%%%%%%%%%%%%%%%%%%%%%%%%%%%%%%%%%
\subsection{Copyright}

Copyright \copyright{} 2017--2018 Niklas Beisert

This work may be distributed and/or modified under the
conditions of the \LaTeX{} Project Public License, either version 1.3
of this license or (at your option) any later version.
The latest version of this license is in
  \url{http://www.latex-project.org/lppl.txt}
and version 1.3 or later is part of all distributions of \LaTeX{}
version 2005/12/01 or later.

This work has the LPPL maintenance status `maintained'.

The Current Maintainer of this work is Niklas Beisert.

This work consists of the files |README.txt|, |childdoc.ins| and |childdoc.dtx|
as well as the derived files |childdoc.def|, |cdocsamp.tex|
with |cdocsch1.tex|, |cdocsch2.tex|, |cdocspt3.tex|, |cdocspt4.tex|,
|cdocsdrf.tex|, |cdocsfn1.tex|, |cdocsfn2.tex|
as well as |childdoc.pdf|.

%%%%%%%%%%%%%%%%%%%%%%%%%%%%%%%%%%%%%%%%%%%%%%%%%%%%%%%%%%%%%%%%%%%%%%%%%%%%%%%%
\subsection{Files and Installation}

The package consists of the files:
%
\begin{center}
\begin{tabular}{ll}
    |README.txt|   & readme file \\
    |childdoc.ins| & installation file \\
    |childdoc.dtx| & source file \\
    |childdoc.def| & definition file \\
    |cdocsamp.tex| & sample main file \\
    |cdocsch1.tex| & sample include file \\
    |cdocsch2.tex| & sample include file \\
    |cdocspt3.tex| & sample part file \\
    |cdocspt4.tex| & sample part file \\
    |cdocsdrf.tex| & sample redirection file \\
    |cdocsfn1.tex| & sample redirection file \\
    |cdocsfn2.tex| & sample redirection file \\
    |childdoc.pdf| & manual
\end{tabular}
\end{center}
%
The distribution consists of the files
|README.txt|, |childdoc.ins| and |childdoc.dtx|.
%
\begin{itemize}
\item
Run (pdf)\LaTeX{} on |childdoc.dtx|
to compile the manual |childdoc.pdf| (this file).
\item
Run \LaTeX{} on |childdoc.ins| to create the definitions file |childdoc.def|
and the sample |cdocsamp.tex| with include files
|cdocsch1.tex|, |cdocsch2.tex|, |cdocspt3.tex|, |cdocspt4.tex|,
|cdocsdrf.tex|, |cdocsfn1.tex|, |cdocsfn2.tex|.
Then copy the file |childdoc.def| to an appropriate directory of your \LaTeX{}
distribution, e.g.\ \textit{texmf-root}|/tex/latex/childdoc|.
\end{itemize}

%%%%%%%%%%%%%%%%%%%%%%%%%%%%%%%%%%%%%%%%%%%%%%%%%%%%%%%%%%%%%%%%%%%%%%%%%%%%%%%%
\subsection{Related CTAN Packages}

There are several other packages which offer a similar functionality:
%
\begin{itemize}
\item
The packages
\href{http://ctan.org/pkg/docmute}{\textsf{docmute}},
\href{http://ctan.org/pkg/includex}{\textsf{includex}} and
\href{http://ctan.org/pkg/standalone}{\textsf{standalone}}
provide commands to include only the document body of
a child file thus allowing both files to be compiled individually.
\item
The packages \href{http://ctan.org/pkg/subdocs}{\textsf{subdocs}}
and \href{http://ctan.org/pkg/subfiles}{\textsf{subfiles}}
provide structures in which the main and child documents can be
encapsulated and allowing them to be compiled individually.
The inclusion mechanism is different from the conventional |\include|.
\item
The package \href{http://ctan.org/pkg/combine}{\textsf{combine}}
is an elaborate solution to combine several documents into one.
\end{itemize}
%
See also the CTAN topic \href{http://ctan.org/topic/subdocs}{\textsf{subdocs}}
for further related packages.
The present package differs from the above solutions in that
a document structure constructed with the conventional |\include| mechanism
just needs two extra commands at the top of every file
such that all constituent files can be compiled individually.

%%%%%%%%%%%%%%%%%%%%%%%%%%%%%%%%%%%%%%%%%%%%%%%%%%%%%%%%%%%%%%%%%%%%%%%%%%%%%%%%
%\subsection{Feature Suggestions}
%
%The following is a list of features which may be useful for future
%versions of this package:
%%
%\begin{itemize}
%\item
%\ldots
%\end{itemize}

%%%%%%%%%%%%%%%%%%%%%%%%%%%%%%%%%%%%%%%%%%%%%%%%%%%%%%%%%%%%%%%%%%%%%%%%%%%%%%%%
\subsection{Revision History}

%%%%%%%%%%%%%%%%%%%%%%%%%%%%%%%%%%%%%%%%
\paragraph{v2.0:} 2018/12/30

\begin{itemize}
\item
immediate forward processing
\item
added |\childdocby| mechanism
\item
manual restructured
\end{itemize}

%%%%%%%%%%%%%%%%%%%%%%%%%%%%%%%%%%%%%%%%
\paragraph{v1.6:} 2018/01/17

\begin{itemize}
\item
application for development of include files
\item
corrections to manual
\end{itemize}

%%%%%%%%%%%%%%%%%%%%%%%%%%%%%%%%%%%%%%%%
\paragraph{v1.5:} 2017/05/21

\begin{itemize}
\item
more complete structuring introduced
\item
|\childdocof| introduced
\item
|\childdoc| renamed to |\childdocmain|
\item
|\childredirect| renamed to |\childdocforward| and |\childdocforwardprefix|
and functionality expanded
\end{itemize}

%%%%%%%%%%%%%%%%%%%%%%%%%%%%%%%%%%%%%%%%
\paragraph{v1.0:} 2017/04/27

\begin{itemize}
\item
manual and install package
\item
first version published on CTAN
\end{itemize}

%%%%%%%%%%%%%%%%%%%%%%%%%%%%%%%%%%%%%%%%
\paragraph{v0.6:} 2017/04/26

\begin{itemize}
\item
redirection mechanism added
\end{itemize}

%%%%%%%%%%%%%%%%%%%%%%%%%%%%%%%%%%%%%%%%
\paragraph{v0.5:} 2017/04/26

\begin{itemize}
\item
functionality in definition file
\end{itemize}


%%%%%%%%%%%%%%%%%%%%%%%%%%%%%%%%%%%%%%%%%%%%%%%%%%%%%%%%%%%%%%%%%%%%%%%%%%%%%%%%
%%%%%%%%%%%%%%%%%%%%%%%%%%%%%%%%%%%%%%%%%%%%%%%%%%%%%%%%%%%%%%%%%%%%%%%%%%%%%%%%
%%%%%%%%%%%%%%%%%%%%%%%%%%%%%%%%%%%%%%%%%%%%%%%%%%%%%%%%%%%%%%%%%%%%%%%%%%%%%%%%
\appendix

\settowidth\MacroIndent{\rmfamily\scriptsize 000\ }

 \DocInput{childdoc.dtx}

\end{document}
%</driver>
% \fi
%
% %%%%%%%%%%%%%%%%%%%%%%%%%%%%%%%%%%%%%%%%%%%%%%%%%%%%%%%%%%%%%%%%%%%%%%%%%%%%%%
% %%%%%%%%%%%%%%%%%%%%%%%%%%%%%%%%%%%%%%%%%%%%%%%%%%%%%%%%%%%%%%%%%%%%%%%%%%%%%%
% \section{Sample}
%\iffalse
%<*samplemain>
%\fi
%
% The following presents a sample document
% with two chapters, two parts, a title page,
% a compile flag as well as three forwarding files to set the flag.
% It consists of eight |.tex| files:
% \begin{center}
% \begin{tabular}{ll}
% |cdocsamp.tex|&main file\\
% |cdocsch1.tex|&include file for chapter 1\\
% |cdocsch2.tex|&include file for chapter 2\\
% |cdocspt3.tex|&include file for part 3\\
% |cdocspt4.tex|&include file for part 4\\
% |cdocsdrf.tex|&forwarding file for main file in draft mode\\
% |cdocsfi1.tex|&forwarding file for final version of chapter 1\\
% |cdocsfi2.tex|&forwarding file for final version of chapter 2\\
% \end{tabular}
% \end{center}
% Each of the eight files can be compiled directly by the \LaTeX{} compiler.
%
% %%%%%%%%%%%%%%%%%%%%%%%%%%%%%%%%%%%%%%
% \paragraph{Main File.}
%
% The main file is called |cdocsamp.tex|.
%
% Load the \textsf{childdoc} definitions and
% declare the filename for the main document:
%    \begin{macrocode}
\input{childdoc.def}
\childdocmain{}
%    \end{macrocode}

% Optional override for |\version| flag:
%    \begin{macrocode}
%%\ifchilddoc\else\providecommand{\version}{draft}\fi
%    \end{macrocode}

% Define the default values for the |\version| flag
% (|final| for the main file and |draft| for childs):
%    \begin{macrocode}
\ifchilddoc
\providecommand{\version}{draft}
\else
\providecommand{\version}{final}
\fi
%    \end{macrocode}

% Load the standard document class:
%    \begin{macrocode}
\documentclass[12pt]{article}
%    \end{macrocode}

% Start the document body:
%    \begin{macrocode}
\begin{document}
%    \end{macrocode}

% Declare a title page.
% Print title, part of document being processed and version flag:
%    \begin{macrocode}
\addtocounter{page}{-1}
\begin{center}
{\LARGE\bfseries{}childdoc example\par}
\vspace{1cm}
\ifchilddoc
\ifchilddocmanual part\else chapter\fi:
`\childdocname' of `\childdocjob'\par
\else
main document: `\childdocjob'\par
\fi
version: \version\par
\end{center}
\newpage
%    \end{macrocode}

% Manually include selected file,
% otherwise process as usual:
%    \begin{macrocode}
\ifchilddocmanual
\section*{part `\childdocname'}
\input{\childdocname}
\else
%    \end{macrocode}

% Include the two chapters:
%    \begin{macrocode}
\include{cdocsch1}
\include{cdocsch2}
%    \end{macrocode}

% Include the two parts unless only chapters should be displayed:
%    \begin{macrocode}
\ifchilddoc\else
\section{part three}
\input{cdocspt3}
\section{part four}
\input{cdocspt4}
\fi
%    \end{macrocode}

% Process as usual until here:
%    \begin{macrocode}
\fi
%    \end{macrocode}

% End of document body:
%    \begin{macrocode}
\end{document}
%    \end{macrocode}
%\iffalse
%</samplemain>
%\fi
%
% %%%%%%%%%%%%%%%%%%%%%%%%%%%%%%%%%%%%%%
% \paragraph{Chapter Include Files.}
%
% The include files are called |cdocsch1.tex| and |cdocsch2.tex|.
%
%\iffalse
%<*samplechap1|samplechap2>
%\fi

% Optional override for |\version| flag:
%    \begin{macrocode}
%%\providecommand{\version}{final}
%    \end{macrocode}

% Include the main document:
%    \begin{macrocode}
\input{childdoc.def}
\childdocof{cdocsamp}
%    \end{macrocode}

%\iffalse
%</samplechap1|samplechap2>
%\fi
%
%\iffalse
%<*samplechap1>
%\fi
% Some text for chapter 1:
%    \begin{macrocode}
\section{one}
some text in chapter one
%    \end{macrocode}

%\iffalse
%</samplechap1>
%\fi
% Some text for chapter 2:
%\iffalse
%<*samplechap2>
%\fi
%    \begin{macrocode}
\section{two}
more text in chapter two
%    \end{macrocode}

%\iffalse
%</samplechap2>
%\fi
%
% %%%%%%%%%%%%%%%%%%%%%%%%%%%%%%%%%%%%%%
% \paragraph{Part Include Files.}
%
% The include files are called |cdocspt3.tex| and |cdocspt4.tex|.
%
%\iffalse
%<*samplepart3|samplepart4>
%\fi

% Optional override for |\version| flag:
%    \begin{macrocode}
%%\providecommand{\version}{final}
%    \end{macrocode}

% Include the main document:
%    \begin{macrocode}
\input{childdoc.def}
\childdocby{cdocsamp}
%    \end{macrocode}

%\iffalse
%</samplepart3|samplepart4>
%\fi
%
%\iffalse
%<*samplepart3>
%\fi
% Some text for part 3:
%    \begin{macrocode}
some text in part three
%    \end{macrocode}

%\iffalse
%</samplepart3>
%\fi
% Some text for part 4:
%\iffalse
%<*samplepart4>
%\fi
%    \begin{macrocode}
more text in part four
%    \end{macrocode}

%\iffalse
%</samplepart4>
%\fi
%
% %%%%%%%%%%%%%%%%%%%%%%%%%%%%%%%%%%%%%%
% \paragraph{Forwarding for a Complete Draft.}
%
% The following forwarding file |cdocsdrf.tex|
% compiles the main document in draft mode:
%\iffalse
%<*sampledraft>
%\fi
%    \begin{macrocode}
\def\version{draft}
\input{childdoc.def}
\childdocforward{cdocsamp}
%    \end{macrocode}

%\iffalse
%</sampledraft>
%\fi
%
% %%%%%%%%%%%%%%%%%%%%%%%%%%%%%%%%%%%%%%
% \paragraph{Forwarding for Final Version of the Chapters.}
%
% The following forwarding files |cdocsfn1.tex| and |cdocsfn2.tex|
% (with identical content)
% compile the final versions of the child documents
% |cdocsch1.tex| and |cdocsch2.tex|, respectively:
%\iffalse
%<*samplefinal>
%\fi
%    \begin{macrocode}
\def\version{final}
\input{childdoc.def}
\childdocforwardprefix[cdocsamp]{cdocsfn}{cdocsch}
%    \end{macrocode}

%\iffalse
%</samplefinal>
%\fi
%
% %%%%%%%%%%%%%%%%%%%%%%%%%%%%%%%%%%%%%%
% \paragraph{Command Line Processing.}
%
% The following three command lines generate the output files
% |cdocscld|, |cdocscl1| and |cdocscl2|
% which should be identical to
% |cdocsdrf|, |cdocsch1| and |cdocsfn2|, respectively:
% \begin{center}
% \begin{tabular}{l}
% |latex -jobname cdocscld \|\\
% |  "\def\version{draft}\input{childdoc.def}\childdocforward{cdocsamp}"|\\
% |latex -jobname cdocscl1 \|\\
% |  "\input{childdoc.def}\childdocforward[cdocsamp]{cdocsch1}"|\\
% |latex -jobname cdocscl2 \|\\
% |  "\def\version{final}\input{childdoc.def}\childdocforward{cdocsch2}"|
% \end{tabular}
% \end{center}
% Note that the trailing backslash on each first line
% merely continues the input to the second line
% (for convenient cut ant paste).
% Furthermore, the command |latex| can be replaced by any
% of its alternative versions such as |pdflatex|.
%
% %%%%%%%%%%%%%%%%%%%%%%%%%%%%%%%%%%%%%%%%%%%%%%%%%%%%%%%%%%%%%%%%%%%%%%%%%%%%%%
% %%%%%%%%%%%%%%%%%%%%%%%%%%%%%%%%%%%%%%%%%%%%%%%%%%%%%%%%%%%%%%%%%%%%%%%%%%%%%%
% \section{Implementation}
%\iffalse
%<*package>
%\fi
%
% This section describes the definitions file |childdoc.def|.

% The definitions cannot be loaded using |\usepackage| or |\RequirePackage|
% which has a mechanism to prevent loading a style file more than once.
% When loading the definitions by means of |\input|
% multiple instances have to be prevented manually:
%\iffalse
%This code needs to be before the `\ProvidesFile' directive
%which is defined at the beginning of this file.
%Therefore it is also placed there and commented out here.
%</package>
%<*discard>
%\fi
%    \begin{macrocode}
\ifdefined\childdocmain\endinput\fi
%    \end{macrocode}
%\iffalse
%</discard>
%<*package>
%\fi
%
% \macro{\ifchilddoc}
% \macro{\ifchilddocmanual}
% The conditional |\ifchilddoc| tells whether a
% child (true) or main (false) document is being compiled.
% The conditional |\ifchilddocmanual| tells whether
% the |\includeonly| mechanism is used (false) or
% the selection of child files must be performed manually (true).
% The definitions initialise to false:
%    \begin{macrocode}
\newif\ifchilddoc
\newif\ifchilddocmanual
%    \end{macrocode}

% \macro{\childdocname}
% \macro{\childdocjob}
% The macro |\childdocname| stores the name of the main document
% to be compiled. The macro |\childdocjob| stores the name of
% the document on which the \LaTeX{} compiler was originally invoked.
% The content of |\jobname| cannot be compared
% to filenames specified in the source due to different catcodes.
% The following code rescans |\jobname|, stores the result
% in |\childdocname| and saves a copy in |\childdocjob|:
%    \begin{macrocode}
\edef\childdocname{\scantokens\expandafter{\jobname\noexpand}}
\let\childdocjob\childdocname
%    \end{macrocode}

% \macro{\childdocdisable}
% The macro |\childdocdisable| prevents the main file
% from being processed more than once.
% At this stage, the main document command |\childdocmain|
% is assumed to be called once again where it should do nothing.
% Any subsequent call to it should prevent
% a secondary processing of the main document
% It overwrites the forwarding commands
% |\childdocof| and |\childdocforward|
% with empty macros to prevent further inclusions of the main document:
%    \begin{macrocode}
\newcommand{\childdocdisable}
{
  \renewcommand{\childdocmain}[1]{\renewcommand{\childdocmain}[1]{\endinput}}
  \renewcommand{\childdocof}[1]{}
  \renewcommand{\childdocby}[2][]{}
  \renewcommand{\childdocforward}[2][]{}
  \renewcommand{\childdocdisable}{}
}
%    \end{macrocode}

% \macro{\childdocmain}
% The macro |\childdocmain| is to be called at the top of the main file
% with nothing or the main filename (without extension) as argument.
% First, it breaks loops.
% If the argument is not empty and does not match |\childdocname|
% (which is set by the first inclusion of |childdoc.def|),
% |\ifchilddoc| is set to true, |\includeonly| is applied to the child file
% and |\jobname| is set to the main file
% (for proper handling of |.aux| files):
%    \begin{macrocode}
\newcommand{\childdocmain}[1]
{
  \childdocdisable\childdocmain{}
  \if?#1?\else
    \begingroup
      \def\childdoctmp{#1}
      \ifx\childdoctmp\childdocname
        \def\childdoctmp{}
      \else
        \def\childdoctmp
        {
          \childdoctrue
          \includeonly{\childdocname}
          \def\childdocjob{#1}
          \def\jobname{#1}
        }
      \fi
      \expandafter
    \endgroup
    \childdoctmp
  \fi
}
%    \end{macrocode}

% \macro{\childdocof}
% The command |\childdocof| redirects
% compilation to the main file |#1|.
%    \begin{macrocode}
\newcommand{\childdocof}[1]
{
  \childdocdisable
  \childdoctrue
  \includeonly{\childdocname}
  \def\jobname{#1}
  \def\childdocjob{#1}
  \input{#1}
}
%    \end{macrocode}

% \macro{\childdocby}
% The command |\childdocby| ....
%    \begin{macrocode}
\newcommand{\childdocby}[2][]
{
  \childdocdisable
  \childdoctrue
  \childdocmanualtrue
  \if?#1?\else
    \def\jobname{#2}
  \fi
  \def\childdocjob{#2}
  \input{#2}
  \endinput
}
%    \end{macrocode}

% \macro{\childdocforward}
% The command |\childdocforward| redirects
% compilation to the main file or
% (if the optional argument is given) a child file.
% Parameters are set as if the main file
% or a child file starting with |\childdocof| was compiled.
% Then compilation is handed over to the main file:
%    \begin{macrocode}
\newcommand{\childdocforward}[2][]
{
  \begingroup
    \if?#1?
      \def\childdoctmp
      {
        \def\childdocname{#2}
        \def\childdocjob{#2}
        \def\jobname{#2}
        \input{#2}
        \endinput
      }
    \else
      \def\childdoctmp
      {
        \childdocdisable
        \def\childdocname{#2}
        \childdoctrue
        \includeonly{#2}
        \def\childdocjob{#1}
        \def\jobname{#1}
        \input{#1}
        \endinput
      }
    \fi
    \expandafter
  \endgroup
  \childdoctmp
}
%    \end{macrocode}

% \macro{\childdocforwardprefix}
% The command |\childdocforwardprefix| redirects
% compilation to the main or a child file by means of a pattern.
% The prefix |#1| in the current filename is replaced by |#2|
% and the suffix of the current filename is kept
% (it is assumed that the filename does not contain the substring `|~~~|'
% which is used as a delimiter).
% Compilation is handed over to the new file by |\childdocforward|:
%    \begin{macrocode}
\newcommand{\childdocforwardprefix}[3][]
{
  \begingroup
    \def\childdocextract #2##1~~~{\def\childdoctmp{\childdocforward[#1]{#3##1}}}
    \expandafter\childdocextract\childdocname~~~
    \expandafter
  \endgroup
  \childdoctmp
}
%    \end{macrocode}

% \macro{\childdoc}
% The deprecated macro |\childdoc| is a legacy version of |\childdocmain|:
%    \begin{macrocode}
\newcommand{\childdoc}{\childdocmain}
%    \end{macrocode}

% \macro{\childdocredirect}
% The deprecated macro |\childdocredirect| is a legacy version
% of |\childdocforward| and |\childdocforwardprefix|:
%    \begin{macrocode}
\newcommand{\childdocredirect}[2][]
{
  \begingroup
    \if?#1?
      \def\childdoctmp{\childdocforward{#2}}
    \else
      \def\childdoctmp{\childdocforwardprefix{#1}{#2}}
    \fi
    \expandafter
  \endgroup
  \childdoctmp
}
%    \end{macrocode}

%\iffalse
%</package>
%\fi
%
\endinput
|\\
|\childdocforward{|\textit{main}|}|\\
\end{tabular}
\end{center}
%
or alternatively with:
%
\begin{center}
\begin{tabular}{l}
|% \iffalse
%
% childdoc.dtx Copyright (C) 2017-2018 Niklas Beisert
%
% This work may be distributed and/or modified under the
% conditions of the LaTeX Project Public License, either version 1.3
% of this license or (at your option) any later version.
% The latest version of this license is in
%   http://www.latex-project.org/lppl.txt
% and version 1.3 or later is part of all distributions of LaTeX
% version 2005/12/01 or later.
%
% This work has the LPPL maintenance status `maintained'.
%
% The Current Maintainer of this work is Niklas Beisert.
%
% This work consists of the files childdoc.dtx and childdoc.ins
% and the derived files childdoc.def and cdocsamp.tex with
% cdocsch1.tex, cdocsch2.tex, cdocsdrf.tex, cdocsfn1.tex, cdocsfn2.tex.
%
%<package>\ifdefined\childdocmain\endinput\fi
%<package>\ProvidesFile{childdoc.def}[2018/12/30 v2.0 child document driver]
%<samplemain>\ProvidesFile{cdocsamp.tex}[2018/12/30 v2.0 sample for childdoc]
%<*driver>
%\ProvidesFile{childdoc.drv}[2018/12/30 v2.0 childdoc reference manual file]
\PassOptionsToClass{10pt,a4paper}{article}
\documentclass{ltxdoc}

\usepackage[margin=35mm]{geometry}
\usepackage{hyperref}
\usepackage{hyperxmp}
\usepackage[usenames]{color}

\hypersetup{colorlinks=true}
\hypersetup{pdfstartview=FitH}
\hypersetup{pdfpagemode=UseNone}
\hypersetup{pdfsource={}}
\hypersetup{pdflang={en-UK}}
\hypersetup{pdfcopyright={Copyright 2017-2018 Niklas Beisert.
  This work may be distributed and/or modified under the
  conditions of the LaTeX Project Public License, either version 1.3
  of this license or (at your option) any later version.}}
\hypersetup{pdflicenseurl={http://www.latex-project.org/lppl.txt}}
\hypersetup{pdfcontactaddress={ETH Zurich, ITP, HIT K,
  Wolfgang-Pauli-Strasse 27}}
\hypersetup{pdfcontactpostcode={8093}}
\hypersetup{pdfcontactcity={Zurich}}
\hypersetup{pdfcontactcountry={Switzerland}}
\hypersetup{pdfcontactemail={nbeisert@itp.phys.ethz.ch}}
\hypersetup{pdfcontacturl={http://people.phys.ethz.ch/\xmptilde nbeisert/}}

\newcommand{\secref}[1]{\hyperref[#1]{section \ref*{#1}}}

\parskip1ex
\parindent0pt
\let\olditemize\itemize
\def\itemize{\olditemize\parskip0pt}

\begin{document}

\title{The \textsf{childdoc} Package}
\hypersetup{pdftitle={The childdoc Package}}
\author{Niklas Beisert\\[2ex]
  Institut f\"ur Theoretische Physik\\
  Eidgen\"ossische Technische Hochschule Z\"urich\\
  Wolfgang-Pauli-Strasse 27, 8093 Z\"urich, Switzerland\\[1ex]
  \href{mailto:nbeisert@itp.phys.ethz.ch}
  {\texttt{nbeisert@itp.phys.ethz.ch}}}
\hypersetup{pdfauthor={Niklas Beisert}}
\hypersetup{pdfsubject={Manual for the LaTeX2e Package childdoc}}
\date{30 December 2018, \textsf{v2.0}}
\maketitle

\begin{abstract}\noindent
\textsf{childdoc} is a \LaTeXe{} package
that enables the direct compilation
of document sections included by |\include|
to individual files.
\end{abstract}

\begingroup
\parskip0ex
\tableofcontents
\endgroup

%%%%%%%%%%%%%%%%%%%%%%%%%%%%%%%%%%%%%%%%%%%%%%%%%%%%%%%%%%%%%%%%%%%%%%%%%%%%%%%%
%%%%%%%%%%%%%%%%%%%%%%%%%%%%%%%%%%%%%%%%%%%%%%%%%%%%%%%%%%%%%%%%%%%%%%%%%%%%%%%%
\section{Introduction}

\LaTeX{} provides a mechanism to structure a large document (such as a book)
into a main file and several child files (containing the chapters)
using the |\include| command.
This mechanism is beneficial for documents
which span hundreds of pages in order to
make the source file(s) more manageable.
Moreover, compilation can be restricted to
selected child files by means of the |\includeonly| command.
The latter feature can be used to reduce the compilation time while editing
(this was significantly more useful in the earlier days of \LaTeX{})
or to generate a smaller document which is easier to navigate.
Another application of |\includeonly| is to generate
documents consisting of selected parts of the complete document.

However, there are a few drawbacks of the plain |\include| mechanism:
\begin{itemize}
\item
The child files cannot be compiled on their own,
they can only be compiled via the main file.
A naive editing environment
(such as a text editor with an option
to have the current file processed by \LaTeX)
may require one to switch to the main file before compiling;
attempting to compile the child file produces errors.
\item
The main file must be modified (each time)
to adjust the |\includeonly| command
to the present needs. This easily leaves the main file in a messy state.
\item
The generated document will always carry the filename
of the main document. This is inconvenient if
several child files are to be compiled and
to be kept for distribution.
\end{itemize}

The present package provides a simple interface
to make child files individually compilable by \LaTeX{}.
Compiling a child file then has the same effect as compiling
the main file with an |\includeonly| command
to select the appropriate child.
Moreover the generated document will carry the name of the child
rather than the main file.
This resolves all three above issues.

This feature is meant to make the editing of books,
thesis documents and lecture notes somewhat more convenient.
However, the package can also be used efficiently for
composing a series of documents (such as exercise sheets)
which are typically distributed individually.
It then assists the author in generating the individual documents
(potentially in different versions)
as well as a document containing the collected series.
Another application is in developing style files
or other kinds of included material
where compilation of the style file could redirect
to a sample or test file.

%%%%%%%%%%%%%%%%%%%%%%%%%%%%%%%%%%%%%%%%%%%%%%%%%%%%%%%%%%%%%%%%%%%%%%%%%%%%%%%%
%%%%%%%%%%%%%%%%%%%%%%%%%%%%%%%%%%%%%%%%%%%%%%%%%%%%%%%%%%%%%%%%%%%%%%%%%%%%%%%%
\section{Usage}

First of all, the package \textsf{childdoc} is \emph{not} a standard
\LaTeXe{} |.sty| style file! Therefore it needs to be invoked in
a non-standard way.

%%%%%%%%%%%%%%%%%%%%%%%%%%%%%%%%%%%%%%%%%%%%%%%%%%%%%%%%%%%%%%%%%%%%%%%%%%%%%%%%
\subsection{Included Files}
\label{sec:include}

%%%%%%%%%%%%%%%%%%%%%%%%%%%%%%%%%%%%%%%%
\DescribeMacro{\childdocmain}
To use the package, add the commands
\begin{center}
\begin{tabular}{l}
|\input{childdoc.def}|\\
|\childdocmain{}|\\
\end{tabular}
\end{center}
at the very top of the main \LaTeX{} file,
in particular \emph{before} the |\documentclass| statement!
The argument of |\childdocmain| should be left empty
(but it must be present).

%%%%%%%%%%%%%%%%%%%%%%%%%%%%%%%%%%%%%%%%
\DescribeMacro{\childdocof}
Furthermore, add the commands
\begin{center}
\begin{tabular}{l}
|\input{childdoc.def}|\\
|\childdocof{|\textit{main}|}|\\
\end{tabular}
\end{center}
at the top of every child file \textit{child}
which is included by |\include{|\textit{child}|}|
from within the main file
(or at least for those files to be compiled individually).
The argument \textit{main} must be the filename of the main file.

There are a couple of
considerations in setting up the main and child documents:

%%%%%%%%%%%%%%%%%%%%%%%%%%%%%%%%%%%%%%%%
\paragraph{Restrictions.}

Please note the following restrictions:
\begin{itemize}
\item
|\childdocmain| must be called with one argument \textit{main}
to ensure compatibility with earlier version of the package.
It must either be empty (|\childdocmain{}|)
or precisely match the filename of the main file in which it is specified.
See \secref{sec:detection} for further information.
\item
The filename \textit{main} must be specified without the |.tex| extension.
\item
The filename \textit{main} is case sensitive
(even in case-insensitive file systems)
due to internal string comparison.
\item
The argument \textit{main} should be fully expanded, it cannot be a macro.
\item
Subdirectories and special characters should be avoided in filenames.
\item
The command |\childdocmain{|\textit{main}|}| must be followed by a whitespace.
It should not be followed immediately by another command
or by a comment mark `|%|'.
This is because the \TeX{} parser reads the token immediately following
the argument of |\childdocmain| and puts it
at the beginning of every child section;
however, a white\-space is ignored.
\end{itemize}

%%%%%%%%%%%%%%%%%%%%%%%%%%%%%%%%%%%%%%%%
\paragraph{Content of Main File.}

It is advisable to place all content in the child files included by |\include|.
Any output contained in the main file will appear in all child documents
unless suppressed manually;
it cannot be suppressed automatically by the |\includeonly| directive
and thus should normally be avoided.
A method to include some content in the main file
by means of conditional processing is described in \secref{sec:conditional}.

%%%%%%%%%%%%%%%%%%%%%%%%%%%%%%%%%%%%%%%%
\paragraph{Page Numbering.}

When only a part of the document is compiled,
the appropriate numbering of pages
(as well as other status parameters)
is determined from the |.aux| files.
The latter contain information from previous passes.
However this information needs to propagate through
all intermediate child documents.
Therefore the page numbering in child documents may well
be inconsistent until the complete document is compiled at least once.

A useful (if unconventional) way to always ensure a consistent
page numbering is to restart the numbering in each child document
and denote the pages by `\textit{child}|.|\textit{page}'
where \textit{child} represents the chapter/section number of the child file.
This can be achieved by the command
|\numberwithin{page}{|\textit{child}|}|
of the \textsf{amsmath} package
where \textit{child} can be |chapter| or |section|
depending on the chosen structuring.
Alternatively, one can modify the macro |\thepage| appropriately
and reset the counter |page| at the start of each child file.

%%%%%%%%%%%%%%%%%%%%%%%%%%%%%%%%%%%%%%%%%%%%%%%%%%%%%%%%%%%%%%%%%%%%%%%%%%%%%%%%
\subsection{Conditional Processing}
\label{sec:conditional}

The package provides a mechanism to compile different versions
of a document. To customise the versions further some conditional processing
can come in handy to distinguish which version is being compiled.
The package provides two macros to describe the compilation context:

%%%%%%%%%%%%%%%%%%%%%%%%%%%%%%%%%%%%%%%%
\DescribeMacro{\ifchilddoc}
The conditional |\ifchilddoc| distinguishes between the compilation of
child documents and the main document:
%
\begin{center}
|\ifchilddoc |\textit{child-code}| |[|\||else |\textit{main-code}]| \||fi|
\end{center}

%%%%%%%%%%%%%%%%%%%%%%%%%%%%%%%%%%%%%%%%
\DescribeMacro{\childdocname}
\DescribeMacro{\childdocjob}
The macro |\childdocname| contains the filename (without extension)
of the main or child file being processed.
Note that |\childdocjob| will always contain the name of the main file.

%%%%%%%%%%%%%%%%%%%%%%%%%%%%%%%%%%%%%%%%
\paragraph{Title Page.}

Conditional processing can be used to include a title or banner page
in the main document when proper precautions are taken.
Importantly, the code in the main file should ensure that the page counter
(as well as other status parameters which are stored in the |.aux| files)
takes the same value after the conditional processing.
Otherwise the page numbers may take divergent values
depending on which part is compiled.

For example, a title page could be declared by:
%
\begin{center}
\begin{tabular}{l}
|\ifchilddoc\||else|\\
|\addtocounter{page}{-1}|\\
\textit{code for title page}\\
|\newpage|\\
|\||fi|
\end{tabular}
\end{center}
%
A banner page for the child documents can be generated by:
%
\begin{center}
\begin{tabular}{l}
|\ifchilddoc|\\
|\addtocounter{page}{-1}|\\
\textit{code for banner page}\\
|\newpage|\\
|\||fi|
\end{tabular}
\end{center}
%
Here one could write a message such as:
\begin{center}
|This is the part \childdocname{} of \childdocjob{}.|
\end{center}

%%%%%%%%%%%%%%%%%%%%%%%%%%%%%%%%%%%%%%%%%%%%%%%%%%%%%%%%%%%%%%%%%%%%%%%%%%%%%%%%
\subsection{Flags}
\label{sec:flags}

The package makes it easy to generate different versions
of the main or child documents.
To this end compilation flags can be defined
and assigned different default values.
They will be particularly useful in conjunction
with the forwarding mechanism described in \secref{sec:forward}.

For example, it may be useful to have a flag |\version|
which can be set to |draft| or |final|.
The document source will contain some conditional code
depending on the value of |\version|.
Suppose further, the flag should default to |final| for the main file
and to |draft| for child files
which is a natural assignment for editing the document.
This is achieved by placing the following code
in the preamble of the main document
(below the |\childdocmain| directive):
%
\begin{center}
\begin{tabular}{l}
|\ifchilddoc|\\
|\providecommand{\version}{draft}|\\
|\||else|\\
|\providecommand{\version}{final}|\\
|\||fi|
\end{tabular}
\end{center}
%
The definition by |\providecommand| makes sure
that previous definitions are not overwritten.
Further statements |\providecommand{\version}{...}|
can thus be added before the above code to override it.

For the main file, one might add a line
(between |\childdocmain| and the above block)
%
\begin{center}
|%\ifchilddoc\||else\providecommand{\version}{draft}\||fi|
\end{center}
%
which can be uncommented to produce a draft version.
Likewise one can add a line to the very top of a child file
(above the |\childdocof{|\textit{main}|}| directive)
%
\begin{center}
|%\providecommand{\version}{final}|
\end{center}
%
which can be uncommented to produce the final version of this child document.

%%%%%%%%%%%%%%%%%%%%%%%%%%%%%%%%%%%%%%%%%%%%%%%%%%%%%%%%%%%%%%%%%%%%%%%%%%%%%%%%
\subsection{Forwarding}
\label{sec:forward}

Different versions of the main or child documents
using compilation flags as described in \secref{sec:flags}
can be (permanently) stored in different files
for convenient compilation, viewing and distribution.
To this end, the package defines a command
to pass on compilation to a different file:

%%%%%%%%%%%%%%%%%%%%%%%%%%%%%%%%%%%%%%%%
\DescribeMacro{\childdocforward}
The command |\childdocforward| redirects processing to
another source file:
%
\begin{center}
\begin{tabular}{l}
|\input{childdoc.def}|\\
|\childdocforward[|\textit{main}|]{|\textit{dest}|}|\\
\end{tabular}
\end{center}
%
The argument \textit{dest} is the destination file
(without extension).
It should be the main file or one of the child files.
Note that further \textsf{childdoc} directives
such as |\childdocof| and |\childdocforward|
in the indicated file will be processed in this form.
The optional argument \textit{main}
passes on directly to the main file \textit{main}
while pretending to compile the child \textit{dest}.
This form behaves as if \textit{dest}
issues |\childdocof{|\textit{main}|}| right away,
and no further \textsf{childdoc} directives will be processed.

%%%%%%%%%%%%%%%%%%%%%%%%%%%%%%%%%%%%%%%%
\DescribeMacro{\...prefix}
In the alternative form |\childdocforwardprefix|,
%
\begin{center}
\begin{tabular}{l}
|\input{childdoc.def}|\\
|\childdocforwardprefix[|\textit{main}|]{|\textit{prefix}|}{|\textit{dest}|}|
\end{tabular}
\end{center}
%
the destination file is determined by a pattern
depending on the current file:
To make this work, the current file must be called
`{\textit{prefix}\hspace{0.2em}\textit{suffix}}'
with \textit{prefix} matching precisely the argument.
Processing is then passed on to the file
`{\textit{dest}\hspace{0.2em}\textit{suffix}}'.
Surely, the same effect is achieved by
directly specifying the
argument `{\textit{dest}\hspace{0.2em}\textit{suffix}}'
in the first form.
However, that requires to set up a different file
for each child. With the alternative form of the command
all these files can have exactly the same content
which simplifies setting them up and maintaining them.

For example, the following file |draft.tex|
with a compilation flag |\version| as described in \secref{sec:flags}
compiles the main document as a draft:
%
\begin{center}
\begin{tabular}{l}
|\def\version{draft}|\\
|\input{childdoc.def}|\\
|\childdocforward{|\textit{main}|}|
\end{tabular}
\end{center}
%
Likewise, the following files |final|\textit{nn}|.tex|
compile the final version of the child document
|child|\textit{nn}|.tex|:
%
\begin{center}
\begin{tabular}{l}
|\def\version{final}|\\
|\input{childdoc.def}|\\
|\childdocforwardprefix{final}{child}|
\end{tabular}
\end{center}
%

Note that when several versions of a main file and/or of each child file
are to be generated, it may be convenient to set up a |Makefile| or
shell script to automatise the process.

%%%%%%%%%%%%%%%%%%%%%%%%%%%%%%%%%%%%%%%%%%%%%%%%%%%%%%%%%%%%%%%%%%%%%%%%%%%%%%%%
\subsection{Command Line Processing}
\label{sec:commandline}

The effect of redirection files can also be achieved by invoking
the \LaTeX{} compiler with a more elaborate command line.
Most conveniently this should be done as part
of a shell script or a |Makefile|.

When using \textsf{childdoc} in the main file, the following
command lines effectively perform a redirection
(note that depending on the shell being used,
backslashes may have to be doubled: `|\|' $\to$ `|\\|'):
%
\begin{center}
|... -jobname "|\textit{target}|" |\\|"|[\textit{flags}]%
|\input{childdoc.def}\childdocforward[|\textit{main}|]{|\textit{dest}|}"|
\end{center}
%
Here \textit{target} is the name of the output file,
\textit{main} is the name of the main file
and \textit{dest} is the name of the main or child file to be processed
(all filenames without extensions).
The optional argument \textit{main} can be omitted
if \textit{main} matches \textit{dest}.
Optionally, compilation \textit{flags} can be defined via |\def| commands.
This command line makes the \TeX{} engine believe
it is compiling the file \textit{target}
whose content is specified as the latter parameter.
The provided code then forwards the processing to
\textit{main} or \textit{dest} as described in \secref{sec:forward}.

%%%%%%%%%%%%%%%%%%%%%%%%%%%%%%%%%%%%%%%%%%%%%%%%%%%%%%%%%%%%%%%%%%%%%%%%%%%%%%%%
\subsection{Include by Input}
\label{sec:input}

Including child documents by |\include| has some restrictions by design.
Most notably, the content of a child document always occupies
its own set of pages; pages cannot be shared between child documents.
Usually, this behaviour makes perfect sense
because each child document contain an essential part of the document.
However, in some situations it may be desirable to compose
a document from a collection of parts
without having mandatory page breaks between then.
For this case, the package
provides a mechanism to include parts
by |\input| which can also be processed individually.
However, by construction this mechanism
requires manual handling of the content to be output.

%%%%%%%%%%%%%%%%%%%%%%%%%%%%%%%%%%%%%%%%
\DescribeMacro{\ifchilddocmanual}
The main file should be prepared as usual, see \secref{sec:include}.
However, the document body must make a distinction
between processing of an individual part and of the main document, e.g.:
%
\begin{center}
\begin{tabular}{l}
|\ifchilddocmanual|\\
|\input{\childdocname}|\\
|\||else|\\
\textit{document body with }|\input{|\textit{part}|}|\\
|\||fi|
\end{tabular}
\end{center}
%
The conditional |\ifchilddocmanual| is true whenever
a part to be included by |\input| is being compiled,
and the name of the part is stored in |\childdocname|.

%%%%%%%%%%%%%%%%%%%%%%%%%%%%%%%%%%%%%%%%
\DescribeMacro{\childdocby}
Each part to be included by |\input| should start with:
%
\begin{center}
\begin{tabular}{l}
|\input{childdoc.def}|\\
|\childdocby{|\textit{main}|}|\\
\end{tabular}
\end{center}
%
The directive |\childdocby| is similar to |\childdocof|
described in \secref{sec:include},
but the subsequent selection of content must be done manually.
To that end, both |\ifchilddoc| and |\ifchilddocmanual|
will be true upon processing of a part,
and the name of the part is stored in |\childdocname|.
Note that |\jobname| will be set to the filename of the current part
so that each part receives an individual |.aux| file
that does not interfere with the |.aux| file(s) of the main document.
This behaviour can be altered by the alternative form
|\childdocby[*]{|\textit{main}|}| (with a non-empty optional argument)
which uses the |.aux| file of the main document
by setting |\jobname| to \textit{main}.

%%%%%%%%%%%%%%%%%%%%%%%%%%%%%%%%%%%%%%%%%%%%%%%%%%%%%%%%%%%%%%%%%%%%%%%%%%%%%%%%
\subsection{Driver Development}
\label{sec:driver}

The \textsf{childdoc} mechanism can also be use for the development
of definition files such as \LaTeX{} styles or classes.
This case differs from the above setup with multiple parts
included by |\include| in that no |\includeonly| should be invoked.
This can be achieved by starting the include file
(before |\ProvidesPackage|) with:
%
\begin{center}
\begin{tabular}{l}
|\input{childdoc.def}|\\
|\childdocforward{|\textit{main}|}|\\
\end{tabular}
\end{center}
%
or alternatively with:
%
\begin{center}
\begin{tabular}{l}
|\input{childdoc.def}|\\
|\childdocby{|\textit{main}|}|\\
\end{tabular}
\end{center}
%
Both forms have slightly different effects as described above.
The main file is prepared as usual, see \secref{sec:include}.

%%%%%%%%%%%%%%%%%%%%%%%%%%%%%%%%%%%%%%%%%%%%%%%%%%%%%%%%%%%%%%%%%%%%%%%%%%%%%%%%
\subsection{Legacy Detection}
\label{sec:detection}

The directive |\childdocmain| in the main file can detect
whether the complete document or merely a child is to be compiled
even without using the directive |\childdocof|.
This method is deprecated because it is less robust
and there is no compelling reason to use it;
it is merely provided for backward compatibility
and it may be removed in future versions.

If the detection mechanism is to be used,
it is mandatory to correctly specify
the filename of the main file as the argument of |\childdocmain|:
%
\begin{center}
\begin{tabular}{l}
|\input{childdoc.def}|\\
|\childdocmain{|\textit{main}|}|\\
\end{tabular}
\end{center}
%
If |\jobname| does not match the argument \textit{main} of |\childdocmain|,
it is assumed that |\jobname| points to the child file to be compiled.
When using |\childdocmain| with the main file specified as argument,
it suffices to start a child file
with just |\input{|\textit{main}|}|
without loading of the package and using |\childdocof|.
If instead all processing is done
with the appropriate \textsf{childdoc} directives,
the argument of \textit{main} of |\childdocmain| can be empty.

An alternative version of the command line processing described
in \secref{sec:commandline} using the detection mechanism reads:
%
\begin{center}
|... -jobname "|\textit{target}|" "|[\textit{flags}]%
[|\def\jobname{|\textit{dest}|}|]|\input{|\textit{main}|}"|
\end{center}

%%%%%%%%%%%%%%%%%%%%%%%%%%%%%%%%%%%%%%%%%%%%%%%%%%%%%%%%%%%%%%%%%%%%%%%%%%%%%%%%
\subsection{Manual Code}
\label{sec:manual}

In case one cannot be certain whether the definitions file |childdoc.def|
is installed on the target \TeX{} distribution
and one prefers not to ship it,
it is conceivable to paste a few relevant commands into the sources.

To that end, drop all statements |\input{childdoc.def}|
and perform the replacements as outlined below.
Instead of |\childdocmain{|\textit{main}|}| add the following code
to the top of the main file:
%
\begin{center}
\begin{tabular}{l}
|\||ifdefined\childdocname\endinput\||fi\newif\ifchilddoc|\\
|\edef\childdocname{\scantokens\expandafter{\jobname\noexpand}}|\\
|\def\childdocmain{|\textit{main}|}\||ifx\childdocmain\childdocname\||else|\\
|\childdoctrue\includeonly{\childdocname}\let\jobname\childdocmain\||fi|\\
\end{tabular}
\end{center}
%
Instead of |\childdocof{|\textit{main}|}| just include the main file
at the top of each child file:
%
\begin{center}
|\input{|\textit{main}|}|
\end{center}
%
A simple redirection |\childdocforward{|\textit{dest}|}| is achieved by:
%
\begin{center}
|\def\jobname{|\textit{dest}|}\input{\jobname}|
\end{center}
%
The redirection with prefix
|\childdocforwardprefix[|\textit{prefix}|]{|\textit{dest}|}|
is accomplished by:
%
\begin{center}
\begin{tabular}{l}
|{\edef\jobname{\scantokens\expandafter{\jobname\noexpand}}|\\
|\def\redirectjob |\textit{prefix}|#1~~~{\gdef\jobname{|\textit{dest}|#1}}|\\
|\expandafter\redirectjob\jobname~~~}\input{\jobname}|
\end{tabular}
\end{center}

In an alternative approach,
child documents can be compiled by a specific command line
without additional code or specific definitions:
%
\begin{center}
|... -jobname "|\textit{target}|" "|[\textit{flags}]%
|\includeonly{|\textit{dest}|}\input{|\textit{main}|}"|
\end{center}
%

%%%%%%%%%%%%%%%%%%%%%%%%%%%%%%%%%%%%%%%%%%%%%%%%%%%%%%%%%%%%%%%%%%%%%%%%%%%%%%%%
%%%%%%%%%%%%%%%%%%%%%%%%%%%%%%%%%%%%%%%%%%%%%%%%%%%%%%%%%%%%%%%%%%%%%%%%%%%%%%%%
\section{Information}

%%%%%%%%%%%%%%%%%%%%%%%%%%%%%%%%%%%%%%%%%%%%%%%%%%%%%%%%%%%%%%%%%%%%%%%%%%%%%%%%
\subsection{Copyright}

Copyright \copyright{} 2017--2018 Niklas Beisert

This work may be distributed and/or modified under the
conditions of the \LaTeX{} Project Public License, either version 1.3
of this license or (at your option) any later version.
The latest version of this license is in
  \url{http://www.latex-project.org/lppl.txt}
and version 1.3 or later is part of all distributions of \LaTeX{}
version 2005/12/01 or later.

This work has the LPPL maintenance status `maintained'.

The Current Maintainer of this work is Niklas Beisert.

This work consists of the files |README.txt|, |childdoc.ins| and |childdoc.dtx|
as well as the derived files |childdoc.def|, |cdocsamp.tex|
with |cdocsch1.tex|, |cdocsch2.tex|, |cdocspt3.tex|, |cdocspt4.tex|,
|cdocsdrf.tex|, |cdocsfn1.tex|, |cdocsfn2.tex|
as well as |childdoc.pdf|.

%%%%%%%%%%%%%%%%%%%%%%%%%%%%%%%%%%%%%%%%%%%%%%%%%%%%%%%%%%%%%%%%%%%%%%%%%%%%%%%%
\subsection{Files and Installation}

The package consists of the files:
%
\begin{center}
\begin{tabular}{ll}
    |README.txt|   & readme file \\
    |childdoc.ins| & installation file \\
    |childdoc.dtx| & source file \\
    |childdoc.def| & definition file \\
    |cdocsamp.tex| & sample main file \\
    |cdocsch1.tex| & sample include file \\
    |cdocsch2.tex| & sample include file \\
    |cdocspt3.tex| & sample part file \\
    |cdocspt4.tex| & sample part file \\
    |cdocsdrf.tex| & sample redirection file \\
    |cdocsfn1.tex| & sample redirection file \\
    |cdocsfn2.tex| & sample redirection file \\
    |childdoc.pdf| & manual
\end{tabular}
\end{center}
%
The distribution consists of the files
|README.txt|, |childdoc.ins| and |childdoc.dtx|.
%
\begin{itemize}
\item
Run (pdf)\LaTeX{} on |childdoc.dtx|
to compile the manual |childdoc.pdf| (this file).
\item
Run \LaTeX{} on |childdoc.ins| to create the definitions file |childdoc.def|
and the sample |cdocsamp.tex| with include files
|cdocsch1.tex|, |cdocsch2.tex|, |cdocspt3.tex|, |cdocspt4.tex|,
|cdocsdrf.tex|, |cdocsfn1.tex|, |cdocsfn2.tex|.
Then copy the file |childdoc.def| to an appropriate directory of your \LaTeX{}
distribution, e.g.\ \textit{texmf-root}|/tex/latex/childdoc|.
\end{itemize}

%%%%%%%%%%%%%%%%%%%%%%%%%%%%%%%%%%%%%%%%%%%%%%%%%%%%%%%%%%%%%%%%%%%%%%%%%%%%%%%%
\subsection{Related CTAN Packages}

There are several other packages which offer a similar functionality:
%
\begin{itemize}
\item
The packages
\href{http://ctan.org/pkg/docmute}{\textsf{docmute}},
\href{http://ctan.org/pkg/includex}{\textsf{includex}} and
\href{http://ctan.org/pkg/standalone}{\textsf{standalone}}
provide commands to include only the document body of
a child file thus allowing both files to be compiled individually.
\item
The packages \href{http://ctan.org/pkg/subdocs}{\textsf{subdocs}}
and \href{http://ctan.org/pkg/subfiles}{\textsf{subfiles}}
provide structures in which the main and child documents can be
encapsulated and allowing them to be compiled individually.
The inclusion mechanism is different from the conventional |\include|.
\item
The package \href{http://ctan.org/pkg/combine}{\textsf{combine}}
is an elaborate solution to combine several documents into one.
\end{itemize}
%
See also the CTAN topic \href{http://ctan.org/topic/subdocs}{\textsf{subdocs}}
for further related packages.
The present package differs from the above solutions in that
a document structure constructed with the conventional |\include| mechanism
just needs two extra commands at the top of every file
such that all constituent files can be compiled individually.

%%%%%%%%%%%%%%%%%%%%%%%%%%%%%%%%%%%%%%%%%%%%%%%%%%%%%%%%%%%%%%%%%%%%%%%%%%%%%%%%
%\subsection{Feature Suggestions}
%
%The following is a list of features which may be useful for future
%versions of this package:
%%
%\begin{itemize}
%\item
%\ldots
%\end{itemize}

%%%%%%%%%%%%%%%%%%%%%%%%%%%%%%%%%%%%%%%%%%%%%%%%%%%%%%%%%%%%%%%%%%%%%%%%%%%%%%%%
\subsection{Revision History}

%%%%%%%%%%%%%%%%%%%%%%%%%%%%%%%%%%%%%%%%
\paragraph{v2.0:} 2018/12/30

\begin{itemize}
\item
immediate forward processing
\item
added |\childdocby| mechanism
\item
manual restructured
\end{itemize}

%%%%%%%%%%%%%%%%%%%%%%%%%%%%%%%%%%%%%%%%
\paragraph{v1.6:} 2018/01/17

\begin{itemize}
\item
application for development of include files
\item
corrections to manual
\end{itemize}

%%%%%%%%%%%%%%%%%%%%%%%%%%%%%%%%%%%%%%%%
\paragraph{v1.5:} 2017/05/21

\begin{itemize}
\item
more complete structuring introduced
\item
|\childdocof| introduced
\item
|\childdoc| renamed to |\childdocmain|
\item
|\childredirect| renamed to |\childdocforward| and |\childdocforwardprefix|
and functionality expanded
\end{itemize}

%%%%%%%%%%%%%%%%%%%%%%%%%%%%%%%%%%%%%%%%
\paragraph{v1.0:} 2017/04/27

\begin{itemize}
\item
manual and install package
\item
first version published on CTAN
\end{itemize}

%%%%%%%%%%%%%%%%%%%%%%%%%%%%%%%%%%%%%%%%
\paragraph{v0.6:} 2017/04/26

\begin{itemize}
\item
redirection mechanism added
\end{itemize}

%%%%%%%%%%%%%%%%%%%%%%%%%%%%%%%%%%%%%%%%
\paragraph{v0.5:} 2017/04/26

\begin{itemize}
\item
functionality in definition file
\end{itemize}


%%%%%%%%%%%%%%%%%%%%%%%%%%%%%%%%%%%%%%%%%%%%%%%%%%%%%%%%%%%%%%%%%%%%%%%%%%%%%%%%
%%%%%%%%%%%%%%%%%%%%%%%%%%%%%%%%%%%%%%%%%%%%%%%%%%%%%%%%%%%%%%%%%%%%%%%%%%%%%%%%
%%%%%%%%%%%%%%%%%%%%%%%%%%%%%%%%%%%%%%%%%%%%%%%%%%%%%%%%%%%%%%%%%%%%%%%%%%%%%%%%
\appendix

\settowidth\MacroIndent{\rmfamily\scriptsize 000\ }

 \DocInput{childdoc.dtx}

\end{document}
%</driver>
% \fi
%
% %%%%%%%%%%%%%%%%%%%%%%%%%%%%%%%%%%%%%%%%%%%%%%%%%%%%%%%%%%%%%%%%%%%%%%%%%%%%%%
% %%%%%%%%%%%%%%%%%%%%%%%%%%%%%%%%%%%%%%%%%%%%%%%%%%%%%%%%%%%%%%%%%%%%%%%%%%%%%%
% \section{Sample}
%\iffalse
%<*samplemain>
%\fi
%
% The following presents a sample document
% with two chapters, two parts, a title page,
% a compile flag as well as three forwarding files to set the flag.
% It consists of eight |.tex| files:
% \begin{center}
% \begin{tabular}{ll}
% |cdocsamp.tex|&main file\\
% |cdocsch1.tex|&include file for chapter 1\\
% |cdocsch2.tex|&include file for chapter 2\\
% |cdocspt3.tex|&include file for part 3\\
% |cdocspt4.tex|&include file for part 4\\
% |cdocsdrf.tex|&forwarding file for main file in draft mode\\
% |cdocsfi1.tex|&forwarding file for final version of chapter 1\\
% |cdocsfi2.tex|&forwarding file for final version of chapter 2\\
% \end{tabular}
% \end{center}
% Each of the eight files can be compiled directly by the \LaTeX{} compiler.
%
% %%%%%%%%%%%%%%%%%%%%%%%%%%%%%%%%%%%%%%
% \paragraph{Main File.}
%
% The main file is called |cdocsamp.tex|.
%
% Load the \textsf{childdoc} definitions and
% declare the filename for the main document:
%    \begin{macrocode}
\input{childdoc.def}
\childdocmain{}
%    \end{macrocode}

% Optional override for |\version| flag:
%    \begin{macrocode}
%%\ifchilddoc\else\providecommand{\version}{draft}\fi
%    \end{macrocode}

% Define the default values for the |\version| flag
% (|final| for the main file and |draft| for childs):
%    \begin{macrocode}
\ifchilddoc
\providecommand{\version}{draft}
\else
\providecommand{\version}{final}
\fi
%    \end{macrocode}

% Load the standard document class:
%    \begin{macrocode}
\documentclass[12pt]{article}
%    \end{macrocode}

% Start the document body:
%    \begin{macrocode}
\begin{document}
%    \end{macrocode}

% Declare a title page.
% Print title, part of document being processed and version flag:
%    \begin{macrocode}
\addtocounter{page}{-1}
\begin{center}
{\LARGE\bfseries{}childdoc example\par}
\vspace{1cm}
\ifchilddoc
\ifchilddocmanual part\else chapter\fi:
`\childdocname' of `\childdocjob'\par
\else
main document: `\childdocjob'\par
\fi
version: \version\par
\end{center}
\newpage
%    \end{macrocode}

% Manually include selected file,
% otherwise process as usual:
%    \begin{macrocode}
\ifchilddocmanual
\section*{part `\childdocname'}
\input{\childdocname}
\else
%    \end{macrocode}

% Include the two chapters:
%    \begin{macrocode}
\include{cdocsch1}
\include{cdocsch2}
%    \end{macrocode}

% Include the two parts unless only chapters should be displayed:
%    \begin{macrocode}
\ifchilddoc\else
\section{part three}
\input{cdocspt3}
\section{part four}
\input{cdocspt4}
\fi
%    \end{macrocode}

% Process as usual until here:
%    \begin{macrocode}
\fi
%    \end{macrocode}

% End of document body:
%    \begin{macrocode}
\end{document}
%    \end{macrocode}
%\iffalse
%</samplemain>
%\fi
%
% %%%%%%%%%%%%%%%%%%%%%%%%%%%%%%%%%%%%%%
% \paragraph{Chapter Include Files.}
%
% The include files are called |cdocsch1.tex| and |cdocsch2.tex|.
%
%\iffalse
%<*samplechap1|samplechap2>
%\fi

% Optional override for |\version| flag:
%    \begin{macrocode}
%%\providecommand{\version}{final}
%    \end{macrocode}

% Include the main document:
%    \begin{macrocode}
\input{childdoc.def}
\childdocof{cdocsamp}
%    \end{macrocode}

%\iffalse
%</samplechap1|samplechap2>
%\fi
%
%\iffalse
%<*samplechap1>
%\fi
% Some text for chapter 1:
%    \begin{macrocode}
\section{one}
some text in chapter one
%    \end{macrocode}

%\iffalse
%</samplechap1>
%\fi
% Some text for chapter 2:
%\iffalse
%<*samplechap2>
%\fi
%    \begin{macrocode}
\section{two}
more text in chapter two
%    \end{macrocode}

%\iffalse
%</samplechap2>
%\fi
%
% %%%%%%%%%%%%%%%%%%%%%%%%%%%%%%%%%%%%%%
% \paragraph{Part Include Files.}
%
% The include files are called |cdocspt3.tex| and |cdocspt4.tex|.
%
%\iffalse
%<*samplepart3|samplepart4>
%\fi

% Optional override for |\version| flag:
%    \begin{macrocode}
%%\providecommand{\version}{final}
%    \end{macrocode}

% Include the main document:
%    \begin{macrocode}
\input{childdoc.def}
\childdocby{cdocsamp}
%    \end{macrocode}

%\iffalse
%</samplepart3|samplepart4>
%\fi
%
%\iffalse
%<*samplepart3>
%\fi
% Some text for part 3:
%    \begin{macrocode}
some text in part three
%    \end{macrocode}

%\iffalse
%</samplepart3>
%\fi
% Some text for part 4:
%\iffalse
%<*samplepart4>
%\fi
%    \begin{macrocode}
more text in part four
%    \end{macrocode}

%\iffalse
%</samplepart4>
%\fi
%
% %%%%%%%%%%%%%%%%%%%%%%%%%%%%%%%%%%%%%%
% \paragraph{Forwarding for a Complete Draft.}
%
% The following forwarding file |cdocsdrf.tex|
% compiles the main document in draft mode:
%\iffalse
%<*sampledraft>
%\fi
%    \begin{macrocode}
\def\version{draft}
\input{childdoc.def}
\childdocforward{cdocsamp}
%    \end{macrocode}

%\iffalse
%</sampledraft>
%\fi
%
% %%%%%%%%%%%%%%%%%%%%%%%%%%%%%%%%%%%%%%
% \paragraph{Forwarding for Final Version of the Chapters.}
%
% The following forwarding files |cdocsfn1.tex| and |cdocsfn2.tex|
% (with identical content)
% compile the final versions of the child documents
% |cdocsch1.tex| and |cdocsch2.tex|, respectively:
%\iffalse
%<*samplefinal>
%\fi
%    \begin{macrocode}
\def\version{final}
\input{childdoc.def}
\childdocforwardprefix[cdocsamp]{cdocsfn}{cdocsch}
%    \end{macrocode}

%\iffalse
%</samplefinal>
%\fi
%
% %%%%%%%%%%%%%%%%%%%%%%%%%%%%%%%%%%%%%%
% \paragraph{Command Line Processing.}
%
% The following three command lines generate the output files
% |cdocscld|, |cdocscl1| and |cdocscl2|
% which should be identical to
% |cdocsdrf|, |cdocsch1| and |cdocsfn2|, respectively:
% \begin{center}
% \begin{tabular}{l}
% |latex -jobname cdocscld \|\\
% |  "\def\version{draft}\input{childdoc.def}\childdocforward{cdocsamp}"|\\
% |latex -jobname cdocscl1 \|\\
% |  "\input{childdoc.def}\childdocforward[cdocsamp]{cdocsch1}"|\\
% |latex -jobname cdocscl2 \|\\
% |  "\def\version{final}\input{childdoc.def}\childdocforward{cdocsch2}"|
% \end{tabular}
% \end{center}
% Note that the trailing backslash on each first line
% merely continues the input to the second line
% (for convenient cut ant paste).
% Furthermore, the command |latex| can be replaced by any
% of its alternative versions such as |pdflatex|.
%
% %%%%%%%%%%%%%%%%%%%%%%%%%%%%%%%%%%%%%%%%%%%%%%%%%%%%%%%%%%%%%%%%%%%%%%%%%%%%%%
% %%%%%%%%%%%%%%%%%%%%%%%%%%%%%%%%%%%%%%%%%%%%%%%%%%%%%%%%%%%%%%%%%%%%%%%%%%%%%%
% \section{Implementation}
%\iffalse
%<*package>
%\fi
%
% This section describes the definitions file |childdoc.def|.

% The definitions cannot be loaded using |\usepackage| or |\RequirePackage|
% which has a mechanism to prevent loading a style file more than once.
% When loading the definitions by means of |\input|
% multiple instances have to be prevented manually:
%\iffalse
%This code needs to be before the `\ProvidesFile' directive
%which is defined at the beginning of this file.
%Therefore it is also placed there and commented out here.
%</package>
%<*discard>
%\fi
%    \begin{macrocode}
\ifdefined\childdocmain\endinput\fi
%    \end{macrocode}
%\iffalse
%</discard>
%<*package>
%\fi
%
% \macro{\ifchilddoc}
% \macro{\ifchilddocmanual}
% The conditional |\ifchilddoc| tells whether a
% child (true) or main (false) document is being compiled.
% The conditional |\ifchilddocmanual| tells whether
% the |\includeonly| mechanism is used (false) or
% the selection of child files must be performed manually (true).
% The definitions initialise to false:
%    \begin{macrocode}
\newif\ifchilddoc
\newif\ifchilddocmanual
%    \end{macrocode}

% \macro{\childdocname}
% \macro{\childdocjob}
% The macro |\childdocname| stores the name of the main document
% to be compiled. The macro |\childdocjob| stores the name of
% the document on which the \LaTeX{} compiler was originally invoked.
% The content of |\jobname| cannot be compared
% to filenames specified in the source due to different catcodes.
% The following code rescans |\jobname|, stores the result
% in |\childdocname| and saves a copy in |\childdocjob|:
%    \begin{macrocode}
\edef\childdocname{\scantokens\expandafter{\jobname\noexpand}}
\let\childdocjob\childdocname
%    \end{macrocode}

% \macro{\childdocdisable}
% The macro |\childdocdisable| prevents the main file
% from being processed more than once.
% At this stage, the main document command |\childdocmain|
% is assumed to be called once again where it should do nothing.
% Any subsequent call to it should prevent
% a secondary processing of the main document
% It overwrites the forwarding commands
% |\childdocof| and |\childdocforward|
% with empty macros to prevent further inclusions of the main document:
%    \begin{macrocode}
\newcommand{\childdocdisable}
{
  \renewcommand{\childdocmain}[1]{\renewcommand{\childdocmain}[1]{\endinput}}
  \renewcommand{\childdocof}[1]{}
  \renewcommand{\childdocby}[2][]{}
  \renewcommand{\childdocforward}[2][]{}
  \renewcommand{\childdocdisable}{}
}
%    \end{macrocode}

% \macro{\childdocmain}
% The macro |\childdocmain| is to be called at the top of the main file
% with nothing or the main filename (without extension) as argument.
% First, it breaks loops.
% If the argument is not empty and does not match |\childdocname|
% (which is set by the first inclusion of |childdoc.def|),
% |\ifchilddoc| is set to true, |\includeonly| is applied to the child file
% and |\jobname| is set to the main file
% (for proper handling of |.aux| files):
%    \begin{macrocode}
\newcommand{\childdocmain}[1]
{
  \childdocdisable\childdocmain{}
  \if?#1?\else
    \begingroup
      \def\childdoctmp{#1}
      \ifx\childdoctmp\childdocname
        \def\childdoctmp{}
      \else
        \def\childdoctmp
        {
          \childdoctrue
          \includeonly{\childdocname}
          \def\childdocjob{#1}
          \def\jobname{#1}
        }
      \fi
      \expandafter
    \endgroup
    \childdoctmp
  \fi
}
%    \end{macrocode}

% \macro{\childdocof}
% The command |\childdocof| redirects
% compilation to the main file |#1|.
%    \begin{macrocode}
\newcommand{\childdocof}[1]
{
  \childdocdisable
  \childdoctrue
  \includeonly{\childdocname}
  \def\jobname{#1}
  \def\childdocjob{#1}
  \input{#1}
}
%    \end{macrocode}

% \macro{\childdocby}
% The command |\childdocby| ....
%    \begin{macrocode}
\newcommand{\childdocby}[2][]
{
  \childdocdisable
  \childdoctrue
  \childdocmanualtrue
  \if?#1?\else
    \def\jobname{#2}
  \fi
  \def\childdocjob{#2}
  \input{#2}
  \endinput
}
%    \end{macrocode}

% \macro{\childdocforward}
% The command |\childdocforward| redirects
% compilation to the main file or
% (if the optional argument is given) a child file.
% Parameters are set as if the main file
% or a child file starting with |\childdocof| was compiled.
% Then compilation is handed over to the main file:
%    \begin{macrocode}
\newcommand{\childdocforward}[2][]
{
  \begingroup
    \if?#1?
      \def\childdoctmp
      {
        \def\childdocname{#2}
        \def\childdocjob{#2}
        \def\jobname{#2}
        \input{#2}
        \endinput
      }
    \else
      \def\childdoctmp
      {
        \childdocdisable
        \def\childdocname{#2}
        \childdoctrue
        \includeonly{#2}
        \def\childdocjob{#1}
        \def\jobname{#1}
        \input{#1}
        \endinput
      }
    \fi
    \expandafter
  \endgroup
  \childdoctmp
}
%    \end{macrocode}

% \macro{\childdocforwardprefix}
% The command |\childdocforwardprefix| redirects
% compilation to the main or a child file by means of a pattern.
% The prefix |#1| in the current filename is replaced by |#2|
% and the suffix of the current filename is kept
% (it is assumed that the filename does not contain the substring `|~~~|'
% which is used as a delimiter).
% Compilation is handed over to the new file by |\childdocforward|:
%    \begin{macrocode}
\newcommand{\childdocforwardprefix}[3][]
{
  \begingroup
    \def\childdocextract #2##1~~~{\def\childdoctmp{\childdocforward[#1]{#3##1}}}
    \expandafter\childdocextract\childdocname~~~
    \expandafter
  \endgroup
  \childdoctmp
}
%    \end{macrocode}

% \macro{\childdoc}
% The deprecated macro |\childdoc| is a legacy version of |\childdocmain|:
%    \begin{macrocode}
\newcommand{\childdoc}{\childdocmain}
%    \end{macrocode}

% \macro{\childdocredirect}
% The deprecated macro |\childdocredirect| is a legacy version
% of |\childdocforward| and |\childdocforwardprefix|:
%    \begin{macrocode}
\newcommand{\childdocredirect}[2][]
{
  \begingroup
    \if?#1?
      \def\childdoctmp{\childdocforward{#2}}
    \else
      \def\childdoctmp{\childdocforwardprefix{#1}{#2}}
    \fi
    \expandafter
  \endgroup
  \childdoctmp
}
%    \end{macrocode}

%\iffalse
%</package>
%\fi
%
\endinput
|\\
|\childdocby{|\textit{main}|}|\\
\end{tabular}
\end{center}
%
Both forms have slightly different effects as described above.
The main file is prepared as usual, see \secref{sec:include}.

%%%%%%%%%%%%%%%%%%%%%%%%%%%%%%%%%%%%%%%%%%%%%%%%%%%%%%%%%%%%%%%%%%%%%%%%%%%%%%%%
\subsection{Legacy Detection}
\label{sec:detection}

The directive |\childdocmain| in the main file can detect
whether the complete document or merely a child is to be compiled
even without using the directive |\childdocof|.
This method is deprecated because it is less robust
and there is no compelling reason to use it;
it is merely provided for backward compatibility
and it may be removed in future versions.

If the detection mechanism is to be used,
it is mandatory to correctly specify
the filename of the main file as the argument of |\childdocmain|:
%
\begin{center}
\begin{tabular}{l}
|% \iffalse
%
% childdoc.dtx Copyright (C) 2017-2018 Niklas Beisert
%
% This work may be distributed and/or modified under the
% conditions of the LaTeX Project Public License, either version 1.3
% of this license or (at your option) any later version.
% The latest version of this license is in
%   http://www.latex-project.org/lppl.txt
% and version 1.3 or later is part of all distributions of LaTeX
% version 2005/12/01 or later.
%
% This work has the LPPL maintenance status `maintained'.
%
% The Current Maintainer of this work is Niklas Beisert.
%
% This work consists of the files childdoc.dtx and childdoc.ins
% and the derived files childdoc.def and cdocsamp.tex with
% cdocsch1.tex, cdocsch2.tex, cdocsdrf.tex, cdocsfn1.tex, cdocsfn2.tex.
%
%<package>\ifdefined\childdocmain\endinput\fi
%<package>\ProvidesFile{childdoc.def}[2018/12/30 v2.0 child document driver]
%<samplemain>\ProvidesFile{cdocsamp.tex}[2018/12/30 v2.0 sample for childdoc]
%<*driver>
%\ProvidesFile{childdoc.drv}[2018/12/30 v2.0 childdoc reference manual file]
\PassOptionsToClass{10pt,a4paper}{article}
\documentclass{ltxdoc}

\usepackage[margin=35mm]{geometry}
\usepackage{hyperref}
\usepackage{hyperxmp}
\usepackage[usenames]{color}

\hypersetup{colorlinks=true}
\hypersetup{pdfstartview=FitH}
\hypersetup{pdfpagemode=UseNone}
\hypersetup{pdfsource={}}
\hypersetup{pdflang={en-UK}}
\hypersetup{pdfcopyright={Copyright 2017-2018 Niklas Beisert.
  This work may be distributed and/or modified under the
  conditions of the LaTeX Project Public License, either version 1.3
  of this license or (at your option) any later version.}}
\hypersetup{pdflicenseurl={http://www.latex-project.org/lppl.txt}}
\hypersetup{pdfcontactaddress={ETH Zurich, ITP, HIT K,
  Wolfgang-Pauli-Strasse 27}}
\hypersetup{pdfcontactpostcode={8093}}
\hypersetup{pdfcontactcity={Zurich}}
\hypersetup{pdfcontactcountry={Switzerland}}
\hypersetup{pdfcontactemail={nbeisert@itp.phys.ethz.ch}}
\hypersetup{pdfcontacturl={http://people.phys.ethz.ch/\xmptilde nbeisert/}}

\newcommand{\secref}[1]{\hyperref[#1]{section \ref*{#1}}}

\parskip1ex
\parindent0pt
\let\olditemize\itemize
\def\itemize{\olditemize\parskip0pt}

\begin{document}

\title{The \textsf{childdoc} Package}
\hypersetup{pdftitle={The childdoc Package}}
\author{Niklas Beisert\\[2ex]
  Institut f\"ur Theoretische Physik\\
  Eidgen\"ossische Technische Hochschule Z\"urich\\
  Wolfgang-Pauli-Strasse 27, 8093 Z\"urich, Switzerland\\[1ex]
  \href{mailto:nbeisert@itp.phys.ethz.ch}
  {\texttt{nbeisert@itp.phys.ethz.ch}}}
\hypersetup{pdfauthor={Niklas Beisert}}
\hypersetup{pdfsubject={Manual for the LaTeX2e Package childdoc}}
\date{30 December 2018, \textsf{v2.0}}
\maketitle

\begin{abstract}\noindent
\textsf{childdoc} is a \LaTeXe{} package
that enables the direct compilation
of document sections included by |\include|
to individual files.
\end{abstract}

\begingroup
\parskip0ex
\tableofcontents
\endgroup

%%%%%%%%%%%%%%%%%%%%%%%%%%%%%%%%%%%%%%%%%%%%%%%%%%%%%%%%%%%%%%%%%%%%%%%%%%%%%%%%
%%%%%%%%%%%%%%%%%%%%%%%%%%%%%%%%%%%%%%%%%%%%%%%%%%%%%%%%%%%%%%%%%%%%%%%%%%%%%%%%
\section{Introduction}

\LaTeX{} provides a mechanism to structure a large document (such as a book)
into a main file and several child files (containing the chapters)
using the |\include| command.
This mechanism is beneficial for documents
which span hundreds of pages in order to
make the source file(s) more manageable.
Moreover, compilation can be restricted to
selected child files by means of the |\includeonly| command.
The latter feature can be used to reduce the compilation time while editing
(this was significantly more useful in the earlier days of \LaTeX{})
or to generate a smaller document which is easier to navigate.
Another application of |\includeonly| is to generate
documents consisting of selected parts of the complete document.

However, there are a few drawbacks of the plain |\include| mechanism:
\begin{itemize}
\item
The child files cannot be compiled on their own,
they can only be compiled via the main file.
A naive editing environment
(such as a text editor with an option
to have the current file processed by \LaTeX)
may require one to switch to the main file before compiling;
attempting to compile the child file produces errors.
\item
The main file must be modified (each time)
to adjust the |\includeonly| command
to the present needs. This easily leaves the main file in a messy state.
\item
The generated document will always carry the filename
of the main document. This is inconvenient if
several child files are to be compiled and
to be kept for distribution.
\end{itemize}

The present package provides a simple interface
to make child files individually compilable by \LaTeX{}.
Compiling a child file then has the same effect as compiling
the main file with an |\includeonly| command
to select the appropriate child.
Moreover the generated document will carry the name of the child
rather than the main file.
This resolves all three above issues.

This feature is meant to make the editing of books,
thesis documents and lecture notes somewhat more convenient.
However, the package can also be used efficiently for
composing a series of documents (such as exercise sheets)
which are typically distributed individually.
It then assists the author in generating the individual documents
(potentially in different versions)
as well as a document containing the collected series.
Another application is in developing style files
or other kinds of included material
where compilation of the style file could redirect
to a sample or test file.

%%%%%%%%%%%%%%%%%%%%%%%%%%%%%%%%%%%%%%%%%%%%%%%%%%%%%%%%%%%%%%%%%%%%%%%%%%%%%%%%
%%%%%%%%%%%%%%%%%%%%%%%%%%%%%%%%%%%%%%%%%%%%%%%%%%%%%%%%%%%%%%%%%%%%%%%%%%%%%%%%
\section{Usage}

First of all, the package \textsf{childdoc} is \emph{not} a standard
\LaTeXe{} |.sty| style file! Therefore it needs to be invoked in
a non-standard way.

%%%%%%%%%%%%%%%%%%%%%%%%%%%%%%%%%%%%%%%%%%%%%%%%%%%%%%%%%%%%%%%%%%%%%%%%%%%%%%%%
\subsection{Included Files}
\label{sec:include}

%%%%%%%%%%%%%%%%%%%%%%%%%%%%%%%%%%%%%%%%
\DescribeMacro{\childdocmain}
To use the package, add the commands
\begin{center}
\begin{tabular}{l}
|\input{childdoc.def}|\\
|\childdocmain{}|\\
\end{tabular}
\end{center}
at the very top of the main \LaTeX{} file,
in particular \emph{before} the |\documentclass| statement!
The argument of |\childdocmain| should be left empty
(but it must be present).

%%%%%%%%%%%%%%%%%%%%%%%%%%%%%%%%%%%%%%%%
\DescribeMacro{\childdocof}
Furthermore, add the commands
\begin{center}
\begin{tabular}{l}
|\input{childdoc.def}|\\
|\childdocof{|\textit{main}|}|\\
\end{tabular}
\end{center}
at the top of every child file \textit{child}
which is included by |\include{|\textit{child}|}|
from within the main file
(or at least for those files to be compiled individually).
The argument \textit{main} must be the filename of the main file.

There are a couple of
considerations in setting up the main and child documents:

%%%%%%%%%%%%%%%%%%%%%%%%%%%%%%%%%%%%%%%%
\paragraph{Restrictions.}

Please note the following restrictions:
\begin{itemize}
\item
|\childdocmain| must be called with one argument \textit{main}
to ensure compatibility with earlier version of the package.
It must either be empty (|\childdocmain{}|)
or precisely match the filename of the main file in which it is specified.
See \secref{sec:detection} for further information.
\item
The filename \textit{main} must be specified without the |.tex| extension.
\item
The filename \textit{main} is case sensitive
(even in case-insensitive file systems)
due to internal string comparison.
\item
The argument \textit{main} should be fully expanded, it cannot be a macro.
\item
Subdirectories and special characters should be avoided in filenames.
\item
The command |\childdocmain{|\textit{main}|}| must be followed by a whitespace.
It should not be followed immediately by another command
or by a comment mark `|%|'.
This is because the \TeX{} parser reads the token immediately following
the argument of |\childdocmain| and puts it
at the beginning of every child section;
however, a white\-space is ignored.
\end{itemize}

%%%%%%%%%%%%%%%%%%%%%%%%%%%%%%%%%%%%%%%%
\paragraph{Content of Main File.}

It is advisable to place all content in the child files included by |\include|.
Any output contained in the main file will appear in all child documents
unless suppressed manually;
it cannot be suppressed automatically by the |\includeonly| directive
and thus should normally be avoided.
A method to include some content in the main file
by means of conditional processing is described in \secref{sec:conditional}.

%%%%%%%%%%%%%%%%%%%%%%%%%%%%%%%%%%%%%%%%
\paragraph{Page Numbering.}

When only a part of the document is compiled,
the appropriate numbering of pages
(as well as other status parameters)
is determined from the |.aux| files.
The latter contain information from previous passes.
However this information needs to propagate through
all intermediate child documents.
Therefore the page numbering in child documents may well
be inconsistent until the complete document is compiled at least once.

A useful (if unconventional) way to always ensure a consistent
page numbering is to restart the numbering in each child document
and denote the pages by `\textit{child}|.|\textit{page}'
where \textit{child} represents the chapter/section number of the child file.
This can be achieved by the command
|\numberwithin{page}{|\textit{child}|}|
of the \textsf{amsmath} package
where \textit{child} can be |chapter| or |section|
depending on the chosen structuring.
Alternatively, one can modify the macro |\thepage| appropriately
and reset the counter |page| at the start of each child file.

%%%%%%%%%%%%%%%%%%%%%%%%%%%%%%%%%%%%%%%%%%%%%%%%%%%%%%%%%%%%%%%%%%%%%%%%%%%%%%%%
\subsection{Conditional Processing}
\label{sec:conditional}

The package provides a mechanism to compile different versions
of a document. To customise the versions further some conditional processing
can come in handy to distinguish which version is being compiled.
The package provides two macros to describe the compilation context:

%%%%%%%%%%%%%%%%%%%%%%%%%%%%%%%%%%%%%%%%
\DescribeMacro{\ifchilddoc}
The conditional |\ifchilddoc| distinguishes between the compilation of
child documents and the main document:
%
\begin{center}
|\ifchilddoc |\textit{child-code}| |[|\||else |\textit{main-code}]| \||fi|
\end{center}

%%%%%%%%%%%%%%%%%%%%%%%%%%%%%%%%%%%%%%%%
\DescribeMacro{\childdocname}
\DescribeMacro{\childdocjob}
The macro |\childdocname| contains the filename (without extension)
of the main or child file being processed.
Note that |\childdocjob| will always contain the name of the main file.

%%%%%%%%%%%%%%%%%%%%%%%%%%%%%%%%%%%%%%%%
\paragraph{Title Page.}

Conditional processing can be used to include a title or banner page
in the main document when proper precautions are taken.
Importantly, the code in the main file should ensure that the page counter
(as well as other status parameters which are stored in the |.aux| files)
takes the same value after the conditional processing.
Otherwise the page numbers may take divergent values
depending on which part is compiled.

For example, a title page could be declared by:
%
\begin{center}
\begin{tabular}{l}
|\ifchilddoc\||else|\\
|\addtocounter{page}{-1}|\\
\textit{code for title page}\\
|\newpage|\\
|\||fi|
\end{tabular}
\end{center}
%
A banner page for the child documents can be generated by:
%
\begin{center}
\begin{tabular}{l}
|\ifchilddoc|\\
|\addtocounter{page}{-1}|\\
\textit{code for banner page}\\
|\newpage|\\
|\||fi|
\end{tabular}
\end{center}
%
Here one could write a message such as:
\begin{center}
|This is the part \childdocname{} of \childdocjob{}.|
\end{center}

%%%%%%%%%%%%%%%%%%%%%%%%%%%%%%%%%%%%%%%%%%%%%%%%%%%%%%%%%%%%%%%%%%%%%%%%%%%%%%%%
\subsection{Flags}
\label{sec:flags}

The package makes it easy to generate different versions
of the main or child documents.
To this end compilation flags can be defined
and assigned different default values.
They will be particularly useful in conjunction
with the forwarding mechanism described in \secref{sec:forward}.

For example, it may be useful to have a flag |\version|
which can be set to |draft| or |final|.
The document source will contain some conditional code
depending on the value of |\version|.
Suppose further, the flag should default to |final| for the main file
and to |draft| for child files
which is a natural assignment for editing the document.
This is achieved by placing the following code
in the preamble of the main document
(below the |\childdocmain| directive):
%
\begin{center}
\begin{tabular}{l}
|\ifchilddoc|\\
|\providecommand{\version}{draft}|\\
|\||else|\\
|\providecommand{\version}{final}|\\
|\||fi|
\end{tabular}
\end{center}
%
The definition by |\providecommand| makes sure
that previous definitions are not overwritten.
Further statements |\providecommand{\version}{...}|
can thus be added before the above code to override it.

For the main file, one might add a line
(between |\childdocmain| and the above block)
%
\begin{center}
|%\ifchilddoc\||else\providecommand{\version}{draft}\||fi|
\end{center}
%
which can be uncommented to produce a draft version.
Likewise one can add a line to the very top of a child file
(above the |\childdocof{|\textit{main}|}| directive)
%
\begin{center}
|%\providecommand{\version}{final}|
\end{center}
%
which can be uncommented to produce the final version of this child document.

%%%%%%%%%%%%%%%%%%%%%%%%%%%%%%%%%%%%%%%%%%%%%%%%%%%%%%%%%%%%%%%%%%%%%%%%%%%%%%%%
\subsection{Forwarding}
\label{sec:forward}

Different versions of the main or child documents
using compilation flags as described in \secref{sec:flags}
can be (permanently) stored in different files
for convenient compilation, viewing and distribution.
To this end, the package defines a command
to pass on compilation to a different file:

%%%%%%%%%%%%%%%%%%%%%%%%%%%%%%%%%%%%%%%%
\DescribeMacro{\childdocforward}
The command |\childdocforward| redirects processing to
another source file:
%
\begin{center}
\begin{tabular}{l}
|\input{childdoc.def}|\\
|\childdocforward[|\textit{main}|]{|\textit{dest}|}|\\
\end{tabular}
\end{center}
%
The argument \textit{dest} is the destination file
(without extension).
It should be the main file or one of the child files.
Note that further \textsf{childdoc} directives
such as |\childdocof| and |\childdocforward|
in the indicated file will be processed in this form.
The optional argument \textit{main}
passes on directly to the main file \textit{main}
while pretending to compile the child \textit{dest}.
This form behaves as if \textit{dest}
issues |\childdocof{|\textit{main}|}| right away,
and no further \textsf{childdoc} directives will be processed.

%%%%%%%%%%%%%%%%%%%%%%%%%%%%%%%%%%%%%%%%
\DescribeMacro{\...prefix}
In the alternative form |\childdocforwardprefix|,
%
\begin{center}
\begin{tabular}{l}
|\input{childdoc.def}|\\
|\childdocforwardprefix[|\textit{main}|]{|\textit{prefix}|}{|\textit{dest}|}|
\end{tabular}
\end{center}
%
the destination file is determined by a pattern
depending on the current file:
To make this work, the current file must be called
`{\textit{prefix}\hspace{0.2em}\textit{suffix}}'
with \textit{prefix} matching precisely the argument.
Processing is then passed on to the file
`{\textit{dest}\hspace{0.2em}\textit{suffix}}'.
Surely, the same effect is achieved by
directly specifying the
argument `{\textit{dest}\hspace{0.2em}\textit{suffix}}'
in the first form.
However, that requires to set up a different file
for each child. With the alternative form of the command
all these files can have exactly the same content
which simplifies setting them up and maintaining them.

For example, the following file |draft.tex|
with a compilation flag |\version| as described in \secref{sec:flags}
compiles the main document as a draft:
%
\begin{center}
\begin{tabular}{l}
|\def\version{draft}|\\
|\input{childdoc.def}|\\
|\childdocforward{|\textit{main}|}|
\end{tabular}
\end{center}
%
Likewise, the following files |final|\textit{nn}|.tex|
compile the final version of the child document
|child|\textit{nn}|.tex|:
%
\begin{center}
\begin{tabular}{l}
|\def\version{final}|\\
|\input{childdoc.def}|\\
|\childdocforwardprefix{final}{child}|
\end{tabular}
\end{center}
%

Note that when several versions of a main file and/or of each child file
are to be generated, it may be convenient to set up a |Makefile| or
shell script to automatise the process.

%%%%%%%%%%%%%%%%%%%%%%%%%%%%%%%%%%%%%%%%%%%%%%%%%%%%%%%%%%%%%%%%%%%%%%%%%%%%%%%%
\subsection{Command Line Processing}
\label{sec:commandline}

The effect of redirection files can also be achieved by invoking
the \LaTeX{} compiler with a more elaborate command line.
Most conveniently this should be done as part
of a shell script or a |Makefile|.

When using \textsf{childdoc} in the main file, the following
command lines effectively perform a redirection
(note that depending on the shell being used,
backslashes may have to be doubled: `|\|' $\to$ `|\\|'):
%
\begin{center}
|... -jobname "|\textit{target}|" |\\|"|[\textit{flags}]%
|\input{childdoc.def}\childdocforward[|\textit{main}|]{|\textit{dest}|}"|
\end{center}
%
Here \textit{target} is the name of the output file,
\textit{main} is the name of the main file
and \textit{dest} is the name of the main or child file to be processed
(all filenames without extensions).
The optional argument \textit{main} can be omitted
if \textit{main} matches \textit{dest}.
Optionally, compilation \textit{flags} can be defined via |\def| commands.
This command line makes the \TeX{} engine believe
it is compiling the file \textit{target}
whose content is specified as the latter parameter.
The provided code then forwards the processing to
\textit{main} or \textit{dest} as described in \secref{sec:forward}.

%%%%%%%%%%%%%%%%%%%%%%%%%%%%%%%%%%%%%%%%%%%%%%%%%%%%%%%%%%%%%%%%%%%%%%%%%%%%%%%%
\subsection{Include by Input}
\label{sec:input}

Including child documents by |\include| has some restrictions by design.
Most notably, the content of a child document always occupies
its own set of pages; pages cannot be shared between child documents.
Usually, this behaviour makes perfect sense
because each child document contain an essential part of the document.
However, in some situations it may be desirable to compose
a document from a collection of parts
without having mandatory page breaks between then.
For this case, the package
provides a mechanism to include parts
by |\input| which can also be processed individually.
However, by construction this mechanism
requires manual handling of the content to be output.

%%%%%%%%%%%%%%%%%%%%%%%%%%%%%%%%%%%%%%%%
\DescribeMacro{\ifchilddocmanual}
The main file should be prepared as usual, see \secref{sec:include}.
However, the document body must make a distinction
between processing of an individual part and of the main document, e.g.:
%
\begin{center}
\begin{tabular}{l}
|\ifchilddocmanual|\\
|\input{\childdocname}|\\
|\||else|\\
\textit{document body with }|\input{|\textit{part}|}|\\
|\||fi|
\end{tabular}
\end{center}
%
The conditional |\ifchilddocmanual| is true whenever
a part to be included by |\input| is being compiled,
and the name of the part is stored in |\childdocname|.

%%%%%%%%%%%%%%%%%%%%%%%%%%%%%%%%%%%%%%%%
\DescribeMacro{\childdocby}
Each part to be included by |\input| should start with:
%
\begin{center}
\begin{tabular}{l}
|\input{childdoc.def}|\\
|\childdocby{|\textit{main}|}|\\
\end{tabular}
\end{center}
%
The directive |\childdocby| is similar to |\childdocof|
described in \secref{sec:include},
but the subsequent selection of content must be done manually.
To that end, both |\ifchilddoc| and |\ifchilddocmanual|
will be true upon processing of a part,
and the name of the part is stored in |\childdocname|.
Note that |\jobname| will be set to the filename of the current part
so that each part receives an individual |.aux| file
that does not interfere with the |.aux| file(s) of the main document.
This behaviour can be altered by the alternative form
|\childdocby[*]{|\textit{main}|}| (with a non-empty optional argument)
which uses the |.aux| file of the main document
by setting |\jobname| to \textit{main}.

%%%%%%%%%%%%%%%%%%%%%%%%%%%%%%%%%%%%%%%%%%%%%%%%%%%%%%%%%%%%%%%%%%%%%%%%%%%%%%%%
\subsection{Driver Development}
\label{sec:driver}

The \textsf{childdoc} mechanism can also be use for the development
of definition files such as \LaTeX{} styles or classes.
This case differs from the above setup with multiple parts
included by |\include| in that no |\includeonly| should be invoked.
This can be achieved by starting the include file
(before |\ProvidesPackage|) with:
%
\begin{center}
\begin{tabular}{l}
|\input{childdoc.def}|\\
|\childdocforward{|\textit{main}|}|\\
\end{tabular}
\end{center}
%
or alternatively with:
%
\begin{center}
\begin{tabular}{l}
|\input{childdoc.def}|\\
|\childdocby{|\textit{main}|}|\\
\end{tabular}
\end{center}
%
Both forms have slightly different effects as described above.
The main file is prepared as usual, see \secref{sec:include}.

%%%%%%%%%%%%%%%%%%%%%%%%%%%%%%%%%%%%%%%%%%%%%%%%%%%%%%%%%%%%%%%%%%%%%%%%%%%%%%%%
\subsection{Legacy Detection}
\label{sec:detection}

The directive |\childdocmain| in the main file can detect
whether the complete document or merely a child is to be compiled
even without using the directive |\childdocof|.
This method is deprecated because it is less robust
and there is no compelling reason to use it;
it is merely provided for backward compatibility
and it may be removed in future versions.

If the detection mechanism is to be used,
it is mandatory to correctly specify
the filename of the main file as the argument of |\childdocmain|:
%
\begin{center}
\begin{tabular}{l}
|\input{childdoc.def}|\\
|\childdocmain{|\textit{main}|}|\\
\end{tabular}
\end{center}
%
If |\jobname| does not match the argument \textit{main} of |\childdocmain|,
it is assumed that |\jobname| points to the child file to be compiled.
When using |\childdocmain| with the main file specified as argument,
it suffices to start a child file
with just |\input{|\textit{main}|}|
without loading of the package and using |\childdocof|.
If instead all processing is done
with the appropriate \textsf{childdoc} directives,
the argument of \textit{main} of |\childdocmain| can be empty.

An alternative version of the command line processing described
in \secref{sec:commandline} using the detection mechanism reads:
%
\begin{center}
|... -jobname "|\textit{target}|" "|[\textit{flags}]%
[|\def\jobname{|\textit{dest}|}|]|\input{|\textit{main}|}"|
\end{center}

%%%%%%%%%%%%%%%%%%%%%%%%%%%%%%%%%%%%%%%%%%%%%%%%%%%%%%%%%%%%%%%%%%%%%%%%%%%%%%%%
\subsection{Manual Code}
\label{sec:manual}

In case one cannot be certain whether the definitions file |childdoc.def|
is installed on the target \TeX{} distribution
and one prefers not to ship it,
it is conceivable to paste a few relevant commands into the sources.

To that end, drop all statements |\input{childdoc.def}|
and perform the replacements as outlined below.
Instead of |\childdocmain{|\textit{main}|}| add the following code
to the top of the main file:
%
\begin{center}
\begin{tabular}{l}
|\||ifdefined\childdocname\endinput\||fi\newif\ifchilddoc|\\
|\edef\childdocname{\scantokens\expandafter{\jobname\noexpand}}|\\
|\def\childdocmain{|\textit{main}|}\||ifx\childdocmain\childdocname\||else|\\
|\childdoctrue\includeonly{\childdocname}\let\jobname\childdocmain\||fi|\\
\end{tabular}
\end{center}
%
Instead of |\childdocof{|\textit{main}|}| just include the main file
at the top of each child file:
%
\begin{center}
|\input{|\textit{main}|}|
\end{center}
%
A simple redirection |\childdocforward{|\textit{dest}|}| is achieved by:
%
\begin{center}
|\def\jobname{|\textit{dest}|}\input{\jobname}|
\end{center}
%
The redirection with prefix
|\childdocforwardprefix[|\textit{prefix}|]{|\textit{dest}|}|
is accomplished by:
%
\begin{center}
\begin{tabular}{l}
|{\edef\jobname{\scantokens\expandafter{\jobname\noexpand}}|\\
|\def\redirectjob |\textit{prefix}|#1~~~{\gdef\jobname{|\textit{dest}|#1}}|\\
|\expandafter\redirectjob\jobname~~~}\input{\jobname}|
\end{tabular}
\end{center}

In an alternative approach,
child documents can be compiled by a specific command line
without additional code or specific definitions:
%
\begin{center}
|... -jobname "|\textit{target}|" "|[\textit{flags}]%
|\includeonly{|\textit{dest}|}\input{|\textit{main}|}"|
\end{center}
%

%%%%%%%%%%%%%%%%%%%%%%%%%%%%%%%%%%%%%%%%%%%%%%%%%%%%%%%%%%%%%%%%%%%%%%%%%%%%%%%%
%%%%%%%%%%%%%%%%%%%%%%%%%%%%%%%%%%%%%%%%%%%%%%%%%%%%%%%%%%%%%%%%%%%%%%%%%%%%%%%%
\section{Information}

%%%%%%%%%%%%%%%%%%%%%%%%%%%%%%%%%%%%%%%%%%%%%%%%%%%%%%%%%%%%%%%%%%%%%%%%%%%%%%%%
\subsection{Copyright}

Copyright \copyright{} 2017--2018 Niklas Beisert

This work may be distributed and/or modified under the
conditions of the \LaTeX{} Project Public License, either version 1.3
of this license or (at your option) any later version.
The latest version of this license is in
  \url{http://www.latex-project.org/lppl.txt}
and version 1.3 or later is part of all distributions of \LaTeX{}
version 2005/12/01 or later.

This work has the LPPL maintenance status `maintained'.

The Current Maintainer of this work is Niklas Beisert.

This work consists of the files |README.txt|, |childdoc.ins| and |childdoc.dtx|
as well as the derived files |childdoc.def|, |cdocsamp.tex|
with |cdocsch1.tex|, |cdocsch2.tex|, |cdocspt3.tex|, |cdocspt4.tex|,
|cdocsdrf.tex|, |cdocsfn1.tex|, |cdocsfn2.tex|
as well as |childdoc.pdf|.

%%%%%%%%%%%%%%%%%%%%%%%%%%%%%%%%%%%%%%%%%%%%%%%%%%%%%%%%%%%%%%%%%%%%%%%%%%%%%%%%
\subsection{Files and Installation}

The package consists of the files:
%
\begin{center}
\begin{tabular}{ll}
    |README.txt|   & readme file \\
    |childdoc.ins| & installation file \\
    |childdoc.dtx| & source file \\
    |childdoc.def| & definition file \\
    |cdocsamp.tex| & sample main file \\
    |cdocsch1.tex| & sample include file \\
    |cdocsch2.tex| & sample include file \\
    |cdocspt3.tex| & sample part file \\
    |cdocspt4.tex| & sample part file \\
    |cdocsdrf.tex| & sample redirection file \\
    |cdocsfn1.tex| & sample redirection file \\
    |cdocsfn2.tex| & sample redirection file \\
    |childdoc.pdf| & manual
\end{tabular}
\end{center}
%
The distribution consists of the files
|README.txt|, |childdoc.ins| and |childdoc.dtx|.
%
\begin{itemize}
\item
Run (pdf)\LaTeX{} on |childdoc.dtx|
to compile the manual |childdoc.pdf| (this file).
\item
Run \LaTeX{} on |childdoc.ins| to create the definitions file |childdoc.def|
and the sample |cdocsamp.tex| with include files
|cdocsch1.tex|, |cdocsch2.tex|, |cdocspt3.tex|, |cdocspt4.tex|,
|cdocsdrf.tex|, |cdocsfn1.tex|, |cdocsfn2.tex|.
Then copy the file |childdoc.def| to an appropriate directory of your \LaTeX{}
distribution, e.g.\ \textit{texmf-root}|/tex/latex/childdoc|.
\end{itemize}

%%%%%%%%%%%%%%%%%%%%%%%%%%%%%%%%%%%%%%%%%%%%%%%%%%%%%%%%%%%%%%%%%%%%%%%%%%%%%%%%
\subsection{Related CTAN Packages}

There are several other packages which offer a similar functionality:
%
\begin{itemize}
\item
The packages
\href{http://ctan.org/pkg/docmute}{\textsf{docmute}},
\href{http://ctan.org/pkg/includex}{\textsf{includex}} and
\href{http://ctan.org/pkg/standalone}{\textsf{standalone}}
provide commands to include only the document body of
a child file thus allowing both files to be compiled individually.
\item
The packages \href{http://ctan.org/pkg/subdocs}{\textsf{subdocs}}
and \href{http://ctan.org/pkg/subfiles}{\textsf{subfiles}}
provide structures in which the main and child documents can be
encapsulated and allowing them to be compiled individually.
The inclusion mechanism is different from the conventional |\include|.
\item
The package \href{http://ctan.org/pkg/combine}{\textsf{combine}}
is an elaborate solution to combine several documents into one.
\end{itemize}
%
See also the CTAN topic \href{http://ctan.org/topic/subdocs}{\textsf{subdocs}}
for further related packages.
The present package differs from the above solutions in that
a document structure constructed with the conventional |\include| mechanism
just needs two extra commands at the top of every file
such that all constituent files can be compiled individually.

%%%%%%%%%%%%%%%%%%%%%%%%%%%%%%%%%%%%%%%%%%%%%%%%%%%%%%%%%%%%%%%%%%%%%%%%%%%%%%%%
%\subsection{Feature Suggestions}
%
%The following is a list of features which may be useful for future
%versions of this package:
%%
%\begin{itemize}
%\item
%\ldots
%\end{itemize}

%%%%%%%%%%%%%%%%%%%%%%%%%%%%%%%%%%%%%%%%%%%%%%%%%%%%%%%%%%%%%%%%%%%%%%%%%%%%%%%%
\subsection{Revision History}

%%%%%%%%%%%%%%%%%%%%%%%%%%%%%%%%%%%%%%%%
\paragraph{v2.0:} 2018/12/30

\begin{itemize}
\item
immediate forward processing
\item
added |\childdocby| mechanism
\item
manual restructured
\end{itemize}

%%%%%%%%%%%%%%%%%%%%%%%%%%%%%%%%%%%%%%%%
\paragraph{v1.6:} 2018/01/17

\begin{itemize}
\item
application for development of include files
\item
corrections to manual
\end{itemize}

%%%%%%%%%%%%%%%%%%%%%%%%%%%%%%%%%%%%%%%%
\paragraph{v1.5:} 2017/05/21

\begin{itemize}
\item
more complete structuring introduced
\item
|\childdocof| introduced
\item
|\childdoc| renamed to |\childdocmain|
\item
|\childredirect| renamed to |\childdocforward| and |\childdocforwardprefix|
and functionality expanded
\end{itemize}

%%%%%%%%%%%%%%%%%%%%%%%%%%%%%%%%%%%%%%%%
\paragraph{v1.0:} 2017/04/27

\begin{itemize}
\item
manual and install package
\item
first version published on CTAN
\end{itemize}

%%%%%%%%%%%%%%%%%%%%%%%%%%%%%%%%%%%%%%%%
\paragraph{v0.6:} 2017/04/26

\begin{itemize}
\item
redirection mechanism added
\end{itemize}

%%%%%%%%%%%%%%%%%%%%%%%%%%%%%%%%%%%%%%%%
\paragraph{v0.5:} 2017/04/26

\begin{itemize}
\item
functionality in definition file
\end{itemize}


%%%%%%%%%%%%%%%%%%%%%%%%%%%%%%%%%%%%%%%%%%%%%%%%%%%%%%%%%%%%%%%%%%%%%%%%%%%%%%%%
%%%%%%%%%%%%%%%%%%%%%%%%%%%%%%%%%%%%%%%%%%%%%%%%%%%%%%%%%%%%%%%%%%%%%%%%%%%%%%%%
%%%%%%%%%%%%%%%%%%%%%%%%%%%%%%%%%%%%%%%%%%%%%%%%%%%%%%%%%%%%%%%%%%%%%%%%%%%%%%%%
\appendix

\settowidth\MacroIndent{\rmfamily\scriptsize 000\ }

 \DocInput{childdoc.dtx}

\end{document}
%</driver>
% \fi
%
% %%%%%%%%%%%%%%%%%%%%%%%%%%%%%%%%%%%%%%%%%%%%%%%%%%%%%%%%%%%%%%%%%%%%%%%%%%%%%%
% %%%%%%%%%%%%%%%%%%%%%%%%%%%%%%%%%%%%%%%%%%%%%%%%%%%%%%%%%%%%%%%%%%%%%%%%%%%%%%
% \section{Sample}
%\iffalse
%<*samplemain>
%\fi
%
% The following presents a sample document
% with two chapters, two parts, a title page,
% a compile flag as well as three forwarding files to set the flag.
% It consists of eight |.tex| files:
% \begin{center}
% \begin{tabular}{ll}
% |cdocsamp.tex|&main file\\
% |cdocsch1.tex|&include file for chapter 1\\
% |cdocsch2.tex|&include file for chapter 2\\
% |cdocspt3.tex|&include file for part 3\\
% |cdocspt4.tex|&include file for part 4\\
% |cdocsdrf.tex|&forwarding file for main file in draft mode\\
% |cdocsfi1.tex|&forwarding file for final version of chapter 1\\
% |cdocsfi2.tex|&forwarding file for final version of chapter 2\\
% \end{tabular}
% \end{center}
% Each of the eight files can be compiled directly by the \LaTeX{} compiler.
%
% %%%%%%%%%%%%%%%%%%%%%%%%%%%%%%%%%%%%%%
% \paragraph{Main File.}
%
% The main file is called |cdocsamp.tex|.
%
% Load the \textsf{childdoc} definitions and
% declare the filename for the main document:
%    \begin{macrocode}
\input{childdoc.def}
\childdocmain{}
%    \end{macrocode}

% Optional override for |\version| flag:
%    \begin{macrocode}
%%\ifchilddoc\else\providecommand{\version}{draft}\fi
%    \end{macrocode}

% Define the default values for the |\version| flag
% (|final| for the main file and |draft| for childs):
%    \begin{macrocode}
\ifchilddoc
\providecommand{\version}{draft}
\else
\providecommand{\version}{final}
\fi
%    \end{macrocode}

% Load the standard document class:
%    \begin{macrocode}
\documentclass[12pt]{article}
%    \end{macrocode}

% Start the document body:
%    \begin{macrocode}
\begin{document}
%    \end{macrocode}

% Declare a title page.
% Print title, part of document being processed and version flag:
%    \begin{macrocode}
\addtocounter{page}{-1}
\begin{center}
{\LARGE\bfseries{}childdoc example\par}
\vspace{1cm}
\ifchilddoc
\ifchilddocmanual part\else chapter\fi:
`\childdocname' of `\childdocjob'\par
\else
main document: `\childdocjob'\par
\fi
version: \version\par
\end{center}
\newpage
%    \end{macrocode}

% Manually include selected file,
% otherwise process as usual:
%    \begin{macrocode}
\ifchilddocmanual
\section*{part `\childdocname'}
\input{\childdocname}
\else
%    \end{macrocode}

% Include the two chapters:
%    \begin{macrocode}
\include{cdocsch1}
\include{cdocsch2}
%    \end{macrocode}

% Include the two parts unless only chapters should be displayed:
%    \begin{macrocode}
\ifchilddoc\else
\section{part three}
\input{cdocspt3}
\section{part four}
\input{cdocspt4}
\fi
%    \end{macrocode}

% Process as usual until here:
%    \begin{macrocode}
\fi
%    \end{macrocode}

% End of document body:
%    \begin{macrocode}
\end{document}
%    \end{macrocode}
%\iffalse
%</samplemain>
%\fi
%
% %%%%%%%%%%%%%%%%%%%%%%%%%%%%%%%%%%%%%%
% \paragraph{Chapter Include Files.}
%
% The include files are called |cdocsch1.tex| and |cdocsch2.tex|.
%
%\iffalse
%<*samplechap1|samplechap2>
%\fi

% Optional override for |\version| flag:
%    \begin{macrocode}
%%\providecommand{\version}{final}
%    \end{macrocode}

% Include the main document:
%    \begin{macrocode}
\input{childdoc.def}
\childdocof{cdocsamp}
%    \end{macrocode}

%\iffalse
%</samplechap1|samplechap2>
%\fi
%
%\iffalse
%<*samplechap1>
%\fi
% Some text for chapter 1:
%    \begin{macrocode}
\section{one}
some text in chapter one
%    \end{macrocode}

%\iffalse
%</samplechap1>
%\fi
% Some text for chapter 2:
%\iffalse
%<*samplechap2>
%\fi
%    \begin{macrocode}
\section{two}
more text in chapter two
%    \end{macrocode}

%\iffalse
%</samplechap2>
%\fi
%
% %%%%%%%%%%%%%%%%%%%%%%%%%%%%%%%%%%%%%%
% \paragraph{Part Include Files.}
%
% The include files are called |cdocspt3.tex| and |cdocspt4.tex|.
%
%\iffalse
%<*samplepart3|samplepart4>
%\fi

% Optional override for |\version| flag:
%    \begin{macrocode}
%%\providecommand{\version}{final}
%    \end{macrocode}

% Include the main document:
%    \begin{macrocode}
\input{childdoc.def}
\childdocby{cdocsamp}
%    \end{macrocode}

%\iffalse
%</samplepart3|samplepart4>
%\fi
%
%\iffalse
%<*samplepart3>
%\fi
% Some text for part 3:
%    \begin{macrocode}
some text in part three
%    \end{macrocode}

%\iffalse
%</samplepart3>
%\fi
% Some text for part 4:
%\iffalse
%<*samplepart4>
%\fi
%    \begin{macrocode}
more text in part four
%    \end{macrocode}

%\iffalse
%</samplepart4>
%\fi
%
% %%%%%%%%%%%%%%%%%%%%%%%%%%%%%%%%%%%%%%
% \paragraph{Forwarding for a Complete Draft.}
%
% The following forwarding file |cdocsdrf.tex|
% compiles the main document in draft mode:
%\iffalse
%<*sampledraft>
%\fi
%    \begin{macrocode}
\def\version{draft}
\input{childdoc.def}
\childdocforward{cdocsamp}
%    \end{macrocode}

%\iffalse
%</sampledraft>
%\fi
%
% %%%%%%%%%%%%%%%%%%%%%%%%%%%%%%%%%%%%%%
% \paragraph{Forwarding for Final Version of the Chapters.}
%
% The following forwarding files |cdocsfn1.tex| and |cdocsfn2.tex|
% (with identical content)
% compile the final versions of the child documents
% |cdocsch1.tex| and |cdocsch2.tex|, respectively:
%\iffalse
%<*samplefinal>
%\fi
%    \begin{macrocode}
\def\version{final}
\input{childdoc.def}
\childdocforwardprefix[cdocsamp]{cdocsfn}{cdocsch}
%    \end{macrocode}

%\iffalse
%</samplefinal>
%\fi
%
% %%%%%%%%%%%%%%%%%%%%%%%%%%%%%%%%%%%%%%
% \paragraph{Command Line Processing.}
%
% The following three command lines generate the output files
% |cdocscld|, |cdocscl1| and |cdocscl2|
% which should be identical to
% |cdocsdrf|, |cdocsch1| and |cdocsfn2|, respectively:
% \begin{center}
% \begin{tabular}{l}
% |latex -jobname cdocscld \|\\
% |  "\def\version{draft}\input{childdoc.def}\childdocforward{cdocsamp}"|\\
% |latex -jobname cdocscl1 \|\\
% |  "\input{childdoc.def}\childdocforward[cdocsamp]{cdocsch1}"|\\
% |latex -jobname cdocscl2 \|\\
% |  "\def\version{final}\input{childdoc.def}\childdocforward{cdocsch2}"|
% \end{tabular}
% \end{center}
% Note that the trailing backslash on each first line
% merely continues the input to the second line
% (for convenient cut ant paste).
% Furthermore, the command |latex| can be replaced by any
% of its alternative versions such as |pdflatex|.
%
% %%%%%%%%%%%%%%%%%%%%%%%%%%%%%%%%%%%%%%%%%%%%%%%%%%%%%%%%%%%%%%%%%%%%%%%%%%%%%%
% %%%%%%%%%%%%%%%%%%%%%%%%%%%%%%%%%%%%%%%%%%%%%%%%%%%%%%%%%%%%%%%%%%%%%%%%%%%%%%
% \section{Implementation}
%\iffalse
%<*package>
%\fi
%
% This section describes the definitions file |childdoc.def|.

% The definitions cannot be loaded using |\usepackage| or |\RequirePackage|
% which has a mechanism to prevent loading a style file more than once.
% When loading the definitions by means of |\input|
% multiple instances have to be prevented manually:
%\iffalse
%This code needs to be before the `\ProvidesFile' directive
%which is defined at the beginning of this file.
%Therefore it is also placed there and commented out here.
%</package>
%<*discard>
%\fi
%    \begin{macrocode}
\ifdefined\childdocmain\endinput\fi
%    \end{macrocode}
%\iffalse
%</discard>
%<*package>
%\fi
%
% \macro{\ifchilddoc}
% \macro{\ifchilddocmanual}
% The conditional |\ifchilddoc| tells whether a
% child (true) or main (false) document is being compiled.
% The conditional |\ifchilddocmanual| tells whether
% the |\includeonly| mechanism is used (false) or
% the selection of child files must be performed manually (true).
% The definitions initialise to false:
%    \begin{macrocode}
\newif\ifchilddoc
\newif\ifchilddocmanual
%    \end{macrocode}

% \macro{\childdocname}
% \macro{\childdocjob}
% The macro |\childdocname| stores the name of the main document
% to be compiled. The macro |\childdocjob| stores the name of
% the document on which the \LaTeX{} compiler was originally invoked.
% The content of |\jobname| cannot be compared
% to filenames specified in the source due to different catcodes.
% The following code rescans |\jobname|, stores the result
% in |\childdocname| and saves a copy in |\childdocjob|:
%    \begin{macrocode}
\edef\childdocname{\scantokens\expandafter{\jobname\noexpand}}
\let\childdocjob\childdocname
%    \end{macrocode}

% \macro{\childdocdisable}
% The macro |\childdocdisable| prevents the main file
% from being processed more than once.
% At this stage, the main document command |\childdocmain|
% is assumed to be called once again where it should do nothing.
% Any subsequent call to it should prevent
% a secondary processing of the main document
% It overwrites the forwarding commands
% |\childdocof| and |\childdocforward|
% with empty macros to prevent further inclusions of the main document:
%    \begin{macrocode}
\newcommand{\childdocdisable}
{
  \renewcommand{\childdocmain}[1]{\renewcommand{\childdocmain}[1]{\endinput}}
  \renewcommand{\childdocof}[1]{}
  \renewcommand{\childdocby}[2][]{}
  \renewcommand{\childdocforward}[2][]{}
  \renewcommand{\childdocdisable}{}
}
%    \end{macrocode}

% \macro{\childdocmain}
% The macro |\childdocmain| is to be called at the top of the main file
% with nothing or the main filename (without extension) as argument.
% First, it breaks loops.
% If the argument is not empty and does not match |\childdocname|
% (which is set by the first inclusion of |childdoc.def|),
% |\ifchilddoc| is set to true, |\includeonly| is applied to the child file
% and |\jobname| is set to the main file
% (for proper handling of |.aux| files):
%    \begin{macrocode}
\newcommand{\childdocmain}[1]
{
  \childdocdisable\childdocmain{}
  \if?#1?\else
    \begingroup
      \def\childdoctmp{#1}
      \ifx\childdoctmp\childdocname
        \def\childdoctmp{}
      \else
        \def\childdoctmp
        {
          \childdoctrue
          \includeonly{\childdocname}
          \def\childdocjob{#1}
          \def\jobname{#1}
        }
      \fi
      \expandafter
    \endgroup
    \childdoctmp
  \fi
}
%    \end{macrocode}

% \macro{\childdocof}
% The command |\childdocof| redirects
% compilation to the main file |#1|.
%    \begin{macrocode}
\newcommand{\childdocof}[1]
{
  \childdocdisable
  \childdoctrue
  \includeonly{\childdocname}
  \def\jobname{#1}
  \def\childdocjob{#1}
  \input{#1}
}
%    \end{macrocode}

% \macro{\childdocby}
% The command |\childdocby| ....
%    \begin{macrocode}
\newcommand{\childdocby}[2][]
{
  \childdocdisable
  \childdoctrue
  \childdocmanualtrue
  \if?#1?\else
    \def\jobname{#2}
  \fi
  \def\childdocjob{#2}
  \input{#2}
  \endinput
}
%    \end{macrocode}

% \macro{\childdocforward}
% The command |\childdocforward| redirects
% compilation to the main file or
% (if the optional argument is given) a child file.
% Parameters are set as if the main file
% or a child file starting with |\childdocof| was compiled.
% Then compilation is handed over to the main file:
%    \begin{macrocode}
\newcommand{\childdocforward}[2][]
{
  \begingroup
    \if?#1?
      \def\childdoctmp
      {
        \def\childdocname{#2}
        \def\childdocjob{#2}
        \def\jobname{#2}
        \input{#2}
        \endinput
      }
    \else
      \def\childdoctmp
      {
        \childdocdisable
        \def\childdocname{#2}
        \childdoctrue
        \includeonly{#2}
        \def\childdocjob{#1}
        \def\jobname{#1}
        \input{#1}
        \endinput
      }
    \fi
    \expandafter
  \endgroup
  \childdoctmp
}
%    \end{macrocode}

% \macro{\childdocforwardprefix}
% The command |\childdocforwardprefix| redirects
% compilation to the main or a child file by means of a pattern.
% The prefix |#1| in the current filename is replaced by |#2|
% and the suffix of the current filename is kept
% (it is assumed that the filename does not contain the substring `|~~~|'
% which is used as a delimiter).
% Compilation is handed over to the new file by |\childdocforward|:
%    \begin{macrocode}
\newcommand{\childdocforwardprefix}[3][]
{
  \begingroup
    \def\childdocextract #2##1~~~{\def\childdoctmp{\childdocforward[#1]{#3##1}}}
    \expandafter\childdocextract\childdocname~~~
    \expandafter
  \endgroup
  \childdoctmp
}
%    \end{macrocode}

% \macro{\childdoc}
% The deprecated macro |\childdoc| is a legacy version of |\childdocmain|:
%    \begin{macrocode}
\newcommand{\childdoc}{\childdocmain}
%    \end{macrocode}

% \macro{\childdocredirect}
% The deprecated macro |\childdocredirect| is a legacy version
% of |\childdocforward| and |\childdocforwardprefix|:
%    \begin{macrocode}
\newcommand{\childdocredirect}[2][]
{
  \begingroup
    \if?#1?
      \def\childdoctmp{\childdocforward{#2}}
    \else
      \def\childdoctmp{\childdocforwardprefix{#1}{#2}}
    \fi
    \expandafter
  \endgroup
  \childdoctmp
}
%    \end{macrocode}

%\iffalse
%</package>
%\fi
%
\endinput
|\\
|\childdocmain{|\textit{main}|}|\\
\end{tabular}
\end{center}
%
If |\jobname| does not match the argument \textit{main} of |\childdocmain|,
it is assumed that |\jobname| points to the child file to be compiled.
When using |\childdocmain| with the main file specified as argument,
it suffices to start a child file
with just |\input{|\textit{main}|}|
without loading of the package and using |\childdocof|.
If instead all processing is done
with the appropriate \textsf{childdoc} directives,
the argument of \textit{main} of |\childdocmain| can be empty.

An alternative version of the command line processing described
in \secref{sec:commandline} using the detection mechanism reads:
%
\begin{center}
|... -jobname "|\textit{target}|" "|[\textit{flags}]%
[|\def\jobname{|\textit{dest}|}|]|\input{|\textit{main}|}"|
\end{center}

%%%%%%%%%%%%%%%%%%%%%%%%%%%%%%%%%%%%%%%%%%%%%%%%%%%%%%%%%%%%%%%%%%%%%%%%%%%%%%%%
\subsection{Manual Code}
\label{sec:manual}

In case one cannot be certain whether the definitions file |childdoc.def|
is installed on the target \TeX{} distribution
and one prefers not to ship it,
it is conceivable to paste a few relevant commands into the sources.

To that end, drop all statements |% \iffalse
%
% childdoc.dtx Copyright (C) 2017-2018 Niklas Beisert
%
% This work may be distributed and/or modified under the
% conditions of the LaTeX Project Public License, either version 1.3
% of this license or (at your option) any later version.
% The latest version of this license is in
%   http://www.latex-project.org/lppl.txt
% and version 1.3 or later is part of all distributions of LaTeX
% version 2005/12/01 or later.
%
% This work has the LPPL maintenance status `maintained'.
%
% The Current Maintainer of this work is Niklas Beisert.
%
% This work consists of the files childdoc.dtx and childdoc.ins
% and the derived files childdoc.def and cdocsamp.tex with
% cdocsch1.tex, cdocsch2.tex, cdocsdrf.tex, cdocsfn1.tex, cdocsfn2.tex.
%
%<package>\ifdefined\childdocmain\endinput\fi
%<package>\ProvidesFile{childdoc.def}[2018/12/30 v2.0 child document driver]
%<samplemain>\ProvidesFile{cdocsamp.tex}[2018/12/30 v2.0 sample for childdoc]
%<*driver>
%\ProvidesFile{childdoc.drv}[2018/12/30 v2.0 childdoc reference manual file]
\PassOptionsToClass{10pt,a4paper}{article}
\documentclass{ltxdoc}

\usepackage[margin=35mm]{geometry}
\usepackage{hyperref}
\usepackage{hyperxmp}
\usepackage[usenames]{color}

\hypersetup{colorlinks=true}
\hypersetup{pdfstartview=FitH}
\hypersetup{pdfpagemode=UseNone}
\hypersetup{pdfsource={}}
\hypersetup{pdflang={en-UK}}
\hypersetup{pdfcopyright={Copyright 2017-2018 Niklas Beisert.
  This work may be distributed and/or modified under the
  conditions of the LaTeX Project Public License, either version 1.3
  of this license or (at your option) any later version.}}
\hypersetup{pdflicenseurl={http://www.latex-project.org/lppl.txt}}
\hypersetup{pdfcontactaddress={ETH Zurich, ITP, HIT K,
  Wolfgang-Pauli-Strasse 27}}
\hypersetup{pdfcontactpostcode={8093}}
\hypersetup{pdfcontactcity={Zurich}}
\hypersetup{pdfcontactcountry={Switzerland}}
\hypersetup{pdfcontactemail={nbeisert@itp.phys.ethz.ch}}
\hypersetup{pdfcontacturl={http://people.phys.ethz.ch/\xmptilde nbeisert/}}

\newcommand{\secref}[1]{\hyperref[#1]{section \ref*{#1}}}

\parskip1ex
\parindent0pt
\let\olditemize\itemize
\def\itemize{\olditemize\parskip0pt}

\begin{document}

\title{The \textsf{childdoc} Package}
\hypersetup{pdftitle={The childdoc Package}}
\author{Niklas Beisert\\[2ex]
  Institut f\"ur Theoretische Physik\\
  Eidgen\"ossische Technische Hochschule Z\"urich\\
  Wolfgang-Pauli-Strasse 27, 8093 Z\"urich, Switzerland\\[1ex]
  \href{mailto:nbeisert@itp.phys.ethz.ch}
  {\texttt{nbeisert@itp.phys.ethz.ch}}}
\hypersetup{pdfauthor={Niklas Beisert}}
\hypersetup{pdfsubject={Manual for the LaTeX2e Package childdoc}}
\date{30 December 2018, \textsf{v2.0}}
\maketitle

\begin{abstract}\noindent
\textsf{childdoc} is a \LaTeXe{} package
that enables the direct compilation
of document sections included by |\include|
to individual files.
\end{abstract}

\begingroup
\parskip0ex
\tableofcontents
\endgroup

%%%%%%%%%%%%%%%%%%%%%%%%%%%%%%%%%%%%%%%%%%%%%%%%%%%%%%%%%%%%%%%%%%%%%%%%%%%%%%%%
%%%%%%%%%%%%%%%%%%%%%%%%%%%%%%%%%%%%%%%%%%%%%%%%%%%%%%%%%%%%%%%%%%%%%%%%%%%%%%%%
\section{Introduction}

\LaTeX{} provides a mechanism to structure a large document (such as a book)
into a main file and several child files (containing the chapters)
using the |\include| command.
This mechanism is beneficial for documents
which span hundreds of pages in order to
make the source file(s) more manageable.
Moreover, compilation can be restricted to
selected child files by means of the |\includeonly| command.
The latter feature can be used to reduce the compilation time while editing
(this was significantly more useful in the earlier days of \LaTeX{})
or to generate a smaller document which is easier to navigate.
Another application of |\includeonly| is to generate
documents consisting of selected parts of the complete document.

However, there are a few drawbacks of the plain |\include| mechanism:
\begin{itemize}
\item
The child files cannot be compiled on their own,
they can only be compiled via the main file.
A naive editing environment
(such as a text editor with an option
to have the current file processed by \LaTeX)
may require one to switch to the main file before compiling;
attempting to compile the child file produces errors.
\item
The main file must be modified (each time)
to adjust the |\includeonly| command
to the present needs. This easily leaves the main file in a messy state.
\item
The generated document will always carry the filename
of the main document. This is inconvenient if
several child files are to be compiled and
to be kept for distribution.
\end{itemize}

The present package provides a simple interface
to make child files individually compilable by \LaTeX{}.
Compiling a child file then has the same effect as compiling
the main file with an |\includeonly| command
to select the appropriate child.
Moreover the generated document will carry the name of the child
rather than the main file.
This resolves all three above issues.

This feature is meant to make the editing of books,
thesis documents and lecture notes somewhat more convenient.
However, the package can also be used efficiently for
composing a series of documents (such as exercise sheets)
which are typically distributed individually.
It then assists the author in generating the individual documents
(potentially in different versions)
as well as a document containing the collected series.
Another application is in developing style files
or other kinds of included material
where compilation of the style file could redirect
to a sample or test file.

%%%%%%%%%%%%%%%%%%%%%%%%%%%%%%%%%%%%%%%%%%%%%%%%%%%%%%%%%%%%%%%%%%%%%%%%%%%%%%%%
%%%%%%%%%%%%%%%%%%%%%%%%%%%%%%%%%%%%%%%%%%%%%%%%%%%%%%%%%%%%%%%%%%%%%%%%%%%%%%%%
\section{Usage}

First of all, the package \textsf{childdoc} is \emph{not} a standard
\LaTeXe{} |.sty| style file! Therefore it needs to be invoked in
a non-standard way.

%%%%%%%%%%%%%%%%%%%%%%%%%%%%%%%%%%%%%%%%%%%%%%%%%%%%%%%%%%%%%%%%%%%%%%%%%%%%%%%%
\subsection{Included Files}
\label{sec:include}

%%%%%%%%%%%%%%%%%%%%%%%%%%%%%%%%%%%%%%%%
\DescribeMacro{\childdocmain}
To use the package, add the commands
\begin{center}
\begin{tabular}{l}
|\input{childdoc.def}|\\
|\childdocmain{}|\\
\end{tabular}
\end{center}
at the very top of the main \LaTeX{} file,
in particular \emph{before} the |\documentclass| statement!
The argument of |\childdocmain| should be left empty
(but it must be present).

%%%%%%%%%%%%%%%%%%%%%%%%%%%%%%%%%%%%%%%%
\DescribeMacro{\childdocof}
Furthermore, add the commands
\begin{center}
\begin{tabular}{l}
|\input{childdoc.def}|\\
|\childdocof{|\textit{main}|}|\\
\end{tabular}
\end{center}
at the top of every child file \textit{child}
which is included by |\include{|\textit{child}|}|
from within the main file
(or at least for those files to be compiled individually).
The argument \textit{main} must be the filename of the main file.

There are a couple of
considerations in setting up the main and child documents:

%%%%%%%%%%%%%%%%%%%%%%%%%%%%%%%%%%%%%%%%
\paragraph{Restrictions.}

Please note the following restrictions:
\begin{itemize}
\item
|\childdocmain| must be called with one argument \textit{main}
to ensure compatibility with earlier version of the package.
It must either be empty (|\childdocmain{}|)
or precisely match the filename of the main file in which it is specified.
See \secref{sec:detection} for further information.
\item
The filename \textit{main} must be specified without the |.tex| extension.
\item
The filename \textit{main} is case sensitive
(even in case-insensitive file systems)
due to internal string comparison.
\item
The argument \textit{main} should be fully expanded, it cannot be a macro.
\item
Subdirectories and special characters should be avoided in filenames.
\item
The command |\childdocmain{|\textit{main}|}| must be followed by a whitespace.
It should not be followed immediately by another command
or by a comment mark `|%|'.
This is because the \TeX{} parser reads the token immediately following
the argument of |\childdocmain| and puts it
at the beginning of every child section;
however, a white\-space is ignored.
\end{itemize}

%%%%%%%%%%%%%%%%%%%%%%%%%%%%%%%%%%%%%%%%
\paragraph{Content of Main File.}

It is advisable to place all content in the child files included by |\include|.
Any output contained in the main file will appear in all child documents
unless suppressed manually;
it cannot be suppressed automatically by the |\includeonly| directive
and thus should normally be avoided.
A method to include some content in the main file
by means of conditional processing is described in \secref{sec:conditional}.

%%%%%%%%%%%%%%%%%%%%%%%%%%%%%%%%%%%%%%%%
\paragraph{Page Numbering.}

When only a part of the document is compiled,
the appropriate numbering of pages
(as well as other status parameters)
is determined from the |.aux| files.
The latter contain information from previous passes.
However this information needs to propagate through
all intermediate child documents.
Therefore the page numbering in child documents may well
be inconsistent until the complete document is compiled at least once.

A useful (if unconventional) way to always ensure a consistent
page numbering is to restart the numbering in each child document
and denote the pages by `\textit{child}|.|\textit{page}'
where \textit{child} represents the chapter/section number of the child file.
This can be achieved by the command
|\numberwithin{page}{|\textit{child}|}|
of the \textsf{amsmath} package
where \textit{child} can be |chapter| or |section|
depending on the chosen structuring.
Alternatively, one can modify the macro |\thepage| appropriately
and reset the counter |page| at the start of each child file.

%%%%%%%%%%%%%%%%%%%%%%%%%%%%%%%%%%%%%%%%%%%%%%%%%%%%%%%%%%%%%%%%%%%%%%%%%%%%%%%%
\subsection{Conditional Processing}
\label{sec:conditional}

The package provides a mechanism to compile different versions
of a document. To customise the versions further some conditional processing
can come in handy to distinguish which version is being compiled.
The package provides two macros to describe the compilation context:

%%%%%%%%%%%%%%%%%%%%%%%%%%%%%%%%%%%%%%%%
\DescribeMacro{\ifchilddoc}
The conditional |\ifchilddoc| distinguishes between the compilation of
child documents and the main document:
%
\begin{center}
|\ifchilddoc |\textit{child-code}| |[|\||else |\textit{main-code}]| \||fi|
\end{center}

%%%%%%%%%%%%%%%%%%%%%%%%%%%%%%%%%%%%%%%%
\DescribeMacro{\childdocname}
\DescribeMacro{\childdocjob}
The macro |\childdocname| contains the filename (without extension)
of the main or child file being processed.
Note that |\childdocjob| will always contain the name of the main file.

%%%%%%%%%%%%%%%%%%%%%%%%%%%%%%%%%%%%%%%%
\paragraph{Title Page.}

Conditional processing can be used to include a title or banner page
in the main document when proper precautions are taken.
Importantly, the code in the main file should ensure that the page counter
(as well as other status parameters which are stored in the |.aux| files)
takes the same value after the conditional processing.
Otherwise the page numbers may take divergent values
depending on which part is compiled.

For example, a title page could be declared by:
%
\begin{center}
\begin{tabular}{l}
|\ifchilddoc\||else|\\
|\addtocounter{page}{-1}|\\
\textit{code for title page}\\
|\newpage|\\
|\||fi|
\end{tabular}
\end{center}
%
A banner page for the child documents can be generated by:
%
\begin{center}
\begin{tabular}{l}
|\ifchilddoc|\\
|\addtocounter{page}{-1}|\\
\textit{code for banner page}\\
|\newpage|\\
|\||fi|
\end{tabular}
\end{center}
%
Here one could write a message such as:
\begin{center}
|This is the part \childdocname{} of \childdocjob{}.|
\end{center}

%%%%%%%%%%%%%%%%%%%%%%%%%%%%%%%%%%%%%%%%%%%%%%%%%%%%%%%%%%%%%%%%%%%%%%%%%%%%%%%%
\subsection{Flags}
\label{sec:flags}

The package makes it easy to generate different versions
of the main or child documents.
To this end compilation flags can be defined
and assigned different default values.
They will be particularly useful in conjunction
with the forwarding mechanism described in \secref{sec:forward}.

For example, it may be useful to have a flag |\version|
which can be set to |draft| or |final|.
The document source will contain some conditional code
depending on the value of |\version|.
Suppose further, the flag should default to |final| for the main file
and to |draft| for child files
which is a natural assignment for editing the document.
This is achieved by placing the following code
in the preamble of the main document
(below the |\childdocmain| directive):
%
\begin{center}
\begin{tabular}{l}
|\ifchilddoc|\\
|\providecommand{\version}{draft}|\\
|\||else|\\
|\providecommand{\version}{final}|\\
|\||fi|
\end{tabular}
\end{center}
%
The definition by |\providecommand| makes sure
that previous definitions are not overwritten.
Further statements |\providecommand{\version}{...}|
can thus be added before the above code to override it.

For the main file, one might add a line
(between |\childdocmain| and the above block)
%
\begin{center}
|%\ifchilddoc\||else\providecommand{\version}{draft}\||fi|
\end{center}
%
which can be uncommented to produce a draft version.
Likewise one can add a line to the very top of a child file
(above the |\childdocof{|\textit{main}|}| directive)
%
\begin{center}
|%\providecommand{\version}{final}|
\end{center}
%
which can be uncommented to produce the final version of this child document.

%%%%%%%%%%%%%%%%%%%%%%%%%%%%%%%%%%%%%%%%%%%%%%%%%%%%%%%%%%%%%%%%%%%%%%%%%%%%%%%%
\subsection{Forwarding}
\label{sec:forward}

Different versions of the main or child documents
using compilation flags as described in \secref{sec:flags}
can be (permanently) stored in different files
for convenient compilation, viewing and distribution.
To this end, the package defines a command
to pass on compilation to a different file:

%%%%%%%%%%%%%%%%%%%%%%%%%%%%%%%%%%%%%%%%
\DescribeMacro{\childdocforward}
The command |\childdocforward| redirects processing to
another source file:
%
\begin{center}
\begin{tabular}{l}
|\input{childdoc.def}|\\
|\childdocforward[|\textit{main}|]{|\textit{dest}|}|\\
\end{tabular}
\end{center}
%
The argument \textit{dest} is the destination file
(without extension).
It should be the main file or one of the child files.
Note that further \textsf{childdoc} directives
such as |\childdocof| and |\childdocforward|
in the indicated file will be processed in this form.
The optional argument \textit{main}
passes on directly to the main file \textit{main}
while pretending to compile the child \textit{dest}.
This form behaves as if \textit{dest}
issues |\childdocof{|\textit{main}|}| right away,
and no further \textsf{childdoc} directives will be processed.

%%%%%%%%%%%%%%%%%%%%%%%%%%%%%%%%%%%%%%%%
\DescribeMacro{\...prefix}
In the alternative form |\childdocforwardprefix|,
%
\begin{center}
\begin{tabular}{l}
|\input{childdoc.def}|\\
|\childdocforwardprefix[|\textit{main}|]{|\textit{prefix}|}{|\textit{dest}|}|
\end{tabular}
\end{center}
%
the destination file is determined by a pattern
depending on the current file:
To make this work, the current file must be called
`{\textit{prefix}\hspace{0.2em}\textit{suffix}}'
with \textit{prefix} matching precisely the argument.
Processing is then passed on to the file
`{\textit{dest}\hspace{0.2em}\textit{suffix}}'.
Surely, the same effect is achieved by
directly specifying the
argument `{\textit{dest}\hspace{0.2em}\textit{suffix}}'
in the first form.
However, that requires to set up a different file
for each child. With the alternative form of the command
all these files can have exactly the same content
which simplifies setting them up and maintaining them.

For example, the following file |draft.tex|
with a compilation flag |\version| as described in \secref{sec:flags}
compiles the main document as a draft:
%
\begin{center}
\begin{tabular}{l}
|\def\version{draft}|\\
|\input{childdoc.def}|\\
|\childdocforward{|\textit{main}|}|
\end{tabular}
\end{center}
%
Likewise, the following files |final|\textit{nn}|.tex|
compile the final version of the child document
|child|\textit{nn}|.tex|:
%
\begin{center}
\begin{tabular}{l}
|\def\version{final}|\\
|\input{childdoc.def}|\\
|\childdocforwardprefix{final}{child}|
\end{tabular}
\end{center}
%

Note that when several versions of a main file and/or of each child file
are to be generated, it may be convenient to set up a |Makefile| or
shell script to automatise the process.

%%%%%%%%%%%%%%%%%%%%%%%%%%%%%%%%%%%%%%%%%%%%%%%%%%%%%%%%%%%%%%%%%%%%%%%%%%%%%%%%
\subsection{Command Line Processing}
\label{sec:commandline}

The effect of redirection files can also be achieved by invoking
the \LaTeX{} compiler with a more elaborate command line.
Most conveniently this should be done as part
of a shell script or a |Makefile|.

When using \textsf{childdoc} in the main file, the following
command lines effectively perform a redirection
(note that depending on the shell being used,
backslashes may have to be doubled: `|\|' $\to$ `|\\|'):
%
\begin{center}
|... -jobname "|\textit{target}|" |\\|"|[\textit{flags}]%
|\input{childdoc.def}\childdocforward[|\textit{main}|]{|\textit{dest}|}"|
\end{center}
%
Here \textit{target} is the name of the output file,
\textit{main} is the name of the main file
and \textit{dest} is the name of the main or child file to be processed
(all filenames without extensions).
The optional argument \textit{main} can be omitted
if \textit{main} matches \textit{dest}.
Optionally, compilation \textit{flags} can be defined via |\def| commands.
This command line makes the \TeX{} engine believe
it is compiling the file \textit{target}
whose content is specified as the latter parameter.
The provided code then forwards the processing to
\textit{main} or \textit{dest} as described in \secref{sec:forward}.

%%%%%%%%%%%%%%%%%%%%%%%%%%%%%%%%%%%%%%%%%%%%%%%%%%%%%%%%%%%%%%%%%%%%%%%%%%%%%%%%
\subsection{Include by Input}
\label{sec:input}

Including child documents by |\include| has some restrictions by design.
Most notably, the content of a child document always occupies
its own set of pages; pages cannot be shared between child documents.
Usually, this behaviour makes perfect sense
because each child document contain an essential part of the document.
However, in some situations it may be desirable to compose
a document from a collection of parts
without having mandatory page breaks between then.
For this case, the package
provides a mechanism to include parts
by |\input| which can also be processed individually.
However, by construction this mechanism
requires manual handling of the content to be output.

%%%%%%%%%%%%%%%%%%%%%%%%%%%%%%%%%%%%%%%%
\DescribeMacro{\ifchilddocmanual}
The main file should be prepared as usual, see \secref{sec:include}.
However, the document body must make a distinction
between processing of an individual part and of the main document, e.g.:
%
\begin{center}
\begin{tabular}{l}
|\ifchilddocmanual|\\
|\input{\childdocname}|\\
|\||else|\\
\textit{document body with }|\input{|\textit{part}|}|\\
|\||fi|
\end{tabular}
\end{center}
%
The conditional |\ifchilddocmanual| is true whenever
a part to be included by |\input| is being compiled,
and the name of the part is stored in |\childdocname|.

%%%%%%%%%%%%%%%%%%%%%%%%%%%%%%%%%%%%%%%%
\DescribeMacro{\childdocby}
Each part to be included by |\input| should start with:
%
\begin{center}
\begin{tabular}{l}
|\input{childdoc.def}|\\
|\childdocby{|\textit{main}|}|\\
\end{tabular}
\end{center}
%
The directive |\childdocby| is similar to |\childdocof|
described in \secref{sec:include},
but the subsequent selection of content must be done manually.
To that end, both |\ifchilddoc| and |\ifchilddocmanual|
will be true upon processing of a part,
and the name of the part is stored in |\childdocname|.
Note that |\jobname| will be set to the filename of the current part
so that each part receives an individual |.aux| file
that does not interfere with the |.aux| file(s) of the main document.
This behaviour can be altered by the alternative form
|\childdocby[*]{|\textit{main}|}| (with a non-empty optional argument)
which uses the |.aux| file of the main document
by setting |\jobname| to \textit{main}.

%%%%%%%%%%%%%%%%%%%%%%%%%%%%%%%%%%%%%%%%%%%%%%%%%%%%%%%%%%%%%%%%%%%%%%%%%%%%%%%%
\subsection{Driver Development}
\label{sec:driver}

The \textsf{childdoc} mechanism can also be use for the development
of definition files such as \LaTeX{} styles or classes.
This case differs from the above setup with multiple parts
included by |\include| in that no |\includeonly| should be invoked.
This can be achieved by starting the include file
(before |\ProvidesPackage|) with:
%
\begin{center}
\begin{tabular}{l}
|\input{childdoc.def}|\\
|\childdocforward{|\textit{main}|}|\\
\end{tabular}
\end{center}
%
or alternatively with:
%
\begin{center}
\begin{tabular}{l}
|\input{childdoc.def}|\\
|\childdocby{|\textit{main}|}|\\
\end{tabular}
\end{center}
%
Both forms have slightly different effects as described above.
The main file is prepared as usual, see \secref{sec:include}.

%%%%%%%%%%%%%%%%%%%%%%%%%%%%%%%%%%%%%%%%%%%%%%%%%%%%%%%%%%%%%%%%%%%%%%%%%%%%%%%%
\subsection{Legacy Detection}
\label{sec:detection}

The directive |\childdocmain| in the main file can detect
whether the complete document or merely a child is to be compiled
even without using the directive |\childdocof|.
This method is deprecated because it is less robust
and there is no compelling reason to use it;
it is merely provided for backward compatibility
and it may be removed in future versions.

If the detection mechanism is to be used,
it is mandatory to correctly specify
the filename of the main file as the argument of |\childdocmain|:
%
\begin{center}
\begin{tabular}{l}
|\input{childdoc.def}|\\
|\childdocmain{|\textit{main}|}|\\
\end{tabular}
\end{center}
%
If |\jobname| does not match the argument \textit{main} of |\childdocmain|,
it is assumed that |\jobname| points to the child file to be compiled.
When using |\childdocmain| with the main file specified as argument,
it suffices to start a child file
with just |\input{|\textit{main}|}|
without loading of the package and using |\childdocof|.
If instead all processing is done
with the appropriate \textsf{childdoc} directives,
the argument of \textit{main} of |\childdocmain| can be empty.

An alternative version of the command line processing described
in \secref{sec:commandline} using the detection mechanism reads:
%
\begin{center}
|... -jobname "|\textit{target}|" "|[\textit{flags}]%
[|\def\jobname{|\textit{dest}|}|]|\input{|\textit{main}|}"|
\end{center}

%%%%%%%%%%%%%%%%%%%%%%%%%%%%%%%%%%%%%%%%%%%%%%%%%%%%%%%%%%%%%%%%%%%%%%%%%%%%%%%%
\subsection{Manual Code}
\label{sec:manual}

In case one cannot be certain whether the definitions file |childdoc.def|
is installed on the target \TeX{} distribution
and one prefers not to ship it,
it is conceivable to paste a few relevant commands into the sources.

To that end, drop all statements |\input{childdoc.def}|
and perform the replacements as outlined below.
Instead of |\childdocmain{|\textit{main}|}| add the following code
to the top of the main file:
%
\begin{center}
\begin{tabular}{l}
|\||ifdefined\childdocname\endinput\||fi\newif\ifchilddoc|\\
|\edef\childdocname{\scantokens\expandafter{\jobname\noexpand}}|\\
|\def\childdocmain{|\textit{main}|}\||ifx\childdocmain\childdocname\||else|\\
|\childdoctrue\includeonly{\childdocname}\let\jobname\childdocmain\||fi|\\
\end{tabular}
\end{center}
%
Instead of |\childdocof{|\textit{main}|}| just include the main file
at the top of each child file:
%
\begin{center}
|\input{|\textit{main}|}|
\end{center}
%
A simple redirection |\childdocforward{|\textit{dest}|}| is achieved by:
%
\begin{center}
|\def\jobname{|\textit{dest}|}\input{\jobname}|
\end{center}
%
The redirection with prefix
|\childdocforwardprefix[|\textit{prefix}|]{|\textit{dest}|}|
is accomplished by:
%
\begin{center}
\begin{tabular}{l}
|{\edef\jobname{\scantokens\expandafter{\jobname\noexpand}}|\\
|\def\redirectjob |\textit{prefix}|#1~~~{\gdef\jobname{|\textit{dest}|#1}}|\\
|\expandafter\redirectjob\jobname~~~}\input{\jobname}|
\end{tabular}
\end{center}

In an alternative approach,
child documents can be compiled by a specific command line
without additional code or specific definitions:
%
\begin{center}
|... -jobname "|\textit{target}|" "|[\textit{flags}]%
|\includeonly{|\textit{dest}|}\input{|\textit{main}|}"|
\end{center}
%

%%%%%%%%%%%%%%%%%%%%%%%%%%%%%%%%%%%%%%%%%%%%%%%%%%%%%%%%%%%%%%%%%%%%%%%%%%%%%%%%
%%%%%%%%%%%%%%%%%%%%%%%%%%%%%%%%%%%%%%%%%%%%%%%%%%%%%%%%%%%%%%%%%%%%%%%%%%%%%%%%
\section{Information}

%%%%%%%%%%%%%%%%%%%%%%%%%%%%%%%%%%%%%%%%%%%%%%%%%%%%%%%%%%%%%%%%%%%%%%%%%%%%%%%%
\subsection{Copyright}

Copyright \copyright{} 2017--2018 Niklas Beisert

This work may be distributed and/or modified under the
conditions of the \LaTeX{} Project Public License, either version 1.3
of this license or (at your option) any later version.
The latest version of this license is in
  \url{http://www.latex-project.org/lppl.txt}
and version 1.3 or later is part of all distributions of \LaTeX{}
version 2005/12/01 or later.

This work has the LPPL maintenance status `maintained'.

The Current Maintainer of this work is Niklas Beisert.

This work consists of the files |README.txt|, |childdoc.ins| and |childdoc.dtx|
as well as the derived files |childdoc.def|, |cdocsamp.tex|
with |cdocsch1.tex|, |cdocsch2.tex|, |cdocspt3.tex|, |cdocspt4.tex|,
|cdocsdrf.tex|, |cdocsfn1.tex|, |cdocsfn2.tex|
as well as |childdoc.pdf|.

%%%%%%%%%%%%%%%%%%%%%%%%%%%%%%%%%%%%%%%%%%%%%%%%%%%%%%%%%%%%%%%%%%%%%%%%%%%%%%%%
\subsection{Files and Installation}

The package consists of the files:
%
\begin{center}
\begin{tabular}{ll}
    |README.txt|   & readme file \\
    |childdoc.ins| & installation file \\
    |childdoc.dtx| & source file \\
    |childdoc.def| & definition file \\
    |cdocsamp.tex| & sample main file \\
    |cdocsch1.tex| & sample include file \\
    |cdocsch2.tex| & sample include file \\
    |cdocspt3.tex| & sample part file \\
    |cdocspt4.tex| & sample part file \\
    |cdocsdrf.tex| & sample redirection file \\
    |cdocsfn1.tex| & sample redirection file \\
    |cdocsfn2.tex| & sample redirection file \\
    |childdoc.pdf| & manual
\end{tabular}
\end{center}
%
The distribution consists of the files
|README.txt|, |childdoc.ins| and |childdoc.dtx|.
%
\begin{itemize}
\item
Run (pdf)\LaTeX{} on |childdoc.dtx|
to compile the manual |childdoc.pdf| (this file).
\item
Run \LaTeX{} on |childdoc.ins| to create the definitions file |childdoc.def|
and the sample |cdocsamp.tex| with include files
|cdocsch1.tex|, |cdocsch2.tex|, |cdocspt3.tex|, |cdocspt4.tex|,
|cdocsdrf.tex|, |cdocsfn1.tex|, |cdocsfn2.tex|.
Then copy the file |childdoc.def| to an appropriate directory of your \LaTeX{}
distribution, e.g.\ \textit{texmf-root}|/tex/latex/childdoc|.
\end{itemize}

%%%%%%%%%%%%%%%%%%%%%%%%%%%%%%%%%%%%%%%%%%%%%%%%%%%%%%%%%%%%%%%%%%%%%%%%%%%%%%%%
\subsection{Related CTAN Packages}

There are several other packages which offer a similar functionality:
%
\begin{itemize}
\item
The packages
\href{http://ctan.org/pkg/docmute}{\textsf{docmute}},
\href{http://ctan.org/pkg/includex}{\textsf{includex}} and
\href{http://ctan.org/pkg/standalone}{\textsf{standalone}}
provide commands to include only the document body of
a child file thus allowing both files to be compiled individually.
\item
The packages \href{http://ctan.org/pkg/subdocs}{\textsf{subdocs}}
and \href{http://ctan.org/pkg/subfiles}{\textsf{subfiles}}
provide structures in which the main and child documents can be
encapsulated and allowing them to be compiled individually.
The inclusion mechanism is different from the conventional |\include|.
\item
The package \href{http://ctan.org/pkg/combine}{\textsf{combine}}
is an elaborate solution to combine several documents into one.
\end{itemize}
%
See also the CTAN topic \href{http://ctan.org/topic/subdocs}{\textsf{subdocs}}
for further related packages.
The present package differs from the above solutions in that
a document structure constructed with the conventional |\include| mechanism
just needs two extra commands at the top of every file
such that all constituent files can be compiled individually.

%%%%%%%%%%%%%%%%%%%%%%%%%%%%%%%%%%%%%%%%%%%%%%%%%%%%%%%%%%%%%%%%%%%%%%%%%%%%%%%%
%\subsection{Feature Suggestions}
%
%The following is a list of features which may be useful for future
%versions of this package:
%%
%\begin{itemize}
%\item
%\ldots
%\end{itemize}

%%%%%%%%%%%%%%%%%%%%%%%%%%%%%%%%%%%%%%%%%%%%%%%%%%%%%%%%%%%%%%%%%%%%%%%%%%%%%%%%
\subsection{Revision History}

%%%%%%%%%%%%%%%%%%%%%%%%%%%%%%%%%%%%%%%%
\paragraph{v2.0:} 2018/12/30

\begin{itemize}
\item
immediate forward processing
\item
added |\childdocby| mechanism
\item
manual restructured
\end{itemize}

%%%%%%%%%%%%%%%%%%%%%%%%%%%%%%%%%%%%%%%%
\paragraph{v1.6:} 2018/01/17

\begin{itemize}
\item
application for development of include files
\item
corrections to manual
\end{itemize}

%%%%%%%%%%%%%%%%%%%%%%%%%%%%%%%%%%%%%%%%
\paragraph{v1.5:} 2017/05/21

\begin{itemize}
\item
more complete structuring introduced
\item
|\childdocof| introduced
\item
|\childdoc| renamed to |\childdocmain|
\item
|\childredirect| renamed to |\childdocforward| and |\childdocforwardprefix|
and functionality expanded
\end{itemize}

%%%%%%%%%%%%%%%%%%%%%%%%%%%%%%%%%%%%%%%%
\paragraph{v1.0:} 2017/04/27

\begin{itemize}
\item
manual and install package
\item
first version published on CTAN
\end{itemize}

%%%%%%%%%%%%%%%%%%%%%%%%%%%%%%%%%%%%%%%%
\paragraph{v0.6:} 2017/04/26

\begin{itemize}
\item
redirection mechanism added
\end{itemize}

%%%%%%%%%%%%%%%%%%%%%%%%%%%%%%%%%%%%%%%%
\paragraph{v0.5:} 2017/04/26

\begin{itemize}
\item
functionality in definition file
\end{itemize}


%%%%%%%%%%%%%%%%%%%%%%%%%%%%%%%%%%%%%%%%%%%%%%%%%%%%%%%%%%%%%%%%%%%%%%%%%%%%%%%%
%%%%%%%%%%%%%%%%%%%%%%%%%%%%%%%%%%%%%%%%%%%%%%%%%%%%%%%%%%%%%%%%%%%%%%%%%%%%%%%%
%%%%%%%%%%%%%%%%%%%%%%%%%%%%%%%%%%%%%%%%%%%%%%%%%%%%%%%%%%%%%%%%%%%%%%%%%%%%%%%%
\appendix

\settowidth\MacroIndent{\rmfamily\scriptsize 000\ }

 \DocInput{childdoc.dtx}

\end{document}
%</driver>
% \fi
%
% %%%%%%%%%%%%%%%%%%%%%%%%%%%%%%%%%%%%%%%%%%%%%%%%%%%%%%%%%%%%%%%%%%%%%%%%%%%%%%
% %%%%%%%%%%%%%%%%%%%%%%%%%%%%%%%%%%%%%%%%%%%%%%%%%%%%%%%%%%%%%%%%%%%%%%%%%%%%%%
% \section{Sample}
%\iffalse
%<*samplemain>
%\fi
%
% The following presents a sample document
% with two chapters, two parts, a title page,
% a compile flag as well as three forwarding files to set the flag.
% It consists of eight |.tex| files:
% \begin{center}
% \begin{tabular}{ll}
% |cdocsamp.tex|&main file\\
% |cdocsch1.tex|&include file for chapter 1\\
% |cdocsch2.tex|&include file for chapter 2\\
% |cdocspt3.tex|&include file for part 3\\
% |cdocspt4.tex|&include file for part 4\\
% |cdocsdrf.tex|&forwarding file for main file in draft mode\\
% |cdocsfi1.tex|&forwarding file for final version of chapter 1\\
% |cdocsfi2.tex|&forwarding file for final version of chapter 2\\
% \end{tabular}
% \end{center}
% Each of the eight files can be compiled directly by the \LaTeX{} compiler.
%
% %%%%%%%%%%%%%%%%%%%%%%%%%%%%%%%%%%%%%%
% \paragraph{Main File.}
%
% The main file is called |cdocsamp.tex|.
%
% Load the \textsf{childdoc} definitions and
% declare the filename for the main document:
%    \begin{macrocode}
\input{childdoc.def}
\childdocmain{}
%    \end{macrocode}

% Optional override for |\version| flag:
%    \begin{macrocode}
%%\ifchilddoc\else\providecommand{\version}{draft}\fi
%    \end{macrocode}

% Define the default values for the |\version| flag
% (|final| for the main file and |draft| for childs):
%    \begin{macrocode}
\ifchilddoc
\providecommand{\version}{draft}
\else
\providecommand{\version}{final}
\fi
%    \end{macrocode}

% Load the standard document class:
%    \begin{macrocode}
\documentclass[12pt]{article}
%    \end{macrocode}

% Start the document body:
%    \begin{macrocode}
\begin{document}
%    \end{macrocode}

% Declare a title page.
% Print title, part of document being processed and version flag:
%    \begin{macrocode}
\addtocounter{page}{-1}
\begin{center}
{\LARGE\bfseries{}childdoc example\par}
\vspace{1cm}
\ifchilddoc
\ifchilddocmanual part\else chapter\fi:
`\childdocname' of `\childdocjob'\par
\else
main document: `\childdocjob'\par
\fi
version: \version\par
\end{center}
\newpage
%    \end{macrocode}

% Manually include selected file,
% otherwise process as usual:
%    \begin{macrocode}
\ifchilddocmanual
\section*{part `\childdocname'}
\input{\childdocname}
\else
%    \end{macrocode}

% Include the two chapters:
%    \begin{macrocode}
\include{cdocsch1}
\include{cdocsch2}
%    \end{macrocode}

% Include the two parts unless only chapters should be displayed:
%    \begin{macrocode}
\ifchilddoc\else
\section{part three}
\input{cdocspt3}
\section{part four}
\input{cdocspt4}
\fi
%    \end{macrocode}

% Process as usual until here:
%    \begin{macrocode}
\fi
%    \end{macrocode}

% End of document body:
%    \begin{macrocode}
\end{document}
%    \end{macrocode}
%\iffalse
%</samplemain>
%\fi
%
% %%%%%%%%%%%%%%%%%%%%%%%%%%%%%%%%%%%%%%
% \paragraph{Chapter Include Files.}
%
% The include files are called |cdocsch1.tex| and |cdocsch2.tex|.
%
%\iffalse
%<*samplechap1|samplechap2>
%\fi

% Optional override for |\version| flag:
%    \begin{macrocode}
%%\providecommand{\version}{final}
%    \end{macrocode}

% Include the main document:
%    \begin{macrocode}
\input{childdoc.def}
\childdocof{cdocsamp}
%    \end{macrocode}

%\iffalse
%</samplechap1|samplechap2>
%\fi
%
%\iffalse
%<*samplechap1>
%\fi
% Some text for chapter 1:
%    \begin{macrocode}
\section{one}
some text in chapter one
%    \end{macrocode}

%\iffalse
%</samplechap1>
%\fi
% Some text for chapter 2:
%\iffalse
%<*samplechap2>
%\fi
%    \begin{macrocode}
\section{two}
more text in chapter two
%    \end{macrocode}

%\iffalse
%</samplechap2>
%\fi
%
% %%%%%%%%%%%%%%%%%%%%%%%%%%%%%%%%%%%%%%
% \paragraph{Part Include Files.}
%
% The include files are called |cdocspt3.tex| and |cdocspt4.tex|.
%
%\iffalse
%<*samplepart3|samplepart4>
%\fi

% Optional override for |\version| flag:
%    \begin{macrocode}
%%\providecommand{\version}{final}
%    \end{macrocode}

% Include the main document:
%    \begin{macrocode}
\input{childdoc.def}
\childdocby{cdocsamp}
%    \end{macrocode}

%\iffalse
%</samplepart3|samplepart4>
%\fi
%
%\iffalse
%<*samplepart3>
%\fi
% Some text for part 3:
%    \begin{macrocode}
some text in part three
%    \end{macrocode}

%\iffalse
%</samplepart3>
%\fi
% Some text for part 4:
%\iffalse
%<*samplepart4>
%\fi
%    \begin{macrocode}
more text in part four
%    \end{macrocode}

%\iffalse
%</samplepart4>
%\fi
%
% %%%%%%%%%%%%%%%%%%%%%%%%%%%%%%%%%%%%%%
% \paragraph{Forwarding for a Complete Draft.}
%
% The following forwarding file |cdocsdrf.tex|
% compiles the main document in draft mode:
%\iffalse
%<*sampledraft>
%\fi
%    \begin{macrocode}
\def\version{draft}
\input{childdoc.def}
\childdocforward{cdocsamp}
%    \end{macrocode}

%\iffalse
%</sampledraft>
%\fi
%
% %%%%%%%%%%%%%%%%%%%%%%%%%%%%%%%%%%%%%%
% \paragraph{Forwarding for Final Version of the Chapters.}
%
% The following forwarding files |cdocsfn1.tex| and |cdocsfn2.tex|
% (with identical content)
% compile the final versions of the child documents
% |cdocsch1.tex| and |cdocsch2.tex|, respectively:
%\iffalse
%<*samplefinal>
%\fi
%    \begin{macrocode}
\def\version{final}
\input{childdoc.def}
\childdocforwardprefix[cdocsamp]{cdocsfn}{cdocsch}
%    \end{macrocode}

%\iffalse
%</samplefinal>
%\fi
%
% %%%%%%%%%%%%%%%%%%%%%%%%%%%%%%%%%%%%%%
% \paragraph{Command Line Processing.}
%
% The following three command lines generate the output files
% |cdocscld|, |cdocscl1| and |cdocscl2|
% which should be identical to
% |cdocsdrf|, |cdocsch1| and |cdocsfn2|, respectively:
% \begin{center}
% \begin{tabular}{l}
% |latex -jobname cdocscld \|\\
% |  "\def\version{draft}\input{childdoc.def}\childdocforward{cdocsamp}"|\\
% |latex -jobname cdocscl1 \|\\
% |  "\input{childdoc.def}\childdocforward[cdocsamp]{cdocsch1}"|\\
% |latex -jobname cdocscl2 \|\\
% |  "\def\version{final}\input{childdoc.def}\childdocforward{cdocsch2}"|
% \end{tabular}
% \end{center}
% Note that the trailing backslash on each first line
% merely continues the input to the second line
% (for convenient cut ant paste).
% Furthermore, the command |latex| can be replaced by any
% of its alternative versions such as |pdflatex|.
%
% %%%%%%%%%%%%%%%%%%%%%%%%%%%%%%%%%%%%%%%%%%%%%%%%%%%%%%%%%%%%%%%%%%%%%%%%%%%%%%
% %%%%%%%%%%%%%%%%%%%%%%%%%%%%%%%%%%%%%%%%%%%%%%%%%%%%%%%%%%%%%%%%%%%%%%%%%%%%%%
% \section{Implementation}
%\iffalse
%<*package>
%\fi
%
% This section describes the definitions file |childdoc.def|.

% The definitions cannot be loaded using |\usepackage| or |\RequirePackage|
% which has a mechanism to prevent loading a style file more than once.
% When loading the definitions by means of |\input|
% multiple instances have to be prevented manually:
%\iffalse
%This code needs to be before the `\ProvidesFile' directive
%which is defined at the beginning of this file.
%Therefore it is also placed there and commented out here.
%</package>
%<*discard>
%\fi
%    \begin{macrocode}
\ifdefined\childdocmain\endinput\fi
%    \end{macrocode}
%\iffalse
%</discard>
%<*package>
%\fi
%
% \macro{\ifchilddoc}
% \macro{\ifchilddocmanual}
% The conditional |\ifchilddoc| tells whether a
% child (true) or main (false) document is being compiled.
% The conditional |\ifchilddocmanual| tells whether
% the |\includeonly| mechanism is used (false) or
% the selection of child files must be performed manually (true).
% The definitions initialise to false:
%    \begin{macrocode}
\newif\ifchilddoc
\newif\ifchilddocmanual
%    \end{macrocode}

% \macro{\childdocname}
% \macro{\childdocjob}
% The macro |\childdocname| stores the name of the main document
% to be compiled. The macro |\childdocjob| stores the name of
% the document on which the \LaTeX{} compiler was originally invoked.
% The content of |\jobname| cannot be compared
% to filenames specified in the source due to different catcodes.
% The following code rescans |\jobname|, stores the result
% in |\childdocname| and saves a copy in |\childdocjob|:
%    \begin{macrocode}
\edef\childdocname{\scantokens\expandafter{\jobname\noexpand}}
\let\childdocjob\childdocname
%    \end{macrocode}

% \macro{\childdocdisable}
% The macro |\childdocdisable| prevents the main file
% from being processed more than once.
% At this stage, the main document command |\childdocmain|
% is assumed to be called once again where it should do nothing.
% Any subsequent call to it should prevent
% a secondary processing of the main document
% It overwrites the forwarding commands
% |\childdocof| and |\childdocforward|
% with empty macros to prevent further inclusions of the main document:
%    \begin{macrocode}
\newcommand{\childdocdisable}
{
  \renewcommand{\childdocmain}[1]{\renewcommand{\childdocmain}[1]{\endinput}}
  \renewcommand{\childdocof}[1]{}
  \renewcommand{\childdocby}[2][]{}
  \renewcommand{\childdocforward}[2][]{}
  \renewcommand{\childdocdisable}{}
}
%    \end{macrocode}

% \macro{\childdocmain}
% The macro |\childdocmain| is to be called at the top of the main file
% with nothing or the main filename (without extension) as argument.
% First, it breaks loops.
% If the argument is not empty and does not match |\childdocname|
% (which is set by the first inclusion of |childdoc.def|),
% |\ifchilddoc| is set to true, |\includeonly| is applied to the child file
% and |\jobname| is set to the main file
% (for proper handling of |.aux| files):
%    \begin{macrocode}
\newcommand{\childdocmain}[1]
{
  \childdocdisable\childdocmain{}
  \if?#1?\else
    \begingroup
      \def\childdoctmp{#1}
      \ifx\childdoctmp\childdocname
        \def\childdoctmp{}
      \else
        \def\childdoctmp
        {
          \childdoctrue
          \includeonly{\childdocname}
          \def\childdocjob{#1}
          \def\jobname{#1}
        }
      \fi
      \expandafter
    \endgroup
    \childdoctmp
  \fi
}
%    \end{macrocode}

% \macro{\childdocof}
% The command |\childdocof| redirects
% compilation to the main file |#1|.
%    \begin{macrocode}
\newcommand{\childdocof}[1]
{
  \childdocdisable
  \childdoctrue
  \includeonly{\childdocname}
  \def\jobname{#1}
  \def\childdocjob{#1}
  \input{#1}
}
%    \end{macrocode}

% \macro{\childdocby}
% The command |\childdocby| ....
%    \begin{macrocode}
\newcommand{\childdocby}[2][]
{
  \childdocdisable
  \childdoctrue
  \childdocmanualtrue
  \if?#1?\else
    \def\jobname{#2}
  \fi
  \def\childdocjob{#2}
  \input{#2}
  \endinput
}
%    \end{macrocode}

% \macro{\childdocforward}
% The command |\childdocforward| redirects
% compilation to the main file or
% (if the optional argument is given) a child file.
% Parameters are set as if the main file
% or a child file starting with |\childdocof| was compiled.
% Then compilation is handed over to the main file:
%    \begin{macrocode}
\newcommand{\childdocforward}[2][]
{
  \begingroup
    \if?#1?
      \def\childdoctmp
      {
        \def\childdocname{#2}
        \def\childdocjob{#2}
        \def\jobname{#2}
        \input{#2}
        \endinput
      }
    \else
      \def\childdoctmp
      {
        \childdocdisable
        \def\childdocname{#2}
        \childdoctrue
        \includeonly{#2}
        \def\childdocjob{#1}
        \def\jobname{#1}
        \input{#1}
        \endinput
      }
    \fi
    \expandafter
  \endgroup
  \childdoctmp
}
%    \end{macrocode}

% \macro{\childdocforwardprefix}
% The command |\childdocforwardprefix| redirects
% compilation to the main or a child file by means of a pattern.
% The prefix |#1| in the current filename is replaced by |#2|
% and the suffix of the current filename is kept
% (it is assumed that the filename does not contain the substring `|~~~|'
% which is used as a delimiter).
% Compilation is handed over to the new file by |\childdocforward|:
%    \begin{macrocode}
\newcommand{\childdocforwardprefix}[3][]
{
  \begingroup
    \def\childdocextract #2##1~~~{\def\childdoctmp{\childdocforward[#1]{#3##1}}}
    \expandafter\childdocextract\childdocname~~~
    \expandafter
  \endgroup
  \childdoctmp
}
%    \end{macrocode}

% \macro{\childdoc}
% The deprecated macro |\childdoc| is a legacy version of |\childdocmain|:
%    \begin{macrocode}
\newcommand{\childdoc}{\childdocmain}
%    \end{macrocode}

% \macro{\childdocredirect}
% The deprecated macro |\childdocredirect| is a legacy version
% of |\childdocforward| and |\childdocforwardprefix|:
%    \begin{macrocode}
\newcommand{\childdocredirect}[2][]
{
  \begingroup
    \if?#1?
      \def\childdoctmp{\childdocforward{#2}}
    \else
      \def\childdoctmp{\childdocforwardprefix{#1}{#2}}
    \fi
    \expandafter
  \endgroup
  \childdoctmp
}
%    \end{macrocode}

%\iffalse
%</package>
%\fi
%
\endinput
|
and perform the replacements as outlined below.
Instead of |\childdocmain{|\textit{main}|}| add the following code
to the top of the main file:
%
\begin{center}
\begin{tabular}{l}
|\||ifdefined\childdocname\endinput\||fi\newif\ifchilddoc|\\
|\edef\childdocname{\scantokens\expandafter{\jobname\noexpand}}|\\
|\def\childdocmain{|\textit{main}|}\||ifx\childdocmain\childdocname\||else|\\
|\childdoctrue\includeonly{\childdocname}\let\jobname\childdocmain\||fi|\\
\end{tabular}
\end{center}
%
Instead of |\childdocof{|\textit{main}|}| just include the main file
at the top of each child file:
%
\begin{center}
|\input{|\textit{main}|}|
\end{center}
%
A simple redirection |\childdocforward{|\textit{dest}|}| is achieved by:
%
\begin{center}
|\def\jobname{|\textit{dest}|}\input{\jobname}|
\end{center}
%
The redirection with prefix
|\childdocforwardprefix[|\textit{prefix}|]{|\textit{dest}|}|
is accomplished by:
%
\begin{center}
\begin{tabular}{l}
|{\edef\jobname{\scantokens\expandafter{\jobname\noexpand}}|\\
|\def\redirectjob |\textit{prefix}|#1~~~{\gdef\jobname{|\textit{dest}|#1}}|\\
|\expandafter\redirectjob\jobname~~~}\input{\jobname}|
\end{tabular}
\end{center}

In an alternative approach,
child documents can be compiled by a specific command line
without additional code or specific definitions:
%
\begin{center}
|... -jobname "|\textit{target}|" "|[\textit{flags}]%
|\includeonly{|\textit{dest}|}\input{|\textit{main}|}"|
\end{center}
%

%%%%%%%%%%%%%%%%%%%%%%%%%%%%%%%%%%%%%%%%%%%%%%%%%%%%%%%%%%%%%%%%%%%%%%%%%%%%%%%%
%%%%%%%%%%%%%%%%%%%%%%%%%%%%%%%%%%%%%%%%%%%%%%%%%%%%%%%%%%%%%%%%%%%%%%%%%%%%%%%%
\section{Information}

%%%%%%%%%%%%%%%%%%%%%%%%%%%%%%%%%%%%%%%%%%%%%%%%%%%%%%%%%%%%%%%%%%%%%%%%%%%%%%%%
\subsection{Copyright}

Copyright \copyright{} 2017--2018 Niklas Beisert

This work may be distributed and/or modified under the
conditions of the \LaTeX{} Project Public License, either version 1.3
of this license or (at your option) any later version.
The latest version of this license is in
  \url{http://www.latex-project.org/lppl.txt}
and version 1.3 or later is part of all distributions of \LaTeX{}
version 2005/12/01 or later.

This work has the LPPL maintenance status `maintained'.

The Current Maintainer of this work is Niklas Beisert.

This work consists of the files |README.txt|, |childdoc.ins| and |childdoc.dtx|
as well as the derived files |childdoc.def|, |cdocsamp.tex|
with |cdocsch1.tex|, |cdocsch2.tex|, |cdocspt3.tex|, |cdocspt4.tex|,
|cdocsdrf.tex|, |cdocsfn1.tex|, |cdocsfn2.tex|
as well as |childdoc.pdf|.

%%%%%%%%%%%%%%%%%%%%%%%%%%%%%%%%%%%%%%%%%%%%%%%%%%%%%%%%%%%%%%%%%%%%%%%%%%%%%%%%
\subsection{Files and Installation}

The package consists of the files:
%
\begin{center}
\begin{tabular}{ll}
    |README.txt|   & readme file \\
    |childdoc.ins| & installation file \\
    |childdoc.dtx| & source file \\
    |childdoc.def| & definition file \\
    |cdocsamp.tex| & sample main file \\
    |cdocsch1.tex| & sample include file \\
    |cdocsch2.tex| & sample include file \\
    |cdocspt3.tex| & sample part file \\
    |cdocspt4.tex| & sample part file \\
    |cdocsdrf.tex| & sample redirection file \\
    |cdocsfn1.tex| & sample redirection file \\
    |cdocsfn2.tex| & sample redirection file \\
    |childdoc.pdf| & manual
\end{tabular}
\end{center}
%
The distribution consists of the files
|README.txt|, |childdoc.ins| and |childdoc.dtx|.
%
\begin{itemize}
\item
Run (pdf)\LaTeX{} on |childdoc.dtx|
to compile the manual |childdoc.pdf| (this file).
\item
Run \LaTeX{} on |childdoc.ins| to create the definitions file |childdoc.def|
and the sample |cdocsamp.tex| with include files
|cdocsch1.tex|, |cdocsch2.tex|, |cdocspt3.tex|, |cdocspt4.tex|,
|cdocsdrf.tex|, |cdocsfn1.tex|, |cdocsfn2.tex|.
Then copy the file |childdoc.def| to an appropriate directory of your \LaTeX{}
distribution, e.g.\ \textit{texmf-root}|/tex/latex/childdoc|.
\end{itemize}

%%%%%%%%%%%%%%%%%%%%%%%%%%%%%%%%%%%%%%%%%%%%%%%%%%%%%%%%%%%%%%%%%%%%%%%%%%%%%%%%
\subsection{Related CTAN Packages}

There are several other packages which offer a similar functionality:
%
\begin{itemize}
\item
The packages
\href{http://ctan.org/pkg/docmute}{\textsf{docmute}},
\href{http://ctan.org/pkg/includex}{\textsf{includex}} and
\href{http://ctan.org/pkg/standalone}{\textsf{standalone}}
provide commands to include only the document body of
a child file thus allowing both files to be compiled individually.
\item
The packages \href{http://ctan.org/pkg/subdocs}{\textsf{subdocs}}
and \href{http://ctan.org/pkg/subfiles}{\textsf{subfiles}}
provide structures in which the main and child documents can be
encapsulated and allowing them to be compiled individually.
The inclusion mechanism is different from the conventional |\include|.
\item
The package \href{http://ctan.org/pkg/combine}{\textsf{combine}}
is an elaborate solution to combine several documents into one.
\end{itemize}
%
See also the CTAN topic \href{http://ctan.org/topic/subdocs}{\textsf{subdocs}}
for further related packages.
The present package differs from the above solutions in that
a document structure constructed with the conventional |\include| mechanism
just needs two extra commands at the top of every file
such that all constituent files can be compiled individually.

%%%%%%%%%%%%%%%%%%%%%%%%%%%%%%%%%%%%%%%%%%%%%%%%%%%%%%%%%%%%%%%%%%%%%%%%%%%%%%%%
%\subsection{Feature Suggestions}
%
%The following is a list of features which may be useful for future
%versions of this package:
%%
%\begin{itemize}
%\item
%\ldots
%\end{itemize}

%%%%%%%%%%%%%%%%%%%%%%%%%%%%%%%%%%%%%%%%%%%%%%%%%%%%%%%%%%%%%%%%%%%%%%%%%%%%%%%%
\subsection{Revision History}

%%%%%%%%%%%%%%%%%%%%%%%%%%%%%%%%%%%%%%%%
\paragraph{v2.0:} 2018/12/30

\begin{itemize}
\item
immediate forward processing
\item
added |\childdocby| mechanism
\item
manual restructured
\end{itemize}

%%%%%%%%%%%%%%%%%%%%%%%%%%%%%%%%%%%%%%%%
\paragraph{v1.6:} 2018/01/17

\begin{itemize}
\item
application for development of include files
\item
corrections to manual
\end{itemize}

%%%%%%%%%%%%%%%%%%%%%%%%%%%%%%%%%%%%%%%%
\paragraph{v1.5:} 2017/05/21

\begin{itemize}
\item
more complete structuring introduced
\item
|\childdocof| introduced
\item
|\childdoc| renamed to |\childdocmain|
\item
|\childredirect| renamed to |\childdocforward| and |\childdocforwardprefix|
and functionality expanded
\end{itemize}

%%%%%%%%%%%%%%%%%%%%%%%%%%%%%%%%%%%%%%%%
\paragraph{v1.0:} 2017/04/27

\begin{itemize}
\item
manual and install package
\item
first version published on CTAN
\end{itemize}

%%%%%%%%%%%%%%%%%%%%%%%%%%%%%%%%%%%%%%%%
\paragraph{v0.6:} 2017/04/26

\begin{itemize}
\item
redirection mechanism added
\end{itemize}

%%%%%%%%%%%%%%%%%%%%%%%%%%%%%%%%%%%%%%%%
\paragraph{v0.5:} 2017/04/26

\begin{itemize}
\item
functionality in definition file
\end{itemize}


%%%%%%%%%%%%%%%%%%%%%%%%%%%%%%%%%%%%%%%%%%%%%%%%%%%%%%%%%%%%%%%%%%%%%%%%%%%%%%%%
%%%%%%%%%%%%%%%%%%%%%%%%%%%%%%%%%%%%%%%%%%%%%%%%%%%%%%%%%%%%%%%%%%%%%%%%%%%%%%%%
%%%%%%%%%%%%%%%%%%%%%%%%%%%%%%%%%%%%%%%%%%%%%%%%%%%%%%%%%%%%%%%%%%%%%%%%%%%%%%%%
\appendix

\settowidth\MacroIndent{\rmfamily\scriptsize 000\ }

 \DocInput{childdoc.dtx}

\end{document}
%</driver>
% \fi
%
% %%%%%%%%%%%%%%%%%%%%%%%%%%%%%%%%%%%%%%%%%%%%%%%%%%%%%%%%%%%%%%%%%%%%%%%%%%%%%%
% %%%%%%%%%%%%%%%%%%%%%%%%%%%%%%%%%%%%%%%%%%%%%%%%%%%%%%%%%%%%%%%%%%%%%%%%%%%%%%
% \section{Sample}
%\iffalse
%<*samplemain>
%\fi
%
% The following presents a sample document
% with two chapters, two parts, a title page,
% a compile flag as well as three forwarding files to set the flag.
% It consists of eight |.tex| files:
% \begin{center}
% \begin{tabular}{ll}
% |cdocsamp.tex|&main file\\
% |cdocsch1.tex|&include file for chapter 1\\
% |cdocsch2.tex|&include file for chapter 2\\
% |cdocspt3.tex|&include file for part 3\\
% |cdocspt4.tex|&include file for part 4\\
% |cdocsdrf.tex|&forwarding file for main file in draft mode\\
% |cdocsfi1.tex|&forwarding file for final version of chapter 1\\
% |cdocsfi2.tex|&forwarding file for final version of chapter 2\\
% \end{tabular}
% \end{center}
% Each of the eight files can be compiled directly by the \LaTeX{} compiler.
%
% %%%%%%%%%%%%%%%%%%%%%%%%%%%%%%%%%%%%%%
% \paragraph{Main File.}
%
% The main file is called |cdocsamp.tex|.
%
% Load the \textsf{childdoc} definitions and
% declare the filename for the main document:
%    \begin{macrocode}
% \iffalse
%
% childdoc.dtx Copyright (C) 2017-2018 Niklas Beisert
%
% This work may be distributed and/or modified under the
% conditions of the LaTeX Project Public License, either version 1.3
% of this license or (at your option) any later version.
% The latest version of this license is in
%   http://www.latex-project.org/lppl.txt
% and version 1.3 or later is part of all distributions of LaTeX
% version 2005/12/01 or later.
%
% This work has the LPPL maintenance status `maintained'.
%
% The Current Maintainer of this work is Niklas Beisert.
%
% This work consists of the files childdoc.dtx and childdoc.ins
% and the derived files childdoc.def and cdocsamp.tex with
% cdocsch1.tex, cdocsch2.tex, cdocsdrf.tex, cdocsfn1.tex, cdocsfn2.tex.
%
%<package>\ifdefined\childdocmain\endinput\fi
%<package>\ProvidesFile{childdoc.def}[2018/12/30 v2.0 child document driver]
%<samplemain>\ProvidesFile{cdocsamp.tex}[2018/12/30 v2.0 sample for childdoc]
%<*driver>
%\ProvidesFile{childdoc.drv}[2018/12/30 v2.0 childdoc reference manual file]
\PassOptionsToClass{10pt,a4paper}{article}
\documentclass{ltxdoc}

\usepackage[margin=35mm]{geometry}
\usepackage{hyperref}
\usepackage{hyperxmp}
\usepackage[usenames]{color}

\hypersetup{colorlinks=true}
\hypersetup{pdfstartview=FitH}
\hypersetup{pdfpagemode=UseNone}
\hypersetup{pdfsource={}}
\hypersetup{pdflang={en-UK}}
\hypersetup{pdfcopyright={Copyright 2017-2018 Niklas Beisert.
  This work may be distributed and/or modified under the
  conditions of the LaTeX Project Public License, either version 1.3
  of this license or (at your option) any later version.}}
\hypersetup{pdflicenseurl={http://www.latex-project.org/lppl.txt}}
\hypersetup{pdfcontactaddress={ETH Zurich, ITP, HIT K,
  Wolfgang-Pauli-Strasse 27}}
\hypersetup{pdfcontactpostcode={8093}}
\hypersetup{pdfcontactcity={Zurich}}
\hypersetup{pdfcontactcountry={Switzerland}}
\hypersetup{pdfcontactemail={nbeisert@itp.phys.ethz.ch}}
\hypersetup{pdfcontacturl={http://people.phys.ethz.ch/\xmptilde nbeisert/}}

\newcommand{\secref}[1]{\hyperref[#1]{section \ref*{#1}}}

\parskip1ex
\parindent0pt
\let\olditemize\itemize
\def\itemize{\olditemize\parskip0pt}

\begin{document}

\title{The \textsf{childdoc} Package}
\hypersetup{pdftitle={The childdoc Package}}
\author{Niklas Beisert\\[2ex]
  Institut f\"ur Theoretische Physik\\
  Eidgen\"ossische Technische Hochschule Z\"urich\\
  Wolfgang-Pauli-Strasse 27, 8093 Z\"urich, Switzerland\\[1ex]
  \href{mailto:nbeisert@itp.phys.ethz.ch}
  {\texttt{nbeisert@itp.phys.ethz.ch}}}
\hypersetup{pdfauthor={Niklas Beisert}}
\hypersetup{pdfsubject={Manual for the LaTeX2e Package childdoc}}
\date{30 December 2018, \textsf{v2.0}}
\maketitle

\begin{abstract}\noindent
\textsf{childdoc} is a \LaTeXe{} package
that enables the direct compilation
of document sections included by |\include|
to individual files.
\end{abstract}

\begingroup
\parskip0ex
\tableofcontents
\endgroup

%%%%%%%%%%%%%%%%%%%%%%%%%%%%%%%%%%%%%%%%%%%%%%%%%%%%%%%%%%%%%%%%%%%%%%%%%%%%%%%%
%%%%%%%%%%%%%%%%%%%%%%%%%%%%%%%%%%%%%%%%%%%%%%%%%%%%%%%%%%%%%%%%%%%%%%%%%%%%%%%%
\section{Introduction}

\LaTeX{} provides a mechanism to structure a large document (such as a book)
into a main file and several child files (containing the chapters)
using the |\include| command.
This mechanism is beneficial for documents
which span hundreds of pages in order to
make the source file(s) more manageable.
Moreover, compilation can be restricted to
selected child files by means of the |\includeonly| command.
The latter feature can be used to reduce the compilation time while editing
(this was significantly more useful in the earlier days of \LaTeX{})
or to generate a smaller document which is easier to navigate.
Another application of |\includeonly| is to generate
documents consisting of selected parts of the complete document.

However, there are a few drawbacks of the plain |\include| mechanism:
\begin{itemize}
\item
The child files cannot be compiled on their own,
they can only be compiled via the main file.
A naive editing environment
(such as a text editor with an option
to have the current file processed by \LaTeX)
may require one to switch to the main file before compiling;
attempting to compile the child file produces errors.
\item
The main file must be modified (each time)
to adjust the |\includeonly| command
to the present needs. This easily leaves the main file in a messy state.
\item
The generated document will always carry the filename
of the main document. This is inconvenient if
several child files are to be compiled and
to be kept for distribution.
\end{itemize}

The present package provides a simple interface
to make child files individually compilable by \LaTeX{}.
Compiling a child file then has the same effect as compiling
the main file with an |\includeonly| command
to select the appropriate child.
Moreover the generated document will carry the name of the child
rather than the main file.
This resolves all three above issues.

This feature is meant to make the editing of books,
thesis documents and lecture notes somewhat more convenient.
However, the package can also be used efficiently for
composing a series of documents (such as exercise sheets)
which are typically distributed individually.
It then assists the author in generating the individual documents
(potentially in different versions)
as well as a document containing the collected series.
Another application is in developing style files
or other kinds of included material
where compilation of the style file could redirect
to a sample or test file.

%%%%%%%%%%%%%%%%%%%%%%%%%%%%%%%%%%%%%%%%%%%%%%%%%%%%%%%%%%%%%%%%%%%%%%%%%%%%%%%%
%%%%%%%%%%%%%%%%%%%%%%%%%%%%%%%%%%%%%%%%%%%%%%%%%%%%%%%%%%%%%%%%%%%%%%%%%%%%%%%%
\section{Usage}

First of all, the package \textsf{childdoc} is \emph{not} a standard
\LaTeXe{} |.sty| style file! Therefore it needs to be invoked in
a non-standard way.

%%%%%%%%%%%%%%%%%%%%%%%%%%%%%%%%%%%%%%%%%%%%%%%%%%%%%%%%%%%%%%%%%%%%%%%%%%%%%%%%
\subsection{Included Files}
\label{sec:include}

%%%%%%%%%%%%%%%%%%%%%%%%%%%%%%%%%%%%%%%%
\DescribeMacro{\childdocmain}
To use the package, add the commands
\begin{center}
\begin{tabular}{l}
|\input{childdoc.def}|\\
|\childdocmain{}|\\
\end{tabular}
\end{center}
at the very top of the main \LaTeX{} file,
in particular \emph{before} the |\documentclass| statement!
The argument of |\childdocmain| should be left empty
(but it must be present).

%%%%%%%%%%%%%%%%%%%%%%%%%%%%%%%%%%%%%%%%
\DescribeMacro{\childdocof}
Furthermore, add the commands
\begin{center}
\begin{tabular}{l}
|\input{childdoc.def}|\\
|\childdocof{|\textit{main}|}|\\
\end{tabular}
\end{center}
at the top of every child file \textit{child}
which is included by |\include{|\textit{child}|}|
from within the main file
(or at least for those files to be compiled individually).
The argument \textit{main} must be the filename of the main file.

There are a couple of
considerations in setting up the main and child documents:

%%%%%%%%%%%%%%%%%%%%%%%%%%%%%%%%%%%%%%%%
\paragraph{Restrictions.}

Please note the following restrictions:
\begin{itemize}
\item
|\childdocmain| must be called with one argument \textit{main}
to ensure compatibility with earlier version of the package.
It must either be empty (|\childdocmain{}|)
or precisely match the filename of the main file in which it is specified.
See \secref{sec:detection} for further information.
\item
The filename \textit{main} must be specified without the |.tex| extension.
\item
The filename \textit{main} is case sensitive
(even in case-insensitive file systems)
due to internal string comparison.
\item
The argument \textit{main} should be fully expanded, it cannot be a macro.
\item
Subdirectories and special characters should be avoided in filenames.
\item
The command |\childdocmain{|\textit{main}|}| must be followed by a whitespace.
It should not be followed immediately by another command
or by a comment mark `|%|'.
This is because the \TeX{} parser reads the token immediately following
the argument of |\childdocmain| and puts it
at the beginning of every child section;
however, a white\-space is ignored.
\end{itemize}

%%%%%%%%%%%%%%%%%%%%%%%%%%%%%%%%%%%%%%%%
\paragraph{Content of Main File.}

It is advisable to place all content in the child files included by |\include|.
Any output contained in the main file will appear in all child documents
unless suppressed manually;
it cannot be suppressed automatically by the |\includeonly| directive
and thus should normally be avoided.
A method to include some content in the main file
by means of conditional processing is described in \secref{sec:conditional}.

%%%%%%%%%%%%%%%%%%%%%%%%%%%%%%%%%%%%%%%%
\paragraph{Page Numbering.}

When only a part of the document is compiled,
the appropriate numbering of pages
(as well as other status parameters)
is determined from the |.aux| files.
The latter contain information from previous passes.
However this information needs to propagate through
all intermediate child documents.
Therefore the page numbering in child documents may well
be inconsistent until the complete document is compiled at least once.

A useful (if unconventional) way to always ensure a consistent
page numbering is to restart the numbering in each child document
and denote the pages by `\textit{child}|.|\textit{page}'
where \textit{child} represents the chapter/section number of the child file.
This can be achieved by the command
|\numberwithin{page}{|\textit{child}|}|
of the \textsf{amsmath} package
where \textit{child} can be |chapter| or |section|
depending on the chosen structuring.
Alternatively, one can modify the macro |\thepage| appropriately
and reset the counter |page| at the start of each child file.

%%%%%%%%%%%%%%%%%%%%%%%%%%%%%%%%%%%%%%%%%%%%%%%%%%%%%%%%%%%%%%%%%%%%%%%%%%%%%%%%
\subsection{Conditional Processing}
\label{sec:conditional}

The package provides a mechanism to compile different versions
of a document. To customise the versions further some conditional processing
can come in handy to distinguish which version is being compiled.
The package provides two macros to describe the compilation context:

%%%%%%%%%%%%%%%%%%%%%%%%%%%%%%%%%%%%%%%%
\DescribeMacro{\ifchilddoc}
The conditional |\ifchilddoc| distinguishes between the compilation of
child documents and the main document:
%
\begin{center}
|\ifchilddoc |\textit{child-code}| |[|\||else |\textit{main-code}]| \||fi|
\end{center}

%%%%%%%%%%%%%%%%%%%%%%%%%%%%%%%%%%%%%%%%
\DescribeMacro{\childdocname}
\DescribeMacro{\childdocjob}
The macro |\childdocname| contains the filename (without extension)
of the main or child file being processed.
Note that |\childdocjob| will always contain the name of the main file.

%%%%%%%%%%%%%%%%%%%%%%%%%%%%%%%%%%%%%%%%
\paragraph{Title Page.}

Conditional processing can be used to include a title or banner page
in the main document when proper precautions are taken.
Importantly, the code in the main file should ensure that the page counter
(as well as other status parameters which are stored in the |.aux| files)
takes the same value after the conditional processing.
Otherwise the page numbers may take divergent values
depending on which part is compiled.

For example, a title page could be declared by:
%
\begin{center}
\begin{tabular}{l}
|\ifchilddoc\||else|\\
|\addtocounter{page}{-1}|\\
\textit{code for title page}\\
|\newpage|\\
|\||fi|
\end{tabular}
\end{center}
%
A banner page for the child documents can be generated by:
%
\begin{center}
\begin{tabular}{l}
|\ifchilddoc|\\
|\addtocounter{page}{-1}|\\
\textit{code for banner page}\\
|\newpage|\\
|\||fi|
\end{tabular}
\end{center}
%
Here one could write a message such as:
\begin{center}
|This is the part \childdocname{} of \childdocjob{}.|
\end{center}

%%%%%%%%%%%%%%%%%%%%%%%%%%%%%%%%%%%%%%%%%%%%%%%%%%%%%%%%%%%%%%%%%%%%%%%%%%%%%%%%
\subsection{Flags}
\label{sec:flags}

The package makes it easy to generate different versions
of the main or child documents.
To this end compilation flags can be defined
and assigned different default values.
They will be particularly useful in conjunction
with the forwarding mechanism described in \secref{sec:forward}.

For example, it may be useful to have a flag |\version|
which can be set to |draft| or |final|.
The document source will contain some conditional code
depending on the value of |\version|.
Suppose further, the flag should default to |final| for the main file
and to |draft| for child files
which is a natural assignment for editing the document.
This is achieved by placing the following code
in the preamble of the main document
(below the |\childdocmain| directive):
%
\begin{center}
\begin{tabular}{l}
|\ifchilddoc|\\
|\providecommand{\version}{draft}|\\
|\||else|\\
|\providecommand{\version}{final}|\\
|\||fi|
\end{tabular}
\end{center}
%
The definition by |\providecommand| makes sure
that previous definitions are not overwritten.
Further statements |\providecommand{\version}{...}|
can thus be added before the above code to override it.

For the main file, one might add a line
(between |\childdocmain| and the above block)
%
\begin{center}
|%\ifchilddoc\||else\providecommand{\version}{draft}\||fi|
\end{center}
%
which can be uncommented to produce a draft version.
Likewise one can add a line to the very top of a child file
(above the |\childdocof{|\textit{main}|}| directive)
%
\begin{center}
|%\providecommand{\version}{final}|
\end{center}
%
which can be uncommented to produce the final version of this child document.

%%%%%%%%%%%%%%%%%%%%%%%%%%%%%%%%%%%%%%%%%%%%%%%%%%%%%%%%%%%%%%%%%%%%%%%%%%%%%%%%
\subsection{Forwarding}
\label{sec:forward}

Different versions of the main or child documents
using compilation flags as described in \secref{sec:flags}
can be (permanently) stored in different files
for convenient compilation, viewing and distribution.
To this end, the package defines a command
to pass on compilation to a different file:

%%%%%%%%%%%%%%%%%%%%%%%%%%%%%%%%%%%%%%%%
\DescribeMacro{\childdocforward}
The command |\childdocforward| redirects processing to
another source file:
%
\begin{center}
\begin{tabular}{l}
|\input{childdoc.def}|\\
|\childdocforward[|\textit{main}|]{|\textit{dest}|}|\\
\end{tabular}
\end{center}
%
The argument \textit{dest} is the destination file
(without extension).
It should be the main file or one of the child files.
Note that further \textsf{childdoc} directives
such as |\childdocof| and |\childdocforward|
in the indicated file will be processed in this form.
The optional argument \textit{main}
passes on directly to the main file \textit{main}
while pretending to compile the child \textit{dest}.
This form behaves as if \textit{dest}
issues |\childdocof{|\textit{main}|}| right away,
and no further \textsf{childdoc} directives will be processed.

%%%%%%%%%%%%%%%%%%%%%%%%%%%%%%%%%%%%%%%%
\DescribeMacro{\...prefix}
In the alternative form |\childdocforwardprefix|,
%
\begin{center}
\begin{tabular}{l}
|\input{childdoc.def}|\\
|\childdocforwardprefix[|\textit{main}|]{|\textit{prefix}|}{|\textit{dest}|}|
\end{tabular}
\end{center}
%
the destination file is determined by a pattern
depending on the current file:
To make this work, the current file must be called
`{\textit{prefix}\hspace{0.2em}\textit{suffix}}'
with \textit{prefix} matching precisely the argument.
Processing is then passed on to the file
`{\textit{dest}\hspace{0.2em}\textit{suffix}}'.
Surely, the same effect is achieved by
directly specifying the
argument `{\textit{dest}\hspace{0.2em}\textit{suffix}}'
in the first form.
However, that requires to set up a different file
for each child. With the alternative form of the command
all these files can have exactly the same content
which simplifies setting them up and maintaining them.

For example, the following file |draft.tex|
with a compilation flag |\version| as described in \secref{sec:flags}
compiles the main document as a draft:
%
\begin{center}
\begin{tabular}{l}
|\def\version{draft}|\\
|\input{childdoc.def}|\\
|\childdocforward{|\textit{main}|}|
\end{tabular}
\end{center}
%
Likewise, the following files |final|\textit{nn}|.tex|
compile the final version of the child document
|child|\textit{nn}|.tex|:
%
\begin{center}
\begin{tabular}{l}
|\def\version{final}|\\
|\input{childdoc.def}|\\
|\childdocforwardprefix{final}{child}|
\end{tabular}
\end{center}
%

Note that when several versions of a main file and/or of each child file
are to be generated, it may be convenient to set up a |Makefile| or
shell script to automatise the process.

%%%%%%%%%%%%%%%%%%%%%%%%%%%%%%%%%%%%%%%%%%%%%%%%%%%%%%%%%%%%%%%%%%%%%%%%%%%%%%%%
\subsection{Command Line Processing}
\label{sec:commandline}

The effect of redirection files can also be achieved by invoking
the \LaTeX{} compiler with a more elaborate command line.
Most conveniently this should be done as part
of a shell script or a |Makefile|.

When using \textsf{childdoc} in the main file, the following
command lines effectively perform a redirection
(note that depending on the shell being used,
backslashes may have to be doubled: `|\|' $\to$ `|\\|'):
%
\begin{center}
|... -jobname "|\textit{target}|" |\\|"|[\textit{flags}]%
|\input{childdoc.def}\childdocforward[|\textit{main}|]{|\textit{dest}|}"|
\end{center}
%
Here \textit{target} is the name of the output file,
\textit{main} is the name of the main file
and \textit{dest} is the name of the main or child file to be processed
(all filenames without extensions).
The optional argument \textit{main} can be omitted
if \textit{main} matches \textit{dest}.
Optionally, compilation \textit{flags} can be defined via |\def| commands.
This command line makes the \TeX{} engine believe
it is compiling the file \textit{target}
whose content is specified as the latter parameter.
The provided code then forwards the processing to
\textit{main} or \textit{dest} as described in \secref{sec:forward}.

%%%%%%%%%%%%%%%%%%%%%%%%%%%%%%%%%%%%%%%%%%%%%%%%%%%%%%%%%%%%%%%%%%%%%%%%%%%%%%%%
\subsection{Include by Input}
\label{sec:input}

Including child documents by |\include| has some restrictions by design.
Most notably, the content of a child document always occupies
its own set of pages; pages cannot be shared between child documents.
Usually, this behaviour makes perfect sense
because each child document contain an essential part of the document.
However, in some situations it may be desirable to compose
a document from a collection of parts
without having mandatory page breaks between then.
For this case, the package
provides a mechanism to include parts
by |\input| which can also be processed individually.
However, by construction this mechanism
requires manual handling of the content to be output.

%%%%%%%%%%%%%%%%%%%%%%%%%%%%%%%%%%%%%%%%
\DescribeMacro{\ifchilddocmanual}
The main file should be prepared as usual, see \secref{sec:include}.
However, the document body must make a distinction
between processing of an individual part and of the main document, e.g.:
%
\begin{center}
\begin{tabular}{l}
|\ifchilddocmanual|\\
|\input{\childdocname}|\\
|\||else|\\
\textit{document body with }|\input{|\textit{part}|}|\\
|\||fi|
\end{tabular}
\end{center}
%
The conditional |\ifchilddocmanual| is true whenever
a part to be included by |\input| is being compiled,
and the name of the part is stored in |\childdocname|.

%%%%%%%%%%%%%%%%%%%%%%%%%%%%%%%%%%%%%%%%
\DescribeMacro{\childdocby}
Each part to be included by |\input| should start with:
%
\begin{center}
\begin{tabular}{l}
|\input{childdoc.def}|\\
|\childdocby{|\textit{main}|}|\\
\end{tabular}
\end{center}
%
The directive |\childdocby| is similar to |\childdocof|
described in \secref{sec:include},
but the subsequent selection of content must be done manually.
To that end, both |\ifchilddoc| and |\ifchilddocmanual|
will be true upon processing of a part,
and the name of the part is stored in |\childdocname|.
Note that |\jobname| will be set to the filename of the current part
so that each part receives an individual |.aux| file
that does not interfere with the |.aux| file(s) of the main document.
This behaviour can be altered by the alternative form
|\childdocby[*]{|\textit{main}|}| (with a non-empty optional argument)
which uses the |.aux| file of the main document
by setting |\jobname| to \textit{main}.

%%%%%%%%%%%%%%%%%%%%%%%%%%%%%%%%%%%%%%%%%%%%%%%%%%%%%%%%%%%%%%%%%%%%%%%%%%%%%%%%
\subsection{Driver Development}
\label{sec:driver}

The \textsf{childdoc} mechanism can also be use for the development
of definition files such as \LaTeX{} styles or classes.
This case differs from the above setup with multiple parts
included by |\include| in that no |\includeonly| should be invoked.
This can be achieved by starting the include file
(before |\ProvidesPackage|) with:
%
\begin{center}
\begin{tabular}{l}
|\input{childdoc.def}|\\
|\childdocforward{|\textit{main}|}|\\
\end{tabular}
\end{center}
%
or alternatively with:
%
\begin{center}
\begin{tabular}{l}
|\input{childdoc.def}|\\
|\childdocby{|\textit{main}|}|\\
\end{tabular}
\end{center}
%
Both forms have slightly different effects as described above.
The main file is prepared as usual, see \secref{sec:include}.

%%%%%%%%%%%%%%%%%%%%%%%%%%%%%%%%%%%%%%%%%%%%%%%%%%%%%%%%%%%%%%%%%%%%%%%%%%%%%%%%
\subsection{Legacy Detection}
\label{sec:detection}

The directive |\childdocmain| in the main file can detect
whether the complete document or merely a child is to be compiled
even without using the directive |\childdocof|.
This method is deprecated because it is less robust
and there is no compelling reason to use it;
it is merely provided for backward compatibility
and it may be removed in future versions.

If the detection mechanism is to be used,
it is mandatory to correctly specify
the filename of the main file as the argument of |\childdocmain|:
%
\begin{center}
\begin{tabular}{l}
|\input{childdoc.def}|\\
|\childdocmain{|\textit{main}|}|\\
\end{tabular}
\end{center}
%
If |\jobname| does not match the argument \textit{main} of |\childdocmain|,
it is assumed that |\jobname| points to the child file to be compiled.
When using |\childdocmain| with the main file specified as argument,
it suffices to start a child file
with just |\input{|\textit{main}|}|
without loading of the package and using |\childdocof|.
If instead all processing is done
with the appropriate \textsf{childdoc} directives,
the argument of \textit{main} of |\childdocmain| can be empty.

An alternative version of the command line processing described
in \secref{sec:commandline} using the detection mechanism reads:
%
\begin{center}
|... -jobname "|\textit{target}|" "|[\textit{flags}]%
[|\def\jobname{|\textit{dest}|}|]|\input{|\textit{main}|}"|
\end{center}

%%%%%%%%%%%%%%%%%%%%%%%%%%%%%%%%%%%%%%%%%%%%%%%%%%%%%%%%%%%%%%%%%%%%%%%%%%%%%%%%
\subsection{Manual Code}
\label{sec:manual}

In case one cannot be certain whether the definitions file |childdoc.def|
is installed on the target \TeX{} distribution
and one prefers not to ship it,
it is conceivable to paste a few relevant commands into the sources.

To that end, drop all statements |\input{childdoc.def}|
and perform the replacements as outlined below.
Instead of |\childdocmain{|\textit{main}|}| add the following code
to the top of the main file:
%
\begin{center}
\begin{tabular}{l}
|\||ifdefined\childdocname\endinput\||fi\newif\ifchilddoc|\\
|\edef\childdocname{\scantokens\expandafter{\jobname\noexpand}}|\\
|\def\childdocmain{|\textit{main}|}\||ifx\childdocmain\childdocname\||else|\\
|\childdoctrue\includeonly{\childdocname}\let\jobname\childdocmain\||fi|\\
\end{tabular}
\end{center}
%
Instead of |\childdocof{|\textit{main}|}| just include the main file
at the top of each child file:
%
\begin{center}
|\input{|\textit{main}|}|
\end{center}
%
A simple redirection |\childdocforward{|\textit{dest}|}| is achieved by:
%
\begin{center}
|\def\jobname{|\textit{dest}|}\input{\jobname}|
\end{center}
%
The redirection with prefix
|\childdocforwardprefix[|\textit{prefix}|]{|\textit{dest}|}|
is accomplished by:
%
\begin{center}
\begin{tabular}{l}
|{\edef\jobname{\scantokens\expandafter{\jobname\noexpand}}|\\
|\def\redirectjob |\textit{prefix}|#1~~~{\gdef\jobname{|\textit{dest}|#1}}|\\
|\expandafter\redirectjob\jobname~~~}\input{\jobname}|
\end{tabular}
\end{center}

In an alternative approach,
child documents can be compiled by a specific command line
without additional code or specific definitions:
%
\begin{center}
|... -jobname "|\textit{target}|" "|[\textit{flags}]%
|\includeonly{|\textit{dest}|}\input{|\textit{main}|}"|
\end{center}
%

%%%%%%%%%%%%%%%%%%%%%%%%%%%%%%%%%%%%%%%%%%%%%%%%%%%%%%%%%%%%%%%%%%%%%%%%%%%%%%%%
%%%%%%%%%%%%%%%%%%%%%%%%%%%%%%%%%%%%%%%%%%%%%%%%%%%%%%%%%%%%%%%%%%%%%%%%%%%%%%%%
\section{Information}

%%%%%%%%%%%%%%%%%%%%%%%%%%%%%%%%%%%%%%%%%%%%%%%%%%%%%%%%%%%%%%%%%%%%%%%%%%%%%%%%
\subsection{Copyright}

Copyright \copyright{} 2017--2018 Niklas Beisert

This work may be distributed and/or modified under the
conditions of the \LaTeX{} Project Public License, either version 1.3
of this license or (at your option) any later version.
The latest version of this license is in
  \url{http://www.latex-project.org/lppl.txt}
and version 1.3 or later is part of all distributions of \LaTeX{}
version 2005/12/01 or later.

This work has the LPPL maintenance status `maintained'.

The Current Maintainer of this work is Niklas Beisert.

This work consists of the files |README.txt|, |childdoc.ins| and |childdoc.dtx|
as well as the derived files |childdoc.def|, |cdocsamp.tex|
with |cdocsch1.tex|, |cdocsch2.tex|, |cdocspt3.tex|, |cdocspt4.tex|,
|cdocsdrf.tex|, |cdocsfn1.tex|, |cdocsfn2.tex|
as well as |childdoc.pdf|.

%%%%%%%%%%%%%%%%%%%%%%%%%%%%%%%%%%%%%%%%%%%%%%%%%%%%%%%%%%%%%%%%%%%%%%%%%%%%%%%%
\subsection{Files and Installation}

The package consists of the files:
%
\begin{center}
\begin{tabular}{ll}
    |README.txt|   & readme file \\
    |childdoc.ins| & installation file \\
    |childdoc.dtx| & source file \\
    |childdoc.def| & definition file \\
    |cdocsamp.tex| & sample main file \\
    |cdocsch1.tex| & sample include file \\
    |cdocsch2.tex| & sample include file \\
    |cdocspt3.tex| & sample part file \\
    |cdocspt4.tex| & sample part file \\
    |cdocsdrf.tex| & sample redirection file \\
    |cdocsfn1.tex| & sample redirection file \\
    |cdocsfn2.tex| & sample redirection file \\
    |childdoc.pdf| & manual
\end{tabular}
\end{center}
%
The distribution consists of the files
|README.txt|, |childdoc.ins| and |childdoc.dtx|.
%
\begin{itemize}
\item
Run (pdf)\LaTeX{} on |childdoc.dtx|
to compile the manual |childdoc.pdf| (this file).
\item
Run \LaTeX{} on |childdoc.ins| to create the definitions file |childdoc.def|
and the sample |cdocsamp.tex| with include files
|cdocsch1.tex|, |cdocsch2.tex|, |cdocspt3.tex|, |cdocspt4.tex|,
|cdocsdrf.tex|, |cdocsfn1.tex|, |cdocsfn2.tex|.
Then copy the file |childdoc.def| to an appropriate directory of your \LaTeX{}
distribution, e.g.\ \textit{texmf-root}|/tex/latex/childdoc|.
\end{itemize}

%%%%%%%%%%%%%%%%%%%%%%%%%%%%%%%%%%%%%%%%%%%%%%%%%%%%%%%%%%%%%%%%%%%%%%%%%%%%%%%%
\subsection{Related CTAN Packages}

There are several other packages which offer a similar functionality:
%
\begin{itemize}
\item
The packages
\href{http://ctan.org/pkg/docmute}{\textsf{docmute}},
\href{http://ctan.org/pkg/includex}{\textsf{includex}} and
\href{http://ctan.org/pkg/standalone}{\textsf{standalone}}
provide commands to include only the document body of
a child file thus allowing both files to be compiled individually.
\item
The packages \href{http://ctan.org/pkg/subdocs}{\textsf{subdocs}}
and \href{http://ctan.org/pkg/subfiles}{\textsf{subfiles}}
provide structures in which the main and child documents can be
encapsulated and allowing them to be compiled individually.
The inclusion mechanism is different from the conventional |\include|.
\item
The package \href{http://ctan.org/pkg/combine}{\textsf{combine}}
is an elaborate solution to combine several documents into one.
\end{itemize}
%
See also the CTAN topic \href{http://ctan.org/topic/subdocs}{\textsf{subdocs}}
for further related packages.
The present package differs from the above solutions in that
a document structure constructed with the conventional |\include| mechanism
just needs two extra commands at the top of every file
such that all constituent files can be compiled individually.

%%%%%%%%%%%%%%%%%%%%%%%%%%%%%%%%%%%%%%%%%%%%%%%%%%%%%%%%%%%%%%%%%%%%%%%%%%%%%%%%
%\subsection{Feature Suggestions}
%
%The following is a list of features which may be useful for future
%versions of this package:
%%
%\begin{itemize}
%\item
%\ldots
%\end{itemize}

%%%%%%%%%%%%%%%%%%%%%%%%%%%%%%%%%%%%%%%%%%%%%%%%%%%%%%%%%%%%%%%%%%%%%%%%%%%%%%%%
\subsection{Revision History}

%%%%%%%%%%%%%%%%%%%%%%%%%%%%%%%%%%%%%%%%
\paragraph{v2.0:} 2018/12/30

\begin{itemize}
\item
immediate forward processing
\item
added |\childdocby| mechanism
\item
manual restructured
\end{itemize}

%%%%%%%%%%%%%%%%%%%%%%%%%%%%%%%%%%%%%%%%
\paragraph{v1.6:} 2018/01/17

\begin{itemize}
\item
application for development of include files
\item
corrections to manual
\end{itemize}

%%%%%%%%%%%%%%%%%%%%%%%%%%%%%%%%%%%%%%%%
\paragraph{v1.5:} 2017/05/21

\begin{itemize}
\item
more complete structuring introduced
\item
|\childdocof| introduced
\item
|\childdoc| renamed to |\childdocmain|
\item
|\childredirect| renamed to |\childdocforward| and |\childdocforwardprefix|
and functionality expanded
\end{itemize}

%%%%%%%%%%%%%%%%%%%%%%%%%%%%%%%%%%%%%%%%
\paragraph{v1.0:} 2017/04/27

\begin{itemize}
\item
manual and install package
\item
first version published on CTAN
\end{itemize}

%%%%%%%%%%%%%%%%%%%%%%%%%%%%%%%%%%%%%%%%
\paragraph{v0.6:} 2017/04/26

\begin{itemize}
\item
redirection mechanism added
\end{itemize}

%%%%%%%%%%%%%%%%%%%%%%%%%%%%%%%%%%%%%%%%
\paragraph{v0.5:} 2017/04/26

\begin{itemize}
\item
functionality in definition file
\end{itemize}


%%%%%%%%%%%%%%%%%%%%%%%%%%%%%%%%%%%%%%%%%%%%%%%%%%%%%%%%%%%%%%%%%%%%%%%%%%%%%%%%
%%%%%%%%%%%%%%%%%%%%%%%%%%%%%%%%%%%%%%%%%%%%%%%%%%%%%%%%%%%%%%%%%%%%%%%%%%%%%%%%
%%%%%%%%%%%%%%%%%%%%%%%%%%%%%%%%%%%%%%%%%%%%%%%%%%%%%%%%%%%%%%%%%%%%%%%%%%%%%%%%
\appendix

\settowidth\MacroIndent{\rmfamily\scriptsize 000\ }

 \DocInput{childdoc.dtx}

\end{document}
%</driver>
% \fi
%
% %%%%%%%%%%%%%%%%%%%%%%%%%%%%%%%%%%%%%%%%%%%%%%%%%%%%%%%%%%%%%%%%%%%%%%%%%%%%%%
% %%%%%%%%%%%%%%%%%%%%%%%%%%%%%%%%%%%%%%%%%%%%%%%%%%%%%%%%%%%%%%%%%%%%%%%%%%%%%%
% \section{Sample}
%\iffalse
%<*samplemain>
%\fi
%
% The following presents a sample document
% with two chapters, two parts, a title page,
% a compile flag as well as three forwarding files to set the flag.
% It consists of eight |.tex| files:
% \begin{center}
% \begin{tabular}{ll}
% |cdocsamp.tex|&main file\\
% |cdocsch1.tex|&include file for chapter 1\\
% |cdocsch2.tex|&include file for chapter 2\\
% |cdocspt3.tex|&include file for part 3\\
% |cdocspt4.tex|&include file for part 4\\
% |cdocsdrf.tex|&forwarding file for main file in draft mode\\
% |cdocsfi1.tex|&forwarding file for final version of chapter 1\\
% |cdocsfi2.tex|&forwarding file for final version of chapter 2\\
% \end{tabular}
% \end{center}
% Each of the eight files can be compiled directly by the \LaTeX{} compiler.
%
% %%%%%%%%%%%%%%%%%%%%%%%%%%%%%%%%%%%%%%
% \paragraph{Main File.}
%
% The main file is called |cdocsamp.tex|.
%
% Load the \textsf{childdoc} definitions and
% declare the filename for the main document:
%    \begin{macrocode}
\input{childdoc.def}
\childdocmain{}
%    \end{macrocode}

% Optional override for |\version| flag:
%    \begin{macrocode}
%%\ifchilddoc\else\providecommand{\version}{draft}\fi
%    \end{macrocode}

% Define the default values for the |\version| flag
% (|final| for the main file and |draft| for childs):
%    \begin{macrocode}
\ifchilddoc
\providecommand{\version}{draft}
\else
\providecommand{\version}{final}
\fi
%    \end{macrocode}

% Load the standard document class:
%    \begin{macrocode}
\documentclass[12pt]{article}
%    \end{macrocode}

% Start the document body:
%    \begin{macrocode}
\begin{document}
%    \end{macrocode}

% Declare a title page.
% Print title, part of document being processed and version flag:
%    \begin{macrocode}
\addtocounter{page}{-1}
\begin{center}
{\LARGE\bfseries{}childdoc example\par}
\vspace{1cm}
\ifchilddoc
\ifchilddocmanual part\else chapter\fi:
`\childdocname' of `\childdocjob'\par
\else
main document: `\childdocjob'\par
\fi
version: \version\par
\end{center}
\newpage
%    \end{macrocode}

% Manually include selected file,
% otherwise process as usual:
%    \begin{macrocode}
\ifchilddocmanual
\section*{part `\childdocname'}
\input{\childdocname}
\else
%    \end{macrocode}

% Include the two chapters:
%    \begin{macrocode}
\include{cdocsch1}
\include{cdocsch2}
%    \end{macrocode}

% Include the two parts unless only chapters should be displayed:
%    \begin{macrocode}
\ifchilddoc\else
\section{part three}
\input{cdocspt3}
\section{part four}
\input{cdocspt4}
\fi
%    \end{macrocode}

% Process as usual until here:
%    \begin{macrocode}
\fi
%    \end{macrocode}

% End of document body:
%    \begin{macrocode}
\end{document}
%    \end{macrocode}
%\iffalse
%</samplemain>
%\fi
%
% %%%%%%%%%%%%%%%%%%%%%%%%%%%%%%%%%%%%%%
% \paragraph{Chapter Include Files.}
%
% The include files are called |cdocsch1.tex| and |cdocsch2.tex|.
%
%\iffalse
%<*samplechap1|samplechap2>
%\fi

% Optional override for |\version| flag:
%    \begin{macrocode}
%%\providecommand{\version}{final}
%    \end{macrocode}

% Include the main document:
%    \begin{macrocode}
\input{childdoc.def}
\childdocof{cdocsamp}
%    \end{macrocode}

%\iffalse
%</samplechap1|samplechap2>
%\fi
%
%\iffalse
%<*samplechap1>
%\fi
% Some text for chapter 1:
%    \begin{macrocode}
\section{one}
some text in chapter one
%    \end{macrocode}

%\iffalse
%</samplechap1>
%\fi
% Some text for chapter 2:
%\iffalse
%<*samplechap2>
%\fi
%    \begin{macrocode}
\section{two}
more text in chapter two
%    \end{macrocode}

%\iffalse
%</samplechap2>
%\fi
%
% %%%%%%%%%%%%%%%%%%%%%%%%%%%%%%%%%%%%%%
% \paragraph{Part Include Files.}
%
% The include files are called |cdocspt3.tex| and |cdocspt4.tex|.
%
%\iffalse
%<*samplepart3|samplepart4>
%\fi

% Optional override for |\version| flag:
%    \begin{macrocode}
%%\providecommand{\version}{final}
%    \end{macrocode}

% Include the main document:
%    \begin{macrocode}
\input{childdoc.def}
\childdocby{cdocsamp}
%    \end{macrocode}

%\iffalse
%</samplepart3|samplepart4>
%\fi
%
%\iffalse
%<*samplepart3>
%\fi
% Some text for part 3:
%    \begin{macrocode}
some text in part three
%    \end{macrocode}

%\iffalse
%</samplepart3>
%\fi
% Some text for part 4:
%\iffalse
%<*samplepart4>
%\fi
%    \begin{macrocode}
more text in part four
%    \end{macrocode}

%\iffalse
%</samplepart4>
%\fi
%
% %%%%%%%%%%%%%%%%%%%%%%%%%%%%%%%%%%%%%%
% \paragraph{Forwarding for a Complete Draft.}
%
% The following forwarding file |cdocsdrf.tex|
% compiles the main document in draft mode:
%\iffalse
%<*sampledraft>
%\fi
%    \begin{macrocode}
\def\version{draft}
\input{childdoc.def}
\childdocforward{cdocsamp}
%    \end{macrocode}

%\iffalse
%</sampledraft>
%\fi
%
% %%%%%%%%%%%%%%%%%%%%%%%%%%%%%%%%%%%%%%
% \paragraph{Forwarding for Final Version of the Chapters.}
%
% The following forwarding files |cdocsfn1.tex| and |cdocsfn2.tex|
% (with identical content)
% compile the final versions of the child documents
% |cdocsch1.tex| and |cdocsch2.tex|, respectively:
%\iffalse
%<*samplefinal>
%\fi
%    \begin{macrocode}
\def\version{final}
\input{childdoc.def}
\childdocforwardprefix[cdocsamp]{cdocsfn}{cdocsch}
%    \end{macrocode}

%\iffalse
%</samplefinal>
%\fi
%
% %%%%%%%%%%%%%%%%%%%%%%%%%%%%%%%%%%%%%%
% \paragraph{Command Line Processing.}
%
% The following three command lines generate the output files
% |cdocscld|, |cdocscl1| and |cdocscl2|
% which should be identical to
% |cdocsdrf|, |cdocsch1| and |cdocsfn2|, respectively:
% \begin{center}
% \begin{tabular}{l}
% |latex -jobname cdocscld \|\\
% |  "\def\version{draft}\input{childdoc.def}\childdocforward{cdocsamp}"|\\
% |latex -jobname cdocscl1 \|\\
% |  "\input{childdoc.def}\childdocforward[cdocsamp]{cdocsch1}"|\\
% |latex -jobname cdocscl2 \|\\
% |  "\def\version{final}\input{childdoc.def}\childdocforward{cdocsch2}"|
% \end{tabular}
% \end{center}
% Note that the trailing backslash on each first line
% merely continues the input to the second line
% (for convenient cut ant paste).
% Furthermore, the command |latex| can be replaced by any
% of its alternative versions such as |pdflatex|.
%
% %%%%%%%%%%%%%%%%%%%%%%%%%%%%%%%%%%%%%%%%%%%%%%%%%%%%%%%%%%%%%%%%%%%%%%%%%%%%%%
% %%%%%%%%%%%%%%%%%%%%%%%%%%%%%%%%%%%%%%%%%%%%%%%%%%%%%%%%%%%%%%%%%%%%%%%%%%%%%%
% \section{Implementation}
%\iffalse
%<*package>
%\fi
%
% This section describes the definitions file |childdoc.def|.

% The definitions cannot be loaded using |\usepackage| or |\RequirePackage|
% which has a mechanism to prevent loading a style file more than once.
% When loading the definitions by means of |\input|
% multiple instances have to be prevented manually:
%\iffalse
%This code needs to be before the `\ProvidesFile' directive
%which is defined at the beginning of this file.
%Therefore it is also placed there and commented out here.
%</package>
%<*discard>
%\fi
%    \begin{macrocode}
\ifdefined\childdocmain\endinput\fi
%    \end{macrocode}
%\iffalse
%</discard>
%<*package>
%\fi
%
% \macro{\ifchilddoc}
% \macro{\ifchilddocmanual}
% The conditional |\ifchilddoc| tells whether a
% child (true) or main (false) document is being compiled.
% The conditional |\ifchilddocmanual| tells whether
% the |\includeonly| mechanism is used (false) or
% the selection of child files must be performed manually (true).
% The definitions initialise to false:
%    \begin{macrocode}
\newif\ifchilddoc
\newif\ifchilddocmanual
%    \end{macrocode}

% \macro{\childdocname}
% \macro{\childdocjob}
% The macro |\childdocname| stores the name of the main document
% to be compiled. The macro |\childdocjob| stores the name of
% the document on which the \LaTeX{} compiler was originally invoked.
% The content of |\jobname| cannot be compared
% to filenames specified in the source due to different catcodes.
% The following code rescans |\jobname|, stores the result
% in |\childdocname| and saves a copy in |\childdocjob|:
%    \begin{macrocode}
\edef\childdocname{\scantokens\expandafter{\jobname\noexpand}}
\let\childdocjob\childdocname
%    \end{macrocode}

% \macro{\childdocdisable}
% The macro |\childdocdisable| prevents the main file
% from being processed more than once.
% At this stage, the main document command |\childdocmain|
% is assumed to be called once again where it should do nothing.
% Any subsequent call to it should prevent
% a secondary processing of the main document
% It overwrites the forwarding commands
% |\childdocof| and |\childdocforward|
% with empty macros to prevent further inclusions of the main document:
%    \begin{macrocode}
\newcommand{\childdocdisable}
{
  \renewcommand{\childdocmain}[1]{\renewcommand{\childdocmain}[1]{\endinput}}
  \renewcommand{\childdocof}[1]{}
  \renewcommand{\childdocby}[2][]{}
  \renewcommand{\childdocforward}[2][]{}
  \renewcommand{\childdocdisable}{}
}
%    \end{macrocode}

% \macro{\childdocmain}
% The macro |\childdocmain| is to be called at the top of the main file
% with nothing or the main filename (without extension) as argument.
% First, it breaks loops.
% If the argument is not empty and does not match |\childdocname|
% (which is set by the first inclusion of |childdoc.def|),
% |\ifchilddoc| is set to true, |\includeonly| is applied to the child file
% and |\jobname| is set to the main file
% (for proper handling of |.aux| files):
%    \begin{macrocode}
\newcommand{\childdocmain}[1]
{
  \childdocdisable\childdocmain{}
  \if?#1?\else
    \begingroup
      \def\childdoctmp{#1}
      \ifx\childdoctmp\childdocname
        \def\childdoctmp{}
      \else
        \def\childdoctmp
        {
          \childdoctrue
          \includeonly{\childdocname}
          \def\childdocjob{#1}
          \def\jobname{#1}
        }
      \fi
      \expandafter
    \endgroup
    \childdoctmp
  \fi
}
%    \end{macrocode}

% \macro{\childdocof}
% The command |\childdocof| redirects
% compilation to the main file |#1|.
%    \begin{macrocode}
\newcommand{\childdocof}[1]
{
  \childdocdisable
  \childdoctrue
  \includeonly{\childdocname}
  \def\jobname{#1}
  \def\childdocjob{#1}
  \input{#1}
}
%    \end{macrocode}

% \macro{\childdocby}
% The command |\childdocby| ....
%    \begin{macrocode}
\newcommand{\childdocby}[2][]
{
  \childdocdisable
  \childdoctrue
  \childdocmanualtrue
  \if?#1?\else
    \def\jobname{#2}
  \fi
  \def\childdocjob{#2}
  \input{#2}
  \endinput
}
%    \end{macrocode}

% \macro{\childdocforward}
% The command |\childdocforward| redirects
% compilation to the main file or
% (if the optional argument is given) a child file.
% Parameters are set as if the main file
% or a child file starting with |\childdocof| was compiled.
% Then compilation is handed over to the main file:
%    \begin{macrocode}
\newcommand{\childdocforward}[2][]
{
  \begingroup
    \if?#1?
      \def\childdoctmp
      {
        \def\childdocname{#2}
        \def\childdocjob{#2}
        \def\jobname{#2}
        \input{#2}
        \endinput
      }
    \else
      \def\childdoctmp
      {
        \childdocdisable
        \def\childdocname{#2}
        \childdoctrue
        \includeonly{#2}
        \def\childdocjob{#1}
        \def\jobname{#1}
        \input{#1}
        \endinput
      }
    \fi
    \expandafter
  \endgroup
  \childdoctmp
}
%    \end{macrocode}

% \macro{\childdocforwardprefix}
% The command |\childdocforwardprefix| redirects
% compilation to the main or a child file by means of a pattern.
% The prefix |#1| in the current filename is replaced by |#2|
% and the suffix of the current filename is kept
% (it is assumed that the filename does not contain the substring `|~~~|'
% which is used as a delimiter).
% Compilation is handed over to the new file by |\childdocforward|:
%    \begin{macrocode}
\newcommand{\childdocforwardprefix}[3][]
{
  \begingroup
    \def\childdocextract #2##1~~~{\def\childdoctmp{\childdocforward[#1]{#3##1}}}
    \expandafter\childdocextract\childdocname~~~
    \expandafter
  \endgroup
  \childdoctmp
}
%    \end{macrocode}

% \macro{\childdoc}
% The deprecated macro |\childdoc| is a legacy version of |\childdocmain|:
%    \begin{macrocode}
\newcommand{\childdoc}{\childdocmain}
%    \end{macrocode}

% \macro{\childdocredirect}
% The deprecated macro |\childdocredirect| is a legacy version
% of |\childdocforward| and |\childdocforwardprefix|:
%    \begin{macrocode}
\newcommand{\childdocredirect}[2][]
{
  \begingroup
    \if?#1?
      \def\childdoctmp{\childdocforward{#2}}
    \else
      \def\childdoctmp{\childdocforwardprefix{#1}{#2}}
    \fi
    \expandafter
  \endgroup
  \childdoctmp
}
%    \end{macrocode}

%\iffalse
%</package>
%\fi
%
\endinput

\childdocmain{}
%    \end{macrocode}

% Optional override for |\version| flag:
%    \begin{macrocode}
%%\ifchilddoc\else\providecommand{\version}{draft}\fi
%    \end{macrocode}

% Define the default values for the |\version| flag
% (|final| for the main file and |draft| for childs):
%    \begin{macrocode}
\ifchilddoc
\providecommand{\version}{draft}
\else
\providecommand{\version}{final}
\fi
%    \end{macrocode}

% Load the standard document class:
%    \begin{macrocode}
\documentclass[12pt]{article}
%    \end{macrocode}

% Start the document body:
%    \begin{macrocode}
\begin{document}
%    \end{macrocode}

% Declare a title page.
% Print title, part of document being processed and version flag:
%    \begin{macrocode}
\addtocounter{page}{-1}
\begin{center}
{\LARGE\bfseries{}childdoc example\par}
\vspace{1cm}
\ifchilddoc
\ifchilddocmanual part\else chapter\fi:
`\childdocname' of `\childdocjob'\par
\else
main document: `\childdocjob'\par
\fi
version: \version\par
\end{center}
\newpage
%    \end{macrocode}

% Manually include selected file,
% otherwise process as usual:
%    \begin{macrocode}
\ifchilddocmanual
\section*{part `\childdocname'}
\input{\childdocname}
\else
%    \end{macrocode}

% Include the two chapters:
%    \begin{macrocode}
\include{cdocsch1}
\include{cdocsch2}
%    \end{macrocode}

% Include the two parts unless only chapters should be displayed:
%    \begin{macrocode}
\ifchilddoc\else
\section{part three}
\input{cdocspt3}
\section{part four}
\input{cdocspt4}
\fi
%    \end{macrocode}

% Process as usual until here:
%    \begin{macrocode}
\fi
%    \end{macrocode}

% End of document body:
%    \begin{macrocode}
\end{document}
%    \end{macrocode}
%\iffalse
%</samplemain>
%\fi
%
% %%%%%%%%%%%%%%%%%%%%%%%%%%%%%%%%%%%%%%
% \paragraph{Chapter Include Files.}
%
% The include files are called |cdocsch1.tex| and |cdocsch2.tex|.
%
%\iffalse
%<*samplechap1|samplechap2>
%\fi

% Optional override for |\version| flag:
%    \begin{macrocode}
%%\providecommand{\version}{final}
%    \end{macrocode}

% Include the main document:
%    \begin{macrocode}
% \iffalse
%
% childdoc.dtx Copyright (C) 2017-2018 Niklas Beisert
%
% This work may be distributed and/or modified under the
% conditions of the LaTeX Project Public License, either version 1.3
% of this license or (at your option) any later version.
% The latest version of this license is in
%   http://www.latex-project.org/lppl.txt
% and version 1.3 or later is part of all distributions of LaTeX
% version 2005/12/01 or later.
%
% This work has the LPPL maintenance status `maintained'.
%
% The Current Maintainer of this work is Niklas Beisert.
%
% This work consists of the files childdoc.dtx and childdoc.ins
% and the derived files childdoc.def and cdocsamp.tex with
% cdocsch1.tex, cdocsch2.tex, cdocsdrf.tex, cdocsfn1.tex, cdocsfn2.tex.
%
%<package>\ifdefined\childdocmain\endinput\fi
%<package>\ProvidesFile{childdoc.def}[2018/12/30 v2.0 child document driver]
%<samplemain>\ProvidesFile{cdocsamp.tex}[2018/12/30 v2.0 sample for childdoc]
%<*driver>
%\ProvidesFile{childdoc.drv}[2018/12/30 v2.0 childdoc reference manual file]
\PassOptionsToClass{10pt,a4paper}{article}
\documentclass{ltxdoc}

\usepackage[margin=35mm]{geometry}
\usepackage{hyperref}
\usepackage{hyperxmp}
\usepackage[usenames]{color}

\hypersetup{colorlinks=true}
\hypersetup{pdfstartview=FitH}
\hypersetup{pdfpagemode=UseNone}
\hypersetup{pdfsource={}}
\hypersetup{pdflang={en-UK}}
\hypersetup{pdfcopyright={Copyright 2017-2018 Niklas Beisert.
  This work may be distributed and/or modified under the
  conditions of the LaTeX Project Public License, either version 1.3
  of this license or (at your option) any later version.}}
\hypersetup{pdflicenseurl={http://www.latex-project.org/lppl.txt}}
\hypersetup{pdfcontactaddress={ETH Zurich, ITP, HIT K,
  Wolfgang-Pauli-Strasse 27}}
\hypersetup{pdfcontactpostcode={8093}}
\hypersetup{pdfcontactcity={Zurich}}
\hypersetup{pdfcontactcountry={Switzerland}}
\hypersetup{pdfcontactemail={nbeisert@itp.phys.ethz.ch}}
\hypersetup{pdfcontacturl={http://people.phys.ethz.ch/\xmptilde nbeisert/}}

\newcommand{\secref}[1]{\hyperref[#1]{section \ref*{#1}}}

\parskip1ex
\parindent0pt
\let\olditemize\itemize
\def\itemize{\olditemize\parskip0pt}

\begin{document}

\title{The \textsf{childdoc} Package}
\hypersetup{pdftitle={The childdoc Package}}
\author{Niklas Beisert\\[2ex]
  Institut f\"ur Theoretische Physik\\
  Eidgen\"ossische Technische Hochschule Z\"urich\\
  Wolfgang-Pauli-Strasse 27, 8093 Z\"urich, Switzerland\\[1ex]
  \href{mailto:nbeisert@itp.phys.ethz.ch}
  {\texttt{nbeisert@itp.phys.ethz.ch}}}
\hypersetup{pdfauthor={Niklas Beisert}}
\hypersetup{pdfsubject={Manual for the LaTeX2e Package childdoc}}
\date{30 December 2018, \textsf{v2.0}}
\maketitle

\begin{abstract}\noindent
\textsf{childdoc} is a \LaTeXe{} package
that enables the direct compilation
of document sections included by |\include|
to individual files.
\end{abstract}

\begingroup
\parskip0ex
\tableofcontents
\endgroup

%%%%%%%%%%%%%%%%%%%%%%%%%%%%%%%%%%%%%%%%%%%%%%%%%%%%%%%%%%%%%%%%%%%%%%%%%%%%%%%%
%%%%%%%%%%%%%%%%%%%%%%%%%%%%%%%%%%%%%%%%%%%%%%%%%%%%%%%%%%%%%%%%%%%%%%%%%%%%%%%%
\section{Introduction}

\LaTeX{} provides a mechanism to structure a large document (such as a book)
into a main file and several child files (containing the chapters)
using the |\include| command.
This mechanism is beneficial for documents
which span hundreds of pages in order to
make the source file(s) more manageable.
Moreover, compilation can be restricted to
selected child files by means of the |\includeonly| command.
The latter feature can be used to reduce the compilation time while editing
(this was significantly more useful in the earlier days of \LaTeX{})
or to generate a smaller document which is easier to navigate.
Another application of |\includeonly| is to generate
documents consisting of selected parts of the complete document.

However, there are a few drawbacks of the plain |\include| mechanism:
\begin{itemize}
\item
The child files cannot be compiled on their own,
they can only be compiled via the main file.
A naive editing environment
(such as a text editor with an option
to have the current file processed by \LaTeX)
may require one to switch to the main file before compiling;
attempting to compile the child file produces errors.
\item
The main file must be modified (each time)
to adjust the |\includeonly| command
to the present needs. This easily leaves the main file in a messy state.
\item
The generated document will always carry the filename
of the main document. This is inconvenient if
several child files are to be compiled and
to be kept for distribution.
\end{itemize}

The present package provides a simple interface
to make child files individually compilable by \LaTeX{}.
Compiling a child file then has the same effect as compiling
the main file with an |\includeonly| command
to select the appropriate child.
Moreover the generated document will carry the name of the child
rather than the main file.
This resolves all three above issues.

This feature is meant to make the editing of books,
thesis documents and lecture notes somewhat more convenient.
However, the package can also be used efficiently for
composing a series of documents (such as exercise sheets)
which are typically distributed individually.
It then assists the author in generating the individual documents
(potentially in different versions)
as well as a document containing the collected series.
Another application is in developing style files
or other kinds of included material
where compilation of the style file could redirect
to a sample or test file.

%%%%%%%%%%%%%%%%%%%%%%%%%%%%%%%%%%%%%%%%%%%%%%%%%%%%%%%%%%%%%%%%%%%%%%%%%%%%%%%%
%%%%%%%%%%%%%%%%%%%%%%%%%%%%%%%%%%%%%%%%%%%%%%%%%%%%%%%%%%%%%%%%%%%%%%%%%%%%%%%%
\section{Usage}

First of all, the package \textsf{childdoc} is \emph{not} a standard
\LaTeXe{} |.sty| style file! Therefore it needs to be invoked in
a non-standard way.

%%%%%%%%%%%%%%%%%%%%%%%%%%%%%%%%%%%%%%%%%%%%%%%%%%%%%%%%%%%%%%%%%%%%%%%%%%%%%%%%
\subsection{Included Files}
\label{sec:include}

%%%%%%%%%%%%%%%%%%%%%%%%%%%%%%%%%%%%%%%%
\DescribeMacro{\childdocmain}
To use the package, add the commands
\begin{center}
\begin{tabular}{l}
|\input{childdoc.def}|\\
|\childdocmain{}|\\
\end{tabular}
\end{center}
at the very top of the main \LaTeX{} file,
in particular \emph{before} the |\documentclass| statement!
The argument of |\childdocmain| should be left empty
(but it must be present).

%%%%%%%%%%%%%%%%%%%%%%%%%%%%%%%%%%%%%%%%
\DescribeMacro{\childdocof}
Furthermore, add the commands
\begin{center}
\begin{tabular}{l}
|\input{childdoc.def}|\\
|\childdocof{|\textit{main}|}|\\
\end{tabular}
\end{center}
at the top of every child file \textit{child}
which is included by |\include{|\textit{child}|}|
from within the main file
(or at least for those files to be compiled individually).
The argument \textit{main} must be the filename of the main file.

There are a couple of
considerations in setting up the main and child documents:

%%%%%%%%%%%%%%%%%%%%%%%%%%%%%%%%%%%%%%%%
\paragraph{Restrictions.}

Please note the following restrictions:
\begin{itemize}
\item
|\childdocmain| must be called with one argument \textit{main}
to ensure compatibility with earlier version of the package.
It must either be empty (|\childdocmain{}|)
or precisely match the filename of the main file in which it is specified.
See \secref{sec:detection} for further information.
\item
The filename \textit{main} must be specified without the |.tex| extension.
\item
The filename \textit{main} is case sensitive
(even in case-insensitive file systems)
due to internal string comparison.
\item
The argument \textit{main} should be fully expanded, it cannot be a macro.
\item
Subdirectories and special characters should be avoided in filenames.
\item
The command |\childdocmain{|\textit{main}|}| must be followed by a whitespace.
It should not be followed immediately by another command
or by a comment mark `|%|'.
This is because the \TeX{} parser reads the token immediately following
the argument of |\childdocmain| and puts it
at the beginning of every child section;
however, a white\-space is ignored.
\end{itemize}

%%%%%%%%%%%%%%%%%%%%%%%%%%%%%%%%%%%%%%%%
\paragraph{Content of Main File.}

It is advisable to place all content in the child files included by |\include|.
Any output contained in the main file will appear in all child documents
unless suppressed manually;
it cannot be suppressed automatically by the |\includeonly| directive
and thus should normally be avoided.
A method to include some content in the main file
by means of conditional processing is described in \secref{sec:conditional}.

%%%%%%%%%%%%%%%%%%%%%%%%%%%%%%%%%%%%%%%%
\paragraph{Page Numbering.}

When only a part of the document is compiled,
the appropriate numbering of pages
(as well as other status parameters)
is determined from the |.aux| files.
The latter contain information from previous passes.
However this information needs to propagate through
all intermediate child documents.
Therefore the page numbering in child documents may well
be inconsistent until the complete document is compiled at least once.

A useful (if unconventional) way to always ensure a consistent
page numbering is to restart the numbering in each child document
and denote the pages by `\textit{child}|.|\textit{page}'
where \textit{child} represents the chapter/section number of the child file.
This can be achieved by the command
|\numberwithin{page}{|\textit{child}|}|
of the \textsf{amsmath} package
where \textit{child} can be |chapter| or |section|
depending on the chosen structuring.
Alternatively, one can modify the macro |\thepage| appropriately
and reset the counter |page| at the start of each child file.

%%%%%%%%%%%%%%%%%%%%%%%%%%%%%%%%%%%%%%%%%%%%%%%%%%%%%%%%%%%%%%%%%%%%%%%%%%%%%%%%
\subsection{Conditional Processing}
\label{sec:conditional}

The package provides a mechanism to compile different versions
of a document. To customise the versions further some conditional processing
can come in handy to distinguish which version is being compiled.
The package provides two macros to describe the compilation context:

%%%%%%%%%%%%%%%%%%%%%%%%%%%%%%%%%%%%%%%%
\DescribeMacro{\ifchilddoc}
The conditional |\ifchilddoc| distinguishes between the compilation of
child documents and the main document:
%
\begin{center}
|\ifchilddoc |\textit{child-code}| |[|\||else |\textit{main-code}]| \||fi|
\end{center}

%%%%%%%%%%%%%%%%%%%%%%%%%%%%%%%%%%%%%%%%
\DescribeMacro{\childdocname}
\DescribeMacro{\childdocjob}
The macro |\childdocname| contains the filename (without extension)
of the main or child file being processed.
Note that |\childdocjob| will always contain the name of the main file.

%%%%%%%%%%%%%%%%%%%%%%%%%%%%%%%%%%%%%%%%
\paragraph{Title Page.}

Conditional processing can be used to include a title or banner page
in the main document when proper precautions are taken.
Importantly, the code in the main file should ensure that the page counter
(as well as other status parameters which are stored in the |.aux| files)
takes the same value after the conditional processing.
Otherwise the page numbers may take divergent values
depending on which part is compiled.

For example, a title page could be declared by:
%
\begin{center}
\begin{tabular}{l}
|\ifchilddoc\||else|\\
|\addtocounter{page}{-1}|\\
\textit{code for title page}\\
|\newpage|\\
|\||fi|
\end{tabular}
\end{center}
%
A banner page for the child documents can be generated by:
%
\begin{center}
\begin{tabular}{l}
|\ifchilddoc|\\
|\addtocounter{page}{-1}|\\
\textit{code for banner page}\\
|\newpage|\\
|\||fi|
\end{tabular}
\end{center}
%
Here one could write a message such as:
\begin{center}
|This is the part \childdocname{} of \childdocjob{}.|
\end{center}

%%%%%%%%%%%%%%%%%%%%%%%%%%%%%%%%%%%%%%%%%%%%%%%%%%%%%%%%%%%%%%%%%%%%%%%%%%%%%%%%
\subsection{Flags}
\label{sec:flags}

The package makes it easy to generate different versions
of the main or child documents.
To this end compilation flags can be defined
and assigned different default values.
They will be particularly useful in conjunction
with the forwarding mechanism described in \secref{sec:forward}.

For example, it may be useful to have a flag |\version|
which can be set to |draft| or |final|.
The document source will contain some conditional code
depending on the value of |\version|.
Suppose further, the flag should default to |final| for the main file
and to |draft| for child files
which is a natural assignment for editing the document.
This is achieved by placing the following code
in the preamble of the main document
(below the |\childdocmain| directive):
%
\begin{center}
\begin{tabular}{l}
|\ifchilddoc|\\
|\providecommand{\version}{draft}|\\
|\||else|\\
|\providecommand{\version}{final}|\\
|\||fi|
\end{tabular}
\end{center}
%
The definition by |\providecommand| makes sure
that previous definitions are not overwritten.
Further statements |\providecommand{\version}{...}|
can thus be added before the above code to override it.

For the main file, one might add a line
(between |\childdocmain| and the above block)
%
\begin{center}
|%\ifchilddoc\||else\providecommand{\version}{draft}\||fi|
\end{center}
%
which can be uncommented to produce a draft version.
Likewise one can add a line to the very top of a child file
(above the |\childdocof{|\textit{main}|}| directive)
%
\begin{center}
|%\providecommand{\version}{final}|
\end{center}
%
which can be uncommented to produce the final version of this child document.

%%%%%%%%%%%%%%%%%%%%%%%%%%%%%%%%%%%%%%%%%%%%%%%%%%%%%%%%%%%%%%%%%%%%%%%%%%%%%%%%
\subsection{Forwarding}
\label{sec:forward}

Different versions of the main or child documents
using compilation flags as described in \secref{sec:flags}
can be (permanently) stored in different files
for convenient compilation, viewing and distribution.
To this end, the package defines a command
to pass on compilation to a different file:

%%%%%%%%%%%%%%%%%%%%%%%%%%%%%%%%%%%%%%%%
\DescribeMacro{\childdocforward}
The command |\childdocforward| redirects processing to
another source file:
%
\begin{center}
\begin{tabular}{l}
|\input{childdoc.def}|\\
|\childdocforward[|\textit{main}|]{|\textit{dest}|}|\\
\end{tabular}
\end{center}
%
The argument \textit{dest} is the destination file
(without extension).
It should be the main file or one of the child files.
Note that further \textsf{childdoc} directives
such as |\childdocof| and |\childdocforward|
in the indicated file will be processed in this form.
The optional argument \textit{main}
passes on directly to the main file \textit{main}
while pretending to compile the child \textit{dest}.
This form behaves as if \textit{dest}
issues |\childdocof{|\textit{main}|}| right away,
and no further \textsf{childdoc} directives will be processed.

%%%%%%%%%%%%%%%%%%%%%%%%%%%%%%%%%%%%%%%%
\DescribeMacro{\...prefix}
In the alternative form |\childdocforwardprefix|,
%
\begin{center}
\begin{tabular}{l}
|\input{childdoc.def}|\\
|\childdocforwardprefix[|\textit{main}|]{|\textit{prefix}|}{|\textit{dest}|}|
\end{tabular}
\end{center}
%
the destination file is determined by a pattern
depending on the current file:
To make this work, the current file must be called
`{\textit{prefix}\hspace{0.2em}\textit{suffix}}'
with \textit{prefix} matching precisely the argument.
Processing is then passed on to the file
`{\textit{dest}\hspace{0.2em}\textit{suffix}}'.
Surely, the same effect is achieved by
directly specifying the
argument `{\textit{dest}\hspace{0.2em}\textit{suffix}}'
in the first form.
However, that requires to set up a different file
for each child. With the alternative form of the command
all these files can have exactly the same content
which simplifies setting them up and maintaining them.

For example, the following file |draft.tex|
with a compilation flag |\version| as described in \secref{sec:flags}
compiles the main document as a draft:
%
\begin{center}
\begin{tabular}{l}
|\def\version{draft}|\\
|\input{childdoc.def}|\\
|\childdocforward{|\textit{main}|}|
\end{tabular}
\end{center}
%
Likewise, the following files |final|\textit{nn}|.tex|
compile the final version of the child document
|child|\textit{nn}|.tex|:
%
\begin{center}
\begin{tabular}{l}
|\def\version{final}|\\
|\input{childdoc.def}|\\
|\childdocforwardprefix{final}{child}|
\end{tabular}
\end{center}
%

Note that when several versions of a main file and/or of each child file
are to be generated, it may be convenient to set up a |Makefile| or
shell script to automatise the process.

%%%%%%%%%%%%%%%%%%%%%%%%%%%%%%%%%%%%%%%%%%%%%%%%%%%%%%%%%%%%%%%%%%%%%%%%%%%%%%%%
\subsection{Command Line Processing}
\label{sec:commandline}

The effect of redirection files can also be achieved by invoking
the \LaTeX{} compiler with a more elaborate command line.
Most conveniently this should be done as part
of a shell script or a |Makefile|.

When using \textsf{childdoc} in the main file, the following
command lines effectively perform a redirection
(note that depending on the shell being used,
backslashes may have to be doubled: `|\|' $\to$ `|\\|'):
%
\begin{center}
|... -jobname "|\textit{target}|" |\\|"|[\textit{flags}]%
|\input{childdoc.def}\childdocforward[|\textit{main}|]{|\textit{dest}|}"|
\end{center}
%
Here \textit{target} is the name of the output file,
\textit{main} is the name of the main file
and \textit{dest} is the name of the main or child file to be processed
(all filenames without extensions).
The optional argument \textit{main} can be omitted
if \textit{main} matches \textit{dest}.
Optionally, compilation \textit{flags} can be defined via |\def| commands.
This command line makes the \TeX{} engine believe
it is compiling the file \textit{target}
whose content is specified as the latter parameter.
The provided code then forwards the processing to
\textit{main} or \textit{dest} as described in \secref{sec:forward}.

%%%%%%%%%%%%%%%%%%%%%%%%%%%%%%%%%%%%%%%%%%%%%%%%%%%%%%%%%%%%%%%%%%%%%%%%%%%%%%%%
\subsection{Include by Input}
\label{sec:input}

Including child documents by |\include| has some restrictions by design.
Most notably, the content of a child document always occupies
its own set of pages; pages cannot be shared between child documents.
Usually, this behaviour makes perfect sense
because each child document contain an essential part of the document.
However, in some situations it may be desirable to compose
a document from a collection of parts
without having mandatory page breaks between then.
For this case, the package
provides a mechanism to include parts
by |\input| which can also be processed individually.
However, by construction this mechanism
requires manual handling of the content to be output.

%%%%%%%%%%%%%%%%%%%%%%%%%%%%%%%%%%%%%%%%
\DescribeMacro{\ifchilddocmanual}
The main file should be prepared as usual, see \secref{sec:include}.
However, the document body must make a distinction
between processing of an individual part and of the main document, e.g.:
%
\begin{center}
\begin{tabular}{l}
|\ifchilddocmanual|\\
|\input{\childdocname}|\\
|\||else|\\
\textit{document body with }|\input{|\textit{part}|}|\\
|\||fi|
\end{tabular}
\end{center}
%
The conditional |\ifchilddocmanual| is true whenever
a part to be included by |\input| is being compiled,
and the name of the part is stored in |\childdocname|.

%%%%%%%%%%%%%%%%%%%%%%%%%%%%%%%%%%%%%%%%
\DescribeMacro{\childdocby}
Each part to be included by |\input| should start with:
%
\begin{center}
\begin{tabular}{l}
|\input{childdoc.def}|\\
|\childdocby{|\textit{main}|}|\\
\end{tabular}
\end{center}
%
The directive |\childdocby| is similar to |\childdocof|
described in \secref{sec:include},
but the subsequent selection of content must be done manually.
To that end, both |\ifchilddoc| and |\ifchilddocmanual|
will be true upon processing of a part,
and the name of the part is stored in |\childdocname|.
Note that |\jobname| will be set to the filename of the current part
so that each part receives an individual |.aux| file
that does not interfere with the |.aux| file(s) of the main document.
This behaviour can be altered by the alternative form
|\childdocby[*]{|\textit{main}|}| (with a non-empty optional argument)
which uses the |.aux| file of the main document
by setting |\jobname| to \textit{main}.

%%%%%%%%%%%%%%%%%%%%%%%%%%%%%%%%%%%%%%%%%%%%%%%%%%%%%%%%%%%%%%%%%%%%%%%%%%%%%%%%
\subsection{Driver Development}
\label{sec:driver}

The \textsf{childdoc} mechanism can also be use for the development
of definition files such as \LaTeX{} styles or classes.
This case differs from the above setup with multiple parts
included by |\include| in that no |\includeonly| should be invoked.
This can be achieved by starting the include file
(before |\ProvidesPackage|) with:
%
\begin{center}
\begin{tabular}{l}
|\input{childdoc.def}|\\
|\childdocforward{|\textit{main}|}|\\
\end{tabular}
\end{center}
%
or alternatively with:
%
\begin{center}
\begin{tabular}{l}
|\input{childdoc.def}|\\
|\childdocby{|\textit{main}|}|\\
\end{tabular}
\end{center}
%
Both forms have slightly different effects as described above.
The main file is prepared as usual, see \secref{sec:include}.

%%%%%%%%%%%%%%%%%%%%%%%%%%%%%%%%%%%%%%%%%%%%%%%%%%%%%%%%%%%%%%%%%%%%%%%%%%%%%%%%
\subsection{Legacy Detection}
\label{sec:detection}

The directive |\childdocmain| in the main file can detect
whether the complete document or merely a child is to be compiled
even without using the directive |\childdocof|.
This method is deprecated because it is less robust
and there is no compelling reason to use it;
it is merely provided for backward compatibility
and it may be removed in future versions.

If the detection mechanism is to be used,
it is mandatory to correctly specify
the filename of the main file as the argument of |\childdocmain|:
%
\begin{center}
\begin{tabular}{l}
|\input{childdoc.def}|\\
|\childdocmain{|\textit{main}|}|\\
\end{tabular}
\end{center}
%
If |\jobname| does not match the argument \textit{main} of |\childdocmain|,
it is assumed that |\jobname| points to the child file to be compiled.
When using |\childdocmain| with the main file specified as argument,
it suffices to start a child file
with just |\input{|\textit{main}|}|
without loading of the package and using |\childdocof|.
If instead all processing is done
with the appropriate \textsf{childdoc} directives,
the argument of \textit{main} of |\childdocmain| can be empty.

An alternative version of the command line processing described
in \secref{sec:commandline} using the detection mechanism reads:
%
\begin{center}
|... -jobname "|\textit{target}|" "|[\textit{flags}]%
[|\def\jobname{|\textit{dest}|}|]|\input{|\textit{main}|}"|
\end{center}

%%%%%%%%%%%%%%%%%%%%%%%%%%%%%%%%%%%%%%%%%%%%%%%%%%%%%%%%%%%%%%%%%%%%%%%%%%%%%%%%
\subsection{Manual Code}
\label{sec:manual}

In case one cannot be certain whether the definitions file |childdoc.def|
is installed on the target \TeX{} distribution
and one prefers not to ship it,
it is conceivable to paste a few relevant commands into the sources.

To that end, drop all statements |\input{childdoc.def}|
and perform the replacements as outlined below.
Instead of |\childdocmain{|\textit{main}|}| add the following code
to the top of the main file:
%
\begin{center}
\begin{tabular}{l}
|\||ifdefined\childdocname\endinput\||fi\newif\ifchilddoc|\\
|\edef\childdocname{\scantokens\expandafter{\jobname\noexpand}}|\\
|\def\childdocmain{|\textit{main}|}\||ifx\childdocmain\childdocname\||else|\\
|\childdoctrue\includeonly{\childdocname}\let\jobname\childdocmain\||fi|\\
\end{tabular}
\end{center}
%
Instead of |\childdocof{|\textit{main}|}| just include the main file
at the top of each child file:
%
\begin{center}
|\input{|\textit{main}|}|
\end{center}
%
A simple redirection |\childdocforward{|\textit{dest}|}| is achieved by:
%
\begin{center}
|\def\jobname{|\textit{dest}|}\input{\jobname}|
\end{center}
%
The redirection with prefix
|\childdocforwardprefix[|\textit{prefix}|]{|\textit{dest}|}|
is accomplished by:
%
\begin{center}
\begin{tabular}{l}
|{\edef\jobname{\scantokens\expandafter{\jobname\noexpand}}|\\
|\def\redirectjob |\textit{prefix}|#1~~~{\gdef\jobname{|\textit{dest}|#1}}|\\
|\expandafter\redirectjob\jobname~~~}\input{\jobname}|
\end{tabular}
\end{center}

In an alternative approach,
child documents can be compiled by a specific command line
without additional code or specific definitions:
%
\begin{center}
|... -jobname "|\textit{target}|" "|[\textit{flags}]%
|\includeonly{|\textit{dest}|}\input{|\textit{main}|}"|
\end{center}
%

%%%%%%%%%%%%%%%%%%%%%%%%%%%%%%%%%%%%%%%%%%%%%%%%%%%%%%%%%%%%%%%%%%%%%%%%%%%%%%%%
%%%%%%%%%%%%%%%%%%%%%%%%%%%%%%%%%%%%%%%%%%%%%%%%%%%%%%%%%%%%%%%%%%%%%%%%%%%%%%%%
\section{Information}

%%%%%%%%%%%%%%%%%%%%%%%%%%%%%%%%%%%%%%%%%%%%%%%%%%%%%%%%%%%%%%%%%%%%%%%%%%%%%%%%
\subsection{Copyright}

Copyright \copyright{} 2017--2018 Niklas Beisert

This work may be distributed and/or modified under the
conditions of the \LaTeX{} Project Public License, either version 1.3
of this license or (at your option) any later version.
The latest version of this license is in
  \url{http://www.latex-project.org/lppl.txt}
and version 1.3 or later is part of all distributions of \LaTeX{}
version 2005/12/01 or later.

This work has the LPPL maintenance status `maintained'.

The Current Maintainer of this work is Niklas Beisert.

This work consists of the files |README.txt|, |childdoc.ins| and |childdoc.dtx|
as well as the derived files |childdoc.def|, |cdocsamp.tex|
with |cdocsch1.tex|, |cdocsch2.tex|, |cdocspt3.tex|, |cdocspt4.tex|,
|cdocsdrf.tex|, |cdocsfn1.tex|, |cdocsfn2.tex|
as well as |childdoc.pdf|.

%%%%%%%%%%%%%%%%%%%%%%%%%%%%%%%%%%%%%%%%%%%%%%%%%%%%%%%%%%%%%%%%%%%%%%%%%%%%%%%%
\subsection{Files and Installation}

The package consists of the files:
%
\begin{center}
\begin{tabular}{ll}
    |README.txt|   & readme file \\
    |childdoc.ins| & installation file \\
    |childdoc.dtx| & source file \\
    |childdoc.def| & definition file \\
    |cdocsamp.tex| & sample main file \\
    |cdocsch1.tex| & sample include file \\
    |cdocsch2.tex| & sample include file \\
    |cdocspt3.tex| & sample part file \\
    |cdocspt4.tex| & sample part file \\
    |cdocsdrf.tex| & sample redirection file \\
    |cdocsfn1.tex| & sample redirection file \\
    |cdocsfn2.tex| & sample redirection file \\
    |childdoc.pdf| & manual
\end{tabular}
\end{center}
%
The distribution consists of the files
|README.txt|, |childdoc.ins| and |childdoc.dtx|.
%
\begin{itemize}
\item
Run (pdf)\LaTeX{} on |childdoc.dtx|
to compile the manual |childdoc.pdf| (this file).
\item
Run \LaTeX{} on |childdoc.ins| to create the definitions file |childdoc.def|
and the sample |cdocsamp.tex| with include files
|cdocsch1.tex|, |cdocsch2.tex|, |cdocspt3.tex|, |cdocspt4.tex|,
|cdocsdrf.tex|, |cdocsfn1.tex|, |cdocsfn2.tex|.
Then copy the file |childdoc.def| to an appropriate directory of your \LaTeX{}
distribution, e.g.\ \textit{texmf-root}|/tex/latex/childdoc|.
\end{itemize}

%%%%%%%%%%%%%%%%%%%%%%%%%%%%%%%%%%%%%%%%%%%%%%%%%%%%%%%%%%%%%%%%%%%%%%%%%%%%%%%%
\subsection{Related CTAN Packages}

There are several other packages which offer a similar functionality:
%
\begin{itemize}
\item
The packages
\href{http://ctan.org/pkg/docmute}{\textsf{docmute}},
\href{http://ctan.org/pkg/includex}{\textsf{includex}} and
\href{http://ctan.org/pkg/standalone}{\textsf{standalone}}
provide commands to include only the document body of
a child file thus allowing both files to be compiled individually.
\item
The packages \href{http://ctan.org/pkg/subdocs}{\textsf{subdocs}}
and \href{http://ctan.org/pkg/subfiles}{\textsf{subfiles}}
provide structures in which the main and child documents can be
encapsulated and allowing them to be compiled individually.
The inclusion mechanism is different from the conventional |\include|.
\item
The package \href{http://ctan.org/pkg/combine}{\textsf{combine}}
is an elaborate solution to combine several documents into one.
\end{itemize}
%
See also the CTAN topic \href{http://ctan.org/topic/subdocs}{\textsf{subdocs}}
for further related packages.
The present package differs from the above solutions in that
a document structure constructed with the conventional |\include| mechanism
just needs two extra commands at the top of every file
such that all constituent files can be compiled individually.

%%%%%%%%%%%%%%%%%%%%%%%%%%%%%%%%%%%%%%%%%%%%%%%%%%%%%%%%%%%%%%%%%%%%%%%%%%%%%%%%
%\subsection{Feature Suggestions}
%
%The following is a list of features which may be useful for future
%versions of this package:
%%
%\begin{itemize}
%\item
%\ldots
%\end{itemize}

%%%%%%%%%%%%%%%%%%%%%%%%%%%%%%%%%%%%%%%%%%%%%%%%%%%%%%%%%%%%%%%%%%%%%%%%%%%%%%%%
\subsection{Revision History}

%%%%%%%%%%%%%%%%%%%%%%%%%%%%%%%%%%%%%%%%
\paragraph{v2.0:} 2018/12/30

\begin{itemize}
\item
immediate forward processing
\item
added |\childdocby| mechanism
\item
manual restructured
\end{itemize}

%%%%%%%%%%%%%%%%%%%%%%%%%%%%%%%%%%%%%%%%
\paragraph{v1.6:} 2018/01/17

\begin{itemize}
\item
application for development of include files
\item
corrections to manual
\end{itemize}

%%%%%%%%%%%%%%%%%%%%%%%%%%%%%%%%%%%%%%%%
\paragraph{v1.5:} 2017/05/21

\begin{itemize}
\item
more complete structuring introduced
\item
|\childdocof| introduced
\item
|\childdoc| renamed to |\childdocmain|
\item
|\childredirect| renamed to |\childdocforward| and |\childdocforwardprefix|
and functionality expanded
\end{itemize}

%%%%%%%%%%%%%%%%%%%%%%%%%%%%%%%%%%%%%%%%
\paragraph{v1.0:} 2017/04/27

\begin{itemize}
\item
manual and install package
\item
first version published on CTAN
\end{itemize}

%%%%%%%%%%%%%%%%%%%%%%%%%%%%%%%%%%%%%%%%
\paragraph{v0.6:} 2017/04/26

\begin{itemize}
\item
redirection mechanism added
\end{itemize}

%%%%%%%%%%%%%%%%%%%%%%%%%%%%%%%%%%%%%%%%
\paragraph{v0.5:} 2017/04/26

\begin{itemize}
\item
functionality in definition file
\end{itemize}


%%%%%%%%%%%%%%%%%%%%%%%%%%%%%%%%%%%%%%%%%%%%%%%%%%%%%%%%%%%%%%%%%%%%%%%%%%%%%%%%
%%%%%%%%%%%%%%%%%%%%%%%%%%%%%%%%%%%%%%%%%%%%%%%%%%%%%%%%%%%%%%%%%%%%%%%%%%%%%%%%
%%%%%%%%%%%%%%%%%%%%%%%%%%%%%%%%%%%%%%%%%%%%%%%%%%%%%%%%%%%%%%%%%%%%%%%%%%%%%%%%
\appendix

\settowidth\MacroIndent{\rmfamily\scriptsize 000\ }

 \DocInput{childdoc.dtx}

\end{document}
%</driver>
% \fi
%
% %%%%%%%%%%%%%%%%%%%%%%%%%%%%%%%%%%%%%%%%%%%%%%%%%%%%%%%%%%%%%%%%%%%%%%%%%%%%%%
% %%%%%%%%%%%%%%%%%%%%%%%%%%%%%%%%%%%%%%%%%%%%%%%%%%%%%%%%%%%%%%%%%%%%%%%%%%%%%%
% \section{Sample}
%\iffalse
%<*samplemain>
%\fi
%
% The following presents a sample document
% with two chapters, two parts, a title page,
% a compile flag as well as three forwarding files to set the flag.
% It consists of eight |.tex| files:
% \begin{center}
% \begin{tabular}{ll}
% |cdocsamp.tex|&main file\\
% |cdocsch1.tex|&include file for chapter 1\\
% |cdocsch2.tex|&include file for chapter 2\\
% |cdocspt3.tex|&include file for part 3\\
% |cdocspt4.tex|&include file for part 4\\
% |cdocsdrf.tex|&forwarding file for main file in draft mode\\
% |cdocsfi1.tex|&forwarding file for final version of chapter 1\\
% |cdocsfi2.tex|&forwarding file for final version of chapter 2\\
% \end{tabular}
% \end{center}
% Each of the eight files can be compiled directly by the \LaTeX{} compiler.
%
% %%%%%%%%%%%%%%%%%%%%%%%%%%%%%%%%%%%%%%
% \paragraph{Main File.}
%
% The main file is called |cdocsamp.tex|.
%
% Load the \textsf{childdoc} definitions and
% declare the filename for the main document:
%    \begin{macrocode}
\input{childdoc.def}
\childdocmain{}
%    \end{macrocode}

% Optional override for |\version| flag:
%    \begin{macrocode}
%%\ifchilddoc\else\providecommand{\version}{draft}\fi
%    \end{macrocode}

% Define the default values for the |\version| flag
% (|final| for the main file and |draft| for childs):
%    \begin{macrocode}
\ifchilddoc
\providecommand{\version}{draft}
\else
\providecommand{\version}{final}
\fi
%    \end{macrocode}

% Load the standard document class:
%    \begin{macrocode}
\documentclass[12pt]{article}
%    \end{macrocode}

% Start the document body:
%    \begin{macrocode}
\begin{document}
%    \end{macrocode}

% Declare a title page.
% Print title, part of document being processed and version flag:
%    \begin{macrocode}
\addtocounter{page}{-1}
\begin{center}
{\LARGE\bfseries{}childdoc example\par}
\vspace{1cm}
\ifchilddoc
\ifchilddocmanual part\else chapter\fi:
`\childdocname' of `\childdocjob'\par
\else
main document: `\childdocjob'\par
\fi
version: \version\par
\end{center}
\newpage
%    \end{macrocode}

% Manually include selected file,
% otherwise process as usual:
%    \begin{macrocode}
\ifchilddocmanual
\section*{part `\childdocname'}
\input{\childdocname}
\else
%    \end{macrocode}

% Include the two chapters:
%    \begin{macrocode}
\include{cdocsch1}
\include{cdocsch2}
%    \end{macrocode}

% Include the two parts unless only chapters should be displayed:
%    \begin{macrocode}
\ifchilddoc\else
\section{part three}
\input{cdocspt3}
\section{part four}
\input{cdocspt4}
\fi
%    \end{macrocode}

% Process as usual until here:
%    \begin{macrocode}
\fi
%    \end{macrocode}

% End of document body:
%    \begin{macrocode}
\end{document}
%    \end{macrocode}
%\iffalse
%</samplemain>
%\fi
%
% %%%%%%%%%%%%%%%%%%%%%%%%%%%%%%%%%%%%%%
% \paragraph{Chapter Include Files.}
%
% The include files are called |cdocsch1.tex| and |cdocsch2.tex|.
%
%\iffalse
%<*samplechap1|samplechap2>
%\fi

% Optional override for |\version| flag:
%    \begin{macrocode}
%%\providecommand{\version}{final}
%    \end{macrocode}

% Include the main document:
%    \begin{macrocode}
\input{childdoc.def}
\childdocof{cdocsamp}
%    \end{macrocode}

%\iffalse
%</samplechap1|samplechap2>
%\fi
%
%\iffalse
%<*samplechap1>
%\fi
% Some text for chapter 1:
%    \begin{macrocode}
\section{one}
some text in chapter one
%    \end{macrocode}

%\iffalse
%</samplechap1>
%\fi
% Some text for chapter 2:
%\iffalse
%<*samplechap2>
%\fi
%    \begin{macrocode}
\section{two}
more text in chapter two
%    \end{macrocode}

%\iffalse
%</samplechap2>
%\fi
%
% %%%%%%%%%%%%%%%%%%%%%%%%%%%%%%%%%%%%%%
% \paragraph{Part Include Files.}
%
% The include files are called |cdocspt3.tex| and |cdocspt4.tex|.
%
%\iffalse
%<*samplepart3|samplepart4>
%\fi

% Optional override for |\version| flag:
%    \begin{macrocode}
%%\providecommand{\version}{final}
%    \end{macrocode}

% Include the main document:
%    \begin{macrocode}
\input{childdoc.def}
\childdocby{cdocsamp}
%    \end{macrocode}

%\iffalse
%</samplepart3|samplepart4>
%\fi
%
%\iffalse
%<*samplepart3>
%\fi
% Some text for part 3:
%    \begin{macrocode}
some text in part three
%    \end{macrocode}

%\iffalse
%</samplepart3>
%\fi
% Some text for part 4:
%\iffalse
%<*samplepart4>
%\fi
%    \begin{macrocode}
more text in part four
%    \end{macrocode}

%\iffalse
%</samplepart4>
%\fi
%
% %%%%%%%%%%%%%%%%%%%%%%%%%%%%%%%%%%%%%%
% \paragraph{Forwarding for a Complete Draft.}
%
% The following forwarding file |cdocsdrf.tex|
% compiles the main document in draft mode:
%\iffalse
%<*sampledraft>
%\fi
%    \begin{macrocode}
\def\version{draft}
\input{childdoc.def}
\childdocforward{cdocsamp}
%    \end{macrocode}

%\iffalse
%</sampledraft>
%\fi
%
% %%%%%%%%%%%%%%%%%%%%%%%%%%%%%%%%%%%%%%
% \paragraph{Forwarding for Final Version of the Chapters.}
%
% The following forwarding files |cdocsfn1.tex| and |cdocsfn2.tex|
% (with identical content)
% compile the final versions of the child documents
% |cdocsch1.tex| and |cdocsch2.tex|, respectively:
%\iffalse
%<*samplefinal>
%\fi
%    \begin{macrocode}
\def\version{final}
\input{childdoc.def}
\childdocforwardprefix[cdocsamp]{cdocsfn}{cdocsch}
%    \end{macrocode}

%\iffalse
%</samplefinal>
%\fi
%
% %%%%%%%%%%%%%%%%%%%%%%%%%%%%%%%%%%%%%%
% \paragraph{Command Line Processing.}
%
% The following three command lines generate the output files
% |cdocscld|, |cdocscl1| and |cdocscl2|
% which should be identical to
% |cdocsdrf|, |cdocsch1| and |cdocsfn2|, respectively:
% \begin{center}
% \begin{tabular}{l}
% |latex -jobname cdocscld \|\\
% |  "\def\version{draft}\input{childdoc.def}\childdocforward{cdocsamp}"|\\
% |latex -jobname cdocscl1 \|\\
% |  "\input{childdoc.def}\childdocforward[cdocsamp]{cdocsch1}"|\\
% |latex -jobname cdocscl2 \|\\
% |  "\def\version{final}\input{childdoc.def}\childdocforward{cdocsch2}"|
% \end{tabular}
% \end{center}
% Note that the trailing backslash on each first line
% merely continues the input to the second line
% (for convenient cut ant paste).
% Furthermore, the command |latex| can be replaced by any
% of its alternative versions such as |pdflatex|.
%
% %%%%%%%%%%%%%%%%%%%%%%%%%%%%%%%%%%%%%%%%%%%%%%%%%%%%%%%%%%%%%%%%%%%%%%%%%%%%%%
% %%%%%%%%%%%%%%%%%%%%%%%%%%%%%%%%%%%%%%%%%%%%%%%%%%%%%%%%%%%%%%%%%%%%%%%%%%%%%%
% \section{Implementation}
%\iffalse
%<*package>
%\fi
%
% This section describes the definitions file |childdoc.def|.

% The definitions cannot be loaded using |\usepackage| or |\RequirePackage|
% which has a mechanism to prevent loading a style file more than once.
% When loading the definitions by means of |\input|
% multiple instances have to be prevented manually:
%\iffalse
%This code needs to be before the `\ProvidesFile' directive
%which is defined at the beginning of this file.
%Therefore it is also placed there and commented out here.
%</package>
%<*discard>
%\fi
%    \begin{macrocode}
\ifdefined\childdocmain\endinput\fi
%    \end{macrocode}
%\iffalse
%</discard>
%<*package>
%\fi
%
% \macro{\ifchilddoc}
% \macro{\ifchilddocmanual}
% The conditional |\ifchilddoc| tells whether a
% child (true) or main (false) document is being compiled.
% The conditional |\ifchilddocmanual| tells whether
% the |\includeonly| mechanism is used (false) or
% the selection of child files must be performed manually (true).
% The definitions initialise to false:
%    \begin{macrocode}
\newif\ifchilddoc
\newif\ifchilddocmanual
%    \end{macrocode}

% \macro{\childdocname}
% \macro{\childdocjob}
% The macro |\childdocname| stores the name of the main document
% to be compiled. The macro |\childdocjob| stores the name of
% the document on which the \LaTeX{} compiler was originally invoked.
% The content of |\jobname| cannot be compared
% to filenames specified in the source due to different catcodes.
% The following code rescans |\jobname|, stores the result
% in |\childdocname| and saves a copy in |\childdocjob|:
%    \begin{macrocode}
\edef\childdocname{\scantokens\expandafter{\jobname\noexpand}}
\let\childdocjob\childdocname
%    \end{macrocode}

% \macro{\childdocdisable}
% The macro |\childdocdisable| prevents the main file
% from being processed more than once.
% At this stage, the main document command |\childdocmain|
% is assumed to be called once again where it should do nothing.
% Any subsequent call to it should prevent
% a secondary processing of the main document
% It overwrites the forwarding commands
% |\childdocof| and |\childdocforward|
% with empty macros to prevent further inclusions of the main document:
%    \begin{macrocode}
\newcommand{\childdocdisable}
{
  \renewcommand{\childdocmain}[1]{\renewcommand{\childdocmain}[1]{\endinput}}
  \renewcommand{\childdocof}[1]{}
  \renewcommand{\childdocby}[2][]{}
  \renewcommand{\childdocforward}[2][]{}
  \renewcommand{\childdocdisable}{}
}
%    \end{macrocode}

% \macro{\childdocmain}
% The macro |\childdocmain| is to be called at the top of the main file
% with nothing or the main filename (without extension) as argument.
% First, it breaks loops.
% If the argument is not empty and does not match |\childdocname|
% (which is set by the first inclusion of |childdoc.def|),
% |\ifchilddoc| is set to true, |\includeonly| is applied to the child file
% and |\jobname| is set to the main file
% (for proper handling of |.aux| files):
%    \begin{macrocode}
\newcommand{\childdocmain}[1]
{
  \childdocdisable\childdocmain{}
  \if?#1?\else
    \begingroup
      \def\childdoctmp{#1}
      \ifx\childdoctmp\childdocname
        \def\childdoctmp{}
      \else
        \def\childdoctmp
        {
          \childdoctrue
          \includeonly{\childdocname}
          \def\childdocjob{#1}
          \def\jobname{#1}
        }
      \fi
      \expandafter
    \endgroup
    \childdoctmp
  \fi
}
%    \end{macrocode}

% \macro{\childdocof}
% The command |\childdocof| redirects
% compilation to the main file |#1|.
%    \begin{macrocode}
\newcommand{\childdocof}[1]
{
  \childdocdisable
  \childdoctrue
  \includeonly{\childdocname}
  \def\jobname{#1}
  \def\childdocjob{#1}
  \input{#1}
}
%    \end{macrocode}

% \macro{\childdocby}
% The command |\childdocby| ....
%    \begin{macrocode}
\newcommand{\childdocby}[2][]
{
  \childdocdisable
  \childdoctrue
  \childdocmanualtrue
  \if?#1?\else
    \def\jobname{#2}
  \fi
  \def\childdocjob{#2}
  \input{#2}
  \endinput
}
%    \end{macrocode}

% \macro{\childdocforward}
% The command |\childdocforward| redirects
% compilation to the main file or
% (if the optional argument is given) a child file.
% Parameters are set as if the main file
% or a child file starting with |\childdocof| was compiled.
% Then compilation is handed over to the main file:
%    \begin{macrocode}
\newcommand{\childdocforward}[2][]
{
  \begingroup
    \if?#1?
      \def\childdoctmp
      {
        \def\childdocname{#2}
        \def\childdocjob{#2}
        \def\jobname{#2}
        \input{#2}
        \endinput
      }
    \else
      \def\childdoctmp
      {
        \childdocdisable
        \def\childdocname{#2}
        \childdoctrue
        \includeonly{#2}
        \def\childdocjob{#1}
        \def\jobname{#1}
        \input{#1}
        \endinput
      }
    \fi
    \expandafter
  \endgroup
  \childdoctmp
}
%    \end{macrocode}

% \macro{\childdocforwardprefix}
% The command |\childdocforwardprefix| redirects
% compilation to the main or a child file by means of a pattern.
% The prefix |#1| in the current filename is replaced by |#2|
% and the suffix of the current filename is kept
% (it is assumed that the filename does not contain the substring `|~~~|'
% which is used as a delimiter).
% Compilation is handed over to the new file by |\childdocforward|:
%    \begin{macrocode}
\newcommand{\childdocforwardprefix}[3][]
{
  \begingroup
    \def\childdocextract #2##1~~~{\def\childdoctmp{\childdocforward[#1]{#3##1}}}
    \expandafter\childdocextract\childdocname~~~
    \expandafter
  \endgroup
  \childdoctmp
}
%    \end{macrocode}

% \macro{\childdoc}
% The deprecated macro |\childdoc| is a legacy version of |\childdocmain|:
%    \begin{macrocode}
\newcommand{\childdoc}{\childdocmain}
%    \end{macrocode}

% \macro{\childdocredirect}
% The deprecated macro |\childdocredirect| is a legacy version
% of |\childdocforward| and |\childdocforwardprefix|:
%    \begin{macrocode}
\newcommand{\childdocredirect}[2][]
{
  \begingroup
    \if?#1?
      \def\childdoctmp{\childdocforward{#2}}
    \else
      \def\childdoctmp{\childdocforwardprefix{#1}{#2}}
    \fi
    \expandafter
  \endgroup
  \childdoctmp
}
%    \end{macrocode}

%\iffalse
%</package>
%\fi
%
\endinput

\childdocof{cdocsamp}
%    \end{macrocode}

%\iffalse
%</samplechap1|samplechap2>
%\fi
%
%\iffalse
%<*samplechap1>
%\fi
% Some text for chapter 1:
%    \begin{macrocode}
\section{one}
some text in chapter one
%    \end{macrocode}

%\iffalse
%</samplechap1>
%\fi
% Some text for chapter 2:
%\iffalse
%<*samplechap2>
%\fi
%    \begin{macrocode}
\section{two}
more text in chapter two
%    \end{macrocode}

%\iffalse
%</samplechap2>
%\fi
%
% %%%%%%%%%%%%%%%%%%%%%%%%%%%%%%%%%%%%%%
% \paragraph{Part Include Files.}
%
% The include files are called |cdocspt3.tex| and |cdocspt4.tex|.
%
%\iffalse
%<*samplepart3|samplepart4>
%\fi

% Optional override for |\version| flag:
%    \begin{macrocode}
%%\providecommand{\version}{final}
%    \end{macrocode}

% Include the main document:
%    \begin{macrocode}
% \iffalse
%
% childdoc.dtx Copyright (C) 2017-2018 Niklas Beisert
%
% This work may be distributed and/or modified under the
% conditions of the LaTeX Project Public License, either version 1.3
% of this license or (at your option) any later version.
% The latest version of this license is in
%   http://www.latex-project.org/lppl.txt
% and version 1.3 or later is part of all distributions of LaTeX
% version 2005/12/01 or later.
%
% This work has the LPPL maintenance status `maintained'.
%
% The Current Maintainer of this work is Niklas Beisert.
%
% This work consists of the files childdoc.dtx and childdoc.ins
% and the derived files childdoc.def and cdocsamp.tex with
% cdocsch1.tex, cdocsch2.tex, cdocsdrf.tex, cdocsfn1.tex, cdocsfn2.tex.
%
%<package>\ifdefined\childdocmain\endinput\fi
%<package>\ProvidesFile{childdoc.def}[2018/12/30 v2.0 child document driver]
%<samplemain>\ProvidesFile{cdocsamp.tex}[2018/12/30 v2.0 sample for childdoc]
%<*driver>
%\ProvidesFile{childdoc.drv}[2018/12/30 v2.0 childdoc reference manual file]
\PassOptionsToClass{10pt,a4paper}{article}
\documentclass{ltxdoc}

\usepackage[margin=35mm]{geometry}
\usepackage{hyperref}
\usepackage{hyperxmp}
\usepackage[usenames]{color}

\hypersetup{colorlinks=true}
\hypersetup{pdfstartview=FitH}
\hypersetup{pdfpagemode=UseNone}
\hypersetup{pdfsource={}}
\hypersetup{pdflang={en-UK}}
\hypersetup{pdfcopyright={Copyright 2017-2018 Niklas Beisert.
  This work may be distributed and/or modified under the
  conditions of the LaTeX Project Public License, either version 1.3
  of this license or (at your option) any later version.}}
\hypersetup{pdflicenseurl={http://www.latex-project.org/lppl.txt}}
\hypersetup{pdfcontactaddress={ETH Zurich, ITP, HIT K,
  Wolfgang-Pauli-Strasse 27}}
\hypersetup{pdfcontactpostcode={8093}}
\hypersetup{pdfcontactcity={Zurich}}
\hypersetup{pdfcontactcountry={Switzerland}}
\hypersetup{pdfcontactemail={nbeisert@itp.phys.ethz.ch}}
\hypersetup{pdfcontacturl={http://people.phys.ethz.ch/\xmptilde nbeisert/}}

\newcommand{\secref}[1]{\hyperref[#1]{section \ref*{#1}}}

\parskip1ex
\parindent0pt
\let\olditemize\itemize
\def\itemize{\olditemize\parskip0pt}

\begin{document}

\title{The \textsf{childdoc} Package}
\hypersetup{pdftitle={The childdoc Package}}
\author{Niklas Beisert\\[2ex]
  Institut f\"ur Theoretische Physik\\
  Eidgen\"ossische Technische Hochschule Z\"urich\\
  Wolfgang-Pauli-Strasse 27, 8093 Z\"urich, Switzerland\\[1ex]
  \href{mailto:nbeisert@itp.phys.ethz.ch}
  {\texttt{nbeisert@itp.phys.ethz.ch}}}
\hypersetup{pdfauthor={Niklas Beisert}}
\hypersetup{pdfsubject={Manual for the LaTeX2e Package childdoc}}
\date{30 December 2018, \textsf{v2.0}}
\maketitle

\begin{abstract}\noindent
\textsf{childdoc} is a \LaTeXe{} package
that enables the direct compilation
of document sections included by |\include|
to individual files.
\end{abstract}

\begingroup
\parskip0ex
\tableofcontents
\endgroup

%%%%%%%%%%%%%%%%%%%%%%%%%%%%%%%%%%%%%%%%%%%%%%%%%%%%%%%%%%%%%%%%%%%%%%%%%%%%%%%%
%%%%%%%%%%%%%%%%%%%%%%%%%%%%%%%%%%%%%%%%%%%%%%%%%%%%%%%%%%%%%%%%%%%%%%%%%%%%%%%%
\section{Introduction}

\LaTeX{} provides a mechanism to structure a large document (such as a book)
into a main file and several child files (containing the chapters)
using the |\include| command.
This mechanism is beneficial for documents
which span hundreds of pages in order to
make the source file(s) more manageable.
Moreover, compilation can be restricted to
selected child files by means of the |\includeonly| command.
The latter feature can be used to reduce the compilation time while editing
(this was significantly more useful in the earlier days of \LaTeX{})
or to generate a smaller document which is easier to navigate.
Another application of |\includeonly| is to generate
documents consisting of selected parts of the complete document.

However, there are a few drawbacks of the plain |\include| mechanism:
\begin{itemize}
\item
The child files cannot be compiled on their own,
they can only be compiled via the main file.
A naive editing environment
(such as a text editor with an option
to have the current file processed by \LaTeX)
may require one to switch to the main file before compiling;
attempting to compile the child file produces errors.
\item
The main file must be modified (each time)
to adjust the |\includeonly| command
to the present needs. This easily leaves the main file in a messy state.
\item
The generated document will always carry the filename
of the main document. This is inconvenient if
several child files are to be compiled and
to be kept for distribution.
\end{itemize}

The present package provides a simple interface
to make child files individually compilable by \LaTeX{}.
Compiling a child file then has the same effect as compiling
the main file with an |\includeonly| command
to select the appropriate child.
Moreover the generated document will carry the name of the child
rather than the main file.
This resolves all three above issues.

This feature is meant to make the editing of books,
thesis documents and lecture notes somewhat more convenient.
However, the package can also be used efficiently for
composing a series of documents (such as exercise sheets)
which are typically distributed individually.
It then assists the author in generating the individual documents
(potentially in different versions)
as well as a document containing the collected series.
Another application is in developing style files
or other kinds of included material
where compilation of the style file could redirect
to a sample or test file.

%%%%%%%%%%%%%%%%%%%%%%%%%%%%%%%%%%%%%%%%%%%%%%%%%%%%%%%%%%%%%%%%%%%%%%%%%%%%%%%%
%%%%%%%%%%%%%%%%%%%%%%%%%%%%%%%%%%%%%%%%%%%%%%%%%%%%%%%%%%%%%%%%%%%%%%%%%%%%%%%%
\section{Usage}

First of all, the package \textsf{childdoc} is \emph{not} a standard
\LaTeXe{} |.sty| style file! Therefore it needs to be invoked in
a non-standard way.

%%%%%%%%%%%%%%%%%%%%%%%%%%%%%%%%%%%%%%%%%%%%%%%%%%%%%%%%%%%%%%%%%%%%%%%%%%%%%%%%
\subsection{Included Files}
\label{sec:include}

%%%%%%%%%%%%%%%%%%%%%%%%%%%%%%%%%%%%%%%%
\DescribeMacro{\childdocmain}
To use the package, add the commands
\begin{center}
\begin{tabular}{l}
|\input{childdoc.def}|\\
|\childdocmain{}|\\
\end{tabular}
\end{center}
at the very top of the main \LaTeX{} file,
in particular \emph{before} the |\documentclass| statement!
The argument of |\childdocmain| should be left empty
(but it must be present).

%%%%%%%%%%%%%%%%%%%%%%%%%%%%%%%%%%%%%%%%
\DescribeMacro{\childdocof}
Furthermore, add the commands
\begin{center}
\begin{tabular}{l}
|\input{childdoc.def}|\\
|\childdocof{|\textit{main}|}|\\
\end{tabular}
\end{center}
at the top of every child file \textit{child}
which is included by |\include{|\textit{child}|}|
from within the main file
(or at least for those files to be compiled individually).
The argument \textit{main} must be the filename of the main file.

There are a couple of
considerations in setting up the main and child documents:

%%%%%%%%%%%%%%%%%%%%%%%%%%%%%%%%%%%%%%%%
\paragraph{Restrictions.}

Please note the following restrictions:
\begin{itemize}
\item
|\childdocmain| must be called with one argument \textit{main}
to ensure compatibility with earlier version of the package.
It must either be empty (|\childdocmain{}|)
or precisely match the filename of the main file in which it is specified.
See \secref{sec:detection} for further information.
\item
The filename \textit{main} must be specified without the |.tex| extension.
\item
The filename \textit{main} is case sensitive
(even in case-insensitive file systems)
due to internal string comparison.
\item
The argument \textit{main} should be fully expanded, it cannot be a macro.
\item
Subdirectories and special characters should be avoided in filenames.
\item
The command |\childdocmain{|\textit{main}|}| must be followed by a whitespace.
It should not be followed immediately by another command
or by a comment mark `|%|'.
This is because the \TeX{} parser reads the token immediately following
the argument of |\childdocmain| and puts it
at the beginning of every child section;
however, a white\-space is ignored.
\end{itemize}

%%%%%%%%%%%%%%%%%%%%%%%%%%%%%%%%%%%%%%%%
\paragraph{Content of Main File.}

It is advisable to place all content in the child files included by |\include|.
Any output contained in the main file will appear in all child documents
unless suppressed manually;
it cannot be suppressed automatically by the |\includeonly| directive
and thus should normally be avoided.
A method to include some content in the main file
by means of conditional processing is described in \secref{sec:conditional}.

%%%%%%%%%%%%%%%%%%%%%%%%%%%%%%%%%%%%%%%%
\paragraph{Page Numbering.}

When only a part of the document is compiled,
the appropriate numbering of pages
(as well as other status parameters)
is determined from the |.aux| files.
The latter contain information from previous passes.
However this information needs to propagate through
all intermediate child documents.
Therefore the page numbering in child documents may well
be inconsistent until the complete document is compiled at least once.

A useful (if unconventional) way to always ensure a consistent
page numbering is to restart the numbering in each child document
and denote the pages by `\textit{child}|.|\textit{page}'
where \textit{child} represents the chapter/section number of the child file.
This can be achieved by the command
|\numberwithin{page}{|\textit{child}|}|
of the \textsf{amsmath} package
where \textit{child} can be |chapter| or |section|
depending on the chosen structuring.
Alternatively, one can modify the macro |\thepage| appropriately
and reset the counter |page| at the start of each child file.

%%%%%%%%%%%%%%%%%%%%%%%%%%%%%%%%%%%%%%%%%%%%%%%%%%%%%%%%%%%%%%%%%%%%%%%%%%%%%%%%
\subsection{Conditional Processing}
\label{sec:conditional}

The package provides a mechanism to compile different versions
of a document. To customise the versions further some conditional processing
can come in handy to distinguish which version is being compiled.
The package provides two macros to describe the compilation context:

%%%%%%%%%%%%%%%%%%%%%%%%%%%%%%%%%%%%%%%%
\DescribeMacro{\ifchilddoc}
The conditional |\ifchilddoc| distinguishes between the compilation of
child documents and the main document:
%
\begin{center}
|\ifchilddoc |\textit{child-code}| |[|\||else |\textit{main-code}]| \||fi|
\end{center}

%%%%%%%%%%%%%%%%%%%%%%%%%%%%%%%%%%%%%%%%
\DescribeMacro{\childdocname}
\DescribeMacro{\childdocjob}
The macro |\childdocname| contains the filename (without extension)
of the main or child file being processed.
Note that |\childdocjob| will always contain the name of the main file.

%%%%%%%%%%%%%%%%%%%%%%%%%%%%%%%%%%%%%%%%
\paragraph{Title Page.}

Conditional processing can be used to include a title or banner page
in the main document when proper precautions are taken.
Importantly, the code in the main file should ensure that the page counter
(as well as other status parameters which are stored in the |.aux| files)
takes the same value after the conditional processing.
Otherwise the page numbers may take divergent values
depending on which part is compiled.

For example, a title page could be declared by:
%
\begin{center}
\begin{tabular}{l}
|\ifchilddoc\||else|\\
|\addtocounter{page}{-1}|\\
\textit{code for title page}\\
|\newpage|\\
|\||fi|
\end{tabular}
\end{center}
%
A banner page for the child documents can be generated by:
%
\begin{center}
\begin{tabular}{l}
|\ifchilddoc|\\
|\addtocounter{page}{-1}|\\
\textit{code for banner page}\\
|\newpage|\\
|\||fi|
\end{tabular}
\end{center}
%
Here one could write a message such as:
\begin{center}
|This is the part \childdocname{} of \childdocjob{}.|
\end{center}

%%%%%%%%%%%%%%%%%%%%%%%%%%%%%%%%%%%%%%%%%%%%%%%%%%%%%%%%%%%%%%%%%%%%%%%%%%%%%%%%
\subsection{Flags}
\label{sec:flags}

The package makes it easy to generate different versions
of the main or child documents.
To this end compilation flags can be defined
and assigned different default values.
They will be particularly useful in conjunction
with the forwarding mechanism described in \secref{sec:forward}.

For example, it may be useful to have a flag |\version|
which can be set to |draft| or |final|.
The document source will contain some conditional code
depending on the value of |\version|.
Suppose further, the flag should default to |final| for the main file
and to |draft| for child files
which is a natural assignment for editing the document.
This is achieved by placing the following code
in the preamble of the main document
(below the |\childdocmain| directive):
%
\begin{center}
\begin{tabular}{l}
|\ifchilddoc|\\
|\providecommand{\version}{draft}|\\
|\||else|\\
|\providecommand{\version}{final}|\\
|\||fi|
\end{tabular}
\end{center}
%
The definition by |\providecommand| makes sure
that previous definitions are not overwritten.
Further statements |\providecommand{\version}{...}|
can thus be added before the above code to override it.

For the main file, one might add a line
(between |\childdocmain| and the above block)
%
\begin{center}
|%\ifchilddoc\||else\providecommand{\version}{draft}\||fi|
\end{center}
%
which can be uncommented to produce a draft version.
Likewise one can add a line to the very top of a child file
(above the |\childdocof{|\textit{main}|}| directive)
%
\begin{center}
|%\providecommand{\version}{final}|
\end{center}
%
which can be uncommented to produce the final version of this child document.

%%%%%%%%%%%%%%%%%%%%%%%%%%%%%%%%%%%%%%%%%%%%%%%%%%%%%%%%%%%%%%%%%%%%%%%%%%%%%%%%
\subsection{Forwarding}
\label{sec:forward}

Different versions of the main or child documents
using compilation flags as described in \secref{sec:flags}
can be (permanently) stored in different files
for convenient compilation, viewing and distribution.
To this end, the package defines a command
to pass on compilation to a different file:

%%%%%%%%%%%%%%%%%%%%%%%%%%%%%%%%%%%%%%%%
\DescribeMacro{\childdocforward}
The command |\childdocforward| redirects processing to
another source file:
%
\begin{center}
\begin{tabular}{l}
|\input{childdoc.def}|\\
|\childdocforward[|\textit{main}|]{|\textit{dest}|}|\\
\end{tabular}
\end{center}
%
The argument \textit{dest} is the destination file
(without extension).
It should be the main file or one of the child files.
Note that further \textsf{childdoc} directives
such as |\childdocof| and |\childdocforward|
in the indicated file will be processed in this form.
The optional argument \textit{main}
passes on directly to the main file \textit{main}
while pretending to compile the child \textit{dest}.
This form behaves as if \textit{dest}
issues |\childdocof{|\textit{main}|}| right away,
and no further \textsf{childdoc} directives will be processed.

%%%%%%%%%%%%%%%%%%%%%%%%%%%%%%%%%%%%%%%%
\DescribeMacro{\...prefix}
In the alternative form |\childdocforwardprefix|,
%
\begin{center}
\begin{tabular}{l}
|\input{childdoc.def}|\\
|\childdocforwardprefix[|\textit{main}|]{|\textit{prefix}|}{|\textit{dest}|}|
\end{tabular}
\end{center}
%
the destination file is determined by a pattern
depending on the current file:
To make this work, the current file must be called
`{\textit{prefix}\hspace{0.2em}\textit{suffix}}'
with \textit{prefix} matching precisely the argument.
Processing is then passed on to the file
`{\textit{dest}\hspace{0.2em}\textit{suffix}}'.
Surely, the same effect is achieved by
directly specifying the
argument `{\textit{dest}\hspace{0.2em}\textit{suffix}}'
in the first form.
However, that requires to set up a different file
for each child. With the alternative form of the command
all these files can have exactly the same content
which simplifies setting them up and maintaining them.

For example, the following file |draft.tex|
with a compilation flag |\version| as described in \secref{sec:flags}
compiles the main document as a draft:
%
\begin{center}
\begin{tabular}{l}
|\def\version{draft}|\\
|\input{childdoc.def}|\\
|\childdocforward{|\textit{main}|}|
\end{tabular}
\end{center}
%
Likewise, the following files |final|\textit{nn}|.tex|
compile the final version of the child document
|child|\textit{nn}|.tex|:
%
\begin{center}
\begin{tabular}{l}
|\def\version{final}|\\
|\input{childdoc.def}|\\
|\childdocforwardprefix{final}{child}|
\end{tabular}
\end{center}
%

Note that when several versions of a main file and/or of each child file
are to be generated, it may be convenient to set up a |Makefile| or
shell script to automatise the process.

%%%%%%%%%%%%%%%%%%%%%%%%%%%%%%%%%%%%%%%%%%%%%%%%%%%%%%%%%%%%%%%%%%%%%%%%%%%%%%%%
\subsection{Command Line Processing}
\label{sec:commandline}

The effect of redirection files can also be achieved by invoking
the \LaTeX{} compiler with a more elaborate command line.
Most conveniently this should be done as part
of a shell script or a |Makefile|.

When using \textsf{childdoc} in the main file, the following
command lines effectively perform a redirection
(note that depending on the shell being used,
backslashes may have to be doubled: `|\|' $\to$ `|\\|'):
%
\begin{center}
|... -jobname "|\textit{target}|" |\\|"|[\textit{flags}]%
|\input{childdoc.def}\childdocforward[|\textit{main}|]{|\textit{dest}|}"|
\end{center}
%
Here \textit{target} is the name of the output file,
\textit{main} is the name of the main file
and \textit{dest} is the name of the main or child file to be processed
(all filenames without extensions).
The optional argument \textit{main} can be omitted
if \textit{main} matches \textit{dest}.
Optionally, compilation \textit{flags} can be defined via |\def| commands.
This command line makes the \TeX{} engine believe
it is compiling the file \textit{target}
whose content is specified as the latter parameter.
The provided code then forwards the processing to
\textit{main} or \textit{dest} as described in \secref{sec:forward}.

%%%%%%%%%%%%%%%%%%%%%%%%%%%%%%%%%%%%%%%%%%%%%%%%%%%%%%%%%%%%%%%%%%%%%%%%%%%%%%%%
\subsection{Include by Input}
\label{sec:input}

Including child documents by |\include| has some restrictions by design.
Most notably, the content of a child document always occupies
its own set of pages; pages cannot be shared between child documents.
Usually, this behaviour makes perfect sense
because each child document contain an essential part of the document.
However, in some situations it may be desirable to compose
a document from a collection of parts
without having mandatory page breaks between then.
For this case, the package
provides a mechanism to include parts
by |\input| which can also be processed individually.
However, by construction this mechanism
requires manual handling of the content to be output.

%%%%%%%%%%%%%%%%%%%%%%%%%%%%%%%%%%%%%%%%
\DescribeMacro{\ifchilddocmanual}
The main file should be prepared as usual, see \secref{sec:include}.
However, the document body must make a distinction
between processing of an individual part and of the main document, e.g.:
%
\begin{center}
\begin{tabular}{l}
|\ifchilddocmanual|\\
|\input{\childdocname}|\\
|\||else|\\
\textit{document body with }|\input{|\textit{part}|}|\\
|\||fi|
\end{tabular}
\end{center}
%
The conditional |\ifchilddocmanual| is true whenever
a part to be included by |\input| is being compiled,
and the name of the part is stored in |\childdocname|.

%%%%%%%%%%%%%%%%%%%%%%%%%%%%%%%%%%%%%%%%
\DescribeMacro{\childdocby}
Each part to be included by |\input| should start with:
%
\begin{center}
\begin{tabular}{l}
|\input{childdoc.def}|\\
|\childdocby{|\textit{main}|}|\\
\end{tabular}
\end{center}
%
The directive |\childdocby| is similar to |\childdocof|
described in \secref{sec:include},
but the subsequent selection of content must be done manually.
To that end, both |\ifchilddoc| and |\ifchilddocmanual|
will be true upon processing of a part,
and the name of the part is stored in |\childdocname|.
Note that |\jobname| will be set to the filename of the current part
so that each part receives an individual |.aux| file
that does not interfere with the |.aux| file(s) of the main document.
This behaviour can be altered by the alternative form
|\childdocby[*]{|\textit{main}|}| (with a non-empty optional argument)
which uses the |.aux| file of the main document
by setting |\jobname| to \textit{main}.

%%%%%%%%%%%%%%%%%%%%%%%%%%%%%%%%%%%%%%%%%%%%%%%%%%%%%%%%%%%%%%%%%%%%%%%%%%%%%%%%
\subsection{Driver Development}
\label{sec:driver}

The \textsf{childdoc} mechanism can also be use for the development
of definition files such as \LaTeX{} styles or classes.
This case differs from the above setup with multiple parts
included by |\include| in that no |\includeonly| should be invoked.
This can be achieved by starting the include file
(before |\ProvidesPackage|) with:
%
\begin{center}
\begin{tabular}{l}
|\input{childdoc.def}|\\
|\childdocforward{|\textit{main}|}|\\
\end{tabular}
\end{center}
%
or alternatively with:
%
\begin{center}
\begin{tabular}{l}
|\input{childdoc.def}|\\
|\childdocby{|\textit{main}|}|\\
\end{tabular}
\end{center}
%
Both forms have slightly different effects as described above.
The main file is prepared as usual, see \secref{sec:include}.

%%%%%%%%%%%%%%%%%%%%%%%%%%%%%%%%%%%%%%%%%%%%%%%%%%%%%%%%%%%%%%%%%%%%%%%%%%%%%%%%
\subsection{Legacy Detection}
\label{sec:detection}

The directive |\childdocmain| in the main file can detect
whether the complete document or merely a child is to be compiled
even without using the directive |\childdocof|.
This method is deprecated because it is less robust
and there is no compelling reason to use it;
it is merely provided for backward compatibility
and it may be removed in future versions.

If the detection mechanism is to be used,
it is mandatory to correctly specify
the filename of the main file as the argument of |\childdocmain|:
%
\begin{center}
\begin{tabular}{l}
|\input{childdoc.def}|\\
|\childdocmain{|\textit{main}|}|\\
\end{tabular}
\end{center}
%
If |\jobname| does not match the argument \textit{main} of |\childdocmain|,
it is assumed that |\jobname| points to the child file to be compiled.
When using |\childdocmain| with the main file specified as argument,
it suffices to start a child file
with just |\input{|\textit{main}|}|
without loading of the package and using |\childdocof|.
If instead all processing is done
with the appropriate \textsf{childdoc} directives,
the argument of \textit{main} of |\childdocmain| can be empty.

An alternative version of the command line processing described
in \secref{sec:commandline} using the detection mechanism reads:
%
\begin{center}
|... -jobname "|\textit{target}|" "|[\textit{flags}]%
[|\def\jobname{|\textit{dest}|}|]|\input{|\textit{main}|}"|
\end{center}

%%%%%%%%%%%%%%%%%%%%%%%%%%%%%%%%%%%%%%%%%%%%%%%%%%%%%%%%%%%%%%%%%%%%%%%%%%%%%%%%
\subsection{Manual Code}
\label{sec:manual}

In case one cannot be certain whether the definitions file |childdoc.def|
is installed on the target \TeX{} distribution
and one prefers not to ship it,
it is conceivable to paste a few relevant commands into the sources.

To that end, drop all statements |\input{childdoc.def}|
and perform the replacements as outlined below.
Instead of |\childdocmain{|\textit{main}|}| add the following code
to the top of the main file:
%
\begin{center}
\begin{tabular}{l}
|\||ifdefined\childdocname\endinput\||fi\newif\ifchilddoc|\\
|\edef\childdocname{\scantokens\expandafter{\jobname\noexpand}}|\\
|\def\childdocmain{|\textit{main}|}\||ifx\childdocmain\childdocname\||else|\\
|\childdoctrue\includeonly{\childdocname}\let\jobname\childdocmain\||fi|\\
\end{tabular}
\end{center}
%
Instead of |\childdocof{|\textit{main}|}| just include the main file
at the top of each child file:
%
\begin{center}
|\input{|\textit{main}|}|
\end{center}
%
A simple redirection |\childdocforward{|\textit{dest}|}| is achieved by:
%
\begin{center}
|\def\jobname{|\textit{dest}|}\input{\jobname}|
\end{center}
%
The redirection with prefix
|\childdocforwardprefix[|\textit{prefix}|]{|\textit{dest}|}|
is accomplished by:
%
\begin{center}
\begin{tabular}{l}
|{\edef\jobname{\scantokens\expandafter{\jobname\noexpand}}|\\
|\def\redirectjob |\textit{prefix}|#1~~~{\gdef\jobname{|\textit{dest}|#1}}|\\
|\expandafter\redirectjob\jobname~~~}\input{\jobname}|
\end{tabular}
\end{center}

In an alternative approach,
child documents can be compiled by a specific command line
without additional code or specific definitions:
%
\begin{center}
|... -jobname "|\textit{target}|" "|[\textit{flags}]%
|\includeonly{|\textit{dest}|}\input{|\textit{main}|}"|
\end{center}
%

%%%%%%%%%%%%%%%%%%%%%%%%%%%%%%%%%%%%%%%%%%%%%%%%%%%%%%%%%%%%%%%%%%%%%%%%%%%%%%%%
%%%%%%%%%%%%%%%%%%%%%%%%%%%%%%%%%%%%%%%%%%%%%%%%%%%%%%%%%%%%%%%%%%%%%%%%%%%%%%%%
\section{Information}

%%%%%%%%%%%%%%%%%%%%%%%%%%%%%%%%%%%%%%%%%%%%%%%%%%%%%%%%%%%%%%%%%%%%%%%%%%%%%%%%
\subsection{Copyright}

Copyright \copyright{} 2017--2018 Niklas Beisert

This work may be distributed and/or modified under the
conditions of the \LaTeX{} Project Public License, either version 1.3
of this license or (at your option) any later version.
The latest version of this license is in
  \url{http://www.latex-project.org/lppl.txt}
and version 1.3 or later is part of all distributions of \LaTeX{}
version 2005/12/01 or later.

This work has the LPPL maintenance status `maintained'.

The Current Maintainer of this work is Niklas Beisert.

This work consists of the files |README.txt|, |childdoc.ins| and |childdoc.dtx|
as well as the derived files |childdoc.def|, |cdocsamp.tex|
with |cdocsch1.tex|, |cdocsch2.tex|, |cdocspt3.tex|, |cdocspt4.tex|,
|cdocsdrf.tex|, |cdocsfn1.tex|, |cdocsfn2.tex|
as well as |childdoc.pdf|.

%%%%%%%%%%%%%%%%%%%%%%%%%%%%%%%%%%%%%%%%%%%%%%%%%%%%%%%%%%%%%%%%%%%%%%%%%%%%%%%%
\subsection{Files and Installation}

The package consists of the files:
%
\begin{center}
\begin{tabular}{ll}
    |README.txt|   & readme file \\
    |childdoc.ins| & installation file \\
    |childdoc.dtx| & source file \\
    |childdoc.def| & definition file \\
    |cdocsamp.tex| & sample main file \\
    |cdocsch1.tex| & sample include file \\
    |cdocsch2.tex| & sample include file \\
    |cdocspt3.tex| & sample part file \\
    |cdocspt4.tex| & sample part file \\
    |cdocsdrf.tex| & sample redirection file \\
    |cdocsfn1.tex| & sample redirection file \\
    |cdocsfn2.tex| & sample redirection file \\
    |childdoc.pdf| & manual
\end{tabular}
\end{center}
%
The distribution consists of the files
|README.txt|, |childdoc.ins| and |childdoc.dtx|.
%
\begin{itemize}
\item
Run (pdf)\LaTeX{} on |childdoc.dtx|
to compile the manual |childdoc.pdf| (this file).
\item
Run \LaTeX{} on |childdoc.ins| to create the definitions file |childdoc.def|
and the sample |cdocsamp.tex| with include files
|cdocsch1.tex|, |cdocsch2.tex|, |cdocspt3.tex|, |cdocspt4.tex|,
|cdocsdrf.tex|, |cdocsfn1.tex|, |cdocsfn2.tex|.
Then copy the file |childdoc.def| to an appropriate directory of your \LaTeX{}
distribution, e.g.\ \textit{texmf-root}|/tex/latex/childdoc|.
\end{itemize}

%%%%%%%%%%%%%%%%%%%%%%%%%%%%%%%%%%%%%%%%%%%%%%%%%%%%%%%%%%%%%%%%%%%%%%%%%%%%%%%%
\subsection{Related CTAN Packages}

There are several other packages which offer a similar functionality:
%
\begin{itemize}
\item
The packages
\href{http://ctan.org/pkg/docmute}{\textsf{docmute}},
\href{http://ctan.org/pkg/includex}{\textsf{includex}} and
\href{http://ctan.org/pkg/standalone}{\textsf{standalone}}
provide commands to include only the document body of
a child file thus allowing both files to be compiled individually.
\item
The packages \href{http://ctan.org/pkg/subdocs}{\textsf{subdocs}}
and \href{http://ctan.org/pkg/subfiles}{\textsf{subfiles}}
provide structures in which the main and child documents can be
encapsulated and allowing them to be compiled individually.
The inclusion mechanism is different from the conventional |\include|.
\item
The package \href{http://ctan.org/pkg/combine}{\textsf{combine}}
is an elaborate solution to combine several documents into one.
\end{itemize}
%
See also the CTAN topic \href{http://ctan.org/topic/subdocs}{\textsf{subdocs}}
for further related packages.
The present package differs from the above solutions in that
a document structure constructed with the conventional |\include| mechanism
just needs two extra commands at the top of every file
such that all constituent files can be compiled individually.

%%%%%%%%%%%%%%%%%%%%%%%%%%%%%%%%%%%%%%%%%%%%%%%%%%%%%%%%%%%%%%%%%%%%%%%%%%%%%%%%
%\subsection{Feature Suggestions}
%
%The following is a list of features which may be useful for future
%versions of this package:
%%
%\begin{itemize}
%\item
%\ldots
%\end{itemize}

%%%%%%%%%%%%%%%%%%%%%%%%%%%%%%%%%%%%%%%%%%%%%%%%%%%%%%%%%%%%%%%%%%%%%%%%%%%%%%%%
\subsection{Revision History}

%%%%%%%%%%%%%%%%%%%%%%%%%%%%%%%%%%%%%%%%
\paragraph{v2.0:} 2018/12/30

\begin{itemize}
\item
immediate forward processing
\item
added |\childdocby| mechanism
\item
manual restructured
\end{itemize}

%%%%%%%%%%%%%%%%%%%%%%%%%%%%%%%%%%%%%%%%
\paragraph{v1.6:} 2018/01/17

\begin{itemize}
\item
application for development of include files
\item
corrections to manual
\end{itemize}

%%%%%%%%%%%%%%%%%%%%%%%%%%%%%%%%%%%%%%%%
\paragraph{v1.5:} 2017/05/21

\begin{itemize}
\item
more complete structuring introduced
\item
|\childdocof| introduced
\item
|\childdoc| renamed to |\childdocmain|
\item
|\childredirect| renamed to |\childdocforward| and |\childdocforwardprefix|
and functionality expanded
\end{itemize}

%%%%%%%%%%%%%%%%%%%%%%%%%%%%%%%%%%%%%%%%
\paragraph{v1.0:} 2017/04/27

\begin{itemize}
\item
manual and install package
\item
first version published on CTAN
\end{itemize}

%%%%%%%%%%%%%%%%%%%%%%%%%%%%%%%%%%%%%%%%
\paragraph{v0.6:} 2017/04/26

\begin{itemize}
\item
redirection mechanism added
\end{itemize}

%%%%%%%%%%%%%%%%%%%%%%%%%%%%%%%%%%%%%%%%
\paragraph{v0.5:} 2017/04/26

\begin{itemize}
\item
functionality in definition file
\end{itemize}


%%%%%%%%%%%%%%%%%%%%%%%%%%%%%%%%%%%%%%%%%%%%%%%%%%%%%%%%%%%%%%%%%%%%%%%%%%%%%%%%
%%%%%%%%%%%%%%%%%%%%%%%%%%%%%%%%%%%%%%%%%%%%%%%%%%%%%%%%%%%%%%%%%%%%%%%%%%%%%%%%
%%%%%%%%%%%%%%%%%%%%%%%%%%%%%%%%%%%%%%%%%%%%%%%%%%%%%%%%%%%%%%%%%%%%%%%%%%%%%%%%
\appendix

\settowidth\MacroIndent{\rmfamily\scriptsize 000\ }

 \DocInput{childdoc.dtx}

\end{document}
%</driver>
% \fi
%
% %%%%%%%%%%%%%%%%%%%%%%%%%%%%%%%%%%%%%%%%%%%%%%%%%%%%%%%%%%%%%%%%%%%%%%%%%%%%%%
% %%%%%%%%%%%%%%%%%%%%%%%%%%%%%%%%%%%%%%%%%%%%%%%%%%%%%%%%%%%%%%%%%%%%%%%%%%%%%%
% \section{Sample}
%\iffalse
%<*samplemain>
%\fi
%
% The following presents a sample document
% with two chapters, two parts, a title page,
% a compile flag as well as three forwarding files to set the flag.
% It consists of eight |.tex| files:
% \begin{center}
% \begin{tabular}{ll}
% |cdocsamp.tex|&main file\\
% |cdocsch1.tex|&include file for chapter 1\\
% |cdocsch2.tex|&include file for chapter 2\\
% |cdocspt3.tex|&include file for part 3\\
% |cdocspt4.tex|&include file for part 4\\
% |cdocsdrf.tex|&forwarding file for main file in draft mode\\
% |cdocsfi1.tex|&forwarding file for final version of chapter 1\\
% |cdocsfi2.tex|&forwarding file for final version of chapter 2\\
% \end{tabular}
% \end{center}
% Each of the eight files can be compiled directly by the \LaTeX{} compiler.
%
% %%%%%%%%%%%%%%%%%%%%%%%%%%%%%%%%%%%%%%
% \paragraph{Main File.}
%
% The main file is called |cdocsamp.tex|.
%
% Load the \textsf{childdoc} definitions and
% declare the filename for the main document:
%    \begin{macrocode}
\input{childdoc.def}
\childdocmain{}
%    \end{macrocode}

% Optional override for |\version| flag:
%    \begin{macrocode}
%%\ifchilddoc\else\providecommand{\version}{draft}\fi
%    \end{macrocode}

% Define the default values for the |\version| flag
% (|final| for the main file and |draft| for childs):
%    \begin{macrocode}
\ifchilddoc
\providecommand{\version}{draft}
\else
\providecommand{\version}{final}
\fi
%    \end{macrocode}

% Load the standard document class:
%    \begin{macrocode}
\documentclass[12pt]{article}
%    \end{macrocode}

% Start the document body:
%    \begin{macrocode}
\begin{document}
%    \end{macrocode}

% Declare a title page.
% Print title, part of document being processed and version flag:
%    \begin{macrocode}
\addtocounter{page}{-1}
\begin{center}
{\LARGE\bfseries{}childdoc example\par}
\vspace{1cm}
\ifchilddoc
\ifchilddocmanual part\else chapter\fi:
`\childdocname' of `\childdocjob'\par
\else
main document: `\childdocjob'\par
\fi
version: \version\par
\end{center}
\newpage
%    \end{macrocode}

% Manually include selected file,
% otherwise process as usual:
%    \begin{macrocode}
\ifchilddocmanual
\section*{part `\childdocname'}
\input{\childdocname}
\else
%    \end{macrocode}

% Include the two chapters:
%    \begin{macrocode}
\include{cdocsch1}
\include{cdocsch2}
%    \end{macrocode}

% Include the two parts unless only chapters should be displayed:
%    \begin{macrocode}
\ifchilddoc\else
\section{part three}
\input{cdocspt3}
\section{part four}
\input{cdocspt4}
\fi
%    \end{macrocode}

% Process as usual until here:
%    \begin{macrocode}
\fi
%    \end{macrocode}

% End of document body:
%    \begin{macrocode}
\end{document}
%    \end{macrocode}
%\iffalse
%</samplemain>
%\fi
%
% %%%%%%%%%%%%%%%%%%%%%%%%%%%%%%%%%%%%%%
% \paragraph{Chapter Include Files.}
%
% The include files are called |cdocsch1.tex| and |cdocsch2.tex|.
%
%\iffalse
%<*samplechap1|samplechap2>
%\fi

% Optional override for |\version| flag:
%    \begin{macrocode}
%%\providecommand{\version}{final}
%    \end{macrocode}

% Include the main document:
%    \begin{macrocode}
\input{childdoc.def}
\childdocof{cdocsamp}
%    \end{macrocode}

%\iffalse
%</samplechap1|samplechap2>
%\fi
%
%\iffalse
%<*samplechap1>
%\fi
% Some text for chapter 1:
%    \begin{macrocode}
\section{one}
some text in chapter one
%    \end{macrocode}

%\iffalse
%</samplechap1>
%\fi
% Some text for chapter 2:
%\iffalse
%<*samplechap2>
%\fi
%    \begin{macrocode}
\section{two}
more text in chapter two
%    \end{macrocode}

%\iffalse
%</samplechap2>
%\fi
%
% %%%%%%%%%%%%%%%%%%%%%%%%%%%%%%%%%%%%%%
% \paragraph{Part Include Files.}
%
% The include files are called |cdocspt3.tex| and |cdocspt4.tex|.
%
%\iffalse
%<*samplepart3|samplepart4>
%\fi

% Optional override for |\version| flag:
%    \begin{macrocode}
%%\providecommand{\version}{final}
%    \end{macrocode}

% Include the main document:
%    \begin{macrocode}
\input{childdoc.def}
\childdocby{cdocsamp}
%    \end{macrocode}

%\iffalse
%</samplepart3|samplepart4>
%\fi
%
%\iffalse
%<*samplepart3>
%\fi
% Some text for part 3:
%    \begin{macrocode}
some text in part three
%    \end{macrocode}

%\iffalse
%</samplepart3>
%\fi
% Some text for part 4:
%\iffalse
%<*samplepart4>
%\fi
%    \begin{macrocode}
more text in part four
%    \end{macrocode}

%\iffalse
%</samplepart4>
%\fi
%
% %%%%%%%%%%%%%%%%%%%%%%%%%%%%%%%%%%%%%%
% \paragraph{Forwarding for a Complete Draft.}
%
% The following forwarding file |cdocsdrf.tex|
% compiles the main document in draft mode:
%\iffalse
%<*sampledraft>
%\fi
%    \begin{macrocode}
\def\version{draft}
\input{childdoc.def}
\childdocforward{cdocsamp}
%    \end{macrocode}

%\iffalse
%</sampledraft>
%\fi
%
% %%%%%%%%%%%%%%%%%%%%%%%%%%%%%%%%%%%%%%
% \paragraph{Forwarding for Final Version of the Chapters.}
%
% The following forwarding files |cdocsfn1.tex| and |cdocsfn2.tex|
% (with identical content)
% compile the final versions of the child documents
% |cdocsch1.tex| and |cdocsch2.tex|, respectively:
%\iffalse
%<*samplefinal>
%\fi
%    \begin{macrocode}
\def\version{final}
\input{childdoc.def}
\childdocforwardprefix[cdocsamp]{cdocsfn}{cdocsch}
%    \end{macrocode}

%\iffalse
%</samplefinal>
%\fi
%
% %%%%%%%%%%%%%%%%%%%%%%%%%%%%%%%%%%%%%%
% \paragraph{Command Line Processing.}
%
% The following three command lines generate the output files
% |cdocscld|, |cdocscl1| and |cdocscl2|
% which should be identical to
% |cdocsdrf|, |cdocsch1| and |cdocsfn2|, respectively:
% \begin{center}
% \begin{tabular}{l}
% |latex -jobname cdocscld \|\\
% |  "\def\version{draft}\input{childdoc.def}\childdocforward{cdocsamp}"|\\
% |latex -jobname cdocscl1 \|\\
% |  "\input{childdoc.def}\childdocforward[cdocsamp]{cdocsch1}"|\\
% |latex -jobname cdocscl2 \|\\
% |  "\def\version{final}\input{childdoc.def}\childdocforward{cdocsch2}"|
% \end{tabular}
% \end{center}
% Note that the trailing backslash on each first line
% merely continues the input to the second line
% (for convenient cut ant paste).
% Furthermore, the command |latex| can be replaced by any
% of its alternative versions such as |pdflatex|.
%
% %%%%%%%%%%%%%%%%%%%%%%%%%%%%%%%%%%%%%%%%%%%%%%%%%%%%%%%%%%%%%%%%%%%%%%%%%%%%%%
% %%%%%%%%%%%%%%%%%%%%%%%%%%%%%%%%%%%%%%%%%%%%%%%%%%%%%%%%%%%%%%%%%%%%%%%%%%%%%%
% \section{Implementation}
%\iffalse
%<*package>
%\fi
%
% This section describes the definitions file |childdoc.def|.

% The definitions cannot be loaded using |\usepackage| or |\RequirePackage|
% which has a mechanism to prevent loading a style file more than once.
% When loading the definitions by means of |\input|
% multiple instances have to be prevented manually:
%\iffalse
%This code needs to be before the `\ProvidesFile' directive
%which is defined at the beginning of this file.
%Therefore it is also placed there and commented out here.
%</package>
%<*discard>
%\fi
%    \begin{macrocode}
\ifdefined\childdocmain\endinput\fi
%    \end{macrocode}
%\iffalse
%</discard>
%<*package>
%\fi
%
% \macro{\ifchilddoc}
% \macro{\ifchilddocmanual}
% The conditional |\ifchilddoc| tells whether a
% child (true) or main (false) document is being compiled.
% The conditional |\ifchilddocmanual| tells whether
% the |\includeonly| mechanism is used (false) or
% the selection of child files must be performed manually (true).
% The definitions initialise to false:
%    \begin{macrocode}
\newif\ifchilddoc
\newif\ifchilddocmanual
%    \end{macrocode}

% \macro{\childdocname}
% \macro{\childdocjob}
% The macro |\childdocname| stores the name of the main document
% to be compiled. The macro |\childdocjob| stores the name of
% the document on which the \LaTeX{} compiler was originally invoked.
% The content of |\jobname| cannot be compared
% to filenames specified in the source due to different catcodes.
% The following code rescans |\jobname|, stores the result
% in |\childdocname| and saves a copy in |\childdocjob|:
%    \begin{macrocode}
\edef\childdocname{\scantokens\expandafter{\jobname\noexpand}}
\let\childdocjob\childdocname
%    \end{macrocode}

% \macro{\childdocdisable}
% The macro |\childdocdisable| prevents the main file
% from being processed more than once.
% At this stage, the main document command |\childdocmain|
% is assumed to be called once again where it should do nothing.
% Any subsequent call to it should prevent
% a secondary processing of the main document
% It overwrites the forwarding commands
% |\childdocof| and |\childdocforward|
% with empty macros to prevent further inclusions of the main document:
%    \begin{macrocode}
\newcommand{\childdocdisable}
{
  \renewcommand{\childdocmain}[1]{\renewcommand{\childdocmain}[1]{\endinput}}
  \renewcommand{\childdocof}[1]{}
  \renewcommand{\childdocby}[2][]{}
  \renewcommand{\childdocforward}[2][]{}
  \renewcommand{\childdocdisable}{}
}
%    \end{macrocode}

% \macro{\childdocmain}
% The macro |\childdocmain| is to be called at the top of the main file
% with nothing or the main filename (without extension) as argument.
% First, it breaks loops.
% If the argument is not empty and does not match |\childdocname|
% (which is set by the first inclusion of |childdoc.def|),
% |\ifchilddoc| is set to true, |\includeonly| is applied to the child file
% and |\jobname| is set to the main file
% (for proper handling of |.aux| files):
%    \begin{macrocode}
\newcommand{\childdocmain}[1]
{
  \childdocdisable\childdocmain{}
  \if?#1?\else
    \begingroup
      \def\childdoctmp{#1}
      \ifx\childdoctmp\childdocname
        \def\childdoctmp{}
      \else
        \def\childdoctmp
        {
          \childdoctrue
          \includeonly{\childdocname}
          \def\childdocjob{#1}
          \def\jobname{#1}
        }
      \fi
      \expandafter
    \endgroup
    \childdoctmp
  \fi
}
%    \end{macrocode}

% \macro{\childdocof}
% The command |\childdocof| redirects
% compilation to the main file |#1|.
%    \begin{macrocode}
\newcommand{\childdocof}[1]
{
  \childdocdisable
  \childdoctrue
  \includeonly{\childdocname}
  \def\jobname{#1}
  \def\childdocjob{#1}
  \input{#1}
}
%    \end{macrocode}

% \macro{\childdocby}
% The command |\childdocby| ....
%    \begin{macrocode}
\newcommand{\childdocby}[2][]
{
  \childdocdisable
  \childdoctrue
  \childdocmanualtrue
  \if?#1?\else
    \def\jobname{#2}
  \fi
  \def\childdocjob{#2}
  \input{#2}
  \endinput
}
%    \end{macrocode}

% \macro{\childdocforward}
% The command |\childdocforward| redirects
% compilation to the main file or
% (if the optional argument is given) a child file.
% Parameters are set as if the main file
% or a child file starting with |\childdocof| was compiled.
% Then compilation is handed over to the main file:
%    \begin{macrocode}
\newcommand{\childdocforward}[2][]
{
  \begingroup
    \if?#1?
      \def\childdoctmp
      {
        \def\childdocname{#2}
        \def\childdocjob{#2}
        \def\jobname{#2}
        \input{#2}
        \endinput
      }
    \else
      \def\childdoctmp
      {
        \childdocdisable
        \def\childdocname{#2}
        \childdoctrue
        \includeonly{#2}
        \def\childdocjob{#1}
        \def\jobname{#1}
        \input{#1}
        \endinput
      }
    \fi
    \expandafter
  \endgroup
  \childdoctmp
}
%    \end{macrocode}

% \macro{\childdocforwardprefix}
% The command |\childdocforwardprefix| redirects
% compilation to the main or a child file by means of a pattern.
% The prefix |#1| in the current filename is replaced by |#2|
% and the suffix of the current filename is kept
% (it is assumed that the filename does not contain the substring `|~~~|'
% which is used as a delimiter).
% Compilation is handed over to the new file by |\childdocforward|:
%    \begin{macrocode}
\newcommand{\childdocforwardprefix}[3][]
{
  \begingroup
    \def\childdocextract #2##1~~~{\def\childdoctmp{\childdocforward[#1]{#3##1}}}
    \expandafter\childdocextract\childdocname~~~
    \expandafter
  \endgroup
  \childdoctmp
}
%    \end{macrocode}

% \macro{\childdoc}
% The deprecated macro |\childdoc| is a legacy version of |\childdocmain|:
%    \begin{macrocode}
\newcommand{\childdoc}{\childdocmain}
%    \end{macrocode}

% \macro{\childdocredirect}
% The deprecated macro |\childdocredirect| is a legacy version
% of |\childdocforward| and |\childdocforwardprefix|:
%    \begin{macrocode}
\newcommand{\childdocredirect}[2][]
{
  \begingroup
    \if?#1?
      \def\childdoctmp{\childdocforward{#2}}
    \else
      \def\childdoctmp{\childdocforwardprefix{#1}{#2}}
    \fi
    \expandafter
  \endgroup
  \childdoctmp
}
%    \end{macrocode}

%\iffalse
%</package>
%\fi
%
\endinput

\childdocby{cdocsamp}
%    \end{macrocode}

%\iffalse
%</samplepart3|samplepart4>
%\fi
%
%\iffalse
%<*samplepart3>
%\fi
% Some text for part 3:
%    \begin{macrocode}
some text in part three
%    \end{macrocode}

%\iffalse
%</samplepart3>
%\fi
% Some text for part 4:
%\iffalse
%<*samplepart4>
%\fi
%    \begin{macrocode}
more text in part four
%    \end{macrocode}

%\iffalse
%</samplepart4>
%\fi
%
% %%%%%%%%%%%%%%%%%%%%%%%%%%%%%%%%%%%%%%
% \paragraph{Forwarding for a Complete Draft.}
%
% The following forwarding file |cdocsdrf.tex|
% compiles the main document in draft mode:
%\iffalse
%<*sampledraft>
%\fi
%    \begin{macrocode}
\def\version{draft}
% \iffalse
%
% childdoc.dtx Copyright (C) 2017-2018 Niklas Beisert
%
% This work may be distributed and/or modified under the
% conditions of the LaTeX Project Public License, either version 1.3
% of this license or (at your option) any later version.
% The latest version of this license is in
%   http://www.latex-project.org/lppl.txt
% and version 1.3 or later is part of all distributions of LaTeX
% version 2005/12/01 or later.
%
% This work has the LPPL maintenance status `maintained'.
%
% The Current Maintainer of this work is Niklas Beisert.
%
% This work consists of the files childdoc.dtx and childdoc.ins
% and the derived files childdoc.def and cdocsamp.tex with
% cdocsch1.tex, cdocsch2.tex, cdocsdrf.tex, cdocsfn1.tex, cdocsfn2.tex.
%
%<package>\ifdefined\childdocmain\endinput\fi
%<package>\ProvidesFile{childdoc.def}[2018/12/30 v2.0 child document driver]
%<samplemain>\ProvidesFile{cdocsamp.tex}[2018/12/30 v2.0 sample for childdoc]
%<*driver>
%\ProvidesFile{childdoc.drv}[2018/12/30 v2.0 childdoc reference manual file]
\PassOptionsToClass{10pt,a4paper}{article}
\documentclass{ltxdoc}

\usepackage[margin=35mm]{geometry}
\usepackage{hyperref}
\usepackage{hyperxmp}
\usepackage[usenames]{color}

\hypersetup{colorlinks=true}
\hypersetup{pdfstartview=FitH}
\hypersetup{pdfpagemode=UseNone}
\hypersetup{pdfsource={}}
\hypersetup{pdflang={en-UK}}
\hypersetup{pdfcopyright={Copyright 2017-2018 Niklas Beisert.
  This work may be distributed and/or modified under the
  conditions of the LaTeX Project Public License, either version 1.3
  of this license or (at your option) any later version.}}
\hypersetup{pdflicenseurl={http://www.latex-project.org/lppl.txt}}
\hypersetup{pdfcontactaddress={ETH Zurich, ITP, HIT K,
  Wolfgang-Pauli-Strasse 27}}
\hypersetup{pdfcontactpostcode={8093}}
\hypersetup{pdfcontactcity={Zurich}}
\hypersetup{pdfcontactcountry={Switzerland}}
\hypersetup{pdfcontactemail={nbeisert@itp.phys.ethz.ch}}
\hypersetup{pdfcontacturl={http://people.phys.ethz.ch/\xmptilde nbeisert/}}

\newcommand{\secref}[1]{\hyperref[#1]{section \ref*{#1}}}

\parskip1ex
\parindent0pt
\let\olditemize\itemize
\def\itemize{\olditemize\parskip0pt}

\begin{document}

\title{The \textsf{childdoc} Package}
\hypersetup{pdftitle={The childdoc Package}}
\author{Niklas Beisert\\[2ex]
  Institut f\"ur Theoretische Physik\\
  Eidgen\"ossische Technische Hochschule Z\"urich\\
  Wolfgang-Pauli-Strasse 27, 8093 Z\"urich, Switzerland\\[1ex]
  \href{mailto:nbeisert@itp.phys.ethz.ch}
  {\texttt{nbeisert@itp.phys.ethz.ch}}}
\hypersetup{pdfauthor={Niklas Beisert}}
\hypersetup{pdfsubject={Manual for the LaTeX2e Package childdoc}}
\date{30 December 2018, \textsf{v2.0}}
\maketitle

\begin{abstract}\noindent
\textsf{childdoc} is a \LaTeXe{} package
that enables the direct compilation
of document sections included by |\include|
to individual files.
\end{abstract}

\begingroup
\parskip0ex
\tableofcontents
\endgroup

%%%%%%%%%%%%%%%%%%%%%%%%%%%%%%%%%%%%%%%%%%%%%%%%%%%%%%%%%%%%%%%%%%%%%%%%%%%%%%%%
%%%%%%%%%%%%%%%%%%%%%%%%%%%%%%%%%%%%%%%%%%%%%%%%%%%%%%%%%%%%%%%%%%%%%%%%%%%%%%%%
\section{Introduction}

\LaTeX{} provides a mechanism to structure a large document (such as a book)
into a main file and several child files (containing the chapters)
using the |\include| command.
This mechanism is beneficial for documents
which span hundreds of pages in order to
make the source file(s) more manageable.
Moreover, compilation can be restricted to
selected child files by means of the |\includeonly| command.
The latter feature can be used to reduce the compilation time while editing
(this was significantly more useful in the earlier days of \LaTeX{})
or to generate a smaller document which is easier to navigate.
Another application of |\includeonly| is to generate
documents consisting of selected parts of the complete document.

However, there are a few drawbacks of the plain |\include| mechanism:
\begin{itemize}
\item
The child files cannot be compiled on their own,
they can only be compiled via the main file.
A naive editing environment
(such as a text editor with an option
to have the current file processed by \LaTeX)
may require one to switch to the main file before compiling;
attempting to compile the child file produces errors.
\item
The main file must be modified (each time)
to adjust the |\includeonly| command
to the present needs. This easily leaves the main file in a messy state.
\item
The generated document will always carry the filename
of the main document. This is inconvenient if
several child files are to be compiled and
to be kept for distribution.
\end{itemize}

The present package provides a simple interface
to make child files individually compilable by \LaTeX{}.
Compiling a child file then has the same effect as compiling
the main file with an |\includeonly| command
to select the appropriate child.
Moreover the generated document will carry the name of the child
rather than the main file.
This resolves all three above issues.

This feature is meant to make the editing of books,
thesis documents and lecture notes somewhat more convenient.
However, the package can also be used efficiently for
composing a series of documents (such as exercise sheets)
which are typically distributed individually.
It then assists the author in generating the individual documents
(potentially in different versions)
as well as a document containing the collected series.
Another application is in developing style files
or other kinds of included material
where compilation of the style file could redirect
to a sample or test file.

%%%%%%%%%%%%%%%%%%%%%%%%%%%%%%%%%%%%%%%%%%%%%%%%%%%%%%%%%%%%%%%%%%%%%%%%%%%%%%%%
%%%%%%%%%%%%%%%%%%%%%%%%%%%%%%%%%%%%%%%%%%%%%%%%%%%%%%%%%%%%%%%%%%%%%%%%%%%%%%%%
\section{Usage}

First of all, the package \textsf{childdoc} is \emph{not} a standard
\LaTeXe{} |.sty| style file! Therefore it needs to be invoked in
a non-standard way.

%%%%%%%%%%%%%%%%%%%%%%%%%%%%%%%%%%%%%%%%%%%%%%%%%%%%%%%%%%%%%%%%%%%%%%%%%%%%%%%%
\subsection{Included Files}
\label{sec:include}

%%%%%%%%%%%%%%%%%%%%%%%%%%%%%%%%%%%%%%%%
\DescribeMacro{\childdocmain}
To use the package, add the commands
\begin{center}
\begin{tabular}{l}
|\input{childdoc.def}|\\
|\childdocmain{}|\\
\end{tabular}
\end{center}
at the very top of the main \LaTeX{} file,
in particular \emph{before} the |\documentclass| statement!
The argument of |\childdocmain| should be left empty
(but it must be present).

%%%%%%%%%%%%%%%%%%%%%%%%%%%%%%%%%%%%%%%%
\DescribeMacro{\childdocof}
Furthermore, add the commands
\begin{center}
\begin{tabular}{l}
|\input{childdoc.def}|\\
|\childdocof{|\textit{main}|}|\\
\end{tabular}
\end{center}
at the top of every child file \textit{child}
which is included by |\include{|\textit{child}|}|
from within the main file
(or at least for those files to be compiled individually).
The argument \textit{main} must be the filename of the main file.

There are a couple of
considerations in setting up the main and child documents:

%%%%%%%%%%%%%%%%%%%%%%%%%%%%%%%%%%%%%%%%
\paragraph{Restrictions.}

Please note the following restrictions:
\begin{itemize}
\item
|\childdocmain| must be called with one argument \textit{main}
to ensure compatibility with earlier version of the package.
It must either be empty (|\childdocmain{}|)
or precisely match the filename of the main file in which it is specified.
See \secref{sec:detection} for further information.
\item
The filename \textit{main} must be specified without the |.tex| extension.
\item
The filename \textit{main} is case sensitive
(even in case-insensitive file systems)
due to internal string comparison.
\item
The argument \textit{main} should be fully expanded, it cannot be a macro.
\item
Subdirectories and special characters should be avoided in filenames.
\item
The command |\childdocmain{|\textit{main}|}| must be followed by a whitespace.
It should not be followed immediately by another command
or by a comment mark `|%|'.
This is because the \TeX{} parser reads the token immediately following
the argument of |\childdocmain| and puts it
at the beginning of every child section;
however, a white\-space is ignored.
\end{itemize}

%%%%%%%%%%%%%%%%%%%%%%%%%%%%%%%%%%%%%%%%
\paragraph{Content of Main File.}

It is advisable to place all content in the child files included by |\include|.
Any output contained in the main file will appear in all child documents
unless suppressed manually;
it cannot be suppressed automatically by the |\includeonly| directive
and thus should normally be avoided.
A method to include some content in the main file
by means of conditional processing is described in \secref{sec:conditional}.

%%%%%%%%%%%%%%%%%%%%%%%%%%%%%%%%%%%%%%%%
\paragraph{Page Numbering.}

When only a part of the document is compiled,
the appropriate numbering of pages
(as well as other status parameters)
is determined from the |.aux| files.
The latter contain information from previous passes.
However this information needs to propagate through
all intermediate child documents.
Therefore the page numbering in child documents may well
be inconsistent until the complete document is compiled at least once.

A useful (if unconventional) way to always ensure a consistent
page numbering is to restart the numbering in each child document
and denote the pages by `\textit{child}|.|\textit{page}'
where \textit{child} represents the chapter/section number of the child file.
This can be achieved by the command
|\numberwithin{page}{|\textit{child}|}|
of the \textsf{amsmath} package
where \textit{child} can be |chapter| or |section|
depending on the chosen structuring.
Alternatively, one can modify the macro |\thepage| appropriately
and reset the counter |page| at the start of each child file.

%%%%%%%%%%%%%%%%%%%%%%%%%%%%%%%%%%%%%%%%%%%%%%%%%%%%%%%%%%%%%%%%%%%%%%%%%%%%%%%%
\subsection{Conditional Processing}
\label{sec:conditional}

The package provides a mechanism to compile different versions
of a document. To customise the versions further some conditional processing
can come in handy to distinguish which version is being compiled.
The package provides two macros to describe the compilation context:

%%%%%%%%%%%%%%%%%%%%%%%%%%%%%%%%%%%%%%%%
\DescribeMacro{\ifchilddoc}
The conditional |\ifchilddoc| distinguishes between the compilation of
child documents and the main document:
%
\begin{center}
|\ifchilddoc |\textit{child-code}| |[|\||else |\textit{main-code}]| \||fi|
\end{center}

%%%%%%%%%%%%%%%%%%%%%%%%%%%%%%%%%%%%%%%%
\DescribeMacro{\childdocname}
\DescribeMacro{\childdocjob}
The macro |\childdocname| contains the filename (without extension)
of the main or child file being processed.
Note that |\childdocjob| will always contain the name of the main file.

%%%%%%%%%%%%%%%%%%%%%%%%%%%%%%%%%%%%%%%%
\paragraph{Title Page.}

Conditional processing can be used to include a title or banner page
in the main document when proper precautions are taken.
Importantly, the code in the main file should ensure that the page counter
(as well as other status parameters which are stored in the |.aux| files)
takes the same value after the conditional processing.
Otherwise the page numbers may take divergent values
depending on which part is compiled.

For example, a title page could be declared by:
%
\begin{center}
\begin{tabular}{l}
|\ifchilddoc\||else|\\
|\addtocounter{page}{-1}|\\
\textit{code for title page}\\
|\newpage|\\
|\||fi|
\end{tabular}
\end{center}
%
A banner page for the child documents can be generated by:
%
\begin{center}
\begin{tabular}{l}
|\ifchilddoc|\\
|\addtocounter{page}{-1}|\\
\textit{code for banner page}\\
|\newpage|\\
|\||fi|
\end{tabular}
\end{center}
%
Here one could write a message such as:
\begin{center}
|This is the part \childdocname{} of \childdocjob{}.|
\end{center}

%%%%%%%%%%%%%%%%%%%%%%%%%%%%%%%%%%%%%%%%%%%%%%%%%%%%%%%%%%%%%%%%%%%%%%%%%%%%%%%%
\subsection{Flags}
\label{sec:flags}

The package makes it easy to generate different versions
of the main or child documents.
To this end compilation flags can be defined
and assigned different default values.
They will be particularly useful in conjunction
with the forwarding mechanism described in \secref{sec:forward}.

For example, it may be useful to have a flag |\version|
which can be set to |draft| or |final|.
The document source will contain some conditional code
depending on the value of |\version|.
Suppose further, the flag should default to |final| for the main file
and to |draft| for child files
which is a natural assignment for editing the document.
This is achieved by placing the following code
in the preamble of the main document
(below the |\childdocmain| directive):
%
\begin{center}
\begin{tabular}{l}
|\ifchilddoc|\\
|\providecommand{\version}{draft}|\\
|\||else|\\
|\providecommand{\version}{final}|\\
|\||fi|
\end{tabular}
\end{center}
%
The definition by |\providecommand| makes sure
that previous definitions are not overwritten.
Further statements |\providecommand{\version}{...}|
can thus be added before the above code to override it.

For the main file, one might add a line
(between |\childdocmain| and the above block)
%
\begin{center}
|%\ifchilddoc\||else\providecommand{\version}{draft}\||fi|
\end{center}
%
which can be uncommented to produce a draft version.
Likewise one can add a line to the very top of a child file
(above the |\childdocof{|\textit{main}|}| directive)
%
\begin{center}
|%\providecommand{\version}{final}|
\end{center}
%
which can be uncommented to produce the final version of this child document.

%%%%%%%%%%%%%%%%%%%%%%%%%%%%%%%%%%%%%%%%%%%%%%%%%%%%%%%%%%%%%%%%%%%%%%%%%%%%%%%%
\subsection{Forwarding}
\label{sec:forward}

Different versions of the main or child documents
using compilation flags as described in \secref{sec:flags}
can be (permanently) stored in different files
for convenient compilation, viewing and distribution.
To this end, the package defines a command
to pass on compilation to a different file:

%%%%%%%%%%%%%%%%%%%%%%%%%%%%%%%%%%%%%%%%
\DescribeMacro{\childdocforward}
The command |\childdocforward| redirects processing to
another source file:
%
\begin{center}
\begin{tabular}{l}
|\input{childdoc.def}|\\
|\childdocforward[|\textit{main}|]{|\textit{dest}|}|\\
\end{tabular}
\end{center}
%
The argument \textit{dest} is the destination file
(without extension).
It should be the main file or one of the child files.
Note that further \textsf{childdoc} directives
such as |\childdocof| and |\childdocforward|
in the indicated file will be processed in this form.
The optional argument \textit{main}
passes on directly to the main file \textit{main}
while pretending to compile the child \textit{dest}.
This form behaves as if \textit{dest}
issues |\childdocof{|\textit{main}|}| right away,
and no further \textsf{childdoc} directives will be processed.

%%%%%%%%%%%%%%%%%%%%%%%%%%%%%%%%%%%%%%%%
\DescribeMacro{\...prefix}
In the alternative form |\childdocforwardprefix|,
%
\begin{center}
\begin{tabular}{l}
|\input{childdoc.def}|\\
|\childdocforwardprefix[|\textit{main}|]{|\textit{prefix}|}{|\textit{dest}|}|
\end{tabular}
\end{center}
%
the destination file is determined by a pattern
depending on the current file:
To make this work, the current file must be called
`{\textit{prefix}\hspace{0.2em}\textit{suffix}}'
with \textit{prefix} matching precisely the argument.
Processing is then passed on to the file
`{\textit{dest}\hspace{0.2em}\textit{suffix}}'.
Surely, the same effect is achieved by
directly specifying the
argument `{\textit{dest}\hspace{0.2em}\textit{suffix}}'
in the first form.
However, that requires to set up a different file
for each child. With the alternative form of the command
all these files can have exactly the same content
which simplifies setting them up and maintaining them.

For example, the following file |draft.tex|
with a compilation flag |\version| as described in \secref{sec:flags}
compiles the main document as a draft:
%
\begin{center}
\begin{tabular}{l}
|\def\version{draft}|\\
|\input{childdoc.def}|\\
|\childdocforward{|\textit{main}|}|
\end{tabular}
\end{center}
%
Likewise, the following files |final|\textit{nn}|.tex|
compile the final version of the child document
|child|\textit{nn}|.tex|:
%
\begin{center}
\begin{tabular}{l}
|\def\version{final}|\\
|\input{childdoc.def}|\\
|\childdocforwardprefix{final}{child}|
\end{tabular}
\end{center}
%

Note that when several versions of a main file and/or of each child file
are to be generated, it may be convenient to set up a |Makefile| or
shell script to automatise the process.

%%%%%%%%%%%%%%%%%%%%%%%%%%%%%%%%%%%%%%%%%%%%%%%%%%%%%%%%%%%%%%%%%%%%%%%%%%%%%%%%
\subsection{Command Line Processing}
\label{sec:commandline}

The effect of redirection files can also be achieved by invoking
the \LaTeX{} compiler with a more elaborate command line.
Most conveniently this should be done as part
of a shell script or a |Makefile|.

When using \textsf{childdoc} in the main file, the following
command lines effectively perform a redirection
(note that depending on the shell being used,
backslashes may have to be doubled: `|\|' $\to$ `|\\|'):
%
\begin{center}
|... -jobname "|\textit{target}|" |\\|"|[\textit{flags}]%
|\input{childdoc.def}\childdocforward[|\textit{main}|]{|\textit{dest}|}"|
\end{center}
%
Here \textit{target} is the name of the output file,
\textit{main} is the name of the main file
and \textit{dest} is the name of the main or child file to be processed
(all filenames without extensions).
The optional argument \textit{main} can be omitted
if \textit{main} matches \textit{dest}.
Optionally, compilation \textit{flags} can be defined via |\def| commands.
This command line makes the \TeX{} engine believe
it is compiling the file \textit{target}
whose content is specified as the latter parameter.
The provided code then forwards the processing to
\textit{main} or \textit{dest} as described in \secref{sec:forward}.

%%%%%%%%%%%%%%%%%%%%%%%%%%%%%%%%%%%%%%%%%%%%%%%%%%%%%%%%%%%%%%%%%%%%%%%%%%%%%%%%
\subsection{Include by Input}
\label{sec:input}

Including child documents by |\include| has some restrictions by design.
Most notably, the content of a child document always occupies
its own set of pages; pages cannot be shared between child documents.
Usually, this behaviour makes perfect sense
because each child document contain an essential part of the document.
However, in some situations it may be desirable to compose
a document from a collection of parts
without having mandatory page breaks between then.
For this case, the package
provides a mechanism to include parts
by |\input| which can also be processed individually.
However, by construction this mechanism
requires manual handling of the content to be output.

%%%%%%%%%%%%%%%%%%%%%%%%%%%%%%%%%%%%%%%%
\DescribeMacro{\ifchilddocmanual}
The main file should be prepared as usual, see \secref{sec:include}.
However, the document body must make a distinction
between processing of an individual part and of the main document, e.g.:
%
\begin{center}
\begin{tabular}{l}
|\ifchilddocmanual|\\
|\input{\childdocname}|\\
|\||else|\\
\textit{document body with }|\input{|\textit{part}|}|\\
|\||fi|
\end{tabular}
\end{center}
%
The conditional |\ifchilddocmanual| is true whenever
a part to be included by |\input| is being compiled,
and the name of the part is stored in |\childdocname|.

%%%%%%%%%%%%%%%%%%%%%%%%%%%%%%%%%%%%%%%%
\DescribeMacro{\childdocby}
Each part to be included by |\input| should start with:
%
\begin{center}
\begin{tabular}{l}
|\input{childdoc.def}|\\
|\childdocby{|\textit{main}|}|\\
\end{tabular}
\end{center}
%
The directive |\childdocby| is similar to |\childdocof|
described in \secref{sec:include},
but the subsequent selection of content must be done manually.
To that end, both |\ifchilddoc| and |\ifchilddocmanual|
will be true upon processing of a part,
and the name of the part is stored in |\childdocname|.
Note that |\jobname| will be set to the filename of the current part
so that each part receives an individual |.aux| file
that does not interfere with the |.aux| file(s) of the main document.
This behaviour can be altered by the alternative form
|\childdocby[*]{|\textit{main}|}| (with a non-empty optional argument)
which uses the |.aux| file of the main document
by setting |\jobname| to \textit{main}.

%%%%%%%%%%%%%%%%%%%%%%%%%%%%%%%%%%%%%%%%%%%%%%%%%%%%%%%%%%%%%%%%%%%%%%%%%%%%%%%%
\subsection{Driver Development}
\label{sec:driver}

The \textsf{childdoc} mechanism can also be use for the development
of definition files such as \LaTeX{} styles or classes.
This case differs from the above setup with multiple parts
included by |\include| in that no |\includeonly| should be invoked.
This can be achieved by starting the include file
(before |\ProvidesPackage|) with:
%
\begin{center}
\begin{tabular}{l}
|\input{childdoc.def}|\\
|\childdocforward{|\textit{main}|}|\\
\end{tabular}
\end{center}
%
or alternatively with:
%
\begin{center}
\begin{tabular}{l}
|\input{childdoc.def}|\\
|\childdocby{|\textit{main}|}|\\
\end{tabular}
\end{center}
%
Both forms have slightly different effects as described above.
The main file is prepared as usual, see \secref{sec:include}.

%%%%%%%%%%%%%%%%%%%%%%%%%%%%%%%%%%%%%%%%%%%%%%%%%%%%%%%%%%%%%%%%%%%%%%%%%%%%%%%%
\subsection{Legacy Detection}
\label{sec:detection}

The directive |\childdocmain| in the main file can detect
whether the complete document or merely a child is to be compiled
even without using the directive |\childdocof|.
This method is deprecated because it is less robust
and there is no compelling reason to use it;
it is merely provided for backward compatibility
and it may be removed in future versions.

If the detection mechanism is to be used,
it is mandatory to correctly specify
the filename of the main file as the argument of |\childdocmain|:
%
\begin{center}
\begin{tabular}{l}
|\input{childdoc.def}|\\
|\childdocmain{|\textit{main}|}|\\
\end{tabular}
\end{center}
%
If |\jobname| does not match the argument \textit{main} of |\childdocmain|,
it is assumed that |\jobname| points to the child file to be compiled.
When using |\childdocmain| with the main file specified as argument,
it suffices to start a child file
with just |\input{|\textit{main}|}|
without loading of the package and using |\childdocof|.
If instead all processing is done
with the appropriate \textsf{childdoc} directives,
the argument of \textit{main} of |\childdocmain| can be empty.

An alternative version of the command line processing described
in \secref{sec:commandline} using the detection mechanism reads:
%
\begin{center}
|... -jobname "|\textit{target}|" "|[\textit{flags}]%
[|\def\jobname{|\textit{dest}|}|]|\input{|\textit{main}|}"|
\end{center}

%%%%%%%%%%%%%%%%%%%%%%%%%%%%%%%%%%%%%%%%%%%%%%%%%%%%%%%%%%%%%%%%%%%%%%%%%%%%%%%%
\subsection{Manual Code}
\label{sec:manual}

In case one cannot be certain whether the definitions file |childdoc.def|
is installed on the target \TeX{} distribution
and one prefers not to ship it,
it is conceivable to paste a few relevant commands into the sources.

To that end, drop all statements |\input{childdoc.def}|
and perform the replacements as outlined below.
Instead of |\childdocmain{|\textit{main}|}| add the following code
to the top of the main file:
%
\begin{center}
\begin{tabular}{l}
|\||ifdefined\childdocname\endinput\||fi\newif\ifchilddoc|\\
|\edef\childdocname{\scantokens\expandafter{\jobname\noexpand}}|\\
|\def\childdocmain{|\textit{main}|}\||ifx\childdocmain\childdocname\||else|\\
|\childdoctrue\includeonly{\childdocname}\let\jobname\childdocmain\||fi|\\
\end{tabular}
\end{center}
%
Instead of |\childdocof{|\textit{main}|}| just include the main file
at the top of each child file:
%
\begin{center}
|\input{|\textit{main}|}|
\end{center}
%
A simple redirection |\childdocforward{|\textit{dest}|}| is achieved by:
%
\begin{center}
|\def\jobname{|\textit{dest}|}\input{\jobname}|
\end{center}
%
The redirection with prefix
|\childdocforwardprefix[|\textit{prefix}|]{|\textit{dest}|}|
is accomplished by:
%
\begin{center}
\begin{tabular}{l}
|{\edef\jobname{\scantokens\expandafter{\jobname\noexpand}}|\\
|\def\redirectjob |\textit{prefix}|#1~~~{\gdef\jobname{|\textit{dest}|#1}}|\\
|\expandafter\redirectjob\jobname~~~}\input{\jobname}|
\end{tabular}
\end{center}

In an alternative approach,
child documents can be compiled by a specific command line
without additional code or specific definitions:
%
\begin{center}
|... -jobname "|\textit{target}|" "|[\textit{flags}]%
|\includeonly{|\textit{dest}|}\input{|\textit{main}|}"|
\end{center}
%

%%%%%%%%%%%%%%%%%%%%%%%%%%%%%%%%%%%%%%%%%%%%%%%%%%%%%%%%%%%%%%%%%%%%%%%%%%%%%%%%
%%%%%%%%%%%%%%%%%%%%%%%%%%%%%%%%%%%%%%%%%%%%%%%%%%%%%%%%%%%%%%%%%%%%%%%%%%%%%%%%
\section{Information}

%%%%%%%%%%%%%%%%%%%%%%%%%%%%%%%%%%%%%%%%%%%%%%%%%%%%%%%%%%%%%%%%%%%%%%%%%%%%%%%%
\subsection{Copyright}

Copyright \copyright{} 2017--2018 Niklas Beisert

This work may be distributed and/or modified under the
conditions of the \LaTeX{} Project Public License, either version 1.3
of this license or (at your option) any later version.
The latest version of this license is in
  \url{http://www.latex-project.org/lppl.txt}
and version 1.3 or later is part of all distributions of \LaTeX{}
version 2005/12/01 or later.

This work has the LPPL maintenance status `maintained'.

The Current Maintainer of this work is Niklas Beisert.

This work consists of the files |README.txt|, |childdoc.ins| and |childdoc.dtx|
as well as the derived files |childdoc.def|, |cdocsamp.tex|
with |cdocsch1.tex|, |cdocsch2.tex|, |cdocspt3.tex|, |cdocspt4.tex|,
|cdocsdrf.tex|, |cdocsfn1.tex|, |cdocsfn2.tex|
as well as |childdoc.pdf|.

%%%%%%%%%%%%%%%%%%%%%%%%%%%%%%%%%%%%%%%%%%%%%%%%%%%%%%%%%%%%%%%%%%%%%%%%%%%%%%%%
\subsection{Files and Installation}

The package consists of the files:
%
\begin{center}
\begin{tabular}{ll}
    |README.txt|   & readme file \\
    |childdoc.ins| & installation file \\
    |childdoc.dtx| & source file \\
    |childdoc.def| & definition file \\
    |cdocsamp.tex| & sample main file \\
    |cdocsch1.tex| & sample include file \\
    |cdocsch2.tex| & sample include file \\
    |cdocspt3.tex| & sample part file \\
    |cdocspt4.tex| & sample part file \\
    |cdocsdrf.tex| & sample redirection file \\
    |cdocsfn1.tex| & sample redirection file \\
    |cdocsfn2.tex| & sample redirection file \\
    |childdoc.pdf| & manual
\end{tabular}
\end{center}
%
The distribution consists of the files
|README.txt|, |childdoc.ins| and |childdoc.dtx|.
%
\begin{itemize}
\item
Run (pdf)\LaTeX{} on |childdoc.dtx|
to compile the manual |childdoc.pdf| (this file).
\item
Run \LaTeX{} on |childdoc.ins| to create the definitions file |childdoc.def|
and the sample |cdocsamp.tex| with include files
|cdocsch1.tex|, |cdocsch2.tex|, |cdocspt3.tex|, |cdocspt4.tex|,
|cdocsdrf.tex|, |cdocsfn1.tex|, |cdocsfn2.tex|.
Then copy the file |childdoc.def| to an appropriate directory of your \LaTeX{}
distribution, e.g.\ \textit{texmf-root}|/tex/latex/childdoc|.
\end{itemize}

%%%%%%%%%%%%%%%%%%%%%%%%%%%%%%%%%%%%%%%%%%%%%%%%%%%%%%%%%%%%%%%%%%%%%%%%%%%%%%%%
\subsection{Related CTAN Packages}

There are several other packages which offer a similar functionality:
%
\begin{itemize}
\item
The packages
\href{http://ctan.org/pkg/docmute}{\textsf{docmute}},
\href{http://ctan.org/pkg/includex}{\textsf{includex}} and
\href{http://ctan.org/pkg/standalone}{\textsf{standalone}}
provide commands to include only the document body of
a child file thus allowing both files to be compiled individually.
\item
The packages \href{http://ctan.org/pkg/subdocs}{\textsf{subdocs}}
and \href{http://ctan.org/pkg/subfiles}{\textsf{subfiles}}
provide structures in which the main and child documents can be
encapsulated and allowing them to be compiled individually.
The inclusion mechanism is different from the conventional |\include|.
\item
The package \href{http://ctan.org/pkg/combine}{\textsf{combine}}
is an elaborate solution to combine several documents into one.
\end{itemize}
%
See also the CTAN topic \href{http://ctan.org/topic/subdocs}{\textsf{subdocs}}
for further related packages.
The present package differs from the above solutions in that
a document structure constructed with the conventional |\include| mechanism
just needs two extra commands at the top of every file
such that all constituent files can be compiled individually.

%%%%%%%%%%%%%%%%%%%%%%%%%%%%%%%%%%%%%%%%%%%%%%%%%%%%%%%%%%%%%%%%%%%%%%%%%%%%%%%%
%\subsection{Feature Suggestions}
%
%The following is a list of features which may be useful for future
%versions of this package:
%%
%\begin{itemize}
%\item
%\ldots
%\end{itemize}

%%%%%%%%%%%%%%%%%%%%%%%%%%%%%%%%%%%%%%%%%%%%%%%%%%%%%%%%%%%%%%%%%%%%%%%%%%%%%%%%
\subsection{Revision History}

%%%%%%%%%%%%%%%%%%%%%%%%%%%%%%%%%%%%%%%%
\paragraph{v2.0:} 2018/12/30

\begin{itemize}
\item
immediate forward processing
\item
added |\childdocby| mechanism
\item
manual restructured
\end{itemize}

%%%%%%%%%%%%%%%%%%%%%%%%%%%%%%%%%%%%%%%%
\paragraph{v1.6:} 2018/01/17

\begin{itemize}
\item
application for development of include files
\item
corrections to manual
\end{itemize}

%%%%%%%%%%%%%%%%%%%%%%%%%%%%%%%%%%%%%%%%
\paragraph{v1.5:} 2017/05/21

\begin{itemize}
\item
more complete structuring introduced
\item
|\childdocof| introduced
\item
|\childdoc| renamed to |\childdocmain|
\item
|\childredirect| renamed to |\childdocforward| and |\childdocforwardprefix|
and functionality expanded
\end{itemize}

%%%%%%%%%%%%%%%%%%%%%%%%%%%%%%%%%%%%%%%%
\paragraph{v1.0:} 2017/04/27

\begin{itemize}
\item
manual and install package
\item
first version published on CTAN
\end{itemize}

%%%%%%%%%%%%%%%%%%%%%%%%%%%%%%%%%%%%%%%%
\paragraph{v0.6:} 2017/04/26

\begin{itemize}
\item
redirection mechanism added
\end{itemize}

%%%%%%%%%%%%%%%%%%%%%%%%%%%%%%%%%%%%%%%%
\paragraph{v0.5:} 2017/04/26

\begin{itemize}
\item
functionality in definition file
\end{itemize}


%%%%%%%%%%%%%%%%%%%%%%%%%%%%%%%%%%%%%%%%%%%%%%%%%%%%%%%%%%%%%%%%%%%%%%%%%%%%%%%%
%%%%%%%%%%%%%%%%%%%%%%%%%%%%%%%%%%%%%%%%%%%%%%%%%%%%%%%%%%%%%%%%%%%%%%%%%%%%%%%%
%%%%%%%%%%%%%%%%%%%%%%%%%%%%%%%%%%%%%%%%%%%%%%%%%%%%%%%%%%%%%%%%%%%%%%%%%%%%%%%%
\appendix

\settowidth\MacroIndent{\rmfamily\scriptsize 000\ }

 \DocInput{childdoc.dtx}

\end{document}
%</driver>
% \fi
%
% %%%%%%%%%%%%%%%%%%%%%%%%%%%%%%%%%%%%%%%%%%%%%%%%%%%%%%%%%%%%%%%%%%%%%%%%%%%%%%
% %%%%%%%%%%%%%%%%%%%%%%%%%%%%%%%%%%%%%%%%%%%%%%%%%%%%%%%%%%%%%%%%%%%%%%%%%%%%%%
% \section{Sample}
%\iffalse
%<*samplemain>
%\fi
%
% The following presents a sample document
% with two chapters, two parts, a title page,
% a compile flag as well as three forwarding files to set the flag.
% It consists of eight |.tex| files:
% \begin{center}
% \begin{tabular}{ll}
% |cdocsamp.tex|&main file\\
% |cdocsch1.tex|&include file for chapter 1\\
% |cdocsch2.tex|&include file for chapter 2\\
% |cdocspt3.tex|&include file for part 3\\
% |cdocspt4.tex|&include file for part 4\\
% |cdocsdrf.tex|&forwarding file for main file in draft mode\\
% |cdocsfi1.tex|&forwarding file for final version of chapter 1\\
% |cdocsfi2.tex|&forwarding file for final version of chapter 2\\
% \end{tabular}
% \end{center}
% Each of the eight files can be compiled directly by the \LaTeX{} compiler.
%
% %%%%%%%%%%%%%%%%%%%%%%%%%%%%%%%%%%%%%%
% \paragraph{Main File.}
%
% The main file is called |cdocsamp.tex|.
%
% Load the \textsf{childdoc} definitions and
% declare the filename for the main document:
%    \begin{macrocode}
\input{childdoc.def}
\childdocmain{}
%    \end{macrocode}

% Optional override for |\version| flag:
%    \begin{macrocode}
%%\ifchilddoc\else\providecommand{\version}{draft}\fi
%    \end{macrocode}

% Define the default values for the |\version| flag
% (|final| for the main file and |draft| for childs):
%    \begin{macrocode}
\ifchilddoc
\providecommand{\version}{draft}
\else
\providecommand{\version}{final}
\fi
%    \end{macrocode}

% Load the standard document class:
%    \begin{macrocode}
\documentclass[12pt]{article}
%    \end{macrocode}

% Start the document body:
%    \begin{macrocode}
\begin{document}
%    \end{macrocode}

% Declare a title page.
% Print title, part of document being processed and version flag:
%    \begin{macrocode}
\addtocounter{page}{-1}
\begin{center}
{\LARGE\bfseries{}childdoc example\par}
\vspace{1cm}
\ifchilddoc
\ifchilddocmanual part\else chapter\fi:
`\childdocname' of `\childdocjob'\par
\else
main document: `\childdocjob'\par
\fi
version: \version\par
\end{center}
\newpage
%    \end{macrocode}

% Manually include selected file,
% otherwise process as usual:
%    \begin{macrocode}
\ifchilddocmanual
\section*{part `\childdocname'}
\input{\childdocname}
\else
%    \end{macrocode}

% Include the two chapters:
%    \begin{macrocode}
\include{cdocsch1}
\include{cdocsch2}
%    \end{macrocode}

% Include the two parts unless only chapters should be displayed:
%    \begin{macrocode}
\ifchilddoc\else
\section{part three}
\input{cdocspt3}
\section{part four}
\input{cdocspt4}
\fi
%    \end{macrocode}

% Process as usual until here:
%    \begin{macrocode}
\fi
%    \end{macrocode}

% End of document body:
%    \begin{macrocode}
\end{document}
%    \end{macrocode}
%\iffalse
%</samplemain>
%\fi
%
% %%%%%%%%%%%%%%%%%%%%%%%%%%%%%%%%%%%%%%
% \paragraph{Chapter Include Files.}
%
% The include files are called |cdocsch1.tex| and |cdocsch2.tex|.
%
%\iffalse
%<*samplechap1|samplechap2>
%\fi

% Optional override for |\version| flag:
%    \begin{macrocode}
%%\providecommand{\version}{final}
%    \end{macrocode}

% Include the main document:
%    \begin{macrocode}
\input{childdoc.def}
\childdocof{cdocsamp}
%    \end{macrocode}

%\iffalse
%</samplechap1|samplechap2>
%\fi
%
%\iffalse
%<*samplechap1>
%\fi
% Some text for chapter 1:
%    \begin{macrocode}
\section{one}
some text in chapter one
%    \end{macrocode}

%\iffalse
%</samplechap1>
%\fi
% Some text for chapter 2:
%\iffalse
%<*samplechap2>
%\fi
%    \begin{macrocode}
\section{two}
more text in chapter two
%    \end{macrocode}

%\iffalse
%</samplechap2>
%\fi
%
% %%%%%%%%%%%%%%%%%%%%%%%%%%%%%%%%%%%%%%
% \paragraph{Part Include Files.}
%
% The include files are called |cdocspt3.tex| and |cdocspt4.tex|.
%
%\iffalse
%<*samplepart3|samplepart4>
%\fi

% Optional override for |\version| flag:
%    \begin{macrocode}
%%\providecommand{\version}{final}
%    \end{macrocode}

% Include the main document:
%    \begin{macrocode}
\input{childdoc.def}
\childdocby{cdocsamp}
%    \end{macrocode}

%\iffalse
%</samplepart3|samplepart4>
%\fi
%
%\iffalse
%<*samplepart3>
%\fi
% Some text for part 3:
%    \begin{macrocode}
some text in part three
%    \end{macrocode}

%\iffalse
%</samplepart3>
%\fi
% Some text for part 4:
%\iffalse
%<*samplepart4>
%\fi
%    \begin{macrocode}
more text in part four
%    \end{macrocode}

%\iffalse
%</samplepart4>
%\fi
%
% %%%%%%%%%%%%%%%%%%%%%%%%%%%%%%%%%%%%%%
% \paragraph{Forwarding for a Complete Draft.}
%
% The following forwarding file |cdocsdrf.tex|
% compiles the main document in draft mode:
%\iffalse
%<*sampledraft>
%\fi
%    \begin{macrocode}
\def\version{draft}
\input{childdoc.def}
\childdocforward{cdocsamp}
%    \end{macrocode}

%\iffalse
%</sampledraft>
%\fi
%
% %%%%%%%%%%%%%%%%%%%%%%%%%%%%%%%%%%%%%%
% \paragraph{Forwarding for Final Version of the Chapters.}
%
% The following forwarding files |cdocsfn1.tex| and |cdocsfn2.tex|
% (with identical content)
% compile the final versions of the child documents
% |cdocsch1.tex| and |cdocsch2.tex|, respectively:
%\iffalse
%<*samplefinal>
%\fi
%    \begin{macrocode}
\def\version{final}
\input{childdoc.def}
\childdocforwardprefix[cdocsamp]{cdocsfn}{cdocsch}
%    \end{macrocode}

%\iffalse
%</samplefinal>
%\fi
%
% %%%%%%%%%%%%%%%%%%%%%%%%%%%%%%%%%%%%%%
% \paragraph{Command Line Processing.}
%
% The following three command lines generate the output files
% |cdocscld|, |cdocscl1| and |cdocscl2|
% which should be identical to
% |cdocsdrf|, |cdocsch1| and |cdocsfn2|, respectively:
% \begin{center}
% \begin{tabular}{l}
% |latex -jobname cdocscld \|\\
% |  "\def\version{draft}\input{childdoc.def}\childdocforward{cdocsamp}"|\\
% |latex -jobname cdocscl1 \|\\
% |  "\input{childdoc.def}\childdocforward[cdocsamp]{cdocsch1}"|\\
% |latex -jobname cdocscl2 \|\\
% |  "\def\version{final}\input{childdoc.def}\childdocforward{cdocsch2}"|
% \end{tabular}
% \end{center}
% Note that the trailing backslash on each first line
% merely continues the input to the second line
% (for convenient cut ant paste).
% Furthermore, the command |latex| can be replaced by any
% of its alternative versions such as |pdflatex|.
%
% %%%%%%%%%%%%%%%%%%%%%%%%%%%%%%%%%%%%%%%%%%%%%%%%%%%%%%%%%%%%%%%%%%%%%%%%%%%%%%
% %%%%%%%%%%%%%%%%%%%%%%%%%%%%%%%%%%%%%%%%%%%%%%%%%%%%%%%%%%%%%%%%%%%%%%%%%%%%%%
% \section{Implementation}
%\iffalse
%<*package>
%\fi
%
% This section describes the definitions file |childdoc.def|.

% The definitions cannot be loaded using |\usepackage| or |\RequirePackage|
% which has a mechanism to prevent loading a style file more than once.
% When loading the definitions by means of |\input|
% multiple instances have to be prevented manually:
%\iffalse
%This code needs to be before the `\ProvidesFile' directive
%which is defined at the beginning of this file.
%Therefore it is also placed there and commented out here.
%</package>
%<*discard>
%\fi
%    \begin{macrocode}
\ifdefined\childdocmain\endinput\fi
%    \end{macrocode}
%\iffalse
%</discard>
%<*package>
%\fi
%
% \macro{\ifchilddoc}
% \macro{\ifchilddocmanual}
% The conditional |\ifchilddoc| tells whether a
% child (true) or main (false) document is being compiled.
% The conditional |\ifchilddocmanual| tells whether
% the |\includeonly| mechanism is used (false) or
% the selection of child files must be performed manually (true).
% The definitions initialise to false:
%    \begin{macrocode}
\newif\ifchilddoc
\newif\ifchilddocmanual
%    \end{macrocode}

% \macro{\childdocname}
% \macro{\childdocjob}
% The macro |\childdocname| stores the name of the main document
% to be compiled. The macro |\childdocjob| stores the name of
% the document on which the \LaTeX{} compiler was originally invoked.
% The content of |\jobname| cannot be compared
% to filenames specified in the source due to different catcodes.
% The following code rescans |\jobname|, stores the result
% in |\childdocname| and saves a copy in |\childdocjob|:
%    \begin{macrocode}
\edef\childdocname{\scantokens\expandafter{\jobname\noexpand}}
\let\childdocjob\childdocname
%    \end{macrocode}

% \macro{\childdocdisable}
% The macro |\childdocdisable| prevents the main file
% from being processed more than once.
% At this stage, the main document command |\childdocmain|
% is assumed to be called once again where it should do nothing.
% Any subsequent call to it should prevent
% a secondary processing of the main document
% It overwrites the forwarding commands
% |\childdocof| and |\childdocforward|
% with empty macros to prevent further inclusions of the main document:
%    \begin{macrocode}
\newcommand{\childdocdisable}
{
  \renewcommand{\childdocmain}[1]{\renewcommand{\childdocmain}[1]{\endinput}}
  \renewcommand{\childdocof}[1]{}
  \renewcommand{\childdocby}[2][]{}
  \renewcommand{\childdocforward}[2][]{}
  \renewcommand{\childdocdisable}{}
}
%    \end{macrocode}

% \macro{\childdocmain}
% The macro |\childdocmain| is to be called at the top of the main file
% with nothing or the main filename (without extension) as argument.
% First, it breaks loops.
% If the argument is not empty and does not match |\childdocname|
% (which is set by the first inclusion of |childdoc.def|),
% |\ifchilddoc| is set to true, |\includeonly| is applied to the child file
% and |\jobname| is set to the main file
% (for proper handling of |.aux| files):
%    \begin{macrocode}
\newcommand{\childdocmain}[1]
{
  \childdocdisable\childdocmain{}
  \if?#1?\else
    \begingroup
      \def\childdoctmp{#1}
      \ifx\childdoctmp\childdocname
        \def\childdoctmp{}
      \else
        \def\childdoctmp
        {
          \childdoctrue
          \includeonly{\childdocname}
          \def\childdocjob{#1}
          \def\jobname{#1}
        }
      \fi
      \expandafter
    \endgroup
    \childdoctmp
  \fi
}
%    \end{macrocode}

% \macro{\childdocof}
% The command |\childdocof| redirects
% compilation to the main file |#1|.
%    \begin{macrocode}
\newcommand{\childdocof}[1]
{
  \childdocdisable
  \childdoctrue
  \includeonly{\childdocname}
  \def\jobname{#1}
  \def\childdocjob{#1}
  \input{#1}
}
%    \end{macrocode}

% \macro{\childdocby}
% The command |\childdocby| ....
%    \begin{macrocode}
\newcommand{\childdocby}[2][]
{
  \childdocdisable
  \childdoctrue
  \childdocmanualtrue
  \if?#1?\else
    \def\jobname{#2}
  \fi
  \def\childdocjob{#2}
  \input{#2}
  \endinput
}
%    \end{macrocode}

% \macro{\childdocforward}
% The command |\childdocforward| redirects
% compilation to the main file or
% (if the optional argument is given) a child file.
% Parameters are set as if the main file
% or a child file starting with |\childdocof| was compiled.
% Then compilation is handed over to the main file:
%    \begin{macrocode}
\newcommand{\childdocforward}[2][]
{
  \begingroup
    \if?#1?
      \def\childdoctmp
      {
        \def\childdocname{#2}
        \def\childdocjob{#2}
        \def\jobname{#2}
        \input{#2}
        \endinput
      }
    \else
      \def\childdoctmp
      {
        \childdocdisable
        \def\childdocname{#2}
        \childdoctrue
        \includeonly{#2}
        \def\childdocjob{#1}
        \def\jobname{#1}
        \input{#1}
        \endinput
      }
    \fi
    \expandafter
  \endgroup
  \childdoctmp
}
%    \end{macrocode}

% \macro{\childdocforwardprefix}
% The command |\childdocforwardprefix| redirects
% compilation to the main or a child file by means of a pattern.
% The prefix |#1| in the current filename is replaced by |#2|
% and the suffix of the current filename is kept
% (it is assumed that the filename does not contain the substring `|~~~|'
% which is used as a delimiter).
% Compilation is handed over to the new file by |\childdocforward|:
%    \begin{macrocode}
\newcommand{\childdocforwardprefix}[3][]
{
  \begingroup
    \def\childdocextract #2##1~~~{\def\childdoctmp{\childdocforward[#1]{#3##1}}}
    \expandafter\childdocextract\childdocname~~~
    \expandafter
  \endgroup
  \childdoctmp
}
%    \end{macrocode}

% \macro{\childdoc}
% The deprecated macro |\childdoc| is a legacy version of |\childdocmain|:
%    \begin{macrocode}
\newcommand{\childdoc}{\childdocmain}
%    \end{macrocode}

% \macro{\childdocredirect}
% The deprecated macro |\childdocredirect| is a legacy version
% of |\childdocforward| and |\childdocforwardprefix|:
%    \begin{macrocode}
\newcommand{\childdocredirect}[2][]
{
  \begingroup
    \if?#1?
      \def\childdoctmp{\childdocforward{#2}}
    \else
      \def\childdoctmp{\childdocforwardprefix{#1}{#2}}
    \fi
    \expandafter
  \endgroup
  \childdoctmp
}
%    \end{macrocode}

%\iffalse
%</package>
%\fi
%
\endinput

\childdocforward{cdocsamp}
%    \end{macrocode}

%\iffalse
%</sampledraft>
%\fi
%
% %%%%%%%%%%%%%%%%%%%%%%%%%%%%%%%%%%%%%%
% \paragraph{Forwarding for Final Version of the Chapters.}
%
% The following forwarding files |cdocsfn1.tex| and |cdocsfn2.tex|
% (with identical content)
% compile the final versions of the child documents
% |cdocsch1.tex| and |cdocsch2.tex|, respectively:
%\iffalse
%<*samplefinal>
%\fi
%    \begin{macrocode}
\def\version{final}
% \iffalse
%
% childdoc.dtx Copyright (C) 2017-2018 Niklas Beisert
%
% This work may be distributed and/or modified under the
% conditions of the LaTeX Project Public License, either version 1.3
% of this license or (at your option) any later version.
% The latest version of this license is in
%   http://www.latex-project.org/lppl.txt
% and version 1.3 or later is part of all distributions of LaTeX
% version 2005/12/01 or later.
%
% This work has the LPPL maintenance status `maintained'.
%
% The Current Maintainer of this work is Niklas Beisert.
%
% This work consists of the files childdoc.dtx and childdoc.ins
% and the derived files childdoc.def and cdocsamp.tex with
% cdocsch1.tex, cdocsch2.tex, cdocsdrf.tex, cdocsfn1.tex, cdocsfn2.tex.
%
%<package>\ifdefined\childdocmain\endinput\fi
%<package>\ProvidesFile{childdoc.def}[2018/12/30 v2.0 child document driver]
%<samplemain>\ProvidesFile{cdocsamp.tex}[2018/12/30 v2.0 sample for childdoc]
%<*driver>
%\ProvidesFile{childdoc.drv}[2018/12/30 v2.0 childdoc reference manual file]
\PassOptionsToClass{10pt,a4paper}{article}
\documentclass{ltxdoc}

\usepackage[margin=35mm]{geometry}
\usepackage{hyperref}
\usepackage{hyperxmp}
\usepackage[usenames]{color}

\hypersetup{colorlinks=true}
\hypersetup{pdfstartview=FitH}
\hypersetup{pdfpagemode=UseNone}
\hypersetup{pdfsource={}}
\hypersetup{pdflang={en-UK}}
\hypersetup{pdfcopyright={Copyright 2017-2018 Niklas Beisert.
  This work may be distributed and/or modified under the
  conditions of the LaTeX Project Public License, either version 1.3
  of this license or (at your option) any later version.}}
\hypersetup{pdflicenseurl={http://www.latex-project.org/lppl.txt}}
\hypersetup{pdfcontactaddress={ETH Zurich, ITP, HIT K,
  Wolfgang-Pauli-Strasse 27}}
\hypersetup{pdfcontactpostcode={8093}}
\hypersetup{pdfcontactcity={Zurich}}
\hypersetup{pdfcontactcountry={Switzerland}}
\hypersetup{pdfcontactemail={nbeisert@itp.phys.ethz.ch}}
\hypersetup{pdfcontacturl={http://people.phys.ethz.ch/\xmptilde nbeisert/}}

\newcommand{\secref}[1]{\hyperref[#1]{section \ref*{#1}}}

\parskip1ex
\parindent0pt
\let\olditemize\itemize
\def\itemize{\olditemize\parskip0pt}

\begin{document}

\title{The \textsf{childdoc} Package}
\hypersetup{pdftitle={The childdoc Package}}
\author{Niklas Beisert\\[2ex]
  Institut f\"ur Theoretische Physik\\
  Eidgen\"ossische Technische Hochschule Z\"urich\\
  Wolfgang-Pauli-Strasse 27, 8093 Z\"urich, Switzerland\\[1ex]
  \href{mailto:nbeisert@itp.phys.ethz.ch}
  {\texttt{nbeisert@itp.phys.ethz.ch}}}
\hypersetup{pdfauthor={Niklas Beisert}}
\hypersetup{pdfsubject={Manual for the LaTeX2e Package childdoc}}
\date{30 December 2018, \textsf{v2.0}}
\maketitle

\begin{abstract}\noindent
\textsf{childdoc} is a \LaTeXe{} package
that enables the direct compilation
of document sections included by |\include|
to individual files.
\end{abstract}

\begingroup
\parskip0ex
\tableofcontents
\endgroup

%%%%%%%%%%%%%%%%%%%%%%%%%%%%%%%%%%%%%%%%%%%%%%%%%%%%%%%%%%%%%%%%%%%%%%%%%%%%%%%%
%%%%%%%%%%%%%%%%%%%%%%%%%%%%%%%%%%%%%%%%%%%%%%%%%%%%%%%%%%%%%%%%%%%%%%%%%%%%%%%%
\section{Introduction}

\LaTeX{} provides a mechanism to structure a large document (such as a book)
into a main file and several child files (containing the chapters)
using the |\include| command.
This mechanism is beneficial for documents
which span hundreds of pages in order to
make the source file(s) more manageable.
Moreover, compilation can be restricted to
selected child files by means of the |\includeonly| command.
The latter feature can be used to reduce the compilation time while editing
(this was significantly more useful in the earlier days of \LaTeX{})
or to generate a smaller document which is easier to navigate.
Another application of |\includeonly| is to generate
documents consisting of selected parts of the complete document.

However, there are a few drawbacks of the plain |\include| mechanism:
\begin{itemize}
\item
The child files cannot be compiled on their own,
they can only be compiled via the main file.
A naive editing environment
(such as a text editor with an option
to have the current file processed by \LaTeX)
may require one to switch to the main file before compiling;
attempting to compile the child file produces errors.
\item
The main file must be modified (each time)
to adjust the |\includeonly| command
to the present needs. This easily leaves the main file in a messy state.
\item
The generated document will always carry the filename
of the main document. This is inconvenient if
several child files are to be compiled and
to be kept for distribution.
\end{itemize}

The present package provides a simple interface
to make child files individually compilable by \LaTeX{}.
Compiling a child file then has the same effect as compiling
the main file with an |\includeonly| command
to select the appropriate child.
Moreover the generated document will carry the name of the child
rather than the main file.
This resolves all three above issues.

This feature is meant to make the editing of books,
thesis documents and lecture notes somewhat more convenient.
However, the package can also be used efficiently for
composing a series of documents (such as exercise sheets)
which are typically distributed individually.
It then assists the author in generating the individual documents
(potentially in different versions)
as well as a document containing the collected series.
Another application is in developing style files
or other kinds of included material
where compilation of the style file could redirect
to a sample or test file.

%%%%%%%%%%%%%%%%%%%%%%%%%%%%%%%%%%%%%%%%%%%%%%%%%%%%%%%%%%%%%%%%%%%%%%%%%%%%%%%%
%%%%%%%%%%%%%%%%%%%%%%%%%%%%%%%%%%%%%%%%%%%%%%%%%%%%%%%%%%%%%%%%%%%%%%%%%%%%%%%%
\section{Usage}

First of all, the package \textsf{childdoc} is \emph{not} a standard
\LaTeXe{} |.sty| style file! Therefore it needs to be invoked in
a non-standard way.

%%%%%%%%%%%%%%%%%%%%%%%%%%%%%%%%%%%%%%%%%%%%%%%%%%%%%%%%%%%%%%%%%%%%%%%%%%%%%%%%
\subsection{Included Files}
\label{sec:include}

%%%%%%%%%%%%%%%%%%%%%%%%%%%%%%%%%%%%%%%%
\DescribeMacro{\childdocmain}
To use the package, add the commands
\begin{center}
\begin{tabular}{l}
|\input{childdoc.def}|\\
|\childdocmain{}|\\
\end{tabular}
\end{center}
at the very top of the main \LaTeX{} file,
in particular \emph{before} the |\documentclass| statement!
The argument of |\childdocmain| should be left empty
(but it must be present).

%%%%%%%%%%%%%%%%%%%%%%%%%%%%%%%%%%%%%%%%
\DescribeMacro{\childdocof}
Furthermore, add the commands
\begin{center}
\begin{tabular}{l}
|\input{childdoc.def}|\\
|\childdocof{|\textit{main}|}|\\
\end{tabular}
\end{center}
at the top of every child file \textit{child}
which is included by |\include{|\textit{child}|}|
from within the main file
(or at least for those files to be compiled individually).
The argument \textit{main} must be the filename of the main file.

There are a couple of
considerations in setting up the main and child documents:

%%%%%%%%%%%%%%%%%%%%%%%%%%%%%%%%%%%%%%%%
\paragraph{Restrictions.}

Please note the following restrictions:
\begin{itemize}
\item
|\childdocmain| must be called with one argument \textit{main}
to ensure compatibility with earlier version of the package.
It must either be empty (|\childdocmain{}|)
or precisely match the filename of the main file in which it is specified.
See \secref{sec:detection} for further information.
\item
The filename \textit{main} must be specified without the |.tex| extension.
\item
The filename \textit{main} is case sensitive
(even in case-insensitive file systems)
due to internal string comparison.
\item
The argument \textit{main} should be fully expanded, it cannot be a macro.
\item
Subdirectories and special characters should be avoided in filenames.
\item
The command |\childdocmain{|\textit{main}|}| must be followed by a whitespace.
It should not be followed immediately by another command
or by a comment mark `|%|'.
This is because the \TeX{} parser reads the token immediately following
the argument of |\childdocmain| and puts it
at the beginning of every child section;
however, a white\-space is ignored.
\end{itemize}

%%%%%%%%%%%%%%%%%%%%%%%%%%%%%%%%%%%%%%%%
\paragraph{Content of Main File.}

It is advisable to place all content in the child files included by |\include|.
Any output contained in the main file will appear in all child documents
unless suppressed manually;
it cannot be suppressed automatically by the |\includeonly| directive
and thus should normally be avoided.
A method to include some content in the main file
by means of conditional processing is described in \secref{sec:conditional}.

%%%%%%%%%%%%%%%%%%%%%%%%%%%%%%%%%%%%%%%%
\paragraph{Page Numbering.}

When only a part of the document is compiled,
the appropriate numbering of pages
(as well as other status parameters)
is determined from the |.aux| files.
The latter contain information from previous passes.
However this information needs to propagate through
all intermediate child documents.
Therefore the page numbering in child documents may well
be inconsistent until the complete document is compiled at least once.

A useful (if unconventional) way to always ensure a consistent
page numbering is to restart the numbering in each child document
and denote the pages by `\textit{child}|.|\textit{page}'
where \textit{child} represents the chapter/section number of the child file.
This can be achieved by the command
|\numberwithin{page}{|\textit{child}|}|
of the \textsf{amsmath} package
where \textit{child} can be |chapter| or |section|
depending on the chosen structuring.
Alternatively, one can modify the macro |\thepage| appropriately
and reset the counter |page| at the start of each child file.

%%%%%%%%%%%%%%%%%%%%%%%%%%%%%%%%%%%%%%%%%%%%%%%%%%%%%%%%%%%%%%%%%%%%%%%%%%%%%%%%
\subsection{Conditional Processing}
\label{sec:conditional}

The package provides a mechanism to compile different versions
of a document. To customise the versions further some conditional processing
can come in handy to distinguish which version is being compiled.
The package provides two macros to describe the compilation context:

%%%%%%%%%%%%%%%%%%%%%%%%%%%%%%%%%%%%%%%%
\DescribeMacro{\ifchilddoc}
The conditional |\ifchilddoc| distinguishes between the compilation of
child documents and the main document:
%
\begin{center}
|\ifchilddoc |\textit{child-code}| |[|\||else |\textit{main-code}]| \||fi|
\end{center}

%%%%%%%%%%%%%%%%%%%%%%%%%%%%%%%%%%%%%%%%
\DescribeMacro{\childdocname}
\DescribeMacro{\childdocjob}
The macro |\childdocname| contains the filename (without extension)
of the main or child file being processed.
Note that |\childdocjob| will always contain the name of the main file.

%%%%%%%%%%%%%%%%%%%%%%%%%%%%%%%%%%%%%%%%
\paragraph{Title Page.}

Conditional processing can be used to include a title or banner page
in the main document when proper precautions are taken.
Importantly, the code in the main file should ensure that the page counter
(as well as other status parameters which are stored in the |.aux| files)
takes the same value after the conditional processing.
Otherwise the page numbers may take divergent values
depending on which part is compiled.

For example, a title page could be declared by:
%
\begin{center}
\begin{tabular}{l}
|\ifchilddoc\||else|\\
|\addtocounter{page}{-1}|\\
\textit{code for title page}\\
|\newpage|\\
|\||fi|
\end{tabular}
\end{center}
%
A banner page for the child documents can be generated by:
%
\begin{center}
\begin{tabular}{l}
|\ifchilddoc|\\
|\addtocounter{page}{-1}|\\
\textit{code for banner page}\\
|\newpage|\\
|\||fi|
\end{tabular}
\end{center}
%
Here one could write a message such as:
\begin{center}
|This is the part \childdocname{} of \childdocjob{}.|
\end{center}

%%%%%%%%%%%%%%%%%%%%%%%%%%%%%%%%%%%%%%%%%%%%%%%%%%%%%%%%%%%%%%%%%%%%%%%%%%%%%%%%
\subsection{Flags}
\label{sec:flags}

The package makes it easy to generate different versions
of the main or child documents.
To this end compilation flags can be defined
and assigned different default values.
They will be particularly useful in conjunction
with the forwarding mechanism described in \secref{sec:forward}.

For example, it may be useful to have a flag |\version|
which can be set to |draft| or |final|.
The document source will contain some conditional code
depending on the value of |\version|.
Suppose further, the flag should default to |final| for the main file
and to |draft| for child files
which is a natural assignment for editing the document.
This is achieved by placing the following code
in the preamble of the main document
(below the |\childdocmain| directive):
%
\begin{center}
\begin{tabular}{l}
|\ifchilddoc|\\
|\providecommand{\version}{draft}|\\
|\||else|\\
|\providecommand{\version}{final}|\\
|\||fi|
\end{tabular}
\end{center}
%
The definition by |\providecommand| makes sure
that previous definitions are not overwritten.
Further statements |\providecommand{\version}{...}|
can thus be added before the above code to override it.

For the main file, one might add a line
(between |\childdocmain| and the above block)
%
\begin{center}
|%\ifchilddoc\||else\providecommand{\version}{draft}\||fi|
\end{center}
%
which can be uncommented to produce a draft version.
Likewise one can add a line to the very top of a child file
(above the |\childdocof{|\textit{main}|}| directive)
%
\begin{center}
|%\providecommand{\version}{final}|
\end{center}
%
which can be uncommented to produce the final version of this child document.

%%%%%%%%%%%%%%%%%%%%%%%%%%%%%%%%%%%%%%%%%%%%%%%%%%%%%%%%%%%%%%%%%%%%%%%%%%%%%%%%
\subsection{Forwarding}
\label{sec:forward}

Different versions of the main or child documents
using compilation flags as described in \secref{sec:flags}
can be (permanently) stored in different files
for convenient compilation, viewing and distribution.
To this end, the package defines a command
to pass on compilation to a different file:

%%%%%%%%%%%%%%%%%%%%%%%%%%%%%%%%%%%%%%%%
\DescribeMacro{\childdocforward}
The command |\childdocforward| redirects processing to
another source file:
%
\begin{center}
\begin{tabular}{l}
|\input{childdoc.def}|\\
|\childdocforward[|\textit{main}|]{|\textit{dest}|}|\\
\end{tabular}
\end{center}
%
The argument \textit{dest} is the destination file
(without extension).
It should be the main file or one of the child files.
Note that further \textsf{childdoc} directives
such as |\childdocof| and |\childdocforward|
in the indicated file will be processed in this form.
The optional argument \textit{main}
passes on directly to the main file \textit{main}
while pretending to compile the child \textit{dest}.
This form behaves as if \textit{dest}
issues |\childdocof{|\textit{main}|}| right away,
and no further \textsf{childdoc} directives will be processed.

%%%%%%%%%%%%%%%%%%%%%%%%%%%%%%%%%%%%%%%%
\DescribeMacro{\...prefix}
In the alternative form |\childdocforwardprefix|,
%
\begin{center}
\begin{tabular}{l}
|\input{childdoc.def}|\\
|\childdocforwardprefix[|\textit{main}|]{|\textit{prefix}|}{|\textit{dest}|}|
\end{tabular}
\end{center}
%
the destination file is determined by a pattern
depending on the current file:
To make this work, the current file must be called
`{\textit{prefix}\hspace{0.2em}\textit{suffix}}'
with \textit{prefix} matching precisely the argument.
Processing is then passed on to the file
`{\textit{dest}\hspace{0.2em}\textit{suffix}}'.
Surely, the same effect is achieved by
directly specifying the
argument `{\textit{dest}\hspace{0.2em}\textit{suffix}}'
in the first form.
However, that requires to set up a different file
for each child. With the alternative form of the command
all these files can have exactly the same content
which simplifies setting them up and maintaining them.

For example, the following file |draft.tex|
with a compilation flag |\version| as described in \secref{sec:flags}
compiles the main document as a draft:
%
\begin{center}
\begin{tabular}{l}
|\def\version{draft}|\\
|\input{childdoc.def}|\\
|\childdocforward{|\textit{main}|}|
\end{tabular}
\end{center}
%
Likewise, the following files |final|\textit{nn}|.tex|
compile the final version of the child document
|child|\textit{nn}|.tex|:
%
\begin{center}
\begin{tabular}{l}
|\def\version{final}|\\
|\input{childdoc.def}|\\
|\childdocforwardprefix{final}{child}|
\end{tabular}
\end{center}
%

Note that when several versions of a main file and/or of each child file
are to be generated, it may be convenient to set up a |Makefile| or
shell script to automatise the process.

%%%%%%%%%%%%%%%%%%%%%%%%%%%%%%%%%%%%%%%%%%%%%%%%%%%%%%%%%%%%%%%%%%%%%%%%%%%%%%%%
\subsection{Command Line Processing}
\label{sec:commandline}

The effect of redirection files can also be achieved by invoking
the \LaTeX{} compiler with a more elaborate command line.
Most conveniently this should be done as part
of a shell script or a |Makefile|.

When using \textsf{childdoc} in the main file, the following
command lines effectively perform a redirection
(note that depending on the shell being used,
backslashes may have to be doubled: `|\|' $\to$ `|\\|'):
%
\begin{center}
|... -jobname "|\textit{target}|" |\\|"|[\textit{flags}]%
|\input{childdoc.def}\childdocforward[|\textit{main}|]{|\textit{dest}|}"|
\end{center}
%
Here \textit{target} is the name of the output file,
\textit{main} is the name of the main file
and \textit{dest} is the name of the main or child file to be processed
(all filenames without extensions).
The optional argument \textit{main} can be omitted
if \textit{main} matches \textit{dest}.
Optionally, compilation \textit{flags} can be defined via |\def| commands.
This command line makes the \TeX{} engine believe
it is compiling the file \textit{target}
whose content is specified as the latter parameter.
The provided code then forwards the processing to
\textit{main} or \textit{dest} as described in \secref{sec:forward}.

%%%%%%%%%%%%%%%%%%%%%%%%%%%%%%%%%%%%%%%%%%%%%%%%%%%%%%%%%%%%%%%%%%%%%%%%%%%%%%%%
\subsection{Include by Input}
\label{sec:input}

Including child documents by |\include| has some restrictions by design.
Most notably, the content of a child document always occupies
its own set of pages; pages cannot be shared between child documents.
Usually, this behaviour makes perfect sense
because each child document contain an essential part of the document.
However, in some situations it may be desirable to compose
a document from a collection of parts
without having mandatory page breaks between then.
For this case, the package
provides a mechanism to include parts
by |\input| which can also be processed individually.
However, by construction this mechanism
requires manual handling of the content to be output.

%%%%%%%%%%%%%%%%%%%%%%%%%%%%%%%%%%%%%%%%
\DescribeMacro{\ifchilddocmanual}
The main file should be prepared as usual, see \secref{sec:include}.
However, the document body must make a distinction
between processing of an individual part and of the main document, e.g.:
%
\begin{center}
\begin{tabular}{l}
|\ifchilddocmanual|\\
|\input{\childdocname}|\\
|\||else|\\
\textit{document body with }|\input{|\textit{part}|}|\\
|\||fi|
\end{tabular}
\end{center}
%
The conditional |\ifchilddocmanual| is true whenever
a part to be included by |\input| is being compiled,
and the name of the part is stored in |\childdocname|.

%%%%%%%%%%%%%%%%%%%%%%%%%%%%%%%%%%%%%%%%
\DescribeMacro{\childdocby}
Each part to be included by |\input| should start with:
%
\begin{center}
\begin{tabular}{l}
|\input{childdoc.def}|\\
|\childdocby{|\textit{main}|}|\\
\end{tabular}
\end{center}
%
The directive |\childdocby| is similar to |\childdocof|
described in \secref{sec:include},
but the subsequent selection of content must be done manually.
To that end, both |\ifchilddoc| and |\ifchilddocmanual|
will be true upon processing of a part,
and the name of the part is stored in |\childdocname|.
Note that |\jobname| will be set to the filename of the current part
so that each part receives an individual |.aux| file
that does not interfere with the |.aux| file(s) of the main document.
This behaviour can be altered by the alternative form
|\childdocby[*]{|\textit{main}|}| (with a non-empty optional argument)
which uses the |.aux| file of the main document
by setting |\jobname| to \textit{main}.

%%%%%%%%%%%%%%%%%%%%%%%%%%%%%%%%%%%%%%%%%%%%%%%%%%%%%%%%%%%%%%%%%%%%%%%%%%%%%%%%
\subsection{Driver Development}
\label{sec:driver}

The \textsf{childdoc} mechanism can also be use for the development
of definition files such as \LaTeX{} styles or classes.
This case differs from the above setup with multiple parts
included by |\include| in that no |\includeonly| should be invoked.
This can be achieved by starting the include file
(before |\ProvidesPackage|) with:
%
\begin{center}
\begin{tabular}{l}
|\input{childdoc.def}|\\
|\childdocforward{|\textit{main}|}|\\
\end{tabular}
\end{center}
%
or alternatively with:
%
\begin{center}
\begin{tabular}{l}
|\input{childdoc.def}|\\
|\childdocby{|\textit{main}|}|\\
\end{tabular}
\end{center}
%
Both forms have slightly different effects as described above.
The main file is prepared as usual, see \secref{sec:include}.

%%%%%%%%%%%%%%%%%%%%%%%%%%%%%%%%%%%%%%%%%%%%%%%%%%%%%%%%%%%%%%%%%%%%%%%%%%%%%%%%
\subsection{Legacy Detection}
\label{sec:detection}

The directive |\childdocmain| in the main file can detect
whether the complete document or merely a child is to be compiled
even without using the directive |\childdocof|.
This method is deprecated because it is less robust
and there is no compelling reason to use it;
it is merely provided for backward compatibility
and it may be removed in future versions.

If the detection mechanism is to be used,
it is mandatory to correctly specify
the filename of the main file as the argument of |\childdocmain|:
%
\begin{center}
\begin{tabular}{l}
|\input{childdoc.def}|\\
|\childdocmain{|\textit{main}|}|\\
\end{tabular}
\end{center}
%
If |\jobname| does not match the argument \textit{main} of |\childdocmain|,
it is assumed that |\jobname| points to the child file to be compiled.
When using |\childdocmain| with the main file specified as argument,
it suffices to start a child file
with just |\input{|\textit{main}|}|
without loading of the package and using |\childdocof|.
If instead all processing is done
with the appropriate \textsf{childdoc} directives,
the argument of \textit{main} of |\childdocmain| can be empty.

An alternative version of the command line processing described
in \secref{sec:commandline} using the detection mechanism reads:
%
\begin{center}
|... -jobname "|\textit{target}|" "|[\textit{flags}]%
[|\def\jobname{|\textit{dest}|}|]|\input{|\textit{main}|}"|
\end{center}

%%%%%%%%%%%%%%%%%%%%%%%%%%%%%%%%%%%%%%%%%%%%%%%%%%%%%%%%%%%%%%%%%%%%%%%%%%%%%%%%
\subsection{Manual Code}
\label{sec:manual}

In case one cannot be certain whether the definitions file |childdoc.def|
is installed on the target \TeX{} distribution
and one prefers not to ship it,
it is conceivable to paste a few relevant commands into the sources.

To that end, drop all statements |\input{childdoc.def}|
and perform the replacements as outlined below.
Instead of |\childdocmain{|\textit{main}|}| add the following code
to the top of the main file:
%
\begin{center}
\begin{tabular}{l}
|\||ifdefined\childdocname\endinput\||fi\newif\ifchilddoc|\\
|\edef\childdocname{\scantokens\expandafter{\jobname\noexpand}}|\\
|\def\childdocmain{|\textit{main}|}\||ifx\childdocmain\childdocname\||else|\\
|\childdoctrue\includeonly{\childdocname}\let\jobname\childdocmain\||fi|\\
\end{tabular}
\end{center}
%
Instead of |\childdocof{|\textit{main}|}| just include the main file
at the top of each child file:
%
\begin{center}
|\input{|\textit{main}|}|
\end{center}
%
A simple redirection |\childdocforward{|\textit{dest}|}| is achieved by:
%
\begin{center}
|\def\jobname{|\textit{dest}|}\input{\jobname}|
\end{center}
%
The redirection with prefix
|\childdocforwardprefix[|\textit{prefix}|]{|\textit{dest}|}|
is accomplished by:
%
\begin{center}
\begin{tabular}{l}
|{\edef\jobname{\scantokens\expandafter{\jobname\noexpand}}|\\
|\def\redirectjob |\textit{prefix}|#1~~~{\gdef\jobname{|\textit{dest}|#1}}|\\
|\expandafter\redirectjob\jobname~~~}\input{\jobname}|
\end{tabular}
\end{center}

In an alternative approach,
child documents can be compiled by a specific command line
without additional code or specific definitions:
%
\begin{center}
|... -jobname "|\textit{target}|" "|[\textit{flags}]%
|\includeonly{|\textit{dest}|}\input{|\textit{main}|}"|
\end{center}
%

%%%%%%%%%%%%%%%%%%%%%%%%%%%%%%%%%%%%%%%%%%%%%%%%%%%%%%%%%%%%%%%%%%%%%%%%%%%%%%%%
%%%%%%%%%%%%%%%%%%%%%%%%%%%%%%%%%%%%%%%%%%%%%%%%%%%%%%%%%%%%%%%%%%%%%%%%%%%%%%%%
\section{Information}

%%%%%%%%%%%%%%%%%%%%%%%%%%%%%%%%%%%%%%%%%%%%%%%%%%%%%%%%%%%%%%%%%%%%%%%%%%%%%%%%
\subsection{Copyright}

Copyright \copyright{} 2017--2018 Niklas Beisert

This work may be distributed and/or modified under the
conditions of the \LaTeX{} Project Public License, either version 1.3
of this license or (at your option) any later version.
The latest version of this license is in
  \url{http://www.latex-project.org/lppl.txt}
and version 1.3 or later is part of all distributions of \LaTeX{}
version 2005/12/01 or later.

This work has the LPPL maintenance status `maintained'.

The Current Maintainer of this work is Niklas Beisert.

This work consists of the files |README.txt|, |childdoc.ins| and |childdoc.dtx|
as well as the derived files |childdoc.def|, |cdocsamp.tex|
with |cdocsch1.tex|, |cdocsch2.tex|, |cdocspt3.tex|, |cdocspt4.tex|,
|cdocsdrf.tex|, |cdocsfn1.tex|, |cdocsfn2.tex|
as well as |childdoc.pdf|.

%%%%%%%%%%%%%%%%%%%%%%%%%%%%%%%%%%%%%%%%%%%%%%%%%%%%%%%%%%%%%%%%%%%%%%%%%%%%%%%%
\subsection{Files and Installation}

The package consists of the files:
%
\begin{center}
\begin{tabular}{ll}
    |README.txt|   & readme file \\
    |childdoc.ins| & installation file \\
    |childdoc.dtx| & source file \\
    |childdoc.def| & definition file \\
    |cdocsamp.tex| & sample main file \\
    |cdocsch1.tex| & sample include file \\
    |cdocsch2.tex| & sample include file \\
    |cdocspt3.tex| & sample part file \\
    |cdocspt4.tex| & sample part file \\
    |cdocsdrf.tex| & sample redirection file \\
    |cdocsfn1.tex| & sample redirection file \\
    |cdocsfn2.tex| & sample redirection file \\
    |childdoc.pdf| & manual
\end{tabular}
\end{center}
%
The distribution consists of the files
|README.txt|, |childdoc.ins| and |childdoc.dtx|.
%
\begin{itemize}
\item
Run (pdf)\LaTeX{} on |childdoc.dtx|
to compile the manual |childdoc.pdf| (this file).
\item
Run \LaTeX{} on |childdoc.ins| to create the definitions file |childdoc.def|
and the sample |cdocsamp.tex| with include files
|cdocsch1.tex|, |cdocsch2.tex|, |cdocspt3.tex|, |cdocspt4.tex|,
|cdocsdrf.tex|, |cdocsfn1.tex|, |cdocsfn2.tex|.
Then copy the file |childdoc.def| to an appropriate directory of your \LaTeX{}
distribution, e.g.\ \textit{texmf-root}|/tex/latex/childdoc|.
\end{itemize}

%%%%%%%%%%%%%%%%%%%%%%%%%%%%%%%%%%%%%%%%%%%%%%%%%%%%%%%%%%%%%%%%%%%%%%%%%%%%%%%%
\subsection{Related CTAN Packages}

There are several other packages which offer a similar functionality:
%
\begin{itemize}
\item
The packages
\href{http://ctan.org/pkg/docmute}{\textsf{docmute}},
\href{http://ctan.org/pkg/includex}{\textsf{includex}} and
\href{http://ctan.org/pkg/standalone}{\textsf{standalone}}
provide commands to include only the document body of
a child file thus allowing both files to be compiled individually.
\item
The packages \href{http://ctan.org/pkg/subdocs}{\textsf{subdocs}}
and \href{http://ctan.org/pkg/subfiles}{\textsf{subfiles}}
provide structures in which the main and child documents can be
encapsulated and allowing them to be compiled individually.
The inclusion mechanism is different from the conventional |\include|.
\item
The package \href{http://ctan.org/pkg/combine}{\textsf{combine}}
is an elaborate solution to combine several documents into one.
\end{itemize}
%
See also the CTAN topic \href{http://ctan.org/topic/subdocs}{\textsf{subdocs}}
for further related packages.
The present package differs from the above solutions in that
a document structure constructed with the conventional |\include| mechanism
just needs two extra commands at the top of every file
such that all constituent files can be compiled individually.

%%%%%%%%%%%%%%%%%%%%%%%%%%%%%%%%%%%%%%%%%%%%%%%%%%%%%%%%%%%%%%%%%%%%%%%%%%%%%%%%
%\subsection{Feature Suggestions}
%
%The following is a list of features which may be useful for future
%versions of this package:
%%
%\begin{itemize}
%\item
%\ldots
%\end{itemize}

%%%%%%%%%%%%%%%%%%%%%%%%%%%%%%%%%%%%%%%%%%%%%%%%%%%%%%%%%%%%%%%%%%%%%%%%%%%%%%%%
\subsection{Revision History}

%%%%%%%%%%%%%%%%%%%%%%%%%%%%%%%%%%%%%%%%
\paragraph{v2.0:} 2018/12/30

\begin{itemize}
\item
immediate forward processing
\item
added |\childdocby| mechanism
\item
manual restructured
\end{itemize}

%%%%%%%%%%%%%%%%%%%%%%%%%%%%%%%%%%%%%%%%
\paragraph{v1.6:} 2018/01/17

\begin{itemize}
\item
application for development of include files
\item
corrections to manual
\end{itemize}

%%%%%%%%%%%%%%%%%%%%%%%%%%%%%%%%%%%%%%%%
\paragraph{v1.5:} 2017/05/21

\begin{itemize}
\item
more complete structuring introduced
\item
|\childdocof| introduced
\item
|\childdoc| renamed to |\childdocmain|
\item
|\childredirect| renamed to |\childdocforward| and |\childdocforwardprefix|
and functionality expanded
\end{itemize}

%%%%%%%%%%%%%%%%%%%%%%%%%%%%%%%%%%%%%%%%
\paragraph{v1.0:} 2017/04/27

\begin{itemize}
\item
manual and install package
\item
first version published on CTAN
\end{itemize}

%%%%%%%%%%%%%%%%%%%%%%%%%%%%%%%%%%%%%%%%
\paragraph{v0.6:} 2017/04/26

\begin{itemize}
\item
redirection mechanism added
\end{itemize}

%%%%%%%%%%%%%%%%%%%%%%%%%%%%%%%%%%%%%%%%
\paragraph{v0.5:} 2017/04/26

\begin{itemize}
\item
functionality in definition file
\end{itemize}


%%%%%%%%%%%%%%%%%%%%%%%%%%%%%%%%%%%%%%%%%%%%%%%%%%%%%%%%%%%%%%%%%%%%%%%%%%%%%%%%
%%%%%%%%%%%%%%%%%%%%%%%%%%%%%%%%%%%%%%%%%%%%%%%%%%%%%%%%%%%%%%%%%%%%%%%%%%%%%%%%
%%%%%%%%%%%%%%%%%%%%%%%%%%%%%%%%%%%%%%%%%%%%%%%%%%%%%%%%%%%%%%%%%%%%%%%%%%%%%%%%
\appendix

\settowidth\MacroIndent{\rmfamily\scriptsize 000\ }

 \DocInput{childdoc.dtx}

\end{document}
%</driver>
% \fi
%
% %%%%%%%%%%%%%%%%%%%%%%%%%%%%%%%%%%%%%%%%%%%%%%%%%%%%%%%%%%%%%%%%%%%%%%%%%%%%%%
% %%%%%%%%%%%%%%%%%%%%%%%%%%%%%%%%%%%%%%%%%%%%%%%%%%%%%%%%%%%%%%%%%%%%%%%%%%%%%%
% \section{Sample}
%\iffalse
%<*samplemain>
%\fi
%
% The following presents a sample document
% with two chapters, two parts, a title page,
% a compile flag as well as three forwarding files to set the flag.
% It consists of eight |.tex| files:
% \begin{center}
% \begin{tabular}{ll}
% |cdocsamp.tex|&main file\\
% |cdocsch1.tex|&include file for chapter 1\\
% |cdocsch2.tex|&include file for chapter 2\\
% |cdocspt3.tex|&include file for part 3\\
% |cdocspt4.tex|&include file for part 4\\
% |cdocsdrf.tex|&forwarding file for main file in draft mode\\
% |cdocsfi1.tex|&forwarding file for final version of chapter 1\\
% |cdocsfi2.tex|&forwarding file for final version of chapter 2\\
% \end{tabular}
% \end{center}
% Each of the eight files can be compiled directly by the \LaTeX{} compiler.
%
% %%%%%%%%%%%%%%%%%%%%%%%%%%%%%%%%%%%%%%
% \paragraph{Main File.}
%
% The main file is called |cdocsamp.tex|.
%
% Load the \textsf{childdoc} definitions and
% declare the filename for the main document:
%    \begin{macrocode}
\input{childdoc.def}
\childdocmain{}
%    \end{macrocode}

% Optional override for |\version| flag:
%    \begin{macrocode}
%%\ifchilddoc\else\providecommand{\version}{draft}\fi
%    \end{macrocode}

% Define the default values for the |\version| flag
% (|final| for the main file and |draft| for childs):
%    \begin{macrocode}
\ifchilddoc
\providecommand{\version}{draft}
\else
\providecommand{\version}{final}
\fi
%    \end{macrocode}

% Load the standard document class:
%    \begin{macrocode}
\documentclass[12pt]{article}
%    \end{macrocode}

% Start the document body:
%    \begin{macrocode}
\begin{document}
%    \end{macrocode}

% Declare a title page.
% Print title, part of document being processed and version flag:
%    \begin{macrocode}
\addtocounter{page}{-1}
\begin{center}
{\LARGE\bfseries{}childdoc example\par}
\vspace{1cm}
\ifchilddoc
\ifchilddocmanual part\else chapter\fi:
`\childdocname' of `\childdocjob'\par
\else
main document: `\childdocjob'\par
\fi
version: \version\par
\end{center}
\newpage
%    \end{macrocode}

% Manually include selected file,
% otherwise process as usual:
%    \begin{macrocode}
\ifchilddocmanual
\section*{part `\childdocname'}
\input{\childdocname}
\else
%    \end{macrocode}

% Include the two chapters:
%    \begin{macrocode}
\include{cdocsch1}
\include{cdocsch2}
%    \end{macrocode}

% Include the two parts unless only chapters should be displayed:
%    \begin{macrocode}
\ifchilddoc\else
\section{part three}
\input{cdocspt3}
\section{part four}
\input{cdocspt4}
\fi
%    \end{macrocode}

% Process as usual until here:
%    \begin{macrocode}
\fi
%    \end{macrocode}

% End of document body:
%    \begin{macrocode}
\end{document}
%    \end{macrocode}
%\iffalse
%</samplemain>
%\fi
%
% %%%%%%%%%%%%%%%%%%%%%%%%%%%%%%%%%%%%%%
% \paragraph{Chapter Include Files.}
%
% The include files are called |cdocsch1.tex| and |cdocsch2.tex|.
%
%\iffalse
%<*samplechap1|samplechap2>
%\fi

% Optional override for |\version| flag:
%    \begin{macrocode}
%%\providecommand{\version}{final}
%    \end{macrocode}

% Include the main document:
%    \begin{macrocode}
\input{childdoc.def}
\childdocof{cdocsamp}
%    \end{macrocode}

%\iffalse
%</samplechap1|samplechap2>
%\fi
%
%\iffalse
%<*samplechap1>
%\fi
% Some text for chapter 1:
%    \begin{macrocode}
\section{one}
some text in chapter one
%    \end{macrocode}

%\iffalse
%</samplechap1>
%\fi
% Some text for chapter 2:
%\iffalse
%<*samplechap2>
%\fi
%    \begin{macrocode}
\section{two}
more text in chapter two
%    \end{macrocode}

%\iffalse
%</samplechap2>
%\fi
%
% %%%%%%%%%%%%%%%%%%%%%%%%%%%%%%%%%%%%%%
% \paragraph{Part Include Files.}
%
% The include files are called |cdocspt3.tex| and |cdocspt4.tex|.
%
%\iffalse
%<*samplepart3|samplepart4>
%\fi

% Optional override for |\version| flag:
%    \begin{macrocode}
%%\providecommand{\version}{final}
%    \end{macrocode}

% Include the main document:
%    \begin{macrocode}
\input{childdoc.def}
\childdocby{cdocsamp}
%    \end{macrocode}

%\iffalse
%</samplepart3|samplepart4>
%\fi
%
%\iffalse
%<*samplepart3>
%\fi
% Some text for part 3:
%    \begin{macrocode}
some text in part three
%    \end{macrocode}

%\iffalse
%</samplepart3>
%\fi
% Some text for part 4:
%\iffalse
%<*samplepart4>
%\fi
%    \begin{macrocode}
more text in part four
%    \end{macrocode}

%\iffalse
%</samplepart4>
%\fi
%
% %%%%%%%%%%%%%%%%%%%%%%%%%%%%%%%%%%%%%%
% \paragraph{Forwarding for a Complete Draft.}
%
% The following forwarding file |cdocsdrf.tex|
% compiles the main document in draft mode:
%\iffalse
%<*sampledraft>
%\fi
%    \begin{macrocode}
\def\version{draft}
\input{childdoc.def}
\childdocforward{cdocsamp}
%    \end{macrocode}

%\iffalse
%</sampledraft>
%\fi
%
% %%%%%%%%%%%%%%%%%%%%%%%%%%%%%%%%%%%%%%
% \paragraph{Forwarding for Final Version of the Chapters.}
%
% The following forwarding files |cdocsfn1.tex| and |cdocsfn2.tex|
% (with identical content)
% compile the final versions of the child documents
% |cdocsch1.tex| and |cdocsch2.tex|, respectively:
%\iffalse
%<*samplefinal>
%\fi
%    \begin{macrocode}
\def\version{final}
\input{childdoc.def}
\childdocforwardprefix[cdocsamp]{cdocsfn}{cdocsch}
%    \end{macrocode}

%\iffalse
%</samplefinal>
%\fi
%
% %%%%%%%%%%%%%%%%%%%%%%%%%%%%%%%%%%%%%%
% \paragraph{Command Line Processing.}
%
% The following three command lines generate the output files
% |cdocscld|, |cdocscl1| and |cdocscl2|
% which should be identical to
% |cdocsdrf|, |cdocsch1| and |cdocsfn2|, respectively:
% \begin{center}
% \begin{tabular}{l}
% |latex -jobname cdocscld \|\\
% |  "\def\version{draft}\input{childdoc.def}\childdocforward{cdocsamp}"|\\
% |latex -jobname cdocscl1 \|\\
% |  "\input{childdoc.def}\childdocforward[cdocsamp]{cdocsch1}"|\\
% |latex -jobname cdocscl2 \|\\
% |  "\def\version{final}\input{childdoc.def}\childdocforward{cdocsch2}"|
% \end{tabular}
% \end{center}
% Note that the trailing backslash on each first line
% merely continues the input to the second line
% (for convenient cut ant paste).
% Furthermore, the command |latex| can be replaced by any
% of its alternative versions such as |pdflatex|.
%
% %%%%%%%%%%%%%%%%%%%%%%%%%%%%%%%%%%%%%%%%%%%%%%%%%%%%%%%%%%%%%%%%%%%%%%%%%%%%%%
% %%%%%%%%%%%%%%%%%%%%%%%%%%%%%%%%%%%%%%%%%%%%%%%%%%%%%%%%%%%%%%%%%%%%%%%%%%%%%%
% \section{Implementation}
%\iffalse
%<*package>
%\fi
%
% This section describes the definitions file |childdoc.def|.

% The definitions cannot be loaded using |\usepackage| or |\RequirePackage|
% which has a mechanism to prevent loading a style file more than once.
% When loading the definitions by means of |\input|
% multiple instances have to be prevented manually:
%\iffalse
%This code needs to be before the `\ProvidesFile' directive
%which is defined at the beginning of this file.
%Therefore it is also placed there and commented out here.
%</package>
%<*discard>
%\fi
%    \begin{macrocode}
\ifdefined\childdocmain\endinput\fi
%    \end{macrocode}
%\iffalse
%</discard>
%<*package>
%\fi
%
% \macro{\ifchilddoc}
% \macro{\ifchilddocmanual}
% The conditional |\ifchilddoc| tells whether a
% child (true) or main (false) document is being compiled.
% The conditional |\ifchilddocmanual| tells whether
% the |\includeonly| mechanism is used (false) or
% the selection of child files must be performed manually (true).
% The definitions initialise to false:
%    \begin{macrocode}
\newif\ifchilddoc
\newif\ifchilddocmanual
%    \end{macrocode}

% \macro{\childdocname}
% \macro{\childdocjob}
% The macro |\childdocname| stores the name of the main document
% to be compiled. The macro |\childdocjob| stores the name of
% the document on which the \LaTeX{} compiler was originally invoked.
% The content of |\jobname| cannot be compared
% to filenames specified in the source due to different catcodes.
% The following code rescans |\jobname|, stores the result
% in |\childdocname| and saves a copy in |\childdocjob|:
%    \begin{macrocode}
\edef\childdocname{\scantokens\expandafter{\jobname\noexpand}}
\let\childdocjob\childdocname
%    \end{macrocode}

% \macro{\childdocdisable}
% The macro |\childdocdisable| prevents the main file
% from being processed more than once.
% At this stage, the main document command |\childdocmain|
% is assumed to be called once again where it should do nothing.
% Any subsequent call to it should prevent
% a secondary processing of the main document
% It overwrites the forwarding commands
% |\childdocof| and |\childdocforward|
% with empty macros to prevent further inclusions of the main document:
%    \begin{macrocode}
\newcommand{\childdocdisable}
{
  \renewcommand{\childdocmain}[1]{\renewcommand{\childdocmain}[1]{\endinput}}
  \renewcommand{\childdocof}[1]{}
  \renewcommand{\childdocby}[2][]{}
  \renewcommand{\childdocforward}[2][]{}
  \renewcommand{\childdocdisable}{}
}
%    \end{macrocode}

% \macro{\childdocmain}
% The macro |\childdocmain| is to be called at the top of the main file
% with nothing or the main filename (without extension) as argument.
% First, it breaks loops.
% If the argument is not empty and does not match |\childdocname|
% (which is set by the first inclusion of |childdoc.def|),
% |\ifchilddoc| is set to true, |\includeonly| is applied to the child file
% and |\jobname| is set to the main file
% (for proper handling of |.aux| files):
%    \begin{macrocode}
\newcommand{\childdocmain}[1]
{
  \childdocdisable\childdocmain{}
  \if?#1?\else
    \begingroup
      \def\childdoctmp{#1}
      \ifx\childdoctmp\childdocname
        \def\childdoctmp{}
      \else
        \def\childdoctmp
        {
          \childdoctrue
          \includeonly{\childdocname}
          \def\childdocjob{#1}
          \def\jobname{#1}
        }
      \fi
      \expandafter
    \endgroup
    \childdoctmp
  \fi
}
%    \end{macrocode}

% \macro{\childdocof}
% The command |\childdocof| redirects
% compilation to the main file |#1|.
%    \begin{macrocode}
\newcommand{\childdocof}[1]
{
  \childdocdisable
  \childdoctrue
  \includeonly{\childdocname}
  \def\jobname{#1}
  \def\childdocjob{#1}
  \input{#1}
}
%    \end{macrocode}

% \macro{\childdocby}
% The command |\childdocby| ....
%    \begin{macrocode}
\newcommand{\childdocby}[2][]
{
  \childdocdisable
  \childdoctrue
  \childdocmanualtrue
  \if?#1?\else
    \def\jobname{#2}
  \fi
  \def\childdocjob{#2}
  \input{#2}
  \endinput
}
%    \end{macrocode}

% \macro{\childdocforward}
% The command |\childdocforward| redirects
% compilation to the main file or
% (if the optional argument is given) a child file.
% Parameters are set as if the main file
% or a child file starting with |\childdocof| was compiled.
% Then compilation is handed over to the main file:
%    \begin{macrocode}
\newcommand{\childdocforward}[2][]
{
  \begingroup
    \if?#1?
      \def\childdoctmp
      {
        \def\childdocname{#2}
        \def\childdocjob{#2}
        \def\jobname{#2}
        \input{#2}
        \endinput
      }
    \else
      \def\childdoctmp
      {
        \childdocdisable
        \def\childdocname{#2}
        \childdoctrue
        \includeonly{#2}
        \def\childdocjob{#1}
        \def\jobname{#1}
        \input{#1}
        \endinput
      }
    \fi
    \expandafter
  \endgroup
  \childdoctmp
}
%    \end{macrocode}

% \macro{\childdocforwardprefix}
% The command |\childdocforwardprefix| redirects
% compilation to the main or a child file by means of a pattern.
% The prefix |#1| in the current filename is replaced by |#2|
% and the suffix of the current filename is kept
% (it is assumed that the filename does not contain the substring `|~~~|'
% which is used as a delimiter).
% Compilation is handed over to the new file by |\childdocforward|:
%    \begin{macrocode}
\newcommand{\childdocforwardprefix}[3][]
{
  \begingroup
    \def\childdocextract #2##1~~~{\def\childdoctmp{\childdocforward[#1]{#3##1}}}
    \expandafter\childdocextract\childdocname~~~
    \expandafter
  \endgroup
  \childdoctmp
}
%    \end{macrocode}

% \macro{\childdoc}
% The deprecated macro |\childdoc| is a legacy version of |\childdocmain|:
%    \begin{macrocode}
\newcommand{\childdoc}{\childdocmain}
%    \end{macrocode}

% \macro{\childdocredirect}
% The deprecated macro |\childdocredirect| is a legacy version
% of |\childdocforward| and |\childdocforwardprefix|:
%    \begin{macrocode}
\newcommand{\childdocredirect}[2][]
{
  \begingroup
    \if?#1?
      \def\childdoctmp{\childdocforward{#2}}
    \else
      \def\childdoctmp{\childdocforwardprefix{#1}{#2}}
    \fi
    \expandafter
  \endgroup
  \childdoctmp
}
%    \end{macrocode}

%\iffalse
%</package>
%\fi
%
\endinput

\childdocforwardprefix[cdocsamp]{cdocsfn}{cdocsch}
%    \end{macrocode}

%\iffalse
%</samplefinal>
%\fi
%
% %%%%%%%%%%%%%%%%%%%%%%%%%%%%%%%%%%%%%%
% \paragraph{Command Line Processing.}
%
% The following three command lines generate the output files
% |cdocscld|, |cdocscl1| and |cdocscl2|
% which should be identical to
% |cdocsdrf|, |cdocsch1| and |cdocsfn2|, respectively:
% \begin{center}
% \begin{tabular}{l}
% |latex -jobname cdocscld \|\\
% |  "\def\version{draft}% \iffalse
%
% childdoc.dtx Copyright (C) 2017-2018 Niklas Beisert
%
% This work may be distributed and/or modified under the
% conditions of the LaTeX Project Public License, either version 1.3
% of this license or (at your option) any later version.
% The latest version of this license is in
%   http://www.latex-project.org/lppl.txt
% and version 1.3 or later is part of all distributions of LaTeX
% version 2005/12/01 or later.
%
% This work has the LPPL maintenance status `maintained'.
%
% The Current Maintainer of this work is Niklas Beisert.
%
% This work consists of the files childdoc.dtx and childdoc.ins
% and the derived files childdoc.def and cdocsamp.tex with
% cdocsch1.tex, cdocsch2.tex, cdocsdrf.tex, cdocsfn1.tex, cdocsfn2.tex.
%
%<package>\ifdefined\childdocmain\endinput\fi
%<package>\ProvidesFile{childdoc.def}[2018/12/30 v2.0 child document driver]
%<samplemain>\ProvidesFile{cdocsamp.tex}[2018/12/30 v2.0 sample for childdoc]
%<*driver>
%\ProvidesFile{childdoc.drv}[2018/12/30 v2.0 childdoc reference manual file]
\PassOptionsToClass{10pt,a4paper}{article}
\documentclass{ltxdoc}

\usepackage[margin=35mm]{geometry}
\usepackage{hyperref}
\usepackage{hyperxmp}
\usepackage[usenames]{color}

\hypersetup{colorlinks=true}
\hypersetup{pdfstartview=FitH}
\hypersetup{pdfpagemode=UseNone}
\hypersetup{pdfsource={}}
\hypersetup{pdflang={en-UK}}
\hypersetup{pdfcopyright={Copyright 2017-2018 Niklas Beisert.
  This work may be distributed and/or modified under the
  conditions of the LaTeX Project Public License, either version 1.3
  of this license or (at your option) any later version.}}
\hypersetup{pdflicenseurl={http://www.latex-project.org/lppl.txt}}
\hypersetup{pdfcontactaddress={ETH Zurich, ITP, HIT K,
  Wolfgang-Pauli-Strasse 27}}
\hypersetup{pdfcontactpostcode={8093}}
\hypersetup{pdfcontactcity={Zurich}}
\hypersetup{pdfcontactcountry={Switzerland}}
\hypersetup{pdfcontactemail={nbeisert@itp.phys.ethz.ch}}
\hypersetup{pdfcontacturl={http://people.phys.ethz.ch/\xmptilde nbeisert/}}

\newcommand{\secref}[1]{\hyperref[#1]{section \ref*{#1}}}

\parskip1ex
\parindent0pt
\let\olditemize\itemize
\def\itemize{\olditemize\parskip0pt}

\begin{document}

\title{The \textsf{childdoc} Package}
\hypersetup{pdftitle={The childdoc Package}}
\author{Niklas Beisert\\[2ex]
  Institut f\"ur Theoretische Physik\\
  Eidgen\"ossische Technische Hochschule Z\"urich\\
  Wolfgang-Pauli-Strasse 27, 8093 Z\"urich, Switzerland\\[1ex]
  \href{mailto:nbeisert@itp.phys.ethz.ch}
  {\texttt{nbeisert@itp.phys.ethz.ch}}}
\hypersetup{pdfauthor={Niklas Beisert}}
\hypersetup{pdfsubject={Manual for the LaTeX2e Package childdoc}}
\date{30 December 2018, \textsf{v2.0}}
\maketitle

\begin{abstract}\noindent
\textsf{childdoc} is a \LaTeXe{} package
that enables the direct compilation
of document sections included by |\include|
to individual files.
\end{abstract}

\begingroup
\parskip0ex
\tableofcontents
\endgroup

%%%%%%%%%%%%%%%%%%%%%%%%%%%%%%%%%%%%%%%%%%%%%%%%%%%%%%%%%%%%%%%%%%%%%%%%%%%%%%%%
%%%%%%%%%%%%%%%%%%%%%%%%%%%%%%%%%%%%%%%%%%%%%%%%%%%%%%%%%%%%%%%%%%%%%%%%%%%%%%%%
\section{Introduction}

\LaTeX{} provides a mechanism to structure a large document (such as a book)
into a main file and several child files (containing the chapters)
using the |\include| command.
This mechanism is beneficial for documents
which span hundreds of pages in order to
make the source file(s) more manageable.
Moreover, compilation can be restricted to
selected child files by means of the |\includeonly| command.
The latter feature can be used to reduce the compilation time while editing
(this was significantly more useful in the earlier days of \LaTeX{})
or to generate a smaller document which is easier to navigate.
Another application of |\includeonly| is to generate
documents consisting of selected parts of the complete document.

However, there are a few drawbacks of the plain |\include| mechanism:
\begin{itemize}
\item
The child files cannot be compiled on their own,
they can only be compiled via the main file.
A naive editing environment
(such as a text editor with an option
to have the current file processed by \LaTeX)
may require one to switch to the main file before compiling;
attempting to compile the child file produces errors.
\item
The main file must be modified (each time)
to adjust the |\includeonly| command
to the present needs. This easily leaves the main file in a messy state.
\item
The generated document will always carry the filename
of the main document. This is inconvenient if
several child files are to be compiled and
to be kept for distribution.
\end{itemize}

The present package provides a simple interface
to make child files individually compilable by \LaTeX{}.
Compiling a child file then has the same effect as compiling
the main file with an |\includeonly| command
to select the appropriate child.
Moreover the generated document will carry the name of the child
rather than the main file.
This resolves all three above issues.

This feature is meant to make the editing of books,
thesis documents and lecture notes somewhat more convenient.
However, the package can also be used efficiently for
composing a series of documents (such as exercise sheets)
which are typically distributed individually.
It then assists the author in generating the individual documents
(potentially in different versions)
as well as a document containing the collected series.
Another application is in developing style files
or other kinds of included material
where compilation of the style file could redirect
to a sample or test file.

%%%%%%%%%%%%%%%%%%%%%%%%%%%%%%%%%%%%%%%%%%%%%%%%%%%%%%%%%%%%%%%%%%%%%%%%%%%%%%%%
%%%%%%%%%%%%%%%%%%%%%%%%%%%%%%%%%%%%%%%%%%%%%%%%%%%%%%%%%%%%%%%%%%%%%%%%%%%%%%%%
\section{Usage}

First of all, the package \textsf{childdoc} is \emph{not} a standard
\LaTeXe{} |.sty| style file! Therefore it needs to be invoked in
a non-standard way.

%%%%%%%%%%%%%%%%%%%%%%%%%%%%%%%%%%%%%%%%%%%%%%%%%%%%%%%%%%%%%%%%%%%%%%%%%%%%%%%%
\subsection{Included Files}
\label{sec:include}

%%%%%%%%%%%%%%%%%%%%%%%%%%%%%%%%%%%%%%%%
\DescribeMacro{\childdocmain}
To use the package, add the commands
\begin{center}
\begin{tabular}{l}
|\input{childdoc.def}|\\
|\childdocmain{}|\\
\end{tabular}
\end{center}
at the very top of the main \LaTeX{} file,
in particular \emph{before} the |\documentclass| statement!
The argument of |\childdocmain| should be left empty
(but it must be present).

%%%%%%%%%%%%%%%%%%%%%%%%%%%%%%%%%%%%%%%%
\DescribeMacro{\childdocof}
Furthermore, add the commands
\begin{center}
\begin{tabular}{l}
|\input{childdoc.def}|\\
|\childdocof{|\textit{main}|}|\\
\end{tabular}
\end{center}
at the top of every child file \textit{child}
which is included by |\include{|\textit{child}|}|
from within the main file
(or at least for those files to be compiled individually).
The argument \textit{main} must be the filename of the main file.

There are a couple of
considerations in setting up the main and child documents:

%%%%%%%%%%%%%%%%%%%%%%%%%%%%%%%%%%%%%%%%
\paragraph{Restrictions.}

Please note the following restrictions:
\begin{itemize}
\item
|\childdocmain| must be called with one argument \textit{main}
to ensure compatibility with earlier version of the package.
It must either be empty (|\childdocmain{}|)
or precisely match the filename of the main file in which it is specified.
See \secref{sec:detection} for further information.
\item
The filename \textit{main} must be specified without the |.tex| extension.
\item
The filename \textit{main} is case sensitive
(even in case-insensitive file systems)
due to internal string comparison.
\item
The argument \textit{main} should be fully expanded, it cannot be a macro.
\item
Subdirectories and special characters should be avoided in filenames.
\item
The command |\childdocmain{|\textit{main}|}| must be followed by a whitespace.
It should not be followed immediately by another command
or by a comment mark `|%|'.
This is because the \TeX{} parser reads the token immediately following
the argument of |\childdocmain| and puts it
at the beginning of every child section;
however, a white\-space is ignored.
\end{itemize}

%%%%%%%%%%%%%%%%%%%%%%%%%%%%%%%%%%%%%%%%
\paragraph{Content of Main File.}

It is advisable to place all content in the child files included by |\include|.
Any output contained in the main file will appear in all child documents
unless suppressed manually;
it cannot be suppressed automatically by the |\includeonly| directive
and thus should normally be avoided.
A method to include some content in the main file
by means of conditional processing is described in \secref{sec:conditional}.

%%%%%%%%%%%%%%%%%%%%%%%%%%%%%%%%%%%%%%%%
\paragraph{Page Numbering.}

When only a part of the document is compiled,
the appropriate numbering of pages
(as well as other status parameters)
is determined from the |.aux| files.
The latter contain information from previous passes.
However this information needs to propagate through
all intermediate child documents.
Therefore the page numbering in child documents may well
be inconsistent until the complete document is compiled at least once.

A useful (if unconventional) way to always ensure a consistent
page numbering is to restart the numbering in each child document
and denote the pages by `\textit{child}|.|\textit{page}'
where \textit{child} represents the chapter/section number of the child file.
This can be achieved by the command
|\numberwithin{page}{|\textit{child}|}|
of the \textsf{amsmath} package
where \textit{child} can be |chapter| or |section|
depending on the chosen structuring.
Alternatively, one can modify the macro |\thepage| appropriately
and reset the counter |page| at the start of each child file.

%%%%%%%%%%%%%%%%%%%%%%%%%%%%%%%%%%%%%%%%%%%%%%%%%%%%%%%%%%%%%%%%%%%%%%%%%%%%%%%%
\subsection{Conditional Processing}
\label{sec:conditional}

The package provides a mechanism to compile different versions
of a document. To customise the versions further some conditional processing
can come in handy to distinguish which version is being compiled.
The package provides two macros to describe the compilation context:

%%%%%%%%%%%%%%%%%%%%%%%%%%%%%%%%%%%%%%%%
\DescribeMacro{\ifchilddoc}
The conditional |\ifchilddoc| distinguishes between the compilation of
child documents and the main document:
%
\begin{center}
|\ifchilddoc |\textit{child-code}| |[|\||else |\textit{main-code}]| \||fi|
\end{center}

%%%%%%%%%%%%%%%%%%%%%%%%%%%%%%%%%%%%%%%%
\DescribeMacro{\childdocname}
\DescribeMacro{\childdocjob}
The macro |\childdocname| contains the filename (without extension)
of the main or child file being processed.
Note that |\childdocjob| will always contain the name of the main file.

%%%%%%%%%%%%%%%%%%%%%%%%%%%%%%%%%%%%%%%%
\paragraph{Title Page.}

Conditional processing can be used to include a title or banner page
in the main document when proper precautions are taken.
Importantly, the code in the main file should ensure that the page counter
(as well as other status parameters which are stored in the |.aux| files)
takes the same value after the conditional processing.
Otherwise the page numbers may take divergent values
depending on which part is compiled.

For example, a title page could be declared by:
%
\begin{center}
\begin{tabular}{l}
|\ifchilddoc\||else|\\
|\addtocounter{page}{-1}|\\
\textit{code for title page}\\
|\newpage|\\
|\||fi|
\end{tabular}
\end{center}
%
A banner page for the child documents can be generated by:
%
\begin{center}
\begin{tabular}{l}
|\ifchilddoc|\\
|\addtocounter{page}{-1}|\\
\textit{code for banner page}\\
|\newpage|\\
|\||fi|
\end{tabular}
\end{center}
%
Here one could write a message such as:
\begin{center}
|This is the part \childdocname{} of \childdocjob{}.|
\end{center}

%%%%%%%%%%%%%%%%%%%%%%%%%%%%%%%%%%%%%%%%%%%%%%%%%%%%%%%%%%%%%%%%%%%%%%%%%%%%%%%%
\subsection{Flags}
\label{sec:flags}

The package makes it easy to generate different versions
of the main or child documents.
To this end compilation flags can be defined
and assigned different default values.
They will be particularly useful in conjunction
with the forwarding mechanism described in \secref{sec:forward}.

For example, it may be useful to have a flag |\version|
which can be set to |draft| or |final|.
The document source will contain some conditional code
depending on the value of |\version|.
Suppose further, the flag should default to |final| for the main file
and to |draft| for child files
which is a natural assignment for editing the document.
This is achieved by placing the following code
in the preamble of the main document
(below the |\childdocmain| directive):
%
\begin{center}
\begin{tabular}{l}
|\ifchilddoc|\\
|\providecommand{\version}{draft}|\\
|\||else|\\
|\providecommand{\version}{final}|\\
|\||fi|
\end{tabular}
\end{center}
%
The definition by |\providecommand| makes sure
that previous definitions are not overwritten.
Further statements |\providecommand{\version}{...}|
can thus be added before the above code to override it.

For the main file, one might add a line
(between |\childdocmain| and the above block)
%
\begin{center}
|%\ifchilddoc\||else\providecommand{\version}{draft}\||fi|
\end{center}
%
which can be uncommented to produce a draft version.
Likewise one can add a line to the very top of a child file
(above the |\childdocof{|\textit{main}|}| directive)
%
\begin{center}
|%\providecommand{\version}{final}|
\end{center}
%
which can be uncommented to produce the final version of this child document.

%%%%%%%%%%%%%%%%%%%%%%%%%%%%%%%%%%%%%%%%%%%%%%%%%%%%%%%%%%%%%%%%%%%%%%%%%%%%%%%%
\subsection{Forwarding}
\label{sec:forward}

Different versions of the main or child documents
using compilation flags as described in \secref{sec:flags}
can be (permanently) stored in different files
for convenient compilation, viewing and distribution.
To this end, the package defines a command
to pass on compilation to a different file:

%%%%%%%%%%%%%%%%%%%%%%%%%%%%%%%%%%%%%%%%
\DescribeMacro{\childdocforward}
The command |\childdocforward| redirects processing to
another source file:
%
\begin{center}
\begin{tabular}{l}
|\input{childdoc.def}|\\
|\childdocforward[|\textit{main}|]{|\textit{dest}|}|\\
\end{tabular}
\end{center}
%
The argument \textit{dest} is the destination file
(without extension).
It should be the main file or one of the child files.
Note that further \textsf{childdoc} directives
such as |\childdocof| and |\childdocforward|
in the indicated file will be processed in this form.
The optional argument \textit{main}
passes on directly to the main file \textit{main}
while pretending to compile the child \textit{dest}.
This form behaves as if \textit{dest}
issues |\childdocof{|\textit{main}|}| right away,
and no further \textsf{childdoc} directives will be processed.

%%%%%%%%%%%%%%%%%%%%%%%%%%%%%%%%%%%%%%%%
\DescribeMacro{\...prefix}
In the alternative form |\childdocforwardprefix|,
%
\begin{center}
\begin{tabular}{l}
|\input{childdoc.def}|\\
|\childdocforwardprefix[|\textit{main}|]{|\textit{prefix}|}{|\textit{dest}|}|
\end{tabular}
\end{center}
%
the destination file is determined by a pattern
depending on the current file:
To make this work, the current file must be called
`{\textit{prefix}\hspace{0.2em}\textit{suffix}}'
with \textit{prefix} matching precisely the argument.
Processing is then passed on to the file
`{\textit{dest}\hspace{0.2em}\textit{suffix}}'.
Surely, the same effect is achieved by
directly specifying the
argument `{\textit{dest}\hspace{0.2em}\textit{suffix}}'
in the first form.
However, that requires to set up a different file
for each child. With the alternative form of the command
all these files can have exactly the same content
which simplifies setting them up and maintaining them.

For example, the following file |draft.tex|
with a compilation flag |\version| as described in \secref{sec:flags}
compiles the main document as a draft:
%
\begin{center}
\begin{tabular}{l}
|\def\version{draft}|\\
|\input{childdoc.def}|\\
|\childdocforward{|\textit{main}|}|
\end{tabular}
\end{center}
%
Likewise, the following files |final|\textit{nn}|.tex|
compile the final version of the child document
|child|\textit{nn}|.tex|:
%
\begin{center}
\begin{tabular}{l}
|\def\version{final}|\\
|\input{childdoc.def}|\\
|\childdocforwardprefix{final}{child}|
\end{tabular}
\end{center}
%

Note that when several versions of a main file and/or of each child file
are to be generated, it may be convenient to set up a |Makefile| or
shell script to automatise the process.

%%%%%%%%%%%%%%%%%%%%%%%%%%%%%%%%%%%%%%%%%%%%%%%%%%%%%%%%%%%%%%%%%%%%%%%%%%%%%%%%
\subsection{Command Line Processing}
\label{sec:commandline}

The effect of redirection files can also be achieved by invoking
the \LaTeX{} compiler with a more elaborate command line.
Most conveniently this should be done as part
of a shell script or a |Makefile|.

When using \textsf{childdoc} in the main file, the following
command lines effectively perform a redirection
(note that depending on the shell being used,
backslashes may have to be doubled: `|\|' $\to$ `|\\|'):
%
\begin{center}
|... -jobname "|\textit{target}|" |\\|"|[\textit{flags}]%
|\input{childdoc.def}\childdocforward[|\textit{main}|]{|\textit{dest}|}"|
\end{center}
%
Here \textit{target} is the name of the output file,
\textit{main} is the name of the main file
and \textit{dest} is the name of the main or child file to be processed
(all filenames without extensions).
The optional argument \textit{main} can be omitted
if \textit{main} matches \textit{dest}.
Optionally, compilation \textit{flags} can be defined via |\def| commands.
This command line makes the \TeX{} engine believe
it is compiling the file \textit{target}
whose content is specified as the latter parameter.
The provided code then forwards the processing to
\textit{main} or \textit{dest} as described in \secref{sec:forward}.

%%%%%%%%%%%%%%%%%%%%%%%%%%%%%%%%%%%%%%%%%%%%%%%%%%%%%%%%%%%%%%%%%%%%%%%%%%%%%%%%
\subsection{Include by Input}
\label{sec:input}

Including child documents by |\include| has some restrictions by design.
Most notably, the content of a child document always occupies
its own set of pages; pages cannot be shared between child documents.
Usually, this behaviour makes perfect sense
because each child document contain an essential part of the document.
However, in some situations it may be desirable to compose
a document from a collection of parts
without having mandatory page breaks between then.
For this case, the package
provides a mechanism to include parts
by |\input| which can also be processed individually.
However, by construction this mechanism
requires manual handling of the content to be output.

%%%%%%%%%%%%%%%%%%%%%%%%%%%%%%%%%%%%%%%%
\DescribeMacro{\ifchilddocmanual}
The main file should be prepared as usual, see \secref{sec:include}.
However, the document body must make a distinction
between processing of an individual part and of the main document, e.g.:
%
\begin{center}
\begin{tabular}{l}
|\ifchilddocmanual|\\
|\input{\childdocname}|\\
|\||else|\\
\textit{document body with }|\input{|\textit{part}|}|\\
|\||fi|
\end{tabular}
\end{center}
%
The conditional |\ifchilddocmanual| is true whenever
a part to be included by |\input| is being compiled,
and the name of the part is stored in |\childdocname|.

%%%%%%%%%%%%%%%%%%%%%%%%%%%%%%%%%%%%%%%%
\DescribeMacro{\childdocby}
Each part to be included by |\input| should start with:
%
\begin{center}
\begin{tabular}{l}
|\input{childdoc.def}|\\
|\childdocby{|\textit{main}|}|\\
\end{tabular}
\end{center}
%
The directive |\childdocby| is similar to |\childdocof|
described in \secref{sec:include},
but the subsequent selection of content must be done manually.
To that end, both |\ifchilddoc| and |\ifchilddocmanual|
will be true upon processing of a part,
and the name of the part is stored in |\childdocname|.
Note that |\jobname| will be set to the filename of the current part
so that each part receives an individual |.aux| file
that does not interfere with the |.aux| file(s) of the main document.
This behaviour can be altered by the alternative form
|\childdocby[*]{|\textit{main}|}| (with a non-empty optional argument)
which uses the |.aux| file of the main document
by setting |\jobname| to \textit{main}.

%%%%%%%%%%%%%%%%%%%%%%%%%%%%%%%%%%%%%%%%%%%%%%%%%%%%%%%%%%%%%%%%%%%%%%%%%%%%%%%%
\subsection{Driver Development}
\label{sec:driver}

The \textsf{childdoc} mechanism can also be use for the development
of definition files such as \LaTeX{} styles or classes.
This case differs from the above setup with multiple parts
included by |\include| in that no |\includeonly| should be invoked.
This can be achieved by starting the include file
(before |\ProvidesPackage|) with:
%
\begin{center}
\begin{tabular}{l}
|\input{childdoc.def}|\\
|\childdocforward{|\textit{main}|}|\\
\end{tabular}
\end{center}
%
or alternatively with:
%
\begin{center}
\begin{tabular}{l}
|\input{childdoc.def}|\\
|\childdocby{|\textit{main}|}|\\
\end{tabular}
\end{center}
%
Both forms have slightly different effects as described above.
The main file is prepared as usual, see \secref{sec:include}.

%%%%%%%%%%%%%%%%%%%%%%%%%%%%%%%%%%%%%%%%%%%%%%%%%%%%%%%%%%%%%%%%%%%%%%%%%%%%%%%%
\subsection{Legacy Detection}
\label{sec:detection}

The directive |\childdocmain| in the main file can detect
whether the complete document or merely a child is to be compiled
even without using the directive |\childdocof|.
This method is deprecated because it is less robust
and there is no compelling reason to use it;
it is merely provided for backward compatibility
and it may be removed in future versions.

If the detection mechanism is to be used,
it is mandatory to correctly specify
the filename of the main file as the argument of |\childdocmain|:
%
\begin{center}
\begin{tabular}{l}
|\input{childdoc.def}|\\
|\childdocmain{|\textit{main}|}|\\
\end{tabular}
\end{center}
%
If |\jobname| does not match the argument \textit{main} of |\childdocmain|,
it is assumed that |\jobname| points to the child file to be compiled.
When using |\childdocmain| with the main file specified as argument,
it suffices to start a child file
with just |\input{|\textit{main}|}|
without loading of the package and using |\childdocof|.
If instead all processing is done
with the appropriate \textsf{childdoc} directives,
the argument of \textit{main} of |\childdocmain| can be empty.

An alternative version of the command line processing described
in \secref{sec:commandline} using the detection mechanism reads:
%
\begin{center}
|... -jobname "|\textit{target}|" "|[\textit{flags}]%
[|\def\jobname{|\textit{dest}|}|]|\input{|\textit{main}|}"|
\end{center}

%%%%%%%%%%%%%%%%%%%%%%%%%%%%%%%%%%%%%%%%%%%%%%%%%%%%%%%%%%%%%%%%%%%%%%%%%%%%%%%%
\subsection{Manual Code}
\label{sec:manual}

In case one cannot be certain whether the definitions file |childdoc.def|
is installed on the target \TeX{} distribution
and one prefers not to ship it,
it is conceivable to paste a few relevant commands into the sources.

To that end, drop all statements |\input{childdoc.def}|
and perform the replacements as outlined below.
Instead of |\childdocmain{|\textit{main}|}| add the following code
to the top of the main file:
%
\begin{center}
\begin{tabular}{l}
|\||ifdefined\childdocname\endinput\||fi\newif\ifchilddoc|\\
|\edef\childdocname{\scantokens\expandafter{\jobname\noexpand}}|\\
|\def\childdocmain{|\textit{main}|}\||ifx\childdocmain\childdocname\||else|\\
|\childdoctrue\includeonly{\childdocname}\let\jobname\childdocmain\||fi|\\
\end{tabular}
\end{center}
%
Instead of |\childdocof{|\textit{main}|}| just include the main file
at the top of each child file:
%
\begin{center}
|\input{|\textit{main}|}|
\end{center}
%
A simple redirection |\childdocforward{|\textit{dest}|}| is achieved by:
%
\begin{center}
|\def\jobname{|\textit{dest}|}\input{\jobname}|
\end{center}
%
The redirection with prefix
|\childdocforwardprefix[|\textit{prefix}|]{|\textit{dest}|}|
is accomplished by:
%
\begin{center}
\begin{tabular}{l}
|{\edef\jobname{\scantokens\expandafter{\jobname\noexpand}}|\\
|\def\redirectjob |\textit{prefix}|#1~~~{\gdef\jobname{|\textit{dest}|#1}}|\\
|\expandafter\redirectjob\jobname~~~}\input{\jobname}|
\end{tabular}
\end{center}

In an alternative approach,
child documents can be compiled by a specific command line
without additional code or specific definitions:
%
\begin{center}
|... -jobname "|\textit{target}|" "|[\textit{flags}]%
|\includeonly{|\textit{dest}|}\input{|\textit{main}|}"|
\end{center}
%

%%%%%%%%%%%%%%%%%%%%%%%%%%%%%%%%%%%%%%%%%%%%%%%%%%%%%%%%%%%%%%%%%%%%%%%%%%%%%%%%
%%%%%%%%%%%%%%%%%%%%%%%%%%%%%%%%%%%%%%%%%%%%%%%%%%%%%%%%%%%%%%%%%%%%%%%%%%%%%%%%
\section{Information}

%%%%%%%%%%%%%%%%%%%%%%%%%%%%%%%%%%%%%%%%%%%%%%%%%%%%%%%%%%%%%%%%%%%%%%%%%%%%%%%%
\subsection{Copyright}

Copyright \copyright{} 2017--2018 Niklas Beisert

This work may be distributed and/or modified under the
conditions of the \LaTeX{} Project Public License, either version 1.3
of this license or (at your option) any later version.
The latest version of this license is in
  \url{http://www.latex-project.org/lppl.txt}
and version 1.3 or later is part of all distributions of \LaTeX{}
version 2005/12/01 or later.

This work has the LPPL maintenance status `maintained'.

The Current Maintainer of this work is Niklas Beisert.

This work consists of the files |README.txt|, |childdoc.ins| and |childdoc.dtx|
as well as the derived files |childdoc.def|, |cdocsamp.tex|
with |cdocsch1.tex|, |cdocsch2.tex|, |cdocspt3.tex|, |cdocspt4.tex|,
|cdocsdrf.tex|, |cdocsfn1.tex|, |cdocsfn2.tex|
as well as |childdoc.pdf|.

%%%%%%%%%%%%%%%%%%%%%%%%%%%%%%%%%%%%%%%%%%%%%%%%%%%%%%%%%%%%%%%%%%%%%%%%%%%%%%%%
\subsection{Files and Installation}

The package consists of the files:
%
\begin{center}
\begin{tabular}{ll}
    |README.txt|   & readme file \\
    |childdoc.ins| & installation file \\
    |childdoc.dtx| & source file \\
    |childdoc.def| & definition file \\
    |cdocsamp.tex| & sample main file \\
    |cdocsch1.tex| & sample include file \\
    |cdocsch2.tex| & sample include file \\
    |cdocspt3.tex| & sample part file \\
    |cdocspt4.tex| & sample part file \\
    |cdocsdrf.tex| & sample redirection file \\
    |cdocsfn1.tex| & sample redirection file \\
    |cdocsfn2.tex| & sample redirection file \\
    |childdoc.pdf| & manual
\end{tabular}
\end{center}
%
The distribution consists of the files
|README.txt|, |childdoc.ins| and |childdoc.dtx|.
%
\begin{itemize}
\item
Run (pdf)\LaTeX{} on |childdoc.dtx|
to compile the manual |childdoc.pdf| (this file).
\item
Run \LaTeX{} on |childdoc.ins| to create the definitions file |childdoc.def|
and the sample |cdocsamp.tex| with include files
|cdocsch1.tex|, |cdocsch2.tex|, |cdocspt3.tex|, |cdocspt4.tex|,
|cdocsdrf.tex|, |cdocsfn1.tex|, |cdocsfn2.tex|.
Then copy the file |childdoc.def| to an appropriate directory of your \LaTeX{}
distribution, e.g.\ \textit{texmf-root}|/tex/latex/childdoc|.
\end{itemize}

%%%%%%%%%%%%%%%%%%%%%%%%%%%%%%%%%%%%%%%%%%%%%%%%%%%%%%%%%%%%%%%%%%%%%%%%%%%%%%%%
\subsection{Related CTAN Packages}

There are several other packages which offer a similar functionality:
%
\begin{itemize}
\item
The packages
\href{http://ctan.org/pkg/docmute}{\textsf{docmute}},
\href{http://ctan.org/pkg/includex}{\textsf{includex}} and
\href{http://ctan.org/pkg/standalone}{\textsf{standalone}}
provide commands to include only the document body of
a child file thus allowing both files to be compiled individually.
\item
The packages \href{http://ctan.org/pkg/subdocs}{\textsf{subdocs}}
and \href{http://ctan.org/pkg/subfiles}{\textsf{subfiles}}
provide structures in which the main and child documents can be
encapsulated and allowing them to be compiled individually.
The inclusion mechanism is different from the conventional |\include|.
\item
The package \href{http://ctan.org/pkg/combine}{\textsf{combine}}
is an elaborate solution to combine several documents into one.
\end{itemize}
%
See also the CTAN topic \href{http://ctan.org/topic/subdocs}{\textsf{subdocs}}
for further related packages.
The present package differs from the above solutions in that
a document structure constructed with the conventional |\include| mechanism
just needs two extra commands at the top of every file
such that all constituent files can be compiled individually.

%%%%%%%%%%%%%%%%%%%%%%%%%%%%%%%%%%%%%%%%%%%%%%%%%%%%%%%%%%%%%%%%%%%%%%%%%%%%%%%%
%\subsection{Feature Suggestions}
%
%The following is a list of features which may be useful for future
%versions of this package:
%%
%\begin{itemize}
%\item
%\ldots
%\end{itemize}

%%%%%%%%%%%%%%%%%%%%%%%%%%%%%%%%%%%%%%%%%%%%%%%%%%%%%%%%%%%%%%%%%%%%%%%%%%%%%%%%
\subsection{Revision History}

%%%%%%%%%%%%%%%%%%%%%%%%%%%%%%%%%%%%%%%%
\paragraph{v2.0:} 2018/12/30

\begin{itemize}
\item
immediate forward processing
\item
added |\childdocby| mechanism
\item
manual restructured
\end{itemize}

%%%%%%%%%%%%%%%%%%%%%%%%%%%%%%%%%%%%%%%%
\paragraph{v1.6:} 2018/01/17

\begin{itemize}
\item
application for development of include files
\item
corrections to manual
\end{itemize}

%%%%%%%%%%%%%%%%%%%%%%%%%%%%%%%%%%%%%%%%
\paragraph{v1.5:} 2017/05/21

\begin{itemize}
\item
more complete structuring introduced
\item
|\childdocof| introduced
\item
|\childdoc| renamed to |\childdocmain|
\item
|\childredirect| renamed to |\childdocforward| and |\childdocforwardprefix|
and functionality expanded
\end{itemize}

%%%%%%%%%%%%%%%%%%%%%%%%%%%%%%%%%%%%%%%%
\paragraph{v1.0:} 2017/04/27

\begin{itemize}
\item
manual and install package
\item
first version published on CTAN
\end{itemize}

%%%%%%%%%%%%%%%%%%%%%%%%%%%%%%%%%%%%%%%%
\paragraph{v0.6:} 2017/04/26

\begin{itemize}
\item
redirection mechanism added
\end{itemize}

%%%%%%%%%%%%%%%%%%%%%%%%%%%%%%%%%%%%%%%%
\paragraph{v0.5:} 2017/04/26

\begin{itemize}
\item
functionality in definition file
\end{itemize}


%%%%%%%%%%%%%%%%%%%%%%%%%%%%%%%%%%%%%%%%%%%%%%%%%%%%%%%%%%%%%%%%%%%%%%%%%%%%%%%%
%%%%%%%%%%%%%%%%%%%%%%%%%%%%%%%%%%%%%%%%%%%%%%%%%%%%%%%%%%%%%%%%%%%%%%%%%%%%%%%%
%%%%%%%%%%%%%%%%%%%%%%%%%%%%%%%%%%%%%%%%%%%%%%%%%%%%%%%%%%%%%%%%%%%%%%%%%%%%%%%%
\appendix

\settowidth\MacroIndent{\rmfamily\scriptsize 000\ }

 \DocInput{childdoc.dtx}

\end{document}
%</driver>
% \fi
%
% %%%%%%%%%%%%%%%%%%%%%%%%%%%%%%%%%%%%%%%%%%%%%%%%%%%%%%%%%%%%%%%%%%%%%%%%%%%%%%
% %%%%%%%%%%%%%%%%%%%%%%%%%%%%%%%%%%%%%%%%%%%%%%%%%%%%%%%%%%%%%%%%%%%%%%%%%%%%%%
% \section{Sample}
%\iffalse
%<*samplemain>
%\fi
%
% The following presents a sample document
% with two chapters, two parts, a title page,
% a compile flag as well as three forwarding files to set the flag.
% It consists of eight |.tex| files:
% \begin{center}
% \begin{tabular}{ll}
% |cdocsamp.tex|&main file\\
% |cdocsch1.tex|&include file for chapter 1\\
% |cdocsch2.tex|&include file for chapter 2\\
% |cdocspt3.tex|&include file for part 3\\
% |cdocspt4.tex|&include file for part 4\\
% |cdocsdrf.tex|&forwarding file for main file in draft mode\\
% |cdocsfi1.tex|&forwarding file for final version of chapter 1\\
% |cdocsfi2.tex|&forwarding file for final version of chapter 2\\
% \end{tabular}
% \end{center}
% Each of the eight files can be compiled directly by the \LaTeX{} compiler.
%
% %%%%%%%%%%%%%%%%%%%%%%%%%%%%%%%%%%%%%%
% \paragraph{Main File.}
%
% The main file is called |cdocsamp.tex|.
%
% Load the \textsf{childdoc} definitions and
% declare the filename for the main document:
%    \begin{macrocode}
\input{childdoc.def}
\childdocmain{}
%    \end{macrocode}

% Optional override for |\version| flag:
%    \begin{macrocode}
%%\ifchilddoc\else\providecommand{\version}{draft}\fi
%    \end{macrocode}

% Define the default values for the |\version| flag
% (|final| for the main file and |draft| for childs):
%    \begin{macrocode}
\ifchilddoc
\providecommand{\version}{draft}
\else
\providecommand{\version}{final}
\fi
%    \end{macrocode}

% Load the standard document class:
%    \begin{macrocode}
\documentclass[12pt]{article}
%    \end{macrocode}

% Start the document body:
%    \begin{macrocode}
\begin{document}
%    \end{macrocode}

% Declare a title page.
% Print title, part of document being processed and version flag:
%    \begin{macrocode}
\addtocounter{page}{-1}
\begin{center}
{\LARGE\bfseries{}childdoc example\par}
\vspace{1cm}
\ifchilddoc
\ifchilddocmanual part\else chapter\fi:
`\childdocname' of `\childdocjob'\par
\else
main document: `\childdocjob'\par
\fi
version: \version\par
\end{center}
\newpage
%    \end{macrocode}

% Manually include selected file,
% otherwise process as usual:
%    \begin{macrocode}
\ifchilddocmanual
\section*{part `\childdocname'}
\input{\childdocname}
\else
%    \end{macrocode}

% Include the two chapters:
%    \begin{macrocode}
\include{cdocsch1}
\include{cdocsch2}
%    \end{macrocode}

% Include the two parts unless only chapters should be displayed:
%    \begin{macrocode}
\ifchilddoc\else
\section{part three}
\input{cdocspt3}
\section{part four}
\input{cdocspt4}
\fi
%    \end{macrocode}

% Process as usual until here:
%    \begin{macrocode}
\fi
%    \end{macrocode}

% End of document body:
%    \begin{macrocode}
\end{document}
%    \end{macrocode}
%\iffalse
%</samplemain>
%\fi
%
% %%%%%%%%%%%%%%%%%%%%%%%%%%%%%%%%%%%%%%
% \paragraph{Chapter Include Files.}
%
% The include files are called |cdocsch1.tex| and |cdocsch2.tex|.
%
%\iffalse
%<*samplechap1|samplechap2>
%\fi

% Optional override for |\version| flag:
%    \begin{macrocode}
%%\providecommand{\version}{final}
%    \end{macrocode}

% Include the main document:
%    \begin{macrocode}
\input{childdoc.def}
\childdocof{cdocsamp}
%    \end{macrocode}

%\iffalse
%</samplechap1|samplechap2>
%\fi
%
%\iffalse
%<*samplechap1>
%\fi
% Some text for chapter 1:
%    \begin{macrocode}
\section{one}
some text in chapter one
%    \end{macrocode}

%\iffalse
%</samplechap1>
%\fi
% Some text for chapter 2:
%\iffalse
%<*samplechap2>
%\fi
%    \begin{macrocode}
\section{two}
more text in chapter two
%    \end{macrocode}

%\iffalse
%</samplechap2>
%\fi
%
% %%%%%%%%%%%%%%%%%%%%%%%%%%%%%%%%%%%%%%
% \paragraph{Part Include Files.}
%
% The include files are called |cdocspt3.tex| and |cdocspt4.tex|.
%
%\iffalse
%<*samplepart3|samplepart4>
%\fi

% Optional override for |\version| flag:
%    \begin{macrocode}
%%\providecommand{\version}{final}
%    \end{macrocode}

% Include the main document:
%    \begin{macrocode}
\input{childdoc.def}
\childdocby{cdocsamp}
%    \end{macrocode}

%\iffalse
%</samplepart3|samplepart4>
%\fi
%
%\iffalse
%<*samplepart3>
%\fi
% Some text for part 3:
%    \begin{macrocode}
some text in part three
%    \end{macrocode}

%\iffalse
%</samplepart3>
%\fi
% Some text for part 4:
%\iffalse
%<*samplepart4>
%\fi
%    \begin{macrocode}
more text in part four
%    \end{macrocode}

%\iffalse
%</samplepart4>
%\fi
%
% %%%%%%%%%%%%%%%%%%%%%%%%%%%%%%%%%%%%%%
% \paragraph{Forwarding for a Complete Draft.}
%
% The following forwarding file |cdocsdrf.tex|
% compiles the main document in draft mode:
%\iffalse
%<*sampledraft>
%\fi
%    \begin{macrocode}
\def\version{draft}
\input{childdoc.def}
\childdocforward{cdocsamp}
%    \end{macrocode}

%\iffalse
%</sampledraft>
%\fi
%
% %%%%%%%%%%%%%%%%%%%%%%%%%%%%%%%%%%%%%%
% \paragraph{Forwarding for Final Version of the Chapters.}
%
% The following forwarding files |cdocsfn1.tex| and |cdocsfn2.tex|
% (with identical content)
% compile the final versions of the child documents
% |cdocsch1.tex| and |cdocsch2.tex|, respectively:
%\iffalse
%<*samplefinal>
%\fi
%    \begin{macrocode}
\def\version{final}
\input{childdoc.def}
\childdocforwardprefix[cdocsamp]{cdocsfn}{cdocsch}
%    \end{macrocode}

%\iffalse
%</samplefinal>
%\fi
%
% %%%%%%%%%%%%%%%%%%%%%%%%%%%%%%%%%%%%%%
% \paragraph{Command Line Processing.}
%
% The following three command lines generate the output files
% |cdocscld|, |cdocscl1| and |cdocscl2|
% which should be identical to
% |cdocsdrf|, |cdocsch1| and |cdocsfn2|, respectively:
% \begin{center}
% \begin{tabular}{l}
% |latex -jobname cdocscld \|\\
% |  "\def\version{draft}\input{childdoc.def}\childdocforward{cdocsamp}"|\\
% |latex -jobname cdocscl1 \|\\
% |  "\input{childdoc.def}\childdocforward[cdocsamp]{cdocsch1}"|\\
% |latex -jobname cdocscl2 \|\\
% |  "\def\version{final}\input{childdoc.def}\childdocforward{cdocsch2}"|
% \end{tabular}
% \end{center}
% Note that the trailing backslash on each first line
% merely continues the input to the second line
% (for convenient cut ant paste).
% Furthermore, the command |latex| can be replaced by any
% of its alternative versions such as |pdflatex|.
%
% %%%%%%%%%%%%%%%%%%%%%%%%%%%%%%%%%%%%%%%%%%%%%%%%%%%%%%%%%%%%%%%%%%%%%%%%%%%%%%
% %%%%%%%%%%%%%%%%%%%%%%%%%%%%%%%%%%%%%%%%%%%%%%%%%%%%%%%%%%%%%%%%%%%%%%%%%%%%%%
% \section{Implementation}
%\iffalse
%<*package>
%\fi
%
% This section describes the definitions file |childdoc.def|.

% The definitions cannot be loaded using |\usepackage| or |\RequirePackage|
% which has a mechanism to prevent loading a style file more than once.
% When loading the definitions by means of |\input|
% multiple instances have to be prevented manually:
%\iffalse
%This code needs to be before the `\ProvidesFile' directive
%which is defined at the beginning of this file.
%Therefore it is also placed there and commented out here.
%</package>
%<*discard>
%\fi
%    \begin{macrocode}
\ifdefined\childdocmain\endinput\fi
%    \end{macrocode}
%\iffalse
%</discard>
%<*package>
%\fi
%
% \macro{\ifchilddoc}
% \macro{\ifchilddocmanual}
% The conditional |\ifchilddoc| tells whether a
% child (true) or main (false) document is being compiled.
% The conditional |\ifchilddocmanual| tells whether
% the |\includeonly| mechanism is used (false) or
% the selection of child files must be performed manually (true).
% The definitions initialise to false:
%    \begin{macrocode}
\newif\ifchilddoc
\newif\ifchilddocmanual
%    \end{macrocode}

% \macro{\childdocname}
% \macro{\childdocjob}
% The macro |\childdocname| stores the name of the main document
% to be compiled. The macro |\childdocjob| stores the name of
% the document on which the \LaTeX{} compiler was originally invoked.
% The content of |\jobname| cannot be compared
% to filenames specified in the source due to different catcodes.
% The following code rescans |\jobname|, stores the result
% in |\childdocname| and saves a copy in |\childdocjob|:
%    \begin{macrocode}
\edef\childdocname{\scantokens\expandafter{\jobname\noexpand}}
\let\childdocjob\childdocname
%    \end{macrocode}

% \macro{\childdocdisable}
% The macro |\childdocdisable| prevents the main file
% from being processed more than once.
% At this stage, the main document command |\childdocmain|
% is assumed to be called once again where it should do nothing.
% Any subsequent call to it should prevent
% a secondary processing of the main document
% It overwrites the forwarding commands
% |\childdocof| and |\childdocforward|
% with empty macros to prevent further inclusions of the main document:
%    \begin{macrocode}
\newcommand{\childdocdisable}
{
  \renewcommand{\childdocmain}[1]{\renewcommand{\childdocmain}[1]{\endinput}}
  \renewcommand{\childdocof}[1]{}
  \renewcommand{\childdocby}[2][]{}
  \renewcommand{\childdocforward}[2][]{}
  \renewcommand{\childdocdisable}{}
}
%    \end{macrocode}

% \macro{\childdocmain}
% The macro |\childdocmain| is to be called at the top of the main file
% with nothing or the main filename (without extension) as argument.
% First, it breaks loops.
% If the argument is not empty and does not match |\childdocname|
% (which is set by the first inclusion of |childdoc.def|),
% |\ifchilddoc| is set to true, |\includeonly| is applied to the child file
% and |\jobname| is set to the main file
% (for proper handling of |.aux| files):
%    \begin{macrocode}
\newcommand{\childdocmain}[1]
{
  \childdocdisable\childdocmain{}
  \if?#1?\else
    \begingroup
      \def\childdoctmp{#1}
      \ifx\childdoctmp\childdocname
        \def\childdoctmp{}
      \else
        \def\childdoctmp
        {
          \childdoctrue
          \includeonly{\childdocname}
          \def\childdocjob{#1}
          \def\jobname{#1}
        }
      \fi
      \expandafter
    \endgroup
    \childdoctmp
  \fi
}
%    \end{macrocode}

% \macro{\childdocof}
% The command |\childdocof| redirects
% compilation to the main file |#1|.
%    \begin{macrocode}
\newcommand{\childdocof}[1]
{
  \childdocdisable
  \childdoctrue
  \includeonly{\childdocname}
  \def\jobname{#1}
  \def\childdocjob{#1}
  \input{#1}
}
%    \end{macrocode}

% \macro{\childdocby}
% The command |\childdocby| ....
%    \begin{macrocode}
\newcommand{\childdocby}[2][]
{
  \childdocdisable
  \childdoctrue
  \childdocmanualtrue
  \if?#1?\else
    \def\jobname{#2}
  \fi
  \def\childdocjob{#2}
  \input{#2}
  \endinput
}
%    \end{macrocode}

% \macro{\childdocforward}
% The command |\childdocforward| redirects
% compilation to the main file or
% (if the optional argument is given) a child file.
% Parameters are set as if the main file
% or a child file starting with |\childdocof| was compiled.
% Then compilation is handed over to the main file:
%    \begin{macrocode}
\newcommand{\childdocforward}[2][]
{
  \begingroup
    \if?#1?
      \def\childdoctmp
      {
        \def\childdocname{#2}
        \def\childdocjob{#2}
        \def\jobname{#2}
        \input{#2}
        \endinput
      }
    \else
      \def\childdoctmp
      {
        \childdocdisable
        \def\childdocname{#2}
        \childdoctrue
        \includeonly{#2}
        \def\childdocjob{#1}
        \def\jobname{#1}
        \input{#1}
        \endinput
      }
    \fi
    \expandafter
  \endgroup
  \childdoctmp
}
%    \end{macrocode}

% \macro{\childdocforwardprefix}
% The command |\childdocforwardprefix| redirects
% compilation to the main or a child file by means of a pattern.
% The prefix |#1| in the current filename is replaced by |#2|
% and the suffix of the current filename is kept
% (it is assumed that the filename does not contain the substring `|~~~|'
% which is used as a delimiter).
% Compilation is handed over to the new file by |\childdocforward|:
%    \begin{macrocode}
\newcommand{\childdocforwardprefix}[3][]
{
  \begingroup
    \def\childdocextract #2##1~~~{\def\childdoctmp{\childdocforward[#1]{#3##1}}}
    \expandafter\childdocextract\childdocname~~~
    \expandafter
  \endgroup
  \childdoctmp
}
%    \end{macrocode}

% \macro{\childdoc}
% The deprecated macro |\childdoc| is a legacy version of |\childdocmain|:
%    \begin{macrocode}
\newcommand{\childdoc}{\childdocmain}
%    \end{macrocode}

% \macro{\childdocredirect}
% The deprecated macro |\childdocredirect| is a legacy version
% of |\childdocforward| and |\childdocforwardprefix|:
%    \begin{macrocode}
\newcommand{\childdocredirect}[2][]
{
  \begingroup
    \if?#1?
      \def\childdoctmp{\childdocforward{#2}}
    \else
      \def\childdoctmp{\childdocforwardprefix{#1}{#2}}
    \fi
    \expandafter
  \endgroup
  \childdoctmp
}
%    \end{macrocode}

%\iffalse
%</package>
%\fi
%
\endinput
\childdocforward{cdocsamp}"|\\
% |latex -jobname cdocscl1 \|\\
% |  "% \iffalse
%
% childdoc.dtx Copyright (C) 2017-2018 Niklas Beisert
%
% This work may be distributed and/or modified under the
% conditions of the LaTeX Project Public License, either version 1.3
% of this license or (at your option) any later version.
% The latest version of this license is in
%   http://www.latex-project.org/lppl.txt
% and version 1.3 or later is part of all distributions of LaTeX
% version 2005/12/01 or later.
%
% This work has the LPPL maintenance status `maintained'.
%
% The Current Maintainer of this work is Niklas Beisert.
%
% This work consists of the files childdoc.dtx and childdoc.ins
% and the derived files childdoc.def and cdocsamp.tex with
% cdocsch1.tex, cdocsch2.tex, cdocsdrf.tex, cdocsfn1.tex, cdocsfn2.tex.
%
%<package>\ifdefined\childdocmain\endinput\fi
%<package>\ProvidesFile{childdoc.def}[2018/12/30 v2.0 child document driver]
%<samplemain>\ProvidesFile{cdocsamp.tex}[2018/12/30 v2.0 sample for childdoc]
%<*driver>
%\ProvidesFile{childdoc.drv}[2018/12/30 v2.0 childdoc reference manual file]
\PassOptionsToClass{10pt,a4paper}{article}
\documentclass{ltxdoc}

\usepackage[margin=35mm]{geometry}
\usepackage{hyperref}
\usepackage{hyperxmp}
\usepackage[usenames]{color}

\hypersetup{colorlinks=true}
\hypersetup{pdfstartview=FitH}
\hypersetup{pdfpagemode=UseNone}
\hypersetup{pdfsource={}}
\hypersetup{pdflang={en-UK}}
\hypersetup{pdfcopyright={Copyright 2017-2018 Niklas Beisert.
  This work may be distributed and/or modified under the
  conditions of the LaTeX Project Public License, either version 1.3
  of this license or (at your option) any later version.}}
\hypersetup{pdflicenseurl={http://www.latex-project.org/lppl.txt}}
\hypersetup{pdfcontactaddress={ETH Zurich, ITP, HIT K,
  Wolfgang-Pauli-Strasse 27}}
\hypersetup{pdfcontactpostcode={8093}}
\hypersetup{pdfcontactcity={Zurich}}
\hypersetup{pdfcontactcountry={Switzerland}}
\hypersetup{pdfcontactemail={nbeisert@itp.phys.ethz.ch}}
\hypersetup{pdfcontacturl={http://people.phys.ethz.ch/\xmptilde nbeisert/}}

\newcommand{\secref}[1]{\hyperref[#1]{section \ref*{#1}}}

\parskip1ex
\parindent0pt
\let\olditemize\itemize
\def\itemize{\olditemize\parskip0pt}

\begin{document}

\title{The \textsf{childdoc} Package}
\hypersetup{pdftitle={The childdoc Package}}
\author{Niklas Beisert\\[2ex]
  Institut f\"ur Theoretische Physik\\
  Eidgen\"ossische Technische Hochschule Z\"urich\\
  Wolfgang-Pauli-Strasse 27, 8093 Z\"urich, Switzerland\\[1ex]
  \href{mailto:nbeisert@itp.phys.ethz.ch}
  {\texttt{nbeisert@itp.phys.ethz.ch}}}
\hypersetup{pdfauthor={Niklas Beisert}}
\hypersetup{pdfsubject={Manual for the LaTeX2e Package childdoc}}
\date{30 December 2018, \textsf{v2.0}}
\maketitle

\begin{abstract}\noindent
\textsf{childdoc} is a \LaTeXe{} package
that enables the direct compilation
of document sections included by |\include|
to individual files.
\end{abstract}

\begingroup
\parskip0ex
\tableofcontents
\endgroup

%%%%%%%%%%%%%%%%%%%%%%%%%%%%%%%%%%%%%%%%%%%%%%%%%%%%%%%%%%%%%%%%%%%%%%%%%%%%%%%%
%%%%%%%%%%%%%%%%%%%%%%%%%%%%%%%%%%%%%%%%%%%%%%%%%%%%%%%%%%%%%%%%%%%%%%%%%%%%%%%%
\section{Introduction}

\LaTeX{} provides a mechanism to structure a large document (such as a book)
into a main file and several child files (containing the chapters)
using the |\include| command.
This mechanism is beneficial for documents
which span hundreds of pages in order to
make the source file(s) more manageable.
Moreover, compilation can be restricted to
selected child files by means of the |\includeonly| command.
The latter feature can be used to reduce the compilation time while editing
(this was significantly more useful in the earlier days of \LaTeX{})
or to generate a smaller document which is easier to navigate.
Another application of |\includeonly| is to generate
documents consisting of selected parts of the complete document.

However, there are a few drawbacks of the plain |\include| mechanism:
\begin{itemize}
\item
The child files cannot be compiled on their own,
they can only be compiled via the main file.
A naive editing environment
(such as a text editor with an option
to have the current file processed by \LaTeX)
may require one to switch to the main file before compiling;
attempting to compile the child file produces errors.
\item
The main file must be modified (each time)
to adjust the |\includeonly| command
to the present needs. This easily leaves the main file in a messy state.
\item
The generated document will always carry the filename
of the main document. This is inconvenient if
several child files are to be compiled and
to be kept for distribution.
\end{itemize}

The present package provides a simple interface
to make child files individually compilable by \LaTeX{}.
Compiling a child file then has the same effect as compiling
the main file with an |\includeonly| command
to select the appropriate child.
Moreover the generated document will carry the name of the child
rather than the main file.
This resolves all three above issues.

This feature is meant to make the editing of books,
thesis documents and lecture notes somewhat more convenient.
However, the package can also be used efficiently for
composing a series of documents (such as exercise sheets)
which are typically distributed individually.
It then assists the author in generating the individual documents
(potentially in different versions)
as well as a document containing the collected series.
Another application is in developing style files
or other kinds of included material
where compilation of the style file could redirect
to a sample or test file.

%%%%%%%%%%%%%%%%%%%%%%%%%%%%%%%%%%%%%%%%%%%%%%%%%%%%%%%%%%%%%%%%%%%%%%%%%%%%%%%%
%%%%%%%%%%%%%%%%%%%%%%%%%%%%%%%%%%%%%%%%%%%%%%%%%%%%%%%%%%%%%%%%%%%%%%%%%%%%%%%%
\section{Usage}

First of all, the package \textsf{childdoc} is \emph{not} a standard
\LaTeXe{} |.sty| style file! Therefore it needs to be invoked in
a non-standard way.

%%%%%%%%%%%%%%%%%%%%%%%%%%%%%%%%%%%%%%%%%%%%%%%%%%%%%%%%%%%%%%%%%%%%%%%%%%%%%%%%
\subsection{Included Files}
\label{sec:include}

%%%%%%%%%%%%%%%%%%%%%%%%%%%%%%%%%%%%%%%%
\DescribeMacro{\childdocmain}
To use the package, add the commands
\begin{center}
\begin{tabular}{l}
|\input{childdoc.def}|\\
|\childdocmain{}|\\
\end{tabular}
\end{center}
at the very top of the main \LaTeX{} file,
in particular \emph{before} the |\documentclass| statement!
The argument of |\childdocmain| should be left empty
(but it must be present).

%%%%%%%%%%%%%%%%%%%%%%%%%%%%%%%%%%%%%%%%
\DescribeMacro{\childdocof}
Furthermore, add the commands
\begin{center}
\begin{tabular}{l}
|\input{childdoc.def}|\\
|\childdocof{|\textit{main}|}|\\
\end{tabular}
\end{center}
at the top of every child file \textit{child}
which is included by |\include{|\textit{child}|}|
from within the main file
(or at least for those files to be compiled individually).
The argument \textit{main} must be the filename of the main file.

There are a couple of
considerations in setting up the main and child documents:

%%%%%%%%%%%%%%%%%%%%%%%%%%%%%%%%%%%%%%%%
\paragraph{Restrictions.}

Please note the following restrictions:
\begin{itemize}
\item
|\childdocmain| must be called with one argument \textit{main}
to ensure compatibility with earlier version of the package.
It must either be empty (|\childdocmain{}|)
or precisely match the filename of the main file in which it is specified.
See \secref{sec:detection} for further information.
\item
The filename \textit{main} must be specified without the |.tex| extension.
\item
The filename \textit{main} is case sensitive
(even in case-insensitive file systems)
due to internal string comparison.
\item
The argument \textit{main} should be fully expanded, it cannot be a macro.
\item
Subdirectories and special characters should be avoided in filenames.
\item
The command |\childdocmain{|\textit{main}|}| must be followed by a whitespace.
It should not be followed immediately by another command
or by a comment mark `|%|'.
This is because the \TeX{} parser reads the token immediately following
the argument of |\childdocmain| and puts it
at the beginning of every child section;
however, a white\-space is ignored.
\end{itemize}

%%%%%%%%%%%%%%%%%%%%%%%%%%%%%%%%%%%%%%%%
\paragraph{Content of Main File.}

It is advisable to place all content in the child files included by |\include|.
Any output contained in the main file will appear in all child documents
unless suppressed manually;
it cannot be suppressed automatically by the |\includeonly| directive
and thus should normally be avoided.
A method to include some content in the main file
by means of conditional processing is described in \secref{sec:conditional}.

%%%%%%%%%%%%%%%%%%%%%%%%%%%%%%%%%%%%%%%%
\paragraph{Page Numbering.}

When only a part of the document is compiled,
the appropriate numbering of pages
(as well as other status parameters)
is determined from the |.aux| files.
The latter contain information from previous passes.
However this information needs to propagate through
all intermediate child documents.
Therefore the page numbering in child documents may well
be inconsistent until the complete document is compiled at least once.

A useful (if unconventional) way to always ensure a consistent
page numbering is to restart the numbering in each child document
and denote the pages by `\textit{child}|.|\textit{page}'
where \textit{child} represents the chapter/section number of the child file.
This can be achieved by the command
|\numberwithin{page}{|\textit{child}|}|
of the \textsf{amsmath} package
where \textit{child} can be |chapter| or |section|
depending on the chosen structuring.
Alternatively, one can modify the macro |\thepage| appropriately
and reset the counter |page| at the start of each child file.

%%%%%%%%%%%%%%%%%%%%%%%%%%%%%%%%%%%%%%%%%%%%%%%%%%%%%%%%%%%%%%%%%%%%%%%%%%%%%%%%
\subsection{Conditional Processing}
\label{sec:conditional}

The package provides a mechanism to compile different versions
of a document. To customise the versions further some conditional processing
can come in handy to distinguish which version is being compiled.
The package provides two macros to describe the compilation context:

%%%%%%%%%%%%%%%%%%%%%%%%%%%%%%%%%%%%%%%%
\DescribeMacro{\ifchilddoc}
The conditional |\ifchilddoc| distinguishes between the compilation of
child documents and the main document:
%
\begin{center}
|\ifchilddoc |\textit{child-code}| |[|\||else |\textit{main-code}]| \||fi|
\end{center}

%%%%%%%%%%%%%%%%%%%%%%%%%%%%%%%%%%%%%%%%
\DescribeMacro{\childdocname}
\DescribeMacro{\childdocjob}
The macro |\childdocname| contains the filename (without extension)
of the main or child file being processed.
Note that |\childdocjob| will always contain the name of the main file.

%%%%%%%%%%%%%%%%%%%%%%%%%%%%%%%%%%%%%%%%
\paragraph{Title Page.}

Conditional processing can be used to include a title or banner page
in the main document when proper precautions are taken.
Importantly, the code in the main file should ensure that the page counter
(as well as other status parameters which are stored in the |.aux| files)
takes the same value after the conditional processing.
Otherwise the page numbers may take divergent values
depending on which part is compiled.

For example, a title page could be declared by:
%
\begin{center}
\begin{tabular}{l}
|\ifchilddoc\||else|\\
|\addtocounter{page}{-1}|\\
\textit{code for title page}\\
|\newpage|\\
|\||fi|
\end{tabular}
\end{center}
%
A banner page for the child documents can be generated by:
%
\begin{center}
\begin{tabular}{l}
|\ifchilddoc|\\
|\addtocounter{page}{-1}|\\
\textit{code for banner page}\\
|\newpage|\\
|\||fi|
\end{tabular}
\end{center}
%
Here one could write a message such as:
\begin{center}
|This is the part \childdocname{} of \childdocjob{}.|
\end{center}

%%%%%%%%%%%%%%%%%%%%%%%%%%%%%%%%%%%%%%%%%%%%%%%%%%%%%%%%%%%%%%%%%%%%%%%%%%%%%%%%
\subsection{Flags}
\label{sec:flags}

The package makes it easy to generate different versions
of the main or child documents.
To this end compilation flags can be defined
and assigned different default values.
They will be particularly useful in conjunction
with the forwarding mechanism described in \secref{sec:forward}.

For example, it may be useful to have a flag |\version|
which can be set to |draft| or |final|.
The document source will contain some conditional code
depending on the value of |\version|.
Suppose further, the flag should default to |final| for the main file
and to |draft| for child files
which is a natural assignment for editing the document.
This is achieved by placing the following code
in the preamble of the main document
(below the |\childdocmain| directive):
%
\begin{center}
\begin{tabular}{l}
|\ifchilddoc|\\
|\providecommand{\version}{draft}|\\
|\||else|\\
|\providecommand{\version}{final}|\\
|\||fi|
\end{tabular}
\end{center}
%
The definition by |\providecommand| makes sure
that previous definitions are not overwritten.
Further statements |\providecommand{\version}{...}|
can thus be added before the above code to override it.

For the main file, one might add a line
(between |\childdocmain| and the above block)
%
\begin{center}
|%\ifchilddoc\||else\providecommand{\version}{draft}\||fi|
\end{center}
%
which can be uncommented to produce a draft version.
Likewise one can add a line to the very top of a child file
(above the |\childdocof{|\textit{main}|}| directive)
%
\begin{center}
|%\providecommand{\version}{final}|
\end{center}
%
which can be uncommented to produce the final version of this child document.

%%%%%%%%%%%%%%%%%%%%%%%%%%%%%%%%%%%%%%%%%%%%%%%%%%%%%%%%%%%%%%%%%%%%%%%%%%%%%%%%
\subsection{Forwarding}
\label{sec:forward}

Different versions of the main or child documents
using compilation flags as described in \secref{sec:flags}
can be (permanently) stored in different files
for convenient compilation, viewing and distribution.
To this end, the package defines a command
to pass on compilation to a different file:

%%%%%%%%%%%%%%%%%%%%%%%%%%%%%%%%%%%%%%%%
\DescribeMacro{\childdocforward}
The command |\childdocforward| redirects processing to
another source file:
%
\begin{center}
\begin{tabular}{l}
|\input{childdoc.def}|\\
|\childdocforward[|\textit{main}|]{|\textit{dest}|}|\\
\end{tabular}
\end{center}
%
The argument \textit{dest} is the destination file
(without extension).
It should be the main file or one of the child files.
Note that further \textsf{childdoc} directives
such as |\childdocof| and |\childdocforward|
in the indicated file will be processed in this form.
The optional argument \textit{main}
passes on directly to the main file \textit{main}
while pretending to compile the child \textit{dest}.
This form behaves as if \textit{dest}
issues |\childdocof{|\textit{main}|}| right away,
and no further \textsf{childdoc} directives will be processed.

%%%%%%%%%%%%%%%%%%%%%%%%%%%%%%%%%%%%%%%%
\DescribeMacro{\...prefix}
In the alternative form |\childdocforwardprefix|,
%
\begin{center}
\begin{tabular}{l}
|\input{childdoc.def}|\\
|\childdocforwardprefix[|\textit{main}|]{|\textit{prefix}|}{|\textit{dest}|}|
\end{tabular}
\end{center}
%
the destination file is determined by a pattern
depending on the current file:
To make this work, the current file must be called
`{\textit{prefix}\hspace{0.2em}\textit{suffix}}'
with \textit{prefix} matching precisely the argument.
Processing is then passed on to the file
`{\textit{dest}\hspace{0.2em}\textit{suffix}}'.
Surely, the same effect is achieved by
directly specifying the
argument `{\textit{dest}\hspace{0.2em}\textit{suffix}}'
in the first form.
However, that requires to set up a different file
for each child. With the alternative form of the command
all these files can have exactly the same content
which simplifies setting them up and maintaining them.

For example, the following file |draft.tex|
with a compilation flag |\version| as described in \secref{sec:flags}
compiles the main document as a draft:
%
\begin{center}
\begin{tabular}{l}
|\def\version{draft}|\\
|\input{childdoc.def}|\\
|\childdocforward{|\textit{main}|}|
\end{tabular}
\end{center}
%
Likewise, the following files |final|\textit{nn}|.tex|
compile the final version of the child document
|child|\textit{nn}|.tex|:
%
\begin{center}
\begin{tabular}{l}
|\def\version{final}|\\
|\input{childdoc.def}|\\
|\childdocforwardprefix{final}{child}|
\end{tabular}
\end{center}
%

Note that when several versions of a main file and/or of each child file
are to be generated, it may be convenient to set up a |Makefile| or
shell script to automatise the process.

%%%%%%%%%%%%%%%%%%%%%%%%%%%%%%%%%%%%%%%%%%%%%%%%%%%%%%%%%%%%%%%%%%%%%%%%%%%%%%%%
\subsection{Command Line Processing}
\label{sec:commandline}

The effect of redirection files can also be achieved by invoking
the \LaTeX{} compiler with a more elaborate command line.
Most conveniently this should be done as part
of a shell script or a |Makefile|.

When using \textsf{childdoc} in the main file, the following
command lines effectively perform a redirection
(note that depending on the shell being used,
backslashes may have to be doubled: `|\|' $\to$ `|\\|'):
%
\begin{center}
|... -jobname "|\textit{target}|" |\\|"|[\textit{flags}]%
|\input{childdoc.def}\childdocforward[|\textit{main}|]{|\textit{dest}|}"|
\end{center}
%
Here \textit{target} is the name of the output file,
\textit{main} is the name of the main file
and \textit{dest} is the name of the main or child file to be processed
(all filenames without extensions).
The optional argument \textit{main} can be omitted
if \textit{main} matches \textit{dest}.
Optionally, compilation \textit{flags} can be defined via |\def| commands.
This command line makes the \TeX{} engine believe
it is compiling the file \textit{target}
whose content is specified as the latter parameter.
The provided code then forwards the processing to
\textit{main} or \textit{dest} as described in \secref{sec:forward}.

%%%%%%%%%%%%%%%%%%%%%%%%%%%%%%%%%%%%%%%%%%%%%%%%%%%%%%%%%%%%%%%%%%%%%%%%%%%%%%%%
\subsection{Include by Input}
\label{sec:input}

Including child documents by |\include| has some restrictions by design.
Most notably, the content of a child document always occupies
its own set of pages; pages cannot be shared between child documents.
Usually, this behaviour makes perfect sense
because each child document contain an essential part of the document.
However, in some situations it may be desirable to compose
a document from a collection of parts
without having mandatory page breaks between then.
For this case, the package
provides a mechanism to include parts
by |\input| which can also be processed individually.
However, by construction this mechanism
requires manual handling of the content to be output.

%%%%%%%%%%%%%%%%%%%%%%%%%%%%%%%%%%%%%%%%
\DescribeMacro{\ifchilddocmanual}
The main file should be prepared as usual, see \secref{sec:include}.
However, the document body must make a distinction
between processing of an individual part and of the main document, e.g.:
%
\begin{center}
\begin{tabular}{l}
|\ifchilddocmanual|\\
|\input{\childdocname}|\\
|\||else|\\
\textit{document body with }|\input{|\textit{part}|}|\\
|\||fi|
\end{tabular}
\end{center}
%
The conditional |\ifchilddocmanual| is true whenever
a part to be included by |\input| is being compiled,
and the name of the part is stored in |\childdocname|.

%%%%%%%%%%%%%%%%%%%%%%%%%%%%%%%%%%%%%%%%
\DescribeMacro{\childdocby}
Each part to be included by |\input| should start with:
%
\begin{center}
\begin{tabular}{l}
|\input{childdoc.def}|\\
|\childdocby{|\textit{main}|}|\\
\end{tabular}
\end{center}
%
The directive |\childdocby| is similar to |\childdocof|
described in \secref{sec:include},
but the subsequent selection of content must be done manually.
To that end, both |\ifchilddoc| and |\ifchilddocmanual|
will be true upon processing of a part,
and the name of the part is stored in |\childdocname|.
Note that |\jobname| will be set to the filename of the current part
so that each part receives an individual |.aux| file
that does not interfere with the |.aux| file(s) of the main document.
This behaviour can be altered by the alternative form
|\childdocby[*]{|\textit{main}|}| (with a non-empty optional argument)
which uses the |.aux| file of the main document
by setting |\jobname| to \textit{main}.

%%%%%%%%%%%%%%%%%%%%%%%%%%%%%%%%%%%%%%%%%%%%%%%%%%%%%%%%%%%%%%%%%%%%%%%%%%%%%%%%
\subsection{Driver Development}
\label{sec:driver}

The \textsf{childdoc} mechanism can also be use for the development
of definition files such as \LaTeX{} styles or classes.
This case differs from the above setup with multiple parts
included by |\include| in that no |\includeonly| should be invoked.
This can be achieved by starting the include file
(before |\ProvidesPackage|) with:
%
\begin{center}
\begin{tabular}{l}
|\input{childdoc.def}|\\
|\childdocforward{|\textit{main}|}|\\
\end{tabular}
\end{center}
%
or alternatively with:
%
\begin{center}
\begin{tabular}{l}
|\input{childdoc.def}|\\
|\childdocby{|\textit{main}|}|\\
\end{tabular}
\end{center}
%
Both forms have slightly different effects as described above.
The main file is prepared as usual, see \secref{sec:include}.

%%%%%%%%%%%%%%%%%%%%%%%%%%%%%%%%%%%%%%%%%%%%%%%%%%%%%%%%%%%%%%%%%%%%%%%%%%%%%%%%
\subsection{Legacy Detection}
\label{sec:detection}

The directive |\childdocmain| in the main file can detect
whether the complete document or merely a child is to be compiled
even without using the directive |\childdocof|.
This method is deprecated because it is less robust
and there is no compelling reason to use it;
it is merely provided for backward compatibility
and it may be removed in future versions.

If the detection mechanism is to be used,
it is mandatory to correctly specify
the filename of the main file as the argument of |\childdocmain|:
%
\begin{center}
\begin{tabular}{l}
|\input{childdoc.def}|\\
|\childdocmain{|\textit{main}|}|\\
\end{tabular}
\end{center}
%
If |\jobname| does not match the argument \textit{main} of |\childdocmain|,
it is assumed that |\jobname| points to the child file to be compiled.
When using |\childdocmain| with the main file specified as argument,
it suffices to start a child file
with just |\input{|\textit{main}|}|
without loading of the package and using |\childdocof|.
If instead all processing is done
with the appropriate \textsf{childdoc} directives,
the argument of \textit{main} of |\childdocmain| can be empty.

An alternative version of the command line processing described
in \secref{sec:commandline} using the detection mechanism reads:
%
\begin{center}
|... -jobname "|\textit{target}|" "|[\textit{flags}]%
[|\def\jobname{|\textit{dest}|}|]|\input{|\textit{main}|}"|
\end{center}

%%%%%%%%%%%%%%%%%%%%%%%%%%%%%%%%%%%%%%%%%%%%%%%%%%%%%%%%%%%%%%%%%%%%%%%%%%%%%%%%
\subsection{Manual Code}
\label{sec:manual}

In case one cannot be certain whether the definitions file |childdoc.def|
is installed on the target \TeX{} distribution
and one prefers not to ship it,
it is conceivable to paste a few relevant commands into the sources.

To that end, drop all statements |\input{childdoc.def}|
and perform the replacements as outlined below.
Instead of |\childdocmain{|\textit{main}|}| add the following code
to the top of the main file:
%
\begin{center}
\begin{tabular}{l}
|\||ifdefined\childdocname\endinput\||fi\newif\ifchilddoc|\\
|\edef\childdocname{\scantokens\expandafter{\jobname\noexpand}}|\\
|\def\childdocmain{|\textit{main}|}\||ifx\childdocmain\childdocname\||else|\\
|\childdoctrue\includeonly{\childdocname}\let\jobname\childdocmain\||fi|\\
\end{tabular}
\end{center}
%
Instead of |\childdocof{|\textit{main}|}| just include the main file
at the top of each child file:
%
\begin{center}
|\input{|\textit{main}|}|
\end{center}
%
A simple redirection |\childdocforward{|\textit{dest}|}| is achieved by:
%
\begin{center}
|\def\jobname{|\textit{dest}|}\input{\jobname}|
\end{center}
%
The redirection with prefix
|\childdocforwardprefix[|\textit{prefix}|]{|\textit{dest}|}|
is accomplished by:
%
\begin{center}
\begin{tabular}{l}
|{\edef\jobname{\scantokens\expandafter{\jobname\noexpand}}|\\
|\def\redirectjob |\textit{prefix}|#1~~~{\gdef\jobname{|\textit{dest}|#1}}|\\
|\expandafter\redirectjob\jobname~~~}\input{\jobname}|
\end{tabular}
\end{center}

In an alternative approach,
child documents can be compiled by a specific command line
without additional code or specific definitions:
%
\begin{center}
|... -jobname "|\textit{target}|" "|[\textit{flags}]%
|\includeonly{|\textit{dest}|}\input{|\textit{main}|}"|
\end{center}
%

%%%%%%%%%%%%%%%%%%%%%%%%%%%%%%%%%%%%%%%%%%%%%%%%%%%%%%%%%%%%%%%%%%%%%%%%%%%%%%%%
%%%%%%%%%%%%%%%%%%%%%%%%%%%%%%%%%%%%%%%%%%%%%%%%%%%%%%%%%%%%%%%%%%%%%%%%%%%%%%%%
\section{Information}

%%%%%%%%%%%%%%%%%%%%%%%%%%%%%%%%%%%%%%%%%%%%%%%%%%%%%%%%%%%%%%%%%%%%%%%%%%%%%%%%
\subsection{Copyright}

Copyright \copyright{} 2017--2018 Niklas Beisert

This work may be distributed and/or modified under the
conditions of the \LaTeX{} Project Public License, either version 1.3
of this license or (at your option) any later version.
The latest version of this license is in
  \url{http://www.latex-project.org/lppl.txt}
and version 1.3 or later is part of all distributions of \LaTeX{}
version 2005/12/01 or later.

This work has the LPPL maintenance status `maintained'.

The Current Maintainer of this work is Niklas Beisert.

This work consists of the files |README.txt|, |childdoc.ins| and |childdoc.dtx|
as well as the derived files |childdoc.def|, |cdocsamp.tex|
with |cdocsch1.tex|, |cdocsch2.tex|, |cdocspt3.tex|, |cdocspt4.tex|,
|cdocsdrf.tex|, |cdocsfn1.tex|, |cdocsfn2.tex|
as well as |childdoc.pdf|.

%%%%%%%%%%%%%%%%%%%%%%%%%%%%%%%%%%%%%%%%%%%%%%%%%%%%%%%%%%%%%%%%%%%%%%%%%%%%%%%%
\subsection{Files and Installation}

The package consists of the files:
%
\begin{center}
\begin{tabular}{ll}
    |README.txt|   & readme file \\
    |childdoc.ins| & installation file \\
    |childdoc.dtx| & source file \\
    |childdoc.def| & definition file \\
    |cdocsamp.tex| & sample main file \\
    |cdocsch1.tex| & sample include file \\
    |cdocsch2.tex| & sample include file \\
    |cdocspt3.tex| & sample part file \\
    |cdocspt4.tex| & sample part file \\
    |cdocsdrf.tex| & sample redirection file \\
    |cdocsfn1.tex| & sample redirection file \\
    |cdocsfn2.tex| & sample redirection file \\
    |childdoc.pdf| & manual
\end{tabular}
\end{center}
%
The distribution consists of the files
|README.txt|, |childdoc.ins| and |childdoc.dtx|.
%
\begin{itemize}
\item
Run (pdf)\LaTeX{} on |childdoc.dtx|
to compile the manual |childdoc.pdf| (this file).
\item
Run \LaTeX{} on |childdoc.ins| to create the definitions file |childdoc.def|
and the sample |cdocsamp.tex| with include files
|cdocsch1.tex|, |cdocsch2.tex|, |cdocspt3.tex|, |cdocspt4.tex|,
|cdocsdrf.tex|, |cdocsfn1.tex|, |cdocsfn2.tex|.
Then copy the file |childdoc.def| to an appropriate directory of your \LaTeX{}
distribution, e.g.\ \textit{texmf-root}|/tex/latex/childdoc|.
\end{itemize}

%%%%%%%%%%%%%%%%%%%%%%%%%%%%%%%%%%%%%%%%%%%%%%%%%%%%%%%%%%%%%%%%%%%%%%%%%%%%%%%%
\subsection{Related CTAN Packages}

There are several other packages which offer a similar functionality:
%
\begin{itemize}
\item
The packages
\href{http://ctan.org/pkg/docmute}{\textsf{docmute}},
\href{http://ctan.org/pkg/includex}{\textsf{includex}} and
\href{http://ctan.org/pkg/standalone}{\textsf{standalone}}
provide commands to include only the document body of
a child file thus allowing both files to be compiled individually.
\item
The packages \href{http://ctan.org/pkg/subdocs}{\textsf{subdocs}}
and \href{http://ctan.org/pkg/subfiles}{\textsf{subfiles}}
provide structures in which the main and child documents can be
encapsulated and allowing them to be compiled individually.
The inclusion mechanism is different from the conventional |\include|.
\item
The package \href{http://ctan.org/pkg/combine}{\textsf{combine}}
is an elaborate solution to combine several documents into one.
\end{itemize}
%
See also the CTAN topic \href{http://ctan.org/topic/subdocs}{\textsf{subdocs}}
for further related packages.
The present package differs from the above solutions in that
a document structure constructed with the conventional |\include| mechanism
just needs two extra commands at the top of every file
such that all constituent files can be compiled individually.

%%%%%%%%%%%%%%%%%%%%%%%%%%%%%%%%%%%%%%%%%%%%%%%%%%%%%%%%%%%%%%%%%%%%%%%%%%%%%%%%
%\subsection{Feature Suggestions}
%
%The following is a list of features which may be useful for future
%versions of this package:
%%
%\begin{itemize}
%\item
%\ldots
%\end{itemize}

%%%%%%%%%%%%%%%%%%%%%%%%%%%%%%%%%%%%%%%%%%%%%%%%%%%%%%%%%%%%%%%%%%%%%%%%%%%%%%%%
\subsection{Revision History}

%%%%%%%%%%%%%%%%%%%%%%%%%%%%%%%%%%%%%%%%
\paragraph{v2.0:} 2018/12/30

\begin{itemize}
\item
immediate forward processing
\item
added |\childdocby| mechanism
\item
manual restructured
\end{itemize}

%%%%%%%%%%%%%%%%%%%%%%%%%%%%%%%%%%%%%%%%
\paragraph{v1.6:} 2018/01/17

\begin{itemize}
\item
application for development of include files
\item
corrections to manual
\end{itemize}

%%%%%%%%%%%%%%%%%%%%%%%%%%%%%%%%%%%%%%%%
\paragraph{v1.5:} 2017/05/21

\begin{itemize}
\item
more complete structuring introduced
\item
|\childdocof| introduced
\item
|\childdoc| renamed to |\childdocmain|
\item
|\childredirect| renamed to |\childdocforward| and |\childdocforwardprefix|
and functionality expanded
\end{itemize}

%%%%%%%%%%%%%%%%%%%%%%%%%%%%%%%%%%%%%%%%
\paragraph{v1.0:} 2017/04/27

\begin{itemize}
\item
manual and install package
\item
first version published on CTAN
\end{itemize}

%%%%%%%%%%%%%%%%%%%%%%%%%%%%%%%%%%%%%%%%
\paragraph{v0.6:} 2017/04/26

\begin{itemize}
\item
redirection mechanism added
\end{itemize}

%%%%%%%%%%%%%%%%%%%%%%%%%%%%%%%%%%%%%%%%
\paragraph{v0.5:} 2017/04/26

\begin{itemize}
\item
functionality in definition file
\end{itemize}


%%%%%%%%%%%%%%%%%%%%%%%%%%%%%%%%%%%%%%%%%%%%%%%%%%%%%%%%%%%%%%%%%%%%%%%%%%%%%%%%
%%%%%%%%%%%%%%%%%%%%%%%%%%%%%%%%%%%%%%%%%%%%%%%%%%%%%%%%%%%%%%%%%%%%%%%%%%%%%%%%
%%%%%%%%%%%%%%%%%%%%%%%%%%%%%%%%%%%%%%%%%%%%%%%%%%%%%%%%%%%%%%%%%%%%%%%%%%%%%%%%
\appendix

\settowidth\MacroIndent{\rmfamily\scriptsize 000\ }

 \DocInput{childdoc.dtx}

\end{document}
%</driver>
% \fi
%
% %%%%%%%%%%%%%%%%%%%%%%%%%%%%%%%%%%%%%%%%%%%%%%%%%%%%%%%%%%%%%%%%%%%%%%%%%%%%%%
% %%%%%%%%%%%%%%%%%%%%%%%%%%%%%%%%%%%%%%%%%%%%%%%%%%%%%%%%%%%%%%%%%%%%%%%%%%%%%%
% \section{Sample}
%\iffalse
%<*samplemain>
%\fi
%
% The following presents a sample document
% with two chapters, two parts, a title page,
% a compile flag as well as three forwarding files to set the flag.
% It consists of eight |.tex| files:
% \begin{center}
% \begin{tabular}{ll}
% |cdocsamp.tex|&main file\\
% |cdocsch1.tex|&include file for chapter 1\\
% |cdocsch2.tex|&include file for chapter 2\\
% |cdocspt3.tex|&include file for part 3\\
% |cdocspt4.tex|&include file for part 4\\
% |cdocsdrf.tex|&forwarding file for main file in draft mode\\
% |cdocsfi1.tex|&forwarding file for final version of chapter 1\\
% |cdocsfi2.tex|&forwarding file for final version of chapter 2\\
% \end{tabular}
% \end{center}
% Each of the eight files can be compiled directly by the \LaTeX{} compiler.
%
% %%%%%%%%%%%%%%%%%%%%%%%%%%%%%%%%%%%%%%
% \paragraph{Main File.}
%
% The main file is called |cdocsamp.tex|.
%
% Load the \textsf{childdoc} definitions and
% declare the filename for the main document:
%    \begin{macrocode}
\input{childdoc.def}
\childdocmain{}
%    \end{macrocode}

% Optional override for |\version| flag:
%    \begin{macrocode}
%%\ifchilddoc\else\providecommand{\version}{draft}\fi
%    \end{macrocode}

% Define the default values for the |\version| flag
% (|final| for the main file and |draft| for childs):
%    \begin{macrocode}
\ifchilddoc
\providecommand{\version}{draft}
\else
\providecommand{\version}{final}
\fi
%    \end{macrocode}

% Load the standard document class:
%    \begin{macrocode}
\documentclass[12pt]{article}
%    \end{macrocode}

% Start the document body:
%    \begin{macrocode}
\begin{document}
%    \end{macrocode}

% Declare a title page.
% Print title, part of document being processed and version flag:
%    \begin{macrocode}
\addtocounter{page}{-1}
\begin{center}
{\LARGE\bfseries{}childdoc example\par}
\vspace{1cm}
\ifchilddoc
\ifchilddocmanual part\else chapter\fi:
`\childdocname' of `\childdocjob'\par
\else
main document: `\childdocjob'\par
\fi
version: \version\par
\end{center}
\newpage
%    \end{macrocode}

% Manually include selected file,
% otherwise process as usual:
%    \begin{macrocode}
\ifchilddocmanual
\section*{part `\childdocname'}
\input{\childdocname}
\else
%    \end{macrocode}

% Include the two chapters:
%    \begin{macrocode}
\include{cdocsch1}
\include{cdocsch2}
%    \end{macrocode}

% Include the two parts unless only chapters should be displayed:
%    \begin{macrocode}
\ifchilddoc\else
\section{part three}
\input{cdocspt3}
\section{part four}
\input{cdocspt4}
\fi
%    \end{macrocode}

% Process as usual until here:
%    \begin{macrocode}
\fi
%    \end{macrocode}

% End of document body:
%    \begin{macrocode}
\end{document}
%    \end{macrocode}
%\iffalse
%</samplemain>
%\fi
%
% %%%%%%%%%%%%%%%%%%%%%%%%%%%%%%%%%%%%%%
% \paragraph{Chapter Include Files.}
%
% The include files are called |cdocsch1.tex| and |cdocsch2.tex|.
%
%\iffalse
%<*samplechap1|samplechap2>
%\fi

% Optional override for |\version| flag:
%    \begin{macrocode}
%%\providecommand{\version}{final}
%    \end{macrocode}

% Include the main document:
%    \begin{macrocode}
\input{childdoc.def}
\childdocof{cdocsamp}
%    \end{macrocode}

%\iffalse
%</samplechap1|samplechap2>
%\fi
%
%\iffalse
%<*samplechap1>
%\fi
% Some text for chapter 1:
%    \begin{macrocode}
\section{one}
some text in chapter one
%    \end{macrocode}

%\iffalse
%</samplechap1>
%\fi
% Some text for chapter 2:
%\iffalse
%<*samplechap2>
%\fi
%    \begin{macrocode}
\section{two}
more text in chapter two
%    \end{macrocode}

%\iffalse
%</samplechap2>
%\fi
%
% %%%%%%%%%%%%%%%%%%%%%%%%%%%%%%%%%%%%%%
% \paragraph{Part Include Files.}
%
% The include files are called |cdocspt3.tex| and |cdocspt4.tex|.
%
%\iffalse
%<*samplepart3|samplepart4>
%\fi

% Optional override for |\version| flag:
%    \begin{macrocode}
%%\providecommand{\version}{final}
%    \end{macrocode}

% Include the main document:
%    \begin{macrocode}
\input{childdoc.def}
\childdocby{cdocsamp}
%    \end{macrocode}

%\iffalse
%</samplepart3|samplepart4>
%\fi
%
%\iffalse
%<*samplepart3>
%\fi
% Some text for part 3:
%    \begin{macrocode}
some text in part three
%    \end{macrocode}

%\iffalse
%</samplepart3>
%\fi
% Some text for part 4:
%\iffalse
%<*samplepart4>
%\fi
%    \begin{macrocode}
more text in part four
%    \end{macrocode}

%\iffalse
%</samplepart4>
%\fi
%
% %%%%%%%%%%%%%%%%%%%%%%%%%%%%%%%%%%%%%%
% \paragraph{Forwarding for a Complete Draft.}
%
% The following forwarding file |cdocsdrf.tex|
% compiles the main document in draft mode:
%\iffalse
%<*sampledraft>
%\fi
%    \begin{macrocode}
\def\version{draft}
\input{childdoc.def}
\childdocforward{cdocsamp}
%    \end{macrocode}

%\iffalse
%</sampledraft>
%\fi
%
% %%%%%%%%%%%%%%%%%%%%%%%%%%%%%%%%%%%%%%
% \paragraph{Forwarding for Final Version of the Chapters.}
%
% The following forwarding files |cdocsfn1.tex| and |cdocsfn2.tex|
% (with identical content)
% compile the final versions of the child documents
% |cdocsch1.tex| and |cdocsch2.tex|, respectively:
%\iffalse
%<*samplefinal>
%\fi
%    \begin{macrocode}
\def\version{final}
\input{childdoc.def}
\childdocforwardprefix[cdocsamp]{cdocsfn}{cdocsch}
%    \end{macrocode}

%\iffalse
%</samplefinal>
%\fi
%
% %%%%%%%%%%%%%%%%%%%%%%%%%%%%%%%%%%%%%%
% \paragraph{Command Line Processing.}
%
% The following three command lines generate the output files
% |cdocscld|, |cdocscl1| and |cdocscl2|
% which should be identical to
% |cdocsdrf|, |cdocsch1| and |cdocsfn2|, respectively:
% \begin{center}
% \begin{tabular}{l}
% |latex -jobname cdocscld \|\\
% |  "\def\version{draft}\input{childdoc.def}\childdocforward{cdocsamp}"|\\
% |latex -jobname cdocscl1 \|\\
% |  "\input{childdoc.def}\childdocforward[cdocsamp]{cdocsch1}"|\\
% |latex -jobname cdocscl2 \|\\
% |  "\def\version{final}\input{childdoc.def}\childdocforward{cdocsch2}"|
% \end{tabular}
% \end{center}
% Note that the trailing backslash on each first line
% merely continues the input to the second line
% (for convenient cut ant paste).
% Furthermore, the command |latex| can be replaced by any
% of its alternative versions such as |pdflatex|.
%
% %%%%%%%%%%%%%%%%%%%%%%%%%%%%%%%%%%%%%%%%%%%%%%%%%%%%%%%%%%%%%%%%%%%%%%%%%%%%%%
% %%%%%%%%%%%%%%%%%%%%%%%%%%%%%%%%%%%%%%%%%%%%%%%%%%%%%%%%%%%%%%%%%%%%%%%%%%%%%%
% \section{Implementation}
%\iffalse
%<*package>
%\fi
%
% This section describes the definitions file |childdoc.def|.

% The definitions cannot be loaded using |\usepackage| or |\RequirePackage|
% which has a mechanism to prevent loading a style file more than once.
% When loading the definitions by means of |\input|
% multiple instances have to be prevented manually:
%\iffalse
%This code needs to be before the `\ProvidesFile' directive
%which is defined at the beginning of this file.
%Therefore it is also placed there and commented out here.
%</package>
%<*discard>
%\fi
%    \begin{macrocode}
\ifdefined\childdocmain\endinput\fi
%    \end{macrocode}
%\iffalse
%</discard>
%<*package>
%\fi
%
% \macro{\ifchilddoc}
% \macro{\ifchilddocmanual}
% The conditional |\ifchilddoc| tells whether a
% child (true) or main (false) document is being compiled.
% The conditional |\ifchilddocmanual| tells whether
% the |\includeonly| mechanism is used (false) or
% the selection of child files must be performed manually (true).
% The definitions initialise to false:
%    \begin{macrocode}
\newif\ifchilddoc
\newif\ifchilddocmanual
%    \end{macrocode}

% \macro{\childdocname}
% \macro{\childdocjob}
% The macro |\childdocname| stores the name of the main document
% to be compiled. The macro |\childdocjob| stores the name of
% the document on which the \LaTeX{} compiler was originally invoked.
% The content of |\jobname| cannot be compared
% to filenames specified in the source due to different catcodes.
% The following code rescans |\jobname|, stores the result
% in |\childdocname| and saves a copy in |\childdocjob|:
%    \begin{macrocode}
\edef\childdocname{\scantokens\expandafter{\jobname\noexpand}}
\let\childdocjob\childdocname
%    \end{macrocode}

% \macro{\childdocdisable}
% The macro |\childdocdisable| prevents the main file
% from being processed more than once.
% At this stage, the main document command |\childdocmain|
% is assumed to be called once again where it should do nothing.
% Any subsequent call to it should prevent
% a secondary processing of the main document
% It overwrites the forwarding commands
% |\childdocof| and |\childdocforward|
% with empty macros to prevent further inclusions of the main document:
%    \begin{macrocode}
\newcommand{\childdocdisable}
{
  \renewcommand{\childdocmain}[1]{\renewcommand{\childdocmain}[1]{\endinput}}
  \renewcommand{\childdocof}[1]{}
  \renewcommand{\childdocby}[2][]{}
  \renewcommand{\childdocforward}[2][]{}
  \renewcommand{\childdocdisable}{}
}
%    \end{macrocode}

% \macro{\childdocmain}
% The macro |\childdocmain| is to be called at the top of the main file
% with nothing or the main filename (without extension) as argument.
% First, it breaks loops.
% If the argument is not empty and does not match |\childdocname|
% (which is set by the first inclusion of |childdoc.def|),
% |\ifchilddoc| is set to true, |\includeonly| is applied to the child file
% and |\jobname| is set to the main file
% (for proper handling of |.aux| files):
%    \begin{macrocode}
\newcommand{\childdocmain}[1]
{
  \childdocdisable\childdocmain{}
  \if?#1?\else
    \begingroup
      \def\childdoctmp{#1}
      \ifx\childdoctmp\childdocname
        \def\childdoctmp{}
      \else
        \def\childdoctmp
        {
          \childdoctrue
          \includeonly{\childdocname}
          \def\childdocjob{#1}
          \def\jobname{#1}
        }
      \fi
      \expandafter
    \endgroup
    \childdoctmp
  \fi
}
%    \end{macrocode}

% \macro{\childdocof}
% The command |\childdocof| redirects
% compilation to the main file |#1|.
%    \begin{macrocode}
\newcommand{\childdocof}[1]
{
  \childdocdisable
  \childdoctrue
  \includeonly{\childdocname}
  \def\jobname{#1}
  \def\childdocjob{#1}
  \input{#1}
}
%    \end{macrocode}

% \macro{\childdocby}
% The command |\childdocby| ....
%    \begin{macrocode}
\newcommand{\childdocby}[2][]
{
  \childdocdisable
  \childdoctrue
  \childdocmanualtrue
  \if?#1?\else
    \def\jobname{#2}
  \fi
  \def\childdocjob{#2}
  \input{#2}
  \endinput
}
%    \end{macrocode}

% \macro{\childdocforward}
% The command |\childdocforward| redirects
% compilation to the main file or
% (if the optional argument is given) a child file.
% Parameters are set as if the main file
% or a child file starting with |\childdocof| was compiled.
% Then compilation is handed over to the main file:
%    \begin{macrocode}
\newcommand{\childdocforward}[2][]
{
  \begingroup
    \if?#1?
      \def\childdoctmp
      {
        \def\childdocname{#2}
        \def\childdocjob{#2}
        \def\jobname{#2}
        \input{#2}
        \endinput
      }
    \else
      \def\childdoctmp
      {
        \childdocdisable
        \def\childdocname{#2}
        \childdoctrue
        \includeonly{#2}
        \def\childdocjob{#1}
        \def\jobname{#1}
        \input{#1}
        \endinput
      }
    \fi
    \expandafter
  \endgroup
  \childdoctmp
}
%    \end{macrocode}

% \macro{\childdocforwardprefix}
% The command |\childdocforwardprefix| redirects
% compilation to the main or a child file by means of a pattern.
% The prefix |#1| in the current filename is replaced by |#2|
% and the suffix of the current filename is kept
% (it is assumed that the filename does not contain the substring `|~~~|'
% which is used as a delimiter).
% Compilation is handed over to the new file by |\childdocforward|:
%    \begin{macrocode}
\newcommand{\childdocforwardprefix}[3][]
{
  \begingroup
    \def\childdocextract #2##1~~~{\def\childdoctmp{\childdocforward[#1]{#3##1}}}
    \expandafter\childdocextract\childdocname~~~
    \expandafter
  \endgroup
  \childdoctmp
}
%    \end{macrocode}

% \macro{\childdoc}
% The deprecated macro |\childdoc| is a legacy version of |\childdocmain|:
%    \begin{macrocode}
\newcommand{\childdoc}{\childdocmain}
%    \end{macrocode}

% \macro{\childdocredirect}
% The deprecated macro |\childdocredirect| is a legacy version
% of |\childdocforward| and |\childdocforwardprefix|:
%    \begin{macrocode}
\newcommand{\childdocredirect}[2][]
{
  \begingroup
    \if?#1?
      \def\childdoctmp{\childdocforward{#2}}
    \else
      \def\childdoctmp{\childdocforwardprefix{#1}{#2}}
    \fi
    \expandafter
  \endgroup
  \childdoctmp
}
%    \end{macrocode}

%\iffalse
%</package>
%\fi
%
\endinput
\childdocforward[cdocsamp]{cdocsch1}"|\\
% |latex -jobname cdocscl2 \|\\
% |  "\def\version{final}% \iffalse
%
% childdoc.dtx Copyright (C) 2017-2018 Niklas Beisert
%
% This work may be distributed and/or modified under the
% conditions of the LaTeX Project Public License, either version 1.3
% of this license or (at your option) any later version.
% The latest version of this license is in
%   http://www.latex-project.org/lppl.txt
% and version 1.3 or later is part of all distributions of LaTeX
% version 2005/12/01 or later.
%
% This work has the LPPL maintenance status `maintained'.
%
% The Current Maintainer of this work is Niklas Beisert.
%
% This work consists of the files childdoc.dtx and childdoc.ins
% and the derived files childdoc.def and cdocsamp.tex with
% cdocsch1.tex, cdocsch2.tex, cdocsdrf.tex, cdocsfn1.tex, cdocsfn2.tex.
%
%<package>\ifdefined\childdocmain\endinput\fi
%<package>\ProvidesFile{childdoc.def}[2018/12/30 v2.0 child document driver]
%<samplemain>\ProvidesFile{cdocsamp.tex}[2018/12/30 v2.0 sample for childdoc]
%<*driver>
%\ProvidesFile{childdoc.drv}[2018/12/30 v2.0 childdoc reference manual file]
\PassOptionsToClass{10pt,a4paper}{article}
\documentclass{ltxdoc}

\usepackage[margin=35mm]{geometry}
\usepackage{hyperref}
\usepackage{hyperxmp}
\usepackage[usenames]{color}

\hypersetup{colorlinks=true}
\hypersetup{pdfstartview=FitH}
\hypersetup{pdfpagemode=UseNone}
\hypersetup{pdfsource={}}
\hypersetup{pdflang={en-UK}}
\hypersetup{pdfcopyright={Copyright 2017-2018 Niklas Beisert.
  This work may be distributed and/or modified under the
  conditions of the LaTeX Project Public License, either version 1.3
  of this license or (at your option) any later version.}}
\hypersetup{pdflicenseurl={http://www.latex-project.org/lppl.txt}}
\hypersetup{pdfcontactaddress={ETH Zurich, ITP, HIT K,
  Wolfgang-Pauli-Strasse 27}}
\hypersetup{pdfcontactpostcode={8093}}
\hypersetup{pdfcontactcity={Zurich}}
\hypersetup{pdfcontactcountry={Switzerland}}
\hypersetup{pdfcontactemail={nbeisert@itp.phys.ethz.ch}}
\hypersetup{pdfcontacturl={http://people.phys.ethz.ch/\xmptilde nbeisert/}}

\newcommand{\secref}[1]{\hyperref[#1]{section \ref*{#1}}}

\parskip1ex
\parindent0pt
\let\olditemize\itemize
\def\itemize{\olditemize\parskip0pt}

\begin{document}

\title{The \textsf{childdoc} Package}
\hypersetup{pdftitle={The childdoc Package}}
\author{Niklas Beisert\\[2ex]
  Institut f\"ur Theoretische Physik\\
  Eidgen\"ossische Technische Hochschule Z\"urich\\
  Wolfgang-Pauli-Strasse 27, 8093 Z\"urich, Switzerland\\[1ex]
  \href{mailto:nbeisert@itp.phys.ethz.ch}
  {\texttt{nbeisert@itp.phys.ethz.ch}}}
\hypersetup{pdfauthor={Niklas Beisert}}
\hypersetup{pdfsubject={Manual for the LaTeX2e Package childdoc}}
\date{30 December 2018, \textsf{v2.0}}
\maketitle

\begin{abstract}\noindent
\textsf{childdoc} is a \LaTeXe{} package
that enables the direct compilation
of document sections included by |\include|
to individual files.
\end{abstract}

\begingroup
\parskip0ex
\tableofcontents
\endgroup

%%%%%%%%%%%%%%%%%%%%%%%%%%%%%%%%%%%%%%%%%%%%%%%%%%%%%%%%%%%%%%%%%%%%%%%%%%%%%%%%
%%%%%%%%%%%%%%%%%%%%%%%%%%%%%%%%%%%%%%%%%%%%%%%%%%%%%%%%%%%%%%%%%%%%%%%%%%%%%%%%
\section{Introduction}

\LaTeX{} provides a mechanism to structure a large document (such as a book)
into a main file and several child files (containing the chapters)
using the |\include| command.
This mechanism is beneficial for documents
which span hundreds of pages in order to
make the source file(s) more manageable.
Moreover, compilation can be restricted to
selected child files by means of the |\includeonly| command.
The latter feature can be used to reduce the compilation time while editing
(this was significantly more useful in the earlier days of \LaTeX{})
or to generate a smaller document which is easier to navigate.
Another application of |\includeonly| is to generate
documents consisting of selected parts of the complete document.

However, there are a few drawbacks of the plain |\include| mechanism:
\begin{itemize}
\item
The child files cannot be compiled on their own,
they can only be compiled via the main file.
A naive editing environment
(such as a text editor with an option
to have the current file processed by \LaTeX)
may require one to switch to the main file before compiling;
attempting to compile the child file produces errors.
\item
The main file must be modified (each time)
to adjust the |\includeonly| command
to the present needs. This easily leaves the main file in a messy state.
\item
The generated document will always carry the filename
of the main document. This is inconvenient if
several child files are to be compiled and
to be kept for distribution.
\end{itemize}

The present package provides a simple interface
to make child files individually compilable by \LaTeX{}.
Compiling a child file then has the same effect as compiling
the main file with an |\includeonly| command
to select the appropriate child.
Moreover the generated document will carry the name of the child
rather than the main file.
This resolves all three above issues.

This feature is meant to make the editing of books,
thesis documents and lecture notes somewhat more convenient.
However, the package can also be used efficiently for
composing a series of documents (such as exercise sheets)
which are typically distributed individually.
It then assists the author in generating the individual documents
(potentially in different versions)
as well as a document containing the collected series.
Another application is in developing style files
or other kinds of included material
where compilation of the style file could redirect
to a sample or test file.

%%%%%%%%%%%%%%%%%%%%%%%%%%%%%%%%%%%%%%%%%%%%%%%%%%%%%%%%%%%%%%%%%%%%%%%%%%%%%%%%
%%%%%%%%%%%%%%%%%%%%%%%%%%%%%%%%%%%%%%%%%%%%%%%%%%%%%%%%%%%%%%%%%%%%%%%%%%%%%%%%
\section{Usage}

First of all, the package \textsf{childdoc} is \emph{not} a standard
\LaTeXe{} |.sty| style file! Therefore it needs to be invoked in
a non-standard way.

%%%%%%%%%%%%%%%%%%%%%%%%%%%%%%%%%%%%%%%%%%%%%%%%%%%%%%%%%%%%%%%%%%%%%%%%%%%%%%%%
\subsection{Included Files}
\label{sec:include}

%%%%%%%%%%%%%%%%%%%%%%%%%%%%%%%%%%%%%%%%
\DescribeMacro{\childdocmain}
To use the package, add the commands
\begin{center}
\begin{tabular}{l}
|\input{childdoc.def}|\\
|\childdocmain{}|\\
\end{tabular}
\end{center}
at the very top of the main \LaTeX{} file,
in particular \emph{before} the |\documentclass| statement!
The argument of |\childdocmain| should be left empty
(but it must be present).

%%%%%%%%%%%%%%%%%%%%%%%%%%%%%%%%%%%%%%%%
\DescribeMacro{\childdocof}
Furthermore, add the commands
\begin{center}
\begin{tabular}{l}
|\input{childdoc.def}|\\
|\childdocof{|\textit{main}|}|\\
\end{tabular}
\end{center}
at the top of every child file \textit{child}
which is included by |\include{|\textit{child}|}|
from within the main file
(or at least for those files to be compiled individually).
The argument \textit{main} must be the filename of the main file.

There are a couple of
considerations in setting up the main and child documents:

%%%%%%%%%%%%%%%%%%%%%%%%%%%%%%%%%%%%%%%%
\paragraph{Restrictions.}

Please note the following restrictions:
\begin{itemize}
\item
|\childdocmain| must be called with one argument \textit{main}
to ensure compatibility with earlier version of the package.
It must either be empty (|\childdocmain{}|)
or precisely match the filename of the main file in which it is specified.
See \secref{sec:detection} for further information.
\item
The filename \textit{main} must be specified without the |.tex| extension.
\item
The filename \textit{main} is case sensitive
(even in case-insensitive file systems)
due to internal string comparison.
\item
The argument \textit{main} should be fully expanded, it cannot be a macro.
\item
Subdirectories and special characters should be avoided in filenames.
\item
The command |\childdocmain{|\textit{main}|}| must be followed by a whitespace.
It should not be followed immediately by another command
or by a comment mark `|%|'.
This is because the \TeX{} parser reads the token immediately following
the argument of |\childdocmain| and puts it
at the beginning of every child section;
however, a white\-space is ignored.
\end{itemize}

%%%%%%%%%%%%%%%%%%%%%%%%%%%%%%%%%%%%%%%%
\paragraph{Content of Main File.}

It is advisable to place all content in the child files included by |\include|.
Any output contained in the main file will appear in all child documents
unless suppressed manually;
it cannot be suppressed automatically by the |\includeonly| directive
and thus should normally be avoided.
A method to include some content in the main file
by means of conditional processing is described in \secref{sec:conditional}.

%%%%%%%%%%%%%%%%%%%%%%%%%%%%%%%%%%%%%%%%
\paragraph{Page Numbering.}

When only a part of the document is compiled,
the appropriate numbering of pages
(as well as other status parameters)
is determined from the |.aux| files.
The latter contain information from previous passes.
However this information needs to propagate through
all intermediate child documents.
Therefore the page numbering in child documents may well
be inconsistent until the complete document is compiled at least once.

A useful (if unconventional) way to always ensure a consistent
page numbering is to restart the numbering in each child document
and denote the pages by `\textit{child}|.|\textit{page}'
where \textit{child} represents the chapter/section number of the child file.
This can be achieved by the command
|\numberwithin{page}{|\textit{child}|}|
of the \textsf{amsmath} package
where \textit{child} can be |chapter| or |section|
depending on the chosen structuring.
Alternatively, one can modify the macro |\thepage| appropriately
and reset the counter |page| at the start of each child file.

%%%%%%%%%%%%%%%%%%%%%%%%%%%%%%%%%%%%%%%%%%%%%%%%%%%%%%%%%%%%%%%%%%%%%%%%%%%%%%%%
\subsection{Conditional Processing}
\label{sec:conditional}

The package provides a mechanism to compile different versions
of a document. To customise the versions further some conditional processing
can come in handy to distinguish which version is being compiled.
The package provides two macros to describe the compilation context:

%%%%%%%%%%%%%%%%%%%%%%%%%%%%%%%%%%%%%%%%
\DescribeMacro{\ifchilddoc}
The conditional |\ifchilddoc| distinguishes between the compilation of
child documents and the main document:
%
\begin{center}
|\ifchilddoc |\textit{child-code}| |[|\||else |\textit{main-code}]| \||fi|
\end{center}

%%%%%%%%%%%%%%%%%%%%%%%%%%%%%%%%%%%%%%%%
\DescribeMacro{\childdocname}
\DescribeMacro{\childdocjob}
The macro |\childdocname| contains the filename (without extension)
of the main or child file being processed.
Note that |\childdocjob| will always contain the name of the main file.

%%%%%%%%%%%%%%%%%%%%%%%%%%%%%%%%%%%%%%%%
\paragraph{Title Page.}

Conditional processing can be used to include a title or banner page
in the main document when proper precautions are taken.
Importantly, the code in the main file should ensure that the page counter
(as well as other status parameters which are stored in the |.aux| files)
takes the same value after the conditional processing.
Otherwise the page numbers may take divergent values
depending on which part is compiled.

For example, a title page could be declared by:
%
\begin{center}
\begin{tabular}{l}
|\ifchilddoc\||else|\\
|\addtocounter{page}{-1}|\\
\textit{code for title page}\\
|\newpage|\\
|\||fi|
\end{tabular}
\end{center}
%
A banner page for the child documents can be generated by:
%
\begin{center}
\begin{tabular}{l}
|\ifchilddoc|\\
|\addtocounter{page}{-1}|\\
\textit{code for banner page}\\
|\newpage|\\
|\||fi|
\end{tabular}
\end{center}
%
Here one could write a message such as:
\begin{center}
|This is the part \childdocname{} of \childdocjob{}.|
\end{center}

%%%%%%%%%%%%%%%%%%%%%%%%%%%%%%%%%%%%%%%%%%%%%%%%%%%%%%%%%%%%%%%%%%%%%%%%%%%%%%%%
\subsection{Flags}
\label{sec:flags}

The package makes it easy to generate different versions
of the main or child documents.
To this end compilation flags can be defined
and assigned different default values.
They will be particularly useful in conjunction
with the forwarding mechanism described in \secref{sec:forward}.

For example, it may be useful to have a flag |\version|
which can be set to |draft| or |final|.
The document source will contain some conditional code
depending on the value of |\version|.
Suppose further, the flag should default to |final| for the main file
and to |draft| for child files
which is a natural assignment for editing the document.
This is achieved by placing the following code
in the preamble of the main document
(below the |\childdocmain| directive):
%
\begin{center}
\begin{tabular}{l}
|\ifchilddoc|\\
|\providecommand{\version}{draft}|\\
|\||else|\\
|\providecommand{\version}{final}|\\
|\||fi|
\end{tabular}
\end{center}
%
The definition by |\providecommand| makes sure
that previous definitions are not overwritten.
Further statements |\providecommand{\version}{...}|
can thus be added before the above code to override it.

For the main file, one might add a line
(between |\childdocmain| and the above block)
%
\begin{center}
|%\ifchilddoc\||else\providecommand{\version}{draft}\||fi|
\end{center}
%
which can be uncommented to produce a draft version.
Likewise one can add a line to the very top of a child file
(above the |\childdocof{|\textit{main}|}| directive)
%
\begin{center}
|%\providecommand{\version}{final}|
\end{center}
%
which can be uncommented to produce the final version of this child document.

%%%%%%%%%%%%%%%%%%%%%%%%%%%%%%%%%%%%%%%%%%%%%%%%%%%%%%%%%%%%%%%%%%%%%%%%%%%%%%%%
\subsection{Forwarding}
\label{sec:forward}

Different versions of the main or child documents
using compilation flags as described in \secref{sec:flags}
can be (permanently) stored in different files
for convenient compilation, viewing and distribution.
To this end, the package defines a command
to pass on compilation to a different file:

%%%%%%%%%%%%%%%%%%%%%%%%%%%%%%%%%%%%%%%%
\DescribeMacro{\childdocforward}
The command |\childdocforward| redirects processing to
another source file:
%
\begin{center}
\begin{tabular}{l}
|\input{childdoc.def}|\\
|\childdocforward[|\textit{main}|]{|\textit{dest}|}|\\
\end{tabular}
\end{center}
%
The argument \textit{dest} is the destination file
(without extension).
It should be the main file or one of the child files.
Note that further \textsf{childdoc} directives
such as |\childdocof| and |\childdocforward|
in the indicated file will be processed in this form.
The optional argument \textit{main}
passes on directly to the main file \textit{main}
while pretending to compile the child \textit{dest}.
This form behaves as if \textit{dest}
issues |\childdocof{|\textit{main}|}| right away,
and no further \textsf{childdoc} directives will be processed.

%%%%%%%%%%%%%%%%%%%%%%%%%%%%%%%%%%%%%%%%
\DescribeMacro{\...prefix}
In the alternative form |\childdocforwardprefix|,
%
\begin{center}
\begin{tabular}{l}
|\input{childdoc.def}|\\
|\childdocforwardprefix[|\textit{main}|]{|\textit{prefix}|}{|\textit{dest}|}|
\end{tabular}
\end{center}
%
the destination file is determined by a pattern
depending on the current file:
To make this work, the current file must be called
`{\textit{prefix}\hspace{0.2em}\textit{suffix}}'
with \textit{prefix} matching precisely the argument.
Processing is then passed on to the file
`{\textit{dest}\hspace{0.2em}\textit{suffix}}'.
Surely, the same effect is achieved by
directly specifying the
argument `{\textit{dest}\hspace{0.2em}\textit{suffix}}'
in the first form.
However, that requires to set up a different file
for each child. With the alternative form of the command
all these files can have exactly the same content
which simplifies setting them up and maintaining them.

For example, the following file |draft.tex|
with a compilation flag |\version| as described in \secref{sec:flags}
compiles the main document as a draft:
%
\begin{center}
\begin{tabular}{l}
|\def\version{draft}|\\
|\input{childdoc.def}|\\
|\childdocforward{|\textit{main}|}|
\end{tabular}
\end{center}
%
Likewise, the following files |final|\textit{nn}|.tex|
compile the final version of the child document
|child|\textit{nn}|.tex|:
%
\begin{center}
\begin{tabular}{l}
|\def\version{final}|\\
|\input{childdoc.def}|\\
|\childdocforwardprefix{final}{child}|
\end{tabular}
\end{center}
%

Note that when several versions of a main file and/or of each child file
are to be generated, it may be convenient to set up a |Makefile| or
shell script to automatise the process.

%%%%%%%%%%%%%%%%%%%%%%%%%%%%%%%%%%%%%%%%%%%%%%%%%%%%%%%%%%%%%%%%%%%%%%%%%%%%%%%%
\subsection{Command Line Processing}
\label{sec:commandline}

The effect of redirection files can also be achieved by invoking
the \LaTeX{} compiler with a more elaborate command line.
Most conveniently this should be done as part
of a shell script or a |Makefile|.

When using \textsf{childdoc} in the main file, the following
command lines effectively perform a redirection
(note that depending on the shell being used,
backslashes may have to be doubled: `|\|' $\to$ `|\\|'):
%
\begin{center}
|... -jobname "|\textit{target}|" |\\|"|[\textit{flags}]%
|\input{childdoc.def}\childdocforward[|\textit{main}|]{|\textit{dest}|}"|
\end{center}
%
Here \textit{target} is the name of the output file,
\textit{main} is the name of the main file
and \textit{dest} is the name of the main or child file to be processed
(all filenames without extensions).
The optional argument \textit{main} can be omitted
if \textit{main} matches \textit{dest}.
Optionally, compilation \textit{flags} can be defined via |\def| commands.
This command line makes the \TeX{} engine believe
it is compiling the file \textit{target}
whose content is specified as the latter parameter.
The provided code then forwards the processing to
\textit{main} or \textit{dest} as described in \secref{sec:forward}.

%%%%%%%%%%%%%%%%%%%%%%%%%%%%%%%%%%%%%%%%%%%%%%%%%%%%%%%%%%%%%%%%%%%%%%%%%%%%%%%%
\subsection{Include by Input}
\label{sec:input}

Including child documents by |\include| has some restrictions by design.
Most notably, the content of a child document always occupies
its own set of pages; pages cannot be shared between child documents.
Usually, this behaviour makes perfect sense
because each child document contain an essential part of the document.
However, in some situations it may be desirable to compose
a document from a collection of parts
without having mandatory page breaks between then.
For this case, the package
provides a mechanism to include parts
by |\input| which can also be processed individually.
However, by construction this mechanism
requires manual handling of the content to be output.

%%%%%%%%%%%%%%%%%%%%%%%%%%%%%%%%%%%%%%%%
\DescribeMacro{\ifchilddocmanual}
The main file should be prepared as usual, see \secref{sec:include}.
However, the document body must make a distinction
between processing of an individual part and of the main document, e.g.:
%
\begin{center}
\begin{tabular}{l}
|\ifchilddocmanual|\\
|\input{\childdocname}|\\
|\||else|\\
\textit{document body with }|\input{|\textit{part}|}|\\
|\||fi|
\end{tabular}
\end{center}
%
The conditional |\ifchilddocmanual| is true whenever
a part to be included by |\input| is being compiled,
and the name of the part is stored in |\childdocname|.

%%%%%%%%%%%%%%%%%%%%%%%%%%%%%%%%%%%%%%%%
\DescribeMacro{\childdocby}
Each part to be included by |\input| should start with:
%
\begin{center}
\begin{tabular}{l}
|\input{childdoc.def}|\\
|\childdocby{|\textit{main}|}|\\
\end{tabular}
\end{center}
%
The directive |\childdocby| is similar to |\childdocof|
described in \secref{sec:include},
but the subsequent selection of content must be done manually.
To that end, both |\ifchilddoc| and |\ifchilddocmanual|
will be true upon processing of a part,
and the name of the part is stored in |\childdocname|.
Note that |\jobname| will be set to the filename of the current part
so that each part receives an individual |.aux| file
that does not interfere with the |.aux| file(s) of the main document.
This behaviour can be altered by the alternative form
|\childdocby[*]{|\textit{main}|}| (with a non-empty optional argument)
which uses the |.aux| file of the main document
by setting |\jobname| to \textit{main}.

%%%%%%%%%%%%%%%%%%%%%%%%%%%%%%%%%%%%%%%%%%%%%%%%%%%%%%%%%%%%%%%%%%%%%%%%%%%%%%%%
\subsection{Driver Development}
\label{sec:driver}

The \textsf{childdoc} mechanism can also be use for the development
of definition files such as \LaTeX{} styles or classes.
This case differs from the above setup with multiple parts
included by |\include| in that no |\includeonly| should be invoked.
This can be achieved by starting the include file
(before |\ProvidesPackage|) with:
%
\begin{center}
\begin{tabular}{l}
|\input{childdoc.def}|\\
|\childdocforward{|\textit{main}|}|\\
\end{tabular}
\end{center}
%
or alternatively with:
%
\begin{center}
\begin{tabular}{l}
|\input{childdoc.def}|\\
|\childdocby{|\textit{main}|}|\\
\end{tabular}
\end{center}
%
Both forms have slightly different effects as described above.
The main file is prepared as usual, see \secref{sec:include}.

%%%%%%%%%%%%%%%%%%%%%%%%%%%%%%%%%%%%%%%%%%%%%%%%%%%%%%%%%%%%%%%%%%%%%%%%%%%%%%%%
\subsection{Legacy Detection}
\label{sec:detection}

The directive |\childdocmain| in the main file can detect
whether the complete document or merely a child is to be compiled
even without using the directive |\childdocof|.
This method is deprecated because it is less robust
and there is no compelling reason to use it;
it is merely provided for backward compatibility
and it may be removed in future versions.

If the detection mechanism is to be used,
it is mandatory to correctly specify
the filename of the main file as the argument of |\childdocmain|:
%
\begin{center}
\begin{tabular}{l}
|\input{childdoc.def}|\\
|\childdocmain{|\textit{main}|}|\\
\end{tabular}
\end{center}
%
If |\jobname| does not match the argument \textit{main} of |\childdocmain|,
it is assumed that |\jobname| points to the child file to be compiled.
When using |\childdocmain| with the main file specified as argument,
it suffices to start a child file
with just |\input{|\textit{main}|}|
without loading of the package and using |\childdocof|.
If instead all processing is done
with the appropriate \textsf{childdoc} directives,
the argument of \textit{main} of |\childdocmain| can be empty.

An alternative version of the command line processing described
in \secref{sec:commandline} using the detection mechanism reads:
%
\begin{center}
|... -jobname "|\textit{target}|" "|[\textit{flags}]%
[|\def\jobname{|\textit{dest}|}|]|\input{|\textit{main}|}"|
\end{center}

%%%%%%%%%%%%%%%%%%%%%%%%%%%%%%%%%%%%%%%%%%%%%%%%%%%%%%%%%%%%%%%%%%%%%%%%%%%%%%%%
\subsection{Manual Code}
\label{sec:manual}

In case one cannot be certain whether the definitions file |childdoc.def|
is installed on the target \TeX{} distribution
and one prefers not to ship it,
it is conceivable to paste a few relevant commands into the sources.

To that end, drop all statements |\input{childdoc.def}|
and perform the replacements as outlined below.
Instead of |\childdocmain{|\textit{main}|}| add the following code
to the top of the main file:
%
\begin{center}
\begin{tabular}{l}
|\||ifdefined\childdocname\endinput\||fi\newif\ifchilddoc|\\
|\edef\childdocname{\scantokens\expandafter{\jobname\noexpand}}|\\
|\def\childdocmain{|\textit{main}|}\||ifx\childdocmain\childdocname\||else|\\
|\childdoctrue\includeonly{\childdocname}\let\jobname\childdocmain\||fi|\\
\end{tabular}
\end{center}
%
Instead of |\childdocof{|\textit{main}|}| just include the main file
at the top of each child file:
%
\begin{center}
|\input{|\textit{main}|}|
\end{center}
%
A simple redirection |\childdocforward{|\textit{dest}|}| is achieved by:
%
\begin{center}
|\def\jobname{|\textit{dest}|}\input{\jobname}|
\end{center}
%
The redirection with prefix
|\childdocforwardprefix[|\textit{prefix}|]{|\textit{dest}|}|
is accomplished by:
%
\begin{center}
\begin{tabular}{l}
|{\edef\jobname{\scantokens\expandafter{\jobname\noexpand}}|\\
|\def\redirectjob |\textit{prefix}|#1~~~{\gdef\jobname{|\textit{dest}|#1}}|\\
|\expandafter\redirectjob\jobname~~~}\input{\jobname}|
\end{tabular}
\end{center}

In an alternative approach,
child documents can be compiled by a specific command line
without additional code or specific definitions:
%
\begin{center}
|... -jobname "|\textit{target}|" "|[\textit{flags}]%
|\includeonly{|\textit{dest}|}\input{|\textit{main}|}"|
\end{center}
%

%%%%%%%%%%%%%%%%%%%%%%%%%%%%%%%%%%%%%%%%%%%%%%%%%%%%%%%%%%%%%%%%%%%%%%%%%%%%%%%%
%%%%%%%%%%%%%%%%%%%%%%%%%%%%%%%%%%%%%%%%%%%%%%%%%%%%%%%%%%%%%%%%%%%%%%%%%%%%%%%%
\section{Information}

%%%%%%%%%%%%%%%%%%%%%%%%%%%%%%%%%%%%%%%%%%%%%%%%%%%%%%%%%%%%%%%%%%%%%%%%%%%%%%%%
\subsection{Copyright}

Copyright \copyright{} 2017--2018 Niklas Beisert

This work may be distributed and/or modified under the
conditions of the \LaTeX{} Project Public License, either version 1.3
of this license or (at your option) any later version.
The latest version of this license is in
  \url{http://www.latex-project.org/lppl.txt}
and version 1.3 or later is part of all distributions of \LaTeX{}
version 2005/12/01 or later.

This work has the LPPL maintenance status `maintained'.

The Current Maintainer of this work is Niklas Beisert.

This work consists of the files |README.txt|, |childdoc.ins| and |childdoc.dtx|
as well as the derived files |childdoc.def|, |cdocsamp.tex|
with |cdocsch1.tex|, |cdocsch2.tex|, |cdocspt3.tex|, |cdocspt4.tex|,
|cdocsdrf.tex|, |cdocsfn1.tex|, |cdocsfn2.tex|
as well as |childdoc.pdf|.

%%%%%%%%%%%%%%%%%%%%%%%%%%%%%%%%%%%%%%%%%%%%%%%%%%%%%%%%%%%%%%%%%%%%%%%%%%%%%%%%
\subsection{Files and Installation}

The package consists of the files:
%
\begin{center}
\begin{tabular}{ll}
    |README.txt|   & readme file \\
    |childdoc.ins| & installation file \\
    |childdoc.dtx| & source file \\
    |childdoc.def| & definition file \\
    |cdocsamp.tex| & sample main file \\
    |cdocsch1.tex| & sample include file \\
    |cdocsch2.tex| & sample include file \\
    |cdocspt3.tex| & sample part file \\
    |cdocspt4.tex| & sample part file \\
    |cdocsdrf.tex| & sample redirection file \\
    |cdocsfn1.tex| & sample redirection file \\
    |cdocsfn2.tex| & sample redirection file \\
    |childdoc.pdf| & manual
\end{tabular}
\end{center}
%
The distribution consists of the files
|README.txt|, |childdoc.ins| and |childdoc.dtx|.
%
\begin{itemize}
\item
Run (pdf)\LaTeX{} on |childdoc.dtx|
to compile the manual |childdoc.pdf| (this file).
\item
Run \LaTeX{} on |childdoc.ins| to create the definitions file |childdoc.def|
and the sample |cdocsamp.tex| with include files
|cdocsch1.tex|, |cdocsch2.tex|, |cdocspt3.tex|, |cdocspt4.tex|,
|cdocsdrf.tex|, |cdocsfn1.tex|, |cdocsfn2.tex|.
Then copy the file |childdoc.def| to an appropriate directory of your \LaTeX{}
distribution, e.g.\ \textit{texmf-root}|/tex/latex/childdoc|.
\end{itemize}

%%%%%%%%%%%%%%%%%%%%%%%%%%%%%%%%%%%%%%%%%%%%%%%%%%%%%%%%%%%%%%%%%%%%%%%%%%%%%%%%
\subsection{Related CTAN Packages}

There are several other packages which offer a similar functionality:
%
\begin{itemize}
\item
The packages
\href{http://ctan.org/pkg/docmute}{\textsf{docmute}},
\href{http://ctan.org/pkg/includex}{\textsf{includex}} and
\href{http://ctan.org/pkg/standalone}{\textsf{standalone}}
provide commands to include only the document body of
a child file thus allowing both files to be compiled individually.
\item
The packages \href{http://ctan.org/pkg/subdocs}{\textsf{subdocs}}
and \href{http://ctan.org/pkg/subfiles}{\textsf{subfiles}}
provide structures in which the main and child documents can be
encapsulated and allowing them to be compiled individually.
The inclusion mechanism is different from the conventional |\include|.
\item
The package \href{http://ctan.org/pkg/combine}{\textsf{combine}}
is an elaborate solution to combine several documents into one.
\end{itemize}
%
See also the CTAN topic \href{http://ctan.org/topic/subdocs}{\textsf{subdocs}}
for further related packages.
The present package differs from the above solutions in that
a document structure constructed with the conventional |\include| mechanism
just needs two extra commands at the top of every file
such that all constituent files can be compiled individually.

%%%%%%%%%%%%%%%%%%%%%%%%%%%%%%%%%%%%%%%%%%%%%%%%%%%%%%%%%%%%%%%%%%%%%%%%%%%%%%%%
%\subsection{Feature Suggestions}
%
%The following is a list of features which may be useful for future
%versions of this package:
%%
%\begin{itemize}
%\item
%\ldots
%\end{itemize}

%%%%%%%%%%%%%%%%%%%%%%%%%%%%%%%%%%%%%%%%%%%%%%%%%%%%%%%%%%%%%%%%%%%%%%%%%%%%%%%%
\subsection{Revision History}

%%%%%%%%%%%%%%%%%%%%%%%%%%%%%%%%%%%%%%%%
\paragraph{v2.0:} 2018/12/30

\begin{itemize}
\item
immediate forward processing
\item
added |\childdocby| mechanism
\item
manual restructured
\end{itemize}

%%%%%%%%%%%%%%%%%%%%%%%%%%%%%%%%%%%%%%%%
\paragraph{v1.6:} 2018/01/17

\begin{itemize}
\item
application for development of include files
\item
corrections to manual
\end{itemize}

%%%%%%%%%%%%%%%%%%%%%%%%%%%%%%%%%%%%%%%%
\paragraph{v1.5:} 2017/05/21

\begin{itemize}
\item
more complete structuring introduced
\item
|\childdocof| introduced
\item
|\childdoc| renamed to |\childdocmain|
\item
|\childredirect| renamed to |\childdocforward| and |\childdocforwardprefix|
and functionality expanded
\end{itemize}

%%%%%%%%%%%%%%%%%%%%%%%%%%%%%%%%%%%%%%%%
\paragraph{v1.0:} 2017/04/27

\begin{itemize}
\item
manual and install package
\item
first version published on CTAN
\end{itemize}

%%%%%%%%%%%%%%%%%%%%%%%%%%%%%%%%%%%%%%%%
\paragraph{v0.6:} 2017/04/26

\begin{itemize}
\item
redirection mechanism added
\end{itemize}

%%%%%%%%%%%%%%%%%%%%%%%%%%%%%%%%%%%%%%%%
\paragraph{v0.5:} 2017/04/26

\begin{itemize}
\item
functionality in definition file
\end{itemize}


%%%%%%%%%%%%%%%%%%%%%%%%%%%%%%%%%%%%%%%%%%%%%%%%%%%%%%%%%%%%%%%%%%%%%%%%%%%%%%%%
%%%%%%%%%%%%%%%%%%%%%%%%%%%%%%%%%%%%%%%%%%%%%%%%%%%%%%%%%%%%%%%%%%%%%%%%%%%%%%%%
%%%%%%%%%%%%%%%%%%%%%%%%%%%%%%%%%%%%%%%%%%%%%%%%%%%%%%%%%%%%%%%%%%%%%%%%%%%%%%%%
\appendix

\settowidth\MacroIndent{\rmfamily\scriptsize 000\ }

 \DocInput{childdoc.dtx}

\end{document}
%</driver>
% \fi
%
% %%%%%%%%%%%%%%%%%%%%%%%%%%%%%%%%%%%%%%%%%%%%%%%%%%%%%%%%%%%%%%%%%%%%%%%%%%%%%%
% %%%%%%%%%%%%%%%%%%%%%%%%%%%%%%%%%%%%%%%%%%%%%%%%%%%%%%%%%%%%%%%%%%%%%%%%%%%%%%
% \section{Sample}
%\iffalse
%<*samplemain>
%\fi
%
% The following presents a sample document
% with two chapters, two parts, a title page,
% a compile flag as well as three forwarding files to set the flag.
% It consists of eight |.tex| files:
% \begin{center}
% \begin{tabular}{ll}
% |cdocsamp.tex|&main file\\
% |cdocsch1.tex|&include file for chapter 1\\
% |cdocsch2.tex|&include file for chapter 2\\
% |cdocspt3.tex|&include file for part 3\\
% |cdocspt4.tex|&include file for part 4\\
% |cdocsdrf.tex|&forwarding file for main file in draft mode\\
% |cdocsfi1.tex|&forwarding file for final version of chapter 1\\
% |cdocsfi2.tex|&forwarding file for final version of chapter 2\\
% \end{tabular}
% \end{center}
% Each of the eight files can be compiled directly by the \LaTeX{} compiler.
%
% %%%%%%%%%%%%%%%%%%%%%%%%%%%%%%%%%%%%%%
% \paragraph{Main File.}
%
% The main file is called |cdocsamp.tex|.
%
% Load the \textsf{childdoc} definitions and
% declare the filename for the main document:
%    \begin{macrocode}
\input{childdoc.def}
\childdocmain{}
%    \end{macrocode}

% Optional override for |\version| flag:
%    \begin{macrocode}
%%\ifchilddoc\else\providecommand{\version}{draft}\fi
%    \end{macrocode}

% Define the default values for the |\version| flag
% (|final| for the main file and |draft| for childs):
%    \begin{macrocode}
\ifchilddoc
\providecommand{\version}{draft}
\else
\providecommand{\version}{final}
\fi
%    \end{macrocode}

% Load the standard document class:
%    \begin{macrocode}
\documentclass[12pt]{article}
%    \end{macrocode}

% Start the document body:
%    \begin{macrocode}
\begin{document}
%    \end{macrocode}

% Declare a title page.
% Print title, part of document being processed and version flag:
%    \begin{macrocode}
\addtocounter{page}{-1}
\begin{center}
{\LARGE\bfseries{}childdoc example\par}
\vspace{1cm}
\ifchilddoc
\ifchilddocmanual part\else chapter\fi:
`\childdocname' of `\childdocjob'\par
\else
main document: `\childdocjob'\par
\fi
version: \version\par
\end{center}
\newpage
%    \end{macrocode}

% Manually include selected file,
% otherwise process as usual:
%    \begin{macrocode}
\ifchilddocmanual
\section*{part `\childdocname'}
\input{\childdocname}
\else
%    \end{macrocode}

% Include the two chapters:
%    \begin{macrocode}
\include{cdocsch1}
\include{cdocsch2}
%    \end{macrocode}

% Include the two parts unless only chapters should be displayed:
%    \begin{macrocode}
\ifchilddoc\else
\section{part three}
\input{cdocspt3}
\section{part four}
\input{cdocspt4}
\fi
%    \end{macrocode}

% Process as usual until here:
%    \begin{macrocode}
\fi
%    \end{macrocode}

% End of document body:
%    \begin{macrocode}
\end{document}
%    \end{macrocode}
%\iffalse
%</samplemain>
%\fi
%
% %%%%%%%%%%%%%%%%%%%%%%%%%%%%%%%%%%%%%%
% \paragraph{Chapter Include Files.}
%
% The include files are called |cdocsch1.tex| and |cdocsch2.tex|.
%
%\iffalse
%<*samplechap1|samplechap2>
%\fi

% Optional override for |\version| flag:
%    \begin{macrocode}
%%\providecommand{\version}{final}
%    \end{macrocode}

% Include the main document:
%    \begin{macrocode}
\input{childdoc.def}
\childdocof{cdocsamp}
%    \end{macrocode}

%\iffalse
%</samplechap1|samplechap2>
%\fi
%
%\iffalse
%<*samplechap1>
%\fi
% Some text for chapter 1:
%    \begin{macrocode}
\section{one}
some text in chapter one
%    \end{macrocode}

%\iffalse
%</samplechap1>
%\fi
% Some text for chapter 2:
%\iffalse
%<*samplechap2>
%\fi
%    \begin{macrocode}
\section{two}
more text in chapter two
%    \end{macrocode}

%\iffalse
%</samplechap2>
%\fi
%
% %%%%%%%%%%%%%%%%%%%%%%%%%%%%%%%%%%%%%%
% \paragraph{Part Include Files.}
%
% The include files are called |cdocspt3.tex| and |cdocspt4.tex|.
%
%\iffalse
%<*samplepart3|samplepart4>
%\fi

% Optional override for |\version| flag:
%    \begin{macrocode}
%%\providecommand{\version}{final}
%    \end{macrocode}

% Include the main document:
%    \begin{macrocode}
\input{childdoc.def}
\childdocby{cdocsamp}
%    \end{macrocode}

%\iffalse
%</samplepart3|samplepart4>
%\fi
%
%\iffalse
%<*samplepart3>
%\fi
% Some text for part 3:
%    \begin{macrocode}
some text in part three
%    \end{macrocode}

%\iffalse
%</samplepart3>
%\fi
% Some text for part 4:
%\iffalse
%<*samplepart4>
%\fi
%    \begin{macrocode}
more text in part four
%    \end{macrocode}

%\iffalse
%</samplepart4>
%\fi
%
% %%%%%%%%%%%%%%%%%%%%%%%%%%%%%%%%%%%%%%
% \paragraph{Forwarding for a Complete Draft.}
%
% The following forwarding file |cdocsdrf.tex|
% compiles the main document in draft mode:
%\iffalse
%<*sampledraft>
%\fi
%    \begin{macrocode}
\def\version{draft}
\input{childdoc.def}
\childdocforward{cdocsamp}
%    \end{macrocode}

%\iffalse
%</sampledraft>
%\fi
%
% %%%%%%%%%%%%%%%%%%%%%%%%%%%%%%%%%%%%%%
% \paragraph{Forwarding for Final Version of the Chapters.}
%
% The following forwarding files |cdocsfn1.tex| and |cdocsfn2.tex|
% (with identical content)
% compile the final versions of the child documents
% |cdocsch1.tex| and |cdocsch2.tex|, respectively:
%\iffalse
%<*samplefinal>
%\fi
%    \begin{macrocode}
\def\version{final}
\input{childdoc.def}
\childdocforwardprefix[cdocsamp]{cdocsfn}{cdocsch}
%    \end{macrocode}

%\iffalse
%</samplefinal>
%\fi
%
% %%%%%%%%%%%%%%%%%%%%%%%%%%%%%%%%%%%%%%
% \paragraph{Command Line Processing.}
%
% The following three command lines generate the output files
% |cdocscld|, |cdocscl1| and |cdocscl2|
% which should be identical to
% |cdocsdrf|, |cdocsch1| and |cdocsfn2|, respectively:
% \begin{center}
% \begin{tabular}{l}
% |latex -jobname cdocscld \|\\
% |  "\def\version{draft}\input{childdoc.def}\childdocforward{cdocsamp}"|\\
% |latex -jobname cdocscl1 \|\\
% |  "\input{childdoc.def}\childdocforward[cdocsamp]{cdocsch1}"|\\
% |latex -jobname cdocscl2 \|\\
% |  "\def\version{final}\input{childdoc.def}\childdocforward{cdocsch2}"|
% \end{tabular}
% \end{center}
% Note that the trailing backslash on each first line
% merely continues the input to the second line
% (for convenient cut ant paste).
% Furthermore, the command |latex| can be replaced by any
% of its alternative versions such as |pdflatex|.
%
% %%%%%%%%%%%%%%%%%%%%%%%%%%%%%%%%%%%%%%%%%%%%%%%%%%%%%%%%%%%%%%%%%%%%%%%%%%%%%%
% %%%%%%%%%%%%%%%%%%%%%%%%%%%%%%%%%%%%%%%%%%%%%%%%%%%%%%%%%%%%%%%%%%%%%%%%%%%%%%
% \section{Implementation}
%\iffalse
%<*package>
%\fi
%
% This section describes the definitions file |childdoc.def|.

% The definitions cannot be loaded using |\usepackage| or |\RequirePackage|
% which has a mechanism to prevent loading a style file more than once.
% When loading the definitions by means of |\input|
% multiple instances have to be prevented manually:
%\iffalse
%This code needs to be before the `\ProvidesFile' directive
%which is defined at the beginning of this file.
%Therefore it is also placed there and commented out here.
%</package>
%<*discard>
%\fi
%    \begin{macrocode}
\ifdefined\childdocmain\endinput\fi
%    \end{macrocode}
%\iffalse
%</discard>
%<*package>
%\fi
%
% \macro{\ifchilddoc}
% \macro{\ifchilddocmanual}
% The conditional |\ifchilddoc| tells whether a
% child (true) or main (false) document is being compiled.
% The conditional |\ifchilddocmanual| tells whether
% the |\includeonly| mechanism is used (false) or
% the selection of child files must be performed manually (true).
% The definitions initialise to false:
%    \begin{macrocode}
\newif\ifchilddoc
\newif\ifchilddocmanual
%    \end{macrocode}

% \macro{\childdocname}
% \macro{\childdocjob}
% The macro |\childdocname| stores the name of the main document
% to be compiled. The macro |\childdocjob| stores the name of
% the document on which the \LaTeX{} compiler was originally invoked.
% The content of |\jobname| cannot be compared
% to filenames specified in the source due to different catcodes.
% The following code rescans |\jobname|, stores the result
% in |\childdocname| and saves a copy in |\childdocjob|:
%    \begin{macrocode}
\edef\childdocname{\scantokens\expandafter{\jobname\noexpand}}
\let\childdocjob\childdocname
%    \end{macrocode}

% \macro{\childdocdisable}
% The macro |\childdocdisable| prevents the main file
% from being processed more than once.
% At this stage, the main document command |\childdocmain|
% is assumed to be called once again where it should do nothing.
% Any subsequent call to it should prevent
% a secondary processing of the main document
% It overwrites the forwarding commands
% |\childdocof| and |\childdocforward|
% with empty macros to prevent further inclusions of the main document:
%    \begin{macrocode}
\newcommand{\childdocdisable}
{
  \renewcommand{\childdocmain}[1]{\renewcommand{\childdocmain}[1]{\endinput}}
  \renewcommand{\childdocof}[1]{}
  \renewcommand{\childdocby}[2][]{}
  \renewcommand{\childdocforward}[2][]{}
  \renewcommand{\childdocdisable}{}
}
%    \end{macrocode}

% \macro{\childdocmain}
% The macro |\childdocmain| is to be called at the top of the main file
% with nothing or the main filename (without extension) as argument.
% First, it breaks loops.
% If the argument is not empty and does not match |\childdocname|
% (which is set by the first inclusion of |childdoc.def|),
% |\ifchilddoc| is set to true, |\includeonly| is applied to the child file
% and |\jobname| is set to the main file
% (for proper handling of |.aux| files):
%    \begin{macrocode}
\newcommand{\childdocmain}[1]
{
  \childdocdisable\childdocmain{}
  \if?#1?\else
    \begingroup
      \def\childdoctmp{#1}
      \ifx\childdoctmp\childdocname
        \def\childdoctmp{}
      \else
        \def\childdoctmp
        {
          \childdoctrue
          \includeonly{\childdocname}
          \def\childdocjob{#1}
          \def\jobname{#1}
        }
      \fi
      \expandafter
    \endgroup
    \childdoctmp
  \fi
}
%    \end{macrocode}

% \macro{\childdocof}
% The command |\childdocof| redirects
% compilation to the main file |#1|.
%    \begin{macrocode}
\newcommand{\childdocof}[1]
{
  \childdocdisable
  \childdoctrue
  \includeonly{\childdocname}
  \def\jobname{#1}
  \def\childdocjob{#1}
  \input{#1}
}
%    \end{macrocode}

% \macro{\childdocby}
% The command |\childdocby| ....
%    \begin{macrocode}
\newcommand{\childdocby}[2][]
{
  \childdocdisable
  \childdoctrue
  \childdocmanualtrue
  \if?#1?\else
    \def\jobname{#2}
  \fi
  \def\childdocjob{#2}
  \input{#2}
  \endinput
}
%    \end{macrocode}

% \macro{\childdocforward}
% The command |\childdocforward| redirects
% compilation to the main file or
% (if the optional argument is given) a child file.
% Parameters are set as if the main file
% or a child file starting with |\childdocof| was compiled.
% Then compilation is handed over to the main file:
%    \begin{macrocode}
\newcommand{\childdocforward}[2][]
{
  \begingroup
    \if?#1?
      \def\childdoctmp
      {
        \def\childdocname{#2}
        \def\childdocjob{#2}
        \def\jobname{#2}
        \input{#2}
        \endinput
      }
    \else
      \def\childdoctmp
      {
        \childdocdisable
        \def\childdocname{#2}
        \childdoctrue
        \includeonly{#2}
        \def\childdocjob{#1}
        \def\jobname{#1}
        \input{#1}
        \endinput
      }
    \fi
    \expandafter
  \endgroup
  \childdoctmp
}
%    \end{macrocode}

% \macro{\childdocforwardprefix}
% The command |\childdocforwardprefix| redirects
% compilation to the main or a child file by means of a pattern.
% The prefix |#1| in the current filename is replaced by |#2|
% and the suffix of the current filename is kept
% (it is assumed that the filename does not contain the substring `|~~~|'
% which is used as a delimiter).
% Compilation is handed over to the new file by |\childdocforward|:
%    \begin{macrocode}
\newcommand{\childdocforwardprefix}[3][]
{
  \begingroup
    \def\childdocextract #2##1~~~{\def\childdoctmp{\childdocforward[#1]{#3##1}}}
    \expandafter\childdocextract\childdocname~~~
    \expandafter
  \endgroup
  \childdoctmp
}
%    \end{macrocode}

% \macro{\childdoc}
% The deprecated macro |\childdoc| is a legacy version of |\childdocmain|:
%    \begin{macrocode}
\newcommand{\childdoc}{\childdocmain}
%    \end{macrocode}

% \macro{\childdocredirect}
% The deprecated macro |\childdocredirect| is a legacy version
% of |\childdocforward| and |\childdocforwardprefix|:
%    \begin{macrocode}
\newcommand{\childdocredirect}[2][]
{
  \begingroup
    \if?#1?
      \def\childdoctmp{\childdocforward{#2}}
    \else
      \def\childdoctmp{\childdocforwardprefix{#1}{#2}}
    \fi
    \expandafter
  \endgroup
  \childdoctmp
}
%    \end{macrocode}

%\iffalse
%</package>
%\fi
%
\endinput
\childdocforward{cdocsch2}"|
% \end{tabular}
% \end{center}
% Note that the trailing backslash on each first line
% merely continues the input to the second line
% (for convenient cut ant paste).
% Furthermore, the command |latex| can be replaced by any
% of its alternative versions such as |pdflatex|.
%
% %%%%%%%%%%%%%%%%%%%%%%%%%%%%%%%%%%%%%%%%%%%%%%%%%%%%%%%%%%%%%%%%%%%%%%%%%%%%%%
% %%%%%%%%%%%%%%%%%%%%%%%%%%%%%%%%%%%%%%%%%%%%%%%%%%%%%%%%%%%%%%%%%%%%%%%%%%%%%%
% \section{Implementation}
%\iffalse
%<*package>
%\fi
%
% This section describes the definitions file |childdoc.def|.

% The definitions cannot be loaded using |\usepackage| or |\RequirePackage|
% which has a mechanism to prevent loading a style file more than once.
% When loading the definitions by means of |\input|
% multiple instances have to be prevented manually:
%\iffalse
%This code needs to be before the `\ProvidesFile' directive
%which is defined at the beginning of this file.
%Therefore it is also placed there and commented out here.
%</package>
%<*discard>
%\fi
%    \begin{macrocode}
\ifdefined\childdocmain\endinput\fi
%    \end{macrocode}
%\iffalse
%</discard>
%<*package>
%\fi
%
% \macro{\ifchilddoc}
% \macro{\ifchilddocmanual}
% The conditional |\ifchilddoc| tells whether a
% child (true) or main (false) document is being compiled.
% The conditional |\ifchilddocmanual| tells whether
% the |\includeonly| mechanism is used (false) or
% the selection of child files must be performed manually (true).
% The definitions initialise to false:
%    \begin{macrocode}
\newif\ifchilddoc
\newif\ifchilddocmanual
%    \end{macrocode}

% \macro{\childdocname}
% \macro{\childdocjob}
% The macro |\childdocname| stores the name of the main document
% to be compiled. The macro |\childdocjob| stores the name of
% the document on which the \LaTeX{} compiler was originally invoked.
% The content of |\jobname| cannot be compared
% to filenames specified in the source due to different catcodes.
% The following code rescans |\jobname|, stores the result
% in |\childdocname| and saves a copy in |\childdocjob|:
%    \begin{macrocode}
\edef\childdocname{\scantokens\expandafter{\jobname\noexpand}}
\let\childdocjob\childdocname
%    \end{macrocode}

% \macro{\childdocdisable}
% The macro |\childdocdisable| prevents the main file
% from being processed more than once.
% At this stage, the main document command |\childdocmain|
% is assumed to be called once again where it should do nothing.
% Any subsequent call to it should prevent
% a secondary processing of the main document
% It overwrites the forwarding commands
% |\childdocof| and |\childdocforward|
% with empty macros to prevent further inclusions of the main document:
%    \begin{macrocode}
\newcommand{\childdocdisable}
{
  \renewcommand{\childdocmain}[1]{\renewcommand{\childdocmain}[1]{\endinput}}
  \renewcommand{\childdocof}[1]{}
  \renewcommand{\childdocby}[2][]{}
  \renewcommand{\childdocforward}[2][]{}
  \renewcommand{\childdocdisable}{}
}
%    \end{macrocode}

% \macro{\childdocmain}
% The macro |\childdocmain| is to be called at the top of the main file
% with nothing or the main filename (without extension) as argument.
% First, it breaks loops.
% If the argument is not empty and does not match |\childdocname|
% (which is set by the first inclusion of |childdoc.def|),
% |\ifchilddoc| is set to true, |\includeonly| is applied to the child file
% and |\jobname| is set to the main file
% (for proper handling of |.aux| files):
%    \begin{macrocode}
\newcommand{\childdocmain}[1]
{
  \childdocdisable\childdocmain{}
  \if?#1?\else
    \begingroup
      \def\childdoctmp{#1}
      \ifx\childdoctmp\childdocname
        \def\childdoctmp{}
      \else
        \def\childdoctmp
        {
          \childdoctrue
          \includeonly{\childdocname}
          \def\childdocjob{#1}
          \def\jobname{#1}
        }
      \fi
      \expandafter
    \endgroup
    \childdoctmp
  \fi
}
%    \end{macrocode}

% \macro{\childdocof}
% The command |\childdocof| redirects
% compilation to the main file |#1|.
%    \begin{macrocode}
\newcommand{\childdocof}[1]
{
  \childdocdisable
  \childdoctrue
  \includeonly{\childdocname}
  \def\jobname{#1}
  \def\childdocjob{#1}
  \input{#1}
}
%    \end{macrocode}

% \macro{\childdocby}
% The command |\childdocby| ....
%    \begin{macrocode}
\newcommand{\childdocby}[2][]
{
  \childdocdisable
  \childdoctrue
  \childdocmanualtrue
  \if?#1?\else
    \def\jobname{#2}
  \fi
  \def\childdocjob{#2}
  \input{#2}
  \endinput
}
%    \end{macrocode}

% \macro{\childdocforward}
% The command |\childdocforward| redirects
% compilation to the main file or
% (if the optional argument is given) a child file.
% Parameters are set as if the main file
% or a child file starting with |\childdocof| was compiled.
% Then compilation is handed over to the main file:
%    \begin{macrocode}
\newcommand{\childdocforward}[2][]
{
  \begingroup
    \if?#1?
      \def\childdoctmp
      {
        \def\childdocname{#2}
        \def\childdocjob{#2}
        \def\jobname{#2}
        \input{#2}
        \endinput
      }
    \else
      \def\childdoctmp
      {
        \childdocdisable
        \def\childdocname{#2}
        \childdoctrue
        \includeonly{#2}
        \def\childdocjob{#1}
        \def\jobname{#1}
        \input{#1}
        \endinput
      }
    \fi
    \expandafter
  \endgroup
  \childdoctmp
}
%    \end{macrocode}

% \macro{\childdocforwardprefix}
% The command |\childdocforwardprefix| redirects
% compilation to the main or a child file by means of a pattern.
% The prefix |#1| in the current filename is replaced by |#2|
% and the suffix of the current filename is kept
% (it is assumed that the filename does not contain the substring `|~~~|'
% which is used as a delimiter).
% Compilation is handed over to the new file by |\childdocforward|:
%    \begin{macrocode}
\newcommand{\childdocforwardprefix}[3][]
{
  \begingroup
    \def\childdocextract #2##1~~~{\def\childdoctmp{\childdocforward[#1]{#3##1}}}
    \expandafter\childdocextract\childdocname~~~
    \expandafter
  \endgroup
  \childdoctmp
}
%    \end{macrocode}

% \macro{\childdoc}
% The deprecated macro |\childdoc| is a legacy version of |\childdocmain|:
%    \begin{macrocode}
\newcommand{\childdoc}{\childdocmain}
%    \end{macrocode}

% \macro{\childdocredirect}
% The deprecated macro |\childdocredirect| is a legacy version
% of |\childdocforward| and |\childdocforwardprefix|:
%    \begin{macrocode}
\newcommand{\childdocredirect}[2][]
{
  \begingroup
    \if?#1?
      \def\childdoctmp{\childdocforward{#2}}
    \else
      \def\childdoctmp{\childdocforwardprefix{#1}{#2}}
    \fi
    \expandafter
  \endgroup
  \childdoctmp
}
%    \end{macrocode}

%\iffalse
%</package>
%\fi
%
\endinput
|\\
|\childdocforward{|\textit{main}|}|
\end{tabular}
\end{center}
%
Likewise, the following files |final|\textit{nn}|.tex|
compile the final version of the child document
|child|\textit{nn}|.tex|:
%
\begin{center}
\begin{tabular}{l}
|\def\version{final}|\\
|% \iffalse
%
% childdoc.dtx Copyright (C) 2017-2018 Niklas Beisert
%
% This work may be distributed and/or modified under the
% conditions of the LaTeX Project Public License, either version 1.3
% of this license or (at your option) any later version.
% The latest version of this license is in
%   http://www.latex-project.org/lppl.txt
% and version 1.3 or later is part of all distributions of LaTeX
% version 2005/12/01 or later.
%
% This work has the LPPL maintenance status `maintained'.
%
% The Current Maintainer of this work is Niklas Beisert.
%
% This work consists of the files childdoc.dtx and childdoc.ins
% and the derived files childdoc.def and cdocsamp.tex with
% cdocsch1.tex, cdocsch2.tex, cdocsdrf.tex, cdocsfn1.tex, cdocsfn2.tex.
%
%<package>\ifdefined\childdocmain\endinput\fi
%<package>\ProvidesFile{childdoc.def}[2018/12/30 v2.0 child document driver]
%<samplemain>\ProvidesFile{cdocsamp.tex}[2018/12/30 v2.0 sample for childdoc]
%<*driver>
%\ProvidesFile{childdoc.drv}[2018/12/30 v2.0 childdoc reference manual file]
\PassOptionsToClass{10pt,a4paper}{article}
\documentclass{ltxdoc}

\usepackage[margin=35mm]{geometry}
\usepackage{hyperref}
\usepackage{hyperxmp}
\usepackage[usenames]{color}

\hypersetup{colorlinks=true}
\hypersetup{pdfstartview=FitH}
\hypersetup{pdfpagemode=UseNone}
\hypersetup{pdfsource={}}
\hypersetup{pdflang={en-UK}}
\hypersetup{pdfcopyright={Copyright 2017-2018 Niklas Beisert.
  This work may be distributed and/or modified under the
  conditions of the LaTeX Project Public License, either version 1.3
  of this license or (at your option) any later version.}}
\hypersetup{pdflicenseurl={http://www.latex-project.org/lppl.txt}}
\hypersetup{pdfcontactaddress={ETH Zurich, ITP, HIT K,
  Wolfgang-Pauli-Strasse 27}}
\hypersetup{pdfcontactpostcode={8093}}
\hypersetup{pdfcontactcity={Zurich}}
\hypersetup{pdfcontactcountry={Switzerland}}
\hypersetup{pdfcontactemail={nbeisert@itp.phys.ethz.ch}}
\hypersetup{pdfcontacturl={http://people.phys.ethz.ch/\xmptilde nbeisert/}}

\newcommand{\secref}[1]{\hyperref[#1]{section \ref*{#1}}}

\parskip1ex
\parindent0pt
\let\olditemize\itemize
\def\itemize{\olditemize\parskip0pt}

\begin{document}

\title{The \textsf{childdoc} Package}
\hypersetup{pdftitle={The childdoc Package}}
\author{Niklas Beisert\\[2ex]
  Institut f\"ur Theoretische Physik\\
  Eidgen\"ossische Technische Hochschule Z\"urich\\
  Wolfgang-Pauli-Strasse 27, 8093 Z\"urich, Switzerland\\[1ex]
  \href{mailto:nbeisert@itp.phys.ethz.ch}
  {\texttt{nbeisert@itp.phys.ethz.ch}}}
\hypersetup{pdfauthor={Niklas Beisert}}
\hypersetup{pdfsubject={Manual for the LaTeX2e Package childdoc}}
\date{30 December 2018, \textsf{v2.0}}
\maketitle

\begin{abstract}\noindent
\textsf{childdoc} is a \LaTeXe{} package
that enables the direct compilation
of document sections included by |\include|
to individual files.
\end{abstract}

\begingroup
\parskip0ex
\tableofcontents
\endgroup

%%%%%%%%%%%%%%%%%%%%%%%%%%%%%%%%%%%%%%%%%%%%%%%%%%%%%%%%%%%%%%%%%%%%%%%%%%%%%%%%
%%%%%%%%%%%%%%%%%%%%%%%%%%%%%%%%%%%%%%%%%%%%%%%%%%%%%%%%%%%%%%%%%%%%%%%%%%%%%%%%
\section{Introduction}

\LaTeX{} provides a mechanism to structure a large document (such as a book)
into a main file and several child files (containing the chapters)
using the |\include| command.
This mechanism is beneficial for documents
which span hundreds of pages in order to
make the source file(s) more manageable.
Moreover, compilation can be restricted to
selected child files by means of the |\includeonly| command.
The latter feature can be used to reduce the compilation time while editing
(this was significantly more useful in the earlier days of \LaTeX{})
or to generate a smaller document which is easier to navigate.
Another application of |\includeonly| is to generate
documents consisting of selected parts of the complete document.

However, there are a few drawbacks of the plain |\include| mechanism:
\begin{itemize}
\item
The child files cannot be compiled on their own,
they can only be compiled via the main file.
A naive editing environment
(such as a text editor with an option
to have the current file processed by \LaTeX)
may require one to switch to the main file before compiling;
attempting to compile the child file produces errors.
\item
The main file must be modified (each time)
to adjust the |\includeonly| command
to the present needs. This easily leaves the main file in a messy state.
\item
The generated document will always carry the filename
of the main document. This is inconvenient if
several child files are to be compiled and
to be kept for distribution.
\end{itemize}

The present package provides a simple interface
to make child files individually compilable by \LaTeX{}.
Compiling a child file then has the same effect as compiling
the main file with an |\includeonly| command
to select the appropriate child.
Moreover the generated document will carry the name of the child
rather than the main file.
This resolves all three above issues.

This feature is meant to make the editing of books,
thesis documents and lecture notes somewhat more convenient.
However, the package can also be used efficiently for
composing a series of documents (such as exercise sheets)
which are typically distributed individually.
It then assists the author in generating the individual documents
(potentially in different versions)
as well as a document containing the collected series.
Another application is in developing style files
or other kinds of included material
where compilation of the style file could redirect
to a sample or test file.

%%%%%%%%%%%%%%%%%%%%%%%%%%%%%%%%%%%%%%%%%%%%%%%%%%%%%%%%%%%%%%%%%%%%%%%%%%%%%%%%
%%%%%%%%%%%%%%%%%%%%%%%%%%%%%%%%%%%%%%%%%%%%%%%%%%%%%%%%%%%%%%%%%%%%%%%%%%%%%%%%
\section{Usage}

First of all, the package \textsf{childdoc} is \emph{not} a standard
\LaTeXe{} |.sty| style file! Therefore it needs to be invoked in
a non-standard way.

%%%%%%%%%%%%%%%%%%%%%%%%%%%%%%%%%%%%%%%%%%%%%%%%%%%%%%%%%%%%%%%%%%%%%%%%%%%%%%%%
\subsection{Included Files}
\label{sec:include}

%%%%%%%%%%%%%%%%%%%%%%%%%%%%%%%%%%%%%%%%
\DescribeMacro{\childdocmain}
To use the package, add the commands
\begin{center}
\begin{tabular}{l}
|% \iffalse
%
% childdoc.dtx Copyright (C) 2017-2018 Niklas Beisert
%
% This work may be distributed and/or modified under the
% conditions of the LaTeX Project Public License, either version 1.3
% of this license or (at your option) any later version.
% The latest version of this license is in
%   http://www.latex-project.org/lppl.txt
% and version 1.3 or later is part of all distributions of LaTeX
% version 2005/12/01 or later.
%
% This work has the LPPL maintenance status `maintained'.
%
% The Current Maintainer of this work is Niklas Beisert.
%
% This work consists of the files childdoc.dtx and childdoc.ins
% and the derived files childdoc.def and cdocsamp.tex with
% cdocsch1.tex, cdocsch2.tex, cdocsdrf.tex, cdocsfn1.tex, cdocsfn2.tex.
%
%<package>\ifdefined\childdocmain\endinput\fi
%<package>\ProvidesFile{childdoc.def}[2018/12/30 v2.0 child document driver]
%<samplemain>\ProvidesFile{cdocsamp.tex}[2018/12/30 v2.0 sample for childdoc]
%<*driver>
%\ProvidesFile{childdoc.drv}[2018/12/30 v2.0 childdoc reference manual file]
\PassOptionsToClass{10pt,a4paper}{article}
\documentclass{ltxdoc}

\usepackage[margin=35mm]{geometry}
\usepackage{hyperref}
\usepackage{hyperxmp}
\usepackage[usenames]{color}

\hypersetup{colorlinks=true}
\hypersetup{pdfstartview=FitH}
\hypersetup{pdfpagemode=UseNone}
\hypersetup{pdfsource={}}
\hypersetup{pdflang={en-UK}}
\hypersetup{pdfcopyright={Copyright 2017-2018 Niklas Beisert.
  This work may be distributed and/or modified under the
  conditions of the LaTeX Project Public License, either version 1.3
  of this license or (at your option) any later version.}}
\hypersetup{pdflicenseurl={http://www.latex-project.org/lppl.txt}}
\hypersetup{pdfcontactaddress={ETH Zurich, ITP, HIT K,
  Wolfgang-Pauli-Strasse 27}}
\hypersetup{pdfcontactpostcode={8093}}
\hypersetup{pdfcontactcity={Zurich}}
\hypersetup{pdfcontactcountry={Switzerland}}
\hypersetup{pdfcontactemail={nbeisert@itp.phys.ethz.ch}}
\hypersetup{pdfcontacturl={http://people.phys.ethz.ch/\xmptilde nbeisert/}}

\newcommand{\secref}[1]{\hyperref[#1]{section \ref*{#1}}}

\parskip1ex
\parindent0pt
\let\olditemize\itemize
\def\itemize{\olditemize\parskip0pt}

\begin{document}

\title{The \textsf{childdoc} Package}
\hypersetup{pdftitle={The childdoc Package}}
\author{Niklas Beisert\\[2ex]
  Institut f\"ur Theoretische Physik\\
  Eidgen\"ossische Technische Hochschule Z\"urich\\
  Wolfgang-Pauli-Strasse 27, 8093 Z\"urich, Switzerland\\[1ex]
  \href{mailto:nbeisert@itp.phys.ethz.ch}
  {\texttt{nbeisert@itp.phys.ethz.ch}}}
\hypersetup{pdfauthor={Niklas Beisert}}
\hypersetup{pdfsubject={Manual for the LaTeX2e Package childdoc}}
\date{30 December 2018, \textsf{v2.0}}
\maketitle

\begin{abstract}\noindent
\textsf{childdoc} is a \LaTeXe{} package
that enables the direct compilation
of document sections included by |\include|
to individual files.
\end{abstract}

\begingroup
\parskip0ex
\tableofcontents
\endgroup

%%%%%%%%%%%%%%%%%%%%%%%%%%%%%%%%%%%%%%%%%%%%%%%%%%%%%%%%%%%%%%%%%%%%%%%%%%%%%%%%
%%%%%%%%%%%%%%%%%%%%%%%%%%%%%%%%%%%%%%%%%%%%%%%%%%%%%%%%%%%%%%%%%%%%%%%%%%%%%%%%
\section{Introduction}

\LaTeX{} provides a mechanism to structure a large document (such as a book)
into a main file and several child files (containing the chapters)
using the |\include| command.
This mechanism is beneficial for documents
which span hundreds of pages in order to
make the source file(s) more manageable.
Moreover, compilation can be restricted to
selected child files by means of the |\includeonly| command.
The latter feature can be used to reduce the compilation time while editing
(this was significantly more useful in the earlier days of \LaTeX{})
or to generate a smaller document which is easier to navigate.
Another application of |\includeonly| is to generate
documents consisting of selected parts of the complete document.

However, there are a few drawbacks of the plain |\include| mechanism:
\begin{itemize}
\item
The child files cannot be compiled on their own,
they can only be compiled via the main file.
A naive editing environment
(such as a text editor with an option
to have the current file processed by \LaTeX)
may require one to switch to the main file before compiling;
attempting to compile the child file produces errors.
\item
The main file must be modified (each time)
to adjust the |\includeonly| command
to the present needs. This easily leaves the main file in a messy state.
\item
The generated document will always carry the filename
of the main document. This is inconvenient if
several child files are to be compiled and
to be kept for distribution.
\end{itemize}

The present package provides a simple interface
to make child files individually compilable by \LaTeX{}.
Compiling a child file then has the same effect as compiling
the main file with an |\includeonly| command
to select the appropriate child.
Moreover the generated document will carry the name of the child
rather than the main file.
This resolves all three above issues.

This feature is meant to make the editing of books,
thesis documents and lecture notes somewhat more convenient.
However, the package can also be used efficiently for
composing a series of documents (such as exercise sheets)
which are typically distributed individually.
It then assists the author in generating the individual documents
(potentially in different versions)
as well as a document containing the collected series.
Another application is in developing style files
or other kinds of included material
where compilation of the style file could redirect
to a sample or test file.

%%%%%%%%%%%%%%%%%%%%%%%%%%%%%%%%%%%%%%%%%%%%%%%%%%%%%%%%%%%%%%%%%%%%%%%%%%%%%%%%
%%%%%%%%%%%%%%%%%%%%%%%%%%%%%%%%%%%%%%%%%%%%%%%%%%%%%%%%%%%%%%%%%%%%%%%%%%%%%%%%
\section{Usage}

First of all, the package \textsf{childdoc} is \emph{not} a standard
\LaTeXe{} |.sty| style file! Therefore it needs to be invoked in
a non-standard way.

%%%%%%%%%%%%%%%%%%%%%%%%%%%%%%%%%%%%%%%%%%%%%%%%%%%%%%%%%%%%%%%%%%%%%%%%%%%%%%%%
\subsection{Included Files}
\label{sec:include}

%%%%%%%%%%%%%%%%%%%%%%%%%%%%%%%%%%%%%%%%
\DescribeMacro{\childdocmain}
To use the package, add the commands
\begin{center}
\begin{tabular}{l}
|\input{childdoc.def}|\\
|\childdocmain{}|\\
\end{tabular}
\end{center}
at the very top of the main \LaTeX{} file,
in particular \emph{before} the |\documentclass| statement!
The argument of |\childdocmain| should be left empty
(but it must be present).

%%%%%%%%%%%%%%%%%%%%%%%%%%%%%%%%%%%%%%%%
\DescribeMacro{\childdocof}
Furthermore, add the commands
\begin{center}
\begin{tabular}{l}
|\input{childdoc.def}|\\
|\childdocof{|\textit{main}|}|\\
\end{tabular}
\end{center}
at the top of every child file \textit{child}
which is included by |\include{|\textit{child}|}|
from within the main file
(or at least for those files to be compiled individually).
The argument \textit{main} must be the filename of the main file.

There are a couple of
considerations in setting up the main and child documents:

%%%%%%%%%%%%%%%%%%%%%%%%%%%%%%%%%%%%%%%%
\paragraph{Restrictions.}

Please note the following restrictions:
\begin{itemize}
\item
|\childdocmain| must be called with one argument \textit{main}
to ensure compatibility with earlier version of the package.
It must either be empty (|\childdocmain{}|)
or precisely match the filename of the main file in which it is specified.
See \secref{sec:detection} for further information.
\item
The filename \textit{main} must be specified without the |.tex| extension.
\item
The filename \textit{main} is case sensitive
(even in case-insensitive file systems)
due to internal string comparison.
\item
The argument \textit{main} should be fully expanded, it cannot be a macro.
\item
Subdirectories and special characters should be avoided in filenames.
\item
The command |\childdocmain{|\textit{main}|}| must be followed by a whitespace.
It should not be followed immediately by another command
or by a comment mark `|%|'.
This is because the \TeX{} parser reads the token immediately following
the argument of |\childdocmain| and puts it
at the beginning of every child section;
however, a white\-space is ignored.
\end{itemize}

%%%%%%%%%%%%%%%%%%%%%%%%%%%%%%%%%%%%%%%%
\paragraph{Content of Main File.}

It is advisable to place all content in the child files included by |\include|.
Any output contained in the main file will appear in all child documents
unless suppressed manually;
it cannot be suppressed automatically by the |\includeonly| directive
and thus should normally be avoided.
A method to include some content in the main file
by means of conditional processing is described in \secref{sec:conditional}.

%%%%%%%%%%%%%%%%%%%%%%%%%%%%%%%%%%%%%%%%
\paragraph{Page Numbering.}

When only a part of the document is compiled,
the appropriate numbering of pages
(as well as other status parameters)
is determined from the |.aux| files.
The latter contain information from previous passes.
However this information needs to propagate through
all intermediate child documents.
Therefore the page numbering in child documents may well
be inconsistent until the complete document is compiled at least once.

A useful (if unconventional) way to always ensure a consistent
page numbering is to restart the numbering in each child document
and denote the pages by `\textit{child}|.|\textit{page}'
where \textit{child} represents the chapter/section number of the child file.
This can be achieved by the command
|\numberwithin{page}{|\textit{child}|}|
of the \textsf{amsmath} package
where \textit{child} can be |chapter| or |section|
depending on the chosen structuring.
Alternatively, one can modify the macro |\thepage| appropriately
and reset the counter |page| at the start of each child file.

%%%%%%%%%%%%%%%%%%%%%%%%%%%%%%%%%%%%%%%%%%%%%%%%%%%%%%%%%%%%%%%%%%%%%%%%%%%%%%%%
\subsection{Conditional Processing}
\label{sec:conditional}

The package provides a mechanism to compile different versions
of a document. To customise the versions further some conditional processing
can come in handy to distinguish which version is being compiled.
The package provides two macros to describe the compilation context:

%%%%%%%%%%%%%%%%%%%%%%%%%%%%%%%%%%%%%%%%
\DescribeMacro{\ifchilddoc}
The conditional |\ifchilddoc| distinguishes between the compilation of
child documents and the main document:
%
\begin{center}
|\ifchilddoc |\textit{child-code}| |[|\||else |\textit{main-code}]| \||fi|
\end{center}

%%%%%%%%%%%%%%%%%%%%%%%%%%%%%%%%%%%%%%%%
\DescribeMacro{\childdocname}
\DescribeMacro{\childdocjob}
The macro |\childdocname| contains the filename (without extension)
of the main or child file being processed.
Note that |\childdocjob| will always contain the name of the main file.

%%%%%%%%%%%%%%%%%%%%%%%%%%%%%%%%%%%%%%%%
\paragraph{Title Page.}

Conditional processing can be used to include a title or banner page
in the main document when proper precautions are taken.
Importantly, the code in the main file should ensure that the page counter
(as well as other status parameters which are stored in the |.aux| files)
takes the same value after the conditional processing.
Otherwise the page numbers may take divergent values
depending on which part is compiled.

For example, a title page could be declared by:
%
\begin{center}
\begin{tabular}{l}
|\ifchilddoc\||else|\\
|\addtocounter{page}{-1}|\\
\textit{code for title page}\\
|\newpage|\\
|\||fi|
\end{tabular}
\end{center}
%
A banner page for the child documents can be generated by:
%
\begin{center}
\begin{tabular}{l}
|\ifchilddoc|\\
|\addtocounter{page}{-1}|\\
\textit{code for banner page}\\
|\newpage|\\
|\||fi|
\end{tabular}
\end{center}
%
Here one could write a message such as:
\begin{center}
|This is the part \childdocname{} of \childdocjob{}.|
\end{center}

%%%%%%%%%%%%%%%%%%%%%%%%%%%%%%%%%%%%%%%%%%%%%%%%%%%%%%%%%%%%%%%%%%%%%%%%%%%%%%%%
\subsection{Flags}
\label{sec:flags}

The package makes it easy to generate different versions
of the main or child documents.
To this end compilation flags can be defined
and assigned different default values.
They will be particularly useful in conjunction
with the forwarding mechanism described in \secref{sec:forward}.

For example, it may be useful to have a flag |\version|
which can be set to |draft| or |final|.
The document source will contain some conditional code
depending on the value of |\version|.
Suppose further, the flag should default to |final| for the main file
and to |draft| for child files
which is a natural assignment for editing the document.
This is achieved by placing the following code
in the preamble of the main document
(below the |\childdocmain| directive):
%
\begin{center}
\begin{tabular}{l}
|\ifchilddoc|\\
|\providecommand{\version}{draft}|\\
|\||else|\\
|\providecommand{\version}{final}|\\
|\||fi|
\end{tabular}
\end{center}
%
The definition by |\providecommand| makes sure
that previous definitions are not overwritten.
Further statements |\providecommand{\version}{...}|
can thus be added before the above code to override it.

For the main file, one might add a line
(between |\childdocmain| and the above block)
%
\begin{center}
|%\ifchilddoc\||else\providecommand{\version}{draft}\||fi|
\end{center}
%
which can be uncommented to produce a draft version.
Likewise one can add a line to the very top of a child file
(above the |\childdocof{|\textit{main}|}| directive)
%
\begin{center}
|%\providecommand{\version}{final}|
\end{center}
%
which can be uncommented to produce the final version of this child document.

%%%%%%%%%%%%%%%%%%%%%%%%%%%%%%%%%%%%%%%%%%%%%%%%%%%%%%%%%%%%%%%%%%%%%%%%%%%%%%%%
\subsection{Forwarding}
\label{sec:forward}

Different versions of the main or child documents
using compilation flags as described in \secref{sec:flags}
can be (permanently) stored in different files
for convenient compilation, viewing and distribution.
To this end, the package defines a command
to pass on compilation to a different file:

%%%%%%%%%%%%%%%%%%%%%%%%%%%%%%%%%%%%%%%%
\DescribeMacro{\childdocforward}
The command |\childdocforward| redirects processing to
another source file:
%
\begin{center}
\begin{tabular}{l}
|\input{childdoc.def}|\\
|\childdocforward[|\textit{main}|]{|\textit{dest}|}|\\
\end{tabular}
\end{center}
%
The argument \textit{dest} is the destination file
(without extension).
It should be the main file or one of the child files.
Note that further \textsf{childdoc} directives
such as |\childdocof| and |\childdocforward|
in the indicated file will be processed in this form.
The optional argument \textit{main}
passes on directly to the main file \textit{main}
while pretending to compile the child \textit{dest}.
This form behaves as if \textit{dest}
issues |\childdocof{|\textit{main}|}| right away,
and no further \textsf{childdoc} directives will be processed.

%%%%%%%%%%%%%%%%%%%%%%%%%%%%%%%%%%%%%%%%
\DescribeMacro{\...prefix}
In the alternative form |\childdocforwardprefix|,
%
\begin{center}
\begin{tabular}{l}
|\input{childdoc.def}|\\
|\childdocforwardprefix[|\textit{main}|]{|\textit{prefix}|}{|\textit{dest}|}|
\end{tabular}
\end{center}
%
the destination file is determined by a pattern
depending on the current file:
To make this work, the current file must be called
`{\textit{prefix}\hspace{0.2em}\textit{suffix}}'
with \textit{prefix} matching precisely the argument.
Processing is then passed on to the file
`{\textit{dest}\hspace{0.2em}\textit{suffix}}'.
Surely, the same effect is achieved by
directly specifying the
argument `{\textit{dest}\hspace{0.2em}\textit{suffix}}'
in the first form.
However, that requires to set up a different file
for each child. With the alternative form of the command
all these files can have exactly the same content
which simplifies setting them up and maintaining them.

For example, the following file |draft.tex|
with a compilation flag |\version| as described in \secref{sec:flags}
compiles the main document as a draft:
%
\begin{center}
\begin{tabular}{l}
|\def\version{draft}|\\
|\input{childdoc.def}|\\
|\childdocforward{|\textit{main}|}|
\end{tabular}
\end{center}
%
Likewise, the following files |final|\textit{nn}|.tex|
compile the final version of the child document
|child|\textit{nn}|.tex|:
%
\begin{center}
\begin{tabular}{l}
|\def\version{final}|\\
|\input{childdoc.def}|\\
|\childdocforwardprefix{final}{child}|
\end{tabular}
\end{center}
%

Note that when several versions of a main file and/or of each child file
are to be generated, it may be convenient to set up a |Makefile| or
shell script to automatise the process.

%%%%%%%%%%%%%%%%%%%%%%%%%%%%%%%%%%%%%%%%%%%%%%%%%%%%%%%%%%%%%%%%%%%%%%%%%%%%%%%%
\subsection{Command Line Processing}
\label{sec:commandline}

The effect of redirection files can also be achieved by invoking
the \LaTeX{} compiler with a more elaborate command line.
Most conveniently this should be done as part
of a shell script or a |Makefile|.

When using \textsf{childdoc} in the main file, the following
command lines effectively perform a redirection
(note that depending on the shell being used,
backslashes may have to be doubled: `|\|' $\to$ `|\\|'):
%
\begin{center}
|... -jobname "|\textit{target}|" |\\|"|[\textit{flags}]%
|\input{childdoc.def}\childdocforward[|\textit{main}|]{|\textit{dest}|}"|
\end{center}
%
Here \textit{target} is the name of the output file,
\textit{main} is the name of the main file
and \textit{dest} is the name of the main or child file to be processed
(all filenames without extensions).
The optional argument \textit{main} can be omitted
if \textit{main} matches \textit{dest}.
Optionally, compilation \textit{flags} can be defined via |\def| commands.
This command line makes the \TeX{} engine believe
it is compiling the file \textit{target}
whose content is specified as the latter parameter.
The provided code then forwards the processing to
\textit{main} or \textit{dest} as described in \secref{sec:forward}.

%%%%%%%%%%%%%%%%%%%%%%%%%%%%%%%%%%%%%%%%%%%%%%%%%%%%%%%%%%%%%%%%%%%%%%%%%%%%%%%%
\subsection{Include by Input}
\label{sec:input}

Including child documents by |\include| has some restrictions by design.
Most notably, the content of a child document always occupies
its own set of pages; pages cannot be shared between child documents.
Usually, this behaviour makes perfect sense
because each child document contain an essential part of the document.
However, in some situations it may be desirable to compose
a document from a collection of parts
without having mandatory page breaks between then.
For this case, the package
provides a mechanism to include parts
by |\input| which can also be processed individually.
However, by construction this mechanism
requires manual handling of the content to be output.

%%%%%%%%%%%%%%%%%%%%%%%%%%%%%%%%%%%%%%%%
\DescribeMacro{\ifchilddocmanual}
The main file should be prepared as usual, see \secref{sec:include}.
However, the document body must make a distinction
between processing of an individual part and of the main document, e.g.:
%
\begin{center}
\begin{tabular}{l}
|\ifchilddocmanual|\\
|\input{\childdocname}|\\
|\||else|\\
\textit{document body with }|\input{|\textit{part}|}|\\
|\||fi|
\end{tabular}
\end{center}
%
The conditional |\ifchilddocmanual| is true whenever
a part to be included by |\input| is being compiled,
and the name of the part is stored in |\childdocname|.

%%%%%%%%%%%%%%%%%%%%%%%%%%%%%%%%%%%%%%%%
\DescribeMacro{\childdocby}
Each part to be included by |\input| should start with:
%
\begin{center}
\begin{tabular}{l}
|\input{childdoc.def}|\\
|\childdocby{|\textit{main}|}|\\
\end{tabular}
\end{center}
%
The directive |\childdocby| is similar to |\childdocof|
described in \secref{sec:include},
but the subsequent selection of content must be done manually.
To that end, both |\ifchilddoc| and |\ifchilddocmanual|
will be true upon processing of a part,
and the name of the part is stored in |\childdocname|.
Note that |\jobname| will be set to the filename of the current part
so that each part receives an individual |.aux| file
that does not interfere with the |.aux| file(s) of the main document.
This behaviour can be altered by the alternative form
|\childdocby[*]{|\textit{main}|}| (with a non-empty optional argument)
which uses the |.aux| file of the main document
by setting |\jobname| to \textit{main}.

%%%%%%%%%%%%%%%%%%%%%%%%%%%%%%%%%%%%%%%%%%%%%%%%%%%%%%%%%%%%%%%%%%%%%%%%%%%%%%%%
\subsection{Driver Development}
\label{sec:driver}

The \textsf{childdoc} mechanism can also be use for the development
of definition files such as \LaTeX{} styles or classes.
This case differs from the above setup with multiple parts
included by |\include| in that no |\includeonly| should be invoked.
This can be achieved by starting the include file
(before |\ProvidesPackage|) with:
%
\begin{center}
\begin{tabular}{l}
|\input{childdoc.def}|\\
|\childdocforward{|\textit{main}|}|\\
\end{tabular}
\end{center}
%
or alternatively with:
%
\begin{center}
\begin{tabular}{l}
|\input{childdoc.def}|\\
|\childdocby{|\textit{main}|}|\\
\end{tabular}
\end{center}
%
Both forms have slightly different effects as described above.
The main file is prepared as usual, see \secref{sec:include}.

%%%%%%%%%%%%%%%%%%%%%%%%%%%%%%%%%%%%%%%%%%%%%%%%%%%%%%%%%%%%%%%%%%%%%%%%%%%%%%%%
\subsection{Legacy Detection}
\label{sec:detection}

The directive |\childdocmain| in the main file can detect
whether the complete document or merely a child is to be compiled
even without using the directive |\childdocof|.
This method is deprecated because it is less robust
and there is no compelling reason to use it;
it is merely provided for backward compatibility
and it may be removed in future versions.

If the detection mechanism is to be used,
it is mandatory to correctly specify
the filename of the main file as the argument of |\childdocmain|:
%
\begin{center}
\begin{tabular}{l}
|\input{childdoc.def}|\\
|\childdocmain{|\textit{main}|}|\\
\end{tabular}
\end{center}
%
If |\jobname| does not match the argument \textit{main} of |\childdocmain|,
it is assumed that |\jobname| points to the child file to be compiled.
When using |\childdocmain| with the main file specified as argument,
it suffices to start a child file
with just |\input{|\textit{main}|}|
without loading of the package and using |\childdocof|.
If instead all processing is done
with the appropriate \textsf{childdoc} directives,
the argument of \textit{main} of |\childdocmain| can be empty.

An alternative version of the command line processing described
in \secref{sec:commandline} using the detection mechanism reads:
%
\begin{center}
|... -jobname "|\textit{target}|" "|[\textit{flags}]%
[|\def\jobname{|\textit{dest}|}|]|\input{|\textit{main}|}"|
\end{center}

%%%%%%%%%%%%%%%%%%%%%%%%%%%%%%%%%%%%%%%%%%%%%%%%%%%%%%%%%%%%%%%%%%%%%%%%%%%%%%%%
\subsection{Manual Code}
\label{sec:manual}

In case one cannot be certain whether the definitions file |childdoc.def|
is installed on the target \TeX{} distribution
and one prefers not to ship it,
it is conceivable to paste a few relevant commands into the sources.

To that end, drop all statements |\input{childdoc.def}|
and perform the replacements as outlined below.
Instead of |\childdocmain{|\textit{main}|}| add the following code
to the top of the main file:
%
\begin{center}
\begin{tabular}{l}
|\||ifdefined\childdocname\endinput\||fi\newif\ifchilddoc|\\
|\edef\childdocname{\scantokens\expandafter{\jobname\noexpand}}|\\
|\def\childdocmain{|\textit{main}|}\||ifx\childdocmain\childdocname\||else|\\
|\childdoctrue\includeonly{\childdocname}\let\jobname\childdocmain\||fi|\\
\end{tabular}
\end{center}
%
Instead of |\childdocof{|\textit{main}|}| just include the main file
at the top of each child file:
%
\begin{center}
|\input{|\textit{main}|}|
\end{center}
%
A simple redirection |\childdocforward{|\textit{dest}|}| is achieved by:
%
\begin{center}
|\def\jobname{|\textit{dest}|}\input{\jobname}|
\end{center}
%
The redirection with prefix
|\childdocforwardprefix[|\textit{prefix}|]{|\textit{dest}|}|
is accomplished by:
%
\begin{center}
\begin{tabular}{l}
|{\edef\jobname{\scantokens\expandafter{\jobname\noexpand}}|\\
|\def\redirectjob |\textit{prefix}|#1~~~{\gdef\jobname{|\textit{dest}|#1}}|\\
|\expandafter\redirectjob\jobname~~~}\input{\jobname}|
\end{tabular}
\end{center}

In an alternative approach,
child documents can be compiled by a specific command line
without additional code or specific definitions:
%
\begin{center}
|... -jobname "|\textit{target}|" "|[\textit{flags}]%
|\includeonly{|\textit{dest}|}\input{|\textit{main}|}"|
\end{center}
%

%%%%%%%%%%%%%%%%%%%%%%%%%%%%%%%%%%%%%%%%%%%%%%%%%%%%%%%%%%%%%%%%%%%%%%%%%%%%%%%%
%%%%%%%%%%%%%%%%%%%%%%%%%%%%%%%%%%%%%%%%%%%%%%%%%%%%%%%%%%%%%%%%%%%%%%%%%%%%%%%%
\section{Information}

%%%%%%%%%%%%%%%%%%%%%%%%%%%%%%%%%%%%%%%%%%%%%%%%%%%%%%%%%%%%%%%%%%%%%%%%%%%%%%%%
\subsection{Copyright}

Copyright \copyright{} 2017--2018 Niklas Beisert

This work may be distributed and/or modified under the
conditions of the \LaTeX{} Project Public License, either version 1.3
of this license or (at your option) any later version.
The latest version of this license is in
  \url{http://www.latex-project.org/lppl.txt}
and version 1.3 or later is part of all distributions of \LaTeX{}
version 2005/12/01 or later.

This work has the LPPL maintenance status `maintained'.

The Current Maintainer of this work is Niklas Beisert.

This work consists of the files |README.txt|, |childdoc.ins| and |childdoc.dtx|
as well as the derived files |childdoc.def|, |cdocsamp.tex|
with |cdocsch1.tex|, |cdocsch2.tex|, |cdocspt3.tex|, |cdocspt4.tex|,
|cdocsdrf.tex|, |cdocsfn1.tex|, |cdocsfn2.tex|
as well as |childdoc.pdf|.

%%%%%%%%%%%%%%%%%%%%%%%%%%%%%%%%%%%%%%%%%%%%%%%%%%%%%%%%%%%%%%%%%%%%%%%%%%%%%%%%
\subsection{Files and Installation}

The package consists of the files:
%
\begin{center}
\begin{tabular}{ll}
    |README.txt|   & readme file \\
    |childdoc.ins| & installation file \\
    |childdoc.dtx| & source file \\
    |childdoc.def| & definition file \\
    |cdocsamp.tex| & sample main file \\
    |cdocsch1.tex| & sample include file \\
    |cdocsch2.tex| & sample include file \\
    |cdocspt3.tex| & sample part file \\
    |cdocspt4.tex| & sample part file \\
    |cdocsdrf.tex| & sample redirection file \\
    |cdocsfn1.tex| & sample redirection file \\
    |cdocsfn2.tex| & sample redirection file \\
    |childdoc.pdf| & manual
\end{tabular}
\end{center}
%
The distribution consists of the files
|README.txt|, |childdoc.ins| and |childdoc.dtx|.
%
\begin{itemize}
\item
Run (pdf)\LaTeX{} on |childdoc.dtx|
to compile the manual |childdoc.pdf| (this file).
\item
Run \LaTeX{} on |childdoc.ins| to create the definitions file |childdoc.def|
and the sample |cdocsamp.tex| with include files
|cdocsch1.tex|, |cdocsch2.tex|, |cdocspt3.tex|, |cdocspt4.tex|,
|cdocsdrf.tex|, |cdocsfn1.tex|, |cdocsfn2.tex|.
Then copy the file |childdoc.def| to an appropriate directory of your \LaTeX{}
distribution, e.g.\ \textit{texmf-root}|/tex/latex/childdoc|.
\end{itemize}

%%%%%%%%%%%%%%%%%%%%%%%%%%%%%%%%%%%%%%%%%%%%%%%%%%%%%%%%%%%%%%%%%%%%%%%%%%%%%%%%
\subsection{Related CTAN Packages}

There are several other packages which offer a similar functionality:
%
\begin{itemize}
\item
The packages
\href{http://ctan.org/pkg/docmute}{\textsf{docmute}},
\href{http://ctan.org/pkg/includex}{\textsf{includex}} and
\href{http://ctan.org/pkg/standalone}{\textsf{standalone}}
provide commands to include only the document body of
a child file thus allowing both files to be compiled individually.
\item
The packages \href{http://ctan.org/pkg/subdocs}{\textsf{subdocs}}
and \href{http://ctan.org/pkg/subfiles}{\textsf{subfiles}}
provide structures in which the main and child documents can be
encapsulated and allowing them to be compiled individually.
The inclusion mechanism is different from the conventional |\include|.
\item
The package \href{http://ctan.org/pkg/combine}{\textsf{combine}}
is an elaborate solution to combine several documents into one.
\end{itemize}
%
See also the CTAN topic \href{http://ctan.org/topic/subdocs}{\textsf{subdocs}}
for further related packages.
The present package differs from the above solutions in that
a document structure constructed with the conventional |\include| mechanism
just needs two extra commands at the top of every file
such that all constituent files can be compiled individually.

%%%%%%%%%%%%%%%%%%%%%%%%%%%%%%%%%%%%%%%%%%%%%%%%%%%%%%%%%%%%%%%%%%%%%%%%%%%%%%%%
%\subsection{Feature Suggestions}
%
%The following is a list of features which may be useful for future
%versions of this package:
%%
%\begin{itemize}
%\item
%\ldots
%\end{itemize}

%%%%%%%%%%%%%%%%%%%%%%%%%%%%%%%%%%%%%%%%%%%%%%%%%%%%%%%%%%%%%%%%%%%%%%%%%%%%%%%%
\subsection{Revision History}

%%%%%%%%%%%%%%%%%%%%%%%%%%%%%%%%%%%%%%%%
\paragraph{v2.0:} 2018/12/30

\begin{itemize}
\item
immediate forward processing
\item
added |\childdocby| mechanism
\item
manual restructured
\end{itemize}

%%%%%%%%%%%%%%%%%%%%%%%%%%%%%%%%%%%%%%%%
\paragraph{v1.6:} 2018/01/17

\begin{itemize}
\item
application for development of include files
\item
corrections to manual
\end{itemize}

%%%%%%%%%%%%%%%%%%%%%%%%%%%%%%%%%%%%%%%%
\paragraph{v1.5:} 2017/05/21

\begin{itemize}
\item
more complete structuring introduced
\item
|\childdocof| introduced
\item
|\childdoc| renamed to |\childdocmain|
\item
|\childredirect| renamed to |\childdocforward| and |\childdocforwardprefix|
and functionality expanded
\end{itemize}

%%%%%%%%%%%%%%%%%%%%%%%%%%%%%%%%%%%%%%%%
\paragraph{v1.0:} 2017/04/27

\begin{itemize}
\item
manual and install package
\item
first version published on CTAN
\end{itemize}

%%%%%%%%%%%%%%%%%%%%%%%%%%%%%%%%%%%%%%%%
\paragraph{v0.6:} 2017/04/26

\begin{itemize}
\item
redirection mechanism added
\end{itemize}

%%%%%%%%%%%%%%%%%%%%%%%%%%%%%%%%%%%%%%%%
\paragraph{v0.5:} 2017/04/26

\begin{itemize}
\item
functionality in definition file
\end{itemize}


%%%%%%%%%%%%%%%%%%%%%%%%%%%%%%%%%%%%%%%%%%%%%%%%%%%%%%%%%%%%%%%%%%%%%%%%%%%%%%%%
%%%%%%%%%%%%%%%%%%%%%%%%%%%%%%%%%%%%%%%%%%%%%%%%%%%%%%%%%%%%%%%%%%%%%%%%%%%%%%%%
%%%%%%%%%%%%%%%%%%%%%%%%%%%%%%%%%%%%%%%%%%%%%%%%%%%%%%%%%%%%%%%%%%%%%%%%%%%%%%%%
\appendix

\settowidth\MacroIndent{\rmfamily\scriptsize 000\ }

 \DocInput{childdoc.dtx}

\end{document}
%</driver>
% \fi
%
% %%%%%%%%%%%%%%%%%%%%%%%%%%%%%%%%%%%%%%%%%%%%%%%%%%%%%%%%%%%%%%%%%%%%%%%%%%%%%%
% %%%%%%%%%%%%%%%%%%%%%%%%%%%%%%%%%%%%%%%%%%%%%%%%%%%%%%%%%%%%%%%%%%%%%%%%%%%%%%
% \section{Sample}
%\iffalse
%<*samplemain>
%\fi
%
% The following presents a sample document
% with two chapters, two parts, a title page,
% a compile flag as well as three forwarding files to set the flag.
% It consists of eight |.tex| files:
% \begin{center}
% \begin{tabular}{ll}
% |cdocsamp.tex|&main file\\
% |cdocsch1.tex|&include file for chapter 1\\
% |cdocsch2.tex|&include file for chapter 2\\
% |cdocspt3.tex|&include file for part 3\\
% |cdocspt4.tex|&include file for part 4\\
% |cdocsdrf.tex|&forwarding file for main file in draft mode\\
% |cdocsfi1.tex|&forwarding file for final version of chapter 1\\
% |cdocsfi2.tex|&forwarding file for final version of chapter 2\\
% \end{tabular}
% \end{center}
% Each of the eight files can be compiled directly by the \LaTeX{} compiler.
%
% %%%%%%%%%%%%%%%%%%%%%%%%%%%%%%%%%%%%%%
% \paragraph{Main File.}
%
% The main file is called |cdocsamp.tex|.
%
% Load the \textsf{childdoc} definitions and
% declare the filename for the main document:
%    \begin{macrocode}
\input{childdoc.def}
\childdocmain{}
%    \end{macrocode}

% Optional override for |\version| flag:
%    \begin{macrocode}
%%\ifchilddoc\else\providecommand{\version}{draft}\fi
%    \end{macrocode}

% Define the default values for the |\version| flag
% (|final| for the main file and |draft| for childs):
%    \begin{macrocode}
\ifchilddoc
\providecommand{\version}{draft}
\else
\providecommand{\version}{final}
\fi
%    \end{macrocode}

% Load the standard document class:
%    \begin{macrocode}
\documentclass[12pt]{article}
%    \end{macrocode}

% Start the document body:
%    \begin{macrocode}
\begin{document}
%    \end{macrocode}

% Declare a title page.
% Print title, part of document being processed and version flag:
%    \begin{macrocode}
\addtocounter{page}{-1}
\begin{center}
{\LARGE\bfseries{}childdoc example\par}
\vspace{1cm}
\ifchilddoc
\ifchilddocmanual part\else chapter\fi:
`\childdocname' of `\childdocjob'\par
\else
main document: `\childdocjob'\par
\fi
version: \version\par
\end{center}
\newpage
%    \end{macrocode}

% Manually include selected file,
% otherwise process as usual:
%    \begin{macrocode}
\ifchilddocmanual
\section*{part `\childdocname'}
\input{\childdocname}
\else
%    \end{macrocode}

% Include the two chapters:
%    \begin{macrocode}
\include{cdocsch1}
\include{cdocsch2}
%    \end{macrocode}

% Include the two parts unless only chapters should be displayed:
%    \begin{macrocode}
\ifchilddoc\else
\section{part three}
\input{cdocspt3}
\section{part four}
\input{cdocspt4}
\fi
%    \end{macrocode}

% Process as usual until here:
%    \begin{macrocode}
\fi
%    \end{macrocode}

% End of document body:
%    \begin{macrocode}
\end{document}
%    \end{macrocode}
%\iffalse
%</samplemain>
%\fi
%
% %%%%%%%%%%%%%%%%%%%%%%%%%%%%%%%%%%%%%%
% \paragraph{Chapter Include Files.}
%
% The include files are called |cdocsch1.tex| and |cdocsch2.tex|.
%
%\iffalse
%<*samplechap1|samplechap2>
%\fi

% Optional override for |\version| flag:
%    \begin{macrocode}
%%\providecommand{\version}{final}
%    \end{macrocode}

% Include the main document:
%    \begin{macrocode}
\input{childdoc.def}
\childdocof{cdocsamp}
%    \end{macrocode}

%\iffalse
%</samplechap1|samplechap2>
%\fi
%
%\iffalse
%<*samplechap1>
%\fi
% Some text for chapter 1:
%    \begin{macrocode}
\section{one}
some text in chapter one
%    \end{macrocode}

%\iffalse
%</samplechap1>
%\fi
% Some text for chapter 2:
%\iffalse
%<*samplechap2>
%\fi
%    \begin{macrocode}
\section{two}
more text in chapter two
%    \end{macrocode}

%\iffalse
%</samplechap2>
%\fi
%
% %%%%%%%%%%%%%%%%%%%%%%%%%%%%%%%%%%%%%%
% \paragraph{Part Include Files.}
%
% The include files are called |cdocspt3.tex| and |cdocspt4.tex|.
%
%\iffalse
%<*samplepart3|samplepart4>
%\fi

% Optional override for |\version| flag:
%    \begin{macrocode}
%%\providecommand{\version}{final}
%    \end{macrocode}

% Include the main document:
%    \begin{macrocode}
\input{childdoc.def}
\childdocby{cdocsamp}
%    \end{macrocode}

%\iffalse
%</samplepart3|samplepart4>
%\fi
%
%\iffalse
%<*samplepart3>
%\fi
% Some text for part 3:
%    \begin{macrocode}
some text in part three
%    \end{macrocode}

%\iffalse
%</samplepart3>
%\fi
% Some text for part 4:
%\iffalse
%<*samplepart4>
%\fi
%    \begin{macrocode}
more text in part four
%    \end{macrocode}

%\iffalse
%</samplepart4>
%\fi
%
% %%%%%%%%%%%%%%%%%%%%%%%%%%%%%%%%%%%%%%
% \paragraph{Forwarding for a Complete Draft.}
%
% The following forwarding file |cdocsdrf.tex|
% compiles the main document in draft mode:
%\iffalse
%<*sampledraft>
%\fi
%    \begin{macrocode}
\def\version{draft}
\input{childdoc.def}
\childdocforward{cdocsamp}
%    \end{macrocode}

%\iffalse
%</sampledraft>
%\fi
%
% %%%%%%%%%%%%%%%%%%%%%%%%%%%%%%%%%%%%%%
% \paragraph{Forwarding for Final Version of the Chapters.}
%
% The following forwarding files |cdocsfn1.tex| and |cdocsfn2.tex|
% (with identical content)
% compile the final versions of the child documents
% |cdocsch1.tex| and |cdocsch2.tex|, respectively:
%\iffalse
%<*samplefinal>
%\fi
%    \begin{macrocode}
\def\version{final}
\input{childdoc.def}
\childdocforwardprefix[cdocsamp]{cdocsfn}{cdocsch}
%    \end{macrocode}

%\iffalse
%</samplefinal>
%\fi
%
% %%%%%%%%%%%%%%%%%%%%%%%%%%%%%%%%%%%%%%
% \paragraph{Command Line Processing.}
%
% The following three command lines generate the output files
% |cdocscld|, |cdocscl1| and |cdocscl2|
% which should be identical to
% |cdocsdrf|, |cdocsch1| and |cdocsfn2|, respectively:
% \begin{center}
% \begin{tabular}{l}
% |latex -jobname cdocscld \|\\
% |  "\def\version{draft}\input{childdoc.def}\childdocforward{cdocsamp}"|\\
% |latex -jobname cdocscl1 \|\\
% |  "\input{childdoc.def}\childdocforward[cdocsamp]{cdocsch1}"|\\
% |latex -jobname cdocscl2 \|\\
% |  "\def\version{final}\input{childdoc.def}\childdocforward{cdocsch2}"|
% \end{tabular}
% \end{center}
% Note that the trailing backslash on each first line
% merely continues the input to the second line
% (for convenient cut ant paste).
% Furthermore, the command |latex| can be replaced by any
% of its alternative versions such as |pdflatex|.
%
% %%%%%%%%%%%%%%%%%%%%%%%%%%%%%%%%%%%%%%%%%%%%%%%%%%%%%%%%%%%%%%%%%%%%%%%%%%%%%%
% %%%%%%%%%%%%%%%%%%%%%%%%%%%%%%%%%%%%%%%%%%%%%%%%%%%%%%%%%%%%%%%%%%%%%%%%%%%%%%
% \section{Implementation}
%\iffalse
%<*package>
%\fi
%
% This section describes the definitions file |childdoc.def|.

% The definitions cannot be loaded using |\usepackage| or |\RequirePackage|
% which has a mechanism to prevent loading a style file more than once.
% When loading the definitions by means of |\input|
% multiple instances have to be prevented manually:
%\iffalse
%This code needs to be before the `\ProvidesFile' directive
%which is defined at the beginning of this file.
%Therefore it is also placed there and commented out here.
%</package>
%<*discard>
%\fi
%    \begin{macrocode}
\ifdefined\childdocmain\endinput\fi
%    \end{macrocode}
%\iffalse
%</discard>
%<*package>
%\fi
%
% \macro{\ifchilddoc}
% \macro{\ifchilddocmanual}
% The conditional |\ifchilddoc| tells whether a
% child (true) or main (false) document is being compiled.
% The conditional |\ifchilddocmanual| tells whether
% the |\includeonly| mechanism is used (false) or
% the selection of child files must be performed manually (true).
% The definitions initialise to false:
%    \begin{macrocode}
\newif\ifchilddoc
\newif\ifchilddocmanual
%    \end{macrocode}

% \macro{\childdocname}
% \macro{\childdocjob}
% The macro |\childdocname| stores the name of the main document
% to be compiled. The macro |\childdocjob| stores the name of
% the document on which the \LaTeX{} compiler was originally invoked.
% The content of |\jobname| cannot be compared
% to filenames specified in the source due to different catcodes.
% The following code rescans |\jobname|, stores the result
% in |\childdocname| and saves a copy in |\childdocjob|:
%    \begin{macrocode}
\edef\childdocname{\scantokens\expandafter{\jobname\noexpand}}
\let\childdocjob\childdocname
%    \end{macrocode}

% \macro{\childdocdisable}
% The macro |\childdocdisable| prevents the main file
% from being processed more than once.
% At this stage, the main document command |\childdocmain|
% is assumed to be called once again where it should do nothing.
% Any subsequent call to it should prevent
% a secondary processing of the main document
% It overwrites the forwarding commands
% |\childdocof| and |\childdocforward|
% with empty macros to prevent further inclusions of the main document:
%    \begin{macrocode}
\newcommand{\childdocdisable}
{
  \renewcommand{\childdocmain}[1]{\renewcommand{\childdocmain}[1]{\endinput}}
  \renewcommand{\childdocof}[1]{}
  \renewcommand{\childdocby}[2][]{}
  \renewcommand{\childdocforward}[2][]{}
  \renewcommand{\childdocdisable}{}
}
%    \end{macrocode}

% \macro{\childdocmain}
% The macro |\childdocmain| is to be called at the top of the main file
% with nothing or the main filename (without extension) as argument.
% First, it breaks loops.
% If the argument is not empty and does not match |\childdocname|
% (which is set by the first inclusion of |childdoc.def|),
% |\ifchilddoc| is set to true, |\includeonly| is applied to the child file
% and |\jobname| is set to the main file
% (for proper handling of |.aux| files):
%    \begin{macrocode}
\newcommand{\childdocmain}[1]
{
  \childdocdisable\childdocmain{}
  \if?#1?\else
    \begingroup
      \def\childdoctmp{#1}
      \ifx\childdoctmp\childdocname
        \def\childdoctmp{}
      \else
        \def\childdoctmp
        {
          \childdoctrue
          \includeonly{\childdocname}
          \def\childdocjob{#1}
          \def\jobname{#1}
        }
      \fi
      \expandafter
    \endgroup
    \childdoctmp
  \fi
}
%    \end{macrocode}

% \macro{\childdocof}
% The command |\childdocof| redirects
% compilation to the main file |#1|.
%    \begin{macrocode}
\newcommand{\childdocof}[1]
{
  \childdocdisable
  \childdoctrue
  \includeonly{\childdocname}
  \def\jobname{#1}
  \def\childdocjob{#1}
  \input{#1}
}
%    \end{macrocode}

% \macro{\childdocby}
% The command |\childdocby| ....
%    \begin{macrocode}
\newcommand{\childdocby}[2][]
{
  \childdocdisable
  \childdoctrue
  \childdocmanualtrue
  \if?#1?\else
    \def\jobname{#2}
  \fi
  \def\childdocjob{#2}
  \input{#2}
  \endinput
}
%    \end{macrocode}

% \macro{\childdocforward}
% The command |\childdocforward| redirects
% compilation to the main file or
% (if the optional argument is given) a child file.
% Parameters are set as if the main file
% or a child file starting with |\childdocof| was compiled.
% Then compilation is handed over to the main file:
%    \begin{macrocode}
\newcommand{\childdocforward}[2][]
{
  \begingroup
    \if?#1?
      \def\childdoctmp
      {
        \def\childdocname{#2}
        \def\childdocjob{#2}
        \def\jobname{#2}
        \input{#2}
        \endinput
      }
    \else
      \def\childdoctmp
      {
        \childdocdisable
        \def\childdocname{#2}
        \childdoctrue
        \includeonly{#2}
        \def\childdocjob{#1}
        \def\jobname{#1}
        \input{#1}
        \endinput
      }
    \fi
    \expandafter
  \endgroup
  \childdoctmp
}
%    \end{macrocode}

% \macro{\childdocforwardprefix}
% The command |\childdocforwardprefix| redirects
% compilation to the main or a child file by means of a pattern.
% The prefix |#1| in the current filename is replaced by |#2|
% and the suffix of the current filename is kept
% (it is assumed that the filename does not contain the substring `|~~~|'
% which is used as a delimiter).
% Compilation is handed over to the new file by |\childdocforward|:
%    \begin{macrocode}
\newcommand{\childdocforwardprefix}[3][]
{
  \begingroup
    \def\childdocextract #2##1~~~{\def\childdoctmp{\childdocforward[#1]{#3##1}}}
    \expandafter\childdocextract\childdocname~~~
    \expandafter
  \endgroup
  \childdoctmp
}
%    \end{macrocode}

% \macro{\childdoc}
% The deprecated macro |\childdoc| is a legacy version of |\childdocmain|:
%    \begin{macrocode}
\newcommand{\childdoc}{\childdocmain}
%    \end{macrocode}

% \macro{\childdocredirect}
% The deprecated macro |\childdocredirect| is a legacy version
% of |\childdocforward| and |\childdocforwardprefix|:
%    \begin{macrocode}
\newcommand{\childdocredirect}[2][]
{
  \begingroup
    \if?#1?
      \def\childdoctmp{\childdocforward{#2}}
    \else
      \def\childdoctmp{\childdocforwardprefix{#1}{#2}}
    \fi
    \expandafter
  \endgroup
  \childdoctmp
}
%    \end{macrocode}

%\iffalse
%</package>
%\fi
%
\endinput
|\\
|\childdocmain{}|\\
\end{tabular}
\end{center}
at the very top of the main \LaTeX{} file,
in particular \emph{before} the |\documentclass| statement!
The argument of |\childdocmain| should be left empty
(but it must be present).

%%%%%%%%%%%%%%%%%%%%%%%%%%%%%%%%%%%%%%%%
\DescribeMacro{\childdocof}
Furthermore, add the commands
\begin{center}
\begin{tabular}{l}
|% \iffalse
%
% childdoc.dtx Copyright (C) 2017-2018 Niklas Beisert
%
% This work may be distributed and/or modified under the
% conditions of the LaTeX Project Public License, either version 1.3
% of this license or (at your option) any later version.
% The latest version of this license is in
%   http://www.latex-project.org/lppl.txt
% and version 1.3 or later is part of all distributions of LaTeX
% version 2005/12/01 or later.
%
% This work has the LPPL maintenance status `maintained'.
%
% The Current Maintainer of this work is Niklas Beisert.
%
% This work consists of the files childdoc.dtx and childdoc.ins
% and the derived files childdoc.def and cdocsamp.tex with
% cdocsch1.tex, cdocsch2.tex, cdocsdrf.tex, cdocsfn1.tex, cdocsfn2.tex.
%
%<package>\ifdefined\childdocmain\endinput\fi
%<package>\ProvidesFile{childdoc.def}[2018/12/30 v2.0 child document driver]
%<samplemain>\ProvidesFile{cdocsamp.tex}[2018/12/30 v2.0 sample for childdoc]
%<*driver>
%\ProvidesFile{childdoc.drv}[2018/12/30 v2.0 childdoc reference manual file]
\PassOptionsToClass{10pt,a4paper}{article}
\documentclass{ltxdoc}

\usepackage[margin=35mm]{geometry}
\usepackage{hyperref}
\usepackage{hyperxmp}
\usepackage[usenames]{color}

\hypersetup{colorlinks=true}
\hypersetup{pdfstartview=FitH}
\hypersetup{pdfpagemode=UseNone}
\hypersetup{pdfsource={}}
\hypersetup{pdflang={en-UK}}
\hypersetup{pdfcopyright={Copyright 2017-2018 Niklas Beisert.
  This work may be distributed and/or modified under the
  conditions of the LaTeX Project Public License, either version 1.3
  of this license or (at your option) any later version.}}
\hypersetup{pdflicenseurl={http://www.latex-project.org/lppl.txt}}
\hypersetup{pdfcontactaddress={ETH Zurich, ITP, HIT K,
  Wolfgang-Pauli-Strasse 27}}
\hypersetup{pdfcontactpostcode={8093}}
\hypersetup{pdfcontactcity={Zurich}}
\hypersetup{pdfcontactcountry={Switzerland}}
\hypersetup{pdfcontactemail={nbeisert@itp.phys.ethz.ch}}
\hypersetup{pdfcontacturl={http://people.phys.ethz.ch/\xmptilde nbeisert/}}

\newcommand{\secref}[1]{\hyperref[#1]{section \ref*{#1}}}

\parskip1ex
\parindent0pt
\let\olditemize\itemize
\def\itemize{\olditemize\parskip0pt}

\begin{document}

\title{The \textsf{childdoc} Package}
\hypersetup{pdftitle={The childdoc Package}}
\author{Niklas Beisert\\[2ex]
  Institut f\"ur Theoretische Physik\\
  Eidgen\"ossische Technische Hochschule Z\"urich\\
  Wolfgang-Pauli-Strasse 27, 8093 Z\"urich, Switzerland\\[1ex]
  \href{mailto:nbeisert@itp.phys.ethz.ch}
  {\texttt{nbeisert@itp.phys.ethz.ch}}}
\hypersetup{pdfauthor={Niklas Beisert}}
\hypersetup{pdfsubject={Manual for the LaTeX2e Package childdoc}}
\date{30 December 2018, \textsf{v2.0}}
\maketitle

\begin{abstract}\noindent
\textsf{childdoc} is a \LaTeXe{} package
that enables the direct compilation
of document sections included by |\include|
to individual files.
\end{abstract}

\begingroup
\parskip0ex
\tableofcontents
\endgroup

%%%%%%%%%%%%%%%%%%%%%%%%%%%%%%%%%%%%%%%%%%%%%%%%%%%%%%%%%%%%%%%%%%%%%%%%%%%%%%%%
%%%%%%%%%%%%%%%%%%%%%%%%%%%%%%%%%%%%%%%%%%%%%%%%%%%%%%%%%%%%%%%%%%%%%%%%%%%%%%%%
\section{Introduction}

\LaTeX{} provides a mechanism to structure a large document (such as a book)
into a main file and several child files (containing the chapters)
using the |\include| command.
This mechanism is beneficial for documents
which span hundreds of pages in order to
make the source file(s) more manageable.
Moreover, compilation can be restricted to
selected child files by means of the |\includeonly| command.
The latter feature can be used to reduce the compilation time while editing
(this was significantly more useful in the earlier days of \LaTeX{})
or to generate a smaller document which is easier to navigate.
Another application of |\includeonly| is to generate
documents consisting of selected parts of the complete document.

However, there are a few drawbacks of the plain |\include| mechanism:
\begin{itemize}
\item
The child files cannot be compiled on their own,
they can only be compiled via the main file.
A naive editing environment
(such as a text editor with an option
to have the current file processed by \LaTeX)
may require one to switch to the main file before compiling;
attempting to compile the child file produces errors.
\item
The main file must be modified (each time)
to adjust the |\includeonly| command
to the present needs. This easily leaves the main file in a messy state.
\item
The generated document will always carry the filename
of the main document. This is inconvenient if
several child files are to be compiled and
to be kept for distribution.
\end{itemize}

The present package provides a simple interface
to make child files individually compilable by \LaTeX{}.
Compiling a child file then has the same effect as compiling
the main file with an |\includeonly| command
to select the appropriate child.
Moreover the generated document will carry the name of the child
rather than the main file.
This resolves all three above issues.

This feature is meant to make the editing of books,
thesis documents and lecture notes somewhat more convenient.
However, the package can also be used efficiently for
composing a series of documents (such as exercise sheets)
which are typically distributed individually.
It then assists the author in generating the individual documents
(potentially in different versions)
as well as a document containing the collected series.
Another application is in developing style files
or other kinds of included material
where compilation of the style file could redirect
to a sample or test file.

%%%%%%%%%%%%%%%%%%%%%%%%%%%%%%%%%%%%%%%%%%%%%%%%%%%%%%%%%%%%%%%%%%%%%%%%%%%%%%%%
%%%%%%%%%%%%%%%%%%%%%%%%%%%%%%%%%%%%%%%%%%%%%%%%%%%%%%%%%%%%%%%%%%%%%%%%%%%%%%%%
\section{Usage}

First of all, the package \textsf{childdoc} is \emph{not} a standard
\LaTeXe{} |.sty| style file! Therefore it needs to be invoked in
a non-standard way.

%%%%%%%%%%%%%%%%%%%%%%%%%%%%%%%%%%%%%%%%%%%%%%%%%%%%%%%%%%%%%%%%%%%%%%%%%%%%%%%%
\subsection{Included Files}
\label{sec:include}

%%%%%%%%%%%%%%%%%%%%%%%%%%%%%%%%%%%%%%%%
\DescribeMacro{\childdocmain}
To use the package, add the commands
\begin{center}
\begin{tabular}{l}
|\input{childdoc.def}|\\
|\childdocmain{}|\\
\end{tabular}
\end{center}
at the very top of the main \LaTeX{} file,
in particular \emph{before} the |\documentclass| statement!
The argument of |\childdocmain| should be left empty
(but it must be present).

%%%%%%%%%%%%%%%%%%%%%%%%%%%%%%%%%%%%%%%%
\DescribeMacro{\childdocof}
Furthermore, add the commands
\begin{center}
\begin{tabular}{l}
|\input{childdoc.def}|\\
|\childdocof{|\textit{main}|}|\\
\end{tabular}
\end{center}
at the top of every child file \textit{child}
which is included by |\include{|\textit{child}|}|
from within the main file
(or at least for those files to be compiled individually).
The argument \textit{main} must be the filename of the main file.

There are a couple of
considerations in setting up the main and child documents:

%%%%%%%%%%%%%%%%%%%%%%%%%%%%%%%%%%%%%%%%
\paragraph{Restrictions.}

Please note the following restrictions:
\begin{itemize}
\item
|\childdocmain| must be called with one argument \textit{main}
to ensure compatibility with earlier version of the package.
It must either be empty (|\childdocmain{}|)
or precisely match the filename of the main file in which it is specified.
See \secref{sec:detection} for further information.
\item
The filename \textit{main} must be specified without the |.tex| extension.
\item
The filename \textit{main} is case sensitive
(even in case-insensitive file systems)
due to internal string comparison.
\item
The argument \textit{main} should be fully expanded, it cannot be a macro.
\item
Subdirectories and special characters should be avoided in filenames.
\item
The command |\childdocmain{|\textit{main}|}| must be followed by a whitespace.
It should not be followed immediately by another command
or by a comment mark `|%|'.
This is because the \TeX{} parser reads the token immediately following
the argument of |\childdocmain| and puts it
at the beginning of every child section;
however, a white\-space is ignored.
\end{itemize}

%%%%%%%%%%%%%%%%%%%%%%%%%%%%%%%%%%%%%%%%
\paragraph{Content of Main File.}

It is advisable to place all content in the child files included by |\include|.
Any output contained in the main file will appear in all child documents
unless suppressed manually;
it cannot be suppressed automatically by the |\includeonly| directive
and thus should normally be avoided.
A method to include some content in the main file
by means of conditional processing is described in \secref{sec:conditional}.

%%%%%%%%%%%%%%%%%%%%%%%%%%%%%%%%%%%%%%%%
\paragraph{Page Numbering.}

When only a part of the document is compiled,
the appropriate numbering of pages
(as well as other status parameters)
is determined from the |.aux| files.
The latter contain information from previous passes.
However this information needs to propagate through
all intermediate child documents.
Therefore the page numbering in child documents may well
be inconsistent until the complete document is compiled at least once.

A useful (if unconventional) way to always ensure a consistent
page numbering is to restart the numbering in each child document
and denote the pages by `\textit{child}|.|\textit{page}'
where \textit{child} represents the chapter/section number of the child file.
This can be achieved by the command
|\numberwithin{page}{|\textit{child}|}|
of the \textsf{amsmath} package
where \textit{child} can be |chapter| or |section|
depending on the chosen structuring.
Alternatively, one can modify the macro |\thepage| appropriately
and reset the counter |page| at the start of each child file.

%%%%%%%%%%%%%%%%%%%%%%%%%%%%%%%%%%%%%%%%%%%%%%%%%%%%%%%%%%%%%%%%%%%%%%%%%%%%%%%%
\subsection{Conditional Processing}
\label{sec:conditional}

The package provides a mechanism to compile different versions
of a document. To customise the versions further some conditional processing
can come in handy to distinguish which version is being compiled.
The package provides two macros to describe the compilation context:

%%%%%%%%%%%%%%%%%%%%%%%%%%%%%%%%%%%%%%%%
\DescribeMacro{\ifchilddoc}
The conditional |\ifchilddoc| distinguishes between the compilation of
child documents and the main document:
%
\begin{center}
|\ifchilddoc |\textit{child-code}| |[|\||else |\textit{main-code}]| \||fi|
\end{center}

%%%%%%%%%%%%%%%%%%%%%%%%%%%%%%%%%%%%%%%%
\DescribeMacro{\childdocname}
\DescribeMacro{\childdocjob}
The macro |\childdocname| contains the filename (without extension)
of the main or child file being processed.
Note that |\childdocjob| will always contain the name of the main file.

%%%%%%%%%%%%%%%%%%%%%%%%%%%%%%%%%%%%%%%%
\paragraph{Title Page.}

Conditional processing can be used to include a title or banner page
in the main document when proper precautions are taken.
Importantly, the code in the main file should ensure that the page counter
(as well as other status parameters which are stored in the |.aux| files)
takes the same value after the conditional processing.
Otherwise the page numbers may take divergent values
depending on which part is compiled.

For example, a title page could be declared by:
%
\begin{center}
\begin{tabular}{l}
|\ifchilddoc\||else|\\
|\addtocounter{page}{-1}|\\
\textit{code for title page}\\
|\newpage|\\
|\||fi|
\end{tabular}
\end{center}
%
A banner page for the child documents can be generated by:
%
\begin{center}
\begin{tabular}{l}
|\ifchilddoc|\\
|\addtocounter{page}{-1}|\\
\textit{code for banner page}\\
|\newpage|\\
|\||fi|
\end{tabular}
\end{center}
%
Here one could write a message such as:
\begin{center}
|This is the part \childdocname{} of \childdocjob{}.|
\end{center}

%%%%%%%%%%%%%%%%%%%%%%%%%%%%%%%%%%%%%%%%%%%%%%%%%%%%%%%%%%%%%%%%%%%%%%%%%%%%%%%%
\subsection{Flags}
\label{sec:flags}

The package makes it easy to generate different versions
of the main or child documents.
To this end compilation flags can be defined
and assigned different default values.
They will be particularly useful in conjunction
with the forwarding mechanism described in \secref{sec:forward}.

For example, it may be useful to have a flag |\version|
which can be set to |draft| or |final|.
The document source will contain some conditional code
depending on the value of |\version|.
Suppose further, the flag should default to |final| for the main file
and to |draft| for child files
which is a natural assignment for editing the document.
This is achieved by placing the following code
in the preamble of the main document
(below the |\childdocmain| directive):
%
\begin{center}
\begin{tabular}{l}
|\ifchilddoc|\\
|\providecommand{\version}{draft}|\\
|\||else|\\
|\providecommand{\version}{final}|\\
|\||fi|
\end{tabular}
\end{center}
%
The definition by |\providecommand| makes sure
that previous definitions are not overwritten.
Further statements |\providecommand{\version}{...}|
can thus be added before the above code to override it.

For the main file, one might add a line
(between |\childdocmain| and the above block)
%
\begin{center}
|%\ifchilddoc\||else\providecommand{\version}{draft}\||fi|
\end{center}
%
which can be uncommented to produce a draft version.
Likewise one can add a line to the very top of a child file
(above the |\childdocof{|\textit{main}|}| directive)
%
\begin{center}
|%\providecommand{\version}{final}|
\end{center}
%
which can be uncommented to produce the final version of this child document.

%%%%%%%%%%%%%%%%%%%%%%%%%%%%%%%%%%%%%%%%%%%%%%%%%%%%%%%%%%%%%%%%%%%%%%%%%%%%%%%%
\subsection{Forwarding}
\label{sec:forward}

Different versions of the main or child documents
using compilation flags as described in \secref{sec:flags}
can be (permanently) stored in different files
for convenient compilation, viewing and distribution.
To this end, the package defines a command
to pass on compilation to a different file:

%%%%%%%%%%%%%%%%%%%%%%%%%%%%%%%%%%%%%%%%
\DescribeMacro{\childdocforward}
The command |\childdocforward| redirects processing to
another source file:
%
\begin{center}
\begin{tabular}{l}
|\input{childdoc.def}|\\
|\childdocforward[|\textit{main}|]{|\textit{dest}|}|\\
\end{tabular}
\end{center}
%
The argument \textit{dest} is the destination file
(without extension).
It should be the main file or one of the child files.
Note that further \textsf{childdoc} directives
such as |\childdocof| and |\childdocforward|
in the indicated file will be processed in this form.
The optional argument \textit{main}
passes on directly to the main file \textit{main}
while pretending to compile the child \textit{dest}.
This form behaves as if \textit{dest}
issues |\childdocof{|\textit{main}|}| right away,
and no further \textsf{childdoc} directives will be processed.

%%%%%%%%%%%%%%%%%%%%%%%%%%%%%%%%%%%%%%%%
\DescribeMacro{\...prefix}
In the alternative form |\childdocforwardprefix|,
%
\begin{center}
\begin{tabular}{l}
|\input{childdoc.def}|\\
|\childdocforwardprefix[|\textit{main}|]{|\textit{prefix}|}{|\textit{dest}|}|
\end{tabular}
\end{center}
%
the destination file is determined by a pattern
depending on the current file:
To make this work, the current file must be called
`{\textit{prefix}\hspace{0.2em}\textit{suffix}}'
with \textit{prefix} matching precisely the argument.
Processing is then passed on to the file
`{\textit{dest}\hspace{0.2em}\textit{suffix}}'.
Surely, the same effect is achieved by
directly specifying the
argument `{\textit{dest}\hspace{0.2em}\textit{suffix}}'
in the first form.
However, that requires to set up a different file
for each child. With the alternative form of the command
all these files can have exactly the same content
which simplifies setting them up and maintaining them.

For example, the following file |draft.tex|
with a compilation flag |\version| as described in \secref{sec:flags}
compiles the main document as a draft:
%
\begin{center}
\begin{tabular}{l}
|\def\version{draft}|\\
|\input{childdoc.def}|\\
|\childdocforward{|\textit{main}|}|
\end{tabular}
\end{center}
%
Likewise, the following files |final|\textit{nn}|.tex|
compile the final version of the child document
|child|\textit{nn}|.tex|:
%
\begin{center}
\begin{tabular}{l}
|\def\version{final}|\\
|\input{childdoc.def}|\\
|\childdocforwardprefix{final}{child}|
\end{tabular}
\end{center}
%

Note that when several versions of a main file and/or of each child file
are to be generated, it may be convenient to set up a |Makefile| or
shell script to automatise the process.

%%%%%%%%%%%%%%%%%%%%%%%%%%%%%%%%%%%%%%%%%%%%%%%%%%%%%%%%%%%%%%%%%%%%%%%%%%%%%%%%
\subsection{Command Line Processing}
\label{sec:commandline}

The effect of redirection files can also be achieved by invoking
the \LaTeX{} compiler with a more elaborate command line.
Most conveniently this should be done as part
of a shell script or a |Makefile|.

When using \textsf{childdoc} in the main file, the following
command lines effectively perform a redirection
(note that depending on the shell being used,
backslashes may have to be doubled: `|\|' $\to$ `|\\|'):
%
\begin{center}
|... -jobname "|\textit{target}|" |\\|"|[\textit{flags}]%
|\input{childdoc.def}\childdocforward[|\textit{main}|]{|\textit{dest}|}"|
\end{center}
%
Here \textit{target} is the name of the output file,
\textit{main} is the name of the main file
and \textit{dest} is the name of the main or child file to be processed
(all filenames without extensions).
The optional argument \textit{main} can be omitted
if \textit{main} matches \textit{dest}.
Optionally, compilation \textit{flags} can be defined via |\def| commands.
This command line makes the \TeX{} engine believe
it is compiling the file \textit{target}
whose content is specified as the latter parameter.
The provided code then forwards the processing to
\textit{main} or \textit{dest} as described in \secref{sec:forward}.

%%%%%%%%%%%%%%%%%%%%%%%%%%%%%%%%%%%%%%%%%%%%%%%%%%%%%%%%%%%%%%%%%%%%%%%%%%%%%%%%
\subsection{Include by Input}
\label{sec:input}

Including child documents by |\include| has some restrictions by design.
Most notably, the content of a child document always occupies
its own set of pages; pages cannot be shared between child documents.
Usually, this behaviour makes perfect sense
because each child document contain an essential part of the document.
However, in some situations it may be desirable to compose
a document from a collection of parts
without having mandatory page breaks between then.
For this case, the package
provides a mechanism to include parts
by |\input| which can also be processed individually.
However, by construction this mechanism
requires manual handling of the content to be output.

%%%%%%%%%%%%%%%%%%%%%%%%%%%%%%%%%%%%%%%%
\DescribeMacro{\ifchilddocmanual}
The main file should be prepared as usual, see \secref{sec:include}.
However, the document body must make a distinction
between processing of an individual part and of the main document, e.g.:
%
\begin{center}
\begin{tabular}{l}
|\ifchilddocmanual|\\
|\input{\childdocname}|\\
|\||else|\\
\textit{document body with }|\input{|\textit{part}|}|\\
|\||fi|
\end{tabular}
\end{center}
%
The conditional |\ifchilddocmanual| is true whenever
a part to be included by |\input| is being compiled,
and the name of the part is stored in |\childdocname|.

%%%%%%%%%%%%%%%%%%%%%%%%%%%%%%%%%%%%%%%%
\DescribeMacro{\childdocby}
Each part to be included by |\input| should start with:
%
\begin{center}
\begin{tabular}{l}
|\input{childdoc.def}|\\
|\childdocby{|\textit{main}|}|\\
\end{tabular}
\end{center}
%
The directive |\childdocby| is similar to |\childdocof|
described in \secref{sec:include},
but the subsequent selection of content must be done manually.
To that end, both |\ifchilddoc| and |\ifchilddocmanual|
will be true upon processing of a part,
and the name of the part is stored in |\childdocname|.
Note that |\jobname| will be set to the filename of the current part
so that each part receives an individual |.aux| file
that does not interfere with the |.aux| file(s) of the main document.
This behaviour can be altered by the alternative form
|\childdocby[*]{|\textit{main}|}| (with a non-empty optional argument)
which uses the |.aux| file of the main document
by setting |\jobname| to \textit{main}.

%%%%%%%%%%%%%%%%%%%%%%%%%%%%%%%%%%%%%%%%%%%%%%%%%%%%%%%%%%%%%%%%%%%%%%%%%%%%%%%%
\subsection{Driver Development}
\label{sec:driver}

The \textsf{childdoc} mechanism can also be use for the development
of definition files such as \LaTeX{} styles or classes.
This case differs from the above setup with multiple parts
included by |\include| in that no |\includeonly| should be invoked.
This can be achieved by starting the include file
(before |\ProvidesPackage|) with:
%
\begin{center}
\begin{tabular}{l}
|\input{childdoc.def}|\\
|\childdocforward{|\textit{main}|}|\\
\end{tabular}
\end{center}
%
or alternatively with:
%
\begin{center}
\begin{tabular}{l}
|\input{childdoc.def}|\\
|\childdocby{|\textit{main}|}|\\
\end{tabular}
\end{center}
%
Both forms have slightly different effects as described above.
The main file is prepared as usual, see \secref{sec:include}.

%%%%%%%%%%%%%%%%%%%%%%%%%%%%%%%%%%%%%%%%%%%%%%%%%%%%%%%%%%%%%%%%%%%%%%%%%%%%%%%%
\subsection{Legacy Detection}
\label{sec:detection}

The directive |\childdocmain| in the main file can detect
whether the complete document or merely a child is to be compiled
even without using the directive |\childdocof|.
This method is deprecated because it is less robust
and there is no compelling reason to use it;
it is merely provided for backward compatibility
and it may be removed in future versions.

If the detection mechanism is to be used,
it is mandatory to correctly specify
the filename of the main file as the argument of |\childdocmain|:
%
\begin{center}
\begin{tabular}{l}
|\input{childdoc.def}|\\
|\childdocmain{|\textit{main}|}|\\
\end{tabular}
\end{center}
%
If |\jobname| does not match the argument \textit{main} of |\childdocmain|,
it is assumed that |\jobname| points to the child file to be compiled.
When using |\childdocmain| with the main file specified as argument,
it suffices to start a child file
with just |\input{|\textit{main}|}|
without loading of the package and using |\childdocof|.
If instead all processing is done
with the appropriate \textsf{childdoc} directives,
the argument of \textit{main} of |\childdocmain| can be empty.

An alternative version of the command line processing described
in \secref{sec:commandline} using the detection mechanism reads:
%
\begin{center}
|... -jobname "|\textit{target}|" "|[\textit{flags}]%
[|\def\jobname{|\textit{dest}|}|]|\input{|\textit{main}|}"|
\end{center}

%%%%%%%%%%%%%%%%%%%%%%%%%%%%%%%%%%%%%%%%%%%%%%%%%%%%%%%%%%%%%%%%%%%%%%%%%%%%%%%%
\subsection{Manual Code}
\label{sec:manual}

In case one cannot be certain whether the definitions file |childdoc.def|
is installed on the target \TeX{} distribution
and one prefers not to ship it,
it is conceivable to paste a few relevant commands into the sources.

To that end, drop all statements |\input{childdoc.def}|
and perform the replacements as outlined below.
Instead of |\childdocmain{|\textit{main}|}| add the following code
to the top of the main file:
%
\begin{center}
\begin{tabular}{l}
|\||ifdefined\childdocname\endinput\||fi\newif\ifchilddoc|\\
|\edef\childdocname{\scantokens\expandafter{\jobname\noexpand}}|\\
|\def\childdocmain{|\textit{main}|}\||ifx\childdocmain\childdocname\||else|\\
|\childdoctrue\includeonly{\childdocname}\let\jobname\childdocmain\||fi|\\
\end{tabular}
\end{center}
%
Instead of |\childdocof{|\textit{main}|}| just include the main file
at the top of each child file:
%
\begin{center}
|\input{|\textit{main}|}|
\end{center}
%
A simple redirection |\childdocforward{|\textit{dest}|}| is achieved by:
%
\begin{center}
|\def\jobname{|\textit{dest}|}\input{\jobname}|
\end{center}
%
The redirection with prefix
|\childdocforwardprefix[|\textit{prefix}|]{|\textit{dest}|}|
is accomplished by:
%
\begin{center}
\begin{tabular}{l}
|{\edef\jobname{\scantokens\expandafter{\jobname\noexpand}}|\\
|\def\redirectjob |\textit{prefix}|#1~~~{\gdef\jobname{|\textit{dest}|#1}}|\\
|\expandafter\redirectjob\jobname~~~}\input{\jobname}|
\end{tabular}
\end{center}

In an alternative approach,
child documents can be compiled by a specific command line
without additional code or specific definitions:
%
\begin{center}
|... -jobname "|\textit{target}|" "|[\textit{flags}]%
|\includeonly{|\textit{dest}|}\input{|\textit{main}|}"|
\end{center}
%

%%%%%%%%%%%%%%%%%%%%%%%%%%%%%%%%%%%%%%%%%%%%%%%%%%%%%%%%%%%%%%%%%%%%%%%%%%%%%%%%
%%%%%%%%%%%%%%%%%%%%%%%%%%%%%%%%%%%%%%%%%%%%%%%%%%%%%%%%%%%%%%%%%%%%%%%%%%%%%%%%
\section{Information}

%%%%%%%%%%%%%%%%%%%%%%%%%%%%%%%%%%%%%%%%%%%%%%%%%%%%%%%%%%%%%%%%%%%%%%%%%%%%%%%%
\subsection{Copyright}

Copyright \copyright{} 2017--2018 Niklas Beisert

This work may be distributed and/or modified under the
conditions of the \LaTeX{} Project Public License, either version 1.3
of this license or (at your option) any later version.
The latest version of this license is in
  \url{http://www.latex-project.org/lppl.txt}
and version 1.3 or later is part of all distributions of \LaTeX{}
version 2005/12/01 or later.

This work has the LPPL maintenance status `maintained'.

The Current Maintainer of this work is Niklas Beisert.

This work consists of the files |README.txt|, |childdoc.ins| and |childdoc.dtx|
as well as the derived files |childdoc.def|, |cdocsamp.tex|
with |cdocsch1.tex|, |cdocsch2.tex|, |cdocspt3.tex|, |cdocspt4.tex|,
|cdocsdrf.tex|, |cdocsfn1.tex|, |cdocsfn2.tex|
as well as |childdoc.pdf|.

%%%%%%%%%%%%%%%%%%%%%%%%%%%%%%%%%%%%%%%%%%%%%%%%%%%%%%%%%%%%%%%%%%%%%%%%%%%%%%%%
\subsection{Files and Installation}

The package consists of the files:
%
\begin{center}
\begin{tabular}{ll}
    |README.txt|   & readme file \\
    |childdoc.ins| & installation file \\
    |childdoc.dtx| & source file \\
    |childdoc.def| & definition file \\
    |cdocsamp.tex| & sample main file \\
    |cdocsch1.tex| & sample include file \\
    |cdocsch2.tex| & sample include file \\
    |cdocspt3.tex| & sample part file \\
    |cdocspt4.tex| & sample part file \\
    |cdocsdrf.tex| & sample redirection file \\
    |cdocsfn1.tex| & sample redirection file \\
    |cdocsfn2.tex| & sample redirection file \\
    |childdoc.pdf| & manual
\end{tabular}
\end{center}
%
The distribution consists of the files
|README.txt|, |childdoc.ins| and |childdoc.dtx|.
%
\begin{itemize}
\item
Run (pdf)\LaTeX{} on |childdoc.dtx|
to compile the manual |childdoc.pdf| (this file).
\item
Run \LaTeX{} on |childdoc.ins| to create the definitions file |childdoc.def|
and the sample |cdocsamp.tex| with include files
|cdocsch1.tex|, |cdocsch2.tex|, |cdocspt3.tex|, |cdocspt4.tex|,
|cdocsdrf.tex|, |cdocsfn1.tex|, |cdocsfn2.tex|.
Then copy the file |childdoc.def| to an appropriate directory of your \LaTeX{}
distribution, e.g.\ \textit{texmf-root}|/tex/latex/childdoc|.
\end{itemize}

%%%%%%%%%%%%%%%%%%%%%%%%%%%%%%%%%%%%%%%%%%%%%%%%%%%%%%%%%%%%%%%%%%%%%%%%%%%%%%%%
\subsection{Related CTAN Packages}

There are several other packages which offer a similar functionality:
%
\begin{itemize}
\item
The packages
\href{http://ctan.org/pkg/docmute}{\textsf{docmute}},
\href{http://ctan.org/pkg/includex}{\textsf{includex}} and
\href{http://ctan.org/pkg/standalone}{\textsf{standalone}}
provide commands to include only the document body of
a child file thus allowing both files to be compiled individually.
\item
The packages \href{http://ctan.org/pkg/subdocs}{\textsf{subdocs}}
and \href{http://ctan.org/pkg/subfiles}{\textsf{subfiles}}
provide structures in which the main and child documents can be
encapsulated and allowing them to be compiled individually.
The inclusion mechanism is different from the conventional |\include|.
\item
The package \href{http://ctan.org/pkg/combine}{\textsf{combine}}
is an elaborate solution to combine several documents into one.
\end{itemize}
%
See also the CTAN topic \href{http://ctan.org/topic/subdocs}{\textsf{subdocs}}
for further related packages.
The present package differs from the above solutions in that
a document structure constructed with the conventional |\include| mechanism
just needs two extra commands at the top of every file
such that all constituent files can be compiled individually.

%%%%%%%%%%%%%%%%%%%%%%%%%%%%%%%%%%%%%%%%%%%%%%%%%%%%%%%%%%%%%%%%%%%%%%%%%%%%%%%%
%\subsection{Feature Suggestions}
%
%The following is a list of features which may be useful for future
%versions of this package:
%%
%\begin{itemize}
%\item
%\ldots
%\end{itemize}

%%%%%%%%%%%%%%%%%%%%%%%%%%%%%%%%%%%%%%%%%%%%%%%%%%%%%%%%%%%%%%%%%%%%%%%%%%%%%%%%
\subsection{Revision History}

%%%%%%%%%%%%%%%%%%%%%%%%%%%%%%%%%%%%%%%%
\paragraph{v2.0:} 2018/12/30

\begin{itemize}
\item
immediate forward processing
\item
added |\childdocby| mechanism
\item
manual restructured
\end{itemize}

%%%%%%%%%%%%%%%%%%%%%%%%%%%%%%%%%%%%%%%%
\paragraph{v1.6:} 2018/01/17

\begin{itemize}
\item
application for development of include files
\item
corrections to manual
\end{itemize}

%%%%%%%%%%%%%%%%%%%%%%%%%%%%%%%%%%%%%%%%
\paragraph{v1.5:} 2017/05/21

\begin{itemize}
\item
more complete structuring introduced
\item
|\childdocof| introduced
\item
|\childdoc| renamed to |\childdocmain|
\item
|\childredirect| renamed to |\childdocforward| and |\childdocforwardprefix|
and functionality expanded
\end{itemize}

%%%%%%%%%%%%%%%%%%%%%%%%%%%%%%%%%%%%%%%%
\paragraph{v1.0:} 2017/04/27

\begin{itemize}
\item
manual and install package
\item
first version published on CTAN
\end{itemize}

%%%%%%%%%%%%%%%%%%%%%%%%%%%%%%%%%%%%%%%%
\paragraph{v0.6:} 2017/04/26

\begin{itemize}
\item
redirection mechanism added
\end{itemize}

%%%%%%%%%%%%%%%%%%%%%%%%%%%%%%%%%%%%%%%%
\paragraph{v0.5:} 2017/04/26

\begin{itemize}
\item
functionality in definition file
\end{itemize}


%%%%%%%%%%%%%%%%%%%%%%%%%%%%%%%%%%%%%%%%%%%%%%%%%%%%%%%%%%%%%%%%%%%%%%%%%%%%%%%%
%%%%%%%%%%%%%%%%%%%%%%%%%%%%%%%%%%%%%%%%%%%%%%%%%%%%%%%%%%%%%%%%%%%%%%%%%%%%%%%%
%%%%%%%%%%%%%%%%%%%%%%%%%%%%%%%%%%%%%%%%%%%%%%%%%%%%%%%%%%%%%%%%%%%%%%%%%%%%%%%%
\appendix

\settowidth\MacroIndent{\rmfamily\scriptsize 000\ }

 \DocInput{childdoc.dtx}

\end{document}
%</driver>
% \fi
%
% %%%%%%%%%%%%%%%%%%%%%%%%%%%%%%%%%%%%%%%%%%%%%%%%%%%%%%%%%%%%%%%%%%%%%%%%%%%%%%
% %%%%%%%%%%%%%%%%%%%%%%%%%%%%%%%%%%%%%%%%%%%%%%%%%%%%%%%%%%%%%%%%%%%%%%%%%%%%%%
% \section{Sample}
%\iffalse
%<*samplemain>
%\fi
%
% The following presents a sample document
% with two chapters, two parts, a title page,
% a compile flag as well as three forwarding files to set the flag.
% It consists of eight |.tex| files:
% \begin{center}
% \begin{tabular}{ll}
% |cdocsamp.tex|&main file\\
% |cdocsch1.tex|&include file for chapter 1\\
% |cdocsch2.tex|&include file for chapter 2\\
% |cdocspt3.tex|&include file for part 3\\
% |cdocspt4.tex|&include file for part 4\\
% |cdocsdrf.tex|&forwarding file for main file in draft mode\\
% |cdocsfi1.tex|&forwarding file for final version of chapter 1\\
% |cdocsfi2.tex|&forwarding file for final version of chapter 2\\
% \end{tabular}
% \end{center}
% Each of the eight files can be compiled directly by the \LaTeX{} compiler.
%
% %%%%%%%%%%%%%%%%%%%%%%%%%%%%%%%%%%%%%%
% \paragraph{Main File.}
%
% The main file is called |cdocsamp.tex|.
%
% Load the \textsf{childdoc} definitions and
% declare the filename for the main document:
%    \begin{macrocode}
\input{childdoc.def}
\childdocmain{}
%    \end{macrocode}

% Optional override for |\version| flag:
%    \begin{macrocode}
%%\ifchilddoc\else\providecommand{\version}{draft}\fi
%    \end{macrocode}

% Define the default values for the |\version| flag
% (|final| for the main file and |draft| for childs):
%    \begin{macrocode}
\ifchilddoc
\providecommand{\version}{draft}
\else
\providecommand{\version}{final}
\fi
%    \end{macrocode}

% Load the standard document class:
%    \begin{macrocode}
\documentclass[12pt]{article}
%    \end{macrocode}

% Start the document body:
%    \begin{macrocode}
\begin{document}
%    \end{macrocode}

% Declare a title page.
% Print title, part of document being processed and version flag:
%    \begin{macrocode}
\addtocounter{page}{-1}
\begin{center}
{\LARGE\bfseries{}childdoc example\par}
\vspace{1cm}
\ifchilddoc
\ifchilddocmanual part\else chapter\fi:
`\childdocname' of `\childdocjob'\par
\else
main document: `\childdocjob'\par
\fi
version: \version\par
\end{center}
\newpage
%    \end{macrocode}

% Manually include selected file,
% otherwise process as usual:
%    \begin{macrocode}
\ifchilddocmanual
\section*{part `\childdocname'}
\input{\childdocname}
\else
%    \end{macrocode}

% Include the two chapters:
%    \begin{macrocode}
\include{cdocsch1}
\include{cdocsch2}
%    \end{macrocode}

% Include the two parts unless only chapters should be displayed:
%    \begin{macrocode}
\ifchilddoc\else
\section{part three}
\input{cdocspt3}
\section{part four}
\input{cdocspt4}
\fi
%    \end{macrocode}

% Process as usual until here:
%    \begin{macrocode}
\fi
%    \end{macrocode}

% End of document body:
%    \begin{macrocode}
\end{document}
%    \end{macrocode}
%\iffalse
%</samplemain>
%\fi
%
% %%%%%%%%%%%%%%%%%%%%%%%%%%%%%%%%%%%%%%
% \paragraph{Chapter Include Files.}
%
% The include files are called |cdocsch1.tex| and |cdocsch2.tex|.
%
%\iffalse
%<*samplechap1|samplechap2>
%\fi

% Optional override for |\version| flag:
%    \begin{macrocode}
%%\providecommand{\version}{final}
%    \end{macrocode}

% Include the main document:
%    \begin{macrocode}
\input{childdoc.def}
\childdocof{cdocsamp}
%    \end{macrocode}

%\iffalse
%</samplechap1|samplechap2>
%\fi
%
%\iffalse
%<*samplechap1>
%\fi
% Some text for chapter 1:
%    \begin{macrocode}
\section{one}
some text in chapter one
%    \end{macrocode}

%\iffalse
%</samplechap1>
%\fi
% Some text for chapter 2:
%\iffalse
%<*samplechap2>
%\fi
%    \begin{macrocode}
\section{two}
more text in chapter two
%    \end{macrocode}

%\iffalse
%</samplechap2>
%\fi
%
% %%%%%%%%%%%%%%%%%%%%%%%%%%%%%%%%%%%%%%
% \paragraph{Part Include Files.}
%
% The include files are called |cdocspt3.tex| and |cdocspt4.tex|.
%
%\iffalse
%<*samplepart3|samplepart4>
%\fi

% Optional override for |\version| flag:
%    \begin{macrocode}
%%\providecommand{\version}{final}
%    \end{macrocode}

% Include the main document:
%    \begin{macrocode}
\input{childdoc.def}
\childdocby{cdocsamp}
%    \end{macrocode}

%\iffalse
%</samplepart3|samplepart4>
%\fi
%
%\iffalse
%<*samplepart3>
%\fi
% Some text for part 3:
%    \begin{macrocode}
some text in part three
%    \end{macrocode}

%\iffalse
%</samplepart3>
%\fi
% Some text for part 4:
%\iffalse
%<*samplepart4>
%\fi
%    \begin{macrocode}
more text in part four
%    \end{macrocode}

%\iffalse
%</samplepart4>
%\fi
%
% %%%%%%%%%%%%%%%%%%%%%%%%%%%%%%%%%%%%%%
% \paragraph{Forwarding for a Complete Draft.}
%
% The following forwarding file |cdocsdrf.tex|
% compiles the main document in draft mode:
%\iffalse
%<*sampledraft>
%\fi
%    \begin{macrocode}
\def\version{draft}
\input{childdoc.def}
\childdocforward{cdocsamp}
%    \end{macrocode}

%\iffalse
%</sampledraft>
%\fi
%
% %%%%%%%%%%%%%%%%%%%%%%%%%%%%%%%%%%%%%%
% \paragraph{Forwarding for Final Version of the Chapters.}
%
% The following forwarding files |cdocsfn1.tex| and |cdocsfn2.tex|
% (with identical content)
% compile the final versions of the child documents
% |cdocsch1.tex| and |cdocsch2.tex|, respectively:
%\iffalse
%<*samplefinal>
%\fi
%    \begin{macrocode}
\def\version{final}
\input{childdoc.def}
\childdocforwardprefix[cdocsamp]{cdocsfn}{cdocsch}
%    \end{macrocode}

%\iffalse
%</samplefinal>
%\fi
%
% %%%%%%%%%%%%%%%%%%%%%%%%%%%%%%%%%%%%%%
% \paragraph{Command Line Processing.}
%
% The following three command lines generate the output files
% |cdocscld|, |cdocscl1| and |cdocscl2|
% which should be identical to
% |cdocsdrf|, |cdocsch1| and |cdocsfn2|, respectively:
% \begin{center}
% \begin{tabular}{l}
% |latex -jobname cdocscld \|\\
% |  "\def\version{draft}\input{childdoc.def}\childdocforward{cdocsamp}"|\\
% |latex -jobname cdocscl1 \|\\
% |  "\input{childdoc.def}\childdocforward[cdocsamp]{cdocsch1}"|\\
% |latex -jobname cdocscl2 \|\\
% |  "\def\version{final}\input{childdoc.def}\childdocforward{cdocsch2}"|
% \end{tabular}
% \end{center}
% Note that the trailing backslash on each first line
% merely continues the input to the second line
% (for convenient cut ant paste).
% Furthermore, the command |latex| can be replaced by any
% of its alternative versions such as |pdflatex|.
%
% %%%%%%%%%%%%%%%%%%%%%%%%%%%%%%%%%%%%%%%%%%%%%%%%%%%%%%%%%%%%%%%%%%%%%%%%%%%%%%
% %%%%%%%%%%%%%%%%%%%%%%%%%%%%%%%%%%%%%%%%%%%%%%%%%%%%%%%%%%%%%%%%%%%%%%%%%%%%%%
% \section{Implementation}
%\iffalse
%<*package>
%\fi
%
% This section describes the definitions file |childdoc.def|.

% The definitions cannot be loaded using |\usepackage| or |\RequirePackage|
% which has a mechanism to prevent loading a style file more than once.
% When loading the definitions by means of |\input|
% multiple instances have to be prevented manually:
%\iffalse
%This code needs to be before the `\ProvidesFile' directive
%which is defined at the beginning of this file.
%Therefore it is also placed there and commented out here.
%</package>
%<*discard>
%\fi
%    \begin{macrocode}
\ifdefined\childdocmain\endinput\fi
%    \end{macrocode}
%\iffalse
%</discard>
%<*package>
%\fi
%
% \macro{\ifchilddoc}
% \macro{\ifchilddocmanual}
% The conditional |\ifchilddoc| tells whether a
% child (true) or main (false) document is being compiled.
% The conditional |\ifchilddocmanual| tells whether
% the |\includeonly| mechanism is used (false) or
% the selection of child files must be performed manually (true).
% The definitions initialise to false:
%    \begin{macrocode}
\newif\ifchilddoc
\newif\ifchilddocmanual
%    \end{macrocode}

% \macro{\childdocname}
% \macro{\childdocjob}
% The macro |\childdocname| stores the name of the main document
% to be compiled. The macro |\childdocjob| stores the name of
% the document on which the \LaTeX{} compiler was originally invoked.
% The content of |\jobname| cannot be compared
% to filenames specified in the source due to different catcodes.
% The following code rescans |\jobname|, stores the result
% in |\childdocname| and saves a copy in |\childdocjob|:
%    \begin{macrocode}
\edef\childdocname{\scantokens\expandafter{\jobname\noexpand}}
\let\childdocjob\childdocname
%    \end{macrocode}

% \macro{\childdocdisable}
% The macro |\childdocdisable| prevents the main file
% from being processed more than once.
% At this stage, the main document command |\childdocmain|
% is assumed to be called once again where it should do nothing.
% Any subsequent call to it should prevent
% a secondary processing of the main document
% It overwrites the forwarding commands
% |\childdocof| and |\childdocforward|
% with empty macros to prevent further inclusions of the main document:
%    \begin{macrocode}
\newcommand{\childdocdisable}
{
  \renewcommand{\childdocmain}[1]{\renewcommand{\childdocmain}[1]{\endinput}}
  \renewcommand{\childdocof}[1]{}
  \renewcommand{\childdocby}[2][]{}
  \renewcommand{\childdocforward}[2][]{}
  \renewcommand{\childdocdisable}{}
}
%    \end{macrocode}

% \macro{\childdocmain}
% The macro |\childdocmain| is to be called at the top of the main file
% with nothing or the main filename (without extension) as argument.
% First, it breaks loops.
% If the argument is not empty and does not match |\childdocname|
% (which is set by the first inclusion of |childdoc.def|),
% |\ifchilddoc| is set to true, |\includeonly| is applied to the child file
% and |\jobname| is set to the main file
% (for proper handling of |.aux| files):
%    \begin{macrocode}
\newcommand{\childdocmain}[1]
{
  \childdocdisable\childdocmain{}
  \if?#1?\else
    \begingroup
      \def\childdoctmp{#1}
      \ifx\childdoctmp\childdocname
        \def\childdoctmp{}
      \else
        \def\childdoctmp
        {
          \childdoctrue
          \includeonly{\childdocname}
          \def\childdocjob{#1}
          \def\jobname{#1}
        }
      \fi
      \expandafter
    \endgroup
    \childdoctmp
  \fi
}
%    \end{macrocode}

% \macro{\childdocof}
% The command |\childdocof| redirects
% compilation to the main file |#1|.
%    \begin{macrocode}
\newcommand{\childdocof}[1]
{
  \childdocdisable
  \childdoctrue
  \includeonly{\childdocname}
  \def\jobname{#1}
  \def\childdocjob{#1}
  \input{#1}
}
%    \end{macrocode}

% \macro{\childdocby}
% The command |\childdocby| ....
%    \begin{macrocode}
\newcommand{\childdocby}[2][]
{
  \childdocdisable
  \childdoctrue
  \childdocmanualtrue
  \if?#1?\else
    \def\jobname{#2}
  \fi
  \def\childdocjob{#2}
  \input{#2}
  \endinput
}
%    \end{macrocode}

% \macro{\childdocforward}
% The command |\childdocforward| redirects
% compilation to the main file or
% (if the optional argument is given) a child file.
% Parameters are set as if the main file
% or a child file starting with |\childdocof| was compiled.
% Then compilation is handed over to the main file:
%    \begin{macrocode}
\newcommand{\childdocforward}[2][]
{
  \begingroup
    \if?#1?
      \def\childdoctmp
      {
        \def\childdocname{#2}
        \def\childdocjob{#2}
        \def\jobname{#2}
        \input{#2}
        \endinput
      }
    \else
      \def\childdoctmp
      {
        \childdocdisable
        \def\childdocname{#2}
        \childdoctrue
        \includeonly{#2}
        \def\childdocjob{#1}
        \def\jobname{#1}
        \input{#1}
        \endinput
      }
    \fi
    \expandafter
  \endgroup
  \childdoctmp
}
%    \end{macrocode}

% \macro{\childdocforwardprefix}
% The command |\childdocforwardprefix| redirects
% compilation to the main or a child file by means of a pattern.
% The prefix |#1| in the current filename is replaced by |#2|
% and the suffix of the current filename is kept
% (it is assumed that the filename does not contain the substring `|~~~|'
% which is used as a delimiter).
% Compilation is handed over to the new file by |\childdocforward|:
%    \begin{macrocode}
\newcommand{\childdocforwardprefix}[3][]
{
  \begingroup
    \def\childdocextract #2##1~~~{\def\childdoctmp{\childdocforward[#1]{#3##1}}}
    \expandafter\childdocextract\childdocname~~~
    \expandafter
  \endgroup
  \childdoctmp
}
%    \end{macrocode}

% \macro{\childdoc}
% The deprecated macro |\childdoc| is a legacy version of |\childdocmain|:
%    \begin{macrocode}
\newcommand{\childdoc}{\childdocmain}
%    \end{macrocode}

% \macro{\childdocredirect}
% The deprecated macro |\childdocredirect| is a legacy version
% of |\childdocforward| and |\childdocforwardprefix|:
%    \begin{macrocode}
\newcommand{\childdocredirect}[2][]
{
  \begingroup
    \if?#1?
      \def\childdoctmp{\childdocforward{#2}}
    \else
      \def\childdoctmp{\childdocforwardprefix{#1}{#2}}
    \fi
    \expandafter
  \endgroup
  \childdoctmp
}
%    \end{macrocode}

%\iffalse
%</package>
%\fi
%
\endinput
|\\
|\childdocof{|\textit{main}|}|\\
\end{tabular}
\end{center}
at the top of every child file \textit{child}
which is included by |\include{|\textit{child}|}|
from within the main file
(or at least for those files to be compiled individually).
The argument \textit{main} must be the filename of the main file.

There are a couple of
considerations in setting up the main and child documents:

%%%%%%%%%%%%%%%%%%%%%%%%%%%%%%%%%%%%%%%%
\paragraph{Restrictions.}

Please note the following restrictions:
\begin{itemize}
\item
|\childdocmain| must be called with one argument \textit{main}
to ensure compatibility with earlier version of the package.
It must either be empty (|\childdocmain{}|)
or precisely match the filename of the main file in which it is specified.
See \secref{sec:detection} for further information.
\item
The filename \textit{main} must be specified without the |.tex| extension.
\item
The filename \textit{main} is case sensitive
(even in case-insensitive file systems)
due to internal string comparison.
\item
The argument \textit{main} should be fully expanded, it cannot be a macro.
\item
Subdirectories and special characters should be avoided in filenames.
\item
The command |\childdocmain{|\textit{main}|}| must be followed by a whitespace.
It should not be followed immediately by another command
or by a comment mark `|%|'.
This is because the \TeX{} parser reads the token immediately following
the argument of |\childdocmain| and puts it
at the beginning of every child section;
however, a white\-space is ignored.
\end{itemize}

%%%%%%%%%%%%%%%%%%%%%%%%%%%%%%%%%%%%%%%%
\paragraph{Content of Main File.}

It is advisable to place all content in the child files included by |\include|.
Any output contained in the main file will appear in all child documents
unless suppressed manually;
it cannot be suppressed automatically by the |\includeonly| directive
and thus should normally be avoided.
A method to include some content in the main file
by means of conditional processing is described in \secref{sec:conditional}.

%%%%%%%%%%%%%%%%%%%%%%%%%%%%%%%%%%%%%%%%
\paragraph{Page Numbering.}

When only a part of the document is compiled,
the appropriate numbering of pages
(as well as other status parameters)
is determined from the |.aux| files.
The latter contain information from previous passes.
However this information needs to propagate through
all intermediate child documents.
Therefore the page numbering in child documents may well
be inconsistent until the complete document is compiled at least once.

A useful (if unconventional) way to always ensure a consistent
page numbering is to restart the numbering in each child document
and denote the pages by `\textit{child}|.|\textit{page}'
where \textit{child} represents the chapter/section number of the child file.
This can be achieved by the command
|\numberwithin{page}{|\textit{child}|}|
of the \textsf{amsmath} package
where \textit{child} can be |chapter| or |section|
depending on the chosen structuring.
Alternatively, one can modify the macro |\thepage| appropriately
and reset the counter |page| at the start of each child file.

%%%%%%%%%%%%%%%%%%%%%%%%%%%%%%%%%%%%%%%%%%%%%%%%%%%%%%%%%%%%%%%%%%%%%%%%%%%%%%%%
\subsection{Conditional Processing}
\label{sec:conditional}

The package provides a mechanism to compile different versions
of a document. To customise the versions further some conditional processing
can come in handy to distinguish which version is being compiled.
The package provides two macros to describe the compilation context:

%%%%%%%%%%%%%%%%%%%%%%%%%%%%%%%%%%%%%%%%
\DescribeMacro{\ifchilddoc}
The conditional |\ifchilddoc| distinguishes between the compilation of
child documents and the main document:
%
\begin{center}
|\ifchilddoc |\textit{child-code}| |[|\||else |\textit{main-code}]| \||fi|
\end{center}

%%%%%%%%%%%%%%%%%%%%%%%%%%%%%%%%%%%%%%%%
\DescribeMacro{\childdocname}
\DescribeMacro{\childdocjob}
The macro |\childdocname| contains the filename (without extension)
of the main or child file being processed.
Note that |\childdocjob| will always contain the name of the main file.

%%%%%%%%%%%%%%%%%%%%%%%%%%%%%%%%%%%%%%%%
\paragraph{Title Page.}

Conditional processing can be used to include a title or banner page
in the main document when proper precautions are taken.
Importantly, the code in the main file should ensure that the page counter
(as well as other status parameters which are stored in the |.aux| files)
takes the same value after the conditional processing.
Otherwise the page numbers may take divergent values
depending on which part is compiled.

For example, a title page could be declared by:
%
\begin{center}
\begin{tabular}{l}
|\ifchilddoc\||else|\\
|\addtocounter{page}{-1}|\\
\textit{code for title page}\\
|\newpage|\\
|\||fi|
\end{tabular}
\end{center}
%
A banner page for the child documents can be generated by:
%
\begin{center}
\begin{tabular}{l}
|\ifchilddoc|\\
|\addtocounter{page}{-1}|\\
\textit{code for banner page}\\
|\newpage|\\
|\||fi|
\end{tabular}
\end{center}
%
Here one could write a message such as:
\begin{center}
|This is the part \childdocname{} of \childdocjob{}.|
\end{center}

%%%%%%%%%%%%%%%%%%%%%%%%%%%%%%%%%%%%%%%%%%%%%%%%%%%%%%%%%%%%%%%%%%%%%%%%%%%%%%%%
\subsection{Flags}
\label{sec:flags}

The package makes it easy to generate different versions
of the main or child documents.
To this end compilation flags can be defined
and assigned different default values.
They will be particularly useful in conjunction
with the forwarding mechanism described in \secref{sec:forward}.

For example, it may be useful to have a flag |\version|
which can be set to |draft| or |final|.
The document source will contain some conditional code
depending on the value of |\version|.
Suppose further, the flag should default to |final| for the main file
and to |draft| for child files
which is a natural assignment for editing the document.
This is achieved by placing the following code
in the preamble of the main document
(below the |\childdocmain| directive):
%
\begin{center}
\begin{tabular}{l}
|\ifchilddoc|\\
|\providecommand{\version}{draft}|\\
|\||else|\\
|\providecommand{\version}{final}|\\
|\||fi|
\end{tabular}
\end{center}
%
The definition by |\providecommand| makes sure
that previous definitions are not overwritten.
Further statements |\providecommand{\version}{...}|
can thus be added before the above code to override it.

For the main file, one might add a line
(between |\childdocmain| and the above block)
%
\begin{center}
|%\ifchilddoc\||else\providecommand{\version}{draft}\||fi|
\end{center}
%
which can be uncommented to produce a draft version.
Likewise one can add a line to the very top of a child file
(above the |\childdocof{|\textit{main}|}| directive)
%
\begin{center}
|%\providecommand{\version}{final}|
\end{center}
%
which can be uncommented to produce the final version of this child document.

%%%%%%%%%%%%%%%%%%%%%%%%%%%%%%%%%%%%%%%%%%%%%%%%%%%%%%%%%%%%%%%%%%%%%%%%%%%%%%%%
\subsection{Forwarding}
\label{sec:forward}

Different versions of the main or child documents
using compilation flags as described in \secref{sec:flags}
can be (permanently) stored in different files
for convenient compilation, viewing and distribution.
To this end, the package defines a command
to pass on compilation to a different file:

%%%%%%%%%%%%%%%%%%%%%%%%%%%%%%%%%%%%%%%%
\DescribeMacro{\childdocforward}
The command |\childdocforward| redirects processing to
another source file:
%
\begin{center}
\begin{tabular}{l}
|% \iffalse
%
% childdoc.dtx Copyright (C) 2017-2018 Niklas Beisert
%
% This work may be distributed and/or modified under the
% conditions of the LaTeX Project Public License, either version 1.3
% of this license or (at your option) any later version.
% The latest version of this license is in
%   http://www.latex-project.org/lppl.txt
% and version 1.3 or later is part of all distributions of LaTeX
% version 2005/12/01 or later.
%
% This work has the LPPL maintenance status `maintained'.
%
% The Current Maintainer of this work is Niklas Beisert.
%
% This work consists of the files childdoc.dtx and childdoc.ins
% and the derived files childdoc.def and cdocsamp.tex with
% cdocsch1.tex, cdocsch2.tex, cdocsdrf.tex, cdocsfn1.tex, cdocsfn2.tex.
%
%<package>\ifdefined\childdocmain\endinput\fi
%<package>\ProvidesFile{childdoc.def}[2018/12/30 v2.0 child document driver]
%<samplemain>\ProvidesFile{cdocsamp.tex}[2018/12/30 v2.0 sample for childdoc]
%<*driver>
%\ProvidesFile{childdoc.drv}[2018/12/30 v2.0 childdoc reference manual file]
\PassOptionsToClass{10pt,a4paper}{article}
\documentclass{ltxdoc}

\usepackage[margin=35mm]{geometry}
\usepackage{hyperref}
\usepackage{hyperxmp}
\usepackage[usenames]{color}

\hypersetup{colorlinks=true}
\hypersetup{pdfstartview=FitH}
\hypersetup{pdfpagemode=UseNone}
\hypersetup{pdfsource={}}
\hypersetup{pdflang={en-UK}}
\hypersetup{pdfcopyright={Copyright 2017-2018 Niklas Beisert.
  This work may be distributed and/or modified under the
  conditions of the LaTeX Project Public License, either version 1.3
  of this license or (at your option) any later version.}}
\hypersetup{pdflicenseurl={http://www.latex-project.org/lppl.txt}}
\hypersetup{pdfcontactaddress={ETH Zurich, ITP, HIT K,
  Wolfgang-Pauli-Strasse 27}}
\hypersetup{pdfcontactpostcode={8093}}
\hypersetup{pdfcontactcity={Zurich}}
\hypersetup{pdfcontactcountry={Switzerland}}
\hypersetup{pdfcontactemail={nbeisert@itp.phys.ethz.ch}}
\hypersetup{pdfcontacturl={http://people.phys.ethz.ch/\xmptilde nbeisert/}}

\newcommand{\secref}[1]{\hyperref[#1]{section \ref*{#1}}}

\parskip1ex
\parindent0pt
\let\olditemize\itemize
\def\itemize{\olditemize\parskip0pt}

\begin{document}

\title{The \textsf{childdoc} Package}
\hypersetup{pdftitle={The childdoc Package}}
\author{Niklas Beisert\\[2ex]
  Institut f\"ur Theoretische Physik\\
  Eidgen\"ossische Technische Hochschule Z\"urich\\
  Wolfgang-Pauli-Strasse 27, 8093 Z\"urich, Switzerland\\[1ex]
  \href{mailto:nbeisert@itp.phys.ethz.ch}
  {\texttt{nbeisert@itp.phys.ethz.ch}}}
\hypersetup{pdfauthor={Niklas Beisert}}
\hypersetup{pdfsubject={Manual for the LaTeX2e Package childdoc}}
\date{30 December 2018, \textsf{v2.0}}
\maketitle

\begin{abstract}\noindent
\textsf{childdoc} is a \LaTeXe{} package
that enables the direct compilation
of document sections included by |\include|
to individual files.
\end{abstract}

\begingroup
\parskip0ex
\tableofcontents
\endgroup

%%%%%%%%%%%%%%%%%%%%%%%%%%%%%%%%%%%%%%%%%%%%%%%%%%%%%%%%%%%%%%%%%%%%%%%%%%%%%%%%
%%%%%%%%%%%%%%%%%%%%%%%%%%%%%%%%%%%%%%%%%%%%%%%%%%%%%%%%%%%%%%%%%%%%%%%%%%%%%%%%
\section{Introduction}

\LaTeX{} provides a mechanism to structure a large document (such as a book)
into a main file and several child files (containing the chapters)
using the |\include| command.
This mechanism is beneficial for documents
which span hundreds of pages in order to
make the source file(s) more manageable.
Moreover, compilation can be restricted to
selected child files by means of the |\includeonly| command.
The latter feature can be used to reduce the compilation time while editing
(this was significantly more useful in the earlier days of \LaTeX{})
or to generate a smaller document which is easier to navigate.
Another application of |\includeonly| is to generate
documents consisting of selected parts of the complete document.

However, there are a few drawbacks of the plain |\include| mechanism:
\begin{itemize}
\item
The child files cannot be compiled on their own,
they can only be compiled via the main file.
A naive editing environment
(such as a text editor with an option
to have the current file processed by \LaTeX)
may require one to switch to the main file before compiling;
attempting to compile the child file produces errors.
\item
The main file must be modified (each time)
to adjust the |\includeonly| command
to the present needs. This easily leaves the main file in a messy state.
\item
The generated document will always carry the filename
of the main document. This is inconvenient if
several child files are to be compiled and
to be kept for distribution.
\end{itemize}

The present package provides a simple interface
to make child files individually compilable by \LaTeX{}.
Compiling a child file then has the same effect as compiling
the main file with an |\includeonly| command
to select the appropriate child.
Moreover the generated document will carry the name of the child
rather than the main file.
This resolves all three above issues.

This feature is meant to make the editing of books,
thesis documents and lecture notes somewhat more convenient.
However, the package can also be used efficiently for
composing a series of documents (such as exercise sheets)
which are typically distributed individually.
It then assists the author in generating the individual documents
(potentially in different versions)
as well as a document containing the collected series.
Another application is in developing style files
or other kinds of included material
where compilation of the style file could redirect
to a sample or test file.

%%%%%%%%%%%%%%%%%%%%%%%%%%%%%%%%%%%%%%%%%%%%%%%%%%%%%%%%%%%%%%%%%%%%%%%%%%%%%%%%
%%%%%%%%%%%%%%%%%%%%%%%%%%%%%%%%%%%%%%%%%%%%%%%%%%%%%%%%%%%%%%%%%%%%%%%%%%%%%%%%
\section{Usage}

First of all, the package \textsf{childdoc} is \emph{not} a standard
\LaTeXe{} |.sty| style file! Therefore it needs to be invoked in
a non-standard way.

%%%%%%%%%%%%%%%%%%%%%%%%%%%%%%%%%%%%%%%%%%%%%%%%%%%%%%%%%%%%%%%%%%%%%%%%%%%%%%%%
\subsection{Included Files}
\label{sec:include}

%%%%%%%%%%%%%%%%%%%%%%%%%%%%%%%%%%%%%%%%
\DescribeMacro{\childdocmain}
To use the package, add the commands
\begin{center}
\begin{tabular}{l}
|\input{childdoc.def}|\\
|\childdocmain{}|\\
\end{tabular}
\end{center}
at the very top of the main \LaTeX{} file,
in particular \emph{before} the |\documentclass| statement!
The argument of |\childdocmain| should be left empty
(but it must be present).

%%%%%%%%%%%%%%%%%%%%%%%%%%%%%%%%%%%%%%%%
\DescribeMacro{\childdocof}
Furthermore, add the commands
\begin{center}
\begin{tabular}{l}
|\input{childdoc.def}|\\
|\childdocof{|\textit{main}|}|\\
\end{tabular}
\end{center}
at the top of every child file \textit{child}
which is included by |\include{|\textit{child}|}|
from within the main file
(or at least for those files to be compiled individually).
The argument \textit{main} must be the filename of the main file.

There are a couple of
considerations in setting up the main and child documents:

%%%%%%%%%%%%%%%%%%%%%%%%%%%%%%%%%%%%%%%%
\paragraph{Restrictions.}

Please note the following restrictions:
\begin{itemize}
\item
|\childdocmain| must be called with one argument \textit{main}
to ensure compatibility with earlier version of the package.
It must either be empty (|\childdocmain{}|)
or precisely match the filename of the main file in which it is specified.
See \secref{sec:detection} for further information.
\item
The filename \textit{main} must be specified without the |.tex| extension.
\item
The filename \textit{main} is case sensitive
(even in case-insensitive file systems)
due to internal string comparison.
\item
The argument \textit{main} should be fully expanded, it cannot be a macro.
\item
Subdirectories and special characters should be avoided in filenames.
\item
The command |\childdocmain{|\textit{main}|}| must be followed by a whitespace.
It should not be followed immediately by another command
or by a comment mark `|%|'.
This is because the \TeX{} parser reads the token immediately following
the argument of |\childdocmain| and puts it
at the beginning of every child section;
however, a white\-space is ignored.
\end{itemize}

%%%%%%%%%%%%%%%%%%%%%%%%%%%%%%%%%%%%%%%%
\paragraph{Content of Main File.}

It is advisable to place all content in the child files included by |\include|.
Any output contained in the main file will appear in all child documents
unless suppressed manually;
it cannot be suppressed automatically by the |\includeonly| directive
and thus should normally be avoided.
A method to include some content in the main file
by means of conditional processing is described in \secref{sec:conditional}.

%%%%%%%%%%%%%%%%%%%%%%%%%%%%%%%%%%%%%%%%
\paragraph{Page Numbering.}

When only a part of the document is compiled,
the appropriate numbering of pages
(as well as other status parameters)
is determined from the |.aux| files.
The latter contain information from previous passes.
However this information needs to propagate through
all intermediate child documents.
Therefore the page numbering in child documents may well
be inconsistent until the complete document is compiled at least once.

A useful (if unconventional) way to always ensure a consistent
page numbering is to restart the numbering in each child document
and denote the pages by `\textit{child}|.|\textit{page}'
where \textit{child} represents the chapter/section number of the child file.
This can be achieved by the command
|\numberwithin{page}{|\textit{child}|}|
of the \textsf{amsmath} package
where \textit{child} can be |chapter| or |section|
depending on the chosen structuring.
Alternatively, one can modify the macro |\thepage| appropriately
and reset the counter |page| at the start of each child file.

%%%%%%%%%%%%%%%%%%%%%%%%%%%%%%%%%%%%%%%%%%%%%%%%%%%%%%%%%%%%%%%%%%%%%%%%%%%%%%%%
\subsection{Conditional Processing}
\label{sec:conditional}

The package provides a mechanism to compile different versions
of a document. To customise the versions further some conditional processing
can come in handy to distinguish which version is being compiled.
The package provides two macros to describe the compilation context:

%%%%%%%%%%%%%%%%%%%%%%%%%%%%%%%%%%%%%%%%
\DescribeMacro{\ifchilddoc}
The conditional |\ifchilddoc| distinguishes between the compilation of
child documents and the main document:
%
\begin{center}
|\ifchilddoc |\textit{child-code}| |[|\||else |\textit{main-code}]| \||fi|
\end{center}

%%%%%%%%%%%%%%%%%%%%%%%%%%%%%%%%%%%%%%%%
\DescribeMacro{\childdocname}
\DescribeMacro{\childdocjob}
The macro |\childdocname| contains the filename (without extension)
of the main or child file being processed.
Note that |\childdocjob| will always contain the name of the main file.

%%%%%%%%%%%%%%%%%%%%%%%%%%%%%%%%%%%%%%%%
\paragraph{Title Page.}

Conditional processing can be used to include a title or banner page
in the main document when proper precautions are taken.
Importantly, the code in the main file should ensure that the page counter
(as well as other status parameters which are stored in the |.aux| files)
takes the same value after the conditional processing.
Otherwise the page numbers may take divergent values
depending on which part is compiled.

For example, a title page could be declared by:
%
\begin{center}
\begin{tabular}{l}
|\ifchilddoc\||else|\\
|\addtocounter{page}{-1}|\\
\textit{code for title page}\\
|\newpage|\\
|\||fi|
\end{tabular}
\end{center}
%
A banner page for the child documents can be generated by:
%
\begin{center}
\begin{tabular}{l}
|\ifchilddoc|\\
|\addtocounter{page}{-1}|\\
\textit{code for banner page}\\
|\newpage|\\
|\||fi|
\end{tabular}
\end{center}
%
Here one could write a message such as:
\begin{center}
|This is the part \childdocname{} of \childdocjob{}.|
\end{center}

%%%%%%%%%%%%%%%%%%%%%%%%%%%%%%%%%%%%%%%%%%%%%%%%%%%%%%%%%%%%%%%%%%%%%%%%%%%%%%%%
\subsection{Flags}
\label{sec:flags}

The package makes it easy to generate different versions
of the main or child documents.
To this end compilation flags can be defined
and assigned different default values.
They will be particularly useful in conjunction
with the forwarding mechanism described in \secref{sec:forward}.

For example, it may be useful to have a flag |\version|
which can be set to |draft| or |final|.
The document source will contain some conditional code
depending on the value of |\version|.
Suppose further, the flag should default to |final| for the main file
and to |draft| for child files
which is a natural assignment for editing the document.
This is achieved by placing the following code
in the preamble of the main document
(below the |\childdocmain| directive):
%
\begin{center}
\begin{tabular}{l}
|\ifchilddoc|\\
|\providecommand{\version}{draft}|\\
|\||else|\\
|\providecommand{\version}{final}|\\
|\||fi|
\end{tabular}
\end{center}
%
The definition by |\providecommand| makes sure
that previous definitions are not overwritten.
Further statements |\providecommand{\version}{...}|
can thus be added before the above code to override it.

For the main file, one might add a line
(between |\childdocmain| and the above block)
%
\begin{center}
|%\ifchilddoc\||else\providecommand{\version}{draft}\||fi|
\end{center}
%
which can be uncommented to produce a draft version.
Likewise one can add a line to the very top of a child file
(above the |\childdocof{|\textit{main}|}| directive)
%
\begin{center}
|%\providecommand{\version}{final}|
\end{center}
%
which can be uncommented to produce the final version of this child document.

%%%%%%%%%%%%%%%%%%%%%%%%%%%%%%%%%%%%%%%%%%%%%%%%%%%%%%%%%%%%%%%%%%%%%%%%%%%%%%%%
\subsection{Forwarding}
\label{sec:forward}

Different versions of the main or child documents
using compilation flags as described in \secref{sec:flags}
can be (permanently) stored in different files
for convenient compilation, viewing and distribution.
To this end, the package defines a command
to pass on compilation to a different file:

%%%%%%%%%%%%%%%%%%%%%%%%%%%%%%%%%%%%%%%%
\DescribeMacro{\childdocforward}
The command |\childdocforward| redirects processing to
another source file:
%
\begin{center}
\begin{tabular}{l}
|\input{childdoc.def}|\\
|\childdocforward[|\textit{main}|]{|\textit{dest}|}|\\
\end{tabular}
\end{center}
%
The argument \textit{dest} is the destination file
(without extension).
It should be the main file or one of the child files.
Note that further \textsf{childdoc} directives
such as |\childdocof| and |\childdocforward|
in the indicated file will be processed in this form.
The optional argument \textit{main}
passes on directly to the main file \textit{main}
while pretending to compile the child \textit{dest}.
This form behaves as if \textit{dest}
issues |\childdocof{|\textit{main}|}| right away,
and no further \textsf{childdoc} directives will be processed.

%%%%%%%%%%%%%%%%%%%%%%%%%%%%%%%%%%%%%%%%
\DescribeMacro{\...prefix}
In the alternative form |\childdocforwardprefix|,
%
\begin{center}
\begin{tabular}{l}
|\input{childdoc.def}|\\
|\childdocforwardprefix[|\textit{main}|]{|\textit{prefix}|}{|\textit{dest}|}|
\end{tabular}
\end{center}
%
the destination file is determined by a pattern
depending on the current file:
To make this work, the current file must be called
`{\textit{prefix}\hspace{0.2em}\textit{suffix}}'
with \textit{prefix} matching precisely the argument.
Processing is then passed on to the file
`{\textit{dest}\hspace{0.2em}\textit{suffix}}'.
Surely, the same effect is achieved by
directly specifying the
argument `{\textit{dest}\hspace{0.2em}\textit{suffix}}'
in the first form.
However, that requires to set up a different file
for each child. With the alternative form of the command
all these files can have exactly the same content
which simplifies setting them up and maintaining them.

For example, the following file |draft.tex|
with a compilation flag |\version| as described in \secref{sec:flags}
compiles the main document as a draft:
%
\begin{center}
\begin{tabular}{l}
|\def\version{draft}|\\
|\input{childdoc.def}|\\
|\childdocforward{|\textit{main}|}|
\end{tabular}
\end{center}
%
Likewise, the following files |final|\textit{nn}|.tex|
compile the final version of the child document
|child|\textit{nn}|.tex|:
%
\begin{center}
\begin{tabular}{l}
|\def\version{final}|\\
|\input{childdoc.def}|\\
|\childdocforwardprefix{final}{child}|
\end{tabular}
\end{center}
%

Note that when several versions of a main file and/or of each child file
are to be generated, it may be convenient to set up a |Makefile| or
shell script to automatise the process.

%%%%%%%%%%%%%%%%%%%%%%%%%%%%%%%%%%%%%%%%%%%%%%%%%%%%%%%%%%%%%%%%%%%%%%%%%%%%%%%%
\subsection{Command Line Processing}
\label{sec:commandline}

The effect of redirection files can also be achieved by invoking
the \LaTeX{} compiler with a more elaborate command line.
Most conveniently this should be done as part
of a shell script or a |Makefile|.

When using \textsf{childdoc} in the main file, the following
command lines effectively perform a redirection
(note that depending on the shell being used,
backslashes may have to be doubled: `|\|' $\to$ `|\\|'):
%
\begin{center}
|... -jobname "|\textit{target}|" |\\|"|[\textit{flags}]%
|\input{childdoc.def}\childdocforward[|\textit{main}|]{|\textit{dest}|}"|
\end{center}
%
Here \textit{target} is the name of the output file,
\textit{main} is the name of the main file
and \textit{dest} is the name of the main or child file to be processed
(all filenames without extensions).
The optional argument \textit{main} can be omitted
if \textit{main} matches \textit{dest}.
Optionally, compilation \textit{flags} can be defined via |\def| commands.
This command line makes the \TeX{} engine believe
it is compiling the file \textit{target}
whose content is specified as the latter parameter.
The provided code then forwards the processing to
\textit{main} or \textit{dest} as described in \secref{sec:forward}.

%%%%%%%%%%%%%%%%%%%%%%%%%%%%%%%%%%%%%%%%%%%%%%%%%%%%%%%%%%%%%%%%%%%%%%%%%%%%%%%%
\subsection{Include by Input}
\label{sec:input}

Including child documents by |\include| has some restrictions by design.
Most notably, the content of a child document always occupies
its own set of pages; pages cannot be shared between child documents.
Usually, this behaviour makes perfect sense
because each child document contain an essential part of the document.
However, in some situations it may be desirable to compose
a document from a collection of parts
without having mandatory page breaks between then.
For this case, the package
provides a mechanism to include parts
by |\input| which can also be processed individually.
However, by construction this mechanism
requires manual handling of the content to be output.

%%%%%%%%%%%%%%%%%%%%%%%%%%%%%%%%%%%%%%%%
\DescribeMacro{\ifchilddocmanual}
The main file should be prepared as usual, see \secref{sec:include}.
However, the document body must make a distinction
between processing of an individual part and of the main document, e.g.:
%
\begin{center}
\begin{tabular}{l}
|\ifchilddocmanual|\\
|\input{\childdocname}|\\
|\||else|\\
\textit{document body with }|\input{|\textit{part}|}|\\
|\||fi|
\end{tabular}
\end{center}
%
The conditional |\ifchilddocmanual| is true whenever
a part to be included by |\input| is being compiled,
and the name of the part is stored in |\childdocname|.

%%%%%%%%%%%%%%%%%%%%%%%%%%%%%%%%%%%%%%%%
\DescribeMacro{\childdocby}
Each part to be included by |\input| should start with:
%
\begin{center}
\begin{tabular}{l}
|\input{childdoc.def}|\\
|\childdocby{|\textit{main}|}|\\
\end{tabular}
\end{center}
%
The directive |\childdocby| is similar to |\childdocof|
described in \secref{sec:include},
but the subsequent selection of content must be done manually.
To that end, both |\ifchilddoc| and |\ifchilddocmanual|
will be true upon processing of a part,
and the name of the part is stored in |\childdocname|.
Note that |\jobname| will be set to the filename of the current part
so that each part receives an individual |.aux| file
that does not interfere with the |.aux| file(s) of the main document.
This behaviour can be altered by the alternative form
|\childdocby[*]{|\textit{main}|}| (with a non-empty optional argument)
which uses the |.aux| file of the main document
by setting |\jobname| to \textit{main}.

%%%%%%%%%%%%%%%%%%%%%%%%%%%%%%%%%%%%%%%%%%%%%%%%%%%%%%%%%%%%%%%%%%%%%%%%%%%%%%%%
\subsection{Driver Development}
\label{sec:driver}

The \textsf{childdoc} mechanism can also be use for the development
of definition files such as \LaTeX{} styles or classes.
This case differs from the above setup with multiple parts
included by |\include| in that no |\includeonly| should be invoked.
This can be achieved by starting the include file
(before |\ProvidesPackage|) with:
%
\begin{center}
\begin{tabular}{l}
|\input{childdoc.def}|\\
|\childdocforward{|\textit{main}|}|\\
\end{tabular}
\end{center}
%
or alternatively with:
%
\begin{center}
\begin{tabular}{l}
|\input{childdoc.def}|\\
|\childdocby{|\textit{main}|}|\\
\end{tabular}
\end{center}
%
Both forms have slightly different effects as described above.
The main file is prepared as usual, see \secref{sec:include}.

%%%%%%%%%%%%%%%%%%%%%%%%%%%%%%%%%%%%%%%%%%%%%%%%%%%%%%%%%%%%%%%%%%%%%%%%%%%%%%%%
\subsection{Legacy Detection}
\label{sec:detection}

The directive |\childdocmain| in the main file can detect
whether the complete document or merely a child is to be compiled
even without using the directive |\childdocof|.
This method is deprecated because it is less robust
and there is no compelling reason to use it;
it is merely provided for backward compatibility
and it may be removed in future versions.

If the detection mechanism is to be used,
it is mandatory to correctly specify
the filename of the main file as the argument of |\childdocmain|:
%
\begin{center}
\begin{tabular}{l}
|\input{childdoc.def}|\\
|\childdocmain{|\textit{main}|}|\\
\end{tabular}
\end{center}
%
If |\jobname| does not match the argument \textit{main} of |\childdocmain|,
it is assumed that |\jobname| points to the child file to be compiled.
When using |\childdocmain| with the main file specified as argument,
it suffices to start a child file
with just |\input{|\textit{main}|}|
without loading of the package and using |\childdocof|.
If instead all processing is done
with the appropriate \textsf{childdoc} directives,
the argument of \textit{main} of |\childdocmain| can be empty.

An alternative version of the command line processing described
in \secref{sec:commandline} using the detection mechanism reads:
%
\begin{center}
|... -jobname "|\textit{target}|" "|[\textit{flags}]%
[|\def\jobname{|\textit{dest}|}|]|\input{|\textit{main}|}"|
\end{center}

%%%%%%%%%%%%%%%%%%%%%%%%%%%%%%%%%%%%%%%%%%%%%%%%%%%%%%%%%%%%%%%%%%%%%%%%%%%%%%%%
\subsection{Manual Code}
\label{sec:manual}

In case one cannot be certain whether the definitions file |childdoc.def|
is installed on the target \TeX{} distribution
and one prefers not to ship it,
it is conceivable to paste a few relevant commands into the sources.

To that end, drop all statements |\input{childdoc.def}|
and perform the replacements as outlined below.
Instead of |\childdocmain{|\textit{main}|}| add the following code
to the top of the main file:
%
\begin{center}
\begin{tabular}{l}
|\||ifdefined\childdocname\endinput\||fi\newif\ifchilddoc|\\
|\edef\childdocname{\scantokens\expandafter{\jobname\noexpand}}|\\
|\def\childdocmain{|\textit{main}|}\||ifx\childdocmain\childdocname\||else|\\
|\childdoctrue\includeonly{\childdocname}\let\jobname\childdocmain\||fi|\\
\end{tabular}
\end{center}
%
Instead of |\childdocof{|\textit{main}|}| just include the main file
at the top of each child file:
%
\begin{center}
|\input{|\textit{main}|}|
\end{center}
%
A simple redirection |\childdocforward{|\textit{dest}|}| is achieved by:
%
\begin{center}
|\def\jobname{|\textit{dest}|}\input{\jobname}|
\end{center}
%
The redirection with prefix
|\childdocforwardprefix[|\textit{prefix}|]{|\textit{dest}|}|
is accomplished by:
%
\begin{center}
\begin{tabular}{l}
|{\edef\jobname{\scantokens\expandafter{\jobname\noexpand}}|\\
|\def\redirectjob |\textit{prefix}|#1~~~{\gdef\jobname{|\textit{dest}|#1}}|\\
|\expandafter\redirectjob\jobname~~~}\input{\jobname}|
\end{tabular}
\end{center}

In an alternative approach,
child documents can be compiled by a specific command line
without additional code or specific definitions:
%
\begin{center}
|... -jobname "|\textit{target}|" "|[\textit{flags}]%
|\includeonly{|\textit{dest}|}\input{|\textit{main}|}"|
\end{center}
%

%%%%%%%%%%%%%%%%%%%%%%%%%%%%%%%%%%%%%%%%%%%%%%%%%%%%%%%%%%%%%%%%%%%%%%%%%%%%%%%%
%%%%%%%%%%%%%%%%%%%%%%%%%%%%%%%%%%%%%%%%%%%%%%%%%%%%%%%%%%%%%%%%%%%%%%%%%%%%%%%%
\section{Information}

%%%%%%%%%%%%%%%%%%%%%%%%%%%%%%%%%%%%%%%%%%%%%%%%%%%%%%%%%%%%%%%%%%%%%%%%%%%%%%%%
\subsection{Copyright}

Copyright \copyright{} 2017--2018 Niklas Beisert

This work may be distributed and/or modified under the
conditions of the \LaTeX{} Project Public License, either version 1.3
of this license or (at your option) any later version.
The latest version of this license is in
  \url{http://www.latex-project.org/lppl.txt}
and version 1.3 or later is part of all distributions of \LaTeX{}
version 2005/12/01 or later.

This work has the LPPL maintenance status `maintained'.

The Current Maintainer of this work is Niklas Beisert.

This work consists of the files |README.txt|, |childdoc.ins| and |childdoc.dtx|
as well as the derived files |childdoc.def|, |cdocsamp.tex|
with |cdocsch1.tex|, |cdocsch2.tex|, |cdocspt3.tex|, |cdocspt4.tex|,
|cdocsdrf.tex|, |cdocsfn1.tex|, |cdocsfn2.tex|
as well as |childdoc.pdf|.

%%%%%%%%%%%%%%%%%%%%%%%%%%%%%%%%%%%%%%%%%%%%%%%%%%%%%%%%%%%%%%%%%%%%%%%%%%%%%%%%
\subsection{Files and Installation}

The package consists of the files:
%
\begin{center}
\begin{tabular}{ll}
    |README.txt|   & readme file \\
    |childdoc.ins| & installation file \\
    |childdoc.dtx| & source file \\
    |childdoc.def| & definition file \\
    |cdocsamp.tex| & sample main file \\
    |cdocsch1.tex| & sample include file \\
    |cdocsch2.tex| & sample include file \\
    |cdocspt3.tex| & sample part file \\
    |cdocspt4.tex| & sample part file \\
    |cdocsdrf.tex| & sample redirection file \\
    |cdocsfn1.tex| & sample redirection file \\
    |cdocsfn2.tex| & sample redirection file \\
    |childdoc.pdf| & manual
\end{tabular}
\end{center}
%
The distribution consists of the files
|README.txt|, |childdoc.ins| and |childdoc.dtx|.
%
\begin{itemize}
\item
Run (pdf)\LaTeX{} on |childdoc.dtx|
to compile the manual |childdoc.pdf| (this file).
\item
Run \LaTeX{} on |childdoc.ins| to create the definitions file |childdoc.def|
and the sample |cdocsamp.tex| with include files
|cdocsch1.tex|, |cdocsch2.tex|, |cdocspt3.tex|, |cdocspt4.tex|,
|cdocsdrf.tex|, |cdocsfn1.tex|, |cdocsfn2.tex|.
Then copy the file |childdoc.def| to an appropriate directory of your \LaTeX{}
distribution, e.g.\ \textit{texmf-root}|/tex/latex/childdoc|.
\end{itemize}

%%%%%%%%%%%%%%%%%%%%%%%%%%%%%%%%%%%%%%%%%%%%%%%%%%%%%%%%%%%%%%%%%%%%%%%%%%%%%%%%
\subsection{Related CTAN Packages}

There are several other packages which offer a similar functionality:
%
\begin{itemize}
\item
The packages
\href{http://ctan.org/pkg/docmute}{\textsf{docmute}},
\href{http://ctan.org/pkg/includex}{\textsf{includex}} and
\href{http://ctan.org/pkg/standalone}{\textsf{standalone}}
provide commands to include only the document body of
a child file thus allowing both files to be compiled individually.
\item
The packages \href{http://ctan.org/pkg/subdocs}{\textsf{subdocs}}
and \href{http://ctan.org/pkg/subfiles}{\textsf{subfiles}}
provide structures in which the main and child documents can be
encapsulated and allowing them to be compiled individually.
The inclusion mechanism is different from the conventional |\include|.
\item
The package \href{http://ctan.org/pkg/combine}{\textsf{combine}}
is an elaborate solution to combine several documents into one.
\end{itemize}
%
See also the CTAN topic \href{http://ctan.org/topic/subdocs}{\textsf{subdocs}}
for further related packages.
The present package differs from the above solutions in that
a document structure constructed with the conventional |\include| mechanism
just needs two extra commands at the top of every file
such that all constituent files can be compiled individually.

%%%%%%%%%%%%%%%%%%%%%%%%%%%%%%%%%%%%%%%%%%%%%%%%%%%%%%%%%%%%%%%%%%%%%%%%%%%%%%%%
%\subsection{Feature Suggestions}
%
%The following is a list of features which may be useful for future
%versions of this package:
%%
%\begin{itemize}
%\item
%\ldots
%\end{itemize}

%%%%%%%%%%%%%%%%%%%%%%%%%%%%%%%%%%%%%%%%%%%%%%%%%%%%%%%%%%%%%%%%%%%%%%%%%%%%%%%%
\subsection{Revision History}

%%%%%%%%%%%%%%%%%%%%%%%%%%%%%%%%%%%%%%%%
\paragraph{v2.0:} 2018/12/30

\begin{itemize}
\item
immediate forward processing
\item
added |\childdocby| mechanism
\item
manual restructured
\end{itemize}

%%%%%%%%%%%%%%%%%%%%%%%%%%%%%%%%%%%%%%%%
\paragraph{v1.6:} 2018/01/17

\begin{itemize}
\item
application for development of include files
\item
corrections to manual
\end{itemize}

%%%%%%%%%%%%%%%%%%%%%%%%%%%%%%%%%%%%%%%%
\paragraph{v1.5:} 2017/05/21

\begin{itemize}
\item
more complete structuring introduced
\item
|\childdocof| introduced
\item
|\childdoc| renamed to |\childdocmain|
\item
|\childredirect| renamed to |\childdocforward| and |\childdocforwardprefix|
and functionality expanded
\end{itemize}

%%%%%%%%%%%%%%%%%%%%%%%%%%%%%%%%%%%%%%%%
\paragraph{v1.0:} 2017/04/27

\begin{itemize}
\item
manual and install package
\item
first version published on CTAN
\end{itemize}

%%%%%%%%%%%%%%%%%%%%%%%%%%%%%%%%%%%%%%%%
\paragraph{v0.6:} 2017/04/26

\begin{itemize}
\item
redirection mechanism added
\end{itemize}

%%%%%%%%%%%%%%%%%%%%%%%%%%%%%%%%%%%%%%%%
\paragraph{v0.5:} 2017/04/26

\begin{itemize}
\item
functionality in definition file
\end{itemize}


%%%%%%%%%%%%%%%%%%%%%%%%%%%%%%%%%%%%%%%%%%%%%%%%%%%%%%%%%%%%%%%%%%%%%%%%%%%%%%%%
%%%%%%%%%%%%%%%%%%%%%%%%%%%%%%%%%%%%%%%%%%%%%%%%%%%%%%%%%%%%%%%%%%%%%%%%%%%%%%%%
%%%%%%%%%%%%%%%%%%%%%%%%%%%%%%%%%%%%%%%%%%%%%%%%%%%%%%%%%%%%%%%%%%%%%%%%%%%%%%%%
\appendix

\settowidth\MacroIndent{\rmfamily\scriptsize 000\ }

 \DocInput{childdoc.dtx}

\end{document}
%</driver>
% \fi
%
% %%%%%%%%%%%%%%%%%%%%%%%%%%%%%%%%%%%%%%%%%%%%%%%%%%%%%%%%%%%%%%%%%%%%%%%%%%%%%%
% %%%%%%%%%%%%%%%%%%%%%%%%%%%%%%%%%%%%%%%%%%%%%%%%%%%%%%%%%%%%%%%%%%%%%%%%%%%%%%
% \section{Sample}
%\iffalse
%<*samplemain>
%\fi
%
% The following presents a sample document
% with two chapters, two parts, a title page,
% a compile flag as well as three forwarding files to set the flag.
% It consists of eight |.tex| files:
% \begin{center}
% \begin{tabular}{ll}
% |cdocsamp.tex|&main file\\
% |cdocsch1.tex|&include file for chapter 1\\
% |cdocsch2.tex|&include file for chapter 2\\
% |cdocspt3.tex|&include file for part 3\\
% |cdocspt4.tex|&include file for part 4\\
% |cdocsdrf.tex|&forwarding file for main file in draft mode\\
% |cdocsfi1.tex|&forwarding file for final version of chapter 1\\
% |cdocsfi2.tex|&forwarding file for final version of chapter 2\\
% \end{tabular}
% \end{center}
% Each of the eight files can be compiled directly by the \LaTeX{} compiler.
%
% %%%%%%%%%%%%%%%%%%%%%%%%%%%%%%%%%%%%%%
% \paragraph{Main File.}
%
% The main file is called |cdocsamp.tex|.
%
% Load the \textsf{childdoc} definitions and
% declare the filename for the main document:
%    \begin{macrocode}
\input{childdoc.def}
\childdocmain{}
%    \end{macrocode}

% Optional override for |\version| flag:
%    \begin{macrocode}
%%\ifchilddoc\else\providecommand{\version}{draft}\fi
%    \end{macrocode}

% Define the default values for the |\version| flag
% (|final| for the main file and |draft| for childs):
%    \begin{macrocode}
\ifchilddoc
\providecommand{\version}{draft}
\else
\providecommand{\version}{final}
\fi
%    \end{macrocode}

% Load the standard document class:
%    \begin{macrocode}
\documentclass[12pt]{article}
%    \end{macrocode}

% Start the document body:
%    \begin{macrocode}
\begin{document}
%    \end{macrocode}

% Declare a title page.
% Print title, part of document being processed and version flag:
%    \begin{macrocode}
\addtocounter{page}{-1}
\begin{center}
{\LARGE\bfseries{}childdoc example\par}
\vspace{1cm}
\ifchilddoc
\ifchilddocmanual part\else chapter\fi:
`\childdocname' of `\childdocjob'\par
\else
main document: `\childdocjob'\par
\fi
version: \version\par
\end{center}
\newpage
%    \end{macrocode}

% Manually include selected file,
% otherwise process as usual:
%    \begin{macrocode}
\ifchilddocmanual
\section*{part `\childdocname'}
\input{\childdocname}
\else
%    \end{macrocode}

% Include the two chapters:
%    \begin{macrocode}
\include{cdocsch1}
\include{cdocsch2}
%    \end{macrocode}

% Include the two parts unless only chapters should be displayed:
%    \begin{macrocode}
\ifchilddoc\else
\section{part three}
\input{cdocspt3}
\section{part four}
\input{cdocspt4}
\fi
%    \end{macrocode}

% Process as usual until here:
%    \begin{macrocode}
\fi
%    \end{macrocode}

% End of document body:
%    \begin{macrocode}
\end{document}
%    \end{macrocode}
%\iffalse
%</samplemain>
%\fi
%
% %%%%%%%%%%%%%%%%%%%%%%%%%%%%%%%%%%%%%%
% \paragraph{Chapter Include Files.}
%
% The include files are called |cdocsch1.tex| and |cdocsch2.tex|.
%
%\iffalse
%<*samplechap1|samplechap2>
%\fi

% Optional override for |\version| flag:
%    \begin{macrocode}
%%\providecommand{\version}{final}
%    \end{macrocode}

% Include the main document:
%    \begin{macrocode}
\input{childdoc.def}
\childdocof{cdocsamp}
%    \end{macrocode}

%\iffalse
%</samplechap1|samplechap2>
%\fi
%
%\iffalse
%<*samplechap1>
%\fi
% Some text for chapter 1:
%    \begin{macrocode}
\section{one}
some text in chapter one
%    \end{macrocode}

%\iffalse
%</samplechap1>
%\fi
% Some text for chapter 2:
%\iffalse
%<*samplechap2>
%\fi
%    \begin{macrocode}
\section{two}
more text in chapter two
%    \end{macrocode}

%\iffalse
%</samplechap2>
%\fi
%
% %%%%%%%%%%%%%%%%%%%%%%%%%%%%%%%%%%%%%%
% \paragraph{Part Include Files.}
%
% The include files are called |cdocspt3.tex| and |cdocspt4.tex|.
%
%\iffalse
%<*samplepart3|samplepart4>
%\fi

% Optional override for |\version| flag:
%    \begin{macrocode}
%%\providecommand{\version}{final}
%    \end{macrocode}

% Include the main document:
%    \begin{macrocode}
\input{childdoc.def}
\childdocby{cdocsamp}
%    \end{macrocode}

%\iffalse
%</samplepart3|samplepart4>
%\fi
%
%\iffalse
%<*samplepart3>
%\fi
% Some text for part 3:
%    \begin{macrocode}
some text in part three
%    \end{macrocode}

%\iffalse
%</samplepart3>
%\fi
% Some text for part 4:
%\iffalse
%<*samplepart4>
%\fi
%    \begin{macrocode}
more text in part four
%    \end{macrocode}

%\iffalse
%</samplepart4>
%\fi
%
% %%%%%%%%%%%%%%%%%%%%%%%%%%%%%%%%%%%%%%
% \paragraph{Forwarding for a Complete Draft.}
%
% The following forwarding file |cdocsdrf.tex|
% compiles the main document in draft mode:
%\iffalse
%<*sampledraft>
%\fi
%    \begin{macrocode}
\def\version{draft}
\input{childdoc.def}
\childdocforward{cdocsamp}
%    \end{macrocode}

%\iffalse
%</sampledraft>
%\fi
%
% %%%%%%%%%%%%%%%%%%%%%%%%%%%%%%%%%%%%%%
% \paragraph{Forwarding for Final Version of the Chapters.}
%
% The following forwarding files |cdocsfn1.tex| and |cdocsfn2.tex|
% (with identical content)
% compile the final versions of the child documents
% |cdocsch1.tex| and |cdocsch2.tex|, respectively:
%\iffalse
%<*samplefinal>
%\fi
%    \begin{macrocode}
\def\version{final}
\input{childdoc.def}
\childdocforwardprefix[cdocsamp]{cdocsfn}{cdocsch}
%    \end{macrocode}

%\iffalse
%</samplefinal>
%\fi
%
% %%%%%%%%%%%%%%%%%%%%%%%%%%%%%%%%%%%%%%
% \paragraph{Command Line Processing.}
%
% The following three command lines generate the output files
% |cdocscld|, |cdocscl1| and |cdocscl2|
% which should be identical to
% |cdocsdrf|, |cdocsch1| and |cdocsfn2|, respectively:
% \begin{center}
% \begin{tabular}{l}
% |latex -jobname cdocscld \|\\
% |  "\def\version{draft}\input{childdoc.def}\childdocforward{cdocsamp}"|\\
% |latex -jobname cdocscl1 \|\\
% |  "\input{childdoc.def}\childdocforward[cdocsamp]{cdocsch1}"|\\
% |latex -jobname cdocscl2 \|\\
% |  "\def\version{final}\input{childdoc.def}\childdocforward{cdocsch2}"|
% \end{tabular}
% \end{center}
% Note that the trailing backslash on each first line
% merely continues the input to the second line
% (for convenient cut ant paste).
% Furthermore, the command |latex| can be replaced by any
% of its alternative versions such as |pdflatex|.
%
% %%%%%%%%%%%%%%%%%%%%%%%%%%%%%%%%%%%%%%%%%%%%%%%%%%%%%%%%%%%%%%%%%%%%%%%%%%%%%%
% %%%%%%%%%%%%%%%%%%%%%%%%%%%%%%%%%%%%%%%%%%%%%%%%%%%%%%%%%%%%%%%%%%%%%%%%%%%%%%
% \section{Implementation}
%\iffalse
%<*package>
%\fi
%
% This section describes the definitions file |childdoc.def|.

% The definitions cannot be loaded using |\usepackage| or |\RequirePackage|
% which has a mechanism to prevent loading a style file more than once.
% When loading the definitions by means of |\input|
% multiple instances have to be prevented manually:
%\iffalse
%This code needs to be before the `\ProvidesFile' directive
%which is defined at the beginning of this file.
%Therefore it is also placed there and commented out here.
%</package>
%<*discard>
%\fi
%    \begin{macrocode}
\ifdefined\childdocmain\endinput\fi
%    \end{macrocode}
%\iffalse
%</discard>
%<*package>
%\fi
%
% \macro{\ifchilddoc}
% \macro{\ifchilddocmanual}
% The conditional |\ifchilddoc| tells whether a
% child (true) or main (false) document is being compiled.
% The conditional |\ifchilddocmanual| tells whether
% the |\includeonly| mechanism is used (false) or
% the selection of child files must be performed manually (true).
% The definitions initialise to false:
%    \begin{macrocode}
\newif\ifchilddoc
\newif\ifchilddocmanual
%    \end{macrocode}

% \macro{\childdocname}
% \macro{\childdocjob}
% The macro |\childdocname| stores the name of the main document
% to be compiled. The macro |\childdocjob| stores the name of
% the document on which the \LaTeX{} compiler was originally invoked.
% The content of |\jobname| cannot be compared
% to filenames specified in the source due to different catcodes.
% The following code rescans |\jobname|, stores the result
% in |\childdocname| and saves a copy in |\childdocjob|:
%    \begin{macrocode}
\edef\childdocname{\scantokens\expandafter{\jobname\noexpand}}
\let\childdocjob\childdocname
%    \end{macrocode}

% \macro{\childdocdisable}
% The macro |\childdocdisable| prevents the main file
% from being processed more than once.
% At this stage, the main document command |\childdocmain|
% is assumed to be called once again where it should do nothing.
% Any subsequent call to it should prevent
% a secondary processing of the main document
% It overwrites the forwarding commands
% |\childdocof| and |\childdocforward|
% with empty macros to prevent further inclusions of the main document:
%    \begin{macrocode}
\newcommand{\childdocdisable}
{
  \renewcommand{\childdocmain}[1]{\renewcommand{\childdocmain}[1]{\endinput}}
  \renewcommand{\childdocof}[1]{}
  \renewcommand{\childdocby}[2][]{}
  \renewcommand{\childdocforward}[2][]{}
  \renewcommand{\childdocdisable}{}
}
%    \end{macrocode}

% \macro{\childdocmain}
% The macro |\childdocmain| is to be called at the top of the main file
% with nothing or the main filename (without extension) as argument.
% First, it breaks loops.
% If the argument is not empty and does not match |\childdocname|
% (which is set by the first inclusion of |childdoc.def|),
% |\ifchilddoc| is set to true, |\includeonly| is applied to the child file
% and |\jobname| is set to the main file
% (for proper handling of |.aux| files):
%    \begin{macrocode}
\newcommand{\childdocmain}[1]
{
  \childdocdisable\childdocmain{}
  \if?#1?\else
    \begingroup
      \def\childdoctmp{#1}
      \ifx\childdoctmp\childdocname
        \def\childdoctmp{}
      \else
        \def\childdoctmp
        {
          \childdoctrue
          \includeonly{\childdocname}
          \def\childdocjob{#1}
          \def\jobname{#1}
        }
      \fi
      \expandafter
    \endgroup
    \childdoctmp
  \fi
}
%    \end{macrocode}

% \macro{\childdocof}
% The command |\childdocof| redirects
% compilation to the main file |#1|.
%    \begin{macrocode}
\newcommand{\childdocof}[1]
{
  \childdocdisable
  \childdoctrue
  \includeonly{\childdocname}
  \def\jobname{#1}
  \def\childdocjob{#1}
  \input{#1}
}
%    \end{macrocode}

% \macro{\childdocby}
% The command |\childdocby| ....
%    \begin{macrocode}
\newcommand{\childdocby}[2][]
{
  \childdocdisable
  \childdoctrue
  \childdocmanualtrue
  \if?#1?\else
    \def\jobname{#2}
  \fi
  \def\childdocjob{#2}
  \input{#2}
  \endinput
}
%    \end{macrocode}

% \macro{\childdocforward}
% The command |\childdocforward| redirects
% compilation to the main file or
% (if the optional argument is given) a child file.
% Parameters are set as if the main file
% or a child file starting with |\childdocof| was compiled.
% Then compilation is handed over to the main file:
%    \begin{macrocode}
\newcommand{\childdocforward}[2][]
{
  \begingroup
    \if?#1?
      \def\childdoctmp
      {
        \def\childdocname{#2}
        \def\childdocjob{#2}
        \def\jobname{#2}
        \input{#2}
        \endinput
      }
    \else
      \def\childdoctmp
      {
        \childdocdisable
        \def\childdocname{#2}
        \childdoctrue
        \includeonly{#2}
        \def\childdocjob{#1}
        \def\jobname{#1}
        \input{#1}
        \endinput
      }
    \fi
    \expandafter
  \endgroup
  \childdoctmp
}
%    \end{macrocode}

% \macro{\childdocforwardprefix}
% The command |\childdocforwardprefix| redirects
% compilation to the main or a child file by means of a pattern.
% The prefix |#1| in the current filename is replaced by |#2|
% and the suffix of the current filename is kept
% (it is assumed that the filename does not contain the substring `|~~~|'
% which is used as a delimiter).
% Compilation is handed over to the new file by |\childdocforward|:
%    \begin{macrocode}
\newcommand{\childdocforwardprefix}[3][]
{
  \begingroup
    \def\childdocextract #2##1~~~{\def\childdoctmp{\childdocforward[#1]{#3##1}}}
    \expandafter\childdocextract\childdocname~~~
    \expandafter
  \endgroup
  \childdoctmp
}
%    \end{macrocode}

% \macro{\childdoc}
% The deprecated macro |\childdoc| is a legacy version of |\childdocmain|:
%    \begin{macrocode}
\newcommand{\childdoc}{\childdocmain}
%    \end{macrocode}

% \macro{\childdocredirect}
% The deprecated macro |\childdocredirect| is a legacy version
% of |\childdocforward| and |\childdocforwardprefix|:
%    \begin{macrocode}
\newcommand{\childdocredirect}[2][]
{
  \begingroup
    \if?#1?
      \def\childdoctmp{\childdocforward{#2}}
    \else
      \def\childdoctmp{\childdocforwardprefix{#1}{#2}}
    \fi
    \expandafter
  \endgroup
  \childdoctmp
}
%    \end{macrocode}

%\iffalse
%</package>
%\fi
%
\endinput
|\\
|\childdocforward[|\textit{main}|]{|\textit{dest}|}|\\
\end{tabular}
\end{center}
%
The argument \textit{dest} is the destination file
(without extension).
It should be the main file or one of the child files.
Note that further \textsf{childdoc} directives
such as |\childdocof| and |\childdocforward|
in the indicated file will be processed in this form.
The optional argument \textit{main}
passes on directly to the main file \textit{main}
while pretending to compile the child \textit{dest}.
This form behaves as if \textit{dest}
issues |\childdocof{|\textit{main}|}| right away,
and no further \textsf{childdoc} directives will be processed.

%%%%%%%%%%%%%%%%%%%%%%%%%%%%%%%%%%%%%%%%
\DescribeMacro{\...prefix}
In the alternative form |\childdocforwardprefix|,
%
\begin{center}
\begin{tabular}{l}
|% \iffalse
%
% childdoc.dtx Copyright (C) 2017-2018 Niklas Beisert
%
% This work may be distributed and/or modified under the
% conditions of the LaTeX Project Public License, either version 1.3
% of this license or (at your option) any later version.
% The latest version of this license is in
%   http://www.latex-project.org/lppl.txt
% and version 1.3 or later is part of all distributions of LaTeX
% version 2005/12/01 or later.
%
% This work has the LPPL maintenance status `maintained'.
%
% The Current Maintainer of this work is Niklas Beisert.
%
% This work consists of the files childdoc.dtx and childdoc.ins
% and the derived files childdoc.def and cdocsamp.tex with
% cdocsch1.tex, cdocsch2.tex, cdocsdrf.tex, cdocsfn1.tex, cdocsfn2.tex.
%
%<package>\ifdefined\childdocmain\endinput\fi
%<package>\ProvidesFile{childdoc.def}[2018/12/30 v2.0 child document driver]
%<samplemain>\ProvidesFile{cdocsamp.tex}[2018/12/30 v2.0 sample for childdoc]
%<*driver>
%\ProvidesFile{childdoc.drv}[2018/12/30 v2.0 childdoc reference manual file]
\PassOptionsToClass{10pt,a4paper}{article}
\documentclass{ltxdoc}

\usepackage[margin=35mm]{geometry}
\usepackage{hyperref}
\usepackage{hyperxmp}
\usepackage[usenames]{color}

\hypersetup{colorlinks=true}
\hypersetup{pdfstartview=FitH}
\hypersetup{pdfpagemode=UseNone}
\hypersetup{pdfsource={}}
\hypersetup{pdflang={en-UK}}
\hypersetup{pdfcopyright={Copyright 2017-2018 Niklas Beisert.
  This work may be distributed and/or modified under the
  conditions of the LaTeX Project Public License, either version 1.3
  of this license or (at your option) any later version.}}
\hypersetup{pdflicenseurl={http://www.latex-project.org/lppl.txt}}
\hypersetup{pdfcontactaddress={ETH Zurich, ITP, HIT K,
  Wolfgang-Pauli-Strasse 27}}
\hypersetup{pdfcontactpostcode={8093}}
\hypersetup{pdfcontactcity={Zurich}}
\hypersetup{pdfcontactcountry={Switzerland}}
\hypersetup{pdfcontactemail={nbeisert@itp.phys.ethz.ch}}
\hypersetup{pdfcontacturl={http://people.phys.ethz.ch/\xmptilde nbeisert/}}

\newcommand{\secref}[1]{\hyperref[#1]{section \ref*{#1}}}

\parskip1ex
\parindent0pt
\let\olditemize\itemize
\def\itemize{\olditemize\parskip0pt}

\begin{document}

\title{The \textsf{childdoc} Package}
\hypersetup{pdftitle={The childdoc Package}}
\author{Niklas Beisert\\[2ex]
  Institut f\"ur Theoretische Physik\\
  Eidgen\"ossische Technische Hochschule Z\"urich\\
  Wolfgang-Pauli-Strasse 27, 8093 Z\"urich, Switzerland\\[1ex]
  \href{mailto:nbeisert@itp.phys.ethz.ch}
  {\texttt{nbeisert@itp.phys.ethz.ch}}}
\hypersetup{pdfauthor={Niklas Beisert}}
\hypersetup{pdfsubject={Manual for the LaTeX2e Package childdoc}}
\date{30 December 2018, \textsf{v2.0}}
\maketitle

\begin{abstract}\noindent
\textsf{childdoc} is a \LaTeXe{} package
that enables the direct compilation
of document sections included by |\include|
to individual files.
\end{abstract}

\begingroup
\parskip0ex
\tableofcontents
\endgroup

%%%%%%%%%%%%%%%%%%%%%%%%%%%%%%%%%%%%%%%%%%%%%%%%%%%%%%%%%%%%%%%%%%%%%%%%%%%%%%%%
%%%%%%%%%%%%%%%%%%%%%%%%%%%%%%%%%%%%%%%%%%%%%%%%%%%%%%%%%%%%%%%%%%%%%%%%%%%%%%%%
\section{Introduction}

\LaTeX{} provides a mechanism to structure a large document (such as a book)
into a main file and several child files (containing the chapters)
using the |\include| command.
This mechanism is beneficial for documents
which span hundreds of pages in order to
make the source file(s) more manageable.
Moreover, compilation can be restricted to
selected child files by means of the |\includeonly| command.
The latter feature can be used to reduce the compilation time while editing
(this was significantly more useful in the earlier days of \LaTeX{})
or to generate a smaller document which is easier to navigate.
Another application of |\includeonly| is to generate
documents consisting of selected parts of the complete document.

However, there are a few drawbacks of the plain |\include| mechanism:
\begin{itemize}
\item
The child files cannot be compiled on their own,
they can only be compiled via the main file.
A naive editing environment
(such as a text editor with an option
to have the current file processed by \LaTeX)
may require one to switch to the main file before compiling;
attempting to compile the child file produces errors.
\item
The main file must be modified (each time)
to adjust the |\includeonly| command
to the present needs. This easily leaves the main file in a messy state.
\item
The generated document will always carry the filename
of the main document. This is inconvenient if
several child files are to be compiled and
to be kept for distribution.
\end{itemize}

The present package provides a simple interface
to make child files individually compilable by \LaTeX{}.
Compiling a child file then has the same effect as compiling
the main file with an |\includeonly| command
to select the appropriate child.
Moreover the generated document will carry the name of the child
rather than the main file.
This resolves all three above issues.

This feature is meant to make the editing of books,
thesis documents and lecture notes somewhat more convenient.
However, the package can also be used efficiently for
composing a series of documents (such as exercise sheets)
which are typically distributed individually.
It then assists the author in generating the individual documents
(potentially in different versions)
as well as a document containing the collected series.
Another application is in developing style files
or other kinds of included material
where compilation of the style file could redirect
to a sample or test file.

%%%%%%%%%%%%%%%%%%%%%%%%%%%%%%%%%%%%%%%%%%%%%%%%%%%%%%%%%%%%%%%%%%%%%%%%%%%%%%%%
%%%%%%%%%%%%%%%%%%%%%%%%%%%%%%%%%%%%%%%%%%%%%%%%%%%%%%%%%%%%%%%%%%%%%%%%%%%%%%%%
\section{Usage}

First of all, the package \textsf{childdoc} is \emph{not} a standard
\LaTeXe{} |.sty| style file! Therefore it needs to be invoked in
a non-standard way.

%%%%%%%%%%%%%%%%%%%%%%%%%%%%%%%%%%%%%%%%%%%%%%%%%%%%%%%%%%%%%%%%%%%%%%%%%%%%%%%%
\subsection{Included Files}
\label{sec:include}

%%%%%%%%%%%%%%%%%%%%%%%%%%%%%%%%%%%%%%%%
\DescribeMacro{\childdocmain}
To use the package, add the commands
\begin{center}
\begin{tabular}{l}
|\input{childdoc.def}|\\
|\childdocmain{}|\\
\end{tabular}
\end{center}
at the very top of the main \LaTeX{} file,
in particular \emph{before} the |\documentclass| statement!
The argument of |\childdocmain| should be left empty
(but it must be present).

%%%%%%%%%%%%%%%%%%%%%%%%%%%%%%%%%%%%%%%%
\DescribeMacro{\childdocof}
Furthermore, add the commands
\begin{center}
\begin{tabular}{l}
|\input{childdoc.def}|\\
|\childdocof{|\textit{main}|}|\\
\end{tabular}
\end{center}
at the top of every child file \textit{child}
which is included by |\include{|\textit{child}|}|
from within the main file
(or at least for those files to be compiled individually).
The argument \textit{main} must be the filename of the main file.

There are a couple of
considerations in setting up the main and child documents:

%%%%%%%%%%%%%%%%%%%%%%%%%%%%%%%%%%%%%%%%
\paragraph{Restrictions.}

Please note the following restrictions:
\begin{itemize}
\item
|\childdocmain| must be called with one argument \textit{main}
to ensure compatibility with earlier version of the package.
It must either be empty (|\childdocmain{}|)
or precisely match the filename of the main file in which it is specified.
See \secref{sec:detection} for further information.
\item
The filename \textit{main} must be specified without the |.tex| extension.
\item
The filename \textit{main} is case sensitive
(even in case-insensitive file systems)
due to internal string comparison.
\item
The argument \textit{main} should be fully expanded, it cannot be a macro.
\item
Subdirectories and special characters should be avoided in filenames.
\item
The command |\childdocmain{|\textit{main}|}| must be followed by a whitespace.
It should not be followed immediately by another command
or by a comment mark `|%|'.
This is because the \TeX{} parser reads the token immediately following
the argument of |\childdocmain| and puts it
at the beginning of every child section;
however, a white\-space is ignored.
\end{itemize}

%%%%%%%%%%%%%%%%%%%%%%%%%%%%%%%%%%%%%%%%
\paragraph{Content of Main File.}

It is advisable to place all content in the child files included by |\include|.
Any output contained in the main file will appear in all child documents
unless suppressed manually;
it cannot be suppressed automatically by the |\includeonly| directive
and thus should normally be avoided.
A method to include some content in the main file
by means of conditional processing is described in \secref{sec:conditional}.

%%%%%%%%%%%%%%%%%%%%%%%%%%%%%%%%%%%%%%%%
\paragraph{Page Numbering.}

When only a part of the document is compiled,
the appropriate numbering of pages
(as well as other status parameters)
is determined from the |.aux| files.
The latter contain information from previous passes.
However this information needs to propagate through
all intermediate child documents.
Therefore the page numbering in child documents may well
be inconsistent until the complete document is compiled at least once.

A useful (if unconventional) way to always ensure a consistent
page numbering is to restart the numbering in each child document
and denote the pages by `\textit{child}|.|\textit{page}'
where \textit{child} represents the chapter/section number of the child file.
This can be achieved by the command
|\numberwithin{page}{|\textit{child}|}|
of the \textsf{amsmath} package
where \textit{child} can be |chapter| or |section|
depending on the chosen structuring.
Alternatively, one can modify the macro |\thepage| appropriately
and reset the counter |page| at the start of each child file.

%%%%%%%%%%%%%%%%%%%%%%%%%%%%%%%%%%%%%%%%%%%%%%%%%%%%%%%%%%%%%%%%%%%%%%%%%%%%%%%%
\subsection{Conditional Processing}
\label{sec:conditional}

The package provides a mechanism to compile different versions
of a document. To customise the versions further some conditional processing
can come in handy to distinguish which version is being compiled.
The package provides two macros to describe the compilation context:

%%%%%%%%%%%%%%%%%%%%%%%%%%%%%%%%%%%%%%%%
\DescribeMacro{\ifchilddoc}
The conditional |\ifchilddoc| distinguishes between the compilation of
child documents and the main document:
%
\begin{center}
|\ifchilddoc |\textit{child-code}| |[|\||else |\textit{main-code}]| \||fi|
\end{center}

%%%%%%%%%%%%%%%%%%%%%%%%%%%%%%%%%%%%%%%%
\DescribeMacro{\childdocname}
\DescribeMacro{\childdocjob}
The macro |\childdocname| contains the filename (without extension)
of the main or child file being processed.
Note that |\childdocjob| will always contain the name of the main file.

%%%%%%%%%%%%%%%%%%%%%%%%%%%%%%%%%%%%%%%%
\paragraph{Title Page.}

Conditional processing can be used to include a title or banner page
in the main document when proper precautions are taken.
Importantly, the code in the main file should ensure that the page counter
(as well as other status parameters which are stored in the |.aux| files)
takes the same value after the conditional processing.
Otherwise the page numbers may take divergent values
depending on which part is compiled.

For example, a title page could be declared by:
%
\begin{center}
\begin{tabular}{l}
|\ifchilddoc\||else|\\
|\addtocounter{page}{-1}|\\
\textit{code for title page}\\
|\newpage|\\
|\||fi|
\end{tabular}
\end{center}
%
A banner page for the child documents can be generated by:
%
\begin{center}
\begin{tabular}{l}
|\ifchilddoc|\\
|\addtocounter{page}{-1}|\\
\textit{code for banner page}\\
|\newpage|\\
|\||fi|
\end{tabular}
\end{center}
%
Here one could write a message such as:
\begin{center}
|This is the part \childdocname{} of \childdocjob{}.|
\end{center}

%%%%%%%%%%%%%%%%%%%%%%%%%%%%%%%%%%%%%%%%%%%%%%%%%%%%%%%%%%%%%%%%%%%%%%%%%%%%%%%%
\subsection{Flags}
\label{sec:flags}

The package makes it easy to generate different versions
of the main or child documents.
To this end compilation flags can be defined
and assigned different default values.
They will be particularly useful in conjunction
with the forwarding mechanism described in \secref{sec:forward}.

For example, it may be useful to have a flag |\version|
which can be set to |draft| or |final|.
The document source will contain some conditional code
depending on the value of |\version|.
Suppose further, the flag should default to |final| for the main file
and to |draft| for child files
which is a natural assignment for editing the document.
This is achieved by placing the following code
in the preamble of the main document
(below the |\childdocmain| directive):
%
\begin{center}
\begin{tabular}{l}
|\ifchilddoc|\\
|\providecommand{\version}{draft}|\\
|\||else|\\
|\providecommand{\version}{final}|\\
|\||fi|
\end{tabular}
\end{center}
%
The definition by |\providecommand| makes sure
that previous definitions are not overwritten.
Further statements |\providecommand{\version}{...}|
can thus be added before the above code to override it.

For the main file, one might add a line
(between |\childdocmain| and the above block)
%
\begin{center}
|%\ifchilddoc\||else\providecommand{\version}{draft}\||fi|
\end{center}
%
which can be uncommented to produce a draft version.
Likewise one can add a line to the very top of a child file
(above the |\childdocof{|\textit{main}|}| directive)
%
\begin{center}
|%\providecommand{\version}{final}|
\end{center}
%
which can be uncommented to produce the final version of this child document.

%%%%%%%%%%%%%%%%%%%%%%%%%%%%%%%%%%%%%%%%%%%%%%%%%%%%%%%%%%%%%%%%%%%%%%%%%%%%%%%%
\subsection{Forwarding}
\label{sec:forward}

Different versions of the main or child documents
using compilation flags as described in \secref{sec:flags}
can be (permanently) stored in different files
for convenient compilation, viewing and distribution.
To this end, the package defines a command
to pass on compilation to a different file:

%%%%%%%%%%%%%%%%%%%%%%%%%%%%%%%%%%%%%%%%
\DescribeMacro{\childdocforward}
The command |\childdocforward| redirects processing to
another source file:
%
\begin{center}
\begin{tabular}{l}
|\input{childdoc.def}|\\
|\childdocforward[|\textit{main}|]{|\textit{dest}|}|\\
\end{tabular}
\end{center}
%
The argument \textit{dest} is the destination file
(without extension).
It should be the main file or one of the child files.
Note that further \textsf{childdoc} directives
such as |\childdocof| and |\childdocforward|
in the indicated file will be processed in this form.
The optional argument \textit{main}
passes on directly to the main file \textit{main}
while pretending to compile the child \textit{dest}.
This form behaves as if \textit{dest}
issues |\childdocof{|\textit{main}|}| right away,
and no further \textsf{childdoc} directives will be processed.

%%%%%%%%%%%%%%%%%%%%%%%%%%%%%%%%%%%%%%%%
\DescribeMacro{\...prefix}
In the alternative form |\childdocforwardprefix|,
%
\begin{center}
\begin{tabular}{l}
|\input{childdoc.def}|\\
|\childdocforwardprefix[|\textit{main}|]{|\textit{prefix}|}{|\textit{dest}|}|
\end{tabular}
\end{center}
%
the destination file is determined by a pattern
depending on the current file:
To make this work, the current file must be called
`{\textit{prefix}\hspace{0.2em}\textit{suffix}}'
with \textit{prefix} matching precisely the argument.
Processing is then passed on to the file
`{\textit{dest}\hspace{0.2em}\textit{suffix}}'.
Surely, the same effect is achieved by
directly specifying the
argument `{\textit{dest}\hspace{0.2em}\textit{suffix}}'
in the first form.
However, that requires to set up a different file
for each child. With the alternative form of the command
all these files can have exactly the same content
which simplifies setting them up and maintaining them.

For example, the following file |draft.tex|
with a compilation flag |\version| as described in \secref{sec:flags}
compiles the main document as a draft:
%
\begin{center}
\begin{tabular}{l}
|\def\version{draft}|\\
|\input{childdoc.def}|\\
|\childdocforward{|\textit{main}|}|
\end{tabular}
\end{center}
%
Likewise, the following files |final|\textit{nn}|.tex|
compile the final version of the child document
|child|\textit{nn}|.tex|:
%
\begin{center}
\begin{tabular}{l}
|\def\version{final}|\\
|\input{childdoc.def}|\\
|\childdocforwardprefix{final}{child}|
\end{tabular}
\end{center}
%

Note that when several versions of a main file and/or of each child file
are to be generated, it may be convenient to set up a |Makefile| or
shell script to automatise the process.

%%%%%%%%%%%%%%%%%%%%%%%%%%%%%%%%%%%%%%%%%%%%%%%%%%%%%%%%%%%%%%%%%%%%%%%%%%%%%%%%
\subsection{Command Line Processing}
\label{sec:commandline}

The effect of redirection files can also be achieved by invoking
the \LaTeX{} compiler with a more elaborate command line.
Most conveniently this should be done as part
of a shell script or a |Makefile|.

When using \textsf{childdoc} in the main file, the following
command lines effectively perform a redirection
(note that depending on the shell being used,
backslashes may have to be doubled: `|\|' $\to$ `|\\|'):
%
\begin{center}
|... -jobname "|\textit{target}|" |\\|"|[\textit{flags}]%
|\input{childdoc.def}\childdocforward[|\textit{main}|]{|\textit{dest}|}"|
\end{center}
%
Here \textit{target} is the name of the output file,
\textit{main} is the name of the main file
and \textit{dest} is the name of the main or child file to be processed
(all filenames without extensions).
The optional argument \textit{main} can be omitted
if \textit{main} matches \textit{dest}.
Optionally, compilation \textit{flags} can be defined via |\def| commands.
This command line makes the \TeX{} engine believe
it is compiling the file \textit{target}
whose content is specified as the latter parameter.
The provided code then forwards the processing to
\textit{main} or \textit{dest} as described in \secref{sec:forward}.

%%%%%%%%%%%%%%%%%%%%%%%%%%%%%%%%%%%%%%%%%%%%%%%%%%%%%%%%%%%%%%%%%%%%%%%%%%%%%%%%
\subsection{Include by Input}
\label{sec:input}

Including child documents by |\include| has some restrictions by design.
Most notably, the content of a child document always occupies
its own set of pages; pages cannot be shared between child documents.
Usually, this behaviour makes perfect sense
because each child document contain an essential part of the document.
However, in some situations it may be desirable to compose
a document from a collection of parts
without having mandatory page breaks between then.
For this case, the package
provides a mechanism to include parts
by |\input| which can also be processed individually.
However, by construction this mechanism
requires manual handling of the content to be output.

%%%%%%%%%%%%%%%%%%%%%%%%%%%%%%%%%%%%%%%%
\DescribeMacro{\ifchilddocmanual}
The main file should be prepared as usual, see \secref{sec:include}.
However, the document body must make a distinction
between processing of an individual part and of the main document, e.g.:
%
\begin{center}
\begin{tabular}{l}
|\ifchilddocmanual|\\
|\input{\childdocname}|\\
|\||else|\\
\textit{document body with }|\input{|\textit{part}|}|\\
|\||fi|
\end{tabular}
\end{center}
%
The conditional |\ifchilddocmanual| is true whenever
a part to be included by |\input| is being compiled,
and the name of the part is stored in |\childdocname|.

%%%%%%%%%%%%%%%%%%%%%%%%%%%%%%%%%%%%%%%%
\DescribeMacro{\childdocby}
Each part to be included by |\input| should start with:
%
\begin{center}
\begin{tabular}{l}
|\input{childdoc.def}|\\
|\childdocby{|\textit{main}|}|\\
\end{tabular}
\end{center}
%
The directive |\childdocby| is similar to |\childdocof|
described in \secref{sec:include},
but the subsequent selection of content must be done manually.
To that end, both |\ifchilddoc| and |\ifchilddocmanual|
will be true upon processing of a part,
and the name of the part is stored in |\childdocname|.
Note that |\jobname| will be set to the filename of the current part
so that each part receives an individual |.aux| file
that does not interfere with the |.aux| file(s) of the main document.
This behaviour can be altered by the alternative form
|\childdocby[*]{|\textit{main}|}| (with a non-empty optional argument)
which uses the |.aux| file of the main document
by setting |\jobname| to \textit{main}.

%%%%%%%%%%%%%%%%%%%%%%%%%%%%%%%%%%%%%%%%%%%%%%%%%%%%%%%%%%%%%%%%%%%%%%%%%%%%%%%%
\subsection{Driver Development}
\label{sec:driver}

The \textsf{childdoc} mechanism can also be use for the development
of definition files such as \LaTeX{} styles or classes.
This case differs from the above setup with multiple parts
included by |\include| in that no |\includeonly| should be invoked.
This can be achieved by starting the include file
(before |\ProvidesPackage|) with:
%
\begin{center}
\begin{tabular}{l}
|\input{childdoc.def}|\\
|\childdocforward{|\textit{main}|}|\\
\end{tabular}
\end{center}
%
or alternatively with:
%
\begin{center}
\begin{tabular}{l}
|\input{childdoc.def}|\\
|\childdocby{|\textit{main}|}|\\
\end{tabular}
\end{center}
%
Both forms have slightly different effects as described above.
The main file is prepared as usual, see \secref{sec:include}.

%%%%%%%%%%%%%%%%%%%%%%%%%%%%%%%%%%%%%%%%%%%%%%%%%%%%%%%%%%%%%%%%%%%%%%%%%%%%%%%%
\subsection{Legacy Detection}
\label{sec:detection}

The directive |\childdocmain| in the main file can detect
whether the complete document or merely a child is to be compiled
even without using the directive |\childdocof|.
This method is deprecated because it is less robust
and there is no compelling reason to use it;
it is merely provided for backward compatibility
and it may be removed in future versions.

If the detection mechanism is to be used,
it is mandatory to correctly specify
the filename of the main file as the argument of |\childdocmain|:
%
\begin{center}
\begin{tabular}{l}
|\input{childdoc.def}|\\
|\childdocmain{|\textit{main}|}|\\
\end{tabular}
\end{center}
%
If |\jobname| does not match the argument \textit{main} of |\childdocmain|,
it is assumed that |\jobname| points to the child file to be compiled.
When using |\childdocmain| with the main file specified as argument,
it suffices to start a child file
with just |\input{|\textit{main}|}|
without loading of the package and using |\childdocof|.
If instead all processing is done
with the appropriate \textsf{childdoc} directives,
the argument of \textit{main} of |\childdocmain| can be empty.

An alternative version of the command line processing described
in \secref{sec:commandline} using the detection mechanism reads:
%
\begin{center}
|... -jobname "|\textit{target}|" "|[\textit{flags}]%
[|\def\jobname{|\textit{dest}|}|]|\input{|\textit{main}|}"|
\end{center}

%%%%%%%%%%%%%%%%%%%%%%%%%%%%%%%%%%%%%%%%%%%%%%%%%%%%%%%%%%%%%%%%%%%%%%%%%%%%%%%%
\subsection{Manual Code}
\label{sec:manual}

In case one cannot be certain whether the definitions file |childdoc.def|
is installed on the target \TeX{} distribution
and one prefers not to ship it,
it is conceivable to paste a few relevant commands into the sources.

To that end, drop all statements |\input{childdoc.def}|
and perform the replacements as outlined below.
Instead of |\childdocmain{|\textit{main}|}| add the following code
to the top of the main file:
%
\begin{center}
\begin{tabular}{l}
|\||ifdefined\childdocname\endinput\||fi\newif\ifchilddoc|\\
|\edef\childdocname{\scantokens\expandafter{\jobname\noexpand}}|\\
|\def\childdocmain{|\textit{main}|}\||ifx\childdocmain\childdocname\||else|\\
|\childdoctrue\includeonly{\childdocname}\let\jobname\childdocmain\||fi|\\
\end{tabular}
\end{center}
%
Instead of |\childdocof{|\textit{main}|}| just include the main file
at the top of each child file:
%
\begin{center}
|\input{|\textit{main}|}|
\end{center}
%
A simple redirection |\childdocforward{|\textit{dest}|}| is achieved by:
%
\begin{center}
|\def\jobname{|\textit{dest}|}\input{\jobname}|
\end{center}
%
The redirection with prefix
|\childdocforwardprefix[|\textit{prefix}|]{|\textit{dest}|}|
is accomplished by:
%
\begin{center}
\begin{tabular}{l}
|{\edef\jobname{\scantokens\expandafter{\jobname\noexpand}}|\\
|\def\redirectjob |\textit{prefix}|#1~~~{\gdef\jobname{|\textit{dest}|#1}}|\\
|\expandafter\redirectjob\jobname~~~}\input{\jobname}|
\end{tabular}
\end{center}

In an alternative approach,
child documents can be compiled by a specific command line
without additional code or specific definitions:
%
\begin{center}
|... -jobname "|\textit{target}|" "|[\textit{flags}]%
|\includeonly{|\textit{dest}|}\input{|\textit{main}|}"|
\end{center}
%

%%%%%%%%%%%%%%%%%%%%%%%%%%%%%%%%%%%%%%%%%%%%%%%%%%%%%%%%%%%%%%%%%%%%%%%%%%%%%%%%
%%%%%%%%%%%%%%%%%%%%%%%%%%%%%%%%%%%%%%%%%%%%%%%%%%%%%%%%%%%%%%%%%%%%%%%%%%%%%%%%
\section{Information}

%%%%%%%%%%%%%%%%%%%%%%%%%%%%%%%%%%%%%%%%%%%%%%%%%%%%%%%%%%%%%%%%%%%%%%%%%%%%%%%%
\subsection{Copyright}

Copyright \copyright{} 2017--2018 Niklas Beisert

This work may be distributed and/or modified under the
conditions of the \LaTeX{} Project Public License, either version 1.3
of this license or (at your option) any later version.
The latest version of this license is in
  \url{http://www.latex-project.org/lppl.txt}
and version 1.3 or later is part of all distributions of \LaTeX{}
version 2005/12/01 or later.

This work has the LPPL maintenance status `maintained'.

The Current Maintainer of this work is Niklas Beisert.

This work consists of the files |README.txt|, |childdoc.ins| and |childdoc.dtx|
as well as the derived files |childdoc.def|, |cdocsamp.tex|
with |cdocsch1.tex|, |cdocsch2.tex|, |cdocspt3.tex|, |cdocspt4.tex|,
|cdocsdrf.tex|, |cdocsfn1.tex|, |cdocsfn2.tex|
as well as |childdoc.pdf|.

%%%%%%%%%%%%%%%%%%%%%%%%%%%%%%%%%%%%%%%%%%%%%%%%%%%%%%%%%%%%%%%%%%%%%%%%%%%%%%%%
\subsection{Files and Installation}

The package consists of the files:
%
\begin{center}
\begin{tabular}{ll}
    |README.txt|   & readme file \\
    |childdoc.ins| & installation file \\
    |childdoc.dtx| & source file \\
    |childdoc.def| & definition file \\
    |cdocsamp.tex| & sample main file \\
    |cdocsch1.tex| & sample include file \\
    |cdocsch2.tex| & sample include file \\
    |cdocspt3.tex| & sample part file \\
    |cdocspt4.tex| & sample part file \\
    |cdocsdrf.tex| & sample redirection file \\
    |cdocsfn1.tex| & sample redirection file \\
    |cdocsfn2.tex| & sample redirection file \\
    |childdoc.pdf| & manual
\end{tabular}
\end{center}
%
The distribution consists of the files
|README.txt|, |childdoc.ins| and |childdoc.dtx|.
%
\begin{itemize}
\item
Run (pdf)\LaTeX{} on |childdoc.dtx|
to compile the manual |childdoc.pdf| (this file).
\item
Run \LaTeX{} on |childdoc.ins| to create the definitions file |childdoc.def|
and the sample |cdocsamp.tex| with include files
|cdocsch1.tex|, |cdocsch2.tex|, |cdocspt3.tex|, |cdocspt4.tex|,
|cdocsdrf.tex|, |cdocsfn1.tex|, |cdocsfn2.tex|.
Then copy the file |childdoc.def| to an appropriate directory of your \LaTeX{}
distribution, e.g.\ \textit{texmf-root}|/tex/latex/childdoc|.
\end{itemize}

%%%%%%%%%%%%%%%%%%%%%%%%%%%%%%%%%%%%%%%%%%%%%%%%%%%%%%%%%%%%%%%%%%%%%%%%%%%%%%%%
\subsection{Related CTAN Packages}

There are several other packages which offer a similar functionality:
%
\begin{itemize}
\item
The packages
\href{http://ctan.org/pkg/docmute}{\textsf{docmute}},
\href{http://ctan.org/pkg/includex}{\textsf{includex}} and
\href{http://ctan.org/pkg/standalone}{\textsf{standalone}}
provide commands to include only the document body of
a child file thus allowing both files to be compiled individually.
\item
The packages \href{http://ctan.org/pkg/subdocs}{\textsf{subdocs}}
and \href{http://ctan.org/pkg/subfiles}{\textsf{subfiles}}
provide structures in which the main and child documents can be
encapsulated and allowing them to be compiled individually.
The inclusion mechanism is different from the conventional |\include|.
\item
The package \href{http://ctan.org/pkg/combine}{\textsf{combine}}
is an elaborate solution to combine several documents into one.
\end{itemize}
%
See also the CTAN topic \href{http://ctan.org/topic/subdocs}{\textsf{subdocs}}
for further related packages.
The present package differs from the above solutions in that
a document structure constructed with the conventional |\include| mechanism
just needs two extra commands at the top of every file
such that all constituent files can be compiled individually.

%%%%%%%%%%%%%%%%%%%%%%%%%%%%%%%%%%%%%%%%%%%%%%%%%%%%%%%%%%%%%%%%%%%%%%%%%%%%%%%%
%\subsection{Feature Suggestions}
%
%The following is a list of features which may be useful for future
%versions of this package:
%%
%\begin{itemize}
%\item
%\ldots
%\end{itemize}

%%%%%%%%%%%%%%%%%%%%%%%%%%%%%%%%%%%%%%%%%%%%%%%%%%%%%%%%%%%%%%%%%%%%%%%%%%%%%%%%
\subsection{Revision History}

%%%%%%%%%%%%%%%%%%%%%%%%%%%%%%%%%%%%%%%%
\paragraph{v2.0:} 2018/12/30

\begin{itemize}
\item
immediate forward processing
\item
added |\childdocby| mechanism
\item
manual restructured
\end{itemize}

%%%%%%%%%%%%%%%%%%%%%%%%%%%%%%%%%%%%%%%%
\paragraph{v1.6:} 2018/01/17

\begin{itemize}
\item
application for development of include files
\item
corrections to manual
\end{itemize}

%%%%%%%%%%%%%%%%%%%%%%%%%%%%%%%%%%%%%%%%
\paragraph{v1.5:} 2017/05/21

\begin{itemize}
\item
more complete structuring introduced
\item
|\childdocof| introduced
\item
|\childdoc| renamed to |\childdocmain|
\item
|\childredirect| renamed to |\childdocforward| and |\childdocforwardprefix|
and functionality expanded
\end{itemize}

%%%%%%%%%%%%%%%%%%%%%%%%%%%%%%%%%%%%%%%%
\paragraph{v1.0:} 2017/04/27

\begin{itemize}
\item
manual and install package
\item
first version published on CTAN
\end{itemize}

%%%%%%%%%%%%%%%%%%%%%%%%%%%%%%%%%%%%%%%%
\paragraph{v0.6:} 2017/04/26

\begin{itemize}
\item
redirection mechanism added
\end{itemize}

%%%%%%%%%%%%%%%%%%%%%%%%%%%%%%%%%%%%%%%%
\paragraph{v0.5:} 2017/04/26

\begin{itemize}
\item
functionality in definition file
\end{itemize}


%%%%%%%%%%%%%%%%%%%%%%%%%%%%%%%%%%%%%%%%%%%%%%%%%%%%%%%%%%%%%%%%%%%%%%%%%%%%%%%%
%%%%%%%%%%%%%%%%%%%%%%%%%%%%%%%%%%%%%%%%%%%%%%%%%%%%%%%%%%%%%%%%%%%%%%%%%%%%%%%%
%%%%%%%%%%%%%%%%%%%%%%%%%%%%%%%%%%%%%%%%%%%%%%%%%%%%%%%%%%%%%%%%%%%%%%%%%%%%%%%%
\appendix

\settowidth\MacroIndent{\rmfamily\scriptsize 000\ }

 \DocInput{childdoc.dtx}

\end{document}
%</driver>
% \fi
%
% %%%%%%%%%%%%%%%%%%%%%%%%%%%%%%%%%%%%%%%%%%%%%%%%%%%%%%%%%%%%%%%%%%%%%%%%%%%%%%
% %%%%%%%%%%%%%%%%%%%%%%%%%%%%%%%%%%%%%%%%%%%%%%%%%%%%%%%%%%%%%%%%%%%%%%%%%%%%%%
% \section{Sample}
%\iffalse
%<*samplemain>
%\fi
%
% The following presents a sample document
% with two chapters, two parts, a title page,
% a compile flag as well as three forwarding files to set the flag.
% It consists of eight |.tex| files:
% \begin{center}
% \begin{tabular}{ll}
% |cdocsamp.tex|&main file\\
% |cdocsch1.tex|&include file for chapter 1\\
% |cdocsch2.tex|&include file for chapter 2\\
% |cdocspt3.tex|&include file for part 3\\
% |cdocspt4.tex|&include file for part 4\\
% |cdocsdrf.tex|&forwarding file for main file in draft mode\\
% |cdocsfi1.tex|&forwarding file for final version of chapter 1\\
% |cdocsfi2.tex|&forwarding file for final version of chapter 2\\
% \end{tabular}
% \end{center}
% Each of the eight files can be compiled directly by the \LaTeX{} compiler.
%
% %%%%%%%%%%%%%%%%%%%%%%%%%%%%%%%%%%%%%%
% \paragraph{Main File.}
%
% The main file is called |cdocsamp.tex|.
%
% Load the \textsf{childdoc} definitions and
% declare the filename for the main document:
%    \begin{macrocode}
\input{childdoc.def}
\childdocmain{}
%    \end{macrocode}

% Optional override for |\version| flag:
%    \begin{macrocode}
%%\ifchilddoc\else\providecommand{\version}{draft}\fi
%    \end{macrocode}

% Define the default values for the |\version| flag
% (|final| for the main file and |draft| for childs):
%    \begin{macrocode}
\ifchilddoc
\providecommand{\version}{draft}
\else
\providecommand{\version}{final}
\fi
%    \end{macrocode}

% Load the standard document class:
%    \begin{macrocode}
\documentclass[12pt]{article}
%    \end{macrocode}

% Start the document body:
%    \begin{macrocode}
\begin{document}
%    \end{macrocode}

% Declare a title page.
% Print title, part of document being processed and version flag:
%    \begin{macrocode}
\addtocounter{page}{-1}
\begin{center}
{\LARGE\bfseries{}childdoc example\par}
\vspace{1cm}
\ifchilddoc
\ifchilddocmanual part\else chapter\fi:
`\childdocname' of `\childdocjob'\par
\else
main document: `\childdocjob'\par
\fi
version: \version\par
\end{center}
\newpage
%    \end{macrocode}

% Manually include selected file,
% otherwise process as usual:
%    \begin{macrocode}
\ifchilddocmanual
\section*{part `\childdocname'}
\input{\childdocname}
\else
%    \end{macrocode}

% Include the two chapters:
%    \begin{macrocode}
\include{cdocsch1}
\include{cdocsch2}
%    \end{macrocode}

% Include the two parts unless only chapters should be displayed:
%    \begin{macrocode}
\ifchilddoc\else
\section{part three}
\input{cdocspt3}
\section{part four}
\input{cdocspt4}
\fi
%    \end{macrocode}

% Process as usual until here:
%    \begin{macrocode}
\fi
%    \end{macrocode}

% End of document body:
%    \begin{macrocode}
\end{document}
%    \end{macrocode}
%\iffalse
%</samplemain>
%\fi
%
% %%%%%%%%%%%%%%%%%%%%%%%%%%%%%%%%%%%%%%
% \paragraph{Chapter Include Files.}
%
% The include files are called |cdocsch1.tex| and |cdocsch2.tex|.
%
%\iffalse
%<*samplechap1|samplechap2>
%\fi

% Optional override for |\version| flag:
%    \begin{macrocode}
%%\providecommand{\version}{final}
%    \end{macrocode}

% Include the main document:
%    \begin{macrocode}
\input{childdoc.def}
\childdocof{cdocsamp}
%    \end{macrocode}

%\iffalse
%</samplechap1|samplechap2>
%\fi
%
%\iffalse
%<*samplechap1>
%\fi
% Some text for chapter 1:
%    \begin{macrocode}
\section{one}
some text in chapter one
%    \end{macrocode}

%\iffalse
%</samplechap1>
%\fi
% Some text for chapter 2:
%\iffalse
%<*samplechap2>
%\fi
%    \begin{macrocode}
\section{two}
more text in chapter two
%    \end{macrocode}

%\iffalse
%</samplechap2>
%\fi
%
% %%%%%%%%%%%%%%%%%%%%%%%%%%%%%%%%%%%%%%
% \paragraph{Part Include Files.}
%
% The include files are called |cdocspt3.tex| and |cdocspt4.tex|.
%
%\iffalse
%<*samplepart3|samplepart4>
%\fi

% Optional override for |\version| flag:
%    \begin{macrocode}
%%\providecommand{\version}{final}
%    \end{macrocode}

% Include the main document:
%    \begin{macrocode}
\input{childdoc.def}
\childdocby{cdocsamp}
%    \end{macrocode}

%\iffalse
%</samplepart3|samplepart4>
%\fi
%
%\iffalse
%<*samplepart3>
%\fi
% Some text for part 3:
%    \begin{macrocode}
some text in part three
%    \end{macrocode}

%\iffalse
%</samplepart3>
%\fi
% Some text for part 4:
%\iffalse
%<*samplepart4>
%\fi
%    \begin{macrocode}
more text in part four
%    \end{macrocode}

%\iffalse
%</samplepart4>
%\fi
%
% %%%%%%%%%%%%%%%%%%%%%%%%%%%%%%%%%%%%%%
% \paragraph{Forwarding for a Complete Draft.}
%
% The following forwarding file |cdocsdrf.tex|
% compiles the main document in draft mode:
%\iffalse
%<*sampledraft>
%\fi
%    \begin{macrocode}
\def\version{draft}
\input{childdoc.def}
\childdocforward{cdocsamp}
%    \end{macrocode}

%\iffalse
%</sampledraft>
%\fi
%
% %%%%%%%%%%%%%%%%%%%%%%%%%%%%%%%%%%%%%%
% \paragraph{Forwarding for Final Version of the Chapters.}
%
% The following forwarding files |cdocsfn1.tex| and |cdocsfn2.tex|
% (with identical content)
% compile the final versions of the child documents
% |cdocsch1.tex| and |cdocsch2.tex|, respectively:
%\iffalse
%<*samplefinal>
%\fi
%    \begin{macrocode}
\def\version{final}
\input{childdoc.def}
\childdocforwardprefix[cdocsamp]{cdocsfn}{cdocsch}
%    \end{macrocode}

%\iffalse
%</samplefinal>
%\fi
%
% %%%%%%%%%%%%%%%%%%%%%%%%%%%%%%%%%%%%%%
% \paragraph{Command Line Processing.}
%
% The following three command lines generate the output files
% |cdocscld|, |cdocscl1| and |cdocscl2|
% which should be identical to
% |cdocsdrf|, |cdocsch1| and |cdocsfn2|, respectively:
% \begin{center}
% \begin{tabular}{l}
% |latex -jobname cdocscld \|\\
% |  "\def\version{draft}\input{childdoc.def}\childdocforward{cdocsamp}"|\\
% |latex -jobname cdocscl1 \|\\
% |  "\input{childdoc.def}\childdocforward[cdocsamp]{cdocsch1}"|\\
% |latex -jobname cdocscl2 \|\\
% |  "\def\version{final}\input{childdoc.def}\childdocforward{cdocsch2}"|
% \end{tabular}
% \end{center}
% Note that the trailing backslash on each first line
% merely continues the input to the second line
% (for convenient cut ant paste).
% Furthermore, the command |latex| can be replaced by any
% of its alternative versions such as |pdflatex|.
%
% %%%%%%%%%%%%%%%%%%%%%%%%%%%%%%%%%%%%%%%%%%%%%%%%%%%%%%%%%%%%%%%%%%%%%%%%%%%%%%
% %%%%%%%%%%%%%%%%%%%%%%%%%%%%%%%%%%%%%%%%%%%%%%%%%%%%%%%%%%%%%%%%%%%%%%%%%%%%%%
% \section{Implementation}
%\iffalse
%<*package>
%\fi
%
% This section describes the definitions file |childdoc.def|.

% The definitions cannot be loaded using |\usepackage| or |\RequirePackage|
% which has a mechanism to prevent loading a style file more than once.
% When loading the definitions by means of |\input|
% multiple instances have to be prevented manually:
%\iffalse
%This code needs to be before the `\ProvidesFile' directive
%which is defined at the beginning of this file.
%Therefore it is also placed there and commented out here.
%</package>
%<*discard>
%\fi
%    \begin{macrocode}
\ifdefined\childdocmain\endinput\fi
%    \end{macrocode}
%\iffalse
%</discard>
%<*package>
%\fi
%
% \macro{\ifchilddoc}
% \macro{\ifchilddocmanual}
% The conditional |\ifchilddoc| tells whether a
% child (true) or main (false) document is being compiled.
% The conditional |\ifchilddocmanual| tells whether
% the |\includeonly| mechanism is used (false) or
% the selection of child files must be performed manually (true).
% The definitions initialise to false:
%    \begin{macrocode}
\newif\ifchilddoc
\newif\ifchilddocmanual
%    \end{macrocode}

% \macro{\childdocname}
% \macro{\childdocjob}
% The macro |\childdocname| stores the name of the main document
% to be compiled. The macro |\childdocjob| stores the name of
% the document on which the \LaTeX{} compiler was originally invoked.
% The content of |\jobname| cannot be compared
% to filenames specified in the source due to different catcodes.
% The following code rescans |\jobname|, stores the result
% in |\childdocname| and saves a copy in |\childdocjob|:
%    \begin{macrocode}
\edef\childdocname{\scantokens\expandafter{\jobname\noexpand}}
\let\childdocjob\childdocname
%    \end{macrocode}

% \macro{\childdocdisable}
% The macro |\childdocdisable| prevents the main file
% from being processed more than once.
% At this stage, the main document command |\childdocmain|
% is assumed to be called once again where it should do nothing.
% Any subsequent call to it should prevent
% a secondary processing of the main document
% It overwrites the forwarding commands
% |\childdocof| and |\childdocforward|
% with empty macros to prevent further inclusions of the main document:
%    \begin{macrocode}
\newcommand{\childdocdisable}
{
  \renewcommand{\childdocmain}[1]{\renewcommand{\childdocmain}[1]{\endinput}}
  \renewcommand{\childdocof}[1]{}
  \renewcommand{\childdocby}[2][]{}
  \renewcommand{\childdocforward}[2][]{}
  \renewcommand{\childdocdisable}{}
}
%    \end{macrocode}

% \macro{\childdocmain}
% The macro |\childdocmain| is to be called at the top of the main file
% with nothing or the main filename (without extension) as argument.
% First, it breaks loops.
% If the argument is not empty and does not match |\childdocname|
% (which is set by the first inclusion of |childdoc.def|),
% |\ifchilddoc| is set to true, |\includeonly| is applied to the child file
% and |\jobname| is set to the main file
% (for proper handling of |.aux| files):
%    \begin{macrocode}
\newcommand{\childdocmain}[1]
{
  \childdocdisable\childdocmain{}
  \if?#1?\else
    \begingroup
      \def\childdoctmp{#1}
      \ifx\childdoctmp\childdocname
        \def\childdoctmp{}
      \else
        \def\childdoctmp
        {
          \childdoctrue
          \includeonly{\childdocname}
          \def\childdocjob{#1}
          \def\jobname{#1}
        }
      \fi
      \expandafter
    \endgroup
    \childdoctmp
  \fi
}
%    \end{macrocode}

% \macro{\childdocof}
% The command |\childdocof| redirects
% compilation to the main file |#1|.
%    \begin{macrocode}
\newcommand{\childdocof}[1]
{
  \childdocdisable
  \childdoctrue
  \includeonly{\childdocname}
  \def\jobname{#1}
  \def\childdocjob{#1}
  \input{#1}
}
%    \end{macrocode}

% \macro{\childdocby}
% The command |\childdocby| ....
%    \begin{macrocode}
\newcommand{\childdocby}[2][]
{
  \childdocdisable
  \childdoctrue
  \childdocmanualtrue
  \if?#1?\else
    \def\jobname{#2}
  \fi
  \def\childdocjob{#2}
  \input{#2}
  \endinput
}
%    \end{macrocode}

% \macro{\childdocforward}
% The command |\childdocforward| redirects
% compilation to the main file or
% (if the optional argument is given) a child file.
% Parameters are set as if the main file
% or a child file starting with |\childdocof| was compiled.
% Then compilation is handed over to the main file:
%    \begin{macrocode}
\newcommand{\childdocforward}[2][]
{
  \begingroup
    \if?#1?
      \def\childdoctmp
      {
        \def\childdocname{#2}
        \def\childdocjob{#2}
        \def\jobname{#2}
        \input{#2}
        \endinput
      }
    \else
      \def\childdoctmp
      {
        \childdocdisable
        \def\childdocname{#2}
        \childdoctrue
        \includeonly{#2}
        \def\childdocjob{#1}
        \def\jobname{#1}
        \input{#1}
        \endinput
      }
    \fi
    \expandafter
  \endgroup
  \childdoctmp
}
%    \end{macrocode}

% \macro{\childdocforwardprefix}
% The command |\childdocforwardprefix| redirects
% compilation to the main or a child file by means of a pattern.
% The prefix |#1| in the current filename is replaced by |#2|
% and the suffix of the current filename is kept
% (it is assumed that the filename does not contain the substring `|~~~|'
% which is used as a delimiter).
% Compilation is handed over to the new file by |\childdocforward|:
%    \begin{macrocode}
\newcommand{\childdocforwardprefix}[3][]
{
  \begingroup
    \def\childdocextract #2##1~~~{\def\childdoctmp{\childdocforward[#1]{#3##1}}}
    \expandafter\childdocextract\childdocname~~~
    \expandafter
  \endgroup
  \childdoctmp
}
%    \end{macrocode}

% \macro{\childdoc}
% The deprecated macro |\childdoc| is a legacy version of |\childdocmain|:
%    \begin{macrocode}
\newcommand{\childdoc}{\childdocmain}
%    \end{macrocode}

% \macro{\childdocredirect}
% The deprecated macro |\childdocredirect| is a legacy version
% of |\childdocforward| and |\childdocforwardprefix|:
%    \begin{macrocode}
\newcommand{\childdocredirect}[2][]
{
  \begingroup
    \if?#1?
      \def\childdoctmp{\childdocforward{#2}}
    \else
      \def\childdoctmp{\childdocforwardprefix{#1}{#2}}
    \fi
    \expandafter
  \endgroup
  \childdoctmp
}
%    \end{macrocode}

%\iffalse
%</package>
%\fi
%
\endinput
|\\
|\childdocforwardprefix[|\textit{main}|]{|\textit{prefix}|}{|\textit{dest}|}|
\end{tabular}
\end{center}
%
the destination file is determined by a pattern
depending on the current file:
To make this work, the current file must be called
`{\textit{prefix}\hspace{0.2em}\textit{suffix}}'
with \textit{prefix} matching precisely the argument.
Processing is then passed on to the file
`{\textit{dest}\hspace{0.2em}\textit{suffix}}'.
Surely, the same effect is achieved by
directly specifying the
argument `{\textit{dest}\hspace{0.2em}\textit{suffix}}'
in the first form.
However, that requires to set up a different file
for each child. With the alternative form of the command
all these files can have exactly the same content
which simplifies setting them up and maintaining them.

For example, the following file |draft.tex|
with a compilation flag |\version| as described in \secref{sec:flags}
compiles the main document as a draft:
%
\begin{center}
\begin{tabular}{l}
|\def\version{draft}|\\
|% \iffalse
%
% childdoc.dtx Copyright (C) 2017-2018 Niklas Beisert
%
% This work may be distributed and/or modified under the
% conditions of the LaTeX Project Public License, either version 1.3
% of this license or (at your option) any later version.
% The latest version of this license is in
%   http://www.latex-project.org/lppl.txt
% and version 1.3 or later is part of all distributions of LaTeX
% version 2005/12/01 or later.
%
% This work has the LPPL maintenance status `maintained'.
%
% The Current Maintainer of this work is Niklas Beisert.
%
% This work consists of the files childdoc.dtx and childdoc.ins
% and the derived files childdoc.def and cdocsamp.tex with
% cdocsch1.tex, cdocsch2.tex, cdocsdrf.tex, cdocsfn1.tex, cdocsfn2.tex.
%
%<package>\ifdefined\childdocmain\endinput\fi
%<package>\ProvidesFile{childdoc.def}[2018/12/30 v2.0 child document driver]
%<samplemain>\ProvidesFile{cdocsamp.tex}[2018/12/30 v2.0 sample for childdoc]
%<*driver>
%\ProvidesFile{childdoc.drv}[2018/12/30 v2.0 childdoc reference manual file]
\PassOptionsToClass{10pt,a4paper}{article}
\documentclass{ltxdoc}

\usepackage[margin=35mm]{geometry}
\usepackage{hyperref}
\usepackage{hyperxmp}
\usepackage[usenames]{color}

\hypersetup{colorlinks=true}
\hypersetup{pdfstartview=FitH}
\hypersetup{pdfpagemode=UseNone}
\hypersetup{pdfsource={}}
\hypersetup{pdflang={en-UK}}
\hypersetup{pdfcopyright={Copyright 2017-2018 Niklas Beisert.
  This work may be distributed and/or modified under the
  conditions of the LaTeX Project Public License, either version 1.3
  of this license or (at your option) any later version.}}
\hypersetup{pdflicenseurl={http://www.latex-project.org/lppl.txt}}
\hypersetup{pdfcontactaddress={ETH Zurich, ITP, HIT K,
  Wolfgang-Pauli-Strasse 27}}
\hypersetup{pdfcontactpostcode={8093}}
\hypersetup{pdfcontactcity={Zurich}}
\hypersetup{pdfcontactcountry={Switzerland}}
\hypersetup{pdfcontactemail={nbeisert@itp.phys.ethz.ch}}
\hypersetup{pdfcontacturl={http://people.phys.ethz.ch/\xmptilde nbeisert/}}

\newcommand{\secref}[1]{\hyperref[#1]{section \ref*{#1}}}

\parskip1ex
\parindent0pt
\let\olditemize\itemize
\def\itemize{\olditemize\parskip0pt}

\begin{document}

\title{The \textsf{childdoc} Package}
\hypersetup{pdftitle={The childdoc Package}}
\author{Niklas Beisert\\[2ex]
  Institut f\"ur Theoretische Physik\\
  Eidgen\"ossische Technische Hochschule Z\"urich\\
  Wolfgang-Pauli-Strasse 27, 8093 Z\"urich, Switzerland\\[1ex]
  \href{mailto:nbeisert@itp.phys.ethz.ch}
  {\texttt{nbeisert@itp.phys.ethz.ch}}}
\hypersetup{pdfauthor={Niklas Beisert}}
\hypersetup{pdfsubject={Manual for the LaTeX2e Package childdoc}}
\date{30 December 2018, \textsf{v2.0}}
\maketitle

\begin{abstract}\noindent
\textsf{childdoc} is a \LaTeXe{} package
that enables the direct compilation
of document sections included by |\include|
to individual files.
\end{abstract}

\begingroup
\parskip0ex
\tableofcontents
\endgroup

%%%%%%%%%%%%%%%%%%%%%%%%%%%%%%%%%%%%%%%%%%%%%%%%%%%%%%%%%%%%%%%%%%%%%%%%%%%%%%%%
%%%%%%%%%%%%%%%%%%%%%%%%%%%%%%%%%%%%%%%%%%%%%%%%%%%%%%%%%%%%%%%%%%%%%%%%%%%%%%%%
\section{Introduction}

\LaTeX{} provides a mechanism to structure a large document (such as a book)
into a main file and several child files (containing the chapters)
using the |\include| command.
This mechanism is beneficial for documents
which span hundreds of pages in order to
make the source file(s) more manageable.
Moreover, compilation can be restricted to
selected child files by means of the |\includeonly| command.
The latter feature can be used to reduce the compilation time while editing
(this was significantly more useful in the earlier days of \LaTeX{})
or to generate a smaller document which is easier to navigate.
Another application of |\includeonly| is to generate
documents consisting of selected parts of the complete document.

However, there are a few drawbacks of the plain |\include| mechanism:
\begin{itemize}
\item
The child files cannot be compiled on their own,
they can only be compiled via the main file.
A naive editing environment
(such as a text editor with an option
to have the current file processed by \LaTeX)
may require one to switch to the main file before compiling;
attempting to compile the child file produces errors.
\item
The main file must be modified (each time)
to adjust the |\includeonly| command
to the present needs. This easily leaves the main file in a messy state.
\item
The generated document will always carry the filename
of the main document. This is inconvenient if
several child files are to be compiled and
to be kept for distribution.
\end{itemize}

The present package provides a simple interface
to make child files individually compilable by \LaTeX{}.
Compiling a child file then has the same effect as compiling
the main file with an |\includeonly| command
to select the appropriate child.
Moreover the generated document will carry the name of the child
rather than the main file.
This resolves all three above issues.

This feature is meant to make the editing of books,
thesis documents and lecture notes somewhat more convenient.
However, the package can also be used efficiently for
composing a series of documents (such as exercise sheets)
which are typically distributed individually.
It then assists the author in generating the individual documents
(potentially in different versions)
as well as a document containing the collected series.
Another application is in developing style files
or other kinds of included material
where compilation of the style file could redirect
to a sample or test file.

%%%%%%%%%%%%%%%%%%%%%%%%%%%%%%%%%%%%%%%%%%%%%%%%%%%%%%%%%%%%%%%%%%%%%%%%%%%%%%%%
%%%%%%%%%%%%%%%%%%%%%%%%%%%%%%%%%%%%%%%%%%%%%%%%%%%%%%%%%%%%%%%%%%%%%%%%%%%%%%%%
\section{Usage}

First of all, the package \textsf{childdoc} is \emph{not} a standard
\LaTeXe{} |.sty| style file! Therefore it needs to be invoked in
a non-standard way.

%%%%%%%%%%%%%%%%%%%%%%%%%%%%%%%%%%%%%%%%%%%%%%%%%%%%%%%%%%%%%%%%%%%%%%%%%%%%%%%%
\subsection{Included Files}
\label{sec:include}

%%%%%%%%%%%%%%%%%%%%%%%%%%%%%%%%%%%%%%%%
\DescribeMacro{\childdocmain}
To use the package, add the commands
\begin{center}
\begin{tabular}{l}
|\input{childdoc.def}|\\
|\childdocmain{}|\\
\end{tabular}
\end{center}
at the very top of the main \LaTeX{} file,
in particular \emph{before} the |\documentclass| statement!
The argument of |\childdocmain| should be left empty
(but it must be present).

%%%%%%%%%%%%%%%%%%%%%%%%%%%%%%%%%%%%%%%%
\DescribeMacro{\childdocof}
Furthermore, add the commands
\begin{center}
\begin{tabular}{l}
|\input{childdoc.def}|\\
|\childdocof{|\textit{main}|}|\\
\end{tabular}
\end{center}
at the top of every child file \textit{child}
which is included by |\include{|\textit{child}|}|
from within the main file
(or at least for those files to be compiled individually).
The argument \textit{main} must be the filename of the main file.

There are a couple of
considerations in setting up the main and child documents:

%%%%%%%%%%%%%%%%%%%%%%%%%%%%%%%%%%%%%%%%
\paragraph{Restrictions.}

Please note the following restrictions:
\begin{itemize}
\item
|\childdocmain| must be called with one argument \textit{main}
to ensure compatibility with earlier version of the package.
It must either be empty (|\childdocmain{}|)
or precisely match the filename of the main file in which it is specified.
See \secref{sec:detection} for further information.
\item
The filename \textit{main} must be specified without the |.tex| extension.
\item
The filename \textit{main} is case sensitive
(even in case-insensitive file systems)
due to internal string comparison.
\item
The argument \textit{main} should be fully expanded, it cannot be a macro.
\item
Subdirectories and special characters should be avoided in filenames.
\item
The command |\childdocmain{|\textit{main}|}| must be followed by a whitespace.
It should not be followed immediately by another command
or by a comment mark `|%|'.
This is because the \TeX{} parser reads the token immediately following
the argument of |\childdocmain| and puts it
at the beginning of every child section;
however, a white\-space is ignored.
\end{itemize}

%%%%%%%%%%%%%%%%%%%%%%%%%%%%%%%%%%%%%%%%
\paragraph{Content of Main File.}

It is advisable to place all content in the child files included by |\include|.
Any output contained in the main file will appear in all child documents
unless suppressed manually;
it cannot be suppressed automatically by the |\includeonly| directive
and thus should normally be avoided.
A method to include some content in the main file
by means of conditional processing is described in \secref{sec:conditional}.

%%%%%%%%%%%%%%%%%%%%%%%%%%%%%%%%%%%%%%%%
\paragraph{Page Numbering.}

When only a part of the document is compiled,
the appropriate numbering of pages
(as well as other status parameters)
is determined from the |.aux| files.
The latter contain information from previous passes.
However this information needs to propagate through
all intermediate child documents.
Therefore the page numbering in child documents may well
be inconsistent until the complete document is compiled at least once.

A useful (if unconventional) way to always ensure a consistent
page numbering is to restart the numbering in each child document
and denote the pages by `\textit{child}|.|\textit{page}'
where \textit{child} represents the chapter/section number of the child file.
This can be achieved by the command
|\numberwithin{page}{|\textit{child}|}|
of the \textsf{amsmath} package
where \textit{child} can be |chapter| or |section|
depending on the chosen structuring.
Alternatively, one can modify the macro |\thepage| appropriately
and reset the counter |page| at the start of each child file.

%%%%%%%%%%%%%%%%%%%%%%%%%%%%%%%%%%%%%%%%%%%%%%%%%%%%%%%%%%%%%%%%%%%%%%%%%%%%%%%%
\subsection{Conditional Processing}
\label{sec:conditional}

The package provides a mechanism to compile different versions
of a document. To customise the versions further some conditional processing
can come in handy to distinguish which version is being compiled.
The package provides two macros to describe the compilation context:

%%%%%%%%%%%%%%%%%%%%%%%%%%%%%%%%%%%%%%%%
\DescribeMacro{\ifchilddoc}
The conditional |\ifchilddoc| distinguishes between the compilation of
child documents and the main document:
%
\begin{center}
|\ifchilddoc |\textit{child-code}| |[|\||else |\textit{main-code}]| \||fi|
\end{center}

%%%%%%%%%%%%%%%%%%%%%%%%%%%%%%%%%%%%%%%%
\DescribeMacro{\childdocname}
\DescribeMacro{\childdocjob}
The macro |\childdocname| contains the filename (without extension)
of the main or child file being processed.
Note that |\childdocjob| will always contain the name of the main file.

%%%%%%%%%%%%%%%%%%%%%%%%%%%%%%%%%%%%%%%%
\paragraph{Title Page.}

Conditional processing can be used to include a title or banner page
in the main document when proper precautions are taken.
Importantly, the code in the main file should ensure that the page counter
(as well as other status parameters which are stored in the |.aux| files)
takes the same value after the conditional processing.
Otherwise the page numbers may take divergent values
depending on which part is compiled.

For example, a title page could be declared by:
%
\begin{center}
\begin{tabular}{l}
|\ifchilddoc\||else|\\
|\addtocounter{page}{-1}|\\
\textit{code for title page}\\
|\newpage|\\
|\||fi|
\end{tabular}
\end{center}
%
A banner page for the child documents can be generated by:
%
\begin{center}
\begin{tabular}{l}
|\ifchilddoc|\\
|\addtocounter{page}{-1}|\\
\textit{code for banner page}\\
|\newpage|\\
|\||fi|
\end{tabular}
\end{center}
%
Here one could write a message such as:
\begin{center}
|This is the part \childdocname{} of \childdocjob{}.|
\end{center}

%%%%%%%%%%%%%%%%%%%%%%%%%%%%%%%%%%%%%%%%%%%%%%%%%%%%%%%%%%%%%%%%%%%%%%%%%%%%%%%%
\subsection{Flags}
\label{sec:flags}

The package makes it easy to generate different versions
of the main or child documents.
To this end compilation flags can be defined
and assigned different default values.
They will be particularly useful in conjunction
with the forwarding mechanism described in \secref{sec:forward}.

For example, it may be useful to have a flag |\version|
which can be set to |draft| or |final|.
The document source will contain some conditional code
depending on the value of |\version|.
Suppose further, the flag should default to |final| for the main file
and to |draft| for child files
which is a natural assignment for editing the document.
This is achieved by placing the following code
in the preamble of the main document
(below the |\childdocmain| directive):
%
\begin{center}
\begin{tabular}{l}
|\ifchilddoc|\\
|\providecommand{\version}{draft}|\\
|\||else|\\
|\providecommand{\version}{final}|\\
|\||fi|
\end{tabular}
\end{center}
%
The definition by |\providecommand| makes sure
that previous definitions are not overwritten.
Further statements |\providecommand{\version}{...}|
can thus be added before the above code to override it.

For the main file, one might add a line
(between |\childdocmain| and the above block)
%
\begin{center}
|%\ifchilddoc\||else\providecommand{\version}{draft}\||fi|
\end{center}
%
which can be uncommented to produce a draft version.
Likewise one can add a line to the very top of a child file
(above the |\childdocof{|\textit{main}|}| directive)
%
\begin{center}
|%\providecommand{\version}{final}|
\end{center}
%
which can be uncommented to produce the final version of this child document.

%%%%%%%%%%%%%%%%%%%%%%%%%%%%%%%%%%%%%%%%%%%%%%%%%%%%%%%%%%%%%%%%%%%%%%%%%%%%%%%%
\subsection{Forwarding}
\label{sec:forward}

Different versions of the main or child documents
using compilation flags as described in \secref{sec:flags}
can be (permanently) stored in different files
for convenient compilation, viewing and distribution.
To this end, the package defines a command
to pass on compilation to a different file:

%%%%%%%%%%%%%%%%%%%%%%%%%%%%%%%%%%%%%%%%
\DescribeMacro{\childdocforward}
The command |\childdocforward| redirects processing to
another source file:
%
\begin{center}
\begin{tabular}{l}
|\input{childdoc.def}|\\
|\childdocforward[|\textit{main}|]{|\textit{dest}|}|\\
\end{tabular}
\end{center}
%
The argument \textit{dest} is the destination file
(without extension).
It should be the main file or one of the child files.
Note that further \textsf{childdoc} directives
such as |\childdocof| and |\childdocforward|
in the indicated file will be processed in this form.
The optional argument \textit{main}
passes on directly to the main file \textit{main}
while pretending to compile the child \textit{dest}.
This form behaves as if \textit{dest}
issues |\childdocof{|\textit{main}|}| right away,
and no further \textsf{childdoc} directives will be processed.

%%%%%%%%%%%%%%%%%%%%%%%%%%%%%%%%%%%%%%%%
\DescribeMacro{\...prefix}
In the alternative form |\childdocforwardprefix|,
%
\begin{center}
\begin{tabular}{l}
|\input{childdoc.def}|\\
|\childdocforwardprefix[|\textit{main}|]{|\textit{prefix}|}{|\textit{dest}|}|
\end{tabular}
\end{center}
%
the destination file is determined by a pattern
depending on the current file:
To make this work, the current file must be called
`{\textit{prefix}\hspace{0.2em}\textit{suffix}}'
with \textit{prefix} matching precisely the argument.
Processing is then passed on to the file
`{\textit{dest}\hspace{0.2em}\textit{suffix}}'.
Surely, the same effect is achieved by
directly specifying the
argument `{\textit{dest}\hspace{0.2em}\textit{suffix}}'
in the first form.
However, that requires to set up a different file
for each child. With the alternative form of the command
all these files can have exactly the same content
which simplifies setting them up and maintaining them.

For example, the following file |draft.tex|
with a compilation flag |\version| as described in \secref{sec:flags}
compiles the main document as a draft:
%
\begin{center}
\begin{tabular}{l}
|\def\version{draft}|\\
|\input{childdoc.def}|\\
|\childdocforward{|\textit{main}|}|
\end{tabular}
\end{center}
%
Likewise, the following files |final|\textit{nn}|.tex|
compile the final version of the child document
|child|\textit{nn}|.tex|:
%
\begin{center}
\begin{tabular}{l}
|\def\version{final}|\\
|\input{childdoc.def}|\\
|\childdocforwardprefix{final}{child}|
\end{tabular}
\end{center}
%

Note that when several versions of a main file and/or of each child file
are to be generated, it may be convenient to set up a |Makefile| or
shell script to automatise the process.

%%%%%%%%%%%%%%%%%%%%%%%%%%%%%%%%%%%%%%%%%%%%%%%%%%%%%%%%%%%%%%%%%%%%%%%%%%%%%%%%
\subsection{Command Line Processing}
\label{sec:commandline}

The effect of redirection files can also be achieved by invoking
the \LaTeX{} compiler with a more elaborate command line.
Most conveniently this should be done as part
of a shell script or a |Makefile|.

When using \textsf{childdoc} in the main file, the following
command lines effectively perform a redirection
(note that depending on the shell being used,
backslashes may have to be doubled: `|\|' $\to$ `|\\|'):
%
\begin{center}
|... -jobname "|\textit{target}|" |\\|"|[\textit{flags}]%
|\input{childdoc.def}\childdocforward[|\textit{main}|]{|\textit{dest}|}"|
\end{center}
%
Here \textit{target} is the name of the output file,
\textit{main} is the name of the main file
and \textit{dest} is the name of the main or child file to be processed
(all filenames without extensions).
The optional argument \textit{main} can be omitted
if \textit{main} matches \textit{dest}.
Optionally, compilation \textit{flags} can be defined via |\def| commands.
This command line makes the \TeX{} engine believe
it is compiling the file \textit{target}
whose content is specified as the latter parameter.
The provided code then forwards the processing to
\textit{main} or \textit{dest} as described in \secref{sec:forward}.

%%%%%%%%%%%%%%%%%%%%%%%%%%%%%%%%%%%%%%%%%%%%%%%%%%%%%%%%%%%%%%%%%%%%%%%%%%%%%%%%
\subsection{Include by Input}
\label{sec:input}

Including child documents by |\include| has some restrictions by design.
Most notably, the content of a child document always occupies
its own set of pages; pages cannot be shared between child documents.
Usually, this behaviour makes perfect sense
because each child document contain an essential part of the document.
However, in some situations it may be desirable to compose
a document from a collection of parts
without having mandatory page breaks between then.
For this case, the package
provides a mechanism to include parts
by |\input| which can also be processed individually.
However, by construction this mechanism
requires manual handling of the content to be output.

%%%%%%%%%%%%%%%%%%%%%%%%%%%%%%%%%%%%%%%%
\DescribeMacro{\ifchilddocmanual}
The main file should be prepared as usual, see \secref{sec:include}.
However, the document body must make a distinction
between processing of an individual part and of the main document, e.g.:
%
\begin{center}
\begin{tabular}{l}
|\ifchilddocmanual|\\
|\input{\childdocname}|\\
|\||else|\\
\textit{document body with }|\input{|\textit{part}|}|\\
|\||fi|
\end{tabular}
\end{center}
%
The conditional |\ifchilddocmanual| is true whenever
a part to be included by |\input| is being compiled,
and the name of the part is stored in |\childdocname|.

%%%%%%%%%%%%%%%%%%%%%%%%%%%%%%%%%%%%%%%%
\DescribeMacro{\childdocby}
Each part to be included by |\input| should start with:
%
\begin{center}
\begin{tabular}{l}
|\input{childdoc.def}|\\
|\childdocby{|\textit{main}|}|\\
\end{tabular}
\end{center}
%
The directive |\childdocby| is similar to |\childdocof|
described in \secref{sec:include},
but the subsequent selection of content must be done manually.
To that end, both |\ifchilddoc| and |\ifchilddocmanual|
will be true upon processing of a part,
and the name of the part is stored in |\childdocname|.
Note that |\jobname| will be set to the filename of the current part
so that each part receives an individual |.aux| file
that does not interfere with the |.aux| file(s) of the main document.
This behaviour can be altered by the alternative form
|\childdocby[*]{|\textit{main}|}| (with a non-empty optional argument)
which uses the |.aux| file of the main document
by setting |\jobname| to \textit{main}.

%%%%%%%%%%%%%%%%%%%%%%%%%%%%%%%%%%%%%%%%%%%%%%%%%%%%%%%%%%%%%%%%%%%%%%%%%%%%%%%%
\subsection{Driver Development}
\label{sec:driver}

The \textsf{childdoc} mechanism can also be use for the development
of definition files such as \LaTeX{} styles or classes.
This case differs from the above setup with multiple parts
included by |\include| in that no |\includeonly| should be invoked.
This can be achieved by starting the include file
(before |\ProvidesPackage|) with:
%
\begin{center}
\begin{tabular}{l}
|\input{childdoc.def}|\\
|\childdocforward{|\textit{main}|}|\\
\end{tabular}
\end{center}
%
or alternatively with:
%
\begin{center}
\begin{tabular}{l}
|\input{childdoc.def}|\\
|\childdocby{|\textit{main}|}|\\
\end{tabular}
\end{center}
%
Both forms have slightly different effects as described above.
The main file is prepared as usual, see \secref{sec:include}.

%%%%%%%%%%%%%%%%%%%%%%%%%%%%%%%%%%%%%%%%%%%%%%%%%%%%%%%%%%%%%%%%%%%%%%%%%%%%%%%%
\subsection{Legacy Detection}
\label{sec:detection}

The directive |\childdocmain| in the main file can detect
whether the complete document or merely a child is to be compiled
even without using the directive |\childdocof|.
This method is deprecated because it is less robust
and there is no compelling reason to use it;
it is merely provided for backward compatibility
and it may be removed in future versions.

If the detection mechanism is to be used,
it is mandatory to correctly specify
the filename of the main file as the argument of |\childdocmain|:
%
\begin{center}
\begin{tabular}{l}
|\input{childdoc.def}|\\
|\childdocmain{|\textit{main}|}|\\
\end{tabular}
\end{center}
%
If |\jobname| does not match the argument \textit{main} of |\childdocmain|,
it is assumed that |\jobname| points to the child file to be compiled.
When using |\childdocmain| with the main file specified as argument,
it suffices to start a child file
with just |\input{|\textit{main}|}|
without loading of the package and using |\childdocof|.
If instead all processing is done
with the appropriate \textsf{childdoc} directives,
the argument of \textit{main} of |\childdocmain| can be empty.

An alternative version of the command line processing described
in \secref{sec:commandline} using the detection mechanism reads:
%
\begin{center}
|... -jobname "|\textit{target}|" "|[\textit{flags}]%
[|\def\jobname{|\textit{dest}|}|]|\input{|\textit{main}|}"|
\end{center}

%%%%%%%%%%%%%%%%%%%%%%%%%%%%%%%%%%%%%%%%%%%%%%%%%%%%%%%%%%%%%%%%%%%%%%%%%%%%%%%%
\subsection{Manual Code}
\label{sec:manual}

In case one cannot be certain whether the definitions file |childdoc.def|
is installed on the target \TeX{} distribution
and one prefers not to ship it,
it is conceivable to paste a few relevant commands into the sources.

To that end, drop all statements |\input{childdoc.def}|
and perform the replacements as outlined below.
Instead of |\childdocmain{|\textit{main}|}| add the following code
to the top of the main file:
%
\begin{center}
\begin{tabular}{l}
|\||ifdefined\childdocname\endinput\||fi\newif\ifchilddoc|\\
|\edef\childdocname{\scantokens\expandafter{\jobname\noexpand}}|\\
|\def\childdocmain{|\textit{main}|}\||ifx\childdocmain\childdocname\||else|\\
|\childdoctrue\includeonly{\childdocname}\let\jobname\childdocmain\||fi|\\
\end{tabular}
\end{center}
%
Instead of |\childdocof{|\textit{main}|}| just include the main file
at the top of each child file:
%
\begin{center}
|\input{|\textit{main}|}|
\end{center}
%
A simple redirection |\childdocforward{|\textit{dest}|}| is achieved by:
%
\begin{center}
|\def\jobname{|\textit{dest}|}\input{\jobname}|
\end{center}
%
The redirection with prefix
|\childdocforwardprefix[|\textit{prefix}|]{|\textit{dest}|}|
is accomplished by:
%
\begin{center}
\begin{tabular}{l}
|{\edef\jobname{\scantokens\expandafter{\jobname\noexpand}}|\\
|\def\redirectjob |\textit{prefix}|#1~~~{\gdef\jobname{|\textit{dest}|#1}}|\\
|\expandafter\redirectjob\jobname~~~}\input{\jobname}|
\end{tabular}
\end{center}

In an alternative approach,
child documents can be compiled by a specific command line
without additional code or specific definitions:
%
\begin{center}
|... -jobname "|\textit{target}|" "|[\textit{flags}]%
|\includeonly{|\textit{dest}|}\input{|\textit{main}|}"|
\end{center}
%

%%%%%%%%%%%%%%%%%%%%%%%%%%%%%%%%%%%%%%%%%%%%%%%%%%%%%%%%%%%%%%%%%%%%%%%%%%%%%%%%
%%%%%%%%%%%%%%%%%%%%%%%%%%%%%%%%%%%%%%%%%%%%%%%%%%%%%%%%%%%%%%%%%%%%%%%%%%%%%%%%
\section{Information}

%%%%%%%%%%%%%%%%%%%%%%%%%%%%%%%%%%%%%%%%%%%%%%%%%%%%%%%%%%%%%%%%%%%%%%%%%%%%%%%%
\subsection{Copyright}

Copyright \copyright{} 2017--2018 Niklas Beisert

This work may be distributed and/or modified under the
conditions of the \LaTeX{} Project Public License, either version 1.3
of this license or (at your option) any later version.
The latest version of this license is in
  \url{http://www.latex-project.org/lppl.txt}
and version 1.3 or later is part of all distributions of \LaTeX{}
version 2005/12/01 or later.

This work has the LPPL maintenance status `maintained'.

The Current Maintainer of this work is Niklas Beisert.

This work consists of the files |README.txt|, |childdoc.ins| and |childdoc.dtx|
as well as the derived files |childdoc.def|, |cdocsamp.tex|
with |cdocsch1.tex|, |cdocsch2.tex|, |cdocspt3.tex|, |cdocspt4.tex|,
|cdocsdrf.tex|, |cdocsfn1.tex|, |cdocsfn2.tex|
as well as |childdoc.pdf|.

%%%%%%%%%%%%%%%%%%%%%%%%%%%%%%%%%%%%%%%%%%%%%%%%%%%%%%%%%%%%%%%%%%%%%%%%%%%%%%%%
\subsection{Files and Installation}

The package consists of the files:
%
\begin{center}
\begin{tabular}{ll}
    |README.txt|   & readme file \\
    |childdoc.ins| & installation file \\
    |childdoc.dtx| & source file \\
    |childdoc.def| & definition file \\
    |cdocsamp.tex| & sample main file \\
    |cdocsch1.tex| & sample include file \\
    |cdocsch2.tex| & sample include file \\
    |cdocspt3.tex| & sample part file \\
    |cdocspt4.tex| & sample part file \\
    |cdocsdrf.tex| & sample redirection file \\
    |cdocsfn1.tex| & sample redirection file \\
    |cdocsfn2.tex| & sample redirection file \\
    |childdoc.pdf| & manual
\end{tabular}
\end{center}
%
The distribution consists of the files
|README.txt|, |childdoc.ins| and |childdoc.dtx|.
%
\begin{itemize}
\item
Run (pdf)\LaTeX{} on |childdoc.dtx|
to compile the manual |childdoc.pdf| (this file).
\item
Run \LaTeX{} on |childdoc.ins| to create the definitions file |childdoc.def|
and the sample |cdocsamp.tex| with include files
|cdocsch1.tex|, |cdocsch2.tex|, |cdocspt3.tex|, |cdocspt4.tex|,
|cdocsdrf.tex|, |cdocsfn1.tex|, |cdocsfn2.tex|.
Then copy the file |childdoc.def| to an appropriate directory of your \LaTeX{}
distribution, e.g.\ \textit{texmf-root}|/tex/latex/childdoc|.
\end{itemize}

%%%%%%%%%%%%%%%%%%%%%%%%%%%%%%%%%%%%%%%%%%%%%%%%%%%%%%%%%%%%%%%%%%%%%%%%%%%%%%%%
\subsection{Related CTAN Packages}

There are several other packages which offer a similar functionality:
%
\begin{itemize}
\item
The packages
\href{http://ctan.org/pkg/docmute}{\textsf{docmute}},
\href{http://ctan.org/pkg/includex}{\textsf{includex}} and
\href{http://ctan.org/pkg/standalone}{\textsf{standalone}}
provide commands to include only the document body of
a child file thus allowing both files to be compiled individually.
\item
The packages \href{http://ctan.org/pkg/subdocs}{\textsf{subdocs}}
and \href{http://ctan.org/pkg/subfiles}{\textsf{subfiles}}
provide structures in which the main and child documents can be
encapsulated and allowing them to be compiled individually.
The inclusion mechanism is different from the conventional |\include|.
\item
The package \href{http://ctan.org/pkg/combine}{\textsf{combine}}
is an elaborate solution to combine several documents into one.
\end{itemize}
%
See also the CTAN topic \href{http://ctan.org/topic/subdocs}{\textsf{subdocs}}
for further related packages.
The present package differs from the above solutions in that
a document structure constructed with the conventional |\include| mechanism
just needs two extra commands at the top of every file
such that all constituent files can be compiled individually.

%%%%%%%%%%%%%%%%%%%%%%%%%%%%%%%%%%%%%%%%%%%%%%%%%%%%%%%%%%%%%%%%%%%%%%%%%%%%%%%%
%\subsection{Feature Suggestions}
%
%The following is a list of features which may be useful for future
%versions of this package:
%%
%\begin{itemize}
%\item
%\ldots
%\end{itemize}

%%%%%%%%%%%%%%%%%%%%%%%%%%%%%%%%%%%%%%%%%%%%%%%%%%%%%%%%%%%%%%%%%%%%%%%%%%%%%%%%
\subsection{Revision History}

%%%%%%%%%%%%%%%%%%%%%%%%%%%%%%%%%%%%%%%%
\paragraph{v2.0:} 2018/12/30

\begin{itemize}
\item
immediate forward processing
\item
added |\childdocby| mechanism
\item
manual restructured
\end{itemize}

%%%%%%%%%%%%%%%%%%%%%%%%%%%%%%%%%%%%%%%%
\paragraph{v1.6:} 2018/01/17

\begin{itemize}
\item
application for development of include files
\item
corrections to manual
\end{itemize}

%%%%%%%%%%%%%%%%%%%%%%%%%%%%%%%%%%%%%%%%
\paragraph{v1.5:} 2017/05/21

\begin{itemize}
\item
more complete structuring introduced
\item
|\childdocof| introduced
\item
|\childdoc| renamed to |\childdocmain|
\item
|\childredirect| renamed to |\childdocforward| and |\childdocforwardprefix|
and functionality expanded
\end{itemize}

%%%%%%%%%%%%%%%%%%%%%%%%%%%%%%%%%%%%%%%%
\paragraph{v1.0:} 2017/04/27

\begin{itemize}
\item
manual and install package
\item
first version published on CTAN
\end{itemize}

%%%%%%%%%%%%%%%%%%%%%%%%%%%%%%%%%%%%%%%%
\paragraph{v0.6:} 2017/04/26

\begin{itemize}
\item
redirection mechanism added
\end{itemize}

%%%%%%%%%%%%%%%%%%%%%%%%%%%%%%%%%%%%%%%%
\paragraph{v0.5:} 2017/04/26

\begin{itemize}
\item
functionality in definition file
\end{itemize}


%%%%%%%%%%%%%%%%%%%%%%%%%%%%%%%%%%%%%%%%%%%%%%%%%%%%%%%%%%%%%%%%%%%%%%%%%%%%%%%%
%%%%%%%%%%%%%%%%%%%%%%%%%%%%%%%%%%%%%%%%%%%%%%%%%%%%%%%%%%%%%%%%%%%%%%%%%%%%%%%%
%%%%%%%%%%%%%%%%%%%%%%%%%%%%%%%%%%%%%%%%%%%%%%%%%%%%%%%%%%%%%%%%%%%%%%%%%%%%%%%%
\appendix

\settowidth\MacroIndent{\rmfamily\scriptsize 000\ }

 \DocInput{childdoc.dtx}

\end{document}
%</driver>
% \fi
%
% %%%%%%%%%%%%%%%%%%%%%%%%%%%%%%%%%%%%%%%%%%%%%%%%%%%%%%%%%%%%%%%%%%%%%%%%%%%%%%
% %%%%%%%%%%%%%%%%%%%%%%%%%%%%%%%%%%%%%%%%%%%%%%%%%%%%%%%%%%%%%%%%%%%%%%%%%%%%%%
% \section{Sample}
%\iffalse
%<*samplemain>
%\fi
%
% The following presents a sample document
% with two chapters, two parts, a title page,
% a compile flag as well as three forwarding files to set the flag.
% It consists of eight |.tex| files:
% \begin{center}
% \begin{tabular}{ll}
% |cdocsamp.tex|&main file\\
% |cdocsch1.tex|&include file for chapter 1\\
% |cdocsch2.tex|&include file for chapter 2\\
% |cdocspt3.tex|&include file for part 3\\
% |cdocspt4.tex|&include file for part 4\\
% |cdocsdrf.tex|&forwarding file for main file in draft mode\\
% |cdocsfi1.tex|&forwarding file for final version of chapter 1\\
% |cdocsfi2.tex|&forwarding file for final version of chapter 2\\
% \end{tabular}
% \end{center}
% Each of the eight files can be compiled directly by the \LaTeX{} compiler.
%
% %%%%%%%%%%%%%%%%%%%%%%%%%%%%%%%%%%%%%%
% \paragraph{Main File.}
%
% The main file is called |cdocsamp.tex|.
%
% Load the \textsf{childdoc} definitions and
% declare the filename for the main document:
%    \begin{macrocode}
\input{childdoc.def}
\childdocmain{}
%    \end{macrocode}

% Optional override for |\version| flag:
%    \begin{macrocode}
%%\ifchilddoc\else\providecommand{\version}{draft}\fi
%    \end{macrocode}

% Define the default values for the |\version| flag
% (|final| for the main file and |draft| for childs):
%    \begin{macrocode}
\ifchilddoc
\providecommand{\version}{draft}
\else
\providecommand{\version}{final}
\fi
%    \end{macrocode}

% Load the standard document class:
%    \begin{macrocode}
\documentclass[12pt]{article}
%    \end{macrocode}

% Start the document body:
%    \begin{macrocode}
\begin{document}
%    \end{macrocode}

% Declare a title page.
% Print title, part of document being processed and version flag:
%    \begin{macrocode}
\addtocounter{page}{-1}
\begin{center}
{\LARGE\bfseries{}childdoc example\par}
\vspace{1cm}
\ifchilddoc
\ifchilddocmanual part\else chapter\fi:
`\childdocname' of `\childdocjob'\par
\else
main document: `\childdocjob'\par
\fi
version: \version\par
\end{center}
\newpage
%    \end{macrocode}

% Manually include selected file,
% otherwise process as usual:
%    \begin{macrocode}
\ifchilddocmanual
\section*{part `\childdocname'}
\input{\childdocname}
\else
%    \end{macrocode}

% Include the two chapters:
%    \begin{macrocode}
\include{cdocsch1}
\include{cdocsch2}
%    \end{macrocode}

% Include the two parts unless only chapters should be displayed:
%    \begin{macrocode}
\ifchilddoc\else
\section{part three}
\input{cdocspt3}
\section{part four}
\input{cdocspt4}
\fi
%    \end{macrocode}

% Process as usual until here:
%    \begin{macrocode}
\fi
%    \end{macrocode}

% End of document body:
%    \begin{macrocode}
\end{document}
%    \end{macrocode}
%\iffalse
%</samplemain>
%\fi
%
% %%%%%%%%%%%%%%%%%%%%%%%%%%%%%%%%%%%%%%
% \paragraph{Chapter Include Files.}
%
% The include files are called |cdocsch1.tex| and |cdocsch2.tex|.
%
%\iffalse
%<*samplechap1|samplechap2>
%\fi

% Optional override for |\version| flag:
%    \begin{macrocode}
%%\providecommand{\version}{final}
%    \end{macrocode}

% Include the main document:
%    \begin{macrocode}
\input{childdoc.def}
\childdocof{cdocsamp}
%    \end{macrocode}

%\iffalse
%</samplechap1|samplechap2>
%\fi
%
%\iffalse
%<*samplechap1>
%\fi
% Some text for chapter 1:
%    \begin{macrocode}
\section{one}
some text in chapter one
%    \end{macrocode}

%\iffalse
%</samplechap1>
%\fi
% Some text for chapter 2:
%\iffalse
%<*samplechap2>
%\fi
%    \begin{macrocode}
\section{two}
more text in chapter two
%    \end{macrocode}

%\iffalse
%</samplechap2>
%\fi
%
% %%%%%%%%%%%%%%%%%%%%%%%%%%%%%%%%%%%%%%
% \paragraph{Part Include Files.}
%
% The include files are called |cdocspt3.tex| and |cdocspt4.tex|.
%
%\iffalse
%<*samplepart3|samplepart4>
%\fi

% Optional override for |\version| flag:
%    \begin{macrocode}
%%\providecommand{\version}{final}
%    \end{macrocode}

% Include the main document:
%    \begin{macrocode}
\input{childdoc.def}
\childdocby{cdocsamp}
%    \end{macrocode}

%\iffalse
%</samplepart3|samplepart4>
%\fi
%
%\iffalse
%<*samplepart3>
%\fi
% Some text for part 3:
%    \begin{macrocode}
some text in part three
%    \end{macrocode}

%\iffalse
%</samplepart3>
%\fi
% Some text for part 4:
%\iffalse
%<*samplepart4>
%\fi
%    \begin{macrocode}
more text in part four
%    \end{macrocode}

%\iffalse
%</samplepart4>
%\fi
%
% %%%%%%%%%%%%%%%%%%%%%%%%%%%%%%%%%%%%%%
% \paragraph{Forwarding for a Complete Draft.}
%
% The following forwarding file |cdocsdrf.tex|
% compiles the main document in draft mode:
%\iffalse
%<*sampledraft>
%\fi
%    \begin{macrocode}
\def\version{draft}
\input{childdoc.def}
\childdocforward{cdocsamp}
%    \end{macrocode}

%\iffalse
%</sampledraft>
%\fi
%
% %%%%%%%%%%%%%%%%%%%%%%%%%%%%%%%%%%%%%%
% \paragraph{Forwarding for Final Version of the Chapters.}
%
% The following forwarding files |cdocsfn1.tex| and |cdocsfn2.tex|
% (with identical content)
% compile the final versions of the child documents
% |cdocsch1.tex| and |cdocsch2.tex|, respectively:
%\iffalse
%<*samplefinal>
%\fi
%    \begin{macrocode}
\def\version{final}
\input{childdoc.def}
\childdocforwardprefix[cdocsamp]{cdocsfn}{cdocsch}
%    \end{macrocode}

%\iffalse
%</samplefinal>
%\fi
%
% %%%%%%%%%%%%%%%%%%%%%%%%%%%%%%%%%%%%%%
% \paragraph{Command Line Processing.}
%
% The following three command lines generate the output files
% |cdocscld|, |cdocscl1| and |cdocscl2|
% which should be identical to
% |cdocsdrf|, |cdocsch1| and |cdocsfn2|, respectively:
% \begin{center}
% \begin{tabular}{l}
% |latex -jobname cdocscld \|\\
% |  "\def\version{draft}\input{childdoc.def}\childdocforward{cdocsamp}"|\\
% |latex -jobname cdocscl1 \|\\
% |  "\input{childdoc.def}\childdocforward[cdocsamp]{cdocsch1}"|\\
% |latex -jobname cdocscl2 \|\\
% |  "\def\version{final}\input{childdoc.def}\childdocforward{cdocsch2}"|
% \end{tabular}
% \end{center}
% Note that the trailing backslash on each first line
% merely continues the input to the second line
% (for convenient cut ant paste).
% Furthermore, the command |latex| can be replaced by any
% of its alternative versions such as |pdflatex|.
%
% %%%%%%%%%%%%%%%%%%%%%%%%%%%%%%%%%%%%%%%%%%%%%%%%%%%%%%%%%%%%%%%%%%%%%%%%%%%%%%
% %%%%%%%%%%%%%%%%%%%%%%%%%%%%%%%%%%%%%%%%%%%%%%%%%%%%%%%%%%%%%%%%%%%%%%%%%%%%%%
% \section{Implementation}
%\iffalse
%<*package>
%\fi
%
% This section describes the definitions file |childdoc.def|.

% The definitions cannot be loaded using |\usepackage| or |\RequirePackage|
% which has a mechanism to prevent loading a style file more than once.
% When loading the definitions by means of |\input|
% multiple instances have to be prevented manually:
%\iffalse
%This code needs to be before the `\ProvidesFile' directive
%which is defined at the beginning of this file.
%Therefore it is also placed there and commented out here.
%</package>
%<*discard>
%\fi
%    \begin{macrocode}
\ifdefined\childdocmain\endinput\fi
%    \end{macrocode}
%\iffalse
%</discard>
%<*package>
%\fi
%
% \macro{\ifchilddoc}
% \macro{\ifchilddocmanual}
% The conditional |\ifchilddoc| tells whether a
% child (true) or main (false) document is being compiled.
% The conditional |\ifchilddocmanual| tells whether
% the |\includeonly| mechanism is used (false) or
% the selection of child files must be performed manually (true).
% The definitions initialise to false:
%    \begin{macrocode}
\newif\ifchilddoc
\newif\ifchilddocmanual
%    \end{macrocode}

% \macro{\childdocname}
% \macro{\childdocjob}
% The macro |\childdocname| stores the name of the main document
% to be compiled. The macro |\childdocjob| stores the name of
% the document on which the \LaTeX{} compiler was originally invoked.
% The content of |\jobname| cannot be compared
% to filenames specified in the source due to different catcodes.
% The following code rescans |\jobname|, stores the result
% in |\childdocname| and saves a copy in |\childdocjob|:
%    \begin{macrocode}
\edef\childdocname{\scantokens\expandafter{\jobname\noexpand}}
\let\childdocjob\childdocname
%    \end{macrocode}

% \macro{\childdocdisable}
% The macro |\childdocdisable| prevents the main file
% from being processed more than once.
% At this stage, the main document command |\childdocmain|
% is assumed to be called once again where it should do nothing.
% Any subsequent call to it should prevent
% a secondary processing of the main document
% It overwrites the forwarding commands
% |\childdocof| and |\childdocforward|
% with empty macros to prevent further inclusions of the main document:
%    \begin{macrocode}
\newcommand{\childdocdisable}
{
  \renewcommand{\childdocmain}[1]{\renewcommand{\childdocmain}[1]{\endinput}}
  \renewcommand{\childdocof}[1]{}
  \renewcommand{\childdocby}[2][]{}
  \renewcommand{\childdocforward}[2][]{}
  \renewcommand{\childdocdisable}{}
}
%    \end{macrocode}

% \macro{\childdocmain}
% The macro |\childdocmain| is to be called at the top of the main file
% with nothing or the main filename (without extension) as argument.
% First, it breaks loops.
% If the argument is not empty and does not match |\childdocname|
% (which is set by the first inclusion of |childdoc.def|),
% |\ifchilddoc| is set to true, |\includeonly| is applied to the child file
% and |\jobname| is set to the main file
% (for proper handling of |.aux| files):
%    \begin{macrocode}
\newcommand{\childdocmain}[1]
{
  \childdocdisable\childdocmain{}
  \if?#1?\else
    \begingroup
      \def\childdoctmp{#1}
      \ifx\childdoctmp\childdocname
        \def\childdoctmp{}
      \else
        \def\childdoctmp
        {
          \childdoctrue
          \includeonly{\childdocname}
          \def\childdocjob{#1}
          \def\jobname{#1}
        }
      \fi
      \expandafter
    \endgroup
    \childdoctmp
  \fi
}
%    \end{macrocode}

% \macro{\childdocof}
% The command |\childdocof| redirects
% compilation to the main file |#1|.
%    \begin{macrocode}
\newcommand{\childdocof}[1]
{
  \childdocdisable
  \childdoctrue
  \includeonly{\childdocname}
  \def\jobname{#1}
  \def\childdocjob{#1}
  \input{#1}
}
%    \end{macrocode}

% \macro{\childdocby}
% The command |\childdocby| ....
%    \begin{macrocode}
\newcommand{\childdocby}[2][]
{
  \childdocdisable
  \childdoctrue
  \childdocmanualtrue
  \if?#1?\else
    \def\jobname{#2}
  \fi
  \def\childdocjob{#2}
  \input{#2}
  \endinput
}
%    \end{macrocode}

% \macro{\childdocforward}
% The command |\childdocforward| redirects
% compilation to the main file or
% (if the optional argument is given) a child file.
% Parameters are set as if the main file
% or a child file starting with |\childdocof| was compiled.
% Then compilation is handed over to the main file:
%    \begin{macrocode}
\newcommand{\childdocforward}[2][]
{
  \begingroup
    \if?#1?
      \def\childdoctmp
      {
        \def\childdocname{#2}
        \def\childdocjob{#2}
        \def\jobname{#2}
        \input{#2}
        \endinput
      }
    \else
      \def\childdoctmp
      {
        \childdocdisable
        \def\childdocname{#2}
        \childdoctrue
        \includeonly{#2}
        \def\childdocjob{#1}
        \def\jobname{#1}
        \input{#1}
        \endinput
      }
    \fi
    \expandafter
  \endgroup
  \childdoctmp
}
%    \end{macrocode}

% \macro{\childdocforwardprefix}
% The command |\childdocforwardprefix| redirects
% compilation to the main or a child file by means of a pattern.
% The prefix |#1| in the current filename is replaced by |#2|
% and the suffix of the current filename is kept
% (it is assumed that the filename does not contain the substring `|~~~|'
% which is used as a delimiter).
% Compilation is handed over to the new file by |\childdocforward|:
%    \begin{macrocode}
\newcommand{\childdocforwardprefix}[3][]
{
  \begingroup
    \def\childdocextract #2##1~~~{\def\childdoctmp{\childdocforward[#1]{#3##1}}}
    \expandafter\childdocextract\childdocname~~~
    \expandafter
  \endgroup
  \childdoctmp
}
%    \end{macrocode}

% \macro{\childdoc}
% The deprecated macro |\childdoc| is a legacy version of |\childdocmain|:
%    \begin{macrocode}
\newcommand{\childdoc}{\childdocmain}
%    \end{macrocode}

% \macro{\childdocredirect}
% The deprecated macro |\childdocredirect| is a legacy version
% of |\childdocforward| and |\childdocforwardprefix|:
%    \begin{macrocode}
\newcommand{\childdocredirect}[2][]
{
  \begingroup
    \if?#1?
      \def\childdoctmp{\childdocforward{#2}}
    \else
      \def\childdoctmp{\childdocforwardprefix{#1}{#2}}
    \fi
    \expandafter
  \endgroup
  \childdoctmp
}
%    \end{macrocode}

%\iffalse
%</package>
%\fi
%
\endinput
|\\
|\childdocforward{|\textit{main}|}|
\end{tabular}
\end{center}
%
Likewise, the following files |final|\textit{nn}|.tex|
compile the final version of the child document
|child|\textit{nn}|.tex|:
%
\begin{center}
\begin{tabular}{l}
|\def\version{final}|\\
|% \iffalse
%
% childdoc.dtx Copyright (C) 2017-2018 Niklas Beisert
%
% This work may be distributed and/or modified under the
% conditions of the LaTeX Project Public License, either version 1.3
% of this license or (at your option) any later version.
% The latest version of this license is in
%   http://www.latex-project.org/lppl.txt
% and version 1.3 or later is part of all distributions of LaTeX
% version 2005/12/01 or later.
%
% This work has the LPPL maintenance status `maintained'.
%
% The Current Maintainer of this work is Niklas Beisert.
%
% This work consists of the files childdoc.dtx and childdoc.ins
% and the derived files childdoc.def and cdocsamp.tex with
% cdocsch1.tex, cdocsch2.tex, cdocsdrf.tex, cdocsfn1.tex, cdocsfn2.tex.
%
%<package>\ifdefined\childdocmain\endinput\fi
%<package>\ProvidesFile{childdoc.def}[2018/12/30 v2.0 child document driver]
%<samplemain>\ProvidesFile{cdocsamp.tex}[2018/12/30 v2.0 sample for childdoc]
%<*driver>
%\ProvidesFile{childdoc.drv}[2018/12/30 v2.0 childdoc reference manual file]
\PassOptionsToClass{10pt,a4paper}{article}
\documentclass{ltxdoc}

\usepackage[margin=35mm]{geometry}
\usepackage{hyperref}
\usepackage{hyperxmp}
\usepackage[usenames]{color}

\hypersetup{colorlinks=true}
\hypersetup{pdfstartview=FitH}
\hypersetup{pdfpagemode=UseNone}
\hypersetup{pdfsource={}}
\hypersetup{pdflang={en-UK}}
\hypersetup{pdfcopyright={Copyright 2017-2018 Niklas Beisert.
  This work may be distributed and/or modified under the
  conditions of the LaTeX Project Public License, either version 1.3
  of this license or (at your option) any later version.}}
\hypersetup{pdflicenseurl={http://www.latex-project.org/lppl.txt}}
\hypersetup{pdfcontactaddress={ETH Zurich, ITP, HIT K,
  Wolfgang-Pauli-Strasse 27}}
\hypersetup{pdfcontactpostcode={8093}}
\hypersetup{pdfcontactcity={Zurich}}
\hypersetup{pdfcontactcountry={Switzerland}}
\hypersetup{pdfcontactemail={nbeisert@itp.phys.ethz.ch}}
\hypersetup{pdfcontacturl={http://people.phys.ethz.ch/\xmptilde nbeisert/}}

\newcommand{\secref}[1]{\hyperref[#1]{section \ref*{#1}}}

\parskip1ex
\parindent0pt
\let\olditemize\itemize
\def\itemize{\olditemize\parskip0pt}

\begin{document}

\title{The \textsf{childdoc} Package}
\hypersetup{pdftitle={The childdoc Package}}
\author{Niklas Beisert\\[2ex]
  Institut f\"ur Theoretische Physik\\
  Eidgen\"ossische Technische Hochschule Z\"urich\\
  Wolfgang-Pauli-Strasse 27, 8093 Z\"urich, Switzerland\\[1ex]
  \href{mailto:nbeisert@itp.phys.ethz.ch}
  {\texttt{nbeisert@itp.phys.ethz.ch}}}
\hypersetup{pdfauthor={Niklas Beisert}}
\hypersetup{pdfsubject={Manual for the LaTeX2e Package childdoc}}
\date{30 December 2018, \textsf{v2.0}}
\maketitle

\begin{abstract}\noindent
\textsf{childdoc} is a \LaTeXe{} package
that enables the direct compilation
of document sections included by |\include|
to individual files.
\end{abstract}

\begingroup
\parskip0ex
\tableofcontents
\endgroup

%%%%%%%%%%%%%%%%%%%%%%%%%%%%%%%%%%%%%%%%%%%%%%%%%%%%%%%%%%%%%%%%%%%%%%%%%%%%%%%%
%%%%%%%%%%%%%%%%%%%%%%%%%%%%%%%%%%%%%%%%%%%%%%%%%%%%%%%%%%%%%%%%%%%%%%%%%%%%%%%%
\section{Introduction}

\LaTeX{} provides a mechanism to structure a large document (such as a book)
into a main file and several child files (containing the chapters)
using the |\include| command.
This mechanism is beneficial for documents
which span hundreds of pages in order to
make the source file(s) more manageable.
Moreover, compilation can be restricted to
selected child files by means of the |\includeonly| command.
The latter feature can be used to reduce the compilation time while editing
(this was significantly more useful in the earlier days of \LaTeX{})
or to generate a smaller document which is easier to navigate.
Another application of |\includeonly| is to generate
documents consisting of selected parts of the complete document.

However, there are a few drawbacks of the plain |\include| mechanism:
\begin{itemize}
\item
The child files cannot be compiled on their own,
they can only be compiled via the main file.
A naive editing environment
(such as a text editor with an option
to have the current file processed by \LaTeX)
may require one to switch to the main file before compiling;
attempting to compile the child file produces errors.
\item
The main file must be modified (each time)
to adjust the |\includeonly| command
to the present needs. This easily leaves the main file in a messy state.
\item
The generated document will always carry the filename
of the main document. This is inconvenient if
several child files are to be compiled and
to be kept for distribution.
\end{itemize}

The present package provides a simple interface
to make child files individually compilable by \LaTeX{}.
Compiling a child file then has the same effect as compiling
the main file with an |\includeonly| command
to select the appropriate child.
Moreover the generated document will carry the name of the child
rather than the main file.
This resolves all three above issues.

This feature is meant to make the editing of books,
thesis documents and lecture notes somewhat more convenient.
However, the package can also be used efficiently for
composing a series of documents (such as exercise sheets)
which are typically distributed individually.
It then assists the author in generating the individual documents
(potentially in different versions)
as well as a document containing the collected series.
Another application is in developing style files
or other kinds of included material
where compilation of the style file could redirect
to a sample or test file.

%%%%%%%%%%%%%%%%%%%%%%%%%%%%%%%%%%%%%%%%%%%%%%%%%%%%%%%%%%%%%%%%%%%%%%%%%%%%%%%%
%%%%%%%%%%%%%%%%%%%%%%%%%%%%%%%%%%%%%%%%%%%%%%%%%%%%%%%%%%%%%%%%%%%%%%%%%%%%%%%%
\section{Usage}

First of all, the package \textsf{childdoc} is \emph{not} a standard
\LaTeXe{} |.sty| style file! Therefore it needs to be invoked in
a non-standard way.

%%%%%%%%%%%%%%%%%%%%%%%%%%%%%%%%%%%%%%%%%%%%%%%%%%%%%%%%%%%%%%%%%%%%%%%%%%%%%%%%
\subsection{Included Files}
\label{sec:include}

%%%%%%%%%%%%%%%%%%%%%%%%%%%%%%%%%%%%%%%%
\DescribeMacro{\childdocmain}
To use the package, add the commands
\begin{center}
\begin{tabular}{l}
|\input{childdoc.def}|\\
|\childdocmain{}|\\
\end{tabular}
\end{center}
at the very top of the main \LaTeX{} file,
in particular \emph{before} the |\documentclass| statement!
The argument of |\childdocmain| should be left empty
(but it must be present).

%%%%%%%%%%%%%%%%%%%%%%%%%%%%%%%%%%%%%%%%
\DescribeMacro{\childdocof}
Furthermore, add the commands
\begin{center}
\begin{tabular}{l}
|\input{childdoc.def}|\\
|\childdocof{|\textit{main}|}|\\
\end{tabular}
\end{center}
at the top of every child file \textit{child}
which is included by |\include{|\textit{child}|}|
from within the main file
(or at least for those files to be compiled individually).
The argument \textit{main} must be the filename of the main file.

There are a couple of
considerations in setting up the main and child documents:

%%%%%%%%%%%%%%%%%%%%%%%%%%%%%%%%%%%%%%%%
\paragraph{Restrictions.}

Please note the following restrictions:
\begin{itemize}
\item
|\childdocmain| must be called with one argument \textit{main}
to ensure compatibility with earlier version of the package.
It must either be empty (|\childdocmain{}|)
or precisely match the filename of the main file in which it is specified.
See \secref{sec:detection} for further information.
\item
The filename \textit{main} must be specified without the |.tex| extension.
\item
The filename \textit{main} is case sensitive
(even in case-insensitive file systems)
due to internal string comparison.
\item
The argument \textit{main} should be fully expanded, it cannot be a macro.
\item
Subdirectories and special characters should be avoided in filenames.
\item
The command |\childdocmain{|\textit{main}|}| must be followed by a whitespace.
It should not be followed immediately by another command
or by a comment mark `|%|'.
This is because the \TeX{} parser reads the token immediately following
the argument of |\childdocmain| and puts it
at the beginning of every child section;
however, a white\-space is ignored.
\end{itemize}

%%%%%%%%%%%%%%%%%%%%%%%%%%%%%%%%%%%%%%%%
\paragraph{Content of Main File.}

It is advisable to place all content in the child files included by |\include|.
Any output contained in the main file will appear in all child documents
unless suppressed manually;
it cannot be suppressed automatically by the |\includeonly| directive
and thus should normally be avoided.
A method to include some content in the main file
by means of conditional processing is described in \secref{sec:conditional}.

%%%%%%%%%%%%%%%%%%%%%%%%%%%%%%%%%%%%%%%%
\paragraph{Page Numbering.}

When only a part of the document is compiled,
the appropriate numbering of pages
(as well as other status parameters)
is determined from the |.aux| files.
The latter contain information from previous passes.
However this information needs to propagate through
all intermediate child documents.
Therefore the page numbering in child documents may well
be inconsistent until the complete document is compiled at least once.

A useful (if unconventional) way to always ensure a consistent
page numbering is to restart the numbering in each child document
and denote the pages by `\textit{child}|.|\textit{page}'
where \textit{child} represents the chapter/section number of the child file.
This can be achieved by the command
|\numberwithin{page}{|\textit{child}|}|
of the \textsf{amsmath} package
where \textit{child} can be |chapter| or |section|
depending on the chosen structuring.
Alternatively, one can modify the macro |\thepage| appropriately
and reset the counter |page| at the start of each child file.

%%%%%%%%%%%%%%%%%%%%%%%%%%%%%%%%%%%%%%%%%%%%%%%%%%%%%%%%%%%%%%%%%%%%%%%%%%%%%%%%
\subsection{Conditional Processing}
\label{sec:conditional}

The package provides a mechanism to compile different versions
of a document. To customise the versions further some conditional processing
can come in handy to distinguish which version is being compiled.
The package provides two macros to describe the compilation context:

%%%%%%%%%%%%%%%%%%%%%%%%%%%%%%%%%%%%%%%%
\DescribeMacro{\ifchilddoc}
The conditional |\ifchilddoc| distinguishes between the compilation of
child documents and the main document:
%
\begin{center}
|\ifchilddoc |\textit{child-code}| |[|\||else |\textit{main-code}]| \||fi|
\end{center}

%%%%%%%%%%%%%%%%%%%%%%%%%%%%%%%%%%%%%%%%
\DescribeMacro{\childdocname}
\DescribeMacro{\childdocjob}
The macro |\childdocname| contains the filename (without extension)
of the main or child file being processed.
Note that |\childdocjob| will always contain the name of the main file.

%%%%%%%%%%%%%%%%%%%%%%%%%%%%%%%%%%%%%%%%
\paragraph{Title Page.}

Conditional processing can be used to include a title or banner page
in the main document when proper precautions are taken.
Importantly, the code in the main file should ensure that the page counter
(as well as other status parameters which are stored in the |.aux| files)
takes the same value after the conditional processing.
Otherwise the page numbers may take divergent values
depending on which part is compiled.

For example, a title page could be declared by:
%
\begin{center}
\begin{tabular}{l}
|\ifchilddoc\||else|\\
|\addtocounter{page}{-1}|\\
\textit{code for title page}\\
|\newpage|\\
|\||fi|
\end{tabular}
\end{center}
%
A banner page for the child documents can be generated by:
%
\begin{center}
\begin{tabular}{l}
|\ifchilddoc|\\
|\addtocounter{page}{-1}|\\
\textit{code for banner page}\\
|\newpage|\\
|\||fi|
\end{tabular}
\end{center}
%
Here one could write a message such as:
\begin{center}
|This is the part \childdocname{} of \childdocjob{}.|
\end{center}

%%%%%%%%%%%%%%%%%%%%%%%%%%%%%%%%%%%%%%%%%%%%%%%%%%%%%%%%%%%%%%%%%%%%%%%%%%%%%%%%
\subsection{Flags}
\label{sec:flags}

The package makes it easy to generate different versions
of the main or child documents.
To this end compilation flags can be defined
and assigned different default values.
They will be particularly useful in conjunction
with the forwarding mechanism described in \secref{sec:forward}.

For example, it may be useful to have a flag |\version|
which can be set to |draft| or |final|.
The document source will contain some conditional code
depending on the value of |\version|.
Suppose further, the flag should default to |final| for the main file
and to |draft| for child files
which is a natural assignment for editing the document.
This is achieved by placing the following code
in the preamble of the main document
(below the |\childdocmain| directive):
%
\begin{center}
\begin{tabular}{l}
|\ifchilddoc|\\
|\providecommand{\version}{draft}|\\
|\||else|\\
|\providecommand{\version}{final}|\\
|\||fi|
\end{tabular}
\end{center}
%
The definition by |\providecommand| makes sure
that previous definitions are not overwritten.
Further statements |\providecommand{\version}{...}|
can thus be added before the above code to override it.

For the main file, one might add a line
(between |\childdocmain| and the above block)
%
\begin{center}
|%\ifchilddoc\||else\providecommand{\version}{draft}\||fi|
\end{center}
%
which can be uncommented to produce a draft version.
Likewise one can add a line to the very top of a child file
(above the |\childdocof{|\textit{main}|}| directive)
%
\begin{center}
|%\providecommand{\version}{final}|
\end{center}
%
which can be uncommented to produce the final version of this child document.

%%%%%%%%%%%%%%%%%%%%%%%%%%%%%%%%%%%%%%%%%%%%%%%%%%%%%%%%%%%%%%%%%%%%%%%%%%%%%%%%
\subsection{Forwarding}
\label{sec:forward}

Different versions of the main or child documents
using compilation flags as described in \secref{sec:flags}
can be (permanently) stored in different files
for convenient compilation, viewing and distribution.
To this end, the package defines a command
to pass on compilation to a different file:

%%%%%%%%%%%%%%%%%%%%%%%%%%%%%%%%%%%%%%%%
\DescribeMacro{\childdocforward}
The command |\childdocforward| redirects processing to
another source file:
%
\begin{center}
\begin{tabular}{l}
|\input{childdoc.def}|\\
|\childdocforward[|\textit{main}|]{|\textit{dest}|}|\\
\end{tabular}
\end{center}
%
The argument \textit{dest} is the destination file
(without extension).
It should be the main file or one of the child files.
Note that further \textsf{childdoc} directives
such as |\childdocof| and |\childdocforward|
in the indicated file will be processed in this form.
The optional argument \textit{main}
passes on directly to the main file \textit{main}
while pretending to compile the child \textit{dest}.
This form behaves as if \textit{dest}
issues |\childdocof{|\textit{main}|}| right away,
and no further \textsf{childdoc} directives will be processed.

%%%%%%%%%%%%%%%%%%%%%%%%%%%%%%%%%%%%%%%%
\DescribeMacro{\...prefix}
In the alternative form |\childdocforwardprefix|,
%
\begin{center}
\begin{tabular}{l}
|\input{childdoc.def}|\\
|\childdocforwardprefix[|\textit{main}|]{|\textit{prefix}|}{|\textit{dest}|}|
\end{tabular}
\end{center}
%
the destination file is determined by a pattern
depending on the current file:
To make this work, the current file must be called
`{\textit{prefix}\hspace{0.2em}\textit{suffix}}'
with \textit{prefix} matching precisely the argument.
Processing is then passed on to the file
`{\textit{dest}\hspace{0.2em}\textit{suffix}}'.
Surely, the same effect is achieved by
directly specifying the
argument `{\textit{dest}\hspace{0.2em}\textit{suffix}}'
in the first form.
However, that requires to set up a different file
for each child. With the alternative form of the command
all these files can have exactly the same content
which simplifies setting them up and maintaining them.

For example, the following file |draft.tex|
with a compilation flag |\version| as described in \secref{sec:flags}
compiles the main document as a draft:
%
\begin{center}
\begin{tabular}{l}
|\def\version{draft}|\\
|\input{childdoc.def}|\\
|\childdocforward{|\textit{main}|}|
\end{tabular}
\end{center}
%
Likewise, the following files |final|\textit{nn}|.tex|
compile the final version of the child document
|child|\textit{nn}|.tex|:
%
\begin{center}
\begin{tabular}{l}
|\def\version{final}|\\
|\input{childdoc.def}|\\
|\childdocforwardprefix{final}{child}|
\end{tabular}
\end{center}
%

Note that when several versions of a main file and/or of each child file
are to be generated, it may be convenient to set up a |Makefile| or
shell script to automatise the process.

%%%%%%%%%%%%%%%%%%%%%%%%%%%%%%%%%%%%%%%%%%%%%%%%%%%%%%%%%%%%%%%%%%%%%%%%%%%%%%%%
\subsection{Command Line Processing}
\label{sec:commandline}

The effect of redirection files can also be achieved by invoking
the \LaTeX{} compiler with a more elaborate command line.
Most conveniently this should be done as part
of a shell script or a |Makefile|.

When using \textsf{childdoc} in the main file, the following
command lines effectively perform a redirection
(note that depending on the shell being used,
backslashes may have to be doubled: `|\|' $\to$ `|\\|'):
%
\begin{center}
|... -jobname "|\textit{target}|" |\\|"|[\textit{flags}]%
|\input{childdoc.def}\childdocforward[|\textit{main}|]{|\textit{dest}|}"|
\end{center}
%
Here \textit{target} is the name of the output file,
\textit{main} is the name of the main file
and \textit{dest} is the name of the main or child file to be processed
(all filenames without extensions).
The optional argument \textit{main} can be omitted
if \textit{main} matches \textit{dest}.
Optionally, compilation \textit{flags} can be defined via |\def| commands.
This command line makes the \TeX{} engine believe
it is compiling the file \textit{target}
whose content is specified as the latter parameter.
The provided code then forwards the processing to
\textit{main} or \textit{dest} as described in \secref{sec:forward}.

%%%%%%%%%%%%%%%%%%%%%%%%%%%%%%%%%%%%%%%%%%%%%%%%%%%%%%%%%%%%%%%%%%%%%%%%%%%%%%%%
\subsection{Include by Input}
\label{sec:input}

Including child documents by |\include| has some restrictions by design.
Most notably, the content of a child document always occupies
its own set of pages; pages cannot be shared between child documents.
Usually, this behaviour makes perfect sense
because each child document contain an essential part of the document.
However, in some situations it may be desirable to compose
a document from a collection of parts
without having mandatory page breaks between then.
For this case, the package
provides a mechanism to include parts
by |\input| which can also be processed individually.
However, by construction this mechanism
requires manual handling of the content to be output.

%%%%%%%%%%%%%%%%%%%%%%%%%%%%%%%%%%%%%%%%
\DescribeMacro{\ifchilddocmanual}
The main file should be prepared as usual, see \secref{sec:include}.
However, the document body must make a distinction
between processing of an individual part and of the main document, e.g.:
%
\begin{center}
\begin{tabular}{l}
|\ifchilddocmanual|\\
|\input{\childdocname}|\\
|\||else|\\
\textit{document body with }|\input{|\textit{part}|}|\\
|\||fi|
\end{tabular}
\end{center}
%
The conditional |\ifchilddocmanual| is true whenever
a part to be included by |\input| is being compiled,
and the name of the part is stored in |\childdocname|.

%%%%%%%%%%%%%%%%%%%%%%%%%%%%%%%%%%%%%%%%
\DescribeMacro{\childdocby}
Each part to be included by |\input| should start with:
%
\begin{center}
\begin{tabular}{l}
|\input{childdoc.def}|\\
|\childdocby{|\textit{main}|}|\\
\end{tabular}
\end{center}
%
The directive |\childdocby| is similar to |\childdocof|
described in \secref{sec:include},
but the subsequent selection of content must be done manually.
To that end, both |\ifchilddoc| and |\ifchilddocmanual|
will be true upon processing of a part,
and the name of the part is stored in |\childdocname|.
Note that |\jobname| will be set to the filename of the current part
so that each part receives an individual |.aux| file
that does not interfere with the |.aux| file(s) of the main document.
This behaviour can be altered by the alternative form
|\childdocby[*]{|\textit{main}|}| (with a non-empty optional argument)
which uses the |.aux| file of the main document
by setting |\jobname| to \textit{main}.

%%%%%%%%%%%%%%%%%%%%%%%%%%%%%%%%%%%%%%%%%%%%%%%%%%%%%%%%%%%%%%%%%%%%%%%%%%%%%%%%
\subsection{Driver Development}
\label{sec:driver}

The \textsf{childdoc} mechanism can also be use for the development
of definition files such as \LaTeX{} styles or classes.
This case differs from the above setup with multiple parts
included by |\include| in that no |\includeonly| should be invoked.
This can be achieved by starting the include file
(before |\ProvidesPackage|) with:
%
\begin{center}
\begin{tabular}{l}
|\input{childdoc.def}|\\
|\childdocforward{|\textit{main}|}|\\
\end{tabular}
\end{center}
%
or alternatively with:
%
\begin{center}
\begin{tabular}{l}
|\input{childdoc.def}|\\
|\childdocby{|\textit{main}|}|\\
\end{tabular}
\end{center}
%
Both forms have slightly different effects as described above.
The main file is prepared as usual, see \secref{sec:include}.

%%%%%%%%%%%%%%%%%%%%%%%%%%%%%%%%%%%%%%%%%%%%%%%%%%%%%%%%%%%%%%%%%%%%%%%%%%%%%%%%
\subsection{Legacy Detection}
\label{sec:detection}

The directive |\childdocmain| in the main file can detect
whether the complete document or merely a child is to be compiled
even without using the directive |\childdocof|.
This method is deprecated because it is less robust
and there is no compelling reason to use it;
it is merely provided for backward compatibility
and it may be removed in future versions.

If the detection mechanism is to be used,
it is mandatory to correctly specify
the filename of the main file as the argument of |\childdocmain|:
%
\begin{center}
\begin{tabular}{l}
|\input{childdoc.def}|\\
|\childdocmain{|\textit{main}|}|\\
\end{tabular}
\end{center}
%
If |\jobname| does not match the argument \textit{main} of |\childdocmain|,
it is assumed that |\jobname| points to the child file to be compiled.
When using |\childdocmain| with the main file specified as argument,
it suffices to start a child file
with just |\input{|\textit{main}|}|
without loading of the package and using |\childdocof|.
If instead all processing is done
with the appropriate \textsf{childdoc} directives,
the argument of \textit{main} of |\childdocmain| can be empty.

An alternative version of the command line processing described
in \secref{sec:commandline} using the detection mechanism reads:
%
\begin{center}
|... -jobname "|\textit{target}|" "|[\textit{flags}]%
[|\def\jobname{|\textit{dest}|}|]|\input{|\textit{main}|}"|
\end{center}

%%%%%%%%%%%%%%%%%%%%%%%%%%%%%%%%%%%%%%%%%%%%%%%%%%%%%%%%%%%%%%%%%%%%%%%%%%%%%%%%
\subsection{Manual Code}
\label{sec:manual}

In case one cannot be certain whether the definitions file |childdoc.def|
is installed on the target \TeX{} distribution
and one prefers not to ship it,
it is conceivable to paste a few relevant commands into the sources.

To that end, drop all statements |\input{childdoc.def}|
and perform the replacements as outlined below.
Instead of |\childdocmain{|\textit{main}|}| add the following code
to the top of the main file:
%
\begin{center}
\begin{tabular}{l}
|\||ifdefined\childdocname\endinput\||fi\newif\ifchilddoc|\\
|\edef\childdocname{\scantokens\expandafter{\jobname\noexpand}}|\\
|\def\childdocmain{|\textit{main}|}\||ifx\childdocmain\childdocname\||else|\\
|\childdoctrue\includeonly{\childdocname}\let\jobname\childdocmain\||fi|\\
\end{tabular}
\end{center}
%
Instead of |\childdocof{|\textit{main}|}| just include the main file
at the top of each child file:
%
\begin{center}
|\input{|\textit{main}|}|
\end{center}
%
A simple redirection |\childdocforward{|\textit{dest}|}| is achieved by:
%
\begin{center}
|\def\jobname{|\textit{dest}|}\input{\jobname}|
\end{center}
%
The redirection with prefix
|\childdocforwardprefix[|\textit{prefix}|]{|\textit{dest}|}|
is accomplished by:
%
\begin{center}
\begin{tabular}{l}
|{\edef\jobname{\scantokens\expandafter{\jobname\noexpand}}|\\
|\def\redirectjob |\textit{prefix}|#1~~~{\gdef\jobname{|\textit{dest}|#1}}|\\
|\expandafter\redirectjob\jobname~~~}\input{\jobname}|
\end{tabular}
\end{center}

In an alternative approach,
child documents can be compiled by a specific command line
without additional code or specific definitions:
%
\begin{center}
|... -jobname "|\textit{target}|" "|[\textit{flags}]%
|\includeonly{|\textit{dest}|}\input{|\textit{main}|}"|
\end{center}
%

%%%%%%%%%%%%%%%%%%%%%%%%%%%%%%%%%%%%%%%%%%%%%%%%%%%%%%%%%%%%%%%%%%%%%%%%%%%%%%%%
%%%%%%%%%%%%%%%%%%%%%%%%%%%%%%%%%%%%%%%%%%%%%%%%%%%%%%%%%%%%%%%%%%%%%%%%%%%%%%%%
\section{Information}

%%%%%%%%%%%%%%%%%%%%%%%%%%%%%%%%%%%%%%%%%%%%%%%%%%%%%%%%%%%%%%%%%%%%%%%%%%%%%%%%
\subsection{Copyright}

Copyright \copyright{} 2017--2018 Niklas Beisert

This work may be distributed and/or modified under the
conditions of the \LaTeX{} Project Public License, either version 1.3
of this license or (at your option) any later version.
The latest version of this license is in
  \url{http://www.latex-project.org/lppl.txt}
and version 1.3 or later is part of all distributions of \LaTeX{}
version 2005/12/01 or later.

This work has the LPPL maintenance status `maintained'.

The Current Maintainer of this work is Niklas Beisert.

This work consists of the files |README.txt|, |childdoc.ins| and |childdoc.dtx|
as well as the derived files |childdoc.def|, |cdocsamp.tex|
with |cdocsch1.tex|, |cdocsch2.tex|, |cdocspt3.tex|, |cdocspt4.tex|,
|cdocsdrf.tex|, |cdocsfn1.tex|, |cdocsfn2.tex|
as well as |childdoc.pdf|.

%%%%%%%%%%%%%%%%%%%%%%%%%%%%%%%%%%%%%%%%%%%%%%%%%%%%%%%%%%%%%%%%%%%%%%%%%%%%%%%%
\subsection{Files and Installation}

The package consists of the files:
%
\begin{center}
\begin{tabular}{ll}
    |README.txt|   & readme file \\
    |childdoc.ins| & installation file \\
    |childdoc.dtx| & source file \\
    |childdoc.def| & definition file \\
    |cdocsamp.tex| & sample main file \\
    |cdocsch1.tex| & sample include file \\
    |cdocsch2.tex| & sample include file \\
    |cdocspt3.tex| & sample part file \\
    |cdocspt4.tex| & sample part file \\
    |cdocsdrf.tex| & sample redirection file \\
    |cdocsfn1.tex| & sample redirection file \\
    |cdocsfn2.tex| & sample redirection file \\
    |childdoc.pdf| & manual
\end{tabular}
\end{center}
%
The distribution consists of the files
|README.txt|, |childdoc.ins| and |childdoc.dtx|.
%
\begin{itemize}
\item
Run (pdf)\LaTeX{} on |childdoc.dtx|
to compile the manual |childdoc.pdf| (this file).
\item
Run \LaTeX{} on |childdoc.ins| to create the definitions file |childdoc.def|
and the sample |cdocsamp.tex| with include files
|cdocsch1.tex|, |cdocsch2.tex|, |cdocspt3.tex|, |cdocspt4.tex|,
|cdocsdrf.tex|, |cdocsfn1.tex|, |cdocsfn2.tex|.
Then copy the file |childdoc.def| to an appropriate directory of your \LaTeX{}
distribution, e.g.\ \textit{texmf-root}|/tex/latex/childdoc|.
\end{itemize}

%%%%%%%%%%%%%%%%%%%%%%%%%%%%%%%%%%%%%%%%%%%%%%%%%%%%%%%%%%%%%%%%%%%%%%%%%%%%%%%%
\subsection{Related CTAN Packages}

There are several other packages which offer a similar functionality:
%
\begin{itemize}
\item
The packages
\href{http://ctan.org/pkg/docmute}{\textsf{docmute}},
\href{http://ctan.org/pkg/includex}{\textsf{includex}} and
\href{http://ctan.org/pkg/standalone}{\textsf{standalone}}
provide commands to include only the document body of
a child file thus allowing both files to be compiled individually.
\item
The packages \href{http://ctan.org/pkg/subdocs}{\textsf{subdocs}}
and \href{http://ctan.org/pkg/subfiles}{\textsf{subfiles}}
provide structures in which the main and child documents can be
encapsulated and allowing them to be compiled individually.
The inclusion mechanism is different from the conventional |\include|.
\item
The package \href{http://ctan.org/pkg/combine}{\textsf{combine}}
is an elaborate solution to combine several documents into one.
\end{itemize}
%
See also the CTAN topic \href{http://ctan.org/topic/subdocs}{\textsf{subdocs}}
for further related packages.
The present package differs from the above solutions in that
a document structure constructed with the conventional |\include| mechanism
just needs two extra commands at the top of every file
such that all constituent files can be compiled individually.

%%%%%%%%%%%%%%%%%%%%%%%%%%%%%%%%%%%%%%%%%%%%%%%%%%%%%%%%%%%%%%%%%%%%%%%%%%%%%%%%
%\subsection{Feature Suggestions}
%
%The following is a list of features which may be useful for future
%versions of this package:
%%
%\begin{itemize}
%\item
%\ldots
%\end{itemize}

%%%%%%%%%%%%%%%%%%%%%%%%%%%%%%%%%%%%%%%%%%%%%%%%%%%%%%%%%%%%%%%%%%%%%%%%%%%%%%%%
\subsection{Revision History}

%%%%%%%%%%%%%%%%%%%%%%%%%%%%%%%%%%%%%%%%
\paragraph{v2.0:} 2018/12/30

\begin{itemize}
\item
immediate forward processing
\item
added |\childdocby| mechanism
\item
manual restructured
\end{itemize}

%%%%%%%%%%%%%%%%%%%%%%%%%%%%%%%%%%%%%%%%
\paragraph{v1.6:} 2018/01/17

\begin{itemize}
\item
application for development of include files
\item
corrections to manual
\end{itemize}

%%%%%%%%%%%%%%%%%%%%%%%%%%%%%%%%%%%%%%%%
\paragraph{v1.5:} 2017/05/21

\begin{itemize}
\item
more complete structuring introduced
\item
|\childdocof| introduced
\item
|\childdoc| renamed to |\childdocmain|
\item
|\childredirect| renamed to |\childdocforward| and |\childdocforwardprefix|
and functionality expanded
\end{itemize}

%%%%%%%%%%%%%%%%%%%%%%%%%%%%%%%%%%%%%%%%
\paragraph{v1.0:} 2017/04/27

\begin{itemize}
\item
manual and install package
\item
first version published on CTAN
\end{itemize}

%%%%%%%%%%%%%%%%%%%%%%%%%%%%%%%%%%%%%%%%
\paragraph{v0.6:} 2017/04/26

\begin{itemize}
\item
redirection mechanism added
\end{itemize}

%%%%%%%%%%%%%%%%%%%%%%%%%%%%%%%%%%%%%%%%
\paragraph{v0.5:} 2017/04/26

\begin{itemize}
\item
functionality in definition file
\end{itemize}


%%%%%%%%%%%%%%%%%%%%%%%%%%%%%%%%%%%%%%%%%%%%%%%%%%%%%%%%%%%%%%%%%%%%%%%%%%%%%%%%
%%%%%%%%%%%%%%%%%%%%%%%%%%%%%%%%%%%%%%%%%%%%%%%%%%%%%%%%%%%%%%%%%%%%%%%%%%%%%%%%
%%%%%%%%%%%%%%%%%%%%%%%%%%%%%%%%%%%%%%%%%%%%%%%%%%%%%%%%%%%%%%%%%%%%%%%%%%%%%%%%
\appendix

\settowidth\MacroIndent{\rmfamily\scriptsize 000\ }

 \DocInput{childdoc.dtx}

\end{document}
%</driver>
% \fi
%
% %%%%%%%%%%%%%%%%%%%%%%%%%%%%%%%%%%%%%%%%%%%%%%%%%%%%%%%%%%%%%%%%%%%%%%%%%%%%%%
% %%%%%%%%%%%%%%%%%%%%%%%%%%%%%%%%%%%%%%%%%%%%%%%%%%%%%%%%%%%%%%%%%%%%%%%%%%%%%%
% \section{Sample}
%\iffalse
%<*samplemain>
%\fi
%
% The following presents a sample document
% with two chapters, two parts, a title page,
% a compile flag as well as three forwarding files to set the flag.
% It consists of eight |.tex| files:
% \begin{center}
% \begin{tabular}{ll}
% |cdocsamp.tex|&main file\\
% |cdocsch1.tex|&include file for chapter 1\\
% |cdocsch2.tex|&include file for chapter 2\\
% |cdocspt3.tex|&include file for part 3\\
% |cdocspt4.tex|&include file for part 4\\
% |cdocsdrf.tex|&forwarding file for main file in draft mode\\
% |cdocsfi1.tex|&forwarding file for final version of chapter 1\\
% |cdocsfi2.tex|&forwarding file for final version of chapter 2\\
% \end{tabular}
% \end{center}
% Each of the eight files can be compiled directly by the \LaTeX{} compiler.
%
% %%%%%%%%%%%%%%%%%%%%%%%%%%%%%%%%%%%%%%
% \paragraph{Main File.}
%
% The main file is called |cdocsamp.tex|.
%
% Load the \textsf{childdoc} definitions and
% declare the filename for the main document:
%    \begin{macrocode}
\input{childdoc.def}
\childdocmain{}
%    \end{macrocode}

% Optional override for |\version| flag:
%    \begin{macrocode}
%%\ifchilddoc\else\providecommand{\version}{draft}\fi
%    \end{macrocode}

% Define the default values for the |\version| flag
% (|final| for the main file and |draft| for childs):
%    \begin{macrocode}
\ifchilddoc
\providecommand{\version}{draft}
\else
\providecommand{\version}{final}
\fi
%    \end{macrocode}

% Load the standard document class:
%    \begin{macrocode}
\documentclass[12pt]{article}
%    \end{macrocode}

% Start the document body:
%    \begin{macrocode}
\begin{document}
%    \end{macrocode}

% Declare a title page.
% Print title, part of document being processed and version flag:
%    \begin{macrocode}
\addtocounter{page}{-1}
\begin{center}
{\LARGE\bfseries{}childdoc example\par}
\vspace{1cm}
\ifchilddoc
\ifchilddocmanual part\else chapter\fi:
`\childdocname' of `\childdocjob'\par
\else
main document: `\childdocjob'\par
\fi
version: \version\par
\end{center}
\newpage
%    \end{macrocode}

% Manually include selected file,
% otherwise process as usual:
%    \begin{macrocode}
\ifchilddocmanual
\section*{part `\childdocname'}
\input{\childdocname}
\else
%    \end{macrocode}

% Include the two chapters:
%    \begin{macrocode}
\include{cdocsch1}
\include{cdocsch2}
%    \end{macrocode}

% Include the two parts unless only chapters should be displayed:
%    \begin{macrocode}
\ifchilddoc\else
\section{part three}
\input{cdocspt3}
\section{part four}
\input{cdocspt4}
\fi
%    \end{macrocode}

% Process as usual until here:
%    \begin{macrocode}
\fi
%    \end{macrocode}

% End of document body:
%    \begin{macrocode}
\end{document}
%    \end{macrocode}
%\iffalse
%</samplemain>
%\fi
%
% %%%%%%%%%%%%%%%%%%%%%%%%%%%%%%%%%%%%%%
% \paragraph{Chapter Include Files.}
%
% The include files are called |cdocsch1.tex| and |cdocsch2.tex|.
%
%\iffalse
%<*samplechap1|samplechap2>
%\fi

% Optional override for |\version| flag:
%    \begin{macrocode}
%%\providecommand{\version}{final}
%    \end{macrocode}

% Include the main document:
%    \begin{macrocode}
\input{childdoc.def}
\childdocof{cdocsamp}
%    \end{macrocode}

%\iffalse
%</samplechap1|samplechap2>
%\fi
%
%\iffalse
%<*samplechap1>
%\fi
% Some text for chapter 1:
%    \begin{macrocode}
\section{one}
some text in chapter one
%    \end{macrocode}

%\iffalse
%</samplechap1>
%\fi
% Some text for chapter 2:
%\iffalse
%<*samplechap2>
%\fi
%    \begin{macrocode}
\section{two}
more text in chapter two
%    \end{macrocode}

%\iffalse
%</samplechap2>
%\fi
%
% %%%%%%%%%%%%%%%%%%%%%%%%%%%%%%%%%%%%%%
% \paragraph{Part Include Files.}
%
% The include files are called |cdocspt3.tex| and |cdocspt4.tex|.
%
%\iffalse
%<*samplepart3|samplepart4>
%\fi

% Optional override for |\version| flag:
%    \begin{macrocode}
%%\providecommand{\version}{final}
%    \end{macrocode}

% Include the main document:
%    \begin{macrocode}
\input{childdoc.def}
\childdocby{cdocsamp}
%    \end{macrocode}

%\iffalse
%</samplepart3|samplepart4>
%\fi
%
%\iffalse
%<*samplepart3>
%\fi
% Some text for part 3:
%    \begin{macrocode}
some text in part three
%    \end{macrocode}

%\iffalse
%</samplepart3>
%\fi
% Some text for part 4:
%\iffalse
%<*samplepart4>
%\fi
%    \begin{macrocode}
more text in part four
%    \end{macrocode}

%\iffalse
%</samplepart4>
%\fi
%
% %%%%%%%%%%%%%%%%%%%%%%%%%%%%%%%%%%%%%%
% \paragraph{Forwarding for a Complete Draft.}
%
% The following forwarding file |cdocsdrf.tex|
% compiles the main document in draft mode:
%\iffalse
%<*sampledraft>
%\fi
%    \begin{macrocode}
\def\version{draft}
\input{childdoc.def}
\childdocforward{cdocsamp}
%    \end{macrocode}

%\iffalse
%</sampledraft>
%\fi
%
% %%%%%%%%%%%%%%%%%%%%%%%%%%%%%%%%%%%%%%
% \paragraph{Forwarding for Final Version of the Chapters.}
%
% The following forwarding files |cdocsfn1.tex| and |cdocsfn2.tex|
% (with identical content)
% compile the final versions of the child documents
% |cdocsch1.tex| and |cdocsch2.tex|, respectively:
%\iffalse
%<*samplefinal>
%\fi
%    \begin{macrocode}
\def\version{final}
\input{childdoc.def}
\childdocforwardprefix[cdocsamp]{cdocsfn}{cdocsch}
%    \end{macrocode}

%\iffalse
%</samplefinal>
%\fi
%
% %%%%%%%%%%%%%%%%%%%%%%%%%%%%%%%%%%%%%%
% \paragraph{Command Line Processing.}
%
% The following three command lines generate the output files
% |cdocscld|, |cdocscl1| and |cdocscl2|
% which should be identical to
% |cdocsdrf|, |cdocsch1| and |cdocsfn2|, respectively:
% \begin{center}
% \begin{tabular}{l}
% |latex -jobname cdocscld \|\\
% |  "\def\version{draft}\input{childdoc.def}\childdocforward{cdocsamp}"|\\
% |latex -jobname cdocscl1 \|\\
% |  "\input{childdoc.def}\childdocforward[cdocsamp]{cdocsch1}"|\\
% |latex -jobname cdocscl2 \|\\
% |  "\def\version{final}\input{childdoc.def}\childdocforward{cdocsch2}"|
% \end{tabular}
% \end{center}
% Note that the trailing backslash on each first line
% merely continues the input to the second line
% (for convenient cut ant paste).
% Furthermore, the command |latex| can be replaced by any
% of its alternative versions such as |pdflatex|.
%
% %%%%%%%%%%%%%%%%%%%%%%%%%%%%%%%%%%%%%%%%%%%%%%%%%%%%%%%%%%%%%%%%%%%%%%%%%%%%%%
% %%%%%%%%%%%%%%%%%%%%%%%%%%%%%%%%%%%%%%%%%%%%%%%%%%%%%%%%%%%%%%%%%%%%%%%%%%%%%%
% \section{Implementation}
%\iffalse
%<*package>
%\fi
%
% This section describes the definitions file |childdoc.def|.

% The definitions cannot be loaded using |\usepackage| or |\RequirePackage|
% which has a mechanism to prevent loading a style file more than once.
% When loading the definitions by means of |\input|
% multiple instances have to be prevented manually:
%\iffalse
%This code needs to be before the `\ProvidesFile' directive
%which is defined at the beginning of this file.
%Therefore it is also placed there and commented out here.
%</package>
%<*discard>
%\fi
%    \begin{macrocode}
\ifdefined\childdocmain\endinput\fi
%    \end{macrocode}
%\iffalse
%</discard>
%<*package>
%\fi
%
% \macro{\ifchilddoc}
% \macro{\ifchilddocmanual}
% The conditional |\ifchilddoc| tells whether a
% child (true) or main (false) document is being compiled.
% The conditional |\ifchilddocmanual| tells whether
% the |\includeonly| mechanism is used (false) or
% the selection of child files must be performed manually (true).
% The definitions initialise to false:
%    \begin{macrocode}
\newif\ifchilddoc
\newif\ifchilddocmanual
%    \end{macrocode}

% \macro{\childdocname}
% \macro{\childdocjob}
% The macro |\childdocname| stores the name of the main document
% to be compiled. The macro |\childdocjob| stores the name of
% the document on which the \LaTeX{} compiler was originally invoked.
% The content of |\jobname| cannot be compared
% to filenames specified in the source due to different catcodes.
% The following code rescans |\jobname|, stores the result
% in |\childdocname| and saves a copy in |\childdocjob|:
%    \begin{macrocode}
\edef\childdocname{\scantokens\expandafter{\jobname\noexpand}}
\let\childdocjob\childdocname
%    \end{macrocode}

% \macro{\childdocdisable}
% The macro |\childdocdisable| prevents the main file
% from being processed more than once.
% At this stage, the main document command |\childdocmain|
% is assumed to be called once again where it should do nothing.
% Any subsequent call to it should prevent
% a secondary processing of the main document
% It overwrites the forwarding commands
% |\childdocof| and |\childdocforward|
% with empty macros to prevent further inclusions of the main document:
%    \begin{macrocode}
\newcommand{\childdocdisable}
{
  \renewcommand{\childdocmain}[1]{\renewcommand{\childdocmain}[1]{\endinput}}
  \renewcommand{\childdocof}[1]{}
  \renewcommand{\childdocby}[2][]{}
  \renewcommand{\childdocforward}[2][]{}
  \renewcommand{\childdocdisable}{}
}
%    \end{macrocode}

% \macro{\childdocmain}
% The macro |\childdocmain| is to be called at the top of the main file
% with nothing or the main filename (without extension) as argument.
% First, it breaks loops.
% If the argument is not empty and does not match |\childdocname|
% (which is set by the first inclusion of |childdoc.def|),
% |\ifchilddoc| is set to true, |\includeonly| is applied to the child file
% and |\jobname| is set to the main file
% (for proper handling of |.aux| files):
%    \begin{macrocode}
\newcommand{\childdocmain}[1]
{
  \childdocdisable\childdocmain{}
  \if?#1?\else
    \begingroup
      \def\childdoctmp{#1}
      \ifx\childdoctmp\childdocname
        \def\childdoctmp{}
      \else
        \def\childdoctmp
        {
          \childdoctrue
          \includeonly{\childdocname}
          \def\childdocjob{#1}
          \def\jobname{#1}
        }
      \fi
      \expandafter
    \endgroup
    \childdoctmp
  \fi
}
%    \end{macrocode}

% \macro{\childdocof}
% The command |\childdocof| redirects
% compilation to the main file |#1|.
%    \begin{macrocode}
\newcommand{\childdocof}[1]
{
  \childdocdisable
  \childdoctrue
  \includeonly{\childdocname}
  \def\jobname{#1}
  \def\childdocjob{#1}
  \input{#1}
}
%    \end{macrocode}

% \macro{\childdocby}
% The command |\childdocby| ....
%    \begin{macrocode}
\newcommand{\childdocby}[2][]
{
  \childdocdisable
  \childdoctrue
  \childdocmanualtrue
  \if?#1?\else
    \def\jobname{#2}
  \fi
  \def\childdocjob{#2}
  \input{#2}
  \endinput
}
%    \end{macrocode}

% \macro{\childdocforward}
% The command |\childdocforward| redirects
% compilation to the main file or
% (if the optional argument is given) a child file.
% Parameters are set as if the main file
% or a child file starting with |\childdocof| was compiled.
% Then compilation is handed over to the main file:
%    \begin{macrocode}
\newcommand{\childdocforward}[2][]
{
  \begingroup
    \if?#1?
      \def\childdoctmp
      {
        \def\childdocname{#2}
        \def\childdocjob{#2}
        \def\jobname{#2}
        \input{#2}
        \endinput
      }
    \else
      \def\childdoctmp
      {
        \childdocdisable
        \def\childdocname{#2}
        \childdoctrue
        \includeonly{#2}
        \def\childdocjob{#1}
        \def\jobname{#1}
        \input{#1}
        \endinput
      }
    \fi
    \expandafter
  \endgroup
  \childdoctmp
}
%    \end{macrocode}

% \macro{\childdocforwardprefix}
% The command |\childdocforwardprefix| redirects
% compilation to the main or a child file by means of a pattern.
% The prefix |#1| in the current filename is replaced by |#2|
% and the suffix of the current filename is kept
% (it is assumed that the filename does not contain the substring `|~~~|'
% which is used as a delimiter).
% Compilation is handed over to the new file by |\childdocforward|:
%    \begin{macrocode}
\newcommand{\childdocforwardprefix}[3][]
{
  \begingroup
    \def\childdocextract #2##1~~~{\def\childdoctmp{\childdocforward[#1]{#3##1}}}
    \expandafter\childdocextract\childdocname~~~
    \expandafter
  \endgroup
  \childdoctmp
}
%    \end{macrocode}

% \macro{\childdoc}
% The deprecated macro |\childdoc| is a legacy version of |\childdocmain|:
%    \begin{macrocode}
\newcommand{\childdoc}{\childdocmain}
%    \end{macrocode}

% \macro{\childdocredirect}
% The deprecated macro |\childdocredirect| is a legacy version
% of |\childdocforward| and |\childdocforwardprefix|:
%    \begin{macrocode}
\newcommand{\childdocredirect}[2][]
{
  \begingroup
    \if?#1?
      \def\childdoctmp{\childdocforward{#2}}
    \else
      \def\childdoctmp{\childdocforwardprefix{#1}{#2}}
    \fi
    \expandafter
  \endgroup
  \childdoctmp
}
%    \end{macrocode}

%\iffalse
%</package>
%\fi
%
\endinput
|\\
|\childdocforwardprefix{final}{child}|
\end{tabular}
\end{center}
%

Note that when several versions of a main file and/or of each child file
are to be generated, it may be convenient to set up a |Makefile| or
shell script to automatise the process.

%%%%%%%%%%%%%%%%%%%%%%%%%%%%%%%%%%%%%%%%%%%%%%%%%%%%%%%%%%%%%%%%%%%%%%%%%%%%%%%%
\subsection{Command Line Processing}
\label{sec:commandline}

The effect of redirection files can also be achieved by invoking
the \LaTeX{} compiler with a more elaborate command line.
Most conveniently this should be done as part
of a shell script or a |Makefile|.

When using \textsf{childdoc} in the main file, the following
command lines effectively perform a redirection
(note that depending on the shell being used,
backslashes may have to be doubled: `|\|' $\to$ `|\\|'):
%
\begin{center}
|... -jobname "|\textit{target}|" |\\|"|[\textit{flags}]%
|% \iffalse
%
% childdoc.dtx Copyright (C) 2017-2018 Niklas Beisert
%
% This work may be distributed and/or modified under the
% conditions of the LaTeX Project Public License, either version 1.3
% of this license or (at your option) any later version.
% The latest version of this license is in
%   http://www.latex-project.org/lppl.txt
% and version 1.3 or later is part of all distributions of LaTeX
% version 2005/12/01 or later.
%
% This work has the LPPL maintenance status `maintained'.
%
% The Current Maintainer of this work is Niklas Beisert.
%
% This work consists of the files childdoc.dtx and childdoc.ins
% and the derived files childdoc.def and cdocsamp.tex with
% cdocsch1.tex, cdocsch2.tex, cdocsdrf.tex, cdocsfn1.tex, cdocsfn2.tex.
%
%<package>\ifdefined\childdocmain\endinput\fi
%<package>\ProvidesFile{childdoc.def}[2018/12/30 v2.0 child document driver]
%<samplemain>\ProvidesFile{cdocsamp.tex}[2018/12/30 v2.0 sample for childdoc]
%<*driver>
%\ProvidesFile{childdoc.drv}[2018/12/30 v2.0 childdoc reference manual file]
\PassOptionsToClass{10pt,a4paper}{article}
\documentclass{ltxdoc}

\usepackage[margin=35mm]{geometry}
\usepackage{hyperref}
\usepackage{hyperxmp}
\usepackage[usenames]{color}

\hypersetup{colorlinks=true}
\hypersetup{pdfstartview=FitH}
\hypersetup{pdfpagemode=UseNone}
\hypersetup{pdfsource={}}
\hypersetup{pdflang={en-UK}}
\hypersetup{pdfcopyright={Copyright 2017-2018 Niklas Beisert.
  This work may be distributed and/or modified under the
  conditions of the LaTeX Project Public License, either version 1.3
  of this license or (at your option) any later version.}}
\hypersetup{pdflicenseurl={http://www.latex-project.org/lppl.txt}}
\hypersetup{pdfcontactaddress={ETH Zurich, ITP, HIT K,
  Wolfgang-Pauli-Strasse 27}}
\hypersetup{pdfcontactpostcode={8093}}
\hypersetup{pdfcontactcity={Zurich}}
\hypersetup{pdfcontactcountry={Switzerland}}
\hypersetup{pdfcontactemail={nbeisert@itp.phys.ethz.ch}}
\hypersetup{pdfcontacturl={http://people.phys.ethz.ch/\xmptilde nbeisert/}}

\newcommand{\secref}[1]{\hyperref[#1]{section \ref*{#1}}}

\parskip1ex
\parindent0pt
\let\olditemize\itemize
\def\itemize{\olditemize\parskip0pt}

\begin{document}

\title{The \textsf{childdoc} Package}
\hypersetup{pdftitle={The childdoc Package}}
\author{Niklas Beisert\\[2ex]
  Institut f\"ur Theoretische Physik\\
  Eidgen\"ossische Technische Hochschule Z\"urich\\
  Wolfgang-Pauli-Strasse 27, 8093 Z\"urich, Switzerland\\[1ex]
  \href{mailto:nbeisert@itp.phys.ethz.ch}
  {\texttt{nbeisert@itp.phys.ethz.ch}}}
\hypersetup{pdfauthor={Niklas Beisert}}
\hypersetup{pdfsubject={Manual for the LaTeX2e Package childdoc}}
\date{30 December 2018, \textsf{v2.0}}
\maketitle

\begin{abstract}\noindent
\textsf{childdoc} is a \LaTeXe{} package
that enables the direct compilation
of document sections included by |\include|
to individual files.
\end{abstract}

\begingroup
\parskip0ex
\tableofcontents
\endgroup

%%%%%%%%%%%%%%%%%%%%%%%%%%%%%%%%%%%%%%%%%%%%%%%%%%%%%%%%%%%%%%%%%%%%%%%%%%%%%%%%
%%%%%%%%%%%%%%%%%%%%%%%%%%%%%%%%%%%%%%%%%%%%%%%%%%%%%%%%%%%%%%%%%%%%%%%%%%%%%%%%
\section{Introduction}

\LaTeX{} provides a mechanism to structure a large document (such as a book)
into a main file and several child files (containing the chapters)
using the |\include| command.
This mechanism is beneficial for documents
which span hundreds of pages in order to
make the source file(s) more manageable.
Moreover, compilation can be restricted to
selected child files by means of the |\includeonly| command.
The latter feature can be used to reduce the compilation time while editing
(this was significantly more useful in the earlier days of \LaTeX{})
or to generate a smaller document which is easier to navigate.
Another application of |\includeonly| is to generate
documents consisting of selected parts of the complete document.

However, there are a few drawbacks of the plain |\include| mechanism:
\begin{itemize}
\item
The child files cannot be compiled on their own,
they can only be compiled via the main file.
A naive editing environment
(such as a text editor with an option
to have the current file processed by \LaTeX)
may require one to switch to the main file before compiling;
attempting to compile the child file produces errors.
\item
The main file must be modified (each time)
to adjust the |\includeonly| command
to the present needs. This easily leaves the main file in a messy state.
\item
The generated document will always carry the filename
of the main document. This is inconvenient if
several child files are to be compiled and
to be kept for distribution.
\end{itemize}

The present package provides a simple interface
to make child files individually compilable by \LaTeX{}.
Compiling a child file then has the same effect as compiling
the main file with an |\includeonly| command
to select the appropriate child.
Moreover the generated document will carry the name of the child
rather than the main file.
This resolves all three above issues.

This feature is meant to make the editing of books,
thesis documents and lecture notes somewhat more convenient.
However, the package can also be used efficiently for
composing a series of documents (such as exercise sheets)
which are typically distributed individually.
It then assists the author in generating the individual documents
(potentially in different versions)
as well as a document containing the collected series.
Another application is in developing style files
or other kinds of included material
where compilation of the style file could redirect
to a sample or test file.

%%%%%%%%%%%%%%%%%%%%%%%%%%%%%%%%%%%%%%%%%%%%%%%%%%%%%%%%%%%%%%%%%%%%%%%%%%%%%%%%
%%%%%%%%%%%%%%%%%%%%%%%%%%%%%%%%%%%%%%%%%%%%%%%%%%%%%%%%%%%%%%%%%%%%%%%%%%%%%%%%
\section{Usage}

First of all, the package \textsf{childdoc} is \emph{not} a standard
\LaTeXe{} |.sty| style file! Therefore it needs to be invoked in
a non-standard way.

%%%%%%%%%%%%%%%%%%%%%%%%%%%%%%%%%%%%%%%%%%%%%%%%%%%%%%%%%%%%%%%%%%%%%%%%%%%%%%%%
\subsection{Included Files}
\label{sec:include}

%%%%%%%%%%%%%%%%%%%%%%%%%%%%%%%%%%%%%%%%
\DescribeMacro{\childdocmain}
To use the package, add the commands
\begin{center}
\begin{tabular}{l}
|\input{childdoc.def}|\\
|\childdocmain{}|\\
\end{tabular}
\end{center}
at the very top of the main \LaTeX{} file,
in particular \emph{before} the |\documentclass| statement!
The argument of |\childdocmain| should be left empty
(but it must be present).

%%%%%%%%%%%%%%%%%%%%%%%%%%%%%%%%%%%%%%%%
\DescribeMacro{\childdocof}
Furthermore, add the commands
\begin{center}
\begin{tabular}{l}
|\input{childdoc.def}|\\
|\childdocof{|\textit{main}|}|\\
\end{tabular}
\end{center}
at the top of every child file \textit{child}
which is included by |\include{|\textit{child}|}|
from within the main file
(or at least for those files to be compiled individually).
The argument \textit{main} must be the filename of the main file.

There are a couple of
considerations in setting up the main and child documents:

%%%%%%%%%%%%%%%%%%%%%%%%%%%%%%%%%%%%%%%%
\paragraph{Restrictions.}

Please note the following restrictions:
\begin{itemize}
\item
|\childdocmain| must be called with one argument \textit{main}
to ensure compatibility with earlier version of the package.
It must either be empty (|\childdocmain{}|)
or precisely match the filename of the main file in which it is specified.
See \secref{sec:detection} for further information.
\item
The filename \textit{main} must be specified without the |.tex| extension.
\item
The filename \textit{main} is case sensitive
(even in case-insensitive file systems)
due to internal string comparison.
\item
The argument \textit{main} should be fully expanded, it cannot be a macro.
\item
Subdirectories and special characters should be avoided in filenames.
\item
The command |\childdocmain{|\textit{main}|}| must be followed by a whitespace.
It should not be followed immediately by another command
or by a comment mark `|%|'.
This is because the \TeX{} parser reads the token immediately following
the argument of |\childdocmain| and puts it
at the beginning of every child section;
however, a white\-space is ignored.
\end{itemize}

%%%%%%%%%%%%%%%%%%%%%%%%%%%%%%%%%%%%%%%%
\paragraph{Content of Main File.}

It is advisable to place all content in the child files included by |\include|.
Any output contained in the main file will appear in all child documents
unless suppressed manually;
it cannot be suppressed automatically by the |\includeonly| directive
and thus should normally be avoided.
A method to include some content in the main file
by means of conditional processing is described in \secref{sec:conditional}.

%%%%%%%%%%%%%%%%%%%%%%%%%%%%%%%%%%%%%%%%
\paragraph{Page Numbering.}

When only a part of the document is compiled,
the appropriate numbering of pages
(as well as other status parameters)
is determined from the |.aux| files.
The latter contain information from previous passes.
However this information needs to propagate through
all intermediate child documents.
Therefore the page numbering in child documents may well
be inconsistent until the complete document is compiled at least once.

A useful (if unconventional) way to always ensure a consistent
page numbering is to restart the numbering in each child document
and denote the pages by `\textit{child}|.|\textit{page}'
where \textit{child} represents the chapter/section number of the child file.
This can be achieved by the command
|\numberwithin{page}{|\textit{child}|}|
of the \textsf{amsmath} package
where \textit{child} can be |chapter| or |section|
depending on the chosen structuring.
Alternatively, one can modify the macro |\thepage| appropriately
and reset the counter |page| at the start of each child file.

%%%%%%%%%%%%%%%%%%%%%%%%%%%%%%%%%%%%%%%%%%%%%%%%%%%%%%%%%%%%%%%%%%%%%%%%%%%%%%%%
\subsection{Conditional Processing}
\label{sec:conditional}

The package provides a mechanism to compile different versions
of a document. To customise the versions further some conditional processing
can come in handy to distinguish which version is being compiled.
The package provides two macros to describe the compilation context:

%%%%%%%%%%%%%%%%%%%%%%%%%%%%%%%%%%%%%%%%
\DescribeMacro{\ifchilddoc}
The conditional |\ifchilddoc| distinguishes between the compilation of
child documents and the main document:
%
\begin{center}
|\ifchilddoc |\textit{child-code}| |[|\||else |\textit{main-code}]| \||fi|
\end{center}

%%%%%%%%%%%%%%%%%%%%%%%%%%%%%%%%%%%%%%%%
\DescribeMacro{\childdocname}
\DescribeMacro{\childdocjob}
The macro |\childdocname| contains the filename (without extension)
of the main or child file being processed.
Note that |\childdocjob| will always contain the name of the main file.

%%%%%%%%%%%%%%%%%%%%%%%%%%%%%%%%%%%%%%%%
\paragraph{Title Page.}

Conditional processing can be used to include a title or banner page
in the main document when proper precautions are taken.
Importantly, the code in the main file should ensure that the page counter
(as well as other status parameters which are stored in the |.aux| files)
takes the same value after the conditional processing.
Otherwise the page numbers may take divergent values
depending on which part is compiled.

For example, a title page could be declared by:
%
\begin{center}
\begin{tabular}{l}
|\ifchilddoc\||else|\\
|\addtocounter{page}{-1}|\\
\textit{code for title page}\\
|\newpage|\\
|\||fi|
\end{tabular}
\end{center}
%
A banner page for the child documents can be generated by:
%
\begin{center}
\begin{tabular}{l}
|\ifchilddoc|\\
|\addtocounter{page}{-1}|\\
\textit{code for banner page}\\
|\newpage|\\
|\||fi|
\end{tabular}
\end{center}
%
Here one could write a message such as:
\begin{center}
|This is the part \childdocname{} of \childdocjob{}.|
\end{center}

%%%%%%%%%%%%%%%%%%%%%%%%%%%%%%%%%%%%%%%%%%%%%%%%%%%%%%%%%%%%%%%%%%%%%%%%%%%%%%%%
\subsection{Flags}
\label{sec:flags}

The package makes it easy to generate different versions
of the main or child documents.
To this end compilation flags can be defined
and assigned different default values.
They will be particularly useful in conjunction
with the forwarding mechanism described in \secref{sec:forward}.

For example, it may be useful to have a flag |\version|
which can be set to |draft| or |final|.
The document source will contain some conditional code
depending on the value of |\version|.
Suppose further, the flag should default to |final| for the main file
and to |draft| for child files
which is a natural assignment for editing the document.
This is achieved by placing the following code
in the preamble of the main document
(below the |\childdocmain| directive):
%
\begin{center}
\begin{tabular}{l}
|\ifchilddoc|\\
|\providecommand{\version}{draft}|\\
|\||else|\\
|\providecommand{\version}{final}|\\
|\||fi|
\end{tabular}
\end{center}
%
The definition by |\providecommand| makes sure
that previous definitions are not overwritten.
Further statements |\providecommand{\version}{...}|
can thus be added before the above code to override it.

For the main file, one might add a line
(between |\childdocmain| and the above block)
%
\begin{center}
|%\ifchilddoc\||else\providecommand{\version}{draft}\||fi|
\end{center}
%
which can be uncommented to produce a draft version.
Likewise one can add a line to the very top of a child file
(above the |\childdocof{|\textit{main}|}| directive)
%
\begin{center}
|%\providecommand{\version}{final}|
\end{center}
%
which can be uncommented to produce the final version of this child document.

%%%%%%%%%%%%%%%%%%%%%%%%%%%%%%%%%%%%%%%%%%%%%%%%%%%%%%%%%%%%%%%%%%%%%%%%%%%%%%%%
\subsection{Forwarding}
\label{sec:forward}

Different versions of the main or child documents
using compilation flags as described in \secref{sec:flags}
can be (permanently) stored in different files
for convenient compilation, viewing and distribution.
To this end, the package defines a command
to pass on compilation to a different file:

%%%%%%%%%%%%%%%%%%%%%%%%%%%%%%%%%%%%%%%%
\DescribeMacro{\childdocforward}
The command |\childdocforward| redirects processing to
another source file:
%
\begin{center}
\begin{tabular}{l}
|\input{childdoc.def}|\\
|\childdocforward[|\textit{main}|]{|\textit{dest}|}|\\
\end{tabular}
\end{center}
%
The argument \textit{dest} is the destination file
(without extension).
It should be the main file or one of the child files.
Note that further \textsf{childdoc} directives
such as |\childdocof| and |\childdocforward|
in the indicated file will be processed in this form.
The optional argument \textit{main}
passes on directly to the main file \textit{main}
while pretending to compile the child \textit{dest}.
This form behaves as if \textit{dest}
issues |\childdocof{|\textit{main}|}| right away,
and no further \textsf{childdoc} directives will be processed.

%%%%%%%%%%%%%%%%%%%%%%%%%%%%%%%%%%%%%%%%
\DescribeMacro{\...prefix}
In the alternative form |\childdocforwardprefix|,
%
\begin{center}
\begin{tabular}{l}
|\input{childdoc.def}|\\
|\childdocforwardprefix[|\textit{main}|]{|\textit{prefix}|}{|\textit{dest}|}|
\end{tabular}
\end{center}
%
the destination file is determined by a pattern
depending on the current file:
To make this work, the current file must be called
`{\textit{prefix}\hspace{0.2em}\textit{suffix}}'
with \textit{prefix} matching precisely the argument.
Processing is then passed on to the file
`{\textit{dest}\hspace{0.2em}\textit{suffix}}'.
Surely, the same effect is achieved by
directly specifying the
argument `{\textit{dest}\hspace{0.2em}\textit{suffix}}'
in the first form.
However, that requires to set up a different file
for each child. With the alternative form of the command
all these files can have exactly the same content
which simplifies setting them up and maintaining them.

For example, the following file |draft.tex|
with a compilation flag |\version| as described in \secref{sec:flags}
compiles the main document as a draft:
%
\begin{center}
\begin{tabular}{l}
|\def\version{draft}|\\
|\input{childdoc.def}|\\
|\childdocforward{|\textit{main}|}|
\end{tabular}
\end{center}
%
Likewise, the following files |final|\textit{nn}|.tex|
compile the final version of the child document
|child|\textit{nn}|.tex|:
%
\begin{center}
\begin{tabular}{l}
|\def\version{final}|\\
|\input{childdoc.def}|\\
|\childdocforwardprefix{final}{child}|
\end{tabular}
\end{center}
%

Note that when several versions of a main file and/or of each child file
are to be generated, it may be convenient to set up a |Makefile| or
shell script to automatise the process.

%%%%%%%%%%%%%%%%%%%%%%%%%%%%%%%%%%%%%%%%%%%%%%%%%%%%%%%%%%%%%%%%%%%%%%%%%%%%%%%%
\subsection{Command Line Processing}
\label{sec:commandline}

The effect of redirection files can also be achieved by invoking
the \LaTeX{} compiler with a more elaborate command line.
Most conveniently this should be done as part
of a shell script or a |Makefile|.

When using \textsf{childdoc} in the main file, the following
command lines effectively perform a redirection
(note that depending on the shell being used,
backslashes may have to be doubled: `|\|' $\to$ `|\\|'):
%
\begin{center}
|... -jobname "|\textit{target}|" |\\|"|[\textit{flags}]%
|\input{childdoc.def}\childdocforward[|\textit{main}|]{|\textit{dest}|}"|
\end{center}
%
Here \textit{target} is the name of the output file,
\textit{main} is the name of the main file
and \textit{dest} is the name of the main or child file to be processed
(all filenames without extensions).
The optional argument \textit{main} can be omitted
if \textit{main} matches \textit{dest}.
Optionally, compilation \textit{flags} can be defined via |\def| commands.
This command line makes the \TeX{} engine believe
it is compiling the file \textit{target}
whose content is specified as the latter parameter.
The provided code then forwards the processing to
\textit{main} or \textit{dest} as described in \secref{sec:forward}.

%%%%%%%%%%%%%%%%%%%%%%%%%%%%%%%%%%%%%%%%%%%%%%%%%%%%%%%%%%%%%%%%%%%%%%%%%%%%%%%%
\subsection{Include by Input}
\label{sec:input}

Including child documents by |\include| has some restrictions by design.
Most notably, the content of a child document always occupies
its own set of pages; pages cannot be shared between child documents.
Usually, this behaviour makes perfect sense
because each child document contain an essential part of the document.
However, in some situations it may be desirable to compose
a document from a collection of parts
without having mandatory page breaks between then.
For this case, the package
provides a mechanism to include parts
by |\input| which can also be processed individually.
However, by construction this mechanism
requires manual handling of the content to be output.

%%%%%%%%%%%%%%%%%%%%%%%%%%%%%%%%%%%%%%%%
\DescribeMacro{\ifchilddocmanual}
The main file should be prepared as usual, see \secref{sec:include}.
However, the document body must make a distinction
between processing of an individual part and of the main document, e.g.:
%
\begin{center}
\begin{tabular}{l}
|\ifchilddocmanual|\\
|\input{\childdocname}|\\
|\||else|\\
\textit{document body with }|\input{|\textit{part}|}|\\
|\||fi|
\end{tabular}
\end{center}
%
The conditional |\ifchilddocmanual| is true whenever
a part to be included by |\input| is being compiled,
and the name of the part is stored in |\childdocname|.

%%%%%%%%%%%%%%%%%%%%%%%%%%%%%%%%%%%%%%%%
\DescribeMacro{\childdocby}
Each part to be included by |\input| should start with:
%
\begin{center}
\begin{tabular}{l}
|\input{childdoc.def}|\\
|\childdocby{|\textit{main}|}|\\
\end{tabular}
\end{center}
%
The directive |\childdocby| is similar to |\childdocof|
described in \secref{sec:include},
but the subsequent selection of content must be done manually.
To that end, both |\ifchilddoc| and |\ifchilddocmanual|
will be true upon processing of a part,
and the name of the part is stored in |\childdocname|.
Note that |\jobname| will be set to the filename of the current part
so that each part receives an individual |.aux| file
that does not interfere with the |.aux| file(s) of the main document.
This behaviour can be altered by the alternative form
|\childdocby[*]{|\textit{main}|}| (with a non-empty optional argument)
which uses the |.aux| file of the main document
by setting |\jobname| to \textit{main}.

%%%%%%%%%%%%%%%%%%%%%%%%%%%%%%%%%%%%%%%%%%%%%%%%%%%%%%%%%%%%%%%%%%%%%%%%%%%%%%%%
\subsection{Driver Development}
\label{sec:driver}

The \textsf{childdoc} mechanism can also be use for the development
of definition files such as \LaTeX{} styles or classes.
This case differs from the above setup with multiple parts
included by |\include| in that no |\includeonly| should be invoked.
This can be achieved by starting the include file
(before |\ProvidesPackage|) with:
%
\begin{center}
\begin{tabular}{l}
|\input{childdoc.def}|\\
|\childdocforward{|\textit{main}|}|\\
\end{tabular}
\end{center}
%
or alternatively with:
%
\begin{center}
\begin{tabular}{l}
|\input{childdoc.def}|\\
|\childdocby{|\textit{main}|}|\\
\end{tabular}
\end{center}
%
Both forms have slightly different effects as described above.
The main file is prepared as usual, see \secref{sec:include}.

%%%%%%%%%%%%%%%%%%%%%%%%%%%%%%%%%%%%%%%%%%%%%%%%%%%%%%%%%%%%%%%%%%%%%%%%%%%%%%%%
\subsection{Legacy Detection}
\label{sec:detection}

The directive |\childdocmain| in the main file can detect
whether the complete document or merely a child is to be compiled
even without using the directive |\childdocof|.
This method is deprecated because it is less robust
and there is no compelling reason to use it;
it is merely provided for backward compatibility
and it may be removed in future versions.

If the detection mechanism is to be used,
it is mandatory to correctly specify
the filename of the main file as the argument of |\childdocmain|:
%
\begin{center}
\begin{tabular}{l}
|\input{childdoc.def}|\\
|\childdocmain{|\textit{main}|}|\\
\end{tabular}
\end{center}
%
If |\jobname| does not match the argument \textit{main} of |\childdocmain|,
it is assumed that |\jobname| points to the child file to be compiled.
When using |\childdocmain| with the main file specified as argument,
it suffices to start a child file
with just |\input{|\textit{main}|}|
without loading of the package and using |\childdocof|.
If instead all processing is done
with the appropriate \textsf{childdoc} directives,
the argument of \textit{main} of |\childdocmain| can be empty.

An alternative version of the command line processing described
in \secref{sec:commandline} using the detection mechanism reads:
%
\begin{center}
|... -jobname "|\textit{target}|" "|[\textit{flags}]%
[|\def\jobname{|\textit{dest}|}|]|\input{|\textit{main}|}"|
\end{center}

%%%%%%%%%%%%%%%%%%%%%%%%%%%%%%%%%%%%%%%%%%%%%%%%%%%%%%%%%%%%%%%%%%%%%%%%%%%%%%%%
\subsection{Manual Code}
\label{sec:manual}

In case one cannot be certain whether the definitions file |childdoc.def|
is installed on the target \TeX{} distribution
and one prefers not to ship it,
it is conceivable to paste a few relevant commands into the sources.

To that end, drop all statements |\input{childdoc.def}|
and perform the replacements as outlined below.
Instead of |\childdocmain{|\textit{main}|}| add the following code
to the top of the main file:
%
\begin{center}
\begin{tabular}{l}
|\||ifdefined\childdocname\endinput\||fi\newif\ifchilddoc|\\
|\edef\childdocname{\scantokens\expandafter{\jobname\noexpand}}|\\
|\def\childdocmain{|\textit{main}|}\||ifx\childdocmain\childdocname\||else|\\
|\childdoctrue\includeonly{\childdocname}\let\jobname\childdocmain\||fi|\\
\end{tabular}
\end{center}
%
Instead of |\childdocof{|\textit{main}|}| just include the main file
at the top of each child file:
%
\begin{center}
|\input{|\textit{main}|}|
\end{center}
%
A simple redirection |\childdocforward{|\textit{dest}|}| is achieved by:
%
\begin{center}
|\def\jobname{|\textit{dest}|}\input{\jobname}|
\end{center}
%
The redirection with prefix
|\childdocforwardprefix[|\textit{prefix}|]{|\textit{dest}|}|
is accomplished by:
%
\begin{center}
\begin{tabular}{l}
|{\edef\jobname{\scantokens\expandafter{\jobname\noexpand}}|\\
|\def\redirectjob |\textit{prefix}|#1~~~{\gdef\jobname{|\textit{dest}|#1}}|\\
|\expandafter\redirectjob\jobname~~~}\input{\jobname}|
\end{tabular}
\end{center}

In an alternative approach,
child documents can be compiled by a specific command line
without additional code or specific definitions:
%
\begin{center}
|... -jobname "|\textit{target}|" "|[\textit{flags}]%
|\includeonly{|\textit{dest}|}\input{|\textit{main}|}"|
\end{center}
%

%%%%%%%%%%%%%%%%%%%%%%%%%%%%%%%%%%%%%%%%%%%%%%%%%%%%%%%%%%%%%%%%%%%%%%%%%%%%%%%%
%%%%%%%%%%%%%%%%%%%%%%%%%%%%%%%%%%%%%%%%%%%%%%%%%%%%%%%%%%%%%%%%%%%%%%%%%%%%%%%%
\section{Information}

%%%%%%%%%%%%%%%%%%%%%%%%%%%%%%%%%%%%%%%%%%%%%%%%%%%%%%%%%%%%%%%%%%%%%%%%%%%%%%%%
\subsection{Copyright}

Copyright \copyright{} 2017--2018 Niklas Beisert

This work may be distributed and/or modified under the
conditions of the \LaTeX{} Project Public License, either version 1.3
of this license or (at your option) any later version.
The latest version of this license is in
  \url{http://www.latex-project.org/lppl.txt}
and version 1.3 or later is part of all distributions of \LaTeX{}
version 2005/12/01 or later.

This work has the LPPL maintenance status `maintained'.

The Current Maintainer of this work is Niklas Beisert.

This work consists of the files |README.txt|, |childdoc.ins| and |childdoc.dtx|
as well as the derived files |childdoc.def|, |cdocsamp.tex|
with |cdocsch1.tex|, |cdocsch2.tex|, |cdocspt3.tex|, |cdocspt4.tex|,
|cdocsdrf.tex|, |cdocsfn1.tex|, |cdocsfn2.tex|
as well as |childdoc.pdf|.

%%%%%%%%%%%%%%%%%%%%%%%%%%%%%%%%%%%%%%%%%%%%%%%%%%%%%%%%%%%%%%%%%%%%%%%%%%%%%%%%
\subsection{Files and Installation}

The package consists of the files:
%
\begin{center}
\begin{tabular}{ll}
    |README.txt|   & readme file \\
    |childdoc.ins| & installation file \\
    |childdoc.dtx| & source file \\
    |childdoc.def| & definition file \\
    |cdocsamp.tex| & sample main file \\
    |cdocsch1.tex| & sample include file \\
    |cdocsch2.tex| & sample include file \\
    |cdocspt3.tex| & sample part file \\
    |cdocspt4.tex| & sample part file \\
    |cdocsdrf.tex| & sample redirection file \\
    |cdocsfn1.tex| & sample redirection file \\
    |cdocsfn2.tex| & sample redirection file \\
    |childdoc.pdf| & manual
\end{tabular}
\end{center}
%
The distribution consists of the files
|README.txt|, |childdoc.ins| and |childdoc.dtx|.
%
\begin{itemize}
\item
Run (pdf)\LaTeX{} on |childdoc.dtx|
to compile the manual |childdoc.pdf| (this file).
\item
Run \LaTeX{} on |childdoc.ins| to create the definitions file |childdoc.def|
and the sample |cdocsamp.tex| with include files
|cdocsch1.tex|, |cdocsch2.tex|, |cdocspt3.tex|, |cdocspt4.tex|,
|cdocsdrf.tex|, |cdocsfn1.tex|, |cdocsfn2.tex|.
Then copy the file |childdoc.def| to an appropriate directory of your \LaTeX{}
distribution, e.g.\ \textit{texmf-root}|/tex/latex/childdoc|.
\end{itemize}

%%%%%%%%%%%%%%%%%%%%%%%%%%%%%%%%%%%%%%%%%%%%%%%%%%%%%%%%%%%%%%%%%%%%%%%%%%%%%%%%
\subsection{Related CTAN Packages}

There are several other packages which offer a similar functionality:
%
\begin{itemize}
\item
The packages
\href{http://ctan.org/pkg/docmute}{\textsf{docmute}},
\href{http://ctan.org/pkg/includex}{\textsf{includex}} and
\href{http://ctan.org/pkg/standalone}{\textsf{standalone}}
provide commands to include only the document body of
a child file thus allowing both files to be compiled individually.
\item
The packages \href{http://ctan.org/pkg/subdocs}{\textsf{subdocs}}
and \href{http://ctan.org/pkg/subfiles}{\textsf{subfiles}}
provide structures in which the main and child documents can be
encapsulated and allowing them to be compiled individually.
The inclusion mechanism is different from the conventional |\include|.
\item
The package \href{http://ctan.org/pkg/combine}{\textsf{combine}}
is an elaborate solution to combine several documents into one.
\end{itemize}
%
See also the CTAN topic \href{http://ctan.org/topic/subdocs}{\textsf{subdocs}}
for further related packages.
The present package differs from the above solutions in that
a document structure constructed with the conventional |\include| mechanism
just needs two extra commands at the top of every file
such that all constituent files can be compiled individually.

%%%%%%%%%%%%%%%%%%%%%%%%%%%%%%%%%%%%%%%%%%%%%%%%%%%%%%%%%%%%%%%%%%%%%%%%%%%%%%%%
%\subsection{Feature Suggestions}
%
%The following is a list of features which may be useful for future
%versions of this package:
%%
%\begin{itemize}
%\item
%\ldots
%\end{itemize}

%%%%%%%%%%%%%%%%%%%%%%%%%%%%%%%%%%%%%%%%%%%%%%%%%%%%%%%%%%%%%%%%%%%%%%%%%%%%%%%%
\subsection{Revision History}

%%%%%%%%%%%%%%%%%%%%%%%%%%%%%%%%%%%%%%%%
\paragraph{v2.0:} 2018/12/30

\begin{itemize}
\item
immediate forward processing
\item
added |\childdocby| mechanism
\item
manual restructured
\end{itemize}

%%%%%%%%%%%%%%%%%%%%%%%%%%%%%%%%%%%%%%%%
\paragraph{v1.6:} 2018/01/17

\begin{itemize}
\item
application for development of include files
\item
corrections to manual
\end{itemize}

%%%%%%%%%%%%%%%%%%%%%%%%%%%%%%%%%%%%%%%%
\paragraph{v1.5:} 2017/05/21

\begin{itemize}
\item
more complete structuring introduced
\item
|\childdocof| introduced
\item
|\childdoc| renamed to |\childdocmain|
\item
|\childredirect| renamed to |\childdocforward| and |\childdocforwardprefix|
and functionality expanded
\end{itemize}

%%%%%%%%%%%%%%%%%%%%%%%%%%%%%%%%%%%%%%%%
\paragraph{v1.0:} 2017/04/27

\begin{itemize}
\item
manual and install package
\item
first version published on CTAN
\end{itemize}

%%%%%%%%%%%%%%%%%%%%%%%%%%%%%%%%%%%%%%%%
\paragraph{v0.6:} 2017/04/26

\begin{itemize}
\item
redirection mechanism added
\end{itemize}

%%%%%%%%%%%%%%%%%%%%%%%%%%%%%%%%%%%%%%%%
\paragraph{v0.5:} 2017/04/26

\begin{itemize}
\item
functionality in definition file
\end{itemize}


%%%%%%%%%%%%%%%%%%%%%%%%%%%%%%%%%%%%%%%%%%%%%%%%%%%%%%%%%%%%%%%%%%%%%%%%%%%%%%%%
%%%%%%%%%%%%%%%%%%%%%%%%%%%%%%%%%%%%%%%%%%%%%%%%%%%%%%%%%%%%%%%%%%%%%%%%%%%%%%%%
%%%%%%%%%%%%%%%%%%%%%%%%%%%%%%%%%%%%%%%%%%%%%%%%%%%%%%%%%%%%%%%%%%%%%%%%%%%%%%%%
\appendix

\settowidth\MacroIndent{\rmfamily\scriptsize 000\ }

 \DocInput{childdoc.dtx}

\end{document}
%</driver>
% \fi
%
% %%%%%%%%%%%%%%%%%%%%%%%%%%%%%%%%%%%%%%%%%%%%%%%%%%%%%%%%%%%%%%%%%%%%%%%%%%%%%%
% %%%%%%%%%%%%%%%%%%%%%%%%%%%%%%%%%%%%%%%%%%%%%%%%%%%%%%%%%%%%%%%%%%%%%%%%%%%%%%
% \section{Sample}
%\iffalse
%<*samplemain>
%\fi
%
% The following presents a sample document
% with two chapters, two parts, a title page,
% a compile flag as well as three forwarding files to set the flag.
% It consists of eight |.tex| files:
% \begin{center}
% \begin{tabular}{ll}
% |cdocsamp.tex|&main file\\
% |cdocsch1.tex|&include file for chapter 1\\
% |cdocsch2.tex|&include file for chapter 2\\
% |cdocspt3.tex|&include file for part 3\\
% |cdocspt4.tex|&include file for part 4\\
% |cdocsdrf.tex|&forwarding file for main file in draft mode\\
% |cdocsfi1.tex|&forwarding file for final version of chapter 1\\
% |cdocsfi2.tex|&forwarding file for final version of chapter 2\\
% \end{tabular}
% \end{center}
% Each of the eight files can be compiled directly by the \LaTeX{} compiler.
%
% %%%%%%%%%%%%%%%%%%%%%%%%%%%%%%%%%%%%%%
% \paragraph{Main File.}
%
% The main file is called |cdocsamp.tex|.
%
% Load the \textsf{childdoc} definitions and
% declare the filename for the main document:
%    \begin{macrocode}
\input{childdoc.def}
\childdocmain{}
%    \end{macrocode}

% Optional override for |\version| flag:
%    \begin{macrocode}
%%\ifchilddoc\else\providecommand{\version}{draft}\fi
%    \end{macrocode}

% Define the default values for the |\version| flag
% (|final| for the main file and |draft| for childs):
%    \begin{macrocode}
\ifchilddoc
\providecommand{\version}{draft}
\else
\providecommand{\version}{final}
\fi
%    \end{macrocode}

% Load the standard document class:
%    \begin{macrocode}
\documentclass[12pt]{article}
%    \end{macrocode}

% Start the document body:
%    \begin{macrocode}
\begin{document}
%    \end{macrocode}

% Declare a title page.
% Print title, part of document being processed and version flag:
%    \begin{macrocode}
\addtocounter{page}{-1}
\begin{center}
{\LARGE\bfseries{}childdoc example\par}
\vspace{1cm}
\ifchilddoc
\ifchilddocmanual part\else chapter\fi:
`\childdocname' of `\childdocjob'\par
\else
main document: `\childdocjob'\par
\fi
version: \version\par
\end{center}
\newpage
%    \end{macrocode}

% Manually include selected file,
% otherwise process as usual:
%    \begin{macrocode}
\ifchilddocmanual
\section*{part `\childdocname'}
\input{\childdocname}
\else
%    \end{macrocode}

% Include the two chapters:
%    \begin{macrocode}
\include{cdocsch1}
\include{cdocsch2}
%    \end{macrocode}

% Include the two parts unless only chapters should be displayed:
%    \begin{macrocode}
\ifchilddoc\else
\section{part three}
\input{cdocspt3}
\section{part four}
\input{cdocspt4}
\fi
%    \end{macrocode}

% Process as usual until here:
%    \begin{macrocode}
\fi
%    \end{macrocode}

% End of document body:
%    \begin{macrocode}
\end{document}
%    \end{macrocode}
%\iffalse
%</samplemain>
%\fi
%
% %%%%%%%%%%%%%%%%%%%%%%%%%%%%%%%%%%%%%%
% \paragraph{Chapter Include Files.}
%
% The include files are called |cdocsch1.tex| and |cdocsch2.tex|.
%
%\iffalse
%<*samplechap1|samplechap2>
%\fi

% Optional override for |\version| flag:
%    \begin{macrocode}
%%\providecommand{\version}{final}
%    \end{macrocode}

% Include the main document:
%    \begin{macrocode}
\input{childdoc.def}
\childdocof{cdocsamp}
%    \end{macrocode}

%\iffalse
%</samplechap1|samplechap2>
%\fi
%
%\iffalse
%<*samplechap1>
%\fi
% Some text for chapter 1:
%    \begin{macrocode}
\section{one}
some text in chapter one
%    \end{macrocode}

%\iffalse
%</samplechap1>
%\fi
% Some text for chapter 2:
%\iffalse
%<*samplechap2>
%\fi
%    \begin{macrocode}
\section{two}
more text in chapter two
%    \end{macrocode}

%\iffalse
%</samplechap2>
%\fi
%
% %%%%%%%%%%%%%%%%%%%%%%%%%%%%%%%%%%%%%%
% \paragraph{Part Include Files.}
%
% The include files are called |cdocspt3.tex| and |cdocspt4.tex|.
%
%\iffalse
%<*samplepart3|samplepart4>
%\fi

% Optional override for |\version| flag:
%    \begin{macrocode}
%%\providecommand{\version}{final}
%    \end{macrocode}

% Include the main document:
%    \begin{macrocode}
\input{childdoc.def}
\childdocby{cdocsamp}
%    \end{macrocode}

%\iffalse
%</samplepart3|samplepart4>
%\fi
%
%\iffalse
%<*samplepart3>
%\fi
% Some text for part 3:
%    \begin{macrocode}
some text in part three
%    \end{macrocode}

%\iffalse
%</samplepart3>
%\fi
% Some text for part 4:
%\iffalse
%<*samplepart4>
%\fi
%    \begin{macrocode}
more text in part four
%    \end{macrocode}

%\iffalse
%</samplepart4>
%\fi
%
% %%%%%%%%%%%%%%%%%%%%%%%%%%%%%%%%%%%%%%
% \paragraph{Forwarding for a Complete Draft.}
%
% The following forwarding file |cdocsdrf.tex|
% compiles the main document in draft mode:
%\iffalse
%<*sampledraft>
%\fi
%    \begin{macrocode}
\def\version{draft}
\input{childdoc.def}
\childdocforward{cdocsamp}
%    \end{macrocode}

%\iffalse
%</sampledraft>
%\fi
%
% %%%%%%%%%%%%%%%%%%%%%%%%%%%%%%%%%%%%%%
% \paragraph{Forwarding for Final Version of the Chapters.}
%
% The following forwarding files |cdocsfn1.tex| and |cdocsfn2.tex|
% (with identical content)
% compile the final versions of the child documents
% |cdocsch1.tex| and |cdocsch2.tex|, respectively:
%\iffalse
%<*samplefinal>
%\fi
%    \begin{macrocode}
\def\version{final}
\input{childdoc.def}
\childdocforwardprefix[cdocsamp]{cdocsfn}{cdocsch}
%    \end{macrocode}

%\iffalse
%</samplefinal>
%\fi
%
% %%%%%%%%%%%%%%%%%%%%%%%%%%%%%%%%%%%%%%
% \paragraph{Command Line Processing.}
%
% The following three command lines generate the output files
% |cdocscld|, |cdocscl1| and |cdocscl2|
% which should be identical to
% |cdocsdrf|, |cdocsch1| and |cdocsfn2|, respectively:
% \begin{center}
% \begin{tabular}{l}
% |latex -jobname cdocscld \|\\
% |  "\def\version{draft}\input{childdoc.def}\childdocforward{cdocsamp}"|\\
% |latex -jobname cdocscl1 \|\\
% |  "\input{childdoc.def}\childdocforward[cdocsamp]{cdocsch1}"|\\
% |latex -jobname cdocscl2 \|\\
% |  "\def\version{final}\input{childdoc.def}\childdocforward{cdocsch2}"|
% \end{tabular}
% \end{center}
% Note that the trailing backslash on each first line
% merely continues the input to the second line
% (for convenient cut ant paste).
% Furthermore, the command |latex| can be replaced by any
% of its alternative versions such as |pdflatex|.
%
% %%%%%%%%%%%%%%%%%%%%%%%%%%%%%%%%%%%%%%%%%%%%%%%%%%%%%%%%%%%%%%%%%%%%%%%%%%%%%%
% %%%%%%%%%%%%%%%%%%%%%%%%%%%%%%%%%%%%%%%%%%%%%%%%%%%%%%%%%%%%%%%%%%%%%%%%%%%%%%
% \section{Implementation}
%\iffalse
%<*package>
%\fi
%
% This section describes the definitions file |childdoc.def|.

% The definitions cannot be loaded using |\usepackage| or |\RequirePackage|
% which has a mechanism to prevent loading a style file more than once.
% When loading the definitions by means of |\input|
% multiple instances have to be prevented manually:
%\iffalse
%This code needs to be before the `\ProvidesFile' directive
%which is defined at the beginning of this file.
%Therefore it is also placed there and commented out here.
%</package>
%<*discard>
%\fi
%    \begin{macrocode}
\ifdefined\childdocmain\endinput\fi
%    \end{macrocode}
%\iffalse
%</discard>
%<*package>
%\fi
%
% \macro{\ifchilddoc}
% \macro{\ifchilddocmanual}
% The conditional |\ifchilddoc| tells whether a
% child (true) or main (false) document is being compiled.
% The conditional |\ifchilddocmanual| tells whether
% the |\includeonly| mechanism is used (false) or
% the selection of child files must be performed manually (true).
% The definitions initialise to false:
%    \begin{macrocode}
\newif\ifchilddoc
\newif\ifchilddocmanual
%    \end{macrocode}

% \macro{\childdocname}
% \macro{\childdocjob}
% The macro |\childdocname| stores the name of the main document
% to be compiled. The macro |\childdocjob| stores the name of
% the document on which the \LaTeX{} compiler was originally invoked.
% The content of |\jobname| cannot be compared
% to filenames specified in the source due to different catcodes.
% The following code rescans |\jobname|, stores the result
% in |\childdocname| and saves a copy in |\childdocjob|:
%    \begin{macrocode}
\edef\childdocname{\scantokens\expandafter{\jobname\noexpand}}
\let\childdocjob\childdocname
%    \end{macrocode}

% \macro{\childdocdisable}
% The macro |\childdocdisable| prevents the main file
% from being processed more than once.
% At this stage, the main document command |\childdocmain|
% is assumed to be called once again where it should do nothing.
% Any subsequent call to it should prevent
% a secondary processing of the main document
% It overwrites the forwarding commands
% |\childdocof| and |\childdocforward|
% with empty macros to prevent further inclusions of the main document:
%    \begin{macrocode}
\newcommand{\childdocdisable}
{
  \renewcommand{\childdocmain}[1]{\renewcommand{\childdocmain}[1]{\endinput}}
  \renewcommand{\childdocof}[1]{}
  \renewcommand{\childdocby}[2][]{}
  \renewcommand{\childdocforward}[2][]{}
  \renewcommand{\childdocdisable}{}
}
%    \end{macrocode}

% \macro{\childdocmain}
% The macro |\childdocmain| is to be called at the top of the main file
% with nothing or the main filename (without extension) as argument.
% First, it breaks loops.
% If the argument is not empty and does not match |\childdocname|
% (which is set by the first inclusion of |childdoc.def|),
% |\ifchilddoc| is set to true, |\includeonly| is applied to the child file
% and |\jobname| is set to the main file
% (for proper handling of |.aux| files):
%    \begin{macrocode}
\newcommand{\childdocmain}[1]
{
  \childdocdisable\childdocmain{}
  \if?#1?\else
    \begingroup
      \def\childdoctmp{#1}
      \ifx\childdoctmp\childdocname
        \def\childdoctmp{}
      \else
        \def\childdoctmp
        {
          \childdoctrue
          \includeonly{\childdocname}
          \def\childdocjob{#1}
          \def\jobname{#1}
        }
      \fi
      \expandafter
    \endgroup
    \childdoctmp
  \fi
}
%    \end{macrocode}

% \macro{\childdocof}
% The command |\childdocof| redirects
% compilation to the main file |#1|.
%    \begin{macrocode}
\newcommand{\childdocof}[1]
{
  \childdocdisable
  \childdoctrue
  \includeonly{\childdocname}
  \def\jobname{#1}
  \def\childdocjob{#1}
  \input{#1}
}
%    \end{macrocode}

% \macro{\childdocby}
% The command |\childdocby| ....
%    \begin{macrocode}
\newcommand{\childdocby}[2][]
{
  \childdocdisable
  \childdoctrue
  \childdocmanualtrue
  \if?#1?\else
    \def\jobname{#2}
  \fi
  \def\childdocjob{#2}
  \input{#2}
  \endinput
}
%    \end{macrocode}

% \macro{\childdocforward}
% The command |\childdocforward| redirects
% compilation to the main file or
% (if the optional argument is given) a child file.
% Parameters are set as if the main file
% or a child file starting with |\childdocof| was compiled.
% Then compilation is handed over to the main file:
%    \begin{macrocode}
\newcommand{\childdocforward}[2][]
{
  \begingroup
    \if?#1?
      \def\childdoctmp
      {
        \def\childdocname{#2}
        \def\childdocjob{#2}
        \def\jobname{#2}
        \input{#2}
        \endinput
      }
    \else
      \def\childdoctmp
      {
        \childdocdisable
        \def\childdocname{#2}
        \childdoctrue
        \includeonly{#2}
        \def\childdocjob{#1}
        \def\jobname{#1}
        \input{#1}
        \endinput
      }
    \fi
    \expandafter
  \endgroup
  \childdoctmp
}
%    \end{macrocode}

% \macro{\childdocforwardprefix}
% The command |\childdocforwardprefix| redirects
% compilation to the main or a child file by means of a pattern.
% The prefix |#1| in the current filename is replaced by |#2|
% and the suffix of the current filename is kept
% (it is assumed that the filename does not contain the substring `|~~~|'
% which is used as a delimiter).
% Compilation is handed over to the new file by |\childdocforward|:
%    \begin{macrocode}
\newcommand{\childdocforwardprefix}[3][]
{
  \begingroup
    \def\childdocextract #2##1~~~{\def\childdoctmp{\childdocforward[#1]{#3##1}}}
    \expandafter\childdocextract\childdocname~~~
    \expandafter
  \endgroup
  \childdoctmp
}
%    \end{macrocode}

% \macro{\childdoc}
% The deprecated macro |\childdoc| is a legacy version of |\childdocmain|:
%    \begin{macrocode}
\newcommand{\childdoc}{\childdocmain}
%    \end{macrocode}

% \macro{\childdocredirect}
% The deprecated macro |\childdocredirect| is a legacy version
% of |\childdocforward| and |\childdocforwardprefix|:
%    \begin{macrocode}
\newcommand{\childdocredirect}[2][]
{
  \begingroup
    \if?#1?
      \def\childdoctmp{\childdocforward{#2}}
    \else
      \def\childdoctmp{\childdocforwardprefix{#1}{#2}}
    \fi
    \expandafter
  \endgroup
  \childdoctmp
}
%    \end{macrocode}

%\iffalse
%</package>
%\fi
%
\endinput
\childdocforward[|\textit{main}|]{|\textit{dest}|}"|
\end{center}
%
Here \textit{target} is the name of the output file,
\textit{main} is the name of the main file
and \textit{dest} is the name of the main or child file to be processed
(all filenames without extensions).
The optional argument \textit{main} can be omitted
if \textit{main} matches \textit{dest}.
Optionally, compilation \textit{flags} can be defined via |\def| commands.
This command line makes the \TeX{} engine believe
it is compiling the file \textit{target}
whose content is specified as the latter parameter.
The provided code then forwards the processing to
\textit{main} or \textit{dest} as described in \secref{sec:forward}.

%%%%%%%%%%%%%%%%%%%%%%%%%%%%%%%%%%%%%%%%%%%%%%%%%%%%%%%%%%%%%%%%%%%%%%%%%%%%%%%%
\subsection{Include by Input}
\label{sec:input}

Including child documents by |\include| has some restrictions by design.
Most notably, the content of a child document always occupies
its own set of pages; pages cannot be shared between child documents.
Usually, this behaviour makes perfect sense
because each child document contain an essential part of the document.
However, in some situations it may be desirable to compose
a document from a collection of parts
without having mandatory page breaks between then.
For this case, the package
provides a mechanism to include parts
by |\input| which can also be processed individually.
However, by construction this mechanism
requires manual handling of the content to be output.

%%%%%%%%%%%%%%%%%%%%%%%%%%%%%%%%%%%%%%%%
\DescribeMacro{\ifchilddocmanual}
The main file should be prepared as usual, see \secref{sec:include}.
However, the document body must make a distinction
between processing of an individual part and of the main document, e.g.:
%
\begin{center}
\begin{tabular}{l}
|\ifchilddocmanual|\\
|\input{\childdocname}|\\
|\||else|\\
\textit{document body with }|\input{|\textit{part}|}|\\
|\||fi|
\end{tabular}
\end{center}
%
The conditional |\ifchilddocmanual| is true whenever
a part to be included by |\input| is being compiled,
and the name of the part is stored in |\childdocname|.

%%%%%%%%%%%%%%%%%%%%%%%%%%%%%%%%%%%%%%%%
\DescribeMacro{\childdocby}
Each part to be included by |\input| should start with:
%
\begin{center}
\begin{tabular}{l}
|% \iffalse
%
% childdoc.dtx Copyright (C) 2017-2018 Niklas Beisert
%
% This work may be distributed and/or modified under the
% conditions of the LaTeX Project Public License, either version 1.3
% of this license or (at your option) any later version.
% The latest version of this license is in
%   http://www.latex-project.org/lppl.txt
% and version 1.3 or later is part of all distributions of LaTeX
% version 2005/12/01 or later.
%
% This work has the LPPL maintenance status `maintained'.
%
% The Current Maintainer of this work is Niklas Beisert.
%
% This work consists of the files childdoc.dtx and childdoc.ins
% and the derived files childdoc.def and cdocsamp.tex with
% cdocsch1.tex, cdocsch2.tex, cdocsdrf.tex, cdocsfn1.tex, cdocsfn2.tex.
%
%<package>\ifdefined\childdocmain\endinput\fi
%<package>\ProvidesFile{childdoc.def}[2018/12/30 v2.0 child document driver]
%<samplemain>\ProvidesFile{cdocsamp.tex}[2018/12/30 v2.0 sample for childdoc]
%<*driver>
%\ProvidesFile{childdoc.drv}[2018/12/30 v2.0 childdoc reference manual file]
\PassOptionsToClass{10pt,a4paper}{article}
\documentclass{ltxdoc}

\usepackage[margin=35mm]{geometry}
\usepackage{hyperref}
\usepackage{hyperxmp}
\usepackage[usenames]{color}

\hypersetup{colorlinks=true}
\hypersetup{pdfstartview=FitH}
\hypersetup{pdfpagemode=UseNone}
\hypersetup{pdfsource={}}
\hypersetup{pdflang={en-UK}}
\hypersetup{pdfcopyright={Copyright 2017-2018 Niklas Beisert.
  This work may be distributed and/or modified under the
  conditions of the LaTeX Project Public License, either version 1.3
  of this license or (at your option) any later version.}}
\hypersetup{pdflicenseurl={http://www.latex-project.org/lppl.txt}}
\hypersetup{pdfcontactaddress={ETH Zurich, ITP, HIT K,
  Wolfgang-Pauli-Strasse 27}}
\hypersetup{pdfcontactpostcode={8093}}
\hypersetup{pdfcontactcity={Zurich}}
\hypersetup{pdfcontactcountry={Switzerland}}
\hypersetup{pdfcontactemail={nbeisert@itp.phys.ethz.ch}}
\hypersetup{pdfcontacturl={http://people.phys.ethz.ch/\xmptilde nbeisert/}}

\newcommand{\secref}[1]{\hyperref[#1]{section \ref*{#1}}}

\parskip1ex
\parindent0pt
\let\olditemize\itemize
\def\itemize{\olditemize\parskip0pt}

\begin{document}

\title{The \textsf{childdoc} Package}
\hypersetup{pdftitle={The childdoc Package}}
\author{Niklas Beisert\\[2ex]
  Institut f\"ur Theoretische Physik\\
  Eidgen\"ossische Technische Hochschule Z\"urich\\
  Wolfgang-Pauli-Strasse 27, 8093 Z\"urich, Switzerland\\[1ex]
  \href{mailto:nbeisert@itp.phys.ethz.ch}
  {\texttt{nbeisert@itp.phys.ethz.ch}}}
\hypersetup{pdfauthor={Niklas Beisert}}
\hypersetup{pdfsubject={Manual for the LaTeX2e Package childdoc}}
\date{30 December 2018, \textsf{v2.0}}
\maketitle

\begin{abstract}\noindent
\textsf{childdoc} is a \LaTeXe{} package
that enables the direct compilation
of document sections included by |\include|
to individual files.
\end{abstract}

\begingroup
\parskip0ex
\tableofcontents
\endgroup

%%%%%%%%%%%%%%%%%%%%%%%%%%%%%%%%%%%%%%%%%%%%%%%%%%%%%%%%%%%%%%%%%%%%%%%%%%%%%%%%
%%%%%%%%%%%%%%%%%%%%%%%%%%%%%%%%%%%%%%%%%%%%%%%%%%%%%%%%%%%%%%%%%%%%%%%%%%%%%%%%
\section{Introduction}

\LaTeX{} provides a mechanism to structure a large document (such as a book)
into a main file and several child files (containing the chapters)
using the |\include| command.
This mechanism is beneficial for documents
which span hundreds of pages in order to
make the source file(s) more manageable.
Moreover, compilation can be restricted to
selected child files by means of the |\includeonly| command.
The latter feature can be used to reduce the compilation time while editing
(this was significantly more useful in the earlier days of \LaTeX{})
or to generate a smaller document which is easier to navigate.
Another application of |\includeonly| is to generate
documents consisting of selected parts of the complete document.

However, there are a few drawbacks of the plain |\include| mechanism:
\begin{itemize}
\item
The child files cannot be compiled on their own,
they can only be compiled via the main file.
A naive editing environment
(such as a text editor with an option
to have the current file processed by \LaTeX)
may require one to switch to the main file before compiling;
attempting to compile the child file produces errors.
\item
The main file must be modified (each time)
to adjust the |\includeonly| command
to the present needs. This easily leaves the main file in a messy state.
\item
The generated document will always carry the filename
of the main document. This is inconvenient if
several child files are to be compiled and
to be kept for distribution.
\end{itemize}

The present package provides a simple interface
to make child files individually compilable by \LaTeX{}.
Compiling a child file then has the same effect as compiling
the main file with an |\includeonly| command
to select the appropriate child.
Moreover the generated document will carry the name of the child
rather than the main file.
This resolves all three above issues.

This feature is meant to make the editing of books,
thesis documents and lecture notes somewhat more convenient.
However, the package can also be used efficiently for
composing a series of documents (such as exercise sheets)
which are typically distributed individually.
It then assists the author in generating the individual documents
(potentially in different versions)
as well as a document containing the collected series.
Another application is in developing style files
or other kinds of included material
where compilation of the style file could redirect
to a sample or test file.

%%%%%%%%%%%%%%%%%%%%%%%%%%%%%%%%%%%%%%%%%%%%%%%%%%%%%%%%%%%%%%%%%%%%%%%%%%%%%%%%
%%%%%%%%%%%%%%%%%%%%%%%%%%%%%%%%%%%%%%%%%%%%%%%%%%%%%%%%%%%%%%%%%%%%%%%%%%%%%%%%
\section{Usage}

First of all, the package \textsf{childdoc} is \emph{not} a standard
\LaTeXe{} |.sty| style file! Therefore it needs to be invoked in
a non-standard way.

%%%%%%%%%%%%%%%%%%%%%%%%%%%%%%%%%%%%%%%%%%%%%%%%%%%%%%%%%%%%%%%%%%%%%%%%%%%%%%%%
\subsection{Included Files}
\label{sec:include}

%%%%%%%%%%%%%%%%%%%%%%%%%%%%%%%%%%%%%%%%
\DescribeMacro{\childdocmain}
To use the package, add the commands
\begin{center}
\begin{tabular}{l}
|\input{childdoc.def}|\\
|\childdocmain{}|\\
\end{tabular}
\end{center}
at the very top of the main \LaTeX{} file,
in particular \emph{before} the |\documentclass| statement!
The argument of |\childdocmain| should be left empty
(but it must be present).

%%%%%%%%%%%%%%%%%%%%%%%%%%%%%%%%%%%%%%%%
\DescribeMacro{\childdocof}
Furthermore, add the commands
\begin{center}
\begin{tabular}{l}
|\input{childdoc.def}|\\
|\childdocof{|\textit{main}|}|\\
\end{tabular}
\end{center}
at the top of every child file \textit{child}
which is included by |\include{|\textit{child}|}|
from within the main file
(or at least for those files to be compiled individually).
The argument \textit{main} must be the filename of the main file.

There are a couple of
considerations in setting up the main and child documents:

%%%%%%%%%%%%%%%%%%%%%%%%%%%%%%%%%%%%%%%%
\paragraph{Restrictions.}

Please note the following restrictions:
\begin{itemize}
\item
|\childdocmain| must be called with one argument \textit{main}
to ensure compatibility with earlier version of the package.
It must either be empty (|\childdocmain{}|)
or precisely match the filename of the main file in which it is specified.
See \secref{sec:detection} for further information.
\item
The filename \textit{main} must be specified without the |.tex| extension.
\item
The filename \textit{main} is case sensitive
(even in case-insensitive file systems)
due to internal string comparison.
\item
The argument \textit{main} should be fully expanded, it cannot be a macro.
\item
Subdirectories and special characters should be avoided in filenames.
\item
The command |\childdocmain{|\textit{main}|}| must be followed by a whitespace.
It should not be followed immediately by another command
or by a comment mark `|%|'.
This is because the \TeX{} parser reads the token immediately following
the argument of |\childdocmain| and puts it
at the beginning of every child section;
however, a white\-space is ignored.
\end{itemize}

%%%%%%%%%%%%%%%%%%%%%%%%%%%%%%%%%%%%%%%%
\paragraph{Content of Main File.}

It is advisable to place all content in the child files included by |\include|.
Any output contained in the main file will appear in all child documents
unless suppressed manually;
it cannot be suppressed automatically by the |\includeonly| directive
and thus should normally be avoided.
A method to include some content in the main file
by means of conditional processing is described in \secref{sec:conditional}.

%%%%%%%%%%%%%%%%%%%%%%%%%%%%%%%%%%%%%%%%
\paragraph{Page Numbering.}

When only a part of the document is compiled,
the appropriate numbering of pages
(as well as other status parameters)
is determined from the |.aux| files.
The latter contain information from previous passes.
However this information needs to propagate through
all intermediate child documents.
Therefore the page numbering in child documents may well
be inconsistent until the complete document is compiled at least once.

A useful (if unconventional) way to always ensure a consistent
page numbering is to restart the numbering in each child document
and denote the pages by `\textit{child}|.|\textit{page}'
where \textit{child} represents the chapter/section number of the child file.
This can be achieved by the command
|\numberwithin{page}{|\textit{child}|}|
of the \textsf{amsmath} package
where \textit{child} can be |chapter| or |section|
depending on the chosen structuring.
Alternatively, one can modify the macro |\thepage| appropriately
and reset the counter |page| at the start of each child file.

%%%%%%%%%%%%%%%%%%%%%%%%%%%%%%%%%%%%%%%%%%%%%%%%%%%%%%%%%%%%%%%%%%%%%%%%%%%%%%%%
\subsection{Conditional Processing}
\label{sec:conditional}

The package provides a mechanism to compile different versions
of a document. To customise the versions further some conditional processing
can come in handy to distinguish which version is being compiled.
The package provides two macros to describe the compilation context:

%%%%%%%%%%%%%%%%%%%%%%%%%%%%%%%%%%%%%%%%
\DescribeMacro{\ifchilddoc}
The conditional |\ifchilddoc| distinguishes between the compilation of
child documents and the main document:
%
\begin{center}
|\ifchilddoc |\textit{child-code}| |[|\||else |\textit{main-code}]| \||fi|
\end{center}

%%%%%%%%%%%%%%%%%%%%%%%%%%%%%%%%%%%%%%%%
\DescribeMacro{\childdocname}
\DescribeMacro{\childdocjob}
The macro |\childdocname| contains the filename (without extension)
of the main or child file being processed.
Note that |\childdocjob| will always contain the name of the main file.

%%%%%%%%%%%%%%%%%%%%%%%%%%%%%%%%%%%%%%%%
\paragraph{Title Page.}

Conditional processing can be used to include a title or banner page
in the main document when proper precautions are taken.
Importantly, the code in the main file should ensure that the page counter
(as well as other status parameters which are stored in the |.aux| files)
takes the same value after the conditional processing.
Otherwise the page numbers may take divergent values
depending on which part is compiled.

For example, a title page could be declared by:
%
\begin{center}
\begin{tabular}{l}
|\ifchilddoc\||else|\\
|\addtocounter{page}{-1}|\\
\textit{code for title page}\\
|\newpage|\\
|\||fi|
\end{tabular}
\end{center}
%
A banner page for the child documents can be generated by:
%
\begin{center}
\begin{tabular}{l}
|\ifchilddoc|\\
|\addtocounter{page}{-1}|\\
\textit{code for banner page}\\
|\newpage|\\
|\||fi|
\end{tabular}
\end{center}
%
Here one could write a message such as:
\begin{center}
|This is the part \childdocname{} of \childdocjob{}.|
\end{center}

%%%%%%%%%%%%%%%%%%%%%%%%%%%%%%%%%%%%%%%%%%%%%%%%%%%%%%%%%%%%%%%%%%%%%%%%%%%%%%%%
\subsection{Flags}
\label{sec:flags}

The package makes it easy to generate different versions
of the main or child documents.
To this end compilation flags can be defined
and assigned different default values.
They will be particularly useful in conjunction
with the forwarding mechanism described in \secref{sec:forward}.

For example, it may be useful to have a flag |\version|
which can be set to |draft| or |final|.
The document source will contain some conditional code
depending on the value of |\version|.
Suppose further, the flag should default to |final| for the main file
and to |draft| for child files
which is a natural assignment for editing the document.
This is achieved by placing the following code
in the preamble of the main document
(below the |\childdocmain| directive):
%
\begin{center}
\begin{tabular}{l}
|\ifchilddoc|\\
|\providecommand{\version}{draft}|\\
|\||else|\\
|\providecommand{\version}{final}|\\
|\||fi|
\end{tabular}
\end{center}
%
The definition by |\providecommand| makes sure
that previous definitions are not overwritten.
Further statements |\providecommand{\version}{...}|
can thus be added before the above code to override it.

For the main file, one might add a line
(between |\childdocmain| and the above block)
%
\begin{center}
|%\ifchilddoc\||else\providecommand{\version}{draft}\||fi|
\end{center}
%
which can be uncommented to produce a draft version.
Likewise one can add a line to the very top of a child file
(above the |\childdocof{|\textit{main}|}| directive)
%
\begin{center}
|%\providecommand{\version}{final}|
\end{center}
%
which can be uncommented to produce the final version of this child document.

%%%%%%%%%%%%%%%%%%%%%%%%%%%%%%%%%%%%%%%%%%%%%%%%%%%%%%%%%%%%%%%%%%%%%%%%%%%%%%%%
\subsection{Forwarding}
\label{sec:forward}

Different versions of the main or child documents
using compilation flags as described in \secref{sec:flags}
can be (permanently) stored in different files
for convenient compilation, viewing and distribution.
To this end, the package defines a command
to pass on compilation to a different file:

%%%%%%%%%%%%%%%%%%%%%%%%%%%%%%%%%%%%%%%%
\DescribeMacro{\childdocforward}
The command |\childdocforward| redirects processing to
another source file:
%
\begin{center}
\begin{tabular}{l}
|\input{childdoc.def}|\\
|\childdocforward[|\textit{main}|]{|\textit{dest}|}|\\
\end{tabular}
\end{center}
%
The argument \textit{dest} is the destination file
(without extension).
It should be the main file or one of the child files.
Note that further \textsf{childdoc} directives
such as |\childdocof| and |\childdocforward|
in the indicated file will be processed in this form.
The optional argument \textit{main}
passes on directly to the main file \textit{main}
while pretending to compile the child \textit{dest}.
This form behaves as if \textit{dest}
issues |\childdocof{|\textit{main}|}| right away,
and no further \textsf{childdoc} directives will be processed.

%%%%%%%%%%%%%%%%%%%%%%%%%%%%%%%%%%%%%%%%
\DescribeMacro{\...prefix}
In the alternative form |\childdocforwardprefix|,
%
\begin{center}
\begin{tabular}{l}
|\input{childdoc.def}|\\
|\childdocforwardprefix[|\textit{main}|]{|\textit{prefix}|}{|\textit{dest}|}|
\end{tabular}
\end{center}
%
the destination file is determined by a pattern
depending on the current file:
To make this work, the current file must be called
`{\textit{prefix}\hspace{0.2em}\textit{suffix}}'
with \textit{prefix} matching precisely the argument.
Processing is then passed on to the file
`{\textit{dest}\hspace{0.2em}\textit{suffix}}'.
Surely, the same effect is achieved by
directly specifying the
argument `{\textit{dest}\hspace{0.2em}\textit{suffix}}'
in the first form.
However, that requires to set up a different file
for each child. With the alternative form of the command
all these files can have exactly the same content
which simplifies setting them up and maintaining them.

For example, the following file |draft.tex|
with a compilation flag |\version| as described in \secref{sec:flags}
compiles the main document as a draft:
%
\begin{center}
\begin{tabular}{l}
|\def\version{draft}|\\
|\input{childdoc.def}|\\
|\childdocforward{|\textit{main}|}|
\end{tabular}
\end{center}
%
Likewise, the following files |final|\textit{nn}|.tex|
compile the final version of the child document
|child|\textit{nn}|.tex|:
%
\begin{center}
\begin{tabular}{l}
|\def\version{final}|\\
|\input{childdoc.def}|\\
|\childdocforwardprefix{final}{child}|
\end{tabular}
\end{center}
%

Note that when several versions of a main file and/or of each child file
are to be generated, it may be convenient to set up a |Makefile| or
shell script to automatise the process.

%%%%%%%%%%%%%%%%%%%%%%%%%%%%%%%%%%%%%%%%%%%%%%%%%%%%%%%%%%%%%%%%%%%%%%%%%%%%%%%%
\subsection{Command Line Processing}
\label{sec:commandline}

The effect of redirection files can also be achieved by invoking
the \LaTeX{} compiler with a more elaborate command line.
Most conveniently this should be done as part
of a shell script or a |Makefile|.

When using \textsf{childdoc} in the main file, the following
command lines effectively perform a redirection
(note that depending on the shell being used,
backslashes may have to be doubled: `|\|' $\to$ `|\\|'):
%
\begin{center}
|... -jobname "|\textit{target}|" |\\|"|[\textit{flags}]%
|\input{childdoc.def}\childdocforward[|\textit{main}|]{|\textit{dest}|}"|
\end{center}
%
Here \textit{target} is the name of the output file,
\textit{main} is the name of the main file
and \textit{dest} is the name of the main or child file to be processed
(all filenames without extensions).
The optional argument \textit{main} can be omitted
if \textit{main} matches \textit{dest}.
Optionally, compilation \textit{flags} can be defined via |\def| commands.
This command line makes the \TeX{} engine believe
it is compiling the file \textit{target}
whose content is specified as the latter parameter.
The provided code then forwards the processing to
\textit{main} or \textit{dest} as described in \secref{sec:forward}.

%%%%%%%%%%%%%%%%%%%%%%%%%%%%%%%%%%%%%%%%%%%%%%%%%%%%%%%%%%%%%%%%%%%%%%%%%%%%%%%%
\subsection{Include by Input}
\label{sec:input}

Including child documents by |\include| has some restrictions by design.
Most notably, the content of a child document always occupies
its own set of pages; pages cannot be shared between child documents.
Usually, this behaviour makes perfect sense
because each child document contain an essential part of the document.
However, in some situations it may be desirable to compose
a document from a collection of parts
without having mandatory page breaks between then.
For this case, the package
provides a mechanism to include parts
by |\input| which can also be processed individually.
However, by construction this mechanism
requires manual handling of the content to be output.

%%%%%%%%%%%%%%%%%%%%%%%%%%%%%%%%%%%%%%%%
\DescribeMacro{\ifchilddocmanual}
The main file should be prepared as usual, see \secref{sec:include}.
However, the document body must make a distinction
between processing of an individual part and of the main document, e.g.:
%
\begin{center}
\begin{tabular}{l}
|\ifchilddocmanual|\\
|\input{\childdocname}|\\
|\||else|\\
\textit{document body with }|\input{|\textit{part}|}|\\
|\||fi|
\end{tabular}
\end{center}
%
The conditional |\ifchilddocmanual| is true whenever
a part to be included by |\input| is being compiled,
and the name of the part is stored in |\childdocname|.

%%%%%%%%%%%%%%%%%%%%%%%%%%%%%%%%%%%%%%%%
\DescribeMacro{\childdocby}
Each part to be included by |\input| should start with:
%
\begin{center}
\begin{tabular}{l}
|\input{childdoc.def}|\\
|\childdocby{|\textit{main}|}|\\
\end{tabular}
\end{center}
%
The directive |\childdocby| is similar to |\childdocof|
described in \secref{sec:include},
but the subsequent selection of content must be done manually.
To that end, both |\ifchilddoc| and |\ifchilddocmanual|
will be true upon processing of a part,
and the name of the part is stored in |\childdocname|.
Note that |\jobname| will be set to the filename of the current part
so that each part receives an individual |.aux| file
that does not interfere with the |.aux| file(s) of the main document.
This behaviour can be altered by the alternative form
|\childdocby[*]{|\textit{main}|}| (with a non-empty optional argument)
which uses the |.aux| file of the main document
by setting |\jobname| to \textit{main}.

%%%%%%%%%%%%%%%%%%%%%%%%%%%%%%%%%%%%%%%%%%%%%%%%%%%%%%%%%%%%%%%%%%%%%%%%%%%%%%%%
\subsection{Driver Development}
\label{sec:driver}

The \textsf{childdoc} mechanism can also be use for the development
of definition files such as \LaTeX{} styles or classes.
This case differs from the above setup with multiple parts
included by |\include| in that no |\includeonly| should be invoked.
This can be achieved by starting the include file
(before |\ProvidesPackage|) with:
%
\begin{center}
\begin{tabular}{l}
|\input{childdoc.def}|\\
|\childdocforward{|\textit{main}|}|\\
\end{tabular}
\end{center}
%
or alternatively with:
%
\begin{center}
\begin{tabular}{l}
|\input{childdoc.def}|\\
|\childdocby{|\textit{main}|}|\\
\end{tabular}
\end{center}
%
Both forms have slightly different effects as described above.
The main file is prepared as usual, see \secref{sec:include}.

%%%%%%%%%%%%%%%%%%%%%%%%%%%%%%%%%%%%%%%%%%%%%%%%%%%%%%%%%%%%%%%%%%%%%%%%%%%%%%%%
\subsection{Legacy Detection}
\label{sec:detection}

The directive |\childdocmain| in the main file can detect
whether the complete document or merely a child is to be compiled
even without using the directive |\childdocof|.
This method is deprecated because it is less robust
and there is no compelling reason to use it;
it is merely provided for backward compatibility
and it may be removed in future versions.

If the detection mechanism is to be used,
it is mandatory to correctly specify
the filename of the main file as the argument of |\childdocmain|:
%
\begin{center}
\begin{tabular}{l}
|\input{childdoc.def}|\\
|\childdocmain{|\textit{main}|}|\\
\end{tabular}
\end{center}
%
If |\jobname| does not match the argument \textit{main} of |\childdocmain|,
it is assumed that |\jobname| points to the child file to be compiled.
When using |\childdocmain| with the main file specified as argument,
it suffices to start a child file
with just |\input{|\textit{main}|}|
without loading of the package and using |\childdocof|.
If instead all processing is done
with the appropriate \textsf{childdoc} directives,
the argument of \textit{main} of |\childdocmain| can be empty.

An alternative version of the command line processing described
in \secref{sec:commandline} using the detection mechanism reads:
%
\begin{center}
|... -jobname "|\textit{target}|" "|[\textit{flags}]%
[|\def\jobname{|\textit{dest}|}|]|\input{|\textit{main}|}"|
\end{center}

%%%%%%%%%%%%%%%%%%%%%%%%%%%%%%%%%%%%%%%%%%%%%%%%%%%%%%%%%%%%%%%%%%%%%%%%%%%%%%%%
\subsection{Manual Code}
\label{sec:manual}

In case one cannot be certain whether the definitions file |childdoc.def|
is installed on the target \TeX{} distribution
and one prefers not to ship it,
it is conceivable to paste a few relevant commands into the sources.

To that end, drop all statements |\input{childdoc.def}|
and perform the replacements as outlined below.
Instead of |\childdocmain{|\textit{main}|}| add the following code
to the top of the main file:
%
\begin{center}
\begin{tabular}{l}
|\||ifdefined\childdocname\endinput\||fi\newif\ifchilddoc|\\
|\edef\childdocname{\scantokens\expandafter{\jobname\noexpand}}|\\
|\def\childdocmain{|\textit{main}|}\||ifx\childdocmain\childdocname\||else|\\
|\childdoctrue\includeonly{\childdocname}\let\jobname\childdocmain\||fi|\\
\end{tabular}
\end{center}
%
Instead of |\childdocof{|\textit{main}|}| just include the main file
at the top of each child file:
%
\begin{center}
|\input{|\textit{main}|}|
\end{center}
%
A simple redirection |\childdocforward{|\textit{dest}|}| is achieved by:
%
\begin{center}
|\def\jobname{|\textit{dest}|}\input{\jobname}|
\end{center}
%
The redirection with prefix
|\childdocforwardprefix[|\textit{prefix}|]{|\textit{dest}|}|
is accomplished by:
%
\begin{center}
\begin{tabular}{l}
|{\edef\jobname{\scantokens\expandafter{\jobname\noexpand}}|\\
|\def\redirectjob |\textit{prefix}|#1~~~{\gdef\jobname{|\textit{dest}|#1}}|\\
|\expandafter\redirectjob\jobname~~~}\input{\jobname}|
\end{tabular}
\end{center}

In an alternative approach,
child documents can be compiled by a specific command line
without additional code or specific definitions:
%
\begin{center}
|... -jobname "|\textit{target}|" "|[\textit{flags}]%
|\includeonly{|\textit{dest}|}\input{|\textit{main}|}"|
\end{center}
%

%%%%%%%%%%%%%%%%%%%%%%%%%%%%%%%%%%%%%%%%%%%%%%%%%%%%%%%%%%%%%%%%%%%%%%%%%%%%%%%%
%%%%%%%%%%%%%%%%%%%%%%%%%%%%%%%%%%%%%%%%%%%%%%%%%%%%%%%%%%%%%%%%%%%%%%%%%%%%%%%%
\section{Information}

%%%%%%%%%%%%%%%%%%%%%%%%%%%%%%%%%%%%%%%%%%%%%%%%%%%%%%%%%%%%%%%%%%%%%%%%%%%%%%%%
\subsection{Copyright}

Copyright \copyright{} 2017--2018 Niklas Beisert

This work may be distributed and/or modified under the
conditions of the \LaTeX{} Project Public License, either version 1.3
of this license or (at your option) any later version.
The latest version of this license is in
  \url{http://www.latex-project.org/lppl.txt}
and version 1.3 or later is part of all distributions of \LaTeX{}
version 2005/12/01 or later.

This work has the LPPL maintenance status `maintained'.

The Current Maintainer of this work is Niklas Beisert.

This work consists of the files |README.txt|, |childdoc.ins| and |childdoc.dtx|
as well as the derived files |childdoc.def|, |cdocsamp.tex|
with |cdocsch1.tex|, |cdocsch2.tex|, |cdocspt3.tex|, |cdocspt4.tex|,
|cdocsdrf.tex|, |cdocsfn1.tex|, |cdocsfn2.tex|
as well as |childdoc.pdf|.

%%%%%%%%%%%%%%%%%%%%%%%%%%%%%%%%%%%%%%%%%%%%%%%%%%%%%%%%%%%%%%%%%%%%%%%%%%%%%%%%
\subsection{Files and Installation}

The package consists of the files:
%
\begin{center}
\begin{tabular}{ll}
    |README.txt|   & readme file \\
    |childdoc.ins| & installation file \\
    |childdoc.dtx| & source file \\
    |childdoc.def| & definition file \\
    |cdocsamp.tex| & sample main file \\
    |cdocsch1.tex| & sample include file \\
    |cdocsch2.tex| & sample include file \\
    |cdocspt3.tex| & sample part file \\
    |cdocspt4.tex| & sample part file \\
    |cdocsdrf.tex| & sample redirection file \\
    |cdocsfn1.tex| & sample redirection file \\
    |cdocsfn2.tex| & sample redirection file \\
    |childdoc.pdf| & manual
\end{tabular}
\end{center}
%
The distribution consists of the files
|README.txt|, |childdoc.ins| and |childdoc.dtx|.
%
\begin{itemize}
\item
Run (pdf)\LaTeX{} on |childdoc.dtx|
to compile the manual |childdoc.pdf| (this file).
\item
Run \LaTeX{} on |childdoc.ins| to create the definitions file |childdoc.def|
and the sample |cdocsamp.tex| with include files
|cdocsch1.tex|, |cdocsch2.tex|, |cdocspt3.tex|, |cdocspt4.tex|,
|cdocsdrf.tex|, |cdocsfn1.tex|, |cdocsfn2.tex|.
Then copy the file |childdoc.def| to an appropriate directory of your \LaTeX{}
distribution, e.g.\ \textit{texmf-root}|/tex/latex/childdoc|.
\end{itemize}

%%%%%%%%%%%%%%%%%%%%%%%%%%%%%%%%%%%%%%%%%%%%%%%%%%%%%%%%%%%%%%%%%%%%%%%%%%%%%%%%
\subsection{Related CTAN Packages}

There are several other packages which offer a similar functionality:
%
\begin{itemize}
\item
The packages
\href{http://ctan.org/pkg/docmute}{\textsf{docmute}},
\href{http://ctan.org/pkg/includex}{\textsf{includex}} and
\href{http://ctan.org/pkg/standalone}{\textsf{standalone}}
provide commands to include only the document body of
a child file thus allowing both files to be compiled individually.
\item
The packages \href{http://ctan.org/pkg/subdocs}{\textsf{subdocs}}
and \href{http://ctan.org/pkg/subfiles}{\textsf{subfiles}}
provide structures in which the main and child documents can be
encapsulated and allowing them to be compiled individually.
The inclusion mechanism is different from the conventional |\include|.
\item
The package \href{http://ctan.org/pkg/combine}{\textsf{combine}}
is an elaborate solution to combine several documents into one.
\end{itemize}
%
See also the CTAN topic \href{http://ctan.org/topic/subdocs}{\textsf{subdocs}}
for further related packages.
The present package differs from the above solutions in that
a document structure constructed with the conventional |\include| mechanism
just needs two extra commands at the top of every file
such that all constituent files can be compiled individually.

%%%%%%%%%%%%%%%%%%%%%%%%%%%%%%%%%%%%%%%%%%%%%%%%%%%%%%%%%%%%%%%%%%%%%%%%%%%%%%%%
%\subsection{Feature Suggestions}
%
%The following is a list of features which may be useful for future
%versions of this package:
%%
%\begin{itemize}
%\item
%\ldots
%\end{itemize}

%%%%%%%%%%%%%%%%%%%%%%%%%%%%%%%%%%%%%%%%%%%%%%%%%%%%%%%%%%%%%%%%%%%%%%%%%%%%%%%%
\subsection{Revision History}

%%%%%%%%%%%%%%%%%%%%%%%%%%%%%%%%%%%%%%%%
\paragraph{v2.0:} 2018/12/30

\begin{itemize}
\item
immediate forward processing
\item
added |\childdocby| mechanism
\item
manual restructured
\end{itemize}

%%%%%%%%%%%%%%%%%%%%%%%%%%%%%%%%%%%%%%%%
\paragraph{v1.6:} 2018/01/17

\begin{itemize}
\item
application for development of include files
\item
corrections to manual
\end{itemize}

%%%%%%%%%%%%%%%%%%%%%%%%%%%%%%%%%%%%%%%%
\paragraph{v1.5:} 2017/05/21

\begin{itemize}
\item
more complete structuring introduced
\item
|\childdocof| introduced
\item
|\childdoc| renamed to |\childdocmain|
\item
|\childredirect| renamed to |\childdocforward| and |\childdocforwardprefix|
and functionality expanded
\end{itemize}

%%%%%%%%%%%%%%%%%%%%%%%%%%%%%%%%%%%%%%%%
\paragraph{v1.0:} 2017/04/27

\begin{itemize}
\item
manual and install package
\item
first version published on CTAN
\end{itemize}

%%%%%%%%%%%%%%%%%%%%%%%%%%%%%%%%%%%%%%%%
\paragraph{v0.6:} 2017/04/26

\begin{itemize}
\item
redirection mechanism added
\end{itemize}

%%%%%%%%%%%%%%%%%%%%%%%%%%%%%%%%%%%%%%%%
\paragraph{v0.5:} 2017/04/26

\begin{itemize}
\item
functionality in definition file
\end{itemize}


%%%%%%%%%%%%%%%%%%%%%%%%%%%%%%%%%%%%%%%%%%%%%%%%%%%%%%%%%%%%%%%%%%%%%%%%%%%%%%%%
%%%%%%%%%%%%%%%%%%%%%%%%%%%%%%%%%%%%%%%%%%%%%%%%%%%%%%%%%%%%%%%%%%%%%%%%%%%%%%%%
%%%%%%%%%%%%%%%%%%%%%%%%%%%%%%%%%%%%%%%%%%%%%%%%%%%%%%%%%%%%%%%%%%%%%%%%%%%%%%%%
\appendix

\settowidth\MacroIndent{\rmfamily\scriptsize 000\ }

 \DocInput{childdoc.dtx}

\end{document}
%</driver>
% \fi
%
% %%%%%%%%%%%%%%%%%%%%%%%%%%%%%%%%%%%%%%%%%%%%%%%%%%%%%%%%%%%%%%%%%%%%%%%%%%%%%%
% %%%%%%%%%%%%%%%%%%%%%%%%%%%%%%%%%%%%%%%%%%%%%%%%%%%%%%%%%%%%%%%%%%%%%%%%%%%%%%
% \section{Sample}
%\iffalse
%<*samplemain>
%\fi
%
% The following presents a sample document
% with two chapters, two parts, a title page,
% a compile flag as well as three forwarding files to set the flag.
% It consists of eight |.tex| files:
% \begin{center}
% \begin{tabular}{ll}
% |cdocsamp.tex|&main file\\
% |cdocsch1.tex|&include file for chapter 1\\
% |cdocsch2.tex|&include file for chapter 2\\
% |cdocspt3.tex|&include file for part 3\\
% |cdocspt4.tex|&include file for part 4\\
% |cdocsdrf.tex|&forwarding file for main file in draft mode\\
% |cdocsfi1.tex|&forwarding file for final version of chapter 1\\
% |cdocsfi2.tex|&forwarding file for final version of chapter 2\\
% \end{tabular}
% \end{center}
% Each of the eight files can be compiled directly by the \LaTeX{} compiler.
%
% %%%%%%%%%%%%%%%%%%%%%%%%%%%%%%%%%%%%%%
% \paragraph{Main File.}
%
% The main file is called |cdocsamp.tex|.
%
% Load the \textsf{childdoc} definitions and
% declare the filename for the main document:
%    \begin{macrocode}
\input{childdoc.def}
\childdocmain{}
%    \end{macrocode}

% Optional override for |\version| flag:
%    \begin{macrocode}
%%\ifchilddoc\else\providecommand{\version}{draft}\fi
%    \end{macrocode}

% Define the default values for the |\version| flag
% (|final| for the main file and |draft| for childs):
%    \begin{macrocode}
\ifchilddoc
\providecommand{\version}{draft}
\else
\providecommand{\version}{final}
\fi
%    \end{macrocode}

% Load the standard document class:
%    \begin{macrocode}
\documentclass[12pt]{article}
%    \end{macrocode}

% Start the document body:
%    \begin{macrocode}
\begin{document}
%    \end{macrocode}

% Declare a title page.
% Print title, part of document being processed and version flag:
%    \begin{macrocode}
\addtocounter{page}{-1}
\begin{center}
{\LARGE\bfseries{}childdoc example\par}
\vspace{1cm}
\ifchilddoc
\ifchilddocmanual part\else chapter\fi:
`\childdocname' of `\childdocjob'\par
\else
main document: `\childdocjob'\par
\fi
version: \version\par
\end{center}
\newpage
%    \end{macrocode}

% Manually include selected file,
% otherwise process as usual:
%    \begin{macrocode}
\ifchilddocmanual
\section*{part `\childdocname'}
\input{\childdocname}
\else
%    \end{macrocode}

% Include the two chapters:
%    \begin{macrocode}
\include{cdocsch1}
\include{cdocsch2}
%    \end{macrocode}

% Include the two parts unless only chapters should be displayed:
%    \begin{macrocode}
\ifchilddoc\else
\section{part three}
\input{cdocspt3}
\section{part four}
\input{cdocspt4}
\fi
%    \end{macrocode}

% Process as usual until here:
%    \begin{macrocode}
\fi
%    \end{macrocode}

% End of document body:
%    \begin{macrocode}
\end{document}
%    \end{macrocode}
%\iffalse
%</samplemain>
%\fi
%
% %%%%%%%%%%%%%%%%%%%%%%%%%%%%%%%%%%%%%%
% \paragraph{Chapter Include Files.}
%
% The include files are called |cdocsch1.tex| and |cdocsch2.tex|.
%
%\iffalse
%<*samplechap1|samplechap2>
%\fi

% Optional override for |\version| flag:
%    \begin{macrocode}
%%\providecommand{\version}{final}
%    \end{macrocode}

% Include the main document:
%    \begin{macrocode}
\input{childdoc.def}
\childdocof{cdocsamp}
%    \end{macrocode}

%\iffalse
%</samplechap1|samplechap2>
%\fi
%
%\iffalse
%<*samplechap1>
%\fi
% Some text for chapter 1:
%    \begin{macrocode}
\section{one}
some text in chapter one
%    \end{macrocode}

%\iffalse
%</samplechap1>
%\fi
% Some text for chapter 2:
%\iffalse
%<*samplechap2>
%\fi
%    \begin{macrocode}
\section{two}
more text in chapter two
%    \end{macrocode}

%\iffalse
%</samplechap2>
%\fi
%
% %%%%%%%%%%%%%%%%%%%%%%%%%%%%%%%%%%%%%%
% \paragraph{Part Include Files.}
%
% The include files are called |cdocspt3.tex| and |cdocspt4.tex|.
%
%\iffalse
%<*samplepart3|samplepart4>
%\fi

% Optional override for |\version| flag:
%    \begin{macrocode}
%%\providecommand{\version}{final}
%    \end{macrocode}

% Include the main document:
%    \begin{macrocode}
\input{childdoc.def}
\childdocby{cdocsamp}
%    \end{macrocode}

%\iffalse
%</samplepart3|samplepart4>
%\fi
%
%\iffalse
%<*samplepart3>
%\fi
% Some text for part 3:
%    \begin{macrocode}
some text in part three
%    \end{macrocode}

%\iffalse
%</samplepart3>
%\fi
% Some text for part 4:
%\iffalse
%<*samplepart4>
%\fi
%    \begin{macrocode}
more text in part four
%    \end{macrocode}

%\iffalse
%</samplepart4>
%\fi
%
% %%%%%%%%%%%%%%%%%%%%%%%%%%%%%%%%%%%%%%
% \paragraph{Forwarding for a Complete Draft.}
%
% The following forwarding file |cdocsdrf.tex|
% compiles the main document in draft mode:
%\iffalse
%<*sampledraft>
%\fi
%    \begin{macrocode}
\def\version{draft}
\input{childdoc.def}
\childdocforward{cdocsamp}
%    \end{macrocode}

%\iffalse
%</sampledraft>
%\fi
%
% %%%%%%%%%%%%%%%%%%%%%%%%%%%%%%%%%%%%%%
% \paragraph{Forwarding for Final Version of the Chapters.}
%
% The following forwarding files |cdocsfn1.tex| and |cdocsfn2.tex|
% (with identical content)
% compile the final versions of the child documents
% |cdocsch1.tex| and |cdocsch2.tex|, respectively:
%\iffalse
%<*samplefinal>
%\fi
%    \begin{macrocode}
\def\version{final}
\input{childdoc.def}
\childdocforwardprefix[cdocsamp]{cdocsfn}{cdocsch}
%    \end{macrocode}

%\iffalse
%</samplefinal>
%\fi
%
% %%%%%%%%%%%%%%%%%%%%%%%%%%%%%%%%%%%%%%
% \paragraph{Command Line Processing.}
%
% The following three command lines generate the output files
% |cdocscld|, |cdocscl1| and |cdocscl2|
% which should be identical to
% |cdocsdrf|, |cdocsch1| and |cdocsfn2|, respectively:
% \begin{center}
% \begin{tabular}{l}
% |latex -jobname cdocscld \|\\
% |  "\def\version{draft}\input{childdoc.def}\childdocforward{cdocsamp}"|\\
% |latex -jobname cdocscl1 \|\\
% |  "\input{childdoc.def}\childdocforward[cdocsamp]{cdocsch1}"|\\
% |latex -jobname cdocscl2 \|\\
% |  "\def\version{final}\input{childdoc.def}\childdocforward{cdocsch2}"|
% \end{tabular}
% \end{center}
% Note that the trailing backslash on each first line
% merely continues the input to the second line
% (for convenient cut ant paste).
% Furthermore, the command |latex| can be replaced by any
% of its alternative versions such as |pdflatex|.
%
% %%%%%%%%%%%%%%%%%%%%%%%%%%%%%%%%%%%%%%%%%%%%%%%%%%%%%%%%%%%%%%%%%%%%%%%%%%%%%%
% %%%%%%%%%%%%%%%%%%%%%%%%%%%%%%%%%%%%%%%%%%%%%%%%%%%%%%%%%%%%%%%%%%%%%%%%%%%%%%
% \section{Implementation}
%\iffalse
%<*package>
%\fi
%
% This section describes the definitions file |childdoc.def|.

% The definitions cannot be loaded using |\usepackage| or |\RequirePackage|
% which has a mechanism to prevent loading a style file more than once.
% When loading the definitions by means of |\input|
% multiple instances have to be prevented manually:
%\iffalse
%This code needs to be before the `\ProvidesFile' directive
%which is defined at the beginning of this file.
%Therefore it is also placed there and commented out here.
%</package>
%<*discard>
%\fi
%    \begin{macrocode}
\ifdefined\childdocmain\endinput\fi
%    \end{macrocode}
%\iffalse
%</discard>
%<*package>
%\fi
%
% \macro{\ifchilddoc}
% \macro{\ifchilddocmanual}
% The conditional |\ifchilddoc| tells whether a
% child (true) or main (false) document is being compiled.
% The conditional |\ifchilddocmanual| tells whether
% the |\includeonly| mechanism is used (false) or
% the selection of child files must be performed manually (true).
% The definitions initialise to false:
%    \begin{macrocode}
\newif\ifchilddoc
\newif\ifchilddocmanual
%    \end{macrocode}

% \macro{\childdocname}
% \macro{\childdocjob}
% The macro |\childdocname| stores the name of the main document
% to be compiled. The macro |\childdocjob| stores the name of
% the document on which the \LaTeX{} compiler was originally invoked.
% The content of |\jobname| cannot be compared
% to filenames specified in the source due to different catcodes.
% The following code rescans |\jobname|, stores the result
% in |\childdocname| and saves a copy in |\childdocjob|:
%    \begin{macrocode}
\edef\childdocname{\scantokens\expandafter{\jobname\noexpand}}
\let\childdocjob\childdocname
%    \end{macrocode}

% \macro{\childdocdisable}
% The macro |\childdocdisable| prevents the main file
% from being processed more than once.
% At this stage, the main document command |\childdocmain|
% is assumed to be called once again where it should do nothing.
% Any subsequent call to it should prevent
% a secondary processing of the main document
% It overwrites the forwarding commands
% |\childdocof| and |\childdocforward|
% with empty macros to prevent further inclusions of the main document:
%    \begin{macrocode}
\newcommand{\childdocdisable}
{
  \renewcommand{\childdocmain}[1]{\renewcommand{\childdocmain}[1]{\endinput}}
  \renewcommand{\childdocof}[1]{}
  \renewcommand{\childdocby}[2][]{}
  \renewcommand{\childdocforward}[2][]{}
  \renewcommand{\childdocdisable}{}
}
%    \end{macrocode}

% \macro{\childdocmain}
% The macro |\childdocmain| is to be called at the top of the main file
% with nothing or the main filename (without extension) as argument.
% First, it breaks loops.
% If the argument is not empty and does not match |\childdocname|
% (which is set by the first inclusion of |childdoc.def|),
% |\ifchilddoc| is set to true, |\includeonly| is applied to the child file
% and |\jobname| is set to the main file
% (for proper handling of |.aux| files):
%    \begin{macrocode}
\newcommand{\childdocmain}[1]
{
  \childdocdisable\childdocmain{}
  \if?#1?\else
    \begingroup
      \def\childdoctmp{#1}
      \ifx\childdoctmp\childdocname
        \def\childdoctmp{}
      \else
        \def\childdoctmp
        {
          \childdoctrue
          \includeonly{\childdocname}
          \def\childdocjob{#1}
          \def\jobname{#1}
        }
      \fi
      \expandafter
    \endgroup
    \childdoctmp
  \fi
}
%    \end{macrocode}

% \macro{\childdocof}
% The command |\childdocof| redirects
% compilation to the main file |#1|.
%    \begin{macrocode}
\newcommand{\childdocof}[1]
{
  \childdocdisable
  \childdoctrue
  \includeonly{\childdocname}
  \def\jobname{#1}
  \def\childdocjob{#1}
  \input{#1}
}
%    \end{macrocode}

% \macro{\childdocby}
% The command |\childdocby| ....
%    \begin{macrocode}
\newcommand{\childdocby}[2][]
{
  \childdocdisable
  \childdoctrue
  \childdocmanualtrue
  \if?#1?\else
    \def\jobname{#2}
  \fi
  \def\childdocjob{#2}
  \input{#2}
  \endinput
}
%    \end{macrocode}

% \macro{\childdocforward}
% The command |\childdocforward| redirects
% compilation to the main file or
% (if the optional argument is given) a child file.
% Parameters are set as if the main file
% or a child file starting with |\childdocof| was compiled.
% Then compilation is handed over to the main file:
%    \begin{macrocode}
\newcommand{\childdocforward}[2][]
{
  \begingroup
    \if?#1?
      \def\childdoctmp
      {
        \def\childdocname{#2}
        \def\childdocjob{#2}
        \def\jobname{#2}
        \input{#2}
        \endinput
      }
    \else
      \def\childdoctmp
      {
        \childdocdisable
        \def\childdocname{#2}
        \childdoctrue
        \includeonly{#2}
        \def\childdocjob{#1}
        \def\jobname{#1}
        \input{#1}
        \endinput
      }
    \fi
    \expandafter
  \endgroup
  \childdoctmp
}
%    \end{macrocode}

% \macro{\childdocforwardprefix}
% The command |\childdocforwardprefix| redirects
% compilation to the main or a child file by means of a pattern.
% The prefix |#1| in the current filename is replaced by |#2|
% and the suffix of the current filename is kept
% (it is assumed that the filename does not contain the substring `|~~~|'
% which is used as a delimiter).
% Compilation is handed over to the new file by |\childdocforward|:
%    \begin{macrocode}
\newcommand{\childdocforwardprefix}[3][]
{
  \begingroup
    \def\childdocextract #2##1~~~{\def\childdoctmp{\childdocforward[#1]{#3##1}}}
    \expandafter\childdocextract\childdocname~~~
    \expandafter
  \endgroup
  \childdoctmp
}
%    \end{macrocode}

% \macro{\childdoc}
% The deprecated macro |\childdoc| is a legacy version of |\childdocmain|:
%    \begin{macrocode}
\newcommand{\childdoc}{\childdocmain}
%    \end{macrocode}

% \macro{\childdocredirect}
% The deprecated macro |\childdocredirect| is a legacy version
% of |\childdocforward| and |\childdocforwardprefix|:
%    \begin{macrocode}
\newcommand{\childdocredirect}[2][]
{
  \begingroup
    \if?#1?
      \def\childdoctmp{\childdocforward{#2}}
    \else
      \def\childdoctmp{\childdocforwardprefix{#1}{#2}}
    \fi
    \expandafter
  \endgroup
  \childdoctmp
}
%    \end{macrocode}

%\iffalse
%</package>
%\fi
%
\endinput
|\\
|\childdocby{|\textit{main}|}|\\
\end{tabular}
\end{center}
%
The directive |\childdocby| is similar to |\childdocof|
described in \secref{sec:include},
but the subsequent selection of content must be done manually.
To that end, both |\ifchilddoc| and |\ifchilddocmanual|
will be true upon processing of a part,
and the name of the part is stored in |\childdocname|.
Note that |\jobname| will be set to the filename of the current part
so that each part receives an individual |.aux| file
that does not interfere with the |.aux| file(s) of the main document.
This behaviour can be altered by the alternative form
|\childdocby[*]{|\textit{main}|}| (with a non-empty optional argument)
which uses the |.aux| file of the main document
by setting |\jobname| to \textit{main}.

%%%%%%%%%%%%%%%%%%%%%%%%%%%%%%%%%%%%%%%%%%%%%%%%%%%%%%%%%%%%%%%%%%%%%%%%%%%%%%%%
\subsection{Driver Development}
\label{sec:driver}

The \textsf{childdoc} mechanism can also be use for the development
of definition files such as \LaTeX{} styles or classes.
This case differs from the above setup with multiple parts
included by |\include| in that no |\includeonly| should be invoked.
This can be achieved by starting the include file
(before |\ProvidesPackage|) with:
%
\begin{center}
\begin{tabular}{l}
|% \iffalse
%
% childdoc.dtx Copyright (C) 2017-2018 Niklas Beisert
%
% This work may be distributed and/or modified under the
% conditions of the LaTeX Project Public License, either version 1.3
% of this license or (at your option) any later version.
% The latest version of this license is in
%   http://www.latex-project.org/lppl.txt
% and version 1.3 or later is part of all distributions of LaTeX
% version 2005/12/01 or later.
%
% This work has the LPPL maintenance status `maintained'.
%
% The Current Maintainer of this work is Niklas Beisert.
%
% This work consists of the files childdoc.dtx and childdoc.ins
% and the derived files childdoc.def and cdocsamp.tex with
% cdocsch1.tex, cdocsch2.tex, cdocsdrf.tex, cdocsfn1.tex, cdocsfn2.tex.
%
%<package>\ifdefined\childdocmain\endinput\fi
%<package>\ProvidesFile{childdoc.def}[2018/12/30 v2.0 child document driver]
%<samplemain>\ProvidesFile{cdocsamp.tex}[2018/12/30 v2.0 sample for childdoc]
%<*driver>
%\ProvidesFile{childdoc.drv}[2018/12/30 v2.0 childdoc reference manual file]
\PassOptionsToClass{10pt,a4paper}{article}
\documentclass{ltxdoc}

\usepackage[margin=35mm]{geometry}
\usepackage{hyperref}
\usepackage{hyperxmp}
\usepackage[usenames]{color}

\hypersetup{colorlinks=true}
\hypersetup{pdfstartview=FitH}
\hypersetup{pdfpagemode=UseNone}
\hypersetup{pdfsource={}}
\hypersetup{pdflang={en-UK}}
\hypersetup{pdfcopyright={Copyright 2017-2018 Niklas Beisert.
  This work may be distributed and/or modified under the
  conditions of the LaTeX Project Public License, either version 1.3
  of this license or (at your option) any later version.}}
\hypersetup{pdflicenseurl={http://www.latex-project.org/lppl.txt}}
\hypersetup{pdfcontactaddress={ETH Zurich, ITP, HIT K,
  Wolfgang-Pauli-Strasse 27}}
\hypersetup{pdfcontactpostcode={8093}}
\hypersetup{pdfcontactcity={Zurich}}
\hypersetup{pdfcontactcountry={Switzerland}}
\hypersetup{pdfcontactemail={nbeisert@itp.phys.ethz.ch}}
\hypersetup{pdfcontacturl={http://people.phys.ethz.ch/\xmptilde nbeisert/}}

\newcommand{\secref}[1]{\hyperref[#1]{section \ref*{#1}}}

\parskip1ex
\parindent0pt
\let\olditemize\itemize
\def\itemize{\olditemize\parskip0pt}

\begin{document}

\title{The \textsf{childdoc} Package}
\hypersetup{pdftitle={The childdoc Package}}
\author{Niklas Beisert\\[2ex]
  Institut f\"ur Theoretische Physik\\
  Eidgen\"ossische Technische Hochschule Z\"urich\\
  Wolfgang-Pauli-Strasse 27, 8093 Z\"urich, Switzerland\\[1ex]
  \href{mailto:nbeisert@itp.phys.ethz.ch}
  {\texttt{nbeisert@itp.phys.ethz.ch}}}
\hypersetup{pdfauthor={Niklas Beisert}}
\hypersetup{pdfsubject={Manual for the LaTeX2e Package childdoc}}
\date{30 December 2018, \textsf{v2.0}}
\maketitle

\begin{abstract}\noindent
\textsf{childdoc} is a \LaTeXe{} package
that enables the direct compilation
of document sections included by |\include|
to individual files.
\end{abstract}

\begingroup
\parskip0ex
\tableofcontents
\endgroup

%%%%%%%%%%%%%%%%%%%%%%%%%%%%%%%%%%%%%%%%%%%%%%%%%%%%%%%%%%%%%%%%%%%%%%%%%%%%%%%%
%%%%%%%%%%%%%%%%%%%%%%%%%%%%%%%%%%%%%%%%%%%%%%%%%%%%%%%%%%%%%%%%%%%%%%%%%%%%%%%%
\section{Introduction}

\LaTeX{} provides a mechanism to structure a large document (such as a book)
into a main file and several child files (containing the chapters)
using the |\include| command.
This mechanism is beneficial for documents
which span hundreds of pages in order to
make the source file(s) more manageable.
Moreover, compilation can be restricted to
selected child files by means of the |\includeonly| command.
The latter feature can be used to reduce the compilation time while editing
(this was significantly more useful in the earlier days of \LaTeX{})
or to generate a smaller document which is easier to navigate.
Another application of |\includeonly| is to generate
documents consisting of selected parts of the complete document.

However, there are a few drawbacks of the plain |\include| mechanism:
\begin{itemize}
\item
The child files cannot be compiled on their own,
they can only be compiled via the main file.
A naive editing environment
(such as a text editor with an option
to have the current file processed by \LaTeX)
may require one to switch to the main file before compiling;
attempting to compile the child file produces errors.
\item
The main file must be modified (each time)
to adjust the |\includeonly| command
to the present needs. This easily leaves the main file in a messy state.
\item
The generated document will always carry the filename
of the main document. This is inconvenient if
several child files are to be compiled and
to be kept for distribution.
\end{itemize}

The present package provides a simple interface
to make child files individually compilable by \LaTeX{}.
Compiling a child file then has the same effect as compiling
the main file with an |\includeonly| command
to select the appropriate child.
Moreover the generated document will carry the name of the child
rather than the main file.
This resolves all three above issues.

This feature is meant to make the editing of books,
thesis documents and lecture notes somewhat more convenient.
However, the package can also be used efficiently for
composing a series of documents (such as exercise sheets)
which are typically distributed individually.
It then assists the author in generating the individual documents
(potentially in different versions)
as well as a document containing the collected series.
Another application is in developing style files
or other kinds of included material
where compilation of the style file could redirect
to a sample or test file.

%%%%%%%%%%%%%%%%%%%%%%%%%%%%%%%%%%%%%%%%%%%%%%%%%%%%%%%%%%%%%%%%%%%%%%%%%%%%%%%%
%%%%%%%%%%%%%%%%%%%%%%%%%%%%%%%%%%%%%%%%%%%%%%%%%%%%%%%%%%%%%%%%%%%%%%%%%%%%%%%%
\section{Usage}

First of all, the package \textsf{childdoc} is \emph{not} a standard
\LaTeXe{} |.sty| style file! Therefore it needs to be invoked in
a non-standard way.

%%%%%%%%%%%%%%%%%%%%%%%%%%%%%%%%%%%%%%%%%%%%%%%%%%%%%%%%%%%%%%%%%%%%%%%%%%%%%%%%
\subsection{Included Files}
\label{sec:include}

%%%%%%%%%%%%%%%%%%%%%%%%%%%%%%%%%%%%%%%%
\DescribeMacro{\childdocmain}
To use the package, add the commands
\begin{center}
\begin{tabular}{l}
|\input{childdoc.def}|\\
|\childdocmain{}|\\
\end{tabular}
\end{center}
at the very top of the main \LaTeX{} file,
in particular \emph{before} the |\documentclass| statement!
The argument of |\childdocmain| should be left empty
(but it must be present).

%%%%%%%%%%%%%%%%%%%%%%%%%%%%%%%%%%%%%%%%
\DescribeMacro{\childdocof}
Furthermore, add the commands
\begin{center}
\begin{tabular}{l}
|\input{childdoc.def}|\\
|\childdocof{|\textit{main}|}|\\
\end{tabular}
\end{center}
at the top of every child file \textit{child}
which is included by |\include{|\textit{child}|}|
from within the main file
(or at least for those files to be compiled individually).
The argument \textit{main} must be the filename of the main file.

There are a couple of
considerations in setting up the main and child documents:

%%%%%%%%%%%%%%%%%%%%%%%%%%%%%%%%%%%%%%%%
\paragraph{Restrictions.}

Please note the following restrictions:
\begin{itemize}
\item
|\childdocmain| must be called with one argument \textit{main}
to ensure compatibility with earlier version of the package.
It must either be empty (|\childdocmain{}|)
or precisely match the filename of the main file in which it is specified.
See \secref{sec:detection} for further information.
\item
The filename \textit{main} must be specified without the |.tex| extension.
\item
The filename \textit{main} is case sensitive
(even in case-insensitive file systems)
due to internal string comparison.
\item
The argument \textit{main} should be fully expanded, it cannot be a macro.
\item
Subdirectories and special characters should be avoided in filenames.
\item
The command |\childdocmain{|\textit{main}|}| must be followed by a whitespace.
It should not be followed immediately by another command
or by a comment mark `|%|'.
This is because the \TeX{} parser reads the token immediately following
the argument of |\childdocmain| and puts it
at the beginning of every child section;
however, a white\-space is ignored.
\end{itemize}

%%%%%%%%%%%%%%%%%%%%%%%%%%%%%%%%%%%%%%%%
\paragraph{Content of Main File.}

It is advisable to place all content in the child files included by |\include|.
Any output contained in the main file will appear in all child documents
unless suppressed manually;
it cannot be suppressed automatically by the |\includeonly| directive
and thus should normally be avoided.
A method to include some content in the main file
by means of conditional processing is described in \secref{sec:conditional}.

%%%%%%%%%%%%%%%%%%%%%%%%%%%%%%%%%%%%%%%%
\paragraph{Page Numbering.}

When only a part of the document is compiled,
the appropriate numbering of pages
(as well as other status parameters)
is determined from the |.aux| files.
The latter contain information from previous passes.
However this information needs to propagate through
all intermediate child documents.
Therefore the page numbering in child documents may well
be inconsistent until the complete document is compiled at least once.

A useful (if unconventional) way to always ensure a consistent
page numbering is to restart the numbering in each child document
and denote the pages by `\textit{child}|.|\textit{page}'
where \textit{child} represents the chapter/section number of the child file.
This can be achieved by the command
|\numberwithin{page}{|\textit{child}|}|
of the \textsf{amsmath} package
where \textit{child} can be |chapter| or |section|
depending on the chosen structuring.
Alternatively, one can modify the macro |\thepage| appropriately
and reset the counter |page| at the start of each child file.

%%%%%%%%%%%%%%%%%%%%%%%%%%%%%%%%%%%%%%%%%%%%%%%%%%%%%%%%%%%%%%%%%%%%%%%%%%%%%%%%
\subsection{Conditional Processing}
\label{sec:conditional}

The package provides a mechanism to compile different versions
of a document. To customise the versions further some conditional processing
can come in handy to distinguish which version is being compiled.
The package provides two macros to describe the compilation context:

%%%%%%%%%%%%%%%%%%%%%%%%%%%%%%%%%%%%%%%%
\DescribeMacro{\ifchilddoc}
The conditional |\ifchilddoc| distinguishes between the compilation of
child documents and the main document:
%
\begin{center}
|\ifchilddoc |\textit{child-code}| |[|\||else |\textit{main-code}]| \||fi|
\end{center}

%%%%%%%%%%%%%%%%%%%%%%%%%%%%%%%%%%%%%%%%
\DescribeMacro{\childdocname}
\DescribeMacro{\childdocjob}
The macro |\childdocname| contains the filename (without extension)
of the main or child file being processed.
Note that |\childdocjob| will always contain the name of the main file.

%%%%%%%%%%%%%%%%%%%%%%%%%%%%%%%%%%%%%%%%
\paragraph{Title Page.}

Conditional processing can be used to include a title or banner page
in the main document when proper precautions are taken.
Importantly, the code in the main file should ensure that the page counter
(as well as other status parameters which are stored in the |.aux| files)
takes the same value after the conditional processing.
Otherwise the page numbers may take divergent values
depending on which part is compiled.

For example, a title page could be declared by:
%
\begin{center}
\begin{tabular}{l}
|\ifchilddoc\||else|\\
|\addtocounter{page}{-1}|\\
\textit{code for title page}\\
|\newpage|\\
|\||fi|
\end{tabular}
\end{center}
%
A banner page for the child documents can be generated by:
%
\begin{center}
\begin{tabular}{l}
|\ifchilddoc|\\
|\addtocounter{page}{-1}|\\
\textit{code for banner page}\\
|\newpage|\\
|\||fi|
\end{tabular}
\end{center}
%
Here one could write a message such as:
\begin{center}
|This is the part \childdocname{} of \childdocjob{}.|
\end{center}

%%%%%%%%%%%%%%%%%%%%%%%%%%%%%%%%%%%%%%%%%%%%%%%%%%%%%%%%%%%%%%%%%%%%%%%%%%%%%%%%
\subsection{Flags}
\label{sec:flags}

The package makes it easy to generate different versions
of the main or child documents.
To this end compilation flags can be defined
and assigned different default values.
They will be particularly useful in conjunction
with the forwarding mechanism described in \secref{sec:forward}.

For example, it may be useful to have a flag |\version|
which can be set to |draft| or |final|.
The document source will contain some conditional code
depending on the value of |\version|.
Suppose further, the flag should default to |final| for the main file
and to |draft| for child files
which is a natural assignment for editing the document.
This is achieved by placing the following code
in the preamble of the main document
(below the |\childdocmain| directive):
%
\begin{center}
\begin{tabular}{l}
|\ifchilddoc|\\
|\providecommand{\version}{draft}|\\
|\||else|\\
|\providecommand{\version}{final}|\\
|\||fi|
\end{tabular}
\end{center}
%
The definition by |\providecommand| makes sure
that previous definitions are not overwritten.
Further statements |\providecommand{\version}{...}|
can thus be added before the above code to override it.

For the main file, one might add a line
(between |\childdocmain| and the above block)
%
\begin{center}
|%\ifchilddoc\||else\providecommand{\version}{draft}\||fi|
\end{center}
%
which can be uncommented to produce a draft version.
Likewise one can add a line to the very top of a child file
(above the |\childdocof{|\textit{main}|}| directive)
%
\begin{center}
|%\providecommand{\version}{final}|
\end{center}
%
which can be uncommented to produce the final version of this child document.

%%%%%%%%%%%%%%%%%%%%%%%%%%%%%%%%%%%%%%%%%%%%%%%%%%%%%%%%%%%%%%%%%%%%%%%%%%%%%%%%
\subsection{Forwarding}
\label{sec:forward}

Different versions of the main or child documents
using compilation flags as described in \secref{sec:flags}
can be (permanently) stored in different files
for convenient compilation, viewing and distribution.
To this end, the package defines a command
to pass on compilation to a different file:

%%%%%%%%%%%%%%%%%%%%%%%%%%%%%%%%%%%%%%%%
\DescribeMacro{\childdocforward}
The command |\childdocforward| redirects processing to
another source file:
%
\begin{center}
\begin{tabular}{l}
|\input{childdoc.def}|\\
|\childdocforward[|\textit{main}|]{|\textit{dest}|}|\\
\end{tabular}
\end{center}
%
The argument \textit{dest} is the destination file
(without extension).
It should be the main file or one of the child files.
Note that further \textsf{childdoc} directives
such as |\childdocof| and |\childdocforward|
in the indicated file will be processed in this form.
The optional argument \textit{main}
passes on directly to the main file \textit{main}
while pretending to compile the child \textit{dest}.
This form behaves as if \textit{dest}
issues |\childdocof{|\textit{main}|}| right away,
and no further \textsf{childdoc} directives will be processed.

%%%%%%%%%%%%%%%%%%%%%%%%%%%%%%%%%%%%%%%%
\DescribeMacro{\...prefix}
In the alternative form |\childdocforwardprefix|,
%
\begin{center}
\begin{tabular}{l}
|\input{childdoc.def}|\\
|\childdocforwardprefix[|\textit{main}|]{|\textit{prefix}|}{|\textit{dest}|}|
\end{tabular}
\end{center}
%
the destination file is determined by a pattern
depending on the current file:
To make this work, the current file must be called
`{\textit{prefix}\hspace{0.2em}\textit{suffix}}'
with \textit{prefix} matching precisely the argument.
Processing is then passed on to the file
`{\textit{dest}\hspace{0.2em}\textit{suffix}}'.
Surely, the same effect is achieved by
directly specifying the
argument `{\textit{dest}\hspace{0.2em}\textit{suffix}}'
in the first form.
However, that requires to set up a different file
for each child. With the alternative form of the command
all these files can have exactly the same content
which simplifies setting them up and maintaining them.

For example, the following file |draft.tex|
with a compilation flag |\version| as described in \secref{sec:flags}
compiles the main document as a draft:
%
\begin{center}
\begin{tabular}{l}
|\def\version{draft}|\\
|\input{childdoc.def}|\\
|\childdocforward{|\textit{main}|}|
\end{tabular}
\end{center}
%
Likewise, the following files |final|\textit{nn}|.tex|
compile the final version of the child document
|child|\textit{nn}|.tex|:
%
\begin{center}
\begin{tabular}{l}
|\def\version{final}|\\
|\input{childdoc.def}|\\
|\childdocforwardprefix{final}{child}|
\end{tabular}
\end{center}
%

Note that when several versions of a main file and/or of each child file
are to be generated, it may be convenient to set up a |Makefile| or
shell script to automatise the process.

%%%%%%%%%%%%%%%%%%%%%%%%%%%%%%%%%%%%%%%%%%%%%%%%%%%%%%%%%%%%%%%%%%%%%%%%%%%%%%%%
\subsection{Command Line Processing}
\label{sec:commandline}

The effect of redirection files can also be achieved by invoking
the \LaTeX{} compiler with a more elaborate command line.
Most conveniently this should be done as part
of a shell script or a |Makefile|.

When using \textsf{childdoc} in the main file, the following
command lines effectively perform a redirection
(note that depending on the shell being used,
backslashes may have to be doubled: `|\|' $\to$ `|\\|'):
%
\begin{center}
|... -jobname "|\textit{target}|" |\\|"|[\textit{flags}]%
|\input{childdoc.def}\childdocforward[|\textit{main}|]{|\textit{dest}|}"|
\end{center}
%
Here \textit{target} is the name of the output file,
\textit{main} is the name of the main file
and \textit{dest} is the name of the main or child file to be processed
(all filenames without extensions).
The optional argument \textit{main} can be omitted
if \textit{main} matches \textit{dest}.
Optionally, compilation \textit{flags} can be defined via |\def| commands.
This command line makes the \TeX{} engine believe
it is compiling the file \textit{target}
whose content is specified as the latter parameter.
The provided code then forwards the processing to
\textit{main} or \textit{dest} as described in \secref{sec:forward}.

%%%%%%%%%%%%%%%%%%%%%%%%%%%%%%%%%%%%%%%%%%%%%%%%%%%%%%%%%%%%%%%%%%%%%%%%%%%%%%%%
\subsection{Include by Input}
\label{sec:input}

Including child documents by |\include| has some restrictions by design.
Most notably, the content of a child document always occupies
its own set of pages; pages cannot be shared between child documents.
Usually, this behaviour makes perfect sense
because each child document contain an essential part of the document.
However, in some situations it may be desirable to compose
a document from a collection of parts
without having mandatory page breaks between then.
For this case, the package
provides a mechanism to include parts
by |\input| which can also be processed individually.
However, by construction this mechanism
requires manual handling of the content to be output.

%%%%%%%%%%%%%%%%%%%%%%%%%%%%%%%%%%%%%%%%
\DescribeMacro{\ifchilddocmanual}
The main file should be prepared as usual, see \secref{sec:include}.
However, the document body must make a distinction
between processing of an individual part and of the main document, e.g.:
%
\begin{center}
\begin{tabular}{l}
|\ifchilddocmanual|\\
|\input{\childdocname}|\\
|\||else|\\
\textit{document body with }|\input{|\textit{part}|}|\\
|\||fi|
\end{tabular}
\end{center}
%
The conditional |\ifchilddocmanual| is true whenever
a part to be included by |\input| is being compiled,
and the name of the part is stored in |\childdocname|.

%%%%%%%%%%%%%%%%%%%%%%%%%%%%%%%%%%%%%%%%
\DescribeMacro{\childdocby}
Each part to be included by |\input| should start with:
%
\begin{center}
\begin{tabular}{l}
|\input{childdoc.def}|\\
|\childdocby{|\textit{main}|}|\\
\end{tabular}
\end{center}
%
The directive |\childdocby| is similar to |\childdocof|
described in \secref{sec:include},
but the subsequent selection of content must be done manually.
To that end, both |\ifchilddoc| and |\ifchilddocmanual|
will be true upon processing of a part,
and the name of the part is stored in |\childdocname|.
Note that |\jobname| will be set to the filename of the current part
so that each part receives an individual |.aux| file
that does not interfere with the |.aux| file(s) of the main document.
This behaviour can be altered by the alternative form
|\childdocby[*]{|\textit{main}|}| (with a non-empty optional argument)
which uses the |.aux| file of the main document
by setting |\jobname| to \textit{main}.

%%%%%%%%%%%%%%%%%%%%%%%%%%%%%%%%%%%%%%%%%%%%%%%%%%%%%%%%%%%%%%%%%%%%%%%%%%%%%%%%
\subsection{Driver Development}
\label{sec:driver}

The \textsf{childdoc} mechanism can also be use for the development
of definition files such as \LaTeX{} styles or classes.
This case differs from the above setup with multiple parts
included by |\include| in that no |\includeonly| should be invoked.
This can be achieved by starting the include file
(before |\ProvidesPackage|) with:
%
\begin{center}
\begin{tabular}{l}
|\input{childdoc.def}|\\
|\childdocforward{|\textit{main}|}|\\
\end{tabular}
\end{center}
%
or alternatively with:
%
\begin{center}
\begin{tabular}{l}
|\input{childdoc.def}|\\
|\childdocby{|\textit{main}|}|\\
\end{tabular}
\end{center}
%
Both forms have slightly different effects as described above.
The main file is prepared as usual, see \secref{sec:include}.

%%%%%%%%%%%%%%%%%%%%%%%%%%%%%%%%%%%%%%%%%%%%%%%%%%%%%%%%%%%%%%%%%%%%%%%%%%%%%%%%
\subsection{Legacy Detection}
\label{sec:detection}

The directive |\childdocmain| in the main file can detect
whether the complete document or merely a child is to be compiled
even without using the directive |\childdocof|.
This method is deprecated because it is less robust
and there is no compelling reason to use it;
it is merely provided for backward compatibility
and it may be removed in future versions.

If the detection mechanism is to be used,
it is mandatory to correctly specify
the filename of the main file as the argument of |\childdocmain|:
%
\begin{center}
\begin{tabular}{l}
|\input{childdoc.def}|\\
|\childdocmain{|\textit{main}|}|\\
\end{tabular}
\end{center}
%
If |\jobname| does not match the argument \textit{main} of |\childdocmain|,
it is assumed that |\jobname| points to the child file to be compiled.
When using |\childdocmain| with the main file specified as argument,
it suffices to start a child file
with just |\input{|\textit{main}|}|
without loading of the package and using |\childdocof|.
If instead all processing is done
with the appropriate \textsf{childdoc} directives,
the argument of \textit{main} of |\childdocmain| can be empty.

An alternative version of the command line processing described
in \secref{sec:commandline} using the detection mechanism reads:
%
\begin{center}
|... -jobname "|\textit{target}|" "|[\textit{flags}]%
[|\def\jobname{|\textit{dest}|}|]|\input{|\textit{main}|}"|
\end{center}

%%%%%%%%%%%%%%%%%%%%%%%%%%%%%%%%%%%%%%%%%%%%%%%%%%%%%%%%%%%%%%%%%%%%%%%%%%%%%%%%
\subsection{Manual Code}
\label{sec:manual}

In case one cannot be certain whether the definitions file |childdoc.def|
is installed on the target \TeX{} distribution
and one prefers not to ship it,
it is conceivable to paste a few relevant commands into the sources.

To that end, drop all statements |\input{childdoc.def}|
and perform the replacements as outlined below.
Instead of |\childdocmain{|\textit{main}|}| add the following code
to the top of the main file:
%
\begin{center}
\begin{tabular}{l}
|\||ifdefined\childdocname\endinput\||fi\newif\ifchilddoc|\\
|\edef\childdocname{\scantokens\expandafter{\jobname\noexpand}}|\\
|\def\childdocmain{|\textit{main}|}\||ifx\childdocmain\childdocname\||else|\\
|\childdoctrue\includeonly{\childdocname}\let\jobname\childdocmain\||fi|\\
\end{tabular}
\end{center}
%
Instead of |\childdocof{|\textit{main}|}| just include the main file
at the top of each child file:
%
\begin{center}
|\input{|\textit{main}|}|
\end{center}
%
A simple redirection |\childdocforward{|\textit{dest}|}| is achieved by:
%
\begin{center}
|\def\jobname{|\textit{dest}|}\input{\jobname}|
\end{center}
%
The redirection with prefix
|\childdocforwardprefix[|\textit{prefix}|]{|\textit{dest}|}|
is accomplished by:
%
\begin{center}
\begin{tabular}{l}
|{\edef\jobname{\scantokens\expandafter{\jobname\noexpand}}|\\
|\def\redirectjob |\textit{prefix}|#1~~~{\gdef\jobname{|\textit{dest}|#1}}|\\
|\expandafter\redirectjob\jobname~~~}\input{\jobname}|
\end{tabular}
\end{center}

In an alternative approach,
child documents can be compiled by a specific command line
without additional code or specific definitions:
%
\begin{center}
|... -jobname "|\textit{target}|" "|[\textit{flags}]%
|\includeonly{|\textit{dest}|}\input{|\textit{main}|}"|
\end{center}
%

%%%%%%%%%%%%%%%%%%%%%%%%%%%%%%%%%%%%%%%%%%%%%%%%%%%%%%%%%%%%%%%%%%%%%%%%%%%%%%%%
%%%%%%%%%%%%%%%%%%%%%%%%%%%%%%%%%%%%%%%%%%%%%%%%%%%%%%%%%%%%%%%%%%%%%%%%%%%%%%%%
\section{Information}

%%%%%%%%%%%%%%%%%%%%%%%%%%%%%%%%%%%%%%%%%%%%%%%%%%%%%%%%%%%%%%%%%%%%%%%%%%%%%%%%
\subsection{Copyright}

Copyright \copyright{} 2017--2018 Niklas Beisert

This work may be distributed and/or modified under the
conditions of the \LaTeX{} Project Public License, either version 1.3
of this license or (at your option) any later version.
The latest version of this license is in
  \url{http://www.latex-project.org/lppl.txt}
and version 1.3 or later is part of all distributions of \LaTeX{}
version 2005/12/01 or later.

This work has the LPPL maintenance status `maintained'.

The Current Maintainer of this work is Niklas Beisert.

This work consists of the files |README.txt|, |childdoc.ins| and |childdoc.dtx|
as well as the derived files |childdoc.def|, |cdocsamp.tex|
with |cdocsch1.tex|, |cdocsch2.tex|, |cdocspt3.tex|, |cdocspt4.tex|,
|cdocsdrf.tex|, |cdocsfn1.tex|, |cdocsfn2.tex|
as well as |childdoc.pdf|.

%%%%%%%%%%%%%%%%%%%%%%%%%%%%%%%%%%%%%%%%%%%%%%%%%%%%%%%%%%%%%%%%%%%%%%%%%%%%%%%%
\subsection{Files and Installation}

The package consists of the files:
%
\begin{center}
\begin{tabular}{ll}
    |README.txt|   & readme file \\
    |childdoc.ins| & installation file \\
    |childdoc.dtx| & source file \\
    |childdoc.def| & definition file \\
    |cdocsamp.tex| & sample main file \\
    |cdocsch1.tex| & sample include file \\
    |cdocsch2.tex| & sample include file \\
    |cdocspt3.tex| & sample part file \\
    |cdocspt4.tex| & sample part file \\
    |cdocsdrf.tex| & sample redirection file \\
    |cdocsfn1.tex| & sample redirection file \\
    |cdocsfn2.tex| & sample redirection file \\
    |childdoc.pdf| & manual
\end{tabular}
\end{center}
%
The distribution consists of the files
|README.txt|, |childdoc.ins| and |childdoc.dtx|.
%
\begin{itemize}
\item
Run (pdf)\LaTeX{} on |childdoc.dtx|
to compile the manual |childdoc.pdf| (this file).
\item
Run \LaTeX{} on |childdoc.ins| to create the definitions file |childdoc.def|
and the sample |cdocsamp.tex| with include files
|cdocsch1.tex|, |cdocsch2.tex|, |cdocspt3.tex|, |cdocspt4.tex|,
|cdocsdrf.tex|, |cdocsfn1.tex|, |cdocsfn2.tex|.
Then copy the file |childdoc.def| to an appropriate directory of your \LaTeX{}
distribution, e.g.\ \textit{texmf-root}|/tex/latex/childdoc|.
\end{itemize}

%%%%%%%%%%%%%%%%%%%%%%%%%%%%%%%%%%%%%%%%%%%%%%%%%%%%%%%%%%%%%%%%%%%%%%%%%%%%%%%%
\subsection{Related CTAN Packages}

There are several other packages which offer a similar functionality:
%
\begin{itemize}
\item
The packages
\href{http://ctan.org/pkg/docmute}{\textsf{docmute}},
\href{http://ctan.org/pkg/includex}{\textsf{includex}} and
\href{http://ctan.org/pkg/standalone}{\textsf{standalone}}
provide commands to include only the document body of
a child file thus allowing both files to be compiled individually.
\item
The packages \href{http://ctan.org/pkg/subdocs}{\textsf{subdocs}}
and \href{http://ctan.org/pkg/subfiles}{\textsf{subfiles}}
provide structures in which the main and child documents can be
encapsulated and allowing them to be compiled individually.
The inclusion mechanism is different from the conventional |\include|.
\item
The package \href{http://ctan.org/pkg/combine}{\textsf{combine}}
is an elaborate solution to combine several documents into one.
\end{itemize}
%
See also the CTAN topic \href{http://ctan.org/topic/subdocs}{\textsf{subdocs}}
for further related packages.
The present package differs from the above solutions in that
a document structure constructed with the conventional |\include| mechanism
just needs two extra commands at the top of every file
such that all constituent files can be compiled individually.

%%%%%%%%%%%%%%%%%%%%%%%%%%%%%%%%%%%%%%%%%%%%%%%%%%%%%%%%%%%%%%%%%%%%%%%%%%%%%%%%
%\subsection{Feature Suggestions}
%
%The following is a list of features which may be useful for future
%versions of this package:
%%
%\begin{itemize}
%\item
%\ldots
%\end{itemize}

%%%%%%%%%%%%%%%%%%%%%%%%%%%%%%%%%%%%%%%%%%%%%%%%%%%%%%%%%%%%%%%%%%%%%%%%%%%%%%%%
\subsection{Revision History}

%%%%%%%%%%%%%%%%%%%%%%%%%%%%%%%%%%%%%%%%
\paragraph{v2.0:} 2018/12/30

\begin{itemize}
\item
immediate forward processing
\item
added |\childdocby| mechanism
\item
manual restructured
\end{itemize}

%%%%%%%%%%%%%%%%%%%%%%%%%%%%%%%%%%%%%%%%
\paragraph{v1.6:} 2018/01/17

\begin{itemize}
\item
application for development of include files
\item
corrections to manual
\end{itemize}

%%%%%%%%%%%%%%%%%%%%%%%%%%%%%%%%%%%%%%%%
\paragraph{v1.5:} 2017/05/21

\begin{itemize}
\item
more complete structuring introduced
\item
|\childdocof| introduced
\item
|\childdoc| renamed to |\childdocmain|
\item
|\childredirect| renamed to |\childdocforward| and |\childdocforwardprefix|
and functionality expanded
\end{itemize}

%%%%%%%%%%%%%%%%%%%%%%%%%%%%%%%%%%%%%%%%
\paragraph{v1.0:} 2017/04/27

\begin{itemize}
\item
manual and install package
\item
first version published on CTAN
\end{itemize}

%%%%%%%%%%%%%%%%%%%%%%%%%%%%%%%%%%%%%%%%
\paragraph{v0.6:} 2017/04/26

\begin{itemize}
\item
redirection mechanism added
\end{itemize}

%%%%%%%%%%%%%%%%%%%%%%%%%%%%%%%%%%%%%%%%
\paragraph{v0.5:} 2017/04/26

\begin{itemize}
\item
functionality in definition file
\end{itemize}


%%%%%%%%%%%%%%%%%%%%%%%%%%%%%%%%%%%%%%%%%%%%%%%%%%%%%%%%%%%%%%%%%%%%%%%%%%%%%%%%
%%%%%%%%%%%%%%%%%%%%%%%%%%%%%%%%%%%%%%%%%%%%%%%%%%%%%%%%%%%%%%%%%%%%%%%%%%%%%%%%
%%%%%%%%%%%%%%%%%%%%%%%%%%%%%%%%%%%%%%%%%%%%%%%%%%%%%%%%%%%%%%%%%%%%%%%%%%%%%%%%
\appendix

\settowidth\MacroIndent{\rmfamily\scriptsize 000\ }

 \DocInput{childdoc.dtx}

\end{document}
%</driver>
% \fi
%
% %%%%%%%%%%%%%%%%%%%%%%%%%%%%%%%%%%%%%%%%%%%%%%%%%%%%%%%%%%%%%%%%%%%%%%%%%%%%%%
% %%%%%%%%%%%%%%%%%%%%%%%%%%%%%%%%%%%%%%%%%%%%%%%%%%%%%%%%%%%%%%%%%%%%%%%%%%%%%%
% \section{Sample}
%\iffalse
%<*samplemain>
%\fi
%
% The following presents a sample document
% with two chapters, two parts, a title page,
% a compile flag as well as three forwarding files to set the flag.
% It consists of eight |.tex| files:
% \begin{center}
% \begin{tabular}{ll}
% |cdocsamp.tex|&main file\\
% |cdocsch1.tex|&include file for chapter 1\\
% |cdocsch2.tex|&include file for chapter 2\\
% |cdocspt3.tex|&include file for part 3\\
% |cdocspt4.tex|&include file for part 4\\
% |cdocsdrf.tex|&forwarding file for main file in draft mode\\
% |cdocsfi1.tex|&forwarding file for final version of chapter 1\\
% |cdocsfi2.tex|&forwarding file for final version of chapter 2\\
% \end{tabular}
% \end{center}
% Each of the eight files can be compiled directly by the \LaTeX{} compiler.
%
% %%%%%%%%%%%%%%%%%%%%%%%%%%%%%%%%%%%%%%
% \paragraph{Main File.}
%
% The main file is called |cdocsamp.tex|.
%
% Load the \textsf{childdoc} definitions and
% declare the filename for the main document:
%    \begin{macrocode}
\input{childdoc.def}
\childdocmain{}
%    \end{macrocode}

% Optional override for |\version| flag:
%    \begin{macrocode}
%%\ifchilddoc\else\providecommand{\version}{draft}\fi
%    \end{macrocode}

% Define the default values for the |\version| flag
% (|final| for the main file and |draft| for childs):
%    \begin{macrocode}
\ifchilddoc
\providecommand{\version}{draft}
\else
\providecommand{\version}{final}
\fi
%    \end{macrocode}

% Load the standard document class:
%    \begin{macrocode}
\documentclass[12pt]{article}
%    \end{macrocode}

% Start the document body:
%    \begin{macrocode}
\begin{document}
%    \end{macrocode}

% Declare a title page.
% Print title, part of document being processed and version flag:
%    \begin{macrocode}
\addtocounter{page}{-1}
\begin{center}
{\LARGE\bfseries{}childdoc example\par}
\vspace{1cm}
\ifchilddoc
\ifchilddocmanual part\else chapter\fi:
`\childdocname' of `\childdocjob'\par
\else
main document: `\childdocjob'\par
\fi
version: \version\par
\end{center}
\newpage
%    \end{macrocode}

% Manually include selected file,
% otherwise process as usual:
%    \begin{macrocode}
\ifchilddocmanual
\section*{part `\childdocname'}
\input{\childdocname}
\else
%    \end{macrocode}

% Include the two chapters:
%    \begin{macrocode}
\include{cdocsch1}
\include{cdocsch2}
%    \end{macrocode}

% Include the two parts unless only chapters should be displayed:
%    \begin{macrocode}
\ifchilddoc\else
\section{part three}
\input{cdocspt3}
\section{part four}
\input{cdocspt4}
\fi
%    \end{macrocode}

% Process as usual until here:
%    \begin{macrocode}
\fi
%    \end{macrocode}

% End of document body:
%    \begin{macrocode}
\end{document}
%    \end{macrocode}
%\iffalse
%</samplemain>
%\fi
%
% %%%%%%%%%%%%%%%%%%%%%%%%%%%%%%%%%%%%%%
% \paragraph{Chapter Include Files.}
%
% The include files are called |cdocsch1.tex| and |cdocsch2.tex|.
%
%\iffalse
%<*samplechap1|samplechap2>
%\fi

% Optional override for |\version| flag:
%    \begin{macrocode}
%%\providecommand{\version}{final}
%    \end{macrocode}

% Include the main document:
%    \begin{macrocode}
\input{childdoc.def}
\childdocof{cdocsamp}
%    \end{macrocode}

%\iffalse
%</samplechap1|samplechap2>
%\fi
%
%\iffalse
%<*samplechap1>
%\fi
% Some text for chapter 1:
%    \begin{macrocode}
\section{one}
some text in chapter one
%    \end{macrocode}

%\iffalse
%</samplechap1>
%\fi
% Some text for chapter 2:
%\iffalse
%<*samplechap2>
%\fi
%    \begin{macrocode}
\section{two}
more text in chapter two
%    \end{macrocode}

%\iffalse
%</samplechap2>
%\fi
%
% %%%%%%%%%%%%%%%%%%%%%%%%%%%%%%%%%%%%%%
% \paragraph{Part Include Files.}
%
% The include files are called |cdocspt3.tex| and |cdocspt4.tex|.
%
%\iffalse
%<*samplepart3|samplepart4>
%\fi

% Optional override for |\version| flag:
%    \begin{macrocode}
%%\providecommand{\version}{final}
%    \end{macrocode}

% Include the main document:
%    \begin{macrocode}
\input{childdoc.def}
\childdocby{cdocsamp}
%    \end{macrocode}

%\iffalse
%</samplepart3|samplepart4>
%\fi
%
%\iffalse
%<*samplepart3>
%\fi
% Some text for part 3:
%    \begin{macrocode}
some text in part three
%    \end{macrocode}

%\iffalse
%</samplepart3>
%\fi
% Some text for part 4:
%\iffalse
%<*samplepart4>
%\fi
%    \begin{macrocode}
more text in part four
%    \end{macrocode}

%\iffalse
%</samplepart4>
%\fi
%
% %%%%%%%%%%%%%%%%%%%%%%%%%%%%%%%%%%%%%%
% \paragraph{Forwarding for a Complete Draft.}
%
% The following forwarding file |cdocsdrf.tex|
% compiles the main document in draft mode:
%\iffalse
%<*sampledraft>
%\fi
%    \begin{macrocode}
\def\version{draft}
\input{childdoc.def}
\childdocforward{cdocsamp}
%    \end{macrocode}

%\iffalse
%</sampledraft>
%\fi
%
% %%%%%%%%%%%%%%%%%%%%%%%%%%%%%%%%%%%%%%
% \paragraph{Forwarding for Final Version of the Chapters.}
%
% The following forwarding files |cdocsfn1.tex| and |cdocsfn2.tex|
% (with identical content)
% compile the final versions of the child documents
% |cdocsch1.tex| and |cdocsch2.tex|, respectively:
%\iffalse
%<*samplefinal>
%\fi
%    \begin{macrocode}
\def\version{final}
\input{childdoc.def}
\childdocforwardprefix[cdocsamp]{cdocsfn}{cdocsch}
%    \end{macrocode}

%\iffalse
%</samplefinal>
%\fi
%
% %%%%%%%%%%%%%%%%%%%%%%%%%%%%%%%%%%%%%%
% \paragraph{Command Line Processing.}
%
% The following three command lines generate the output files
% |cdocscld|, |cdocscl1| and |cdocscl2|
% which should be identical to
% |cdocsdrf|, |cdocsch1| and |cdocsfn2|, respectively:
% \begin{center}
% \begin{tabular}{l}
% |latex -jobname cdocscld \|\\
% |  "\def\version{draft}\input{childdoc.def}\childdocforward{cdocsamp}"|\\
% |latex -jobname cdocscl1 \|\\
% |  "\input{childdoc.def}\childdocforward[cdocsamp]{cdocsch1}"|\\
% |latex -jobname cdocscl2 \|\\
% |  "\def\version{final}\input{childdoc.def}\childdocforward{cdocsch2}"|
% \end{tabular}
% \end{center}
% Note that the trailing backslash on each first line
% merely continues the input to the second line
% (for convenient cut ant paste).
% Furthermore, the command |latex| can be replaced by any
% of its alternative versions such as |pdflatex|.
%
% %%%%%%%%%%%%%%%%%%%%%%%%%%%%%%%%%%%%%%%%%%%%%%%%%%%%%%%%%%%%%%%%%%%%%%%%%%%%%%
% %%%%%%%%%%%%%%%%%%%%%%%%%%%%%%%%%%%%%%%%%%%%%%%%%%%%%%%%%%%%%%%%%%%%%%%%%%%%%%
% \section{Implementation}
%\iffalse
%<*package>
%\fi
%
% This section describes the definitions file |childdoc.def|.

% The definitions cannot be loaded using |\usepackage| or |\RequirePackage|
% which has a mechanism to prevent loading a style file more than once.
% When loading the definitions by means of |\input|
% multiple instances have to be prevented manually:
%\iffalse
%This code needs to be before the `\ProvidesFile' directive
%which is defined at the beginning of this file.
%Therefore it is also placed there and commented out here.
%</package>
%<*discard>
%\fi
%    \begin{macrocode}
\ifdefined\childdocmain\endinput\fi
%    \end{macrocode}
%\iffalse
%</discard>
%<*package>
%\fi
%
% \macro{\ifchilddoc}
% \macro{\ifchilddocmanual}
% The conditional |\ifchilddoc| tells whether a
% child (true) or main (false) document is being compiled.
% The conditional |\ifchilddocmanual| tells whether
% the |\includeonly| mechanism is used (false) or
% the selection of child files must be performed manually (true).
% The definitions initialise to false:
%    \begin{macrocode}
\newif\ifchilddoc
\newif\ifchilddocmanual
%    \end{macrocode}

% \macro{\childdocname}
% \macro{\childdocjob}
% The macro |\childdocname| stores the name of the main document
% to be compiled. The macro |\childdocjob| stores the name of
% the document on which the \LaTeX{} compiler was originally invoked.
% The content of |\jobname| cannot be compared
% to filenames specified in the source due to different catcodes.
% The following code rescans |\jobname|, stores the result
% in |\childdocname| and saves a copy in |\childdocjob|:
%    \begin{macrocode}
\edef\childdocname{\scantokens\expandafter{\jobname\noexpand}}
\let\childdocjob\childdocname
%    \end{macrocode}

% \macro{\childdocdisable}
% The macro |\childdocdisable| prevents the main file
% from being processed more than once.
% At this stage, the main document command |\childdocmain|
% is assumed to be called once again where it should do nothing.
% Any subsequent call to it should prevent
% a secondary processing of the main document
% It overwrites the forwarding commands
% |\childdocof| and |\childdocforward|
% with empty macros to prevent further inclusions of the main document:
%    \begin{macrocode}
\newcommand{\childdocdisable}
{
  \renewcommand{\childdocmain}[1]{\renewcommand{\childdocmain}[1]{\endinput}}
  \renewcommand{\childdocof}[1]{}
  \renewcommand{\childdocby}[2][]{}
  \renewcommand{\childdocforward}[2][]{}
  \renewcommand{\childdocdisable}{}
}
%    \end{macrocode}

% \macro{\childdocmain}
% The macro |\childdocmain| is to be called at the top of the main file
% with nothing or the main filename (without extension) as argument.
% First, it breaks loops.
% If the argument is not empty and does not match |\childdocname|
% (which is set by the first inclusion of |childdoc.def|),
% |\ifchilddoc| is set to true, |\includeonly| is applied to the child file
% and |\jobname| is set to the main file
% (for proper handling of |.aux| files):
%    \begin{macrocode}
\newcommand{\childdocmain}[1]
{
  \childdocdisable\childdocmain{}
  \if?#1?\else
    \begingroup
      \def\childdoctmp{#1}
      \ifx\childdoctmp\childdocname
        \def\childdoctmp{}
      \else
        \def\childdoctmp
        {
          \childdoctrue
          \includeonly{\childdocname}
          \def\childdocjob{#1}
          \def\jobname{#1}
        }
      \fi
      \expandafter
    \endgroup
    \childdoctmp
  \fi
}
%    \end{macrocode}

% \macro{\childdocof}
% The command |\childdocof| redirects
% compilation to the main file |#1|.
%    \begin{macrocode}
\newcommand{\childdocof}[1]
{
  \childdocdisable
  \childdoctrue
  \includeonly{\childdocname}
  \def\jobname{#1}
  \def\childdocjob{#1}
  \input{#1}
}
%    \end{macrocode}

% \macro{\childdocby}
% The command |\childdocby| ....
%    \begin{macrocode}
\newcommand{\childdocby}[2][]
{
  \childdocdisable
  \childdoctrue
  \childdocmanualtrue
  \if?#1?\else
    \def\jobname{#2}
  \fi
  \def\childdocjob{#2}
  \input{#2}
  \endinput
}
%    \end{macrocode}

% \macro{\childdocforward}
% The command |\childdocforward| redirects
% compilation to the main file or
% (if the optional argument is given) a child file.
% Parameters are set as if the main file
% or a child file starting with |\childdocof| was compiled.
% Then compilation is handed over to the main file:
%    \begin{macrocode}
\newcommand{\childdocforward}[2][]
{
  \begingroup
    \if?#1?
      \def\childdoctmp
      {
        \def\childdocname{#2}
        \def\childdocjob{#2}
        \def\jobname{#2}
        \input{#2}
        \endinput
      }
    \else
      \def\childdoctmp
      {
        \childdocdisable
        \def\childdocname{#2}
        \childdoctrue
        \includeonly{#2}
        \def\childdocjob{#1}
        \def\jobname{#1}
        \input{#1}
        \endinput
      }
    \fi
    \expandafter
  \endgroup
  \childdoctmp
}
%    \end{macrocode}

% \macro{\childdocforwardprefix}
% The command |\childdocforwardprefix| redirects
% compilation to the main or a child file by means of a pattern.
% The prefix |#1| in the current filename is replaced by |#2|
% and the suffix of the current filename is kept
% (it is assumed that the filename does not contain the substring `|~~~|'
% which is used as a delimiter).
% Compilation is handed over to the new file by |\childdocforward|:
%    \begin{macrocode}
\newcommand{\childdocforwardprefix}[3][]
{
  \begingroup
    \def\childdocextract #2##1~~~{\def\childdoctmp{\childdocforward[#1]{#3##1}}}
    \expandafter\childdocextract\childdocname~~~
    \expandafter
  \endgroup
  \childdoctmp
}
%    \end{macrocode}

% \macro{\childdoc}
% The deprecated macro |\childdoc| is a legacy version of |\childdocmain|:
%    \begin{macrocode}
\newcommand{\childdoc}{\childdocmain}
%    \end{macrocode}

% \macro{\childdocredirect}
% The deprecated macro |\childdocredirect| is a legacy version
% of |\childdocforward| and |\childdocforwardprefix|:
%    \begin{macrocode}
\newcommand{\childdocredirect}[2][]
{
  \begingroup
    \if?#1?
      \def\childdoctmp{\childdocforward{#2}}
    \else
      \def\childdoctmp{\childdocforwardprefix{#1}{#2}}
    \fi
    \expandafter
  \endgroup
  \childdoctmp
}
%    \end{macrocode}

%\iffalse
%</package>
%\fi
%
\endinput
|\\
|\childdocforward{|\textit{main}|}|\\
\end{tabular}
\end{center}
%
or alternatively with:
%
\begin{center}
\begin{tabular}{l}
|% \iffalse
%
% childdoc.dtx Copyright (C) 2017-2018 Niklas Beisert
%
% This work may be distributed and/or modified under the
% conditions of the LaTeX Project Public License, either version 1.3
% of this license or (at your option) any later version.
% The latest version of this license is in
%   http://www.latex-project.org/lppl.txt
% and version 1.3 or later is part of all distributions of LaTeX
% version 2005/12/01 or later.
%
% This work has the LPPL maintenance status `maintained'.
%
% The Current Maintainer of this work is Niklas Beisert.
%
% This work consists of the files childdoc.dtx and childdoc.ins
% and the derived files childdoc.def and cdocsamp.tex with
% cdocsch1.tex, cdocsch2.tex, cdocsdrf.tex, cdocsfn1.tex, cdocsfn2.tex.
%
%<package>\ifdefined\childdocmain\endinput\fi
%<package>\ProvidesFile{childdoc.def}[2018/12/30 v2.0 child document driver]
%<samplemain>\ProvidesFile{cdocsamp.tex}[2018/12/30 v2.0 sample for childdoc]
%<*driver>
%\ProvidesFile{childdoc.drv}[2018/12/30 v2.0 childdoc reference manual file]
\PassOptionsToClass{10pt,a4paper}{article}
\documentclass{ltxdoc}

\usepackage[margin=35mm]{geometry}
\usepackage{hyperref}
\usepackage{hyperxmp}
\usepackage[usenames]{color}

\hypersetup{colorlinks=true}
\hypersetup{pdfstartview=FitH}
\hypersetup{pdfpagemode=UseNone}
\hypersetup{pdfsource={}}
\hypersetup{pdflang={en-UK}}
\hypersetup{pdfcopyright={Copyright 2017-2018 Niklas Beisert.
  This work may be distributed and/or modified under the
  conditions of the LaTeX Project Public License, either version 1.3
  of this license or (at your option) any later version.}}
\hypersetup{pdflicenseurl={http://www.latex-project.org/lppl.txt}}
\hypersetup{pdfcontactaddress={ETH Zurich, ITP, HIT K,
  Wolfgang-Pauli-Strasse 27}}
\hypersetup{pdfcontactpostcode={8093}}
\hypersetup{pdfcontactcity={Zurich}}
\hypersetup{pdfcontactcountry={Switzerland}}
\hypersetup{pdfcontactemail={nbeisert@itp.phys.ethz.ch}}
\hypersetup{pdfcontacturl={http://people.phys.ethz.ch/\xmptilde nbeisert/}}

\newcommand{\secref}[1]{\hyperref[#1]{section \ref*{#1}}}

\parskip1ex
\parindent0pt
\let\olditemize\itemize
\def\itemize{\olditemize\parskip0pt}

\begin{document}

\title{The \textsf{childdoc} Package}
\hypersetup{pdftitle={The childdoc Package}}
\author{Niklas Beisert\\[2ex]
  Institut f\"ur Theoretische Physik\\
  Eidgen\"ossische Technische Hochschule Z\"urich\\
  Wolfgang-Pauli-Strasse 27, 8093 Z\"urich, Switzerland\\[1ex]
  \href{mailto:nbeisert@itp.phys.ethz.ch}
  {\texttt{nbeisert@itp.phys.ethz.ch}}}
\hypersetup{pdfauthor={Niklas Beisert}}
\hypersetup{pdfsubject={Manual for the LaTeX2e Package childdoc}}
\date{30 December 2018, \textsf{v2.0}}
\maketitle

\begin{abstract}\noindent
\textsf{childdoc} is a \LaTeXe{} package
that enables the direct compilation
of document sections included by |\include|
to individual files.
\end{abstract}

\begingroup
\parskip0ex
\tableofcontents
\endgroup

%%%%%%%%%%%%%%%%%%%%%%%%%%%%%%%%%%%%%%%%%%%%%%%%%%%%%%%%%%%%%%%%%%%%%%%%%%%%%%%%
%%%%%%%%%%%%%%%%%%%%%%%%%%%%%%%%%%%%%%%%%%%%%%%%%%%%%%%%%%%%%%%%%%%%%%%%%%%%%%%%
\section{Introduction}

\LaTeX{} provides a mechanism to structure a large document (such as a book)
into a main file and several child files (containing the chapters)
using the |\include| command.
This mechanism is beneficial for documents
which span hundreds of pages in order to
make the source file(s) more manageable.
Moreover, compilation can be restricted to
selected child files by means of the |\includeonly| command.
The latter feature can be used to reduce the compilation time while editing
(this was significantly more useful in the earlier days of \LaTeX{})
or to generate a smaller document which is easier to navigate.
Another application of |\includeonly| is to generate
documents consisting of selected parts of the complete document.

However, there are a few drawbacks of the plain |\include| mechanism:
\begin{itemize}
\item
The child files cannot be compiled on their own,
they can only be compiled via the main file.
A naive editing environment
(such as a text editor with an option
to have the current file processed by \LaTeX)
may require one to switch to the main file before compiling;
attempting to compile the child file produces errors.
\item
The main file must be modified (each time)
to adjust the |\includeonly| command
to the present needs. This easily leaves the main file in a messy state.
\item
The generated document will always carry the filename
of the main document. This is inconvenient if
several child files are to be compiled and
to be kept for distribution.
\end{itemize}

The present package provides a simple interface
to make child files individually compilable by \LaTeX{}.
Compiling a child file then has the same effect as compiling
the main file with an |\includeonly| command
to select the appropriate child.
Moreover the generated document will carry the name of the child
rather than the main file.
This resolves all three above issues.

This feature is meant to make the editing of books,
thesis documents and lecture notes somewhat more convenient.
However, the package can also be used efficiently for
composing a series of documents (such as exercise sheets)
which are typically distributed individually.
It then assists the author in generating the individual documents
(potentially in different versions)
as well as a document containing the collected series.
Another application is in developing style files
or other kinds of included material
where compilation of the style file could redirect
to a sample or test file.

%%%%%%%%%%%%%%%%%%%%%%%%%%%%%%%%%%%%%%%%%%%%%%%%%%%%%%%%%%%%%%%%%%%%%%%%%%%%%%%%
%%%%%%%%%%%%%%%%%%%%%%%%%%%%%%%%%%%%%%%%%%%%%%%%%%%%%%%%%%%%%%%%%%%%%%%%%%%%%%%%
\section{Usage}

First of all, the package \textsf{childdoc} is \emph{not} a standard
\LaTeXe{} |.sty| style file! Therefore it needs to be invoked in
a non-standard way.

%%%%%%%%%%%%%%%%%%%%%%%%%%%%%%%%%%%%%%%%%%%%%%%%%%%%%%%%%%%%%%%%%%%%%%%%%%%%%%%%
\subsection{Included Files}
\label{sec:include}

%%%%%%%%%%%%%%%%%%%%%%%%%%%%%%%%%%%%%%%%
\DescribeMacro{\childdocmain}
To use the package, add the commands
\begin{center}
\begin{tabular}{l}
|\input{childdoc.def}|\\
|\childdocmain{}|\\
\end{tabular}
\end{center}
at the very top of the main \LaTeX{} file,
in particular \emph{before} the |\documentclass| statement!
The argument of |\childdocmain| should be left empty
(but it must be present).

%%%%%%%%%%%%%%%%%%%%%%%%%%%%%%%%%%%%%%%%
\DescribeMacro{\childdocof}
Furthermore, add the commands
\begin{center}
\begin{tabular}{l}
|\input{childdoc.def}|\\
|\childdocof{|\textit{main}|}|\\
\end{tabular}
\end{center}
at the top of every child file \textit{child}
which is included by |\include{|\textit{child}|}|
from within the main file
(or at least for those files to be compiled individually).
The argument \textit{main} must be the filename of the main file.

There are a couple of
considerations in setting up the main and child documents:

%%%%%%%%%%%%%%%%%%%%%%%%%%%%%%%%%%%%%%%%
\paragraph{Restrictions.}

Please note the following restrictions:
\begin{itemize}
\item
|\childdocmain| must be called with one argument \textit{main}
to ensure compatibility with earlier version of the package.
It must either be empty (|\childdocmain{}|)
or precisely match the filename of the main file in which it is specified.
See \secref{sec:detection} for further information.
\item
The filename \textit{main} must be specified without the |.tex| extension.
\item
The filename \textit{main} is case sensitive
(even in case-insensitive file systems)
due to internal string comparison.
\item
The argument \textit{main} should be fully expanded, it cannot be a macro.
\item
Subdirectories and special characters should be avoided in filenames.
\item
The command |\childdocmain{|\textit{main}|}| must be followed by a whitespace.
It should not be followed immediately by another command
or by a comment mark `|%|'.
This is because the \TeX{} parser reads the token immediately following
the argument of |\childdocmain| and puts it
at the beginning of every child section;
however, a white\-space is ignored.
\end{itemize}

%%%%%%%%%%%%%%%%%%%%%%%%%%%%%%%%%%%%%%%%
\paragraph{Content of Main File.}

It is advisable to place all content in the child files included by |\include|.
Any output contained in the main file will appear in all child documents
unless suppressed manually;
it cannot be suppressed automatically by the |\includeonly| directive
and thus should normally be avoided.
A method to include some content in the main file
by means of conditional processing is described in \secref{sec:conditional}.

%%%%%%%%%%%%%%%%%%%%%%%%%%%%%%%%%%%%%%%%
\paragraph{Page Numbering.}

When only a part of the document is compiled,
the appropriate numbering of pages
(as well as other status parameters)
is determined from the |.aux| files.
The latter contain information from previous passes.
However this information needs to propagate through
all intermediate child documents.
Therefore the page numbering in child documents may well
be inconsistent until the complete document is compiled at least once.

A useful (if unconventional) way to always ensure a consistent
page numbering is to restart the numbering in each child document
and denote the pages by `\textit{child}|.|\textit{page}'
where \textit{child} represents the chapter/section number of the child file.
This can be achieved by the command
|\numberwithin{page}{|\textit{child}|}|
of the \textsf{amsmath} package
where \textit{child} can be |chapter| or |section|
depending on the chosen structuring.
Alternatively, one can modify the macro |\thepage| appropriately
and reset the counter |page| at the start of each child file.

%%%%%%%%%%%%%%%%%%%%%%%%%%%%%%%%%%%%%%%%%%%%%%%%%%%%%%%%%%%%%%%%%%%%%%%%%%%%%%%%
\subsection{Conditional Processing}
\label{sec:conditional}

The package provides a mechanism to compile different versions
of a document. To customise the versions further some conditional processing
can come in handy to distinguish which version is being compiled.
The package provides two macros to describe the compilation context:

%%%%%%%%%%%%%%%%%%%%%%%%%%%%%%%%%%%%%%%%
\DescribeMacro{\ifchilddoc}
The conditional |\ifchilddoc| distinguishes between the compilation of
child documents and the main document:
%
\begin{center}
|\ifchilddoc |\textit{child-code}| |[|\||else |\textit{main-code}]| \||fi|
\end{center}

%%%%%%%%%%%%%%%%%%%%%%%%%%%%%%%%%%%%%%%%
\DescribeMacro{\childdocname}
\DescribeMacro{\childdocjob}
The macro |\childdocname| contains the filename (without extension)
of the main or child file being processed.
Note that |\childdocjob| will always contain the name of the main file.

%%%%%%%%%%%%%%%%%%%%%%%%%%%%%%%%%%%%%%%%
\paragraph{Title Page.}

Conditional processing can be used to include a title or banner page
in the main document when proper precautions are taken.
Importantly, the code in the main file should ensure that the page counter
(as well as other status parameters which are stored in the |.aux| files)
takes the same value after the conditional processing.
Otherwise the page numbers may take divergent values
depending on which part is compiled.

For example, a title page could be declared by:
%
\begin{center}
\begin{tabular}{l}
|\ifchilddoc\||else|\\
|\addtocounter{page}{-1}|\\
\textit{code for title page}\\
|\newpage|\\
|\||fi|
\end{tabular}
\end{center}
%
A banner page for the child documents can be generated by:
%
\begin{center}
\begin{tabular}{l}
|\ifchilddoc|\\
|\addtocounter{page}{-1}|\\
\textit{code for banner page}\\
|\newpage|\\
|\||fi|
\end{tabular}
\end{center}
%
Here one could write a message such as:
\begin{center}
|This is the part \childdocname{} of \childdocjob{}.|
\end{center}

%%%%%%%%%%%%%%%%%%%%%%%%%%%%%%%%%%%%%%%%%%%%%%%%%%%%%%%%%%%%%%%%%%%%%%%%%%%%%%%%
\subsection{Flags}
\label{sec:flags}

The package makes it easy to generate different versions
of the main or child documents.
To this end compilation flags can be defined
and assigned different default values.
They will be particularly useful in conjunction
with the forwarding mechanism described in \secref{sec:forward}.

For example, it may be useful to have a flag |\version|
which can be set to |draft| or |final|.
The document source will contain some conditional code
depending on the value of |\version|.
Suppose further, the flag should default to |final| for the main file
and to |draft| for child files
which is a natural assignment for editing the document.
This is achieved by placing the following code
in the preamble of the main document
(below the |\childdocmain| directive):
%
\begin{center}
\begin{tabular}{l}
|\ifchilddoc|\\
|\providecommand{\version}{draft}|\\
|\||else|\\
|\providecommand{\version}{final}|\\
|\||fi|
\end{tabular}
\end{center}
%
The definition by |\providecommand| makes sure
that previous definitions are not overwritten.
Further statements |\providecommand{\version}{...}|
can thus be added before the above code to override it.

For the main file, one might add a line
(between |\childdocmain| and the above block)
%
\begin{center}
|%\ifchilddoc\||else\providecommand{\version}{draft}\||fi|
\end{center}
%
which can be uncommented to produce a draft version.
Likewise one can add a line to the very top of a child file
(above the |\childdocof{|\textit{main}|}| directive)
%
\begin{center}
|%\providecommand{\version}{final}|
\end{center}
%
which can be uncommented to produce the final version of this child document.

%%%%%%%%%%%%%%%%%%%%%%%%%%%%%%%%%%%%%%%%%%%%%%%%%%%%%%%%%%%%%%%%%%%%%%%%%%%%%%%%
\subsection{Forwarding}
\label{sec:forward}

Different versions of the main or child documents
using compilation flags as described in \secref{sec:flags}
can be (permanently) stored in different files
for convenient compilation, viewing and distribution.
To this end, the package defines a command
to pass on compilation to a different file:

%%%%%%%%%%%%%%%%%%%%%%%%%%%%%%%%%%%%%%%%
\DescribeMacro{\childdocforward}
The command |\childdocforward| redirects processing to
another source file:
%
\begin{center}
\begin{tabular}{l}
|\input{childdoc.def}|\\
|\childdocforward[|\textit{main}|]{|\textit{dest}|}|\\
\end{tabular}
\end{center}
%
The argument \textit{dest} is the destination file
(without extension).
It should be the main file or one of the child files.
Note that further \textsf{childdoc} directives
such as |\childdocof| and |\childdocforward|
in the indicated file will be processed in this form.
The optional argument \textit{main}
passes on directly to the main file \textit{main}
while pretending to compile the child \textit{dest}.
This form behaves as if \textit{dest}
issues |\childdocof{|\textit{main}|}| right away,
and no further \textsf{childdoc} directives will be processed.

%%%%%%%%%%%%%%%%%%%%%%%%%%%%%%%%%%%%%%%%
\DescribeMacro{\...prefix}
In the alternative form |\childdocforwardprefix|,
%
\begin{center}
\begin{tabular}{l}
|\input{childdoc.def}|\\
|\childdocforwardprefix[|\textit{main}|]{|\textit{prefix}|}{|\textit{dest}|}|
\end{tabular}
\end{center}
%
the destination file is determined by a pattern
depending on the current file:
To make this work, the current file must be called
`{\textit{prefix}\hspace{0.2em}\textit{suffix}}'
with \textit{prefix} matching precisely the argument.
Processing is then passed on to the file
`{\textit{dest}\hspace{0.2em}\textit{suffix}}'.
Surely, the same effect is achieved by
directly specifying the
argument `{\textit{dest}\hspace{0.2em}\textit{suffix}}'
in the first form.
However, that requires to set up a different file
for each child. With the alternative form of the command
all these files can have exactly the same content
which simplifies setting them up and maintaining them.

For example, the following file |draft.tex|
with a compilation flag |\version| as described in \secref{sec:flags}
compiles the main document as a draft:
%
\begin{center}
\begin{tabular}{l}
|\def\version{draft}|\\
|\input{childdoc.def}|\\
|\childdocforward{|\textit{main}|}|
\end{tabular}
\end{center}
%
Likewise, the following files |final|\textit{nn}|.tex|
compile the final version of the child document
|child|\textit{nn}|.tex|:
%
\begin{center}
\begin{tabular}{l}
|\def\version{final}|\\
|\input{childdoc.def}|\\
|\childdocforwardprefix{final}{child}|
\end{tabular}
\end{center}
%

Note that when several versions of a main file and/or of each child file
are to be generated, it may be convenient to set up a |Makefile| or
shell script to automatise the process.

%%%%%%%%%%%%%%%%%%%%%%%%%%%%%%%%%%%%%%%%%%%%%%%%%%%%%%%%%%%%%%%%%%%%%%%%%%%%%%%%
\subsection{Command Line Processing}
\label{sec:commandline}

The effect of redirection files can also be achieved by invoking
the \LaTeX{} compiler with a more elaborate command line.
Most conveniently this should be done as part
of a shell script or a |Makefile|.

When using \textsf{childdoc} in the main file, the following
command lines effectively perform a redirection
(note that depending on the shell being used,
backslashes may have to be doubled: `|\|' $\to$ `|\\|'):
%
\begin{center}
|... -jobname "|\textit{target}|" |\\|"|[\textit{flags}]%
|\input{childdoc.def}\childdocforward[|\textit{main}|]{|\textit{dest}|}"|
\end{center}
%
Here \textit{target} is the name of the output file,
\textit{main} is the name of the main file
and \textit{dest} is the name of the main or child file to be processed
(all filenames without extensions).
The optional argument \textit{main} can be omitted
if \textit{main} matches \textit{dest}.
Optionally, compilation \textit{flags} can be defined via |\def| commands.
This command line makes the \TeX{} engine believe
it is compiling the file \textit{target}
whose content is specified as the latter parameter.
The provided code then forwards the processing to
\textit{main} or \textit{dest} as described in \secref{sec:forward}.

%%%%%%%%%%%%%%%%%%%%%%%%%%%%%%%%%%%%%%%%%%%%%%%%%%%%%%%%%%%%%%%%%%%%%%%%%%%%%%%%
\subsection{Include by Input}
\label{sec:input}

Including child documents by |\include| has some restrictions by design.
Most notably, the content of a child document always occupies
its own set of pages; pages cannot be shared between child documents.
Usually, this behaviour makes perfect sense
because each child document contain an essential part of the document.
However, in some situations it may be desirable to compose
a document from a collection of parts
without having mandatory page breaks between then.
For this case, the package
provides a mechanism to include parts
by |\input| which can also be processed individually.
However, by construction this mechanism
requires manual handling of the content to be output.

%%%%%%%%%%%%%%%%%%%%%%%%%%%%%%%%%%%%%%%%
\DescribeMacro{\ifchilddocmanual}
The main file should be prepared as usual, see \secref{sec:include}.
However, the document body must make a distinction
between processing of an individual part and of the main document, e.g.:
%
\begin{center}
\begin{tabular}{l}
|\ifchilddocmanual|\\
|\input{\childdocname}|\\
|\||else|\\
\textit{document body with }|\input{|\textit{part}|}|\\
|\||fi|
\end{tabular}
\end{center}
%
The conditional |\ifchilddocmanual| is true whenever
a part to be included by |\input| is being compiled,
and the name of the part is stored in |\childdocname|.

%%%%%%%%%%%%%%%%%%%%%%%%%%%%%%%%%%%%%%%%
\DescribeMacro{\childdocby}
Each part to be included by |\input| should start with:
%
\begin{center}
\begin{tabular}{l}
|\input{childdoc.def}|\\
|\childdocby{|\textit{main}|}|\\
\end{tabular}
\end{center}
%
The directive |\childdocby| is similar to |\childdocof|
described in \secref{sec:include},
but the subsequent selection of content must be done manually.
To that end, both |\ifchilddoc| and |\ifchilddocmanual|
will be true upon processing of a part,
and the name of the part is stored in |\childdocname|.
Note that |\jobname| will be set to the filename of the current part
so that each part receives an individual |.aux| file
that does not interfere with the |.aux| file(s) of the main document.
This behaviour can be altered by the alternative form
|\childdocby[*]{|\textit{main}|}| (with a non-empty optional argument)
which uses the |.aux| file of the main document
by setting |\jobname| to \textit{main}.

%%%%%%%%%%%%%%%%%%%%%%%%%%%%%%%%%%%%%%%%%%%%%%%%%%%%%%%%%%%%%%%%%%%%%%%%%%%%%%%%
\subsection{Driver Development}
\label{sec:driver}

The \textsf{childdoc} mechanism can also be use for the development
of definition files such as \LaTeX{} styles or classes.
This case differs from the above setup with multiple parts
included by |\include| in that no |\includeonly| should be invoked.
This can be achieved by starting the include file
(before |\ProvidesPackage|) with:
%
\begin{center}
\begin{tabular}{l}
|\input{childdoc.def}|\\
|\childdocforward{|\textit{main}|}|\\
\end{tabular}
\end{center}
%
or alternatively with:
%
\begin{center}
\begin{tabular}{l}
|\input{childdoc.def}|\\
|\childdocby{|\textit{main}|}|\\
\end{tabular}
\end{center}
%
Both forms have slightly different effects as described above.
The main file is prepared as usual, see \secref{sec:include}.

%%%%%%%%%%%%%%%%%%%%%%%%%%%%%%%%%%%%%%%%%%%%%%%%%%%%%%%%%%%%%%%%%%%%%%%%%%%%%%%%
\subsection{Legacy Detection}
\label{sec:detection}

The directive |\childdocmain| in the main file can detect
whether the complete document or merely a child is to be compiled
even without using the directive |\childdocof|.
This method is deprecated because it is less robust
and there is no compelling reason to use it;
it is merely provided for backward compatibility
and it may be removed in future versions.

If the detection mechanism is to be used,
it is mandatory to correctly specify
the filename of the main file as the argument of |\childdocmain|:
%
\begin{center}
\begin{tabular}{l}
|\input{childdoc.def}|\\
|\childdocmain{|\textit{main}|}|\\
\end{tabular}
\end{center}
%
If |\jobname| does not match the argument \textit{main} of |\childdocmain|,
it is assumed that |\jobname| points to the child file to be compiled.
When using |\childdocmain| with the main file specified as argument,
it suffices to start a child file
with just |\input{|\textit{main}|}|
without loading of the package and using |\childdocof|.
If instead all processing is done
with the appropriate \textsf{childdoc} directives,
the argument of \textit{main} of |\childdocmain| can be empty.

An alternative version of the command line processing described
in \secref{sec:commandline} using the detection mechanism reads:
%
\begin{center}
|... -jobname "|\textit{target}|" "|[\textit{flags}]%
[|\def\jobname{|\textit{dest}|}|]|\input{|\textit{main}|}"|
\end{center}

%%%%%%%%%%%%%%%%%%%%%%%%%%%%%%%%%%%%%%%%%%%%%%%%%%%%%%%%%%%%%%%%%%%%%%%%%%%%%%%%
\subsection{Manual Code}
\label{sec:manual}

In case one cannot be certain whether the definitions file |childdoc.def|
is installed on the target \TeX{} distribution
and one prefers not to ship it,
it is conceivable to paste a few relevant commands into the sources.

To that end, drop all statements |\input{childdoc.def}|
and perform the replacements as outlined below.
Instead of |\childdocmain{|\textit{main}|}| add the following code
to the top of the main file:
%
\begin{center}
\begin{tabular}{l}
|\||ifdefined\childdocname\endinput\||fi\newif\ifchilddoc|\\
|\edef\childdocname{\scantokens\expandafter{\jobname\noexpand}}|\\
|\def\childdocmain{|\textit{main}|}\||ifx\childdocmain\childdocname\||else|\\
|\childdoctrue\includeonly{\childdocname}\let\jobname\childdocmain\||fi|\\
\end{tabular}
\end{center}
%
Instead of |\childdocof{|\textit{main}|}| just include the main file
at the top of each child file:
%
\begin{center}
|\input{|\textit{main}|}|
\end{center}
%
A simple redirection |\childdocforward{|\textit{dest}|}| is achieved by:
%
\begin{center}
|\def\jobname{|\textit{dest}|}\input{\jobname}|
\end{center}
%
The redirection with prefix
|\childdocforwardprefix[|\textit{prefix}|]{|\textit{dest}|}|
is accomplished by:
%
\begin{center}
\begin{tabular}{l}
|{\edef\jobname{\scantokens\expandafter{\jobname\noexpand}}|\\
|\def\redirectjob |\textit{prefix}|#1~~~{\gdef\jobname{|\textit{dest}|#1}}|\\
|\expandafter\redirectjob\jobname~~~}\input{\jobname}|
\end{tabular}
\end{center}

In an alternative approach,
child documents can be compiled by a specific command line
without additional code or specific definitions:
%
\begin{center}
|... -jobname "|\textit{target}|" "|[\textit{flags}]%
|\includeonly{|\textit{dest}|}\input{|\textit{main}|}"|
\end{center}
%

%%%%%%%%%%%%%%%%%%%%%%%%%%%%%%%%%%%%%%%%%%%%%%%%%%%%%%%%%%%%%%%%%%%%%%%%%%%%%%%%
%%%%%%%%%%%%%%%%%%%%%%%%%%%%%%%%%%%%%%%%%%%%%%%%%%%%%%%%%%%%%%%%%%%%%%%%%%%%%%%%
\section{Information}

%%%%%%%%%%%%%%%%%%%%%%%%%%%%%%%%%%%%%%%%%%%%%%%%%%%%%%%%%%%%%%%%%%%%%%%%%%%%%%%%
\subsection{Copyright}

Copyright \copyright{} 2017--2018 Niklas Beisert

This work may be distributed and/or modified under the
conditions of the \LaTeX{} Project Public License, either version 1.3
of this license or (at your option) any later version.
The latest version of this license is in
  \url{http://www.latex-project.org/lppl.txt}
and version 1.3 or later is part of all distributions of \LaTeX{}
version 2005/12/01 or later.

This work has the LPPL maintenance status `maintained'.

The Current Maintainer of this work is Niklas Beisert.

This work consists of the files |README.txt|, |childdoc.ins| and |childdoc.dtx|
as well as the derived files |childdoc.def|, |cdocsamp.tex|
with |cdocsch1.tex|, |cdocsch2.tex|, |cdocspt3.tex|, |cdocspt4.tex|,
|cdocsdrf.tex|, |cdocsfn1.tex|, |cdocsfn2.tex|
as well as |childdoc.pdf|.

%%%%%%%%%%%%%%%%%%%%%%%%%%%%%%%%%%%%%%%%%%%%%%%%%%%%%%%%%%%%%%%%%%%%%%%%%%%%%%%%
\subsection{Files and Installation}

The package consists of the files:
%
\begin{center}
\begin{tabular}{ll}
    |README.txt|   & readme file \\
    |childdoc.ins| & installation file \\
    |childdoc.dtx| & source file \\
    |childdoc.def| & definition file \\
    |cdocsamp.tex| & sample main file \\
    |cdocsch1.tex| & sample include file \\
    |cdocsch2.tex| & sample include file \\
    |cdocspt3.tex| & sample part file \\
    |cdocspt4.tex| & sample part file \\
    |cdocsdrf.tex| & sample redirection file \\
    |cdocsfn1.tex| & sample redirection file \\
    |cdocsfn2.tex| & sample redirection file \\
    |childdoc.pdf| & manual
\end{tabular}
\end{center}
%
The distribution consists of the files
|README.txt|, |childdoc.ins| and |childdoc.dtx|.
%
\begin{itemize}
\item
Run (pdf)\LaTeX{} on |childdoc.dtx|
to compile the manual |childdoc.pdf| (this file).
\item
Run \LaTeX{} on |childdoc.ins| to create the definitions file |childdoc.def|
and the sample |cdocsamp.tex| with include files
|cdocsch1.tex|, |cdocsch2.tex|, |cdocspt3.tex|, |cdocspt4.tex|,
|cdocsdrf.tex|, |cdocsfn1.tex|, |cdocsfn2.tex|.
Then copy the file |childdoc.def| to an appropriate directory of your \LaTeX{}
distribution, e.g.\ \textit{texmf-root}|/tex/latex/childdoc|.
\end{itemize}

%%%%%%%%%%%%%%%%%%%%%%%%%%%%%%%%%%%%%%%%%%%%%%%%%%%%%%%%%%%%%%%%%%%%%%%%%%%%%%%%
\subsection{Related CTAN Packages}

There are several other packages which offer a similar functionality:
%
\begin{itemize}
\item
The packages
\href{http://ctan.org/pkg/docmute}{\textsf{docmute}},
\href{http://ctan.org/pkg/includex}{\textsf{includex}} and
\href{http://ctan.org/pkg/standalone}{\textsf{standalone}}
provide commands to include only the document body of
a child file thus allowing both files to be compiled individually.
\item
The packages \href{http://ctan.org/pkg/subdocs}{\textsf{subdocs}}
and \href{http://ctan.org/pkg/subfiles}{\textsf{subfiles}}
provide structures in which the main and child documents can be
encapsulated and allowing them to be compiled individually.
The inclusion mechanism is different from the conventional |\include|.
\item
The package \href{http://ctan.org/pkg/combine}{\textsf{combine}}
is an elaborate solution to combine several documents into one.
\end{itemize}
%
See also the CTAN topic \href{http://ctan.org/topic/subdocs}{\textsf{subdocs}}
for further related packages.
The present package differs from the above solutions in that
a document structure constructed with the conventional |\include| mechanism
just needs two extra commands at the top of every file
such that all constituent files can be compiled individually.

%%%%%%%%%%%%%%%%%%%%%%%%%%%%%%%%%%%%%%%%%%%%%%%%%%%%%%%%%%%%%%%%%%%%%%%%%%%%%%%%
%\subsection{Feature Suggestions}
%
%The following is a list of features which may be useful for future
%versions of this package:
%%
%\begin{itemize}
%\item
%\ldots
%\end{itemize}

%%%%%%%%%%%%%%%%%%%%%%%%%%%%%%%%%%%%%%%%%%%%%%%%%%%%%%%%%%%%%%%%%%%%%%%%%%%%%%%%
\subsection{Revision History}

%%%%%%%%%%%%%%%%%%%%%%%%%%%%%%%%%%%%%%%%
\paragraph{v2.0:} 2018/12/30

\begin{itemize}
\item
immediate forward processing
\item
added |\childdocby| mechanism
\item
manual restructured
\end{itemize}

%%%%%%%%%%%%%%%%%%%%%%%%%%%%%%%%%%%%%%%%
\paragraph{v1.6:} 2018/01/17

\begin{itemize}
\item
application for development of include files
\item
corrections to manual
\end{itemize}

%%%%%%%%%%%%%%%%%%%%%%%%%%%%%%%%%%%%%%%%
\paragraph{v1.5:} 2017/05/21

\begin{itemize}
\item
more complete structuring introduced
\item
|\childdocof| introduced
\item
|\childdoc| renamed to |\childdocmain|
\item
|\childredirect| renamed to |\childdocforward| and |\childdocforwardprefix|
and functionality expanded
\end{itemize}

%%%%%%%%%%%%%%%%%%%%%%%%%%%%%%%%%%%%%%%%
\paragraph{v1.0:} 2017/04/27

\begin{itemize}
\item
manual and install package
\item
first version published on CTAN
\end{itemize}

%%%%%%%%%%%%%%%%%%%%%%%%%%%%%%%%%%%%%%%%
\paragraph{v0.6:} 2017/04/26

\begin{itemize}
\item
redirection mechanism added
\end{itemize}

%%%%%%%%%%%%%%%%%%%%%%%%%%%%%%%%%%%%%%%%
\paragraph{v0.5:} 2017/04/26

\begin{itemize}
\item
functionality in definition file
\end{itemize}


%%%%%%%%%%%%%%%%%%%%%%%%%%%%%%%%%%%%%%%%%%%%%%%%%%%%%%%%%%%%%%%%%%%%%%%%%%%%%%%%
%%%%%%%%%%%%%%%%%%%%%%%%%%%%%%%%%%%%%%%%%%%%%%%%%%%%%%%%%%%%%%%%%%%%%%%%%%%%%%%%
%%%%%%%%%%%%%%%%%%%%%%%%%%%%%%%%%%%%%%%%%%%%%%%%%%%%%%%%%%%%%%%%%%%%%%%%%%%%%%%%
\appendix

\settowidth\MacroIndent{\rmfamily\scriptsize 000\ }

 \DocInput{childdoc.dtx}

\end{document}
%</driver>
% \fi
%
% %%%%%%%%%%%%%%%%%%%%%%%%%%%%%%%%%%%%%%%%%%%%%%%%%%%%%%%%%%%%%%%%%%%%%%%%%%%%%%
% %%%%%%%%%%%%%%%%%%%%%%%%%%%%%%%%%%%%%%%%%%%%%%%%%%%%%%%%%%%%%%%%%%%%%%%%%%%%%%
% \section{Sample}
%\iffalse
%<*samplemain>
%\fi
%
% The following presents a sample document
% with two chapters, two parts, a title page,
% a compile flag as well as three forwarding files to set the flag.
% It consists of eight |.tex| files:
% \begin{center}
% \begin{tabular}{ll}
% |cdocsamp.tex|&main file\\
% |cdocsch1.tex|&include file for chapter 1\\
% |cdocsch2.tex|&include file for chapter 2\\
% |cdocspt3.tex|&include file for part 3\\
% |cdocspt4.tex|&include file for part 4\\
% |cdocsdrf.tex|&forwarding file for main file in draft mode\\
% |cdocsfi1.tex|&forwarding file for final version of chapter 1\\
% |cdocsfi2.tex|&forwarding file for final version of chapter 2\\
% \end{tabular}
% \end{center}
% Each of the eight files can be compiled directly by the \LaTeX{} compiler.
%
% %%%%%%%%%%%%%%%%%%%%%%%%%%%%%%%%%%%%%%
% \paragraph{Main File.}
%
% The main file is called |cdocsamp.tex|.
%
% Load the \textsf{childdoc} definitions and
% declare the filename for the main document:
%    \begin{macrocode}
\input{childdoc.def}
\childdocmain{}
%    \end{macrocode}

% Optional override for |\version| flag:
%    \begin{macrocode}
%%\ifchilddoc\else\providecommand{\version}{draft}\fi
%    \end{macrocode}

% Define the default values for the |\version| flag
% (|final| for the main file and |draft| for childs):
%    \begin{macrocode}
\ifchilddoc
\providecommand{\version}{draft}
\else
\providecommand{\version}{final}
\fi
%    \end{macrocode}

% Load the standard document class:
%    \begin{macrocode}
\documentclass[12pt]{article}
%    \end{macrocode}

% Start the document body:
%    \begin{macrocode}
\begin{document}
%    \end{macrocode}

% Declare a title page.
% Print title, part of document being processed and version flag:
%    \begin{macrocode}
\addtocounter{page}{-1}
\begin{center}
{\LARGE\bfseries{}childdoc example\par}
\vspace{1cm}
\ifchilddoc
\ifchilddocmanual part\else chapter\fi:
`\childdocname' of `\childdocjob'\par
\else
main document: `\childdocjob'\par
\fi
version: \version\par
\end{center}
\newpage
%    \end{macrocode}

% Manually include selected file,
% otherwise process as usual:
%    \begin{macrocode}
\ifchilddocmanual
\section*{part `\childdocname'}
\input{\childdocname}
\else
%    \end{macrocode}

% Include the two chapters:
%    \begin{macrocode}
\include{cdocsch1}
\include{cdocsch2}
%    \end{macrocode}

% Include the two parts unless only chapters should be displayed:
%    \begin{macrocode}
\ifchilddoc\else
\section{part three}
\input{cdocspt3}
\section{part four}
\input{cdocspt4}
\fi
%    \end{macrocode}

% Process as usual until here:
%    \begin{macrocode}
\fi
%    \end{macrocode}

% End of document body:
%    \begin{macrocode}
\end{document}
%    \end{macrocode}
%\iffalse
%</samplemain>
%\fi
%
% %%%%%%%%%%%%%%%%%%%%%%%%%%%%%%%%%%%%%%
% \paragraph{Chapter Include Files.}
%
% The include files are called |cdocsch1.tex| and |cdocsch2.tex|.
%
%\iffalse
%<*samplechap1|samplechap2>
%\fi

% Optional override for |\version| flag:
%    \begin{macrocode}
%%\providecommand{\version}{final}
%    \end{macrocode}

% Include the main document:
%    \begin{macrocode}
\input{childdoc.def}
\childdocof{cdocsamp}
%    \end{macrocode}

%\iffalse
%</samplechap1|samplechap2>
%\fi
%
%\iffalse
%<*samplechap1>
%\fi
% Some text for chapter 1:
%    \begin{macrocode}
\section{one}
some text in chapter one
%    \end{macrocode}

%\iffalse
%</samplechap1>
%\fi
% Some text for chapter 2:
%\iffalse
%<*samplechap2>
%\fi
%    \begin{macrocode}
\section{two}
more text in chapter two
%    \end{macrocode}

%\iffalse
%</samplechap2>
%\fi
%
% %%%%%%%%%%%%%%%%%%%%%%%%%%%%%%%%%%%%%%
% \paragraph{Part Include Files.}
%
% The include files are called |cdocspt3.tex| and |cdocspt4.tex|.
%
%\iffalse
%<*samplepart3|samplepart4>
%\fi

% Optional override for |\version| flag:
%    \begin{macrocode}
%%\providecommand{\version}{final}
%    \end{macrocode}

% Include the main document:
%    \begin{macrocode}
\input{childdoc.def}
\childdocby{cdocsamp}
%    \end{macrocode}

%\iffalse
%</samplepart3|samplepart4>
%\fi
%
%\iffalse
%<*samplepart3>
%\fi
% Some text for part 3:
%    \begin{macrocode}
some text in part three
%    \end{macrocode}

%\iffalse
%</samplepart3>
%\fi
% Some text for part 4:
%\iffalse
%<*samplepart4>
%\fi
%    \begin{macrocode}
more text in part four
%    \end{macrocode}

%\iffalse
%</samplepart4>
%\fi
%
% %%%%%%%%%%%%%%%%%%%%%%%%%%%%%%%%%%%%%%
% \paragraph{Forwarding for a Complete Draft.}
%
% The following forwarding file |cdocsdrf.tex|
% compiles the main document in draft mode:
%\iffalse
%<*sampledraft>
%\fi
%    \begin{macrocode}
\def\version{draft}
\input{childdoc.def}
\childdocforward{cdocsamp}
%    \end{macrocode}

%\iffalse
%</sampledraft>
%\fi
%
% %%%%%%%%%%%%%%%%%%%%%%%%%%%%%%%%%%%%%%
% \paragraph{Forwarding for Final Version of the Chapters.}
%
% The following forwarding files |cdocsfn1.tex| and |cdocsfn2.tex|
% (with identical content)
% compile the final versions of the child documents
% |cdocsch1.tex| and |cdocsch2.tex|, respectively:
%\iffalse
%<*samplefinal>
%\fi
%    \begin{macrocode}
\def\version{final}
\input{childdoc.def}
\childdocforwardprefix[cdocsamp]{cdocsfn}{cdocsch}
%    \end{macrocode}

%\iffalse
%</samplefinal>
%\fi
%
% %%%%%%%%%%%%%%%%%%%%%%%%%%%%%%%%%%%%%%
% \paragraph{Command Line Processing.}
%
% The following three command lines generate the output files
% |cdocscld|, |cdocscl1| and |cdocscl2|
% which should be identical to
% |cdocsdrf|, |cdocsch1| and |cdocsfn2|, respectively:
% \begin{center}
% \begin{tabular}{l}
% |latex -jobname cdocscld \|\\
% |  "\def\version{draft}\input{childdoc.def}\childdocforward{cdocsamp}"|\\
% |latex -jobname cdocscl1 \|\\
% |  "\input{childdoc.def}\childdocforward[cdocsamp]{cdocsch1}"|\\
% |latex -jobname cdocscl2 \|\\
% |  "\def\version{final}\input{childdoc.def}\childdocforward{cdocsch2}"|
% \end{tabular}
% \end{center}
% Note that the trailing backslash on each first line
% merely continues the input to the second line
% (for convenient cut ant paste).
% Furthermore, the command |latex| can be replaced by any
% of its alternative versions such as |pdflatex|.
%
% %%%%%%%%%%%%%%%%%%%%%%%%%%%%%%%%%%%%%%%%%%%%%%%%%%%%%%%%%%%%%%%%%%%%%%%%%%%%%%
% %%%%%%%%%%%%%%%%%%%%%%%%%%%%%%%%%%%%%%%%%%%%%%%%%%%%%%%%%%%%%%%%%%%%%%%%%%%%%%
% \section{Implementation}
%\iffalse
%<*package>
%\fi
%
% This section describes the definitions file |childdoc.def|.

% The definitions cannot be loaded using |\usepackage| or |\RequirePackage|
% which has a mechanism to prevent loading a style file more than once.
% When loading the definitions by means of |\input|
% multiple instances have to be prevented manually:
%\iffalse
%This code needs to be before the `\ProvidesFile' directive
%which is defined at the beginning of this file.
%Therefore it is also placed there and commented out here.
%</package>
%<*discard>
%\fi
%    \begin{macrocode}
\ifdefined\childdocmain\endinput\fi
%    \end{macrocode}
%\iffalse
%</discard>
%<*package>
%\fi
%
% \macro{\ifchilddoc}
% \macro{\ifchilddocmanual}
% The conditional |\ifchilddoc| tells whether a
% child (true) or main (false) document is being compiled.
% The conditional |\ifchilddocmanual| tells whether
% the |\includeonly| mechanism is used (false) or
% the selection of child files must be performed manually (true).
% The definitions initialise to false:
%    \begin{macrocode}
\newif\ifchilddoc
\newif\ifchilddocmanual
%    \end{macrocode}

% \macro{\childdocname}
% \macro{\childdocjob}
% The macro |\childdocname| stores the name of the main document
% to be compiled. The macro |\childdocjob| stores the name of
% the document on which the \LaTeX{} compiler was originally invoked.
% The content of |\jobname| cannot be compared
% to filenames specified in the source due to different catcodes.
% The following code rescans |\jobname|, stores the result
% in |\childdocname| and saves a copy in |\childdocjob|:
%    \begin{macrocode}
\edef\childdocname{\scantokens\expandafter{\jobname\noexpand}}
\let\childdocjob\childdocname
%    \end{macrocode}

% \macro{\childdocdisable}
% The macro |\childdocdisable| prevents the main file
% from being processed more than once.
% At this stage, the main document command |\childdocmain|
% is assumed to be called once again where it should do nothing.
% Any subsequent call to it should prevent
% a secondary processing of the main document
% It overwrites the forwarding commands
% |\childdocof| and |\childdocforward|
% with empty macros to prevent further inclusions of the main document:
%    \begin{macrocode}
\newcommand{\childdocdisable}
{
  \renewcommand{\childdocmain}[1]{\renewcommand{\childdocmain}[1]{\endinput}}
  \renewcommand{\childdocof}[1]{}
  \renewcommand{\childdocby}[2][]{}
  \renewcommand{\childdocforward}[2][]{}
  \renewcommand{\childdocdisable}{}
}
%    \end{macrocode}

% \macro{\childdocmain}
% The macro |\childdocmain| is to be called at the top of the main file
% with nothing or the main filename (without extension) as argument.
% First, it breaks loops.
% If the argument is not empty and does not match |\childdocname|
% (which is set by the first inclusion of |childdoc.def|),
% |\ifchilddoc| is set to true, |\includeonly| is applied to the child file
% and |\jobname| is set to the main file
% (for proper handling of |.aux| files):
%    \begin{macrocode}
\newcommand{\childdocmain}[1]
{
  \childdocdisable\childdocmain{}
  \if?#1?\else
    \begingroup
      \def\childdoctmp{#1}
      \ifx\childdoctmp\childdocname
        \def\childdoctmp{}
      \else
        \def\childdoctmp
        {
          \childdoctrue
          \includeonly{\childdocname}
          \def\childdocjob{#1}
          \def\jobname{#1}
        }
      \fi
      \expandafter
    \endgroup
    \childdoctmp
  \fi
}
%    \end{macrocode}

% \macro{\childdocof}
% The command |\childdocof| redirects
% compilation to the main file |#1|.
%    \begin{macrocode}
\newcommand{\childdocof}[1]
{
  \childdocdisable
  \childdoctrue
  \includeonly{\childdocname}
  \def\jobname{#1}
  \def\childdocjob{#1}
  \input{#1}
}
%    \end{macrocode}

% \macro{\childdocby}
% The command |\childdocby| ....
%    \begin{macrocode}
\newcommand{\childdocby}[2][]
{
  \childdocdisable
  \childdoctrue
  \childdocmanualtrue
  \if?#1?\else
    \def\jobname{#2}
  \fi
  \def\childdocjob{#2}
  \input{#2}
  \endinput
}
%    \end{macrocode}

% \macro{\childdocforward}
% The command |\childdocforward| redirects
% compilation to the main file or
% (if the optional argument is given) a child file.
% Parameters are set as if the main file
% or a child file starting with |\childdocof| was compiled.
% Then compilation is handed over to the main file:
%    \begin{macrocode}
\newcommand{\childdocforward}[2][]
{
  \begingroup
    \if?#1?
      \def\childdoctmp
      {
        \def\childdocname{#2}
        \def\childdocjob{#2}
        \def\jobname{#2}
        \input{#2}
        \endinput
      }
    \else
      \def\childdoctmp
      {
        \childdocdisable
        \def\childdocname{#2}
        \childdoctrue
        \includeonly{#2}
        \def\childdocjob{#1}
        \def\jobname{#1}
        \input{#1}
        \endinput
      }
    \fi
    \expandafter
  \endgroup
  \childdoctmp
}
%    \end{macrocode}

% \macro{\childdocforwardprefix}
% The command |\childdocforwardprefix| redirects
% compilation to the main or a child file by means of a pattern.
% The prefix |#1| in the current filename is replaced by |#2|
% and the suffix of the current filename is kept
% (it is assumed that the filename does not contain the substring `|~~~|'
% which is used as a delimiter).
% Compilation is handed over to the new file by |\childdocforward|:
%    \begin{macrocode}
\newcommand{\childdocforwardprefix}[3][]
{
  \begingroup
    \def\childdocextract #2##1~~~{\def\childdoctmp{\childdocforward[#1]{#3##1}}}
    \expandafter\childdocextract\childdocname~~~
    \expandafter
  \endgroup
  \childdoctmp
}
%    \end{macrocode}

% \macro{\childdoc}
% The deprecated macro |\childdoc| is a legacy version of |\childdocmain|:
%    \begin{macrocode}
\newcommand{\childdoc}{\childdocmain}
%    \end{macrocode}

% \macro{\childdocredirect}
% The deprecated macro |\childdocredirect| is a legacy version
% of |\childdocforward| and |\childdocforwardprefix|:
%    \begin{macrocode}
\newcommand{\childdocredirect}[2][]
{
  \begingroup
    \if?#1?
      \def\childdoctmp{\childdocforward{#2}}
    \else
      \def\childdoctmp{\childdocforwardprefix{#1}{#2}}
    \fi
    \expandafter
  \endgroup
  \childdoctmp
}
%    \end{macrocode}

%\iffalse
%</package>
%\fi
%
\endinput
|\\
|\childdocby{|\textit{main}|}|\\
\end{tabular}
\end{center}
%
Both forms have slightly different effects as described above.
The main file is prepared as usual, see \secref{sec:include}.

%%%%%%%%%%%%%%%%%%%%%%%%%%%%%%%%%%%%%%%%%%%%%%%%%%%%%%%%%%%%%%%%%%%%%%%%%%%%%%%%
\subsection{Legacy Detection}
\label{sec:detection}

The directive |\childdocmain| in the main file can detect
whether the complete document or merely a child is to be compiled
even without using the directive |\childdocof|.
This method is deprecated because it is less robust
and there is no compelling reason to use it;
it is merely provided for backward compatibility
and it may be removed in future versions.

If the detection mechanism is to be used,
it is mandatory to correctly specify
the filename of the main file as the argument of |\childdocmain|:
%
\begin{center}
\begin{tabular}{l}
|% \iffalse
%
% childdoc.dtx Copyright (C) 2017-2018 Niklas Beisert
%
% This work may be distributed and/or modified under the
% conditions of the LaTeX Project Public License, either version 1.3
% of this license or (at your option) any later version.
% The latest version of this license is in
%   http://www.latex-project.org/lppl.txt
% and version 1.3 or later is part of all distributions of LaTeX
% version 2005/12/01 or later.
%
% This work has the LPPL maintenance status `maintained'.
%
% The Current Maintainer of this work is Niklas Beisert.
%
% This work consists of the files childdoc.dtx and childdoc.ins
% and the derived files childdoc.def and cdocsamp.tex with
% cdocsch1.tex, cdocsch2.tex, cdocsdrf.tex, cdocsfn1.tex, cdocsfn2.tex.
%
%<package>\ifdefined\childdocmain\endinput\fi
%<package>\ProvidesFile{childdoc.def}[2018/12/30 v2.0 child document driver]
%<samplemain>\ProvidesFile{cdocsamp.tex}[2018/12/30 v2.0 sample for childdoc]
%<*driver>
%\ProvidesFile{childdoc.drv}[2018/12/30 v2.0 childdoc reference manual file]
\PassOptionsToClass{10pt,a4paper}{article}
\documentclass{ltxdoc}

\usepackage[margin=35mm]{geometry}
\usepackage{hyperref}
\usepackage{hyperxmp}
\usepackage[usenames]{color}

\hypersetup{colorlinks=true}
\hypersetup{pdfstartview=FitH}
\hypersetup{pdfpagemode=UseNone}
\hypersetup{pdfsource={}}
\hypersetup{pdflang={en-UK}}
\hypersetup{pdfcopyright={Copyright 2017-2018 Niklas Beisert.
  This work may be distributed and/or modified under the
  conditions of the LaTeX Project Public License, either version 1.3
  of this license or (at your option) any later version.}}
\hypersetup{pdflicenseurl={http://www.latex-project.org/lppl.txt}}
\hypersetup{pdfcontactaddress={ETH Zurich, ITP, HIT K,
  Wolfgang-Pauli-Strasse 27}}
\hypersetup{pdfcontactpostcode={8093}}
\hypersetup{pdfcontactcity={Zurich}}
\hypersetup{pdfcontactcountry={Switzerland}}
\hypersetup{pdfcontactemail={nbeisert@itp.phys.ethz.ch}}
\hypersetup{pdfcontacturl={http://people.phys.ethz.ch/\xmptilde nbeisert/}}

\newcommand{\secref}[1]{\hyperref[#1]{section \ref*{#1}}}

\parskip1ex
\parindent0pt
\let\olditemize\itemize
\def\itemize{\olditemize\parskip0pt}

\begin{document}

\title{The \textsf{childdoc} Package}
\hypersetup{pdftitle={The childdoc Package}}
\author{Niklas Beisert\\[2ex]
  Institut f\"ur Theoretische Physik\\
  Eidgen\"ossische Technische Hochschule Z\"urich\\
  Wolfgang-Pauli-Strasse 27, 8093 Z\"urich, Switzerland\\[1ex]
  \href{mailto:nbeisert@itp.phys.ethz.ch}
  {\texttt{nbeisert@itp.phys.ethz.ch}}}
\hypersetup{pdfauthor={Niklas Beisert}}
\hypersetup{pdfsubject={Manual for the LaTeX2e Package childdoc}}
\date{30 December 2018, \textsf{v2.0}}
\maketitle

\begin{abstract}\noindent
\textsf{childdoc} is a \LaTeXe{} package
that enables the direct compilation
of document sections included by |\include|
to individual files.
\end{abstract}

\begingroup
\parskip0ex
\tableofcontents
\endgroup

%%%%%%%%%%%%%%%%%%%%%%%%%%%%%%%%%%%%%%%%%%%%%%%%%%%%%%%%%%%%%%%%%%%%%%%%%%%%%%%%
%%%%%%%%%%%%%%%%%%%%%%%%%%%%%%%%%%%%%%%%%%%%%%%%%%%%%%%%%%%%%%%%%%%%%%%%%%%%%%%%
\section{Introduction}

\LaTeX{} provides a mechanism to structure a large document (such as a book)
into a main file and several child files (containing the chapters)
using the |\include| command.
This mechanism is beneficial for documents
which span hundreds of pages in order to
make the source file(s) more manageable.
Moreover, compilation can be restricted to
selected child files by means of the |\includeonly| command.
The latter feature can be used to reduce the compilation time while editing
(this was significantly more useful in the earlier days of \LaTeX{})
or to generate a smaller document which is easier to navigate.
Another application of |\includeonly| is to generate
documents consisting of selected parts of the complete document.

However, there are a few drawbacks of the plain |\include| mechanism:
\begin{itemize}
\item
The child files cannot be compiled on their own,
they can only be compiled via the main file.
A naive editing environment
(such as a text editor with an option
to have the current file processed by \LaTeX)
may require one to switch to the main file before compiling;
attempting to compile the child file produces errors.
\item
The main file must be modified (each time)
to adjust the |\includeonly| command
to the present needs. This easily leaves the main file in a messy state.
\item
The generated document will always carry the filename
of the main document. This is inconvenient if
several child files are to be compiled and
to be kept for distribution.
\end{itemize}

The present package provides a simple interface
to make child files individually compilable by \LaTeX{}.
Compiling a child file then has the same effect as compiling
the main file with an |\includeonly| command
to select the appropriate child.
Moreover the generated document will carry the name of the child
rather than the main file.
This resolves all three above issues.

This feature is meant to make the editing of books,
thesis documents and lecture notes somewhat more convenient.
However, the package can also be used efficiently for
composing a series of documents (such as exercise sheets)
which are typically distributed individually.
It then assists the author in generating the individual documents
(potentially in different versions)
as well as a document containing the collected series.
Another application is in developing style files
or other kinds of included material
where compilation of the style file could redirect
to a sample or test file.

%%%%%%%%%%%%%%%%%%%%%%%%%%%%%%%%%%%%%%%%%%%%%%%%%%%%%%%%%%%%%%%%%%%%%%%%%%%%%%%%
%%%%%%%%%%%%%%%%%%%%%%%%%%%%%%%%%%%%%%%%%%%%%%%%%%%%%%%%%%%%%%%%%%%%%%%%%%%%%%%%
\section{Usage}

First of all, the package \textsf{childdoc} is \emph{not} a standard
\LaTeXe{} |.sty| style file! Therefore it needs to be invoked in
a non-standard way.

%%%%%%%%%%%%%%%%%%%%%%%%%%%%%%%%%%%%%%%%%%%%%%%%%%%%%%%%%%%%%%%%%%%%%%%%%%%%%%%%
\subsection{Included Files}
\label{sec:include}

%%%%%%%%%%%%%%%%%%%%%%%%%%%%%%%%%%%%%%%%
\DescribeMacro{\childdocmain}
To use the package, add the commands
\begin{center}
\begin{tabular}{l}
|\input{childdoc.def}|\\
|\childdocmain{}|\\
\end{tabular}
\end{center}
at the very top of the main \LaTeX{} file,
in particular \emph{before} the |\documentclass| statement!
The argument of |\childdocmain| should be left empty
(but it must be present).

%%%%%%%%%%%%%%%%%%%%%%%%%%%%%%%%%%%%%%%%
\DescribeMacro{\childdocof}
Furthermore, add the commands
\begin{center}
\begin{tabular}{l}
|\input{childdoc.def}|\\
|\childdocof{|\textit{main}|}|\\
\end{tabular}
\end{center}
at the top of every child file \textit{child}
which is included by |\include{|\textit{child}|}|
from within the main file
(or at least for those files to be compiled individually).
The argument \textit{main} must be the filename of the main file.

There are a couple of
considerations in setting up the main and child documents:

%%%%%%%%%%%%%%%%%%%%%%%%%%%%%%%%%%%%%%%%
\paragraph{Restrictions.}

Please note the following restrictions:
\begin{itemize}
\item
|\childdocmain| must be called with one argument \textit{main}
to ensure compatibility with earlier version of the package.
It must either be empty (|\childdocmain{}|)
or precisely match the filename of the main file in which it is specified.
See \secref{sec:detection} for further information.
\item
The filename \textit{main} must be specified without the |.tex| extension.
\item
The filename \textit{main} is case sensitive
(even in case-insensitive file systems)
due to internal string comparison.
\item
The argument \textit{main} should be fully expanded, it cannot be a macro.
\item
Subdirectories and special characters should be avoided in filenames.
\item
The command |\childdocmain{|\textit{main}|}| must be followed by a whitespace.
It should not be followed immediately by another command
or by a comment mark `|%|'.
This is because the \TeX{} parser reads the token immediately following
the argument of |\childdocmain| and puts it
at the beginning of every child section;
however, a white\-space is ignored.
\end{itemize}

%%%%%%%%%%%%%%%%%%%%%%%%%%%%%%%%%%%%%%%%
\paragraph{Content of Main File.}

It is advisable to place all content in the child files included by |\include|.
Any output contained in the main file will appear in all child documents
unless suppressed manually;
it cannot be suppressed automatically by the |\includeonly| directive
and thus should normally be avoided.
A method to include some content in the main file
by means of conditional processing is described in \secref{sec:conditional}.

%%%%%%%%%%%%%%%%%%%%%%%%%%%%%%%%%%%%%%%%
\paragraph{Page Numbering.}

When only a part of the document is compiled,
the appropriate numbering of pages
(as well as other status parameters)
is determined from the |.aux| files.
The latter contain information from previous passes.
However this information needs to propagate through
all intermediate child documents.
Therefore the page numbering in child documents may well
be inconsistent until the complete document is compiled at least once.

A useful (if unconventional) way to always ensure a consistent
page numbering is to restart the numbering in each child document
and denote the pages by `\textit{child}|.|\textit{page}'
where \textit{child} represents the chapter/section number of the child file.
This can be achieved by the command
|\numberwithin{page}{|\textit{child}|}|
of the \textsf{amsmath} package
where \textit{child} can be |chapter| or |section|
depending on the chosen structuring.
Alternatively, one can modify the macro |\thepage| appropriately
and reset the counter |page| at the start of each child file.

%%%%%%%%%%%%%%%%%%%%%%%%%%%%%%%%%%%%%%%%%%%%%%%%%%%%%%%%%%%%%%%%%%%%%%%%%%%%%%%%
\subsection{Conditional Processing}
\label{sec:conditional}

The package provides a mechanism to compile different versions
of a document. To customise the versions further some conditional processing
can come in handy to distinguish which version is being compiled.
The package provides two macros to describe the compilation context:

%%%%%%%%%%%%%%%%%%%%%%%%%%%%%%%%%%%%%%%%
\DescribeMacro{\ifchilddoc}
The conditional |\ifchilddoc| distinguishes between the compilation of
child documents and the main document:
%
\begin{center}
|\ifchilddoc |\textit{child-code}| |[|\||else |\textit{main-code}]| \||fi|
\end{center}

%%%%%%%%%%%%%%%%%%%%%%%%%%%%%%%%%%%%%%%%
\DescribeMacro{\childdocname}
\DescribeMacro{\childdocjob}
The macro |\childdocname| contains the filename (without extension)
of the main or child file being processed.
Note that |\childdocjob| will always contain the name of the main file.

%%%%%%%%%%%%%%%%%%%%%%%%%%%%%%%%%%%%%%%%
\paragraph{Title Page.}

Conditional processing can be used to include a title or banner page
in the main document when proper precautions are taken.
Importantly, the code in the main file should ensure that the page counter
(as well as other status parameters which are stored in the |.aux| files)
takes the same value after the conditional processing.
Otherwise the page numbers may take divergent values
depending on which part is compiled.

For example, a title page could be declared by:
%
\begin{center}
\begin{tabular}{l}
|\ifchilddoc\||else|\\
|\addtocounter{page}{-1}|\\
\textit{code for title page}\\
|\newpage|\\
|\||fi|
\end{tabular}
\end{center}
%
A banner page for the child documents can be generated by:
%
\begin{center}
\begin{tabular}{l}
|\ifchilddoc|\\
|\addtocounter{page}{-1}|\\
\textit{code for banner page}\\
|\newpage|\\
|\||fi|
\end{tabular}
\end{center}
%
Here one could write a message such as:
\begin{center}
|This is the part \childdocname{} of \childdocjob{}.|
\end{center}

%%%%%%%%%%%%%%%%%%%%%%%%%%%%%%%%%%%%%%%%%%%%%%%%%%%%%%%%%%%%%%%%%%%%%%%%%%%%%%%%
\subsection{Flags}
\label{sec:flags}

The package makes it easy to generate different versions
of the main or child documents.
To this end compilation flags can be defined
and assigned different default values.
They will be particularly useful in conjunction
with the forwarding mechanism described in \secref{sec:forward}.

For example, it may be useful to have a flag |\version|
which can be set to |draft| or |final|.
The document source will contain some conditional code
depending on the value of |\version|.
Suppose further, the flag should default to |final| for the main file
and to |draft| for child files
which is a natural assignment for editing the document.
This is achieved by placing the following code
in the preamble of the main document
(below the |\childdocmain| directive):
%
\begin{center}
\begin{tabular}{l}
|\ifchilddoc|\\
|\providecommand{\version}{draft}|\\
|\||else|\\
|\providecommand{\version}{final}|\\
|\||fi|
\end{tabular}
\end{center}
%
The definition by |\providecommand| makes sure
that previous definitions are not overwritten.
Further statements |\providecommand{\version}{...}|
can thus be added before the above code to override it.

For the main file, one might add a line
(between |\childdocmain| and the above block)
%
\begin{center}
|%\ifchilddoc\||else\providecommand{\version}{draft}\||fi|
\end{center}
%
which can be uncommented to produce a draft version.
Likewise one can add a line to the very top of a child file
(above the |\childdocof{|\textit{main}|}| directive)
%
\begin{center}
|%\providecommand{\version}{final}|
\end{center}
%
which can be uncommented to produce the final version of this child document.

%%%%%%%%%%%%%%%%%%%%%%%%%%%%%%%%%%%%%%%%%%%%%%%%%%%%%%%%%%%%%%%%%%%%%%%%%%%%%%%%
\subsection{Forwarding}
\label{sec:forward}

Different versions of the main or child documents
using compilation flags as described in \secref{sec:flags}
can be (permanently) stored in different files
for convenient compilation, viewing and distribution.
To this end, the package defines a command
to pass on compilation to a different file:

%%%%%%%%%%%%%%%%%%%%%%%%%%%%%%%%%%%%%%%%
\DescribeMacro{\childdocforward}
The command |\childdocforward| redirects processing to
another source file:
%
\begin{center}
\begin{tabular}{l}
|\input{childdoc.def}|\\
|\childdocforward[|\textit{main}|]{|\textit{dest}|}|\\
\end{tabular}
\end{center}
%
The argument \textit{dest} is the destination file
(without extension).
It should be the main file or one of the child files.
Note that further \textsf{childdoc} directives
such as |\childdocof| and |\childdocforward|
in the indicated file will be processed in this form.
The optional argument \textit{main}
passes on directly to the main file \textit{main}
while pretending to compile the child \textit{dest}.
This form behaves as if \textit{dest}
issues |\childdocof{|\textit{main}|}| right away,
and no further \textsf{childdoc} directives will be processed.

%%%%%%%%%%%%%%%%%%%%%%%%%%%%%%%%%%%%%%%%
\DescribeMacro{\...prefix}
In the alternative form |\childdocforwardprefix|,
%
\begin{center}
\begin{tabular}{l}
|\input{childdoc.def}|\\
|\childdocforwardprefix[|\textit{main}|]{|\textit{prefix}|}{|\textit{dest}|}|
\end{tabular}
\end{center}
%
the destination file is determined by a pattern
depending on the current file:
To make this work, the current file must be called
`{\textit{prefix}\hspace{0.2em}\textit{suffix}}'
with \textit{prefix} matching precisely the argument.
Processing is then passed on to the file
`{\textit{dest}\hspace{0.2em}\textit{suffix}}'.
Surely, the same effect is achieved by
directly specifying the
argument `{\textit{dest}\hspace{0.2em}\textit{suffix}}'
in the first form.
However, that requires to set up a different file
for each child. With the alternative form of the command
all these files can have exactly the same content
which simplifies setting them up and maintaining them.

For example, the following file |draft.tex|
with a compilation flag |\version| as described in \secref{sec:flags}
compiles the main document as a draft:
%
\begin{center}
\begin{tabular}{l}
|\def\version{draft}|\\
|\input{childdoc.def}|\\
|\childdocforward{|\textit{main}|}|
\end{tabular}
\end{center}
%
Likewise, the following files |final|\textit{nn}|.tex|
compile the final version of the child document
|child|\textit{nn}|.tex|:
%
\begin{center}
\begin{tabular}{l}
|\def\version{final}|\\
|\input{childdoc.def}|\\
|\childdocforwardprefix{final}{child}|
\end{tabular}
\end{center}
%

Note that when several versions of a main file and/or of each child file
are to be generated, it may be convenient to set up a |Makefile| or
shell script to automatise the process.

%%%%%%%%%%%%%%%%%%%%%%%%%%%%%%%%%%%%%%%%%%%%%%%%%%%%%%%%%%%%%%%%%%%%%%%%%%%%%%%%
\subsection{Command Line Processing}
\label{sec:commandline}

The effect of redirection files can also be achieved by invoking
the \LaTeX{} compiler with a more elaborate command line.
Most conveniently this should be done as part
of a shell script or a |Makefile|.

When using \textsf{childdoc} in the main file, the following
command lines effectively perform a redirection
(note that depending on the shell being used,
backslashes may have to be doubled: `|\|' $\to$ `|\\|'):
%
\begin{center}
|... -jobname "|\textit{target}|" |\\|"|[\textit{flags}]%
|\input{childdoc.def}\childdocforward[|\textit{main}|]{|\textit{dest}|}"|
\end{center}
%
Here \textit{target} is the name of the output file,
\textit{main} is the name of the main file
and \textit{dest} is the name of the main or child file to be processed
(all filenames without extensions).
The optional argument \textit{main} can be omitted
if \textit{main} matches \textit{dest}.
Optionally, compilation \textit{flags} can be defined via |\def| commands.
This command line makes the \TeX{} engine believe
it is compiling the file \textit{target}
whose content is specified as the latter parameter.
The provided code then forwards the processing to
\textit{main} or \textit{dest} as described in \secref{sec:forward}.

%%%%%%%%%%%%%%%%%%%%%%%%%%%%%%%%%%%%%%%%%%%%%%%%%%%%%%%%%%%%%%%%%%%%%%%%%%%%%%%%
\subsection{Include by Input}
\label{sec:input}

Including child documents by |\include| has some restrictions by design.
Most notably, the content of a child document always occupies
its own set of pages; pages cannot be shared between child documents.
Usually, this behaviour makes perfect sense
because each child document contain an essential part of the document.
However, in some situations it may be desirable to compose
a document from a collection of parts
without having mandatory page breaks between then.
For this case, the package
provides a mechanism to include parts
by |\input| which can also be processed individually.
However, by construction this mechanism
requires manual handling of the content to be output.

%%%%%%%%%%%%%%%%%%%%%%%%%%%%%%%%%%%%%%%%
\DescribeMacro{\ifchilddocmanual}
The main file should be prepared as usual, see \secref{sec:include}.
However, the document body must make a distinction
between processing of an individual part and of the main document, e.g.:
%
\begin{center}
\begin{tabular}{l}
|\ifchilddocmanual|\\
|\input{\childdocname}|\\
|\||else|\\
\textit{document body with }|\input{|\textit{part}|}|\\
|\||fi|
\end{tabular}
\end{center}
%
The conditional |\ifchilddocmanual| is true whenever
a part to be included by |\input| is being compiled,
and the name of the part is stored in |\childdocname|.

%%%%%%%%%%%%%%%%%%%%%%%%%%%%%%%%%%%%%%%%
\DescribeMacro{\childdocby}
Each part to be included by |\input| should start with:
%
\begin{center}
\begin{tabular}{l}
|\input{childdoc.def}|\\
|\childdocby{|\textit{main}|}|\\
\end{tabular}
\end{center}
%
The directive |\childdocby| is similar to |\childdocof|
described in \secref{sec:include},
but the subsequent selection of content must be done manually.
To that end, both |\ifchilddoc| and |\ifchilddocmanual|
will be true upon processing of a part,
and the name of the part is stored in |\childdocname|.
Note that |\jobname| will be set to the filename of the current part
so that each part receives an individual |.aux| file
that does not interfere with the |.aux| file(s) of the main document.
This behaviour can be altered by the alternative form
|\childdocby[*]{|\textit{main}|}| (with a non-empty optional argument)
which uses the |.aux| file of the main document
by setting |\jobname| to \textit{main}.

%%%%%%%%%%%%%%%%%%%%%%%%%%%%%%%%%%%%%%%%%%%%%%%%%%%%%%%%%%%%%%%%%%%%%%%%%%%%%%%%
\subsection{Driver Development}
\label{sec:driver}

The \textsf{childdoc} mechanism can also be use for the development
of definition files such as \LaTeX{} styles or classes.
This case differs from the above setup with multiple parts
included by |\include| in that no |\includeonly| should be invoked.
This can be achieved by starting the include file
(before |\ProvidesPackage|) with:
%
\begin{center}
\begin{tabular}{l}
|\input{childdoc.def}|\\
|\childdocforward{|\textit{main}|}|\\
\end{tabular}
\end{center}
%
or alternatively with:
%
\begin{center}
\begin{tabular}{l}
|\input{childdoc.def}|\\
|\childdocby{|\textit{main}|}|\\
\end{tabular}
\end{center}
%
Both forms have slightly different effects as described above.
The main file is prepared as usual, see \secref{sec:include}.

%%%%%%%%%%%%%%%%%%%%%%%%%%%%%%%%%%%%%%%%%%%%%%%%%%%%%%%%%%%%%%%%%%%%%%%%%%%%%%%%
\subsection{Legacy Detection}
\label{sec:detection}

The directive |\childdocmain| in the main file can detect
whether the complete document or merely a child is to be compiled
even without using the directive |\childdocof|.
This method is deprecated because it is less robust
and there is no compelling reason to use it;
it is merely provided for backward compatibility
and it may be removed in future versions.

If the detection mechanism is to be used,
it is mandatory to correctly specify
the filename of the main file as the argument of |\childdocmain|:
%
\begin{center}
\begin{tabular}{l}
|\input{childdoc.def}|\\
|\childdocmain{|\textit{main}|}|\\
\end{tabular}
\end{center}
%
If |\jobname| does not match the argument \textit{main} of |\childdocmain|,
it is assumed that |\jobname| points to the child file to be compiled.
When using |\childdocmain| with the main file specified as argument,
it suffices to start a child file
with just |\input{|\textit{main}|}|
without loading of the package and using |\childdocof|.
If instead all processing is done
with the appropriate \textsf{childdoc} directives,
the argument of \textit{main} of |\childdocmain| can be empty.

An alternative version of the command line processing described
in \secref{sec:commandline} using the detection mechanism reads:
%
\begin{center}
|... -jobname "|\textit{target}|" "|[\textit{flags}]%
[|\def\jobname{|\textit{dest}|}|]|\input{|\textit{main}|}"|
\end{center}

%%%%%%%%%%%%%%%%%%%%%%%%%%%%%%%%%%%%%%%%%%%%%%%%%%%%%%%%%%%%%%%%%%%%%%%%%%%%%%%%
\subsection{Manual Code}
\label{sec:manual}

In case one cannot be certain whether the definitions file |childdoc.def|
is installed on the target \TeX{} distribution
and one prefers not to ship it,
it is conceivable to paste a few relevant commands into the sources.

To that end, drop all statements |\input{childdoc.def}|
and perform the replacements as outlined below.
Instead of |\childdocmain{|\textit{main}|}| add the following code
to the top of the main file:
%
\begin{center}
\begin{tabular}{l}
|\||ifdefined\childdocname\endinput\||fi\newif\ifchilddoc|\\
|\edef\childdocname{\scantokens\expandafter{\jobname\noexpand}}|\\
|\def\childdocmain{|\textit{main}|}\||ifx\childdocmain\childdocname\||else|\\
|\childdoctrue\includeonly{\childdocname}\let\jobname\childdocmain\||fi|\\
\end{tabular}
\end{center}
%
Instead of |\childdocof{|\textit{main}|}| just include the main file
at the top of each child file:
%
\begin{center}
|\input{|\textit{main}|}|
\end{center}
%
A simple redirection |\childdocforward{|\textit{dest}|}| is achieved by:
%
\begin{center}
|\def\jobname{|\textit{dest}|}\input{\jobname}|
\end{center}
%
The redirection with prefix
|\childdocforwardprefix[|\textit{prefix}|]{|\textit{dest}|}|
is accomplished by:
%
\begin{center}
\begin{tabular}{l}
|{\edef\jobname{\scantokens\expandafter{\jobname\noexpand}}|\\
|\def\redirectjob |\textit{prefix}|#1~~~{\gdef\jobname{|\textit{dest}|#1}}|\\
|\expandafter\redirectjob\jobname~~~}\input{\jobname}|
\end{tabular}
\end{center}

In an alternative approach,
child documents can be compiled by a specific command line
without additional code or specific definitions:
%
\begin{center}
|... -jobname "|\textit{target}|" "|[\textit{flags}]%
|\includeonly{|\textit{dest}|}\input{|\textit{main}|}"|
\end{center}
%

%%%%%%%%%%%%%%%%%%%%%%%%%%%%%%%%%%%%%%%%%%%%%%%%%%%%%%%%%%%%%%%%%%%%%%%%%%%%%%%%
%%%%%%%%%%%%%%%%%%%%%%%%%%%%%%%%%%%%%%%%%%%%%%%%%%%%%%%%%%%%%%%%%%%%%%%%%%%%%%%%
\section{Information}

%%%%%%%%%%%%%%%%%%%%%%%%%%%%%%%%%%%%%%%%%%%%%%%%%%%%%%%%%%%%%%%%%%%%%%%%%%%%%%%%
\subsection{Copyright}

Copyright \copyright{} 2017--2018 Niklas Beisert

This work may be distributed and/or modified under the
conditions of the \LaTeX{} Project Public License, either version 1.3
of this license or (at your option) any later version.
The latest version of this license is in
  \url{http://www.latex-project.org/lppl.txt}
and version 1.3 or later is part of all distributions of \LaTeX{}
version 2005/12/01 or later.

This work has the LPPL maintenance status `maintained'.

The Current Maintainer of this work is Niklas Beisert.

This work consists of the files |README.txt|, |childdoc.ins| and |childdoc.dtx|
as well as the derived files |childdoc.def|, |cdocsamp.tex|
with |cdocsch1.tex|, |cdocsch2.tex|, |cdocspt3.tex|, |cdocspt4.tex|,
|cdocsdrf.tex|, |cdocsfn1.tex|, |cdocsfn2.tex|
as well as |childdoc.pdf|.

%%%%%%%%%%%%%%%%%%%%%%%%%%%%%%%%%%%%%%%%%%%%%%%%%%%%%%%%%%%%%%%%%%%%%%%%%%%%%%%%
\subsection{Files and Installation}

The package consists of the files:
%
\begin{center}
\begin{tabular}{ll}
    |README.txt|   & readme file \\
    |childdoc.ins| & installation file \\
    |childdoc.dtx| & source file \\
    |childdoc.def| & definition file \\
    |cdocsamp.tex| & sample main file \\
    |cdocsch1.tex| & sample include file \\
    |cdocsch2.tex| & sample include file \\
    |cdocspt3.tex| & sample part file \\
    |cdocspt4.tex| & sample part file \\
    |cdocsdrf.tex| & sample redirection file \\
    |cdocsfn1.tex| & sample redirection file \\
    |cdocsfn2.tex| & sample redirection file \\
    |childdoc.pdf| & manual
\end{tabular}
\end{center}
%
The distribution consists of the files
|README.txt|, |childdoc.ins| and |childdoc.dtx|.
%
\begin{itemize}
\item
Run (pdf)\LaTeX{} on |childdoc.dtx|
to compile the manual |childdoc.pdf| (this file).
\item
Run \LaTeX{} on |childdoc.ins| to create the definitions file |childdoc.def|
and the sample |cdocsamp.tex| with include files
|cdocsch1.tex|, |cdocsch2.tex|, |cdocspt3.tex|, |cdocspt4.tex|,
|cdocsdrf.tex|, |cdocsfn1.tex|, |cdocsfn2.tex|.
Then copy the file |childdoc.def| to an appropriate directory of your \LaTeX{}
distribution, e.g.\ \textit{texmf-root}|/tex/latex/childdoc|.
\end{itemize}

%%%%%%%%%%%%%%%%%%%%%%%%%%%%%%%%%%%%%%%%%%%%%%%%%%%%%%%%%%%%%%%%%%%%%%%%%%%%%%%%
\subsection{Related CTAN Packages}

There are several other packages which offer a similar functionality:
%
\begin{itemize}
\item
The packages
\href{http://ctan.org/pkg/docmute}{\textsf{docmute}},
\href{http://ctan.org/pkg/includex}{\textsf{includex}} and
\href{http://ctan.org/pkg/standalone}{\textsf{standalone}}
provide commands to include only the document body of
a child file thus allowing both files to be compiled individually.
\item
The packages \href{http://ctan.org/pkg/subdocs}{\textsf{subdocs}}
and \href{http://ctan.org/pkg/subfiles}{\textsf{subfiles}}
provide structures in which the main and child documents can be
encapsulated and allowing them to be compiled individually.
The inclusion mechanism is different from the conventional |\include|.
\item
The package \href{http://ctan.org/pkg/combine}{\textsf{combine}}
is an elaborate solution to combine several documents into one.
\end{itemize}
%
See also the CTAN topic \href{http://ctan.org/topic/subdocs}{\textsf{subdocs}}
for further related packages.
The present package differs from the above solutions in that
a document structure constructed with the conventional |\include| mechanism
just needs two extra commands at the top of every file
such that all constituent files can be compiled individually.

%%%%%%%%%%%%%%%%%%%%%%%%%%%%%%%%%%%%%%%%%%%%%%%%%%%%%%%%%%%%%%%%%%%%%%%%%%%%%%%%
%\subsection{Feature Suggestions}
%
%The following is a list of features which may be useful for future
%versions of this package:
%%
%\begin{itemize}
%\item
%\ldots
%\end{itemize}

%%%%%%%%%%%%%%%%%%%%%%%%%%%%%%%%%%%%%%%%%%%%%%%%%%%%%%%%%%%%%%%%%%%%%%%%%%%%%%%%
\subsection{Revision History}

%%%%%%%%%%%%%%%%%%%%%%%%%%%%%%%%%%%%%%%%
\paragraph{v2.0:} 2018/12/30

\begin{itemize}
\item
immediate forward processing
\item
added |\childdocby| mechanism
\item
manual restructured
\end{itemize}

%%%%%%%%%%%%%%%%%%%%%%%%%%%%%%%%%%%%%%%%
\paragraph{v1.6:} 2018/01/17

\begin{itemize}
\item
application for development of include files
\item
corrections to manual
\end{itemize}

%%%%%%%%%%%%%%%%%%%%%%%%%%%%%%%%%%%%%%%%
\paragraph{v1.5:} 2017/05/21

\begin{itemize}
\item
more complete structuring introduced
\item
|\childdocof| introduced
\item
|\childdoc| renamed to |\childdocmain|
\item
|\childredirect| renamed to |\childdocforward| and |\childdocforwardprefix|
and functionality expanded
\end{itemize}

%%%%%%%%%%%%%%%%%%%%%%%%%%%%%%%%%%%%%%%%
\paragraph{v1.0:} 2017/04/27

\begin{itemize}
\item
manual and install package
\item
first version published on CTAN
\end{itemize}

%%%%%%%%%%%%%%%%%%%%%%%%%%%%%%%%%%%%%%%%
\paragraph{v0.6:} 2017/04/26

\begin{itemize}
\item
redirection mechanism added
\end{itemize}

%%%%%%%%%%%%%%%%%%%%%%%%%%%%%%%%%%%%%%%%
\paragraph{v0.5:} 2017/04/26

\begin{itemize}
\item
functionality in definition file
\end{itemize}


%%%%%%%%%%%%%%%%%%%%%%%%%%%%%%%%%%%%%%%%%%%%%%%%%%%%%%%%%%%%%%%%%%%%%%%%%%%%%%%%
%%%%%%%%%%%%%%%%%%%%%%%%%%%%%%%%%%%%%%%%%%%%%%%%%%%%%%%%%%%%%%%%%%%%%%%%%%%%%%%%
%%%%%%%%%%%%%%%%%%%%%%%%%%%%%%%%%%%%%%%%%%%%%%%%%%%%%%%%%%%%%%%%%%%%%%%%%%%%%%%%
\appendix

\settowidth\MacroIndent{\rmfamily\scriptsize 000\ }

 \DocInput{childdoc.dtx}

\end{document}
%</driver>
% \fi
%
% %%%%%%%%%%%%%%%%%%%%%%%%%%%%%%%%%%%%%%%%%%%%%%%%%%%%%%%%%%%%%%%%%%%%%%%%%%%%%%
% %%%%%%%%%%%%%%%%%%%%%%%%%%%%%%%%%%%%%%%%%%%%%%%%%%%%%%%%%%%%%%%%%%%%%%%%%%%%%%
% \section{Sample}
%\iffalse
%<*samplemain>
%\fi
%
% The following presents a sample document
% with two chapters, two parts, a title page,
% a compile flag as well as three forwarding files to set the flag.
% It consists of eight |.tex| files:
% \begin{center}
% \begin{tabular}{ll}
% |cdocsamp.tex|&main file\\
% |cdocsch1.tex|&include file for chapter 1\\
% |cdocsch2.tex|&include file for chapter 2\\
% |cdocspt3.tex|&include file for part 3\\
% |cdocspt4.tex|&include file for part 4\\
% |cdocsdrf.tex|&forwarding file for main file in draft mode\\
% |cdocsfi1.tex|&forwarding file for final version of chapter 1\\
% |cdocsfi2.tex|&forwarding file for final version of chapter 2\\
% \end{tabular}
% \end{center}
% Each of the eight files can be compiled directly by the \LaTeX{} compiler.
%
% %%%%%%%%%%%%%%%%%%%%%%%%%%%%%%%%%%%%%%
% \paragraph{Main File.}
%
% The main file is called |cdocsamp.tex|.
%
% Load the \textsf{childdoc} definitions and
% declare the filename for the main document:
%    \begin{macrocode}
\input{childdoc.def}
\childdocmain{}
%    \end{macrocode}

% Optional override for |\version| flag:
%    \begin{macrocode}
%%\ifchilddoc\else\providecommand{\version}{draft}\fi
%    \end{macrocode}

% Define the default values for the |\version| flag
% (|final| for the main file and |draft| for childs):
%    \begin{macrocode}
\ifchilddoc
\providecommand{\version}{draft}
\else
\providecommand{\version}{final}
\fi
%    \end{macrocode}

% Load the standard document class:
%    \begin{macrocode}
\documentclass[12pt]{article}
%    \end{macrocode}

% Start the document body:
%    \begin{macrocode}
\begin{document}
%    \end{macrocode}

% Declare a title page.
% Print title, part of document being processed and version flag:
%    \begin{macrocode}
\addtocounter{page}{-1}
\begin{center}
{\LARGE\bfseries{}childdoc example\par}
\vspace{1cm}
\ifchilddoc
\ifchilddocmanual part\else chapter\fi:
`\childdocname' of `\childdocjob'\par
\else
main document: `\childdocjob'\par
\fi
version: \version\par
\end{center}
\newpage
%    \end{macrocode}

% Manually include selected file,
% otherwise process as usual:
%    \begin{macrocode}
\ifchilddocmanual
\section*{part `\childdocname'}
\input{\childdocname}
\else
%    \end{macrocode}

% Include the two chapters:
%    \begin{macrocode}
\include{cdocsch1}
\include{cdocsch2}
%    \end{macrocode}

% Include the two parts unless only chapters should be displayed:
%    \begin{macrocode}
\ifchilddoc\else
\section{part three}
\input{cdocspt3}
\section{part four}
\input{cdocspt4}
\fi
%    \end{macrocode}

% Process as usual until here:
%    \begin{macrocode}
\fi
%    \end{macrocode}

% End of document body:
%    \begin{macrocode}
\end{document}
%    \end{macrocode}
%\iffalse
%</samplemain>
%\fi
%
% %%%%%%%%%%%%%%%%%%%%%%%%%%%%%%%%%%%%%%
% \paragraph{Chapter Include Files.}
%
% The include files are called |cdocsch1.tex| and |cdocsch2.tex|.
%
%\iffalse
%<*samplechap1|samplechap2>
%\fi

% Optional override for |\version| flag:
%    \begin{macrocode}
%%\providecommand{\version}{final}
%    \end{macrocode}

% Include the main document:
%    \begin{macrocode}
\input{childdoc.def}
\childdocof{cdocsamp}
%    \end{macrocode}

%\iffalse
%</samplechap1|samplechap2>
%\fi
%
%\iffalse
%<*samplechap1>
%\fi
% Some text for chapter 1:
%    \begin{macrocode}
\section{one}
some text in chapter one
%    \end{macrocode}

%\iffalse
%</samplechap1>
%\fi
% Some text for chapter 2:
%\iffalse
%<*samplechap2>
%\fi
%    \begin{macrocode}
\section{two}
more text in chapter two
%    \end{macrocode}

%\iffalse
%</samplechap2>
%\fi
%
% %%%%%%%%%%%%%%%%%%%%%%%%%%%%%%%%%%%%%%
% \paragraph{Part Include Files.}
%
% The include files are called |cdocspt3.tex| and |cdocspt4.tex|.
%
%\iffalse
%<*samplepart3|samplepart4>
%\fi

% Optional override for |\version| flag:
%    \begin{macrocode}
%%\providecommand{\version}{final}
%    \end{macrocode}

% Include the main document:
%    \begin{macrocode}
\input{childdoc.def}
\childdocby{cdocsamp}
%    \end{macrocode}

%\iffalse
%</samplepart3|samplepart4>
%\fi
%
%\iffalse
%<*samplepart3>
%\fi
% Some text for part 3:
%    \begin{macrocode}
some text in part three
%    \end{macrocode}

%\iffalse
%</samplepart3>
%\fi
% Some text for part 4:
%\iffalse
%<*samplepart4>
%\fi
%    \begin{macrocode}
more text in part four
%    \end{macrocode}

%\iffalse
%</samplepart4>
%\fi
%
% %%%%%%%%%%%%%%%%%%%%%%%%%%%%%%%%%%%%%%
% \paragraph{Forwarding for a Complete Draft.}
%
% The following forwarding file |cdocsdrf.tex|
% compiles the main document in draft mode:
%\iffalse
%<*sampledraft>
%\fi
%    \begin{macrocode}
\def\version{draft}
\input{childdoc.def}
\childdocforward{cdocsamp}
%    \end{macrocode}

%\iffalse
%</sampledraft>
%\fi
%
% %%%%%%%%%%%%%%%%%%%%%%%%%%%%%%%%%%%%%%
% \paragraph{Forwarding for Final Version of the Chapters.}
%
% The following forwarding files |cdocsfn1.tex| and |cdocsfn2.tex|
% (with identical content)
% compile the final versions of the child documents
% |cdocsch1.tex| and |cdocsch2.tex|, respectively:
%\iffalse
%<*samplefinal>
%\fi
%    \begin{macrocode}
\def\version{final}
\input{childdoc.def}
\childdocforwardprefix[cdocsamp]{cdocsfn}{cdocsch}
%    \end{macrocode}

%\iffalse
%</samplefinal>
%\fi
%
% %%%%%%%%%%%%%%%%%%%%%%%%%%%%%%%%%%%%%%
% \paragraph{Command Line Processing.}
%
% The following three command lines generate the output files
% |cdocscld|, |cdocscl1| and |cdocscl2|
% which should be identical to
% |cdocsdrf|, |cdocsch1| and |cdocsfn2|, respectively:
% \begin{center}
% \begin{tabular}{l}
% |latex -jobname cdocscld \|\\
% |  "\def\version{draft}\input{childdoc.def}\childdocforward{cdocsamp}"|\\
% |latex -jobname cdocscl1 \|\\
% |  "\input{childdoc.def}\childdocforward[cdocsamp]{cdocsch1}"|\\
% |latex -jobname cdocscl2 \|\\
% |  "\def\version{final}\input{childdoc.def}\childdocforward{cdocsch2}"|
% \end{tabular}
% \end{center}
% Note that the trailing backslash on each first line
% merely continues the input to the second line
% (for convenient cut ant paste).
% Furthermore, the command |latex| can be replaced by any
% of its alternative versions such as |pdflatex|.
%
% %%%%%%%%%%%%%%%%%%%%%%%%%%%%%%%%%%%%%%%%%%%%%%%%%%%%%%%%%%%%%%%%%%%%%%%%%%%%%%
% %%%%%%%%%%%%%%%%%%%%%%%%%%%%%%%%%%%%%%%%%%%%%%%%%%%%%%%%%%%%%%%%%%%%%%%%%%%%%%
% \section{Implementation}
%\iffalse
%<*package>
%\fi
%
% This section describes the definitions file |childdoc.def|.

% The definitions cannot be loaded using |\usepackage| or |\RequirePackage|
% which has a mechanism to prevent loading a style file more than once.
% When loading the definitions by means of |\input|
% multiple instances have to be prevented manually:
%\iffalse
%This code needs to be before the `\ProvidesFile' directive
%which is defined at the beginning of this file.
%Therefore it is also placed there and commented out here.
%</package>
%<*discard>
%\fi
%    \begin{macrocode}
\ifdefined\childdocmain\endinput\fi
%    \end{macrocode}
%\iffalse
%</discard>
%<*package>
%\fi
%
% \macro{\ifchilddoc}
% \macro{\ifchilddocmanual}
% The conditional |\ifchilddoc| tells whether a
% child (true) or main (false) document is being compiled.
% The conditional |\ifchilddocmanual| tells whether
% the |\includeonly| mechanism is used (false) or
% the selection of child files must be performed manually (true).
% The definitions initialise to false:
%    \begin{macrocode}
\newif\ifchilddoc
\newif\ifchilddocmanual
%    \end{macrocode}

% \macro{\childdocname}
% \macro{\childdocjob}
% The macro |\childdocname| stores the name of the main document
% to be compiled. The macro |\childdocjob| stores the name of
% the document on which the \LaTeX{} compiler was originally invoked.
% The content of |\jobname| cannot be compared
% to filenames specified in the source due to different catcodes.
% The following code rescans |\jobname|, stores the result
% in |\childdocname| and saves a copy in |\childdocjob|:
%    \begin{macrocode}
\edef\childdocname{\scantokens\expandafter{\jobname\noexpand}}
\let\childdocjob\childdocname
%    \end{macrocode}

% \macro{\childdocdisable}
% The macro |\childdocdisable| prevents the main file
% from being processed more than once.
% At this stage, the main document command |\childdocmain|
% is assumed to be called once again where it should do nothing.
% Any subsequent call to it should prevent
% a secondary processing of the main document
% It overwrites the forwarding commands
% |\childdocof| and |\childdocforward|
% with empty macros to prevent further inclusions of the main document:
%    \begin{macrocode}
\newcommand{\childdocdisable}
{
  \renewcommand{\childdocmain}[1]{\renewcommand{\childdocmain}[1]{\endinput}}
  \renewcommand{\childdocof}[1]{}
  \renewcommand{\childdocby}[2][]{}
  \renewcommand{\childdocforward}[2][]{}
  \renewcommand{\childdocdisable}{}
}
%    \end{macrocode}

% \macro{\childdocmain}
% The macro |\childdocmain| is to be called at the top of the main file
% with nothing or the main filename (without extension) as argument.
% First, it breaks loops.
% If the argument is not empty and does not match |\childdocname|
% (which is set by the first inclusion of |childdoc.def|),
% |\ifchilddoc| is set to true, |\includeonly| is applied to the child file
% and |\jobname| is set to the main file
% (for proper handling of |.aux| files):
%    \begin{macrocode}
\newcommand{\childdocmain}[1]
{
  \childdocdisable\childdocmain{}
  \if?#1?\else
    \begingroup
      \def\childdoctmp{#1}
      \ifx\childdoctmp\childdocname
        \def\childdoctmp{}
      \else
        \def\childdoctmp
        {
          \childdoctrue
          \includeonly{\childdocname}
          \def\childdocjob{#1}
          \def\jobname{#1}
        }
      \fi
      \expandafter
    \endgroup
    \childdoctmp
  \fi
}
%    \end{macrocode}

% \macro{\childdocof}
% The command |\childdocof| redirects
% compilation to the main file |#1|.
%    \begin{macrocode}
\newcommand{\childdocof}[1]
{
  \childdocdisable
  \childdoctrue
  \includeonly{\childdocname}
  \def\jobname{#1}
  \def\childdocjob{#1}
  \input{#1}
}
%    \end{macrocode}

% \macro{\childdocby}
% The command |\childdocby| ....
%    \begin{macrocode}
\newcommand{\childdocby}[2][]
{
  \childdocdisable
  \childdoctrue
  \childdocmanualtrue
  \if?#1?\else
    \def\jobname{#2}
  \fi
  \def\childdocjob{#2}
  \input{#2}
  \endinput
}
%    \end{macrocode}

% \macro{\childdocforward}
% The command |\childdocforward| redirects
% compilation to the main file or
% (if the optional argument is given) a child file.
% Parameters are set as if the main file
% or a child file starting with |\childdocof| was compiled.
% Then compilation is handed over to the main file:
%    \begin{macrocode}
\newcommand{\childdocforward}[2][]
{
  \begingroup
    \if?#1?
      \def\childdoctmp
      {
        \def\childdocname{#2}
        \def\childdocjob{#2}
        \def\jobname{#2}
        \input{#2}
        \endinput
      }
    \else
      \def\childdoctmp
      {
        \childdocdisable
        \def\childdocname{#2}
        \childdoctrue
        \includeonly{#2}
        \def\childdocjob{#1}
        \def\jobname{#1}
        \input{#1}
        \endinput
      }
    \fi
    \expandafter
  \endgroup
  \childdoctmp
}
%    \end{macrocode}

% \macro{\childdocforwardprefix}
% The command |\childdocforwardprefix| redirects
% compilation to the main or a child file by means of a pattern.
% The prefix |#1| in the current filename is replaced by |#2|
% and the suffix of the current filename is kept
% (it is assumed that the filename does not contain the substring `|~~~|'
% which is used as a delimiter).
% Compilation is handed over to the new file by |\childdocforward|:
%    \begin{macrocode}
\newcommand{\childdocforwardprefix}[3][]
{
  \begingroup
    \def\childdocextract #2##1~~~{\def\childdoctmp{\childdocforward[#1]{#3##1}}}
    \expandafter\childdocextract\childdocname~~~
    \expandafter
  \endgroup
  \childdoctmp
}
%    \end{macrocode}

% \macro{\childdoc}
% The deprecated macro |\childdoc| is a legacy version of |\childdocmain|:
%    \begin{macrocode}
\newcommand{\childdoc}{\childdocmain}
%    \end{macrocode}

% \macro{\childdocredirect}
% The deprecated macro |\childdocredirect| is a legacy version
% of |\childdocforward| and |\childdocforwardprefix|:
%    \begin{macrocode}
\newcommand{\childdocredirect}[2][]
{
  \begingroup
    \if?#1?
      \def\childdoctmp{\childdocforward{#2}}
    \else
      \def\childdoctmp{\childdocforwardprefix{#1}{#2}}
    \fi
    \expandafter
  \endgroup
  \childdoctmp
}
%    \end{macrocode}

%\iffalse
%</package>
%\fi
%
\endinput
|\\
|\childdocmain{|\textit{main}|}|\\
\end{tabular}
\end{center}
%
If |\jobname| does not match the argument \textit{main} of |\childdocmain|,
it is assumed that |\jobname| points to the child file to be compiled.
When using |\childdocmain| with the main file specified as argument,
it suffices to start a child file
with just |\input{|\textit{main}|}|
without loading of the package and using |\childdocof|.
If instead all processing is done
with the appropriate \textsf{childdoc} directives,
the argument of \textit{main} of |\childdocmain| can be empty.

An alternative version of the command line processing described
in \secref{sec:commandline} using the detection mechanism reads:
%
\begin{center}
|... -jobname "|\textit{target}|" "|[\textit{flags}]%
[|\def\jobname{|\textit{dest}|}|]|\input{|\textit{main}|}"|
\end{center}

%%%%%%%%%%%%%%%%%%%%%%%%%%%%%%%%%%%%%%%%%%%%%%%%%%%%%%%%%%%%%%%%%%%%%%%%%%%%%%%%
\subsection{Manual Code}
\label{sec:manual}

In case one cannot be certain whether the definitions file |childdoc.def|
is installed on the target \TeX{} distribution
and one prefers not to ship it,
it is conceivable to paste a few relevant commands into the sources.

To that end, drop all statements |% \iffalse
%
% childdoc.dtx Copyright (C) 2017-2018 Niklas Beisert
%
% This work may be distributed and/or modified under the
% conditions of the LaTeX Project Public License, either version 1.3
% of this license or (at your option) any later version.
% The latest version of this license is in
%   http://www.latex-project.org/lppl.txt
% and version 1.3 or later is part of all distributions of LaTeX
% version 2005/12/01 or later.
%
% This work has the LPPL maintenance status `maintained'.
%
% The Current Maintainer of this work is Niklas Beisert.
%
% This work consists of the files childdoc.dtx and childdoc.ins
% and the derived files childdoc.def and cdocsamp.tex with
% cdocsch1.tex, cdocsch2.tex, cdocsdrf.tex, cdocsfn1.tex, cdocsfn2.tex.
%
%<package>\ifdefined\childdocmain\endinput\fi
%<package>\ProvidesFile{childdoc.def}[2018/12/30 v2.0 child document driver]
%<samplemain>\ProvidesFile{cdocsamp.tex}[2018/12/30 v2.0 sample for childdoc]
%<*driver>
%\ProvidesFile{childdoc.drv}[2018/12/30 v2.0 childdoc reference manual file]
\PassOptionsToClass{10pt,a4paper}{article}
\documentclass{ltxdoc}

\usepackage[margin=35mm]{geometry}
\usepackage{hyperref}
\usepackage{hyperxmp}
\usepackage[usenames]{color}

\hypersetup{colorlinks=true}
\hypersetup{pdfstartview=FitH}
\hypersetup{pdfpagemode=UseNone}
\hypersetup{pdfsource={}}
\hypersetup{pdflang={en-UK}}
\hypersetup{pdfcopyright={Copyright 2017-2018 Niklas Beisert.
  This work may be distributed and/or modified under the
  conditions of the LaTeX Project Public License, either version 1.3
  of this license or (at your option) any later version.}}
\hypersetup{pdflicenseurl={http://www.latex-project.org/lppl.txt}}
\hypersetup{pdfcontactaddress={ETH Zurich, ITP, HIT K,
  Wolfgang-Pauli-Strasse 27}}
\hypersetup{pdfcontactpostcode={8093}}
\hypersetup{pdfcontactcity={Zurich}}
\hypersetup{pdfcontactcountry={Switzerland}}
\hypersetup{pdfcontactemail={nbeisert@itp.phys.ethz.ch}}
\hypersetup{pdfcontacturl={http://people.phys.ethz.ch/\xmptilde nbeisert/}}

\newcommand{\secref}[1]{\hyperref[#1]{section \ref*{#1}}}

\parskip1ex
\parindent0pt
\let\olditemize\itemize
\def\itemize{\olditemize\parskip0pt}

\begin{document}

\title{The \textsf{childdoc} Package}
\hypersetup{pdftitle={The childdoc Package}}
\author{Niklas Beisert\\[2ex]
  Institut f\"ur Theoretische Physik\\
  Eidgen\"ossische Technische Hochschule Z\"urich\\
  Wolfgang-Pauli-Strasse 27, 8093 Z\"urich, Switzerland\\[1ex]
  \href{mailto:nbeisert@itp.phys.ethz.ch}
  {\texttt{nbeisert@itp.phys.ethz.ch}}}
\hypersetup{pdfauthor={Niklas Beisert}}
\hypersetup{pdfsubject={Manual for the LaTeX2e Package childdoc}}
\date{30 December 2018, \textsf{v2.0}}
\maketitle

\begin{abstract}\noindent
\textsf{childdoc} is a \LaTeXe{} package
that enables the direct compilation
of document sections included by |\include|
to individual files.
\end{abstract}

\begingroup
\parskip0ex
\tableofcontents
\endgroup

%%%%%%%%%%%%%%%%%%%%%%%%%%%%%%%%%%%%%%%%%%%%%%%%%%%%%%%%%%%%%%%%%%%%%%%%%%%%%%%%
%%%%%%%%%%%%%%%%%%%%%%%%%%%%%%%%%%%%%%%%%%%%%%%%%%%%%%%%%%%%%%%%%%%%%%%%%%%%%%%%
\section{Introduction}

\LaTeX{} provides a mechanism to structure a large document (such as a book)
into a main file and several child files (containing the chapters)
using the |\include| command.
This mechanism is beneficial for documents
which span hundreds of pages in order to
make the source file(s) more manageable.
Moreover, compilation can be restricted to
selected child files by means of the |\includeonly| command.
The latter feature can be used to reduce the compilation time while editing
(this was significantly more useful in the earlier days of \LaTeX{})
or to generate a smaller document which is easier to navigate.
Another application of |\includeonly| is to generate
documents consisting of selected parts of the complete document.

However, there are a few drawbacks of the plain |\include| mechanism:
\begin{itemize}
\item
The child files cannot be compiled on their own,
they can only be compiled via the main file.
A naive editing environment
(such as a text editor with an option
to have the current file processed by \LaTeX)
may require one to switch to the main file before compiling;
attempting to compile the child file produces errors.
\item
The main file must be modified (each time)
to adjust the |\includeonly| command
to the present needs. This easily leaves the main file in a messy state.
\item
The generated document will always carry the filename
of the main document. This is inconvenient if
several child files are to be compiled and
to be kept for distribution.
\end{itemize}

The present package provides a simple interface
to make child files individually compilable by \LaTeX{}.
Compiling a child file then has the same effect as compiling
the main file with an |\includeonly| command
to select the appropriate child.
Moreover the generated document will carry the name of the child
rather than the main file.
This resolves all three above issues.

This feature is meant to make the editing of books,
thesis documents and lecture notes somewhat more convenient.
However, the package can also be used efficiently for
composing a series of documents (such as exercise sheets)
which are typically distributed individually.
It then assists the author in generating the individual documents
(potentially in different versions)
as well as a document containing the collected series.
Another application is in developing style files
or other kinds of included material
where compilation of the style file could redirect
to a sample or test file.

%%%%%%%%%%%%%%%%%%%%%%%%%%%%%%%%%%%%%%%%%%%%%%%%%%%%%%%%%%%%%%%%%%%%%%%%%%%%%%%%
%%%%%%%%%%%%%%%%%%%%%%%%%%%%%%%%%%%%%%%%%%%%%%%%%%%%%%%%%%%%%%%%%%%%%%%%%%%%%%%%
\section{Usage}

First of all, the package \textsf{childdoc} is \emph{not} a standard
\LaTeXe{} |.sty| style file! Therefore it needs to be invoked in
a non-standard way.

%%%%%%%%%%%%%%%%%%%%%%%%%%%%%%%%%%%%%%%%%%%%%%%%%%%%%%%%%%%%%%%%%%%%%%%%%%%%%%%%
\subsection{Included Files}
\label{sec:include}

%%%%%%%%%%%%%%%%%%%%%%%%%%%%%%%%%%%%%%%%
\DescribeMacro{\childdocmain}
To use the package, add the commands
\begin{center}
\begin{tabular}{l}
|\input{childdoc.def}|\\
|\childdocmain{}|\\
\end{tabular}
\end{center}
at the very top of the main \LaTeX{} file,
in particular \emph{before} the |\documentclass| statement!
The argument of |\childdocmain| should be left empty
(but it must be present).

%%%%%%%%%%%%%%%%%%%%%%%%%%%%%%%%%%%%%%%%
\DescribeMacro{\childdocof}
Furthermore, add the commands
\begin{center}
\begin{tabular}{l}
|\input{childdoc.def}|\\
|\childdocof{|\textit{main}|}|\\
\end{tabular}
\end{center}
at the top of every child file \textit{child}
which is included by |\include{|\textit{child}|}|
from within the main file
(or at least for those files to be compiled individually).
The argument \textit{main} must be the filename of the main file.

There are a couple of
considerations in setting up the main and child documents:

%%%%%%%%%%%%%%%%%%%%%%%%%%%%%%%%%%%%%%%%
\paragraph{Restrictions.}

Please note the following restrictions:
\begin{itemize}
\item
|\childdocmain| must be called with one argument \textit{main}
to ensure compatibility with earlier version of the package.
It must either be empty (|\childdocmain{}|)
or precisely match the filename of the main file in which it is specified.
See \secref{sec:detection} for further information.
\item
The filename \textit{main} must be specified without the |.tex| extension.
\item
The filename \textit{main} is case sensitive
(even in case-insensitive file systems)
due to internal string comparison.
\item
The argument \textit{main} should be fully expanded, it cannot be a macro.
\item
Subdirectories and special characters should be avoided in filenames.
\item
The command |\childdocmain{|\textit{main}|}| must be followed by a whitespace.
It should not be followed immediately by another command
or by a comment mark `|%|'.
This is because the \TeX{} parser reads the token immediately following
the argument of |\childdocmain| and puts it
at the beginning of every child section;
however, a white\-space is ignored.
\end{itemize}

%%%%%%%%%%%%%%%%%%%%%%%%%%%%%%%%%%%%%%%%
\paragraph{Content of Main File.}

It is advisable to place all content in the child files included by |\include|.
Any output contained in the main file will appear in all child documents
unless suppressed manually;
it cannot be suppressed automatically by the |\includeonly| directive
and thus should normally be avoided.
A method to include some content in the main file
by means of conditional processing is described in \secref{sec:conditional}.

%%%%%%%%%%%%%%%%%%%%%%%%%%%%%%%%%%%%%%%%
\paragraph{Page Numbering.}

When only a part of the document is compiled,
the appropriate numbering of pages
(as well as other status parameters)
is determined from the |.aux| files.
The latter contain information from previous passes.
However this information needs to propagate through
all intermediate child documents.
Therefore the page numbering in child documents may well
be inconsistent until the complete document is compiled at least once.

A useful (if unconventional) way to always ensure a consistent
page numbering is to restart the numbering in each child document
and denote the pages by `\textit{child}|.|\textit{page}'
where \textit{child} represents the chapter/section number of the child file.
This can be achieved by the command
|\numberwithin{page}{|\textit{child}|}|
of the \textsf{amsmath} package
where \textit{child} can be |chapter| or |section|
depending on the chosen structuring.
Alternatively, one can modify the macro |\thepage| appropriately
and reset the counter |page| at the start of each child file.

%%%%%%%%%%%%%%%%%%%%%%%%%%%%%%%%%%%%%%%%%%%%%%%%%%%%%%%%%%%%%%%%%%%%%%%%%%%%%%%%
\subsection{Conditional Processing}
\label{sec:conditional}

The package provides a mechanism to compile different versions
of a document. To customise the versions further some conditional processing
can come in handy to distinguish which version is being compiled.
The package provides two macros to describe the compilation context:

%%%%%%%%%%%%%%%%%%%%%%%%%%%%%%%%%%%%%%%%
\DescribeMacro{\ifchilddoc}
The conditional |\ifchilddoc| distinguishes between the compilation of
child documents and the main document:
%
\begin{center}
|\ifchilddoc |\textit{child-code}| |[|\||else |\textit{main-code}]| \||fi|
\end{center}

%%%%%%%%%%%%%%%%%%%%%%%%%%%%%%%%%%%%%%%%
\DescribeMacro{\childdocname}
\DescribeMacro{\childdocjob}
The macro |\childdocname| contains the filename (without extension)
of the main or child file being processed.
Note that |\childdocjob| will always contain the name of the main file.

%%%%%%%%%%%%%%%%%%%%%%%%%%%%%%%%%%%%%%%%
\paragraph{Title Page.}

Conditional processing can be used to include a title or banner page
in the main document when proper precautions are taken.
Importantly, the code in the main file should ensure that the page counter
(as well as other status parameters which are stored in the |.aux| files)
takes the same value after the conditional processing.
Otherwise the page numbers may take divergent values
depending on which part is compiled.

For example, a title page could be declared by:
%
\begin{center}
\begin{tabular}{l}
|\ifchilddoc\||else|\\
|\addtocounter{page}{-1}|\\
\textit{code for title page}\\
|\newpage|\\
|\||fi|
\end{tabular}
\end{center}
%
A banner page for the child documents can be generated by:
%
\begin{center}
\begin{tabular}{l}
|\ifchilddoc|\\
|\addtocounter{page}{-1}|\\
\textit{code for banner page}\\
|\newpage|\\
|\||fi|
\end{tabular}
\end{center}
%
Here one could write a message such as:
\begin{center}
|This is the part \childdocname{} of \childdocjob{}.|
\end{center}

%%%%%%%%%%%%%%%%%%%%%%%%%%%%%%%%%%%%%%%%%%%%%%%%%%%%%%%%%%%%%%%%%%%%%%%%%%%%%%%%
\subsection{Flags}
\label{sec:flags}

The package makes it easy to generate different versions
of the main or child documents.
To this end compilation flags can be defined
and assigned different default values.
They will be particularly useful in conjunction
with the forwarding mechanism described in \secref{sec:forward}.

For example, it may be useful to have a flag |\version|
which can be set to |draft| or |final|.
The document source will contain some conditional code
depending on the value of |\version|.
Suppose further, the flag should default to |final| for the main file
and to |draft| for child files
which is a natural assignment for editing the document.
This is achieved by placing the following code
in the preamble of the main document
(below the |\childdocmain| directive):
%
\begin{center}
\begin{tabular}{l}
|\ifchilddoc|\\
|\providecommand{\version}{draft}|\\
|\||else|\\
|\providecommand{\version}{final}|\\
|\||fi|
\end{tabular}
\end{center}
%
The definition by |\providecommand| makes sure
that previous definitions are not overwritten.
Further statements |\providecommand{\version}{...}|
can thus be added before the above code to override it.

For the main file, one might add a line
(between |\childdocmain| and the above block)
%
\begin{center}
|%\ifchilddoc\||else\providecommand{\version}{draft}\||fi|
\end{center}
%
which can be uncommented to produce a draft version.
Likewise one can add a line to the very top of a child file
(above the |\childdocof{|\textit{main}|}| directive)
%
\begin{center}
|%\providecommand{\version}{final}|
\end{center}
%
which can be uncommented to produce the final version of this child document.

%%%%%%%%%%%%%%%%%%%%%%%%%%%%%%%%%%%%%%%%%%%%%%%%%%%%%%%%%%%%%%%%%%%%%%%%%%%%%%%%
\subsection{Forwarding}
\label{sec:forward}

Different versions of the main or child documents
using compilation flags as described in \secref{sec:flags}
can be (permanently) stored in different files
for convenient compilation, viewing and distribution.
To this end, the package defines a command
to pass on compilation to a different file:

%%%%%%%%%%%%%%%%%%%%%%%%%%%%%%%%%%%%%%%%
\DescribeMacro{\childdocforward}
The command |\childdocforward| redirects processing to
another source file:
%
\begin{center}
\begin{tabular}{l}
|\input{childdoc.def}|\\
|\childdocforward[|\textit{main}|]{|\textit{dest}|}|\\
\end{tabular}
\end{center}
%
The argument \textit{dest} is the destination file
(without extension).
It should be the main file or one of the child files.
Note that further \textsf{childdoc} directives
such as |\childdocof| and |\childdocforward|
in the indicated file will be processed in this form.
The optional argument \textit{main}
passes on directly to the main file \textit{main}
while pretending to compile the child \textit{dest}.
This form behaves as if \textit{dest}
issues |\childdocof{|\textit{main}|}| right away,
and no further \textsf{childdoc} directives will be processed.

%%%%%%%%%%%%%%%%%%%%%%%%%%%%%%%%%%%%%%%%
\DescribeMacro{\...prefix}
In the alternative form |\childdocforwardprefix|,
%
\begin{center}
\begin{tabular}{l}
|\input{childdoc.def}|\\
|\childdocforwardprefix[|\textit{main}|]{|\textit{prefix}|}{|\textit{dest}|}|
\end{tabular}
\end{center}
%
the destination file is determined by a pattern
depending on the current file:
To make this work, the current file must be called
`{\textit{prefix}\hspace{0.2em}\textit{suffix}}'
with \textit{prefix} matching precisely the argument.
Processing is then passed on to the file
`{\textit{dest}\hspace{0.2em}\textit{suffix}}'.
Surely, the same effect is achieved by
directly specifying the
argument `{\textit{dest}\hspace{0.2em}\textit{suffix}}'
in the first form.
However, that requires to set up a different file
for each child. With the alternative form of the command
all these files can have exactly the same content
which simplifies setting them up and maintaining them.

For example, the following file |draft.tex|
with a compilation flag |\version| as described in \secref{sec:flags}
compiles the main document as a draft:
%
\begin{center}
\begin{tabular}{l}
|\def\version{draft}|\\
|\input{childdoc.def}|\\
|\childdocforward{|\textit{main}|}|
\end{tabular}
\end{center}
%
Likewise, the following files |final|\textit{nn}|.tex|
compile the final version of the child document
|child|\textit{nn}|.tex|:
%
\begin{center}
\begin{tabular}{l}
|\def\version{final}|\\
|\input{childdoc.def}|\\
|\childdocforwardprefix{final}{child}|
\end{tabular}
\end{center}
%

Note that when several versions of a main file and/or of each child file
are to be generated, it may be convenient to set up a |Makefile| or
shell script to automatise the process.

%%%%%%%%%%%%%%%%%%%%%%%%%%%%%%%%%%%%%%%%%%%%%%%%%%%%%%%%%%%%%%%%%%%%%%%%%%%%%%%%
\subsection{Command Line Processing}
\label{sec:commandline}

The effect of redirection files can also be achieved by invoking
the \LaTeX{} compiler with a more elaborate command line.
Most conveniently this should be done as part
of a shell script or a |Makefile|.

When using \textsf{childdoc} in the main file, the following
command lines effectively perform a redirection
(note that depending on the shell being used,
backslashes may have to be doubled: `|\|' $\to$ `|\\|'):
%
\begin{center}
|... -jobname "|\textit{target}|" |\\|"|[\textit{flags}]%
|\input{childdoc.def}\childdocforward[|\textit{main}|]{|\textit{dest}|}"|
\end{center}
%
Here \textit{target} is the name of the output file,
\textit{main} is the name of the main file
and \textit{dest} is the name of the main or child file to be processed
(all filenames without extensions).
The optional argument \textit{main} can be omitted
if \textit{main} matches \textit{dest}.
Optionally, compilation \textit{flags} can be defined via |\def| commands.
This command line makes the \TeX{} engine believe
it is compiling the file \textit{target}
whose content is specified as the latter parameter.
The provided code then forwards the processing to
\textit{main} or \textit{dest} as described in \secref{sec:forward}.

%%%%%%%%%%%%%%%%%%%%%%%%%%%%%%%%%%%%%%%%%%%%%%%%%%%%%%%%%%%%%%%%%%%%%%%%%%%%%%%%
\subsection{Include by Input}
\label{sec:input}

Including child documents by |\include| has some restrictions by design.
Most notably, the content of a child document always occupies
its own set of pages; pages cannot be shared between child documents.
Usually, this behaviour makes perfect sense
because each child document contain an essential part of the document.
However, in some situations it may be desirable to compose
a document from a collection of parts
without having mandatory page breaks between then.
For this case, the package
provides a mechanism to include parts
by |\input| which can also be processed individually.
However, by construction this mechanism
requires manual handling of the content to be output.

%%%%%%%%%%%%%%%%%%%%%%%%%%%%%%%%%%%%%%%%
\DescribeMacro{\ifchilddocmanual}
The main file should be prepared as usual, see \secref{sec:include}.
However, the document body must make a distinction
between processing of an individual part and of the main document, e.g.:
%
\begin{center}
\begin{tabular}{l}
|\ifchilddocmanual|\\
|\input{\childdocname}|\\
|\||else|\\
\textit{document body with }|\input{|\textit{part}|}|\\
|\||fi|
\end{tabular}
\end{center}
%
The conditional |\ifchilddocmanual| is true whenever
a part to be included by |\input| is being compiled,
and the name of the part is stored in |\childdocname|.

%%%%%%%%%%%%%%%%%%%%%%%%%%%%%%%%%%%%%%%%
\DescribeMacro{\childdocby}
Each part to be included by |\input| should start with:
%
\begin{center}
\begin{tabular}{l}
|\input{childdoc.def}|\\
|\childdocby{|\textit{main}|}|\\
\end{tabular}
\end{center}
%
The directive |\childdocby| is similar to |\childdocof|
described in \secref{sec:include},
but the subsequent selection of content must be done manually.
To that end, both |\ifchilddoc| and |\ifchilddocmanual|
will be true upon processing of a part,
and the name of the part is stored in |\childdocname|.
Note that |\jobname| will be set to the filename of the current part
so that each part receives an individual |.aux| file
that does not interfere with the |.aux| file(s) of the main document.
This behaviour can be altered by the alternative form
|\childdocby[*]{|\textit{main}|}| (with a non-empty optional argument)
which uses the |.aux| file of the main document
by setting |\jobname| to \textit{main}.

%%%%%%%%%%%%%%%%%%%%%%%%%%%%%%%%%%%%%%%%%%%%%%%%%%%%%%%%%%%%%%%%%%%%%%%%%%%%%%%%
\subsection{Driver Development}
\label{sec:driver}

The \textsf{childdoc} mechanism can also be use for the development
of definition files such as \LaTeX{} styles or classes.
This case differs from the above setup with multiple parts
included by |\include| in that no |\includeonly| should be invoked.
This can be achieved by starting the include file
(before |\ProvidesPackage|) with:
%
\begin{center}
\begin{tabular}{l}
|\input{childdoc.def}|\\
|\childdocforward{|\textit{main}|}|\\
\end{tabular}
\end{center}
%
or alternatively with:
%
\begin{center}
\begin{tabular}{l}
|\input{childdoc.def}|\\
|\childdocby{|\textit{main}|}|\\
\end{tabular}
\end{center}
%
Both forms have slightly different effects as described above.
The main file is prepared as usual, see \secref{sec:include}.

%%%%%%%%%%%%%%%%%%%%%%%%%%%%%%%%%%%%%%%%%%%%%%%%%%%%%%%%%%%%%%%%%%%%%%%%%%%%%%%%
\subsection{Legacy Detection}
\label{sec:detection}

The directive |\childdocmain| in the main file can detect
whether the complete document or merely a child is to be compiled
even without using the directive |\childdocof|.
This method is deprecated because it is less robust
and there is no compelling reason to use it;
it is merely provided for backward compatibility
and it may be removed in future versions.

If the detection mechanism is to be used,
it is mandatory to correctly specify
the filename of the main file as the argument of |\childdocmain|:
%
\begin{center}
\begin{tabular}{l}
|\input{childdoc.def}|\\
|\childdocmain{|\textit{main}|}|\\
\end{tabular}
\end{center}
%
If |\jobname| does not match the argument \textit{main} of |\childdocmain|,
it is assumed that |\jobname| points to the child file to be compiled.
When using |\childdocmain| with the main file specified as argument,
it suffices to start a child file
with just |\input{|\textit{main}|}|
without loading of the package and using |\childdocof|.
If instead all processing is done
with the appropriate \textsf{childdoc} directives,
the argument of \textit{main} of |\childdocmain| can be empty.

An alternative version of the command line processing described
in \secref{sec:commandline} using the detection mechanism reads:
%
\begin{center}
|... -jobname "|\textit{target}|" "|[\textit{flags}]%
[|\def\jobname{|\textit{dest}|}|]|\input{|\textit{main}|}"|
\end{center}

%%%%%%%%%%%%%%%%%%%%%%%%%%%%%%%%%%%%%%%%%%%%%%%%%%%%%%%%%%%%%%%%%%%%%%%%%%%%%%%%
\subsection{Manual Code}
\label{sec:manual}

In case one cannot be certain whether the definitions file |childdoc.def|
is installed on the target \TeX{} distribution
and one prefers not to ship it,
it is conceivable to paste a few relevant commands into the sources.

To that end, drop all statements |\input{childdoc.def}|
and perform the replacements as outlined below.
Instead of |\childdocmain{|\textit{main}|}| add the following code
to the top of the main file:
%
\begin{center}
\begin{tabular}{l}
|\||ifdefined\childdocname\endinput\||fi\newif\ifchilddoc|\\
|\edef\childdocname{\scantokens\expandafter{\jobname\noexpand}}|\\
|\def\childdocmain{|\textit{main}|}\||ifx\childdocmain\childdocname\||else|\\
|\childdoctrue\includeonly{\childdocname}\let\jobname\childdocmain\||fi|\\
\end{tabular}
\end{center}
%
Instead of |\childdocof{|\textit{main}|}| just include the main file
at the top of each child file:
%
\begin{center}
|\input{|\textit{main}|}|
\end{center}
%
A simple redirection |\childdocforward{|\textit{dest}|}| is achieved by:
%
\begin{center}
|\def\jobname{|\textit{dest}|}\input{\jobname}|
\end{center}
%
The redirection with prefix
|\childdocforwardprefix[|\textit{prefix}|]{|\textit{dest}|}|
is accomplished by:
%
\begin{center}
\begin{tabular}{l}
|{\edef\jobname{\scantokens\expandafter{\jobname\noexpand}}|\\
|\def\redirectjob |\textit{prefix}|#1~~~{\gdef\jobname{|\textit{dest}|#1}}|\\
|\expandafter\redirectjob\jobname~~~}\input{\jobname}|
\end{tabular}
\end{center}

In an alternative approach,
child documents can be compiled by a specific command line
without additional code or specific definitions:
%
\begin{center}
|... -jobname "|\textit{target}|" "|[\textit{flags}]%
|\includeonly{|\textit{dest}|}\input{|\textit{main}|}"|
\end{center}
%

%%%%%%%%%%%%%%%%%%%%%%%%%%%%%%%%%%%%%%%%%%%%%%%%%%%%%%%%%%%%%%%%%%%%%%%%%%%%%%%%
%%%%%%%%%%%%%%%%%%%%%%%%%%%%%%%%%%%%%%%%%%%%%%%%%%%%%%%%%%%%%%%%%%%%%%%%%%%%%%%%
\section{Information}

%%%%%%%%%%%%%%%%%%%%%%%%%%%%%%%%%%%%%%%%%%%%%%%%%%%%%%%%%%%%%%%%%%%%%%%%%%%%%%%%
\subsection{Copyright}

Copyright \copyright{} 2017--2018 Niklas Beisert

This work may be distributed and/or modified under the
conditions of the \LaTeX{} Project Public License, either version 1.3
of this license or (at your option) any later version.
The latest version of this license is in
  \url{http://www.latex-project.org/lppl.txt}
and version 1.3 or later is part of all distributions of \LaTeX{}
version 2005/12/01 or later.

This work has the LPPL maintenance status `maintained'.

The Current Maintainer of this work is Niklas Beisert.

This work consists of the files |README.txt|, |childdoc.ins| and |childdoc.dtx|
as well as the derived files |childdoc.def|, |cdocsamp.tex|
with |cdocsch1.tex|, |cdocsch2.tex|, |cdocspt3.tex|, |cdocspt4.tex|,
|cdocsdrf.tex|, |cdocsfn1.tex|, |cdocsfn2.tex|
as well as |childdoc.pdf|.

%%%%%%%%%%%%%%%%%%%%%%%%%%%%%%%%%%%%%%%%%%%%%%%%%%%%%%%%%%%%%%%%%%%%%%%%%%%%%%%%
\subsection{Files and Installation}

The package consists of the files:
%
\begin{center}
\begin{tabular}{ll}
    |README.txt|   & readme file \\
    |childdoc.ins| & installation file \\
    |childdoc.dtx| & source file \\
    |childdoc.def| & definition file \\
    |cdocsamp.tex| & sample main file \\
    |cdocsch1.tex| & sample include file \\
    |cdocsch2.tex| & sample include file \\
    |cdocspt3.tex| & sample part file \\
    |cdocspt4.tex| & sample part file \\
    |cdocsdrf.tex| & sample redirection file \\
    |cdocsfn1.tex| & sample redirection file \\
    |cdocsfn2.tex| & sample redirection file \\
    |childdoc.pdf| & manual
\end{tabular}
\end{center}
%
The distribution consists of the files
|README.txt|, |childdoc.ins| and |childdoc.dtx|.
%
\begin{itemize}
\item
Run (pdf)\LaTeX{} on |childdoc.dtx|
to compile the manual |childdoc.pdf| (this file).
\item
Run \LaTeX{} on |childdoc.ins| to create the definitions file |childdoc.def|
and the sample |cdocsamp.tex| with include files
|cdocsch1.tex|, |cdocsch2.tex|, |cdocspt3.tex|, |cdocspt4.tex|,
|cdocsdrf.tex|, |cdocsfn1.tex|, |cdocsfn2.tex|.
Then copy the file |childdoc.def| to an appropriate directory of your \LaTeX{}
distribution, e.g.\ \textit{texmf-root}|/tex/latex/childdoc|.
\end{itemize}

%%%%%%%%%%%%%%%%%%%%%%%%%%%%%%%%%%%%%%%%%%%%%%%%%%%%%%%%%%%%%%%%%%%%%%%%%%%%%%%%
\subsection{Related CTAN Packages}

There are several other packages which offer a similar functionality:
%
\begin{itemize}
\item
The packages
\href{http://ctan.org/pkg/docmute}{\textsf{docmute}},
\href{http://ctan.org/pkg/includex}{\textsf{includex}} and
\href{http://ctan.org/pkg/standalone}{\textsf{standalone}}
provide commands to include only the document body of
a child file thus allowing both files to be compiled individually.
\item
The packages \href{http://ctan.org/pkg/subdocs}{\textsf{subdocs}}
and \href{http://ctan.org/pkg/subfiles}{\textsf{subfiles}}
provide structures in which the main and child documents can be
encapsulated and allowing them to be compiled individually.
The inclusion mechanism is different from the conventional |\include|.
\item
The package \href{http://ctan.org/pkg/combine}{\textsf{combine}}
is an elaborate solution to combine several documents into one.
\end{itemize}
%
See also the CTAN topic \href{http://ctan.org/topic/subdocs}{\textsf{subdocs}}
for further related packages.
The present package differs from the above solutions in that
a document structure constructed with the conventional |\include| mechanism
just needs two extra commands at the top of every file
such that all constituent files can be compiled individually.

%%%%%%%%%%%%%%%%%%%%%%%%%%%%%%%%%%%%%%%%%%%%%%%%%%%%%%%%%%%%%%%%%%%%%%%%%%%%%%%%
%\subsection{Feature Suggestions}
%
%The following is a list of features which may be useful for future
%versions of this package:
%%
%\begin{itemize}
%\item
%\ldots
%\end{itemize}

%%%%%%%%%%%%%%%%%%%%%%%%%%%%%%%%%%%%%%%%%%%%%%%%%%%%%%%%%%%%%%%%%%%%%%%%%%%%%%%%
\subsection{Revision History}

%%%%%%%%%%%%%%%%%%%%%%%%%%%%%%%%%%%%%%%%
\paragraph{v2.0:} 2018/12/30

\begin{itemize}
\item
immediate forward processing
\item
added |\childdocby| mechanism
\item
manual restructured
\end{itemize}

%%%%%%%%%%%%%%%%%%%%%%%%%%%%%%%%%%%%%%%%
\paragraph{v1.6:} 2018/01/17

\begin{itemize}
\item
application for development of include files
\item
corrections to manual
\end{itemize}

%%%%%%%%%%%%%%%%%%%%%%%%%%%%%%%%%%%%%%%%
\paragraph{v1.5:} 2017/05/21

\begin{itemize}
\item
more complete structuring introduced
\item
|\childdocof| introduced
\item
|\childdoc| renamed to |\childdocmain|
\item
|\childredirect| renamed to |\childdocforward| and |\childdocforwardprefix|
and functionality expanded
\end{itemize}

%%%%%%%%%%%%%%%%%%%%%%%%%%%%%%%%%%%%%%%%
\paragraph{v1.0:} 2017/04/27

\begin{itemize}
\item
manual and install package
\item
first version published on CTAN
\end{itemize}

%%%%%%%%%%%%%%%%%%%%%%%%%%%%%%%%%%%%%%%%
\paragraph{v0.6:} 2017/04/26

\begin{itemize}
\item
redirection mechanism added
\end{itemize}

%%%%%%%%%%%%%%%%%%%%%%%%%%%%%%%%%%%%%%%%
\paragraph{v0.5:} 2017/04/26

\begin{itemize}
\item
functionality in definition file
\end{itemize}


%%%%%%%%%%%%%%%%%%%%%%%%%%%%%%%%%%%%%%%%%%%%%%%%%%%%%%%%%%%%%%%%%%%%%%%%%%%%%%%%
%%%%%%%%%%%%%%%%%%%%%%%%%%%%%%%%%%%%%%%%%%%%%%%%%%%%%%%%%%%%%%%%%%%%%%%%%%%%%%%%
%%%%%%%%%%%%%%%%%%%%%%%%%%%%%%%%%%%%%%%%%%%%%%%%%%%%%%%%%%%%%%%%%%%%%%%%%%%%%%%%
\appendix

\settowidth\MacroIndent{\rmfamily\scriptsize 000\ }

 \DocInput{childdoc.dtx}

\end{document}
%</driver>
% \fi
%
% %%%%%%%%%%%%%%%%%%%%%%%%%%%%%%%%%%%%%%%%%%%%%%%%%%%%%%%%%%%%%%%%%%%%%%%%%%%%%%
% %%%%%%%%%%%%%%%%%%%%%%%%%%%%%%%%%%%%%%%%%%%%%%%%%%%%%%%%%%%%%%%%%%%%%%%%%%%%%%
% \section{Sample}
%\iffalse
%<*samplemain>
%\fi
%
% The following presents a sample document
% with two chapters, two parts, a title page,
% a compile flag as well as three forwarding files to set the flag.
% It consists of eight |.tex| files:
% \begin{center}
% \begin{tabular}{ll}
% |cdocsamp.tex|&main file\\
% |cdocsch1.tex|&include file for chapter 1\\
% |cdocsch2.tex|&include file for chapter 2\\
% |cdocspt3.tex|&include file for part 3\\
% |cdocspt4.tex|&include file for part 4\\
% |cdocsdrf.tex|&forwarding file for main file in draft mode\\
% |cdocsfi1.tex|&forwarding file for final version of chapter 1\\
% |cdocsfi2.tex|&forwarding file for final version of chapter 2\\
% \end{tabular}
% \end{center}
% Each of the eight files can be compiled directly by the \LaTeX{} compiler.
%
% %%%%%%%%%%%%%%%%%%%%%%%%%%%%%%%%%%%%%%
% \paragraph{Main File.}
%
% The main file is called |cdocsamp.tex|.
%
% Load the \textsf{childdoc} definitions and
% declare the filename for the main document:
%    \begin{macrocode}
\input{childdoc.def}
\childdocmain{}
%    \end{macrocode}

% Optional override for |\version| flag:
%    \begin{macrocode}
%%\ifchilddoc\else\providecommand{\version}{draft}\fi
%    \end{macrocode}

% Define the default values for the |\version| flag
% (|final| for the main file and |draft| for childs):
%    \begin{macrocode}
\ifchilddoc
\providecommand{\version}{draft}
\else
\providecommand{\version}{final}
\fi
%    \end{macrocode}

% Load the standard document class:
%    \begin{macrocode}
\documentclass[12pt]{article}
%    \end{macrocode}

% Start the document body:
%    \begin{macrocode}
\begin{document}
%    \end{macrocode}

% Declare a title page.
% Print title, part of document being processed and version flag:
%    \begin{macrocode}
\addtocounter{page}{-1}
\begin{center}
{\LARGE\bfseries{}childdoc example\par}
\vspace{1cm}
\ifchilddoc
\ifchilddocmanual part\else chapter\fi:
`\childdocname' of `\childdocjob'\par
\else
main document: `\childdocjob'\par
\fi
version: \version\par
\end{center}
\newpage
%    \end{macrocode}

% Manually include selected file,
% otherwise process as usual:
%    \begin{macrocode}
\ifchilddocmanual
\section*{part `\childdocname'}
\input{\childdocname}
\else
%    \end{macrocode}

% Include the two chapters:
%    \begin{macrocode}
\include{cdocsch1}
\include{cdocsch2}
%    \end{macrocode}

% Include the two parts unless only chapters should be displayed:
%    \begin{macrocode}
\ifchilddoc\else
\section{part three}
\input{cdocspt3}
\section{part four}
\input{cdocspt4}
\fi
%    \end{macrocode}

% Process as usual until here:
%    \begin{macrocode}
\fi
%    \end{macrocode}

% End of document body:
%    \begin{macrocode}
\end{document}
%    \end{macrocode}
%\iffalse
%</samplemain>
%\fi
%
% %%%%%%%%%%%%%%%%%%%%%%%%%%%%%%%%%%%%%%
% \paragraph{Chapter Include Files.}
%
% The include files are called |cdocsch1.tex| and |cdocsch2.tex|.
%
%\iffalse
%<*samplechap1|samplechap2>
%\fi

% Optional override for |\version| flag:
%    \begin{macrocode}
%%\providecommand{\version}{final}
%    \end{macrocode}

% Include the main document:
%    \begin{macrocode}
\input{childdoc.def}
\childdocof{cdocsamp}
%    \end{macrocode}

%\iffalse
%</samplechap1|samplechap2>
%\fi
%
%\iffalse
%<*samplechap1>
%\fi
% Some text for chapter 1:
%    \begin{macrocode}
\section{one}
some text in chapter one
%    \end{macrocode}

%\iffalse
%</samplechap1>
%\fi
% Some text for chapter 2:
%\iffalse
%<*samplechap2>
%\fi
%    \begin{macrocode}
\section{two}
more text in chapter two
%    \end{macrocode}

%\iffalse
%</samplechap2>
%\fi
%
% %%%%%%%%%%%%%%%%%%%%%%%%%%%%%%%%%%%%%%
% \paragraph{Part Include Files.}
%
% The include files are called |cdocspt3.tex| and |cdocspt4.tex|.
%
%\iffalse
%<*samplepart3|samplepart4>
%\fi

% Optional override for |\version| flag:
%    \begin{macrocode}
%%\providecommand{\version}{final}
%    \end{macrocode}

% Include the main document:
%    \begin{macrocode}
\input{childdoc.def}
\childdocby{cdocsamp}
%    \end{macrocode}

%\iffalse
%</samplepart3|samplepart4>
%\fi
%
%\iffalse
%<*samplepart3>
%\fi
% Some text for part 3:
%    \begin{macrocode}
some text in part three
%    \end{macrocode}

%\iffalse
%</samplepart3>
%\fi
% Some text for part 4:
%\iffalse
%<*samplepart4>
%\fi
%    \begin{macrocode}
more text in part four
%    \end{macrocode}

%\iffalse
%</samplepart4>
%\fi
%
% %%%%%%%%%%%%%%%%%%%%%%%%%%%%%%%%%%%%%%
% \paragraph{Forwarding for a Complete Draft.}
%
% The following forwarding file |cdocsdrf.tex|
% compiles the main document in draft mode:
%\iffalse
%<*sampledraft>
%\fi
%    \begin{macrocode}
\def\version{draft}
\input{childdoc.def}
\childdocforward{cdocsamp}
%    \end{macrocode}

%\iffalse
%</sampledraft>
%\fi
%
% %%%%%%%%%%%%%%%%%%%%%%%%%%%%%%%%%%%%%%
% \paragraph{Forwarding for Final Version of the Chapters.}
%
% The following forwarding files |cdocsfn1.tex| and |cdocsfn2.tex|
% (with identical content)
% compile the final versions of the child documents
% |cdocsch1.tex| and |cdocsch2.tex|, respectively:
%\iffalse
%<*samplefinal>
%\fi
%    \begin{macrocode}
\def\version{final}
\input{childdoc.def}
\childdocforwardprefix[cdocsamp]{cdocsfn}{cdocsch}
%    \end{macrocode}

%\iffalse
%</samplefinal>
%\fi
%
% %%%%%%%%%%%%%%%%%%%%%%%%%%%%%%%%%%%%%%
% \paragraph{Command Line Processing.}
%
% The following three command lines generate the output files
% |cdocscld|, |cdocscl1| and |cdocscl2|
% which should be identical to
% |cdocsdrf|, |cdocsch1| and |cdocsfn2|, respectively:
% \begin{center}
% \begin{tabular}{l}
% |latex -jobname cdocscld \|\\
% |  "\def\version{draft}\input{childdoc.def}\childdocforward{cdocsamp}"|\\
% |latex -jobname cdocscl1 \|\\
% |  "\input{childdoc.def}\childdocforward[cdocsamp]{cdocsch1}"|\\
% |latex -jobname cdocscl2 \|\\
% |  "\def\version{final}\input{childdoc.def}\childdocforward{cdocsch2}"|
% \end{tabular}
% \end{center}
% Note that the trailing backslash on each first line
% merely continues the input to the second line
% (for convenient cut ant paste).
% Furthermore, the command |latex| can be replaced by any
% of its alternative versions such as |pdflatex|.
%
% %%%%%%%%%%%%%%%%%%%%%%%%%%%%%%%%%%%%%%%%%%%%%%%%%%%%%%%%%%%%%%%%%%%%%%%%%%%%%%
% %%%%%%%%%%%%%%%%%%%%%%%%%%%%%%%%%%%%%%%%%%%%%%%%%%%%%%%%%%%%%%%%%%%%%%%%%%%%%%
% \section{Implementation}
%\iffalse
%<*package>
%\fi
%
% This section describes the definitions file |childdoc.def|.

% The definitions cannot be loaded using |\usepackage| or |\RequirePackage|
% which has a mechanism to prevent loading a style file more than once.
% When loading the definitions by means of |\input|
% multiple instances have to be prevented manually:
%\iffalse
%This code needs to be before the `\ProvidesFile' directive
%which is defined at the beginning of this file.
%Therefore it is also placed there and commented out here.
%</package>
%<*discard>
%\fi
%    \begin{macrocode}
\ifdefined\childdocmain\endinput\fi
%    \end{macrocode}
%\iffalse
%</discard>
%<*package>
%\fi
%
% \macro{\ifchilddoc}
% \macro{\ifchilddocmanual}
% The conditional |\ifchilddoc| tells whether a
% child (true) or main (false) document is being compiled.
% The conditional |\ifchilddocmanual| tells whether
% the |\includeonly| mechanism is used (false) or
% the selection of child files must be performed manually (true).
% The definitions initialise to false:
%    \begin{macrocode}
\newif\ifchilddoc
\newif\ifchilddocmanual
%    \end{macrocode}

% \macro{\childdocname}
% \macro{\childdocjob}
% The macro |\childdocname| stores the name of the main document
% to be compiled. The macro |\childdocjob| stores the name of
% the document on which the \LaTeX{} compiler was originally invoked.
% The content of |\jobname| cannot be compared
% to filenames specified in the source due to different catcodes.
% The following code rescans |\jobname|, stores the result
% in |\childdocname| and saves a copy in |\childdocjob|:
%    \begin{macrocode}
\edef\childdocname{\scantokens\expandafter{\jobname\noexpand}}
\let\childdocjob\childdocname
%    \end{macrocode}

% \macro{\childdocdisable}
% The macro |\childdocdisable| prevents the main file
% from being processed more than once.
% At this stage, the main document command |\childdocmain|
% is assumed to be called once again where it should do nothing.
% Any subsequent call to it should prevent
% a secondary processing of the main document
% It overwrites the forwarding commands
% |\childdocof| and |\childdocforward|
% with empty macros to prevent further inclusions of the main document:
%    \begin{macrocode}
\newcommand{\childdocdisable}
{
  \renewcommand{\childdocmain}[1]{\renewcommand{\childdocmain}[1]{\endinput}}
  \renewcommand{\childdocof}[1]{}
  \renewcommand{\childdocby}[2][]{}
  \renewcommand{\childdocforward}[2][]{}
  \renewcommand{\childdocdisable}{}
}
%    \end{macrocode}

% \macro{\childdocmain}
% The macro |\childdocmain| is to be called at the top of the main file
% with nothing or the main filename (without extension) as argument.
% First, it breaks loops.
% If the argument is not empty and does not match |\childdocname|
% (which is set by the first inclusion of |childdoc.def|),
% |\ifchilddoc| is set to true, |\includeonly| is applied to the child file
% and |\jobname| is set to the main file
% (for proper handling of |.aux| files):
%    \begin{macrocode}
\newcommand{\childdocmain}[1]
{
  \childdocdisable\childdocmain{}
  \if?#1?\else
    \begingroup
      \def\childdoctmp{#1}
      \ifx\childdoctmp\childdocname
        \def\childdoctmp{}
      \else
        \def\childdoctmp
        {
          \childdoctrue
          \includeonly{\childdocname}
          \def\childdocjob{#1}
          \def\jobname{#1}
        }
      \fi
      \expandafter
    \endgroup
    \childdoctmp
  \fi
}
%    \end{macrocode}

% \macro{\childdocof}
% The command |\childdocof| redirects
% compilation to the main file |#1|.
%    \begin{macrocode}
\newcommand{\childdocof}[1]
{
  \childdocdisable
  \childdoctrue
  \includeonly{\childdocname}
  \def\jobname{#1}
  \def\childdocjob{#1}
  \input{#1}
}
%    \end{macrocode}

% \macro{\childdocby}
% The command |\childdocby| ....
%    \begin{macrocode}
\newcommand{\childdocby}[2][]
{
  \childdocdisable
  \childdoctrue
  \childdocmanualtrue
  \if?#1?\else
    \def\jobname{#2}
  \fi
  \def\childdocjob{#2}
  \input{#2}
  \endinput
}
%    \end{macrocode}

% \macro{\childdocforward}
% The command |\childdocforward| redirects
% compilation to the main file or
% (if the optional argument is given) a child file.
% Parameters are set as if the main file
% or a child file starting with |\childdocof| was compiled.
% Then compilation is handed over to the main file:
%    \begin{macrocode}
\newcommand{\childdocforward}[2][]
{
  \begingroup
    \if?#1?
      \def\childdoctmp
      {
        \def\childdocname{#2}
        \def\childdocjob{#2}
        \def\jobname{#2}
        \input{#2}
        \endinput
      }
    \else
      \def\childdoctmp
      {
        \childdocdisable
        \def\childdocname{#2}
        \childdoctrue
        \includeonly{#2}
        \def\childdocjob{#1}
        \def\jobname{#1}
        \input{#1}
        \endinput
      }
    \fi
    \expandafter
  \endgroup
  \childdoctmp
}
%    \end{macrocode}

% \macro{\childdocforwardprefix}
% The command |\childdocforwardprefix| redirects
% compilation to the main or a child file by means of a pattern.
% The prefix |#1| in the current filename is replaced by |#2|
% and the suffix of the current filename is kept
% (it is assumed that the filename does not contain the substring `|~~~|'
% which is used as a delimiter).
% Compilation is handed over to the new file by |\childdocforward|:
%    \begin{macrocode}
\newcommand{\childdocforwardprefix}[3][]
{
  \begingroup
    \def\childdocextract #2##1~~~{\def\childdoctmp{\childdocforward[#1]{#3##1}}}
    \expandafter\childdocextract\childdocname~~~
    \expandafter
  \endgroup
  \childdoctmp
}
%    \end{macrocode}

% \macro{\childdoc}
% The deprecated macro |\childdoc| is a legacy version of |\childdocmain|:
%    \begin{macrocode}
\newcommand{\childdoc}{\childdocmain}
%    \end{macrocode}

% \macro{\childdocredirect}
% The deprecated macro |\childdocredirect| is a legacy version
% of |\childdocforward| and |\childdocforwardprefix|:
%    \begin{macrocode}
\newcommand{\childdocredirect}[2][]
{
  \begingroup
    \if?#1?
      \def\childdoctmp{\childdocforward{#2}}
    \else
      \def\childdoctmp{\childdocforwardprefix{#1}{#2}}
    \fi
    \expandafter
  \endgroup
  \childdoctmp
}
%    \end{macrocode}

%\iffalse
%</package>
%\fi
%
\endinput
|
and perform the replacements as outlined below.
Instead of |\childdocmain{|\textit{main}|}| add the following code
to the top of the main file:
%
\begin{center}
\begin{tabular}{l}
|\||ifdefined\childdocname\endinput\||fi\newif\ifchilddoc|\\
|\edef\childdocname{\scantokens\expandafter{\jobname\noexpand}}|\\
|\def\childdocmain{|\textit{main}|}\||ifx\childdocmain\childdocname\||else|\\
|\childdoctrue\includeonly{\childdocname}\let\jobname\childdocmain\||fi|\\
\end{tabular}
\end{center}
%
Instead of |\childdocof{|\textit{main}|}| just include the main file
at the top of each child file:
%
\begin{center}
|\input{|\textit{main}|}|
\end{center}
%
A simple redirection |\childdocforward{|\textit{dest}|}| is achieved by:
%
\begin{center}
|\def\jobname{|\textit{dest}|}\input{\jobname}|
\end{center}
%
The redirection with prefix
|\childdocforwardprefix[|\textit{prefix}|]{|\textit{dest}|}|
is accomplished by:
%
\begin{center}
\begin{tabular}{l}
|{\edef\jobname{\scantokens\expandafter{\jobname\noexpand}}|\\
|\def\redirectjob |\textit{prefix}|#1~~~{\gdef\jobname{|\textit{dest}|#1}}|\\
|\expandafter\redirectjob\jobname~~~}\input{\jobname}|
\end{tabular}
\end{center}

In an alternative approach,
child documents can be compiled by a specific command line
without additional code or specific definitions:
%
\begin{center}
|... -jobname "|\textit{target}|" "|[\textit{flags}]%
|\includeonly{|\textit{dest}|}\input{|\textit{main}|}"|
\end{center}
%

%%%%%%%%%%%%%%%%%%%%%%%%%%%%%%%%%%%%%%%%%%%%%%%%%%%%%%%%%%%%%%%%%%%%%%%%%%%%%%%%
%%%%%%%%%%%%%%%%%%%%%%%%%%%%%%%%%%%%%%%%%%%%%%%%%%%%%%%%%%%%%%%%%%%%%%%%%%%%%%%%
\section{Information}

%%%%%%%%%%%%%%%%%%%%%%%%%%%%%%%%%%%%%%%%%%%%%%%%%%%%%%%%%%%%%%%%%%%%%%%%%%%%%%%%
\subsection{Copyright}

Copyright \copyright{} 2017--2018 Niklas Beisert

This work may be distributed and/or modified under the
conditions of the \LaTeX{} Project Public License, either version 1.3
of this license or (at your option) any later version.
The latest version of this license is in
  \url{http://www.latex-project.org/lppl.txt}
and version 1.3 or later is part of all distributions of \LaTeX{}
version 2005/12/01 or later.

This work has the LPPL maintenance status `maintained'.

The Current Maintainer of this work is Niklas Beisert.

This work consists of the files |README.txt|, |childdoc.ins| and |childdoc.dtx|
as well as the derived files |childdoc.def|, |cdocsamp.tex|
with |cdocsch1.tex|, |cdocsch2.tex|, |cdocspt3.tex|, |cdocspt4.tex|,
|cdocsdrf.tex|, |cdocsfn1.tex|, |cdocsfn2.tex|
as well as |childdoc.pdf|.

%%%%%%%%%%%%%%%%%%%%%%%%%%%%%%%%%%%%%%%%%%%%%%%%%%%%%%%%%%%%%%%%%%%%%%%%%%%%%%%%
\subsection{Files and Installation}

The package consists of the files:
%
\begin{center}
\begin{tabular}{ll}
    |README.txt|   & readme file \\
    |childdoc.ins| & installation file \\
    |childdoc.dtx| & source file \\
    |childdoc.def| & definition file \\
    |cdocsamp.tex| & sample main file \\
    |cdocsch1.tex| & sample include file \\
    |cdocsch2.tex| & sample include file \\
    |cdocspt3.tex| & sample part file \\
    |cdocspt4.tex| & sample part file \\
    |cdocsdrf.tex| & sample redirection file \\
    |cdocsfn1.tex| & sample redirection file \\
    |cdocsfn2.tex| & sample redirection file \\
    |childdoc.pdf| & manual
\end{tabular}
\end{center}
%
The distribution consists of the files
|README.txt|, |childdoc.ins| and |childdoc.dtx|.
%
\begin{itemize}
\item
Run (pdf)\LaTeX{} on |childdoc.dtx|
to compile the manual |childdoc.pdf| (this file).
\item
Run \LaTeX{} on |childdoc.ins| to create the definitions file |childdoc.def|
and the sample |cdocsamp.tex| with include files
|cdocsch1.tex|, |cdocsch2.tex|, |cdocspt3.tex|, |cdocspt4.tex|,
|cdocsdrf.tex|, |cdocsfn1.tex|, |cdocsfn2.tex|.
Then copy the file |childdoc.def| to an appropriate directory of your \LaTeX{}
distribution, e.g.\ \textit{texmf-root}|/tex/latex/childdoc|.
\end{itemize}

%%%%%%%%%%%%%%%%%%%%%%%%%%%%%%%%%%%%%%%%%%%%%%%%%%%%%%%%%%%%%%%%%%%%%%%%%%%%%%%%
\subsection{Related CTAN Packages}

There are several other packages which offer a similar functionality:
%
\begin{itemize}
\item
The packages
\href{http://ctan.org/pkg/docmute}{\textsf{docmute}},
\href{http://ctan.org/pkg/includex}{\textsf{includex}} and
\href{http://ctan.org/pkg/standalone}{\textsf{standalone}}
provide commands to include only the document body of
a child file thus allowing both files to be compiled individually.
\item
The packages \href{http://ctan.org/pkg/subdocs}{\textsf{subdocs}}
and \href{http://ctan.org/pkg/subfiles}{\textsf{subfiles}}
provide structures in which the main and child documents can be
encapsulated and allowing them to be compiled individually.
The inclusion mechanism is different from the conventional |\include|.
\item
The package \href{http://ctan.org/pkg/combine}{\textsf{combine}}
is an elaborate solution to combine several documents into one.
\end{itemize}
%
See also the CTAN topic \href{http://ctan.org/topic/subdocs}{\textsf{subdocs}}
for further related packages.
The present package differs from the above solutions in that
a document structure constructed with the conventional |\include| mechanism
just needs two extra commands at the top of every file
such that all constituent files can be compiled individually.

%%%%%%%%%%%%%%%%%%%%%%%%%%%%%%%%%%%%%%%%%%%%%%%%%%%%%%%%%%%%%%%%%%%%%%%%%%%%%%%%
%\subsection{Feature Suggestions}
%
%The following is a list of features which may be useful for future
%versions of this package:
%%
%\begin{itemize}
%\item
%\ldots
%\end{itemize}

%%%%%%%%%%%%%%%%%%%%%%%%%%%%%%%%%%%%%%%%%%%%%%%%%%%%%%%%%%%%%%%%%%%%%%%%%%%%%%%%
\subsection{Revision History}

%%%%%%%%%%%%%%%%%%%%%%%%%%%%%%%%%%%%%%%%
\paragraph{v2.0:} 2018/12/30

\begin{itemize}
\item
immediate forward processing
\item
added |\childdocby| mechanism
\item
manual restructured
\end{itemize}

%%%%%%%%%%%%%%%%%%%%%%%%%%%%%%%%%%%%%%%%
\paragraph{v1.6:} 2018/01/17

\begin{itemize}
\item
application for development of include files
\item
corrections to manual
\end{itemize}

%%%%%%%%%%%%%%%%%%%%%%%%%%%%%%%%%%%%%%%%
\paragraph{v1.5:} 2017/05/21

\begin{itemize}
\item
more complete structuring introduced
\item
|\childdocof| introduced
\item
|\childdoc| renamed to |\childdocmain|
\item
|\childredirect| renamed to |\childdocforward| and |\childdocforwardprefix|
and functionality expanded
\end{itemize}

%%%%%%%%%%%%%%%%%%%%%%%%%%%%%%%%%%%%%%%%
\paragraph{v1.0:} 2017/04/27

\begin{itemize}
\item
manual and install package
\item
first version published on CTAN
\end{itemize}

%%%%%%%%%%%%%%%%%%%%%%%%%%%%%%%%%%%%%%%%
\paragraph{v0.6:} 2017/04/26

\begin{itemize}
\item
redirection mechanism added
\end{itemize}

%%%%%%%%%%%%%%%%%%%%%%%%%%%%%%%%%%%%%%%%
\paragraph{v0.5:} 2017/04/26

\begin{itemize}
\item
functionality in definition file
\end{itemize}


%%%%%%%%%%%%%%%%%%%%%%%%%%%%%%%%%%%%%%%%%%%%%%%%%%%%%%%%%%%%%%%%%%%%%%%%%%%%%%%%
%%%%%%%%%%%%%%%%%%%%%%%%%%%%%%%%%%%%%%%%%%%%%%%%%%%%%%%%%%%%%%%%%%%%%%%%%%%%%%%%
%%%%%%%%%%%%%%%%%%%%%%%%%%%%%%%%%%%%%%%%%%%%%%%%%%%%%%%%%%%%%%%%%%%%%%%%%%%%%%%%
\appendix

\settowidth\MacroIndent{\rmfamily\scriptsize 000\ }

 \DocInput{childdoc.dtx}

\end{document}
%</driver>
% \fi
%
% %%%%%%%%%%%%%%%%%%%%%%%%%%%%%%%%%%%%%%%%%%%%%%%%%%%%%%%%%%%%%%%%%%%%%%%%%%%%%%
% %%%%%%%%%%%%%%%%%%%%%%%%%%%%%%%%%%%%%%%%%%%%%%%%%%%%%%%%%%%%%%%%%%%%%%%%%%%%%%
% \section{Sample}
%\iffalse
%<*samplemain>
%\fi
%
% The following presents a sample document
% with two chapters, two parts, a title page,
% a compile flag as well as three forwarding files to set the flag.
% It consists of eight |.tex| files:
% \begin{center}
% \begin{tabular}{ll}
% |cdocsamp.tex|&main file\\
% |cdocsch1.tex|&include file for chapter 1\\
% |cdocsch2.tex|&include file for chapter 2\\
% |cdocspt3.tex|&include file for part 3\\
% |cdocspt4.tex|&include file for part 4\\
% |cdocsdrf.tex|&forwarding file for main file in draft mode\\
% |cdocsfi1.tex|&forwarding file for final version of chapter 1\\
% |cdocsfi2.tex|&forwarding file for final version of chapter 2\\
% \end{tabular}
% \end{center}
% Each of the eight files can be compiled directly by the \LaTeX{} compiler.
%
% %%%%%%%%%%%%%%%%%%%%%%%%%%%%%%%%%%%%%%
% \paragraph{Main File.}
%
% The main file is called |cdocsamp.tex|.
%
% Load the \textsf{childdoc} definitions and
% declare the filename for the main document:
%    \begin{macrocode}
% \iffalse
%
% childdoc.dtx Copyright (C) 2017-2018 Niklas Beisert
%
% This work may be distributed and/or modified under the
% conditions of the LaTeX Project Public License, either version 1.3
% of this license or (at your option) any later version.
% The latest version of this license is in
%   http://www.latex-project.org/lppl.txt
% and version 1.3 or later is part of all distributions of LaTeX
% version 2005/12/01 or later.
%
% This work has the LPPL maintenance status `maintained'.
%
% The Current Maintainer of this work is Niklas Beisert.
%
% This work consists of the files childdoc.dtx and childdoc.ins
% and the derived files childdoc.def and cdocsamp.tex with
% cdocsch1.tex, cdocsch2.tex, cdocsdrf.tex, cdocsfn1.tex, cdocsfn2.tex.
%
%<package>\ifdefined\childdocmain\endinput\fi
%<package>\ProvidesFile{childdoc.def}[2018/12/30 v2.0 child document driver]
%<samplemain>\ProvidesFile{cdocsamp.tex}[2018/12/30 v2.0 sample for childdoc]
%<*driver>
%\ProvidesFile{childdoc.drv}[2018/12/30 v2.0 childdoc reference manual file]
\PassOptionsToClass{10pt,a4paper}{article}
\documentclass{ltxdoc}

\usepackage[margin=35mm]{geometry}
\usepackage{hyperref}
\usepackage{hyperxmp}
\usepackage[usenames]{color}

\hypersetup{colorlinks=true}
\hypersetup{pdfstartview=FitH}
\hypersetup{pdfpagemode=UseNone}
\hypersetup{pdfsource={}}
\hypersetup{pdflang={en-UK}}
\hypersetup{pdfcopyright={Copyright 2017-2018 Niklas Beisert.
  This work may be distributed and/or modified under the
  conditions of the LaTeX Project Public License, either version 1.3
  of this license or (at your option) any later version.}}
\hypersetup{pdflicenseurl={http://www.latex-project.org/lppl.txt}}
\hypersetup{pdfcontactaddress={ETH Zurich, ITP, HIT K,
  Wolfgang-Pauli-Strasse 27}}
\hypersetup{pdfcontactpostcode={8093}}
\hypersetup{pdfcontactcity={Zurich}}
\hypersetup{pdfcontactcountry={Switzerland}}
\hypersetup{pdfcontactemail={nbeisert@itp.phys.ethz.ch}}
\hypersetup{pdfcontacturl={http://people.phys.ethz.ch/\xmptilde nbeisert/}}

\newcommand{\secref}[1]{\hyperref[#1]{section \ref*{#1}}}

\parskip1ex
\parindent0pt
\let\olditemize\itemize
\def\itemize{\olditemize\parskip0pt}

\begin{document}

\title{The \textsf{childdoc} Package}
\hypersetup{pdftitle={The childdoc Package}}
\author{Niklas Beisert\\[2ex]
  Institut f\"ur Theoretische Physik\\
  Eidgen\"ossische Technische Hochschule Z\"urich\\
  Wolfgang-Pauli-Strasse 27, 8093 Z\"urich, Switzerland\\[1ex]
  \href{mailto:nbeisert@itp.phys.ethz.ch}
  {\texttt{nbeisert@itp.phys.ethz.ch}}}
\hypersetup{pdfauthor={Niklas Beisert}}
\hypersetup{pdfsubject={Manual for the LaTeX2e Package childdoc}}
\date{30 December 2018, \textsf{v2.0}}
\maketitle

\begin{abstract}\noindent
\textsf{childdoc} is a \LaTeXe{} package
that enables the direct compilation
of document sections included by |\include|
to individual files.
\end{abstract}

\begingroup
\parskip0ex
\tableofcontents
\endgroup

%%%%%%%%%%%%%%%%%%%%%%%%%%%%%%%%%%%%%%%%%%%%%%%%%%%%%%%%%%%%%%%%%%%%%%%%%%%%%%%%
%%%%%%%%%%%%%%%%%%%%%%%%%%%%%%%%%%%%%%%%%%%%%%%%%%%%%%%%%%%%%%%%%%%%%%%%%%%%%%%%
\section{Introduction}

\LaTeX{} provides a mechanism to structure a large document (such as a book)
into a main file and several child files (containing the chapters)
using the |\include| command.
This mechanism is beneficial for documents
which span hundreds of pages in order to
make the source file(s) more manageable.
Moreover, compilation can be restricted to
selected child files by means of the |\includeonly| command.
The latter feature can be used to reduce the compilation time while editing
(this was significantly more useful in the earlier days of \LaTeX{})
or to generate a smaller document which is easier to navigate.
Another application of |\includeonly| is to generate
documents consisting of selected parts of the complete document.

However, there are a few drawbacks of the plain |\include| mechanism:
\begin{itemize}
\item
The child files cannot be compiled on their own,
they can only be compiled via the main file.
A naive editing environment
(such as a text editor with an option
to have the current file processed by \LaTeX)
may require one to switch to the main file before compiling;
attempting to compile the child file produces errors.
\item
The main file must be modified (each time)
to adjust the |\includeonly| command
to the present needs. This easily leaves the main file in a messy state.
\item
The generated document will always carry the filename
of the main document. This is inconvenient if
several child files are to be compiled and
to be kept for distribution.
\end{itemize}

The present package provides a simple interface
to make child files individually compilable by \LaTeX{}.
Compiling a child file then has the same effect as compiling
the main file with an |\includeonly| command
to select the appropriate child.
Moreover the generated document will carry the name of the child
rather than the main file.
This resolves all three above issues.

This feature is meant to make the editing of books,
thesis documents and lecture notes somewhat more convenient.
However, the package can also be used efficiently for
composing a series of documents (such as exercise sheets)
which are typically distributed individually.
It then assists the author in generating the individual documents
(potentially in different versions)
as well as a document containing the collected series.
Another application is in developing style files
or other kinds of included material
where compilation of the style file could redirect
to a sample or test file.

%%%%%%%%%%%%%%%%%%%%%%%%%%%%%%%%%%%%%%%%%%%%%%%%%%%%%%%%%%%%%%%%%%%%%%%%%%%%%%%%
%%%%%%%%%%%%%%%%%%%%%%%%%%%%%%%%%%%%%%%%%%%%%%%%%%%%%%%%%%%%%%%%%%%%%%%%%%%%%%%%
\section{Usage}

First of all, the package \textsf{childdoc} is \emph{not} a standard
\LaTeXe{} |.sty| style file! Therefore it needs to be invoked in
a non-standard way.

%%%%%%%%%%%%%%%%%%%%%%%%%%%%%%%%%%%%%%%%%%%%%%%%%%%%%%%%%%%%%%%%%%%%%%%%%%%%%%%%
\subsection{Included Files}
\label{sec:include}

%%%%%%%%%%%%%%%%%%%%%%%%%%%%%%%%%%%%%%%%
\DescribeMacro{\childdocmain}
To use the package, add the commands
\begin{center}
\begin{tabular}{l}
|\input{childdoc.def}|\\
|\childdocmain{}|\\
\end{tabular}
\end{center}
at the very top of the main \LaTeX{} file,
in particular \emph{before} the |\documentclass| statement!
The argument of |\childdocmain| should be left empty
(but it must be present).

%%%%%%%%%%%%%%%%%%%%%%%%%%%%%%%%%%%%%%%%
\DescribeMacro{\childdocof}
Furthermore, add the commands
\begin{center}
\begin{tabular}{l}
|\input{childdoc.def}|\\
|\childdocof{|\textit{main}|}|\\
\end{tabular}
\end{center}
at the top of every child file \textit{child}
which is included by |\include{|\textit{child}|}|
from within the main file
(or at least for those files to be compiled individually).
The argument \textit{main} must be the filename of the main file.

There are a couple of
considerations in setting up the main and child documents:

%%%%%%%%%%%%%%%%%%%%%%%%%%%%%%%%%%%%%%%%
\paragraph{Restrictions.}

Please note the following restrictions:
\begin{itemize}
\item
|\childdocmain| must be called with one argument \textit{main}
to ensure compatibility with earlier version of the package.
It must either be empty (|\childdocmain{}|)
or precisely match the filename of the main file in which it is specified.
See \secref{sec:detection} for further information.
\item
The filename \textit{main} must be specified without the |.tex| extension.
\item
The filename \textit{main} is case sensitive
(even in case-insensitive file systems)
due to internal string comparison.
\item
The argument \textit{main} should be fully expanded, it cannot be a macro.
\item
Subdirectories and special characters should be avoided in filenames.
\item
The command |\childdocmain{|\textit{main}|}| must be followed by a whitespace.
It should not be followed immediately by another command
or by a comment mark `|%|'.
This is because the \TeX{} parser reads the token immediately following
the argument of |\childdocmain| and puts it
at the beginning of every child section;
however, a white\-space is ignored.
\end{itemize}

%%%%%%%%%%%%%%%%%%%%%%%%%%%%%%%%%%%%%%%%
\paragraph{Content of Main File.}

It is advisable to place all content in the child files included by |\include|.
Any output contained in the main file will appear in all child documents
unless suppressed manually;
it cannot be suppressed automatically by the |\includeonly| directive
and thus should normally be avoided.
A method to include some content in the main file
by means of conditional processing is described in \secref{sec:conditional}.

%%%%%%%%%%%%%%%%%%%%%%%%%%%%%%%%%%%%%%%%
\paragraph{Page Numbering.}

When only a part of the document is compiled,
the appropriate numbering of pages
(as well as other status parameters)
is determined from the |.aux| files.
The latter contain information from previous passes.
However this information needs to propagate through
all intermediate child documents.
Therefore the page numbering in child documents may well
be inconsistent until the complete document is compiled at least once.

A useful (if unconventional) way to always ensure a consistent
page numbering is to restart the numbering in each child document
and denote the pages by `\textit{child}|.|\textit{page}'
where \textit{child} represents the chapter/section number of the child file.
This can be achieved by the command
|\numberwithin{page}{|\textit{child}|}|
of the \textsf{amsmath} package
where \textit{child} can be |chapter| or |section|
depending on the chosen structuring.
Alternatively, one can modify the macro |\thepage| appropriately
and reset the counter |page| at the start of each child file.

%%%%%%%%%%%%%%%%%%%%%%%%%%%%%%%%%%%%%%%%%%%%%%%%%%%%%%%%%%%%%%%%%%%%%%%%%%%%%%%%
\subsection{Conditional Processing}
\label{sec:conditional}

The package provides a mechanism to compile different versions
of a document. To customise the versions further some conditional processing
can come in handy to distinguish which version is being compiled.
The package provides two macros to describe the compilation context:

%%%%%%%%%%%%%%%%%%%%%%%%%%%%%%%%%%%%%%%%
\DescribeMacro{\ifchilddoc}
The conditional |\ifchilddoc| distinguishes between the compilation of
child documents and the main document:
%
\begin{center}
|\ifchilddoc |\textit{child-code}| |[|\||else |\textit{main-code}]| \||fi|
\end{center}

%%%%%%%%%%%%%%%%%%%%%%%%%%%%%%%%%%%%%%%%
\DescribeMacro{\childdocname}
\DescribeMacro{\childdocjob}
The macro |\childdocname| contains the filename (without extension)
of the main or child file being processed.
Note that |\childdocjob| will always contain the name of the main file.

%%%%%%%%%%%%%%%%%%%%%%%%%%%%%%%%%%%%%%%%
\paragraph{Title Page.}

Conditional processing can be used to include a title or banner page
in the main document when proper precautions are taken.
Importantly, the code in the main file should ensure that the page counter
(as well as other status parameters which are stored in the |.aux| files)
takes the same value after the conditional processing.
Otherwise the page numbers may take divergent values
depending on which part is compiled.

For example, a title page could be declared by:
%
\begin{center}
\begin{tabular}{l}
|\ifchilddoc\||else|\\
|\addtocounter{page}{-1}|\\
\textit{code for title page}\\
|\newpage|\\
|\||fi|
\end{tabular}
\end{center}
%
A banner page for the child documents can be generated by:
%
\begin{center}
\begin{tabular}{l}
|\ifchilddoc|\\
|\addtocounter{page}{-1}|\\
\textit{code for banner page}\\
|\newpage|\\
|\||fi|
\end{tabular}
\end{center}
%
Here one could write a message such as:
\begin{center}
|This is the part \childdocname{} of \childdocjob{}.|
\end{center}

%%%%%%%%%%%%%%%%%%%%%%%%%%%%%%%%%%%%%%%%%%%%%%%%%%%%%%%%%%%%%%%%%%%%%%%%%%%%%%%%
\subsection{Flags}
\label{sec:flags}

The package makes it easy to generate different versions
of the main or child documents.
To this end compilation flags can be defined
and assigned different default values.
They will be particularly useful in conjunction
with the forwarding mechanism described in \secref{sec:forward}.

For example, it may be useful to have a flag |\version|
which can be set to |draft| or |final|.
The document source will contain some conditional code
depending on the value of |\version|.
Suppose further, the flag should default to |final| for the main file
and to |draft| for child files
which is a natural assignment for editing the document.
This is achieved by placing the following code
in the preamble of the main document
(below the |\childdocmain| directive):
%
\begin{center}
\begin{tabular}{l}
|\ifchilddoc|\\
|\providecommand{\version}{draft}|\\
|\||else|\\
|\providecommand{\version}{final}|\\
|\||fi|
\end{tabular}
\end{center}
%
The definition by |\providecommand| makes sure
that previous definitions are not overwritten.
Further statements |\providecommand{\version}{...}|
can thus be added before the above code to override it.

For the main file, one might add a line
(between |\childdocmain| and the above block)
%
\begin{center}
|%\ifchilddoc\||else\providecommand{\version}{draft}\||fi|
\end{center}
%
which can be uncommented to produce a draft version.
Likewise one can add a line to the very top of a child file
(above the |\childdocof{|\textit{main}|}| directive)
%
\begin{center}
|%\providecommand{\version}{final}|
\end{center}
%
which can be uncommented to produce the final version of this child document.

%%%%%%%%%%%%%%%%%%%%%%%%%%%%%%%%%%%%%%%%%%%%%%%%%%%%%%%%%%%%%%%%%%%%%%%%%%%%%%%%
\subsection{Forwarding}
\label{sec:forward}

Different versions of the main or child documents
using compilation flags as described in \secref{sec:flags}
can be (permanently) stored in different files
for convenient compilation, viewing and distribution.
To this end, the package defines a command
to pass on compilation to a different file:

%%%%%%%%%%%%%%%%%%%%%%%%%%%%%%%%%%%%%%%%
\DescribeMacro{\childdocforward}
The command |\childdocforward| redirects processing to
another source file:
%
\begin{center}
\begin{tabular}{l}
|\input{childdoc.def}|\\
|\childdocforward[|\textit{main}|]{|\textit{dest}|}|\\
\end{tabular}
\end{center}
%
The argument \textit{dest} is the destination file
(without extension).
It should be the main file or one of the child files.
Note that further \textsf{childdoc} directives
such as |\childdocof| and |\childdocforward|
in the indicated file will be processed in this form.
The optional argument \textit{main}
passes on directly to the main file \textit{main}
while pretending to compile the child \textit{dest}.
This form behaves as if \textit{dest}
issues |\childdocof{|\textit{main}|}| right away,
and no further \textsf{childdoc} directives will be processed.

%%%%%%%%%%%%%%%%%%%%%%%%%%%%%%%%%%%%%%%%
\DescribeMacro{\...prefix}
In the alternative form |\childdocforwardprefix|,
%
\begin{center}
\begin{tabular}{l}
|\input{childdoc.def}|\\
|\childdocforwardprefix[|\textit{main}|]{|\textit{prefix}|}{|\textit{dest}|}|
\end{tabular}
\end{center}
%
the destination file is determined by a pattern
depending on the current file:
To make this work, the current file must be called
`{\textit{prefix}\hspace{0.2em}\textit{suffix}}'
with \textit{prefix} matching precisely the argument.
Processing is then passed on to the file
`{\textit{dest}\hspace{0.2em}\textit{suffix}}'.
Surely, the same effect is achieved by
directly specifying the
argument `{\textit{dest}\hspace{0.2em}\textit{suffix}}'
in the first form.
However, that requires to set up a different file
for each child. With the alternative form of the command
all these files can have exactly the same content
which simplifies setting them up and maintaining them.

For example, the following file |draft.tex|
with a compilation flag |\version| as described in \secref{sec:flags}
compiles the main document as a draft:
%
\begin{center}
\begin{tabular}{l}
|\def\version{draft}|\\
|\input{childdoc.def}|\\
|\childdocforward{|\textit{main}|}|
\end{tabular}
\end{center}
%
Likewise, the following files |final|\textit{nn}|.tex|
compile the final version of the child document
|child|\textit{nn}|.tex|:
%
\begin{center}
\begin{tabular}{l}
|\def\version{final}|\\
|\input{childdoc.def}|\\
|\childdocforwardprefix{final}{child}|
\end{tabular}
\end{center}
%

Note that when several versions of a main file and/or of each child file
are to be generated, it may be convenient to set up a |Makefile| or
shell script to automatise the process.

%%%%%%%%%%%%%%%%%%%%%%%%%%%%%%%%%%%%%%%%%%%%%%%%%%%%%%%%%%%%%%%%%%%%%%%%%%%%%%%%
\subsection{Command Line Processing}
\label{sec:commandline}

The effect of redirection files can also be achieved by invoking
the \LaTeX{} compiler with a more elaborate command line.
Most conveniently this should be done as part
of a shell script or a |Makefile|.

When using \textsf{childdoc} in the main file, the following
command lines effectively perform a redirection
(note that depending on the shell being used,
backslashes may have to be doubled: `|\|' $\to$ `|\\|'):
%
\begin{center}
|... -jobname "|\textit{target}|" |\\|"|[\textit{flags}]%
|\input{childdoc.def}\childdocforward[|\textit{main}|]{|\textit{dest}|}"|
\end{center}
%
Here \textit{target} is the name of the output file,
\textit{main} is the name of the main file
and \textit{dest} is the name of the main or child file to be processed
(all filenames without extensions).
The optional argument \textit{main} can be omitted
if \textit{main} matches \textit{dest}.
Optionally, compilation \textit{flags} can be defined via |\def| commands.
This command line makes the \TeX{} engine believe
it is compiling the file \textit{target}
whose content is specified as the latter parameter.
The provided code then forwards the processing to
\textit{main} or \textit{dest} as described in \secref{sec:forward}.

%%%%%%%%%%%%%%%%%%%%%%%%%%%%%%%%%%%%%%%%%%%%%%%%%%%%%%%%%%%%%%%%%%%%%%%%%%%%%%%%
\subsection{Include by Input}
\label{sec:input}

Including child documents by |\include| has some restrictions by design.
Most notably, the content of a child document always occupies
its own set of pages; pages cannot be shared between child documents.
Usually, this behaviour makes perfect sense
because each child document contain an essential part of the document.
However, in some situations it may be desirable to compose
a document from a collection of parts
without having mandatory page breaks between then.
For this case, the package
provides a mechanism to include parts
by |\input| which can also be processed individually.
However, by construction this mechanism
requires manual handling of the content to be output.

%%%%%%%%%%%%%%%%%%%%%%%%%%%%%%%%%%%%%%%%
\DescribeMacro{\ifchilddocmanual}
The main file should be prepared as usual, see \secref{sec:include}.
However, the document body must make a distinction
between processing of an individual part and of the main document, e.g.:
%
\begin{center}
\begin{tabular}{l}
|\ifchilddocmanual|\\
|\input{\childdocname}|\\
|\||else|\\
\textit{document body with }|\input{|\textit{part}|}|\\
|\||fi|
\end{tabular}
\end{center}
%
The conditional |\ifchilddocmanual| is true whenever
a part to be included by |\input| is being compiled,
and the name of the part is stored in |\childdocname|.

%%%%%%%%%%%%%%%%%%%%%%%%%%%%%%%%%%%%%%%%
\DescribeMacro{\childdocby}
Each part to be included by |\input| should start with:
%
\begin{center}
\begin{tabular}{l}
|\input{childdoc.def}|\\
|\childdocby{|\textit{main}|}|\\
\end{tabular}
\end{center}
%
The directive |\childdocby| is similar to |\childdocof|
described in \secref{sec:include},
but the subsequent selection of content must be done manually.
To that end, both |\ifchilddoc| and |\ifchilddocmanual|
will be true upon processing of a part,
and the name of the part is stored in |\childdocname|.
Note that |\jobname| will be set to the filename of the current part
so that each part receives an individual |.aux| file
that does not interfere with the |.aux| file(s) of the main document.
This behaviour can be altered by the alternative form
|\childdocby[*]{|\textit{main}|}| (with a non-empty optional argument)
which uses the |.aux| file of the main document
by setting |\jobname| to \textit{main}.

%%%%%%%%%%%%%%%%%%%%%%%%%%%%%%%%%%%%%%%%%%%%%%%%%%%%%%%%%%%%%%%%%%%%%%%%%%%%%%%%
\subsection{Driver Development}
\label{sec:driver}

The \textsf{childdoc} mechanism can also be use for the development
of definition files such as \LaTeX{} styles or classes.
This case differs from the above setup with multiple parts
included by |\include| in that no |\includeonly| should be invoked.
This can be achieved by starting the include file
(before |\ProvidesPackage|) with:
%
\begin{center}
\begin{tabular}{l}
|\input{childdoc.def}|\\
|\childdocforward{|\textit{main}|}|\\
\end{tabular}
\end{center}
%
or alternatively with:
%
\begin{center}
\begin{tabular}{l}
|\input{childdoc.def}|\\
|\childdocby{|\textit{main}|}|\\
\end{tabular}
\end{center}
%
Both forms have slightly different effects as described above.
The main file is prepared as usual, see \secref{sec:include}.

%%%%%%%%%%%%%%%%%%%%%%%%%%%%%%%%%%%%%%%%%%%%%%%%%%%%%%%%%%%%%%%%%%%%%%%%%%%%%%%%
\subsection{Legacy Detection}
\label{sec:detection}

The directive |\childdocmain| in the main file can detect
whether the complete document or merely a child is to be compiled
even without using the directive |\childdocof|.
This method is deprecated because it is less robust
and there is no compelling reason to use it;
it is merely provided for backward compatibility
and it may be removed in future versions.

If the detection mechanism is to be used,
it is mandatory to correctly specify
the filename of the main file as the argument of |\childdocmain|:
%
\begin{center}
\begin{tabular}{l}
|\input{childdoc.def}|\\
|\childdocmain{|\textit{main}|}|\\
\end{tabular}
\end{center}
%
If |\jobname| does not match the argument \textit{main} of |\childdocmain|,
it is assumed that |\jobname| points to the child file to be compiled.
When using |\childdocmain| with the main file specified as argument,
it suffices to start a child file
with just |\input{|\textit{main}|}|
without loading of the package and using |\childdocof|.
If instead all processing is done
with the appropriate \textsf{childdoc} directives,
the argument of \textit{main} of |\childdocmain| can be empty.

An alternative version of the command line processing described
in \secref{sec:commandline} using the detection mechanism reads:
%
\begin{center}
|... -jobname "|\textit{target}|" "|[\textit{flags}]%
[|\def\jobname{|\textit{dest}|}|]|\input{|\textit{main}|}"|
\end{center}

%%%%%%%%%%%%%%%%%%%%%%%%%%%%%%%%%%%%%%%%%%%%%%%%%%%%%%%%%%%%%%%%%%%%%%%%%%%%%%%%
\subsection{Manual Code}
\label{sec:manual}

In case one cannot be certain whether the definitions file |childdoc.def|
is installed on the target \TeX{} distribution
and one prefers not to ship it,
it is conceivable to paste a few relevant commands into the sources.

To that end, drop all statements |\input{childdoc.def}|
and perform the replacements as outlined below.
Instead of |\childdocmain{|\textit{main}|}| add the following code
to the top of the main file:
%
\begin{center}
\begin{tabular}{l}
|\||ifdefined\childdocname\endinput\||fi\newif\ifchilddoc|\\
|\edef\childdocname{\scantokens\expandafter{\jobname\noexpand}}|\\
|\def\childdocmain{|\textit{main}|}\||ifx\childdocmain\childdocname\||else|\\
|\childdoctrue\includeonly{\childdocname}\let\jobname\childdocmain\||fi|\\
\end{tabular}
\end{center}
%
Instead of |\childdocof{|\textit{main}|}| just include the main file
at the top of each child file:
%
\begin{center}
|\input{|\textit{main}|}|
\end{center}
%
A simple redirection |\childdocforward{|\textit{dest}|}| is achieved by:
%
\begin{center}
|\def\jobname{|\textit{dest}|}\input{\jobname}|
\end{center}
%
The redirection with prefix
|\childdocforwardprefix[|\textit{prefix}|]{|\textit{dest}|}|
is accomplished by:
%
\begin{center}
\begin{tabular}{l}
|{\edef\jobname{\scantokens\expandafter{\jobname\noexpand}}|\\
|\def\redirectjob |\textit{prefix}|#1~~~{\gdef\jobname{|\textit{dest}|#1}}|\\
|\expandafter\redirectjob\jobname~~~}\input{\jobname}|
\end{tabular}
\end{center}

In an alternative approach,
child documents can be compiled by a specific command line
without additional code or specific definitions:
%
\begin{center}
|... -jobname "|\textit{target}|" "|[\textit{flags}]%
|\includeonly{|\textit{dest}|}\input{|\textit{main}|}"|
\end{center}
%

%%%%%%%%%%%%%%%%%%%%%%%%%%%%%%%%%%%%%%%%%%%%%%%%%%%%%%%%%%%%%%%%%%%%%%%%%%%%%%%%
%%%%%%%%%%%%%%%%%%%%%%%%%%%%%%%%%%%%%%%%%%%%%%%%%%%%%%%%%%%%%%%%%%%%%%%%%%%%%%%%
\section{Information}

%%%%%%%%%%%%%%%%%%%%%%%%%%%%%%%%%%%%%%%%%%%%%%%%%%%%%%%%%%%%%%%%%%%%%%%%%%%%%%%%
\subsection{Copyright}

Copyright \copyright{} 2017--2018 Niklas Beisert

This work may be distributed and/or modified under the
conditions of the \LaTeX{} Project Public License, either version 1.3
of this license or (at your option) any later version.
The latest version of this license is in
  \url{http://www.latex-project.org/lppl.txt}
and version 1.3 or later is part of all distributions of \LaTeX{}
version 2005/12/01 or later.

This work has the LPPL maintenance status `maintained'.

The Current Maintainer of this work is Niklas Beisert.

This work consists of the files |README.txt|, |childdoc.ins| and |childdoc.dtx|
as well as the derived files |childdoc.def|, |cdocsamp.tex|
with |cdocsch1.tex|, |cdocsch2.tex|, |cdocspt3.tex|, |cdocspt4.tex|,
|cdocsdrf.tex|, |cdocsfn1.tex|, |cdocsfn2.tex|
as well as |childdoc.pdf|.

%%%%%%%%%%%%%%%%%%%%%%%%%%%%%%%%%%%%%%%%%%%%%%%%%%%%%%%%%%%%%%%%%%%%%%%%%%%%%%%%
\subsection{Files and Installation}

The package consists of the files:
%
\begin{center}
\begin{tabular}{ll}
    |README.txt|   & readme file \\
    |childdoc.ins| & installation file \\
    |childdoc.dtx| & source file \\
    |childdoc.def| & definition file \\
    |cdocsamp.tex| & sample main file \\
    |cdocsch1.tex| & sample include file \\
    |cdocsch2.tex| & sample include file \\
    |cdocspt3.tex| & sample part file \\
    |cdocspt4.tex| & sample part file \\
    |cdocsdrf.tex| & sample redirection file \\
    |cdocsfn1.tex| & sample redirection file \\
    |cdocsfn2.tex| & sample redirection file \\
    |childdoc.pdf| & manual
\end{tabular}
\end{center}
%
The distribution consists of the files
|README.txt|, |childdoc.ins| and |childdoc.dtx|.
%
\begin{itemize}
\item
Run (pdf)\LaTeX{} on |childdoc.dtx|
to compile the manual |childdoc.pdf| (this file).
\item
Run \LaTeX{} on |childdoc.ins| to create the definitions file |childdoc.def|
and the sample |cdocsamp.tex| with include files
|cdocsch1.tex|, |cdocsch2.tex|, |cdocspt3.tex|, |cdocspt4.tex|,
|cdocsdrf.tex|, |cdocsfn1.tex|, |cdocsfn2.tex|.
Then copy the file |childdoc.def| to an appropriate directory of your \LaTeX{}
distribution, e.g.\ \textit{texmf-root}|/tex/latex/childdoc|.
\end{itemize}

%%%%%%%%%%%%%%%%%%%%%%%%%%%%%%%%%%%%%%%%%%%%%%%%%%%%%%%%%%%%%%%%%%%%%%%%%%%%%%%%
\subsection{Related CTAN Packages}

There are several other packages which offer a similar functionality:
%
\begin{itemize}
\item
The packages
\href{http://ctan.org/pkg/docmute}{\textsf{docmute}},
\href{http://ctan.org/pkg/includex}{\textsf{includex}} and
\href{http://ctan.org/pkg/standalone}{\textsf{standalone}}
provide commands to include only the document body of
a child file thus allowing both files to be compiled individually.
\item
The packages \href{http://ctan.org/pkg/subdocs}{\textsf{subdocs}}
and \href{http://ctan.org/pkg/subfiles}{\textsf{subfiles}}
provide structures in which the main and child documents can be
encapsulated and allowing them to be compiled individually.
The inclusion mechanism is different from the conventional |\include|.
\item
The package \href{http://ctan.org/pkg/combine}{\textsf{combine}}
is an elaborate solution to combine several documents into one.
\end{itemize}
%
See also the CTAN topic \href{http://ctan.org/topic/subdocs}{\textsf{subdocs}}
for further related packages.
The present package differs from the above solutions in that
a document structure constructed with the conventional |\include| mechanism
just needs two extra commands at the top of every file
such that all constituent files can be compiled individually.

%%%%%%%%%%%%%%%%%%%%%%%%%%%%%%%%%%%%%%%%%%%%%%%%%%%%%%%%%%%%%%%%%%%%%%%%%%%%%%%%
%\subsection{Feature Suggestions}
%
%The following is a list of features which may be useful for future
%versions of this package:
%%
%\begin{itemize}
%\item
%\ldots
%\end{itemize}

%%%%%%%%%%%%%%%%%%%%%%%%%%%%%%%%%%%%%%%%%%%%%%%%%%%%%%%%%%%%%%%%%%%%%%%%%%%%%%%%
\subsection{Revision History}

%%%%%%%%%%%%%%%%%%%%%%%%%%%%%%%%%%%%%%%%
\paragraph{v2.0:} 2018/12/30

\begin{itemize}
\item
immediate forward processing
\item
added |\childdocby| mechanism
\item
manual restructured
\end{itemize}

%%%%%%%%%%%%%%%%%%%%%%%%%%%%%%%%%%%%%%%%
\paragraph{v1.6:} 2018/01/17

\begin{itemize}
\item
application for development of include files
\item
corrections to manual
\end{itemize}

%%%%%%%%%%%%%%%%%%%%%%%%%%%%%%%%%%%%%%%%
\paragraph{v1.5:} 2017/05/21

\begin{itemize}
\item
more complete structuring introduced
\item
|\childdocof| introduced
\item
|\childdoc| renamed to |\childdocmain|
\item
|\childredirect| renamed to |\childdocforward| and |\childdocforwardprefix|
and functionality expanded
\end{itemize}

%%%%%%%%%%%%%%%%%%%%%%%%%%%%%%%%%%%%%%%%
\paragraph{v1.0:} 2017/04/27

\begin{itemize}
\item
manual and install package
\item
first version published on CTAN
\end{itemize}

%%%%%%%%%%%%%%%%%%%%%%%%%%%%%%%%%%%%%%%%
\paragraph{v0.6:} 2017/04/26

\begin{itemize}
\item
redirection mechanism added
\end{itemize}

%%%%%%%%%%%%%%%%%%%%%%%%%%%%%%%%%%%%%%%%
\paragraph{v0.5:} 2017/04/26

\begin{itemize}
\item
functionality in definition file
\end{itemize}


%%%%%%%%%%%%%%%%%%%%%%%%%%%%%%%%%%%%%%%%%%%%%%%%%%%%%%%%%%%%%%%%%%%%%%%%%%%%%%%%
%%%%%%%%%%%%%%%%%%%%%%%%%%%%%%%%%%%%%%%%%%%%%%%%%%%%%%%%%%%%%%%%%%%%%%%%%%%%%%%%
%%%%%%%%%%%%%%%%%%%%%%%%%%%%%%%%%%%%%%%%%%%%%%%%%%%%%%%%%%%%%%%%%%%%%%%%%%%%%%%%
\appendix

\settowidth\MacroIndent{\rmfamily\scriptsize 000\ }

 \DocInput{childdoc.dtx}

\end{document}
%</driver>
% \fi
%
% %%%%%%%%%%%%%%%%%%%%%%%%%%%%%%%%%%%%%%%%%%%%%%%%%%%%%%%%%%%%%%%%%%%%%%%%%%%%%%
% %%%%%%%%%%%%%%%%%%%%%%%%%%%%%%%%%%%%%%%%%%%%%%%%%%%%%%%%%%%%%%%%%%%%%%%%%%%%%%
% \section{Sample}
%\iffalse
%<*samplemain>
%\fi
%
% The following presents a sample document
% with two chapters, two parts, a title page,
% a compile flag as well as three forwarding files to set the flag.
% It consists of eight |.tex| files:
% \begin{center}
% \begin{tabular}{ll}
% |cdocsamp.tex|&main file\\
% |cdocsch1.tex|&include file for chapter 1\\
% |cdocsch2.tex|&include file for chapter 2\\
% |cdocspt3.tex|&include file for part 3\\
% |cdocspt4.tex|&include file for part 4\\
% |cdocsdrf.tex|&forwarding file for main file in draft mode\\
% |cdocsfi1.tex|&forwarding file for final version of chapter 1\\
% |cdocsfi2.tex|&forwarding file for final version of chapter 2\\
% \end{tabular}
% \end{center}
% Each of the eight files can be compiled directly by the \LaTeX{} compiler.
%
% %%%%%%%%%%%%%%%%%%%%%%%%%%%%%%%%%%%%%%
% \paragraph{Main File.}
%
% The main file is called |cdocsamp.tex|.
%
% Load the \textsf{childdoc} definitions and
% declare the filename for the main document:
%    \begin{macrocode}
\input{childdoc.def}
\childdocmain{}
%    \end{macrocode}

% Optional override for |\version| flag:
%    \begin{macrocode}
%%\ifchilddoc\else\providecommand{\version}{draft}\fi
%    \end{macrocode}

% Define the default values for the |\version| flag
% (|final| for the main file and |draft| for childs):
%    \begin{macrocode}
\ifchilddoc
\providecommand{\version}{draft}
\else
\providecommand{\version}{final}
\fi
%    \end{macrocode}

% Load the standard document class:
%    \begin{macrocode}
\documentclass[12pt]{article}
%    \end{macrocode}

% Start the document body:
%    \begin{macrocode}
\begin{document}
%    \end{macrocode}

% Declare a title page.
% Print title, part of document being processed and version flag:
%    \begin{macrocode}
\addtocounter{page}{-1}
\begin{center}
{\LARGE\bfseries{}childdoc example\par}
\vspace{1cm}
\ifchilddoc
\ifchilddocmanual part\else chapter\fi:
`\childdocname' of `\childdocjob'\par
\else
main document: `\childdocjob'\par
\fi
version: \version\par
\end{center}
\newpage
%    \end{macrocode}

% Manually include selected file,
% otherwise process as usual:
%    \begin{macrocode}
\ifchilddocmanual
\section*{part `\childdocname'}
\input{\childdocname}
\else
%    \end{macrocode}

% Include the two chapters:
%    \begin{macrocode}
\include{cdocsch1}
\include{cdocsch2}
%    \end{macrocode}

% Include the two parts unless only chapters should be displayed:
%    \begin{macrocode}
\ifchilddoc\else
\section{part three}
\input{cdocspt3}
\section{part four}
\input{cdocspt4}
\fi
%    \end{macrocode}

% Process as usual until here:
%    \begin{macrocode}
\fi
%    \end{macrocode}

% End of document body:
%    \begin{macrocode}
\end{document}
%    \end{macrocode}
%\iffalse
%</samplemain>
%\fi
%
% %%%%%%%%%%%%%%%%%%%%%%%%%%%%%%%%%%%%%%
% \paragraph{Chapter Include Files.}
%
% The include files are called |cdocsch1.tex| and |cdocsch2.tex|.
%
%\iffalse
%<*samplechap1|samplechap2>
%\fi

% Optional override for |\version| flag:
%    \begin{macrocode}
%%\providecommand{\version}{final}
%    \end{macrocode}

% Include the main document:
%    \begin{macrocode}
\input{childdoc.def}
\childdocof{cdocsamp}
%    \end{macrocode}

%\iffalse
%</samplechap1|samplechap2>
%\fi
%
%\iffalse
%<*samplechap1>
%\fi
% Some text for chapter 1:
%    \begin{macrocode}
\section{one}
some text in chapter one
%    \end{macrocode}

%\iffalse
%</samplechap1>
%\fi
% Some text for chapter 2:
%\iffalse
%<*samplechap2>
%\fi
%    \begin{macrocode}
\section{two}
more text in chapter two
%    \end{macrocode}

%\iffalse
%</samplechap2>
%\fi
%
% %%%%%%%%%%%%%%%%%%%%%%%%%%%%%%%%%%%%%%
% \paragraph{Part Include Files.}
%
% The include files are called |cdocspt3.tex| and |cdocspt4.tex|.
%
%\iffalse
%<*samplepart3|samplepart4>
%\fi

% Optional override for |\version| flag:
%    \begin{macrocode}
%%\providecommand{\version}{final}
%    \end{macrocode}

% Include the main document:
%    \begin{macrocode}
\input{childdoc.def}
\childdocby{cdocsamp}
%    \end{macrocode}

%\iffalse
%</samplepart3|samplepart4>
%\fi
%
%\iffalse
%<*samplepart3>
%\fi
% Some text for part 3:
%    \begin{macrocode}
some text in part three
%    \end{macrocode}

%\iffalse
%</samplepart3>
%\fi
% Some text for part 4:
%\iffalse
%<*samplepart4>
%\fi
%    \begin{macrocode}
more text in part four
%    \end{macrocode}

%\iffalse
%</samplepart4>
%\fi
%
% %%%%%%%%%%%%%%%%%%%%%%%%%%%%%%%%%%%%%%
% \paragraph{Forwarding for a Complete Draft.}
%
% The following forwarding file |cdocsdrf.tex|
% compiles the main document in draft mode:
%\iffalse
%<*sampledraft>
%\fi
%    \begin{macrocode}
\def\version{draft}
\input{childdoc.def}
\childdocforward{cdocsamp}
%    \end{macrocode}

%\iffalse
%</sampledraft>
%\fi
%
% %%%%%%%%%%%%%%%%%%%%%%%%%%%%%%%%%%%%%%
% \paragraph{Forwarding for Final Version of the Chapters.}
%
% The following forwarding files |cdocsfn1.tex| and |cdocsfn2.tex|
% (with identical content)
% compile the final versions of the child documents
% |cdocsch1.tex| and |cdocsch2.tex|, respectively:
%\iffalse
%<*samplefinal>
%\fi
%    \begin{macrocode}
\def\version{final}
\input{childdoc.def}
\childdocforwardprefix[cdocsamp]{cdocsfn}{cdocsch}
%    \end{macrocode}

%\iffalse
%</samplefinal>
%\fi
%
% %%%%%%%%%%%%%%%%%%%%%%%%%%%%%%%%%%%%%%
% \paragraph{Command Line Processing.}
%
% The following three command lines generate the output files
% |cdocscld|, |cdocscl1| and |cdocscl2|
% which should be identical to
% |cdocsdrf|, |cdocsch1| and |cdocsfn2|, respectively:
% \begin{center}
% \begin{tabular}{l}
% |latex -jobname cdocscld \|\\
% |  "\def\version{draft}\input{childdoc.def}\childdocforward{cdocsamp}"|\\
% |latex -jobname cdocscl1 \|\\
% |  "\input{childdoc.def}\childdocforward[cdocsamp]{cdocsch1}"|\\
% |latex -jobname cdocscl2 \|\\
% |  "\def\version{final}\input{childdoc.def}\childdocforward{cdocsch2}"|
% \end{tabular}
% \end{center}
% Note that the trailing backslash on each first line
% merely continues the input to the second line
% (for convenient cut ant paste).
% Furthermore, the command |latex| can be replaced by any
% of its alternative versions such as |pdflatex|.
%
% %%%%%%%%%%%%%%%%%%%%%%%%%%%%%%%%%%%%%%%%%%%%%%%%%%%%%%%%%%%%%%%%%%%%%%%%%%%%%%
% %%%%%%%%%%%%%%%%%%%%%%%%%%%%%%%%%%%%%%%%%%%%%%%%%%%%%%%%%%%%%%%%%%%%%%%%%%%%%%
% \section{Implementation}
%\iffalse
%<*package>
%\fi
%
% This section describes the definitions file |childdoc.def|.

% The definitions cannot be loaded using |\usepackage| or |\RequirePackage|
% which has a mechanism to prevent loading a style file more than once.
% When loading the definitions by means of |\input|
% multiple instances have to be prevented manually:
%\iffalse
%This code needs to be before the `\ProvidesFile' directive
%which is defined at the beginning of this file.
%Therefore it is also placed there and commented out here.
%</package>
%<*discard>
%\fi
%    \begin{macrocode}
\ifdefined\childdocmain\endinput\fi
%    \end{macrocode}
%\iffalse
%</discard>
%<*package>
%\fi
%
% \macro{\ifchilddoc}
% \macro{\ifchilddocmanual}
% The conditional |\ifchilddoc| tells whether a
% child (true) or main (false) document is being compiled.
% The conditional |\ifchilddocmanual| tells whether
% the |\includeonly| mechanism is used (false) or
% the selection of child files must be performed manually (true).
% The definitions initialise to false:
%    \begin{macrocode}
\newif\ifchilddoc
\newif\ifchilddocmanual
%    \end{macrocode}

% \macro{\childdocname}
% \macro{\childdocjob}
% The macro |\childdocname| stores the name of the main document
% to be compiled. The macro |\childdocjob| stores the name of
% the document on which the \LaTeX{} compiler was originally invoked.
% The content of |\jobname| cannot be compared
% to filenames specified in the source due to different catcodes.
% The following code rescans |\jobname|, stores the result
% in |\childdocname| and saves a copy in |\childdocjob|:
%    \begin{macrocode}
\edef\childdocname{\scantokens\expandafter{\jobname\noexpand}}
\let\childdocjob\childdocname
%    \end{macrocode}

% \macro{\childdocdisable}
% The macro |\childdocdisable| prevents the main file
% from being processed more than once.
% At this stage, the main document command |\childdocmain|
% is assumed to be called once again where it should do nothing.
% Any subsequent call to it should prevent
% a secondary processing of the main document
% It overwrites the forwarding commands
% |\childdocof| and |\childdocforward|
% with empty macros to prevent further inclusions of the main document:
%    \begin{macrocode}
\newcommand{\childdocdisable}
{
  \renewcommand{\childdocmain}[1]{\renewcommand{\childdocmain}[1]{\endinput}}
  \renewcommand{\childdocof}[1]{}
  \renewcommand{\childdocby}[2][]{}
  \renewcommand{\childdocforward}[2][]{}
  \renewcommand{\childdocdisable}{}
}
%    \end{macrocode}

% \macro{\childdocmain}
% The macro |\childdocmain| is to be called at the top of the main file
% with nothing or the main filename (without extension) as argument.
% First, it breaks loops.
% If the argument is not empty and does not match |\childdocname|
% (which is set by the first inclusion of |childdoc.def|),
% |\ifchilddoc| is set to true, |\includeonly| is applied to the child file
% and |\jobname| is set to the main file
% (for proper handling of |.aux| files):
%    \begin{macrocode}
\newcommand{\childdocmain}[1]
{
  \childdocdisable\childdocmain{}
  \if?#1?\else
    \begingroup
      \def\childdoctmp{#1}
      \ifx\childdoctmp\childdocname
        \def\childdoctmp{}
      \else
        \def\childdoctmp
        {
          \childdoctrue
          \includeonly{\childdocname}
          \def\childdocjob{#1}
          \def\jobname{#1}
        }
      \fi
      \expandafter
    \endgroup
    \childdoctmp
  \fi
}
%    \end{macrocode}

% \macro{\childdocof}
% The command |\childdocof| redirects
% compilation to the main file |#1|.
%    \begin{macrocode}
\newcommand{\childdocof}[1]
{
  \childdocdisable
  \childdoctrue
  \includeonly{\childdocname}
  \def\jobname{#1}
  \def\childdocjob{#1}
  \input{#1}
}
%    \end{macrocode}

% \macro{\childdocby}
% The command |\childdocby| ....
%    \begin{macrocode}
\newcommand{\childdocby}[2][]
{
  \childdocdisable
  \childdoctrue
  \childdocmanualtrue
  \if?#1?\else
    \def\jobname{#2}
  \fi
  \def\childdocjob{#2}
  \input{#2}
  \endinput
}
%    \end{macrocode}

% \macro{\childdocforward}
% The command |\childdocforward| redirects
% compilation to the main file or
% (if the optional argument is given) a child file.
% Parameters are set as if the main file
% or a child file starting with |\childdocof| was compiled.
% Then compilation is handed over to the main file:
%    \begin{macrocode}
\newcommand{\childdocforward}[2][]
{
  \begingroup
    \if?#1?
      \def\childdoctmp
      {
        \def\childdocname{#2}
        \def\childdocjob{#2}
        \def\jobname{#2}
        \input{#2}
        \endinput
      }
    \else
      \def\childdoctmp
      {
        \childdocdisable
        \def\childdocname{#2}
        \childdoctrue
        \includeonly{#2}
        \def\childdocjob{#1}
        \def\jobname{#1}
        \input{#1}
        \endinput
      }
    \fi
    \expandafter
  \endgroup
  \childdoctmp
}
%    \end{macrocode}

% \macro{\childdocforwardprefix}
% The command |\childdocforwardprefix| redirects
% compilation to the main or a child file by means of a pattern.
% The prefix |#1| in the current filename is replaced by |#2|
% and the suffix of the current filename is kept
% (it is assumed that the filename does not contain the substring `|~~~|'
% which is used as a delimiter).
% Compilation is handed over to the new file by |\childdocforward|:
%    \begin{macrocode}
\newcommand{\childdocforwardprefix}[3][]
{
  \begingroup
    \def\childdocextract #2##1~~~{\def\childdoctmp{\childdocforward[#1]{#3##1}}}
    \expandafter\childdocextract\childdocname~~~
    \expandafter
  \endgroup
  \childdoctmp
}
%    \end{macrocode}

% \macro{\childdoc}
% The deprecated macro |\childdoc| is a legacy version of |\childdocmain|:
%    \begin{macrocode}
\newcommand{\childdoc}{\childdocmain}
%    \end{macrocode}

% \macro{\childdocredirect}
% The deprecated macro |\childdocredirect| is a legacy version
% of |\childdocforward| and |\childdocforwardprefix|:
%    \begin{macrocode}
\newcommand{\childdocredirect}[2][]
{
  \begingroup
    \if?#1?
      \def\childdoctmp{\childdocforward{#2}}
    \else
      \def\childdoctmp{\childdocforwardprefix{#1}{#2}}
    \fi
    \expandafter
  \endgroup
  \childdoctmp
}
%    \end{macrocode}

%\iffalse
%</package>
%\fi
%
\endinput

\childdocmain{}
%    \end{macrocode}

% Optional override for |\version| flag:
%    \begin{macrocode}
%%\ifchilddoc\else\providecommand{\version}{draft}\fi
%    \end{macrocode}

% Define the default values for the |\version| flag
% (|final| for the main file and |draft| for childs):
%    \begin{macrocode}
\ifchilddoc
\providecommand{\version}{draft}
\else
\providecommand{\version}{final}
\fi
%    \end{macrocode}

% Load the standard document class:
%    \begin{macrocode}
\documentclass[12pt]{article}
%    \end{macrocode}

% Start the document body:
%    \begin{macrocode}
\begin{document}
%    \end{macrocode}

% Declare a title page.
% Print title, part of document being processed and version flag:
%    \begin{macrocode}
\addtocounter{page}{-1}
\begin{center}
{\LARGE\bfseries{}childdoc example\par}
\vspace{1cm}
\ifchilddoc
\ifchilddocmanual part\else chapter\fi:
`\childdocname' of `\childdocjob'\par
\else
main document: `\childdocjob'\par
\fi
version: \version\par
\end{center}
\newpage
%    \end{macrocode}

% Manually include selected file,
% otherwise process as usual:
%    \begin{macrocode}
\ifchilddocmanual
\section*{part `\childdocname'}
\input{\childdocname}
\else
%    \end{macrocode}

% Include the two chapters:
%    \begin{macrocode}
\include{cdocsch1}
\include{cdocsch2}
%    \end{macrocode}

% Include the two parts unless only chapters should be displayed:
%    \begin{macrocode}
\ifchilddoc\else
\section{part three}
\input{cdocspt3}
\section{part four}
\input{cdocspt4}
\fi
%    \end{macrocode}

% Process as usual until here:
%    \begin{macrocode}
\fi
%    \end{macrocode}

% End of document body:
%    \begin{macrocode}
\end{document}
%    \end{macrocode}
%\iffalse
%</samplemain>
%\fi
%
% %%%%%%%%%%%%%%%%%%%%%%%%%%%%%%%%%%%%%%
% \paragraph{Chapter Include Files.}
%
% The include files are called |cdocsch1.tex| and |cdocsch2.tex|.
%
%\iffalse
%<*samplechap1|samplechap2>
%\fi

% Optional override for |\version| flag:
%    \begin{macrocode}
%%\providecommand{\version}{final}
%    \end{macrocode}

% Include the main document:
%    \begin{macrocode}
% \iffalse
%
% childdoc.dtx Copyright (C) 2017-2018 Niklas Beisert
%
% This work may be distributed and/or modified under the
% conditions of the LaTeX Project Public License, either version 1.3
% of this license or (at your option) any later version.
% The latest version of this license is in
%   http://www.latex-project.org/lppl.txt
% and version 1.3 or later is part of all distributions of LaTeX
% version 2005/12/01 or later.
%
% This work has the LPPL maintenance status `maintained'.
%
% The Current Maintainer of this work is Niklas Beisert.
%
% This work consists of the files childdoc.dtx and childdoc.ins
% and the derived files childdoc.def and cdocsamp.tex with
% cdocsch1.tex, cdocsch2.tex, cdocsdrf.tex, cdocsfn1.tex, cdocsfn2.tex.
%
%<package>\ifdefined\childdocmain\endinput\fi
%<package>\ProvidesFile{childdoc.def}[2018/12/30 v2.0 child document driver]
%<samplemain>\ProvidesFile{cdocsamp.tex}[2018/12/30 v2.0 sample for childdoc]
%<*driver>
%\ProvidesFile{childdoc.drv}[2018/12/30 v2.0 childdoc reference manual file]
\PassOptionsToClass{10pt,a4paper}{article}
\documentclass{ltxdoc}

\usepackage[margin=35mm]{geometry}
\usepackage{hyperref}
\usepackage{hyperxmp}
\usepackage[usenames]{color}

\hypersetup{colorlinks=true}
\hypersetup{pdfstartview=FitH}
\hypersetup{pdfpagemode=UseNone}
\hypersetup{pdfsource={}}
\hypersetup{pdflang={en-UK}}
\hypersetup{pdfcopyright={Copyright 2017-2018 Niklas Beisert.
  This work may be distributed and/or modified under the
  conditions of the LaTeX Project Public License, either version 1.3
  of this license or (at your option) any later version.}}
\hypersetup{pdflicenseurl={http://www.latex-project.org/lppl.txt}}
\hypersetup{pdfcontactaddress={ETH Zurich, ITP, HIT K,
  Wolfgang-Pauli-Strasse 27}}
\hypersetup{pdfcontactpostcode={8093}}
\hypersetup{pdfcontactcity={Zurich}}
\hypersetup{pdfcontactcountry={Switzerland}}
\hypersetup{pdfcontactemail={nbeisert@itp.phys.ethz.ch}}
\hypersetup{pdfcontacturl={http://people.phys.ethz.ch/\xmptilde nbeisert/}}

\newcommand{\secref}[1]{\hyperref[#1]{section \ref*{#1}}}

\parskip1ex
\parindent0pt
\let\olditemize\itemize
\def\itemize{\olditemize\parskip0pt}

\begin{document}

\title{The \textsf{childdoc} Package}
\hypersetup{pdftitle={The childdoc Package}}
\author{Niklas Beisert\\[2ex]
  Institut f\"ur Theoretische Physik\\
  Eidgen\"ossische Technische Hochschule Z\"urich\\
  Wolfgang-Pauli-Strasse 27, 8093 Z\"urich, Switzerland\\[1ex]
  \href{mailto:nbeisert@itp.phys.ethz.ch}
  {\texttt{nbeisert@itp.phys.ethz.ch}}}
\hypersetup{pdfauthor={Niklas Beisert}}
\hypersetup{pdfsubject={Manual for the LaTeX2e Package childdoc}}
\date{30 December 2018, \textsf{v2.0}}
\maketitle

\begin{abstract}\noindent
\textsf{childdoc} is a \LaTeXe{} package
that enables the direct compilation
of document sections included by |\include|
to individual files.
\end{abstract}

\begingroup
\parskip0ex
\tableofcontents
\endgroup

%%%%%%%%%%%%%%%%%%%%%%%%%%%%%%%%%%%%%%%%%%%%%%%%%%%%%%%%%%%%%%%%%%%%%%%%%%%%%%%%
%%%%%%%%%%%%%%%%%%%%%%%%%%%%%%%%%%%%%%%%%%%%%%%%%%%%%%%%%%%%%%%%%%%%%%%%%%%%%%%%
\section{Introduction}

\LaTeX{} provides a mechanism to structure a large document (such as a book)
into a main file and several child files (containing the chapters)
using the |\include| command.
This mechanism is beneficial for documents
which span hundreds of pages in order to
make the source file(s) more manageable.
Moreover, compilation can be restricted to
selected child files by means of the |\includeonly| command.
The latter feature can be used to reduce the compilation time while editing
(this was significantly more useful in the earlier days of \LaTeX{})
or to generate a smaller document which is easier to navigate.
Another application of |\includeonly| is to generate
documents consisting of selected parts of the complete document.

However, there are a few drawbacks of the plain |\include| mechanism:
\begin{itemize}
\item
The child files cannot be compiled on their own,
they can only be compiled via the main file.
A naive editing environment
(such as a text editor with an option
to have the current file processed by \LaTeX)
may require one to switch to the main file before compiling;
attempting to compile the child file produces errors.
\item
The main file must be modified (each time)
to adjust the |\includeonly| command
to the present needs. This easily leaves the main file in a messy state.
\item
The generated document will always carry the filename
of the main document. This is inconvenient if
several child files are to be compiled and
to be kept for distribution.
\end{itemize}

The present package provides a simple interface
to make child files individually compilable by \LaTeX{}.
Compiling a child file then has the same effect as compiling
the main file with an |\includeonly| command
to select the appropriate child.
Moreover the generated document will carry the name of the child
rather than the main file.
This resolves all three above issues.

This feature is meant to make the editing of books,
thesis documents and lecture notes somewhat more convenient.
However, the package can also be used efficiently for
composing a series of documents (such as exercise sheets)
which are typically distributed individually.
It then assists the author in generating the individual documents
(potentially in different versions)
as well as a document containing the collected series.
Another application is in developing style files
or other kinds of included material
where compilation of the style file could redirect
to a sample or test file.

%%%%%%%%%%%%%%%%%%%%%%%%%%%%%%%%%%%%%%%%%%%%%%%%%%%%%%%%%%%%%%%%%%%%%%%%%%%%%%%%
%%%%%%%%%%%%%%%%%%%%%%%%%%%%%%%%%%%%%%%%%%%%%%%%%%%%%%%%%%%%%%%%%%%%%%%%%%%%%%%%
\section{Usage}

First of all, the package \textsf{childdoc} is \emph{not} a standard
\LaTeXe{} |.sty| style file! Therefore it needs to be invoked in
a non-standard way.

%%%%%%%%%%%%%%%%%%%%%%%%%%%%%%%%%%%%%%%%%%%%%%%%%%%%%%%%%%%%%%%%%%%%%%%%%%%%%%%%
\subsection{Included Files}
\label{sec:include}

%%%%%%%%%%%%%%%%%%%%%%%%%%%%%%%%%%%%%%%%
\DescribeMacro{\childdocmain}
To use the package, add the commands
\begin{center}
\begin{tabular}{l}
|\input{childdoc.def}|\\
|\childdocmain{}|\\
\end{tabular}
\end{center}
at the very top of the main \LaTeX{} file,
in particular \emph{before} the |\documentclass| statement!
The argument of |\childdocmain| should be left empty
(but it must be present).

%%%%%%%%%%%%%%%%%%%%%%%%%%%%%%%%%%%%%%%%
\DescribeMacro{\childdocof}
Furthermore, add the commands
\begin{center}
\begin{tabular}{l}
|\input{childdoc.def}|\\
|\childdocof{|\textit{main}|}|\\
\end{tabular}
\end{center}
at the top of every child file \textit{child}
which is included by |\include{|\textit{child}|}|
from within the main file
(or at least for those files to be compiled individually).
The argument \textit{main} must be the filename of the main file.

There are a couple of
considerations in setting up the main and child documents:

%%%%%%%%%%%%%%%%%%%%%%%%%%%%%%%%%%%%%%%%
\paragraph{Restrictions.}

Please note the following restrictions:
\begin{itemize}
\item
|\childdocmain| must be called with one argument \textit{main}
to ensure compatibility with earlier version of the package.
It must either be empty (|\childdocmain{}|)
or precisely match the filename of the main file in which it is specified.
See \secref{sec:detection} for further information.
\item
The filename \textit{main} must be specified without the |.tex| extension.
\item
The filename \textit{main} is case sensitive
(even in case-insensitive file systems)
due to internal string comparison.
\item
The argument \textit{main} should be fully expanded, it cannot be a macro.
\item
Subdirectories and special characters should be avoided in filenames.
\item
The command |\childdocmain{|\textit{main}|}| must be followed by a whitespace.
It should not be followed immediately by another command
or by a comment mark `|%|'.
This is because the \TeX{} parser reads the token immediately following
the argument of |\childdocmain| and puts it
at the beginning of every child section;
however, a white\-space is ignored.
\end{itemize}

%%%%%%%%%%%%%%%%%%%%%%%%%%%%%%%%%%%%%%%%
\paragraph{Content of Main File.}

It is advisable to place all content in the child files included by |\include|.
Any output contained in the main file will appear in all child documents
unless suppressed manually;
it cannot be suppressed automatically by the |\includeonly| directive
and thus should normally be avoided.
A method to include some content in the main file
by means of conditional processing is described in \secref{sec:conditional}.

%%%%%%%%%%%%%%%%%%%%%%%%%%%%%%%%%%%%%%%%
\paragraph{Page Numbering.}

When only a part of the document is compiled,
the appropriate numbering of pages
(as well as other status parameters)
is determined from the |.aux| files.
The latter contain information from previous passes.
However this information needs to propagate through
all intermediate child documents.
Therefore the page numbering in child documents may well
be inconsistent until the complete document is compiled at least once.

A useful (if unconventional) way to always ensure a consistent
page numbering is to restart the numbering in each child document
and denote the pages by `\textit{child}|.|\textit{page}'
where \textit{child} represents the chapter/section number of the child file.
This can be achieved by the command
|\numberwithin{page}{|\textit{child}|}|
of the \textsf{amsmath} package
where \textit{child} can be |chapter| or |section|
depending on the chosen structuring.
Alternatively, one can modify the macro |\thepage| appropriately
and reset the counter |page| at the start of each child file.

%%%%%%%%%%%%%%%%%%%%%%%%%%%%%%%%%%%%%%%%%%%%%%%%%%%%%%%%%%%%%%%%%%%%%%%%%%%%%%%%
\subsection{Conditional Processing}
\label{sec:conditional}

The package provides a mechanism to compile different versions
of a document. To customise the versions further some conditional processing
can come in handy to distinguish which version is being compiled.
The package provides two macros to describe the compilation context:

%%%%%%%%%%%%%%%%%%%%%%%%%%%%%%%%%%%%%%%%
\DescribeMacro{\ifchilddoc}
The conditional |\ifchilddoc| distinguishes between the compilation of
child documents and the main document:
%
\begin{center}
|\ifchilddoc |\textit{child-code}| |[|\||else |\textit{main-code}]| \||fi|
\end{center}

%%%%%%%%%%%%%%%%%%%%%%%%%%%%%%%%%%%%%%%%
\DescribeMacro{\childdocname}
\DescribeMacro{\childdocjob}
The macro |\childdocname| contains the filename (without extension)
of the main or child file being processed.
Note that |\childdocjob| will always contain the name of the main file.

%%%%%%%%%%%%%%%%%%%%%%%%%%%%%%%%%%%%%%%%
\paragraph{Title Page.}

Conditional processing can be used to include a title or banner page
in the main document when proper precautions are taken.
Importantly, the code in the main file should ensure that the page counter
(as well as other status parameters which are stored in the |.aux| files)
takes the same value after the conditional processing.
Otherwise the page numbers may take divergent values
depending on which part is compiled.

For example, a title page could be declared by:
%
\begin{center}
\begin{tabular}{l}
|\ifchilddoc\||else|\\
|\addtocounter{page}{-1}|\\
\textit{code for title page}\\
|\newpage|\\
|\||fi|
\end{tabular}
\end{center}
%
A banner page for the child documents can be generated by:
%
\begin{center}
\begin{tabular}{l}
|\ifchilddoc|\\
|\addtocounter{page}{-1}|\\
\textit{code for banner page}\\
|\newpage|\\
|\||fi|
\end{tabular}
\end{center}
%
Here one could write a message such as:
\begin{center}
|This is the part \childdocname{} of \childdocjob{}.|
\end{center}

%%%%%%%%%%%%%%%%%%%%%%%%%%%%%%%%%%%%%%%%%%%%%%%%%%%%%%%%%%%%%%%%%%%%%%%%%%%%%%%%
\subsection{Flags}
\label{sec:flags}

The package makes it easy to generate different versions
of the main or child documents.
To this end compilation flags can be defined
and assigned different default values.
They will be particularly useful in conjunction
with the forwarding mechanism described in \secref{sec:forward}.

For example, it may be useful to have a flag |\version|
which can be set to |draft| or |final|.
The document source will contain some conditional code
depending on the value of |\version|.
Suppose further, the flag should default to |final| for the main file
and to |draft| for child files
which is a natural assignment for editing the document.
This is achieved by placing the following code
in the preamble of the main document
(below the |\childdocmain| directive):
%
\begin{center}
\begin{tabular}{l}
|\ifchilddoc|\\
|\providecommand{\version}{draft}|\\
|\||else|\\
|\providecommand{\version}{final}|\\
|\||fi|
\end{tabular}
\end{center}
%
The definition by |\providecommand| makes sure
that previous definitions are not overwritten.
Further statements |\providecommand{\version}{...}|
can thus be added before the above code to override it.

For the main file, one might add a line
(between |\childdocmain| and the above block)
%
\begin{center}
|%\ifchilddoc\||else\providecommand{\version}{draft}\||fi|
\end{center}
%
which can be uncommented to produce a draft version.
Likewise one can add a line to the very top of a child file
(above the |\childdocof{|\textit{main}|}| directive)
%
\begin{center}
|%\providecommand{\version}{final}|
\end{center}
%
which can be uncommented to produce the final version of this child document.

%%%%%%%%%%%%%%%%%%%%%%%%%%%%%%%%%%%%%%%%%%%%%%%%%%%%%%%%%%%%%%%%%%%%%%%%%%%%%%%%
\subsection{Forwarding}
\label{sec:forward}

Different versions of the main or child documents
using compilation flags as described in \secref{sec:flags}
can be (permanently) stored in different files
for convenient compilation, viewing and distribution.
To this end, the package defines a command
to pass on compilation to a different file:

%%%%%%%%%%%%%%%%%%%%%%%%%%%%%%%%%%%%%%%%
\DescribeMacro{\childdocforward}
The command |\childdocforward| redirects processing to
another source file:
%
\begin{center}
\begin{tabular}{l}
|\input{childdoc.def}|\\
|\childdocforward[|\textit{main}|]{|\textit{dest}|}|\\
\end{tabular}
\end{center}
%
The argument \textit{dest} is the destination file
(without extension).
It should be the main file or one of the child files.
Note that further \textsf{childdoc} directives
such as |\childdocof| and |\childdocforward|
in the indicated file will be processed in this form.
The optional argument \textit{main}
passes on directly to the main file \textit{main}
while pretending to compile the child \textit{dest}.
This form behaves as if \textit{dest}
issues |\childdocof{|\textit{main}|}| right away,
and no further \textsf{childdoc} directives will be processed.

%%%%%%%%%%%%%%%%%%%%%%%%%%%%%%%%%%%%%%%%
\DescribeMacro{\...prefix}
In the alternative form |\childdocforwardprefix|,
%
\begin{center}
\begin{tabular}{l}
|\input{childdoc.def}|\\
|\childdocforwardprefix[|\textit{main}|]{|\textit{prefix}|}{|\textit{dest}|}|
\end{tabular}
\end{center}
%
the destination file is determined by a pattern
depending on the current file:
To make this work, the current file must be called
`{\textit{prefix}\hspace{0.2em}\textit{suffix}}'
with \textit{prefix} matching precisely the argument.
Processing is then passed on to the file
`{\textit{dest}\hspace{0.2em}\textit{suffix}}'.
Surely, the same effect is achieved by
directly specifying the
argument `{\textit{dest}\hspace{0.2em}\textit{suffix}}'
in the first form.
However, that requires to set up a different file
for each child. With the alternative form of the command
all these files can have exactly the same content
which simplifies setting them up and maintaining them.

For example, the following file |draft.tex|
with a compilation flag |\version| as described in \secref{sec:flags}
compiles the main document as a draft:
%
\begin{center}
\begin{tabular}{l}
|\def\version{draft}|\\
|\input{childdoc.def}|\\
|\childdocforward{|\textit{main}|}|
\end{tabular}
\end{center}
%
Likewise, the following files |final|\textit{nn}|.tex|
compile the final version of the child document
|child|\textit{nn}|.tex|:
%
\begin{center}
\begin{tabular}{l}
|\def\version{final}|\\
|\input{childdoc.def}|\\
|\childdocforwardprefix{final}{child}|
\end{tabular}
\end{center}
%

Note that when several versions of a main file and/or of each child file
are to be generated, it may be convenient to set up a |Makefile| or
shell script to automatise the process.

%%%%%%%%%%%%%%%%%%%%%%%%%%%%%%%%%%%%%%%%%%%%%%%%%%%%%%%%%%%%%%%%%%%%%%%%%%%%%%%%
\subsection{Command Line Processing}
\label{sec:commandline}

The effect of redirection files can also be achieved by invoking
the \LaTeX{} compiler with a more elaborate command line.
Most conveniently this should be done as part
of a shell script or a |Makefile|.

When using \textsf{childdoc} in the main file, the following
command lines effectively perform a redirection
(note that depending on the shell being used,
backslashes may have to be doubled: `|\|' $\to$ `|\\|'):
%
\begin{center}
|... -jobname "|\textit{target}|" |\\|"|[\textit{flags}]%
|\input{childdoc.def}\childdocforward[|\textit{main}|]{|\textit{dest}|}"|
\end{center}
%
Here \textit{target} is the name of the output file,
\textit{main} is the name of the main file
and \textit{dest} is the name of the main or child file to be processed
(all filenames without extensions).
The optional argument \textit{main} can be omitted
if \textit{main} matches \textit{dest}.
Optionally, compilation \textit{flags} can be defined via |\def| commands.
This command line makes the \TeX{} engine believe
it is compiling the file \textit{target}
whose content is specified as the latter parameter.
The provided code then forwards the processing to
\textit{main} or \textit{dest} as described in \secref{sec:forward}.

%%%%%%%%%%%%%%%%%%%%%%%%%%%%%%%%%%%%%%%%%%%%%%%%%%%%%%%%%%%%%%%%%%%%%%%%%%%%%%%%
\subsection{Include by Input}
\label{sec:input}

Including child documents by |\include| has some restrictions by design.
Most notably, the content of a child document always occupies
its own set of pages; pages cannot be shared between child documents.
Usually, this behaviour makes perfect sense
because each child document contain an essential part of the document.
However, in some situations it may be desirable to compose
a document from a collection of parts
without having mandatory page breaks between then.
For this case, the package
provides a mechanism to include parts
by |\input| which can also be processed individually.
However, by construction this mechanism
requires manual handling of the content to be output.

%%%%%%%%%%%%%%%%%%%%%%%%%%%%%%%%%%%%%%%%
\DescribeMacro{\ifchilddocmanual}
The main file should be prepared as usual, see \secref{sec:include}.
However, the document body must make a distinction
between processing of an individual part and of the main document, e.g.:
%
\begin{center}
\begin{tabular}{l}
|\ifchilddocmanual|\\
|\input{\childdocname}|\\
|\||else|\\
\textit{document body with }|\input{|\textit{part}|}|\\
|\||fi|
\end{tabular}
\end{center}
%
The conditional |\ifchilddocmanual| is true whenever
a part to be included by |\input| is being compiled,
and the name of the part is stored in |\childdocname|.

%%%%%%%%%%%%%%%%%%%%%%%%%%%%%%%%%%%%%%%%
\DescribeMacro{\childdocby}
Each part to be included by |\input| should start with:
%
\begin{center}
\begin{tabular}{l}
|\input{childdoc.def}|\\
|\childdocby{|\textit{main}|}|\\
\end{tabular}
\end{center}
%
The directive |\childdocby| is similar to |\childdocof|
described in \secref{sec:include},
but the subsequent selection of content must be done manually.
To that end, both |\ifchilddoc| and |\ifchilddocmanual|
will be true upon processing of a part,
and the name of the part is stored in |\childdocname|.
Note that |\jobname| will be set to the filename of the current part
so that each part receives an individual |.aux| file
that does not interfere with the |.aux| file(s) of the main document.
This behaviour can be altered by the alternative form
|\childdocby[*]{|\textit{main}|}| (with a non-empty optional argument)
which uses the |.aux| file of the main document
by setting |\jobname| to \textit{main}.

%%%%%%%%%%%%%%%%%%%%%%%%%%%%%%%%%%%%%%%%%%%%%%%%%%%%%%%%%%%%%%%%%%%%%%%%%%%%%%%%
\subsection{Driver Development}
\label{sec:driver}

The \textsf{childdoc} mechanism can also be use for the development
of definition files such as \LaTeX{} styles or classes.
This case differs from the above setup with multiple parts
included by |\include| in that no |\includeonly| should be invoked.
This can be achieved by starting the include file
(before |\ProvidesPackage|) with:
%
\begin{center}
\begin{tabular}{l}
|\input{childdoc.def}|\\
|\childdocforward{|\textit{main}|}|\\
\end{tabular}
\end{center}
%
or alternatively with:
%
\begin{center}
\begin{tabular}{l}
|\input{childdoc.def}|\\
|\childdocby{|\textit{main}|}|\\
\end{tabular}
\end{center}
%
Both forms have slightly different effects as described above.
The main file is prepared as usual, see \secref{sec:include}.

%%%%%%%%%%%%%%%%%%%%%%%%%%%%%%%%%%%%%%%%%%%%%%%%%%%%%%%%%%%%%%%%%%%%%%%%%%%%%%%%
\subsection{Legacy Detection}
\label{sec:detection}

The directive |\childdocmain| in the main file can detect
whether the complete document or merely a child is to be compiled
even without using the directive |\childdocof|.
This method is deprecated because it is less robust
and there is no compelling reason to use it;
it is merely provided for backward compatibility
and it may be removed in future versions.

If the detection mechanism is to be used,
it is mandatory to correctly specify
the filename of the main file as the argument of |\childdocmain|:
%
\begin{center}
\begin{tabular}{l}
|\input{childdoc.def}|\\
|\childdocmain{|\textit{main}|}|\\
\end{tabular}
\end{center}
%
If |\jobname| does not match the argument \textit{main} of |\childdocmain|,
it is assumed that |\jobname| points to the child file to be compiled.
When using |\childdocmain| with the main file specified as argument,
it suffices to start a child file
with just |\input{|\textit{main}|}|
without loading of the package and using |\childdocof|.
If instead all processing is done
with the appropriate \textsf{childdoc} directives,
the argument of \textit{main} of |\childdocmain| can be empty.

An alternative version of the command line processing described
in \secref{sec:commandline} using the detection mechanism reads:
%
\begin{center}
|... -jobname "|\textit{target}|" "|[\textit{flags}]%
[|\def\jobname{|\textit{dest}|}|]|\input{|\textit{main}|}"|
\end{center}

%%%%%%%%%%%%%%%%%%%%%%%%%%%%%%%%%%%%%%%%%%%%%%%%%%%%%%%%%%%%%%%%%%%%%%%%%%%%%%%%
\subsection{Manual Code}
\label{sec:manual}

In case one cannot be certain whether the definitions file |childdoc.def|
is installed on the target \TeX{} distribution
and one prefers not to ship it,
it is conceivable to paste a few relevant commands into the sources.

To that end, drop all statements |\input{childdoc.def}|
and perform the replacements as outlined below.
Instead of |\childdocmain{|\textit{main}|}| add the following code
to the top of the main file:
%
\begin{center}
\begin{tabular}{l}
|\||ifdefined\childdocname\endinput\||fi\newif\ifchilddoc|\\
|\edef\childdocname{\scantokens\expandafter{\jobname\noexpand}}|\\
|\def\childdocmain{|\textit{main}|}\||ifx\childdocmain\childdocname\||else|\\
|\childdoctrue\includeonly{\childdocname}\let\jobname\childdocmain\||fi|\\
\end{tabular}
\end{center}
%
Instead of |\childdocof{|\textit{main}|}| just include the main file
at the top of each child file:
%
\begin{center}
|\input{|\textit{main}|}|
\end{center}
%
A simple redirection |\childdocforward{|\textit{dest}|}| is achieved by:
%
\begin{center}
|\def\jobname{|\textit{dest}|}\input{\jobname}|
\end{center}
%
The redirection with prefix
|\childdocforwardprefix[|\textit{prefix}|]{|\textit{dest}|}|
is accomplished by:
%
\begin{center}
\begin{tabular}{l}
|{\edef\jobname{\scantokens\expandafter{\jobname\noexpand}}|\\
|\def\redirectjob |\textit{prefix}|#1~~~{\gdef\jobname{|\textit{dest}|#1}}|\\
|\expandafter\redirectjob\jobname~~~}\input{\jobname}|
\end{tabular}
\end{center}

In an alternative approach,
child documents can be compiled by a specific command line
without additional code or specific definitions:
%
\begin{center}
|... -jobname "|\textit{target}|" "|[\textit{flags}]%
|\includeonly{|\textit{dest}|}\input{|\textit{main}|}"|
\end{center}
%

%%%%%%%%%%%%%%%%%%%%%%%%%%%%%%%%%%%%%%%%%%%%%%%%%%%%%%%%%%%%%%%%%%%%%%%%%%%%%%%%
%%%%%%%%%%%%%%%%%%%%%%%%%%%%%%%%%%%%%%%%%%%%%%%%%%%%%%%%%%%%%%%%%%%%%%%%%%%%%%%%
\section{Information}

%%%%%%%%%%%%%%%%%%%%%%%%%%%%%%%%%%%%%%%%%%%%%%%%%%%%%%%%%%%%%%%%%%%%%%%%%%%%%%%%
\subsection{Copyright}

Copyright \copyright{} 2017--2018 Niklas Beisert

This work may be distributed and/or modified under the
conditions of the \LaTeX{} Project Public License, either version 1.3
of this license or (at your option) any later version.
The latest version of this license is in
  \url{http://www.latex-project.org/lppl.txt}
and version 1.3 or later is part of all distributions of \LaTeX{}
version 2005/12/01 or later.

This work has the LPPL maintenance status `maintained'.

The Current Maintainer of this work is Niklas Beisert.

This work consists of the files |README.txt|, |childdoc.ins| and |childdoc.dtx|
as well as the derived files |childdoc.def|, |cdocsamp.tex|
with |cdocsch1.tex|, |cdocsch2.tex|, |cdocspt3.tex|, |cdocspt4.tex|,
|cdocsdrf.tex|, |cdocsfn1.tex|, |cdocsfn2.tex|
as well as |childdoc.pdf|.

%%%%%%%%%%%%%%%%%%%%%%%%%%%%%%%%%%%%%%%%%%%%%%%%%%%%%%%%%%%%%%%%%%%%%%%%%%%%%%%%
\subsection{Files and Installation}

The package consists of the files:
%
\begin{center}
\begin{tabular}{ll}
    |README.txt|   & readme file \\
    |childdoc.ins| & installation file \\
    |childdoc.dtx| & source file \\
    |childdoc.def| & definition file \\
    |cdocsamp.tex| & sample main file \\
    |cdocsch1.tex| & sample include file \\
    |cdocsch2.tex| & sample include file \\
    |cdocspt3.tex| & sample part file \\
    |cdocspt4.tex| & sample part file \\
    |cdocsdrf.tex| & sample redirection file \\
    |cdocsfn1.tex| & sample redirection file \\
    |cdocsfn2.tex| & sample redirection file \\
    |childdoc.pdf| & manual
\end{tabular}
\end{center}
%
The distribution consists of the files
|README.txt|, |childdoc.ins| and |childdoc.dtx|.
%
\begin{itemize}
\item
Run (pdf)\LaTeX{} on |childdoc.dtx|
to compile the manual |childdoc.pdf| (this file).
\item
Run \LaTeX{} on |childdoc.ins| to create the definitions file |childdoc.def|
and the sample |cdocsamp.tex| with include files
|cdocsch1.tex|, |cdocsch2.tex|, |cdocspt3.tex|, |cdocspt4.tex|,
|cdocsdrf.tex|, |cdocsfn1.tex|, |cdocsfn2.tex|.
Then copy the file |childdoc.def| to an appropriate directory of your \LaTeX{}
distribution, e.g.\ \textit{texmf-root}|/tex/latex/childdoc|.
\end{itemize}

%%%%%%%%%%%%%%%%%%%%%%%%%%%%%%%%%%%%%%%%%%%%%%%%%%%%%%%%%%%%%%%%%%%%%%%%%%%%%%%%
\subsection{Related CTAN Packages}

There are several other packages which offer a similar functionality:
%
\begin{itemize}
\item
The packages
\href{http://ctan.org/pkg/docmute}{\textsf{docmute}},
\href{http://ctan.org/pkg/includex}{\textsf{includex}} and
\href{http://ctan.org/pkg/standalone}{\textsf{standalone}}
provide commands to include only the document body of
a child file thus allowing both files to be compiled individually.
\item
The packages \href{http://ctan.org/pkg/subdocs}{\textsf{subdocs}}
and \href{http://ctan.org/pkg/subfiles}{\textsf{subfiles}}
provide structures in which the main and child documents can be
encapsulated and allowing them to be compiled individually.
The inclusion mechanism is different from the conventional |\include|.
\item
The package \href{http://ctan.org/pkg/combine}{\textsf{combine}}
is an elaborate solution to combine several documents into one.
\end{itemize}
%
See also the CTAN topic \href{http://ctan.org/topic/subdocs}{\textsf{subdocs}}
for further related packages.
The present package differs from the above solutions in that
a document structure constructed with the conventional |\include| mechanism
just needs two extra commands at the top of every file
such that all constituent files can be compiled individually.

%%%%%%%%%%%%%%%%%%%%%%%%%%%%%%%%%%%%%%%%%%%%%%%%%%%%%%%%%%%%%%%%%%%%%%%%%%%%%%%%
%\subsection{Feature Suggestions}
%
%The following is a list of features which may be useful for future
%versions of this package:
%%
%\begin{itemize}
%\item
%\ldots
%\end{itemize}

%%%%%%%%%%%%%%%%%%%%%%%%%%%%%%%%%%%%%%%%%%%%%%%%%%%%%%%%%%%%%%%%%%%%%%%%%%%%%%%%
\subsection{Revision History}

%%%%%%%%%%%%%%%%%%%%%%%%%%%%%%%%%%%%%%%%
\paragraph{v2.0:} 2018/12/30

\begin{itemize}
\item
immediate forward processing
\item
added |\childdocby| mechanism
\item
manual restructured
\end{itemize}

%%%%%%%%%%%%%%%%%%%%%%%%%%%%%%%%%%%%%%%%
\paragraph{v1.6:} 2018/01/17

\begin{itemize}
\item
application for development of include files
\item
corrections to manual
\end{itemize}

%%%%%%%%%%%%%%%%%%%%%%%%%%%%%%%%%%%%%%%%
\paragraph{v1.5:} 2017/05/21

\begin{itemize}
\item
more complete structuring introduced
\item
|\childdocof| introduced
\item
|\childdoc| renamed to |\childdocmain|
\item
|\childredirect| renamed to |\childdocforward| and |\childdocforwardprefix|
and functionality expanded
\end{itemize}

%%%%%%%%%%%%%%%%%%%%%%%%%%%%%%%%%%%%%%%%
\paragraph{v1.0:} 2017/04/27

\begin{itemize}
\item
manual and install package
\item
first version published on CTAN
\end{itemize}

%%%%%%%%%%%%%%%%%%%%%%%%%%%%%%%%%%%%%%%%
\paragraph{v0.6:} 2017/04/26

\begin{itemize}
\item
redirection mechanism added
\end{itemize}

%%%%%%%%%%%%%%%%%%%%%%%%%%%%%%%%%%%%%%%%
\paragraph{v0.5:} 2017/04/26

\begin{itemize}
\item
functionality in definition file
\end{itemize}


%%%%%%%%%%%%%%%%%%%%%%%%%%%%%%%%%%%%%%%%%%%%%%%%%%%%%%%%%%%%%%%%%%%%%%%%%%%%%%%%
%%%%%%%%%%%%%%%%%%%%%%%%%%%%%%%%%%%%%%%%%%%%%%%%%%%%%%%%%%%%%%%%%%%%%%%%%%%%%%%%
%%%%%%%%%%%%%%%%%%%%%%%%%%%%%%%%%%%%%%%%%%%%%%%%%%%%%%%%%%%%%%%%%%%%%%%%%%%%%%%%
\appendix

\settowidth\MacroIndent{\rmfamily\scriptsize 000\ }

 \DocInput{childdoc.dtx}

\end{document}
%</driver>
% \fi
%
% %%%%%%%%%%%%%%%%%%%%%%%%%%%%%%%%%%%%%%%%%%%%%%%%%%%%%%%%%%%%%%%%%%%%%%%%%%%%%%
% %%%%%%%%%%%%%%%%%%%%%%%%%%%%%%%%%%%%%%%%%%%%%%%%%%%%%%%%%%%%%%%%%%%%%%%%%%%%%%
% \section{Sample}
%\iffalse
%<*samplemain>
%\fi
%
% The following presents a sample document
% with two chapters, two parts, a title page,
% a compile flag as well as three forwarding files to set the flag.
% It consists of eight |.tex| files:
% \begin{center}
% \begin{tabular}{ll}
% |cdocsamp.tex|&main file\\
% |cdocsch1.tex|&include file for chapter 1\\
% |cdocsch2.tex|&include file for chapter 2\\
% |cdocspt3.tex|&include file for part 3\\
% |cdocspt4.tex|&include file for part 4\\
% |cdocsdrf.tex|&forwarding file for main file in draft mode\\
% |cdocsfi1.tex|&forwarding file for final version of chapter 1\\
% |cdocsfi2.tex|&forwarding file for final version of chapter 2\\
% \end{tabular}
% \end{center}
% Each of the eight files can be compiled directly by the \LaTeX{} compiler.
%
% %%%%%%%%%%%%%%%%%%%%%%%%%%%%%%%%%%%%%%
% \paragraph{Main File.}
%
% The main file is called |cdocsamp.tex|.
%
% Load the \textsf{childdoc} definitions and
% declare the filename for the main document:
%    \begin{macrocode}
\input{childdoc.def}
\childdocmain{}
%    \end{macrocode}

% Optional override for |\version| flag:
%    \begin{macrocode}
%%\ifchilddoc\else\providecommand{\version}{draft}\fi
%    \end{macrocode}

% Define the default values for the |\version| flag
% (|final| for the main file and |draft| for childs):
%    \begin{macrocode}
\ifchilddoc
\providecommand{\version}{draft}
\else
\providecommand{\version}{final}
\fi
%    \end{macrocode}

% Load the standard document class:
%    \begin{macrocode}
\documentclass[12pt]{article}
%    \end{macrocode}

% Start the document body:
%    \begin{macrocode}
\begin{document}
%    \end{macrocode}

% Declare a title page.
% Print title, part of document being processed and version flag:
%    \begin{macrocode}
\addtocounter{page}{-1}
\begin{center}
{\LARGE\bfseries{}childdoc example\par}
\vspace{1cm}
\ifchilddoc
\ifchilddocmanual part\else chapter\fi:
`\childdocname' of `\childdocjob'\par
\else
main document: `\childdocjob'\par
\fi
version: \version\par
\end{center}
\newpage
%    \end{macrocode}

% Manually include selected file,
% otherwise process as usual:
%    \begin{macrocode}
\ifchilddocmanual
\section*{part `\childdocname'}
\input{\childdocname}
\else
%    \end{macrocode}

% Include the two chapters:
%    \begin{macrocode}
\include{cdocsch1}
\include{cdocsch2}
%    \end{macrocode}

% Include the two parts unless only chapters should be displayed:
%    \begin{macrocode}
\ifchilddoc\else
\section{part three}
\input{cdocspt3}
\section{part four}
\input{cdocspt4}
\fi
%    \end{macrocode}

% Process as usual until here:
%    \begin{macrocode}
\fi
%    \end{macrocode}

% End of document body:
%    \begin{macrocode}
\end{document}
%    \end{macrocode}
%\iffalse
%</samplemain>
%\fi
%
% %%%%%%%%%%%%%%%%%%%%%%%%%%%%%%%%%%%%%%
% \paragraph{Chapter Include Files.}
%
% The include files are called |cdocsch1.tex| and |cdocsch2.tex|.
%
%\iffalse
%<*samplechap1|samplechap2>
%\fi

% Optional override for |\version| flag:
%    \begin{macrocode}
%%\providecommand{\version}{final}
%    \end{macrocode}

% Include the main document:
%    \begin{macrocode}
\input{childdoc.def}
\childdocof{cdocsamp}
%    \end{macrocode}

%\iffalse
%</samplechap1|samplechap2>
%\fi
%
%\iffalse
%<*samplechap1>
%\fi
% Some text for chapter 1:
%    \begin{macrocode}
\section{one}
some text in chapter one
%    \end{macrocode}

%\iffalse
%</samplechap1>
%\fi
% Some text for chapter 2:
%\iffalse
%<*samplechap2>
%\fi
%    \begin{macrocode}
\section{two}
more text in chapter two
%    \end{macrocode}

%\iffalse
%</samplechap2>
%\fi
%
% %%%%%%%%%%%%%%%%%%%%%%%%%%%%%%%%%%%%%%
% \paragraph{Part Include Files.}
%
% The include files are called |cdocspt3.tex| and |cdocspt4.tex|.
%
%\iffalse
%<*samplepart3|samplepart4>
%\fi

% Optional override for |\version| flag:
%    \begin{macrocode}
%%\providecommand{\version}{final}
%    \end{macrocode}

% Include the main document:
%    \begin{macrocode}
\input{childdoc.def}
\childdocby{cdocsamp}
%    \end{macrocode}

%\iffalse
%</samplepart3|samplepart4>
%\fi
%
%\iffalse
%<*samplepart3>
%\fi
% Some text for part 3:
%    \begin{macrocode}
some text in part three
%    \end{macrocode}

%\iffalse
%</samplepart3>
%\fi
% Some text for part 4:
%\iffalse
%<*samplepart4>
%\fi
%    \begin{macrocode}
more text in part four
%    \end{macrocode}

%\iffalse
%</samplepart4>
%\fi
%
% %%%%%%%%%%%%%%%%%%%%%%%%%%%%%%%%%%%%%%
% \paragraph{Forwarding for a Complete Draft.}
%
% The following forwarding file |cdocsdrf.tex|
% compiles the main document in draft mode:
%\iffalse
%<*sampledraft>
%\fi
%    \begin{macrocode}
\def\version{draft}
\input{childdoc.def}
\childdocforward{cdocsamp}
%    \end{macrocode}

%\iffalse
%</sampledraft>
%\fi
%
% %%%%%%%%%%%%%%%%%%%%%%%%%%%%%%%%%%%%%%
% \paragraph{Forwarding for Final Version of the Chapters.}
%
% The following forwarding files |cdocsfn1.tex| and |cdocsfn2.tex|
% (with identical content)
% compile the final versions of the child documents
% |cdocsch1.tex| and |cdocsch2.tex|, respectively:
%\iffalse
%<*samplefinal>
%\fi
%    \begin{macrocode}
\def\version{final}
\input{childdoc.def}
\childdocforwardprefix[cdocsamp]{cdocsfn}{cdocsch}
%    \end{macrocode}

%\iffalse
%</samplefinal>
%\fi
%
% %%%%%%%%%%%%%%%%%%%%%%%%%%%%%%%%%%%%%%
% \paragraph{Command Line Processing.}
%
% The following three command lines generate the output files
% |cdocscld|, |cdocscl1| and |cdocscl2|
% which should be identical to
% |cdocsdrf|, |cdocsch1| and |cdocsfn2|, respectively:
% \begin{center}
% \begin{tabular}{l}
% |latex -jobname cdocscld \|\\
% |  "\def\version{draft}\input{childdoc.def}\childdocforward{cdocsamp}"|\\
% |latex -jobname cdocscl1 \|\\
% |  "\input{childdoc.def}\childdocforward[cdocsamp]{cdocsch1}"|\\
% |latex -jobname cdocscl2 \|\\
% |  "\def\version{final}\input{childdoc.def}\childdocforward{cdocsch2}"|
% \end{tabular}
% \end{center}
% Note that the trailing backslash on each first line
% merely continues the input to the second line
% (for convenient cut ant paste).
% Furthermore, the command |latex| can be replaced by any
% of its alternative versions such as |pdflatex|.
%
% %%%%%%%%%%%%%%%%%%%%%%%%%%%%%%%%%%%%%%%%%%%%%%%%%%%%%%%%%%%%%%%%%%%%%%%%%%%%%%
% %%%%%%%%%%%%%%%%%%%%%%%%%%%%%%%%%%%%%%%%%%%%%%%%%%%%%%%%%%%%%%%%%%%%%%%%%%%%%%
% \section{Implementation}
%\iffalse
%<*package>
%\fi
%
% This section describes the definitions file |childdoc.def|.

% The definitions cannot be loaded using |\usepackage| or |\RequirePackage|
% which has a mechanism to prevent loading a style file more than once.
% When loading the definitions by means of |\input|
% multiple instances have to be prevented manually:
%\iffalse
%This code needs to be before the `\ProvidesFile' directive
%which is defined at the beginning of this file.
%Therefore it is also placed there and commented out here.
%</package>
%<*discard>
%\fi
%    \begin{macrocode}
\ifdefined\childdocmain\endinput\fi
%    \end{macrocode}
%\iffalse
%</discard>
%<*package>
%\fi
%
% \macro{\ifchilddoc}
% \macro{\ifchilddocmanual}
% The conditional |\ifchilddoc| tells whether a
% child (true) or main (false) document is being compiled.
% The conditional |\ifchilddocmanual| tells whether
% the |\includeonly| mechanism is used (false) or
% the selection of child files must be performed manually (true).
% The definitions initialise to false:
%    \begin{macrocode}
\newif\ifchilddoc
\newif\ifchilddocmanual
%    \end{macrocode}

% \macro{\childdocname}
% \macro{\childdocjob}
% The macro |\childdocname| stores the name of the main document
% to be compiled. The macro |\childdocjob| stores the name of
% the document on which the \LaTeX{} compiler was originally invoked.
% The content of |\jobname| cannot be compared
% to filenames specified in the source due to different catcodes.
% The following code rescans |\jobname|, stores the result
% in |\childdocname| and saves a copy in |\childdocjob|:
%    \begin{macrocode}
\edef\childdocname{\scantokens\expandafter{\jobname\noexpand}}
\let\childdocjob\childdocname
%    \end{macrocode}

% \macro{\childdocdisable}
% The macro |\childdocdisable| prevents the main file
% from being processed more than once.
% At this stage, the main document command |\childdocmain|
% is assumed to be called once again where it should do nothing.
% Any subsequent call to it should prevent
% a secondary processing of the main document
% It overwrites the forwarding commands
% |\childdocof| and |\childdocforward|
% with empty macros to prevent further inclusions of the main document:
%    \begin{macrocode}
\newcommand{\childdocdisable}
{
  \renewcommand{\childdocmain}[1]{\renewcommand{\childdocmain}[1]{\endinput}}
  \renewcommand{\childdocof}[1]{}
  \renewcommand{\childdocby}[2][]{}
  \renewcommand{\childdocforward}[2][]{}
  \renewcommand{\childdocdisable}{}
}
%    \end{macrocode}

% \macro{\childdocmain}
% The macro |\childdocmain| is to be called at the top of the main file
% with nothing or the main filename (without extension) as argument.
% First, it breaks loops.
% If the argument is not empty and does not match |\childdocname|
% (which is set by the first inclusion of |childdoc.def|),
% |\ifchilddoc| is set to true, |\includeonly| is applied to the child file
% and |\jobname| is set to the main file
% (for proper handling of |.aux| files):
%    \begin{macrocode}
\newcommand{\childdocmain}[1]
{
  \childdocdisable\childdocmain{}
  \if?#1?\else
    \begingroup
      \def\childdoctmp{#1}
      \ifx\childdoctmp\childdocname
        \def\childdoctmp{}
      \else
        \def\childdoctmp
        {
          \childdoctrue
          \includeonly{\childdocname}
          \def\childdocjob{#1}
          \def\jobname{#1}
        }
      \fi
      \expandafter
    \endgroup
    \childdoctmp
  \fi
}
%    \end{macrocode}

% \macro{\childdocof}
% The command |\childdocof| redirects
% compilation to the main file |#1|.
%    \begin{macrocode}
\newcommand{\childdocof}[1]
{
  \childdocdisable
  \childdoctrue
  \includeonly{\childdocname}
  \def\jobname{#1}
  \def\childdocjob{#1}
  \input{#1}
}
%    \end{macrocode}

% \macro{\childdocby}
% The command |\childdocby| ....
%    \begin{macrocode}
\newcommand{\childdocby}[2][]
{
  \childdocdisable
  \childdoctrue
  \childdocmanualtrue
  \if?#1?\else
    \def\jobname{#2}
  \fi
  \def\childdocjob{#2}
  \input{#2}
  \endinput
}
%    \end{macrocode}

% \macro{\childdocforward}
% The command |\childdocforward| redirects
% compilation to the main file or
% (if the optional argument is given) a child file.
% Parameters are set as if the main file
% or a child file starting with |\childdocof| was compiled.
% Then compilation is handed over to the main file:
%    \begin{macrocode}
\newcommand{\childdocforward}[2][]
{
  \begingroup
    \if?#1?
      \def\childdoctmp
      {
        \def\childdocname{#2}
        \def\childdocjob{#2}
        \def\jobname{#2}
        \input{#2}
        \endinput
      }
    \else
      \def\childdoctmp
      {
        \childdocdisable
        \def\childdocname{#2}
        \childdoctrue
        \includeonly{#2}
        \def\childdocjob{#1}
        \def\jobname{#1}
        \input{#1}
        \endinput
      }
    \fi
    \expandafter
  \endgroup
  \childdoctmp
}
%    \end{macrocode}

% \macro{\childdocforwardprefix}
% The command |\childdocforwardprefix| redirects
% compilation to the main or a child file by means of a pattern.
% The prefix |#1| in the current filename is replaced by |#2|
% and the suffix of the current filename is kept
% (it is assumed that the filename does not contain the substring `|~~~|'
% which is used as a delimiter).
% Compilation is handed over to the new file by |\childdocforward|:
%    \begin{macrocode}
\newcommand{\childdocforwardprefix}[3][]
{
  \begingroup
    \def\childdocextract #2##1~~~{\def\childdoctmp{\childdocforward[#1]{#3##1}}}
    \expandafter\childdocextract\childdocname~~~
    \expandafter
  \endgroup
  \childdoctmp
}
%    \end{macrocode}

% \macro{\childdoc}
% The deprecated macro |\childdoc| is a legacy version of |\childdocmain|:
%    \begin{macrocode}
\newcommand{\childdoc}{\childdocmain}
%    \end{macrocode}

% \macro{\childdocredirect}
% The deprecated macro |\childdocredirect| is a legacy version
% of |\childdocforward| and |\childdocforwardprefix|:
%    \begin{macrocode}
\newcommand{\childdocredirect}[2][]
{
  \begingroup
    \if?#1?
      \def\childdoctmp{\childdocforward{#2}}
    \else
      \def\childdoctmp{\childdocforwardprefix{#1}{#2}}
    \fi
    \expandafter
  \endgroup
  \childdoctmp
}
%    \end{macrocode}

%\iffalse
%</package>
%\fi
%
\endinput

\childdocof{cdocsamp}
%    \end{macrocode}

%\iffalse
%</samplechap1|samplechap2>
%\fi
%
%\iffalse
%<*samplechap1>
%\fi
% Some text for chapter 1:
%    \begin{macrocode}
\section{one}
some text in chapter one
%    \end{macrocode}

%\iffalse
%</samplechap1>
%\fi
% Some text for chapter 2:
%\iffalse
%<*samplechap2>
%\fi
%    \begin{macrocode}
\section{two}
more text in chapter two
%    \end{macrocode}

%\iffalse
%</samplechap2>
%\fi
%
% %%%%%%%%%%%%%%%%%%%%%%%%%%%%%%%%%%%%%%
% \paragraph{Part Include Files.}
%
% The include files are called |cdocspt3.tex| and |cdocspt4.tex|.
%
%\iffalse
%<*samplepart3|samplepart4>
%\fi

% Optional override for |\version| flag:
%    \begin{macrocode}
%%\providecommand{\version}{final}
%    \end{macrocode}

% Include the main document:
%    \begin{macrocode}
% \iffalse
%
% childdoc.dtx Copyright (C) 2017-2018 Niklas Beisert
%
% This work may be distributed and/or modified under the
% conditions of the LaTeX Project Public License, either version 1.3
% of this license or (at your option) any later version.
% The latest version of this license is in
%   http://www.latex-project.org/lppl.txt
% and version 1.3 or later is part of all distributions of LaTeX
% version 2005/12/01 or later.
%
% This work has the LPPL maintenance status `maintained'.
%
% The Current Maintainer of this work is Niklas Beisert.
%
% This work consists of the files childdoc.dtx and childdoc.ins
% and the derived files childdoc.def and cdocsamp.tex with
% cdocsch1.tex, cdocsch2.tex, cdocsdrf.tex, cdocsfn1.tex, cdocsfn2.tex.
%
%<package>\ifdefined\childdocmain\endinput\fi
%<package>\ProvidesFile{childdoc.def}[2018/12/30 v2.0 child document driver]
%<samplemain>\ProvidesFile{cdocsamp.tex}[2018/12/30 v2.0 sample for childdoc]
%<*driver>
%\ProvidesFile{childdoc.drv}[2018/12/30 v2.0 childdoc reference manual file]
\PassOptionsToClass{10pt,a4paper}{article}
\documentclass{ltxdoc}

\usepackage[margin=35mm]{geometry}
\usepackage{hyperref}
\usepackage{hyperxmp}
\usepackage[usenames]{color}

\hypersetup{colorlinks=true}
\hypersetup{pdfstartview=FitH}
\hypersetup{pdfpagemode=UseNone}
\hypersetup{pdfsource={}}
\hypersetup{pdflang={en-UK}}
\hypersetup{pdfcopyright={Copyright 2017-2018 Niklas Beisert.
  This work may be distributed and/or modified under the
  conditions of the LaTeX Project Public License, either version 1.3
  of this license or (at your option) any later version.}}
\hypersetup{pdflicenseurl={http://www.latex-project.org/lppl.txt}}
\hypersetup{pdfcontactaddress={ETH Zurich, ITP, HIT K,
  Wolfgang-Pauli-Strasse 27}}
\hypersetup{pdfcontactpostcode={8093}}
\hypersetup{pdfcontactcity={Zurich}}
\hypersetup{pdfcontactcountry={Switzerland}}
\hypersetup{pdfcontactemail={nbeisert@itp.phys.ethz.ch}}
\hypersetup{pdfcontacturl={http://people.phys.ethz.ch/\xmptilde nbeisert/}}

\newcommand{\secref}[1]{\hyperref[#1]{section \ref*{#1}}}

\parskip1ex
\parindent0pt
\let\olditemize\itemize
\def\itemize{\olditemize\parskip0pt}

\begin{document}

\title{The \textsf{childdoc} Package}
\hypersetup{pdftitle={The childdoc Package}}
\author{Niklas Beisert\\[2ex]
  Institut f\"ur Theoretische Physik\\
  Eidgen\"ossische Technische Hochschule Z\"urich\\
  Wolfgang-Pauli-Strasse 27, 8093 Z\"urich, Switzerland\\[1ex]
  \href{mailto:nbeisert@itp.phys.ethz.ch}
  {\texttt{nbeisert@itp.phys.ethz.ch}}}
\hypersetup{pdfauthor={Niklas Beisert}}
\hypersetup{pdfsubject={Manual for the LaTeX2e Package childdoc}}
\date{30 December 2018, \textsf{v2.0}}
\maketitle

\begin{abstract}\noindent
\textsf{childdoc} is a \LaTeXe{} package
that enables the direct compilation
of document sections included by |\include|
to individual files.
\end{abstract}

\begingroup
\parskip0ex
\tableofcontents
\endgroup

%%%%%%%%%%%%%%%%%%%%%%%%%%%%%%%%%%%%%%%%%%%%%%%%%%%%%%%%%%%%%%%%%%%%%%%%%%%%%%%%
%%%%%%%%%%%%%%%%%%%%%%%%%%%%%%%%%%%%%%%%%%%%%%%%%%%%%%%%%%%%%%%%%%%%%%%%%%%%%%%%
\section{Introduction}

\LaTeX{} provides a mechanism to structure a large document (such as a book)
into a main file and several child files (containing the chapters)
using the |\include| command.
This mechanism is beneficial for documents
which span hundreds of pages in order to
make the source file(s) more manageable.
Moreover, compilation can be restricted to
selected child files by means of the |\includeonly| command.
The latter feature can be used to reduce the compilation time while editing
(this was significantly more useful in the earlier days of \LaTeX{})
or to generate a smaller document which is easier to navigate.
Another application of |\includeonly| is to generate
documents consisting of selected parts of the complete document.

However, there are a few drawbacks of the plain |\include| mechanism:
\begin{itemize}
\item
The child files cannot be compiled on their own,
they can only be compiled via the main file.
A naive editing environment
(such as a text editor with an option
to have the current file processed by \LaTeX)
may require one to switch to the main file before compiling;
attempting to compile the child file produces errors.
\item
The main file must be modified (each time)
to adjust the |\includeonly| command
to the present needs. This easily leaves the main file in a messy state.
\item
The generated document will always carry the filename
of the main document. This is inconvenient if
several child files are to be compiled and
to be kept for distribution.
\end{itemize}

The present package provides a simple interface
to make child files individually compilable by \LaTeX{}.
Compiling a child file then has the same effect as compiling
the main file with an |\includeonly| command
to select the appropriate child.
Moreover the generated document will carry the name of the child
rather than the main file.
This resolves all three above issues.

This feature is meant to make the editing of books,
thesis documents and lecture notes somewhat more convenient.
However, the package can also be used efficiently for
composing a series of documents (such as exercise sheets)
which are typically distributed individually.
It then assists the author in generating the individual documents
(potentially in different versions)
as well as a document containing the collected series.
Another application is in developing style files
or other kinds of included material
where compilation of the style file could redirect
to a sample or test file.

%%%%%%%%%%%%%%%%%%%%%%%%%%%%%%%%%%%%%%%%%%%%%%%%%%%%%%%%%%%%%%%%%%%%%%%%%%%%%%%%
%%%%%%%%%%%%%%%%%%%%%%%%%%%%%%%%%%%%%%%%%%%%%%%%%%%%%%%%%%%%%%%%%%%%%%%%%%%%%%%%
\section{Usage}

First of all, the package \textsf{childdoc} is \emph{not} a standard
\LaTeXe{} |.sty| style file! Therefore it needs to be invoked in
a non-standard way.

%%%%%%%%%%%%%%%%%%%%%%%%%%%%%%%%%%%%%%%%%%%%%%%%%%%%%%%%%%%%%%%%%%%%%%%%%%%%%%%%
\subsection{Included Files}
\label{sec:include}

%%%%%%%%%%%%%%%%%%%%%%%%%%%%%%%%%%%%%%%%
\DescribeMacro{\childdocmain}
To use the package, add the commands
\begin{center}
\begin{tabular}{l}
|\input{childdoc.def}|\\
|\childdocmain{}|\\
\end{tabular}
\end{center}
at the very top of the main \LaTeX{} file,
in particular \emph{before} the |\documentclass| statement!
The argument of |\childdocmain| should be left empty
(but it must be present).

%%%%%%%%%%%%%%%%%%%%%%%%%%%%%%%%%%%%%%%%
\DescribeMacro{\childdocof}
Furthermore, add the commands
\begin{center}
\begin{tabular}{l}
|\input{childdoc.def}|\\
|\childdocof{|\textit{main}|}|\\
\end{tabular}
\end{center}
at the top of every child file \textit{child}
which is included by |\include{|\textit{child}|}|
from within the main file
(or at least for those files to be compiled individually).
The argument \textit{main} must be the filename of the main file.

There are a couple of
considerations in setting up the main and child documents:

%%%%%%%%%%%%%%%%%%%%%%%%%%%%%%%%%%%%%%%%
\paragraph{Restrictions.}

Please note the following restrictions:
\begin{itemize}
\item
|\childdocmain| must be called with one argument \textit{main}
to ensure compatibility with earlier version of the package.
It must either be empty (|\childdocmain{}|)
or precisely match the filename of the main file in which it is specified.
See \secref{sec:detection} for further information.
\item
The filename \textit{main} must be specified without the |.tex| extension.
\item
The filename \textit{main} is case sensitive
(even in case-insensitive file systems)
due to internal string comparison.
\item
The argument \textit{main} should be fully expanded, it cannot be a macro.
\item
Subdirectories and special characters should be avoided in filenames.
\item
The command |\childdocmain{|\textit{main}|}| must be followed by a whitespace.
It should not be followed immediately by another command
or by a comment mark `|%|'.
This is because the \TeX{} parser reads the token immediately following
the argument of |\childdocmain| and puts it
at the beginning of every child section;
however, a white\-space is ignored.
\end{itemize}

%%%%%%%%%%%%%%%%%%%%%%%%%%%%%%%%%%%%%%%%
\paragraph{Content of Main File.}

It is advisable to place all content in the child files included by |\include|.
Any output contained in the main file will appear in all child documents
unless suppressed manually;
it cannot be suppressed automatically by the |\includeonly| directive
and thus should normally be avoided.
A method to include some content in the main file
by means of conditional processing is described in \secref{sec:conditional}.

%%%%%%%%%%%%%%%%%%%%%%%%%%%%%%%%%%%%%%%%
\paragraph{Page Numbering.}

When only a part of the document is compiled,
the appropriate numbering of pages
(as well as other status parameters)
is determined from the |.aux| files.
The latter contain information from previous passes.
However this information needs to propagate through
all intermediate child documents.
Therefore the page numbering in child documents may well
be inconsistent until the complete document is compiled at least once.

A useful (if unconventional) way to always ensure a consistent
page numbering is to restart the numbering in each child document
and denote the pages by `\textit{child}|.|\textit{page}'
where \textit{child} represents the chapter/section number of the child file.
This can be achieved by the command
|\numberwithin{page}{|\textit{child}|}|
of the \textsf{amsmath} package
where \textit{child} can be |chapter| or |section|
depending on the chosen structuring.
Alternatively, one can modify the macro |\thepage| appropriately
and reset the counter |page| at the start of each child file.

%%%%%%%%%%%%%%%%%%%%%%%%%%%%%%%%%%%%%%%%%%%%%%%%%%%%%%%%%%%%%%%%%%%%%%%%%%%%%%%%
\subsection{Conditional Processing}
\label{sec:conditional}

The package provides a mechanism to compile different versions
of a document. To customise the versions further some conditional processing
can come in handy to distinguish which version is being compiled.
The package provides two macros to describe the compilation context:

%%%%%%%%%%%%%%%%%%%%%%%%%%%%%%%%%%%%%%%%
\DescribeMacro{\ifchilddoc}
The conditional |\ifchilddoc| distinguishes between the compilation of
child documents and the main document:
%
\begin{center}
|\ifchilddoc |\textit{child-code}| |[|\||else |\textit{main-code}]| \||fi|
\end{center}

%%%%%%%%%%%%%%%%%%%%%%%%%%%%%%%%%%%%%%%%
\DescribeMacro{\childdocname}
\DescribeMacro{\childdocjob}
The macro |\childdocname| contains the filename (without extension)
of the main or child file being processed.
Note that |\childdocjob| will always contain the name of the main file.

%%%%%%%%%%%%%%%%%%%%%%%%%%%%%%%%%%%%%%%%
\paragraph{Title Page.}

Conditional processing can be used to include a title or banner page
in the main document when proper precautions are taken.
Importantly, the code in the main file should ensure that the page counter
(as well as other status parameters which are stored in the |.aux| files)
takes the same value after the conditional processing.
Otherwise the page numbers may take divergent values
depending on which part is compiled.

For example, a title page could be declared by:
%
\begin{center}
\begin{tabular}{l}
|\ifchilddoc\||else|\\
|\addtocounter{page}{-1}|\\
\textit{code for title page}\\
|\newpage|\\
|\||fi|
\end{tabular}
\end{center}
%
A banner page for the child documents can be generated by:
%
\begin{center}
\begin{tabular}{l}
|\ifchilddoc|\\
|\addtocounter{page}{-1}|\\
\textit{code for banner page}\\
|\newpage|\\
|\||fi|
\end{tabular}
\end{center}
%
Here one could write a message such as:
\begin{center}
|This is the part \childdocname{} of \childdocjob{}.|
\end{center}

%%%%%%%%%%%%%%%%%%%%%%%%%%%%%%%%%%%%%%%%%%%%%%%%%%%%%%%%%%%%%%%%%%%%%%%%%%%%%%%%
\subsection{Flags}
\label{sec:flags}

The package makes it easy to generate different versions
of the main or child documents.
To this end compilation flags can be defined
and assigned different default values.
They will be particularly useful in conjunction
with the forwarding mechanism described in \secref{sec:forward}.

For example, it may be useful to have a flag |\version|
which can be set to |draft| or |final|.
The document source will contain some conditional code
depending on the value of |\version|.
Suppose further, the flag should default to |final| for the main file
and to |draft| for child files
which is a natural assignment for editing the document.
This is achieved by placing the following code
in the preamble of the main document
(below the |\childdocmain| directive):
%
\begin{center}
\begin{tabular}{l}
|\ifchilddoc|\\
|\providecommand{\version}{draft}|\\
|\||else|\\
|\providecommand{\version}{final}|\\
|\||fi|
\end{tabular}
\end{center}
%
The definition by |\providecommand| makes sure
that previous definitions are not overwritten.
Further statements |\providecommand{\version}{...}|
can thus be added before the above code to override it.

For the main file, one might add a line
(between |\childdocmain| and the above block)
%
\begin{center}
|%\ifchilddoc\||else\providecommand{\version}{draft}\||fi|
\end{center}
%
which can be uncommented to produce a draft version.
Likewise one can add a line to the very top of a child file
(above the |\childdocof{|\textit{main}|}| directive)
%
\begin{center}
|%\providecommand{\version}{final}|
\end{center}
%
which can be uncommented to produce the final version of this child document.

%%%%%%%%%%%%%%%%%%%%%%%%%%%%%%%%%%%%%%%%%%%%%%%%%%%%%%%%%%%%%%%%%%%%%%%%%%%%%%%%
\subsection{Forwarding}
\label{sec:forward}

Different versions of the main or child documents
using compilation flags as described in \secref{sec:flags}
can be (permanently) stored in different files
for convenient compilation, viewing and distribution.
To this end, the package defines a command
to pass on compilation to a different file:

%%%%%%%%%%%%%%%%%%%%%%%%%%%%%%%%%%%%%%%%
\DescribeMacro{\childdocforward}
The command |\childdocforward| redirects processing to
another source file:
%
\begin{center}
\begin{tabular}{l}
|\input{childdoc.def}|\\
|\childdocforward[|\textit{main}|]{|\textit{dest}|}|\\
\end{tabular}
\end{center}
%
The argument \textit{dest} is the destination file
(without extension).
It should be the main file or one of the child files.
Note that further \textsf{childdoc} directives
such as |\childdocof| and |\childdocforward|
in the indicated file will be processed in this form.
The optional argument \textit{main}
passes on directly to the main file \textit{main}
while pretending to compile the child \textit{dest}.
This form behaves as if \textit{dest}
issues |\childdocof{|\textit{main}|}| right away,
and no further \textsf{childdoc} directives will be processed.

%%%%%%%%%%%%%%%%%%%%%%%%%%%%%%%%%%%%%%%%
\DescribeMacro{\...prefix}
In the alternative form |\childdocforwardprefix|,
%
\begin{center}
\begin{tabular}{l}
|\input{childdoc.def}|\\
|\childdocforwardprefix[|\textit{main}|]{|\textit{prefix}|}{|\textit{dest}|}|
\end{tabular}
\end{center}
%
the destination file is determined by a pattern
depending on the current file:
To make this work, the current file must be called
`{\textit{prefix}\hspace{0.2em}\textit{suffix}}'
with \textit{prefix} matching precisely the argument.
Processing is then passed on to the file
`{\textit{dest}\hspace{0.2em}\textit{suffix}}'.
Surely, the same effect is achieved by
directly specifying the
argument `{\textit{dest}\hspace{0.2em}\textit{suffix}}'
in the first form.
However, that requires to set up a different file
for each child. With the alternative form of the command
all these files can have exactly the same content
which simplifies setting them up and maintaining them.

For example, the following file |draft.tex|
with a compilation flag |\version| as described in \secref{sec:flags}
compiles the main document as a draft:
%
\begin{center}
\begin{tabular}{l}
|\def\version{draft}|\\
|\input{childdoc.def}|\\
|\childdocforward{|\textit{main}|}|
\end{tabular}
\end{center}
%
Likewise, the following files |final|\textit{nn}|.tex|
compile the final version of the child document
|child|\textit{nn}|.tex|:
%
\begin{center}
\begin{tabular}{l}
|\def\version{final}|\\
|\input{childdoc.def}|\\
|\childdocforwardprefix{final}{child}|
\end{tabular}
\end{center}
%

Note that when several versions of a main file and/or of each child file
are to be generated, it may be convenient to set up a |Makefile| or
shell script to automatise the process.

%%%%%%%%%%%%%%%%%%%%%%%%%%%%%%%%%%%%%%%%%%%%%%%%%%%%%%%%%%%%%%%%%%%%%%%%%%%%%%%%
\subsection{Command Line Processing}
\label{sec:commandline}

The effect of redirection files can also be achieved by invoking
the \LaTeX{} compiler with a more elaborate command line.
Most conveniently this should be done as part
of a shell script or a |Makefile|.

When using \textsf{childdoc} in the main file, the following
command lines effectively perform a redirection
(note that depending on the shell being used,
backslashes may have to be doubled: `|\|' $\to$ `|\\|'):
%
\begin{center}
|... -jobname "|\textit{target}|" |\\|"|[\textit{flags}]%
|\input{childdoc.def}\childdocforward[|\textit{main}|]{|\textit{dest}|}"|
\end{center}
%
Here \textit{target} is the name of the output file,
\textit{main} is the name of the main file
and \textit{dest} is the name of the main or child file to be processed
(all filenames without extensions).
The optional argument \textit{main} can be omitted
if \textit{main} matches \textit{dest}.
Optionally, compilation \textit{flags} can be defined via |\def| commands.
This command line makes the \TeX{} engine believe
it is compiling the file \textit{target}
whose content is specified as the latter parameter.
The provided code then forwards the processing to
\textit{main} or \textit{dest} as described in \secref{sec:forward}.

%%%%%%%%%%%%%%%%%%%%%%%%%%%%%%%%%%%%%%%%%%%%%%%%%%%%%%%%%%%%%%%%%%%%%%%%%%%%%%%%
\subsection{Include by Input}
\label{sec:input}

Including child documents by |\include| has some restrictions by design.
Most notably, the content of a child document always occupies
its own set of pages; pages cannot be shared between child documents.
Usually, this behaviour makes perfect sense
because each child document contain an essential part of the document.
However, in some situations it may be desirable to compose
a document from a collection of parts
without having mandatory page breaks between then.
For this case, the package
provides a mechanism to include parts
by |\input| which can also be processed individually.
However, by construction this mechanism
requires manual handling of the content to be output.

%%%%%%%%%%%%%%%%%%%%%%%%%%%%%%%%%%%%%%%%
\DescribeMacro{\ifchilddocmanual}
The main file should be prepared as usual, see \secref{sec:include}.
However, the document body must make a distinction
between processing of an individual part and of the main document, e.g.:
%
\begin{center}
\begin{tabular}{l}
|\ifchilddocmanual|\\
|\input{\childdocname}|\\
|\||else|\\
\textit{document body with }|\input{|\textit{part}|}|\\
|\||fi|
\end{tabular}
\end{center}
%
The conditional |\ifchilddocmanual| is true whenever
a part to be included by |\input| is being compiled,
and the name of the part is stored in |\childdocname|.

%%%%%%%%%%%%%%%%%%%%%%%%%%%%%%%%%%%%%%%%
\DescribeMacro{\childdocby}
Each part to be included by |\input| should start with:
%
\begin{center}
\begin{tabular}{l}
|\input{childdoc.def}|\\
|\childdocby{|\textit{main}|}|\\
\end{tabular}
\end{center}
%
The directive |\childdocby| is similar to |\childdocof|
described in \secref{sec:include},
but the subsequent selection of content must be done manually.
To that end, both |\ifchilddoc| and |\ifchilddocmanual|
will be true upon processing of a part,
and the name of the part is stored in |\childdocname|.
Note that |\jobname| will be set to the filename of the current part
so that each part receives an individual |.aux| file
that does not interfere with the |.aux| file(s) of the main document.
This behaviour can be altered by the alternative form
|\childdocby[*]{|\textit{main}|}| (with a non-empty optional argument)
which uses the |.aux| file of the main document
by setting |\jobname| to \textit{main}.

%%%%%%%%%%%%%%%%%%%%%%%%%%%%%%%%%%%%%%%%%%%%%%%%%%%%%%%%%%%%%%%%%%%%%%%%%%%%%%%%
\subsection{Driver Development}
\label{sec:driver}

The \textsf{childdoc} mechanism can also be use for the development
of definition files such as \LaTeX{} styles or classes.
This case differs from the above setup with multiple parts
included by |\include| in that no |\includeonly| should be invoked.
This can be achieved by starting the include file
(before |\ProvidesPackage|) with:
%
\begin{center}
\begin{tabular}{l}
|\input{childdoc.def}|\\
|\childdocforward{|\textit{main}|}|\\
\end{tabular}
\end{center}
%
or alternatively with:
%
\begin{center}
\begin{tabular}{l}
|\input{childdoc.def}|\\
|\childdocby{|\textit{main}|}|\\
\end{tabular}
\end{center}
%
Both forms have slightly different effects as described above.
The main file is prepared as usual, see \secref{sec:include}.

%%%%%%%%%%%%%%%%%%%%%%%%%%%%%%%%%%%%%%%%%%%%%%%%%%%%%%%%%%%%%%%%%%%%%%%%%%%%%%%%
\subsection{Legacy Detection}
\label{sec:detection}

The directive |\childdocmain| in the main file can detect
whether the complete document or merely a child is to be compiled
even without using the directive |\childdocof|.
This method is deprecated because it is less robust
and there is no compelling reason to use it;
it is merely provided for backward compatibility
and it may be removed in future versions.

If the detection mechanism is to be used,
it is mandatory to correctly specify
the filename of the main file as the argument of |\childdocmain|:
%
\begin{center}
\begin{tabular}{l}
|\input{childdoc.def}|\\
|\childdocmain{|\textit{main}|}|\\
\end{tabular}
\end{center}
%
If |\jobname| does not match the argument \textit{main} of |\childdocmain|,
it is assumed that |\jobname| points to the child file to be compiled.
When using |\childdocmain| with the main file specified as argument,
it suffices to start a child file
with just |\input{|\textit{main}|}|
without loading of the package and using |\childdocof|.
If instead all processing is done
with the appropriate \textsf{childdoc} directives,
the argument of \textit{main} of |\childdocmain| can be empty.

An alternative version of the command line processing described
in \secref{sec:commandline} using the detection mechanism reads:
%
\begin{center}
|... -jobname "|\textit{target}|" "|[\textit{flags}]%
[|\def\jobname{|\textit{dest}|}|]|\input{|\textit{main}|}"|
\end{center}

%%%%%%%%%%%%%%%%%%%%%%%%%%%%%%%%%%%%%%%%%%%%%%%%%%%%%%%%%%%%%%%%%%%%%%%%%%%%%%%%
\subsection{Manual Code}
\label{sec:manual}

In case one cannot be certain whether the definitions file |childdoc.def|
is installed on the target \TeX{} distribution
and one prefers not to ship it,
it is conceivable to paste a few relevant commands into the sources.

To that end, drop all statements |\input{childdoc.def}|
and perform the replacements as outlined below.
Instead of |\childdocmain{|\textit{main}|}| add the following code
to the top of the main file:
%
\begin{center}
\begin{tabular}{l}
|\||ifdefined\childdocname\endinput\||fi\newif\ifchilddoc|\\
|\edef\childdocname{\scantokens\expandafter{\jobname\noexpand}}|\\
|\def\childdocmain{|\textit{main}|}\||ifx\childdocmain\childdocname\||else|\\
|\childdoctrue\includeonly{\childdocname}\let\jobname\childdocmain\||fi|\\
\end{tabular}
\end{center}
%
Instead of |\childdocof{|\textit{main}|}| just include the main file
at the top of each child file:
%
\begin{center}
|\input{|\textit{main}|}|
\end{center}
%
A simple redirection |\childdocforward{|\textit{dest}|}| is achieved by:
%
\begin{center}
|\def\jobname{|\textit{dest}|}\input{\jobname}|
\end{center}
%
The redirection with prefix
|\childdocforwardprefix[|\textit{prefix}|]{|\textit{dest}|}|
is accomplished by:
%
\begin{center}
\begin{tabular}{l}
|{\edef\jobname{\scantokens\expandafter{\jobname\noexpand}}|\\
|\def\redirectjob |\textit{prefix}|#1~~~{\gdef\jobname{|\textit{dest}|#1}}|\\
|\expandafter\redirectjob\jobname~~~}\input{\jobname}|
\end{tabular}
\end{center}

In an alternative approach,
child documents can be compiled by a specific command line
without additional code or specific definitions:
%
\begin{center}
|... -jobname "|\textit{target}|" "|[\textit{flags}]%
|\includeonly{|\textit{dest}|}\input{|\textit{main}|}"|
\end{center}
%

%%%%%%%%%%%%%%%%%%%%%%%%%%%%%%%%%%%%%%%%%%%%%%%%%%%%%%%%%%%%%%%%%%%%%%%%%%%%%%%%
%%%%%%%%%%%%%%%%%%%%%%%%%%%%%%%%%%%%%%%%%%%%%%%%%%%%%%%%%%%%%%%%%%%%%%%%%%%%%%%%
\section{Information}

%%%%%%%%%%%%%%%%%%%%%%%%%%%%%%%%%%%%%%%%%%%%%%%%%%%%%%%%%%%%%%%%%%%%%%%%%%%%%%%%
\subsection{Copyright}

Copyright \copyright{} 2017--2018 Niklas Beisert

This work may be distributed and/or modified under the
conditions of the \LaTeX{} Project Public License, either version 1.3
of this license or (at your option) any later version.
The latest version of this license is in
  \url{http://www.latex-project.org/lppl.txt}
and version 1.3 or later is part of all distributions of \LaTeX{}
version 2005/12/01 or later.

This work has the LPPL maintenance status `maintained'.

The Current Maintainer of this work is Niklas Beisert.

This work consists of the files |README.txt|, |childdoc.ins| and |childdoc.dtx|
as well as the derived files |childdoc.def|, |cdocsamp.tex|
with |cdocsch1.tex|, |cdocsch2.tex|, |cdocspt3.tex|, |cdocspt4.tex|,
|cdocsdrf.tex|, |cdocsfn1.tex|, |cdocsfn2.tex|
as well as |childdoc.pdf|.

%%%%%%%%%%%%%%%%%%%%%%%%%%%%%%%%%%%%%%%%%%%%%%%%%%%%%%%%%%%%%%%%%%%%%%%%%%%%%%%%
\subsection{Files and Installation}

The package consists of the files:
%
\begin{center}
\begin{tabular}{ll}
    |README.txt|   & readme file \\
    |childdoc.ins| & installation file \\
    |childdoc.dtx| & source file \\
    |childdoc.def| & definition file \\
    |cdocsamp.tex| & sample main file \\
    |cdocsch1.tex| & sample include file \\
    |cdocsch2.tex| & sample include file \\
    |cdocspt3.tex| & sample part file \\
    |cdocspt4.tex| & sample part file \\
    |cdocsdrf.tex| & sample redirection file \\
    |cdocsfn1.tex| & sample redirection file \\
    |cdocsfn2.tex| & sample redirection file \\
    |childdoc.pdf| & manual
\end{tabular}
\end{center}
%
The distribution consists of the files
|README.txt|, |childdoc.ins| and |childdoc.dtx|.
%
\begin{itemize}
\item
Run (pdf)\LaTeX{} on |childdoc.dtx|
to compile the manual |childdoc.pdf| (this file).
\item
Run \LaTeX{} on |childdoc.ins| to create the definitions file |childdoc.def|
and the sample |cdocsamp.tex| with include files
|cdocsch1.tex|, |cdocsch2.tex|, |cdocspt3.tex|, |cdocspt4.tex|,
|cdocsdrf.tex|, |cdocsfn1.tex|, |cdocsfn2.tex|.
Then copy the file |childdoc.def| to an appropriate directory of your \LaTeX{}
distribution, e.g.\ \textit{texmf-root}|/tex/latex/childdoc|.
\end{itemize}

%%%%%%%%%%%%%%%%%%%%%%%%%%%%%%%%%%%%%%%%%%%%%%%%%%%%%%%%%%%%%%%%%%%%%%%%%%%%%%%%
\subsection{Related CTAN Packages}

There are several other packages which offer a similar functionality:
%
\begin{itemize}
\item
The packages
\href{http://ctan.org/pkg/docmute}{\textsf{docmute}},
\href{http://ctan.org/pkg/includex}{\textsf{includex}} and
\href{http://ctan.org/pkg/standalone}{\textsf{standalone}}
provide commands to include only the document body of
a child file thus allowing both files to be compiled individually.
\item
The packages \href{http://ctan.org/pkg/subdocs}{\textsf{subdocs}}
and \href{http://ctan.org/pkg/subfiles}{\textsf{subfiles}}
provide structures in which the main and child documents can be
encapsulated and allowing them to be compiled individually.
The inclusion mechanism is different from the conventional |\include|.
\item
The package \href{http://ctan.org/pkg/combine}{\textsf{combine}}
is an elaborate solution to combine several documents into one.
\end{itemize}
%
See also the CTAN topic \href{http://ctan.org/topic/subdocs}{\textsf{subdocs}}
for further related packages.
The present package differs from the above solutions in that
a document structure constructed with the conventional |\include| mechanism
just needs two extra commands at the top of every file
such that all constituent files can be compiled individually.

%%%%%%%%%%%%%%%%%%%%%%%%%%%%%%%%%%%%%%%%%%%%%%%%%%%%%%%%%%%%%%%%%%%%%%%%%%%%%%%%
%\subsection{Feature Suggestions}
%
%The following is a list of features which may be useful for future
%versions of this package:
%%
%\begin{itemize}
%\item
%\ldots
%\end{itemize}

%%%%%%%%%%%%%%%%%%%%%%%%%%%%%%%%%%%%%%%%%%%%%%%%%%%%%%%%%%%%%%%%%%%%%%%%%%%%%%%%
\subsection{Revision History}

%%%%%%%%%%%%%%%%%%%%%%%%%%%%%%%%%%%%%%%%
\paragraph{v2.0:} 2018/12/30

\begin{itemize}
\item
immediate forward processing
\item
added |\childdocby| mechanism
\item
manual restructured
\end{itemize}

%%%%%%%%%%%%%%%%%%%%%%%%%%%%%%%%%%%%%%%%
\paragraph{v1.6:} 2018/01/17

\begin{itemize}
\item
application for development of include files
\item
corrections to manual
\end{itemize}

%%%%%%%%%%%%%%%%%%%%%%%%%%%%%%%%%%%%%%%%
\paragraph{v1.5:} 2017/05/21

\begin{itemize}
\item
more complete structuring introduced
\item
|\childdocof| introduced
\item
|\childdoc| renamed to |\childdocmain|
\item
|\childredirect| renamed to |\childdocforward| and |\childdocforwardprefix|
and functionality expanded
\end{itemize}

%%%%%%%%%%%%%%%%%%%%%%%%%%%%%%%%%%%%%%%%
\paragraph{v1.0:} 2017/04/27

\begin{itemize}
\item
manual and install package
\item
first version published on CTAN
\end{itemize}

%%%%%%%%%%%%%%%%%%%%%%%%%%%%%%%%%%%%%%%%
\paragraph{v0.6:} 2017/04/26

\begin{itemize}
\item
redirection mechanism added
\end{itemize}

%%%%%%%%%%%%%%%%%%%%%%%%%%%%%%%%%%%%%%%%
\paragraph{v0.5:} 2017/04/26

\begin{itemize}
\item
functionality in definition file
\end{itemize}


%%%%%%%%%%%%%%%%%%%%%%%%%%%%%%%%%%%%%%%%%%%%%%%%%%%%%%%%%%%%%%%%%%%%%%%%%%%%%%%%
%%%%%%%%%%%%%%%%%%%%%%%%%%%%%%%%%%%%%%%%%%%%%%%%%%%%%%%%%%%%%%%%%%%%%%%%%%%%%%%%
%%%%%%%%%%%%%%%%%%%%%%%%%%%%%%%%%%%%%%%%%%%%%%%%%%%%%%%%%%%%%%%%%%%%%%%%%%%%%%%%
\appendix

\settowidth\MacroIndent{\rmfamily\scriptsize 000\ }

 \DocInput{childdoc.dtx}

\end{document}
%</driver>
% \fi
%
% %%%%%%%%%%%%%%%%%%%%%%%%%%%%%%%%%%%%%%%%%%%%%%%%%%%%%%%%%%%%%%%%%%%%%%%%%%%%%%
% %%%%%%%%%%%%%%%%%%%%%%%%%%%%%%%%%%%%%%%%%%%%%%%%%%%%%%%%%%%%%%%%%%%%%%%%%%%%%%
% \section{Sample}
%\iffalse
%<*samplemain>
%\fi
%
% The following presents a sample document
% with two chapters, two parts, a title page,
% a compile flag as well as three forwarding files to set the flag.
% It consists of eight |.tex| files:
% \begin{center}
% \begin{tabular}{ll}
% |cdocsamp.tex|&main file\\
% |cdocsch1.tex|&include file for chapter 1\\
% |cdocsch2.tex|&include file for chapter 2\\
% |cdocspt3.tex|&include file for part 3\\
% |cdocspt4.tex|&include file for part 4\\
% |cdocsdrf.tex|&forwarding file for main file in draft mode\\
% |cdocsfi1.tex|&forwarding file for final version of chapter 1\\
% |cdocsfi2.tex|&forwarding file for final version of chapter 2\\
% \end{tabular}
% \end{center}
% Each of the eight files can be compiled directly by the \LaTeX{} compiler.
%
% %%%%%%%%%%%%%%%%%%%%%%%%%%%%%%%%%%%%%%
% \paragraph{Main File.}
%
% The main file is called |cdocsamp.tex|.
%
% Load the \textsf{childdoc} definitions and
% declare the filename for the main document:
%    \begin{macrocode}
\input{childdoc.def}
\childdocmain{}
%    \end{macrocode}

% Optional override for |\version| flag:
%    \begin{macrocode}
%%\ifchilddoc\else\providecommand{\version}{draft}\fi
%    \end{macrocode}

% Define the default values for the |\version| flag
% (|final| for the main file and |draft| for childs):
%    \begin{macrocode}
\ifchilddoc
\providecommand{\version}{draft}
\else
\providecommand{\version}{final}
\fi
%    \end{macrocode}

% Load the standard document class:
%    \begin{macrocode}
\documentclass[12pt]{article}
%    \end{macrocode}

% Start the document body:
%    \begin{macrocode}
\begin{document}
%    \end{macrocode}

% Declare a title page.
% Print title, part of document being processed and version flag:
%    \begin{macrocode}
\addtocounter{page}{-1}
\begin{center}
{\LARGE\bfseries{}childdoc example\par}
\vspace{1cm}
\ifchilddoc
\ifchilddocmanual part\else chapter\fi:
`\childdocname' of `\childdocjob'\par
\else
main document: `\childdocjob'\par
\fi
version: \version\par
\end{center}
\newpage
%    \end{macrocode}

% Manually include selected file,
% otherwise process as usual:
%    \begin{macrocode}
\ifchilddocmanual
\section*{part `\childdocname'}
\input{\childdocname}
\else
%    \end{macrocode}

% Include the two chapters:
%    \begin{macrocode}
\include{cdocsch1}
\include{cdocsch2}
%    \end{macrocode}

% Include the two parts unless only chapters should be displayed:
%    \begin{macrocode}
\ifchilddoc\else
\section{part three}
\input{cdocspt3}
\section{part four}
\input{cdocspt4}
\fi
%    \end{macrocode}

% Process as usual until here:
%    \begin{macrocode}
\fi
%    \end{macrocode}

% End of document body:
%    \begin{macrocode}
\end{document}
%    \end{macrocode}
%\iffalse
%</samplemain>
%\fi
%
% %%%%%%%%%%%%%%%%%%%%%%%%%%%%%%%%%%%%%%
% \paragraph{Chapter Include Files.}
%
% The include files are called |cdocsch1.tex| and |cdocsch2.tex|.
%
%\iffalse
%<*samplechap1|samplechap2>
%\fi

% Optional override for |\version| flag:
%    \begin{macrocode}
%%\providecommand{\version}{final}
%    \end{macrocode}

% Include the main document:
%    \begin{macrocode}
\input{childdoc.def}
\childdocof{cdocsamp}
%    \end{macrocode}

%\iffalse
%</samplechap1|samplechap2>
%\fi
%
%\iffalse
%<*samplechap1>
%\fi
% Some text for chapter 1:
%    \begin{macrocode}
\section{one}
some text in chapter one
%    \end{macrocode}

%\iffalse
%</samplechap1>
%\fi
% Some text for chapter 2:
%\iffalse
%<*samplechap2>
%\fi
%    \begin{macrocode}
\section{two}
more text in chapter two
%    \end{macrocode}

%\iffalse
%</samplechap2>
%\fi
%
% %%%%%%%%%%%%%%%%%%%%%%%%%%%%%%%%%%%%%%
% \paragraph{Part Include Files.}
%
% The include files are called |cdocspt3.tex| and |cdocspt4.tex|.
%
%\iffalse
%<*samplepart3|samplepart4>
%\fi

% Optional override for |\version| flag:
%    \begin{macrocode}
%%\providecommand{\version}{final}
%    \end{macrocode}

% Include the main document:
%    \begin{macrocode}
\input{childdoc.def}
\childdocby{cdocsamp}
%    \end{macrocode}

%\iffalse
%</samplepart3|samplepart4>
%\fi
%
%\iffalse
%<*samplepart3>
%\fi
% Some text for part 3:
%    \begin{macrocode}
some text in part three
%    \end{macrocode}

%\iffalse
%</samplepart3>
%\fi
% Some text for part 4:
%\iffalse
%<*samplepart4>
%\fi
%    \begin{macrocode}
more text in part four
%    \end{macrocode}

%\iffalse
%</samplepart4>
%\fi
%
% %%%%%%%%%%%%%%%%%%%%%%%%%%%%%%%%%%%%%%
% \paragraph{Forwarding for a Complete Draft.}
%
% The following forwarding file |cdocsdrf.tex|
% compiles the main document in draft mode:
%\iffalse
%<*sampledraft>
%\fi
%    \begin{macrocode}
\def\version{draft}
\input{childdoc.def}
\childdocforward{cdocsamp}
%    \end{macrocode}

%\iffalse
%</sampledraft>
%\fi
%
% %%%%%%%%%%%%%%%%%%%%%%%%%%%%%%%%%%%%%%
% \paragraph{Forwarding for Final Version of the Chapters.}
%
% The following forwarding files |cdocsfn1.tex| and |cdocsfn2.tex|
% (with identical content)
% compile the final versions of the child documents
% |cdocsch1.tex| and |cdocsch2.tex|, respectively:
%\iffalse
%<*samplefinal>
%\fi
%    \begin{macrocode}
\def\version{final}
\input{childdoc.def}
\childdocforwardprefix[cdocsamp]{cdocsfn}{cdocsch}
%    \end{macrocode}

%\iffalse
%</samplefinal>
%\fi
%
% %%%%%%%%%%%%%%%%%%%%%%%%%%%%%%%%%%%%%%
% \paragraph{Command Line Processing.}
%
% The following three command lines generate the output files
% |cdocscld|, |cdocscl1| and |cdocscl2|
% which should be identical to
% |cdocsdrf|, |cdocsch1| and |cdocsfn2|, respectively:
% \begin{center}
% \begin{tabular}{l}
% |latex -jobname cdocscld \|\\
% |  "\def\version{draft}\input{childdoc.def}\childdocforward{cdocsamp}"|\\
% |latex -jobname cdocscl1 \|\\
% |  "\input{childdoc.def}\childdocforward[cdocsamp]{cdocsch1}"|\\
% |latex -jobname cdocscl2 \|\\
% |  "\def\version{final}\input{childdoc.def}\childdocforward{cdocsch2}"|
% \end{tabular}
% \end{center}
% Note that the trailing backslash on each first line
% merely continues the input to the second line
% (for convenient cut ant paste).
% Furthermore, the command |latex| can be replaced by any
% of its alternative versions such as |pdflatex|.
%
% %%%%%%%%%%%%%%%%%%%%%%%%%%%%%%%%%%%%%%%%%%%%%%%%%%%%%%%%%%%%%%%%%%%%%%%%%%%%%%
% %%%%%%%%%%%%%%%%%%%%%%%%%%%%%%%%%%%%%%%%%%%%%%%%%%%%%%%%%%%%%%%%%%%%%%%%%%%%%%
% \section{Implementation}
%\iffalse
%<*package>
%\fi
%
% This section describes the definitions file |childdoc.def|.

% The definitions cannot be loaded using |\usepackage| or |\RequirePackage|
% which has a mechanism to prevent loading a style file more than once.
% When loading the definitions by means of |\input|
% multiple instances have to be prevented manually:
%\iffalse
%This code needs to be before the `\ProvidesFile' directive
%which is defined at the beginning of this file.
%Therefore it is also placed there and commented out here.
%</package>
%<*discard>
%\fi
%    \begin{macrocode}
\ifdefined\childdocmain\endinput\fi
%    \end{macrocode}
%\iffalse
%</discard>
%<*package>
%\fi
%
% \macro{\ifchilddoc}
% \macro{\ifchilddocmanual}
% The conditional |\ifchilddoc| tells whether a
% child (true) or main (false) document is being compiled.
% The conditional |\ifchilddocmanual| tells whether
% the |\includeonly| mechanism is used (false) or
% the selection of child files must be performed manually (true).
% The definitions initialise to false:
%    \begin{macrocode}
\newif\ifchilddoc
\newif\ifchilddocmanual
%    \end{macrocode}

% \macro{\childdocname}
% \macro{\childdocjob}
% The macro |\childdocname| stores the name of the main document
% to be compiled. The macro |\childdocjob| stores the name of
% the document on which the \LaTeX{} compiler was originally invoked.
% The content of |\jobname| cannot be compared
% to filenames specified in the source due to different catcodes.
% The following code rescans |\jobname|, stores the result
% in |\childdocname| and saves a copy in |\childdocjob|:
%    \begin{macrocode}
\edef\childdocname{\scantokens\expandafter{\jobname\noexpand}}
\let\childdocjob\childdocname
%    \end{macrocode}

% \macro{\childdocdisable}
% The macro |\childdocdisable| prevents the main file
% from being processed more than once.
% At this stage, the main document command |\childdocmain|
% is assumed to be called once again where it should do nothing.
% Any subsequent call to it should prevent
% a secondary processing of the main document
% It overwrites the forwarding commands
% |\childdocof| and |\childdocforward|
% with empty macros to prevent further inclusions of the main document:
%    \begin{macrocode}
\newcommand{\childdocdisable}
{
  \renewcommand{\childdocmain}[1]{\renewcommand{\childdocmain}[1]{\endinput}}
  \renewcommand{\childdocof}[1]{}
  \renewcommand{\childdocby}[2][]{}
  \renewcommand{\childdocforward}[2][]{}
  \renewcommand{\childdocdisable}{}
}
%    \end{macrocode}

% \macro{\childdocmain}
% The macro |\childdocmain| is to be called at the top of the main file
% with nothing or the main filename (without extension) as argument.
% First, it breaks loops.
% If the argument is not empty and does not match |\childdocname|
% (which is set by the first inclusion of |childdoc.def|),
% |\ifchilddoc| is set to true, |\includeonly| is applied to the child file
% and |\jobname| is set to the main file
% (for proper handling of |.aux| files):
%    \begin{macrocode}
\newcommand{\childdocmain}[1]
{
  \childdocdisable\childdocmain{}
  \if?#1?\else
    \begingroup
      \def\childdoctmp{#1}
      \ifx\childdoctmp\childdocname
        \def\childdoctmp{}
      \else
        \def\childdoctmp
        {
          \childdoctrue
          \includeonly{\childdocname}
          \def\childdocjob{#1}
          \def\jobname{#1}
        }
      \fi
      \expandafter
    \endgroup
    \childdoctmp
  \fi
}
%    \end{macrocode}

% \macro{\childdocof}
% The command |\childdocof| redirects
% compilation to the main file |#1|.
%    \begin{macrocode}
\newcommand{\childdocof}[1]
{
  \childdocdisable
  \childdoctrue
  \includeonly{\childdocname}
  \def\jobname{#1}
  \def\childdocjob{#1}
  \input{#1}
}
%    \end{macrocode}

% \macro{\childdocby}
% The command |\childdocby| ....
%    \begin{macrocode}
\newcommand{\childdocby}[2][]
{
  \childdocdisable
  \childdoctrue
  \childdocmanualtrue
  \if?#1?\else
    \def\jobname{#2}
  \fi
  \def\childdocjob{#2}
  \input{#2}
  \endinput
}
%    \end{macrocode}

% \macro{\childdocforward}
% The command |\childdocforward| redirects
% compilation to the main file or
% (if the optional argument is given) a child file.
% Parameters are set as if the main file
% or a child file starting with |\childdocof| was compiled.
% Then compilation is handed over to the main file:
%    \begin{macrocode}
\newcommand{\childdocforward}[2][]
{
  \begingroup
    \if?#1?
      \def\childdoctmp
      {
        \def\childdocname{#2}
        \def\childdocjob{#2}
        \def\jobname{#2}
        \input{#2}
        \endinput
      }
    \else
      \def\childdoctmp
      {
        \childdocdisable
        \def\childdocname{#2}
        \childdoctrue
        \includeonly{#2}
        \def\childdocjob{#1}
        \def\jobname{#1}
        \input{#1}
        \endinput
      }
    \fi
    \expandafter
  \endgroup
  \childdoctmp
}
%    \end{macrocode}

% \macro{\childdocforwardprefix}
% The command |\childdocforwardprefix| redirects
% compilation to the main or a child file by means of a pattern.
% The prefix |#1| in the current filename is replaced by |#2|
% and the suffix of the current filename is kept
% (it is assumed that the filename does not contain the substring `|~~~|'
% which is used as a delimiter).
% Compilation is handed over to the new file by |\childdocforward|:
%    \begin{macrocode}
\newcommand{\childdocforwardprefix}[3][]
{
  \begingroup
    \def\childdocextract #2##1~~~{\def\childdoctmp{\childdocforward[#1]{#3##1}}}
    \expandafter\childdocextract\childdocname~~~
    \expandafter
  \endgroup
  \childdoctmp
}
%    \end{macrocode}

% \macro{\childdoc}
% The deprecated macro |\childdoc| is a legacy version of |\childdocmain|:
%    \begin{macrocode}
\newcommand{\childdoc}{\childdocmain}
%    \end{macrocode}

% \macro{\childdocredirect}
% The deprecated macro |\childdocredirect| is a legacy version
% of |\childdocforward| and |\childdocforwardprefix|:
%    \begin{macrocode}
\newcommand{\childdocredirect}[2][]
{
  \begingroup
    \if?#1?
      \def\childdoctmp{\childdocforward{#2}}
    \else
      \def\childdoctmp{\childdocforwardprefix{#1}{#2}}
    \fi
    \expandafter
  \endgroup
  \childdoctmp
}
%    \end{macrocode}

%\iffalse
%</package>
%\fi
%
\endinput

\childdocby{cdocsamp}
%    \end{macrocode}

%\iffalse
%</samplepart3|samplepart4>
%\fi
%
%\iffalse
%<*samplepart3>
%\fi
% Some text for part 3:
%    \begin{macrocode}
some text in part three
%    \end{macrocode}

%\iffalse
%</samplepart3>
%\fi
% Some text for part 4:
%\iffalse
%<*samplepart4>
%\fi
%    \begin{macrocode}
more text in part four
%    \end{macrocode}

%\iffalse
%</samplepart4>
%\fi
%
% %%%%%%%%%%%%%%%%%%%%%%%%%%%%%%%%%%%%%%
% \paragraph{Forwarding for a Complete Draft.}
%
% The following forwarding file |cdocsdrf.tex|
% compiles the main document in draft mode:
%\iffalse
%<*sampledraft>
%\fi
%    \begin{macrocode}
\def\version{draft}
% \iffalse
%
% childdoc.dtx Copyright (C) 2017-2018 Niklas Beisert
%
% This work may be distributed and/or modified under the
% conditions of the LaTeX Project Public License, either version 1.3
% of this license or (at your option) any later version.
% The latest version of this license is in
%   http://www.latex-project.org/lppl.txt
% and version 1.3 or later is part of all distributions of LaTeX
% version 2005/12/01 or later.
%
% This work has the LPPL maintenance status `maintained'.
%
% The Current Maintainer of this work is Niklas Beisert.
%
% This work consists of the files childdoc.dtx and childdoc.ins
% and the derived files childdoc.def and cdocsamp.tex with
% cdocsch1.tex, cdocsch2.tex, cdocsdrf.tex, cdocsfn1.tex, cdocsfn2.tex.
%
%<package>\ifdefined\childdocmain\endinput\fi
%<package>\ProvidesFile{childdoc.def}[2018/12/30 v2.0 child document driver]
%<samplemain>\ProvidesFile{cdocsamp.tex}[2018/12/30 v2.0 sample for childdoc]
%<*driver>
%\ProvidesFile{childdoc.drv}[2018/12/30 v2.0 childdoc reference manual file]
\PassOptionsToClass{10pt,a4paper}{article}
\documentclass{ltxdoc}

\usepackage[margin=35mm]{geometry}
\usepackage{hyperref}
\usepackage{hyperxmp}
\usepackage[usenames]{color}

\hypersetup{colorlinks=true}
\hypersetup{pdfstartview=FitH}
\hypersetup{pdfpagemode=UseNone}
\hypersetup{pdfsource={}}
\hypersetup{pdflang={en-UK}}
\hypersetup{pdfcopyright={Copyright 2017-2018 Niklas Beisert.
  This work may be distributed and/or modified under the
  conditions of the LaTeX Project Public License, either version 1.3
  of this license or (at your option) any later version.}}
\hypersetup{pdflicenseurl={http://www.latex-project.org/lppl.txt}}
\hypersetup{pdfcontactaddress={ETH Zurich, ITP, HIT K,
  Wolfgang-Pauli-Strasse 27}}
\hypersetup{pdfcontactpostcode={8093}}
\hypersetup{pdfcontactcity={Zurich}}
\hypersetup{pdfcontactcountry={Switzerland}}
\hypersetup{pdfcontactemail={nbeisert@itp.phys.ethz.ch}}
\hypersetup{pdfcontacturl={http://people.phys.ethz.ch/\xmptilde nbeisert/}}

\newcommand{\secref}[1]{\hyperref[#1]{section \ref*{#1}}}

\parskip1ex
\parindent0pt
\let\olditemize\itemize
\def\itemize{\olditemize\parskip0pt}

\begin{document}

\title{The \textsf{childdoc} Package}
\hypersetup{pdftitle={The childdoc Package}}
\author{Niklas Beisert\\[2ex]
  Institut f\"ur Theoretische Physik\\
  Eidgen\"ossische Technische Hochschule Z\"urich\\
  Wolfgang-Pauli-Strasse 27, 8093 Z\"urich, Switzerland\\[1ex]
  \href{mailto:nbeisert@itp.phys.ethz.ch}
  {\texttt{nbeisert@itp.phys.ethz.ch}}}
\hypersetup{pdfauthor={Niklas Beisert}}
\hypersetup{pdfsubject={Manual for the LaTeX2e Package childdoc}}
\date{30 December 2018, \textsf{v2.0}}
\maketitle

\begin{abstract}\noindent
\textsf{childdoc} is a \LaTeXe{} package
that enables the direct compilation
of document sections included by |\include|
to individual files.
\end{abstract}

\begingroup
\parskip0ex
\tableofcontents
\endgroup

%%%%%%%%%%%%%%%%%%%%%%%%%%%%%%%%%%%%%%%%%%%%%%%%%%%%%%%%%%%%%%%%%%%%%%%%%%%%%%%%
%%%%%%%%%%%%%%%%%%%%%%%%%%%%%%%%%%%%%%%%%%%%%%%%%%%%%%%%%%%%%%%%%%%%%%%%%%%%%%%%
\section{Introduction}

\LaTeX{} provides a mechanism to structure a large document (such as a book)
into a main file and several child files (containing the chapters)
using the |\include| command.
This mechanism is beneficial for documents
which span hundreds of pages in order to
make the source file(s) more manageable.
Moreover, compilation can be restricted to
selected child files by means of the |\includeonly| command.
The latter feature can be used to reduce the compilation time while editing
(this was significantly more useful in the earlier days of \LaTeX{})
or to generate a smaller document which is easier to navigate.
Another application of |\includeonly| is to generate
documents consisting of selected parts of the complete document.

However, there are a few drawbacks of the plain |\include| mechanism:
\begin{itemize}
\item
The child files cannot be compiled on their own,
they can only be compiled via the main file.
A naive editing environment
(such as a text editor with an option
to have the current file processed by \LaTeX)
may require one to switch to the main file before compiling;
attempting to compile the child file produces errors.
\item
The main file must be modified (each time)
to adjust the |\includeonly| command
to the present needs. This easily leaves the main file in a messy state.
\item
The generated document will always carry the filename
of the main document. This is inconvenient if
several child files are to be compiled and
to be kept for distribution.
\end{itemize}

The present package provides a simple interface
to make child files individually compilable by \LaTeX{}.
Compiling a child file then has the same effect as compiling
the main file with an |\includeonly| command
to select the appropriate child.
Moreover the generated document will carry the name of the child
rather than the main file.
This resolves all three above issues.

This feature is meant to make the editing of books,
thesis documents and lecture notes somewhat more convenient.
However, the package can also be used efficiently for
composing a series of documents (such as exercise sheets)
which are typically distributed individually.
It then assists the author in generating the individual documents
(potentially in different versions)
as well as a document containing the collected series.
Another application is in developing style files
or other kinds of included material
where compilation of the style file could redirect
to a sample or test file.

%%%%%%%%%%%%%%%%%%%%%%%%%%%%%%%%%%%%%%%%%%%%%%%%%%%%%%%%%%%%%%%%%%%%%%%%%%%%%%%%
%%%%%%%%%%%%%%%%%%%%%%%%%%%%%%%%%%%%%%%%%%%%%%%%%%%%%%%%%%%%%%%%%%%%%%%%%%%%%%%%
\section{Usage}

First of all, the package \textsf{childdoc} is \emph{not} a standard
\LaTeXe{} |.sty| style file! Therefore it needs to be invoked in
a non-standard way.

%%%%%%%%%%%%%%%%%%%%%%%%%%%%%%%%%%%%%%%%%%%%%%%%%%%%%%%%%%%%%%%%%%%%%%%%%%%%%%%%
\subsection{Included Files}
\label{sec:include}

%%%%%%%%%%%%%%%%%%%%%%%%%%%%%%%%%%%%%%%%
\DescribeMacro{\childdocmain}
To use the package, add the commands
\begin{center}
\begin{tabular}{l}
|\input{childdoc.def}|\\
|\childdocmain{}|\\
\end{tabular}
\end{center}
at the very top of the main \LaTeX{} file,
in particular \emph{before} the |\documentclass| statement!
The argument of |\childdocmain| should be left empty
(but it must be present).

%%%%%%%%%%%%%%%%%%%%%%%%%%%%%%%%%%%%%%%%
\DescribeMacro{\childdocof}
Furthermore, add the commands
\begin{center}
\begin{tabular}{l}
|\input{childdoc.def}|\\
|\childdocof{|\textit{main}|}|\\
\end{tabular}
\end{center}
at the top of every child file \textit{child}
which is included by |\include{|\textit{child}|}|
from within the main file
(or at least for those files to be compiled individually).
The argument \textit{main} must be the filename of the main file.

There are a couple of
considerations in setting up the main and child documents:

%%%%%%%%%%%%%%%%%%%%%%%%%%%%%%%%%%%%%%%%
\paragraph{Restrictions.}

Please note the following restrictions:
\begin{itemize}
\item
|\childdocmain| must be called with one argument \textit{main}
to ensure compatibility with earlier version of the package.
It must either be empty (|\childdocmain{}|)
or precisely match the filename of the main file in which it is specified.
See \secref{sec:detection} for further information.
\item
The filename \textit{main} must be specified without the |.tex| extension.
\item
The filename \textit{main} is case sensitive
(even in case-insensitive file systems)
due to internal string comparison.
\item
The argument \textit{main} should be fully expanded, it cannot be a macro.
\item
Subdirectories and special characters should be avoided in filenames.
\item
The command |\childdocmain{|\textit{main}|}| must be followed by a whitespace.
It should not be followed immediately by another command
or by a comment mark `|%|'.
This is because the \TeX{} parser reads the token immediately following
the argument of |\childdocmain| and puts it
at the beginning of every child section;
however, a white\-space is ignored.
\end{itemize}

%%%%%%%%%%%%%%%%%%%%%%%%%%%%%%%%%%%%%%%%
\paragraph{Content of Main File.}

It is advisable to place all content in the child files included by |\include|.
Any output contained in the main file will appear in all child documents
unless suppressed manually;
it cannot be suppressed automatically by the |\includeonly| directive
and thus should normally be avoided.
A method to include some content in the main file
by means of conditional processing is described in \secref{sec:conditional}.

%%%%%%%%%%%%%%%%%%%%%%%%%%%%%%%%%%%%%%%%
\paragraph{Page Numbering.}

When only a part of the document is compiled,
the appropriate numbering of pages
(as well as other status parameters)
is determined from the |.aux| files.
The latter contain information from previous passes.
However this information needs to propagate through
all intermediate child documents.
Therefore the page numbering in child documents may well
be inconsistent until the complete document is compiled at least once.

A useful (if unconventional) way to always ensure a consistent
page numbering is to restart the numbering in each child document
and denote the pages by `\textit{child}|.|\textit{page}'
where \textit{child} represents the chapter/section number of the child file.
This can be achieved by the command
|\numberwithin{page}{|\textit{child}|}|
of the \textsf{amsmath} package
where \textit{child} can be |chapter| or |section|
depending on the chosen structuring.
Alternatively, one can modify the macro |\thepage| appropriately
and reset the counter |page| at the start of each child file.

%%%%%%%%%%%%%%%%%%%%%%%%%%%%%%%%%%%%%%%%%%%%%%%%%%%%%%%%%%%%%%%%%%%%%%%%%%%%%%%%
\subsection{Conditional Processing}
\label{sec:conditional}

The package provides a mechanism to compile different versions
of a document. To customise the versions further some conditional processing
can come in handy to distinguish which version is being compiled.
The package provides two macros to describe the compilation context:

%%%%%%%%%%%%%%%%%%%%%%%%%%%%%%%%%%%%%%%%
\DescribeMacro{\ifchilddoc}
The conditional |\ifchilddoc| distinguishes between the compilation of
child documents and the main document:
%
\begin{center}
|\ifchilddoc |\textit{child-code}| |[|\||else |\textit{main-code}]| \||fi|
\end{center}

%%%%%%%%%%%%%%%%%%%%%%%%%%%%%%%%%%%%%%%%
\DescribeMacro{\childdocname}
\DescribeMacro{\childdocjob}
The macro |\childdocname| contains the filename (without extension)
of the main or child file being processed.
Note that |\childdocjob| will always contain the name of the main file.

%%%%%%%%%%%%%%%%%%%%%%%%%%%%%%%%%%%%%%%%
\paragraph{Title Page.}

Conditional processing can be used to include a title or banner page
in the main document when proper precautions are taken.
Importantly, the code in the main file should ensure that the page counter
(as well as other status parameters which are stored in the |.aux| files)
takes the same value after the conditional processing.
Otherwise the page numbers may take divergent values
depending on which part is compiled.

For example, a title page could be declared by:
%
\begin{center}
\begin{tabular}{l}
|\ifchilddoc\||else|\\
|\addtocounter{page}{-1}|\\
\textit{code for title page}\\
|\newpage|\\
|\||fi|
\end{tabular}
\end{center}
%
A banner page for the child documents can be generated by:
%
\begin{center}
\begin{tabular}{l}
|\ifchilddoc|\\
|\addtocounter{page}{-1}|\\
\textit{code for banner page}\\
|\newpage|\\
|\||fi|
\end{tabular}
\end{center}
%
Here one could write a message such as:
\begin{center}
|This is the part \childdocname{} of \childdocjob{}.|
\end{center}

%%%%%%%%%%%%%%%%%%%%%%%%%%%%%%%%%%%%%%%%%%%%%%%%%%%%%%%%%%%%%%%%%%%%%%%%%%%%%%%%
\subsection{Flags}
\label{sec:flags}

The package makes it easy to generate different versions
of the main or child documents.
To this end compilation flags can be defined
and assigned different default values.
They will be particularly useful in conjunction
with the forwarding mechanism described in \secref{sec:forward}.

For example, it may be useful to have a flag |\version|
which can be set to |draft| or |final|.
The document source will contain some conditional code
depending on the value of |\version|.
Suppose further, the flag should default to |final| for the main file
and to |draft| for child files
which is a natural assignment for editing the document.
This is achieved by placing the following code
in the preamble of the main document
(below the |\childdocmain| directive):
%
\begin{center}
\begin{tabular}{l}
|\ifchilddoc|\\
|\providecommand{\version}{draft}|\\
|\||else|\\
|\providecommand{\version}{final}|\\
|\||fi|
\end{tabular}
\end{center}
%
The definition by |\providecommand| makes sure
that previous definitions are not overwritten.
Further statements |\providecommand{\version}{...}|
can thus be added before the above code to override it.

For the main file, one might add a line
(between |\childdocmain| and the above block)
%
\begin{center}
|%\ifchilddoc\||else\providecommand{\version}{draft}\||fi|
\end{center}
%
which can be uncommented to produce a draft version.
Likewise one can add a line to the very top of a child file
(above the |\childdocof{|\textit{main}|}| directive)
%
\begin{center}
|%\providecommand{\version}{final}|
\end{center}
%
which can be uncommented to produce the final version of this child document.

%%%%%%%%%%%%%%%%%%%%%%%%%%%%%%%%%%%%%%%%%%%%%%%%%%%%%%%%%%%%%%%%%%%%%%%%%%%%%%%%
\subsection{Forwarding}
\label{sec:forward}

Different versions of the main or child documents
using compilation flags as described in \secref{sec:flags}
can be (permanently) stored in different files
for convenient compilation, viewing and distribution.
To this end, the package defines a command
to pass on compilation to a different file:

%%%%%%%%%%%%%%%%%%%%%%%%%%%%%%%%%%%%%%%%
\DescribeMacro{\childdocforward}
The command |\childdocforward| redirects processing to
another source file:
%
\begin{center}
\begin{tabular}{l}
|\input{childdoc.def}|\\
|\childdocforward[|\textit{main}|]{|\textit{dest}|}|\\
\end{tabular}
\end{center}
%
The argument \textit{dest} is the destination file
(without extension).
It should be the main file or one of the child files.
Note that further \textsf{childdoc} directives
such as |\childdocof| and |\childdocforward|
in the indicated file will be processed in this form.
The optional argument \textit{main}
passes on directly to the main file \textit{main}
while pretending to compile the child \textit{dest}.
This form behaves as if \textit{dest}
issues |\childdocof{|\textit{main}|}| right away,
and no further \textsf{childdoc} directives will be processed.

%%%%%%%%%%%%%%%%%%%%%%%%%%%%%%%%%%%%%%%%
\DescribeMacro{\...prefix}
In the alternative form |\childdocforwardprefix|,
%
\begin{center}
\begin{tabular}{l}
|\input{childdoc.def}|\\
|\childdocforwardprefix[|\textit{main}|]{|\textit{prefix}|}{|\textit{dest}|}|
\end{tabular}
\end{center}
%
the destination file is determined by a pattern
depending on the current file:
To make this work, the current file must be called
`{\textit{prefix}\hspace{0.2em}\textit{suffix}}'
with \textit{prefix} matching precisely the argument.
Processing is then passed on to the file
`{\textit{dest}\hspace{0.2em}\textit{suffix}}'.
Surely, the same effect is achieved by
directly specifying the
argument `{\textit{dest}\hspace{0.2em}\textit{suffix}}'
in the first form.
However, that requires to set up a different file
for each child. With the alternative form of the command
all these files can have exactly the same content
which simplifies setting them up and maintaining them.

For example, the following file |draft.tex|
with a compilation flag |\version| as described in \secref{sec:flags}
compiles the main document as a draft:
%
\begin{center}
\begin{tabular}{l}
|\def\version{draft}|\\
|\input{childdoc.def}|\\
|\childdocforward{|\textit{main}|}|
\end{tabular}
\end{center}
%
Likewise, the following files |final|\textit{nn}|.tex|
compile the final version of the child document
|child|\textit{nn}|.tex|:
%
\begin{center}
\begin{tabular}{l}
|\def\version{final}|\\
|\input{childdoc.def}|\\
|\childdocforwardprefix{final}{child}|
\end{tabular}
\end{center}
%

Note that when several versions of a main file and/or of each child file
are to be generated, it may be convenient to set up a |Makefile| or
shell script to automatise the process.

%%%%%%%%%%%%%%%%%%%%%%%%%%%%%%%%%%%%%%%%%%%%%%%%%%%%%%%%%%%%%%%%%%%%%%%%%%%%%%%%
\subsection{Command Line Processing}
\label{sec:commandline}

The effect of redirection files can also be achieved by invoking
the \LaTeX{} compiler with a more elaborate command line.
Most conveniently this should be done as part
of a shell script or a |Makefile|.

When using \textsf{childdoc} in the main file, the following
command lines effectively perform a redirection
(note that depending on the shell being used,
backslashes may have to be doubled: `|\|' $\to$ `|\\|'):
%
\begin{center}
|... -jobname "|\textit{target}|" |\\|"|[\textit{flags}]%
|\input{childdoc.def}\childdocforward[|\textit{main}|]{|\textit{dest}|}"|
\end{center}
%
Here \textit{target} is the name of the output file,
\textit{main} is the name of the main file
and \textit{dest} is the name of the main or child file to be processed
(all filenames without extensions).
The optional argument \textit{main} can be omitted
if \textit{main} matches \textit{dest}.
Optionally, compilation \textit{flags} can be defined via |\def| commands.
This command line makes the \TeX{} engine believe
it is compiling the file \textit{target}
whose content is specified as the latter parameter.
The provided code then forwards the processing to
\textit{main} or \textit{dest} as described in \secref{sec:forward}.

%%%%%%%%%%%%%%%%%%%%%%%%%%%%%%%%%%%%%%%%%%%%%%%%%%%%%%%%%%%%%%%%%%%%%%%%%%%%%%%%
\subsection{Include by Input}
\label{sec:input}

Including child documents by |\include| has some restrictions by design.
Most notably, the content of a child document always occupies
its own set of pages; pages cannot be shared between child documents.
Usually, this behaviour makes perfect sense
because each child document contain an essential part of the document.
However, in some situations it may be desirable to compose
a document from a collection of parts
without having mandatory page breaks between then.
For this case, the package
provides a mechanism to include parts
by |\input| which can also be processed individually.
However, by construction this mechanism
requires manual handling of the content to be output.

%%%%%%%%%%%%%%%%%%%%%%%%%%%%%%%%%%%%%%%%
\DescribeMacro{\ifchilddocmanual}
The main file should be prepared as usual, see \secref{sec:include}.
However, the document body must make a distinction
between processing of an individual part and of the main document, e.g.:
%
\begin{center}
\begin{tabular}{l}
|\ifchilddocmanual|\\
|\input{\childdocname}|\\
|\||else|\\
\textit{document body with }|\input{|\textit{part}|}|\\
|\||fi|
\end{tabular}
\end{center}
%
The conditional |\ifchilddocmanual| is true whenever
a part to be included by |\input| is being compiled,
and the name of the part is stored in |\childdocname|.

%%%%%%%%%%%%%%%%%%%%%%%%%%%%%%%%%%%%%%%%
\DescribeMacro{\childdocby}
Each part to be included by |\input| should start with:
%
\begin{center}
\begin{tabular}{l}
|\input{childdoc.def}|\\
|\childdocby{|\textit{main}|}|\\
\end{tabular}
\end{center}
%
The directive |\childdocby| is similar to |\childdocof|
described in \secref{sec:include},
but the subsequent selection of content must be done manually.
To that end, both |\ifchilddoc| and |\ifchilddocmanual|
will be true upon processing of a part,
and the name of the part is stored in |\childdocname|.
Note that |\jobname| will be set to the filename of the current part
so that each part receives an individual |.aux| file
that does not interfere with the |.aux| file(s) of the main document.
This behaviour can be altered by the alternative form
|\childdocby[*]{|\textit{main}|}| (with a non-empty optional argument)
which uses the |.aux| file of the main document
by setting |\jobname| to \textit{main}.

%%%%%%%%%%%%%%%%%%%%%%%%%%%%%%%%%%%%%%%%%%%%%%%%%%%%%%%%%%%%%%%%%%%%%%%%%%%%%%%%
\subsection{Driver Development}
\label{sec:driver}

The \textsf{childdoc} mechanism can also be use for the development
of definition files such as \LaTeX{} styles or classes.
This case differs from the above setup with multiple parts
included by |\include| in that no |\includeonly| should be invoked.
This can be achieved by starting the include file
(before |\ProvidesPackage|) with:
%
\begin{center}
\begin{tabular}{l}
|\input{childdoc.def}|\\
|\childdocforward{|\textit{main}|}|\\
\end{tabular}
\end{center}
%
or alternatively with:
%
\begin{center}
\begin{tabular}{l}
|\input{childdoc.def}|\\
|\childdocby{|\textit{main}|}|\\
\end{tabular}
\end{center}
%
Both forms have slightly different effects as described above.
The main file is prepared as usual, see \secref{sec:include}.

%%%%%%%%%%%%%%%%%%%%%%%%%%%%%%%%%%%%%%%%%%%%%%%%%%%%%%%%%%%%%%%%%%%%%%%%%%%%%%%%
\subsection{Legacy Detection}
\label{sec:detection}

The directive |\childdocmain| in the main file can detect
whether the complete document or merely a child is to be compiled
even without using the directive |\childdocof|.
This method is deprecated because it is less robust
and there is no compelling reason to use it;
it is merely provided for backward compatibility
and it may be removed in future versions.

If the detection mechanism is to be used,
it is mandatory to correctly specify
the filename of the main file as the argument of |\childdocmain|:
%
\begin{center}
\begin{tabular}{l}
|\input{childdoc.def}|\\
|\childdocmain{|\textit{main}|}|\\
\end{tabular}
\end{center}
%
If |\jobname| does not match the argument \textit{main} of |\childdocmain|,
it is assumed that |\jobname| points to the child file to be compiled.
When using |\childdocmain| with the main file specified as argument,
it suffices to start a child file
with just |\input{|\textit{main}|}|
without loading of the package and using |\childdocof|.
If instead all processing is done
with the appropriate \textsf{childdoc} directives,
the argument of \textit{main} of |\childdocmain| can be empty.

An alternative version of the command line processing described
in \secref{sec:commandline} using the detection mechanism reads:
%
\begin{center}
|... -jobname "|\textit{target}|" "|[\textit{flags}]%
[|\def\jobname{|\textit{dest}|}|]|\input{|\textit{main}|}"|
\end{center}

%%%%%%%%%%%%%%%%%%%%%%%%%%%%%%%%%%%%%%%%%%%%%%%%%%%%%%%%%%%%%%%%%%%%%%%%%%%%%%%%
\subsection{Manual Code}
\label{sec:manual}

In case one cannot be certain whether the definitions file |childdoc.def|
is installed on the target \TeX{} distribution
and one prefers not to ship it,
it is conceivable to paste a few relevant commands into the sources.

To that end, drop all statements |\input{childdoc.def}|
and perform the replacements as outlined below.
Instead of |\childdocmain{|\textit{main}|}| add the following code
to the top of the main file:
%
\begin{center}
\begin{tabular}{l}
|\||ifdefined\childdocname\endinput\||fi\newif\ifchilddoc|\\
|\edef\childdocname{\scantokens\expandafter{\jobname\noexpand}}|\\
|\def\childdocmain{|\textit{main}|}\||ifx\childdocmain\childdocname\||else|\\
|\childdoctrue\includeonly{\childdocname}\let\jobname\childdocmain\||fi|\\
\end{tabular}
\end{center}
%
Instead of |\childdocof{|\textit{main}|}| just include the main file
at the top of each child file:
%
\begin{center}
|\input{|\textit{main}|}|
\end{center}
%
A simple redirection |\childdocforward{|\textit{dest}|}| is achieved by:
%
\begin{center}
|\def\jobname{|\textit{dest}|}\input{\jobname}|
\end{center}
%
The redirection with prefix
|\childdocforwardprefix[|\textit{prefix}|]{|\textit{dest}|}|
is accomplished by:
%
\begin{center}
\begin{tabular}{l}
|{\edef\jobname{\scantokens\expandafter{\jobname\noexpand}}|\\
|\def\redirectjob |\textit{prefix}|#1~~~{\gdef\jobname{|\textit{dest}|#1}}|\\
|\expandafter\redirectjob\jobname~~~}\input{\jobname}|
\end{tabular}
\end{center}

In an alternative approach,
child documents can be compiled by a specific command line
without additional code or specific definitions:
%
\begin{center}
|... -jobname "|\textit{target}|" "|[\textit{flags}]%
|\includeonly{|\textit{dest}|}\input{|\textit{main}|}"|
\end{center}
%

%%%%%%%%%%%%%%%%%%%%%%%%%%%%%%%%%%%%%%%%%%%%%%%%%%%%%%%%%%%%%%%%%%%%%%%%%%%%%%%%
%%%%%%%%%%%%%%%%%%%%%%%%%%%%%%%%%%%%%%%%%%%%%%%%%%%%%%%%%%%%%%%%%%%%%%%%%%%%%%%%
\section{Information}

%%%%%%%%%%%%%%%%%%%%%%%%%%%%%%%%%%%%%%%%%%%%%%%%%%%%%%%%%%%%%%%%%%%%%%%%%%%%%%%%
\subsection{Copyright}

Copyright \copyright{} 2017--2018 Niklas Beisert

This work may be distributed and/or modified under the
conditions of the \LaTeX{} Project Public License, either version 1.3
of this license or (at your option) any later version.
The latest version of this license is in
  \url{http://www.latex-project.org/lppl.txt}
and version 1.3 or later is part of all distributions of \LaTeX{}
version 2005/12/01 or later.

This work has the LPPL maintenance status `maintained'.

The Current Maintainer of this work is Niklas Beisert.

This work consists of the files |README.txt|, |childdoc.ins| and |childdoc.dtx|
as well as the derived files |childdoc.def|, |cdocsamp.tex|
with |cdocsch1.tex|, |cdocsch2.tex|, |cdocspt3.tex|, |cdocspt4.tex|,
|cdocsdrf.tex|, |cdocsfn1.tex|, |cdocsfn2.tex|
as well as |childdoc.pdf|.

%%%%%%%%%%%%%%%%%%%%%%%%%%%%%%%%%%%%%%%%%%%%%%%%%%%%%%%%%%%%%%%%%%%%%%%%%%%%%%%%
\subsection{Files and Installation}

The package consists of the files:
%
\begin{center}
\begin{tabular}{ll}
    |README.txt|   & readme file \\
    |childdoc.ins| & installation file \\
    |childdoc.dtx| & source file \\
    |childdoc.def| & definition file \\
    |cdocsamp.tex| & sample main file \\
    |cdocsch1.tex| & sample include file \\
    |cdocsch2.tex| & sample include file \\
    |cdocspt3.tex| & sample part file \\
    |cdocspt4.tex| & sample part file \\
    |cdocsdrf.tex| & sample redirection file \\
    |cdocsfn1.tex| & sample redirection file \\
    |cdocsfn2.tex| & sample redirection file \\
    |childdoc.pdf| & manual
\end{tabular}
\end{center}
%
The distribution consists of the files
|README.txt|, |childdoc.ins| and |childdoc.dtx|.
%
\begin{itemize}
\item
Run (pdf)\LaTeX{} on |childdoc.dtx|
to compile the manual |childdoc.pdf| (this file).
\item
Run \LaTeX{} on |childdoc.ins| to create the definitions file |childdoc.def|
and the sample |cdocsamp.tex| with include files
|cdocsch1.tex|, |cdocsch2.tex|, |cdocspt3.tex|, |cdocspt4.tex|,
|cdocsdrf.tex|, |cdocsfn1.tex|, |cdocsfn2.tex|.
Then copy the file |childdoc.def| to an appropriate directory of your \LaTeX{}
distribution, e.g.\ \textit{texmf-root}|/tex/latex/childdoc|.
\end{itemize}

%%%%%%%%%%%%%%%%%%%%%%%%%%%%%%%%%%%%%%%%%%%%%%%%%%%%%%%%%%%%%%%%%%%%%%%%%%%%%%%%
\subsection{Related CTAN Packages}

There are several other packages which offer a similar functionality:
%
\begin{itemize}
\item
The packages
\href{http://ctan.org/pkg/docmute}{\textsf{docmute}},
\href{http://ctan.org/pkg/includex}{\textsf{includex}} and
\href{http://ctan.org/pkg/standalone}{\textsf{standalone}}
provide commands to include only the document body of
a child file thus allowing both files to be compiled individually.
\item
The packages \href{http://ctan.org/pkg/subdocs}{\textsf{subdocs}}
and \href{http://ctan.org/pkg/subfiles}{\textsf{subfiles}}
provide structures in which the main and child documents can be
encapsulated and allowing them to be compiled individually.
The inclusion mechanism is different from the conventional |\include|.
\item
The package \href{http://ctan.org/pkg/combine}{\textsf{combine}}
is an elaborate solution to combine several documents into one.
\end{itemize}
%
See also the CTAN topic \href{http://ctan.org/topic/subdocs}{\textsf{subdocs}}
for further related packages.
The present package differs from the above solutions in that
a document structure constructed with the conventional |\include| mechanism
just needs two extra commands at the top of every file
such that all constituent files can be compiled individually.

%%%%%%%%%%%%%%%%%%%%%%%%%%%%%%%%%%%%%%%%%%%%%%%%%%%%%%%%%%%%%%%%%%%%%%%%%%%%%%%%
%\subsection{Feature Suggestions}
%
%The following is a list of features which may be useful for future
%versions of this package:
%%
%\begin{itemize}
%\item
%\ldots
%\end{itemize}

%%%%%%%%%%%%%%%%%%%%%%%%%%%%%%%%%%%%%%%%%%%%%%%%%%%%%%%%%%%%%%%%%%%%%%%%%%%%%%%%
\subsection{Revision History}

%%%%%%%%%%%%%%%%%%%%%%%%%%%%%%%%%%%%%%%%
\paragraph{v2.0:} 2018/12/30

\begin{itemize}
\item
immediate forward processing
\item
added |\childdocby| mechanism
\item
manual restructured
\end{itemize}

%%%%%%%%%%%%%%%%%%%%%%%%%%%%%%%%%%%%%%%%
\paragraph{v1.6:} 2018/01/17

\begin{itemize}
\item
application for development of include files
\item
corrections to manual
\end{itemize}

%%%%%%%%%%%%%%%%%%%%%%%%%%%%%%%%%%%%%%%%
\paragraph{v1.5:} 2017/05/21

\begin{itemize}
\item
more complete structuring introduced
\item
|\childdocof| introduced
\item
|\childdoc| renamed to |\childdocmain|
\item
|\childredirect| renamed to |\childdocforward| and |\childdocforwardprefix|
and functionality expanded
\end{itemize}

%%%%%%%%%%%%%%%%%%%%%%%%%%%%%%%%%%%%%%%%
\paragraph{v1.0:} 2017/04/27

\begin{itemize}
\item
manual and install package
\item
first version published on CTAN
\end{itemize}

%%%%%%%%%%%%%%%%%%%%%%%%%%%%%%%%%%%%%%%%
\paragraph{v0.6:} 2017/04/26

\begin{itemize}
\item
redirection mechanism added
\end{itemize}

%%%%%%%%%%%%%%%%%%%%%%%%%%%%%%%%%%%%%%%%
\paragraph{v0.5:} 2017/04/26

\begin{itemize}
\item
functionality in definition file
\end{itemize}


%%%%%%%%%%%%%%%%%%%%%%%%%%%%%%%%%%%%%%%%%%%%%%%%%%%%%%%%%%%%%%%%%%%%%%%%%%%%%%%%
%%%%%%%%%%%%%%%%%%%%%%%%%%%%%%%%%%%%%%%%%%%%%%%%%%%%%%%%%%%%%%%%%%%%%%%%%%%%%%%%
%%%%%%%%%%%%%%%%%%%%%%%%%%%%%%%%%%%%%%%%%%%%%%%%%%%%%%%%%%%%%%%%%%%%%%%%%%%%%%%%
\appendix

\settowidth\MacroIndent{\rmfamily\scriptsize 000\ }

 \DocInput{childdoc.dtx}

\end{document}
%</driver>
% \fi
%
% %%%%%%%%%%%%%%%%%%%%%%%%%%%%%%%%%%%%%%%%%%%%%%%%%%%%%%%%%%%%%%%%%%%%%%%%%%%%%%
% %%%%%%%%%%%%%%%%%%%%%%%%%%%%%%%%%%%%%%%%%%%%%%%%%%%%%%%%%%%%%%%%%%%%%%%%%%%%%%
% \section{Sample}
%\iffalse
%<*samplemain>
%\fi
%
% The following presents a sample document
% with two chapters, two parts, a title page,
% a compile flag as well as three forwarding files to set the flag.
% It consists of eight |.tex| files:
% \begin{center}
% \begin{tabular}{ll}
% |cdocsamp.tex|&main file\\
% |cdocsch1.tex|&include file for chapter 1\\
% |cdocsch2.tex|&include file for chapter 2\\
% |cdocspt3.tex|&include file for part 3\\
% |cdocspt4.tex|&include file for part 4\\
% |cdocsdrf.tex|&forwarding file for main file in draft mode\\
% |cdocsfi1.tex|&forwarding file for final version of chapter 1\\
% |cdocsfi2.tex|&forwarding file for final version of chapter 2\\
% \end{tabular}
% \end{center}
% Each of the eight files can be compiled directly by the \LaTeX{} compiler.
%
% %%%%%%%%%%%%%%%%%%%%%%%%%%%%%%%%%%%%%%
% \paragraph{Main File.}
%
% The main file is called |cdocsamp.tex|.
%
% Load the \textsf{childdoc} definitions and
% declare the filename for the main document:
%    \begin{macrocode}
\input{childdoc.def}
\childdocmain{}
%    \end{macrocode}

% Optional override for |\version| flag:
%    \begin{macrocode}
%%\ifchilddoc\else\providecommand{\version}{draft}\fi
%    \end{macrocode}

% Define the default values for the |\version| flag
% (|final| for the main file and |draft| for childs):
%    \begin{macrocode}
\ifchilddoc
\providecommand{\version}{draft}
\else
\providecommand{\version}{final}
\fi
%    \end{macrocode}

% Load the standard document class:
%    \begin{macrocode}
\documentclass[12pt]{article}
%    \end{macrocode}

% Start the document body:
%    \begin{macrocode}
\begin{document}
%    \end{macrocode}

% Declare a title page.
% Print title, part of document being processed and version flag:
%    \begin{macrocode}
\addtocounter{page}{-1}
\begin{center}
{\LARGE\bfseries{}childdoc example\par}
\vspace{1cm}
\ifchilddoc
\ifchilddocmanual part\else chapter\fi:
`\childdocname' of `\childdocjob'\par
\else
main document: `\childdocjob'\par
\fi
version: \version\par
\end{center}
\newpage
%    \end{macrocode}

% Manually include selected file,
% otherwise process as usual:
%    \begin{macrocode}
\ifchilddocmanual
\section*{part `\childdocname'}
\input{\childdocname}
\else
%    \end{macrocode}

% Include the two chapters:
%    \begin{macrocode}
\include{cdocsch1}
\include{cdocsch2}
%    \end{macrocode}

% Include the two parts unless only chapters should be displayed:
%    \begin{macrocode}
\ifchilddoc\else
\section{part three}
\input{cdocspt3}
\section{part four}
\input{cdocspt4}
\fi
%    \end{macrocode}

% Process as usual until here:
%    \begin{macrocode}
\fi
%    \end{macrocode}

% End of document body:
%    \begin{macrocode}
\end{document}
%    \end{macrocode}
%\iffalse
%</samplemain>
%\fi
%
% %%%%%%%%%%%%%%%%%%%%%%%%%%%%%%%%%%%%%%
% \paragraph{Chapter Include Files.}
%
% The include files are called |cdocsch1.tex| and |cdocsch2.tex|.
%
%\iffalse
%<*samplechap1|samplechap2>
%\fi

% Optional override for |\version| flag:
%    \begin{macrocode}
%%\providecommand{\version}{final}
%    \end{macrocode}

% Include the main document:
%    \begin{macrocode}
\input{childdoc.def}
\childdocof{cdocsamp}
%    \end{macrocode}

%\iffalse
%</samplechap1|samplechap2>
%\fi
%
%\iffalse
%<*samplechap1>
%\fi
% Some text for chapter 1:
%    \begin{macrocode}
\section{one}
some text in chapter one
%    \end{macrocode}

%\iffalse
%</samplechap1>
%\fi
% Some text for chapter 2:
%\iffalse
%<*samplechap2>
%\fi
%    \begin{macrocode}
\section{two}
more text in chapter two
%    \end{macrocode}

%\iffalse
%</samplechap2>
%\fi
%
% %%%%%%%%%%%%%%%%%%%%%%%%%%%%%%%%%%%%%%
% \paragraph{Part Include Files.}
%
% The include files are called |cdocspt3.tex| and |cdocspt4.tex|.
%
%\iffalse
%<*samplepart3|samplepart4>
%\fi

% Optional override for |\version| flag:
%    \begin{macrocode}
%%\providecommand{\version}{final}
%    \end{macrocode}

% Include the main document:
%    \begin{macrocode}
\input{childdoc.def}
\childdocby{cdocsamp}
%    \end{macrocode}

%\iffalse
%</samplepart3|samplepart4>
%\fi
%
%\iffalse
%<*samplepart3>
%\fi
% Some text for part 3:
%    \begin{macrocode}
some text in part three
%    \end{macrocode}

%\iffalse
%</samplepart3>
%\fi
% Some text for part 4:
%\iffalse
%<*samplepart4>
%\fi
%    \begin{macrocode}
more text in part four
%    \end{macrocode}

%\iffalse
%</samplepart4>
%\fi
%
% %%%%%%%%%%%%%%%%%%%%%%%%%%%%%%%%%%%%%%
% \paragraph{Forwarding for a Complete Draft.}
%
% The following forwarding file |cdocsdrf.tex|
% compiles the main document in draft mode:
%\iffalse
%<*sampledraft>
%\fi
%    \begin{macrocode}
\def\version{draft}
\input{childdoc.def}
\childdocforward{cdocsamp}
%    \end{macrocode}

%\iffalse
%</sampledraft>
%\fi
%
% %%%%%%%%%%%%%%%%%%%%%%%%%%%%%%%%%%%%%%
% \paragraph{Forwarding for Final Version of the Chapters.}
%
% The following forwarding files |cdocsfn1.tex| and |cdocsfn2.tex|
% (with identical content)
% compile the final versions of the child documents
% |cdocsch1.tex| and |cdocsch2.tex|, respectively:
%\iffalse
%<*samplefinal>
%\fi
%    \begin{macrocode}
\def\version{final}
\input{childdoc.def}
\childdocforwardprefix[cdocsamp]{cdocsfn}{cdocsch}
%    \end{macrocode}

%\iffalse
%</samplefinal>
%\fi
%
% %%%%%%%%%%%%%%%%%%%%%%%%%%%%%%%%%%%%%%
% \paragraph{Command Line Processing.}
%
% The following three command lines generate the output files
% |cdocscld|, |cdocscl1| and |cdocscl2|
% which should be identical to
% |cdocsdrf|, |cdocsch1| and |cdocsfn2|, respectively:
% \begin{center}
% \begin{tabular}{l}
% |latex -jobname cdocscld \|\\
% |  "\def\version{draft}\input{childdoc.def}\childdocforward{cdocsamp}"|\\
% |latex -jobname cdocscl1 \|\\
% |  "\input{childdoc.def}\childdocforward[cdocsamp]{cdocsch1}"|\\
% |latex -jobname cdocscl2 \|\\
% |  "\def\version{final}\input{childdoc.def}\childdocforward{cdocsch2}"|
% \end{tabular}
% \end{center}
% Note that the trailing backslash on each first line
% merely continues the input to the second line
% (for convenient cut ant paste).
% Furthermore, the command |latex| can be replaced by any
% of its alternative versions such as |pdflatex|.
%
% %%%%%%%%%%%%%%%%%%%%%%%%%%%%%%%%%%%%%%%%%%%%%%%%%%%%%%%%%%%%%%%%%%%%%%%%%%%%%%
% %%%%%%%%%%%%%%%%%%%%%%%%%%%%%%%%%%%%%%%%%%%%%%%%%%%%%%%%%%%%%%%%%%%%%%%%%%%%%%
% \section{Implementation}
%\iffalse
%<*package>
%\fi
%
% This section describes the definitions file |childdoc.def|.

% The definitions cannot be loaded using |\usepackage| or |\RequirePackage|
% which has a mechanism to prevent loading a style file more than once.
% When loading the definitions by means of |\input|
% multiple instances have to be prevented manually:
%\iffalse
%This code needs to be before the `\ProvidesFile' directive
%which is defined at the beginning of this file.
%Therefore it is also placed there and commented out here.
%</package>
%<*discard>
%\fi
%    \begin{macrocode}
\ifdefined\childdocmain\endinput\fi
%    \end{macrocode}
%\iffalse
%</discard>
%<*package>
%\fi
%
% \macro{\ifchilddoc}
% \macro{\ifchilddocmanual}
% The conditional |\ifchilddoc| tells whether a
% child (true) or main (false) document is being compiled.
% The conditional |\ifchilddocmanual| tells whether
% the |\includeonly| mechanism is used (false) or
% the selection of child files must be performed manually (true).
% The definitions initialise to false:
%    \begin{macrocode}
\newif\ifchilddoc
\newif\ifchilddocmanual
%    \end{macrocode}

% \macro{\childdocname}
% \macro{\childdocjob}
% The macro |\childdocname| stores the name of the main document
% to be compiled. The macro |\childdocjob| stores the name of
% the document on which the \LaTeX{} compiler was originally invoked.
% The content of |\jobname| cannot be compared
% to filenames specified in the source due to different catcodes.
% The following code rescans |\jobname|, stores the result
% in |\childdocname| and saves a copy in |\childdocjob|:
%    \begin{macrocode}
\edef\childdocname{\scantokens\expandafter{\jobname\noexpand}}
\let\childdocjob\childdocname
%    \end{macrocode}

% \macro{\childdocdisable}
% The macro |\childdocdisable| prevents the main file
% from being processed more than once.
% At this stage, the main document command |\childdocmain|
% is assumed to be called once again where it should do nothing.
% Any subsequent call to it should prevent
% a secondary processing of the main document
% It overwrites the forwarding commands
% |\childdocof| and |\childdocforward|
% with empty macros to prevent further inclusions of the main document:
%    \begin{macrocode}
\newcommand{\childdocdisable}
{
  \renewcommand{\childdocmain}[1]{\renewcommand{\childdocmain}[1]{\endinput}}
  \renewcommand{\childdocof}[1]{}
  \renewcommand{\childdocby}[2][]{}
  \renewcommand{\childdocforward}[2][]{}
  \renewcommand{\childdocdisable}{}
}
%    \end{macrocode}

% \macro{\childdocmain}
% The macro |\childdocmain| is to be called at the top of the main file
% with nothing or the main filename (without extension) as argument.
% First, it breaks loops.
% If the argument is not empty and does not match |\childdocname|
% (which is set by the first inclusion of |childdoc.def|),
% |\ifchilddoc| is set to true, |\includeonly| is applied to the child file
% and |\jobname| is set to the main file
% (for proper handling of |.aux| files):
%    \begin{macrocode}
\newcommand{\childdocmain}[1]
{
  \childdocdisable\childdocmain{}
  \if?#1?\else
    \begingroup
      \def\childdoctmp{#1}
      \ifx\childdoctmp\childdocname
        \def\childdoctmp{}
      \else
        \def\childdoctmp
        {
          \childdoctrue
          \includeonly{\childdocname}
          \def\childdocjob{#1}
          \def\jobname{#1}
        }
      \fi
      \expandafter
    \endgroup
    \childdoctmp
  \fi
}
%    \end{macrocode}

% \macro{\childdocof}
% The command |\childdocof| redirects
% compilation to the main file |#1|.
%    \begin{macrocode}
\newcommand{\childdocof}[1]
{
  \childdocdisable
  \childdoctrue
  \includeonly{\childdocname}
  \def\jobname{#1}
  \def\childdocjob{#1}
  \input{#1}
}
%    \end{macrocode}

% \macro{\childdocby}
% The command |\childdocby| ....
%    \begin{macrocode}
\newcommand{\childdocby}[2][]
{
  \childdocdisable
  \childdoctrue
  \childdocmanualtrue
  \if?#1?\else
    \def\jobname{#2}
  \fi
  \def\childdocjob{#2}
  \input{#2}
  \endinput
}
%    \end{macrocode}

% \macro{\childdocforward}
% The command |\childdocforward| redirects
% compilation to the main file or
% (if the optional argument is given) a child file.
% Parameters are set as if the main file
% or a child file starting with |\childdocof| was compiled.
% Then compilation is handed over to the main file:
%    \begin{macrocode}
\newcommand{\childdocforward}[2][]
{
  \begingroup
    \if?#1?
      \def\childdoctmp
      {
        \def\childdocname{#2}
        \def\childdocjob{#2}
        \def\jobname{#2}
        \input{#2}
        \endinput
      }
    \else
      \def\childdoctmp
      {
        \childdocdisable
        \def\childdocname{#2}
        \childdoctrue
        \includeonly{#2}
        \def\childdocjob{#1}
        \def\jobname{#1}
        \input{#1}
        \endinput
      }
    \fi
    \expandafter
  \endgroup
  \childdoctmp
}
%    \end{macrocode}

% \macro{\childdocforwardprefix}
% The command |\childdocforwardprefix| redirects
% compilation to the main or a child file by means of a pattern.
% The prefix |#1| in the current filename is replaced by |#2|
% and the suffix of the current filename is kept
% (it is assumed that the filename does not contain the substring `|~~~|'
% which is used as a delimiter).
% Compilation is handed over to the new file by |\childdocforward|:
%    \begin{macrocode}
\newcommand{\childdocforwardprefix}[3][]
{
  \begingroup
    \def\childdocextract #2##1~~~{\def\childdoctmp{\childdocforward[#1]{#3##1}}}
    \expandafter\childdocextract\childdocname~~~
    \expandafter
  \endgroup
  \childdoctmp
}
%    \end{macrocode}

% \macro{\childdoc}
% The deprecated macro |\childdoc| is a legacy version of |\childdocmain|:
%    \begin{macrocode}
\newcommand{\childdoc}{\childdocmain}
%    \end{macrocode}

% \macro{\childdocredirect}
% The deprecated macro |\childdocredirect| is a legacy version
% of |\childdocforward| and |\childdocforwardprefix|:
%    \begin{macrocode}
\newcommand{\childdocredirect}[2][]
{
  \begingroup
    \if?#1?
      \def\childdoctmp{\childdocforward{#2}}
    \else
      \def\childdoctmp{\childdocforwardprefix{#1}{#2}}
    \fi
    \expandafter
  \endgroup
  \childdoctmp
}
%    \end{macrocode}

%\iffalse
%</package>
%\fi
%
\endinput

\childdocforward{cdocsamp}
%    \end{macrocode}

%\iffalse
%</sampledraft>
%\fi
%
% %%%%%%%%%%%%%%%%%%%%%%%%%%%%%%%%%%%%%%
% \paragraph{Forwarding for Final Version of the Chapters.}
%
% The following forwarding files |cdocsfn1.tex| and |cdocsfn2.tex|
% (with identical content)
% compile the final versions of the child documents
% |cdocsch1.tex| and |cdocsch2.tex|, respectively:
%\iffalse
%<*samplefinal>
%\fi
%    \begin{macrocode}
\def\version{final}
% \iffalse
%
% childdoc.dtx Copyright (C) 2017-2018 Niklas Beisert
%
% This work may be distributed and/or modified under the
% conditions of the LaTeX Project Public License, either version 1.3
% of this license or (at your option) any later version.
% The latest version of this license is in
%   http://www.latex-project.org/lppl.txt
% and version 1.3 or later is part of all distributions of LaTeX
% version 2005/12/01 or later.
%
% This work has the LPPL maintenance status `maintained'.
%
% The Current Maintainer of this work is Niklas Beisert.
%
% This work consists of the files childdoc.dtx and childdoc.ins
% and the derived files childdoc.def and cdocsamp.tex with
% cdocsch1.tex, cdocsch2.tex, cdocsdrf.tex, cdocsfn1.tex, cdocsfn2.tex.
%
%<package>\ifdefined\childdocmain\endinput\fi
%<package>\ProvidesFile{childdoc.def}[2018/12/30 v2.0 child document driver]
%<samplemain>\ProvidesFile{cdocsamp.tex}[2018/12/30 v2.0 sample for childdoc]
%<*driver>
%\ProvidesFile{childdoc.drv}[2018/12/30 v2.0 childdoc reference manual file]
\PassOptionsToClass{10pt,a4paper}{article}
\documentclass{ltxdoc}

\usepackage[margin=35mm]{geometry}
\usepackage{hyperref}
\usepackage{hyperxmp}
\usepackage[usenames]{color}

\hypersetup{colorlinks=true}
\hypersetup{pdfstartview=FitH}
\hypersetup{pdfpagemode=UseNone}
\hypersetup{pdfsource={}}
\hypersetup{pdflang={en-UK}}
\hypersetup{pdfcopyright={Copyright 2017-2018 Niklas Beisert.
  This work may be distributed and/or modified under the
  conditions of the LaTeX Project Public License, either version 1.3
  of this license or (at your option) any later version.}}
\hypersetup{pdflicenseurl={http://www.latex-project.org/lppl.txt}}
\hypersetup{pdfcontactaddress={ETH Zurich, ITP, HIT K,
  Wolfgang-Pauli-Strasse 27}}
\hypersetup{pdfcontactpostcode={8093}}
\hypersetup{pdfcontactcity={Zurich}}
\hypersetup{pdfcontactcountry={Switzerland}}
\hypersetup{pdfcontactemail={nbeisert@itp.phys.ethz.ch}}
\hypersetup{pdfcontacturl={http://people.phys.ethz.ch/\xmptilde nbeisert/}}

\newcommand{\secref}[1]{\hyperref[#1]{section \ref*{#1}}}

\parskip1ex
\parindent0pt
\let\olditemize\itemize
\def\itemize{\olditemize\parskip0pt}

\begin{document}

\title{The \textsf{childdoc} Package}
\hypersetup{pdftitle={The childdoc Package}}
\author{Niklas Beisert\\[2ex]
  Institut f\"ur Theoretische Physik\\
  Eidgen\"ossische Technische Hochschule Z\"urich\\
  Wolfgang-Pauli-Strasse 27, 8093 Z\"urich, Switzerland\\[1ex]
  \href{mailto:nbeisert@itp.phys.ethz.ch}
  {\texttt{nbeisert@itp.phys.ethz.ch}}}
\hypersetup{pdfauthor={Niklas Beisert}}
\hypersetup{pdfsubject={Manual for the LaTeX2e Package childdoc}}
\date{30 December 2018, \textsf{v2.0}}
\maketitle

\begin{abstract}\noindent
\textsf{childdoc} is a \LaTeXe{} package
that enables the direct compilation
of document sections included by |\include|
to individual files.
\end{abstract}

\begingroup
\parskip0ex
\tableofcontents
\endgroup

%%%%%%%%%%%%%%%%%%%%%%%%%%%%%%%%%%%%%%%%%%%%%%%%%%%%%%%%%%%%%%%%%%%%%%%%%%%%%%%%
%%%%%%%%%%%%%%%%%%%%%%%%%%%%%%%%%%%%%%%%%%%%%%%%%%%%%%%%%%%%%%%%%%%%%%%%%%%%%%%%
\section{Introduction}

\LaTeX{} provides a mechanism to structure a large document (such as a book)
into a main file and several child files (containing the chapters)
using the |\include| command.
This mechanism is beneficial for documents
which span hundreds of pages in order to
make the source file(s) more manageable.
Moreover, compilation can be restricted to
selected child files by means of the |\includeonly| command.
The latter feature can be used to reduce the compilation time while editing
(this was significantly more useful in the earlier days of \LaTeX{})
or to generate a smaller document which is easier to navigate.
Another application of |\includeonly| is to generate
documents consisting of selected parts of the complete document.

However, there are a few drawbacks of the plain |\include| mechanism:
\begin{itemize}
\item
The child files cannot be compiled on their own,
they can only be compiled via the main file.
A naive editing environment
(such as a text editor with an option
to have the current file processed by \LaTeX)
may require one to switch to the main file before compiling;
attempting to compile the child file produces errors.
\item
The main file must be modified (each time)
to adjust the |\includeonly| command
to the present needs. This easily leaves the main file in a messy state.
\item
The generated document will always carry the filename
of the main document. This is inconvenient if
several child files are to be compiled and
to be kept for distribution.
\end{itemize}

The present package provides a simple interface
to make child files individually compilable by \LaTeX{}.
Compiling a child file then has the same effect as compiling
the main file with an |\includeonly| command
to select the appropriate child.
Moreover the generated document will carry the name of the child
rather than the main file.
This resolves all three above issues.

This feature is meant to make the editing of books,
thesis documents and lecture notes somewhat more convenient.
However, the package can also be used efficiently for
composing a series of documents (such as exercise sheets)
which are typically distributed individually.
It then assists the author in generating the individual documents
(potentially in different versions)
as well as a document containing the collected series.
Another application is in developing style files
or other kinds of included material
where compilation of the style file could redirect
to a sample or test file.

%%%%%%%%%%%%%%%%%%%%%%%%%%%%%%%%%%%%%%%%%%%%%%%%%%%%%%%%%%%%%%%%%%%%%%%%%%%%%%%%
%%%%%%%%%%%%%%%%%%%%%%%%%%%%%%%%%%%%%%%%%%%%%%%%%%%%%%%%%%%%%%%%%%%%%%%%%%%%%%%%
\section{Usage}

First of all, the package \textsf{childdoc} is \emph{not} a standard
\LaTeXe{} |.sty| style file! Therefore it needs to be invoked in
a non-standard way.

%%%%%%%%%%%%%%%%%%%%%%%%%%%%%%%%%%%%%%%%%%%%%%%%%%%%%%%%%%%%%%%%%%%%%%%%%%%%%%%%
\subsection{Included Files}
\label{sec:include}

%%%%%%%%%%%%%%%%%%%%%%%%%%%%%%%%%%%%%%%%
\DescribeMacro{\childdocmain}
To use the package, add the commands
\begin{center}
\begin{tabular}{l}
|\input{childdoc.def}|\\
|\childdocmain{}|\\
\end{tabular}
\end{center}
at the very top of the main \LaTeX{} file,
in particular \emph{before} the |\documentclass| statement!
The argument of |\childdocmain| should be left empty
(but it must be present).

%%%%%%%%%%%%%%%%%%%%%%%%%%%%%%%%%%%%%%%%
\DescribeMacro{\childdocof}
Furthermore, add the commands
\begin{center}
\begin{tabular}{l}
|\input{childdoc.def}|\\
|\childdocof{|\textit{main}|}|\\
\end{tabular}
\end{center}
at the top of every child file \textit{child}
which is included by |\include{|\textit{child}|}|
from within the main file
(or at least for those files to be compiled individually).
The argument \textit{main} must be the filename of the main file.

There are a couple of
considerations in setting up the main and child documents:

%%%%%%%%%%%%%%%%%%%%%%%%%%%%%%%%%%%%%%%%
\paragraph{Restrictions.}

Please note the following restrictions:
\begin{itemize}
\item
|\childdocmain| must be called with one argument \textit{main}
to ensure compatibility with earlier version of the package.
It must either be empty (|\childdocmain{}|)
or precisely match the filename of the main file in which it is specified.
See \secref{sec:detection} for further information.
\item
The filename \textit{main} must be specified without the |.tex| extension.
\item
The filename \textit{main} is case sensitive
(even in case-insensitive file systems)
due to internal string comparison.
\item
The argument \textit{main} should be fully expanded, it cannot be a macro.
\item
Subdirectories and special characters should be avoided in filenames.
\item
The command |\childdocmain{|\textit{main}|}| must be followed by a whitespace.
It should not be followed immediately by another command
or by a comment mark `|%|'.
This is because the \TeX{} parser reads the token immediately following
the argument of |\childdocmain| and puts it
at the beginning of every child section;
however, a white\-space is ignored.
\end{itemize}

%%%%%%%%%%%%%%%%%%%%%%%%%%%%%%%%%%%%%%%%
\paragraph{Content of Main File.}

It is advisable to place all content in the child files included by |\include|.
Any output contained in the main file will appear in all child documents
unless suppressed manually;
it cannot be suppressed automatically by the |\includeonly| directive
and thus should normally be avoided.
A method to include some content in the main file
by means of conditional processing is described in \secref{sec:conditional}.

%%%%%%%%%%%%%%%%%%%%%%%%%%%%%%%%%%%%%%%%
\paragraph{Page Numbering.}

When only a part of the document is compiled,
the appropriate numbering of pages
(as well as other status parameters)
is determined from the |.aux| files.
The latter contain information from previous passes.
However this information needs to propagate through
all intermediate child documents.
Therefore the page numbering in child documents may well
be inconsistent until the complete document is compiled at least once.

A useful (if unconventional) way to always ensure a consistent
page numbering is to restart the numbering in each child document
and denote the pages by `\textit{child}|.|\textit{page}'
where \textit{child} represents the chapter/section number of the child file.
This can be achieved by the command
|\numberwithin{page}{|\textit{child}|}|
of the \textsf{amsmath} package
where \textit{child} can be |chapter| or |section|
depending on the chosen structuring.
Alternatively, one can modify the macro |\thepage| appropriately
and reset the counter |page| at the start of each child file.

%%%%%%%%%%%%%%%%%%%%%%%%%%%%%%%%%%%%%%%%%%%%%%%%%%%%%%%%%%%%%%%%%%%%%%%%%%%%%%%%
\subsection{Conditional Processing}
\label{sec:conditional}

The package provides a mechanism to compile different versions
of a document. To customise the versions further some conditional processing
can come in handy to distinguish which version is being compiled.
The package provides two macros to describe the compilation context:

%%%%%%%%%%%%%%%%%%%%%%%%%%%%%%%%%%%%%%%%
\DescribeMacro{\ifchilddoc}
The conditional |\ifchilddoc| distinguishes between the compilation of
child documents and the main document:
%
\begin{center}
|\ifchilddoc |\textit{child-code}| |[|\||else |\textit{main-code}]| \||fi|
\end{center}

%%%%%%%%%%%%%%%%%%%%%%%%%%%%%%%%%%%%%%%%
\DescribeMacro{\childdocname}
\DescribeMacro{\childdocjob}
The macro |\childdocname| contains the filename (without extension)
of the main or child file being processed.
Note that |\childdocjob| will always contain the name of the main file.

%%%%%%%%%%%%%%%%%%%%%%%%%%%%%%%%%%%%%%%%
\paragraph{Title Page.}

Conditional processing can be used to include a title or banner page
in the main document when proper precautions are taken.
Importantly, the code in the main file should ensure that the page counter
(as well as other status parameters which are stored in the |.aux| files)
takes the same value after the conditional processing.
Otherwise the page numbers may take divergent values
depending on which part is compiled.

For example, a title page could be declared by:
%
\begin{center}
\begin{tabular}{l}
|\ifchilddoc\||else|\\
|\addtocounter{page}{-1}|\\
\textit{code for title page}\\
|\newpage|\\
|\||fi|
\end{tabular}
\end{center}
%
A banner page for the child documents can be generated by:
%
\begin{center}
\begin{tabular}{l}
|\ifchilddoc|\\
|\addtocounter{page}{-1}|\\
\textit{code for banner page}\\
|\newpage|\\
|\||fi|
\end{tabular}
\end{center}
%
Here one could write a message such as:
\begin{center}
|This is the part \childdocname{} of \childdocjob{}.|
\end{center}

%%%%%%%%%%%%%%%%%%%%%%%%%%%%%%%%%%%%%%%%%%%%%%%%%%%%%%%%%%%%%%%%%%%%%%%%%%%%%%%%
\subsection{Flags}
\label{sec:flags}

The package makes it easy to generate different versions
of the main or child documents.
To this end compilation flags can be defined
and assigned different default values.
They will be particularly useful in conjunction
with the forwarding mechanism described in \secref{sec:forward}.

For example, it may be useful to have a flag |\version|
which can be set to |draft| or |final|.
The document source will contain some conditional code
depending on the value of |\version|.
Suppose further, the flag should default to |final| for the main file
and to |draft| for child files
which is a natural assignment for editing the document.
This is achieved by placing the following code
in the preamble of the main document
(below the |\childdocmain| directive):
%
\begin{center}
\begin{tabular}{l}
|\ifchilddoc|\\
|\providecommand{\version}{draft}|\\
|\||else|\\
|\providecommand{\version}{final}|\\
|\||fi|
\end{tabular}
\end{center}
%
The definition by |\providecommand| makes sure
that previous definitions are not overwritten.
Further statements |\providecommand{\version}{...}|
can thus be added before the above code to override it.

For the main file, one might add a line
(between |\childdocmain| and the above block)
%
\begin{center}
|%\ifchilddoc\||else\providecommand{\version}{draft}\||fi|
\end{center}
%
which can be uncommented to produce a draft version.
Likewise one can add a line to the very top of a child file
(above the |\childdocof{|\textit{main}|}| directive)
%
\begin{center}
|%\providecommand{\version}{final}|
\end{center}
%
which can be uncommented to produce the final version of this child document.

%%%%%%%%%%%%%%%%%%%%%%%%%%%%%%%%%%%%%%%%%%%%%%%%%%%%%%%%%%%%%%%%%%%%%%%%%%%%%%%%
\subsection{Forwarding}
\label{sec:forward}

Different versions of the main or child documents
using compilation flags as described in \secref{sec:flags}
can be (permanently) stored in different files
for convenient compilation, viewing and distribution.
To this end, the package defines a command
to pass on compilation to a different file:

%%%%%%%%%%%%%%%%%%%%%%%%%%%%%%%%%%%%%%%%
\DescribeMacro{\childdocforward}
The command |\childdocforward| redirects processing to
another source file:
%
\begin{center}
\begin{tabular}{l}
|\input{childdoc.def}|\\
|\childdocforward[|\textit{main}|]{|\textit{dest}|}|\\
\end{tabular}
\end{center}
%
The argument \textit{dest} is the destination file
(without extension).
It should be the main file or one of the child files.
Note that further \textsf{childdoc} directives
such as |\childdocof| and |\childdocforward|
in the indicated file will be processed in this form.
The optional argument \textit{main}
passes on directly to the main file \textit{main}
while pretending to compile the child \textit{dest}.
This form behaves as if \textit{dest}
issues |\childdocof{|\textit{main}|}| right away,
and no further \textsf{childdoc} directives will be processed.

%%%%%%%%%%%%%%%%%%%%%%%%%%%%%%%%%%%%%%%%
\DescribeMacro{\...prefix}
In the alternative form |\childdocforwardprefix|,
%
\begin{center}
\begin{tabular}{l}
|\input{childdoc.def}|\\
|\childdocforwardprefix[|\textit{main}|]{|\textit{prefix}|}{|\textit{dest}|}|
\end{tabular}
\end{center}
%
the destination file is determined by a pattern
depending on the current file:
To make this work, the current file must be called
`{\textit{prefix}\hspace{0.2em}\textit{suffix}}'
with \textit{prefix} matching precisely the argument.
Processing is then passed on to the file
`{\textit{dest}\hspace{0.2em}\textit{suffix}}'.
Surely, the same effect is achieved by
directly specifying the
argument `{\textit{dest}\hspace{0.2em}\textit{suffix}}'
in the first form.
However, that requires to set up a different file
for each child. With the alternative form of the command
all these files can have exactly the same content
which simplifies setting them up and maintaining them.

For example, the following file |draft.tex|
with a compilation flag |\version| as described in \secref{sec:flags}
compiles the main document as a draft:
%
\begin{center}
\begin{tabular}{l}
|\def\version{draft}|\\
|\input{childdoc.def}|\\
|\childdocforward{|\textit{main}|}|
\end{tabular}
\end{center}
%
Likewise, the following files |final|\textit{nn}|.tex|
compile the final version of the child document
|child|\textit{nn}|.tex|:
%
\begin{center}
\begin{tabular}{l}
|\def\version{final}|\\
|\input{childdoc.def}|\\
|\childdocforwardprefix{final}{child}|
\end{tabular}
\end{center}
%

Note that when several versions of a main file and/or of each child file
are to be generated, it may be convenient to set up a |Makefile| or
shell script to automatise the process.

%%%%%%%%%%%%%%%%%%%%%%%%%%%%%%%%%%%%%%%%%%%%%%%%%%%%%%%%%%%%%%%%%%%%%%%%%%%%%%%%
\subsection{Command Line Processing}
\label{sec:commandline}

The effect of redirection files can also be achieved by invoking
the \LaTeX{} compiler with a more elaborate command line.
Most conveniently this should be done as part
of a shell script or a |Makefile|.

When using \textsf{childdoc} in the main file, the following
command lines effectively perform a redirection
(note that depending on the shell being used,
backslashes may have to be doubled: `|\|' $\to$ `|\\|'):
%
\begin{center}
|... -jobname "|\textit{target}|" |\\|"|[\textit{flags}]%
|\input{childdoc.def}\childdocforward[|\textit{main}|]{|\textit{dest}|}"|
\end{center}
%
Here \textit{target} is the name of the output file,
\textit{main} is the name of the main file
and \textit{dest} is the name of the main or child file to be processed
(all filenames without extensions).
The optional argument \textit{main} can be omitted
if \textit{main} matches \textit{dest}.
Optionally, compilation \textit{flags} can be defined via |\def| commands.
This command line makes the \TeX{} engine believe
it is compiling the file \textit{target}
whose content is specified as the latter parameter.
The provided code then forwards the processing to
\textit{main} or \textit{dest} as described in \secref{sec:forward}.

%%%%%%%%%%%%%%%%%%%%%%%%%%%%%%%%%%%%%%%%%%%%%%%%%%%%%%%%%%%%%%%%%%%%%%%%%%%%%%%%
\subsection{Include by Input}
\label{sec:input}

Including child documents by |\include| has some restrictions by design.
Most notably, the content of a child document always occupies
its own set of pages; pages cannot be shared between child documents.
Usually, this behaviour makes perfect sense
because each child document contain an essential part of the document.
However, in some situations it may be desirable to compose
a document from a collection of parts
without having mandatory page breaks between then.
For this case, the package
provides a mechanism to include parts
by |\input| which can also be processed individually.
However, by construction this mechanism
requires manual handling of the content to be output.

%%%%%%%%%%%%%%%%%%%%%%%%%%%%%%%%%%%%%%%%
\DescribeMacro{\ifchilddocmanual}
The main file should be prepared as usual, see \secref{sec:include}.
However, the document body must make a distinction
between processing of an individual part and of the main document, e.g.:
%
\begin{center}
\begin{tabular}{l}
|\ifchilddocmanual|\\
|\input{\childdocname}|\\
|\||else|\\
\textit{document body with }|\input{|\textit{part}|}|\\
|\||fi|
\end{tabular}
\end{center}
%
The conditional |\ifchilddocmanual| is true whenever
a part to be included by |\input| is being compiled,
and the name of the part is stored in |\childdocname|.

%%%%%%%%%%%%%%%%%%%%%%%%%%%%%%%%%%%%%%%%
\DescribeMacro{\childdocby}
Each part to be included by |\input| should start with:
%
\begin{center}
\begin{tabular}{l}
|\input{childdoc.def}|\\
|\childdocby{|\textit{main}|}|\\
\end{tabular}
\end{center}
%
The directive |\childdocby| is similar to |\childdocof|
described in \secref{sec:include},
but the subsequent selection of content must be done manually.
To that end, both |\ifchilddoc| and |\ifchilddocmanual|
will be true upon processing of a part,
and the name of the part is stored in |\childdocname|.
Note that |\jobname| will be set to the filename of the current part
so that each part receives an individual |.aux| file
that does not interfere with the |.aux| file(s) of the main document.
This behaviour can be altered by the alternative form
|\childdocby[*]{|\textit{main}|}| (with a non-empty optional argument)
which uses the |.aux| file of the main document
by setting |\jobname| to \textit{main}.

%%%%%%%%%%%%%%%%%%%%%%%%%%%%%%%%%%%%%%%%%%%%%%%%%%%%%%%%%%%%%%%%%%%%%%%%%%%%%%%%
\subsection{Driver Development}
\label{sec:driver}

The \textsf{childdoc} mechanism can also be use for the development
of definition files such as \LaTeX{} styles or classes.
This case differs from the above setup with multiple parts
included by |\include| in that no |\includeonly| should be invoked.
This can be achieved by starting the include file
(before |\ProvidesPackage|) with:
%
\begin{center}
\begin{tabular}{l}
|\input{childdoc.def}|\\
|\childdocforward{|\textit{main}|}|\\
\end{tabular}
\end{center}
%
or alternatively with:
%
\begin{center}
\begin{tabular}{l}
|\input{childdoc.def}|\\
|\childdocby{|\textit{main}|}|\\
\end{tabular}
\end{center}
%
Both forms have slightly different effects as described above.
The main file is prepared as usual, see \secref{sec:include}.

%%%%%%%%%%%%%%%%%%%%%%%%%%%%%%%%%%%%%%%%%%%%%%%%%%%%%%%%%%%%%%%%%%%%%%%%%%%%%%%%
\subsection{Legacy Detection}
\label{sec:detection}

The directive |\childdocmain| in the main file can detect
whether the complete document or merely a child is to be compiled
even without using the directive |\childdocof|.
This method is deprecated because it is less robust
and there is no compelling reason to use it;
it is merely provided for backward compatibility
and it may be removed in future versions.

If the detection mechanism is to be used,
it is mandatory to correctly specify
the filename of the main file as the argument of |\childdocmain|:
%
\begin{center}
\begin{tabular}{l}
|\input{childdoc.def}|\\
|\childdocmain{|\textit{main}|}|\\
\end{tabular}
\end{center}
%
If |\jobname| does not match the argument \textit{main} of |\childdocmain|,
it is assumed that |\jobname| points to the child file to be compiled.
When using |\childdocmain| with the main file specified as argument,
it suffices to start a child file
with just |\input{|\textit{main}|}|
without loading of the package and using |\childdocof|.
If instead all processing is done
with the appropriate \textsf{childdoc} directives,
the argument of \textit{main} of |\childdocmain| can be empty.

An alternative version of the command line processing described
in \secref{sec:commandline} using the detection mechanism reads:
%
\begin{center}
|... -jobname "|\textit{target}|" "|[\textit{flags}]%
[|\def\jobname{|\textit{dest}|}|]|\input{|\textit{main}|}"|
\end{center}

%%%%%%%%%%%%%%%%%%%%%%%%%%%%%%%%%%%%%%%%%%%%%%%%%%%%%%%%%%%%%%%%%%%%%%%%%%%%%%%%
\subsection{Manual Code}
\label{sec:manual}

In case one cannot be certain whether the definitions file |childdoc.def|
is installed on the target \TeX{} distribution
and one prefers not to ship it,
it is conceivable to paste a few relevant commands into the sources.

To that end, drop all statements |\input{childdoc.def}|
and perform the replacements as outlined below.
Instead of |\childdocmain{|\textit{main}|}| add the following code
to the top of the main file:
%
\begin{center}
\begin{tabular}{l}
|\||ifdefined\childdocname\endinput\||fi\newif\ifchilddoc|\\
|\edef\childdocname{\scantokens\expandafter{\jobname\noexpand}}|\\
|\def\childdocmain{|\textit{main}|}\||ifx\childdocmain\childdocname\||else|\\
|\childdoctrue\includeonly{\childdocname}\let\jobname\childdocmain\||fi|\\
\end{tabular}
\end{center}
%
Instead of |\childdocof{|\textit{main}|}| just include the main file
at the top of each child file:
%
\begin{center}
|\input{|\textit{main}|}|
\end{center}
%
A simple redirection |\childdocforward{|\textit{dest}|}| is achieved by:
%
\begin{center}
|\def\jobname{|\textit{dest}|}\input{\jobname}|
\end{center}
%
The redirection with prefix
|\childdocforwardprefix[|\textit{prefix}|]{|\textit{dest}|}|
is accomplished by:
%
\begin{center}
\begin{tabular}{l}
|{\edef\jobname{\scantokens\expandafter{\jobname\noexpand}}|\\
|\def\redirectjob |\textit{prefix}|#1~~~{\gdef\jobname{|\textit{dest}|#1}}|\\
|\expandafter\redirectjob\jobname~~~}\input{\jobname}|
\end{tabular}
\end{center}

In an alternative approach,
child documents can be compiled by a specific command line
without additional code or specific definitions:
%
\begin{center}
|... -jobname "|\textit{target}|" "|[\textit{flags}]%
|\includeonly{|\textit{dest}|}\input{|\textit{main}|}"|
\end{center}
%

%%%%%%%%%%%%%%%%%%%%%%%%%%%%%%%%%%%%%%%%%%%%%%%%%%%%%%%%%%%%%%%%%%%%%%%%%%%%%%%%
%%%%%%%%%%%%%%%%%%%%%%%%%%%%%%%%%%%%%%%%%%%%%%%%%%%%%%%%%%%%%%%%%%%%%%%%%%%%%%%%
\section{Information}

%%%%%%%%%%%%%%%%%%%%%%%%%%%%%%%%%%%%%%%%%%%%%%%%%%%%%%%%%%%%%%%%%%%%%%%%%%%%%%%%
\subsection{Copyright}

Copyright \copyright{} 2017--2018 Niklas Beisert

This work may be distributed and/or modified under the
conditions of the \LaTeX{} Project Public License, either version 1.3
of this license or (at your option) any later version.
The latest version of this license is in
  \url{http://www.latex-project.org/lppl.txt}
and version 1.3 or later is part of all distributions of \LaTeX{}
version 2005/12/01 or later.

This work has the LPPL maintenance status `maintained'.

The Current Maintainer of this work is Niklas Beisert.

This work consists of the files |README.txt|, |childdoc.ins| and |childdoc.dtx|
as well as the derived files |childdoc.def|, |cdocsamp.tex|
with |cdocsch1.tex|, |cdocsch2.tex|, |cdocspt3.tex|, |cdocspt4.tex|,
|cdocsdrf.tex|, |cdocsfn1.tex|, |cdocsfn2.tex|
as well as |childdoc.pdf|.

%%%%%%%%%%%%%%%%%%%%%%%%%%%%%%%%%%%%%%%%%%%%%%%%%%%%%%%%%%%%%%%%%%%%%%%%%%%%%%%%
\subsection{Files and Installation}

The package consists of the files:
%
\begin{center}
\begin{tabular}{ll}
    |README.txt|   & readme file \\
    |childdoc.ins| & installation file \\
    |childdoc.dtx| & source file \\
    |childdoc.def| & definition file \\
    |cdocsamp.tex| & sample main file \\
    |cdocsch1.tex| & sample include file \\
    |cdocsch2.tex| & sample include file \\
    |cdocspt3.tex| & sample part file \\
    |cdocspt4.tex| & sample part file \\
    |cdocsdrf.tex| & sample redirection file \\
    |cdocsfn1.tex| & sample redirection file \\
    |cdocsfn2.tex| & sample redirection file \\
    |childdoc.pdf| & manual
\end{tabular}
\end{center}
%
The distribution consists of the files
|README.txt|, |childdoc.ins| and |childdoc.dtx|.
%
\begin{itemize}
\item
Run (pdf)\LaTeX{} on |childdoc.dtx|
to compile the manual |childdoc.pdf| (this file).
\item
Run \LaTeX{} on |childdoc.ins| to create the definitions file |childdoc.def|
and the sample |cdocsamp.tex| with include files
|cdocsch1.tex|, |cdocsch2.tex|, |cdocspt3.tex|, |cdocspt4.tex|,
|cdocsdrf.tex|, |cdocsfn1.tex|, |cdocsfn2.tex|.
Then copy the file |childdoc.def| to an appropriate directory of your \LaTeX{}
distribution, e.g.\ \textit{texmf-root}|/tex/latex/childdoc|.
\end{itemize}

%%%%%%%%%%%%%%%%%%%%%%%%%%%%%%%%%%%%%%%%%%%%%%%%%%%%%%%%%%%%%%%%%%%%%%%%%%%%%%%%
\subsection{Related CTAN Packages}

There are several other packages which offer a similar functionality:
%
\begin{itemize}
\item
The packages
\href{http://ctan.org/pkg/docmute}{\textsf{docmute}},
\href{http://ctan.org/pkg/includex}{\textsf{includex}} and
\href{http://ctan.org/pkg/standalone}{\textsf{standalone}}
provide commands to include only the document body of
a child file thus allowing both files to be compiled individually.
\item
The packages \href{http://ctan.org/pkg/subdocs}{\textsf{subdocs}}
and \href{http://ctan.org/pkg/subfiles}{\textsf{subfiles}}
provide structures in which the main and child documents can be
encapsulated and allowing them to be compiled individually.
The inclusion mechanism is different from the conventional |\include|.
\item
The package \href{http://ctan.org/pkg/combine}{\textsf{combine}}
is an elaborate solution to combine several documents into one.
\end{itemize}
%
See also the CTAN topic \href{http://ctan.org/topic/subdocs}{\textsf{subdocs}}
for further related packages.
The present package differs from the above solutions in that
a document structure constructed with the conventional |\include| mechanism
just needs two extra commands at the top of every file
such that all constituent files can be compiled individually.

%%%%%%%%%%%%%%%%%%%%%%%%%%%%%%%%%%%%%%%%%%%%%%%%%%%%%%%%%%%%%%%%%%%%%%%%%%%%%%%%
%\subsection{Feature Suggestions}
%
%The following is a list of features which may be useful for future
%versions of this package:
%%
%\begin{itemize}
%\item
%\ldots
%\end{itemize}

%%%%%%%%%%%%%%%%%%%%%%%%%%%%%%%%%%%%%%%%%%%%%%%%%%%%%%%%%%%%%%%%%%%%%%%%%%%%%%%%
\subsection{Revision History}

%%%%%%%%%%%%%%%%%%%%%%%%%%%%%%%%%%%%%%%%
\paragraph{v2.0:} 2018/12/30

\begin{itemize}
\item
immediate forward processing
\item
added |\childdocby| mechanism
\item
manual restructured
\end{itemize}

%%%%%%%%%%%%%%%%%%%%%%%%%%%%%%%%%%%%%%%%
\paragraph{v1.6:} 2018/01/17

\begin{itemize}
\item
application for development of include files
\item
corrections to manual
\end{itemize}

%%%%%%%%%%%%%%%%%%%%%%%%%%%%%%%%%%%%%%%%
\paragraph{v1.5:} 2017/05/21

\begin{itemize}
\item
more complete structuring introduced
\item
|\childdocof| introduced
\item
|\childdoc| renamed to |\childdocmain|
\item
|\childredirect| renamed to |\childdocforward| and |\childdocforwardprefix|
and functionality expanded
\end{itemize}

%%%%%%%%%%%%%%%%%%%%%%%%%%%%%%%%%%%%%%%%
\paragraph{v1.0:} 2017/04/27

\begin{itemize}
\item
manual and install package
\item
first version published on CTAN
\end{itemize}

%%%%%%%%%%%%%%%%%%%%%%%%%%%%%%%%%%%%%%%%
\paragraph{v0.6:} 2017/04/26

\begin{itemize}
\item
redirection mechanism added
\end{itemize}

%%%%%%%%%%%%%%%%%%%%%%%%%%%%%%%%%%%%%%%%
\paragraph{v0.5:} 2017/04/26

\begin{itemize}
\item
functionality in definition file
\end{itemize}


%%%%%%%%%%%%%%%%%%%%%%%%%%%%%%%%%%%%%%%%%%%%%%%%%%%%%%%%%%%%%%%%%%%%%%%%%%%%%%%%
%%%%%%%%%%%%%%%%%%%%%%%%%%%%%%%%%%%%%%%%%%%%%%%%%%%%%%%%%%%%%%%%%%%%%%%%%%%%%%%%
%%%%%%%%%%%%%%%%%%%%%%%%%%%%%%%%%%%%%%%%%%%%%%%%%%%%%%%%%%%%%%%%%%%%%%%%%%%%%%%%
\appendix

\settowidth\MacroIndent{\rmfamily\scriptsize 000\ }

 \DocInput{childdoc.dtx}

\end{document}
%</driver>
% \fi
%
% %%%%%%%%%%%%%%%%%%%%%%%%%%%%%%%%%%%%%%%%%%%%%%%%%%%%%%%%%%%%%%%%%%%%%%%%%%%%%%
% %%%%%%%%%%%%%%%%%%%%%%%%%%%%%%%%%%%%%%%%%%%%%%%%%%%%%%%%%%%%%%%%%%%%%%%%%%%%%%
% \section{Sample}
%\iffalse
%<*samplemain>
%\fi
%
% The following presents a sample document
% with two chapters, two parts, a title page,
% a compile flag as well as three forwarding files to set the flag.
% It consists of eight |.tex| files:
% \begin{center}
% \begin{tabular}{ll}
% |cdocsamp.tex|&main file\\
% |cdocsch1.tex|&include file for chapter 1\\
% |cdocsch2.tex|&include file for chapter 2\\
% |cdocspt3.tex|&include file for part 3\\
% |cdocspt4.tex|&include file for part 4\\
% |cdocsdrf.tex|&forwarding file for main file in draft mode\\
% |cdocsfi1.tex|&forwarding file for final version of chapter 1\\
% |cdocsfi2.tex|&forwarding file for final version of chapter 2\\
% \end{tabular}
% \end{center}
% Each of the eight files can be compiled directly by the \LaTeX{} compiler.
%
% %%%%%%%%%%%%%%%%%%%%%%%%%%%%%%%%%%%%%%
% \paragraph{Main File.}
%
% The main file is called |cdocsamp.tex|.
%
% Load the \textsf{childdoc} definitions and
% declare the filename for the main document:
%    \begin{macrocode}
\input{childdoc.def}
\childdocmain{}
%    \end{macrocode}

% Optional override for |\version| flag:
%    \begin{macrocode}
%%\ifchilddoc\else\providecommand{\version}{draft}\fi
%    \end{macrocode}

% Define the default values for the |\version| flag
% (|final| for the main file and |draft| for childs):
%    \begin{macrocode}
\ifchilddoc
\providecommand{\version}{draft}
\else
\providecommand{\version}{final}
\fi
%    \end{macrocode}

% Load the standard document class:
%    \begin{macrocode}
\documentclass[12pt]{article}
%    \end{macrocode}

% Start the document body:
%    \begin{macrocode}
\begin{document}
%    \end{macrocode}

% Declare a title page.
% Print title, part of document being processed and version flag:
%    \begin{macrocode}
\addtocounter{page}{-1}
\begin{center}
{\LARGE\bfseries{}childdoc example\par}
\vspace{1cm}
\ifchilddoc
\ifchilddocmanual part\else chapter\fi:
`\childdocname' of `\childdocjob'\par
\else
main document: `\childdocjob'\par
\fi
version: \version\par
\end{center}
\newpage
%    \end{macrocode}

% Manually include selected file,
% otherwise process as usual:
%    \begin{macrocode}
\ifchilddocmanual
\section*{part `\childdocname'}
\input{\childdocname}
\else
%    \end{macrocode}

% Include the two chapters:
%    \begin{macrocode}
\include{cdocsch1}
\include{cdocsch2}
%    \end{macrocode}

% Include the two parts unless only chapters should be displayed:
%    \begin{macrocode}
\ifchilddoc\else
\section{part three}
\input{cdocspt3}
\section{part four}
\input{cdocspt4}
\fi
%    \end{macrocode}

% Process as usual until here:
%    \begin{macrocode}
\fi
%    \end{macrocode}

% End of document body:
%    \begin{macrocode}
\end{document}
%    \end{macrocode}
%\iffalse
%</samplemain>
%\fi
%
% %%%%%%%%%%%%%%%%%%%%%%%%%%%%%%%%%%%%%%
% \paragraph{Chapter Include Files.}
%
% The include files are called |cdocsch1.tex| and |cdocsch2.tex|.
%
%\iffalse
%<*samplechap1|samplechap2>
%\fi

% Optional override for |\version| flag:
%    \begin{macrocode}
%%\providecommand{\version}{final}
%    \end{macrocode}

% Include the main document:
%    \begin{macrocode}
\input{childdoc.def}
\childdocof{cdocsamp}
%    \end{macrocode}

%\iffalse
%</samplechap1|samplechap2>
%\fi
%
%\iffalse
%<*samplechap1>
%\fi
% Some text for chapter 1:
%    \begin{macrocode}
\section{one}
some text in chapter one
%    \end{macrocode}

%\iffalse
%</samplechap1>
%\fi
% Some text for chapter 2:
%\iffalse
%<*samplechap2>
%\fi
%    \begin{macrocode}
\section{two}
more text in chapter two
%    \end{macrocode}

%\iffalse
%</samplechap2>
%\fi
%
% %%%%%%%%%%%%%%%%%%%%%%%%%%%%%%%%%%%%%%
% \paragraph{Part Include Files.}
%
% The include files are called |cdocspt3.tex| and |cdocspt4.tex|.
%
%\iffalse
%<*samplepart3|samplepart4>
%\fi

% Optional override for |\version| flag:
%    \begin{macrocode}
%%\providecommand{\version}{final}
%    \end{macrocode}

% Include the main document:
%    \begin{macrocode}
\input{childdoc.def}
\childdocby{cdocsamp}
%    \end{macrocode}

%\iffalse
%</samplepart3|samplepart4>
%\fi
%
%\iffalse
%<*samplepart3>
%\fi
% Some text for part 3:
%    \begin{macrocode}
some text in part three
%    \end{macrocode}

%\iffalse
%</samplepart3>
%\fi
% Some text for part 4:
%\iffalse
%<*samplepart4>
%\fi
%    \begin{macrocode}
more text in part four
%    \end{macrocode}

%\iffalse
%</samplepart4>
%\fi
%
% %%%%%%%%%%%%%%%%%%%%%%%%%%%%%%%%%%%%%%
% \paragraph{Forwarding for a Complete Draft.}
%
% The following forwarding file |cdocsdrf.tex|
% compiles the main document in draft mode:
%\iffalse
%<*sampledraft>
%\fi
%    \begin{macrocode}
\def\version{draft}
\input{childdoc.def}
\childdocforward{cdocsamp}
%    \end{macrocode}

%\iffalse
%</sampledraft>
%\fi
%
% %%%%%%%%%%%%%%%%%%%%%%%%%%%%%%%%%%%%%%
% \paragraph{Forwarding for Final Version of the Chapters.}
%
% The following forwarding files |cdocsfn1.tex| and |cdocsfn2.tex|
% (with identical content)
% compile the final versions of the child documents
% |cdocsch1.tex| and |cdocsch2.tex|, respectively:
%\iffalse
%<*samplefinal>
%\fi
%    \begin{macrocode}
\def\version{final}
\input{childdoc.def}
\childdocforwardprefix[cdocsamp]{cdocsfn}{cdocsch}
%    \end{macrocode}

%\iffalse
%</samplefinal>
%\fi
%
% %%%%%%%%%%%%%%%%%%%%%%%%%%%%%%%%%%%%%%
% \paragraph{Command Line Processing.}
%
% The following three command lines generate the output files
% |cdocscld|, |cdocscl1| and |cdocscl2|
% which should be identical to
% |cdocsdrf|, |cdocsch1| and |cdocsfn2|, respectively:
% \begin{center}
% \begin{tabular}{l}
% |latex -jobname cdocscld \|\\
% |  "\def\version{draft}\input{childdoc.def}\childdocforward{cdocsamp}"|\\
% |latex -jobname cdocscl1 \|\\
% |  "\input{childdoc.def}\childdocforward[cdocsamp]{cdocsch1}"|\\
% |latex -jobname cdocscl2 \|\\
% |  "\def\version{final}\input{childdoc.def}\childdocforward{cdocsch2}"|
% \end{tabular}
% \end{center}
% Note that the trailing backslash on each first line
% merely continues the input to the second line
% (for convenient cut ant paste).
% Furthermore, the command |latex| can be replaced by any
% of its alternative versions such as |pdflatex|.
%
% %%%%%%%%%%%%%%%%%%%%%%%%%%%%%%%%%%%%%%%%%%%%%%%%%%%%%%%%%%%%%%%%%%%%%%%%%%%%%%
% %%%%%%%%%%%%%%%%%%%%%%%%%%%%%%%%%%%%%%%%%%%%%%%%%%%%%%%%%%%%%%%%%%%%%%%%%%%%%%
% \section{Implementation}
%\iffalse
%<*package>
%\fi
%
% This section describes the definitions file |childdoc.def|.

% The definitions cannot be loaded using |\usepackage| or |\RequirePackage|
% which has a mechanism to prevent loading a style file more than once.
% When loading the definitions by means of |\input|
% multiple instances have to be prevented manually:
%\iffalse
%This code needs to be before the `\ProvidesFile' directive
%which is defined at the beginning of this file.
%Therefore it is also placed there and commented out here.
%</package>
%<*discard>
%\fi
%    \begin{macrocode}
\ifdefined\childdocmain\endinput\fi
%    \end{macrocode}
%\iffalse
%</discard>
%<*package>
%\fi
%
% \macro{\ifchilddoc}
% \macro{\ifchilddocmanual}
% The conditional |\ifchilddoc| tells whether a
% child (true) or main (false) document is being compiled.
% The conditional |\ifchilddocmanual| tells whether
% the |\includeonly| mechanism is used (false) or
% the selection of child files must be performed manually (true).
% The definitions initialise to false:
%    \begin{macrocode}
\newif\ifchilddoc
\newif\ifchilddocmanual
%    \end{macrocode}

% \macro{\childdocname}
% \macro{\childdocjob}
% The macro |\childdocname| stores the name of the main document
% to be compiled. The macro |\childdocjob| stores the name of
% the document on which the \LaTeX{} compiler was originally invoked.
% The content of |\jobname| cannot be compared
% to filenames specified in the source due to different catcodes.
% The following code rescans |\jobname|, stores the result
% in |\childdocname| and saves a copy in |\childdocjob|:
%    \begin{macrocode}
\edef\childdocname{\scantokens\expandafter{\jobname\noexpand}}
\let\childdocjob\childdocname
%    \end{macrocode}

% \macro{\childdocdisable}
% The macro |\childdocdisable| prevents the main file
% from being processed more than once.
% At this stage, the main document command |\childdocmain|
% is assumed to be called once again where it should do nothing.
% Any subsequent call to it should prevent
% a secondary processing of the main document
% It overwrites the forwarding commands
% |\childdocof| and |\childdocforward|
% with empty macros to prevent further inclusions of the main document:
%    \begin{macrocode}
\newcommand{\childdocdisable}
{
  \renewcommand{\childdocmain}[1]{\renewcommand{\childdocmain}[1]{\endinput}}
  \renewcommand{\childdocof}[1]{}
  \renewcommand{\childdocby}[2][]{}
  \renewcommand{\childdocforward}[2][]{}
  \renewcommand{\childdocdisable}{}
}
%    \end{macrocode}

% \macro{\childdocmain}
% The macro |\childdocmain| is to be called at the top of the main file
% with nothing or the main filename (without extension) as argument.
% First, it breaks loops.
% If the argument is not empty and does not match |\childdocname|
% (which is set by the first inclusion of |childdoc.def|),
% |\ifchilddoc| is set to true, |\includeonly| is applied to the child file
% and |\jobname| is set to the main file
% (for proper handling of |.aux| files):
%    \begin{macrocode}
\newcommand{\childdocmain}[1]
{
  \childdocdisable\childdocmain{}
  \if?#1?\else
    \begingroup
      \def\childdoctmp{#1}
      \ifx\childdoctmp\childdocname
        \def\childdoctmp{}
      \else
        \def\childdoctmp
        {
          \childdoctrue
          \includeonly{\childdocname}
          \def\childdocjob{#1}
          \def\jobname{#1}
        }
      \fi
      \expandafter
    \endgroup
    \childdoctmp
  \fi
}
%    \end{macrocode}

% \macro{\childdocof}
% The command |\childdocof| redirects
% compilation to the main file |#1|.
%    \begin{macrocode}
\newcommand{\childdocof}[1]
{
  \childdocdisable
  \childdoctrue
  \includeonly{\childdocname}
  \def\jobname{#1}
  \def\childdocjob{#1}
  \input{#1}
}
%    \end{macrocode}

% \macro{\childdocby}
% The command |\childdocby| ....
%    \begin{macrocode}
\newcommand{\childdocby}[2][]
{
  \childdocdisable
  \childdoctrue
  \childdocmanualtrue
  \if?#1?\else
    \def\jobname{#2}
  \fi
  \def\childdocjob{#2}
  \input{#2}
  \endinput
}
%    \end{macrocode}

% \macro{\childdocforward}
% The command |\childdocforward| redirects
% compilation to the main file or
% (if the optional argument is given) a child file.
% Parameters are set as if the main file
% or a child file starting with |\childdocof| was compiled.
% Then compilation is handed over to the main file:
%    \begin{macrocode}
\newcommand{\childdocforward}[2][]
{
  \begingroup
    \if?#1?
      \def\childdoctmp
      {
        \def\childdocname{#2}
        \def\childdocjob{#2}
        \def\jobname{#2}
        \input{#2}
        \endinput
      }
    \else
      \def\childdoctmp
      {
        \childdocdisable
        \def\childdocname{#2}
        \childdoctrue
        \includeonly{#2}
        \def\childdocjob{#1}
        \def\jobname{#1}
        \input{#1}
        \endinput
      }
    \fi
    \expandafter
  \endgroup
  \childdoctmp
}
%    \end{macrocode}

% \macro{\childdocforwardprefix}
% The command |\childdocforwardprefix| redirects
% compilation to the main or a child file by means of a pattern.
% The prefix |#1| in the current filename is replaced by |#2|
% and the suffix of the current filename is kept
% (it is assumed that the filename does not contain the substring `|~~~|'
% which is used as a delimiter).
% Compilation is handed over to the new file by |\childdocforward|:
%    \begin{macrocode}
\newcommand{\childdocforwardprefix}[3][]
{
  \begingroup
    \def\childdocextract #2##1~~~{\def\childdoctmp{\childdocforward[#1]{#3##1}}}
    \expandafter\childdocextract\childdocname~~~
    \expandafter
  \endgroup
  \childdoctmp
}
%    \end{macrocode}

% \macro{\childdoc}
% The deprecated macro |\childdoc| is a legacy version of |\childdocmain|:
%    \begin{macrocode}
\newcommand{\childdoc}{\childdocmain}
%    \end{macrocode}

% \macro{\childdocredirect}
% The deprecated macro |\childdocredirect| is a legacy version
% of |\childdocforward| and |\childdocforwardprefix|:
%    \begin{macrocode}
\newcommand{\childdocredirect}[2][]
{
  \begingroup
    \if?#1?
      \def\childdoctmp{\childdocforward{#2}}
    \else
      \def\childdoctmp{\childdocforwardprefix{#1}{#2}}
    \fi
    \expandafter
  \endgroup
  \childdoctmp
}
%    \end{macrocode}

%\iffalse
%</package>
%\fi
%
\endinput

\childdocforwardprefix[cdocsamp]{cdocsfn}{cdocsch}
%    \end{macrocode}

%\iffalse
%</samplefinal>
%\fi
%
% %%%%%%%%%%%%%%%%%%%%%%%%%%%%%%%%%%%%%%
% \paragraph{Command Line Processing.}
%
% The following three command lines generate the output files
% |cdocscld|, |cdocscl1| and |cdocscl2|
% which should be identical to
% |cdocsdrf|, |cdocsch1| and |cdocsfn2|, respectively:
% \begin{center}
% \begin{tabular}{l}
% |latex -jobname cdocscld \|\\
% |  "\def\version{draft}% \iffalse
%
% childdoc.dtx Copyright (C) 2017-2018 Niklas Beisert
%
% This work may be distributed and/or modified under the
% conditions of the LaTeX Project Public License, either version 1.3
% of this license or (at your option) any later version.
% The latest version of this license is in
%   http://www.latex-project.org/lppl.txt
% and version 1.3 or later is part of all distributions of LaTeX
% version 2005/12/01 or later.
%
% This work has the LPPL maintenance status `maintained'.
%
% The Current Maintainer of this work is Niklas Beisert.
%
% This work consists of the files childdoc.dtx and childdoc.ins
% and the derived files childdoc.def and cdocsamp.tex with
% cdocsch1.tex, cdocsch2.tex, cdocsdrf.tex, cdocsfn1.tex, cdocsfn2.tex.
%
%<package>\ifdefined\childdocmain\endinput\fi
%<package>\ProvidesFile{childdoc.def}[2018/12/30 v2.0 child document driver]
%<samplemain>\ProvidesFile{cdocsamp.tex}[2018/12/30 v2.0 sample for childdoc]
%<*driver>
%\ProvidesFile{childdoc.drv}[2018/12/30 v2.0 childdoc reference manual file]
\PassOptionsToClass{10pt,a4paper}{article}
\documentclass{ltxdoc}

\usepackage[margin=35mm]{geometry}
\usepackage{hyperref}
\usepackage{hyperxmp}
\usepackage[usenames]{color}

\hypersetup{colorlinks=true}
\hypersetup{pdfstartview=FitH}
\hypersetup{pdfpagemode=UseNone}
\hypersetup{pdfsource={}}
\hypersetup{pdflang={en-UK}}
\hypersetup{pdfcopyright={Copyright 2017-2018 Niklas Beisert.
  This work may be distributed and/or modified under the
  conditions of the LaTeX Project Public License, either version 1.3
  of this license or (at your option) any later version.}}
\hypersetup{pdflicenseurl={http://www.latex-project.org/lppl.txt}}
\hypersetup{pdfcontactaddress={ETH Zurich, ITP, HIT K,
  Wolfgang-Pauli-Strasse 27}}
\hypersetup{pdfcontactpostcode={8093}}
\hypersetup{pdfcontactcity={Zurich}}
\hypersetup{pdfcontactcountry={Switzerland}}
\hypersetup{pdfcontactemail={nbeisert@itp.phys.ethz.ch}}
\hypersetup{pdfcontacturl={http://people.phys.ethz.ch/\xmptilde nbeisert/}}

\newcommand{\secref}[1]{\hyperref[#1]{section \ref*{#1}}}

\parskip1ex
\parindent0pt
\let\olditemize\itemize
\def\itemize{\olditemize\parskip0pt}

\begin{document}

\title{The \textsf{childdoc} Package}
\hypersetup{pdftitle={The childdoc Package}}
\author{Niklas Beisert\\[2ex]
  Institut f\"ur Theoretische Physik\\
  Eidgen\"ossische Technische Hochschule Z\"urich\\
  Wolfgang-Pauli-Strasse 27, 8093 Z\"urich, Switzerland\\[1ex]
  \href{mailto:nbeisert@itp.phys.ethz.ch}
  {\texttt{nbeisert@itp.phys.ethz.ch}}}
\hypersetup{pdfauthor={Niklas Beisert}}
\hypersetup{pdfsubject={Manual for the LaTeX2e Package childdoc}}
\date{30 December 2018, \textsf{v2.0}}
\maketitle

\begin{abstract}\noindent
\textsf{childdoc} is a \LaTeXe{} package
that enables the direct compilation
of document sections included by |\include|
to individual files.
\end{abstract}

\begingroup
\parskip0ex
\tableofcontents
\endgroup

%%%%%%%%%%%%%%%%%%%%%%%%%%%%%%%%%%%%%%%%%%%%%%%%%%%%%%%%%%%%%%%%%%%%%%%%%%%%%%%%
%%%%%%%%%%%%%%%%%%%%%%%%%%%%%%%%%%%%%%%%%%%%%%%%%%%%%%%%%%%%%%%%%%%%%%%%%%%%%%%%
\section{Introduction}

\LaTeX{} provides a mechanism to structure a large document (such as a book)
into a main file and several child files (containing the chapters)
using the |\include| command.
This mechanism is beneficial for documents
which span hundreds of pages in order to
make the source file(s) more manageable.
Moreover, compilation can be restricted to
selected child files by means of the |\includeonly| command.
The latter feature can be used to reduce the compilation time while editing
(this was significantly more useful in the earlier days of \LaTeX{})
or to generate a smaller document which is easier to navigate.
Another application of |\includeonly| is to generate
documents consisting of selected parts of the complete document.

However, there are a few drawbacks of the plain |\include| mechanism:
\begin{itemize}
\item
The child files cannot be compiled on their own,
they can only be compiled via the main file.
A naive editing environment
(such as a text editor with an option
to have the current file processed by \LaTeX)
may require one to switch to the main file before compiling;
attempting to compile the child file produces errors.
\item
The main file must be modified (each time)
to adjust the |\includeonly| command
to the present needs. This easily leaves the main file in a messy state.
\item
The generated document will always carry the filename
of the main document. This is inconvenient if
several child files are to be compiled and
to be kept for distribution.
\end{itemize}

The present package provides a simple interface
to make child files individually compilable by \LaTeX{}.
Compiling a child file then has the same effect as compiling
the main file with an |\includeonly| command
to select the appropriate child.
Moreover the generated document will carry the name of the child
rather than the main file.
This resolves all three above issues.

This feature is meant to make the editing of books,
thesis documents and lecture notes somewhat more convenient.
However, the package can also be used efficiently for
composing a series of documents (such as exercise sheets)
which are typically distributed individually.
It then assists the author in generating the individual documents
(potentially in different versions)
as well as a document containing the collected series.
Another application is in developing style files
or other kinds of included material
where compilation of the style file could redirect
to a sample or test file.

%%%%%%%%%%%%%%%%%%%%%%%%%%%%%%%%%%%%%%%%%%%%%%%%%%%%%%%%%%%%%%%%%%%%%%%%%%%%%%%%
%%%%%%%%%%%%%%%%%%%%%%%%%%%%%%%%%%%%%%%%%%%%%%%%%%%%%%%%%%%%%%%%%%%%%%%%%%%%%%%%
\section{Usage}

First of all, the package \textsf{childdoc} is \emph{not} a standard
\LaTeXe{} |.sty| style file! Therefore it needs to be invoked in
a non-standard way.

%%%%%%%%%%%%%%%%%%%%%%%%%%%%%%%%%%%%%%%%%%%%%%%%%%%%%%%%%%%%%%%%%%%%%%%%%%%%%%%%
\subsection{Included Files}
\label{sec:include}

%%%%%%%%%%%%%%%%%%%%%%%%%%%%%%%%%%%%%%%%
\DescribeMacro{\childdocmain}
To use the package, add the commands
\begin{center}
\begin{tabular}{l}
|\input{childdoc.def}|\\
|\childdocmain{}|\\
\end{tabular}
\end{center}
at the very top of the main \LaTeX{} file,
in particular \emph{before} the |\documentclass| statement!
The argument of |\childdocmain| should be left empty
(but it must be present).

%%%%%%%%%%%%%%%%%%%%%%%%%%%%%%%%%%%%%%%%
\DescribeMacro{\childdocof}
Furthermore, add the commands
\begin{center}
\begin{tabular}{l}
|\input{childdoc.def}|\\
|\childdocof{|\textit{main}|}|\\
\end{tabular}
\end{center}
at the top of every child file \textit{child}
which is included by |\include{|\textit{child}|}|
from within the main file
(or at least for those files to be compiled individually).
The argument \textit{main} must be the filename of the main file.

There are a couple of
considerations in setting up the main and child documents:

%%%%%%%%%%%%%%%%%%%%%%%%%%%%%%%%%%%%%%%%
\paragraph{Restrictions.}

Please note the following restrictions:
\begin{itemize}
\item
|\childdocmain| must be called with one argument \textit{main}
to ensure compatibility with earlier version of the package.
It must either be empty (|\childdocmain{}|)
or precisely match the filename of the main file in which it is specified.
See \secref{sec:detection} for further information.
\item
The filename \textit{main} must be specified without the |.tex| extension.
\item
The filename \textit{main} is case sensitive
(even in case-insensitive file systems)
due to internal string comparison.
\item
The argument \textit{main} should be fully expanded, it cannot be a macro.
\item
Subdirectories and special characters should be avoided in filenames.
\item
The command |\childdocmain{|\textit{main}|}| must be followed by a whitespace.
It should not be followed immediately by another command
or by a comment mark `|%|'.
This is because the \TeX{} parser reads the token immediately following
the argument of |\childdocmain| and puts it
at the beginning of every child section;
however, a white\-space is ignored.
\end{itemize}

%%%%%%%%%%%%%%%%%%%%%%%%%%%%%%%%%%%%%%%%
\paragraph{Content of Main File.}

It is advisable to place all content in the child files included by |\include|.
Any output contained in the main file will appear in all child documents
unless suppressed manually;
it cannot be suppressed automatically by the |\includeonly| directive
and thus should normally be avoided.
A method to include some content in the main file
by means of conditional processing is described in \secref{sec:conditional}.

%%%%%%%%%%%%%%%%%%%%%%%%%%%%%%%%%%%%%%%%
\paragraph{Page Numbering.}

When only a part of the document is compiled,
the appropriate numbering of pages
(as well as other status parameters)
is determined from the |.aux| files.
The latter contain information from previous passes.
However this information needs to propagate through
all intermediate child documents.
Therefore the page numbering in child documents may well
be inconsistent until the complete document is compiled at least once.

A useful (if unconventional) way to always ensure a consistent
page numbering is to restart the numbering in each child document
and denote the pages by `\textit{child}|.|\textit{page}'
where \textit{child} represents the chapter/section number of the child file.
This can be achieved by the command
|\numberwithin{page}{|\textit{child}|}|
of the \textsf{amsmath} package
where \textit{child} can be |chapter| or |section|
depending on the chosen structuring.
Alternatively, one can modify the macro |\thepage| appropriately
and reset the counter |page| at the start of each child file.

%%%%%%%%%%%%%%%%%%%%%%%%%%%%%%%%%%%%%%%%%%%%%%%%%%%%%%%%%%%%%%%%%%%%%%%%%%%%%%%%
\subsection{Conditional Processing}
\label{sec:conditional}

The package provides a mechanism to compile different versions
of a document. To customise the versions further some conditional processing
can come in handy to distinguish which version is being compiled.
The package provides two macros to describe the compilation context:

%%%%%%%%%%%%%%%%%%%%%%%%%%%%%%%%%%%%%%%%
\DescribeMacro{\ifchilddoc}
The conditional |\ifchilddoc| distinguishes between the compilation of
child documents and the main document:
%
\begin{center}
|\ifchilddoc |\textit{child-code}| |[|\||else |\textit{main-code}]| \||fi|
\end{center}

%%%%%%%%%%%%%%%%%%%%%%%%%%%%%%%%%%%%%%%%
\DescribeMacro{\childdocname}
\DescribeMacro{\childdocjob}
The macro |\childdocname| contains the filename (without extension)
of the main or child file being processed.
Note that |\childdocjob| will always contain the name of the main file.

%%%%%%%%%%%%%%%%%%%%%%%%%%%%%%%%%%%%%%%%
\paragraph{Title Page.}

Conditional processing can be used to include a title or banner page
in the main document when proper precautions are taken.
Importantly, the code in the main file should ensure that the page counter
(as well as other status parameters which are stored in the |.aux| files)
takes the same value after the conditional processing.
Otherwise the page numbers may take divergent values
depending on which part is compiled.

For example, a title page could be declared by:
%
\begin{center}
\begin{tabular}{l}
|\ifchilddoc\||else|\\
|\addtocounter{page}{-1}|\\
\textit{code for title page}\\
|\newpage|\\
|\||fi|
\end{tabular}
\end{center}
%
A banner page for the child documents can be generated by:
%
\begin{center}
\begin{tabular}{l}
|\ifchilddoc|\\
|\addtocounter{page}{-1}|\\
\textit{code for banner page}\\
|\newpage|\\
|\||fi|
\end{tabular}
\end{center}
%
Here one could write a message such as:
\begin{center}
|This is the part \childdocname{} of \childdocjob{}.|
\end{center}

%%%%%%%%%%%%%%%%%%%%%%%%%%%%%%%%%%%%%%%%%%%%%%%%%%%%%%%%%%%%%%%%%%%%%%%%%%%%%%%%
\subsection{Flags}
\label{sec:flags}

The package makes it easy to generate different versions
of the main or child documents.
To this end compilation flags can be defined
and assigned different default values.
They will be particularly useful in conjunction
with the forwarding mechanism described in \secref{sec:forward}.

For example, it may be useful to have a flag |\version|
which can be set to |draft| or |final|.
The document source will contain some conditional code
depending on the value of |\version|.
Suppose further, the flag should default to |final| for the main file
and to |draft| for child files
which is a natural assignment for editing the document.
This is achieved by placing the following code
in the preamble of the main document
(below the |\childdocmain| directive):
%
\begin{center}
\begin{tabular}{l}
|\ifchilddoc|\\
|\providecommand{\version}{draft}|\\
|\||else|\\
|\providecommand{\version}{final}|\\
|\||fi|
\end{tabular}
\end{center}
%
The definition by |\providecommand| makes sure
that previous definitions are not overwritten.
Further statements |\providecommand{\version}{...}|
can thus be added before the above code to override it.

For the main file, one might add a line
(between |\childdocmain| and the above block)
%
\begin{center}
|%\ifchilddoc\||else\providecommand{\version}{draft}\||fi|
\end{center}
%
which can be uncommented to produce a draft version.
Likewise one can add a line to the very top of a child file
(above the |\childdocof{|\textit{main}|}| directive)
%
\begin{center}
|%\providecommand{\version}{final}|
\end{center}
%
which can be uncommented to produce the final version of this child document.

%%%%%%%%%%%%%%%%%%%%%%%%%%%%%%%%%%%%%%%%%%%%%%%%%%%%%%%%%%%%%%%%%%%%%%%%%%%%%%%%
\subsection{Forwarding}
\label{sec:forward}

Different versions of the main or child documents
using compilation flags as described in \secref{sec:flags}
can be (permanently) stored in different files
for convenient compilation, viewing and distribution.
To this end, the package defines a command
to pass on compilation to a different file:

%%%%%%%%%%%%%%%%%%%%%%%%%%%%%%%%%%%%%%%%
\DescribeMacro{\childdocforward}
The command |\childdocforward| redirects processing to
another source file:
%
\begin{center}
\begin{tabular}{l}
|\input{childdoc.def}|\\
|\childdocforward[|\textit{main}|]{|\textit{dest}|}|\\
\end{tabular}
\end{center}
%
The argument \textit{dest} is the destination file
(without extension).
It should be the main file or one of the child files.
Note that further \textsf{childdoc} directives
such as |\childdocof| and |\childdocforward|
in the indicated file will be processed in this form.
The optional argument \textit{main}
passes on directly to the main file \textit{main}
while pretending to compile the child \textit{dest}.
This form behaves as if \textit{dest}
issues |\childdocof{|\textit{main}|}| right away,
and no further \textsf{childdoc} directives will be processed.

%%%%%%%%%%%%%%%%%%%%%%%%%%%%%%%%%%%%%%%%
\DescribeMacro{\...prefix}
In the alternative form |\childdocforwardprefix|,
%
\begin{center}
\begin{tabular}{l}
|\input{childdoc.def}|\\
|\childdocforwardprefix[|\textit{main}|]{|\textit{prefix}|}{|\textit{dest}|}|
\end{tabular}
\end{center}
%
the destination file is determined by a pattern
depending on the current file:
To make this work, the current file must be called
`{\textit{prefix}\hspace{0.2em}\textit{suffix}}'
with \textit{prefix} matching precisely the argument.
Processing is then passed on to the file
`{\textit{dest}\hspace{0.2em}\textit{suffix}}'.
Surely, the same effect is achieved by
directly specifying the
argument `{\textit{dest}\hspace{0.2em}\textit{suffix}}'
in the first form.
However, that requires to set up a different file
for each child. With the alternative form of the command
all these files can have exactly the same content
which simplifies setting them up and maintaining them.

For example, the following file |draft.tex|
with a compilation flag |\version| as described in \secref{sec:flags}
compiles the main document as a draft:
%
\begin{center}
\begin{tabular}{l}
|\def\version{draft}|\\
|\input{childdoc.def}|\\
|\childdocforward{|\textit{main}|}|
\end{tabular}
\end{center}
%
Likewise, the following files |final|\textit{nn}|.tex|
compile the final version of the child document
|child|\textit{nn}|.tex|:
%
\begin{center}
\begin{tabular}{l}
|\def\version{final}|\\
|\input{childdoc.def}|\\
|\childdocforwardprefix{final}{child}|
\end{tabular}
\end{center}
%

Note that when several versions of a main file and/or of each child file
are to be generated, it may be convenient to set up a |Makefile| or
shell script to automatise the process.

%%%%%%%%%%%%%%%%%%%%%%%%%%%%%%%%%%%%%%%%%%%%%%%%%%%%%%%%%%%%%%%%%%%%%%%%%%%%%%%%
\subsection{Command Line Processing}
\label{sec:commandline}

The effect of redirection files can also be achieved by invoking
the \LaTeX{} compiler with a more elaborate command line.
Most conveniently this should be done as part
of a shell script or a |Makefile|.

When using \textsf{childdoc} in the main file, the following
command lines effectively perform a redirection
(note that depending on the shell being used,
backslashes may have to be doubled: `|\|' $\to$ `|\\|'):
%
\begin{center}
|... -jobname "|\textit{target}|" |\\|"|[\textit{flags}]%
|\input{childdoc.def}\childdocforward[|\textit{main}|]{|\textit{dest}|}"|
\end{center}
%
Here \textit{target} is the name of the output file,
\textit{main} is the name of the main file
and \textit{dest} is the name of the main or child file to be processed
(all filenames without extensions).
The optional argument \textit{main} can be omitted
if \textit{main} matches \textit{dest}.
Optionally, compilation \textit{flags} can be defined via |\def| commands.
This command line makes the \TeX{} engine believe
it is compiling the file \textit{target}
whose content is specified as the latter parameter.
The provided code then forwards the processing to
\textit{main} or \textit{dest} as described in \secref{sec:forward}.

%%%%%%%%%%%%%%%%%%%%%%%%%%%%%%%%%%%%%%%%%%%%%%%%%%%%%%%%%%%%%%%%%%%%%%%%%%%%%%%%
\subsection{Include by Input}
\label{sec:input}

Including child documents by |\include| has some restrictions by design.
Most notably, the content of a child document always occupies
its own set of pages; pages cannot be shared between child documents.
Usually, this behaviour makes perfect sense
because each child document contain an essential part of the document.
However, in some situations it may be desirable to compose
a document from a collection of parts
without having mandatory page breaks between then.
For this case, the package
provides a mechanism to include parts
by |\input| which can also be processed individually.
However, by construction this mechanism
requires manual handling of the content to be output.

%%%%%%%%%%%%%%%%%%%%%%%%%%%%%%%%%%%%%%%%
\DescribeMacro{\ifchilddocmanual}
The main file should be prepared as usual, see \secref{sec:include}.
However, the document body must make a distinction
between processing of an individual part and of the main document, e.g.:
%
\begin{center}
\begin{tabular}{l}
|\ifchilddocmanual|\\
|\input{\childdocname}|\\
|\||else|\\
\textit{document body with }|\input{|\textit{part}|}|\\
|\||fi|
\end{tabular}
\end{center}
%
The conditional |\ifchilddocmanual| is true whenever
a part to be included by |\input| is being compiled,
and the name of the part is stored in |\childdocname|.

%%%%%%%%%%%%%%%%%%%%%%%%%%%%%%%%%%%%%%%%
\DescribeMacro{\childdocby}
Each part to be included by |\input| should start with:
%
\begin{center}
\begin{tabular}{l}
|\input{childdoc.def}|\\
|\childdocby{|\textit{main}|}|\\
\end{tabular}
\end{center}
%
The directive |\childdocby| is similar to |\childdocof|
described in \secref{sec:include},
but the subsequent selection of content must be done manually.
To that end, both |\ifchilddoc| and |\ifchilddocmanual|
will be true upon processing of a part,
and the name of the part is stored in |\childdocname|.
Note that |\jobname| will be set to the filename of the current part
so that each part receives an individual |.aux| file
that does not interfere with the |.aux| file(s) of the main document.
This behaviour can be altered by the alternative form
|\childdocby[*]{|\textit{main}|}| (with a non-empty optional argument)
which uses the |.aux| file of the main document
by setting |\jobname| to \textit{main}.

%%%%%%%%%%%%%%%%%%%%%%%%%%%%%%%%%%%%%%%%%%%%%%%%%%%%%%%%%%%%%%%%%%%%%%%%%%%%%%%%
\subsection{Driver Development}
\label{sec:driver}

The \textsf{childdoc} mechanism can also be use for the development
of definition files such as \LaTeX{} styles or classes.
This case differs from the above setup with multiple parts
included by |\include| in that no |\includeonly| should be invoked.
This can be achieved by starting the include file
(before |\ProvidesPackage|) with:
%
\begin{center}
\begin{tabular}{l}
|\input{childdoc.def}|\\
|\childdocforward{|\textit{main}|}|\\
\end{tabular}
\end{center}
%
or alternatively with:
%
\begin{center}
\begin{tabular}{l}
|\input{childdoc.def}|\\
|\childdocby{|\textit{main}|}|\\
\end{tabular}
\end{center}
%
Both forms have slightly different effects as described above.
The main file is prepared as usual, see \secref{sec:include}.

%%%%%%%%%%%%%%%%%%%%%%%%%%%%%%%%%%%%%%%%%%%%%%%%%%%%%%%%%%%%%%%%%%%%%%%%%%%%%%%%
\subsection{Legacy Detection}
\label{sec:detection}

The directive |\childdocmain| in the main file can detect
whether the complete document or merely a child is to be compiled
even without using the directive |\childdocof|.
This method is deprecated because it is less robust
and there is no compelling reason to use it;
it is merely provided for backward compatibility
and it may be removed in future versions.

If the detection mechanism is to be used,
it is mandatory to correctly specify
the filename of the main file as the argument of |\childdocmain|:
%
\begin{center}
\begin{tabular}{l}
|\input{childdoc.def}|\\
|\childdocmain{|\textit{main}|}|\\
\end{tabular}
\end{center}
%
If |\jobname| does not match the argument \textit{main} of |\childdocmain|,
it is assumed that |\jobname| points to the child file to be compiled.
When using |\childdocmain| with the main file specified as argument,
it suffices to start a child file
with just |\input{|\textit{main}|}|
without loading of the package and using |\childdocof|.
If instead all processing is done
with the appropriate \textsf{childdoc} directives,
the argument of \textit{main} of |\childdocmain| can be empty.

An alternative version of the command line processing described
in \secref{sec:commandline} using the detection mechanism reads:
%
\begin{center}
|... -jobname "|\textit{target}|" "|[\textit{flags}]%
[|\def\jobname{|\textit{dest}|}|]|\input{|\textit{main}|}"|
\end{center}

%%%%%%%%%%%%%%%%%%%%%%%%%%%%%%%%%%%%%%%%%%%%%%%%%%%%%%%%%%%%%%%%%%%%%%%%%%%%%%%%
\subsection{Manual Code}
\label{sec:manual}

In case one cannot be certain whether the definitions file |childdoc.def|
is installed on the target \TeX{} distribution
and one prefers not to ship it,
it is conceivable to paste a few relevant commands into the sources.

To that end, drop all statements |\input{childdoc.def}|
and perform the replacements as outlined below.
Instead of |\childdocmain{|\textit{main}|}| add the following code
to the top of the main file:
%
\begin{center}
\begin{tabular}{l}
|\||ifdefined\childdocname\endinput\||fi\newif\ifchilddoc|\\
|\edef\childdocname{\scantokens\expandafter{\jobname\noexpand}}|\\
|\def\childdocmain{|\textit{main}|}\||ifx\childdocmain\childdocname\||else|\\
|\childdoctrue\includeonly{\childdocname}\let\jobname\childdocmain\||fi|\\
\end{tabular}
\end{center}
%
Instead of |\childdocof{|\textit{main}|}| just include the main file
at the top of each child file:
%
\begin{center}
|\input{|\textit{main}|}|
\end{center}
%
A simple redirection |\childdocforward{|\textit{dest}|}| is achieved by:
%
\begin{center}
|\def\jobname{|\textit{dest}|}\input{\jobname}|
\end{center}
%
The redirection with prefix
|\childdocforwardprefix[|\textit{prefix}|]{|\textit{dest}|}|
is accomplished by:
%
\begin{center}
\begin{tabular}{l}
|{\edef\jobname{\scantokens\expandafter{\jobname\noexpand}}|\\
|\def\redirectjob |\textit{prefix}|#1~~~{\gdef\jobname{|\textit{dest}|#1}}|\\
|\expandafter\redirectjob\jobname~~~}\input{\jobname}|
\end{tabular}
\end{center}

In an alternative approach,
child documents can be compiled by a specific command line
without additional code or specific definitions:
%
\begin{center}
|... -jobname "|\textit{target}|" "|[\textit{flags}]%
|\includeonly{|\textit{dest}|}\input{|\textit{main}|}"|
\end{center}
%

%%%%%%%%%%%%%%%%%%%%%%%%%%%%%%%%%%%%%%%%%%%%%%%%%%%%%%%%%%%%%%%%%%%%%%%%%%%%%%%%
%%%%%%%%%%%%%%%%%%%%%%%%%%%%%%%%%%%%%%%%%%%%%%%%%%%%%%%%%%%%%%%%%%%%%%%%%%%%%%%%
\section{Information}

%%%%%%%%%%%%%%%%%%%%%%%%%%%%%%%%%%%%%%%%%%%%%%%%%%%%%%%%%%%%%%%%%%%%%%%%%%%%%%%%
\subsection{Copyright}

Copyright \copyright{} 2017--2018 Niklas Beisert

This work may be distributed and/or modified under the
conditions of the \LaTeX{} Project Public License, either version 1.3
of this license or (at your option) any later version.
The latest version of this license is in
  \url{http://www.latex-project.org/lppl.txt}
and version 1.3 or later is part of all distributions of \LaTeX{}
version 2005/12/01 or later.

This work has the LPPL maintenance status `maintained'.

The Current Maintainer of this work is Niklas Beisert.

This work consists of the files |README.txt|, |childdoc.ins| and |childdoc.dtx|
as well as the derived files |childdoc.def|, |cdocsamp.tex|
with |cdocsch1.tex|, |cdocsch2.tex|, |cdocspt3.tex|, |cdocspt4.tex|,
|cdocsdrf.tex|, |cdocsfn1.tex|, |cdocsfn2.tex|
as well as |childdoc.pdf|.

%%%%%%%%%%%%%%%%%%%%%%%%%%%%%%%%%%%%%%%%%%%%%%%%%%%%%%%%%%%%%%%%%%%%%%%%%%%%%%%%
\subsection{Files and Installation}

The package consists of the files:
%
\begin{center}
\begin{tabular}{ll}
    |README.txt|   & readme file \\
    |childdoc.ins| & installation file \\
    |childdoc.dtx| & source file \\
    |childdoc.def| & definition file \\
    |cdocsamp.tex| & sample main file \\
    |cdocsch1.tex| & sample include file \\
    |cdocsch2.tex| & sample include file \\
    |cdocspt3.tex| & sample part file \\
    |cdocspt4.tex| & sample part file \\
    |cdocsdrf.tex| & sample redirection file \\
    |cdocsfn1.tex| & sample redirection file \\
    |cdocsfn2.tex| & sample redirection file \\
    |childdoc.pdf| & manual
\end{tabular}
\end{center}
%
The distribution consists of the files
|README.txt|, |childdoc.ins| and |childdoc.dtx|.
%
\begin{itemize}
\item
Run (pdf)\LaTeX{} on |childdoc.dtx|
to compile the manual |childdoc.pdf| (this file).
\item
Run \LaTeX{} on |childdoc.ins| to create the definitions file |childdoc.def|
and the sample |cdocsamp.tex| with include files
|cdocsch1.tex|, |cdocsch2.tex|, |cdocspt3.tex|, |cdocspt4.tex|,
|cdocsdrf.tex|, |cdocsfn1.tex|, |cdocsfn2.tex|.
Then copy the file |childdoc.def| to an appropriate directory of your \LaTeX{}
distribution, e.g.\ \textit{texmf-root}|/tex/latex/childdoc|.
\end{itemize}

%%%%%%%%%%%%%%%%%%%%%%%%%%%%%%%%%%%%%%%%%%%%%%%%%%%%%%%%%%%%%%%%%%%%%%%%%%%%%%%%
\subsection{Related CTAN Packages}

There are several other packages which offer a similar functionality:
%
\begin{itemize}
\item
The packages
\href{http://ctan.org/pkg/docmute}{\textsf{docmute}},
\href{http://ctan.org/pkg/includex}{\textsf{includex}} and
\href{http://ctan.org/pkg/standalone}{\textsf{standalone}}
provide commands to include only the document body of
a child file thus allowing both files to be compiled individually.
\item
The packages \href{http://ctan.org/pkg/subdocs}{\textsf{subdocs}}
and \href{http://ctan.org/pkg/subfiles}{\textsf{subfiles}}
provide structures in which the main and child documents can be
encapsulated and allowing them to be compiled individually.
The inclusion mechanism is different from the conventional |\include|.
\item
The package \href{http://ctan.org/pkg/combine}{\textsf{combine}}
is an elaborate solution to combine several documents into one.
\end{itemize}
%
See also the CTAN topic \href{http://ctan.org/topic/subdocs}{\textsf{subdocs}}
for further related packages.
The present package differs from the above solutions in that
a document structure constructed with the conventional |\include| mechanism
just needs two extra commands at the top of every file
such that all constituent files can be compiled individually.

%%%%%%%%%%%%%%%%%%%%%%%%%%%%%%%%%%%%%%%%%%%%%%%%%%%%%%%%%%%%%%%%%%%%%%%%%%%%%%%%
%\subsection{Feature Suggestions}
%
%The following is a list of features which may be useful for future
%versions of this package:
%%
%\begin{itemize}
%\item
%\ldots
%\end{itemize}

%%%%%%%%%%%%%%%%%%%%%%%%%%%%%%%%%%%%%%%%%%%%%%%%%%%%%%%%%%%%%%%%%%%%%%%%%%%%%%%%
\subsection{Revision History}

%%%%%%%%%%%%%%%%%%%%%%%%%%%%%%%%%%%%%%%%
\paragraph{v2.0:} 2018/12/30

\begin{itemize}
\item
immediate forward processing
\item
added |\childdocby| mechanism
\item
manual restructured
\end{itemize}

%%%%%%%%%%%%%%%%%%%%%%%%%%%%%%%%%%%%%%%%
\paragraph{v1.6:} 2018/01/17

\begin{itemize}
\item
application for development of include files
\item
corrections to manual
\end{itemize}

%%%%%%%%%%%%%%%%%%%%%%%%%%%%%%%%%%%%%%%%
\paragraph{v1.5:} 2017/05/21

\begin{itemize}
\item
more complete structuring introduced
\item
|\childdocof| introduced
\item
|\childdoc| renamed to |\childdocmain|
\item
|\childredirect| renamed to |\childdocforward| and |\childdocforwardprefix|
and functionality expanded
\end{itemize}

%%%%%%%%%%%%%%%%%%%%%%%%%%%%%%%%%%%%%%%%
\paragraph{v1.0:} 2017/04/27

\begin{itemize}
\item
manual and install package
\item
first version published on CTAN
\end{itemize}

%%%%%%%%%%%%%%%%%%%%%%%%%%%%%%%%%%%%%%%%
\paragraph{v0.6:} 2017/04/26

\begin{itemize}
\item
redirection mechanism added
\end{itemize}

%%%%%%%%%%%%%%%%%%%%%%%%%%%%%%%%%%%%%%%%
\paragraph{v0.5:} 2017/04/26

\begin{itemize}
\item
functionality in definition file
\end{itemize}


%%%%%%%%%%%%%%%%%%%%%%%%%%%%%%%%%%%%%%%%%%%%%%%%%%%%%%%%%%%%%%%%%%%%%%%%%%%%%%%%
%%%%%%%%%%%%%%%%%%%%%%%%%%%%%%%%%%%%%%%%%%%%%%%%%%%%%%%%%%%%%%%%%%%%%%%%%%%%%%%%
%%%%%%%%%%%%%%%%%%%%%%%%%%%%%%%%%%%%%%%%%%%%%%%%%%%%%%%%%%%%%%%%%%%%%%%%%%%%%%%%
\appendix

\settowidth\MacroIndent{\rmfamily\scriptsize 000\ }

 \DocInput{childdoc.dtx}

\end{document}
%</driver>
% \fi
%
% %%%%%%%%%%%%%%%%%%%%%%%%%%%%%%%%%%%%%%%%%%%%%%%%%%%%%%%%%%%%%%%%%%%%%%%%%%%%%%
% %%%%%%%%%%%%%%%%%%%%%%%%%%%%%%%%%%%%%%%%%%%%%%%%%%%%%%%%%%%%%%%%%%%%%%%%%%%%%%
% \section{Sample}
%\iffalse
%<*samplemain>
%\fi
%
% The following presents a sample document
% with two chapters, two parts, a title page,
% a compile flag as well as three forwarding files to set the flag.
% It consists of eight |.tex| files:
% \begin{center}
% \begin{tabular}{ll}
% |cdocsamp.tex|&main file\\
% |cdocsch1.tex|&include file for chapter 1\\
% |cdocsch2.tex|&include file for chapter 2\\
% |cdocspt3.tex|&include file for part 3\\
% |cdocspt4.tex|&include file for part 4\\
% |cdocsdrf.tex|&forwarding file for main file in draft mode\\
% |cdocsfi1.tex|&forwarding file for final version of chapter 1\\
% |cdocsfi2.tex|&forwarding file for final version of chapter 2\\
% \end{tabular}
% \end{center}
% Each of the eight files can be compiled directly by the \LaTeX{} compiler.
%
% %%%%%%%%%%%%%%%%%%%%%%%%%%%%%%%%%%%%%%
% \paragraph{Main File.}
%
% The main file is called |cdocsamp.tex|.
%
% Load the \textsf{childdoc} definitions and
% declare the filename for the main document:
%    \begin{macrocode}
\input{childdoc.def}
\childdocmain{}
%    \end{macrocode}

% Optional override for |\version| flag:
%    \begin{macrocode}
%%\ifchilddoc\else\providecommand{\version}{draft}\fi
%    \end{macrocode}

% Define the default values for the |\version| flag
% (|final| for the main file and |draft| for childs):
%    \begin{macrocode}
\ifchilddoc
\providecommand{\version}{draft}
\else
\providecommand{\version}{final}
\fi
%    \end{macrocode}

% Load the standard document class:
%    \begin{macrocode}
\documentclass[12pt]{article}
%    \end{macrocode}

% Start the document body:
%    \begin{macrocode}
\begin{document}
%    \end{macrocode}

% Declare a title page.
% Print title, part of document being processed and version flag:
%    \begin{macrocode}
\addtocounter{page}{-1}
\begin{center}
{\LARGE\bfseries{}childdoc example\par}
\vspace{1cm}
\ifchilddoc
\ifchilddocmanual part\else chapter\fi:
`\childdocname' of `\childdocjob'\par
\else
main document: `\childdocjob'\par
\fi
version: \version\par
\end{center}
\newpage
%    \end{macrocode}

% Manually include selected file,
% otherwise process as usual:
%    \begin{macrocode}
\ifchilddocmanual
\section*{part `\childdocname'}
\input{\childdocname}
\else
%    \end{macrocode}

% Include the two chapters:
%    \begin{macrocode}
\include{cdocsch1}
\include{cdocsch2}
%    \end{macrocode}

% Include the two parts unless only chapters should be displayed:
%    \begin{macrocode}
\ifchilddoc\else
\section{part three}
\input{cdocspt3}
\section{part four}
\input{cdocspt4}
\fi
%    \end{macrocode}

% Process as usual until here:
%    \begin{macrocode}
\fi
%    \end{macrocode}

% End of document body:
%    \begin{macrocode}
\end{document}
%    \end{macrocode}
%\iffalse
%</samplemain>
%\fi
%
% %%%%%%%%%%%%%%%%%%%%%%%%%%%%%%%%%%%%%%
% \paragraph{Chapter Include Files.}
%
% The include files are called |cdocsch1.tex| and |cdocsch2.tex|.
%
%\iffalse
%<*samplechap1|samplechap2>
%\fi

% Optional override for |\version| flag:
%    \begin{macrocode}
%%\providecommand{\version}{final}
%    \end{macrocode}

% Include the main document:
%    \begin{macrocode}
\input{childdoc.def}
\childdocof{cdocsamp}
%    \end{macrocode}

%\iffalse
%</samplechap1|samplechap2>
%\fi
%
%\iffalse
%<*samplechap1>
%\fi
% Some text for chapter 1:
%    \begin{macrocode}
\section{one}
some text in chapter one
%    \end{macrocode}

%\iffalse
%</samplechap1>
%\fi
% Some text for chapter 2:
%\iffalse
%<*samplechap2>
%\fi
%    \begin{macrocode}
\section{two}
more text in chapter two
%    \end{macrocode}

%\iffalse
%</samplechap2>
%\fi
%
% %%%%%%%%%%%%%%%%%%%%%%%%%%%%%%%%%%%%%%
% \paragraph{Part Include Files.}
%
% The include files are called |cdocspt3.tex| and |cdocspt4.tex|.
%
%\iffalse
%<*samplepart3|samplepart4>
%\fi

% Optional override for |\version| flag:
%    \begin{macrocode}
%%\providecommand{\version}{final}
%    \end{macrocode}

% Include the main document:
%    \begin{macrocode}
\input{childdoc.def}
\childdocby{cdocsamp}
%    \end{macrocode}

%\iffalse
%</samplepart3|samplepart4>
%\fi
%
%\iffalse
%<*samplepart3>
%\fi
% Some text for part 3:
%    \begin{macrocode}
some text in part three
%    \end{macrocode}

%\iffalse
%</samplepart3>
%\fi
% Some text for part 4:
%\iffalse
%<*samplepart4>
%\fi
%    \begin{macrocode}
more text in part four
%    \end{macrocode}

%\iffalse
%</samplepart4>
%\fi
%
% %%%%%%%%%%%%%%%%%%%%%%%%%%%%%%%%%%%%%%
% \paragraph{Forwarding for a Complete Draft.}
%
% The following forwarding file |cdocsdrf.tex|
% compiles the main document in draft mode:
%\iffalse
%<*sampledraft>
%\fi
%    \begin{macrocode}
\def\version{draft}
\input{childdoc.def}
\childdocforward{cdocsamp}
%    \end{macrocode}

%\iffalse
%</sampledraft>
%\fi
%
% %%%%%%%%%%%%%%%%%%%%%%%%%%%%%%%%%%%%%%
% \paragraph{Forwarding for Final Version of the Chapters.}
%
% The following forwarding files |cdocsfn1.tex| and |cdocsfn2.tex|
% (with identical content)
% compile the final versions of the child documents
% |cdocsch1.tex| and |cdocsch2.tex|, respectively:
%\iffalse
%<*samplefinal>
%\fi
%    \begin{macrocode}
\def\version{final}
\input{childdoc.def}
\childdocforwardprefix[cdocsamp]{cdocsfn}{cdocsch}
%    \end{macrocode}

%\iffalse
%</samplefinal>
%\fi
%
% %%%%%%%%%%%%%%%%%%%%%%%%%%%%%%%%%%%%%%
% \paragraph{Command Line Processing.}
%
% The following three command lines generate the output files
% |cdocscld|, |cdocscl1| and |cdocscl2|
% which should be identical to
% |cdocsdrf|, |cdocsch1| and |cdocsfn2|, respectively:
% \begin{center}
% \begin{tabular}{l}
% |latex -jobname cdocscld \|\\
% |  "\def\version{draft}\input{childdoc.def}\childdocforward{cdocsamp}"|\\
% |latex -jobname cdocscl1 \|\\
% |  "\input{childdoc.def}\childdocforward[cdocsamp]{cdocsch1}"|\\
% |latex -jobname cdocscl2 \|\\
% |  "\def\version{final}\input{childdoc.def}\childdocforward{cdocsch2}"|
% \end{tabular}
% \end{center}
% Note that the trailing backslash on each first line
% merely continues the input to the second line
% (for convenient cut ant paste).
% Furthermore, the command |latex| can be replaced by any
% of its alternative versions such as |pdflatex|.
%
% %%%%%%%%%%%%%%%%%%%%%%%%%%%%%%%%%%%%%%%%%%%%%%%%%%%%%%%%%%%%%%%%%%%%%%%%%%%%%%
% %%%%%%%%%%%%%%%%%%%%%%%%%%%%%%%%%%%%%%%%%%%%%%%%%%%%%%%%%%%%%%%%%%%%%%%%%%%%%%
% \section{Implementation}
%\iffalse
%<*package>
%\fi
%
% This section describes the definitions file |childdoc.def|.

% The definitions cannot be loaded using |\usepackage| or |\RequirePackage|
% which has a mechanism to prevent loading a style file more than once.
% When loading the definitions by means of |\input|
% multiple instances have to be prevented manually:
%\iffalse
%This code needs to be before the `\ProvidesFile' directive
%which is defined at the beginning of this file.
%Therefore it is also placed there and commented out here.
%</package>
%<*discard>
%\fi
%    \begin{macrocode}
\ifdefined\childdocmain\endinput\fi
%    \end{macrocode}
%\iffalse
%</discard>
%<*package>
%\fi
%
% \macro{\ifchilddoc}
% \macro{\ifchilddocmanual}
% The conditional |\ifchilddoc| tells whether a
% child (true) or main (false) document is being compiled.
% The conditional |\ifchilddocmanual| tells whether
% the |\includeonly| mechanism is used (false) or
% the selection of child files must be performed manually (true).
% The definitions initialise to false:
%    \begin{macrocode}
\newif\ifchilddoc
\newif\ifchilddocmanual
%    \end{macrocode}

% \macro{\childdocname}
% \macro{\childdocjob}
% The macro |\childdocname| stores the name of the main document
% to be compiled. The macro |\childdocjob| stores the name of
% the document on which the \LaTeX{} compiler was originally invoked.
% The content of |\jobname| cannot be compared
% to filenames specified in the source due to different catcodes.
% The following code rescans |\jobname|, stores the result
% in |\childdocname| and saves a copy in |\childdocjob|:
%    \begin{macrocode}
\edef\childdocname{\scantokens\expandafter{\jobname\noexpand}}
\let\childdocjob\childdocname
%    \end{macrocode}

% \macro{\childdocdisable}
% The macro |\childdocdisable| prevents the main file
% from being processed more than once.
% At this stage, the main document command |\childdocmain|
% is assumed to be called once again where it should do nothing.
% Any subsequent call to it should prevent
% a secondary processing of the main document
% It overwrites the forwarding commands
% |\childdocof| and |\childdocforward|
% with empty macros to prevent further inclusions of the main document:
%    \begin{macrocode}
\newcommand{\childdocdisable}
{
  \renewcommand{\childdocmain}[1]{\renewcommand{\childdocmain}[1]{\endinput}}
  \renewcommand{\childdocof}[1]{}
  \renewcommand{\childdocby}[2][]{}
  \renewcommand{\childdocforward}[2][]{}
  \renewcommand{\childdocdisable}{}
}
%    \end{macrocode}

% \macro{\childdocmain}
% The macro |\childdocmain| is to be called at the top of the main file
% with nothing or the main filename (without extension) as argument.
% First, it breaks loops.
% If the argument is not empty and does not match |\childdocname|
% (which is set by the first inclusion of |childdoc.def|),
% |\ifchilddoc| is set to true, |\includeonly| is applied to the child file
% and |\jobname| is set to the main file
% (for proper handling of |.aux| files):
%    \begin{macrocode}
\newcommand{\childdocmain}[1]
{
  \childdocdisable\childdocmain{}
  \if?#1?\else
    \begingroup
      \def\childdoctmp{#1}
      \ifx\childdoctmp\childdocname
        \def\childdoctmp{}
      \else
        \def\childdoctmp
        {
          \childdoctrue
          \includeonly{\childdocname}
          \def\childdocjob{#1}
          \def\jobname{#1}
        }
      \fi
      \expandafter
    \endgroup
    \childdoctmp
  \fi
}
%    \end{macrocode}

% \macro{\childdocof}
% The command |\childdocof| redirects
% compilation to the main file |#1|.
%    \begin{macrocode}
\newcommand{\childdocof}[1]
{
  \childdocdisable
  \childdoctrue
  \includeonly{\childdocname}
  \def\jobname{#1}
  \def\childdocjob{#1}
  \input{#1}
}
%    \end{macrocode}

% \macro{\childdocby}
% The command |\childdocby| ....
%    \begin{macrocode}
\newcommand{\childdocby}[2][]
{
  \childdocdisable
  \childdoctrue
  \childdocmanualtrue
  \if?#1?\else
    \def\jobname{#2}
  \fi
  \def\childdocjob{#2}
  \input{#2}
  \endinput
}
%    \end{macrocode}

% \macro{\childdocforward}
% The command |\childdocforward| redirects
% compilation to the main file or
% (if the optional argument is given) a child file.
% Parameters are set as if the main file
% or a child file starting with |\childdocof| was compiled.
% Then compilation is handed over to the main file:
%    \begin{macrocode}
\newcommand{\childdocforward}[2][]
{
  \begingroup
    \if?#1?
      \def\childdoctmp
      {
        \def\childdocname{#2}
        \def\childdocjob{#2}
        \def\jobname{#2}
        \input{#2}
        \endinput
      }
    \else
      \def\childdoctmp
      {
        \childdocdisable
        \def\childdocname{#2}
        \childdoctrue
        \includeonly{#2}
        \def\childdocjob{#1}
        \def\jobname{#1}
        \input{#1}
        \endinput
      }
    \fi
    \expandafter
  \endgroup
  \childdoctmp
}
%    \end{macrocode}

% \macro{\childdocforwardprefix}
% The command |\childdocforwardprefix| redirects
% compilation to the main or a child file by means of a pattern.
% The prefix |#1| in the current filename is replaced by |#2|
% and the suffix of the current filename is kept
% (it is assumed that the filename does not contain the substring `|~~~|'
% which is used as a delimiter).
% Compilation is handed over to the new file by |\childdocforward|:
%    \begin{macrocode}
\newcommand{\childdocforwardprefix}[3][]
{
  \begingroup
    \def\childdocextract #2##1~~~{\def\childdoctmp{\childdocforward[#1]{#3##1}}}
    \expandafter\childdocextract\childdocname~~~
    \expandafter
  \endgroup
  \childdoctmp
}
%    \end{macrocode}

% \macro{\childdoc}
% The deprecated macro |\childdoc| is a legacy version of |\childdocmain|:
%    \begin{macrocode}
\newcommand{\childdoc}{\childdocmain}
%    \end{macrocode}

% \macro{\childdocredirect}
% The deprecated macro |\childdocredirect| is a legacy version
% of |\childdocforward| and |\childdocforwardprefix|:
%    \begin{macrocode}
\newcommand{\childdocredirect}[2][]
{
  \begingroup
    \if?#1?
      \def\childdoctmp{\childdocforward{#2}}
    \else
      \def\childdoctmp{\childdocforwardprefix{#1}{#2}}
    \fi
    \expandafter
  \endgroup
  \childdoctmp
}
%    \end{macrocode}

%\iffalse
%</package>
%\fi
%
\endinput
\childdocforward{cdocsamp}"|\\
% |latex -jobname cdocscl1 \|\\
% |  "% \iffalse
%
% childdoc.dtx Copyright (C) 2017-2018 Niklas Beisert
%
% This work may be distributed and/or modified under the
% conditions of the LaTeX Project Public License, either version 1.3
% of this license or (at your option) any later version.
% The latest version of this license is in
%   http://www.latex-project.org/lppl.txt
% and version 1.3 or later is part of all distributions of LaTeX
% version 2005/12/01 or later.
%
% This work has the LPPL maintenance status `maintained'.
%
% The Current Maintainer of this work is Niklas Beisert.
%
% This work consists of the files childdoc.dtx and childdoc.ins
% and the derived files childdoc.def and cdocsamp.tex with
% cdocsch1.tex, cdocsch2.tex, cdocsdrf.tex, cdocsfn1.tex, cdocsfn2.tex.
%
%<package>\ifdefined\childdocmain\endinput\fi
%<package>\ProvidesFile{childdoc.def}[2018/12/30 v2.0 child document driver]
%<samplemain>\ProvidesFile{cdocsamp.tex}[2018/12/30 v2.0 sample for childdoc]
%<*driver>
%\ProvidesFile{childdoc.drv}[2018/12/30 v2.0 childdoc reference manual file]
\PassOptionsToClass{10pt,a4paper}{article}
\documentclass{ltxdoc}

\usepackage[margin=35mm]{geometry}
\usepackage{hyperref}
\usepackage{hyperxmp}
\usepackage[usenames]{color}

\hypersetup{colorlinks=true}
\hypersetup{pdfstartview=FitH}
\hypersetup{pdfpagemode=UseNone}
\hypersetup{pdfsource={}}
\hypersetup{pdflang={en-UK}}
\hypersetup{pdfcopyright={Copyright 2017-2018 Niklas Beisert.
  This work may be distributed and/or modified under the
  conditions of the LaTeX Project Public License, either version 1.3
  of this license or (at your option) any later version.}}
\hypersetup{pdflicenseurl={http://www.latex-project.org/lppl.txt}}
\hypersetup{pdfcontactaddress={ETH Zurich, ITP, HIT K,
  Wolfgang-Pauli-Strasse 27}}
\hypersetup{pdfcontactpostcode={8093}}
\hypersetup{pdfcontactcity={Zurich}}
\hypersetup{pdfcontactcountry={Switzerland}}
\hypersetup{pdfcontactemail={nbeisert@itp.phys.ethz.ch}}
\hypersetup{pdfcontacturl={http://people.phys.ethz.ch/\xmptilde nbeisert/}}

\newcommand{\secref}[1]{\hyperref[#1]{section \ref*{#1}}}

\parskip1ex
\parindent0pt
\let\olditemize\itemize
\def\itemize{\olditemize\parskip0pt}

\begin{document}

\title{The \textsf{childdoc} Package}
\hypersetup{pdftitle={The childdoc Package}}
\author{Niklas Beisert\\[2ex]
  Institut f\"ur Theoretische Physik\\
  Eidgen\"ossische Technische Hochschule Z\"urich\\
  Wolfgang-Pauli-Strasse 27, 8093 Z\"urich, Switzerland\\[1ex]
  \href{mailto:nbeisert@itp.phys.ethz.ch}
  {\texttt{nbeisert@itp.phys.ethz.ch}}}
\hypersetup{pdfauthor={Niklas Beisert}}
\hypersetup{pdfsubject={Manual for the LaTeX2e Package childdoc}}
\date{30 December 2018, \textsf{v2.0}}
\maketitle

\begin{abstract}\noindent
\textsf{childdoc} is a \LaTeXe{} package
that enables the direct compilation
of document sections included by |\include|
to individual files.
\end{abstract}

\begingroup
\parskip0ex
\tableofcontents
\endgroup

%%%%%%%%%%%%%%%%%%%%%%%%%%%%%%%%%%%%%%%%%%%%%%%%%%%%%%%%%%%%%%%%%%%%%%%%%%%%%%%%
%%%%%%%%%%%%%%%%%%%%%%%%%%%%%%%%%%%%%%%%%%%%%%%%%%%%%%%%%%%%%%%%%%%%%%%%%%%%%%%%
\section{Introduction}

\LaTeX{} provides a mechanism to structure a large document (such as a book)
into a main file and several child files (containing the chapters)
using the |\include| command.
This mechanism is beneficial for documents
which span hundreds of pages in order to
make the source file(s) more manageable.
Moreover, compilation can be restricted to
selected child files by means of the |\includeonly| command.
The latter feature can be used to reduce the compilation time while editing
(this was significantly more useful in the earlier days of \LaTeX{})
or to generate a smaller document which is easier to navigate.
Another application of |\includeonly| is to generate
documents consisting of selected parts of the complete document.

However, there are a few drawbacks of the plain |\include| mechanism:
\begin{itemize}
\item
The child files cannot be compiled on their own,
they can only be compiled via the main file.
A naive editing environment
(such as a text editor with an option
to have the current file processed by \LaTeX)
may require one to switch to the main file before compiling;
attempting to compile the child file produces errors.
\item
The main file must be modified (each time)
to adjust the |\includeonly| command
to the present needs. This easily leaves the main file in a messy state.
\item
The generated document will always carry the filename
of the main document. This is inconvenient if
several child files are to be compiled and
to be kept for distribution.
\end{itemize}

The present package provides a simple interface
to make child files individually compilable by \LaTeX{}.
Compiling a child file then has the same effect as compiling
the main file with an |\includeonly| command
to select the appropriate child.
Moreover the generated document will carry the name of the child
rather than the main file.
This resolves all three above issues.

This feature is meant to make the editing of books,
thesis documents and lecture notes somewhat more convenient.
However, the package can also be used efficiently for
composing a series of documents (such as exercise sheets)
which are typically distributed individually.
It then assists the author in generating the individual documents
(potentially in different versions)
as well as a document containing the collected series.
Another application is in developing style files
or other kinds of included material
where compilation of the style file could redirect
to a sample or test file.

%%%%%%%%%%%%%%%%%%%%%%%%%%%%%%%%%%%%%%%%%%%%%%%%%%%%%%%%%%%%%%%%%%%%%%%%%%%%%%%%
%%%%%%%%%%%%%%%%%%%%%%%%%%%%%%%%%%%%%%%%%%%%%%%%%%%%%%%%%%%%%%%%%%%%%%%%%%%%%%%%
\section{Usage}

First of all, the package \textsf{childdoc} is \emph{not} a standard
\LaTeXe{} |.sty| style file! Therefore it needs to be invoked in
a non-standard way.

%%%%%%%%%%%%%%%%%%%%%%%%%%%%%%%%%%%%%%%%%%%%%%%%%%%%%%%%%%%%%%%%%%%%%%%%%%%%%%%%
\subsection{Included Files}
\label{sec:include}

%%%%%%%%%%%%%%%%%%%%%%%%%%%%%%%%%%%%%%%%
\DescribeMacro{\childdocmain}
To use the package, add the commands
\begin{center}
\begin{tabular}{l}
|\input{childdoc.def}|\\
|\childdocmain{}|\\
\end{tabular}
\end{center}
at the very top of the main \LaTeX{} file,
in particular \emph{before} the |\documentclass| statement!
The argument of |\childdocmain| should be left empty
(but it must be present).

%%%%%%%%%%%%%%%%%%%%%%%%%%%%%%%%%%%%%%%%
\DescribeMacro{\childdocof}
Furthermore, add the commands
\begin{center}
\begin{tabular}{l}
|\input{childdoc.def}|\\
|\childdocof{|\textit{main}|}|\\
\end{tabular}
\end{center}
at the top of every child file \textit{child}
which is included by |\include{|\textit{child}|}|
from within the main file
(or at least for those files to be compiled individually).
The argument \textit{main} must be the filename of the main file.

There are a couple of
considerations in setting up the main and child documents:

%%%%%%%%%%%%%%%%%%%%%%%%%%%%%%%%%%%%%%%%
\paragraph{Restrictions.}

Please note the following restrictions:
\begin{itemize}
\item
|\childdocmain| must be called with one argument \textit{main}
to ensure compatibility with earlier version of the package.
It must either be empty (|\childdocmain{}|)
or precisely match the filename of the main file in which it is specified.
See \secref{sec:detection} for further information.
\item
The filename \textit{main} must be specified without the |.tex| extension.
\item
The filename \textit{main} is case sensitive
(even in case-insensitive file systems)
due to internal string comparison.
\item
The argument \textit{main} should be fully expanded, it cannot be a macro.
\item
Subdirectories and special characters should be avoided in filenames.
\item
The command |\childdocmain{|\textit{main}|}| must be followed by a whitespace.
It should not be followed immediately by another command
or by a comment mark `|%|'.
This is because the \TeX{} parser reads the token immediately following
the argument of |\childdocmain| and puts it
at the beginning of every child section;
however, a white\-space is ignored.
\end{itemize}

%%%%%%%%%%%%%%%%%%%%%%%%%%%%%%%%%%%%%%%%
\paragraph{Content of Main File.}

It is advisable to place all content in the child files included by |\include|.
Any output contained in the main file will appear in all child documents
unless suppressed manually;
it cannot be suppressed automatically by the |\includeonly| directive
and thus should normally be avoided.
A method to include some content in the main file
by means of conditional processing is described in \secref{sec:conditional}.

%%%%%%%%%%%%%%%%%%%%%%%%%%%%%%%%%%%%%%%%
\paragraph{Page Numbering.}

When only a part of the document is compiled,
the appropriate numbering of pages
(as well as other status parameters)
is determined from the |.aux| files.
The latter contain information from previous passes.
However this information needs to propagate through
all intermediate child documents.
Therefore the page numbering in child documents may well
be inconsistent until the complete document is compiled at least once.

A useful (if unconventional) way to always ensure a consistent
page numbering is to restart the numbering in each child document
and denote the pages by `\textit{child}|.|\textit{page}'
where \textit{child} represents the chapter/section number of the child file.
This can be achieved by the command
|\numberwithin{page}{|\textit{child}|}|
of the \textsf{amsmath} package
where \textit{child} can be |chapter| or |section|
depending on the chosen structuring.
Alternatively, one can modify the macro |\thepage| appropriately
and reset the counter |page| at the start of each child file.

%%%%%%%%%%%%%%%%%%%%%%%%%%%%%%%%%%%%%%%%%%%%%%%%%%%%%%%%%%%%%%%%%%%%%%%%%%%%%%%%
\subsection{Conditional Processing}
\label{sec:conditional}

The package provides a mechanism to compile different versions
of a document. To customise the versions further some conditional processing
can come in handy to distinguish which version is being compiled.
The package provides two macros to describe the compilation context:

%%%%%%%%%%%%%%%%%%%%%%%%%%%%%%%%%%%%%%%%
\DescribeMacro{\ifchilddoc}
The conditional |\ifchilddoc| distinguishes between the compilation of
child documents and the main document:
%
\begin{center}
|\ifchilddoc |\textit{child-code}| |[|\||else |\textit{main-code}]| \||fi|
\end{center}

%%%%%%%%%%%%%%%%%%%%%%%%%%%%%%%%%%%%%%%%
\DescribeMacro{\childdocname}
\DescribeMacro{\childdocjob}
The macro |\childdocname| contains the filename (without extension)
of the main or child file being processed.
Note that |\childdocjob| will always contain the name of the main file.

%%%%%%%%%%%%%%%%%%%%%%%%%%%%%%%%%%%%%%%%
\paragraph{Title Page.}

Conditional processing can be used to include a title or banner page
in the main document when proper precautions are taken.
Importantly, the code in the main file should ensure that the page counter
(as well as other status parameters which are stored in the |.aux| files)
takes the same value after the conditional processing.
Otherwise the page numbers may take divergent values
depending on which part is compiled.

For example, a title page could be declared by:
%
\begin{center}
\begin{tabular}{l}
|\ifchilddoc\||else|\\
|\addtocounter{page}{-1}|\\
\textit{code for title page}\\
|\newpage|\\
|\||fi|
\end{tabular}
\end{center}
%
A banner page for the child documents can be generated by:
%
\begin{center}
\begin{tabular}{l}
|\ifchilddoc|\\
|\addtocounter{page}{-1}|\\
\textit{code for banner page}\\
|\newpage|\\
|\||fi|
\end{tabular}
\end{center}
%
Here one could write a message such as:
\begin{center}
|This is the part \childdocname{} of \childdocjob{}.|
\end{center}

%%%%%%%%%%%%%%%%%%%%%%%%%%%%%%%%%%%%%%%%%%%%%%%%%%%%%%%%%%%%%%%%%%%%%%%%%%%%%%%%
\subsection{Flags}
\label{sec:flags}

The package makes it easy to generate different versions
of the main or child documents.
To this end compilation flags can be defined
and assigned different default values.
They will be particularly useful in conjunction
with the forwarding mechanism described in \secref{sec:forward}.

For example, it may be useful to have a flag |\version|
which can be set to |draft| or |final|.
The document source will contain some conditional code
depending on the value of |\version|.
Suppose further, the flag should default to |final| for the main file
and to |draft| for child files
which is a natural assignment for editing the document.
This is achieved by placing the following code
in the preamble of the main document
(below the |\childdocmain| directive):
%
\begin{center}
\begin{tabular}{l}
|\ifchilddoc|\\
|\providecommand{\version}{draft}|\\
|\||else|\\
|\providecommand{\version}{final}|\\
|\||fi|
\end{tabular}
\end{center}
%
The definition by |\providecommand| makes sure
that previous definitions are not overwritten.
Further statements |\providecommand{\version}{...}|
can thus be added before the above code to override it.

For the main file, one might add a line
(between |\childdocmain| and the above block)
%
\begin{center}
|%\ifchilddoc\||else\providecommand{\version}{draft}\||fi|
\end{center}
%
which can be uncommented to produce a draft version.
Likewise one can add a line to the very top of a child file
(above the |\childdocof{|\textit{main}|}| directive)
%
\begin{center}
|%\providecommand{\version}{final}|
\end{center}
%
which can be uncommented to produce the final version of this child document.

%%%%%%%%%%%%%%%%%%%%%%%%%%%%%%%%%%%%%%%%%%%%%%%%%%%%%%%%%%%%%%%%%%%%%%%%%%%%%%%%
\subsection{Forwarding}
\label{sec:forward}

Different versions of the main or child documents
using compilation flags as described in \secref{sec:flags}
can be (permanently) stored in different files
for convenient compilation, viewing and distribution.
To this end, the package defines a command
to pass on compilation to a different file:

%%%%%%%%%%%%%%%%%%%%%%%%%%%%%%%%%%%%%%%%
\DescribeMacro{\childdocforward}
The command |\childdocforward| redirects processing to
another source file:
%
\begin{center}
\begin{tabular}{l}
|\input{childdoc.def}|\\
|\childdocforward[|\textit{main}|]{|\textit{dest}|}|\\
\end{tabular}
\end{center}
%
The argument \textit{dest} is the destination file
(without extension).
It should be the main file or one of the child files.
Note that further \textsf{childdoc} directives
such as |\childdocof| and |\childdocforward|
in the indicated file will be processed in this form.
The optional argument \textit{main}
passes on directly to the main file \textit{main}
while pretending to compile the child \textit{dest}.
This form behaves as if \textit{dest}
issues |\childdocof{|\textit{main}|}| right away,
and no further \textsf{childdoc} directives will be processed.

%%%%%%%%%%%%%%%%%%%%%%%%%%%%%%%%%%%%%%%%
\DescribeMacro{\...prefix}
In the alternative form |\childdocforwardprefix|,
%
\begin{center}
\begin{tabular}{l}
|\input{childdoc.def}|\\
|\childdocforwardprefix[|\textit{main}|]{|\textit{prefix}|}{|\textit{dest}|}|
\end{tabular}
\end{center}
%
the destination file is determined by a pattern
depending on the current file:
To make this work, the current file must be called
`{\textit{prefix}\hspace{0.2em}\textit{suffix}}'
with \textit{prefix} matching precisely the argument.
Processing is then passed on to the file
`{\textit{dest}\hspace{0.2em}\textit{suffix}}'.
Surely, the same effect is achieved by
directly specifying the
argument `{\textit{dest}\hspace{0.2em}\textit{suffix}}'
in the first form.
However, that requires to set up a different file
for each child. With the alternative form of the command
all these files can have exactly the same content
which simplifies setting them up and maintaining them.

For example, the following file |draft.tex|
with a compilation flag |\version| as described in \secref{sec:flags}
compiles the main document as a draft:
%
\begin{center}
\begin{tabular}{l}
|\def\version{draft}|\\
|\input{childdoc.def}|\\
|\childdocforward{|\textit{main}|}|
\end{tabular}
\end{center}
%
Likewise, the following files |final|\textit{nn}|.tex|
compile the final version of the child document
|child|\textit{nn}|.tex|:
%
\begin{center}
\begin{tabular}{l}
|\def\version{final}|\\
|\input{childdoc.def}|\\
|\childdocforwardprefix{final}{child}|
\end{tabular}
\end{center}
%

Note that when several versions of a main file and/or of each child file
are to be generated, it may be convenient to set up a |Makefile| or
shell script to automatise the process.

%%%%%%%%%%%%%%%%%%%%%%%%%%%%%%%%%%%%%%%%%%%%%%%%%%%%%%%%%%%%%%%%%%%%%%%%%%%%%%%%
\subsection{Command Line Processing}
\label{sec:commandline}

The effect of redirection files can also be achieved by invoking
the \LaTeX{} compiler with a more elaborate command line.
Most conveniently this should be done as part
of a shell script or a |Makefile|.

When using \textsf{childdoc} in the main file, the following
command lines effectively perform a redirection
(note that depending on the shell being used,
backslashes may have to be doubled: `|\|' $\to$ `|\\|'):
%
\begin{center}
|... -jobname "|\textit{target}|" |\\|"|[\textit{flags}]%
|\input{childdoc.def}\childdocforward[|\textit{main}|]{|\textit{dest}|}"|
\end{center}
%
Here \textit{target} is the name of the output file,
\textit{main} is the name of the main file
and \textit{dest} is the name of the main or child file to be processed
(all filenames without extensions).
The optional argument \textit{main} can be omitted
if \textit{main} matches \textit{dest}.
Optionally, compilation \textit{flags} can be defined via |\def| commands.
This command line makes the \TeX{} engine believe
it is compiling the file \textit{target}
whose content is specified as the latter parameter.
The provided code then forwards the processing to
\textit{main} or \textit{dest} as described in \secref{sec:forward}.

%%%%%%%%%%%%%%%%%%%%%%%%%%%%%%%%%%%%%%%%%%%%%%%%%%%%%%%%%%%%%%%%%%%%%%%%%%%%%%%%
\subsection{Include by Input}
\label{sec:input}

Including child documents by |\include| has some restrictions by design.
Most notably, the content of a child document always occupies
its own set of pages; pages cannot be shared between child documents.
Usually, this behaviour makes perfect sense
because each child document contain an essential part of the document.
However, in some situations it may be desirable to compose
a document from a collection of parts
without having mandatory page breaks between then.
For this case, the package
provides a mechanism to include parts
by |\input| which can also be processed individually.
However, by construction this mechanism
requires manual handling of the content to be output.

%%%%%%%%%%%%%%%%%%%%%%%%%%%%%%%%%%%%%%%%
\DescribeMacro{\ifchilddocmanual}
The main file should be prepared as usual, see \secref{sec:include}.
However, the document body must make a distinction
between processing of an individual part and of the main document, e.g.:
%
\begin{center}
\begin{tabular}{l}
|\ifchilddocmanual|\\
|\input{\childdocname}|\\
|\||else|\\
\textit{document body with }|\input{|\textit{part}|}|\\
|\||fi|
\end{tabular}
\end{center}
%
The conditional |\ifchilddocmanual| is true whenever
a part to be included by |\input| is being compiled,
and the name of the part is stored in |\childdocname|.

%%%%%%%%%%%%%%%%%%%%%%%%%%%%%%%%%%%%%%%%
\DescribeMacro{\childdocby}
Each part to be included by |\input| should start with:
%
\begin{center}
\begin{tabular}{l}
|\input{childdoc.def}|\\
|\childdocby{|\textit{main}|}|\\
\end{tabular}
\end{center}
%
The directive |\childdocby| is similar to |\childdocof|
described in \secref{sec:include},
but the subsequent selection of content must be done manually.
To that end, both |\ifchilddoc| and |\ifchilddocmanual|
will be true upon processing of a part,
and the name of the part is stored in |\childdocname|.
Note that |\jobname| will be set to the filename of the current part
so that each part receives an individual |.aux| file
that does not interfere with the |.aux| file(s) of the main document.
This behaviour can be altered by the alternative form
|\childdocby[*]{|\textit{main}|}| (with a non-empty optional argument)
which uses the |.aux| file of the main document
by setting |\jobname| to \textit{main}.

%%%%%%%%%%%%%%%%%%%%%%%%%%%%%%%%%%%%%%%%%%%%%%%%%%%%%%%%%%%%%%%%%%%%%%%%%%%%%%%%
\subsection{Driver Development}
\label{sec:driver}

The \textsf{childdoc} mechanism can also be use for the development
of definition files such as \LaTeX{} styles or classes.
This case differs from the above setup with multiple parts
included by |\include| in that no |\includeonly| should be invoked.
This can be achieved by starting the include file
(before |\ProvidesPackage|) with:
%
\begin{center}
\begin{tabular}{l}
|\input{childdoc.def}|\\
|\childdocforward{|\textit{main}|}|\\
\end{tabular}
\end{center}
%
or alternatively with:
%
\begin{center}
\begin{tabular}{l}
|\input{childdoc.def}|\\
|\childdocby{|\textit{main}|}|\\
\end{tabular}
\end{center}
%
Both forms have slightly different effects as described above.
The main file is prepared as usual, see \secref{sec:include}.

%%%%%%%%%%%%%%%%%%%%%%%%%%%%%%%%%%%%%%%%%%%%%%%%%%%%%%%%%%%%%%%%%%%%%%%%%%%%%%%%
\subsection{Legacy Detection}
\label{sec:detection}

The directive |\childdocmain| in the main file can detect
whether the complete document or merely a child is to be compiled
even without using the directive |\childdocof|.
This method is deprecated because it is less robust
and there is no compelling reason to use it;
it is merely provided for backward compatibility
and it may be removed in future versions.

If the detection mechanism is to be used,
it is mandatory to correctly specify
the filename of the main file as the argument of |\childdocmain|:
%
\begin{center}
\begin{tabular}{l}
|\input{childdoc.def}|\\
|\childdocmain{|\textit{main}|}|\\
\end{tabular}
\end{center}
%
If |\jobname| does not match the argument \textit{main} of |\childdocmain|,
it is assumed that |\jobname| points to the child file to be compiled.
When using |\childdocmain| with the main file specified as argument,
it suffices to start a child file
with just |\input{|\textit{main}|}|
without loading of the package and using |\childdocof|.
If instead all processing is done
with the appropriate \textsf{childdoc} directives,
the argument of \textit{main} of |\childdocmain| can be empty.

An alternative version of the command line processing described
in \secref{sec:commandline} using the detection mechanism reads:
%
\begin{center}
|... -jobname "|\textit{target}|" "|[\textit{flags}]%
[|\def\jobname{|\textit{dest}|}|]|\input{|\textit{main}|}"|
\end{center}

%%%%%%%%%%%%%%%%%%%%%%%%%%%%%%%%%%%%%%%%%%%%%%%%%%%%%%%%%%%%%%%%%%%%%%%%%%%%%%%%
\subsection{Manual Code}
\label{sec:manual}

In case one cannot be certain whether the definitions file |childdoc.def|
is installed on the target \TeX{} distribution
and one prefers not to ship it,
it is conceivable to paste a few relevant commands into the sources.

To that end, drop all statements |\input{childdoc.def}|
and perform the replacements as outlined below.
Instead of |\childdocmain{|\textit{main}|}| add the following code
to the top of the main file:
%
\begin{center}
\begin{tabular}{l}
|\||ifdefined\childdocname\endinput\||fi\newif\ifchilddoc|\\
|\edef\childdocname{\scantokens\expandafter{\jobname\noexpand}}|\\
|\def\childdocmain{|\textit{main}|}\||ifx\childdocmain\childdocname\||else|\\
|\childdoctrue\includeonly{\childdocname}\let\jobname\childdocmain\||fi|\\
\end{tabular}
\end{center}
%
Instead of |\childdocof{|\textit{main}|}| just include the main file
at the top of each child file:
%
\begin{center}
|\input{|\textit{main}|}|
\end{center}
%
A simple redirection |\childdocforward{|\textit{dest}|}| is achieved by:
%
\begin{center}
|\def\jobname{|\textit{dest}|}\input{\jobname}|
\end{center}
%
The redirection with prefix
|\childdocforwardprefix[|\textit{prefix}|]{|\textit{dest}|}|
is accomplished by:
%
\begin{center}
\begin{tabular}{l}
|{\edef\jobname{\scantokens\expandafter{\jobname\noexpand}}|\\
|\def\redirectjob |\textit{prefix}|#1~~~{\gdef\jobname{|\textit{dest}|#1}}|\\
|\expandafter\redirectjob\jobname~~~}\input{\jobname}|
\end{tabular}
\end{center}

In an alternative approach,
child documents can be compiled by a specific command line
without additional code or specific definitions:
%
\begin{center}
|... -jobname "|\textit{target}|" "|[\textit{flags}]%
|\includeonly{|\textit{dest}|}\input{|\textit{main}|}"|
\end{center}
%

%%%%%%%%%%%%%%%%%%%%%%%%%%%%%%%%%%%%%%%%%%%%%%%%%%%%%%%%%%%%%%%%%%%%%%%%%%%%%%%%
%%%%%%%%%%%%%%%%%%%%%%%%%%%%%%%%%%%%%%%%%%%%%%%%%%%%%%%%%%%%%%%%%%%%%%%%%%%%%%%%
\section{Information}

%%%%%%%%%%%%%%%%%%%%%%%%%%%%%%%%%%%%%%%%%%%%%%%%%%%%%%%%%%%%%%%%%%%%%%%%%%%%%%%%
\subsection{Copyright}

Copyright \copyright{} 2017--2018 Niklas Beisert

This work may be distributed and/or modified under the
conditions of the \LaTeX{} Project Public License, either version 1.3
of this license or (at your option) any later version.
The latest version of this license is in
  \url{http://www.latex-project.org/lppl.txt}
and version 1.3 or later is part of all distributions of \LaTeX{}
version 2005/12/01 or later.

This work has the LPPL maintenance status `maintained'.

The Current Maintainer of this work is Niklas Beisert.

This work consists of the files |README.txt|, |childdoc.ins| and |childdoc.dtx|
as well as the derived files |childdoc.def|, |cdocsamp.tex|
with |cdocsch1.tex|, |cdocsch2.tex|, |cdocspt3.tex|, |cdocspt4.tex|,
|cdocsdrf.tex|, |cdocsfn1.tex|, |cdocsfn2.tex|
as well as |childdoc.pdf|.

%%%%%%%%%%%%%%%%%%%%%%%%%%%%%%%%%%%%%%%%%%%%%%%%%%%%%%%%%%%%%%%%%%%%%%%%%%%%%%%%
\subsection{Files and Installation}

The package consists of the files:
%
\begin{center}
\begin{tabular}{ll}
    |README.txt|   & readme file \\
    |childdoc.ins| & installation file \\
    |childdoc.dtx| & source file \\
    |childdoc.def| & definition file \\
    |cdocsamp.tex| & sample main file \\
    |cdocsch1.tex| & sample include file \\
    |cdocsch2.tex| & sample include file \\
    |cdocspt3.tex| & sample part file \\
    |cdocspt4.tex| & sample part file \\
    |cdocsdrf.tex| & sample redirection file \\
    |cdocsfn1.tex| & sample redirection file \\
    |cdocsfn2.tex| & sample redirection file \\
    |childdoc.pdf| & manual
\end{tabular}
\end{center}
%
The distribution consists of the files
|README.txt|, |childdoc.ins| and |childdoc.dtx|.
%
\begin{itemize}
\item
Run (pdf)\LaTeX{} on |childdoc.dtx|
to compile the manual |childdoc.pdf| (this file).
\item
Run \LaTeX{} on |childdoc.ins| to create the definitions file |childdoc.def|
and the sample |cdocsamp.tex| with include files
|cdocsch1.tex|, |cdocsch2.tex|, |cdocspt3.tex|, |cdocspt4.tex|,
|cdocsdrf.tex|, |cdocsfn1.tex|, |cdocsfn2.tex|.
Then copy the file |childdoc.def| to an appropriate directory of your \LaTeX{}
distribution, e.g.\ \textit{texmf-root}|/tex/latex/childdoc|.
\end{itemize}

%%%%%%%%%%%%%%%%%%%%%%%%%%%%%%%%%%%%%%%%%%%%%%%%%%%%%%%%%%%%%%%%%%%%%%%%%%%%%%%%
\subsection{Related CTAN Packages}

There are several other packages which offer a similar functionality:
%
\begin{itemize}
\item
The packages
\href{http://ctan.org/pkg/docmute}{\textsf{docmute}},
\href{http://ctan.org/pkg/includex}{\textsf{includex}} and
\href{http://ctan.org/pkg/standalone}{\textsf{standalone}}
provide commands to include only the document body of
a child file thus allowing both files to be compiled individually.
\item
The packages \href{http://ctan.org/pkg/subdocs}{\textsf{subdocs}}
and \href{http://ctan.org/pkg/subfiles}{\textsf{subfiles}}
provide structures in which the main and child documents can be
encapsulated and allowing them to be compiled individually.
The inclusion mechanism is different from the conventional |\include|.
\item
The package \href{http://ctan.org/pkg/combine}{\textsf{combine}}
is an elaborate solution to combine several documents into one.
\end{itemize}
%
See also the CTAN topic \href{http://ctan.org/topic/subdocs}{\textsf{subdocs}}
for further related packages.
The present package differs from the above solutions in that
a document structure constructed with the conventional |\include| mechanism
just needs two extra commands at the top of every file
such that all constituent files can be compiled individually.

%%%%%%%%%%%%%%%%%%%%%%%%%%%%%%%%%%%%%%%%%%%%%%%%%%%%%%%%%%%%%%%%%%%%%%%%%%%%%%%%
%\subsection{Feature Suggestions}
%
%The following is a list of features which may be useful for future
%versions of this package:
%%
%\begin{itemize}
%\item
%\ldots
%\end{itemize}

%%%%%%%%%%%%%%%%%%%%%%%%%%%%%%%%%%%%%%%%%%%%%%%%%%%%%%%%%%%%%%%%%%%%%%%%%%%%%%%%
\subsection{Revision History}

%%%%%%%%%%%%%%%%%%%%%%%%%%%%%%%%%%%%%%%%
\paragraph{v2.0:} 2018/12/30

\begin{itemize}
\item
immediate forward processing
\item
added |\childdocby| mechanism
\item
manual restructured
\end{itemize}

%%%%%%%%%%%%%%%%%%%%%%%%%%%%%%%%%%%%%%%%
\paragraph{v1.6:} 2018/01/17

\begin{itemize}
\item
application for development of include files
\item
corrections to manual
\end{itemize}

%%%%%%%%%%%%%%%%%%%%%%%%%%%%%%%%%%%%%%%%
\paragraph{v1.5:} 2017/05/21

\begin{itemize}
\item
more complete structuring introduced
\item
|\childdocof| introduced
\item
|\childdoc| renamed to |\childdocmain|
\item
|\childredirect| renamed to |\childdocforward| and |\childdocforwardprefix|
and functionality expanded
\end{itemize}

%%%%%%%%%%%%%%%%%%%%%%%%%%%%%%%%%%%%%%%%
\paragraph{v1.0:} 2017/04/27

\begin{itemize}
\item
manual and install package
\item
first version published on CTAN
\end{itemize}

%%%%%%%%%%%%%%%%%%%%%%%%%%%%%%%%%%%%%%%%
\paragraph{v0.6:} 2017/04/26

\begin{itemize}
\item
redirection mechanism added
\end{itemize}

%%%%%%%%%%%%%%%%%%%%%%%%%%%%%%%%%%%%%%%%
\paragraph{v0.5:} 2017/04/26

\begin{itemize}
\item
functionality in definition file
\end{itemize}


%%%%%%%%%%%%%%%%%%%%%%%%%%%%%%%%%%%%%%%%%%%%%%%%%%%%%%%%%%%%%%%%%%%%%%%%%%%%%%%%
%%%%%%%%%%%%%%%%%%%%%%%%%%%%%%%%%%%%%%%%%%%%%%%%%%%%%%%%%%%%%%%%%%%%%%%%%%%%%%%%
%%%%%%%%%%%%%%%%%%%%%%%%%%%%%%%%%%%%%%%%%%%%%%%%%%%%%%%%%%%%%%%%%%%%%%%%%%%%%%%%
\appendix

\settowidth\MacroIndent{\rmfamily\scriptsize 000\ }

 \DocInput{childdoc.dtx}

\end{document}
%</driver>
% \fi
%
% %%%%%%%%%%%%%%%%%%%%%%%%%%%%%%%%%%%%%%%%%%%%%%%%%%%%%%%%%%%%%%%%%%%%%%%%%%%%%%
% %%%%%%%%%%%%%%%%%%%%%%%%%%%%%%%%%%%%%%%%%%%%%%%%%%%%%%%%%%%%%%%%%%%%%%%%%%%%%%
% \section{Sample}
%\iffalse
%<*samplemain>
%\fi
%
% The following presents a sample document
% with two chapters, two parts, a title page,
% a compile flag as well as three forwarding files to set the flag.
% It consists of eight |.tex| files:
% \begin{center}
% \begin{tabular}{ll}
% |cdocsamp.tex|&main file\\
% |cdocsch1.tex|&include file for chapter 1\\
% |cdocsch2.tex|&include file for chapter 2\\
% |cdocspt3.tex|&include file for part 3\\
% |cdocspt4.tex|&include file for part 4\\
% |cdocsdrf.tex|&forwarding file for main file in draft mode\\
% |cdocsfi1.tex|&forwarding file for final version of chapter 1\\
% |cdocsfi2.tex|&forwarding file for final version of chapter 2\\
% \end{tabular}
% \end{center}
% Each of the eight files can be compiled directly by the \LaTeX{} compiler.
%
% %%%%%%%%%%%%%%%%%%%%%%%%%%%%%%%%%%%%%%
% \paragraph{Main File.}
%
% The main file is called |cdocsamp.tex|.
%
% Load the \textsf{childdoc} definitions and
% declare the filename for the main document:
%    \begin{macrocode}
\input{childdoc.def}
\childdocmain{}
%    \end{macrocode}

% Optional override for |\version| flag:
%    \begin{macrocode}
%%\ifchilddoc\else\providecommand{\version}{draft}\fi
%    \end{macrocode}

% Define the default values for the |\version| flag
% (|final| for the main file and |draft| for childs):
%    \begin{macrocode}
\ifchilddoc
\providecommand{\version}{draft}
\else
\providecommand{\version}{final}
\fi
%    \end{macrocode}

% Load the standard document class:
%    \begin{macrocode}
\documentclass[12pt]{article}
%    \end{macrocode}

% Start the document body:
%    \begin{macrocode}
\begin{document}
%    \end{macrocode}

% Declare a title page.
% Print title, part of document being processed and version flag:
%    \begin{macrocode}
\addtocounter{page}{-1}
\begin{center}
{\LARGE\bfseries{}childdoc example\par}
\vspace{1cm}
\ifchilddoc
\ifchilddocmanual part\else chapter\fi:
`\childdocname' of `\childdocjob'\par
\else
main document: `\childdocjob'\par
\fi
version: \version\par
\end{center}
\newpage
%    \end{macrocode}

% Manually include selected file,
% otherwise process as usual:
%    \begin{macrocode}
\ifchilddocmanual
\section*{part `\childdocname'}
\input{\childdocname}
\else
%    \end{macrocode}

% Include the two chapters:
%    \begin{macrocode}
\include{cdocsch1}
\include{cdocsch2}
%    \end{macrocode}

% Include the two parts unless only chapters should be displayed:
%    \begin{macrocode}
\ifchilddoc\else
\section{part three}
\input{cdocspt3}
\section{part four}
\input{cdocspt4}
\fi
%    \end{macrocode}

% Process as usual until here:
%    \begin{macrocode}
\fi
%    \end{macrocode}

% End of document body:
%    \begin{macrocode}
\end{document}
%    \end{macrocode}
%\iffalse
%</samplemain>
%\fi
%
% %%%%%%%%%%%%%%%%%%%%%%%%%%%%%%%%%%%%%%
% \paragraph{Chapter Include Files.}
%
% The include files are called |cdocsch1.tex| and |cdocsch2.tex|.
%
%\iffalse
%<*samplechap1|samplechap2>
%\fi

% Optional override for |\version| flag:
%    \begin{macrocode}
%%\providecommand{\version}{final}
%    \end{macrocode}

% Include the main document:
%    \begin{macrocode}
\input{childdoc.def}
\childdocof{cdocsamp}
%    \end{macrocode}

%\iffalse
%</samplechap1|samplechap2>
%\fi
%
%\iffalse
%<*samplechap1>
%\fi
% Some text for chapter 1:
%    \begin{macrocode}
\section{one}
some text in chapter one
%    \end{macrocode}

%\iffalse
%</samplechap1>
%\fi
% Some text for chapter 2:
%\iffalse
%<*samplechap2>
%\fi
%    \begin{macrocode}
\section{two}
more text in chapter two
%    \end{macrocode}

%\iffalse
%</samplechap2>
%\fi
%
% %%%%%%%%%%%%%%%%%%%%%%%%%%%%%%%%%%%%%%
% \paragraph{Part Include Files.}
%
% The include files are called |cdocspt3.tex| and |cdocspt4.tex|.
%
%\iffalse
%<*samplepart3|samplepart4>
%\fi

% Optional override for |\version| flag:
%    \begin{macrocode}
%%\providecommand{\version}{final}
%    \end{macrocode}

% Include the main document:
%    \begin{macrocode}
\input{childdoc.def}
\childdocby{cdocsamp}
%    \end{macrocode}

%\iffalse
%</samplepart3|samplepart4>
%\fi
%
%\iffalse
%<*samplepart3>
%\fi
% Some text for part 3:
%    \begin{macrocode}
some text in part three
%    \end{macrocode}

%\iffalse
%</samplepart3>
%\fi
% Some text for part 4:
%\iffalse
%<*samplepart4>
%\fi
%    \begin{macrocode}
more text in part four
%    \end{macrocode}

%\iffalse
%</samplepart4>
%\fi
%
% %%%%%%%%%%%%%%%%%%%%%%%%%%%%%%%%%%%%%%
% \paragraph{Forwarding for a Complete Draft.}
%
% The following forwarding file |cdocsdrf.tex|
% compiles the main document in draft mode:
%\iffalse
%<*sampledraft>
%\fi
%    \begin{macrocode}
\def\version{draft}
\input{childdoc.def}
\childdocforward{cdocsamp}
%    \end{macrocode}

%\iffalse
%</sampledraft>
%\fi
%
% %%%%%%%%%%%%%%%%%%%%%%%%%%%%%%%%%%%%%%
% \paragraph{Forwarding for Final Version of the Chapters.}
%
% The following forwarding files |cdocsfn1.tex| and |cdocsfn2.tex|
% (with identical content)
% compile the final versions of the child documents
% |cdocsch1.tex| and |cdocsch2.tex|, respectively:
%\iffalse
%<*samplefinal>
%\fi
%    \begin{macrocode}
\def\version{final}
\input{childdoc.def}
\childdocforwardprefix[cdocsamp]{cdocsfn}{cdocsch}
%    \end{macrocode}

%\iffalse
%</samplefinal>
%\fi
%
% %%%%%%%%%%%%%%%%%%%%%%%%%%%%%%%%%%%%%%
% \paragraph{Command Line Processing.}
%
% The following three command lines generate the output files
% |cdocscld|, |cdocscl1| and |cdocscl2|
% which should be identical to
% |cdocsdrf|, |cdocsch1| and |cdocsfn2|, respectively:
% \begin{center}
% \begin{tabular}{l}
% |latex -jobname cdocscld \|\\
% |  "\def\version{draft}\input{childdoc.def}\childdocforward{cdocsamp}"|\\
% |latex -jobname cdocscl1 \|\\
% |  "\input{childdoc.def}\childdocforward[cdocsamp]{cdocsch1}"|\\
% |latex -jobname cdocscl2 \|\\
% |  "\def\version{final}\input{childdoc.def}\childdocforward{cdocsch2}"|
% \end{tabular}
% \end{center}
% Note that the trailing backslash on each first line
% merely continues the input to the second line
% (for convenient cut ant paste).
% Furthermore, the command |latex| can be replaced by any
% of its alternative versions such as |pdflatex|.
%
% %%%%%%%%%%%%%%%%%%%%%%%%%%%%%%%%%%%%%%%%%%%%%%%%%%%%%%%%%%%%%%%%%%%%%%%%%%%%%%
% %%%%%%%%%%%%%%%%%%%%%%%%%%%%%%%%%%%%%%%%%%%%%%%%%%%%%%%%%%%%%%%%%%%%%%%%%%%%%%
% \section{Implementation}
%\iffalse
%<*package>
%\fi
%
% This section describes the definitions file |childdoc.def|.

% The definitions cannot be loaded using |\usepackage| or |\RequirePackage|
% which has a mechanism to prevent loading a style file more than once.
% When loading the definitions by means of |\input|
% multiple instances have to be prevented manually:
%\iffalse
%This code needs to be before the `\ProvidesFile' directive
%which is defined at the beginning of this file.
%Therefore it is also placed there and commented out here.
%</package>
%<*discard>
%\fi
%    \begin{macrocode}
\ifdefined\childdocmain\endinput\fi
%    \end{macrocode}
%\iffalse
%</discard>
%<*package>
%\fi
%
% \macro{\ifchilddoc}
% \macro{\ifchilddocmanual}
% The conditional |\ifchilddoc| tells whether a
% child (true) or main (false) document is being compiled.
% The conditional |\ifchilddocmanual| tells whether
% the |\includeonly| mechanism is used (false) or
% the selection of child files must be performed manually (true).
% The definitions initialise to false:
%    \begin{macrocode}
\newif\ifchilddoc
\newif\ifchilddocmanual
%    \end{macrocode}

% \macro{\childdocname}
% \macro{\childdocjob}
% The macro |\childdocname| stores the name of the main document
% to be compiled. The macro |\childdocjob| stores the name of
% the document on which the \LaTeX{} compiler was originally invoked.
% The content of |\jobname| cannot be compared
% to filenames specified in the source due to different catcodes.
% The following code rescans |\jobname|, stores the result
% in |\childdocname| and saves a copy in |\childdocjob|:
%    \begin{macrocode}
\edef\childdocname{\scantokens\expandafter{\jobname\noexpand}}
\let\childdocjob\childdocname
%    \end{macrocode}

% \macro{\childdocdisable}
% The macro |\childdocdisable| prevents the main file
% from being processed more than once.
% At this stage, the main document command |\childdocmain|
% is assumed to be called once again where it should do nothing.
% Any subsequent call to it should prevent
% a secondary processing of the main document
% It overwrites the forwarding commands
% |\childdocof| and |\childdocforward|
% with empty macros to prevent further inclusions of the main document:
%    \begin{macrocode}
\newcommand{\childdocdisable}
{
  \renewcommand{\childdocmain}[1]{\renewcommand{\childdocmain}[1]{\endinput}}
  \renewcommand{\childdocof}[1]{}
  \renewcommand{\childdocby}[2][]{}
  \renewcommand{\childdocforward}[2][]{}
  \renewcommand{\childdocdisable}{}
}
%    \end{macrocode}

% \macro{\childdocmain}
% The macro |\childdocmain| is to be called at the top of the main file
% with nothing or the main filename (without extension) as argument.
% First, it breaks loops.
% If the argument is not empty and does not match |\childdocname|
% (which is set by the first inclusion of |childdoc.def|),
% |\ifchilddoc| is set to true, |\includeonly| is applied to the child file
% and |\jobname| is set to the main file
% (for proper handling of |.aux| files):
%    \begin{macrocode}
\newcommand{\childdocmain}[1]
{
  \childdocdisable\childdocmain{}
  \if?#1?\else
    \begingroup
      \def\childdoctmp{#1}
      \ifx\childdoctmp\childdocname
        \def\childdoctmp{}
      \else
        \def\childdoctmp
        {
          \childdoctrue
          \includeonly{\childdocname}
          \def\childdocjob{#1}
          \def\jobname{#1}
        }
      \fi
      \expandafter
    \endgroup
    \childdoctmp
  \fi
}
%    \end{macrocode}

% \macro{\childdocof}
% The command |\childdocof| redirects
% compilation to the main file |#1|.
%    \begin{macrocode}
\newcommand{\childdocof}[1]
{
  \childdocdisable
  \childdoctrue
  \includeonly{\childdocname}
  \def\jobname{#1}
  \def\childdocjob{#1}
  \input{#1}
}
%    \end{macrocode}

% \macro{\childdocby}
% The command |\childdocby| ....
%    \begin{macrocode}
\newcommand{\childdocby}[2][]
{
  \childdocdisable
  \childdoctrue
  \childdocmanualtrue
  \if?#1?\else
    \def\jobname{#2}
  \fi
  \def\childdocjob{#2}
  \input{#2}
  \endinput
}
%    \end{macrocode}

% \macro{\childdocforward}
% The command |\childdocforward| redirects
% compilation to the main file or
% (if the optional argument is given) a child file.
% Parameters are set as if the main file
% or a child file starting with |\childdocof| was compiled.
% Then compilation is handed over to the main file:
%    \begin{macrocode}
\newcommand{\childdocforward}[2][]
{
  \begingroup
    \if?#1?
      \def\childdoctmp
      {
        \def\childdocname{#2}
        \def\childdocjob{#2}
        \def\jobname{#2}
        \input{#2}
        \endinput
      }
    \else
      \def\childdoctmp
      {
        \childdocdisable
        \def\childdocname{#2}
        \childdoctrue
        \includeonly{#2}
        \def\childdocjob{#1}
        \def\jobname{#1}
        \input{#1}
        \endinput
      }
    \fi
    \expandafter
  \endgroup
  \childdoctmp
}
%    \end{macrocode}

% \macro{\childdocforwardprefix}
% The command |\childdocforwardprefix| redirects
% compilation to the main or a child file by means of a pattern.
% The prefix |#1| in the current filename is replaced by |#2|
% and the suffix of the current filename is kept
% (it is assumed that the filename does not contain the substring `|~~~|'
% which is used as a delimiter).
% Compilation is handed over to the new file by |\childdocforward|:
%    \begin{macrocode}
\newcommand{\childdocforwardprefix}[3][]
{
  \begingroup
    \def\childdocextract #2##1~~~{\def\childdoctmp{\childdocforward[#1]{#3##1}}}
    \expandafter\childdocextract\childdocname~~~
    \expandafter
  \endgroup
  \childdoctmp
}
%    \end{macrocode}

% \macro{\childdoc}
% The deprecated macro |\childdoc| is a legacy version of |\childdocmain|:
%    \begin{macrocode}
\newcommand{\childdoc}{\childdocmain}
%    \end{macrocode}

% \macro{\childdocredirect}
% The deprecated macro |\childdocredirect| is a legacy version
% of |\childdocforward| and |\childdocforwardprefix|:
%    \begin{macrocode}
\newcommand{\childdocredirect}[2][]
{
  \begingroup
    \if?#1?
      \def\childdoctmp{\childdocforward{#2}}
    \else
      \def\childdoctmp{\childdocforwardprefix{#1}{#2}}
    \fi
    \expandafter
  \endgroup
  \childdoctmp
}
%    \end{macrocode}

%\iffalse
%</package>
%\fi
%
\endinput
\childdocforward[cdocsamp]{cdocsch1}"|\\
% |latex -jobname cdocscl2 \|\\
% |  "\def\version{final}% \iffalse
%
% childdoc.dtx Copyright (C) 2017-2018 Niklas Beisert
%
% This work may be distributed and/or modified under the
% conditions of the LaTeX Project Public License, either version 1.3
% of this license or (at your option) any later version.
% The latest version of this license is in
%   http://www.latex-project.org/lppl.txt
% and version 1.3 or later is part of all distributions of LaTeX
% version 2005/12/01 or later.
%
% This work has the LPPL maintenance status `maintained'.
%
% The Current Maintainer of this work is Niklas Beisert.
%
% This work consists of the files childdoc.dtx and childdoc.ins
% and the derived files childdoc.def and cdocsamp.tex with
% cdocsch1.tex, cdocsch2.tex, cdocsdrf.tex, cdocsfn1.tex, cdocsfn2.tex.
%
%<package>\ifdefined\childdocmain\endinput\fi
%<package>\ProvidesFile{childdoc.def}[2018/12/30 v2.0 child document driver]
%<samplemain>\ProvidesFile{cdocsamp.tex}[2018/12/30 v2.0 sample for childdoc]
%<*driver>
%\ProvidesFile{childdoc.drv}[2018/12/30 v2.0 childdoc reference manual file]
\PassOptionsToClass{10pt,a4paper}{article}
\documentclass{ltxdoc}

\usepackage[margin=35mm]{geometry}
\usepackage{hyperref}
\usepackage{hyperxmp}
\usepackage[usenames]{color}

\hypersetup{colorlinks=true}
\hypersetup{pdfstartview=FitH}
\hypersetup{pdfpagemode=UseNone}
\hypersetup{pdfsource={}}
\hypersetup{pdflang={en-UK}}
\hypersetup{pdfcopyright={Copyright 2017-2018 Niklas Beisert.
  This work may be distributed and/or modified under the
  conditions of the LaTeX Project Public License, either version 1.3
  of this license or (at your option) any later version.}}
\hypersetup{pdflicenseurl={http://www.latex-project.org/lppl.txt}}
\hypersetup{pdfcontactaddress={ETH Zurich, ITP, HIT K,
  Wolfgang-Pauli-Strasse 27}}
\hypersetup{pdfcontactpostcode={8093}}
\hypersetup{pdfcontactcity={Zurich}}
\hypersetup{pdfcontactcountry={Switzerland}}
\hypersetup{pdfcontactemail={nbeisert@itp.phys.ethz.ch}}
\hypersetup{pdfcontacturl={http://people.phys.ethz.ch/\xmptilde nbeisert/}}

\newcommand{\secref}[1]{\hyperref[#1]{section \ref*{#1}}}

\parskip1ex
\parindent0pt
\let\olditemize\itemize
\def\itemize{\olditemize\parskip0pt}

\begin{document}

\title{The \textsf{childdoc} Package}
\hypersetup{pdftitle={The childdoc Package}}
\author{Niklas Beisert\\[2ex]
  Institut f\"ur Theoretische Physik\\
  Eidgen\"ossische Technische Hochschule Z\"urich\\
  Wolfgang-Pauli-Strasse 27, 8093 Z\"urich, Switzerland\\[1ex]
  \href{mailto:nbeisert@itp.phys.ethz.ch}
  {\texttt{nbeisert@itp.phys.ethz.ch}}}
\hypersetup{pdfauthor={Niklas Beisert}}
\hypersetup{pdfsubject={Manual for the LaTeX2e Package childdoc}}
\date{30 December 2018, \textsf{v2.0}}
\maketitle

\begin{abstract}\noindent
\textsf{childdoc} is a \LaTeXe{} package
that enables the direct compilation
of document sections included by |\include|
to individual files.
\end{abstract}

\begingroup
\parskip0ex
\tableofcontents
\endgroup

%%%%%%%%%%%%%%%%%%%%%%%%%%%%%%%%%%%%%%%%%%%%%%%%%%%%%%%%%%%%%%%%%%%%%%%%%%%%%%%%
%%%%%%%%%%%%%%%%%%%%%%%%%%%%%%%%%%%%%%%%%%%%%%%%%%%%%%%%%%%%%%%%%%%%%%%%%%%%%%%%
\section{Introduction}

\LaTeX{} provides a mechanism to structure a large document (such as a book)
into a main file and several child files (containing the chapters)
using the |\include| command.
This mechanism is beneficial for documents
which span hundreds of pages in order to
make the source file(s) more manageable.
Moreover, compilation can be restricted to
selected child files by means of the |\includeonly| command.
The latter feature can be used to reduce the compilation time while editing
(this was significantly more useful in the earlier days of \LaTeX{})
or to generate a smaller document which is easier to navigate.
Another application of |\includeonly| is to generate
documents consisting of selected parts of the complete document.

However, there are a few drawbacks of the plain |\include| mechanism:
\begin{itemize}
\item
The child files cannot be compiled on their own,
they can only be compiled via the main file.
A naive editing environment
(such as a text editor with an option
to have the current file processed by \LaTeX)
may require one to switch to the main file before compiling;
attempting to compile the child file produces errors.
\item
The main file must be modified (each time)
to adjust the |\includeonly| command
to the present needs. This easily leaves the main file in a messy state.
\item
The generated document will always carry the filename
of the main document. This is inconvenient if
several child files are to be compiled and
to be kept for distribution.
\end{itemize}

The present package provides a simple interface
to make child files individually compilable by \LaTeX{}.
Compiling a child file then has the same effect as compiling
the main file with an |\includeonly| command
to select the appropriate child.
Moreover the generated document will carry the name of the child
rather than the main file.
This resolves all three above issues.

This feature is meant to make the editing of books,
thesis documents and lecture notes somewhat more convenient.
However, the package can also be used efficiently for
composing a series of documents (such as exercise sheets)
which are typically distributed individually.
It then assists the author in generating the individual documents
(potentially in different versions)
as well as a document containing the collected series.
Another application is in developing style files
or other kinds of included material
where compilation of the style file could redirect
to a sample or test file.

%%%%%%%%%%%%%%%%%%%%%%%%%%%%%%%%%%%%%%%%%%%%%%%%%%%%%%%%%%%%%%%%%%%%%%%%%%%%%%%%
%%%%%%%%%%%%%%%%%%%%%%%%%%%%%%%%%%%%%%%%%%%%%%%%%%%%%%%%%%%%%%%%%%%%%%%%%%%%%%%%
\section{Usage}

First of all, the package \textsf{childdoc} is \emph{not} a standard
\LaTeXe{} |.sty| style file! Therefore it needs to be invoked in
a non-standard way.

%%%%%%%%%%%%%%%%%%%%%%%%%%%%%%%%%%%%%%%%%%%%%%%%%%%%%%%%%%%%%%%%%%%%%%%%%%%%%%%%
\subsection{Included Files}
\label{sec:include}

%%%%%%%%%%%%%%%%%%%%%%%%%%%%%%%%%%%%%%%%
\DescribeMacro{\childdocmain}
To use the package, add the commands
\begin{center}
\begin{tabular}{l}
|\input{childdoc.def}|\\
|\childdocmain{}|\\
\end{tabular}
\end{center}
at the very top of the main \LaTeX{} file,
in particular \emph{before} the |\documentclass| statement!
The argument of |\childdocmain| should be left empty
(but it must be present).

%%%%%%%%%%%%%%%%%%%%%%%%%%%%%%%%%%%%%%%%
\DescribeMacro{\childdocof}
Furthermore, add the commands
\begin{center}
\begin{tabular}{l}
|\input{childdoc.def}|\\
|\childdocof{|\textit{main}|}|\\
\end{tabular}
\end{center}
at the top of every child file \textit{child}
which is included by |\include{|\textit{child}|}|
from within the main file
(or at least for those files to be compiled individually).
The argument \textit{main} must be the filename of the main file.

There are a couple of
considerations in setting up the main and child documents:

%%%%%%%%%%%%%%%%%%%%%%%%%%%%%%%%%%%%%%%%
\paragraph{Restrictions.}

Please note the following restrictions:
\begin{itemize}
\item
|\childdocmain| must be called with one argument \textit{main}
to ensure compatibility with earlier version of the package.
It must either be empty (|\childdocmain{}|)
or precisely match the filename of the main file in which it is specified.
See \secref{sec:detection} for further information.
\item
The filename \textit{main} must be specified without the |.tex| extension.
\item
The filename \textit{main} is case sensitive
(even in case-insensitive file systems)
due to internal string comparison.
\item
The argument \textit{main} should be fully expanded, it cannot be a macro.
\item
Subdirectories and special characters should be avoided in filenames.
\item
The command |\childdocmain{|\textit{main}|}| must be followed by a whitespace.
It should not be followed immediately by another command
or by a comment mark `|%|'.
This is because the \TeX{} parser reads the token immediately following
the argument of |\childdocmain| and puts it
at the beginning of every child section;
however, a white\-space is ignored.
\end{itemize}

%%%%%%%%%%%%%%%%%%%%%%%%%%%%%%%%%%%%%%%%
\paragraph{Content of Main File.}

It is advisable to place all content in the child files included by |\include|.
Any output contained in the main file will appear in all child documents
unless suppressed manually;
it cannot be suppressed automatically by the |\includeonly| directive
and thus should normally be avoided.
A method to include some content in the main file
by means of conditional processing is described in \secref{sec:conditional}.

%%%%%%%%%%%%%%%%%%%%%%%%%%%%%%%%%%%%%%%%
\paragraph{Page Numbering.}

When only a part of the document is compiled,
the appropriate numbering of pages
(as well as other status parameters)
is determined from the |.aux| files.
The latter contain information from previous passes.
However this information needs to propagate through
all intermediate child documents.
Therefore the page numbering in child documents may well
be inconsistent until the complete document is compiled at least once.

A useful (if unconventional) way to always ensure a consistent
page numbering is to restart the numbering in each child document
and denote the pages by `\textit{child}|.|\textit{page}'
where \textit{child} represents the chapter/section number of the child file.
This can be achieved by the command
|\numberwithin{page}{|\textit{child}|}|
of the \textsf{amsmath} package
where \textit{child} can be |chapter| or |section|
depending on the chosen structuring.
Alternatively, one can modify the macro |\thepage| appropriately
and reset the counter |page| at the start of each child file.

%%%%%%%%%%%%%%%%%%%%%%%%%%%%%%%%%%%%%%%%%%%%%%%%%%%%%%%%%%%%%%%%%%%%%%%%%%%%%%%%
\subsection{Conditional Processing}
\label{sec:conditional}

The package provides a mechanism to compile different versions
of a document. To customise the versions further some conditional processing
can come in handy to distinguish which version is being compiled.
The package provides two macros to describe the compilation context:

%%%%%%%%%%%%%%%%%%%%%%%%%%%%%%%%%%%%%%%%
\DescribeMacro{\ifchilddoc}
The conditional |\ifchilddoc| distinguishes between the compilation of
child documents and the main document:
%
\begin{center}
|\ifchilddoc |\textit{child-code}| |[|\||else |\textit{main-code}]| \||fi|
\end{center}

%%%%%%%%%%%%%%%%%%%%%%%%%%%%%%%%%%%%%%%%
\DescribeMacro{\childdocname}
\DescribeMacro{\childdocjob}
The macro |\childdocname| contains the filename (without extension)
of the main or child file being processed.
Note that |\childdocjob| will always contain the name of the main file.

%%%%%%%%%%%%%%%%%%%%%%%%%%%%%%%%%%%%%%%%
\paragraph{Title Page.}

Conditional processing can be used to include a title or banner page
in the main document when proper precautions are taken.
Importantly, the code in the main file should ensure that the page counter
(as well as other status parameters which are stored in the |.aux| files)
takes the same value after the conditional processing.
Otherwise the page numbers may take divergent values
depending on which part is compiled.

For example, a title page could be declared by:
%
\begin{center}
\begin{tabular}{l}
|\ifchilddoc\||else|\\
|\addtocounter{page}{-1}|\\
\textit{code for title page}\\
|\newpage|\\
|\||fi|
\end{tabular}
\end{center}
%
A banner page for the child documents can be generated by:
%
\begin{center}
\begin{tabular}{l}
|\ifchilddoc|\\
|\addtocounter{page}{-1}|\\
\textit{code for banner page}\\
|\newpage|\\
|\||fi|
\end{tabular}
\end{center}
%
Here one could write a message such as:
\begin{center}
|This is the part \childdocname{} of \childdocjob{}.|
\end{center}

%%%%%%%%%%%%%%%%%%%%%%%%%%%%%%%%%%%%%%%%%%%%%%%%%%%%%%%%%%%%%%%%%%%%%%%%%%%%%%%%
\subsection{Flags}
\label{sec:flags}

The package makes it easy to generate different versions
of the main or child documents.
To this end compilation flags can be defined
and assigned different default values.
They will be particularly useful in conjunction
with the forwarding mechanism described in \secref{sec:forward}.

For example, it may be useful to have a flag |\version|
which can be set to |draft| or |final|.
The document source will contain some conditional code
depending on the value of |\version|.
Suppose further, the flag should default to |final| for the main file
and to |draft| for child files
which is a natural assignment for editing the document.
This is achieved by placing the following code
in the preamble of the main document
(below the |\childdocmain| directive):
%
\begin{center}
\begin{tabular}{l}
|\ifchilddoc|\\
|\providecommand{\version}{draft}|\\
|\||else|\\
|\providecommand{\version}{final}|\\
|\||fi|
\end{tabular}
\end{center}
%
The definition by |\providecommand| makes sure
that previous definitions are not overwritten.
Further statements |\providecommand{\version}{...}|
can thus be added before the above code to override it.

For the main file, one might add a line
(between |\childdocmain| and the above block)
%
\begin{center}
|%\ifchilddoc\||else\providecommand{\version}{draft}\||fi|
\end{center}
%
which can be uncommented to produce a draft version.
Likewise one can add a line to the very top of a child file
(above the |\childdocof{|\textit{main}|}| directive)
%
\begin{center}
|%\providecommand{\version}{final}|
\end{center}
%
which can be uncommented to produce the final version of this child document.

%%%%%%%%%%%%%%%%%%%%%%%%%%%%%%%%%%%%%%%%%%%%%%%%%%%%%%%%%%%%%%%%%%%%%%%%%%%%%%%%
\subsection{Forwarding}
\label{sec:forward}

Different versions of the main or child documents
using compilation flags as described in \secref{sec:flags}
can be (permanently) stored in different files
for convenient compilation, viewing and distribution.
To this end, the package defines a command
to pass on compilation to a different file:

%%%%%%%%%%%%%%%%%%%%%%%%%%%%%%%%%%%%%%%%
\DescribeMacro{\childdocforward}
The command |\childdocforward| redirects processing to
another source file:
%
\begin{center}
\begin{tabular}{l}
|\input{childdoc.def}|\\
|\childdocforward[|\textit{main}|]{|\textit{dest}|}|\\
\end{tabular}
\end{center}
%
The argument \textit{dest} is the destination file
(without extension).
It should be the main file or one of the child files.
Note that further \textsf{childdoc} directives
such as |\childdocof| and |\childdocforward|
in the indicated file will be processed in this form.
The optional argument \textit{main}
passes on directly to the main file \textit{main}
while pretending to compile the child \textit{dest}.
This form behaves as if \textit{dest}
issues |\childdocof{|\textit{main}|}| right away,
and no further \textsf{childdoc} directives will be processed.

%%%%%%%%%%%%%%%%%%%%%%%%%%%%%%%%%%%%%%%%
\DescribeMacro{\...prefix}
In the alternative form |\childdocforwardprefix|,
%
\begin{center}
\begin{tabular}{l}
|\input{childdoc.def}|\\
|\childdocforwardprefix[|\textit{main}|]{|\textit{prefix}|}{|\textit{dest}|}|
\end{tabular}
\end{center}
%
the destination file is determined by a pattern
depending on the current file:
To make this work, the current file must be called
`{\textit{prefix}\hspace{0.2em}\textit{suffix}}'
with \textit{prefix} matching precisely the argument.
Processing is then passed on to the file
`{\textit{dest}\hspace{0.2em}\textit{suffix}}'.
Surely, the same effect is achieved by
directly specifying the
argument `{\textit{dest}\hspace{0.2em}\textit{suffix}}'
in the first form.
However, that requires to set up a different file
for each child. With the alternative form of the command
all these files can have exactly the same content
which simplifies setting them up and maintaining them.

For example, the following file |draft.tex|
with a compilation flag |\version| as described in \secref{sec:flags}
compiles the main document as a draft:
%
\begin{center}
\begin{tabular}{l}
|\def\version{draft}|\\
|\input{childdoc.def}|\\
|\childdocforward{|\textit{main}|}|
\end{tabular}
\end{center}
%
Likewise, the following files |final|\textit{nn}|.tex|
compile the final version of the child document
|child|\textit{nn}|.tex|:
%
\begin{center}
\begin{tabular}{l}
|\def\version{final}|\\
|\input{childdoc.def}|\\
|\childdocforwardprefix{final}{child}|
\end{tabular}
\end{center}
%

Note that when several versions of a main file and/or of each child file
are to be generated, it may be convenient to set up a |Makefile| or
shell script to automatise the process.

%%%%%%%%%%%%%%%%%%%%%%%%%%%%%%%%%%%%%%%%%%%%%%%%%%%%%%%%%%%%%%%%%%%%%%%%%%%%%%%%
\subsection{Command Line Processing}
\label{sec:commandline}

The effect of redirection files can also be achieved by invoking
the \LaTeX{} compiler with a more elaborate command line.
Most conveniently this should be done as part
of a shell script or a |Makefile|.

When using \textsf{childdoc} in the main file, the following
command lines effectively perform a redirection
(note that depending on the shell being used,
backslashes may have to be doubled: `|\|' $\to$ `|\\|'):
%
\begin{center}
|... -jobname "|\textit{target}|" |\\|"|[\textit{flags}]%
|\input{childdoc.def}\childdocforward[|\textit{main}|]{|\textit{dest}|}"|
\end{center}
%
Here \textit{target} is the name of the output file,
\textit{main} is the name of the main file
and \textit{dest} is the name of the main or child file to be processed
(all filenames without extensions).
The optional argument \textit{main} can be omitted
if \textit{main} matches \textit{dest}.
Optionally, compilation \textit{flags} can be defined via |\def| commands.
This command line makes the \TeX{} engine believe
it is compiling the file \textit{target}
whose content is specified as the latter parameter.
The provided code then forwards the processing to
\textit{main} or \textit{dest} as described in \secref{sec:forward}.

%%%%%%%%%%%%%%%%%%%%%%%%%%%%%%%%%%%%%%%%%%%%%%%%%%%%%%%%%%%%%%%%%%%%%%%%%%%%%%%%
\subsection{Include by Input}
\label{sec:input}

Including child documents by |\include| has some restrictions by design.
Most notably, the content of a child document always occupies
its own set of pages; pages cannot be shared between child documents.
Usually, this behaviour makes perfect sense
because each child document contain an essential part of the document.
However, in some situations it may be desirable to compose
a document from a collection of parts
without having mandatory page breaks between then.
For this case, the package
provides a mechanism to include parts
by |\input| which can also be processed individually.
However, by construction this mechanism
requires manual handling of the content to be output.

%%%%%%%%%%%%%%%%%%%%%%%%%%%%%%%%%%%%%%%%
\DescribeMacro{\ifchilddocmanual}
The main file should be prepared as usual, see \secref{sec:include}.
However, the document body must make a distinction
between processing of an individual part and of the main document, e.g.:
%
\begin{center}
\begin{tabular}{l}
|\ifchilddocmanual|\\
|\input{\childdocname}|\\
|\||else|\\
\textit{document body with }|\input{|\textit{part}|}|\\
|\||fi|
\end{tabular}
\end{center}
%
The conditional |\ifchilddocmanual| is true whenever
a part to be included by |\input| is being compiled,
and the name of the part is stored in |\childdocname|.

%%%%%%%%%%%%%%%%%%%%%%%%%%%%%%%%%%%%%%%%
\DescribeMacro{\childdocby}
Each part to be included by |\input| should start with:
%
\begin{center}
\begin{tabular}{l}
|\input{childdoc.def}|\\
|\childdocby{|\textit{main}|}|\\
\end{tabular}
\end{center}
%
The directive |\childdocby| is similar to |\childdocof|
described in \secref{sec:include},
but the subsequent selection of content must be done manually.
To that end, both |\ifchilddoc| and |\ifchilddocmanual|
will be true upon processing of a part,
and the name of the part is stored in |\childdocname|.
Note that |\jobname| will be set to the filename of the current part
so that each part receives an individual |.aux| file
that does not interfere with the |.aux| file(s) of the main document.
This behaviour can be altered by the alternative form
|\childdocby[*]{|\textit{main}|}| (with a non-empty optional argument)
which uses the |.aux| file of the main document
by setting |\jobname| to \textit{main}.

%%%%%%%%%%%%%%%%%%%%%%%%%%%%%%%%%%%%%%%%%%%%%%%%%%%%%%%%%%%%%%%%%%%%%%%%%%%%%%%%
\subsection{Driver Development}
\label{sec:driver}

The \textsf{childdoc} mechanism can also be use for the development
of definition files such as \LaTeX{} styles or classes.
This case differs from the above setup with multiple parts
included by |\include| in that no |\includeonly| should be invoked.
This can be achieved by starting the include file
(before |\ProvidesPackage|) with:
%
\begin{center}
\begin{tabular}{l}
|\input{childdoc.def}|\\
|\childdocforward{|\textit{main}|}|\\
\end{tabular}
\end{center}
%
or alternatively with:
%
\begin{center}
\begin{tabular}{l}
|\input{childdoc.def}|\\
|\childdocby{|\textit{main}|}|\\
\end{tabular}
\end{center}
%
Both forms have slightly different effects as described above.
The main file is prepared as usual, see \secref{sec:include}.

%%%%%%%%%%%%%%%%%%%%%%%%%%%%%%%%%%%%%%%%%%%%%%%%%%%%%%%%%%%%%%%%%%%%%%%%%%%%%%%%
\subsection{Legacy Detection}
\label{sec:detection}

The directive |\childdocmain| in the main file can detect
whether the complete document or merely a child is to be compiled
even without using the directive |\childdocof|.
This method is deprecated because it is less robust
and there is no compelling reason to use it;
it is merely provided for backward compatibility
and it may be removed in future versions.

If the detection mechanism is to be used,
it is mandatory to correctly specify
the filename of the main file as the argument of |\childdocmain|:
%
\begin{center}
\begin{tabular}{l}
|\input{childdoc.def}|\\
|\childdocmain{|\textit{main}|}|\\
\end{tabular}
\end{center}
%
If |\jobname| does not match the argument \textit{main} of |\childdocmain|,
it is assumed that |\jobname| points to the child file to be compiled.
When using |\childdocmain| with the main file specified as argument,
it suffices to start a child file
with just |\input{|\textit{main}|}|
without loading of the package and using |\childdocof|.
If instead all processing is done
with the appropriate \textsf{childdoc} directives,
the argument of \textit{main} of |\childdocmain| can be empty.

An alternative version of the command line processing described
in \secref{sec:commandline} using the detection mechanism reads:
%
\begin{center}
|... -jobname "|\textit{target}|" "|[\textit{flags}]%
[|\def\jobname{|\textit{dest}|}|]|\input{|\textit{main}|}"|
\end{center}

%%%%%%%%%%%%%%%%%%%%%%%%%%%%%%%%%%%%%%%%%%%%%%%%%%%%%%%%%%%%%%%%%%%%%%%%%%%%%%%%
\subsection{Manual Code}
\label{sec:manual}

In case one cannot be certain whether the definitions file |childdoc.def|
is installed on the target \TeX{} distribution
and one prefers not to ship it,
it is conceivable to paste a few relevant commands into the sources.

To that end, drop all statements |\input{childdoc.def}|
and perform the replacements as outlined below.
Instead of |\childdocmain{|\textit{main}|}| add the following code
to the top of the main file:
%
\begin{center}
\begin{tabular}{l}
|\||ifdefined\childdocname\endinput\||fi\newif\ifchilddoc|\\
|\edef\childdocname{\scantokens\expandafter{\jobname\noexpand}}|\\
|\def\childdocmain{|\textit{main}|}\||ifx\childdocmain\childdocname\||else|\\
|\childdoctrue\includeonly{\childdocname}\let\jobname\childdocmain\||fi|\\
\end{tabular}
\end{center}
%
Instead of |\childdocof{|\textit{main}|}| just include the main file
at the top of each child file:
%
\begin{center}
|\input{|\textit{main}|}|
\end{center}
%
A simple redirection |\childdocforward{|\textit{dest}|}| is achieved by:
%
\begin{center}
|\def\jobname{|\textit{dest}|}\input{\jobname}|
\end{center}
%
The redirection with prefix
|\childdocforwardprefix[|\textit{prefix}|]{|\textit{dest}|}|
is accomplished by:
%
\begin{center}
\begin{tabular}{l}
|{\edef\jobname{\scantokens\expandafter{\jobname\noexpand}}|\\
|\def\redirectjob |\textit{prefix}|#1~~~{\gdef\jobname{|\textit{dest}|#1}}|\\
|\expandafter\redirectjob\jobname~~~}\input{\jobname}|
\end{tabular}
\end{center}

In an alternative approach,
child documents can be compiled by a specific command line
without additional code or specific definitions:
%
\begin{center}
|... -jobname "|\textit{target}|" "|[\textit{flags}]%
|\includeonly{|\textit{dest}|}\input{|\textit{main}|}"|
\end{center}
%

%%%%%%%%%%%%%%%%%%%%%%%%%%%%%%%%%%%%%%%%%%%%%%%%%%%%%%%%%%%%%%%%%%%%%%%%%%%%%%%%
%%%%%%%%%%%%%%%%%%%%%%%%%%%%%%%%%%%%%%%%%%%%%%%%%%%%%%%%%%%%%%%%%%%%%%%%%%%%%%%%
\section{Information}

%%%%%%%%%%%%%%%%%%%%%%%%%%%%%%%%%%%%%%%%%%%%%%%%%%%%%%%%%%%%%%%%%%%%%%%%%%%%%%%%
\subsection{Copyright}

Copyright \copyright{} 2017--2018 Niklas Beisert

This work may be distributed and/or modified under the
conditions of the \LaTeX{} Project Public License, either version 1.3
of this license or (at your option) any later version.
The latest version of this license is in
  \url{http://www.latex-project.org/lppl.txt}
and version 1.3 or later is part of all distributions of \LaTeX{}
version 2005/12/01 or later.

This work has the LPPL maintenance status `maintained'.

The Current Maintainer of this work is Niklas Beisert.

This work consists of the files |README.txt|, |childdoc.ins| and |childdoc.dtx|
as well as the derived files |childdoc.def|, |cdocsamp.tex|
with |cdocsch1.tex|, |cdocsch2.tex|, |cdocspt3.tex|, |cdocspt4.tex|,
|cdocsdrf.tex|, |cdocsfn1.tex|, |cdocsfn2.tex|
as well as |childdoc.pdf|.

%%%%%%%%%%%%%%%%%%%%%%%%%%%%%%%%%%%%%%%%%%%%%%%%%%%%%%%%%%%%%%%%%%%%%%%%%%%%%%%%
\subsection{Files and Installation}

The package consists of the files:
%
\begin{center}
\begin{tabular}{ll}
    |README.txt|   & readme file \\
    |childdoc.ins| & installation file \\
    |childdoc.dtx| & source file \\
    |childdoc.def| & definition file \\
    |cdocsamp.tex| & sample main file \\
    |cdocsch1.tex| & sample include file \\
    |cdocsch2.tex| & sample include file \\
    |cdocspt3.tex| & sample part file \\
    |cdocspt4.tex| & sample part file \\
    |cdocsdrf.tex| & sample redirection file \\
    |cdocsfn1.tex| & sample redirection file \\
    |cdocsfn2.tex| & sample redirection file \\
    |childdoc.pdf| & manual
\end{tabular}
\end{center}
%
The distribution consists of the files
|README.txt|, |childdoc.ins| and |childdoc.dtx|.
%
\begin{itemize}
\item
Run (pdf)\LaTeX{} on |childdoc.dtx|
to compile the manual |childdoc.pdf| (this file).
\item
Run \LaTeX{} on |childdoc.ins| to create the definitions file |childdoc.def|
and the sample |cdocsamp.tex| with include files
|cdocsch1.tex|, |cdocsch2.tex|, |cdocspt3.tex|, |cdocspt4.tex|,
|cdocsdrf.tex|, |cdocsfn1.tex|, |cdocsfn2.tex|.
Then copy the file |childdoc.def| to an appropriate directory of your \LaTeX{}
distribution, e.g.\ \textit{texmf-root}|/tex/latex/childdoc|.
\end{itemize}

%%%%%%%%%%%%%%%%%%%%%%%%%%%%%%%%%%%%%%%%%%%%%%%%%%%%%%%%%%%%%%%%%%%%%%%%%%%%%%%%
\subsection{Related CTAN Packages}

There are several other packages which offer a similar functionality:
%
\begin{itemize}
\item
The packages
\href{http://ctan.org/pkg/docmute}{\textsf{docmute}},
\href{http://ctan.org/pkg/includex}{\textsf{includex}} and
\href{http://ctan.org/pkg/standalone}{\textsf{standalone}}
provide commands to include only the document body of
a child file thus allowing both files to be compiled individually.
\item
The packages \href{http://ctan.org/pkg/subdocs}{\textsf{subdocs}}
and \href{http://ctan.org/pkg/subfiles}{\textsf{subfiles}}
provide structures in which the main and child documents can be
encapsulated and allowing them to be compiled individually.
The inclusion mechanism is different from the conventional |\include|.
\item
The package \href{http://ctan.org/pkg/combine}{\textsf{combine}}
is an elaborate solution to combine several documents into one.
\end{itemize}
%
See also the CTAN topic \href{http://ctan.org/topic/subdocs}{\textsf{subdocs}}
for further related packages.
The present package differs from the above solutions in that
a document structure constructed with the conventional |\include| mechanism
just needs two extra commands at the top of every file
such that all constituent files can be compiled individually.

%%%%%%%%%%%%%%%%%%%%%%%%%%%%%%%%%%%%%%%%%%%%%%%%%%%%%%%%%%%%%%%%%%%%%%%%%%%%%%%%
%\subsection{Feature Suggestions}
%
%The following is a list of features which may be useful for future
%versions of this package:
%%
%\begin{itemize}
%\item
%\ldots
%\end{itemize}

%%%%%%%%%%%%%%%%%%%%%%%%%%%%%%%%%%%%%%%%%%%%%%%%%%%%%%%%%%%%%%%%%%%%%%%%%%%%%%%%
\subsection{Revision History}

%%%%%%%%%%%%%%%%%%%%%%%%%%%%%%%%%%%%%%%%
\paragraph{v2.0:} 2018/12/30

\begin{itemize}
\item
immediate forward processing
\item
added |\childdocby| mechanism
\item
manual restructured
\end{itemize}

%%%%%%%%%%%%%%%%%%%%%%%%%%%%%%%%%%%%%%%%
\paragraph{v1.6:} 2018/01/17

\begin{itemize}
\item
application for development of include files
\item
corrections to manual
\end{itemize}

%%%%%%%%%%%%%%%%%%%%%%%%%%%%%%%%%%%%%%%%
\paragraph{v1.5:} 2017/05/21

\begin{itemize}
\item
more complete structuring introduced
\item
|\childdocof| introduced
\item
|\childdoc| renamed to |\childdocmain|
\item
|\childredirect| renamed to |\childdocforward| and |\childdocforwardprefix|
and functionality expanded
\end{itemize}

%%%%%%%%%%%%%%%%%%%%%%%%%%%%%%%%%%%%%%%%
\paragraph{v1.0:} 2017/04/27

\begin{itemize}
\item
manual and install package
\item
first version published on CTAN
\end{itemize}

%%%%%%%%%%%%%%%%%%%%%%%%%%%%%%%%%%%%%%%%
\paragraph{v0.6:} 2017/04/26

\begin{itemize}
\item
redirection mechanism added
\end{itemize}

%%%%%%%%%%%%%%%%%%%%%%%%%%%%%%%%%%%%%%%%
\paragraph{v0.5:} 2017/04/26

\begin{itemize}
\item
functionality in definition file
\end{itemize}


%%%%%%%%%%%%%%%%%%%%%%%%%%%%%%%%%%%%%%%%%%%%%%%%%%%%%%%%%%%%%%%%%%%%%%%%%%%%%%%%
%%%%%%%%%%%%%%%%%%%%%%%%%%%%%%%%%%%%%%%%%%%%%%%%%%%%%%%%%%%%%%%%%%%%%%%%%%%%%%%%
%%%%%%%%%%%%%%%%%%%%%%%%%%%%%%%%%%%%%%%%%%%%%%%%%%%%%%%%%%%%%%%%%%%%%%%%%%%%%%%%
\appendix

\settowidth\MacroIndent{\rmfamily\scriptsize 000\ }

 \DocInput{childdoc.dtx}

\end{document}
%</driver>
% \fi
%
% %%%%%%%%%%%%%%%%%%%%%%%%%%%%%%%%%%%%%%%%%%%%%%%%%%%%%%%%%%%%%%%%%%%%%%%%%%%%%%
% %%%%%%%%%%%%%%%%%%%%%%%%%%%%%%%%%%%%%%%%%%%%%%%%%%%%%%%%%%%%%%%%%%%%%%%%%%%%%%
% \section{Sample}
%\iffalse
%<*samplemain>
%\fi
%
% The following presents a sample document
% with two chapters, two parts, a title page,
% a compile flag as well as three forwarding files to set the flag.
% It consists of eight |.tex| files:
% \begin{center}
% \begin{tabular}{ll}
% |cdocsamp.tex|&main file\\
% |cdocsch1.tex|&include file for chapter 1\\
% |cdocsch2.tex|&include file for chapter 2\\
% |cdocspt3.tex|&include file for part 3\\
% |cdocspt4.tex|&include file for part 4\\
% |cdocsdrf.tex|&forwarding file for main file in draft mode\\
% |cdocsfi1.tex|&forwarding file for final version of chapter 1\\
% |cdocsfi2.tex|&forwarding file for final version of chapter 2\\
% \end{tabular}
% \end{center}
% Each of the eight files can be compiled directly by the \LaTeX{} compiler.
%
% %%%%%%%%%%%%%%%%%%%%%%%%%%%%%%%%%%%%%%
% \paragraph{Main File.}
%
% The main file is called |cdocsamp.tex|.
%
% Load the \textsf{childdoc} definitions and
% declare the filename for the main document:
%    \begin{macrocode}
\input{childdoc.def}
\childdocmain{}
%    \end{macrocode}

% Optional override for |\version| flag:
%    \begin{macrocode}
%%\ifchilddoc\else\providecommand{\version}{draft}\fi
%    \end{macrocode}

% Define the default values for the |\version| flag
% (|final| for the main file and |draft| for childs):
%    \begin{macrocode}
\ifchilddoc
\providecommand{\version}{draft}
\else
\providecommand{\version}{final}
\fi
%    \end{macrocode}

% Load the standard document class:
%    \begin{macrocode}
\documentclass[12pt]{article}
%    \end{macrocode}

% Start the document body:
%    \begin{macrocode}
\begin{document}
%    \end{macrocode}

% Declare a title page.
% Print title, part of document being processed and version flag:
%    \begin{macrocode}
\addtocounter{page}{-1}
\begin{center}
{\LARGE\bfseries{}childdoc example\par}
\vspace{1cm}
\ifchilddoc
\ifchilddocmanual part\else chapter\fi:
`\childdocname' of `\childdocjob'\par
\else
main document: `\childdocjob'\par
\fi
version: \version\par
\end{center}
\newpage
%    \end{macrocode}

% Manually include selected file,
% otherwise process as usual:
%    \begin{macrocode}
\ifchilddocmanual
\section*{part `\childdocname'}
\input{\childdocname}
\else
%    \end{macrocode}

% Include the two chapters:
%    \begin{macrocode}
\include{cdocsch1}
\include{cdocsch2}
%    \end{macrocode}

% Include the two parts unless only chapters should be displayed:
%    \begin{macrocode}
\ifchilddoc\else
\section{part three}
\input{cdocspt3}
\section{part four}
\input{cdocspt4}
\fi
%    \end{macrocode}

% Process as usual until here:
%    \begin{macrocode}
\fi
%    \end{macrocode}

% End of document body:
%    \begin{macrocode}
\end{document}
%    \end{macrocode}
%\iffalse
%</samplemain>
%\fi
%
% %%%%%%%%%%%%%%%%%%%%%%%%%%%%%%%%%%%%%%
% \paragraph{Chapter Include Files.}
%
% The include files are called |cdocsch1.tex| and |cdocsch2.tex|.
%
%\iffalse
%<*samplechap1|samplechap2>
%\fi

% Optional override for |\version| flag:
%    \begin{macrocode}
%%\providecommand{\version}{final}
%    \end{macrocode}

% Include the main document:
%    \begin{macrocode}
\input{childdoc.def}
\childdocof{cdocsamp}
%    \end{macrocode}

%\iffalse
%</samplechap1|samplechap2>
%\fi
%
%\iffalse
%<*samplechap1>
%\fi
% Some text for chapter 1:
%    \begin{macrocode}
\section{one}
some text in chapter one
%    \end{macrocode}

%\iffalse
%</samplechap1>
%\fi
% Some text for chapter 2:
%\iffalse
%<*samplechap2>
%\fi
%    \begin{macrocode}
\section{two}
more text in chapter two
%    \end{macrocode}

%\iffalse
%</samplechap2>
%\fi
%
% %%%%%%%%%%%%%%%%%%%%%%%%%%%%%%%%%%%%%%
% \paragraph{Part Include Files.}
%
% The include files are called |cdocspt3.tex| and |cdocspt4.tex|.
%
%\iffalse
%<*samplepart3|samplepart4>
%\fi

% Optional override for |\version| flag:
%    \begin{macrocode}
%%\providecommand{\version}{final}
%    \end{macrocode}

% Include the main document:
%    \begin{macrocode}
\input{childdoc.def}
\childdocby{cdocsamp}
%    \end{macrocode}

%\iffalse
%</samplepart3|samplepart4>
%\fi
%
%\iffalse
%<*samplepart3>
%\fi
% Some text for part 3:
%    \begin{macrocode}
some text in part three
%    \end{macrocode}

%\iffalse
%</samplepart3>
%\fi
% Some text for part 4:
%\iffalse
%<*samplepart4>
%\fi
%    \begin{macrocode}
more text in part four
%    \end{macrocode}

%\iffalse
%</samplepart4>
%\fi
%
% %%%%%%%%%%%%%%%%%%%%%%%%%%%%%%%%%%%%%%
% \paragraph{Forwarding for a Complete Draft.}
%
% The following forwarding file |cdocsdrf.tex|
% compiles the main document in draft mode:
%\iffalse
%<*sampledraft>
%\fi
%    \begin{macrocode}
\def\version{draft}
\input{childdoc.def}
\childdocforward{cdocsamp}
%    \end{macrocode}

%\iffalse
%</sampledraft>
%\fi
%
% %%%%%%%%%%%%%%%%%%%%%%%%%%%%%%%%%%%%%%
% \paragraph{Forwarding for Final Version of the Chapters.}
%
% The following forwarding files |cdocsfn1.tex| and |cdocsfn2.tex|
% (with identical content)
% compile the final versions of the child documents
% |cdocsch1.tex| and |cdocsch2.tex|, respectively:
%\iffalse
%<*samplefinal>
%\fi
%    \begin{macrocode}
\def\version{final}
\input{childdoc.def}
\childdocforwardprefix[cdocsamp]{cdocsfn}{cdocsch}
%    \end{macrocode}

%\iffalse
%</samplefinal>
%\fi
%
% %%%%%%%%%%%%%%%%%%%%%%%%%%%%%%%%%%%%%%
% \paragraph{Command Line Processing.}
%
% The following three command lines generate the output files
% |cdocscld|, |cdocscl1| and |cdocscl2|
% which should be identical to
% |cdocsdrf|, |cdocsch1| and |cdocsfn2|, respectively:
% \begin{center}
% \begin{tabular}{l}
% |latex -jobname cdocscld \|\\
% |  "\def\version{draft}\input{childdoc.def}\childdocforward{cdocsamp}"|\\
% |latex -jobname cdocscl1 \|\\
% |  "\input{childdoc.def}\childdocforward[cdocsamp]{cdocsch1}"|\\
% |latex -jobname cdocscl2 \|\\
% |  "\def\version{final}\input{childdoc.def}\childdocforward{cdocsch2}"|
% \end{tabular}
% \end{center}
% Note that the trailing backslash on each first line
% merely continues the input to the second line
% (for convenient cut ant paste).
% Furthermore, the command |latex| can be replaced by any
% of its alternative versions such as |pdflatex|.
%
% %%%%%%%%%%%%%%%%%%%%%%%%%%%%%%%%%%%%%%%%%%%%%%%%%%%%%%%%%%%%%%%%%%%%%%%%%%%%%%
% %%%%%%%%%%%%%%%%%%%%%%%%%%%%%%%%%%%%%%%%%%%%%%%%%%%%%%%%%%%%%%%%%%%%%%%%%%%%%%
% \section{Implementation}
%\iffalse
%<*package>
%\fi
%
% This section describes the definitions file |childdoc.def|.

% The definitions cannot be loaded using |\usepackage| or |\RequirePackage|
% which has a mechanism to prevent loading a style file more than once.
% When loading the definitions by means of |\input|
% multiple instances have to be prevented manually:
%\iffalse
%This code needs to be before the `\ProvidesFile' directive
%which is defined at the beginning of this file.
%Therefore it is also placed there and commented out here.
%</package>
%<*discard>
%\fi
%    \begin{macrocode}
\ifdefined\childdocmain\endinput\fi
%    \end{macrocode}
%\iffalse
%</discard>
%<*package>
%\fi
%
% \macro{\ifchilddoc}
% \macro{\ifchilddocmanual}
% The conditional |\ifchilddoc| tells whether a
% child (true) or main (false) document is being compiled.
% The conditional |\ifchilddocmanual| tells whether
% the |\includeonly| mechanism is used (false) or
% the selection of child files must be performed manually (true).
% The definitions initialise to false:
%    \begin{macrocode}
\newif\ifchilddoc
\newif\ifchilddocmanual
%    \end{macrocode}

% \macro{\childdocname}
% \macro{\childdocjob}
% The macro |\childdocname| stores the name of the main document
% to be compiled. The macro |\childdocjob| stores the name of
% the document on which the \LaTeX{} compiler was originally invoked.
% The content of |\jobname| cannot be compared
% to filenames specified in the source due to different catcodes.
% The following code rescans |\jobname|, stores the result
% in |\childdocname| and saves a copy in |\childdocjob|:
%    \begin{macrocode}
\edef\childdocname{\scantokens\expandafter{\jobname\noexpand}}
\let\childdocjob\childdocname
%    \end{macrocode}

% \macro{\childdocdisable}
% The macro |\childdocdisable| prevents the main file
% from being processed more than once.
% At this stage, the main document command |\childdocmain|
% is assumed to be called once again where it should do nothing.
% Any subsequent call to it should prevent
% a secondary processing of the main document
% It overwrites the forwarding commands
% |\childdocof| and |\childdocforward|
% with empty macros to prevent further inclusions of the main document:
%    \begin{macrocode}
\newcommand{\childdocdisable}
{
  \renewcommand{\childdocmain}[1]{\renewcommand{\childdocmain}[1]{\endinput}}
  \renewcommand{\childdocof}[1]{}
  \renewcommand{\childdocby}[2][]{}
  \renewcommand{\childdocforward}[2][]{}
  \renewcommand{\childdocdisable}{}
}
%    \end{macrocode}

% \macro{\childdocmain}
% The macro |\childdocmain| is to be called at the top of the main file
% with nothing or the main filename (without extension) as argument.
% First, it breaks loops.
% If the argument is not empty and does not match |\childdocname|
% (which is set by the first inclusion of |childdoc.def|),
% |\ifchilddoc| is set to true, |\includeonly| is applied to the child file
% and |\jobname| is set to the main file
% (for proper handling of |.aux| files):
%    \begin{macrocode}
\newcommand{\childdocmain}[1]
{
  \childdocdisable\childdocmain{}
  \if?#1?\else
    \begingroup
      \def\childdoctmp{#1}
      \ifx\childdoctmp\childdocname
        \def\childdoctmp{}
      \else
        \def\childdoctmp
        {
          \childdoctrue
          \includeonly{\childdocname}
          \def\childdocjob{#1}
          \def\jobname{#1}
        }
      \fi
      \expandafter
    \endgroup
    \childdoctmp
  \fi
}
%    \end{macrocode}

% \macro{\childdocof}
% The command |\childdocof| redirects
% compilation to the main file |#1|.
%    \begin{macrocode}
\newcommand{\childdocof}[1]
{
  \childdocdisable
  \childdoctrue
  \includeonly{\childdocname}
  \def\jobname{#1}
  \def\childdocjob{#1}
  \input{#1}
}
%    \end{macrocode}

% \macro{\childdocby}
% The command |\childdocby| ....
%    \begin{macrocode}
\newcommand{\childdocby}[2][]
{
  \childdocdisable
  \childdoctrue
  \childdocmanualtrue
  \if?#1?\else
    \def\jobname{#2}
  \fi
  \def\childdocjob{#2}
  \input{#2}
  \endinput
}
%    \end{macrocode}

% \macro{\childdocforward}
% The command |\childdocforward| redirects
% compilation to the main file or
% (if the optional argument is given) a child file.
% Parameters are set as if the main file
% or a child file starting with |\childdocof| was compiled.
% Then compilation is handed over to the main file:
%    \begin{macrocode}
\newcommand{\childdocforward}[2][]
{
  \begingroup
    \if?#1?
      \def\childdoctmp
      {
        \def\childdocname{#2}
        \def\childdocjob{#2}
        \def\jobname{#2}
        \input{#2}
        \endinput
      }
    \else
      \def\childdoctmp
      {
        \childdocdisable
        \def\childdocname{#2}
        \childdoctrue
        \includeonly{#2}
        \def\childdocjob{#1}
        \def\jobname{#1}
        \input{#1}
        \endinput
      }
    \fi
    \expandafter
  \endgroup
  \childdoctmp
}
%    \end{macrocode}

% \macro{\childdocforwardprefix}
% The command |\childdocforwardprefix| redirects
% compilation to the main or a child file by means of a pattern.
% The prefix |#1| in the current filename is replaced by |#2|
% and the suffix of the current filename is kept
% (it is assumed that the filename does not contain the substring `|~~~|'
% which is used as a delimiter).
% Compilation is handed over to the new file by |\childdocforward|:
%    \begin{macrocode}
\newcommand{\childdocforwardprefix}[3][]
{
  \begingroup
    \def\childdocextract #2##1~~~{\def\childdoctmp{\childdocforward[#1]{#3##1}}}
    \expandafter\childdocextract\childdocname~~~
    \expandafter
  \endgroup
  \childdoctmp
}
%    \end{macrocode}

% \macro{\childdoc}
% The deprecated macro |\childdoc| is a legacy version of |\childdocmain|:
%    \begin{macrocode}
\newcommand{\childdoc}{\childdocmain}
%    \end{macrocode}

% \macro{\childdocredirect}
% The deprecated macro |\childdocredirect| is a legacy version
% of |\childdocforward| and |\childdocforwardprefix|:
%    \begin{macrocode}
\newcommand{\childdocredirect}[2][]
{
  \begingroup
    \if?#1?
      \def\childdoctmp{\childdocforward{#2}}
    \else
      \def\childdoctmp{\childdocforwardprefix{#1}{#2}}
    \fi
    \expandafter
  \endgroup
  \childdoctmp
}
%    \end{macrocode}

%\iffalse
%</package>
%\fi
%
\endinput
\childdocforward{cdocsch2}"|
% \end{tabular}
% \end{center}
% Note that the trailing backslash on each first line
% merely continues the input to the second line
% (for convenient cut ant paste).
% Furthermore, the command |latex| can be replaced by any
% of its alternative versions such as |pdflatex|.
%
% %%%%%%%%%%%%%%%%%%%%%%%%%%%%%%%%%%%%%%%%%%%%%%%%%%%%%%%%%%%%%%%%%%%%%%%%%%%%%%
% %%%%%%%%%%%%%%%%%%%%%%%%%%%%%%%%%%%%%%%%%%%%%%%%%%%%%%%%%%%%%%%%%%%%%%%%%%%%%%
% \section{Implementation}
%\iffalse
%<*package>
%\fi
%
% This section describes the definitions file |childdoc.def|.

% The definitions cannot be loaded using |\usepackage| or |\RequirePackage|
% which has a mechanism to prevent loading a style file more than once.
% When loading the definitions by means of |\input|
% multiple instances have to be prevented manually:
%\iffalse
%This code needs to be before the `\ProvidesFile' directive
%which is defined at the beginning of this file.
%Therefore it is also placed there and commented out here.
%</package>
%<*discard>
%\fi
%    \begin{macrocode}
\ifdefined\childdocmain\endinput\fi
%    \end{macrocode}
%\iffalse
%</discard>
%<*package>
%\fi
%
% \macro{\ifchilddoc}
% \macro{\ifchilddocmanual}
% The conditional |\ifchilddoc| tells whether a
% child (true) or main (false) document is being compiled.
% The conditional |\ifchilddocmanual| tells whether
% the |\includeonly| mechanism is used (false) or
% the selection of child files must be performed manually (true).
% The definitions initialise to false:
%    \begin{macrocode}
\newif\ifchilddoc
\newif\ifchilddocmanual
%    \end{macrocode}

% \macro{\childdocname}
% \macro{\childdocjob}
% The macro |\childdocname| stores the name of the main document
% to be compiled. The macro |\childdocjob| stores the name of
% the document on which the \LaTeX{} compiler was originally invoked.
% The content of |\jobname| cannot be compared
% to filenames specified in the source due to different catcodes.
% The following code rescans |\jobname|, stores the result
% in |\childdocname| and saves a copy in |\childdocjob|:
%    \begin{macrocode}
\edef\childdocname{\scantokens\expandafter{\jobname\noexpand}}
\let\childdocjob\childdocname
%    \end{macrocode}

% \macro{\childdocdisable}
% The macro |\childdocdisable| prevents the main file
% from being processed more than once.
% At this stage, the main document command |\childdocmain|
% is assumed to be called once again where it should do nothing.
% Any subsequent call to it should prevent
% a secondary processing of the main document
% It overwrites the forwarding commands
% |\childdocof| and |\childdocforward|
% with empty macros to prevent further inclusions of the main document:
%    \begin{macrocode}
\newcommand{\childdocdisable}
{
  \renewcommand{\childdocmain}[1]{\renewcommand{\childdocmain}[1]{\endinput}}
  \renewcommand{\childdocof}[1]{}
  \renewcommand{\childdocby}[2][]{}
  \renewcommand{\childdocforward}[2][]{}
  \renewcommand{\childdocdisable}{}
}
%    \end{macrocode}

% \macro{\childdocmain}
% The macro |\childdocmain| is to be called at the top of the main file
% with nothing or the main filename (without extension) as argument.
% First, it breaks loops.
% If the argument is not empty and does not match |\childdocname|
% (which is set by the first inclusion of |childdoc.def|),
% |\ifchilddoc| is set to true, |\includeonly| is applied to the child file
% and |\jobname| is set to the main file
% (for proper handling of |.aux| files):
%    \begin{macrocode}
\newcommand{\childdocmain}[1]
{
  \childdocdisable\childdocmain{}
  \if?#1?\else
    \begingroup
      \def\childdoctmp{#1}
      \ifx\childdoctmp\childdocname
        \def\childdoctmp{}
      \else
        \def\childdoctmp
        {
          \childdoctrue
          \includeonly{\childdocname}
          \def\childdocjob{#1}
          \def\jobname{#1}
        }
      \fi
      \expandafter
    \endgroup
    \childdoctmp
  \fi
}
%    \end{macrocode}

% \macro{\childdocof}
% The command |\childdocof| redirects
% compilation to the main file |#1|.
%    \begin{macrocode}
\newcommand{\childdocof}[1]
{
  \childdocdisable
  \childdoctrue
  \includeonly{\childdocname}
  \def\jobname{#1}
  \def\childdocjob{#1}
  \input{#1}
}
%    \end{macrocode}

% \macro{\childdocby}
% The command |\childdocby| ....
%    \begin{macrocode}
\newcommand{\childdocby}[2][]
{
  \childdocdisable
  \childdoctrue
  \childdocmanualtrue
  \if?#1?\else
    \def\jobname{#2}
  \fi
  \def\childdocjob{#2}
  \input{#2}
  \endinput
}
%    \end{macrocode}

% \macro{\childdocforward}
% The command |\childdocforward| redirects
% compilation to the main file or
% (if the optional argument is given) a child file.
% Parameters are set as if the main file
% or a child file starting with |\childdocof| was compiled.
% Then compilation is handed over to the main file:
%    \begin{macrocode}
\newcommand{\childdocforward}[2][]
{
  \begingroup
    \if?#1?
      \def\childdoctmp
      {
        \def\childdocname{#2}
        \def\childdocjob{#2}
        \def\jobname{#2}
        \input{#2}
        \endinput
      }
    \else
      \def\childdoctmp
      {
        \childdocdisable
        \def\childdocname{#2}
        \childdoctrue
        \includeonly{#2}
        \def\childdocjob{#1}
        \def\jobname{#1}
        \input{#1}
        \endinput
      }
    \fi
    \expandafter
  \endgroup
  \childdoctmp
}
%    \end{macrocode}

% \macro{\childdocforwardprefix}
% The command |\childdocforwardprefix| redirects
% compilation to the main or a child file by means of a pattern.
% The prefix |#1| in the current filename is replaced by |#2|
% and the suffix of the current filename is kept
% (it is assumed that the filename does not contain the substring `|~~~|'
% which is used as a delimiter).
% Compilation is handed over to the new file by |\childdocforward|:
%    \begin{macrocode}
\newcommand{\childdocforwardprefix}[3][]
{
  \begingroup
    \def\childdocextract #2##1~~~{\def\childdoctmp{\childdocforward[#1]{#3##1}}}
    \expandafter\childdocextract\childdocname~~~
    \expandafter
  \endgroup
  \childdoctmp
}
%    \end{macrocode}

% \macro{\childdoc}
% The deprecated macro |\childdoc| is a legacy version of |\childdocmain|:
%    \begin{macrocode}
\newcommand{\childdoc}{\childdocmain}
%    \end{macrocode}

% \macro{\childdocredirect}
% The deprecated macro |\childdocredirect| is a legacy version
% of |\childdocforward| and |\childdocforwardprefix|:
%    \begin{macrocode}
\newcommand{\childdocredirect}[2][]
{
  \begingroup
    \if?#1?
      \def\childdoctmp{\childdocforward{#2}}
    \else
      \def\childdoctmp{\childdocforwardprefix{#1}{#2}}
    \fi
    \expandafter
  \endgroup
  \childdoctmp
}
%    \end{macrocode}

%\iffalse
%</package>
%\fi
%
\endinput
|\\
|\childdocforwardprefix{final}{child}|
\end{tabular}
\end{center}
%

Note that when several versions of a main file and/or of each child file
are to be generated, it may be convenient to set up a |Makefile| or
shell script to automatise the process.

%%%%%%%%%%%%%%%%%%%%%%%%%%%%%%%%%%%%%%%%%%%%%%%%%%%%%%%%%%%%%%%%%%%%%%%%%%%%%%%%
\subsection{Command Line Processing}
\label{sec:commandline}

The effect of redirection files can also be achieved by invoking
the \LaTeX{} compiler with a more elaborate command line.
Most conveniently this should be done as part
of a shell script or a |Makefile|.

When using \textsf{childdoc} in the main file, the following
command lines effectively perform a redirection
(note that depending on the shell being used,
backslashes may have to be doubled: `|\|' $\to$ `|\\|'):
%
\begin{center}
|... -jobname "|\textit{target}|" |\\|"|[\textit{flags}]%
|% \iffalse
%
% childdoc.dtx Copyright (C) 2017-2018 Niklas Beisert
%
% This work may be distributed and/or modified under the
% conditions of the LaTeX Project Public License, either version 1.3
% of this license or (at your option) any later version.
% The latest version of this license is in
%   http://www.latex-project.org/lppl.txt
% and version 1.3 or later is part of all distributions of LaTeX
% version 2005/12/01 or later.
%
% This work has the LPPL maintenance status `maintained'.
%
% The Current Maintainer of this work is Niklas Beisert.
%
% This work consists of the files childdoc.dtx and childdoc.ins
% and the derived files childdoc.def and cdocsamp.tex with
% cdocsch1.tex, cdocsch2.tex, cdocsdrf.tex, cdocsfn1.tex, cdocsfn2.tex.
%
%<package>\ifdefined\childdocmain\endinput\fi
%<package>\ProvidesFile{childdoc.def}[2018/12/30 v2.0 child document driver]
%<samplemain>\ProvidesFile{cdocsamp.tex}[2018/12/30 v2.0 sample for childdoc]
%<*driver>
%\ProvidesFile{childdoc.drv}[2018/12/30 v2.0 childdoc reference manual file]
\PassOptionsToClass{10pt,a4paper}{article}
\documentclass{ltxdoc}

\usepackage[margin=35mm]{geometry}
\usepackage{hyperref}
\usepackage{hyperxmp}
\usepackage[usenames]{color}

\hypersetup{colorlinks=true}
\hypersetup{pdfstartview=FitH}
\hypersetup{pdfpagemode=UseNone}
\hypersetup{pdfsource={}}
\hypersetup{pdflang={en-UK}}
\hypersetup{pdfcopyright={Copyright 2017-2018 Niklas Beisert.
  This work may be distributed and/or modified under the
  conditions of the LaTeX Project Public License, either version 1.3
  of this license or (at your option) any later version.}}
\hypersetup{pdflicenseurl={http://www.latex-project.org/lppl.txt}}
\hypersetup{pdfcontactaddress={ETH Zurich, ITP, HIT K,
  Wolfgang-Pauli-Strasse 27}}
\hypersetup{pdfcontactpostcode={8093}}
\hypersetup{pdfcontactcity={Zurich}}
\hypersetup{pdfcontactcountry={Switzerland}}
\hypersetup{pdfcontactemail={nbeisert@itp.phys.ethz.ch}}
\hypersetup{pdfcontacturl={http://people.phys.ethz.ch/\xmptilde nbeisert/}}

\newcommand{\secref}[1]{\hyperref[#1]{section \ref*{#1}}}

\parskip1ex
\parindent0pt
\let\olditemize\itemize
\def\itemize{\olditemize\parskip0pt}

\begin{document}

\title{The \textsf{childdoc} Package}
\hypersetup{pdftitle={The childdoc Package}}
\author{Niklas Beisert\\[2ex]
  Institut f\"ur Theoretische Physik\\
  Eidgen\"ossische Technische Hochschule Z\"urich\\
  Wolfgang-Pauli-Strasse 27, 8093 Z\"urich, Switzerland\\[1ex]
  \href{mailto:nbeisert@itp.phys.ethz.ch}
  {\texttt{nbeisert@itp.phys.ethz.ch}}}
\hypersetup{pdfauthor={Niklas Beisert}}
\hypersetup{pdfsubject={Manual for the LaTeX2e Package childdoc}}
\date{30 December 2018, \textsf{v2.0}}
\maketitle

\begin{abstract}\noindent
\textsf{childdoc} is a \LaTeXe{} package
that enables the direct compilation
of document sections included by |\include|
to individual files.
\end{abstract}

\begingroup
\parskip0ex
\tableofcontents
\endgroup

%%%%%%%%%%%%%%%%%%%%%%%%%%%%%%%%%%%%%%%%%%%%%%%%%%%%%%%%%%%%%%%%%%%%%%%%%%%%%%%%
%%%%%%%%%%%%%%%%%%%%%%%%%%%%%%%%%%%%%%%%%%%%%%%%%%%%%%%%%%%%%%%%%%%%%%%%%%%%%%%%
\section{Introduction}

\LaTeX{} provides a mechanism to structure a large document (such as a book)
into a main file and several child files (containing the chapters)
using the |\include| command.
This mechanism is beneficial for documents
which span hundreds of pages in order to
make the source file(s) more manageable.
Moreover, compilation can be restricted to
selected child files by means of the |\includeonly| command.
The latter feature can be used to reduce the compilation time while editing
(this was significantly more useful in the earlier days of \LaTeX{})
or to generate a smaller document which is easier to navigate.
Another application of |\includeonly| is to generate
documents consisting of selected parts of the complete document.

However, there are a few drawbacks of the plain |\include| mechanism:
\begin{itemize}
\item
The child files cannot be compiled on their own,
they can only be compiled via the main file.
A naive editing environment
(such as a text editor with an option
to have the current file processed by \LaTeX)
may require one to switch to the main file before compiling;
attempting to compile the child file produces errors.
\item
The main file must be modified (each time)
to adjust the |\includeonly| command
to the present needs. This easily leaves the main file in a messy state.
\item
The generated document will always carry the filename
of the main document. This is inconvenient if
several child files are to be compiled and
to be kept for distribution.
\end{itemize}

The present package provides a simple interface
to make child files individually compilable by \LaTeX{}.
Compiling a child file then has the same effect as compiling
the main file with an |\includeonly| command
to select the appropriate child.
Moreover the generated document will carry the name of the child
rather than the main file.
This resolves all three above issues.

This feature is meant to make the editing of books,
thesis documents and lecture notes somewhat more convenient.
However, the package can also be used efficiently for
composing a series of documents (such as exercise sheets)
which are typically distributed individually.
It then assists the author in generating the individual documents
(potentially in different versions)
as well as a document containing the collected series.
Another application is in developing style files
or other kinds of included material
where compilation of the style file could redirect
to a sample or test file.

%%%%%%%%%%%%%%%%%%%%%%%%%%%%%%%%%%%%%%%%%%%%%%%%%%%%%%%%%%%%%%%%%%%%%%%%%%%%%%%%
%%%%%%%%%%%%%%%%%%%%%%%%%%%%%%%%%%%%%%%%%%%%%%%%%%%%%%%%%%%%%%%%%%%%%%%%%%%%%%%%
\section{Usage}

First of all, the package \textsf{childdoc} is \emph{not} a standard
\LaTeXe{} |.sty| style file! Therefore it needs to be invoked in
a non-standard way.

%%%%%%%%%%%%%%%%%%%%%%%%%%%%%%%%%%%%%%%%%%%%%%%%%%%%%%%%%%%%%%%%%%%%%%%%%%%%%%%%
\subsection{Included Files}
\label{sec:include}

%%%%%%%%%%%%%%%%%%%%%%%%%%%%%%%%%%%%%%%%
\DescribeMacro{\childdocmain}
To use the package, add the commands
\begin{center}
\begin{tabular}{l}
|% \iffalse
%
% childdoc.dtx Copyright (C) 2017-2018 Niklas Beisert
%
% This work may be distributed and/or modified under the
% conditions of the LaTeX Project Public License, either version 1.3
% of this license or (at your option) any later version.
% The latest version of this license is in
%   http://www.latex-project.org/lppl.txt
% and version 1.3 or later is part of all distributions of LaTeX
% version 2005/12/01 or later.
%
% This work has the LPPL maintenance status `maintained'.
%
% The Current Maintainer of this work is Niklas Beisert.
%
% This work consists of the files childdoc.dtx and childdoc.ins
% and the derived files childdoc.def and cdocsamp.tex with
% cdocsch1.tex, cdocsch2.tex, cdocsdrf.tex, cdocsfn1.tex, cdocsfn2.tex.
%
%<package>\ifdefined\childdocmain\endinput\fi
%<package>\ProvidesFile{childdoc.def}[2018/12/30 v2.0 child document driver]
%<samplemain>\ProvidesFile{cdocsamp.tex}[2018/12/30 v2.0 sample for childdoc]
%<*driver>
%\ProvidesFile{childdoc.drv}[2018/12/30 v2.0 childdoc reference manual file]
\PassOptionsToClass{10pt,a4paper}{article}
\documentclass{ltxdoc}

\usepackage[margin=35mm]{geometry}
\usepackage{hyperref}
\usepackage{hyperxmp}
\usepackage[usenames]{color}

\hypersetup{colorlinks=true}
\hypersetup{pdfstartview=FitH}
\hypersetup{pdfpagemode=UseNone}
\hypersetup{pdfsource={}}
\hypersetup{pdflang={en-UK}}
\hypersetup{pdfcopyright={Copyright 2017-2018 Niklas Beisert.
  This work may be distributed and/or modified under the
  conditions of the LaTeX Project Public License, either version 1.3
  of this license or (at your option) any later version.}}
\hypersetup{pdflicenseurl={http://www.latex-project.org/lppl.txt}}
\hypersetup{pdfcontactaddress={ETH Zurich, ITP, HIT K,
  Wolfgang-Pauli-Strasse 27}}
\hypersetup{pdfcontactpostcode={8093}}
\hypersetup{pdfcontactcity={Zurich}}
\hypersetup{pdfcontactcountry={Switzerland}}
\hypersetup{pdfcontactemail={nbeisert@itp.phys.ethz.ch}}
\hypersetup{pdfcontacturl={http://people.phys.ethz.ch/\xmptilde nbeisert/}}

\newcommand{\secref}[1]{\hyperref[#1]{section \ref*{#1}}}

\parskip1ex
\parindent0pt
\let\olditemize\itemize
\def\itemize{\olditemize\parskip0pt}

\begin{document}

\title{The \textsf{childdoc} Package}
\hypersetup{pdftitle={The childdoc Package}}
\author{Niklas Beisert\\[2ex]
  Institut f\"ur Theoretische Physik\\
  Eidgen\"ossische Technische Hochschule Z\"urich\\
  Wolfgang-Pauli-Strasse 27, 8093 Z\"urich, Switzerland\\[1ex]
  \href{mailto:nbeisert@itp.phys.ethz.ch}
  {\texttt{nbeisert@itp.phys.ethz.ch}}}
\hypersetup{pdfauthor={Niklas Beisert}}
\hypersetup{pdfsubject={Manual for the LaTeX2e Package childdoc}}
\date{30 December 2018, \textsf{v2.0}}
\maketitle

\begin{abstract}\noindent
\textsf{childdoc} is a \LaTeXe{} package
that enables the direct compilation
of document sections included by |\include|
to individual files.
\end{abstract}

\begingroup
\parskip0ex
\tableofcontents
\endgroup

%%%%%%%%%%%%%%%%%%%%%%%%%%%%%%%%%%%%%%%%%%%%%%%%%%%%%%%%%%%%%%%%%%%%%%%%%%%%%%%%
%%%%%%%%%%%%%%%%%%%%%%%%%%%%%%%%%%%%%%%%%%%%%%%%%%%%%%%%%%%%%%%%%%%%%%%%%%%%%%%%
\section{Introduction}

\LaTeX{} provides a mechanism to structure a large document (such as a book)
into a main file and several child files (containing the chapters)
using the |\include| command.
This mechanism is beneficial for documents
which span hundreds of pages in order to
make the source file(s) more manageable.
Moreover, compilation can be restricted to
selected child files by means of the |\includeonly| command.
The latter feature can be used to reduce the compilation time while editing
(this was significantly more useful in the earlier days of \LaTeX{})
or to generate a smaller document which is easier to navigate.
Another application of |\includeonly| is to generate
documents consisting of selected parts of the complete document.

However, there are a few drawbacks of the plain |\include| mechanism:
\begin{itemize}
\item
The child files cannot be compiled on their own,
they can only be compiled via the main file.
A naive editing environment
(such as a text editor with an option
to have the current file processed by \LaTeX)
may require one to switch to the main file before compiling;
attempting to compile the child file produces errors.
\item
The main file must be modified (each time)
to adjust the |\includeonly| command
to the present needs. This easily leaves the main file in a messy state.
\item
The generated document will always carry the filename
of the main document. This is inconvenient if
several child files are to be compiled and
to be kept for distribution.
\end{itemize}

The present package provides a simple interface
to make child files individually compilable by \LaTeX{}.
Compiling a child file then has the same effect as compiling
the main file with an |\includeonly| command
to select the appropriate child.
Moreover the generated document will carry the name of the child
rather than the main file.
This resolves all three above issues.

This feature is meant to make the editing of books,
thesis documents and lecture notes somewhat more convenient.
However, the package can also be used efficiently for
composing a series of documents (such as exercise sheets)
which are typically distributed individually.
It then assists the author in generating the individual documents
(potentially in different versions)
as well as a document containing the collected series.
Another application is in developing style files
or other kinds of included material
where compilation of the style file could redirect
to a sample or test file.

%%%%%%%%%%%%%%%%%%%%%%%%%%%%%%%%%%%%%%%%%%%%%%%%%%%%%%%%%%%%%%%%%%%%%%%%%%%%%%%%
%%%%%%%%%%%%%%%%%%%%%%%%%%%%%%%%%%%%%%%%%%%%%%%%%%%%%%%%%%%%%%%%%%%%%%%%%%%%%%%%
\section{Usage}

First of all, the package \textsf{childdoc} is \emph{not} a standard
\LaTeXe{} |.sty| style file! Therefore it needs to be invoked in
a non-standard way.

%%%%%%%%%%%%%%%%%%%%%%%%%%%%%%%%%%%%%%%%%%%%%%%%%%%%%%%%%%%%%%%%%%%%%%%%%%%%%%%%
\subsection{Included Files}
\label{sec:include}

%%%%%%%%%%%%%%%%%%%%%%%%%%%%%%%%%%%%%%%%
\DescribeMacro{\childdocmain}
To use the package, add the commands
\begin{center}
\begin{tabular}{l}
|\input{childdoc.def}|\\
|\childdocmain{}|\\
\end{tabular}
\end{center}
at the very top of the main \LaTeX{} file,
in particular \emph{before} the |\documentclass| statement!
The argument of |\childdocmain| should be left empty
(but it must be present).

%%%%%%%%%%%%%%%%%%%%%%%%%%%%%%%%%%%%%%%%
\DescribeMacro{\childdocof}
Furthermore, add the commands
\begin{center}
\begin{tabular}{l}
|\input{childdoc.def}|\\
|\childdocof{|\textit{main}|}|\\
\end{tabular}
\end{center}
at the top of every child file \textit{child}
which is included by |\include{|\textit{child}|}|
from within the main file
(or at least for those files to be compiled individually).
The argument \textit{main} must be the filename of the main file.

There are a couple of
considerations in setting up the main and child documents:

%%%%%%%%%%%%%%%%%%%%%%%%%%%%%%%%%%%%%%%%
\paragraph{Restrictions.}

Please note the following restrictions:
\begin{itemize}
\item
|\childdocmain| must be called with one argument \textit{main}
to ensure compatibility with earlier version of the package.
It must either be empty (|\childdocmain{}|)
or precisely match the filename of the main file in which it is specified.
See \secref{sec:detection} for further information.
\item
The filename \textit{main} must be specified without the |.tex| extension.
\item
The filename \textit{main} is case sensitive
(even in case-insensitive file systems)
due to internal string comparison.
\item
The argument \textit{main} should be fully expanded, it cannot be a macro.
\item
Subdirectories and special characters should be avoided in filenames.
\item
The command |\childdocmain{|\textit{main}|}| must be followed by a whitespace.
It should not be followed immediately by another command
or by a comment mark `|%|'.
This is because the \TeX{} parser reads the token immediately following
the argument of |\childdocmain| and puts it
at the beginning of every child section;
however, a white\-space is ignored.
\end{itemize}

%%%%%%%%%%%%%%%%%%%%%%%%%%%%%%%%%%%%%%%%
\paragraph{Content of Main File.}

It is advisable to place all content in the child files included by |\include|.
Any output contained in the main file will appear in all child documents
unless suppressed manually;
it cannot be suppressed automatically by the |\includeonly| directive
and thus should normally be avoided.
A method to include some content in the main file
by means of conditional processing is described in \secref{sec:conditional}.

%%%%%%%%%%%%%%%%%%%%%%%%%%%%%%%%%%%%%%%%
\paragraph{Page Numbering.}

When only a part of the document is compiled,
the appropriate numbering of pages
(as well as other status parameters)
is determined from the |.aux| files.
The latter contain information from previous passes.
However this information needs to propagate through
all intermediate child documents.
Therefore the page numbering in child documents may well
be inconsistent until the complete document is compiled at least once.

A useful (if unconventional) way to always ensure a consistent
page numbering is to restart the numbering in each child document
and denote the pages by `\textit{child}|.|\textit{page}'
where \textit{child} represents the chapter/section number of the child file.
This can be achieved by the command
|\numberwithin{page}{|\textit{child}|}|
of the \textsf{amsmath} package
where \textit{child} can be |chapter| or |section|
depending on the chosen structuring.
Alternatively, one can modify the macro |\thepage| appropriately
and reset the counter |page| at the start of each child file.

%%%%%%%%%%%%%%%%%%%%%%%%%%%%%%%%%%%%%%%%%%%%%%%%%%%%%%%%%%%%%%%%%%%%%%%%%%%%%%%%
\subsection{Conditional Processing}
\label{sec:conditional}

The package provides a mechanism to compile different versions
of a document. To customise the versions further some conditional processing
can come in handy to distinguish which version is being compiled.
The package provides two macros to describe the compilation context:

%%%%%%%%%%%%%%%%%%%%%%%%%%%%%%%%%%%%%%%%
\DescribeMacro{\ifchilddoc}
The conditional |\ifchilddoc| distinguishes between the compilation of
child documents and the main document:
%
\begin{center}
|\ifchilddoc |\textit{child-code}| |[|\||else |\textit{main-code}]| \||fi|
\end{center}

%%%%%%%%%%%%%%%%%%%%%%%%%%%%%%%%%%%%%%%%
\DescribeMacro{\childdocname}
\DescribeMacro{\childdocjob}
The macro |\childdocname| contains the filename (without extension)
of the main or child file being processed.
Note that |\childdocjob| will always contain the name of the main file.

%%%%%%%%%%%%%%%%%%%%%%%%%%%%%%%%%%%%%%%%
\paragraph{Title Page.}

Conditional processing can be used to include a title or banner page
in the main document when proper precautions are taken.
Importantly, the code in the main file should ensure that the page counter
(as well as other status parameters which are stored in the |.aux| files)
takes the same value after the conditional processing.
Otherwise the page numbers may take divergent values
depending on which part is compiled.

For example, a title page could be declared by:
%
\begin{center}
\begin{tabular}{l}
|\ifchilddoc\||else|\\
|\addtocounter{page}{-1}|\\
\textit{code for title page}\\
|\newpage|\\
|\||fi|
\end{tabular}
\end{center}
%
A banner page for the child documents can be generated by:
%
\begin{center}
\begin{tabular}{l}
|\ifchilddoc|\\
|\addtocounter{page}{-1}|\\
\textit{code for banner page}\\
|\newpage|\\
|\||fi|
\end{tabular}
\end{center}
%
Here one could write a message such as:
\begin{center}
|This is the part \childdocname{} of \childdocjob{}.|
\end{center}

%%%%%%%%%%%%%%%%%%%%%%%%%%%%%%%%%%%%%%%%%%%%%%%%%%%%%%%%%%%%%%%%%%%%%%%%%%%%%%%%
\subsection{Flags}
\label{sec:flags}

The package makes it easy to generate different versions
of the main or child documents.
To this end compilation flags can be defined
and assigned different default values.
They will be particularly useful in conjunction
with the forwarding mechanism described in \secref{sec:forward}.

For example, it may be useful to have a flag |\version|
which can be set to |draft| or |final|.
The document source will contain some conditional code
depending on the value of |\version|.
Suppose further, the flag should default to |final| for the main file
and to |draft| for child files
which is a natural assignment for editing the document.
This is achieved by placing the following code
in the preamble of the main document
(below the |\childdocmain| directive):
%
\begin{center}
\begin{tabular}{l}
|\ifchilddoc|\\
|\providecommand{\version}{draft}|\\
|\||else|\\
|\providecommand{\version}{final}|\\
|\||fi|
\end{tabular}
\end{center}
%
The definition by |\providecommand| makes sure
that previous definitions are not overwritten.
Further statements |\providecommand{\version}{...}|
can thus be added before the above code to override it.

For the main file, one might add a line
(between |\childdocmain| and the above block)
%
\begin{center}
|%\ifchilddoc\||else\providecommand{\version}{draft}\||fi|
\end{center}
%
which can be uncommented to produce a draft version.
Likewise one can add a line to the very top of a child file
(above the |\childdocof{|\textit{main}|}| directive)
%
\begin{center}
|%\providecommand{\version}{final}|
\end{center}
%
which can be uncommented to produce the final version of this child document.

%%%%%%%%%%%%%%%%%%%%%%%%%%%%%%%%%%%%%%%%%%%%%%%%%%%%%%%%%%%%%%%%%%%%%%%%%%%%%%%%
\subsection{Forwarding}
\label{sec:forward}

Different versions of the main or child documents
using compilation flags as described in \secref{sec:flags}
can be (permanently) stored in different files
for convenient compilation, viewing and distribution.
To this end, the package defines a command
to pass on compilation to a different file:

%%%%%%%%%%%%%%%%%%%%%%%%%%%%%%%%%%%%%%%%
\DescribeMacro{\childdocforward}
The command |\childdocforward| redirects processing to
another source file:
%
\begin{center}
\begin{tabular}{l}
|\input{childdoc.def}|\\
|\childdocforward[|\textit{main}|]{|\textit{dest}|}|\\
\end{tabular}
\end{center}
%
The argument \textit{dest} is the destination file
(without extension).
It should be the main file or one of the child files.
Note that further \textsf{childdoc} directives
such as |\childdocof| and |\childdocforward|
in the indicated file will be processed in this form.
The optional argument \textit{main}
passes on directly to the main file \textit{main}
while pretending to compile the child \textit{dest}.
This form behaves as if \textit{dest}
issues |\childdocof{|\textit{main}|}| right away,
and no further \textsf{childdoc} directives will be processed.

%%%%%%%%%%%%%%%%%%%%%%%%%%%%%%%%%%%%%%%%
\DescribeMacro{\...prefix}
In the alternative form |\childdocforwardprefix|,
%
\begin{center}
\begin{tabular}{l}
|\input{childdoc.def}|\\
|\childdocforwardprefix[|\textit{main}|]{|\textit{prefix}|}{|\textit{dest}|}|
\end{tabular}
\end{center}
%
the destination file is determined by a pattern
depending on the current file:
To make this work, the current file must be called
`{\textit{prefix}\hspace{0.2em}\textit{suffix}}'
with \textit{prefix} matching precisely the argument.
Processing is then passed on to the file
`{\textit{dest}\hspace{0.2em}\textit{suffix}}'.
Surely, the same effect is achieved by
directly specifying the
argument `{\textit{dest}\hspace{0.2em}\textit{suffix}}'
in the first form.
However, that requires to set up a different file
for each child. With the alternative form of the command
all these files can have exactly the same content
which simplifies setting them up and maintaining them.

For example, the following file |draft.tex|
with a compilation flag |\version| as described in \secref{sec:flags}
compiles the main document as a draft:
%
\begin{center}
\begin{tabular}{l}
|\def\version{draft}|\\
|\input{childdoc.def}|\\
|\childdocforward{|\textit{main}|}|
\end{tabular}
\end{center}
%
Likewise, the following files |final|\textit{nn}|.tex|
compile the final version of the child document
|child|\textit{nn}|.tex|:
%
\begin{center}
\begin{tabular}{l}
|\def\version{final}|\\
|\input{childdoc.def}|\\
|\childdocforwardprefix{final}{child}|
\end{tabular}
\end{center}
%

Note that when several versions of a main file and/or of each child file
are to be generated, it may be convenient to set up a |Makefile| or
shell script to automatise the process.

%%%%%%%%%%%%%%%%%%%%%%%%%%%%%%%%%%%%%%%%%%%%%%%%%%%%%%%%%%%%%%%%%%%%%%%%%%%%%%%%
\subsection{Command Line Processing}
\label{sec:commandline}

The effect of redirection files can also be achieved by invoking
the \LaTeX{} compiler with a more elaborate command line.
Most conveniently this should be done as part
of a shell script or a |Makefile|.

When using \textsf{childdoc} in the main file, the following
command lines effectively perform a redirection
(note that depending on the shell being used,
backslashes may have to be doubled: `|\|' $\to$ `|\\|'):
%
\begin{center}
|... -jobname "|\textit{target}|" |\\|"|[\textit{flags}]%
|\input{childdoc.def}\childdocforward[|\textit{main}|]{|\textit{dest}|}"|
\end{center}
%
Here \textit{target} is the name of the output file,
\textit{main} is the name of the main file
and \textit{dest} is the name of the main or child file to be processed
(all filenames without extensions).
The optional argument \textit{main} can be omitted
if \textit{main} matches \textit{dest}.
Optionally, compilation \textit{flags} can be defined via |\def| commands.
This command line makes the \TeX{} engine believe
it is compiling the file \textit{target}
whose content is specified as the latter parameter.
The provided code then forwards the processing to
\textit{main} or \textit{dest} as described in \secref{sec:forward}.

%%%%%%%%%%%%%%%%%%%%%%%%%%%%%%%%%%%%%%%%%%%%%%%%%%%%%%%%%%%%%%%%%%%%%%%%%%%%%%%%
\subsection{Include by Input}
\label{sec:input}

Including child documents by |\include| has some restrictions by design.
Most notably, the content of a child document always occupies
its own set of pages; pages cannot be shared between child documents.
Usually, this behaviour makes perfect sense
because each child document contain an essential part of the document.
However, in some situations it may be desirable to compose
a document from a collection of parts
without having mandatory page breaks between then.
For this case, the package
provides a mechanism to include parts
by |\input| which can also be processed individually.
However, by construction this mechanism
requires manual handling of the content to be output.

%%%%%%%%%%%%%%%%%%%%%%%%%%%%%%%%%%%%%%%%
\DescribeMacro{\ifchilddocmanual}
The main file should be prepared as usual, see \secref{sec:include}.
However, the document body must make a distinction
between processing of an individual part and of the main document, e.g.:
%
\begin{center}
\begin{tabular}{l}
|\ifchilddocmanual|\\
|\input{\childdocname}|\\
|\||else|\\
\textit{document body with }|\input{|\textit{part}|}|\\
|\||fi|
\end{tabular}
\end{center}
%
The conditional |\ifchilddocmanual| is true whenever
a part to be included by |\input| is being compiled,
and the name of the part is stored in |\childdocname|.

%%%%%%%%%%%%%%%%%%%%%%%%%%%%%%%%%%%%%%%%
\DescribeMacro{\childdocby}
Each part to be included by |\input| should start with:
%
\begin{center}
\begin{tabular}{l}
|\input{childdoc.def}|\\
|\childdocby{|\textit{main}|}|\\
\end{tabular}
\end{center}
%
The directive |\childdocby| is similar to |\childdocof|
described in \secref{sec:include},
but the subsequent selection of content must be done manually.
To that end, both |\ifchilddoc| and |\ifchilddocmanual|
will be true upon processing of a part,
and the name of the part is stored in |\childdocname|.
Note that |\jobname| will be set to the filename of the current part
so that each part receives an individual |.aux| file
that does not interfere with the |.aux| file(s) of the main document.
This behaviour can be altered by the alternative form
|\childdocby[*]{|\textit{main}|}| (with a non-empty optional argument)
which uses the |.aux| file of the main document
by setting |\jobname| to \textit{main}.

%%%%%%%%%%%%%%%%%%%%%%%%%%%%%%%%%%%%%%%%%%%%%%%%%%%%%%%%%%%%%%%%%%%%%%%%%%%%%%%%
\subsection{Driver Development}
\label{sec:driver}

The \textsf{childdoc} mechanism can also be use for the development
of definition files such as \LaTeX{} styles or classes.
This case differs from the above setup with multiple parts
included by |\include| in that no |\includeonly| should be invoked.
This can be achieved by starting the include file
(before |\ProvidesPackage|) with:
%
\begin{center}
\begin{tabular}{l}
|\input{childdoc.def}|\\
|\childdocforward{|\textit{main}|}|\\
\end{tabular}
\end{center}
%
or alternatively with:
%
\begin{center}
\begin{tabular}{l}
|\input{childdoc.def}|\\
|\childdocby{|\textit{main}|}|\\
\end{tabular}
\end{center}
%
Both forms have slightly different effects as described above.
The main file is prepared as usual, see \secref{sec:include}.

%%%%%%%%%%%%%%%%%%%%%%%%%%%%%%%%%%%%%%%%%%%%%%%%%%%%%%%%%%%%%%%%%%%%%%%%%%%%%%%%
\subsection{Legacy Detection}
\label{sec:detection}

The directive |\childdocmain| in the main file can detect
whether the complete document or merely a child is to be compiled
even without using the directive |\childdocof|.
This method is deprecated because it is less robust
and there is no compelling reason to use it;
it is merely provided for backward compatibility
and it may be removed in future versions.

If the detection mechanism is to be used,
it is mandatory to correctly specify
the filename of the main file as the argument of |\childdocmain|:
%
\begin{center}
\begin{tabular}{l}
|\input{childdoc.def}|\\
|\childdocmain{|\textit{main}|}|\\
\end{tabular}
\end{center}
%
If |\jobname| does not match the argument \textit{main} of |\childdocmain|,
it is assumed that |\jobname| points to the child file to be compiled.
When using |\childdocmain| with the main file specified as argument,
it suffices to start a child file
with just |\input{|\textit{main}|}|
without loading of the package and using |\childdocof|.
If instead all processing is done
with the appropriate \textsf{childdoc} directives,
the argument of \textit{main} of |\childdocmain| can be empty.

An alternative version of the command line processing described
in \secref{sec:commandline} using the detection mechanism reads:
%
\begin{center}
|... -jobname "|\textit{target}|" "|[\textit{flags}]%
[|\def\jobname{|\textit{dest}|}|]|\input{|\textit{main}|}"|
\end{center}

%%%%%%%%%%%%%%%%%%%%%%%%%%%%%%%%%%%%%%%%%%%%%%%%%%%%%%%%%%%%%%%%%%%%%%%%%%%%%%%%
\subsection{Manual Code}
\label{sec:manual}

In case one cannot be certain whether the definitions file |childdoc.def|
is installed on the target \TeX{} distribution
and one prefers not to ship it,
it is conceivable to paste a few relevant commands into the sources.

To that end, drop all statements |\input{childdoc.def}|
and perform the replacements as outlined below.
Instead of |\childdocmain{|\textit{main}|}| add the following code
to the top of the main file:
%
\begin{center}
\begin{tabular}{l}
|\||ifdefined\childdocname\endinput\||fi\newif\ifchilddoc|\\
|\edef\childdocname{\scantokens\expandafter{\jobname\noexpand}}|\\
|\def\childdocmain{|\textit{main}|}\||ifx\childdocmain\childdocname\||else|\\
|\childdoctrue\includeonly{\childdocname}\let\jobname\childdocmain\||fi|\\
\end{tabular}
\end{center}
%
Instead of |\childdocof{|\textit{main}|}| just include the main file
at the top of each child file:
%
\begin{center}
|\input{|\textit{main}|}|
\end{center}
%
A simple redirection |\childdocforward{|\textit{dest}|}| is achieved by:
%
\begin{center}
|\def\jobname{|\textit{dest}|}\input{\jobname}|
\end{center}
%
The redirection with prefix
|\childdocforwardprefix[|\textit{prefix}|]{|\textit{dest}|}|
is accomplished by:
%
\begin{center}
\begin{tabular}{l}
|{\edef\jobname{\scantokens\expandafter{\jobname\noexpand}}|\\
|\def\redirectjob |\textit{prefix}|#1~~~{\gdef\jobname{|\textit{dest}|#1}}|\\
|\expandafter\redirectjob\jobname~~~}\input{\jobname}|
\end{tabular}
\end{center}

In an alternative approach,
child documents can be compiled by a specific command line
without additional code or specific definitions:
%
\begin{center}
|... -jobname "|\textit{target}|" "|[\textit{flags}]%
|\includeonly{|\textit{dest}|}\input{|\textit{main}|}"|
\end{center}
%

%%%%%%%%%%%%%%%%%%%%%%%%%%%%%%%%%%%%%%%%%%%%%%%%%%%%%%%%%%%%%%%%%%%%%%%%%%%%%%%%
%%%%%%%%%%%%%%%%%%%%%%%%%%%%%%%%%%%%%%%%%%%%%%%%%%%%%%%%%%%%%%%%%%%%%%%%%%%%%%%%
\section{Information}

%%%%%%%%%%%%%%%%%%%%%%%%%%%%%%%%%%%%%%%%%%%%%%%%%%%%%%%%%%%%%%%%%%%%%%%%%%%%%%%%
\subsection{Copyright}

Copyright \copyright{} 2017--2018 Niklas Beisert

This work may be distributed and/or modified under the
conditions of the \LaTeX{} Project Public License, either version 1.3
of this license or (at your option) any later version.
The latest version of this license is in
  \url{http://www.latex-project.org/lppl.txt}
and version 1.3 or later is part of all distributions of \LaTeX{}
version 2005/12/01 or later.

This work has the LPPL maintenance status `maintained'.

The Current Maintainer of this work is Niklas Beisert.

This work consists of the files |README.txt|, |childdoc.ins| and |childdoc.dtx|
as well as the derived files |childdoc.def|, |cdocsamp.tex|
with |cdocsch1.tex|, |cdocsch2.tex|, |cdocspt3.tex|, |cdocspt4.tex|,
|cdocsdrf.tex|, |cdocsfn1.tex|, |cdocsfn2.tex|
as well as |childdoc.pdf|.

%%%%%%%%%%%%%%%%%%%%%%%%%%%%%%%%%%%%%%%%%%%%%%%%%%%%%%%%%%%%%%%%%%%%%%%%%%%%%%%%
\subsection{Files and Installation}

The package consists of the files:
%
\begin{center}
\begin{tabular}{ll}
    |README.txt|   & readme file \\
    |childdoc.ins| & installation file \\
    |childdoc.dtx| & source file \\
    |childdoc.def| & definition file \\
    |cdocsamp.tex| & sample main file \\
    |cdocsch1.tex| & sample include file \\
    |cdocsch2.tex| & sample include file \\
    |cdocspt3.tex| & sample part file \\
    |cdocspt4.tex| & sample part file \\
    |cdocsdrf.tex| & sample redirection file \\
    |cdocsfn1.tex| & sample redirection file \\
    |cdocsfn2.tex| & sample redirection file \\
    |childdoc.pdf| & manual
\end{tabular}
\end{center}
%
The distribution consists of the files
|README.txt|, |childdoc.ins| and |childdoc.dtx|.
%
\begin{itemize}
\item
Run (pdf)\LaTeX{} on |childdoc.dtx|
to compile the manual |childdoc.pdf| (this file).
\item
Run \LaTeX{} on |childdoc.ins| to create the definitions file |childdoc.def|
and the sample |cdocsamp.tex| with include files
|cdocsch1.tex|, |cdocsch2.tex|, |cdocspt3.tex|, |cdocspt4.tex|,
|cdocsdrf.tex|, |cdocsfn1.tex|, |cdocsfn2.tex|.
Then copy the file |childdoc.def| to an appropriate directory of your \LaTeX{}
distribution, e.g.\ \textit{texmf-root}|/tex/latex/childdoc|.
\end{itemize}

%%%%%%%%%%%%%%%%%%%%%%%%%%%%%%%%%%%%%%%%%%%%%%%%%%%%%%%%%%%%%%%%%%%%%%%%%%%%%%%%
\subsection{Related CTAN Packages}

There are several other packages which offer a similar functionality:
%
\begin{itemize}
\item
The packages
\href{http://ctan.org/pkg/docmute}{\textsf{docmute}},
\href{http://ctan.org/pkg/includex}{\textsf{includex}} and
\href{http://ctan.org/pkg/standalone}{\textsf{standalone}}
provide commands to include only the document body of
a child file thus allowing both files to be compiled individually.
\item
The packages \href{http://ctan.org/pkg/subdocs}{\textsf{subdocs}}
and \href{http://ctan.org/pkg/subfiles}{\textsf{subfiles}}
provide structures in which the main and child documents can be
encapsulated and allowing them to be compiled individually.
The inclusion mechanism is different from the conventional |\include|.
\item
The package \href{http://ctan.org/pkg/combine}{\textsf{combine}}
is an elaborate solution to combine several documents into one.
\end{itemize}
%
See also the CTAN topic \href{http://ctan.org/topic/subdocs}{\textsf{subdocs}}
for further related packages.
The present package differs from the above solutions in that
a document structure constructed with the conventional |\include| mechanism
just needs two extra commands at the top of every file
such that all constituent files can be compiled individually.

%%%%%%%%%%%%%%%%%%%%%%%%%%%%%%%%%%%%%%%%%%%%%%%%%%%%%%%%%%%%%%%%%%%%%%%%%%%%%%%%
%\subsection{Feature Suggestions}
%
%The following is a list of features which may be useful for future
%versions of this package:
%%
%\begin{itemize}
%\item
%\ldots
%\end{itemize}

%%%%%%%%%%%%%%%%%%%%%%%%%%%%%%%%%%%%%%%%%%%%%%%%%%%%%%%%%%%%%%%%%%%%%%%%%%%%%%%%
\subsection{Revision History}

%%%%%%%%%%%%%%%%%%%%%%%%%%%%%%%%%%%%%%%%
\paragraph{v2.0:} 2018/12/30

\begin{itemize}
\item
immediate forward processing
\item
added |\childdocby| mechanism
\item
manual restructured
\end{itemize}

%%%%%%%%%%%%%%%%%%%%%%%%%%%%%%%%%%%%%%%%
\paragraph{v1.6:} 2018/01/17

\begin{itemize}
\item
application for development of include files
\item
corrections to manual
\end{itemize}

%%%%%%%%%%%%%%%%%%%%%%%%%%%%%%%%%%%%%%%%
\paragraph{v1.5:} 2017/05/21

\begin{itemize}
\item
more complete structuring introduced
\item
|\childdocof| introduced
\item
|\childdoc| renamed to |\childdocmain|
\item
|\childredirect| renamed to |\childdocforward| and |\childdocforwardprefix|
and functionality expanded
\end{itemize}

%%%%%%%%%%%%%%%%%%%%%%%%%%%%%%%%%%%%%%%%
\paragraph{v1.0:} 2017/04/27

\begin{itemize}
\item
manual and install package
\item
first version published on CTAN
\end{itemize}

%%%%%%%%%%%%%%%%%%%%%%%%%%%%%%%%%%%%%%%%
\paragraph{v0.6:} 2017/04/26

\begin{itemize}
\item
redirection mechanism added
\end{itemize}

%%%%%%%%%%%%%%%%%%%%%%%%%%%%%%%%%%%%%%%%
\paragraph{v0.5:} 2017/04/26

\begin{itemize}
\item
functionality in definition file
\end{itemize}


%%%%%%%%%%%%%%%%%%%%%%%%%%%%%%%%%%%%%%%%%%%%%%%%%%%%%%%%%%%%%%%%%%%%%%%%%%%%%%%%
%%%%%%%%%%%%%%%%%%%%%%%%%%%%%%%%%%%%%%%%%%%%%%%%%%%%%%%%%%%%%%%%%%%%%%%%%%%%%%%%
%%%%%%%%%%%%%%%%%%%%%%%%%%%%%%%%%%%%%%%%%%%%%%%%%%%%%%%%%%%%%%%%%%%%%%%%%%%%%%%%
\appendix

\settowidth\MacroIndent{\rmfamily\scriptsize 000\ }

 \DocInput{childdoc.dtx}

\end{document}
%</driver>
% \fi
%
% %%%%%%%%%%%%%%%%%%%%%%%%%%%%%%%%%%%%%%%%%%%%%%%%%%%%%%%%%%%%%%%%%%%%%%%%%%%%%%
% %%%%%%%%%%%%%%%%%%%%%%%%%%%%%%%%%%%%%%%%%%%%%%%%%%%%%%%%%%%%%%%%%%%%%%%%%%%%%%
% \section{Sample}
%\iffalse
%<*samplemain>
%\fi
%
% The following presents a sample document
% with two chapters, two parts, a title page,
% a compile flag as well as three forwarding files to set the flag.
% It consists of eight |.tex| files:
% \begin{center}
% \begin{tabular}{ll}
% |cdocsamp.tex|&main file\\
% |cdocsch1.tex|&include file for chapter 1\\
% |cdocsch2.tex|&include file for chapter 2\\
% |cdocspt3.tex|&include file for part 3\\
% |cdocspt4.tex|&include file for part 4\\
% |cdocsdrf.tex|&forwarding file for main file in draft mode\\
% |cdocsfi1.tex|&forwarding file for final version of chapter 1\\
% |cdocsfi2.tex|&forwarding file for final version of chapter 2\\
% \end{tabular}
% \end{center}
% Each of the eight files can be compiled directly by the \LaTeX{} compiler.
%
% %%%%%%%%%%%%%%%%%%%%%%%%%%%%%%%%%%%%%%
% \paragraph{Main File.}
%
% The main file is called |cdocsamp.tex|.
%
% Load the \textsf{childdoc} definitions and
% declare the filename for the main document:
%    \begin{macrocode}
\input{childdoc.def}
\childdocmain{}
%    \end{macrocode}

% Optional override for |\version| flag:
%    \begin{macrocode}
%%\ifchilddoc\else\providecommand{\version}{draft}\fi
%    \end{macrocode}

% Define the default values for the |\version| flag
% (|final| for the main file and |draft| for childs):
%    \begin{macrocode}
\ifchilddoc
\providecommand{\version}{draft}
\else
\providecommand{\version}{final}
\fi
%    \end{macrocode}

% Load the standard document class:
%    \begin{macrocode}
\documentclass[12pt]{article}
%    \end{macrocode}

% Start the document body:
%    \begin{macrocode}
\begin{document}
%    \end{macrocode}

% Declare a title page.
% Print title, part of document being processed and version flag:
%    \begin{macrocode}
\addtocounter{page}{-1}
\begin{center}
{\LARGE\bfseries{}childdoc example\par}
\vspace{1cm}
\ifchilddoc
\ifchilddocmanual part\else chapter\fi:
`\childdocname' of `\childdocjob'\par
\else
main document: `\childdocjob'\par
\fi
version: \version\par
\end{center}
\newpage
%    \end{macrocode}

% Manually include selected file,
% otherwise process as usual:
%    \begin{macrocode}
\ifchilddocmanual
\section*{part `\childdocname'}
\input{\childdocname}
\else
%    \end{macrocode}

% Include the two chapters:
%    \begin{macrocode}
\include{cdocsch1}
\include{cdocsch2}
%    \end{macrocode}

% Include the two parts unless only chapters should be displayed:
%    \begin{macrocode}
\ifchilddoc\else
\section{part three}
\input{cdocspt3}
\section{part four}
\input{cdocspt4}
\fi
%    \end{macrocode}

% Process as usual until here:
%    \begin{macrocode}
\fi
%    \end{macrocode}

% End of document body:
%    \begin{macrocode}
\end{document}
%    \end{macrocode}
%\iffalse
%</samplemain>
%\fi
%
% %%%%%%%%%%%%%%%%%%%%%%%%%%%%%%%%%%%%%%
% \paragraph{Chapter Include Files.}
%
% The include files are called |cdocsch1.tex| and |cdocsch2.tex|.
%
%\iffalse
%<*samplechap1|samplechap2>
%\fi

% Optional override for |\version| flag:
%    \begin{macrocode}
%%\providecommand{\version}{final}
%    \end{macrocode}

% Include the main document:
%    \begin{macrocode}
\input{childdoc.def}
\childdocof{cdocsamp}
%    \end{macrocode}

%\iffalse
%</samplechap1|samplechap2>
%\fi
%
%\iffalse
%<*samplechap1>
%\fi
% Some text for chapter 1:
%    \begin{macrocode}
\section{one}
some text in chapter one
%    \end{macrocode}

%\iffalse
%</samplechap1>
%\fi
% Some text for chapter 2:
%\iffalse
%<*samplechap2>
%\fi
%    \begin{macrocode}
\section{two}
more text in chapter two
%    \end{macrocode}

%\iffalse
%</samplechap2>
%\fi
%
% %%%%%%%%%%%%%%%%%%%%%%%%%%%%%%%%%%%%%%
% \paragraph{Part Include Files.}
%
% The include files are called |cdocspt3.tex| and |cdocspt4.tex|.
%
%\iffalse
%<*samplepart3|samplepart4>
%\fi

% Optional override for |\version| flag:
%    \begin{macrocode}
%%\providecommand{\version}{final}
%    \end{macrocode}

% Include the main document:
%    \begin{macrocode}
\input{childdoc.def}
\childdocby{cdocsamp}
%    \end{macrocode}

%\iffalse
%</samplepart3|samplepart4>
%\fi
%
%\iffalse
%<*samplepart3>
%\fi
% Some text for part 3:
%    \begin{macrocode}
some text in part three
%    \end{macrocode}

%\iffalse
%</samplepart3>
%\fi
% Some text for part 4:
%\iffalse
%<*samplepart4>
%\fi
%    \begin{macrocode}
more text in part four
%    \end{macrocode}

%\iffalse
%</samplepart4>
%\fi
%
% %%%%%%%%%%%%%%%%%%%%%%%%%%%%%%%%%%%%%%
% \paragraph{Forwarding for a Complete Draft.}
%
% The following forwarding file |cdocsdrf.tex|
% compiles the main document in draft mode:
%\iffalse
%<*sampledraft>
%\fi
%    \begin{macrocode}
\def\version{draft}
\input{childdoc.def}
\childdocforward{cdocsamp}
%    \end{macrocode}

%\iffalse
%</sampledraft>
%\fi
%
% %%%%%%%%%%%%%%%%%%%%%%%%%%%%%%%%%%%%%%
% \paragraph{Forwarding for Final Version of the Chapters.}
%
% The following forwarding files |cdocsfn1.tex| and |cdocsfn2.tex|
% (with identical content)
% compile the final versions of the child documents
% |cdocsch1.tex| and |cdocsch2.tex|, respectively:
%\iffalse
%<*samplefinal>
%\fi
%    \begin{macrocode}
\def\version{final}
\input{childdoc.def}
\childdocforwardprefix[cdocsamp]{cdocsfn}{cdocsch}
%    \end{macrocode}

%\iffalse
%</samplefinal>
%\fi
%
% %%%%%%%%%%%%%%%%%%%%%%%%%%%%%%%%%%%%%%
% \paragraph{Command Line Processing.}
%
% The following three command lines generate the output files
% |cdocscld|, |cdocscl1| and |cdocscl2|
% which should be identical to
% |cdocsdrf|, |cdocsch1| and |cdocsfn2|, respectively:
% \begin{center}
% \begin{tabular}{l}
% |latex -jobname cdocscld \|\\
% |  "\def\version{draft}\input{childdoc.def}\childdocforward{cdocsamp}"|\\
% |latex -jobname cdocscl1 \|\\
% |  "\input{childdoc.def}\childdocforward[cdocsamp]{cdocsch1}"|\\
% |latex -jobname cdocscl2 \|\\
% |  "\def\version{final}\input{childdoc.def}\childdocforward{cdocsch2}"|
% \end{tabular}
% \end{center}
% Note that the trailing backslash on each first line
% merely continues the input to the second line
% (for convenient cut ant paste).
% Furthermore, the command |latex| can be replaced by any
% of its alternative versions such as |pdflatex|.
%
% %%%%%%%%%%%%%%%%%%%%%%%%%%%%%%%%%%%%%%%%%%%%%%%%%%%%%%%%%%%%%%%%%%%%%%%%%%%%%%
% %%%%%%%%%%%%%%%%%%%%%%%%%%%%%%%%%%%%%%%%%%%%%%%%%%%%%%%%%%%%%%%%%%%%%%%%%%%%%%
% \section{Implementation}
%\iffalse
%<*package>
%\fi
%
% This section describes the definitions file |childdoc.def|.

% The definitions cannot be loaded using |\usepackage| or |\RequirePackage|
% which has a mechanism to prevent loading a style file more than once.
% When loading the definitions by means of |\input|
% multiple instances have to be prevented manually:
%\iffalse
%This code needs to be before the `\ProvidesFile' directive
%which is defined at the beginning of this file.
%Therefore it is also placed there and commented out here.
%</package>
%<*discard>
%\fi
%    \begin{macrocode}
\ifdefined\childdocmain\endinput\fi
%    \end{macrocode}
%\iffalse
%</discard>
%<*package>
%\fi
%
% \macro{\ifchilddoc}
% \macro{\ifchilddocmanual}
% The conditional |\ifchilddoc| tells whether a
% child (true) or main (false) document is being compiled.
% The conditional |\ifchilddocmanual| tells whether
% the |\includeonly| mechanism is used (false) or
% the selection of child files must be performed manually (true).
% The definitions initialise to false:
%    \begin{macrocode}
\newif\ifchilddoc
\newif\ifchilddocmanual
%    \end{macrocode}

% \macro{\childdocname}
% \macro{\childdocjob}
% The macro |\childdocname| stores the name of the main document
% to be compiled. The macro |\childdocjob| stores the name of
% the document on which the \LaTeX{} compiler was originally invoked.
% The content of |\jobname| cannot be compared
% to filenames specified in the source due to different catcodes.
% The following code rescans |\jobname|, stores the result
% in |\childdocname| and saves a copy in |\childdocjob|:
%    \begin{macrocode}
\edef\childdocname{\scantokens\expandafter{\jobname\noexpand}}
\let\childdocjob\childdocname
%    \end{macrocode}

% \macro{\childdocdisable}
% The macro |\childdocdisable| prevents the main file
% from being processed more than once.
% At this stage, the main document command |\childdocmain|
% is assumed to be called once again where it should do nothing.
% Any subsequent call to it should prevent
% a secondary processing of the main document
% It overwrites the forwarding commands
% |\childdocof| and |\childdocforward|
% with empty macros to prevent further inclusions of the main document:
%    \begin{macrocode}
\newcommand{\childdocdisable}
{
  \renewcommand{\childdocmain}[1]{\renewcommand{\childdocmain}[1]{\endinput}}
  \renewcommand{\childdocof}[1]{}
  \renewcommand{\childdocby}[2][]{}
  \renewcommand{\childdocforward}[2][]{}
  \renewcommand{\childdocdisable}{}
}
%    \end{macrocode}

% \macro{\childdocmain}
% The macro |\childdocmain| is to be called at the top of the main file
% with nothing or the main filename (without extension) as argument.
% First, it breaks loops.
% If the argument is not empty and does not match |\childdocname|
% (which is set by the first inclusion of |childdoc.def|),
% |\ifchilddoc| is set to true, |\includeonly| is applied to the child file
% and |\jobname| is set to the main file
% (for proper handling of |.aux| files):
%    \begin{macrocode}
\newcommand{\childdocmain}[1]
{
  \childdocdisable\childdocmain{}
  \if?#1?\else
    \begingroup
      \def\childdoctmp{#1}
      \ifx\childdoctmp\childdocname
        \def\childdoctmp{}
      \else
        \def\childdoctmp
        {
          \childdoctrue
          \includeonly{\childdocname}
          \def\childdocjob{#1}
          \def\jobname{#1}
        }
      \fi
      \expandafter
    \endgroup
    \childdoctmp
  \fi
}
%    \end{macrocode}

% \macro{\childdocof}
% The command |\childdocof| redirects
% compilation to the main file |#1|.
%    \begin{macrocode}
\newcommand{\childdocof}[1]
{
  \childdocdisable
  \childdoctrue
  \includeonly{\childdocname}
  \def\jobname{#1}
  \def\childdocjob{#1}
  \input{#1}
}
%    \end{macrocode}

% \macro{\childdocby}
% The command |\childdocby| ....
%    \begin{macrocode}
\newcommand{\childdocby}[2][]
{
  \childdocdisable
  \childdoctrue
  \childdocmanualtrue
  \if?#1?\else
    \def\jobname{#2}
  \fi
  \def\childdocjob{#2}
  \input{#2}
  \endinput
}
%    \end{macrocode}

% \macro{\childdocforward}
% The command |\childdocforward| redirects
% compilation to the main file or
% (if the optional argument is given) a child file.
% Parameters are set as if the main file
% or a child file starting with |\childdocof| was compiled.
% Then compilation is handed over to the main file:
%    \begin{macrocode}
\newcommand{\childdocforward}[2][]
{
  \begingroup
    \if?#1?
      \def\childdoctmp
      {
        \def\childdocname{#2}
        \def\childdocjob{#2}
        \def\jobname{#2}
        \input{#2}
        \endinput
      }
    \else
      \def\childdoctmp
      {
        \childdocdisable
        \def\childdocname{#2}
        \childdoctrue
        \includeonly{#2}
        \def\childdocjob{#1}
        \def\jobname{#1}
        \input{#1}
        \endinput
      }
    \fi
    \expandafter
  \endgroup
  \childdoctmp
}
%    \end{macrocode}

% \macro{\childdocforwardprefix}
% The command |\childdocforwardprefix| redirects
% compilation to the main or a child file by means of a pattern.
% The prefix |#1| in the current filename is replaced by |#2|
% and the suffix of the current filename is kept
% (it is assumed that the filename does not contain the substring `|~~~|'
% which is used as a delimiter).
% Compilation is handed over to the new file by |\childdocforward|:
%    \begin{macrocode}
\newcommand{\childdocforwardprefix}[3][]
{
  \begingroup
    \def\childdocextract #2##1~~~{\def\childdoctmp{\childdocforward[#1]{#3##1}}}
    \expandafter\childdocextract\childdocname~~~
    \expandafter
  \endgroup
  \childdoctmp
}
%    \end{macrocode}

% \macro{\childdoc}
% The deprecated macro |\childdoc| is a legacy version of |\childdocmain|:
%    \begin{macrocode}
\newcommand{\childdoc}{\childdocmain}
%    \end{macrocode}

% \macro{\childdocredirect}
% The deprecated macro |\childdocredirect| is a legacy version
% of |\childdocforward| and |\childdocforwardprefix|:
%    \begin{macrocode}
\newcommand{\childdocredirect}[2][]
{
  \begingroup
    \if?#1?
      \def\childdoctmp{\childdocforward{#2}}
    \else
      \def\childdoctmp{\childdocforwardprefix{#1}{#2}}
    \fi
    \expandafter
  \endgroup
  \childdoctmp
}
%    \end{macrocode}

%\iffalse
%</package>
%\fi
%
\endinput
|\\
|\childdocmain{}|\\
\end{tabular}
\end{center}
at the very top of the main \LaTeX{} file,
in particular \emph{before} the |\documentclass| statement!
The argument of |\childdocmain| should be left empty
(but it must be present).

%%%%%%%%%%%%%%%%%%%%%%%%%%%%%%%%%%%%%%%%
\DescribeMacro{\childdocof}
Furthermore, add the commands
\begin{center}
\begin{tabular}{l}
|% \iffalse
%
% childdoc.dtx Copyright (C) 2017-2018 Niklas Beisert
%
% This work may be distributed and/or modified under the
% conditions of the LaTeX Project Public License, either version 1.3
% of this license or (at your option) any later version.
% The latest version of this license is in
%   http://www.latex-project.org/lppl.txt
% and version 1.3 or later is part of all distributions of LaTeX
% version 2005/12/01 or later.
%
% This work has the LPPL maintenance status `maintained'.
%
% The Current Maintainer of this work is Niklas Beisert.
%
% This work consists of the files childdoc.dtx and childdoc.ins
% and the derived files childdoc.def and cdocsamp.tex with
% cdocsch1.tex, cdocsch2.tex, cdocsdrf.tex, cdocsfn1.tex, cdocsfn2.tex.
%
%<package>\ifdefined\childdocmain\endinput\fi
%<package>\ProvidesFile{childdoc.def}[2018/12/30 v2.0 child document driver]
%<samplemain>\ProvidesFile{cdocsamp.tex}[2018/12/30 v2.0 sample for childdoc]
%<*driver>
%\ProvidesFile{childdoc.drv}[2018/12/30 v2.0 childdoc reference manual file]
\PassOptionsToClass{10pt,a4paper}{article}
\documentclass{ltxdoc}

\usepackage[margin=35mm]{geometry}
\usepackage{hyperref}
\usepackage{hyperxmp}
\usepackage[usenames]{color}

\hypersetup{colorlinks=true}
\hypersetup{pdfstartview=FitH}
\hypersetup{pdfpagemode=UseNone}
\hypersetup{pdfsource={}}
\hypersetup{pdflang={en-UK}}
\hypersetup{pdfcopyright={Copyright 2017-2018 Niklas Beisert.
  This work may be distributed and/or modified under the
  conditions of the LaTeX Project Public License, either version 1.3
  of this license or (at your option) any later version.}}
\hypersetup{pdflicenseurl={http://www.latex-project.org/lppl.txt}}
\hypersetup{pdfcontactaddress={ETH Zurich, ITP, HIT K,
  Wolfgang-Pauli-Strasse 27}}
\hypersetup{pdfcontactpostcode={8093}}
\hypersetup{pdfcontactcity={Zurich}}
\hypersetup{pdfcontactcountry={Switzerland}}
\hypersetup{pdfcontactemail={nbeisert@itp.phys.ethz.ch}}
\hypersetup{pdfcontacturl={http://people.phys.ethz.ch/\xmptilde nbeisert/}}

\newcommand{\secref}[1]{\hyperref[#1]{section \ref*{#1}}}

\parskip1ex
\parindent0pt
\let\olditemize\itemize
\def\itemize{\olditemize\parskip0pt}

\begin{document}

\title{The \textsf{childdoc} Package}
\hypersetup{pdftitle={The childdoc Package}}
\author{Niklas Beisert\\[2ex]
  Institut f\"ur Theoretische Physik\\
  Eidgen\"ossische Technische Hochschule Z\"urich\\
  Wolfgang-Pauli-Strasse 27, 8093 Z\"urich, Switzerland\\[1ex]
  \href{mailto:nbeisert@itp.phys.ethz.ch}
  {\texttt{nbeisert@itp.phys.ethz.ch}}}
\hypersetup{pdfauthor={Niklas Beisert}}
\hypersetup{pdfsubject={Manual for the LaTeX2e Package childdoc}}
\date{30 December 2018, \textsf{v2.0}}
\maketitle

\begin{abstract}\noindent
\textsf{childdoc} is a \LaTeXe{} package
that enables the direct compilation
of document sections included by |\include|
to individual files.
\end{abstract}

\begingroup
\parskip0ex
\tableofcontents
\endgroup

%%%%%%%%%%%%%%%%%%%%%%%%%%%%%%%%%%%%%%%%%%%%%%%%%%%%%%%%%%%%%%%%%%%%%%%%%%%%%%%%
%%%%%%%%%%%%%%%%%%%%%%%%%%%%%%%%%%%%%%%%%%%%%%%%%%%%%%%%%%%%%%%%%%%%%%%%%%%%%%%%
\section{Introduction}

\LaTeX{} provides a mechanism to structure a large document (such as a book)
into a main file and several child files (containing the chapters)
using the |\include| command.
This mechanism is beneficial for documents
which span hundreds of pages in order to
make the source file(s) more manageable.
Moreover, compilation can be restricted to
selected child files by means of the |\includeonly| command.
The latter feature can be used to reduce the compilation time while editing
(this was significantly more useful in the earlier days of \LaTeX{})
or to generate a smaller document which is easier to navigate.
Another application of |\includeonly| is to generate
documents consisting of selected parts of the complete document.

However, there are a few drawbacks of the plain |\include| mechanism:
\begin{itemize}
\item
The child files cannot be compiled on their own,
they can only be compiled via the main file.
A naive editing environment
(such as a text editor with an option
to have the current file processed by \LaTeX)
may require one to switch to the main file before compiling;
attempting to compile the child file produces errors.
\item
The main file must be modified (each time)
to adjust the |\includeonly| command
to the present needs. This easily leaves the main file in a messy state.
\item
The generated document will always carry the filename
of the main document. This is inconvenient if
several child files are to be compiled and
to be kept for distribution.
\end{itemize}

The present package provides a simple interface
to make child files individually compilable by \LaTeX{}.
Compiling a child file then has the same effect as compiling
the main file with an |\includeonly| command
to select the appropriate child.
Moreover the generated document will carry the name of the child
rather than the main file.
This resolves all three above issues.

This feature is meant to make the editing of books,
thesis documents and lecture notes somewhat more convenient.
However, the package can also be used efficiently for
composing a series of documents (such as exercise sheets)
which are typically distributed individually.
It then assists the author in generating the individual documents
(potentially in different versions)
as well as a document containing the collected series.
Another application is in developing style files
or other kinds of included material
where compilation of the style file could redirect
to a sample or test file.

%%%%%%%%%%%%%%%%%%%%%%%%%%%%%%%%%%%%%%%%%%%%%%%%%%%%%%%%%%%%%%%%%%%%%%%%%%%%%%%%
%%%%%%%%%%%%%%%%%%%%%%%%%%%%%%%%%%%%%%%%%%%%%%%%%%%%%%%%%%%%%%%%%%%%%%%%%%%%%%%%
\section{Usage}

First of all, the package \textsf{childdoc} is \emph{not} a standard
\LaTeXe{} |.sty| style file! Therefore it needs to be invoked in
a non-standard way.

%%%%%%%%%%%%%%%%%%%%%%%%%%%%%%%%%%%%%%%%%%%%%%%%%%%%%%%%%%%%%%%%%%%%%%%%%%%%%%%%
\subsection{Included Files}
\label{sec:include}

%%%%%%%%%%%%%%%%%%%%%%%%%%%%%%%%%%%%%%%%
\DescribeMacro{\childdocmain}
To use the package, add the commands
\begin{center}
\begin{tabular}{l}
|\input{childdoc.def}|\\
|\childdocmain{}|\\
\end{tabular}
\end{center}
at the very top of the main \LaTeX{} file,
in particular \emph{before} the |\documentclass| statement!
The argument of |\childdocmain| should be left empty
(but it must be present).

%%%%%%%%%%%%%%%%%%%%%%%%%%%%%%%%%%%%%%%%
\DescribeMacro{\childdocof}
Furthermore, add the commands
\begin{center}
\begin{tabular}{l}
|\input{childdoc.def}|\\
|\childdocof{|\textit{main}|}|\\
\end{tabular}
\end{center}
at the top of every child file \textit{child}
which is included by |\include{|\textit{child}|}|
from within the main file
(or at least for those files to be compiled individually).
The argument \textit{main} must be the filename of the main file.

There are a couple of
considerations in setting up the main and child documents:

%%%%%%%%%%%%%%%%%%%%%%%%%%%%%%%%%%%%%%%%
\paragraph{Restrictions.}

Please note the following restrictions:
\begin{itemize}
\item
|\childdocmain| must be called with one argument \textit{main}
to ensure compatibility with earlier version of the package.
It must either be empty (|\childdocmain{}|)
or precisely match the filename of the main file in which it is specified.
See \secref{sec:detection} for further information.
\item
The filename \textit{main} must be specified without the |.tex| extension.
\item
The filename \textit{main} is case sensitive
(even in case-insensitive file systems)
due to internal string comparison.
\item
The argument \textit{main} should be fully expanded, it cannot be a macro.
\item
Subdirectories and special characters should be avoided in filenames.
\item
The command |\childdocmain{|\textit{main}|}| must be followed by a whitespace.
It should not be followed immediately by another command
or by a comment mark `|%|'.
This is because the \TeX{} parser reads the token immediately following
the argument of |\childdocmain| and puts it
at the beginning of every child section;
however, a white\-space is ignored.
\end{itemize}

%%%%%%%%%%%%%%%%%%%%%%%%%%%%%%%%%%%%%%%%
\paragraph{Content of Main File.}

It is advisable to place all content in the child files included by |\include|.
Any output contained in the main file will appear in all child documents
unless suppressed manually;
it cannot be suppressed automatically by the |\includeonly| directive
and thus should normally be avoided.
A method to include some content in the main file
by means of conditional processing is described in \secref{sec:conditional}.

%%%%%%%%%%%%%%%%%%%%%%%%%%%%%%%%%%%%%%%%
\paragraph{Page Numbering.}

When only a part of the document is compiled,
the appropriate numbering of pages
(as well as other status parameters)
is determined from the |.aux| files.
The latter contain information from previous passes.
However this information needs to propagate through
all intermediate child documents.
Therefore the page numbering in child documents may well
be inconsistent until the complete document is compiled at least once.

A useful (if unconventional) way to always ensure a consistent
page numbering is to restart the numbering in each child document
and denote the pages by `\textit{child}|.|\textit{page}'
where \textit{child} represents the chapter/section number of the child file.
This can be achieved by the command
|\numberwithin{page}{|\textit{child}|}|
of the \textsf{amsmath} package
where \textit{child} can be |chapter| or |section|
depending on the chosen structuring.
Alternatively, one can modify the macro |\thepage| appropriately
and reset the counter |page| at the start of each child file.

%%%%%%%%%%%%%%%%%%%%%%%%%%%%%%%%%%%%%%%%%%%%%%%%%%%%%%%%%%%%%%%%%%%%%%%%%%%%%%%%
\subsection{Conditional Processing}
\label{sec:conditional}

The package provides a mechanism to compile different versions
of a document. To customise the versions further some conditional processing
can come in handy to distinguish which version is being compiled.
The package provides two macros to describe the compilation context:

%%%%%%%%%%%%%%%%%%%%%%%%%%%%%%%%%%%%%%%%
\DescribeMacro{\ifchilddoc}
The conditional |\ifchilddoc| distinguishes between the compilation of
child documents and the main document:
%
\begin{center}
|\ifchilddoc |\textit{child-code}| |[|\||else |\textit{main-code}]| \||fi|
\end{center}

%%%%%%%%%%%%%%%%%%%%%%%%%%%%%%%%%%%%%%%%
\DescribeMacro{\childdocname}
\DescribeMacro{\childdocjob}
The macro |\childdocname| contains the filename (without extension)
of the main or child file being processed.
Note that |\childdocjob| will always contain the name of the main file.

%%%%%%%%%%%%%%%%%%%%%%%%%%%%%%%%%%%%%%%%
\paragraph{Title Page.}

Conditional processing can be used to include a title or banner page
in the main document when proper precautions are taken.
Importantly, the code in the main file should ensure that the page counter
(as well as other status parameters which are stored in the |.aux| files)
takes the same value after the conditional processing.
Otherwise the page numbers may take divergent values
depending on which part is compiled.

For example, a title page could be declared by:
%
\begin{center}
\begin{tabular}{l}
|\ifchilddoc\||else|\\
|\addtocounter{page}{-1}|\\
\textit{code for title page}\\
|\newpage|\\
|\||fi|
\end{tabular}
\end{center}
%
A banner page for the child documents can be generated by:
%
\begin{center}
\begin{tabular}{l}
|\ifchilddoc|\\
|\addtocounter{page}{-1}|\\
\textit{code for banner page}\\
|\newpage|\\
|\||fi|
\end{tabular}
\end{center}
%
Here one could write a message such as:
\begin{center}
|This is the part \childdocname{} of \childdocjob{}.|
\end{center}

%%%%%%%%%%%%%%%%%%%%%%%%%%%%%%%%%%%%%%%%%%%%%%%%%%%%%%%%%%%%%%%%%%%%%%%%%%%%%%%%
\subsection{Flags}
\label{sec:flags}

The package makes it easy to generate different versions
of the main or child documents.
To this end compilation flags can be defined
and assigned different default values.
They will be particularly useful in conjunction
with the forwarding mechanism described in \secref{sec:forward}.

For example, it may be useful to have a flag |\version|
which can be set to |draft| or |final|.
The document source will contain some conditional code
depending on the value of |\version|.
Suppose further, the flag should default to |final| for the main file
and to |draft| for child files
which is a natural assignment for editing the document.
This is achieved by placing the following code
in the preamble of the main document
(below the |\childdocmain| directive):
%
\begin{center}
\begin{tabular}{l}
|\ifchilddoc|\\
|\providecommand{\version}{draft}|\\
|\||else|\\
|\providecommand{\version}{final}|\\
|\||fi|
\end{tabular}
\end{center}
%
The definition by |\providecommand| makes sure
that previous definitions are not overwritten.
Further statements |\providecommand{\version}{...}|
can thus be added before the above code to override it.

For the main file, one might add a line
(between |\childdocmain| and the above block)
%
\begin{center}
|%\ifchilddoc\||else\providecommand{\version}{draft}\||fi|
\end{center}
%
which can be uncommented to produce a draft version.
Likewise one can add a line to the very top of a child file
(above the |\childdocof{|\textit{main}|}| directive)
%
\begin{center}
|%\providecommand{\version}{final}|
\end{center}
%
which can be uncommented to produce the final version of this child document.

%%%%%%%%%%%%%%%%%%%%%%%%%%%%%%%%%%%%%%%%%%%%%%%%%%%%%%%%%%%%%%%%%%%%%%%%%%%%%%%%
\subsection{Forwarding}
\label{sec:forward}

Different versions of the main or child documents
using compilation flags as described in \secref{sec:flags}
can be (permanently) stored in different files
for convenient compilation, viewing and distribution.
To this end, the package defines a command
to pass on compilation to a different file:

%%%%%%%%%%%%%%%%%%%%%%%%%%%%%%%%%%%%%%%%
\DescribeMacro{\childdocforward}
The command |\childdocforward| redirects processing to
another source file:
%
\begin{center}
\begin{tabular}{l}
|\input{childdoc.def}|\\
|\childdocforward[|\textit{main}|]{|\textit{dest}|}|\\
\end{tabular}
\end{center}
%
The argument \textit{dest} is the destination file
(without extension).
It should be the main file or one of the child files.
Note that further \textsf{childdoc} directives
such as |\childdocof| and |\childdocforward|
in the indicated file will be processed in this form.
The optional argument \textit{main}
passes on directly to the main file \textit{main}
while pretending to compile the child \textit{dest}.
This form behaves as if \textit{dest}
issues |\childdocof{|\textit{main}|}| right away,
and no further \textsf{childdoc} directives will be processed.

%%%%%%%%%%%%%%%%%%%%%%%%%%%%%%%%%%%%%%%%
\DescribeMacro{\...prefix}
In the alternative form |\childdocforwardprefix|,
%
\begin{center}
\begin{tabular}{l}
|\input{childdoc.def}|\\
|\childdocforwardprefix[|\textit{main}|]{|\textit{prefix}|}{|\textit{dest}|}|
\end{tabular}
\end{center}
%
the destination file is determined by a pattern
depending on the current file:
To make this work, the current file must be called
`{\textit{prefix}\hspace{0.2em}\textit{suffix}}'
with \textit{prefix} matching precisely the argument.
Processing is then passed on to the file
`{\textit{dest}\hspace{0.2em}\textit{suffix}}'.
Surely, the same effect is achieved by
directly specifying the
argument `{\textit{dest}\hspace{0.2em}\textit{suffix}}'
in the first form.
However, that requires to set up a different file
for each child. With the alternative form of the command
all these files can have exactly the same content
which simplifies setting them up and maintaining them.

For example, the following file |draft.tex|
with a compilation flag |\version| as described in \secref{sec:flags}
compiles the main document as a draft:
%
\begin{center}
\begin{tabular}{l}
|\def\version{draft}|\\
|\input{childdoc.def}|\\
|\childdocforward{|\textit{main}|}|
\end{tabular}
\end{center}
%
Likewise, the following files |final|\textit{nn}|.tex|
compile the final version of the child document
|child|\textit{nn}|.tex|:
%
\begin{center}
\begin{tabular}{l}
|\def\version{final}|\\
|\input{childdoc.def}|\\
|\childdocforwardprefix{final}{child}|
\end{tabular}
\end{center}
%

Note that when several versions of a main file and/or of each child file
are to be generated, it may be convenient to set up a |Makefile| or
shell script to automatise the process.

%%%%%%%%%%%%%%%%%%%%%%%%%%%%%%%%%%%%%%%%%%%%%%%%%%%%%%%%%%%%%%%%%%%%%%%%%%%%%%%%
\subsection{Command Line Processing}
\label{sec:commandline}

The effect of redirection files can also be achieved by invoking
the \LaTeX{} compiler with a more elaborate command line.
Most conveniently this should be done as part
of a shell script or a |Makefile|.

When using \textsf{childdoc} in the main file, the following
command lines effectively perform a redirection
(note that depending on the shell being used,
backslashes may have to be doubled: `|\|' $\to$ `|\\|'):
%
\begin{center}
|... -jobname "|\textit{target}|" |\\|"|[\textit{flags}]%
|\input{childdoc.def}\childdocforward[|\textit{main}|]{|\textit{dest}|}"|
\end{center}
%
Here \textit{target} is the name of the output file,
\textit{main} is the name of the main file
and \textit{dest} is the name of the main or child file to be processed
(all filenames without extensions).
The optional argument \textit{main} can be omitted
if \textit{main} matches \textit{dest}.
Optionally, compilation \textit{flags} can be defined via |\def| commands.
This command line makes the \TeX{} engine believe
it is compiling the file \textit{target}
whose content is specified as the latter parameter.
The provided code then forwards the processing to
\textit{main} or \textit{dest} as described in \secref{sec:forward}.

%%%%%%%%%%%%%%%%%%%%%%%%%%%%%%%%%%%%%%%%%%%%%%%%%%%%%%%%%%%%%%%%%%%%%%%%%%%%%%%%
\subsection{Include by Input}
\label{sec:input}

Including child documents by |\include| has some restrictions by design.
Most notably, the content of a child document always occupies
its own set of pages; pages cannot be shared between child documents.
Usually, this behaviour makes perfect sense
because each child document contain an essential part of the document.
However, in some situations it may be desirable to compose
a document from a collection of parts
without having mandatory page breaks between then.
For this case, the package
provides a mechanism to include parts
by |\input| which can also be processed individually.
However, by construction this mechanism
requires manual handling of the content to be output.

%%%%%%%%%%%%%%%%%%%%%%%%%%%%%%%%%%%%%%%%
\DescribeMacro{\ifchilddocmanual}
The main file should be prepared as usual, see \secref{sec:include}.
However, the document body must make a distinction
between processing of an individual part and of the main document, e.g.:
%
\begin{center}
\begin{tabular}{l}
|\ifchilddocmanual|\\
|\input{\childdocname}|\\
|\||else|\\
\textit{document body with }|\input{|\textit{part}|}|\\
|\||fi|
\end{tabular}
\end{center}
%
The conditional |\ifchilddocmanual| is true whenever
a part to be included by |\input| is being compiled,
and the name of the part is stored in |\childdocname|.

%%%%%%%%%%%%%%%%%%%%%%%%%%%%%%%%%%%%%%%%
\DescribeMacro{\childdocby}
Each part to be included by |\input| should start with:
%
\begin{center}
\begin{tabular}{l}
|\input{childdoc.def}|\\
|\childdocby{|\textit{main}|}|\\
\end{tabular}
\end{center}
%
The directive |\childdocby| is similar to |\childdocof|
described in \secref{sec:include},
but the subsequent selection of content must be done manually.
To that end, both |\ifchilddoc| and |\ifchilddocmanual|
will be true upon processing of a part,
and the name of the part is stored in |\childdocname|.
Note that |\jobname| will be set to the filename of the current part
so that each part receives an individual |.aux| file
that does not interfere with the |.aux| file(s) of the main document.
This behaviour can be altered by the alternative form
|\childdocby[*]{|\textit{main}|}| (with a non-empty optional argument)
which uses the |.aux| file of the main document
by setting |\jobname| to \textit{main}.

%%%%%%%%%%%%%%%%%%%%%%%%%%%%%%%%%%%%%%%%%%%%%%%%%%%%%%%%%%%%%%%%%%%%%%%%%%%%%%%%
\subsection{Driver Development}
\label{sec:driver}

The \textsf{childdoc} mechanism can also be use for the development
of definition files such as \LaTeX{} styles or classes.
This case differs from the above setup with multiple parts
included by |\include| in that no |\includeonly| should be invoked.
This can be achieved by starting the include file
(before |\ProvidesPackage|) with:
%
\begin{center}
\begin{tabular}{l}
|\input{childdoc.def}|\\
|\childdocforward{|\textit{main}|}|\\
\end{tabular}
\end{center}
%
or alternatively with:
%
\begin{center}
\begin{tabular}{l}
|\input{childdoc.def}|\\
|\childdocby{|\textit{main}|}|\\
\end{tabular}
\end{center}
%
Both forms have slightly different effects as described above.
The main file is prepared as usual, see \secref{sec:include}.

%%%%%%%%%%%%%%%%%%%%%%%%%%%%%%%%%%%%%%%%%%%%%%%%%%%%%%%%%%%%%%%%%%%%%%%%%%%%%%%%
\subsection{Legacy Detection}
\label{sec:detection}

The directive |\childdocmain| in the main file can detect
whether the complete document or merely a child is to be compiled
even without using the directive |\childdocof|.
This method is deprecated because it is less robust
and there is no compelling reason to use it;
it is merely provided for backward compatibility
and it may be removed in future versions.

If the detection mechanism is to be used,
it is mandatory to correctly specify
the filename of the main file as the argument of |\childdocmain|:
%
\begin{center}
\begin{tabular}{l}
|\input{childdoc.def}|\\
|\childdocmain{|\textit{main}|}|\\
\end{tabular}
\end{center}
%
If |\jobname| does not match the argument \textit{main} of |\childdocmain|,
it is assumed that |\jobname| points to the child file to be compiled.
When using |\childdocmain| with the main file specified as argument,
it suffices to start a child file
with just |\input{|\textit{main}|}|
without loading of the package and using |\childdocof|.
If instead all processing is done
with the appropriate \textsf{childdoc} directives,
the argument of \textit{main} of |\childdocmain| can be empty.

An alternative version of the command line processing described
in \secref{sec:commandline} using the detection mechanism reads:
%
\begin{center}
|... -jobname "|\textit{target}|" "|[\textit{flags}]%
[|\def\jobname{|\textit{dest}|}|]|\input{|\textit{main}|}"|
\end{center}

%%%%%%%%%%%%%%%%%%%%%%%%%%%%%%%%%%%%%%%%%%%%%%%%%%%%%%%%%%%%%%%%%%%%%%%%%%%%%%%%
\subsection{Manual Code}
\label{sec:manual}

In case one cannot be certain whether the definitions file |childdoc.def|
is installed on the target \TeX{} distribution
and one prefers not to ship it,
it is conceivable to paste a few relevant commands into the sources.

To that end, drop all statements |\input{childdoc.def}|
and perform the replacements as outlined below.
Instead of |\childdocmain{|\textit{main}|}| add the following code
to the top of the main file:
%
\begin{center}
\begin{tabular}{l}
|\||ifdefined\childdocname\endinput\||fi\newif\ifchilddoc|\\
|\edef\childdocname{\scantokens\expandafter{\jobname\noexpand}}|\\
|\def\childdocmain{|\textit{main}|}\||ifx\childdocmain\childdocname\||else|\\
|\childdoctrue\includeonly{\childdocname}\let\jobname\childdocmain\||fi|\\
\end{tabular}
\end{center}
%
Instead of |\childdocof{|\textit{main}|}| just include the main file
at the top of each child file:
%
\begin{center}
|\input{|\textit{main}|}|
\end{center}
%
A simple redirection |\childdocforward{|\textit{dest}|}| is achieved by:
%
\begin{center}
|\def\jobname{|\textit{dest}|}\input{\jobname}|
\end{center}
%
The redirection with prefix
|\childdocforwardprefix[|\textit{prefix}|]{|\textit{dest}|}|
is accomplished by:
%
\begin{center}
\begin{tabular}{l}
|{\edef\jobname{\scantokens\expandafter{\jobname\noexpand}}|\\
|\def\redirectjob |\textit{prefix}|#1~~~{\gdef\jobname{|\textit{dest}|#1}}|\\
|\expandafter\redirectjob\jobname~~~}\input{\jobname}|
\end{tabular}
\end{center}

In an alternative approach,
child documents can be compiled by a specific command line
without additional code or specific definitions:
%
\begin{center}
|... -jobname "|\textit{target}|" "|[\textit{flags}]%
|\includeonly{|\textit{dest}|}\input{|\textit{main}|}"|
\end{center}
%

%%%%%%%%%%%%%%%%%%%%%%%%%%%%%%%%%%%%%%%%%%%%%%%%%%%%%%%%%%%%%%%%%%%%%%%%%%%%%%%%
%%%%%%%%%%%%%%%%%%%%%%%%%%%%%%%%%%%%%%%%%%%%%%%%%%%%%%%%%%%%%%%%%%%%%%%%%%%%%%%%
\section{Information}

%%%%%%%%%%%%%%%%%%%%%%%%%%%%%%%%%%%%%%%%%%%%%%%%%%%%%%%%%%%%%%%%%%%%%%%%%%%%%%%%
\subsection{Copyright}

Copyright \copyright{} 2017--2018 Niklas Beisert

This work may be distributed and/or modified under the
conditions of the \LaTeX{} Project Public License, either version 1.3
of this license or (at your option) any later version.
The latest version of this license is in
  \url{http://www.latex-project.org/lppl.txt}
and version 1.3 or later is part of all distributions of \LaTeX{}
version 2005/12/01 or later.

This work has the LPPL maintenance status `maintained'.

The Current Maintainer of this work is Niklas Beisert.

This work consists of the files |README.txt|, |childdoc.ins| and |childdoc.dtx|
as well as the derived files |childdoc.def|, |cdocsamp.tex|
with |cdocsch1.tex|, |cdocsch2.tex|, |cdocspt3.tex|, |cdocspt4.tex|,
|cdocsdrf.tex|, |cdocsfn1.tex|, |cdocsfn2.tex|
as well as |childdoc.pdf|.

%%%%%%%%%%%%%%%%%%%%%%%%%%%%%%%%%%%%%%%%%%%%%%%%%%%%%%%%%%%%%%%%%%%%%%%%%%%%%%%%
\subsection{Files and Installation}

The package consists of the files:
%
\begin{center}
\begin{tabular}{ll}
    |README.txt|   & readme file \\
    |childdoc.ins| & installation file \\
    |childdoc.dtx| & source file \\
    |childdoc.def| & definition file \\
    |cdocsamp.tex| & sample main file \\
    |cdocsch1.tex| & sample include file \\
    |cdocsch2.tex| & sample include file \\
    |cdocspt3.tex| & sample part file \\
    |cdocspt4.tex| & sample part file \\
    |cdocsdrf.tex| & sample redirection file \\
    |cdocsfn1.tex| & sample redirection file \\
    |cdocsfn2.tex| & sample redirection file \\
    |childdoc.pdf| & manual
\end{tabular}
\end{center}
%
The distribution consists of the files
|README.txt|, |childdoc.ins| and |childdoc.dtx|.
%
\begin{itemize}
\item
Run (pdf)\LaTeX{} on |childdoc.dtx|
to compile the manual |childdoc.pdf| (this file).
\item
Run \LaTeX{} on |childdoc.ins| to create the definitions file |childdoc.def|
and the sample |cdocsamp.tex| with include files
|cdocsch1.tex|, |cdocsch2.tex|, |cdocspt3.tex|, |cdocspt4.tex|,
|cdocsdrf.tex|, |cdocsfn1.tex|, |cdocsfn2.tex|.
Then copy the file |childdoc.def| to an appropriate directory of your \LaTeX{}
distribution, e.g.\ \textit{texmf-root}|/tex/latex/childdoc|.
\end{itemize}

%%%%%%%%%%%%%%%%%%%%%%%%%%%%%%%%%%%%%%%%%%%%%%%%%%%%%%%%%%%%%%%%%%%%%%%%%%%%%%%%
\subsection{Related CTAN Packages}

There are several other packages which offer a similar functionality:
%
\begin{itemize}
\item
The packages
\href{http://ctan.org/pkg/docmute}{\textsf{docmute}},
\href{http://ctan.org/pkg/includex}{\textsf{includex}} and
\href{http://ctan.org/pkg/standalone}{\textsf{standalone}}
provide commands to include only the document body of
a child file thus allowing both files to be compiled individually.
\item
The packages \href{http://ctan.org/pkg/subdocs}{\textsf{subdocs}}
and \href{http://ctan.org/pkg/subfiles}{\textsf{subfiles}}
provide structures in which the main and child documents can be
encapsulated and allowing them to be compiled individually.
The inclusion mechanism is different from the conventional |\include|.
\item
The package \href{http://ctan.org/pkg/combine}{\textsf{combine}}
is an elaborate solution to combine several documents into one.
\end{itemize}
%
See also the CTAN topic \href{http://ctan.org/topic/subdocs}{\textsf{subdocs}}
for further related packages.
The present package differs from the above solutions in that
a document structure constructed with the conventional |\include| mechanism
just needs two extra commands at the top of every file
such that all constituent files can be compiled individually.

%%%%%%%%%%%%%%%%%%%%%%%%%%%%%%%%%%%%%%%%%%%%%%%%%%%%%%%%%%%%%%%%%%%%%%%%%%%%%%%%
%\subsection{Feature Suggestions}
%
%The following is a list of features which may be useful for future
%versions of this package:
%%
%\begin{itemize}
%\item
%\ldots
%\end{itemize}

%%%%%%%%%%%%%%%%%%%%%%%%%%%%%%%%%%%%%%%%%%%%%%%%%%%%%%%%%%%%%%%%%%%%%%%%%%%%%%%%
\subsection{Revision History}

%%%%%%%%%%%%%%%%%%%%%%%%%%%%%%%%%%%%%%%%
\paragraph{v2.0:} 2018/12/30

\begin{itemize}
\item
immediate forward processing
\item
added |\childdocby| mechanism
\item
manual restructured
\end{itemize}

%%%%%%%%%%%%%%%%%%%%%%%%%%%%%%%%%%%%%%%%
\paragraph{v1.6:} 2018/01/17

\begin{itemize}
\item
application for development of include files
\item
corrections to manual
\end{itemize}

%%%%%%%%%%%%%%%%%%%%%%%%%%%%%%%%%%%%%%%%
\paragraph{v1.5:} 2017/05/21

\begin{itemize}
\item
more complete structuring introduced
\item
|\childdocof| introduced
\item
|\childdoc| renamed to |\childdocmain|
\item
|\childredirect| renamed to |\childdocforward| and |\childdocforwardprefix|
and functionality expanded
\end{itemize}

%%%%%%%%%%%%%%%%%%%%%%%%%%%%%%%%%%%%%%%%
\paragraph{v1.0:} 2017/04/27

\begin{itemize}
\item
manual and install package
\item
first version published on CTAN
\end{itemize}

%%%%%%%%%%%%%%%%%%%%%%%%%%%%%%%%%%%%%%%%
\paragraph{v0.6:} 2017/04/26

\begin{itemize}
\item
redirection mechanism added
\end{itemize}

%%%%%%%%%%%%%%%%%%%%%%%%%%%%%%%%%%%%%%%%
\paragraph{v0.5:} 2017/04/26

\begin{itemize}
\item
functionality in definition file
\end{itemize}


%%%%%%%%%%%%%%%%%%%%%%%%%%%%%%%%%%%%%%%%%%%%%%%%%%%%%%%%%%%%%%%%%%%%%%%%%%%%%%%%
%%%%%%%%%%%%%%%%%%%%%%%%%%%%%%%%%%%%%%%%%%%%%%%%%%%%%%%%%%%%%%%%%%%%%%%%%%%%%%%%
%%%%%%%%%%%%%%%%%%%%%%%%%%%%%%%%%%%%%%%%%%%%%%%%%%%%%%%%%%%%%%%%%%%%%%%%%%%%%%%%
\appendix

\settowidth\MacroIndent{\rmfamily\scriptsize 000\ }

 \DocInput{childdoc.dtx}

\end{document}
%</driver>
% \fi
%
% %%%%%%%%%%%%%%%%%%%%%%%%%%%%%%%%%%%%%%%%%%%%%%%%%%%%%%%%%%%%%%%%%%%%%%%%%%%%%%
% %%%%%%%%%%%%%%%%%%%%%%%%%%%%%%%%%%%%%%%%%%%%%%%%%%%%%%%%%%%%%%%%%%%%%%%%%%%%%%
% \section{Sample}
%\iffalse
%<*samplemain>
%\fi
%
% The following presents a sample document
% with two chapters, two parts, a title page,
% a compile flag as well as three forwarding files to set the flag.
% It consists of eight |.tex| files:
% \begin{center}
% \begin{tabular}{ll}
% |cdocsamp.tex|&main file\\
% |cdocsch1.tex|&include file for chapter 1\\
% |cdocsch2.tex|&include file for chapter 2\\
% |cdocspt3.tex|&include file for part 3\\
% |cdocspt4.tex|&include file for part 4\\
% |cdocsdrf.tex|&forwarding file for main file in draft mode\\
% |cdocsfi1.tex|&forwarding file for final version of chapter 1\\
% |cdocsfi2.tex|&forwarding file for final version of chapter 2\\
% \end{tabular}
% \end{center}
% Each of the eight files can be compiled directly by the \LaTeX{} compiler.
%
% %%%%%%%%%%%%%%%%%%%%%%%%%%%%%%%%%%%%%%
% \paragraph{Main File.}
%
% The main file is called |cdocsamp.tex|.
%
% Load the \textsf{childdoc} definitions and
% declare the filename for the main document:
%    \begin{macrocode}
\input{childdoc.def}
\childdocmain{}
%    \end{macrocode}

% Optional override for |\version| flag:
%    \begin{macrocode}
%%\ifchilddoc\else\providecommand{\version}{draft}\fi
%    \end{macrocode}

% Define the default values for the |\version| flag
% (|final| for the main file and |draft| for childs):
%    \begin{macrocode}
\ifchilddoc
\providecommand{\version}{draft}
\else
\providecommand{\version}{final}
\fi
%    \end{macrocode}

% Load the standard document class:
%    \begin{macrocode}
\documentclass[12pt]{article}
%    \end{macrocode}

% Start the document body:
%    \begin{macrocode}
\begin{document}
%    \end{macrocode}

% Declare a title page.
% Print title, part of document being processed and version flag:
%    \begin{macrocode}
\addtocounter{page}{-1}
\begin{center}
{\LARGE\bfseries{}childdoc example\par}
\vspace{1cm}
\ifchilddoc
\ifchilddocmanual part\else chapter\fi:
`\childdocname' of `\childdocjob'\par
\else
main document: `\childdocjob'\par
\fi
version: \version\par
\end{center}
\newpage
%    \end{macrocode}

% Manually include selected file,
% otherwise process as usual:
%    \begin{macrocode}
\ifchilddocmanual
\section*{part `\childdocname'}
\input{\childdocname}
\else
%    \end{macrocode}

% Include the two chapters:
%    \begin{macrocode}
\include{cdocsch1}
\include{cdocsch2}
%    \end{macrocode}

% Include the two parts unless only chapters should be displayed:
%    \begin{macrocode}
\ifchilddoc\else
\section{part three}
\input{cdocspt3}
\section{part four}
\input{cdocspt4}
\fi
%    \end{macrocode}

% Process as usual until here:
%    \begin{macrocode}
\fi
%    \end{macrocode}

% End of document body:
%    \begin{macrocode}
\end{document}
%    \end{macrocode}
%\iffalse
%</samplemain>
%\fi
%
% %%%%%%%%%%%%%%%%%%%%%%%%%%%%%%%%%%%%%%
% \paragraph{Chapter Include Files.}
%
% The include files are called |cdocsch1.tex| and |cdocsch2.tex|.
%
%\iffalse
%<*samplechap1|samplechap2>
%\fi

% Optional override for |\version| flag:
%    \begin{macrocode}
%%\providecommand{\version}{final}
%    \end{macrocode}

% Include the main document:
%    \begin{macrocode}
\input{childdoc.def}
\childdocof{cdocsamp}
%    \end{macrocode}

%\iffalse
%</samplechap1|samplechap2>
%\fi
%
%\iffalse
%<*samplechap1>
%\fi
% Some text for chapter 1:
%    \begin{macrocode}
\section{one}
some text in chapter one
%    \end{macrocode}

%\iffalse
%</samplechap1>
%\fi
% Some text for chapter 2:
%\iffalse
%<*samplechap2>
%\fi
%    \begin{macrocode}
\section{two}
more text in chapter two
%    \end{macrocode}

%\iffalse
%</samplechap2>
%\fi
%
% %%%%%%%%%%%%%%%%%%%%%%%%%%%%%%%%%%%%%%
% \paragraph{Part Include Files.}
%
% The include files are called |cdocspt3.tex| and |cdocspt4.tex|.
%
%\iffalse
%<*samplepart3|samplepart4>
%\fi

% Optional override for |\version| flag:
%    \begin{macrocode}
%%\providecommand{\version}{final}
%    \end{macrocode}

% Include the main document:
%    \begin{macrocode}
\input{childdoc.def}
\childdocby{cdocsamp}
%    \end{macrocode}

%\iffalse
%</samplepart3|samplepart4>
%\fi
%
%\iffalse
%<*samplepart3>
%\fi
% Some text for part 3:
%    \begin{macrocode}
some text in part three
%    \end{macrocode}

%\iffalse
%</samplepart3>
%\fi
% Some text for part 4:
%\iffalse
%<*samplepart4>
%\fi
%    \begin{macrocode}
more text in part four
%    \end{macrocode}

%\iffalse
%</samplepart4>
%\fi
%
% %%%%%%%%%%%%%%%%%%%%%%%%%%%%%%%%%%%%%%
% \paragraph{Forwarding for a Complete Draft.}
%
% The following forwarding file |cdocsdrf.tex|
% compiles the main document in draft mode:
%\iffalse
%<*sampledraft>
%\fi
%    \begin{macrocode}
\def\version{draft}
\input{childdoc.def}
\childdocforward{cdocsamp}
%    \end{macrocode}

%\iffalse
%</sampledraft>
%\fi
%
% %%%%%%%%%%%%%%%%%%%%%%%%%%%%%%%%%%%%%%
% \paragraph{Forwarding for Final Version of the Chapters.}
%
% The following forwarding files |cdocsfn1.tex| and |cdocsfn2.tex|
% (with identical content)
% compile the final versions of the child documents
% |cdocsch1.tex| and |cdocsch2.tex|, respectively:
%\iffalse
%<*samplefinal>
%\fi
%    \begin{macrocode}
\def\version{final}
\input{childdoc.def}
\childdocforwardprefix[cdocsamp]{cdocsfn}{cdocsch}
%    \end{macrocode}

%\iffalse
%</samplefinal>
%\fi
%
% %%%%%%%%%%%%%%%%%%%%%%%%%%%%%%%%%%%%%%
% \paragraph{Command Line Processing.}
%
% The following three command lines generate the output files
% |cdocscld|, |cdocscl1| and |cdocscl2|
% which should be identical to
% |cdocsdrf|, |cdocsch1| and |cdocsfn2|, respectively:
% \begin{center}
% \begin{tabular}{l}
% |latex -jobname cdocscld \|\\
% |  "\def\version{draft}\input{childdoc.def}\childdocforward{cdocsamp}"|\\
% |latex -jobname cdocscl1 \|\\
% |  "\input{childdoc.def}\childdocforward[cdocsamp]{cdocsch1}"|\\
% |latex -jobname cdocscl2 \|\\
% |  "\def\version{final}\input{childdoc.def}\childdocforward{cdocsch2}"|
% \end{tabular}
% \end{center}
% Note that the trailing backslash on each first line
% merely continues the input to the second line
% (for convenient cut ant paste).
% Furthermore, the command |latex| can be replaced by any
% of its alternative versions such as |pdflatex|.
%
% %%%%%%%%%%%%%%%%%%%%%%%%%%%%%%%%%%%%%%%%%%%%%%%%%%%%%%%%%%%%%%%%%%%%%%%%%%%%%%
% %%%%%%%%%%%%%%%%%%%%%%%%%%%%%%%%%%%%%%%%%%%%%%%%%%%%%%%%%%%%%%%%%%%%%%%%%%%%%%
% \section{Implementation}
%\iffalse
%<*package>
%\fi
%
% This section describes the definitions file |childdoc.def|.

% The definitions cannot be loaded using |\usepackage| or |\RequirePackage|
% which has a mechanism to prevent loading a style file more than once.
% When loading the definitions by means of |\input|
% multiple instances have to be prevented manually:
%\iffalse
%This code needs to be before the `\ProvidesFile' directive
%which is defined at the beginning of this file.
%Therefore it is also placed there and commented out here.
%</package>
%<*discard>
%\fi
%    \begin{macrocode}
\ifdefined\childdocmain\endinput\fi
%    \end{macrocode}
%\iffalse
%</discard>
%<*package>
%\fi
%
% \macro{\ifchilddoc}
% \macro{\ifchilddocmanual}
% The conditional |\ifchilddoc| tells whether a
% child (true) or main (false) document is being compiled.
% The conditional |\ifchilddocmanual| tells whether
% the |\includeonly| mechanism is used (false) or
% the selection of child files must be performed manually (true).
% The definitions initialise to false:
%    \begin{macrocode}
\newif\ifchilddoc
\newif\ifchilddocmanual
%    \end{macrocode}

% \macro{\childdocname}
% \macro{\childdocjob}
% The macro |\childdocname| stores the name of the main document
% to be compiled. The macro |\childdocjob| stores the name of
% the document on which the \LaTeX{} compiler was originally invoked.
% The content of |\jobname| cannot be compared
% to filenames specified in the source due to different catcodes.
% The following code rescans |\jobname|, stores the result
% in |\childdocname| and saves a copy in |\childdocjob|:
%    \begin{macrocode}
\edef\childdocname{\scantokens\expandafter{\jobname\noexpand}}
\let\childdocjob\childdocname
%    \end{macrocode}

% \macro{\childdocdisable}
% The macro |\childdocdisable| prevents the main file
% from being processed more than once.
% At this stage, the main document command |\childdocmain|
% is assumed to be called once again where it should do nothing.
% Any subsequent call to it should prevent
% a secondary processing of the main document
% It overwrites the forwarding commands
% |\childdocof| and |\childdocforward|
% with empty macros to prevent further inclusions of the main document:
%    \begin{macrocode}
\newcommand{\childdocdisable}
{
  \renewcommand{\childdocmain}[1]{\renewcommand{\childdocmain}[1]{\endinput}}
  \renewcommand{\childdocof}[1]{}
  \renewcommand{\childdocby}[2][]{}
  \renewcommand{\childdocforward}[2][]{}
  \renewcommand{\childdocdisable}{}
}
%    \end{macrocode}

% \macro{\childdocmain}
% The macro |\childdocmain| is to be called at the top of the main file
% with nothing or the main filename (without extension) as argument.
% First, it breaks loops.
% If the argument is not empty and does not match |\childdocname|
% (which is set by the first inclusion of |childdoc.def|),
% |\ifchilddoc| is set to true, |\includeonly| is applied to the child file
% and |\jobname| is set to the main file
% (for proper handling of |.aux| files):
%    \begin{macrocode}
\newcommand{\childdocmain}[1]
{
  \childdocdisable\childdocmain{}
  \if?#1?\else
    \begingroup
      \def\childdoctmp{#1}
      \ifx\childdoctmp\childdocname
        \def\childdoctmp{}
      \else
        \def\childdoctmp
        {
          \childdoctrue
          \includeonly{\childdocname}
          \def\childdocjob{#1}
          \def\jobname{#1}
        }
      \fi
      \expandafter
    \endgroup
    \childdoctmp
  \fi
}
%    \end{macrocode}

% \macro{\childdocof}
% The command |\childdocof| redirects
% compilation to the main file |#1|.
%    \begin{macrocode}
\newcommand{\childdocof}[1]
{
  \childdocdisable
  \childdoctrue
  \includeonly{\childdocname}
  \def\jobname{#1}
  \def\childdocjob{#1}
  \input{#1}
}
%    \end{macrocode}

% \macro{\childdocby}
% The command |\childdocby| ....
%    \begin{macrocode}
\newcommand{\childdocby}[2][]
{
  \childdocdisable
  \childdoctrue
  \childdocmanualtrue
  \if?#1?\else
    \def\jobname{#2}
  \fi
  \def\childdocjob{#2}
  \input{#2}
  \endinput
}
%    \end{macrocode}

% \macro{\childdocforward}
% The command |\childdocforward| redirects
% compilation to the main file or
% (if the optional argument is given) a child file.
% Parameters are set as if the main file
% or a child file starting with |\childdocof| was compiled.
% Then compilation is handed over to the main file:
%    \begin{macrocode}
\newcommand{\childdocforward}[2][]
{
  \begingroup
    \if?#1?
      \def\childdoctmp
      {
        \def\childdocname{#2}
        \def\childdocjob{#2}
        \def\jobname{#2}
        \input{#2}
        \endinput
      }
    \else
      \def\childdoctmp
      {
        \childdocdisable
        \def\childdocname{#2}
        \childdoctrue
        \includeonly{#2}
        \def\childdocjob{#1}
        \def\jobname{#1}
        \input{#1}
        \endinput
      }
    \fi
    \expandafter
  \endgroup
  \childdoctmp
}
%    \end{macrocode}

% \macro{\childdocforwardprefix}
% The command |\childdocforwardprefix| redirects
% compilation to the main or a child file by means of a pattern.
% The prefix |#1| in the current filename is replaced by |#2|
% and the suffix of the current filename is kept
% (it is assumed that the filename does not contain the substring `|~~~|'
% which is used as a delimiter).
% Compilation is handed over to the new file by |\childdocforward|:
%    \begin{macrocode}
\newcommand{\childdocforwardprefix}[3][]
{
  \begingroup
    \def\childdocextract #2##1~~~{\def\childdoctmp{\childdocforward[#1]{#3##1}}}
    \expandafter\childdocextract\childdocname~~~
    \expandafter
  \endgroup
  \childdoctmp
}
%    \end{macrocode}

% \macro{\childdoc}
% The deprecated macro |\childdoc| is a legacy version of |\childdocmain|:
%    \begin{macrocode}
\newcommand{\childdoc}{\childdocmain}
%    \end{macrocode}

% \macro{\childdocredirect}
% The deprecated macro |\childdocredirect| is a legacy version
% of |\childdocforward| and |\childdocforwardprefix|:
%    \begin{macrocode}
\newcommand{\childdocredirect}[2][]
{
  \begingroup
    \if?#1?
      \def\childdoctmp{\childdocforward{#2}}
    \else
      \def\childdoctmp{\childdocforwardprefix{#1}{#2}}
    \fi
    \expandafter
  \endgroup
  \childdoctmp
}
%    \end{macrocode}

%\iffalse
%</package>
%\fi
%
\endinput
|\\
|\childdocof{|\textit{main}|}|\\
\end{tabular}
\end{center}
at the top of every child file \textit{child}
which is included by |\include{|\textit{child}|}|
from within the main file
(or at least for those files to be compiled individually).
The argument \textit{main} must be the filename of the main file.

There are a couple of
considerations in setting up the main and child documents:

%%%%%%%%%%%%%%%%%%%%%%%%%%%%%%%%%%%%%%%%
\paragraph{Restrictions.}

Please note the following restrictions:
\begin{itemize}
\item
|\childdocmain| must be called with one argument \textit{main}
to ensure compatibility with earlier version of the package.
It must either be empty (|\childdocmain{}|)
or precisely match the filename of the main file in which it is specified.
See \secref{sec:detection} for further information.
\item
The filename \textit{main} must be specified without the |.tex| extension.
\item
The filename \textit{main} is case sensitive
(even in case-insensitive file systems)
due to internal string comparison.
\item
The argument \textit{main} should be fully expanded, it cannot be a macro.
\item
Subdirectories and special characters should be avoided in filenames.
\item
The command |\childdocmain{|\textit{main}|}| must be followed by a whitespace.
It should not be followed immediately by another command
or by a comment mark `|%|'.
This is because the \TeX{} parser reads the token immediately following
the argument of |\childdocmain| and puts it
at the beginning of every child section;
however, a white\-space is ignored.
\end{itemize}

%%%%%%%%%%%%%%%%%%%%%%%%%%%%%%%%%%%%%%%%
\paragraph{Content of Main File.}

It is advisable to place all content in the child files included by |\include|.
Any output contained in the main file will appear in all child documents
unless suppressed manually;
it cannot be suppressed automatically by the |\includeonly| directive
and thus should normally be avoided.
A method to include some content in the main file
by means of conditional processing is described in \secref{sec:conditional}.

%%%%%%%%%%%%%%%%%%%%%%%%%%%%%%%%%%%%%%%%
\paragraph{Page Numbering.}

When only a part of the document is compiled,
the appropriate numbering of pages
(as well as other status parameters)
is determined from the |.aux| files.
The latter contain information from previous passes.
However this information needs to propagate through
all intermediate child documents.
Therefore the page numbering in child documents may well
be inconsistent until the complete document is compiled at least once.

A useful (if unconventional) way to always ensure a consistent
page numbering is to restart the numbering in each child document
and denote the pages by `\textit{child}|.|\textit{page}'
where \textit{child} represents the chapter/section number of the child file.
This can be achieved by the command
|\numberwithin{page}{|\textit{child}|}|
of the \textsf{amsmath} package
where \textit{child} can be |chapter| or |section|
depending on the chosen structuring.
Alternatively, one can modify the macro |\thepage| appropriately
and reset the counter |page| at the start of each child file.

%%%%%%%%%%%%%%%%%%%%%%%%%%%%%%%%%%%%%%%%%%%%%%%%%%%%%%%%%%%%%%%%%%%%%%%%%%%%%%%%
\subsection{Conditional Processing}
\label{sec:conditional}

The package provides a mechanism to compile different versions
of a document. To customise the versions further some conditional processing
can come in handy to distinguish which version is being compiled.
The package provides two macros to describe the compilation context:

%%%%%%%%%%%%%%%%%%%%%%%%%%%%%%%%%%%%%%%%
\DescribeMacro{\ifchilddoc}
The conditional |\ifchilddoc| distinguishes between the compilation of
child documents and the main document:
%
\begin{center}
|\ifchilddoc |\textit{child-code}| |[|\||else |\textit{main-code}]| \||fi|
\end{center}

%%%%%%%%%%%%%%%%%%%%%%%%%%%%%%%%%%%%%%%%
\DescribeMacro{\childdocname}
\DescribeMacro{\childdocjob}
The macro |\childdocname| contains the filename (without extension)
of the main or child file being processed.
Note that |\childdocjob| will always contain the name of the main file.

%%%%%%%%%%%%%%%%%%%%%%%%%%%%%%%%%%%%%%%%
\paragraph{Title Page.}

Conditional processing can be used to include a title or banner page
in the main document when proper precautions are taken.
Importantly, the code in the main file should ensure that the page counter
(as well as other status parameters which are stored in the |.aux| files)
takes the same value after the conditional processing.
Otherwise the page numbers may take divergent values
depending on which part is compiled.

For example, a title page could be declared by:
%
\begin{center}
\begin{tabular}{l}
|\ifchilddoc\||else|\\
|\addtocounter{page}{-1}|\\
\textit{code for title page}\\
|\newpage|\\
|\||fi|
\end{tabular}
\end{center}
%
A banner page for the child documents can be generated by:
%
\begin{center}
\begin{tabular}{l}
|\ifchilddoc|\\
|\addtocounter{page}{-1}|\\
\textit{code for banner page}\\
|\newpage|\\
|\||fi|
\end{tabular}
\end{center}
%
Here one could write a message such as:
\begin{center}
|This is the part \childdocname{} of \childdocjob{}.|
\end{center}

%%%%%%%%%%%%%%%%%%%%%%%%%%%%%%%%%%%%%%%%%%%%%%%%%%%%%%%%%%%%%%%%%%%%%%%%%%%%%%%%
\subsection{Flags}
\label{sec:flags}

The package makes it easy to generate different versions
of the main or child documents.
To this end compilation flags can be defined
and assigned different default values.
They will be particularly useful in conjunction
with the forwarding mechanism described in \secref{sec:forward}.

For example, it may be useful to have a flag |\version|
which can be set to |draft| or |final|.
The document source will contain some conditional code
depending on the value of |\version|.
Suppose further, the flag should default to |final| for the main file
and to |draft| for child files
which is a natural assignment for editing the document.
This is achieved by placing the following code
in the preamble of the main document
(below the |\childdocmain| directive):
%
\begin{center}
\begin{tabular}{l}
|\ifchilddoc|\\
|\providecommand{\version}{draft}|\\
|\||else|\\
|\providecommand{\version}{final}|\\
|\||fi|
\end{tabular}
\end{center}
%
The definition by |\providecommand| makes sure
that previous definitions are not overwritten.
Further statements |\providecommand{\version}{...}|
can thus be added before the above code to override it.

For the main file, one might add a line
(between |\childdocmain| and the above block)
%
\begin{center}
|%\ifchilddoc\||else\providecommand{\version}{draft}\||fi|
\end{center}
%
which can be uncommented to produce a draft version.
Likewise one can add a line to the very top of a child file
(above the |\childdocof{|\textit{main}|}| directive)
%
\begin{center}
|%\providecommand{\version}{final}|
\end{center}
%
which can be uncommented to produce the final version of this child document.

%%%%%%%%%%%%%%%%%%%%%%%%%%%%%%%%%%%%%%%%%%%%%%%%%%%%%%%%%%%%%%%%%%%%%%%%%%%%%%%%
\subsection{Forwarding}
\label{sec:forward}

Different versions of the main or child documents
using compilation flags as described in \secref{sec:flags}
can be (permanently) stored in different files
for convenient compilation, viewing and distribution.
To this end, the package defines a command
to pass on compilation to a different file:

%%%%%%%%%%%%%%%%%%%%%%%%%%%%%%%%%%%%%%%%
\DescribeMacro{\childdocforward}
The command |\childdocforward| redirects processing to
another source file:
%
\begin{center}
\begin{tabular}{l}
|% \iffalse
%
% childdoc.dtx Copyright (C) 2017-2018 Niklas Beisert
%
% This work may be distributed and/or modified under the
% conditions of the LaTeX Project Public License, either version 1.3
% of this license or (at your option) any later version.
% The latest version of this license is in
%   http://www.latex-project.org/lppl.txt
% and version 1.3 or later is part of all distributions of LaTeX
% version 2005/12/01 or later.
%
% This work has the LPPL maintenance status `maintained'.
%
% The Current Maintainer of this work is Niklas Beisert.
%
% This work consists of the files childdoc.dtx and childdoc.ins
% and the derived files childdoc.def and cdocsamp.tex with
% cdocsch1.tex, cdocsch2.tex, cdocsdrf.tex, cdocsfn1.tex, cdocsfn2.tex.
%
%<package>\ifdefined\childdocmain\endinput\fi
%<package>\ProvidesFile{childdoc.def}[2018/12/30 v2.0 child document driver]
%<samplemain>\ProvidesFile{cdocsamp.tex}[2018/12/30 v2.0 sample for childdoc]
%<*driver>
%\ProvidesFile{childdoc.drv}[2018/12/30 v2.0 childdoc reference manual file]
\PassOptionsToClass{10pt,a4paper}{article}
\documentclass{ltxdoc}

\usepackage[margin=35mm]{geometry}
\usepackage{hyperref}
\usepackage{hyperxmp}
\usepackage[usenames]{color}

\hypersetup{colorlinks=true}
\hypersetup{pdfstartview=FitH}
\hypersetup{pdfpagemode=UseNone}
\hypersetup{pdfsource={}}
\hypersetup{pdflang={en-UK}}
\hypersetup{pdfcopyright={Copyright 2017-2018 Niklas Beisert.
  This work may be distributed and/or modified under the
  conditions of the LaTeX Project Public License, either version 1.3
  of this license or (at your option) any later version.}}
\hypersetup{pdflicenseurl={http://www.latex-project.org/lppl.txt}}
\hypersetup{pdfcontactaddress={ETH Zurich, ITP, HIT K,
  Wolfgang-Pauli-Strasse 27}}
\hypersetup{pdfcontactpostcode={8093}}
\hypersetup{pdfcontactcity={Zurich}}
\hypersetup{pdfcontactcountry={Switzerland}}
\hypersetup{pdfcontactemail={nbeisert@itp.phys.ethz.ch}}
\hypersetup{pdfcontacturl={http://people.phys.ethz.ch/\xmptilde nbeisert/}}

\newcommand{\secref}[1]{\hyperref[#1]{section \ref*{#1}}}

\parskip1ex
\parindent0pt
\let\olditemize\itemize
\def\itemize{\olditemize\parskip0pt}

\begin{document}

\title{The \textsf{childdoc} Package}
\hypersetup{pdftitle={The childdoc Package}}
\author{Niklas Beisert\\[2ex]
  Institut f\"ur Theoretische Physik\\
  Eidgen\"ossische Technische Hochschule Z\"urich\\
  Wolfgang-Pauli-Strasse 27, 8093 Z\"urich, Switzerland\\[1ex]
  \href{mailto:nbeisert@itp.phys.ethz.ch}
  {\texttt{nbeisert@itp.phys.ethz.ch}}}
\hypersetup{pdfauthor={Niklas Beisert}}
\hypersetup{pdfsubject={Manual for the LaTeX2e Package childdoc}}
\date{30 December 2018, \textsf{v2.0}}
\maketitle

\begin{abstract}\noindent
\textsf{childdoc} is a \LaTeXe{} package
that enables the direct compilation
of document sections included by |\include|
to individual files.
\end{abstract}

\begingroup
\parskip0ex
\tableofcontents
\endgroup

%%%%%%%%%%%%%%%%%%%%%%%%%%%%%%%%%%%%%%%%%%%%%%%%%%%%%%%%%%%%%%%%%%%%%%%%%%%%%%%%
%%%%%%%%%%%%%%%%%%%%%%%%%%%%%%%%%%%%%%%%%%%%%%%%%%%%%%%%%%%%%%%%%%%%%%%%%%%%%%%%
\section{Introduction}

\LaTeX{} provides a mechanism to structure a large document (such as a book)
into a main file and several child files (containing the chapters)
using the |\include| command.
This mechanism is beneficial for documents
which span hundreds of pages in order to
make the source file(s) more manageable.
Moreover, compilation can be restricted to
selected child files by means of the |\includeonly| command.
The latter feature can be used to reduce the compilation time while editing
(this was significantly more useful in the earlier days of \LaTeX{})
or to generate a smaller document which is easier to navigate.
Another application of |\includeonly| is to generate
documents consisting of selected parts of the complete document.

However, there are a few drawbacks of the plain |\include| mechanism:
\begin{itemize}
\item
The child files cannot be compiled on their own,
they can only be compiled via the main file.
A naive editing environment
(such as a text editor with an option
to have the current file processed by \LaTeX)
may require one to switch to the main file before compiling;
attempting to compile the child file produces errors.
\item
The main file must be modified (each time)
to adjust the |\includeonly| command
to the present needs. This easily leaves the main file in a messy state.
\item
The generated document will always carry the filename
of the main document. This is inconvenient if
several child files are to be compiled and
to be kept for distribution.
\end{itemize}

The present package provides a simple interface
to make child files individually compilable by \LaTeX{}.
Compiling a child file then has the same effect as compiling
the main file with an |\includeonly| command
to select the appropriate child.
Moreover the generated document will carry the name of the child
rather than the main file.
This resolves all three above issues.

This feature is meant to make the editing of books,
thesis documents and lecture notes somewhat more convenient.
However, the package can also be used efficiently for
composing a series of documents (such as exercise sheets)
which are typically distributed individually.
It then assists the author in generating the individual documents
(potentially in different versions)
as well as a document containing the collected series.
Another application is in developing style files
or other kinds of included material
where compilation of the style file could redirect
to a sample or test file.

%%%%%%%%%%%%%%%%%%%%%%%%%%%%%%%%%%%%%%%%%%%%%%%%%%%%%%%%%%%%%%%%%%%%%%%%%%%%%%%%
%%%%%%%%%%%%%%%%%%%%%%%%%%%%%%%%%%%%%%%%%%%%%%%%%%%%%%%%%%%%%%%%%%%%%%%%%%%%%%%%
\section{Usage}

First of all, the package \textsf{childdoc} is \emph{not} a standard
\LaTeXe{} |.sty| style file! Therefore it needs to be invoked in
a non-standard way.

%%%%%%%%%%%%%%%%%%%%%%%%%%%%%%%%%%%%%%%%%%%%%%%%%%%%%%%%%%%%%%%%%%%%%%%%%%%%%%%%
\subsection{Included Files}
\label{sec:include}

%%%%%%%%%%%%%%%%%%%%%%%%%%%%%%%%%%%%%%%%
\DescribeMacro{\childdocmain}
To use the package, add the commands
\begin{center}
\begin{tabular}{l}
|\input{childdoc.def}|\\
|\childdocmain{}|\\
\end{tabular}
\end{center}
at the very top of the main \LaTeX{} file,
in particular \emph{before} the |\documentclass| statement!
The argument of |\childdocmain| should be left empty
(but it must be present).

%%%%%%%%%%%%%%%%%%%%%%%%%%%%%%%%%%%%%%%%
\DescribeMacro{\childdocof}
Furthermore, add the commands
\begin{center}
\begin{tabular}{l}
|\input{childdoc.def}|\\
|\childdocof{|\textit{main}|}|\\
\end{tabular}
\end{center}
at the top of every child file \textit{child}
which is included by |\include{|\textit{child}|}|
from within the main file
(or at least for those files to be compiled individually).
The argument \textit{main} must be the filename of the main file.

There are a couple of
considerations in setting up the main and child documents:

%%%%%%%%%%%%%%%%%%%%%%%%%%%%%%%%%%%%%%%%
\paragraph{Restrictions.}

Please note the following restrictions:
\begin{itemize}
\item
|\childdocmain| must be called with one argument \textit{main}
to ensure compatibility with earlier version of the package.
It must either be empty (|\childdocmain{}|)
or precisely match the filename of the main file in which it is specified.
See \secref{sec:detection} for further information.
\item
The filename \textit{main} must be specified without the |.tex| extension.
\item
The filename \textit{main} is case sensitive
(even in case-insensitive file systems)
due to internal string comparison.
\item
The argument \textit{main} should be fully expanded, it cannot be a macro.
\item
Subdirectories and special characters should be avoided in filenames.
\item
The command |\childdocmain{|\textit{main}|}| must be followed by a whitespace.
It should not be followed immediately by another command
or by a comment mark `|%|'.
This is because the \TeX{} parser reads the token immediately following
the argument of |\childdocmain| and puts it
at the beginning of every child section;
however, a white\-space is ignored.
\end{itemize}

%%%%%%%%%%%%%%%%%%%%%%%%%%%%%%%%%%%%%%%%
\paragraph{Content of Main File.}

It is advisable to place all content in the child files included by |\include|.
Any output contained in the main file will appear in all child documents
unless suppressed manually;
it cannot be suppressed automatically by the |\includeonly| directive
and thus should normally be avoided.
A method to include some content in the main file
by means of conditional processing is described in \secref{sec:conditional}.

%%%%%%%%%%%%%%%%%%%%%%%%%%%%%%%%%%%%%%%%
\paragraph{Page Numbering.}

When only a part of the document is compiled,
the appropriate numbering of pages
(as well as other status parameters)
is determined from the |.aux| files.
The latter contain information from previous passes.
However this information needs to propagate through
all intermediate child documents.
Therefore the page numbering in child documents may well
be inconsistent until the complete document is compiled at least once.

A useful (if unconventional) way to always ensure a consistent
page numbering is to restart the numbering in each child document
and denote the pages by `\textit{child}|.|\textit{page}'
where \textit{child} represents the chapter/section number of the child file.
This can be achieved by the command
|\numberwithin{page}{|\textit{child}|}|
of the \textsf{amsmath} package
where \textit{child} can be |chapter| or |section|
depending on the chosen structuring.
Alternatively, one can modify the macro |\thepage| appropriately
and reset the counter |page| at the start of each child file.

%%%%%%%%%%%%%%%%%%%%%%%%%%%%%%%%%%%%%%%%%%%%%%%%%%%%%%%%%%%%%%%%%%%%%%%%%%%%%%%%
\subsection{Conditional Processing}
\label{sec:conditional}

The package provides a mechanism to compile different versions
of a document. To customise the versions further some conditional processing
can come in handy to distinguish which version is being compiled.
The package provides two macros to describe the compilation context:

%%%%%%%%%%%%%%%%%%%%%%%%%%%%%%%%%%%%%%%%
\DescribeMacro{\ifchilddoc}
The conditional |\ifchilddoc| distinguishes between the compilation of
child documents and the main document:
%
\begin{center}
|\ifchilddoc |\textit{child-code}| |[|\||else |\textit{main-code}]| \||fi|
\end{center}

%%%%%%%%%%%%%%%%%%%%%%%%%%%%%%%%%%%%%%%%
\DescribeMacro{\childdocname}
\DescribeMacro{\childdocjob}
The macro |\childdocname| contains the filename (without extension)
of the main or child file being processed.
Note that |\childdocjob| will always contain the name of the main file.

%%%%%%%%%%%%%%%%%%%%%%%%%%%%%%%%%%%%%%%%
\paragraph{Title Page.}

Conditional processing can be used to include a title or banner page
in the main document when proper precautions are taken.
Importantly, the code in the main file should ensure that the page counter
(as well as other status parameters which are stored in the |.aux| files)
takes the same value after the conditional processing.
Otherwise the page numbers may take divergent values
depending on which part is compiled.

For example, a title page could be declared by:
%
\begin{center}
\begin{tabular}{l}
|\ifchilddoc\||else|\\
|\addtocounter{page}{-1}|\\
\textit{code for title page}\\
|\newpage|\\
|\||fi|
\end{tabular}
\end{center}
%
A banner page for the child documents can be generated by:
%
\begin{center}
\begin{tabular}{l}
|\ifchilddoc|\\
|\addtocounter{page}{-1}|\\
\textit{code for banner page}\\
|\newpage|\\
|\||fi|
\end{tabular}
\end{center}
%
Here one could write a message such as:
\begin{center}
|This is the part \childdocname{} of \childdocjob{}.|
\end{center}

%%%%%%%%%%%%%%%%%%%%%%%%%%%%%%%%%%%%%%%%%%%%%%%%%%%%%%%%%%%%%%%%%%%%%%%%%%%%%%%%
\subsection{Flags}
\label{sec:flags}

The package makes it easy to generate different versions
of the main or child documents.
To this end compilation flags can be defined
and assigned different default values.
They will be particularly useful in conjunction
with the forwarding mechanism described in \secref{sec:forward}.

For example, it may be useful to have a flag |\version|
which can be set to |draft| or |final|.
The document source will contain some conditional code
depending on the value of |\version|.
Suppose further, the flag should default to |final| for the main file
and to |draft| for child files
which is a natural assignment for editing the document.
This is achieved by placing the following code
in the preamble of the main document
(below the |\childdocmain| directive):
%
\begin{center}
\begin{tabular}{l}
|\ifchilddoc|\\
|\providecommand{\version}{draft}|\\
|\||else|\\
|\providecommand{\version}{final}|\\
|\||fi|
\end{tabular}
\end{center}
%
The definition by |\providecommand| makes sure
that previous definitions are not overwritten.
Further statements |\providecommand{\version}{...}|
can thus be added before the above code to override it.

For the main file, one might add a line
(between |\childdocmain| and the above block)
%
\begin{center}
|%\ifchilddoc\||else\providecommand{\version}{draft}\||fi|
\end{center}
%
which can be uncommented to produce a draft version.
Likewise one can add a line to the very top of a child file
(above the |\childdocof{|\textit{main}|}| directive)
%
\begin{center}
|%\providecommand{\version}{final}|
\end{center}
%
which can be uncommented to produce the final version of this child document.

%%%%%%%%%%%%%%%%%%%%%%%%%%%%%%%%%%%%%%%%%%%%%%%%%%%%%%%%%%%%%%%%%%%%%%%%%%%%%%%%
\subsection{Forwarding}
\label{sec:forward}

Different versions of the main or child documents
using compilation flags as described in \secref{sec:flags}
can be (permanently) stored in different files
for convenient compilation, viewing and distribution.
To this end, the package defines a command
to pass on compilation to a different file:

%%%%%%%%%%%%%%%%%%%%%%%%%%%%%%%%%%%%%%%%
\DescribeMacro{\childdocforward}
The command |\childdocforward| redirects processing to
another source file:
%
\begin{center}
\begin{tabular}{l}
|\input{childdoc.def}|\\
|\childdocforward[|\textit{main}|]{|\textit{dest}|}|\\
\end{tabular}
\end{center}
%
The argument \textit{dest} is the destination file
(without extension).
It should be the main file or one of the child files.
Note that further \textsf{childdoc} directives
such as |\childdocof| and |\childdocforward|
in the indicated file will be processed in this form.
The optional argument \textit{main}
passes on directly to the main file \textit{main}
while pretending to compile the child \textit{dest}.
This form behaves as if \textit{dest}
issues |\childdocof{|\textit{main}|}| right away,
and no further \textsf{childdoc} directives will be processed.

%%%%%%%%%%%%%%%%%%%%%%%%%%%%%%%%%%%%%%%%
\DescribeMacro{\...prefix}
In the alternative form |\childdocforwardprefix|,
%
\begin{center}
\begin{tabular}{l}
|\input{childdoc.def}|\\
|\childdocforwardprefix[|\textit{main}|]{|\textit{prefix}|}{|\textit{dest}|}|
\end{tabular}
\end{center}
%
the destination file is determined by a pattern
depending on the current file:
To make this work, the current file must be called
`{\textit{prefix}\hspace{0.2em}\textit{suffix}}'
with \textit{prefix} matching precisely the argument.
Processing is then passed on to the file
`{\textit{dest}\hspace{0.2em}\textit{suffix}}'.
Surely, the same effect is achieved by
directly specifying the
argument `{\textit{dest}\hspace{0.2em}\textit{suffix}}'
in the first form.
However, that requires to set up a different file
for each child. With the alternative form of the command
all these files can have exactly the same content
which simplifies setting them up and maintaining them.

For example, the following file |draft.tex|
with a compilation flag |\version| as described in \secref{sec:flags}
compiles the main document as a draft:
%
\begin{center}
\begin{tabular}{l}
|\def\version{draft}|\\
|\input{childdoc.def}|\\
|\childdocforward{|\textit{main}|}|
\end{tabular}
\end{center}
%
Likewise, the following files |final|\textit{nn}|.tex|
compile the final version of the child document
|child|\textit{nn}|.tex|:
%
\begin{center}
\begin{tabular}{l}
|\def\version{final}|\\
|\input{childdoc.def}|\\
|\childdocforwardprefix{final}{child}|
\end{tabular}
\end{center}
%

Note that when several versions of a main file and/or of each child file
are to be generated, it may be convenient to set up a |Makefile| or
shell script to automatise the process.

%%%%%%%%%%%%%%%%%%%%%%%%%%%%%%%%%%%%%%%%%%%%%%%%%%%%%%%%%%%%%%%%%%%%%%%%%%%%%%%%
\subsection{Command Line Processing}
\label{sec:commandline}

The effect of redirection files can also be achieved by invoking
the \LaTeX{} compiler with a more elaborate command line.
Most conveniently this should be done as part
of a shell script or a |Makefile|.

When using \textsf{childdoc} in the main file, the following
command lines effectively perform a redirection
(note that depending on the shell being used,
backslashes may have to be doubled: `|\|' $\to$ `|\\|'):
%
\begin{center}
|... -jobname "|\textit{target}|" |\\|"|[\textit{flags}]%
|\input{childdoc.def}\childdocforward[|\textit{main}|]{|\textit{dest}|}"|
\end{center}
%
Here \textit{target} is the name of the output file,
\textit{main} is the name of the main file
and \textit{dest} is the name of the main or child file to be processed
(all filenames without extensions).
The optional argument \textit{main} can be omitted
if \textit{main} matches \textit{dest}.
Optionally, compilation \textit{flags} can be defined via |\def| commands.
This command line makes the \TeX{} engine believe
it is compiling the file \textit{target}
whose content is specified as the latter parameter.
The provided code then forwards the processing to
\textit{main} or \textit{dest} as described in \secref{sec:forward}.

%%%%%%%%%%%%%%%%%%%%%%%%%%%%%%%%%%%%%%%%%%%%%%%%%%%%%%%%%%%%%%%%%%%%%%%%%%%%%%%%
\subsection{Include by Input}
\label{sec:input}

Including child documents by |\include| has some restrictions by design.
Most notably, the content of a child document always occupies
its own set of pages; pages cannot be shared between child documents.
Usually, this behaviour makes perfect sense
because each child document contain an essential part of the document.
However, in some situations it may be desirable to compose
a document from a collection of parts
without having mandatory page breaks between then.
For this case, the package
provides a mechanism to include parts
by |\input| which can also be processed individually.
However, by construction this mechanism
requires manual handling of the content to be output.

%%%%%%%%%%%%%%%%%%%%%%%%%%%%%%%%%%%%%%%%
\DescribeMacro{\ifchilddocmanual}
The main file should be prepared as usual, see \secref{sec:include}.
However, the document body must make a distinction
between processing of an individual part and of the main document, e.g.:
%
\begin{center}
\begin{tabular}{l}
|\ifchilddocmanual|\\
|\input{\childdocname}|\\
|\||else|\\
\textit{document body with }|\input{|\textit{part}|}|\\
|\||fi|
\end{tabular}
\end{center}
%
The conditional |\ifchilddocmanual| is true whenever
a part to be included by |\input| is being compiled,
and the name of the part is stored in |\childdocname|.

%%%%%%%%%%%%%%%%%%%%%%%%%%%%%%%%%%%%%%%%
\DescribeMacro{\childdocby}
Each part to be included by |\input| should start with:
%
\begin{center}
\begin{tabular}{l}
|\input{childdoc.def}|\\
|\childdocby{|\textit{main}|}|\\
\end{tabular}
\end{center}
%
The directive |\childdocby| is similar to |\childdocof|
described in \secref{sec:include},
but the subsequent selection of content must be done manually.
To that end, both |\ifchilddoc| and |\ifchilddocmanual|
will be true upon processing of a part,
and the name of the part is stored in |\childdocname|.
Note that |\jobname| will be set to the filename of the current part
so that each part receives an individual |.aux| file
that does not interfere with the |.aux| file(s) of the main document.
This behaviour can be altered by the alternative form
|\childdocby[*]{|\textit{main}|}| (with a non-empty optional argument)
which uses the |.aux| file of the main document
by setting |\jobname| to \textit{main}.

%%%%%%%%%%%%%%%%%%%%%%%%%%%%%%%%%%%%%%%%%%%%%%%%%%%%%%%%%%%%%%%%%%%%%%%%%%%%%%%%
\subsection{Driver Development}
\label{sec:driver}

The \textsf{childdoc} mechanism can also be use for the development
of definition files such as \LaTeX{} styles or classes.
This case differs from the above setup with multiple parts
included by |\include| in that no |\includeonly| should be invoked.
This can be achieved by starting the include file
(before |\ProvidesPackage|) with:
%
\begin{center}
\begin{tabular}{l}
|\input{childdoc.def}|\\
|\childdocforward{|\textit{main}|}|\\
\end{tabular}
\end{center}
%
or alternatively with:
%
\begin{center}
\begin{tabular}{l}
|\input{childdoc.def}|\\
|\childdocby{|\textit{main}|}|\\
\end{tabular}
\end{center}
%
Both forms have slightly different effects as described above.
The main file is prepared as usual, see \secref{sec:include}.

%%%%%%%%%%%%%%%%%%%%%%%%%%%%%%%%%%%%%%%%%%%%%%%%%%%%%%%%%%%%%%%%%%%%%%%%%%%%%%%%
\subsection{Legacy Detection}
\label{sec:detection}

The directive |\childdocmain| in the main file can detect
whether the complete document or merely a child is to be compiled
even without using the directive |\childdocof|.
This method is deprecated because it is less robust
and there is no compelling reason to use it;
it is merely provided for backward compatibility
and it may be removed in future versions.

If the detection mechanism is to be used,
it is mandatory to correctly specify
the filename of the main file as the argument of |\childdocmain|:
%
\begin{center}
\begin{tabular}{l}
|\input{childdoc.def}|\\
|\childdocmain{|\textit{main}|}|\\
\end{tabular}
\end{center}
%
If |\jobname| does not match the argument \textit{main} of |\childdocmain|,
it is assumed that |\jobname| points to the child file to be compiled.
When using |\childdocmain| with the main file specified as argument,
it suffices to start a child file
with just |\input{|\textit{main}|}|
without loading of the package and using |\childdocof|.
If instead all processing is done
with the appropriate \textsf{childdoc} directives,
the argument of \textit{main} of |\childdocmain| can be empty.

An alternative version of the command line processing described
in \secref{sec:commandline} using the detection mechanism reads:
%
\begin{center}
|... -jobname "|\textit{target}|" "|[\textit{flags}]%
[|\def\jobname{|\textit{dest}|}|]|\input{|\textit{main}|}"|
\end{center}

%%%%%%%%%%%%%%%%%%%%%%%%%%%%%%%%%%%%%%%%%%%%%%%%%%%%%%%%%%%%%%%%%%%%%%%%%%%%%%%%
\subsection{Manual Code}
\label{sec:manual}

In case one cannot be certain whether the definitions file |childdoc.def|
is installed on the target \TeX{} distribution
and one prefers not to ship it,
it is conceivable to paste a few relevant commands into the sources.

To that end, drop all statements |\input{childdoc.def}|
and perform the replacements as outlined below.
Instead of |\childdocmain{|\textit{main}|}| add the following code
to the top of the main file:
%
\begin{center}
\begin{tabular}{l}
|\||ifdefined\childdocname\endinput\||fi\newif\ifchilddoc|\\
|\edef\childdocname{\scantokens\expandafter{\jobname\noexpand}}|\\
|\def\childdocmain{|\textit{main}|}\||ifx\childdocmain\childdocname\||else|\\
|\childdoctrue\includeonly{\childdocname}\let\jobname\childdocmain\||fi|\\
\end{tabular}
\end{center}
%
Instead of |\childdocof{|\textit{main}|}| just include the main file
at the top of each child file:
%
\begin{center}
|\input{|\textit{main}|}|
\end{center}
%
A simple redirection |\childdocforward{|\textit{dest}|}| is achieved by:
%
\begin{center}
|\def\jobname{|\textit{dest}|}\input{\jobname}|
\end{center}
%
The redirection with prefix
|\childdocforwardprefix[|\textit{prefix}|]{|\textit{dest}|}|
is accomplished by:
%
\begin{center}
\begin{tabular}{l}
|{\edef\jobname{\scantokens\expandafter{\jobname\noexpand}}|\\
|\def\redirectjob |\textit{prefix}|#1~~~{\gdef\jobname{|\textit{dest}|#1}}|\\
|\expandafter\redirectjob\jobname~~~}\input{\jobname}|
\end{tabular}
\end{center}

In an alternative approach,
child documents can be compiled by a specific command line
without additional code or specific definitions:
%
\begin{center}
|... -jobname "|\textit{target}|" "|[\textit{flags}]%
|\includeonly{|\textit{dest}|}\input{|\textit{main}|}"|
\end{center}
%

%%%%%%%%%%%%%%%%%%%%%%%%%%%%%%%%%%%%%%%%%%%%%%%%%%%%%%%%%%%%%%%%%%%%%%%%%%%%%%%%
%%%%%%%%%%%%%%%%%%%%%%%%%%%%%%%%%%%%%%%%%%%%%%%%%%%%%%%%%%%%%%%%%%%%%%%%%%%%%%%%
\section{Information}

%%%%%%%%%%%%%%%%%%%%%%%%%%%%%%%%%%%%%%%%%%%%%%%%%%%%%%%%%%%%%%%%%%%%%%%%%%%%%%%%
\subsection{Copyright}

Copyright \copyright{} 2017--2018 Niklas Beisert

This work may be distributed and/or modified under the
conditions of the \LaTeX{} Project Public License, either version 1.3
of this license or (at your option) any later version.
The latest version of this license is in
  \url{http://www.latex-project.org/lppl.txt}
and version 1.3 or later is part of all distributions of \LaTeX{}
version 2005/12/01 or later.

This work has the LPPL maintenance status `maintained'.

The Current Maintainer of this work is Niklas Beisert.

This work consists of the files |README.txt|, |childdoc.ins| and |childdoc.dtx|
as well as the derived files |childdoc.def|, |cdocsamp.tex|
with |cdocsch1.tex|, |cdocsch2.tex|, |cdocspt3.tex|, |cdocspt4.tex|,
|cdocsdrf.tex|, |cdocsfn1.tex|, |cdocsfn2.tex|
as well as |childdoc.pdf|.

%%%%%%%%%%%%%%%%%%%%%%%%%%%%%%%%%%%%%%%%%%%%%%%%%%%%%%%%%%%%%%%%%%%%%%%%%%%%%%%%
\subsection{Files and Installation}

The package consists of the files:
%
\begin{center}
\begin{tabular}{ll}
    |README.txt|   & readme file \\
    |childdoc.ins| & installation file \\
    |childdoc.dtx| & source file \\
    |childdoc.def| & definition file \\
    |cdocsamp.tex| & sample main file \\
    |cdocsch1.tex| & sample include file \\
    |cdocsch2.tex| & sample include file \\
    |cdocspt3.tex| & sample part file \\
    |cdocspt4.tex| & sample part file \\
    |cdocsdrf.tex| & sample redirection file \\
    |cdocsfn1.tex| & sample redirection file \\
    |cdocsfn2.tex| & sample redirection file \\
    |childdoc.pdf| & manual
\end{tabular}
\end{center}
%
The distribution consists of the files
|README.txt|, |childdoc.ins| and |childdoc.dtx|.
%
\begin{itemize}
\item
Run (pdf)\LaTeX{} on |childdoc.dtx|
to compile the manual |childdoc.pdf| (this file).
\item
Run \LaTeX{} on |childdoc.ins| to create the definitions file |childdoc.def|
and the sample |cdocsamp.tex| with include files
|cdocsch1.tex|, |cdocsch2.tex|, |cdocspt3.tex|, |cdocspt4.tex|,
|cdocsdrf.tex|, |cdocsfn1.tex|, |cdocsfn2.tex|.
Then copy the file |childdoc.def| to an appropriate directory of your \LaTeX{}
distribution, e.g.\ \textit{texmf-root}|/tex/latex/childdoc|.
\end{itemize}

%%%%%%%%%%%%%%%%%%%%%%%%%%%%%%%%%%%%%%%%%%%%%%%%%%%%%%%%%%%%%%%%%%%%%%%%%%%%%%%%
\subsection{Related CTAN Packages}

There are several other packages which offer a similar functionality:
%
\begin{itemize}
\item
The packages
\href{http://ctan.org/pkg/docmute}{\textsf{docmute}},
\href{http://ctan.org/pkg/includex}{\textsf{includex}} and
\href{http://ctan.org/pkg/standalone}{\textsf{standalone}}
provide commands to include only the document body of
a child file thus allowing both files to be compiled individually.
\item
The packages \href{http://ctan.org/pkg/subdocs}{\textsf{subdocs}}
and \href{http://ctan.org/pkg/subfiles}{\textsf{subfiles}}
provide structures in which the main and child documents can be
encapsulated and allowing them to be compiled individually.
The inclusion mechanism is different from the conventional |\include|.
\item
The package \href{http://ctan.org/pkg/combine}{\textsf{combine}}
is an elaborate solution to combine several documents into one.
\end{itemize}
%
See also the CTAN topic \href{http://ctan.org/topic/subdocs}{\textsf{subdocs}}
for further related packages.
The present package differs from the above solutions in that
a document structure constructed with the conventional |\include| mechanism
just needs two extra commands at the top of every file
such that all constituent files can be compiled individually.

%%%%%%%%%%%%%%%%%%%%%%%%%%%%%%%%%%%%%%%%%%%%%%%%%%%%%%%%%%%%%%%%%%%%%%%%%%%%%%%%
%\subsection{Feature Suggestions}
%
%The following is a list of features which may be useful for future
%versions of this package:
%%
%\begin{itemize}
%\item
%\ldots
%\end{itemize}

%%%%%%%%%%%%%%%%%%%%%%%%%%%%%%%%%%%%%%%%%%%%%%%%%%%%%%%%%%%%%%%%%%%%%%%%%%%%%%%%
\subsection{Revision History}

%%%%%%%%%%%%%%%%%%%%%%%%%%%%%%%%%%%%%%%%
\paragraph{v2.0:} 2018/12/30

\begin{itemize}
\item
immediate forward processing
\item
added |\childdocby| mechanism
\item
manual restructured
\end{itemize}

%%%%%%%%%%%%%%%%%%%%%%%%%%%%%%%%%%%%%%%%
\paragraph{v1.6:} 2018/01/17

\begin{itemize}
\item
application for development of include files
\item
corrections to manual
\end{itemize}

%%%%%%%%%%%%%%%%%%%%%%%%%%%%%%%%%%%%%%%%
\paragraph{v1.5:} 2017/05/21

\begin{itemize}
\item
more complete structuring introduced
\item
|\childdocof| introduced
\item
|\childdoc| renamed to |\childdocmain|
\item
|\childredirect| renamed to |\childdocforward| and |\childdocforwardprefix|
and functionality expanded
\end{itemize}

%%%%%%%%%%%%%%%%%%%%%%%%%%%%%%%%%%%%%%%%
\paragraph{v1.0:} 2017/04/27

\begin{itemize}
\item
manual and install package
\item
first version published on CTAN
\end{itemize}

%%%%%%%%%%%%%%%%%%%%%%%%%%%%%%%%%%%%%%%%
\paragraph{v0.6:} 2017/04/26

\begin{itemize}
\item
redirection mechanism added
\end{itemize}

%%%%%%%%%%%%%%%%%%%%%%%%%%%%%%%%%%%%%%%%
\paragraph{v0.5:} 2017/04/26

\begin{itemize}
\item
functionality in definition file
\end{itemize}


%%%%%%%%%%%%%%%%%%%%%%%%%%%%%%%%%%%%%%%%%%%%%%%%%%%%%%%%%%%%%%%%%%%%%%%%%%%%%%%%
%%%%%%%%%%%%%%%%%%%%%%%%%%%%%%%%%%%%%%%%%%%%%%%%%%%%%%%%%%%%%%%%%%%%%%%%%%%%%%%%
%%%%%%%%%%%%%%%%%%%%%%%%%%%%%%%%%%%%%%%%%%%%%%%%%%%%%%%%%%%%%%%%%%%%%%%%%%%%%%%%
\appendix

\settowidth\MacroIndent{\rmfamily\scriptsize 000\ }

 \DocInput{childdoc.dtx}

\end{document}
%</driver>
% \fi
%
% %%%%%%%%%%%%%%%%%%%%%%%%%%%%%%%%%%%%%%%%%%%%%%%%%%%%%%%%%%%%%%%%%%%%%%%%%%%%%%
% %%%%%%%%%%%%%%%%%%%%%%%%%%%%%%%%%%%%%%%%%%%%%%%%%%%%%%%%%%%%%%%%%%%%%%%%%%%%%%
% \section{Sample}
%\iffalse
%<*samplemain>
%\fi
%
% The following presents a sample document
% with two chapters, two parts, a title page,
% a compile flag as well as three forwarding files to set the flag.
% It consists of eight |.tex| files:
% \begin{center}
% \begin{tabular}{ll}
% |cdocsamp.tex|&main file\\
% |cdocsch1.tex|&include file for chapter 1\\
% |cdocsch2.tex|&include file for chapter 2\\
% |cdocspt3.tex|&include file for part 3\\
% |cdocspt4.tex|&include file for part 4\\
% |cdocsdrf.tex|&forwarding file for main file in draft mode\\
% |cdocsfi1.tex|&forwarding file for final version of chapter 1\\
% |cdocsfi2.tex|&forwarding file for final version of chapter 2\\
% \end{tabular}
% \end{center}
% Each of the eight files can be compiled directly by the \LaTeX{} compiler.
%
% %%%%%%%%%%%%%%%%%%%%%%%%%%%%%%%%%%%%%%
% \paragraph{Main File.}
%
% The main file is called |cdocsamp.tex|.
%
% Load the \textsf{childdoc} definitions and
% declare the filename for the main document:
%    \begin{macrocode}
\input{childdoc.def}
\childdocmain{}
%    \end{macrocode}

% Optional override for |\version| flag:
%    \begin{macrocode}
%%\ifchilddoc\else\providecommand{\version}{draft}\fi
%    \end{macrocode}

% Define the default values for the |\version| flag
% (|final| for the main file and |draft| for childs):
%    \begin{macrocode}
\ifchilddoc
\providecommand{\version}{draft}
\else
\providecommand{\version}{final}
\fi
%    \end{macrocode}

% Load the standard document class:
%    \begin{macrocode}
\documentclass[12pt]{article}
%    \end{macrocode}

% Start the document body:
%    \begin{macrocode}
\begin{document}
%    \end{macrocode}

% Declare a title page.
% Print title, part of document being processed and version flag:
%    \begin{macrocode}
\addtocounter{page}{-1}
\begin{center}
{\LARGE\bfseries{}childdoc example\par}
\vspace{1cm}
\ifchilddoc
\ifchilddocmanual part\else chapter\fi:
`\childdocname' of `\childdocjob'\par
\else
main document: `\childdocjob'\par
\fi
version: \version\par
\end{center}
\newpage
%    \end{macrocode}

% Manually include selected file,
% otherwise process as usual:
%    \begin{macrocode}
\ifchilddocmanual
\section*{part `\childdocname'}
\input{\childdocname}
\else
%    \end{macrocode}

% Include the two chapters:
%    \begin{macrocode}
\include{cdocsch1}
\include{cdocsch2}
%    \end{macrocode}

% Include the two parts unless only chapters should be displayed:
%    \begin{macrocode}
\ifchilddoc\else
\section{part three}
\input{cdocspt3}
\section{part four}
\input{cdocspt4}
\fi
%    \end{macrocode}

% Process as usual until here:
%    \begin{macrocode}
\fi
%    \end{macrocode}

% End of document body:
%    \begin{macrocode}
\end{document}
%    \end{macrocode}
%\iffalse
%</samplemain>
%\fi
%
% %%%%%%%%%%%%%%%%%%%%%%%%%%%%%%%%%%%%%%
% \paragraph{Chapter Include Files.}
%
% The include files are called |cdocsch1.tex| and |cdocsch2.tex|.
%
%\iffalse
%<*samplechap1|samplechap2>
%\fi

% Optional override for |\version| flag:
%    \begin{macrocode}
%%\providecommand{\version}{final}
%    \end{macrocode}

% Include the main document:
%    \begin{macrocode}
\input{childdoc.def}
\childdocof{cdocsamp}
%    \end{macrocode}

%\iffalse
%</samplechap1|samplechap2>
%\fi
%
%\iffalse
%<*samplechap1>
%\fi
% Some text for chapter 1:
%    \begin{macrocode}
\section{one}
some text in chapter one
%    \end{macrocode}

%\iffalse
%</samplechap1>
%\fi
% Some text for chapter 2:
%\iffalse
%<*samplechap2>
%\fi
%    \begin{macrocode}
\section{two}
more text in chapter two
%    \end{macrocode}

%\iffalse
%</samplechap2>
%\fi
%
% %%%%%%%%%%%%%%%%%%%%%%%%%%%%%%%%%%%%%%
% \paragraph{Part Include Files.}
%
% The include files are called |cdocspt3.tex| and |cdocspt4.tex|.
%
%\iffalse
%<*samplepart3|samplepart4>
%\fi

% Optional override for |\version| flag:
%    \begin{macrocode}
%%\providecommand{\version}{final}
%    \end{macrocode}

% Include the main document:
%    \begin{macrocode}
\input{childdoc.def}
\childdocby{cdocsamp}
%    \end{macrocode}

%\iffalse
%</samplepart3|samplepart4>
%\fi
%
%\iffalse
%<*samplepart3>
%\fi
% Some text for part 3:
%    \begin{macrocode}
some text in part three
%    \end{macrocode}

%\iffalse
%</samplepart3>
%\fi
% Some text for part 4:
%\iffalse
%<*samplepart4>
%\fi
%    \begin{macrocode}
more text in part four
%    \end{macrocode}

%\iffalse
%</samplepart4>
%\fi
%
% %%%%%%%%%%%%%%%%%%%%%%%%%%%%%%%%%%%%%%
% \paragraph{Forwarding for a Complete Draft.}
%
% The following forwarding file |cdocsdrf.tex|
% compiles the main document in draft mode:
%\iffalse
%<*sampledraft>
%\fi
%    \begin{macrocode}
\def\version{draft}
\input{childdoc.def}
\childdocforward{cdocsamp}
%    \end{macrocode}

%\iffalse
%</sampledraft>
%\fi
%
% %%%%%%%%%%%%%%%%%%%%%%%%%%%%%%%%%%%%%%
% \paragraph{Forwarding for Final Version of the Chapters.}
%
% The following forwarding files |cdocsfn1.tex| and |cdocsfn2.tex|
% (with identical content)
% compile the final versions of the child documents
% |cdocsch1.tex| and |cdocsch2.tex|, respectively:
%\iffalse
%<*samplefinal>
%\fi
%    \begin{macrocode}
\def\version{final}
\input{childdoc.def}
\childdocforwardprefix[cdocsamp]{cdocsfn}{cdocsch}
%    \end{macrocode}

%\iffalse
%</samplefinal>
%\fi
%
% %%%%%%%%%%%%%%%%%%%%%%%%%%%%%%%%%%%%%%
% \paragraph{Command Line Processing.}
%
% The following three command lines generate the output files
% |cdocscld|, |cdocscl1| and |cdocscl2|
% which should be identical to
% |cdocsdrf|, |cdocsch1| and |cdocsfn2|, respectively:
% \begin{center}
% \begin{tabular}{l}
% |latex -jobname cdocscld \|\\
% |  "\def\version{draft}\input{childdoc.def}\childdocforward{cdocsamp}"|\\
% |latex -jobname cdocscl1 \|\\
% |  "\input{childdoc.def}\childdocforward[cdocsamp]{cdocsch1}"|\\
% |latex -jobname cdocscl2 \|\\
% |  "\def\version{final}\input{childdoc.def}\childdocforward{cdocsch2}"|
% \end{tabular}
% \end{center}
% Note that the trailing backslash on each first line
% merely continues the input to the second line
% (for convenient cut ant paste).
% Furthermore, the command |latex| can be replaced by any
% of its alternative versions such as |pdflatex|.
%
% %%%%%%%%%%%%%%%%%%%%%%%%%%%%%%%%%%%%%%%%%%%%%%%%%%%%%%%%%%%%%%%%%%%%%%%%%%%%%%
% %%%%%%%%%%%%%%%%%%%%%%%%%%%%%%%%%%%%%%%%%%%%%%%%%%%%%%%%%%%%%%%%%%%%%%%%%%%%%%
% \section{Implementation}
%\iffalse
%<*package>
%\fi
%
% This section describes the definitions file |childdoc.def|.

% The definitions cannot be loaded using |\usepackage| or |\RequirePackage|
% which has a mechanism to prevent loading a style file more than once.
% When loading the definitions by means of |\input|
% multiple instances have to be prevented manually:
%\iffalse
%This code needs to be before the `\ProvidesFile' directive
%which is defined at the beginning of this file.
%Therefore it is also placed there and commented out here.
%</package>
%<*discard>
%\fi
%    \begin{macrocode}
\ifdefined\childdocmain\endinput\fi
%    \end{macrocode}
%\iffalse
%</discard>
%<*package>
%\fi
%
% \macro{\ifchilddoc}
% \macro{\ifchilddocmanual}
% The conditional |\ifchilddoc| tells whether a
% child (true) or main (false) document is being compiled.
% The conditional |\ifchilddocmanual| tells whether
% the |\includeonly| mechanism is used (false) or
% the selection of child files must be performed manually (true).
% The definitions initialise to false:
%    \begin{macrocode}
\newif\ifchilddoc
\newif\ifchilddocmanual
%    \end{macrocode}

% \macro{\childdocname}
% \macro{\childdocjob}
% The macro |\childdocname| stores the name of the main document
% to be compiled. The macro |\childdocjob| stores the name of
% the document on which the \LaTeX{} compiler was originally invoked.
% The content of |\jobname| cannot be compared
% to filenames specified in the source due to different catcodes.
% The following code rescans |\jobname|, stores the result
% in |\childdocname| and saves a copy in |\childdocjob|:
%    \begin{macrocode}
\edef\childdocname{\scantokens\expandafter{\jobname\noexpand}}
\let\childdocjob\childdocname
%    \end{macrocode}

% \macro{\childdocdisable}
% The macro |\childdocdisable| prevents the main file
% from being processed more than once.
% At this stage, the main document command |\childdocmain|
% is assumed to be called once again where it should do nothing.
% Any subsequent call to it should prevent
% a secondary processing of the main document
% It overwrites the forwarding commands
% |\childdocof| and |\childdocforward|
% with empty macros to prevent further inclusions of the main document:
%    \begin{macrocode}
\newcommand{\childdocdisable}
{
  \renewcommand{\childdocmain}[1]{\renewcommand{\childdocmain}[1]{\endinput}}
  \renewcommand{\childdocof}[1]{}
  \renewcommand{\childdocby}[2][]{}
  \renewcommand{\childdocforward}[2][]{}
  \renewcommand{\childdocdisable}{}
}
%    \end{macrocode}

% \macro{\childdocmain}
% The macro |\childdocmain| is to be called at the top of the main file
% with nothing or the main filename (without extension) as argument.
% First, it breaks loops.
% If the argument is not empty and does not match |\childdocname|
% (which is set by the first inclusion of |childdoc.def|),
% |\ifchilddoc| is set to true, |\includeonly| is applied to the child file
% and |\jobname| is set to the main file
% (for proper handling of |.aux| files):
%    \begin{macrocode}
\newcommand{\childdocmain}[1]
{
  \childdocdisable\childdocmain{}
  \if?#1?\else
    \begingroup
      \def\childdoctmp{#1}
      \ifx\childdoctmp\childdocname
        \def\childdoctmp{}
      \else
        \def\childdoctmp
        {
          \childdoctrue
          \includeonly{\childdocname}
          \def\childdocjob{#1}
          \def\jobname{#1}
        }
      \fi
      \expandafter
    \endgroup
    \childdoctmp
  \fi
}
%    \end{macrocode}

% \macro{\childdocof}
% The command |\childdocof| redirects
% compilation to the main file |#1|.
%    \begin{macrocode}
\newcommand{\childdocof}[1]
{
  \childdocdisable
  \childdoctrue
  \includeonly{\childdocname}
  \def\jobname{#1}
  \def\childdocjob{#1}
  \input{#1}
}
%    \end{macrocode}

% \macro{\childdocby}
% The command |\childdocby| ....
%    \begin{macrocode}
\newcommand{\childdocby}[2][]
{
  \childdocdisable
  \childdoctrue
  \childdocmanualtrue
  \if?#1?\else
    \def\jobname{#2}
  \fi
  \def\childdocjob{#2}
  \input{#2}
  \endinput
}
%    \end{macrocode}

% \macro{\childdocforward}
% The command |\childdocforward| redirects
% compilation to the main file or
% (if the optional argument is given) a child file.
% Parameters are set as if the main file
% or a child file starting with |\childdocof| was compiled.
% Then compilation is handed over to the main file:
%    \begin{macrocode}
\newcommand{\childdocforward}[2][]
{
  \begingroup
    \if?#1?
      \def\childdoctmp
      {
        \def\childdocname{#2}
        \def\childdocjob{#2}
        \def\jobname{#2}
        \input{#2}
        \endinput
      }
    \else
      \def\childdoctmp
      {
        \childdocdisable
        \def\childdocname{#2}
        \childdoctrue
        \includeonly{#2}
        \def\childdocjob{#1}
        \def\jobname{#1}
        \input{#1}
        \endinput
      }
    \fi
    \expandafter
  \endgroup
  \childdoctmp
}
%    \end{macrocode}

% \macro{\childdocforwardprefix}
% The command |\childdocforwardprefix| redirects
% compilation to the main or a child file by means of a pattern.
% The prefix |#1| in the current filename is replaced by |#2|
% and the suffix of the current filename is kept
% (it is assumed that the filename does not contain the substring `|~~~|'
% which is used as a delimiter).
% Compilation is handed over to the new file by |\childdocforward|:
%    \begin{macrocode}
\newcommand{\childdocforwardprefix}[3][]
{
  \begingroup
    \def\childdocextract #2##1~~~{\def\childdoctmp{\childdocforward[#1]{#3##1}}}
    \expandafter\childdocextract\childdocname~~~
    \expandafter
  \endgroup
  \childdoctmp
}
%    \end{macrocode}

% \macro{\childdoc}
% The deprecated macro |\childdoc| is a legacy version of |\childdocmain|:
%    \begin{macrocode}
\newcommand{\childdoc}{\childdocmain}
%    \end{macrocode}

% \macro{\childdocredirect}
% The deprecated macro |\childdocredirect| is a legacy version
% of |\childdocforward| and |\childdocforwardprefix|:
%    \begin{macrocode}
\newcommand{\childdocredirect}[2][]
{
  \begingroup
    \if?#1?
      \def\childdoctmp{\childdocforward{#2}}
    \else
      \def\childdoctmp{\childdocforwardprefix{#1}{#2}}
    \fi
    \expandafter
  \endgroup
  \childdoctmp
}
%    \end{macrocode}

%\iffalse
%</package>
%\fi
%
\endinput
|\\
|\childdocforward[|\textit{main}|]{|\textit{dest}|}|\\
\end{tabular}
\end{center}
%
The argument \textit{dest} is the destination file
(without extension).
It should be the main file or one of the child files.
Note that further \textsf{childdoc} directives
such as |\childdocof| and |\childdocforward|
in the indicated file will be processed in this form.
The optional argument \textit{main}
passes on directly to the main file \textit{main}
while pretending to compile the child \textit{dest}.
This form behaves as if \textit{dest}
issues |\childdocof{|\textit{main}|}| right away,
and no further \textsf{childdoc} directives will be processed.

%%%%%%%%%%%%%%%%%%%%%%%%%%%%%%%%%%%%%%%%
\DescribeMacro{\...prefix}
In the alternative form |\childdocforwardprefix|,
%
\begin{center}
\begin{tabular}{l}
|% \iffalse
%
% childdoc.dtx Copyright (C) 2017-2018 Niklas Beisert
%
% This work may be distributed and/or modified under the
% conditions of the LaTeX Project Public License, either version 1.3
% of this license or (at your option) any later version.
% The latest version of this license is in
%   http://www.latex-project.org/lppl.txt
% and version 1.3 or later is part of all distributions of LaTeX
% version 2005/12/01 or later.
%
% This work has the LPPL maintenance status `maintained'.
%
% The Current Maintainer of this work is Niklas Beisert.
%
% This work consists of the files childdoc.dtx and childdoc.ins
% and the derived files childdoc.def and cdocsamp.tex with
% cdocsch1.tex, cdocsch2.tex, cdocsdrf.tex, cdocsfn1.tex, cdocsfn2.tex.
%
%<package>\ifdefined\childdocmain\endinput\fi
%<package>\ProvidesFile{childdoc.def}[2018/12/30 v2.0 child document driver]
%<samplemain>\ProvidesFile{cdocsamp.tex}[2018/12/30 v2.0 sample for childdoc]
%<*driver>
%\ProvidesFile{childdoc.drv}[2018/12/30 v2.0 childdoc reference manual file]
\PassOptionsToClass{10pt,a4paper}{article}
\documentclass{ltxdoc}

\usepackage[margin=35mm]{geometry}
\usepackage{hyperref}
\usepackage{hyperxmp}
\usepackage[usenames]{color}

\hypersetup{colorlinks=true}
\hypersetup{pdfstartview=FitH}
\hypersetup{pdfpagemode=UseNone}
\hypersetup{pdfsource={}}
\hypersetup{pdflang={en-UK}}
\hypersetup{pdfcopyright={Copyright 2017-2018 Niklas Beisert.
  This work may be distributed and/or modified under the
  conditions of the LaTeX Project Public License, either version 1.3
  of this license or (at your option) any later version.}}
\hypersetup{pdflicenseurl={http://www.latex-project.org/lppl.txt}}
\hypersetup{pdfcontactaddress={ETH Zurich, ITP, HIT K,
  Wolfgang-Pauli-Strasse 27}}
\hypersetup{pdfcontactpostcode={8093}}
\hypersetup{pdfcontactcity={Zurich}}
\hypersetup{pdfcontactcountry={Switzerland}}
\hypersetup{pdfcontactemail={nbeisert@itp.phys.ethz.ch}}
\hypersetup{pdfcontacturl={http://people.phys.ethz.ch/\xmptilde nbeisert/}}

\newcommand{\secref}[1]{\hyperref[#1]{section \ref*{#1}}}

\parskip1ex
\parindent0pt
\let\olditemize\itemize
\def\itemize{\olditemize\parskip0pt}

\begin{document}

\title{The \textsf{childdoc} Package}
\hypersetup{pdftitle={The childdoc Package}}
\author{Niklas Beisert\\[2ex]
  Institut f\"ur Theoretische Physik\\
  Eidgen\"ossische Technische Hochschule Z\"urich\\
  Wolfgang-Pauli-Strasse 27, 8093 Z\"urich, Switzerland\\[1ex]
  \href{mailto:nbeisert@itp.phys.ethz.ch}
  {\texttt{nbeisert@itp.phys.ethz.ch}}}
\hypersetup{pdfauthor={Niklas Beisert}}
\hypersetup{pdfsubject={Manual for the LaTeX2e Package childdoc}}
\date{30 December 2018, \textsf{v2.0}}
\maketitle

\begin{abstract}\noindent
\textsf{childdoc} is a \LaTeXe{} package
that enables the direct compilation
of document sections included by |\include|
to individual files.
\end{abstract}

\begingroup
\parskip0ex
\tableofcontents
\endgroup

%%%%%%%%%%%%%%%%%%%%%%%%%%%%%%%%%%%%%%%%%%%%%%%%%%%%%%%%%%%%%%%%%%%%%%%%%%%%%%%%
%%%%%%%%%%%%%%%%%%%%%%%%%%%%%%%%%%%%%%%%%%%%%%%%%%%%%%%%%%%%%%%%%%%%%%%%%%%%%%%%
\section{Introduction}

\LaTeX{} provides a mechanism to structure a large document (such as a book)
into a main file and several child files (containing the chapters)
using the |\include| command.
This mechanism is beneficial for documents
which span hundreds of pages in order to
make the source file(s) more manageable.
Moreover, compilation can be restricted to
selected child files by means of the |\includeonly| command.
The latter feature can be used to reduce the compilation time while editing
(this was significantly more useful in the earlier days of \LaTeX{})
or to generate a smaller document which is easier to navigate.
Another application of |\includeonly| is to generate
documents consisting of selected parts of the complete document.

However, there are a few drawbacks of the plain |\include| mechanism:
\begin{itemize}
\item
The child files cannot be compiled on their own,
they can only be compiled via the main file.
A naive editing environment
(such as a text editor with an option
to have the current file processed by \LaTeX)
may require one to switch to the main file before compiling;
attempting to compile the child file produces errors.
\item
The main file must be modified (each time)
to adjust the |\includeonly| command
to the present needs. This easily leaves the main file in a messy state.
\item
The generated document will always carry the filename
of the main document. This is inconvenient if
several child files are to be compiled and
to be kept for distribution.
\end{itemize}

The present package provides a simple interface
to make child files individually compilable by \LaTeX{}.
Compiling a child file then has the same effect as compiling
the main file with an |\includeonly| command
to select the appropriate child.
Moreover the generated document will carry the name of the child
rather than the main file.
This resolves all three above issues.

This feature is meant to make the editing of books,
thesis documents and lecture notes somewhat more convenient.
However, the package can also be used efficiently for
composing a series of documents (such as exercise sheets)
which are typically distributed individually.
It then assists the author in generating the individual documents
(potentially in different versions)
as well as a document containing the collected series.
Another application is in developing style files
or other kinds of included material
where compilation of the style file could redirect
to a sample or test file.

%%%%%%%%%%%%%%%%%%%%%%%%%%%%%%%%%%%%%%%%%%%%%%%%%%%%%%%%%%%%%%%%%%%%%%%%%%%%%%%%
%%%%%%%%%%%%%%%%%%%%%%%%%%%%%%%%%%%%%%%%%%%%%%%%%%%%%%%%%%%%%%%%%%%%%%%%%%%%%%%%
\section{Usage}

First of all, the package \textsf{childdoc} is \emph{not} a standard
\LaTeXe{} |.sty| style file! Therefore it needs to be invoked in
a non-standard way.

%%%%%%%%%%%%%%%%%%%%%%%%%%%%%%%%%%%%%%%%%%%%%%%%%%%%%%%%%%%%%%%%%%%%%%%%%%%%%%%%
\subsection{Included Files}
\label{sec:include}

%%%%%%%%%%%%%%%%%%%%%%%%%%%%%%%%%%%%%%%%
\DescribeMacro{\childdocmain}
To use the package, add the commands
\begin{center}
\begin{tabular}{l}
|\input{childdoc.def}|\\
|\childdocmain{}|\\
\end{tabular}
\end{center}
at the very top of the main \LaTeX{} file,
in particular \emph{before} the |\documentclass| statement!
The argument of |\childdocmain| should be left empty
(but it must be present).

%%%%%%%%%%%%%%%%%%%%%%%%%%%%%%%%%%%%%%%%
\DescribeMacro{\childdocof}
Furthermore, add the commands
\begin{center}
\begin{tabular}{l}
|\input{childdoc.def}|\\
|\childdocof{|\textit{main}|}|\\
\end{tabular}
\end{center}
at the top of every child file \textit{child}
which is included by |\include{|\textit{child}|}|
from within the main file
(or at least for those files to be compiled individually).
The argument \textit{main} must be the filename of the main file.

There are a couple of
considerations in setting up the main and child documents:

%%%%%%%%%%%%%%%%%%%%%%%%%%%%%%%%%%%%%%%%
\paragraph{Restrictions.}

Please note the following restrictions:
\begin{itemize}
\item
|\childdocmain| must be called with one argument \textit{main}
to ensure compatibility with earlier version of the package.
It must either be empty (|\childdocmain{}|)
or precisely match the filename of the main file in which it is specified.
See \secref{sec:detection} for further information.
\item
The filename \textit{main} must be specified without the |.tex| extension.
\item
The filename \textit{main} is case sensitive
(even in case-insensitive file systems)
due to internal string comparison.
\item
The argument \textit{main} should be fully expanded, it cannot be a macro.
\item
Subdirectories and special characters should be avoided in filenames.
\item
The command |\childdocmain{|\textit{main}|}| must be followed by a whitespace.
It should not be followed immediately by another command
or by a comment mark `|%|'.
This is because the \TeX{} parser reads the token immediately following
the argument of |\childdocmain| and puts it
at the beginning of every child section;
however, a white\-space is ignored.
\end{itemize}

%%%%%%%%%%%%%%%%%%%%%%%%%%%%%%%%%%%%%%%%
\paragraph{Content of Main File.}

It is advisable to place all content in the child files included by |\include|.
Any output contained in the main file will appear in all child documents
unless suppressed manually;
it cannot be suppressed automatically by the |\includeonly| directive
and thus should normally be avoided.
A method to include some content in the main file
by means of conditional processing is described in \secref{sec:conditional}.

%%%%%%%%%%%%%%%%%%%%%%%%%%%%%%%%%%%%%%%%
\paragraph{Page Numbering.}

When only a part of the document is compiled,
the appropriate numbering of pages
(as well as other status parameters)
is determined from the |.aux| files.
The latter contain information from previous passes.
However this information needs to propagate through
all intermediate child documents.
Therefore the page numbering in child documents may well
be inconsistent until the complete document is compiled at least once.

A useful (if unconventional) way to always ensure a consistent
page numbering is to restart the numbering in each child document
and denote the pages by `\textit{child}|.|\textit{page}'
where \textit{child} represents the chapter/section number of the child file.
This can be achieved by the command
|\numberwithin{page}{|\textit{child}|}|
of the \textsf{amsmath} package
where \textit{child} can be |chapter| or |section|
depending on the chosen structuring.
Alternatively, one can modify the macro |\thepage| appropriately
and reset the counter |page| at the start of each child file.

%%%%%%%%%%%%%%%%%%%%%%%%%%%%%%%%%%%%%%%%%%%%%%%%%%%%%%%%%%%%%%%%%%%%%%%%%%%%%%%%
\subsection{Conditional Processing}
\label{sec:conditional}

The package provides a mechanism to compile different versions
of a document. To customise the versions further some conditional processing
can come in handy to distinguish which version is being compiled.
The package provides two macros to describe the compilation context:

%%%%%%%%%%%%%%%%%%%%%%%%%%%%%%%%%%%%%%%%
\DescribeMacro{\ifchilddoc}
The conditional |\ifchilddoc| distinguishes between the compilation of
child documents and the main document:
%
\begin{center}
|\ifchilddoc |\textit{child-code}| |[|\||else |\textit{main-code}]| \||fi|
\end{center}

%%%%%%%%%%%%%%%%%%%%%%%%%%%%%%%%%%%%%%%%
\DescribeMacro{\childdocname}
\DescribeMacro{\childdocjob}
The macro |\childdocname| contains the filename (without extension)
of the main or child file being processed.
Note that |\childdocjob| will always contain the name of the main file.

%%%%%%%%%%%%%%%%%%%%%%%%%%%%%%%%%%%%%%%%
\paragraph{Title Page.}

Conditional processing can be used to include a title or banner page
in the main document when proper precautions are taken.
Importantly, the code in the main file should ensure that the page counter
(as well as other status parameters which are stored in the |.aux| files)
takes the same value after the conditional processing.
Otherwise the page numbers may take divergent values
depending on which part is compiled.

For example, a title page could be declared by:
%
\begin{center}
\begin{tabular}{l}
|\ifchilddoc\||else|\\
|\addtocounter{page}{-1}|\\
\textit{code for title page}\\
|\newpage|\\
|\||fi|
\end{tabular}
\end{center}
%
A banner page for the child documents can be generated by:
%
\begin{center}
\begin{tabular}{l}
|\ifchilddoc|\\
|\addtocounter{page}{-1}|\\
\textit{code for banner page}\\
|\newpage|\\
|\||fi|
\end{tabular}
\end{center}
%
Here one could write a message such as:
\begin{center}
|This is the part \childdocname{} of \childdocjob{}.|
\end{center}

%%%%%%%%%%%%%%%%%%%%%%%%%%%%%%%%%%%%%%%%%%%%%%%%%%%%%%%%%%%%%%%%%%%%%%%%%%%%%%%%
\subsection{Flags}
\label{sec:flags}

The package makes it easy to generate different versions
of the main or child documents.
To this end compilation flags can be defined
and assigned different default values.
They will be particularly useful in conjunction
with the forwarding mechanism described in \secref{sec:forward}.

For example, it may be useful to have a flag |\version|
which can be set to |draft| or |final|.
The document source will contain some conditional code
depending on the value of |\version|.
Suppose further, the flag should default to |final| for the main file
and to |draft| for child files
which is a natural assignment for editing the document.
This is achieved by placing the following code
in the preamble of the main document
(below the |\childdocmain| directive):
%
\begin{center}
\begin{tabular}{l}
|\ifchilddoc|\\
|\providecommand{\version}{draft}|\\
|\||else|\\
|\providecommand{\version}{final}|\\
|\||fi|
\end{tabular}
\end{center}
%
The definition by |\providecommand| makes sure
that previous definitions are not overwritten.
Further statements |\providecommand{\version}{...}|
can thus be added before the above code to override it.

For the main file, one might add a line
(between |\childdocmain| and the above block)
%
\begin{center}
|%\ifchilddoc\||else\providecommand{\version}{draft}\||fi|
\end{center}
%
which can be uncommented to produce a draft version.
Likewise one can add a line to the very top of a child file
(above the |\childdocof{|\textit{main}|}| directive)
%
\begin{center}
|%\providecommand{\version}{final}|
\end{center}
%
which can be uncommented to produce the final version of this child document.

%%%%%%%%%%%%%%%%%%%%%%%%%%%%%%%%%%%%%%%%%%%%%%%%%%%%%%%%%%%%%%%%%%%%%%%%%%%%%%%%
\subsection{Forwarding}
\label{sec:forward}

Different versions of the main or child documents
using compilation flags as described in \secref{sec:flags}
can be (permanently) stored in different files
for convenient compilation, viewing and distribution.
To this end, the package defines a command
to pass on compilation to a different file:

%%%%%%%%%%%%%%%%%%%%%%%%%%%%%%%%%%%%%%%%
\DescribeMacro{\childdocforward}
The command |\childdocforward| redirects processing to
another source file:
%
\begin{center}
\begin{tabular}{l}
|\input{childdoc.def}|\\
|\childdocforward[|\textit{main}|]{|\textit{dest}|}|\\
\end{tabular}
\end{center}
%
The argument \textit{dest} is the destination file
(without extension).
It should be the main file or one of the child files.
Note that further \textsf{childdoc} directives
such as |\childdocof| and |\childdocforward|
in the indicated file will be processed in this form.
The optional argument \textit{main}
passes on directly to the main file \textit{main}
while pretending to compile the child \textit{dest}.
This form behaves as if \textit{dest}
issues |\childdocof{|\textit{main}|}| right away,
and no further \textsf{childdoc} directives will be processed.

%%%%%%%%%%%%%%%%%%%%%%%%%%%%%%%%%%%%%%%%
\DescribeMacro{\...prefix}
In the alternative form |\childdocforwardprefix|,
%
\begin{center}
\begin{tabular}{l}
|\input{childdoc.def}|\\
|\childdocforwardprefix[|\textit{main}|]{|\textit{prefix}|}{|\textit{dest}|}|
\end{tabular}
\end{center}
%
the destination file is determined by a pattern
depending on the current file:
To make this work, the current file must be called
`{\textit{prefix}\hspace{0.2em}\textit{suffix}}'
with \textit{prefix} matching precisely the argument.
Processing is then passed on to the file
`{\textit{dest}\hspace{0.2em}\textit{suffix}}'.
Surely, the same effect is achieved by
directly specifying the
argument `{\textit{dest}\hspace{0.2em}\textit{suffix}}'
in the first form.
However, that requires to set up a different file
for each child. With the alternative form of the command
all these files can have exactly the same content
which simplifies setting them up and maintaining them.

For example, the following file |draft.tex|
with a compilation flag |\version| as described in \secref{sec:flags}
compiles the main document as a draft:
%
\begin{center}
\begin{tabular}{l}
|\def\version{draft}|\\
|\input{childdoc.def}|\\
|\childdocforward{|\textit{main}|}|
\end{tabular}
\end{center}
%
Likewise, the following files |final|\textit{nn}|.tex|
compile the final version of the child document
|child|\textit{nn}|.tex|:
%
\begin{center}
\begin{tabular}{l}
|\def\version{final}|\\
|\input{childdoc.def}|\\
|\childdocforwardprefix{final}{child}|
\end{tabular}
\end{center}
%

Note that when several versions of a main file and/or of each child file
are to be generated, it may be convenient to set up a |Makefile| or
shell script to automatise the process.

%%%%%%%%%%%%%%%%%%%%%%%%%%%%%%%%%%%%%%%%%%%%%%%%%%%%%%%%%%%%%%%%%%%%%%%%%%%%%%%%
\subsection{Command Line Processing}
\label{sec:commandline}

The effect of redirection files can also be achieved by invoking
the \LaTeX{} compiler with a more elaborate command line.
Most conveniently this should be done as part
of a shell script or a |Makefile|.

When using \textsf{childdoc} in the main file, the following
command lines effectively perform a redirection
(note that depending on the shell being used,
backslashes may have to be doubled: `|\|' $\to$ `|\\|'):
%
\begin{center}
|... -jobname "|\textit{target}|" |\\|"|[\textit{flags}]%
|\input{childdoc.def}\childdocforward[|\textit{main}|]{|\textit{dest}|}"|
\end{center}
%
Here \textit{target} is the name of the output file,
\textit{main} is the name of the main file
and \textit{dest} is the name of the main or child file to be processed
(all filenames without extensions).
The optional argument \textit{main} can be omitted
if \textit{main} matches \textit{dest}.
Optionally, compilation \textit{flags} can be defined via |\def| commands.
This command line makes the \TeX{} engine believe
it is compiling the file \textit{target}
whose content is specified as the latter parameter.
The provided code then forwards the processing to
\textit{main} or \textit{dest} as described in \secref{sec:forward}.

%%%%%%%%%%%%%%%%%%%%%%%%%%%%%%%%%%%%%%%%%%%%%%%%%%%%%%%%%%%%%%%%%%%%%%%%%%%%%%%%
\subsection{Include by Input}
\label{sec:input}

Including child documents by |\include| has some restrictions by design.
Most notably, the content of a child document always occupies
its own set of pages; pages cannot be shared between child documents.
Usually, this behaviour makes perfect sense
because each child document contain an essential part of the document.
However, in some situations it may be desirable to compose
a document from a collection of parts
without having mandatory page breaks between then.
For this case, the package
provides a mechanism to include parts
by |\input| which can also be processed individually.
However, by construction this mechanism
requires manual handling of the content to be output.

%%%%%%%%%%%%%%%%%%%%%%%%%%%%%%%%%%%%%%%%
\DescribeMacro{\ifchilddocmanual}
The main file should be prepared as usual, see \secref{sec:include}.
However, the document body must make a distinction
between processing of an individual part and of the main document, e.g.:
%
\begin{center}
\begin{tabular}{l}
|\ifchilddocmanual|\\
|\input{\childdocname}|\\
|\||else|\\
\textit{document body with }|\input{|\textit{part}|}|\\
|\||fi|
\end{tabular}
\end{center}
%
The conditional |\ifchilddocmanual| is true whenever
a part to be included by |\input| is being compiled,
and the name of the part is stored in |\childdocname|.

%%%%%%%%%%%%%%%%%%%%%%%%%%%%%%%%%%%%%%%%
\DescribeMacro{\childdocby}
Each part to be included by |\input| should start with:
%
\begin{center}
\begin{tabular}{l}
|\input{childdoc.def}|\\
|\childdocby{|\textit{main}|}|\\
\end{tabular}
\end{center}
%
The directive |\childdocby| is similar to |\childdocof|
described in \secref{sec:include},
but the subsequent selection of content must be done manually.
To that end, both |\ifchilddoc| and |\ifchilddocmanual|
will be true upon processing of a part,
and the name of the part is stored in |\childdocname|.
Note that |\jobname| will be set to the filename of the current part
so that each part receives an individual |.aux| file
that does not interfere with the |.aux| file(s) of the main document.
This behaviour can be altered by the alternative form
|\childdocby[*]{|\textit{main}|}| (with a non-empty optional argument)
which uses the |.aux| file of the main document
by setting |\jobname| to \textit{main}.

%%%%%%%%%%%%%%%%%%%%%%%%%%%%%%%%%%%%%%%%%%%%%%%%%%%%%%%%%%%%%%%%%%%%%%%%%%%%%%%%
\subsection{Driver Development}
\label{sec:driver}

The \textsf{childdoc} mechanism can also be use for the development
of definition files such as \LaTeX{} styles or classes.
This case differs from the above setup with multiple parts
included by |\include| in that no |\includeonly| should be invoked.
This can be achieved by starting the include file
(before |\ProvidesPackage|) with:
%
\begin{center}
\begin{tabular}{l}
|\input{childdoc.def}|\\
|\childdocforward{|\textit{main}|}|\\
\end{tabular}
\end{center}
%
or alternatively with:
%
\begin{center}
\begin{tabular}{l}
|\input{childdoc.def}|\\
|\childdocby{|\textit{main}|}|\\
\end{tabular}
\end{center}
%
Both forms have slightly different effects as described above.
The main file is prepared as usual, see \secref{sec:include}.

%%%%%%%%%%%%%%%%%%%%%%%%%%%%%%%%%%%%%%%%%%%%%%%%%%%%%%%%%%%%%%%%%%%%%%%%%%%%%%%%
\subsection{Legacy Detection}
\label{sec:detection}

The directive |\childdocmain| in the main file can detect
whether the complete document or merely a child is to be compiled
even without using the directive |\childdocof|.
This method is deprecated because it is less robust
and there is no compelling reason to use it;
it is merely provided for backward compatibility
and it may be removed in future versions.

If the detection mechanism is to be used,
it is mandatory to correctly specify
the filename of the main file as the argument of |\childdocmain|:
%
\begin{center}
\begin{tabular}{l}
|\input{childdoc.def}|\\
|\childdocmain{|\textit{main}|}|\\
\end{tabular}
\end{center}
%
If |\jobname| does not match the argument \textit{main} of |\childdocmain|,
it is assumed that |\jobname| points to the child file to be compiled.
When using |\childdocmain| with the main file specified as argument,
it suffices to start a child file
with just |\input{|\textit{main}|}|
without loading of the package and using |\childdocof|.
If instead all processing is done
with the appropriate \textsf{childdoc} directives,
the argument of \textit{main} of |\childdocmain| can be empty.

An alternative version of the command line processing described
in \secref{sec:commandline} using the detection mechanism reads:
%
\begin{center}
|... -jobname "|\textit{target}|" "|[\textit{flags}]%
[|\def\jobname{|\textit{dest}|}|]|\input{|\textit{main}|}"|
\end{center}

%%%%%%%%%%%%%%%%%%%%%%%%%%%%%%%%%%%%%%%%%%%%%%%%%%%%%%%%%%%%%%%%%%%%%%%%%%%%%%%%
\subsection{Manual Code}
\label{sec:manual}

In case one cannot be certain whether the definitions file |childdoc.def|
is installed on the target \TeX{} distribution
and one prefers not to ship it,
it is conceivable to paste a few relevant commands into the sources.

To that end, drop all statements |\input{childdoc.def}|
and perform the replacements as outlined below.
Instead of |\childdocmain{|\textit{main}|}| add the following code
to the top of the main file:
%
\begin{center}
\begin{tabular}{l}
|\||ifdefined\childdocname\endinput\||fi\newif\ifchilddoc|\\
|\edef\childdocname{\scantokens\expandafter{\jobname\noexpand}}|\\
|\def\childdocmain{|\textit{main}|}\||ifx\childdocmain\childdocname\||else|\\
|\childdoctrue\includeonly{\childdocname}\let\jobname\childdocmain\||fi|\\
\end{tabular}
\end{center}
%
Instead of |\childdocof{|\textit{main}|}| just include the main file
at the top of each child file:
%
\begin{center}
|\input{|\textit{main}|}|
\end{center}
%
A simple redirection |\childdocforward{|\textit{dest}|}| is achieved by:
%
\begin{center}
|\def\jobname{|\textit{dest}|}\input{\jobname}|
\end{center}
%
The redirection with prefix
|\childdocforwardprefix[|\textit{prefix}|]{|\textit{dest}|}|
is accomplished by:
%
\begin{center}
\begin{tabular}{l}
|{\edef\jobname{\scantokens\expandafter{\jobname\noexpand}}|\\
|\def\redirectjob |\textit{prefix}|#1~~~{\gdef\jobname{|\textit{dest}|#1}}|\\
|\expandafter\redirectjob\jobname~~~}\input{\jobname}|
\end{tabular}
\end{center}

In an alternative approach,
child documents can be compiled by a specific command line
without additional code or specific definitions:
%
\begin{center}
|... -jobname "|\textit{target}|" "|[\textit{flags}]%
|\includeonly{|\textit{dest}|}\input{|\textit{main}|}"|
\end{center}
%

%%%%%%%%%%%%%%%%%%%%%%%%%%%%%%%%%%%%%%%%%%%%%%%%%%%%%%%%%%%%%%%%%%%%%%%%%%%%%%%%
%%%%%%%%%%%%%%%%%%%%%%%%%%%%%%%%%%%%%%%%%%%%%%%%%%%%%%%%%%%%%%%%%%%%%%%%%%%%%%%%
\section{Information}

%%%%%%%%%%%%%%%%%%%%%%%%%%%%%%%%%%%%%%%%%%%%%%%%%%%%%%%%%%%%%%%%%%%%%%%%%%%%%%%%
\subsection{Copyright}

Copyright \copyright{} 2017--2018 Niklas Beisert

This work may be distributed and/or modified under the
conditions of the \LaTeX{} Project Public License, either version 1.3
of this license or (at your option) any later version.
The latest version of this license is in
  \url{http://www.latex-project.org/lppl.txt}
and version 1.3 or later is part of all distributions of \LaTeX{}
version 2005/12/01 or later.

This work has the LPPL maintenance status `maintained'.

The Current Maintainer of this work is Niklas Beisert.

This work consists of the files |README.txt|, |childdoc.ins| and |childdoc.dtx|
as well as the derived files |childdoc.def|, |cdocsamp.tex|
with |cdocsch1.tex|, |cdocsch2.tex|, |cdocspt3.tex|, |cdocspt4.tex|,
|cdocsdrf.tex|, |cdocsfn1.tex|, |cdocsfn2.tex|
as well as |childdoc.pdf|.

%%%%%%%%%%%%%%%%%%%%%%%%%%%%%%%%%%%%%%%%%%%%%%%%%%%%%%%%%%%%%%%%%%%%%%%%%%%%%%%%
\subsection{Files and Installation}

The package consists of the files:
%
\begin{center}
\begin{tabular}{ll}
    |README.txt|   & readme file \\
    |childdoc.ins| & installation file \\
    |childdoc.dtx| & source file \\
    |childdoc.def| & definition file \\
    |cdocsamp.tex| & sample main file \\
    |cdocsch1.tex| & sample include file \\
    |cdocsch2.tex| & sample include file \\
    |cdocspt3.tex| & sample part file \\
    |cdocspt4.tex| & sample part file \\
    |cdocsdrf.tex| & sample redirection file \\
    |cdocsfn1.tex| & sample redirection file \\
    |cdocsfn2.tex| & sample redirection file \\
    |childdoc.pdf| & manual
\end{tabular}
\end{center}
%
The distribution consists of the files
|README.txt|, |childdoc.ins| and |childdoc.dtx|.
%
\begin{itemize}
\item
Run (pdf)\LaTeX{} on |childdoc.dtx|
to compile the manual |childdoc.pdf| (this file).
\item
Run \LaTeX{} on |childdoc.ins| to create the definitions file |childdoc.def|
and the sample |cdocsamp.tex| with include files
|cdocsch1.tex|, |cdocsch2.tex|, |cdocspt3.tex|, |cdocspt4.tex|,
|cdocsdrf.tex|, |cdocsfn1.tex|, |cdocsfn2.tex|.
Then copy the file |childdoc.def| to an appropriate directory of your \LaTeX{}
distribution, e.g.\ \textit{texmf-root}|/tex/latex/childdoc|.
\end{itemize}

%%%%%%%%%%%%%%%%%%%%%%%%%%%%%%%%%%%%%%%%%%%%%%%%%%%%%%%%%%%%%%%%%%%%%%%%%%%%%%%%
\subsection{Related CTAN Packages}

There are several other packages which offer a similar functionality:
%
\begin{itemize}
\item
The packages
\href{http://ctan.org/pkg/docmute}{\textsf{docmute}},
\href{http://ctan.org/pkg/includex}{\textsf{includex}} and
\href{http://ctan.org/pkg/standalone}{\textsf{standalone}}
provide commands to include only the document body of
a child file thus allowing both files to be compiled individually.
\item
The packages \href{http://ctan.org/pkg/subdocs}{\textsf{subdocs}}
and \href{http://ctan.org/pkg/subfiles}{\textsf{subfiles}}
provide structures in which the main and child documents can be
encapsulated and allowing them to be compiled individually.
The inclusion mechanism is different from the conventional |\include|.
\item
The package \href{http://ctan.org/pkg/combine}{\textsf{combine}}
is an elaborate solution to combine several documents into one.
\end{itemize}
%
See also the CTAN topic \href{http://ctan.org/topic/subdocs}{\textsf{subdocs}}
for further related packages.
The present package differs from the above solutions in that
a document structure constructed with the conventional |\include| mechanism
just needs two extra commands at the top of every file
such that all constituent files can be compiled individually.

%%%%%%%%%%%%%%%%%%%%%%%%%%%%%%%%%%%%%%%%%%%%%%%%%%%%%%%%%%%%%%%%%%%%%%%%%%%%%%%%
%\subsection{Feature Suggestions}
%
%The following is a list of features which may be useful for future
%versions of this package:
%%
%\begin{itemize}
%\item
%\ldots
%\end{itemize}

%%%%%%%%%%%%%%%%%%%%%%%%%%%%%%%%%%%%%%%%%%%%%%%%%%%%%%%%%%%%%%%%%%%%%%%%%%%%%%%%
\subsection{Revision History}

%%%%%%%%%%%%%%%%%%%%%%%%%%%%%%%%%%%%%%%%
\paragraph{v2.0:} 2018/12/30

\begin{itemize}
\item
immediate forward processing
\item
added |\childdocby| mechanism
\item
manual restructured
\end{itemize}

%%%%%%%%%%%%%%%%%%%%%%%%%%%%%%%%%%%%%%%%
\paragraph{v1.6:} 2018/01/17

\begin{itemize}
\item
application for development of include files
\item
corrections to manual
\end{itemize}

%%%%%%%%%%%%%%%%%%%%%%%%%%%%%%%%%%%%%%%%
\paragraph{v1.5:} 2017/05/21

\begin{itemize}
\item
more complete structuring introduced
\item
|\childdocof| introduced
\item
|\childdoc| renamed to |\childdocmain|
\item
|\childredirect| renamed to |\childdocforward| and |\childdocforwardprefix|
and functionality expanded
\end{itemize}

%%%%%%%%%%%%%%%%%%%%%%%%%%%%%%%%%%%%%%%%
\paragraph{v1.0:} 2017/04/27

\begin{itemize}
\item
manual and install package
\item
first version published on CTAN
\end{itemize}

%%%%%%%%%%%%%%%%%%%%%%%%%%%%%%%%%%%%%%%%
\paragraph{v0.6:} 2017/04/26

\begin{itemize}
\item
redirection mechanism added
\end{itemize}

%%%%%%%%%%%%%%%%%%%%%%%%%%%%%%%%%%%%%%%%
\paragraph{v0.5:} 2017/04/26

\begin{itemize}
\item
functionality in definition file
\end{itemize}


%%%%%%%%%%%%%%%%%%%%%%%%%%%%%%%%%%%%%%%%%%%%%%%%%%%%%%%%%%%%%%%%%%%%%%%%%%%%%%%%
%%%%%%%%%%%%%%%%%%%%%%%%%%%%%%%%%%%%%%%%%%%%%%%%%%%%%%%%%%%%%%%%%%%%%%%%%%%%%%%%
%%%%%%%%%%%%%%%%%%%%%%%%%%%%%%%%%%%%%%%%%%%%%%%%%%%%%%%%%%%%%%%%%%%%%%%%%%%%%%%%
\appendix

\settowidth\MacroIndent{\rmfamily\scriptsize 000\ }

 \DocInput{childdoc.dtx}

\end{document}
%</driver>
% \fi
%
% %%%%%%%%%%%%%%%%%%%%%%%%%%%%%%%%%%%%%%%%%%%%%%%%%%%%%%%%%%%%%%%%%%%%%%%%%%%%%%
% %%%%%%%%%%%%%%%%%%%%%%%%%%%%%%%%%%%%%%%%%%%%%%%%%%%%%%%%%%%%%%%%%%%%%%%%%%%%%%
% \section{Sample}
%\iffalse
%<*samplemain>
%\fi
%
% The following presents a sample document
% with two chapters, two parts, a title page,
% a compile flag as well as three forwarding files to set the flag.
% It consists of eight |.tex| files:
% \begin{center}
% \begin{tabular}{ll}
% |cdocsamp.tex|&main file\\
% |cdocsch1.tex|&include file for chapter 1\\
% |cdocsch2.tex|&include file for chapter 2\\
% |cdocspt3.tex|&include file for part 3\\
% |cdocspt4.tex|&include file for part 4\\
% |cdocsdrf.tex|&forwarding file for main file in draft mode\\
% |cdocsfi1.tex|&forwarding file for final version of chapter 1\\
% |cdocsfi2.tex|&forwarding file for final version of chapter 2\\
% \end{tabular}
% \end{center}
% Each of the eight files can be compiled directly by the \LaTeX{} compiler.
%
% %%%%%%%%%%%%%%%%%%%%%%%%%%%%%%%%%%%%%%
% \paragraph{Main File.}
%
% The main file is called |cdocsamp.tex|.
%
% Load the \textsf{childdoc} definitions and
% declare the filename for the main document:
%    \begin{macrocode}
\input{childdoc.def}
\childdocmain{}
%    \end{macrocode}

% Optional override for |\version| flag:
%    \begin{macrocode}
%%\ifchilddoc\else\providecommand{\version}{draft}\fi
%    \end{macrocode}

% Define the default values for the |\version| flag
% (|final| for the main file and |draft| for childs):
%    \begin{macrocode}
\ifchilddoc
\providecommand{\version}{draft}
\else
\providecommand{\version}{final}
\fi
%    \end{macrocode}

% Load the standard document class:
%    \begin{macrocode}
\documentclass[12pt]{article}
%    \end{macrocode}

% Start the document body:
%    \begin{macrocode}
\begin{document}
%    \end{macrocode}

% Declare a title page.
% Print title, part of document being processed and version flag:
%    \begin{macrocode}
\addtocounter{page}{-1}
\begin{center}
{\LARGE\bfseries{}childdoc example\par}
\vspace{1cm}
\ifchilddoc
\ifchilddocmanual part\else chapter\fi:
`\childdocname' of `\childdocjob'\par
\else
main document: `\childdocjob'\par
\fi
version: \version\par
\end{center}
\newpage
%    \end{macrocode}

% Manually include selected file,
% otherwise process as usual:
%    \begin{macrocode}
\ifchilddocmanual
\section*{part `\childdocname'}
\input{\childdocname}
\else
%    \end{macrocode}

% Include the two chapters:
%    \begin{macrocode}
\include{cdocsch1}
\include{cdocsch2}
%    \end{macrocode}

% Include the two parts unless only chapters should be displayed:
%    \begin{macrocode}
\ifchilddoc\else
\section{part three}
\input{cdocspt3}
\section{part four}
\input{cdocspt4}
\fi
%    \end{macrocode}

% Process as usual until here:
%    \begin{macrocode}
\fi
%    \end{macrocode}

% End of document body:
%    \begin{macrocode}
\end{document}
%    \end{macrocode}
%\iffalse
%</samplemain>
%\fi
%
% %%%%%%%%%%%%%%%%%%%%%%%%%%%%%%%%%%%%%%
% \paragraph{Chapter Include Files.}
%
% The include files are called |cdocsch1.tex| and |cdocsch2.tex|.
%
%\iffalse
%<*samplechap1|samplechap2>
%\fi

% Optional override for |\version| flag:
%    \begin{macrocode}
%%\providecommand{\version}{final}
%    \end{macrocode}

% Include the main document:
%    \begin{macrocode}
\input{childdoc.def}
\childdocof{cdocsamp}
%    \end{macrocode}

%\iffalse
%</samplechap1|samplechap2>
%\fi
%
%\iffalse
%<*samplechap1>
%\fi
% Some text for chapter 1:
%    \begin{macrocode}
\section{one}
some text in chapter one
%    \end{macrocode}

%\iffalse
%</samplechap1>
%\fi
% Some text for chapter 2:
%\iffalse
%<*samplechap2>
%\fi
%    \begin{macrocode}
\section{two}
more text in chapter two
%    \end{macrocode}

%\iffalse
%</samplechap2>
%\fi
%
% %%%%%%%%%%%%%%%%%%%%%%%%%%%%%%%%%%%%%%
% \paragraph{Part Include Files.}
%
% The include files are called |cdocspt3.tex| and |cdocspt4.tex|.
%
%\iffalse
%<*samplepart3|samplepart4>
%\fi

% Optional override for |\version| flag:
%    \begin{macrocode}
%%\providecommand{\version}{final}
%    \end{macrocode}

% Include the main document:
%    \begin{macrocode}
\input{childdoc.def}
\childdocby{cdocsamp}
%    \end{macrocode}

%\iffalse
%</samplepart3|samplepart4>
%\fi
%
%\iffalse
%<*samplepart3>
%\fi
% Some text for part 3:
%    \begin{macrocode}
some text in part three
%    \end{macrocode}

%\iffalse
%</samplepart3>
%\fi
% Some text for part 4:
%\iffalse
%<*samplepart4>
%\fi
%    \begin{macrocode}
more text in part four
%    \end{macrocode}

%\iffalse
%</samplepart4>
%\fi
%
% %%%%%%%%%%%%%%%%%%%%%%%%%%%%%%%%%%%%%%
% \paragraph{Forwarding for a Complete Draft.}
%
% The following forwarding file |cdocsdrf.tex|
% compiles the main document in draft mode:
%\iffalse
%<*sampledraft>
%\fi
%    \begin{macrocode}
\def\version{draft}
\input{childdoc.def}
\childdocforward{cdocsamp}
%    \end{macrocode}

%\iffalse
%</sampledraft>
%\fi
%
% %%%%%%%%%%%%%%%%%%%%%%%%%%%%%%%%%%%%%%
% \paragraph{Forwarding for Final Version of the Chapters.}
%
% The following forwarding files |cdocsfn1.tex| and |cdocsfn2.tex|
% (with identical content)
% compile the final versions of the child documents
% |cdocsch1.tex| and |cdocsch2.tex|, respectively:
%\iffalse
%<*samplefinal>
%\fi
%    \begin{macrocode}
\def\version{final}
\input{childdoc.def}
\childdocforwardprefix[cdocsamp]{cdocsfn}{cdocsch}
%    \end{macrocode}

%\iffalse
%</samplefinal>
%\fi
%
% %%%%%%%%%%%%%%%%%%%%%%%%%%%%%%%%%%%%%%
% \paragraph{Command Line Processing.}
%
% The following three command lines generate the output files
% |cdocscld|, |cdocscl1| and |cdocscl2|
% which should be identical to
% |cdocsdrf|, |cdocsch1| and |cdocsfn2|, respectively:
% \begin{center}
% \begin{tabular}{l}
% |latex -jobname cdocscld \|\\
% |  "\def\version{draft}\input{childdoc.def}\childdocforward{cdocsamp}"|\\
% |latex -jobname cdocscl1 \|\\
% |  "\input{childdoc.def}\childdocforward[cdocsamp]{cdocsch1}"|\\
% |latex -jobname cdocscl2 \|\\
% |  "\def\version{final}\input{childdoc.def}\childdocforward{cdocsch2}"|
% \end{tabular}
% \end{center}
% Note that the trailing backslash on each first line
% merely continues the input to the second line
% (for convenient cut ant paste).
% Furthermore, the command |latex| can be replaced by any
% of its alternative versions such as |pdflatex|.
%
% %%%%%%%%%%%%%%%%%%%%%%%%%%%%%%%%%%%%%%%%%%%%%%%%%%%%%%%%%%%%%%%%%%%%%%%%%%%%%%
% %%%%%%%%%%%%%%%%%%%%%%%%%%%%%%%%%%%%%%%%%%%%%%%%%%%%%%%%%%%%%%%%%%%%%%%%%%%%%%
% \section{Implementation}
%\iffalse
%<*package>
%\fi
%
% This section describes the definitions file |childdoc.def|.

% The definitions cannot be loaded using |\usepackage| or |\RequirePackage|
% which has a mechanism to prevent loading a style file more than once.
% When loading the definitions by means of |\input|
% multiple instances have to be prevented manually:
%\iffalse
%This code needs to be before the `\ProvidesFile' directive
%which is defined at the beginning of this file.
%Therefore it is also placed there and commented out here.
%</package>
%<*discard>
%\fi
%    \begin{macrocode}
\ifdefined\childdocmain\endinput\fi
%    \end{macrocode}
%\iffalse
%</discard>
%<*package>
%\fi
%
% \macro{\ifchilddoc}
% \macro{\ifchilddocmanual}
% The conditional |\ifchilddoc| tells whether a
% child (true) or main (false) document is being compiled.
% The conditional |\ifchilddocmanual| tells whether
% the |\includeonly| mechanism is used (false) or
% the selection of child files must be performed manually (true).
% The definitions initialise to false:
%    \begin{macrocode}
\newif\ifchilddoc
\newif\ifchilddocmanual
%    \end{macrocode}

% \macro{\childdocname}
% \macro{\childdocjob}
% The macro |\childdocname| stores the name of the main document
% to be compiled. The macro |\childdocjob| stores the name of
% the document on which the \LaTeX{} compiler was originally invoked.
% The content of |\jobname| cannot be compared
% to filenames specified in the source due to different catcodes.
% The following code rescans |\jobname|, stores the result
% in |\childdocname| and saves a copy in |\childdocjob|:
%    \begin{macrocode}
\edef\childdocname{\scantokens\expandafter{\jobname\noexpand}}
\let\childdocjob\childdocname
%    \end{macrocode}

% \macro{\childdocdisable}
% The macro |\childdocdisable| prevents the main file
% from being processed more than once.
% At this stage, the main document command |\childdocmain|
% is assumed to be called once again where it should do nothing.
% Any subsequent call to it should prevent
% a secondary processing of the main document
% It overwrites the forwarding commands
% |\childdocof| and |\childdocforward|
% with empty macros to prevent further inclusions of the main document:
%    \begin{macrocode}
\newcommand{\childdocdisable}
{
  \renewcommand{\childdocmain}[1]{\renewcommand{\childdocmain}[1]{\endinput}}
  \renewcommand{\childdocof}[1]{}
  \renewcommand{\childdocby}[2][]{}
  \renewcommand{\childdocforward}[2][]{}
  \renewcommand{\childdocdisable}{}
}
%    \end{macrocode}

% \macro{\childdocmain}
% The macro |\childdocmain| is to be called at the top of the main file
% with nothing or the main filename (without extension) as argument.
% First, it breaks loops.
% If the argument is not empty and does not match |\childdocname|
% (which is set by the first inclusion of |childdoc.def|),
% |\ifchilddoc| is set to true, |\includeonly| is applied to the child file
% and |\jobname| is set to the main file
% (for proper handling of |.aux| files):
%    \begin{macrocode}
\newcommand{\childdocmain}[1]
{
  \childdocdisable\childdocmain{}
  \if?#1?\else
    \begingroup
      \def\childdoctmp{#1}
      \ifx\childdoctmp\childdocname
        \def\childdoctmp{}
      \else
        \def\childdoctmp
        {
          \childdoctrue
          \includeonly{\childdocname}
          \def\childdocjob{#1}
          \def\jobname{#1}
        }
      \fi
      \expandafter
    \endgroup
    \childdoctmp
  \fi
}
%    \end{macrocode}

% \macro{\childdocof}
% The command |\childdocof| redirects
% compilation to the main file |#1|.
%    \begin{macrocode}
\newcommand{\childdocof}[1]
{
  \childdocdisable
  \childdoctrue
  \includeonly{\childdocname}
  \def\jobname{#1}
  \def\childdocjob{#1}
  \input{#1}
}
%    \end{macrocode}

% \macro{\childdocby}
% The command |\childdocby| ....
%    \begin{macrocode}
\newcommand{\childdocby}[2][]
{
  \childdocdisable
  \childdoctrue
  \childdocmanualtrue
  \if?#1?\else
    \def\jobname{#2}
  \fi
  \def\childdocjob{#2}
  \input{#2}
  \endinput
}
%    \end{macrocode}

% \macro{\childdocforward}
% The command |\childdocforward| redirects
% compilation to the main file or
% (if the optional argument is given) a child file.
% Parameters are set as if the main file
% or a child file starting with |\childdocof| was compiled.
% Then compilation is handed over to the main file:
%    \begin{macrocode}
\newcommand{\childdocforward}[2][]
{
  \begingroup
    \if?#1?
      \def\childdoctmp
      {
        \def\childdocname{#2}
        \def\childdocjob{#2}
        \def\jobname{#2}
        \input{#2}
        \endinput
      }
    \else
      \def\childdoctmp
      {
        \childdocdisable
        \def\childdocname{#2}
        \childdoctrue
        \includeonly{#2}
        \def\childdocjob{#1}
        \def\jobname{#1}
        \input{#1}
        \endinput
      }
    \fi
    \expandafter
  \endgroup
  \childdoctmp
}
%    \end{macrocode}

% \macro{\childdocforwardprefix}
% The command |\childdocforwardprefix| redirects
% compilation to the main or a child file by means of a pattern.
% The prefix |#1| in the current filename is replaced by |#2|
% and the suffix of the current filename is kept
% (it is assumed that the filename does not contain the substring `|~~~|'
% which is used as a delimiter).
% Compilation is handed over to the new file by |\childdocforward|:
%    \begin{macrocode}
\newcommand{\childdocforwardprefix}[3][]
{
  \begingroup
    \def\childdocextract #2##1~~~{\def\childdoctmp{\childdocforward[#1]{#3##1}}}
    \expandafter\childdocextract\childdocname~~~
    \expandafter
  \endgroup
  \childdoctmp
}
%    \end{macrocode}

% \macro{\childdoc}
% The deprecated macro |\childdoc| is a legacy version of |\childdocmain|:
%    \begin{macrocode}
\newcommand{\childdoc}{\childdocmain}
%    \end{macrocode}

% \macro{\childdocredirect}
% The deprecated macro |\childdocredirect| is a legacy version
% of |\childdocforward| and |\childdocforwardprefix|:
%    \begin{macrocode}
\newcommand{\childdocredirect}[2][]
{
  \begingroup
    \if?#1?
      \def\childdoctmp{\childdocforward{#2}}
    \else
      \def\childdoctmp{\childdocforwardprefix{#1}{#2}}
    \fi
    \expandafter
  \endgroup
  \childdoctmp
}
%    \end{macrocode}

%\iffalse
%</package>
%\fi
%
\endinput
|\\
|\childdocforwardprefix[|\textit{main}|]{|\textit{prefix}|}{|\textit{dest}|}|
\end{tabular}
\end{center}
%
the destination file is determined by a pattern
depending on the current file:
To make this work, the current file must be called
`{\textit{prefix}\hspace{0.2em}\textit{suffix}}'
with \textit{prefix} matching precisely the argument.
Processing is then passed on to the file
`{\textit{dest}\hspace{0.2em}\textit{suffix}}'.
Surely, the same effect is achieved by
directly specifying the
argument `{\textit{dest}\hspace{0.2em}\textit{suffix}}'
in the first form.
However, that requires to set up a different file
for each child. With the alternative form of the command
all these files can have exactly the same content
which simplifies setting them up and maintaining them.

For example, the following file |draft.tex|
with a compilation flag |\version| as described in \secref{sec:flags}
compiles the main document as a draft:
%
\begin{center}
\begin{tabular}{l}
|\def\version{draft}|\\
|% \iffalse
%
% childdoc.dtx Copyright (C) 2017-2018 Niklas Beisert
%
% This work may be distributed and/or modified under the
% conditions of the LaTeX Project Public License, either version 1.3
% of this license or (at your option) any later version.
% The latest version of this license is in
%   http://www.latex-project.org/lppl.txt
% and version 1.3 or later is part of all distributions of LaTeX
% version 2005/12/01 or later.
%
% This work has the LPPL maintenance status `maintained'.
%
% The Current Maintainer of this work is Niklas Beisert.
%
% This work consists of the files childdoc.dtx and childdoc.ins
% and the derived files childdoc.def and cdocsamp.tex with
% cdocsch1.tex, cdocsch2.tex, cdocsdrf.tex, cdocsfn1.tex, cdocsfn2.tex.
%
%<package>\ifdefined\childdocmain\endinput\fi
%<package>\ProvidesFile{childdoc.def}[2018/12/30 v2.0 child document driver]
%<samplemain>\ProvidesFile{cdocsamp.tex}[2018/12/30 v2.0 sample for childdoc]
%<*driver>
%\ProvidesFile{childdoc.drv}[2018/12/30 v2.0 childdoc reference manual file]
\PassOptionsToClass{10pt,a4paper}{article}
\documentclass{ltxdoc}

\usepackage[margin=35mm]{geometry}
\usepackage{hyperref}
\usepackage{hyperxmp}
\usepackage[usenames]{color}

\hypersetup{colorlinks=true}
\hypersetup{pdfstartview=FitH}
\hypersetup{pdfpagemode=UseNone}
\hypersetup{pdfsource={}}
\hypersetup{pdflang={en-UK}}
\hypersetup{pdfcopyright={Copyright 2017-2018 Niklas Beisert.
  This work may be distributed and/or modified under the
  conditions of the LaTeX Project Public License, either version 1.3
  of this license or (at your option) any later version.}}
\hypersetup{pdflicenseurl={http://www.latex-project.org/lppl.txt}}
\hypersetup{pdfcontactaddress={ETH Zurich, ITP, HIT K,
  Wolfgang-Pauli-Strasse 27}}
\hypersetup{pdfcontactpostcode={8093}}
\hypersetup{pdfcontactcity={Zurich}}
\hypersetup{pdfcontactcountry={Switzerland}}
\hypersetup{pdfcontactemail={nbeisert@itp.phys.ethz.ch}}
\hypersetup{pdfcontacturl={http://people.phys.ethz.ch/\xmptilde nbeisert/}}

\newcommand{\secref}[1]{\hyperref[#1]{section \ref*{#1}}}

\parskip1ex
\parindent0pt
\let\olditemize\itemize
\def\itemize{\olditemize\parskip0pt}

\begin{document}

\title{The \textsf{childdoc} Package}
\hypersetup{pdftitle={The childdoc Package}}
\author{Niklas Beisert\\[2ex]
  Institut f\"ur Theoretische Physik\\
  Eidgen\"ossische Technische Hochschule Z\"urich\\
  Wolfgang-Pauli-Strasse 27, 8093 Z\"urich, Switzerland\\[1ex]
  \href{mailto:nbeisert@itp.phys.ethz.ch}
  {\texttt{nbeisert@itp.phys.ethz.ch}}}
\hypersetup{pdfauthor={Niklas Beisert}}
\hypersetup{pdfsubject={Manual for the LaTeX2e Package childdoc}}
\date{30 December 2018, \textsf{v2.0}}
\maketitle

\begin{abstract}\noindent
\textsf{childdoc} is a \LaTeXe{} package
that enables the direct compilation
of document sections included by |\include|
to individual files.
\end{abstract}

\begingroup
\parskip0ex
\tableofcontents
\endgroup

%%%%%%%%%%%%%%%%%%%%%%%%%%%%%%%%%%%%%%%%%%%%%%%%%%%%%%%%%%%%%%%%%%%%%%%%%%%%%%%%
%%%%%%%%%%%%%%%%%%%%%%%%%%%%%%%%%%%%%%%%%%%%%%%%%%%%%%%%%%%%%%%%%%%%%%%%%%%%%%%%
\section{Introduction}

\LaTeX{} provides a mechanism to structure a large document (such as a book)
into a main file and several child files (containing the chapters)
using the |\include| command.
This mechanism is beneficial for documents
which span hundreds of pages in order to
make the source file(s) more manageable.
Moreover, compilation can be restricted to
selected child files by means of the |\includeonly| command.
The latter feature can be used to reduce the compilation time while editing
(this was significantly more useful in the earlier days of \LaTeX{})
or to generate a smaller document which is easier to navigate.
Another application of |\includeonly| is to generate
documents consisting of selected parts of the complete document.

However, there are a few drawbacks of the plain |\include| mechanism:
\begin{itemize}
\item
The child files cannot be compiled on their own,
they can only be compiled via the main file.
A naive editing environment
(such as a text editor with an option
to have the current file processed by \LaTeX)
may require one to switch to the main file before compiling;
attempting to compile the child file produces errors.
\item
The main file must be modified (each time)
to adjust the |\includeonly| command
to the present needs. This easily leaves the main file in a messy state.
\item
The generated document will always carry the filename
of the main document. This is inconvenient if
several child files are to be compiled and
to be kept for distribution.
\end{itemize}

The present package provides a simple interface
to make child files individually compilable by \LaTeX{}.
Compiling a child file then has the same effect as compiling
the main file with an |\includeonly| command
to select the appropriate child.
Moreover the generated document will carry the name of the child
rather than the main file.
This resolves all three above issues.

This feature is meant to make the editing of books,
thesis documents and lecture notes somewhat more convenient.
However, the package can also be used efficiently for
composing a series of documents (such as exercise sheets)
which are typically distributed individually.
It then assists the author in generating the individual documents
(potentially in different versions)
as well as a document containing the collected series.
Another application is in developing style files
or other kinds of included material
where compilation of the style file could redirect
to a sample or test file.

%%%%%%%%%%%%%%%%%%%%%%%%%%%%%%%%%%%%%%%%%%%%%%%%%%%%%%%%%%%%%%%%%%%%%%%%%%%%%%%%
%%%%%%%%%%%%%%%%%%%%%%%%%%%%%%%%%%%%%%%%%%%%%%%%%%%%%%%%%%%%%%%%%%%%%%%%%%%%%%%%
\section{Usage}

First of all, the package \textsf{childdoc} is \emph{not} a standard
\LaTeXe{} |.sty| style file! Therefore it needs to be invoked in
a non-standard way.

%%%%%%%%%%%%%%%%%%%%%%%%%%%%%%%%%%%%%%%%%%%%%%%%%%%%%%%%%%%%%%%%%%%%%%%%%%%%%%%%
\subsection{Included Files}
\label{sec:include}

%%%%%%%%%%%%%%%%%%%%%%%%%%%%%%%%%%%%%%%%
\DescribeMacro{\childdocmain}
To use the package, add the commands
\begin{center}
\begin{tabular}{l}
|\input{childdoc.def}|\\
|\childdocmain{}|\\
\end{tabular}
\end{center}
at the very top of the main \LaTeX{} file,
in particular \emph{before} the |\documentclass| statement!
The argument of |\childdocmain| should be left empty
(but it must be present).

%%%%%%%%%%%%%%%%%%%%%%%%%%%%%%%%%%%%%%%%
\DescribeMacro{\childdocof}
Furthermore, add the commands
\begin{center}
\begin{tabular}{l}
|\input{childdoc.def}|\\
|\childdocof{|\textit{main}|}|\\
\end{tabular}
\end{center}
at the top of every child file \textit{child}
which is included by |\include{|\textit{child}|}|
from within the main file
(or at least for those files to be compiled individually).
The argument \textit{main} must be the filename of the main file.

There are a couple of
considerations in setting up the main and child documents:

%%%%%%%%%%%%%%%%%%%%%%%%%%%%%%%%%%%%%%%%
\paragraph{Restrictions.}

Please note the following restrictions:
\begin{itemize}
\item
|\childdocmain| must be called with one argument \textit{main}
to ensure compatibility with earlier version of the package.
It must either be empty (|\childdocmain{}|)
or precisely match the filename of the main file in which it is specified.
See \secref{sec:detection} for further information.
\item
The filename \textit{main} must be specified without the |.tex| extension.
\item
The filename \textit{main} is case sensitive
(even in case-insensitive file systems)
due to internal string comparison.
\item
The argument \textit{main} should be fully expanded, it cannot be a macro.
\item
Subdirectories and special characters should be avoided in filenames.
\item
The command |\childdocmain{|\textit{main}|}| must be followed by a whitespace.
It should not be followed immediately by another command
or by a comment mark `|%|'.
This is because the \TeX{} parser reads the token immediately following
the argument of |\childdocmain| and puts it
at the beginning of every child section;
however, a white\-space is ignored.
\end{itemize}

%%%%%%%%%%%%%%%%%%%%%%%%%%%%%%%%%%%%%%%%
\paragraph{Content of Main File.}

It is advisable to place all content in the child files included by |\include|.
Any output contained in the main file will appear in all child documents
unless suppressed manually;
it cannot be suppressed automatically by the |\includeonly| directive
and thus should normally be avoided.
A method to include some content in the main file
by means of conditional processing is described in \secref{sec:conditional}.

%%%%%%%%%%%%%%%%%%%%%%%%%%%%%%%%%%%%%%%%
\paragraph{Page Numbering.}

When only a part of the document is compiled,
the appropriate numbering of pages
(as well as other status parameters)
is determined from the |.aux| files.
The latter contain information from previous passes.
However this information needs to propagate through
all intermediate child documents.
Therefore the page numbering in child documents may well
be inconsistent until the complete document is compiled at least once.

A useful (if unconventional) way to always ensure a consistent
page numbering is to restart the numbering in each child document
and denote the pages by `\textit{child}|.|\textit{page}'
where \textit{child} represents the chapter/section number of the child file.
This can be achieved by the command
|\numberwithin{page}{|\textit{child}|}|
of the \textsf{amsmath} package
where \textit{child} can be |chapter| or |section|
depending on the chosen structuring.
Alternatively, one can modify the macro |\thepage| appropriately
and reset the counter |page| at the start of each child file.

%%%%%%%%%%%%%%%%%%%%%%%%%%%%%%%%%%%%%%%%%%%%%%%%%%%%%%%%%%%%%%%%%%%%%%%%%%%%%%%%
\subsection{Conditional Processing}
\label{sec:conditional}

The package provides a mechanism to compile different versions
of a document. To customise the versions further some conditional processing
can come in handy to distinguish which version is being compiled.
The package provides two macros to describe the compilation context:

%%%%%%%%%%%%%%%%%%%%%%%%%%%%%%%%%%%%%%%%
\DescribeMacro{\ifchilddoc}
The conditional |\ifchilddoc| distinguishes between the compilation of
child documents and the main document:
%
\begin{center}
|\ifchilddoc |\textit{child-code}| |[|\||else |\textit{main-code}]| \||fi|
\end{center}

%%%%%%%%%%%%%%%%%%%%%%%%%%%%%%%%%%%%%%%%
\DescribeMacro{\childdocname}
\DescribeMacro{\childdocjob}
The macro |\childdocname| contains the filename (without extension)
of the main or child file being processed.
Note that |\childdocjob| will always contain the name of the main file.

%%%%%%%%%%%%%%%%%%%%%%%%%%%%%%%%%%%%%%%%
\paragraph{Title Page.}

Conditional processing can be used to include a title or banner page
in the main document when proper precautions are taken.
Importantly, the code in the main file should ensure that the page counter
(as well as other status parameters which are stored in the |.aux| files)
takes the same value after the conditional processing.
Otherwise the page numbers may take divergent values
depending on which part is compiled.

For example, a title page could be declared by:
%
\begin{center}
\begin{tabular}{l}
|\ifchilddoc\||else|\\
|\addtocounter{page}{-1}|\\
\textit{code for title page}\\
|\newpage|\\
|\||fi|
\end{tabular}
\end{center}
%
A banner page for the child documents can be generated by:
%
\begin{center}
\begin{tabular}{l}
|\ifchilddoc|\\
|\addtocounter{page}{-1}|\\
\textit{code for banner page}\\
|\newpage|\\
|\||fi|
\end{tabular}
\end{center}
%
Here one could write a message such as:
\begin{center}
|This is the part \childdocname{} of \childdocjob{}.|
\end{center}

%%%%%%%%%%%%%%%%%%%%%%%%%%%%%%%%%%%%%%%%%%%%%%%%%%%%%%%%%%%%%%%%%%%%%%%%%%%%%%%%
\subsection{Flags}
\label{sec:flags}

The package makes it easy to generate different versions
of the main or child documents.
To this end compilation flags can be defined
and assigned different default values.
They will be particularly useful in conjunction
with the forwarding mechanism described in \secref{sec:forward}.

For example, it may be useful to have a flag |\version|
which can be set to |draft| or |final|.
The document source will contain some conditional code
depending on the value of |\version|.
Suppose further, the flag should default to |final| for the main file
and to |draft| for child files
which is a natural assignment for editing the document.
This is achieved by placing the following code
in the preamble of the main document
(below the |\childdocmain| directive):
%
\begin{center}
\begin{tabular}{l}
|\ifchilddoc|\\
|\providecommand{\version}{draft}|\\
|\||else|\\
|\providecommand{\version}{final}|\\
|\||fi|
\end{tabular}
\end{center}
%
The definition by |\providecommand| makes sure
that previous definitions are not overwritten.
Further statements |\providecommand{\version}{...}|
can thus be added before the above code to override it.

For the main file, one might add a line
(between |\childdocmain| and the above block)
%
\begin{center}
|%\ifchilddoc\||else\providecommand{\version}{draft}\||fi|
\end{center}
%
which can be uncommented to produce a draft version.
Likewise one can add a line to the very top of a child file
(above the |\childdocof{|\textit{main}|}| directive)
%
\begin{center}
|%\providecommand{\version}{final}|
\end{center}
%
which can be uncommented to produce the final version of this child document.

%%%%%%%%%%%%%%%%%%%%%%%%%%%%%%%%%%%%%%%%%%%%%%%%%%%%%%%%%%%%%%%%%%%%%%%%%%%%%%%%
\subsection{Forwarding}
\label{sec:forward}

Different versions of the main or child documents
using compilation flags as described in \secref{sec:flags}
can be (permanently) stored in different files
for convenient compilation, viewing and distribution.
To this end, the package defines a command
to pass on compilation to a different file:

%%%%%%%%%%%%%%%%%%%%%%%%%%%%%%%%%%%%%%%%
\DescribeMacro{\childdocforward}
The command |\childdocforward| redirects processing to
another source file:
%
\begin{center}
\begin{tabular}{l}
|\input{childdoc.def}|\\
|\childdocforward[|\textit{main}|]{|\textit{dest}|}|\\
\end{tabular}
\end{center}
%
The argument \textit{dest} is the destination file
(without extension).
It should be the main file or one of the child files.
Note that further \textsf{childdoc} directives
such as |\childdocof| and |\childdocforward|
in the indicated file will be processed in this form.
The optional argument \textit{main}
passes on directly to the main file \textit{main}
while pretending to compile the child \textit{dest}.
This form behaves as if \textit{dest}
issues |\childdocof{|\textit{main}|}| right away,
and no further \textsf{childdoc} directives will be processed.

%%%%%%%%%%%%%%%%%%%%%%%%%%%%%%%%%%%%%%%%
\DescribeMacro{\...prefix}
In the alternative form |\childdocforwardprefix|,
%
\begin{center}
\begin{tabular}{l}
|\input{childdoc.def}|\\
|\childdocforwardprefix[|\textit{main}|]{|\textit{prefix}|}{|\textit{dest}|}|
\end{tabular}
\end{center}
%
the destination file is determined by a pattern
depending on the current file:
To make this work, the current file must be called
`{\textit{prefix}\hspace{0.2em}\textit{suffix}}'
with \textit{prefix} matching precisely the argument.
Processing is then passed on to the file
`{\textit{dest}\hspace{0.2em}\textit{suffix}}'.
Surely, the same effect is achieved by
directly specifying the
argument `{\textit{dest}\hspace{0.2em}\textit{suffix}}'
in the first form.
However, that requires to set up a different file
for each child. With the alternative form of the command
all these files can have exactly the same content
which simplifies setting them up and maintaining them.

For example, the following file |draft.tex|
with a compilation flag |\version| as described in \secref{sec:flags}
compiles the main document as a draft:
%
\begin{center}
\begin{tabular}{l}
|\def\version{draft}|\\
|\input{childdoc.def}|\\
|\childdocforward{|\textit{main}|}|
\end{tabular}
\end{center}
%
Likewise, the following files |final|\textit{nn}|.tex|
compile the final version of the child document
|child|\textit{nn}|.tex|:
%
\begin{center}
\begin{tabular}{l}
|\def\version{final}|\\
|\input{childdoc.def}|\\
|\childdocforwardprefix{final}{child}|
\end{tabular}
\end{center}
%

Note that when several versions of a main file and/or of each child file
are to be generated, it may be convenient to set up a |Makefile| or
shell script to automatise the process.

%%%%%%%%%%%%%%%%%%%%%%%%%%%%%%%%%%%%%%%%%%%%%%%%%%%%%%%%%%%%%%%%%%%%%%%%%%%%%%%%
\subsection{Command Line Processing}
\label{sec:commandline}

The effect of redirection files can also be achieved by invoking
the \LaTeX{} compiler with a more elaborate command line.
Most conveniently this should be done as part
of a shell script or a |Makefile|.

When using \textsf{childdoc} in the main file, the following
command lines effectively perform a redirection
(note that depending on the shell being used,
backslashes may have to be doubled: `|\|' $\to$ `|\\|'):
%
\begin{center}
|... -jobname "|\textit{target}|" |\\|"|[\textit{flags}]%
|\input{childdoc.def}\childdocforward[|\textit{main}|]{|\textit{dest}|}"|
\end{center}
%
Here \textit{target} is the name of the output file,
\textit{main} is the name of the main file
and \textit{dest} is the name of the main or child file to be processed
(all filenames without extensions).
The optional argument \textit{main} can be omitted
if \textit{main} matches \textit{dest}.
Optionally, compilation \textit{flags} can be defined via |\def| commands.
This command line makes the \TeX{} engine believe
it is compiling the file \textit{target}
whose content is specified as the latter parameter.
The provided code then forwards the processing to
\textit{main} or \textit{dest} as described in \secref{sec:forward}.

%%%%%%%%%%%%%%%%%%%%%%%%%%%%%%%%%%%%%%%%%%%%%%%%%%%%%%%%%%%%%%%%%%%%%%%%%%%%%%%%
\subsection{Include by Input}
\label{sec:input}

Including child documents by |\include| has some restrictions by design.
Most notably, the content of a child document always occupies
its own set of pages; pages cannot be shared between child documents.
Usually, this behaviour makes perfect sense
because each child document contain an essential part of the document.
However, in some situations it may be desirable to compose
a document from a collection of parts
without having mandatory page breaks between then.
For this case, the package
provides a mechanism to include parts
by |\input| which can also be processed individually.
However, by construction this mechanism
requires manual handling of the content to be output.

%%%%%%%%%%%%%%%%%%%%%%%%%%%%%%%%%%%%%%%%
\DescribeMacro{\ifchilddocmanual}
The main file should be prepared as usual, see \secref{sec:include}.
However, the document body must make a distinction
between processing of an individual part and of the main document, e.g.:
%
\begin{center}
\begin{tabular}{l}
|\ifchilddocmanual|\\
|\input{\childdocname}|\\
|\||else|\\
\textit{document body with }|\input{|\textit{part}|}|\\
|\||fi|
\end{tabular}
\end{center}
%
The conditional |\ifchilddocmanual| is true whenever
a part to be included by |\input| is being compiled,
and the name of the part is stored in |\childdocname|.

%%%%%%%%%%%%%%%%%%%%%%%%%%%%%%%%%%%%%%%%
\DescribeMacro{\childdocby}
Each part to be included by |\input| should start with:
%
\begin{center}
\begin{tabular}{l}
|\input{childdoc.def}|\\
|\childdocby{|\textit{main}|}|\\
\end{tabular}
\end{center}
%
The directive |\childdocby| is similar to |\childdocof|
described in \secref{sec:include},
but the subsequent selection of content must be done manually.
To that end, both |\ifchilddoc| and |\ifchilddocmanual|
will be true upon processing of a part,
and the name of the part is stored in |\childdocname|.
Note that |\jobname| will be set to the filename of the current part
so that each part receives an individual |.aux| file
that does not interfere with the |.aux| file(s) of the main document.
This behaviour can be altered by the alternative form
|\childdocby[*]{|\textit{main}|}| (with a non-empty optional argument)
which uses the |.aux| file of the main document
by setting |\jobname| to \textit{main}.

%%%%%%%%%%%%%%%%%%%%%%%%%%%%%%%%%%%%%%%%%%%%%%%%%%%%%%%%%%%%%%%%%%%%%%%%%%%%%%%%
\subsection{Driver Development}
\label{sec:driver}

The \textsf{childdoc} mechanism can also be use for the development
of definition files such as \LaTeX{} styles or classes.
This case differs from the above setup with multiple parts
included by |\include| in that no |\includeonly| should be invoked.
This can be achieved by starting the include file
(before |\ProvidesPackage|) with:
%
\begin{center}
\begin{tabular}{l}
|\input{childdoc.def}|\\
|\childdocforward{|\textit{main}|}|\\
\end{tabular}
\end{center}
%
or alternatively with:
%
\begin{center}
\begin{tabular}{l}
|\input{childdoc.def}|\\
|\childdocby{|\textit{main}|}|\\
\end{tabular}
\end{center}
%
Both forms have slightly different effects as described above.
The main file is prepared as usual, see \secref{sec:include}.

%%%%%%%%%%%%%%%%%%%%%%%%%%%%%%%%%%%%%%%%%%%%%%%%%%%%%%%%%%%%%%%%%%%%%%%%%%%%%%%%
\subsection{Legacy Detection}
\label{sec:detection}

The directive |\childdocmain| in the main file can detect
whether the complete document or merely a child is to be compiled
even without using the directive |\childdocof|.
This method is deprecated because it is less robust
and there is no compelling reason to use it;
it is merely provided for backward compatibility
and it may be removed in future versions.

If the detection mechanism is to be used,
it is mandatory to correctly specify
the filename of the main file as the argument of |\childdocmain|:
%
\begin{center}
\begin{tabular}{l}
|\input{childdoc.def}|\\
|\childdocmain{|\textit{main}|}|\\
\end{tabular}
\end{center}
%
If |\jobname| does not match the argument \textit{main} of |\childdocmain|,
it is assumed that |\jobname| points to the child file to be compiled.
When using |\childdocmain| with the main file specified as argument,
it suffices to start a child file
with just |\input{|\textit{main}|}|
without loading of the package and using |\childdocof|.
If instead all processing is done
with the appropriate \textsf{childdoc} directives,
the argument of \textit{main} of |\childdocmain| can be empty.

An alternative version of the command line processing described
in \secref{sec:commandline} using the detection mechanism reads:
%
\begin{center}
|... -jobname "|\textit{target}|" "|[\textit{flags}]%
[|\def\jobname{|\textit{dest}|}|]|\input{|\textit{main}|}"|
\end{center}

%%%%%%%%%%%%%%%%%%%%%%%%%%%%%%%%%%%%%%%%%%%%%%%%%%%%%%%%%%%%%%%%%%%%%%%%%%%%%%%%
\subsection{Manual Code}
\label{sec:manual}

In case one cannot be certain whether the definitions file |childdoc.def|
is installed on the target \TeX{} distribution
and one prefers not to ship it,
it is conceivable to paste a few relevant commands into the sources.

To that end, drop all statements |\input{childdoc.def}|
and perform the replacements as outlined below.
Instead of |\childdocmain{|\textit{main}|}| add the following code
to the top of the main file:
%
\begin{center}
\begin{tabular}{l}
|\||ifdefined\childdocname\endinput\||fi\newif\ifchilddoc|\\
|\edef\childdocname{\scantokens\expandafter{\jobname\noexpand}}|\\
|\def\childdocmain{|\textit{main}|}\||ifx\childdocmain\childdocname\||else|\\
|\childdoctrue\includeonly{\childdocname}\let\jobname\childdocmain\||fi|\\
\end{tabular}
\end{center}
%
Instead of |\childdocof{|\textit{main}|}| just include the main file
at the top of each child file:
%
\begin{center}
|\input{|\textit{main}|}|
\end{center}
%
A simple redirection |\childdocforward{|\textit{dest}|}| is achieved by:
%
\begin{center}
|\def\jobname{|\textit{dest}|}\input{\jobname}|
\end{center}
%
The redirection with prefix
|\childdocforwardprefix[|\textit{prefix}|]{|\textit{dest}|}|
is accomplished by:
%
\begin{center}
\begin{tabular}{l}
|{\edef\jobname{\scantokens\expandafter{\jobname\noexpand}}|\\
|\def\redirectjob |\textit{prefix}|#1~~~{\gdef\jobname{|\textit{dest}|#1}}|\\
|\expandafter\redirectjob\jobname~~~}\input{\jobname}|
\end{tabular}
\end{center}

In an alternative approach,
child documents can be compiled by a specific command line
without additional code or specific definitions:
%
\begin{center}
|... -jobname "|\textit{target}|" "|[\textit{flags}]%
|\includeonly{|\textit{dest}|}\input{|\textit{main}|}"|
\end{center}
%

%%%%%%%%%%%%%%%%%%%%%%%%%%%%%%%%%%%%%%%%%%%%%%%%%%%%%%%%%%%%%%%%%%%%%%%%%%%%%%%%
%%%%%%%%%%%%%%%%%%%%%%%%%%%%%%%%%%%%%%%%%%%%%%%%%%%%%%%%%%%%%%%%%%%%%%%%%%%%%%%%
\section{Information}

%%%%%%%%%%%%%%%%%%%%%%%%%%%%%%%%%%%%%%%%%%%%%%%%%%%%%%%%%%%%%%%%%%%%%%%%%%%%%%%%
\subsection{Copyright}

Copyright \copyright{} 2017--2018 Niklas Beisert

This work may be distributed and/or modified under the
conditions of the \LaTeX{} Project Public License, either version 1.3
of this license or (at your option) any later version.
The latest version of this license is in
  \url{http://www.latex-project.org/lppl.txt}
and version 1.3 or later is part of all distributions of \LaTeX{}
version 2005/12/01 or later.

This work has the LPPL maintenance status `maintained'.

The Current Maintainer of this work is Niklas Beisert.

This work consists of the files |README.txt|, |childdoc.ins| and |childdoc.dtx|
as well as the derived files |childdoc.def|, |cdocsamp.tex|
with |cdocsch1.tex|, |cdocsch2.tex|, |cdocspt3.tex|, |cdocspt4.tex|,
|cdocsdrf.tex|, |cdocsfn1.tex|, |cdocsfn2.tex|
as well as |childdoc.pdf|.

%%%%%%%%%%%%%%%%%%%%%%%%%%%%%%%%%%%%%%%%%%%%%%%%%%%%%%%%%%%%%%%%%%%%%%%%%%%%%%%%
\subsection{Files and Installation}

The package consists of the files:
%
\begin{center}
\begin{tabular}{ll}
    |README.txt|   & readme file \\
    |childdoc.ins| & installation file \\
    |childdoc.dtx| & source file \\
    |childdoc.def| & definition file \\
    |cdocsamp.tex| & sample main file \\
    |cdocsch1.tex| & sample include file \\
    |cdocsch2.tex| & sample include file \\
    |cdocspt3.tex| & sample part file \\
    |cdocspt4.tex| & sample part file \\
    |cdocsdrf.tex| & sample redirection file \\
    |cdocsfn1.tex| & sample redirection file \\
    |cdocsfn2.tex| & sample redirection file \\
    |childdoc.pdf| & manual
\end{tabular}
\end{center}
%
The distribution consists of the files
|README.txt|, |childdoc.ins| and |childdoc.dtx|.
%
\begin{itemize}
\item
Run (pdf)\LaTeX{} on |childdoc.dtx|
to compile the manual |childdoc.pdf| (this file).
\item
Run \LaTeX{} on |childdoc.ins| to create the definitions file |childdoc.def|
and the sample |cdocsamp.tex| with include files
|cdocsch1.tex|, |cdocsch2.tex|, |cdocspt3.tex|, |cdocspt4.tex|,
|cdocsdrf.tex|, |cdocsfn1.tex|, |cdocsfn2.tex|.
Then copy the file |childdoc.def| to an appropriate directory of your \LaTeX{}
distribution, e.g.\ \textit{texmf-root}|/tex/latex/childdoc|.
\end{itemize}

%%%%%%%%%%%%%%%%%%%%%%%%%%%%%%%%%%%%%%%%%%%%%%%%%%%%%%%%%%%%%%%%%%%%%%%%%%%%%%%%
\subsection{Related CTAN Packages}

There are several other packages which offer a similar functionality:
%
\begin{itemize}
\item
The packages
\href{http://ctan.org/pkg/docmute}{\textsf{docmute}},
\href{http://ctan.org/pkg/includex}{\textsf{includex}} and
\href{http://ctan.org/pkg/standalone}{\textsf{standalone}}
provide commands to include only the document body of
a child file thus allowing both files to be compiled individually.
\item
The packages \href{http://ctan.org/pkg/subdocs}{\textsf{subdocs}}
and \href{http://ctan.org/pkg/subfiles}{\textsf{subfiles}}
provide structures in which the main and child documents can be
encapsulated and allowing them to be compiled individually.
The inclusion mechanism is different from the conventional |\include|.
\item
The package \href{http://ctan.org/pkg/combine}{\textsf{combine}}
is an elaborate solution to combine several documents into one.
\end{itemize}
%
See also the CTAN topic \href{http://ctan.org/topic/subdocs}{\textsf{subdocs}}
for further related packages.
The present package differs from the above solutions in that
a document structure constructed with the conventional |\include| mechanism
just needs two extra commands at the top of every file
such that all constituent files can be compiled individually.

%%%%%%%%%%%%%%%%%%%%%%%%%%%%%%%%%%%%%%%%%%%%%%%%%%%%%%%%%%%%%%%%%%%%%%%%%%%%%%%%
%\subsection{Feature Suggestions}
%
%The following is a list of features which may be useful for future
%versions of this package:
%%
%\begin{itemize}
%\item
%\ldots
%\end{itemize}

%%%%%%%%%%%%%%%%%%%%%%%%%%%%%%%%%%%%%%%%%%%%%%%%%%%%%%%%%%%%%%%%%%%%%%%%%%%%%%%%
\subsection{Revision History}

%%%%%%%%%%%%%%%%%%%%%%%%%%%%%%%%%%%%%%%%
\paragraph{v2.0:} 2018/12/30

\begin{itemize}
\item
immediate forward processing
\item
added |\childdocby| mechanism
\item
manual restructured
\end{itemize}

%%%%%%%%%%%%%%%%%%%%%%%%%%%%%%%%%%%%%%%%
\paragraph{v1.6:} 2018/01/17

\begin{itemize}
\item
application for development of include files
\item
corrections to manual
\end{itemize}

%%%%%%%%%%%%%%%%%%%%%%%%%%%%%%%%%%%%%%%%
\paragraph{v1.5:} 2017/05/21

\begin{itemize}
\item
more complete structuring introduced
\item
|\childdocof| introduced
\item
|\childdoc| renamed to |\childdocmain|
\item
|\childredirect| renamed to |\childdocforward| and |\childdocforwardprefix|
and functionality expanded
\end{itemize}

%%%%%%%%%%%%%%%%%%%%%%%%%%%%%%%%%%%%%%%%
\paragraph{v1.0:} 2017/04/27

\begin{itemize}
\item
manual and install package
\item
first version published on CTAN
\end{itemize}

%%%%%%%%%%%%%%%%%%%%%%%%%%%%%%%%%%%%%%%%
\paragraph{v0.6:} 2017/04/26

\begin{itemize}
\item
redirection mechanism added
\end{itemize}

%%%%%%%%%%%%%%%%%%%%%%%%%%%%%%%%%%%%%%%%
\paragraph{v0.5:} 2017/04/26

\begin{itemize}
\item
functionality in definition file
\end{itemize}


%%%%%%%%%%%%%%%%%%%%%%%%%%%%%%%%%%%%%%%%%%%%%%%%%%%%%%%%%%%%%%%%%%%%%%%%%%%%%%%%
%%%%%%%%%%%%%%%%%%%%%%%%%%%%%%%%%%%%%%%%%%%%%%%%%%%%%%%%%%%%%%%%%%%%%%%%%%%%%%%%
%%%%%%%%%%%%%%%%%%%%%%%%%%%%%%%%%%%%%%%%%%%%%%%%%%%%%%%%%%%%%%%%%%%%%%%%%%%%%%%%
\appendix

\settowidth\MacroIndent{\rmfamily\scriptsize 000\ }

 \DocInput{childdoc.dtx}

\end{document}
%</driver>
% \fi
%
% %%%%%%%%%%%%%%%%%%%%%%%%%%%%%%%%%%%%%%%%%%%%%%%%%%%%%%%%%%%%%%%%%%%%%%%%%%%%%%
% %%%%%%%%%%%%%%%%%%%%%%%%%%%%%%%%%%%%%%%%%%%%%%%%%%%%%%%%%%%%%%%%%%%%%%%%%%%%%%
% \section{Sample}
%\iffalse
%<*samplemain>
%\fi
%
% The following presents a sample document
% with two chapters, two parts, a title page,
% a compile flag as well as three forwarding files to set the flag.
% It consists of eight |.tex| files:
% \begin{center}
% \begin{tabular}{ll}
% |cdocsamp.tex|&main file\\
% |cdocsch1.tex|&include file for chapter 1\\
% |cdocsch2.tex|&include file for chapter 2\\
% |cdocspt3.tex|&include file for part 3\\
% |cdocspt4.tex|&include file for part 4\\
% |cdocsdrf.tex|&forwarding file for main file in draft mode\\
% |cdocsfi1.tex|&forwarding file for final version of chapter 1\\
% |cdocsfi2.tex|&forwarding file for final version of chapter 2\\
% \end{tabular}
% \end{center}
% Each of the eight files can be compiled directly by the \LaTeX{} compiler.
%
% %%%%%%%%%%%%%%%%%%%%%%%%%%%%%%%%%%%%%%
% \paragraph{Main File.}
%
% The main file is called |cdocsamp.tex|.
%
% Load the \textsf{childdoc} definitions and
% declare the filename for the main document:
%    \begin{macrocode}
\input{childdoc.def}
\childdocmain{}
%    \end{macrocode}

% Optional override for |\version| flag:
%    \begin{macrocode}
%%\ifchilddoc\else\providecommand{\version}{draft}\fi
%    \end{macrocode}

% Define the default values for the |\version| flag
% (|final| for the main file and |draft| for childs):
%    \begin{macrocode}
\ifchilddoc
\providecommand{\version}{draft}
\else
\providecommand{\version}{final}
\fi
%    \end{macrocode}

% Load the standard document class:
%    \begin{macrocode}
\documentclass[12pt]{article}
%    \end{macrocode}

% Start the document body:
%    \begin{macrocode}
\begin{document}
%    \end{macrocode}

% Declare a title page.
% Print title, part of document being processed and version flag:
%    \begin{macrocode}
\addtocounter{page}{-1}
\begin{center}
{\LARGE\bfseries{}childdoc example\par}
\vspace{1cm}
\ifchilddoc
\ifchilddocmanual part\else chapter\fi:
`\childdocname' of `\childdocjob'\par
\else
main document: `\childdocjob'\par
\fi
version: \version\par
\end{center}
\newpage
%    \end{macrocode}

% Manually include selected file,
% otherwise process as usual:
%    \begin{macrocode}
\ifchilddocmanual
\section*{part `\childdocname'}
\input{\childdocname}
\else
%    \end{macrocode}

% Include the two chapters:
%    \begin{macrocode}
\include{cdocsch1}
\include{cdocsch2}
%    \end{macrocode}

% Include the two parts unless only chapters should be displayed:
%    \begin{macrocode}
\ifchilddoc\else
\section{part three}
\input{cdocspt3}
\section{part four}
\input{cdocspt4}
\fi
%    \end{macrocode}

% Process as usual until here:
%    \begin{macrocode}
\fi
%    \end{macrocode}

% End of document body:
%    \begin{macrocode}
\end{document}
%    \end{macrocode}
%\iffalse
%</samplemain>
%\fi
%
% %%%%%%%%%%%%%%%%%%%%%%%%%%%%%%%%%%%%%%
% \paragraph{Chapter Include Files.}
%
% The include files are called |cdocsch1.tex| and |cdocsch2.tex|.
%
%\iffalse
%<*samplechap1|samplechap2>
%\fi

% Optional override for |\version| flag:
%    \begin{macrocode}
%%\providecommand{\version}{final}
%    \end{macrocode}

% Include the main document:
%    \begin{macrocode}
\input{childdoc.def}
\childdocof{cdocsamp}
%    \end{macrocode}

%\iffalse
%</samplechap1|samplechap2>
%\fi
%
%\iffalse
%<*samplechap1>
%\fi
% Some text for chapter 1:
%    \begin{macrocode}
\section{one}
some text in chapter one
%    \end{macrocode}

%\iffalse
%</samplechap1>
%\fi
% Some text for chapter 2:
%\iffalse
%<*samplechap2>
%\fi
%    \begin{macrocode}
\section{two}
more text in chapter two
%    \end{macrocode}

%\iffalse
%</samplechap2>
%\fi
%
% %%%%%%%%%%%%%%%%%%%%%%%%%%%%%%%%%%%%%%
% \paragraph{Part Include Files.}
%
% The include files are called |cdocspt3.tex| and |cdocspt4.tex|.
%
%\iffalse
%<*samplepart3|samplepart4>
%\fi

% Optional override for |\version| flag:
%    \begin{macrocode}
%%\providecommand{\version}{final}
%    \end{macrocode}

% Include the main document:
%    \begin{macrocode}
\input{childdoc.def}
\childdocby{cdocsamp}
%    \end{macrocode}

%\iffalse
%</samplepart3|samplepart4>
%\fi
%
%\iffalse
%<*samplepart3>
%\fi
% Some text for part 3:
%    \begin{macrocode}
some text in part three
%    \end{macrocode}

%\iffalse
%</samplepart3>
%\fi
% Some text for part 4:
%\iffalse
%<*samplepart4>
%\fi
%    \begin{macrocode}
more text in part four
%    \end{macrocode}

%\iffalse
%</samplepart4>
%\fi
%
% %%%%%%%%%%%%%%%%%%%%%%%%%%%%%%%%%%%%%%
% \paragraph{Forwarding for a Complete Draft.}
%
% The following forwarding file |cdocsdrf.tex|
% compiles the main document in draft mode:
%\iffalse
%<*sampledraft>
%\fi
%    \begin{macrocode}
\def\version{draft}
\input{childdoc.def}
\childdocforward{cdocsamp}
%    \end{macrocode}

%\iffalse
%</sampledraft>
%\fi
%
% %%%%%%%%%%%%%%%%%%%%%%%%%%%%%%%%%%%%%%
% \paragraph{Forwarding for Final Version of the Chapters.}
%
% The following forwarding files |cdocsfn1.tex| and |cdocsfn2.tex|
% (with identical content)
% compile the final versions of the child documents
% |cdocsch1.tex| and |cdocsch2.tex|, respectively:
%\iffalse
%<*samplefinal>
%\fi
%    \begin{macrocode}
\def\version{final}
\input{childdoc.def}
\childdocforwardprefix[cdocsamp]{cdocsfn}{cdocsch}
%    \end{macrocode}

%\iffalse
%</samplefinal>
%\fi
%
% %%%%%%%%%%%%%%%%%%%%%%%%%%%%%%%%%%%%%%
% \paragraph{Command Line Processing.}
%
% The following three command lines generate the output files
% |cdocscld|, |cdocscl1| and |cdocscl2|
% which should be identical to
% |cdocsdrf|, |cdocsch1| and |cdocsfn2|, respectively:
% \begin{center}
% \begin{tabular}{l}
% |latex -jobname cdocscld \|\\
% |  "\def\version{draft}\input{childdoc.def}\childdocforward{cdocsamp}"|\\
% |latex -jobname cdocscl1 \|\\
% |  "\input{childdoc.def}\childdocforward[cdocsamp]{cdocsch1}"|\\
% |latex -jobname cdocscl2 \|\\
% |  "\def\version{final}\input{childdoc.def}\childdocforward{cdocsch2}"|
% \end{tabular}
% \end{center}
% Note that the trailing backslash on each first line
% merely continues the input to the second line
% (for convenient cut ant paste).
% Furthermore, the command |latex| can be replaced by any
% of its alternative versions such as |pdflatex|.
%
% %%%%%%%%%%%%%%%%%%%%%%%%%%%%%%%%%%%%%%%%%%%%%%%%%%%%%%%%%%%%%%%%%%%%%%%%%%%%%%
% %%%%%%%%%%%%%%%%%%%%%%%%%%%%%%%%%%%%%%%%%%%%%%%%%%%%%%%%%%%%%%%%%%%%%%%%%%%%%%
% \section{Implementation}
%\iffalse
%<*package>
%\fi
%
% This section describes the definitions file |childdoc.def|.

% The definitions cannot be loaded using |\usepackage| or |\RequirePackage|
% which has a mechanism to prevent loading a style file more than once.
% When loading the definitions by means of |\input|
% multiple instances have to be prevented manually:
%\iffalse
%This code needs to be before the `\ProvidesFile' directive
%which is defined at the beginning of this file.
%Therefore it is also placed there and commented out here.
%</package>
%<*discard>
%\fi
%    \begin{macrocode}
\ifdefined\childdocmain\endinput\fi
%    \end{macrocode}
%\iffalse
%</discard>
%<*package>
%\fi
%
% \macro{\ifchilddoc}
% \macro{\ifchilddocmanual}
% The conditional |\ifchilddoc| tells whether a
% child (true) or main (false) document is being compiled.
% The conditional |\ifchilddocmanual| tells whether
% the |\includeonly| mechanism is used (false) or
% the selection of child files must be performed manually (true).
% The definitions initialise to false:
%    \begin{macrocode}
\newif\ifchilddoc
\newif\ifchilddocmanual
%    \end{macrocode}

% \macro{\childdocname}
% \macro{\childdocjob}
% The macro |\childdocname| stores the name of the main document
% to be compiled. The macro |\childdocjob| stores the name of
% the document on which the \LaTeX{} compiler was originally invoked.
% The content of |\jobname| cannot be compared
% to filenames specified in the source due to different catcodes.
% The following code rescans |\jobname|, stores the result
% in |\childdocname| and saves a copy in |\childdocjob|:
%    \begin{macrocode}
\edef\childdocname{\scantokens\expandafter{\jobname\noexpand}}
\let\childdocjob\childdocname
%    \end{macrocode}

% \macro{\childdocdisable}
% The macro |\childdocdisable| prevents the main file
% from being processed more than once.
% At this stage, the main document command |\childdocmain|
% is assumed to be called once again where it should do nothing.
% Any subsequent call to it should prevent
% a secondary processing of the main document
% It overwrites the forwarding commands
% |\childdocof| and |\childdocforward|
% with empty macros to prevent further inclusions of the main document:
%    \begin{macrocode}
\newcommand{\childdocdisable}
{
  \renewcommand{\childdocmain}[1]{\renewcommand{\childdocmain}[1]{\endinput}}
  \renewcommand{\childdocof}[1]{}
  \renewcommand{\childdocby}[2][]{}
  \renewcommand{\childdocforward}[2][]{}
  \renewcommand{\childdocdisable}{}
}
%    \end{macrocode}

% \macro{\childdocmain}
% The macro |\childdocmain| is to be called at the top of the main file
% with nothing or the main filename (without extension) as argument.
% First, it breaks loops.
% If the argument is not empty and does not match |\childdocname|
% (which is set by the first inclusion of |childdoc.def|),
% |\ifchilddoc| is set to true, |\includeonly| is applied to the child file
% and |\jobname| is set to the main file
% (for proper handling of |.aux| files):
%    \begin{macrocode}
\newcommand{\childdocmain}[1]
{
  \childdocdisable\childdocmain{}
  \if?#1?\else
    \begingroup
      \def\childdoctmp{#1}
      \ifx\childdoctmp\childdocname
        \def\childdoctmp{}
      \else
        \def\childdoctmp
        {
          \childdoctrue
          \includeonly{\childdocname}
          \def\childdocjob{#1}
          \def\jobname{#1}
        }
      \fi
      \expandafter
    \endgroup
    \childdoctmp
  \fi
}
%    \end{macrocode}

% \macro{\childdocof}
% The command |\childdocof| redirects
% compilation to the main file |#1|.
%    \begin{macrocode}
\newcommand{\childdocof}[1]
{
  \childdocdisable
  \childdoctrue
  \includeonly{\childdocname}
  \def\jobname{#1}
  \def\childdocjob{#1}
  \input{#1}
}
%    \end{macrocode}

% \macro{\childdocby}
% The command |\childdocby| ....
%    \begin{macrocode}
\newcommand{\childdocby}[2][]
{
  \childdocdisable
  \childdoctrue
  \childdocmanualtrue
  \if?#1?\else
    \def\jobname{#2}
  \fi
  \def\childdocjob{#2}
  \input{#2}
  \endinput
}
%    \end{macrocode}

% \macro{\childdocforward}
% The command |\childdocforward| redirects
% compilation to the main file or
% (if the optional argument is given) a child file.
% Parameters are set as if the main file
% or a child file starting with |\childdocof| was compiled.
% Then compilation is handed over to the main file:
%    \begin{macrocode}
\newcommand{\childdocforward}[2][]
{
  \begingroup
    \if?#1?
      \def\childdoctmp
      {
        \def\childdocname{#2}
        \def\childdocjob{#2}
        \def\jobname{#2}
        \input{#2}
        \endinput
      }
    \else
      \def\childdoctmp
      {
        \childdocdisable
        \def\childdocname{#2}
        \childdoctrue
        \includeonly{#2}
        \def\childdocjob{#1}
        \def\jobname{#1}
        \input{#1}
        \endinput
      }
    \fi
    \expandafter
  \endgroup
  \childdoctmp
}
%    \end{macrocode}

% \macro{\childdocforwardprefix}
% The command |\childdocforwardprefix| redirects
% compilation to the main or a child file by means of a pattern.
% The prefix |#1| in the current filename is replaced by |#2|
% and the suffix of the current filename is kept
% (it is assumed that the filename does not contain the substring `|~~~|'
% which is used as a delimiter).
% Compilation is handed over to the new file by |\childdocforward|:
%    \begin{macrocode}
\newcommand{\childdocforwardprefix}[3][]
{
  \begingroup
    \def\childdocextract #2##1~~~{\def\childdoctmp{\childdocforward[#1]{#3##1}}}
    \expandafter\childdocextract\childdocname~~~
    \expandafter
  \endgroup
  \childdoctmp
}
%    \end{macrocode}

% \macro{\childdoc}
% The deprecated macro |\childdoc| is a legacy version of |\childdocmain|:
%    \begin{macrocode}
\newcommand{\childdoc}{\childdocmain}
%    \end{macrocode}

% \macro{\childdocredirect}
% The deprecated macro |\childdocredirect| is a legacy version
% of |\childdocforward| and |\childdocforwardprefix|:
%    \begin{macrocode}
\newcommand{\childdocredirect}[2][]
{
  \begingroup
    \if?#1?
      \def\childdoctmp{\childdocforward{#2}}
    \else
      \def\childdoctmp{\childdocforwardprefix{#1}{#2}}
    \fi
    \expandafter
  \endgroup
  \childdoctmp
}
%    \end{macrocode}

%\iffalse
%</package>
%\fi
%
\endinput
|\\
|\childdocforward{|\textit{main}|}|
\end{tabular}
\end{center}
%
Likewise, the following files |final|\textit{nn}|.tex|
compile the final version of the child document
|child|\textit{nn}|.tex|:
%
\begin{center}
\begin{tabular}{l}
|\def\version{final}|\\
|% \iffalse
%
% childdoc.dtx Copyright (C) 2017-2018 Niklas Beisert
%
% This work may be distributed and/or modified under the
% conditions of the LaTeX Project Public License, either version 1.3
% of this license or (at your option) any later version.
% The latest version of this license is in
%   http://www.latex-project.org/lppl.txt
% and version 1.3 or later is part of all distributions of LaTeX
% version 2005/12/01 or later.
%
% This work has the LPPL maintenance status `maintained'.
%
% The Current Maintainer of this work is Niklas Beisert.
%
% This work consists of the files childdoc.dtx and childdoc.ins
% and the derived files childdoc.def and cdocsamp.tex with
% cdocsch1.tex, cdocsch2.tex, cdocsdrf.tex, cdocsfn1.tex, cdocsfn2.tex.
%
%<package>\ifdefined\childdocmain\endinput\fi
%<package>\ProvidesFile{childdoc.def}[2018/12/30 v2.0 child document driver]
%<samplemain>\ProvidesFile{cdocsamp.tex}[2018/12/30 v2.0 sample for childdoc]
%<*driver>
%\ProvidesFile{childdoc.drv}[2018/12/30 v2.0 childdoc reference manual file]
\PassOptionsToClass{10pt,a4paper}{article}
\documentclass{ltxdoc}

\usepackage[margin=35mm]{geometry}
\usepackage{hyperref}
\usepackage{hyperxmp}
\usepackage[usenames]{color}

\hypersetup{colorlinks=true}
\hypersetup{pdfstartview=FitH}
\hypersetup{pdfpagemode=UseNone}
\hypersetup{pdfsource={}}
\hypersetup{pdflang={en-UK}}
\hypersetup{pdfcopyright={Copyright 2017-2018 Niklas Beisert.
  This work may be distributed and/or modified under the
  conditions of the LaTeX Project Public License, either version 1.3
  of this license or (at your option) any later version.}}
\hypersetup{pdflicenseurl={http://www.latex-project.org/lppl.txt}}
\hypersetup{pdfcontactaddress={ETH Zurich, ITP, HIT K,
  Wolfgang-Pauli-Strasse 27}}
\hypersetup{pdfcontactpostcode={8093}}
\hypersetup{pdfcontactcity={Zurich}}
\hypersetup{pdfcontactcountry={Switzerland}}
\hypersetup{pdfcontactemail={nbeisert@itp.phys.ethz.ch}}
\hypersetup{pdfcontacturl={http://people.phys.ethz.ch/\xmptilde nbeisert/}}

\newcommand{\secref}[1]{\hyperref[#1]{section \ref*{#1}}}

\parskip1ex
\parindent0pt
\let\olditemize\itemize
\def\itemize{\olditemize\parskip0pt}

\begin{document}

\title{The \textsf{childdoc} Package}
\hypersetup{pdftitle={The childdoc Package}}
\author{Niklas Beisert\\[2ex]
  Institut f\"ur Theoretische Physik\\
  Eidgen\"ossische Technische Hochschule Z\"urich\\
  Wolfgang-Pauli-Strasse 27, 8093 Z\"urich, Switzerland\\[1ex]
  \href{mailto:nbeisert@itp.phys.ethz.ch}
  {\texttt{nbeisert@itp.phys.ethz.ch}}}
\hypersetup{pdfauthor={Niklas Beisert}}
\hypersetup{pdfsubject={Manual for the LaTeX2e Package childdoc}}
\date{30 December 2018, \textsf{v2.0}}
\maketitle

\begin{abstract}\noindent
\textsf{childdoc} is a \LaTeXe{} package
that enables the direct compilation
of document sections included by |\include|
to individual files.
\end{abstract}

\begingroup
\parskip0ex
\tableofcontents
\endgroup

%%%%%%%%%%%%%%%%%%%%%%%%%%%%%%%%%%%%%%%%%%%%%%%%%%%%%%%%%%%%%%%%%%%%%%%%%%%%%%%%
%%%%%%%%%%%%%%%%%%%%%%%%%%%%%%%%%%%%%%%%%%%%%%%%%%%%%%%%%%%%%%%%%%%%%%%%%%%%%%%%
\section{Introduction}

\LaTeX{} provides a mechanism to structure a large document (such as a book)
into a main file and several child files (containing the chapters)
using the |\include| command.
This mechanism is beneficial for documents
which span hundreds of pages in order to
make the source file(s) more manageable.
Moreover, compilation can be restricted to
selected child files by means of the |\includeonly| command.
The latter feature can be used to reduce the compilation time while editing
(this was significantly more useful in the earlier days of \LaTeX{})
or to generate a smaller document which is easier to navigate.
Another application of |\includeonly| is to generate
documents consisting of selected parts of the complete document.

However, there are a few drawbacks of the plain |\include| mechanism:
\begin{itemize}
\item
The child files cannot be compiled on their own,
they can only be compiled via the main file.
A naive editing environment
(such as a text editor with an option
to have the current file processed by \LaTeX)
may require one to switch to the main file before compiling;
attempting to compile the child file produces errors.
\item
The main file must be modified (each time)
to adjust the |\includeonly| command
to the present needs. This easily leaves the main file in a messy state.
\item
The generated document will always carry the filename
of the main document. This is inconvenient if
several child files are to be compiled and
to be kept for distribution.
\end{itemize}

The present package provides a simple interface
to make child files individually compilable by \LaTeX{}.
Compiling a child file then has the same effect as compiling
the main file with an |\includeonly| command
to select the appropriate child.
Moreover the generated document will carry the name of the child
rather than the main file.
This resolves all three above issues.

This feature is meant to make the editing of books,
thesis documents and lecture notes somewhat more convenient.
However, the package can also be used efficiently for
composing a series of documents (such as exercise sheets)
which are typically distributed individually.
It then assists the author in generating the individual documents
(potentially in different versions)
as well as a document containing the collected series.
Another application is in developing style files
or other kinds of included material
where compilation of the style file could redirect
to a sample or test file.

%%%%%%%%%%%%%%%%%%%%%%%%%%%%%%%%%%%%%%%%%%%%%%%%%%%%%%%%%%%%%%%%%%%%%%%%%%%%%%%%
%%%%%%%%%%%%%%%%%%%%%%%%%%%%%%%%%%%%%%%%%%%%%%%%%%%%%%%%%%%%%%%%%%%%%%%%%%%%%%%%
\section{Usage}

First of all, the package \textsf{childdoc} is \emph{not} a standard
\LaTeXe{} |.sty| style file! Therefore it needs to be invoked in
a non-standard way.

%%%%%%%%%%%%%%%%%%%%%%%%%%%%%%%%%%%%%%%%%%%%%%%%%%%%%%%%%%%%%%%%%%%%%%%%%%%%%%%%
\subsection{Included Files}
\label{sec:include}

%%%%%%%%%%%%%%%%%%%%%%%%%%%%%%%%%%%%%%%%
\DescribeMacro{\childdocmain}
To use the package, add the commands
\begin{center}
\begin{tabular}{l}
|\input{childdoc.def}|\\
|\childdocmain{}|\\
\end{tabular}
\end{center}
at the very top of the main \LaTeX{} file,
in particular \emph{before} the |\documentclass| statement!
The argument of |\childdocmain| should be left empty
(but it must be present).

%%%%%%%%%%%%%%%%%%%%%%%%%%%%%%%%%%%%%%%%
\DescribeMacro{\childdocof}
Furthermore, add the commands
\begin{center}
\begin{tabular}{l}
|\input{childdoc.def}|\\
|\childdocof{|\textit{main}|}|\\
\end{tabular}
\end{center}
at the top of every child file \textit{child}
which is included by |\include{|\textit{child}|}|
from within the main file
(or at least for those files to be compiled individually).
The argument \textit{main} must be the filename of the main file.

There are a couple of
considerations in setting up the main and child documents:

%%%%%%%%%%%%%%%%%%%%%%%%%%%%%%%%%%%%%%%%
\paragraph{Restrictions.}

Please note the following restrictions:
\begin{itemize}
\item
|\childdocmain| must be called with one argument \textit{main}
to ensure compatibility with earlier version of the package.
It must either be empty (|\childdocmain{}|)
or precisely match the filename of the main file in which it is specified.
See \secref{sec:detection} for further information.
\item
The filename \textit{main} must be specified without the |.tex| extension.
\item
The filename \textit{main} is case sensitive
(even in case-insensitive file systems)
due to internal string comparison.
\item
The argument \textit{main} should be fully expanded, it cannot be a macro.
\item
Subdirectories and special characters should be avoided in filenames.
\item
The command |\childdocmain{|\textit{main}|}| must be followed by a whitespace.
It should not be followed immediately by another command
or by a comment mark `|%|'.
This is because the \TeX{} parser reads the token immediately following
the argument of |\childdocmain| and puts it
at the beginning of every child section;
however, a white\-space is ignored.
\end{itemize}

%%%%%%%%%%%%%%%%%%%%%%%%%%%%%%%%%%%%%%%%
\paragraph{Content of Main File.}

It is advisable to place all content in the child files included by |\include|.
Any output contained in the main file will appear in all child documents
unless suppressed manually;
it cannot be suppressed automatically by the |\includeonly| directive
and thus should normally be avoided.
A method to include some content in the main file
by means of conditional processing is described in \secref{sec:conditional}.

%%%%%%%%%%%%%%%%%%%%%%%%%%%%%%%%%%%%%%%%
\paragraph{Page Numbering.}

When only a part of the document is compiled,
the appropriate numbering of pages
(as well as other status parameters)
is determined from the |.aux| files.
The latter contain information from previous passes.
However this information needs to propagate through
all intermediate child documents.
Therefore the page numbering in child documents may well
be inconsistent until the complete document is compiled at least once.

A useful (if unconventional) way to always ensure a consistent
page numbering is to restart the numbering in each child document
and denote the pages by `\textit{child}|.|\textit{page}'
where \textit{child} represents the chapter/section number of the child file.
This can be achieved by the command
|\numberwithin{page}{|\textit{child}|}|
of the \textsf{amsmath} package
where \textit{child} can be |chapter| or |section|
depending on the chosen structuring.
Alternatively, one can modify the macro |\thepage| appropriately
and reset the counter |page| at the start of each child file.

%%%%%%%%%%%%%%%%%%%%%%%%%%%%%%%%%%%%%%%%%%%%%%%%%%%%%%%%%%%%%%%%%%%%%%%%%%%%%%%%
\subsection{Conditional Processing}
\label{sec:conditional}

The package provides a mechanism to compile different versions
of a document. To customise the versions further some conditional processing
can come in handy to distinguish which version is being compiled.
The package provides two macros to describe the compilation context:

%%%%%%%%%%%%%%%%%%%%%%%%%%%%%%%%%%%%%%%%
\DescribeMacro{\ifchilddoc}
The conditional |\ifchilddoc| distinguishes between the compilation of
child documents and the main document:
%
\begin{center}
|\ifchilddoc |\textit{child-code}| |[|\||else |\textit{main-code}]| \||fi|
\end{center}

%%%%%%%%%%%%%%%%%%%%%%%%%%%%%%%%%%%%%%%%
\DescribeMacro{\childdocname}
\DescribeMacro{\childdocjob}
The macro |\childdocname| contains the filename (without extension)
of the main or child file being processed.
Note that |\childdocjob| will always contain the name of the main file.

%%%%%%%%%%%%%%%%%%%%%%%%%%%%%%%%%%%%%%%%
\paragraph{Title Page.}

Conditional processing can be used to include a title or banner page
in the main document when proper precautions are taken.
Importantly, the code in the main file should ensure that the page counter
(as well as other status parameters which are stored in the |.aux| files)
takes the same value after the conditional processing.
Otherwise the page numbers may take divergent values
depending on which part is compiled.

For example, a title page could be declared by:
%
\begin{center}
\begin{tabular}{l}
|\ifchilddoc\||else|\\
|\addtocounter{page}{-1}|\\
\textit{code for title page}\\
|\newpage|\\
|\||fi|
\end{tabular}
\end{center}
%
A banner page for the child documents can be generated by:
%
\begin{center}
\begin{tabular}{l}
|\ifchilddoc|\\
|\addtocounter{page}{-1}|\\
\textit{code for banner page}\\
|\newpage|\\
|\||fi|
\end{tabular}
\end{center}
%
Here one could write a message such as:
\begin{center}
|This is the part \childdocname{} of \childdocjob{}.|
\end{center}

%%%%%%%%%%%%%%%%%%%%%%%%%%%%%%%%%%%%%%%%%%%%%%%%%%%%%%%%%%%%%%%%%%%%%%%%%%%%%%%%
\subsection{Flags}
\label{sec:flags}

The package makes it easy to generate different versions
of the main or child documents.
To this end compilation flags can be defined
and assigned different default values.
They will be particularly useful in conjunction
with the forwarding mechanism described in \secref{sec:forward}.

For example, it may be useful to have a flag |\version|
which can be set to |draft| or |final|.
The document source will contain some conditional code
depending on the value of |\version|.
Suppose further, the flag should default to |final| for the main file
and to |draft| for child files
which is a natural assignment for editing the document.
This is achieved by placing the following code
in the preamble of the main document
(below the |\childdocmain| directive):
%
\begin{center}
\begin{tabular}{l}
|\ifchilddoc|\\
|\providecommand{\version}{draft}|\\
|\||else|\\
|\providecommand{\version}{final}|\\
|\||fi|
\end{tabular}
\end{center}
%
The definition by |\providecommand| makes sure
that previous definitions are not overwritten.
Further statements |\providecommand{\version}{...}|
can thus be added before the above code to override it.

For the main file, one might add a line
(between |\childdocmain| and the above block)
%
\begin{center}
|%\ifchilddoc\||else\providecommand{\version}{draft}\||fi|
\end{center}
%
which can be uncommented to produce a draft version.
Likewise one can add a line to the very top of a child file
(above the |\childdocof{|\textit{main}|}| directive)
%
\begin{center}
|%\providecommand{\version}{final}|
\end{center}
%
which can be uncommented to produce the final version of this child document.

%%%%%%%%%%%%%%%%%%%%%%%%%%%%%%%%%%%%%%%%%%%%%%%%%%%%%%%%%%%%%%%%%%%%%%%%%%%%%%%%
\subsection{Forwarding}
\label{sec:forward}

Different versions of the main or child documents
using compilation flags as described in \secref{sec:flags}
can be (permanently) stored in different files
for convenient compilation, viewing and distribution.
To this end, the package defines a command
to pass on compilation to a different file:

%%%%%%%%%%%%%%%%%%%%%%%%%%%%%%%%%%%%%%%%
\DescribeMacro{\childdocforward}
The command |\childdocforward| redirects processing to
another source file:
%
\begin{center}
\begin{tabular}{l}
|\input{childdoc.def}|\\
|\childdocforward[|\textit{main}|]{|\textit{dest}|}|\\
\end{tabular}
\end{center}
%
The argument \textit{dest} is the destination file
(without extension).
It should be the main file or one of the child files.
Note that further \textsf{childdoc} directives
such as |\childdocof| and |\childdocforward|
in the indicated file will be processed in this form.
The optional argument \textit{main}
passes on directly to the main file \textit{main}
while pretending to compile the child \textit{dest}.
This form behaves as if \textit{dest}
issues |\childdocof{|\textit{main}|}| right away,
and no further \textsf{childdoc} directives will be processed.

%%%%%%%%%%%%%%%%%%%%%%%%%%%%%%%%%%%%%%%%
\DescribeMacro{\...prefix}
In the alternative form |\childdocforwardprefix|,
%
\begin{center}
\begin{tabular}{l}
|\input{childdoc.def}|\\
|\childdocforwardprefix[|\textit{main}|]{|\textit{prefix}|}{|\textit{dest}|}|
\end{tabular}
\end{center}
%
the destination file is determined by a pattern
depending on the current file:
To make this work, the current file must be called
`{\textit{prefix}\hspace{0.2em}\textit{suffix}}'
with \textit{prefix} matching precisely the argument.
Processing is then passed on to the file
`{\textit{dest}\hspace{0.2em}\textit{suffix}}'.
Surely, the same effect is achieved by
directly specifying the
argument `{\textit{dest}\hspace{0.2em}\textit{suffix}}'
in the first form.
However, that requires to set up a different file
for each child. With the alternative form of the command
all these files can have exactly the same content
which simplifies setting them up and maintaining them.

For example, the following file |draft.tex|
with a compilation flag |\version| as described in \secref{sec:flags}
compiles the main document as a draft:
%
\begin{center}
\begin{tabular}{l}
|\def\version{draft}|\\
|\input{childdoc.def}|\\
|\childdocforward{|\textit{main}|}|
\end{tabular}
\end{center}
%
Likewise, the following files |final|\textit{nn}|.tex|
compile the final version of the child document
|child|\textit{nn}|.tex|:
%
\begin{center}
\begin{tabular}{l}
|\def\version{final}|\\
|\input{childdoc.def}|\\
|\childdocforwardprefix{final}{child}|
\end{tabular}
\end{center}
%

Note that when several versions of a main file and/or of each child file
are to be generated, it may be convenient to set up a |Makefile| or
shell script to automatise the process.

%%%%%%%%%%%%%%%%%%%%%%%%%%%%%%%%%%%%%%%%%%%%%%%%%%%%%%%%%%%%%%%%%%%%%%%%%%%%%%%%
\subsection{Command Line Processing}
\label{sec:commandline}

The effect of redirection files can also be achieved by invoking
the \LaTeX{} compiler with a more elaborate command line.
Most conveniently this should be done as part
of a shell script or a |Makefile|.

When using \textsf{childdoc} in the main file, the following
command lines effectively perform a redirection
(note that depending on the shell being used,
backslashes may have to be doubled: `|\|' $\to$ `|\\|'):
%
\begin{center}
|... -jobname "|\textit{target}|" |\\|"|[\textit{flags}]%
|\input{childdoc.def}\childdocforward[|\textit{main}|]{|\textit{dest}|}"|
\end{center}
%
Here \textit{target} is the name of the output file,
\textit{main} is the name of the main file
and \textit{dest} is the name of the main or child file to be processed
(all filenames without extensions).
The optional argument \textit{main} can be omitted
if \textit{main} matches \textit{dest}.
Optionally, compilation \textit{flags} can be defined via |\def| commands.
This command line makes the \TeX{} engine believe
it is compiling the file \textit{target}
whose content is specified as the latter parameter.
The provided code then forwards the processing to
\textit{main} or \textit{dest} as described in \secref{sec:forward}.

%%%%%%%%%%%%%%%%%%%%%%%%%%%%%%%%%%%%%%%%%%%%%%%%%%%%%%%%%%%%%%%%%%%%%%%%%%%%%%%%
\subsection{Include by Input}
\label{sec:input}

Including child documents by |\include| has some restrictions by design.
Most notably, the content of a child document always occupies
its own set of pages; pages cannot be shared between child documents.
Usually, this behaviour makes perfect sense
because each child document contain an essential part of the document.
However, in some situations it may be desirable to compose
a document from a collection of parts
without having mandatory page breaks between then.
For this case, the package
provides a mechanism to include parts
by |\input| which can also be processed individually.
However, by construction this mechanism
requires manual handling of the content to be output.

%%%%%%%%%%%%%%%%%%%%%%%%%%%%%%%%%%%%%%%%
\DescribeMacro{\ifchilddocmanual}
The main file should be prepared as usual, see \secref{sec:include}.
However, the document body must make a distinction
between processing of an individual part and of the main document, e.g.:
%
\begin{center}
\begin{tabular}{l}
|\ifchilddocmanual|\\
|\input{\childdocname}|\\
|\||else|\\
\textit{document body with }|\input{|\textit{part}|}|\\
|\||fi|
\end{tabular}
\end{center}
%
The conditional |\ifchilddocmanual| is true whenever
a part to be included by |\input| is being compiled,
and the name of the part is stored in |\childdocname|.

%%%%%%%%%%%%%%%%%%%%%%%%%%%%%%%%%%%%%%%%
\DescribeMacro{\childdocby}
Each part to be included by |\input| should start with:
%
\begin{center}
\begin{tabular}{l}
|\input{childdoc.def}|\\
|\childdocby{|\textit{main}|}|\\
\end{tabular}
\end{center}
%
The directive |\childdocby| is similar to |\childdocof|
described in \secref{sec:include},
but the subsequent selection of content must be done manually.
To that end, both |\ifchilddoc| and |\ifchilddocmanual|
will be true upon processing of a part,
and the name of the part is stored in |\childdocname|.
Note that |\jobname| will be set to the filename of the current part
so that each part receives an individual |.aux| file
that does not interfere with the |.aux| file(s) of the main document.
This behaviour can be altered by the alternative form
|\childdocby[*]{|\textit{main}|}| (with a non-empty optional argument)
which uses the |.aux| file of the main document
by setting |\jobname| to \textit{main}.

%%%%%%%%%%%%%%%%%%%%%%%%%%%%%%%%%%%%%%%%%%%%%%%%%%%%%%%%%%%%%%%%%%%%%%%%%%%%%%%%
\subsection{Driver Development}
\label{sec:driver}

The \textsf{childdoc} mechanism can also be use for the development
of definition files such as \LaTeX{} styles or classes.
This case differs from the above setup with multiple parts
included by |\include| in that no |\includeonly| should be invoked.
This can be achieved by starting the include file
(before |\ProvidesPackage|) with:
%
\begin{center}
\begin{tabular}{l}
|\input{childdoc.def}|\\
|\childdocforward{|\textit{main}|}|\\
\end{tabular}
\end{center}
%
or alternatively with:
%
\begin{center}
\begin{tabular}{l}
|\input{childdoc.def}|\\
|\childdocby{|\textit{main}|}|\\
\end{tabular}
\end{center}
%
Both forms have slightly different effects as described above.
The main file is prepared as usual, see \secref{sec:include}.

%%%%%%%%%%%%%%%%%%%%%%%%%%%%%%%%%%%%%%%%%%%%%%%%%%%%%%%%%%%%%%%%%%%%%%%%%%%%%%%%
\subsection{Legacy Detection}
\label{sec:detection}

The directive |\childdocmain| in the main file can detect
whether the complete document or merely a child is to be compiled
even without using the directive |\childdocof|.
This method is deprecated because it is less robust
and there is no compelling reason to use it;
it is merely provided for backward compatibility
and it may be removed in future versions.

If the detection mechanism is to be used,
it is mandatory to correctly specify
the filename of the main file as the argument of |\childdocmain|:
%
\begin{center}
\begin{tabular}{l}
|\input{childdoc.def}|\\
|\childdocmain{|\textit{main}|}|\\
\end{tabular}
\end{center}
%
If |\jobname| does not match the argument \textit{main} of |\childdocmain|,
it is assumed that |\jobname| points to the child file to be compiled.
When using |\childdocmain| with the main file specified as argument,
it suffices to start a child file
with just |\input{|\textit{main}|}|
without loading of the package and using |\childdocof|.
If instead all processing is done
with the appropriate \textsf{childdoc} directives,
the argument of \textit{main} of |\childdocmain| can be empty.

An alternative version of the command line processing described
in \secref{sec:commandline} using the detection mechanism reads:
%
\begin{center}
|... -jobname "|\textit{target}|" "|[\textit{flags}]%
[|\def\jobname{|\textit{dest}|}|]|\input{|\textit{main}|}"|
\end{center}

%%%%%%%%%%%%%%%%%%%%%%%%%%%%%%%%%%%%%%%%%%%%%%%%%%%%%%%%%%%%%%%%%%%%%%%%%%%%%%%%
\subsection{Manual Code}
\label{sec:manual}

In case one cannot be certain whether the definitions file |childdoc.def|
is installed on the target \TeX{} distribution
and one prefers not to ship it,
it is conceivable to paste a few relevant commands into the sources.

To that end, drop all statements |\input{childdoc.def}|
and perform the replacements as outlined below.
Instead of |\childdocmain{|\textit{main}|}| add the following code
to the top of the main file:
%
\begin{center}
\begin{tabular}{l}
|\||ifdefined\childdocname\endinput\||fi\newif\ifchilddoc|\\
|\edef\childdocname{\scantokens\expandafter{\jobname\noexpand}}|\\
|\def\childdocmain{|\textit{main}|}\||ifx\childdocmain\childdocname\||else|\\
|\childdoctrue\includeonly{\childdocname}\let\jobname\childdocmain\||fi|\\
\end{tabular}
\end{center}
%
Instead of |\childdocof{|\textit{main}|}| just include the main file
at the top of each child file:
%
\begin{center}
|\input{|\textit{main}|}|
\end{center}
%
A simple redirection |\childdocforward{|\textit{dest}|}| is achieved by:
%
\begin{center}
|\def\jobname{|\textit{dest}|}\input{\jobname}|
\end{center}
%
The redirection with prefix
|\childdocforwardprefix[|\textit{prefix}|]{|\textit{dest}|}|
is accomplished by:
%
\begin{center}
\begin{tabular}{l}
|{\edef\jobname{\scantokens\expandafter{\jobname\noexpand}}|\\
|\def\redirectjob |\textit{prefix}|#1~~~{\gdef\jobname{|\textit{dest}|#1}}|\\
|\expandafter\redirectjob\jobname~~~}\input{\jobname}|
\end{tabular}
\end{center}

In an alternative approach,
child documents can be compiled by a specific command line
without additional code or specific definitions:
%
\begin{center}
|... -jobname "|\textit{target}|" "|[\textit{flags}]%
|\includeonly{|\textit{dest}|}\input{|\textit{main}|}"|
\end{center}
%

%%%%%%%%%%%%%%%%%%%%%%%%%%%%%%%%%%%%%%%%%%%%%%%%%%%%%%%%%%%%%%%%%%%%%%%%%%%%%%%%
%%%%%%%%%%%%%%%%%%%%%%%%%%%%%%%%%%%%%%%%%%%%%%%%%%%%%%%%%%%%%%%%%%%%%%%%%%%%%%%%
\section{Information}

%%%%%%%%%%%%%%%%%%%%%%%%%%%%%%%%%%%%%%%%%%%%%%%%%%%%%%%%%%%%%%%%%%%%%%%%%%%%%%%%
\subsection{Copyright}

Copyright \copyright{} 2017--2018 Niklas Beisert

This work may be distributed and/or modified under the
conditions of the \LaTeX{} Project Public License, either version 1.3
of this license or (at your option) any later version.
The latest version of this license is in
  \url{http://www.latex-project.org/lppl.txt}
and version 1.3 or later is part of all distributions of \LaTeX{}
version 2005/12/01 or later.

This work has the LPPL maintenance status `maintained'.

The Current Maintainer of this work is Niklas Beisert.

This work consists of the files |README.txt|, |childdoc.ins| and |childdoc.dtx|
as well as the derived files |childdoc.def|, |cdocsamp.tex|
with |cdocsch1.tex|, |cdocsch2.tex|, |cdocspt3.tex|, |cdocspt4.tex|,
|cdocsdrf.tex|, |cdocsfn1.tex|, |cdocsfn2.tex|
as well as |childdoc.pdf|.

%%%%%%%%%%%%%%%%%%%%%%%%%%%%%%%%%%%%%%%%%%%%%%%%%%%%%%%%%%%%%%%%%%%%%%%%%%%%%%%%
\subsection{Files and Installation}

The package consists of the files:
%
\begin{center}
\begin{tabular}{ll}
    |README.txt|   & readme file \\
    |childdoc.ins| & installation file \\
    |childdoc.dtx| & source file \\
    |childdoc.def| & definition file \\
    |cdocsamp.tex| & sample main file \\
    |cdocsch1.tex| & sample include file \\
    |cdocsch2.tex| & sample include file \\
    |cdocspt3.tex| & sample part file \\
    |cdocspt4.tex| & sample part file \\
    |cdocsdrf.tex| & sample redirection file \\
    |cdocsfn1.tex| & sample redirection file \\
    |cdocsfn2.tex| & sample redirection file \\
    |childdoc.pdf| & manual
\end{tabular}
\end{center}
%
The distribution consists of the files
|README.txt|, |childdoc.ins| and |childdoc.dtx|.
%
\begin{itemize}
\item
Run (pdf)\LaTeX{} on |childdoc.dtx|
to compile the manual |childdoc.pdf| (this file).
\item
Run \LaTeX{} on |childdoc.ins| to create the definitions file |childdoc.def|
and the sample |cdocsamp.tex| with include files
|cdocsch1.tex|, |cdocsch2.tex|, |cdocspt3.tex|, |cdocspt4.tex|,
|cdocsdrf.tex|, |cdocsfn1.tex|, |cdocsfn2.tex|.
Then copy the file |childdoc.def| to an appropriate directory of your \LaTeX{}
distribution, e.g.\ \textit{texmf-root}|/tex/latex/childdoc|.
\end{itemize}

%%%%%%%%%%%%%%%%%%%%%%%%%%%%%%%%%%%%%%%%%%%%%%%%%%%%%%%%%%%%%%%%%%%%%%%%%%%%%%%%
\subsection{Related CTAN Packages}

There are several other packages which offer a similar functionality:
%
\begin{itemize}
\item
The packages
\href{http://ctan.org/pkg/docmute}{\textsf{docmute}},
\href{http://ctan.org/pkg/includex}{\textsf{includex}} and
\href{http://ctan.org/pkg/standalone}{\textsf{standalone}}
provide commands to include only the document body of
a child file thus allowing both files to be compiled individually.
\item
The packages \href{http://ctan.org/pkg/subdocs}{\textsf{subdocs}}
and \href{http://ctan.org/pkg/subfiles}{\textsf{subfiles}}
provide structures in which the main and child documents can be
encapsulated and allowing them to be compiled individually.
The inclusion mechanism is different from the conventional |\include|.
\item
The package \href{http://ctan.org/pkg/combine}{\textsf{combine}}
is an elaborate solution to combine several documents into one.
\end{itemize}
%
See also the CTAN topic \href{http://ctan.org/topic/subdocs}{\textsf{subdocs}}
for further related packages.
The present package differs from the above solutions in that
a document structure constructed with the conventional |\include| mechanism
just needs two extra commands at the top of every file
such that all constituent files can be compiled individually.

%%%%%%%%%%%%%%%%%%%%%%%%%%%%%%%%%%%%%%%%%%%%%%%%%%%%%%%%%%%%%%%%%%%%%%%%%%%%%%%%
%\subsection{Feature Suggestions}
%
%The following is a list of features which may be useful for future
%versions of this package:
%%
%\begin{itemize}
%\item
%\ldots
%\end{itemize}

%%%%%%%%%%%%%%%%%%%%%%%%%%%%%%%%%%%%%%%%%%%%%%%%%%%%%%%%%%%%%%%%%%%%%%%%%%%%%%%%
\subsection{Revision History}

%%%%%%%%%%%%%%%%%%%%%%%%%%%%%%%%%%%%%%%%
\paragraph{v2.0:} 2018/12/30

\begin{itemize}
\item
immediate forward processing
\item
added |\childdocby| mechanism
\item
manual restructured
\end{itemize}

%%%%%%%%%%%%%%%%%%%%%%%%%%%%%%%%%%%%%%%%
\paragraph{v1.6:} 2018/01/17

\begin{itemize}
\item
application for development of include files
\item
corrections to manual
\end{itemize}

%%%%%%%%%%%%%%%%%%%%%%%%%%%%%%%%%%%%%%%%
\paragraph{v1.5:} 2017/05/21

\begin{itemize}
\item
more complete structuring introduced
\item
|\childdocof| introduced
\item
|\childdoc| renamed to |\childdocmain|
\item
|\childredirect| renamed to |\childdocforward| and |\childdocforwardprefix|
and functionality expanded
\end{itemize}

%%%%%%%%%%%%%%%%%%%%%%%%%%%%%%%%%%%%%%%%
\paragraph{v1.0:} 2017/04/27

\begin{itemize}
\item
manual and install package
\item
first version published on CTAN
\end{itemize}

%%%%%%%%%%%%%%%%%%%%%%%%%%%%%%%%%%%%%%%%
\paragraph{v0.6:} 2017/04/26

\begin{itemize}
\item
redirection mechanism added
\end{itemize}

%%%%%%%%%%%%%%%%%%%%%%%%%%%%%%%%%%%%%%%%
\paragraph{v0.5:} 2017/04/26

\begin{itemize}
\item
functionality in definition file
\end{itemize}


%%%%%%%%%%%%%%%%%%%%%%%%%%%%%%%%%%%%%%%%%%%%%%%%%%%%%%%%%%%%%%%%%%%%%%%%%%%%%%%%
%%%%%%%%%%%%%%%%%%%%%%%%%%%%%%%%%%%%%%%%%%%%%%%%%%%%%%%%%%%%%%%%%%%%%%%%%%%%%%%%
%%%%%%%%%%%%%%%%%%%%%%%%%%%%%%%%%%%%%%%%%%%%%%%%%%%%%%%%%%%%%%%%%%%%%%%%%%%%%%%%
\appendix

\settowidth\MacroIndent{\rmfamily\scriptsize 000\ }

 \DocInput{childdoc.dtx}

\end{document}
%</driver>
% \fi
%
% %%%%%%%%%%%%%%%%%%%%%%%%%%%%%%%%%%%%%%%%%%%%%%%%%%%%%%%%%%%%%%%%%%%%%%%%%%%%%%
% %%%%%%%%%%%%%%%%%%%%%%%%%%%%%%%%%%%%%%%%%%%%%%%%%%%%%%%%%%%%%%%%%%%%%%%%%%%%%%
% \section{Sample}
%\iffalse
%<*samplemain>
%\fi
%
% The following presents a sample document
% with two chapters, two parts, a title page,
% a compile flag as well as three forwarding files to set the flag.
% It consists of eight |.tex| files:
% \begin{center}
% \begin{tabular}{ll}
% |cdocsamp.tex|&main file\\
% |cdocsch1.tex|&include file for chapter 1\\
% |cdocsch2.tex|&include file for chapter 2\\
% |cdocspt3.tex|&include file for part 3\\
% |cdocspt4.tex|&include file for part 4\\
% |cdocsdrf.tex|&forwarding file for main file in draft mode\\
% |cdocsfi1.tex|&forwarding file for final version of chapter 1\\
% |cdocsfi2.tex|&forwarding file for final version of chapter 2\\
% \end{tabular}
% \end{center}
% Each of the eight files can be compiled directly by the \LaTeX{} compiler.
%
% %%%%%%%%%%%%%%%%%%%%%%%%%%%%%%%%%%%%%%
% \paragraph{Main File.}
%
% The main file is called |cdocsamp.tex|.
%
% Load the \textsf{childdoc} definitions and
% declare the filename for the main document:
%    \begin{macrocode}
\input{childdoc.def}
\childdocmain{}
%    \end{macrocode}

% Optional override for |\version| flag:
%    \begin{macrocode}
%%\ifchilddoc\else\providecommand{\version}{draft}\fi
%    \end{macrocode}

% Define the default values for the |\version| flag
% (|final| for the main file and |draft| for childs):
%    \begin{macrocode}
\ifchilddoc
\providecommand{\version}{draft}
\else
\providecommand{\version}{final}
\fi
%    \end{macrocode}

% Load the standard document class:
%    \begin{macrocode}
\documentclass[12pt]{article}
%    \end{macrocode}

% Start the document body:
%    \begin{macrocode}
\begin{document}
%    \end{macrocode}

% Declare a title page.
% Print title, part of document being processed and version flag:
%    \begin{macrocode}
\addtocounter{page}{-1}
\begin{center}
{\LARGE\bfseries{}childdoc example\par}
\vspace{1cm}
\ifchilddoc
\ifchilddocmanual part\else chapter\fi:
`\childdocname' of `\childdocjob'\par
\else
main document: `\childdocjob'\par
\fi
version: \version\par
\end{center}
\newpage
%    \end{macrocode}

% Manually include selected file,
% otherwise process as usual:
%    \begin{macrocode}
\ifchilddocmanual
\section*{part `\childdocname'}
\input{\childdocname}
\else
%    \end{macrocode}

% Include the two chapters:
%    \begin{macrocode}
\include{cdocsch1}
\include{cdocsch2}
%    \end{macrocode}

% Include the two parts unless only chapters should be displayed:
%    \begin{macrocode}
\ifchilddoc\else
\section{part three}
\input{cdocspt3}
\section{part four}
\input{cdocspt4}
\fi
%    \end{macrocode}

% Process as usual until here:
%    \begin{macrocode}
\fi
%    \end{macrocode}

% End of document body:
%    \begin{macrocode}
\end{document}
%    \end{macrocode}
%\iffalse
%</samplemain>
%\fi
%
% %%%%%%%%%%%%%%%%%%%%%%%%%%%%%%%%%%%%%%
% \paragraph{Chapter Include Files.}
%
% The include files are called |cdocsch1.tex| and |cdocsch2.tex|.
%
%\iffalse
%<*samplechap1|samplechap2>
%\fi

% Optional override for |\version| flag:
%    \begin{macrocode}
%%\providecommand{\version}{final}
%    \end{macrocode}

% Include the main document:
%    \begin{macrocode}
\input{childdoc.def}
\childdocof{cdocsamp}
%    \end{macrocode}

%\iffalse
%</samplechap1|samplechap2>
%\fi
%
%\iffalse
%<*samplechap1>
%\fi
% Some text for chapter 1:
%    \begin{macrocode}
\section{one}
some text in chapter one
%    \end{macrocode}

%\iffalse
%</samplechap1>
%\fi
% Some text for chapter 2:
%\iffalse
%<*samplechap2>
%\fi
%    \begin{macrocode}
\section{two}
more text in chapter two
%    \end{macrocode}

%\iffalse
%</samplechap2>
%\fi
%
% %%%%%%%%%%%%%%%%%%%%%%%%%%%%%%%%%%%%%%
% \paragraph{Part Include Files.}
%
% The include files are called |cdocspt3.tex| and |cdocspt4.tex|.
%
%\iffalse
%<*samplepart3|samplepart4>
%\fi

% Optional override for |\version| flag:
%    \begin{macrocode}
%%\providecommand{\version}{final}
%    \end{macrocode}

% Include the main document:
%    \begin{macrocode}
\input{childdoc.def}
\childdocby{cdocsamp}
%    \end{macrocode}

%\iffalse
%</samplepart3|samplepart4>
%\fi
%
%\iffalse
%<*samplepart3>
%\fi
% Some text for part 3:
%    \begin{macrocode}
some text in part three
%    \end{macrocode}

%\iffalse
%</samplepart3>
%\fi
% Some text for part 4:
%\iffalse
%<*samplepart4>
%\fi
%    \begin{macrocode}
more text in part four
%    \end{macrocode}

%\iffalse
%</samplepart4>
%\fi
%
% %%%%%%%%%%%%%%%%%%%%%%%%%%%%%%%%%%%%%%
% \paragraph{Forwarding for a Complete Draft.}
%
% The following forwarding file |cdocsdrf.tex|
% compiles the main document in draft mode:
%\iffalse
%<*sampledraft>
%\fi
%    \begin{macrocode}
\def\version{draft}
\input{childdoc.def}
\childdocforward{cdocsamp}
%    \end{macrocode}

%\iffalse
%</sampledraft>
%\fi
%
% %%%%%%%%%%%%%%%%%%%%%%%%%%%%%%%%%%%%%%
% \paragraph{Forwarding for Final Version of the Chapters.}
%
% The following forwarding files |cdocsfn1.tex| and |cdocsfn2.tex|
% (with identical content)
% compile the final versions of the child documents
% |cdocsch1.tex| and |cdocsch2.tex|, respectively:
%\iffalse
%<*samplefinal>
%\fi
%    \begin{macrocode}
\def\version{final}
\input{childdoc.def}
\childdocforwardprefix[cdocsamp]{cdocsfn}{cdocsch}
%    \end{macrocode}

%\iffalse
%</samplefinal>
%\fi
%
% %%%%%%%%%%%%%%%%%%%%%%%%%%%%%%%%%%%%%%
% \paragraph{Command Line Processing.}
%
% The following three command lines generate the output files
% |cdocscld|, |cdocscl1| and |cdocscl2|
% which should be identical to
% |cdocsdrf|, |cdocsch1| and |cdocsfn2|, respectively:
% \begin{center}
% \begin{tabular}{l}
% |latex -jobname cdocscld \|\\
% |  "\def\version{draft}\input{childdoc.def}\childdocforward{cdocsamp}"|\\
% |latex -jobname cdocscl1 \|\\
% |  "\input{childdoc.def}\childdocforward[cdocsamp]{cdocsch1}"|\\
% |latex -jobname cdocscl2 \|\\
% |  "\def\version{final}\input{childdoc.def}\childdocforward{cdocsch2}"|
% \end{tabular}
% \end{center}
% Note that the trailing backslash on each first line
% merely continues the input to the second line
% (for convenient cut ant paste).
% Furthermore, the command |latex| can be replaced by any
% of its alternative versions such as |pdflatex|.
%
% %%%%%%%%%%%%%%%%%%%%%%%%%%%%%%%%%%%%%%%%%%%%%%%%%%%%%%%%%%%%%%%%%%%%%%%%%%%%%%
% %%%%%%%%%%%%%%%%%%%%%%%%%%%%%%%%%%%%%%%%%%%%%%%%%%%%%%%%%%%%%%%%%%%%%%%%%%%%%%
% \section{Implementation}
%\iffalse
%<*package>
%\fi
%
% This section describes the definitions file |childdoc.def|.

% The definitions cannot be loaded using |\usepackage| or |\RequirePackage|
% which has a mechanism to prevent loading a style file more than once.
% When loading the definitions by means of |\input|
% multiple instances have to be prevented manually:
%\iffalse
%This code needs to be before the `\ProvidesFile' directive
%which is defined at the beginning of this file.
%Therefore it is also placed there and commented out here.
%</package>
%<*discard>
%\fi
%    \begin{macrocode}
\ifdefined\childdocmain\endinput\fi
%    \end{macrocode}
%\iffalse
%</discard>
%<*package>
%\fi
%
% \macro{\ifchilddoc}
% \macro{\ifchilddocmanual}
% The conditional |\ifchilddoc| tells whether a
% child (true) or main (false) document is being compiled.
% The conditional |\ifchilddocmanual| tells whether
% the |\includeonly| mechanism is used (false) or
% the selection of child files must be performed manually (true).
% The definitions initialise to false:
%    \begin{macrocode}
\newif\ifchilddoc
\newif\ifchilddocmanual
%    \end{macrocode}

% \macro{\childdocname}
% \macro{\childdocjob}
% The macro |\childdocname| stores the name of the main document
% to be compiled. The macro |\childdocjob| stores the name of
% the document on which the \LaTeX{} compiler was originally invoked.
% The content of |\jobname| cannot be compared
% to filenames specified in the source due to different catcodes.
% The following code rescans |\jobname|, stores the result
% in |\childdocname| and saves a copy in |\childdocjob|:
%    \begin{macrocode}
\edef\childdocname{\scantokens\expandafter{\jobname\noexpand}}
\let\childdocjob\childdocname
%    \end{macrocode}

% \macro{\childdocdisable}
% The macro |\childdocdisable| prevents the main file
% from being processed more than once.
% At this stage, the main document command |\childdocmain|
% is assumed to be called once again where it should do nothing.
% Any subsequent call to it should prevent
% a secondary processing of the main document
% It overwrites the forwarding commands
% |\childdocof| and |\childdocforward|
% with empty macros to prevent further inclusions of the main document:
%    \begin{macrocode}
\newcommand{\childdocdisable}
{
  \renewcommand{\childdocmain}[1]{\renewcommand{\childdocmain}[1]{\endinput}}
  \renewcommand{\childdocof}[1]{}
  \renewcommand{\childdocby}[2][]{}
  \renewcommand{\childdocforward}[2][]{}
  \renewcommand{\childdocdisable}{}
}
%    \end{macrocode}

% \macro{\childdocmain}
% The macro |\childdocmain| is to be called at the top of the main file
% with nothing or the main filename (without extension) as argument.
% First, it breaks loops.
% If the argument is not empty and does not match |\childdocname|
% (which is set by the first inclusion of |childdoc.def|),
% |\ifchilddoc| is set to true, |\includeonly| is applied to the child file
% and |\jobname| is set to the main file
% (for proper handling of |.aux| files):
%    \begin{macrocode}
\newcommand{\childdocmain}[1]
{
  \childdocdisable\childdocmain{}
  \if?#1?\else
    \begingroup
      \def\childdoctmp{#1}
      \ifx\childdoctmp\childdocname
        \def\childdoctmp{}
      \else
        \def\childdoctmp
        {
          \childdoctrue
          \includeonly{\childdocname}
          \def\childdocjob{#1}
          \def\jobname{#1}
        }
      \fi
      \expandafter
    \endgroup
    \childdoctmp
  \fi
}
%    \end{macrocode}

% \macro{\childdocof}
% The command |\childdocof| redirects
% compilation to the main file |#1|.
%    \begin{macrocode}
\newcommand{\childdocof}[1]
{
  \childdocdisable
  \childdoctrue
  \includeonly{\childdocname}
  \def\jobname{#1}
  \def\childdocjob{#1}
  \input{#1}
}
%    \end{macrocode}

% \macro{\childdocby}
% The command |\childdocby| ....
%    \begin{macrocode}
\newcommand{\childdocby}[2][]
{
  \childdocdisable
  \childdoctrue
  \childdocmanualtrue
  \if?#1?\else
    \def\jobname{#2}
  \fi
  \def\childdocjob{#2}
  \input{#2}
  \endinput
}
%    \end{macrocode}

% \macro{\childdocforward}
% The command |\childdocforward| redirects
% compilation to the main file or
% (if the optional argument is given) a child file.
% Parameters are set as if the main file
% or a child file starting with |\childdocof| was compiled.
% Then compilation is handed over to the main file:
%    \begin{macrocode}
\newcommand{\childdocforward}[2][]
{
  \begingroup
    \if?#1?
      \def\childdoctmp
      {
        \def\childdocname{#2}
        \def\childdocjob{#2}
        \def\jobname{#2}
        \input{#2}
        \endinput
      }
    \else
      \def\childdoctmp
      {
        \childdocdisable
        \def\childdocname{#2}
        \childdoctrue
        \includeonly{#2}
        \def\childdocjob{#1}
        \def\jobname{#1}
        \input{#1}
        \endinput
      }
    \fi
    \expandafter
  \endgroup
  \childdoctmp
}
%    \end{macrocode}

% \macro{\childdocforwardprefix}
% The command |\childdocforwardprefix| redirects
% compilation to the main or a child file by means of a pattern.
% The prefix |#1| in the current filename is replaced by |#2|
% and the suffix of the current filename is kept
% (it is assumed that the filename does not contain the substring `|~~~|'
% which is used as a delimiter).
% Compilation is handed over to the new file by |\childdocforward|:
%    \begin{macrocode}
\newcommand{\childdocforwardprefix}[3][]
{
  \begingroup
    \def\childdocextract #2##1~~~{\def\childdoctmp{\childdocforward[#1]{#3##1}}}
    \expandafter\childdocextract\childdocname~~~
    \expandafter
  \endgroup
  \childdoctmp
}
%    \end{macrocode}

% \macro{\childdoc}
% The deprecated macro |\childdoc| is a legacy version of |\childdocmain|:
%    \begin{macrocode}
\newcommand{\childdoc}{\childdocmain}
%    \end{macrocode}

% \macro{\childdocredirect}
% The deprecated macro |\childdocredirect| is a legacy version
% of |\childdocforward| and |\childdocforwardprefix|:
%    \begin{macrocode}
\newcommand{\childdocredirect}[2][]
{
  \begingroup
    \if?#1?
      \def\childdoctmp{\childdocforward{#2}}
    \else
      \def\childdoctmp{\childdocforwardprefix{#1}{#2}}
    \fi
    \expandafter
  \endgroup
  \childdoctmp
}
%    \end{macrocode}

%\iffalse
%</package>
%\fi
%
\endinput
|\\
|\childdocforwardprefix{final}{child}|
\end{tabular}
\end{center}
%

Note that when several versions of a main file and/or of each child file
are to be generated, it may be convenient to set up a |Makefile| or
shell script to automatise the process.

%%%%%%%%%%%%%%%%%%%%%%%%%%%%%%%%%%%%%%%%%%%%%%%%%%%%%%%%%%%%%%%%%%%%%%%%%%%%%%%%
\subsection{Command Line Processing}
\label{sec:commandline}

The effect of redirection files can also be achieved by invoking
the \LaTeX{} compiler with a more elaborate command line.
Most conveniently this should be done as part
of a shell script or a |Makefile|.

When using \textsf{childdoc} in the main file, the following
command lines effectively perform a redirection
(note that depending on the shell being used,
backslashes may have to be doubled: `|\|' $\to$ `|\\|'):
%
\begin{center}
|... -jobname "|\textit{target}|" |\\|"|[\textit{flags}]%
|% \iffalse
%
% childdoc.dtx Copyright (C) 2017-2018 Niklas Beisert
%
% This work may be distributed and/or modified under the
% conditions of the LaTeX Project Public License, either version 1.3
% of this license or (at your option) any later version.
% The latest version of this license is in
%   http://www.latex-project.org/lppl.txt
% and version 1.3 or later is part of all distributions of LaTeX
% version 2005/12/01 or later.
%
% This work has the LPPL maintenance status `maintained'.
%
% The Current Maintainer of this work is Niklas Beisert.
%
% This work consists of the files childdoc.dtx and childdoc.ins
% and the derived files childdoc.def and cdocsamp.tex with
% cdocsch1.tex, cdocsch2.tex, cdocsdrf.tex, cdocsfn1.tex, cdocsfn2.tex.
%
%<package>\ifdefined\childdocmain\endinput\fi
%<package>\ProvidesFile{childdoc.def}[2018/12/30 v2.0 child document driver]
%<samplemain>\ProvidesFile{cdocsamp.tex}[2018/12/30 v2.0 sample for childdoc]
%<*driver>
%\ProvidesFile{childdoc.drv}[2018/12/30 v2.0 childdoc reference manual file]
\PassOptionsToClass{10pt,a4paper}{article}
\documentclass{ltxdoc}

\usepackage[margin=35mm]{geometry}
\usepackage{hyperref}
\usepackage{hyperxmp}
\usepackage[usenames]{color}

\hypersetup{colorlinks=true}
\hypersetup{pdfstartview=FitH}
\hypersetup{pdfpagemode=UseNone}
\hypersetup{pdfsource={}}
\hypersetup{pdflang={en-UK}}
\hypersetup{pdfcopyright={Copyright 2017-2018 Niklas Beisert.
  This work may be distributed and/or modified under the
  conditions of the LaTeX Project Public License, either version 1.3
  of this license or (at your option) any later version.}}
\hypersetup{pdflicenseurl={http://www.latex-project.org/lppl.txt}}
\hypersetup{pdfcontactaddress={ETH Zurich, ITP, HIT K,
  Wolfgang-Pauli-Strasse 27}}
\hypersetup{pdfcontactpostcode={8093}}
\hypersetup{pdfcontactcity={Zurich}}
\hypersetup{pdfcontactcountry={Switzerland}}
\hypersetup{pdfcontactemail={nbeisert@itp.phys.ethz.ch}}
\hypersetup{pdfcontacturl={http://people.phys.ethz.ch/\xmptilde nbeisert/}}

\newcommand{\secref}[1]{\hyperref[#1]{section \ref*{#1}}}

\parskip1ex
\parindent0pt
\let\olditemize\itemize
\def\itemize{\olditemize\parskip0pt}

\begin{document}

\title{The \textsf{childdoc} Package}
\hypersetup{pdftitle={The childdoc Package}}
\author{Niklas Beisert\\[2ex]
  Institut f\"ur Theoretische Physik\\
  Eidgen\"ossische Technische Hochschule Z\"urich\\
  Wolfgang-Pauli-Strasse 27, 8093 Z\"urich, Switzerland\\[1ex]
  \href{mailto:nbeisert@itp.phys.ethz.ch}
  {\texttt{nbeisert@itp.phys.ethz.ch}}}
\hypersetup{pdfauthor={Niklas Beisert}}
\hypersetup{pdfsubject={Manual for the LaTeX2e Package childdoc}}
\date{30 December 2018, \textsf{v2.0}}
\maketitle

\begin{abstract}\noindent
\textsf{childdoc} is a \LaTeXe{} package
that enables the direct compilation
of document sections included by |\include|
to individual files.
\end{abstract}

\begingroup
\parskip0ex
\tableofcontents
\endgroup

%%%%%%%%%%%%%%%%%%%%%%%%%%%%%%%%%%%%%%%%%%%%%%%%%%%%%%%%%%%%%%%%%%%%%%%%%%%%%%%%
%%%%%%%%%%%%%%%%%%%%%%%%%%%%%%%%%%%%%%%%%%%%%%%%%%%%%%%%%%%%%%%%%%%%%%%%%%%%%%%%
\section{Introduction}

\LaTeX{} provides a mechanism to structure a large document (such as a book)
into a main file and several child files (containing the chapters)
using the |\include| command.
This mechanism is beneficial for documents
which span hundreds of pages in order to
make the source file(s) more manageable.
Moreover, compilation can be restricted to
selected child files by means of the |\includeonly| command.
The latter feature can be used to reduce the compilation time while editing
(this was significantly more useful in the earlier days of \LaTeX{})
or to generate a smaller document which is easier to navigate.
Another application of |\includeonly| is to generate
documents consisting of selected parts of the complete document.

However, there are a few drawbacks of the plain |\include| mechanism:
\begin{itemize}
\item
The child files cannot be compiled on their own,
they can only be compiled via the main file.
A naive editing environment
(such as a text editor with an option
to have the current file processed by \LaTeX)
may require one to switch to the main file before compiling;
attempting to compile the child file produces errors.
\item
The main file must be modified (each time)
to adjust the |\includeonly| command
to the present needs. This easily leaves the main file in a messy state.
\item
The generated document will always carry the filename
of the main document. This is inconvenient if
several child files are to be compiled and
to be kept for distribution.
\end{itemize}

The present package provides a simple interface
to make child files individually compilable by \LaTeX{}.
Compiling a child file then has the same effect as compiling
the main file with an |\includeonly| command
to select the appropriate child.
Moreover the generated document will carry the name of the child
rather than the main file.
This resolves all three above issues.

This feature is meant to make the editing of books,
thesis documents and lecture notes somewhat more convenient.
However, the package can also be used efficiently for
composing a series of documents (such as exercise sheets)
which are typically distributed individually.
It then assists the author in generating the individual documents
(potentially in different versions)
as well as a document containing the collected series.
Another application is in developing style files
or other kinds of included material
where compilation of the style file could redirect
to a sample or test file.

%%%%%%%%%%%%%%%%%%%%%%%%%%%%%%%%%%%%%%%%%%%%%%%%%%%%%%%%%%%%%%%%%%%%%%%%%%%%%%%%
%%%%%%%%%%%%%%%%%%%%%%%%%%%%%%%%%%%%%%%%%%%%%%%%%%%%%%%%%%%%%%%%%%%%%%%%%%%%%%%%
\section{Usage}

First of all, the package \textsf{childdoc} is \emph{not} a standard
\LaTeXe{} |.sty| style file! Therefore it needs to be invoked in
a non-standard way.

%%%%%%%%%%%%%%%%%%%%%%%%%%%%%%%%%%%%%%%%%%%%%%%%%%%%%%%%%%%%%%%%%%%%%%%%%%%%%%%%
\subsection{Included Files}
\label{sec:include}

%%%%%%%%%%%%%%%%%%%%%%%%%%%%%%%%%%%%%%%%
\DescribeMacro{\childdocmain}
To use the package, add the commands
\begin{center}
\begin{tabular}{l}
|\input{childdoc.def}|\\
|\childdocmain{}|\\
\end{tabular}
\end{center}
at the very top of the main \LaTeX{} file,
in particular \emph{before} the |\documentclass| statement!
The argument of |\childdocmain| should be left empty
(but it must be present).

%%%%%%%%%%%%%%%%%%%%%%%%%%%%%%%%%%%%%%%%
\DescribeMacro{\childdocof}
Furthermore, add the commands
\begin{center}
\begin{tabular}{l}
|\input{childdoc.def}|\\
|\childdocof{|\textit{main}|}|\\
\end{tabular}
\end{center}
at the top of every child file \textit{child}
which is included by |\include{|\textit{child}|}|
from within the main file
(or at least for those files to be compiled individually).
The argument \textit{main} must be the filename of the main file.

There are a couple of
considerations in setting up the main and child documents:

%%%%%%%%%%%%%%%%%%%%%%%%%%%%%%%%%%%%%%%%
\paragraph{Restrictions.}

Please note the following restrictions:
\begin{itemize}
\item
|\childdocmain| must be called with one argument \textit{main}
to ensure compatibility with earlier version of the package.
It must either be empty (|\childdocmain{}|)
or precisely match the filename of the main file in which it is specified.
See \secref{sec:detection} for further information.
\item
The filename \textit{main} must be specified without the |.tex| extension.
\item
The filename \textit{main} is case sensitive
(even in case-insensitive file systems)
due to internal string comparison.
\item
The argument \textit{main} should be fully expanded, it cannot be a macro.
\item
Subdirectories and special characters should be avoided in filenames.
\item
The command |\childdocmain{|\textit{main}|}| must be followed by a whitespace.
It should not be followed immediately by another command
or by a comment mark `|%|'.
This is because the \TeX{} parser reads the token immediately following
the argument of |\childdocmain| and puts it
at the beginning of every child section;
however, a white\-space is ignored.
\end{itemize}

%%%%%%%%%%%%%%%%%%%%%%%%%%%%%%%%%%%%%%%%
\paragraph{Content of Main File.}

It is advisable to place all content in the child files included by |\include|.
Any output contained in the main file will appear in all child documents
unless suppressed manually;
it cannot be suppressed automatically by the |\includeonly| directive
and thus should normally be avoided.
A method to include some content in the main file
by means of conditional processing is described in \secref{sec:conditional}.

%%%%%%%%%%%%%%%%%%%%%%%%%%%%%%%%%%%%%%%%
\paragraph{Page Numbering.}

When only a part of the document is compiled,
the appropriate numbering of pages
(as well as other status parameters)
is determined from the |.aux| files.
The latter contain information from previous passes.
However this information needs to propagate through
all intermediate child documents.
Therefore the page numbering in child documents may well
be inconsistent until the complete document is compiled at least once.

A useful (if unconventional) way to always ensure a consistent
page numbering is to restart the numbering in each child document
and denote the pages by `\textit{child}|.|\textit{page}'
where \textit{child} represents the chapter/section number of the child file.
This can be achieved by the command
|\numberwithin{page}{|\textit{child}|}|
of the \textsf{amsmath} package
where \textit{child} can be |chapter| or |section|
depending on the chosen structuring.
Alternatively, one can modify the macro |\thepage| appropriately
and reset the counter |page| at the start of each child file.

%%%%%%%%%%%%%%%%%%%%%%%%%%%%%%%%%%%%%%%%%%%%%%%%%%%%%%%%%%%%%%%%%%%%%%%%%%%%%%%%
\subsection{Conditional Processing}
\label{sec:conditional}

The package provides a mechanism to compile different versions
of a document. To customise the versions further some conditional processing
can come in handy to distinguish which version is being compiled.
The package provides two macros to describe the compilation context:

%%%%%%%%%%%%%%%%%%%%%%%%%%%%%%%%%%%%%%%%
\DescribeMacro{\ifchilddoc}
The conditional |\ifchilddoc| distinguishes between the compilation of
child documents and the main document:
%
\begin{center}
|\ifchilddoc |\textit{child-code}| |[|\||else |\textit{main-code}]| \||fi|
\end{center}

%%%%%%%%%%%%%%%%%%%%%%%%%%%%%%%%%%%%%%%%
\DescribeMacro{\childdocname}
\DescribeMacro{\childdocjob}
The macro |\childdocname| contains the filename (without extension)
of the main or child file being processed.
Note that |\childdocjob| will always contain the name of the main file.

%%%%%%%%%%%%%%%%%%%%%%%%%%%%%%%%%%%%%%%%
\paragraph{Title Page.}

Conditional processing can be used to include a title or banner page
in the main document when proper precautions are taken.
Importantly, the code in the main file should ensure that the page counter
(as well as other status parameters which are stored in the |.aux| files)
takes the same value after the conditional processing.
Otherwise the page numbers may take divergent values
depending on which part is compiled.

For example, a title page could be declared by:
%
\begin{center}
\begin{tabular}{l}
|\ifchilddoc\||else|\\
|\addtocounter{page}{-1}|\\
\textit{code for title page}\\
|\newpage|\\
|\||fi|
\end{tabular}
\end{center}
%
A banner page for the child documents can be generated by:
%
\begin{center}
\begin{tabular}{l}
|\ifchilddoc|\\
|\addtocounter{page}{-1}|\\
\textit{code for banner page}\\
|\newpage|\\
|\||fi|
\end{tabular}
\end{center}
%
Here one could write a message such as:
\begin{center}
|This is the part \childdocname{} of \childdocjob{}.|
\end{center}

%%%%%%%%%%%%%%%%%%%%%%%%%%%%%%%%%%%%%%%%%%%%%%%%%%%%%%%%%%%%%%%%%%%%%%%%%%%%%%%%
\subsection{Flags}
\label{sec:flags}

The package makes it easy to generate different versions
of the main or child documents.
To this end compilation flags can be defined
and assigned different default values.
They will be particularly useful in conjunction
with the forwarding mechanism described in \secref{sec:forward}.

For example, it may be useful to have a flag |\version|
which can be set to |draft| or |final|.
The document source will contain some conditional code
depending on the value of |\version|.
Suppose further, the flag should default to |final| for the main file
and to |draft| for child files
which is a natural assignment for editing the document.
This is achieved by placing the following code
in the preamble of the main document
(below the |\childdocmain| directive):
%
\begin{center}
\begin{tabular}{l}
|\ifchilddoc|\\
|\providecommand{\version}{draft}|\\
|\||else|\\
|\providecommand{\version}{final}|\\
|\||fi|
\end{tabular}
\end{center}
%
The definition by |\providecommand| makes sure
that previous definitions are not overwritten.
Further statements |\providecommand{\version}{...}|
can thus be added before the above code to override it.

For the main file, one might add a line
(between |\childdocmain| and the above block)
%
\begin{center}
|%\ifchilddoc\||else\providecommand{\version}{draft}\||fi|
\end{center}
%
which can be uncommented to produce a draft version.
Likewise one can add a line to the very top of a child file
(above the |\childdocof{|\textit{main}|}| directive)
%
\begin{center}
|%\providecommand{\version}{final}|
\end{center}
%
which can be uncommented to produce the final version of this child document.

%%%%%%%%%%%%%%%%%%%%%%%%%%%%%%%%%%%%%%%%%%%%%%%%%%%%%%%%%%%%%%%%%%%%%%%%%%%%%%%%
\subsection{Forwarding}
\label{sec:forward}

Different versions of the main or child documents
using compilation flags as described in \secref{sec:flags}
can be (permanently) stored in different files
for convenient compilation, viewing and distribution.
To this end, the package defines a command
to pass on compilation to a different file:

%%%%%%%%%%%%%%%%%%%%%%%%%%%%%%%%%%%%%%%%
\DescribeMacro{\childdocforward}
The command |\childdocforward| redirects processing to
another source file:
%
\begin{center}
\begin{tabular}{l}
|\input{childdoc.def}|\\
|\childdocforward[|\textit{main}|]{|\textit{dest}|}|\\
\end{tabular}
\end{center}
%
The argument \textit{dest} is the destination file
(without extension).
It should be the main file or one of the child files.
Note that further \textsf{childdoc} directives
such as |\childdocof| and |\childdocforward|
in the indicated file will be processed in this form.
The optional argument \textit{main}
passes on directly to the main file \textit{main}
while pretending to compile the child \textit{dest}.
This form behaves as if \textit{dest}
issues |\childdocof{|\textit{main}|}| right away,
and no further \textsf{childdoc} directives will be processed.

%%%%%%%%%%%%%%%%%%%%%%%%%%%%%%%%%%%%%%%%
\DescribeMacro{\...prefix}
In the alternative form |\childdocforwardprefix|,
%
\begin{center}
\begin{tabular}{l}
|\input{childdoc.def}|\\
|\childdocforwardprefix[|\textit{main}|]{|\textit{prefix}|}{|\textit{dest}|}|
\end{tabular}
\end{center}
%
the destination file is determined by a pattern
depending on the current file:
To make this work, the current file must be called
`{\textit{prefix}\hspace{0.2em}\textit{suffix}}'
with \textit{prefix} matching precisely the argument.
Processing is then passed on to the file
`{\textit{dest}\hspace{0.2em}\textit{suffix}}'.
Surely, the same effect is achieved by
directly specifying the
argument `{\textit{dest}\hspace{0.2em}\textit{suffix}}'
in the first form.
However, that requires to set up a different file
for each child. With the alternative form of the command
all these files can have exactly the same content
which simplifies setting them up and maintaining them.

For example, the following file |draft.tex|
with a compilation flag |\version| as described in \secref{sec:flags}
compiles the main document as a draft:
%
\begin{center}
\begin{tabular}{l}
|\def\version{draft}|\\
|\input{childdoc.def}|\\
|\childdocforward{|\textit{main}|}|
\end{tabular}
\end{center}
%
Likewise, the following files |final|\textit{nn}|.tex|
compile the final version of the child document
|child|\textit{nn}|.tex|:
%
\begin{center}
\begin{tabular}{l}
|\def\version{final}|\\
|\input{childdoc.def}|\\
|\childdocforwardprefix{final}{child}|
\end{tabular}
\end{center}
%

Note that when several versions of a main file and/or of each child file
are to be generated, it may be convenient to set up a |Makefile| or
shell script to automatise the process.

%%%%%%%%%%%%%%%%%%%%%%%%%%%%%%%%%%%%%%%%%%%%%%%%%%%%%%%%%%%%%%%%%%%%%%%%%%%%%%%%
\subsection{Command Line Processing}
\label{sec:commandline}

The effect of redirection files can also be achieved by invoking
the \LaTeX{} compiler with a more elaborate command line.
Most conveniently this should be done as part
of a shell script or a |Makefile|.

When using \textsf{childdoc} in the main file, the following
command lines effectively perform a redirection
(note that depending on the shell being used,
backslashes may have to be doubled: `|\|' $\to$ `|\\|'):
%
\begin{center}
|... -jobname "|\textit{target}|" |\\|"|[\textit{flags}]%
|\input{childdoc.def}\childdocforward[|\textit{main}|]{|\textit{dest}|}"|
\end{center}
%
Here \textit{target} is the name of the output file,
\textit{main} is the name of the main file
and \textit{dest} is the name of the main or child file to be processed
(all filenames without extensions).
The optional argument \textit{main} can be omitted
if \textit{main} matches \textit{dest}.
Optionally, compilation \textit{flags} can be defined via |\def| commands.
This command line makes the \TeX{} engine believe
it is compiling the file \textit{target}
whose content is specified as the latter parameter.
The provided code then forwards the processing to
\textit{main} or \textit{dest} as described in \secref{sec:forward}.

%%%%%%%%%%%%%%%%%%%%%%%%%%%%%%%%%%%%%%%%%%%%%%%%%%%%%%%%%%%%%%%%%%%%%%%%%%%%%%%%
\subsection{Include by Input}
\label{sec:input}

Including child documents by |\include| has some restrictions by design.
Most notably, the content of a child document always occupies
its own set of pages; pages cannot be shared between child documents.
Usually, this behaviour makes perfect sense
because each child document contain an essential part of the document.
However, in some situations it may be desirable to compose
a document from a collection of parts
without having mandatory page breaks between then.
For this case, the package
provides a mechanism to include parts
by |\input| which can also be processed individually.
However, by construction this mechanism
requires manual handling of the content to be output.

%%%%%%%%%%%%%%%%%%%%%%%%%%%%%%%%%%%%%%%%
\DescribeMacro{\ifchilddocmanual}
The main file should be prepared as usual, see \secref{sec:include}.
However, the document body must make a distinction
between processing of an individual part and of the main document, e.g.:
%
\begin{center}
\begin{tabular}{l}
|\ifchilddocmanual|\\
|\input{\childdocname}|\\
|\||else|\\
\textit{document body with }|\input{|\textit{part}|}|\\
|\||fi|
\end{tabular}
\end{center}
%
The conditional |\ifchilddocmanual| is true whenever
a part to be included by |\input| is being compiled,
and the name of the part is stored in |\childdocname|.

%%%%%%%%%%%%%%%%%%%%%%%%%%%%%%%%%%%%%%%%
\DescribeMacro{\childdocby}
Each part to be included by |\input| should start with:
%
\begin{center}
\begin{tabular}{l}
|\input{childdoc.def}|\\
|\childdocby{|\textit{main}|}|\\
\end{tabular}
\end{center}
%
The directive |\childdocby| is similar to |\childdocof|
described in \secref{sec:include},
but the subsequent selection of content must be done manually.
To that end, both |\ifchilddoc| and |\ifchilddocmanual|
will be true upon processing of a part,
and the name of the part is stored in |\childdocname|.
Note that |\jobname| will be set to the filename of the current part
so that each part receives an individual |.aux| file
that does not interfere with the |.aux| file(s) of the main document.
This behaviour can be altered by the alternative form
|\childdocby[*]{|\textit{main}|}| (with a non-empty optional argument)
which uses the |.aux| file of the main document
by setting |\jobname| to \textit{main}.

%%%%%%%%%%%%%%%%%%%%%%%%%%%%%%%%%%%%%%%%%%%%%%%%%%%%%%%%%%%%%%%%%%%%%%%%%%%%%%%%
\subsection{Driver Development}
\label{sec:driver}

The \textsf{childdoc} mechanism can also be use for the development
of definition files such as \LaTeX{} styles or classes.
This case differs from the above setup with multiple parts
included by |\include| in that no |\includeonly| should be invoked.
This can be achieved by starting the include file
(before |\ProvidesPackage|) with:
%
\begin{center}
\begin{tabular}{l}
|\input{childdoc.def}|\\
|\childdocforward{|\textit{main}|}|\\
\end{tabular}
\end{center}
%
or alternatively with:
%
\begin{center}
\begin{tabular}{l}
|\input{childdoc.def}|\\
|\childdocby{|\textit{main}|}|\\
\end{tabular}
\end{center}
%
Both forms have slightly different effects as described above.
The main file is prepared as usual, see \secref{sec:include}.

%%%%%%%%%%%%%%%%%%%%%%%%%%%%%%%%%%%%%%%%%%%%%%%%%%%%%%%%%%%%%%%%%%%%%%%%%%%%%%%%
\subsection{Legacy Detection}
\label{sec:detection}

The directive |\childdocmain| in the main file can detect
whether the complete document or merely a child is to be compiled
even without using the directive |\childdocof|.
This method is deprecated because it is less robust
and there is no compelling reason to use it;
it is merely provided for backward compatibility
and it may be removed in future versions.

If the detection mechanism is to be used,
it is mandatory to correctly specify
the filename of the main file as the argument of |\childdocmain|:
%
\begin{center}
\begin{tabular}{l}
|\input{childdoc.def}|\\
|\childdocmain{|\textit{main}|}|\\
\end{tabular}
\end{center}
%
If |\jobname| does not match the argument \textit{main} of |\childdocmain|,
it is assumed that |\jobname| points to the child file to be compiled.
When using |\childdocmain| with the main file specified as argument,
it suffices to start a child file
with just |\input{|\textit{main}|}|
without loading of the package and using |\childdocof|.
If instead all processing is done
with the appropriate \textsf{childdoc} directives,
the argument of \textit{main} of |\childdocmain| can be empty.

An alternative version of the command line processing described
in \secref{sec:commandline} using the detection mechanism reads:
%
\begin{center}
|... -jobname "|\textit{target}|" "|[\textit{flags}]%
[|\def\jobname{|\textit{dest}|}|]|\input{|\textit{main}|}"|
\end{center}

%%%%%%%%%%%%%%%%%%%%%%%%%%%%%%%%%%%%%%%%%%%%%%%%%%%%%%%%%%%%%%%%%%%%%%%%%%%%%%%%
\subsection{Manual Code}
\label{sec:manual}

In case one cannot be certain whether the definitions file |childdoc.def|
is installed on the target \TeX{} distribution
and one prefers not to ship it,
it is conceivable to paste a few relevant commands into the sources.

To that end, drop all statements |\input{childdoc.def}|
and perform the replacements as outlined below.
Instead of |\childdocmain{|\textit{main}|}| add the following code
to the top of the main file:
%
\begin{center}
\begin{tabular}{l}
|\||ifdefined\childdocname\endinput\||fi\newif\ifchilddoc|\\
|\edef\childdocname{\scantokens\expandafter{\jobname\noexpand}}|\\
|\def\childdocmain{|\textit{main}|}\||ifx\childdocmain\childdocname\||else|\\
|\childdoctrue\includeonly{\childdocname}\let\jobname\childdocmain\||fi|\\
\end{tabular}
\end{center}
%
Instead of |\childdocof{|\textit{main}|}| just include the main file
at the top of each child file:
%
\begin{center}
|\input{|\textit{main}|}|
\end{center}
%
A simple redirection |\childdocforward{|\textit{dest}|}| is achieved by:
%
\begin{center}
|\def\jobname{|\textit{dest}|}\input{\jobname}|
\end{center}
%
The redirection with prefix
|\childdocforwardprefix[|\textit{prefix}|]{|\textit{dest}|}|
is accomplished by:
%
\begin{center}
\begin{tabular}{l}
|{\edef\jobname{\scantokens\expandafter{\jobname\noexpand}}|\\
|\def\redirectjob |\textit{prefix}|#1~~~{\gdef\jobname{|\textit{dest}|#1}}|\\
|\expandafter\redirectjob\jobname~~~}\input{\jobname}|
\end{tabular}
\end{center}

In an alternative approach,
child documents can be compiled by a specific command line
without additional code or specific definitions:
%
\begin{center}
|... -jobname "|\textit{target}|" "|[\textit{flags}]%
|\includeonly{|\textit{dest}|}\input{|\textit{main}|}"|
\end{center}
%

%%%%%%%%%%%%%%%%%%%%%%%%%%%%%%%%%%%%%%%%%%%%%%%%%%%%%%%%%%%%%%%%%%%%%%%%%%%%%%%%
%%%%%%%%%%%%%%%%%%%%%%%%%%%%%%%%%%%%%%%%%%%%%%%%%%%%%%%%%%%%%%%%%%%%%%%%%%%%%%%%
\section{Information}

%%%%%%%%%%%%%%%%%%%%%%%%%%%%%%%%%%%%%%%%%%%%%%%%%%%%%%%%%%%%%%%%%%%%%%%%%%%%%%%%
\subsection{Copyright}

Copyright \copyright{} 2017--2018 Niklas Beisert

This work may be distributed and/or modified under the
conditions of the \LaTeX{} Project Public License, either version 1.3
of this license or (at your option) any later version.
The latest version of this license is in
  \url{http://www.latex-project.org/lppl.txt}
and version 1.3 or later is part of all distributions of \LaTeX{}
version 2005/12/01 or later.

This work has the LPPL maintenance status `maintained'.

The Current Maintainer of this work is Niklas Beisert.

This work consists of the files |README.txt|, |childdoc.ins| and |childdoc.dtx|
as well as the derived files |childdoc.def|, |cdocsamp.tex|
with |cdocsch1.tex|, |cdocsch2.tex|, |cdocspt3.tex|, |cdocspt4.tex|,
|cdocsdrf.tex|, |cdocsfn1.tex|, |cdocsfn2.tex|
as well as |childdoc.pdf|.

%%%%%%%%%%%%%%%%%%%%%%%%%%%%%%%%%%%%%%%%%%%%%%%%%%%%%%%%%%%%%%%%%%%%%%%%%%%%%%%%
\subsection{Files and Installation}

The package consists of the files:
%
\begin{center}
\begin{tabular}{ll}
    |README.txt|   & readme file \\
    |childdoc.ins| & installation file \\
    |childdoc.dtx| & source file \\
    |childdoc.def| & definition file \\
    |cdocsamp.tex| & sample main file \\
    |cdocsch1.tex| & sample include file \\
    |cdocsch2.tex| & sample include file \\
    |cdocspt3.tex| & sample part file \\
    |cdocspt4.tex| & sample part file \\
    |cdocsdrf.tex| & sample redirection file \\
    |cdocsfn1.tex| & sample redirection file \\
    |cdocsfn2.tex| & sample redirection file \\
    |childdoc.pdf| & manual
\end{tabular}
\end{center}
%
The distribution consists of the files
|README.txt|, |childdoc.ins| and |childdoc.dtx|.
%
\begin{itemize}
\item
Run (pdf)\LaTeX{} on |childdoc.dtx|
to compile the manual |childdoc.pdf| (this file).
\item
Run \LaTeX{} on |childdoc.ins| to create the definitions file |childdoc.def|
and the sample |cdocsamp.tex| with include files
|cdocsch1.tex|, |cdocsch2.tex|, |cdocspt3.tex|, |cdocspt4.tex|,
|cdocsdrf.tex|, |cdocsfn1.tex|, |cdocsfn2.tex|.
Then copy the file |childdoc.def| to an appropriate directory of your \LaTeX{}
distribution, e.g.\ \textit{texmf-root}|/tex/latex/childdoc|.
\end{itemize}

%%%%%%%%%%%%%%%%%%%%%%%%%%%%%%%%%%%%%%%%%%%%%%%%%%%%%%%%%%%%%%%%%%%%%%%%%%%%%%%%
\subsection{Related CTAN Packages}

There are several other packages which offer a similar functionality:
%
\begin{itemize}
\item
The packages
\href{http://ctan.org/pkg/docmute}{\textsf{docmute}},
\href{http://ctan.org/pkg/includex}{\textsf{includex}} and
\href{http://ctan.org/pkg/standalone}{\textsf{standalone}}
provide commands to include only the document body of
a child file thus allowing both files to be compiled individually.
\item
The packages \href{http://ctan.org/pkg/subdocs}{\textsf{subdocs}}
and \href{http://ctan.org/pkg/subfiles}{\textsf{subfiles}}
provide structures in which the main and child documents can be
encapsulated and allowing them to be compiled individually.
The inclusion mechanism is different from the conventional |\include|.
\item
The package \href{http://ctan.org/pkg/combine}{\textsf{combine}}
is an elaborate solution to combine several documents into one.
\end{itemize}
%
See also the CTAN topic \href{http://ctan.org/topic/subdocs}{\textsf{subdocs}}
for further related packages.
The present package differs from the above solutions in that
a document structure constructed with the conventional |\include| mechanism
just needs two extra commands at the top of every file
such that all constituent files can be compiled individually.

%%%%%%%%%%%%%%%%%%%%%%%%%%%%%%%%%%%%%%%%%%%%%%%%%%%%%%%%%%%%%%%%%%%%%%%%%%%%%%%%
%\subsection{Feature Suggestions}
%
%The following is a list of features which may be useful for future
%versions of this package:
%%
%\begin{itemize}
%\item
%\ldots
%\end{itemize}

%%%%%%%%%%%%%%%%%%%%%%%%%%%%%%%%%%%%%%%%%%%%%%%%%%%%%%%%%%%%%%%%%%%%%%%%%%%%%%%%
\subsection{Revision History}

%%%%%%%%%%%%%%%%%%%%%%%%%%%%%%%%%%%%%%%%
\paragraph{v2.0:} 2018/12/30

\begin{itemize}
\item
immediate forward processing
\item
added |\childdocby| mechanism
\item
manual restructured
\end{itemize}

%%%%%%%%%%%%%%%%%%%%%%%%%%%%%%%%%%%%%%%%
\paragraph{v1.6:} 2018/01/17

\begin{itemize}
\item
application for development of include files
\item
corrections to manual
\end{itemize}

%%%%%%%%%%%%%%%%%%%%%%%%%%%%%%%%%%%%%%%%
\paragraph{v1.5:} 2017/05/21

\begin{itemize}
\item
more complete structuring introduced
\item
|\childdocof| introduced
\item
|\childdoc| renamed to |\childdocmain|
\item
|\childredirect| renamed to |\childdocforward| and |\childdocforwardprefix|
and functionality expanded
\end{itemize}

%%%%%%%%%%%%%%%%%%%%%%%%%%%%%%%%%%%%%%%%
\paragraph{v1.0:} 2017/04/27

\begin{itemize}
\item
manual and install package
\item
first version published on CTAN
\end{itemize}

%%%%%%%%%%%%%%%%%%%%%%%%%%%%%%%%%%%%%%%%
\paragraph{v0.6:} 2017/04/26

\begin{itemize}
\item
redirection mechanism added
\end{itemize}

%%%%%%%%%%%%%%%%%%%%%%%%%%%%%%%%%%%%%%%%
\paragraph{v0.5:} 2017/04/26

\begin{itemize}
\item
functionality in definition file
\end{itemize}


%%%%%%%%%%%%%%%%%%%%%%%%%%%%%%%%%%%%%%%%%%%%%%%%%%%%%%%%%%%%%%%%%%%%%%%%%%%%%%%%
%%%%%%%%%%%%%%%%%%%%%%%%%%%%%%%%%%%%%%%%%%%%%%%%%%%%%%%%%%%%%%%%%%%%%%%%%%%%%%%%
%%%%%%%%%%%%%%%%%%%%%%%%%%%%%%%%%%%%%%%%%%%%%%%%%%%%%%%%%%%%%%%%%%%%%%%%%%%%%%%%
\appendix

\settowidth\MacroIndent{\rmfamily\scriptsize 000\ }

 \DocInput{childdoc.dtx}

\end{document}
%</driver>
% \fi
%
% %%%%%%%%%%%%%%%%%%%%%%%%%%%%%%%%%%%%%%%%%%%%%%%%%%%%%%%%%%%%%%%%%%%%%%%%%%%%%%
% %%%%%%%%%%%%%%%%%%%%%%%%%%%%%%%%%%%%%%%%%%%%%%%%%%%%%%%%%%%%%%%%%%%%%%%%%%%%%%
% \section{Sample}
%\iffalse
%<*samplemain>
%\fi
%
% The following presents a sample document
% with two chapters, two parts, a title page,
% a compile flag as well as three forwarding files to set the flag.
% It consists of eight |.tex| files:
% \begin{center}
% \begin{tabular}{ll}
% |cdocsamp.tex|&main file\\
% |cdocsch1.tex|&include file for chapter 1\\
% |cdocsch2.tex|&include file for chapter 2\\
% |cdocspt3.tex|&include file for part 3\\
% |cdocspt4.tex|&include file for part 4\\
% |cdocsdrf.tex|&forwarding file for main file in draft mode\\
% |cdocsfi1.tex|&forwarding file for final version of chapter 1\\
% |cdocsfi2.tex|&forwarding file for final version of chapter 2\\
% \end{tabular}
% \end{center}
% Each of the eight files can be compiled directly by the \LaTeX{} compiler.
%
% %%%%%%%%%%%%%%%%%%%%%%%%%%%%%%%%%%%%%%
% \paragraph{Main File.}
%
% The main file is called |cdocsamp.tex|.
%
% Load the \textsf{childdoc} definitions and
% declare the filename for the main document:
%    \begin{macrocode}
\input{childdoc.def}
\childdocmain{}
%    \end{macrocode}

% Optional override for |\version| flag:
%    \begin{macrocode}
%%\ifchilddoc\else\providecommand{\version}{draft}\fi
%    \end{macrocode}

% Define the default values for the |\version| flag
% (|final| for the main file and |draft| for childs):
%    \begin{macrocode}
\ifchilddoc
\providecommand{\version}{draft}
\else
\providecommand{\version}{final}
\fi
%    \end{macrocode}

% Load the standard document class:
%    \begin{macrocode}
\documentclass[12pt]{article}
%    \end{macrocode}

% Start the document body:
%    \begin{macrocode}
\begin{document}
%    \end{macrocode}

% Declare a title page.
% Print title, part of document being processed and version flag:
%    \begin{macrocode}
\addtocounter{page}{-1}
\begin{center}
{\LARGE\bfseries{}childdoc example\par}
\vspace{1cm}
\ifchilddoc
\ifchilddocmanual part\else chapter\fi:
`\childdocname' of `\childdocjob'\par
\else
main document: `\childdocjob'\par
\fi
version: \version\par
\end{center}
\newpage
%    \end{macrocode}

% Manually include selected file,
% otherwise process as usual:
%    \begin{macrocode}
\ifchilddocmanual
\section*{part `\childdocname'}
\input{\childdocname}
\else
%    \end{macrocode}

% Include the two chapters:
%    \begin{macrocode}
\include{cdocsch1}
\include{cdocsch2}
%    \end{macrocode}

% Include the two parts unless only chapters should be displayed:
%    \begin{macrocode}
\ifchilddoc\else
\section{part three}
\input{cdocspt3}
\section{part four}
\input{cdocspt4}
\fi
%    \end{macrocode}

% Process as usual until here:
%    \begin{macrocode}
\fi
%    \end{macrocode}

% End of document body:
%    \begin{macrocode}
\end{document}
%    \end{macrocode}
%\iffalse
%</samplemain>
%\fi
%
% %%%%%%%%%%%%%%%%%%%%%%%%%%%%%%%%%%%%%%
% \paragraph{Chapter Include Files.}
%
% The include files are called |cdocsch1.tex| and |cdocsch2.tex|.
%
%\iffalse
%<*samplechap1|samplechap2>
%\fi

% Optional override for |\version| flag:
%    \begin{macrocode}
%%\providecommand{\version}{final}
%    \end{macrocode}

% Include the main document:
%    \begin{macrocode}
\input{childdoc.def}
\childdocof{cdocsamp}
%    \end{macrocode}

%\iffalse
%</samplechap1|samplechap2>
%\fi
%
%\iffalse
%<*samplechap1>
%\fi
% Some text for chapter 1:
%    \begin{macrocode}
\section{one}
some text in chapter one
%    \end{macrocode}

%\iffalse
%</samplechap1>
%\fi
% Some text for chapter 2:
%\iffalse
%<*samplechap2>
%\fi
%    \begin{macrocode}
\section{two}
more text in chapter two
%    \end{macrocode}

%\iffalse
%</samplechap2>
%\fi
%
% %%%%%%%%%%%%%%%%%%%%%%%%%%%%%%%%%%%%%%
% \paragraph{Part Include Files.}
%
% The include files are called |cdocspt3.tex| and |cdocspt4.tex|.
%
%\iffalse
%<*samplepart3|samplepart4>
%\fi

% Optional override for |\version| flag:
%    \begin{macrocode}
%%\providecommand{\version}{final}
%    \end{macrocode}

% Include the main document:
%    \begin{macrocode}
\input{childdoc.def}
\childdocby{cdocsamp}
%    \end{macrocode}

%\iffalse
%</samplepart3|samplepart4>
%\fi
%
%\iffalse
%<*samplepart3>
%\fi
% Some text for part 3:
%    \begin{macrocode}
some text in part three
%    \end{macrocode}

%\iffalse
%</samplepart3>
%\fi
% Some text for part 4:
%\iffalse
%<*samplepart4>
%\fi
%    \begin{macrocode}
more text in part four
%    \end{macrocode}

%\iffalse
%</samplepart4>
%\fi
%
% %%%%%%%%%%%%%%%%%%%%%%%%%%%%%%%%%%%%%%
% \paragraph{Forwarding for a Complete Draft.}
%
% The following forwarding file |cdocsdrf.tex|
% compiles the main document in draft mode:
%\iffalse
%<*sampledraft>
%\fi
%    \begin{macrocode}
\def\version{draft}
\input{childdoc.def}
\childdocforward{cdocsamp}
%    \end{macrocode}

%\iffalse
%</sampledraft>
%\fi
%
% %%%%%%%%%%%%%%%%%%%%%%%%%%%%%%%%%%%%%%
% \paragraph{Forwarding for Final Version of the Chapters.}
%
% The following forwarding files |cdocsfn1.tex| and |cdocsfn2.tex|
% (with identical content)
% compile the final versions of the child documents
% |cdocsch1.tex| and |cdocsch2.tex|, respectively:
%\iffalse
%<*samplefinal>
%\fi
%    \begin{macrocode}
\def\version{final}
\input{childdoc.def}
\childdocforwardprefix[cdocsamp]{cdocsfn}{cdocsch}
%    \end{macrocode}

%\iffalse
%</samplefinal>
%\fi
%
% %%%%%%%%%%%%%%%%%%%%%%%%%%%%%%%%%%%%%%
% \paragraph{Command Line Processing.}
%
% The following three command lines generate the output files
% |cdocscld|, |cdocscl1| and |cdocscl2|
% which should be identical to
% |cdocsdrf|, |cdocsch1| and |cdocsfn2|, respectively:
% \begin{center}
% \begin{tabular}{l}
% |latex -jobname cdocscld \|\\
% |  "\def\version{draft}\input{childdoc.def}\childdocforward{cdocsamp}"|\\
% |latex -jobname cdocscl1 \|\\
% |  "\input{childdoc.def}\childdocforward[cdocsamp]{cdocsch1}"|\\
% |latex -jobname cdocscl2 \|\\
% |  "\def\version{final}\input{childdoc.def}\childdocforward{cdocsch2}"|
% \end{tabular}
% \end{center}
% Note that the trailing backslash on each first line
% merely continues the input to the second line
% (for convenient cut ant paste).
% Furthermore, the command |latex| can be replaced by any
% of its alternative versions such as |pdflatex|.
%
% %%%%%%%%%%%%%%%%%%%%%%%%%%%%%%%%%%%%%%%%%%%%%%%%%%%%%%%%%%%%%%%%%%%%%%%%%%%%%%
% %%%%%%%%%%%%%%%%%%%%%%%%%%%%%%%%%%%%%%%%%%%%%%%%%%%%%%%%%%%%%%%%%%%%%%%%%%%%%%
% \section{Implementation}
%\iffalse
%<*package>
%\fi
%
% This section describes the definitions file |childdoc.def|.

% The definitions cannot be loaded using |\usepackage| or |\RequirePackage|
% which has a mechanism to prevent loading a style file more than once.
% When loading the definitions by means of |\input|
% multiple instances have to be prevented manually:
%\iffalse
%This code needs to be before the `\ProvidesFile' directive
%which is defined at the beginning of this file.
%Therefore it is also placed there and commented out here.
%</package>
%<*discard>
%\fi
%    \begin{macrocode}
\ifdefined\childdocmain\endinput\fi
%    \end{macrocode}
%\iffalse
%</discard>
%<*package>
%\fi
%
% \macro{\ifchilddoc}
% \macro{\ifchilddocmanual}
% The conditional |\ifchilddoc| tells whether a
% child (true) or main (false) document is being compiled.
% The conditional |\ifchilddocmanual| tells whether
% the |\includeonly| mechanism is used (false) or
% the selection of child files must be performed manually (true).
% The definitions initialise to false:
%    \begin{macrocode}
\newif\ifchilddoc
\newif\ifchilddocmanual
%    \end{macrocode}

% \macro{\childdocname}
% \macro{\childdocjob}
% The macro |\childdocname| stores the name of the main document
% to be compiled. The macro |\childdocjob| stores the name of
% the document on which the \LaTeX{} compiler was originally invoked.
% The content of |\jobname| cannot be compared
% to filenames specified in the source due to different catcodes.
% The following code rescans |\jobname|, stores the result
% in |\childdocname| and saves a copy in |\childdocjob|:
%    \begin{macrocode}
\edef\childdocname{\scantokens\expandafter{\jobname\noexpand}}
\let\childdocjob\childdocname
%    \end{macrocode}

% \macro{\childdocdisable}
% The macro |\childdocdisable| prevents the main file
% from being processed more than once.
% At this stage, the main document command |\childdocmain|
% is assumed to be called once again where it should do nothing.
% Any subsequent call to it should prevent
% a secondary processing of the main document
% It overwrites the forwarding commands
% |\childdocof| and |\childdocforward|
% with empty macros to prevent further inclusions of the main document:
%    \begin{macrocode}
\newcommand{\childdocdisable}
{
  \renewcommand{\childdocmain}[1]{\renewcommand{\childdocmain}[1]{\endinput}}
  \renewcommand{\childdocof}[1]{}
  \renewcommand{\childdocby}[2][]{}
  \renewcommand{\childdocforward}[2][]{}
  \renewcommand{\childdocdisable}{}
}
%    \end{macrocode}

% \macro{\childdocmain}
% The macro |\childdocmain| is to be called at the top of the main file
% with nothing or the main filename (without extension) as argument.
% First, it breaks loops.
% If the argument is not empty and does not match |\childdocname|
% (which is set by the first inclusion of |childdoc.def|),
% |\ifchilddoc| is set to true, |\includeonly| is applied to the child file
% and |\jobname| is set to the main file
% (for proper handling of |.aux| files):
%    \begin{macrocode}
\newcommand{\childdocmain}[1]
{
  \childdocdisable\childdocmain{}
  \if?#1?\else
    \begingroup
      \def\childdoctmp{#1}
      \ifx\childdoctmp\childdocname
        \def\childdoctmp{}
      \else
        \def\childdoctmp
        {
          \childdoctrue
          \includeonly{\childdocname}
          \def\childdocjob{#1}
          \def\jobname{#1}
        }
      \fi
      \expandafter
    \endgroup
    \childdoctmp
  \fi
}
%    \end{macrocode}

% \macro{\childdocof}
% The command |\childdocof| redirects
% compilation to the main file |#1|.
%    \begin{macrocode}
\newcommand{\childdocof}[1]
{
  \childdocdisable
  \childdoctrue
  \includeonly{\childdocname}
  \def\jobname{#1}
  \def\childdocjob{#1}
  \input{#1}
}
%    \end{macrocode}

% \macro{\childdocby}
% The command |\childdocby| ....
%    \begin{macrocode}
\newcommand{\childdocby}[2][]
{
  \childdocdisable
  \childdoctrue
  \childdocmanualtrue
  \if?#1?\else
    \def\jobname{#2}
  \fi
  \def\childdocjob{#2}
  \input{#2}
  \endinput
}
%    \end{macrocode}

% \macro{\childdocforward}
% The command |\childdocforward| redirects
% compilation to the main file or
% (if the optional argument is given) a child file.
% Parameters are set as if the main file
% or a child file starting with |\childdocof| was compiled.
% Then compilation is handed over to the main file:
%    \begin{macrocode}
\newcommand{\childdocforward}[2][]
{
  \begingroup
    \if?#1?
      \def\childdoctmp
      {
        \def\childdocname{#2}
        \def\childdocjob{#2}
        \def\jobname{#2}
        \input{#2}
        \endinput
      }
    \else
      \def\childdoctmp
      {
        \childdocdisable
        \def\childdocname{#2}
        \childdoctrue
        \includeonly{#2}
        \def\childdocjob{#1}
        \def\jobname{#1}
        \input{#1}
        \endinput
      }
    \fi
    \expandafter
  \endgroup
  \childdoctmp
}
%    \end{macrocode}

% \macro{\childdocforwardprefix}
% The command |\childdocforwardprefix| redirects
% compilation to the main or a child file by means of a pattern.
% The prefix |#1| in the current filename is replaced by |#2|
% and the suffix of the current filename is kept
% (it is assumed that the filename does not contain the substring `|~~~|'
% which is used as a delimiter).
% Compilation is handed over to the new file by |\childdocforward|:
%    \begin{macrocode}
\newcommand{\childdocforwardprefix}[3][]
{
  \begingroup
    \def\childdocextract #2##1~~~{\def\childdoctmp{\childdocforward[#1]{#3##1}}}
    \expandafter\childdocextract\childdocname~~~
    \expandafter
  \endgroup
  \childdoctmp
}
%    \end{macrocode}

% \macro{\childdoc}
% The deprecated macro |\childdoc| is a legacy version of |\childdocmain|:
%    \begin{macrocode}
\newcommand{\childdoc}{\childdocmain}
%    \end{macrocode}

% \macro{\childdocredirect}
% The deprecated macro |\childdocredirect| is a legacy version
% of |\childdocforward| and |\childdocforwardprefix|:
%    \begin{macrocode}
\newcommand{\childdocredirect}[2][]
{
  \begingroup
    \if?#1?
      \def\childdoctmp{\childdocforward{#2}}
    \else
      \def\childdoctmp{\childdocforwardprefix{#1}{#2}}
    \fi
    \expandafter
  \endgroup
  \childdoctmp
}
%    \end{macrocode}

%\iffalse
%</package>
%\fi
%
\endinput
\childdocforward[|\textit{main}|]{|\textit{dest}|}"|
\end{center}
%
Here \textit{target} is the name of the output file,
\textit{main} is the name of the main file
and \textit{dest} is the name of the main or child file to be processed
(all filenames without extensions).
The optional argument \textit{main} can be omitted
if \textit{main} matches \textit{dest}.
Optionally, compilation \textit{flags} can be defined via |\def| commands.
This command line makes the \TeX{} engine believe
it is compiling the file \textit{target}
whose content is specified as the latter parameter.
The provided code then forwards the processing to
\textit{main} or \textit{dest} as described in \secref{sec:forward}.

%%%%%%%%%%%%%%%%%%%%%%%%%%%%%%%%%%%%%%%%%%%%%%%%%%%%%%%%%%%%%%%%%%%%%%%%%%%%%%%%
\subsection{Include by Input}
\label{sec:input}

Including child documents by |\include| has some restrictions by design.
Most notably, the content of a child document always occupies
its own set of pages; pages cannot be shared between child documents.
Usually, this behaviour makes perfect sense
because each child document contain an essential part of the document.
However, in some situations it may be desirable to compose
a document from a collection of parts
without having mandatory page breaks between then.
For this case, the package
provides a mechanism to include parts
by |\input| which can also be processed individually.
However, by construction this mechanism
requires manual handling of the content to be output.

%%%%%%%%%%%%%%%%%%%%%%%%%%%%%%%%%%%%%%%%
\DescribeMacro{\ifchilddocmanual}
The main file should be prepared as usual, see \secref{sec:include}.
However, the document body must make a distinction
between processing of an individual part and of the main document, e.g.:
%
\begin{center}
\begin{tabular}{l}
|\ifchilddocmanual|\\
|\input{\childdocname}|\\
|\||else|\\
\textit{document body with }|\input{|\textit{part}|}|\\
|\||fi|
\end{tabular}
\end{center}
%
The conditional |\ifchilddocmanual| is true whenever
a part to be included by |\input| is being compiled,
and the name of the part is stored in |\childdocname|.

%%%%%%%%%%%%%%%%%%%%%%%%%%%%%%%%%%%%%%%%
\DescribeMacro{\childdocby}
Each part to be included by |\input| should start with:
%
\begin{center}
\begin{tabular}{l}
|% \iffalse
%
% childdoc.dtx Copyright (C) 2017-2018 Niklas Beisert
%
% This work may be distributed and/or modified under the
% conditions of the LaTeX Project Public License, either version 1.3
% of this license or (at your option) any later version.
% The latest version of this license is in
%   http://www.latex-project.org/lppl.txt
% and version 1.3 or later is part of all distributions of LaTeX
% version 2005/12/01 or later.
%
% This work has the LPPL maintenance status `maintained'.
%
% The Current Maintainer of this work is Niklas Beisert.
%
% This work consists of the files childdoc.dtx and childdoc.ins
% and the derived files childdoc.def and cdocsamp.tex with
% cdocsch1.tex, cdocsch2.tex, cdocsdrf.tex, cdocsfn1.tex, cdocsfn2.tex.
%
%<package>\ifdefined\childdocmain\endinput\fi
%<package>\ProvidesFile{childdoc.def}[2018/12/30 v2.0 child document driver]
%<samplemain>\ProvidesFile{cdocsamp.tex}[2018/12/30 v2.0 sample for childdoc]
%<*driver>
%\ProvidesFile{childdoc.drv}[2018/12/30 v2.0 childdoc reference manual file]
\PassOptionsToClass{10pt,a4paper}{article}
\documentclass{ltxdoc}

\usepackage[margin=35mm]{geometry}
\usepackage{hyperref}
\usepackage{hyperxmp}
\usepackage[usenames]{color}

\hypersetup{colorlinks=true}
\hypersetup{pdfstartview=FitH}
\hypersetup{pdfpagemode=UseNone}
\hypersetup{pdfsource={}}
\hypersetup{pdflang={en-UK}}
\hypersetup{pdfcopyright={Copyright 2017-2018 Niklas Beisert.
  This work may be distributed and/or modified under the
  conditions of the LaTeX Project Public License, either version 1.3
  of this license or (at your option) any later version.}}
\hypersetup{pdflicenseurl={http://www.latex-project.org/lppl.txt}}
\hypersetup{pdfcontactaddress={ETH Zurich, ITP, HIT K,
  Wolfgang-Pauli-Strasse 27}}
\hypersetup{pdfcontactpostcode={8093}}
\hypersetup{pdfcontactcity={Zurich}}
\hypersetup{pdfcontactcountry={Switzerland}}
\hypersetup{pdfcontactemail={nbeisert@itp.phys.ethz.ch}}
\hypersetup{pdfcontacturl={http://people.phys.ethz.ch/\xmptilde nbeisert/}}

\newcommand{\secref}[1]{\hyperref[#1]{section \ref*{#1}}}

\parskip1ex
\parindent0pt
\let\olditemize\itemize
\def\itemize{\olditemize\parskip0pt}

\begin{document}

\title{The \textsf{childdoc} Package}
\hypersetup{pdftitle={The childdoc Package}}
\author{Niklas Beisert\\[2ex]
  Institut f\"ur Theoretische Physik\\
  Eidgen\"ossische Technische Hochschule Z\"urich\\
  Wolfgang-Pauli-Strasse 27, 8093 Z\"urich, Switzerland\\[1ex]
  \href{mailto:nbeisert@itp.phys.ethz.ch}
  {\texttt{nbeisert@itp.phys.ethz.ch}}}
\hypersetup{pdfauthor={Niklas Beisert}}
\hypersetup{pdfsubject={Manual for the LaTeX2e Package childdoc}}
\date{30 December 2018, \textsf{v2.0}}
\maketitle

\begin{abstract}\noindent
\textsf{childdoc} is a \LaTeXe{} package
that enables the direct compilation
of document sections included by |\include|
to individual files.
\end{abstract}

\begingroup
\parskip0ex
\tableofcontents
\endgroup

%%%%%%%%%%%%%%%%%%%%%%%%%%%%%%%%%%%%%%%%%%%%%%%%%%%%%%%%%%%%%%%%%%%%%%%%%%%%%%%%
%%%%%%%%%%%%%%%%%%%%%%%%%%%%%%%%%%%%%%%%%%%%%%%%%%%%%%%%%%%%%%%%%%%%%%%%%%%%%%%%
\section{Introduction}

\LaTeX{} provides a mechanism to structure a large document (such as a book)
into a main file and several child files (containing the chapters)
using the |\include| command.
This mechanism is beneficial for documents
which span hundreds of pages in order to
make the source file(s) more manageable.
Moreover, compilation can be restricted to
selected child files by means of the |\includeonly| command.
The latter feature can be used to reduce the compilation time while editing
(this was significantly more useful in the earlier days of \LaTeX{})
or to generate a smaller document which is easier to navigate.
Another application of |\includeonly| is to generate
documents consisting of selected parts of the complete document.

However, there are a few drawbacks of the plain |\include| mechanism:
\begin{itemize}
\item
The child files cannot be compiled on their own,
they can only be compiled via the main file.
A naive editing environment
(such as a text editor with an option
to have the current file processed by \LaTeX)
may require one to switch to the main file before compiling;
attempting to compile the child file produces errors.
\item
The main file must be modified (each time)
to adjust the |\includeonly| command
to the present needs. This easily leaves the main file in a messy state.
\item
The generated document will always carry the filename
of the main document. This is inconvenient if
several child files are to be compiled and
to be kept for distribution.
\end{itemize}

The present package provides a simple interface
to make child files individually compilable by \LaTeX{}.
Compiling a child file then has the same effect as compiling
the main file with an |\includeonly| command
to select the appropriate child.
Moreover the generated document will carry the name of the child
rather than the main file.
This resolves all three above issues.

This feature is meant to make the editing of books,
thesis documents and lecture notes somewhat more convenient.
However, the package can also be used efficiently for
composing a series of documents (such as exercise sheets)
which are typically distributed individually.
It then assists the author in generating the individual documents
(potentially in different versions)
as well as a document containing the collected series.
Another application is in developing style files
or other kinds of included material
where compilation of the style file could redirect
to a sample or test file.

%%%%%%%%%%%%%%%%%%%%%%%%%%%%%%%%%%%%%%%%%%%%%%%%%%%%%%%%%%%%%%%%%%%%%%%%%%%%%%%%
%%%%%%%%%%%%%%%%%%%%%%%%%%%%%%%%%%%%%%%%%%%%%%%%%%%%%%%%%%%%%%%%%%%%%%%%%%%%%%%%
\section{Usage}

First of all, the package \textsf{childdoc} is \emph{not} a standard
\LaTeXe{} |.sty| style file! Therefore it needs to be invoked in
a non-standard way.

%%%%%%%%%%%%%%%%%%%%%%%%%%%%%%%%%%%%%%%%%%%%%%%%%%%%%%%%%%%%%%%%%%%%%%%%%%%%%%%%
\subsection{Included Files}
\label{sec:include}

%%%%%%%%%%%%%%%%%%%%%%%%%%%%%%%%%%%%%%%%
\DescribeMacro{\childdocmain}
To use the package, add the commands
\begin{center}
\begin{tabular}{l}
|\input{childdoc.def}|\\
|\childdocmain{}|\\
\end{tabular}
\end{center}
at the very top of the main \LaTeX{} file,
in particular \emph{before} the |\documentclass| statement!
The argument of |\childdocmain| should be left empty
(but it must be present).

%%%%%%%%%%%%%%%%%%%%%%%%%%%%%%%%%%%%%%%%
\DescribeMacro{\childdocof}
Furthermore, add the commands
\begin{center}
\begin{tabular}{l}
|\input{childdoc.def}|\\
|\childdocof{|\textit{main}|}|\\
\end{tabular}
\end{center}
at the top of every child file \textit{child}
which is included by |\include{|\textit{child}|}|
from within the main file
(or at least for those files to be compiled individually).
The argument \textit{main} must be the filename of the main file.

There are a couple of
considerations in setting up the main and child documents:

%%%%%%%%%%%%%%%%%%%%%%%%%%%%%%%%%%%%%%%%
\paragraph{Restrictions.}

Please note the following restrictions:
\begin{itemize}
\item
|\childdocmain| must be called with one argument \textit{main}
to ensure compatibility with earlier version of the package.
It must either be empty (|\childdocmain{}|)
or precisely match the filename of the main file in which it is specified.
See \secref{sec:detection} for further information.
\item
The filename \textit{main} must be specified without the |.tex| extension.
\item
The filename \textit{main} is case sensitive
(even in case-insensitive file systems)
due to internal string comparison.
\item
The argument \textit{main} should be fully expanded, it cannot be a macro.
\item
Subdirectories and special characters should be avoided in filenames.
\item
The command |\childdocmain{|\textit{main}|}| must be followed by a whitespace.
It should not be followed immediately by another command
or by a comment mark `|%|'.
This is because the \TeX{} parser reads the token immediately following
the argument of |\childdocmain| and puts it
at the beginning of every child section;
however, a white\-space is ignored.
\end{itemize}

%%%%%%%%%%%%%%%%%%%%%%%%%%%%%%%%%%%%%%%%
\paragraph{Content of Main File.}

It is advisable to place all content in the child files included by |\include|.
Any output contained in the main file will appear in all child documents
unless suppressed manually;
it cannot be suppressed automatically by the |\includeonly| directive
and thus should normally be avoided.
A method to include some content in the main file
by means of conditional processing is described in \secref{sec:conditional}.

%%%%%%%%%%%%%%%%%%%%%%%%%%%%%%%%%%%%%%%%
\paragraph{Page Numbering.}

When only a part of the document is compiled,
the appropriate numbering of pages
(as well as other status parameters)
is determined from the |.aux| files.
The latter contain information from previous passes.
However this information needs to propagate through
all intermediate child documents.
Therefore the page numbering in child documents may well
be inconsistent until the complete document is compiled at least once.

A useful (if unconventional) way to always ensure a consistent
page numbering is to restart the numbering in each child document
and denote the pages by `\textit{child}|.|\textit{page}'
where \textit{child} represents the chapter/section number of the child file.
This can be achieved by the command
|\numberwithin{page}{|\textit{child}|}|
of the \textsf{amsmath} package
where \textit{child} can be |chapter| or |section|
depending on the chosen structuring.
Alternatively, one can modify the macro |\thepage| appropriately
and reset the counter |page| at the start of each child file.

%%%%%%%%%%%%%%%%%%%%%%%%%%%%%%%%%%%%%%%%%%%%%%%%%%%%%%%%%%%%%%%%%%%%%%%%%%%%%%%%
\subsection{Conditional Processing}
\label{sec:conditional}

The package provides a mechanism to compile different versions
of a document. To customise the versions further some conditional processing
can come in handy to distinguish which version is being compiled.
The package provides two macros to describe the compilation context:

%%%%%%%%%%%%%%%%%%%%%%%%%%%%%%%%%%%%%%%%
\DescribeMacro{\ifchilddoc}
The conditional |\ifchilddoc| distinguishes between the compilation of
child documents and the main document:
%
\begin{center}
|\ifchilddoc |\textit{child-code}| |[|\||else |\textit{main-code}]| \||fi|
\end{center}

%%%%%%%%%%%%%%%%%%%%%%%%%%%%%%%%%%%%%%%%
\DescribeMacro{\childdocname}
\DescribeMacro{\childdocjob}
The macro |\childdocname| contains the filename (without extension)
of the main or child file being processed.
Note that |\childdocjob| will always contain the name of the main file.

%%%%%%%%%%%%%%%%%%%%%%%%%%%%%%%%%%%%%%%%
\paragraph{Title Page.}

Conditional processing can be used to include a title or banner page
in the main document when proper precautions are taken.
Importantly, the code in the main file should ensure that the page counter
(as well as other status parameters which are stored in the |.aux| files)
takes the same value after the conditional processing.
Otherwise the page numbers may take divergent values
depending on which part is compiled.

For example, a title page could be declared by:
%
\begin{center}
\begin{tabular}{l}
|\ifchilddoc\||else|\\
|\addtocounter{page}{-1}|\\
\textit{code for title page}\\
|\newpage|\\
|\||fi|
\end{tabular}
\end{center}
%
A banner page for the child documents can be generated by:
%
\begin{center}
\begin{tabular}{l}
|\ifchilddoc|\\
|\addtocounter{page}{-1}|\\
\textit{code for banner page}\\
|\newpage|\\
|\||fi|
\end{tabular}
\end{center}
%
Here one could write a message such as:
\begin{center}
|This is the part \childdocname{} of \childdocjob{}.|
\end{center}

%%%%%%%%%%%%%%%%%%%%%%%%%%%%%%%%%%%%%%%%%%%%%%%%%%%%%%%%%%%%%%%%%%%%%%%%%%%%%%%%
\subsection{Flags}
\label{sec:flags}

The package makes it easy to generate different versions
of the main or child documents.
To this end compilation flags can be defined
and assigned different default values.
They will be particularly useful in conjunction
with the forwarding mechanism described in \secref{sec:forward}.

For example, it may be useful to have a flag |\version|
which can be set to |draft| or |final|.
The document source will contain some conditional code
depending on the value of |\version|.
Suppose further, the flag should default to |final| for the main file
and to |draft| for child files
which is a natural assignment for editing the document.
This is achieved by placing the following code
in the preamble of the main document
(below the |\childdocmain| directive):
%
\begin{center}
\begin{tabular}{l}
|\ifchilddoc|\\
|\providecommand{\version}{draft}|\\
|\||else|\\
|\providecommand{\version}{final}|\\
|\||fi|
\end{tabular}
\end{center}
%
The definition by |\providecommand| makes sure
that previous definitions are not overwritten.
Further statements |\providecommand{\version}{...}|
can thus be added before the above code to override it.

For the main file, one might add a line
(between |\childdocmain| and the above block)
%
\begin{center}
|%\ifchilddoc\||else\providecommand{\version}{draft}\||fi|
\end{center}
%
which can be uncommented to produce a draft version.
Likewise one can add a line to the very top of a child file
(above the |\childdocof{|\textit{main}|}| directive)
%
\begin{center}
|%\providecommand{\version}{final}|
\end{center}
%
which can be uncommented to produce the final version of this child document.

%%%%%%%%%%%%%%%%%%%%%%%%%%%%%%%%%%%%%%%%%%%%%%%%%%%%%%%%%%%%%%%%%%%%%%%%%%%%%%%%
\subsection{Forwarding}
\label{sec:forward}

Different versions of the main or child documents
using compilation flags as described in \secref{sec:flags}
can be (permanently) stored in different files
for convenient compilation, viewing and distribution.
To this end, the package defines a command
to pass on compilation to a different file:

%%%%%%%%%%%%%%%%%%%%%%%%%%%%%%%%%%%%%%%%
\DescribeMacro{\childdocforward}
The command |\childdocforward| redirects processing to
another source file:
%
\begin{center}
\begin{tabular}{l}
|\input{childdoc.def}|\\
|\childdocforward[|\textit{main}|]{|\textit{dest}|}|\\
\end{tabular}
\end{center}
%
The argument \textit{dest} is the destination file
(without extension).
It should be the main file or one of the child files.
Note that further \textsf{childdoc} directives
such as |\childdocof| and |\childdocforward|
in the indicated file will be processed in this form.
The optional argument \textit{main}
passes on directly to the main file \textit{main}
while pretending to compile the child \textit{dest}.
This form behaves as if \textit{dest}
issues |\childdocof{|\textit{main}|}| right away,
and no further \textsf{childdoc} directives will be processed.

%%%%%%%%%%%%%%%%%%%%%%%%%%%%%%%%%%%%%%%%
\DescribeMacro{\...prefix}
In the alternative form |\childdocforwardprefix|,
%
\begin{center}
\begin{tabular}{l}
|\input{childdoc.def}|\\
|\childdocforwardprefix[|\textit{main}|]{|\textit{prefix}|}{|\textit{dest}|}|
\end{tabular}
\end{center}
%
the destination file is determined by a pattern
depending on the current file:
To make this work, the current file must be called
`{\textit{prefix}\hspace{0.2em}\textit{suffix}}'
with \textit{prefix} matching precisely the argument.
Processing is then passed on to the file
`{\textit{dest}\hspace{0.2em}\textit{suffix}}'.
Surely, the same effect is achieved by
directly specifying the
argument `{\textit{dest}\hspace{0.2em}\textit{suffix}}'
in the first form.
However, that requires to set up a different file
for each child. With the alternative form of the command
all these files can have exactly the same content
which simplifies setting them up and maintaining them.

For example, the following file |draft.tex|
with a compilation flag |\version| as described in \secref{sec:flags}
compiles the main document as a draft:
%
\begin{center}
\begin{tabular}{l}
|\def\version{draft}|\\
|\input{childdoc.def}|\\
|\childdocforward{|\textit{main}|}|
\end{tabular}
\end{center}
%
Likewise, the following files |final|\textit{nn}|.tex|
compile the final version of the child document
|child|\textit{nn}|.tex|:
%
\begin{center}
\begin{tabular}{l}
|\def\version{final}|\\
|\input{childdoc.def}|\\
|\childdocforwardprefix{final}{child}|
\end{tabular}
\end{center}
%

Note that when several versions of a main file and/or of each child file
are to be generated, it may be convenient to set up a |Makefile| or
shell script to automatise the process.

%%%%%%%%%%%%%%%%%%%%%%%%%%%%%%%%%%%%%%%%%%%%%%%%%%%%%%%%%%%%%%%%%%%%%%%%%%%%%%%%
\subsection{Command Line Processing}
\label{sec:commandline}

The effect of redirection files can also be achieved by invoking
the \LaTeX{} compiler with a more elaborate command line.
Most conveniently this should be done as part
of a shell script or a |Makefile|.

When using \textsf{childdoc} in the main file, the following
command lines effectively perform a redirection
(note that depending on the shell being used,
backslashes may have to be doubled: `|\|' $\to$ `|\\|'):
%
\begin{center}
|... -jobname "|\textit{target}|" |\\|"|[\textit{flags}]%
|\input{childdoc.def}\childdocforward[|\textit{main}|]{|\textit{dest}|}"|
\end{center}
%
Here \textit{target} is the name of the output file,
\textit{main} is the name of the main file
and \textit{dest} is the name of the main or child file to be processed
(all filenames without extensions).
The optional argument \textit{main} can be omitted
if \textit{main} matches \textit{dest}.
Optionally, compilation \textit{flags} can be defined via |\def| commands.
This command line makes the \TeX{} engine believe
it is compiling the file \textit{target}
whose content is specified as the latter parameter.
The provided code then forwards the processing to
\textit{main} or \textit{dest} as described in \secref{sec:forward}.

%%%%%%%%%%%%%%%%%%%%%%%%%%%%%%%%%%%%%%%%%%%%%%%%%%%%%%%%%%%%%%%%%%%%%%%%%%%%%%%%
\subsection{Include by Input}
\label{sec:input}

Including child documents by |\include| has some restrictions by design.
Most notably, the content of a child document always occupies
its own set of pages; pages cannot be shared between child documents.
Usually, this behaviour makes perfect sense
because each child document contain an essential part of the document.
However, in some situations it may be desirable to compose
a document from a collection of parts
without having mandatory page breaks between then.
For this case, the package
provides a mechanism to include parts
by |\input| which can also be processed individually.
However, by construction this mechanism
requires manual handling of the content to be output.

%%%%%%%%%%%%%%%%%%%%%%%%%%%%%%%%%%%%%%%%
\DescribeMacro{\ifchilddocmanual}
The main file should be prepared as usual, see \secref{sec:include}.
However, the document body must make a distinction
between processing of an individual part and of the main document, e.g.:
%
\begin{center}
\begin{tabular}{l}
|\ifchilddocmanual|\\
|\input{\childdocname}|\\
|\||else|\\
\textit{document body with }|\input{|\textit{part}|}|\\
|\||fi|
\end{tabular}
\end{center}
%
The conditional |\ifchilddocmanual| is true whenever
a part to be included by |\input| is being compiled,
and the name of the part is stored in |\childdocname|.

%%%%%%%%%%%%%%%%%%%%%%%%%%%%%%%%%%%%%%%%
\DescribeMacro{\childdocby}
Each part to be included by |\input| should start with:
%
\begin{center}
\begin{tabular}{l}
|\input{childdoc.def}|\\
|\childdocby{|\textit{main}|}|\\
\end{tabular}
\end{center}
%
The directive |\childdocby| is similar to |\childdocof|
described in \secref{sec:include},
but the subsequent selection of content must be done manually.
To that end, both |\ifchilddoc| and |\ifchilddocmanual|
will be true upon processing of a part,
and the name of the part is stored in |\childdocname|.
Note that |\jobname| will be set to the filename of the current part
so that each part receives an individual |.aux| file
that does not interfere with the |.aux| file(s) of the main document.
This behaviour can be altered by the alternative form
|\childdocby[*]{|\textit{main}|}| (with a non-empty optional argument)
which uses the |.aux| file of the main document
by setting |\jobname| to \textit{main}.

%%%%%%%%%%%%%%%%%%%%%%%%%%%%%%%%%%%%%%%%%%%%%%%%%%%%%%%%%%%%%%%%%%%%%%%%%%%%%%%%
\subsection{Driver Development}
\label{sec:driver}

The \textsf{childdoc} mechanism can also be use for the development
of definition files such as \LaTeX{} styles or classes.
This case differs from the above setup with multiple parts
included by |\include| in that no |\includeonly| should be invoked.
This can be achieved by starting the include file
(before |\ProvidesPackage|) with:
%
\begin{center}
\begin{tabular}{l}
|\input{childdoc.def}|\\
|\childdocforward{|\textit{main}|}|\\
\end{tabular}
\end{center}
%
or alternatively with:
%
\begin{center}
\begin{tabular}{l}
|\input{childdoc.def}|\\
|\childdocby{|\textit{main}|}|\\
\end{tabular}
\end{center}
%
Both forms have slightly different effects as described above.
The main file is prepared as usual, see \secref{sec:include}.

%%%%%%%%%%%%%%%%%%%%%%%%%%%%%%%%%%%%%%%%%%%%%%%%%%%%%%%%%%%%%%%%%%%%%%%%%%%%%%%%
\subsection{Legacy Detection}
\label{sec:detection}

The directive |\childdocmain| in the main file can detect
whether the complete document or merely a child is to be compiled
even without using the directive |\childdocof|.
This method is deprecated because it is less robust
and there is no compelling reason to use it;
it is merely provided for backward compatibility
and it may be removed in future versions.

If the detection mechanism is to be used,
it is mandatory to correctly specify
the filename of the main file as the argument of |\childdocmain|:
%
\begin{center}
\begin{tabular}{l}
|\input{childdoc.def}|\\
|\childdocmain{|\textit{main}|}|\\
\end{tabular}
\end{center}
%
If |\jobname| does not match the argument \textit{main} of |\childdocmain|,
it is assumed that |\jobname| points to the child file to be compiled.
When using |\childdocmain| with the main file specified as argument,
it suffices to start a child file
with just |\input{|\textit{main}|}|
without loading of the package and using |\childdocof|.
If instead all processing is done
with the appropriate \textsf{childdoc} directives,
the argument of \textit{main} of |\childdocmain| can be empty.

An alternative version of the command line processing described
in \secref{sec:commandline} using the detection mechanism reads:
%
\begin{center}
|... -jobname "|\textit{target}|" "|[\textit{flags}]%
[|\def\jobname{|\textit{dest}|}|]|\input{|\textit{main}|}"|
\end{center}

%%%%%%%%%%%%%%%%%%%%%%%%%%%%%%%%%%%%%%%%%%%%%%%%%%%%%%%%%%%%%%%%%%%%%%%%%%%%%%%%
\subsection{Manual Code}
\label{sec:manual}

In case one cannot be certain whether the definitions file |childdoc.def|
is installed on the target \TeX{} distribution
and one prefers not to ship it,
it is conceivable to paste a few relevant commands into the sources.

To that end, drop all statements |\input{childdoc.def}|
and perform the replacements as outlined below.
Instead of |\childdocmain{|\textit{main}|}| add the following code
to the top of the main file:
%
\begin{center}
\begin{tabular}{l}
|\||ifdefined\childdocname\endinput\||fi\newif\ifchilddoc|\\
|\edef\childdocname{\scantokens\expandafter{\jobname\noexpand}}|\\
|\def\childdocmain{|\textit{main}|}\||ifx\childdocmain\childdocname\||else|\\
|\childdoctrue\includeonly{\childdocname}\let\jobname\childdocmain\||fi|\\
\end{tabular}
\end{center}
%
Instead of |\childdocof{|\textit{main}|}| just include the main file
at the top of each child file:
%
\begin{center}
|\input{|\textit{main}|}|
\end{center}
%
A simple redirection |\childdocforward{|\textit{dest}|}| is achieved by:
%
\begin{center}
|\def\jobname{|\textit{dest}|}\input{\jobname}|
\end{center}
%
The redirection with prefix
|\childdocforwardprefix[|\textit{prefix}|]{|\textit{dest}|}|
is accomplished by:
%
\begin{center}
\begin{tabular}{l}
|{\edef\jobname{\scantokens\expandafter{\jobname\noexpand}}|\\
|\def\redirectjob |\textit{prefix}|#1~~~{\gdef\jobname{|\textit{dest}|#1}}|\\
|\expandafter\redirectjob\jobname~~~}\input{\jobname}|
\end{tabular}
\end{center}

In an alternative approach,
child documents can be compiled by a specific command line
without additional code or specific definitions:
%
\begin{center}
|... -jobname "|\textit{target}|" "|[\textit{flags}]%
|\includeonly{|\textit{dest}|}\input{|\textit{main}|}"|
\end{center}
%

%%%%%%%%%%%%%%%%%%%%%%%%%%%%%%%%%%%%%%%%%%%%%%%%%%%%%%%%%%%%%%%%%%%%%%%%%%%%%%%%
%%%%%%%%%%%%%%%%%%%%%%%%%%%%%%%%%%%%%%%%%%%%%%%%%%%%%%%%%%%%%%%%%%%%%%%%%%%%%%%%
\section{Information}

%%%%%%%%%%%%%%%%%%%%%%%%%%%%%%%%%%%%%%%%%%%%%%%%%%%%%%%%%%%%%%%%%%%%%%%%%%%%%%%%
\subsection{Copyright}

Copyright \copyright{} 2017--2018 Niklas Beisert

This work may be distributed and/or modified under the
conditions of the \LaTeX{} Project Public License, either version 1.3
of this license or (at your option) any later version.
The latest version of this license is in
  \url{http://www.latex-project.org/lppl.txt}
and version 1.3 or later is part of all distributions of \LaTeX{}
version 2005/12/01 or later.

This work has the LPPL maintenance status `maintained'.

The Current Maintainer of this work is Niklas Beisert.

This work consists of the files |README.txt|, |childdoc.ins| and |childdoc.dtx|
as well as the derived files |childdoc.def|, |cdocsamp.tex|
with |cdocsch1.tex|, |cdocsch2.tex|, |cdocspt3.tex|, |cdocspt4.tex|,
|cdocsdrf.tex|, |cdocsfn1.tex|, |cdocsfn2.tex|
as well as |childdoc.pdf|.

%%%%%%%%%%%%%%%%%%%%%%%%%%%%%%%%%%%%%%%%%%%%%%%%%%%%%%%%%%%%%%%%%%%%%%%%%%%%%%%%
\subsection{Files and Installation}

The package consists of the files:
%
\begin{center}
\begin{tabular}{ll}
    |README.txt|   & readme file \\
    |childdoc.ins| & installation file \\
    |childdoc.dtx| & source file \\
    |childdoc.def| & definition file \\
    |cdocsamp.tex| & sample main file \\
    |cdocsch1.tex| & sample include file \\
    |cdocsch2.tex| & sample include file \\
    |cdocspt3.tex| & sample part file \\
    |cdocspt4.tex| & sample part file \\
    |cdocsdrf.tex| & sample redirection file \\
    |cdocsfn1.tex| & sample redirection file \\
    |cdocsfn2.tex| & sample redirection file \\
    |childdoc.pdf| & manual
\end{tabular}
\end{center}
%
The distribution consists of the files
|README.txt|, |childdoc.ins| and |childdoc.dtx|.
%
\begin{itemize}
\item
Run (pdf)\LaTeX{} on |childdoc.dtx|
to compile the manual |childdoc.pdf| (this file).
\item
Run \LaTeX{} on |childdoc.ins| to create the definitions file |childdoc.def|
and the sample |cdocsamp.tex| with include files
|cdocsch1.tex|, |cdocsch2.tex|, |cdocspt3.tex|, |cdocspt4.tex|,
|cdocsdrf.tex|, |cdocsfn1.tex|, |cdocsfn2.tex|.
Then copy the file |childdoc.def| to an appropriate directory of your \LaTeX{}
distribution, e.g.\ \textit{texmf-root}|/tex/latex/childdoc|.
\end{itemize}

%%%%%%%%%%%%%%%%%%%%%%%%%%%%%%%%%%%%%%%%%%%%%%%%%%%%%%%%%%%%%%%%%%%%%%%%%%%%%%%%
\subsection{Related CTAN Packages}

There are several other packages which offer a similar functionality:
%
\begin{itemize}
\item
The packages
\href{http://ctan.org/pkg/docmute}{\textsf{docmute}},
\href{http://ctan.org/pkg/includex}{\textsf{includex}} and
\href{http://ctan.org/pkg/standalone}{\textsf{standalone}}
provide commands to include only the document body of
a child file thus allowing both files to be compiled individually.
\item
The packages \href{http://ctan.org/pkg/subdocs}{\textsf{subdocs}}
and \href{http://ctan.org/pkg/subfiles}{\textsf{subfiles}}
provide structures in which the main and child documents can be
encapsulated and allowing them to be compiled individually.
The inclusion mechanism is different from the conventional |\include|.
\item
The package \href{http://ctan.org/pkg/combine}{\textsf{combine}}
is an elaborate solution to combine several documents into one.
\end{itemize}
%
See also the CTAN topic \href{http://ctan.org/topic/subdocs}{\textsf{subdocs}}
for further related packages.
The present package differs from the above solutions in that
a document structure constructed with the conventional |\include| mechanism
just needs two extra commands at the top of every file
such that all constituent files can be compiled individually.

%%%%%%%%%%%%%%%%%%%%%%%%%%%%%%%%%%%%%%%%%%%%%%%%%%%%%%%%%%%%%%%%%%%%%%%%%%%%%%%%
%\subsection{Feature Suggestions}
%
%The following is a list of features which may be useful for future
%versions of this package:
%%
%\begin{itemize}
%\item
%\ldots
%\end{itemize}

%%%%%%%%%%%%%%%%%%%%%%%%%%%%%%%%%%%%%%%%%%%%%%%%%%%%%%%%%%%%%%%%%%%%%%%%%%%%%%%%
\subsection{Revision History}

%%%%%%%%%%%%%%%%%%%%%%%%%%%%%%%%%%%%%%%%
\paragraph{v2.0:} 2018/12/30

\begin{itemize}
\item
immediate forward processing
\item
added |\childdocby| mechanism
\item
manual restructured
\end{itemize}

%%%%%%%%%%%%%%%%%%%%%%%%%%%%%%%%%%%%%%%%
\paragraph{v1.6:} 2018/01/17

\begin{itemize}
\item
application for development of include files
\item
corrections to manual
\end{itemize}

%%%%%%%%%%%%%%%%%%%%%%%%%%%%%%%%%%%%%%%%
\paragraph{v1.5:} 2017/05/21

\begin{itemize}
\item
more complete structuring introduced
\item
|\childdocof| introduced
\item
|\childdoc| renamed to |\childdocmain|
\item
|\childredirect| renamed to |\childdocforward| and |\childdocforwardprefix|
and functionality expanded
\end{itemize}

%%%%%%%%%%%%%%%%%%%%%%%%%%%%%%%%%%%%%%%%
\paragraph{v1.0:} 2017/04/27

\begin{itemize}
\item
manual and install package
\item
first version published on CTAN
\end{itemize}

%%%%%%%%%%%%%%%%%%%%%%%%%%%%%%%%%%%%%%%%
\paragraph{v0.6:} 2017/04/26

\begin{itemize}
\item
redirection mechanism added
\end{itemize}

%%%%%%%%%%%%%%%%%%%%%%%%%%%%%%%%%%%%%%%%
\paragraph{v0.5:} 2017/04/26

\begin{itemize}
\item
functionality in definition file
\end{itemize}


%%%%%%%%%%%%%%%%%%%%%%%%%%%%%%%%%%%%%%%%%%%%%%%%%%%%%%%%%%%%%%%%%%%%%%%%%%%%%%%%
%%%%%%%%%%%%%%%%%%%%%%%%%%%%%%%%%%%%%%%%%%%%%%%%%%%%%%%%%%%%%%%%%%%%%%%%%%%%%%%%
%%%%%%%%%%%%%%%%%%%%%%%%%%%%%%%%%%%%%%%%%%%%%%%%%%%%%%%%%%%%%%%%%%%%%%%%%%%%%%%%
\appendix

\settowidth\MacroIndent{\rmfamily\scriptsize 000\ }

 \DocInput{childdoc.dtx}

\end{document}
%</driver>
% \fi
%
% %%%%%%%%%%%%%%%%%%%%%%%%%%%%%%%%%%%%%%%%%%%%%%%%%%%%%%%%%%%%%%%%%%%%%%%%%%%%%%
% %%%%%%%%%%%%%%%%%%%%%%%%%%%%%%%%%%%%%%%%%%%%%%%%%%%%%%%%%%%%%%%%%%%%%%%%%%%%%%
% \section{Sample}
%\iffalse
%<*samplemain>
%\fi
%
% The following presents a sample document
% with two chapters, two parts, a title page,
% a compile flag as well as three forwarding files to set the flag.
% It consists of eight |.tex| files:
% \begin{center}
% \begin{tabular}{ll}
% |cdocsamp.tex|&main file\\
% |cdocsch1.tex|&include file for chapter 1\\
% |cdocsch2.tex|&include file for chapter 2\\
% |cdocspt3.tex|&include file for part 3\\
% |cdocspt4.tex|&include file for part 4\\
% |cdocsdrf.tex|&forwarding file for main file in draft mode\\
% |cdocsfi1.tex|&forwarding file for final version of chapter 1\\
% |cdocsfi2.tex|&forwarding file for final version of chapter 2\\
% \end{tabular}
% \end{center}
% Each of the eight files can be compiled directly by the \LaTeX{} compiler.
%
% %%%%%%%%%%%%%%%%%%%%%%%%%%%%%%%%%%%%%%
% \paragraph{Main File.}
%
% The main file is called |cdocsamp.tex|.
%
% Load the \textsf{childdoc} definitions and
% declare the filename for the main document:
%    \begin{macrocode}
\input{childdoc.def}
\childdocmain{}
%    \end{macrocode}

% Optional override for |\version| flag:
%    \begin{macrocode}
%%\ifchilddoc\else\providecommand{\version}{draft}\fi
%    \end{macrocode}

% Define the default values for the |\version| flag
% (|final| for the main file and |draft| for childs):
%    \begin{macrocode}
\ifchilddoc
\providecommand{\version}{draft}
\else
\providecommand{\version}{final}
\fi
%    \end{macrocode}

% Load the standard document class:
%    \begin{macrocode}
\documentclass[12pt]{article}
%    \end{macrocode}

% Start the document body:
%    \begin{macrocode}
\begin{document}
%    \end{macrocode}

% Declare a title page.
% Print title, part of document being processed and version flag:
%    \begin{macrocode}
\addtocounter{page}{-1}
\begin{center}
{\LARGE\bfseries{}childdoc example\par}
\vspace{1cm}
\ifchilddoc
\ifchilddocmanual part\else chapter\fi:
`\childdocname' of `\childdocjob'\par
\else
main document: `\childdocjob'\par
\fi
version: \version\par
\end{center}
\newpage
%    \end{macrocode}

% Manually include selected file,
% otherwise process as usual:
%    \begin{macrocode}
\ifchilddocmanual
\section*{part `\childdocname'}
\input{\childdocname}
\else
%    \end{macrocode}

% Include the two chapters:
%    \begin{macrocode}
\include{cdocsch1}
\include{cdocsch2}
%    \end{macrocode}

% Include the two parts unless only chapters should be displayed:
%    \begin{macrocode}
\ifchilddoc\else
\section{part three}
\input{cdocspt3}
\section{part four}
\input{cdocspt4}
\fi
%    \end{macrocode}

% Process as usual until here:
%    \begin{macrocode}
\fi
%    \end{macrocode}

% End of document body:
%    \begin{macrocode}
\end{document}
%    \end{macrocode}
%\iffalse
%</samplemain>
%\fi
%
% %%%%%%%%%%%%%%%%%%%%%%%%%%%%%%%%%%%%%%
% \paragraph{Chapter Include Files.}
%
% The include files are called |cdocsch1.tex| and |cdocsch2.tex|.
%
%\iffalse
%<*samplechap1|samplechap2>
%\fi

% Optional override for |\version| flag:
%    \begin{macrocode}
%%\providecommand{\version}{final}
%    \end{macrocode}

% Include the main document:
%    \begin{macrocode}
\input{childdoc.def}
\childdocof{cdocsamp}
%    \end{macrocode}

%\iffalse
%</samplechap1|samplechap2>
%\fi
%
%\iffalse
%<*samplechap1>
%\fi
% Some text for chapter 1:
%    \begin{macrocode}
\section{one}
some text in chapter one
%    \end{macrocode}

%\iffalse
%</samplechap1>
%\fi
% Some text for chapter 2:
%\iffalse
%<*samplechap2>
%\fi
%    \begin{macrocode}
\section{two}
more text in chapter two
%    \end{macrocode}

%\iffalse
%</samplechap2>
%\fi
%
% %%%%%%%%%%%%%%%%%%%%%%%%%%%%%%%%%%%%%%
% \paragraph{Part Include Files.}
%
% The include files are called |cdocspt3.tex| and |cdocspt4.tex|.
%
%\iffalse
%<*samplepart3|samplepart4>
%\fi

% Optional override for |\version| flag:
%    \begin{macrocode}
%%\providecommand{\version}{final}
%    \end{macrocode}

% Include the main document:
%    \begin{macrocode}
\input{childdoc.def}
\childdocby{cdocsamp}
%    \end{macrocode}

%\iffalse
%</samplepart3|samplepart4>
%\fi
%
%\iffalse
%<*samplepart3>
%\fi
% Some text for part 3:
%    \begin{macrocode}
some text in part three
%    \end{macrocode}

%\iffalse
%</samplepart3>
%\fi
% Some text for part 4:
%\iffalse
%<*samplepart4>
%\fi
%    \begin{macrocode}
more text in part four
%    \end{macrocode}

%\iffalse
%</samplepart4>
%\fi
%
% %%%%%%%%%%%%%%%%%%%%%%%%%%%%%%%%%%%%%%
% \paragraph{Forwarding for a Complete Draft.}
%
% The following forwarding file |cdocsdrf.tex|
% compiles the main document in draft mode:
%\iffalse
%<*sampledraft>
%\fi
%    \begin{macrocode}
\def\version{draft}
\input{childdoc.def}
\childdocforward{cdocsamp}
%    \end{macrocode}

%\iffalse
%</sampledraft>
%\fi
%
% %%%%%%%%%%%%%%%%%%%%%%%%%%%%%%%%%%%%%%
% \paragraph{Forwarding for Final Version of the Chapters.}
%
% The following forwarding files |cdocsfn1.tex| and |cdocsfn2.tex|
% (with identical content)
% compile the final versions of the child documents
% |cdocsch1.tex| and |cdocsch2.tex|, respectively:
%\iffalse
%<*samplefinal>
%\fi
%    \begin{macrocode}
\def\version{final}
\input{childdoc.def}
\childdocforwardprefix[cdocsamp]{cdocsfn}{cdocsch}
%    \end{macrocode}

%\iffalse
%</samplefinal>
%\fi
%
% %%%%%%%%%%%%%%%%%%%%%%%%%%%%%%%%%%%%%%
% \paragraph{Command Line Processing.}
%
% The following three command lines generate the output files
% |cdocscld|, |cdocscl1| and |cdocscl2|
% which should be identical to
% |cdocsdrf|, |cdocsch1| and |cdocsfn2|, respectively:
% \begin{center}
% \begin{tabular}{l}
% |latex -jobname cdocscld \|\\
% |  "\def\version{draft}\input{childdoc.def}\childdocforward{cdocsamp}"|\\
% |latex -jobname cdocscl1 \|\\
% |  "\input{childdoc.def}\childdocforward[cdocsamp]{cdocsch1}"|\\
% |latex -jobname cdocscl2 \|\\
% |  "\def\version{final}\input{childdoc.def}\childdocforward{cdocsch2}"|
% \end{tabular}
% \end{center}
% Note that the trailing backslash on each first line
% merely continues the input to the second line
% (for convenient cut ant paste).
% Furthermore, the command |latex| can be replaced by any
% of its alternative versions such as |pdflatex|.
%
% %%%%%%%%%%%%%%%%%%%%%%%%%%%%%%%%%%%%%%%%%%%%%%%%%%%%%%%%%%%%%%%%%%%%%%%%%%%%%%
% %%%%%%%%%%%%%%%%%%%%%%%%%%%%%%%%%%%%%%%%%%%%%%%%%%%%%%%%%%%%%%%%%%%%%%%%%%%%%%
% \section{Implementation}
%\iffalse
%<*package>
%\fi
%
% This section describes the definitions file |childdoc.def|.

% The definitions cannot be loaded using |\usepackage| or |\RequirePackage|
% which has a mechanism to prevent loading a style file more than once.
% When loading the definitions by means of |\input|
% multiple instances have to be prevented manually:
%\iffalse
%This code needs to be before the `\ProvidesFile' directive
%which is defined at the beginning of this file.
%Therefore it is also placed there and commented out here.
%</package>
%<*discard>
%\fi
%    \begin{macrocode}
\ifdefined\childdocmain\endinput\fi
%    \end{macrocode}
%\iffalse
%</discard>
%<*package>
%\fi
%
% \macro{\ifchilddoc}
% \macro{\ifchilddocmanual}
% The conditional |\ifchilddoc| tells whether a
% child (true) or main (false) document is being compiled.
% The conditional |\ifchilddocmanual| tells whether
% the |\includeonly| mechanism is used (false) or
% the selection of child files must be performed manually (true).
% The definitions initialise to false:
%    \begin{macrocode}
\newif\ifchilddoc
\newif\ifchilddocmanual
%    \end{macrocode}

% \macro{\childdocname}
% \macro{\childdocjob}
% The macro |\childdocname| stores the name of the main document
% to be compiled. The macro |\childdocjob| stores the name of
% the document on which the \LaTeX{} compiler was originally invoked.
% The content of |\jobname| cannot be compared
% to filenames specified in the source due to different catcodes.
% The following code rescans |\jobname|, stores the result
% in |\childdocname| and saves a copy in |\childdocjob|:
%    \begin{macrocode}
\edef\childdocname{\scantokens\expandafter{\jobname\noexpand}}
\let\childdocjob\childdocname
%    \end{macrocode}

% \macro{\childdocdisable}
% The macro |\childdocdisable| prevents the main file
% from being processed more than once.
% At this stage, the main document command |\childdocmain|
% is assumed to be called once again where it should do nothing.
% Any subsequent call to it should prevent
% a secondary processing of the main document
% It overwrites the forwarding commands
% |\childdocof| and |\childdocforward|
% with empty macros to prevent further inclusions of the main document:
%    \begin{macrocode}
\newcommand{\childdocdisable}
{
  \renewcommand{\childdocmain}[1]{\renewcommand{\childdocmain}[1]{\endinput}}
  \renewcommand{\childdocof}[1]{}
  \renewcommand{\childdocby}[2][]{}
  \renewcommand{\childdocforward}[2][]{}
  \renewcommand{\childdocdisable}{}
}
%    \end{macrocode}

% \macro{\childdocmain}
% The macro |\childdocmain| is to be called at the top of the main file
% with nothing or the main filename (without extension) as argument.
% First, it breaks loops.
% If the argument is not empty and does not match |\childdocname|
% (which is set by the first inclusion of |childdoc.def|),
% |\ifchilddoc| is set to true, |\includeonly| is applied to the child file
% and |\jobname| is set to the main file
% (for proper handling of |.aux| files):
%    \begin{macrocode}
\newcommand{\childdocmain}[1]
{
  \childdocdisable\childdocmain{}
  \if?#1?\else
    \begingroup
      \def\childdoctmp{#1}
      \ifx\childdoctmp\childdocname
        \def\childdoctmp{}
      \else
        \def\childdoctmp
        {
          \childdoctrue
          \includeonly{\childdocname}
          \def\childdocjob{#1}
          \def\jobname{#1}
        }
      \fi
      \expandafter
    \endgroup
    \childdoctmp
  \fi
}
%    \end{macrocode}

% \macro{\childdocof}
% The command |\childdocof| redirects
% compilation to the main file |#1|.
%    \begin{macrocode}
\newcommand{\childdocof}[1]
{
  \childdocdisable
  \childdoctrue
  \includeonly{\childdocname}
  \def\jobname{#1}
  \def\childdocjob{#1}
  \input{#1}
}
%    \end{macrocode}

% \macro{\childdocby}
% The command |\childdocby| ....
%    \begin{macrocode}
\newcommand{\childdocby}[2][]
{
  \childdocdisable
  \childdoctrue
  \childdocmanualtrue
  \if?#1?\else
    \def\jobname{#2}
  \fi
  \def\childdocjob{#2}
  \input{#2}
  \endinput
}
%    \end{macrocode}

% \macro{\childdocforward}
% The command |\childdocforward| redirects
% compilation to the main file or
% (if the optional argument is given) a child file.
% Parameters are set as if the main file
% or a child file starting with |\childdocof| was compiled.
% Then compilation is handed over to the main file:
%    \begin{macrocode}
\newcommand{\childdocforward}[2][]
{
  \begingroup
    \if?#1?
      \def\childdoctmp
      {
        \def\childdocname{#2}
        \def\childdocjob{#2}
        \def\jobname{#2}
        \input{#2}
        \endinput
      }
    \else
      \def\childdoctmp
      {
        \childdocdisable
        \def\childdocname{#2}
        \childdoctrue
        \includeonly{#2}
        \def\childdocjob{#1}
        \def\jobname{#1}
        \input{#1}
        \endinput
      }
    \fi
    \expandafter
  \endgroup
  \childdoctmp
}
%    \end{macrocode}

% \macro{\childdocforwardprefix}
% The command |\childdocforwardprefix| redirects
% compilation to the main or a child file by means of a pattern.
% The prefix |#1| in the current filename is replaced by |#2|
% and the suffix of the current filename is kept
% (it is assumed that the filename does not contain the substring `|~~~|'
% which is used as a delimiter).
% Compilation is handed over to the new file by |\childdocforward|:
%    \begin{macrocode}
\newcommand{\childdocforwardprefix}[3][]
{
  \begingroup
    \def\childdocextract #2##1~~~{\def\childdoctmp{\childdocforward[#1]{#3##1}}}
    \expandafter\childdocextract\childdocname~~~
    \expandafter
  \endgroup
  \childdoctmp
}
%    \end{macrocode}

% \macro{\childdoc}
% The deprecated macro |\childdoc| is a legacy version of |\childdocmain|:
%    \begin{macrocode}
\newcommand{\childdoc}{\childdocmain}
%    \end{macrocode}

% \macro{\childdocredirect}
% The deprecated macro |\childdocredirect| is a legacy version
% of |\childdocforward| and |\childdocforwardprefix|:
%    \begin{macrocode}
\newcommand{\childdocredirect}[2][]
{
  \begingroup
    \if?#1?
      \def\childdoctmp{\childdocforward{#2}}
    \else
      \def\childdoctmp{\childdocforwardprefix{#1}{#2}}
    \fi
    \expandafter
  \endgroup
  \childdoctmp
}
%    \end{macrocode}

%\iffalse
%</package>
%\fi
%
\endinput
|\\
|\childdocby{|\textit{main}|}|\\
\end{tabular}
\end{center}
%
The directive |\childdocby| is similar to |\childdocof|
described in \secref{sec:include},
but the subsequent selection of content must be done manually.
To that end, both |\ifchilddoc| and |\ifchilddocmanual|
will be true upon processing of a part,
and the name of the part is stored in |\childdocname|.
Note that |\jobname| will be set to the filename of the current part
so that each part receives an individual |.aux| file
that does not interfere with the |.aux| file(s) of the main document.
This behaviour can be altered by the alternative form
|\childdocby[*]{|\textit{main}|}| (with a non-empty optional argument)
which uses the |.aux| file of the main document
by setting |\jobname| to \textit{main}.

%%%%%%%%%%%%%%%%%%%%%%%%%%%%%%%%%%%%%%%%%%%%%%%%%%%%%%%%%%%%%%%%%%%%%%%%%%%%%%%%
\subsection{Driver Development}
\label{sec:driver}

The \textsf{childdoc} mechanism can also be use for the development
of definition files such as \LaTeX{} styles or classes.
This case differs from the above setup with multiple parts
included by |\include| in that no |\includeonly| should be invoked.
This can be achieved by starting the include file
(before |\ProvidesPackage|) with:
%
\begin{center}
\begin{tabular}{l}
|% \iffalse
%
% childdoc.dtx Copyright (C) 2017-2018 Niklas Beisert
%
% This work may be distributed and/or modified under the
% conditions of the LaTeX Project Public License, either version 1.3
% of this license or (at your option) any later version.
% The latest version of this license is in
%   http://www.latex-project.org/lppl.txt
% and version 1.3 or later is part of all distributions of LaTeX
% version 2005/12/01 or later.
%
% This work has the LPPL maintenance status `maintained'.
%
% The Current Maintainer of this work is Niklas Beisert.
%
% This work consists of the files childdoc.dtx and childdoc.ins
% and the derived files childdoc.def and cdocsamp.tex with
% cdocsch1.tex, cdocsch2.tex, cdocsdrf.tex, cdocsfn1.tex, cdocsfn2.tex.
%
%<package>\ifdefined\childdocmain\endinput\fi
%<package>\ProvidesFile{childdoc.def}[2018/12/30 v2.0 child document driver]
%<samplemain>\ProvidesFile{cdocsamp.tex}[2018/12/30 v2.0 sample for childdoc]
%<*driver>
%\ProvidesFile{childdoc.drv}[2018/12/30 v2.0 childdoc reference manual file]
\PassOptionsToClass{10pt,a4paper}{article}
\documentclass{ltxdoc}

\usepackage[margin=35mm]{geometry}
\usepackage{hyperref}
\usepackage{hyperxmp}
\usepackage[usenames]{color}

\hypersetup{colorlinks=true}
\hypersetup{pdfstartview=FitH}
\hypersetup{pdfpagemode=UseNone}
\hypersetup{pdfsource={}}
\hypersetup{pdflang={en-UK}}
\hypersetup{pdfcopyright={Copyright 2017-2018 Niklas Beisert.
  This work may be distributed and/or modified under the
  conditions of the LaTeX Project Public License, either version 1.3
  of this license or (at your option) any later version.}}
\hypersetup{pdflicenseurl={http://www.latex-project.org/lppl.txt}}
\hypersetup{pdfcontactaddress={ETH Zurich, ITP, HIT K,
  Wolfgang-Pauli-Strasse 27}}
\hypersetup{pdfcontactpostcode={8093}}
\hypersetup{pdfcontactcity={Zurich}}
\hypersetup{pdfcontactcountry={Switzerland}}
\hypersetup{pdfcontactemail={nbeisert@itp.phys.ethz.ch}}
\hypersetup{pdfcontacturl={http://people.phys.ethz.ch/\xmptilde nbeisert/}}

\newcommand{\secref}[1]{\hyperref[#1]{section \ref*{#1}}}

\parskip1ex
\parindent0pt
\let\olditemize\itemize
\def\itemize{\olditemize\parskip0pt}

\begin{document}

\title{The \textsf{childdoc} Package}
\hypersetup{pdftitle={The childdoc Package}}
\author{Niklas Beisert\\[2ex]
  Institut f\"ur Theoretische Physik\\
  Eidgen\"ossische Technische Hochschule Z\"urich\\
  Wolfgang-Pauli-Strasse 27, 8093 Z\"urich, Switzerland\\[1ex]
  \href{mailto:nbeisert@itp.phys.ethz.ch}
  {\texttt{nbeisert@itp.phys.ethz.ch}}}
\hypersetup{pdfauthor={Niklas Beisert}}
\hypersetup{pdfsubject={Manual for the LaTeX2e Package childdoc}}
\date{30 December 2018, \textsf{v2.0}}
\maketitle

\begin{abstract}\noindent
\textsf{childdoc} is a \LaTeXe{} package
that enables the direct compilation
of document sections included by |\include|
to individual files.
\end{abstract}

\begingroup
\parskip0ex
\tableofcontents
\endgroup

%%%%%%%%%%%%%%%%%%%%%%%%%%%%%%%%%%%%%%%%%%%%%%%%%%%%%%%%%%%%%%%%%%%%%%%%%%%%%%%%
%%%%%%%%%%%%%%%%%%%%%%%%%%%%%%%%%%%%%%%%%%%%%%%%%%%%%%%%%%%%%%%%%%%%%%%%%%%%%%%%
\section{Introduction}

\LaTeX{} provides a mechanism to structure a large document (such as a book)
into a main file and several child files (containing the chapters)
using the |\include| command.
This mechanism is beneficial for documents
which span hundreds of pages in order to
make the source file(s) more manageable.
Moreover, compilation can be restricted to
selected child files by means of the |\includeonly| command.
The latter feature can be used to reduce the compilation time while editing
(this was significantly more useful in the earlier days of \LaTeX{})
or to generate a smaller document which is easier to navigate.
Another application of |\includeonly| is to generate
documents consisting of selected parts of the complete document.

However, there are a few drawbacks of the plain |\include| mechanism:
\begin{itemize}
\item
The child files cannot be compiled on their own,
they can only be compiled via the main file.
A naive editing environment
(such as a text editor with an option
to have the current file processed by \LaTeX)
may require one to switch to the main file before compiling;
attempting to compile the child file produces errors.
\item
The main file must be modified (each time)
to adjust the |\includeonly| command
to the present needs. This easily leaves the main file in a messy state.
\item
The generated document will always carry the filename
of the main document. This is inconvenient if
several child files are to be compiled and
to be kept for distribution.
\end{itemize}

The present package provides a simple interface
to make child files individually compilable by \LaTeX{}.
Compiling a child file then has the same effect as compiling
the main file with an |\includeonly| command
to select the appropriate child.
Moreover the generated document will carry the name of the child
rather than the main file.
This resolves all three above issues.

This feature is meant to make the editing of books,
thesis documents and lecture notes somewhat more convenient.
However, the package can also be used efficiently for
composing a series of documents (such as exercise sheets)
which are typically distributed individually.
It then assists the author in generating the individual documents
(potentially in different versions)
as well as a document containing the collected series.
Another application is in developing style files
or other kinds of included material
where compilation of the style file could redirect
to a sample or test file.

%%%%%%%%%%%%%%%%%%%%%%%%%%%%%%%%%%%%%%%%%%%%%%%%%%%%%%%%%%%%%%%%%%%%%%%%%%%%%%%%
%%%%%%%%%%%%%%%%%%%%%%%%%%%%%%%%%%%%%%%%%%%%%%%%%%%%%%%%%%%%%%%%%%%%%%%%%%%%%%%%
\section{Usage}

First of all, the package \textsf{childdoc} is \emph{not} a standard
\LaTeXe{} |.sty| style file! Therefore it needs to be invoked in
a non-standard way.

%%%%%%%%%%%%%%%%%%%%%%%%%%%%%%%%%%%%%%%%%%%%%%%%%%%%%%%%%%%%%%%%%%%%%%%%%%%%%%%%
\subsection{Included Files}
\label{sec:include}

%%%%%%%%%%%%%%%%%%%%%%%%%%%%%%%%%%%%%%%%
\DescribeMacro{\childdocmain}
To use the package, add the commands
\begin{center}
\begin{tabular}{l}
|\input{childdoc.def}|\\
|\childdocmain{}|\\
\end{tabular}
\end{center}
at the very top of the main \LaTeX{} file,
in particular \emph{before} the |\documentclass| statement!
The argument of |\childdocmain| should be left empty
(but it must be present).

%%%%%%%%%%%%%%%%%%%%%%%%%%%%%%%%%%%%%%%%
\DescribeMacro{\childdocof}
Furthermore, add the commands
\begin{center}
\begin{tabular}{l}
|\input{childdoc.def}|\\
|\childdocof{|\textit{main}|}|\\
\end{tabular}
\end{center}
at the top of every child file \textit{child}
which is included by |\include{|\textit{child}|}|
from within the main file
(or at least for those files to be compiled individually).
The argument \textit{main} must be the filename of the main file.

There are a couple of
considerations in setting up the main and child documents:

%%%%%%%%%%%%%%%%%%%%%%%%%%%%%%%%%%%%%%%%
\paragraph{Restrictions.}

Please note the following restrictions:
\begin{itemize}
\item
|\childdocmain| must be called with one argument \textit{main}
to ensure compatibility with earlier version of the package.
It must either be empty (|\childdocmain{}|)
or precisely match the filename of the main file in which it is specified.
See \secref{sec:detection} for further information.
\item
The filename \textit{main} must be specified without the |.tex| extension.
\item
The filename \textit{main} is case sensitive
(even in case-insensitive file systems)
due to internal string comparison.
\item
The argument \textit{main} should be fully expanded, it cannot be a macro.
\item
Subdirectories and special characters should be avoided in filenames.
\item
The command |\childdocmain{|\textit{main}|}| must be followed by a whitespace.
It should not be followed immediately by another command
or by a comment mark `|%|'.
This is because the \TeX{} parser reads the token immediately following
the argument of |\childdocmain| and puts it
at the beginning of every child section;
however, a white\-space is ignored.
\end{itemize}

%%%%%%%%%%%%%%%%%%%%%%%%%%%%%%%%%%%%%%%%
\paragraph{Content of Main File.}

It is advisable to place all content in the child files included by |\include|.
Any output contained in the main file will appear in all child documents
unless suppressed manually;
it cannot be suppressed automatically by the |\includeonly| directive
and thus should normally be avoided.
A method to include some content in the main file
by means of conditional processing is described in \secref{sec:conditional}.

%%%%%%%%%%%%%%%%%%%%%%%%%%%%%%%%%%%%%%%%
\paragraph{Page Numbering.}

When only a part of the document is compiled,
the appropriate numbering of pages
(as well as other status parameters)
is determined from the |.aux| files.
The latter contain information from previous passes.
However this information needs to propagate through
all intermediate child documents.
Therefore the page numbering in child documents may well
be inconsistent until the complete document is compiled at least once.

A useful (if unconventional) way to always ensure a consistent
page numbering is to restart the numbering in each child document
and denote the pages by `\textit{child}|.|\textit{page}'
where \textit{child} represents the chapter/section number of the child file.
This can be achieved by the command
|\numberwithin{page}{|\textit{child}|}|
of the \textsf{amsmath} package
where \textit{child} can be |chapter| or |section|
depending on the chosen structuring.
Alternatively, one can modify the macro |\thepage| appropriately
and reset the counter |page| at the start of each child file.

%%%%%%%%%%%%%%%%%%%%%%%%%%%%%%%%%%%%%%%%%%%%%%%%%%%%%%%%%%%%%%%%%%%%%%%%%%%%%%%%
\subsection{Conditional Processing}
\label{sec:conditional}

The package provides a mechanism to compile different versions
of a document. To customise the versions further some conditional processing
can come in handy to distinguish which version is being compiled.
The package provides two macros to describe the compilation context:

%%%%%%%%%%%%%%%%%%%%%%%%%%%%%%%%%%%%%%%%
\DescribeMacro{\ifchilddoc}
The conditional |\ifchilddoc| distinguishes between the compilation of
child documents and the main document:
%
\begin{center}
|\ifchilddoc |\textit{child-code}| |[|\||else |\textit{main-code}]| \||fi|
\end{center}

%%%%%%%%%%%%%%%%%%%%%%%%%%%%%%%%%%%%%%%%
\DescribeMacro{\childdocname}
\DescribeMacro{\childdocjob}
The macro |\childdocname| contains the filename (without extension)
of the main or child file being processed.
Note that |\childdocjob| will always contain the name of the main file.

%%%%%%%%%%%%%%%%%%%%%%%%%%%%%%%%%%%%%%%%
\paragraph{Title Page.}

Conditional processing can be used to include a title or banner page
in the main document when proper precautions are taken.
Importantly, the code in the main file should ensure that the page counter
(as well as other status parameters which are stored in the |.aux| files)
takes the same value after the conditional processing.
Otherwise the page numbers may take divergent values
depending on which part is compiled.

For example, a title page could be declared by:
%
\begin{center}
\begin{tabular}{l}
|\ifchilddoc\||else|\\
|\addtocounter{page}{-1}|\\
\textit{code for title page}\\
|\newpage|\\
|\||fi|
\end{tabular}
\end{center}
%
A banner page for the child documents can be generated by:
%
\begin{center}
\begin{tabular}{l}
|\ifchilddoc|\\
|\addtocounter{page}{-1}|\\
\textit{code for banner page}\\
|\newpage|\\
|\||fi|
\end{tabular}
\end{center}
%
Here one could write a message such as:
\begin{center}
|This is the part \childdocname{} of \childdocjob{}.|
\end{center}

%%%%%%%%%%%%%%%%%%%%%%%%%%%%%%%%%%%%%%%%%%%%%%%%%%%%%%%%%%%%%%%%%%%%%%%%%%%%%%%%
\subsection{Flags}
\label{sec:flags}

The package makes it easy to generate different versions
of the main or child documents.
To this end compilation flags can be defined
and assigned different default values.
They will be particularly useful in conjunction
with the forwarding mechanism described in \secref{sec:forward}.

For example, it may be useful to have a flag |\version|
which can be set to |draft| or |final|.
The document source will contain some conditional code
depending on the value of |\version|.
Suppose further, the flag should default to |final| for the main file
and to |draft| for child files
which is a natural assignment for editing the document.
This is achieved by placing the following code
in the preamble of the main document
(below the |\childdocmain| directive):
%
\begin{center}
\begin{tabular}{l}
|\ifchilddoc|\\
|\providecommand{\version}{draft}|\\
|\||else|\\
|\providecommand{\version}{final}|\\
|\||fi|
\end{tabular}
\end{center}
%
The definition by |\providecommand| makes sure
that previous definitions are not overwritten.
Further statements |\providecommand{\version}{...}|
can thus be added before the above code to override it.

For the main file, one might add a line
(between |\childdocmain| and the above block)
%
\begin{center}
|%\ifchilddoc\||else\providecommand{\version}{draft}\||fi|
\end{center}
%
which can be uncommented to produce a draft version.
Likewise one can add a line to the very top of a child file
(above the |\childdocof{|\textit{main}|}| directive)
%
\begin{center}
|%\providecommand{\version}{final}|
\end{center}
%
which can be uncommented to produce the final version of this child document.

%%%%%%%%%%%%%%%%%%%%%%%%%%%%%%%%%%%%%%%%%%%%%%%%%%%%%%%%%%%%%%%%%%%%%%%%%%%%%%%%
\subsection{Forwarding}
\label{sec:forward}

Different versions of the main or child documents
using compilation flags as described in \secref{sec:flags}
can be (permanently) stored in different files
for convenient compilation, viewing and distribution.
To this end, the package defines a command
to pass on compilation to a different file:

%%%%%%%%%%%%%%%%%%%%%%%%%%%%%%%%%%%%%%%%
\DescribeMacro{\childdocforward}
The command |\childdocforward| redirects processing to
another source file:
%
\begin{center}
\begin{tabular}{l}
|\input{childdoc.def}|\\
|\childdocforward[|\textit{main}|]{|\textit{dest}|}|\\
\end{tabular}
\end{center}
%
The argument \textit{dest} is the destination file
(without extension).
It should be the main file or one of the child files.
Note that further \textsf{childdoc} directives
such as |\childdocof| and |\childdocforward|
in the indicated file will be processed in this form.
The optional argument \textit{main}
passes on directly to the main file \textit{main}
while pretending to compile the child \textit{dest}.
This form behaves as if \textit{dest}
issues |\childdocof{|\textit{main}|}| right away,
and no further \textsf{childdoc} directives will be processed.

%%%%%%%%%%%%%%%%%%%%%%%%%%%%%%%%%%%%%%%%
\DescribeMacro{\...prefix}
In the alternative form |\childdocforwardprefix|,
%
\begin{center}
\begin{tabular}{l}
|\input{childdoc.def}|\\
|\childdocforwardprefix[|\textit{main}|]{|\textit{prefix}|}{|\textit{dest}|}|
\end{tabular}
\end{center}
%
the destination file is determined by a pattern
depending on the current file:
To make this work, the current file must be called
`{\textit{prefix}\hspace{0.2em}\textit{suffix}}'
with \textit{prefix} matching precisely the argument.
Processing is then passed on to the file
`{\textit{dest}\hspace{0.2em}\textit{suffix}}'.
Surely, the same effect is achieved by
directly specifying the
argument `{\textit{dest}\hspace{0.2em}\textit{suffix}}'
in the first form.
However, that requires to set up a different file
for each child. With the alternative form of the command
all these files can have exactly the same content
which simplifies setting them up and maintaining them.

For example, the following file |draft.tex|
with a compilation flag |\version| as described in \secref{sec:flags}
compiles the main document as a draft:
%
\begin{center}
\begin{tabular}{l}
|\def\version{draft}|\\
|\input{childdoc.def}|\\
|\childdocforward{|\textit{main}|}|
\end{tabular}
\end{center}
%
Likewise, the following files |final|\textit{nn}|.tex|
compile the final version of the child document
|child|\textit{nn}|.tex|:
%
\begin{center}
\begin{tabular}{l}
|\def\version{final}|\\
|\input{childdoc.def}|\\
|\childdocforwardprefix{final}{child}|
\end{tabular}
\end{center}
%

Note that when several versions of a main file and/or of each child file
are to be generated, it may be convenient to set up a |Makefile| or
shell script to automatise the process.

%%%%%%%%%%%%%%%%%%%%%%%%%%%%%%%%%%%%%%%%%%%%%%%%%%%%%%%%%%%%%%%%%%%%%%%%%%%%%%%%
\subsection{Command Line Processing}
\label{sec:commandline}

The effect of redirection files can also be achieved by invoking
the \LaTeX{} compiler with a more elaborate command line.
Most conveniently this should be done as part
of a shell script or a |Makefile|.

When using \textsf{childdoc} in the main file, the following
command lines effectively perform a redirection
(note that depending on the shell being used,
backslashes may have to be doubled: `|\|' $\to$ `|\\|'):
%
\begin{center}
|... -jobname "|\textit{target}|" |\\|"|[\textit{flags}]%
|\input{childdoc.def}\childdocforward[|\textit{main}|]{|\textit{dest}|}"|
\end{center}
%
Here \textit{target} is the name of the output file,
\textit{main} is the name of the main file
and \textit{dest} is the name of the main or child file to be processed
(all filenames without extensions).
The optional argument \textit{main} can be omitted
if \textit{main} matches \textit{dest}.
Optionally, compilation \textit{flags} can be defined via |\def| commands.
This command line makes the \TeX{} engine believe
it is compiling the file \textit{target}
whose content is specified as the latter parameter.
The provided code then forwards the processing to
\textit{main} or \textit{dest} as described in \secref{sec:forward}.

%%%%%%%%%%%%%%%%%%%%%%%%%%%%%%%%%%%%%%%%%%%%%%%%%%%%%%%%%%%%%%%%%%%%%%%%%%%%%%%%
\subsection{Include by Input}
\label{sec:input}

Including child documents by |\include| has some restrictions by design.
Most notably, the content of a child document always occupies
its own set of pages; pages cannot be shared between child documents.
Usually, this behaviour makes perfect sense
because each child document contain an essential part of the document.
However, in some situations it may be desirable to compose
a document from a collection of parts
without having mandatory page breaks between then.
For this case, the package
provides a mechanism to include parts
by |\input| which can also be processed individually.
However, by construction this mechanism
requires manual handling of the content to be output.

%%%%%%%%%%%%%%%%%%%%%%%%%%%%%%%%%%%%%%%%
\DescribeMacro{\ifchilddocmanual}
The main file should be prepared as usual, see \secref{sec:include}.
However, the document body must make a distinction
between processing of an individual part and of the main document, e.g.:
%
\begin{center}
\begin{tabular}{l}
|\ifchilddocmanual|\\
|\input{\childdocname}|\\
|\||else|\\
\textit{document body with }|\input{|\textit{part}|}|\\
|\||fi|
\end{tabular}
\end{center}
%
The conditional |\ifchilddocmanual| is true whenever
a part to be included by |\input| is being compiled,
and the name of the part is stored in |\childdocname|.

%%%%%%%%%%%%%%%%%%%%%%%%%%%%%%%%%%%%%%%%
\DescribeMacro{\childdocby}
Each part to be included by |\input| should start with:
%
\begin{center}
\begin{tabular}{l}
|\input{childdoc.def}|\\
|\childdocby{|\textit{main}|}|\\
\end{tabular}
\end{center}
%
The directive |\childdocby| is similar to |\childdocof|
described in \secref{sec:include},
but the subsequent selection of content must be done manually.
To that end, both |\ifchilddoc| and |\ifchilddocmanual|
will be true upon processing of a part,
and the name of the part is stored in |\childdocname|.
Note that |\jobname| will be set to the filename of the current part
so that each part receives an individual |.aux| file
that does not interfere with the |.aux| file(s) of the main document.
This behaviour can be altered by the alternative form
|\childdocby[*]{|\textit{main}|}| (with a non-empty optional argument)
which uses the |.aux| file of the main document
by setting |\jobname| to \textit{main}.

%%%%%%%%%%%%%%%%%%%%%%%%%%%%%%%%%%%%%%%%%%%%%%%%%%%%%%%%%%%%%%%%%%%%%%%%%%%%%%%%
\subsection{Driver Development}
\label{sec:driver}

The \textsf{childdoc} mechanism can also be use for the development
of definition files such as \LaTeX{} styles or classes.
This case differs from the above setup with multiple parts
included by |\include| in that no |\includeonly| should be invoked.
This can be achieved by starting the include file
(before |\ProvidesPackage|) with:
%
\begin{center}
\begin{tabular}{l}
|\input{childdoc.def}|\\
|\childdocforward{|\textit{main}|}|\\
\end{tabular}
\end{center}
%
or alternatively with:
%
\begin{center}
\begin{tabular}{l}
|\input{childdoc.def}|\\
|\childdocby{|\textit{main}|}|\\
\end{tabular}
\end{center}
%
Both forms have slightly different effects as described above.
The main file is prepared as usual, see \secref{sec:include}.

%%%%%%%%%%%%%%%%%%%%%%%%%%%%%%%%%%%%%%%%%%%%%%%%%%%%%%%%%%%%%%%%%%%%%%%%%%%%%%%%
\subsection{Legacy Detection}
\label{sec:detection}

The directive |\childdocmain| in the main file can detect
whether the complete document or merely a child is to be compiled
even without using the directive |\childdocof|.
This method is deprecated because it is less robust
and there is no compelling reason to use it;
it is merely provided for backward compatibility
and it may be removed in future versions.

If the detection mechanism is to be used,
it is mandatory to correctly specify
the filename of the main file as the argument of |\childdocmain|:
%
\begin{center}
\begin{tabular}{l}
|\input{childdoc.def}|\\
|\childdocmain{|\textit{main}|}|\\
\end{tabular}
\end{center}
%
If |\jobname| does not match the argument \textit{main} of |\childdocmain|,
it is assumed that |\jobname| points to the child file to be compiled.
When using |\childdocmain| with the main file specified as argument,
it suffices to start a child file
with just |\input{|\textit{main}|}|
without loading of the package and using |\childdocof|.
If instead all processing is done
with the appropriate \textsf{childdoc} directives,
the argument of \textit{main} of |\childdocmain| can be empty.

An alternative version of the command line processing described
in \secref{sec:commandline} using the detection mechanism reads:
%
\begin{center}
|... -jobname "|\textit{target}|" "|[\textit{flags}]%
[|\def\jobname{|\textit{dest}|}|]|\input{|\textit{main}|}"|
\end{center}

%%%%%%%%%%%%%%%%%%%%%%%%%%%%%%%%%%%%%%%%%%%%%%%%%%%%%%%%%%%%%%%%%%%%%%%%%%%%%%%%
\subsection{Manual Code}
\label{sec:manual}

In case one cannot be certain whether the definitions file |childdoc.def|
is installed on the target \TeX{} distribution
and one prefers not to ship it,
it is conceivable to paste a few relevant commands into the sources.

To that end, drop all statements |\input{childdoc.def}|
and perform the replacements as outlined below.
Instead of |\childdocmain{|\textit{main}|}| add the following code
to the top of the main file:
%
\begin{center}
\begin{tabular}{l}
|\||ifdefined\childdocname\endinput\||fi\newif\ifchilddoc|\\
|\edef\childdocname{\scantokens\expandafter{\jobname\noexpand}}|\\
|\def\childdocmain{|\textit{main}|}\||ifx\childdocmain\childdocname\||else|\\
|\childdoctrue\includeonly{\childdocname}\let\jobname\childdocmain\||fi|\\
\end{tabular}
\end{center}
%
Instead of |\childdocof{|\textit{main}|}| just include the main file
at the top of each child file:
%
\begin{center}
|\input{|\textit{main}|}|
\end{center}
%
A simple redirection |\childdocforward{|\textit{dest}|}| is achieved by:
%
\begin{center}
|\def\jobname{|\textit{dest}|}\input{\jobname}|
\end{center}
%
The redirection with prefix
|\childdocforwardprefix[|\textit{prefix}|]{|\textit{dest}|}|
is accomplished by:
%
\begin{center}
\begin{tabular}{l}
|{\edef\jobname{\scantokens\expandafter{\jobname\noexpand}}|\\
|\def\redirectjob |\textit{prefix}|#1~~~{\gdef\jobname{|\textit{dest}|#1}}|\\
|\expandafter\redirectjob\jobname~~~}\input{\jobname}|
\end{tabular}
\end{center}

In an alternative approach,
child documents can be compiled by a specific command line
without additional code or specific definitions:
%
\begin{center}
|... -jobname "|\textit{target}|" "|[\textit{flags}]%
|\includeonly{|\textit{dest}|}\input{|\textit{main}|}"|
\end{center}
%

%%%%%%%%%%%%%%%%%%%%%%%%%%%%%%%%%%%%%%%%%%%%%%%%%%%%%%%%%%%%%%%%%%%%%%%%%%%%%%%%
%%%%%%%%%%%%%%%%%%%%%%%%%%%%%%%%%%%%%%%%%%%%%%%%%%%%%%%%%%%%%%%%%%%%%%%%%%%%%%%%
\section{Information}

%%%%%%%%%%%%%%%%%%%%%%%%%%%%%%%%%%%%%%%%%%%%%%%%%%%%%%%%%%%%%%%%%%%%%%%%%%%%%%%%
\subsection{Copyright}

Copyright \copyright{} 2017--2018 Niklas Beisert

This work may be distributed and/or modified under the
conditions of the \LaTeX{} Project Public License, either version 1.3
of this license or (at your option) any later version.
The latest version of this license is in
  \url{http://www.latex-project.org/lppl.txt}
and version 1.3 or later is part of all distributions of \LaTeX{}
version 2005/12/01 or later.

This work has the LPPL maintenance status `maintained'.

The Current Maintainer of this work is Niklas Beisert.

This work consists of the files |README.txt|, |childdoc.ins| and |childdoc.dtx|
as well as the derived files |childdoc.def|, |cdocsamp.tex|
with |cdocsch1.tex|, |cdocsch2.tex|, |cdocspt3.tex|, |cdocspt4.tex|,
|cdocsdrf.tex|, |cdocsfn1.tex|, |cdocsfn2.tex|
as well as |childdoc.pdf|.

%%%%%%%%%%%%%%%%%%%%%%%%%%%%%%%%%%%%%%%%%%%%%%%%%%%%%%%%%%%%%%%%%%%%%%%%%%%%%%%%
\subsection{Files and Installation}

The package consists of the files:
%
\begin{center}
\begin{tabular}{ll}
    |README.txt|   & readme file \\
    |childdoc.ins| & installation file \\
    |childdoc.dtx| & source file \\
    |childdoc.def| & definition file \\
    |cdocsamp.tex| & sample main file \\
    |cdocsch1.tex| & sample include file \\
    |cdocsch2.tex| & sample include file \\
    |cdocspt3.tex| & sample part file \\
    |cdocspt4.tex| & sample part file \\
    |cdocsdrf.tex| & sample redirection file \\
    |cdocsfn1.tex| & sample redirection file \\
    |cdocsfn2.tex| & sample redirection file \\
    |childdoc.pdf| & manual
\end{tabular}
\end{center}
%
The distribution consists of the files
|README.txt|, |childdoc.ins| and |childdoc.dtx|.
%
\begin{itemize}
\item
Run (pdf)\LaTeX{} on |childdoc.dtx|
to compile the manual |childdoc.pdf| (this file).
\item
Run \LaTeX{} on |childdoc.ins| to create the definitions file |childdoc.def|
and the sample |cdocsamp.tex| with include files
|cdocsch1.tex|, |cdocsch2.tex|, |cdocspt3.tex|, |cdocspt4.tex|,
|cdocsdrf.tex|, |cdocsfn1.tex|, |cdocsfn2.tex|.
Then copy the file |childdoc.def| to an appropriate directory of your \LaTeX{}
distribution, e.g.\ \textit{texmf-root}|/tex/latex/childdoc|.
\end{itemize}

%%%%%%%%%%%%%%%%%%%%%%%%%%%%%%%%%%%%%%%%%%%%%%%%%%%%%%%%%%%%%%%%%%%%%%%%%%%%%%%%
\subsection{Related CTAN Packages}

There are several other packages which offer a similar functionality:
%
\begin{itemize}
\item
The packages
\href{http://ctan.org/pkg/docmute}{\textsf{docmute}},
\href{http://ctan.org/pkg/includex}{\textsf{includex}} and
\href{http://ctan.org/pkg/standalone}{\textsf{standalone}}
provide commands to include only the document body of
a child file thus allowing both files to be compiled individually.
\item
The packages \href{http://ctan.org/pkg/subdocs}{\textsf{subdocs}}
and \href{http://ctan.org/pkg/subfiles}{\textsf{subfiles}}
provide structures in which the main and child documents can be
encapsulated and allowing them to be compiled individually.
The inclusion mechanism is different from the conventional |\include|.
\item
The package \href{http://ctan.org/pkg/combine}{\textsf{combine}}
is an elaborate solution to combine several documents into one.
\end{itemize}
%
See also the CTAN topic \href{http://ctan.org/topic/subdocs}{\textsf{subdocs}}
for further related packages.
The present package differs from the above solutions in that
a document structure constructed with the conventional |\include| mechanism
just needs two extra commands at the top of every file
such that all constituent files can be compiled individually.

%%%%%%%%%%%%%%%%%%%%%%%%%%%%%%%%%%%%%%%%%%%%%%%%%%%%%%%%%%%%%%%%%%%%%%%%%%%%%%%%
%\subsection{Feature Suggestions}
%
%The following is a list of features which may be useful for future
%versions of this package:
%%
%\begin{itemize}
%\item
%\ldots
%\end{itemize}

%%%%%%%%%%%%%%%%%%%%%%%%%%%%%%%%%%%%%%%%%%%%%%%%%%%%%%%%%%%%%%%%%%%%%%%%%%%%%%%%
\subsection{Revision History}

%%%%%%%%%%%%%%%%%%%%%%%%%%%%%%%%%%%%%%%%
\paragraph{v2.0:} 2018/12/30

\begin{itemize}
\item
immediate forward processing
\item
added |\childdocby| mechanism
\item
manual restructured
\end{itemize}

%%%%%%%%%%%%%%%%%%%%%%%%%%%%%%%%%%%%%%%%
\paragraph{v1.6:} 2018/01/17

\begin{itemize}
\item
application for development of include files
\item
corrections to manual
\end{itemize}

%%%%%%%%%%%%%%%%%%%%%%%%%%%%%%%%%%%%%%%%
\paragraph{v1.5:} 2017/05/21

\begin{itemize}
\item
more complete structuring introduced
\item
|\childdocof| introduced
\item
|\childdoc| renamed to |\childdocmain|
\item
|\childredirect| renamed to |\childdocforward| and |\childdocforwardprefix|
and functionality expanded
\end{itemize}

%%%%%%%%%%%%%%%%%%%%%%%%%%%%%%%%%%%%%%%%
\paragraph{v1.0:} 2017/04/27

\begin{itemize}
\item
manual and install package
\item
first version published on CTAN
\end{itemize}

%%%%%%%%%%%%%%%%%%%%%%%%%%%%%%%%%%%%%%%%
\paragraph{v0.6:} 2017/04/26

\begin{itemize}
\item
redirection mechanism added
\end{itemize}

%%%%%%%%%%%%%%%%%%%%%%%%%%%%%%%%%%%%%%%%
\paragraph{v0.5:} 2017/04/26

\begin{itemize}
\item
functionality in definition file
\end{itemize}


%%%%%%%%%%%%%%%%%%%%%%%%%%%%%%%%%%%%%%%%%%%%%%%%%%%%%%%%%%%%%%%%%%%%%%%%%%%%%%%%
%%%%%%%%%%%%%%%%%%%%%%%%%%%%%%%%%%%%%%%%%%%%%%%%%%%%%%%%%%%%%%%%%%%%%%%%%%%%%%%%
%%%%%%%%%%%%%%%%%%%%%%%%%%%%%%%%%%%%%%%%%%%%%%%%%%%%%%%%%%%%%%%%%%%%%%%%%%%%%%%%
\appendix

\settowidth\MacroIndent{\rmfamily\scriptsize 000\ }

 \DocInput{childdoc.dtx}

\end{document}
%</driver>
% \fi
%
% %%%%%%%%%%%%%%%%%%%%%%%%%%%%%%%%%%%%%%%%%%%%%%%%%%%%%%%%%%%%%%%%%%%%%%%%%%%%%%
% %%%%%%%%%%%%%%%%%%%%%%%%%%%%%%%%%%%%%%%%%%%%%%%%%%%%%%%%%%%%%%%%%%%%%%%%%%%%%%
% \section{Sample}
%\iffalse
%<*samplemain>
%\fi
%
% The following presents a sample document
% with two chapters, two parts, a title page,
% a compile flag as well as three forwarding files to set the flag.
% It consists of eight |.tex| files:
% \begin{center}
% \begin{tabular}{ll}
% |cdocsamp.tex|&main file\\
% |cdocsch1.tex|&include file for chapter 1\\
% |cdocsch2.tex|&include file for chapter 2\\
% |cdocspt3.tex|&include file for part 3\\
% |cdocspt4.tex|&include file for part 4\\
% |cdocsdrf.tex|&forwarding file for main file in draft mode\\
% |cdocsfi1.tex|&forwarding file for final version of chapter 1\\
% |cdocsfi2.tex|&forwarding file for final version of chapter 2\\
% \end{tabular}
% \end{center}
% Each of the eight files can be compiled directly by the \LaTeX{} compiler.
%
% %%%%%%%%%%%%%%%%%%%%%%%%%%%%%%%%%%%%%%
% \paragraph{Main File.}
%
% The main file is called |cdocsamp.tex|.
%
% Load the \textsf{childdoc} definitions and
% declare the filename for the main document:
%    \begin{macrocode}
\input{childdoc.def}
\childdocmain{}
%    \end{macrocode}

% Optional override for |\version| flag:
%    \begin{macrocode}
%%\ifchilddoc\else\providecommand{\version}{draft}\fi
%    \end{macrocode}

% Define the default values for the |\version| flag
% (|final| for the main file and |draft| for childs):
%    \begin{macrocode}
\ifchilddoc
\providecommand{\version}{draft}
\else
\providecommand{\version}{final}
\fi
%    \end{macrocode}

% Load the standard document class:
%    \begin{macrocode}
\documentclass[12pt]{article}
%    \end{macrocode}

% Start the document body:
%    \begin{macrocode}
\begin{document}
%    \end{macrocode}

% Declare a title page.
% Print title, part of document being processed and version flag:
%    \begin{macrocode}
\addtocounter{page}{-1}
\begin{center}
{\LARGE\bfseries{}childdoc example\par}
\vspace{1cm}
\ifchilddoc
\ifchilddocmanual part\else chapter\fi:
`\childdocname' of `\childdocjob'\par
\else
main document: `\childdocjob'\par
\fi
version: \version\par
\end{center}
\newpage
%    \end{macrocode}

% Manually include selected file,
% otherwise process as usual:
%    \begin{macrocode}
\ifchilddocmanual
\section*{part `\childdocname'}
\input{\childdocname}
\else
%    \end{macrocode}

% Include the two chapters:
%    \begin{macrocode}
\include{cdocsch1}
\include{cdocsch2}
%    \end{macrocode}

% Include the two parts unless only chapters should be displayed:
%    \begin{macrocode}
\ifchilddoc\else
\section{part three}
\input{cdocspt3}
\section{part four}
\input{cdocspt4}
\fi
%    \end{macrocode}

% Process as usual until here:
%    \begin{macrocode}
\fi
%    \end{macrocode}

% End of document body:
%    \begin{macrocode}
\end{document}
%    \end{macrocode}
%\iffalse
%</samplemain>
%\fi
%
% %%%%%%%%%%%%%%%%%%%%%%%%%%%%%%%%%%%%%%
% \paragraph{Chapter Include Files.}
%
% The include files are called |cdocsch1.tex| and |cdocsch2.tex|.
%
%\iffalse
%<*samplechap1|samplechap2>
%\fi

% Optional override for |\version| flag:
%    \begin{macrocode}
%%\providecommand{\version}{final}
%    \end{macrocode}

% Include the main document:
%    \begin{macrocode}
\input{childdoc.def}
\childdocof{cdocsamp}
%    \end{macrocode}

%\iffalse
%</samplechap1|samplechap2>
%\fi
%
%\iffalse
%<*samplechap1>
%\fi
% Some text for chapter 1:
%    \begin{macrocode}
\section{one}
some text in chapter one
%    \end{macrocode}

%\iffalse
%</samplechap1>
%\fi
% Some text for chapter 2:
%\iffalse
%<*samplechap2>
%\fi
%    \begin{macrocode}
\section{two}
more text in chapter two
%    \end{macrocode}

%\iffalse
%</samplechap2>
%\fi
%
% %%%%%%%%%%%%%%%%%%%%%%%%%%%%%%%%%%%%%%
% \paragraph{Part Include Files.}
%
% The include files are called |cdocspt3.tex| and |cdocspt4.tex|.
%
%\iffalse
%<*samplepart3|samplepart4>
%\fi

% Optional override for |\version| flag:
%    \begin{macrocode}
%%\providecommand{\version}{final}
%    \end{macrocode}

% Include the main document:
%    \begin{macrocode}
\input{childdoc.def}
\childdocby{cdocsamp}
%    \end{macrocode}

%\iffalse
%</samplepart3|samplepart4>
%\fi
%
%\iffalse
%<*samplepart3>
%\fi
% Some text for part 3:
%    \begin{macrocode}
some text in part three
%    \end{macrocode}

%\iffalse
%</samplepart3>
%\fi
% Some text for part 4:
%\iffalse
%<*samplepart4>
%\fi
%    \begin{macrocode}
more text in part four
%    \end{macrocode}

%\iffalse
%</samplepart4>
%\fi
%
% %%%%%%%%%%%%%%%%%%%%%%%%%%%%%%%%%%%%%%
% \paragraph{Forwarding for a Complete Draft.}
%
% The following forwarding file |cdocsdrf.tex|
% compiles the main document in draft mode:
%\iffalse
%<*sampledraft>
%\fi
%    \begin{macrocode}
\def\version{draft}
\input{childdoc.def}
\childdocforward{cdocsamp}
%    \end{macrocode}

%\iffalse
%</sampledraft>
%\fi
%
% %%%%%%%%%%%%%%%%%%%%%%%%%%%%%%%%%%%%%%
% \paragraph{Forwarding for Final Version of the Chapters.}
%
% The following forwarding files |cdocsfn1.tex| and |cdocsfn2.tex|
% (with identical content)
% compile the final versions of the child documents
% |cdocsch1.tex| and |cdocsch2.tex|, respectively:
%\iffalse
%<*samplefinal>
%\fi
%    \begin{macrocode}
\def\version{final}
\input{childdoc.def}
\childdocforwardprefix[cdocsamp]{cdocsfn}{cdocsch}
%    \end{macrocode}

%\iffalse
%</samplefinal>
%\fi
%
% %%%%%%%%%%%%%%%%%%%%%%%%%%%%%%%%%%%%%%
% \paragraph{Command Line Processing.}
%
% The following three command lines generate the output files
% |cdocscld|, |cdocscl1| and |cdocscl2|
% which should be identical to
% |cdocsdrf|, |cdocsch1| and |cdocsfn2|, respectively:
% \begin{center}
% \begin{tabular}{l}
% |latex -jobname cdocscld \|\\
% |  "\def\version{draft}\input{childdoc.def}\childdocforward{cdocsamp}"|\\
% |latex -jobname cdocscl1 \|\\
% |  "\input{childdoc.def}\childdocforward[cdocsamp]{cdocsch1}"|\\
% |latex -jobname cdocscl2 \|\\
% |  "\def\version{final}\input{childdoc.def}\childdocforward{cdocsch2}"|
% \end{tabular}
% \end{center}
% Note that the trailing backslash on each first line
% merely continues the input to the second line
% (for convenient cut ant paste).
% Furthermore, the command |latex| can be replaced by any
% of its alternative versions such as |pdflatex|.
%
% %%%%%%%%%%%%%%%%%%%%%%%%%%%%%%%%%%%%%%%%%%%%%%%%%%%%%%%%%%%%%%%%%%%%%%%%%%%%%%
% %%%%%%%%%%%%%%%%%%%%%%%%%%%%%%%%%%%%%%%%%%%%%%%%%%%%%%%%%%%%%%%%%%%%%%%%%%%%%%
% \section{Implementation}
%\iffalse
%<*package>
%\fi
%
% This section describes the definitions file |childdoc.def|.

% The definitions cannot be loaded using |\usepackage| or |\RequirePackage|
% which has a mechanism to prevent loading a style file more than once.
% When loading the definitions by means of |\input|
% multiple instances have to be prevented manually:
%\iffalse
%This code needs to be before the `\ProvidesFile' directive
%which is defined at the beginning of this file.
%Therefore it is also placed there and commented out here.
%</package>
%<*discard>
%\fi
%    \begin{macrocode}
\ifdefined\childdocmain\endinput\fi
%    \end{macrocode}
%\iffalse
%</discard>
%<*package>
%\fi
%
% \macro{\ifchilddoc}
% \macro{\ifchilddocmanual}
% The conditional |\ifchilddoc| tells whether a
% child (true) or main (false) document is being compiled.
% The conditional |\ifchilddocmanual| tells whether
% the |\includeonly| mechanism is used (false) or
% the selection of child files must be performed manually (true).
% The definitions initialise to false:
%    \begin{macrocode}
\newif\ifchilddoc
\newif\ifchilddocmanual
%    \end{macrocode}

% \macro{\childdocname}
% \macro{\childdocjob}
% The macro |\childdocname| stores the name of the main document
% to be compiled. The macro |\childdocjob| stores the name of
% the document on which the \LaTeX{} compiler was originally invoked.
% The content of |\jobname| cannot be compared
% to filenames specified in the source due to different catcodes.
% The following code rescans |\jobname|, stores the result
% in |\childdocname| and saves a copy in |\childdocjob|:
%    \begin{macrocode}
\edef\childdocname{\scantokens\expandafter{\jobname\noexpand}}
\let\childdocjob\childdocname
%    \end{macrocode}

% \macro{\childdocdisable}
% The macro |\childdocdisable| prevents the main file
% from being processed more than once.
% At this stage, the main document command |\childdocmain|
% is assumed to be called once again where it should do nothing.
% Any subsequent call to it should prevent
% a secondary processing of the main document
% It overwrites the forwarding commands
% |\childdocof| and |\childdocforward|
% with empty macros to prevent further inclusions of the main document:
%    \begin{macrocode}
\newcommand{\childdocdisable}
{
  \renewcommand{\childdocmain}[1]{\renewcommand{\childdocmain}[1]{\endinput}}
  \renewcommand{\childdocof}[1]{}
  \renewcommand{\childdocby}[2][]{}
  \renewcommand{\childdocforward}[2][]{}
  \renewcommand{\childdocdisable}{}
}
%    \end{macrocode}

% \macro{\childdocmain}
% The macro |\childdocmain| is to be called at the top of the main file
% with nothing or the main filename (without extension) as argument.
% First, it breaks loops.
% If the argument is not empty and does not match |\childdocname|
% (which is set by the first inclusion of |childdoc.def|),
% |\ifchilddoc| is set to true, |\includeonly| is applied to the child file
% and |\jobname| is set to the main file
% (for proper handling of |.aux| files):
%    \begin{macrocode}
\newcommand{\childdocmain}[1]
{
  \childdocdisable\childdocmain{}
  \if?#1?\else
    \begingroup
      \def\childdoctmp{#1}
      \ifx\childdoctmp\childdocname
        \def\childdoctmp{}
      \else
        \def\childdoctmp
        {
          \childdoctrue
          \includeonly{\childdocname}
          \def\childdocjob{#1}
          \def\jobname{#1}
        }
      \fi
      \expandafter
    \endgroup
    \childdoctmp
  \fi
}
%    \end{macrocode}

% \macro{\childdocof}
% The command |\childdocof| redirects
% compilation to the main file |#1|.
%    \begin{macrocode}
\newcommand{\childdocof}[1]
{
  \childdocdisable
  \childdoctrue
  \includeonly{\childdocname}
  \def\jobname{#1}
  \def\childdocjob{#1}
  \input{#1}
}
%    \end{macrocode}

% \macro{\childdocby}
% The command |\childdocby| ....
%    \begin{macrocode}
\newcommand{\childdocby}[2][]
{
  \childdocdisable
  \childdoctrue
  \childdocmanualtrue
  \if?#1?\else
    \def\jobname{#2}
  \fi
  \def\childdocjob{#2}
  \input{#2}
  \endinput
}
%    \end{macrocode}

% \macro{\childdocforward}
% The command |\childdocforward| redirects
% compilation to the main file or
% (if the optional argument is given) a child file.
% Parameters are set as if the main file
% or a child file starting with |\childdocof| was compiled.
% Then compilation is handed over to the main file:
%    \begin{macrocode}
\newcommand{\childdocforward}[2][]
{
  \begingroup
    \if?#1?
      \def\childdoctmp
      {
        \def\childdocname{#2}
        \def\childdocjob{#2}
        \def\jobname{#2}
        \input{#2}
        \endinput
      }
    \else
      \def\childdoctmp
      {
        \childdocdisable
        \def\childdocname{#2}
        \childdoctrue
        \includeonly{#2}
        \def\childdocjob{#1}
        \def\jobname{#1}
        \input{#1}
        \endinput
      }
    \fi
    \expandafter
  \endgroup
  \childdoctmp
}
%    \end{macrocode}

% \macro{\childdocforwardprefix}
% The command |\childdocforwardprefix| redirects
% compilation to the main or a child file by means of a pattern.
% The prefix |#1| in the current filename is replaced by |#2|
% and the suffix of the current filename is kept
% (it is assumed that the filename does not contain the substring `|~~~|'
% which is used as a delimiter).
% Compilation is handed over to the new file by |\childdocforward|:
%    \begin{macrocode}
\newcommand{\childdocforwardprefix}[3][]
{
  \begingroup
    \def\childdocextract #2##1~~~{\def\childdoctmp{\childdocforward[#1]{#3##1}}}
    \expandafter\childdocextract\childdocname~~~
    \expandafter
  \endgroup
  \childdoctmp
}
%    \end{macrocode}

% \macro{\childdoc}
% The deprecated macro |\childdoc| is a legacy version of |\childdocmain|:
%    \begin{macrocode}
\newcommand{\childdoc}{\childdocmain}
%    \end{macrocode}

% \macro{\childdocredirect}
% The deprecated macro |\childdocredirect| is a legacy version
% of |\childdocforward| and |\childdocforwardprefix|:
%    \begin{macrocode}
\newcommand{\childdocredirect}[2][]
{
  \begingroup
    \if?#1?
      \def\childdoctmp{\childdocforward{#2}}
    \else
      \def\childdoctmp{\childdocforwardprefix{#1}{#2}}
    \fi
    \expandafter
  \endgroup
  \childdoctmp
}
%    \end{macrocode}

%\iffalse
%</package>
%\fi
%
\endinput
|\\
|\childdocforward{|\textit{main}|}|\\
\end{tabular}
\end{center}
%
or alternatively with:
%
\begin{center}
\begin{tabular}{l}
|% \iffalse
%
% childdoc.dtx Copyright (C) 2017-2018 Niklas Beisert
%
% This work may be distributed and/or modified under the
% conditions of the LaTeX Project Public License, either version 1.3
% of this license or (at your option) any later version.
% The latest version of this license is in
%   http://www.latex-project.org/lppl.txt
% and version 1.3 or later is part of all distributions of LaTeX
% version 2005/12/01 or later.
%
% This work has the LPPL maintenance status `maintained'.
%
% The Current Maintainer of this work is Niklas Beisert.
%
% This work consists of the files childdoc.dtx and childdoc.ins
% and the derived files childdoc.def and cdocsamp.tex with
% cdocsch1.tex, cdocsch2.tex, cdocsdrf.tex, cdocsfn1.tex, cdocsfn2.tex.
%
%<package>\ifdefined\childdocmain\endinput\fi
%<package>\ProvidesFile{childdoc.def}[2018/12/30 v2.0 child document driver]
%<samplemain>\ProvidesFile{cdocsamp.tex}[2018/12/30 v2.0 sample for childdoc]
%<*driver>
%\ProvidesFile{childdoc.drv}[2018/12/30 v2.0 childdoc reference manual file]
\PassOptionsToClass{10pt,a4paper}{article}
\documentclass{ltxdoc}

\usepackage[margin=35mm]{geometry}
\usepackage{hyperref}
\usepackage{hyperxmp}
\usepackage[usenames]{color}

\hypersetup{colorlinks=true}
\hypersetup{pdfstartview=FitH}
\hypersetup{pdfpagemode=UseNone}
\hypersetup{pdfsource={}}
\hypersetup{pdflang={en-UK}}
\hypersetup{pdfcopyright={Copyright 2017-2018 Niklas Beisert.
  This work may be distributed and/or modified under the
  conditions of the LaTeX Project Public License, either version 1.3
  of this license or (at your option) any later version.}}
\hypersetup{pdflicenseurl={http://www.latex-project.org/lppl.txt}}
\hypersetup{pdfcontactaddress={ETH Zurich, ITP, HIT K,
  Wolfgang-Pauli-Strasse 27}}
\hypersetup{pdfcontactpostcode={8093}}
\hypersetup{pdfcontactcity={Zurich}}
\hypersetup{pdfcontactcountry={Switzerland}}
\hypersetup{pdfcontactemail={nbeisert@itp.phys.ethz.ch}}
\hypersetup{pdfcontacturl={http://people.phys.ethz.ch/\xmptilde nbeisert/}}

\newcommand{\secref}[1]{\hyperref[#1]{section \ref*{#1}}}

\parskip1ex
\parindent0pt
\let\olditemize\itemize
\def\itemize{\olditemize\parskip0pt}

\begin{document}

\title{The \textsf{childdoc} Package}
\hypersetup{pdftitle={The childdoc Package}}
\author{Niklas Beisert\\[2ex]
  Institut f\"ur Theoretische Physik\\
  Eidgen\"ossische Technische Hochschule Z\"urich\\
  Wolfgang-Pauli-Strasse 27, 8093 Z\"urich, Switzerland\\[1ex]
  \href{mailto:nbeisert@itp.phys.ethz.ch}
  {\texttt{nbeisert@itp.phys.ethz.ch}}}
\hypersetup{pdfauthor={Niklas Beisert}}
\hypersetup{pdfsubject={Manual for the LaTeX2e Package childdoc}}
\date{30 December 2018, \textsf{v2.0}}
\maketitle

\begin{abstract}\noindent
\textsf{childdoc} is a \LaTeXe{} package
that enables the direct compilation
of document sections included by |\include|
to individual files.
\end{abstract}

\begingroup
\parskip0ex
\tableofcontents
\endgroup

%%%%%%%%%%%%%%%%%%%%%%%%%%%%%%%%%%%%%%%%%%%%%%%%%%%%%%%%%%%%%%%%%%%%%%%%%%%%%%%%
%%%%%%%%%%%%%%%%%%%%%%%%%%%%%%%%%%%%%%%%%%%%%%%%%%%%%%%%%%%%%%%%%%%%%%%%%%%%%%%%
\section{Introduction}

\LaTeX{} provides a mechanism to structure a large document (such as a book)
into a main file and several child files (containing the chapters)
using the |\include| command.
This mechanism is beneficial for documents
which span hundreds of pages in order to
make the source file(s) more manageable.
Moreover, compilation can be restricted to
selected child files by means of the |\includeonly| command.
The latter feature can be used to reduce the compilation time while editing
(this was significantly more useful in the earlier days of \LaTeX{})
or to generate a smaller document which is easier to navigate.
Another application of |\includeonly| is to generate
documents consisting of selected parts of the complete document.

However, there are a few drawbacks of the plain |\include| mechanism:
\begin{itemize}
\item
The child files cannot be compiled on their own,
they can only be compiled via the main file.
A naive editing environment
(such as a text editor with an option
to have the current file processed by \LaTeX)
may require one to switch to the main file before compiling;
attempting to compile the child file produces errors.
\item
The main file must be modified (each time)
to adjust the |\includeonly| command
to the present needs. This easily leaves the main file in a messy state.
\item
The generated document will always carry the filename
of the main document. This is inconvenient if
several child files are to be compiled and
to be kept for distribution.
\end{itemize}

The present package provides a simple interface
to make child files individually compilable by \LaTeX{}.
Compiling a child file then has the same effect as compiling
the main file with an |\includeonly| command
to select the appropriate child.
Moreover the generated document will carry the name of the child
rather than the main file.
This resolves all three above issues.

This feature is meant to make the editing of books,
thesis documents and lecture notes somewhat more convenient.
However, the package can also be used efficiently for
composing a series of documents (such as exercise sheets)
which are typically distributed individually.
It then assists the author in generating the individual documents
(potentially in different versions)
as well as a document containing the collected series.
Another application is in developing style files
or other kinds of included material
where compilation of the style file could redirect
to a sample or test file.

%%%%%%%%%%%%%%%%%%%%%%%%%%%%%%%%%%%%%%%%%%%%%%%%%%%%%%%%%%%%%%%%%%%%%%%%%%%%%%%%
%%%%%%%%%%%%%%%%%%%%%%%%%%%%%%%%%%%%%%%%%%%%%%%%%%%%%%%%%%%%%%%%%%%%%%%%%%%%%%%%
\section{Usage}

First of all, the package \textsf{childdoc} is \emph{not} a standard
\LaTeXe{} |.sty| style file! Therefore it needs to be invoked in
a non-standard way.

%%%%%%%%%%%%%%%%%%%%%%%%%%%%%%%%%%%%%%%%%%%%%%%%%%%%%%%%%%%%%%%%%%%%%%%%%%%%%%%%
\subsection{Included Files}
\label{sec:include}

%%%%%%%%%%%%%%%%%%%%%%%%%%%%%%%%%%%%%%%%
\DescribeMacro{\childdocmain}
To use the package, add the commands
\begin{center}
\begin{tabular}{l}
|\input{childdoc.def}|\\
|\childdocmain{}|\\
\end{tabular}
\end{center}
at the very top of the main \LaTeX{} file,
in particular \emph{before} the |\documentclass| statement!
The argument of |\childdocmain| should be left empty
(but it must be present).

%%%%%%%%%%%%%%%%%%%%%%%%%%%%%%%%%%%%%%%%
\DescribeMacro{\childdocof}
Furthermore, add the commands
\begin{center}
\begin{tabular}{l}
|\input{childdoc.def}|\\
|\childdocof{|\textit{main}|}|\\
\end{tabular}
\end{center}
at the top of every child file \textit{child}
which is included by |\include{|\textit{child}|}|
from within the main file
(or at least for those files to be compiled individually).
The argument \textit{main} must be the filename of the main file.

There are a couple of
considerations in setting up the main and child documents:

%%%%%%%%%%%%%%%%%%%%%%%%%%%%%%%%%%%%%%%%
\paragraph{Restrictions.}

Please note the following restrictions:
\begin{itemize}
\item
|\childdocmain| must be called with one argument \textit{main}
to ensure compatibility with earlier version of the package.
It must either be empty (|\childdocmain{}|)
or precisely match the filename of the main file in which it is specified.
See \secref{sec:detection} for further information.
\item
The filename \textit{main} must be specified without the |.tex| extension.
\item
The filename \textit{main} is case sensitive
(even in case-insensitive file systems)
due to internal string comparison.
\item
The argument \textit{main} should be fully expanded, it cannot be a macro.
\item
Subdirectories and special characters should be avoided in filenames.
\item
The command |\childdocmain{|\textit{main}|}| must be followed by a whitespace.
It should not be followed immediately by another command
or by a comment mark `|%|'.
This is because the \TeX{} parser reads the token immediately following
the argument of |\childdocmain| and puts it
at the beginning of every child section;
however, a white\-space is ignored.
\end{itemize}

%%%%%%%%%%%%%%%%%%%%%%%%%%%%%%%%%%%%%%%%
\paragraph{Content of Main File.}

It is advisable to place all content in the child files included by |\include|.
Any output contained in the main file will appear in all child documents
unless suppressed manually;
it cannot be suppressed automatically by the |\includeonly| directive
and thus should normally be avoided.
A method to include some content in the main file
by means of conditional processing is described in \secref{sec:conditional}.

%%%%%%%%%%%%%%%%%%%%%%%%%%%%%%%%%%%%%%%%
\paragraph{Page Numbering.}

When only a part of the document is compiled,
the appropriate numbering of pages
(as well as other status parameters)
is determined from the |.aux| files.
The latter contain information from previous passes.
However this information needs to propagate through
all intermediate child documents.
Therefore the page numbering in child documents may well
be inconsistent until the complete document is compiled at least once.

A useful (if unconventional) way to always ensure a consistent
page numbering is to restart the numbering in each child document
and denote the pages by `\textit{child}|.|\textit{page}'
where \textit{child} represents the chapter/section number of the child file.
This can be achieved by the command
|\numberwithin{page}{|\textit{child}|}|
of the \textsf{amsmath} package
where \textit{child} can be |chapter| or |section|
depending on the chosen structuring.
Alternatively, one can modify the macro |\thepage| appropriately
and reset the counter |page| at the start of each child file.

%%%%%%%%%%%%%%%%%%%%%%%%%%%%%%%%%%%%%%%%%%%%%%%%%%%%%%%%%%%%%%%%%%%%%%%%%%%%%%%%
\subsection{Conditional Processing}
\label{sec:conditional}

The package provides a mechanism to compile different versions
of a document. To customise the versions further some conditional processing
can come in handy to distinguish which version is being compiled.
The package provides two macros to describe the compilation context:

%%%%%%%%%%%%%%%%%%%%%%%%%%%%%%%%%%%%%%%%
\DescribeMacro{\ifchilddoc}
The conditional |\ifchilddoc| distinguishes between the compilation of
child documents and the main document:
%
\begin{center}
|\ifchilddoc |\textit{child-code}| |[|\||else |\textit{main-code}]| \||fi|
\end{center}

%%%%%%%%%%%%%%%%%%%%%%%%%%%%%%%%%%%%%%%%
\DescribeMacro{\childdocname}
\DescribeMacro{\childdocjob}
The macro |\childdocname| contains the filename (without extension)
of the main or child file being processed.
Note that |\childdocjob| will always contain the name of the main file.

%%%%%%%%%%%%%%%%%%%%%%%%%%%%%%%%%%%%%%%%
\paragraph{Title Page.}

Conditional processing can be used to include a title or banner page
in the main document when proper precautions are taken.
Importantly, the code in the main file should ensure that the page counter
(as well as other status parameters which are stored in the |.aux| files)
takes the same value after the conditional processing.
Otherwise the page numbers may take divergent values
depending on which part is compiled.

For example, a title page could be declared by:
%
\begin{center}
\begin{tabular}{l}
|\ifchilddoc\||else|\\
|\addtocounter{page}{-1}|\\
\textit{code for title page}\\
|\newpage|\\
|\||fi|
\end{tabular}
\end{center}
%
A banner page for the child documents can be generated by:
%
\begin{center}
\begin{tabular}{l}
|\ifchilddoc|\\
|\addtocounter{page}{-1}|\\
\textit{code for banner page}\\
|\newpage|\\
|\||fi|
\end{tabular}
\end{center}
%
Here one could write a message such as:
\begin{center}
|This is the part \childdocname{} of \childdocjob{}.|
\end{center}

%%%%%%%%%%%%%%%%%%%%%%%%%%%%%%%%%%%%%%%%%%%%%%%%%%%%%%%%%%%%%%%%%%%%%%%%%%%%%%%%
\subsection{Flags}
\label{sec:flags}

The package makes it easy to generate different versions
of the main or child documents.
To this end compilation flags can be defined
and assigned different default values.
They will be particularly useful in conjunction
with the forwarding mechanism described in \secref{sec:forward}.

For example, it may be useful to have a flag |\version|
which can be set to |draft| or |final|.
The document source will contain some conditional code
depending on the value of |\version|.
Suppose further, the flag should default to |final| for the main file
and to |draft| for child files
which is a natural assignment for editing the document.
This is achieved by placing the following code
in the preamble of the main document
(below the |\childdocmain| directive):
%
\begin{center}
\begin{tabular}{l}
|\ifchilddoc|\\
|\providecommand{\version}{draft}|\\
|\||else|\\
|\providecommand{\version}{final}|\\
|\||fi|
\end{tabular}
\end{center}
%
The definition by |\providecommand| makes sure
that previous definitions are not overwritten.
Further statements |\providecommand{\version}{...}|
can thus be added before the above code to override it.

For the main file, one might add a line
(between |\childdocmain| and the above block)
%
\begin{center}
|%\ifchilddoc\||else\providecommand{\version}{draft}\||fi|
\end{center}
%
which can be uncommented to produce a draft version.
Likewise one can add a line to the very top of a child file
(above the |\childdocof{|\textit{main}|}| directive)
%
\begin{center}
|%\providecommand{\version}{final}|
\end{center}
%
which can be uncommented to produce the final version of this child document.

%%%%%%%%%%%%%%%%%%%%%%%%%%%%%%%%%%%%%%%%%%%%%%%%%%%%%%%%%%%%%%%%%%%%%%%%%%%%%%%%
\subsection{Forwarding}
\label{sec:forward}

Different versions of the main or child documents
using compilation flags as described in \secref{sec:flags}
can be (permanently) stored in different files
for convenient compilation, viewing and distribution.
To this end, the package defines a command
to pass on compilation to a different file:

%%%%%%%%%%%%%%%%%%%%%%%%%%%%%%%%%%%%%%%%
\DescribeMacro{\childdocforward}
The command |\childdocforward| redirects processing to
another source file:
%
\begin{center}
\begin{tabular}{l}
|\input{childdoc.def}|\\
|\childdocforward[|\textit{main}|]{|\textit{dest}|}|\\
\end{tabular}
\end{center}
%
The argument \textit{dest} is the destination file
(without extension).
It should be the main file or one of the child files.
Note that further \textsf{childdoc} directives
such as |\childdocof| and |\childdocforward|
in the indicated file will be processed in this form.
The optional argument \textit{main}
passes on directly to the main file \textit{main}
while pretending to compile the child \textit{dest}.
This form behaves as if \textit{dest}
issues |\childdocof{|\textit{main}|}| right away,
and no further \textsf{childdoc} directives will be processed.

%%%%%%%%%%%%%%%%%%%%%%%%%%%%%%%%%%%%%%%%
\DescribeMacro{\...prefix}
In the alternative form |\childdocforwardprefix|,
%
\begin{center}
\begin{tabular}{l}
|\input{childdoc.def}|\\
|\childdocforwardprefix[|\textit{main}|]{|\textit{prefix}|}{|\textit{dest}|}|
\end{tabular}
\end{center}
%
the destination file is determined by a pattern
depending on the current file:
To make this work, the current file must be called
`{\textit{prefix}\hspace{0.2em}\textit{suffix}}'
with \textit{prefix} matching precisely the argument.
Processing is then passed on to the file
`{\textit{dest}\hspace{0.2em}\textit{suffix}}'.
Surely, the same effect is achieved by
directly specifying the
argument `{\textit{dest}\hspace{0.2em}\textit{suffix}}'
in the first form.
However, that requires to set up a different file
for each child. With the alternative form of the command
all these files can have exactly the same content
which simplifies setting them up and maintaining them.

For example, the following file |draft.tex|
with a compilation flag |\version| as described in \secref{sec:flags}
compiles the main document as a draft:
%
\begin{center}
\begin{tabular}{l}
|\def\version{draft}|\\
|\input{childdoc.def}|\\
|\childdocforward{|\textit{main}|}|
\end{tabular}
\end{center}
%
Likewise, the following files |final|\textit{nn}|.tex|
compile the final version of the child document
|child|\textit{nn}|.tex|:
%
\begin{center}
\begin{tabular}{l}
|\def\version{final}|\\
|\input{childdoc.def}|\\
|\childdocforwardprefix{final}{child}|
\end{tabular}
\end{center}
%

Note that when several versions of a main file and/or of each child file
are to be generated, it may be convenient to set up a |Makefile| or
shell script to automatise the process.

%%%%%%%%%%%%%%%%%%%%%%%%%%%%%%%%%%%%%%%%%%%%%%%%%%%%%%%%%%%%%%%%%%%%%%%%%%%%%%%%
\subsection{Command Line Processing}
\label{sec:commandline}

The effect of redirection files can also be achieved by invoking
the \LaTeX{} compiler with a more elaborate command line.
Most conveniently this should be done as part
of a shell script or a |Makefile|.

When using \textsf{childdoc} in the main file, the following
command lines effectively perform a redirection
(note that depending on the shell being used,
backslashes may have to be doubled: `|\|' $\to$ `|\\|'):
%
\begin{center}
|... -jobname "|\textit{target}|" |\\|"|[\textit{flags}]%
|\input{childdoc.def}\childdocforward[|\textit{main}|]{|\textit{dest}|}"|
\end{center}
%
Here \textit{target} is the name of the output file,
\textit{main} is the name of the main file
and \textit{dest} is the name of the main or child file to be processed
(all filenames without extensions).
The optional argument \textit{main} can be omitted
if \textit{main} matches \textit{dest}.
Optionally, compilation \textit{flags} can be defined via |\def| commands.
This command line makes the \TeX{} engine believe
it is compiling the file \textit{target}
whose content is specified as the latter parameter.
The provided code then forwards the processing to
\textit{main} or \textit{dest} as described in \secref{sec:forward}.

%%%%%%%%%%%%%%%%%%%%%%%%%%%%%%%%%%%%%%%%%%%%%%%%%%%%%%%%%%%%%%%%%%%%%%%%%%%%%%%%
\subsection{Include by Input}
\label{sec:input}

Including child documents by |\include| has some restrictions by design.
Most notably, the content of a child document always occupies
its own set of pages; pages cannot be shared between child documents.
Usually, this behaviour makes perfect sense
because each child document contain an essential part of the document.
However, in some situations it may be desirable to compose
a document from a collection of parts
without having mandatory page breaks between then.
For this case, the package
provides a mechanism to include parts
by |\input| which can also be processed individually.
However, by construction this mechanism
requires manual handling of the content to be output.

%%%%%%%%%%%%%%%%%%%%%%%%%%%%%%%%%%%%%%%%
\DescribeMacro{\ifchilddocmanual}
The main file should be prepared as usual, see \secref{sec:include}.
However, the document body must make a distinction
between processing of an individual part and of the main document, e.g.:
%
\begin{center}
\begin{tabular}{l}
|\ifchilddocmanual|\\
|\input{\childdocname}|\\
|\||else|\\
\textit{document body with }|\input{|\textit{part}|}|\\
|\||fi|
\end{tabular}
\end{center}
%
The conditional |\ifchilddocmanual| is true whenever
a part to be included by |\input| is being compiled,
and the name of the part is stored in |\childdocname|.

%%%%%%%%%%%%%%%%%%%%%%%%%%%%%%%%%%%%%%%%
\DescribeMacro{\childdocby}
Each part to be included by |\input| should start with:
%
\begin{center}
\begin{tabular}{l}
|\input{childdoc.def}|\\
|\childdocby{|\textit{main}|}|\\
\end{tabular}
\end{center}
%
The directive |\childdocby| is similar to |\childdocof|
described in \secref{sec:include},
but the subsequent selection of content must be done manually.
To that end, both |\ifchilddoc| and |\ifchilddocmanual|
will be true upon processing of a part,
and the name of the part is stored in |\childdocname|.
Note that |\jobname| will be set to the filename of the current part
so that each part receives an individual |.aux| file
that does not interfere with the |.aux| file(s) of the main document.
This behaviour can be altered by the alternative form
|\childdocby[*]{|\textit{main}|}| (with a non-empty optional argument)
which uses the |.aux| file of the main document
by setting |\jobname| to \textit{main}.

%%%%%%%%%%%%%%%%%%%%%%%%%%%%%%%%%%%%%%%%%%%%%%%%%%%%%%%%%%%%%%%%%%%%%%%%%%%%%%%%
\subsection{Driver Development}
\label{sec:driver}

The \textsf{childdoc} mechanism can also be use for the development
of definition files such as \LaTeX{} styles or classes.
This case differs from the above setup with multiple parts
included by |\include| in that no |\includeonly| should be invoked.
This can be achieved by starting the include file
(before |\ProvidesPackage|) with:
%
\begin{center}
\begin{tabular}{l}
|\input{childdoc.def}|\\
|\childdocforward{|\textit{main}|}|\\
\end{tabular}
\end{center}
%
or alternatively with:
%
\begin{center}
\begin{tabular}{l}
|\input{childdoc.def}|\\
|\childdocby{|\textit{main}|}|\\
\end{tabular}
\end{center}
%
Both forms have slightly different effects as described above.
The main file is prepared as usual, see \secref{sec:include}.

%%%%%%%%%%%%%%%%%%%%%%%%%%%%%%%%%%%%%%%%%%%%%%%%%%%%%%%%%%%%%%%%%%%%%%%%%%%%%%%%
\subsection{Legacy Detection}
\label{sec:detection}

The directive |\childdocmain| in the main file can detect
whether the complete document or merely a child is to be compiled
even without using the directive |\childdocof|.
This method is deprecated because it is less robust
and there is no compelling reason to use it;
it is merely provided for backward compatibility
and it may be removed in future versions.

If the detection mechanism is to be used,
it is mandatory to correctly specify
the filename of the main file as the argument of |\childdocmain|:
%
\begin{center}
\begin{tabular}{l}
|\input{childdoc.def}|\\
|\childdocmain{|\textit{main}|}|\\
\end{tabular}
\end{center}
%
If |\jobname| does not match the argument \textit{main} of |\childdocmain|,
it is assumed that |\jobname| points to the child file to be compiled.
When using |\childdocmain| with the main file specified as argument,
it suffices to start a child file
with just |\input{|\textit{main}|}|
without loading of the package and using |\childdocof|.
If instead all processing is done
with the appropriate \textsf{childdoc} directives,
the argument of \textit{main} of |\childdocmain| can be empty.

An alternative version of the command line processing described
in \secref{sec:commandline} using the detection mechanism reads:
%
\begin{center}
|... -jobname "|\textit{target}|" "|[\textit{flags}]%
[|\def\jobname{|\textit{dest}|}|]|\input{|\textit{main}|}"|
\end{center}

%%%%%%%%%%%%%%%%%%%%%%%%%%%%%%%%%%%%%%%%%%%%%%%%%%%%%%%%%%%%%%%%%%%%%%%%%%%%%%%%
\subsection{Manual Code}
\label{sec:manual}

In case one cannot be certain whether the definitions file |childdoc.def|
is installed on the target \TeX{} distribution
and one prefers not to ship it,
it is conceivable to paste a few relevant commands into the sources.

To that end, drop all statements |\input{childdoc.def}|
and perform the replacements as outlined below.
Instead of |\childdocmain{|\textit{main}|}| add the following code
to the top of the main file:
%
\begin{center}
\begin{tabular}{l}
|\||ifdefined\childdocname\endinput\||fi\newif\ifchilddoc|\\
|\edef\childdocname{\scantokens\expandafter{\jobname\noexpand}}|\\
|\def\childdocmain{|\textit{main}|}\||ifx\childdocmain\childdocname\||else|\\
|\childdoctrue\includeonly{\childdocname}\let\jobname\childdocmain\||fi|\\
\end{tabular}
\end{center}
%
Instead of |\childdocof{|\textit{main}|}| just include the main file
at the top of each child file:
%
\begin{center}
|\input{|\textit{main}|}|
\end{center}
%
A simple redirection |\childdocforward{|\textit{dest}|}| is achieved by:
%
\begin{center}
|\def\jobname{|\textit{dest}|}\input{\jobname}|
\end{center}
%
The redirection with prefix
|\childdocforwardprefix[|\textit{prefix}|]{|\textit{dest}|}|
is accomplished by:
%
\begin{center}
\begin{tabular}{l}
|{\edef\jobname{\scantokens\expandafter{\jobname\noexpand}}|\\
|\def\redirectjob |\textit{prefix}|#1~~~{\gdef\jobname{|\textit{dest}|#1}}|\\
|\expandafter\redirectjob\jobname~~~}\input{\jobname}|
\end{tabular}
\end{center}

In an alternative approach,
child documents can be compiled by a specific command line
without additional code or specific definitions:
%
\begin{center}
|... -jobname "|\textit{target}|" "|[\textit{flags}]%
|\includeonly{|\textit{dest}|}\input{|\textit{main}|}"|
\end{center}
%

%%%%%%%%%%%%%%%%%%%%%%%%%%%%%%%%%%%%%%%%%%%%%%%%%%%%%%%%%%%%%%%%%%%%%%%%%%%%%%%%
%%%%%%%%%%%%%%%%%%%%%%%%%%%%%%%%%%%%%%%%%%%%%%%%%%%%%%%%%%%%%%%%%%%%%%%%%%%%%%%%
\section{Information}

%%%%%%%%%%%%%%%%%%%%%%%%%%%%%%%%%%%%%%%%%%%%%%%%%%%%%%%%%%%%%%%%%%%%%%%%%%%%%%%%
\subsection{Copyright}

Copyright \copyright{} 2017--2018 Niklas Beisert

This work may be distributed and/or modified under the
conditions of the \LaTeX{} Project Public License, either version 1.3
of this license or (at your option) any later version.
The latest version of this license is in
  \url{http://www.latex-project.org/lppl.txt}
and version 1.3 or later is part of all distributions of \LaTeX{}
version 2005/12/01 or later.

This work has the LPPL maintenance status `maintained'.

The Current Maintainer of this work is Niklas Beisert.

This work consists of the files |README.txt|, |childdoc.ins| and |childdoc.dtx|
as well as the derived files |childdoc.def|, |cdocsamp.tex|
with |cdocsch1.tex|, |cdocsch2.tex|, |cdocspt3.tex|, |cdocspt4.tex|,
|cdocsdrf.tex|, |cdocsfn1.tex|, |cdocsfn2.tex|
as well as |childdoc.pdf|.

%%%%%%%%%%%%%%%%%%%%%%%%%%%%%%%%%%%%%%%%%%%%%%%%%%%%%%%%%%%%%%%%%%%%%%%%%%%%%%%%
\subsection{Files and Installation}

The package consists of the files:
%
\begin{center}
\begin{tabular}{ll}
    |README.txt|   & readme file \\
    |childdoc.ins| & installation file \\
    |childdoc.dtx| & source file \\
    |childdoc.def| & definition file \\
    |cdocsamp.tex| & sample main file \\
    |cdocsch1.tex| & sample include file \\
    |cdocsch2.tex| & sample include file \\
    |cdocspt3.tex| & sample part file \\
    |cdocspt4.tex| & sample part file \\
    |cdocsdrf.tex| & sample redirection file \\
    |cdocsfn1.tex| & sample redirection file \\
    |cdocsfn2.tex| & sample redirection file \\
    |childdoc.pdf| & manual
\end{tabular}
\end{center}
%
The distribution consists of the files
|README.txt|, |childdoc.ins| and |childdoc.dtx|.
%
\begin{itemize}
\item
Run (pdf)\LaTeX{} on |childdoc.dtx|
to compile the manual |childdoc.pdf| (this file).
\item
Run \LaTeX{} on |childdoc.ins| to create the definitions file |childdoc.def|
and the sample |cdocsamp.tex| with include files
|cdocsch1.tex|, |cdocsch2.tex|, |cdocspt3.tex|, |cdocspt4.tex|,
|cdocsdrf.tex|, |cdocsfn1.tex|, |cdocsfn2.tex|.
Then copy the file |childdoc.def| to an appropriate directory of your \LaTeX{}
distribution, e.g.\ \textit{texmf-root}|/tex/latex/childdoc|.
\end{itemize}

%%%%%%%%%%%%%%%%%%%%%%%%%%%%%%%%%%%%%%%%%%%%%%%%%%%%%%%%%%%%%%%%%%%%%%%%%%%%%%%%
\subsection{Related CTAN Packages}

There are several other packages which offer a similar functionality:
%
\begin{itemize}
\item
The packages
\href{http://ctan.org/pkg/docmute}{\textsf{docmute}},
\href{http://ctan.org/pkg/includex}{\textsf{includex}} and
\href{http://ctan.org/pkg/standalone}{\textsf{standalone}}
provide commands to include only the document body of
a child file thus allowing both files to be compiled individually.
\item
The packages \href{http://ctan.org/pkg/subdocs}{\textsf{subdocs}}
and \href{http://ctan.org/pkg/subfiles}{\textsf{subfiles}}
provide structures in which the main and child documents can be
encapsulated and allowing them to be compiled individually.
The inclusion mechanism is different from the conventional |\include|.
\item
The package \href{http://ctan.org/pkg/combine}{\textsf{combine}}
is an elaborate solution to combine several documents into one.
\end{itemize}
%
See also the CTAN topic \href{http://ctan.org/topic/subdocs}{\textsf{subdocs}}
for further related packages.
The present package differs from the above solutions in that
a document structure constructed with the conventional |\include| mechanism
just needs two extra commands at the top of every file
such that all constituent files can be compiled individually.

%%%%%%%%%%%%%%%%%%%%%%%%%%%%%%%%%%%%%%%%%%%%%%%%%%%%%%%%%%%%%%%%%%%%%%%%%%%%%%%%
%\subsection{Feature Suggestions}
%
%The following is a list of features which may be useful for future
%versions of this package:
%%
%\begin{itemize}
%\item
%\ldots
%\end{itemize}

%%%%%%%%%%%%%%%%%%%%%%%%%%%%%%%%%%%%%%%%%%%%%%%%%%%%%%%%%%%%%%%%%%%%%%%%%%%%%%%%
\subsection{Revision History}

%%%%%%%%%%%%%%%%%%%%%%%%%%%%%%%%%%%%%%%%
\paragraph{v2.0:} 2018/12/30

\begin{itemize}
\item
immediate forward processing
\item
added |\childdocby| mechanism
\item
manual restructured
\end{itemize}

%%%%%%%%%%%%%%%%%%%%%%%%%%%%%%%%%%%%%%%%
\paragraph{v1.6:} 2018/01/17

\begin{itemize}
\item
application for development of include files
\item
corrections to manual
\end{itemize}

%%%%%%%%%%%%%%%%%%%%%%%%%%%%%%%%%%%%%%%%
\paragraph{v1.5:} 2017/05/21

\begin{itemize}
\item
more complete structuring introduced
\item
|\childdocof| introduced
\item
|\childdoc| renamed to |\childdocmain|
\item
|\childredirect| renamed to |\childdocforward| and |\childdocforwardprefix|
and functionality expanded
\end{itemize}

%%%%%%%%%%%%%%%%%%%%%%%%%%%%%%%%%%%%%%%%
\paragraph{v1.0:} 2017/04/27

\begin{itemize}
\item
manual and install package
\item
first version published on CTAN
\end{itemize}

%%%%%%%%%%%%%%%%%%%%%%%%%%%%%%%%%%%%%%%%
\paragraph{v0.6:} 2017/04/26

\begin{itemize}
\item
redirection mechanism added
\end{itemize}

%%%%%%%%%%%%%%%%%%%%%%%%%%%%%%%%%%%%%%%%
\paragraph{v0.5:} 2017/04/26

\begin{itemize}
\item
functionality in definition file
\end{itemize}


%%%%%%%%%%%%%%%%%%%%%%%%%%%%%%%%%%%%%%%%%%%%%%%%%%%%%%%%%%%%%%%%%%%%%%%%%%%%%%%%
%%%%%%%%%%%%%%%%%%%%%%%%%%%%%%%%%%%%%%%%%%%%%%%%%%%%%%%%%%%%%%%%%%%%%%%%%%%%%%%%
%%%%%%%%%%%%%%%%%%%%%%%%%%%%%%%%%%%%%%%%%%%%%%%%%%%%%%%%%%%%%%%%%%%%%%%%%%%%%%%%
\appendix

\settowidth\MacroIndent{\rmfamily\scriptsize 000\ }

 \DocInput{childdoc.dtx}

\end{document}
%</driver>
% \fi
%
% %%%%%%%%%%%%%%%%%%%%%%%%%%%%%%%%%%%%%%%%%%%%%%%%%%%%%%%%%%%%%%%%%%%%%%%%%%%%%%
% %%%%%%%%%%%%%%%%%%%%%%%%%%%%%%%%%%%%%%%%%%%%%%%%%%%%%%%%%%%%%%%%%%%%%%%%%%%%%%
% \section{Sample}
%\iffalse
%<*samplemain>
%\fi
%
% The following presents a sample document
% with two chapters, two parts, a title page,
% a compile flag as well as three forwarding files to set the flag.
% It consists of eight |.tex| files:
% \begin{center}
% \begin{tabular}{ll}
% |cdocsamp.tex|&main file\\
% |cdocsch1.tex|&include file for chapter 1\\
% |cdocsch2.tex|&include file for chapter 2\\
% |cdocspt3.tex|&include file for part 3\\
% |cdocspt4.tex|&include file for part 4\\
% |cdocsdrf.tex|&forwarding file for main file in draft mode\\
% |cdocsfi1.tex|&forwarding file for final version of chapter 1\\
% |cdocsfi2.tex|&forwarding file for final version of chapter 2\\
% \end{tabular}
% \end{center}
% Each of the eight files can be compiled directly by the \LaTeX{} compiler.
%
% %%%%%%%%%%%%%%%%%%%%%%%%%%%%%%%%%%%%%%
% \paragraph{Main File.}
%
% The main file is called |cdocsamp.tex|.
%
% Load the \textsf{childdoc} definitions and
% declare the filename for the main document:
%    \begin{macrocode}
\input{childdoc.def}
\childdocmain{}
%    \end{macrocode}

% Optional override for |\version| flag:
%    \begin{macrocode}
%%\ifchilddoc\else\providecommand{\version}{draft}\fi
%    \end{macrocode}

% Define the default values for the |\version| flag
% (|final| for the main file and |draft| for childs):
%    \begin{macrocode}
\ifchilddoc
\providecommand{\version}{draft}
\else
\providecommand{\version}{final}
\fi
%    \end{macrocode}

% Load the standard document class:
%    \begin{macrocode}
\documentclass[12pt]{article}
%    \end{macrocode}

% Start the document body:
%    \begin{macrocode}
\begin{document}
%    \end{macrocode}

% Declare a title page.
% Print title, part of document being processed and version flag:
%    \begin{macrocode}
\addtocounter{page}{-1}
\begin{center}
{\LARGE\bfseries{}childdoc example\par}
\vspace{1cm}
\ifchilddoc
\ifchilddocmanual part\else chapter\fi:
`\childdocname' of `\childdocjob'\par
\else
main document: `\childdocjob'\par
\fi
version: \version\par
\end{center}
\newpage
%    \end{macrocode}

% Manually include selected file,
% otherwise process as usual:
%    \begin{macrocode}
\ifchilddocmanual
\section*{part `\childdocname'}
\input{\childdocname}
\else
%    \end{macrocode}

% Include the two chapters:
%    \begin{macrocode}
\include{cdocsch1}
\include{cdocsch2}
%    \end{macrocode}

% Include the two parts unless only chapters should be displayed:
%    \begin{macrocode}
\ifchilddoc\else
\section{part three}
\input{cdocspt3}
\section{part four}
\input{cdocspt4}
\fi
%    \end{macrocode}

% Process as usual until here:
%    \begin{macrocode}
\fi
%    \end{macrocode}

% End of document body:
%    \begin{macrocode}
\end{document}
%    \end{macrocode}
%\iffalse
%</samplemain>
%\fi
%
% %%%%%%%%%%%%%%%%%%%%%%%%%%%%%%%%%%%%%%
% \paragraph{Chapter Include Files.}
%
% The include files are called |cdocsch1.tex| and |cdocsch2.tex|.
%
%\iffalse
%<*samplechap1|samplechap2>
%\fi

% Optional override for |\version| flag:
%    \begin{macrocode}
%%\providecommand{\version}{final}
%    \end{macrocode}

% Include the main document:
%    \begin{macrocode}
\input{childdoc.def}
\childdocof{cdocsamp}
%    \end{macrocode}

%\iffalse
%</samplechap1|samplechap2>
%\fi
%
%\iffalse
%<*samplechap1>
%\fi
% Some text for chapter 1:
%    \begin{macrocode}
\section{one}
some text in chapter one
%    \end{macrocode}

%\iffalse
%</samplechap1>
%\fi
% Some text for chapter 2:
%\iffalse
%<*samplechap2>
%\fi
%    \begin{macrocode}
\section{two}
more text in chapter two
%    \end{macrocode}

%\iffalse
%</samplechap2>
%\fi
%
% %%%%%%%%%%%%%%%%%%%%%%%%%%%%%%%%%%%%%%
% \paragraph{Part Include Files.}
%
% The include files are called |cdocspt3.tex| and |cdocspt4.tex|.
%
%\iffalse
%<*samplepart3|samplepart4>
%\fi

% Optional override for |\version| flag:
%    \begin{macrocode}
%%\providecommand{\version}{final}
%    \end{macrocode}

% Include the main document:
%    \begin{macrocode}
\input{childdoc.def}
\childdocby{cdocsamp}
%    \end{macrocode}

%\iffalse
%</samplepart3|samplepart4>
%\fi
%
%\iffalse
%<*samplepart3>
%\fi
% Some text for part 3:
%    \begin{macrocode}
some text in part three
%    \end{macrocode}

%\iffalse
%</samplepart3>
%\fi
% Some text for part 4:
%\iffalse
%<*samplepart4>
%\fi
%    \begin{macrocode}
more text in part four
%    \end{macrocode}

%\iffalse
%</samplepart4>
%\fi
%
% %%%%%%%%%%%%%%%%%%%%%%%%%%%%%%%%%%%%%%
% \paragraph{Forwarding for a Complete Draft.}
%
% The following forwarding file |cdocsdrf.tex|
% compiles the main document in draft mode:
%\iffalse
%<*sampledraft>
%\fi
%    \begin{macrocode}
\def\version{draft}
\input{childdoc.def}
\childdocforward{cdocsamp}
%    \end{macrocode}

%\iffalse
%</sampledraft>
%\fi
%
% %%%%%%%%%%%%%%%%%%%%%%%%%%%%%%%%%%%%%%
% \paragraph{Forwarding for Final Version of the Chapters.}
%
% The following forwarding files |cdocsfn1.tex| and |cdocsfn2.tex|
% (with identical content)
% compile the final versions of the child documents
% |cdocsch1.tex| and |cdocsch2.tex|, respectively:
%\iffalse
%<*samplefinal>
%\fi
%    \begin{macrocode}
\def\version{final}
\input{childdoc.def}
\childdocforwardprefix[cdocsamp]{cdocsfn}{cdocsch}
%    \end{macrocode}

%\iffalse
%</samplefinal>
%\fi
%
% %%%%%%%%%%%%%%%%%%%%%%%%%%%%%%%%%%%%%%
% \paragraph{Command Line Processing.}
%
% The following three command lines generate the output files
% |cdocscld|, |cdocscl1| and |cdocscl2|
% which should be identical to
% |cdocsdrf|, |cdocsch1| and |cdocsfn2|, respectively:
% \begin{center}
% \begin{tabular}{l}
% |latex -jobname cdocscld \|\\
% |  "\def\version{draft}\input{childdoc.def}\childdocforward{cdocsamp}"|\\
% |latex -jobname cdocscl1 \|\\
% |  "\input{childdoc.def}\childdocforward[cdocsamp]{cdocsch1}"|\\
% |latex -jobname cdocscl2 \|\\
% |  "\def\version{final}\input{childdoc.def}\childdocforward{cdocsch2}"|
% \end{tabular}
% \end{center}
% Note that the trailing backslash on each first line
% merely continues the input to the second line
% (for convenient cut ant paste).
% Furthermore, the command |latex| can be replaced by any
% of its alternative versions such as |pdflatex|.
%
% %%%%%%%%%%%%%%%%%%%%%%%%%%%%%%%%%%%%%%%%%%%%%%%%%%%%%%%%%%%%%%%%%%%%%%%%%%%%%%
% %%%%%%%%%%%%%%%%%%%%%%%%%%%%%%%%%%%%%%%%%%%%%%%%%%%%%%%%%%%%%%%%%%%%%%%%%%%%%%
% \section{Implementation}
%\iffalse
%<*package>
%\fi
%
% This section describes the definitions file |childdoc.def|.

% The definitions cannot be loaded using |\usepackage| or |\RequirePackage|
% which has a mechanism to prevent loading a style file more than once.
% When loading the definitions by means of |\input|
% multiple instances have to be prevented manually:
%\iffalse
%This code needs to be before the `\ProvidesFile' directive
%which is defined at the beginning of this file.
%Therefore it is also placed there and commented out here.
%</package>
%<*discard>
%\fi
%    \begin{macrocode}
\ifdefined\childdocmain\endinput\fi
%    \end{macrocode}
%\iffalse
%</discard>
%<*package>
%\fi
%
% \macro{\ifchilddoc}
% \macro{\ifchilddocmanual}
% The conditional |\ifchilddoc| tells whether a
% child (true) or main (false) document is being compiled.
% The conditional |\ifchilddocmanual| tells whether
% the |\includeonly| mechanism is used (false) or
% the selection of child files must be performed manually (true).
% The definitions initialise to false:
%    \begin{macrocode}
\newif\ifchilddoc
\newif\ifchilddocmanual
%    \end{macrocode}

% \macro{\childdocname}
% \macro{\childdocjob}
% The macro |\childdocname| stores the name of the main document
% to be compiled. The macro |\childdocjob| stores the name of
% the document on which the \LaTeX{} compiler was originally invoked.
% The content of |\jobname| cannot be compared
% to filenames specified in the source due to different catcodes.
% The following code rescans |\jobname|, stores the result
% in |\childdocname| and saves a copy in |\childdocjob|:
%    \begin{macrocode}
\edef\childdocname{\scantokens\expandafter{\jobname\noexpand}}
\let\childdocjob\childdocname
%    \end{macrocode}

% \macro{\childdocdisable}
% The macro |\childdocdisable| prevents the main file
% from being processed more than once.
% At this stage, the main document command |\childdocmain|
% is assumed to be called once again where it should do nothing.
% Any subsequent call to it should prevent
% a secondary processing of the main document
% It overwrites the forwarding commands
% |\childdocof| and |\childdocforward|
% with empty macros to prevent further inclusions of the main document:
%    \begin{macrocode}
\newcommand{\childdocdisable}
{
  \renewcommand{\childdocmain}[1]{\renewcommand{\childdocmain}[1]{\endinput}}
  \renewcommand{\childdocof}[1]{}
  \renewcommand{\childdocby}[2][]{}
  \renewcommand{\childdocforward}[2][]{}
  \renewcommand{\childdocdisable}{}
}
%    \end{macrocode}

% \macro{\childdocmain}
% The macro |\childdocmain| is to be called at the top of the main file
% with nothing or the main filename (without extension) as argument.
% First, it breaks loops.
% If the argument is not empty and does not match |\childdocname|
% (which is set by the first inclusion of |childdoc.def|),
% |\ifchilddoc| is set to true, |\includeonly| is applied to the child file
% and |\jobname| is set to the main file
% (for proper handling of |.aux| files):
%    \begin{macrocode}
\newcommand{\childdocmain}[1]
{
  \childdocdisable\childdocmain{}
  \if?#1?\else
    \begingroup
      \def\childdoctmp{#1}
      \ifx\childdoctmp\childdocname
        \def\childdoctmp{}
      \else
        \def\childdoctmp
        {
          \childdoctrue
          \includeonly{\childdocname}
          \def\childdocjob{#1}
          \def\jobname{#1}
        }
      \fi
      \expandafter
    \endgroup
    \childdoctmp
  \fi
}
%    \end{macrocode}

% \macro{\childdocof}
% The command |\childdocof| redirects
% compilation to the main file |#1|.
%    \begin{macrocode}
\newcommand{\childdocof}[1]
{
  \childdocdisable
  \childdoctrue
  \includeonly{\childdocname}
  \def\jobname{#1}
  \def\childdocjob{#1}
  \input{#1}
}
%    \end{macrocode}

% \macro{\childdocby}
% The command |\childdocby| ....
%    \begin{macrocode}
\newcommand{\childdocby}[2][]
{
  \childdocdisable
  \childdoctrue
  \childdocmanualtrue
  \if?#1?\else
    \def\jobname{#2}
  \fi
  \def\childdocjob{#2}
  \input{#2}
  \endinput
}
%    \end{macrocode}

% \macro{\childdocforward}
% The command |\childdocforward| redirects
% compilation to the main file or
% (if the optional argument is given) a child file.
% Parameters are set as if the main file
% or a child file starting with |\childdocof| was compiled.
% Then compilation is handed over to the main file:
%    \begin{macrocode}
\newcommand{\childdocforward}[2][]
{
  \begingroup
    \if?#1?
      \def\childdoctmp
      {
        \def\childdocname{#2}
        \def\childdocjob{#2}
        \def\jobname{#2}
        \input{#2}
        \endinput
      }
    \else
      \def\childdoctmp
      {
        \childdocdisable
        \def\childdocname{#2}
        \childdoctrue
        \includeonly{#2}
        \def\childdocjob{#1}
        \def\jobname{#1}
        \input{#1}
        \endinput
      }
    \fi
    \expandafter
  \endgroup
  \childdoctmp
}
%    \end{macrocode}

% \macro{\childdocforwardprefix}
% The command |\childdocforwardprefix| redirects
% compilation to the main or a child file by means of a pattern.
% The prefix |#1| in the current filename is replaced by |#2|
% and the suffix of the current filename is kept
% (it is assumed that the filename does not contain the substring `|~~~|'
% which is used as a delimiter).
% Compilation is handed over to the new file by |\childdocforward|:
%    \begin{macrocode}
\newcommand{\childdocforwardprefix}[3][]
{
  \begingroup
    \def\childdocextract #2##1~~~{\def\childdoctmp{\childdocforward[#1]{#3##1}}}
    \expandafter\childdocextract\childdocname~~~
    \expandafter
  \endgroup
  \childdoctmp
}
%    \end{macrocode}

% \macro{\childdoc}
% The deprecated macro |\childdoc| is a legacy version of |\childdocmain|:
%    \begin{macrocode}
\newcommand{\childdoc}{\childdocmain}
%    \end{macrocode}

% \macro{\childdocredirect}
% The deprecated macro |\childdocredirect| is a legacy version
% of |\childdocforward| and |\childdocforwardprefix|:
%    \begin{macrocode}
\newcommand{\childdocredirect}[2][]
{
  \begingroup
    \if?#1?
      \def\childdoctmp{\childdocforward{#2}}
    \else
      \def\childdoctmp{\childdocforwardprefix{#1}{#2}}
    \fi
    \expandafter
  \endgroup
  \childdoctmp
}
%    \end{macrocode}

%\iffalse
%</package>
%\fi
%
\endinput
|\\
|\childdocby{|\textit{main}|}|\\
\end{tabular}
\end{center}
%
Both forms have slightly different effects as described above.
The main file is prepared as usual, see \secref{sec:include}.

%%%%%%%%%%%%%%%%%%%%%%%%%%%%%%%%%%%%%%%%%%%%%%%%%%%%%%%%%%%%%%%%%%%%%%%%%%%%%%%%
\subsection{Legacy Detection}
\label{sec:detection}

The directive |\childdocmain| in the main file can detect
whether the complete document or merely a child is to be compiled
even without using the directive |\childdocof|.
This method is deprecated because it is less robust
and there is no compelling reason to use it;
it is merely provided for backward compatibility
and it may be removed in future versions.

If the detection mechanism is to be used,
it is mandatory to correctly specify
the filename of the main file as the argument of |\childdocmain|:
%
\begin{center}
\begin{tabular}{l}
|% \iffalse
%
% childdoc.dtx Copyright (C) 2017-2018 Niklas Beisert
%
% This work may be distributed and/or modified under the
% conditions of the LaTeX Project Public License, either version 1.3
% of this license or (at your option) any later version.
% The latest version of this license is in
%   http://www.latex-project.org/lppl.txt
% and version 1.3 or later is part of all distributions of LaTeX
% version 2005/12/01 or later.
%
% This work has the LPPL maintenance status `maintained'.
%
% The Current Maintainer of this work is Niklas Beisert.
%
% This work consists of the files childdoc.dtx and childdoc.ins
% and the derived files childdoc.def and cdocsamp.tex with
% cdocsch1.tex, cdocsch2.tex, cdocsdrf.tex, cdocsfn1.tex, cdocsfn2.tex.
%
%<package>\ifdefined\childdocmain\endinput\fi
%<package>\ProvidesFile{childdoc.def}[2018/12/30 v2.0 child document driver]
%<samplemain>\ProvidesFile{cdocsamp.tex}[2018/12/30 v2.0 sample for childdoc]
%<*driver>
%\ProvidesFile{childdoc.drv}[2018/12/30 v2.0 childdoc reference manual file]
\PassOptionsToClass{10pt,a4paper}{article}
\documentclass{ltxdoc}

\usepackage[margin=35mm]{geometry}
\usepackage{hyperref}
\usepackage{hyperxmp}
\usepackage[usenames]{color}

\hypersetup{colorlinks=true}
\hypersetup{pdfstartview=FitH}
\hypersetup{pdfpagemode=UseNone}
\hypersetup{pdfsource={}}
\hypersetup{pdflang={en-UK}}
\hypersetup{pdfcopyright={Copyright 2017-2018 Niklas Beisert.
  This work may be distributed and/or modified under the
  conditions of the LaTeX Project Public License, either version 1.3
  of this license or (at your option) any later version.}}
\hypersetup{pdflicenseurl={http://www.latex-project.org/lppl.txt}}
\hypersetup{pdfcontactaddress={ETH Zurich, ITP, HIT K,
  Wolfgang-Pauli-Strasse 27}}
\hypersetup{pdfcontactpostcode={8093}}
\hypersetup{pdfcontactcity={Zurich}}
\hypersetup{pdfcontactcountry={Switzerland}}
\hypersetup{pdfcontactemail={nbeisert@itp.phys.ethz.ch}}
\hypersetup{pdfcontacturl={http://people.phys.ethz.ch/\xmptilde nbeisert/}}

\newcommand{\secref}[1]{\hyperref[#1]{section \ref*{#1}}}

\parskip1ex
\parindent0pt
\let\olditemize\itemize
\def\itemize{\olditemize\parskip0pt}

\begin{document}

\title{The \textsf{childdoc} Package}
\hypersetup{pdftitle={The childdoc Package}}
\author{Niklas Beisert\\[2ex]
  Institut f\"ur Theoretische Physik\\
  Eidgen\"ossische Technische Hochschule Z\"urich\\
  Wolfgang-Pauli-Strasse 27, 8093 Z\"urich, Switzerland\\[1ex]
  \href{mailto:nbeisert@itp.phys.ethz.ch}
  {\texttt{nbeisert@itp.phys.ethz.ch}}}
\hypersetup{pdfauthor={Niklas Beisert}}
\hypersetup{pdfsubject={Manual for the LaTeX2e Package childdoc}}
\date{30 December 2018, \textsf{v2.0}}
\maketitle

\begin{abstract}\noindent
\textsf{childdoc} is a \LaTeXe{} package
that enables the direct compilation
of document sections included by |\include|
to individual files.
\end{abstract}

\begingroup
\parskip0ex
\tableofcontents
\endgroup

%%%%%%%%%%%%%%%%%%%%%%%%%%%%%%%%%%%%%%%%%%%%%%%%%%%%%%%%%%%%%%%%%%%%%%%%%%%%%%%%
%%%%%%%%%%%%%%%%%%%%%%%%%%%%%%%%%%%%%%%%%%%%%%%%%%%%%%%%%%%%%%%%%%%%%%%%%%%%%%%%
\section{Introduction}

\LaTeX{} provides a mechanism to structure a large document (such as a book)
into a main file and several child files (containing the chapters)
using the |\include| command.
This mechanism is beneficial for documents
which span hundreds of pages in order to
make the source file(s) more manageable.
Moreover, compilation can be restricted to
selected child files by means of the |\includeonly| command.
The latter feature can be used to reduce the compilation time while editing
(this was significantly more useful in the earlier days of \LaTeX{})
or to generate a smaller document which is easier to navigate.
Another application of |\includeonly| is to generate
documents consisting of selected parts of the complete document.

However, there are a few drawbacks of the plain |\include| mechanism:
\begin{itemize}
\item
The child files cannot be compiled on their own,
they can only be compiled via the main file.
A naive editing environment
(such as a text editor with an option
to have the current file processed by \LaTeX)
may require one to switch to the main file before compiling;
attempting to compile the child file produces errors.
\item
The main file must be modified (each time)
to adjust the |\includeonly| command
to the present needs. This easily leaves the main file in a messy state.
\item
The generated document will always carry the filename
of the main document. This is inconvenient if
several child files are to be compiled and
to be kept for distribution.
\end{itemize}

The present package provides a simple interface
to make child files individually compilable by \LaTeX{}.
Compiling a child file then has the same effect as compiling
the main file with an |\includeonly| command
to select the appropriate child.
Moreover the generated document will carry the name of the child
rather than the main file.
This resolves all three above issues.

This feature is meant to make the editing of books,
thesis documents and lecture notes somewhat more convenient.
However, the package can also be used efficiently for
composing a series of documents (such as exercise sheets)
which are typically distributed individually.
It then assists the author in generating the individual documents
(potentially in different versions)
as well as a document containing the collected series.
Another application is in developing style files
or other kinds of included material
where compilation of the style file could redirect
to a sample or test file.

%%%%%%%%%%%%%%%%%%%%%%%%%%%%%%%%%%%%%%%%%%%%%%%%%%%%%%%%%%%%%%%%%%%%%%%%%%%%%%%%
%%%%%%%%%%%%%%%%%%%%%%%%%%%%%%%%%%%%%%%%%%%%%%%%%%%%%%%%%%%%%%%%%%%%%%%%%%%%%%%%
\section{Usage}

First of all, the package \textsf{childdoc} is \emph{not} a standard
\LaTeXe{} |.sty| style file! Therefore it needs to be invoked in
a non-standard way.

%%%%%%%%%%%%%%%%%%%%%%%%%%%%%%%%%%%%%%%%%%%%%%%%%%%%%%%%%%%%%%%%%%%%%%%%%%%%%%%%
\subsection{Included Files}
\label{sec:include}

%%%%%%%%%%%%%%%%%%%%%%%%%%%%%%%%%%%%%%%%
\DescribeMacro{\childdocmain}
To use the package, add the commands
\begin{center}
\begin{tabular}{l}
|\input{childdoc.def}|\\
|\childdocmain{}|\\
\end{tabular}
\end{center}
at the very top of the main \LaTeX{} file,
in particular \emph{before} the |\documentclass| statement!
The argument of |\childdocmain| should be left empty
(but it must be present).

%%%%%%%%%%%%%%%%%%%%%%%%%%%%%%%%%%%%%%%%
\DescribeMacro{\childdocof}
Furthermore, add the commands
\begin{center}
\begin{tabular}{l}
|\input{childdoc.def}|\\
|\childdocof{|\textit{main}|}|\\
\end{tabular}
\end{center}
at the top of every child file \textit{child}
which is included by |\include{|\textit{child}|}|
from within the main file
(or at least for those files to be compiled individually).
The argument \textit{main} must be the filename of the main file.

There are a couple of
considerations in setting up the main and child documents:

%%%%%%%%%%%%%%%%%%%%%%%%%%%%%%%%%%%%%%%%
\paragraph{Restrictions.}

Please note the following restrictions:
\begin{itemize}
\item
|\childdocmain| must be called with one argument \textit{main}
to ensure compatibility with earlier version of the package.
It must either be empty (|\childdocmain{}|)
or precisely match the filename of the main file in which it is specified.
See \secref{sec:detection} for further information.
\item
The filename \textit{main} must be specified without the |.tex| extension.
\item
The filename \textit{main} is case sensitive
(even in case-insensitive file systems)
due to internal string comparison.
\item
The argument \textit{main} should be fully expanded, it cannot be a macro.
\item
Subdirectories and special characters should be avoided in filenames.
\item
The command |\childdocmain{|\textit{main}|}| must be followed by a whitespace.
It should not be followed immediately by another command
or by a comment mark `|%|'.
This is because the \TeX{} parser reads the token immediately following
the argument of |\childdocmain| and puts it
at the beginning of every child section;
however, a white\-space is ignored.
\end{itemize}

%%%%%%%%%%%%%%%%%%%%%%%%%%%%%%%%%%%%%%%%
\paragraph{Content of Main File.}

It is advisable to place all content in the child files included by |\include|.
Any output contained in the main file will appear in all child documents
unless suppressed manually;
it cannot be suppressed automatically by the |\includeonly| directive
and thus should normally be avoided.
A method to include some content in the main file
by means of conditional processing is described in \secref{sec:conditional}.

%%%%%%%%%%%%%%%%%%%%%%%%%%%%%%%%%%%%%%%%
\paragraph{Page Numbering.}

When only a part of the document is compiled,
the appropriate numbering of pages
(as well as other status parameters)
is determined from the |.aux| files.
The latter contain information from previous passes.
However this information needs to propagate through
all intermediate child documents.
Therefore the page numbering in child documents may well
be inconsistent until the complete document is compiled at least once.

A useful (if unconventional) way to always ensure a consistent
page numbering is to restart the numbering in each child document
and denote the pages by `\textit{child}|.|\textit{page}'
where \textit{child} represents the chapter/section number of the child file.
This can be achieved by the command
|\numberwithin{page}{|\textit{child}|}|
of the \textsf{amsmath} package
where \textit{child} can be |chapter| or |section|
depending on the chosen structuring.
Alternatively, one can modify the macro |\thepage| appropriately
and reset the counter |page| at the start of each child file.

%%%%%%%%%%%%%%%%%%%%%%%%%%%%%%%%%%%%%%%%%%%%%%%%%%%%%%%%%%%%%%%%%%%%%%%%%%%%%%%%
\subsection{Conditional Processing}
\label{sec:conditional}

The package provides a mechanism to compile different versions
of a document. To customise the versions further some conditional processing
can come in handy to distinguish which version is being compiled.
The package provides two macros to describe the compilation context:

%%%%%%%%%%%%%%%%%%%%%%%%%%%%%%%%%%%%%%%%
\DescribeMacro{\ifchilddoc}
The conditional |\ifchilddoc| distinguishes between the compilation of
child documents and the main document:
%
\begin{center}
|\ifchilddoc |\textit{child-code}| |[|\||else |\textit{main-code}]| \||fi|
\end{center}

%%%%%%%%%%%%%%%%%%%%%%%%%%%%%%%%%%%%%%%%
\DescribeMacro{\childdocname}
\DescribeMacro{\childdocjob}
The macro |\childdocname| contains the filename (without extension)
of the main or child file being processed.
Note that |\childdocjob| will always contain the name of the main file.

%%%%%%%%%%%%%%%%%%%%%%%%%%%%%%%%%%%%%%%%
\paragraph{Title Page.}

Conditional processing can be used to include a title or banner page
in the main document when proper precautions are taken.
Importantly, the code in the main file should ensure that the page counter
(as well as other status parameters which are stored in the |.aux| files)
takes the same value after the conditional processing.
Otherwise the page numbers may take divergent values
depending on which part is compiled.

For example, a title page could be declared by:
%
\begin{center}
\begin{tabular}{l}
|\ifchilddoc\||else|\\
|\addtocounter{page}{-1}|\\
\textit{code for title page}\\
|\newpage|\\
|\||fi|
\end{tabular}
\end{center}
%
A banner page for the child documents can be generated by:
%
\begin{center}
\begin{tabular}{l}
|\ifchilddoc|\\
|\addtocounter{page}{-1}|\\
\textit{code for banner page}\\
|\newpage|\\
|\||fi|
\end{tabular}
\end{center}
%
Here one could write a message such as:
\begin{center}
|This is the part \childdocname{} of \childdocjob{}.|
\end{center}

%%%%%%%%%%%%%%%%%%%%%%%%%%%%%%%%%%%%%%%%%%%%%%%%%%%%%%%%%%%%%%%%%%%%%%%%%%%%%%%%
\subsection{Flags}
\label{sec:flags}

The package makes it easy to generate different versions
of the main or child documents.
To this end compilation flags can be defined
and assigned different default values.
They will be particularly useful in conjunction
with the forwarding mechanism described in \secref{sec:forward}.

For example, it may be useful to have a flag |\version|
which can be set to |draft| or |final|.
The document source will contain some conditional code
depending on the value of |\version|.
Suppose further, the flag should default to |final| for the main file
and to |draft| for child files
which is a natural assignment for editing the document.
This is achieved by placing the following code
in the preamble of the main document
(below the |\childdocmain| directive):
%
\begin{center}
\begin{tabular}{l}
|\ifchilddoc|\\
|\providecommand{\version}{draft}|\\
|\||else|\\
|\providecommand{\version}{final}|\\
|\||fi|
\end{tabular}
\end{center}
%
The definition by |\providecommand| makes sure
that previous definitions are not overwritten.
Further statements |\providecommand{\version}{...}|
can thus be added before the above code to override it.

For the main file, one might add a line
(between |\childdocmain| and the above block)
%
\begin{center}
|%\ifchilddoc\||else\providecommand{\version}{draft}\||fi|
\end{center}
%
which can be uncommented to produce a draft version.
Likewise one can add a line to the very top of a child file
(above the |\childdocof{|\textit{main}|}| directive)
%
\begin{center}
|%\providecommand{\version}{final}|
\end{center}
%
which can be uncommented to produce the final version of this child document.

%%%%%%%%%%%%%%%%%%%%%%%%%%%%%%%%%%%%%%%%%%%%%%%%%%%%%%%%%%%%%%%%%%%%%%%%%%%%%%%%
\subsection{Forwarding}
\label{sec:forward}

Different versions of the main or child documents
using compilation flags as described in \secref{sec:flags}
can be (permanently) stored in different files
for convenient compilation, viewing and distribution.
To this end, the package defines a command
to pass on compilation to a different file:

%%%%%%%%%%%%%%%%%%%%%%%%%%%%%%%%%%%%%%%%
\DescribeMacro{\childdocforward}
The command |\childdocforward| redirects processing to
another source file:
%
\begin{center}
\begin{tabular}{l}
|\input{childdoc.def}|\\
|\childdocforward[|\textit{main}|]{|\textit{dest}|}|\\
\end{tabular}
\end{center}
%
The argument \textit{dest} is the destination file
(without extension).
It should be the main file or one of the child files.
Note that further \textsf{childdoc} directives
such as |\childdocof| and |\childdocforward|
in the indicated file will be processed in this form.
The optional argument \textit{main}
passes on directly to the main file \textit{main}
while pretending to compile the child \textit{dest}.
This form behaves as if \textit{dest}
issues |\childdocof{|\textit{main}|}| right away,
and no further \textsf{childdoc} directives will be processed.

%%%%%%%%%%%%%%%%%%%%%%%%%%%%%%%%%%%%%%%%
\DescribeMacro{\...prefix}
In the alternative form |\childdocforwardprefix|,
%
\begin{center}
\begin{tabular}{l}
|\input{childdoc.def}|\\
|\childdocforwardprefix[|\textit{main}|]{|\textit{prefix}|}{|\textit{dest}|}|
\end{tabular}
\end{center}
%
the destination file is determined by a pattern
depending on the current file:
To make this work, the current file must be called
`{\textit{prefix}\hspace{0.2em}\textit{suffix}}'
with \textit{prefix} matching precisely the argument.
Processing is then passed on to the file
`{\textit{dest}\hspace{0.2em}\textit{suffix}}'.
Surely, the same effect is achieved by
directly specifying the
argument `{\textit{dest}\hspace{0.2em}\textit{suffix}}'
in the first form.
However, that requires to set up a different file
for each child. With the alternative form of the command
all these files can have exactly the same content
which simplifies setting them up and maintaining them.

For example, the following file |draft.tex|
with a compilation flag |\version| as described in \secref{sec:flags}
compiles the main document as a draft:
%
\begin{center}
\begin{tabular}{l}
|\def\version{draft}|\\
|\input{childdoc.def}|\\
|\childdocforward{|\textit{main}|}|
\end{tabular}
\end{center}
%
Likewise, the following files |final|\textit{nn}|.tex|
compile the final version of the child document
|child|\textit{nn}|.tex|:
%
\begin{center}
\begin{tabular}{l}
|\def\version{final}|\\
|\input{childdoc.def}|\\
|\childdocforwardprefix{final}{child}|
\end{tabular}
\end{center}
%

Note that when several versions of a main file and/or of each child file
are to be generated, it may be convenient to set up a |Makefile| or
shell script to automatise the process.

%%%%%%%%%%%%%%%%%%%%%%%%%%%%%%%%%%%%%%%%%%%%%%%%%%%%%%%%%%%%%%%%%%%%%%%%%%%%%%%%
\subsection{Command Line Processing}
\label{sec:commandline}

The effect of redirection files can also be achieved by invoking
the \LaTeX{} compiler with a more elaborate command line.
Most conveniently this should be done as part
of a shell script or a |Makefile|.

When using \textsf{childdoc} in the main file, the following
command lines effectively perform a redirection
(note that depending on the shell being used,
backslashes may have to be doubled: `|\|' $\to$ `|\\|'):
%
\begin{center}
|... -jobname "|\textit{target}|" |\\|"|[\textit{flags}]%
|\input{childdoc.def}\childdocforward[|\textit{main}|]{|\textit{dest}|}"|
\end{center}
%
Here \textit{target} is the name of the output file,
\textit{main} is the name of the main file
and \textit{dest} is the name of the main or child file to be processed
(all filenames without extensions).
The optional argument \textit{main} can be omitted
if \textit{main} matches \textit{dest}.
Optionally, compilation \textit{flags} can be defined via |\def| commands.
This command line makes the \TeX{} engine believe
it is compiling the file \textit{target}
whose content is specified as the latter parameter.
The provided code then forwards the processing to
\textit{main} or \textit{dest} as described in \secref{sec:forward}.

%%%%%%%%%%%%%%%%%%%%%%%%%%%%%%%%%%%%%%%%%%%%%%%%%%%%%%%%%%%%%%%%%%%%%%%%%%%%%%%%
\subsection{Include by Input}
\label{sec:input}

Including child documents by |\include| has some restrictions by design.
Most notably, the content of a child document always occupies
its own set of pages; pages cannot be shared between child documents.
Usually, this behaviour makes perfect sense
because each child document contain an essential part of the document.
However, in some situations it may be desirable to compose
a document from a collection of parts
without having mandatory page breaks between then.
For this case, the package
provides a mechanism to include parts
by |\input| which can also be processed individually.
However, by construction this mechanism
requires manual handling of the content to be output.

%%%%%%%%%%%%%%%%%%%%%%%%%%%%%%%%%%%%%%%%
\DescribeMacro{\ifchilddocmanual}
The main file should be prepared as usual, see \secref{sec:include}.
However, the document body must make a distinction
between processing of an individual part and of the main document, e.g.:
%
\begin{center}
\begin{tabular}{l}
|\ifchilddocmanual|\\
|\input{\childdocname}|\\
|\||else|\\
\textit{document body with }|\input{|\textit{part}|}|\\
|\||fi|
\end{tabular}
\end{center}
%
The conditional |\ifchilddocmanual| is true whenever
a part to be included by |\input| is being compiled,
and the name of the part is stored in |\childdocname|.

%%%%%%%%%%%%%%%%%%%%%%%%%%%%%%%%%%%%%%%%
\DescribeMacro{\childdocby}
Each part to be included by |\input| should start with:
%
\begin{center}
\begin{tabular}{l}
|\input{childdoc.def}|\\
|\childdocby{|\textit{main}|}|\\
\end{tabular}
\end{center}
%
The directive |\childdocby| is similar to |\childdocof|
described in \secref{sec:include},
but the subsequent selection of content must be done manually.
To that end, both |\ifchilddoc| and |\ifchilddocmanual|
will be true upon processing of a part,
and the name of the part is stored in |\childdocname|.
Note that |\jobname| will be set to the filename of the current part
so that each part receives an individual |.aux| file
that does not interfere with the |.aux| file(s) of the main document.
This behaviour can be altered by the alternative form
|\childdocby[*]{|\textit{main}|}| (with a non-empty optional argument)
which uses the |.aux| file of the main document
by setting |\jobname| to \textit{main}.

%%%%%%%%%%%%%%%%%%%%%%%%%%%%%%%%%%%%%%%%%%%%%%%%%%%%%%%%%%%%%%%%%%%%%%%%%%%%%%%%
\subsection{Driver Development}
\label{sec:driver}

The \textsf{childdoc} mechanism can also be use for the development
of definition files such as \LaTeX{} styles or classes.
This case differs from the above setup with multiple parts
included by |\include| in that no |\includeonly| should be invoked.
This can be achieved by starting the include file
(before |\ProvidesPackage|) with:
%
\begin{center}
\begin{tabular}{l}
|\input{childdoc.def}|\\
|\childdocforward{|\textit{main}|}|\\
\end{tabular}
\end{center}
%
or alternatively with:
%
\begin{center}
\begin{tabular}{l}
|\input{childdoc.def}|\\
|\childdocby{|\textit{main}|}|\\
\end{tabular}
\end{center}
%
Both forms have slightly different effects as described above.
The main file is prepared as usual, see \secref{sec:include}.

%%%%%%%%%%%%%%%%%%%%%%%%%%%%%%%%%%%%%%%%%%%%%%%%%%%%%%%%%%%%%%%%%%%%%%%%%%%%%%%%
\subsection{Legacy Detection}
\label{sec:detection}

The directive |\childdocmain| in the main file can detect
whether the complete document or merely a child is to be compiled
even without using the directive |\childdocof|.
This method is deprecated because it is less robust
and there is no compelling reason to use it;
it is merely provided for backward compatibility
and it may be removed in future versions.

If the detection mechanism is to be used,
it is mandatory to correctly specify
the filename of the main file as the argument of |\childdocmain|:
%
\begin{center}
\begin{tabular}{l}
|\input{childdoc.def}|\\
|\childdocmain{|\textit{main}|}|\\
\end{tabular}
\end{center}
%
If |\jobname| does not match the argument \textit{main} of |\childdocmain|,
it is assumed that |\jobname| points to the child file to be compiled.
When using |\childdocmain| with the main file specified as argument,
it suffices to start a child file
with just |\input{|\textit{main}|}|
without loading of the package and using |\childdocof|.
If instead all processing is done
with the appropriate \textsf{childdoc} directives,
the argument of \textit{main} of |\childdocmain| can be empty.

An alternative version of the command line processing described
in \secref{sec:commandline} using the detection mechanism reads:
%
\begin{center}
|... -jobname "|\textit{target}|" "|[\textit{flags}]%
[|\def\jobname{|\textit{dest}|}|]|\input{|\textit{main}|}"|
\end{center}

%%%%%%%%%%%%%%%%%%%%%%%%%%%%%%%%%%%%%%%%%%%%%%%%%%%%%%%%%%%%%%%%%%%%%%%%%%%%%%%%
\subsection{Manual Code}
\label{sec:manual}

In case one cannot be certain whether the definitions file |childdoc.def|
is installed on the target \TeX{} distribution
and one prefers not to ship it,
it is conceivable to paste a few relevant commands into the sources.

To that end, drop all statements |\input{childdoc.def}|
and perform the replacements as outlined below.
Instead of |\childdocmain{|\textit{main}|}| add the following code
to the top of the main file:
%
\begin{center}
\begin{tabular}{l}
|\||ifdefined\childdocname\endinput\||fi\newif\ifchilddoc|\\
|\edef\childdocname{\scantokens\expandafter{\jobname\noexpand}}|\\
|\def\childdocmain{|\textit{main}|}\||ifx\childdocmain\childdocname\||else|\\
|\childdoctrue\includeonly{\childdocname}\let\jobname\childdocmain\||fi|\\
\end{tabular}
\end{center}
%
Instead of |\childdocof{|\textit{main}|}| just include the main file
at the top of each child file:
%
\begin{center}
|\input{|\textit{main}|}|
\end{center}
%
A simple redirection |\childdocforward{|\textit{dest}|}| is achieved by:
%
\begin{center}
|\def\jobname{|\textit{dest}|}\input{\jobname}|
\end{center}
%
The redirection with prefix
|\childdocforwardprefix[|\textit{prefix}|]{|\textit{dest}|}|
is accomplished by:
%
\begin{center}
\begin{tabular}{l}
|{\edef\jobname{\scantokens\expandafter{\jobname\noexpand}}|\\
|\def\redirectjob |\textit{prefix}|#1~~~{\gdef\jobname{|\textit{dest}|#1}}|\\
|\expandafter\redirectjob\jobname~~~}\input{\jobname}|
\end{tabular}
\end{center}

In an alternative approach,
child documents can be compiled by a specific command line
without additional code or specific definitions:
%
\begin{center}
|... -jobname "|\textit{target}|" "|[\textit{flags}]%
|\includeonly{|\textit{dest}|}\input{|\textit{main}|}"|
\end{center}
%

%%%%%%%%%%%%%%%%%%%%%%%%%%%%%%%%%%%%%%%%%%%%%%%%%%%%%%%%%%%%%%%%%%%%%%%%%%%%%%%%
%%%%%%%%%%%%%%%%%%%%%%%%%%%%%%%%%%%%%%%%%%%%%%%%%%%%%%%%%%%%%%%%%%%%%%%%%%%%%%%%
\section{Information}

%%%%%%%%%%%%%%%%%%%%%%%%%%%%%%%%%%%%%%%%%%%%%%%%%%%%%%%%%%%%%%%%%%%%%%%%%%%%%%%%
\subsection{Copyright}

Copyright \copyright{} 2017--2018 Niklas Beisert

This work may be distributed and/or modified under the
conditions of the \LaTeX{} Project Public License, either version 1.3
of this license or (at your option) any later version.
The latest version of this license is in
  \url{http://www.latex-project.org/lppl.txt}
and version 1.3 or later is part of all distributions of \LaTeX{}
version 2005/12/01 or later.

This work has the LPPL maintenance status `maintained'.

The Current Maintainer of this work is Niklas Beisert.

This work consists of the files |README.txt|, |childdoc.ins| and |childdoc.dtx|
as well as the derived files |childdoc.def|, |cdocsamp.tex|
with |cdocsch1.tex|, |cdocsch2.tex|, |cdocspt3.tex|, |cdocspt4.tex|,
|cdocsdrf.tex|, |cdocsfn1.tex|, |cdocsfn2.tex|
as well as |childdoc.pdf|.

%%%%%%%%%%%%%%%%%%%%%%%%%%%%%%%%%%%%%%%%%%%%%%%%%%%%%%%%%%%%%%%%%%%%%%%%%%%%%%%%
\subsection{Files and Installation}

The package consists of the files:
%
\begin{center}
\begin{tabular}{ll}
    |README.txt|   & readme file \\
    |childdoc.ins| & installation file \\
    |childdoc.dtx| & source file \\
    |childdoc.def| & definition file \\
    |cdocsamp.tex| & sample main file \\
    |cdocsch1.tex| & sample include file \\
    |cdocsch2.tex| & sample include file \\
    |cdocspt3.tex| & sample part file \\
    |cdocspt4.tex| & sample part file \\
    |cdocsdrf.tex| & sample redirection file \\
    |cdocsfn1.tex| & sample redirection file \\
    |cdocsfn2.tex| & sample redirection file \\
    |childdoc.pdf| & manual
\end{tabular}
\end{center}
%
The distribution consists of the files
|README.txt|, |childdoc.ins| and |childdoc.dtx|.
%
\begin{itemize}
\item
Run (pdf)\LaTeX{} on |childdoc.dtx|
to compile the manual |childdoc.pdf| (this file).
\item
Run \LaTeX{} on |childdoc.ins| to create the definitions file |childdoc.def|
and the sample |cdocsamp.tex| with include files
|cdocsch1.tex|, |cdocsch2.tex|, |cdocspt3.tex|, |cdocspt4.tex|,
|cdocsdrf.tex|, |cdocsfn1.tex|, |cdocsfn2.tex|.
Then copy the file |childdoc.def| to an appropriate directory of your \LaTeX{}
distribution, e.g.\ \textit{texmf-root}|/tex/latex/childdoc|.
\end{itemize}

%%%%%%%%%%%%%%%%%%%%%%%%%%%%%%%%%%%%%%%%%%%%%%%%%%%%%%%%%%%%%%%%%%%%%%%%%%%%%%%%
\subsection{Related CTAN Packages}

There are several other packages which offer a similar functionality:
%
\begin{itemize}
\item
The packages
\href{http://ctan.org/pkg/docmute}{\textsf{docmute}},
\href{http://ctan.org/pkg/includex}{\textsf{includex}} and
\href{http://ctan.org/pkg/standalone}{\textsf{standalone}}
provide commands to include only the document body of
a child file thus allowing both files to be compiled individually.
\item
The packages \href{http://ctan.org/pkg/subdocs}{\textsf{subdocs}}
and \href{http://ctan.org/pkg/subfiles}{\textsf{subfiles}}
provide structures in which the main and child documents can be
encapsulated and allowing them to be compiled individually.
The inclusion mechanism is different from the conventional |\include|.
\item
The package \href{http://ctan.org/pkg/combine}{\textsf{combine}}
is an elaborate solution to combine several documents into one.
\end{itemize}
%
See also the CTAN topic \href{http://ctan.org/topic/subdocs}{\textsf{subdocs}}
for further related packages.
The present package differs from the above solutions in that
a document structure constructed with the conventional |\include| mechanism
just needs two extra commands at the top of every file
such that all constituent files can be compiled individually.

%%%%%%%%%%%%%%%%%%%%%%%%%%%%%%%%%%%%%%%%%%%%%%%%%%%%%%%%%%%%%%%%%%%%%%%%%%%%%%%%
%\subsection{Feature Suggestions}
%
%The following is a list of features which may be useful for future
%versions of this package:
%%
%\begin{itemize}
%\item
%\ldots
%\end{itemize}

%%%%%%%%%%%%%%%%%%%%%%%%%%%%%%%%%%%%%%%%%%%%%%%%%%%%%%%%%%%%%%%%%%%%%%%%%%%%%%%%
\subsection{Revision History}

%%%%%%%%%%%%%%%%%%%%%%%%%%%%%%%%%%%%%%%%
\paragraph{v2.0:} 2018/12/30

\begin{itemize}
\item
immediate forward processing
\item
added |\childdocby| mechanism
\item
manual restructured
\end{itemize}

%%%%%%%%%%%%%%%%%%%%%%%%%%%%%%%%%%%%%%%%
\paragraph{v1.6:} 2018/01/17

\begin{itemize}
\item
application for development of include files
\item
corrections to manual
\end{itemize}

%%%%%%%%%%%%%%%%%%%%%%%%%%%%%%%%%%%%%%%%
\paragraph{v1.5:} 2017/05/21

\begin{itemize}
\item
more complete structuring introduced
\item
|\childdocof| introduced
\item
|\childdoc| renamed to |\childdocmain|
\item
|\childredirect| renamed to |\childdocforward| and |\childdocforwardprefix|
and functionality expanded
\end{itemize}

%%%%%%%%%%%%%%%%%%%%%%%%%%%%%%%%%%%%%%%%
\paragraph{v1.0:} 2017/04/27

\begin{itemize}
\item
manual and install package
\item
first version published on CTAN
\end{itemize}

%%%%%%%%%%%%%%%%%%%%%%%%%%%%%%%%%%%%%%%%
\paragraph{v0.6:} 2017/04/26

\begin{itemize}
\item
redirection mechanism added
\end{itemize}

%%%%%%%%%%%%%%%%%%%%%%%%%%%%%%%%%%%%%%%%
\paragraph{v0.5:} 2017/04/26

\begin{itemize}
\item
functionality in definition file
\end{itemize}


%%%%%%%%%%%%%%%%%%%%%%%%%%%%%%%%%%%%%%%%%%%%%%%%%%%%%%%%%%%%%%%%%%%%%%%%%%%%%%%%
%%%%%%%%%%%%%%%%%%%%%%%%%%%%%%%%%%%%%%%%%%%%%%%%%%%%%%%%%%%%%%%%%%%%%%%%%%%%%%%%
%%%%%%%%%%%%%%%%%%%%%%%%%%%%%%%%%%%%%%%%%%%%%%%%%%%%%%%%%%%%%%%%%%%%%%%%%%%%%%%%
\appendix

\settowidth\MacroIndent{\rmfamily\scriptsize 000\ }

 \DocInput{childdoc.dtx}

\end{document}
%</driver>
% \fi
%
% %%%%%%%%%%%%%%%%%%%%%%%%%%%%%%%%%%%%%%%%%%%%%%%%%%%%%%%%%%%%%%%%%%%%%%%%%%%%%%
% %%%%%%%%%%%%%%%%%%%%%%%%%%%%%%%%%%%%%%%%%%%%%%%%%%%%%%%%%%%%%%%%%%%%%%%%%%%%%%
% \section{Sample}
%\iffalse
%<*samplemain>
%\fi
%
% The following presents a sample document
% with two chapters, two parts, a title page,
% a compile flag as well as three forwarding files to set the flag.
% It consists of eight |.tex| files:
% \begin{center}
% \begin{tabular}{ll}
% |cdocsamp.tex|&main file\\
% |cdocsch1.tex|&include file for chapter 1\\
% |cdocsch2.tex|&include file for chapter 2\\
% |cdocspt3.tex|&include file for part 3\\
% |cdocspt4.tex|&include file for part 4\\
% |cdocsdrf.tex|&forwarding file for main file in draft mode\\
% |cdocsfi1.tex|&forwarding file for final version of chapter 1\\
% |cdocsfi2.tex|&forwarding file for final version of chapter 2\\
% \end{tabular}
% \end{center}
% Each of the eight files can be compiled directly by the \LaTeX{} compiler.
%
% %%%%%%%%%%%%%%%%%%%%%%%%%%%%%%%%%%%%%%
% \paragraph{Main File.}
%
% The main file is called |cdocsamp.tex|.
%
% Load the \textsf{childdoc} definitions and
% declare the filename for the main document:
%    \begin{macrocode}
\input{childdoc.def}
\childdocmain{}
%    \end{macrocode}

% Optional override for |\version| flag:
%    \begin{macrocode}
%%\ifchilddoc\else\providecommand{\version}{draft}\fi
%    \end{macrocode}

% Define the default values for the |\version| flag
% (|final| for the main file and |draft| for childs):
%    \begin{macrocode}
\ifchilddoc
\providecommand{\version}{draft}
\else
\providecommand{\version}{final}
\fi
%    \end{macrocode}

% Load the standard document class:
%    \begin{macrocode}
\documentclass[12pt]{article}
%    \end{macrocode}

% Start the document body:
%    \begin{macrocode}
\begin{document}
%    \end{macrocode}

% Declare a title page.
% Print title, part of document being processed and version flag:
%    \begin{macrocode}
\addtocounter{page}{-1}
\begin{center}
{\LARGE\bfseries{}childdoc example\par}
\vspace{1cm}
\ifchilddoc
\ifchilddocmanual part\else chapter\fi:
`\childdocname' of `\childdocjob'\par
\else
main document: `\childdocjob'\par
\fi
version: \version\par
\end{center}
\newpage
%    \end{macrocode}

% Manually include selected file,
% otherwise process as usual:
%    \begin{macrocode}
\ifchilddocmanual
\section*{part `\childdocname'}
\input{\childdocname}
\else
%    \end{macrocode}

% Include the two chapters:
%    \begin{macrocode}
\include{cdocsch1}
\include{cdocsch2}
%    \end{macrocode}

% Include the two parts unless only chapters should be displayed:
%    \begin{macrocode}
\ifchilddoc\else
\section{part three}
\input{cdocspt3}
\section{part four}
\input{cdocspt4}
\fi
%    \end{macrocode}

% Process as usual until here:
%    \begin{macrocode}
\fi
%    \end{macrocode}

% End of document body:
%    \begin{macrocode}
\end{document}
%    \end{macrocode}
%\iffalse
%</samplemain>
%\fi
%
% %%%%%%%%%%%%%%%%%%%%%%%%%%%%%%%%%%%%%%
% \paragraph{Chapter Include Files.}
%
% The include files are called |cdocsch1.tex| and |cdocsch2.tex|.
%
%\iffalse
%<*samplechap1|samplechap2>
%\fi

% Optional override for |\version| flag:
%    \begin{macrocode}
%%\providecommand{\version}{final}
%    \end{macrocode}

% Include the main document:
%    \begin{macrocode}
\input{childdoc.def}
\childdocof{cdocsamp}
%    \end{macrocode}

%\iffalse
%</samplechap1|samplechap2>
%\fi
%
%\iffalse
%<*samplechap1>
%\fi
% Some text for chapter 1:
%    \begin{macrocode}
\section{one}
some text in chapter one
%    \end{macrocode}

%\iffalse
%</samplechap1>
%\fi
% Some text for chapter 2:
%\iffalse
%<*samplechap2>
%\fi
%    \begin{macrocode}
\section{two}
more text in chapter two
%    \end{macrocode}

%\iffalse
%</samplechap2>
%\fi
%
% %%%%%%%%%%%%%%%%%%%%%%%%%%%%%%%%%%%%%%
% \paragraph{Part Include Files.}
%
% The include files are called |cdocspt3.tex| and |cdocspt4.tex|.
%
%\iffalse
%<*samplepart3|samplepart4>
%\fi

% Optional override for |\version| flag:
%    \begin{macrocode}
%%\providecommand{\version}{final}
%    \end{macrocode}

% Include the main document:
%    \begin{macrocode}
\input{childdoc.def}
\childdocby{cdocsamp}
%    \end{macrocode}

%\iffalse
%</samplepart3|samplepart4>
%\fi
%
%\iffalse
%<*samplepart3>
%\fi
% Some text for part 3:
%    \begin{macrocode}
some text in part three
%    \end{macrocode}

%\iffalse
%</samplepart3>
%\fi
% Some text for part 4:
%\iffalse
%<*samplepart4>
%\fi
%    \begin{macrocode}
more text in part four
%    \end{macrocode}

%\iffalse
%</samplepart4>
%\fi
%
% %%%%%%%%%%%%%%%%%%%%%%%%%%%%%%%%%%%%%%
% \paragraph{Forwarding for a Complete Draft.}
%
% The following forwarding file |cdocsdrf.tex|
% compiles the main document in draft mode:
%\iffalse
%<*sampledraft>
%\fi
%    \begin{macrocode}
\def\version{draft}
\input{childdoc.def}
\childdocforward{cdocsamp}
%    \end{macrocode}

%\iffalse
%</sampledraft>
%\fi
%
% %%%%%%%%%%%%%%%%%%%%%%%%%%%%%%%%%%%%%%
% \paragraph{Forwarding for Final Version of the Chapters.}
%
% The following forwarding files |cdocsfn1.tex| and |cdocsfn2.tex|
% (with identical content)
% compile the final versions of the child documents
% |cdocsch1.tex| and |cdocsch2.tex|, respectively:
%\iffalse
%<*samplefinal>
%\fi
%    \begin{macrocode}
\def\version{final}
\input{childdoc.def}
\childdocforwardprefix[cdocsamp]{cdocsfn}{cdocsch}
%    \end{macrocode}

%\iffalse
%</samplefinal>
%\fi
%
% %%%%%%%%%%%%%%%%%%%%%%%%%%%%%%%%%%%%%%
% \paragraph{Command Line Processing.}
%
% The following three command lines generate the output files
% |cdocscld|, |cdocscl1| and |cdocscl2|
% which should be identical to
% |cdocsdrf|, |cdocsch1| and |cdocsfn2|, respectively:
% \begin{center}
% \begin{tabular}{l}
% |latex -jobname cdocscld \|\\
% |  "\def\version{draft}\input{childdoc.def}\childdocforward{cdocsamp}"|\\
% |latex -jobname cdocscl1 \|\\
% |  "\input{childdoc.def}\childdocforward[cdocsamp]{cdocsch1}"|\\
% |latex -jobname cdocscl2 \|\\
% |  "\def\version{final}\input{childdoc.def}\childdocforward{cdocsch2}"|
% \end{tabular}
% \end{center}
% Note that the trailing backslash on each first line
% merely continues the input to the second line
% (for convenient cut ant paste).
% Furthermore, the command |latex| can be replaced by any
% of its alternative versions such as |pdflatex|.
%
% %%%%%%%%%%%%%%%%%%%%%%%%%%%%%%%%%%%%%%%%%%%%%%%%%%%%%%%%%%%%%%%%%%%%%%%%%%%%%%
% %%%%%%%%%%%%%%%%%%%%%%%%%%%%%%%%%%%%%%%%%%%%%%%%%%%%%%%%%%%%%%%%%%%%%%%%%%%%%%
% \section{Implementation}
%\iffalse
%<*package>
%\fi
%
% This section describes the definitions file |childdoc.def|.

% The definitions cannot be loaded using |\usepackage| or |\RequirePackage|
% which has a mechanism to prevent loading a style file more than once.
% When loading the definitions by means of |\input|
% multiple instances have to be prevented manually:
%\iffalse
%This code needs to be before the `\ProvidesFile' directive
%which is defined at the beginning of this file.
%Therefore it is also placed there and commented out here.
%</package>
%<*discard>
%\fi
%    \begin{macrocode}
\ifdefined\childdocmain\endinput\fi
%    \end{macrocode}
%\iffalse
%</discard>
%<*package>
%\fi
%
% \macro{\ifchilddoc}
% \macro{\ifchilddocmanual}
% The conditional |\ifchilddoc| tells whether a
% child (true) or main (false) document is being compiled.
% The conditional |\ifchilddocmanual| tells whether
% the |\includeonly| mechanism is used (false) or
% the selection of child files must be performed manually (true).
% The definitions initialise to false:
%    \begin{macrocode}
\newif\ifchilddoc
\newif\ifchilddocmanual
%    \end{macrocode}

% \macro{\childdocname}
% \macro{\childdocjob}
% The macro |\childdocname| stores the name of the main document
% to be compiled. The macro |\childdocjob| stores the name of
% the document on which the \LaTeX{} compiler was originally invoked.
% The content of |\jobname| cannot be compared
% to filenames specified in the source due to different catcodes.
% The following code rescans |\jobname|, stores the result
% in |\childdocname| and saves a copy in |\childdocjob|:
%    \begin{macrocode}
\edef\childdocname{\scantokens\expandafter{\jobname\noexpand}}
\let\childdocjob\childdocname
%    \end{macrocode}

% \macro{\childdocdisable}
% The macro |\childdocdisable| prevents the main file
% from being processed more than once.
% At this stage, the main document command |\childdocmain|
% is assumed to be called once again where it should do nothing.
% Any subsequent call to it should prevent
% a secondary processing of the main document
% It overwrites the forwarding commands
% |\childdocof| and |\childdocforward|
% with empty macros to prevent further inclusions of the main document:
%    \begin{macrocode}
\newcommand{\childdocdisable}
{
  \renewcommand{\childdocmain}[1]{\renewcommand{\childdocmain}[1]{\endinput}}
  \renewcommand{\childdocof}[1]{}
  \renewcommand{\childdocby}[2][]{}
  \renewcommand{\childdocforward}[2][]{}
  \renewcommand{\childdocdisable}{}
}
%    \end{macrocode}

% \macro{\childdocmain}
% The macro |\childdocmain| is to be called at the top of the main file
% with nothing or the main filename (without extension) as argument.
% First, it breaks loops.
% If the argument is not empty and does not match |\childdocname|
% (which is set by the first inclusion of |childdoc.def|),
% |\ifchilddoc| is set to true, |\includeonly| is applied to the child file
% and |\jobname| is set to the main file
% (for proper handling of |.aux| files):
%    \begin{macrocode}
\newcommand{\childdocmain}[1]
{
  \childdocdisable\childdocmain{}
  \if?#1?\else
    \begingroup
      \def\childdoctmp{#1}
      \ifx\childdoctmp\childdocname
        \def\childdoctmp{}
      \else
        \def\childdoctmp
        {
          \childdoctrue
          \includeonly{\childdocname}
          \def\childdocjob{#1}
          \def\jobname{#1}
        }
      \fi
      \expandafter
    \endgroup
    \childdoctmp
  \fi
}
%    \end{macrocode}

% \macro{\childdocof}
% The command |\childdocof| redirects
% compilation to the main file |#1|.
%    \begin{macrocode}
\newcommand{\childdocof}[1]
{
  \childdocdisable
  \childdoctrue
  \includeonly{\childdocname}
  \def\jobname{#1}
  \def\childdocjob{#1}
  \input{#1}
}
%    \end{macrocode}

% \macro{\childdocby}
% The command |\childdocby| ....
%    \begin{macrocode}
\newcommand{\childdocby}[2][]
{
  \childdocdisable
  \childdoctrue
  \childdocmanualtrue
  \if?#1?\else
    \def\jobname{#2}
  \fi
  \def\childdocjob{#2}
  \input{#2}
  \endinput
}
%    \end{macrocode}

% \macro{\childdocforward}
% The command |\childdocforward| redirects
% compilation to the main file or
% (if the optional argument is given) a child file.
% Parameters are set as if the main file
% or a child file starting with |\childdocof| was compiled.
% Then compilation is handed over to the main file:
%    \begin{macrocode}
\newcommand{\childdocforward}[2][]
{
  \begingroup
    \if?#1?
      \def\childdoctmp
      {
        \def\childdocname{#2}
        \def\childdocjob{#2}
        \def\jobname{#2}
        \input{#2}
        \endinput
      }
    \else
      \def\childdoctmp
      {
        \childdocdisable
        \def\childdocname{#2}
        \childdoctrue
        \includeonly{#2}
        \def\childdocjob{#1}
        \def\jobname{#1}
        \input{#1}
        \endinput
      }
    \fi
    \expandafter
  \endgroup
  \childdoctmp
}
%    \end{macrocode}

% \macro{\childdocforwardprefix}
% The command |\childdocforwardprefix| redirects
% compilation to the main or a child file by means of a pattern.
% The prefix |#1| in the current filename is replaced by |#2|
% and the suffix of the current filename is kept
% (it is assumed that the filename does not contain the substring `|~~~|'
% which is used as a delimiter).
% Compilation is handed over to the new file by |\childdocforward|:
%    \begin{macrocode}
\newcommand{\childdocforwardprefix}[3][]
{
  \begingroup
    \def\childdocextract #2##1~~~{\def\childdoctmp{\childdocforward[#1]{#3##1}}}
    \expandafter\childdocextract\childdocname~~~
    \expandafter
  \endgroup
  \childdoctmp
}
%    \end{macrocode}

% \macro{\childdoc}
% The deprecated macro |\childdoc| is a legacy version of |\childdocmain|:
%    \begin{macrocode}
\newcommand{\childdoc}{\childdocmain}
%    \end{macrocode}

% \macro{\childdocredirect}
% The deprecated macro |\childdocredirect| is a legacy version
% of |\childdocforward| and |\childdocforwardprefix|:
%    \begin{macrocode}
\newcommand{\childdocredirect}[2][]
{
  \begingroup
    \if?#1?
      \def\childdoctmp{\childdocforward{#2}}
    \else
      \def\childdoctmp{\childdocforwardprefix{#1}{#2}}
    \fi
    \expandafter
  \endgroup
  \childdoctmp
}
%    \end{macrocode}

%\iffalse
%</package>
%\fi
%
\endinput
|\\
|\childdocmain{|\textit{main}|}|\\
\end{tabular}
\end{center}
%
If |\jobname| does not match the argument \textit{main} of |\childdocmain|,
it is assumed that |\jobname| points to the child file to be compiled.
When using |\childdocmain| with the main file specified as argument,
it suffices to start a child file
with just |\input{|\textit{main}|}|
without loading of the package and using |\childdocof|.
If instead all processing is done
with the appropriate \textsf{childdoc} directives,
the argument of \textit{main} of |\childdocmain| can be empty.

An alternative version of the command line processing described
in \secref{sec:commandline} using the detection mechanism reads:
%
\begin{center}
|... -jobname "|\textit{target}|" "|[\textit{flags}]%
[|\def\jobname{|\textit{dest}|}|]|\input{|\textit{main}|}"|
\end{center}

%%%%%%%%%%%%%%%%%%%%%%%%%%%%%%%%%%%%%%%%%%%%%%%%%%%%%%%%%%%%%%%%%%%%%%%%%%%%%%%%
\subsection{Manual Code}
\label{sec:manual}

In case one cannot be certain whether the definitions file |childdoc.def|
is installed on the target \TeX{} distribution
and one prefers not to ship it,
it is conceivable to paste a few relevant commands into the sources.

To that end, drop all statements |% \iffalse
%
% childdoc.dtx Copyright (C) 2017-2018 Niklas Beisert
%
% This work may be distributed and/or modified under the
% conditions of the LaTeX Project Public License, either version 1.3
% of this license or (at your option) any later version.
% The latest version of this license is in
%   http://www.latex-project.org/lppl.txt
% and version 1.3 or later is part of all distributions of LaTeX
% version 2005/12/01 or later.
%
% This work has the LPPL maintenance status `maintained'.
%
% The Current Maintainer of this work is Niklas Beisert.
%
% This work consists of the files childdoc.dtx and childdoc.ins
% and the derived files childdoc.def and cdocsamp.tex with
% cdocsch1.tex, cdocsch2.tex, cdocsdrf.tex, cdocsfn1.tex, cdocsfn2.tex.
%
%<package>\ifdefined\childdocmain\endinput\fi
%<package>\ProvidesFile{childdoc.def}[2018/12/30 v2.0 child document driver]
%<samplemain>\ProvidesFile{cdocsamp.tex}[2018/12/30 v2.0 sample for childdoc]
%<*driver>
%\ProvidesFile{childdoc.drv}[2018/12/30 v2.0 childdoc reference manual file]
\PassOptionsToClass{10pt,a4paper}{article}
\documentclass{ltxdoc}

\usepackage[margin=35mm]{geometry}
\usepackage{hyperref}
\usepackage{hyperxmp}
\usepackage[usenames]{color}

\hypersetup{colorlinks=true}
\hypersetup{pdfstartview=FitH}
\hypersetup{pdfpagemode=UseNone}
\hypersetup{pdfsource={}}
\hypersetup{pdflang={en-UK}}
\hypersetup{pdfcopyright={Copyright 2017-2018 Niklas Beisert.
  This work may be distributed and/or modified under the
  conditions of the LaTeX Project Public License, either version 1.3
  of this license or (at your option) any later version.}}
\hypersetup{pdflicenseurl={http://www.latex-project.org/lppl.txt}}
\hypersetup{pdfcontactaddress={ETH Zurich, ITP, HIT K,
  Wolfgang-Pauli-Strasse 27}}
\hypersetup{pdfcontactpostcode={8093}}
\hypersetup{pdfcontactcity={Zurich}}
\hypersetup{pdfcontactcountry={Switzerland}}
\hypersetup{pdfcontactemail={nbeisert@itp.phys.ethz.ch}}
\hypersetup{pdfcontacturl={http://people.phys.ethz.ch/\xmptilde nbeisert/}}

\newcommand{\secref}[1]{\hyperref[#1]{section \ref*{#1}}}

\parskip1ex
\parindent0pt
\let\olditemize\itemize
\def\itemize{\olditemize\parskip0pt}

\begin{document}

\title{The \textsf{childdoc} Package}
\hypersetup{pdftitle={The childdoc Package}}
\author{Niklas Beisert\\[2ex]
  Institut f\"ur Theoretische Physik\\
  Eidgen\"ossische Technische Hochschule Z\"urich\\
  Wolfgang-Pauli-Strasse 27, 8093 Z\"urich, Switzerland\\[1ex]
  \href{mailto:nbeisert@itp.phys.ethz.ch}
  {\texttt{nbeisert@itp.phys.ethz.ch}}}
\hypersetup{pdfauthor={Niklas Beisert}}
\hypersetup{pdfsubject={Manual for the LaTeX2e Package childdoc}}
\date{30 December 2018, \textsf{v2.0}}
\maketitle

\begin{abstract}\noindent
\textsf{childdoc} is a \LaTeXe{} package
that enables the direct compilation
of document sections included by |\include|
to individual files.
\end{abstract}

\begingroup
\parskip0ex
\tableofcontents
\endgroup

%%%%%%%%%%%%%%%%%%%%%%%%%%%%%%%%%%%%%%%%%%%%%%%%%%%%%%%%%%%%%%%%%%%%%%%%%%%%%%%%
%%%%%%%%%%%%%%%%%%%%%%%%%%%%%%%%%%%%%%%%%%%%%%%%%%%%%%%%%%%%%%%%%%%%%%%%%%%%%%%%
\section{Introduction}

\LaTeX{} provides a mechanism to structure a large document (such as a book)
into a main file and several child files (containing the chapters)
using the |\include| command.
This mechanism is beneficial for documents
which span hundreds of pages in order to
make the source file(s) more manageable.
Moreover, compilation can be restricted to
selected child files by means of the |\includeonly| command.
The latter feature can be used to reduce the compilation time while editing
(this was significantly more useful in the earlier days of \LaTeX{})
or to generate a smaller document which is easier to navigate.
Another application of |\includeonly| is to generate
documents consisting of selected parts of the complete document.

However, there are a few drawbacks of the plain |\include| mechanism:
\begin{itemize}
\item
The child files cannot be compiled on their own,
they can only be compiled via the main file.
A naive editing environment
(such as a text editor with an option
to have the current file processed by \LaTeX)
may require one to switch to the main file before compiling;
attempting to compile the child file produces errors.
\item
The main file must be modified (each time)
to adjust the |\includeonly| command
to the present needs. This easily leaves the main file in a messy state.
\item
The generated document will always carry the filename
of the main document. This is inconvenient if
several child files are to be compiled and
to be kept for distribution.
\end{itemize}

The present package provides a simple interface
to make child files individually compilable by \LaTeX{}.
Compiling a child file then has the same effect as compiling
the main file with an |\includeonly| command
to select the appropriate child.
Moreover the generated document will carry the name of the child
rather than the main file.
This resolves all three above issues.

This feature is meant to make the editing of books,
thesis documents and lecture notes somewhat more convenient.
However, the package can also be used efficiently for
composing a series of documents (such as exercise sheets)
which are typically distributed individually.
It then assists the author in generating the individual documents
(potentially in different versions)
as well as a document containing the collected series.
Another application is in developing style files
or other kinds of included material
where compilation of the style file could redirect
to a sample or test file.

%%%%%%%%%%%%%%%%%%%%%%%%%%%%%%%%%%%%%%%%%%%%%%%%%%%%%%%%%%%%%%%%%%%%%%%%%%%%%%%%
%%%%%%%%%%%%%%%%%%%%%%%%%%%%%%%%%%%%%%%%%%%%%%%%%%%%%%%%%%%%%%%%%%%%%%%%%%%%%%%%
\section{Usage}

First of all, the package \textsf{childdoc} is \emph{not} a standard
\LaTeXe{} |.sty| style file! Therefore it needs to be invoked in
a non-standard way.

%%%%%%%%%%%%%%%%%%%%%%%%%%%%%%%%%%%%%%%%%%%%%%%%%%%%%%%%%%%%%%%%%%%%%%%%%%%%%%%%
\subsection{Included Files}
\label{sec:include}

%%%%%%%%%%%%%%%%%%%%%%%%%%%%%%%%%%%%%%%%
\DescribeMacro{\childdocmain}
To use the package, add the commands
\begin{center}
\begin{tabular}{l}
|\input{childdoc.def}|\\
|\childdocmain{}|\\
\end{tabular}
\end{center}
at the very top of the main \LaTeX{} file,
in particular \emph{before} the |\documentclass| statement!
The argument of |\childdocmain| should be left empty
(but it must be present).

%%%%%%%%%%%%%%%%%%%%%%%%%%%%%%%%%%%%%%%%
\DescribeMacro{\childdocof}
Furthermore, add the commands
\begin{center}
\begin{tabular}{l}
|\input{childdoc.def}|\\
|\childdocof{|\textit{main}|}|\\
\end{tabular}
\end{center}
at the top of every child file \textit{child}
which is included by |\include{|\textit{child}|}|
from within the main file
(or at least for those files to be compiled individually).
The argument \textit{main} must be the filename of the main file.

There are a couple of
considerations in setting up the main and child documents:

%%%%%%%%%%%%%%%%%%%%%%%%%%%%%%%%%%%%%%%%
\paragraph{Restrictions.}

Please note the following restrictions:
\begin{itemize}
\item
|\childdocmain| must be called with one argument \textit{main}
to ensure compatibility with earlier version of the package.
It must either be empty (|\childdocmain{}|)
or precisely match the filename of the main file in which it is specified.
See \secref{sec:detection} for further information.
\item
The filename \textit{main} must be specified without the |.tex| extension.
\item
The filename \textit{main} is case sensitive
(even in case-insensitive file systems)
due to internal string comparison.
\item
The argument \textit{main} should be fully expanded, it cannot be a macro.
\item
Subdirectories and special characters should be avoided in filenames.
\item
The command |\childdocmain{|\textit{main}|}| must be followed by a whitespace.
It should not be followed immediately by another command
or by a comment mark `|%|'.
This is because the \TeX{} parser reads the token immediately following
the argument of |\childdocmain| and puts it
at the beginning of every child section;
however, a white\-space is ignored.
\end{itemize}

%%%%%%%%%%%%%%%%%%%%%%%%%%%%%%%%%%%%%%%%
\paragraph{Content of Main File.}

It is advisable to place all content in the child files included by |\include|.
Any output contained in the main file will appear in all child documents
unless suppressed manually;
it cannot be suppressed automatically by the |\includeonly| directive
and thus should normally be avoided.
A method to include some content in the main file
by means of conditional processing is described in \secref{sec:conditional}.

%%%%%%%%%%%%%%%%%%%%%%%%%%%%%%%%%%%%%%%%
\paragraph{Page Numbering.}

When only a part of the document is compiled,
the appropriate numbering of pages
(as well as other status parameters)
is determined from the |.aux| files.
The latter contain information from previous passes.
However this information needs to propagate through
all intermediate child documents.
Therefore the page numbering in child documents may well
be inconsistent until the complete document is compiled at least once.

A useful (if unconventional) way to always ensure a consistent
page numbering is to restart the numbering in each child document
and denote the pages by `\textit{child}|.|\textit{page}'
where \textit{child} represents the chapter/section number of the child file.
This can be achieved by the command
|\numberwithin{page}{|\textit{child}|}|
of the \textsf{amsmath} package
where \textit{child} can be |chapter| or |section|
depending on the chosen structuring.
Alternatively, one can modify the macro |\thepage| appropriately
and reset the counter |page| at the start of each child file.

%%%%%%%%%%%%%%%%%%%%%%%%%%%%%%%%%%%%%%%%%%%%%%%%%%%%%%%%%%%%%%%%%%%%%%%%%%%%%%%%
\subsection{Conditional Processing}
\label{sec:conditional}

The package provides a mechanism to compile different versions
of a document. To customise the versions further some conditional processing
can come in handy to distinguish which version is being compiled.
The package provides two macros to describe the compilation context:

%%%%%%%%%%%%%%%%%%%%%%%%%%%%%%%%%%%%%%%%
\DescribeMacro{\ifchilddoc}
The conditional |\ifchilddoc| distinguishes between the compilation of
child documents and the main document:
%
\begin{center}
|\ifchilddoc |\textit{child-code}| |[|\||else |\textit{main-code}]| \||fi|
\end{center}

%%%%%%%%%%%%%%%%%%%%%%%%%%%%%%%%%%%%%%%%
\DescribeMacro{\childdocname}
\DescribeMacro{\childdocjob}
The macro |\childdocname| contains the filename (without extension)
of the main or child file being processed.
Note that |\childdocjob| will always contain the name of the main file.

%%%%%%%%%%%%%%%%%%%%%%%%%%%%%%%%%%%%%%%%
\paragraph{Title Page.}

Conditional processing can be used to include a title or banner page
in the main document when proper precautions are taken.
Importantly, the code in the main file should ensure that the page counter
(as well as other status parameters which are stored in the |.aux| files)
takes the same value after the conditional processing.
Otherwise the page numbers may take divergent values
depending on which part is compiled.

For example, a title page could be declared by:
%
\begin{center}
\begin{tabular}{l}
|\ifchilddoc\||else|\\
|\addtocounter{page}{-1}|\\
\textit{code for title page}\\
|\newpage|\\
|\||fi|
\end{tabular}
\end{center}
%
A banner page for the child documents can be generated by:
%
\begin{center}
\begin{tabular}{l}
|\ifchilddoc|\\
|\addtocounter{page}{-1}|\\
\textit{code for banner page}\\
|\newpage|\\
|\||fi|
\end{tabular}
\end{center}
%
Here one could write a message such as:
\begin{center}
|This is the part \childdocname{} of \childdocjob{}.|
\end{center}

%%%%%%%%%%%%%%%%%%%%%%%%%%%%%%%%%%%%%%%%%%%%%%%%%%%%%%%%%%%%%%%%%%%%%%%%%%%%%%%%
\subsection{Flags}
\label{sec:flags}

The package makes it easy to generate different versions
of the main or child documents.
To this end compilation flags can be defined
and assigned different default values.
They will be particularly useful in conjunction
with the forwarding mechanism described in \secref{sec:forward}.

For example, it may be useful to have a flag |\version|
which can be set to |draft| or |final|.
The document source will contain some conditional code
depending on the value of |\version|.
Suppose further, the flag should default to |final| for the main file
and to |draft| for child files
which is a natural assignment for editing the document.
This is achieved by placing the following code
in the preamble of the main document
(below the |\childdocmain| directive):
%
\begin{center}
\begin{tabular}{l}
|\ifchilddoc|\\
|\providecommand{\version}{draft}|\\
|\||else|\\
|\providecommand{\version}{final}|\\
|\||fi|
\end{tabular}
\end{center}
%
The definition by |\providecommand| makes sure
that previous definitions are not overwritten.
Further statements |\providecommand{\version}{...}|
can thus be added before the above code to override it.

For the main file, one might add a line
(between |\childdocmain| and the above block)
%
\begin{center}
|%\ifchilddoc\||else\providecommand{\version}{draft}\||fi|
\end{center}
%
which can be uncommented to produce a draft version.
Likewise one can add a line to the very top of a child file
(above the |\childdocof{|\textit{main}|}| directive)
%
\begin{center}
|%\providecommand{\version}{final}|
\end{center}
%
which can be uncommented to produce the final version of this child document.

%%%%%%%%%%%%%%%%%%%%%%%%%%%%%%%%%%%%%%%%%%%%%%%%%%%%%%%%%%%%%%%%%%%%%%%%%%%%%%%%
\subsection{Forwarding}
\label{sec:forward}

Different versions of the main or child documents
using compilation flags as described in \secref{sec:flags}
can be (permanently) stored in different files
for convenient compilation, viewing and distribution.
To this end, the package defines a command
to pass on compilation to a different file:

%%%%%%%%%%%%%%%%%%%%%%%%%%%%%%%%%%%%%%%%
\DescribeMacro{\childdocforward}
The command |\childdocforward| redirects processing to
another source file:
%
\begin{center}
\begin{tabular}{l}
|\input{childdoc.def}|\\
|\childdocforward[|\textit{main}|]{|\textit{dest}|}|\\
\end{tabular}
\end{center}
%
The argument \textit{dest} is the destination file
(without extension).
It should be the main file or one of the child files.
Note that further \textsf{childdoc} directives
such as |\childdocof| and |\childdocforward|
in the indicated file will be processed in this form.
The optional argument \textit{main}
passes on directly to the main file \textit{main}
while pretending to compile the child \textit{dest}.
This form behaves as if \textit{dest}
issues |\childdocof{|\textit{main}|}| right away,
and no further \textsf{childdoc} directives will be processed.

%%%%%%%%%%%%%%%%%%%%%%%%%%%%%%%%%%%%%%%%
\DescribeMacro{\...prefix}
In the alternative form |\childdocforwardprefix|,
%
\begin{center}
\begin{tabular}{l}
|\input{childdoc.def}|\\
|\childdocforwardprefix[|\textit{main}|]{|\textit{prefix}|}{|\textit{dest}|}|
\end{tabular}
\end{center}
%
the destination file is determined by a pattern
depending on the current file:
To make this work, the current file must be called
`{\textit{prefix}\hspace{0.2em}\textit{suffix}}'
with \textit{prefix} matching precisely the argument.
Processing is then passed on to the file
`{\textit{dest}\hspace{0.2em}\textit{suffix}}'.
Surely, the same effect is achieved by
directly specifying the
argument `{\textit{dest}\hspace{0.2em}\textit{suffix}}'
in the first form.
However, that requires to set up a different file
for each child. With the alternative form of the command
all these files can have exactly the same content
which simplifies setting them up and maintaining them.

For example, the following file |draft.tex|
with a compilation flag |\version| as described in \secref{sec:flags}
compiles the main document as a draft:
%
\begin{center}
\begin{tabular}{l}
|\def\version{draft}|\\
|\input{childdoc.def}|\\
|\childdocforward{|\textit{main}|}|
\end{tabular}
\end{center}
%
Likewise, the following files |final|\textit{nn}|.tex|
compile the final version of the child document
|child|\textit{nn}|.tex|:
%
\begin{center}
\begin{tabular}{l}
|\def\version{final}|\\
|\input{childdoc.def}|\\
|\childdocforwardprefix{final}{child}|
\end{tabular}
\end{center}
%

Note that when several versions of a main file and/or of each child file
are to be generated, it may be convenient to set up a |Makefile| or
shell script to automatise the process.

%%%%%%%%%%%%%%%%%%%%%%%%%%%%%%%%%%%%%%%%%%%%%%%%%%%%%%%%%%%%%%%%%%%%%%%%%%%%%%%%
\subsection{Command Line Processing}
\label{sec:commandline}

The effect of redirection files can also be achieved by invoking
the \LaTeX{} compiler with a more elaborate command line.
Most conveniently this should be done as part
of a shell script or a |Makefile|.

When using \textsf{childdoc} in the main file, the following
command lines effectively perform a redirection
(note that depending on the shell being used,
backslashes may have to be doubled: `|\|' $\to$ `|\\|'):
%
\begin{center}
|... -jobname "|\textit{target}|" |\\|"|[\textit{flags}]%
|\input{childdoc.def}\childdocforward[|\textit{main}|]{|\textit{dest}|}"|
\end{center}
%
Here \textit{target} is the name of the output file,
\textit{main} is the name of the main file
and \textit{dest} is the name of the main or child file to be processed
(all filenames without extensions).
The optional argument \textit{main} can be omitted
if \textit{main} matches \textit{dest}.
Optionally, compilation \textit{flags} can be defined via |\def| commands.
This command line makes the \TeX{} engine believe
it is compiling the file \textit{target}
whose content is specified as the latter parameter.
The provided code then forwards the processing to
\textit{main} or \textit{dest} as described in \secref{sec:forward}.

%%%%%%%%%%%%%%%%%%%%%%%%%%%%%%%%%%%%%%%%%%%%%%%%%%%%%%%%%%%%%%%%%%%%%%%%%%%%%%%%
\subsection{Include by Input}
\label{sec:input}

Including child documents by |\include| has some restrictions by design.
Most notably, the content of a child document always occupies
its own set of pages; pages cannot be shared between child documents.
Usually, this behaviour makes perfect sense
because each child document contain an essential part of the document.
However, in some situations it may be desirable to compose
a document from a collection of parts
without having mandatory page breaks between then.
For this case, the package
provides a mechanism to include parts
by |\input| which can also be processed individually.
However, by construction this mechanism
requires manual handling of the content to be output.

%%%%%%%%%%%%%%%%%%%%%%%%%%%%%%%%%%%%%%%%
\DescribeMacro{\ifchilddocmanual}
The main file should be prepared as usual, see \secref{sec:include}.
However, the document body must make a distinction
between processing of an individual part and of the main document, e.g.:
%
\begin{center}
\begin{tabular}{l}
|\ifchilddocmanual|\\
|\input{\childdocname}|\\
|\||else|\\
\textit{document body with }|\input{|\textit{part}|}|\\
|\||fi|
\end{tabular}
\end{center}
%
The conditional |\ifchilddocmanual| is true whenever
a part to be included by |\input| is being compiled,
and the name of the part is stored in |\childdocname|.

%%%%%%%%%%%%%%%%%%%%%%%%%%%%%%%%%%%%%%%%
\DescribeMacro{\childdocby}
Each part to be included by |\input| should start with:
%
\begin{center}
\begin{tabular}{l}
|\input{childdoc.def}|\\
|\childdocby{|\textit{main}|}|\\
\end{tabular}
\end{center}
%
The directive |\childdocby| is similar to |\childdocof|
described in \secref{sec:include},
but the subsequent selection of content must be done manually.
To that end, both |\ifchilddoc| and |\ifchilddocmanual|
will be true upon processing of a part,
and the name of the part is stored in |\childdocname|.
Note that |\jobname| will be set to the filename of the current part
so that each part receives an individual |.aux| file
that does not interfere with the |.aux| file(s) of the main document.
This behaviour can be altered by the alternative form
|\childdocby[*]{|\textit{main}|}| (with a non-empty optional argument)
which uses the |.aux| file of the main document
by setting |\jobname| to \textit{main}.

%%%%%%%%%%%%%%%%%%%%%%%%%%%%%%%%%%%%%%%%%%%%%%%%%%%%%%%%%%%%%%%%%%%%%%%%%%%%%%%%
\subsection{Driver Development}
\label{sec:driver}

The \textsf{childdoc} mechanism can also be use for the development
of definition files such as \LaTeX{} styles or classes.
This case differs from the above setup with multiple parts
included by |\include| in that no |\includeonly| should be invoked.
This can be achieved by starting the include file
(before |\ProvidesPackage|) with:
%
\begin{center}
\begin{tabular}{l}
|\input{childdoc.def}|\\
|\childdocforward{|\textit{main}|}|\\
\end{tabular}
\end{center}
%
or alternatively with:
%
\begin{center}
\begin{tabular}{l}
|\input{childdoc.def}|\\
|\childdocby{|\textit{main}|}|\\
\end{tabular}
\end{center}
%
Both forms have slightly different effects as described above.
The main file is prepared as usual, see \secref{sec:include}.

%%%%%%%%%%%%%%%%%%%%%%%%%%%%%%%%%%%%%%%%%%%%%%%%%%%%%%%%%%%%%%%%%%%%%%%%%%%%%%%%
\subsection{Legacy Detection}
\label{sec:detection}

The directive |\childdocmain| in the main file can detect
whether the complete document or merely a child is to be compiled
even without using the directive |\childdocof|.
This method is deprecated because it is less robust
and there is no compelling reason to use it;
it is merely provided for backward compatibility
and it may be removed in future versions.

If the detection mechanism is to be used,
it is mandatory to correctly specify
the filename of the main file as the argument of |\childdocmain|:
%
\begin{center}
\begin{tabular}{l}
|\input{childdoc.def}|\\
|\childdocmain{|\textit{main}|}|\\
\end{tabular}
\end{center}
%
If |\jobname| does not match the argument \textit{main} of |\childdocmain|,
it is assumed that |\jobname| points to the child file to be compiled.
When using |\childdocmain| with the main file specified as argument,
it suffices to start a child file
with just |\input{|\textit{main}|}|
without loading of the package and using |\childdocof|.
If instead all processing is done
with the appropriate \textsf{childdoc} directives,
the argument of \textit{main} of |\childdocmain| can be empty.

An alternative version of the command line processing described
in \secref{sec:commandline} using the detection mechanism reads:
%
\begin{center}
|... -jobname "|\textit{target}|" "|[\textit{flags}]%
[|\def\jobname{|\textit{dest}|}|]|\input{|\textit{main}|}"|
\end{center}

%%%%%%%%%%%%%%%%%%%%%%%%%%%%%%%%%%%%%%%%%%%%%%%%%%%%%%%%%%%%%%%%%%%%%%%%%%%%%%%%
\subsection{Manual Code}
\label{sec:manual}

In case one cannot be certain whether the definitions file |childdoc.def|
is installed on the target \TeX{} distribution
and one prefers not to ship it,
it is conceivable to paste a few relevant commands into the sources.

To that end, drop all statements |\input{childdoc.def}|
and perform the replacements as outlined below.
Instead of |\childdocmain{|\textit{main}|}| add the following code
to the top of the main file:
%
\begin{center}
\begin{tabular}{l}
|\||ifdefined\childdocname\endinput\||fi\newif\ifchilddoc|\\
|\edef\childdocname{\scantokens\expandafter{\jobname\noexpand}}|\\
|\def\childdocmain{|\textit{main}|}\||ifx\childdocmain\childdocname\||else|\\
|\childdoctrue\includeonly{\childdocname}\let\jobname\childdocmain\||fi|\\
\end{tabular}
\end{center}
%
Instead of |\childdocof{|\textit{main}|}| just include the main file
at the top of each child file:
%
\begin{center}
|\input{|\textit{main}|}|
\end{center}
%
A simple redirection |\childdocforward{|\textit{dest}|}| is achieved by:
%
\begin{center}
|\def\jobname{|\textit{dest}|}\input{\jobname}|
\end{center}
%
The redirection with prefix
|\childdocforwardprefix[|\textit{prefix}|]{|\textit{dest}|}|
is accomplished by:
%
\begin{center}
\begin{tabular}{l}
|{\edef\jobname{\scantokens\expandafter{\jobname\noexpand}}|\\
|\def\redirectjob |\textit{prefix}|#1~~~{\gdef\jobname{|\textit{dest}|#1}}|\\
|\expandafter\redirectjob\jobname~~~}\input{\jobname}|
\end{tabular}
\end{center}

In an alternative approach,
child documents can be compiled by a specific command line
without additional code or specific definitions:
%
\begin{center}
|... -jobname "|\textit{target}|" "|[\textit{flags}]%
|\includeonly{|\textit{dest}|}\input{|\textit{main}|}"|
\end{center}
%

%%%%%%%%%%%%%%%%%%%%%%%%%%%%%%%%%%%%%%%%%%%%%%%%%%%%%%%%%%%%%%%%%%%%%%%%%%%%%%%%
%%%%%%%%%%%%%%%%%%%%%%%%%%%%%%%%%%%%%%%%%%%%%%%%%%%%%%%%%%%%%%%%%%%%%%%%%%%%%%%%
\section{Information}

%%%%%%%%%%%%%%%%%%%%%%%%%%%%%%%%%%%%%%%%%%%%%%%%%%%%%%%%%%%%%%%%%%%%%%%%%%%%%%%%
\subsection{Copyright}

Copyright \copyright{} 2017--2018 Niklas Beisert

This work may be distributed and/or modified under the
conditions of the \LaTeX{} Project Public License, either version 1.3
of this license or (at your option) any later version.
The latest version of this license is in
  \url{http://www.latex-project.org/lppl.txt}
and version 1.3 or later is part of all distributions of \LaTeX{}
version 2005/12/01 or later.

This work has the LPPL maintenance status `maintained'.

The Current Maintainer of this work is Niklas Beisert.

This work consists of the files |README.txt|, |childdoc.ins| and |childdoc.dtx|
as well as the derived files |childdoc.def|, |cdocsamp.tex|
with |cdocsch1.tex|, |cdocsch2.tex|, |cdocspt3.tex|, |cdocspt4.tex|,
|cdocsdrf.tex|, |cdocsfn1.tex|, |cdocsfn2.tex|
as well as |childdoc.pdf|.

%%%%%%%%%%%%%%%%%%%%%%%%%%%%%%%%%%%%%%%%%%%%%%%%%%%%%%%%%%%%%%%%%%%%%%%%%%%%%%%%
\subsection{Files and Installation}

The package consists of the files:
%
\begin{center}
\begin{tabular}{ll}
    |README.txt|   & readme file \\
    |childdoc.ins| & installation file \\
    |childdoc.dtx| & source file \\
    |childdoc.def| & definition file \\
    |cdocsamp.tex| & sample main file \\
    |cdocsch1.tex| & sample include file \\
    |cdocsch2.tex| & sample include file \\
    |cdocspt3.tex| & sample part file \\
    |cdocspt4.tex| & sample part file \\
    |cdocsdrf.tex| & sample redirection file \\
    |cdocsfn1.tex| & sample redirection file \\
    |cdocsfn2.tex| & sample redirection file \\
    |childdoc.pdf| & manual
\end{tabular}
\end{center}
%
The distribution consists of the files
|README.txt|, |childdoc.ins| and |childdoc.dtx|.
%
\begin{itemize}
\item
Run (pdf)\LaTeX{} on |childdoc.dtx|
to compile the manual |childdoc.pdf| (this file).
\item
Run \LaTeX{} on |childdoc.ins| to create the definitions file |childdoc.def|
and the sample |cdocsamp.tex| with include files
|cdocsch1.tex|, |cdocsch2.tex|, |cdocspt3.tex|, |cdocspt4.tex|,
|cdocsdrf.tex|, |cdocsfn1.tex|, |cdocsfn2.tex|.
Then copy the file |childdoc.def| to an appropriate directory of your \LaTeX{}
distribution, e.g.\ \textit{texmf-root}|/tex/latex/childdoc|.
\end{itemize}

%%%%%%%%%%%%%%%%%%%%%%%%%%%%%%%%%%%%%%%%%%%%%%%%%%%%%%%%%%%%%%%%%%%%%%%%%%%%%%%%
\subsection{Related CTAN Packages}

There are several other packages which offer a similar functionality:
%
\begin{itemize}
\item
The packages
\href{http://ctan.org/pkg/docmute}{\textsf{docmute}},
\href{http://ctan.org/pkg/includex}{\textsf{includex}} and
\href{http://ctan.org/pkg/standalone}{\textsf{standalone}}
provide commands to include only the document body of
a child file thus allowing both files to be compiled individually.
\item
The packages \href{http://ctan.org/pkg/subdocs}{\textsf{subdocs}}
and \href{http://ctan.org/pkg/subfiles}{\textsf{subfiles}}
provide structures in which the main and child documents can be
encapsulated and allowing them to be compiled individually.
The inclusion mechanism is different from the conventional |\include|.
\item
The package \href{http://ctan.org/pkg/combine}{\textsf{combine}}
is an elaborate solution to combine several documents into one.
\end{itemize}
%
See also the CTAN topic \href{http://ctan.org/topic/subdocs}{\textsf{subdocs}}
for further related packages.
The present package differs from the above solutions in that
a document structure constructed with the conventional |\include| mechanism
just needs two extra commands at the top of every file
such that all constituent files can be compiled individually.

%%%%%%%%%%%%%%%%%%%%%%%%%%%%%%%%%%%%%%%%%%%%%%%%%%%%%%%%%%%%%%%%%%%%%%%%%%%%%%%%
%\subsection{Feature Suggestions}
%
%The following is a list of features which may be useful for future
%versions of this package:
%%
%\begin{itemize}
%\item
%\ldots
%\end{itemize}

%%%%%%%%%%%%%%%%%%%%%%%%%%%%%%%%%%%%%%%%%%%%%%%%%%%%%%%%%%%%%%%%%%%%%%%%%%%%%%%%
\subsection{Revision History}

%%%%%%%%%%%%%%%%%%%%%%%%%%%%%%%%%%%%%%%%
\paragraph{v2.0:} 2018/12/30

\begin{itemize}
\item
immediate forward processing
\item
added |\childdocby| mechanism
\item
manual restructured
\end{itemize}

%%%%%%%%%%%%%%%%%%%%%%%%%%%%%%%%%%%%%%%%
\paragraph{v1.6:} 2018/01/17

\begin{itemize}
\item
application for development of include files
\item
corrections to manual
\end{itemize}

%%%%%%%%%%%%%%%%%%%%%%%%%%%%%%%%%%%%%%%%
\paragraph{v1.5:} 2017/05/21

\begin{itemize}
\item
more complete structuring introduced
\item
|\childdocof| introduced
\item
|\childdoc| renamed to |\childdocmain|
\item
|\childredirect| renamed to |\childdocforward| and |\childdocforwardprefix|
and functionality expanded
\end{itemize}

%%%%%%%%%%%%%%%%%%%%%%%%%%%%%%%%%%%%%%%%
\paragraph{v1.0:} 2017/04/27

\begin{itemize}
\item
manual and install package
\item
first version published on CTAN
\end{itemize}

%%%%%%%%%%%%%%%%%%%%%%%%%%%%%%%%%%%%%%%%
\paragraph{v0.6:} 2017/04/26

\begin{itemize}
\item
redirection mechanism added
\end{itemize}

%%%%%%%%%%%%%%%%%%%%%%%%%%%%%%%%%%%%%%%%
\paragraph{v0.5:} 2017/04/26

\begin{itemize}
\item
functionality in definition file
\end{itemize}


%%%%%%%%%%%%%%%%%%%%%%%%%%%%%%%%%%%%%%%%%%%%%%%%%%%%%%%%%%%%%%%%%%%%%%%%%%%%%%%%
%%%%%%%%%%%%%%%%%%%%%%%%%%%%%%%%%%%%%%%%%%%%%%%%%%%%%%%%%%%%%%%%%%%%%%%%%%%%%%%%
%%%%%%%%%%%%%%%%%%%%%%%%%%%%%%%%%%%%%%%%%%%%%%%%%%%%%%%%%%%%%%%%%%%%%%%%%%%%%%%%
\appendix

\settowidth\MacroIndent{\rmfamily\scriptsize 000\ }

 \DocInput{childdoc.dtx}

\end{document}
%</driver>
% \fi
%
% %%%%%%%%%%%%%%%%%%%%%%%%%%%%%%%%%%%%%%%%%%%%%%%%%%%%%%%%%%%%%%%%%%%%%%%%%%%%%%
% %%%%%%%%%%%%%%%%%%%%%%%%%%%%%%%%%%%%%%%%%%%%%%%%%%%%%%%%%%%%%%%%%%%%%%%%%%%%%%
% \section{Sample}
%\iffalse
%<*samplemain>
%\fi
%
% The following presents a sample document
% with two chapters, two parts, a title page,
% a compile flag as well as three forwarding files to set the flag.
% It consists of eight |.tex| files:
% \begin{center}
% \begin{tabular}{ll}
% |cdocsamp.tex|&main file\\
% |cdocsch1.tex|&include file for chapter 1\\
% |cdocsch2.tex|&include file for chapter 2\\
% |cdocspt3.tex|&include file for part 3\\
% |cdocspt4.tex|&include file for part 4\\
% |cdocsdrf.tex|&forwarding file for main file in draft mode\\
% |cdocsfi1.tex|&forwarding file for final version of chapter 1\\
% |cdocsfi2.tex|&forwarding file for final version of chapter 2\\
% \end{tabular}
% \end{center}
% Each of the eight files can be compiled directly by the \LaTeX{} compiler.
%
% %%%%%%%%%%%%%%%%%%%%%%%%%%%%%%%%%%%%%%
% \paragraph{Main File.}
%
% The main file is called |cdocsamp.tex|.
%
% Load the \textsf{childdoc} definitions and
% declare the filename for the main document:
%    \begin{macrocode}
\input{childdoc.def}
\childdocmain{}
%    \end{macrocode}

% Optional override for |\version| flag:
%    \begin{macrocode}
%%\ifchilddoc\else\providecommand{\version}{draft}\fi
%    \end{macrocode}

% Define the default values for the |\version| flag
% (|final| for the main file and |draft| for childs):
%    \begin{macrocode}
\ifchilddoc
\providecommand{\version}{draft}
\else
\providecommand{\version}{final}
\fi
%    \end{macrocode}

% Load the standard document class:
%    \begin{macrocode}
\documentclass[12pt]{article}
%    \end{macrocode}

% Start the document body:
%    \begin{macrocode}
\begin{document}
%    \end{macrocode}

% Declare a title page.
% Print title, part of document being processed and version flag:
%    \begin{macrocode}
\addtocounter{page}{-1}
\begin{center}
{\LARGE\bfseries{}childdoc example\par}
\vspace{1cm}
\ifchilddoc
\ifchilddocmanual part\else chapter\fi:
`\childdocname' of `\childdocjob'\par
\else
main document: `\childdocjob'\par
\fi
version: \version\par
\end{center}
\newpage
%    \end{macrocode}

% Manually include selected file,
% otherwise process as usual:
%    \begin{macrocode}
\ifchilddocmanual
\section*{part `\childdocname'}
\input{\childdocname}
\else
%    \end{macrocode}

% Include the two chapters:
%    \begin{macrocode}
\include{cdocsch1}
\include{cdocsch2}
%    \end{macrocode}

% Include the two parts unless only chapters should be displayed:
%    \begin{macrocode}
\ifchilddoc\else
\section{part three}
\input{cdocspt3}
\section{part four}
\input{cdocspt4}
\fi
%    \end{macrocode}

% Process as usual until here:
%    \begin{macrocode}
\fi
%    \end{macrocode}

% End of document body:
%    \begin{macrocode}
\end{document}
%    \end{macrocode}
%\iffalse
%</samplemain>
%\fi
%
% %%%%%%%%%%%%%%%%%%%%%%%%%%%%%%%%%%%%%%
% \paragraph{Chapter Include Files.}
%
% The include files are called |cdocsch1.tex| and |cdocsch2.tex|.
%
%\iffalse
%<*samplechap1|samplechap2>
%\fi

% Optional override for |\version| flag:
%    \begin{macrocode}
%%\providecommand{\version}{final}
%    \end{macrocode}

% Include the main document:
%    \begin{macrocode}
\input{childdoc.def}
\childdocof{cdocsamp}
%    \end{macrocode}

%\iffalse
%</samplechap1|samplechap2>
%\fi
%
%\iffalse
%<*samplechap1>
%\fi
% Some text for chapter 1:
%    \begin{macrocode}
\section{one}
some text in chapter one
%    \end{macrocode}

%\iffalse
%</samplechap1>
%\fi
% Some text for chapter 2:
%\iffalse
%<*samplechap2>
%\fi
%    \begin{macrocode}
\section{two}
more text in chapter two
%    \end{macrocode}

%\iffalse
%</samplechap2>
%\fi
%
% %%%%%%%%%%%%%%%%%%%%%%%%%%%%%%%%%%%%%%
% \paragraph{Part Include Files.}
%
% The include files are called |cdocspt3.tex| and |cdocspt4.tex|.
%
%\iffalse
%<*samplepart3|samplepart4>
%\fi

% Optional override for |\version| flag:
%    \begin{macrocode}
%%\providecommand{\version}{final}
%    \end{macrocode}

% Include the main document:
%    \begin{macrocode}
\input{childdoc.def}
\childdocby{cdocsamp}
%    \end{macrocode}

%\iffalse
%</samplepart3|samplepart4>
%\fi
%
%\iffalse
%<*samplepart3>
%\fi
% Some text for part 3:
%    \begin{macrocode}
some text in part three
%    \end{macrocode}

%\iffalse
%</samplepart3>
%\fi
% Some text for part 4:
%\iffalse
%<*samplepart4>
%\fi
%    \begin{macrocode}
more text in part four
%    \end{macrocode}

%\iffalse
%</samplepart4>
%\fi
%
% %%%%%%%%%%%%%%%%%%%%%%%%%%%%%%%%%%%%%%
% \paragraph{Forwarding for a Complete Draft.}
%
% The following forwarding file |cdocsdrf.tex|
% compiles the main document in draft mode:
%\iffalse
%<*sampledraft>
%\fi
%    \begin{macrocode}
\def\version{draft}
\input{childdoc.def}
\childdocforward{cdocsamp}
%    \end{macrocode}

%\iffalse
%</sampledraft>
%\fi
%
% %%%%%%%%%%%%%%%%%%%%%%%%%%%%%%%%%%%%%%
% \paragraph{Forwarding for Final Version of the Chapters.}
%
% The following forwarding files |cdocsfn1.tex| and |cdocsfn2.tex|
% (with identical content)
% compile the final versions of the child documents
% |cdocsch1.tex| and |cdocsch2.tex|, respectively:
%\iffalse
%<*samplefinal>
%\fi
%    \begin{macrocode}
\def\version{final}
\input{childdoc.def}
\childdocforwardprefix[cdocsamp]{cdocsfn}{cdocsch}
%    \end{macrocode}

%\iffalse
%</samplefinal>
%\fi
%
% %%%%%%%%%%%%%%%%%%%%%%%%%%%%%%%%%%%%%%
% \paragraph{Command Line Processing.}
%
% The following three command lines generate the output files
% |cdocscld|, |cdocscl1| and |cdocscl2|
% which should be identical to
% |cdocsdrf|, |cdocsch1| and |cdocsfn2|, respectively:
% \begin{center}
% \begin{tabular}{l}
% |latex -jobname cdocscld \|\\
% |  "\def\version{draft}\input{childdoc.def}\childdocforward{cdocsamp}"|\\
% |latex -jobname cdocscl1 \|\\
% |  "\input{childdoc.def}\childdocforward[cdocsamp]{cdocsch1}"|\\
% |latex -jobname cdocscl2 \|\\
% |  "\def\version{final}\input{childdoc.def}\childdocforward{cdocsch2}"|
% \end{tabular}
% \end{center}
% Note that the trailing backslash on each first line
% merely continues the input to the second line
% (for convenient cut ant paste).
% Furthermore, the command |latex| can be replaced by any
% of its alternative versions such as |pdflatex|.
%
% %%%%%%%%%%%%%%%%%%%%%%%%%%%%%%%%%%%%%%%%%%%%%%%%%%%%%%%%%%%%%%%%%%%%%%%%%%%%%%
% %%%%%%%%%%%%%%%%%%%%%%%%%%%%%%%%%%%%%%%%%%%%%%%%%%%%%%%%%%%%%%%%%%%%%%%%%%%%%%
% \section{Implementation}
%\iffalse
%<*package>
%\fi
%
% This section describes the definitions file |childdoc.def|.

% The definitions cannot be loaded using |\usepackage| or |\RequirePackage|
% which has a mechanism to prevent loading a style file more than once.
% When loading the definitions by means of |\input|
% multiple instances have to be prevented manually:
%\iffalse
%This code needs to be before the `\ProvidesFile' directive
%which is defined at the beginning of this file.
%Therefore it is also placed there and commented out here.
%</package>
%<*discard>
%\fi
%    \begin{macrocode}
\ifdefined\childdocmain\endinput\fi
%    \end{macrocode}
%\iffalse
%</discard>
%<*package>
%\fi
%
% \macro{\ifchilddoc}
% \macro{\ifchilddocmanual}
% The conditional |\ifchilddoc| tells whether a
% child (true) or main (false) document is being compiled.
% The conditional |\ifchilddocmanual| tells whether
% the |\includeonly| mechanism is used (false) or
% the selection of child files must be performed manually (true).
% The definitions initialise to false:
%    \begin{macrocode}
\newif\ifchilddoc
\newif\ifchilddocmanual
%    \end{macrocode}

% \macro{\childdocname}
% \macro{\childdocjob}
% The macro |\childdocname| stores the name of the main document
% to be compiled. The macro |\childdocjob| stores the name of
% the document on which the \LaTeX{} compiler was originally invoked.
% The content of |\jobname| cannot be compared
% to filenames specified in the source due to different catcodes.
% The following code rescans |\jobname|, stores the result
% in |\childdocname| and saves a copy in |\childdocjob|:
%    \begin{macrocode}
\edef\childdocname{\scantokens\expandafter{\jobname\noexpand}}
\let\childdocjob\childdocname
%    \end{macrocode}

% \macro{\childdocdisable}
% The macro |\childdocdisable| prevents the main file
% from being processed more than once.
% At this stage, the main document command |\childdocmain|
% is assumed to be called once again where it should do nothing.
% Any subsequent call to it should prevent
% a secondary processing of the main document
% It overwrites the forwarding commands
% |\childdocof| and |\childdocforward|
% with empty macros to prevent further inclusions of the main document:
%    \begin{macrocode}
\newcommand{\childdocdisable}
{
  \renewcommand{\childdocmain}[1]{\renewcommand{\childdocmain}[1]{\endinput}}
  \renewcommand{\childdocof}[1]{}
  \renewcommand{\childdocby}[2][]{}
  \renewcommand{\childdocforward}[2][]{}
  \renewcommand{\childdocdisable}{}
}
%    \end{macrocode}

% \macro{\childdocmain}
% The macro |\childdocmain| is to be called at the top of the main file
% with nothing or the main filename (without extension) as argument.
% First, it breaks loops.
% If the argument is not empty and does not match |\childdocname|
% (which is set by the first inclusion of |childdoc.def|),
% |\ifchilddoc| is set to true, |\includeonly| is applied to the child file
% and |\jobname| is set to the main file
% (for proper handling of |.aux| files):
%    \begin{macrocode}
\newcommand{\childdocmain}[1]
{
  \childdocdisable\childdocmain{}
  \if?#1?\else
    \begingroup
      \def\childdoctmp{#1}
      \ifx\childdoctmp\childdocname
        \def\childdoctmp{}
      \else
        \def\childdoctmp
        {
          \childdoctrue
          \includeonly{\childdocname}
          \def\childdocjob{#1}
          \def\jobname{#1}
        }
      \fi
      \expandafter
    \endgroup
    \childdoctmp
  \fi
}
%    \end{macrocode}

% \macro{\childdocof}
% The command |\childdocof| redirects
% compilation to the main file |#1|.
%    \begin{macrocode}
\newcommand{\childdocof}[1]
{
  \childdocdisable
  \childdoctrue
  \includeonly{\childdocname}
  \def\jobname{#1}
  \def\childdocjob{#1}
  \input{#1}
}
%    \end{macrocode}

% \macro{\childdocby}
% The command |\childdocby| ....
%    \begin{macrocode}
\newcommand{\childdocby}[2][]
{
  \childdocdisable
  \childdoctrue
  \childdocmanualtrue
  \if?#1?\else
    \def\jobname{#2}
  \fi
  \def\childdocjob{#2}
  \input{#2}
  \endinput
}
%    \end{macrocode}

% \macro{\childdocforward}
% The command |\childdocforward| redirects
% compilation to the main file or
% (if the optional argument is given) a child file.
% Parameters are set as if the main file
% or a child file starting with |\childdocof| was compiled.
% Then compilation is handed over to the main file:
%    \begin{macrocode}
\newcommand{\childdocforward}[2][]
{
  \begingroup
    \if?#1?
      \def\childdoctmp
      {
        \def\childdocname{#2}
        \def\childdocjob{#2}
        \def\jobname{#2}
        \input{#2}
        \endinput
      }
    \else
      \def\childdoctmp
      {
        \childdocdisable
        \def\childdocname{#2}
        \childdoctrue
        \includeonly{#2}
        \def\childdocjob{#1}
        \def\jobname{#1}
        \input{#1}
        \endinput
      }
    \fi
    \expandafter
  \endgroup
  \childdoctmp
}
%    \end{macrocode}

% \macro{\childdocforwardprefix}
% The command |\childdocforwardprefix| redirects
% compilation to the main or a child file by means of a pattern.
% The prefix |#1| in the current filename is replaced by |#2|
% and the suffix of the current filename is kept
% (it is assumed that the filename does not contain the substring `|~~~|'
% which is used as a delimiter).
% Compilation is handed over to the new file by |\childdocforward|:
%    \begin{macrocode}
\newcommand{\childdocforwardprefix}[3][]
{
  \begingroup
    \def\childdocextract #2##1~~~{\def\childdoctmp{\childdocforward[#1]{#3##1}}}
    \expandafter\childdocextract\childdocname~~~
    \expandafter
  \endgroup
  \childdoctmp
}
%    \end{macrocode}

% \macro{\childdoc}
% The deprecated macro |\childdoc| is a legacy version of |\childdocmain|:
%    \begin{macrocode}
\newcommand{\childdoc}{\childdocmain}
%    \end{macrocode}

% \macro{\childdocredirect}
% The deprecated macro |\childdocredirect| is a legacy version
% of |\childdocforward| and |\childdocforwardprefix|:
%    \begin{macrocode}
\newcommand{\childdocredirect}[2][]
{
  \begingroup
    \if?#1?
      \def\childdoctmp{\childdocforward{#2}}
    \else
      \def\childdoctmp{\childdocforwardprefix{#1}{#2}}
    \fi
    \expandafter
  \endgroup
  \childdoctmp
}
%    \end{macrocode}

%\iffalse
%</package>
%\fi
%
\endinput
|
and perform the replacements as outlined below.
Instead of |\childdocmain{|\textit{main}|}| add the following code
to the top of the main file:
%
\begin{center}
\begin{tabular}{l}
|\||ifdefined\childdocname\endinput\||fi\newif\ifchilddoc|\\
|\edef\childdocname{\scantokens\expandafter{\jobname\noexpand}}|\\
|\def\childdocmain{|\textit{main}|}\||ifx\childdocmain\childdocname\||else|\\
|\childdoctrue\includeonly{\childdocname}\let\jobname\childdocmain\||fi|\\
\end{tabular}
\end{center}
%
Instead of |\childdocof{|\textit{main}|}| just include the main file
at the top of each child file:
%
\begin{center}
|\input{|\textit{main}|}|
\end{center}
%
A simple redirection |\childdocforward{|\textit{dest}|}| is achieved by:
%
\begin{center}
|\def\jobname{|\textit{dest}|}\input{\jobname}|
\end{center}
%
The redirection with prefix
|\childdocforwardprefix[|\textit{prefix}|]{|\textit{dest}|}|
is accomplished by:
%
\begin{center}
\begin{tabular}{l}
|{\edef\jobname{\scantokens\expandafter{\jobname\noexpand}}|\\
|\def\redirectjob |\textit{prefix}|#1~~~{\gdef\jobname{|\textit{dest}|#1}}|\\
|\expandafter\redirectjob\jobname~~~}\input{\jobname}|
\end{tabular}
\end{center}

In an alternative approach,
child documents can be compiled by a specific command line
without additional code or specific definitions:
%
\begin{center}
|... -jobname "|\textit{target}|" "|[\textit{flags}]%
|\includeonly{|\textit{dest}|}\input{|\textit{main}|}"|
\end{center}
%

%%%%%%%%%%%%%%%%%%%%%%%%%%%%%%%%%%%%%%%%%%%%%%%%%%%%%%%%%%%%%%%%%%%%%%%%%%%%%%%%
%%%%%%%%%%%%%%%%%%%%%%%%%%%%%%%%%%%%%%%%%%%%%%%%%%%%%%%%%%%%%%%%%%%%%%%%%%%%%%%%
\section{Information}

%%%%%%%%%%%%%%%%%%%%%%%%%%%%%%%%%%%%%%%%%%%%%%%%%%%%%%%%%%%%%%%%%%%%%%%%%%%%%%%%
\subsection{Copyright}

Copyright \copyright{} 2017--2018 Niklas Beisert

This work may be distributed and/or modified under the
conditions of the \LaTeX{} Project Public License, either version 1.3
of this license or (at your option) any later version.
The latest version of this license is in
  \url{http://www.latex-project.org/lppl.txt}
and version 1.3 or later is part of all distributions of \LaTeX{}
version 2005/12/01 or later.

This work has the LPPL maintenance status `maintained'.

The Current Maintainer of this work is Niklas Beisert.

This work consists of the files |README.txt|, |childdoc.ins| and |childdoc.dtx|
as well as the derived files |childdoc.def|, |cdocsamp.tex|
with |cdocsch1.tex|, |cdocsch2.tex|, |cdocspt3.tex|, |cdocspt4.tex|,
|cdocsdrf.tex|, |cdocsfn1.tex|, |cdocsfn2.tex|
as well as |childdoc.pdf|.

%%%%%%%%%%%%%%%%%%%%%%%%%%%%%%%%%%%%%%%%%%%%%%%%%%%%%%%%%%%%%%%%%%%%%%%%%%%%%%%%
\subsection{Files and Installation}

The package consists of the files:
%
\begin{center}
\begin{tabular}{ll}
    |README.txt|   & readme file \\
    |childdoc.ins| & installation file \\
    |childdoc.dtx| & source file \\
    |childdoc.def| & definition file \\
    |cdocsamp.tex| & sample main file \\
    |cdocsch1.tex| & sample include file \\
    |cdocsch2.tex| & sample include file \\
    |cdocspt3.tex| & sample part file \\
    |cdocspt4.tex| & sample part file \\
    |cdocsdrf.tex| & sample redirection file \\
    |cdocsfn1.tex| & sample redirection file \\
    |cdocsfn2.tex| & sample redirection file \\
    |childdoc.pdf| & manual
\end{tabular}
\end{center}
%
The distribution consists of the files
|README.txt|, |childdoc.ins| and |childdoc.dtx|.
%
\begin{itemize}
\item
Run (pdf)\LaTeX{} on |childdoc.dtx|
to compile the manual |childdoc.pdf| (this file).
\item
Run \LaTeX{} on |childdoc.ins| to create the definitions file |childdoc.def|
and the sample |cdocsamp.tex| with include files
|cdocsch1.tex|, |cdocsch2.tex|, |cdocspt3.tex|, |cdocspt4.tex|,
|cdocsdrf.tex|, |cdocsfn1.tex|, |cdocsfn2.tex|.
Then copy the file |childdoc.def| to an appropriate directory of your \LaTeX{}
distribution, e.g.\ \textit{texmf-root}|/tex/latex/childdoc|.
\end{itemize}

%%%%%%%%%%%%%%%%%%%%%%%%%%%%%%%%%%%%%%%%%%%%%%%%%%%%%%%%%%%%%%%%%%%%%%%%%%%%%%%%
\subsection{Related CTAN Packages}

There are several other packages which offer a similar functionality:
%
\begin{itemize}
\item
The packages
\href{http://ctan.org/pkg/docmute}{\textsf{docmute}},
\href{http://ctan.org/pkg/includex}{\textsf{includex}} and
\href{http://ctan.org/pkg/standalone}{\textsf{standalone}}
provide commands to include only the document body of
a child file thus allowing both files to be compiled individually.
\item
The packages \href{http://ctan.org/pkg/subdocs}{\textsf{subdocs}}
and \href{http://ctan.org/pkg/subfiles}{\textsf{subfiles}}
provide structures in which the main and child documents can be
encapsulated and allowing them to be compiled individually.
The inclusion mechanism is different from the conventional |\include|.
\item
The package \href{http://ctan.org/pkg/combine}{\textsf{combine}}
is an elaborate solution to combine several documents into one.
\end{itemize}
%
See also the CTAN topic \href{http://ctan.org/topic/subdocs}{\textsf{subdocs}}
for further related packages.
The present package differs from the above solutions in that
a document structure constructed with the conventional |\include| mechanism
just needs two extra commands at the top of every file
such that all constituent files can be compiled individually.

%%%%%%%%%%%%%%%%%%%%%%%%%%%%%%%%%%%%%%%%%%%%%%%%%%%%%%%%%%%%%%%%%%%%%%%%%%%%%%%%
%\subsection{Feature Suggestions}
%
%The following is a list of features which may be useful for future
%versions of this package:
%%
%\begin{itemize}
%\item
%\ldots
%\end{itemize}

%%%%%%%%%%%%%%%%%%%%%%%%%%%%%%%%%%%%%%%%%%%%%%%%%%%%%%%%%%%%%%%%%%%%%%%%%%%%%%%%
\subsection{Revision History}

%%%%%%%%%%%%%%%%%%%%%%%%%%%%%%%%%%%%%%%%
\paragraph{v2.0:} 2018/12/30

\begin{itemize}
\item
immediate forward processing
\item
added |\childdocby| mechanism
\item
manual restructured
\end{itemize}

%%%%%%%%%%%%%%%%%%%%%%%%%%%%%%%%%%%%%%%%
\paragraph{v1.6:} 2018/01/17

\begin{itemize}
\item
application for development of include files
\item
corrections to manual
\end{itemize}

%%%%%%%%%%%%%%%%%%%%%%%%%%%%%%%%%%%%%%%%
\paragraph{v1.5:} 2017/05/21

\begin{itemize}
\item
more complete structuring introduced
\item
|\childdocof| introduced
\item
|\childdoc| renamed to |\childdocmain|
\item
|\childredirect| renamed to |\childdocforward| and |\childdocforwardprefix|
and functionality expanded
\end{itemize}

%%%%%%%%%%%%%%%%%%%%%%%%%%%%%%%%%%%%%%%%
\paragraph{v1.0:} 2017/04/27

\begin{itemize}
\item
manual and install package
\item
first version published on CTAN
\end{itemize}

%%%%%%%%%%%%%%%%%%%%%%%%%%%%%%%%%%%%%%%%
\paragraph{v0.6:} 2017/04/26

\begin{itemize}
\item
redirection mechanism added
\end{itemize}

%%%%%%%%%%%%%%%%%%%%%%%%%%%%%%%%%%%%%%%%
\paragraph{v0.5:} 2017/04/26

\begin{itemize}
\item
functionality in definition file
\end{itemize}


%%%%%%%%%%%%%%%%%%%%%%%%%%%%%%%%%%%%%%%%%%%%%%%%%%%%%%%%%%%%%%%%%%%%%%%%%%%%%%%%
%%%%%%%%%%%%%%%%%%%%%%%%%%%%%%%%%%%%%%%%%%%%%%%%%%%%%%%%%%%%%%%%%%%%%%%%%%%%%%%%
%%%%%%%%%%%%%%%%%%%%%%%%%%%%%%%%%%%%%%%%%%%%%%%%%%%%%%%%%%%%%%%%%%%%%%%%%%%%%%%%
\appendix

\settowidth\MacroIndent{\rmfamily\scriptsize 000\ }

 \DocInput{childdoc.dtx}

\end{document}
%</driver>
% \fi
%
% %%%%%%%%%%%%%%%%%%%%%%%%%%%%%%%%%%%%%%%%%%%%%%%%%%%%%%%%%%%%%%%%%%%%%%%%%%%%%%
% %%%%%%%%%%%%%%%%%%%%%%%%%%%%%%%%%%%%%%%%%%%%%%%%%%%%%%%%%%%%%%%%%%%%%%%%%%%%%%
% \section{Sample}
%\iffalse
%<*samplemain>
%\fi
%
% The following presents a sample document
% with two chapters, two parts, a title page,
% a compile flag as well as three forwarding files to set the flag.
% It consists of eight |.tex| files:
% \begin{center}
% \begin{tabular}{ll}
% |cdocsamp.tex|&main file\\
% |cdocsch1.tex|&include file for chapter 1\\
% |cdocsch2.tex|&include file for chapter 2\\
% |cdocspt3.tex|&include file for part 3\\
% |cdocspt4.tex|&include file for part 4\\
% |cdocsdrf.tex|&forwarding file for main file in draft mode\\
% |cdocsfi1.tex|&forwarding file for final version of chapter 1\\
% |cdocsfi2.tex|&forwarding file for final version of chapter 2\\
% \end{tabular}
% \end{center}
% Each of the eight files can be compiled directly by the \LaTeX{} compiler.
%
% %%%%%%%%%%%%%%%%%%%%%%%%%%%%%%%%%%%%%%
% \paragraph{Main File.}
%
% The main file is called |cdocsamp.tex|.
%
% Load the \textsf{childdoc} definitions and
% declare the filename for the main document:
%    \begin{macrocode}
% \iffalse
%
% childdoc.dtx Copyright (C) 2017-2018 Niklas Beisert
%
% This work may be distributed and/or modified under the
% conditions of the LaTeX Project Public License, either version 1.3
% of this license or (at your option) any later version.
% The latest version of this license is in
%   http://www.latex-project.org/lppl.txt
% and version 1.3 or later is part of all distributions of LaTeX
% version 2005/12/01 or later.
%
% This work has the LPPL maintenance status `maintained'.
%
% The Current Maintainer of this work is Niklas Beisert.
%
% This work consists of the files childdoc.dtx and childdoc.ins
% and the derived files childdoc.def and cdocsamp.tex with
% cdocsch1.tex, cdocsch2.tex, cdocsdrf.tex, cdocsfn1.tex, cdocsfn2.tex.
%
%<package>\ifdefined\childdocmain\endinput\fi
%<package>\ProvidesFile{childdoc.def}[2018/12/30 v2.0 child document driver]
%<samplemain>\ProvidesFile{cdocsamp.tex}[2018/12/30 v2.0 sample for childdoc]
%<*driver>
%\ProvidesFile{childdoc.drv}[2018/12/30 v2.0 childdoc reference manual file]
\PassOptionsToClass{10pt,a4paper}{article}
\documentclass{ltxdoc}

\usepackage[margin=35mm]{geometry}
\usepackage{hyperref}
\usepackage{hyperxmp}
\usepackage[usenames]{color}

\hypersetup{colorlinks=true}
\hypersetup{pdfstartview=FitH}
\hypersetup{pdfpagemode=UseNone}
\hypersetup{pdfsource={}}
\hypersetup{pdflang={en-UK}}
\hypersetup{pdfcopyright={Copyright 2017-2018 Niklas Beisert.
  This work may be distributed and/or modified under the
  conditions of the LaTeX Project Public License, either version 1.3
  of this license or (at your option) any later version.}}
\hypersetup{pdflicenseurl={http://www.latex-project.org/lppl.txt}}
\hypersetup{pdfcontactaddress={ETH Zurich, ITP, HIT K,
  Wolfgang-Pauli-Strasse 27}}
\hypersetup{pdfcontactpostcode={8093}}
\hypersetup{pdfcontactcity={Zurich}}
\hypersetup{pdfcontactcountry={Switzerland}}
\hypersetup{pdfcontactemail={nbeisert@itp.phys.ethz.ch}}
\hypersetup{pdfcontacturl={http://people.phys.ethz.ch/\xmptilde nbeisert/}}

\newcommand{\secref}[1]{\hyperref[#1]{section \ref*{#1}}}

\parskip1ex
\parindent0pt
\let\olditemize\itemize
\def\itemize{\olditemize\parskip0pt}

\begin{document}

\title{The \textsf{childdoc} Package}
\hypersetup{pdftitle={The childdoc Package}}
\author{Niklas Beisert\\[2ex]
  Institut f\"ur Theoretische Physik\\
  Eidgen\"ossische Technische Hochschule Z\"urich\\
  Wolfgang-Pauli-Strasse 27, 8093 Z\"urich, Switzerland\\[1ex]
  \href{mailto:nbeisert@itp.phys.ethz.ch}
  {\texttt{nbeisert@itp.phys.ethz.ch}}}
\hypersetup{pdfauthor={Niklas Beisert}}
\hypersetup{pdfsubject={Manual for the LaTeX2e Package childdoc}}
\date{30 December 2018, \textsf{v2.0}}
\maketitle

\begin{abstract}\noindent
\textsf{childdoc} is a \LaTeXe{} package
that enables the direct compilation
of document sections included by |\include|
to individual files.
\end{abstract}

\begingroup
\parskip0ex
\tableofcontents
\endgroup

%%%%%%%%%%%%%%%%%%%%%%%%%%%%%%%%%%%%%%%%%%%%%%%%%%%%%%%%%%%%%%%%%%%%%%%%%%%%%%%%
%%%%%%%%%%%%%%%%%%%%%%%%%%%%%%%%%%%%%%%%%%%%%%%%%%%%%%%%%%%%%%%%%%%%%%%%%%%%%%%%
\section{Introduction}

\LaTeX{} provides a mechanism to structure a large document (such as a book)
into a main file and several child files (containing the chapters)
using the |\include| command.
This mechanism is beneficial for documents
which span hundreds of pages in order to
make the source file(s) more manageable.
Moreover, compilation can be restricted to
selected child files by means of the |\includeonly| command.
The latter feature can be used to reduce the compilation time while editing
(this was significantly more useful in the earlier days of \LaTeX{})
or to generate a smaller document which is easier to navigate.
Another application of |\includeonly| is to generate
documents consisting of selected parts of the complete document.

However, there are a few drawbacks of the plain |\include| mechanism:
\begin{itemize}
\item
The child files cannot be compiled on their own,
they can only be compiled via the main file.
A naive editing environment
(such as a text editor with an option
to have the current file processed by \LaTeX)
may require one to switch to the main file before compiling;
attempting to compile the child file produces errors.
\item
The main file must be modified (each time)
to adjust the |\includeonly| command
to the present needs. This easily leaves the main file in a messy state.
\item
The generated document will always carry the filename
of the main document. This is inconvenient if
several child files are to be compiled and
to be kept for distribution.
\end{itemize}

The present package provides a simple interface
to make child files individually compilable by \LaTeX{}.
Compiling a child file then has the same effect as compiling
the main file with an |\includeonly| command
to select the appropriate child.
Moreover the generated document will carry the name of the child
rather than the main file.
This resolves all three above issues.

This feature is meant to make the editing of books,
thesis documents and lecture notes somewhat more convenient.
However, the package can also be used efficiently for
composing a series of documents (such as exercise sheets)
which are typically distributed individually.
It then assists the author in generating the individual documents
(potentially in different versions)
as well as a document containing the collected series.
Another application is in developing style files
or other kinds of included material
where compilation of the style file could redirect
to a sample or test file.

%%%%%%%%%%%%%%%%%%%%%%%%%%%%%%%%%%%%%%%%%%%%%%%%%%%%%%%%%%%%%%%%%%%%%%%%%%%%%%%%
%%%%%%%%%%%%%%%%%%%%%%%%%%%%%%%%%%%%%%%%%%%%%%%%%%%%%%%%%%%%%%%%%%%%%%%%%%%%%%%%
\section{Usage}

First of all, the package \textsf{childdoc} is \emph{not} a standard
\LaTeXe{} |.sty| style file! Therefore it needs to be invoked in
a non-standard way.

%%%%%%%%%%%%%%%%%%%%%%%%%%%%%%%%%%%%%%%%%%%%%%%%%%%%%%%%%%%%%%%%%%%%%%%%%%%%%%%%
\subsection{Included Files}
\label{sec:include}

%%%%%%%%%%%%%%%%%%%%%%%%%%%%%%%%%%%%%%%%
\DescribeMacro{\childdocmain}
To use the package, add the commands
\begin{center}
\begin{tabular}{l}
|\input{childdoc.def}|\\
|\childdocmain{}|\\
\end{tabular}
\end{center}
at the very top of the main \LaTeX{} file,
in particular \emph{before} the |\documentclass| statement!
The argument of |\childdocmain| should be left empty
(but it must be present).

%%%%%%%%%%%%%%%%%%%%%%%%%%%%%%%%%%%%%%%%
\DescribeMacro{\childdocof}
Furthermore, add the commands
\begin{center}
\begin{tabular}{l}
|\input{childdoc.def}|\\
|\childdocof{|\textit{main}|}|\\
\end{tabular}
\end{center}
at the top of every child file \textit{child}
which is included by |\include{|\textit{child}|}|
from within the main file
(or at least for those files to be compiled individually).
The argument \textit{main} must be the filename of the main file.

There are a couple of
considerations in setting up the main and child documents:

%%%%%%%%%%%%%%%%%%%%%%%%%%%%%%%%%%%%%%%%
\paragraph{Restrictions.}

Please note the following restrictions:
\begin{itemize}
\item
|\childdocmain| must be called with one argument \textit{main}
to ensure compatibility with earlier version of the package.
It must either be empty (|\childdocmain{}|)
or precisely match the filename of the main file in which it is specified.
See \secref{sec:detection} for further information.
\item
The filename \textit{main} must be specified without the |.tex| extension.
\item
The filename \textit{main} is case sensitive
(even in case-insensitive file systems)
due to internal string comparison.
\item
The argument \textit{main} should be fully expanded, it cannot be a macro.
\item
Subdirectories and special characters should be avoided in filenames.
\item
The command |\childdocmain{|\textit{main}|}| must be followed by a whitespace.
It should not be followed immediately by another command
or by a comment mark `|%|'.
This is because the \TeX{} parser reads the token immediately following
the argument of |\childdocmain| and puts it
at the beginning of every child section;
however, a white\-space is ignored.
\end{itemize}

%%%%%%%%%%%%%%%%%%%%%%%%%%%%%%%%%%%%%%%%
\paragraph{Content of Main File.}

It is advisable to place all content in the child files included by |\include|.
Any output contained in the main file will appear in all child documents
unless suppressed manually;
it cannot be suppressed automatically by the |\includeonly| directive
and thus should normally be avoided.
A method to include some content in the main file
by means of conditional processing is described in \secref{sec:conditional}.

%%%%%%%%%%%%%%%%%%%%%%%%%%%%%%%%%%%%%%%%
\paragraph{Page Numbering.}

When only a part of the document is compiled,
the appropriate numbering of pages
(as well as other status parameters)
is determined from the |.aux| files.
The latter contain information from previous passes.
However this information needs to propagate through
all intermediate child documents.
Therefore the page numbering in child documents may well
be inconsistent until the complete document is compiled at least once.

A useful (if unconventional) way to always ensure a consistent
page numbering is to restart the numbering in each child document
and denote the pages by `\textit{child}|.|\textit{page}'
where \textit{child} represents the chapter/section number of the child file.
This can be achieved by the command
|\numberwithin{page}{|\textit{child}|}|
of the \textsf{amsmath} package
where \textit{child} can be |chapter| or |section|
depending on the chosen structuring.
Alternatively, one can modify the macro |\thepage| appropriately
and reset the counter |page| at the start of each child file.

%%%%%%%%%%%%%%%%%%%%%%%%%%%%%%%%%%%%%%%%%%%%%%%%%%%%%%%%%%%%%%%%%%%%%%%%%%%%%%%%
\subsection{Conditional Processing}
\label{sec:conditional}

The package provides a mechanism to compile different versions
of a document. To customise the versions further some conditional processing
can come in handy to distinguish which version is being compiled.
The package provides two macros to describe the compilation context:

%%%%%%%%%%%%%%%%%%%%%%%%%%%%%%%%%%%%%%%%
\DescribeMacro{\ifchilddoc}
The conditional |\ifchilddoc| distinguishes between the compilation of
child documents and the main document:
%
\begin{center}
|\ifchilddoc |\textit{child-code}| |[|\||else |\textit{main-code}]| \||fi|
\end{center}

%%%%%%%%%%%%%%%%%%%%%%%%%%%%%%%%%%%%%%%%
\DescribeMacro{\childdocname}
\DescribeMacro{\childdocjob}
The macro |\childdocname| contains the filename (without extension)
of the main or child file being processed.
Note that |\childdocjob| will always contain the name of the main file.

%%%%%%%%%%%%%%%%%%%%%%%%%%%%%%%%%%%%%%%%
\paragraph{Title Page.}

Conditional processing can be used to include a title or banner page
in the main document when proper precautions are taken.
Importantly, the code in the main file should ensure that the page counter
(as well as other status parameters which are stored in the |.aux| files)
takes the same value after the conditional processing.
Otherwise the page numbers may take divergent values
depending on which part is compiled.

For example, a title page could be declared by:
%
\begin{center}
\begin{tabular}{l}
|\ifchilddoc\||else|\\
|\addtocounter{page}{-1}|\\
\textit{code for title page}\\
|\newpage|\\
|\||fi|
\end{tabular}
\end{center}
%
A banner page for the child documents can be generated by:
%
\begin{center}
\begin{tabular}{l}
|\ifchilddoc|\\
|\addtocounter{page}{-1}|\\
\textit{code for banner page}\\
|\newpage|\\
|\||fi|
\end{tabular}
\end{center}
%
Here one could write a message such as:
\begin{center}
|This is the part \childdocname{} of \childdocjob{}.|
\end{center}

%%%%%%%%%%%%%%%%%%%%%%%%%%%%%%%%%%%%%%%%%%%%%%%%%%%%%%%%%%%%%%%%%%%%%%%%%%%%%%%%
\subsection{Flags}
\label{sec:flags}

The package makes it easy to generate different versions
of the main or child documents.
To this end compilation flags can be defined
and assigned different default values.
They will be particularly useful in conjunction
with the forwarding mechanism described in \secref{sec:forward}.

For example, it may be useful to have a flag |\version|
which can be set to |draft| or |final|.
The document source will contain some conditional code
depending on the value of |\version|.
Suppose further, the flag should default to |final| for the main file
and to |draft| for child files
which is a natural assignment for editing the document.
This is achieved by placing the following code
in the preamble of the main document
(below the |\childdocmain| directive):
%
\begin{center}
\begin{tabular}{l}
|\ifchilddoc|\\
|\providecommand{\version}{draft}|\\
|\||else|\\
|\providecommand{\version}{final}|\\
|\||fi|
\end{tabular}
\end{center}
%
The definition by |\providecommand| makes sure
that previous definitions are not overwritten.
Further statements |\providecommand{\version}{...}|
can thus be added before the above code to override it.

For the main file, one might add a line
(between |\childdocmain| and the above block)
%
\begin{center}
|%\ifchilddoc\||else\providecommand{\version}{draft}\||fi|
\end{center}
%
which can be uncommented to produce a draft version.
Likewise one can add a line to the very top of a child file
(above the |\childdocof{|\textit{main}|}| directive)
%
\begin{center}
|%\providecommand{\version}{final}|
\end{center}
%
which can be uncommented to produce the final version of this child document.

%%%%%%%%%%%%%%%%%%%%%%%%%%%%%%%%%%%%%%%%%%%%%%%%%%%%%%%%%%%%%%%%%%%%%%%%%%%%%%%%
\subsection{Forwarding}
\label{sec:forward}

Different versions of the main or child documents
using compilation flags as described in \secref{sec:flags}
can be (permanently) stored in different files
for convenient compilation, viewing and distribution.
To this end, the package defines a command
to pass on compilation to a different file:

%%%%%%%%%%%%%%%%%%%%%%%%%%%%%%%%%%%%%%%%
\DescribeMacro{\childdocforward}
The command |\childdocforward| redirects processing to
another source file:
%
\begin{center}
\begin{tabular}{l}
|\input{childdoc.def}|\\
|\childdocforward[|\textit{main}|]{|\textit{dest}|}|\\
\end{tabular}
\end{center}
%
The argument \textit{dest} is the destination file
(without extension).
It should be the main file or one of the child files.
Note that further \textsf{childdoc} directives
such as |\childdocof| and |\childdocforward|
in the indicated file will be processed in this form.
The optional argument \textit{main}
passes on directly to the main file \textit{main}
while pretending to compile the child \textit{dest}.
This form behaves as if \textit{dest}
issues |\childdocof{|\textit{main}|}| right away,
and no further \textsf{childdoc} directives will be processed.

%%%%%%%%%%%%%%%%%%%%%%%%%%%%%%%%%%%%%%%%
\DescribeMacro{\...prefix}
In the alternative form |\childdocforwardprefix|,
%
\begin{center}
\begin{tabular}{l}
|\input{childdoc.def}|\\
|\childdocforwardprefix[|\textit{main}|]{|\textit{prefix}|}{|\textit{dest}|}|
\end{tabular}
\end{center}
%
the destination file is determined by a pattern
depending on the current file:
To make this work, the current file must be called
`{\textit{prefix}\hspace{0.2em}\textit{suffix}}'
with \textit{prefix} matching precisely the argument.
Processing is then passed on to the file
`{\textit{dest}\hspace{0.2em}\textit{suffix}}'.
Surely, the same effect is achieved by
directly specifying the
argument `{\textit{dest}\hspace{0.2em}\textit{suffix}}'
in the first form.
However, that requires to set up a different file
for each child. With the alternative form of the command
all these files can have exactly the same content
which simplifies setting them up and maintaining them.

For example, the following file |draft.tex|
with a compilation flag |\version| as described in \secref{sec:flags}
compiles the main document as a draft:
%
\begin{center}
\begin{tabular}{l}
|\def\version{draft}|\\
|\input{childdoc.def}|\\
|\childdocforward{|\textit{main}|}|
\end{tabular}
\end{center}
%
Likewise, the following files |final|\textit{nn}|.tex|
compile the final version of the child document
|child|\textit{nn}|.tex|:
%
\begin{center}
\begin{tabular}{l}
|\def\version{final}|\\
|\input{childdoc.def}|\\
|\childdocforwardprefix{final}{child}|
\end{tabular}
\end{center}
%

Note that when several versions of a main file and/or of each child file
are to be generated, it may be convenient to set up a |Makefile| or
shell script to automatise the process.

%%%%%%%%%%%%%%%%%%%%%%%%%%%%%%%%%%%%%%%%%%%%%%%%%%%%%%%%%%%%%%%%%%%%%%%%%%%%%%%%
\subsection{Command Line Processing}
\label{sec:commandline}

The effect of redirection files can also be achieved by invoking
the \LaTeX{} compiler with a more elaborate command line.
Most conveniently this should be done as part
of a shell script or a |Makefile|.

When using \textsf{childdoc} in the main file, the following
command lines effectively perform a redirection
(note that depending on the shell being used,
backslashes may have to be doubled: `|\|' $\to$ `|\\|'):
%
\begin{center}
|... -jobname "|\textit{target}|" |\\|"|[\textit{flags}]%
|\input{childdoc.def}\childdocforward[|\textit{main}|]{|\textit{dest}|}"|
\end{center}
%
Here \textit{target} is the name of the output file,
\textit{main} is the name of the main file
and \textit{dest} is the name of the main or child file to be processed
(all filenames without extensions).
The optional argument \textit{main} can be omitted
if \textit{main} matches \textit{dest}.
Optionally, compilation \textit{flags} can be defined via |\def| commands.
This command line makes the \TeX{} engine believe
it is compiling the file \textit{target}
whose content is specified as the latter parameter.
The provided code then forwards the processing to
\textit{main} or \textit{dest} as described in \secref{sec:forward}.

%%%%%%%%%%%%%%%%%%%%%%%%%%%%%%%%%%%%%%%%%%%%%%%%%%%%%%%%%%%%%%%%%%%%%%%%%%%%%%%%
\subsection{Include by Input}
\label{sec:input}

Including child documents by |\include| has some restrictions by design.
Most notably, the content of a child document always occupies
its own set of pages; pages cannot be shared between child documents.
Usually, this behaviour makes perfect sense
because each child document contain an essential part of the document.
However, in some situations it may be desirable to compose
a document from a collection of parts
without having mandatory page breaks between then.
For this case, the package
provides a mechanism to include parts
by |\input| which can also be processed individually.
However, by construction this mechanism
requires manual handling of the content to be output.

%%%%%%%%%%%%%%%%%%%%%%%%%%%%%%%%%%%%%%%%
\DescribeMacro{\ifchilddocmanual}
The main file should be prepared as usual, see \secref{sec:include}.
However, the document body must make a distinction
between processing of an individual part and of the main document, e.g.:
%
\begin{center}
\begin{tabular}{l}
|\ifchilddocmanual|\\
|\input{\childdocname}|\\
|\||else|\\
\textit{document body with }|\input{|\textit{part}|}|\\
|\||fi|
\end{tabular}
\end{center}
%
The conditional |\ifchilddocmanual| is true whenever
a part to be included by |\input| is being compiled,
and the name of the part is stored in |\childdocname|.

%%%%%%%%%%%%%%%%%%%%%%%%%%%%%%%%%%%%%%%%
\DescribeMacro{\childdocby}
Each part to be included by |\input| should start with:
%
\begin{center}
\begin{tabular}{l}
|\input{childdoc.def}|\\
|\childdocby{|\textit{main}|}|\\
\end{tabular}
\end{center}
%
The directive |\childdocby| is similar to |\childdocof|
described in \secref{sec:include},
but the subsequent selection of content must be done manually.
To that end, both |\ifchilddoc| and |\ifchilddocmanual|
will be true upon processing of a part,
and the name of the part is stored in |\childdocname|.
Note that |\jobname| will be set to the filename of the current part
so that each part receives an individual |.aux| file
that does not interfere with the |.aux| file(s) of the main document.
This behaviour can be altered by the alternative form
|\childdocby[*]{|\textit{main}|}| (with a non-empty optional argument)
which uses the |.aux| file of the main document
by setting |\jobname| to \textit{main}.

%%%%%%%%%%%%%%%%%%%%%%%%%%%%%%%%%%%%%%%%%%%%%%%%%%%%%%%%%%%%%%%%%%%%%%%%%%%%%%%%
\subsection{Driver Development}
\label{sec:driver}

The \textsf{childdoc} mechanism can also be use for the development
of definition files such as \LaTeX{} styles or classes.
This case differs from the above setup with multiple parts
included by |\include| in that no |\includeonly| should be invoked.
This can be achieved by starting the include file
(before |\ProvidesPackage|) with:
%
\begin{center}
\begin{tabular}{l}
|\input{childdoc.def}|\\
|\childdocforward{|\textit{main}|}|\\
\end{tabular}
\end{center}
%
or alternatively with:
%
\begin{center}
\begin{tabular}{l}
|\input{childdoc.def}|\\
|\childdocby{|\textit{main}|}|\\
\end{tabular}
\end{center}
%
Both forms have slightly different effects as described above.
The main file is prepared as usual, see \secref{sec:include}.

%%%%%%%%%%%%%%%%%%%%%%%%%%%%%%%%%%%%%%%%%%%%%%%%%%%%%%%%%%%%%%%%%%%%%%%%%%%%%%%%
\subsection{Legacy Detection}
\label{sec:detection}

The directive |\childdocmain| in the main file can detect
whether the complete document or merely a child is to be compiled
even without using the directive |\childdocof|.
This method is deprecated because it is less robust
and there is no compelling reason to use it;
it is merely provided for backward compatibility
and it may be removed in future versions.

If the detection mechanism is to be used,
it is mandatory to correctly specify
the filename of the main file as the argument of |\childdocmain|:
%
\begin{center}
\begin{tabular}{l}
|\input{childdoc.def}|\\
|\childdocmain{|\textit{main}|}|\\
\end{tabular}
\end{center}
%
If |\jobname| does not match the argument \textit{main} of |\childdocmain|,
it is assumed that |\jobname| points to the child file to be compiled.
When using |\childdocmain| with the main file specified as argument,
it suffices to start a child file
with just |\input{|\textit{main}|}|
without loading of the package and using |\childdocof|.
If instead all processing is done
with the appropriate \textsf{childdoc} directives,
the argument of \textit{main} of |\childdocmain| can be empty.

An alternative version of the command line processing described
in \secref{sec:commandline} using the detection mechanism reads:
%
\begin{center}
|... -jobname "|\textit{target}|" "|[\textit{flags}]%
[|\def\jobname{|\textit{dest}|}|]|\input{|\textit{main}|}"|
\end{center}

%%%%%%%%%%%%%%%%%%%%%%%%%%%%%%%%%%%%%%%%%%%%%%%%%%%%%%%%%%%%%%%%%%%%%%%%%%%%%%%%
\subsection{Manual Code}
\label{sec:manual}

In case one cannot be certain whether the definitions file |childdoc.def|
is installed on the target \TeX{} distribution
and one prefers not to ship it,
it is conceivable to paste a few relevant commands into the sources.

To that end, drop all statements |\input{childdoc.def}|
and perform the replacements as outlined below.
Instead of |\childdocmain{|\textit{main}|}| add the following code
to the top of the main file:
%
\begin{center}
\begin{tabular}{l}
|\||ifdefined\childdocname\endinput\||fi\newif\ifchilddoc|\\
|\edef\childdocname{\scantokens\expandafter{\jobname\noexpand}}|\\
|\def\childdocmain{|\textit{main}|}\||ifx\childdocmain\childdocname\||else|\\
|\childdoctrue\includeonly{\childdocname}\let\jobname\childdocmain\||fi|\\
\end{tabular}
\end{center}
%
Instead of |\childdocof{|\textit{main}|}| just include the main file
at the top of each child file:
%
\begin{center}
|\input{|\textit{main}|}|
\end{center}
%
A simple redirection |\childdocforward{|\textit{dest}|}| is achieved by:
%
\begin{center}
|\def\jobname{|\textit{dest}|}\input{\jobname}|
\end{center}
%
The redirection with prefix
|\childdocforwardprefix[|\textit{prefix}|]{|\textit{dest}|}|
is accomplished by:
%
\begin{center}
\begin{tabular}{l}
|{\edef\jobname{\scantokens\expandafter{\jobname\noexpand}}|\\
|\def\redirectjob |\textit{prefix}|#1~~~{\gdef\jobname{|\textit{dest}|#1}}|\\
|\expandafter\redirectjob\jobname~~~}\input{\jobname}|
\end{tabular}
\end{center}

In an alternative approach,
child documents can be compiled by a specific command line
without additional code or specific definitions:
%
\begin{center}
|... -jobname "|\textit{target}|" "|[\textit{flags}]%
|\includeonly{|\textit{dest}|}\input{|\textit{main}|}"|
\end{center}
%

%%%%%%%%%%%%%%%%%%%%%%%%%%%%%%%%%%%%%%%%%%%%%%%%%%%%%%%%%%%%%%%%%%%%%%%%%%%%%%%%
%%%%%%%%%%%%%%%%%%%%%%%%%%%%%%%%%%%%%%%%%%%%%%%%%%%%%%%%%%%%%%%%%%%%%%%%%%%%%%%%
\section{Information}

%%%%%%%%%%%%%%%%%%%%%%%%%%%%%%%%%%%%%%%%%%%%%%%%%%%%%%%%%%%%%%%%%%%%%%%%%%%%%%%%
\subsection{Copyright}

Copyright \copyright{} 2017--2018 Niklas Beisert

This work may be distributed and/or modified under the
conditions of the \LaTeX{} Project Public License, either version 1.3
of this license or (at your option) any later version.
The latest version of this license is in
  \url{http://www.latex-project.org/lppl.txt}
and version 1.3 or later is part of all distributions of \LaTeX{}
version 2005/12/01 or later.

This work has the LPPL maintenance status `maintained'.

The Current Maintainer of this work is Niklas Beisert.

This work consists of the files |README.txt|, |childdoc.ins| and |childdoc.dtx|
as well as the derived files |childdoc.def|, |cdocsamp.tex|
with |cdocsch1.tex|, |cdocsch2.tex|, |cdocspt3.tex|, |cdocspt4.tex|,
|cdocsdrf.tex|, |cdocsfn1.tex|, |cdocsfn2.tex|
as well as |childdoc.pdf|.

%%%%%%%%%%%%%%%%%%%%%%%%%%%%%%%%%%%%%%%%%%%%%%%%%%%%%%%%%%%%%%%%%%%%%%%%%%%%%%%%
\subsection{Files and Installation}

The package consists of the files:
%
\begin{center}
\begin{tabular}{ll}
    |README.txt|   & readme file \\
    |childdoc.ins| & installation file \\
    |childdoc.dtx| & source file \\
    |childdoc.def| & definition file \\
    |cdocsamp.tex| & sample main file \\
    |cdocsch1.tex| & sample include file \\
    |cdocsch2.tex| & sample include file \\
    |cdocspt3.tex| & sample part file \\
    |cdocspt4.tex| & sample part file \\
    |cdocsdrf.tex| & sample redirection file \\
    |cdocsfn1.tex| & sample redirection file \\
    |cdocsfn2.tex| & sample redirection file \\
    |childdoc.pdf| & manual
\end{tabular}
\end{center}
%
The distribution consists of the files
|README.txt|, |childdoc.ins| and |childdoc.dtx|.
%
\begin{itemize}
\item
Run (pdf)\LaTeX{} on |childdoc.dtx|
to compile the manual |childdoc.pdf| (this file).
\item
Run \LaTeX{} on |childdoc.ins| to create the definitions file |childdoc.def|
and the sample |cdocsamp.tex| with include files
|cdocsch1.tex|, |cdocsch2.tex|, |cdocspt3.tex|, |cdocspt4.tex|,
|cdocsdrf.tex|, |cdocsfn1.tex|, |cdocsfn2.tex|.
Then copy the file |childdoc.def| to an appropriate directory of your \LaTeX{}
distribution, e.g.\ \textit{texmf-root}|/tex/latex/childdoc|.
\end{itemize}

%%%%%%%%%%%%%%%%%%%%%%%%%%%%%%%%%%%%%%%%%%%%%%%%%%%%%%%%%%%%%%%%%%%%%%%%%%%%%%%%
\subsection{Related CTAN Packages}

There are several other packages which offer a similar functionality:
%
\begin{itemize}
\item
The packages
\href{http://ctan.org/pkg/docmute}{\textsf{docmute}},
\href{http://ctan.org/pkg/includex}{\textsf{includex}} and
\href{http://ctan.org/pkg/standalone}{\textsf{standalone}}
provide commands to include only the document body of
a child file thus allowing both files to be compiled individually.
\item
The packages \href{http://ctan.org/pkg/subdocs}{\textsf{subdocs}}
and \href{http://ctan.org/pkg/subfiles}{\textsf{subfiles}}
provide structures in which the main and child documents can be
encapsulated and allowing them to be compiled individually.
The inclusion mechanism is different from the conventional |\include|.
\item
The package \href{http://ctan.org/pkg/combine}{\textsf{combine}}
is an elaborate solution to combine several documents into one.
\end{itemize}
%
See also the CTAN topic \href{http://ctan.org/topic/subdocs}{\textsf{subdocs}}
for further related packages.
The present package differs from the above solutions in that
a document structure constructed with the conventional |\include| mechanism
just needs two extra commands at the top of every file
such that all constituent files can be compiled individually.

%%%%%%%%%%%%%%%%%%%%%%%%%%%%%%%%%%%%%%%%%%%%%%%%%%%%%%%%%%%%%%%%%%%%%%%%%%%%%%%%
%\subsection{Feature Suggestions}
%
%The following is a list of features which may be useful for future
%versions of this package:
%%
%\begin{itemize}
%\item
%\ldots
%\end{itemize}

%%%%%%%%%%%%%%%%%%%%%%%%%%%%%%%%%%%%%%%%%%%%%%%%%%%%%%%%%%%%%%%%%%%%%%%%%%%%%%%%
\subsection{Revision History}

%%%%%%%%%%%%%%%%%%%%%%%%%%%%%%%%%%%%%%%%
\paragraph{v2.0:} 2018/12/30

\begin{itemize}
\item
immediate forward processing
\item
added |\childdocby| mechanism
\item
manual restructured
\end{itemize}

%%%%%%%%%%%%%%%%%%%%%%%%%%%%%%%%%%%%%%%%
\paragraph{v1.6:} 2018/01/17

\begin{itemize}
\item
application for development of include files
\item
corrections to manual
\end{itemize}

%%%%%%%%%%%%%%%%%%%%%%%%%%%%%%%%%%%%%%%%
\paragraph{v1.5:} 2017/05/21

\begin{itemize}
\item
more complete structuring introduced
\item
|\childdocof| introduced
\item
|\childdoc| renamed to |\childdocmain|
\item
|\childredirect| renamed to |\childdocforward| and |\childdocforwardprefix|
and functionality expanded
\end{itemize}

%%%%%%%%%%%%%%%%%%%%%%%%%%%%%%%%%%%%%%%%
\paragraph{v1.0:} 2017/04/27

\begin{itemize}
\item
manual and install package
\item
first version published on CTAN
\end{itemize}

%%%%%%%%%%%%%%%%%%%%%%%%%%%%%%%%%%%%%%%%
\paragraph{v0.6:} 2017/04/26

\begin{itemize}
\item
redirection mechanism added
\end{itemize}

%%%%%%%%%%%%%%%%%%%%%%%%%%%%%%%%%%%%%%%%
\paragraph{v0.5:} 2017/04/26

\begin{itemize}
\item
functionality in definition file
\end{itemize}


%%%%%%%%%%%%%%%%%%%%%%%%%%%%%%%%%%%%%%%%%%%%%%%%%%%%%%%%%%%%%%%%%%%%%%%%%%%%%%%%
%%%%%%%%%%%%%%%%%%%%%%%%%%%%%%%%%%%%%%%%%%%%%%%%%%%%%%%%%%%%%%%%%%%%%%%%%%%%%%%%
%%%%%%%%%%%%%%%%%%%%%%%%%%%%%%%%%%%%%%%%%%%%%%%%%%%%%%%%%%%%%%%%%%%%%%%%%%%%%%%%
\appendix

\settowidth\MacroIndent{\rmfamily\scriptsize 000\ }

 \DocInput{childdoc.dtx}

\end{document}
%</driver>
% \fi
%
% %%%%%%%%%%%%%%%%%%%%%%%%%%%%%%%%%%%%%%%%%%%%%%%%%%%%%%%%%%%%%%%%%%%%%%%%%%%%%%
% %%%%%%%%%%%%%%%%%%%%%%%%%%%%%%%%%%%%%%%%%%%%%%%%%%%%%%%%%%%%%%%%%%%%%%%%%%%%%%
% \section{Sample}
%\iffalse
%<*samplemain>
%\fi
%
% The following presents a sample document
% with two chapters, two parts, a title page,
% a compile flag as well as three forwarding files to set the flag.
% It consists of eight |.tex| files:
% \begin{center}
% \begin{tabular}{ll}
% |cdocsamp.tex|&main file\\
% |cdocsch1.tex|&include file for chapter 1\\
% |cdocsch2.tex|&include file for chapter 2\\
% |cdocspt3.tex|&include file for part 3\\
% |cdocspt4.tex|&include file for part 4\\
% |cdocsdrf.tex|&forwarding file for main file in draft mode\\
% |cdocsfi1.tex|&forwarding file for final version of chapter 1\\
% |cdocsfi2.tex|&forwarding file for final version of chapter 2\\
% \end{tabular}
% \end{center}
% Each of the eight files can be compiled directly by the \LaTeX{} compiler.
%
% %%%%%%%%%%%%%%%%%%%%%%%%%%%%%%%%%%%%%%
% \paragraph{Main File.}
%
% The main file is called |cdocsamp.tex|.
%
% Load the \textsf{childdoc} definitions and
% declare the filename for the main document:
%    \begin{macrocode}
\input{childdoc.def}
\childdocmain{}
%    \end{macrocode}

% Optional override for |\version| flag:
%    \begin{macrocode}
%%\ifchilddoc\else\providecommand{\version}{draft}\fi
%    \end{macrocode}

% Define the default values for the |\version| flag
% (|final| for the main file and |draft| for childs):
%    \begin{macrocode}
\ifchilddoc
\providecommand{\version}{draft}
\else
\providecommand{\version}{final}
\fi
%    \end{macrocode}

% Load the standard document class:
%    \begin{macrocode}
\documentclass[12pt]{article}
%    \end{macrocode}

% Start the document body:
%    \begin{macrocode}
\begin{document}
%    \end{macrocode}

% Declare a title page.
% Print title, part of document being processed and version flag:
%    \begin{macrocode}
\addtocounter{page}{-1}
\begin{center}
{\LARGE\bfseries{}childdoc example\par}
\vspace{1cm}
\ifchilddoc
\ifchilddocmanual part\else chapter\fi:
`\childdocname' of `\childdocjob'\par
\else
main document: `\childdocjob'\par
\fi
version: \version\par
\end{center}
\newpage
%    \end{macrocode}

% Manually include selected file,
% otherwise process as usual:
%    \begin{macrocode}
\ifchilddocmanual
\section*{part `\childdocname'}
\input{\childdocname}
\else
%    \end{macrocode}

% Include the two chapters:
%    \begin{macrocode}
\include{cdocsch1}
\include{cdocsch2}
%    \end{macrocode}

% Include the two parts unless only chapters should be displayed:
%    \begin{macrocode}
\ifchilddoc\else
\section{part three}
\input{cdocspt3}
\section{part four}
\input{cdocspt4}
\fi
%    \end{macrocode}

% Process as usual until here:
%    \begin{macrocode}
\fi
%    \end{macrocode}

% End of document body:
%    \begin{macrocode}
\end{document}
%    \end{macrocode}
%\iffalse
%</samplemain>
%\fi
%
% %%%%%%%%%%%%%%%%%%%%%%%%%%%%%%%%%%%%%%
% \paragraph{Chapter Include Files.}
%
% The include files are called |cdocsch1.tex| and |cdocsch2.tex|.
%
%\iffalse
%<*samplechap1|samplechap2>
%\fi

% Optional override for |\version| flag:
%    \begin{macrocode}
%%\providecommand{\version}{final}
%    \end{macrocode}

% Include the main document:
%    \begin{macrocode}
\input{childdoc.def}
\childdocof{cdocsamp}
%    \end{macrocode}

%\iffalse
%</samplechap1|samplechap2>
%\fi
%
%\iffalse
%<*samplechap1>
%\fi
% Some text for chapter 1:
%    \begin{macrocode}
\section{one}
some text in chapter one
%    \end{macrocode}

%\iffalse
%</samplechap1>
%\fi
% Some text for chapter 2:
%\iffalse
%<*samplechap2>
%\fi
%    \begin{macrocode}
\section{two}
more text in chapter two
%    \end{macrocode}

%\iffalse
%</samplechap2>
%\fi
%
% %%%%%%%%%%%%%%%%%%%%%%%%%%%%%%%%%%%%%%
% \paragraph{Part Include Files.}
%
% The include files are called |cdocspt3.tex| and |cdocspt4.tex|.
%
%\iffalse
%<*samplepart3|samplepart4>
%\fi

% Optional override for |\version| flag:
%    \begin{macrocode}
%%\providecommand{\version}{final}
%    \end{macrocode}

% Include the main document:
%    \begin{macrocode}
\input{childdoc.def}
\childdocby{cdocsamp}
%    \end{macrocode}

%\iffalse
%</samplepart3|samplepart4>
%\fi
%
%\iffalse
%<*samplepart3>
%\fi
% Some text for part 3:
%    \begin{macrocode}
some text in part three
%    \end{macrocode}

%\iffalse
%</samplepart3>
%\fi
% Some text for part 4:
%\iffalse
%<*samplepart4>
%\fi
%    \begin{macrocode}
more text in part four
%    \end{macrocode}

%\iffalse
%</samplepart4>
%\fi
%
% %%%%%%%%%%%%%%%%%%%%%%%%%%%%%%%%%%%%%%
% \paragraph{Forwarding for a Complete Draft.}
%
% The following forwarding file |cdocsdrf.tex|
% compiles the main document in draft mode:
%\iffalse
%<*sampledraft>
%\fi
%    \begin{macrocode}
\def\version{draft}
\input{childdoc.def}
\childdocforward{cdocsamp}
%    \end{macrocode}

%\iffalse
%</sampledraft>
%\fi
%
% %%%%%%%%%%%%%%%%%%%%%%%%%%%%%%%%%%%%%%
% \paragraph{Forwarding for Final Version of the Chapters.}
%
% The following forwarding files |cdocsfn1.tex| and |cdocsfn2.tex|
% (with identical content)
% compile the final versions of the child documents
% |cdocsch1.tex| and |cdocsch2.tex|, respectively:
%\iffalse
%<*samplefinal>
%\fi
%    \begin{macrocode}
\def\version{final}
\input{childdoc.def}
\childdocforwardprefix[cdocsamp]{cdocsfn}{cdocsch}
%    \end{macrocode}

%\iffalse
%</samplefinal>
%\fi
%
% %%%%%%%%%%%%%%%%%%%%%%%%%%%%%%%%%%%%%%
% \paragraph{Command Line Processing.}
%
% The following three command lines generate the output files
% |cdocscld|, |cdocscl1| and |cdocscl2|
% which should be identical to
% |cdocsdrf|, |cdocsch1| and |cdocsfn2|, respectively:
% \begin{center}
% \begin{tabular}{l}
% |latex -jobname cdocscld \|\\
% |  "\def\version{draft}\input{childdoc.def}\childdocforward{cdocsamp}"|\\
% |latex -jobname cdocscl1 \|\\
% |  "\input{childdoc.def}\childdocforward[cdocsamp]{cdocsch1}"|\\
% |latex -jobname cdocscl2 \|\\
% |  "\def\version{final}\input{childdoc.def}\childdocforward{cdocsch2}"|
% \end{tabular}
% \end{center}
% Note that the trailing backslash on each first line
% merely continues the input to the second line
% (for convenient cut ant paste).
% Furthermore, the command |latex| can be replaced by any
% of its alternative versions such as |pdflatex|.
%
% %%%%%%%%%%%%%%%%%%%%%%%%%%%%%%%%%%%%%%%%%%%%%%%%%%%%%%%%%%%%%%%%%%%%%%%%%%%%%%
% %%%%%%%%%%%%%%%%%%%%%%%%%%%%%%%%%%%%%%%%%%%%%%%%%%%%%%%%%%%%%%%%%%%%%%%%%%%%%%
% \section{Implementation}
%\iffalse
%<*package>
%\fi
%
% This section describes the definitions file |childdoc.def|.

% The definitions cannot be loaded using |\usepackage| or |\RequirePackage|
% which has a mechanism to prevent loading a style file more than once.
% When loading the definitions by means of |\input|
% multiple instances have to be prevented manually:
%\iffalse
%This code needs to be before the `\ProvidesFile' directive
%which is defined at the beginning of this file.
%Therefore it is also placed there and commented out here.
%</package>
%<*discard>
%\fi
%    \begin{macrocode}
\ifdefined\childdocmain\endinput\fi
%    \end{macrocode}
%\iffalse
%</discard>
%<*package>
%\fi
%
% \macro{\ifchilddoc}
% \macro{\ifchilddocmanual}
% The conditional |\ifchilddoc| tells whether a
% child (true) or main (false) document is being compiled.
% The conditional |\ifchilddocmanual| tells whether
% the |\includeonly| mechanism is used (false) or
% the selection of child files must be performed manually (true).
% The definitions initialise to false:
%    \begin{macrocode}
\newif\ifchilddoc
\newif\ifchilddocmanual
%    \end{macrocode}

% \macro{\childdocname}
% \macro{\childdocjob}
% The macro |\childdocname| stores the name of the main document
% to be compiled. The macro |\childdocjob| stores the name of
% the document on which the \LaTeX{} compiler was originally invoked.
% The content of |\jobname| cannot be compared
% to filenames specified in the source due to different catcodes.
% The following code rescans |\jobname|, stores the result
% in |\childdocname| and saves a copy in |\childdocjob|:
%    \begin{macrocode}
\edef\childdocname{\scantokens\expandafter{\jobname\noexpand}}
\let\childdocjob\childdocname
%    \end{macrocode}

% \macro{\childdocdisable}
% The macro |\childdocdisable| prevents the main file
% from being processed more than once.
% At this stage, the main document command |\childdocmain|
% is assumed to be called once again where it should do nothing.
% Any subsequent call to it should prevent
% a secondary processing of the main document
% It overwrites the forwarding commands
% |\childdocof| and |\childdocforward|
% with empty macros to prevent further inclusions of the main document:
%    \begin{macrocode}
\newcommand{\childdocdisable}
{
  \renewcommand{\childdocmain}[1]{\renewcommand{\childdocmain}[1]{\endinput}}
  \renewcommand{\childdocof}[1]{}
  \renewcommand{\childdocby}[2][]{}
  \renewcommand{\childdocforward}[2][]{}
  \renewcommand{\childdocdisable}{}
}
%    \end{macrocode}

% \macro{\childdocmain}
% The macro |\childdocmain| is to be called at the top of the main file
% with nothing or the main filename (without extension) as argument.
% First, it breaks loops.
% If the argument is not empty and does not match |\childdocname|
% (which is set by the first inclusion of |childdoc.def|),
% |\ifchilddoc| is set to true, |\includeonly| is applied to the child file
% and |\jobname| is set to the main file
% (for proper handling of |.aux| files):
%    \begin{macrocode}
\newcommand{\childdocmain}[1]
{
  \childdocdisable\childdocmain{}
  \if?#1?\else
    \begingroup
      \def\childdoctmp{#1}
      \ifx\childdoctmp\childdocname
        \def\childdoctmp{}
      \else
        \def\childdoctmp
        {
          \childdoctrue
          \includeonly{\childdocname}
          \def\childdocjob{#1}
          \def\jobname{#1}
        }
      \fi
      \expandafter
    \endgroup
    \childdoctmp
  \fi
}
%    \end{macrocode}

% \macro{\childdocof}
% The command |\childdocof| redirects
% compilation to the main file |#1|.
%    \begin{macrocode}
\newcommand{\childdocof}[1]
{
  \childdocdisable
  \childdoctrue
  \includeonly{\childdocname}
  \def\jobname{#1}
  \def\childdocjob{#1}
  \input{#1}
}
%    \end{macrocode}

% \macro{\childdocby}
% The command |\childdocby| ....
%    \begin{macrocode}
\newcommand{\childdocby}[2][]
{
  \childdocdisable
  \childdoctrue
  \childdocmanualtrue
  \if?#1?\else
    \def\jobname{#2}
  \fi
  \def\childdocjob{#2}
  \input{#2}
  \endinput
}
%    \end{macrocode}

% \macro{\childdocforward}
% The command |\childdocforward| redirects
% compilation to the main file or
% (if the optional argument is given) a child file.
% Parameters are set as if the main file
% or a child file starting with |\childdocof| was compiled.
% Then compilation is handed over to the main file:
%    \begin{macrocode}
\newcommand{\childdocforward}[2][]
{
  \begingroup
    \if?#1?
      \def\childdoctmp
      {
        \def\childdocname{#2}
        \def\childdocjob{#2}
        \def\jobname{#2}
        \input{#2}
        \endinput
      }
    \else
      \def\childdoctmp
      {
        \childdocdisable
        \def\childdocname{#2}
        \childdoctrue
        \includeonly{#2}
        \def\childdocjob{#1}
        \def\jobname{#1}
        \input{#1}
        \endinput
      }
    \fi
    \expandafter
  \endgroup
  \childdoctmp
}
%    \end{macrocode}

% \macro{\childdocforwardprefix}
% The command |\childdocforwardprefix| redirects
% compilation to the main or a child file by means of a pattern.
% The prefix |#1| in the current filename is replaced by |#2|
% and the suffix of the current filename is kept
% (it is assumed that the filename does not contain the substring `|~~~|'
% which is used as a delimiter).
% Compilation is handed over to the new file by |\childdocforward|:
%    \begin{macrocode}
\newcommand{\childdocforwardprefix}[3][]
{
  \begingroup
    \def\childdocextract #2##1~~~{\def\childdoctmp{\childdocforward[#1]{#3##1}}}
    \expandafter\childdocextract\childdocname~~~
    \expandafter
  \endgroup
  \childdoctmp
}
%    \end{macrocode}

% \macro{\childdoc}
% The deprecated macro |\childdoc| is a legacy version of |\childdocmain|:
%    \begin{macrocode}
\newcommand{\childdoc}{\childdocmain}
%    \end{macrocode}

% \macro{\childdocredirect}
% The deprecated macro |\childdocredirect| is a legacy version
% of |\childdocforward| and |\childdocforwardprefix|:
%    \begin{macrocode}
\newcommand{\childdocredirect}[2][]
{
  \begingroup
    \if?#1?
      \def\childdoctmp{\childdocforward{#2}}
    \else
      \def\childdoctmp{\childdocforwardprefix{#1}{#2}}
    \fi
    \expandafter
  \endgroup
  \childdoctmp
}
%    \end{macrocode}

%\iffalse
%</package>
%\fi
%
\endinput

\childdocmain{}
%    \end{macrocode}

% Optional override for |\version| flag:
%    \begin{macrocode}
%%\ifchilddoc\else\providecommand{\version}{draft}\fi
%    \end{macrocode}

% Define the default values for the |\version| flag
% (|final| for the main file and |draft| for childs):
%    \begin{macrocode}
\ifchilddoc
\providecommand{\version}{draft}
\else
\providecommand{\version}{final}
\fi
%    \end{macrocode}

% Load the standard document class:
%    \begin{macrocode}
\documentclass[12pt]{article}
%    \end{macrocode}

% Start the document body:
%    \begin{macrocode}
\begin{document}
%    \end{macrocode}

% Declare a title page.
% Print title, part of document being processed and version flag:
%    \begin{macrocode}
\addtocounter{page}{-1}
\begin{center}
{\LARGE\bfseries{}childdoc example\par}
\vspace{1cm}
\ifchilddoc
\ifchilddocmanual part\else chapter\fi:
`\childdocname' of `\childdocjob'\par
\else
main document: `\childdocjob'\par
\fi
version: \version\par
\end{center}
\newpage
%    \end{macrocode}

% Manually include selected file,
% otherwise process as usual:
%    \begin{macrocode}
\ifchilddocmanual
\section*{part `\childdocname'}
\input{\childdocname}
\else
%    \end{macrocode}

% Include the two chapters:
%    \begin{macrocode}
\include{cdocsch1}
\include{cdocsch2}
%    \end{macrocode}

% Include the two parts unless only chapters should be displayed:
%    \begin{macrocode}
\ifchilddoc\else
\section{part three}
\input{cdocspt3}
\section{part four}
\input{cdocspt4}
\fi
%    \end{macrocode}

% Process as usual until here:
%    \begin{macrocode}
\fi
%    \end{macrocode}

% End of document body:
%    \begin{macrocode}
\end{document}
%    \end{macrocode}
%\iffalse
%</samplemain>
%\fi
%
% %%%%%%%%%%%%%%%%%%%%%%%%%%%%%%%%%%%%%%
% \paragraph{Chapter Include Files.}
%
% The include files are called |cdocsch1.tex| and |cdocsch2.tex|.
%
%\iffalse
%<*samplechap1|samplechap2>
%\fi

% Optional override for |\version| flag:
%    \begin{macrocode}
%%\providecommand{\version}{final}
%    \end{macrocode}

% Include the main document:
%    \begin{macrocode}
% \iffalse
%
% childdoc.dtx Copyright (C) 2017-2018 Niklas Beisert
%
% This work may be distributed and/or modified under the
% conditions of the LaTeX Project Public License, either version 1.3
% of this license or (at your option) any later version.
% The latest version of this license is in
%   http://www.latex-project.org/lppl.txt
% and version 1.3 or later is part of all distributions of LaTeX
% version 2005/12/01 or later.
%
% This work has the LPPL maintenance status `maintained'.
%
% The Current Maintainer of this work is Niklas Beisert.
%
% This work consists of the files childdoc.dtx and childdoc.ins
% and the derived files childdoc.def and cdocsamp.tex with
% cdocsch1.tex, cdocsch2.tex, cdocsdrf.tex, cdocsfn1.tex, cdocsfn2.tex.
%
%<package>\ifdefined\childdocmain\endinput\fi
%<package>\ProvidesFile{childdoc.def}[2018/12/30 v2.0 child document driver]
%<samplemain>\ProvidesFile{cdocsamp.tex}[2018/12/30 v2.0 sample for childdoc]
%<*driver>
%\ProvidesFile{childdoc.drv}[2018/12/30 v2.0 childdoc reference manual file]
\PassOptionsToClass{10pt,a4paper}{article}
\documentclass{ltxdoc}

\usepackage[margin=35mm]{geometry}
\usepackage{hyperref}
\usepackage{hyperxmp}
\usepackage[usenames]{color}

\hypersetup{colorlinks=true}
\hypersetup{pdfstartview=FitH}
\hypersetup{pdfpagemode=UseNone}
\hypersetup{pdfsource={}}
\hypersetup{pdflang={en-UK}}
\hypersetup{pdfcopyright={Copyright 2017-2018 Niklas Beisert.
  This work may be distributed and/or modified under the
  conditions of the LaTeX Project Public License, either version 1.3
  of this license or (at your option) any later version.}}
\hypersetup{pdflicenseurl={http://www.latex-project.org/lppl.txt}}
\hypersetup{pdfcontactaddress={ETH Zurich, ITP, HIT K,
  Wolfgang-Pauli-Strasse 27}}
\hypersetup{pdfcontactpostcode={8093}}
\hypersetup{pdfcontactcity={Zurich}}
\hypersetup{pdfcontactcountry={Switzerland}}
\hypersetup{pdfcontactemail={nbeisert@itp.phys.ethz.ch}}
\hypersetup{pdfcontacturl={http://people.phys.ethz.ch/\xmptilde nbeisert/}}

\newcommand{\secref}[1]{\hyperref[#1]{section \ref*{#1}}}

\parskip1ex
\parindent0pt
\let\olditemize\itemize
\def\itemize{\olditemize\parskip0pt}

\begin{document}

\title{The \textsf{childdoc} Package}
\hypersetup{pdftitle={The childdoc Package}}
\author{Niklas Beisert\\[2ex]
  Institut f\"ur Theoretische Physik\\
  Eidgen\"ossische Technische Hochschule Z\"urich\\
  Wolfgang-Pauli-Strasse 27, 8093 Z\"urich, Switzerland\\[1ex]
  \href{mailto:nbeisert@itp.phys.ethz.ch}
  {\texttt{nbeisert@itp.phys.ethz.ch}}}
\hypersetup{pdfauthor={Niklas Beisert}}
\hypersetup{pdfsubject={Manual for the LaTeX2e Package childdoc}}
\date{30 December 2018, \textsf{v2.0}}
\maketitle

\begin{abstract}\noindent
\textsf{childdoc} is a \LaTeXe{} package
that enables the direct compilation
of document sections included by |\include|
to individual files.
\end{abstract}

\begingroup
\parskip0ex
\tableofcontents
\endgroup

%%%%%%%%%%%%%%%%%%%%%%%%%%%%%%%%%%%%%%%%%%%%%%%%%%%%%%%%%%%%%%%%%%%%%%%%%%%%%%%%
%%%%%%%%%%%%%%%%%%%%%%%%%%%%%%%%%%%%%%%%%%%%%%%%%%%%%%%%%%%%%%%%%%%%%%%%%%%%%%%%
\section{Introduction}

\LaTeX{} provides a mechanism to structure a large document (such as a book)
into a main file and several child files (containing the chapters)
using the |\include| command.
This mechanism is beneficial for documents
which span hundreds of pages in order to
make the source file(s) more manageable.
Moreover, compilation can be restricted to
selected child files by means of the |\includeonly| command.
The latter feature can be used to reduce the compilation time while editing
(this was significantly more useful in the earlier days of \LaTeX{})
or to generate a smaller document which is easier to navigate.
Another application of |\includeonly| is to generate
documents consisting of selected parts of the complete document.

However, there are a few drawbacks of the plain |\include| mechanism:
\begin{itemize}
\item
The child files cannot be compiled on their own,
they can only be compiled via the main file.
A naive editing environment
(such as a text editor with an option
to have the current file processed by \LaTeX)
may require one to switch to the main file before compiling;
attempting to compile the child file produces errors.
\item
The main file must be modified (each time)
to adjust the |\includeonly| command
to the present needs. This easily leaves the main file in a messy state.
\item
The generated document will always carry the filename
of the main document. This is inconvenient if
several child files are to be compiled and
to be kept for distribution.
\end{itemize}

The present package provides a simple interface
to make child files individually compilable by \LaTeX{}.
Compiling a child file then has the same effect as compiling
the main file with an |\includeonly| command
to select the appropriate child.
Moreover the generated document will carry the name of the child
rather than the main file.
This resolves all three above issues.

This feature is meant to make the editing of books,
thesis documents and lecture notes somewhat more convenient.
However, the package can also be used efficiently for
composing a series of documents (such as exercise sheets)
which are typically distributed individually.
It then assists the author in generating the individual documents
(potentially in different versions)
as well as a document containing the collected series.
Another application is in developing style files
or other kinds of included material
where compilation of the style file could redirect
to a sample or test file.

%%%%%%%%%%%%%%%%%%%%%%%%%%%%%%%%%%%%%%%%%%%%%%%%%%%%%%%%%%%%%%%%%%%%%%%%%%%%%%%%
%%%%%%%%%%%%%%%%%%%%%%%%%%%%%%%%%%%%%%%%%%%%%%%%%%%%%%%%%%%%%%%%%%%%%%%%%%%%%%%%
\section{Usage}

First of all, the package \textsf{childdoc} is \emph{not} a standard
\LaTeXe{} |.sty| style file! Therefore it needs to be invoked in
a non-standard way.

%%%%%%%%%%%%%%%%%%%%%%%%%%%%%%%%%%%%%%%%%%%%%%%%%%%%%%%%%%%%%%%%%%%%%%%%%%%%%%%%
\subsection{Included Files}
\label{sec:include}

%%%%%%%%%%%%%%%%%%%%%%%%%%%%%%%%%%%%%%%%
\DescribeMacro{\childdocmain}
To use the package, add the commands
\begin{center}
\begin{tabular}{l}
|\input{childdoc.def}|\\
|\childdocmain{}|\\
\end{tabular}
\end{center}
at the very top of the main \LaTeX{} file,
in particular \emph{before} the |\documentclass| statement!
The argument of |\childdocmain| should be left empty
(but it must be present).

%%%%%%%%%%%%%%%%%%%%%%%%%%%%%%%%%%%%%%%%
\DescribeMacro{\childdocof}
Furthermore, add the commands
\begin{center}
\begin{tabular}{l}
|\input{childdoc.def}|\\
|\childdocof{|\textit{main}|}|\\
\end{tabular}
\end{center}
at the top of every child file \textit{child}
which is included by |\include{|\textit{child}|}|
from within the main file
(or at least for those files to be compiled individually).
The argument \textit{main} must be the filename of the main file.

There are a couple of
considerations in setting up the main and child documents:

%%%%%%%%%%%%%%%%%%%%%%%%%%%%%%%%%%%%%%%%
\paragraph{Restrictions.}

Please note the following restrictions:
\begin{itemize}
\item
|\childdocmain| must be called with one argument \textit{main}
to ensure compatibility with earlier version of the package.
It must either be empty (|\childdocmain{}|)
or precisely match the filename of the main file in which it is specified.
See \secref{sec:detection} for further information.
\item
The filename \textit{main} must be specified without the |.tex| extension.
\item
The filename \textit{main} is case sensitive
(even in case-insensitive file systems)
due to internal string comparison.
\item
The argument \textit{main} should be fully expanded, it cannot be a macro.
\item
Subdirectories and special characters should be avoided in filenames.
\item
The command |\childdocmain{|\textit{main}|}| must be followed by a whitespace.
It should not be followed immediately by another command
or by a comment mark `|%|'.
This is because the \TeX{} parser reads the token immediately following
the argument of |\childdocmain| and puts it
at the beginning of every child section;
however, a white\-space is ignored.
\end{itemize}

%%%%%%%%%%%%%%%%%%%%%%%%%%%%%%%%%%%%%%%%
\paragraph{Content of Main File.}

It is advisable to place all content in the child files included by |\include|.
Any output contained in the main file will appear in all child documents
unless suppressed manually;
it cannot be suppressed automatically by the |\includeonly| directive
and thus should normally be avoided.
A method to include some content in the main file
by means of conditional processing is described in \secref{sec:conditional}.

%%%%%%%%%%%%%%%%%%%%%%%%%%%%%%%%%%%%%%%%
\paragraph{Page Numbering.}

When only a part of the document is compiled,
the appropriate numbering of pages
(as well as other status parameters)
is determined from the |.aux| files.
The latter contain information from previous passes.
However this information needs to propagate through
all intermediate child documents.
Therefore the page numbering in child documents may well
be inconsistent until the complete document is compiled at least once.

A useful (if unconventional) way to always ensure a consistent
page numbering is to restart the numbering in each child document
and denote the pages by `\textit{child}|.|\textit{page}'
where \textit{child} represents the chapter/section number of the child file.
This can be achieved by the command
|\numberwithin{page}{|\textit{child}|}|
of the \textsf{amsmath} package
where \textit{child} can be |chapter| or |section|
depending on the chosen structuring.
Alternatively, one can modify the macro |\thepage| appropriately
and reset the counter |page| at the start of each child file.

%%%%%%%%%%%%%%%%%%%%%%%%%%%%%%%%%%%%%%%%%%%%%%%%%%%%%%%%%%%%%%%%%%%%%%%%%%%%%%%%
\subsection{Conditional Processing}
\label{sec:conditional}

The package provides a mechanism to compile different versions
of a document. To customise the versions further some conditional processing
can come in handy to distinguish which version is being compiled.
The package provides two macros to describe the compilation context:

%%%%%%%%%%%%%%%%%%%%%%%%%%%%%%%%%%%%%%%%
\DescribeMacro{\ifchilddoc}
The conditional |\ifchilddoc| distinguishes between the compilation of
child documents and the main document:
%
\begin{center}
|\ifchilddoc |\textit{child-code}| |[|\||else |\textit{main-code}]| \||fi|
\end{center}

%%%%%%%%%%%%%%%%%%%%%%%%%%%%%%%%%%%%%%%%
\DescribeMacro{\childdocname}
\DescribeMacro{\childdocjob}
The macro |\childdocname| contains the filename (without extension)
of the main or child file being processed.
Note that |\childdocjob| will always contain the name of the main file.

%%%%%%%%%%%%%%%%%%%%%%%%%%%%%%%%%%%%%%%%
\paragraph{Title Page.}

Conditional processing can be used to include a title or banner page
in the main document when proper precautions are taken.
Importantly, the code in the main file should ensure that the page counter
(as well as other status parameters which are stored in the |.aux| files)
takes the same value after the conditional processing.
Otherwise the page numbers may take divergent values
depending on which part is compiled.

For example, a title page could be declared by:
%
\begin{center}
\begin{tabular}{l}
|\ifchilddoc\||else|\\
|\addtocounter{page}{-1}|\\
\textit{code for title page}\\
|\newpage|\\
|\||fi|
\end{tabular}
\end{center}
%
A banner page for the child documents can be generated by:
%
\begin{center}
\begin{tabular}{l}
|\ifchilddoc|\\
|\addtocounter{page}{-1}|\\
\textit{code for banner page}\\
|\newpage|\\
|\||fi|
\end{tabular}
\end{center}
%
Here one could write a message such as:
\begin{center}
|This is the part \childdocname{} of \childdocjob{}.|
\end{center}

%%%%%%%%%%%%%%%%%%%%%%%%%%%%%%%%%%%%%%%%%%%%%%%%%%%%%%%%%%%%%%%%%%%%%%%%%%%%%%%%
\subsection{Flags}
\label{sec:flags}

The package makes it easy to generate different versions
of the main or child documents.
To this end compilation flags can be defined
and assigned different default values.
They will be particularly useful in conjunction
with the forwarding mechanism described in \secref{sec:forward}.

For example, it may be useful to have a flag |\version|
which can be set to |draft| or |final|.
The document source will contain some conditional code
depending on the value of |\version|.
Suppose further, the flag should default to |final| for the main file
and to |draft| for child files
which is a natural assignment for editing the document.
This is achieved by placing the following code
in the preamble of the main document
(below the |\childdocmain| directive):
%
\begin{center}
\begin{tabular}{l}
|\ifchilddoc|\\
|\providecommand{\version}{draft}|\\
|\||else|\\
|\providecommand{\version}{final}|\\
|\||fi|
\end{tabular}
\end{center}
%
The definition by |\providecommand| makes sure
that previous definitions are not overwritten.
Further statements |\providecommand{\version}{...}|
can thus be added before the above code to override it.

For the main file, one might add a line
(between |\childdocmain| and the above block)
%
\begin{center}
|%\ifchilddoc\||else\providecommand{\version}{draft}\||fi|
\end{center}
%
which can be uncommented to produce a draft version.
Likewise one can add a line to the very top of a child file
(above the |\childdocof{|\textit{main}|}| directive)
%
\begin{center}
|%\providecommand{\version}{final}|
\end{center}
%
which can be uncommented to produce the final version of this child document.

%%%%%%%%%%%%%%%%%%%%%%%%%%%%%%%%%%%%%%%%%%%%%%%%%%%%%%%%%%%%%%%%%%%%%%%%%%%%%%%%
\subsection{Forwarding}
\label{sec:forward}

Different versions of the main or child documents
using compilation flags as described in \secref{sec:flags}
can be (permanently) stored in different files
for convenient compilation, viewing and distribution.
To this end, the package defines a command
to pass on compilation to a different file:

%%%%%%%%%%%%%%%%%%%%%%%%%%%%%%%%%%%%%%%%
\DescribeMacro{\childdocforward}
The command |\childdocforward| redirects processing to
another source file:
%
\begin{center}
\begin{tabular}{l}
|\input{childdoc.def}|\\
|\childdocforward[|\textit{main}|]{|\textit{dest}|}|\\
\end{tabular}
\end{center}
%
The argument \textit{dest} is the destination file
(without extension).
It should be the main file or one of the child files.
Note that further \textsf{childdoc} directives
such as |\childdocof| and |\childdocforward|
in the indicated file will be processed in this form.
The optional argument \textit{main}
passes on directly to the main file \textit{main}
while pretending to compile the child \textit{dest}.
This form behaves as if \textit{dest}
issues |\childdocof{|\textit{main}|}| right away,
and no further \textsf{childdoc} directives will be processed.

%%%%%%%%%%%%%%%%%%%%%%%%%%%%%%%%%%%%%%%%
\DescribeMacro{\...prefix}
In the alternative form |\childdocforwardprefix|,
%
\begin{center}
\begin{tabular}{l}
|\input{childdoc.def}|\\
|\childdocforwardprefix[|\textit{main}|]{|\textit{prefix}|}{|\textit{dest}|}|
\end{tabular}
\end{center}
%
the destination file is determined by a pattern
depending on the current file:
To make this work, the current file must be called
`{\textit{prefix}\hspace{0.2em}\textit{suffix}}'
with \textit{prefix} matching precisely the argument.
Processing is then passed on to the file
`{\textit{dest}\hspace{0.2em}\textit{suffix}}'.
Surely, the same effect is achieved by
directly specifying the
argument `{\textit{dest}\hspace{0.2em}\textit{suffix}}'
in the first form.
However, that requires to set up a different file
for each child. With the alternative form of the command
all these files can have exactly the same content
which simplifies setting them up and maintaining them.

For example, the following file |draft.tex|
with a compilation flag |\version| as described in \secref{sec:flags}
compiles the main document as a draft:
%
\begin{center}
\begin{tabular}{l}
|\def\version{draft}|\\
|\input{childdoc.def}|\\
|\childdocforward{|\textit{main}|}|
\end{tabular}
\end{center}
%
Likewise, the following files |final|\textit{nn}|.tex|
compile the final version of the child document
|child|\textit{nn}|.tex|:
%
\begin{center}
\begin{tabular}{l}
|\def\version{final}|\\
|\input{childdoc.def}|\\
|\childdocforwardprefix{final}{child}|
\end{tabular}
\end{center}
%

Note that when several versions of a main file and/or of each child file
are to be generated, it may be convenient to set up a |Makefile| or
shell script to automatise the process.

%%%%%%%%%%%%%%%%%%%%%%%%%%%%%%%%%%%%%%%%%%%%%%%%%%%%%%%%%%%%%%%%%%%%%%%%%%%%%%%%
\subsection{Command Line Processing}
\label{sec:commandline}

The effect of redirection files can also be achieved by invoking
the \LaTeX{} compiler with a more elaborate command line.
Most conveniently this should be done as part
of a shell script or a |Makefile|.

When using \textsf{childdoc} in the main file, the following
command lines effectively perform a redirection
(note that depending on the shell being used,
backslashes may have to be doubled: `|\|' $\to$ `|\\|'):
%
\begin{center}
|... -jobname "|\textit{target}|" |\\|"|[\textit{flags}]%
|\input{childdoc.def}\childdocforward[|\textit{main}|]{|\textit{dest}|}"|
\end{center}
%
Here \textit{target} is the name of the output file,
\textit{main} is the name of the main file
and \textit{dest} is the name of the main or child file to be processed
(all filenames without extensions).
The optional argument \textit{main} can be omitted
if \textit{main} matches \textit{dest}.
Optionally, compilation \textit{flags} can be defined via |\def| commands.
This command line makes the \TeX{} engine believe
it is compiling the file \textit{target}
whose content is specified as the latter parameter.
The provided code then forwards the processing to
\textit{main} or \textit{dest} as described in \secref{sec:forward}.

%%%%%%%%%%%%%%%%%%%%%%%%%%%%%%%%%%%%%%%%%%%%%%%%%%%%%%%%%%%%%%%%%%%%%%%%%%%%%%%%
\subsection{Include by Input}
\label{sec:input}

Including child documents by |\include| has some restrictions by design.
Most notably, the content of a child document always occupies
its own set of pages; pages cannot be shared between child documents.
Usually, this behaviour makes perfect sense
because each child document contain an essential part of the document.
However, in some situations it may be desirable to compose
a document from a collection of parts
without having mandatory page breaks between then.
For this case, the package
provides a mechanism to include parts
by |\input| which can also be processed individually.
However, by construction this mechanism
requires manual handling of the content to be output.

%%%%%%%%%%%%%%%%%%%%%%%%%%%%%%%%%%%%%%%%
\DescribeMacro{\ifchilddocmanual}
The main file should be prepared as usual, see \secref{sec:include}.
However, the document body must make a distinction
between processing of an individual part and of the main document, e.g.:
%
\begin{center}
\begin{tabular}{l}
|\ifchilddocmanual|\\
|\input{\childdocname}|\\
|\||else|\\
\textit{document body with }|\input{|\textit{part}|}|\\
|\||fi|
\end{tabular}
\end{center}
%
The conditional |\ifchilddocmanual| is true whenever
a part to be included by |\input| is being compiled,
and the name of the part is stored in |\childdocname|.

%%%%%%%%%%%%%%%%%%%%%%%%%%%%%%%%%%%%%%%%
\DescribeMacro{\childdocby}
Each part to be included by |\input| should start with:
%
\begin{center}
\begin{tabular}{l}
|\input{childdoc.def}|\\
|\childdocby{|\textit{main}|}|\\
\end{tabular}
\end{center}
%
The directive |\childdocby| is similar to |\childdocof|
described in \secref{sec:include},
but the subsequent selection of content must be done manually.
To that end, both |\ifchilddoc| and |\ifchilddocmanual|
will be true upon processing of a part,
and the name of the part is stored in |\childdocname|.
Note that |\jobname| will be set to the filename of the current part
so that each part receives an individual |.aux| file
that does not interfere with the |.aux| file(s) of the main document.
This behaviour can be altered by the alternative form
|\childdocby[*]{|\textit{main}|}| (with a non-empty optional argument)
which uses the |.aux| file of the main document
by setting |\jobname| to \textit{main}.

%%%%%%%%%%%%%%%%%%%%%%%%%%%%%%%%%%%%%%%%%%%%%%%%%%%%%%%%%%%%%%%%%%%%%%%%%%%%%%%%
\subsection{Driver Development}
\label{sec:driver}

The \textsf{childdoc} mechanism can also be use for the development
of definition files such as \LaTeX{} styles or classes.
This case differs from the above setup with multiple parts
included by |\include| in that no |\includeonly| should be invoked.
This can be achieved by starting the include file
(before |\ProvidesPackage|) with:
%
\begin{center}
\begin{tabular}{l}
|\input{childdoc.def}|\\
|\childdocforward{|\textit{main}|}|\\
\end{tabular}
\end{center}
%
or alternatively with:
%
\begin{center}
\begin{tabular}{l}
|\input{childdoc.def}|\\
|\childdocby{|\textit{main}|}|\\
\end{tabular}
\end{center}
%
Both forms have slightly different effects as described above.
The main file is prepared as usual, see \secref{sec:include}.

%%%%%%%%%%%%%%%%%%%%%%%%%%%%%%%%%%%%%%%%%%%%%%%%%%%%%%%%%%%%%%%%%%%%%%%%%%%%%%%%
\subsection{Legacy Detection}
\label{sec:detection}

The directive |\childdocmain| in the main file can detect
whether the complete document or merely a child is to be compiled
even without using the directive |\childdocof|.
This method is deprecated because it is less robust
and there is no compelling reason to use it;
it is merely provided for backward compatibility
and it may be removed in future versions.

If the detection mechanism is to be used,
it is mandatory to correctly specify
the filename of the main file as the argument of |\childdocmain|:
%
\begin{center}
\begin{tabular}{l}
|\input{childdoc.def}|\\
|\childdocmain{|\textit{main}|}|\\
\end{tabular}
\end{center}
%
If |\jobname| does not match the argument \textit{main} of |\childdocmain|,
it is assumed that |\jobname| points to the child file to be compiled.
When using |\childdocmain| with the main file specified as argument,
it suffices to start a child file
with just |\input{|\textit{main}|}|
without loading of the package and using |\childdocof|.
If instead all processing is done
with the appropriate \textsf{childdoc} directives,
the argument of \textit{main} of |\childdocmain| can be empty.

An alternative version of the command line processing described
in \secref{sec:commandline} using the detection mechanism reads:
%
\begin{center}
|... -jobname "|\textit{target}|" "|[\textit{flags}]%
[|\def\jobname{|\textit{dest}|}|]|\input{|\textit{main}|}"|
\end{center}

%%%%%%%%%%%%%%%%%%%%%%%%%%%%%%%%%%%%%%%%%%%%%%%%%%%%%%%%%%%%%%%%%%%%%%%%%%%%%%%%
\subsection{Manual Code}
\label{sec:manual}

In case one cannot be certain whether the definitions file |childdoc.def|
is installed on the target \TeX{} distribution
and one prefers not to ship it,
it is conceivable to paste a few relevant commands into the sources.

To that end, drop all statements |\input{childdoc.def}|
and perform the replacements as outlined below.
Instead of |\childdocmain{|\textit{main}|}| add the following code
to the top of the main file:
%
\begin{center}
\begin{tabular}{l}
|\||ifdefined\childdocname\endinput\||fi\newif\ifchilddoc|\\
|\edef\childdocname{\scantokens\expandafter{\jobname\noexpand}}|\\
|\def\childdocmain{|\textit{main}|}\||ifx\childdocmain\childdocname\||else|\\
|\childdoctrue\includeonly{\childdocname}\let\jobname\childdocmain\||fi|\\
\end{tabular}
\end{center}
%
Instead of |\childdocof{|\textit{main}|}| just include the main file
at the top of each child file:
%
\begin{center}
|\input{|\textit{main}|}|
\end{center}
%
A simple redirection |\childdocforward{|\textit{dest}|}| is achieved by:
%
\begin{center}
|\def\jobname{|\textit{dest}|}\input{\jobname}|
\end{center}
%
The redirection with prefix
|\childdocforwardprefix[|\textit{prefix}|]{|\textit{dest}|}|
is accomplished by:
%
\begin{center}
\begin{tabular}{l}
|{\edef\jobname{\scantokens\expandafter{\jobname\noexpand}}|\\
|\def\redirectjob |\textit{prefix}|#1~~~{\gdef\jobname{|\textit{dest}|#1}}|\\
|\expandafter\redirectjob\jobname~~~}\input{\jobname}|
\end{tabular}
\end{center}

In an alternative approach,
child documents can be compiled by a specific command line
without additional code or specific definitions:
%
\begin{center}
|... -jobname "|\textit{target}|" "|[\textit{flags}]%
|\includeonly{|\textit{dest}|}\input{|\textit{main}|}"|
\end{center}
%

%%%%%%%%%%%%%%%%%%%%%%%%%%%%%%%%%%%%%%%%%%%%%%%%%%%%%%%%%%%%%%%%%%%%%%%%%%%%%%%%
%%%%%%%%%%%%%%%%%%%%%%%%%%%%%%%%%%%%%%%%%%%%%%%%%%%%%%%%%%%%%%%%%%%%%%%%%%%%%%%%
\section{Information}

%%%%%%%%%%%%%%%%%%%%%%%%%%%%%%%%%%%%%%%%%%%%%%%%%%%%%%%%%%%%%%%%%%%%%%%%%%%%%%%%
\subsection{Copyright}

Copyright \copyright{} 2017--2018 Niklas Beisert

This work may be distributed and/or modified under the
conditions of the \LaTeX{} Project Public License, either version 1.3
of this license or (at your option) any later version.
The latest version of this license is in
  \url{http://www.latex-project.org/lppl.txt}
and version 1.3 or later is part of all distributions of \LaTeX{}
version 2005/12/01 or later.

This work has the LPPL maintenance status `maintained'.

The Current Maintainer of this work is Niklas Beisert.

This work consists of the files |README.txt|, |childdoc.ins| and |childdoc.dtx|
as well as the derived files |childdoc.def|, |cdocsamp.tex|
with |cdocsch1.tex|, |cdocsch2.tex|, |cdocspt3.tex|, |cdocspt4.tex|,
|cdocsdrf.tex|, |cdocsfn1.tex|, |cdocsfn2.tex|
as well as |childdoc.pdf|.

%%%%%%%%%%%%%%%%%%%%%%%%%%%%%%%%%%%%%%%%%%%%%%%%%%%%%%%%%%%%%%%%%%%%%%%%%%%%%%%%
\subsection{Files and Installation}

The package consists of the files:
%
\begin{center}
\begin{tabular}{ll}
    |README.txt|   & readme file \\
    |childdoc.ins| & installation file \\
    |childdoc.dtx| & source file \\
    |childdoc.def| & definition file \\
    |cdocsamp.tex| & sample main file \\
    |cdocsch1.tex| & sample include file \\
    |cdocsch2.tex| & sample include file \\
    |cdocspt3.tex| & sample part file \\
    |cdocspt4.tex| & sample part file \\
    |cdocsdrf.tex| & sample redirection file \\
    |cdocsfn1.tex| & sample redirection file \\
    |cdocsfn2.tex| & sample redirection file \\
    |childdoc.pdf| & manual
\end{tabular}
\end{center}
%
The distribution consists of the files
|README.txt|, |childdoc.ins| and |childdoc.dtx|.
%
\begin{itemize}
\item
Run (pdf)\LaTeX{} on |childdoc.dtx|
to compile the manual |childdoc.pdf| (this file).
\item
Run \LaTeX{} on |childdoc.ins| to create the definitions file |childdoc.def|
and the sample |cdocsamp.tex| with include files
|cdocsch1.tex|, |cdocsch2.tex|, |cdocspt3.tex|, |cdocspt4.tex|,
|cdocsdrf.tex|, |cdocsfn1.tex|, |cdocsfn2.tex|.
Then copy the file |childdoc.def| to an appropriate directory of your \LaTeX{}
distribution, e.g.\ \textit{texmf-root}|/tex/latex/childdoc|.
\end{itemize}

%%%%%%%%%%%%%%%%%%%%%%%%%%%%%%%%%%%%%%%%%%%%%%%%%%%%%%%%%%%%%%%%%%%%%%%%%%%%%%%%
\subsection{Related CTAN Packages}

There are several other packages which offer a similar functionality:
%
\begin{itemize}
\item
The packages
\href{http://ctan.org/pkg/docmute}{\textsf{docmute}},
\href{http://ctan.org/pkg/includex}{\textsf{includex}} and
\href{http://ctan.org/pkg/standalone}{\textsf{standalone}}
provide commands to include only the document body of
a child file thus allowing both files to be compiled individually.
\item
The packages \href{http://ctan.org/pkg/subdocs}{\textsf{subdocs}}
and \href{http://ctan.org/pkg/subfiles}{\textsf{subfiles}}
provide structures in which the main and child documents can be
encapsulated and allowing them to be compiled individually.
The inclusion mechanism is different from the conventional |\include|.
\item
The package \href{http://ctan.org/pkg/combine}{\textsf{combine}}
is an elaborate solution to combine several documents into one.
\end{itemize}
%
See also the CTAN topic \href{http://ctan.org/topic/subdocs}{\textsf{subdocs}}
for further related packages.
The present package differs from the above solutions in that
a document structure constructed with the conventional |\include| mechanism
just needs two extra commands at the top of every file
such that all constituent files can be compiled individually.

%%%%%%%%%%%%%%%%%%%%%%%%%%%%%%%%%%%%%%%%%%%%%%%%%%%%%%%%%%%%%%%%%%%%%%%%%%%%%%%%
%\subsection{Feature Suggestions}
%
%The following is a list of features which may be useful for future
%versions of this package:
%%
%\begin{itemize}
%\item
%\ldots
%\end{itemize}

%%%%%%%%%%%%%%%%%%%%%%%%%%%%%%%%%%%%%%%%%%%%%%%%%%%%%%%%%%%%%%%%%%%%%%%%%%%%%%%%
\subsection{Revision History}

%%%%%%%%%%%%%%%%%%%%%%%%%%%%%%%%%%%%%%%%
\paragraph{v2.0:} 2018/12/30

\begin{itemize}
\item
immediate forward processing
\item
added |\childdocby| mechanism
\item
manual restructured
\end{itemize}

%%%%%%%%%%%%%%%%%%%%%%%%%%%%%%%%%%%%%%%%
\paragraph{v1.6:} 2018/01/17

\begin{itemize}
\item
application for development of include files
\item
corrections to manual
\end{itemize}

%%%%%%%%%%%%%%%%%%%%%%%%%%%%%%%%%%%%%%%%
\paragraph{v1.5:} 2017/05/21

\begin{itemize}
\item
more complete structuring introduced
\item
|\childdocof| introduced
\item
|\childdoc| renamed to |\childdocmain|
\item
|\childredirect| renamed to |\childdocforward| and |\childdocforwardprefix|
and functionality expanded
\end{itemize}

%%%%%%%%%%%%%%%%%%%%%%%%%%%%%%%%%%%%%%%%
\paragraph{v1.0:} 2017/04/27

\begin{itemize}
\item
manual and install package
\item
first version published on CTAN
\end{itemize}

%%%%%%%%%%%%%%%%%%%%%%%%%%%%%%%%%%%%%%%%
\paragraph{v0.6:} 2017/04/26

\begin{itemize}
\item
redirection mechanism added
\end{itemize}

%%%%%%%%%%%%%%%%%%%%%%%%%%%%%%%%%%%%%%%%
\paragraph{v0.5:} 2017/04/26

\begin{itemize}
\item
functionality in definition file
\end{itemize}


%%%%%%%%%%%%%%%%%%%%%%%%%%%%%%%%%%%%%%%%%%%%%%%%%%%%%%%%%%%%%%%%%%%%%%%%%%%%%%%%
%%%%%%%%%%%%%%%%%%%%%%%%%%%%%%%%%%%%%%%%%%%%%%%%%%%%%%%%%%%%%%%%%%%%%%%%%%%%%%%%
%%%%%%%%%%%%%%%%%%%%%%%%%%%%%%%%%%%%%%%%%%%%%%%%%%%%%%%%%%%%%%%%%%%%%%%%%%%%%%%%
\appendix

\settowidth\MacroIndent{\rmfamily\scriptsize 000\ }

 \DocInput{childdoc.dtx}

\end{document}
%</driver>
% \fi
%
% %%%%%%%%%%%%%%%%%%%%%%%%%%%%%%%%%%%%%%%%%%%%%%%%%%%%%%%%%%%%%%%%%%%%%%%%%%%%%%
% %%%%%%%%%%%%%%%%%%%%%%%%%%%%%%%%%%%%%%%%%%%%%%%%%%%%%%%%%%%%%%%%%%%%%%%%%%%%%%
% \section{Sample}
%\iffalse
%<*samplemain>
%\fi
%
% The following presents a sample document
% with two chapters, two parts, a title page,
% a compile flag as well as three forwarding files to set the flag.
% It consists of eight |.tex| files:
% \begin{center}
% \begin{tabular}{ll}
% |cdocsamp.tex|&main file\\
% |cdocsch1.tex|&include file for chapter 1\\
% |cdocsch2.tex|&include file for chapter 2\\
% |cdocspt3.tex|&include file for part 3\\
% |cdocspt4.tex|&include file for part 4\\
% |cdocsdrf.tex|&forwarding file for main file in draft mode\\
% |cdocsfi1.tex|&forwarding file for final version of chapter 1\\
% |cdocsfi2.tex|&forwarding file for final version of chapter 2\\
% \end{tabular}
% \end{center}
% Each of the eight files can be compiled directly by the \LaTeX{} compiler.
%
% %%%%%%%%%%%%%%%%%%%%%%%%%%%%%%%%%%%%%%
% \paragraph{Main File.}
%
% The main file is called |cdocsamp.tex|.
%
% Load the \textsf{childdoc} definitions and
% declare the filename for the main document:
%    \begin{macrocode}
\input{childdoc.def}
\childdocmain{}
%    \end{macrocode}

% Optional override for |\version| flag:
%    \begin{macrocode}
%%\ifchilddoc\else\providecommand{\version}{draft}\fi
%    \end{macrocode}

% Define the default values for the |\version| flag
% (|final| for the main file and |draft| for childs):
%    \begin{macrocode}
\ifchilddoc
\providecommand{\version}{draft}
\else
\providecommand{\version}{final}
\fi
%    \end{macrocode}

% Load the standard document class:
%    \begin{macrocode}
\documentclass[12pt]{article}
%    \end{macrocode}

% Start the document body:
%    \begin{macrocode}
\begin{document}
%    \end{macrocode}

% Declare a title page.
% Print title, part of document being processed and version flag:
%    \begin{macrocode}
\addtocounter{page}{-1}
\begin{center}
{\LARGE\bfseries{}childdoc example\par}
\vspace{1cm}
\ifchilddoc
\ifchilddocmanual part\else chapter\fi:
`\childdocname' of `\childdocjob'\par
\else
main document: `\childdocjob'\par
\fi
version: \version\par
\end{center}
\newpage
%    \end{macrocode}

% Manually include selected file,
% otherwise process as usual:
%    \begin{macrocode}
\ifchilddocmanual
\section*{part `\childdocname'}
\input{\childdocname}
\else
%    \end{macrocode}

% Include the two chapters:
%    \begin{macrocode}
\include{cdocsch1}
\include{cdocsch2}
%    \end{macrocode}

% Include the two parts unless only chapters should be displayed:
%    \begin{macrocode}
\ifchilddoc\else
\section{part three}
\input{cdocspt3}
\section{part four}
\input{cdocspt4}
\fi
%    \end{macrocode}

% Process as usual until here:
%    \begin{macrocode}
\fi
%    \end{macrocode}

% End of document body:
%    \begin{macrocode}
\end{document}
%    \end{macrocode}
%\iffalse
%</samplemain>
%\fi
%
% %%%%%%%%%%%%%%%%%%%%%%%%%%%%%%%%%%%%%%
% \paragraph{Chapter Include Files.}
%
% The include files are called |cdocsch1.tex| and |cdocsch2.tex|.
%
%\iffalse
%<*samplechap1|samplechap2>
%\fi

% Optional override for |\version| flag:
%    \begin{macrocode}
%%\providecommand{\version}{final}
%    \end{macrocode}

% Include the main document:
%    \begin{macrocode}
\input{childdoc.def}
\childdocof{cdocsamp}
%    \end{macrocode}

%\iffalse
%</samplechap1|samplechap2>
%\fi
%
%\iffalse
%<*samplechap1>
%\fi
% Some text for chapter 1:
%    \begin{macrocode}
\section{one}
some text in chapter one
%    \end{macrocode}

%\iffalse
%</samplechap1>
%\fi
% Some text for chapter 2:
%\iffalse
%<*samplechap2>
%\fi
%    \begin{macrocode}
\section{two}
more text in chapter two
%    \end{macrocode}

%\iffalse
%</samplechap2>
%\fi
%
% %%%%%%%%%%%%%%%%%%%%%%%%%%%%%%%%%%%%%%
% \paragraph{Part Include Files.}
%
% The include files are called |cdocspt3.tex| and |cdocspt4.tex|.
%
%\iffalse
%<*samplepart3|samplepart4>
%\fi

% Optional override for |\version| flag:
%    \begin{macrocode}
%%\providecommand{\version}{final}
%    \end{macrocode}

% Include the main document:
%    \begin{macrocode}
\input{childdoc.def}
\childdocby{cdocsamp}
%    \end{macrocode}

%\iffalse
%</samplepart3|samplepart4>
%\fi
%
%\iffalse
%<*samplepart3>
%\fi
% Some text for part 3:
%    \begin{macrocode}
some text in part three
%    \end{macrocode}

%\iffalse
%</samplepart3>
%\fi
% Some text for part 4:
%\iffalse
%<*samplepart4>
%\fi
%    \begin{macrocode}
more text in part four
%    \end{macrocode}

%\iffalse
%</samplepart4>
%\fi
%
% %%%%%%%%%%%%%%%%%%%%%%%%%%%%%%%%%%%%%%
% \paragraph{Forwarding for a Complete Draft.}
%
% The following forwarding file |cdocsdrf.tex|
% compiles the main document in draft mode:
%\iffalse
%<*sampledraft>
%\fi
%    \begin{macrocode}
\def\version{draft}
\input{childdoc.def}
\childdocforward{cdocsamp}
%    \end{macrocode}

%\iffalse
%</sampledraft>
%\fi
%
% %%%%%%%%%%%%%%%%%%%%%%%%%%%%%%%%%%%%%%
% \paragraph{Forwarding for Final Version of the Chapters.}
%
% The following forwarding files |cdocsfn1.tex| and |cdocsfn2.tex|
% (with identical content)
% compile the final versions of the child documents
% |cdocsch1.tex| and |cdocsch2.tex|, respectively:
%\iffalse
%<*samplefinal>
%\fi
%    \begin{macrocode}
\def\version{final}
\input{childdoc.def}
\childdocforwardprefix[cdocsamp]{cdocsfn}{cdocsch}
%    \end{macrocode}

%\iffalse
%</samplefinal>
%\fi
%
% %%%%%%%%%%%%%%%%%%%%%%%%%%%%%%%%%%%%%%
% \paragraph{Command Line Processing.}
%
% The following three command lines generate the output files
% |cdocscld|, |cdocscl1| and |cdocscl2|
% which should be identical to
% |cdocsdrf|, |cdocsch1| and |cdocsfn2|, respectively:
% \begin{center}
% \begin{tabular}{l}
% |latex -jobname cdocscld \|\\
% |  "\def\version{draft}\input{childdoc.def}\childdocforward{cdocsamp}"|\\
% |latex -jobname cdocscl1 \|\\
% |  "\input{childdoc.def}\childdocforward[cdocsamp]{cdocsch1}"|\\
% |latex -jobname cdocscl2 \|\\
% |  "\def\version{final}\input{childdoc.def}\childdocforward{cdocsch2}"|
% \end{tabular}
% \end{center}
% Note that the trailing backslash on each first line
% merely continues the input to the second line
% (for convenient cut ant paste).
% Furthermore, the command |latex| can be replaced by any
% of its alternative versions such as |pdflatex|.
%
% %%%%%%%%%%%%%%%%%%%%%%%%%%%%%%%%%%%%%%%%%%%%%%%%%%%%%%%%%%%%%%%%%%%%%%%%%%%%%%
% %%%%%%%%%%%%%%%%%%%%%%%%%%%%%%%%%%%%%%%%%%%%%%%%%%%%%%%%%%%%%%%%%%%%%%%%%%%%%%
% \section{Implementation}
%\iffalse
%<*package>
%\fi
%
% This section describes the definitions file |childdoc.def|.

% The definitions cannot be loaded using |\usepackage| or |\RequirePackage|
% which has a mechanism to prevent loading a style file more than once.
% When loading the definitions by means of |\input|
% multiple instances have to be prevented manually:
%\iffalse
%This code needs to be before the `\ProvidesFile' directive
%which is defined at the beginning of this file.
%Therefore it is also placed there and commented out here.
%</package>
%<*discard>
%\fi
%    \begin{macrocode}
\ifdefined\childdocmain\endinput\fi
%    \end{macrocode}
%\iffalse
%</discard>
%<*package>
%\fi
%
% \macro{\ifchilddoc}
% \macro{\ifchilddocmanual}
% The conditional |\ifchilddoc| tells whether a
% child (true) or main (false) document is being compiled.
% The conditional |\ifchilddocmanual| tells whether
% the |\includeonly| mechanism is used (false) or
% the selection of child files must be performed manually (true).
% The definitions initialise to false:
%    \begin{macrocode}
\newif\ifchilddoc
\newif\ifchilddocmanual
%    \end{macrocode}

% \macro{\childdocname}
% \macro{\childdocjob}
% The macro |\childdocname| stores the name of the main document
% to be compiled. The macro |\childdocjob| stores the name of
% the document on which the \LaTeX{} compiler was originally invoked.
% The content of |\jobname| cannot be compared
% to filenames specified in the source due to different catcodes.
% The following code rescans |\jobname|, stores the result
% in |\childdocname| and saves a copy in |\childdocjob|:
%    \begin{macrocode}
\edef\childdocname{\scantokens\expandafter{\jobname\noexpand}}
\let\childdocjob\childdocname
%    \end{macrocode}

% \macro{\childdocdisable}
% The macro |\childdocdisable| prevents the main file
% from being processed more than once.
% At this stage, the main document command |\childdocmain|
% is assumed to be called once again where it should do nothing.
% Any subsequent call to it should prevent
% a secondary processing of the main document
% It overwrites the forwarding commands
% |\childdocof| and |\childdocforward|
% with empty macros to prevent further inclusions of the main document:
%    \begin{macrocode}
\newcommand{\childdocdisable}
{
  \renewcommand{\childdocmain}[1]{\renewcommand{\childdocmain}[1]{\endinput}}
  \renewcommand{\childdocof}[1]{}
  \renewcommand{\childdocby}[2][]{}
  \renewcommand{\childdocforward}[2][]{}
  \renewcommand{\childdocdisable}{}
}
%    \end{macrocode}

% \macro{\childdocmain}
% The macro |\childdocmain| is to be called at the top of the main file
% with nothing or the main filename (without extension) as argument.
% First, it breaks loops.
% If the argument is not empty and does not match |\childdocname|
% (which is set by the first inclusion of |childdoc.def|),
% |\ifchilddoc| is set to true, |\includeonly| is applied to the child file
% and |\jobname| is set to the main file
% (for proper handling of |.aux| files):
%    \begin{macrocode}
\newcommand{\childdocmain}[1]
{
  \childdocdisable\childdocmain{}
  \if?#1?\else
    \begingroup
      \def\childdoctmp{#1}
      \ifx\childdoctmp\childdocname
        \def\childdoctmp{}
      \else
        \def\childdoctmp
        {
          \childdoctrue
          \includeonly{\childdocname}
          \def\childdocjob{#1}
          \def\jobname{#1}
        }
      \fi
      \expandafter
    \endgroup
    \childdoctmp
  \fi
}
%    \end{macrocode}

% \macro{\childdocof}
% The command |\childdocof| redirects
% compilation to the main file |#1|.
%    \begin{macrocode}
\newcommand{\childdocof}[1]
{
  \childdocdisable
  \childdoctrue
  \includeonly{\childdocname}
  \def\jobname{#1}
  \def\childdocjob{#1}
  \input{#1}
}
%    \end{macrocode}

% \macro{\childdocby}
% The command |\childdocby| ....
%    \begin{macrocode}
\newcommand{\childdocby}[2][]
{
  \childdocdisable
  \childdoctrue
  \childdocmanualtrue
  \if?#1?\else
    \def\jobname{#2}
  \fi
  \def\childdocjob{#2}
  \input{#2}
  \endinput
}
%    \end{macrocode}

% \macro{\childdocforward}
% The command |\childdocforward| redirects
% compilation to the main file or
% (if the optional argument is given) a child file.
% Parameters are set as if the main file
% or a child file starting with |\childdocof| was compiled.
% Then compilation is handed over to the main file:
%    \begin{macrocode}
\newcommand{\childdocforward}[2][]
{
  \begingroup
    \if?#1?
      \def\childdoctmp
      {
        \def\childdocname{#2}
        \def\childdocjob{#2}
        \def\jobname{#2}
        \input{#2}
        \endinput
      }
    \else
      \def\childdoctmp
      {
        \childdocdisable
        \def\childdocname{#2}
        \childdoctrue
        \includeonly{#2}
        \def\childdocjob{#1}
        \def\jobname{#1}
        \input{#1}
        \endinput
      }
    \fi
    \expandafter
  \endgroup
  \childdoctmp
}
%    \end{macrocode}

% \macro{\childdocforwardprefix}
% The command |\childdocforwardprefix| redirects
% compilation to the main or a child file by means of a pattern.
% The prefix |#1| in the current filename is replaced by |#2|
% and the suffix of the current filename is kept
% (it is assumed that the filename does not contain the substring `|~~~|'
% which is used as a delimiter).
% Compilation is handed over to the new file by |\childdocforward|:
%    \begin{macrocode}
\newcommand{\childdocforwardprefix}[3][]
{
  \begingroup
    \def\childdocextract #2##1~~~{\def\childdoctmp{\childdocforward[#1]{#3##1}}}
    \expandafter\childdocextract\childdocname~~~
    \expandafter
  \endgroup
  \childdoctmp
}
%    \end{macrocode}

% \macro{\childdoc}
% The deprecated macro |\childdoc| is a legacy version of |\childdocmain|:
%    \begin{macrocode}
\newcommand{\childdoc}{\childdocmain}
%    \end{macrocode}

% \macro{\childdocredirect}
% The deprecated macro |\childdocredirect| is a legacy version
% of |\childdocforward| and |\childdocforwardprefix|:
%    \begin{macrocode}
\newcommand{\childdocredirect}[2][]
{
  \begingroup
    \if?#1?
      \def\childdoctmp{\childdocforward{#2}}
    \else
      \def\childdoctmp{\childdocforwardprefix{#1}{#2}}
    \fi
    \expandafter
  \endgroup
  \childdoctmp
}
%    \end{macrocode}

%\iffalse
%</package>
%\fi
%
\endinput

\childdocof{cdocsamp}
%    \end{macrocode}

%\iffalse
%</samplechap1|samplechap2>
%\fi
%
%\iffalse
%<*samplechap1>
%\fi
% Some text for chapter 1:
%    \begin{macrocode}
\section{one}
some text in chapter one
%    \end{macrocode}

%\iffalse
%</samplechap1>
%\fi
% Some text for chapter 2:
%\iffalse
%<*samplechap2>
%\fi
%    \begin{macrocode}
\section{two}
more text in chapter two
%    \end{macrocode}

%\iffalse
%</samplechap2>
%\fi
%
% %%%%%%%%%%%%%%%%%%%%%%%%%%%%%%%%%%%%%%
% \paragraph{Part Include Files.}
%
% The include files are called |cdocspt3.tex| and |cdocspt4.tex|.
%
%\iffalse
%<*samplepart3|samplepart4>
%\fi

% Optional override for |\version| flag:
%    \begin{macrocode}
%%\providecommand{\version}{final}
%    \end{macrocode}

% Include the main document:
%    \begin{macrocode}
% \iffalse
%
% childdoc.dtx Copyright (C) 2017-2018 Niklas Beisert
%
% This work may be distributed and/or modified under the
% conditions of the LaTeX Project Public License, either version 1.3
% of this license or (at your option) any later version.
% The latest version of this license is in
%   http://www.latex-project.org/lppl.txt
% and version 1.3 or later is part of all distributions of LaTeX
% version 2005/12/01 or later.
%
% This work has the LPPL maintenance status `maintained'.
%
% The Current Maintainer of this work is Niklas Beisert.
%
% This work consists of the files childdoc.dtx and childdoc.ins
% and the derived files childdoc.def and cdocsamp.tex with
% cdocsch1.tex, cdocsch2.tex, cdocsdrf.tex, cdocsfn1.tex, cdocsfn2.tex.
%
%<package>\ifdefined\childdocmain\endinput\fi
%<package>\ProvidesFile{childdoc.def}[2018/12/30 v2.0 child document driver]
%<samplemain>\ProvidesFile{cdocsamp.tex}[2018/12/30 v2.0 sample for childdoc]
%<*driver>
%\ProvidesFile{childdoc.drv}[2018/12/30 v2.0 childdoc reference manual file]
\PassOptionsToClass{10pt,a4paper}{article}
\documentclass{ltxdoc}

\usepackage[margin=35mm]{geometry}
\usepackage{hyperref}
\usepackage{hyperxmp}
\usepackage[usenames]{color}

\hypersetup{colorlinks=true}
\hypersetup{pdfstartview=FitH}
\hypersetup{pdfpagemode=UseNone}
\hypersetup{pdfsource={}}
\hypersetup{pdflang={en-UK}}
\hypersetup{pdfcopyright={Copyright 2017-2018 Niklas Beisert.
  This work may be distributed and/or modified under the
  conditions of the LaTeX Project Public License, either version 1.3
  of this license or (at your option) any later version.}}
\hypersetup{pdflicenseurl={http://www.latex-project.org/lppl.txt}}
\hypersetup{pdfcontactaddress={ETH Zurich, ITP, HIT K,
  Wolfgang-Pauli-Strasse 27}}
\hypersetup{pdfcontactpostcode={8093}}
\hypersetup{pdfcontactcity={Zurich}}
\hypersetup{pdfcontactcountry={Switzerland}}
\hypersetup{pdfcontactemail={nbeisert@itp.phys.ethz.ch}}
\hypersetup{pdfcontacturl={http://people.phys.ethz.ch/\xmptilde nbeisert/}}

\newcommand{\secref}[1]{\hyperref[#1]{section \ref*{#1}}}

\parskip1ex
\parindent0pt
\let\olditemize\itemize
\def\itemize{\olditemize\parskip0pt}

\begin{document}

\title{The \textsf{childdoc} Package}
\hypersetup{pdftitle={The childdoc Package}}
\author{Niklas Beisert\\[2ex]
  Institut f\"ur Theoretische Physik\\
  Eidgen\"ossische Technische Hochschule Z\"urich\\
  Wolfgang-Pauli-Strasse 27, 8093 Z\"urich, Switzerland\\[1ex]
  \href{mailto:nbeisert@itp.phys.ethz.ch}
  {\texttt{nbeisert@itp.phys.ethz.ch}}}
\hypersetup{pdfauthor={Niklas Beisert}}
\hypersetup{pdfsubject={Manual for the LaTeX2e Package childdoc}}
\date{30 December 2018, \textsf{v2.0}}
\maketitle

\begin{abstract}\noindent
\textsf{childdoc} is a \LaTeXe{} package
that enables the direct compilation
of document sections included by |\include|
to individual files.
\end{abstract}

\begingroup
\parskip0ex
\tableofcontents
\endgroup

%%%%%%%%%%%%%%%%%%%%%%%%%%%%%%%%%%%%%%%%%%%%%%%%%%%%%%%%%%%%%%%%%%%%%%%%%%%%%%%%
%%%%%%%%%%%%%%%%%%%%%%%%%%%%%%%%%%%%%%%%%%%%%%%%%%%%%%%%%%%%%%%%%%%%%%%%%%%%%%%%
\section{Introduction}

\LaTeX{} provides a mechanism to structure a large document (such as a book)
into a main file and several child files (containing the chapters)
using the |\include| command.
This mechanism is beneficial for documents
which span hundreds of pages in order to
make the source file(s) more manageable.
Moreover, compilation can be restricted to
selected child files by means of the |\includeonly| command.
The latter feature can be used to reduce the compilation time while editing
(this was significantly more useful in the earlier days of \LaTeX{})
or to generate a smaller document which is easier to navigate.
Another application of |\includeonly| is to generate
documents consisting of selected parts of the complete document.

However, there are a few drawbacks of the plain |\include| mechanism:
\begin{itemize}
\item
The child files cannot be compiled on their own,
they can only be compiled via the main file.
A naive editing environment
(such as a text editor with an option
to have the current file processed by \LaTeX)
may require one to switch to the main file before compiling;
attempting to compile the child file produces errors.
\item
The main file must be modified (each time)
to adjust the |\includeonly| command
to the present needs. This easily leaves the main file in a messy state.
\item
The generated document will always carry the filename
of the main document. This is inconvenient if
several child files are to be compiled and
to be kept for distribution.
\end{itemize}

The present package provides a simple interface
to make child files individually compilable by \LaTeX{}.
Compiling a child file then has the same effect as compiling
the main file with an |\includeonly| command
to select the appropriate child.
Moreover the generated document will carry the name of the child
rather than the main file.
This resolves all three above issues.

This feature is meant to make the editing of books,
thesis documents and lecture notes somewhat more convenient.
However, the package can also be used efficiently for
composing a series of documents (such as exercise sheets)
which are typically distributed individually.
It then assists the author in generating the individual documents
(potentially in different versions)
as well as a document containing the collected series.
Another application is in developing style files
or other kinds of included material
where compilation of the style file could redirect
to a sample or test file.

%%%%%%%%%%%%%%%%%%%%%%%%%%%%%%%%%%%%%%%%%%%%%%%%%%%%%%%%%%%%%%%%%%%%%%%%%%%%%%%%
%%%%%%%%%%%%%%%%%%%%%%%%%%%%%%%%%%%%%%%%%%%%%%%%%%%%%%%%%%%%%%%%%%%%%%%%%%%%%%%%
\section{Usage}

First of all, the package \textsf{childdoc} is \emph{not} a standard
\LaTeXe{} |.sty| style file! Therefore it needs to be invoked in
a non-standard way.

%%%%%%%%%%%%%%%%%%%%%%%%%%%%%%%%%%%%%%%%%%%%%%%%%%%%%%%%%%%%%%%%%%%%%%%%%%%%%%%%
\subsection{Included Files}
\label{sec:include}

%%%%%%%%%%%%%%%%%%%%%%%%%%%%%%%%%%%%%%%%
\DescribeMacro{\childdocmain}
To use the package, add the commands
\begin{center}
\begin{tabular}{l}
|\input{childdoc.def}|\\
|\childdocmain{}|\\
\end{tabular}
\end{center}
at the very top of the main \LaTeX{} file,
in particular \emph{before} the |\documentclass| statement!
The argument of |\childdocmain| should be left empty
(but it must be present).

%%%%%%%%%%%%%%%%%%%%%%%%%%%%%%%%%%%%%%%%
\DescribeMacro{\childdocof}
Furthermore, add the commands
\begin{center}
\begin{tabular}{l}
|\input{childdoc.def}|\\
|\childdocof{|\textit{main}|}|\\
\end{tabular}
\end{center}
at the top of every child file \textit{child}
which is included by |\include{|\textit{child}|}|
from within the main file
(or at least for those files to be compiled individually).
The argument \textit{main} must be the filename of the main file.

There are a couple of
considerations in setting up the main and child documents:

%%%%%%%%%%%%%%%%%%%%%%%%%%%%%%%%%%%%%%%%
\paragraph{Restrictions.}

Please note the following restrictions:
\begin{itemize}
\item
|\childdocmain| must be called with one argument \textit{main}
to ensure compatibility with earlier version of the package.
It must either be empty (|\childdocmain{}|)
or precisely match the filename of the main file in which it is specified.
See \secref{sec:detection} for further information.
\item
The filename \textit{main} must be specified without the |.tex| extension.
\item
The filename \textit{main} is case sensitive
(even in case-insensitive file systems)
due to internal string comparison.
\item
The argument \textit{main} should be fully expanded, it cannot be a macro.
\item
Subdirectories and special characters should be avoided in filenames.
\item
The command |\childdocmain{|\textit{main}|}| must be followed by a whitespace.
It should not be followed immediately by another command
or by a comment mark `|%|'.
This is because the \TeX{} parser reads the token immediately following
the argument of |\childdocmain| and puts it
at the beginning of every child section;
however, a white\-space is ignored.
\end{itemize}

%%%%%%%%%%%%%%%%%%%%%%%%%%%%%%%%%%%%%%%%
\paragraph{Content of Main File.}

It is advisable to place all content in the child files included by |\include|.
Any output contained in the main file will appear in all child documents
unless suppressed manually;
it cannot be suppressed automatically by the |\includeonly| directive
and thus should normally be avoided.
A method to include some content in the main file
by means of conditional processing is described in \secref{sec:conditional}.

%%%%%%%%%%%%%%%%%%%%%%%%%%%%%%%%%%%%%%%%
\paragraph{Page Numbering.}

When only a part of the document is compiled,
the appropriate numbering of pages
(as well as other status parameters)
is determined from the |.aux| files.
The latter contain information from previous passes.
However this information needs to propagate through
all intermediate child documents.
Therefore the page numbering in child documents may well
be inconsistent until the complete document is compiled at least once.

A useful (if unconventional) way to always ensure a consistent
page numbering is to restart the numbering in each child document
and denote the pages by `\textit{child}|.|\textit{page}'
where \textit{child} represents the chapter/section number of the child file.
This can be achieved by the command
|\numberwithin{page}{|\textit{child}|}|
of the \textsf{amsmath} package
where \textit{child} can be |chapter| or |section|
depending on the chosen structuring.
Alternatively, one can modify the macro |\thepage| appropriately
and reset the counter |page| at the start of each child file.

%%%%%%%%%%%%%%%%%%%%%%%%%%%%%%%%%%%%%%%%%%%%%%%%%%%%%%%%%%%%%%%%%%%%%%%%%%%%%%%%
\subsection{Conditional Processing}
\label{sec:conditional}

The package provides a mechanism to compile different versions
of a document. To customise the versions further some conditional processing
can come in handy to distinguish which version is being compiled.
The package provides two macros to describe the compilation context:

%%%%%%%%%%%%%%%%%%%%%%%%%%%%%%%%%%%%%%%%
\DescribeMacro{\ifchilddoc}
The conditional |\ifchilddoc| distinguishes between the compilation of
child documents and the main document:
%
\begin{center}
|\ifchilddoc |\textit{child-code}| |[|\||else |\textit{main-code}]| \||fi|
\end{center}

%%%%%%%%%%%%%%%%%%%%%%%%%%%%%%%%%%%%%%%%
\DescribeMacro{\childdocname}
\DescribeMacro{\childdocjob}
The macro |\childdocname| contains the filename (without extension)
of the main or child file being processed.
Note that |\childdocjob| will always contain the name of the main file.

%%%%%%%%%%%%%%%%%%%%%%%%%%%%%%%%%%%%%%%%
\paragraph{Title Page.}

Conditional processing can be used to include a title or banner page
in the main document when proper precautions are taken.
Importantly, the code in the main file should ensure that the page counter
(as well as other status parameters which are stored in the |.aux| files)
takes the same value after the conditional processing.
Otherwise the page numbers may take divergent values
depending on which part is compiled.

For example, a title page could be declared by:
%
\begin{center}
\begin{tabular}{l}
|\ifchilddoc\||else|\\
|\addtocounter{page}{-1}|\\
\textit{code for title page}\\
|\newpage|\\
|\||fi|
\end{tabular}
\end{center}
%
A banner page for the child documents can be generated by:
%
\begin{center}
\begin{tabular}{l}
|\ifchilddoc|\\
|\addtocounter{page}{-1}|\\
\textit{code for banner page}\\
|\newpage|\\
|\||fi|
\end{tabular}
\end{center}
%
Here one could write a message such as:
\begin{center}
|This is the part \childdocname{} of \childdocjob{}.|
\end{center}

%%%%%%%%%%%%%%%%%%%%%%%%%%%%%%%%%%%%%%%%%%%%%%%%%%%%%%%%%%%%%%%%%%%%%%%%%%%%%%%%
\subsection{Flags}
\label{sec:flags}

The package makes it easy to generate different versions
of the main or child documents.
To this end compilation flags can be defined
and assigned different default values.
They will be particularly useful in conjunction
with the forwarding mechanism described in \secref{sec:forward}.

For example, it may be useful to have a flag |\version|
which can be set to |draft| or |final|.
The document source will contain some conditional code
depending on the value of |\version|.
Suppose further, the flag should default to |final| for the main file
and to |draft| for child files
which is a natural assignment for editing the document.
This is achieved by placing the following code
in the preamble of the main document
(below the |\childdocmain| directive):
%
\begin{center}
\begin{tabular}{l}
|\ifchilddoc|\\
|\providecommand{\version}{draft}|\\
|\||else|\\
|\providecommand{\version}{final}|\\
|\||fi|
\end{tabular}
\end{center}
%
The definition by |\providecommand| makes sure
that previous definitions are not overwritten.
Further statements |\providecommand{\version}{...}|
can thus be added before the above code to override it.

For the main file, one might add a line
(between |\childdocmain| and the above block)
%
\begin{center}
|%\ifchilddoc\||else\providecommand{\version}{draft}\||fi|
\end{center}
%
which can be uncommented to produce a draft version.
Likewise one can add a line to the very top of a child file
(above the |\childdocof{|\textit{main}|}| directive)
%
\begin{center}
|%\providecommand{\version}{final}|
\end{center}
%
which can be uncommented to produce the final version of this child document.

%%%%%%%%%%%%%%%%%%%%%%%%%%%%%%%%%%%%%%%%%%%%%%%%%%%%%%%%%%%%%%%%%%%%%%%%%%%%%%%%
\subsection{Forwarding}
\label{sec:forward}

Different versions of the main or child documents
using compilation flags as described in \secref{sec:flags}
can be (permanently) stored in different files
for convenient compilation, viewing and distribution.
To this end, the package defines a command
to pass on compilation to a different file:

%%%%%%%%%%%%%%%%%%%%%%%%%%%%%%%%%%%%%%%%
\DescribeMacro{\childdocforward}
The command |\childdocforward| redirects processing to
another source file:
%
\begin{center}
\begin{tabular}{l}
|\input{childdoc.def}|\\
|\childdocforward[|\textit{main}|]{|\textit{dest}|}|\\
\end{tabular}
\end{center}
%
The argument \textit{dest} is the destination file
(without extension).
It should be the main file or one of the child files.
Note that further \textsf{childdoc} directives
such as |\childdocof| and |\childdocforward|
in the indicated file will be processed in this form.
The optional argument \textit{main}
passes on directly to the main file \textit{main}
while pretending to compile the child \textit{dest}.
This form behaves as if \textit{dest}
issues |\childdocof{|\textit{main}|}| right away,
and no further \textsf{childdoc} directives will be processed.

%%%%%%%%%%%%%%%%%%%%%%%%%%%%%%%%%%%%%%%%
\DescribeMacro{\...prefix}
In the alternative form |\childdocforwardprefix|,
%
\begin{center}
\begin{tabular}{l}
|\input{childdoc.def}|\\
|\childdocforwardprefix[|\textit{main}|]{|\textit{prefix}|}{|\textit{dest}|}|
\end{tabular}
\end{center}
%
the destination file is determined by a pattern
depending on the current file:
To make this work, the current file must be called
`{\textit{prefix}\hspace{0.2em}\textit{suffix}}'
with \textit{prefix} matching precisely the argument.
Processing is then passed on to the file
`{\textit{dest}\hspace{0.2em}\textit{suffix}}'.
Surely, the same effect is achieved by
directly specifying the
argument `{\textit{dest}\hspace{0.2em}\textit{suffix}}'
in the first form.
However, that requires to set up a different file
for each child. With the alternative form of the command
all these files can have exactly the same content
which simplifies setting them up and maintaining them.

For example, the following file |draft.tex|
with a compilation flag |\version| as described in \secref{sec:flags}
compiles the main document as a draft:
%
\begin{center}
\begin{tabular}{l}
|\def\version{draft}|\\
|\input{childdoc.def}|\\
|\childdocforward{|\textit{main}|}|
\end{tabular}
\end{center}
%
Likewise, the following files |final|\textit{nn}|.tex|
compile the final version of the child document
|child|\textit{nn}|.tex|:
%
\begin{center}
\begin{tabular}{l}
|\def\version{final}|\\
|\input{childdoc.def}|\\
|\childdocforwardprefix{final}{child}|
\end{tabular}
\end{center}
%

Note that when several versions of a main file and/or of each child file
are to be generated, it may be convenient to set up a |Makefile| or
shell script to automatise the process.

%%%%%%%%%%%%%%%%%%%%%%%%%%%%%%%%%%%%%%%%%%%%%%%%%%%%%%%%%%%%%%%%%%%%%%%%%%%%%%%%
\subsection{Command Line Processing}
\label{sec:commandline}

The effect of redirection files can also be achieved by invoking
the \LaTeX{} compiler with a more elaborate command line.
Most conveniently this should be done as part
of a shell script or a |Makefile|.

When using \textsf{childdoc} in the main file, the following
command lines effectively perform a redirection
(note that depending on the shell being used,
backslashes may have to be doubled: `|\|' $\to$ `|\\|'):
%
\begin{center}
|... -jobname "|\textit{target}|" |\\|"|[\textit{flags}]%
|\input{childdoc.def}\childdocforward[|\textit{main}|]{|\textit{dest}|}"|
\end{center}
%
Here \textit{target} is the name of the output file,
\textit{main} is the name of the main file
and \textit{dest} is the name of the main or child file to be processed
(all filenames without extensions).
The optional argument \textit{main} can be omitted
if \textit{main} matches \textit{dest}.
Optionally, compilation \textit{flags} can be defined via |\def| commands.
This command line makes the \TeX{} engine believe
it is compiling the file \textit{target}
whose content is specified as the latter parameter.
The provided code then forwards the processing to
\textit{main} or \textit{dest} as described in \secref{sec:forward}.

%%%%%%%%%%%%%%%%%%%%%%%%%%%%%%%%%%%%%%%%%%%%%%%%%%%%%%%%%%%%%%%%%%%%%%%%%%%%%%%%
\subsection{Include by Input}
\label{sec:input}

Including child documents by |\include| has some restrictions by design.
Most notably, the content of a child document always occupies
its own set of pages; pages cannot be shared between child documents.
Usually, this behaviour makes perfect sense
because each child document contain an essential part of the document.
However, in some situations it may be desirable to compose
a document from a collection of parts
without having mandatory page breaks between then.
For this case, the package
provides a mechanism to include parts
by |\input| which can also be processed individually.
However, by construction this mechanism
requires manual handling of the content to be output.

%%%%%%%%%%%%%%%%%%%%%%%%%%%%%%%%%%%%%%%%
\DescribeMacro{\ifchilddocmanual}
The main file should be prepared as usual, see \secref{sec:include}.
However, the document body must make a distinction
between processing of an individual part and of the main document, e.g.:
%
\begin{center}
\begin{tabular}{l}
|\ifchilddocmanual|\\
|\input{\childdocname}|\\
|\||else|\\
\textit{document body with }|\input{|\textit{part}|}|\\
|\||fi|
\end{tabular}
\end{center}
%
The conditional |\ifchilddocmanual| is true whenever
a part to be included by |\input| is being compiled,
and the name of the part is stored in |\childdocname|.

%%%%%%%%%%%%%%%%%%%%%%%%%%%%%%%%%%%%%%%%
\DescribeMacro{\childdocby}
Each part to be included by |\input| should start with:
%
\begin{center}
\begin{tabular}{l}
|\input{childdoc.def}|\\
|\childdocby{|\textit{main}|}|\\
\end{tabular}
\end{center}
%
The directive |\childdocby| is similar to |\childdocof|
described in \secref{sec:include},
but the subsequent selection of content must be done manually.
To that end, both |\ifchilddoc| and |\ifchilddocmanual|
will be true upon processing of a part,
and the name of the part is stored in |\childdocname|.
Note that |\jobname| will be set to the filename of the current part
so that each part receives an individual |.aux| file
that does not interfere with the |.aux| file(s) of the main document.
This behaviour can be altered by the alternative form
|\childdocby[*]{|\textit{main}|}| (with a non-empty optional argument)
which uses the |.aux| file of the main document
by setting |\jobname| to \textit{main}.

%%%%%%%%%%%%%%%%%%%%%%%%%%%%%%%%%%%%%%%%%%%%%%%%%%%%%%%%%%%%%%%%%%%%%%%%%%%%%%%%
\subsection{Driver Development}
\label{sec:driver}

The \textsf{childdoc} mechanism can also be use for the development
of definition files such as \LaTeX{} styles or classes.
This case differs from the above setup with multiple parts
included by |\include| in that no |\includeonly| should be invoked.
This can be achieved by starting the include file
(before |\ProvidesPackage|) with:
%
\begin{center}
\begin{tabular}{l}
|\input{childdoc.def}|\\
|\childdocforward{|\textit{main}|}|\\
\end{tabular}
\end{center}
%
or alternatively with:
%
\begin{center}
\begin{tabular}{l}
|\input{childdoc.def}|\\
|\childdocby{|\textit{main}|}|\\
\end{tabular}
\end{center}
%
Both forms have slightly different effects as described above.
The main file is prepared as usual, see \secref{sec:include}.

%%%%%%%%%%%%%%%%%%%%%%%%%%%%%%%%%%%%%%%%%%%%%%%%%%%%%%%%%%%%%%%%%%%%%%%%%%%%%%%%
\subsection{Legacy Detection}
\label{sec:detection}

The directive |\childdocmain| in the main file can detect
whether the complete document or merely a child is to be compiled
even without using the directive |\childdocof|.
This method is deprecated because it is less robust
and there is no compelling reason to use it;
it is merely provided for backward compatibility
and it may be removed in future versions.

If the detection mechanism is to be used,
it is mandatory to correctly specify
the filename of the main file as the argument of |\childdocmain|:
%
\begin{center}
\begin{tabular}{l}
|\input{childdoc.def}|\\
|\childdocmain{|\textit{main}|}|\\
\end{tabular}
\end{center}
%
If |\jobname| does not match the argument \textit{main} of |\childdocmain|,
it is assumed that |\jobname| points to the child file to be compiled.
When using |\childdocmain| with the main file specified as argument,
it suffices to start a child file
with just |\input{|\textit{main}|}|
without loading of the package and using |\childdocof|.
If instead all processing is done
with the appropriate \textsf{childdoc} directives,
the argument of \textit{main} of |\childdocmain| can be empty.

An alternative version of the command line processing described
in \secref{sec:commandline} using the detection mechanism reads:
%
\begin{center}
|... -jobname "|\textit{target}|" "|[\textit{flags}]%
[|\def\jobname{|\textit{dest}|}|]|\input{|\textit{main}|}"|
\end{center}

%%%%%%%%%%%%%%%%%%%%%%%%%%%%%%%%%%%%%%%%%%%%%%%%%%%%%%%%%%%%%%%%%%%%%%%%%%%%%%%%
\subsection{Manual Code}
\label{sec:manual}

In case one cannot be certain whether the definitions file |childdoc.def|
is installed on the target \TeX{} distribution
and one prefers not to ship it,
it is conceivable to paste a few relevant commands into the sources.

To that end, drop all statements |\input{childdoc.def}|
and perform the replacements as outlined below.
Instead of |\childdocmain{|\textit{main}|}| add the following code
to the top of the main file:
%
\begin{center}
\begin{tabular}{l}
|\||ifdefined\childdocname\endinput\||fi\newif\ifchilddoc|\\
|\edef\childdocname{\scantokens\expandafter{\jobname\noexpand}}|\\
|\def\childdocmain{|\textit{main}|}\||ifx\childdocmain\childdocname\||else|\\
|\childdoctrue\includeonly{\childdocname}\let\jobname\childdocmain\||fi|\\
\end{tabular}
\end{center}
%
Instead of |\childdocof{|\textit{main}|}| just include the main file
at the top of each child file:
%
\begin{center}
|\input{|\textit{main}|}|
\end{center}
%
A simple redirection |\childdocforward{|\textit{dest}|}| is achieved by:
%
\begin{center}
|\def\jobname{|\textit{dest}|}\input{\jobname}|
\end{center}
%
The redirection with prefix
|\childdocforwardprefix[|\textit{prefix}|]{|\textit{dest}|}|
is accomplished by:
%
\begin{center}
\begin{tabular}{l}
|{\edef\jobname{\scantokens\expandafter{\jobname\noexpand}}|\\
|\def\redirectjob |\textit{prefix}|#1~~~{\gdef\jobname{|\textit{dest}|#1}}|\\
|\expandafter\redirectjob\jobname~~~}\input{\jobname}|
\end{tabular}
\end{center}

In an alternative approach,
child documents can be compiled by a specific command line
without additional code or specific definitions:
%
\begin{center}
|... -jobname "|\textit{target}|" "|[\textit{flags}]%
|\includeonly{|\textit{dest}|}\input{|\textit{main}|}"|
\end{center}
%

%%%%%%%%%%%%%%%%%%%%%%%%%%%%%%%%%%%%%%%%%%%%%%%%%%%%%%%%%%%%%%%%%%%%%%%%%%%%%%%%
%%%%%%%%%%%%%%%%%%%%%%%%%%%%%%%%%%%%%%%%%%%%%%%%%%%%%%%%%%%%%%%%%%%%%%%%%%%%%%%%
\section{Information}

%%%%%%%%%%%%%%%%%%%%%%%%%%%%%%%%%%%%%%%%%%%%%%%%%%%%%%%%%%%%%%%%%%%%%%%%%%%%%%%%
\subsection{Copyright}

Copyright \copyright{} 2017--2018 Niklas Beisert

This work may be distributed and/or modified under the
conditions of the \LaTeX{} Project Public License, either version 1.3
of this license or (at your option) any later version.
The latest version of this license is in
  \url{http://www.latex-project.org/lppl.txt}
and version 1.3 or later is part of all distributions of \LaTeX{}
version 2005/12/01 or later.

This work has the LPPL maintenance status `maintained'.

The Current Maintainer of this work is Niklas Beisert.

This work consists of the files |README.txt|, |childdoc.ins| and |childdoc.dtx|
as well as the derived files |childdoc.def|, |cdocsamp.tex|
with |cdocsch1.tex|, |cdocsch2.tex|, |cdocspt3.tex|, |cdocspt4.tex|,
|cdocsdrf.tex|, |cdocsfn1.tex|, |cdocsfn2.tex|
as well as |childdoc.pdf|.

%%%%%%%%%%%%%%%%%%%%%%%%%%%%%%%%%%%%%%%%%%%%%%%%%%%%%%%%%%%%%%%%%%%%%%%%%%%%%%%%
\subsection{Files and Installation}

The package consists of the files:
%
\begin{center}
\begin{tabular}{ll}
    |README.txt|   & readme file \\
    |childdoc.ins| & installation file \\
    |childdoc.dtx| & source file \\
    |childdoc.def| & definition file \\
    |cdocsamp.tex| & sample main file \\
    |cdocsch1.tex| & sample include file \\
    |cdocsch2.tex| & sample include file \\
    |cdocspt3.tex| & sample part file \\
    |cdocspt4.tex| & sample part file \\
    |cdocsdrf.tex| & sample redirection file \\
    |cdocsfn1.tex| & sample redirection file \\
    |cdocsfn2.tex| & sample redirection file \\
    |childdoc.pdf| & manual
\end{tabular}
\end{center}
%
The distribution consists of the files
|README.txt|, |childdoc.ins| and |childdoc.dtx|.
%
\begin{itemize}
\item
Run (pdf)\LaTeX{} on |childdoc.dtx|
to compile the manual |childdoc.pdf| (this file).
\item
Run \LaTeX{} on |childdoc.ins| to create the definitions file |childdoc.def|
and the sample |cdocsamp.tex| with include files
|cdocsch1.tex|, |cdocsch2.tex|, |cdocspt3.tex|, |cdocspt4.tex|,
|cdocsdrf.tex|, |cdocsfn1.tex|, |cdocsfn2.tex|.
Then copy the file |childdoc.def| to an appropriate directory of your \LaTeX{}
distribution, e.g.\ \textit{texmf-root}|/tex/latex/childdoc|.
\end{itemize}

%%%%%%%%%%%%%%%%%%%%%%%%%%%%%%%%%%%%%%%%%%%%%%%%%%%%%%%%%%%%%%%%%%%%%%%%%%%%%%%%
\subsection{Related CTAN Packages}

There are several other packages which offer a similar functionality:
%
\begin{itemize}
\item
The packages
\href{http://ctan.org/pkg/docmute}{\textsf{docmute}},
\href{http://ctan.org/pkg/includex}{\textsf{includex}} and
\href{http://ctan.org/pkg/standalone}{\textsf{standalone}}
provide commands to include only the document body of
a child file thus allowing both files to be compiled individually.
\item
The packages \href{http://ctan.org/pkg/subdocs}{\textsf{subdocs}}
and \href{http://ctan.org/pkg/subfiles}{\textsf{subfiles}}
provide structures in which the main and child documents can be
encapsulated and allowing them to be compiled individually.
The inclusion mechanism is different from the conventional |\include|.
\item
The package \href{http://ctan.org/pkg/combine}{\textsf{combine}}
is an elaborate solution to combine several documents into one.
\end{itemize}
%
See also the CTAN topic \href{http://ctan.org/topic/subdocs}{\textsf{subdocs}}
for further related packages.
The present package differs from the above solutions in that
a document structure constructed with the conventional |\include| mechanism
just needs two extra commands at the top of every file
such that all constituent files can be compiled individually.

%%%%%%%%%%%%%%%%%%%%%%%%%%%%%%%%%%%%%%%%%%%%%%%%%%%%%%%%%%%%%%%%%%%%%%%%%%%%%%%%
%\subsection{Feature Suggestions}
%
%The following is a list of features which may be useful for future
%versions of this package:
%%
%\begin{itemize}
%\item
%\ldots
%\end{itemize}

%%%%%%%%%%%%%%%%%%%%%%%%%%%%%%%%%%%%%%%%%%%%%%%%%%%%%%%%%%%%%%%%%%%%%%%%%%%%%%%%
\subsection{Revision History}

%%%%%%%%%%%%%%%%%%%%%%%%%%%%%%%%%%%%%%%%
\paragraph{v2.0:} 2018/12/30

\begin{itemize}
\item
immediate forward processing
\item
added |\childdocby| mechanism
\item
manual restructured
\end{itemize}

%%%%%%%%%%%%%%%%%%%%%%%%%%%%%%%%%%%%%%%%
\paragraph{v1.6:} 2018/01/17

\begin{itemize}
\item
application for development of include files
\item
corrections to manual
\end{itemize}

%%%%%%%%%%%%%%%%%%%%%%%%%%%%%%%%%%%%%%%%
\paragraph{v1.5:} 2017/05/21

\begin{itemize}
\item
more complete structuring introduced
\item
|\childdocof| introduced
\item
|\childdoc| renamed to |\childdocmain|
\item
|\childredirect| renamed to |\childdocforward| and |\childdocforwardprefix|
and functionality expanded
\end{itemize}

%%%%%%%%%%%%%%%%%%%%%%%%%%%%%%%%%%%%%%%%
\paragraph{v1.0:} 2017/04/27

\begin{itemize}
\item
manual and install package
\item
first version published on CTAN
\end{itemize}

%%%%%%%%%%%%%%%%%%%%%%%%%%%%%%%%%%%%%%%%
\paragraph{v0.6:} 2017/04/26

\begin{itemize}
\item
redirection mechanism added
\end{itemize}

%%%%%%%%%%%%%%%%%%%%%%%%%%%%%%%%%%%%%%%%
\paragraph{v0.5:} 2017/04/26

\begin{itemize}
\item
functionality in definition file
\end{itemize}


%%%%%%%%%%%%%%%%%%%%%%%%%%%%%%%%%%%%%%%%%%%%%%%%%%%%%%%%%%%%%%%%%%%%%%%%%%%%%%%%
%%%%%%%%%%%%%%%%%%%%%%%%%%%%%%%%%%%%%%%%%%%%%%%%%%%%%%%%%%%%%%%%%%%%%%%%%%%%%%%%
%%%%%%%%%%%%%%%%%%%%%%%%%%%%%%%%%%%%%%%%%%%%%%%%%%%%%%%%%%%%%%%%%%%%%%%%%%%%%%%%
\appendix

\settowidth\MacroIndent{\rmfamily\scriptsize 000\ }

 \DocInput{childdoc.dtx}

\end{document}
%</driver>
% \fi
%
% %%%%%%%%%%%%%%%%%%%%%%%%%%%%%%%%%%%%%%%%%%%%%%%%%%%%%%%%%%%%%%%%%%%%%%%%%%%%%%
% %%%%%%%%%%%%%%%%%%%%%%%%%%%%%%%%%%%%%%%%%%%%%%%%%%%%%%%%%%%%%%%%%%%%%%%%%%%%%%
% \section{Sample}
%\iffalse
%<*samplemain>
%\fi
%
% The following presents a sample document
% with two chapters, two parts, a title page,
% a compile flag as well as three forwarding files to set the flag.
% It consists of eight |.tex| files:
% \begin{center}
% \begin{tabular}{ll}
% |cdocsamp.tex|&main file\\
% |cdocsch1.tex|&include file for chapter 1\\
% |cdocsch2.tex|&include file for chapter 2\\
% |cdocspt3.tex|&include file for part 3\\
% |cdocspt4.tex|&include file for part 4\\
% |cdocsdrf.tex|&forwarding file for main file in draft mode\\
% |cdocsfi1.tex|&forwarding file for final version of chapter 1\\
% |cdocsfi2.tex|&forwarding file for final version of chapter 2\\
% \end{tabular}
% \end{center}
% Each of the eight files can be compiled directly by the \LaTeX{} compiler.
%
% %%%%%%%%%%%%%%%%%%%%%%%%%%%%%%%%%%%%%%
% \paragraph{Main File.}
%
% The main file is called |cdocsamp.tex|.
%
% Load the \textsf{childdoc} definitions and
% declare the filename for the main document:
%    \begin{macrocode}
\input{childdoc.def}
\childdocmain{}
%    \end{macrocode}

% Optional override for |\version| flag:
%    \begin{macrocode}
%%\ifchilddoc\else\providecommand{\version}{draft}\fi
%    \end{macrocode}

% Define the default values for the |\version| flag
% (|final| for the main file and |draft| for childs):
%    \begin{macrocode}
\ifchilddoc
\providecommand{\version}{draft}
\else
\providecommand{\version}{final}
\fi
%    \end{macrocode}

% Load the standard document class:
%    \begin{macrocode}
\documentclass[12pt]{article}
%    \end{macrocode}

% Start the document body:
%    \begin{macrocode}
\begin{document}
%    \end{macrocode}

% Declare a title page.
% Print title, part of document being processed and version flag:
%    \begin{macrocode}
\addtocounter{page}{-1}
\begin{center}
{\LARGE\bfseries{}childdoc example\par}
\vspace{1cm}
\ifchilddoc
\ifchilddocmanual part\else chapter\fi:
`\childdocname' of `\childdocjob'\par
\else
main document: `\childdocjob'\par
\fi
version: \version\par
\end{center}
\newpage
%    \end{macrocode}

% Manually include selected file,
% otherwise process as usual:
%    \begin{macrocode}
\ifchilddocmanual
\section*{part `\childdocname'}
\input{\childdocname}
\else
%    \end{macrocode}

% Include the two chapters:
%    \begin{macrocode}
\include{cdocsch1}
\include{cdocsch2}
%    \end{macrocode}

% Include the two parts unless only chapters should be displayed:
%    \begin{macrocode}
\ifchilddoc\else
\section{part three}
\input{cdocspt3}
\section{part four}
\input{cdocspt4}
\fi
%    \end{macrocode}

% Process as usual until here:
%    \begin{macrocode}
\fi
%    \end{macrocode}

% End of document body:
%    \begin{macrocode}
\end{document}
%    \end{macrocode}
%\iffalse
%</samplemain>
%\fi
%
% %%%%%%%%%%%%%%%%%%%%%%%%%%%%%%%%%%%%%%
% \paragraph{Chapter Include Files.}
%
% The include files are called |cdocsch1.tex| and |cdocsch2.tex|.
%
%\iffalse
%<*samplechap1|samplechap2>
%\fi

% Optional override for |\version| flag:
%    \begin{macrocode}
%%\providecommand{\version}{final}
%    \end{macrocode}

% Include the main document:
%    \begin{macrocode}
\input{childdoc.def}
\childdocof{cdocsamp}
%    \end{macrocode}

%\iffalse
%</samplechap1|samplechap2>
%\fi
%
%\iffalse
%<*samplechap1>
%\fi
% Some text for chapter 1:
%    \begin{macrocode}
\section{one}
some text in chapter one
%    \end{macrocode}

%\iffalse
%</samplechap1>
%\fi
% Some text for chapter 2:
%\iffalse
%<*samplechap2>
%\fi
%    \begin{macrocode}
\section{two}
more text in chapter two
%    \end{macrocode}

%\iffalse
%</samplechap2>
%\fi
%
% %%%%%%%%%%%%%%%%%%%%%%%%%%%%%%%%%%%%%%
% \paragraph{Part Include Files.}
%
% The include files are called |cdocspt3.tex| and |cdocspt4.tex|.
%
%\iffalse
%<*samplepart3|samplepart4>
%\fi

% Optional override for |\version| flag:
%    \begin{macrocode}
%%\providecommand{\version}{final}
%    \end{macrocode}

% Include the main document:
%    \begin{macrocode}
\input{childdoc.def}
\childdocby{cdocsamp}
%    \end{macrocode}

%\iffalse
%</samplepart3|samplepart4>
%\fi
%
%\iffalse
%<*samplepart3>
%\fi
% Some text for part 3:
%    \begin{macrocode}
some text in part three
%    \end{macrocode}

%\iffalse
%</samplepart3>
%\fi
% Some text for part 4:
%\iffalse
%<*samplepart4>
%\fi
%    \begin{macrocode}
more text in part four
%    \end{macrocode}

%\iffalse
%</samplepart4>
%\fi
%
% %%%%%%%%%%%%%%%%%%%%%%%%%%%%%%%%%%%%%%
% \paragraph{Forwarding for a Complete Draft.}
%
% The following forwarding file |cdocsdrf.tex|
% compiles the main document in draft mode:
%\iffalse
%<*sampledraft>
%\fi
%    \begin{macrocode}
\def\version{draft}
\input{childdoc.def}
\childdocforward{cdocsamp}
%    \end{macrocode}

%\iffalse
%</sampledraft>
%\fi
%
% %%%%%%%%%%%%%%%%%%%%%%%%%%%%%%%%%%%%%%
% \paragraph{Forwarding for Final Version of the Chapters.}
%
% The following forwarding files |cdocsfn1.tex| and |cdocsfn2.tex|
% (with identical content)
% compile the final versions of the child documents
% |cdocsch1.tex| and |cdocsch2.tex|, respectively:
%\iffalse
%<*samplefinal>
%\fi
%    \begin{macrocode}
\def\version{final}
\input{childdoc.def}
\childdocforwardprefix[cdocsamp]{cdocsfn}{cdocsch}
%    \end{macrocode}

%\iffalse
%</samplefinal>
%\fi
%
% %%%%%%%%%%%%%%%%%%%%%%%%%%%%%%%%%%%%%%
% \paragraph{Command Line Processing.}
%
% The following three command lines generate the output files
% |cdocscld|, |cdocscl1| and |cdocscl2|
% which should be identical to
% |cdocsdrf|, |cdocsch1| and |cdocsfn2|, respectively:
% \begin{center}
% \begin{tabular}{l}
% |latex -jobname cdocscld \|\\
% |  "\def\version{draft}\input{childdoc.def}\childdocforward{cdocsamp}"|\\
% |latex -jobname cdocscl1 \|\\
% |  "\input{childdoc.def}\childdocforward[cdocsamp]{cdocsch1}"|\\
% |latex -jobname cdocscl2 \|\\
% |  "\def\version{final}\input{childdoc.def}\childdocforward{cdocsch2}"|
% \end{tabular}
% \end{center}
% Note that the trailing backslash on each first line
% merely continues the input to the second line
% (for convenient cut ant paste).
% Furthermore, the command |latex| can be replaced by any
% of its alternative versions such as |pdflatex|.
%
% %%%%%%%%%%%%%%%%%%%%%%%%%%%%%%%%%%%%%%%%%%%%%%%%%%%%%%%%%%%%%%%%%%%%%%%%%%%%%%
% %%%%%%%%%%%%%%%%%%%%%%%%%%%%%%%%%%%%%%%%%%%%%%%%%%%%%%%%%%%%%%%%%%%%%%%%%%%%%%
% \section{Implementation}
%\iffalse
%<*package>
%\fi
%
% This section describes the definitions file |childdoc.def|.

% The definitions cannot be loaded using |\usepackage| or |\RequirePackage|
% which has a mechanism to prevent loading a style file more than once.
% When loading the definitions by means of |\input|
% multiple instances have to be prevented manually:
%\iffalse
%This code needs to be before the `\ProvidesFile' directive
%which is defined at the beginning of this file.
%Therefore it is also placed there and commented out here.
%</package>
%<*discard>
%\fi
%    \begin{macrocode}
\ifdefined\childdocmain\endinput\fi
%    \end{macrocode}
%\iffalse
%</discard>
%<*package>
%\fi
%
% \macro{\ifchilddoc}
% \macro{\ifchilddocmanual}
% The conditional |\ifchilddoc| tells whether a
% child (true) or main (false) document is being compiled.
% The conditional |\ifchilddocmanual| tells whether
% the |\includeonly| mechanism is used (false) or
% the selection of child files must be performed manually (true).
% The definitions initialise to false:
%    \begin{macrocode}
\newif\ifchilddoc
\newif\ifchilddocmanual
%    \end{macrocode}

% \macro{\childdocname}
% \macro{\childdocjob}
% The macro |\childdocname| stores the name of the main document
% to be compiled. The macro |\childdocjob| stores the name of
% the document on which the \LaTeX{} compiler was originally invoked.
% The content of |\jobname| cannot be compared
% to filenames specified in the source due to different catcodes.
% The following code rescans |\jobname|, stores the result
% in |\childdocname| and saves a copy in |\childdocjob|:
%    \begin{macrocode}
\edef\childdocname{\scantokens\expandafter{\jobname\noexpand}}
\let\childdocjob\childdocname
%    \end{macrocode}

% \macro{\childdocdisable}
% The macro |\childdocdisable| prevents the main file
% from being processed more than once.
% At this stage, the main document command |\childdocmain|
% is assumed to be called once again where it should do nothing.
% Any subsequent call to it should prevent
% a secondary processing of the main document
% It overwrites the forwarding commands
% |\childdocof| and |\childdocforward|
% with empty macros to prevent further inclusions of the main document:
%    \begin{macrocode}
\newcommand{\childdocdisable}
{
  \renewcommand{\childdocmain}[1]{\renewcommand{\childdocmain}[1]{\endinput}}
  \renewcommand{\childdocof}[1]{}
  \renewcommand{\childdocby}[2][]{}
  \renewcommand{\childdocforward}[2][]{}
  \renewcommand{\childdocdisable}{}
}
%    \end{macrocode}

% \macro{\childdocmain}
% The macro |\childdocmain| is to be called at the top of the main file
% with nothing or the main filename (without extension) as argument.
% First, it breaks loops.
% If the argument is not empty and does not match |\childdocname|
% (which is set by the first inclusion of |childdoc.def|),
% |\ifchilddoc| is set to true, |\includeonly| is applied to the child file
% and |\jobname| is set to the main file
% (for proper handling of |.aux| files):
%    \begin{macrocode}
\newcommand{\childdocmain}[1]
{
  \childdocdisable\childdocmain{}
  \if?#1?\else
    \begingroup
      \def\childdoctmp{#1}
      \ifx\childdoctmp\childdocname
        \def\childdoctmp{}
      \else
        \def\childdoctmp
        {
          \childdoctrue
          \includeonly{\childdocname}
          \def\childdocjob{#1}
          \def\jobname{#1}
        }
      \fi
      \expandafter
    \endgroup
    \childdoctmp
  \fi
}
%    \end{macrocode}

% \macro{\childdocof}
% The command |\childdocof| redirects
% compilation to the main file |#1|.
%    \begin{macrocode}
\newcommand{\childdocof}[1]
{
  \childdocdisable
  \childdoctrue
  \includeonly{\childdocname}
  \def\jobname{#1}
  \def\childdocjob{#1}
  \input{#1}
}
%    \end{macrocode}

% \macro{\childdocby}
% The command |\childdocby| ....
%    \begin{macrocode}
\newcommand{\childdocby}[2][]
{
  \childdocdisable
  \childdoctrue
  \childdocmanualtrue
  \if?#1?\else
    \def\jobname{#2}
  \fi
  \def\childdocjob{#2}
  \input{#2}
  \endinput
}
%    \end{macrocode}

% \macro{\childdocforward}
% The command |\childdocforward| redirects
% compilation to the main file or
% (if the optional argument is given) a child file.
% Parameters are set as if the main file
% or a child file starting with |\childdocof| was compiled.
% Then compilation is handed over to the main file:
%    \begin{macrocode}
\newcommand{\childdocforward}[2][]
{
  \begingroup
    \if?#1?
      \def\childdoctmp
      {
        \def\childdocname{#2}
        \def\childdocjob{#2}
        \def\jobname{#2}
        \input{#2}
        \endinput
      }
    \else
      \def\childdoctmp
      {
        \childdocdisable
        \def\childdocname{#2}
        \childdoctrue
        \includeonly{#2}
        \def\childdocjob{#1}
        \def\jobname{#1}
        \input{#1}
        \endinput
      }
    \fi
    \expandafter
  \endgroup
  \childdoctmp
}
%    \end{macrocode}

% \macro{\childdocforwardprefix}
% The command |\childdocforwardprefix| redirects
% compilation to the main or a child file by means of a pattern.
% The prefix |#1| in the current filename is replaced by |#2|
% and the suffix of the current filename is kept
% (it is assumed that the filename does not contain the substring `|~~~|'
% which is used as a delimiter).
% Compilation is handed over to the new file by |\childdocforward|:
%    \begin{macrocode}
\newcommand{\childdocforwardprefix}[3][]
{
  \begingroup
    \def\childdocextract #2##1~~~{\def\childdoctmp{\childdocforward[#1]{#3##1}}}
    \expandafter\childdocextract\childdocname~~~
    \expandafter
  \endgroup
  \childdoctmp
}
%    \end{macrocode}

% \macro{\childdoc}
% The deprecated macro |\childdoc| is a legacy version of |\childdocmain|:
%    \begin{macrocode}
\newcommand{\childdoc}{\childdocmain}
%    \end{macrocode}

% \macro{\childdocredirect}
% The deprecated macro |\childdocredirect| is a legacy version
% of |\childdocforward| and |\childdocforwardprefix|:
%    \begin{macrocode}
\newcommand{\childdocredirect}[2][]
{
  \begingroup
    \if?#1?
      \def\childdoctmp{\childdocforward{#2}}
    \else
      \def\childdoctmp{\childdocforwardprefix{#1}{#2}}
    \fi
    \expandafter
  \endgroup
  \childdoctmp
}
%    \end{macrocode}

%\iffalse
%</package>
%\fi
%
\endinput

\childdocby{cdocsamp}
%    \end{macrocode}

%\iffalse
%</samplepart3|samplepart4>
%\fi
%
%\iffalse
%<*samplepart3>
%\fi
% Some text for part 3:
%    \begin{macrocode}
some text in part three
%    \end{macrocode}

%\iffalse
%</samplepart3>
%\fi
% Some text for part 4:
%\iffalse
%<*samplepart4>
%\fi
%    \begin{macrocode}
more text in part four
%    \end{macrocode}

%\iffalse
%</samplepart4>
%\fi
%
% %%%%%%%%%%%%%%%%%%%%%%%%%%%%%%%%%%%%%%
% \paragraph{Forwarding for a Complete Draft.}
%
% The following forwarding file |cdocsdrf.tex|
% compiles the main document in draft mode:
%\iffalse
%<*sampledraft>
%\fi
%    \begin{macrocode}
\def\version{draft}
% \iffalse
%
% childdoc.dtx Copyright (C) 2017-2018 Niklas Beisert
%
% This work may be distributed and/or modified under the
% conditions of the LaTeX Project Public License, either version 1.3
% of this license or (at your option) any later version.
% The latest version of this license is in
%   http://www.latex-project.org/lppl.txt
% and version 1.3 or later is part of all distributions of LaTeX
% version 2005/12/01 or later.
%
% This work has the LPPL maintenance status `maintained'.
%
% The Current Maintainer of this work is Niklas Beisert.
%
% This work consists of the files childdoc.dtx and childdoc.ins
% and the derived files childdoc.def and cdocsamp.tex with
% cdocsch1.tex, cdocsch2.tex, cdocsdrf.tex, cdocsfn1.tex, cdocsfn2.tex.
%
%<package>\ifdefined\childdocmain\endinput\fi
%<package>\ProvidesFile{childdoc.def}[2018/12/30 v2.0 child document driver]
%<samplemain>\ProvidesFile{cdocsamp.tex}[2018/12/30 v2.0 sample for childdoc]
%<*driver>
%\ProvidesFile{childdoc.drv}[2018/12/30 v2.0 childdoc reference manual file]
\PassOptionsToClass{10pt,a4paper}{article}
\documentclass{ltxdoc}

\usepackage[margin=35mm]{geometry}
\usepackage{hyperref}
\usepackage{hyperxmp}
\usepackage[usenames]{color}

\hypersetup{colorlinks=true}
\hypersetup{pdfstartview=FitH}
\hypersetup{pdfpagemode=UseNone}
\hypersetup{pdfsource={}}
\hypersetup{pdflang={en-UK}}
\hypersetup{pdfcopyright={Copyright 2017-2018 Niklas Beisert.
  This work may be distributed and/or modified under the
  conditions of the LaTeX Project Public License, either version 1.3
  of this license or (at your option) any later version.}}
\hypersetup{pdflicenseurl={http://www.latex-project.org/lppl.txt}}
\hypersetup{pdfcontactaddress={ETH Zurich, ITP, HIT K,
  Wolfgang-Pauli-Strasse 27}}
\hypersetup{pdfcontactpostcode={8093}}
\hypersetup{pdfcontactcity={Zurich}}
\hypersetup{pdfcontactcountry={Switzerland}}
\hypersetup{pdfcontactemail={nbeisert@itp.phys.ethz.ch}}
\hypersetup{pdfcontacturl={http://people.phys.ethz.ch/\xmptilde nbeisert/}}

\newcommand{\secref}[1]{\hyperref[#1]{section \ref*{#1}}}

\parskip1ex
\parindent0pt
\let\olditemize\itemize
\def\itemize{\olditemize\parskip0pt}

\begin{document}

\title{The \textsf{childdoc} Package}
\hypersetup{pdftitle={The childdoc Package}}
\author{Niklas Beisert\\[2ex]
  Institut f\"ur Theoretische Physik\\
  Eidgen\"ossische Technische Hochschule Z\"urich\\
  Wolfgang-Pauli-Strasse 27, 8093 Z\"urich, Switzerland\\[1ex]
  \href{mailto:nbeisert@itp.phys.ethz.ch}
  {\texttt{nbeisert@itp.phys.ethz.ch}}}
\hypersetup{pdfauthor={Niklas Beisert}}
\hypersetup{pdfsubject={Manual for the LaTeX2e Package childdoc}}
\date{30 December 2018, \textsf{v2.0}}
\maketitle

\begin{abstract}\noindent
\textsf{childdoc} is a \LaTeXe{} package
that enables the direct compilation
of document sections included by |\include|
to individual files.
\end{abstract}

\begingroup
\parskip0ex
\tableofcontents
\endgroup

%%%%%%%%%%%%%%%%%%%%%%%%%%%%%%%%%%%%%%%%%%%%%%%%%%%%%%%%%%%%%%%%%%%%%%%%%%%%%%%%
%%%%%%%%%%%%%%%%%%%%%%%%%%%%%%%%%%%%%%%%%%%%%%%%%%%%%%%%%%%%%%%%%%%%%%%%%%%%%%%%
\section{Introduction}

\LaTeX{} provides a mechanism to structure a large document (such as a book)
into a main file and several child files (containing the chapters)
using the |\include| command.
This mechanism is beneficial for documents
which span hundreds of pages in order to
make the source file(s) more manageable.
Moreover, compilation can be restricted to
selected child files by means of the |\includeonly| command.
The latter feature can be used to reduce the compilation time while editing
(this was significantly more useful in the earlier days of \LaTeX{})
or to generate a smaller document which is easier to navigate.
Another application of |\includeonly| is to generate
documents consisting of selected parts of the complete document.

However, there are a few drawbacks of the plain |\include| mechanism:
\begin{itemize}
\item
The child files cannot be compiled on their own,
they can only be compiled via the main file.
A naive editing environment
(such as a text editor with an option
to have the current file processed by \LaTeX)
may require one to switch to the main file before compiling;
attempting to compile the child file produces errors.
\item
The main file must be modified (each time)
to adjust the |\includeonly| command
to the present needs. This easily leaves the main file in a messy state.
\item
The generated document will always carry the filename
of the main document. This is inconvenient if
several child files are to be compiled and
to be kept for distribution.
\end{itemize}

The present package provides a simple interface
to make child files individually compilable by \LaTeX{}.
Compiling a child file then has the same effect as compiling
the main file with an |\includeonly| command
to select the appropriate child.
Moreover the generated document will carry the name of the child
rather than the main file.
This resolves all three above issues.

This feature is meant to make the editing of books,
thesis documents and lecture notes somewhat more convenient.
However, the package can also be used efficiently for
composing a series of documents (such as exercise sheets)
which are typically distributed individually.
It then assists the author in generating the individual documents
(potentially in different versions)
as well as a document containing the collected series.
Another application is in developing style files
or other kinds of included material
where compilation of the style file could redirect
to a sample or test file.

%%%%%%%%%%%%%%%%%%%%%%%%%%%%%%%%%%%%%%%%%%%%%%%%%%%%%%%%%%%%%%%%%%%%%%%%%%%%%%%%
%%%%%%%%%%%%%%%%%%%%%%%%%%%%%%%%%%%%%%%%%%%%%%%%%%%%%%%%%%%%%%%%%%%%%%%%%%%%%%%%
\section{Usage}

First of all, the package \textsf{childdoc} is \emph{not} a standard
\LaTeXe{} |.sty| style file! Therefore it needs to be invoked in
a non-standard way.

%%%%%%%%%%%%%%%%%%%%%%%%%%%%%%%%%%%%%%%%%%%%%%%%%%%%%%%%%%%%%%%%%%%%%%%%%%%%%%%%
\subsection{Included Files}
\label{sec:include}

%%%%%%%%%%%%%%%%%%%%%%%%%%%%%%%%%%%%%%%%
\DescribeMacro{\childdocmain}
To use the package, add the commands
\begin{center}
\begin{tabular}{l}
|\input{childdoc.def}|\\
|\childdocmain{}|\\
\end{tabular}
\end{center}
at the very top of the main \LaTeX{} file,
in particular \emph{before} the |\documentclass| statement!
The argument of |\childdocmain| should be left empty
(but it must be present).

%%%%%%%%%%%%%%%%%%%%%%%%%%%%%%%%%%%%%%%%
\DescribeMacro{\childdocof}
Furthermore, add the commands
\begin{center}
\begin{tabular}{l}
|\input{childdoc.def}|\\
|\childdocof{|\textit{main}|}|\\
\end{tabular}
\end{center}
at the top of every child file \textit{child}
which is included by |\include{|\textit{child}|}|
from within the main file
(or at least for those files to be compiled individually).
The argument \textit{main} must be the filename of the main file.

There are a couple of
considerations in setting up the main and child documents:

%%%%%%%%%%%%%%%%%%%%%%%%%%%%%%%%%%%%%%%%
\paragraph{Restrictions.}

Please note the following restrictions:
\begin{itemize}
\item
|\childdocmain| must be called with one argument \textit{main}
to ensure compatibility with earlier version of the package.
It must either be empty (|\childdocmain{}|)
or precisely match the filename of the main file in which it is specified.
See \secref{sec:detection} for further information.
\item
The filename \textit{main} must be specified without the |.tex| extension.
\item
The filename \textit{main} is case sensitive
(even in case-insensitive file systems)
due to internal string comparison.
\item
The argument \textit{main} should be fully expanded, it cannot be a macro.
\item
Subdirectories and special characters should be avoided in filenames.
\item
The command |\childdocmain{|\textit{main}|}| must be followed by a whitespace.
It should not be followed immediately by another command
or by a comment mark `|%|'.
This is because the \TeX{} parser reads the token immediately following
the argument of |\childdocmain| and puts it
at the beginning of every child section;
however, a white\-space is ignored.
\end{itemize}

%%%%%%%%%%%%%%%%%%%%%%%%%%%%%%%%%%%%%%%%
\paragraph{Content of Main File.}

It is advisable to place all content in the child files included by |\include|.
Any output contained in the main file will appear in all child documents
unless suppressed manually;
it cannot be suppressed automatically by the |\includeonly| directive
and thus should normally be avoided.
A method to include some content in the main file
by means of conditional processing is described in \secref{sec:conditional}.

%%%%%%%%%%%%%%%%%%%%%%%%%%%%%%%%%%%%%%%%
\paragraph{Page Numbering.}

When only a part of the document is compiled,
the appropriate numbering of pages
(as well as other status parameters)
is determined from the |.aux| files.
The latter contain information from previous passes.
However this information needs to propagate through
all intermediate child documents.
Therefore the page numbering in child documents may well
be inconsistent until the complete document is compiled at least once.

A useful (if unconventional) way to always ensure a consistent
page numbering is to restart the numbering in each child document
and denote the pages by `\textit{child}|.|\textit{page}'
where \textit{child} represents the chapter/section number of the child file.
This can be achieved by the command
|\numberwithin{page}{|\textit{child}|}|
of the \textsf{amsmath} package
where \textit{child} can be |chapter| or |section|
depending on the chosen structuring.
Alternatively, one can modify the macro |\thepage| appropriately
and reset the counter |page| at the start of each child file.

%%%%%%%%%%%%%%%%%%%%%%%%%%%%%%%%%%%%%%%%%%%%%%%%%%%%%%%%%%%%%%%%%%%%%%%%%%%%%%%%
\subsection{Conditional Processing}
\label{sec:conditional}

The package provides a mechanism to compile different versions
of a document. To customise the versions further some conditional processing
can come in handy to distinguish which version is being compiled.
The package provides two macros to describe the compilation context:

%%%%%%%%%%%%%%%%%%%%%%%%%%%%%%%%%%%%%%%%
\DescribeMacro{\ifchilddoc}
The conditional |\ifchilddoc| distinguishes between the compilation of
child documents and the main document:
%
\begin{center}
|\ifchilddoc |\textit{child-code}| |[|\||else |\textit{main-code}]| \||fi|
\end{center}

%%%%%%%%%%%%%%%%%%%%%%%%%%%%%%%%%%%%%%%%
\DescribeMacro{\childdocname}
\DescribeMacro{\childdocjob}
The macro |\childdocname| contains the filename (without extension)
of the main or child file being processed.
Note that |\childdocjob| will always contain the name of the main file.

%%%%%%%%%%%%%%%%%%%%%%%%%%%%%%%%%%%%%%%%
\paragraph{Title Page.}

Conditional processing can be used to include a title or banner page
in the main document when proper precautions are taken.
Importantly, the code in the main file should ensure that the page counter
(as well as other status parameters which are stored in the |.aux| files)
takes the same value after the conditional processing.
Otherwise the page numbers may take divergent values
depending on which part is compiled.

For example, a title page could be declared by:
%
\begin{center}
\begin{tabular}{l}
|\ifchilddoc\||else|\\
|\addtocounter{page}{-1}|\\
\textit{code for title page}\\
|\newpage|\\
|\||fi|
\end{tabular}
\end{center}
%
A banner page for the child documents can be generated by:
%
\begin{center}
\begin{tabular}{l}
|\ifchilddoc|\\
|\addtocounter{page}{-1}|\\
\textit{code for banner page}\\
|\newpage|\\
|\||fi|
\end{tabular}
\end{center}
%
Here one could write a message such as:
\begin{center}
|This is the part \childdocname{} of \childdocjob{}.|
\end{center}

%%%%%%%%%%%%%%%%%%%%%%%%%%%%%%%%%%%%%%%%%%%%%%%%%%%%%%%%%%%%%%%%%%%%%%%%%%%%%%%%
\subsection{Flags}
\label{sec:flags}

The package makes it easy to generate different versions
of the main or child documents.
To this end compilation flags can be defined
and assigned different default values.
They will be particularly useful in conjunction
with the forwarding mechanism described in \secref{sec:forward}.

For example, it may be useful to have a flag |\version|
which can be set to |draft| or |final|.
The document source will contain some conditional code
depending on the value of |\version|.
Suppose further, the flag should default to |final| for the main file
and to |draft| for child files
which is a natural assignment for editing the document.
This is achieved by placing the following code
in the preamble of the main document
(below the |\childdocmain| directive):
%
\begin{center}
\begin{tabular}{l}
|\ifchilddoc|\\
|\providecommand{\version}{draft}|\\
|\||else|\\
|\providecommand{\version}{final}|\\
|\||fi|
\end{tabular}
\end{center}
%
The definition by |\providecommand| makes sure
that previous definitions are not overwritten.
Further statements |\providecommand{\version}{...}|
can thus be added before the above code to override it.

For the main file, one might add a line
(between |\childdocmain| and the above block)
%
\begin{center}
|%\ifchilddoc\||else\providecommand{\version}{draft}\||fi|
\end{center}
%
which can be uncommented to produce a draft version.
Likewise one can add a line to the very top of a child file
(above the |\childdocof{|\textit{main}|}| directive)
%
\begin{center}
|%\providecommand{\version}{final}|
\end{center}
%
which can be uncommented to produce the final version of this child document.

%%%%%%%%%%%%%%%%%%%%%%%%%%%%%%%%%%%%%%%%%%%%%%%%%%%%%%%%%%%%%%%%%%%%%%%%%%%%%%%%
\subsection{Forwarding}
\label{sec:forward}

Different versions of the main or child documents
using compilation flags as described in \secref{sec:flags}
can be (permanently) stored in different files
for convenient compilation, viewing and distribution.
To this end, the package defines a command
to pass on compilation to a different file:

%%%%%%%%%%%%%%%%%%%%%%%%%%%%%%%%%%%%%%%%
\DescribeMacro{\childdocforward}
The command |\childdocforward| redirects processing to
another source file:
%
\begin{center}
\begin{tabular}{l}
|\input{childdoc.def}|\\
|\childdocforward[|\textit{main}|]{|\textit{dest}|}|\\
\end{tabular}
\end{center}
%
The argument \textit{dest} is the destination file
(without extension).
It should be the main file or one of the child files.
Note that further \textsf{childdoc} directives
such as |\childdocof| and |\childdocforward|
in the indicated file will be processed in this form.
The optional argument \textit{main}
passes on directly to the main file \textit{main}
while pretending to compile the child \textit{dest}.
This form behaves as if \textit{dest}
issues |\childdocof{|\textit{main}|}| right away,
and no further \textsf{childdoc} directives will be processed.

%%%%%%%%%%%%%%%%%%%%%%%%%%%%%%%%%%%%%%%%
\DescribeMacro{\...prefix}
In the alternative form |\childdocforwardprefix|,
%
\begin{center}
\begin{tabular}{l}
|\input{childdoc.def}|\\
|\childdocforwardprefix[|\textit{main}|]{|\textit{prefix}|}{|\textit{dest}|}|
\end{tabular}
\end{center}
%
the destination file is determined by a pattern
depending on the current file:
To make this work, the current file must be called
`{\textit{prefix}\hspace{0.2em}\textit{suffix}}'
with \textit{prefix} matching precisely the argument.
Processing is then passed on to the file
`{\textit{dest}\hspace{0.2em}\textit{suffix}}'.
Surely, the same effect is achieved by
directly specifying the
argument `{\textit{dest}\hspace{0.2em}\textit{suffix}}'
in the first form.
However, that requires to set up a different file
for each child. With the alternative form of the command
all these files can have exactly the same content
which simplifies setting them up and maintaining them.

For example, the following file |draft.tex|
with a compilation flag |\version| as described in \secref{sec:flags}
compiles the main document as a draft:
%
\begin{center}
\begin{tabular}{l}
|\def\version{draft}|\\
|\input{childdoc.def}|\\
|\childdocforward{|\textit{main}|}|
\end{tabular}
\end{center}
%
Likewise, the following files |final|\textit{nn}|.tex|
compile the final version of the child document
|child|\textit{nn}|.tex|:
%
\begin{center}
\begin{tabular}{l}
|\def\version{final}|\\
|\input{childdoc.def}|\\
|\childdocforwardprefix{final}{child}|
\end{tabular}
\end{center}
%

Note that when several versions of a main file and/or of each child file
are to be generated, it may be convenient to set up a |Makefile| or
shell script to automatise the process.

%%%%%%%%%%%%%%%%%%%%%%%%%%%%%%%%%%%%%%%%%%%%%%%%%%%%%%%%%%%%%%%%%%%%%%%%%%%%%%%%
\subsection{Command Line Processing}
\label{sec:commandline}

The effect of redirection files can also be achieved by invoking
the \LaTeX{} compiler with a more elaborate command line.
Most conveniently this should be done as part
of a shell script or a |Makefile|.

When using \textsf{childdoc} in the main file, the following
command lines effectively perform a redirection
(note that depending on the shell being used,
backslashes may have to be doubled: `|\|' $\to$ `|\\|'):
%
\begin{center}
|... -jobname "|\textit{target}|" |\\|"|[\textit{flags}]%
|\input{childdoc.def}\childdocforward[|\textit{main}|]{|\textit{dest}|}"|
\end{center}
%
Here \textit{target} is the name of the output file,
\textit{main} is the name of the main file
and \textit{dest} is the name of the main or child file to be processed
(all filenames without extensions).
The optional argument \textit{main} can be omitted
if \textit{main} matches \textit{dest}.
Optionally, compilation \textit{flags} can be defined via |\def| commands.
This command line makes the \TeX{} engine believe
it is compiling the file \textit{target}
whose content is specified as the latter parameter.
The provided code then forwards the processing to
\textit{main} or \textit{dest} as described in \secref{sec:forward}.

%%%%%%%%%%%%%%%%%%%%%%%%%%%%%%%%%%%%%%%%%%%%%%%%%%%%%%%%%%%%%%%%%%%%%%%%%%%%%%%%
\subsection{Include by Input}
\label{sec:input}

Including child documents by |\include| has some restrictions by design.
Most notably, the content of a child document always occupies
its own set of pages; pages cannot be shared between child documents.
Usually, this behaviour makes perfect sense
because each child document contain an essential part of the document.
However, in some situations it may be desirable to compose
a document from a collection of parts
without having mandatory page breaks between then.
For this case, the package
provides a mechanism to include parts
by |\input| which can also be processed individually.
However, by construction this mechanism
requires manual handling of the content to be output.

%%%%%%%%%%%%%%%%%%%%%%%%%%%%%%%%%%%%%%%%
\DescribeMacro{\ifchilddocmanual}
The main file should be prepared as usual, see \secref{sec:include}.
However, the document body must make a distinction
between processing of an individual part and of the main document, e.g.:
%
\begin{center}
\begin{tabular}{l}
|\ifchilddocmanual|\\
|\input{\childdocname}|\\
|\||else|\\
\textit{document body with }|\input{|\textit{part}|}|\\
|\||fi|
\end{tabular}
\end{center}
%
The conditional |\ifchilddocmanual| is true whenever
a part to be included by |\input| is being compiled,
and the name of the part is stored in |\childdocname|.

%%%%%%%%%%%%%%%%%%%%%%%%%%%%%%%%%%%%%%%%
\DescribeMacro{\childdocby}
Each part to be included by |\input| should start with:
%
\begin{center}
\begin{tabular}{l}
|\input{childdoc.def}|\\
|\childdocby{|\textit{main}|}|\\
\end{tabular}
\end{center}
%
The directive |\childdocby| is similar to |\childdocof|
described in \secref{sec:include},
but the subsequent selection of content must be done manually.
To that end, both |\ifchilddoc| and |\ifchilddocmanual|
will be true upon processing of a part,
and the name of the part is stored in |\childdocname|.
Note that |\jobname| will be set to the filename of the current part
so that each part receives an individual |.aux| file
that does not interfere with the |.aux| file(s) of the main document.
This behaviour can be altered by the alternative form
|\childdocby[*]{|\textit{main}|}| (with a non-empty optional argument)
which uses the |.aux| file of the main document
by setting |\jobname| to \textit{main}.

%%%%%%%%%%%%%%%%%%%%%%%%%%%%%%%%%%%%%%%%%%%%%%%%%%%%%%%%%%%%%%%%%%%%%%%%%%%%%%%%
\subsection{Driver Development}
\label{sec:driver}

The \textsf{childdoc} mechanism can also be use for the development
of definition files such as \LaTeX{} styles or classes.
This case differs from the above setup with multiple parts
included by |\include| in that no |\includeonly| should be invoked.
This can be achieved by starting the include file
(before |\ProvidesPackage|) with:
%
\begin{center}
\begin{tabular}{l}
|\input{childdoc.def}|\\
|\childdocforward{|\textit{main}|}|\\
\end{tabular}
\end{center}
%
or alternatively with:
%
\begin{center}
\begin{tabular}{l}
|\input{childdoc.def}|\\
|\childdocby{|\textit{main}|}|\\
\end{tabular}
\end{center}
%
Both forms have slightly different effects as described above.
The main file is prepared as usual, see \secref{sec:include}.

%%%%%%%%%%%%%%%%%%%%%%%%%%%%%%%%%%%%%%%%%%%%%%%%%%%%%%%%%%%%%%%%%%%%%%%%%%%%%%%%
\subsection{Legacy Detection}
\label{sec:detection}

The directive |\childdocmain| in the main file can detect
whether the complete document or merely a child is to be compiled
even without using the directive |\childdocof|.
This method is deprecated because it is less robust
and there is no compelling reason to use it;
it is merely provided for backward compatibility
and it may be removed in future versions.

If the detection mechanism is to be used,
it is mandatory to correctly specify
the filename of the main file as the argument of |\childdocmain|:
%
\begin{center}
\begin{tabular}{l}
|\input{childdoc.def}|\\
|\childdocmain{|\textit{main}|}|\\
\end{tabular}
\end{center}
%
If |\jobname| does not match the argument \textit{main} of |\childdocmain|,
it is assumed that |\jobname| points to the child file to be compiled.
When using |\childdocmain| with the main file specified as argument,
it suffices to start a child file
with just |\input{|\textit{main}|}|
without loading of the package and using |\childdocof|.
If instead all processing is done
with the appropriate \textsf{childdoc} directives,
the argument of \textit{main} of |\childdocmain| can be empty.

An alternative version of the command line processing described
in \secref{sec:commandline} using the detection mechanism reads:
%
\begin{center}
|... -jobname "|\textit{target}|" "|[\textit{flags}]%
[|\def\jobname{|\textit{dest}|}|]|\input{|\textit{main}|}"|
\end{center}

%%%%%%%%%%%%%%%%%%%%%%%%%%%%%%%%%%%%%%%%%%%%%%%%%%%%%%%%%%%%%%%%%%%%%%%%%%%%%%%%
\subsection{Manual Code}
\label{sec:manual}

In case one cannot be certain whether the definitions file |childdoc.def|
is installed on the target \TeX{} distribution
and one prefers not to ship it,
it is conceivable to paste a few relevant commands into the sources.

To that end, drop all statements |\input{childdoc.def}|
and perform the replacements as outlined below.
Instead of |\childdocmain{|\textit{main}|}| add the following code
to the top of the main file:
%
\begin{center}
\begin{tabular}{l}
|\||ifdefined\childdocname\endinput\||fi\newif\ifchilddoc|\\
|\edef\childdocname{\scantokens\expandafter{\jobname\noexpand}}|\\
|\def\childdocmain{|\textit{main}|}\||ifx\childdocmain\childdocname\||else|\\
|\childdoctrue\includeonly{\childdocname}\let\jobname\childdocmain\||fi|\\
\end{tabular}
\end{center}
%
Instead of |\childdocof{|\textit{main}|}| just include the main file
at the top of each child file:
%
\begin{center}
|\input{|\textit{main}|}|
\end{center}
%
A simple redirection |\childdocforward{|\textit{dest}|}| is achieved by:
%
\begin{center}
|\def\jobname{|\textit{dest}|}\input{\jobname}|
\end{center}
%
The redirection with prefix
|\childdocforwardprefix[|\textit{prefix}|]{|\textit{dest}|}|
is accomplished by:
%
\begin{center}
\begin{tabular}{l}
|{\edef\jobname{\scantokens\expandafter{\jobname\noexpand}}|\\
|\def\redirectjob |\textit{prefix}|#1~~~{\gdef\jobname{|\textit{dest}|#1}}|\\
|\expandafter\redirectjob\jobname~~~}\input{\jobname}|
\end{tabular}
\end{center}

In an alternative approach,
child documents can be compiled by a specific command line
without additional code or specific definitions:
%
\begin{center}
|... -jobname "|\textit{target}|" "|[\textit{flags}]%
|\includeonly{|\textit{dest}|}\input{|\textit{main}|}"|
\end{center}
%

%%%%%%%%%%%%%%%%%%%%%%%%%%%%%%%%%%%%%%%%%%%%%%%%%%%%%%%%%%%%%%%%%%%%%%%%%%%%%%%%
%%%%%%%%%%%%%%%%%%%%%%%%%%%%%%%%%%%%%%%%%%%%%%%%%%%%%%%%%%%%%%%%%%%%%%%%%%%%%%%%
\section{Information}

%%%%%%%%%%%%%%%%%%%%%%%%%%%%%%%%%%%%%%%%%%%%%%%%%%%%%%%%%%%%%%%%%%%%%%%%%%%%%%%%
\subsection{Copyright}

Copyright \copyright{} 2017--2018 Niklas Beisert

This work may be distributed and/or modified under the
conditions of the \LaTeX{} Project Public License, either version 1.3
of this license or (at your option) any later version.
The latest version of this license is in
  \url{http://www.latex-project.org/lppl.txt}
and version 1.3 or later is part of all distributions of \LaTeX{}
version 2005/12/01 or later.

This work has the LPPL maintenance status `maintained'.

The Current Maintainer of this work is Niklas Beisert.

This work consists of the files |README.txt|, |childdoc.ins| and |childdoc.dtx|
as well as the derived files |childdoc.def|, |cdocsamp.tex|
with |cdocsch1.tex|, |cdocsch2.tex|, |cdocspt3.tex|, |cdocspt4.tex|,
|cdocsdrf.tex|, |cdocsfn1.tex|, |cdocsfn2.tex|
as well as |childdoc.pdf|.

%%%%%%%%%%%%%%%%%%%%%%%%%%%%%%%%%%%%%%%%%%%%%%%%%%%%%%%%%%%%%%%%%%%%%%%%%%%%%%%%
\subsection{Files and Installation}

The package consists of the files:
%
\begin{center}
\begin{tabular}{ll}
    |README.txt|   & readme file \\
    |childdoc.ins| & installation file \\
    |childdoc.dtx| & source file \\
    |childdoc.def| & definition file \\
    |cdocsamp.tex| & sample main file \\
    |cdocsch1.tex| & sample include file \\
    |cdocsch2.tex| & sample include file \\
    |cdocspt3.tex| & sample part file \\
    |cdocspt4.tex| & sample part file \\
    |cdocsdrf.tex| & sample redirection file \\
    |cdocsfn1.tex| & sample redirection file \\
    |cdocsfn2.tex| & sample redirection file \\
    |childdoc.pdf| & manual
\end{tabular}
\end{center}
%
The distribution consists of the files
|README.txt|, |childdoc.ins| and |childdoc.dtx|.
%
\begin{itemize}
\item
Run (pdf)\LaTeX{} on |childdoc.dtx|
to compile the manual |childdoc.pdf| (this file).
\item
Run \LaTeX{} on |childdoc.ins| to create the definitions file |childdoc.def|
and the sample |cdocsamp.tex| with include files
|cdocsch1.tex|, |cdocsch2.tex|, |cdocspt3.tex|, |cdocspt4.tex|,
|cdocsdrf.tex|, |cdocsfn1.tex|, |cdocsfn2.tex|.
Then copy the file |childdoc.def| to an appropriate directory of your \LaTeX{}
distribution, e.g.\ \textit{texmf-root}|/tex/latex/childdoc|.
\end{itemize}

%%%%%%%%%%%%%%%%%%%%%%%%%%%%%%%%%%%%%%%%%%%%%%%%%%%%%%%%%%%%%%%%%%%%%%%%%%%%%%%%
\subsection{Related CTAN Packages}

There are several other packages which offer a similar functionality:
%
\begin{itemize}
\item
The packages
\href{http://ctan.org/pkg/docmute}{\textsf{docmute}},
\href{http://ctan.org/pkg/includex}{\textsf{includex}} and
\href{http://ctan.org/pkg/standalone}{\textsf{standalone}}
provide commands to include only the document body of
a child file thus allowing both files to be compiled individually.
\item
The packages \href{http://ctan.org/pkg/subdocs}{\textsf{subdocs}}
and \href{http://ctan.org/pkg/subfiles}{\textsf{subfiles}}
provide structures in which the main and child documents can be
encapsulated and allowing them to be compiled individually.
The inclusion mechanism is different from the conventional |\include|.
\item
The package \href{http://ctan.org/pkg/combine}{\textsf{combine}}
is an elaborate solution to combine several documents into one.
\end{itemize}
%
See also the CTAN topic \href{http://ctan.org/topic/subdocs}{\textsf{subdocs}}
for further related packages.
The present package differs from the above solutions in that
a document structure constructed with the conventional |\include| mechanism
just needs two extra commands at the top of every file
such that all constituent files can be compiled individually.

%%%%%%%%%%%%%%%%%%%%%%%%%%%%%%%%%%%%%%%%%%%%%%%%%%%%%%%%%%%%%%%%%%%%%%%%%%%%%%%%
%\subsection{Feature Suggestions}
%
%The following is a list of features which may be useful for future
%versions of this package:
%%
%\begin{itemize}
%\item
%\ldots
%\end{itemize}

%%%%%%%%%%%%%%%%%%%%%%%%%%%%%%%%%%%%%%%%%%%%%%%%%%%%%%%%%%%%%%%%%%%%%%%%%%%%%%%%
\subsection{Revision History}

%%%%%%%%%%%%%%%%%%%%%%%%%%%%%%%%%%%%%%%%
\paragraph{v2.0:} 2018/12/30

\begin{itemize}
\item
immediate forward processing
\item
added |\childdocby| mechanism
\item
manual restructured
\end{itemize}

%%%%%%%%%%%%%%%%%%%%%%%%%%%%%%%%%%%%%%%%
\paragraph{v1.6:} 2018/01/17

\begin{itemize}
\item
application for development of include files
\item
corrections to manual
\end{itemize}

%%%%%%%%%%%%%%%%%%%%%%%%%%%%%%%%%%%%%%%%
\paragraph{v1.5:} 2017/05/21

\begin{itemize}
\item
more complete structuring introduced
\item
|\childdocof| introduced
\item
|\childdoc| renamed to |\childdocmain|
\item
|\childredirect| renamed to |\childdocforward| and |\childdocforwardprefix|
and functionality expanded
\end{itemize}

%%%%%%%%%%%%%%%%%%%%%%%%%%%%%%%%%%%%%%%%
\paragraph{v1.0:} 2017/04/27

\begin{itemize}
\item
manual and install package
\item
first version published on CTAN
\end{itemize}

%%%%%%%%%%%%%%%%%%%%%%%%%%%%%%%%%%%%%%%%
\paragraph{v0.6:} 2017/04/26

\begin{itemize}
\item
redirection mechanism added
\end{itemize}

%%%%%%%%%%%%%%%%%%%%%%%%%%%%%%%%%%%%%%%%
\paragraph{v0.5:} 2017/04/26

\begin{itemize}
\item
functionality in definition file
\end{itemize}


%%%%%%%%%%%%%%%%%%%%%%%%%%%%%%%%%%%%%%%%%%%%%%%%%%%%%%%%%%%%%%%%%%%%%%%%%%%%%%%%
%%%%%%%%%%%%%%%%%%%%%%%%%%%%%%%%%%%%%%%%%%%%%%%%%%%%%%%%%%%%%%%%%%%%%%%%%%%%%%%%
%%%%%%%%%%%%%%%%%%%%%%%%%%%%%%%%%%%%%%%%%%%%%%%%%%%%%%%%%%%%%%%%%%%%%%%%%%%%%%%%
\appendix

\settowidth\MacroIndent{\rmfamily\scriptsize 000\ }

 \DocInput{childdoc.dtx}

\end{document}
%</driver>
% \fi
%
% %%%%%%%%%%%%%%%%%%%%%%%%%%%%%%%%%%%%%%%%%%%%%%%%%%%%%%%%%%%%%%%%%%%%%%%%%%%%%%
% %%%%%%%%%%%%%%%%%%%%%%%%%%%%%%%%%%%%%%%%%%%%%%%%%%%%%%%%%%%%%%%%%%%%%%%%%%%%%%
% \section{Sample}
%\iffalse
%<*samplemain>
%\fi
%
% The following presents a sample document
% with two chapters, two parts, a title page,
% a compile flag as well as three forwarding files to set the flag.
% It consists of eight |.tex| files:
% \begin{center}
% \begin{tabular}{ll}
% |cdocsamp.tex|&main file\\
% |cdocsch1.tex|&include file for chapter 1\\
% |cdocsch2.tex|&include file for chapter 2\\
% |cdocspt3.tex|&include file for part 3\\
% |cdocspt4.tex|&include file for part 4\\
% |cdocsdrf.tex|&forwarding file for main file in draft mode\\
% |cdocsfi1.tex|&forwarding file for final version of chapter 1\\
% |cdocsfi2.tex|&forwarding file for final version of chapter 2\\
% \end{tabular}
% \end{center}
% Each of the eight files can be compiled directly by the \LaTeX{} compiler.
%
% %%%%%%%%%%%%%%%%%%%%%%%%%%%%%%%%%%%%%%
% \paragraph{Main File.}
%
% The main file is called |cdocsamp.tex|.
%
% Load the \textsf{childdoc} definitions and
% declare the filename for the main document:
%    \begin{macrocode}
\input{childdoc.def}
\childdocmain{}
%    \end{macrocode}

% Optional override for |\version| flag:
%    \begin{macrocode}
%%\ifchilddoc\else\providecommand{\version}{draft}\fi
%    \end{macrocode}

% Define the default values for the |\version| flag
% (|final| for the main file and |draft| for childs):
%    \begin{macrocode}
\ifchilddoc
\providecommand{\version}{draft}
\else
\providecommand{\version}{final}
\fi
%    \end{macrocode}

% Load the standard document class:
%    \begin{macrocode}
\documentclass[12pt]{article}
%    \end{macrocode}

% Start the document body:
%    \begin{macrocode}
\begin{document}
%    \end{macrocode}

% Declare a title page.
% Print title, part of document being processed and version flag:
%    \begin{macrocode}
\addtocounter{page}{-1}
\begin{center}
{\LARGE\bfseries{}childdoc example\par}
\vspace{1cm}
\ifchilddoc
\ifchilddocmanual part\else chapter\fi:
`\childdocname' of `\childdocjob'\par
\else
main document: `\childdocjob'\par
\fi
version: \version\par
\end{center}
\newpage
%    \end{macrocode}

% Manually include selected file,
% otherwise process as usual:
%    \begin{macrocode}
\ifchilddocmanual
\section*{part `\childdocname'}
\input{\childdocname}
\else
%    \end{macrocode}

% Include the two chapters:
%    \begin{macrocode}
\include{cdocsch1}
\include{cdocsch2}
%    \end{macrocode}

% Include the two parts unless only chapters should be displayed:
%    \begin{macrocode}
\ifchilddoc\else
\section{part three}
\input{cdocspt3}
\section{part four}
\input{cdocspt4}
\fi
%    \end{macrocode}

% Process as usual until here:
%    \begin{macrocode}
\fi
%    \end{macrocode}

% End of document body:
%    \begin{macrocode}
\end{document}
%    \end{macrocode}
%\iffalse
%</samplemain>
%\fi
%
% %%%%%%%%%%%%%%%%%%%%%%%%%%%%%%%%%%%%%%
% \paragraph{Chapter Include Files.}
%
% The include files are called |cdocsch1.tex| and |cdocsch2.tex|.
%
%\iffalse
%<*samplechap1|samplechap2>
%\fi

% Optional override for |\version| flag:
%    \begin{macrocode}
%%\providecommand{\version}{final}
%    \end{macrocode}

% Include the main document:
%    \begin{macrocode}
\input{childdoc.def}
\childdocof{cdocsamp}
%    \end{macrocode}

%\iffalse
%</samplechap1|samplechap2>
%\fi
%
%\iffalse
%<*samplechap1>
%\fi
% Some text for chapter 1:
%    \begin{macrocode}
\section{one}
some text in chapter one
%    \end{macrocode}

%\iffalse
%</samplechap1>
%\fi
% Some text for chapter 2:
%\iffalse
%<*samplechap2>
%\fi
%    \begin{macrocode}
\section{two}
more text in chapter two
%    \end{macrocode}

%\iffalse
%</samplechap2>
%\fi
%
% %%%%%%%%%%%%%%%%%%%%%%%%%%%%%%%%%%%%%%
% \paragraph{Part Include Files.}
%
% The include files are called |cdocspt3.tex| and |cdocspt4.tex|.
%
%\iffalse
%<*samplepart3|samplepart4>
%\fi

% Optional override for |\version| flag:
%    \begin{macrocode}
%%\providecommand{\version}{final}
%    \end{macrocode}

% Include the main document:
%    \begin{macrocode}
\input{childdoc.def}
\childdocby{cdocsamp}
%    \end{macrocode}

%\iffalse
%</samplepart3|samplepart4>
%\fi
%
%\iffalse
%<*samplepart3>
%\fi
% Some text for part 3:
%    \begin{macrocode}
some text in part three
%    \end{macrocode}

%\iffalse
%</samplepart3>
%\fi
% Some text for part 4:
%\iffalse
%<*samplepart4>
%\fi
%    \begin{macrocode}
more text in part four
%    \end{macrocode}

%\iffalse
%</samplepart4>
%\fi
%
% %%%%%%%%%%%%%%%%%%%%%%%%%%%%%%%%%%%%%%
% \paragraph{Forwarding for a Complete Draft.}
%
% The following forwarding file |cdocsdrf.tex|
% compiles the main document in draft mode:
%\iffalse
%<*sampledraft>
%\fi
%    \begin{macrocode}
\def\version{draft}
\input{childdoc.def}
\childdocforward{cdocsamp}
%    \end{macrocode}

%\iffalse
%</sampledraft>
%\fi
%
% %%%%%%%%%%%%%%%%%%%%%%%%%%%%%%%%%%%%%%
% \paragraph{Forwarding for Final Version of the Chapters.}
%
% The following forwarding files |cdocsfn1.tex| and |cdocsfn2.tex|
% (with identical content)
% compile the final versions of the child documents
% |cdocsch1.tex| and |cdocsch2.tex|, respectively:
%\iffalse
%<*samplefinal>
%\fi
%    \begin{macrocode}
\def\version{final}
\input{childdoc.def}
\childdocforwardprefix[cdocsamp]{cdocsfn}{cdocsch}
%    \end{macrocode}

%\iffalse
%</samplefinal>
%\fi
%
% %%%%%%%%%%%%%%%%%%%%%%%%%%%%%%%%%%%%%%
% \paragraph{Command Line Processing.}
%
% The following three command lines generate the output files
% |cdocscld|, |cdocscl1| and |cdocscl2|
% which should be identical to
% |cdocsdrf|, |cdocsch1| and |cdocsfn2|, respectively:
% \begin{center}
% \begin{tabular}{l}
% |latex -jobname cdocscld \|\\
% |  "\def\version{draft}\input{childdoc.def}\childdocforward{cdocsamp}"|\\
% |latex -jobname cdocscl1 \|\\
% |  "\input{childdoc.def}\childdocforward[cdocsamp]{cdocsch1}"|\\
% |latex -jobname cdocscl2 \|\\
% |  "\def\version{final}\input{childdoc.def}\childdocforward{cdocsch2}"|
% \end{tabular}
% \end{center}
% Note that the trailing backslash on each first line
% merely continues the input to the second line
% (for convenient cut ant paste).
% Furthermore, the command |latex| can be replaced by any
% of its alternative versions such as |pdflatex|.
%
% %%%%%%%%%%%%%%%%%%%%%%%%%%%%%%%%%%%%%%%%%%%%%%%%%%%%%%%%%%%%%%%%%%%%%%%%%%%%%%
% %%%%%%%%%%%%%%%%%%%%%%%%%%%%%%%%%%%%%%%%%%%%%%%%%%%%%%%%%%%%%%%%%%%%%%%%%%%%%%
% \section{Implementation}
%\iffalse
%<*package>
%\fi
%
% This section describes the definitions file |childdoc.def|.

% The definitions cannot be loaded using |\usepackage| or |\RequirePackage|
% which has a mechanism to prevent loading a style file more than once.
% When loading the definitions by means of |\input|
% multiple instances have to be prevented manually:
%\iffalse
%This code needs to be before the `\ProvidesFile' directive
%which is defined at the beginning of this file.
%Therefore it is also placed there and commented out here.
%</package>
%<*discard>
%\fi
%    \begin{macrocode}
\ifdefined\childdocmain\endinput\fi
%    \end{macrocode}
%\iffalse
%</discard>
%<*package>
%\fi
%
% \macro{\ifchilddoc}
% \macro{\ifchilddocmanual}
% The conditional |\ifchilddoc| tells whether a
% child (true) or main (false) document is being compiled.
% The conditional |\ifchilddocmanual| tells whether
% the |\includeonly| mechanism is used (false) or
% the selection of child files must be performed manually (true).
% The definitions initialise to false:
%    \begin{macrocode}
\newif\ifchilddoc
\newif\ifchilddocmanual
%    \end{macrocode}

% \macro{\childdocname}
% \macro{\childdocjob}
% The macro |\childdocname| stores the name of the main document
% to be compiled. The macro |\childdocjob| stores the name of
% the document on which the \LaTeX{} compiler was originally invoked.
% The content of |\jobname| cannot be compared
% to filenames specified in the source due to different catcodes.
% The following code rescans |\jobname|, stores the result
% in |\childdocname| and saves a copy in |\childdocjob|:
%    \begin{macrocode}
\edef\childdocname{\scantokens\expandafter{\jobname\noexpand}}
\let\childdocjob\childdocname
%    \end{macrocode}

% \macro{\childdocdisable}
% The macro |\childdocdisable| prevents the main file
% from being processed more than once.
% At this stage, the main document command |\childdocmain|
% is assumed to be called once again where it should do nothing.
% Any subsequent call to it should prevent
% a secondary processing of the main document
% It overwrites the forwarding commands
% |\childdocof| and |\childdocforward|
% with empty macros to prevent further inclusions of the main document:
%    \begin{macrocode}
\newcommand{\childdocdisable}
{
  \renewcommand{\childdocmain}[1]{\renewcommand{\childdocmain}[1]{\endinput}}
  \renewcommand{\childdocof}[1]{}
  \renewcommand{\childdocby}[2][]{}
  \renewcommand{\childdocforward}[2][]{}
  \renewcommand{\childdocdisable}{}
}
%    \end{macrocode}

% \macro{\childdocmain}
% The macro |\childdocmain| is to be called at the top of the main file
% with nothing or the main filename (without extension) as argument.
% First, it breaks loops.
% If the argument is not empty and does not match |\childdocname|
% (which is set by the first inclusion of |childdoc.def|),
% |\ifchilddoc| is set to true, |\includeonly| is applied to the child file
% and |\jobname| is set to the main file
% (for proper handling of |.aux| files):
%    \begin{macrocode}
\newcommand{\childdocmain}[1]
{
  \childdocdisable\childdocmain{}
  \if?#1?\else
    \begingroup
      \def\childdoctmp{#1}
      \ifx\childdoctmp\childdocname
        \def\childdoctmp{}
      \else
        \def\childdoctmp
        {
          \childdoctrue
          \includeonly{\childdocname}
          \def\childdocjob{#1}
          \def\jobname{#1}
        }
      \fi
      \expandafter
    \endgroup
    \childdoctmp
  \fi
}
%    \end{macrocode}

% \macro{\childdocof}
% The command |\childdocof| redirects
% compilation to the main file |#1|.
%    \begin{macrocode}
\newcommand{\childdocof}[1]
{
  \childdocdisable
  \childdoctrue
  \includeonly{\childdocname}
  \def\jobname{#1}
  \def\childdocjob{#1}
  \input{#1}
}
%    \end{macrocode}

% \macro{\childdocby}
% The command |\childdocby| ....
%    \begin{macrocode}
\newcommand{\childdocby}[2][]
{
  \childdocdisable
  \childdoctrue
  \childdocmanualtrue
  \if?#1?\else
    \def\jobname{#2}
  \fi
  \def\childdocjob{#2}
  \input{#2}
  \endinput
}
%    \end{macrocode}

% \macro{\childdocforward}
% The command |\childdocforward| redirects
% compilation to the main file or
% (if the optional argument is given) a child file.
% Parameters are set as if the main file
% or a child file starting with |\childdocof| was compiled.
% Then compilation is handed over to the main file:
%    \begin{macrocode}
\newcommand{\childdocforward}[2][]
{
  \begingroup
    \if?#1?
      \def\childdoctmp
      {
        \def\childdocname{#2}
        \def\childdocjob{#2}
        \def\jobname{#2}
        \input{#2}
        \endinput
      }
    \else
      \def\childdoctmp
      {
        \childdocdisable
        \def\childdocname{#2}
        \childdoctrue
        \includeonly{#2}
        \def\childdocjob{#1}
        \def\jobname{#1}
        \input{#1}
        \endinput
      }
    \fi
    \expandafter
  \endgroup
  \childdoctmp
}
%    \end{macrocode}

% \macro{\childdocforwardprefix}
% The command |\childdocforwardprefix| redirects
% compilation to the main or a child file by means of a pattern.
% The prefix |#1| in the current filename is replaced by |#2|
% and the suffix of the current filename is kept
% (it is assumed that the filename does not contain the substring `|~~~|'
% which is used as a delimiter).
% Compilation is handed over to the new file by |\childdocforward|:
%    \begin{macrocode}
\newcommand{\childdocforwardprefix}[3][]
{
  \begingroup
    \def\childdocextract #2##1~~~{\def\childdoctmp{\childdocforward[#1]{#3##1}}}
    \expandafter\childdocextract\childdocname~~~
    \expandafter
  \endgroup
  \childdoctmp
}
%    \end{macrocode}

% \macro{\childdoc}
% The deprecated macro |\childdoc| is a legacy version of |\childdocmain|:
%    \begin{macrocode}
\newcommand{\childdoc}{\childdocmain}
%    \end{macrocode}

% \macro{\childdocredirect}
% The deprecated macro |\childdocredirect| is a legacy version
% of |\childdocforward| and |\childdocforwardprefix|:
%    \begin{macrocode}
\newcommand{\childdocredirect}[2][]
{
  \begingroup
    \if?#1?
      \def\childdoctmp{\childdocforward{#2}}
    \else
      \def\childdoctmp{\childdocforwardprefix{#1}{#2}}
    \fi
    \expandafter
  \endgroup
  \childdoctmp
}
%    \end{macrocode}

%\iffalse
%</package>
%\fi
%
\endinput

\childdocforward{cdocsamp}
%    \end{macrocode}

%\iffalse
%</sampledraft>
%\fi
%
% %%%%%%%%%%%%%%%%%%%%%%%%%%%%%%%%%%%%%%
% \paragraph{Forwarding for Final Version of the Chapters.}
%
% The following forwarding files |cdocsfn1.tex| and |cdocsfn2.tex|
% (with identical content)
% compile the final versions of the child documents
% |cdocsch1.tex| and |cdocsch2.tex|, respectively:
%\iffalse
%<*samplefinal>
%\fi
%    \begin{macrocode}
\def\version{final}
% \iffalse
%
% childdoc.dtx Copyright (C) 2017-2018 Niklas Beisert
%
% This work may be distributed and/or modified under the
% conditions of the LaTeX Project Public License, either version 1.3
% of this license or (at your option) any later version.
% The latest version of this license is in
%   http://www.latex-project.org/lppl.txt
% and version 1.3 or later is part of all distributions of LaTeX
% version 2005/12/01 or later.
%
% This work has the LPPL maintenance status `maintained'.
%
% The Current Maintainer of this work is Niklas Beisert.
%
% This work consists of the files childdoc.dtx and childdoc.ins
% and the derived files childdoc.def and cdocsamp.tex with
% cdocsch1.tex, cdocsch2.tex, cdocsdrf.tex, cdocsfn1.tex, cdocsfn2.tex.
%
%<package>\ifdefined\childdocmain\endinput\fi
%<package>\ProvidesFile{childdoc.def}[2018/12/30 v2.0 child document driver]
%<samplemain>\ProvidesFile{cdocsamp.tex}[2018/12/30 v2.0 sample for childdoc]
%<*driver>
%\ProvidesFile{childdoc.drv}[2018/12/30 v2.0 childdoc reference manual file]
\PassOptionsToClass{10pt,a4paper}{article}
\documentclass{ltxdoc}

\usepackage[margin=35mm]{geometry}
\usepackage{hyperref}
\usepackage{hyperxmp}
\usepackage[usenames]{color}

\hypersetup{colorlinks=true}
\hypersetup{pdfstartview=FitH}
\hypersetup{pdfpagemode=UseNone}
\hypersetup{pdfsource={}}
\hypersetup{pdflang={en-UK}}
\hypersetup{pdfcopyright={Copyright 2017-2018 Niklas Beisert.
  This work may be distributed and/or modified under the
  conditions of the LaTeX Project Public License, either version 1.3
  of this license or (at your option) any later version.}}
\hypersetup{pdflicenseurl={http://www.latex-project.org/lppl.txt}}
\hypersetup{pdfcontactaddress={ETH Zurich, ITP, HIT K,
  Wolfgang-Pauli-Strasse 27}}
\hypersetup{pdfcontactpostcode={8093}}
\hypersetup{pdfcontactcity={Zurich}}
\hypersetup{pdfcontactcountry={Switzerland}}
\hypersetup{pdfcontactemail={nbeisert@itp.phys.ethz.ch}}
\hypersetup{pdfcontacturl={http://people.phys.ethz.ch/\xmptilde nbeisert/}}

\newcommand{\secref}[1]{\hyperref[#1]{section \ref*{#1}}}

\parskip1ex
\parindent0pt
\let\olditemize\itemize
\def\itemize{\olditemize\parskip0pt}

\begin{document}

\title{The \textsf{childdoc} Package}
\hypersetup{pdftitle={The childdoc Package}}
\author{Niklas Beisert\\[2ex]
  Institut f\"ur Theoretische Physik\\
  Eidgen\"ossische Technische Hochschule Z\"urich\\
  Wolfgang-Pauli-Strasse 27, 8093 Z\"urich, Switzerland\\[1ex]
  \href{mailto:nbeisert@itp.phys.ethz.ch}
  {\texttt{nbeisert@itp.phys.ethz.ch}}}
\hypersetup{pdfauthor={Niklas Beisert}}
\hypersetup{pdfsubject={Manual for the LaTeX2e Package childdoc}}
\date{30 December 2018, \textsf{v2.0}}
\maketitle

\begin{abstract}\noindent
\textsf{childdoc} is a \LaTeXe{} package
that enables the direct compilation
of document sections included by |\include|
to individual files.
\end{abstract}

\begingroup
\parskip0ex
\tableofcontents
\endgroup

%%%%%%%%%%%%%%%%%%%%%%%%%%%%%%%%%%%%%%%%%%%%%%%%%%%%%%%%%%%%%%%%%%%%%%%%%%%%%%%%
%%%%%%%%%%%%%%%%%%%%%%%%%%%%%%%%%%%%%%%%%%%%%%%%%%%%%%%%%%%%%%%%%%%%%%%%%%%%%%%%
\section{Introduction}

\LaTeX{} provides a mechanism to structure a large document (such as a book)
into a main file and several child files (containing the chapters)
using the |\include| command.
This mechanism is beneficial for documents
which span hundreds of pages in order to
make the source file(s) more manageable.
Moreover, compilation can be restricted to
selected child files by means of the |\includeonly| command.
The latter feature can be used to reduce the compilation time while editing
(this was significantly more useful in the earlier days of \LaTeX{})
or to generate a smaller document which is easier to navigate.
Another application of |\includeonly| is to generate
documents consisting of selected parts of the complete document.

However, there are a few drawbacks of the plain |\include| mechanism:
\begin{itemize}
\item
The child files cannot be compiled on their own,
they can only be compiled via the main file.
A naive editing environment
(such as a text editor with an option
to have the current file processed by \LaTeX)
may require one to switch to the main file before compiling;
attempting to compile the child file produces errors.
\item
The main file must be modified (each time)
to adjust the |\includeonly| command
to the present needs. This easily leaves the main file in a messy state.
\item
The generated document will always carry the filename
of the main document. This is inconvenient if
several child files are to be compiled and
to be kept for distribution.
\end{itemize}

The present package provides a simple interface
to make child files individually compilable by \LaTeX{}.
Compiling a child file then has the same effect as compiling
the main file with an |\includeonly| command
to select the appropriate child.
Moreover the generated document will carry the name of the child
rather than the main file.
This resolves all three above issues.

This feature is meant to make the editing of books,
thesis documents and lecture notes somewhat more convenient.
However, the package can also be used efficiently for
composing a series of documents (such as exercise sheets)
which are typically distributed individually.
It then assists the author in generating the individual documents
(potentially in different versions)
as well as a document containing the collected series.
Another application is in developing style files
or other kinds of included material
where compilation of the style file could redirect
to a sample or test file.

%%%%%%%%%%%%%%%%%%%%%%%%%%%%%%%%%%%%%%%%%%%%%%%%%%%%%%%%%%%%%%%%%%%%%%%%%%%%%%%%
%%%%%%%%%%%%%%%%%%%%%%%%%%%%%%%%%%%%%%%%%%%%%%%%%%%%%%%%%%%%%%%%%%%%%%%%%%%%%%%%
\section{Usage}

First of all, the package \textsf{childdoc} is \emph{not} a standard
\LaTeXe{} |.sty| style file! Therefore it needs to be invoked in
a non-standard way.

%%%%%%%%%%%%%%%%%%%%%%%%%%%%%%%%%%%%%%%%%%%%%%%%%%%%%%%%%%%%%%%%%%%%%%%%%%%%%%%%
\subsection{Included Files}
\label{sec:include}

%%%%%%%%%%%%%%%%%%%%%%%%%%%%%%%%%%%%%%%%
\DescribeMacro{\childdocmain}
To use the package, add the commands
\begin{center}
\begin{tabular}{l}
|\input{childdoc.def}|\\
|\childdocmain{}|\\
\end{tabular}
\end{center}
at the very top of the main \LaTeX{} file,
in particular \emph{before} the |\documentclass| statement!
The argument of |\childdocmain| should be left empty
(but it must be present).

%%%%%%%%%%%%%%%%%%%%%%%%%%%%%%%%%%%%%%%%
\DescribeMacro{\childdocof}
Furthermore, add the commands
\begin{center}
\begin{tabular}{l}
|\input{childdoc.def}|\\
|\childdocof{|\textit{main}|}|\\
\end{tabular}
\end{center}
at the top of every child file \textit{child}
which is included by |\include{|\textit{child}|}|
from within the main file
(or at least for those files to be compiled individually).
The argument \textit{main} must be the filename of the main file.

There are a couple of
considerations in setting up the main and child documents:

%%%%%%%%%%%%%%%%%%%%%%%%%%%%%%%%%%%%%%%%
\paragraph{Restrictions.}

Please note the following restrictions:
\begin{itemize}
\item
|\childdocmain| must be called with one argument \textit{main}
to ensure compatibility with earlier version of the package.
It must either be empty (|\childdocmain{}|)
or precisely match the filename of the main file in which it is specified.
See \secref{sec:detection} for further information.
\item
The filename \textit{main} must be specified without the |.tex| extension.
\item
The filename \textit{main} is case sensitive
(even in case-insensitive file systems)
due to internal string comparison.
\item
The argument \textit{main} should be fully expanded, it cannot be a macro.
\item
Subdirectories and special characters should be avoided in filenames.
\item
The command |\childdocmain{|\textit{main}|}| must be followed by a whitespace.
It should not be followed immediately by another command
or by a comment mark `|%|'.
This is because the \TeX{} parser reads the token immediately following
the argument of |\childdocmain| and puts it
at the beginning of every child section;
however, a white\-space is ignored.
\end{itemize}

%%%%%%%%%%%%%%%%%%%%%%%%%%%%%%%%%%%%%%%%
\paragraph{Content of Main File.}

It is advisable to place all content in the child files included by |\include|.
Any output contained in the main file will appear in all child documents
unless suppressed manually;
it cannot be suppressed automatically by the |\includeonly| directive
and thus should normally be avoided.
A method to include some content in the main file
by means of conditional processing is described in \secref{sec:conditional}.

%%%%%%%%%%%%%%%%%%%%%%%%%%%%%%%%%%%%%%%%
\paragraph{Page Numbering.}

When only a part of the document is compiled,
the appropriate numbering of pages
(as well as other status parameters)
is determined from the |.aux| files.
The latter contain information from previous passes.
However this information needs to propagate through
all intermediate child documents.
Therefore the page numbering in child documents may well
be inconsistent until the complete document is compiled at least once.

A useful (if unconventional) way to always ensure a consistent
page numbering is to restart the numbering in each child document
and denote the pages by `\textit{child}|.|\textit{page}'
where \textit{child} represents the chapter/section number of the child file.
This can be achieved by the command
|\numberwithin{page}{|\textit{child}|}|
of the \textsf{amsmath} package
where \textit{child} can be |chapter| or |section|
depending on the chosen structuring.
Alternatively, one can modify the macro |\thepage| appropriately
and reset the counter |page| at the start of each child file.

%%%%%%%%%%%%%%%%%%%%%%%%%%%%%%%%%%%%%%%%%%%%%%%%%%%%%%%%%%%%%%%%%%%%%%%%%%%%%%%%
\subsection{Conditional Processing}
\label{sec:conditional}

The package provides a mechanism to compile different versions
of a document. To customise the versions further some conditional processing
can come in handy to distinguish which version is being compiled.
The package provides two macros to describe the compilation context:

%%%%%%%%%%%%%%%%%%%%%%%%%%%%%%%%%%%%%%%%
\DescribeMacro{\ifchilddoc}
The conditional |\ifchilddoc| distinguishes between the compilation of
child documents and the main document:
%
\begin{center}
|\ifchilddoc |\textit{child-code}| |[|\||else |\textit{main-code}]| \||fi|
\end{center}

%%%%%%%%%%%%%%%%%%%%%%%%%%%%%%%%%%%%%%%%
\DescribeMacro{\childdocname}
\DescribeMacro{\childdocjob}
The macro |\childdocname| contains the filename (without extension)
of the main or child file being processed.
Note that |\childdocjob| will always contain the name of the main file.

%%%%%%%%%%%%%%%%%%%%%%%%%%%%%%%%%%%%%%%%
\paragraph{Title Page.}

Conditional processing can be used to include a title or banner page
in the main document when proper precautions are taken.
Importantly, the code in the main file should ensure that the page counter
(as well as other status parameters which are stored in the |.aux| files)
takes the same value after the conditional processing.
Otherwise the page numbers may take divergent values
depending on which part is compiled.

For example, a title page could be declared by:
%
\begin{center}
\begin{tabular}{l}
|\ifchilddoc\||else|\\
|\addtocounter{page}{-1}|\\
\textit{code for title page}\\
|\newpage|\\
|\||fi|
\end{tabular}
\end{center}
%
A banner page for the child documents can be generated by:
%
\begin{center}
\begin{tabular}{l}
|\ifchilddoc|\\
|\addtocounter{page}{-1}|\\
\textit{code for banner page}\\
|\newpage|\\
|\||fi|
\end{tabular}
\end{center}
%
Here one could write a message such as:
\begin{center}
|This is the part \childdocname{} of \childdocjob{}.|
\end{center}

%%%%%%%%%%%%%%%%%%%%%%%%%%%%%%%%%%%%%%%%%%%%%%%%%%%%%%%%%%%%%%%%%%%%%%%%%%%%%%%%
\subsection{Flags}
\label{sec:flags}

The package makes it easy to generate different versions
of the main or child documents.
To this end compilation flags can be defined
and assigned different default values.
They will be particularly useful in conjunction
with the forwarding mechanism described in \secref{sec:forward}.

For example, it may be useful to have a flag |\version|
which can be set to |draft| or |final|.
The document source will contain some conditional code
depending on the value of |\version|.
Suppose further, the flag should default to |final| for the main file
and to |draft| for child files
which is a natural assignment for editing the document.
This is achieved by placing the following code
in the preamble of the main document
(below the |\childdocmain| directive):
%
\begin{center}
\begin{tabular}{l}
|\ifchilddoc|\\
|\providecommand{\version}{draft}|\\
|\||else|\\
|\providecommand{\version}{final}|\\
|\||fi|
\end{tabular}
\end{center}
%
The definition by |\providecommand| makes sure
that previous definitions are not overwritten.
Further statements |\providecommand{\version}{...}|
can thus be added before the above code to override it.

For the main file, one might add a line
(between |\childdocmain| and the above block)
%
\begin{center}
|%\ifchilddoc\||else\providecommand{\version}{draft}\||fi|
\end{center}
%
which can be uncommented to produce a draft version.
Likewise one can add a line to the very top of a child file
(above the |\childdocof{|\textit{main}|}| directive)
%
\begin{center}
|%\providecommand{\version}{final}|
\end{center}
%
which can be uncommented to produce the final version of this child document.

%%%%%%%%%%%%%%%%%%%%%%%%%%%%%%%%%%%%%%%%%%%%%%%%%%%%%%%%%%%%%%%%%%%%%%%%%%%%%%%%
\subsection{Forwarding}
\label{sec:forward}

Different versions of the main or child documents
using compilation flags as described in \secref{sec:flags}
can be (permanently) stored in different files
for convenient compilation, viewing and distribution.
To this end, the package defines a command
to pass on compilation to a different file:

%%%%%%%%%%%%%%%%%%%%%%%%%%%%%%%%%%%%%%%%
\DescribeMacro{\childdocforward}
The command |\childdocforward| redirects processing to
another source file:
%
\begin{center}
\begin{tabular}{l}
|\input{childdoc.def}|\\
|\childdocforward[|\textit{main}|]{|\textit{dest}|}|\\
\end{tabular}
\end{center}
%
The argument \textit{dest} is the destination file
(without extension).
It should be the main file or one of the child files.
Note that further \textsf{childdoc} directives
such as |\childdocof| and |\childdocforward|
in the indicated file will be processed in this form.
The optional argument \textit{main}
passes on directly to the main file \textit{main}
while pretending to compile the child \textit{dest}.
This form behaves as if \textit{dest}
issues |\childdocof{|\textit{main}|}| right away,
and no further \textsf{childdoc} directives will be processed.

%%%%%%%%%%%%%%%%%%%%%%%%%%%%%%%%%%%%%%%%
\DescribeMacro{\...prefix}
In the alternative form |\childdocforwardprefix|,
%
\begin{center}
\begin{tabular}{l}
|\input{childdoc.def}|\\
|\childdocforwardprefix[|\textit{main}|]{|\textit{prefix}|}{|\textit{dest}|}|
\end{tabular}
\end{center}
%
the destination file is determined by a pattern
depending on the current file:
To make this work, the current file must be called
`{\textit{prefix}\hspace{0.2em}\textit{suffix}}'
with \textit{prefix} matching precisely the argument.
Processing is then passed on to the file
`{\textit{dest}\hspace{0.2em}\textit{suffix}}'.
Surely, the same effect is achieved by
directly specifying the
argument `{\textit{dest}\hspace{0.2em}\textit{suffix}}'
in the first form.
However, that requires to set up a different file
for each child. With the alternative form of the command
all these files can have exactly the same content
which simplifies setting them up and maintaining them.

For example, the following file |draft.tex|
with a compilation flag |\version| as described in \secref{sec:flags}
compiles the main document as a draft:
%
\begin{center}
\begin{tabular}{l}
|\def\version{draft}|\\
|\input{childdoc.def}|\\
|\childdocforward{|\textit{main}|}|
\end{tabular}
\end{center}
%
Likewise, the following files |final|\textit{nn}|.tex|
compile the final version of the child document
|child|\textit{nn}|.tex|:
%
\begin{center}
\begin{tabular}{l}
|\def\version{final}|\\
|\input{childdoc.def}|\\
|\childdocforwardprefix{final}{child}|
\end{tabular}
\end{center}
%

Note that when several versions of a main file and/or of each child file
are to be generated, it may be convenient to set up a |Makefile| or
shell script to automatise the process.

%%%%%%%%%%%%%%%%%%%%%%%%%%%%%%%%%%%%%%%%%%%%%%%%%%%%%%%%%%%%%%%%%%%%%%%%%%%%%%%%
\subsection{Command Line Processing}
\label{sec:commandline}

The effect of redirection files can also be achieved by invoking
the \LaTeX{} compiler with a more elaborate command line.
Most conveniently this should be done as part
of a shell script or a |Makefile|.

When using \textsf{childdoc} in the main file, the following
command lines effectively perform a redirection
(note that depending on the shell being used,
backslashes may have to be doubled: `|\|' $\to$ `|\\|'):
%
\begin{center}
|... -jobname "|\textit{target}|" |\\|"|[\textit{flags}]%
|\input{childdoc.def}\childdocforward[|\textit{main}|]{|\textit{dest}|}"|
\end{center}
%
Here \textit{target} is the name of the output file,
\textit{main} is the name of the main file
and \textit{dest} is the name of the main or child file to be processed
(all filenames without extensions).
The optional argument \textit{main} can be omitted
if \textit{main} matches \textit{dest}.
Optionally, compilation \textit{flags} can be defined via |\def| commands.
This command line makes the \TeX{} engine believe
it is compiling the file \textit{target}
whose content is specified as the latter parameter.
The provided code then forwards the processing to
\textit{main} or \textit{dest} as described in \secref{sec:forward}.

%%%%%%%%%%%%%%%%%%%%%%%%%%%%%%%%%%%%%%%%%%%%%%%%%%%%%%%%%%%%%%%%%%%%%%%%%%%%%%%%
\subsection{Include by Input}
\label{sec:input}

Including child documents by |\include| has some restrictions by design.
Most notably, the content of a child document always occupies
its own set of pages; pages cannot be shared between child documents.
Usually, this behaviour makes perfect sense
because each child document contain an essential part of the document.
However, in some situations it may be desirable to compose
a document from a collection of parts
without having mandatory page breaks between then.
For this case, the package
provides a mechanism to include parts
by |\input| which can also be processed individually.
However, by construction this mechanism
requires manual handling of the content to be output.

%%%%%%%%%%%%%%%%%%%%%%%%%%%%%%%%%%%%%%%%
\DescribeMacro{\ifchilddocmanual}
The main file should be prepared as usual, see \secref{sec:include}.
However, the document body must make a distinction
between processing of an individual part and of the main document, e.g.:
%
\begin{center}
\begin{tabular}{l}
|\ifchilddocmanual|\\
|\input{\childdocname}|\\
|\||else|\\
\textit{document body with }|\input{|\textit{part}|}|\\
|\||fi|
\end{tabular}
\end{center}
%
The conditional |\ifchilddocmanual| is true whenever
a part to be included by |\input| is being compiled,
and the name of the part is stored in |\childdocname|.

%%%%%%%%%%%%%%%%%%%%%%%%%%%%%%%%%%%%%%%%
\DescribeMacro{\childdocby}
Each part to be included by |\input| should start with:
%
\begin{center}
\begin{tabular}{l}
|\input{childdoc.def}|\\
|\childdocby{|\textit{main}|}|\\
\end{tabular}
\end{center}
%
The directive |\childdocby| is similar to |\childdocof|
described in \secref{sec:include},
but the subsequent selection of content must be done manually.
To that end, both |\ifchilddoc| and |\ifchilddocmanual|
will be true upon processing of a part,
and the name of the part is stored in |\childdocname|.
Note that |\jobname| will be set to the filename of the current part
so that each part receives an individual |.aux| file
that does not interfere with the |.aux| file(s) of the main document.
This behaviour can be altered by the alternative form
|\childdocby[*]{|\textit{main}|}| (with a non-empty optional argument)
which uses the |.aux| file of the main document
by setting |\jobname| to \textit{main}.

%%%%%%%%%%%%%%%%%%%%%%%%%%%%%%%%%%%%%%%%%%%%%%%%%%%%%%%%%%%%%%%%%%%%%%%%%%%%%%%%
\subsection{Driver Development}
\label{sec:driver}

The \textsf{childdoc} mechanism can also be use for the development
of definition files such as \LaTeX{} styles or classes.
This case differs from the above setup with multiple parts
included by |\include| in that no |\includeonly| should be invoked.
This can be achieved by starting the include file
(before |\ProvidesPackage|) with:
%
\begin{center}
\begin{tabular}{l}
|\input{childdoc.def}|\\
|\childdocforward{|\textit{main}|}|\\
\end{tabular}
\end{center}
%
or alternatively with:
%
\begin{center}
\begin{tabular}{l}
|\input{childdoc.def}|\\
|\childdocby{|\textit{main}|}|\\
\end{tabular}
\end{center}
%
Both forms have slightly different effects as described above.
The main file is prepared as usual, see \secref{sec:include}.

%%%%%%%%%%%%%%%%%%%%%%%%%%%%%%%%%%%%%%%%%%%%%%%%%%%%%%%%%%%%%%%%%%%%%%%%%%%%%%%%
\subsection{Legacy Detection}
\label{sec:detection}

The directive |\childdocmain| in the main file can detect
whether the complete document or merely a child is to be compiled
even without using the directive |\childdocof|.
This method is deprecated because it is less robust
and there is no compelling reason to use it;
it is merely provided for backward compatibility
and it may be removed in future versions.

If the detection mechanism is to be used,
it is mandatory to correctly specify
the filename of the main file as the argument of |\childdocmain|:
%
\begin{center}
\begin{tabular}{l}
|\input{childdoc.def}|\\
|\childdocmain{|\textit{main}|}|\\
\end{tabular}
\end{center}
%
If |\jobname| does not match the argument \textit{main} of |\childdocmain|,
it is assumed that |\jobname| points to the child file to be compiled.
When using |\childdocmain| with the main file specified as argument,
it suffices to start a child file
with just |\input{|\textit{main}|}|
without loading of the package and using |\childdocof|.
If instead all processing is done
with the appropriate \textsf{childdoc} directives,
the argument of \textit{main} of |\childdocmain| can be empty.

An alternative version of the command line processing described
in \secref{sec:commandline} using the detection mechanism reads:
%
\begin{center}
|... -jobname "|\textit{target}|" "|[\textit{flags}]%
[|\def\jobname{|\textit{dest}|}|]|\input{|\textit{main}|}"|
\end{center}

%%%%%%%%%%%%%%%%%%%%%%%%%%%%%%%%%%%%%%%%%%%%%%%%%%%%%%%%%%%%%%%%%%%%%%%%%%%%%%%%
\subsection{Manual Code}
\label{sec:manual}

In case one cannot be certain whether the definitions file |childdoc.def|
is installed on the target \TeX{} distribution
and one prefers not to ship it,
it is conceivable to paste a few relevant commands into the sources.

To that end, drop all statements |\input{childdoc.def}|
and perform the replacements as outlined below.
Instead of |\childdocmain{|\textit{main}|}| add the following code
to the top of the main file:
%
\begin{center}
\begin{tabular}{l}
|\||ifdefined\childdocname\endinput\||fi\newif\ifchilddoc|\\
|\edef\childdocname{\scantokens\expandafter{\jobname\noexpand}}|\\
|\def\childdocmain{|\textit{main}|}\||ifx\childdocmain\childdocname\||else|\\
|\childdoctrue\includeonly{\childdocname}\let\jobname\childdocmain\||fi|\\
\end{tabular}
\end{center}
%
Instead of |\childdocof{|\textit{main}|}| just include the main file
at the top of each child file:
%
\begin{center}
|\input{|\textit{main}|}|
\end{center}
%
A simple redirection |\childdocforward{|\textit{dest}|}| is achieved by:
%
\begin{center}
|\def\jobname{|\textit{dest}|}\input{\jobname}|
\end{center}
%
The redirection with prefix
|\childdocforwardprefix[|\textit{prefix}|]{|\textit{dest}|}|
is accomplished by:
%
\begin{center}
\begin{tabular}{l}
|{\edef\jobname{\scantokens\expandafter{\jobname\noexpand}}|\\
|\def\redirectjob |\textit{prefix}|#1~~~{\gdef\jobname{|\textit{dest}|#1}}|\\
|\expandafter\redirectjob\jobname~~~}\input{\jobname}|
\end{tabular}
\end{center}

In an alternative approach,
child documents can be compiled by a specific command line
without additional code or specific definitions:
%
\begin{center}
|... -jobname "|\textit{target}|" "|[\textit{flags}]%
|\includeonly{|\textit{dest}|}\input{|\textit{main}|}"|
\end{center}
%

%%%%%%%%%%%%%%%%%%%%%%%%%%%%%%%%%%%%%%%%%%%%%%%%%%%%%%%%%%%%%%%%%%%%%%%%%%%%%%%%
%%%%%%%%%%%%%%%%%%%%%%%%%%%%%%%%%%%%%%%%%%%%%%%%%%%%%%%%%%%%%%%%%%%%%%%%%%%%%%%%
\section{Information}

%%%%%%%%%%%%%%%%%%%%%%%%%%%%%%%%%%%%%%%%%%%%%%%%%%%%%%%%%%%%%%%%%%%%%%%%%%%%%%%%
\subsection{Copyright}

Copyright \copyright{} 2017--2018 Niklas Beisert

This work may be distributed and/or modified under the
conditions of the \LaTeX{} Project Public License, either version 1.3
of this license or (at your option) any later version.
The latest version of this license is in
  \url{http://www.latex-project.org/lppl.txt}
and version 1.3 or later is part of all distributions of \LaTeX{}
version 2005/12/01 or later.

This work has the LPPL maintenance status `maintained'.

The Current Maintainer of this work is Niklas Beisert.

This work consists of the files |README.txt|, |childdoc.ins| and |childdoc.dtx|
as well as the derived files |childdoc.def|, |cdocsamp.tex|
with |cdocsch1.tex|, |cdocsch2.tex|, |cdocspt3.tex|, |cdocspt4.tex|,
|cdocsdrf.tex|, |cdocsfn1.tex|, |cdocsfn2.tex|
as well as |childdoc.pdf|.

%%%%%%%%%%%%%%%%%%%%%%%%%%%%%%%%%%%%%%%%%%%%%%%%%%%%%%%%%%%%%%%%%%%%%%%%%%%%%%%%
\subsection{Files and Installation}

The package consists of the files:
%
\begin{center}
\begin{tabular}{ll}
    |README.txt|   & readme file \\
    |childdoc.ins| & installation file \\
    |childdoc.dtx| & source file \\
    |childdoc.def| & definition file \\
    |cdocsamp.tex| & sample main file \\
    |cdocsch1.tex| & sample include file \\
    |cdocsch2.tex| & sample include file \\
    |cdocspt3.tex| & sample part file \\
    |cdocspt4.tex| & sample part file \\
    |cdocsdrf.tex| & sample redirection file \\
    |cdocsfn1.tex| & sample redirection file \\
    |cdocsfn2.tex| & sample redirection file \\
    |childdoc.pdf| & manual
\end{tabular}
\end{center}
%
The distribution consists of the files
|README.txt|, |childdoc.ins| and |childdoc.dtx|.
%
\begin{itemize}
\item
Run (pdf)\LaTeX{} on |childdoc.dtx|
to compile the manual |childdoc.pdf| (this file).
\item
Run \LaTeX{} on |childdoc.ins| to create the definitions file |childdoc.def|
and the sample |cdocsamp.tex| with include files
|cdocsch1.tex|, |cdocsch2.tex|, |cdocspt3.tex|, |cdocspt4.tex|,
|cdocsdrf.tex|, |cdocsfn1.tex|, |cdocsfn2.tex|.
Then copy the file |childdoc.def| to an appropriate directory of your \LaTeX{}
distribution, e.g.\ \textit{texmf-root}|/tex/latex/childdoc|.
\end{itemize}

%%%%%%%%%%%%%%%%%%%%%%%%%%%%%%%%%%%%%%%%%%%%%%%%%%%%%%%%%%%%%%%%%%%%%%%%%%%%%%%%
\subsection{Related CTAN Packages}

There are several other packages which offer a similar functionality:
%
\begin{itemize}
\item
The packages
\href{http://ctan.org/pkg/docmute}{\textsf{docmute}},
\href{http://ctan.org/pkg/includex}{\textsf{includex}} and
\href{http://ctan.org/pkg/standalone}{\textsf{standalone}}
provide commands to include only the document body of
a child file thus allowing both files to be compiled individually.
\item
The packages \href{http://ctan.org/pkg/subdocs}{\textsf{subdocs}}
and \href{http://ctan.org/pkg/subfiles}{\textsf{subfiles}}
provide structures in which the main and child documents can be
encapsulated and allowing them to be compiled individually.
The inclusion mechanism is different from the conventional |\include|.
\item
The package \href{http://ctan.org/pkg/combine}{\textsf{combine}}
is an elaborate solution to combine several documents into one.
\end{itemize}
%
See also the CTAN topic \href{http://ctan.org/topic/subdocs}{\textsf{subdocs}}
for further related packages.
The present package differs from the above solutions in that
a document structure constructed with the conventional |\include| mechanism
just needs two extra commands at the top of every file
such that all constituent files can be compiled individually.

%%%%%%%%%%%%%%%%%%%%%%%%%%%%%%%%%%%%%%%%%%%%%%%%%%%%%%%%%%%%%%%%%%%%%%%%%%%%%%%%
%\subsection{Feature Suggestions}
%
%The following is a list of features which may be useful for future
%versions of this package:
%%
%\begin{itemize}
%\item
%\ldots
%\end{itemize}

%%%%%%%%%%%%%%%%%%%%%%%%%%%%%%%%%%%%%%%%%%%%%%%%%%%%%%%%%%%%%%%%%%%%%%%%%%%%%%%%
\subsection{Revision History}

%%%%%%%%%%%%%%%%%%%%%%%%%%%%%%%%%%%%%%%%
\paragraph{v2.0:} 2018/12/30

\begin{itemize}
\item
immediate forward processing
\item
added |\childdocby| mechanism
\item
manual restructured
\end{itemize}

%%%%%%%%%%%%%%%%%%%%%%%%%%%%%%%%%%%%%%%%
\paragraph{v1.6:} 2018/01/17

\begin{itemize}
\item
application for development of include files
\item
corrections to manual
\end{itemize}

%%%%%%%%%%%%%%%%%%%%%%%%%%%%%%%%%%%%%%%%
\paragraph{v1.5:} 2017/05/21

\begin{itemize}
\item
more complete structuring introduced
\item
|\childdocof| introduced
\item
|\childdoc| renamed to |\childdocmain|
\item
|\childredirect| renamed to |\childdocforward| and |\childdocforwardprefix|
and functionality expanded
\end{itemize}

%%%%%%%%%%%%%%%%%%%%%%%%%%%%%%%%%%%%%%%%
\paragraph{v1.0:} 2017/04/27

\begin{itemize}
\item
manual and install package
\item
first version published on CTAN
\end{itemize}

%%%%%%%%%%%%%%%%%%%%%%%%%%%%%%%%%%%%%%%%
\paragraph{v0.6:} 2017/04/26

\begin{itemize}
\item
redirection mechanism added
\end{itemize}

%%%%%%%%%%%%%%%%%%%%%%%%%%%%%%%%%%%%%%%%
\paragraph{v0.5:} 2017/04/26

\begin{itemize}
\item
functionality in definition file
\end{itemize}


%%%%%%%%%%%%%%%%%%%%%%%%%%%%%%%%%%%%%%%%%%%%%%%%%%%%%%%%%%%%%%%%%%%%%%%%%%%%%%%%
%%%%%%%%%%%%%%%%%%%%%%%%%%%%%%%%%%%%%%%%%%%%%%%%%%%%%%%%%%%%%%%%%%%%%%%%%%%%%%%%
%%%%%%%%%%%%%%%%%%%%%%%%%%%%%%%%%%%%%%%%%%%%%%%%%%%%%%%%%%%%%%%%%%%%%%%%%%%%%%%%
\appendix

\settowidth\MacroIndent{\rmfamily\scriptsize 000\ }

 \DocInput{childdoc.dtx}

\end{document}
%</driver>
% \fi
%
% %%%%%%%%%%%%%%%%%%%%%%%%%%%%%%%%%%%%%%%%%%%%%%%%%%%%%%%%%%%%%%%%%%%%%%%%%%%%%%
% %%%%%%%%%%%%%%%%%%%%%%%%%%%%%%%%%%%%%%%%%%%%%%%%%%%%%%%%%%%%%%%%%%%%%%%%%%%%%%
% \section{Sample}
%\iffalse
%<*samplemain>
%\fi
%
% The following presents a sample document
% with two chapters, two parts, a title page,
% a compile flag as well as three forwarding files to set the flag.
% It consists of eight |.tex| files:
% \begin{center}
% \begin{tabular}{ll}
% |cdocsamp.tex|&main file\\
% |cdocsch1.tex|&include file for chapter 1\\
% |cdocsch2.tex|&include file for chapter 2\\
% |cdocspt3.tex|&include file for part 3\\
% |cdocspt4.tex|&include file for part 4\\
% |cdocsdrf.tex|&forwarding file for main file in draft mode\\
% |cdocsfi1.tex|&forwarding file for final version of chapter 1\\
% |cdocsfi2.tex|&forwarding file for final version of chapter 2\\
% \end{tabular}
% \end{center}
% Each of the eight files can be compiled directly by the \LaTeX{} compiler.
%
% %%%%%%%%%%%%%%%%%%%%%%%%%%%%%%%%%%%%%%
% \paragraph{Main File.}
%
% The main file is called |cdocsamp.tex|.
%
% Load the \textsf{childdoc} definitions and
% declare the filename for the main document:
%    \begin{macrocode}
\input{childdoc.def}
\childdocmain{}
%    \end{macrocode}

% Optional override for |\version| flag:
%    \begin{macrocode}
%%\ifchilddoc\else\providecommand{\version}{draft}\fi
%    \end{macrocode}

% Define the default values for the |\version| flag
% (|final| for the main file and |draft| for childs):
%    \begin{macrocode}
\ifchilddoc
\providecommand{\version}{draft}
\else
\providecommand{\version}{final}
\fi
%    \end{macrocode}

% Load the standard document class:
%    \begin{macrocode}
\documentclass[12pt]{article}
%    \end{macrocode}

% Start the document body:
%    \begin{macrocode}
\begin{document}
%    \end{macrocode}

% Declare a title page.
% Print title, part of document being processed and version flag:
%    \begin{macrocode}
\addtocounter{page}{-1}
\begin{center}
{\LARGE\bfseries{}childdoc example\par}
\vspace{1cm}
\ifchilddoc
\ifchilddocmanual part\else chapter\fi:
`\childdocname' of `\childdocjob'\par
\else
main document: `\childdocjob'\par
\fi
version: \version\par
\end{center}
\newpage
%    \end{macrocode}

% Manually include selected file,
% otherwise process as usual:
%    \begin{macrocode}
\ifchilddocmanual
\section*{part `\childdocname'}
\input{\childdocname}
\else
%    \end{macrocode}

% Include the two chapters:
%    \begin{macrocode}
\include{cdocsch1}
\include{cdocsch2}
%    \end{macrocode}

% Include the two parts unless only chapters should be displayed:
%    \begin{macrocode}
\ifchilddoc\else
\section{part three}
\input{cdocspt3}
\section{part four}
\input{cdocspt4}
\fi
%    \end{macrocode}

% Process as usual until here:
%    \begin{macrocode}
\fi
%    \end{macrocode}

% End of document body:
%    \begin{macrocode}
\end{document}
%    \end{macrocode}
%\iffalse
%</samplemain>
%\fi
%
% %%%%%%%%%%%%%%%%%%%%%%%%%%%%%%%%%%%%%%
% \paragraph{Chapter Include Files.}
%
% The include files are called |cdocsch1.tex| and |cdocsch2.tex|.
%
%\iffalse
%<*samplechap1|samplechap2>
%\fi

% Optional override for |\version| flag:
%    \begin{macrocode}
%%\providecommand{\version}{final}
%    \end{macrocode}

% Include the main document:
%    \begin{macrocode}
\input{childdoc.def}
\childdocof{cdocsamp}
%    \end{macrocode}

%\iffalse
%</samplechap1|samplechap2>
%\fi
%
%\iffalse
%<*samplechap1>
%\fi
% Some text for chapter 1:
%    \begin{macrocode}
\section{one}
some text in chapter one
%    \end{macrocode}

%\iffalse
%</samplechap1>
%\fi
% Some text for chapter 2:
%\iffalse
%<*samplechap2>
%\fi
%    \begin{macrocode}
\section{two}
more text in chapter two
%    \end{macrocode}

%\iffalse
%</samplechap2>
%\fi
%
% %%%%%%%%%%%%%%%%%%%%%%%%%%%%%%%%%%%%%%
% \paragraph{Part Include Files.}
%
% The include files are called |cdocspt3.tex| and |cdocspt4.tex|.
%
%\iffalse
%<*samplepart3|samplepart4>
%\fi

% Optional override for |\version| flag:
%    \begin{macrocode}
%%\providecommand{\version}{final}
%    \end{macrocode}

% Include the main document:
%    \begin{macrocode}
\input{childdoc.def}
\childdocby{cdocsamp}
%    \end{macrocode}

%\iffalse
%</samplepart3|samplepart4>
%\fi
%
%\iffalse
%<*samplepart3>
%\fi
% Some text for part 3:
%    \begin{macrocode}
some text in part three
%    \end{macrocode}

%\iffalse
%</samplepart3>
%\fi
% Some text for part 4:
%\iffalse
%<*samplepart4>
%\fi
%    \begin{macrocode}
more text in part four
%    \end{macrocode}

%\iffalse
%</samplepart4>
%\fi
%
% %%%%%%%%%%%%%%%%%%%%%%%%%%%%%%%%%%%%%%
% \paragraph{Forwarding for a Complete Draft.}
%
% The following forwarding file |cdocsdrf.tex|
% compiles the main document in draft mode:
%\iffalse
%<*sampledraft>
%\fi
%    \begin{macrocode}
\def\version{draft}
\input{childdoc.def}
\childdocforward{cdocsamp}
%    \end{macrocode}

%\iffalse
%</sampledraft>
%\fi
%
% %%%%%%%%%%%%%%%%%%%%%%%%%%%%%%%%%%%%%%
% \paragraph{Forwarding for Final Version of the Chapters.}
%
% The following forwarding files |cdocsfn1.tex| and |cdocsfn2.tex|
% (with identical content)
% compile the final versions of the child documents
% |cdocsch1.tex| and |cdocsch2.tex|, respectively:
%\iffalse
%<*samplefinal>
%\fi
%    \begin{macrocode}
\def\version{final}
\input{childdoc.def}
\childdocforwardprefix[cdocsamp]{cdocsfn}{cdocsch}
%    \end{macrocode}

%\iffalse
%</samplefinal>
%\fi
%
% %%%%%%%%%%%%%%%%%%%%%%%%%%%%%%%%%%%%%%
% \paragraph{Command Line Processing.}
%
% The following three command lines generate the output files
% |cdocscld|, |cdocscl1| and |cdocscl2|
% which should be identical to
% |cdocsdrf|, |cdocsch1| and |cdocsfn2|, respectively:
% \begin{center}
% \begin{tabular}{l}
% |latex -jobname cdocscld \|\\
% |  "\def\version{draft}\input{childdoc.def}\childdocforward{cdocsamp}"|\\
% |latex -jobname cdocscl1 \|\\
% |  "\input{childdoc.def}\childdocforward[cdocsamp]{cdocsch1}"|\\
% |latex -jobname cdocscl2 \|\\
% |  "\def\version{final}\input{childdoc.def}\childdocforward{cdocsch2}"|
% \end{tabular}
% \end{center}
% Note that the trailing backslash on each first line
% merely continues the input to the second line
% (for convenient cut ant paste).
% Furthermore, the command |latex| can be replaced by any
% of its alternative versions such as |pdflatex|.
%
% %%%%%%%%%%%%%%%%%%%%%%%%%%%%%%%%%%%%%%%%%%%%%%%%%%%%%%%%%%%%%%%%%%%%%%%%%%%%%%
% %%%%%%%%%%%%%%%%%%%%%%%%%%%%%%%%%%%%%%%%%%%%%%%%%%%%%%%%%%%%%%%%%%%%%%%%%%%%%%
% \section{Implementation}
%\iffalse
%<*package>
%\fi
%
% This section describes the definitions file |childdoc.def|.

% The definitions cannot be loaded using |\usepackage| or |\RequirePackage|
% which has a mechanism to prevent loading a style file more than once.
% When loading the definitions by means of |\input|
% multiple instances have to be prevented manually:
%\iffalse
%This code needs to be before the `\ProvidesFile' directive
%which is defined at the beginning of this file.
%Therefore it is also placed there and commented out here.
%</package>
%<*discard>
%\fi
%    \begin{macrocode}
\ifdefined\childdocmain\endinput\fi
%    \end{macrocode}
%\iffalse
%</discard>
%<*package>
%\fi
%
% \macro{\ifchilddoc}
% \macro{\ifchilddocmanual}
% The conditional |\ifchilddoc| tells whether a
% child (true) or main (false) document is being compiled.
% The conditional |\ifchilddocmanual| tells whether
% the |\includeonly| mechanism is used (false) or
% the selection of child files must be performed manually (true).
% The definitions initialise to false:
%    \begin{macrocode}
\newif\ifchilddoc
\newif\ifchilddocmanual
%    \end{macrocode}

% \macro{\childdocname}
% \macro{\childdocjob}
% The macro |\childdocname| stores the name of the main document
% to be compiled. The macro |\childdocjob| stores the name of
% the document on which the \LaTeX{} compiler was originally invoked.
% The content of |\jobname| cannot be compared
% to filenames specified in the source due to different catcodes.
% The following code rescans |\jobname|, stores the result
% in |\childdocname| and saves a copy in |\childdocjob|:
%    \begin{macrocode}
\edef\childdocname{\scantokens\expandafter{\jobname\noexpand}}
\let\childdocjob\childdocname
%    \end{macrocode}

% \macro{\childdocdisable}
% The macro |\childdocdisable| prevents the main file
% from being processed more than once.
% At this stage, the main document command |\childdocmain|
% is assumed to be called once again where it should do nothing.
% Any subsequent call to it should prevent
% a secondary processing of the main document
% It overwrites the forwarding commands
% |\childdocof| and |\childdocforward|
% with empty macros to prevent further inclusions of the main document:
%    \begin{macrocode}
\newcommand{\childdocdisable}
{
  \renewcommand{\childdocmain}[1]{\renewcommand{\childdocmain}[1]{\endinput}}
  \renewcommand{\childdocof}[1]{}
  \renewcommand{\childdocby}[2][]{}
  \renewcommand{\childdocforward}[2][]{}
  \renewcommand{\childdocdisable}{}
}
%    \end{macrocode}

% \macro{\childdocmain}
% The macro |\childdocmain| is to be called at the top of the main file
% with nothing or the main filename (without extension) as argument.
% First, it breaks loops.
% If the argument is not empty and does not match |\childdocname|
% (which is set by the first inclusion of |childdoc.def|),
% |\ifchilddoc| is set to true, |\includeonly| is applied to the child file
% and |\jobname| is set to the main file
% (for proper handling of |.aux| files):
%    \begin{macrocode}
\newcommand{\childdocmain}[1]
{
  \childdocdisable\childdocmain{}
  \if?#1?\else
    \begingroup
      \def\childdoctmp{#1}
      \ifx\childdoctmp\childdocname
        \def\childdoctmp{}
      \else
        \def\childdoctmp
        {
          \childdoctrue
          \includeonly{\childdocname}
          \def\childdocjob{#1}
          \def\jobname{#1}
        }
      \fi
      \expandafter
    \endgroup
    \childdoctmp
  \fi
}
%    \end{macrocode}

% \macro{\childdocof}
% The command |\childdocof| redirects
% compilation to the main file |#1|.
%    \begin{macrocode}
\newcommand{\childdocof}[1]
{
  \childdocdisable
  \childdoctrue
  \includeonly{\childdocname}
  \def\jobname{#1}
  \def\childdocjob{#1}
  \input{#1}
}
%    \end{macrocode}

% \macro{\childdocby}
% The command |\childdocby| ....
%    \begin{macrocode}
\newcommand{\childdocby}[2][]
{
  \childdocdisable
  \childdoctrue
  \childdocmanualtrue
  \if?#1?\else
    \def\jobname{#2}
  \fi
  \def\childdocjob{#2}
  \input{#2}
  \endinput
}
%    \end{macrocode}

% \macro{\childdocforward}
% The command |\childdocforward| redirects
% compilation to the main file or
% (if the optional argument is given) a child file.
% Parameters are set as if the main file
% or a child file starting with |\childdocof| was compiled.
% Then compilation is handed over to the main file:
%    \begin{macrocode}
\newcommand{\childdocforward}[2][]
{
  \begingroup
    \if?#1?
      \def\childdoctmp
      {
        \def\childdocname{#2}
        \def\childdocjob{#2}
        \def\jobname{#2}
        \input{#2}
        \endinput
      }
    \else
      \def\childdoctmp
      {
        \childdocdisable
        \def\childdocname{#2}
        \childdoctrue
        \includeonly{#2}
        \def\childdocjob{#1}
        \def\jobname{#1}
        \input{#1}
        \endinput
      }
    \fi
    \expandafter
  \endgroup
  \childdoctmp
}
%    \end{macrocode}

% \macro{\childdocforwardprefix}
% The command |\childdocforwardprefix| redirects
% compilation to the main or a child file by means of a pattern.
% The prefix |#1| in the current filename is replaced by |#2|
% and the suffix of the current filename is kept
% (it is assumed that the filename does not contain the substring `|~~~|'
% which is used as a delimiter).
% Compilation is handed over to the new file by |\childdocforward|:
%    \begin{macrocode}
\newcommand{\childdocforwardprefix}[3][]
{
  \begingroup
    \def\childdocextract #2##1~~~{\def\childdoctmp{\childdocforward[#1]{#3##1}}}
    \expandafter\childdocextract\childdocname~~~
    \expandafter
  \endgroup
  \childdoctmp
}
%    \end{macrocode}

% \macro{\childdoc}
% The deprecated macro |\childdoc| is a legacy version of |\childdocmain|:
%    \begin{macrocode}
\newcommand{\childdoc}{\childdocmain}
%    \end{macrocode}

% \macro{\childdocredirect}
% The deprecated macro |\childdocredirect| is a legacy version
% of |\childdocforward| and |\childdocforwardprefix|:
%    \begin{macrocode}
\newcommand{\childdocredirect}[2][]
{
  \begingroup
    \if?#1?
      \def\childdoctmp{\childdocforward{#2}}
    \else
      \def\childdoctmp{\childdocforwardprefix{#1}{#2}}
    \fi
    \expandafter
  \endgroup
  \childdoctmp
}
%    \end{macrocode}

%\iffalse
%</package>
%\fi
%
\endinput

\childdocforwardprefix[cdocsamp]{cdocsfn}{cdocsch}
%    \end{macrocode}

%\iffalse
%</samplefinal>
%\fi
%
% %%%%%%%%%%%%%%%%%%%%%%%%%%%%%%%%%%%%%%
% \paragraph{Command Line Processing.}
%
% The following three command lines generate the output files
% |cdocscld|, |cdocscl1| and |cdocscl2|
% which should be identical to
% |cdocsdrf|, |cdocsch1| and |cdocsfn2|, respectively:
% \begin{center}
% \begin{tabular}{l}
% |latex -jobname cdocscld \|\\
% |  "\def\version{draft}% \iffalse
%
% childdoc.dtx Copyright (C) 2017-2018 Niklas Beisert
%
% This work may be distributed and/or modified under the
% conditions of the LaTeX Project Public License, either version 1.3
% of this license or (at your option) any later version.
% The latest version of this license is in
%   http://www.latex-project.org/lppl.txt
% and version 1.3 or later is part of all distributions of LaTeX
% version 2005/12/01 or later.
%
% This work has the LPPL maintenance status `maintained'.
%
% The Current Maintainer of this work is Niklas Beisert.
%
% This work consists of the files childdoc.dtx and childdoc.ins
% and the derived files childdoc.def and cdocsamp.tex with
% cdocsch1.tex, cdocsch2.tex, cdocsdrf.tex, cdocsfn1.tex, cdocsfn2.tex.
%
%<package>\ifdefined\childdocmain\endinput\fi
%<package>\ProvidesFile{childdoc.def}[2018/12/30 v2.0 child document driver]
%<samplemain>\ProvidesFile{cdocsamp.tex}[2018/12/30 v2.0 sample for childdoc]
%<*driver>
%\ProvidesFile{childdoc.drv}[2018/12/30 v2.0 childdoc reference manual file]
\PassOptionsToClass{10pt,a4paper}{article}
\documentclass{ltxdoc}

\usepackage[margin=35mm]{geometry}
\usepackage{hyperref}
\usepackage{hyperxmp}
\usepackage[usenames]{color}

\hypersetup{colorlinks=true}
\hypersetup{pdfstartview=FitH}
\hypersetup{pdfpagemode=UseNone}
\hypersetup{pdfsource={}}
\hypersetup{pdflang={en-UK}}
\hypersetup{pdfcopyright={Copyright 2017-2018 Niklas Beisert.
  This work may be distributed and/or modified under the
  conditions of the LaTeX Project Public License, either version 1.3
  of this license or (at your option) any later version.}}
\hypersetup{pdflicenseurl={http://www.latex-project.org/lppl.txt}}
\hypersetup{pdfcontactaddress={ETH Zurich, ITP, HIT K,
  Wolfgang-Pauli-Strasse 27}}
\hypersetup{pdfcontactpostcode={8093}}
\hypersetup{pdfcontactcity={Zurich}}
\hypersetup{pdfcontactcountry={Switzerland}}
\hypersetup{pdfcontactemail={nbeisert@itp.phys.ethz.ch}}
\hypersetup{pdfcontacturl={http://people.phys.ethz.ch/\xmptilde nbeisert/}}

\newcommand{\secref}[1]{\hyperref[#1]{section \ref*{#1}}}

\parskip1ex
\parindent0pt
\let\olditemize\itemize
\def\itemize{\olditemize\parskip0pt}

\begin{document}

\title{The \textsf{childdoc} Package}
\hypersetup{pdftitle={The childdoc Package}}
\author{Niklas Beisert\\[2ex]
  Institut f\"ur Theoretische Physik\\
  Eidgen\"ossische Technische Hochschule Z\"urich\\
  Wolfgang-Pauli-Strasse 27, 8093 Z\"urich, Switzerland\\[1ex]
  \href{mailto:nbeisert@itp.phys.ethz.ch}
  {\texttt{nbeisert@itp.phys.ethz.ch}}}
\hypersetup{pdfauthor={Niklas Beisert}}
\hypersetup{pdfsubject={Manual for the LaTeX2e Package childdoc}}
\date{30 December 2018, \textsf{v2.0}}
\maketitle

\begin{abstract}\noindent
\textsf{childdoc} is a \LaTeXe{} package
that enables the direct compilation
of document sections included by |\include|
to individual files.
\end{abstract}

\begingroup
\parskip0ex
\tableofcontents
\endgroup

%%%%%%%%%%%%%%%%%%%%%%%%%%%%%%%%%%%%%%%%%%%%%%%%%%%%%%%%%%%%%%%%%%%%%%%%%%%%%%%%
%%%%%%%%%%%%%%%%%%%%%%%%%%%%%%%%%%%%%%%%%%%%%%%%%%%%%%%%%%%%%%%%%%%%%%%%%%%%%%%%
\section{Introduction}

\LaTeX{} provides a mechanism to structure a large document (such as a book)
into a main file and several child files (containing the chapters)
using the |\include| command.
This mechanism is beneficial for documents
which span hundreds of pages in order to
make the source file(s) more manageable.
Moreover, compilation can be restricted to
selected child files by means of the |\includeonly| command.
The latter feature can be used to reduce the compilation time while editing
(this was significantly more useful in the earlier days of \LaTeX{})
or to generate a smaller document which is easier to navigate.
Another application of |\includeonly| is to generate
documents consisting of selected parts of the complete document.

However, there are a few drawbacks of the plain |\include| mechanism:
\begin{itemize}
\item
The child files cannot be compiled on their own,
they can only be compiled via the main file.
A naive editing environment
(such as a text editor with an option
to have the current file processed by \LaTeX)
may require one to switch to the main file before compiling;
attempting to compile the child file produces errors.
\item
The main file must be modified (each time)
to adjust the |\includeonly| command
to the present needs. This easily leaves the main file in a messy state.
\item
The generated document will always carry the filename
of the main document. This is inconvenient if
several child files are to be compiled and
to be kept for distribution.
\end{itemize}

The present package provides a simple interface
to make child files individually compilable by \LaTeX{}.
Compiling a child file then has the same effect as compiling
the main file with an |\includeonly| command
to select the appropriate child.
Moreover the generated document will carry the name of the child
rather than the main file.
This resolves all three above issues.

This feature is meant to make the editing of books,
thesis documents and lecture notes somewhat more convenient.
However, the package can also be used efficiently for
composing a series of documents (such as exercise sheets)
which are typically distributed individually.
It then assists the author in generating the individual documents
(potentially in different versions)
as well as a document containing the collected series.
Another application is in developing style files
or other kinds of included material
where compilation of the style file could redirect
to a sample or test file.

%%%%%%%%%%%%%%%%%%%%%%%%%%%%%%%%%%%%%%%%%%%%%%%%%%%%%%%%%%%%%%%%%%%%%%%%%%%%%%%%
%%%%%%%%%%%%%%%%%%%%%%%%%%%%%%%%%%%%%%%%%%%%%%%%%%%%%%%%%%%%%%%%%%%%%%%%%%%%%%%%
\section{Usage}

First of all, the package \textsf{childdoc} is \emph{not} a standard
\LaTeXe{} |.sty| style file! Therefore it needs to be invoked in
a non-standard way.

%%%%%%%%%%%%%%%%%%%%%%%%%%%%%%%%%%%%%%%%%%%%%%%%%%%%%%%%%%%%%%%%%%%%%%%%%%%%%%%%
\subsection{Included Files}
\label{sec:include}

%%%%%%%%%%%%%%%%%%%%%%%%%%%%%%%%%%%%%%%%
\DescribeMacro{\childdocmain}
To use the package, add the commands
\begin{center}
\begin{tabular}{l}
|\input{childdoc.def}|\\
|\childdocmain{}|\\
\end{tabular}
\end{center}
at the very top of the main \LaTeX{} file,
in particular \emph{before} the |\documentclass| statement!
The argument of |\childdocmain| should be left empty
(but it must be present).

%%%%%%%%%%%%%%%%%%%%%%%%%%%%%%%%%%%%%%%%
\DescribeMacro{\childdocof}
Furthermore, add the commands
\begin{center}
\begin{tabular}{l}
|\input{childdoc.def}|\\
|\childdocof{|\textit{main}|}|\\
\end{tabular}
\end{center}
at the top of every child file \textit{child}
which is included by |\include{|\textit{child}|}|
from within the main file
(or at least for those files to be compiled individually).
The argument \textit{main} must be the filename of the main file.

There are a couple of
considerations in setting up the main and child documents:

%%%%%%%%%%%%%%%%%%%%%%%%%%%%%%%%%%%%%%%%
\paragraph{Restrictions.}

Please note the following restrictions:
\begin{itemize}
\item
|\childdocmain| must be called with one argument \textit{main}
to ensure compatibility with earlier version of the package.
It must either be empty (|\childdocmain{}|)
or precisely match the filename of the main file in which it is specified.
See \secref{sec:detection} for further information.
\item
The filename \textit{main} must be specified without the |.tex| extension.
\item
The filename \textit{main} is case sensitive
(even in case-insensitive file systems)
due to internal string comparison.
\item
The argument \textit{main} should be fully expanded, it cannot be a macro.
\item
Subdirectories and special characters should be avoided in filenames.
\item
The command |\childdocmain{|\textit{main}|}| must be followed by a whitespace.
It should not be followed immediately by another command
or by a comment mark `|%|'.
This is because the \TeX{} parser reads the token immediately following
the argument of |\childdocmain| and puts it
at the beginning of every child section;
however, a white\-space is ignored.
\end{itemize}

%%%%%%%%%%%%%%%%%%%%%%%%%%%%%%%%%%%%%%%%
\paragraph{Content of Main File.}

It is advisable to place all content in the child files included by |\include|.
Any output contained in the main file will appear in all child documents
unless suppressed manually;
it cannot be suppressed automatically by the |\includeonly| directive
and thus should normally be avoided.
A method to include some content in the main file
by means of conditional processing is described in \secref{sec:conditional}.

%%%%%%%%%%%%%%%%%%%%%%%%%%%%%%%%%%%%%%%%
\paragraph{Page Numbering.}

When only a part of the document is compiled,
the appropriate numbering of pages
(as well as other status parameters)
is determined from the |.aux| files.
The latter contain information from previous passes.
However this information needs to propagate through
all intermediate child documents.
Therefore the page numbering in child documents may well
be inconsistent until the complete document is compiled at least once.

A useful (if unconventional) way to always ensure a consistent
page numbering is to restart the numbering in each child document
and denote the pages by `\textit{child}|.|\textit{page}'
where \textit{child} represents the chapter/section number of the child file.
This can be achieved by the command
|\numberwithin{page}{|\textit{child}|}|
of the \textsf{amsmath} package
where \textit{child} can be |chapter| or |section|
depending on the chosen structuring.
Alternatively, one can modify the macro |\thepage| appropriately
and reset the counter |page| at the start of each child file.

%%%%%%%%%%%%%%%%%%%%%%%%%%%%%%%%%%%%%%%%%%%%%%%%%%%%%%%%%%%%%%%%%%%%%%%%%%%%%%%%
\subsection{Conditional Processing}
\label{sec:conditional}

The package provides a mechanism to compile different versions
of a document. To customise the versions further some conditional processing
can come in handy to distinguish which version is being compiled.
The package provides two macros to describe the compilation context:

%%%%%%%%%%%%%%%%%%%%%%%%%%%%%%%%%%%%%%%%
\DescribeMacro{\ifchilddoc}
The conditional |\ifchilddoc| distinguishes between the compilation of
child documents and the main document:
%
\begin{center}
|\ifchilddoc |\textit{child-code}| |[|\||else |\textit{main-code}]| \||fi|
\end{center}

%%%%%%%%%%%%%%%%%%%%%%%%%%%%%%%%%%%%%%%%
\DescribeMacro{\childdocname}
\DescribeMacro{\childdocjob}
The macro |\childdocname| contains the filename (without extension)
of the main or child file being processed.
Note that |\childdocjob| will always contain the name of the main file.

%%%%%%%%%%%%%%%%%%%%%%%%%%%%%%%%%%%%%%%%
\paragraph{Title Page.}

Conditional processing can be used to include a title or banner page
in the main document when proper precautions are taken.
Importantly, the code in the main file should ensure that the page counter
(as well as other status parameters which are stored in the |.aux| files)
takes the same value after the conditional processing.
Otherwise the page numbers may take divergent values
depending on which part is compiled.

For example, a title page could be declared by:
%
\begin{center}
\begin{tabular}{l}
|\ifchilddoc\||else|\\
|\addtocounter{page}{-1}|\\
\textit{code for title page}\\
|\newpage|\\
|\||fi|
\end{tabular}
\end{center}
%
A banner page for the child documents can be generated by:
%
\begin{center}
\begin{tabular}{l}
|\ifchilddoc|\\
|\addtocounter{page}{-1}|\\
\textit{code for banner page}\\
|\newpage|\\
|\||fi|
\end{tabular}
\end{center}
%
Here one could write a message such as:
\begin{center}
|This is the part \childdocname{} of \childdocjob{}.|
\end{center}

%%%%%%%%%%%%%%%%%%%%%%%%%%%%%%%%%%%%%%%%%%%%%%%%%%%%%%%%%%%%%%%%%%%%%%%%%%%%%%%%
\subsection{Flags}
\label{sec:flags}

The package makes it easy to generate different versions
of the main or child documents.
To this end compilation flags can be defined
and assigned different default values.
They will be particularly useful in conjunction
with the forwarding mechanism described in \secref{sec:forward}.

For example, it may be useful to have a flag |\version|
which can be set to |draft| or |final|.
The document source will contain some conditional code
depending on the value of |\version|.
Suppose further, the flag should default to |final| for the main file
and to |draft| for child files
which is a natural assignment for editing the document.
This is achieved by placing the following code
in the preamble of the main document
(below the |\childdocmain| directive):
%
\begin{center}
\begin{tabular}{l}
|\ifchilddoc|\\
|\providecommand{\version}{draft}|\\
|\||else|\\
|\providecommand{\version}{final}|\\
|\||fi|
\end{tabular}
\end{center}
%
The definition by |\providecommand| makes sure
that previous definitions are not overwritten.
Further statements |\providecommand{\version}{...}|
can thus be added before the above code to override it.

For the main file, one might add a line
(between |\childdocmain| and the above block)
%
\begin{center}
|%\ifchilddoc\||else\providecommand{\version}{draft}\||fi|
\end{center}
%
which can be uncommented to produce a draft version.
Likewise one can add a line to the very top of a child file
(above the |\childdocof{|\textit{main}|}| directive)
%
\begin{center}
|%\providecommand{\version}{final}|
\end{center}
%
which can be uncommented to produce the final version of this child document.

%%%%%%%%%%%%%%%%%%%%%%%%%%%%%%%%%%%%%%%%%%%%%%%%%%%%%%%%%%%%%%%%%%%%%%%%%%%%%%%%
\subsection{Forwarding}
\label{sec:forward}

Different versions of the main or child documents
using compilation flags as described in \secref{sec:flags}
can be (permanently) stored in different files
for convenient compilation, viewing and distribution.
To this end, the package defines a command
to pass on compilation to a different file:

%%%%%%%%%%%%%%%%%%%%%%%%%%%%%%%%%%%%%%%%
\DescribeMacro{\childdocforward}
The command |\childdocforward| redirects processing to
another source file:
%
\begin{center}
\begin{tabular}{l}
|\input{childdoc.def}|\\
|\childdocforward[|\textit{main}|]{|\textit{dest}|}|\\
\end{tabular}
\end{center}
%
The argument \textit{dest} is the destination file
(without extension).
It should be the main file or one of the child files.
Note that further \textsf{childdoc} directives
such as |\childdocof| and |\childdocforward|
in the indicated file will be processed in this form.
The optional argument \textit{main}
passes on directly to the main file \textit{main}
while pretending to compile the child \textit{dest}.
This form behaves as if \textit{dest}
issues |\childdocof{|\textit{main}|}| right away,
and no further \textsf{childdoc} directives will be processed.

%%%%%%%%%%%%%%%%%%%%%%%%%%%%%%%%%%%%%%%%
\DescribeMacro{\...prefix}
In the alternative form |\childdocforwardprefix|,
%
\begin{center}
\begin{tabular}{l}
|\input{childdoc.def}|\\
|\childdocforwardprefix[|\textit{main}|]{|\textit{prefix}|}{|\textit{dest}|}|
\end{tabular}
\end{center}
%
the destination file is determined by a pattern
depending on the current file:
To make this work, the current file must be called
`{\textit{prefix}\hspace{0.2em}\textit{suffix}}'
with \textit{prefix} matching precisely the argument.
Processing is then passed on to the file
`{\textit{dest}\hspace{0.2em}\textit{suffix}}'.
Surely, the same effect is achieved by
directly specifying the
argument `{\textit{dest}\hspace{0.2em}\textit{suffix}}'
in the first form.
However, that requires to set up a different file
for each child. With the alternative form of the command
all these files can have exactly the same content
which simplifies setting them up and maintaining them.

For example, the following file |draft.tex|
with a compilation flag |\version| as described in \secref{sec:flags}
compiles the main document as a draft:
%
\begin{center}
\begin{tabular}{l}
|\def\version{draft}|\\
|\input{childdoc.def}|\\
|\childdocforward{|\textit{main}|}|
\end{tabular}
\end{center}
%
Likewise, the following files |final|\textit{nn}|.tex|
compile the final version of the child document
|child|\textit{nn}|.tex|:
%
\begin{center}
\begin{tabular}{l}
|\def\version{final}|\\
|\input{childdoc.def}|\\
|\childdocforwardprefix{final}{child}|
\end{tabular}
\end{center}
%

Note that when several versions of a main file and/or of each child file
are to be generated, it may be convenient to set up a |Makefile| or
shell script to automatise the process.

%%%%%%%%%%%%%%%%%%%%%%%%%%%%%%%%%%%%%%%%%%%%%%%%%%%%%%%%%%%%%%%%%%%%%%%%%%%%%%%%
\subsection{Command Line Processing}
\label{sec:commandline}

The effect of redirection files can also be achieved by invoking
the \LaTeX{} compiler with a more elaborate command line.
Most conveniently this should be done as part
of a shell script or a |Makefile|.

When using \textsf{childdoc} in the main file, the following
command lines effectively perform a redirection
(note that depending on the shell being used,
backslashes may have to be doubled: `|\|' $\to$ `|\\|'):
%
\begin{center}
|... -jobname "|\textit{target}|" |\\|"|[\textit{flags}]%
|\input{childdoc.def}\childdocforward[|\textit{main}|]{|\textit{dest}|}"|
\end{center}
%
Here \textit{target} is the name of the output file,
\textit{main} is the name of the main file
and \textit{dest} is the name of the main or child file to be processed
(all filenames without extensions).
The optional argument \textit{main} can be omitted
if \textit{main} matches \textit{dest}.
Optionally, compilation \textit{flags} can be defined via |\def| commands.
This command line makes the \TeX{} engine believe
it is compiling the file \textit{target}
whose content is specified as the latter parameter.
The provided code then forwards the processing to
\textit{main} or \textit{dest} as described in \secref{sec:forward}.

%%%%%%%%%%%%%%%%%%%%%%%%%%%%%%%%%%%%%%%%%%%%%%%%%%%%%%%%%%%%%%%%%%%%%%%%%%%%%%%%
\subsection{Include by Input}
\label{sec:input}

Including child documents by |\include| has some restrictions by design.
Most notably, the content of a child document always occupies
its own set of pages; pages cannot be shared between child documents.
Usually, this behaviour makes perfect sense
because each child document contain an essential part of the document.
However, in some situations it may be desirable to compose
a document from a collection of parts
without having mandatory page breaks between then.
For this case, the package
provides a mechanism to include parts
by |\input| which can also be processed individually.
However, by construction this mechanism
requires manual handling of the content to be output.

%%%%%%%%%%%%%%%%%%%%%%%%%%%%%%%%%%%%%%%%
\DescribeMacro{\ifchilddocmanual}
The main file should be prepared as usual, see \secref{sec:include}.
However, the document body must make a distinction
between processing of an individual part and of the main document, e.g.:
%
\begin{center}
\begin{tabular}{l}
|\ifchilddocmanual|\\
|\input{\childdocname}|\\
|\||else|\\
\textit{document body with }|\input{|\textit{part}|}|\\
|\||fi|
\end{tabular}
\end{center}
%
The conditional |\ifchilddocmanual| is true whenever
a part to be included by |\input| is being compiled,
and the name of the part is stored in |\childdocname|.

%%%%%%%%%%%%%%%%%%%%%%%%%%%%%%%%%%%%%%%%
\DescribeMacro{\childdocby}
Each part to be included by |\input| should start with:
%
\begin{center}
\begin{tabular}{l}
|\input{childdoc.def}|\\
|\childdocby{|\textit{main}|}|\\
\end{tabular}
\end{center}
%
The directive |\childdocby| is similar to |\childdocof|
described in \secref{sec:include},
but the subsequent selection of content must be done manually.
To that end, both |\ifchilddoc| and |\ifchilddocmanual|
will be true upon processing of a part,
and the name of the part is stored in |\childdocname|.
Note that |\jobname| will be set to the filename of the current part
so that each part receives an individual |.aux| file
that does not interfere with the |.aux| file(s) of the main document.
This behaviour can be altered by the alternative form
|\childdocby[*]{|\textit{main}|}| (with a non-empty optional argument)
which uses the |.aux| file of the main document
by setting |\jobname| to \textit{main}.

%%%%%%%%%%%%%%%%%%%%%%%%%%%%%%%%%%%%%%%%%%%%%%%%%%%%%%%%%%%%%%%%%%%%%%%%%%%%%%%%
\subsection{Driver Development}
\label{sec:driver}

The \textsf{childdoc} mechanism can also be use for the development
of definition files such as \LaTeX{} styles or classes.
This case differs from the above setup with multiple parts
included by |\include| in that no |\includeonly| should be invoked.
This can be achieved by starting the include file
(before |\ProvidesPackage|) with:
%
\begin{center}
\begin{tabular}{l}
|\input{childdoc.def}|\\
|\childdocforward{|\textit{main}|}|\\
\end{tabular}
\end{center}
%
or alternatively with:
%
\begin{center}
\begin{tabular}{l}
|\input{childdoc.def}|\\
|\childdocby{|\textit{main}|}|\\
\end{tabular}
\end{center}
%
Both forms have slightly different effects as described above.
The main file is prepared as usual, see \secref{sec:include}.

%%%%%%%%%%%%%%%%%%%%%%%%%%%%%%%%%%%%%%%%%%%%%%%%%%%%%%%%%%%%%%%%%%%%%%%%%%%%%%%%
\subsection{Legacy Detection}
\label{sec:detection}

The directive |\childdocmain| in the main file can detect
whether the complete document or merely a child is to be compiled
even without using the directive |\childdocof|.
This method is deprecated because it is less robust
and there is no compelling reason to use it;
it is merely provided for backward compatibility
and it may be removed in future versions.

If the detection mechanism is to be used,
it is mandatory to correctly specify
the filename of the main file as the argument of |\childdocmain|:
%
\begin{center}
\begin{tabular}{l}
|\input{childdoc.def}|\\
|\childdocmain{|\textit{main}|}|\\
\end{tabular}
\end{center}
%
If |\jobname| does not match the argument \textit{main} of |\childdocmain|,
it is assumed that |\jobname| points to the child file to be compiled.
When using |\childdocmain| with the main file specified as argument,
it suffices to start a child file
with just |\input{|\textit{main}|}|
without loading of the package and using |\childdocof|.
If instead all processing is done
with the appropriate \textsf{childdoc} directives,
the argument of \textit{main} of |\childdocmain| can be empty.

An alternative version of the command line processing described
in \secref{sec:commandline} using the detection mechanism reads:
%
\begin{center}
|... -jobname "|\textit{target}|" "|[\textit{flags}]%
[|\def\jobname{|\textit{dest}|}|]|\input{|\textit{main}|}"|
\end{center}

%%%%%%%%%%%%%%%%%%%%%%%%%%%%%%%%%%%%%%%%%%%%%%%%%%%%%%%%%%%%%%%%%%%%%%%%%%%%%%%%
\subsection{Manual Code}
\label{sec:manual}

In case one cannot be certain whether the definitions file |childdoc.def|
is installed on the target \TeX{} distribution
and one prefers not to ship it,
it is conceivable to paste a few relevant commands into the sources.

To that end, drop all statements |\input{childdoc.def}|
and perform the replacements as outlined below.
Instead of |\childdocmain{|\textit{main}|}| add the following code
to the top of the main file:
%
\begin{center}
\begin{tabular}{l}
|\||ifdefined\childdocname\endinput\||fi\newif\ifchilddoc|\\
|\edef\childdocname{\scantokens\expandafter{\jobname\noexpand}}|\\
|\def\childdocmain{|\textit{main}|}\||ifx\childdocmain\childdocname\||else|\\
|\childdoctrue\includeonly{\childdocname}\let\jobname\childdocmain\||fi|\\
\end{tabular}
\end{center}
%
Instead of |\childdocof{|\textit{main}|}| just include the main file
at the top of each child file:
%
\begin{center}
|\input{|\textit{main}|}|
\end{center}
%
A simple redirection |\childdocforward{|\textit{dest}|}| is achieved by:
%
\begin{center}
|\def\jobname{|\textit{dest}|}\input{\jobname}|
\end{center}
%
The redirection with prefix
|\childdocforwardprefix[|\textit{prefix}|]{|\textit{dest}|}|
is accomplished by:
%
\begin{center}
\begin{tabular}{l}
|{\edef\jobname{\scantokens\expandafter{\jobname\noexpand}}|\\
|\def\redirectjob |\textit{prefix}|#1~~~{\gdef\jobname{|\textit{dest}|#1}}|\\
|\expandafter\redirectjob\jobname~~~}\input{\jobname}|
\end{tabular}
\end{center}

In an alternative approach,
child documents can be compiled by a specific command line
without additional code or specific definitions:
%
\begin{center}
|... -jobname "|\textit{target}|" "|[\textit{flags}]%
|\includeonly{|\textit{dest}|}\input{|\textit{main}|}"|
\end{center}
%

%%%%%%%%%%%%%%%%%%%%%%%%%%%%%%%%%%%%%%%%%%%%%%%%%%%%%%%%%%%%%%%%%%%%%%%%%%%%%%%%
%%%%%%%%%%%%%%%%%%%%%%%%%%%%%%%%%%%%%%%%%%%%%%%%%%%%%%%%%%%%%%%%%%%%%%%%%%%%%%%%
\section{Information}

%%%%%%%%%%%%%%%%%%%%%%%%%%%%%%%%%%%%%%%%%%%%%%%%%%%%%%%%%%%%%%%%%%%%%%%%%%%%%%%%
\subsection{Copyright}

Copyright \copyright{} 2017--2018 Niklas Beisert

This work may be distributed and/or modified under the
conditions of the \LaTeX{} Project Public License, either version 1.3
of this license or (at your option) any later version.
The latest version of this license is in
  \url{http://www.latex-project.org/lppl.txt}
and version 1.3 or later is part of all distributions of \LaTeX{}
version 2005/12/01 or later.

This work has the LPPL maintenance status `maintained'.

The Current Maintainer of this work is Niklas Beisert.

This work consists of the files |README.txt|, |childdoc.ins| and |childdoc.dtx|
as well as the derived files |childdoc.def|, |cdocsamp.tex|
with |cdocsch1.tex|, |cdocsch2.tex|, |cdocspt3.tex|, |cdocspt4.tex|,
|cdocsdrf.tex|, |cdocsfn1.tex|, |cdocsfn2.tex|
as well as |childdoc.pdf|.

%%%%%%%%%%%%%%%%%%%%%%%%%%%%%%%%%%%%%%%%%%%%%%%%%%%%%%%%%%%%%%%%%%%%%%%%%%%%%%%%
\subsection{Files and Installation}

The package consists of the files:
%
\begin{center}
\begin{tabular}{ll}
    |README.txt|   & readme file \\
    |childdoc.ins| & installation file \\
    |childdoc.dtx| & source file \\
    |childdoc.def| & definition file \\
    |cdocsamp.tex| & sample main file \\
    |cdocsch1.tex| & sample include file \\
    |cdocsch2.tex| & sample include file \\
    |cdocspt3.tex| & sample part file \\
    |cdocspt4.tex| & sample part file \\
    |cdocsdrf.tex| & sample redirection file \\
    |cdocsfn1.tex| & sample redirection file \\
    |cdocsfn2.tex| & sample redirection file \\
    |childdoc.pdf| & manual
\end{tabular}
\end{center}
%
The distribution consists of the files
|README.txt|, |childdoc.ins| and |childdoc.dtx|.
%
\begin{itemize}
\item
Run (pdf)\LaTeX{} on |childdoc.dtx|
to compile the manual |childdoc.pdf| (this file).
\item
Run \LaTeX{} on |childdoc.ins| to create the definitions file |childdoc.def|
and the sample |cdocsamp.tex| with include files
|cdocsch1.tex|, |cdocsch2.tex|, |cdocspt3.tex|, |cdocspt4.tex|,
|cdocsdrf.tex|, |cdocsfn1.tex|, |cdocsfn2.tex|.
Then copy the file |childdoc.def| to an appropriate directory of your \LaTeX{}
distribution, e.g.\ \textit{texmf-root}|/tex/latex/childdoc|.
\end{itemize}

%%%%%%%%%%%%%%%%%%%%%%%%%%%%%%%%%%%%%%%%%%%%%%%%%%%%%%%%%%%%%%%%%%%%%%%%%%%%%%%%
\subsection{Related CTAN Packages}

There are several other packages which offer a similar functionality:
%
\begin{itemize}
\item
The packages
\href{http://ctan.org/pkg/docmute}{\textsf{docmute}},
\href{http://ctan.org/pkg/includex}{\textsf{includex}} and
\href{http://ctan.org/pkg/standalone}{\textsf{standalone}}
provide commands to include only the document body of
a child file thus allowing both files to be compiled individually.
\item
The packages \href{http://ctan.org/pkg/subdocs}{\textsf{subdocs}}
and \href{http://ctan.org/pkg/subfiles}{\textsf{subfiles}}
provide structures in which the main and child documents can be
encapsulated and allowing them to be compiled individually.
The inclusion mechanism is different from the conventional |\include|.
\item
The package \href{http://ctan.org/pkg/combine}{\textsf{combine}}
is an elaborate solution to combine several documents into one.
\end{itemize}
%
See also the CTAN topic \href{http://ctan.org/topic/subdocs}{\textsf{subdocs}}
for further related packages.
The present package differs from the above solutions in that
a document structure constructed with the conventional |\include| mechanism
just needs two extra commands at the top of every file
such that all constituent files can be compiled individually.

%%%%%%%%%%%%%%%%%%%%%%%%%%%%%%%%%%%%%%%%%%%%%%%%%%%%%%%%%%%%%%%%%%%%%%%%%%%%%%%%
%\subsection{Feature Suggestions}
%
%The following is a list of features which may be useful for future
%versions of this package:
%%
%\begin{itemize}
%\item
%\ldots
%\end{itemize}

%%%%%%%%%%%%%%%%%%%%%%%%%%%%%%%%%%%%%%%%%%%%%%%%%%%%%%%%%%%%%%%%%%%%%%%%%%%%%%%%
\subsection{Revision History}

%%%%%%%%%%%%%%%%%%%%%%%%%%%%%%%%%%%%%%%%
\paragraph{v2.0:} 2018/12/30

\begin{itemize}
\item
immediate forward processing
\item
added |\childdocby| mechanism
\item
manual restructured
\end{itemize}

%%%%%%%%%%%%%%%%%%%%%%%%%%%%%%%%%%%%%%%%
\paragraph{v1.6:} 2018/01/17

\begin{itemize}
\item
application for development of include files
\item
corrections to manual
\end{itemize}

%%%%%%%%%%%%%%%%%%%%%%%%%%%%%%%%%%%%%%%%
\paragraph{v1.5:} 2017/05/21

\begin{itemize}
\item
more complete structuring introduced
\item
|\childdocof| introduced
\item
|\childdoc| renamed to |\childdocmain|
\item
|\childredirect| renamed to |\childdocforward| and |\childdocforwardprefix|
and functionality expanded
\end{itemize}

%%%%%%%%%%%%%%%%%%%%%%%%%%%%%%%%%%%%%%%%
\paragraph{v1.0:} 2017/04/27

\begin{itemize}
\item
manual and install package
\item
first version published on CTAN
\end{itemize}

%%%%%%%%%%%%%%%%%%%%%%%%%%%%%%%%%%%%%%%%
\paragraph{v0.6:} 2017/04/26

\begin{itemize}
\item
redirection mechanism added
\end{itemize}

%%%%%%%%%%%%%%%%%%%%%%%%%%%%%%%%%%%%%%%%
\paragraph{v0.5:} 2017/04/26

\begin{itemize}
\item
functionality in definition file
\end{itemize}


%%%%%%%%%%%%%%%%%%%%%%%%%%%%%%%%%%%%%%%%%%%%%%%%%%%%%%%%%%%%%%%%%%%%%%%%%%%%%%%%
%%%%%%%%%%%%%%%%%%%%%%%%%%%%%%%%%%%%%%%%%%%%%%%%%%%%%%%%%%%%%%%%%%%%%%%%%%%%%%%%
%%%%%%%%%%%%%%%%%%%%%%%%%%%%%%%%%%%%%%%%%%%%%%%%%%%%%%%%%%%%%%%%%%%%%%%%%%%%%%%%
\appendix

\settowidth\MacroIndent{\rmfamily\scriptsize 000\ }

 \DocInput{childdoc.dtx}

\end{document}
%</driver>
% \fi
%
% %%%%%%%%%%%%%%%%%%%%%%%%%%%%%%%%%%%%%%%%%%%%%%%%%%%%%%%%%%%%%%%%%%%%%%%%%%%%%%
% %%%%%%%%%%%%%%%%%%%%%%%%%%%%%%%%%%%%%%%%%%%%%%%%%%%%%%%%%%%%%%%%%%%%%%%%%%%%%%
% \section{Sample}
%\iffalse
%<*samplemain>
%\fi
%
% The following presents a sample document
% with two chapters, two parts, a title page,
% a compile flag as well as three forwarding files to set the flag.
% It consists of eight |.tex| files:
% \begin{center}
% \begin{tabular}{ll}
% |cdocsamp.tex|&main file\\
% |cdocsch1.tex|&include file for chapter 1\\
% |cdocsch2.tex|&include file for chapter 2\\
% |cdocspt3.tex|&include file for part 3\\
% |cdocspt4.tex|&include file for part 4\\
% |cdocsdrf.tex|&forwarding file for main file in draft mode\\
% |cdocsfi1.tex|&forwarding file for final version of chapter 1\\
% |cdocsfi2.tex|&forwarding file for final version of chapter 2\\
% \end{tabular}
% \end{center}
% Each of the eight files can be compiled directly by the \LaTeX{} compiler.
%
% %%%%%%%%%%%%%%%%%%%%%%%%%%%%%%%%%%%%%%
% \paragraph{Main File.}
%
% The main file is called |cdocsamp.tex|.
%
% Load the \textsf{childdoc} definitions and
% declare the filename for the main document:
%    \begin{macrocode}
\input{childdoc.def}
\childdocmain{}
%    \end{macrocode}

% Optional override for |\version| flag:
%    \begin{macrocode}
%%\ifchilddoc\else\providecommand{\version}{draft}\fi
%    \end{macrocode}

% Define the default values for the |\version| flag
% (|final| for the main file and |draft| for childs):
%    \begin{macrocode}
\ifchilddoc
\providecommand{\version}{draft}
\else
\providecommand{\version}{final}
\fi
%    \end{macrocode}

% Load the standard document class:
%    \begin{macrocode}
\documentclass[12pt]{article}
%    \end{macrocode}

% Start the document body:
%    \begin{macrocode}
\begin{document}
%    \end{macrocode}

% Declare a title page.
% Print title, part of document being processed and version flag:
%    \begin{macrocode}
\addtocounter{page}{-1}
\begin{center}
{\LARGE\bfseries{}childdoc example\par}
\vspace{1cm}
\ifchilddoc
\ifchilddocmanual part\else chapter\fi:
`\childdocname' of `\childdocjob'\par
\else
main document: `\childdocjob'\par
\fi
version: \version\par
\end{center}
\newpage
%    \end{macrocode}

% Manually include selected file,
% otherwise process as usual:
%    \begin{macrocode}
\ifchilddocmanual
\section*{part `\childdocname'}
\input{\childdocname}
\else
%    \end{macrocode}

% Include the two chapters:
%    \begin{macrocode}
\include{cdocsch1}
\include{cdocsch2}
%    \end{macrocode}

% Include the two parts unless only chapters should be displayed:
%    \begin{macrocode}
\ifchilddoc\else
\section{part three}
\input{cdocspt3}
\section{part four}
\input{cdocspt4}
\fi
%    \end{macrocode}

% Process as usual until here:
%    \begin{macrocode}
\fi
%    \end{macrocode}

% End of document body:
%    \begin{macrocode}
\end{document}
%    \end{macrocode}
%\iffalse
%</samplemain>
%\fi
%
% %%%%%%%%%%%%%%%%%%%%%%%%%%%%%%%%%%%%%%
% \paragraph{Chapter Include Files.}
%
% The include files are called |cdocsch1.tex| and |cdocsch2.tex|.
%
%\iffalse
%<*samplechap1|samplechap2>
%\fi

% Optional override for |\version| flag:
%    \begin{macrocode}
%%\providecommand{\version}{final}
%    \end{macrocode}

% Include the main document:
%    \begin{macrocode}
\input{childdoc.def}
\childdocof{cdocsamp}
%    \end{macrocode}

%\iffalse
%</samplechap1|samplechap2>
%\fi
%
%\iffalse
%<*samplechap1>
%\fi
% Some text for chapter 1:
%    \begin{macrocode}
\section{one}
some text in chapter one
%    \end{macrocode}

%\iffalse
%</samplechap1>
%\fi
% Some text for chapter 2:
%\iffalse
%<*samplechap2>
%\fi
%    \begin{macrocode}
\section{two}
more text in chapter two
%    \end{macrocode}

%\iffalse
%</samplechap2>
%\fi
%
% %%%%%%%%%%%%%%%%%%%%%%%%%%%%%%%%%%%%%%
% \paragraph{Part Include Files.}
%
% The include files are called |cdocspt3.tex| and |cdocspt4.tex|.
%
%\iffalse
%<*samplepart3|samplepart4>
%\fi

% Optional override for |\version| flag:
%    \begin{macrocode}
%%\providecommand{\version}{final}
%    \end{macrocode}

% Include the main document:
%    \begin{macrocode}
\input{childdoc.def}
\childdocby{cdocsamp}
%    \end{macrocode}

%\iffalse
%</samplepart3|samplepart4>
%\fi
%
%\iffalse
%<*samplepart3>
%\fi
% Some text for part 3:
%    \begin{macrocode}
some text in part three
%    \end{macrocode}

%\iffalse
%</samplepart3>
%\fi
% Some text for part 4:
%\iffalse
%<*samplepart4>
%\fi
%    \begin{macrocode}
more text in part four
%    \end{macrocode}

%\iffalse
%</samplepart4>
%\fi
%
% %%%%%%%%%%%%%%%%%%%%%%%%%%%%%%%%%%%%%%
% \paragraph{Forwarding for a Complete Draft.}
%
% The following forwarding file |cdocsdrf.tex|
% compiles the main document in draft mode:
%\iffalse
%<*sampledraft>
%\fi
%    \begin{macrocode}
\def\version{draft}
\input{childdoc.def}
\childdocforward{cdocsamp}
%    \end{macrocode}

%\iffalse
%</sampledraft>
%\fi
%
% %%%%%%%%%%%%%%%%%%%%%%%%%%%%%%%%%%%%%%
% \paragraph{Forwarding for Final Version of the Chapters.}
%
% The following forwarding files |cdocsfn1.tex| and |cdocsfn2.tex|
% (with identical content)
% compile the final versions of the child documents
% |cdocsch1.tex| and |cdocsch2.tex|, respectively:
%\iffalse
%<*samplefinal>
%\fi
%    \begin{macrocode}
\def\version{final}
\input{childdoc.def}
\childdocforwardprefix[cdocsamp]{cdocsfn}{cdocsch}
%    \end{macrocode}

%\iffalse
%</samplefinal>
%\fi
%
% %%%%%%%%%%%%%%%%%%%%%%%%%%%%%%%%%%%%%%
% \paragraph{Command Line Processing.}
%
% The following three command lines generate the output files
% |cdocscld|, |cdocscl1| and |cdocscl2|
% which should be identical to
% |cdocsdrf|, |cdocsch1| and |cdocsfn2|, respectively:
% \begin{center}
% \begin{tabular}{l}
% |latex -jobname cdocscld \|\\
% |  "\def\version{draft}\input{childdoc.def}\childdocforward{cdocsamp}"|\\
% |latex -jobname cdocscl1 \|\\
% |  "\input{childdoc.def}\childdocforward[cdocsamp]{cdocsch1}"|\\
% |latex -jobname cdocscl2 \|\\
% |  "\def\version{final}\input{childdoc.def}\childdocforward{cdocsch2}"|
% \end{tabular}
% \end{center}
% Note that the trailing backslash on each first line
% merely continues the input to the second line
% (for convenient cut ant paste).
% Furthermore, the command |latex| can be replaced by any
% of its alternative versions such as |pdflatex|.
%
% %%%%%%%%%%%%%%%%%%%%%%%%%%%%%%%%%%%%%%%%%%%%%%%%%%%%%%%%%%%%%%%%%%%%%%%%%%%%%%
% %%%%%%%%%%%%%%%%%%%%%%%%%%%%%%%%%%%%%%%%%%%%%%%%%%%%%%%%%%%%%%%%%%%%%%%%%%%%%%
% \section{Implementation}
%\iffalse
%<*package>
%\fi
%
% This section describes the definitions file |childdoc.def|.

% The definitions cannot be loaded using |\usepackage| or |\RequirePackage|
% which has a mechanism to prevent loading a style file more than once.
% When loading the definitions by means of |\input|
% multiple instances have to be prevented manually:
%\iffalse
%This code needs to be before the `\ProvidesFile' directive
%which is defined at the beginning of this file.
%Therefore it is also placed there and commented out here.
%</package>
%<*discard>
%\fi
%    \begin{macrocode}
\ifdefined\childdocmain\endinput\fi
%    \end{macrocode}
%\iffalse
%</discard>
%<*package>
%\fi
%
% \macro{\ifchilddoc}
% \macro{\ifchilddocmanual}
% The conditional |\ifchilddoc| tells whether a
% child (true) or main (false) document is being compiled.
% The conditional |\ifchilddocmanual| tells whether
% the |\includeonly| mechanism is used (false) or
% the selection of child files must be performed manually (true).
% The definitions initialise to false:
%    \begin{macrocode}
\newif\ifchilddoc
\newif\ifchilddocmanual
%    \end{macrocode}

% \macro{\childdocname}
% \macro{\childdocjob}
% The macro |\childdocname| stores the name of the main document
% to be compiled. The macro |\childdocjob| stores the name of
% the document on which the \LaTeX{} compiler was originally invoked.
% The content of |\jobname| cannot be compared
% to filenames specified in the source due to different catcodes.
% The following code rescans |\jobname|, stores the result
% in |\childdocname| and saves a copy in |\childdocjob|:
%    \begin{macrocode}
\edef\childdocname{\scantokens\expandafter{\jobname\noexpand}}
\let\childdocjob\childdocname
%    \end{macrocode}

% \macro{\childdocdisable}
% The macro |\childdocdisable| prevents the main file
% from being processed more than once.
% At this stage, the main document command |\childdocmain|
% is assumed to be called once again where it should do nothing.
% Any subsequent call to it should prevent
% a secondary processing of the main document
% It overwrites the forwarding commands
% |\childdocof| and |\childdocforward|
% with empty macros to prevent further inclusions of the main document:
%    \begin{macrocode}
\newcommand{\childdocdisable}
{
  \renewcommand{\childdocmain}[1]{\renewcommand{\childdocmain}[1]{\endinput}}
  \renewcommand{\childdocof}[1]{}
  \renewcommand{\childdocby}[2][]{}
  \renewcommand{\childdocforward}[2][]{}
  \renewcommand{\childdocdisable}{}
}
%    \end{macrocode}

% \macro{\childdocmain}
% The macro |\childdocmain| is to be called at the top of the main file
% with nothing or the main filename (without extension) as argument.
% First, it breaks loops.
% If the argument is not empty and does not match |\childdocname|
% (which is set by the first inclusion of |childdoc.def|),
% |\ifchilddoc| is set to true, |\includeonly| is applied to the child file
% and |\jobname| is set to the main file
% (for proper handling of |.aux| files):
%    \begin{macrocode}
\newcommand{\childdocmain}[1]
{
  \childdocdisable\childdocmain{}
  \if?#1?\else
    \begingroup
      \def\childdoctmp{#1}
      \ifx\childdoctmp\childdocname
        \def\childdoctmp{}
      \else
        \def\childdoctmp
        {
          \childdoctrue
          \includeonly{\childdocname}
          \def\childdocjob{#1}
          \def\jobname{#1}
        }
      \fi
      \expandafter
    \endgroup
    \childdoctmp
  \fi
}
%    \end{macrocode}

% \macro{\childdocof}
% The command |\childdocof| redirects
% compilation to the main file |#1|.
%    \begin{macrocode}
\newcommand{\childdocof}[1]
{
  \childdocdisable
  \childdoctrue
  \includeonly{\childdocname}
  \def\jobname{#1}
  \def\childdocjob{#1}
  \input{#1}
}
%    \end{macrocode}

% \macro{\childdocby}
% The command |\childdocby| ....
%    \begin{macrocode}
\newcommand{\childdocby}[2][]
{
  \childdocdisable
  \childdoctrue
  \childdocmanualtrue
  \if?#1?\else
    \def\jobname{#2}
  \fi
  \def\childdocjob{#2}
  \input{#2}
  \endinput
}
%    \end{macrocode}

% \macro{\childdocforward}
% The command |\childdocforward| redirects
% compilation to the main file or
% (if the optional argument is given) a child file.
% Parameters are set as if the main file
% or a child file starting with |\childdocof| was compiled.
% Then compilation is handed over to the main file:
%    \begin{macrocode}
\newcommand{\childdocforward}[2][]
{
  \begingroup
    \if?#1?
      \def\childdoctmp
      {
        \def\childdocname{#2}
        \def\childdocjob{#2}
        \def\jobname{#2}
        \input{#2}
        \endinput
      }
    \else
      \def\childdoctmp
      {
        \childdocdisable
        \def\childdocname{#2}
        \childdoctrue
        \includeonly{#2}
        \def\childdocjob{#1}
        \def\jobname{#1}
        \input{#1}
        \endinput
      }
    \fi
    \expandafter
  \endgroup
  \childdoctmp
}
%    \end{macrocode}

% \macro{\childdocforwardprefix}
% The command |\childdocforwardprefix| redirects
% compilation to the main or a child file by means of a pattern.
% The prefix |#1| in the current filename is replaced by |#2|
% and the suffix of the current filename is kept
% (it is assumed that the filename does not contain the substring `|~~~|'
% which is used as a delimiter).
% Compilation is handed over to the new file by |\childdocforward|:
%    \begin{macrocode}
\newcommand{\childdocforwardprefix}[3][]
{
  \begingroup
    \def\childdocextract #2##1~~~{\def\childdoctmp{\childdocforward[#1]{#3##1}}}
    \expandafter\childdocextract\childdocname~~~
    \expandafter
  \endgroup
  \childdoctmp
}
%    \end{macrocode}

% \macro{\childdoc}
% The deprecated macro |\childdoc| is a legacy version of |\childdocmain|:
%    \begin{macrocode}
\newcommand{\childdoc}{\childdocmain}
%    \end{macrocode}

% \macro{\childdocredirect}
% The deprecated macro |\childdocredirect| is a legacy version
% of |\childdocforward| and |\childdocforwardprefix|:
%    \begin{macrocode}
\newcommand{\childdocredirect}[2][]
{
  \begingroup
    \if?#1?
      \def\childdoctmp{\childdocforward{#2}}
    \else
      \def\childdoctmp{\childdocforwardprefix{#1}{#2}}
    \fi
    \expandafter
  \endgroup
  \childdoctmp
}
%    \end{macrocode}

%\iffalse
%</package>
%\fi
%
\endinput
\childdocforward{cdocsamp}"|\\
% |latex -jobname cdocscl1 \|\\
% |  "% \iffalse
%
% childdoc.dtx Copyright (C) 2017-2018 Niklas Beisert
%
% This work may be distributed and/or modified under the
% conditions of the LaTeX Project Public License, either version 1.3
% of this license or (at your option) any later version.
% The latest version of this license is in
%   http://www.latex-project.org/lppl.txt
% and version 1.3 or later is part of all distributions of LaTeX
% version 2005/12/01 or later.
%
% This work has the LPPL maintenance status `maintained'.
%
% The Current Maintainer of this work is Niklas Beisert.
%
% This work consists of the files childdoc.dtx and childdoc.ins
% and the derived files childdoc.def and cdocsamp.tex with
% cdocsch1.tex, cdocsch2.tex, cdocsdrf.tex, cdocsfn1.tex, cdocsfn2.tex.
%
%<package>\ifdefined\childdocmain\endinput\fi
%<package>\ProvidesFile{childdoc.def}[2018/12/30 v2.0 child document driver]
%<samplemain>\ProvidesFile{cdocsamp.tex}[2018/12/30 v2.0 sample for childdoc]
%<*driver>
%\ProvidesFile{childdoc.drv}[2018/12/30 v2.0 childdoc reference manual file]
\PassOptionsToClass{10pt,a4paper}{article}
\documentclass{ltxdoc}

\usepackage[margin=35mm]{geometry}
\usepackage{hyperref}
\usepackage{hyperxmp}
\usepackage[usenames]{color}

\hypersetup{colorlinks=true}
\hypersetup{pdfstartview=FitH}
\hypersetup{pdfpagemode=UseNone}
\hypersetup{pdfsource={}}
\hypersetup{pdflang={en-UK}}
\hypersetup{pdfcopyright={Copyright 2017-2018 Niklas Beisert.
  This work may be distributed and/or modified under the
  conditions of the LaTeX Project Public License, either version 1.3
  of this license or (at your option) any later version.}}
\hypersetup{pdflicenseurl={http://www.latex-project.org/lppl.txt}}
\hypersetup{pdfcontactaddress={ETH Zurich, ITP, HIT K,
  Wolfgang-Pauli-Strasse 27}}
\hypersetup{pdfcontactpostcode={8093}}
\hypersetup{pdfcontactcity={Zurich}}
\hypersetup{pdfcontactcountry={Switzerland}}
\hypersetup{pdfcontactemail={nbeisert@itp.phys.ethz.ch}}
\hypersetup{pdfcontacturl={http://people.phys.ethz.ch/\xmptilde nbeisert/}}

\newcommand{\secref}[1]{\hyperref[#1]{section \ref*{#1}}}

\parskip1ex
\parindent0pt
\let\olditemize\itemize
\def\itemize{\olditemize\parskip0pt}

\begin{document}

\title{The \textsf{childdoc} Package}
\hypersetup{pdftitle={The childdoc Package}}
\author{Niklas Beisert\\[2ex]
  Institut f\"ur Theoretische Physik\\
  Eidgen\"ossische Technische Hochschule Z\"urich\\
  Wolfgang-Pauli-Strasse 27, 8093 Z\"urich, Switzerland\\[1ex]
  \href{mailto:nbeisert@itp.phys.ethz.ch}
  {\texttt{nbeisert@itp.phys.ethz.ch}}}
\hypersetup{pdfauthor={Niklas Beisert}}
\hypersetup{pdfsubject={Manual for the LaTeX2e Package childdoc}}
\date{30 December 2018, \textsf{v2.0}}
\maketitle

\begin{abstract}\noindent
\textsf{childdoc} is a \LaTeXe{} package
that enables the direct compilation
of document sections included by |\include|
to individual files.
\end{abstract}

\begingroup
\parskip0ex
\tableofcontents
\endgroup

%%%%%%%%%%%%%%%%%%%%%%%%%%%%%%%%%%%%%%%%%%%%%%%%%%%%%%%%%%%%%%%%%%%%%%%%%%%%%%%%
%%%%%%%%%%%%%%%%%%%%%%%%%%%%%%%%%%%%%%%%%%%%%%%%%%%%%%%%%%%%%%%%%%%%%%%%%%%%%%%%
\section{Introduction}

\LaTeX{} provides a mechanism to structure a large document (such as a book)
into a main file and several child files (containing the chapters)
using the |\include| command.
This mechanism is beneficial for documents
which span hundreds of pages in order to
make the source file(s) more manageable.
Moreover, compilation can be restricted to
selected child files by means of the |\includeonly| command.
The latter feature can be used to reduce the compilation time while editing
(this was significantly more useful in the earlier days of \LaTeX{})
or to generate a smaller document which is easier to navigate.
Another application of |\includeonly| is to generate
documents consisting of selected parts of the complete document.

However, there are a few drawbacks of the plain |\include| mechanism:
\begin{itemize}
\item
The child files cannot be compiled on their own,
they can only be compiled via the main file.
A naive editing environment
(such as a text editor with an option
to have the current file processed by \LaTeX)
may require one to switch to the main file before compiling;
attempting to compile the child file produces errors.
\item
The main file must be modified (each time)
to adjust the |\includeonly| command
to the present needs. This easily leaves the main file in a messy state.
\item
The generated document will always carry the filename
of the main document. This is inconvenient if
several child files are to be compiled and
to be kept for distribution.
\end{itemize}

The present package provides a simple interface
to make child files individually compilable by \LaTeX{}.
Compiling a child file then has the same effect as compiling
the main file with an |\includeonly| command
to select the appropriate child.
Moreover the generated document will carry the name of the child
rather than the main file.
This resolves all three above issues.

This feature is meant to make the editing of books,
thesis documents and lecture notes somewhat more convenient.
However, the package can also be used efficiently for
composing a series of documents (such as exercise sheets)
which are typically distributed individually.
It then assists the author in generating the individual documents
(potentially in different versions)
as well as a document containing the collected series.
Another application is in developing style files
or other kinds of included material
where compilation of the style file could redirect
to a sample or test file.

%%%%%%%%%%%%%%%%%%%%%%%%%%%%%%%%%%%%%%%%%%%%%%%%%%%%%%%%%%%%%%%%%%%%%%%%%%%%%%%%
%%%%%%%%%%%%%%%%%%%%%%%%%%%%%%%%%%%%%%%%%%%%%%%%%%%%%%%%%%%%%%%%%%%%%%%%%%%%%%%%
\section{Usage}

First of all, the package \textsf{childdoc} is \emph{not} a standard
\LaTeXe{} |.sty| style file! Therefore it needs to be invoked in
a non-standard way.

%%%%%%%%%%%%%%%%%%%%%%%%%%%%%%%%%%%%%%%%%%%%%%%%%%%%%%%%%%%%%%%%%%%%%%%%%%%%%%%%
\subsection{Included Files}
\label{sec:include}

%%%%%%%%%%%%%%%%%%%%%%%%%%%%%%%%%%%%%%%%
\DescribeMacro{\childdocmain}
To use the package, add the commands
\begin{center}
\begin{tabular}{l}
|\input{childdoc.def}|\\
|\childdocmain{}|\\
\end{tabular}
\end{center}
at the very top of the main \LaTeX{} file,
in particular \emph{before} the |\documentclass| statement!
The argument of |\childdocmain| should be left empty
(but it must be present).

%%%%%%%%%%%%%%%%%%%%%%%%%%%%%%%%%%%%%%%%
\DescribeMacro{\childdocof}
Furthermore, add the commands
\begin{center}
\begin{tabular}{l}
|\input{childdoc.def}|\\
|\childdocof{|\textit{main}|}|\\
\end{tabular}
\end{center}
at the top of every child file \textit{child}
which is included by |\include{|\textit{child}|}|
from within the main file
(or at least for those files to be compiled individually).
The argument \textit{main} must be the filename of the main file.

There are a couple of
considerations in setting up the main and child documents:

%%%%%%%%%%%%%%%%%%%%%%%%%%%%%%%%%%%%%%%%
\paragraph{Restrictions.}

Please note the following restrictions:
\begin{itemize}
\item
|\childdocmain| must be called with one argument \textit{main}
to ensure compatibility with earlier version of the package.
It must either be empty (|\childdocmain{}|)
or precisely match the filename of the main file in which it is specified.
See \secref{sec:detection} for further information.
\item
The filename \textit{main} must be specified without the |.tex| extension.
\item
The filename \textit{main} is case sensitive
(even in case-insensitive file systems)
due to internal string comparison.
\item
The argument \textit{main} should be fully expanded, it cannot be a macro.
\item
Subdirectories and special characters should be avoided in filenames.
\item
The command |\childdocmain{|\textit{main}|}| must be followed by a whitespace.
It should not be followed immediately by another command
or by a comment mark `|%|'.
This is because the \TeX{} parser reads the token immediately following
the argument of |\childdocmain| and puts it
at the beginning of every child section;
however, a white\-space is ignored.
\end{itemize}

%%%%%%%%%%%%%%%%%%%%%%%%%%%%%%%%%%%%%%%%
\paragraph{Content of Main File.}

It is advisable to place all content in the child files included by |\include|.
Any output contained in the main file will appear in all child documents
unless suppressed manually;
it cannot be suppressed automatically by the |\includeonly| directive
and thus should normally be avoided.
A method to include some content in the main file
by means of conditional processing is described in \secref{sec:conditional}.

%%%%%%%%%%%%%%%%%%%%%%%%%%%%%%%%%%%%%%%%
\paragraph{Page Numbering.}

When only a part of the document is compiled,
the appropriate numbering of pages
(as well as other status parameters)
is determined from the |.aux| files.
The latter contain information from previous passes.
However this information needs to propagate through
all intermediate child documents.
Therefore the page numbering in child documents may well
be inconsistent until the complete document is compiled at least once.

A useful (if unconventional) way to always ensure a consistent
page numbering is to restart the numbering in each child document
and denote the pages by `\textit{child}|.|\textit{page}'
where \textit{child} represents the chapter/section number of the child file.
This can be achieved by the command
|\numberwithin{page}{|\textit{child}|}|
of the \textsf{amsmath} package
where \textit{child} can be |chapter| or |section|
depending on the chosen structuring.
Alternatively, one can modify the macro |\thepage| appropriately
and reset the counter |page| at the start of each child file.

%%%%%%%%%%%%%%%%%%%%%%%%%%%%%%%%%%%%%%%%%%%%%%%%%%%%%%%%%%%%%%%%%%%%%%%%%%%%%%%%
\subsection{Conditional Processing}
\label{sec:conditional}

The package provides a mechanism to compile different versions
of a document. To customise the versions further some conditional processing
can come in handy to distinguish which version is being compiled.
The package provides two macros to describe the compilation context:

%%%%%%%%%%%%%%%%%%%%%%%%%%%%%%%%%%%%%%%%
\DescribeMacro{\ifchilddoc}
The conditional |\ifchilddoc| distinguishes between the compilation of
child documents and the main document:
%
\begin{center}
|\ifchilddoc |\textit{child-code}| |[|\||else |\textit{main-code}]| \||fi|
\end{center}

%%%%%%%%%%%%%%%%%%%%%%%%%%%%%%%%%%%%%%%%
\DescribeMacro{\childdocname}
\DescribeMacro{\childdocjob}
The macro |\childdocname| contains the filename (without extension)
of the main or child file being processed.
Note that |\childdocjob| will always contain the name of the main file.

%%%%%%%%%%%%%%%%%%%%%%%%%%%%%%%%%%%%%%%%
\paragraph{Title Page.}

Conditional processing can be used to include a title or banner page
in the main document when proper precautions are taken.
Importantly, the code in the main file should ensure that the page counter
(as well as other status parameters which are stored in the |.aux| files)
takes the same value after the conditional processing.
Otherwise the page numbers may take divergent values
depending on which part is compiled.

For example, a title page could be declared by:
%
\begin{center}
\begin{tabular}{l}
|\ifchilddoc\||else|\\
|\addtocounter{page}{-1}|\\
\textit{code for title page}\\
|\newpage|\\
|\||fi|
\end{tabular}
\end{center}
%
A banner page for the child documents can be generated by:
%
\begin{center}
\begin{tabular}{l}
|\ifchilddoc|\\
|\addtocounter{page}{-1}|\\
\textit{code for banner page}\\
|\newpage|\\
|\||fi|
\end{tabular}
\end{center}
%
Here one could write a message such as:
\begin{center}
|This is the part \childdocname{} of \childdocjob{}.|
\end{center}

%%%%%%%%%%%%%%%%%%%%%%%%%%%%%%%%%%%%%%%%%%%%%%%%%%%%%%%%%%%%%%%%%%%%%%%%%%%%%%%%
\subsection{Flags}
\label{sec:flags}

The package makes it easy to generate different versions
of the main or child documents.
To this end compilation flags can be defined
and assigned different default values.
They will be particularly useful in conjunction
with the forwarding mechanism described in \secref{sec:forward}.

For example, it may be useful to have a flag |\version|
which can be set to |draft| or |final|.
The document source will contain some conditional code
depending on the value of |\version|.
Suppose further, the flag should default to |final| for the main file
and to |draft| for child files
which is a natural assignment for editing the document.
This is achieved by placing the following code
in the preamble of the main document
(below the |\childdocmain| directive):
%
\begin{center}
\begin{tabular}{l}
|\ifchilddoc|\\
|\providecommand{\version}{draft}|\\
|\||else|\\
|\providecommand{\version}{final}|\\
|\||fi|
\end{tabular}
\end{center}
%
The definition by |\providecommand| makes sure
that previous definitions are not overwritten.
Further statements |\providecommand{\version}{...}|
can thus be added before the above code to override it.

For the main file, one might add a line
(between |\childdocmain| and the above block)
%
\begin{center}
|%\ifchilddoc\||else\providecommand{\version}{draft}\||fi|
\end{center}
%
which can be uncommented to produce a draft version.
Likewise one can add a line to the very top of a child file
(above the |\childdocof{|\textit{main}|}| directive)
%
\begin{center}
|%\providecommand{\version}{final}|
\end{center}
%
which can be uncommented to produce the final version of this child document.

%%%%%%%%%%%%%%%%%%%%%%%%%%%%%%%%%%%%%%%%%%%%%%%%%%%%%%%%%%%%%%%%%%%%%%%%%%%%%%%%
\subsection{Forwarding}
\label{sec:forward}

Different versions of the main or child documents
using compilation flags as described in \secref{sec:flags}
can be (permanently) stored in different files
for convenient compilation, viewing and distribution.
To this end, the package defines a command
to pass on compilation to a different file:

%%%%%%%%%%%%%%%%%%%%%%%%%%%%%%%%%%%%%%%%
\DescribeMacro{\childdocforward}
The command |\childdocforward| redirects processing to
another source file:
%
\begin{center}
\begin{tabular}{l}
|\input{childdoc.def}|\\
|\childdocforward[|\textit{main}|]{|\textit{dest}|}|\\
\end{tabular}
\end{center}
%
The argument \textit{dest} is the destination file
(without extension).
It should be the main file or one of the child files.
Note that further \textsf{childdoc} directives
such as |\childdocof| and |\childdocforward|
in the indicated file will be processed in this form.
The optional argument \textit{main}
passes on directly to the main file \textit{main}
while pretending to compile the child \textit{dest}.
This form behaves as if \textit{dest}
issues |\childdocof{|\textit{main}|}| right away,
and no further \textsf{childdoc} directives will be processed.

%%%%%%%%%%%%%%%%%%%%%%%%%%%%%%%%%%%%%%%%
\DescribeMacro{\...prefix}
In the alternative form |\childdocforwardprefix|,
%
\begin{center}
\begin{tabular}{l}
|\input{childdoc.def}|\\
|\childdocforwardprefix[|\textit{main}|]{|\textit{prefix}|}{|\textit{dest}|}|
\end{tabular}
\end{center}
%
the destination file is determined by a pattern
depending on the current file:
To make this work, the current file must be called
`{\textit{prefix}\hspace{0.2em}\textit{suffix}}'
with \textit{prefix} matching precisely the argument.
Processing is then passed on to the file
`{\textit{dest}\hspace{0.2em}\textit{suffix}}'.
Surely, the same effect is achieved by
directly specifying the
argument `{\textit{dest}\hspace{0.2em}\textit{suffix}}'
in the first form.
However, that requires to set up a different file
for each child. With the alternative form of the command
all these files can have exactly the same content
which simplifies setting them up and maintaining them.

For example, the following file |draft.tex|
with a compilation flag |\version| as described in \secref{sec:flags}
compiles the main document as a draft:
%
\begin{center}
\begin{tabular}{l}
|\def\version{draft}|\\
|\input{childdoc.def}|\\
|\childdocforward{|\textit{main}|}|
\end{tabular}
\end{center}
%
Likewise, the following files |final|\textit{nn}|.tex|
compile the final version of the child document
|child|\textit{nn}|.tex|:
%
\begin{center}
\begin{tabular}{l}
|\def\version{final}|\\
|\input{childdoc.def}|\\
|\childdocforwardprefix{final}{child}|
\end{tabular}
\end{center}
%

Note that when several versions of a main file and/or of each child file
are to be generated, it may be convenient to set up a |Makefile| or
shell script to automatise the process.

%%%%%%%%%%%%%%%%%%%%%%%%%%%%%%%%%%%%%%%%%%%%%%%%%%%%%%%%%%%%%%%%%%%%%%%%%%%%%%%%
\subsection{Command Line Processing}
\label{sec:commandline}

The effect of redirection files can also be achieved by invoking
the \LaTeX{} compiler with a more elaborate command line.
Most conveniently this should be done as part
of a shell script or a |Makefile|.

When using \textsf{childdoc} in the main file, the following
command lines effectively perform a redirection
(note that depending on the shell being used,
backslashes may have to be doubled: `|\|' $\to$ `|\\|'):
%
\begin{center}
|... -jobname "|\textit{target}|" |\\|"|[\textit{flags}]%
|\input{childdoc.def}\childdocforward[|\textit{main}|]{|\textit{dest}|}"|
\end{center}
%
Here \textit{target} is the name of the output file,
\textit{main} is the name of the main file
and \textit{dest} is the name of the main or child file to be processed
(all filenames without extensions).
The optional argument \textit{main} can be omitted
if \textit{main} matches \textit{dest}.
Optionally, compilation \textit{flags} can be defined via |\def| commands.
This command line makes the \TeX{} engine believe
it is compiling the file \textit{target}
whose content is specified as the latter parameter.
The provided code then forwards the processing to
\textit{main} or \textit{dest} as described in \secref{sec:forward}.

%%%%%%%%%%%%%%%%%%%%%%%%%%%%%%%%%%%%%%%%%%%%%%%%%%%%%%%%%%%%%%%%%%%%%%%%%%%%%%%%
\subsection{Include by Input}
\label{sec:input}

Including child documents by |\include| has some restrictions by design.
Most notably, the content of a child document always occupies
its own set of pages; pages cannot be shared between child documents.
Usually, this behaviour makes perfect sense
because each child document contain an essential part of the document.
However, in some situations it may be desirable to compose
a document from a collection of parts
without having mandatory page breaks between then.
For this case, the package
provides a mechanism to include parts
by |\input| which can also be processed individually.
However, by construction this mechanism
requires manual handling of the content to be output.

%%%%%%%%%%%%%%%%%%%%%%%%%%%%%%%%%%%%%%%%
\DescribeMacro{\ifchilddocmanual}
The main file should be prepared as usual, see \secref{sec:include}.
However, the document body must make a distinction
between processing of an individual part and of the main document, e.g.:
%
\begin{center}
\begin{tabular}{l}
|\ifchilddocmanual|\\
|\input{\childdocname}|\\
|\||else|\\
\textit{document body with }|\input{|\textit{part}|}|\\
|\||fi|
\end{tabular}
\end{center}
%
The conditional |\ifchilddocmanual| is true whenever
a part to be included by |\input| is being compiled,
and the name of the part is stored in |\childdocname|.

%%%%%%%%%%%%%%%%%%%%%%%%%%%%%%%%%%%%%%%%
\DescribeMacro{\childdocby}
Each part to be included by |\input| should start with:
%
\begin{center}
\begin{tabular}{l}
|\input{childdoc.def}|\\
|\childdocby{|\textit{main}|}|\\
\end{tabular}
\end{center}
%
The directive |\childdocby| is similar to |\childdocof|
described in \secref{sec:include},
but the subsequent selection of content must be done manually.
To that end, both |\ifchilddoc| and |\ifchilddocmanual|
will be true upon processing of a part,
and the name of the part is stored in |\childdocname|.
Note that |\jobname| will be set to the filename of the current part
so that each part receives an individual |.aux| file
that does not interfere with the |.aux| file(s) of the main document.
This behaviour can be altered by the alternative form
|\childdocby[*]{|\textit{main}|}| (with a non-empty optional argument)
which uses the |.aux| file of the main document
by setting |\jobname| to \textit{main}.

%%%%%%%%%%%%%%%%%%%%%%%%%%%%%%%%%%%%%%%%%%%%%%%%%%%%%%%%%%%%%%%%%%%%%%%%%%%%%%%%
\subsection{Driver Development}
\label{sec:driver}

The \textsf{childdoc} mechanism can also be use for the development
of definition files such as \LaTeX{} styles or classes.
This case differs from the above setup with multiple parts
included by |\include| in that no |\includeonly| should be invoked.
This can be achieved by starting the include file
(before |\ProvidesPackage|) with:
%
\begin{center}
\begin{tabular}{l}
|\input{childdoc.def}|\\
|\childdocforward{|\textit{main}|}|\\
\end{tabular}
\end{center}
%
or alternatively with:
%
\begin{center}
\begin{tabular}{l}
|\input{childdoc.def}|\\
|\childdocby{|\textit{main}|}|\\
\end{tabular}
\end{center}
%
Both forms have slightly different effects as described above.
The main file is prepared as usual, see \secref{sec:include}.

%%%%%%%%%%%%%%%%%%%%%%%%%%%%%%%%%%%%%%%%%%%%%%%%%%%%%%%%%%%%%%%%%%%%%%%%%%%%%%%%
\subsection{Legacy Detection}
\label{sec:detection}

The directive |\childdocmain| in the main file can detect
whether the complete document or merely a child is to be compiled
even without using the directive |\childdocof|.
This method is deprecated because it is less robust
and there is no compelling reason to use it;
it is merely provided for backward compatibility
and it may be removed in future versions.

If the detection mechanism is to be used,
it is mandatory to correctly specify
the filename of the main file as the argument of |\childdocmain|:
%
\begin{center}
\begin{tabular}{l}
|\input{childdoc.def}|\\
|\childdocmain{|\textit{main}|}|\\
\end{tabular}
\end{center}
%
If |\jobname| does not match the argument \textit{main} of |\childdocmain|,
it is assumed that |\jobname| points to the child file to be compiled.
When using |\childdocmain| with the main file specified as argument,
it suffices to start a child file
with just |\input{|\textit{main}|}|
without loading of the package and using |\childdocof|.
If instead all processing is done
with the appropriate \textsf{childdoc} directives,
the argument of \textit{main} of |\childdocmain| can be empty.

An alternative version of the command line processing described
in \secref{sec:commandline} using the detection mechanism reads:
%
\begin{center}
|... -jobname "|\textit{target}|" "|[\textit{flags}]%
[|\def\jobname{|\textit{dest}|}|]|\input{|\textit{main}|}"|
\end{center}

%%%%%%%%%%%%%%%%%%%%%%%%%%%%%%%%%%%%%%%%%%%%%%%%%%%%%%%%%%%%%%%%%%%%%%%%%%%%%%%%
\subsection{Manual Code}
\label{sec:manual}

In case one cannot be certain whether the definitions file |childdoc.def|
is installed on the target \TeX{} distribution
and one prefers not to ship it,
it is conceivable to paste a few relevant commands into the sources.

To that end, drop all statements |\input{childdoc.def}|
and perform the replacements as outlined below.
Instead of |\childdocmain{|\textit{main}|}| add the following code
to the top of the main file:
%
\begin{center}
\begin{tabular}{l}
|\||ifdefined\childdocname\endinput\||fi\newif\ifchilddoc|\\
|\edef\childdocname{\scantokens\expandafter{\jobname\noexpand}}|\\
|\def\childdocmain{|\textit{main}|}\||ifx\childdocmain\childdocname\||else|\\
|\childdoctrue\includeonly{\childdocname}\let\jobname\childdocmain\||fi|\\
\end{tabular}
\end{center}
%
Instead of |\childdocof{|\textit{main}|}| just include the main file
at the top of each child file:
%
\begin{center}
|\input{|\textit{main}|}|
\end{center}
%
A simple redirection |\childdocforward{|\textit{dest}|}| is achieved by:
%
\begin{center}
|\def\jobname{|\textit{dest}|}\input{\jobname}|
\end{center}
%
The redirection with prefix
|\childdocforwardprefix[|\textit{prefix}|]{|\textit{dest}|}|
is accomplished by:
%
\begin{center}
\begin{tabular}{l}
|{\edef\jobname{\scantokens\expandafter{\jobname\noexpand}}|\\
|\def\redirectjob |\textit{prefix}|#1~~~{\gdef\jobname{|\textit{dest}|#1}}|\\
|\expandafter\redirectjob\jobname~~~}\input{\jobname}|
\end{tabular}
\end{center}

In an alternative approach,
child documents can be compiled by a specific command line
without additional code or specific definitions:
%
\begin{center}
|... -jobname "|\textit{target}|" "|[\textit{flags}]%
|\includeonly{|\textit{dest}|}\input{|\textit{main}|}"|
\end{center}
%

%%%%%%%%%%%%%%%%%%%%%%%%%%%%%%%%%%%%%%%%%%%%%%%%%%%%%%%%%%%%%%%%%%%%%%%%%%%%%%%%
%%%%%%%%%%%%%%%%%%%%%%%%%%%%%%%%%%%%%%%%%%%%%%%%%%%%%%%%%%%%%%%%%%%%%%%%%%%%%%%%
\section{Information}

%%%%%%%%%%%%%%%%%%%%%%%%%%%%%%%%%%%%%%%%%%%%%%%%%%%%%%%%%%%%%%%%%%%%%%%%%%%%%%%%
\subsection{Copyright}

Copyright \copyright{} 2017--2018 Niklas Beisert

This work may be distributed and/or modified under the
conditions of the \LaTeX{} Project Public License, either version 1.3
of this license or (at your option) any later version.
The latest version of this license is in
  \url{http://www.latex-project.org/lppl.txt}
and version 1.3 or later is part of all distributions of \LaTeX{}
version 2005/12/01 or later.

This work has the LPPL maintenance status `maintained'.

The Current Maintainer of this work is Niklas Beisert.

This work consists of the files |README.txt|, |childdoc.ins| and |childdoc.dtx|
as well as the derived files |childdoc.def|, |cdocsamp.tex|
with |cdocsch1.tex|, |cdocsch2.tex|, |cdocspt3.tex|, |cdocspt4.tex|,
|cdocsdrf.tex|, |cdocsfn1.tex|, |cdocsfn2.tex|
as well as |childdoc.pdf|.

%%%%%%%%%%%%%%%%%%%%%%%%%%%%%%%%%%%%%%%%%%%%%%%%%%%%%%%%%%%%%%%%%%%%%%%%%%%%%%%%
\subsection{Files and Installation}

The package consists of the files:
%
\begin{center}
\begin{tabular}{ll}
    |README.txt|   & readme file \\
    |childdoc.ins| & installation file \\
    |childdoc.dtx| & source file \\
    |childdoc.def| & definition file \\
    |cdocsamp.tex| & sample main file \\
    |cdocsch1.tex| & sample include file \\
    |cdocsch2.tex| & sample include file \\
    |cdocspt3.tex| & sample part file \\
    |cdocspt4.tex| & sample part file \\
    |cdocsdrf.tex| & sample redirection file \\
    |cdocsfn1.tex| & sample redirection file \\
    |cdocsfn2.tex| & sample redirection file \\
    |childdoc.pdf| & manual
\end{tabular}
\end{center}
%
The distribution consists of the files
|README.txt|, |childdoc.ins| and |childdoc.dtx|.
%
\begin{itemize}
\item
Run (pdf)\LaTeX{} on |childdoc.dtx|
to compile the manual |childdoc.pdf| (this file).
\item
Run \LaTeX{} on |childdoc.ins| to create the definitions file |childdoc.def|
and the sample |cdocsamp.tex| with include files
|cdocsch1.tex|, |cdocsch2.tex|, |cdocspt3.tex|, |cdocspt4.tex|,
|cdocsdrf.tex|, |cdocsfn1.tex|, |cdocsfn2.tex|.
Then copy the file |childdoc.def| to an appropriate directory of your \LaTeX{}
distribution, e.g.\ \textit{texmf-root}|/tex/latex/childdoc|.
\end{itemize}

%%%%%%%%%%%%%%%%%%%%%%%%%%%%%%%%%%%%%%%%%%%%%%%%%%%%%%%%%%%%%%%%%%%%%%%%%%%%%%%%
\subsection{Related CTAN Packages}

There are several other packages which offer a similar functionality:
%
\begin{itemize}
\item
The packages
\href{http://ctan.org/pkg/docmute}{\textsf{docmute}},
\href{http://ctan.org/pkg/includex}{\textsf{includex}} and
\href{http://ctan.org/pkg/standalone}{\textsf{standalone}}
provide commands to include only the document body of
a child file thus allowing both files to be compiled individually.
\item
The packages \href{http://ctan.org/pkg/subdocs}{\textsf{subdocs}}
and \href{http://ctan.org/pkg/subfiles}{\textsf{subfiles}}
provide structures in which the main and child documents can be
encapsulated and allowing them to be compiled individually.
The inclusion mechanism is different from the conventional |\include|.
\item
The package \href{http://ctan.org/pkg/combine}{\textsf{combine}}
is an elaborate solution to combine several documents into one.
\end{itemize}
%
See also the CTAN topic \href{http://ctan.org/topic/subdocs}{\textsf{subdocs}}
for further related packages.
The present package differs from the above solutions in that
a document structure constructed with the conventional |\include| mechanism
just needs two extra commands at the top of every file
such that all constituent files can be compiled individually.

%%%%%%%%%%%%%%%%%%%%%%%%%%%%%%%%%%%%%%%%%%%%%%%%%%%%%%%%%%%%%%%%%%%%%%%%%%%%%%%%
%\subsection{Feature Suggestions}
%
%The following is a list of features which may be useful for future
%versions of this package:
%%
%\begin{itemize}
%\item
%\ldots
%\end{itemize}

%%%%%%%%%%%%%%%%%%%%%%%%%%%%%%%%%%%%%%%%%%%%%%%%%%%%%%%%%%%%%%%%%%%%%%%%%%%%%%%%
\subsection{Revision History}

%%%%%%%%%%%%%%%%%%%%%%%%%%%%%%%%%%%%%%%%
\paragraph{v2.0:} 2018/12/30

\begin{itemize}
\item
immediate forward processing
\item
added |\childdocby| mechanism
\item
manual restructured
\end{itemize}

%%%%%%%%%%%%%%%%%%%%%%%%%%%%%%%%%%%%%%%%
\paragraph{v1.6:} 2018/01/17

\begin{itemize}
\item
application for development of include files
\item
corrections to manual
\end{itemize}

%%%%%%%%%%%%%%%%%%%%%%%%%%%%%%%%%%%%%%%%
\paragraph{v1.5:} 2017/05/21

\begin{itemize}
\item
more complete structuring introduced
\item
|\childdocof| introduced
\item
|\childdoc| renamed to |\childdocmain|
\item
|\childredirect| renamed to |\childdocforward| and |\childdocforwardprefix|
and functionality expanded
\end{itemize}

%%%%%%%%%%%%%%%%%%%%%%%%%%%%%%%%%%%%%%%%
\paragraph{v1.0:} 2017/04/27

\begin{itemize}
\item
manual and install package
\item
first version published on CTAN
\end{itemize}

%%%%%%%%%%%%%%%%%%%%%%%%%%%%%%%%%%%%%%%%
\paragraph{v0.6:} 2017/04/26

\begin{itemize}
\item
redirection mechanism added
\end{itemize}

%%%%%%%%%%%%%%%%%%%%%%%%%%%%%%%%%%%%%%%%
\paragraph{v0.5:} 2017/04/26

\begin{itemize}
\item
functionality in definition file
\end{itemize}


%%%%%%%%%%%%%%%%%%%%%%%%%%%%%%%%%%%%%%%%%%%%%%%%%%%%%%%%%%%%%%%%%%%%%%%%%%%%%%%%
%%%%%%%%%%%%%%%%%%%%%%%%%%%%%%%%%%%%%%%%%%%%%%%%%%%%%%%%%%%%%%%%%%%%%%%%%%%%%%%%
%%%%%%%%%%%%%%%%%%%%%%%%%%%%%%%%%%%%%%%%%%%%%%%%%%%%%%%%%%%%%%%%%%%%%%%%%%%%%%%%
\appendix

\settowidth\MacroIndent{\rmfamily\scriptsize 000\ }

 \DocInput{childdoc.dtx}

\end{document}
%</driver>
% \fi
%
% %%%%%%%%%%%%%%%%%%%%%%%%%%%%%%%%%%%%%%%%%%%%%%%%%%%%%%%%%%%%%%%%%%%%%%%%%%%%%%
% %%%%%%%%%%%%%%%%%%%%%%%%%%%%%%%%%%%%%%%%%%%%%%%%%%%%%%%%%%%%%%%%%%%%%%%%%%%%%%
% \section{Sample}
%\iffalse
%<*samplemain>
%\fi
%
% The following presents a sample document
% with two chapters, two parts, a title page,
% a compile flag as well as three forwarding files to set the flag.
% It consists of eight |.tex| files:
% \begin{center}
% \begin{tabular}{ll}
% |cdocsamp.tex|&main file\\
% |cdocsch1.tex|&include file for chapter 1\\
% |cdocsch2.tex|&include file for chapter 2\\
% |cdocspt3.tex|&include file for part 3\\
% |cdocspt4.tex|&include file for part 4\\
% |cdocsdrf.tex|&forwarding file for main file in draft mode\\
% |cdocsfi1.tex|&forwarding file for final version of chapter 1\\
% |cdocsfi2.tex|&forwarding file for final version of chapter 2\\
% \end{tabular}
% \end{center}
% Each of the eight files can be compiled directly by the \LaTeX{} compiler.
%
% %%%%%%%%%%%%%%%%%%%%%%%%%%%%%%%%%%%%%%
% \paragraph{Main File.}
%
% The main file is called |cdocsamp.tex|.
%
% Load the \textsf{childdoc} definitions and
% declare the filename for the main document:
%    \begin{macrocode}
\input{childdoc.def}
\childdocmain{}
%    \end{macrocode}

% Optional override for |\version| flag:
%    \begin{macrocode}
%%\ifchilddoc\else\providecommand{\version}{draft}\fi
%    \end{macrocode}

% Define the default values for the |\version| flag
% (|final| for the main file and |draft| for childs):
%    \begin{macrocode}
\ifchilddoc
\providecommand{\version}{draft}
\else
\providecommand{\version}{final}
\fi
%    \end{macrocode}

% Load the standard document class:
%    \begin{macrocode}
\documentclass[12pt]{article}
%    \end{macrocode}

% Start the document body:
%    \begin{macrocode}
\begin{document}
%    \end{macrocode}

% Declare a title page.
% Print title, part of document being processed and version flag:
%    \begin{macrocode}
\addtocounter{page}{-1}
\begin{center}
{\LARGE\bfseries{}childdoc example\par}
\vspace{1cm}
\ifchilddoc
\ifchilddocmanual part\else chapter\fi:
`\childdocname' of `\childdocjob'\par
\else
main document: `\childdocjob'\par
\fi
version: \version\par
\end{center}
\newpage
%    \end{macrocode}

% Manually include selected file,
% otherwise process as usual:
%    \begin{macrocode}
\ifchilddocmanual
\section*{part `\childdocname'}
\input{\childdocname}
\else
%    \end{macrocode}

% Include the two chapters:
%    \begin{macrocode}
\include{cdocsch1}
\include{cdocsch2}
%    \end{macrocode}

% Include the two parts unless only chapters should be displayed:
%    \begin{macrocode}
\ifchilddoc\else
\section{part three}
\input{cdocspt3}
\section{part four}
\input{cdocspt4}
\fi
%    \end{macrocode}

% Process as usual until here:
%    \begin{macrocode}
\fi
%    \end{macrocode}

% End of document body:
%    \begin{macrocode}
\end{document}
%    \end{macrocode}
%\iffalse
%</samplemain>
%\fi
%
% %%%%%%%%%%%%%%%%%%%%%%%%%%%%%%%%%%%%%%
% \paragraph{Chapter Include Files.}
%
% The include files are called |cdocsch1.tex| and |cdocsch2.tex|.
%
%\iffalse
%<*samplechap1|samplechap2>
%\fi

% Optional override for |\version| flag:
%    \begin{macrocode}
%%\providecommand{\version}{final}
%    \end{macrocode}

% Include the main document:
%    \begin{macrocode}
\input{childdoc.def}
\childdocof{cdocsamp}
%    \end{macrocode}

%\iffalse
%</samplechap1|samplechap2>
%\fi
%
%\iffalse
%<*samplechap1>
%\fi
% Some text for chapter 1:
%    \begin{macrocode}
\section{one}
some text in chapter one
%    \end{macrocode}

%\iffalse
%</samplechap1>
%\fi
% Some text for chapter 2:
%\iffalse
%<*samplechap2>
%\fi
%    \begin{macrocode}
\section{two}
more text in chapter two
%    \end{macrocode}

%\iffalse
%</samplechap2>
%\fi
%
% %%%%%%%%%%%%%%%%%%%%%%%%%%%%%%%%%%%%%%
% \paragraph{Part Include Files.}
%
% The include files are called |cdocspt3.tex| and |cdocspt4.tex|.
%
%\iffalse
%<*samplepart3|samplepart4>
%\fi

% Optional override for |\version| flag:
%    \begin{macrocode}
%%\providecommand{\version}{final}
%    \end{macrocode}

% Include the main document:
%    \begin{macrocode}
\input{childdoc.def}
\childdocby{cdocsamp}
%    \end{macrocode}

%\iffalse
%</samplepart3|samplepart4>
%\fi
%
%\iffalse
%<*samplepart3>
%\fi
% Some text for part 3:
%    \begin{macrocode}
some text in part three
%    \end{macrocode}

%\iffalse
%</samplepart3>
%\fi
% Some text for part 4:
%\iffalse
%<*samplepart4>
%\fi
%    \begin{macrocode}
more text in part four
%    \end{macrocode}

%\iffalse
%</samplepart4>
%\fi
%
% %%%%%%%%%%%%%%%%%%%%%%%%%%%%%%%%%%%%%%
% \paragraph{Forwarding for a Complete Draft.}
%
% The following forwarding file |cdocsdrf.tex|
% compiles the main document in draft mode:
%\iffalse
%<*sampledraft>
%\fi
%    \begin{macrocode}
\def\version{draft}
\input{childdoc.def}
\childdocforward{cdocsamp}
%    \end{macrocode}

%\iffalse
%</sampledraft>
%\fi
%
% %%%%%%%%%%%%%%%%%%%%%%%%%%%%%%%%%%%%%%
% \paragraph{Forwarding for Final Version of the Chapters.}
%
% The following forwarding files |cdocsfn1.tex| and |cdocsfn2.tex|
% (with identical content)
% compile the final versions of the child documents
% |cdocsch1.tex| and |cdocsch2.tex|, respectively:
%\iffalse
%<*samplefinal>
%\fi
%    \begin{macrocode}
\def\version{final}
\input{childdoc.def}
\childdocforwardprefix[cdocsamp]{cdocsfn}{cdocsch}
%    \end{macrocode}

%\iffalse
%</samplefinal>
%\fi
%
% %%%%%%%%%%%%%%%%%%%%%%%%%%%%%%%%%%%%%%
% \paragraph{Command Line Processing.}
%
% The following three command lines generate the output files
% |cdocscld|, |cdocscl1| and |cdocscl2|
% which should be identical to
% |cdocsdrf|, |cdocsch1| and |cdocsfn2|, respectively:
% \begin{center}
% \begin{tabular}{l}
% |latex -jobname cdocscld \|\\
% |  "\def\version{draft}\input{childdoc.def}\childdocforward{cdocsamp}"|\\
% |latex -jobname cdocscl1 \|\\
% |  "\input{childdoc.def}\childdocforward[cdocsamp]{cdocsch1}"|\\
% |latex -jobname cdocscl2 \|\\
% |  "\def\version{final}\input{childdoc.def}\childdocforward{cdocsch2}"|
% \end{tabular}
% \end{center}
% Note that the trailing backslash on each first line
% merely continues the input to the second line
% (for convenient cut ant paste).
% Furthermore, the command |latex| can be replaced by any
% of its alternative versions such as |pdflatex|.
%
% %%%%%%%%%%%%%%%%%%%%%%%%%%%%%%%%%%%%%%%%%%%%%%%%%%%%%%%%%%%%%%%%%%%%%%%%%%%%%%
% %%%%%%%%%%%%%%%%%%%%%%%%%%%%%%%%%%%%%%%%%%%%%%%%%%%%%%%%%%%%%%%%%%%%%%%%%%%%%%
% \section{Implementation}
%\iffalse
%<*package>
%\fi
%
% This section describes the definitions file |childdoc.def|.

% The definitions cannot be loaded using |\usepackage| or |\RequirePackage|
% which has a mechanism to prevent loading a style file more than once.
% When loading the definitions by means of |\input|
% multiple instances have to be prevented manually:
%\iffalse
%This code needs to be before the `\ProvidesFile' directive
%which is defined at the beginning of this file.
%Therefore it is also placed there and commented out here.
%</package>
%<*discard>
%\fi
%    \begin{macrocode}
\ifdefined\childdocmain\endinput\fi
%    \end{macrocode}
%\iffalse
%</discard>
%<*package>
%\fi
%
% \macro{\ifchilddoc}
% \macro{\ifchilddocmanual}
% The conditional |\ifchilddoc| tells whether a
% child (true) or main (false) document is being compiled.
% The conditional |\ifchilddocmanual| tells whether
% the |\includeonly| mechanism is used (false) or
% the selection of child files must be performed manually (true).
% The definitions initialise to false:
%    \begin{macrocode}
\newif\ifchilddoc
\newif\ifchilddocmanual
%    \end{macrocode}

% \macro{\childdocname}
% \macro{\childdocjob}
% The macro |\childdocname| stores the name of the main document
% to be compiled. The macro |\childdocjob| stores the name of
% the document on which the \LaTeX{} compiler was originally invoked.
% The content of |\jobname| cannot be compared
% to filenames specified in the source due to different catcodes.
% The following code rescans |\jobname|, stores the result
% in |\childdocname| and saves a copy in |\childdocjob|:
%    \begin{macrocode}
\edef\childdocname{\scantokens\expandafter{\jobname\noexpand}}
\let\childdocjob\childdocname
%    \end{macrocode}

% \macro{\childdocdisable}
% The macro |\childdocdisable| prevents the main file
% from being processed more than once.
% At this stage, the main document command |\childdocmain|
% is assumed to be called once again where it should do nothing.
% Any subsequent call to it should prevent
% a secondary processing of the main document
% It overwrites the forwarding commands
% |\childdocof| and |\childdocforward|
% with empty macros to prevent further inclusions of the main document:
%    \begin{macrocode}
\newcommand{\childdocdisable}
{
  \renewcommand{\childdocmain}[1]{\renewcommand{\childdocmain}[1]{\endinput}}
  \renewcommand{\childdocof}[1]{}
  \renewcommand{\childdocby}[2][]{}
  \renewcommand{\childdocforward}[2][]{}
  \renewcommand{\childdocdisable}{}
}
%    \end{macrocode}

% \macro{\childdocmain}
% The macro |\childdocmain| is to be called at the top of the main file
% with nothing or the main filename (without extension) as argument.
% First, it breaks loops.
% If the argument is not empty and does not match |\childdocname|
% (which is set by the first inclusion of |childdoc.def|),
% |\ifchilddoc| is set to true, |\includeonly| is applied to the child file
% and |\jobname| is set to the main file
% (for proper handling of |.aux| files):
%    \begin{macrocode}
\newcommand{\childdocmain}[1]
{
  \childdocdisable\childdocmain{}
  \if?#1?\else
    \begingroup
      \def\childdoctmp{#1}
      \ifx\childdoctmp\childdocname
        \def\childdoctmp{}
      \else
        \def\childdoctmp
        {
          \childdoctrue
          \includeonly{\childdocname}
          \def\childdocjob{#1}
          \def\jobname{#1}
        }
      \fi
      \expandafter
    \endgroup
    \childdoctmp
  \fi
}
%    \end{macrocode}

% \macro{\childdocof}
% The command |\childdocof| redirects
% compilation to the main file |#1|.
%    \begin{macrocode}
\newcommand{\childdocof}[1]
{
  \childdocdisable
  \childdoctrue
  \includeonly{\childdocname}
  \def\jobname{#1}
  \def\childdocjob{#1}
  \input{#1}
}
%    \end{macrocode}

% \macro{\childdocby}
% The command |\childdocby| ....
%    \begin{macrocode}
\newcommand{\childdocby}[2][]
{
  \childdocdisable
  \childdoctrue
  \childdocmanualtrue
  \if?#1?\else
    \def\jobname{#2}
  \fi
  \def\childdocjob{#2}
  \input{#2}
  \endinput
}
%    \end{macrocode}

% \macro{\childdocforward}
% The command |\childdocforward| redirects
% compilation to the main file or
% (if the optional argument is given) a child file.
% Parameters are set as if the main file
% or a child file starting with |\childdocof| was compiled.
% Then compilation is handed over to the main file:
%    \begin{macrocode}
\newcommand{\childdocforward}[2][]
{
  \begingroup
    \if?#1?
      \def\childdoctmp
      {
        \def\childdocname{#2}
        \def\childdocjob{#2}
        \def\jobname{#2}
        \input{#2}
        \endinput
      }
    \else
      \def\childdoctmp
      {
        \childdocdisable
        \def\childdocname{#2}
        \childdoctrue
        \includeonly{#2}
        \def\childdocjob{#1}
        \def\jobname{#1}
        \input{#1}
        \endinput
      }
    \fi
    \expandafter
  \endgroup
  \childdoctmp
}
%    \end{macrocode}

% \macro{\childdocforwardprefix}
% The command |\childdocforwardprefix| redirects
% compilation to the main or a child file by means of a pattern.
% The prefix |#1| in the current filename is replaced by |#2|
% and the suffix of the current filename is kept
% (it is assumed that the filename does not contain the substring `|~~~|'
% which is used as a delimiter).
% Compilation is handed over to the new file by |\childdocforward|:
%    \begin{macrocode}
\newcommand{\childdocforwardprefix}[3][]
{
  \begingroup
    \def\childdocextract #2##1~~~{\def\childdoctmp{\childdocforward[#1]{#3##1}}}
    \expandafter\childdocextract\childdocname~~~
    \expandafter
  \endgroup
  \childdoctmp
}
%    \end{macrocode}

% \macro{\childdoc}
% The deprecated macro |\childdoc| is a legacy version of |\childdocmain|:
%    \begin{macrocode}
\newcommand{\childdoc}{\childdocmain}
%    \end{macrocode}

% \macro{\childdocredirect}
% The deprecated macro |\childdocredirect| is a legacy version
% of |\childdocforward| and |\childdocforwardprefix|:
%    \begin{macrocode}
\newcommand{\childdocredirect}[2][]
{
  \begingroup
    \if?#1?
      \def\childdoctmp{\childdocforward{#2}}
    \else
      \def\childdoctmp{\childdocforwardprefix{#1}{#2}}
    \fi
    \expandafter
  \endgroup
  \childdoctmp
}
%    \end{macrocode}

%\iffalse
%</package>
%\fi
%
\endinput
\childdocforward[cdocsamp]{cdocsch1}"|\\
% |latex -jobname cdocscl2 \|\\
% |  "\def\version{final}% \iffalse
%
% childdoc.dtx Copyright (C) 2017-2018 Niklas Beisert
%
% This work may be distributed and/or modified under the
% conditions of the LaTeX Project Public License, either version 1.3
% of this license or (at your option) any later version.
% The latest version of this license is in
%   http://www.latex-project.org/lppl.txt
% and version 1.3 or later is part of all distributions of LaTeX
% version 2005/12/01 or later.
%
% This work has the LPPL maintenance status `maintained'.
%
% The Current Maintainer of this work is Niklas Beisert.
%
% This work consists of the files childdoc.dtx and childdoc.ins
% and the derived files childdoc.def and cdocsamp.tex with
% cdocsch1.tex, cdocsch2.tex, cdocsdrf.tex, cdocsfn1.tex, cdocsfn2.tex.
%
%<package>\ifdefined\childdocmain\endinput\fi
%<package>\ProvidesFile{childdoc.def}[2018/12/30 v2.0 child document driver]
%<samplemain>\ProvidesFile{cdocsamp.tex}[2018/12/30 v2.0 sample for childdoc]
%<*driver>
%\ProvidesFile{childdoc.drv}[2018/12/30 v2.0 childdoc reference manual file]
\PassOptionsToClass{10pt,a4paper}{article}
\documentclass{ltxdoc}

\usepackage[margin=35mm]{geometry}
\usepackage{hyperref}
\usepackage{hyperxmp}
\usepackage[usenames]{color}

\hypersetup{colorlinks=true}
\hypersetup{pdfstartview=FitH}
\hypersetup{pdfpagemode=UseNone}
\hypersetup{pdfsource={}}
\hypersetup{pdflang={en-UK}}
\hypersetup{pdfcopyright={Copyright 2017-2018 Niklas Beisert.
  This work may be distributed and/or modified under the
  conditions of the LaTeX Project Public License, either version 1.3
  of this license or (at your option) any later version.}}
\hypersetup{pdflicenseurl={http://www.latex-project.org/lppl.txt}}
\hypersetup{pdfcontactaddress={ETH Zurich, ITP, HIT K,
  Wolfgang-Pauli-Strasse 27}}
\hypersetup{pdfcontactpostcode={8093}}
\hypersetup{pdfcontactcity={Zurich}}
\hypersetup{pdfcontactcountry={Switzerland}}
\hypersetup{pdfcontactemail={nbeisert@itp.phys.ethz.ch}}
\hypersetup{pdfcontacturl={http://people.phys.ethz.ch/\xmptilde nbeisert/}}

\newcommand{\secref}[1]{\hyperref[#1]{section \ref*{#1}}}

\parskip1ex
\parindent0pt
\let\olditemize\itemize
\def\itemize{\olditemize\parskip0pt}

\begin{document}

\title{The \textsf{childdoc} Package}
\hypersetup{pdftitle={The childdoc Package}}
\author{Niklas Beisert\\[2ex]
  Institut f\"ur Theoretische Physik\\
  Eidgen\"ossische Technische Hochschule Z\"urich\\
  Wolfgang-Pauli-Strasse 27, 8093 Z\"urich, Switzerland\\[1ex]
  \href{mailto:nbeisert@itp.phys.ethz.ch}
  {\texttt{nbeisert@itp.phys.ethz.ch}}}
\hypersetup{pdfauthor={Niklas Beisert}}
\hypersetup{pdfsubject={Manual for the LaTeX2e Package childdoc}}
\date{30 December 2018, \textsf{v2.0}}
\maketitle

\begin{abstract}\noindent
\textsf{childdoc} is a \LaTeXe{} package
that enables the direct compilation
of document sections included by |\include|
to individual files.
\end{abstract}

\begingroup
\parskip0ex
\tableofcontents
\endgroup

%%%%%%%%%%%%%%%%%%%%%%%%%%%%%%%%%%%%%%%%%%%%%%%%%%%%%%%%%%%%%%%%%%%%%%%%%%%%%%%%
%%%%%%%%%%%%%%%%%%%%%%%%%%%%%%%%%%%%%%%%%%%%%%%%%%%%%%%%%%%%%%%%%%%%%%%%%%%%%%%%
\section{Introduction}

\LaTeX{} provides a mechanism to structure a large document (such as a book)
into a main file and several child files (containing the chapters)
using the |\include| command.
This mechanism is beneficial for documents
which span hundreds of pages in order to
make the source file(s) more manageable.
Moreover, compilation can be restricted to
selected child files by means of the |\includeonly| command.
The latter feature can be used to reduce the compilation time while editing
(this was significantly more useful in the earlier days of \LaTeX{})
or to generate a smaller document which is easier to navigate.
Another application of |\includeonly| is to generate
documents consisting of selected parts of the complete document.

However, there are a few drawbacks of the plain |\include| mechanism:
\begin{itemize}
\item
The child files cannot be compiled on their own,
they can only be compiled via the main file.
A naive editing environment
(such as a text editor with an option
to have the current file processed by \LaTeX)
may require one to switch to the main file before compiling;
attempting to compile the child file produces errors.
\item
The main file must be modified (each time)
to adjust the |\includeonly| command
to the present needs. This easily leaves the main file in a messy state.
\item
The generated document will always carry the filename
of the main document. This is inconvenient if
several child files are to be compiled and
to be kept for distribution.
\end{itemize}

The present package provides a simple interface
to make child files individually compilable by \LaTeX{}.
Compiling a child file then has the same effect as compiling
the main file with an |\includeonly| command
to select the appropriate child.
Moreover the generated document will carry the name of the child
rather than the main file.
This resolves all three above issues.

This feature is meant to make the editing of books,
thesis documents and lecture notes somewhat more convenient.
However, the package can also be used efficiently for
composing a series of documents (such as exercise sheets)
which are typically distributed individually.
It then assists the author in generating the individual documents
(potentially in different versions)
as well as a document containing the collected series.
Another application is in developing style files
or other kinds of included material
where compilation of the style file could redirect
to a sample or test file.

%%%%%%%%%%%%%%%%%%%%%%%%%%%%%%%%%%%%%%%%%%%%%%%%%%%%%%%%%%%%%%%%%%%%%%%%%%%%%%%%
%%%%%%%%%%%%%%%%%%%%%%%%%%%%%%%%%%%%%%%%%%%%%%%%%%%%%%%%%%%%%%%%%%%%%%%%%%%%%%%%
\section{Usage}

First of all, the package \textsf{childdoc} is \emph{not} a standard
\LaTeXe{} |.sty| style file! Therefore it needs to be invoked in
a non-standard way.

%%%%%%%%%%%%%%%%%%%%%%%%%%%%%%%%%%%%%%%%%%%%%%%%%%%%%%%%%%%%%%%%%%%%%%%%%%%%%%%%
\subsection{Included Files}
\label{sec:include}

%%%%%%%%%%%%%%%%%%%%%%%%%%%%%%%%%%%%%%%%
\DescribeMacro{\childdocmain}
To use the package, add the commands
\begin{center}
\begin{tabular}{l}
|\input{childdoc.def}|\\
|\childdocmain{}|\\
\end{tabular}
\end{center}
at the very top of the main \LaTeX{} file,
in particular \emph{before} the |\documentclass| statement!
The argument of |\childdocmain| should be left empty
(but it must be present).

%%%%%%%%%%%%%%%%%%%%%%%%%%%%%%%%%%%%%%%%
\DescribeMacro{\childdocof}
Furthermore, add the commands
\begin{center}
\begin{tabular}{l}
|\input{childdoc.def}|\\
|\childdocof{|\textit{main}|}|\\
\end{tabular}
\end{center}
at the top of every child file \textit{child}
which is included by |\include{|\textit{child}|}|
from within the main file
(or at least for those files to be compiled individually).
The argument \textit{main} must be the filename of the main file.

There are a couple of
considerations in setting up the main and child documents:

%%%%%%%%%%%%%%%%%%%%%%%%%%%%%%%%%%%%%%%%
\paragraph{Restrictions.}

Please note the following restrictions:
\begin{itemize}
\item
|\childdocmain| must be called with one argument \textit{main}
to ensure compatibility with earlier version of the package.
It must either be empty (|\childdocmain{}|)
or precisely match the filename of the main file in which it is specified.
See \secref{sec:detection} for further information.
\item
The filename \textit{main} must be specified without the |.tex| extension.
\item
The filename \textit{main} is case sensitive
(even in case-insensitive file systems)
due to internal string comparison.
\item
The argument \textit{main} should be fully expanded, it cannot be a macro.
\item
Subdirectories and special characters should be avoided in filenames.
\item
The command |\childdocmain{|\textit{main}|}| must be followed by a whitespace.
It should not be followed immediately by another command
or by a comment mark `|%|'.
This is because the \TeX{} parser reads the token immediately following
the argument of |\childdocmain| and puts it
at the beginning of every child section;
however, a white\-space is ignored.
\end{itemize}

%%%%%%%%%%%%%%%%%%%%%%%%%%%%%%%%%%%%%%%%
\paragraph{Content of Main File.}

It is advisable to place all content in the child files included by |\include|.
Any output contained in the main file will appear in all child documents
unless suppressed manually;
it cannot be suppressed automatically by the |\includeonly| directive
and thus should normally be avoided.
A method to include some content in the main file
by means of conditional processing is described in \secref{sec:conditional}.

%%%%%%%%%%%%%%%%%%%%%%%%%%%%%%%%%%%%%%%%
\paragraph{Page Numbering.}

When only a part of the document is compiled,
the appropriate numbering of pages
(as well as other status parameters)
is determined from the |.aux| files.
The latter contain information from previous passes.
However this information needs to propagate through
all intermediate child documents.
Therefore the page numbering in child documents may well
be inconsistent until the complete document is compiled at least once.

A useful (if unconventional) way to always ensure a consistent
page numbering is to restart the numbering in each child document
and denote the pages by `\textit{child}|.|\textit{page}'
where \textit{child} represents the chapter/section number of the child file.
This can be achieved by the command
|\numberwithin{page}{|\textit{child}|}|
of the \textsf{amsmath} package
where \textit{child} can be |chapter| or |section|
depending on the chosen structuring.
Alternatively, one can modify the macro |\thepage| appropriately
and reset the counter |page| at the start of each child file.

%%%%%%%%%%%%%%%%%%%%%%%%%%%%%%%%%%%%%%%%%%%%%%%%%%%%%%%%%%%%%%%%%%%%%%%%%%%%%%%%
\subsection{Conditional Processing}
\label{sec:conditional}

The package provides a mechanism to compile different versions
of a document. To customise the versions further some conditional processing
can come in handy to distinguish which version is being compiled.
The package provides two macros to describe the compilation context:

%%%%%%%%%%%%%%%%%%%%%%%%%%%%%%%%%%%%%%%%
\DescribeMacro{\ifchilddoc}
The conditional |\ifchilddoc| distinguishes between the compilation of
child documents and the main document:
%
\begin{center}
|\ifchilddoc |\textit{child-code}| |[|\||else |\textit{main-code}]| \||fi|
\end{center}

%%%%%%%%%%%%%%%%%%%%%%%%%%%%%%%%%%%%%%%%
\DescribeMacro{\childdocname}
\DescribeMacro{\childdocjob}
The macro |\childdocname| contains the filename (without extension)
of the main or child file being processed.
Note that |\childdocjob| will always contain the name of the main file.

%%%%%%%%%%%%%%%%%%%%%%%%%%%%%%%%%%%%%%%%
\paragraph{Title Page.}

Conditional processing can be used to include a title or banner page
in the main document when proper precautions are taken.
Importantly, the code in the main file should ensure that the page counter
(as well as other status parameters which are stored in the |.aux| files)
takes the same value after the conditional processing.
Otherwise the page numbers may take divergent values
depending on which part is compiled.

For example, a title page could be declared by:
%
\begin{center}
\begin{tabular}{l}
|\ifchilddoc\||else|\\
|\addtocounter{page}{-1}|\\
\textit{code for title page}\\
|\newpage|\\
|\||fi|
\end{tabular}
\end{center}
%
A banner page for the child documents can be generated by:
%
\begin{center}
\begin{tabular}{l}
|\ifchilddoc|\\
|\addtocounter{page}{-1}|\\
\textit{code for banner page}\\
|\newpage|\\
|\||fi|
\end{tabular}
\end{center}
%
Here one could write a message such as:
\begin{center}
|This is the part \childdocname{} of \childdocjob{}.|
\end{center}

%%%%%%%%%%%%%%%%%%%%%%%%%%%%%%%%%%%%%%%%%%%%%%%%%%%%%%%%%%%%%%%%%%%%%%%%%%%%%%%%
\subsection{Flags}
\label{sec:flags}

The package makes it easy to generate different versions
of the main or child documents.
To this end compilation flags can be defined
and assigned different default values.
They will be particularly useful in conjunction
with the forwarding mechanism described in \secref{sec:forward}.

For example, it may be useful to have a flag |\version|
which can be set to |draft| or |final|.
The document source will contain some conditional code
depending on the value of |\version|.
Suppose further, the flag should default to |final| for the main file
and to |draft| for child files
which is a natural assignment for editing the document.
This is achieved by placing the following code
in the preamble of the main document
(below the |\childdocmain| directive):
%
\begin{center}
\begin{tabular}{l}
|\ifchilddoc|\\
|\providecommand{\version}{draft}|\\
|\||else|\\
|\providecommand{\version}{final}|\\
|\||fi|
\end{tabular}
\end{center}
%
The definition by |\providecommand| makes sure
that previous definitions are not overwritten.
Further statements |\providecommand{\version}{...}|
can thus be added before the above code to override it.

For the main file, one might add a line
(between |\childdocmain| and the above block)
%
\begin{center}
|%\ifchilddoc\||else\providecommand{\version}{draft}\||fi|
\end{center}
%
which can be uncommented to produce a draft version.
Likewise one can add a line to the very top of a child file
(above the |\childdocof{|\textit{main}|}| directive)
%
\begin{center}
|%\providecommand{\version}{final}|
\end{center}
%
which can be uncommented to produce the final version of this child document.

%%%%%%%%%%%%%%%%%%%%%%%%%%%%%%%%%%%%%%%%%%%%%%%%%%%%%%%%%%%%%%%%%%%%%%%%%%%%%%%%
\subsection{Forwarding}
\label{sec:forward}

Different versions of the main or child documents
using compilation flags as described in \secref{sec:flags}
can be (permanently) stored in different files
for convenient compilation, viewing and distribution.
To this end, the package defines a command
to pass on compilation to a different file:

%%%%%%%%%%%%%%%%%%%%%%%%%%%%%%%%%%%%%%%%
\DescribeMacro{\childdocforward}
The command |\childdocforward| redirects processing to
another source file:
%
\begin{center}
\begin{tabular}{l}
|\input{childdoc.def}|\\
|\childdocforward[|\textit{main}|]{|\textit{dest}|}|\\
\end{tabular}
\end{center}
%
The argument \textit{dest} is the destination file
(without extension).
It should be the main file or one of the child files.
Note that further \textsf{childdoc} directives
such as |\childdocof| and |\childdocforward|
in the indicated file will be processed in this form.
The optional argument \textit{main}
passes on directly to the main file \textit{main}
while pretending to compile the child \textit{dest}.
This form behaves as if \textit{dest}
issues |\childdocof{|\textit{main}|}| right away,
and no further \textsf{childdoc} directives will be processed.

%%%%%%%%%%%%%%%%%%%%%%%%%%%%%%%%%%%%%%%%
\DescribeMacro{\...prefix}
In the alternative form |\childdocforwardprefix|,
%
\begin{center}
\begin{tabular}{l}
|\input{childdoc.def}|\\
|\childdocforwardprefix[|\textit{main}|]{|\textit{prefix}|}{|\textit{dest}|}|
\end{tabular}
\end{center}
%
the destination file is determined by a pattern
depending on the current file:
To make this work, the current file must be called
`{\textit{prefix}\hspace{0.2em}\textit{suffix}}'
with \textit{prefix} matching precisely the argument.
Processing is then passed on to the file
`{\textit{dest}\hspace{0.2em}\textit{suffix}}'.
Surely, the same effect is achieved by
directly specifying the
argument `{\textit{dest}\hspace{0.2em}\textit{suffix}}'
in the first form.
However, that requires to set up a different file
for each child. With the alternative form of the command
all these files can have exactly the same content
which simplifies setting them up and maintaining them.

For example, the following file |draft.tex|
with a compilation flag |\version| as described in \secref{sec:flags}
compiles the main document as a draft:
%
\begin{center}
\begin{tabular}{l}
|\def\version{draft}|\\
|\input{childdoc.def}|\\
|\childdocforward{|\textit{main}|}|
\end{tabular}
\end{center}
%
Likewise, the following files |final|\textit{nn}|.tex|
compile the final version of the child document
|child|\textit{nn}|.tex|:
%
\begin{center}
\begin{tabular}{l}
|\def\version{final}|\\
|\input{childdoc.def}|\\
|\childdocforwardprefix{final}{child}|
\end{tabular}
\end{center}
%

Note that when several versions of a main file and/or of each child file
are to be generated, it may be convenient to set up a |Makefile| or
shell script to automatise the process.

%%%%%%%%%%%%%%%%%%%%%%%%%%%%%%%%%%%%%%%%%%%%%%%%%%%%%%%%%%%%%%%%%%%%%%%%%%%%%%%%
\subsection{Command Line Processing}
\label{sec:commandline}

The effect of redirection files can also be achieved by invoking
the \LaTeX{} compiler with a more elaborate command line.
Most conveniently this should be done as part
of a shell script or a |Makefile|.

When using \textsf{childdoc} in the main file, the following
command lines effectively perform a redirection
(note that depending on the shell being used,
backslashes may have to be doubled: `|\|' $\to$ `|\\|'):
%
\begin{center}
|... -jobname "|\textit{target}|" |\\|"|[\textit{flags}]%
|\input{childdoc.def}\childdocforward[|\textit{main}|]{|\textit{dest}|}"|
\end{center}
%
Here \textit{target} is the name of the output file,
\textit{main} is the name of the main file
and \textit{dest} is the name of the main or child file to be processed
(all filenames without extensions).
The optional argument \textit{main} can be omitted
if \textit{main} matches \textit{dest}.
Optionally, compilation \textit{flags} can be defined via |\def| commands.
This command line makes the \TeX{} engine believe
it is compiling the file \textit{target}
whose content is specified as the latter parameter.
The provided code then forwards the processing to
\textit{main} or \textit{dest} as described in \secref{sec:forward}.

%%%%%%%%%%%%%%%%%%%%%%%%%%%%%%%%%%%%%%%%%%%%%%%%%%%%%%%%%%%%%%%%%%%%%%%%%%%%%%%%
\subsection{Include by Input}
\label{sec:input}

Including child documents by |\include| has some restrictions by design.
Most notably, the content of a child document always occupies
its own set of pages; pages cannot be shared between child documents.
Usually, this behaviour makes perfect sense
because each child document contain an essential part of the document.
However, in some situations it may be desirable to compose
a document from a collection of parts
without having mandatory page breaks between then.
For this case, the package
provides a mechanism to include parts
by |\input| which can also be processed individually.
However, by construction this mechanism
requires manual handling of the content to be output.

%%%%%%%%%%%%%%%%%%%%%%%%%%%%%%%%%%%%%%%%
\DescribeMacro{\ifchilddocmanual}
The main file should be prepared as usual, see \secref{sec:include}.
However, the document body must make a distinction
between processing of an individual part and of the main document, e.g.:
%
\begin{center}
\begin{tabular}{l}
|\ifchilddocmanual|\\
|\input{\childdocname}|\\
|\||else|\\
\textit{document body with }|\input{|\textit{part}|}|\\
|\||fi|
\end{tabular}
\end{center}
%
The conditional |\ifchilddocmanual| is true whenever
a part to be included by |\input| is being compiled,
and the name of the part is stored in |\childdocname|.

%%%%%%%%%%%%%%%%%%%%%%%%%%%%%%%%%%%%%%%%
\DescribeMacro{\childdocby}
Each part to be included by |\input| should start with:
%
\begin{center}
\begin{tabular}{l}
|\input{childdoc.def}|\\
|\childdocby{|\textit{main}|}|\\
\end{tabular}
\end{center}
%
The directive |\childdocby| is similar to |\childdocof|
described in \secref{sec:include},
but the subsequent selection of content must be done manually.
To that end, both |\ifchilddoc| and |\ifchilddocmanual|
will be true upon processing of a part,
and the name of the part is stored in |\childdocname|.
Note that |\jobname| will be set to the filename of the current part
so that each part receives an individual |.aux| file
that does not interfere with the |.aux| file(s) of the main document.
This behaviour can be altered by the alternative form
|\childdocby[*]{|\textit{main}|}| (with a non-empty optional argument)
which uses the |.aux| file of the main document
by setting |\jobname| to \textit{main}.

%%%%%%%%%%%%%%%%%%%%%%%%%%%%%%%%%%%%%%%%%%%%%%%%%%%%%%%%%%%%%%%%%%%%%%%%%%%%%%%%
\subsection{Driver Development}
\label{sec:driver}

The \textsf{childdoc} mechanism can also be use for the development
of definition files such as \LaTeX{} styles or classes.
This case differs from the above setup with multiple parts
included by |\include| in that no |\includeonly| should be invoked.
This can be achieved by starting the include file
(before |\ProvidesPackage|) with:
%
\begin{center}
\begin{tabular}{l}
|\input{childdoc.def}|\\
|\childdocforward{|\textit{main}|}|\\
\end{tabular}
\end{center}
%
or alternatively with:
%
\begin{center}
\begin{tabular}{l}
|\input{childdoc.def}|\\
|\childdocby{|\textit{main}|}|\\
\end{tabular}
\end{center}
%
Both forms have slightly different effects as described above.
The main file is prepared as usual, see \secref{sec:include}.

%%%%%%%%%%%%%%%%%%%%%%%%%%%%%%%%%%%%%%%%%%%%%%%%%%%%%%%%%%%%%%%%%%%%%%%%%%%%%%%%
\subsection{Legacy Detection}
\label{sec:detection}

The directive |\childdocmain| in the main file can detect
whether the complete document or merely a child is to be compiled
even without using the directive |\childdocof|.
This method is deprecated because it is less robust
and there is no compelling reason to use it;
it is merely provided for backward compatibility
and it may be removed in future versions.

If the detection mechanism is to be used,
it is mandatory to correctly specify
the filename of the main file as the argument of |\childdocmain|:
%
\begin{center}
\begin{tabular}{l}
|\input{childdoc.def}|\\
|\childdocmain{|\textit{main}|}|\\
\end{tabular}
\end{center}
%
If |\jobname| does not match the argument \textit{main} of |\childdocmain|,
it is assumed that |\jobname| points to the child file to be compiled.
When using |\childdocmain| with the main file specified as argument,
it suffices to start a child file
with just |\input{|\textit{main}|}|
without loading of the package and using |\childdocof|.
If instead all processing is done
with the appropriate \textsf{childdoc} directives,
the argument of \textit{main} of |\childdocmain| can be empty.

An alternative version of the command line processing described
in \secref{sec:commandline} using the detection mechanism reads:
%
\begin{center}
|... -jobname "|\textit{target}|" "|[\textit{flags}]%
[|\def\jobname{|\textit{dest}|}|]|\input{|\textit{main}|}"|
\end{center}

%%%%%%%%%%%%%%%%%%%%%%%%%%%%%%%%%%%%%%%%%%%%%%%%%%%%%%%%%%%%%%%%%%%%%%%%%%%%%%%%
\subsection{Manual Code}
\label{sec:manual}

In case one cannot be certain whether the definitions file |childdoc.def|
is installed on the target \TeX{} distribution
and one prefers not to ship it,
it is conceivable to paste a few relevant commands into the sources.

To that end, drop all statements |\input{childdoc.def}|
and perform the replacements as outlined below.
Instead of |\childdocmain{|\textit{main}|}| add the following code
to the top of the main file:
%
\begin{center}
\begin{tabular}{l}
|\||ifdefined\childdocname\endinput\||fi\newif\ifchilddoc|\\
|\edef\childdocname{\scantokens\expandafter{\jobname\noexpand}}|\\
|\def\childdocmain{|\textit{main}|}\||ifx\childdocmain\childdocname\||else|\\
|\childdoctrue\includeonly{\childdocname}\let\jobname\childdocmain\||fi|\\
\end{tabular}
\end{center}
%
Instead of |\childdocof{|\textit{main}|}| just include the main file
at the top of each child file:
%
\begin{center}
|\input{|\textit{main}|}|
\end{center}
%
A simple redirection |\childdocforward{|\textit{dest}|}| is achieved by:
%
\begin{center}
|\def\jobname{|\textit{dest}|}\input{\jobname}|
\end{center}
%
The redirection with prefix
|\childdocforwardprefix[|\textit{prefix}|]{|\textit{dest}|}|
is accomplished by:
%
\begin{center}
\begin{tabular}{l}
|{\edef\jobname{\scantokens\expandafter{\jobname\noexpand}}|\\
|\def\redirectjob |\textit{prefix}|#1~~~{\gdef\jobname{|\textit{dest}|#1}}|\\
|\expandafter\redirectjob\jobname~~~}\input{\jobname}|
\end{tabular}
\end{center}

In an alternative approach,
child documents can be compiled by a specific command line
without additional code or specific definitions:
%
\begin{center}
|... -jobname "|\textit{target}|" "|[\textit{flags}]%
|\includeonly{|\textit{dest}|}\input{|\textit{main}|}"|
\end{center}
%

%%%%%%%%%%%%%%%%%%%%%%%%%%%%%%%%%%%%%%%%%%%%%%%%%%%%%%%%%%%%%%%%%%%%%%%%%%%%%%%%
%%%%%%%%%%%%%%%%%%%%%%%%%%%%%%%%%%%%%%%%%%%%%%%%%%%%%%%%%%%%%%%%%%%%%%%%%%%%%%%%
\section{Information}

%%%%%%%%%%%%%%%%%%%%%%%%%%%%%%%%%%%%%%%%%%%%%%%%%%%%%%%%%%%%%%%%%%%%%%%%%%%%%%%%
\subsection{Copyright}

Copyright \copyright{} 2017--2018 Niklas Beisert

This work may be distributed and/or modified under the
conditions of the \LaTeX{} Project Public License, either version 1.3
of this license or (at your option) any later version.
The latest version of this license is in
  \url{http://www.latex-project.org/lppl.txt}
and version 1.3 or later is part of all distributions of \LaTeX{}
version 2005/12/01 or later.

This work has the LPPL maintenance status `maintained'.

The Current Maintainer of this work is Niklas Beisert.

This work consists of the files |README.txt|, |childdoc.ins| and |childdoc.dtx|
as well as the derived files |childdoc.def|, |cdocsamp.tex|
with |cdocsch1.tex|, |cdocsch2.tex|, |cdocspt3.tex|, |cdocspt4.tex|,
|cdocsdrf.tex|, |cdocsfn1.tex|, |cdocsfn2.tex|
as well as |childdoc.pdf|.

%%%%%%%%%%%%%%%%%%%%%%%%%%%%%%%%%%%%%%%%%%%%%%%%%%%%%%%%%%%%%%%%%%%%%%%%%%%%%%%%
\subsection{Files and Installation}

The package consists of the files:
%
\begin{center}
\begin{tabular}{ll}
    |README.txt|   & readme file \\
    |childdoc.ins| & installation file \\
    |childdoc.dtx| & source file \\
    |childdoc.def| & definition file \\
    |cdocsamp.tex| & sample main file \\
    |cdocsch1.tex| & sample include file \\
    |cdocsch2.tex| & sample include file \\
    |cdocspt3.tex| & sample part file \\
    |cdocspt4.tex| & sample part file \\
    |cdocsdrf.tex| & sample redirection file \\
    |cdocsfn1.tex| & sample redirection file \\
    |cdocsfn2.tex| & sample redirection file \\
    |childdoc.pdf| & manual
\end{tabular}
\end{center}
%
The distribution consists of the files
|README.txt|, |childdoc.ins| and |childdoc.dtx|.
%
\begin{itemize}
\item
Run (pdf)\LaTeX{} on |childdoc.dtx|
to compile the manual |childdoc.pdf| (this file).
\item
Run \LaTeX{} on |childdoc.ins| to create the definitions file |childdoc.def|
and the sample |cdocsamp.tex| with include files
|cdocsch1.tex|, |cdocsch2.tex|, |cdocspt3.tex|, |cdocspt4.tex|,
|cdocsdrf.tex|, |cdocsfn1.tex|, |cdocsfn2.tex|.
Then copy the file |childdoc.def| to an appropriate directory of your \LaTeX{}
distribution, e.g.\ \textit{texmf-root}|/tex/latex/childdoc|.
\end{itemize}

%%%%%%%%%%%%%%%%%%%%%%%%%%%%%%%%%%%%%%%%%%%%%%%%%%%%%%%%%%%%%%%%%%%%%%%%%%%%%%%%
\subsection{Related CTAN Packages}

There are several other packages which offer a similar functionality:
%
\begin{itemize}
\item
The packages
\href{http://ctan.org/pkg/docmute}{\textsf{docmute}},
\href{http://ctan.org/pkg/includex}{\textsf{includex}} and
\href{http://ctan.org/pkg/standalone}{\textsf{standalone}}
provide commands to include only the document body of
a child file thus allowing both files to be compiled individually.
\item
The packages \href{http://ctan.org/pkg/subdocs}{\textsf{subdocs}}
and \href{http://ctan.org/pkg/subfiles}{\textsf{subfiles}}
provide structures in which the main and child documents can be
encapsulated and allowing them to be compiled individually.
The inclusion mechanism is different from the conventional |\include|.
\item
The package \href{http://ctan.org/pkg/combine}{\textsf{combine}}
is an elaborate solution to combine several documents into one.
\end{itemize}
%
See also the CTAN topic \href{http://ctan.org/topic/subdocs}{\textsf{subdocs}}
for further related packages.
The present package differs from the above solutions in that
a document structure constructed with the conventional |\include| mechanism
just needs two extra commands at the top of every file
such that all constituent files can be compiled individually.

%%%%%%%%%%%%%%%%%%%%%%%%%%%%%%%%%%%%%%%%%%%%%%%%%%%%%%%%%%%%%%%%%%%%%%%%%%%%%%%%
%\subsection{Feature Suggestions}
%
%The following is a list of features which may be useful for future
%versions of this package:
%%
%\begin{itemize}
%\item
%\ldots
%\end{itemize}

%%%%%%%%%%%%%%%%%%%%%%%%%%%%%%%%%%%%%%%%%%%%%%%%%%%%%%%%%%%%%%%%%%%%%%%%%%%%%%%%
\subsection{Revision History}

%%%%%%%%%%%%%%%%%%%%%%%%%%%%%%%%%%%%%%%%
\paragraph{v2.0:} 2018/12/30

\begin{itemize}
\item
immediate forward processing
\item
added |\childdocby| mechanism
\item
manual restructured
\end{itemize}

%%%%%%%%%%%%%%%%%%%%%%%%%%%%%%%%%%%%%%%%
\paragraph{v1.6:} 2018/01/17

\begin{itemize}
\item
application for development of include files
\item
corrections to manual
\end{itemize}

%%%%%%%%%%%%%%%%%%%%%%%%%%%%%%%%%%%%%%%%
\paragraph{v1.5:} 2017/05/21

\begin{itemize}
\item
more complete structuring introduced
\item
|\childdocof| introduced
\item
|\childdoc| renamed to |\childdocmain|
\item
|\childredirect| renamed to |\childdocforward| and |\childdocforwardprefix|
and functionality expanded
\end{itemize}

%%%%%%%%%%%%%%%%%%%%%%%%%%%%%%%%%%%%%%%%
\paragraph{v1.0:} 2017/04/27

\begin{itemize}
\item
manual and install package
\item
first version published on CTAN
\end{itemize}

%%%%%%%%%%%%%%%%%%%%%%%%%%%%%%%%%%%%%%%%
\paragraph{v0.6:} 2017/04/26

\begin{itemize}
\item
redirection mechanism added
\end{itemize}

%%%%%%%%%%%%%%%%%%%%%%%%%%%%%%%%%%%%%%%%
\paragraph{v0.5:} 2017/04/26

\begin{itemize}
\item
functionality in definition file
\end{itemize}


%%%%%%%%%%%%%%%%%%%%%%%%%%%%%%%%%%%%%%%%%%%%%%%%%%%%%%%%%%%%%%%%%%%%%%%%%%%%%%%%
%%%%%%%%%%%%%%%%%%%%%%%%%%%%%%%%%%%%%%%%%%%%%%%%%%%%%%%%%%%%%%%%%%%%%%%%%%%%%%%%
%%%%%%%%%%%%%%%%%%%%%%%%%%%%%%%%%%%%%%%%%%%%%%%%%%%%%%%%%%%%%%%%%%%%%%%%%%%%%%%%
\appendix

\settowidth\MacroIndent{\rmfamily\scriptsize 000\ }

 \DocInput{childdoc.dtx}

\end{document}
%</driver>
% \fi
%
% %%%%%%%%%%%%%%%%%%%%%%%%%%%%%%%%%%%%%%%%%%%%%%%%%%%%%%%%%%%%%%%%%%%%%%%%%%%%%%
% %%%%%%%%%%%%%%%%%%%%%%%%%%%%%%%%%%%%%%%%%%%%%%%%%%%%%%%%%%%%%%%%%%%%%%%%%%%%%%
% \section{Sample}
%\iffalse
%<*samplemain>
%\fi
%
% The following presents a sample document
% with two chapters, two parts, a title page,
% a compile flag as well as three forwarding files to set the flag.
% It consists of eight |.tex| files:
% \begin{center}
% \begin{tabular}{ll}
% |cdocsamp.tex|&main file\\
% |cdocsch1.tex|&include file for chapter 1\\
% |cdocsch2.tex|&include file for chapter 2\\
% |cdocspt3.tex|&include file for part 3\\
% |cdocspt4.tex|&include file for part 4\\
% |cdocsdrf.tex|&forwarding file for main file in draft mode\\
% |cdocsfi1.tex|&forwarding file for final version of chapter 1\\
% |cdocsfi2.tex|&forwarding file for final version of chapter 2\\
% \end{tabular}
% \end{center}
% Each of the eight files can be compiled directly by the \LaTeX{} compiler.
%
% %%%%%%%%%%%%%%%%%%%%%%%%%%%%%%%%%%%%%%
% \paragraph{Main File.}
%
% The main file is called |cdocsamp.tex|.
%
% Load the \textsf{childdoc} definitions and
% declare the filename for the main document:
%    \begin{macrocode}
\input{childdoc.def}
\childdocmain{}
%    \end{macrocode}

% Optional override for |\version| flag:
%    \begin{macrocode}
%%\ifchilddoc\else\providecommand{\version}{draft}\fi
%    \end{macrocode}

% Define the default values for the |\version| flag
% (|final| for the main file and |draft| for childs):
%    \begin{macrocode}
\ifchilddoc
\providecommand{\version}{draft}
\else
\providecommand{\version}{final}
\fi
%    \end{macrocode}

% Load the standard document class:
%    \begin{macrocode}
\documentclass[12pt]{article}
%    \end{macrocode}

% Start the document body:
%    \begin{macrocode}
\begin{document}
%    \end{macrocode}

% Declare a title page.
% Print title, part of document being processed and version flag:
%    \begin{macrocode}
\addtocounter{page}{-1}
\begin{center}
{\LARGE\bfseries{}childdoc example\par}
\vspace{1cm}
\ifchilddoc
\ifchilddocmanual part\else chapter\fi:
`\childdocname' of `\childdocjob'\par
\else
main document: `\childdocjob'\par
\fi
version: \version\par
\end{center}
\newpage
%    \end{macrocode}

% Manually include selected file,
% otherwise process as usual:
%    \begin{macrocode}
\ifchilddocmanual
\section*{part `\childdocname'}
\input{\childdocname}
\else
%    \end{macrocode}

% Include the two chapters:
%    \begin{macrocode}
\include{cdocsch1}
\include{cdocsch2}
%    \end{macrocode}

% Include the two parts unless only chapters should be displayed:
%    \begin{macrocode}
\ifchilddoc\else
\section{part three}
\input{cdocspt3}
\section{part four}
\input{cdocspt4}
\fi
%    \end{macrocode}

% Process as usual until here:
%    \begin{macrocode}
\fi
%    \end{macrocode}

% End of document body:
%    \begin{macrocode}
\end{document}
%    \end{macrocode}
%\iffalse
%</samplemain>
%\fi
%
% %%%%%%%%%%%%%%%%%%%%%%%%%%%%%%%%%%%%%%
% \paragraph{Chapter Include Files.}
%
% The include files are called |cdocsch1.tex| and |cdocsch2.tex|.
%
%\iffalse
%<*samplechap1|samplechap2>
%\fi

% Optional override for |\version| flag:
%    \begin{macrocode}
%%\providecommand{\version}{final}
%    \end{macrocode}

% Include the main document:
%    \begin{macrocode}
\input{childdoc.def}
\childdocof{cdocsamp}
%    \end{macrocode}

%\iffalse
%</samplechap1|samplechap2>
%\fi
%
%\iffalse
%<*samplechap1>
%\fi
% Some text for chapter 1:
%    \begin{macrocode}
\section{one}
some text in chapter one
%    \end{macrocode}

%\iffalse
%</samplechap1>
%\fi
% Some text for chapter 2:
%\iffalse
%<*samplechap2>
%\fi
%    \begin{macrocode}
\section{two}
more text in chapter two
%    \end{macrocode}

%\iffalse
%</samplechap2>
%\fi
%
% %%%%%%%%%%%%%%%%%%%%%%%%%%%%%%%%%%%%%%
% \paragraph{Part Include Files.}
%
% The include files are called |cdocspt3.tex| and |cdocspt4.tex|.
%
%\iffalse
%<*samplepart3|samplepart4>
%\fi

% Optional override for |\version| flag:
%    \begin{macrocode}
%%\providecommand{\version}{final}
%    \end{macrocode}

% Include the main document:
%    \begin{macrocode}
\input{childdoc.def}
\childdocby{cdocsamp}
%    \end{macrocode}

%\iffalse
%</samplepart3|samplepart4>
%\fi
%
%\iffalse
%<*samplepart3>
%\fi
% Some text for part 3:
%    \begin{macrocode}
some text in part three
%    \end{macrocode}

%\iffalse
%</samplepart3>
%\fi
% Some text for part 4:
%\iffalse
%<*samplepart4>
%\fi
%    \begin{macrocode}
more text in part four
%    \end{macrocode}

%\iffalse
%</samplepart4>
%\fi
%
% %%%%%%%%%%%%%%%%%%%%%%%%%%%%%%%%%%%%%%
% \paragraph{Forwarding for a Complete Draft.}
%
% The following forwarding file |cdocsdrf.tex|
% compiles the main document in draft mode:
%\iffalse
%<*sampledraft>
%\fi
%    \begin{macrocode}
\def\version{draft}
\input{childdoc.def}
\childdocforward{cdocsamp}
%    \end{macrocode}

%\iffalse
%</sampledraft>
%\fi
%
% %%%%%%%%%%%%%%%%%%%%%%%%%%%%%%%%%%%%%%
% \paragraph{Forwarding for Final Version of the Chapters.}
%
% The following forwarding files |cdocsfn1.tex| and |cdocsfn2.tex|
% (with identical content)
% compile the final versions of the child documents
% |cdocsch1.tex| and |cdocsch2.tex|, respectively:
%\iffalse
%<*samplefinal>
%\fi
%    \begin{macrocode}
\def\version{final}
\input{childdoc.def}
\childdocforwardprefix[cdocsamp]{cdocsfn}{cdocsch}
%    \end{macrocode}

%\iffalse
%</samplefinal>
%\fi
%
% %%%%%%%%%%%%%%%%%%%%%%%%%%%%%%%%%%%%%%
% \paragraph{Command Line Processing.}
%
% The following three command lines generate the output files
% |cdocscld|, |cdocscl1| and |cdocscl2|
% which should be identical to
% |cdocsdrf|, |cdocsch1| and |cdocsfn2|, respectively:
% \begin{center}
% \begin{tabular}{l}
% |latex -jobname cdocscld \|\\
% |  "\def\version{draft}\input{childdoc.def}\childdocforward{cdocsamp}"|\\
% |latex -jobname cdocscl1 \|\\
% |  "\input{childdoc.def}\childdocforward[cdocsamp]{cdocsch1}"|\\
% |latex -jobname cdocscl2 \|\\
% |  "\def\version{final}\input{childdoc.def}\childdocforward{cdocsch2}"|
% \end{tabular}
% \end{center}
% Note that the trailing backslash on each first line
% merely continues the input to the second line
% (for convenient cut ant paste).
% Furthermore, the command |latex| can be replaced by any
% of its alternative versions such as |pdflatex|.
%
% %%%%%%%%%%%%%%%%%%%%%%%%%%%%%%%%%%%%%%%%%%%%%%%%%%%%%%%%%%%%%%%%%%%%%%%%%%%%%%
% %%%%%%%%%%%%%%%%%%%%%%%%%%%%%%%%%%%%%%%%%%%%%%%%%%%%%%%%%%%%%%%%%%%%%%%%%%%%%%
% \section{Implementation}
%\iffalse
%<*package>
%\fi
%
% This section describes the definitions file |childdoc.def|.

% The definitions cannot be loaded using |\usepackage| or |\RequirePackage|
% which has a mechanism to prevent loading a style file more than once.
% When loading the definitions by means of |\input|
% multiple instances have to be prevented manually:
%\iffalse
%This code needs to be before the `\ProvidesFile' directive
%which is defined at the beginning of this file.
%Therefore it is also placed there and commented out here.
%</package>
%<*discard>
%\fi
%    \begin{macrocode}
\ifdefined\childdocmain\endinput\fi
%    \end{macrocode}
%\iffalse
%</discard>
%<*package>
%\fi
%
% \macro{\ifchilddoc}
% \macro{\ifchilddocmanual}
% The conditional |\ifchilddoc| tells whether a
% child (true) or main (false) document is being compiled.
% The conditional |\ifchilddocmanual| tells whether
% the |\includeonly| mechanism is used (false) or
% the selection of child files must be performed manually (true).
% The definitions initialise to false:
%    \begin{macrocode}
\newif\ifchilddoc
\newif\ifchilddocmanual
%    \end{macrocode}

% \macro{\childdocname}
% \macro{\childdocjob}
% The macro |\childdocname| stores the name of the main document
% to be compiled. The macro |\childdocjob| stores the name of
% the document on which the \LaTeX{} compiler was originally invoked.
% The content of |\jobname| cannot be compared
% to filenames specified in the source due to different catcodes.
% The following code rescans |\jobname|, stores the result
% in |\childdocname| and saves a copy in |\childdocjob|:
%    \begin{macrocode}
\edef\childdocname{\scantokens\expandafter{\jobname\noexpand}}
\let\childdocjob\childdocname
%    \end{macrocode}

% \macro{\childdocdisable}
% The macro |\childdocdisable| prevents the main file
% from being processed more than once.
% At this stage, the main document command |\childdocmain|
% is assumed to be called once again where it should do nothing.
% Any subsequent call to it should prevent
% a secondary processing of the main document
% It overwrites the forwarding commands
% |\childdocof| and |\childdocforward|
% with empty macros to prevent further inclusions of the main document:
%    \begin{macrocode}
\newcommand{\childdocdisable}
{
  \renewcommand{\childdocmain}[1]{\renewcommand{\childdocmain}[1]{\endinput}}
  \renewcommand{\childdocof}[1]{}
  \renewcommand{\childdocby}[2][]{}
  \renewcommand{\childdocforward}[2][]{}
  \renewcommand{\childdocdisable}{}
}
%    \end{macrocode}

% \macro{\childdocmain}
% The macro |\childdocmain| is to be called at the top of the main file
% with nothing or the main filename (without extension) as argument.
% First, it breaks loops.
% If the argument is not empty and does not match |\childdocname|
% (which is set by the first inclusion of |childdoc.def|),
% |\ifchilddoc| is set to true, |\includeonly| is applied to the child file
% and |\jobname| is set to the main file
% (for proper handling of |.aux| files):
%    \begin{macrocode}
\newcommand{\childdocmain}[1]
{
  \childdocdisable\childdocmain{}
  \if?#1?\else
    \begingroup
      \def\childdoctmp{#1}
      \ifx\childdoctmp\childdocname
        \def\childdoctmp{}
      \else
        \def\childdoctmp
        {
          \childdoctrue
          \includeonly{\childdocname}
          \def\childdocjob{#1}
          \def\jobname{#1}
        }
      \fi
      \expandafter
    \endgroup
    \childdoctmp
  \fi
}
%    \end{macrocode}

% \macro{\childdocof}
% The command |\childdocof| redirects
% compilation to the main file |#1|.
%    \begin{macrocode}
\newcommand{\childdocof}[1]
{
  \childdocdisable
  \childdoctrue
  \includeonly{\childdocname}
  \def\jobname{#1}
  \def\childdocjob{#1}
  \input{#1}
}
%    \end{macrocode}

% \macro{\childdocby}
% The command |\childdocby| ....
%    \begin{macrocode}
\newcommand{\childdocby}[2][]
{
  \childdocdisable
  \childdoctrue
  \childdocmanualtrue
  \if?#1?\else
    \def\jobname{#2}
  \fi
  \def\childdocjob{#2}
  \input{#2}
  \endinput
}
%    \end{macrocode}

% \macro{\childdocforward}
% The command |\childdocforward| redirects
% compilation to the main file or
% (if the optional argument is given) a child file.
% Parameters are set as if the main file
% or a child file starting with |\childdocof| was compiled.
% Then compilation is handed over to the main file:
%    \begin{macrocode}
\newcommand{\childdocforward}[2][]
{
  \begingroup
    \if?#1?
      \def\childdoctmp
      {
        \def\childdocname{#2}
        \def\childdocjob{#2}
        \def\jobname{#2}
        \input{#2}
        \endinput
      }
    \else
      \def\childdoctmp
      {
        \childdocdisable
        \def\childdocname{#2}
        \childdoctrue
        \includeonly{#2}
        \def\childdocjob{#1}
        \def\jobname{#1}
        \input{#1}
        \endinput
      }
    \fi
    \expandafter
  \endgroup
  \childdoctmp
}
%    \end{macrocode}

% \macro{\childdocforwardprefix}
% The command |\childdocforwardprefix| redirects
% compilation to the main or a child file by means of a pattern.
% The prefix |#1| in the current filename is replaced by |#2|
% and the suffix of the current filename is kept
% (it is assumed that the filename does not contain the substring `|~~~|'
% which is used as a delimiter).
% Compilation is handed over to the new file by |\childdocforward|:
%    \begin{macrocode}
\newcommand{\childdocforwardprefix}[3][]
{
  \begingroup
    \def\childdocextract #2##1~~~{\def\childdoctmp{\childdocforward[#1]{#3##1}}}
    \expandafter\childdocextract\childdocname~~~
    \expandafter
  \endgroup
  \childdoctmp
}
%    \end{macrocode}

% \macro{\childdoc}
% The deprecated macro |\childdoc| is a legacy version of |\childdocmain|:
%    \begin{macrocode}
\newcommand{\childdoc}{\childdocmain}
%    \end{macrocode}

% \macro{\childdocredirect}
% The deprecated macro |\childdocredirect| is a legacy version
% of |\childdocforward| and |\childdocforwardprefix|:
%    \begin{macrocode}
\newcommand{\childdocredirect}[2][]
{
  \begingroup
    \if?#1?
      \def\childdoctmp{\childdocforward{#2}}
    \else
      \def\childdoctmp{\childdocforwardprefix{#1}{#2}}
    \fi
    \expandafter
  \endgroup
  \childdoctmp
}
%    \end{macrocode}

%\iffalse
%</package>
%\fi
%
\endinput
\childdocforward{cdocsch2}"|
% \end{tabular}
% \end{center}
% Note that the trailing backslash on each first line
% merely continues the input to the second line
% (for convenient cut ant paste).
% Furthermore, the command |latex| can be replaced by any
% of its alternative versions such as |pdflatex|.
%
% %%%%%%%%%%%%%%%%%%%%%%%%%%%%%%%%%%%%%%%%%%%%%%%%%%%%%%%%%%%%%%%%%%%%%%%%%%%%%%
% %%%%%%%%%%%%%%%%%%%%%%%%%%%%%%%%%%%%%%%%%%%%%%%%%%%%%%%%%%%%%%%%%%%%%%%%%%%%%%
% \section{Implementation}
%\iffalse
%<*package>
%\fi
%
% This section describes the definitions file |childdoc.def|.

% The definitions cannot be loaded using |\usepackage| or |\RequirePackage|
% which has a mechanism to prevent loading a style file more than once.
% When loading the definitions by means of |\input|
% multiple instances have to be prevented manually:
%\iffalse
%This code needs to be before the `\ProvidesFile' directive
%which is defined at the beginning of this file.
%Therefore it is also placed there and commented out here.
%</package>
%<*discard>
%\fi
%    \begin{macrocode}
\ifdefined\childdocmain\endinput\fi
%    \end{macrocode}
%\iffalse
%</discard>
%<*package>
%\fi
%
% \macro{\ifchilddoc}
% \macro{\ifchilddocmanual}
% The conditional |\ifchilddoc| tells whether a
% child (true) or main (false) document is being compiled.
% The conditional |\ifchilddocmanual| tells whether
% the |\includeonly| mechanism is used (false) or
% the selection of child files must be performed manually (true).
% The definitions initialise to false:
%    \begin{macrocode}
\newif\ifchilddoc
\newif\ifchilddocmanual
%    \end{macrocode}

% \macro{\childdocname}
% \macro{\childdocjob}
% The macro |\childdocname| stores the name of the main document
% to be compiled. The macro |\childdocjob| stores the name of
% the document on which the \LaTeX{} compiler was originally invoked.
% The content of |\jobname| cannot be compared
% to filenames specified in the source due to different catcodes.
% The following code rescans |\jobname|, stores the result
% in |\childdocname| and saves a copy in |\childdocjob|:
%    \begin{macrocode}
\edef\childdocname{\scantokens\expandafter{\jobname\noexpand}}
\let\childdocjob\childdocname
%    \end{macrocode}

% \macro{\childdocdisable}
% The macro |\childdocdisable| prevents the main file
% from being processed more than once.
% At this stage, the main document command |\childdocmain|
% is assumed to be called once again where it should do nothing.
% Any subsequent call to it should prevent
% a secondary processing of the main document
% It overwrites the forwarding commands
% |\childdocof| and |\childdocforward|
% with empty macros to prevent further inclusions of the main document:
%    \begin{macrocode}
\newcommand{\childdocdisable}
{
  \renewcommand{\childdocmain}[1]{\renewcommand{\childdocmain}[1]{\endinput}}
  \renewcommand{\childdocof}[1]{}
  \renewcommand{\childdocby}[2][]{}
  \renewcommand{\childdocforward}[2][]{}
  \renewcommand{\childdocdisable}{}
}
%    \end{macrocode}

% \macro{\childdocmain}
% The macro |\childdocmain| is to be called at the top of the main file
% with nothing or the main filename (without extension) as argument.
% First, it breaks loops.
% If the argument is not empty and does not match |\childdocname|
% (which is set by the first inclusion of |childdoc.def|),
% |\ifchilddoc| is set to true, |\includeonly| is applied to the child file
% and |\jobname| is set to the main file
% (for proper handling of |.aux| files):
%    \begin{macrocode}
\newcommand{\childdocmain}[1]
{
  \childdocdisable\childdocmain{}
  \if?#1?\else
    \begingroup
      \def\childdoctmp{#1}
      \ifx\childdoctmp\childdocname
        \def\childdoctmp{}
      \else
        \def\childdoctmp
        {
          \childdoctrue
          \includeonly{\childdocname}
          \def\childdocjob{#1}
          \def\jobname{#1}
        }
      \fi
      \expandafter
    \endgroup
    \childdoctmp
  \fi
}
%    \end{macrocode}

% \macro{\childdocof}
% The command |\childdocof| redirects
% compilation to the main file |#1|.
%    \begin{macrocode}
\newcommand{\childdocof}[1]
{
  \childdocdisable
  \childdoctrue
  \includeonly{\childdocname}
  \def\jobname{#1}
  \def\childdocjob{#1}
  \input{#1}
}
%    \end{macrocode}

% \macro{\childdocby}
% The command |\childdocby| ....
%    \begin{macrocode}
\newcommand{\childdocby}[2][]
{
  \childdocdisable
  \childdoctrue
  \childdocmanualtrue
  \if?#1?\else
    \def\jobname{#2}
  \fi
  \def\childdocjob{#2}
  \input{#2}
  \endinput
}
%    \end{macrocode}

% \macro{\childdocforward}
% The command |\childdocforward| redirects
% compilation to the main file or
% (if the optional argument is given) a child file.
% Parameters are set as if the main file
% or a child file starting with |\childdocof| was compiled.
% Then compilation is handed over to the main file:
%    \begin{macrocode}
\newcommand{\childdocforward}[2][]
{
  \begingroup
    \if?#1?
      \def\childdoctmp
      {
        \def\childdocname{#2}
        \def\childdocjob{#2}
        \def\jobname{#2}
        \input{#2}
        \endinput
      }
    \else
      \def\childdoctmp
      {
        \childdocdisable
        \def\childdocname{#2}
        \childdoctrue
        \includeonly{#2}
        \def\childdocjob{#1}
        \def\jobname{#1}
        \input{#1}
        \endinput
      }
    \fi
    \expandafter
  \endgroup
  \childdoctmp
}
%    \end{macrocode}

% \macro{\childdocforwardprefix}
% The command |\childdocforwardprefix| redirects
% compilation to the main or a child file by means of a pattern.
% The prefix |#1| in the current filename is replaced by |#2|
% and the suffix of the current filename is kept
% (it is assumed that the filename does not contain the substring `|~~~|'
% which is used as a delimiter).
% Compilation is handed over to the new file by |\childdocforward|:
%    \begin{macrocode}
\newcommand{\childdocforwardprefix}[3][]
{
  \begingroup
    \def\childdocextract #2##1~~~{\def\childdoctmp{\childdocforward[#1]{#3##1}}}
    \expandafter\childdocextract\childdocname~~~
    \expandafter
  \endgroup
  \childdoctmp
}
%    \end{macrocode}

% \macro{\childdoc}
% The deprecated macro |\childdoc| is a legacy version of |\childdocmain|:
%    \begin{macrocode}
\newcommand{\childdoc}{\childdocmain}
%    \end{macrocode}

% \macro{\childdocredirect}
% The deprecated macro |\childdocredirect| is a legacy version
% of |\childdocforward| and |\childdocforwardprefix|:
%    \begin{macrocode}
\newcommand{\childdocredirect}[2][]
{
  \begingroup
    \if?#1?
      \def\childdoctmp{\childdocforward{#2}}
    \else
      \def\childdoctmp{\childdocforwardprefix{#1}{#2}}
    \fi
    \expandafter
  \endgroup
  \childdoctmp
}
%    \end{macrocode}

%\iffalse
%</package>
%\fi
%
\endinput
\childdocforward[|\textit{main}|]{|\textit{dest}|}"|
\end{center}
%
Here \textit{target} is the name of the output file,
\textit{main} is the name of the main file
and \textit{dest} is the name of the main or child file to be processed
(all filenames without extensions).
The optional argument \textit{main} can be omitted
if \textit{main} matches \textit{dest}.
Optionally, compilation \textit{flags} can be defined via |\def| commands.
This command line makes the \TeX{} engine believe
it is compiling the file \textit{target}
whose content is specified as the latter parameter.
The provided code then forwards the processing to
\textit{main} or \textit{dest} as described in \secref{sec:forward}.

%%%%%%%%%%%%%%%%%%%%%%%%%%%%%%%%%%%%%%%%%%%%%%%%%%%%%%%%%%%%%%%%%%%%%%%%%%%%%%%%
\subsection{Include by Input}
\label{sec:input}

Including child documents by |\include| has some restrictions by design.
Most notably, the content of a child document always occupies
its own set of pages; pages cannot be shared between child documents.
Usually, this behaviour makes perfect sense
because each child document contain an essential part of the document.
However, in some situations it may be desirable to compose
a document from a collection of parts
without having mandatory page breaks between then.
For this case, the package
provides a mechanism to include parts
by |\input| which can also be processed individually.
However, by construction this mechanism
requires manual handling of the content to be output.

%%%%%%%%%%%%%%%%%%%%%%%%%%%%%%%%%%%%%%%%
\DescribeMacro{\ifchilddocmanual}
The main file should be prepared as usual, see \secref{sec:include}.
However, the document body must make a distinction
between processing of an individual part and of the main document, e.g.:
%
\begin{center}
\begin{tabular}{l}
|\ifchilddocmanual|\\
|\input{\childdocname}|\\
|\||else|\\
\textit{document body with }|\input{|\textit{part}|}|\\
|\||fi|
\end{tabular}
\end{center}
%
The conditional |\ifchilddocmanual| is true whenever
a part to be included by |\input| is being compiled,
and the name of the part is stored in |\childdocname|.

%%%%%%%%%%%%%%%%%%%%%%%%%%%%%%%%%%%%%%%%
\DescribeMacro{\childdocby}
Each part to be included by |\input| should start with:
%
\begin{center}
\begin{tabular}{l}
|% \iffalse
%
% childdoc.dtx Copyright (C) 2017-2018 Niklas Beisert
%
% This work may be distributed and/or modified under the
% conditions of the LaTeX Project Public License, either version 1.3
% of this license or (at your option) any later version.
% The latest version of this license is in
%   http://www.latex-project.org/lppl.txt
% and version 1.3 or later is part of all distributions of LaTeX
% version 2005/12/01 or later.
%
% This work has the LPPL maintenance status `maintained'.
%
% The Current Maintainer of this work is Niklas Beisert.
%
% This work consists of the files childdoc.dtx and childdoc.ins
% and the derived files childdoc.def and cdocsamp.tex with
% cdocsch1.tex, cdocsch2.tex, cdocsdrf.tex, cdocsfn1.tex, cdocsfn2.tex.
%
%<package>\ifdefined\childdocmain\endinput\fi
%<package>\ProvidesFile{childdoc.def}[2018/12/30 v2.0 child document driver]
%<samplemain>\ProvidesFile{cdocsamp.tex}[2018/12/30 v2.0 sample for childdoc]
%<*driver>
%\ProvidesFile{childdoc.drv}[2018/12/30 v2.0 childdoc reference manual file]
\PassOptionsToClass{10pt,a4paper}{article}
\documentclass{ltxdoc}

\usepackage[margin=35mm]{geometry}
\usepackage{hyperref}
\usepackage{hyperxmp}
\usepackage[usenames]{color}

\hypersetup{colorlinks=true}
\hypersetup{pdfstartview=FitH}
\hypersetup{pdfpagemode=UseNone}
\hypersetup{pdfsource={}}
\hypersetup{pdflang={en-UK}}
\hypersetup{pdfcopyright={Copyright 2017-2018 Niklas Beisert.
  This work may be distributed and/or modified under the
  conditions of the LaTeX Project Public License, either version 1.3
  of this license or (at your option) any later version.}}
\hypersetup{pdflicenseurl={http://www.latex-project.org/lppl.txt}}
\hypersetup{pdfcontactaddress={ETH Zurich, ITP, HIT K,
  Wolfgang-Pauli-Strasse 27}}
\hypersetup{pdfcontactpostcode={8093}}
\hypersetup{pdfcontactcity={Zurich}}
\hypersetup{pdfcontactcountry={Switzerland}}
\hypersetup{pdfcontactemail={nbeisert@itp.phys.ethz.ch}}
\hypersetup{pdfcontacturl={http://people.phys.ethz.ch/\xmptilde nbeisert/}}

\newcommand{\secref}[1]{\hyperref[#1]{section \ref*{#1}}}

\parskip1ex
\parindent0pt
\let\olditemize\itemize
\def\itemize{\olditemize\parskip0pt}

\begin{document}

\title{The \textsf{childdoc} Package}
\hypersetup{pdftitle={The childdoc Package}}
\author{Niklas Beisert\\[2ex]
  Institut f\"ur Theoretische Physik\\
  Eidgen\"ossische Technische Hochschule Z\"urich\\
  Wolfgang-Pauli-Strasse 27, 8093 Z\"urich, Switzerland\\[1ex]
  \href{mailto:nbeisert@itp.phys.ethz.ch}
  {\texttt{nbeisert@itp.phys.ethz.ch}}}
\hypersetup{pdfauthor={Niklas Beisert}}
\hypersetup{pdfsubject={Manual for the LaTeX2e Package childdoc}}
\date{30 December 2018, \textsf{v2.0}}
\maketitle

\begin{abstract}\noindent
\textsf{childdoc} is a \LaTeXe{} package
that enables the direct compilation
of document sections included by |\include|
to individual files.
\end{abstract}

\begingroup
\parskip0ex
\tableofcontents
\endgroup

%%%%%%%%%%%%%%%%%%%%%%%%%%%%%%%%%%%%%%%%%%%%%%%%%%%%%%%%%%%%%%%%%%%%%%%%%%%%%%%%
%%%%%%%%%%%%%%%%%%%%%%%%%%%%%%%%%%%%%%%%%%%%%%%%%%%%%%%%%%%%%%%%%%%%%%%%%%%%%%%%
\section{Introduction}

\LaTeX{} provides a mechanism to structure a large document (such as a book)
into a main file and several child files (containing the chapters)
using the |\include| command.
This mechanism is beneficial for documents
which span hundreds of pages in order to
make the source file(s) more manageable.
Moreover, compilation can be restricted to
selected child files by means of the |\includeonly| command.
The latter feature can be used to reduce the compilation time while editing
(this was significantly more useful in the earlier days of \LaTeX{})
or to generate a smaller document which is easier to navigate.
Another application of |\includeonly| is to generate
documents consisting of selected parts of the complete document.

However, there are a few drawbacks of the plain |\include| mechanism:
\begin{itemize}
\item
The child files cannot be compiled on their own,
they can only be compiled via the main file.
A naive editing environment
(such as a text editor with an option
to have the current file processed by \LaTeX)
may require one to switch to the main file before compiling;
attempting to compile the child file produces errors.
\item
The main file must be modified (each time)
to adjust the |\includeonly| command
to the present needs. This easily leaves the main file in a messy state.
\item
The generated document will always carry the filename
of the main document. This is inconvenient if
several child files are to be compiled and
to be kept for distribution.
\end{itemize}

The present package provides a simple interface
to make child files individually compilable by \LaTeX{}.
Compiling a child file then has the same effect as compiling
the main file with an |\includeonly| command
to select the appropriate child.
Moreover the generated document will carry the name of the child
rather than the main file.
This resolves all three above issues.

This feature is meant to make the editing of books,
thesis documents and lecture notes somewhat more convenient.
However, the package can also be used efficiently for
composing a series of documents (such as exercise sheets)
which are typically distributed individually.
It then assists the author in generating the individual documents
(potentially in different versions)
as well as a document containing the collected series.
Another application is in developing style files
or other kinds of included material
where compilation of the style file could redirect
to a sample or test file.

%%%%%%%%%%%%%%%%%%%%%%%%%%%%%%%%%%%%%%%%%%%%%%%%%%%%%%%%%%%%%%%%%%%%%%%%%%%%%%%%
%%%%%%%%%%%%%%%%%%%%%%%%%%%%%%%%%%%%%%%%%%%%%%%%%%%%%%%%%%%%%%%%%%%%%%%%%%%%%%%%
\section{Usage}

First of all, the package \textsf{childdoc} is \emph{not} a standard
\LaTeXe{} |.sty| style file! Therefore it needs to be invoked in
a non-standard way.

%%%%%%%%%%%%%%%%%%%%%%%%%%%%%%%%%%%%%%%%%%%%%%%%%%%%%%%%%%%%%%%%%%%%%%%%%%%%%%%%
\subsection{Included Files}
\label{sec:include}

%%%%%%%%%%%%%%%%%%%%%%%%%%%%%%%%%%%%%%%%
\DescribeMacro{\childdocmain}
To use the package, add the commands
\begin{center}
\begin{tabular}{l}
|% \iffalse
%
% childdoc.dtx Copyright (C) 2017-2018 Niklas Beisert
%
% This work may be distributed and/or modified under the
% conditions of the LaTeX Project Public License, either version 1.3
% of this license or (at your option) any later version.
% The latest version of this license is in
%   http://www.latex-project.org/lppl.txt
% and version 1.3 or later is part of all distributions of LaTeX
% version 2005/12/01 or later.
%
% This work has the LPPL maintenance status `maintained'.
%
% The Current Maintainer of this work is Niklas Beisert.
%
% This work consists of the files childdoc.dtx and childdoc.ins
% and the derived files childdoc.def and cdocsamp.tex with
% cdocsch1.tex, cdocsch2.tex, cdocsdrf.tex, cdocsfn1.tex, cdocsfn2.tex.
%
%<package>\ifdefined\childdocmain\endinput\fi
%<package>\ProvidesFile{childdoc.def}[2018/12/30 v2.0 child document driver]
%<samplemain>\ProvidesFile{cdocsamp.tex}[2018/12/30 v2.0 sample for childdoc]
%<*driver>
%\ProvidesFile{childdoc.drv}[2018/12/30 v2.0 childdoc reference manual file]
\PassOptionsToClass{10pt,a4paper}{article}
\documentclass{ltxdoc}

\usepackage[margin=35mm]{geometry}
\usepackage{hyperref}
\usepackage{hyperxmp}
\usepackage[usenames]{color}

\hypersetup{colorlinks=true}
\hypersetup{pdfstartview=FitH}
\hypersetup{pdfpagemode=UseNone}
\hypersetup{pdfsource={}}
\hypersetup{pdflang={en-UK}}
\hypersetup{pdfcopyright={Copyright 2017-2018 Niklas Beisert.
  This work may be distributed and/or modified under the
  conditions of the LaTeX Project Public License, either version 1.3
  of this license or (at your option) any later version.}}
\hypersetup{pdflicenseurl={http://www.latex-project.org/lppl.txt}}
\hypersetup{pdfcontactaddress={ETH Zurich, ITP, HIT K,
  Wolfgang-Pauli-Strasse 27}}
\hypersetup{pdfcontactpostcode={8093}}
\hypersetup{pdfcontactcity={Zurich}}
\hypersetup{pdfcontactcountry={Switzerland}}
\hypersetup{pdfcontactemail={nbeisert@itp.phys.ethz.ch}}
\hypersetup{pdfcontacturl={http://people.phys.ethz.ch/\xmptilde nbeisert/}}

\newcommand{\secref}[1]{\hyperref[#1]{section \ref*{#1}}}

\parskip1ex
\parindent0pt
\let\olditemize\itemize
\def\itemize{\olditemize\parskip0pt}

\begin{document}

\title{The \textsf{childdoc} Package}
\hypersetup{pdftitle={The childdoc Package}}
\author{Niklas Beisert\\[2ex]
  Institut f\"ur Theoretische Physik\\
  Eidgen\"ossische Technische Hochschule Z\"urich\\
  Wolfgang-Pauli-Strasse 27, 8093 Z\"urich, Switzerland\\[1ex]
  \href{mailto:nbeisert@itp.phys.ethz.ch}
  {\texttt{nbeisert@itp.phys.ethz.ch}}}
\hypersetup{pdfauthor={Niklas Beisert}}
\hypersetup{pdfsubject={Manual for the LaTeX2e Package childdoc}}
\date{30 December 2018, \textsf{v2.0}}
\maketitle

\begin{abstract}\noindent
\textsf{childdoc} is a \LaTeXe{} package
that enables the direct compilation
of document sections included by |\include|
to individual files.
\end{abstract}

\begingroup
\parskip0ex
\tableofcontents
\endgroup

%%%%%%%%%%%%%%%%%%%%%%%%%%%%%%%%%%%%%%%%%%%%%%%%%%%%%%%%%%%%%%%%%%%%%%%%%%%%%%%%
%%%%%%%%%%%%%%%%%%%%%%%%%%%%%%%%%%%%%%%%%%%%%%%%%%%%%%%%%%%%%%%%%%%%%%%%%%%%%%%%
\section{Introduction}

\LaTeX{} provides a mechanism to structure a large document (such as a book)
into a main file and several child files (containing the chapters)
using the |\include| command.
This mechanism is beneficial for documents
which span hundreds of pages in order to
make the source file(s) more manageable.
Moreover, compilation can be restricted to
selected child files by means of the |\includeonly| command.
The latter feature can be used to reduce the compilation time while editing
(this was significantly more useful in the earlier days of \LaTeX{})
or to generate a smaller document which is easier to navigate.
Another application of |\includeonly| is to generate
documents consisting of selected parts of the complete document.

However, there are a few drawbacks of the plain |\include| mechanism:
\begin{itemize}
\item
The child files cannot be compiled on their own,
they can only be compiled via the main file.
A naive editing environment
(such as a text editor with an option
to have the current file processed by \LaTeX)
may require one to switch to the main file before compiling;
attempting to compile the child file produces errors.
\item
The main file must be modified (each time)
to adjust the |\includeonly| command
to the present needs. This easily leaves the main file in a messy state.
\item
The generated document will always carry the filename
of the main document. This is inconvenient if
several child files are to be compiled and
to be kept for distribution.
\end{itemize}

The present package provides a simple interface
to make child files individually compilable by \LaTeX{}.
Compiling a child file then has the same effect as compiling
the main file with an |\includeonly| command
to select the appropriate child.
Moreover the generated document will carry the name of the child
rather than the main file.
This resolves all three above issues.

This feature is meant to make the editing of books,
thesis documents and lecture notes somewhat more convenient.
However, the package can also be used efficiently for
composing a series of documents (such as exercise sheets)
which are typically distributed individually.
It then assists the author in generating the individual documents
(potentially in different versions)
as well as a document containing the collected series.
Another application is in developing style files
or other kinds of included material
where compilation of the style file could redirect
to a sample or test file.

%%%%%%%%%%%%%%%%%%%%%%%%%%%%%%%%%%%%%%%%%%%%%%%%%%%%%%%%%%%%%%%%%%%%%%%%%%%%%%%%
%%%%%%%%%%%%%%%%%%%%%%%%%%%%%%%%%%%%%%%%%%%%%%%%%%%%%%%%%%%%%%%%%%%%%%%%%%%%%%%%
\section{Usage}

First of all, the package \textsf{childdoc} is \emph{not} a standard
\LaTeXe{} |.sty| style file! Therefore it needs to be invoked in
a non-standard way.

%%%%%%%%%%%%%%%%%%%%%%%%%%%%%%%%%%%%%%%%%%%%%%%%%%%%%%%%%%%%%%%%%%%%%%%%%%%%%%%%
\subsection{Included Files}
\label{sec:include}

%%%%%%%%%%%%%%%%%%%%%%%%%%%%%%%%%%%%%%%%
\DescribeMacro{\childdocmain}
To use the package, add the commands
\begin{center}
\begin{tabular}{l}
|\input{childdoc.def}|\\
|\childdocmain{}|\\
\end{tabular}
\end{center}
at the very top of the main \LaTeX{} file,
in particular \emph{before} the |\documentclass| statement!
The argument of |\childdocmain| should be left empty
(but it must be present).

%%%%%%%%%%%%%%%%%%%%%%%%%%%%%%%%%%%%%%%%
\DescribeMacro{\childdocof}
Furthermore, add the commands
\begin{center}
\begin{tabular}{l}
|\input{childdoc.def}|\\
|\childdocof{|\textit{main}|}|\\
\end{tabular}
\end{center}
at the top of every child file \textit{child}
which is included by |\include{|\textit{child}|}|
from within the main file
(or at least for those files to be compiled individually).
The argument \textit{main} must be the filename of the main file.

There are a couple of
considerations in setting up the main and child documents:

%%%%%%%%%%%%%%%%%%%%%%%%%%%%%%%%%%%%%%%%
\paragraph{Restrictions.}

Please note the following restrictions:
\begin{itemize}
\item
|\childdocmain| must be called with one argument \textit{main}
to ensure compatibility with earlier version of the package.
It must either be empty (|\childdocmain{}|)
or precisely match the filename of the main file in which it is specified.
See \secref{sec:detection} for further information.
\item
The filename \textit{main} must be specified without the |.tex| extension.
\item
The filename \textit{main} is case sensitive
(even in case-insensitive file systems)
due to internal string comparison.
\item
The argument \textit{main} should be fully expanded, it cannot be a macro.
\item
Subdirectories and special characters should be avoided in filenames.
\item
The command |\childdocmain{|\textit{main}|}| must be followed by a whitespace.
It should not be followed immediately by another command
or by a comment mark `|%|'.
This is because the \TeX{} parser reads the token immediately following
the argument of |\childdocmain| and puts it
at the beginning of every child section;
however, a white\-space is ignored.
\end{itemize}

%%%%%%%%%%%%%%%%%%%%%%%%%%%%%%%%%%%%%%%%
\paragraph{Content of Main File.}

It is advisable to place all content in the child files included by |\include|.
Any output contained in the main file will appear in all child documents
unless suppressed manually;
it cannot be suppressed automatically by the |\includeonly| directive
and thus should normally be avoided.
A method to include some content in the main file
by means of conditional processing is described in \secref{sec:conditional}.

%%%%%%%%%%%%%%%%%%%%%%%%%%%%%%%%%%%%%%%%
\paragraph{Page Numbering.}

When only a part of the document is compiled,
the appropriate numbering of pages
(as well as other status parameters)
is determined from the |.aux| files.
The latter contain information from previous passes.
However this information needs to propagate through
all intermediate child documents.
Therefore the page numbering in child documents may well
be inconsistent until the complete document is compiled at least once.

A useful (if unconventional) way to always ensure a consistent
page numbering is to restart the numbering in each child document
and denote the pages by `\textit{child}|.|\textit{page}'
where \textit{child} represents the chapter/section number of the child file.
This can be achieved by the command
|\numberwithin{page}{|\textit{child}|}|
of the \textsf{amsmath} package
where \textit{child} can be |chapter| or |section|
depending on the chosen structuring.
Alternatively, one can modify the macro |\thepage| appropriately
and reset the counter |page| at the start of each child file.

%%%%%%%%%%%%%%%%%%%%%%%%%%%%%%%%%%%%%%%%%%%%%%%%%%%%%%%%%%%%%%%%%%%%%%%%%%%%%%%%
\subsection{Conditional Processing}
\label{sec:conditional}

The package provides a mechanism to compile different versions
of a document. To customise the versions further some conditional processing
can come in handy to distinguish which version is being compiled.
The package provides two macros to describe the compilation context:

%%%%%%%%%%%%%%%%%%%%%%%%%%%%%%%%%%%%%%%%
\DescribeMacro{\ifchilddoc}
The conditional |\ifchilddoc| distinguishes between the compilation of
child documents and the main document:
%
\begin{center}
|\ifchilddoc |\textit{child-code}| |[|\||else |\textit{main-code}]| \||fi|
\end{center}

%%%%%%%%%%%%%%%%%%%%%%%%%%%%%%%%%%%%%%%%
\DescribeMacro{\childdocname}
\DescribeMacro{\childdocjob}
The macro |\childdocname| contains the filename (without extension)
of the main or child file being processed.
Note that |\childdocjob| will always contain the name of the main file.

%%%%%%%%%%%%%%%%%%%%%%%%%%%%%%%%%%%%%%%%
\paragraph{Title Page.}

Conditional processing can be used to include a title or banner page
in the main document when proper precautions are taken.
Importantly, the code in the main file should ensure that the page counter
(as well as other status parameters which are stored in the |.aux| files)
takes the same value after the conditional processing.
Otherwise the page numbers may take divergent values
depending on which part is compiled.

For example, a title page could be declared by:
%
\begin{center}
\begin{tabular}{l}
|\ifchilddoc\||else|\\
|\addtocounter{page}{-1}|\\
\textit{code for title page}\\
|\newpage|\\
|\||fi|
\end{tabular}
\end{center}
%
A banner page for the child documents can be generated by:
%
\begin{center}
\begin{tabular}{l}
|\ifchilddoc|\\
|\addtocounter{page}{-1}|\\
\textit{code for banner page}\\
|\newpage|\\
|\||fi|
\end{tabular}
\end{center}
%
Here one could write a message such as:
\begin{center}
|This is the part \childdocname{} of \childdocjob{}.|
\end{center}

%%%%%%%%%%%%%%%%%%%%%%%%%%%%%%%%%%%%%%%%%%%%%%%%%%%%%%%%%%%%%%%%%%%%%%%%%%%%%%%%
\subsection{Flags}
\label{sec:flags}

The package makes it easy to generate different versions
of the main or child documents.
To this end compilation flags can be defined
and assigned different default values.
They will be particularly useful in conjunction
with the forwarding mechanism described in \secref{sec:forward}.

For example, it may be useful to have a flag |\version|
which can be set to |draft| or |final|.
The document source will contain some conditional code
depending on the value of |\version|.
Suppose further, the flag should default to |final| for the main file
and to |draft| for child files
which is a natural assignment for editing the document.
This is achieved by placing the following code
in the preamble of the main document
(below the |\childdocmain| directive):
%
\begin{center}
\begin{tabular}{l}
|\ifchilddoc|\\
|\providecommand{\version}{draft}|\\
|\||else|\\
|\providecommand{\version}{final}|\\
|\||fi|
\end{tabular}
\end{center}
%
The definition by |\providecommand| makes sure
that previous definitions are not overwritten.
Further statements |\providecommand{\version}{...}|
can thus be added before the above code to override it.

For the main file, one might add a line
(between |\childdocmain| and the above block)
%
\begin{center}
|%\ifchilddoc\||else\providecommand{\version}{draft}\||fi|
\end{center}
%
which can be uncommented to produce a draft version.
Likewise one can add a line to the very top of a child file
(above the |\childdocof{|\textit{main}|}| directive)
%
\begin{center}
|%\providecommand{\version}{final}|
\end{center}
%
which can be uncommented to produce the final version of this child document.

%%%%%%%%%%%%%%%%%%%%%%%%%%%%%%%%%%%%%%%%%%%%%%%%%%%%%%%%%%%%%%%%%%%%%%%%%%%%%%%%
\subsection{Forwarding}
\label{sec:forward}

Different versions of the main or child documents
using compilation flags as described in \secref{sec:flags}
can be (permanently) stored in different files
for convenient compilation, viewing and distribution.
To this end, the package defines a command
to pass on compilation to a different file:

%%%%%%%%%%%%%%%%%%%%%%%%%%%%%%%%%%%%%%%%
\DescribeMacro{\childdocforward}
The command |\childdocforward| redirects processing to
another source file:
%
\begin{center}
\begin{tabular}{l}
|\input{childdoc.def}|\\
|\childdocforward[|\textit{main}|]{|\textit{dest}|}|\\
\end{tabular}
\end{center}
%
The argument \textit{dest} is the destination file
(without extension).
It should be the main file or one of the child files.
Note that further \textsf{childdoc} directives
such as |\childdocof| and |\childdocforward|
in the indicated file will be processed in this form.
The optional argument \textit{main}
passes on directly to the main file \textit{main}
while pretending to compile the child \textit{dest}.
This form behaves as if \textit{dest}
issues |\childdocof{|\textit{main}|}| right away,
and no further \textsf{childdoc} directives will be processed.

%%%%%%%%%%%%%%%%%%%%%%%%%%%%%%%%%%%%%%%%
\DescribeMacro{\...prefix}
In the alternative form |\childdocforwardprefix|,
%
\begin{center}
\begin{tabular}{l}
|\input{childdoc.def}|\\
|\childdocforwardprefix[|\textit{main}|]{|\textit{prefix}|}{|\textit{dest}|}|
\end{tabular}
\end{center}
%
the destination file is determined by a pattern
depending on the current file:
To make this work, the current file must be called
`{\textit{prefix}\hspace{0.2em}\textit{suffix}}'
with \textit{prefix} matching precisely the argument.
Processing is then passed on to the file
`{\textit{dest}\hspace{0.2em}\textit{suffix}}'.
Surely, the same effect is achieved by
directly specifying the
argument `{\textit{dest}\hspace{0.2em}\textit{suffix}}'
in the first form.
However, that requires to set up a different file
for each child. With the alternative form of the command
all these files can have exactly the same content
which simplifies setting them up and maintaining them.

For example, the following file |draft.tex|
with a compilation flag |\version| as described in \secref{sec:flags}
compiles the main document as a draft:
%
\begin{center}
\begin{tabular}{l}
|\def\version{draft}|\\
|\input{childdoc.def}|\\
|\childdocforward{|\textit{main}|}|
\end{tabular}
\end{center}
%
Likewise, the following files |final|\textit{nn}|.tex|
compile the final version of the child document
|child|\textit{nn}|.tex|:
%
\begin{center}
\begin{tabular}{l}
|\def\version{final}|\\
|\input{childdoc.def}|\\
|\childdocforwardprefix{final}{child}|
\end{tabular}
\end{center}
%

Note that when several versions of a main file and/or of each child file
are to be generated, it may be convenient to set up a |Makefile| or
shell script to automatise the process.

%%%%%%%%%%%%%%%%%%%%%%%%%%%%%%%%%%%%%%%%%%%%%%%%%%%%%%%%%%%%%%%%%%%%%%%%%%%%%%%%
\subsection{Command Line Processing}
\label{sec:commandline}

The effect of redirection files can also be achieved by invoking
the \LaTeX{} compiler with a more elaborate command line.
Most conveniently this should be done as part
of a shell script or a |Makefile|.

When using \textsf{childdoc} in the main file, the following
command lines effectively perform a redirection
(note that depending on the shell being used,
backslashes may have to be doubled: `|\|' $\to$ `|\\|'):
%
\begin{center}
|... -jobname "|\textit{target}|" |\\|"|[\textit{flags}]%
|\input{childdoc.def}\childdocforward[|\textit{main}|]{|\textit{dest}|}"|
\end{center}
%
Here \textit{target} is the name of the output file,
\textit{main} is the name of the main file
and \textit{dest} is the name of the main or child file to be processed
(all filenames without extensions).
The optional argument \textit{main} can be omitted
if \textit{main} matches \textit{dest}.
Optionally, compilation \textit{flags} can be defined via |\def| commands.
This command line makes the \TeX{} engine believe
it is compiling the file \textit{target}
whose content is specified as the latter parameter.
The provided code then forwards the processing to
\textit{main} or \textit{dest} as described in \secref{sec:forward}.

%%%%%%%%%%%%%%%%%%%%%%%%%%%%%%%%%%%%%%%%%%%%%%%%%%%%%%%%%%%%%%%%%%%%%%%%%%%%%%%%
\subsection{Include by Input}
\label{sec:input}

Including child documents by |\include| has some restrictions by design.
Most notably, the content of a child document always occupies
its own set of pages; pages cannot be shared between child documents.
Usually, this behaviour makes perfect sense
because each child document contain an essential part of the document.
However, in some situations it may be desirable to compose
a document from a collection of parts
without having mandatory page breaks between then.
For this case, the package
provides a mechanism to include parts
by |\input| which can also be processed individually.
However, by construction this mechanism
requires manual handling of the content to be output.

%%%%%%%%%%%%%%%%%%%%%%%%%%%%%%%%%%%%%%%%
\DescribeMacro{\ifchilddocmanual}
The main file should be prepared as usual, see \secref{sec:include}.
However, the document body must make a distinction
between processing of an individual part and of the main document, e.g.:
%
\begin{center}
\begin{tabular}{l}
|\ifchilddocmanual|\\
|\input{\childdocname}|\\
|\||else|\\
\textit{document body with }|\input{|\textit{part}|}|\\
|\||fi|
\end{tabular}
\end{center}
%
The conditional |\ifchilddocmanual| is true whenever
a part to be included by |\input| is being compiled,
and the name of the part is stored in |\childdocname|.

%%%%%%%%%%%%%%%%%%%%%%%%%%%%%%%%%%%%%%%%
\DescribeMacro{\childdocby}
Each part to be included by |\input| should start with:
%
\begin{center}
\begin{tabular}{l}
|\input{childdoc.def}|\\
|\childdocby{|\textit{main}|}|\\
\end{tabular}
\end{center}
%
The directive |\childdocby| is similar to |\childdocof|
described in \secref{sec:include},
but the subsequent selection of content must be done manually.
To that end, both |\ifchilddoc| and |\ifchilddocmanual|
will be true upon processing of a part,
and the name of the part is stored in |\childdocname|.
Note that |\jobname| will be set to the filename of the current part
so that each part receives an individual |.aux| file
that does not interfere with the |.aux| file(s) of the main document.
This behaviour can be altered by the alternative form
|\childdocby[*]{|\textit{main}|}| (with a non-empty optional argument)
which uses the |.aux| file of the main document
by setting |\jobname| to \textit{main}.

%%%%%%%%%%%%%%%%%%%%%%%%%%%%%%%%%%%%%%%%%%%%%%%%%%%%%%%%%%%%%%%%%%%%%%%%%%%%%%%%
\subsection{Driver Development}
\label{sec:driver}

The \textsf{childdoc} mechanism can also be use for the development
of definition files such as \LaTeX{} styles or classes.
This case differs from the above setup with multiple parts
included by |\include| in that no |\includeonly| should be invoked.
This can be achieved by starting the include file
(before |\ProvidesPackage|) with:
%
\begin{center}
\begin{tabular}{l}
|\input{childdoc.def}|\\
|\childdocforward{|\textit{main}|}|\\
\end{tabular}
\end{center}
%
or alternatively with:
%
\begin{center}
\begin{tabular}{l}
|\input{childdoc.def}|\\
|\childdocby{|\textit{main}|}|\\
\end{tabular}
\end{center}
%
Both forms have slightly different effects as described above.
The main file is prepared as usual, see \secref{sec:include}.

%%%%%%%%%%%%%%%%%%%%%%%%%%%%%%%%%%%%%%%%%%%%%%%%%%%%%%%%%%%%%%%%%%%%%%%%%%%%%%%%
\subsection{Legacy Detection}
\label{sec:detection}

The directive |\childdocmain| in the main file can detect
whether the complete document or merely a child is to be compiled
even without using the directive |\childdocof|.
This method is deprecated because it is less robust
and there is no compelling reason to use it;
it is merely provided for backward compatibility
and it may be removed in future versions.

If the detection mechanism is to be used,
it is mandatory to correctly specify
the filename of the main file as the argument of |\childdocmain|:
%
\begin{center}
\begin{tabular}{l}
|\input{childdoc.def}|\\
|\childdocmain{|\textit{main}|}|\\
\end{tabular}
\end{center}
%
If |\jobname| does not match the argument \textit{main} of |\childdocmain|,
it is assumed that |\jobname| points to the child file to be compiled.
When using |\childdocmain| with the main file specified as argument,
it suffices to start a child file
with just |\input{|\textit{main}|}|
without loading of the package and using |\childdocof|.
If instead all processing is done
with the appropriate \textsf{childdoc} directives,
the argument of \textit{main} of |\childdocmain| can be empty.

An alternative version of the command line processing described
in \secref{sec:commandline} using the detection mechanism reads:
%
\begin{center}
|... -jobname "|\textit{target}|" "|[\textit{flags}]%
[|\def\jobname{|\textit{dest}|}|]|\input{|\textit{main}|}"|
\end{center}

%%%%%%%%%%%%%%%%%%%%%%%%%%%%%%%%%%%%%%%%%%%%%%%%%%%%%%%%%%%%%%%%%%%%%%%%%%%%%%%%
\subsection{Manual Code}
\label{sec:manual}

In case one cannot be certain whether the definitions file |childdoc.def|
is installed on the target \TeX{} distribution
and one prefers not to ship it,
it is conceivable to paste a few relevant commands into the sources.

To that end, drop all statements |\input{childdoc.def}|
and perform the replacements as outlined below.
Instead of |\childdocmain{|\textit{main}|}| add the following code
to the top of the main file:
%
\begin{center}
\begin{tabular}{l}
|\||ifdefined\childdocname\endinput\||fi\newif\ifchilddoc|\\
|\edef\childdocname{\scantokens\expandafter{\jobname\noexpand}}|\\
|\def\childdocmain{|\textit{main}|}\||ifx\childdocmain\childdocname\||else|\\
|\childdoctrue\includeonly{\childdocname}\let\jobname\childdocmain\||fi|\\
\end{tabular}
\end{center}
%
Instead of |\childdocof{|\textit{main}|}| just include the main file
at the top of each child file:
%
\begin{center}
|\input{|\textit{main}|}|
\end{center}
%
A simple redirection |\childdocforward{|\textit{dest}|}| is achieved by:
%
\begin{center}
|\def\jobname{|\textit{dest}|}\input{\jobname}|
\end{center}
%
The redirection with prefix
|\childdocforwardprefix[|\textit{prefix}|]{|\textit{dest}|}|
is accomplished by:
%
\begin{center}
\begin{tabular}{l}
|{\edef\jobname{\scantokens\expandafter{\jobname\noexpand}}|\\
|\def\redirectjob |\textit{prefix}|#1~~~{\gdef\jobname{|\textit{dest}|#1}}|\\
|\expandafter\redirectjob\jobname~~~}\input{\jobname}|
\end{tabular}
\end{center}

In an alternative approach,
child documents can be compiled by a specific command line
without additional code or specific definitions:
%
\begin{center}
|... -jobname "|\textit{target}|" "|[\textit{flags}]%
|\includeonly{|\textit{dest}|}\input{|\textit{main}|}"|
\end{center}
%

%%%%%%%%%%%%%%%%%%%%%%%%%%%%%%%%%%%%%%%%%%%%%%%%%%%%%%%%%%%%%%%%%%%%%%%%%%%%%%%%
%%%%%%%%%%%%%%%%%%%%%%%%%%%%%%%%%%%%%%%%%%%%%%%%%%%%%%%%%%%%%%%%%%%%%%%%%%%%%%%%
\section{Information}

%%%%%%%%%%%%%%%%%%%%%%%%%%%%%%%%%%%%%%%%%%%%%%%%%%%%%%%%%%%%%%%%%%%%%%%%%%%%%%%%
\subsection{Copyright}

Copyright \copyright{} 2017--2018 Niklas Beisert

This work may be distributed and/or modified under the
conditions of the \LaTeX{} Project Public License, either version 1.3
of this license or (at your option) any later version.
The latest version of this license is in
  \url{http://www.latex-project.org/lppl.txt}
and version 1.3 or later is part of all distributions of \LaTeX{}
version 2005/12/01 or later.

This work has the LPPL maintenance status `maintained'.

The Current Maintainer of this work is Niklas Beisert.

This work consists of the files |README.txt|, |childdoc.ins| and |childdoc.dtx|
as well as the derived files |childdoc.def|, |cdocsamp.tex|
with |cdocsch1.tex|, |cdocsch2.tex|, |cdocspt3.tex|, |cdocspt4.tex|,
|cdocsdrf.tex|, |cdocsfn1.tex|, |cdocsfn2.tex|
as well as |childdoc.pdf|.

%%%%%%%%%%%%%%%%%%%%%%%%%%%%%%%%%%%%%%%%%%%%%%%%%%%%%%%%%%%%%%%%%%%%%%%%%%%%%%%%
\subsection{Files and Installation}

The package consists of the files:
%
\begin{center}
\begin{tabular}{ll}
    |README.txt|   & readme file \\
    |childdoc.ins| & installation file \\
    |childdoc.dtx| & source file \\
    |childdoc.def| & definition file \\
    |cdocsamp.tex| & sample main file \\
    |cdocsch1.tex| & sample include file \\
    |cdocsch2.tex| & sample include file \\
    |cdocspt3.tex| & sample part file \\
    |cdocspt4.tex| & sample part file \\
    |cdocsdrf.tex| & sample redirection file \\
    |cdocsfn1.tex| & sample redirection file \\
    |cdocsfn2.tex| & sample redirection file \\
    |childdoc.pdf| & manual
\end{tabular}
\end{center}
%
The distribution consists of the files
|README.txt|, |childdoc.ins| and |childdoc.dtx|.
%
\begin{itemize}
\item
Run (pdf)\LaTeX{} on |childdoc.dtx|
to compile the manual |childdoc.pdf| (this file).
\item
Run \LaTeX{} on |childdoc.ins| to create the definitions file |childdoc.def|
and the sample |cdocsamp.tex| with include files
|cdocsch1.tex|, |cdocsch2.tex|, |cdocspt3.tex|, |cdocspt4.tex|,
|cdocsdrf.tex|, |cdocsfn1.tex|, |cdocsfn2.tex|.
Then copy the file |childdoc.def| to an appropriate directory of your \LaTeX{}
distribution, e.g.\ \textit{texmf-root}|/tex/latex/childdoc|.
\end{itemize}

%%%%%%%%%%%%%%%%%%%%%%%%%%%%%%%%%%%%%%%%%%%%%%%%%%%%%%%%%%%%%%%%%%%%%%%%%%%%%%%%
\subsection{Related CTAN Packages}

There are several other packages which offer a similar functionality:
%
\begin{itemize}
\item
The packages
\href{http://ctan.org/pkg/docmute}{\textsf{docmute}},
\href{http://ctan.org/pkg/includex}{\textsf{includex}} and
\href{http://ctan.org/pkg/standalone}{\textsf{standalone}}
provide commands to include only the document body of
a child file thus allowing both files to be compiled individually.
\item
The packages \href{http://ctan.org/pkg/subdocs}{\textsf{subdocs}}
and \href{http://ctan.org/pkg/subfiles}{\textsf{subfiles}}
provide structures in which the main and child documents can be
encapsulated and allowing them to be compiled individually.
The inclusion mechanism is different from the conventional |\include|.
\item
The package \href{http://ctan.org/pkg/combine}{\textsf{combine}}
is an elaborate solution to combine several documents into one.
\end{itemize}
%
See also the CTAN topic \href{http://ctan.org/topic/subdocs}{\textsf{subdocs}}
for further related packages.
The present package differs from the above solutions in that
a document structure constructed with the conventional |\include| mechanism
just needs two extra commands at the top of every file
such that all constituent files can be compiled individually.

%%%%%%%%%%%%%%%%%%%%%%%%%%%%%%%%%%%%%%%%%%%%%%%%%%%%%%%%%%%%%%%%%%%%%%%%%%%%%%%%
%\subsection{Feature Suggestions}
%
%The following is a list of features which may be useful for future
%versions of this package:
%%
%\begin{itemize}
%\item
%\ldots
%\end{itemize}

%%%%%%%%%%%%%%%%%%%%%%%%%%%%%%%%%%%%%%%%%%%%%%%%%%%%%%%%%%%%%%%%%%%%%%%%%%%%%%%%
\subsection{Revision History}

%%%%%%%%%%%%%%%%%%%%%%%%%%%%%%%%%%%%%%%%
\paragraph{v2.0:} 2018/12/30

\begin{itemize}
\item
immediate forward processing
\item
added |\childdocby| mechanism
\item
manual restructured
\end{itemize}

%%%%%%%%%%%%%%%%%%%%%%%%%%%%%%%%%%%%%%%%
\paragraph{v1.6:} 2018/01/17

\begin{itemize}
\item
application for development of include files
\item
corrections to manual
\end{itemize}

%%%%%%%%%%%%%%%%%%%%%%%%%%%%%%%%%%%%%%%%
\paragraph{v1.5:} 2017/05/21

\begin{itemize}
\item
more complete structuring introduced
\item
|\childdocof| introduced
\item
|\childdoc| renamed to |\childdocmain|
\item
|\childredirect| renamed to |\childdocforward| and |\childdocforwardprefix|
and functionality expanded
\end{itemize}

%%%%%%%%%%%%%%%%%%%%%%%%%%%%%%%%%%%%%%%%
\paragraph{v1.0:} 2017/04/27

\begin{itemize}
\item
manual and install package
\item
first version published on CTAN
\end{itemize}

%%%%%%%%%%%%%%%%%%%%%%%%%%%%%%%%%%%%%%%%
\paragraph{v0.6:} 2017/04/26

\begin{itemize}
\item
redirection mechanism added
\end{itemize}

%%%%%%%%%%%%%%%%%%%%%%%%%%%%%%%%%%%%%%%%
\paragraph{v0.5:} 2017/04/26

\begin{itemize}
\item
functionality in definition file
\end{itemize}


%%%%%%%%%%%%%%%%%%%%%%%%%%%%%%%%%%%%%%%%%%%%%%%%%%%%%%%%%%%%%%%%%%%%%%%%%%%%%%%%
%%%%%%%%%%%%%%%%%%%%%%%%%%%%%%%%%%%%%%%%%%%%%%%%%%%%%%%%%%%%%%%%%%%%%%%%%%%%%%%%
%%%%%%%%%%%%%%%%%%%%%%%%%%%%%%%%%%%%%%%%%%%%%%%%%%%%%%%%%%%%%%%%%%%%%%%%%%%%%%%%
\appendix

\settowidth\MacroIndent{\rmfamily\scriptsize 000\ }

 \DocInput{childdoc.dtx}

\end{document}
%</driver>
% \fi
%
% %%%%%%%%%%%%%%%%%%%%%%%%%%%%%%%%%%%%%%%%%%%%%%%%%%%%%%%%%%%%%%%%%%%%%%%%%%%%%%
% %%%%%%%%%%%%%%%%%%%%%%%%%%%%%%%%%%%%%%%%%%%%%%%%%%%%%%%%%%%%%%%%%%%%%%%%%%%%%%
% \section{Sample}
%\iffalse
%<*samplemain>
%\fi
%
% The following presents a sample document
% with two chapters, two parts, a title page,
% a compile flag as well as three forwarding files to set the flag.
% It consists of eight |.tex| files:
% \begin{center}
% \begin{tabular}{ll}
% |cdocsamp.tex|&main file\\
% |cdocsch1.tex|&include file for chapter 1\\
% |cdocsch2.tex|&include file for chapter 2\\
% |cdocspt3.tex|&include file for part 3\\
% |cdocspt4.tex|&include file for part 4\\
% |cdocsdrf.tex|&forwarding file for main file in draft mode\\
% |cdocsfi1.tex|&forwarding file for final version of chapter 1\\
% |cdocsfi2.tex|&forwarding file for final version of chapter 2\\
% \end{tabular}
% \end{center}
% Each of the eight files can be compiled directly by the \LaTeX{} compiler.
%
% %%%%%%%%%%%%%%%%%%%%%%%%%%%%%%%%%%%%%%
% \paragraph{Main File.}
%
% The main file is called |cdocsamp.tex|.
%
% Load the \textsf{childdoc} definitions and
% declare the filename for the main document:
%    \begin{macrocode}
\input{childdoc.def}
\childdocmain{}
%    \end{macrocode}

% Optional override for |\version| flag:
%    \begin{macrocode}
%%\ifchilddoc\else\providecommand{\version}{draft}\fi
%    \end{macrocode}

% Define the default values for the |\version| flag
% (|final| for the main file and |draft| for childs):
%    \begin{macrocode}
\ifchilddoc
\providecommand{\version}{draft}
\else
\providecommand{\version}{final}
\fi
%    \end{macrocode}

% Load the standard document class:
%    \begin{macrocode}
\documentclass[12pt]{article}
%    \end{macrocode}

% Start the document body:
%    \begin{macrocode}
\begin{document}
%    \end{macrocode}

% Declare a title page.
% Print title, part of document being processed and version flag:
%    \begin{macrocode}
\addtocounter{page}{-1}
\begin{center}
{\LARGE\bfseries{}childdoc example\par}
\vspace{1cm}
\ifchilddoc
\ifchilddocmanual part\else chapter\fi:
`\childdocname' of `\childdocjob'\par
\else
main document: `\childdocjob'\par
\fi
version: \version\par
\end{center}
\newpage
%    \end{macrocode}

% Manually include selected file,
% otherwise process as usual:
%    \begin{macrocode}
\ifchilddocmanual
\section*{part `\childdocname'}
\input{\childdocname}
\else
%    \end{macrocode}

% Include the two chapters:
%    \begin{macrocode}
\include{cdocsch1}
\include{cdocsch2}
%    \end{macrocode}

% Include the two parts unless only chapters should be displayed:
%    \begin{macrocode}
\ifchilddoc\else
\section{part three}
\input{cdocspt3}
\section{part four}
\input{cdocspt4}
\fi
%    \end{macrocode}

% Process as usual until here:
%    \begin{macrocode}
\fi
%    \end{macrocode}

% End of document body:
%    \begin{macrocode}
\end{document}
%    \end{macrocode}
%\iffalse
%</samplemain>
%\fi
%
% %%%%%%%%%%%%%%%%%%%%%%%%%%%%%%%%%%%%%%
% \paragraph{Chapter Include Files.}
%
% The include files are called |cdocsch1.tex| and |cdocsch2.tex|.
%
%\iffalse
%<*samplechap1|samplechap2>
%\fi

% Optional override for |\version| flag:
%    \begin{macrocode}
%%\providecommand{\version}{final}
%    \end{macrocode}

% Include the main document:
%    \begin{macrocode}
\input{childdoc.def}
\childdocof{cdocsamp}
%    \end{macrocode}

%\iffalse
%</samplechap1|samplechap2>
%\fi
%
%\iffalse
%<*samplechap1>
%\fi
% Some text for chapter 1:
%    \begin{macrocode}
\section{one}
some text in chapter one
%    \end{macrocode}

%\iffalse
%</samplechap1>
%\fi
% Some text for chapter 2:
%\iffalse
%<*samplechap2>
%\fi
%    \begin{macrocode}
\section{two}
more text in chapter two
%    \end{macrocode}

%\iffalse
%</samplechap2>
%\fi
%
% %%%%%%%%%%%%%%%%%%%%%%%%%%%%%%%%%%%%%%
% \paragraph{Part Include Files.}
%
% The include files are called |cdocspt3.tex| and |cdocspt4.tex|.
%
%\iffalse
%<*samplepart3|samplepart4>
%\fi

% Optional override for |\version| flag:
%    \begin{macrocode}
%%\providecommand{\version}{final}
%    \end{macrocode}

% Include the main document:
%    \begin{macrocode}
\input{childdoc.def}
\childdocby{cdocsamp}
%    \end{macrocode}

%\iffalse
%</samplepart3|samplepart4>
%\fi
%
%\iffalse
%<*samplepart3>
%\fi
% Some text for part 3:
%    \begin{macrocode}
some text in part three
%    \end{macrocode}

%\iffalse
%</samplepart3>
%\fi
% Some text for part 4:
%\iffalse
%<*samplepart4>
%\fi
%    \begin{macrocode}
more text in part four
%    \end{macrocode}

%\iffalse
%</samplepart4>
%\fi
%
% %%%%%%%%%%%%%%%%%%%%%%%%%%%%%%%%%%%%%%
% \paragraph{Forwarding for a Complete Draft.}
%
% The following forwarding file |cdocsdrf.tex|
% compiles the main document in draft mode:
%\iffalse
%<*sampledraft>
%\fi
%    \begin{macrocode}
\def\version{draft}
\input{childdoc.def}
\childdocforward{cdocsamp}
%    \end{macrocode}

%\iffalse
%</sampledraft>
%\fi
%
% %%%%%%%%%%%%%%%%%%%%%%%%%%%%%%%%%%%%%%
% \paragraph{Forwarding for Final Version of the Chapters.}
%
% The following forwarding files |cdocsfn1.tex| and |cdocsfn2.tex|
% (with identical content)
% compile the final versions of the child documents
% |cdocsch1.tex| and |cdocsch2.tex|, respectively:
%\iffalse
%<*samplefinal>
%\fi
%    \begin{macrocode}
\def\version{final}
\input{childdoc.def}
\childdocforwardprefix[cdocsamp]{cdocsfn}{cdocsch}
%    \end{macrocode}

%\iffalse
%</samplefinal>
%\fi
%
% %%%%%%%%%%%%%%%%%%%%%%%%%%%%%%%%%%%%%%
% \paragraph{Command Line Processing.}
%
% The following three command lines generate the output files
% |cdocscld|, |cdocscl1| and |cdocscl2|
% which should be identical to
% |cdocsdrf|, |cdocsch1| and |cdocsfn2|, respectively:
% \begin{center}
% \begin{tabular}{l}
% |latex -jobname cdocscld \|\\
% |  "\def\version{draft}\input{childdoc.def}\childdocforward{cdocsamp}"|\\
% |latex -jobname cdocscl1 \|\\
% |  "\input{childdoc.def}\childdocforward[cdocsamp]{cdocsch1}"|\\
% |latex -jobname cdocscl2 \|\\
% |  "\def\version{final}\input{childdoc.def}\childdocforward{cdocsch2}"|
% \end{tabular}
% \end{center}
% Note that the trailing backslash on each first line
% merely continues the input to the second line
% (for convenient cut ant paste).
% Furthermore, the command |latex| can be replaced by any
% of its alternative versions such as |pdflatex|.
%
% %%%%%%%%%%%%%%%%%%%%%%%%%%%%%%%%%%%%%%%%%%%%%%%%%%%%%%%%%%%%%%%%%%%%%%%%%%%%%%
% %%%%%%%%%%%%%%%%%%%%%%%%%%%%%%%%%%%%%%%%%%%%%%%%%%%%%%%%%%%%%%%%%%%%%%%%%%%%%%
% \section{Implementation}
%\iffalse
%<*package>
%\fi
%
% This section describes the definitions file |childdoc.def|.

% The definitions cannot be loaded using |\usepackage| or |\RequirePackage|
% which has a mechanism to prevent loading a style file more than once.
% When loading the definitions by means of |\input|
% multiple instances have to be prevented manually:
%\iffalse
%This code needs to be before the `\ProvidesFile' directive
%which is defined at the beginning of this file.
%Therefore it is also placed there and commented out here.
%</package>
%<*discard>
%\fi
%    \begin{macrocode}
\ifdefined\childdocmain\endinput\fi
%    \end{macrocode}
%\iffalse
%</discard>
%<*package>
%\fi
%
% \macro{\ifchilddoc}
% \macro{\ifchilddocmanual}
% The conditional |\ifchilddoc| tells whether a
% child (true) or main (false) document is being compiled.
% The conditional |\ifchilddocmanual| tells whether
% the |\includeonly| mechanism is used (false) or
% the selection of child files must be performed manually (true).
% The definitions initialise to false:
%    \begin{macrocode}
\newif\ifchilddoc
\newif\ifchilddocmanual
%    \end{macrocode}

% \macro{\childdocname}
% \macro{\childdocjob}
% The macro |\childdocname| stores the name of the main document
% to be compiled. The macro |\childdocjob| stores the name of
% the document on which the \LaTeX{} compiler was originally invoked.
% The content of |\jobname| cannot be compared
% to filenames specified in the source due to different catcodes.
% The following code rescans |\jobname|, stores the result
% in |\childdocname| and saves a copy in |\childdocjob|:
%    \begin{macrocode}
\edef\childdocname{\scantokens\expandafter{\jobname\noexpand}}
\let\childdocjob\childdocname
%    \end{macrocode}

% \macro{\childdocdisable}
% The macro |\childdocdisable| prevents the main file
% from being processed more than once.
% At this stage, the main document command |\childdocmain|
% is assumed to be called once again where it should do nothing.
% Any subsequent call to it should prevent
% a secondary processing of the main document
% It overwrites the forwarding commands
% |\childdocof| and |\childdocforward|
% with empty macros to prevent further inclusions of the main document:
%    \begin{macrocode}
\newcommand{\childdocdisable}
{
  \renewcommand{\childdocmain}[1]{\renewcommand{\childdocmain}[1]{\endinput}}
  \renewcommand{\childdocof}[1]{}
  \renewcommand{\childdocby}[2][]{}
  \renewcommand{\childdocforward}[2][]{}
  \renewcommand{\childdocdisable}{}
}
%    \end{macrocode}

% \macro{\childdocmain}
% The macro |\childdocmain| is to be called at the top of the main file
% with nothing or the main filename (without extension) as argument.
% First, it breaks loops.
% If the argument is not empty and does not match |\childdocname|
% (which is set by the first inclusion of |childdoc.def|),
% |\ifchilddoc| is set to true, |\includeonly| is applied to the child file
% and |\jobname| is set to the main file
% (for proper handling of |.aux| files):
%    \begin{macrocode}
\newcommand{\childdocmain}[1]
{
  \childdocdisable\childdocmain{}
  \if?#1?\else
    \begingroup
      \def\childdoctmp{#1}
      \ifx\childdoctmp\childdocname
        \def\childdoctmp{}
      \else
        \def\childdoctmp
        {
          \childdoctrue
          \includeonly{\childdocname}
          \def\childdocjob{#1}
          \def\jobname{#1}
        }
      \fi
      \expandafter
    \endgroup
    \childdoctmp
  \fi
}
%    \end{macrocode}

% \macro{\childdocof}
% The command |\childdocof| redirects
% compilation to the main file |#1|.
%    \begin{macrocode}
\newcommand{\childdocof}[1]
{
  \childdocdisable
  \childdoctrue
  \includeonly{\childdocname}
  \def\jobname{#1}
  \def\childdocjob{#1}
  \input{#1}
}
%    \end{macrocode}

% \macro{\childdocby}
% The command |\childdocby| ....
%    \begin{macrocode}
\newcommand{\childdocby}[2][]
{
  \childdocdisable
  \childdoctrue
  \childdocmanualtrue
  \if?#1?\else
    \def\jobname{#2}
  \fi
  \def\childdocjob{#2}
  \input{#2}
  \endinput
}
%    \end{macrocode}

% \macro{\childdocforward}
% The command |\childdocforward| redirects
% compilation to the main file or
% (if the optional argument is given) a child file.
% Parameters are set as if the main file
% or a child file starting with |\childdocof| was compiled.
% Then compilation is handed over to the main file:
%    \begin{macrocode}
\newcommand{\childdocforward}[2][]
{
  \begingroup
    \if?#1?
      \def\childdoctmp
      {
        \def\childdocname{#2}
        \def\childdocjob{#2}
        \def\jobname{#2}
        \input{#2}
        \endinput
      }
    \else
      \def\childdoctmp
      {
        \childdocdisable
        \def\childdocname{#2}
        \childdoctrue
        \includeonly{#2}
        \def\childdocjob{#1}
        \def\jobname{#1}
        \input{#1}
        \endinput
      }
    \fi
    \expandafter
  \endgroup
  \childdoctmp
}
%    \end{macrocode}

% \macro{\childdocforwardprefix}
% The command |\childdocforwardprefix| redirects
% compilation to the main or a child file by means of a pattern.
% The prefix |#1| in the current filename is replaced by |#2|
% and the suffix of the current filename is kept
% (it is assumed that the filename does not contain the substring `|~~~|'
% which is used as a delimiter).
% Compilation is handed over to the new file by |\childdocforward|:
%    \begin{macrocode}
\newcommand{\childdocforwardprefix}[3][]
{
  \begingroup
    \def\childdocextract #2##1~~~{\def\childdoctmp{\childdocforward[#1]{#3##1}}}
    \expandafter\childdocextract\childdocname~~~
    \expandafter
  \endgroup
  \childdoctmp
}
%    \end{macrocode}

% \macro{\childdoc}
% The deprecated macro |\childdoc| is a legacy version of |\childdocmain|:
%    \begin{macrocode}
\newcommand{\childdoc}{\childdocmain}
%    \end{macrocode}

% \macro{\childdocredirect}
% The deprecated macro |\childdocredirect| is a legacy version
% of |\childdocforward| and |\childdocforwardprefix|:
%    \begin{macrocode}
\newcommand{\childdocredirect}[2][]
{
  \begingroup
    \if?#1?
      \def\childdoctmp{\childdocforward{#2}}
    \else
      \def\childdoctmp{\childdocforwardprefix{#1}{#2}}
    \fi
    \expandafter
  \endgroup
  \childdoctmp
}
%    \end{macrocode}

%\iffalse
%</package>
%\fi
%
\endinput
|\\
|\childdocmain{}|\\
\end{tabular}
\end{center}
at the very top of the main \LaTeX{} file,
in particular \emph{before} the |\documentclass| statement!
The argument of |\childdocmain| should be left empty
(but it must be present).

%%%%%%%%%%%%%%%%%%%%%%%%%%%%%%%%%%%%%%%%
\DescribeMacro{\childdocof}
Furthermore, add the commands
\begin{center}
\begin{tabular}{l}
|% \iffalse
%
% childdoc.dtx Copyright (C) 2017-2018 Niklas Beisert
%
% This work may be distributed and/or modified under the
% conditions of the LaTeX Project Public License, either version 1.3
% of this license or (at your option) any later version.
% The latest version of this license is in
%   http://www.latex-project.org/lppl.txt
% and version 1.3 or later is part of all distributions of LaTeX
% version 2005/12/01 or later.
%
% This work has the LPPL maintenance status `maintained'.
%
% The Current Maintainer of this work is Niklas Beisert.
%
% This work consists of the files childdoc.dtx and childdoc.ins
% and the derived files childdoc.def and cdocsamp.tex with
% cdocsch1.tex, cdocsch2.tex, cdocsdrf.tex, cdocsfn1.tex, cdocsfn2.tex.
%
%<package>\ifdefined\childdocmain\endinput\fi
%<package>\ProvidesFile{childdoc.def}[2018/12/30 v2.0 child document driver]
%<samplemain>\ProvidesFile{cdocsamp.tex}[2018/12/30 v2.0 sample for childdoc]
%<*driver>
%\ProvidesFile{childdoc.drv}[2018/12/30 v2.0 childdoc reference manual file]
\PassOptionsToClass{10pt,a4paper}{article}
\documentclass{ltxdoc}

\usepackage[margin=35mm]{geometry}
\usepackage{hyperref}
\usepackage{hyperxmp}
\usepackage[usenames]{color}

\hypersetup{colorlinks=true}
\hypersetup{pdfstartview=FitH}
\hypersetup{pdfpagemode=UseNone}
\hypersetup{pdfsource={}}
\hypersetup{pdflang={en-UK}}
\hypersetup{pdfcopyright={Copyright 2017-2018 Niklas Beisert.
  This work may be distributed and/or modified under the
  conditions of the LaTeX Project Public License, either version 1.3
  of this license or (at your option) any later version.}}
\hypersetup{pdflicenseurl={http://www.latex-project.org/lppl.txt}}
\hypersetup{pdfcontactaddress={ETH Zurich, ITP, HIT K,
  Wolfgang-Pauli-Strasse 27}}
\hypersetup{pdfcontactpostcode={8093}}
\hypersetup{pdfcontactcity={Zurich}}
\hypersetup{pdfcontactcountry={Switzerland}}
\hypersetup{pdfcontactemail={nbeisert@itp.phys.ethz.ch}}
\hypersetup{pdfcontacturl={http://people.phys.ethz.ch/\xmptilde nbeisert/}}

\newcommand{\secref}[1]{\hyperref[#1]{section \ref*{#1}}}

\parskip1ex
\parindent0pt
\let\olditemize\itemize
\def\itemize{\olditemize\parskip0pt}

\begin{document}

\title{The \textsf{childdoc} Package}
\hypersetup{pdftitle={The childdoc Package}}
\author{Niklas Beisert\\[2ex]
  Institut f\"ur Theoretische Physik\\
  Eidgen\"ossische Technische Hochschule Z\"urich\\
  Wolfgang-Pauli-Strasse 27, 8093 Z\"urich, Switzerland\\[1ex]
  \href{mailto:nbeisert@itp.phys.ethz.ch}
  {\texttt{nbeisert@itp.phys.ethz.ch}}}
\hypersetup{pdfauthor={Niklas Beisert}}
\hypersetup{pdfsubject={Manual for the LaTeX2e Package childdoc}}
\date{30 December 2018, \textsf{v2.0}}
\maketitle

\begin{abstract}\noindent
\textsf{childdoc} is a \LaTeXe{} package
that enables the direct compilation
of document sections included by |\include|
to individual files.
\end{abstract}

\begingroup
\parskip0ex
\tableofcontents
\endgroup

%%%%%%%%%%%%%%%%%%%%%%%%%%%%%%%%%%%%%%%%%%%%%%%%%%%%%%%%%%%%%%%%%%%%%%%%%%%%%%%%
%%%%%%%%%%%%%%%%%%%%%%%%%%%%%%%%%%%%%%%%%%%%%%%%%%%%%%%%%%%%%%%%%%%%%%%%%%%%%%%%
\section{Introduction}

\LaTeX{} provides a mechanism to structure a large document (such as a book)
into a main file and several child files (containing the chapters)
using the |\include| command.
This mechanism is beneficial for documents
which span hundreds of pages in order to
make the source file(s) more manageable.
Moreover, compilation can be restricted to
selected child files by means of the |\includeonly| command.
The latter feature can be used to reduce the compilation time while editing
(this was significantly more useful in the earlier days of \LaTeX{})
or to generate a smaller document which is easier to navigate.
Another application of |\includeonly| is to generate
documents consisting of selected parts of the complete document.

However, there are a few drawbacks of the plain |\include| mechanism:
\begin{itemize}
\item
The child files cannot be compiled on their own,
they can only be compiled via the main file.
A naive editing environment
(such as a text editor with an option
to have the current file processed by \LaTeX)
may require one to switch to the main file before compiling;
attempting to compile the child file produces errors.
\item
The main file must be modified (each time)
to adjust the |\includeonly| command
to the present needs. This easily leaves the main file in a messy state.
\item
The generated document will always carry the filename
of the main document. This is inconvenient if
several child files are to be compiled and
to be kept for distribution.
\end{itemize}

The present package provides a simple interface
to make child files individually compilable by \LaTeX{}.
Compiling a child file then has the same effect as compiling
the main file with an |\includeonly| command
to select the appropriate child.
Moreover the generated document will carry the name of the child
rather than the main file.
This resolves all three above issues.

This feature is meant to make the editing of books,
thesis documents and lecture notes somewhat more convenient.
However, the package can also be used efficiently for
composing a series of documents (such as exercise sheets)
which are typically distributed individually.
It then assists the author in generating the individual documents
(potentially in different versions)
as well as a document containing the collected series.
Another application is in developing style files
or other kinds of included material
where compilation of the style file could redirect
to a sample or test file.

%%%%%%%%%%%%%%%%%%%%%%%%%%%%%%%%%%%%%%%%%%%%%%%%%%%%%%%%%%%%%%%%%%%%%%%%%%%%%%%%
%%%%%%%%%%%%%%%%%%%%%%%%%%%%%%%%%%%%%%%%%%%%%%%%%%%%%%%%%%%%%%%%%%%%%%%%%%%%%%%%
\section{Usage}

First of all, the package \textsf{childdoc} is \emph{not} a standard
\LaTeXe{} |.sty| style file! Therefore it needs to be invoked in
a non-standard way.

%%%%%%%%%%%%%%%%%%%%%%%%%%%%%%%%%%%%%%%%%%%%%%%%%%%%%%%%%%%%%%%%%%%%%%%%%%%%%%%%
\subsection{Included Files}
\label{sec:include}

%%%%%%%%%%%%%%%%%%%%%%%%%%%%%%%%%%%%%%%%
\DescribeMacro{\childdocmain}
To use the package, add the commands
\begin{center}
\begin{tabular}{l}
|\input{childdoc.def}|\\
|\childdocmain{}|\\
\end{tabular}
\end{center}
at the very top of the main \LaTeX{} file,
in particular \emph{before} the |\documentclass| statement!
The argument of |\childdocmain| should be left empty
(but it must be present).

%%%%%%%%%%%%%%%%%%%%%%%%%%%%%%%%%%%%%%%%
\DescribeMacro{\childdocof}
Furthermore, add the commands
\begin{center}
\begin{tabular}{l}
|\input{childdoc.def}|\\
|\childdocof{|\textit{main}|}|\\
\end{tabular}
\end{center}
at the top of every child file \textit{child}
which is included by |\include{|\textit{child}|}|
from within the main file
(or at least for those files to be compiled individually).
The argument \textit{main} must be the filename of the main file.

There are a couple of
considerations in setting up the main and child documents:

%%%%%%%%%%%%%%%%%%%%%%%%%%%%%%%%%%%%%%%%
\paragraph{Restrictions.}

Please note the following restrictions:
\begin{itemize}
\item
|\childdocmain| must be called with one argument \textit{main}
to ensure compatibility with earlier version of the package.
It must either be empty (|\childdocmain{}|)
or precisely match the filename of the main file in which it is specified.
See \secref{sec:detection} for further information.
\item
The filename \textit{main} must be specified without the |.tex| extension.
\item
The filename \textit{main} is case sensitive
(even in case-insensitive file systems)
due to internal string comparison.
\item
The argument \textit{main} should be fully expanded, it cannot be a macro.
\item
Subdirectories and special characters should be avoided in filenames.
\item
The command |\childdocmain{|\textit{main}|}| must be followed by a whitespace.
It should not be followed immediately by another command
or by a comment mark `|%|'.
This is because the \TeX{} parser reads the token immediately following
the argument of |\childdocmain| and puts it
at the beginning of every child section;
however, a white\-space is ignored.
\end{itemize}

%%%%%%%%%%%%%%%%%%%%%%%%%%%%%%%%%%%%%%%%
\paragraph{Content of Main File.}

It is advisable to place all content in the child files included by |\include|.
Any output contained in the main file will appear in all child documents
unless suppressed manually;
it cannot be suppressed automatically by the |\includeonly| directive
and thus should normally be avoided.
A method to include some content in the main file
by means of conditional processing is described in \secref{sec:conditional}.

%%%%%%%%%%%%%%%%%%%%%%%%%%%%%%%%%%%%%%%%
\paragraph{Page Numbering.}

When only a part of the document is compiled,
the appropriate numbering of pages
(as well as other status parameters)
is determined from the |.aux| files.
The latter contain information from previous passes.
However this information needs to propagate through
all intermediate child documents.
Therefore the page numbering in child documents may well
be inconsistent until the complete document is compiled at least once.

A useful (if unconventional) way to always ensure a consistent
page numbering is to restart the numbering in each child document
and denote the pages by `\textit{child}|.|\textit{page}'
where \textit{child} represents the chapter/section number of the child file.
This can be achieved by the command
|\numberwithin{page}{|\textit{child}|}|
of the \textsf{amsmath} package
where \textit{child} can be |chapter| or |section|
depending on the chosen structuring.
Alternatively, one can modify the macro |\thepage| appropriately
and reset the counter |page| at the start of each child file.

%%%%%%%%%%%%%%%%%%%%%%%%%%%%%%%%%%%%%%%%%%%%%%%%%%%%%%%%%%%%%%%%%%%%%%%%%%%%%%%%
\subsection{Conditional Processing}
\label{sec:conditional}

The package provides a mechanism to compile different versions
of a document. To customise the versions further some conditional processing
can come in handy to distinguish which version is being compiled.
The package provides two macros to describe the compilation context:

%%%%%%%%%%%%%%%%%%%%%%%%%%%%%%%%%%%%%%%%
\DescribeMacro{\ifchilddoc}
The conditional |\ifchilddoc| distinguishes between the compilation of
child documents and the main document:
%
\begin{center}
|\ifchilddoc |\textit{child-code}| |[|\||else |\textit{main-code}]| \||fi|
\end{center}

%%%%%%%%%%%%%%%%%%%%%%%%%%%%%%%%%%%%%%%%
\DescribeMacro{\childdocname}
\DescribeMacro{\childdocjob}
The macro |\childdocname| contains the filename (without extension)
of the main or child file being processed.
Note that |\childdocjob| will always contain the name of the main file.

%%%%%%%%%%%%%%%%%%%%%%%%%%%%%%%%%%%%%%%%
\paragraph{Title Page.}

Conditional processing can be used to include a title or banner page
in the main document when proper precautions are taken.
Importantly, the code in the main file should ensure that the page counter
(as well as other status parameters which are stored in the |.aux| files)
takes the same value after the conditional processing.
Otherwise the page numbers may take divergent values
depending on which part is compiled.

For example, a title page could be declared by:
%
\begin{center}
\begin{tabular}{l}
|\ifchilddoc\||else|\\
|\addtocounter{page}{-1}|\\
\textit{code for title page}\\
|\newpage|\\
|\||fi|
\end{tabular}
\end{center}
%
A banner page for the child documents can be generated by:
%
\begin{center}
\begin{tabular}{l}
|\ifchilddoc|\\
|\addtocounter{page}{-1}|\\
\textit{code for banner page}\\
|\newpage|\\
|\||fi|
\end{tabular}
\end{center}
%
Here one could write a message such as:
\begin{center}
|This is the part \childdocname{} of \childdocjob{}.|
\end{center}

%%%%%%%%%%%%%%%%%%%%%%%%%%%%%%%%%%%%%%%%%%%%%%%%%%%%%%%%%%%%%%%%%%%%%%%%%%%%%%%%
\subsection{Flags}
\label{sec:flags}

The package makes it easy to generate different versions
of the main or child documents.
To this end compilation flags can be defined
and assigned different default values.
They will be particularly useful in conjunction
with the forwarding mechanism described in \secref{sec:forward}.

For example, it may be useful to have a flag |\version|
which can be set to |draft| or |final|.
The document source will contain some conditional code
depending on the value of |\version|.
Suppose further, the flag should default to |final| for the main file
and to |draft| for child files
which is a natural assignment for editing the document.
This is achieved by placing the following code
in the preamble of the main document
(below the |\childdocmain| directive):
%
\begin{center}
\begin{tabular}{l}
|\ifchilddoc|\\
|\providecommand{\version}{draft}|\\
|\||else|\\
|\providecommand{\version}{final}|\\
|\||fi|
\end{tabular}
\end{center}
%
The definition by |\providecommand| makes sure
that previous definitions are not overwritten.
Further statements |\providecommand{\version}{...}|
can thus be added before the above code to override it.

For the main file, one might add a line
(between |\childdocmain| and the above block)
%
\begin{center}
|%\ifchilddoc\||else\providecommand{\version}{draft}\||fi|
\end{center}
%
which can be uncommented to produce a draft version.
Likewise one can add a line to the very top of a child file
(above the |\childdocof{|\textit{main}|}| directive)
%
\begin{center}
|%\providecommand{\version}{final}|
\end{center}
%
which can be uncommented to produce the final version of this child document.

%%%%%%%%%%%%%%%%%%%%%%%%%%%%%%%%%%%%%%%%%%%%%%%%%%%%%%%%%%%%%%%%%%%%%%%%%%%%%%%%
\subsection{Forwarding}
\label{sec:forward}

Different versions of the main or child documents
using compilation flags as described in \secref{sec:flags}
can be (permanently) stored in different files
for convenient compilation, viewing and distribution.
To this end, the package defines a command
to pass on compilation to a different file:

%%%%%%%%%%%%%%%%%%%%%%%%%%%%%%%%%%%%%%%%
\DescribeMacro{\childdocforward}
The command |\childdocforward| redirects processing to
another source file:
%
\begin{center}
\begin{tabular}{l}
|\input{childdoc.def}|\\
|\childdocforward[|\textit{main}|]{|\textit{dest}|}|\\
\end{tabular}
\end{center}
%
The argument \textit{dest} is the destination file
(without extension).
It should be the main file or one of the child files.
Note that further \textsf{childdoc} directives
such as |\childdocof| and |\childdocforward|
in the indicated file will be processed in this form.
The optional argument \textit{main}
passes on directly to the main file \textit{main}
while pretending to compile the child \textit{dest}.
This form behaves as if \textit{dest}
issues |\childdocof{|\textit{main}|}| right away,
and no further \textsf{childdoc} directives will be processed.

%%%%%%%%%%%%%%%%%%%%%%%%%%%%%%%%%%%%%%%%
\DescribeMacro{\...prefix}
In the alternative form |\childdocforwardprefix|,
%
\begin{center}
\begin{tabular}{l}
|\input{childdoc.def}|\\
|\childdocforwardprefix[|\textit{main}|]{|\textit{prefix}|}{|\textit{dest}|}|
\end{tabular}
\end{center}
%
the destination file is determined by a pattern
depending on the current file:
To make this work, the current file must be called
`{\textit{prefix}\hspace{0.2em}\textit{suffix}}'
with \textit{prefix} matching precisely the argument.
Processing is then passed on to the file
`{\textit{dest}\hspace{0.2em}\textit{suffix}}'.
Surely, the same effect is achieved by
directly specifying the
argument `{\textit{dest}\hspace{0.2em}\textit{suffix}}'
in the first form.
However, that requires to set up a different file
for each child. With the alternative form of the command
all these files can have exactly the same content
which simplifies setting them up and maintaining them.

For example, the following file |draft.tex|
with a compilation flag |\version| as described in \secref{sec:flags}
compiles the main document as a draft:
%
\begin{center}
\begin{tabular}{l}
|\def\version{draft}|\\
|\input{childdoc.def}|\\
|\childdocforward{|\textit{main}|}|
\end{tabular}
\end{center}
%
Likewise, the following files |final|\textit{nn}|.tex|
compile the final version of the child document
|child|\textit{nn}|.tex|:
%
\begin{center}
\begin{tabular}{l}
|\def\version{final}|\\
|\input{childdoc.def}|\\
|\childdocforwardprefix{final}{child}|
\end{tabular}
\end{center}
%

Note that when several versions of a main file and/or of each child file
are to be generated, it may be convenient to set up a |Makefile| or
shell script to automatise the process.

%%%%%%%%%%%%%%%%%%%%%%%%%%%%%%%%%%%%%%%%%%%%%%%%%%%%%%%%%%%%%%%%%%%%%%%%%%%%%%%%
\subsection{Command Line Processing}
\label{sec:commandline}

The effect of redirection files can also be achieved by invoking
the \LaTeX{} compiler with a more elaborate command line.
Most conveniently this should be done as part
of a shell script or a |Makefile|.

When using \textsf{childdoc} in the main file, the following
command lines effectively perform a redirection
(note that depending on the shell being used,
backslashes may have to be doubled: `|\|' $\to$ `|\\|'):
%
\begin{center}
|... -jobname "|\textit{target}|" |\\|"|[\textit{flags}]%
|\input{childdoc.def}\childdocforward[|\textit{main}|]{|\textit{dest}|}"|
\end{center}
%
Here \textit{target} is the name of the output file,
\textit{main} is the name of the main file
and \textit{dest} is the name of the main or child file to be processed
(all filenames without extensions).
The optional argument \textit{main} can be omitted
if \textit{main} matches \textit{dest}.
Optionally, compilation \textit{flags} can be defined via |\def| commands.
This command line makes the \TeX{} engine believe
it is compiling the file \textit{target}
whose content is specified as the latter parameter.
The provided code then forwards the processing to
\textit{main} or \textit{dest} as described in \secref{sec:forward}.

%%%%%%%%%%%%%%%%%%%%%%%%%%%%%%%%%%%%%%%%%%%%%%%%%%%%%%%%%%%%%%%%%%%%%%%%%%%%%%%%
\subsection{Include by Input}
\label{sec:input}

Including child documents by |\include| has some restrictions by design.
Most notably, the content of a child document always occupies
its own set of pages; pages cannot be shared between child documents.
Usually, this behaviour makes perfect sense
because each child document contain an essential part of the document.
However, in some situations it may be desirable to compose
a document from a collection of parts
without having mandatory page breaks between then.
For this case, the package
provides a mechanism to include parts
by |\input| which can also be processed individually.
However, by construction this mechanism
requires manual handling of the content to be output.

%%%%%%%%%%%%%%%%%%%%%%%%%%%%%%%%%%%%%%%%
\DescribeMacro{\ifchilddocmanual}
The main file should be prepared as usual, see \secref{sec:include}.
However, the document body must make a distinction
between processing of an individual part and of the main document, e.g.:
%
\begin{center}
\begin{tabular}{l}
|\ifchilddocmanual|\\
|\input{\childdocname}|\\
|\||else|\\
\textit{document body with }|\input{|\textit{part}|}|\\
|\||fi|
\end{tabular}
\end{center}
%
The conditional |\ifchilddocmanual| is true whenever
a part to be included by |\input| is being compiled,
and the name of the part is stored in |\childdocname|.

%%%%%%%%%%%%%%%%%%%%%%%%%%%%%%%%%%%%%%%%
\DescribeMacro{\childdocby}
Each part to be included by |\input| should start with:
%
\begin{center}
\begin{tabular}{l}
|\input{childdoc.def}|\\
|\childdocby{|\textit{main}|}|\\
\end{tabular}
\end{center}
%
The directive |\childdocby| is similar to |\childdocof|
described in \secref{sec:include},
but the subsequent selection of content must be done manually.
To that end, both |\ifchilddoc| and |\ifchilddocmanual|
will be true upon processing of a part,
and the name of the part is stored in |\childdocname|.
Note that |\jobname| will be set to the filename of the current part
so that each part receives an individual |.aux| file
that does not interfere with the |.aux| file(s) of the main document.
This behaviour can be altered by the alternative form
|\childdocby[*]{|\textit{main}|}| (with a non-empty optional argument)
which uses the |.aux| file of the main document
by setting |\jobname| to \textit{main}.

%%%%%%%%%%%%%%%%%%%%%%%%%%%%%%%%%%%%%%%%%%%%%%%%%%%%%%%%%%%%%%%%%%%%%%%%%%%%%%%%
\subsection{Driver Development}
\label{sec:driver}

The \textsf{childdoc} mechanism can also be use for the development
of definition files such as \LaTeX{} styles or classes.
This case differs from the above setup with multiple parts
included by |\include| in that no |\includeonly| should be invoked.
This can be achieved by starting the include file
(before |\ProvidesPackage|) with:
%
\begin{center}
\begin{tabular}{l}
|\input{childdoc.def}|\\
|\childdocforward{|\textit{main}|}|\\
\end{tabular}
\end{center}
%
or alternatively with:
%
\begin{center}
\begin{tabular}{l}
|\input{childdoc.def}|\\
|\childdocby{|\textit{main}|}|\\
\end{tabular}
\end{center}
%
Both forms have slightly different effects as described above.
The main file is prepared as usual, see \secref{sec:include}.

%%%%%%%%%%%%%%%%%%%%%%%%%%%%%%%%%%%%%%%%%%%%%%%%%%%%%%%%%%%%%%%%%%%%%%%%%%%%%%%%
\subsection{Legacy Detection}
\label{sec:detection}

The directive |\childdocmain| in the main file can detect
whether the complete document or merely a child is to be compiled
even without using the directive |\childdocof|.
This method is deprecated because it is less robust
and there is no compelling reason to use it;
it is merely provided for backward compatibility
and it may be removed in future versions.

If the detection mechanism is to be used,
it is mandatory to correctly specify
the filename of the main file as the argument of |\childdocmain|:
%
\begin{center}
\begin{tabular}{l}
|\input{childdoc.def}|\\
|\childdocmain{|\textit{main}|}|\\
\end{tabular}
\end{center}
%
If |\jobname| does not match the argument \textit{main} of |\childdocmain|,
it is assumed that |\jobname| points to the child file to be compiled.
When using |\childdocmain| with the main file specified as argument,
it suffices to start a child file
with just |\input{|\textit{main}|}|
without loading of the package and using |\childdocof|.
If instead all processing is done
with the appropriate \textsf{childdoc} directives,
the argument of \textit{main} of |\childdocmain| can be empty.

An alternative version of the command line processing described
in \secref{sec:commandline} using the detection mechanism reads:
%
\begin{center}
|... -jobname "|\textit{target}|" "|[\textit{flags}]%
[|\def\jobname{|\textit{dest}|}|]|\input{|\textit{main}|}"|
\end{center}

%%%%%%%%%%%%%%%%%%%%%%%%%%%%%%%%%%%%%%%%%%%%%%%%%%%%%%%%%%%%%%%%%%%%%%%%%%%%%%%%
\subsection{Manual Code}
\label{sec:manual}

In case one cannot be certain whether the definitions file |childdoc.def|
is installed on the target \TeX{} distribution
and one prefers not to ship it,
it is conceivable to paste a few relevant commands into the sources.

To that end, drop all statements |\input{childdoc.def}|
and perform the replacements as outlined below.
Instead of |\childdocmain{|\textit{main}|}| add the following code
to the top of the main file:
%
\begin{center}
\begin{tabular}{l}
|\||ifdefined\childdocname\endinput\||fi\newif\ifchilddoc|\\
|\edef\childdocname{\scantokens\expandafter{\jobname\noexpand}}|\\
|\def\childdocmain{|\textit{main}|}\||ifx\childdocmain\childdocname\||else|\\
|\childdoctrue\includeonly{\childdocname}\let\jobname\childdocmain\||fi|\\
\end{tabular}
\end{center}
%
Instead of |\childdocof{|\textit{main}|}| just include the main file
at the top of each child file:
%
\begin{center}
|\input{|\textit{main}|}|
\end{center}
%
A simple redirection |\childdocforward{|\textit{dest}|}| is achieved by:
%
\begin{center}
|\def\jobname{|\textit{dest}|}\input{\jobname}|
\end{center}
%
The redirection with prefix
|\childdocforwardprefix[|\textit{prefix}|]{|\textit{dest}|}|
is accomplished by:
%
\begin{center}
\begin{tabular}{l}
|{\edef\jobname{\scantokens\expandafter{\jobname\noexpand}}|\\
|\def\redirectjob |\textit{prefix}|#1~~~{\gdef\jobname{|\textit{dest}|#1}}|\\
|\expandafter\redirectjob\jobname~~~}\input{\jobname}|
\end{tabular}
\end{center}

In an alternative approach,
child documents can be compiled by a specific command line
without additional code or specific definitions:
%
\begin{center}
|... -jobname "|\textit{target}|" "|[\textit{flags}]%
|\includeonly{|\textit{dest}|}\input{|\textit{main}|}"|
\end{center}
%

%%%%%%%%%%%%%%%%%%%%%%%%%%%%%%%%%%%%%%%%%%%%%%%%%%%%%%%%%%%%%%%%%%%%%%%%%%%%%%%%
%%%%%%%%%%%%%%%%%%%%%%%%%%%%%%%%%%%%%%%%%%%%%%%%%%%%%%%%%%%%%%%%%%%%%%%%%%%%%%%%
\section{Information}

%%%%%%%%%%%%%%%%%%%%%%%%%%%%%%%%%%%%%%%%%%%%%%%%%%%%%%%%%%%%%%%%%%%%%%%%%%%%%%%%
\subsection{Copyright}

Copyright \copyright{} 2017--2018 Niklas Beisert

This work may be distributed and/or modified under the
conditions of the \LaTeX{} Project Public License, either version 1.3
of this license or (at your option) any later version.
The latest version of this license is in
  \url{http://www.latex-project.org/lppl.txt}
and version 1.3 or later is part of all distributions of \LaTeX{}
version 2005/12/01 or later.

This work has the LPPL maintenance status `maintained'.

The Current Maintainer of this work is Niklas Beisert.

This work consists of the files |README.txt|, |childdoc.ins| and |childdoc.dtx|
as well as the derived files |childdoc.def|, |cdocsamp.tex|
with |cdocsch1.tex|, |cdocsch2.tex|, |cdocspt3.tex|, |cdocspt4.tex|,
|cdocsdrf.tex|, |cdocsfn1.tex|, |cdocsfn2.tex|
as well as |childdoc.pdf|.

%%%%%%%%%%%%%%%%%%%%%%%%%%%%%%%%%%%%%%%%%%%%%%%%%%%%%%%%%%%%%%%%%%%%%%%%%%%%%%%%
\subsection{Files and Installation}

The package consists of the files:
%
\begin{center}
\begin{tabular}{ll}
    |README.txt|   & readme file \\
    |childdoc.ins| & installation file \\
    |childdoc.dtx| & source file \\
    |childdoc.def| & definition file \\
    |cdocsamp.tex| & sample main file \\
    |cdocsch1.tex| & sample include file \\
    |cdocsch2.tex| & sample include file \\
    |cdocspt3.tex| & sample part file \\
    |cdocspt4.tex| & sample part file \\
    |cdocsdrf.tex| & sample redirection file \\
    |cdocsfn1.tex| & sample redirection file \\
    |cdocsfn2.tex| & sample redirection file \\
    |childdoc.pdf| & manual
\end{tabular}
\end{center}
%
The distribution consists of the files
|README.txt|, |childdoc.ins| and |childdoc.dtx|.
%
\begin{itemize}
\item
Run (pdf)\LaTeX{} on |childdoc.dtx|
to compile the manual |childdoc.pdf| (this file).
\item
Run \LaTeX{} on |childdoc.ins| to create the definitions file |childdoc.def|
and the sample |cdocsamp.tex| with include files
|cdocsch1.tex|, |cdocsch2.tex|, |cdocspt3.tex|, |cdocspt4.tex|,
|cdocsdrf.tex|, |cdocsfn1.tex|, |cdocsfn2.tex|.
Then copy the file |childdoc.def| to an appropriate directory of your \LaTeX{}
distribution, e.g.\ \textit{texmf-root}|/tex/latex/childdoc|.
\end{itemize}

%%%%%%%%%%%%%%%%%%%%%%%%%%%%%%%%%%%%%%%%%%%%%%%%%%%%%%%%%%%%%%%%%%%%%%%%%%%%%%%%
\subsection{Related CTAN Packages}

There are several other packages which offer a similar functionality:
%
\begin{itemize}
\item
The packages
\href{http://ctan.org/pkg/docmute}{\textsf{docmute}},
\href{http://ctan.org/pkg/includex}{\textsf{includex}} and
\href{http://ctan.org/pkg/standalone}{\textsf{standalone}}
provide commands to include only the document body of
a child file thus allowing both files to be compiled individually.
\item
The packages \href{http://ctan.org/pkg/subdocs}{\textsf{subdocs}}
and \href{http://ctan.org/pkg/subfiles}{\textsf{subfiles}}
provide structures in which the main and child documents can be
encapsulated and allowing them to be compiled individually.
The inclusion mechanism is different from the conventional |\include|.
\item
The package \href{http://ctan.org/pkg/combine}{\textsf{combine}}
is an elaborate solution to combine several documents into one.
\end{itemize}
%
See also the CTAN topic \href{http://ctan.org/topic/subdocs}{\textsf{subdocs}}
for further related packages.
The present package differs from the above solutions in that
a document structure constructed with the conventional |\include| mechanism
just needs two extra commands at the top of every file
such that all constituent files can be compiled individually.

%%%%%%%%%%%%%%%%%%%%%%%%%%%%%%%%%%%%%%%%%%%%%%%%%%%%%%%%%%%%%%%%%%%%%%%%%%%%%%%%
%\subsection{Feature Suggestions}
%
%The following is a list of features which may be useful for future
%versions of this package:
%%
%\begin{itemize}
%\item
%\ldots
%\end{itemize}

%%%%%%%%%%%%%%%%%%%%%%%%%%%%%%%%%%%%%%%%%%%%%%%%%%%%%%%%%%%%%%%%%%%%%%%%%%%%%%%%
\subsection{Revision History}

%%%%%%%%%%%%%%%%%%%%%%%%%%%%%%%%%%%%%%%%
\paragraph{v2.0:} 2018/12/30

\begin{itemize}
\item
immediate forward processing
\item
added |\childdocby| mechanism
\item
manual restructured
\end{itemize}

%%%%%%%%%%%%%%%%%%%%%%%%%%%%%%%%%%%%%%%%
\paragraph{v1.6:} 2018/01/17

\begin{itemize}
\item
application for development of include files
\item
corrections to manual
\end{itemize}

%%%%%%%%%%%%%%%%%%%%%%%%%%%%%%%%%%%%%%%%
\paragraph{v1.5:} 2017/05/21

\begin{itemize}
\item
more complete structuring introduced
\item
|\childdocof| introduced
\item
|\childdoc| renamed to |\childdocmain|
\item
|\childredirect| renamed to |\childdocforward| and |\childdocforwardprefix|
and functionality expanded
\end{itemize}

%%%%%%%%%%%%%%%%%%%%%%%%%%%%%%%%%%%%%%%%
\paragraph{v1.0:} 2017/04/27

\begin{itemize}
\item
manual and install package
\item
first version published on CTAN
\end{itemize}

%%%%%%%%%%%%%%%%%%%%%%%%%%%%%%%%%%%%%%%%
\paragraph{v0.6:} 2017/04/26

\begin{itemize}
\item
redirection mechanism added
\end{itemize}

%%%%%%%%%%%%%%%%%%%%%%%%%%%%%%%%%%%%%%%%
\paragraph{v0.5:} 2017/04/26

\begin{itemize}
\item
functionality in definition file
\end{itemize}


%%%%%%%%%%%%%%%%%%%%%%%%%%%%%%%%%%%%%%%%%%%%%%%%%%%%%%%%%%%%%%%%%%%%%%%%%%%%%%%%
%%%%%%%%%%%%%%%%%%%%%%%%%%%%%%%%%%%%%%%%%%%%%%%%%%%%%%%%%%%%%%%%%%%%%%%%%%%%%%%%
%%%%%%%%%%%%%%%%%%%%%%%%%%%%%%%%%%%%%%%%%%%%%%%%%%%%%%%%%%%%%%%%%%%%%%%%%%%%%%%%
\appendix

\settowidth\MacroIndent{\rmfamily\scriptsize 000\ }

 \DocInput{childdoc.dtx}

\end{document}
%</driver>
% \fi
%
% %%%%%%%%%%%%%%%%%%%%%%%%%%%%%%%%%%%%%%%%%%%%%%%%%%%%%%%%%%%%%%%%%%%%%%%%%%%%%%
% %%%%%%%%%%%%%%%%%%%%%%%%%%%%%%%%%%%%%%%%%%%%%%%%%%%%%%%%%%%%%%%%%%%%%%%%%%%%%%
% \section{Sample}
%\iffalse
%<*samplemain>
%\fi
%
% The following presents a sample document
% with two chapters, two parts, a title page,
% a compile flag as well as three forwarding files to set the flag.
% It consists of eight |.tex| files:
% \begin{center}
% \begin{tabular}{ll}
% |cdocsamp.tex|&main file\\
% |cdocsch1.tex|&include file for chapter 1\\
% |cdocsch2.tex|&include file for chapter 2\\
% |cdocspt3.tex|&include file for part 3\\
% |cdocspt4.tex|&include file for part 4\\
% |cdocsdrf.tex|&forwarding file for main file in draft mode\\
% |cdocsfi1.tex|&forwarding file for final version of chapter 1\\
% |cdocsfi2.tex|&forwarding file for final version of chapter 2\\
% \end{tabular}
% \end{center}
% Each of the eight files can be compiled directly by the \LaTeX{} compiler.
%
% %%%%%%%%%%%%%%%%%%%%%%%%%%%%%%%%%%%%%%
% \paragraph{Main File.}
%
% The main file is called |cdocsamp.tex|.
%
% Load the \textsf{childdoc} definitions and
% declare the filename for the main document:
%    \begin{macrocode}
\input{childdoc.def}
\childdocmain{}
%    \end{macrocode}

% Optional override for |\version| flag:
%    \begin{macrocode}
%%\ifchilddoc\else\providecommand{\version}{draft}\fi
%    \end{macrocode}

% Define the default values for the |\version| flag
% (|final| for the main file and |draft| for childs):
%    \begin{macrocode}
\ifchilddoc
\providecommand{\version}{draft}
\else
\providecommand{\version}{final}
\fi
%    \end{macrocode}

% Load the standard document class:
%    \begin{macrocode}
\documentclass[12pt]{article}
%    \end{macrocode}

% Start the document body:
%    \begin{macrocode}
\begin{document}
%    \end{macrocode}

% Declare a title page.
% Print title, part of document being processed and version flag:
%    \begin{macrocode}
\addtocounter{page}{-1}
\begin{center}
{\LARGE\bfseries{}childdoc example\par}
\vspace{1cm}
\ifchilddoc
\ifchilddocmanual part\else chapter\fi:
`\childdocname' of `\childdocjob'\par
\else
main document: `\childdocjob'\par
\fi
version: \version\par
\end{center}
\newpage
%    \end{macrocode}

% Manually include selected file,
% otherwise process as usual:
%    \begin{macrocode}
\ifchilddocmanual
\section*{part `\childdocname'}
\input{\childdocname}
\else
%    \end{macrocode}

% Include the two chapters:
%    \begin{macrocode}
\include{cdocsch1}
\include{cdocsch2}
%    \end{macrocode}

% Include the two parts unless only chapters should be displayed:
%    \begin{macrocode}
\ifchilddoc\else
\section{part three}
\input{cdocspt3}
\section{part four}
\input{cdocspt4}
\fi
%    \end{macrocode}

% Process as usual until here:
%    \begin{macrocode}
\fi
%    \end{macrocode}

% End of document body:
%    \begin{macrocode}
\end{document}
%    \end{macrocode}
%\iffalse
%</samplemain>
%\fi
%
% %%%%%%%%%%%%%%%%%%%%%%%%%%%%%%%%%%%%%%
% \paragraph{Chapter Include Files.}
%
% The include files are called |cdocsch1.tex| and |cdocsch2.tex|.
%
%\iffalse
%<*samplechap1|samplechap2>
%\fi

% Optional override for |\version| flag:
%    \begin{macrocode}
%%\providecommand{\version}{final}
%    \end{macrocode}

% Include the main document:
%    \begin{macrocode}
\input{childdoc.def}
\childdocof{cdocsamp}
%    \end{macrocode}

%\iffalse
%</samplechap1|samplechap2>
%\fi
%
%\iffalse
%<*samplechap1>
%\fi
% Some text for chapter 1:
%    \begin{macrocode}
\section{one}
some text in chapter one
%    \end{macrocode}

%\iffalse
%</samplechap1>
%\fi
% Some text for chapter 2:
%\iffalse
%<*samplechap2>
%\fi
%    \begin{macrocode}
\section{two}
more text in chapter two
%    \end{macrocode}

%\iffalse
%</samplechap2>
%\fi
%
% %%%%%%%%%%%%%%%%%%%%%%%%%%%%%%%%%%%%%%
% \paragraph{Part Include Files.}
%
% The include files are called |cdocspt3.tex| and |cdocspt4.tex|.
%
%\iffalse
%<*samplepart3|samplepart4>
%\fi

% Optional override for |\version| flag:
%    \begin{macrocode}
%%\providecommand{\version}{final}
%    \end{macrocode}

% Include the main document:
%    \begin{macrocode}
\input{childdoc.def}
\childdocby{cdocsamp}
%    \end{macrocode}

%\iffalse
%</samplepart3|samplepart4>
%\fi
%
%\iffalse
%<*samplepart3>
%\fi
% Some text for part 3:
%    \begin{macrocode}
some text in part three
%    \end{macrocode}

%\iffalse
%</samplepart3>
%\fi
% Some text for part 4:
%\iffalse
%<*samplepart4>
%\fi
%    \begin{macrocode}
more text in part four
%    \end{macrocode}

%\iffalse
%</samplepart4>
%\fi
%
% %%%%%%%%%%%%%%%%%%%%%%%%%%%%%%%%%%%%%%
% \paragraph{Forwarding for a Complete Draft.}
%
% The following forwarding file |cdocsdrf.tex|
% compiles the main document in draft mode:
%\iffalse
%<*sampledraft>
%\fi
%    \begin{macrocode}
\def\version{draft}
\input{childdoc.def}
\childdocforward{cdocsamp}
%    \end{macrocode}

%\iffalse
%</sampledraft>
%\fi
%
% %%%%%%%%%%%%%%%%%%%%%%%%%%%%%%%%%%%%%%
% \paragraph{Forwarding for Final Version of the Chapters.}
%
% The following forwarding files |cdocsfn1.tex| and |cdocsfn2.tex|
% (with identical content)
% compile the final versions of the child documents
% |cdocsch1.tex| and |cdocsch2.tex|, respectively:
%\iffalse
%<*samplefinal>
%\fi
%    \begin{macrocode}
\def\version{final}
\input{childdoc.def}
\childdocforwardprefix[cdocsamp]{cdocsfn}{cdocsch}
%    \end{macrocode}

%\iffalse
%</samplefinal>
%\fi
%
% %%%%%%%%%%%%%%%%%%%%%%%%%%%%%%%%%%%%%%
% \paragraph{Command Line Processing.}
%
% The following three command lines generate the output files
% |cdocscld|, |cdocscl1| and |cdocscl2|
% which should be identical to
% |cdocsdrf|, |cdocsch1| and |cdocsfn2|, respectively:
% \begin{center}
% \begin{tabular}{l}
% |latex -jobname cdocscld \|\\
% |  "\def\version{draft}\input{childdoc.def}\childdocforward{cdocsamp}"|\\
% |latex -jobname cdocscl1 \|\\
% |  "\input{childdoc.def}\childdocforward[cdocsamp]{cdocsch1}"|\\
% |latex -jobname cdocscl2 \|\\
% |  "\def\version{final}\input{childdoc.def}\childdocforward{cdocsch2}"|
% \end{tabular}
% \end{center}
% Note that the trailing backslash on each first line
% merely continues the input to the second line
% (for convenient cut ant paste).
% Furthermore, the command |latex| can be replaced by any
% of its alternative versions such as |pdflatex|.
%
% %%%%%%%%%%%%%%%%%%%%%%%%%%%%%%%%%%%%%%%%%%%%%%%%%%%%%%%%%%%%%%%%%%%%%%%%%%%%%%
% %%%%%%%%%%%%%%%%%%%%%%%%%%%%%%%%%%%%%%%%%%%%%%%%%%%%%%%%%%%%%%%%%%%%%%%%%%%%%%
% \section{Implementation}
%\iffalse
%<*package>
%\fi
%
% This section describes the definitions file |childdoc.def|.

% The definitions cannot be loaded using |\usepackage| or |\RequirePackage|
% which has a mechanism to prevent loading a style file more than once.
% When loading the definitions by means of |\input|
% multiple instances have to be prevented manually:
%\iffalse
%This code needs to be before the `\ProvidesFile' directive
%which is defined at the beginning of this file.
%Therefore it is also placed there and commented out here.
%</package>
%<*discard>
%\fi
%    \begin{macrocode}
\ifdefined\childdocmain\endinput\fi
%    \end{macrocode}
%\iffalse
%</discard>
%<*package>
%\fi
%
% \macro{\ifchilddoc}
% \macro{\ifchilddocmanual}
% The conditional |\ifchilddoc| tells whether a
% child (true) or main (false) document is being compiled.
% The conditional |\ifchilddocmanual| tells whether
% the |\includeonly| mechanism is used (false) or
% the selection of child files must be performed manually (true).
% The definitions initialise to false:
%    \begin{macrocode}
\newif\ifchilddoc
\newif\ifchilddocmanual
%    \end{macrocode}

% \macro{\childdocname}
% \macro{\childdocjob}
% The macro |\childdocname| stores the name of the main document
% to be compiled. The macro |\childdocjob| stores the name of
% the document on which the \LaTeX{} compiler was originally invoked.
% The content of |\jobname| cannot be compared
% to filenames specified in the source due to different catcodes.
% The following code rescans |\jobname|, stores the result
% in |\childdocname| and saves a copy in |\childdocjob|:
%    \begin{macrocode}
\edef\childdocname{\scantokens\expandafter{\jobname\noexpand}}
\let\childdocjob\childdocname
%    \end{macrocode}

% \macro{\childdocdisable}
% The macro |\childdocdisable| prevents the main file
% from being processed more than once.
% At this stage, the main document command |\childdocmain|
% is assumed to be called once again where it should do nothing.
% Any subsequent call to it should prevent
% a secondary processing of the main document
% It overwrites the forwarding commands
% |\childdocof| and |\childdocforward|
% with empty macros to prevent further inclusions of the main document:
%    \begin{macrocode}
\newcommand{\childdocdisable}
{
  \renewcommand{\childdocmain}[1]{\renewcommand{\childdocmain}[1]{\endinput}}
  \renewcommand{\childdocof}[1]{}
  \renewcommand{\childdocby}[2][]{}
  \renewcommand{\childdocforward}[2][]{}
  \renewcommand{\childdocdisable}{}
}
%    \end{macrocode}

% \macro{\childdocmain}
% The macro |\childdocmain| is to be called at the top of the main file
% with nothing or the main filename (without extension) as argument.
% First, it breaks loops.
% If the argument is not empty and does not match |\childdocname|
% (which is set by the first inclusion of |childdoc.def|),
% |\ifchilddoc| is set to true, |\includeonly| is applied to the child file
% and |\jobname| is set to the main file
% (for proper handling of |.aux| files):
%    \begin{macrocode}
\newcommand{\childdocmain}[1]
{
  \childdocdisable\childdocmain{}
  \if?#1?\else
    \begingroup
      \def\childdoctmp{#1}
      \ifx\childdoctmp\childdocname
        \def\childdoctmp{}
      \else
        \def\childdoctmp
        {
          \childdoctrue
          \includeonly{\childdocname}
          \def\childdocjob{#1}
          \def\jobname{#1}
        }
      \fi
      \expandafter
    \endgroup
    \childdoctmp
  \fi
}
%    \end{macrocode}

% \macro{\childdocof}
% The command |\childdocof| redirects
% compilation to the main file |#1|.
%    \begin{macrocode}
\newcommand{\childdocof}[1]
{
  \childdocdisable
  \childdoctrue
  \includeonly{\childdocname}
  \def\jobname{#1}
  \def\childdocjob{#1}
  \input{#1}
}
%    \end{macrocode}

% \macro{\childdocby}
% The command |\childdocby| ....
%    \begin{macrocode}
\newcommand{\childdocby}[2][]
{
  \childdocdisable
  \childdoctrue
  \childdocmanualtrue
  \if?#1?\else
    \def\jobname{#2}
  \fi
  \def\childdocjob{#2}
  \input{#2}
  \endinput
}
%    \end{macrocode}

% \macro{\childdocforward}
% The command |\childdocforward| redirects
% compilation to the main file or
% (if the optional argument is given) a child file.
% Parameters are set as if the main file
% or a child file starting with |\childdocof| was compiled.
% Then compilation is handed over to the main file:
%    \begin{macrocode}
\newcommand{\childdocforward}[2][]
{
  \begingroup
    \if?#1?
      \def\childdoctmp
      {
        \def\childdocname{#2}
        \def\childdocjob{#2}
        \def\jobname{#2}
        \input{#2}
        \endinput
      }
    \else
      \def\childdoctmp
      {
        \childdocdisable
        \def\childdocname{#2}
        \childdoctrue
        \includeonly{#2}
        \def\childdocjob{#1}
        \def\jobname{#1}
        \input{#1}
        \endinput
      }
    \fi
    \expandafter
  \endgroup
  \childdoctmp
}
%    \end{macrocode}

% \macro{\childdocforwardprefix}
% The command |\childdocforwardprefix| redirects
% compilation to the main or a child file by means of a pattern.
% The prefix |#1| in the current filename is replaced by |#2|
% and the suffix of the current filename is kept
% (it is assumed that the filename does not contain the substring `|~~~|'
% which is used as a delimiter).
% Compilation is handed over to the new file by |\childdocforward|:
%    \begin{macrocode}
\newcommand{\childdocforwardprefix}[3][]
{
  \begingroup
    \def\childdocextract #2##1~~~{\def\childdoctmp{\childdocforward[#1]{#3##1}}}
    \expandafter\childdocextract\childdocname~~~
    \expandafter
  \endgroup
  \childdoctmp
}
%    \end{macrocode}

% \macro{\childdoc}
% The deprecated macro |\childdoc| is a legacy version of |\childdocmain|:
%    \begin{macrocode}
\newcommand{\childdoc}{\childdocmain}
%    \end{macrocode}

% \macro{\childdocredirect}
% The deprecated macro |\childdocredirect| is a legacy version
% of |\childdocforward| and |\childdocforwardprefix|:
%    \begin{macrocode}
\newcommand{\childdocredirect}[2][]
{
  \begingroup
    \if?#1?
      \def\childdoctmp{\childdocforward{#2}}
    \else
      \def\childdoctmp{\childdocforwardprefix{#1}{#2}}
    \fi
    \expandafter
  \endgroup
  \childdoctmp
}
%    \end{macrocode}

%\iffalse
%</package>
%\fi
%
\endinput
|\\
|\childdocof{|\textit{main}|}|\\
\end{tabular}
\end{center}
at the top of every child file \textit{child}
which is included by |\include{|\textit{child}|}|
from within the main file
(or at least for those files to be compiled individually).
The argument \textit{main} must be the filename of the main file.

There are a couple of
considerations in setting up the main and child documents:

%%%%%%%%%%%%%%%%%%%%%%%%%%%%%%%%%%%%%%%%
\paragraph{Restrictions.}

Please note the following restrictions:
\begin{itemize}
\item
|\childdocmain| must be called with one argument \textit{main}
to ensure compatibility with earlier version of the package.
It must either be empty (|\childdocmain{}|)
or precisely match the filename of the main file in which it is specified.
See \secref{sec:detection} for further information.
\item
The filename \textit{main} must be specified without the |.tex| extension.
\item
The filename \textit{main} is case sensitive
(even in case-insensitive file systems)
due to internal string comparison.
\item
The argument \textit{main} should be fully expanded, it cannot be a macro.
\item
Subdirectories and special characters should be avoided in filenames.
\item
The command |\childdocmain{|\textit{main}|}| must be followed by a whitespace.
It should not be followed immediately by another command
or by a comment mark `|%|'.
This is because the \TeX{} parser reads the token immediately following
the argument of |\childdocmain| and puts it
at the beginning of every child section;
however, a white\-space is ignored.
\end{itemize}

%%%%%%%%%%%%%%%%%%%%%%%%%%%%%%%%%%%%%%%%
\paragraph{Content of Main File.}

It is advisable to place all content in the child files included by |\include|.
Any output contained in the main file will appear in all child documents
unless suppressed manually;
it cannot be suppressed automatically by the |\includeonly| directive
and thus should normally be avoided.
A method to include some content in the main file
by means of conditional processing is described in \secref{sec:conditional}.

%%%%%%%%%%%%%%%%%%%%%%%%%%%%%%%%%%%%%%%%
\paragraph{Page Numbering.}

When only a part of the document is compiled,
the appropriate numbering of pages
(as well as other status parameters)
is determined from the |.aux| files.
The latter contain information from previous passes.
However this information needs to propagate through
all intermediate child documents.
Therefore the page numbering in child documents may well
be inconsistent until the complete document is compiled at least once.

A useful (if unconventional) way to always ensure a consistent
page numbering is to restart the numbering in each child document
and denote the pages by `\textit{child}|.|\textit{page}'
where \textit{child} represents the chapter/section number of the child file.
This can be achieved by the command
|\numberwithin{page}{|\textit{child}|}|
of the \textsf{amsmath} package
where \textit{child} can be |chapter| or |section|
depending on the chosen structuring.
Alternatively, one can modify the macro |\thepage| appropriately
and reset the counter |page| at the start of each child file.

%%%%%%%%%%%%%%%%%%%%%%%%%%%%%%%%%%%%%%%%%%%%%%%%%%%%%%%%%%%%%%%%%%%%%%%%%%%%%%%%
\subsection{Conditional Processing}
\label{sec:conditional}

The package provides a mechanism to compile different versions
of a document. To customise the versions further some conditional processing
can come in handy to distinguish which version is being compiled.
The package provides two macros to describe the compilation context:

%%%%%%%%%%%%%%%%%%%%%%%%%%%%%%%%%%%%%%%%
\DescribeMacro{\ifchilddoc}
The conditional |\ifchilddoc| distinguishes between the compilation of
child documents and the main document:
%
\begin{center}
|\ifchilddoc |\textit{child-code}| |[|\||else |\textit{main-code}]| \||fi|
\end{center}

%%%%%%%%%%%%%%%%%%%%%%%%%%%%%%%%%%%%%%%%
\DescribeMacro{\childdocname}
\DescribeMacro{\childdocjob}
The macro |\childdocname| contains the filename (without extension)
of the main or child file being processed.
Note that |\childdocjob| will always contain the name of the main file.

%%%%%%%%%%%%%%%%%%%%%%%%%%%%%%%%%%%%%%%%
\paragraph{Title Page.}

Conditional processing can be used to include a title or banner page
in the main document when proper precautions are taken.
Importantly, the code in the main file should ensure that the page counter
(as well as other status parameters which are stored in the |.aux| files)
takes the same value after the conditional processing.
Otherwise the page numbers may take divergent values
depending on which part is compiled.

For example, a title page could be declared by:
%
\begin{center}
\begin{tabular}{l}
|\ifchilddoc\||else|\\
|\addtocounter{page}{-1}|\\
\textit{code for title page}\\
|\newpage|\\
|\||fi|
\end{tabular}
\end{center}
%
A banner page for the child documents can be generated by:
%
\begin{center}
\begin{tabular}{l}
|\ifchilddoc|\\
|\addtocounter{page}{-1}|\\
\textit{code for banner page}\\
|\newpage|\\
|\||fi|
\end{tabular}
\end{center}
%
Here one could write a message such as:
\begin{center}
|This is the part \childdocname{} of \childdocjob{}.|
\end{center}

%%%%%%%%%%%%%%%%%%%%%%%%%%%%%%%%%%%%%%%%%%%%%%%%%%%%%%%%%%%%%%%%%%%%%%%%%%%%%%%%
\subsection{Flags}
\label{sec:flags}

The package makes it easy to generate different versions
of the main or child documents.
To this end compilation flags can be defined
and assigned different default values.
They will be particularly useful in conjunction
with the forwarding mechanism described in \secref{sec:forward}.

For example, it may be useful to have a flag |\version|
which can be set to |draft| or |final|.
The document source will contain some conditional code
depending on the value of |\version|.
Suppose further, the flag should default to |final| for the main file
and to |draft| for child files
which is a natural assignment for editing the document.
This is achieved by placing the following code
in the preamble of the main document
(below the |\childdocmain| directive):
%
\begin{center}
\begin{tabular}{l}
|\ifchilddoc|\\
|\providecommand{\version}{draft}|\\
|\||else|\\
|\providecommand{\version}{final}|\\
|\||fi|
\end{tabular}
\end{center}
%
The definition by |\providecommand| makes sure
that previous definitions are not overwritten.
Further statements |\providecommand{\version}{...}|
can thus be added before the above code to override it.

For the main file, one might add a line
(between |\childdocmain| and the above block)
%
\begin{center}
|%\ifchilddoc\||else\providecommand{\version}{draft}\||fi|
\end{center}
%
which can be uncommented to produce a draft version.
Likewise one can add a line to the very top of a child file
(above the |\childdocof{|\textit{main}|}| directive)
%
\begin{center}
|%\providecommand{\version}{final}|
\end{center}
%
which can be uncommented to produce the final version of this child document.

%%%%%%%%%%%%%%%%%%%%%%%%%%%%%%%%%%%%%%%%%%%%%%%%%%%%%%%%%%%%%%%%%%%%%%%%%%%%%%%%
\subsection{Forwarding}
\label{sec:forward}

Different versions of the main or child documents
using compilation flags as described in \secref{sec:flags}
can be (permanently) stored in different files
for convenient compilation, viewing and distribution.
To this end, the package defines a command
to pass on compilation to a different file:

%%%%%%%%%%%%%%%%%%%%%%%%%%%%%%%%%%%%%%%%
\DescribeMacro{\childdocforward}
The command |\childdocforward| redirects processing to
another source file:
%
\begin{center}
\begin{tabular}{l}
|% \iffalse
%
% childdoc.dtx Copyright (C) 2017-2018 Niklas Beisert
%
% This work may be distributed and/or modified under the
% conditions of the LaTeX Project Public License, either version 1.3
% of this license or (at your option) any later version.
% The latest version of this license is in
%   http://www.latex-project.org/lppl.txt
% and version 1.3 or later is part of all distributions of LaTeX
% version 2005/12/01 or later.
%
% This work has the LPPL maintenance status `maintained'.
%
% The Current Maintainer of this work is Niklas Beisert.
%
% This work consists of the files childdoc.dtx and childdoc.ins
% and the derived files childdoc.def and cdocsamp.tex with
% cdocsch1.tex, cdocsch2.tex, cdocsdrf.tex, cdocsfn1.tex, cdocsfn2.tex.
%
%<package>\ifdefined\childdocmain\endinput\fi
%<package>\ProvidesFile{childdoc.def}[2018/12/30 v2.0 child document driver]
%<samplemain>\ProvidesFile{cdocsamp.tex}[2018/12/30 v2.0 sample for childdoc]
%<*driver>
%\ProvidesFile{childdoc.drv}[2018/12/30 v2.0 childdoc reference manual file]
\PassOptionsToClass{10pt,a4paper}{article}
\documentclass{ltxdoc}

\usepackage[margin=35mm]{geometry}
\usepackage{hyperref}
\usepackage{hyperxmp}
\usepackage[usenames]{color}

\hypersetup{colorlinks=true}
\hypersetup{pdfstartview=FitH}
\hypersetup{pdfpagemode=UseNone}
\hypersetup{pdfsource={}}
\hypersetup{pdflang={en-UK}}
\hypersetup{pdfcopyright={Copyright 2017-2018 Niklas Beisert.
  This work may be distributed and/or modified under the
  conditions of the LaTeX Project Public License, either version 1.3
  of this license or (at your option) any later version.}}
\hypersetup{pdflicenseurl={http://www.latex-project.org/lppl.txt}}
\hypersetup{pdfcontactaddress={ETH Zurich, ITP, HIT K,
  Wolfgang-Pauli-Strasse 27}}
\hypersetup{pdfcontactpostcode={8093}}
\hypersetup{pdfcontactcity={Zurich}}
\hypersetup{pdfcontactcountry={Switzerland}}
\hypersetup{pdfcontactemail={nbeisert@itp.phys.ethz.ch}}
\hypersetup{pdfcontacturl={http://people.phys.ethz.ch/\xmptilde nbeisert/}}

\newcommand{\secref}[1]{\hyperref[#1]{section \ref*{#1}}}

\parskip1ex
\parindent0pt
\let\olditemize\itemize
\def\itemize{\olditemize\parskip0pt}

\begin{document}

\title{The \textsf{childdoc} Package}
\hypersetup{pdftitle={The childdoc Package}}
\author{Niklas Beisert\\[2ex]
  Institut f\"ur Theoretische Physik\\
  Eidgen\"ossische Technische Hochschule Z\"urich\\
  Wolfgang-Pauli-Strasse 27, 8093 Z\"urich, Switzerland\\[1ex]
  \href{mailto:nbeisert@itp.phys.ethz.ch}
  {\texttt{nbeisert@itp.phys.ethz.ch}}}
\hypersetup{pdfauthor={Niklas Beisert}}
\hypersetup{pdfsubject={Manual for the LaTeX2e Package childdoc}}
\date{30 December 2018, \textsf{v2.0}}
\maketitle

\begin{abstract}\noindent
\textsf{childdoc} is a \LaTeXe{} package
that enables the direct compilation
of document sections included by |\include|
to individual files.
\end{abstract}

\begingroup
\parskip0ex
\tableofcontents
\endgroup

%%%%%%%%%%%%%%%%%%%%%%%%%%%%%%%%%%%%%%%%%%%%%%%%%%%%%%%%%%%%%%%%%%%%%%%%%%%%%%%%
%%%%%%%%%%%%%%%%%%%%%%%%%%%%%%%%%%%%%%%%%%%%%%%%%%%%%%%%%%%%%%%%%%%%%%%%%%%%%%%%
\section{Introduction}

\LaTeX{} provides a mechanism to structure a large document (such as a book)
into a main file and several child files (containing the chapters)
using the |\include| command.
This mechanism is beneficial for documents
which span hundreds of pages in order to
make the source file(s) more manageable.
Moreover, compilation can be restricted to
selected child files by means of the |\includeonly| command.
The latter feature can be used to reduce the compilation time while editing
(this was significantly more useful in the earlier days of \LaTeX{})
or to generate a smaller document which is easier to navigate.
Another application of |\includeonly| is to generate
documents consisting of selected parts of the complete document.

However, there are a few drawbacks of the plain |\include| mechanism:
\begin{itemize}
\item
The child files cannot be compiled on their own,
they can only be compiled via the main file.
A naive editing environment
(such as a text editor with an option
to have the current file processed by \LaTeX)
may require one to switch to the main file before compiling;
attempting to compile the child file produces errors.
\item
The main file must be modified (each time)
to adjust the |\includeonly| command
to the present needs. This easily leaves the main file in a messy state.
\item
The generated document will always carry the filename
of the main document. This is inconvenient if
several child files are to be compiled and
to be kept for distribution.
\end{itemize}

The present package provides a simple interface
to make child files individually compilable by \LaTeX{}.
Compiling a child file then has the same effect as compiling
the main file with an |\includeonly| command
to select the appropriate child.
Moreover the generated document will carry the name of the child
rather than the main file.
This resolves all three above issues.

This feature is meant to make the editing of books,
thesis documents and lecture notes somewhat more convenient.
However, the package can also be used efficiently for
composing a series of documents (such as exercise sheets)
which are typically distributed individually.
It then assists the author in generating the individual documents
(potentially in different versions)
as well as a document containing the collected series.
Another application is in developing style files
or other kinds of included material
where compilation of the style file could redirect
to a sample or test file.

%%%%%%%%%%%%%%%%%%%%%%%%%%%%%%%%%%%%%%%%%%%%%%%%%%%%%%%%%%%%%%%%%%%%%%%%%%%%%%%%
%%%%%%%%%%%%%%%%%%%%%%%%%%%%%%%%%%%%%%%%%%%%%%%%%%%%%%%%%%%%%%%%%%%%%%%%%%%%%%%%
\section{Usage}

First of all, the package \textsf{childdoc} is \emph{not} a standard
\LaTeXe{} |.sty| style file! Therefore it needs to be invoked in
a non-standard way.

%%%%%%%%%%%%%%%%%%%%%%%%%%%%%%%%%%%%%%%%%%%%%%%%%%%%%%%%%%%%%%%%%%%%%%%%%%%%%%%%
\subsection{Included Files}
\label{sec:include}

%%%%%%%%%%%%%%%%%%%%%%%%%%%%%%%%%%%%%%%%
\DescribeMacro{\childdocmain}
To use the package, add the commands
\begin{center}
\begin{tabular}{l}
|\input{childdoc.def}|\\
|\childdocmain{}|\\
\end{tabular}
\end{center}
at the very top of the main \LaTeX{} file,
in particular \emph{before} the |\documentclass| statement!
The argument of |\childdocmain| should be left empty
(but it must be present).

%%%%%%%%%%%%%%%%%%%%%%%%%%%%%%%%%%%%%%%%
\DescribeMacro{\childdocof}
Furthermore, add the commands
\begin{center}
\begin{tabular}{l}
|\input{childdoc.def}|\\
|\childdocof{|\textit{main}|}|\\
\end{tabular}
\end{center}
at the top of every child file \textit{child}
which is included by |\include{|\textit{child}|}|
from within the main file
(or at least for those files to be compiled individually).
The argument \textit{main} must be the filename of the main file.

There are a couple of
considerations in setting up the main and child documents:

%%%%%%%%%%%%%%%%%%%%%%%%%%%%%%%%%%%%%%%%
\paragraph{Restrictions.}

Please note the following restrictions:
\begin{itemize}
\item
|\childdocmain| must be called with one argument \textit{main}
to ensure compatibility with earlier version of the package.
It must either be empty (|\childdocmain{}|)
or precisely match the filename of the main file in which it is specified.
See \secref{sec:detection} for further information.
\item
The filename \textit{main} must be specified without the |.tex| extension.
\item
The filename \textit{main} is case sensitive
(even in case-insensitive file systems)
due to internal string comparison.
\item
The argument \textit{main} should be fully expanded, it cannot be a macro.
\item
Subdirectories and special characters should be avoided in filenames.
\item
The command |\childdocmain{|\textit{main}|}| must be followed by a whitespace.
It should not be followed immediately by another command
or by a comment mark `|%|'.
This is because the \TeX{} parser reads the token immediately following
the argument of |\childdocmain| and puts it
at the beginning of every child section;
however, a white\-space is ignored.
\end{itemize}

%%%%%%%%%%%%%%%%%%%%%%%%%%%%%%%%%%%%%%%%
\paragraph{Content of Main File.}

It is advisable to place all content in the child files included by |\include|.
Any output contained in the main file will appear in all child documents
unless suppressed manually;
it cannot be suppressed automatically by the |\includeonly| directive
and thus should normally be avoided.
A method to include some content in the main file
by means of conditional processing is described in \secref{sec:conditional}.

%%%%%%%%%%%%%%%%%%%%%%%%%%%%%%%%%%%%%%%%
\paragraph{Page Numbering.}

When only a part of the document is compiled,
the appropriate numbering of pages
(as well as other status parameters)
is determined from the |.aux| files.
The latter contain information from previous passes.
However this information needs to propagate through
all intermediate child documents.
Therefore the page numbering in child documents may well
be inconsistent until the complete document is compiled at least once.

A useful (if unconventional) way to always ensure a consistent
page numbering is to restart the numbering in each child document
and denote the pages by `\textit{child}|.|\textit{page}'
where \textit{child} represents the chapter/section number of the child file.
This can be achieved by the command
|\numberwithin{page}{|\textit{child}|}|
of the \textsf{amsmath} package
where \textit{child} can be |chapter| or |section|
depending on the chosen structuring.
Alternatively, one can modify the macro |\thepage| appropriately
and reset the counter |page| at the start of each child file.

%%%%%%%%%%%%%%%%%%%%%%%%%%%%%%%%%%%%%%%%%%%%%%%%%%%%%%%%%%%%%%%%%%%%%%%%%%%%%%%%
\subsection{Conditional Processing}
\label{sec:conditional}

The package provides a mechanism to compile different versions
of a document. To customise the versions further some conditional processing
can come in handy to distinguish which version is being compiled.
The package provides two macros to describe the compilation context:

%%%%%%%%%%%%%%%%%%%%%%%%%%%%%%%%%%%%%%%%
\DescribeMacro{\ifchilddoc}
The conditional |\ifchilddoc| distinguishes between the compilation of
child documents and the main document:
%
\begin{center}
|\ifchilddoc |\textit{child-code}| |[|\||else |\textit{main-code}]| \||fi|
\end{center}

%%%%%%%%%%%%%%%%%%%%%%%%%%%%%%%%%%%%%%%%
\DescribeMacro{\childdocname}
\DescribeMacro{\childdocjob}
The macro |\childdocname| contains the filename (without extension)
of the main or child file being processed.
Note that |\childdocjob| will always contain the name of the main file.

%%%%%%%%%%%%%%%%%%%%%%%%%%%%%%%%%%%%%%%%
\paragraph{Title Page.}

Conditional processing can be used to include a title or banner page
in the main document when proper precautions are taken.
Importantly, the code in the main file should ensure that the page counter
(as well as other status parameters which are stored in the |.aux| files)
takes the same value after the conditional processing.
Otherwise the page numbers may take divergent values
depending on which part is compiled.

For example, a title page could be declared by:
%
\begin{center}
\begin{tabular}{l}
|\ifchilddoc\||else|\\
|\addtocounter{page}{-1}|\\
\textit{code for title page}\\
|\newpage|\\
|\||fi|
\end{tabular}
\end{center}
%
A banner page for the child documents can be generated by:
%
\begin{center}
\begin{tabular}{l}
|\ifchilddoc|\\
|\addtocounter{page}{-1}|\\
\textit{code for banner page}\\
|\newpage|\\
|\||fi|
\end{tabular}
\end{center}
%
Here one could write a message such as:
\begin{center}
|This is the part \childdocname{} of \childdocjob{}.|
\end{center}

%%%%%%%%%%%%%%%%%%%%%%%%%%%%%%%%%%%%%%%%%%%%%%%%%%%%%%%%%%%%%%%%%%%%%%%%%%%%%%%%
\subsection{Flags}
\label{sec:flags}

The package makes it easy to generate different versions
of the main or child documents.
To this end compilation flags can be defined
and assigned different default values.
They will be particularly useful in conjunction
with the forwarding mechanism described in \secref{sec:forward}.

For example, it may be useful to have a flag |\version|
which can be set to |draft| or |final|.
The document source will contain some conditional code
depending on the value of |\version|.
Suppose further, the flag should default to |final| for the main file
and to |draft| for child files
which is a natural assignment for editing the document.
This is achieved by placing the following code
in the preamble of the main document
(below the |\childdocmain| directive):
%
\begin{center}
\begin{tabular}{l}
|\ifchilddoc|\\
|\providecommand{\version}{draft}|\\
|\||else|\\
|\providecommand{\version}{final}|\\
|\||fi|
\end{tabular}
\end{center}
%
The definition by |\providecommand| makes sure
that previous definitions are not overwritten.
Further statements |\providecommand{\version}{...}|
can thus be added before the above code to override it.

For the main file, one might add a line
(between |\childdocmain| and the above block)
%
\begin{center}
|%\ifchilddoc\||else\providecommand{\version}{draft}\||fi|
\end{center}
%
which can be uncommented to produce a draft version.
Likewise one can add a line to the very top of a child file
(above the |\childdocof{|\textit{main}|}| directive)
%
\begin{center}
|%\providecommand{\version}{final}|
\end{center}
%
which can be uncommented to produce the final version of this child document.

%%%%%%%%%%%%%%%%%%%%%%%%%%%%%%%%%%%%%%%%%%%%%%%%%%%%%%%%%%%%%%%%%%%%%%%%%%%%%%%%
\subsection{Forwarding}
\label{sec:forward}

Different versions of the main or child documents
using compilation flags as described in \secref{sec:flags}
can be (permanently) stored in different files
for convenient compilation, viewing and distribution.
To this end, the package defines a command
to pass on compilation to a different file:

%%%%%%%%%%%%%%%%%%%%%%%%%%%%%%%%%%%%%%%%
\DescribeMacro{\childdocforward}
The command |\childdocforward| redirects processing to
another source file:
%
\begin{center}
\begin{tabular}{l}
|\input{childdoc.def}|\\
|\childdocforward[|\textit{main}|]{|\textit{dest}|}|\\
\end{tabular}
\end{center}
%
The argument \textit{dest} is the destination file
(without extension).
It should be the main file or one of the child files.
Note that further \textsf{childdoc} directives
such as |\childdocof| and |\childdocforward|
in the indicated file will be processed in this form.
The optional argument \textit{main}
passes on directly to the main file \textit{main}
while pretending to compile the child \textit{dest}.
This form behaves as if \textit{dest}
issues |\childdocof{|\textit{main}|}| right away,
and no further \textsf{childdoc} directives will be processed.

%%%%%%%%%%%%%%%%%%%%%%%%%%%%%%%%%%%%%%%%
\DescribeMacro{\...prefix}
In the alternative form |\childdocforwardprefix|,
%
\begin{center}
\begin{tabular}{l}
|\input{childdoc.def}|\\
|\childdocforwardprefix[|\textit{main}|]{|\textit{prefix}|}{|\textit{dest}|}|
\end{tabular}
\end{center}
%
the destination file is determined by a pattern
depending on the current file:
To make this work, the current file must be called
`{\textit{prefix}\hspace{0.2em}\textit{suffix}}'
with \textit{prefix} matching precisely the argument.
Processing is then passed on to the file
`{\textit{dest}\hspace{0.2em}\textit{suffix}}'.
Surely, the same effect is achieved by
directly specifying the
argument `{\textit{dest}\hspace{0.2em}\textit{suffix}}'
in the first form.
However, that requires to set up a different file
for each child. With the alternative form of the command
all these files can have exactly the same content
which simplifies setting them up and maintaining them.

For example, the following file |draft.tex|
with a compilation flag |\version| as described in \secref{sec:flags}
compiles the main document as a draft:
%
\begin{center}
\begin{tabular}{l}
|\def\version{draft}|\\
|\input{childdoc.def}|\\
|\childdocforward{|\textit{main}|}|
\end{tabular}
\end{center}
%
Likewise, the following files |final|\textit{nn}|.tex|
compile the final version of the child document
|child|\textit{nn}|.tex|:
%
\begin{center}
\begin{tabular}{l}
|\def\version{final}|\\
|\input{childdoc.def}|\\
|\childdocforwardprefix{final}{child}|
\end{tabular}
\end{center}
%

Note that when several versions of a main file and/or of each child file
are to be generated, it may be convenient to set up a |Makefile| or
shell script to automatise the process.

%%%%%%%%%%%%%%%%%%%%%%%%%%%%%%%%%%%%%%%%%%%%%%%%%%%%%%%%%%%%%%%%%%%%%%%%%%%%%%%%
\subsection{Command Line Processing}
\label{sec:commandline}

The effect of redirection files can also be achieved by invoking
the \LaTeX{} compiler with a more elaborate command line.
Most conveniently this should be done as part
of a shell script or a |Makefile|.

When using \textsf{childdoc} in the main file, the following
command lines effectively perform a redirection
(note that depending on the shell being used,
backslashes may have to be doubled: `|\|' $\to$ `|\\|'):
%
\begin{center}
|... -jobname "|\textit{target}|" |\\|"|[\textit{flags}]%
|\input{childdoc.def}\childdocforward[|\textit{main}|]{|\textit{dest}|}"|
\end{center}
%
Here \textit{target} is the name of the output file,
\textit{main} is the name of the main file
and \textit{dest} is the name of the main or child file to be processed
(all filenames without extensions).
The optional argument \textit{main} can be omitted
if \textit{main} matches \textit{dest}.
Optionally, compilation \textit{flags} can be defined via |\def| commands.
This command line makes the \TeX{} engine believe
it is compiling the file \textit{target}
whose content is specified as the latter parameter.
The provided code then forwards the processing to
\textit{main} or \textit{dest} as described in \secref{sec:forward}.

%%%%%%%%%%%%%%%%%%%%%%%%%%%%%%%%%%%%%%%%%%%%%%%%%%%%%%%%%%%%%%%%%%%%%%%%%%%%%%%%
\subsection{Include by Input}
\label{sec:input}

Including child documents by |\include| has some restrictions by design.
Most notably, the content of a child document always occupies
its own set of pages; pages cannot be shared between child documents.
Usually, this behaviour makes perfect sense
because each child document contain an essential part of the document.
However, in some situations it may be desirable to compose
a document from a collection of parts
without having mandatory page breaks between then.
For this case, the package
provides a mechanism to include parts
by |\input| which can also be processed individually.
However, by construction this mechanism
requires manual handling of the content to be output.

%%%%%%%%%%%%%%%%%%%%%%%%%%%%%%%%%%%%%%%%
\DescribeMacro{\ifchilddocmanual}
The main file should be prepared as usual, see \secref{sec:include}.
However, the document body must make a distinction
between processing of an individual part and of the main document, e.g.:
%
\begin{center}
\begin{tabular}{l}
|\ifchilddocmanual|\\
|\input{\childdocname}|\\
|\||else|\\
\textit{document body with }|\input{|\textit{part}|}|\\
|\||fi|
\end{tabular}
\end{center}
%
The conditional |\ifchilddocmanual| is true whenever
a part to be included by |\input| is being compiled,
and the name of the part is stored in |\childdocname|.

%%%%%%%%%%%%%%%%%%%%%%%%%%%%%%%%%%%%%%%%
\DescribeMacro{\childdocby}
Each part to be included by |\input| should start with:
%
\begin{center}
\begin{tabular}{l}
|\input{childdoc.def}|\\
|\childdocby{|\textit{main}|}|\\
\end{tabular}
\end{center}
%
The directive |\childdocby| is similar to |\childdocof|
described in \secref{sec:include},
but the subsequent selection of content must be done manually.
To that end, both |\ifchilddoc| and |\ifchilddocmanual|
will be true upon processing of a part,
and the name of the part is stored in |\childdocname|.
Note that |\jobname| will be set to the filename of the current part
so that each part receives an individual |.aux| file
that does not interfere with the |.aux| file(s) of the main document.
This behaviour can be altered by the alternative form
|\childdocby[*]{|\textit{main}|}| (with a non-empty optional argument)
which uses the |.aux| file of the main document
by setting |\jobname| to \textit{main}.

%%%%%%%%%%%%%%%%%%%%%%%%%%%%%%%%%%%%%%%%%%%%%%%%%%%%%%%%%%%%%%%%%%%%%%%%%%%%%%%%
\subsection{Driver Development}
\label{sec:driver}

The \textsf{childdoc} mechanism can also be use for the development
of definition files such as \LaTeX{} styles or classes.
This case differs from the above setup with multiple parts
included by |\include| in that no |\includeonly| should be invoked.
This can be achieved by starting the include file
(before |\ProvidesPackage|) with:
%
\begin{center}
\begin{tabular}{l}
|\input{childdoc.def}|\\
|\childdocforward{|\textit{main}|}|\\
\end{tabular}
\end{center}
%
or alternatively with:
%
\begin{center}
\begin{tabular}{l}
|\input{childdoc.def}|\\
|\childdocby{|\textit{main}|}|\\
\end{tabular}
\end{center}
%
Both forms have slightly different effects as described above.
The main file is prepared as usual, see \secref{sec:include}.

%%%%%%%%%%%%%%%%%%%%%%%%%%%%%%%%%%%%%%%%%%%%%%%%%%%%%%%%%%%%%%%%%%%%%%%%%%%%%%%%
\subsection{Legacy Detection}
\label{sec:detection}

The directive |\childdocmain| in the main file can detect
whether the complete document or merely a child is to be compiled
even without using the directive |\childdocof|.
This method is deprecated because it is less robust
and there is no compelling reason to use it;
it is merely provided for backward compatibility
and it may be removed in future versions.

If the detection mechanism is to be used,
it is mandatory to correctly specify
the filename of the main file as the argument of |\childdocmain|:
%
\begin{center}
\begin{tabular}{l}
|\input{childdoc.def}|\\
|\childdocmain{|\textit{main}|}|\\
\end{tabular}
\end{center}
%
If |\jobname| does not match the argument \textit{main} of |\childdocmain|,
it is assumed that |\jobname| points to the child file to be compiled.
When using |\childdocmain| with the main file specified as argument,
it suffices to start a child file
with just |\input{|\textit{main}|}|
without loading of the package and using |\childdocof|.
If instead all processing is done
with the appropriate \textsf{childdoc} directives,
the argument of \textit{main} of |\childdocmain| can be empty.

An alternative version of the command line processing described
in \secref{sec:commandline} using the detection mechanism reads:
%
\begin{center}
|... -jobname "|\textit{target}|" "|[\textit{flags}]%
[|\def\jobname{|\textit{dest}|}|]|\input{|\textit{main}|}"|
\end{center}

%%%%%%%%%%%%%%%%%%%%%%%%%%%%%%%%%%%%%%%%%%%%%%%%%%%%%%%%%%%%%%%%%%%%%%%%%%%%%%%%
\subsection{Manual Code}
\label{sec:manual}

In case one cannot be certain whether the definitions file |childdoc.def|
is installed on the target \TeX{} distribution
and one prefers not to ship it,
it is conceivable to paste a few relevant commands into the sources.

To that end, drop all statements |\input{childdoc.def}|
and perform the replacements as outlined below.
Instead of |\childdocmain{|\textit{main}|}| add the following code
to the top of the main file:
%
\begin{center}
\begin{tabular}{l}
|\||ifdefined\childdocname\endinput\||fi\newif\ifchilddoc|\\
|\edef\childdocname{\scantokens\expandafter{\jobname\noexpand}}|\\
|\def\childdocmain{|\textit{main}|}\||ifx\childdocmain\childdocname\||else|\\
|\childdoctrue\includeonly{\childdocname}\let\jobname\childdocmain\||fi|\\
\end{tabular}
\end{center}
%
Instead of |\childdocof{|\textit{main}|}| just include the main file
at the top of each child file:
%
\begin{center}
|\input{|\textit{main}|}|
\end{center}
%
A simple redirection |\childdocforward{|\textit{dest}|}| is achieved by:
%
\begin{center}
|\def\jobname{|\textit{dest}|}\input{\jobname}|
\end{center}
%
The redirection with prefix
|\childdocforwardprefix[|\textit{prefix}|]{|\textit{dest}|}|
is accomplished by:
%
\begin{center}
\begin{tabular}{l}
|{\edef\jobname{\scantokens\expandafter{\jobname\noexpand}}|\\
|\def\redirectjob |\textit{prefix}|#1~~~{\gdef\jobname{|\textit{dest}|#1}}|\\
|\expandafter\redirectjob\jobname~~~}\input{\jobname}|
\end{tabular}
\end{center}

In an alternative approach,
child documents can be compiled by a specific command line
without additional code or specific definitions:
%
\begin{center}
|... -jobname "|\textit{target}|" "|[\textit{flags}]%
|\includeonly{|\textit{dest}|}\input{|\textit{main}|}"|
\end{center}
%

%%%%%%%%%%%%%%%%%%%%%%%%%%%%%%%%%%%%%%%%%%%%%%%%%%%%%%%%%%%%%%%%%%%%%%%%%%%%%%%%
%%%%%%%%%%%%%%%%%%%%%%%%%%%%%%%%%%%%%%%%%%%%%%%%%%%%%%%%%%%%%%%%%%%%%%%%%%%%%%%%
\section{Information}

%%%%%%%%%%%%%%%%%%%%%%%%%%%%%%%%%%%%%%%%%%%%%%%%%%%%%%%%%%%%%%%%%%%%%%%%%%%%%%%%
\subsection{Copyright}

Copyright \copyright{} 2017--2018 Niklas Beisert

This work may be distributed and/or modified under the
conditions of the \LaTeX{} Project Public License, either version 1.3
of this license or (at your option) any later version.
The latest version of this license is in
  \url{http://www.latex-project.org/lppl.txt}
and version 1.3 or later is part of all distributions of \LaTeX{}
version 2005/12/01 or later.

This work has the LPPL maintenance status `maintained'.

The Current Maintainer of this work is Niklas Beisert.

This work consists of the files |README.txt|, |childdoc.ins| and |childdoc.dtx|
as well as the derived files |childdoc.def|, |cdocsamp.tex|
with |cdocsch1.tex|, |cdocsch2.tex|, |cdocspt3.tex|, |cdocspt4.tex|,
|cdocsdrf.tex|, |cdocsfn1.tex|, |cdocsfn2.tex|
as well as |childdoc.pdf|.

%%%%%%%%%%%%%%%%%%%%%%%%%%%%%%%%%%%%%%%%%%%%%%%%%%%%%%%%%%%%%%%%%%%%%%%%%%%%%%%%
\subsection{Files and Installation}

The package consists of the files:
%
\begin{center}
\begin{tabular}{ll}
    |README.txt|   & readme file \\
    |childdoc.ins| & installation file \\
    |childdoc.dtx| & source file \\
    |childdoc.def| & definition file \\
    |cdocsamp.tex| & sample main file \\
    |cdocsch1.tex| & sample include file \\
    |cdocsch2.tex| & sample include file \\
    |cdocspt3.tex| & sample part file \\
    |cdocspt4.tex| & sample part file \\
    |cdocsdrf.tex| & sample redirection file \\
    |cdocsfn1.tex| & sample redirection file \\
    |cdocsfn2.tex| & sample redirection file \\
    |childdoc.pdf| & manual
\end{tabular}
\end{center}
%
The distribution consists of the files
|README.txt|, |childdoc.ins| and |childdoc.dtx|.
%
\begin{itemize}
\item
Run (pdf)\LaTeX{} on |childdoc.dtx|
to compile the manual |childdoc.pdf| (this file).
\item
Run \LaTeX{} on |childdoc.ins| to create the definitions file |childdoc.def|
and the sample |cdocsamp.tex| with include files
|cdocsch1.tex|, |cdocsch2.tex|, |cdocspt3.tex|, |cdocspt4.tex|,
|cdocsdrf.tex|, |cdocsfn1.tex|, |cdocsfn2.tex|.
Then copy the file |childdoc.def| to an appropriate directory of your \LaTeX{}
distribution, e.g.\ \textit{texmf-root}|/tex/latex/childdoc|.
\end{itemize}

%%%%%%%%%%%%%%%%%%%%%%%%%%%%%%%%%%%%%%%%%%%%%%%%%%%%%%%%%%%%%%%%%%%%%%%%%%%%%%%%
\subsection{Related CTAN Packages}

There are several other packages which offer a similar functionality:
%
\begin{itemize}
\item
The packages
\href{http://ctan.org/pkg/docmute}{\textsf{docmute}},
\href{http://ctan.org/pkg/includex}{\textsf{includex}} and
\href{http://ctan.org/pkg/standalone}{\textsf{standalone}}
provide commands to include only the document body of
a child file thus allowing both files to be compiled individually.
\item
The packages \href{http://ctan.org/pkg/subdocs}{\textsf{subdocs}}
and \href{http://ctan.org/pkg/subfiles}{\textsf{subfiles}}
provide structures in which the main and child documents can be
encapsulated and allowing them to be compiled individually.
The inclusion mechanism is different from the conventional |\include|.
\item
The package \href{http://ctan.org/pkg/combine}{\textsf{combine}}
is an elaborate solution to combine several documents into one.
\end{itemize}
%
See also the CTAN topic \href{http://ctan.org/topic/subdocs}{\textsf{subdocs}}
for further related packages.
The present package differs from the above solutions in that
a document structure constructed with the conventional |\include| mechanism
just needs two extra commands at the top of every file
such that all constituent files can be compiled individually.

%%%%%%%%%%%%%%%%%%%%%%%%%%%%%%%%%%%%%%%%%%%%%%%%%%%%%%%%%%%%%%%%%%%%%%%%%%%%%%%%
%\subsection{Feature Suggestions}
%
%The following is a list of features which may be useful for future
%versions of this package:
%%
%\begin{itemize}
%\item
%\ldots
%\end{itemize}

%%%%%%%%%%%%%%%%%%%%%%%%%%%%%%%%%%%%%%%%%%%%%%%%%%%%%%%%%%%%%%%%%%%%%%%%%%%%%%%%
\subsection{Revision History}

%%%%%%%%%%%%%%%%%%%%%%%%%%%%%%%%%%%%%%%%
\paragraph{v2.0:} 2018/12/30

\begin{itemize}
\item
immediate forward processing
\item
added |\childdocby| mechanism
\item
manual restructured
\end{itemize}

%%%%%%%%%%%%%%%%%%%%%%%%%%%%%%%%%%%%%%%%
\paragraph{v1.6:} 2018/01/17

\begin{itemize}
\item
application for development of include files
\item
corrections to manual
\end{itemize}

%%%%%%%%%%%%%%%%%%%%%%%%%%%%%%%%%%%%%%%%
\paragraph{v1.5:} 2017/05/21

\begin{itemize}
\item
more complete structuring introduced
\item
|\childdocof| introduced
\item
|\childdoc| renamed to |\childdocmain|
\item
|\childredirect| renamed to |\childdocforward| and |\childdocforwardprefix|
and functionality expanded
\end{itemize}

%%%%%%%%%%%%%%%%%%%%%%%%%%%%%%%%%%%%%%%%
\paragraph{v1.0:} 2017/04/27

\begin{itemize}
\item
manual and install package
\item
first version published on CTAN
\end{itemize}

%%%%%%%%%%%%%%%%%%%%%%%%%%%%%%%%%%%%%%%%
\paragraph{v0.6:} 2017/04/26

\begin{itemize}
\item
redirection mechanism added
\end{itemize}

%%%%%%%%%%%%%%%%%%%%%%%%%%%%%%%%%%%%%%%%
\paragraph{v0.5:} 2017/04/26

\begin{itemize}
\item
functionality in definition file
\end{itemize}


%%%%%%%%%%%%%%%%%%%%%%%%%%%%%%%%%%%%%%%%%%%%%%%%%%%%%%%%%%%%%%%%%%%%%%%%%%%%%%%%
%%%%%%%%%%%%%%%%%%%%%%%%%%%%%%%%%%%%%%%%%%%%%%%%%%%%%%%%%%%%%%%%%%%%%%%%%%%%%%%%
%%%%%%%%%%%%%%%%%%%%%%%%%%%%%%%%%%%%%%%%%%%%%%%%%%%%%%%%%%%%%%%%%%%%%%%%%%%%%%%%
\appendix

\settowidth\MacroIndent{\rmfamily\scriptsize 000\ }

 \DocInput{childdoc.dtx}

\end{document}
%</driver>
% \fi
%
% %%%%%%%%%%%%%%%%%%%%%%%%%%%%%%%%%%%%%%%%%%%%%%%%%%%%%%%%%%%%%%%%%%%%%%%%%%%%%%
% %%%%%%%%%%%%%%%%%%%%%%%%%%%%%%%%%%%%%%%%%%%%%%%%%%%%%%%%%%%%%%%%%%%%%%%%%%%%%%
% \section{Sample}
%\iffalse
%<*samplemain>
%\fi
%
% The following presents a sample document
% with two chapters, two parts, a title page,
% a compile flag as well as three forwarding files to set the flag.
% It consists of eight |.tex| files:
% \begin{center}
% \begin{tabular}{ll}
% |cdocsamp.tex|&main file\\
% |cdocsch1.tex|&include file for chapter 1\\
% |cdocsch2.tex|&include file for chapter 2\\
% |cdocspt3.tex|&include file for part 3\\
% |cdocspt4.tex|&include file for part 4\\
% |cdocsdrf.tex|&forwarding file for main file in draft mode\\
% |cdocsfi1.tex|&forwarding file for final version of chapter 1\\
% |cdocsfi2.tex|&forwarding file for final version of chapter 2\\
% \end{tabular}
% \end{center}
% Each of the eight files can be compiled directly by the \LaTeX{} compiler.
%
% %%%%%%%%%%%%%%%%%%%%%%%%%%%%%%%%%%%%%%
% \paragraph{Main File.}
%
% The main file is called |cdocsamp.tex|.
%
% Load the \textsf{childdoc} definitions and
% declare the filename for the main document:
%    \begin{macrocode}
\input{childdoc.def}
\childdocmain{}
%    \end{macrocode}

% Optional override for |\version| flag:
%    \begin{macrocode}
%%\ifchilddoc\else\providecommand{\version}{draft}\fi
%    \end{macrocode}

% Define the default values for the |\version| flag
% (|final| for the main file and |draft| for childs):
%    \begin{macrocode}
\ifchilddoc
\providecommand{\version}{draft}
\else
\providecommand{\version}{final}
\fi
%    \end{macrocode}

% Load the standard document class:
%    \begin{macrocode}
\documentclass[12pt]{article}
%    \end{macrocode}

% Start the document body:
%    \begin{macrocode}
\begin{document}
%    \end{macrocode}

% Declare a title page.
% Print title, part of document being processed and version flag:
%    \begin{macrocode}
\addtocounter{page}{-1}
\begin{center}
{\LARGE\bfseries{}childdoc example\par}
\vspace{1cm}
\ifchilddoc
\ifchilddocmanual part\else chapter\fi:
`\childdocname' of `\childdocjob'\par
\else
main document: `\childdocjob'\par
\fi
version: \version\par
\end{center}
\newpage
%    \end{macrocode}

% Manually include selected file,
% otherwise process as usual:
%    \begin{macrocode}
\ifchilddocmanual
\section*{part `\childdocname'}
\input{\childdocname}
\else
%    \end{macrocode}

% Include the two chapters:
%    \begin{macrocode}
\include{cdocsch1}
\include{cdocsch2}
%    \end{macrocode}

% Include the two parts unless only chapters should be displayed:
%    \begin{macrocode}
\ifchilddoc\else
\section{part three}
\input{cdocspt3}
\section{part four}
\input{cdocspt4}
\fi
%    \end{macrocode}

% Process as usual until here:
%    \begin{macrocode}
\fi
%    \end{macrocode}

% End of document body:
%    \begin{macrocode}
\end{document}
%    \end{macrocode}
%\iffalse
%</samplemain>
%\fi
%
% %%%%%%%%%%%%%%%%%%%%%%%%%%%%%%%%%%%%%%
% \paragraph{Chapter Include Files.}
%
% The include files are called |cdocsch1.tex| and |cdocsch2.tex|.
%
%\iffalse
%<*samplechap1|samplechap2>
%\fi

% Optional override for |\version| flag:
%    \begin{macrocode}
%%\providecommand{\version}{final}
%    \end{macrocode}

% Include the main document:
%    \begin{macrocode}
\input{childdoc.def}
\childdocof{cdocsamp}
%    \end{macrocode}

%\iffalse
%</samplechap1|samplechap2>
%\fi
%
%\iffalse
%<*samplechap1>
%\fi
% Some text for chapter 1:
%    \begin{macrocode}
\section{one}
some text in chapter one
%    \end{macrocode}

%\iffalse
%</samplechap1>
%\fi
% Some text for chapter 2:
%\iffalse
%<*samplechap2>
%\fi
%    \begin{macrocode}
\section{two}
more text in chapter two
%    \end{macrocode}

%\iffalse
%</samplechap2>
%\fi
%
% %%%%%%%%%%%%%%%%%%%%%%%%%%%%%%%%%%%%%%
% \paragraph{Part Include Files.}
%
% The include files are called |cdocspt3.tex| and |cdocspt4.tex|.
%
%\iffalse
%<*samplepart3|samplepart4>
%\fi

% Optional override for |\version| flag:
%    \begin{macrocode}
%%\providecommand{\version}{final}
%    \end{macrocode}

% Include the main document:
%    \begin{macrocode}
\input{childdoc.def}
\childdocby{cdocsamp}
%    \end{macrocode}

%\iffalse
%</samplepart3|samplepart4>
%\fi
%
%\iffalse
%<*samplepart3>
%\fi
% Some text for part 3:
%    \begin{macrocode}
some text in part three
%    \end{macrocode}

%\iffalse
%</samplepart3>
%\fi
% Some text for part 4:
%\iffalse
%<*samplepart4>
%\fi
%    \begin{macrocode}
more text in part four
%    \end{macrocode}

%\iffalse
%</samplepart4>
%\fi
%
% %%%%%%%%%%%%%%%%%%%%%%%%%%%%%%%%%%%%%%
% \paragraph{Forwarding for a Complete Draft.}
%
% The following forwarding file |cdocsdrf.tex|
% compiles the main document in draft mode:
%\iffalse
%<*sampledraft>
%\fi
%    \begin{macrocode}
\def\version{draft}
\input{childdoc.def}
\childdocforward{cdocsamp}
%    \end{macrocode}

%\iffalse
%</sampledraft>
%\fi
%
% %%%%%%%%%%%%%%%%%%%%%%%%%%%%%%%%%%%%%%
% \paragraph{Forwarding for Final Version of the Chapters.}
%
% The following forwarding files |cdocsfn1.tex| and |cdocsfn2.tex|
% (with identical content)
% compile the final versions of the child documents
% |cdocsch1.tex| and |cdocsch2.tex|, respectively:
%\iffalse
%<*samplefinal>
%\fi
%    \begin{macrocode}
\def\version{final}
\input{childdoc.def}
\childdocforwardprefix[cdocsamp]{cdocsfn}{cdocsch}
%    \end{macrocode}

%\iffalse
%</samplefinal>
%\fi
%
% %%%%%%%%%%%%%%%%%%%%%%%%%%%%%%%%%%%%%%
% \paragraph{Command Line Processing.}
%
% The following three command lines generate the output files
% |cdocscld|, |cdocscl1| and |cdocscl2|
% which should be identical to
% |cdocsdrf|, |cdocsch1| and |cdocsfn2|, respectively:
% \begin{center}
% \begin{tabular}{l}
% |latex -jobname cdocscld \|\\
% |  "\def\version{draft}\input{childdoc.def}\childdocforward{cdocsamp}"|\\
% |latex -jobname cdocscl1 \|\\
% |  "\input{childdoc.def}\childdocforward[cdocsamp]{cdocsch1}"|\\
% |latex -jobname cdocscl2 \|\\
% |  "\def\version{final}\input{childdoc.def}\childdocforward{cdocsch2}"|
% \end{tabular}
% \end{center}
% Note that the trailing backslash on each first line
% merely continues the input to the second line
% (for convenient cut ant paste).
% Furthermore, the command |latex| can be replaced by any
% of its alternative versions such as |pdflatex|.
%
% %%%%%%%%%%%%%%%%%%%%%%%%%%%%%%%%%%%%%%%%%%%%%%%%%%%%%%%%%%%%%%%%%%%%%%%%%%%%%%
% %%%%%%%%%%%%%%%%%%%%%%%%%%%%%%%%%%%%%%%%%%%%%%%%%%%%%%%%%%%%%%%%%%%%%%%%%%%%%%
% \section{Implementation}
%\iffalse
%<*package>
%\fi
%
% This section describes the definitions file |childdoc.def|.

% The definitions cannot be loaded using |\usepackage| or |\RequirePackage|
% which has a mechanism to prevent loading a style file more than once.
% When loading the definitions by means of |\input|
% multiple instances have to be prevented manually:
%\iffalse
%This code needs to be before the `\ProvidesFile' directive
%which is defined at the beginning of this file.
%Therefore it is also placed there and commented out here.
%</package>
%<*discard>
%\fi
%    \begin{macrocode}
\ifdefined\childdocmain\endinput\fi
%    \end{macrocode}
%\iffalse
%</discard>
%<*package>
%\fi
%
% \macro{\ifchilddoc}
% \macro{\ifchilddocmanual}
% The conditional |\ifchilddoc| tells whether a
% child (true) or main (false) document is being compiled.
% The conditional |\ifchilddocmanual| tells whether
% the |\includeonly| mechanism is used (false) or
% the selection of child files must be performed manually (true).
% The definitions initialise to false:
%    \begin{macrocode}
\newif\ifchilddoc
\newif\ifchilddocmanual
%    \end{macrocode}

% \macro{\childdocname}
% \macro{\childdocjob}
% The macro |\childdocname| stores the name of the main document
% to be compiled. The macro |\childdocjob| stores the name of
% the document on which the \LaTeX{} compiler was originally invoked.
% The content of |\jobname| cannot be compared
% to filenames specified in the source due to different catcodes.
% The following code rescans |\jobname|, stores the result
% in |\childdocname| and saves a copy in |\childdocjob|:
%    \begin{macrocode}
\edef\childdocname{\scantokens\expandafter{\jobname\noexpand}}
\let\childdocjob\childdocname
%    \end{macrocode}

% \macro{\childdocdisable}
% The macro |\childdocdisable| prevents the main file
% from being processed more than once.
% At this stage, the main document command |\childdocmain|
% is assumed to be called once again where it should do nothing.
% Any subsequent call to it should prevent
% a secondary processing of the main document
% It overwrites the forwarding commands
% |\childdocof| and |\childdocforward|
% with empty macros to prevent further inclusions of the main document:
%    \begin{macrocode}
\newcommand{\childdocdisable}
{
  \renewcommand{\childdocmain}[1]{\renewcommand{\childdocmain}[1]{\endinput}}
  \renewcommand{\childdocof}[1]{}
  \renewcommand{\childdocby}[2][]{}
  \renewcommand{\childdocforward}[2][]{}
  \renewcommand{\childdocdisable}{}
}
%    \end{macrocode}

% \macro{\childdocmain}
% The macro |\childdocmain| is to be called at the top of the main file
% with nothing or the main filename (without extension) as argument.
% First, it breaks loops.
% If the argument is not empty and does not match |\childdocname|
% (which is set by the first inclusion of |childdoc.def|),
% |\ifchilddoc| is set to true, |\includeonly| is applied to the child file
% and |\jobname| is set to the main file
% (for proper handling of |.aux| files):
%    \begin{macrocode}
\newcommand{\childdocmain}[1]
{
  \childdocdisable\childdocmain{}
  \if?#1?\else
    \begingroup
      \def\childdoctmp{#1}
      \ifx\childdoctmp\childdocname
        \def\childdoctmp{}
      \else
        \def\childdoctmp
        {
          \childdoctrue
          \includeonly{\childdocname}
          \def\childdocjob{#1}
          \def\jobname{#1}
        }
      \fi
      \expandafter
    \endgroup
    \childdoctmp
  \fi
}
%    \end{macrocode}

% \macro{\childdocof}
% The command |\childdocof| redirects
% compilation to the main file |#1|.
%    \begin{macrocode}
\newcommand{\childdocof}[1]
{
  \childdocdisable
  \childdoctrue
  \includeonly{\childdocname}
  \def\jobname{#1}
  \def\childdocjob{#1}
  \input{#1}
}
%    \end{macrocode}

% \macro{\childdocby}
% The command |\childdocby| ....
%    \begin{macrocode}
\newcommand{\childdocby}[2][]
{
  \childdocdisable
  \childdoctrue
  \childdocmanualtrue
  \if?#1?\else
    \def\jobname{#2}
  \fi
  \def\childdocjob{#2}
  \input{#2}
  \endinput
}
%    \end{macrocode}

% \macro{\childdocforward}
% The command |\childdocforward| redirects
% compilation to the main file or
% (if the optional argument is given) a child file.
% Parameters are set as if the main file
% or a child file starting with |\childdocof| was compiled.
% Then compilation is handed over to the main file:
%    \begin{macrocode}
\newcommand{\childdocforward}[2][]
{
  \begingroup
    \if?#1?
      \def\childdoctmp
      {
        \def\childdocname{#2}
        \def\childdocjob{#2}
        \def\jobname{#2}
        \input{#2}
        \endinput
      }
    \else
      \def\childdoctmp
      {
        \childdocdisable
        \def\childdocname{#2}
        \childdoctrue
        \includeonly{#2}
        \def\childdocjob{#1}
        \def\jobname{#1}
        \input{#1}
        \endinput
      }
    \fi
    \expandafter
  \endgroup
  \childdoctmp
}
%    \end{macrocode}

% \macro{\childdocforwardprefix}
% The command |\childdocforwardprefix| redirects
% compilation to the main or a child file by means of a pattern.
% The prefix |#1| in the current filename is replaced by |#2|
% and the suffix of the current filename is kept
% (it is assumed that the filename does not contain the substring `|~~~|'
% which is used as a delimiter).
% Compilation is handed over to the new file by |\childdocforward|:
%    \begin{macrocode}
\newcommand{\childdocforwardprefix}[3][]
{
  \begingroup
    \def\childdocextract #2##1~~~{\def\childdoctmp{\childdocforward[#1]{#3##1}}}
    \expandafter\childdocextract\childdocname~~~
    \expandafter
  \endgroup
  \childdoctmp
}
%    \end{macrocode}

% \macro{\childdoc}
% The deprecated macro |\childdoc| is a legacy version of |\childdocmain|:
%    \begin{macrocode}
\newcommand{\childdoc}{\childdocmain}
%    \end{macrocode}

% \macro{\childdocredirect}
% The deprecated macro |\childdocredirect| is a legacy version
% of |\childdocforward| and |\childdocforwardprefix|:
%    \begin{macrocode}
\newcommand{\childdocredirect}[2][]
{
  \begingroup
    \if?#1?
      \def\childdoctmp{\childdocforward{#2}}
    \else
      \def\childdoctmp{\childdocforwardprefix{#1}{#2}}
    \fi
    \expandafter
  \endgroup
  \childdoctmp
}
%    \end{macrocode}

%\iffalse
%</package>
%\fi
%
\endinput
|\\
|\childdocforward[|\textit{main}|]{|\textit{dest}|}|\\
\end{tabular}
\end{center}
%
The argument \textit{dest} is the destination file
(without extension).
It should be the main file or one of the child files.
Note that further \textsf{childdoc} directives
such as |\childdocof| and |\childdocforward|
in the indicated file will be processed in this form.
The optional argument \textit{main}
passes on directly to the main file \textit{main}
while pretending to compile the child \textit{dest}.
This form behaves as if \textit{dest}
issues |\childdocof{|\textit{main}|}| right away,
and no further \textsf{childdoc} directives will be processed.

%%%%%%%%%%%%%%%%%%%%%%%%%%%%%%%%%%%%%%%%
\DescribeMacro{\...prefix}
In the alternative form |\childdocforwardprefix|,
%
\begin{center}
\begin{tabular}{l}
|% \iffalse
%
% childdoc.dtx Copyright (C) 2017-2018 Niklas Beisert
%
% This work may be distributed and/or modified under the
% conditions of the LaTeX Project Public License, either version 1.3
% of this license or (at your option) any later version.
% The latest version of this license is in
%   http://www.latex-project.org/lppl.txt
% and version 1.3 or later is part of all distributions of LaTeX
% version 2005/12/01 or later.
%
% This work has the LPPL maintenance status `maintained'.
%
% The Current Maintainer of this work is Niklas Beisert.
%
% This work consists of the files childdoc.dtx and childdoc.ins
% and the derived files childdoc.def and cdocsamp.tex with
% cdocsch1.tex, cdocsch2.tex, cdocsdrf.tex, cdocsfn1.tex, cdocsfn2.tex.
%
%<package>\ifdefined\childdocmain\endinput\fi
%<package>\ProvidesFile{childdoc.def}[2018/12/30 v2.0 child document driver]
%<samplemain>\ProvidesFile{cdocsamp.tex}[2018/12/30 v2.0 sample for childdoc]
%<*driver>
%\ProvidesFile{childdoc.drv}[2018/12/30 v2.0 childdoc reference manual file]
\PassOptionsToClass{10pt,a4paper}{article}
\documentclass{ltxdoc}

\usepackage[margin=35mm]{geometry}
\usepackage{hyperref}
\usepackage{hyperxmp}
\usepackage[usenames]{color}

\hypersetup{colorlinks=true}
\hypersetup{pdfstartview=FitH}
\hypersetup{pdfpagemode=UseNone}
\hypersetup{pdfsource={}}
\hypersetup{pdflang={en-UK}}
\hypersetup{pdfcopyright={Copyright 2017-2018 Niklas Beisert.
  This work may be distributed and/or modified under the
  conditions of the LaTeX Project Public License, either version 1.3
  of this license or (at your option) any later version.}}
\hypersetup{pdflicenseurl={http://www.latex-project.org/lppl.txt}}
\hypersetup{pdfcontactaddress={ETH Zurich, ITP, HIT K,
  Wolfgang-Pauli-Strasse 27}}
\hypersetup{pdfcontactpostcode={8093}}
\hypersetup{pdfcontactcity={Zurich}}
\hypersetup{pdfcontactcountry={Switzerland}}
\hypersetup{pdfcontactemail={nbeisert@itp.phys.ethz.ch}}
\hypersetup{pdfcontacturl={http://people.phys.ethz.ch/\xmptilde nbeisert/}}

\newcommand{\secref}[1]{\hyperref[#1]{section \ref*{#1}}}

\parskip1ex
\parindent0pt
\let\olditemize\itemize
\def\itemize{\olditemize\parskip0pt}

\begin{document}

\title{The \textsf{childdoc} Package}
\hypersetup{pdftitle={The childdoc Package}}
\author{Niklas Beisert\\[2ex]
  Institut f\"ur Theoretische Physik\\
  Eidgen\"ossische Technische Hochschule Z\"urich\\
  Wolfgang-Pauli-Strasse 27, 8093 Z\"urich, Switzerland\\[1ex]
  \href{mailto:nbeisert@itp.phys.ethz.ch}
  {\texttt{nbeisert@itp.phys.ethz.ch}}}
\hypersetup{pdfauthor={Niklas Beisert}}
\hypersetup{pdfsubject={Manual for the LaTeX2e Package childdoc}}
\date{30 December 2018, \textsf{v2.0}}
\maketitle

\begin{abstract}\noindent
\textsf{childdoc} is a \LaTeXe{} package
that enables the direct compilation
of document sections included by |\include|
to individual files.
\end{abstract}

\begingroup
\parskip0ex
\tableofcontents
\endgroup

%%%%%%%%%%%%%%%%%%%%%%%%%%%%%%%%%%%%%%%%%%%%%%%%%%%%%%%%%%%%%%%%%%%%%%%%%%%%%%%%
%%%%%%%%%%%%%%%%%%%%%%%%%%%%%%%%%%%%%%%%%%%%%%%%%%%%%%%%%%%%%%%%%%%%%%%%%%%%%%%%
\section{Introduction}

\LaTeX{} provides a mechanism to structure a large document (such as a book)
into a main file and several child files (containing the chapters)
using the |\include| command.
This mechanism is beneficial for documents
which span hundreds of pages in order to
make the source file(s) more manageable.
Moreover, compilation can be restricted to
selected child files by means of the |\includeonly| command.
The latter feature can be used to reduce the compilation time while editing
(this was significantly more useful in the earlier days of \LaTeX{})
or to generate a smaller document which is easier to navigate.
Another application of |\includeonly| is to generate
documents consisting of selected parts of the complete document.

However, there are a few drawbacks of the plain |\include| mechanism:
\begin{itemize}
\item
The child files cannot be compiled on their own,
they can only be compiled via the main file.
A naive editing environment
(such as a text editor with an option
to have the current file processed by \LaTeX)
may require one to switch to the main file before compiling;
attempting to compile the child file produces errors.
\item
The main file must be modified (each time)
to adjust the |\includeonly| command
to the present needs. This easily leaves the main file in a messy state.
\item
The generated document will always carry the filename
of the main document. This is inconvenient if
several child files are to be compiled and
to be kept for distribution.
\end{itemize}

The present package provides a simple interface
to make child files individually compilable by \LaTeX{}.
Compiling a child file then has the same effect as compiling
the main file with an |\includeonly| command
to select the appropriate child.
Moreover the generated document will carry the name of the child
rather than the main file.
This resolves all three above issues.

This feature is meant to make the editing of books,
thesis documents and lecture notes somewhat more convenient.
However, the package can also be used efficiently for
composing a series of documents (such as exercise sheets)
which are typically distributed individually.
It then assists the author in generating the individual documents
(potentially in different versions)
as well as a document containing the collected series.
Another application is in developing style files
or other kinds of included material
where compilation of the style file could redirect
to a sample or test file.

%%%%%%%%%%%%%%%%%%%%%%%%%%%%%%%%%%%%%%%%%%%%%%%%%%%%%%%%%%%%%%%%%%%%%%%%%%%%%%%%
%%%%%%%%%%%%%%%%%%%%%%%%%%%%%%%%%%%%%%%%%%%%%%%%%%%%%%%%%%%%%%%%%%%%%%%%%%%%%%%%
\section{Usage}

First of all, the package \textsf{childdoc} is \emph{not} a standard
\LaTeXe{} |.sty| style file! Therefore it needs to be invoked in
a non-standard way.

%%%%%%%%%%%%%%%%%%%%%%%%%%%%%%%%%%%%%%%%%%%%%%%%%%%%%%%%%%%%%%%%%%%%%%%%%%%%%%%%
\subsection{Included Files}
\label{sec:include}

%%%%%%%%%%%%%%%%%%%%%%%%%%%%%%%%%%%%%%%%
\DescribeMacro{\childdocmain}
To use the package, add the commands
\begin{center}
\begin{tabular}{l}
|\input{childdoc.def}|\\
|\childdocmain{}|\\
\end{tabular}
\end{center}
at the very top of the main \LaTeX{} file,
in particular \emph{before} the |\documentclass| statement!
The argument of |\childdocmain| should be left empty
(but it must be present).

%%%%%%%%%%%%%%%%%%%%%%%%%%%%%%%%%%%%%%%%
\DescribeMacro{\childdocof}
Furthermore, add the commands
\begin{center}
\begin{tabular}{l}
|\input{childdoc.def}|\\
|\childdocof{|\textit{main}|}|\\
\end{tabular}
\end{center}
at the top of every child file \textit{child}
which is included by |\include{|\textit{child}|}|
from within the main file
(or at least for those files to be compiled individually).
The argument \textit{main} must be the filename of the main file.

There are a couple of
considerations in setting up the main and child documents:

%%%%%%%%%%%%%%%%%%%%%%%%%%%%%%%%%%%%%%%%
\paragraph{Restrictions.}

Please note the following restrictions:
\begin{itemize}
\item
|\childdocmain| must be called with one argument \textit{main}
to ensure compatibility with earlier version of the package.
It must either be empty (|\childdocmain{}|)
or precisely match the filename of the main file in which it is specified.
See \secref{sec:detection} for further information.
\item
The filename \textit{main} must be specified without the |.tex| extension.
\item
The filename \textit{main} is case sensitive
(even in case-insensitive file systems)
due to internal string comparison.
\item
The argument \textit{main} should be fully expanded, it cannot be a macro.
\item
Subdirectories and special characters should be avoided in filenames.
\item
The command |\childdocmain{|\textit{main}|}| must be followed by a whitespace.
It should not be followed immediately by another command
or by a comment mark `|%|'.
This is because the \TeX{} parser reads the token immediately following
the argument of |\childdocmain| and puts it
at the beginning of every child section;
however, a white\-space is ignored.
\end{itemize}

%%%%%%%%%%%%%%%%%%%%%%%%%%%%%%%%%%%%%%%%
\paragraph{Content of Main File.}

It is advisable to place all content in the child files included by |\include|.
Any output contained in the main file will appear in all child documents
unless suppressed manually;
it cannot be suppressed automatically by the |\includeonly| directive
and thus should normally be avoided.
A method to include some content in the main file
by means of conditional processing is described in \secref{sec:conditional}.

%%%%%%%%%%%%%%%%%%%%%%%%%%%%%%%%%%%%%%%%
\paragraph{Page Numbering.}

When only a part of the document is compiled,
the appropriate numbering of pages
(as well as other status parameters)
is determined from the |.aux| files.
The latter contain information from previous passes.
However this information needs to propagate through
all intermediate child documents.
Therefore the page numbering in child documents may well
be inconsistent until the complete document is compiled at least once.

A useful (if unconventional) way to always ensure a consistent
page numbering is to restart the numbering in each child document
and denote the pages by `\textit{child}|.|\textit{page}'
where \textit{child} represents the chapter/section number of the child file.
This can be achieved by the command
|\numberwithin{page}{|\textit{child}|}|
of the \textsf{amsmath} package
where \textit{child} can be |chapter| or |section|
depending on the chosen structuring.
Alternatively, one can modify the macro |\thepage| appropriately
and reset the counter |page| at the start of each child file.

%%%%%%%%%%%%%%%%%%%%%%%%%%%%%%%%%%%%%%%%%%%%%%%%%%%%%%%%%%%%%%%%%%%%%%%%%%%%%%%%
\subsection{Conditional Processing}
\label{sec:conditional}

The package provides a mechanism to compile different versions
of a document. To customise the versions further some conditional processing
can come in handy to distinguish which version is being compiled.
The package provides two macros to describe the compilation context:

%%%%%%%%%%%%%%%%%%%%%%%%%%%%%%%%%%%%%%%%
\DescribeMacro{\ifchilddoc}
The conditional |\ifchilddoc| distinguishes between the compilation of
child documents and the main document:
%
\begin{center}
|\ifchilddoc |\textit{child-code}| |[|\||else |\textit{main-code}]| \||fi|
\end{center}

%%%%%%%%%%%%%%%%%%%%%%%%%%%%%%%%%%%%%%%%
\DescribeMacro{\childdocname}
\DescribeMacro{\childdocjob}
The macro |\childdocname| contains the filename (without extension)
of the main or child file being processed.
Note that |\childdocjob| will always contain the name of the main file.

%%%%%%%%%%%%%%%%%%%%%%%%%%%%%%%%%%%%%%%%
\paragraph{Title Page.}

Conditional processing can be used to include a title or banner page
in the main document when proper precautions are taken.
Importantly, the code in the main file should ensure that the page counter
(as well as other status parameters which are stored in the |.aux| files)
takes the same value after the conditional processing.
Otherwise the page numbers may take divergent values
depending on which part is compiled.

For example, a title page could be declared by:
%
\begin{center}
\begin{tabular}{l}
|\ifchilddoc\||else|\\
|\addtocounter{page}{-1}|\\
\textit{code for title page}\\
|\newpage|\\
|\||fi|
\end{tabular}
\end{center}
%
A banner page for the child documents can be generated by:
%
\begin{center}
\begin{tabular}{l}
|\ifchilddoc|\\
|\addtocounter{page}{-1}|\\
\textit{code for banner page}\\
|\newpage|\\
|\||fi|
\end{tabular}
\end{center}
%
Here one could write a message such as:
\begin{center}
|This is the part \childdocname{} of \childdocjob{}.|
\end{center}

%%%%%%%%%%%%%%%%%%%%%%%%%%%%%%%%%%%%%%%%%%%%%%%%%%%%%%%%%%%%%%%%%%%%%%%%%%%%%%%%
\subsection{Flags}
\label{sec:flags}

The package makes it easy to generate different versions
of the main or child documents.
To this end compilation flags can be defined
and assigned different default values.
They will be particularly useful in conjunction
with the forwarding mechanism described in \secref{sec:forward}.

For example, it may be useful to have a flag |\version|
which can be set to |draft| or |final|.
The document source will contain some conditional code
depending on the value of |\version|.
Suppose further, the flag should default to |final| for the main file
and to |draft| for child files
which is a natural assignment for editing the document.
This is achieved by placing the following code
in the preamble of the main document
(below the |\childdocmain| directive):
%
\begin{center}
\begin{tabular}{l}
|\ifchilddoc|\\
|\providecommand{\version}{draft}|\\
|\||else|\\
|\providecommand{\version}{final}|\\
|\||fi|
\end{tabular}
\end{center}
%
The definition by |\providecommand| makes sure
that previous definitions are not overwritten.
Further statements |\providecommand{\version}{...}|
can thus be added before the above code to override it.

For the main file, one might add a line
(between |\childdocmain| and the above block)
%
\begin{center}
|%\ifchilddoc\||else\providecommand{\version}{draft}\||fi|
\end{center}
%
which can be uncommented to produce a draft version.
Likewise one can add a line to the very top of a child file
(above the |\childdocof{|\textit{main}|}| directive)
%
\begin{center}
|%\providecommand{\version}{final}|
\end{center}
%
which can be uncommented to produce the final version of this child document.

%%%%%%%%%%%%%%%%%%%%%%%%%%%%%%%%%%%%%%%%%%%%%%%%%%%%%%%%%%%%%%%%%%%%%%%%%%%%%%%%
\subsection{Forwarding}
\label{sec:forward}

Different versions of the main or child documents
using compilation flags as described in \secref{sec:flags}
can be (permanently) stored in different files
for convenient compilation, viewing and distribution.
To this end, the package defines a command
to pass on compilation to a different file:

%%%%%%%%%%%%%%%%%%%%%%%%%%%%%%%%%%%%%%%%
\DescribeMacro{\childdocforward}
The command |\childdocforward| redirects processing to
another source file:
%
\begin{center}
\begin{tabular}{l}
|\input{childdoc.def}|\\
|\childdocforward[|\textit{main}|]{|\textit{dest}|}|\\
\end{tabular}
\end{center}
%
The argument \textit{dest} is the destination file
(without extension).
It should be the main file or one of the child files.
Note that further \textsf{childdoc} directives
such as |\childdocof| and |\childdocforward|
in the indicated file will be processed in this form.
The optional argument \textit{main}
passes on directly to the main file \textit{main}
while pretending to compile the child \textit{dest}.
This form behaves as if \textit{dest}
issues |\childdocof{|\textit{main}|}| right away,
and no further \textsf{childdoc} directives will be processed.

%%%%%%%%%%%%%%%%%%%%%%%%%%%%%%%%%%%%%%%%
\DescribeMacro{\...prefix}
In the alternative form |\childdocforwardprefix|,
%
\begin{center}
\begin{tabular}{l}
|\input{childdoc.def}|\\
|\childdocforwardprefix[|\textit{main}|]{|\textit{prefix}|}{|\textit{dest}|}|
\end{tabular}
\end{center}
%
the destination file is determined by a pattern
depending on the current file:
To make this work, the current file must be called
`{\textit{prefix}\hspace{0.2em}\textit{suffix}}'
with \textit{prefix} matching precisely the argument.
Processing is then passed on to the file
`{\textit{dest}\hspace{0.2em}\textit{suffix}}'.
Surely, the same effect is achieved by
directly specifying the
argument `{\textit{dest}\hspace{0.2em}\textit{suffix}}'
in the first form.
However, that requires to set up a different file
for each child. With the alternative form of the command
all these files can have exactly the same content
which simplifies setting them up and maintaining them.

For example, the following file |draft.tex|
with a compilation flag |\version| as described in \secref{sec:flags}
compiles the main document as a draft:
%
\begin{center}
\begin{tabular}{l}
|\def\version{draft}|\\
|\input{childdoc.def}|\\
|\childdocforward{|\textit{main}|}|
\end{tabular}
\end{center}
%
Likewise, the following files |final|\textit{nn}|.tex|
compile the final version of the child document
|child|\textit{nn}|.tex|:
%
\begin{center}
\begin{tabular}{l}
|\def\version{final}|\\
|\input{childdoc.def}|\\
|\childdocforwardprefix{final}{child}|
\end{tabular}
\end{center}
%

Note that when several versions of a main file and/or of each child file
are to be generated, it may be convenient to set up a |Makefile| or
shell script to automatise the process.

%%%%%%%%%%%%%%%%%%%%%%%%%%%%%%%%%%%%%%%%%%%%%%%%%%%%%%%%%%%%%%%%%%%%%%%%%%%%%%%%
\subsection{Command Line Processing}
\label{sec:commandline}

The effect of redirection files can also be achieved by invoking
the \LaTeX{} compiler with a more elaborate command line.
Most conveniently this should be done as part
of a shell script or a |Makefile|.

When using \textsf{childdoc} in the main file, the following
command lines effectively perform a redirection
(note that depending on the shell being used,
backslashes may have to be doubled: `|\|' $\to$ `|\\|'):
%
\begin{center}
|... -jobname "|\textit{target}|" |\\|"|[\textit{flags}]%
|\input{childdoc.def}\childdocforward[|\textit{main}|]{|\textit{dest}|}"|
\end{center}
%
Here \textit{target} is the name of the output file,
\textit{main} is the name of the main file
and \textit{dest} is the name of the main or child file to be processed
(all filenames without extensions).
The optional argument \textit{main} can be omitted
if \textit{main} matches \textit{dest}.
Optionally, compilation \textit{flags} can be defined via |\def| commands.
This command line makes the \TeX{} engine believe
it is compiling the file \textit{target}
whose content is specified as the latter parameter.
The provided code then forwards the processing to
\textit{main} or \textit{dest} as described in \secref{sec:forward}.

%%%%%%%%%%%%%%%%%%%%%%%%%%%%%%%%%%%%%%%%%%%%%%%%%%%%%%%%%%%%%%%%%%%%%%%%%%%%%%%%
\subsection{Include by Input}
\label{sec:input}

Including child documents by |\include| has some restrictions by design.
Most notably, the content of a child document always occupies
its own set of pages; pages cannot be shared between child documents.
Usually, this behaviour makes perfect sense
because each child document contain an essential part of the document.
However, in some situations it may be desirable to compose
a document from a collection of parts
without having mandatory page breaks between then.
For this case, the package
provides a mechanism to include parts
by |\input| which can also be processed individually.
However, by construction this mechanism
requires manual handling of the content to be output.

%%%%%%%%%%%%%%%%%%%%%%%%%%%%%%%%%%%%%%%%
\DescribeMacro{\ifchilddocmanual}
The main file should be prepared as usual, see \secref{sec:include}.
However, the document body must make a distinction
between processing of an individual part and of the main document, e.g.:
%
\begin{center}
\begin{tabular}{l}
|\ifchilddocmanual|\\
|\input{\childdocname}|\\
|\||else|\\
\textit{document body with }|\input{|\textit{part}|}|\\
|\||fi|
\end{tabular}
\end{center}
%
The conditional |\ifchilddocmanual| is true whenever
a part to be included by |\input| is being compiled,
and the name of the part is stored in |\childdocname|.

%%%%%%%%%%%%%%%%%%%%%%%%%%%%%%%%%%%%%%%%
\DescribeMacro{\childdocby}
Each part to be included by |\input| should start with:
%
\begin{center}
\begin{tabular}{l}
|\input{childdoc.def}|\\
|\childdocby{|\textit{main}|}|\\
\end{tabular}
\end{center}
%
The directive |\childdocby| is similar to |\childdocof|
described in \secref{sec:include},
but the subsequent selection of content must be done manually.
To that end, both |\ifchilddoc| and |\ifchilddocmanual|
will be true upon processing of a part,
and the name of the part is stored in |\childdocname|.
Note that |\jobname| will be set to the filename of the current part
so that each part receives an individual |.aux| file
that does not interfere with the |.aux| file(s) of the main document.
This behaviour can be altered by the alternative form
|\childdocby[*]{|\textit{main}|}| (with a non-empty optional argument)
which uses the |.aux| file of the main document
by setting |\jobname| to \textit{main}.

%%%%%%%%%%%%%%%%%%%%%%%%%%%%%%%%%%%%%%%%%%%%%%%%%%%%%%%%%%%%%%%%%%%%%%%%%%%%%%%%
\subsection{Driver Development}
\label{sec:driver}

The \textsf{childdoc} mechanism can also be use for the development
of definition files such as \LaTeX{} styles or classes.
This case differs from the above setup with multiple parts
included by |\include| in that no |\includeonly| should be invoked.
This can be achieved by starting the include file
(before |\ProvidesPackage|) with:
%
\begin{center}
\begin{tabular}{l}
|\input{childdoc.def}|\\
|\childdocforward{|\textit{main}|}|\\
\end{tabular}
\end{center}
%
or alternatively with:
%
\begin{center}
\begin{tabular}{l}
|\input{childdoc.def}|\\
|\childdocby{|\textit{main}|}|\\
\end{tabular}
\end{center}
%
Both forms have slightly different effects as described above.
The main file is prepared as usual, see \secref{sec:include}.

%%%%%%%%%%%%%%%%%%%%%%%%%%%%%%%%%%%%%%%%%%%%%%%%%%%%%%%%%%%%%%%%%%%%%%%%%%%%%%%%
\subsection{Legacy Detection}
\label{sec:detection}

The directive |\childdocmain| in the main file can detect
whether the complete document or merely a child is to be compiled
even without using the directive |\childdocof|.
This method is deprecated because it is less robust
and there is no compelling reason to use it;
it is merely provided for backward compatibility
and it may be removed in future versions.

If the detection mechanism is to be used,
it is mandatory to correctly specify
the filename of the main file as the argument of |\childdocmain|:
%
\begin{center}
\begin{tabular}{l}
|\input{childdoc.def}|\\
|\childdocmain{|\textit{main}|}|\\
\end{tabular}
\end{center}
%
If |\jobname| does not match the argument \textit{main} of |\childdocmain|,
it is assumed that |\jobname| points to the child file to be compiled.
When using |\childdocmain| with the main file specified as argument,
it suffices to start a child file
with just |\input{|\textit{main}|}|
without loading of the package and using |\childdocof|.
If instead all processing is done
with the appropriate \textsf{childdoc} directives,
the argument of \textit{main} of |\childdocmain| can be empty.

An alternative version of the command line processing described
in \secref{sec:commandline} using the detection mechanism reads:
%
\begin{center}
|... -jobname "|\textit{target}|" "|[\textit{flags}]%
[|\def\jobname{|\textit{dest}|}|]|\input{|\textit{main}|}"|
\end{center}

%%%%%%%%%%%%%%%%%%%%%%%%%%%%%%%%%%%%%%%%%%%%%%%%%%%%%%%%%%%%%%%%%%%%%%%%%%%%%%%%
\subsection{Manual Code}
\label{sec:manual}

In case one cannot be certain whether the definitions file |childdoc.def|
is installed on the target \TeX{} distribution
and one prefers not to ship it,
it is conceivable to paste a few relevant commands into the sources.

To that end, drop all statements |\input{childdoc.def}|
and perform the replacements as outlined below.
Instead of |\childdocmain{|\textit{main}|}| add the following code
to the top of the main file:
%
\begin{center}
\begin{tabular}{l}
|\||ifdefined\childdocname\endinput\||fi\newif\ifchilddoc|\\
|\edef\childdocname{\scantokens\expandafter{\jobname\noexpand}}|\\
|\def\childdocmain{|\textit{main}|}\||ifx\childdocmain\childdocname\||else|\\
|\childdoctrue\includeonly{\childdocname}\let\jobname\childdocmain\||fi|\\
\end{tabular}
\end{center}
%
Instead of |\childdocof{|\textit{main}|}| just include the main file
at the top of each child file:
%
\begin{center}
|\input{|\textit{main}|}|
\end{center}
%
A simple redirection |\childdocforward{|\textit{dest}|}| is achieved by:
%
\begin{center}
|\def\jobname{|\textit{dest}|}\input{\jobname}|
\end{center}
%
The redirection with prefix
|\childdocforwardprefix[|\textit{prefix}|]{|\textit{dest}|}|
is accomplished by:
%
\begin{center}
\begin{tabular}{l}
|{\edef\jobname{\scantokens\expandafter{\jobname\noexpand}}|\\
|\def\redirectjob |\textit{prefix}|#1~~~{\gdef\jobname{|\textit{dest}|#1}}|\\
|\expandafter\redirectjob\jobname~~~}\input{\jobname}|
\end{tabular}
\end{center}

In an alternative approach,
child documents can be compiled by a specific command line
without additional code or specific definitions:
%
\begin{center}
|... -jobname "|\textit{target}|" "|[\textit{flags}]%
|\includeonly{|\textit{dest}|}\input{|\textit{main}|}"|
\end{center}
%

%%%%%%%%%%%%%%%%%%%%%%%%%%%%%%%%%%%%%%%%%%%%%%%%%%%%%%%%%%%%%%%%%%%%%%%%%%%%%%%%
%%%%%%%%%%%%%%%%%%%%%%%%%%%%%%%%%%%%%%%%%%%%%%%%%%%%%%%%%%%%%%%%%%%%%%%%%%%%%%%%
\section{Information}

%%%%%%%%%%%%%%%%%%%%%%%%%%%%%%%%%%%%%%%%%%%%%%%%%%%%%%%%%%%%%%%%%%%%%%%%%%%%%%%%
\subsection{Copyright}

Copyright \copyright{} 2017--2018 Niklas Beisert

This work may be distributed and/or modified under the
conditions of the \LaTeX{} Project Public License, either version 1.3
of this license or (at your option) any later version.
The latest version of this license is in
  \url{http://www.latex-project.org/lppl.txt}
and version 1.3 or later is part of all distributions of \LaTeX{}
version 2005/12/01 or later.

This work has the LPPL maintenance status `maintained'.

The Current Maintainer of this work is Niklas Beisert.

This work consists of the files |README.txt|, |childdoc.ins| and |childdoc.dtx|
as well as the derived files |childdoc.def|, |cdocsamp.tex|
with |cdocsch1.tex|, |cdocsch2.tex|, |cdocspt3.tex|, |cdocspt4.tex|,
|cdocsdrf.tex|, |cdocsfn1.tex|, |cdocsfn2.tex|
as well as |childdoc.pdf|.

%%%%%%%%%%%%%%%%%%%%%%%%%%%%%%%%%%%%%%%%%%%%%%%%%%%%%%%%%%%%%%%%%%%%%%%%%%%%%%%%
\subsection{Files and Installation}

The package consists of the files:
%
\begin{center}
\begin{tabular}{ll}
    |README.txt|   & readme file \\
    |childdoc.ins| & installation file \\
    |childdoc.dtx| & source file \\
    |childdoc.def| & definition file \\
    |cdocsamp.tex| & sample main file \\
    |cdocsch1.tex| & sample include file \\
    |cdocsch2.tex| & sample include file \\
    |cdocspt3.tex| & sample part file \\
    |cdocspt4.tex| & sample part file \\
    |cdocsdrf.tex| & sample redirection file \\
    |cdocsfn1.tex| & sample redirection file \\
    |cdocsfn2.tex| & sample redirection file \\
    |childdoc.pdf| & manual
\end{tabular}
\end{center}
%
The distribution consists of the files
|README.txt|, |childdoc.ins| and |childdoc.dtx|.
%
\begin{itemize}
\item
Run (pdf)\LaTeX{} on |childdoc.dtx|
to compile the manual |childdoc.pdf| (this file).
\item
Run \LaTeX{} on |childdoc.ins| to create the definitions file |childdoc.def|
and the sample |cdocsamp.tex| with include files
|cdocsch1.tex|, |cdocsch2.tex|, |cdocspt3.tex|, |cdocspt4.tex|,
|cdocsdrf.tex|, |cdocsfn1.tex|, |cdocsfn2.tex|.
Then copy the file |childdoc.def| to an appropriate directory of your \LaTeX{}
distribution, e.g.\ \textit{texmf-root}|/tex/latex/childdoc|.
\end{itemize}

%%%%%%%%%%%%%%%%%%%%%%%%%%%%%%%%%%%%%%%%%%%%%%%%%%%%%%%%%%%%%%%%%%%%%%%%%%%%%%%%
\subsection{Related CTAN Packages}

There are several other packages which offer a similar functionality:
%
\begin{itemize}
\item
The packages
\href{http://ctan.org/pkg/docmute}{\textsf{docmute}},
\href{http://ctan.org/pkg/includex}{\textsf{includex}} and
\href{http://ctan.org/pkg/standalone}{\textsf{standalone}}
provide commands to include only the document body of
a child file thus allowing both files to be compiled individually.
\item
The packages \href{http://ctan.org/pkg/subdocs}{\textsf{subdocs}}
and \href{http://ctan.org/pkg/subfiles}{\textsf{subfiles}}
provide structures in which the main and child documents can be
encapsulated and allowing them to be compiled individually.
The inclusion mechanism is different from the conventional |\include|.
\item
The package \href{http://ctan.org/pkg/combine}{\textsf{combine}}
is an elaborate solution to combine several documents into one.
\end{itemize}
%
See also the CTAN topic \href{http://ctan.org/topic/subdocs}{\textsf{subdocs}}
for further related packages.
The present package differs from the above solutions in that
a document structure constructed with the conventional |\include| mechanism
just needs two extra commands at the top of every file
such that all constituent files can be compiled individually.

%%%%%%%%%%%%%%%%%%%%%%%%%%%%%%%%%%%%%%%%%%%%%%%%%%%%%%%%%%%%%%%%%%%%%%%%%%%%%%%%
%\subsection{Feature Suggestions}
%
%The following is a list of features which may be useful for future
%versions of this package:
%%
%\begin{itemize}
%\item
%\ldots
%\end{itemize}

%%%%%%%%%%%%%%%%%%%%%%%%%%%%%%%%%%%%%%%%%%%%%%%%%%%%%%%%%%%%%%%%%%%%%%%%%%%%%%%%
\subsection{Revision History}

%%%%%%%%%%%%%%%%%%%%%%%%%%%%%%%%%%%%%%%%
\paragraph{v2.0:} 2018/12/30

\begin{itemize}
\item
immediate forward processing
\item
added |\childdocby| mechanism
\item
manual restructured
\end{itemize}

%%%%%%%%%%%%%%%%%%%%%%%%%%%%%%%%%%%%%%%%
\paragraph{v1.6:} 2018/01/17

\begin{itemize}
\item
application for development of include files
\item
corrections to manual
\end{itemize}

%%%%%%%%%%%%%%%%%%%%%%%%%%%%%%%%%%%%%%%%
\paragraph{v1.5:} 2017/05/21

\begin{itemize}
\item
more complete structuring introduced
\item
|\childdocof| introduced
\item
|\childdoc| renamed to |\childdocmain|
\item
|\childredirect| renamed to |\childdocforward| and |\childdocforwardprefix|
and functionality expanded
\end{itemize}

%%%%%%%%%%%%%%%%%%%%%%%%%%%%%%%%%%%%%%%%
\paragraph{v1.0:} 2017/04/27

\begin{itemize}
\item
manual and install package
\item
first version published on CTAN
\end{itemize}

%%%%%%%%%%%%%%%%%%%%%%%%%%%%%%%%%%%%%%%%
\paragraph{v0.6:} 2017/04/26

\begin{itemize}
\item
redirection mechanism added
\end{itemize}

%%%%%%%%%%%%%%%%%%%%%%%%%%%%%%%%%%%%%%%%
\paragraph{v0.5:} 2017/04/26

\begin{itemize}
\item
functionality in definition file
\end{itemize}


%%%%%%%%%%%%%%%%%%%%%%%%%%%%%%%%%%%%%%%%%%%%%%%%%%%%%%%%%%%%%%%%%%%%%%%%%%%%%%%%
%%%%%%%%%%%%%%%%%%%%%%%%%%%%%%%%%%%%%%%%%%%%%%%%%%%%%%%%%%%%%%%%%%%%%%%%%%%%%%%%
%%%%%%%%%%%%%%%%%%%%%%%%%%%%%%%%%%%%%%%%%%%%%%%%%%%%%%%%%%%%%%%%%%%%%%%%%%%%%%%%
\appendix

\settowidth\MacroIndent{\rmfamily\scriptsize 000\ }

 \DocInput{childdoc.dtx}

\end{document}
%</driver>
% \fi
%
% %%%%%%%%%%%%%%%%%%%%%%%%%%%%%%%%%%%%%%%%%%%%%%%%%%%%%%%%%%%%%%%%%%%%%%%%%%%%%%
% %%%%%%%%%%%%%%%%%%%%%%%%%%%%%%%%%%%%%%%%%%%%%%%%%%%%%%%%%%%%%%%%%%%%%%%%%%%%%%
% \section{Sample}
%\iffalse
%<*samplemain>
%\fi
%
% The following presents a sample document
% with two chapters, two parts, a title page,
% a compile flag as well as three forwarding files to set the flag.
% It consists of eight |.tex| files:
% \begin{center}
% \begin{tabular}{ll}
% |cdocsamp.tex|&main file\\
% |cdocsch1.tex|&include file for chapter 1\\
% |cdocsch2.tex|&include file for chapter 2\\
% |cdocspt3.tex|&include file for part 3\\
% |cdocspt4.tex|&include file for part 4\\
% |cdocsdrf.tex|&forwarding file for main file in draft mode\\
% |cdocsfi1.tex|&forwarding file for final version of chapter 1\\
% |cdocsfi2.tex|&forwarding file for final version of chapter 2\\
% \end{tabular}
% \end{center}
% Each of the eight files can be compiled directly by the \LaTeX{} compiler.
%
% %%%%%%%%%%%%%%%%%%%%%%%%%%%%%%%%%%%%%%
% \paragraph{Main File.}
%
% The main file is called |cdocsamp.tex|.
%
% Load the \textsf{childdoc} definitions and
% declare the filename for the main document:
%    \begin{macrocode}
\input{childdoc.def}
\childdocmain{}
%    \end{macrocode}

% Optional override for |\version| flag:
%    \begin{macrocode}
%%\ifchilddoc\else\providecommand{\version}{draft}\fi
%    \end{macrocode}

% Define the default values for the |\version| flag
% (|final| for the main file and |draft| for childs):
%    \begin{macrocode}
\ifchilddoc
\providecommand{\version}{draft}
\else
\providecommand{\version}{final}
\fi
%    \end{macrocode}

% Load the standard document class:
%    \begin{macrocode}
\documentclass[12pt]{article}
%    \end{macrocode}

% Start the document body:
%    \begin{macrocode}
\begin{document}
%    \end{macrocode}

% Declare a title page.
% Print title, part of document being processed and version flag:
%    \begin{macrocode}
\addtocounter{page}{-1}
\begin{center}
{\LARGE\bfseries{}childdoc example\par}
\vspace{1cm}
\ifchilddoc
\ifchilddocmanual part\else chapter\fi:
`\childdocname' of `\childdocjob'\par
\else
main document: `\childdocjob'\par
\fi
version: \version\par
\end{center}
\newpage
%    \end{macrocode}

% Manually include selected file,
% otherwise process as usual:
%    \begin{macrocode}
\ifchilddocmanual
\section*{part `\childdocname'}
\input{\childdocname}
\else
%    \end{macrocode}

% Include the two chapters:
%    \begin{macrocode}
\include{cdocsch1}
\include{cdocsch2}
%    \end{macrocode}

% Include the two parts unless only chapters should be displayed:
%    \begin{macrocode}
\ifchilddoc\else
\section{part three}
\input{cdocspt3}
\section{part four}
\input{cdocspt4}
\fi
%    \end{macrocode}

% Process as usual until here:
%    \begin{macrocode}
\fi
%    \end{macrocode}

% End of document body:
%    \begin{macrocode}
\end{document}
%    \end{macrocode}
%\iffalse
%</samplemain>
%\fi
%
% %%%%%%%%%%%%%%%%%%%%%%%%%%%%%%%%%%%%%%
% \paragraph{Chapter Include Files.}
%
% The include files are called |cdocsch1.tex| and |cdocsch2.tex|.
%
%\iffalse
%<*samplechap1|samplechap2>
%\fi

% Optional override for |\version| flag:
%    \begin{macrocode}
%%\providecommand{\version}{final}
%    \end{macrocode}

% Include the main document:
%    \begin{macrocode}
\input{childdoc.def}
\childdocof{cdocsamp}
%    \end{macrocode}

%\iffalse
%</samplechap1|samplechap2>
%\fi
%
%\iffalse
%<*samplechap1>
%\fi
% Some text for chapter 1:
%    \begin{macrocode}
\section{one}
some text in chapter one
%    \end{macrocode}

%\iffalse
%</samplechap1>
%\fi
% Some text for chapter 2:
%\iffalse
%<*samplechap2>
%\fi
%    \begin{macrocode}
\section{two}
more text in chapter two
%    \end{macrocode}

%\iffalse
%</samplechap2>
%\fi
%
% %%%%%%%%%%%%%%%%%%%%%%%%%%%%%%%%%%%%%%
% \paragraph{Part Include Files.}
%
% The include files are called |cdocspt3.tex| and |cdocspt4.tex|.
%
%\iffalse
%<*samplepart3|samplepart4>
%\fi

% Optional override for |\version| flag:
%    \begin{macrocode}
%%\providecommand{\version}{final}
%    \end{macrocode}

% Include the main document:
%    \begin{macrocode}
\input{childdoc.def}
\childdocby{cdocsamp}
%    \end{macrocode}

%\iffalse
%</samplepart3|samplepart4>
%\fi
%
%\iffalse
%<*samplepart3>
%\fi
% Some text for part 3:
%    \begin{macrocode}
some text in part three
%    \end{macrocode}

%\iffalse
%</samplepart3>
%\fi
% Some text for part 4:
%\iffalse
%<*samplepart4>
%\fi
%    \begin{macrocode}
more text in part four
%    \end{macrocode}

%\iffalse
%</samplepart4>
%\fi
%
% %%%%%%%%%%%%%%%%%%%%%%%%%%%%%%%%%%%%%%
% \paragraph{Forwarding for a Complete Draft.}
%
% The following forwarding file |cdocsdrf.tex|
% compiles the main document in draft mode:
%\iffalse
%<*sampledraft>
%\fi
%    \begin{macrocode}
\def\version{draft}
\input{childdoc.def}
\childdocforward{cdocsamp}
%    \end{macrocode}

%\iffalse
%</sampledraft>
%\fi
%
% %%%%%%%%%%%%%%%%%%%%%%%%%%%%%%%%%%%%%%
% \paragraph{Forwarding for Final Version of the Chapters.}
%
% The following forwarding files |cdocsfn1.tex| and |cdocsfn2.tex|
% (with identical content)
% compile the final versions of the child documents
% |cdocsch1.tex| and |cdocsch2.tex|, respectively:
%\iffalse
%<*samplefinal>
%\fi
%    \begin{macrocode}
\def\version{final}
\input{childdoc.def}
\childdocforwardprefix[cdocsamp]{cdocsfn}{cdocsch}
%    \end{macrocode}

%\iffalse
%</samplefinal>
%\fi
%
% %%%%%%%%%%%%%%%%%%%%%%%%%%%%%%%%%%%%%%
% \paragraph{Command Line Processing.}
%
% The following three command lines generate the output files
% |cdocscld|, |cdocscl1| and |cdocscl2|
% which should be identical to
% |cdocsdrf|, |cdocsch1| and |cdocsfn2|, respectively:
% \begin{center}
% \begin{tabular}{l}
% |latex -jobname cdocscld \|\\
% |  "\def\version{draft}\input{childdoc.def}\childdocforward{cdocsamp}"|\\
% |latex -jobname cdocscl1 \|\\
% |  "\input{childdoc.def}\childdocforward[cdocsamp]{cdocsch1}"|\\
% |latex -jobname cdocscl2 \|\\
% |  "\def\version{final}\input{childdoc.def}\childdocforward{cdocsch2}"|
% \end{tabular}
% \end{center}
% Note that the trailing backslash on each first line
% merely continues the input to the second line
% (for convenient cut ant paste).
% Furthermore, the command |latex| can be replaced by any
% of its alternative versions such as |pdflatex|.
%
% %%%%%%%%%%%%%%%%%%%%%%%%%%%%%%%%%%%%%%%%%%%%%%%%%%%%%%%%%%%%%%%%%%%%%%%%%%%%%%
% %%%%%%%%%%%%%%%%%%%%%%%%%%%%%%%%%%%%%%%%%%%%%%%%%%%%%%%%%%%%%%%%%%%%%%%%%%%%%%
% \section{Implementation}
%\iffalse
%<*package>
%\fi
%
% This section describes the definitions file |childdoc.def|.

% The definitions cannot be loaded using |\usepackage| or |\RequirePackage|
% which has a mechanism to prevent loading a style file more than once.
% When loading the definitions by means of |\input|
% multiple instances have to be prevented manually:
%\iffalse
%This code needs to be before the `\ProvidesFile' directive
%which is defined at the beginning of this file.
%Therefore it is also placed there and commented out here.
%</package>
%<*discard>
%\fi
%    \begin{macrocode}
\ifdefined\childdocmain\endinput\fi
%    \end{macrocode}
%\iffalse
%</discard>
%<*package>
%\fi
%
% \macro{\ifchilddoc}
% \macro{\ifchilddocmanual}
% The conditional |\ifchilddoc| tells whether a
% child (true) or main (false) document is being compiled.
% The conditional |\ifchilddocmanual| tells whether
% the |\includeonly| mechanism is used (false) or
% the selection of child files must be performed manually (true).
% The definitions initialise to false:
%    \begin{macrocode}
\newif\ifchilddoc
\newif\ifchilddocmanual
%    \end{macrocode}

% \macro{\childdocname}
% \macro{\childdocjob}
% The macro |\childdocname| stores the name of the main document
% to be compiled. The macro |\childdocjob| stores the name of
% the document on which the \LaTeX{} compiler was originally invoked.
% The content of |\jobname| cannot be compared
% to filenames specified in the source due to different catcodes.
% The following code rescans |\jobname|, stores the result
% in |\childdocname| and saves a copy in |\childdocjob|:
%    \begin{macrocode}
\edef\childdocname{\scantokens\expandafter{\jobname\noexpand}}
\let\childdocjob\childdocname
%    \end{macrocode}

% \macro{\childdocdisable}
% The macro |\childdocdisable| prevents the main file
% from being processed more than once.
% At this stage, the main document command |\childdocmain|
% is assumed to be called once again where it should do nothing.
% Any subsequent call to it should prevent
% a secondary processing of the main document
% It overwrites the forwarding commands
% |\childdocof| and |\childdocforward|
% with empty macros to prevent further inclusions of the main document:
%    \begin{macrocode}
\newcommand{\childdocdisable}
{
  \renewcommand{\childdocmain}[1]{\renewcommand{\childdocmain}[1]{\endinput}}
  \renewcommand{\childdocof}[1]{}
  \renewcommand{\childdocby}[2][]{}
  \renewcommand{\childdocforward}[2][]{}
  \renewcommand{\childdocdisable}{}
}
%    \end{macrocode}

% \macro{\childdocmain}
% The macro |\childdocmain| is to be called at the top of the main file
% with nothing or the main filename (without extension) as argument.
% First, it breaks loops.
% If the argument is not empty and does not match |\childdocname|
% (which is set by the first inclusion of |childdoc.def|),
% |\ifchilddoc| is set to true, |\includeonly| is applied to the child file
% and |\jobname| is set to the main file
% (for proper handling of |.aux| files):
%    \begin{macrocode}
\newcommand{\childdocmain}[1]
{
  \childdocdisable\childdocmain{}
  \if?#1?\else
    \begingroup
      \def\childdoctmp{#1}
      \ifx\childdoctmp\childdocname
        \def\childdoctmp{}
      \else
        \def\childdoctmp
        {
          \childdoctrue
          \includeonly{\childdocname}
          \def\childdocjob{#1}
          \def\jobname{#1}
        }
      \fi
      \expandafter
    \endgroup
    \childdoctmp
  \fi
}
%    \end{macrocode}

% \macro{\childdocof}
% The command |\childdocof| redirects
% compilation to the main file |#1|.
%    \begin{macrocode}
\newcommand{\childdocof}[1]
{
  \childdocdisable
  \childdoctrue
  \includeonly{\childdocname}
  \def\jobname{#1}
  \def\childdocjob{#1}
  \input{#1}
}
%    \end{macrocode}

% \macro{\childdocby}
% The command |\childdocby| ....
%    \begin{macrocode}
\newcommand{\childdocby}[2][]
{
  \childdocdisable
  \childdoctrue
  \childdocmanualtrue
  \if?#1?\else
    \def\jobname{#2}
  \fi
  \def\childdocjob{#2}
  \input{#2}
  \endinput
}
%    \end{macrocode}

% \macro{\childdocforward}
% The command |\childdocforward| redirects
% compilation to the main file or
% (if the optional argument is given) a child file.
% Parameters are set as if the main file
% or a child file starting with |\childdocof| was compiled.
% Then compilation is handed over to the main file:
%    \begin{macrocode}
\newcommand{\childdocforward}[2][]
{
  \begingroup
    \if?#1?
      \def\childdoctmp
      {
        \def\childdocname{#2}
        \def\childdocjob{#2}
        \def\jobname{#2}
        \input{#2}
        \endinput
      }
    \else
      \def\childdoctmp
      {
        \childdocdisable
        \def\childdocname{#2}
        \childdoctrue
        \includeonly{#2}
        \def\childdocjob{#1}
        \def\jobname{#1}
        \input{#1}
        \endinput
      }
    \fi
    \expandafter
  \endgroup
  \childdoctmp
}
%    \end{macrocode}

% \macro{\childdocforwardprefix}
% The command |\childdocforwardprefix| redirects
% compilation to the main or a child file by means of a pattern.
% The prefix |#1| in the current filename is replaced by |#2|
% and the suffix of the current filename is kept
% (it is assumed that the filename does not contain the substring `|~~~|'
% which is used as a delimiter).
% Compilation is handed over to the new file by |\childdocforward|:
%    \begin{macrocode}
\newcommand{\childdocforwardprefix}[3][]
{
  \begingroup
    \def\childdocextract #2##1~~~{\def\childdoctmp{\childdocforward[#1]{#3##1}}}
    \expandafter\childdocextract\childdocname~~~
    \expandafter
  \endgroup
  \childdoctmp
}
%    \end{macrocode}

% \macro{\childdoc}
% The deprecated macro |\childdoc| is a legacy version of |\childdocmain|:
%    \begin{macrocode}
\newcommand{\childdoc}{\childdocmain}
%    \end{macrocode}

% \macro{\childdocredirect}
% The deprecated macro |\childdocredirect| is a legacy version
% of |\childdocforward| and |\childdocforwardprefix|:
%    \begin{macrocode}
\newcommand{\childdocredirect}[2][]
{
  \begingroup
    \if?#1?
      \def\childdoctmp{\childdocforward{#2}}
    \else
      \def\childdoctmp{\childdocforwardprefix{#1}{#2}}
    \fi
    \expandafter
  \endgroup
  \childdoctmp
}
%    \end{macrocode}

%\iffalse
%</package>
%\fi
%
\endinput
|\\
|\childdocforwardprefix[|\textit{main}|]{|\textit{prefix}|}{|\textit{dest}|}|
\end{tabular}
\end{center}
%
the destination file is determined by a pattern
depending on the current file:
To make this work, the current file must be called
`{\textit{prefix}\hspace{0.2em}\textit{suffix}}'
with \textit{prefix} matching precisely the argument.
Processing is then passed on to the file
`{\textit{dest}\hspace{0.2em}\textit{suffix}}'.
Surely, the same effect is achieved by
directly specifying the
argument `{\textit{dest}\hspace{0.2em}\textit{suffix}}'
in the first form.
However, that requires to set up a different file
for each child. With the alternative form of the command
all these files can have exactly the same content
which simplifies setting them up and maintaining them.

For example, the following file |draft.tex|
with a compilation flag |\version| as described in \secref{sec:flags}
compiles the main document as a draft:
%
\begin{center}
\begin{tabular}{l}
|\def\version{draft}|\\
|% \iffalse
%
% childdoc.dtx Copyright (C) 2017-2018 Niklas Beisert
%
% This work may be distributed and/or modified under the
% conditions of the LaTeX Project Public License, either version 1.3
% of this license or (at your option) any later version.
% The latest version of this license is in
%   http://www.latex-project.org/lppl.txt
% and version 1.3 or later is part of all distributions of LaTeX
% version 2005/12/01 or later.
%
% This work has the LPPL maintenance status `maintained'.
%
% The Current Maintainer of this work is Niklas Beisert.
%
% This work consists of the files childdoc.dtx and childdoc.ins
% and the derived files childdoc.def and cdocsamp.tex with
% cdocsch1.tex, cdocsch2.tex, cdocsdrf.tex, cdocsfn1.tex, cdocsfn2.tex.
%
%<package>\ifdefined\childdocmain\endinput\fi
%<package>\ProvidesFile{childdoc.def}[2018/12/30 v2.0 child document driver]
%<samplemain>\ProvidesFile{cdocsamp.tex}[2018/12/30 v2.0 sample for childdoc]
%<*driver>
%\ProvidesFile{childdoc.drv}[2018/12/30 v2.0 childdoc reference manual file]
\PassOptionsToClass{10pt,a4paper}{article}
\documentclass{ltxdoc}

\usepackage[margin=35mm]{geometry}
\usepackage{hyperref}
\usepackage{hyperxmp}
\usepackage[usenames]{color}

\hypersetup{colorlinks=true}
\hypersetup{pdfstartview=FitH}
\hypersetup{pdfpagemode=UseNone}
\hypersetup{pdfsource={}}
\hypersetup{pdflang={en-UK}}
\hypersetup{pdfcopyright={Copyright 2017-2018 Niklas Beisert.
  This work may be distributed and/or modified under the
  conditions of the LaTeX Project Public License, either version 1.3
  of this license or (at your option) any later version.}}
\hypersetup{pdflicenseurl={http://www.latex-project.org/lppl.txt}}
\hypersetup{pdfcontactaddress={ETH Zurich, ITP, HIT K,
  Wolfgang-Pauli-Strasse 27}}
\hypersetup{pdfcontactpostcode={8093}}
\hypersetup{pdfcontactcity={Zurich}}
\hypersetup{pdfcontactcountry={Switzerland}}
\hypersetup{pdfcontactemail={nbeisert@itp.phys.ethz.ch}}
\hypersetup{pdfcontacturl={http://people.phys.ethz.ch/\xmptilde nbeisert/}}

\newcommand{\secref}[1]{\hyperref[#1]{section \ref*{#1}}}

\parskip1ex
\parindent0pt
\let\olditemize\itemize
\def\itemize{\olditemize\parskip0pt}

\begin{document}

\title{The \textsf{childdoc} Package}
\hypersetup{pdftitle={The childdoc Package}}
\author{Niklas Beisert\\[2ex]
  Institut f\"ur Theoretische Physik\\
  Eidgen\"ossische Technische Hochschule Z\"urich\\
  Wolfgang-Pauli-Strasse 27, 8093 Z\"urich, Switzerland\\[1ex]
  \href{mailto:nbeisert@itp.phys.ethz.ch}
  {\texttt{nbeisert@itp.phys.ethz.ch}}}
\hypersetup{pdfauthor={Niklas Beisert}}
\hypersetup{pdfsubject={Manual for the LaTeX2e Package childdoc}}
\date{30 December 2018, \textsf{v2.0}}
\maketitle

\begin{abstract}\noindent
\textsf{childdoc} is a \LaTeXe{} package
that enables the direct compilation
of document sections included by |\include|
to individual files.
\end{abstract}

\begingroup
\parskip0ex
\tableofcontents
\endgroup

%%%%%%%%%%%%%%%%%%%%%%%%%%%%%%%%%%%%%%%%%%%%%%%%%%%%%%%%%%%%%%%%%%%%%%%%%%%%%%%%
%%%%%%%%%%%%%%%%%%%%%%%%%%%%%%%%%%%%%%%%%%%%%%%%%%%%%%%%%%%%%%%%%%%%%%%%%%%%%%%%
\section{Introduction}

\LaTeX{} provides a mechanism to structure a large document (such as a book)
into a main file and several child files (containing the chapters)
using the |\include| command.
This mechanism is beneficial for documents
which span hundreds of pages in order to
make the source file(s) more manageable.
Moreover, compilation can be restricted to
selected child files by means of the |\includeonly| command.
The latter feature can be used to reduce the compilation time while editing
(this was significantly more useful in the earlier days of \LaTeX{})
or to generate a smaller document which is easier to navigate.
Another application of |\includeonly| is to generate
documents consisting of selected parts of the complete document.

However, there are a few drawbacks of the plain |\include| mechanism:
\begin{itemize}
\item
The child files cannot be compiled on their own,
they can only be compiled via the main file.
A naive editing environment
(such as a text editor with an option
to have the current file processed by \LaTeX)
may require one to switch to the main file before compiling;
attempting to compile the child file produces errors.
\item
The main file must be modified (each time)
to adjust the |\includeonly| command
to the present needs. This easily leaves the main file in a messy state.
\item
The generated document will always carry the filename
of the main document. This is inconvenient if
several child files are to be compiled and
to be kept for distribution.
\end{itemize}

The present package provides a simple interface
to make child files individually compilable by \LaTeX{}.
Compiling a child file then has the same effect as compiling
the main file with an |\includeonly| command
to select the appropriate child.
Moreover the generated document will carry the name of the child
rather than the main file.
This resolves all three above issues.

This feature is meant to make the editing of books,
thesis documents and lecture notes somewhat more convenient.
However, the package can also be used efficiently for
composing a series of documents (such as exercise sheets)
which are typically distributed individually.
It then assists the author in generating the individual documents
(potentially in different versions)
as well as a document containing the collected series.
Another application is in developing style files
or other kinds of included material
where compilation of the style file could redirect
to a sample or test file.

%%%%%%%%%%%%%%%%%%%%%%%%%%%%%%%%%%%%%%%%%%%%%%%%%%%%%%%%%%%%%%%%%%%%%%%%%%%%%%%%
%%%%%%%%%%%%%%%%%%%%%%%%%%%%%%%%%%%%%%%%%%%%%%%%%%%%%%%%%%%%%%%%%%%%%%%%%%%%%%%%
\section{Usage}

First of all, the package \textsf{childdoc} is \emph{not} a standard
\LaTeXe{} |.sty| style file! Therefore it needs to be invoked in
a non-standard way.

%%%%%%%%%%%%%%%%%%%%%%%%%%%%%%%%%%%%%%%%%%%%%%%%%%%%%%%%%%%%%%%%%%%%%%%%%%%%%%%%
\subsection{Included Files}
\label{sec:include}

%%%%%%%%%%%%%%%%%%%%%%%%%%%%%%%%%%%%%%%%
\DescribeMacro{\childdocmain}
To use the package, add the commands
\begin{center}
\begin{tabular}{l}
|\input{childdoc.def}|\\
|\childdocmain{}|\\
\end{tabular}
\end{center}
at the very top of the main \LaTeX{} file,
in particular \emph{before} the |\documentclass| statement!
The argument of |\childdocmain| should be left empty
(but it must be present).

%%%%%%%%%%%%%%%%%%%%%%%%%%%%%%%%%%%%%%%%
\DescribeMacro{\childdocof}
Furthermore, add the commands
\begin{center}
\begin{tabular}{l}
|\input{childdoc.def}|\\
|\childdocof{|\textit{main}|}|\\
\end{tabular}
\end{center}
at the top of every child file \textit{child}
which is included by |\include{|\textit{child}|}|
from within the main file
(or at least for those files to be compiled individually).
The argument \textit{main} must be the filename of the main file.

There are a couple of
considerations in setting up the main and child documents:

%%%%%%%%%%%%%%%%%%%%%%%%%%%%%%%%%%%%%%%%
\paragraph{Restrictions.}

Please note the following restrictions:
\begin{itemize}
\item
|\childdocmain| must be called with one argument \textit{main}
to ensure compatibility with earlier version of the package.
It must either be empty (|\childdocmain{}|)
or precisely match the filename of the main file in which it is specified.
See \secref{sec:detection} for further information.
\item
The filename \textit{main} must be specified without the |.tex| extension.
\item
The filename \textit{main} is case sensitive
(even in case-insensitive file systems)
due to internal string comparison.
\item
The argument \textit{main} should be fully expanded, it cannot be a macro.
\item
Subdirectories and special characters should be avoided in filenames.
\item
The command |\childdocmain{|\textit{main}|}| must be followed by a whitespace.
It should not be followed immediately by another command
or by a comment mark `|%|'.
This is because the \TeX{} parser reads the token immediately following
the argument of |\childdocmain| and puts it
at the beginning of every child section;
however, a white\-space is ignored.
\end{itemize}

%%%%%%%%%%%%%%%%%%%%%%%%%%%%%%%%%%%%%%%%
\paragraph{Content of Main File.}

It is advisable to place all content in the child files included by |\include|.
Any output contained in the main file will appear in all child documents
unless suppressed manually;
it cannot be suppressed automatically by the |\includeonly| directive
and thus should normally be avoided.
A method to include some content in the main file
by means of conditional processing is described in \secref{sec:conditional}.

%%%%%%%%%%%%%%%%%%%%%%%%%%%%%%%%%%%%%%%%
\paragraph{Page Numbering.}

When only a part of the document is compiled,
the appropriate numbering of pages
(as well as other status parameters)
is determined from the |.aux| files.
The latter contain information from previous passes.
However this information needs to propagate through
all intermediate child documents.
Therefore the page numbering in child documents may well
be inconsistent until the complete document is compiled at least once.

A useful (if unconventional) way to always ensure a consistent
page numbering is to restart the numbering in each child document
and denote the pages by `\textit{child}|.|\textit{page}'
where \textit{child} represents the chapter/section number of the child file.
This can be achieved by the command
|\numberwithin{page}{|\textit{child}|}|
of the \textsf{amsmath} package
where \textit{child} can be |chapter| or |section|
depending on the chosen structuring.
Alternatively, one can modify the macro |\thepage| appropriately
and reset the counter |page| at the start of each child file.

%%%%%%%%%%%%%%%%%%%%%%%%%%%%%%%%%%%%%%%%%%%%%%%%%%%%%%%%%%%%%%%%%%%%%%%%%%%%%%%%
\subsection{Conditional Processing}
\label{sec:conditional}

The package provides a mechanism to compile different versions
of a document. To customise the versions further some conditional processing
can come in handy to distinguish which version is being compiled.
The package provides two macros to describe the compilation context:

%%%%%%%%%%%%%%%%%%%%%%%%%%%%%%%%%%%%%%%%
\DescribeMacro{\ifchilddoc}
The conditional |\ifchilddoc| distinguishes between the compilation of
child documents and the main document:
%
\begin{center}
|\ifchilddoc |\textit{child-code}| |[|\||else |\textit{main-code}]| \||fi|
\end{center}

%%%%%%%%%%%%%%%%%%%%%%%%%%%%%%%%%%%%%%%%
\DescribeMacro{\childdocname}
\DescribeMacro{\childdocjob}
The macro |\childdocname| contains the filename (without extension)
of the main or child file being processed.
Note that |\childdocjob| will always contain the name of the main file.

%%%%%%%%%%%%%%%%%%%%%%%%%%%%%%%%%%%%%%%%
\paragraph{Title Page.}

Conditional processing can be used to include a title or banner page
in the main document when proper precautions are taken.
Importantly, the code in the main file should ensure that the page counter
(as well as other status parameters which are stored in the |.aux| files)
takes the same value after the conditional processing.
Otherwise the page numbers may take divergent values
depending on which part is compiled.

For example, a title page could be declared by:
%
\begin{center}
\begin{tabular}{l}
|\ifchilddoc\||else|\\
|\addtocounter{page}{-1}|\\
\textit{code for title page}\\
|\newpage|\\
|\||fi|
\end{tabular}
\end{center}
%
A banner page for the child documents can be generated by:
%
\begin{center}
\begin{tabular}{l}
|\ifchilddoc|\\
|\addtocounter{page}{-1}|\\
\textit{code for banner page}\\
|\newpage|\\
|\||fi|
\end{tabular}
\end{center}
%
Here one could write a message such as:
\begin{center}
|This is the part \childdocname{} of \childdocjob{}.|
\end{center}

%%%%%%%%%%%%%%%%%%%%%%%%%%%%%%%%%%%%%%%%%%%%%%%%%%%%%%%%%%%%%%%%%%%%%%%%%%%%%%%%
\subsection{Flags}
\label{sec:flags}

The package makes it easy to generate different versions
of the main or child documents.
To this end compilation flags can be defined
and assigned different default values.
They will be particularly useful in conjunction
with the forwarding mechanism described in \secref{sec:forward}.

For example, it may be useful to have a flag |\version|
which can be set to |draft| or |final|.
The document source will contain some conditional code
depending on the value of |\version|.
Suppose further, the flag should default to |final| for the main file
and to |draft| for child files
which is a natural assignment for editing the document.
This is achieved by placing the following code
in the preamble of the main document
(below the |\childdocmain| directive):
%
\begin{center}
\begin{tabular}{l}
|\ifchilddoc|\\
|\providecommand{\version}{draft}|\\
|\||else|\\
|\providecommand{\version}{final}|\\
|\||fi|
\end{tabular}
\end{center}
%
The definition by |\providecommand| makes sure
that previous definitions are not overwritten.
Further statements |\providecommand{\version}{...}|
can thus be added before the above code to override it.

For the main file, one might add a line
(between |\childdocmain| and the above block)
%
\begin{center}
|%\ifchilddoc\||else\providecommand{\version}{draft}\||fi|
\end{center}
%
which can be uncommented to produce a draft version.
Likewise one can add a line to the very top of a child file
(above the |\childdocof{|\textit{main}|}| directive)
%
\begin{center}
|%\providecommand{\version}{final}|
\end{center}
%
which can be uncommented to produce the final version of this child document.

%%%%%%%%%%%%%%%%%%%%%%%%%%%%%%%%%%%%%%%%%%%%%%%%%%%%%%%%%%%%%%%%%%%%%%%%%%%%%%%%
\subsection{Forwarding}
\label{sec:forward}

Different versions of the main or child documents
using compilation flags as described in \secref{sec:flags}
can be (permanently) stored in different files
for convenient compilation, viewing and distribution.
To this end, the package defines a command
to pass on compilation to a different file:

%%%%%%%%%%%%%%%%%%%%%%%%%%%%%%%%%%%%%%%%
\DescribeMacro{\childdocforward}
The command |\childdocforward| redirects processing to
another source file:
%
\begin{center}
\begin{tabular}{l}
|\input{childdoc.def}|\\
|\childdocforward[|\textit{main}|]{|\textit{dest}|}|\\
\end{tabular}
\end{center}
%
The argument \textit{dest} is the destination file
(without extension).
It should be the main file or one of the child files.
Note that further \textsf{childdoc} directives
such as |\childdocof| and |\childdocforward|
in the indicated file will be processed in this form.
The optional argument \textit{main}
passes on directly to the main file \textit{main}
while pretending to compile the child \textit{dest}.
This form behaves as if \textit{dest}
issues |\childdocof{|\textit{main}|}| right away,
and no further \textsf{childdoc} directives will be processed.

%%%%%%%%%%%%%%%%%%%%%%%%%%%%%%%%%%%%%%%%
\DescribeMacro{\...prefix}
In the alternative form |\childdocforwardprefix|,
%
\begin{center}
\begin{tabular}{l}
|\input{childdoc.def}|\\
|\childdocforwardprefix[|\textit{main}|]{|\textit{prefix}|}{|\textit{dest}|}|
\end{tabular}
\end{center}
%
the destination file is determined by a pattern
depending on the current file:
To make this work, the current file must be called
`{\textit{prefix}\hspace{0.2em}\textit{suffix}}'
with \textit{prefix} matching precisely the argument.
Processing is then passed on to the file
`{\textit{dest}\hspace{0.2em}\textit{suffix}}'.
Surely, the same effect is achieved by
directly specifying the
argument `{\textit{dest}\hspace{0.2em}\textit{suffix}}'
in the first form.
However, that requires to set up a different file
for each child. With the alternative form of the command
all these files can have exactly the same content
which simplifies setting them up and maintaining them.

For example, the following file |draft.tex|
with a compilation flag |\version| as described in \secref{sec:flags}
compiles the main document as a draft:
%
\begin{center}
\begin{tabular}{l}
|\def\version{draft}|\\
|\input{childdoc.def}|\\
|\childdocforward{|\textit{main}|}|
\end{tabular}
\end{center}
%
Likewise, the following files |final|\textit{nn}|.tex|
compile the final version of the child document
|child|\textit{nn}|.tex|:
%
\begin{center}
\begin{tabular}{l}
|\def\version{final}|\\
|\input{childdoc.def}|\\
|\childdocforwardprefix{final}{child}|
\end{tabular}
\end{center}
%

Note that when several versions of a main file and/or of each child file
are to be generated, it may be convenient to set up a |Makefile| or
shell script to automatise the process.

%%%%%%%%%%%%%%%%%%%%%%%%%%%%%%%%%%%%%%%%%%%%%%%%%%%%%%%%%%%%%%%%%%%%%%%%%%%%%%%%
\subsection{Command Line Processing}
\label{sec:commandline}

The effect of redirection files can also be achieved by invoking
the \LaTeX{} compiler with a more elaborate command line.
Most conveniently this should be done as part
of a shell script or a |Makefile|.

When using \textsf{childdoc} in the main file, the following
command lines effectively perform a redirection
(note that depending on the shell being used,
backslashes may have to be doubled: `|\|' $\to$ `|\\|'):
%
\begin{center}
|... -jobname "|\textit{target}|" |\\|"|[\textit{flags}]%
|\input{childdoc.def}\childdocforward[|\textit{main}|]{|\textit{dest}|}"|
\end{center}
%
Here \textit{target} is the name of the output file,
\textit{main} is the name of the main file
and \textit{dest} is the name of the main or child file to be processed
(all filenames without extensions).
The optional argument \textit{main} can be omitted
if \textit{main} matches \textit{dest}.
Optionally, compilation \textit{flags} can be defined via |\def| commands.
This command line makes the \TeX{} engine believe
it is compiling the file \textit{target}
whose content is specified as the latter parameter.
The provided code then forwards the processing to
\textit{main} or \textit{dest} as described in \secref{sec:forward}.

%%%%%%%%%%%%%%%%%%%%%%%%%%%%%%%%%%%%%%%%%%%%%%%%%%%%%%%%%%%%%%%%%%%%%%%%%%%%%%%%
\subsection{Include by Input}
\label{sec:input}

Including child documents by |\include| has some restrictions by design.
Most notably, the content of a child document always occupies
its own set of pages; pages cannot be shared between child documents.
Usually, this behaviour makes perfect sense
because each child document contain an essential part of the document.
However, in some situations it may be desirable to compose
a document from a collection of parts
without having mandatory page breaks between then.
For this case, the package
provides a mechanism to include parts
by |\input| which can also be processed individually.
However, by construction this mechanism
requires manual handling of the content to be output.

%%%%%%%%%%%%%%%%%%%%%%%%%%%%%%%%%%%%%%%%
\DescribeMacro{\ifchilddocmanual}
The main file should be prepared as usual, see \secref{sec:include}.
However, the document body must make a distinction
between processing of an individual part and of the main document, e.g.:
%
\begin{center}
\begin{tabular}{l}
|\ifchilddocmanual|\\
|\input{\childdocname}|\\
|\||else|\\
\textit{document body with }|\input{|\textit{part}|}|\\
|\||fi|
\end{tabular}
\end{center}
%
The conditional |\ifchilddocmanual| is true whenever
a part to be included by |\input| is being compiled,
and the name of the part is stored in |\childdocname|.

%%%%%%%%%%%%%%%%%%%%%%%%%%%%%%%%%%%%%%%%
\DescribeMacro{\childdocby}
Each part to be included by |\input| should start with:
%
\begin{center}
\begin{tabular}{l}
|\input{childdoc.def}|\\
|\childdocby{|\textit{main}|}|\\
\end{tabular}
\end{center}
%
The directive |\childdocby| is similar to |\childdocof|
described in \secref{sec:include},
but the subsequent selection of content must be done manually.
To that end, both |\ifchilddoc| and |\ifchilddocmanual|
will be true upon processing of a part,
and the name of the part is stored in |\childdocname|.
Note that |\jobname| will be set to the filename of the current part
so that each part receives an individual |.aux| file
that does not interfere with the |.aux| file(s) of the main document.
This behaviour can be altered by the alternative form
|\childdocby[*]{|\textit{main}|}| (with a non-empty optional argument)
which uses the |.aux| file of the main document
by setting |\jobname| to \textit{main}.

%%%%%%%%%%%%%%%%%%%%%%%%%%%%%%%%%%%%%%%%%%%%%%%%%%%%%%%%%%%%%%%%%%%%%%%%%%%%%%%%
\subsection{Driver Development}
\label{sec:driver}

The \textsf{childdoc} mechanism can also be use for the development
of definition files such as \LaTeX{} styles or classes.
This case differs from the above setup with multiple parts
included by |\include| in that no |\includeonly| should be invoked.
This can be achieved by starting the include file
(before |\ProvidesPackage|) with:
%
\begin{center}
\begin{tabular}{l}
|\input{childdoc.def}|\\
|\childdocforward{|\textit{main}|}|\\
\end{tabular}
\end{center}
%
or alternatively with:
%
\begin{center}
\begin{tabular}{l}
|\input{childdoc.def}|\\
|\childdocby{|\textit{main}|}|\\
\end{tabular}
\end{center}
%
Both forms have slightly different effects as described above.
The main file is prepared as usual, see \secref{sec:include}.

%%%%%%%%%%%%%%%%%%%%%%%%%%%%%%%%%%%%%%%%%%%%%%%%%%%%%%%%%%%%%%%%%%%%%%%%%%%%%%%%
\subsection{Legacy Detection}
\label{sec:detection}

The directive |\childdocmain| in the main file can detect
whether the complete document or merely a child is to be compiled
even without using the directive |\childdocof|.
This method is deprecated because it is less robust
and there is no compelling reason to use it;
it is merely provided for backward compatibility
and it may be removed in future versions.

If the detection mechanism is to be used,
it is mandatory to correctly specify
the filename of the main file as the argument of |\childdocmain|:
%
\begin{center}
\begin{tabular}{l}
|\input{childdoc.def}|\\
|\childdocmain{|\textit{main}|}|\\
\end{tabular}
\end{center}
%
If |\jobname| does not match the argument \textit{main} of |\childdocmain|,
it is assumed that |\jobname| points to the child file to be compiled.
When using |\childdocmain| with the main file specified as argument,
it suffices to start a child file
with just |\input{|\textit{main}|}|
without loading of the package and using |\childdocof|.
If instead all processing is done
with the appropriate \textsf{childdoc} directives,
the argument of \textit{main} of |\childdocmain| can be empty.

An alternative version of the command line processing described
in \secref{sec:commandline} using the detection mechanism reads:
%
\begin{center}
|... -jobname "|\textit{target}|" "|[\textit{flags}]%
[|\def\jobname{|\textit{dest}|}|]|\input{|\textit{main}|}"|
\end{center}

%%%%%%%%%%%%%%%%%%%%%%%%%%%%%%%%%%%%%%%%%%%%%%%%%%%%%%%%%%%%%%%%%%%%%%%%%%%%%%%%
\subsection{Manual Code}
\label{sec:manual}

In case one cannot be certain whether the definitions file |childdoc.def|
is installed on the target \TeX{} distribution
and one prefers not to ship it,
it is conceivable to paste a few relevant commands into the sources.

To that end, drop all statements |\input{childdoc.def}|
and perform the replacements as outlined below.
Instead of |\childdocmain{|\textit{main}|}| add the following code
to the top of the main file:
%
\begin{center}
\begin{tabular}{l}
|\||ifdefined\childdocname\endinput\||fi\newif\ifchilddoc|\\
|\edef\childdocname{\scantokens\expandafter{\jobname\noexpand}}|\\
|\def\childdocmain{|\textit{main}|}\||ifx\childdocmain\childdocname\||else|\\
|\childdoctrue\includeonly{\childdocname}\let\jobname\childdocmain\||fi|\\
\end{tabular}
\end{center}
%
Instead of |\childdocof{|\textit{main}|}| just include the main file
at the top of each child file:
%
\begin{center}
|\input{|\textit{main}|}|
\end{center}
%
A simple redirection |\childdocforward{|\textit{dest}|}| is achieved by:
%
\begin{center}
|\def\jobname{|\textit{dest}|}\input{\jobname}|
\end{center}
%
The redirection with prefix
|\childdocforwardprefix[|\textit{prefix}|]{|\textit{dest}|}|
is accomplished by:
%
\begin{center}
\begin{tabular}{l}
|{\edef\jobname{\scantokens\expandafter{\jobname\noexpand}}|\\
|\def\redirectjob |\textit{prefix}|#1~~~{\gdef\jobname{|\textit{dest}|#1}}|\\
|\expandafter\redirectjob\jobname~~~}\input{\jobname}|
\end{tabular}
\end{center}

In an alternative approach,
child documents can be compiled by a specific command line
without additional code or specific definitions:
%
\begin{center}
|... -jobname "|\textit{target}|" "|[\textit{flags}]%
|\includeonly{|\textit{dest}|}\input{|\textit{main}|}"|
\end{center}
%

%%%%%%%%%%%%%%%%%%%%%%%%%%%%%%%%%%%%%%%%%%%%%%%%%%%%%%%%%%%%%%%%%%%%%%%%%%%%%%%%
%%%%%%%%%%%%%%%%%%%%%%%%%%%%%%%%%%%%%%%%%%%%%%%%%%%%%%%%%%%%%%%%%%%%%%%%%%%%%%%%
\section{Information}

%%%%%%%%%%%%%%%%%%%%%%%%%%%%%%%%%%%%%%%%%%%%%%%%%%%%%%%%%%%%%%%%%%%%%%%%%%%%%%%%
\subsection{Copyright}

Copyright \copyright{} 2017--2018 Niklas Beisert

This work may be distributed and/or modified under the
conditions of the \LaTeX{} Project Public License, either version 1.3
of this license or (at your option) any later version.
The latest version of this license is in
  \url{http://www.latex-project.org/lppl.txt}
and version 1.3 or later is part of all distributions of \LaTeX{}
version 2005/12/01 or later.

This work has the LPPL maintenance status `maintained'.

The Current Maintainer of this work is Niklas Beisert.

This work consists of the files |README.txt|, |childdoc.ins| and |childdoc.dtx|
as well as the derived files |childdoc.def|, |cdocsamp.tex|
with |cdocsch1.tex|, |cdocsch2.tex|, |cdocspt3.tex|, |cdocspt4.tex|,
|cdocsdrf.tex|, |cdocsfn1.tex|, |cdocsfn2.tex|
as well as |childdoc.pdf|.

%%%%%%%%%%%%%%%%%%%%%%%%%%%%%%%%%%%%%%%%%%%%%%%%%%%%%%%%%%%%%%%%%%%%%%%%%%%%%%%%
\subsection{Files and Installation}

The package consists of the files:
%
\begin{center}
\begin{tabular}{ll}
    |README.txt|   & readme file \\
    |childdoc.ins| & installation file \\
    |childdoc.dtx| & source file \\
    |childdoc.def| & definition file \\
    |cdocsamp.tex| & sample main file \\
    |cdocsch1.tex| & sample include file \\
    |cdocsch2.tex| & sample include file \\
    |cdocspt3.tex| & sample part file \\
    |cdocspt4.tex| & sample part file \\
    |cdocsdrf.tex| & sample redirection file \\
    |cdocsfn1.tex| & sample redirection file \\
    |cdocsfn2.tex| & sample redirection file \\
    |childdoc.pdf| & manual
\end{tabular}
\end{center}
%
The distribution consists of the files
|README.txt|, |childdoc.ins| and |childdoc.dtx|.
%
\begin{itemize}
\item
Run (pdf)\LaTeX{} on |childdoc.dtx|
to compile the manual |childdoc.pdf| (this file).
\item
Run \LaTeX{} on |childdoc.ins| to create the definitions file |childdoc.def|
and the sample |cdocsamp.tex| with include files
|cdocsch1.tex|, |cdocsch2.tex|, |cdocspt3.tex|, |cdocspt4.tex|,
|cdocsdrf.tex|, |cdocsfn1.tex|, |cdocsfn2.tex|.
Then copy the file |childdoc.def| to an appropriate directory of your \LaTeX{}
distribution, e.g.\ \textit{texmf-root}|/tex/latex/childdoc|.
\end{itemize}

%%%%%%%%%%%%%%%%%%%%%%%%%%%%%%%%%%%%%%%%%%%%%%%%%%%%%%%%%%%%%%%%%%%%%%%%%%%%%%%%
\subsection{Related CTAN Packages}

There are several other packages which offer a similar functionality:
%
\begin{itemize}
\item
The packages
\href{http://ctan.org/pkg/docmute}{\textsf{docmute}},
\href{http://ctan.org/pkg/includex}{\textsf{includex}} and
\href{http://ctan.org/pkg/standalone}{\textsf{standalone}}
provide commands to include only the document body of
a child file thus allowing both files to be compiled individually.
\item
The packages \href{http://ctan.org/pkg/subdocs}{\textsf{subdocs}}
and \href{http://ctan.org/pkg/subfiles}{\textsf{subfiles}}
provide structures in which the main and child documents can be
encapsulated and allowing them to be compiled individually.
The inclusion mechanism is different from the conventional |\include|.
\item
The package \href{http://ctan.org/pkg/combine}{\textsf{combine}}
is an elaborate solution to combine several documents into one.
\end{itemize}
%
See also the CTAN topic \href{http://ctan.org/topic/subdocs}{\textsf{subdocs}}
for further related packages.
The present package differs from the above solutions in that
a document structure constructed with the conventional |\include| mechanism
just needs two extra commands at the top of every file
such that all constituent files can be compiled individually.

%%%%%%%%%%%%%%%%%%%%%%%%%%%%%%%%%%%%%%%%%%%%%%%%%%%%%%%%%%%%%%%%%%%%%%%%%%%%%%%%
%\subsection{Feature Suggestions}
%
%The following is a list of features which may be useful for future
%versions of this package:
%%
%\begin{itemize}
%\item
%\ldots
%\end{itemize}

%%%%%%%%%%%%%%%%%%%%%%%%%%%%%%%%%%%%%%%%%%%%%%%%%%%%%%%%%%%%%%%%%%%%%%%%%%%%%%%%
\subsection{Revision History}

%%%%%%%%%%%%%%%%%%%%%%%%%%%%%%%%%%%%%%%%
\paragraph{v2.0:} 2018/12/30

\begin{itemize}
\item
immediate forward processing
\item
added |\childdocby| mechanism
\item
manual restructured
\end{itemize}

%%%%%%%%%%%%%%%%%%%%%%%%%%%%%%%%%%%%%%%%
\paragraph{v1.6:} 2018/01/17

\begin{itemize}
\item
application for development of include files
\item
corrections to manual
\end{itemize}

%%%%%%%%%%%%%%%%%%%%%%%%%%%%%%%%%%%%%%%%
\paragraph{v1.5:} 2017/05/21

\begin{itemize}
\item
more complete structuring introduced
\item
|\childdocof| introduced
\item
|\childdoc| renamed to |\childdocmain|
\item
|\childredirect| renamed to |\childdocforward| and |\childdocforwardprefix|
and functionality expanded
\end{itemize}

%%%%%%%%%%%%%%%%%%%%%%%%%%%%%%%%%%%%%%%%
\paragraph{v1.0:} 2017/04/27

\begin{itemize}
\item
manual and install package
\item
first version published on CTAN
\end{itemize}

%%%%%%%%%%%%%%%%%%%%%%%%%%%%%%%%%%%%%%%%
\paragraph{v0.6:} 2017/04/26

\begin{itemize}
\item
redirection mechanism added
\end{itemize}

%%%%%%%%%%%%%%%%%%%%%%%%%%%%%%%%%%%%%%%%
\paragraph{v0.5:} 2017/04/26

\begin{itemize}
\item
functionality in definition file
\end{itemize}


%%%%%%%%%%%%%%%%%%%%%%%%%%%%%%%%%%%%%%%%%%%%%%%%%%%%%%%%%%%%%%%%%%%%%%%%%%%%%%%%
%%%%%%%%%%%%%%%%%%%%%%%%%%%%%%%%%%%%%%%%%%%%%%%%%%%%%%%%%%%%%%%%%%%%%%%%%%%%%%%%
%%%%%%%%%%%%%%%%%%%%%%%%%%%%%%%%%%%%%%%%%%%%%%%%%%%%%%%%%%%%%%%%%%%%%%%%%%%%%%%%
\appendix

\settowidth\MacroIndent{\rmfamily\scriptsize 000\ }

 \DocInput{childdoc.dtx}

\end{document}
%</driver>
% \fi
%
% %%%%%%%%%%%%%%%%%%%%%%%%%%%%%%%%%%%%%%%%%%%%%%%%%%%%%%%%%%%%%%%%%%%%%%%%%%%%%%
% %%%%%%%%%%%%%%%%%%%%%%%%%%%%%%%%%%%%%%%%%%%%%%%%%%%%%%%%%%%%%%%%%%%%%%%%%%%%%%
% \section{Sample}
%\iffalse
%<*samplemain>
%\fi
%
% The following presents a sample document
% with two chapters, two parts, a title page,
% a compile flag as well as three forwarding files to set the flag.
% It consists of eight |.tex| files:
% \begin{center}
% \begin{tabular}{ll}
% |cdocsamp.tex|&main file\\
% |cdocsch1.tex|&include file for chapter 1\\
% |cdocsch2.tex|&include file for chapter 2\\
% |cdocspt3.tex|&include file for part 3\\
% |cdocspt4.tex|&include file for part 4\\
% |cdocsdrf.tex|&forwarding file for main file in draft mode\\
% |cdocsfi1.tex|&forwarding file for final version of chapter 1\\
% |cdocsfi2.tex|&forwarding file for final version of chapter 2\\
% \end{tabular}
% \end{center}
% Each of the eight files can be compiled directly by the \LaTeX{} compiler.
%
% %%%%%%%%%%%%%%%%%%%%%%%%%%%%%%%%%%%%%%
% \paragraph{Main File.}
%
% The main file is called |cdocsamp.tex|.
%
% Load the \textsf{childdoc} definitions and
% declare the filename for the main document:
%    \begin{macrocode}
\input{childdoc.def}
\childdocmain{}
%    \end{macrocode}

% Optional override for |\version| flag:
%    \begin{macrocode}
%%\ifchilddoc\else\providecommand{\version}{draft}\fi
%    \end{macrocode}

% Define the default values for the |\version| flag
% (|final| for the main file and |draft| for childs):
%    \begin{macrocode}
\ifchilddoc
\providecommand{\version}{draft}
\else
\providecommand{\version}{final}
\fi
%    \end{macrocode}

% Load the standard document class:
%    \begin{macrocode}
\documentclass[12pt]{article}
%    \end{macrocode}

% Start the document body:
%    \begin{macrocode}
\begin{document}
%    \end{macrocode}

% Declare a title page.
% Print title, part of document being processed and version flag:
%    \begin{macrocode}
\addtocounter{page}{-1}
\begin{center}
{\LARGE\bfseries{}childdoc example\par}
\vspace{1cm}
\ifchilddoc
\ifchilddocmanual part\else chapter\fi:
`\childdocname' of `\childdocjob'\par
\else
main document: `\childdocjob'\par
\fi
version: \version\par
\end{center}
\newpage
%    \end{macrocode}

% Manually include selected file,
% otherwise process as usual:
%    \begin{macrocode}
\ifchilddocmanual
\section*{part `\childdocname'}
\input{\childdocname}
\else
%    \end{macrocode}

% Include the two chapters:
%    \begin{macrocode}
\include{cdocsch1}
\include{cdocsch2}
%    \end{macrocode}

% Include the two parts unless only chapters should be displayed:
%    \begin{macrocode}
\ifchilddoc\else
\section{part three}
\input{cdocspt3}
\section{part four}
\input{cdocspt4}
\fi
%    \end{macrocode}

% Process as usual until here:
%    \begin{macrocode}
\fi
%    \end{macrocode}

% End of document body:
%    \begin{macrocode}
\end{document}
%    \end{macrocode}
%\iffalse
%</samplemain>
%\fi
%
% %%%%%%%%%%%%%%%%%%%%%%%%%%%%%%%%%%%%%%
% \paragraph{Chapter Include Files.}
%
% The include files are called |cdocsch1.tex| and |cdocsch2.tex|.
%
%\iffalse
%<*samplechap1|samplechap2>
%\fi

% Optional override for |\version| flag:
%    \begin{macrocode}
%%\providecommand{\version}{final}
%    \end{macrocode}

% Include the main document:
%    \begin{macrocode}
\input{childdoc.def}
\childdocof{cdocsamp}
%    \end{macrocode}

%\iffalse
%</samplechap1|samplechap2>
%\fi
%
%\iffalse
%<*samplechap1>
%\fi
% Some text for chapter 1:
%    \begin{macrocode}
\section{one}
some text in chapter one
%    \end{macrocode}

%\iffalse
%</samplechap1>
%\fi
% Some text for chapter 2:
%\iffalse
%<*samplechap2>
%\fi
%    \begin{macrocode}
\section{two}
more text in chapter two
%    \end{macrocode}

%\iffalse
%</samplechap2>
%\fi
%
% %%%%%%%%%%%%%%%%%%%%%%%%%%%%%%%%%%%%%%
% \paragraph{Part Include Files.}
%
% The include files are called |cdocspt3.tex| and |cdocspt4.tex|.
%
%\iffalse
%<*samplepart3|samplepart4>
%\fi

% Optional override for |\version| flag:
%    \begin{macrocode}
%%\providecommand{\version}{final}
%    \end{macrocode}

% Include the main document:
%    \begin{macrocode}
\input{childdoc.def}
\childdocby{cdocsamp}
%    \end{macrocode}

%\iffalse
%</samplepart3|samplepart4>
%\fi
%
%\iffalse
%<*samplepart3>
%\fi
% Some text for part 3:
%    \begin{macrocode}
some text in part three
%    \end{macrocode}

%\iffalse
%</samplepart3>
%\fi
% Some text for part 4:
%\iffalse
%<*samplepart4>
%\fi
%    \begin{macrocode}
more text in part four
%    \end{macrocode}

%\iffalse
%</samplepart4>
%\fi
%
% %%%%%%%%%%%%%%%%%%%%%%%%%%%%%%%%%%%%%%
% \paragraph{Forwarding for a Complete Draft.}
%
% The following forwarding file |cdocsdrf.tex|
% compiles the main document in draft mode:
%\iffalse
%<*sampledraft>
%\fi
%    \begin{macrocode}
\def\version{draft}
\input{childdoc.def}
\childdocforward{cdocsamp}
%    \end{macrocode}

%\iffalse
%</sampledraft>
%\fi
%
% %%%%%%%%%%%%%%%%%%%%%%%%%%%%%%%%%%%%%%
% \paragraph{Forwarding for Final Version of the Chapters.}
%
% The following forwarding files |cdocsfn1.tex| and |cdocsfn2.tex|
% (with identical content)
% compile the final versions of the child documents
% |cdocsch1.tex| and |cdocsch2.tex|, respectively:
%\iffalse
%<*samplefinal>
%\fi
%    \begin{macrocode}
\def\version{final}
\input{childdoc.def}
\childdocforwardprefix[cdocsamp]{cdocsfn}{cdocsch}
%    \end{macrocode}

%\iffalse
%</samplefinal>
%\fi
%
% %%%%%%%%%%%%%%%%%%%%%%%%%%%%%%%%%%%%%%
% \paragraph{Command Line Processing.}
%
% The following three command lines generate the output files
% |cdocscld|, |cdocscl1| and |cdocscl2|
% which should be identical to
% |cdocsdrf|, |cdocsch1| and |cdocsfn2|, respectively:
% \begin{center}
% \begin{tabular}{l}
% |latex -jobname cdocscld \|\\
% |  "\def\version{draft}\input{childdoc.def}\childdocforward{cdocsamp}"|\\
% |latex -jobname cdocscl1 \|\\
% |  "\input{childdoc.def}\childdocforward[cdocsamp]{cdocsch1}"|\\
% |latex -jobname cdocscl2 \|\\
% |  "\def\version{final}\input{childdoc.def}\childdocforward{cdocsch2}"|
% \end{tabular}
% \end{center}
% Note that the trailing backslash on each first line
% merely continues the input to the second line
% (for convenient cut ant paste).
% Furthermore, the command |latex| can be replaced by any
% of its alternative versions such as |pdflatex|.
%
% %%%%%%%%%%%%%%%%%%%%%%%%%%%%%%%%%%%%%%%%%%%%%%%%%%%%%%%%%%%%%%%%%%%%%%%%%%%%%%
% %%%%%%%%%%%%%%%%%%%%%%%%%%%%%%%%%%%%%%%%%%%%%%%%%%%%%%%%%%%%%%%%%%%%%%%%%%%%%%
% \section{Implementation}
%\iffalse
%<*package>
%\fi
%
% This section describes the definitions file |childdoc.def|.

% The definitions cannot be loaded using |\usepackage| or |\RequirePackage|
% which has a mechanism to prevent loading a style file more than once.
% When loading the definitions by means of |\input|
% multiple instances have to be prevented manually:
%\iffalse
%This code needs to be before the `\ProvidesFile' directive
%which is defined at the beginning of this file.
%Therefore it is also placed there and commented out here.
%</package>
%<*discard>
%\fi
%    \begin{macrocode}
\ifdefined\childdocmain\endinput\fi
%    \end{macrocode}
%\iffalse
%</discard>
%<*package>
%\fi
%
% \macro{\ifchilddoc}
% \macro{\ifchilddocmanual}
% The conditional |\ifchilddoc| tells whether a
% child (true) or main (false) document is being compiled.
% The conditional |\ifchilddocmanual| tells whether
% the |\includeonly| mechanism is used (false) or
% the selection of child files must be performed manually (true).
% The definitions initialise to false:
%    \begin{macrocode}
\newif\ifchilddoc
\newif\ifchilddocmanual
%    \end{macrocode}

% \macro{\childdocname}
% \macro{\childdocjob}
% The macro |\childdocname| stores the name of the main document
% to be compiled. The macro |\childdocjob| stores the name of
% the document on which the \LaTeX{} compiler was originally invoked.
% The content of |\jobname| cannot be compared
% to filenames specified in the source due to different catcodes.
% The following code rescans |\jobname|, stores the result
% in |\childdocname| and saves a copy in |\childdocjob|:
%    \begin{macrocode}
\edef\childdocname{\scantokens\expandafter{\jobname\noexpand}}
\let\childdocjob\childdocname
%    \end{macrocode}

% \macro{\childdocdisable}
% The macro |\childdocdisable| prevents the main file
% from being processed more than once.
% At this stage, the main document command |\childdocmain|
% is assumed to be called once again where it should do nothing.
% Any subsequent call to it should prevent
% a secondary processing of the main document
% It overwrites the forwarding commands
% |\childdocof| and |\childdocforward|
% with empty macros to prevent further inclusions of the main document:
%    \begin{macrocode}
\newcommand{\childdocdisable}
{
  \renewcommand{\childdocmain}[1]{\renewcommand{\childdocmain}[1]{\endinput}}
  \renewcommand{\childdocof}[1]{}
  \renewcommand{\childdocby}[2][]{}
  \renewcommand{\childdocforward}[2][]{}
  \renewcommand{\childdocdisable}{}
}
%    \end{macrocode}

% \macro{\childdocmain}
% The macro |\childdocmain| is to be called at the top of the main file
% with nothing or the main filename (without extension) as argument.
% First, it breaks loops.
% If the argument is not empty and does not match |\childdocname|
% (which is set by the first inclusion of |childdoc.def|),
% |\ifchilddoc| is set to true, |\includeonly| is applied to the child file
% and |\jobname| is set to the main file
% (for proper handling of |.aux| files):
%    \begin{macrocode}
\newcommand{\childdocmain}[1]
{
  \childdocdisable\childdocmain{}
  \if?#1?\else
    \begingroup
      \def\childdoctmp{#1}
      \ifx\childdoctmp\childdocname
        \def\childdoctmp{}
      \else
        \def\childdoctmp
        {
          \childdoctrue
          \includeonly{\childdocname}
          \def\childdocjob{#1}
          \def\jobname{#1}
        }
      \fi
      \expandafter
    \endgroup
    \childdoctmp
  \fi
}
%    \end{macrocode}

% \macro{\childdocof}
% The command |\childdocof| redirects
% compilation to the main file |#1|.
%    \begin{macrocode}
\newcommand{\childdocof}[1]
{
  \childdocdisable
  \childdoctrue
  \includeonly{\childdocname}
  \def\jobname{#1}
  \def\childdocjob{#1}
  \input{#1}
}
%    \end{macrocode}

% \macro{\childdocby}
% The command |\childdocby| ....
%    \begin{macrocode}
\newcommand{\childdocby}[2][]
{
  \childdocdisable
  \childdoctrue
  \childdocmanualtrue
  \if?#1?\else
    \def\jobname{#2}
  \fi
  \def\childdocjob{#2}
  \input{#2}
  \endinput
}
%    \end{macrocode}

% \macro{\childdocforward}
% The command |\childdocforward| redirects
% compilation to the main file or
% (if the optional argument is given) a child file.
% Parameters are set as if the main file
% or a child file starting with |\childdocof| was compiled.
% Then compilation is handed over to the main file:
%    \begin{macrocode}
\newcommand{\childdocforward}[2][]
{
  \begingroup
    \if?#1?
      \def\childdoctmp
      {
        \def\childdocname{#2}
        \def\childdocjob{#2}
        \def\jobname{#2}
        \input{#2}
        \endinput
      }
    \else
      \def\childdoctmp
      {
        \childdocdisable
        \def\childdocname{#2}
        \childdoctrue
        \includeonly{#2}
        \def\childdocjob{#1}
        \def\jobname{#1}
        \input{#1}
        \endinput
      }
    \fi
    \expandafter
  \endgroup
  \childdoctmp
}
%    \end{macrocode}

% \macro{\childdocforwardprefix}
% The command |\childdocforwardprefix| redirects
% compilation to the main or a child file by means of a pattern.
% The prefix |#1| in the current filename is replaced by |#2|
% and the suffix of the current filename is kept
% (it is assumed that the filename does not contain the substring `|~~~|'
% which is used as a delimiter).
% Compilation is handed over to the new file by |\childdocforward|:
%    \begin{macrocode}
\newcommand{\childdocforwardprefix}[3][]
{
  \begingroup
    \def\childdocextract #2##1~~~{\def\childdoctmp{\childdocforward[#1]{#3##1}}}
    \expandafter\childdocextract\childdocname~~~
    \expandafter
  \endgroup
  \childdoctmp
}
%    \end{macrocode}

% \macro{\childdoc}
% The deprecated macro |\childdoc| is a legacy version of |\childdocmain|:
%    \begin{macrocode}
\newcommand{\childdoc}{\childdocmain}
%    \end{macrocode}

% \macro{\childdocredirect}
% The deprecated macro |\childdocredirect| is a legacy version
% of |\childdocforward| and |\childdocforwardprefix|:
%    \begin{macrocode}
\newcommand{\childdocredirect}[2][]
{
  \begingroup
    \if?#1?
      \def\childdoctmp{\childdocforward{#2}}
    \else
      \def\childdoctmp{\childdocforwardprefix{#1}{#2}}
    \fi
    \expandafter
  \endgroup
  \childdoctmp
}
%    \end{macrocode}

%\iffalse
%</package>
%\fi
%
\endinput
|\\
|\childdocforward{|\textit{main}|}|
\end{tabular}
\end{center}
%
Likewise, the following files |final|\textit{nn}|.tex|
compile the final version of the child document
|child|\textit{nn}|.tex|:
%
\begin{center}
\begin{tabular}{l}
|\def\version{final}|\\
|% \iffalse
%
% childdoc.dtx Copyright (C) 2017-2018 Niklas Beisert
%
% This work may be distributed and/or modified under the
% conditions of the LaTeX Project Public License, either version 1.3
% of this license or (at your option) any later version.
% The latest version of this license is in
%   http://www.latex-project.org/lppl.txt
% and version 1.3 or later is part of all distributions of LaTeX
% version 2005/12/01 or later.
%
% This work has the LPPL maintenance status `maintained'.
%
% The Current Maintainer of this work is Niklas Beisert.
%
% This work consists of the files childdoc.dtx and childdoc.ins
% and the derived files childdoc.def and cdocsamp.tex with
% cdocsch1.tex, cdocsch2.tex, cdocsdrf.tex, cdocsfn1.tex, cdocsfn2.tex.
%
%<package>\ifdefined\childdocmain\endinput\fi
%<package>\ProvidesFile{childdoc.def}[2018/12/30 v2.0 child document driver]
%<samplemain>\ProvidesFile{cdocsamp.tex}[2018/12/30 v2.0 sample for childdoc]
%<*driver>
%\ProvidesFile{childdoc.drv}[2018/12/30 v2.0 childdoc reference manual file]
\PassOptionsToClass{10pt,a4paper}{article}
\documentclass{ltxdoc}

\usepackage[margin=35mm]{geometry}
\usepackage{hyperref}
\usepackage{hyperxmp}
\usepackage[usenames]{color}

\hypersetup{colorlinks=true}
\hypersetup{pdfstartview=FitH}
\hypersetup{pdfpagemode=UseNone}
\hypersetup{pdfsource={}}
\hypersetup{pdflang={en-UK}}
\hypersetup{pdfcopyright={Copyright 2017-2018 Niklas Beisert.
  This work may be distributed and/or modified under the
  conditions of the LaTeX Project Public License, either version 1.3
  of this license or (at your option) any later version.}}
\hypersetup{pdflicenseurl={http://www.latex-project.org/lppl.txt}}
\hypersetup{pdfcontactaddress={ETH Zurich, ITP, HIT K,
  Wolfgang-Pauli-Strasse 27}}
\hypersetup{pdfcontactpostcode={8093}}
\hypersetup{pdfcontactcity={Zurich}}
\hypersetup{pdfcontactcountry={Switzerland}}
\hypersetup{pdfcontactemail={nbeisert@itp.phys.ethz.ch}}
\hypersetup{pdfcontacturl={http://people.phys.ethz.ch/\xmptilde nbeisert/}}

\newcommand{\secref}[1]{\hyperref[#1]{section \ref*{#1}}}

\parskip1ex
\parindent0pt
\let\olditemize\itemize
\def\itemize{\olditemize\parskip0pt}

\begin{document}

\title{The \textsf{childdoc} Package}
\hypersetup{pdftitle={The childdoc Package}}
\author{Niklas Beisert\\[2ex]
  Institut f\"ur Theoretische Physik\\
  Eidgen\"ossische Technische Hochschule Z\"urich\\
  Wolfgang-Pauli-Strasse 27, 8093 Z\"urich, Switzerland\\[1ex]
  \href{mailto:nbeisert@itp.phys.ethz.ch}
  {\texttt{nbeisert@itp.phys.ethz.ch}}}
\hypersetup{pdfauthor={Niklas Beisert}}
\hypersetup{pdfsubject={Manual for the LaTeX2e Package childdoc}}
\date{30 December 2018, \textsf{v2.0}}
\maketitle

\begin{abstract}\noindent
\textsf{childdoc} is a \LaTeXe{} package
that enables the direct compilation
of document sections included by |\include|
to individual files.
\end{abstract}

\begingroup
\parskip0ex
\tableofcontents
\endgroup

%%%%%%%%%%%%%%%%%%%%%%%%%%%%%%%%%%%%%%%%%%%%%%%%%%%%%%%%%%%%%%%%%%%%%%%%%%%%%%%%
%%%%%%%%%%%%%%%%%%%%%%%%%%%%%%%%%%%%%%%%%%%%%%%%%%%%%%%%%%%%%%%%%%%%%%%%%%%%%%%%
\section{Introduction}

\LaTeX{} provides a mechanism to structure a large document (such as a book)
into a main file and several child files (containing the chapters)
using the |\include| command.
This mechanism is beneficial for documents
which span hundreds of pages in order to
make the source file(s) more manageable.
Moreover, compilation can be restricted to
selected child files by means of the |\includeonly| command.
The latter feature can be used to reduce the compilation time while editing
(this was significantly more useful in the earlier days of \LaTeX{})
or to generate a smaller document which is easier to navigate.
Another application of |\includeonly| is to generate
documents consisting of selected parts of the complete document.

However, there are a few drawbacks of the plain |\include| mechanism:
\begin{itemize}
\item
The child files cannot be compiled on their own,
they can only be compiled via the main file.
A naive editing environment
(such as a text editor with an option
to have the current file processed by \LaTeX)
may require one to switch to the main file before compiling;
attempting to compile the child file produces errors.
\item
The main file must be modified (each time)
to adjust the |\includeonly| command
to the present needs. This easily leaves the main file in a messy state.
\item
The generated document will always carry the filename
of the main document. This is inconvenient if
several child files are to be compiled and
to be kept for distribution.
\end{itemize}

The present package provides a simple interface
to make child files individually compilable by \LaTeX{}.
Compiling a child file then has the same effect as compiling
the main file with an |\includeonly| command
to select the appropriate child.
Moreover the generated document will carry the name of the child
rather than the main file.
This resolves all three above issues.

This feature is meant to make the editing of books,
thesis documents and lecture notes somewhat more convenient.
However, the package can also be used efficiently for
composing a series of documents (such as exercise sheets)
which are typically distributed individually.
It then assists the author in generating the individual documents
(potentially in different versions)
as well as a document containing the collected series.
Another application is in developing style files
or other kinds of included material
where compilation of the style file could redirect
to a sample or test file.

%%%%%%%%%%%%%%%%%%%%%%%%%%%%%%%%%%%%%%%%%%%%%%%%%%%%%%%%%%%%%%%%%%%%%%%%%%%%%%%%
%%%%%%%%%%%%%%%%%%%%%%%%%%%%%%%%%%%%%%%%%%%%%%%%%%%%%%%%%%%%%%%%%%%%%%%%%%%%%%%%
\section{Usage}

First of all, the package \textsf{childdoc} is \emph{not} a standard
\LaTeXe{} |.sty| style file! Therefore it needs to be invoked in
a non-standard way.

%%%%%%%%%%%%%%%%%%%%%%%%%%%%%%%%%%%%%%%%%%%%%%%%%%%%%%%%%%%%%%%%%%%%%%%%%%%%%%%%
\subsection{Included Files}
\label{sec:include}

%%%%%%%%%%%%%%%%%%%%%%%%%%%%%%%%%%%%%%%%
\DescribeMacro{\childdocmain}
To use the package, add the commands
\begin{center}
\begin{tabular}{l}
|\input{childdoc.def}|\\
|\childdocmain{}|\\
\end{tabular}
\end{center}
at the very top of the main \LaTeX{} file,
in particular \emph{before} the |\documentclass| statement!
The argument of |\childdocmain| should be left empty
(but it must be present).

%%%%%%%%%%%%%%%%%%%%%%%%%%%%%%%%%%%%%%%%
\DescribeMacro{\childdocof}
Furthermore, add the commands
\begin{center}
\begin{tabular}{l}
|\input{childdoc.def}|\\
|\childdocof{|\textit{main}|}|\\
\end{tabular}
\end{center}
at the top of every child file \textit{child}
which is included by |\include{|\textit{child}|}|
from within the main file
(or at least for those files to be compiled individually).
The argument \textit{main} must be the filename of the main file.

There are a couple of
considerations in setting up the main and child documents:

%%%%%%%%%%%%%%%%%%%%%%%%%%%%%%%%%%%%%%%%
\paragraph{Restrictions.}

Please note the following restrictions:
\begin{itemize}
\item
|\childdocmain| must be called with one argument \textit{main}
to ensure compatibility with earlier version of the package.
It must either be empty (|\childdocmain{}|)
or precisely match the filename of the main file in which it is specified.
See \secref{sec:detection} for further information.
\item
The filename \textit{main} must be specified without the |.tex| extension.
\item
The filename \textit{main} is case sensitive
(even in case-insensitive file systems)
due to internal string comparison.
\item
The argument \textit{main} should be fully expanded, it cannot be a macro.
\item
Subdirectories and special characters should be avoided in filenames.
\item
The command |\childdocmain{|\textit{main}|}| must be followed by a whitespace.
It should not be followed immediately by another command
or by a comment mark `|%|'.
This is because the \TeX{} parser reads the token immediately following
the argument of |\childdocmain| and puts it
at the beginning of every child section;
however, a white\-space is ignored.
\end{itemize}

%%%%%%%%%%%%%%%%%%%%%%%%%%%%%%%%%%%%%%%%
\paragraph{Content of Main File.}

It is advisable to place all content in the child files included by |\include|.
Any output contained in the main file will appear in all child documents
unless suppressed manually;
it cannot be suppressed automatically by the |\includeonly| directive
and thus should normally be avoided.
A method to include some content in the main file
by means of conditional processing is described in \secref{sec:conditional}.

%%%%%%%%%%%%%%%%%%%%%%%%%%%%%%%%%%%%%%%%
\paragraph{Page Numbering.}

When only a part of the document is compiled,
the appropriate numbering of pages
(as well as other status parameters)
is determined from the |.aux| files.
The latter contain information from previous passes.
However this information needs to propagate through
all intermediate child documents.
Therefore the page numbering in child documents may well
be inconsistent until the complete document is compiled at least once.

A useful (if unconventional) way to always ensure a consistent
page numbering is to restart the numbering in each child document
and denote the pages by `\textit{child}|.|\textit{page}'
where \textit{child} represents the chapter/section number of the child file.
This can be achieved by the command
|\numberwithin{page}{|\textit{child}|}|
of the \textsf{amsmath} package
where \textit{child} can be |chapter| or |section|
depending on the chosen structuring.
Alternatively, one can modify the macro |\thepage| appropriately
and reset the counter |page| at the start of each child file.

%%%%%%%%%%%%%%%%%%%%%%%%%%%%%%%%%%%%%%%%%%%%%%%%%%%%%%%%%%%%%%%%%%%%%%%%%%%%%%%%
\subsection{Conditional Processing}
\label{sec:conditional}

The package provides a mechanism to compile different versions
of a document. To customise the versions further some conditional processing
can come in handy to distinguish which version is being compiled.
The package provides two macros to describe the compilation context:

%%%%%%%%%%%%%%%%%%%%%%%%%%%%%%%%%%%%%%%%
\DescribeMacro{\ifchilddoc}
The conditional |\ifchilddoc| distinguishes between the compilation of
child documents and the main document:
%
\begin{center}
|\ifchilddoc |\textit{child-code}| |[|\||else |\textit{main-code}]| \||fi|
\end{center}

%%%%%%%%%%%%%%%%%%%%%%%%%%%%%%%%%%%%%%%%
\DescribeMacro{\childdocname}
\DescribeMacro{\childdocjob}
The macro |\childdocname| contains the filename (without extension)
of the main or child file being processed.
Note that |\childdocjob| will always contain the name of the main file.

%%%%%%%%%%%%%%%%%%%%%%%%%%%%%%%%%%%%%%%%
\paragraph{Title Page.}

Conditional processing can be used to include a title or banner page
in the main document when proper precautions are taken.
Importantly, the code in the main file should ensure that the page counter
(as well as other status parameters which are stored in the |.aux| files)
takes the same value after the conditional processing.
Otherwise the page numbers may take divergent values
depending on which part is compiled.

For example, a title page could be declared by:
%
\begin{center}
\begin{tabular}{l}
|\ifchilddoc\||else|\\
|\addtocounter{page}{-1}|\\
\textit{code for title page}\\
|\newpage|\\
|\||fi|
\end{tabular}
\end{center}
%
A banner page for the child documents can be generated by:
%
\begin{center}
\begin{tabular}{l}
|\ifchilddoc|\\
|\addtocounter{page}{-1}|\\
\textit{code for banner page}\\
|\newpage|\\
|\||fi|
\end{tabular}
\end{center}
%
Here one could write a message such as:
\begin{center}
|This is the part \childdocname{} of \childdocjob{}.|
\end{center}

%%%%%%%%%%%%%%%%%%%%%%%%%%%%%%%%%%%%%%%%%%%%%%%%%%%%%%%%%%%%%%%%%%%%%%%%%%%%%%%%
\subsection{Flags}
\label{sec:flags}

The package makes it easy to generate different versions
of the main or child documents.
To this end compilation flags can be defined
and assigned different default values.
They will be particularly useful in conjunction
with the forwarding mechanism described in \secref{sec:forward}.

For example, it may be useful to have a flag |\version|
which can be set to |draft| or |final|.
The document source will contain some conditional code
depending on the value of |\version|.
Suppose further, the flag should default to |final| for the main file
and to |draft| for child files
which is a natural assignment for editing the document.
This is achieved by placing the following code
in the preamble of the main document
(below the |\childdocmain| directive):
%
\begin{center}
\begin{tabular}{l}
|\ifchilddoc|\\
|\providecommand{\version}{draft}|\\
|\||else|\\
|\providecommand{\version}{final}|\\
|\||fi|
\end{tabular}
\end{center}
%
The definition by |\providecommand| makes sure
that previous definitions are not overwritten.
Further statements |\providecommand{\version}{...}|
can thus be added before the above code to override it.

For the main file, one might add a line
(between |\childdocmain| and the above block)
%
\begin{center}
|%\ifchilddoc\||else\providecommand{\version}{draft}\||fi|
\end{center}
%
which can be uncommented to produce a draft version.
Likewise one can add a line to the very top of a child file
(above the |\childdocof{|\textit{main}|}| directive)
%
\begin{center}
|%\providecommand{\version}{final}|
\end{center}
%
which can be uncommented to produce the final version of this child document.

%%%%%%%%%%%%%%%%%%%%%%%%%%%%%%%%%%%%%%%%%%%%%%%%%%%%%%%%%%%%%%%%%%%%%%%%%%%%%%%%
\subsection{Forwarding}
\label{sec:forward}

Different versions of the main or child documents
using compilation flags as described in \secref{sec:flags}
can be (permanently) stored in different files
for convenient compilation, viewing and distribution.
To this end, the package defines a command
to pass on compilation to a different file:

%%%%%%%%%%%%%%%%%%%%%%%%%%%%%%%%%%%%%%%%
\DescribeMacro{\childdocforward}
The command |\childdocforward| redirects processing to
another source file:
%
\begin{center}
\begin{tabular}{l}
|\input{childdoc.def}|\\
|\childdocforward[|\textit{main}|]{|\textit{dest}|}|\\
\end{tabular}
\end{center}
%
The argument \textit{dest} is the destination file
(without extension).
It should be the main file or one of the child files.
Note that further \textsf{childdoc} directives
such as |\childdocof| and |\childdocforward|
in the indicated file will be processed in this form.
The optional argument \textit{main}
passes on directly to the main file \textit{main}
while pretending to compile the child \textit{dest}.
This form behaves as if \textit{dest}
issues |\childdocof{|\textit{main}|}| right away,
and no further \textsf{childdoc} directives will be processed.

%%%%%%%%%%%%%%%%%%%%%%%%%%%%%%%%%%%%%%%%
\DescribeMacro{\...prefix}
In the alternative form |\childdocforwardprefix|,
%
\begin{center}
\begin{tabular}{l}
|\input{childdoc.def}|\\
|\childdocforwardprefix[|\textit{main}|]{|\textit{prefix}|}{|\textit{dest}|}|
\end{tabular}
\end{center}
%
the destination file is determined by a pattern
depending on the current file:
To make this work, the current file must be called
`{\textit{prefix}\hspace{0.2em}\textit{suffix}}'
with \textit{prefix} matching precisely the argument.
Processing is then passed on to the file
`{\textit{dest}\hspace{0.2em}\textit{suffix}}'.
Surely, the same effect is achieved by
directly specifying the
argument `{\textit{dest}\hspace{0.2em}\textit{suffix}}'
in the first form.
However, that requires to set up a different file
for each child. With the alternative form of the command
all these files can have exactly the same content
which simplifies setting them up and maintaining them.

For example, the following file |draft.tex|
with a compilation flag |\version| as described in \secref{sec:flags}
compiles the main document as a draft:
%
\begin{center}
\begin{tabular}{l}
|\def\version{draft}|\\
|\input{childdoc.def}|\\
|\childdocforward{|\textit{main}|}|
\end{tabular}
\end{center}
%
Likewise, the following files |final|\textit{nn}|.tex|
compile the final version of the child document
|child|\textit{nn}|.tex|:
%
\begin{center}
\begin{tabular}{l}
|\def\version{final}|\\
|\input{childdoc.def}|\\
|\childdocforwardprefix{final}{child}|
\end{tabular}
\end{center}
%

Note that when several versions of a main file and/or of each child file
are to be generated, it may be convenient to set up a |Makefile| or
shell script to automatise the process.

%%%%%%%%%%%%%%%%%%%%%%%%%%%%%%%%%%%%%%%%%%%%%%%%%%%%%%%%%%%%%%%%%%%%%%%%%%%%%%%%
\subsection{Command Line Processing}
\label{sec:commandline}

The effect of redirection files can also be achieved by invoking
the \LaTeX{} compiler with a more elaborate command line.
Most conveniently this should be done as part
of a shell script or a |Makefile|.

When using \textsf{childdoc} in the main file, the following
command lines effectively perform a redirection
(note that depending on the shell being used,
backslashes may have to be doubled: `|\|' $\to$ `|\\|'):
%
\begin{center}
|... -jobname "|\textit{target}|" |\\|"|[\textit{flags}]%
|\input{childdoc.def}\childdocforward[|\textit{main}|]{|\textit{dest}|}"|
\end{center}
%
Here \textit{target} is the name of the output file,
\textit{main} is the name of the main file
and \textit{dest} is the name of the main or child file to be processed
(all filenames without extensions).
The optional argument \textit{main} can be omitted
if \textit{main} matches \textit{dest}.
Optionally, compilation \textit{flags} can be defined via |\def| commands.
This command line makes the \TeX{} engine believe
it is compiling the file \textit{target}
whose content is specified as the latter parameter.
The provided code then forwards the processing to
\textit{main} or \textit{dest} as described in \secref{sec:forward}.

%%%%%%%%%%%%%%%%%%%%%%%%%%%%%%%%%%%%%%%%%%%%%%%%%%%%%%%%%%%%%%%%%%%%%%%%%%%%%%%%
\subsection{Include by Input}
\label{sec:input}

Including child documents by |\include| has some restrictions by design.
Most notably, the content of a child document always occupies
its own set of pages; pages cannot be shared between child documents.
Usually, this behaviour makes perfect sense
because each child document contain an essential part of the document.
However, in some situations it may be desirable to compose
a document from a collection of parts
without having mandatory page breaks between then.
For this case, the package
provides a mechanism to include parts
by |\input| which can also be processed individually.
However, by construction this mechanism
requires manual handling of the content to be output.

%%%%%%%%%%%%%%%%%%%%%%%%%%%%%%%%%%%%%%%%
\DescribeMacro{\ifchilddocmanual}
The main file should be prepared as usual, see \secref{sec:include}.
However, the document body must make a distinction
between processing of an individual part and of the main document, e.g.:
%
\begin{center}
\begin{tabular}{l}
|\ifchilddocmanual|\\
|\input{\childdocname}|\\
|\||else|\\
\textit{document body with }|\input{|\textit{part}|}|\\
|\||fi|
\end{tabular}
\end{center}
%
The conditional |\ifchilddocmanual| is true whenever
a part to be included by |\input| is being compiled,
and the name of the part is stored in |\childdocname|.

%%%%%%%%%%%%%%%%%%%%%%%%%%%%%%%%%%%%%%%%
\DescribeMacro{\childdocby}
Each part to be included by |\input| should start with:
%
\begin{center}
\begin{tabular}{l}
|\input{childdoc.def}|\\
|\childdocby{|\textit{main}|}|\\
\end{tabular}
\end{center}
%
The directive |\childdocby| is similar to |\childdocof|
described in \secref{sec:include},
but the subsequent selection of content must be done manually.
To that end, both |\ifchilddoc| and |\ifchilddocmanual|
will be true upon processing of a part,
and the name of the part is stored in |\childdocname|.
Note that |\jobname| will be set to the filename of the current part
so that each part receives an individual |.aux| file
that does not interfere with the |.aux| file(s) of the main document.
This behaviour can be altered by the alternative form
|\childdocby[*]{|\textit{main}|}| (with a non-empty optional argument)
which uses the |.aux| file of the main document
by setting |\jobname| to \textit{main}.

%%%%%%%%%%%%%%%%%%%%%%%%%%%%%%%%%%%%%%%%%%%%%%%%%%%%%%%%%%%%%%%%%%%%%%%%%%%%%%%%
\subsection{Driver Development}
\label{sec:driver}

The \textsf{childdoc} mechanism can also be use for the development
of definition files such as \LaTeX{} styles or classes.
This case differs from the above setup with multiple parts
included by |\include| in that no |\includeonly| should be invoked.
This can be achieved by starting the include file
(before |\ProvidesPackage|) with:
%
\begin{center}
\begin{tabular}{l}
|\input{childdoc.def}|\\
|\childdocforward{|\textit{main}|}|\\
\end{tabular}
\end{center}
%
or alternatively with:
%
\begin{center}
\begin{tabular}{l}
|\input{childdoc.def}|\\
|\childdocby{|\textit{main}|}|\\
\end{tabular}
\end{center}
%
Both forms have slightly different effects as described above.
The main file is prepared as usual, see \secref{sec:include}.

%%%%%%%%%%%%%%%%%%%%%%%%%%%%%%%%%%%%%%%%%%%%%%%%%%%%%%%%%%%%%%%%%%%%%%%%%%%%%%%%
\subsection{Legacy Detection}
\label{sec:detection}

The directive |\childdocmain| in the main file can detect
whether the complete document or merely a child is to be compiled
even without using the directive |\childdocof|.
This method is deprecated because it is less robust
and there is no compelling reason to use it;
it is merely provided for backward compatibility
and it may be removed in future versions.

If the detection mechanism is to be used,
it is mandatory to correctly specify
the filename of the main file as the argument of |\childdocmain|:
%
\begin{center}
\begin{tabular}{l}
|\input{childdoc.def}|\\
|\childdocmain{|\textit{main}|}|\\
\end{tabular}
\end{center}
%
If |\jobname| does not match the argument \textit{main} of |\childdocmain|,
it is assumed that |\jobname| points to the child file to be compiled.
When using |\childdocmain| with the main file specified as argument,
it suffices to start a child file
with just |\input{|\textit{main}|}|
without loading of the package and using |\childdocof|.
If instead all processing is done
with the appropriate \textsf{childdoc} directives,
the argument of \textit{main} of |\childdocmain| can be empty.

An alternative version of the command line processing described
in \secref{sec:commandline} using the detection mechanism reads:
%
\begin{center}
|... -jobname "|\textit{target}|" "|[\textit{flags}]%
[|\def\jobname{|\textit{dest}|}|]|\input{|\textit{main}|}"|
\end{center}

%%%%%%%%%%%%%%%%%%%%%%%%%%%%%%%%%%%%%%%%%%%%%%%%%%%%%%%%%%%%%%%%%%%%%%%%%%%%%%%%
\subsection{Manual Code}
\label{sec:manual}

In case one cannot be certain whether the definitions file |childdoc.def|
is installed on the target \TeX{} distribution
and one prefers not to ship it,
it is conceivable to paste a few relevant commands into the sources.

To that end, drop all statements |\input{childdoc.def}|
and perform the replacements as outlined below.
Instead of |\childdocmain{|\textit{main}|}| add the following code
to the top of the main file:
%
\begin{center}
\begin{tabular}{l}
|\||ifdefined\childdocname\endinput\||fi\newif\ifchilddoc|\\
|\edef\childdocname{\scantokens\expandafter{\jobname\noexpand}}|\\
|\def\childdocmain{|\textit{main}|}\||ifx\childdocmain\childdocname\||else|\\
|\childdoctrue\includeonly{\childdocname}\let\jobname\childdocmain\||fi|\\
\end{tabular}
\end{center}
%
Instead of |\childdocof{|\textit{main}|}| just include the main file
at the top of each child file:
%
\begin{center}
|\input{|\textit{main}|}|
\end{center}
%
A simple redirection |\childdocforward{|\textit{dest}|}| is achieved by:
%
\begin{center}
|\def\jobname{|\textit{dest}|}\input{\jobname}|
\end{center}
%
The redirection with prefix
|\childdocforwardprefix[|\textit{prefix}|]{|\textit{dest}|}|
is accomplished by:
%
\begin{center}
\begin{tabular}{l}
|{\edef\jobname{\scantokens\expandafter{\jobname\noexpand}}|\\
|\def\redirectjob |\textit{prefix}|#1~~~{\gdef\jobname{|\textit{dest}|#1}}|\\
|\expandafter\redirectjob\jobname~~~}\input{\jobname}|
\end{tabular}
\end{center}

In an alternative approach,
child documents can be compiled by a specific command line
without additional code or specific definitions:
%
\begin{center}
|... -jobname "|\textit{target}|" "|[\textit{flags}]%
|\includeonly{|\textit{dest}|}\input{|\textit{main}|}"|
\end{center}
%

%%%%%%%%%%%%%%%%%%%%%%%%%%%%%%%%%%%%%%%%%%%%%%%%%%%%%%%%%%%%%%%%%%%%%%%%%%%%%%%%
%%%%%%%%%%%%%%%%%%%%%%%%%%%%%%%%%%%%%%%%%%%%%%%%%%%%%%%%%%%%%%%%%%%%%%%%%%%%%%%%
\section{Information}

%%%%%%%%%%%%%%%%%%%%%%%%%%%%%%%%%%%%%%%%%%%%%%%%%%%%%%%%%%%%%%%%%%%%%%%%%%%%%%%%
\subsection{Copyright}

Copyright \copyright{} 2017--2018 Niklas Beisert

This work may be distributed and/or modified under the
conditions of the \LaTeX{} Project Public License, either version 1.3
of this license or (at your option) any later version.
The latest version of this license is in
  \url{http://www.latex-project.org/lppl.txt}
and version 1.3 or later is part of all distributions of \LaTeX{}
version 2005/12/01 or later.

This work has the LPPL maintenance status `maintained'.

The Current Maintainer of this work is Niklas Beisert.

This work consists of the files |README.txt|, |childdoc.ins| and |childdoc.dtx|
as well as the derived files |childdoc.def|, |cdocsamp.tex|
with |cdocsch1.tex|, |cdocsch2.tex|, |cdocspt3.tex|, |cdocspt4.tex|,
|cdocsdrf.tex|, |cdocsfn1.tex|, |cdocsfn2.tex|
as well as |childdoc.pdf|.

%%%%%%%%%%%%%%%%%%%%%%%%%%%%%%%%%%%%%%%%%%%%%%%%%%%%%%%%%%%%%%%%%%%%%%%%%%%%%%%%
\subsection{Files and Installation}

The package consists of the files:
%
\begin{center}
\begin{tabular}{ll}
    |README.txt|   & readme file \\
    |childdoc.ins| & installation file \\
    |childdoc.dtx| & source file \\
    |childdoc.def| & definition file \\
    |cdocsamp.tex| & sample main file \\
    |cdocsch1.tex| & sample include file \\
    |cdocsch2.tex| & sample include file \\
    |cdocspt3.tex| & sample part file \\
    |cdocspt4.tex| & sample part file \\
    |cdocsdrf.tex| & sample redirection file \\
    |cdocsfn1.tex| & sample redirection file \\
    |cdocsfn2.tex| & sample redirection file \\
    |childdoc.pdf| & manual
\end{tabular}
\end{center}
%
The distribution consists of the files
|README.txt|, |childdoc.ins| and |childdoc.dtx|.
%
\begin{itemize}
\item
Run (pdf)\LaTeX{} on |childdoc.dtx|
to compile the manual |childdoc.pdf| (this file).
\item
Run \LaTeX{} on |childdoc.ins| to create the definitions file |childdoc.def|
and the sample |cdocsamp.tex| with include files
|cdocsch1.tex|, |cdocsch2.tex|, |cdocspt3.tex|, |cdocspt4.tex|,
|cdocsdrf.tex|, |cdocsfn1.tex|, |cdocsfn2.tex|.
Then copy the file |childdoc.def| to an appropriate directory of your \LaTeX{}
distribution, e.g.\ \textit{texmf-root}|/tex/latex/childdoc|.
\end{itemize}

%%%%%%%%%%%%%%%%%%%%%%%%%%%%%%%%%%%%%%%%%%%%%%%%%%%%%%%%%%%%%%%%%%%%%%%%%%%%%%%%
\subsection{Related CTAN Packages}

There are several other packages which offer a similar functionality:
%
\begin{itemize}
\item
The packages
\href{http://ctan.org/pkg/docmute}{\textsf{docmute}},
\href{http://ctan.org/pkg/includex}{\textsf{includex}} and
\href{http://ctan.org/pkg/standalone}{\textsf{standalone}}
provide commands to include only the document body of
a child file thus allowing both files to be compiled individually.
\item
The packages \href{http://ctan.org/pkg/subdocs}{\textsf{subdocs}}
and \href{http://ctan.org/pkg/subfiles}{\textsf{subfiles}}
provide structures in which the main and child documents can be
encapsulated and allowing them to be compiled individually.
The inclusion mechanism is different from the conventional |\include|.
\item
The package \href{http://ctan.org/pkg/combine}{\textsf{combine}}
is an elaborate solution to combine several documents into one.
\end{itemize}
%
See also the CTAN topic \href{http://ctan.org/topic/subdocs}{\textsf{subdocs}}
for further related packages.
The present package differs from the above solutions in that
a document structure constructed with the conventional |\include| mechanism
just needs two extra commands at the top of every file
such that all constituent files can be compiled individually.

%%%%%%%%%%%%%%%%%%%%%%%%%%%%%%%%%%%%%%%%%%%%%%%%%%%%%%%%%%%%%%%%%%%%%%%%%%%%%%%%
%\subsection{Feature Suggestions}
%
%The following is a list of features which may be useful for future
%versions of this package:
%%
%\begin{itemize}
%\item
%\ldots
%\end{itemize}

%%%%%%%%%%%%%%%%%%%%%%%%%%%%%%%%%%%%%%%%%%%%%%%%%%%%%%%%%%%%%%%%%%%%%%%%%%%%%%%%
\subsection{Revision History}

%%%%%%%%%%%%%%%%%%%%%%%%%%%%%%%%%%%%%%%%
\paragraph{v2.0:} 2018/12/30

\begin{itemize}
\item
immediate forward processing
\item
added |\childdocby| mechanism
\item
manual restructured
\end{itemize}

%%%%%%%%%%%%%%%%%%%%%%%%%%%%%%%%%%%%%%%%
\paragraph{v1.6:} 2018/01/17

\begin{itemize}
\item
application for development of include files
\item
corrections to manual
\end{itemize}

%%%%%%%%%%%%%%%%%%%%%%%%%%%%%%%%%%%%%%%%
\paragraph{v1.5:} 2017/05/21

\begin{itemize}
\item
more complete structuring introduced
\item
|\childdocof| introduced
\item
|\childdoc| renamed to |\childdocmain|
\item
|\childredirect| renamed to |\childdocforward| and |\childdocforwardprefix|
and functionality expanded
\end{itemize}

%%%%%%%%%%%%%%%%%%%%%%%%%%%%%%%%%%%%%%%%
\paragraph{v1.0:} 2017/04/27

\begin{itemize}
\item
manual and install package
\item
first version published on CTAN
\end{itemize}

%%%%%%%%%%%%%%%%%%%%%%%%%%%%%%%%%%%%%%%%
\paragraph{v0.6:} 2017/04/26

\begin{itemize}
\item
redirection mechanism added
\end{itemize}

%%%%%%%%%%%%%%%%%%%%%%%%%%%%%%%%%%%%%%%%
\paragraph{v0.5:} 2017/04/26

\begin{itemize}
\item
functionality in definition file
\end{itemize}


%%%%%%%%%%%%%%%%%%%%%%%%%%%%%%%%%%%%%%%%%%%%%%%%%%%%%%%%%%%%%%%%%%%%%%%%%%%%%%%%
%%%%%%%%%%%%%%%%%%%%%%%%%%%%%%%%%%%%%%%%%%%%%%%%%%%%%%%%%%%%%%%%%%%%%%%%%%%%%%%%
%%%%%%%%%%%%%%%%%%%%%%%%%%%%%%%%%%%%%%%%%%%%%%%%%%%%%%%%%%%%%%%%%%%%%%%%%%%%%%%%
\appendix

\settowidth\MacroIndent{\rmfamily\scriptsize 000\ }

 \DocInput{childdoc.dtx}

\end{document}
%</driver>
% \fi
%
% %%%%%%%%%%%%%%%%%%%%%%%%%%%%%%%%%%%%%%%%%%%%%%%%%%%%%%%%%%%%%%%%%%%%%%%%%%%%%%
% %%%%%%%%%%%%%%%%%%%%%%%%%%%%%%%%%%%%%%%%%%%%%%%%%%%%%%%%%%%%%%%%%%%%%%%%%%%%%%
% \section{Sample}
%\iffalse
%<*samplemain>
%\fi
%
% The following presents a sample document
% with two chapters, two parts, a title page,
% a compile flag as well as three forwarding files to set the flag.
% It consists of eight |.tex| files:
% \begin{center}
% \begin{tabular}{ll}
% |cdocsamp.tex|&main file\\
% |cdocsch1.tex|&include file for chapter 1\\
% |cdocsch2.tex|&include file for chapter 2\\
% |cdocspt3.tex|&include file for part 3\\
% |cdocspt4.tex|&include file for part 4\\
% |cdocsdrf.tex|&forwarding file for main file in draft mode\\
% |cdocsfi1.tex|&forwarding file for final version of chapter 1\\
% |cdocsfi2.tex|&forwarding file for final version of chapter 2\\
% \end{tabular}
% \end{center}
% Each of the eight files can be compiled directly by the \LaTeX{} compiler.
%
% %%%%%%%%%%%%%%%%%%%%%%%%%%%%%%%%%%%%%%
% \paragraph{Main File.}
%
% The main file is called |cdocsamp.tex|.
%
% Load the \textsf{childdoc} definitions and
% declare the filename for the main document:
%    \begin{macrocode}
\input{childdoc.def}
\childdocmain{}
%    \end{macrocode}

% Optional override for |\version| flag:
%    \begin{macrocode}
%%\ifchilddoc\else\providecommand{\version}{draft}\fi
%    \end{macrocode}

% Define the default values for the |\version| flag
% (|final| for the main file and |draft| for childs):
%    \begin{macrocode}
\ifchilddoc
\providecommand{\version}{draft}
\else
\providecommand{\version}{final}
\fi
%    \end{macrocode}

% Load the standard document class:
%    \begin{macrocode}
\documentclass[12pt]{article}
%    \end{macrocode}

% Start the document body:
%    \begin{macrocode}
\begin{document}
%    \end{macrocode}

% Declare a title page.
% Print title, part of document being processed and version flag:
%    \begin{macrocode}
\addtocounter{page}{-1}
\begin{center}
{\LARGE\bfseries{}childdoc example\par}
\vspace{1cm}
\ifchilddoc
\ifchilddocmanual part\else chapter\fi:
`\childdocname' of `\childdocjob'\par
\else
main document: `\childdocjob'\par
\fi
version: \version\par
\end{center}
\newpage
%    \end{macrocode}

% Manually include selected file,
% otherwise process as usual:
%    \begin{macrocode}
\ifchilddocmanual
\section*{part `\childdocname'}
\input{\childdocname}
\else
%    \end{macrocode}

% Include the two chapters:
%    \begin{macrocode}
\include{cdocsch1}
\include{cdocsch2}
%    \end{macrocode}

% Include the two parts unless only chapters should be displayed:
%    \begin{macrocode}
\ifchilddoc\else
\section{part three}
\input{cdocspt3}
\section{part four}
\input{cdocspt4}
\fi
%    \end{macrocode}

% Process as usual until here:
%    \begin{macrocode}
\fi
%    \end{macrocode}

% End of document body:
%    \begin{macrocode}
\end{document}
%    \end{macrocode}
%\iffalse
%</samplemain>
%\fi
%
% %%%%%%%%%%%%%%%%%%%%%%%%%%%%%%%%%%%%%%
% \paragraph{Chapter Include Files.}
%
% The include files are called |cdocsch1.tex| and |cdocsch2.tex|.
%
%\iffalse
%<*samplechap1|samplechap2>
%\fi

% Optional override for |\version| flag:
%    \begin{macrocode}
%%\providecommand{\version}{final}
%    \end{macrocode}

% Include the main document:
%    \begin{macrocode}
\input{childdoc.def}
\childdocof{cdocsamp}
%    \end{macrocode}

%\iffalse
%</samplechap1|samplechap2>
%\fi
%
%\iffalse
%<*samplechap1>
%\fi
% Some text for chapter 1:
%    \begin{macrocode}
\section{one}
some text in chapter one
%    \end{macrocode}

%\iffalse
%</samplechap1>
%\fi
% Some text for chapter 2:
%\iffalse
%<*samplechap2>
%\fi
%    \begin{macrocode}
\section{two}
more text in chapter two
%    \end{macrocode}

%\iffalse
%</samplechap2>
%\fi
%
% %%%%%%%%%%%%%%%%%%%%%%%%%%%%%%%%%%%%%%
% \paragraph{Part Include Files.}
%
% The include files are called |cdocspt3.tex| and |cdocspt4.tex|.
%
%\iffalse
%<*samplepart3|samplepart4>
%\fi

% Optional override for |\version| flag:
%    \begin{macrocode}
%%\providecommand{\version}{final}
%    \end{macrocode}

% Include the main document:
%    \begin{macrocode}
\input{childdoc.def}
\childdocby{cdocsamp}
%    \end{macrocode}

%\iffalse
%</samplepart3|samplepart4>
%\fi
%
%\iffalse
%<*samplepart3>
%\fi
% Some text for part 3:
%    \begin{macrocode}
some text in part three
%    \end{macrocode}

%\iffalse
%</samplepart3>
%\fi
% Some text for part 4:
%\iffalse
%<*samplepart4>
%\fi
%    \begin{macrocode}
more text in part four
%    \end{macrocode}

%\iffalse
%</samplepart4>
%\fi
%
% %%%%%%%%%%%%%%%%%%%%%%%%%%%%%%%%%%%%%%
% \paragraph{Forwarding for a Complete Draft.}
%
% The following forwarding file |cdocsdrf.tex|
% compiles the main document in draft mode:
%\iffalse
%<*sampledraft>
%\fi
%    \begin{macrocode}
\def\version{draft}
\input{childdoc.def}
\childdocforward{cdocsamp}
%    \end{macrocode}

%\iffalse
%</sampledraft>
%\fi
%
% %%%%%%%%%%%%%%%%%%%%%%%%%%%%%%%%%%%%%%
% \paragraph{Forwarding for Final Version of the Chapters.}
%
% The following forwarding files |cdocsfn1.tex| and |cdocsfn2.tex|
% (with identical content)
% compile the final versions of the child documents
% |cdocsch1.tex| and |cdocsch2.tex|, respectively:
%\iffalse
%<*samplefinal>
%\fi
%    \begin{macrocode}
\def\version{final}
\input{childdoc.def}
\childdocforwardprefix[cdocsamp]{cdocsfn}{cdocsch}
%    \end{macrocode}

%\iffalse
%</samplefinal>
%\fi
%
% %%%%%%%%%%%%%%%%%%%%%%%%%%%%%%%%%%%%%%
% \paragraph{Command Line Processing.}
%
% The following three command lines generate the output files
% |cdocscld|, |cdocscl1| and |cdocscl2|
% which should be identical to
% |cdocsdrf|, |cdocsch1| and |cdocsfn2|, respectively:
% \begin{center}
% \begin{tabular}{l}
% |latex -jobname cdocscld \|\\
% |  "\def\version{draft}\input{childdoc.def}\childdocforward{cdocsamp}"|\\
% |latex -jobname cdocscl1 \|\\
% |  "\input{childdoc.def}\childdocforward[cdocsamp]{cdocsch1}"|\\
% |latex -jobname cdocscl2 \|\\
% |  "\def\version{final}\input{childdoc.def}\childdocforward{cdocsch2}"|
% \end{tabular}
% \end{center}
% Note that the trailing backslash on each first line
% merely continues the input to the second line
% (for convenient cut ant paste).
% Furthermore, the command |latex| can be replaced by any
% of its alternative versions such as |pdflatex|.
%
% %%%%%%%%%%%%%%%%%%%%%%%%%%%%%%%%%%%%%%%%%%%%%%%%%%%%%%%%%%%%%%%%%%%%%%%%%%%%%%
% %%%%%%%%%%%%%%%%%%%%%%%%%%%%%%%%%%%%%%%%%%%%%%%%%%%%%%%%%%%%%%%%%%%%%%%%%%%%%%
% \section{Implementation}
%\iffalse
%<*package>
%\fi
%
% This section describes the definitions file |childdoc.def|.

% The definitions cannot be loaded using |\usepackage| or |\RequirePackage|
% which has a mechanism to prevent loading a style file more than once.
% When loading the definitions by means of |\input|
% multiple instances have to be prevented manually:
%\iffalse
%This code needs to be before the `\ProvidesFile' directive
%which is defined at the beginning of this file.
%Therefore it is also placed there and commented out here.
%</package>
%<*discard>
%\fi
%    \begin{macrocode}
\ifdefined\childdocmain\endinput\fi
%    \end{macrocode}
%\iffalse
%</discard>
%<*package>
%\fi
%
% \macro{\ifchilddoc}
% \macro{\ifchilddocmanual}
% The conditional |\ifchilddoc| tells whether a
% child (true) or main (false) document is being compiled.
% The conditional |\ifchilddocmanual| tells whether
% the |\includeonly| mechanism is used (false) or
% the selection of child files must be performed manually (true).
% The definitions initialise to false:
%    \begin{macrocode}
\newif\ifchilddoc
\newif\ifchilddocmanual
%    \end{macrocode}

% \macro{\childdocname}
% \macro{\childdocjob}
% The macro |\childdocname| stores the name of the main document
% to be compiled. The macro |\childdocjob| stores the name of
% the document on which the \LaTeX{} compiler was originally invoked.
% The content of |\jobname| cannot be compared
% to filenames specified in the source due to different catcodes.
% The following code rescans |\jobname|, stores the result
% in |\childdocname| and saves a copy in |\childdocjob|:
%    \begin{macrocode}
\edef\childdocname{\scantokens\expandafter{\jobname\noexpand}}
\let\childdocjob\childdocname
%    \end{macrocode}

% \macro{\childdocdisable}
% The macro |\childdocdisable| prevents the main file
% from being processed more than once.
% At this stage, the main document command |\childdocmain|
% is assumed to be called once again where it should do nothing.
% Any subsequent call to it should prevent
% a secondary processing of the main document
% It overwrites the forwarding commands
% |\childdocof| and |\childdocforward|
% with empty macros to prevent further inclusions of the main document:
%    \begin{macrocode}
\newcommand{\childdocdisable}
{
  \renewcommand{\childdocmain}[1]{\renewcommand{\childdocmain}[1]{\endinput}}
  \renewcommand{\childdocof}[1]{}
  \renewcommand{\childdocby}[2][]{}
  \renewcommand{\childdocforward}[2][]{}
  \renewcommand{\childdocdisable}{}
}
%    \end{macrocode}

% \macro{\childdocmain}
% The macro |\childdocmain| is to be called at the top of the main file
% with nothing or the main filename (without extension) as argument.
% First, it breaks loops.
% If the argument is not empty and does not match |\childdocname|
% (which is set by the first inclusion of |childdoc.def|),
% |\ifchilddoc| is set to true, |\includeonly| is applied to the child file
% and |\jobname| is set to the main file
% (for proper handling of |.aux| files):
%    \begin{macrocode}
\newcommand{\childdocmain}[1]
{
  \childdocdisable\childdocmain{}
  \if?#1?\else
    \begingroup
      \def\childdoctmp{#1}
      \ifx\childdoctmp\childdocname
        \def\childdoctmp{}
      \else
        \def\childdoctmp
        {
          \childdoctrue
          \includeonly{\childdocname}
          \def\childdocjob{#1}
          \def\jobname{#1}
        }
      \fi
      \expandafter
    \endgroup
    \childdoctmp
  \fi
}
%    \end{macrocode}

% \macro{\childdocof}
% The command |\childdocof| redirects
% compilation to the main file |#1|.
%    \begin{macrocode}
\newcommand{\childdocof}[1]
{
  \childdocdisable
  \childdoctrue
  \includeonly{\childdocname}
  \def\jobname{#1}
  \def\childdocjob{#1}
  \input{#1}
}
%    \end{macrocode}

% \macro{\childdocby}
% The command |\childdocby| ....
%    \begin{macrocode}
\newcommand{\childdocby}[2][]
{
  \childdocdisable
  \childdoctrue
  \childdocmanualtrue
  \if?#1?\else
    \def\jobname{#2}
  \fi
  \def\childdocjob{#2}
  \input{#2}
  \endinput
}
%    \end{macrocode}

% \macro{\childdocforward}
% The command |\childdocforward| redirects
% compilation to the main file or
% (if the optional argument is given) a child file.
% Parameters are set as if the main file
% or a child file starting with |\childdocof| was compiled.
% Then compilation is handed over to the main file:
%    \begin{macrocode}
\newcommand{\childdocforward}[2][]
{
  \begingroup
    \if?#1?
      \def\childdoctmp
      {
        \def\childdocname{#2}
        \def\childdocjob{#2}
        \def\jobname{#2}
        \input{#2}
        \endinput
      }
    \else
      \def\childdoctmp
      {
        \childdocdisable
        \def\childdocname{#2}
        \childdoctrue
        \includeonly{#2}
        \def\childdocjob{#1}
        \def\jobname{#1}
        \input{#1}
        \endinput
      }
    \fi
    \expandafter
  \endgroup
  \childdoctmp
}
%    \end{macrocode}

% \macro{\childdocforwardprefix}
% The command |\childdocforwardprefix| redirects
% compilation to the main or a child file by means of a pattern.
% The prefix |#1| in the current filename is replaced by |#2|
% and the suffix of the current filename is kept
% (it is assumed that the filename does not contain the substring `|~~~|'
% which is used as a delimiter).
% Compilation is handed over to the new file by |\childdocforward|:
%    \begin{macrocode}
\newcommand{\childdocforwardprefix}[3][]
{
  \begingroup
    \def\childdocextract #2##1~~~{\def\childdoctmp{\childdocforward[#1]{#3##1}}}
    \expandafter\childdocextract\childdocname~~~
    \expandafter
  \endgroup
  \childdoctmp
}
%    \end{macrocode}

% \macro{\childdoc}
% The deprecated macro |\childdoc| is a legacy version of |\childdocmain|:
%    \begin{macrocode}
\newcommand{\childdoc}{\childdocmain}
%    \end{macrocode}

% \macro{\childdocredirect}
% The deprecated macro |\childdocredirect| is a legacy version
% of |\childdocforward| and |\childdocforwardprefix|:
%    \begin{macrocode}
\newcommand{\childdocredirect}[2][]
{
  \begingroup
    \if?#1?
      \def\childdoctmp{\childdocforward{#2}}
    \else
      \def\childdoctmp{\childdocforwardprefix{#1}{#2}}
    \fi
    \expandafter
  \endgroup
  \childdoctmp
}
%    \end{macrocode}

%\iffalse
%</package>
%\fi
%
\endinput
|\\
|\childdocforwardprefix{final}{child}|
\end{tabular}
\end{center}
%

Note that when several versions of a main file and/or of each child file
are to be generated, it may be convenient to set up a |Makefile| or
shell script to automatise the process.

%%%%%%%%%%%%%%%%%%%%%%%%%%%%%%%%%%%%%%%%%%%%%%%%%%%%%%%%%%%%%%%%%%%%%%%%%%%%%%%%
\subsection{Command Line Processing}
\label{sec:commandline}

The effect of redirection files can also be achieved by invoking
the \LaTeX{} compiler with a more elaborate command line.
Most conveniently this should be done as part
of a shell script or a |Makefile|.

When using \textsf{childdoc} in the main file, the following
command lines effectively perform a redirection
(note that depending on the shell being used,
backslashes may have to be doubled: `|\|' $\to$ `|\\|'):
%
\begin{center}
|... -jobname "|\textit{target}|" |\\|"|[\textit{flags}]%
|% \iffalse
%
% childdoc.dtx Copyright (C) 2017-2018 Niklas Beisert
%
% This work may be distributed and/or modified under the
% conditions of the LaTeX Project Public License, either version 1.3
% of this license or (at your option) any later version.
% The latest version of this license is in
%   http://www.latex-project.org/lppl.txt
% and version 1.3 or later is part of all distributions of LaTeX
% version 2005/12/01 or later.
%
% This work has the LPPL maintenance status `maintained'.
%
% The Current Maintainer of this work is Niklas Beisert.
%
% This work consists of the files childdoc.dtx and childdoc.ins
% and the derived files childdoc.def and cdocsamp.tex with
% cdocsch1.tex, cdocsch2.tex, cdocsdrf.tex, cdocsfn1.tex, cdocsfn2.tex.
%
%<package>\ifdefined\childdocmain\endinput\fi
%<package>\ProvidesFile{childdoc.def}[2018/12/30 v2.0 child document driver]
%<samplemain>\ProvidesFile{cdocsamp.tex}[2018/12/30 v2.0 sample for childdoc]
%<*driver>
%\ProvidesFile{childdoc.drv}[2018/12/30 v2.0 childdoc reference manual file]
\PassOptionsToClass{10pt,a4paper}{article}
\documentclass{ltxdoc}

\usepackage[margin=35mm]{geometry}
\usepackage{hyperref}
\usepackage{hyperxmp}
\usepackage[usenames]{color}

\hypersetup{colorlinks=true}
\hypersetup{pdfstartview=FitH}
\hypersetup{pdfpagemode=UseNone}
\hypersetup{pdfsource={}}
\hypersetup{pdflang={en-UK}}
\hypersetup{pdfcopyright={Copyright 2017-2018 Niklas Beisert.
  This work may be distributed and/or modified under the
  conditions of the LaTeX Project Public License, either version 1.3
  of this license or (at your option) any later version.}}
\hypersetup{pdflicenseurl={http://www.latex-project.org/lppl.txt}}
\hypersetup{pdfcontactaddress={ETH Zurich, ITP, HIT K,
  Wolfgang-Pauli-Strasse 27}}
\hypersetup{pdfcontactpostcode={8093}}
\hypersetup{pdfcontactcity={Zurich}}
\hypersetup{pdfcontactcountry={Switzerland}}
\hypersetup{pdfcontactemail={nbeisert@itp.phys.ethz.ch}}
\hypersetup{pdfcontacturl={http://people.phys.ethz.ch/\xmptilde nbeisert/}}

\newcommand{\secref}[1]{\hyperref[#1]{section \ref*{#1}}}

\parskip1ex
\parindent0pt
\let\olditemize\itemize
\def\itemize{\olditemize\parskip0pt}

\begin{document}

\title{The \textsf{childdoc} Package}
\hypersetup{pdftitle={The childdoc Package}}
\author{Niklas Beisert\\[2ex]
  Institut f\"ur Theoretische Physik\\
  Eidgen\"ossische Technische Hochschule Z\"urich\\
  Wolfgang-Pauli-Strasse 27, 8093 Z\"urich, Switzerland\\[1ex]
  \href{mailto:nbeisert@itp.phys.ethz.ch}
  {\texttt{nbeisert@itp.phys.ethz.ch}}}
\hypersetup{pdfauthor={Niklas Beisert}}
\hypersetup{pdfsubject={Manual for the LaTeX2e Package childdoc}}
\date{30 December 2018, \textsf{v2.0}}
\maketitle

\begin{abstract}\noindent
\textsf{childdoc} is a \LaTeXe{} package
that enables the direct compilation
of document sections included by |\include|
to individual files.
\end{abstract}

\begingroup
\parskip0ex
\tableofcontents
\endgroup

%%%%%%%%%%%%%%%%%%%%%%%%%%%%%%%%%%%%%%%%%%%%%%%%%%%%%%%%%%%%%%%%%%%%%%%%%%%%%%%%
%%%%%%%%%%%%%%%%%%%%%%%%%%%%%%%%%%%%%%%%%%%%%%%%%%%%%%%%%%%%%%%%%%%%%%%%%%%%%%%%
\section{Introduction}

\LaTeX{} provides a mechanism to structure a large document (such as a book)
into a main file and several child files (containing the chapters)
using the |\include| command.
This mechanism is beneficial for documents
which span hundreds of pages in order to
make the source file(s) more manageable.
Moreover, compilation can be restricted to
selected child files by means of the |\includeonly| command.
The latter feature can be used to reduce the compilation time while editing
(this was significantly more useful in the earlier days of \LaTeX{})
or to generate a smaller document which is easier to navigate.
Another application of |\includeonly| is to generate
documents consisting of selected parts of the complete document.

However, there are a few drawbacks of the plain |\include| mechanism:
\begin{itemize}
\item
The child files cannot be compiled on their own,
they can only be compiled via the main file.
A naive editing environment
(such as a text editor with an option
to have the current file processed by \LaTeX)
may require one to switch to the main file before compiling;
attempting to compile the child file produces errors.
\item
The main file must be modified (each time)
to adjust the |\includeonly| command
to the present needs. This easily leaves the main file in a messy state.
\item
The generated document will always carry the filename
of the main document. This is inconvenient if
several child files are to be compiled and
to be kept for distribution.
\end{itemize}

The present package provides a simple interface
to make child files individually compilable by \LaTeX{}.
Compiling a child file then has the same effect as compiling
the main file with an |\includeonly| command
to select the appropriate child.
Moreover the generated document will carry the name of the child
rather than the main file.
This resolves all three above issues.

This feature is meant to make the editing of books,
thesis documents and lecture notes somewhat more convenient.
However, the package can also be used efficiently for
composing a series of documents (such as exercise sheets)
which are typically distributed individually.
It then assists the author in generating the individual documents
(potentially in different versions)
as well as a document containing the collected series.
Another application is in developing style files
or other kinds of included material
where compilation of the style file could redirect
to a sample or test file.

%%%%%%%%%%%%%%%%%%%%%%%%%%%%%%%%%%%%%%%%%%%%%%%%%%%%%%%%%%%%%%%%%%%%%%%%%%%%%%%%
%%%%%%%%%%%%%%%%%%%%%%%%%%%%%%%%%%%%%%%%%%%%%%%%%%%%%%%%%%%%%%%%%%%%%%%%%%%%%%%%
\section{Usage}

First of all, the package \textsf{childdoc} is \emph{not} a standard
\LaTeXe{} |.sty| style file! Therefore it needs to be invoked in
a non-standard way.

%%%%%%%%%%%%%%%%%%%%%%%%%%%%%%%%%%%%%%%%%%%%%%%%%%%%%%%%%%%%%%%%%%%%%%%%%%%%%%%%
\subsection{Included Files}
\label{sec:include}

%%%%%%%%%%%%%%%%%%%%%%%%%%%%%%%%%%%%%%%%
\DescribeMacro{\childdocmain}
To use the package, add the commands
\begin{center}
\begin{tabular}{l}
|\input{childdoc.def}|\\
|\childdocmain{}|\\
\end{tabular}
\end{center}
at the very top of the main \LaTeX{} file,
in particular \emph{before} the |\documentclass| statement!
The argument of |\childdocmain| should be left empty
(but it must be present).

%%%%%%%%%%%%%%%%%%%%%%%%%%%%%%%%%%%%%%%%
\DescribeMacro{\childdocof}
Furthermore, add the commands
\begin{center}
\begin{tabular}{l}
|\input{childdoc.def}|\\
|\childdocof{|\textit{main}|}|\\
\end{tabular}
\end{center}
at the top of every child file \textit{child}
which is included by |\include{|\textit{child}|}|
from within the main file
(or at least for those files to be compiled individually).
The argument \textit{main} must be the filename of the main file.

There are a couple of
considerations in setting up the main and child documents:

%%%%%%%%%%%%%%%%%%%%%%%%%%%%%%%%%%%%%%%%
\paragraph{Restrictions.}

Please note the following restrictions:
\begin{itemize}
\item
|\childdocmain| must be called with one argument \textit{main}
to ensure compatibility with earlier version of the package.
It must either be empty (|\childdocmain{}|)
or precisely match the filename of the main file in which it is specified.
See \secref{sec:detection} for further information.
\item
The filename \textit{main} must be specified without the |.tex| extension.
\item
The filename \textit{main} is case sensitive
(even in case-insensitive file systems)
due to internal string comparison.
\item
The argument \textit{main} should be fully expanded, it cannot be a macro.
\item
Subdirectories and special characters should be avoided in filenames.
\item
The command |\childdocmain{|\textit{main}|}| must be followed by a whitespace.
It should not be followed immediately by another command
or by a comment mark `|%|'.
This is because the \TeX{} parser reads the token immediately following
the argument of |\childdocmain| and puts it
at the beginning of every child section;
however, a white\-space is ignored.
\end{itemize}

%%%%%%%%%%%%%%%%%%%%%%%%%%%%%%%%%%%%%%%%
\paragraph{Content of Main File.}

It is advisable to place all content in the child files included by |\include|.
Any output contained in the main file will appear in all child documents
unless suppressed manually;
it cannot be suppressed automatically by the |\includeonly| directive
and thus should normally be avoided.
A method to include some content in the main file
by means of conditional processing is described in \secref{sec:conditional}.

%%%%%%%%%%%%%%%%%%%%%%%%%%%%%%%%%%%%%%%%
\paragraph{Page Numbering.}

When only a part of the document is compiled,
the appropriate numbering of pages
(as well as other status parameters)
is determined from the |.aux| files.
The latter contain information from previous passes.
However this information needs to propagate through
all intermediate child documents.
Therefore the page numbering in child documents may well
be inconsistent until the complete document is compiled at least once.

A useful (if unconventional) way to always ensure a consistent
page numbering is to restart the numbering in each child document
and denote the pages by `\textit{child}|.|\textit{page}'
where \textit{child} represents the chapter/section number of the child file.
This can be achieved by the command
|\numberwithin{page}{|\textit{child}|}|
of the \textsf{amsmath} package
where \textit{child} can be |chapter| or |section|
depending on the chosen structuring.
Alternatively, one can modify the macro |\thepage| appropriately
and reset the counter |page| at the start of each child file.

%%%%%%%%%%%%%%%%%%%%%%%%%%%%%%%%%%%%%%%%%%%%%%%%%%%%%%%%%%%%%%%%%%%%%%%%%%%%%%%%
\subsection{Conditional Processing}
\label{sec:conditional}

The package provides a mechanism to compile different versions
of a document. To customise the versions further some conditional processing
can come in handy to distinguish which version is being compiled.
The package provides two macros to describe the compilation context:

%%%%%%%%%%%%%%%%%%%%%%%%%%%%%%%%%%%%%%%%
\DescribeMacro{\ifchilddoc}
The conditional |\ifchilddoc| distinguishes between the compilation of
child documents and the main document:
%
\begin{center}
|\ifchilddoc |\textit{child-code}| |[|\||else |\textit{main-code}]| \||fi|
\end{center}

%%%%%%%%%%%%%%%%%%%%%%%%%%%%%%%%%%%%%%%%
\DescribeMacro{\childdocname}
\DescribeMacro{\childdocjob}
The macro |\childdocname| contains the filename (without extension)
of the main or child file being processed.
Note that |\childdocjob| will always contain the name of the main file.

%%%%%%%%%%%%%%%%%%%%%%%%%%%%%%%%%%%%%%%%
\paragraph{Title Page.}

Conditional processing can be used to include a title or banner page
in the main document when proper precautions are taken.
Importantly, the code in the main file should ensure that the page counter
(as well as other status parameters which are stored in the |.aux| files)
takes the same value after the conditional processing.
Otherwise the page numbers may take divergent values
depending on which part is compiled.

For example, a title page could be declared by:
%
\begin{center}
\begin{tabular}{l}
|\ifchilddoc\||else|\\
|\addtocounter{page}{-1}|\\
\textit{code for title page}\\
|\newpage|\\
|\||fi|
\end{tabular}
\end{center}
%
A banner page for the child documents can be generated by:
%
\begin{center}
\begin{tabular}{l}
|\ifchilddoc|\\
|\addtocounter{page}{-1}|\\
\textit{code for banner page}\\
|\newpage|\\
|\||fi|
\end{tabular}
\end{center}
%
Here one could write a message such as:
\begin{center}
|This is the part \childdocname{} of \childdocjob{}.|
\end{center}

%%%%%%%%%%%%%%%%%%%%%%%%%%%%%%%%%%%%%%%%%%%%%%%%%%%%%%%%%%%%%%%%%%%%%%%%%%%%%%%%
\subsection{Flags}
\label{sec:flags}

The package makes it easy to generate different versions
of the main or child documents.
To this end compilation flags can be defined
and assigned different default values.
They will be particularly useful in conjunction
with the forwarding mechanism described in \secref{sec:forward}.

For example, it may be useful to have a flag |\version|
which can be set to |draft| or |final|.
The document source will contain some conditional code
depending on the value of |\version|.
Suppose further, the flag should default to |final| for the main file
and to |draft| for child files
which is a natural assignment for editing the document.
This is achieved by placing the following code
in the preamble of the main document
(below the |\childdocmain| directive):
%
\begin{center}
\begin{tabular}{l}
|\ifchilddoc|\\
|\providecommand{\version}{draft}|\\
|\||else|\\
|\providecommand{\version}{final}|\\
|\||fi|
\end{tabular}
\end{center}
%
The definition by |\providecommand| makes sure
that previous definitions are not overwritten.
Further statements |\providecommand{\version}{...}|
can thus be added before the above code to override it.

For the main file, one might add a line
(between |\childdocmain| and the above block)
%
\begin{center}
|%\ifchilddoc\||else\providecommand{\version}{draft}\||fi|
\end{center}
%
which can be uncommented to produce a draft version.
Likewise one can add a line to the very top of a child file
(above the |\childdocof{|\textit{main}|}| directive)
%
\begin{center}
|%\providecommand{\version}{final}|
\end{center}
%
which can be uncommented to produce the final version of this child document.

%%%%%%%%%%%%%%%%%%%%%%%%%%%%%%%%%%%%%%%%%%%%%%%%%%%%%%%%%%%%%%%%%%%%%%%%%%%%%%%%
\subsection{Forwarding}
\label{sec:forward}

Different versions of the main or child documents
using compilation flags as described in \secref{sec:flags}
can be (permanently) stored in different files
for convenient compilation, viewing and distribution.
To this end, the package defines a command
to pass on compilation to a different file:

%%%%%%%%%%%%%%%%%%%%%%%%%%%%%%%%%%%%%%%%
\DescribeMacro{\childdocforward}
The command |\childdocforward| redirects processing to
another source file:
%
\begin{center}
\begin{tabular}{l}
|\input{childdoc.def}|\\
|\childdocforward[|\textit{main}|]{|\textit{dest}|}|\\
\end{tabular}
\end{center}
%
The argument \textit{dest} is the destination file
(without extension).
It should be the main file or one of the child files.
Note that further \textsf{childdoc} directives
such as |\childdocof| and |\childdocforward|
in the indicated file will be processed in this form.
The optional argument \textit{main}
passes on directly to the main file \textit{main}
while pretending to compile the child \textit{dest}.
This form behaves as if \textit{dest}
issues |\childdocof{|\textit{main}|}| right away,
and no further \textsf{childdoc} directives will be processed.

%%%%%%%%%%%%%%%%%%%%%%%%%%%%%%%%%%%%%%%%
\DescribeMacro{\...prefix}
In the alternative form |\childdocforwardprefix|,
%
\begin{center}
\begin{tabular}{l}
|\input{childdoc.def}|\\
|\childdocforwardprefix[|\textit{main}|]{|\textit{prefix}|}{|\textit{dest}|}|
\end{tabular}
\end{center}
%
the destination file is determined by a pattern
depending on the current file:
To make this work, the current file must be called
`{\textit{prefix}\hspace{0.2em}\textit{suffix}}'
with \textit{prefix} matching precisely the argument.
Processing is then passed on to the file
`{\textit{dest}\hspace{0.2em}\textit{suffix}}'.
Surely, the same effect is achieved by
directly specifying the
argument `{\textit{dest}\hspace{0.2em}\textit{suffix}}'
in the first form.
However, that requires to set up a different file
for each child. With the alternative form of the command
all these files can have exactly the same content
which simplifies setting them up and maintaining them.

For example, the following file |draft.tex|
with a compilation flag |\version| as described in \secref{sec:flags}
compiles the main document as a draft:
%
\begin{center}
\begin{tabular}{l}
|\def\version{draft}|\\
|\input{childdoc.def}|\\
|\childdocforward{|\textit{main}|}|
\end{tabular}
\end{center}
%
Likewise, the following files |final|\textit{nn}|.tex|
compile the final version of the child document
|child|\textit{nn}|.tex|:
%
\begin{center}
\begin{tabular}{l}
|\def\version{final}|\\
|\input{childdoc.def}|\\
|\childdocforwardprefix{final}{child}|
\end{tabular}
\end{center}
%

Note that when several versions of a main file and/or of each child file
are to be generated, it may be convenient to set up a |Makefile| or
shell script to automatise the process.

%%%%%%%%%%%%%%%%%%%%%%%%%%%%%%%%%%%%%%%%%%%%%%%%%%%%%%%%%%%%%%%%%%%%%%%%%%%%%%%%
\subsection{Command Line Processing}
\label{sec:commandline}

The effect of redirection files can also be achieved by invoking
the \LaTeX{} compiler with a more elaborate command line.
Most conveniently this should be done as part
of a shell script or a |Makefile|.

When using \textsf{childdoc} in the main file, the following
command lines effectively perform a redirection
(note that depending on the shell being used,
backslashes may have to be doubled: `|\|' $\to$ `|\\|'):
%
\begin{center}
|... -jobname "|\textit{target}|" |\\|"|[\textit{flags}]%
|\input{childdoc.def}\childdocforward[|\textit{main}|]{|\textit{dest}|}"|
\end{center}
%
Here \textit{target} is the name of the output file,
\textit{main} is the name of the main file
and \textit{dest} is the name of the main or child file to be processed
(all filenames without extensions).
The optional argument \textit{main} can be omitted
if \textit{main} matches \textit{dest}.
Optionally, compilation \textit{flags} can be defined via |\def| commands.
This command line makes the \TeX{} engine believe
it is compiling the file \textit{target}
whose content is specified as the latter parameter.
The provided code then forwards the processing to
\textit{main} or \textit{dest} as described in \secref{sec:forward}.

%%%%%%%%%%%%%%%%%%%%%%%%%%%%%%%%%%%%%%%%%%%%%%%%%%%%%%%%%%%%%%%%%%%%%%%%%%%%%%%%
\subsection{Include by Input}
\label{sec:input}

Including child documents by |\include| has some restrictions by design.
Most notably, the content of a child document always occupies
its own set of pages; pages cannot be shared between child documents.
Usually, this behaviour makes perfect sense
because each child document contain an essential part of the document.
However, in some situations it may be desirable to compose
a document from a collection of parts
without having mandatory page breaks between then.
For this case, the package
provides a mechanism to include parts
by |\input| which can also be processed individually.
However, by construction this mechanism
requires manual handling of the content to be output.

%%%%%%%%%%%%%%%%%%%%%%%%%%%%%%%%%%%%%%%%
\DescribeMacro{\ifchilddocmanual}
The main file should be prepared as usual, see \secref{sec:include}.
However, the document body must make a distinction
between processing of an individual part and of the main document, e.g.:
%
\begin{center}
\begin{tabular}{l}
|\ifchilddocmanual|\\
|\input{\childdocname}|\\
|\||else|\\
\textit{document body with }|\input{|\textit{part}|}|\\
|\||fi|
\end{tabular}
\end{center}
%
The conditional |\ifchilddocmanual| is true whenever
a part to be included by |\input| is being compiled,
and the name of the part is stored in |\childdocname|.

%%%%%%%%%%%%%%%%%%%%%%%%%%%%%%%%%%%%%%%%
\DescribeMacro{\childdocby}
Each part to be included by |\input| should start with:
%
\begin{center}
\begin{tabular}{l}
|\input{childdoc.def}|\\
|\childdocby{|\textit{main}|}|\\
\end{tabular}
\end{center}
%
The directive |\childdocby| is similar to |\childdocof|
described in \secref{sec:include},
but the subsequent selection of content must be done manually.
To that end, both |\ifchilddoc| and |\ifchilddocmanual|
will be true upon processing of a part,
and the name of the part is stored in |\childdocname|.
Note that |\jobname| will be set to the filename of the current part
so that each part receives an individual |.aux| file
that does not interfere with the |.aux| file(s) of the main document.
This behaviour can be altered by the alternative form
|\childdocby[*]{|\textit{main}|}| (with a non-empty optional argument)
which uses the |.aux| file of the main document
by setting |\jobname| to \textit{main}.

%%%%%%%%%%%%%%%%%%%%%%%%%%%%%%%%%%%%%%%%%%%%%%%%%%%%%%%%%%%%%%%%%%%%%%%%%%%%%%%%
\subsection{Driver Development}
\label{sec:driver}

The \textsf{childdoc} mechanism can also be use for the development
of definition files such as \LaTeX{} styles or classes.
This case differs from the above setup with multiple parts
included by |\include| in that no |\includeonly| should be invoked.
This can be achieved by starting the include file
(before |\ProvidesPackage|) with:
%
\begin{center}
\begin{tabular}{l}
|\input{childdoc.def}|\\
|\childdocforward{|\textit{main}|}|\\
\end{tabular}
\end{center}
%
or alternatively with:
%
\begin{center}
\begin{tabular}{l}
|\input{childdoc.def}|\\
|\childdocby{|\textit{main}|}|\\
\end{tabular}
\end{center}
%
Both forms have slightly different effects as described above.
The main file is prepared as usual, see \secref{sec:include}.

%%%%%%%%%%%%%%%%%%%%%%%%%%%%%%%%%%%%%%%%%%%%%%%%%%%%%%%%%%%%%%%%%%%%%%%%%%%%%%%%
\subsection{Legacy Detection}
\label{sec:detection}

The directive |\childdocmain| in the main file can detect
whether the complete document or merely a child is to be compiled
even without using the directive |\childdocof|.
This method is deprecated because it is less robust
and there is no compelling reason to use it;
it is merely provided for backward compatibility
and it may be removed in future versions.

If the detection mechanism is to be used,
it is mandatory to correctly specify
the filename of the main file as the argument of |\childdocmain|:
%
\begin{center}
\begin{tabular}{l}
|\input{childdoc.def}|\\
|\childdocmain{|\textit{main}|}|\\
\end{tabular}
\end{center}
%
If |\jobname| does not match the argument \textit{main} of |\childdocmain|,
it is assumed that |\jobname| points to the child file to be compiled.
When using |\childdocmain| with the main file specified as argument,
it suffices to start a child file
with just |\input{|\textit{main}|}|
without loading of the package and using |\childdocof|.
If instead all processing is done
with the appropriate \textsf{childdoc} directives,
the argument of \textit{main} of |\childdocmain| can be empty.

An alternative version of the command line processing described
in \secref{sec:commandline} using the detection mechanism reads:
%
\begin{center}
|... -jobname "|\textit{target}|" "|[\textit{flags}]%
[|\def\jobname{|\textit{dest}|}|]|\input{|\textit{main}|}"|
\end{center}

%%%%%%%%%%%%%%%%%%%%%%%%%%%%%%%%%%%%%%%%%%%%%%%%%%%%%%%%%%%%%%%%%%%%%%%%%%%%%%%%
\subsection{Manual Code}
\label{sec:manual}

In case one cannot be certain whether the definitions file |childdoc.def|
is installed on the target \TeX{} distribution
and one prefers not to ship it,
it is conceivable to paste a few relevant commands into the sources.

To that end, drop all statements |\input{childdoc.def}|
and perform the replacements as outlined below.
Instead of |\childdocmain{|\textit{main}|}| add the following code
to the top of the main file:
%
\begin{center}
\begin{tabular}{l}
|\||ifdefined\childdocname\endinput\||fi\newif\ifchilddoc|\\
|\edef\childdocname{\scantokens\expandafter{\jobname\noexpand}}|\\
|\def\childdocmain{|\textit{main}|}\||ifx\childdocmain\childdocname\||else|\\
|\childdoctrue\includeonly{\childdocname}\let\jobname\childdocmain\||fi|\\
\end{tabular}
\end{center}
%
Instead of |\childdocof{|\textit{main}|}| just include the main file
at the top of each child file:
%
\begin{center}
|\input{|\textit{main}|}|
\end{center}
%
A simple redirection |\childdocforward{|\textit{dest}|}| is achieved by:
%
\begin{center}
|\def\jobname{|\textit{dest}|}\input{\jobname}|
\end{center}
%
The redirection with prefix
|\childdocforwardprefix[|\textit{prefix}|]{|\textit{dest}|}|
is accomplished by:
%
\begin{center}
\begin{tabular}{l}
|{\edef\jobname{\scantokens\expandafter{\jobname\noexpand}}|\\
|\def\redirectjob |\textit{prefix}|#1~~~{\gdef\jobname{|\textit{dest}|#1}}|\\
|\expandafter\redirectjob\jobname~~~}\input{\jobname}|
\end{tabular}
\end{center}

In an alternative approach,
child documents can be compiled by a specific command line
without additional code or specific definitions:
%
\begin{center}
|... -jobname "|\textit{target}|" "|[\textit{flags}]%
|\includeonly{|\textit{dest}|}\input{|\textit{main}|}"|
\end{center}
%

%%%%%%%%%%%%%%%%%%%%%%%%%%%%%%%%%%%%%%%%%%%%%%%%%%%%%%%%%%%%%%%%%%%%%%%%%%%%%%%%
%%%%%%%%%%%%%%%%%%%%%%%%%%%%%%%%%%%%%%%%%%%%%%%%%%%%%%%%%%%%%%%%%%%%%%%%%%%%%%%%
\section{Information}

%%%%%%%%%%%%%%%%%%%%%%%%%%%%%%%%%%%%%%%%%%%%%%%%%%%%%%%%%%%%%%%%%%%%%%%%%%%%%%%%
\subsection{Copyright}

Copyright \copyright{} 2017--2018 Niklas Beisert

This work may be distributed and/or modified under the
conditions of the \LaTeX{} Project Public License, either version 1.3
of this license or (at your option) any later version.
The latest version of this license is in
  \url{http://www.latex-project.org/lppl.txt}
and version 1.3 or later is part of all distributions of \LaTeX{}
version 2005/12/01 or later.

This work has the LPPL maintenance status `maintained'.

The Current Maintainer of this work is Niklas Beisert.

This work consists of the files |README.txt|, |childdoc.ins| and |childdoc.dtx|
as well as the derived files |childdoc.def|, |cdocsamp.tex|
with |cdocsch1.tex|, |cdocsch2.tex|, |cdocspt3.tex|, |cdocspt4.tex|,
|cdocsdrf.tex|, |cdocsfn1.tex|, |cdocsfn2.tex|
as well as |childdoc.pdf|.

%%%%%%%%%%%%%%%%%%%%%%%%%%%%%%%%%%%%%%%%%%%%%%%%%%%%%%%%%%%%%%%%%%%%%%%%%%%%%%%%
\subsection{Files and Installation}

The package consists of the files:
%
\begin{center}
\begin{tabular}{ll}
    |README.txt|   & readme file \\
    |childdoc.ins| & installation file \\
    |childdoc.dtx| & source file \\
    |childdoc.def| & definition file \\
    |cdocsamp.tex| & sample main file \\
    |cdocsch1.tex| & sample include file \\
    |cdocsch2.tex| & sample include file \\
    |cdocspt3.tex| & sample part file \\
    |cdocspt4.tex| & sample part file \\
    |cdocsdrf.tex| & sample redirection file \\
    |cdocsfn1.tex| & sample redirection file \\
    |cdocsfn2.tex| & sample redirection file \\
    |childdoc.pdf| & manual
\end{tabular}
\end{center}
%
The distribution consists of the files
|README.txt|, |childdoc.ins| and |childdoc.dtx|.
%
\begin{itemize}
\item
Run (pdf)\LaTeX{} on |childdoc.dtx|
to compile the manual |childdoc.pdf| (this file).
\item
Run \LaTeX{} on |childdoc.ins| to create the definitions file |childdoc.def|
and the sample |cdocsamp.tex| with include files
|cdocsch1.tex|, |cdocsch2.tex|, |cdocspt3.tex|, |cdocspt4.tex|,
|cdocsdrf.tex|, |cdocsfn1.tex|, |cdocsfn2.tex|.
Then copy the file |childdoc.def| to an appropriate directory of your \LaTeX{}
distribution, e.g.\ \textit{texmf-root}|/tex/latex/childdoc|.
\end{itemize}

%%%%%%%%%%%%%%%%%%%%%%%%%%%%%%%%%%%%%%%%%%%%%%%%%%%%%%%%%%%%%%%%%%%%%%%%%%%%%%%%
\subsection{Related CTAN Packages}

There are several other packages which offer a similar functionality:
%
\begin{itemize}
\item
The packages
\href{http://ctan.org/pkg/docmute}{\textsf{docmute}},
\href{http://ctan.org/pkg/includex}{\textsf{includex}} and
\href{http://ctan.org/pkg/standalone}{\textsf{standalone}}
provide commands to include only the document body of
a child file thus allowing both files to be compiled individually.
\item
The packages \href{http://ctan.org/pkg/subdocs}{\textsf{subdocs}}
and \href{http://ctan.org/pkg/subfiles}{\textsf{subfiles}}
provide structures in which the main and child documents can be
encapsulated and allowing them to be compiled individually.
The inclusion mechanism is different from the conventional |\include|.
\item
The package \href{http://ctan.org/pkg/combine}{\textsf{combine}}
is an elaborate solution to combine several documents into one.
\end{itemize}
%
See also the CTAN topic \href{http://ctan.org/topic/subdocs}{\textsf{subdocs}}
for further related packages.
The present package differs from the above solutions in that
a document structure constructed with the conventional |\include| mechanism
just needs two extra commands at the top of every file
such that all constituent files can be compiled individually.

%%%%%%%%%%%%%%%%%%%%%%%%%%%%%%%%%%%%%%%%%%%%%%%%%%%%%%%%%%%%%%%%%%%%%%%%%%%%%%%%
%\subsection{Feature Suggestions}
%
%The following is a list of features which may be useful for future
%versions of this package:
%%
%\begin{itemize}
%\item
%\ldots
%\end{itemize}

%%%%%%%%%%%%%%%%%%%%%%%%%%%%%%%%%%%%%%%%%%%%%%%%%%%%%%%%%%%%%%%%%%%%%%%%%%%%%%%%
\subsection{Revision History}

%%%%%%%%%%%%%%%%%%%%%%%%%%%%%%%%%%%%%%%%
\paragraph{v2.0:} 2018/12/30

\begin{itemize}
\item
immediate forward processing
\item
added |\childdocby| mechanism
\item
manual restructured
\end{itemize}

%%%%%%%%%%%%%%%%%%%%%%%%%%%%%%%%%%%%%%%%
\paragraph{v1.6:} 2018/01/17

\begin{itemize}
\item
application for development of include files
\item
corrections to manual
\end{itemize}

%%%%%%%%%%%%%%%%%%%%%%%%%%%%%%%%%%%%%%%%
\paragraph{v1.5:} 2017/05/21

\begin{itemize}
\item
more complete structuring introduced
\item
|\childdocof| introduced
\item
|\childdoc| renamed to |\childdocmain|
\item
|\childredirect| renamed to |\childdocforward| and |\childdocforwardprefix|
and functionality expanded
\end{itemize}

%%%%%%%%%%%%%%%%%%%%%%%%%%%%%%%%%%%%%%%%
\paragraph{v1.0:} 2017/04/27

\begin{itemize}
\item
manual and install package
\item
first version published on CTAN
\end{itemize}

%%%%%%%%%%%%%%%%%%%%%%%%%%%%%%%%%%%%%%%%
\paragraph{v0.6:} 2017/04/26

\begin{itemize}
\item
redirection mechanism added
\end{itemize}

%%%%%%%%%%%%%%%%%%%%%%%%%%%%%%%%%%%%%%%%
\paragraph{v0.5:} 2017/04/26

\begin{itemize}
\item
functionality in definition file
\end{itemize}


%%%%%%%%%%%%%%%%%%%%%%%%%%%%%%%%%%%%%%%%%%%%%%%%%%%%%%%%%%%%%%%%%%%%%%%%%%%%%%%%
%%%%%%%%%%%%%%%%%%%%%%%%%%%%%%%%%%%%%%%%%%%%%%%%%%%%%%%%%%%%%%%%%%%%%%%%%%%%%%%%
%%%%%%%%%%%%%%%%%%%%%%%%%%%%%%%%%%%%%%%%%%%%%%%%%%%%%%%%%%%%%%%%%%%%%%%%%%%%%%%%
\appendix

\settowidth\MacroIndent{\rmfamily\scriptsize 000\ }

 \DocInput{childdoc.dtx}

\end{document}
%</driver>
% \fi
%
% %%%%%%%%%%%%%%%%%%%%%%%%%%%%%%%%%%%%%%%%%%%%%%%%%%%%%%%%%%%%%%%%%%%%%%%%%%%%%%
% %%%%%%%%%%%%%%%%%%%%%%%%%%%%%%%%%%%%%%%%%%%%%%%%%%%%%%%%%%%%%%%%%%%%%%%%%%%%%%
% \section{Sample}
%\iffalse
%<*samplemain>
%\fi
%
% The following presents a sample document
% with two chapters, two parts, a title page,
% a compile flag as well as three forwarding files to set the flag.
% It consists of eight |.tex| files:
% \begin{center}
% \begin{tabular}{ll}
% |cdocsamp.tex|&main file\\
% |cdocsch1.tex|&include file for chapter 1\\
% |cdocsch2.tex|&include file for chapter 2\\
% |cdocspt3.tex|&include file for part 3\\
% |cdocspt4.tex|&include file for part 4\\
% |cdocsdrf.tex|&forwarding file for main file in draft mode\\
% |cdocsfi1.tex|&forwarding file for final version of chapter 1\\
% |cdocsfi2.tex|&forwarding file for final version of chapter 2\\
% \end{tabular}
% \end{center}
% Each of the eight files can be compiled directly by the \LaTeX{} compiler.
%
% %%%%%%%%%%%%%%%%%%%%%%%%%%%%%%%%%%%%%%
% \paragraph{Main File.}
%
% The main file is called |cdocsamp.tex|.
%
% Load the \textsf{childdoc} definitions and
% declare the filename for the main document:
%    \begin{macrocode}
\input{childdoc.def}
\childdocmain{}
%    \end{macrocode}

% Optional override for |\version| flag:
%    \begin{macrocode}
%%\ifchilddoc\else\providecommand{\version}{draft}\fi
%    \end{macrocode}

% Define the default values for the |\version| flag
% (|final| for the main file and |draft| for childs):
%    \begin{macrocode}
\ifchilddoc
\providecommand{\version}{draft}
\else
\providecommand{\version}{final}
\fi
%    \end{macrocode}

% Load the standard document class:
%    \begin{macrocode}
\documentclass[12pt]{article}
%    \end{macrocode}

% Start the document body:
%    \begin{macrocode}
\begin{document}
%    \end{macrocode}

% Declare a title page.
% Print title, part of document being processed and version flag:
%    \begin{macrocode}
\addtocounter{page}{-1}
\begin{center}
{\LARGE\bfseries{}childdoc example\par}
\vspace{1cm}
\ifchilddoc
\ifchilddocmanual part\else chapter\fi:
`\childdocname' of `\childdocjob'\par
\else
main document: `\childdocjob'\par
\fi
version: \version\par
\end{center}
\newpage
%    \end{macrocode}

% Manually include selected file,
% otherwise process as usual:
%    \begin{macrocode}
\ifchilddocmanual
\section*{part `\childdocname'}
\input{\childdocname}
\else
%    \end{macrocode}

% Include the two chapters:
%    \begin{macrocode}
\include{cdocsch1}
\include{cdocsch2}
%    \end{macrocode}

% Include the two parts unless only chapters should be displayed:
%    \begin{macrocode}
\ifchilddoc\else
\section{part three}
\input{cdocspt3}
\section{part four}
\input{cdocspt4}
\fi
%    \end{macrocode}

% Process as usual until here:
%    \begin{macrocode}
\fi
%    \end{macrocode}

% End of document body:
%    \begin{macrocode}
\end{document}
%    \end{macrocode}
%\iffalse
%</samplemain>
%\fi
%
% %%%%%%%%%%%%%%%%%%%%%%%%%%%%%%%%%%%%%%
% \paragraph{Chapter Include Files.}
%
% The include files are called |cdocsch1.tex| and |cdocsch2.tex|.
%
%\iffalse
%<*samplechap1|samplechap2>
%\fi

% Optional override for |\version| flag:
%    \begin{macrocode}
%%\providecommand{\version}{final}
%    \end{macrocode}

% Include the main document:
%    \begin{macrocode}
\input{childdoc.def}
\childdocof{cdocsamp}
%    \end{macrocode}

%\iffalse
%</samplechap1|samplechap2>
%\fi
%
%\iffalse
%<*samplechap1>
%\fi
% Some text for chapter 1:
%    \begin{macrocode}
\section{one}
some text in chapter one
%    \end{macrocode}

%\iffalse
%</samplechap1>
%\fi
% Some text for chapter 2:
%\iffalse
%<*samplechap2>
%\fi
%    \begin{macrocode}
\section{two}
more text in chapter two
%    \end{macrocode}

%\iffalse
%</samplechap2>
%\fi
%
% %%%%%%%%%%%%%%%%%%%%%%%%%%%%%%%%%%%%%%
% \paragraph{Part Include Files.}
%
% The include files are called |cdocspt3.tex| and |cdocspt4.tex|.
%
%\iffalse
%<*samplepart3|samplepart4>
%\fi

% Optional override for |\version| flag:
%    \begin{macrocode}
%%\providecommand{\version}{final}
%    \end{macrocode}

% Include the main document:
%    \begin{macrocode}
\input{childdoc.def}
\childdocby{cdocsamp}
%    \end{macrocode}

%\iffalse
%</samplepart3|samplepart4>
%\fi
%
%\iffalse
%<*samplepart3>
%\fi
% Some text for part 3:
%    \begin{macrocode}
some text in part three
%    \end{macrocode}

%\iffalse
%</samplepart3>
%\fi
% Some text for part 4:
%\iffalse
%<*samplepart4>
%\fi
%    \begin{macrocode}
more text in part four
%    \end{macrocode}

%\iffalse
%</samplepart4>
%\fi
%
% %%%%%%%%%%%%%%%%%%%%%%%%%%%%%%%%%%%%%%
% \paragraph{Forwarding for a Complete Draft.}
%
% The following forwarding file |cdocsdrf.tex|
% compiles the main document in draft mode:
%\iffalse
%<*sampledraft>
%\fi
%    \begin{macrocode}
\def\version{draft}
\input{childdoc.def}
\childdocforward{cdocsamp}
%    \end{macrocode}

%\iffalse
%</sampledraft>
%\fi
%
% %%%%%%%%%%%%%%%%%%%%%%%%%%%%%%%%%%%%%%
% \paragraph{Forwarding for Final Version of the Chapters.}
%
% The following forwarding files |cdocsfn1.tex| and |cdocsfn2.tex|
% (with identical content)
% compile the final versions of the child documents
% |cdocsch1.tex| and |cdocsch2.tex|, respectively:
%\iffalse
%<*samplefinal>
%\fi
%    \begin{macrocode}
\def\version{final}
\input{childdoc.def}
\childdocforwardprefix[cdocsamp]{cdocsfn}{cdocsch}
%    \end{macrocode}

%\iffalse
%</samplefinal>
%\fi
%
% %%%%%%%%%%%%%%%%%%%%%%%%%%%%%%%%%%%%%%
% \paragraph{Command Line Processing.}
%
% The following three command lines generate the output files
% |cdocscld|, |cdocscl1| and |cdocscl2|
% which should be identical to
% |cdocsdrf|, |cdocsch1| and |cdocsfn2|, respectively:
% \begin{center}
% \begin{tabular}{l}
% |latex -jobname cdocscld \|\\
% |  "\def\version{draft}\input{childdoc.def}\childdocforward{cdocsamp}"|\\
% |latex -jobname cdocscl1 \|\\
% |  "\input{childdoc.def}\childdocforward[cdocsamp]{cdocsch1}"|\\
% |latex -jobname cdocscl2 \|\\
% |  "\def\version{final}\input{childdoc.def}\childdocforward{cdocsch2}"|
% \end{tabular}
% \end{center}
% Note that the trailing backslash on each first line
% merely continues the input to the second line
% (for convenient cut ant paste).
% Furthermore, the command |latex| can be replaced by any
% of its alternative versions such as |pdflatex|.
%
% %%%%%%%%%%%%%%%%%%%%%%%%%%%%%%%%%%%%%%%%%%%%%%%%%%%%%%%%%%%%%%%%%%%%%%%%%%%%%%
% %%%%%%%%%%%%%%%%%%%%%%%%%%%%%%%%%%%%%%%%%%%%%%%%%%%%%%%%%%%%%%%%%%%%%%%%%%%%%%
% \section{Implementation}
%\iffalse
%<*package>
%\fi
%
% This section describes the definitions file |childdoc.def|.

% The definitions cannot be loaded using |\usepackage| or |\RequirePackage|
% which has a mechanism to prevent loading a style file more than once.
% When loading the definitions by means of |\input|
% multiple instances have to be prevented manually:
%\iffalse
%This code needs to be before the `\ProvidesFile' directive
%which is defined at the beginning of this file.
%Therefore it is also placed there and commented out here.
%</package>
%<*discard>
%\fi
%    \begin{macrocode}
\ifdefined\childdocmain\endinput\fi
%    \end{macrocode}
%\iffalse
%</discard>
%<*package>
%\fi
%
% \macro{\ifchilddoc}
% \macro{\ifchilddocmanual}
% The conditional |\ifchilddoc| tells whether a
% child (true) or main (false) document is being compiled.
% The conditional |\ifchilddocmanual| tells whether
% the |\includeonly| mechanism is used (false) or
% the selection of child files must be performed manually (true).
% The definitions initialise to false:
%    \begin{macrocode}
\newif\ifchilddoc
\newif\ifchilddocmanual
%    \end{macrocode}

% \macro{\childdocname}
% \macro{\childdocjob}
% The macro |\childdocname| stores the name of the main document
% to be compiled. The macro |\childdocjob| stores the name of
% the document on which the \LaTeX{} compiler was originally invoked.
% The content of |\jobname| cannot be compared
% to filenames specified in the source due to different catcodes.
% The following code rescans |\jobname|, stores the result
% in |\childdocname| and saves a copy in |\childdocjob|:
%    \begin{macrocode}
\edef\childdocname{\scantokens\expandafter{\jobname\noexpand}}
\let\childdocjob\childdocname
%    \end{macrocode}

% \macro{\childdocdisable}
% The macro |\childdocdisable| prevents the main file
% from being processed more than once.
% At this stage, the main document command |\childdocmain|
% is assumed to be called once again where it should do nothing.
% Any subsequent call to it should prevent
% a secondary processing of the main document
% It overwrites the forwarding commands
% |\childdocof| and |\childdocforward|
% with empty macros to prevent further inclusions of the main document:
%    \begin{macrocode}
\newcommand{\childdocdisable}
{
  \renewcommand{\childdocmain}[1]{\renewcommand{\childdocmain}[1]{\endinput}}
  \renewcommand{\childdocof}[1]{}
  \renewcommand{\childdocby}[2][]{}
  \renewcommand{\childdocforward}[2][]{}
  \renewcommand{\childdocdisable}{}
}
%    \end{macrocode}

% \macro{\childdocmain}
% The macro |\childdocmain| is to be called at the top of the main file
% with nothing or the main filename (without extension) as argument.
% First, it breaks loops.
% If the argument is not empty and does not match |\childdocname|
% (which is set by the first inclusion of |childdoc.def|),
% |\ifchilddoc| is set to true, |\includeonly| is applied to the child file
% and |\jobname| is set to the main file
% (for proper handling of |.aux| files):
%    \begin{macrocode}
\newcommand{\childdocmain}[1]
{
  \childdocdisable\childdocmain{}
  \if?#1?\else
    \begingroup
      \def\childdoctmp{#1}
      \ifx\childdoctmp\childdocname
        \def\childdoctmp{}
      \else
        \def\childdoctmp
        {
          \childdoctrue
          \includeonly{\childdocname}
          \def\childdocjob{#1}
          \def\jobname{#1}
        }
      \fi
      \expandafter
    \endgroup
    \childdoctmp
  \fi
}
%    \end{macrocode}

% \macro{\childdocof}
% The command |\childdocof| redirects
% compilation to the main file |#1|.
%    \begin{macrocode}
\newcommand{\childdocof}[1]
{
  \childdocdisable
  \childdoctrue
  \includeonly{\childdocname}
  \def\jobname{#1}
  \def\childdocjob{#1}
  \input{#1}
}
%    \end{macrocode}

% \macro{\childdocby}
% The command |\childdocby| ....
%    \begin{macrocode}
\newcommand{\childdocby}[2][]
{
  \childdocdisable
  \childdoctrue
  \childdocmanualtrue
  \if?#1?\else
    \def\jobname{#2}
  \fi
  \def\childdocjob{#2}
  \input{#2}
  \endinput
}
%    \end{macrocode}

% \macro{\childdocforward}
% The command |\childdocforward| redirects
% compilation to the main file or
% (if the optional argument is given) a child file.
% Parameters are set as if the main file
% or a child file starting with |\childdocof| was compiled.
% Then compilation is handed over to the main file:
%    \begin{macrocode}
\newcommand{\childdocforward}[2][]
{
  \begingroup
    \if?#1?
      \def\childdoctmp
      {
        \def\childdocname{#2}
        \def\childdocjob{#2}
        \def\jobname{#2}
        \input{#2}
        \endinput
      }
    \else
      \def\childdoctmp
      {
        \childdocdisable
        \def\childdocname{#2}
        \childdoctrue
        \includeonly{#2}
        \def\childdocjob{#1}
        \def\jobname{#1}
        \input{#1}
        \endinput
      }
    \fi
    \expandafter
  \endgroup
  \childdoctmp
}
%    \end{macrocode}

% \macro{\childdocforwardprefix}
% The command |\childdocforwardprefix| redirects
% compilation to the main or a child file by means of a pattern.
% The prefix |#1| in the current filename is replaced by |#2|
% and the suffix of the current filename is kept
% (it is assumed that the filename does not contain the substring `|~~~|'
% which is used as a delimiter).
% Compilation is handed over to the new file by |\childdocforward|:
%    \begin{macrocode}
\newcommand{\childdocforwardprefix}[3][]
{
  \begingroup
    \def\childdocextract #2##1~~~{\def\childdoctmp{\childdocforward[#1]{#3##1}}}
    \expandafter\childdocextract\childdocname~~~
    \expandafter
  \endgroup
  \childdoctmp
}
%    \end{macrocode}

% \macro{\childdoc}
% The deprecated macro |\childdoc| is a legacy version of |\childdocmain|:
%    \begin{macrocode}
\newcommand{\childdoc}{\childdocmain}
%    \end{macrocode}

% \macro{\childdocredirect}
% The deprecated macro |\childdocredirect| is a legacy version
% of |\childdocforward| and |\childdocforwardprefix|:
%    \begin{macrocode}
\newcommand{\childdocredirect}[2][]
{
  \begingroup
    \if?#1?
      \def\childdoctmp{\childdocforward{#2}}
    \else
      \def\childdoctmp{\childdocforwardprefix{#1}{#2}}
    \fi
    \expandafter
  \endgroup
  \childdoctmp
}
%    \end{macrocode}

%\iffalse
%</package>
%\fi
%
\endinput
\childdocforward[|\textit{main}|]{|\textit{dest}|}"|
\end{center}
%
Here \textit{target} is the name of the output file,
\textit{main} is the name of the main file
and \textit{dest} is the name of the main or child file to be processed
(all filenames without extensions).
The optional argument \textit{main} can be omitted
if \textit{main} matches \textit{dest}.
Optionally, compilation \textit{flags} can be defined via |\def| commands.
This command line makes the \TeX{} engine believe
it is compiling the file \textit{target}
whose content is specified as the latter parameter.
The provided code then forwards the processing to
\textit{main} or \textit{dest} as described in \secref{sec:forward}.

%%%%%%%%%%%%%%%%%%%%%%%%%%%%%%%%%%%%%%%%%%%%%%%%%%%%%%%%%%%%%%%%%%%%%%%%%%%%%%%%
\subsection{Include by Input}
\label{sec:input}

Including child documents by |\include| has some restrictions by design.
Most notably, the content of a child document always occupies
its own set of pages; pages cannot be shared between child documents.
Usually, this behaviour makes perfect sense
because each child document contain an essential part of the document.
However, in some situations it may be desirable to compose
a document from a collection of parts
without having mandatory page breaks between then.
For this case, the package
provides a mechanism to include parts
by |\input| which can also be processed individually.
However, by construction this mechanism
requires manual handling of the content to be output.

%%%%%%%%%%%%%%%%%%%%%%%%%%%%%%%%%%%%%%%%
\DescribeMacro{\ifchilddocmanual}
The main file should be prepared as usual, see \secref{sec:include}.
However, the document body must make a distinction
between processing of an individual part and of the main document, e.g.:
%
\begin{center}
\begin{tabular}{l}
|\ifchilddocmanual|\\
|\input{\childdocname}|\\
|\||else|\\
\textit{document body with }|\input{|\textit{part}|}|\\
|\||fi|
\end{tabular}
\end{center}
%
The conditional |\ifchilddocmanual| is true whenever
a part to be included by |\input| is being compiled,
and the name of the part is stored in |\childdocname|.

%%%%%%%%%%%%%%%%%%%%%%%%%%%%%%%%%%%%%%%%
\DescribeMacro{\childdocby}
Each part to be included by |\input| should start with:
%
\begin{center}
\begin{tabular}{l}
|% \iffalse
%
% childdoc.dtx Copyright (C) 2017-2018 Niklas Beisert
%
% This work may be distributed and/or modified under the
% conditions of the LaTeX Project Public License, either version 1.3
% of this license or (at your option) any later version.
% The latest version of this license is in
%   http://www.latex-project.org/lppl.txt
% and version 1.3 or later is part of all distributions of LaTeX
% version 2005/12/01 or later.
%
% This work has the LPPL maintenance status `maintained'.
%
% The Current Maintainer of this work is Niklas Beisert.
%
% This work consists of the files childdoc.dtx and childdoc.ins
% and the derived files childdoc.def and cdocsamp.tex with
% cdocsch1.tex, cdocsch2.tex, cdocsdrf.tex, cdocsfn1.tex, cdocsfn2.tex.
%
%<package>\ifdefined\childdocmain\endinput\fi
%<package>\ProvidesFile{childdoc.def}[2018/12/30 v2.0 child document driver]
%<samplemain>\ProvidesFile{cdocsamp.tex}[2018/12/30 v2.0 sample for childdoc]
%<*driver>
%\ProvidesFile{childdoc.drv}[2018/12/30 v2.0 childdoc reference manual file]
\PassOptionsToClass{10pt,a4paper}{article}
\documentclass{ltxdoc}

\usepackage[margin=35mm]{geometry}
\usepackage{hyperref}
\usepackage{hyperxmp}
\usepackage[usenames]{color}

\hypersetup{colorlinks=true}
\hypersetup{pdfstartview=FitH}
\hypersetup{pdfpagemode=UseNone}
\hypersetup{pdfsource={}}
\hypersetup{pdflang={en-UK}}
\hypersetup{pdfcopyright={Copyright 2017-2018 Niklas Beisert.
  This work may be distributed and/or modified under the
  conditions of the LaTeX Project Public License, either version 1.3
  of this license or (at your option) any later version.}}
\hypersetup{pdflicenseurl={http://www.latex-project.org/lppl.txt}}
\hypersetup{pdfcontactaddress={ETH Zurich, ITP, HIT K,
  Wolfgang-Pauli-Strasse 27}}
\hypersetup{pdfcontactpostcode={8093}}
\hypersetup{pdfcontactcity={Zurich}}
\hypersetup{pdfcontactcountry={Switzerland}}
\hypersetup{pdfcontactemail={nbeisert@itp.phys.ethz.ch}}
\hypersetup{pdfcontacturl={http://people.phys.ethz.ch/\xmptilde nbeisert/}}

\newcommand{\secref}[1]{\hyperref[#1]{section \ref*{#1}}}

\parskip1ex
\parindent0pt
\let\olditemize\itemize
\def\itemize{\olditemize\parskip0pt}

\begin{document}

\title{The \textsf{childdoc} Package}
\hypersetup{pdftitle={The childdoc Package}}
\author{Niklas Beisert\\[2ex]
  Institut f\"ur Theoretische Physik\\
  Eidgen\"ossische Technische Hochschule Z\"urich\\
  Wolfgang-Pauli-Strasse 27, 8093 Z\"urich, Switzerland\\[1ex]
  \href{mailto:nbeisert@itp.phys.ethz.ch}
  {\texttt{nbeisert@itp.phys.ethz.ch}}}
\hypersetup{pdfauthor={Niklas Beisert}}
\hypersetup{pdfsubject={Manual for the LaTeX2e Package childdoc}}
\date{30 December 2018, \textsf{v2.0}}
\maketitle

\begin{abstract}\noindent
\textsf{childdoc} is a \LaTeXe{} package
that enables the direct compilation
of document sections included by |\include|
to individual files.
\end{abstract}

\begingroup
\parskip0ex
\tableofcontents
\endgroup

%%%%%%%%%%%%%%%%%%%%%%%%%%%%%%%%%%%%%%%%%%%%%%%%%%%%%%%%%%%%%%%%%%%%%%%%%%%%%%%%
%%%%%%%%%%%%%%%%%%%%%%%%%%%%%%%%%%%%%%%%%%%%%%%%%%%%%%%%%%%%%%%%%%%%%%%%%%%%%%%%
\section{Introduction}

\LaTeX{} provides a mechanism to structure a large document (such as a book)
into a main file and several child files (containing the chapters)
using the |\include| command.
This mechanism is beneficial for documents
which span hundreds of pages in order to
make the source file(s) more manageable.
Moreover, compilation can be restricted to
selected child files by means of the |\includeonly| command.
The latter feature can be used to reduce the compilation time while editing
(this was significantly more useful in the earlier days of \LaTeX{})
or to generate a smaller document which is easier to navigate.
Another application of |\includeonly| is to generate
documents consisting of selected parts of the complete document.

However, there are a few drawbacks of the plain |\include| mechanism:
\begin{itemize}
\item
The child files cannot be compiled on their own,
they can only be compiled via the main file.
A naive editing environment
(such as a text editor with an option
to have the current file processed by \LaTeX)
may require one to switch to the main file before compiling;
attempting to compile the child file produces errors.
\item
The main file must be modified (each time)
to adjust the |\includeonly| command
to the present needs. This easily leaves the main file in a messy state.
\item
The generated document will always carry the filename
of the main document. This is inconvenient if
several child files are to be compiled and
to be kept for distribution.
\end{itemize}

The present package provides a simple interface
to make child files individually compilable by \LaTeX{}.
Compiling a child file then has the same effect as compiling
the main file with an |\includeonly| command
to select the appropriate child.
Moreover the generated document will carry the name of the child
rather than the main file.
This resolves all three above issues.

This feature is meant to make the editing of books,
thesis documents and lecture notes somewhat more convenient.
However, the package can also be used efficiently for
composing a series of documents (such as exercise sheets)
which are typically distributed individually.
It then assists the author in generating the individual documents
(potentially in different versions)
as well as a document containing the collected series.
Another application is in developing style files
or other kinds of included material
where compilation of the style file could redirect
to a sample or test file.

%%%%%%%%%%%%%%%%%%%%%%%%%%%%%%%%%%%%%%%%%%%%%%%%%%%%%%%%%%%%%%%%%%%%%%%%%%%%%%%%
%%%%%%%%%%%%%%%%%%%%%%%%%%%%%%%%%%%%%%%%%%%%%%%%%%%%%%%%%%%%%%%%%%%%%%%%%%%%%%%%
\section{Usage}

First of all, the package \textsf{childdoc} is \emph{not} a standard
\LaTeXe{} |.sty| style file! Therefore it needs to be invoked in
a non-standard way.

%%%%%%%%%%%%%%%%%%%%%%%%%%%%%%%%%%%%%%%%%%%%%%%%%%%%%%%%%%%%%%%%%%%%%%%%%%%%%%%%
\subsection{Included Files}
\label{sec:include}

%%%%%%%%%%%%%%%%%%%%%%%%%%%%%%%%%%%%%%%%
\DescribeMacro{\childdocmain}
To use the package, add the commands
\begin{center}
\begin{tabular}{l}
|\input{childdoc.def}|\\
|\childdocmain{}|\\
\end{tabular}
\end{center}
at the very top of the main \LaTeX{} file,
in particular \emph{before} the |\documentclass| statement!
The argument of |\childdocmain| should be left empty
(but it must be present).

%%%%%%%%%%%%%%%%%%%%%%%%%%%%%%%%%%%%%%%%
\DescribeMacro{\childdocof}
Furthermore, add the commands
\begin{center}
\begin{tabular}{l}
|\input{childdoc.def}|\\
|\childdocof{|\textit{main}|}|\\
\end{tabular}
\end{center}
at the top of every child file \textit{child}
which is included by |\include{|\textit{child}|}|
from within the main file
(or at least for those files to be compiled individually).
The argument \textit{main} must be the filename of the main file.

There are a couple of
considerations in setting up the main and child documents:

%%%%%%%%%%%%%%%%%%%%%%%%%%%%%%%%%%%%%%%%
\paragraph{Restrictions.}

Please note the following restrictions:
\begin{itemize}
\item
|\childdocmain| must be called with one argument \textit{main}
to ensure compatibility with earlier version of the package.
It must either be empty (|\childdocmain{}|)
or precisely match the filename of the main file in which it is specified.
See \secref{sec:detection} for further information.
\item
The filename \textit{main} must be specified without the |.tex| extension.
\item
The filename \textit{main} is case sensitive
(even in case-insensitive file systems)
due to internal string comparison.
\item
The argument \textit{main} should be fully expanded, it cannot be a macro.
\item
Subdirectories and special characters should be avoided in filenames.
\item
The command |\childdocmain{|\textit{main}|}| must be followed by a whitespace.
It should not be followed immediately by another command
or by a comment mark `|%|'.
This is because the \TeX{} parser reads the token immediately following
the argument of |\childdocmain| and puts it
at the beginning of every child section;
however, a white\-space is ignored.
\end{itemize}

%%%%%%%%%%%%%%%%%%%%%%%%%%%%%%%%%%%%%%%%
\paragraph{Content of Main File.}

It is advisable to place all content in the child files included by |\include|.
Any output contained in the main file will appear in all child documents
unless suppressed manually;
it cannot be suppressed automatically by the |\includeonly| directive
and thus should normally be avoided.
A method to include some content in the main file
by means of conditional processing is described in \secref{sec:conditional}.

%%%%%%%%%%%%%%%%%%%%%%%%%%%%%%%%%%%%%%%%
\paragraph{Page Numbering.}

When only a part of the document is compiled,
the appropriate numbering of pages
(as well as other status parameters)
is determined from the |.aux| files.
The latter contain information from previous passes.
However this information needs to propagate through
all intermediate child documents.
Therefore the page numbering in child documents may well
be inconsistent until the complete document is compiled at least once.

A useful (if unconventional) way to always ensure a consistent
page numbering is to restart the numbering in each child document
and denote the pages by `\textit{child}|.|\textit{page}'
where \textit{child} represents the chapter/section number of the child file.
This can be achieved by the command
|\numberwithin{page}{|\textit{child}|}|
of the \textsf{amsmath} package
where \textit{child} can be |chapter| or |section|
depending on the chosen structuring.
Alternatively, one can modify the macro |\thepage| appropriately
and reset the counter |page| at the start of each child file.

%%%%%%%%%%%%%%%%%%%%%%%%%%%%%%%%%%%%%%%%%%%%%%%%%%%%%%%%%%%%%%%%%%%%%%%%%%%%%%%%
\subsection{Conditional Processing}
\label{sec:conditional}

The package provides a mechanism to compile different versions
of a document. To customise the versions further some conditional processing
can come in handy to distinguish which version is being compiled.
The package provides two macros to describe the compilation context:

%%%%%%%%%%%%%%%%%%%%%%%%%%%%%%%%%%%%%%%%
\DescribeMacro{\ifchilddoc}
The conditional |\ifchilddoc| distinguishes between the compilation of
child documents and the main document:
%
\begin{center}
|\ifchilddoc |\textit{child-code}| |[|\||else |\textit{main-code}]| \||fi|
\end{center}

%%%%%%%%%%%%%%%%%%%%%%%%%%%%%%%%%%%%%%%%
\DescribeMacro{\childdocname}
\DescribeMacro{\childdocjob}
The macro |\childdocname| contains the filename (without extension)
of the main or child file being processed.
Note that |\childdocjob| will always contain the name of the main file.

%%%%%%%%%%%%%%%%%%%%%%%%%%%%%%%%%%%%%%%%
\paragraph{Title Page.}

Conditional processing can be used to include a title or banner page
in the main document when proper precautions are taken.
Importantly, the code in the main file should ensure that the page counter
(as well as other status parameters which are stored in the |.aux| files)
takes the same value after the conditional processing.
Otherwise the page numbers may take divergent values
depending on which part is compiled.

For example, a title page could be declared by:
%
\begin{center}
\begin{tabular}{l}
|\ifchilddoc\||else|\\
|\addtocounter{page}{-1}|\\
\textit{code for title page}\\
|\newpage|\\
|\||fi|
\end{tabular}
\end{center}
%
A banner page for the child documents can be generated by:
%
\begin{center}
\begin{tabular}{l}
|\ifchilddoc|\\
|\addtocounter{page}{-1}|\\
\textit{code for banner page}\\
|\newpage|\\
|\||fi|
\end{tabular}
\end{center}
%
Here one could write a message such as:
\begin{center}
|This is the part \childdocname{} of \childdocjob{}.|
\end{center}

%%%%%%%%%%%%%%%%%%%%%%%%%%%%%%%%%%%%%%%%%%%%%%%%%%%%%%%%%%%%%%%%%%%%%%%%%%%%%%%%
\subsection{Flags}
\label{sec:flags}

The package makes it easy to generate different versions
of the main or child documents.
To this end compilation flags can be defined
and assigned different default values.
They will be particularly useful in conjunction
with the forwarding mechanism described in \secref{sec:forward}.

For example, it may be useful to have a flag |\version|
which can be set to |draft| or |final|.
The document source will contain some conditional code
depending on the value of |\version|.
Suppose further, the flag should default to |final| for the main file
and to |draft| for child files
which is a natural assignment for editing the document.
This is achieved by placing the following code
in the preamble of the main document
(below the |\childdocmain| directive):
%
\begin{center}
\begin{tabular}{l}
|\ifchilddoc|\\
|\providecommand{\version}{draft}|\\
|\||else|\\
|\providecommand{\version}{final}|\\
|\||fi|
\end{tabular}
\end{center}
%
The definition by |\providecommand| makes sure
that previous definitions are not overwritten.
Further statements |\providecommand{\version}{...}|
can thus be added before the above code to override it.

For the main file, one might add a line
(between |\childdocmain| and the above block)
%
\begin{center}
|%\ifchilddoc\||else\providecommand{\version}{draft}\||fi|
\end{center}
%
which can be uncommented to produce a draft version.
Likewise one can add a line to the very top of a child file
(above the |\childdocof{|\textit{main}|}| directive)
%
\begin{center}
|%\providecommand{\version}{final}|
\end{center}
%
which can be uncommented to produce the final version of this child document.

%%%%%%%%%%%%%%%%%%%%%%%%%%%%%%%%%%%%%%%%%%%%%%%%%%%%%%%%%%%%%%%%%%%%%%%%%%%%%%%%
\subsection{Forwarding}
\label{sec:forward}

Different versions of the main or child documents
using compilation flags as described in \secref{sec:flags}
can be (permanently) stored in different files
for convenient compilation, viewing and distribution.
To this end, the package defines a command
to pass on compilation to a different file:

%%%%%%%%%%%%%%%%%%%%%%%%%%%%%%%%%%%%%%%%
\DescribeMacro{\childdocforward}
The command |\childdocforward| redirects processing to
another source file:
%
\begin{center}
\begin{tabular}{l}
|\input{childdoc.def}|\\
|\childdocforward[|\textit{main}|]{|\textit{dest}|}|\\
\end{tabular}
\end{center}
%
The argument \textit{dest} is the destination file
(without extension).
It should be the main file or one of the child files.
Note that further \textsf{childdoc} directives
such as |\childdocof| and |\childdocforward|
in the indicated file will be processed in this form.
The optional argument \textit{main}
passes on directly to the main file \textit{main}
while pretending to compile the child \textit{dest}.
This form behaves as if \textit{dest}
issues |\childdocof{|\textit{main}|}| right away,
and no further \textsf{childdoc} directives will be processed.

%%%%%%%%%%%%%%%%%%%%%%%%%%%%%%%%%%%%%%%%
\DescribeMacro{\...prefix}
In the alternative form |\childdocforwardprefix|,
%
\begin{center}
\begin{tabular}{l}
|\input{childdoc.def}|\\
|\childdocforwardprefix[|\textit{main}|]{|\textit{prefix}|}{|\textit{dest}|}|
\end{tabular}
\end{center}
%
the destination file is determined by a pattern
depending on the current file:
To make this work, the current file must be called
`{\textit{prefix}\hspace{0.2em}\textit{suffix}}'
with \textit{prefix} matching precisely the argument.
Processing is then passed on to the file
`{\textit{dest}\hspace{0.2em}\textit{suffix}}'.
Surely, the same effect is achieved by
directly specifying the
argument `{\textit{dest}\hspace{0.2em}\textit{suffix}}'
in the first form.
However, that requires to set up a different file
for each child. With the alternative form of the command
all these files can have exactly the same content
which simplifies setting them up and maintaining them.

For example, the following file |draft.tex|
with a compilation flag |\version| as described in \secref{sec:flags}
compiles the main document as a draft:
%
\begin{center}
\begin{tabular}{l}
|\def\version{draft}|\\
|\input{childdoc.def}|\\
|\childdocforward{|\textit{main}|}|
\end{tabular}
\end{center}
%
Likewise, the following files |final|\textit{nn}|.tex|
compile the final version of the child document
|child|\textit{nn}|.tex|:
%
\begin{center}
\begin{tabular}{l}
|\def\version{final}|\\
|\input{childdoc.def}|\\
|\childdocforwardprefix{final}{child}|
\end{tabular}
\end{center}
%

Note that when several versions of a main file and/or of each child file
are to be generated, it may be convenient to set up a |Makefile| or
shell script to automatise the process.

%%%%%%%%%%%%%%%%%%%%%%%%%%%%%%%%%%%%%%%%%%%%%%%%%%%%%%%%%%%%%%%%%%%%%%%%%%%%%%%%
\subsection{Command Line Processing}
\label{sec:commandline}

The effect of redirection files can also be achieved by invoking
the \LaTeX{} compiler with a more elaborate command line.
Most conveniently this should be done as part
of a shell script or a |Makefile|.

When using \textsf{childdoc} in the main file, the following
command lines effectively perform a redirection
(note that depending on the shell being used,
backslashes may have to be doubled: `|\|' $\to$ `|\\|'):
%
\begin{center}
|... -jobname "|\textit{target}|" |\\|"|[\textit{flags}]%
|\input{childdoc.def}\childdocforward[|\textit{main}|]{|\textit{dest}|}"|
\end{center}
%
Here \textit{target} is the name of the output file,
\textit{main} is the name of the main file
and \textit{dest} is the name of the main or child file to be processed
(all filenames without extensions).
The optional argument \textit{main} can be omitted
if \textit{main} matches \textit{dest}.
Optionally, compilation \textit{flags} can be defined via |\def| commands.
This command line makes the \TeX{} engine believe
it is compiling the file \textit{target}
whose content is specified as the latter parameter.
The provided code then forwards the processing to
\textit{main} or \textit{dest} as described in \secref{sec:forward}.

%%%%%%%%%%%%%%%%%%%%%%%%%%%%%%%%%%%%%%%%%%%%%%%%%%%%%%%%%%%%%%%%%%%%%%%%%%%%%%%%
\subsection{Include by Input}
\label{sec:input}

Including child documents by |\include| has some restrictions by design.
Most notably, the content of a child document always occupies
its own set of pages; pages cannot be shared between child documents.
Usually, this behaviour makes perfect sense
because each child document contain an essential part of the document.
However, in some situations it may be desirable to compose
a document from a collection of parts
without having mandatory page breaks between then.
For this case, the package
provides a mechanism to include parts
by |\input| which can also be processed individually.
However, by construction this mechanism
requires manual handling of the content to be output.

%%%%%%%%%%%%%%%%%%%%%%%%%%%%%%%%%%%%%%%%
\DescribeMacro{\ifchilddocmanual}
The main file should be prepared as usual, see \secref{sec:include}.
However, the document body must make a distinction
between processing of an individual part and of the main document, e.g.:
%
\begin{center}
\begin{tabular}{l}
|\ifchilddocmanual|\\
|\input{\childdocname}|\\
|\||else|\\
\textit{document body with }|\input{|\textit{part}|}|\\
|\||fi|
\end{tabular}
\end{center}
%
The conditional |\ifchilddocmanual| is true whenever
a part to be included by |\input| is being compiled,
and the name of the part is stored in |\childdocname|.

%%%%%%%%%%%%%%%%%%%%%%%%%%%%%%%%%%%%%%%%
\DescribeMacro{\childdocby}
Each part to be included by |\input| should start with:
%
\begin{center}
\begin{tabular}{l}
|\input{childdoc.def}|\\
|\childdocby{|\textit{main}|}|\\
\end{tabular}
\end{center}
%
The directive |\childdocby| is similar to |\childdocof|
described in \secref{sec:include},
but the subsequent selection of content must be done manually.
To that end, both |\ifchilddoc| and |\ifchilddocmanual|
will be true upon processing of a part,
and the name of the part is stored in |\childdocname|.
Note that |\jobname| will be set to the filename of the current part
so that each part receives an individual |.aux| file
that does not interfere with the |.aux| file(s) of the main document.
This behaviour can be altered by the alternative form
|\childdocby[*]{|\textit{main}|}| (with a non-empty optional argument)
which uses the |.aux| file of the main document
by setting |\jobname| to \textit{main}.

%%%%%%%%%%%%%%%%%%%%%%%%%%%%%%%%%%%%%%%%%%%%%%%%%%%%%%%%%%%%%%%%%%%%%%%%%%%%%%%%
\subsection{Driver Development}
\label{sec:driver}

The \textsf{childdoc} mechanism can also be use for the development
of definition files such as \LaTeX{} styles or classes.
This case differs from the above setup with multiple parts
included by |\include| in that no |\includeonly| should be invoked.
This can be achieved by starting the include file
(before |\ProvidesPackage|) with:
%
\begin{center}
\begin{tabular}{l}
|\input{childdoc.def}|\\
|\childdocforward{|\textit{main}|}|\\
\end{tabular}
\end{center}
%
or alternatively with:
%
\begin{center}
\begin{tabular}{l}
|\input{childdoc.def}|\\
|\childdocby{|\textit{main}|}|\\
\end{tabular}
\end{center}
%
Both forms have slightly different effects as described above.
The main file is prepared as usual, see \secref{sec:include}.

%%%%%%%%%%%%%%%%%%%%%%%%%%%%%%%%%%%%%%%%%%%%%%%%%%%%%%%%%%%%%%%%%%%%%%%%%%%%%%%%
\subsection{Legacy Detection}
\label{sec:detection}

The directive |\childdocmain| in the main file can detect
whether the complete document or merely a child is to be compiled
even without using the directive |\childdocof|.
This method is deprecated because it is less robust
and there is no compelling reason to use it;
it is merely provided for backward compatibility
and it may be removed in future versions.

If the detection mechanism is to be used,
it is mandatory to correctly specify
the filename of the main file as the argument of |\childdocmain|:
%
\begin{center}
\begin{tabular}{l}
|\input{childdoc.def}|\\
|\childdocmain{|\textit{main}|}|\\
\end{tabular}
\end{center}
%
If |\jobname| does not match the argument \textit{main} of |\childdocmain|,
it is assumed that |\jobname| points to the child file to be compiled.
When using |\childdocmain| with the main file specified as argument,
it suffices to start a child file
with just |\input{|\textit{main}|}|
without loading of the package and using |\childdocof|.
If instead all processing is done
with the appropriate \textsf{childdoc} directives,
the argument of \textit{main} of |\childdocmain| can be empty.

An alternative version of the command line processing described
in \secref{sec:commandline} using the detection mechanism reads:
%
\begin{center}
|... -jobname "|\textit{target}|" "|[\textit{flags}]%
[|\def\jobname{|\textit{dest}|}|]|\input{|\textit{main}|}"|
\end{center}

%%%%%%%%%%%%%%%%%%%%%%%%%%%%%%%%%%%%%%%%%%%%%%%%%%%%%%%%%%%%%%%%%%%%%%%%%%%%%%%%
\subsection{Manual Code}
\label{sec:manual}

In case one cannot be certain whether the definitions file |childdoc.def|
is installed on the target \TeX{} distribution
and one prefers not to ship it,
it is conceivable to paste a few relevant commands into the sources.

To that end, drop all statements |\input{childdoc.def}|
and perform the replacements as outlined below.
Instead of |\childdocmain{|\textit{main}|}| add the following code
to the top of the main file:
%
\begin{center}
\begin{tabular}{l}
|\||ifdefined\childdocname\endinput\||fi\newif\ifchilddoc|\\
|\edef\childdocname{\scantokens\expandafter{\jobname\noexpand}}|\\
|\def\childdocmain{|\textit{main}|}\||ifx\childdocmain\childdocname\||else|\\
|\childdoctrue\includeonly{\childdocname}\let\jobname\childdocmain\||fi|\\
\end{tabular}
\end{center}
%
Instead of |\childdocof{|\textit{main}|}| just include the main file
at the top of each child file:
%
\begin{center}
|\input{|\textit{main}|}|
\end{center}
%
A simple redirection |\childdocforward{|\textit{dest}|}| is achieved by:
%
\begin{center}
|\def\jobname{|\textit{dest}|}\input{\jobname}|
\end{center}
%
The redirection with prefix
|\childdocforwardprefix[|\textit{prefix}|]{|\textit{dest}|}|
is accomplished by:
%
\begin{center}
\begin{tabular}{l}
|{\edef\jobname{\scantokens\expandafter{\jobname\noexpand}}|\\
|\def\redirectjob |\textit{prefix}|#1~~~{\gdef\jobname{|\textit{dest}|#1}}|\\
|\expandafter\redirectjob\jobname~~~}\input{\jobname}|
\end{tabular}
\end{center}

In an alternative approach,
child documents can be compiled by a specific command line
without additional code or specific definitions:
%
\begin{center}
|... -jobname "|\textit{target}|" "|[\textit{flags}]%
|\includeonly{|\textit{dest}|}\input{|\textit{main}|}"|
\end{center}
%

%%%%%%%%%%%%%%%%%%%%%%%%%%%%%%%%%%%%%%%%%%%%%%%%%%%%%%%%%%%%%%%%%%%%%%%%%%%%%%%%
%%%%%%%%%%%%%%%%%%%%%%%%%%%%%%%%%%%%%%%%%%%%%%%%%%%%%%%%%%%%%%%%%%%%%%%%%%%%%%%%
\section{Information}

%%%%%%%%%%%%%%%%%%%%%%%%%%%%%%%%%%%%%%%%%%%%%%%%%%%%%%%%%%%%%%%%%%%%%%%%%%%%%%%%
\subsection{Copyright}

Copyright \copyright{} 2017--2018 Niklas Beisert

This work may be distributed and/or modified under the
conditions of the \LaTeX{} Project Public License, either version 1.3
of this license or (at your option) any later version.
The latest version of this license is in
  \url{http://www.latex-project.org/lppl.txt}
and version 1.3 or later is part of all distributions of \LaTeX{}
version 2005/12/01 or later.

This work has the LPPL maintenance status `maintained'.

The Current Maintainer of this work is Niklas Beisert.

This work consists of the files |README.txt|, |childdoc.ins| and |childdoc.dtx|
as well as the derived files |childdoc.def|, |cdocsamp.tex|
with |cdocsch1.tex|, |cdocsch2.tex|, |cdocspt3.tex|, |cdocspt4.tex|,
|cdocsdrf.tex|, |cdocsfn1.tex|, |cdocsfn2.tex|
as well as |childdoc.pdf|.

%%%%%%%%%%%%%%%%%%%%%%%%%%%%%%%%%%%%%%%%%%%%%%%%%%%%%%%%%%%%%%%%%%%%%%%%%%%%%%%%
\subsection{Files and Installation}

The package consists of the files:
%
\begin{center}
\begin{tabular}{ll}
    |README.txt|   & readme file \\
    |childdoc.ins| & installation file \\
    |childdoc.dtx| & source file \\
    |childdoc.def| & definition file \\
    |cdocsamp.tex| & sample main file \\
    |cdocsch1.tex| & sample include file \\
    |cdocsch2.tex| & sample include file \\
    |cdocspt3.tex| & sample part file \\
    |cdocspt4.tex| & sample part file \\
    |cdocsdrf.tex| & sample redirection file \\
    |cdocsfn1.tex| & sample redirection file \\
    |cdocsfn2.tex| & sample redirection file \\
    |childdoc.pdf| & manual
\end{tabular}
\end{center}
%
The distribution consists of the files
|README.txt|, |childdoc.ins| and |childdoc.dtx|.
%
\begin{itemize}
\item
Run (pdf)\LaTeX{} on |childdoc.dtx|
to compile the manual |childdoc.pdf| (this file).
\item
Run \LaTeX{} on |childdoc.ins| to create the definitions file |childdoc.def|
and the sample |cdocsamp.tex| with include files
|cdocsch1.tex|, |cdocsch2.tex|, |cdocspt3.tex|, |cdocspt4.tex|,
|cdocsdrf.tex|, |cdocsfn1.tex|, |cdocsfn2.tex|.
Then copy the file |childdoc.def| to an appropriate directory of your \LaTeX{}
distribution, e.g.\ \textit{texmf-root}|/tex/latex/childdoc|.
\end{itemize}

%%%%%%%%%%%%%%%%%%%%%%%%%%%%%%%%%%%%%%%%%%%%%%%%%%%%%%%%%%%%%%%%%%%%%%%%%%%%%%%%
\subsection{Related CTAN Packages}

There are several other packages which offer a similar functionality:
%
\begin{itemize}
\item
The packages
\href{http://ctan.org/pkg/docmute}{\textsf{docmute}},
\href{http://ctan.org/pkg/includex}{\textsf{includex}} and
\href{http://ctan.org/pkg/standalone}{\textsf{standalone}}
provide commands to include only the document body of
a child file thus allowing both files to be compiled individually.
\item
The packages \href{http://ctan.org/pkg/subdocs}{\textsf{subdocs}}
and \href{http://ctan.org/pkg/subfiles}{\textsf{subfiles}}
provide structures in which the main and child documents can be
encapsulated and allowing them to be compiled individually.
The inclusion mechanism is different from the conventional |\include|.
\item
The package \href{http://ctan.org/pkg/combine}{\textsf{combine}}
is an elaborate solution to combine several documents into one.
\end{itemize}
%
See also the CTAN topic \href{http://ctan.org/topic/subdocs}{\textsf{subdocs}}
for further related packages.
The present package differs from the above solutions in that
a document structure constructed with the conventional |\include| mechanism
just needs two extra commands at the top of every file
such that all constituent files can be compiled individually.

%%%%%%%%%%%%%%%%%%%%%%%%%%%%%%%%%%%%%%%%%%%%%%%%%%%%%%%%%%%%%%%%%%%%%%%%%%%%%%%%
%\subsection{Feature Suggestions}
%
%The following is a list of features which may be useful for future
%versions of this package:
%%
%\begin{itemize}
%\item
%\ldots
%\end{itemize}

%%%%%%%%%%%%%%%%%%%%%%%%%%%%%%%%%%%%%%%%%%%%%%%%%%%%%%%%%%%%%%%%%%%%%%%%%%%%%%%%
\subsection{Revision History}

%%%%%%%%%%%%%%%%%%%%%%%%%%%%%%%%%%%%%%%%
\paragraph{v2.0:} 2018/12/30

\begin{itemize}
\item
immediate forward processing
\item
added |\childdocby| mechanism
\item
manual restructured
\end{itemize}

%%%%%%%%%%%%%%%%%%%%%%%%%%%%%%%%%%%%%%%%
\paragraph{v1.6:} 2018/01/17

\begin{itemize}
\item
application for development of include files
\item
corrections to manual
\end{itemize}

%%%%%%%%%%%%%%%%%%%%%%%%%%%%%%%%%%%%%%%%
\paragraph{v1.5:} 2017/05/21

\begin{itemize}
\item
more complete structuring introduced
\item
|\childdocof| introduced
\item
|\childdoc| renamed to |\childdocmain|
\item
|\childredirect| renamed to |\childdocforward| and |\childdocforwardprefix|
and functionality expanded
\end{itemize}

%%%%%%%%%%%%%%%%%%%%%%%%%%%%%%%%%%%%%%%%
\paragraph{v1.0:} 2017/04/27

\begin{itemize}
\item
manual and install package
\item
first version published on CTAN
\end{itemize}

%%%%%%%%%%%%%%%%%%%%%%%%%%%%%%%%%%%%%%%%
\paragraph{v0.6:} 2017/04/26

\begin{itemize}
\item
redirection mechanism added
\end{itemize}

%%%%%%%%%%%%%%%%%%%%%%%%%%%%%%%%%%%%%%%%
\paragraph{v0.5:} 2017/04/26

\begin{itemize}
\item
functionality in definition file
\end{itemize}


%%%%%%%%%%%%%%%%%%%%%%%%%%%%%%%%%%%%%%%%%%%%%%%%%%%%%%%%%%%%%%%%%%%%%%%%%%%%%%%%
%%%%%%%%%%%%%%%%%%%%%%%%%%%%%%%%%%%%%%%%%%%%%%%%%%%%%%%%%%%%%%%%%%%%%%%%%%%%%%%%
%%%%%%%%%%%%%%%%%%%%%%%%%%%%%%%%%%%%%%%%%%%%%%%%%%%%%%%%%%%%%%%%%%%%%%%%%%%%%%%%
\appendix

\settowidth\MacroIndent{\rmfamily\scriptsize 000\ }

 \DocInput{childdoc.dtx}

\end{document}
%</driver>
% \fi
%
% %%%%%%%%%%%%%%%%%%%%%%%%%%%%%%%%%%%%%%%%%%%%%%%%%%%%%%%%%%%%%%%%%%%%%%%%%%%%%%
% %%%%%%%%%%%%%%%%%%%%%%%%%%%%%%%%%%%%%%%%%%%%%%%%%%%%%%%%%%%%%%%%%%%%%%%%%%%%%%
% \section{Sample}
%\iffalse
%<*samplemain>
%\fi
%
% The following presents a sample document
% with two chapters, two parts, a title page,
% a compile flag as well as three forwarding files to set the flag.
% It consists of eight |.tex| files:
% \begin{center}
% \begin{tabular}{ll}
% |cdocsamp.tex|&main file\\
% |cdocsch1.tex|&include file for chapter 1\\
% |cdocsch2.tex|&include file for chapter 2\\
% |cdocspt3.tex|&include file for part 3\\
% |cdocspt4.tex|&include file for part 4\\
% |cdocsdrf.tex|&forwarding file for main file in draft mode\\
% |cdocsfi1.tex|&forwarding file for final version of chapter 1\\
% |cdocsfi2.tex|&forwarding file for final version of chapter 2\\
% \end{tabular}
% \end{center}
% Each of the eight files can be compiled directly by the \LaTeX{} compiler.
%
% %%%%%%%%%%%%%%%%%%%%%%%%%%%%%%%%%%%%%%
% \paragraph{Main File.}
%
% The main file is called |cdocsamp.tex|.
%
% Load the \textsf{childdoc} definitions and
% declare the filename for the main document:
%    \begin{macrocode}
\input{childdoc.def}
\childdocmain{}
%    \end{macrocode}

% Optional override for |\version| flag:
%    \begin{macrocode}
%%\ifchilddoc\else\providecommand{\version}{draft}\fi
%    \end{macrocode}

% Define the default values for the |\version| flag
% (|final| for the main file and |draft| for childs):
%    \begin{macrocode}
\ifchilddoc
\providecommand{\version}{draft}
\else
\providecommand{\version}{final}
\fi
%    \end{macrocode}

% Load the standard document class:
%    \begin{macrocode}
\documentclass[12pt]{article}
%    \end{macrocode}

% Start the document body:
%    \begin{macrocode}
\begin{document}
%    \end{macrocode}

% Declare a title page.
% Print title, part of document being processed and version flag:
%    \begin{macrocode}
\addtocounter{page}{-1}
\begin{center}
{\LARGE\bfseries{}childdoc example\par}
\vspace{1cm}
\ifchilddoc
\ifchilddocmanual part\else chapter\fi:
`\childdocname' of `\childdocjob'\par
\else
main document: `\childdocjob'\par
\fi
version: \version\par
\end{center}
\newpage
%    \end{macrocode}

% Manually include selected file,
% otherwise process as usual:
%    \begin{macrocode}
\ifchilddocmanual
\section*{part `\childdocname'}
\input{\childdocname}
\else
%    \end{macrocode}

% Include the two chapters:
%    \begin{macrocode}
\include{cdocsch1}
\include{cdocsch2}
%    \end{macrocode}

% Include the two parts unless only chapters should be displayed:
%    \begin{macrocode}
\ifchilddoc\else
\section{part three}
\input{cdocspt3}
\section{part four}
\input{cdocspt4}
\fi
%    \end{macrocode}

% Process as usual until here:
%    \begin{macrocode}
\fi
%    \end{macrocode}

% End of document body:
%    \begin{macrocode}
\end{document}
%    \end{macrocode}
%\iffalse
%</samplemain>
%\fi
%
% %%%%%%%%%%%%%%%%%%%%%%%%%%%%%%%%%%%%%%
% \paragraph{Chapter Include Files.}
%
% The include files are called |cdocsch1.tex| and |cdocsch2.tex|.
%
%\iffalse
%<*samplechap1|samplechap2>
%\fi

% Optional override for |\version| flag:
%    \begin{macrocode}
%%\providecommand{\version}{final}
%    \end{macrocode}

% Include the main document:
%    \begin{macrocode}
\input{childdoc.def}
\childdocof{cdocsamp}
%    \end{macrocode}

%\iffalse
%</samplechap1|samplechap2>
%\fi
%
%\iffalse
%<*samplechap1>
%\fi
% Some text for chapter 1:
%    \begin{macrocode}
\section{one}
some text in chapter one
%    \end{macrocode}

%\iffalse
%</samplechap1>
%\fi
% Some text for chapter 2:
%\iffalse
%<*samplechap2>
%\fi
%    \begin{macrocode}
\section{two}
more text in chapter two
%    \end{macrocode}

%\iffalse
%</samplechap2>
%\fi
%
% %%%%%%%%%%%%%%%%%%%%%%%%%%%%%%%%%%%%%%
% \paragraph{Part Include Files.}
%
% The include files are called |cdocspt3.tex| and |cdocspt4.tex|.
%
%\iffalse
%<*samplepart3|samplepart4>
%\fi

% Optional override for |\version| flag:
%    \begin{macrocode}
%%\providecommand{\version}{final}
%    \end{macrocode}

% Include the main document:
%    \begin{macrocode}
\input{childdoc.def}
\childdocby{cdocsamp}
%    \end{macrocode}

%\iffalse
%</samplepart3|samplepart4>
%\fi
%
%\iffalse
%<*samplepart3>
%\fi
% Some text for part 3:
%    \begin{macrocode}
some text in part three
%    \end{macrocode}

%\iffalse
%</samplepart3>
%\fi
% Some text for part 4:
%\iffalse
%<*samplepart4>
%\fi
%    \begin{macrocode}
more text in part four
%    \end{macrocode}

%\iffalse
%</samplepart4>
%\fi
%
% %%%%%%%%%%%%%%%%%%%%%%%%%%%%%%%%%%%%%%
% \paragraph{Forwarding for a Complete Draft.}
%
% The following forwarding file |cdocsdrf.tex|
% compiles the main document in draft mode:
%\iffalse
%<*sampledraft>
%\fi
%    \begin{macrocode}
\def\version{draft}
\input{childdoc.def}
\childdocforward{cdocsamp}
%    \end{macrocode}

%\iffalse
%</sampledraft>
%\fi
%
% %%%%%%%%%%%%%%%%%%%%%%%%%%%%%%%%%%%%%%
% \paragraph{Forwarding for Final Version of the Chapters.}
%
% The following forwarding files |cdocsfn1.tex| and |cdocsfn2.tex|
% (with identical content)
% compile the final versions of the child documents
% |cdocsch1.tex| and |cdocsch2.tex|, respectively:
%\iffalse
%<*samplefinal>
%\fi
%    \begin{macrocode}
\def\version{final}
\input{childdoc.def}
\childdocforwardprefix[cdocsamp]{cdocsfn}{cdocsch}
%    \end{macrocode}

%\iffalse
%</samplefinal>
%\fi
%
% %%%%%%%%%%%%%%%%%%%%%%%%%%%%%%%%%%%%%%
% \paragraph{Command Line Processing.}
%
% The following three command lines generate the output files
% |cdocscld|, |cdocscl1| and |cdocscl2|
% which should be identical to
% |cdocsdrf|, |cdocsch1| and |cdocsfn2|, respectively:
% \begin{center}
% \begin{tabular}{l}
% |latex -jobname cdocscld \|\\
% |  "\def\version{draft}\input{childdoc.def}\childdocforward{cdocsamp}"|\\
% |latex -jobname cdocscl1 \|\\
% |  "\input{childdoc.def}\childdocforward[cdocsamp]{cdocsch1}"|\\
% |latex -jobname cdocscl2 \|\\
% |  "\def\version{final}\input{childdoc.def}\childdocforward{cdocsch2}"|
% \end{tabular}
% \end{center}
% Note that the trailing backslash on each first line
% merely continues the input to the second line
% (for convenient cut ant paste).
% Furthermore, the command |latex| can be replaced by any
% of its alternative versions such as |pdflatex|.
%
% %%%%%%%%%%%%%%%%%%%%%%%%%%%%%%%%%%%%%%%%%%%%%%%%%%%%%%%%%%%%%%%%%%%%%%%%%%%%%%
% %%%%%%%%%%%%%%%%%%%%%%%%%%%%%%%%%%%%%%%%%%%%%%%%%%%%%%%%%%%%%%%%%%%%%%%%%%%%%%
% \section{Implementation}
%\iffalse
%<*package>
%\fi
%
% This section describes the definitions file |childdoc.def|.

% The definitions cannot be loaded using |\usepackage| or |\RequirePackage|
% which has a mechanism to prevent loading a style file more than once.
% When loading the definitions by means of |\input|
% multiple instances have to be prevented manually:
%\iffalse
%This code needs to be before the `\ProvidesFile' directive
%which is defined at the beginning of this file.
%Therefore it is also placed there and commented out here.
%</package>
%<*discard>
%\fi
%    \begin{macrocode}
\ifdefined\childdocmain\endinput\fi
%    \end{macrocode}
%\iffalse
%</discard>
%<*package>
%\fi
%
% \macro{\ifchilddoc}
% \macro{\ifchilddocmanual}
% The conditional |\ifchilddoc| tells whether a
% child (true) or main (false) document is being compiled.
% The conditional |\ifchilddocmanual| tells whether
% the |\includeonly| mechanism is used (false) or
% the selection of child files must be performed manually (true).
% The definitions initialise to false:
%    \begin{macrocode}
\newif\ifchilddoc
\newif\ifchilddocmanual
%    \end{macrocode}

% \macro{\childdocname}
% \macro{\childdocjob}
% The macro |\childdocname| stores the name of the main document
% to be compiled. The macro |\childdocjob| stores the name of
% the document on which the \LaTeX{} compiler was originally invoked.
% The content of |\jobname| cannot be compared
% to filenames specified in the source due to different catcodes.
% The following code rescans |\jobname|, stores the result
% in |\childdocname| and saves a copy in |\childdocjob|:
%    \begin{macrocode}
\edef\childdocname{\scantokens\expandafter{\jobname\noexpand}}
\let\childdocjob\childdocname
%    \end{macrocode}

% \macro{\childdocdisable}
% The macro |\childdocdisable| prevents the main file
% from being processed more than once.
% At this stage, the main document command |\childdocmain|
% is assumed to be called once again where it should do nothing.
% Any subsequent call to it should prevent
% a secondary processing of the main document
% It overwrites the forwarding commands
% |\childdocof| and |\childdocforward|
% with empty macros to prevent further inclusions of the main document:
%    \begin{macrocode}
\newcommand{\childdocdisable}
{
  \renewcommand{\childdocmain}[1]{\renewcommand{\childdocmain}[1]{\endinput}}
  \renewcommand{\childdocof}[1]{}
  \renewcommand{\childdocby}[2][]{}
  \renewcommand{\childdocforward}[2][]{}
  \renewcommand{\childdocdisable}{}
}
%    \end{macrocode}

% \macro{\childdocmain}
% The macro |\childdocmain| is to be called at the top of the main file
% with nothing or the main filename (without extension) as argument.
% First, it breaks loops.
% If the argument is not empty and does not match |\childdocname|
% (which is set by the first inclusion of |childdoc.def|),
% |\ifchilddoc| is set to true, |\includeonly| is applied to the child file
% and |\jobname| is set to the main file
% (for proper handling of |.aux| files):
%    \begin{macrocode}
\newcommand{\childdocmain}[1]
{
  \childdocdisable\childdocmain{}
  \if?#1?\else
    \begingroup
      \def\childdoctmp{#1}
      \ifx\childdoctmp\childdocname
        \def\childdoctmp{}
      \else
        \def\childdoctmp
        {
          \childdoctrue
          \includeonly{\childdocname}
          \def\childdocjob{#1}
          \def\jobname{#1}
        }
      \fi
      \expandafter
    \endgroup
    \childdoctmp
  \fi
}
%    \end{macrocode}

% \macro{\childdocof}
% The command |\childdocof| redirects
% compilation to the main file |#1|.
%    \begin{macrocode}
\newcommand{\childdocof}[1]
{
  \childdocdisable
  \childdoctrue
  \includeonly{\childdocname}
  \def\jobname{#1}
  \def\childdocjob{#1}
  \input{#1}
}
%    \end{macrocode}

% \macro{\childdocby}
% The command |\childdocby| ....
%    \begin{macrocode}
\newcommand{\childdocby}[2][]
{
  \childdocdisable
  \childdoctrue
  \childdocmanualtrue
  \if?#1?\else
    \def\jobname{#2}
  \fi
  \def\childdocjob{#2}
  \input{#2}
  \endinput
}
%    \end{macrocode}

% \macro{\childdocforward}
% The command |\childdocforward| redirects
% compilation to the main file or
% (if the optional argument is given) a child file.
% Parameters are set as if the main file
% or a child file starting with |\childdocof| was compiled.
% Then compilation is handed over to the main file:
%    \begin{macrocode}
\newcommand{\childdocforward}[2][]
{
  \begingroup
    \if?#1?
      \def\childdoctmp
      {
        \def\childdocname{#2}
        \def\childdocjob{#2}
        \def\jobname{#2}
        \input{#2}
        \endinput
      }
    \else
      \def\childdoctmp
      {
        \childdocdisable
        \def\childdocname{#2}
        \childdoctrue
        \includeonly{#2}
        \def\childdocjob{#1}
        \def\jobname{#1}
        \input{#1}
        \endinput
      }
    \fi
    \expandafter
  \endgroup
  \childdoctmp
}
%    \end{macrocode}

% \macro{\childdocforwardprefix}
% The command |\childdocforwardprefix| redirects
% compilation to the main or a child file by means of a pattern.
% The prefix |#1| in the current filename is replaced by |#2|
% and the suffix of the current filename is kept
% (it is assumed that the filename does not contain the substring `|~~~|'
% which is used as a delimiter).
% Compilation is handed over to the new file by |\childdocforward|:
%    \begin{macrocode}
\newcommand{\childdocforwardprefix}[3][]
{
  \begingroup
    \def\childdocextract #2##1~~~{\def\childdoctmp{\childdocforward[#1]{#3##1}}}
    \expandafter\childdocextract\childdocname~~~
    \expandafter
  \endgroup
  \childdoctmp
}
%    \end{macrocode}

% \macro{\childdoc}
% The deprecated macro |\childdoc| is a legacy version of |\childdocmain|:
%    \begin{macrocode}
\newcommand{\childdoc}{\childdocmain}
%    \end{macrocode}

% \macro{\childdocredirect}
% The deprecated macro |\childdocredirect| is a legacy version
% of |\childdocforward| and |\childdocforwardprefix|:
%    \begin{macrocode}
\newcommand{\childdocredirect}[2][]
{
  \begingroup
    \if?#1?
      \def\childdoctmp{\childdocforward{#2}}
    \else
      \def\childdoctmp{\childdocforwardprefix{#1}{#2}}
    \fi
    \expandafter
  \endgroup
  \childdoctmp
}
%    \end{macrocode}

%\iffalse
%</package>
%\fi
%
\endinput
|\\
|\childdocby{|\textit{main}|}|\\
\end{tabular}
\end{center}
%
The directive |\childdocby| is similar to |\childdocof|
described in \secref{sec:include},
but the subsequent selection of content must be done manually.
To that end, both |\ifchilddoc| and |\ifchilddocmanual|
will be true upon processing of a part,
and the name of the part is stored in |\childdocname|.
Note that |\jobname| will be set to the filename of the current part
so that each part receives an individual |.aux| file
that does not interfere with the |.aux| file(s) of the main document.
This behaviour can be altered by the alternative form
|\childdocby[*]{|\textit{main}|}| (with a non-empty optional argument)
which uses the |.aux| file of the main document
by setting |\jobname| to \textit{main}.

%%%%%%%%%%%%%%%%%%%%%%%%%%%%%%%%%%%%%%%%%%%%%%%%%%%%%%%%%%%%%%%%%%%%%%%%%%%%%%%%
\subsection{Driver Development}
\label{sec:driver}

The \textsf{childdoc} mechanism can also be use for the development
of definition files such as \LaTeX{} styles or classes.
This case differs from the above setup with multiple parts
included by |\include| in that no |\includeonly| should be invoked.
This can be achieved by starting the include file
(before |\ProvidesPackage|) with:
%
\begin{center}
\begin{tabular}{l}
|% \iffalse
%
% childdoc.dtx Copyright (C) 2017-2018 Niklas Beisert
%
% This work may be distributed and/or modified under the
% conditions of the LaTeX Project Public License, either version 1.3
% of this license or (at your option) any later version.
% The latest version of this license is in
%   http://www.latex-project.org/lppl.txt
% and version 1.3 or later is part of all distributions of LaTeX
% version 2005/12/01 or later.
%
% This work has the LPPL maintenance status `maintained'.
%
% The Current Maintainer of this work is Niklas Beisert.
%
% This work consists of the files childdoc.dtx and childdoc.ins
% and the derived files childdoc.def and cdocsamp.tex with
% cdocsch1.tex, cdocsch2.tex, cdocsdrf.tex, cdocsfn1.tex, cdocsfn2.tex.
%
%<package>\ifdefined\childdocmain\endinput\fi
%<package>\ProvidesFile{childdoc.def}[2018/12/30 v2.0 child document driver]
%<samplemain>\ProvidesFile{cdocsamp.tex}[2018/12/30 v2.0 sample for childdoc]
%<*driver>
%\ProvidesFile{childdoc.drv}[2018/12/30 v2.0 childdoc reference manual file]
\PassOptionsToClass{10pt,a4paper}{article}
\documentclass{ltxdoc}

\usepackage[margin=35mm]{geometry}
\usepackage{hyperref}
\usepackage{hyperxmp}
\usepackage[usenames]{color}

\hypersetup{colorlinks=true}
\hypersetup{pdfstartview=FitH}
\hypersetup{pdfpagemode=UseNone}
\hypersetup{pdfsource={}}
\hypersetup{pdflang={en-UK}}
\hypersetup{pdfcopyright={Copyright 2017-2018 Niklas Beisert.
  This work may be distributed and/or modified under the
  conditions of the LaTeX Project Public License, either version 1.3
  of this license or (at your option) any later version.}}
\hypersetup{pdflicenseurl={http://www.latex-project.org/lppl.txt}}
\hypersetup{pdfcontactaddress={ETH Zurich, ITP, HIT K,
  Wolfgang-Pauli-Strasse 27}}
\hypersetup{pdfcontactpostcode={8093}}
\hypersetup{pdfcontactcity={Zurich}}
\hypersetup{pdfcontactcountry={Switzerland}}
\hypersetup{pdfcontactemail={nbeisert@itp.phys.ethz.ch}}
\hypersetup{pdfcontacturl={http://people.phys.ethz.ch/\xmptilde nbeisert/}}

\newcommand{\secref}[1]{\hyperref[#1]{section \ref*{#1}}}

\parskip1ex
\parindent0pt
\let\olditemize\itemize
\def\itemize{\olditemize\parskip0pt}

\begin{document}

\title{The \textsf{childdoc} Package}
\hypersetup{pdftitle={The childdoc Package}}
\author{Niklas Beisert\\[2ex]
  Institut f\"ur Theoretische Physik\\
  Eidgen\"ossische Technische Hochschule Z\"urich\\
  Wolfgang-Pauli-Strasse 27, 8093 Z\"urich, Switzerland\\[1ex]
  \href{mailto:nbeisert@itp.phys.ethz.ch}
  {\texttt{nbeisert@itp.phys.ethz.ch}}}
\hypersetup{pdfauthor={Niklas Beisert}}
\hypersetup{pdfsubject={Manual for the LaTeX2e Package childdoc}}
\date{30 December 2018, \textsf{v2.0}}
\maketitle

\begin{abstract}\noindent
\textsf{childdoc} is a \LaTeXe{} package
that enables the direct compilation
of document sections included by |\include|
to individual files.
\end{abstract}

\begingroup
\parskip0ex
\tableofcontents
\endgroup

%%%%%%%%%%%%%%%%%%%%%%%%%%%%%%%%%%%%%%%%%%%%%%%%%%%%%%%%%%%%%%%%%%%%%%%%%%%%%%%%
%%%%%%%%%%%%%%%%%%%%%%%%%%%%%%%%%%%%%%%%%%%%%%%%%%%%%%%%%%%%%%%%%%%%%%%%%%%%%%%%
\section{Introduction}

\LaTeX{} provides a mechanism to structure a large document (such as a book)
into a main file and several child files (containing the chapters)
using the |\include| command.
This mechanism is beneficial for documents
which span hundreds of pages in order to
make the source file(s) more manageable.
Moreover, compilation can be restricted to
selected child files by means of the |\includeonly| command.
The latter feature can be used to reduce the compilation time while editing
(this was significantly more useful in the earlier days of \LaTeX{})
or to generate a smaller document which is easier to navigate.
Another application of |\includeonly| is to generate
documents consisting of selected parts of the complete document.

However, there are a few drawbacks of the plain |\include| mechanism:
\begin{itemize}
\item
The child files cannot be compiled on their own,
they can only be compiled via the main file.
A naive editing environment
(such as a text editor with an option
to have the current file processed by \LaTeX)
may require one to switch to the main file before compiling;
attempting to compile the child file produces errors.
\item
The main file must be modified (each time)
to adjust the |\includeonly| command
to the present needs. This easily leaves the main file in a messy state.
\item
The generated document will always carry the filename
of the main document. This is inconvenient if
several child files are to be compiled and
to be kept for distribution.
\end{itemize}

The present package provides a simple interface
to make child files individually compilable by \LaTeX{}.
Compiling a child file then has the same effect as compiling
the main file with an |\includeonly| command
to select the appropriate child.
Moreover the generated document will carry the name of the child
rather than the main file.
This resolves all three above issues.

This feature is meant to make the editing of books,
thesis documents and lecture notes somewhat more convenient.
However, the package can also be used efficiently for
composing a series of documents (such as exercise sheets)
which are typically distributed individually.
It then assists the author in generating the individual documents
(potentially in different versions)
as well as a document containing the collected series.
Another application is in developing style files
or other kinds of included material
where compilation of the style file could redirect
to a sample or test file.

%%%%%%%%%%%%%%%%%%%%%%%%%%%%%%%%%%%%%%%%%%%%%%%%%%%%%%%%%%%%%%%%%%%%%%%%%%%%%%%%
%%%%%%%%%%%%%%%%%%%%%%%%%%%%%%%%%%%%%%%%%%%%%%%%%%%%%%%%%%%%%%%%%%%%%%%%%%%%%%%%
\section{Usage}

First of all, the package \textsf{childdoc} is \emph{not} a standard
\LaTeXe{} |.sty| style file! Therefore it needs to be invoked in
a non-standard way.

%%%%%%%%%%%%%%%%%%%%%%%%%%%%%%%%%%%%%%%%%%%%%%%%%%%%%%%%%%%%%%%%%%%%%%%%%%%%%%%%
\subsection{Included Files}
\label{sec:include}

%%%%%%%%%%%%%%%%%%%%%%%%%%%%%%%%%%%%%%%%
\DescribeMacro{\childdocmain}
To use the package, add the commands
\begin{center}
\begin{tabular}{l}
|\input{childdoc.def}|\\
|\childdocmain{}|\\
\end{tabular}
\end{center}
at the very top of the main \LaTeX{} file,
in particular \emph{before} the |\documentclass| statement!
The argument of |\childdocmain| should be left empty
(but it must be present).

%%%%%%%%%%%%%%%%%%%%%%%%%%%%%%%%%%%%%%%%
\DescribeMacro{\childdocof}
Furthermore, add the commands
\begin{center}
\begin{tabular}{l}
|\input{childdoc.def}|\\
|\childdocof{|\textit{main}|}|\\
\end{tabular}
\end{center}
at the top of every child file \textit{child}
which is included by |\include{|\textit{child}|}|
from within the main file
(or at least for those files to be compiled individually).
The argument \textit{main} must be the filename of the main file.

There are a couple of
considerations in setting up the main and child documents:

%%%%%%%%%%%%%%%%%%%%%%%%%%%%%%%%%%%%%%%%
\paragraph{Restrictions.}

Please note the following restrictions:
\begin{itemize}
\item
|\childdocmain| must be called with one argument \textit{main}
to ensure compatibility with earlier version of the package.
It must either be empty (|\childdocmain{}|)
or precisely match the filename of the main file in which it is specified.
See \secref{sec:detection} for further information.
\item
The filename \textit{main} must be specified without the |.tex| extension.
\item
The filename \textit{main} is case sensitive
(even in case-insensitive file systems)
due to internal string comparison.
\item
The argument \textit{main} should be fully expanded, it cannot be a macro.
\item
Subdirectories and special characters should be avoided in filenames.
\item
The command |\childdocmain{|\textit{main}|}| must be followed by a whitespace.
It should not be followed immediately by another command
or by a comment mark `|%|'.
This is because the \TeX{} parser reads the token immediately following
the argument of |\childdocmain| and puts it
at the beginning of every child section;
however, a white\-space is ignored.
\end{itemize}

%%%%%%%%%%%%%%%%%%%%%%%%%%%%%%%%%%%%%%%%
\paragraph{Content of Main File.}

It is advisable to place all content in the child files included by |\include|.
Any output contained in the main file will appear in all child documents
unless suppressed manually;
it cannot be suppressed automatically by the |\includeonly| directive
and thus should normally be avoided.
A method to include some content in the main file
by means of conditional processing is described in \secref{sec:conditional}.

%%%%%%%%%%%%%%%%%%%%%%%%%%%%%%%%%%%%%%%%
\paragraph{Page Numbering.}

When only a part of the document is compiled,
the appropriate numbering of pages
(as well as other status parameters)
is determined from the |.aux| files.
The latter contain information from previous passes.
However this information needs to propagate through
all intermediate child documents.
Therefore the page numbering in child documents may well
be inconsistent until the complete document is compiled at least once.

A useful (if unconventional) way to always ensure a consistent
page numbering is to restart the numbering in each child document
and denote the pages by `\textit{child}|.|\textit{page}'
where \textit{child} represents the chapter/section number of the child file.
This can be achieved by the command
|\numberwithin{page}{|\textit{child}|}|
of the \textsf{amsmath} package
where \textit{child} can be |chapter| or |section|
depending on the chosen structuring.
Alternatively, one can modify the macro |\thepage| appropriately
and reset the counter |page| at the start of each child file.

%%%%%%%%%%%%%%%%%%%%%%%%%%%%%%%%%%%%%%%%%%%%%%%%%%%%%%%%%%%%%%%%%%%%%%%%%%%%%%%%
\subsection{Conditional Processing}
\label{sec:conditional}

The package provides a mechanism to compile different versions
of a document. To customise the versions further some conditional processing
can come in handy to distinguish which version is being compiled.
The package provides two macros to describe the compilation context:

%%%%%%%%%%%%%%%%%%%%%%%%%%%%%%%%%%%%%%%%
\DescribeMacro{\ifchilddoc}
The conditional |\ifchilddoc| distinguishes between the compilation of
child documents and the main document:
%
\begin{center}
|\ifchilddoc |\textit{child-code}| |[|\||else |\textit{main-code}]| \||fi|
\end{center}

%%%%%%%%%%%%%%%%%%%%%%%%%%%%%%%%%%%%%%%%
\DescribeMacro{\childdocname}
\DescribeMacro{\childdocjob}
The macro |\childdocname| contains the filename (without extension)
of the main or child file being processed.
Note that |\childdocjob| will always contain the name of the main file.

%%%%%%%%%%%%%%%%%%%%%%%%%%%%%%%%%%%%%%%%
\paragraph{Title Page.}

Conditional processing can be used to include a title or banner page
in the main document when proper precautions are taken.
Importantly, the code in the main file should ensure that the page counter
(as well as other status parameters which are stored in the |.aux| files)
takes the same value after the conditional processing.
Otherwise the page numbers may take divergent values
depending on which part is compiled.

For example, a title page could be declared by:
%
\begin{center}
\begin{tabular}{l}
|\ifchilddoc\||else|\\
|\addtocounter{page}{-1}|\\
\textit{code for title page}\\
|\newpage|\\
|\||fi|
\end{tabular}
\end{center}
%
A banner page for the child documents can be generated by:
%
\begin{center}
\begin{tabular}{l}
|\ifchilddoc|\\
|\addtocounter{page}{-1}|\\
\textit{code for banner page}\\
|\newpage|\\
|\||fi|
\end{tabular}
\end{center}
%
Here one could write a message such as:
\begin{center}
|This is the part \childdocname{} of \childdocjob{}.|
\end{center}

%%%%%%%%%%%%%%%%%%%%%%%%%%%%%%%%%%%%%%%%%%%%%%%%%%%%%%%%%%%%%%%%%%%%%%%%%%%%%%%%
\subsection{Flags}
\label{sec:flags}

The package makes it easy to generate different versions
of the main or child documents.
To this end compilation flags can be defined
and assigned different default values.
They will be particularly useful in conjunction
with the forwarding mechanism described in \secref{sec:forward}.

For example, it may be useful to have a flag |\version|
which can be set to |draft| or |final|.
The document source will contain some conditional code
depending on the value of |\version|.
Suppose further, the flag should default to |final| for the main file
and to |draft| for child files
which is a natural assignment for editing the document.
This is achieved by placing the following code
in the preamble of the main document
(below the |\childdocmain| directive):
%
\begin{center}
\begin{tabular}{l}
|\ifchilddoc|\\
|\providecommand{\version}{draft}|\\
|\||else|\\
|\providecommand{\version}{final}|\\
|\||fi|
\end{tabular}
\end{center}
%
The definition by |\providecommand| makes sure
that previous definitions are not overwritten.
Further statements |\providecommand{\version}{...}|
can thus be added before the above code to override it.

For the main file, one might add a line
(between |\childdocmain| and the above block)
%
\begin{center}
|%\ifchilddoc\||else\providecommand{\version}{draft}\||fi|
\end{center}
%
which can be uncommented to produce a draft version.
Likewise one can add a line to the very top of a child file
(above the |\childdocof{|\textit{main}|}| directive)
%
\begin{center}
|%\providecommand{\version}{final}|
\end{center}
%
which can be uncommented to produce the final version of this child document.

%%%%%%%%%%%%%%%%%%%%%%%%%%%%%%%%%%%%%%%%%%%%%%%%%%%%%%%%%%%%%%%%%%%%%%%%%%%%%%%%
\subsection{Forwarding}
\label{sec:forward}

Different versions of the main or child documents
using compilation flags as described in \secref{sec:flags}
can be (permanently) stored in different files
for convenient compilation, viewing and distribution.
To this end, the package defines a command
to pass on compilation to a different file:

%%%%%%%%%%%%%%%%%%%%%%%%%%%%%%%%%%%%%%%%
\DescribeMacro{\childdocforward}
The command |\childdocforward| redirects processing to
another source file:
%
\begin{center}
\begin{tabular}{l}
|\input{childdoc.def}|\\
|\childdocforward[|\textit{main}|]{|\textit{dest}|}|\\
\end{tabular}
\end{center}
%
The argument \textit{dest} is the destination file
(without extension).
It should be the main file or one of the child files.
Note that further \textsf{childdoc} directives
such as |\childdocof| and |\childdocforward|
in the indicated file will be processed in this form.
The optional argument \textit{main}
passes on directly to the main file \textit{main}
while pretending to compile the child \textit{dest}.
This form behaves as if \textit{dest}
issues |\childdocof{|\textit{main}|}| right away,
and no further \textsf{childdoc} directives will be processed.

%%%%%%%%%%%%%%%%%%%%%%%%%%%%%%%%%%%%%%%%
\DescribeMacro{\...prefix}
In the alternative form |\childdocforwardprefix|,
%
\begin{center}
\begin{tabular}{l}
|\input{childdoc.def}|\\
|\childdocforwardprefix[|\textit{main}|]{|\textit{prefix}|}{|\textit{dest}|}|
\end{tabular}
\end{center}
%
the destination file is determined by a pattern
depending on the current file:
To make this work, the current file must be called
`{\textit{prefix}\hspace{0.2em}\textit{suffix}}'
with \textit{prefix} matching precisely the argument.
Processing is then passed on to the file
`{\textit{dest}\hspace{0.2em}\textit{suffix}}'.
Surely, the same effect is achieved by
directly specifying the
argument `{\textit{dest}\hspace{0.2em}\textit{suffix}}'
in the first form.
However, that requires to set up a different file
for each child. With the alternative form of the command
all these files can have exactly the same content
which simplifies setting them up and maintaining them.

For example, the following file |draft.tex|
with a compilation flag |\version| as described in \secref{sec:flags}
compiles the main document as a draft:
%
\begin{center}
\begin{tabular}{l}
|\def\version{draft}|\\
|\input{childdoc.def}|\\
|\childdocforward{|\textit{main}|}|
\end{tabular}
\end{center}
%
Likewise, the following files |final|\textit{nn}|.tex|
compile the final version of the child document
|child|\textit{nn}|.tex|:
%
\begin{center}
\begin{tabular}{l}
|\def\version{final}|\\
|\input{childdoc.def}|\\
|\childdocforwardprefix{final}{child}|
\end{tabular}
\end{center}
%

Note that when several versions of a main file and/or of each child file
are to be generated, it may be convenient to set up a |Makefile| or
shell script to automatise the process.

%%%%%%%%%%%%%%%%%%%%%%%%%%%%%%%%%%%%%%%%%%%%%%%%%%%%%%%%%%%%%%%%%%%%%%%%%%%%%%%%
\subsection{Command Line Processing}
\label{sec:commandline}

The effect of redirection files can also be achieved by invoking
the \LaTeX{} compiler with a more elaborate command line.
Most conveniently this should be done as part
of a shell script or a |Makefile|.

When using \textsf{childdoc} in the main file, the following
command lines effectively perform a redirection
(note that depending on the shell being used,
backslashes may have to be doubled: `|\|' $\to$ `|\\|'):
%
\begin{center}
|... -jobname "|\textit{target}|" |\\|"|[\textit{flags}]%
|\input{childdoc.def}\childdocforward[|\textit{main}|]{|\textit{dest}|}"|
\end{center}
%
Here \textit{target} is the name of the output file,
\textit{main} is the name of the main file
and \textit{dest} is the name of the main or child file to be processed
(all filenames without extensions).
The optional argument \textit{main} can be omitted
if \textit{main} matches \textit{dest}.
Optionally, compilation \textit{flags} can be defined via |\def| commands.
This command line makes the \TeX{} engine believe
it is compiling the file \textit{target}
whose content is specified as the latter parameter.
The provided code then forwards the processing to
\textit{main} or \textit{dest} as described in \secref{sec:forward}.

%%%%%%%%%%%%%%%%%%%%%%%%%%%%%%%%%%%%%%%%%%%%%%%%%%%%%%%%%%%%%%%%%%%%%%%%%%%%%%%%
\subsection{Include by Input}
\label{sec:input}

Including child documents by |\include| has some restrictions by design.
Most notably, the content of a child document always occupies
its own set of pages; pages cannot be shared between child documents.
Usually, this behaviour makes perfect sense
because each child document contain an essential part of the document.
However, in some situations it may be desirable to compose
a document from a collection of parts
without having mandatory page breaks between then.
For this case, the package
provides a mechanism to include parts
by |\input| which can also be processed individually.
However, by construction this mechanism
requires manual handling of the content to be output.

%%%%%%%%%%%%%%%%%%%%%%%%%%%%%%%%%%%%%%%%
\DescribeMacro{\ifchilddocmanual}
The main file should be prepared as usual, see \secref{sec:include}.
However, the document body must make a distinction
between processing of an individual part and of the main document, e.g.:
%
\begin{center}
\begin{tabular}{l}
|\ifchilddocmanual|\\
|\input{\childdocname}|\\
|\||else|\\
\textit{document body with }|\input{|\textit{part}|}|\\
|\||fi|
\end{tabular}
\end{center}
%
The conditional |\ifchilddocmanual| is true whenever
a part to be included by |\input| is being compiled,
and the name of the part is stored in |\childdocname|.

%%%%%%%%%%%%%%%%%%%%%%%%%%%%%%%%%%%%%%%%
\DescribeMacro{\childdocby}
Each part to be included by |\input| should start with:
%
\begin{center}
\begin{tabular}{l}
|\input{childdoc.def}|\\
|\childdocby{|\textit{main}|}|\\
\end{tabular}
\end{center}
%
The directive |\childdocby| is similar to |\childdocof|
described in \secref{sec:include},
but the subsequent selection of content must be done manually.
To that end, both |\ifchilddoc| and |\ifchilddocmanual|
will be true upon processing of a part,
and the name of the part is stored in |\childdocname|.
Note that |\jobname| will be set to the filename of the current part
so that each part receives an individual |.aux| file
that does not interfere with the |.aux| file(s) of the main document.
This behaviour can be altered by the alternative form
|\childdocby[*]{|\textit{main}|}| (with a non-empty optional argument)
which uses the |.aux| file of the main document
by setting |\jobname| to \textit{main}.

%%%%%%%%%%%%%%%%%%%%%%%%%%%%%%%%%%%%%%%%%%%%%%%%%%%%%%%%%%%%%%%%%%%%%%%%%%%%%%%%
\subsection{Driver Development}
\label{sec:driver}

The \textsf{childdoc} mechanism can also be use for the development
of definition files such as \LaTeX{} styles or classes.
This case differs from the above setup with multiple parts
included by |\include| in that no |\includeonly| should be invoked.
This can be achieved by starting the include file
(before |\ProvidesPackage|) with:
%
\begin{center}
\begin{tabular}{l}
|\input{childdoc.def}|\\
|\childdocforward{|\textit{main}|}|\\
\end{tabular}
\end{center}
%
or alternatively with:
%
\begin{center}
\begin{tabular}{l}
|\input{childdoc.def}|\\
|\childdocby{|\textit{main}|}|\\
\end{tabular}
\end{center}
%
Both forms have slightly different effects as described above.
The main file is prepared as usual, see \secref{sec:include}.

%%%%%%%%%%%%%%%%%%%%%%%%%%%%%%%%%%%%%%%%%%%%%%%%%%%%%%%%%%%%%%%%%%%%%%%%%%%%%%%%
\subsection{Legacy Detection}
\label{sec:detection}

The directive |\childdocmain| in the main file can detect
whether the complete document or merely a child is to be compiled
even without using the directive |\childdocof|.
This method is deprecated because it is less robust
and there is no compelling reason to use it;
it is merely provided for backward compatibility
and it may be removed in future versions.

If the detection mechanism is to be used,
it is mandatory to correctly specify
the filename of the main file as the argument of |\childdocmain|:
%
\begin{center}
\begin{tabular}{l}
|\input{childdoc.def}|\\
|\childdocmain{|\textit{main}|}|\\
\end{tabular}
\end{center}
%
If |\jobname| does not match the argument \textit{main} of |\childdocmain|,
it is assumed that |\jobname| points to the child file to be compiled.
When using |\childdocmain| with the main file specified as argument,
it suffices to start a child file
with just |\input{|\textit{main}|}|
without loading of the package and using |\childdocof|.
If instead all processing is done
with the appropriate \textsf{childdoc} directives,
the argument of \textit{main} of |\childdocmain| can be empty.

An alternative version of the command line processing described
in \secref{sec:commandline} using the detection mechanism reads:
%
\begin{center}
|... -jobname "|\textit{target}|" "|[\textit{flags}]%
[|\def\jobname{|\textit{dest}|}|]|\input{|\textit{main}|}"|
\end{center}

%%%%%%%%%%%%%%%%%%%%%%%%%%%%%%%%%%%%%%%%%%%%%%%%%%%%%%%%%%%%%%%%%%%%%%%%%%%%%%%%
\subsection{Manual Code}
\label{sec:manual}

In case one cannot be certain whether the definitions file |childdoc.def|
is installed on the target \TeX{} distribution
and one prefers not to ship it,
it is conceivable to paste a few relevant commands into the sources.

To that end, drop all statements |\input{childdoc.def}|
and perform the replacements as outlined below.
Instead of |\childdocmain{|\textit{main}|}| add the following code
to the top of the main file:
%
\begin{center}
\begin{tabular}{l}
|\||ifdefined\childdocname\endinput\||fi\newif\ifchilddoc|\\
|\edef\childdocname{\scantokens\expandafter{\jobname\noexpand}}|\\
|\def\childdocmain{|\textit{main}|}\||ifx\childdocmain\childdocname\||else|\\
|\childdoctrue\includeonly{\childdocname}\let\jobname\childdocmain\||fi|\\
\end{tabular}
\end{center}
%
Instead of |\childdocof{|\textit{main}|}| just include the main file
at the top of each child file:
%
\begin{center}
|\input{|\textit{main}|}|
\end{center}
%
A simple redirection |\childdocforward{|\textit{dest}|}| is achieved by:
%
\begin{center}
|\def\jobname{|\textit{dest}|}\input{\jobname}|
\end{center}
%
The redirection with prefix
|\childdocforwardprefix[|\textit{prefix}|]{|\textit{dest}|}|
is accomplished by:
%
\begin{center}
\begin{tabular}{l}
|{\edef\jobname{\scantokens\expandafter{\jobname\noexpand}}|\\
|\def\redirectjob |\textit{prefix}|#1~~~{\gdef\jobname{|\textit{dest}|#1}}|\\
|\expandafter\redirectjob\jobname~~~}\input{\jobname}|
\end{tabular}
\end{center}

In an alternative approach,
child documents can be compiled by a specific command line
without additional code or specific definitions:
%
\begin{center}
|... -jobname "|\textit{target}|" "|[\textit{flags}]%
|\includeonly{|\textit{dest}|}\input{|\textit{main}|}"|
\end{center}
%

%%%%%%%%%%%%%%%%%%%%%%%%%%%%%%%%%%%%%%%%%%%%%%%%%%%%%%%%%%%%%%%%%%%%%%%%%%%%%%%%
%%%%%%%%%%%%%%%%%%%%%%%%%%%%%%%%%%%%%%%%%%%%%%%%%%%%%%%%%%%%%%%%%%%%%%%%%%%%%%%%
\section{Information}

%%%%%%%%%%%%%%%%%%%%%%%%%%%%%%%%%%%%%%%%%%%%%%%%%%%%%%%%%%%%%%%%%%%%%%%%%%%%%%%%
\subsection{Copyright}

Copyright \copyright{} 2017--2018 Niklas Beisert

This work may be distributed and/or modified under the
conditions of the \LaTeX{} Project Public License, either version 1.3
of this license or (at your option) any later version.
The latest version of this license is in
  \url{http://www.latex-project.org/lppl.txt}
and version 1.3 or later is part of all distributions of \LaTeX{}
version 2005/12/01 or later.

This work has the LPPL maintenance status `maintained'.

The Current Maintainer of this work is Niklas Beisert.

This work consists of the files |README.txt|, |childdoc.ins| and |childdoc.dtx|
as well as the derived files |childdoc.def|, |cdocsamp.tex|
with |cdocsch1.tex|, |cdocsch2.tex|, |cdocspt3.tex|, |cdocspt4.tex|,
|cdocsdrf.tex|, |cdocsfn1.tex|, |cdocsfn2.tex|
as well as |childdoc.pdf|.

%%%%%%%%%%%%%%%%%%%%%%%%%%%%%%%%%%%%%%%%%%%%%%%%%%%%%%%%%%%%%%%%%%%%%%%%%%%%%%%%
\subsection{Files and Installation}

The package consists of the files:
%
\begin{center}
\begin{tabular}{ll}
    |README.txt|   & readme file \\
    |childdoc.ins| & installation file \\
    |childdoc.dtx| & source file \\
    |childdoc.def| & definition file \\
    |cdocsamp.tex| & sample main file \\
    |cdocsch1.tex| & sample include file \\
    |cdocsch2.tex| & sample include file \\
    |cdocspt3.tex| & sample part file \\
    |cdocspt4.tex| & sample part file \\
    |cdocsdrf.tex| & sample redirection file \\
    |cdocsfn1.tex| & sample redirection file \\
    |cdocsfn2.tex| & sample redirection file \\
    |childdoc.pdf| & manual
\end{tabular}
\end{center}
%
The distribution consists of the files
|README.txt|, |childdoc.ins| and |childdoc.dtx|.
%
\begin{itemize}
\item
Run (pdf)\LaTeX{} on |childdoc.dtx|
to compile the manual |childdoc.pdf| (this file).
\item
Run \LaTeX{} on |childdoc.ins| to create the definitions file |childdoc.def|
and the sample |cdocsamp.tex| with include files
|cdocsch1.tex|, |cdocsch2.tex|, |cdocspt3.tex|, |cdocspt4.tex|,
|cdocsdrf.tex|, |cdocsfn1.tex|, |cdocsfn2.tex|.
Then copy the file |childdoc.def| to an appropriate directory of your \LaTeX{}
distribution, e.g.\ \textit{texmf-root}|/tex/latex/childdoc|.
\end{itemize}

%%%%%%%%%%%%%%%%%%%%%%%%%%%%%%%%%%%%%%%%%%%%%%%%%%%%%%%%%%%%%%%%%%%%%%%%%%%%%%%%
\subsection{Related CTAN Packages}

There are several other packages which offer a similar functionality:
%
\begin{itemize}
\item
The packages
\href{http://ctan.org/pkg/docmute}{\textsf{docmute}},
\href{http://ctan.org/pkg/includex}{\textsf{includex}} and
\href{http://ctan.org/pkg/standalone}{\textsf{standalone}}
provide commands to include only the document body of
a child file thus allowing both files to be compiled individually.
\item
The packages \href{http://ctan.org/pkg/subdocs}{\textsf{subdocs}}
and \href{http://ctan.org/pkg/subfiles}{\textsf{subfiles}}
provide structures in which the main and child documents can be
encapsulated and allowing them to be compiled individually.
The inclusion mechanism is different from the conventional |\include|.
\item
The package \href{http://ctan.org/pkg/combine}{\textsf{combine}}
is an elaborate solution to combine several documents into one.
\end{itemize}
%
See also the CTAN topic \href{http://ctan.org/topic/subdocs}{\textsf{subdocs}}
for further related packages.
The present package differs from the above solutions in that
a document structure constructed with the conventional |\include| mechanism
just needs two extra commands at the top of every file
such that all constituent files can be compiled individually.

%%%%%%%%%%%%%%%%%%%%%%%%%%%%%%%%%%%%%%%%%%%%%%%%%%%%%%%%%%%%%%%%%%%%%%%%%%%%%%%%
%\subsection{Feature Suggestions}
%
%The following is a list of features which may be useful for future
%versions of this package:
%%
%\begin{itemize}
%\item
%\ldots
%\end{itemize}

%%%%%%%%%%%%%%%%%%%%%%%%%%%%%%%%%%%%%%%%%%%%%%%%%%%%%%%%%%%%%%%%%%%%%%%%%%%%%%%%
\subsection{Revision History}

%%%%%%%%%%%%%%%%%%%%%%%%%%%%%%%%%%%%%%%%
\paragraph{v2.0:} 2018/12/30

\begin{itemize}
\item
immediate forward processing
\item
added |\childdocby| mechanism
\item
manual restructured
\end{itemize}

%%%%%%%%%%%%%%%%%%%%%%%%%%%%%%%%%%%%%%%%
\paragraph{v1.6:} 2018/01/17

\begin{itemize}
\item
application for development of include files
\item
corrections to manual
\end{itemize}

%%%%%%%%%%%%%%%%%%%%%%%%%%%%%%%%%%%%%%%%
\paragraph{v1.5:} 2017/05/21

\begin{itemize}
\item
more complete structuring introduced
\item
|\childdocof| introduced
\item
|\childdoc| renamed to |\childdocmain|
\item
|\childredirect| renamed to |\childdocforward| and |\childdocforwardprefix|
and functionality expanded
\end{itemize}

%%%%%%%%%%%%%%%%%%%%%%%%%%%%%%%%%%%%%%%%
\paragraph{v1.0:} 2017/04/27

\begin{itemize}
\item
manual and install package
\item
first version published on CTAN
\end{itemize}

%%%%%%%%%%%%%%%%%%%%%%%%%%%%%%%%%%%%%%%%
\paragraph{v0.6:} 2017/04/26

\begin{itemize}
\item
redirection mechanism added
\end{itemize}

%%%%%%%%%%%%%%%%%%%%%%%%%%%%%%%%%%%%%%%%
\paragraph{v0.5:} 2017/04/26

\begin{itemize}
\item
functionality in definition file
\end{itemize}


%%%%%%%%%%%%%%%%%%%%%%%%%%%%%%%%%%%%%%%%%%%%%%%%%%%%%%%%%%%%%%%%%%%%%%%%%%%%%%%%
%%%%%%%%%%%%%%%%%%%%%%%%%%%%%%%%%%%%%%%%%%%%%%%%%%%%%%%%%%%%%%%%%%%%%%%%%%%%%%%%
%%%%%%%%%%%%%%%%%%%%%%%%%%%%%%%%%%%%%%%%%%%%%%%%%%%%%%%%%%%%%%%%%%%%%%%%%%%%%%%%
\appendix

\settowidth\MacroIndent{\rmfamily\scriptsize 000\ }

 \DocInput{childdoc.dtx}

\end{document}
%</driver>
% \fi
%
% %%%%%%%%%%%%%%%%%%%%%%%%%%%%%%%%%%%%%%%%%%%%%%%%%%%%%%%%%%%%%%%%%%%%%%%%%%%%%%
% %%%%%%%%%%%%%%%%%%%%%%%%%%%%%%%%%%%%%%%%%%%%%%%%%%%%%%%%%%%%%%%%%%%%%%%%%%%%%%
% \section{Sample}
%\iffalse
%<*samplemain>
%\fi
%
% The following presents a sample document
% with two chapters, two parts, a title page,
% a compile flag as well as three forwarding files to set the flag.
% It consists of eight |.tex| files:
% \begin{center}
% \begin{tabular}{ll}
% |cdocsamp.tex|&main file\\
% |cdocsch1.tex|&include file for chapter 1\\
% |cdocsch2.tex|&include file for chapter 2\\
% |cdocspt3.tex|&include file for part 3\\
% |cdocspt4.tex|&include file for part 4\\
% |cdocsdrf.tex|&forwarding file for main file in draft mode\\
% |cdocsfi1.tex|&forwarding file for final version of chapter 1\\
% |cdocsfi2.tex|&forwarding file for final version of chapter 2\\
% \end{tabular}
% \end{center}
% Each of the eight files can be compiled directly by the \LaTeX{} compiler.
%
% %%%%%%%%%%%%%%%%%%%%%%%%%%%%%%%%%%%%%%
% \paragraph{Main File.}
%
% The main file is called |cdocsamp.tex|.
%
% Load the \textsf{childdoc} definitions and
% declare the filename for the main document:
%    \begin{macrocode}
\input{childdoc.def}
\childdocmain{}
%    \end{macrocode}

% Optional override for |\version| flag:
%    \begin{macrocode}
%%\ifchilddoc\else\providecommand{\version}{draft}\fi
%    \end{macrocode}

% Define the default values for the |\version| flag
% (|final| for the main file and |draft| for childs):
%    \begin{macrocode}
\ifchilddoc
\providecommand{\version}{draft}
\else
\providecommand{\version}{final}
\fi
%    \end{macrocode}

% Load the standard document class:
%    \begin{macrocode}
\documentclass[12pt]{article}
%    \end{macrocode}

% Start the document body:
%    \begin{macrocode}
\begin{document}
%    \end{macrocode}

% Declare a title page.
% Print title, part of document being processed and version flag:
%    \begin{macrocode}
\addtocounter{page}{-1}
\begin{center}
{\LARGE\bfseries{}childdoc example\par}
\vspace{1cm}
\ifchilddoc
\ifchilddocmanual part\else chapter\fi:
`\childdocname' of `\childdocjob'\par
\else
main document: `\childdocjob'\par
\fi
version: \version\par
\end{center}
\newpage
%    \end{macrocode}

% Manually include selected file,
% otherwise process as usual:
%    \begin{macrocode}
\ifchilddocmanual
\section*{part `\childdocname'}
\input{\childdocname}
\else
%    \end{macrocode}

% Include the two chapters:
%    \begin{macrocode}
\include{cdocsch1}
\include{cdocsch2}
%    \end{macrocode}

% Include the two parts unless only chapters should be displayed:
%    \begin{macrocode}
\ifchilddoc\else
\section{part three}
\input{cdocspt3}
\section{part four}
\input{cdocspt4}
\fi
%    \end{macrocode}

% Process as usual until here:
%    \begin{macrocode}
\fi
%    \end{macrocode}

% End of document body:
%    \begin{macrocode}
\end{document}
%    \end{macrocode}
%\iffalse
%</samplemain>
%\fi
%
% %%%%%%%%%%%%%%%%%%%%%%%%%%%%%%%%%%%%%%
% \paragraph{Chapter Include Files.}
%
% The include files are called |cdocsch1.tex| and |cdocsch2.tex|.
%
%\iffalse
%<*samplechap1|samplechap2>
%\fi

% Optional override for |\version| flag:
%    \begin{macrocode}
%%\providecommand{\version}{final}
%    \end{macrocode}

% Include the main document:
%    \begin{macrocode}
\input{childdoc.def}
\childdocof{cdocsamp}
%    \end{macrocode}

%\iffalse
%</samplechap1|samplechap2>
%\fi
%
%\iffalse
%<*samplechap1>
%\fi
% Some text for chapter 1:
%    \begin{macrocode}
\section{one}
some text in chapter one
%    \end{macrocode}

%\iffalse
%</samplechap1>
%\fi
% Some text for chapter 2:
%\iffalse
%<*samplechap2>
%\fi
%    \begin{macrocode}
\section{two}
more text in chapter two
%    \end{macrocode}

%\iffalse
%</samplechap2>
%\fi
%
% %%%%%%%%%%%%%%%%%%%%%%%%%%%%%%%%%%%%%%
% \paragraph{Part Include Files.}
%
% The include files are called |cdocspt3.tex| and |cdocspt4.tex|.
%
%\iffalse
%<*samplepart3|samplepart4>
%\fi

% Optional override for |\version| flag:
%    \begin{macrocode}
%%\providecommand{\version}{final}
%    \end{macrocode}

% Include the main document:
%    \begin{macrocode}
\input{childdoc.def}
\childdocby{cdocsamp}
%    \end{macrocode}

%\iffalse
%</samplepart3|samplepart4>
%\fi
%
%\iffalse
%<*samplepart3>
%\fi
% Some text for part 3:
%    \begin{macrocode}
some text in part three
%    \end{macrocode}

%\iffalse
%</samplepart3>
%\fi
% Some text for part 4:
%\iffalse
%<*samplepart4>
%\fi
%    \begin{macrocode}
more text in part four
%    \end{macrocode}

%\iffalse
%</samplepart4>
%\fi
%
% %%%%%%%%%%%%%%%%%%%%%%%%%%%%%%%%%%%%%%
% \paragraph{Forwarding for a Complete Draft.}
%
% The following forwarding file |cdocsdrf.tex|
% compiles the main document in draft mode:
%\iffalse
%<*sampledraft>
%\fi
%    \begin{macrocode}
\def\version{draft}
\input{childdoc.def}
\childdocforward{cdocsamp}
%    \end{macrocode}

%\iffalse
%</sampledraft>
%\fi
%
% %%%%%%%%%%%%%%%%%%%%%%%%%%%%%%%%%%%%%%
% \paragraph{Forwarding for Final Version of the Chapters.}
%
% The following forwarding files |cdocsfn1.tex| and |cdocsfn2.tex|
% (with identical content)
% compile the final versions of the child documents
% |cdocsch1.tex| and |cdocsch2.tex|, respectively:
%\iffalse
%<*samplefinal>
%\fi
%    \begin{macrocode}
\def\version{final}
\input{childdoc.def}
\childdocforwardprefix[cdocsamp]{cdocsfn}{cdocsch}
%    \end{macrocode}

%\iffalse
%</samplefinal>
%\fi
%
% %%%%%%%%%%%%%%%%%%%%%%%%%%%%%%%%%%%%%%
% \paragraph{Command Line Processing.}
%
% The following three command lines generate the output files
% |cdocscld|, |cdocscl1| and |cdocscl2|
% which should be identical to
% |cdocsdrf|, |cdocsch1| and |cdocsfn2|, respectively:
% \begin{center}
% \begin{tabular}{l}
% |latex -jobname cdocscld \|\\
% |  "\def\version{draft}\input{childdoc.def}\childdocforward{cdocsamp}"|\\
% |latex -jobname cdocscl1 \|\\
% |  "\input{childdoc.def}\childdocforward[cdocsamp]{cdocsch1}"|\\
% |latex -jobname cdocscl2 \|\\
% |  "\def\version{final}\input{childdoc.def}\childdocforward{cdocsch2}"|
% \end{tabular}
% \end{center}
% Note that the trailing backslash on each first line
% merely continues the input to the second line
% (for convenient cut ant paste).
% Furthermore, the command |latex| can be replaced by any
% of its alternative versions such as |pdflatex|.
%
% %%%%%%%%%%%%%%%%%%%%%%%%%%%%%%%%%%%%%%%%%%%%%%%%%%%%%%%%%%%%%%%%%%%%%%%%%%%%%%
% %%%%%%%%%%%%%%%%%%%%%%%%%%%%%%%%%%%%%%%%%%%%%%%%%%%%%%%%%%%%%%%%%%%%%%%%%%%%%%
% \section{Implementation}
%\iffalse
%<*package>
%\fi
%
% This section describes the definitions file |childdoc.def|.

% The definitions cannot be loaded using |\usepackage| or |\RequirePackage|
% which has a mechanism to prevent loading a style file more than once.
% When loading the definitions by means of |\input|
% multiple instances have to be prevented manually:
%\iffalse
%This code needs to be before the `\ProvidesFile' directive
%which is defined at the beginning of this file.
%Therefore it is also placed there and commented out here.
%</package>
%<*discard>
%\fi
%    \begin{macrocode}
\ifdefined\childdocmain\endinput\fi
%    \end{macrocode}
%\iffalse
%</discard>
%<*package>
%\fi
%
% \macro{\ifchilddoc}
% \macro{\ifchilddocmanual}
% The conditional |\ifchilddoc| tells whether a
% child (true) or main (false) document is being compiled.
% The conditional |\ifchilddocmanual| tells whether
% the |\includeonly| mechanism is used (false) or
% the selection of child files must be performed manually (true).
% The definitions initialise to false:
%    \begin{macrocode}
\newif\ifchilddoc
\newif\ifchilddocmanual
%    \end{macrocode}

% \macro{\childdocname}
% \macro{\childdocjob}
% The macro |\childdocname| stores the name of the main document
% to be compiled. The macro |\childdocjob| stores the name of
% the document on which the \LaTeX{} compiler was originally invoked.
% The content of |\jobname| cannot be compared
% to filenames specified in the source due to different catcodes.
% The following code rescans |\jobname|, stores the result
% in |\childdocname| and saves a copy in |\childdocjob|:
%    \begin{macrocode}
\edef\childdocname{\scantokens\expandafter{\jobname\noexpand}}
\let\childdocjob\childdocname
%    \end{macrocode}

% \macro{\childdocdisable}
% The macro |\childdocdisable| prevents the main file
% from being processed more than once.
% At this stage, the main document command |\childdocmain|
% is assumed to be called once again where it should do nothing.
% Any subsequent call to it should prevent
% a secondary processing of the main document
% It overwrites the forwarding commands
% |\childdocof| and |\childdocforward|
% with empty macros to prevent further inclusions of the main document:
%    \begin{macrocode}
\newcommand{\childdocdisable}
{
  \renewcommand{\childdocmain}[1]{\renewcommand{\childdocmain}[1]{\endinput}}
  \renewcommand{\childdocof}[1]{}
  \renewcommand{\childdocby}[2][]{}
  \renewcommand{\childdocforward}[2][]{}
  \renewcommand{\childdocdisable}{}
}
%    \end{macrocode}

% \macro{\childdocmain}
% The macro |\childdocmain| is to be called at the top of the main file
% with nothing or the main filename (without extension) as argument.
% First, it breaks loops.
% If the argument is not empty and does not match |\childdocname|
% (which is set by the first inclusion of |childdoc.def|),
% |\ifchilddoc| is set to true, |\includeonly| is applied to the child file
% and |\jobname| is set to the main file
% (for proper handling of |.aux| files):
%    \begin{macrocode}
\newcommand{\childdocmain}[1]
{
  \childdocdisable\childdocmain{}
  \if?#1?\else
    \begingroup
      \def\childdoctmp{#1}
      \ifx\childdoctmp\childdocname
        \def\childdoctmp{}
      \else
        \def\childdoctmp
        {
          \childdoctrue
          \includeonly{\childdocname}
          \def\childdocjob{#1}
          \def\jobname{#1}
        }
      \fi
      \expandafter
    \endgroup
    \childdoctmp
  \fi
}
%    \end{macrocode}

% \macro{\childdocof}
% The command |\childdocof| redirects
% compilation to the main file |#1|.
%    \begin{macrocode}
\newcommand{\childdocof}[1]
{
  \childdocdisable
  \childdoctrue
  \includeonly{\childdocname}
  \def\jobname{#1}
  \def\childdocjob{#1}
  \input{#1}
}
%    \end{macrocode}

% \macro{\childdocby}
% The command |\childdocby| ....
%    \begin{macrocode}
\newcommand{\childdocby}[2][]
{
  \childdocdisable
  \childdoctrue
  \childdocmanualtrue
  \if?#1?\else
    \def\jobname{#2}
  \fi
  \def\childdocjob{#2}
  \input{#2}
  \endinput
}
%    \end{macrocode}

% \macro{\childdocforward}
% The command |\childdocforward| redirects
% compilation to the main file or
% (if the optional argument is given) a child file.
% Parameters are set as if the main file
% or a child file starting with |\childdocof| was compiled.
% Then compilation is handed over to the main file:
%    \begin{macrocode}
\newcommand{\childdocforward}[2][]
{
  \begingroup
    \if?#1?
      \def\childdoctmp
      {
        \def\childdocname{#2}
        \def\childdocjob{#2}
        \def\jobname{#2}
        \input{#2}
        \endinput
      }
    \else
      \def\childdoctmp
      {
        \childdocdisable
        \def\childdocname{#2}
        \childdoctrue
        \includeonly{#2}
        \def\childdocjob{#1}
        \def\jobname{#1}
        \input{#1}
        \endinput
      }
    \fi
    \expandafter
  \endgroup
  \childdoctmp
}
%    \end{macrocode}

% \macro{\childdocforwardprefix}
% The command |\childdocforwardprefix| redirects
% compilation to the main or a child file by means of a pattern.
% The prefix |#1| in the current filename is replaced by |#2|
% and the suffix of the current filename is kept
% (it is assumed that the filename does not contain the substring `|~~~|'
% which is used as a delimiter).
% Compilation is handed over to the new file by |\childdocforward|:
%    \begin{macrocode}
\newcommand{\childdocforwardprefix}[3][]
{
  \begingroup
    \def\childdocextract #2##1~~~{\def\childdoctmp{\childdocforward[#1]{#3##1}}}
    \expandafter\childdocextract\childdocname~~~
    \expandafter
  \endgroup
  \childdoctmp
}
%    \end{macrocode}

% \macro{\childdoc}
% The deprecated macro |\childdoc| is a legacy version of |\childdocmain|:
%    \begin{macrocode}
\newcommand{\childdoc}{\childdocmain}
%    \end{macrocode}

% \macro{\childdocredirect}
% The deprecated macro |\childdocredirect| is a legacy version
% of |\childdocforward| and |\childdocforwardprefix|:
%    \begin{macrocode}
\newcommand{\childdocredirect}[2][]
{
  \begingroup
    \if?#1?
      \def\childdoctmp{\childdocforward{#2}}
    \else
      \def\childdoctmp{\childdocforwardprefix{#1}{#2}}
    \fi
    \expandafter
  \endgroup
  \childdoctmp
}
%    \end{macrocode}

%\iffalse
%</package>
%\fi
%
\endinput
|\\
|\childdocforward{|\textit{main}|}|\\
\end{tabular}
\end{center}
%
or alternatively with:
%
\begin{center}
\begin{tabular}{l}
|% \iffalse
%
% childdoc.dtx Copyright (C) 2017-2018 Niklas Beisert
%
% This work may be distributed and/or modified under the
% conditions of the LaTeX Project Public License, either version 1.3
% of this license or (at your option) any later version.
% The latest version of this license is in
%   http://www.latex-project.org/lppl.txt
% and version 1.3 or later is part of all distributions of LaTeX
% version 2005/12/01 or later.
%
% This work has the LPPL maintenance status `maintained'.
%
% The Current Maintainer of this work is Niklas Beisert.
%
% This work consists of the files childdoc.dtx and childdoc.ins
% and the derived files childdoc.def and cdocsamp.tex with
% cdocsch1.tex, cdocsch2.tex, cdocsdrf.tex, cdocsfn1.tex, cdocsfn2.tex.
%
%<package>\ifdefined\childdocmain\endinput\fi
%<package>\ProvidesFile{childdoc.def}[2018/12/30 v2.0 child document driver]
%<samplemain>\ProvidesFile{cdocsamp.tex}[2018/12/30 v2.0 sample for childdoc]
%<*driver>
%\ProvidesFile{childdoc.drv}[2018/12/30 v2.0 childdoc reference manual file]
\PassOptionsToClass{10pt,a4paper}{article}
\documentclass{ltxdoc}

\usepackage[margin=35mm]{geometry}
\usepackage{hyperref}
\usepackage{hyperxmp}
\usepackage[usenames]{color}

\hypersetup{colorlinks=true}
\hypersetup{pdfstartview=FitH}
\hypersetup{pdfpagemode=UseNone}
\hypersetup{pdfsource={}}
\hypersetup{pdflang={en-UK}}
\hypersetup{pdfcopyright={Copyright 2017-2018 Niklas Beisert.
  This work may be distributed and/or modified under the
  conditions of the LaTeX Project Public License, either version 1.3
  of this license or (at your option) any later version.}}
\hypersetup{pdflicenseurl={http://www.latex-project.org/lppl.txt}}
\hypersetup{pdfcontactaddress={ETH Zurich, ITP, HIT K,
  Wolfgang-Pauli-Strasse 27}}
\hypersetup{pdfcontactpostcode={8093}}
\hypersetup{pdfcontactcity={Zurich}}
\hypersetup{pdfcontactcountry={Switzerland}}
\hypersetup{pdfcontactemail={nbeisert@itp.phys.ethz.ch}}
\hypersetup{pdfcontacturl={http://people.phys.ethz.ch/\xmptilde nbeisert/}}

\newcommand{\secref}[1]{\hyperref[#1]{section \ref*{#1}}}

\parskip1ex
\parindent0pt
\let\olditemize\itemize
\def\itemize{\olditemize\parskip0pt}

\begin{document}

\title{The \textsf{childdoc} Package}
\hypersetup{pdftitle={The childdoc Package}}
\author{Niklas Beisert\\[2ex]
  Institut f\"ur Theoretische Physik\\
  Eidgen\"ossische Technische Hochschule Z\"urich\\
  Wolfgang-Pauli-Strasse 27, 8093 Z\"urich, Switzerland\\[1ex]
  \href{mailto:nbeisert@itp.phys.ethz.ch}
  {\texttt{nbeisert@itp.phys.ethz.ch}}}
\hypersetup{pdfauthor={Niklas Beisert}}
\hypersetup{pdfsubject={Manual for the LaTeX2e Package childdoc}}
\date{30 December 2018, \textsf{v2.0}}
\maketitle

\begin{abstract}\noindent
\textsf{childdoc} is a \LaTeXe{} package
that enables the direct compilation
of document sections included by |\include|
to individual files.
\end{abstract}

\begingroup
\parskip0ex
\tableofcontents
\endgroup

%%%%%%%%%%%%%%%%%%%%%%%%%%%%%%%%%%%%%%%%%%%%%%%%%%%%%%%%%%%%%%%%%%%%%%%%%%%%%%%%
%%%%%%%%%%%%%%%%%%%%%%%%%%%%%%%%%%%%%%%%%%%%%%%%%%%%%%%%%%%%%%%%%%%%%%%%%%%%%%%%
\section{Introduction}

\LaTeX{} provides a mechanism to structure a large document (such as a book)
into a main file and several child files (containing the chapters)
using the |\include| command.
This mechanism is beneficial for documents
which span hundreds of pages in order to
make the source file(s) more manageable.
Moreover, compilation can be restricted to
selected child files by means of the |\includeonly| command.
The latter feature can be used to reduce the compilation time while editing
(this was significantly more useful in the earlier days of \LaTeX{})
or to generate a smaller document which is easier to navigate.
Another application of |\includeonly| is to generate
documents consisting of selected parts of the complete document.

However, there are a few drawbacks of the plain |\include| mechanism:
\begin{itemize}
\item
The child files cannot be compiled on their own,
they can only be compiled via the main file.
A naive editing environment
(such as a text editor with an option
to have the current file processed by \LaTeX)
may require one to switch to the main file before compiling;
attempting to compile the child file produces errors.
\item
The main file must be modified (each time)
to adjust the |\includeonly| command
to the present needs. This easily leaves the main file in a messy state.
\item
The generated document will always carry the filename
of the main document. This is inconvenient if
several child files are to be compiled and
to be kept for distribution.
\end{itemize}

The present package provides a simple interface
to make child files individually compilable by \LaTeX{}.
Compiling a child file then has the same effect as compiling
the main file with an |\includeonly| command
to select the appropriate child.
Moreover the generated document will carry the name of the child
rather than the main file.
This resolves all three above issues.

This feature is meant to make the editing of books,
thesis documents and lecture notes somewhat more convenient.
However, the package can also be used efficiently for
composing a series of documents (such as exercise sheets)
which are typically distributed individually.
It then assists the author in generating the individual documents
(potentially in different versions)
as well as a document containing the collected series.
Another application is in developing style files
or other kinds of included material
where compilation of the style file could redirect
to a sample or test file.

%%%%%%%%%%%%%%%%%%%%%%%%%%%%%%%%%%%%%%%%%%%%%%%%%%%%%%%%%%%%%%%%%%%%%%%%%%%%%%%%
%%%%%%%%%%%%%%%%%%%%%%%%%%%%%%%%%%%%%%%%%%%%%%%%%%%%%%%%%%%%%%%%%%%%%%%%%%%%%%%%
\section{Usage}

First of all, the package \textsf{childdoc} is \emph{not} a standard
\LaTeXe{} |.sty| style file! Therefore it needs to be invoked in
a non-standard way.

%%%%%%%%%%%%%%%%%%%%%%%%%%%%%%%%%%%%%%%%%%%%%%%%%%%%%%%%%%%%%%%%%%%%%%%%%%%%%%%%
\subsection{Included Files}
\label{sec:include}

%%%%%%%%%%%%%%%%%%%%%%%%%%%%%%%%%%%%%%%%
\DescribeMacro{\childdocmain}
To use the package, add the commands
\begin{center}
\begin{tabular}{l}
|\input{childdoc.def}|\\
|\childdocmain{}|\\
\end{tabular}
\end{center}
at the very top of the main \LaTeX{} file,
in particular \emph{before} the |\documentclass| statement!
The argument of |\childdocmain| should be left empty
(but it must be present).

%%%%%%%%%%%%%%%%%%%%%%%%%%%%%%%%%%%%%%%%
\DescribeMacro{\childdocof}
Furthermore, add the commands
\begin{center}
\begin{tabular}{l}
|\input{childdoc.def}|\\
|\childdocof{|\textit{main}|}|\\
\end{tabular}
\end{center}
at the top of every child file \textit{child}
which is included by |\include{|\textit{child}|}|
from within the main file
(or at least for those files to be compiled individually).
The argument \textit{main} must be the filename of the main file.

There are a couple of
considerations in setting up the main and child documents:

%%%%%%%%%%%%%%%%%%%%%%%%%%%%%%%%%%%%%%%%
\paragraph{Restrictions.}

Please note the following restrictions:
\begin{itemize}
\item
|\childdocmain| must be called with one argument \textit{main}
to ensure compatibility with earlier version of the package.
It must either be empty (|\childdocmain{}|)
or precisely match the filename of the main file in which it is specified.
See \secref{sec:detection} for further information.
\item
The filename \textit{main} must be specified without the |.tex| extension.
\item
The filename \textit{main} is case sensitive
(even in case-insensitive file systems)
due to internal string comparison.
\item
The argument \textit{main} should be fully expanded, it cannot be a macro.
\item
Subdirectories and special characters should be avoided in filenames.
\item
The command |\childdocmain{|\textit{main}|}| must be followed by a whitespace.
It should not be followed immediately by another command
or by a comment mark `|%|'.
This is because the \TeX{} parser reads the token immediately following
the argument of |\childdocmain| and puts it
at the beginning of every child section;
however, a white\-space is ignored.
\end{itemize}

%%%%%%%%%%%%%%%%%%%%%%%%%%%%%%%%%%%%%%%%
\paragraph{Content of Main File.}

It is advisable to place all content in the child files included by |\include|.
Any output contained in the main file will appear in all child documents
unless suppressed manually;
it cannot be suppressed automatically by the |\includeonly| directive
and thus should normally be avoided.
A method to include some content in the main file
by means of conditional processing is described in \secref{sec:conditional}.

%%%%%%%%%%%%%%%%%%%%%%%%%%%%%%%%%%%%%%%%
\paragraph{Page Numbering.}

When only a part of the document is compiled,
the appropriate numbering of pages
(as well as other status parameters)
is determined from the |.aux| files.
The latter contain information from previous passes.
However this information needs to propagate through
all intermediate child documents.
Therefore the page numbering in child documents may well
be inconsistent until the complete document is compiled at least once.

A useful (if unconventional) way to always ensure a consistent
page numbering is to restart the numbering in each child document
and denote the pages by `\textit{child}|.|\textit{page}'
where \textit{child} represents the chapter/section number of the child file.
This can be achieved by the command
|\numberwithin{page}{|\textit{child}|}|
of the \textsf{amsmath} package
where \textit{child} can be |chapter| or |section|
depending on the chosen structuring.
Alternatively, one can modify the macro |\thepage| appropriately
and reset the counter |page| at the start of each child file.

%%%%%%%%%%%%%%%%%%%%%%%%%%%%%%%%%%%%%%%%%%%%%%%%%%%%%%%%%%%%%%%%%%%%%%%%%%%%%%%%
\subsection{Conditional Processing}
\label{sec:conditional}

The package provides a mechanism to compile different versions
of a document. To customise the versions further some conditional processing
can come in handy to distinguish which version is being compiled.
The package provides two macros to describe the compilation context:

%%%%%%%%%%%%%%%%%%%%%%%%%%%%%%%%%%%%%%%%
\DescribeMacro{\ifchilddoc}
The conditional |\ifchilddoc| distinguishes between the compilation of
child documents and the main document:
%
\begin{center}
|\ifchilddoc |\textit{child-code}| |[|\||else |\textit{main-code}]| \||fi|
\end{center}

%%%%%%%%%%%%%%%%%%%%%%%%%%%%%%%%%%%%%%%%
\DescribeMacro{\childdocname}
\DescribeMacro{\childdocjob}
The macro |\childdocname| contains the filename (without extension)
of the main or child file being processed.
Note that |\childdocjob| will always contain the name of the main file.

%%%%%%%%%%%%%%%%%%%%%%%%%%%%%%%%%%%%%%%%
\paragraph{Title Page.}

Conditional processing can be used to include a title or banner page
in the main document when proper precautions are taken.
Importantly, the code in the main file should ensure that the page counter
(as well as other status parameters which are stored in the |.aux| files)
takes the same value after the conditional processing.
Otherwise the page numbers may take divergent values
depending on which part is compiled.

For example, a title page could be declared by:
%
\begin{center}
\begin{tabular}{l}
|\ifchilddoc\||else|\\
|\addtocounter{page}{-1}|\\
\textit{code for title page}\\
|\newpage|\\
|\||fi|
\end{tabular}
\end{center}
%
A banner page for the child documents can be generated by:
%
\begin{center}
\begin{tabular}{l}
|\ifchilddoc|\\
|\addtocounter{page}{-1}|\\
\textit{code for banner page}\\
|\newpage|\\
|\||fi|
\end{tabular}
\end{center}
%
Here one could write a message such as:
\begin{center}
|This is the part \childdocname{} of \childdocjob{}.|
\end{center}

%%%%%%%%%%%%%%%%%%%%%%%%%%%%%%%%%%%%%%%%%%%%%%%%%%%%%%%%%%%%%%%%%%%%%%%%%%%%%%%%
\subsection{Flags}
\label{sec:flags}

The package makes it easy to generate different versions
of the main or child documents.
To this end compilation flags can be defined
and assigned different default values.
They will be particularly useful in conjunction
with the forwarding mechanism described in \secref{sec:forward}.

For example, it may be useful to have a flag |\version|
which can be set to |draft| or |final|.
The document source will contain some conditional code
depending on the value of |\version|.
Suppose further, the flag should default to |final| for the main file
and to |draft| for child files
which is a natural assignment for editing the document.
This is achieved by placing the following code
in the preamble of the main document
(below the |\childdocmain| directive):
%
\begin{center}
\begin{tabular}{l}
|\ifchilddoc|\\
|\providecommand{\version}{draft}|\\
|\||else|\\
|\providecommand{\version}{final}|\\
|\||fi|
\end{tabular}
\end{center}
%
The definition by |\providecommand| makes sure
that previous definitions are not overwritten.
Further statements |\providecommand{\version}{...}|
can thus be added before the above code to override it.

For the main file, one might add a line
(between |\childdocmain| and the above block)
%
\begin{center}
|%\ifchilddoc\||else\providecommand{\version}{draft}\||fi|
\end{center}
%
which can be uncommented to produce a draft version.
Likewise one can add a line to the very top of a child file
(above the |\childdocof{|\textit{main}|}| directive)
%
\begin{center}
|%\providecommand{\version}{final}|
\end{center}
%
which can be uncommented to produce the final version of this child document.

%%%%%%%%%%%%%%%%%%%%%%%%%%%%%%%%%%%%%%%%%%%%%%%%%%%%%%%%%%%%%%%%%%%%%%%%%%%%%%%%
\subsection{Forwarding}
\label{sec:forward}

Different versions of the main or child documents
using compilation flags as described in \secref{sec:flags}
can be (permanently) stored in different files
for convenient compilation, viewing and distribution.
To this end, the package defines a command
to pass on compilation to a different file:

%%%%%%%%%%%%%%%%%%%%%%%%%%%%%%%%%%%%%%%%
\DescribeMacro{\childdocforward}
The command |\childdocforward| redirects processing to
another source file:
%
\begin{center}
\begin{tabular}{l}
|\input{childdoc.def}|\\
|\childdocforward[|\textit{main}|]{|\textit{dest}|}|\\
\end{tabular}
\end{center}
%
The argument \textit{dest} is the destination file
(without extension).
It should be the main file or one of the child files.
Note that further \textsf{childdoc} directives
such as |\childdocof| and |\childdocforward|
in the indicated file will be processed in this form.
The optional argument \textit{main}
passes on directly to the main file \textit{main}
while pretending to compile the child \textit{dest}.
This form behaves as if \textit{dest}
issues |\childdocof{|\textit{main}|}| right away,
and no further \textsf{childdoc} directives will be processed.

%%%%%%%%%%%%%%%%%%%%%%%%%%%%%%%%%%%%%%%%
\DescribeMacro{\...prefix}
In the alternative form |\childdocforwardprefix|,
%
\begin{center}
\begin{tabular}{l}
|\input{childdoc.def}|\\
|\childdocforwardprefix[|\textit{main}|]{|\textit{prefix}|}{|\textit{dest}|}|
\end{tabular}
\end{center}
%
the destination file is determined by a pattern
depending on the current file:
To make this work, the current file must be called
`{\textit{prefix}\hspace{0.2em}\textit{suffix}}'
with \textit{prefix} matching precisely the argument.
Processing is then passed on to the file
`{\textit{dest}\hspace{0.2em}\textit{suffix}}'.
Surely, the same effect is achieved by
directly specifying the
argument `{\textit{dest}\hspace{0.2em}\textit{suffix}}'
in the first form.
However, that requires to set up a different file
for each child. With the alternative form of the command
all these files can have exactly the same content
which simplifies setting them up and maintaining them.

For example, the following file |draft.tex|
with a compilation flag |\version| as described in \secref{sec:flags}
compiles the main document as a draft:
%
\begin{center}
\begin{tabular}{l}
|\def\version{draft}|\\
|\input{childdoc.def}|\\
|\childdocforward{|\textit{main}|}|
\end{tabular}
\end{center}
%
Likewise, the following files |final|\textit{nn}|.tex|
compile the final version of the child document
|child|\textit{nn}|.tex|:
%
\begin{center}
\begin{tabular}{l}
|\def\version{final}|\\
|\input{childdoc.def}|\\
|\childdocforwardprefix{final}{child}|
\end{tabular}
\end{center}
%

Note that when several versions of a main file and/or of each child file
are to be generated, it may be convenient to set up a |Makefile| or
shell script to automatise the process.

%%%%%%%%%%%%%%%%%%%%%%%%%%%%%%%%%%%%%%%%%%%%%%%%%%%%%%%%%%%%%%%%%%%%%%%%%%%%%%%%
\subsection{Command Line Processing}
\label{sec:commandline}

The effect of redirection files can also be achieved by invoking
the \LaTeX{} compiler with a more elaborate command line.
Most conveniently this should be done as part
of a shell script or a |Makefile|.

When using \textsf{childdoc} in the main file, the following
command lines effectively perform a redirection
(note that depending on the shell being used,
backslashes may have to be doubled: `|\|' $\to$ `|\\|'):
%
\begin{center}
|... -jobname "|\textit{target}|" |\\|"|[\textit{flags}]%
|\input{childdoc.def}\childdocforward[|\textit{main}|]{|\textit{dest}|}"|
\end{center}
%
Here \textit{target} is the name of the output file,
\textit{main} is the name of the main file
and \textit{dest} is the name of the main or child file to be processed
(all filenames without extensions).
The optional argument \textit{main} can be omitted
if \textit{main} matches \textit{dest}.
Optionally, compilation \textit{flags} can be defined via |\def| commands.
This command line makes the \TeX{} engine believe
it is compiling the file \textit{target}
whose content is specified as the latter parameter.
The provided code then forwards the processing to
\textit{main} or \textit{dest} as described in \secref{sec:forward}.

%%%%%%%%%%%%%%%%%%%%%%%%%%%%%%%%%%%%%%%%%%%%%%%%%%%%%%%%%%%%%%%%%%%%%%%%%%%%%%%%
\subsection{Include by Input}
\label{sec:input}

Including child documents by |\include| has some restrictions by design.
Most notably, the content of a child document always occupies
its own set of pages; pages cannot be shared between child documents.
Usually, this behaviour makes perfect sense
because each child document contain an essential part of the document.
However, in some situations it may be desirable to compose
a document from a collection of parts
without having mandatory page breaks between then.
For this case, the package
provides a mechanism to include parts
by |\input| which can also be processed individually.
However, by construction this mechanism
requires manual handling of the content to be output.

%%%%%%%%%%%%%%%%%%%%%%%%%%%%%%%%%%%%%%%%
\DescribeMacro{\ifchilddocmanual}
The main file should be prepared as usual, see \secref{sec:include}.
However, the document body must make a distinction
between processing of an individual part and of the main document, e.g.:
%
\begin{center}
\begin{tabular}{l}
|\ifchilddocmanual|\\
|\input{\childdocname}|\\
|\||else|\\
\textit{document body with }|\input{|\textit{part}|}|\\
|\||fi|
\end{tabular}
\end{center}
%
The conditional |\ifchilddocmanual| is true whenever
a part to be included by |\input| is being compiled,
and the name of the part is stored in |\childdocname|.

%%%%%%%%%%%%%%%%%%%%%%%%%%%%%%%%%%%%%%%%
\DescribeMacro{\childdocby}
Each part to be included by |\input| should start with:
%
\begin{center}
\begin{tabular}{l}
|\input{childdoc.def}|\\
|\childdocby{|\textit{main}|}|\\
\end{tabular}
\end{center}
%
The directive |\childdocby| is similar to |\childdocof|
described in \secref{sec:include},
but the subsequent selection of content must be done manually.
To that end, both |\ifchilddoc| and |\ifchilddocmanual|
will be true upon processing of a part,
and the name of the part is stored in |\childdocname|.
Note that |\jobname| will be set to the filename of the current part
so that each part receives an individual |.aux| file
that does not interfere with the |.aux| file(s) of the main document.
This behaviour can be altered by the alternative form
|\childdocby[*]{|\textit{main}|}| (with a non-empty optional argument)
which uses the |.aux| file of the main document
by setting |\jobname| to \textit{main}.

%%%%%%%%%%%%%%%%%%%%%%%%%%%%%%%%%%%%%%%%%%%%%%%%%%%%%%%%%%%%%%%%%%%%%%%%%%%%%%%%
\subsection{Driver Development}
\label{sec:driver}

The \textsf{childdoc} mechanism can also be use for the development
of definition files such as \LaTeX{} styles or classes.
This case differs from the above setup with multiple parts
included by |\include| in that no |\includeonly| should be invoked.
This can be achieved by starting the include file
(before |\ProvidesPackage|) with:
%
\begin{center}
\begin{tabular}{l}
|\input{childdoc.def}|\\
|\childdocforward{|\textit{main}|}|\\
\end{tabular}
\end{center}
%
or alternatively with:
%
\begin{center}
\begin{tabular}{l}
|\input{childdoc.def}|\\
|\childdocby{|\textit{main}|}|\\
\end{tabular}
\end{center}
%
Both forms have slightly different effects as described above.
The main file is prepared as usual, see \secref{sec:include}.

%%%%%%%%%%%%%%%%%%%%%%%%%%%%%%%%%%%%%%%%%%%%%%%%%%%%%%%%%%%%%%%%%%%%%%%%%%%%%%%%
\subsection{Legacy Detection}
\label{sec:detection}

The directive |\childdocmain| in the main file can detect
whether the complete document or merely a child is to be compiled
even without using the directive |\childdocof|.
This method is deprecated because it is less robust
and there is no compelling reason to use it;
it is merely provided for backward compatibility
and it may be removed in future versions.

If the detection mechanism is to be used,
it is mandatory to correctly specify
the filename of the main file as the argument of |\childdocmain|:
%
\begin{center}
\begin{tabular}{l}
|\input{childdoc.def}|\\
|\childdocmain{|\textit{main}|}|\\
\end{tabular}
\end{center}
%
If |\jobname| does not match the argument \textit{main} of |\childdocmain|,
it is assumed that |\jobname| points to the child file to be compiled.
When using |\childdocmain| with the main file specified as argument,
it suffices to start a child file
with just |\input{|\textit{main}|}|
without loading of the package and using |\childdocof|.
If instead all processing is done
with the appropriate \textsf{childdoc} directives,
the argument of \textit{main} of |\childdocmain| can be empty.

An alternative version of the command line processing described
in \secref{sec:commandline} using the detection mechanism reads:
%
\begin{center}
|... -jobname "|\textit{target}|" "|[\textit{flags}]%
[|\def\jobname{|\textit{dest}|}|]|\input{|\textit{main}|}"|
\end{center}

%%%%%%%%%%%%%%%%%%%%%%%%%%%%%%%%%%%%%%%%%%%%%%%%%%%%%%%%%%%%%%%%%%%%%%%%%%%%%%%%
\subsection{Manual Code}
\label{sec:manual}

In case one cannot be certain whether the definitions file |childdoc.def|
is installed on the target \TeX{} distribution
and one prefers not to ship it,
it is conceivable to paste a few relevant commands into the sources.

To that end, drop all statements |\input{childdoc.def}|
and perform the replacements as outlined below.
Instead of |\childdocmain{|\textit{main}|}| add the following code
to the top of the main file:
%
\begin{center}
\begin{tabular}{l}
|\||ifdefined\childdocname\endinput\||fi\newif\ifchilddoc|\\
|\edef\childdocname{\scantokens\expandafter{\jobname\noexpand}}|\\
|\def\childdocmain{|\textit{main}|}\||ifx\childdocmain\childdocname\||else|\\
|\childdoctrue\includeonly{\childdocname}\let\jobname\childdocmain\||fi|\\
\end{tabular}
\end{center}
%
Instead of |\childdocof{|\textit{main}|}| just include the main file
at the top of each child file:
%
\begin{center}
|\input{|\textit{main}|}|
\end{center}
%
A simple redirection |\childdocforward{|\textit{dest}|}| is achieved by:
%
\begin{center}
|\def\jobname{|\textit{dest}|}\input{\jobname}|
\end{center}
%
The redirection with prefix
|\childdocforwardprefix[|\textit{prefix}|]{|\textit{dest}|}|
is accomplished by:
%
\begin{center}
\begin{tabular}{l}
|{\edef\jobname{\scantokens\expandafter{\jobname\noexpand}}|\\
|\def\redirectjob |\textit{prefix}|#1~~~{\gdef\jobname{|\textit{dest}|#1}}|\\
|\expandafter\redirectjob\jobname~~~}\input{\jobname}|
\end{tabular}
\end{center}

In an alternative approach,
child documents can be compiled by a specific command line
without additional code or specific definitions:
%
\begin{center}
|... -jobname "|\textit{target}|" "|[\textit{flags}]%
|\includeonly{|\textit{dest}|}\input{|\textit{main}|}"|
\end{center}
%

%%%%%%%%%%%%%%%%%%%%%%%%%%%%%%%%%%%%%%%%%%%%%%%%%%%%%%%%%%%%%%%%%%%%%%%%%%%%%%%%
%%%%%%%%%%%%%%%%%%%%%%%%%%%%%%%%%%%%%%%%%%%%%%%%%%%%%%%%%%%%%%%%%%%%%%%%%%%%%%%%
\section{Information}

%%%%%%%%%%%%%%%%%%%%%%%%%%%%%%%%%%%%%%%%%%%%%%%%%%%%%%%%%%%%%%%%%%%%%%%%%%%%%%%%
\subsection{Copyright}

Copyright \copyright{} 2017--2018 Niklas Beisert

This work may be distributed and/or modified under the
conditions of the \LaTeX{} Project Public License, either version 1.3
of this license or (at your option) any later version.
The latest version of this license is in
  \url{http://www.latex-project.org/lppl.txt}
and version 1.3 or later is part of all distributions of \LaTeX{}
version 2005/12/01 or later.

This work has the LPPL maintenance status `maintained'.

The Current Maintainer of this work is Niklas Beisert.

This work consists of the files |README.txt|, |childdoc.ins| and |childdoc.dtx|
as well as the derived files |childdoc.def|, |cdocsamp.tex|
with |cdocsch1.tex|, |cdocsch2.tex|, |cdocspt3.tex|, |cdocspt4.tex|,
|cdocsdrf.tex|, |cdocsfn1.tex|, |cdocsfn2.tex|
as well as |childdoc.pdf|.

%%%%%%%%%%%%%%%%%%%%%%%%%%%%%%%%%%%%%%%%%%%%%%%%%%%%%%%%%%%%%%%%%%%%%%%%%%%%%%%%
\subsection{Files and Installation}

The package consists of the files:
%
\begin{center}
\begin{tabular}{ll}
    |README.txt|   & readme file \\
    |childdoc.ins| & installation file \\
    |childdoc.dtx| & source file \\
    |childdoc.def| & definition file \\
    |cdocsamp.tex| & sample main file \\
    |cdocsch1.tex| & sample include file \\
    |cdocsch2.tex| & sample include file \\
    |cdocspt3.tex| & sample part file \\
    |cdocspt4.tex| & sample part file \\
    |cdocsdrf.tex| & sample redirection file \\
    |cdocsfn1.tex| & sample redirection file \\
    |cdocsfn2.tex| & sample redirection file \\
    |childdoc.pdf| & manual
\end{tabular}
\end{center}
%
The distribution consists of the files
|README.txt|, |childdoc.ins| and |childdoc.dtx|.
%
\begin{itemize}
\item
Run (pdf)\LaTeX{} on |childdoc.dtx|
to compile the manual |childdoc.pdf| (this file).
\item
Run \LaTeX{} on |childdoc.ins| to create the definitions file |childdoc.def|
and the sample |cdocsamp.tex| with include files
|cdocsch1.tex|, |cdocsch2.tex|, |cdocspt3.tex|, |cdocspt4.tex|,
|cdocsdrf.tex|, |cdocsfn1.tex|, |cdocsfn2.tex|.
Then copy the file |childdoc.def| to an appropriate directory of your \LaTeX{}
distribution, e.g.\ \textit{texmf-root}|/tex/latex/childdoc|.
\end{itemize}

%%%%%%%%%%%%%%%%%%%%%%%%%%%%%%%%%%%%%%%%%%%%%%%%%%%%%%%%%%%%%%%%%%%%%%%%%%%%%%%%
\subsection{Related CTAN Packages}

There are several other packages which offer a similar functionality:
%
\begin{itemize}
\item
The packages
\href{http://ctan.org/pkg/docmute}{\textsf{docmute}},
\href{http://ctan.org/pkg/includex}{\textsf{includex}} and
\href{http://ctan.org/pkg/standalone}{\textsf{standalone}}
provide commands to include only the document body of
a child file thus allowing both files to be compiled individually.
\item
The packages \href{http://ctan.org/pkg/subdocs}{\textsf{subdocs}}
and \href{http://ctan.org/pkg/subfiles}{\textsf{subfiles}}
provide structures in which the main and child documents can be
encapsulated and allowing them to be compiled individually.
The inclusion mechanism is different from the conventional |\include|.
\item
The package \href{http://ctan.org/pkg/combine}{\textsf{combine}}
is an elaborate solution to combine several documents into one.
\end{itemize}
%
See also the CTAN topic \href{http://ctan.org/topic/subdocs}{\textsf{subdocs}}
for further related packages.
The present package differs from the above solutions in that
a document structure constructed with the conventional |\include| mechanism
just needs two extra commands at the top of every file
such that all constituent files can be compiled individually.

%%%%%%%%%%%%%%%%%%%%%%%%%%%%%%%%%%%%%%%%%%%%%%%%%%%%%%%%%%%%%%%%%%%%%%%%%%%%%%%%
%\subsection{Feature Suggestions}
%
%The following is a list of features which may be useful for future
%versions of this package:
%%
%\begin{itemize}
%\item
%\ldots
%\end{itemize}

%%%%%%%%%%%%%%%%%%%%%%%%%%%%%%%%%%%%%%%%%%%%%%%%%%%%%%%%%%%%%%%%%%%%%%%%%%%%%%%%
\subsection{Revision History}

%%%%%%%%%%%%%%%%%%%%%%%%%%%%%%%%%%%%%%%%
\paragraph{v2.0:} 2018/12/30

\begin{itemize}
\item
immediate forward processing
\item
added |\childdocby| mechanism
\item
manual restructured
\end{itemize}

%%%%%%%%%%%%%%%%%%%%%%%%%%%%%%%%%%%%%%%%
\paragraph{v1.6:} 2018/01/17

\begin{itemize}
\item
application for development of include files
\item
corrections to manual
\end{itemize}

%%%%%%%%%%%%%%%%%%%%%%%%%%%%%%%%%%%%%%%%
\paragraph{v1.5:} 2017/05/21

\begin{itemize}
\item
more complete structuring introduced
\item
|\childdocof| introduced
\item
|\childdoc| renamed to |\childdocmain|
\item
|\childredirect| renamed to |\childdocforward| and |\childdocforwardprefix|
and functionality expanded
\end{itemize}

%%%%%%%%%%%%%%%%%%%%%%%%%%%%%%%%%%%%%%%%
\paragraph{v1.0:} 2017/04/27

\begin{itemize}
\item
manual and install package
\item
first version published on CTAN
\end{itemize}

%%%%%%%%%%%%%%%%%%%%%%%%%%%%%%%%%%%%%%%%
\paragraph{v0.6:} 2017/04/26

\begin{itemize}
\item
redirection mechanism added
\end{itemize}

%%%%%%%%%%%%%%%%%%%%%%%%%%%%%%%%%%%%%%%%
\paragraph{v0.5:} 2017/04/26

\begin{itemize}
\item
functionality in definition file
\end{itemize}


%%%%%%%%%%%%%%%%%%%%%%%%%%%%%%%%%%%%%%%%%%%%%%%%%%%%%%%%%%%%%%%%%%%%%%%%%%%%%%%%
%%%%%%%%%%%%%%%%%%%%%%%%%%%%%%%%%%%%%%%%%%%%%%%%%%%%%%%%%%%%%%%%%%%%%%%%%%%%%%%%
%%%%%%%%%%%%%%%%%%%%%%%%%%%%%%%%%%%%%%%%%%%%%%%%%%%%%%%%%%%%%%%%%%%%%%%%%%%%%%%%
\appendix

\settowidth\MacroIndent{\rmfamily\scriptsize 000\ }

 \DocInput{childdoc.dtx}

\end{document}
%</driver>
% \fi
%
% %%%%%%%%%%%%%%%%%%%%%%%%%%%%%%%%%%%%%%%%%%%%%%%%%%%%%%%%%%%%%%%%%%%%%%%%%%%%%%
% %%%%%%%%%%%%%%%%%%%%%%%%%%%%%%%%%%%%%%%%%%%%%%%%%%%%%%%%%%%%%%%%%%%%%%%%%%%%%%
% \section{Sample}
%\iffalse
%<*samplemain>
%\fi
%
% The following presents a sample document
% with two chapters, two parts, a title page,
% a compile flag as well as three forwarding files to set the flag.
% It consists of eight |.tex| files:
% \begin{center}
% \begin{tabular}{ll}
% |cdocsamp.tex|&main file\\
% |cdocsch1.tex|&include file for chapter 1\\
% |cdocsch2.tex|&include file for chapter 2\\
% |cdocspt3.tex|&include file for part 3\\
% |cdocspt4.tex|&include file for part 4\\
% |cdocsdrf.tex|&forwarding file for main file in draft mode\\
% |cdocsfi1.tex|&forwarding file for final version of chapter 1\\
% |cdocsfi2.tex|&forwarding file for final version of chapter 2\\
% \end{tabular}
% \end{center}
% Each of the eight files can be compiled directly by the \LaTeX{} compiler.
%
% %%%%%%%%%%%%%%%%%%%%%%%%%%%%%%%%%%%%%%
% \paragraph{Main File.}
%
% The main file is called |cdocsamp.tex|.
%
% Load the \textsf{childdoc} definitions and
% declare the filename for the main document:
%    \begin{macrocode}
\input{childdoc.def}
\childdocmain{}
%    \end{macrocode}

% Optional override for |\version| flag:
%    \begin{macrocode}
%%\ifchilddoc\else\providecommand{\version}{draft}\fi
%    \end{macrocode}

% Define the default values for the |\version| flag
% (|final| for the main file and |draft| for childs):
%    \begin{macrocode}
\ifchilddoc
\providecommand{\version}{draft}
\else
\providecommand{\version}{final}
\fi
%    \end{macrocode}

% Load the standard document class:
%    \begin{macrocode}
\documentclass[12pt]{article}
%    \end{macrocode}

% Start the document body:
%    \begin{macrocode}
\begin{document}
%    \end{macrocode}

% Declare a title page.
% Print title, part of document being processed and version flag:
%    \begin{macrocode}
\addtocounter{page}{-1}
\begin{center}
{\LARGE\bfseries{}childdoc example\par}
\vspace{1cm}
\ifchilddoc
\ifchilddocmanual part\else chapter\fi:
`\childdocname' of `\childdocjob'\par
\else
main document: `\childdocjob'\par
\fi
version: \version\par
\end{center}
\newpage
%    \end{macrocode}

% Manually include selected file,
% otherwise process as usual:
%    \begin{macrocode}
\ifchilddocmanual
\section*{part `\childdocname'}
\input{\childdocname}
\else
%    \end{macrocode}

% Include the two chapters:
%    \begin{macrocode}
\include{cdocsch1}
\include{cdocsch2}
%    \end{macrocode}

% Include the two parts unless only chapters should be displayed:
%    \begin{macrocode}
\ifchilddoc\else
\section{part three}
\input{cdocspt3}
\section{part four}
\input{cdocspt4}
\fi
%    \end{macrocode}

% Process as usual until here:
%    \begin{macrocode}
\fi
%    \end{macrocode}

% End of document body:
%    \begin{macrocode}
\end{document}
%    \end{macrocode}
%\iffalse
%</samplemain>
%\fi
%
% %%%%%%%%%%%%%%%%%%%%%%%%%%%%%%%%%%%%%%
% \paragraph{Chapter Include Files.}
%
% The include files are called |cdocsch1.tex| and |cdocsch2.tex|.
%
%\iffalse
%<*samplechap1|samplechap2>
%\fi

% Optional override for |\version| flag:
%    \begin{macrocode}
%%\providecommand{\version}{final}
%    \end{macrocode}

% Include the main document:
%    \begin{macrocode}
\input{childdoc.def}
\childdocof{cdocsamp}
%    \end{macrocode}

%\iffalse
%</samplechap1|samplechap2>
%\fi
%
%\iffalse
%<*samplechap1>
%\fi
% Some text for chapter 1:
%    \begin{macrocode}
\section{one}
some text in chapter one
%    \end{macrocode}

%\iffalse
%</samplechap1>
%\fi
% Some text for chapter 2:
%\iffalse
%<*samplechap2>
%\fi
%    \begin{macrocode}
\section{two}
more text in chapter two
%    \end{macrocode}

%\iffalse
%</samplechap2>
%\fi
%
% %%%%%%%%%%%%%%%%%%%%%%%%%%%%%%%%%%%%%%
% \paragraph{Part Include Files.}
%
% The include files are called |cdocspt3.tex| and |cdocspt4.tex|.
%
%\iffalse
%<*samplepart3|samplepart4>
%\fi

% Optional override for |\version| flag:
%    \begin{macrocode}
%%\providecommand{\version}{final}
%    \end{macrocode}

% Include the main document:
%    \begin{macrocode}
\input{childdoc.def}
\childdocby{cdocsamp}
%    \end{macrocode}

%\iffalse
%</samplepart3|samplepart4>
%\fi
%
%\iffalse
%<*samplepart3>
%\fi
% Some text for part 3:
%    \begin{macrocode}
some text in part three
%    \end{macrocode}

%\iffalse
%</samplepart3>
%\fi
% Some text for part 4:
%\iffalse
%<*samplepart4>
%\fi
%    \begin{macrocode}
more text in part four
%    \end{macrocode}

%\iffalse
%</samplepart4>
%\fi
%
% %%%%%%%%%%%%%%%%%%%%%%%%%%%%%%%%%%%%%%
% \paragraph{Forwarding for a Complete Draft.}
%
% The following forwarding file |cdocsdrf.tex|
% compiles the main document in draft mode:
%\iffalse
%<*sampledraft>
%\fi
%    \begin{macrocode}
\def\version{draft}
\input{childdoc.def}
\childdocforward{cdocsamp}
%    \end{macrocode}

%\iffalse
%</sampledraft>
%\fi
%
% %%%%%%%%%%%%%%%%%%%%%%%%%%%%%%%%%%%%%%
% \paragraph{Forwarding for Final Version of the Chapters.}
%
% The following forwarding files |cdocsfn1.tex| and |cdocsfn2.tex|
% (with identical content)
% compile the final versions of the child documents
% |cdocsch1.tex| and |cdocsch2.tex|, respectively:
%\iffalse
%<*samplefinal>
%\fi
%    \begin{macrocode}
\def\version{final}
\input{childdoc.def}
\childdocforwardprefix[cdocsamp]{cdocsfn}{cdocsch}
%    \end{macrocode}

%\iffalse
%</samplefinal>
%\fi
%
% %%%%%%%%%%%%%%%%%%%%%%%%%%%%%%%%%%%%%%
% \paragraph{Command Line Processing.}
%
% The following three command lines generate the output files
% |cdocscld|, |cdocscl1| and |cdocscl2|
% which should be identical to
% |cdocsdrf|, |cdocsch1| and |cdocsfn2|, respectively:
% \begin{center}
% \begin{tabular}{l}
% |latex -jobname cdocscld \|\\
% |  "\def\version{draft}\input{childdoc.def}\childdocforward{cdocsamp}"|\\
% |latex -jobname cdocscl1 \|\\
% |  "\input{childdoc.def}\childdocforward[cdocsamp]{cdocsch1}"|\\
% |latex -jobname cdocscl2 \|\\
% |  "\def\version{final}\input{childdoc.def}\childdocforward{cdocsch2}"|
% \end{tabular}
% \end{center}
% Note that the trailing backslash on each first line
% merely continues the input to the second line
% (for convenient cut ant paste).
% Furthermore, the command |latex| can be replaced by any
% of its alternative versions such as |pdflatex|.
%
% %%%%%%%%%%%%%%%%%%%%%%%%%%%%%%%%%%%%%%%%%%%%%%%%%%%%%%%%%%%%%%%%%%%%%%%%%%%%%%
% %%%%%%%%%%%%%%%%%%%%%%%%%%%%%%%%%%%%%%%%%%%%%%%%%%%%%%%%%%%%%%%%%%%%%%%%%%%%%%
% \section{Implementation}
%\iffalse
%<*package>
%\fi
%
% This section describes the definitions file |childdoc.def|.

% The definitions cannot be loaded using |\usepackage| or |\RequirePackage|
% which has a mechanism to prevent loading a style file more than once.
% When loading the definitions by means of |\input|
% multiple instances have to be prevented manually:
%\iffalse
%This code needs to be before the `\ProvidesFile' directive
%which is defined at the beginning of this file.
%Therefore it is also placed there and commented out here.
%</package>
%<*discard>
%\fi
%    \begin{macrocode}
\ifdefined\childdocmain\endinput\fi
%    \end{macrocode}
%\iffalse
%</discard>
%<*package>
%\fi
%
% \macro{\ifchilddoc}
% \macro{\ifchilddocmanual}
% The conditional |\ifchilddoc| tells whether a
% child (true) or main (false) document is being compiled.
% The conditional |\ifchilddocmanual| tells whether
% the |\includeonly| mechanism is used (false) or
% the selection of child files must be performed manually (true).
% The definitions initialise to false:
%    \begin{macrocode}
\newif\ifchilddoc
\newif\ifchilddocmanual
%    \end{macrocode}

% \macro{\childdocname}
% \macro{\childdocjob}
% The macro |\childdocname| stores the name of the main document
% to be compiled. The macro |\childdocjob| stores the name of
% the document on which the \LaTeX{} compiler was originally invoked.
% The content of |\jobname| cannot be compared
% to filenames specified in the source due to different catcodes.
% The following code rescans |\jobname|, stores the result
% in |\childdocname| and saves a copy in |\childdocjob|:
%    \begin{macrocode}
\edef\childdocname{\scantokens\expandafter{\jobname\noexpand}}
\let\childdocjob\childdocname
%    \end{macrocode}

% \macro{\childdocdisable}
% The macro |\childdocdisable| prevents the main file
% from being processed more than once.
% At this stage, the main document command |\childdocmain|
% is assumed to be called once again where it should do nothing.
% Any subsequent call to it should prevent
% a secondary processing of the main document
% It overwrites the forwarding commands
% |\childdocof| and |\childdocforward|
% with empty macros to prevent further inclusions of the main document:
%    \begin{macrocode}
\newcommand{\childdocdisable}
{
  \renewcommand{\childdocmain}[1]{\renewcommand{\childdocmain}[1]{\endinput}}
  \renewcommand{\childdocof}[1]{}
  \renewcommand{\childdocby}[2][]{}
  \renewcommand{\childdocforward}[2][]{}
  \renewcommand{\childdocdisable}{}
}
%    \end{macrocode}

% \macro{\childdocmain}
% The macro |\childdocmain| is to be called at the top of the main file
% with nothing or the main filename (without extension) as argument.
% First, it breaks loops.
% If the argument is not empty and does not match |\childdocname|
% (which is set by the first inclusion of |childdoc.def|),
% |\ifchilddoc| is set to true, |\includeonly| is applied to the child file
% and |\jobname| is set to the main file
% (for proper handling of |.aux| files):
%    \begin{macrocode}
\newcommand{\childdocmain}[1]
{
  \childdocdisable\childdocmain{}
  \if?#1?\else
    \begingroup
      \def\childdoctmp{#1}
      \ifx\childdoctmp\childdocname
        \def\childdoctmp{}
      \else
        \def\childdoctmp
        {
          \childdoctrue
          \includeonly{\childdocname}
          \def\childdocjob{#1}
          \def\jobname{#1}
        }
      \fi
      \expandafter
    \endgroup
    \childdoctmp
  \fi
}
%    \end{macrocode}

% \macro{\childdocof}
% The command |\childdocof| redirects
% compilation to the main file |#1|.
%    \begin{macrocode}
\newcommand{\childdocof}[1]
{
  \childdocdisable
  \childdoctrue
  \includeonly{\childdocname}
  \def\jobname{#1}
  \def\childdocjob{#1}
  \input{#1}
}
%    \end{macrocode}

% \macro{\childdocby}
% The command |\childdocby| ....
%    \begin{macrocode}
\newcommand{\childdocby}[2][]
{
  \childdocdisable
  \childdoctrue
  \childdocmanualtrue
  \if?#1?\else
    \def\jobname{#2}
  \fi
  \def\childdocjob{#2}
  \input{#2}
  \endinput
}
%    \end{macrocode}

% \macro{\childdocforward}
% The command |\childdocforward| redirects
% compilation to the main file or
% (if the optional argument is given) a child file.
% Parameters are set as if the main file
% or a child file starting with |\childdocof| was compiled.
% Then compilation is handed over to the main file:
%    \begin{macrocode}
\newcommand{\childdocforward}[2][]
{
  \begingroup
    \if?#1?
      \def\childdoctmp
      {
        \def\childdocname{#2}
        \def\childdocjob{#2}
        \def\jobname{#2}
        \input{#2}
        \endinput
      }
    \else
      \def\childdoctmp
      {
        \childdocdisable
        \def\childdocname{#2}
        \childdoctrue
        \includeonly{#2}
        \def\childdocjob{#1}
        \def\jobname{#1}
        \input{#1}
        \endinput
      }
    \fi
    \expandafter
  \endgroup
  \childdoctmp
}
%    \end{macrocode}

% \macro{\childdocforwardprefix}
% The command |\childdocforwardprefix| redirects
% compilation to the main or a child file by means of a pattern.
% The prefix |#1| in the current filename is replaced by |#2|
% and the suffix of the current filename is kept
% (it is assumed that the filename does not contain the substring `|~~~|'
% which is used as a delimiter).
% Compilation is handed over to the new file by |\childdocforward|:
%    \begin{macrocode}
\newcommand{\childdocforwardprefix}[3][]
{
  \begingroup
    \def\childdocextract #2##1~~~{\def\childdoctmp{\childdocforward[#1]{#3##1}}}
    \expandafter\childdocextract\childdocname~~~
    \expandafter
  \endgroup
  \childdoctmp
}
%    \end{macrocode}

% \macro{\childdoc}
% The deprecated macro |\childdoc| is a legacy version of |\childdocmain|:
%    \begin{macrocode}
\newcommand{\childdoc}{\childdocmain}
%    \end{macrocode}

% \macro{\childdocredirect}
% The deprecated macro |\childdocredirect| is a legacy version
% of |\childdocforward| and |\childdocforwardprefix|:
%    \begin{macrocode}
\newcommand{\childdocredirect}[2][]
{
  \begingroup
    \if?#1?
      \def\childdoctmp{\childdocforward{#2}}
    \else
      \def\childdoctmp{\childdocforwardprefix{#1}{#2}}
    \fi
    \expandafter
  \endgroup
  \childdoctmp
}
%    \end{macrocode}

%\iffalse
%</package>
%\fi
%
\endinput
|\\
|\childdocby{|\textit{main}|}|\\
\end{tabular}
\end{center}
%
Both forms have slightly different effects as described above.
The main file is prepared as usual, see \secref{sec:include}.

%%%%%%%%%%%%%%%%%%%%%%%%%%%%%%%%%%%%%%%%%%%%%%%%%%%%%%%%%%%%%%%%%%%%%%%%%%%%%%%%
\subsection{Legacy Detection}
\label{sec:detection}

The directive |\childdocmain| in the main file can detect
whether the complete document or merely a child is to be compiled
even without using the directive |\childdocof|.
This method is deprecated because it is less robust
and there is no compelling reason to use it;
it is merely provided for backward compatibility
and it may be removed in future versions.

If the detection mechanism is to be used,
it is mandatory to correctly specify
the filename of the main file as the argument of |\childdocmain|:
%
\begin{center}
\begin{tabular}{l}
|% \iffalse
%
% childdoc.dtx Copyright (C) 2017-2018 Niklas Beisert
%
% This work may be distributed and/or modified under the
% conditions of the LaTeX Project Public License, either version 1.3
% of this license or (at your option) any later version.
% The latest version of this license is in
%   http://www.latex-project.org/lppl.txt
% and version 1.3 or later is part of all distributions of LaTeX
% version 2005/12/01 or later.
%
% This work has the LPPL maintenance status `maintained'.
%
% The Current Maintainer of this work is Niklas Beisert.
%
% This work consists of the files childdoc.dtx and childdoc.ins
% and the derived files childdoc.def and cdocsamp.tex with
% cdocsch1.tex, cdocsch2.tex, cdocsdrf.tex, cdocsfn1.tex, cdocsfn2.tex.
%
%<package>\ifdefined\childdocmain\endinput\fi
%<package>\ProvidesFile{childdoc.def}[2018/12/30 v2.0 child document driver]
%<samplemain>\ProvidesFile{cdocsamp.tex}[2018/12/30 v2.0 sample for childdoc]
%<*driver>
%\ProvidesFile{childdoc.drv}[2018/12/30 v2.0 childdoc reference manual file]
\PassOptionsToClass{10pt,a4paper}{article}
\documentclass{ltxdoc}

\usepackage[margin=35mm]{geometry}
\usepackage{hyperref}
\usepackage{hyperxmp}
\usepackage[usenames]{color}

\hypersetup{colorlinks=true}
\hypersetup{pdfstartview=FitH}
\hypersetup{pdfpagemode=UseNone}
\hypersetup{pdfsource={}}
\hypersetup{pdflang={en-UK}}
\hypersetup{pdfcopyright={Copyright 2017-2018 Niklas Beisert.
  This work may be distributed and/or modified under the
  conditions of the LaTeX Project Public License, either version 1.3
  of this license or (at your option) any later version.}}
\hypersetup{pdflicenseurl={http://www.latex-project.org/lppl.txt}}
\hypersetup{pdfcontactaddress={ETH Zurich, ITP, HIT K,
  Wolfgang-Pauli-Strasse 27}}
\hypersetup{pdfcontactpostcode={8093}}
\hypersetup{pdfcontactcity={Zurich}}
\hypersetup{pdfcontactcountry={Switzerland}}
\hypersetup{pdfcontactemail={nbeisert@itp.phys.ethz.ch}}
\hypersetup{pdfcontacturl={http://people.phys.ethz.ch/\xmptilde nbeisert/}}

\newcommand{\secref}[1]{\hyperref[#1]{section \ref*{#1}}}

\parskip1ex
\parindent0pt
\let\olditemize\itemize
\def\itemize{\olditemize\parskip0pt}

\begin{document}

\title{The \textsf{childdoc} Package}
\hypersetup{pdftitle={The childdoc Package}}
\author{Niklas Beisert\\[2ex]
  Institut f\"ur Theoretische Physik\\
  Eidgen\"ossische Technische Hochschule Z\"urich\\
  Wolfgang-Pauli-Strasse 27, 8093 Z\"urich, Switzerland\\[1ex]
  \href{mailto:nbeisert@itp.phys.ethz.ch}
  {\texttt{nbeisert@itp.phys.ethz.ch}}}
\hypersetup{pdfauthor={Niklas Beisert}}
\hypersetup{pdfsubject={Manual for the LaTeX2e Package childdoc}}
\date{30 December 2018, \textsf{v2.0}}
\maketitle

\begin{abstract}\noindent
\textsf{childdoc} is a \LaTeXe{} package
that enables the direct compilation
of document sections included by |\include|
to individual files.
\end{abstract}

\begingroup
\parskip0ex
\tableofcontents
\endgroup

%%%%%%%%%%%%%%%%%%%%%%%%%%%%%%%%%%%%%%%%%%%%%%%%%%%%%%%%%%%%%%%%%%%%%%%%%%%%%%%%
%%%%%%%%%%%%%%%%%%%%%%%%%%%%%%%%%%%%%%%%%%%%%%%%%%%%%%%%%%%%%%%%%%%%%%%%%%%%%%%%
\section{Introduction}

\LaTeX{} provides a mechanism to structure a large document (such as a book)
into a main file and several child files (containing the chapters)
using the |\include| command.
This mechanism is beneficial for documents
which span hundreds of pages in order to
make the source file(s) more manageable.
Moreover, compilation can be restricted to
selected child files by means of the |\includeonly| command.
The latter feature can be used to reduce the compilation time while editing
(this was significantly more useful in the earlier days of \LaTeX{})
or to generate a smaller document which is easier to navigate.
Another application of |\includeonly| is to generate
documents consisting of selected parts of the complete document.

However, there are a few drawbacks of the plain |\include| mechanism:
\begin{itemize}
\item
The child files cannot be compiled on their own,
they can only be compiled via the main file.
A naive editing environment
(such as a text editor with an option
to have the current file processed by \LaTeX)
may require one to switch to the main file before compiling;
attempting to compile the child file produces errors.
\item
The main file must be modified (each time)
to adjust the |\includeonly| command
to the present needs. This easily leaves the main file in a messy state.
\item
The generated document will always carry the filename
of the main document. This is inconvenient if
several child files are to be compiled and
to be kept for distribution.
\end{itemize}

The present package provides a simple interface
to make child files individually compilable by \LaTeX{}.
Compiling a child file then has the same effect as compiling
the main file with an |\includeonly| command
to select the appropriate child.
Moreover the generated document will carry the name of the child
rather than the main file.
This resolves all three above issues.

This feature is meant to make the editing of books,
thesis documents and lecture notes somewhat more convenient.
However, the package can also be used efficiently for
composing a series of documents (such as exercise sheets)
which are typically distributed individually.
It then assists the author in generating the individual documents
(potentially in different versions)
as well as a document containing the collected series.
Another application is in developing style files
or other kinds of included material
where compilation of the style file could redirect
to a sample or test file.

%%%%%%%%%%%%%%%%%%%%%%%%%%%%%%%%%%%%%%%%%%%%%%%%%%%%%%%%%%%%%%%%%%%%%%%%%%%%%%%%
%%%%%%%%%%%%%%%%%%%%%%%%%%%%%%%%%%%%%%%%%%%%%%%%%%%%%%%%%%%%%%%%%%%%%%%%%%%%%%%%
\section{Usage}

First of all, the package \textsf{childdoc} is \emph{not} a standard
\LaTeXe{} |.sty| style file! Therefore it needs to be invoked in
a non-standard way.

%%%%%%%%%%%%%%%%%%%%%%%%%%%%%%%%%%%%%%%%%%%%%%%%%%%%%%%%%%%%%%%%%%%%%%%%%%%%%%%%
\subsection{Included Files}
\label{sec:include}

%%%%%%%%%%%%%%%%%%%%%%%%%%%%%%%%%%%%%%%%
\DescribeMacro{\childdocmain}
To use the package, add the commands
\begin{center}
\begin{tabular}{l}
|\input{childdoc.def}|\\
|\childdocmain{}|\\
\end{tabular}
\end{center}
at the very top of the main \LaTeX{} file,
in particular \emph{before} the |\documentclass| statement!
The argument of |\childdocmain| should be left empty
(but it must be present).

%%%%%%%%%%%%%%%%%%%%%%%%%%%%%%%%%%%%%%%%
\DescribeMacro{\childdocof}
Furthermore, add the commands
\begin{center}
\begin{tabular}{l}
|\input{childdoc.def}|\\
|\childdocof{|\textit{main}|}|\\
\end{tabular}
\end{center}
at the top of every child file \textit{child}
which is included by |\include{|\textit{child}|}|
from within the main file
(or at least for those files to be compiled individually).
The argument \textit{main} must be the filename of the main file.

There are a couple of
considerations in setting up the main and child documents:

%%%%%%%%%%%%%%%%%%%%%%%%%%%%%%%%%%%%%%%%
\paragraph{Restrictions.}

Please note the following restrictions:
\begin{itemize}
\item
|\childdocmain| must be called with one argument \textit{main}
to ensure compatibility with earlier version of the package.
It must either be empty (|\childdocmain{}|)
or precisely match the filename of the main file in which it is specified.
See \secref{sec:detection} for further information.
\item
The filename \textit{main} must be specified without the |.tex| extension.
\item
The filename \textit{main} is case sensitive
(even in case-insensitive file systems)
due to internal string comparison.
\item
The argument \textit{main} should be fully expanded, it cannot be a macro.
\item
Subdirectories and special characters should be avoided in filenames.
\item
The command |\childdocmain{|\textit{main}|}| must be followed by a whitespace.
It should not be followed immediately by another command
or by a comment mark `|%|'.
This is because the \TeX{} parser reads the token immediately following
the argument of |\childdocmain| and puts it
at the beginning of every child section;
however, a white\-space is ignored.
\end{itemize}

%%%%%%%%%%%%%%%%%%%%%%%%%%%%%%%%%%%%%%%%
\paragraph{Content of Main File.}

It is advisable to place all content in the child files included by |\include|.
Any output contained in the main file will appear in all child documents
unless suppressed manually;
it cannot be suppressed automatically by the |\includeonly| directive
and thus should normally be avoided.
A method to include some content in the main file
by means of conditional processing is described in \secref{sec:conditional}.

%%%%%%%%%%%%%%%%%%%%%%%%%%%%%%%%%%%%%%%%
\paragraph{Page Numbering.}

When only a part of the document is compiled,
the appropriate numbering of pages
(as well as other status parameters)
is determined from the |.aux| files.
The latter contain information from previous passes.
However this information needs to propagate through
all intermediate child documents.
Therefore the page numbering in child documents may well
be inconsistent until the complete document is compiled at least once.

A useful (if unconventional) way to always ensure a consistent
page numbering is to restart the numbering in each child document
and denote the pages by `\textit{child}|.|\textit{page}'
where \textit{child} represents the chapter/section number of the child file.
This can be achieved by the command
|\numberwithin{page}{|\textit{child}|}|
of the \textsf{amsmath} package
where \textit{child} can be |chapter| or |section|
depending on the chosen structuring.
Alternatively, one can modify the macro |\thepage| appropriately
and reset the counter |page| at the start of each child file.

%%%%%%%%%%%%%%%%%%%%%%%%%%%%%%%%%%%%%%%%%%%%%%%%%%%%%%%%%%%%%%%%%%%%%%%%%%%%%%%%
\subsection{Conditional Processing}
\label{sec:conditional}

The package provides a mechanism to compile different versions
of a document. To customise the versions further some conditional processing
can come in handy to distinguish which version is being compiled.
The package provides two macros to describe the compilation context:

%%%%%%%%%%%%%%%%%%%%%%%%%%%%%%%%%%%%%%%%
\DescribeMacro{\ifchilddoc}
The conditional |\ifchilddoc| distinguishes between the compilation of
child documents and the main document:
%
\begin{center}
|\ifchilddoc |\textit{child-code}| |[|\||else |\textit{main-code}]| \||fi|
\end{center}

%%%%%%%%%%%%%%%%%%%%%%%%%%%%%%%%%%%%%%%%
\DescribeMacro{\childdocname}
\DescribeMacro{\childdocjob}
The macro |\childdocname| contains the filename (without extension)
of the main or child file being processed.
Note that |\childdocjob| will always contain the name of the main file.

%%%%%%%%%%%%%%%%%%%%%%%%%%%%%%%%%%%%%%%%
\paragraph{Title Page.}

Conditional processing can be used to include a title or banner page
in the main document when proper precautions are taken.
Importantly, the code in the main file should ensure that the page counter
(as well as other status parameters which are stored in the |.aux| files)
takes the same value after the conditional processing.
Otherwise the page numbers may take divergent values
depending on which part is compiled.

For example, a title page could be declared by:
%
\begin{center}
\begin{tabular}{l}
|\ifchilddoc\||else|\\
|\addtocounter{page}{-1}|\\
\textit{code for title page}\\
|\newpage|\\
|\||fi|
\end{tabular}
\end{center}
%
A banner page for the child documents can be generated by:
%
\begin{center}
\begin{tabular}{l}
|\ifchilddoc|\\
|\addtocounter{page}{-1}|\\
\textit{code for banner page}\\
|\newpage|\\
|\||fi|
\end{tabular}
\end{center}
%
Here one could write a message such as:
\begin{center}
|This is the part \childdocname{} of \childdocjob{}.|
\end{center}

%%%%%%%%%%%%%%%%%%%%%%%%%%%%%%%%%%%%%%%%%%%%%%%%%%%%%%%%%%%%%%%%%%%%%%%%%%%%%%%%
\subsection{Flags}
\label{sec:flags}

The package makes it easy to generate different versions
of the main or child documents.
To this end compilation flags can be defined
and assigned different default values.
They will be particularly useful in conjunction
with the forwarding mechanism described in \secref{sec:forward}.

For example, it may be useful to have a flag |\version|
which can be set to |draft| or |final|.
The document source will contain some conditional code
depending on the value of |\version|.
Suppose further, the flag should default to |final| for the main file
and to |draft| for child files
which is a natural assignment for editing the document.
This is achieved by placing the following code
in the preamble of the main document
(below the |\childdocmain| directive):
%
\begin{center}
\begin{tabular}{l}
|\ifchilddoc|\\
|\providecommand{\version}{draft}|\\
|\||else|\\
|\providecommand{\version}{final}|\\
|\||fi|
\end{tabular}
\end{center}
%
The definition by |\providecommand| makes sure
that previous definitions are not overwritten.
Further statements |\providecommand{\version}{...}|
can thus be added before the above code to override it.

For the main file, one might add a line
(between |\childdocmain| and the above block)
%
\begin{center}
|%\ifchilddoc\||else\providecommand{\version}{draft}\||fi|
\end{center}
%
which can be uncommented to produce a draft version.
Likewise one can add a line to the very top of a child file
(above the |\childdocof{|\textit{main}|}| directive)
%
\begin{center}
|%\providecommand{\version}{final}|
\end{center}
%
which can be uncommented to produce the final version of this child document.

%%%%%%%%%%%%%%%%%%%%%%%%%%%%%%%%%%%%%%%%%%%%%%%%%%%%%%%%%%%%%%%%%%%%%%%%%%%%%%%%
\subsection{Forwarding}
\label{sec:forward}

Different versions of the main or child documents
using compilation flags as described in \secref{sec:flags}
can be (permanently) stored in different files
for convenient compilation, viewing and distribution.
To this end, the package defines a command
to pass on compilation to a different file:

%%%%%%%%%%%%%%%%%%%%%%%%%%%%%%%%%%%%%%%%
\DescribeMacro{\childdocforward}
The command |\childdocforward| redirects processing to
another source file:
%
\begin{center}
\begin{tabular}{l}
|\input{childdoc.def}|\\
|\childdocforward[|\textit{main}|]{|\textit{dest}|}|\\
\end{tabular}
\end{center}
%
The argument \textit{dest} is the destination file
(without extension).
It should be the main file or one of the child files.
Note that further \textsf{childdoc} directives
such as |\childdocof| and |\childdocforward|
in the indicated file will be processed in this form.
The optional argument \textit{main}
passes on directly to the main file \textit{main}
while pretending to compile the child \textit{dest}.
This form behaves as if \textit{dest}
issues |\childdocof{|\textit{main}|}| right away,
and no further \textsf{childdoc} directives will be processed.

%%%%%%%%%%%%%%%%%%%%%%%%%%%%%%%%%%%%%%%%
\DescribeMacro{\...prefix}
In the alternative form |\childdocforwardprefix|,
%
\begin{center}
\begin{tabular}{l}
|\input{childdoc.def}|\\
|\childdocforwardprefix[|\textit{main}|]{|\textit{prefix}|}{|\textit{dest}|}|
\end{tabular}
\end{center}
%
the destination file is determined by a pattern
depending on the current file:
To make this work, the current file must be called
`{\textit{prefix}\hspace{0.2em}\textit{suffix}}'
with \textit{prefix} matching precisely the argument.
Processing is then passed on to the file
`{\textit{dest}\hspace{0.2em}\textit{suffix}}'.
Surely, the same effect is achieved by
directly specifying the
argument `{\textit{dest}\hspace{0.2em}\textit{suffix}}'
in the first form.
However, that requires to set up a different file
for each child. With the alternative form of the command
all these files can have exactly the same content
which simplifies setting them up and maintaining them.

For example, the following file |draft.tex|
with a compilation flag |\version| as described in \secref{sec:flags}
compiles the main document as a draft:
%
\begin{center}
\begin{tabular}{l}
|\def\version{draft}|\\
|\input{childdoc.def}|\\
|\childdocforward{|\textit{main}|}|
\end{tabular}
\end{center}
%
Likewise, the following files |final|\textit{nn}|.tex|
compile the final version of the child document
|child|\textit{nn}|.tex|:
%
\begin{center}
\begin{tabular}{l}
|\def\version{final}|\\
|\input{childdoc.def}|\\
|\childdocforwardprefix{final}{child}|
\end{tabular}
\end{center}
%

Note that when several versions of a main file and/or of each child file
are to be generated, it may be convenient to set up a |Makefile| or
shell script to automatise the process.

%%%%%%%%%%%%%%%%%%%%%%%%%%%%%%%%%%%%%%%%%%%%%%%%%%%%%%%%%%%%%%%%%%%%%%%%%%%%%%%%
\subsection{Command Line Processing}
\label{sec:commandline}

The effect of redirection files can also be achieved by invoking
the \LaTeX{} compiler with a more elaborate command line.
Most conveniently this should be done as part
of a shell script or a |Makefile|.

When using \textsf{childdoc} in the main file, the following
command lines effectively perform a redirection
(note that depending on the shell being used,
backslashes may have to be doubled: `|\|' $\to$ `|\\|'):
%
\begin{center}
|... -jobname "|\textit{target}|" |\\|"|[\textit{flags}]%
|\input{childdoc.def}\childdocforward[|\textit{main}|]{|\textit{dest}|}"|
\end{center}
%
Here \textit{target} is the name of the output file,
\textit{main} is the name of the main file
and \textit{dest} is the name of the main or child file to be processed
(all filenames without extensions).
The optional argument \textit{main} can be omitted
if \textit{main} matches \textit{dest}.
Optionally, compilation \textit{flags} can be defined via |\def| commands.
This command line makes the \TeX{} engine believe
it is compiling the file \textit{target}
whose content is specified as the latter parameter.
The provided code then forwards the processing to
\textit{main} or \textit{dest} as described in \secref{sec:forward}.

%%%%%%%%%%%%%%%%%%%%%%%%%%%%%%%%%%%%%%%%%%%%%%%%%%%%%%%%%%%%%%%%%%%%%%%%%%%%%%%%
\subsection{Include by Input}
\label{sec:input}

Including child documents by |\include| has some restrictions by design.
Most notably, the content of a child document always occupies
its own set of pages; pages cannot be shared between child documents.
Usually, this behaviour makes perfect sense
because each child document contain an essential part of the document.
However, in some situations it may be desirable to compose
a document from a collection of parts
without having mandatory page breaks between then.
For this case, the package
provides a mechanism to include parts
by |\input| which can also be processed individually.
However, by construction this mechanism
requires manual handling of the content to be output.

%%%%%%%%%%%%%%%%%%%%%%%%%%%%%%%%%%%%%%%%
\DescribeMacro{\ifchilddocmanual}
The main file should be prepared as usual, see \secref{sec:include}.
However, the document body must make a distinction
between processing of an individual part and of the main document, e.g.:
%
\begin{center}
\begin{tabular}{l}
|\ifchilddocmanual|\\
|\input{\childdocname}|\\
|\||else|\\
\textit{document body with }|\input{|\textit{part}|}|\\
|\||fi|
\end{tabular}
\end{center}
%
The conditional |\ifchilddocmanual| is true whenever
a part to be included by |\input| is being compiled,
and the name of the part is stored in |\childdocname|.

%%%%%%%%%%%%%%%%%%%%%%%%%%%%%%%%%%%%%%%%
\DescribeMacro{\childdocby}
Each part to be included by |\input| should start with:
%
\begin{center}
\begin{tabular}{l}
|\input{childdoc.def}|\\
|\childdocby{|\textit{main}|}|\\
\end{tabular}
\end{center}
%
The directive |\childdocby| is similar to |\childdocof|
described in \secref{sec:include},
but the subsequent selection of content must be done manually.
To that end, both |\ifchilddoc| and |\ifchilddocmanual|
will be true upon processing of a part,
and the name of the part is stored in |\childdocname|.
Note that |\jobname| will be set to the filename of the current part
so that each part receives an individual |.aux| file
that does not interfere with the |.aux| file(s) of the main document.
This behaviour can be altered by the alternative form
|\childdocby[*]{|\textit{main}|}| (with a non-empty optional argument)
which uses the |.aux| file of the main document
by setting |\jobname| to \textit{main}.

%%%%%%%%%%%%%%%%%%%%%%%%%%%%%%%%%%%%%%%%%%%%%%%%%%%%%%%%%%%%%%%%%%%%%%%%%%%%%%%%
\subsection{Driver Development}
\label{sec:driver}

The \textsf{childdoc} mechanism can also be use for the development
of definition files such as \LaTeX{} styles or classes.
This case differs from the above setup with multiple parts
included by |\include| in that no |\includeonly| should be invoked.
This can be achieved by starting the include file
(before |\ProvidesPackage|) with:
%
\begin{center}
\begin{tabular}{l}
|\input{childdoc.def}|\\
|\childdocforward{|\textit{main}|}|\\
\end{tabular}
\end{center}
%
or alternatively with:
%
\begin{center}
\begin{tabular}{l}
|\input{childdoc.def}|\\
|\childdocby{|\textit{main}|}|\\
\end{tabular}
\end{center}
%
Both forms have slightly different effects as described above.
The main file is prepared as usual, see \secref{sec:include}.

%%%%%%%%%%%%%%%%%%%%%%%%%%%%%%%%%%%%%%%%%%%%%%%%%%%%%%%%%%%%%%%%%%%%%%%%%%%%%%%%
\subsection{Legacy Detection}
\label{sec:detection}

The directive |\childdocmain| in the main file can detect
whether the complete document or merely a child is to be compiled
even without using the directive |\childdocof|.
This method is deprecated because it is less robust
and there is no compelling reason to use it;
it is merely provided for backward compatibility
and it may be removed in future versions.

If the detection mechanism is to be used,
it is mandatory to correctly specify
the filename of the main file as the argument of |\childdocmain|:
%
\begin{center}
\begin{tabular}{l}
|\input{childdoc.def}|\\
|\childdocmain{|\textit{main}|}|\\
\end{tabular}
\end{center}
%
If |\jobname| does not match the argument \textit{main} of |\childdocmain|,
it is assumed that |\jobname| points to the child file to be compiled.
When using |\childdocmain| with the main file specified as argument,
it suffices to start a child file
with just |\input{|\textit{main}|}|
without loading of the package and using |\childdocof|.
If instead all processing is done
with the appropriate \textsf{childdoc} directives,
the argument of \textit{main} of |\childdocmain| can be empty.

An alternative version of the command line processing described
in \secref{sec:commandline} using the detection mechanism reads:
%
\begin{center}
|... -jobname "|\textit{target}|" "|[\textit{flags}]%
[|\def\jobname{|\textit{dest}|}|]|\input{|\textit{main}|}"|
\end{center}

%%%%%%%%%%%%%%%%%%%%%%%%%%%%%%%%%%%%%%%%%%%%%%%%%%%%%%%%%%%%%%%%%%%%%%%%%%%%%%%%
\subsection{Manual Code}
\label{sec:manual}

In case one cannot be certain whether the definitions file |childdoc.def|
is installed on the target \TeX{} distribution
and one prefers not to ship it,
it is conceivable to paste a few relevant commands into the sources.

To that end, drop all statements |\input{childdoc.def}|
and perform the replacements as outlined below.
Instead of |\childdocmain{|\textit{main}|}| add the following code
to the top of the main file:
%
\begin{center}
\begin{tabular}{l}
|\||ifdefined\childdocname\endinput\||fi\newif\ifchilddoc|\\
|\edef\childdocname{\scantokens\expandafter{\jobname\noexpand}}|\\
|\def\childdocmain{|\textit{main}|}\||ifx\childdocmain\childdocname\||else|\\
|\childdoctrue\includeonly{\childdocname}\let\jobname\childdocmain\||fi|\\
\end{tabular}
\end{center}
%
Instead of |\childdocof{|\textit{main}|}| just include the main file
at the top of each child file:
%
\begin{center}
|\input{|\textit{main}|}|
\end{center}
%
A simple redirection |\childdocforward{|\textit{dest}|}| is achieved by:
%
\begin{center}
|\def\jobname{|\textit{dest}|}\input{\jobname}|
\end{center}
%
The redirection with prefix
|\childdocforwardprefix[|\textit{prefix}|]{|\textit{dest}|}|
is accomplished by:
%
\begin{center}
\begin{tabular}{l}
|{\edef\jobname{\scantokens\expandafter{\jobname\noexpand}}|\\
|\def\redirectjob |\textit{prefix}|#1~~~{\gdef\jobname{|\textit{dest}|#1}}|\\
|\expandafter\redirectjob\jobname~~~}\input{\jobname}|
\end{tabular}
\end{center}

In an alternative approach,
child documents can be compiled by a specific command line
without additional code or specific definitions:
%
\begin{center}
|... -jobname "|\textit{target}|" "|[\textit{flags}]%
|\includeonly{|\textit{dest}|}\input{|\textit{main}|}"|
\end{center}
%

%%%%%%%%%%%%%%%%%%%%%%%%%%%%%%%%%%%%%%%%%%%%%%%%%%%%%%%%%%%%%%%%%%%%%%%%%%%%%%%%
%%%%%%%%%%%%%%%%%%%%%%%%%%%%%%%%%%%%%%%%%%%%%%%%%%%%%%%%%%%%%%%%%%%%%%%%%%%%%%%%
\section{Information}

%%%%%%%%%%%%%%%%%%%%%%%%%%%%%%%%%%%%%%%%%%%%%%%%%%%%%%%%%%%%%%%%%%%%%%%%%%%%%%%%
\subsection{Copyright}

Copyright \copyright{} 2017--2018 Niklas Beisert

This work may be distributed and/or modified under the
conditions of the \LaTeX{} Project Public License, either version 1.3
of this license or (at your option) any later version.
The latest version of this license is in
  \url{http://www.latex-project.org/lppl.txt}
and version 1.3 or later is part of all distributions of \LaTeX{}
version 2005/12/01 or later.

This work has the LPPL maintenance status `maintained'.

The Current Maintainer of this work is Niklas Beisert.

This work consists of the files |README.txt|, |childdoc.ins| and |childdoc.dtx|
as well as the derived files |childdoc.def|, |cdocsamp.tex|
with |cdocsch1.tex|, |cdocsch2.tex|, |cdocspt3.tex|, |cdocspt4.tex|,
|cdocsdrf.tex|, |cdocsfn1.tex|, |cdocsfn2.tex|
as well as |childdoc.pdf|.

%%%%%%%%%%%%%%%%%%%%%%%%%%%%%%%%%%%%%%%%%%%%%%%%%%%%%%%%%%%%%%%%%%%%%%%%%%%%%%%%
\subsection{Files and Installation}

The package consists of the files:
%
\begin{center}
\begin{tabular}{ll}
    |README.txt|   & readme file \\
    |childdoc.ins| & installation file \\
    |childdoc.dtx| & source file \\
    |childdoc.def| & definition file \\
    |cdocsamp.tex| & sample main file \\
    |cdocsch1.tex| & sample include file \\
    |cdocsch2.tex| & sample include file \\
    |cdocspt3.tex| & sample part file \\
    |cdocspt4.tex| & sample part file \\
    |cdocsdrf.tex| & sample redirection file \\
    |cdocsfn1.tex| & sample redirection file \\
    |cdocsfn2.tex| & sample redirection file \\
    |childdoc.pdf| & manual
\end{tabular}
\end{center}
%
The distribution consists of the files
|README.txt|, |childdoc.ins| and |childdoc.dtx|.
%
\begin{itemize}
\item
Run (pdf)\LaTeX{} on |childdoc.dtx|
to compile the manual |childdoc.pdf| (this file).
\item
Run \LaTeX{} on |childdoc.ins| to create the definitions file |childdoc.def|
and the sample |cdocsamp.tex| with include files
|cdocsch1.tex|, |cdocsch2.tex|, |cdocspt3.tex|, |cdocspt4.tex|,
|cdocsdrf.tex|, |cdocsfn1.tex|, |cdocsfn2.tex|.
Then copy the file |childdoc.def| to an appropriate directory of your \LaTeX{}
distribution, e.g.\ \textit{texmf-root}|/tex/latex/childdoc|.
\end{itemize}

%%%%%%%%%%%%%%%%%%%%%%%%%%%%%%%%%%%%%%%%%%%%%%%%%%%%%%%%%%%%%%%%%%%%%%%%%%%%%%%%
\subsection{Related CTAN Packages}

There are several other packages which offer a similar functionality:
%
\begin{itemize}
\item
The packages
\href{http://ctan.org/pkg/docmute}{\textsf{docmute}},
\href{http://ctan.org/pkg/includex}{\textsf{includex}} and
\href{http://ctan.org/pkg/standalone}{\textsf{standalone}}
provide commands to include only the document body of
a child file thus allowing both files to be compiled individually.
\item
The packages \href{http://ctan.org/pkg/subdocs}{\textsf{subdocs}}
and \href{http://ctan.org/pkg/subfiles}{\textsf{subfiles}}
provide structures in which the main and child documents can be
encapsulated and allowing them to be compiled individually.
The inclusion mechanism is different from the conventional |\include|.
\item
The package \href{http://ctan.org/pkg/combine}{\textsf{combine}}
is an elaborate solution to combine several documents into one.
\end{itemize}
%
See also the CTAN topic \href{http://ctan.org/topic/subdocs}{\textsf{subdocs}}
for further related packages.
The present package differs from the above solutions in that
a document structure constructed with the conventional |\include| mechanism
just needs two extra commands at the top of every file
such that all constituent files can be compiled individually.

%%%%%%%%%%%%%%%%%%%%%%%%%%%%%%%%%%%%%%%%%%%%%%%%%%%%%%%%%%%%%%%%%%%%%%%%%%%%%%%%
%\subsection{Feature Suggestions}
%
%The following is a list of features which may be useful for future
%versions of this package:
%%
%\begin{itemize}
%\item
%\ldots
%\end{itemize}

%%%%%%%%%%%%%%%%%%%%%%%%%%%%%%%%%%%%%%%%%%%%%%%%%%%%%%%%%%%%%%%%%%%%%%%%%%%%%%%%
\subsection{Revision History}

%%%%%%%%%%%%%%%%%%%%%%%%%%%%%%%%%%%%%%%%
\paragraph{v2.0:} 2018/12/30

\begin{itemize}
\item
immediate forward processing
\item
added |\childdocby| mechanism
\item
manual restructured
\end{itemize}

%%%%%%%%%%%%%%%%%%%%%%%%%%%%%%%%%%%%%%%%
\paragraph{v1.6:} 2018/01/17

\begin{itemize}
\item
application for development of include files
\item
corrections to manual
\end{itemize}

%%%%%%%%%%%%%%%%%%%%%%%%%%%%%%%%%%%%%%%%
\paragraph{v1.5:} 2017/05/21

\begin{itemize}
\item
more complete structuring introduced
\item
|\childdocof| introduced
\item
|\childdoc| renamed to |\childdocmain|
\item
|\childredirect| renamed to |\childdocforward| and |\childdocforwardprefix|
and functionality expanded
\end{itemize}

%%%%%%%%%%%%%%%%%%%%%%%%%%%%%%%%%%%%%%%%
\paragraph{v1.0:} 2017/04/27

\begin{itemize}
\item
manual and install package
\item
first version published on CTAN
\end{itemize}

%%%%%%%%%%%%%%%%%%%%%%%%%%%%%%%%%%%%%%%%
\paragraph{v0.6:} 2017/04/26

\begin{itemize}
\item
redirection mechanism added
\end{itemize}

%%%%%%%%%%%%%%%%%%%%%%%%%%%%%%%%%%%%%%%%
\paragraph{v0.5:} 2017/04/26

\begin{itemize}
\item
functionality in definition file
\end{itemize}


%%%%%%%%%%%%%%%%%%%%%%%%%%%%%%%%%%%%%%%%%%%%%%%%%%%%%%%%%%%%%%%%%%%%%%%%%%%%%%%%
%%%%%%%%%%%%%%%%%%%%%%%%%%%%%%%%%%%%%%%%%%%%%%%%%%%%%%%%%%%%%%%%%%%%%%%%%%%%%%%%
%%%%%%%%%%%%%%%%%%%%%%%%%%%%%%%%%%%%%%%%%%%%%%%%%%%%%%%%%%%%%%%%%%%%%%%%%%%%%%%%
\appendix

\settowidth\MacroIndent{\rmfamily\scriptsize 000\ }

 \DocInput{childdoc.dtx}

\end{document}
%</driver>
% \fi
%
% %%%%%%%%%%%%%%%%%%%%%%%%%%%%%%%%%%%%%%%%%%%%%%%%%%%%%%%%%%%%%%%%%%%%%%%%%%%%%%
% %%%%%%%%%%%%%%%%%%%%%%%%%%%%%%%%%%%%%%%%%%%%%%%%%%%%%%%%%%%%%%%%%%%%%%%%%%%%%%
% \section{Sample}
%\iffalse
%<*samplemain>
%\fi
%
% The following presents a sample document
% with two chapters, two parts, a title page,
% a compile flag as well as three forwarding files to set the flag.
% It consists of eight |.tex| files:
% \begin{center}
% \begin{tabular}{ll}
% |cdocsamp.tex|&main file\\
% |cdocsch1.tex|&include file for chapter 1\\
% |cdocsch2.tex|&include file for chapter 2\\
% |cdocspt3.tex|&include file for part 3\\
% |cdocspt4.tex|&include file for part 4\\
% |cdocsdrf.tex|&forwarding file for main file in draft mode\\
% |cdocsfi1.tex|&forwarding file for final version of chapter 1\\
% |cdocsfi2.tex|&forwarding file for final version of chapter 2\\
% \end{tabular}
% \end{center}
% Each of the eight files can be compiled directly by the \LaTeX{} compiler.
%
% %%%%%%%%%%%%%%%%%%%%%%%%%%%%%%%%%%%%%%
% \paragraph{Main File.}
%
% The main file is called |cdocsamp.tex|.
%
% Load the \textsf{childdoc} definitions and
% declare the filename for the main document:
%    \begin{macrocode}
\input{childdoc.def}
\childdocmain{}
%    \end{macrocode}

% Optional override for |\version| flag:
%    \begin{macrocode}
%%\ifchilddoc\else\providecommand{\version}{draft}\fi
%    \end{macrocode}

% Define the default values for the |\version| flag
% (|final| for the main file and |draft| for childs):
%    \begin{macrocode}
\ifchilddoc
\providecommand{\version}{draft}
\else
\providecommand{\version}{final}
\fi
%    \end{macrocode}

% Load the standard document class:
%    \begin{macrocode}
\documentclass[12pt]{article}
%    \end{macrocode}

% Start the document body:
%    \begin{macrocode}
\begin{document}
%    \end{macrocode}

% Declare a title page.
% Print title, part of document being processed and version flag:
%    \begin{macrocode}
\addtocounter{page}{-1}
\begin{center}
{\LARGE\bfseries{}childdoc example\par}
\vspace{1cm}
\ifchilddoc
\ifchilddocmanual part\else chapter\fi:
`\childdocname' of `\childdocjob'\par
\else
main document: `\childdocjob'\par
\fi
version: \version\par
\end{center}
\newpage
%    \end{macrocode}

% Manually include selected file,
% otherwise process as usual:
%    \begin{macrocode}
\ifchilddocmanual
\section*{part `\childdocname'}
\input{\childdocname}
\else
%    \end{macrocode}

% Include the two chapters:
%    \begin{macrocode}
\include{cdocsch1}
\include{cdocsch2}
%    \end{macrocode}

% Include the two parts unless only chapters should be displayed:
%    \begin{macrocode}
\ifchilddoc\else
\section{part three}
\input{cdocspt3}
\section{part four}
\input{cdocspt4}
\fi
%    \end{macrocode}

% Process as usual until here:
%    \begin{macrocode}
\fi
%    \end{macrocode}

% End of document body:
%    \begin{macrocode}
\end{document}
%    \end{macrocode}
%\iffalse
%</samplemain>
%\fi
%
% %%%%%%%%%%%%%%%%%%%%%%%%%%%%%%%%%%%%%%
% \paragraph{Chapter Include Files.}
%
% The include files are called |cdocsch1.tex| and |cdocsch2.tex|.
%
%\iffalse
%<*samplechap1|samplechap2>
%\fi

% Optional override for |\version| flag:
%    \begin{macrocode}
%%\providecommand{\version}{final}
%    \end{macrocode}

% Include the main document:
%    \begin{macrocode}
\input{childdoc.def}
\childdocof{cdocsamp}
%    \end{macrocode}

%\iffalse
%</samplechap1|samplechap2>
%\fi
%
%\iffalse
%<*samplechap1>
%\fi
% Some text for chapter 1:
%    \begin{macrocode}
\section{one}
some text in chapter one
%    \end{macrocode}

%\iffalse
%</samplechap1>
%\fi
% Some text for chapter 2:
%\iffalse
%<*samplechap2>
%\fi
%    \begin{macrocode}
\section{two}
more text in chapter two
%    \end{macrocode}

%\iffalse
%</samplechap2>
%\fi
%
% %%%%%%%%%%%%%%%%%%%%%%%%%%%%%%%%%%%%%%
% \paragraph{Part Include Files.}
%
% The include files are called |cdocspt3.tex| and |cdocspt4.tex|.
%
%\iffalse
%<*samplepart3|samplepart4>
%\fi

% Optional override for |\version| flag:
%    \begin{macrocode}
%%\providecommand{\version}{final}
%    \end{macrocode}

% Include the main document:
%    \begin{macrocode}
\input{childdoc.def}
\childdocby{cdocsamp}
%    \end{macrocode}

%\iffalse
%</samplepart3|samplepart4>
%\fi
%
%\iffalse
%<*samplepart3>
%\fi
% Some text for part 3:
%    \begin{macrocode}
some text in part three
%    \end{macrocode}

%\iffalse
%</samplepart3>
%\fi
% Some text for part 4:
%\iffalse
%<*samplepart4>
%\fi
%    \begin{macrocode}
more text in part four
%    \end{macrocode}

%\iffalse
%</samplepart4>
%\fi
%
% %%%%%%%%%%%%%%%%%%%%%%%%%%%%%%%%%%%%%%
% \paragraph{Forwarding for a Complete Draft.}
%
% The following forwarding file |cdocsdrf.tex|
% compiles the main document in draft mode:
%\iffalse
%<*sampledraft>
%\fi
%    \begin{macrocode}
\def\version{draft}
\input{childdoc.def}
\childdocforward{cdocsamp}
%    \end{macrocode}

%\iffalse
%</sampledraft>
%\fi
%
% %%%%%%%%%%%%%%%%%%%%%%%%%%%%%%%%%%%%%%
% \paragraph{Forwarding for Final Version of the Chapters.}
%
% The following forwarding files |cdocsfn1.tex| and |cdocsfn2.tex|
% (with identical content)
% compile the final versions of the child documents
% |cdocsch1.tex| and |cdocsch2.tex|, respectively:
%\iffalse
%<*samplefinal>
%\fi
%    \begin{macrocode}
\def\version{final}
\input{childdoc.def}
\childdocforwardprefix[cdocsamp]{cdocsfn}{cdocsch}
%    \end{macrocode}

%\iffalse
%</samplefinal>
%\fi
%
% %%%%%%%%%%%%%%%%%%%%%%%%%%%%%%%%%%%%%%
% \paragraph{Command Line Processing.}
%
% The following three command lines generate the output files
% |cdocscld|, |cdocscl1| and |cdocscl2|
% which should be identical to
% |cdocsdrf|, |cdocsch1| and |cdocsfn2|, respectively:
% \begin{center}
% \begin{tabular}{l}
% |latex -jobname cdocscld \|\\
% |  "\def\version{draft}\input{childdoc.def}\childdocforward{cdocsamp}"|\\
% |latex -jobname cdocscl1 \|\\
% |  "\input{childdoc.def}\childdocforward[cdocsamp]{cdocsch1}"|\\
% |latex -jobname cdocscl2 \|\\
% |  "\def\version{final}\input{childdoc.def}\childdocforward{cdocsch2}"|
% \end{tabular}
% \end{center}
% Note that the trailing backslash on each first line
% merely continues the input to the second line
% (for convenient cut ant paste).
% Furthermore, the command |latex| can be replaced by any
% of its alternative versions such as |pdflatex|.
%
% %%%%%%%%%%%%%%%%%%%%%%%%%%%%%%%%%%%%%%%%%%%%%%%%%%%%%%%%%%%%%%%%%%%%%%%%%%%%%%
% %%%%%%%%%%%%%%%%%%%%%%%%%%%%%%%%%%%%%%%%%%%%%%%%%%%%%%%%%%%%%%%%%%%%%%%%%%%%%%
% \section{Implementation}
%\iffalse
%<*package>
%\fi
%
% This section describes the definitions file |childdoc.def|.

% The definitions cannot be loaded using |\usepackage| or |\RequirePackage|
% which has a mechanism to prevent loading a style file more than once.
% When loading the definitions by means of |\input|
% multiple instances have to be prevented manually:
%\iffalse
%This code needs to be before the `\ProvidesFile' directive
%which is defined at the beginning of this file.
%Therefore it is also placed there and commented out here.
%</package>
%<*discard>
%\fi
%    \begin{macrocode}
\ifdefined\childdocmain\endinput\fi
%    \end{macrocode}
%\iffalse
%</discard>
%<*package>
%\fi
%
% \macro{\ifchilddoc}
% \macro{\ifchilddocmanual}
% The conditional |\ifchilddoc| tells whether a
% child (true) or main (false) document is being compiled.
% The conditional |\ifchilddocmanual| tells whether
% the |\includeonly| mechanism is used (false) or
% the selection of child files must be performed manually (true).
% The definitions initialise to false:
%    \begin{macrocode}
\newif\ifchilddoc
\newif\ifchilddocmanual
%    \end{macrocode}

% \macro{\childdocname}
% \macro{\childdocjob}
% The macro |\childdocname| stores the name of the main document
% to be compiled. The macro |\childdocjob| stores the name of
% the document on which the \LaTeX{} compiler was originally invoked.
% The content of |\jobname| cannot be compared
% to filenames specified in the source due to different catcodes.
% The following code rescans |\jobname|, stores the result
% in |\childdocname| and saves a copy in |\childdocjob|:
%    \begin{macrocode}
\edef\childdocname{\scantokens\expandafter{\jobname\noexpand}}
\let\childdocjob\childdocname
%    \end{macrocode}

% \macro{\childdocdisable}
% The macro |\childdocdisable| prevents the main file
% from being processed more than once.
% At this stage, the main document command |\childdocmain|
% is assumed to be called once again where it should do nothing.
% Any subsequent call to it should prevent
% a secondary processing of the main document
% It overwrites the forwarding commands
% |\childdocof| and |\childdocforward|
% with empty macros to prevent further inclusions of the main document:
%    \begin{macrocode}
\newcommand{\childdocdisable}
{
  \renewcommand{\childdocmain}[1]{\renewcommand{\childdocmain}[1]{\endinput}}
  \renewcommand{\childdocof}[1]{}
  \renewcommand{\childdocby}[2][]{}
  \renewcommand{\childdocforward}[2][]{}
  \renewcommand{\childdocdisable}{}
}
%    \end{macrocode}

% \macro{\childdocmain}
% The macro |\childdocmain| is to be called at the top of the main file
% with nothing or the main filename (without extension) as argument.
% First, it breaks loops.
% If the argument is not empty and does not match |\childdocname|
% (which is set by the first inclusion of |childdoc.def|),
% |\ifchilddoc| is set to true, |\includeonly| is applied to the child file
% and |\jobname| is set to the main file
% (for proper handling of |.aux| files):
%    \begin{macrocode}
\newcommand{\childdocmain}[1]
{
  \childdocdisable\childdocmain{}
  \if?#1?\else
    \begingroup
      \def\childdoctmp{#1}
      \ifx\childdoctmp\childdocname
        \def\childdoctmp{}
      \else
        \def\childdoctmp
        {
          \childdoctrue
          \includeonly{\childdocname}
          \def\childdocjob{#1}
          \def\jobname{#1}
        }
      \fi
      \expandafter
    \endgroup
    \childdoctmp
  \fi
}
%    \end{macrocode}

% \macro{\childdocof}
% The command |\childdocof| redirects
% compilation to the main file |#1|.
%    \begin{macrocode}
\newcommand{\childdocof}[1]
{
  \childdocdisable
  \childdoctrue
  \includeonly{\childdocname}
  \def\jobname{#1}
  \def\childdocjob{#1}
  \input{#1}
}
%    \end{macrocode}

% \macro{\childdocby}
% The command |\childdocby| ....
%    \begin{macrocode}
\newcommand{\childdocby}[2][]
{
  \childdocdisable
  \childdoctrue
  \childdocmanualtrue
  \if?#1?\else
    \def\jobname{#2}
  \fi
  \def\childdocjob{#2}
  \input{#2}
  \endinput
}
%    \end{macrocode}

% \macro{\childdocforward}
% The command |\childdocforward| redirects
% compilation to the main file or
% (if the optional argument is given) a child file.
% Parameters are set as if the main file
% or a child file starting with |\childdocof| was compiled.
% Then compilation is handed over to the main file:
%    \begin{macrocode}
\newcommand{\childdocforward}[2][]
{
  \begingroup
    \if?#1?
      \def\childdoctmp
      {
        \def\childdocname{#2}
        \def\childdocjob{#2}
        \def\jobname{#2}
        \input{#2}
        \endinput
      }
    \else
      \def\childdoctmp
      {
        \childdocdisable
        \def\childdocname{#2}
        \childdoctrue
        \includeonly{#2}
        \def\childdocjob{#1}
        \def\jobname{#1}
        \input{#1}
        \endinput
      }
    \fi
    \expandafter
  \endgroup
  \childdoctmp
}
%    \end{macrocode}

% \macro{\childdocforwardprefix}
% The command |\childdocforwardprefix| redirects
% compilation to the main or a child file by means of a pattern.
% The prefix |#1| in the current filename is replaced by |#2|
% and the suffix of the current filename is kept
% (it is assumed that the filename does not contain the substring `|~~~|'
% which is used as a delimiter).
% Compilation is handed over to the new file by |\childdocforward|:
%    \begin{macrocode}
\newcommand{\childdocforwardprefix}[3][]
{
  \begingroup
    \def\childdocextract #2##1~~~{\def\childdoctmp{\childdocforward[#1]{#3##1}}}
    \expandafter\childdocextract\childdocname~~~
    \expandafter
  \endgroup
  \childdoctmp
}
%    \end{macrocode}

% \macro{\childdoc}
% The deprecated macro |\childdoc| is a legacy version of |\childdocmain|:
%    \begin{macrocode}
\newcommand{\childdoc}{\childdocmain}
%    \end{macrocode}

% \macro{\childdocredirect}
% The deprecated macro |\childdocredirect| is a legacy version
% of |\childdocforward| and |\childdocforwardprefix|:
%    \begin{macrocode}
\newcommand{\childdocredirect}[2][]
{
  \begingroup
    \if?#1?
      \def\childdoctmp{\childdocforward{#2}}
    \else
      \def\childdoctmp{\childdocforwardprefix{#1}{#2}}
    \fi
    \expandafter
  \endgroup
  \childdoctmp
}
%    \end{macrocode}

%\iffalse
%</package>
%\fi
%
\endinput
|\\
|\childdocmain{|\textit{main}|}|\\
\end{tabular}
\end{center}
%
If |\jobname| does not match the argument \textit{main} of |\childdocmain|,
it is assumed that |\jobname| points to the child file to be compiled.
When using |\childdocmain| with the main file specified as argument,
it suffices to start a child file
with just |\input{|\textit{main}|}|
without loading of the package and using |\childdocof|.
If instead all processing is done
with the appropriate \textsf{childdoc} directives,
the argument of \textit{main} of |\childdocmain| can be empty.

An alternative version of the command line processing described
in \secref{sec:commandline} using the detection mechanism reads:
%
\begin{center}
|... -jobname "|\textit{target}|" "|[\textit{flags}]%
[|\def\jobname{|\textit{dest}|}|]|\input{|\textit{main}|}"|
\end{center}

%%%%%%%%%%%%%%%%%%%%%%%%%%%%%%%%%%%%%%%%%%%%%%%%%%%%%%%%%%%%%%%%%%%%%%%%%%%%%%%%
\subsection{Manual Code}
\label{sec:manual}

In case one cannot be certain whether the definitions file |childdoc.def|
is installed on the target \TeX{} distribution
and one prefers not to ship it,
it is conceivable to paste a few relevant commands into the sources.

To that end, drop all statements |% \iffalse
%
% childdoc.dtx Copyright (C) 2017-2018 Niklas Beisert
%
% This work may be distributed and/or modified under the
% conditions of the LaTeX Project Public License, either version 1.3
% of this license or (at your option) any later version.
% The latest version of this license is in
%   http://www.latex-project.org/lppl.txt
% and version 1.3 or later is part of all distributions of LaTeX
% version 2005/12/01 or later.
%
% This work has the LPPL maintenance status `maintained'.
%
% The Current Maintainer of this work is Niklas Beisert.
%
% This work consists of the files childdoc.dtx and childdoc.ins
% and the derived files childdoc.def and cdocsamp.tex with
% cdocsch1.tex, cdocsch2.tex, cdocsdrf.tex, cdocsfn1.tex, cdocsfn2.tex.
%
%<package>\ifdefined\childdocmain\endinput\fi
%<package>\ProvidesFile{childdoc.def}[2018/12/30 v2.0 child document driver]
%<samplemain>\ProvidesFile{cdocsamp.tex}[2018/12/30 v2.0 sample for childdoc]
%<*driver>
%\ProvidesFile{childdoc.drv}[2018/12/30 v2.0 childdoc reference manual file]
\PassOptionsToClass{10pt,a4paper}{article}
\documentclass{ltxdoc}

\usepackage[margin=35mm]{geometry}
\usepackage{hyperref}
\usepackage{hyperxmp}
\usepackage[usenames]{color}

\hypersetup{colorlinks=true}
\hypersetup{pdfstartview=FitH}
\hypersetup{pdfpagemode=UseNone}
\hypersetup{pdfsource={}}
\hypersetup{pdflang={en-UK}}
\hypersetup{pdfcopyright={Copyright 2017-2018 Niklas Beisert.
  This work may be distributed and/or modified under the
  conditions of the LaTeX Project Public License, either version 1.3
  of this license or (at your option) any later version.}}
\hypersetup{pdflicenseurl={http://www.latex-project.org/lppl.txt}}
\hypersetup{pdfcontactaddress={ETH Zurich, ITP, HIT K,
  Wolfgang-Pauli-Strasse 27}}
\hypersetup{pdfcontactpostcode={8093}}
\hypersetup{pdfcontactcity={Zurich}}
\hypersetup{pdfcontactcountry={Switzerland}}
\hypersetup{pdfcontactemail={nbeisert@itp.phys.ethz.ch}}
\hypersetup{pdfcontacturl={http://people.phys.ethz.ch/\xmptilde nbeisert/}}

\newcommand{\secref}[1]{\hyperref[#1]{section \ref*{#1}}}

\parskip1ex
\parindent0pt
\let\olditemize\itemize
\def\itemize{\olditemize\parskip0pt}

\begin{document}

\title{The \textsf{childdoc} Package}
\hypersetup{pdftitle={The childdoc Package}}
\author{Niklas Beisert\\[2ex]
  Institut f\"ur Theoretische Physik\\
  Eidgen\"ossische Technische Hochschule Z\"urich\\
  Wolfgang-Pauli-Strasse 27, 8093 Z\"urich, Switzerland\\[1ex]
  \href{mailto:nbeisert@itp.phys.ethz.ch}
  {\texttt{nbeisert@itp.phys.ethz.ch}}}
\hypersetup{pdfauthor={Niklas Beisert}}
\hypersetup{pdfsubject={Manual for the LaTeX2e Package childdoc}}
\date{30 December 2018, \textsf{v2.0}}
\maketitle

\begin{abstract}\noindent
\textsf{childdoc} is a \LaTeXe{} package
that enables the direct compilation
of document sections included by |\include|
to individual files.
\end{abstract}

\begingroup
\parskip0ex
\tableofcontents
\endgroup

%%%%%%%%%%%%%%%%%%%%%%%%%%%%%%%%%%%%%%%%%%%%%%%%%%%%%%%%%%%%%%%%%%%%%%%%%%%%%%%%
%%%%%%%%%%%%%%%%%%%%%%%%%%%%%%%%%%%%%%%%%%%%%%%%%%%%%%%%%%%%%%%%%%%%%%%%%%%%%%%%
\section{Introduction}

\LaTeX{} provides a mechanism to structure a large document (such as a book)
into a main file and several child files (containing the chapters)
using the |\include| command.
This mechanism is beneficial for documents
which span hundreds of pages in order to
make the source file(s) more manageable.
Moreover, compilation can be restricted to
selected child files by means of the |\includeonly| command.
The latter feature can be used to reduce the compilation time while editing
(this was significantly more useful in the earlier days of \LaTeX{})
or to generate a smaller document which is easier to navigate.
Another application of |\includeonly| is to generate
documents consisting of selected parts of the complete document.

However, there are a few drawbacks of the plain |\include| mechanism:
\begin{itemize}
\item
The child files cannot be compiled on their own,
they can only be compiled via the main file.
A naive editing environment
(such as a text editor with an option
to have the current file processed by \LaTeX)
may require one to switch to the main file before compiling;
attempting to compile the child file produces errors.
\item
The main file must be modified (each time)
to adjust the |\includeonly| command
to the present needs. This easily leaves the main file in a messy state.
\item
The generated document will always carry the filename
of the main document. This is inconvenient if
several child files are to be compiled and
to be kept for distribution.
\end{itemize}

The present package provides a simple interface
to make child files individually compilable by \LaTeX{}.
Compiling a child file then has the same effect as compiling
the main file with an |\includeonly| command
to select the appropriate child.
Moreover the generated document will carry the name of the child
rather than the main file.
This resolves all three above issues.

This feature is meant to make the editing of books,
thesis documents and lecture notes somewhat more convenient.
However, the package can also be used efficiently for
composing a series of documents (such as exercise sheets)
which are typically distributed individually.
It then assists the author in generating the individual documents
(potentially in different versions)
as well as a document containing the collected series.
Another application is in developing style files
or other kinds of included material
where compilation of the style file could redirect
to a sample or test file.

%%%%%%%%%%%%%%%%%%%%%%%%%%%%%%%%%%%%%%%%%%%%%%%%%%%%%%%%%%%%%%%%%%%%%%%%%%%%%%%%
%%%%%%%%%%%%%%%%%%%%%%%%%%%%%%%%%%%%%%%%%%%%%%%%%%%%%%%%%%%%%%%%%%%%%%%%%%%%%%%%
\section{Usage}

First of all, the package \textsf{childdoc} is \emph{not} a standard
\LaTeXe{} |.sty| style file! Therefore it needs to be invoked in
a non-standard way.

%%%%%%%%%%%%%%%%%%%%%%%%%%%%%%%%%%%%%%%%%%%%%%%%%%%%%%%%%%%%%%%%%%%%%%%%%%%%%%%%
\subsection{Included Files}
\label{sec:include}

%%%%%%%%%%%%%%%%%%%%%%%%%%%%%%%%%%%%%%%%
\DescribeMacro{\childdocmain}
To use the package, add the commands
\begin{center}
\begin{tabular}{l}
|\input{childdoc.def}|\\
|\childdocmain{}|\\
\end{tabular}
\end{center}
at the very top of the main \LaTeX{} file,
in particular \emph{before} the |\documentclass| statement!
The argument of |\childdocmain| should be left empty
(but it must be present).

%%%%%%%%%%%%%%%%%%%%%%%%%%%%%%%%%%%%%%%%
\DescribeMacro{\childdocof}
Furthermore, add the commands
\begin{center}
\begin{tabular}{l}
|\input{childdoc.def}|\\
|\childdocof{|\textit{main}|}|\\
\end{tabular}
\end{center}
at the top of every child file \textit{child}
which is included by |\include{|\textit{child}|}|
from within the main file
(or at least for those files to be compiled individually).
The argument \textit{main} must be the filename of the main file.

There are a couple of
considerations in setting up the main and child documents:

%%%%%%%%%%%%%%%%%%%%%%%%%%%%%%%%%%%%%%%%
\paragraph{Restrictions.}

Please note the following restrictions:
\begin{itemize}
\item
|\childdocmain| must be called with one argument \textit{main}
to ensure compatibility with earlier version of the package.
It must either be empty (|\childdocmain{}|)
or precisely match the filename of the main file in which it is specified.
See \secref{sec:detection} for further information.
\item
The filename \textit{main} must be specified without the |.tex| extension.
\item
The filename \textit{main} is case sensitive
(even in case-insensitive file systems)
due to internal string comparison.
\item
The argument \textit{main} should be fully expanded, it cannot be a macro.
\item
Subdirectories and special characters should be avoided in filenames.
\item
The command |\childdocmain{|\textit{main}|}| must be followed by a whitespace.
It should not be followed immediately by another command
or by a comment mark `|%|'.
This is because the \TeX{} parser reads the token immediately following
the argument of |\childdocmain| and puts it
at the beginning of every child section;
however, a white\-space is ignored.
\end{itemize}

%%%%%%%%%%%%%%%%%%%%%%%%%%%%%%%%%%%%%%%%
\paragraph{Content of Main File.}

It is advisable to place all content in the child files included by |\include|.
Any output contained in the main file will appear in all child documents
unless suppressed manually;
it cannot be suppressed automatically by the |\includeonly| directive
and thus should normally be avoided.
A method to include some content in the main file
by means of conditional processing is described in \secref{sec:conditional}.

%%%%%%%%%%%%%%%%%%%%%%%%%%%%%%%%%%%%%%%%
\paragraph{Page Numbering.}

When only a part of the document is compiled,
the appropriate numbering of pages
(as well as other status parameters)
is determined from the |.aux| files.
The latter contain information from previous passes.
However this information needs to propagate through
all intermediate child documents.
Therefore the page numbering in child documents may well
be inconsistent until the complete document is compiled at least once.

A useful (if unconventional) way to always ensure a consistent
page numbering is to restart the numbering in each child document
and denote the pages by `\textit{child}|.|\textit{page}'
where \textit{child} represents the chapter/section number of the child file.
This can be achieved by the command
|\numberwithin{page}{|\textit{child}|}|
of the \textsf{amsmath} package
where \textit{child} can be |chapter| or |section|
depending on the chosen structuring.
Alternatively, one can modify the macro |\thepage| appropriately
and reset the counter |page| at the start of each child file.

%%%%%%%%%%%%%%%%%%%%%%%%%%%%%%%%%%%%%%%%%%%%%%%%%%%%%%%%%%%%%%%%%%%%%%%%%%%%%%%%
\subsection{Conditional Processing}
\label{sec:conditional}

The package provides a mechanism to compile different versions
of a document. To customise the versions further some conditional processing
can come in handy to distinguish which version is being compiled.
The package provides two macros to describe the compilation context:

%%%%%%%%%%%%%%%%%%%%%%%%%%%%%%%%%%%%%%%%
\DescribeMacro{\ifchilddoc}
The conditional |\ifchilddoc| distinguishes between the compilation of
child documents and the main document:
%
\begin{center}
|\ifchilddoc |\textit{child-code}| |[|\||else |\textit{main-code}]| \||fi|
\end{center}

%%%%%%%%%%%%%%%%%%%%%%%%%%%%%%%%%%%%%%%%
\DescribeMacro{\childdocname}
\DescribeMacro{\childdocjob}
The macro |\childdocname| contains the filename (without extension)
of the main or child file being processed.
Note that |\childdocjob| will always contain the name of the main file.

%%%%%%%%%%%%%%%%%%%%%%%%%%%%%%%%%%%%%%%%
\paragraph{Title Page.}

Conditional processing can be used to include a title or banner page
in the main document when proper precautions are taken.
Importantly, the code in the main file should ensure that the page counter
(as well as other status parameters which are stored in the |.aux| files)
takes the same value after the conditional processing.
Otherwise the page numbers may take divergent values
depending on which part is compiled.

For example, a title page could be declared by:
%
\begin{center}
\begin{tabular}{l}
|\ifchilddoc\||else|\\
|\addtocounter{page}{-1}|\\
\textit{code for title page}\\
|\newpage|\\
|\||fi|
\end{tabular}
\end{center}
%
A banner page for the child documents can be generated by:
%
\begin{center}
\begin{tabular}{l}
|\ifchilddoc|\\
|\addtocounter{page}{-1}|\\
\textit{code for banner page}\\
|\newpage|\\
|\||fi|
\end{tabular}
\end{center}
%
Here one could write a message such as:
\begin{center}
|This is the part \childdocname{} of \childdocjob{}.|
\end{center}

%%%%%%%%%%%%%%%%%%%%%%%%%%%%%%%%%%%%%%%%%%%%%%%%%%%%%%%%%%%%%%%%%%%%%%%%%%%%%%%%
\subsection{Flags}
\label{sec:flags}

The package makes it easy to generate different versions
of the main or child documents.
To this end compilation flags can be defined
and assigned different default values.
They will be particularly useful in conjunction
with the forwarding mechanism described in \secref{sec:forward}.

For example, it may be useful to have a flag |\version|
which can be set to |draft| or |final|.
The document source will contain some conditional code
depending on the value of |\version|.
Suppose further, the flag should default to |final| for the main file
and to |draft| for child files
which is a natural assignment for editing the document.
This is achieved by placing the following code
in the preamble of the main document
(below the |\childdocmain| directive):
%
\begin{center}
\begin{tabular}{l}
|\ifchilddoc|\\
|\providecommand{\version}{draft}|\\
|\||else|\\
|\providecommand{\version}{final}|\\
|\||fi|
\end{tabular}
\end{center}
%
The definition by |\providecommand| makes sure
that previous definitions are not overwritten.
Further statements |\providecommand{\version}{...}|
can thus be added before the above code to override it.

For the main file, one might add a line
(between |\childdocmain| and the above block)
%
\begin{center}
|%\ifchilddoc\||else\providecommand{\version}{draft}\||fi|
\end{center}
%
which can be uncommented to produce a draft version.
Likewise one can add a line to the very top of a child file
(above the |\childdocof{|\textit{main}|}| directive)
%
\begin{center}
|%\providecommand{\version}{final}|
\end{center}
%
which can be uncommented to produce the final version of this child document.

%%%%%%%%%%%%%%%%%%%%%%%%%%%%%%%%%%%%%%%%%%%%%%%%%%%%%%%%%%%%%%%%%%%%%%%%%%%%%%%%
\subsection{Forwarding}
\label{sec:forward}

Different versions of the main or child documents
using compilation flags as described in \secref{sec:flags}
can be (permanently) stored in different files
for convenient compilation, viewing and distribution.
To this end, the package defines a command
to pass on compilation to a different file:

%%%%%%%%%%%%%%%%%%%%%%%%%%%%%%%%%%%%%%%%
\DescribeMacro{\childdocforward}
The command |\childdocforward| redirects processing to
another source file:
%
\begin{center}
\begin{tabular}{l}
|\input{childdoc.def}|\\
|\childdocforward[|\textit{main}|]{|\textit{dest}|}|\\
\end{tabular}
\end{center}
%
The argument \textit{dest} is the destination file
(without extension).
It should be the main file or one of the child files.
Note that further \textsf{childdoc} directives
such as |\childdocof| and |\childdocforward|
in the indicated file will be processed in this form.
The optional argument \textit{main}
passes on directly to the main file \textit{main}
while pretending to compile the child \textit{dest}.
This form behaves as if \textit{dest}
issues |\childdocof{|\textit{main}|}| right away,
and no further \textsf{childdoc} directives will be processed.

%%%%%%%%%%%%%%%%%%%%%%%%%%%%%%%%%%%%%%%%
\DescribeMacro{\...prefix}
In the alternative form |\childdocforwardprefix|,
%
\begin{center}
\begin{tabular}{l}
|\input{childdoc.def}|\\
|\childdocforwardprefix[|\textit{main}|]{|\textit{prefix}|}{|\textit{dest}|}|
\end{tabular}
\end{center}
%
the destination file is determined by a pattern
depending on the current file:
To make this work, the current file must be called
`{\textit{prefix}\hspace{0.2em}\textit{suffix}}'
with \textit{prefix} matching precisely the argument.
Processing is then passed on to the file
`{\textit{dest}\hspace{0.2em}\textit{suffix}}'.
Surely, the same effect is achieved by
directly specifying the
argument `{\textit{dest}\hspace{0.2em}\textit{suffix}}'
in the first form.
However, that requires to set up a different file
for each child. With the alternative form of the command
all these files can have exactly the same content
which simplifies setting them up and maintaining them.

For example, the following file |draft.tex|
with a compilation flag |\version| as described in \secref{sec:flags}
compiles the main document as a draft:
%
\begin{center}
\begin{tabular}{l}
|\def\version{draft}|\\
|\input{childdoc.def}|\\
|\childdocforward{|\textit{main}|}|
\end{tabular}
\end{center}
%
Likewise, the following files |final|\textit{nn}|.tex|
compile the final version of the child document
|child|\textit{nn}|.tex|:
%
\begin{center}
\begin{tabular}{l}
|\def\version{final}|\\
|\input{childdoc.def}|\\
|\childdocforwardprefix{final}{child}|
\end{tabular}
\end{center}
%

Note that when several versions of a main file and/or of each child file
are to be generated, it may be convenient to set up a |Makefile| or
shell script to automatise the process.

%%%%%%%%%%%%%%%%%%%%%%%%%%%%%%%%%%%%%%%%%%%%%%%%%%%%%%%%%%%%%%%%%%%%%%%%%%%%%%%%
\subsection{Command Line Processing}
\label{sec:commandline}

The effect of redirection files can also be achieved by invoking
the \LaTeX{} compiler with a more elaborate command line.
Most conveniently this should be done as part
of a shell script or a |Makefile|.

When using \textsf{childdoc} in the main file, the following
command lines effectively perform a redirection
(note that depending on the shell being used,
backslashes may have to be doubled: `|\|' $\to$ `|\\|'):
%
\begin{center}
|... -jobname "|\textit{target}|" |\\|"|[\textit{flags}]%
|\input{childdoc.def}\childdocforward[|\textit{main}|]{|\textit{dest}|}"|
\end{center}
%
Here \textit{target} is the name of the output file,
\textit{main} is the name of the main file
and \textit{dest} is the name of the main or child file to be processed
(all filenames without extensions).
The optional argument \textit{main} can be omitted
if \textit{main} matches \textit{dest}.
Optionally, compilation \textit{flags} can be defined via |\def| commands.
This command line makes the \TeX{} engine believe
it is compiling the file \textit{target}
whose content is specified as the latter parameter.
The provided code then forwards the processing to
\textit{main} or \textit{dest} as described in \secref{sec:forward}.

%%%%%%%%%%%%%%%%%%%%%%%%%%%%%%%%%%%%%%%%%%%%%%%%%%%%%%%%%%%%%%%%%%%%%%%%%%%%%%%%
\subsection{Include by Input}
\label{sec:input}

Including child documents by |\include| has some restrictions by design.
Most notably, the content of a child document always occupies
its own set of pages; pages cannot be shared between child documents.
Usually, this behaviour makes perfect sense
because each child document contain an essential part of the document.
However, in some situations it may be desirable to compose
a document from a collection of parts
without having mandatory page breaks between then.
For this case, the package
provides a mechanism to include parts
by |\input| which can also be processed individually.
However, by construction this mechanism
requires manual handling of the content to be output.

%%%%%%%%%%%%%%%%%%%%%%%%%%%%%%%%%%%%%%%%
\DescribeMacro{\ifchilddocmanual}
The main file should be prepared as usual, see \secref{sec:include}.
However, the document body must make a distinction
between processing of an individual part and of the main document, e.g.:
%
\begin{center}
\begin{tabular}{l}
|\ifchilddocmanual|\\
|\input{\childdocname}|\\
|\||else|\\
\textit{document body with }|\input{|\textit{part}|}|\\
|\||fi|
\end{tabular}
\end{center}
%
The conditional |\ifchilddocmanual| is true whenever
a part to be included by |\input| is being compiled,
and the name of the part is stored in |\childdocname|.

%%%%%%%%%%%%%%%%%%%%%%%%%%%%%%%%%%%%%%%%
\DescribeMacro{\childdocby}
Each part to be included by |\input| should start with:
%
\begin{center}
\begin{tabular}{l}
|\input{childdoc.def}|\\
|\childdocby{|\textit{main}|}|\\
\end{tabular}
\end{center}
%
The directive |\childdocby| is similar to |\childdocof|
described in \secref{sec:include},
but the subsequent selection of content must be done manually.
To that end, both |\ifchilddoc| and |\ifchilddocmanual|
will be true upon processing of a part,
and the name of the part is stored in |\childdocname|.
Note that |\jobname| will be set to the filename of the current part
so that each part receives an individual |.aux| file
that does not interfere with the |.aux| file(s) of the main document.
This behaviour can be altered by the alternative form
|\childdocby[*]{|\textit{main}|}| (with a non-empty optional argument)
which uses the |.aux| file of the main document
by setting |\jobname| to \textit{main}.

%%%%%%%%%%%%%%%%%%%%%%%%%%%%%%%%%%%%%%%%%%%%%%%%%%%%%%%%%%%%%%%%%%%%%%%%%%%%%%%%
\subsection{Driver Development}
\label{sec:driver}

The \textsf{childdoc} mechanism can also be use for the development
of definition files such as \LaTeX{} styles or classes.
This case differs from the above setup with multiple parts
included by |\include| in that no |\includeonly| should be invoked.
This can be achieved by starting the include file
(before |\ProvidesPackage|) with:
%
\begin{center}
\begin{tabular}{l}
|\input{childdoc.def}|\\
|\childdocforward{|\textit{main}|}|\\
\end{tabular}
\end{center}
%
or alternatively with:
%
\begin{center}
\begin{tabular}{l}
|\input{childdoc.def}|\\
|\childdocby{|\textit{main}|}|\\
\end{tabular}
\end{center}
%
Both forms have slightly different effects as described above.
The main file is prepared as usual, see \secref{sec:include}.

%%%%%%%%%%%%%%%%%%%%%%%%%%%%%%%%%%%%%%%%%%%%%%%%%%%%%%%%%%%%%%%%%%%%%%%%%%%%%%%%
\subsection{Legacy Detection}
\label{sec:detection}

The directive |\childdocmain| in the main file can detect
whether the complete document or merely a child is to be compiled
even without using the directive |\childdocof|.
This method is deprecated because it is less robust
and there is no compelling reason to use it;
it is merely provided for backward compatibility
and it may be removed in future versions.

If the detection mechanism is to be used,
it is mandatory to correctly specify
the filename of the main file as the argument of |\childdocmain|:
%
\begin{center}
\begin{tabular}{l}
|\input{childdoc.def}|\\
|\childdocmain{|\textit{main}|}|\\
\end{tabular}
\end{center}
%
If |\jobname| does not match the argument \textit{main} of |\childdocmain|,
it is assumed that |\jobname| points to the child file to be compiled.
When using |\childdocmain| with the main file specified as argument,
it suffices to start a child file
with just |\input{|\textit{main}|}|
without loading of the package and using |\childdocof|.
If instead all processing is done
with the appropriate \textsf{childdoc} directives,
the argument of \textit{main} of |\childdocmain| can be empty.

An alternative version of the command line processing described
in \secref{sec:commandline} using the detection mechanism reads:
%
\begin{center}
|... -jobname "|\textit{target}|" "|[\textit{flags}]%
[|\def\jobname{|\textit{dest}|}|]|\input{|\textit{main}|}"|
\end{center}

%%%%%%%%%%%%%%%%%%%%%%%%%%%%%%%%%%%%%%%%%%%%%%%%%%%%%%%%%%%%%%%%%%%%%%%%%%%%%%%%
\subsection{Manual Code}
\label{sec:manual}

In case one cannot be certain whether the definitions file |childdoc.def|
is installed on the target \TeX{} distribution
and one prefers not to ship it,
it is conceivable to paste a few relevant commands into the sources.

To that end, drop all statements |\input{childdoc.def}|
and perform the replacements as outlined below.
Instead of |\childdocmain{|\textit{main}|}| add the following code
to the top of the main file:
%
\begin{center}
\begin{tabular}{l}
|\||ifdefined\childdocname\endinput\||fi\newif\ifchilddoc|\\
|\edef\childdocname{\scantokens\expandafter{\jobname\noexpand}}|\\
|\def\childdocmain{|\textit{main}|}\||ifx\childdocmain\childdocname\||else|\\
|\childdoctrue\includeonly{\childdocname}\let\jobname\childdocmain\||fi|\\
\end{tabular}
\end{center}
%
Instead of |\childdocof{|\textit{main}|}| just include the main file
at the top of each child file:
%
\begin{center}
|\input{|\textit{main}|}|
\end{center}
%
A simple redirection |\childdocforward{|\textit{dest}|}| is achieved by:
%
\begin{center}
|\def\jobname{|\textit{dest}|}\input{\jobname}|
\end{center}
%
The redirection with prefix
|\childdocforwardprefix[|\textit{prefix}|]{|\textit{dest}|}|
is accomplished by:
%
\begin{center}
\begin{tabular}{l}
|{\edef\jobname{\scantokens\expandafter{\jobname\noexpand}}|\\
|\def\redirectjob |\textit{prefix}|#1~~~{\gdef\jobname{|\textit{dest}|#1}}|\\
|\expandafter\redirectjob\jobname~~~}\input{\jobname}|
\end{tabular}
\end{center}

In an alternative approach,
child documents can be compiled by a specific command line
without additional code or specific definitions:
%
\begin{center}
|... -jobname "|\textit{target}|" "|[\textit{flags}]%
|\includeonly{|\textit{dest}|}\input{|\textit{main}|}"|
\end{center}
%

%%%%%%%%%%%%%%%%%%%%%%%%%%%%%%%%%%%%%%%%%%%%%%%%%%%%%%%%%%%%%%%%%%%%%%%%%%%%%%%%
%%%%%%%%%%%%%%%%%%%%%%%%%%%%%%%%%%%%%%%%%%%%%%%%%%%%%%%%%%%%%%%%%%%%%%%%%%%%%%%%
\section{Information}

%%%%%%%%%%%%%%%%%%%%%%%%%%%%%%%%%%%%%%%%%%%%%%%%%%%%%%%%%%%%%%%%%%%%%%%%%%%%%%%%
\subsection{Copyright}

Copyright \copyright{} 2017--2018 Niklas Beisert

This work may be distributed and/or modified under the
conditions of the \LaTeX{} Project Public License, either version 1.3
of this license or (at your option) any later version.
The latest version of this license is in
  \url{http://www.latex-project.org/lppl.txt}
and version 1.3 or later is part of all distributions of \LaTeX{}
version 2005/12/01 or later.

This work has the LPPL maintenance status `maintained'.

The Current Maintainer of this work is Niklas Beisert.

This work consists of the files |README.txt|, |childdoc.ins| and |childdoc.dtx|
as well as the derived files |childdoc.def|, |cdocsamp.tex|
with |cdocsch1.tex|, |cdocsch2.tex|, |cdocspt3.tex|, |cdocspt4.tex|,
|cdocsdrf.tex|, |cdocsfn1.tex|, |cdocsfn2.tex|
as well as |childdoc.pdf|.

%%%%%%%%%%%%%%%%%%%%%%%%%%%%%%%%%%%%%%%%%%%%%%%%%%%%%%%%%%%%%%%%%%%%%%%%%%%%%%%%
\subsection{Files and Installation}

The package consists of the files:
%
\begin{center}
\begin{tabular}{ll}
    |README.txt|   & readme file \\
    |childdoc.ins| & installation file \\
    |childdoc.dtx| & source file \\
    |childdoc.def| & definition file \\
    |cdocsamp.tex| & sample main file \\
    |cdocsch1.tex| & sample include file \\
    |cdocsch2.tex| & sample include file \\
    |cdocspt3.tex| & sample part file \\
    |cdocspt4.tex| & sample part file \\
    |cdocsdrf.tex| & sample redirection file \\
    |cdocsfn1.tex| & sample redirection file \\
    |cdocsfn2.tex| & sample redirection file \\
    |childdoc.pdf| & manual
\end{tabular}
\end{center}
%
The distribution consists of the files
|README.txt|, |childdoc.ins| and |childdoc.dtx|.
%
\begin{itemize}
\item
Run (pdf)\LaTeX{} on |childdoc.dtx|
to compile the manual |childdoc.pdf| (this file).
\item
Run \LaTeX{} on |childdoc.ins| to create the definitions file |childdoc.def|
and the sample |cdocsamp.tex| with include files
|cdocsch1.tex|, |cdocsch2.tex|, |cdocspt3.tex|, |cdocspt4.tex|,
|cdocsdrf.tex|, |cdocsfn1.tex|, |cdocsfn2.tex|.
Then copy the file |childdoc.def| to an appropriate directory of your \LaTeX{}
distribution, e.g.\ \textit{texmf-root}|/tex/latex/childdoc|.
\end{itemize}

%%%%%%%%%%%%%%%%%%%%%%%%%%%%%%%%%%%%%%%%%%%%%%%%%%%%%%%%%%%%%%%%%%%%%%%%%%%%%%%%
\subsection{Related CTAN Packages}

There are several other packages which offer a similar functionality:
%
\begin{itemize}
\item
The packages
\href{http://ctan.org/pkg/docmute}{\textsf{docmute}},
\href{http://ctan.org/pkg/includex}{\textsf{includex}} and
\href{http://ctan.org/pkg/standalone}{\textsf{standalone}}
provide commands to include only the document body of
a child file thus allowing both files to be compiled individually.
\item
The packages \href{http://ctan.org/pkg/subdocs}{\textsf{subdocs}}
and \href{http://ctan.org/pkg/subfiles}{\textsf{subfiles}}
provide structures in which the main and child documents can be
encapsulated and allowing them to be compiled individually.
The inclusion mechanism is different from the conventional |\include|.
\item
The package \href{http://ctan.org/pkg/combine}{\textsf{combine}}
is an elaborate solution to combine several documents into one.
\end{itemize}
%
See also the CTAN topic \href{http://ctan.org/topic/subdocs}{\textsf{subdocs}}
for further related packages.
The present package differs from the above solutions in that
a document structure constructed with the conventional |\include| mechanism
just needs two extra commands at the top of every file
such that all constituent files can be compiled individually.

%%%%%%%%%%%%%%%%%%%%%%%%%%%%%%%%%%%%%%%%%%%%%%%%%%%%%%%%%%%%%%%%%%%%%%%%%%%%%%%%
%\subsection{Feature Suggestions}
%
%The following is a list of features which may be useful for future
%versions of this package:
%%
%\begin{itemize}
%\item
%\ldots
%\end{itemize}

%%%%%%%%%%%%%%%%%%%%%%%%%%%%%%%%%%%%%%%%%%%%%%%%%%%%%%%%%%%%%%%%%%%%%%%%%%%%%%%%
\subsection{Revision History}

%%%%%%%%%%%%%%%%%%%%%%%%%%%%%%%%%%%%%%%%
\paragraph{v2.0:} 2018/12/30

\begin{itemize}
\item
immediate forward processing
\item
added |\childdocby| mechanism
\item
manual restructured
\end{itemize}

%%%%%%%%%%%%%%%%%%%%%%%%%%%%%%%%%%%%%%%%
\paragraph{v1.6:} 2018/01/17

\begin{itemize}
\item
application for development of include files
\item
corrections to manual
\end{itemize}

%%%%%%%%%%%%%%%%%%%%%%%%%%%%%%%%%%%%%%%%
\paragraph{v1.5:} 2017/05/21

\begin{itemize}
\item
more complete structuring introduced
\item
|\childdocof| introduced
\item
|\childdoc| renamed to |\childdocmain|
\item
|\childredirect| renamed to |\childdocforward| and |\childdocforwardprefix|
and functionality expanded
\end{itemize}

%%%%%%%%%%%%%%%%%%%%%%%%%%%%%%%%%%%%%%%%
\paragraph{v1.0:} 2017/04/27

\begin{itemize}
\item
manual and install package
\item
first version published on CTAN
\end{itemize}

%%%%%%%%%%%%%%%%%%%%%%%%%%%%%%%%%%%%%%%%
\paragraph{v0.6:} 2017/04/26

\begin{itemize}
\item
redirection mechanism added
\end{itemize}

%%%%%%%%%%%%%%%%%%%%%%%%%%%%%%%%%%%%%%%%
\paragraph{v0.5:} 2017/04/26

\begin{itemize}
\item
functionality in definition file
\end{itemize}


%%%%%%%%%%%%%%%%%%%%%%%%%%%%%%%%%%%%%%%%%%%%%%%%%%%%%%%%%%%%%%%%%%%%%%%%%%%%%%%%
%%%%%%%%%%%%%%%%%%%%%%%%%%%%%%%%%%%%%%%%%%%%%%%%%%%%%%%%%%%%%%%%%%%%%%%%%%%%%%%%
%%%%%%%%%%%%%%%%%%%%%%%%%%%%%%%%%%%%%%%%%%%%%%%%%%%%%%%%%%%%%%%%%%%%%%%%%%%%%%%%
\appendix

\settowidth\MacroIndent{\rmfamily\scriptsize 000\ }

 \DocInput{childdoc.dtx}

\end{document}
%</driver>
% \fi
%
% %%%%%%%%%%%%%%%%%%%%%%%%%%%%%%%%%%%%%%%%%%%%%%%%%%%%%%%%%%%%%%%%%%%%%%%%%%%%%%
% %%%%%%%%%%%%%%%%%%%%%%%%%%%%%%%%%%%%%%%%%%%%%%%%%%%%%%%%%%%%%%%%%%%%%%%%%%%%%%
% \section{Sample}
%\iffalse
%<*samplemain>
%\fi
%
% The following presents a sample document
% with two chapters, two parts, a title page,
% a compile flag as well as three forwarding files to set the flag.
% It consists of eight |.tex| files:
% \begin{center}
% \begin{tabular}{ll}
% |cdocsamp.tex|&main file\\
% |cdocsch1.tex|&include file for chapter 1\\
% |cdocsch2.tex|&include file for chapter 2\\
% |cdocspt3.tex|&include file for part 3\\
% |cdocspt4.tex|&include file for part 4\\
% |cdocsdrf.tex|&forwarding file for main file in draft mode\\
% |cdocsfi1.tex|&forwarding file for final version of chapter 1\\
% |cdocsfi2.tex|&forwarding file for final version of chapter 2\\
% \end{tabular}
% \end{center}
% Each of the eight files can be compiled directly by the \LaTeX{} compiler.
%
% %%%%%%%%%%%%%%%%%%%%%%%%%%%%%%%%%%%%%%
% \paragraph{Main File.}
%
% The main file is called |cdocsamp.tex|.
%
% Load the \textsf{childdoc} definitions and
% declare the filename for the main document:
%    \begin{macrocode}
\input{childdoc.def}
\childdocmain{}
%    \end{macrocode}

% Optional override for |\version| flag:
%    \begin{macrocode}
%%\ifchilddoc\else\providecommand{\version}{draft}\fi
%    \end{macrocode}

% Define the default values for the |\version| flag
% (|final| for the main file and |draft| for childs):
%    \begin{macrocode}
\ifchilddoc
\providecommand{\version}{draft}
\else
\providecommand{\version}{final}
\fi
%    \end{macrocode}

% Load the standard document class:
%    \begin{macrocode}
\documentclass[12pt]{article}
%    \end{macrocode}

% Start the document body:
%    \begin{macrocode}
\begin{document}
%    \end{macrocode}

% Declare a title page.
% Print title, part of document being processed and version flag:
%    \begin{macrocode}
\addtocounter{page}{-1}
\begin{center}
{\LARGE\bfseries{}childdoc example\par}
\vspace{1cm}
\ifchilddoc
\ifchilddocmanual part\else chapter\fi:
`\childdocname' of `\childdocjob'\par
\else
main document: `\childdocjob'\par
\fi
version: \version\par
\end{center}
\newpage
%    \end{macrocode}

% Manually include selected file,
% otherwise process as usual:
%    \begin{macrocode}
\ifchilddocmanual
\section*{part `\childdocname'}
\input{\childdocname}
\else
%    \end{macrocode}

% Include the two chapters:
%    \begin{macrocode}
\include{cdocsch1}
\include{cdocsch2}
%    \end{macrocode}

% Include the two parts unless only chapters should be displayed:
%    \begin{macrocode}
\ifchilddoc\else
\section{part three}
\input{cdocspt3}
\section{part four}
\input{cdocspt4}
\fi
%    \end{macrocode}

% Process as usual until here:
%    \begin{macrocode}
\fi
%    \end{macrocode}

% End of document body:
%    \begin{macrocode}
\end{document}
%    \end{macrocode}
%\iffalse
%</samplemain>
%\fi
%
% %%%%%%%%%%%%%%%%%%%%%%%%%%%%%%%%%%%%%%
% \paragraph{Chapter Include Files.}
%
% The include files are called |cdocsch1.tex| and |cdocsch2.tex|.
%
%\iffalse
%<*samplechap1|samplechap2>
%\fi

% Optional override for |\version| flag:
%    \begin{macrocode}
%%\providecommand{\version}{final}
%    \end{macrocode}

% Include the main document:
%    \begin{macrocode}
\input{childdoc.def}
\childdocof{cdocsamp}
%    \end{macrocode}

%\iffalse
%</samplechap1|samplechap2>
%\fi
%
%\iffalse
%<*samplechap1>
%\fi
% Some text for chapter 1:
%    \begin{macrocode}
\section{one}
some text in chapter one
%    \end{macrocode}

%\iffalse
%</samplechap1>
%\fi
% Some text for chapter 2:
%\iffalse
%<*samplechap2>
%\fi
%    \begin{macrocode}
\section{two}
more text in chapter two
%    \end{macrocode}

%\iffalse
%</samplechap2>
%\fi
%
% %%%%%%%%%%%%%%%%%%%%%%%%%%%%%%%%%%%%%%
% \paragraph{Part Include Files.}
%
% The include files are called |cdocspt3.tex| and |cdocspt4.tex|.
%
%\iffalse
%<*samplepart3|samplepart4>
%\fi

% Optional override for |\version| flag:
%    \begin{macrocode}
%%\providecommand{\version}{final}
%    \end{macrocode}

% Include the main document:
%    \begin{macrocode}
\input{childdoc.def}
\childdocby{cdocsamp}
%    \end{macrocode}

%\iffalse
%</samplepart3|samplepart4>
%\fi
%
%\iffalse
%<*samplepart3>
%\fi
% Some text for part 3:
%    \begin{macrocode}
some text in part three
%    \end{macrocode}

%\iffalse
%</samplepart3>
%\fi
% Some text for part 4:
%\iffalse
%<*samplepart4>
%\fi
%    \begin{macrocode}
more text in part four
%    \end{macrocode}

%\iffalse
%</samplepart4>
%\fi
%
% %%%%%%%%%%%%%%%%%%%%%%%%%%%%%%%%%%%%%%
% \paragraph{Forwarding for a Complete Draft.}
%
% The following forwarding file |cdocsdrf.tex|
% compiles the main document in draft mode:
%\iffalse
%<*sampledraft>
%\fi
%    \begin{macrocode}
\def\version{draft}
\input{childdoc.def}
\childdocforward{cdocsamp}
%    \end{macrocode}

%\iffalse
%</sampledraft>
%\fi
%
% %%%%%%%%%%%%%%%%%%%%%%%%%%%%%%%%%%%%%%
% \paragraph{Forwarding for Final Version of the Chapters.}
%
% The following forwarding files |cdocsfn1.tex| and |cdocsfn2.tex|
% (with identical content)
% compile the final versions of the child documents
% |cdocsch1.tex| and |cdocsch2.tex|, respectively:
%\iffalse
%<*samplefinal>
%\fi
%    \begin{macrocode}
\def\version{final}
\input{childdoc.def}
\childdocforwardprefix[cdocsamp]{cdocsfn}{cdocsch}
%    \end{macrocode}

%\iffalse
%</samplefinal>
%\fi
%
% %%%%%%%%%%%%%%%%%%%%%%%%%%%%%%%%%%%%%%
% \paragraph{Command Line Processing.}
%
% The following three command lines generate the output files
% |cdocscld|, |cdocscl1| and |cdocscl2|
% which should be identical to
% |cdocsdrf|, |cdocsch1| and |cdocsfn2|, respectively:
% \begin{center}
% \begin{tabular}{l}
% |latex -jobname cdocscld \|\\
% |  "\def\version{draft}\input{childdoc.def}\childdocforward{cdocsamp}"|\\
% |latex -jobname cdocscl1 \|\\
% |  "\input{childdoc.def}\childdocforward[cdocsamp]{cdocsch1}"|\\
% |latex -jobname cdocscl2 \|\\
% |  "\def\version{final}\input{childdoc.def}\childdocforward{cdocsch2}"|
% \end{tabular}
% \end{center}
% Note that the trailing backslash on each first line
% merely continues the input to the second line
% (for convenient cut ant paste).
% Furthermore, the command |latex| can be replaced by any
% of its alternative versions such as |pdflatex|.
%
% %%%%%%%%%%%%%%%%%%%%%%%%%%%%%%%%%%%%%%%%%%%%%%%%%%%%%%%%%%%%%%%%%%%%%%%%%%%%%%
% %%%%%%%%%%%%%%%%%%%%%%%%%%%%%%%%%%%%%%%%%%%%%%%%%%%%%%%%%%%%%%%%%%%%%%%%%%%%%%
% \section{Implementation}
%\iffalse
%<*package>
%\fi
%
% This section describes the definitions file |childdoc.def|.

% The definitions cannot be loaded using |\usepackage| or |\RequirePackage|
% which has a mechanism to prevent loading a style file more than once.
% When loading the definitions by means of |\input|
% multiple instances have to be prevented manually:
%\iffalse
%This code needs to be before the `\ProvidesFile' directive
%which is defined at the beginning of this file.
%Therefore it is also placed there and commented out here.
%</package>
%<*discard>
%\fi
%    \begin{macrocode}
\ifdefined\childdocmain\endinput\fi
%    \end{macrocode}
%\iffalse
%</discard>
%<*package>
%\fi
%
% \macro{\ifchilddoc}
% \macro{\ifchilddocmanual}
% The conditional |\ifchilddoc| tells whether a
% child (true) or main (false) document is being compiled.
% The conditional |\ifchilddocmanual| tells whether
% the |\includeonly| mechanism is used (false) or
% the selection of child files must be performed manually (true).
% The definitions initialise to false:
%    \begin{macrocode}
\newif\ifchilddoc
\newif\ifchilddocmanual
%    \end{macrocode}

% \macro{\childdocname}
% \macro{\childdocjob}
% The macro |\childdocname| stores the name of the main document
% to be compiled. The macro |\childdocjob| stores the name of
% the document on which the \LaTeX{} compiler was originally invoked.
% The content of |\jobname| cannot be compared
% to filenames specified in the source due to different catcodes.
% The following code rescans |\jobname|, stores the result
% in |\childdocname| and saves a copy in |\childdocjob|:
%    \begin{macrocode}
\edef\childdocname{\scantokens\expandafter{\jobname\noexpand}}
\let\childdocjob\childdocname
%    \end{macrocode}

% \macro{\childdocdisable}
% The macro |\childdocdisable| prevents the main file
% from being processed more than once.
% At this stage, the main document command |\childdocmain|
% is assumed to be called once again where it should do nothing.
% Any subsequent call to it should prevent
% a secondary processing of the main document
% It overwrites the forwarding commands
% |\childdocof| and |\childdocforward|
% with empty macros to prevent further inclusions of the main document:
%    \begin{macrocode}
\newcommand{\childdocdisable}
{
  \renewcommand{\childdocmain}[1]{\renewcommand{\childdocmain}[1]{\endinput}}
  \renewcommand{\childdocof}[1]{}
  \renewcommand{\childdocby}[2][]{}
  \renewcommand{\childdocforward}[2][]{}
  \renewcommand{\childdocdisable}{}
}
%    \end{macrocode}

% \macro{\childdocmain}
% The macro |\childdocmain| is to be called at the top of the main file
% with nothing or the main filename (without extension) as argument.
% First, it breaks loops.
% If the argument is not empty and does not match |\childdocname|
% (which is set by the first inclusion of |childdoc.def|),
% |\ifchilddoc| is set to true, |\includeonly| is applied to the child file
% and |\jobname| is set to the main file
% (for proper handling of |.aux| files):
%    \begin{macrocode}
\newcommand{\childdocmain}[1]
{
  \childdocdisable\childdocmain{}
  \if?#1?\else
    \begingroup
      \def\childdoctmp{#1}
      \ifx\childdoctmp\childdocname
        \def\childdoctmp{}
      \else
        \def\childdoctmp
        {
          \childdoctrue
          \includeonly{\childdocname}
          \def\childdocjob{#1}
          \def\jobname{#1}
        }
      \fi
      \expandafter
    \endgroup
    \childdoctmp
  \fi
}
%    \end{macrocode}

% \macro{\childdocof}
% The command |\childdocof| redirects
% compilation to the main file |#1|.
%    \begin{macrocode}
\newcommand{\childdocof}[1]
{
  \childdocdisable
  \childdoctrue
  \includeonly{\childdocname}
  \def\jobname{#1}
  \def\childdocjob{#1}
  \input{#1}
}
%    \end{macrocode}

% \macro{\childdocby}
% The command |\childdocby| ....
%    \begin{macrocode}
\newcommand{\childdocby}[2][]
{
  \childdocdisable
  \childdoctrue
  \childdocmanualtrue
  \if?#1?\else
    \def\jobname{#2}
  \fi
  \def\childdocjob{#2}
  \input{#2}
  \endinput
}
%    \end{macrocode}

% \macro{\childdocforward}
% The command |\childdocforward| redirects
% compilation to the main file or
% (if the optional argument is given) a child file.
% Parameters are set as if the main file
% or a child file starting with |\childdocof| was compiled.
% Then compilation is handed over to the main file:
%    \begin{macrocode}
\newcommand{\childdocforward}[2][]
{
  \begingroup
    \if?#1?
      \def\childdoctmp
      {
        \def\childdocname{#2}
        \def\childdocjob{#2}
        \def\jobname{#2}
        \input{#2}
        \endinput
      }
    \else
      \def\childdoctmp
      {
        \childdocdisable
        \def\childdocname{#2}
        \childdoctrue
        \includeonly{#2}
        \def\childdocjob{#1}
        \def\jobname{#1}
        \input{#1}
        \endinput
      }
    \fi
    \expandafter
  \endgroup
  \childdoctmp
}
%    \end{macrocode}

% \macro{\childdocforwardprefix}
% The command |\childdocforwardprefix| redirects
% compilation to the main or a child file by means of a pattern.
% The prefix |#1| in the current filename is replaced by |#2|
% and the suffix of the current filename is kept
% (it is assumed that the filename does not contain the substring `|~~~|'
% which is used as a delimiter).
% Compilation is handed over to the new file by |\childdocforward|:
%    \begin{macrocode}
\newcommand{\childdocforwardprefix}[3][]
{
  \begingroup
    \def\childdocextract #2##1~~~{\def\childdoctmp{\childdocforward[#1]{#3##1}}}
    \expandafter\childdocextract\childdocname~~~
    \expandafter
  \endgroup
  \childdoctmp
}
%    \end{macrocode}

% \macro{\childdoc}
% The deprecated macro |\childdoc| is a legacy version of |\childdocmain|:
%    \begin{macrocode}
\newcommand{\childdoc}{\childdocmain}
%    \end{macrocode}

% \macro{\childdocredirect}
% The deprecated macro |\childdocredirect| is a legacy version
% of |\childdocforward| and |\childdocforwardprefix|:
%    \begin{macrocode}
\newcommand{\childdocredirect}[2][]
{
  \begingroup
    \if?#1?
      \def\childdoctmp{\childdocforward{#2}}
    \else
      \def\childdoctmp{\childdocforwardprefix{#1}{#2}}
    \fi
    \expandafter
  \endgroup
  \childdoctmp
}
%    \end{macrocode}

%\iffalse
%</package>
%\fi
%
\endinput
|
and perform the replacements as outlined below.
Instead of |\childdocmain{|\textit{main}|}| add the following code
to the top of the main file:
%
\begin{center}
\begin{tabular}{l}
|\||ifdefined\childdocname\endinput\||fi\newif\ifchilddoc|\\
|\edef\childdocname{\scantokens\expandafter{\jobname\noexpand}}|\\
|\def\childdocmain{|\textit{main}|}\||ifx\childdocmain\childdocname\||else|\\
|\childdoctrue\includeonly{\childdocname}\let\jobname\childdocmain\||fi|\\
\end{tabular}
\end{center}
%
Instead of |\childdocof{|\textit{main}|}| just include the main file
at the top of each child file:
%
\begin{center}
|\input{|\textit{main}|}|
\end{center}
%
A simple redirection |\childdocforward{|\textit{dest}|}| is achieved by:
%
\begin{center}
|\def\jobname{|\textit{dest}|}\input{\jobname}|
\end{center}
%
The redirection with prefix
|\childdocforwardprefix[|\textit{prefix}|]{|\textit{dest}|}|
is accomplished by:
%
\begin{center}
\begin{tabular}{l}
|{\edef\jobname{\scantokens\expandafter{\jobname\noexpand}}|\\
|\def\redirectjob |\textit{prefix}|#1~~~{\gdef\jobname{|\textit{dest}|#1}}|\\
|\expandafter\redirectjob\jobname~~~}\input{\jobname}|
\end{tabular}
\end{center}

In an alternative approach,
child documents can be compiled by a specific command line
without additional code or specific definitions:
%
\begin{center}
|... -jobname "|\textit{target}|" "|[\textit{flags}]%
|\includeonly{|\textit{dest}|}\input{|\textit{main}|}"|
\end{center}
%

%%%%%%%%%%%%%%%%%%%%%%%%%%%%%%%%%%%%%%%%%%%%%%%%%%%%%%%%%%%%%%%%%%%%%%%%%%%%%%%%
%%%%%%%%%%%%%%%%%%%%%%%%%%%%%%%%%%%%%%%%%%%%%%%%%%%%%%%%%%%%%%%%%%%%%%%%%%%%%%%%
\section{Information}

%%%%%%%%%%%%%%%%%%%%%%%%%%%%%%%%%%%%%%%%%%%%%%%%%%%%%%%%%%%%%%%%%%%%%%%%%%%%%%%%
\subsection{Copyright}

Copyright \copyright{} 2017--2018 Niklas Beisert

This work may be distributed and/or modified under the
conditions of the \LaTeX{} Project Public License, either version 1.3
of this license or (at your option) any later version.
The latest version of this license is in
  \url{http://www.latex-project.org/lppl.txt}
and version 1.3 or later is part of all distributions of \LaTeX{}
version 2005/12/01 or later.

This work has the LPPL maintenance status `maintained'.

The Current Maintainer of this work is Niklas Beisert.

This work consists of the files |README.txt|, |childdoc.ins| and |childdoc.dtx|
as well as the derived files |childdoc.def|, |cdocsamp.tex|
with |cdocsch1.tex|, |cdocsch2.tex|, |cdocspt3.tex|, |cdocspt4.tex|,
|cdocsdrf.tex|, |cdocsfn1.tex|, |cdocsfn2.tex|
as well as |childdoc.pdf|.

%%%%%%%%%%%%%%%%%%%%%%%%%%%%%%%%%%%%%%%%%%%%%%%%%%%%%%%%%%%%%%%%%%%%%%%%%%%%%%%%
\subsection{Files and Installation}

The package consists of the files:
%
\begin{center}
\begin{tabular}{ll}
    |README.txt|   & readme file \\
    |childdoc.ins| & installation file \\
    |childdoc.dtx| & source file \\
    |childdoc.def| & definition file \\
    |cdocsamp.tex| & sample main file \\
    |cdocsch1.tex| & sample include file \\
    |cdocsch2.tex| & sample include file \\
    |cdocspt3.tex| & sample part file \\
    |cdocspt4.tex| & sample part file \\
    |cdocsdrf.tex| & sample redirection file \\
    |cdocsfn1.tex| & sample redirection file \\
    |cdocsfn2.tex| & sample redirection file \\
    |childdoc.pdf| & manual
\end{tabular}
\end{center}
%
The distribution consists of the files
|README.txt|, |childdoc.ins| and |childdoc.dtx|.
%
\begin{itemize}
\item
Run (pdf)\LaTeX{} on |childdoc.dtx|
to compile the manual |childdoc.pdf| (this file).
\item
Run \LaTeX{} on |childdoc.ins| to create the definitions file |childdoc.def|
and the sample |cdocsamp.tex| with include files
|cdocsch1.tex|, |cdocsch2.tex|, |cdocspt3.tex|, |cdocspt4.tex|,
|cdocsdrf.tex|, |cdocsfn1.tex|, |cdocsfn2.tex|.
Then copy the file |childdoc.def| to an appropriate directory of your \LaTeX{}
distribution, e.g.\ \textit{texmf-root}|/tex/latex/childdoc|.
\end{itemize}

%%%%%%%%%%%%%%%%%%%%%%%%%%%%%%%%%%%%%%%%%%%%%%%%%%%%%%%%%%%%%%%%%%%%%%%%%%%%%%%%
\subsection{Related CTAN Packages}

There are several other packages which offer a similar functionality:
%
\begin{itemize}
\item
The packages
\href{http://ctan.org/pkg/docmute}{\textsf{docmute}},
\href{http://ctan.org/pkg/includex}{\textsf{includex}} and
\href{http://ctan.org/pkg/standalone}{\textsf{standalone}}
provide commands to include only the document body of
a child file thus allowing both files to be compiled individually.
\item
The packages \href{http://ctan.org/pkg/subdocs}{\textsf{subdocs}}
and \href{http://ctan.org/pkg/subfiles}{\textsf{subfiles}}
provide structures in which the main and child documents can be
encapsulated and allowing them to be compiled individually.
The inclusion mechanism is different from the conventional |\include|.
\item
The package \href{http://ctan.org/pkg/combine}{\textsf{combine}}
is an elaborate solution to combine several documents into one.
\end{itemize}
%
See also the CTAN topic \href{http://ctan.org/topic/subdocs}{\textsf{subdocs}}
for further related packages.
The present package differs from the above solutions in that
a document structure constructed with the conventional |\include| mechanism
just needs two extra commands at the top of every file
such that all constituent files can be compiled individually.

%%%%%%%%%%%%%%%%%%%%%%%%%%%%%%%%%%%%%%%%%%%%%%%%%%%%%%%%%%%%%%%%%%%%%%%%%%%%%%%%
%\subsection{Feature Suggestions}
%
%The following is a list of features which may be useful for future
%versions of this package:
%%
%\begin{itemize}
%\item
%\ldots
%\end{itemize}

%%%%%%%%%%%%%%%%%%%%%%%%%%%%%%%%%%%%%%%%%%%%%%%%%%%%%%%%%%%%%%%%%%%%%%%%%%%%%%%%
\subsection{Revision History}

%%%%%%%%%%%%%%%%%%%%%%%%%%%%%%%%%%%%%%%%
\paragraph{v2.0:} 2018/12/30

\begin{itemize}
\item
immediate forward processing
\item
added |\childdocby| mechanism
\item
manual restructured
\end{itemize}

%%%%%%%%%%%%%%%%%%%%%%%%%%%%%%%%%%%%%%%%
\paragraph{v1.6:} 2018/01/17

\begin{itemize}
\item
application for development of include files
\item
corrections to manual
\end{itemize}

%%%%%%%%%%%%%%%%%%%%%%%%%%%%%%%%%%%%%%%%
\paragraph{v1.5:} 2017/05/21

\begin{itemize}
\item
more complete structuring introduced
\item
|\childdocof| introduced
\item
|\childdoc| renamed to |\childdocmain|
\item
|\childredirect| renamed to |\childdocforward| and |\childdocforwardprefix|
and functionality expanded
\end{itemize}

%%%%%%%%%%%%%%%%%%%%%%%%%%%%%%%%%%%%%%%%
\paragraph{v1.0:} 2017/04/27

\begin{itemize}
\item
manual and install package
\item
first version published on CTAN
\end{itemize}

%%%%%%%%%%%%%%%%%%%%%%%%%%%%%%%%%%%%%%%%
\paragraph{v0.6:} 2017/04/26

\begin{itemize}
\item
redirection mechanism added
\end{itemize}

%%%%%%%%%%%%%%%%%%%%%%%%%%%%%%%%%%%%%%%%
\paragraph{v0.5:} 2017/04/26

\begin{itemize}
\item
functionality in definition file
\end{itemize}


%%%%%%%%%%%%%%%%%%%%%%%%%%%%%%%%%%%%%%%%%%%%%%%%%%%%%%%%%%%%%%%%%%%%%%%%%%%%%%%%
%%%%%%%%%%%%%%%%%%%%%%%%%%%%%%%%%%%%%%%%%%%%%%%%%%%%%%%%%%%%%%%%%%%%%%%%%%%%%%%%
%%%%%%%%%%%%%%%%%%%%%%%%%%%%%%%%%%%%%%%%%%%%%%%%%%%%%%%%%%%%%%%%%%%%%%%%%%%%%%%%
\appendix

\settowidth\MacroIndent{\rmfamily\scriptsize 000\ }

 \DocInput{childdoc.dtx}

\end{document}
%</driver>
% \fi
%
% %%%%%%%%%%%%%%%%%%%%%%%%%%%%%%%%%%%%%%%%%%%%%%%%%%%%%%%%%%%%%%%%%%%%%%%%%%%%%%
% %%%%%%%%%%%%%%%%%%%%%%%%%%%%%%%%%%%%%%%%%%%%%%%%%%%%%%%%%%%%%%%%%%%%%%%%%%%%%%
% \section{Sample}
%\iffalse
%<*samplemain>
%\fi
%
% The following presents a sample document
% with two chapters, two parts, a title page,
% a compile flag as well as three forwarding files to set the flag.
% It consists of eight |.tex| files:
% \begin{center}
% \begin{tabular}{ll}
% |cdocsamp.tex|&main file\\
% |cdocsch1.tex|&include file for chapter 1\\
% |cdocsch2.tex|&include file for chapter 2\\
% |cdocspt3.tex|&include file for part 3\\
% |cdocspt4.tex|&include file for part 4\\
% |cdocsdrf.tex|&forwarding file for main file in draft mode\\
% |cdocsfi1.tex|&forwarding file for final version of chapter 1\\
% |cdocsfi2.tex|&forwarding file for final version of chapter 2\\
% \end{tabular}
% \end{center}
% Each of the eight files can be compiled directly by the \LaTeX{} compiler.
%
% %%%%%%%%%%%%%%%%%%%%%%%%%%%%%%%%%%%%%%
% \paragraph{Main File.}
%
% The main file is called |cdocsamp.tex|.
%
% Load the \textsf{childdoc} definitions and
% declare the filename for the main document:
%    \begin{macrocode}
% \iffalse
%
% childdoc.dtx Copyright (C) 2017-2018 Niklas Beisert
%
% This work may be distributed and/or modified under the
% conditions of the LaTeX Project Public License, either version 1.3
% of this license or (at your option) any later version.
% The latest version of this license is in
%   http://www.latex-project.org/lppl.txt
% and version 1.3 or later is part of all distributions of LaTeX
% version 2005/12/01 or later.
%
% This work has the LPPL maintenance status `maintained'.
%
% The Current Maintainer of this work is Niklas Beisert.
%
% This work consists of the files childdoc.dtx and childdoc.ins
% and the derived files childdoc.def and cdocsamp.tex with
% cdocsch1.tex, cdocsch2.tex, cdocsdrf.tex, cdocsfn1.tex, cdocsfn2.tex.
%
%<package>\ifdefined\childdocmain\endinput\fi
%<package>\ProvidesFile{childdoc.def}[2018/12/30 v2.0 child document driver]
%<samplemain>\ProvidesFile{cdocsamp.tex}[2018/12/30 v2.0 sample for childdoc]
%<*driver>
%\ProvidesFile{childdoc.drv}[2018/12/30 v2.0 childdoc reference manual file]
\PassOptionsToClass{10pt,a4paper}{article}
\documentclass{ltxdoc}

\usepackage[margin=35mm]{geometry}
\usepackage{hyperref}
\usepackage{hyperxmp}
\usepackage[usenames]{color}

\hypersetup{colorlinks=true}
\hypersetup{pdfstartview=FitH}
\hypersetup{pdfpagemode=UseNone}
\hypersetup{pdfsource={}}
\hypersetup{pdflang={en-UK}}
\hypersetup{pdfcopyright={Copyright 2017-2018 Niklas Beisert.
  This work may be distributed and/or modified under the
  conditions of the LaTeX Project Public License, either version 1.3
  of this license or (at your option) any later version.}}
\hypersetup{pdflicenseurl={http://www.latex-project.org/lppl.txt}}
\hypersetup{pdfcontactaddress={ETH Zurich, ITP, HIT K,
  Wolfgang-Pauli-Strasse 27}}
\hypersetup{pdfcontactpostcode={8093}}
\hypersetup{pdfcontactcity={Zurich}}
\hypersetup{pdfcontactcountry={Switzerland}}
\hypersetup{pdfcontactemail={nbeisert@itp.phys.ethz.ch}}
\hypersetup{pdfcontacturl={http://people.phys.ethz.ch/\xmptilde nbeisert/}}

\newcommand{\secref}[1]{\hyperref[#1]{section \ref*{#1}}}

\parskip1ex
\parindent0pt
\let\olditemize\itemize
\def\itemize{\olditemize\parskip0pt}

\begin{document}

\title{The \textsf{childdoc} Package}
\hypersetup{pdftitle={The childdoc Package}}
\author{Niklas Beisert\\[2ex]
  Institut f\"ur Theoretische Physik\\
  Eidgen\"ossische Technische Hochschule Z\"urich\\
  Wolfgang-Pauli-Strasse 27, 8093 Z\"urich, Switzerland\\[1ex]
  \href{mailto:nbeisert@itp.phys.ethz.ch}
  {\texttt{nbeisert@itp.phys.ethz.ch}}}
\hypersetup{pdfauthor={Niklas Beisert}}
\hypersetup{pdfsubject={Manual for the LaTeX2e Package childdoc}}
\date{30 December 2018, \textsf{v2.0}}
\maketitle

\begin{abstract}\noindent
\textsf{childdoc} is a \LaTeXe{} package
that enables the direct compilation
of document sections included by |\include|
to individual files.
\end{abstract}

\begingroup
\parskip0ex
\tableofcontents
\endgroup

%%%%%%%%%%%%%%%%%%%%%%%%%%%%%%%%%%%%%%%%%%%%%%%%%%%%%%%%%%%%%%%%%%%%%%%%%%%%%%%%
%%%%%%%%%%%%%%%%%%%%%%%%%%%%%%%%%%%%%%%%%%%%%%%%%%%%%%%%%%%%%%%%%%%%%%%%%%%%%%%%
\section{Introduction}

\LaTeX{} provides a mechanism to structure a large document (such as a book)
into a main file and several child files (containing the chapters)
using the |\include| command.
This mechanism is beneficial for documents
which span hundreds of pages in order to
make the source file(s) more manageable.
Moreover, compilation can be restricted to
selected child files by means of the |\includeonly| command.
The latter feature can be used to reduce the compilation time while editing
(this was significantly more useful in the earlier days of \LaTeX{})
or to generate a smaller document which is easier to navigate.
Another application of |\includeonly| is to generate
documents consisting of selected parts of the complete document.

However, there are a few drawbacks of the plain |\include| mechanism:
\begin{itemize}
\item
The child files cannot be compiled on their own,
they can only be compiled via the main file.
A naive editing environment
(such as a text editor with an option
to have the current file processed by \LaTeX)
may require one to switch to the main file before compiling;
attempting to compile the child file produces errors.
\item
The main file must be modified (each time)
to adjust the |\includeonly| command
to the present needs. This easily leaves the main file in a messy state.
\item
The generated document will always carry the filename
of the main document. This is inconvenient if
several child files are to be compiled and
to be kept for distribution.
\end{itemize}

The present package provides a simple interface
to make child files individually compilable by \LaTeX{}.
Compiling a child file then has the same effect as compiling
the main file with an |\includeonly| command
to select the appropriate child.
Moreover the generated document will carry the name of the child
rather than the main file.
This resolves all three above issues.

This feature is meant to make the editing of books,
thesis documents and lecture notes somewhat more convenient.
However, the package can also be used efficiently for
composing a series of documents (such as exercise sheets)
which are typically distributed individually.
It then assists the author in generating the individual documents
(potentially in different versions)
as well as a document containing the collected series.
Another application is in developing style files
or other kinds of included material
where compilation of the style file could redirect
to a sample or test file.

%%%%%%%%%%%%%%%%%%%%%%%%%%%%%%%%%%%%%%%%%%%%%%%%%%%%%%%%%%%%%%%%%%%%%%%%%%%%%%%%
%%%%%%%%%%%%%%%%%%%%%%%%%%%%%%%%%%%%%%%%%%%%%%%%%%%%%%%%%%%%%%%%%%%%%%%%%%%%%%%%
\section{Usage}

First of all, the package \textsf{childdoc} is \emph{not} a standard
\LaTeXe{} |.sty| style file! Therefore it needs to be invoked in
a non-standard way.

%%%%%%%%%%%%%%%%%%%%%%%%%%%%%%%%%%%%%%%%%%%%%%%%%%%%%%%%%%%%%%%%%%%%%%%%%%%%%%%%
\subsection{Included Files}
\label{sec:include}

%%%%%%%%%%%%%%%%%%%%%%%%%%%%%%%%%%%%%%%%
\DescribeMacro{\childdocmain}
To use the package, add the commands
\begin{center}
\begin{tabular}{l}
|\input{childdoc.def}|\\
|\childdocmain{}|\\
\end{tabular}
\end{center}
at the very top of the main \LaTeX{} file,
in particular \emph{before} the |\documentclass| statement!
The argument of |\childdocmain| should be left empty
(but it must be present).

%%%%%%%%%%%%%%%%%%%%%%%%%%%%%%%%%%%%%%%%
\DescribeMacro{\childdocof}
Furthermore, add the commands
\begin{center}
\begin{tabular}{l}
|\input{childdoc.def}|\\
|\childdocof{|\textit{main}|}|\\
\end{tabular}
\end{center}
at the top of every child file \textit{child}
which is included by |\include{|\textit{child}|}|
from within the main file
(or at least for those files to be compiled individually).
The argument \textit{main} must be the filename of the main file.

There are a couple of
considerations in setting up the main and child documents:

%%%%%%%%%%%%%%%%%%%%%%%%%%%%%%%%%%%%%%%%
\paragraph{Restrictions.}

Please note the following restrictions:
\begin{itemize}
\item
|\childdocmain| must be called with one argument \textit{main}
to ensure compatibility with earlier version of the package.
It must either be empty (|\childdocmain{}|)
or precisely match the filename of the main file in which it is specified.
See \secref{sec:detection} for further information.
\item
The filename \textit{main} must be specified without the |.tex| extension.
\item
The filename \textit{main} is case sensitive
(even in case-insensitive file systems)
due to internal string comparison.
\item
The argument \textit{main} should be fully expanded, it cannot be a macro.
\item
Subdirectories and special characters should be avoided in filenames.
\item
The command |\childdocmain{|\textit{main}|}| must be followed by a whitespace.
It should not be followed immediately by another command
or by a comment mark `|%|'.
This is because the \TeX{} parser reads the token immediately following
the argument of |\childdocmain| and puts it
at the beginning of every child section;
however, a white\-space is ignored.
\end{itemize}

%%%%%%%%%%%%%%%%%%%%%%%%%%%%%%%%%%%%%%%%
\paragraph{Content of Main File.}

It is advisable to place all content in the child files included by |\include|.
Any output contained in the main file will appear in all child documents
unless suppressed manually;
it cannot be suppressed automatically by the |\includeonly| directive
and thus should normally be avoided.
A method to include some content in the main file
by means of conditional processing is described in \secref{sec:conditional}.

%%%%%%%%%%%%%%%%%%%%%%%%%%%%%%%%%%%%%%%%
\paragraph{Page Numbering.}

When only a part of the document is compiled,
the appropriate numbering of pages
(as well as other status parameters)
is determined from the |.aux| files.
The latter contain information from previous passes.
However this information needs to propagate through
all intermediate child documents.
Therefore the page numbering in child documents may well
be inconsistent until the complete document is compiled at least once.

A useful (if unconventional) way to always ensure a consistent
page numbering is to restart the numbering in each child document
and denote the pages by `\textit{child}|.|\textit{page}'
where \textit{child} represents the chapter/section number of the child file.
This can be achieved by the command
|\numberwithin{page}{|\textit{child}|}|
of the \textsf{amsmath} package
where \textit{child} can be |chapter| or |section|
depending on the chosen structuring.
Alternatively, one can modify the macro |\thepage| appropriately
and reset the counter |page| at the start of each child file.

%%%%%%%%%%%%%%%%%%%%%%%%%%%%%%%%%%%%%%%%%%%%%%%%%%%%%%%%%%%%%%%%%%%%%%%%%%%%%%%%
\subsection{Conditional Processing}
\label{sec:conditional}

The package provides a mechanism to compile different versions
of a document. To customise the versions further some conditional processing
can come in handy to distinguish which version is being compiled.
The package provides two macros to describe the compilation context:

%%%%%%%%%%%%%%%%%%%%%%%%%%%%%%%%%%%%%%%%
\DescribeMacro{\ifchilddoc}
The conditional |\ifchilddoc| distinguishes between the compilation of
child documents and the main document:
%
\begin{center}
|\ifchilddoc |\textit{child-code}| |[|\||else |\textit{main-code}]| \||fi|
\end{center}

%%%%%%%%%%%%%%%%%%%%%%%%%%%%%%%%%%%%%%%%
\DescribeMacro{\childdocname}
\DescribeMacro{\childdocjob}
The macro |\childdocname| contains the filename (without extension)
of the main or child file being processed.
Note that |\childdocjob| will always contain the name of the main file.

%%%%%%%%%%%%%%%%%%%%%%%%%%%%%%%%%%%%%%%%
\paragraph{Title Page.}

Conditional processing can be used to include a title or banner page
in the main document when proper precautions are taken.
Importantly, the code in the main file should ensure that the page counter
(as well as other status parameters which are stored in the |.aux| files)
takes the same value after the conditional processing.
Otherwise the page numbers may take divergent values
depending on which part is compiled.

For example, a title page could be declared by:
%
\begin{center}
\begin{tabular}{l}
|\ifchilddoc\||else|\\
|\addtocounter{page}{-1}|\\
\textit{code for title page}\\
|\newpage|\\
|\||fi|
\end{tabular}
\end{center}
%
A banner page for the child documents can be generated by:
%
\begin{center}
\begin{tabular}{l}
|\ifchilddoc|\\
|\addtocounter{page}{-1}|\\
\textit{code for banner page}\\
|\newpage|\\
|\||fi|
\end{tabular}
\end{center}
%
Here one could write a message such as:
\begin{center}
|This is the part \childdocname{} of \childdocjob{}.|
\end{center}

%%%%%%%%%%%%%%%%%%%%%%%%%%%%%%%%%%%%%%%%%%%%%%%%%%%%%%%%%%%%%%%%%%%%%%%%%%%%%%%%
\subsection{Flags}
\label{sec:flags}

The package makes it easy to generate different versions
of the main or child documents.
To this end compilation flags can be defined
and assigned different default values.
They will be particularly useful in conjunction
with the forwarding mechanism described in \secref{sec:forward}.

For example, it may be useful to have a flag |\version|
which can be set to |draft| or |final|.
The document source will contain some conditional code
depending on the value of |\version|.
Suppose further, the flag should default to |final| for the main file
and to |draft| for child files
which is a natural assignment for editing the document.
This is achieved by placing the following code
in the preamble of the main document
(below the |\childdocmain| directive):
%
\begin{center}
\begin{tabular}{l}
|\ifchilddoc|\\
|\providecommand{\version}{draft}|\\
|\||else|\\
|\providecommand{\version}{final}|\\
|\||fi|
\end{tabular}
\end{center}
%
The definition by |\providecommand| makes sure
that previous definitions are not overwritten.
Further statements |\providecommand{\version}{...}|
can thus be added before the above code to override it.

For the main file, one might add a line
(between |\childdocmain| and the above block)
%
\begin{center}
|%\ifchilddoc\||else\providecommand{\version}{draft}\||fi|
\end{center}
%
which can be uncommented to produce a draft version.
Likewise one can add a line to the very top of a child file
(above the |\childdocof{|\textit{main}|}| directive)
%
\begin{center}
|%\providecommand{\version}{final}|
\end{center}
%
which can be uncommented to produce the final version of this child document.

%%%%%%%%%%%%%%%%%%%%%%%%%%%%%%%%%%%%%%%%%%%%%%%%%%%%%%%%%%%%%%%%%%%%%%%%%%%%%%%%
\subsection{Forwarding}
\label{sec:forward}

Different versions of the main or child documents
using compilation flags as described in \secref{sec:flags}
can be (permanently) stored in different files
for convenient compilation, viewing and distribution.
To this end, the package defines a command
to pass on compilation to a different file:

%%%%%%%%%%%%%%%%%%%%%%%%%%%%%%%%%%%%%%%%
\DescribeMacro{\childdocforward}
The command |\childdocforward| redirects processing to
another source file:
%
\begin{center}
\begin{tabular}{l}
|\input{childdoc.def}|\\
|\childdocforward[|\textit{main}|]{|\textit{dest}|}|\\
\end{tabular}
\end{center}
%
The argument \textit{dest} is the destination file
(without extension).
It should be the main file or one of the child files.
Note that further \textsf{childdoc} directives
such as |\childdocof| and |\childdocforward|
in the indicated file will be processed in this form.
The optional argument \textit{main}
passes on directly to the main file \textit{main}
while pretending to compile the child \textit{dest}.
This form behaves as if \textit{dest}
issues |\childdocof{|\textit{main}|}| right away,
and no further \textsf{childdoc} directives will be processed.

%%%%%%%%%%%%%%%%%%%%%%%%%%%%%%%%%%%%%%%%
\DescribeMacro{\...prefix}
In the alternative form |\childdocforwardprefix|,
%
\begin{center}
\begin{tabular}{l}
|\input{childdoc.def}|\\
|\childdocforwardprefix[|\textit{main}|]{|\textit{prefix}|}{|\textit{dest}|}|
\end{tabular}
\end{center}
%
the destination file is determined by a pattern
depending on the current file:
To make this work, the current file must be called
`{\textit{prefix}\hspace{0.2em}\textit{suffix}}'
with \textit{prefix} matching precisely the argument.
Processing is then passed on to the file
`{\textit{dest}\hspace{0.2em}\textit{suffix}}'.
Surely, the same effect is achieved by
directly specifying the
argument `{\textit{dest}\hspace{0.2em}\textit{suffix}}'
in the first form.
However, that requires to set up a different file
for each child. With the alternative form of the command
all these files can have exactly the same content
which simplifies setting them up and maintaining them.

For example, the following file |draft.tex|
with a compilation flag |\version| as described in \secref{sec:flags}
compiles the main document as a draft:
%
\begin{center}
\begin{tabular}{l}
|\def\version{draft}|\\
|\input{childdoc.def}|\\
|\childdocforward{|\textit{main}|}|
\end{tabular}
\end{center}
%
Likewise, the following files |final|\textit{nn}|.tex|
compile the final version of the child document
|child|\textit{nn}|.tex|:
%
\begin{center}
\begin{tabular}{l}
|\def\version{final}|\\
|\input{childdoc.def}|\\
|\childdocforwardprefix{final}{child}|
\end{tabular}
\end{center}
%

Note that when several versions of a main file and/or of each child file
are to be generated, it may be convenient to set up a |Makefile| or
shell script to automatise the process.

%%%%%%%%%%%%%%%%%%%%%%%%%%%%%%%%%%%%%%%%%%%%%%%%%%%%%%%%%%%%%%%%%%%%%%%%%%%%%%%%
\subsection{Command Line Processing}
\label{sec:commandline}

The effect of redirection files can also be achieved by invoking
the \LaTeX{} compiler with a more elaborate command line.
Most conveniently this should be done as part
of a shell script or a |Makefile|.

When using \textsf{childdoc} in the main file, the following
command lines effectively perform a redirection
(note that depending on the shell being used,
backslashes may have to be doubled: `|\|' $\to$ `|\\|'):
%
\begin{center}
|... -jobname "|\textit{target}|" |\\|"|[\textit{flags}]%
|\input{childdoc.def}\childdocforward[|\textit{main}|]{|\textit{dest}|}"|
\end{center}
%
Here \textit{target} is the name of the output file,
\textit{main} is the name of the main file
and \textit{dest} is the name of the main or child file to be processed
(all filenames without extensions).
The optional argument \textit{main} can be omitted
if \textit{main} matches \textit{dest}.
Optionally, compilation \textit{flags} can be defined via |\def| commands.
This command line makes the \TeX{} engine believe
it is compiling the file \textit{target}
whose content is specified as the latter parameter.
The provided code then forwards the processing to
\textit{main} or \textit{dest} as described in \secref{sec:forward}.

%%%%%%%%%%%%%%%%%%%%%%%%%%%%%%%%%%%%%%%%%%%%%%%%%%%%%%%%%%%%%%%%%%%%%%%%%%%%%%%%
\subsection{Include by Input}
\label{sec:input}

Including child documents by |\include| has some restrictions by design.
Most notably, the content of a child document always occupies
its own set of pages; pages cannot be shared between child documents.
Usually, this behaviour makes perfect sense
because each child document contain an essential part of the document.
However, in some situations it may be desirable to compose
a document from a collection of parts
without having mandatory page breaks between then.
For this case, the package
provides a mechanism to include parts
by |\input| which can also be processed individually.
However, by construction this mechanism
requires manual handling of the content to be output.

%%%%%%%%%%%%%%%%%%%%%%%%%%%%%%%%%%%%%%%%
\DescribeMacro{\ifchilddocmanual}
The main file should be prepared as usual, see \secref{sec:include}.
However, the document body must make a distinction
between processing of an individual part and of the main document, e.g.:
%
\begin{center}
\begin{tabular}{l}
|\ifchilddocmanual|\\
|\input{\childdocname}|\\
|\||else|\\
\textit{document body with }|\input{|\textit{part}|}|\\
|\||fi|
\end{tabular}
\end{center}
%
The conditional |\ifchilddocmanual| is true whenever
a part to be included by |\input| is being compiled,
and the name of the part is stored in |\childdocname|.

%%%%%%%%%%%%%%%%%%%%%%%%%%%%%%%%%%%%%%%%
\DescribeMacro{\childdocby}
Each part to be included by |\input| should start with:
%
\begin{center}
\begin{tabular}{l}
|\input{childdoc.def}|\\
|\childdocby{|\textit{main}|}|\\
\end{tabular}
\end{center}
%
The directive |\childdocby| is similar to |\childdocof|
described in \secref{sec:include},
but the subsequent selection of content must be done manually.
To that end, both |\ifchilddoc| and |\ifchilddocmanual|
will be true upon processing of a part,
and the name of the part is stored in |\childdocname|.
Note that |\jobname| will be set to the filename of the current part
so that each part receives an individual |.aux| file
that does not interfere with the |.aux| file(s) of the main document.
This behaviour can be altered by the alternative form
|\childdocby[*]{|\textit{main}|}| (with a non-empty optional argument)
which uses the |.aux| file of the main document
by setting |\jobname| to \textit{main}.

%%%%%%%%%%%%%%%%%%%%%%%%%%%%%%%%%%%%%%%%%%%%%%%%%%%%%%%%%%%%%%%%%%%%%%%%%%%%%%%%
\subsection{Driver Development}
\label{sec:driver}

The \textsf{childdoc} mechanism can also be use for the development
of definition files such as \LaTeX{} styles or classes.
This case differs from the above setup with multiple parts
included by |\include| in that no |\includeonly| should be invoked.
This can be achieved by starting the include file
(before |\ProvidesPackage|) with:
%
\begin{center}
\begin{tabular}{l}
|\input{childdoc.def}|\\
|\childdocforward{|\textit{main}|}|\\
\end{tabular}
\end{center}
%
or alternatively with:
%
\begin{center}
\begin{tabular}{l}
|\input{childdoc.def}|\\
|\childdocby{|\textit{main}|}|\\
\end{tabular}
\end{center}
%
Both forms have slightly different effects as described above.
The main file is prepared as usual, see \secref{sec:include}.

%%%%%%%%%%%%%%%%%%%%%%%%%%%%%%%%%%%%%%%%%%%%%%%%%%%%%%%%%%%%%%%%%%%%%%%%%%%%%%%%
\subsection{Legacy Detection}
\label{sec:detection}

The directive |\childdocmain| in the main file can detect
whether the complete document or merely a child is to be compiled
even without using the directive |\childdocof|.
This method is deprecated because it is less robust
and there is no compelling reason to use it;
it is merely provided for backward compatibility
and it may be removed in future versions.

If the detection mechanism is to be used,
it is mandatory to correctly specify
the filename of the main file as the argument of |\childdocmain|:
%
\begin{center}
\begin{tabular}{l}
|\input{childdoc.def}|\\
|\childdocmain{|\textit{main}|}|\\
\end{tabular}
\end{center}
%
If |\jobname| does not match the argument \textit{main} of |\childdocmain|,
it is assumed that |\jobname| points to the child file to be compiled.
When using |\childdocmain| with the main file specified as argument,
it suffices to start a child file
with just |\input{|\textit{main}|}|
without loading of the package and using |\childdocof|.
If instead all processing is done
with the appropriate \textsf{childdoc} directives,
the argument of \textit{main} of |\childdocmain| can be empty.

An alternative version of the command line processing described
in \secref{sec:commandline} using the detection mechanism reads:
%
\begin{center}
|... -jobname "|\textit{target}|" "|[\textit{flags}]%
[|\def\jobname{|\textit{dest}|}|]|\input{|\textit{main}|}"|
\end{center}

%%%%%%%%%%%%%%%%%%%%%%%%%%%%%%%%%%%%%%%%%%%%%%%%%%%%%%%%%%%%%%%%%%%%%%%%%%%%%%%%
\subsection{Manual Code}
\label{sec:manual}

In case one cannot be certain whether the definitions file |childdoc.def|
is installed on the target \TeX{} distribution
and one prefers not to ship it,
it is conceivable to paste a few relevant commands into the sources.

To that end, drop all statements |\input{childdoc.def}|
and perform the replacements as outlined below.
Instead of |\childdocmain{|\textit{main}|}| add the following code
to the top of the main file:
%
\begin{center}
\begin{tabular}{l}
|\||ifdefined\childdocname\endinput\||fi\newif\ifchilddoc|\\
|\edef\childdocname{\scantokens\expandafter{\jobname\noexpand}}|\\
|\def\childdocmain{|\textit{main}|}\||ifx\childdocmain\childdocname\||else|\\
|\childdoctrue\includeonly{\childdocname}\let\jobname\childdocmain\||fi|\\
\end{tabular}
\end{center}
%
Instead of |\childdocof{|\textit{main}|}| just include the main file
at the top of each child file:
%
\begin{center}
|\input{|\textit{main}|}|
\end{center}
%
A simple redirection |\childdocforward{|\textit{dest}|}| is achieved by:
%
\begin{center}
|\def\jobname{|\textit{dest}|}\input{\jobname}|
\end{center}
%
The redirection with prefix
|\childdocforwardprefix[|\textit{prefix}|]{|\textit{dest}|}|
is accomplished by:
%
\begin{center}
\begin{tabular}{l}
|{\edef\jobname{\scantokens\expandafter{\jobname\noexpand}}|\\
|\def\redirectjob |\textit{prefix}|#1~~~{\gdef\jobname{|\textit{dest}|#1}}|\\
|\expandafter\redirectjob\jobname~~~}\input{\jobname}|
\end{tabular}
\end{center}

In an alternative approach,
child documents can be compiled by a specific command line
without additional code or specific definitions:
%
\begin{center}
|... -jobname "|\textit{target}|" "|[\textit{flags}]%
|\includeonly{|\textit{dest}|}\input{|\textit{main}|}"|
\end{center}
%

%%%%%%%%%%%%%%%%%%%%%%%%%%%%%%%%%%%%%%%%%%%%%%%%%%%%%%%%%%%%%%%%%%%%%%%%%%%%%%%%
%%%%%%%%%%%%%%%%%%%%%%%%%%%%%%%%%%%%%%%%%%%%%%%%%%%%%%%%%%%%%%%%%%%%%%%%%%%%%%%%
\section{Information}

%%%%%%%%%%%%%%%%%%%%%%%%%%%%%%%%%%%%%%%%%%%%%%%%%%%%%%%%%%%%%%%%%%%%%%%%%%%%%%%%
\subsection{Copyright}

Copyright \copyright{} 2017--2018 Niklas Beisert

This work may be distributed and/or modified under the
conditions of the \LaTeX{} Project Public License, either version 1.3
of this license or (at your option) any later version.
The latest version of this license is in
  \url{http://www.latex-project.org/lppl.txt}
and version 1.3 or later is part of all distributions of \LaTeX{}
version 2005/12/01 or later.

This work has the LPPL maintenance status `maintained'.

The Current Maintainer of this work is Niklas Beisert.

This work consists of the files |README.txt|, |childdoc.ins| and |childdoc.dtx|
as well as the derived files |childdoc.def|, |cdocsamp.tex|
with |cdocsch1.tex|, |cdocsch2.tex|, |cdocspt3.tex|, |cdocspt4.tex|,
|cdocsdrf.tex|, |cdocsfn1.tex|, |cdocsfn2.tex|
as well as |childdoc.pdf|.

%%%%%%%%%%%%%%%%%%%%%%%%%%%%%%%%%%%%%%%%%%%%%%%%%%%%%%%%%%%%%%%%%%%%%%%%%%%%%%%%
\subsection{Files and Installation}

The package consists of the files:
%
\begin{center}
\begin{tabular}{ll}
    |README.txt|   & readme file \\
    |childdoc.ins| & installation file \\
    |childdoc.dtx| & source file \\
    |childdoc.def| & definition file \\
    |cdocsamp.tex| & sample main file \\
    |cdocsch1.tex| & sample include file \\
    |cdocsch2.tex| & sample include file \\
    |cdocspt3.tex| & sample part file \\
    |cdocspt4.tex| & sample part file \\
    |cdocsdrf.tex| & sample redirection file \\
    |cdocsfn1.tex| & sample redirection file \\
    |cdocsfn2.tex| & sample redirection file \\
    |childdoc.pdf| & manual
\end{tabular}
\end{center}
%
The distribution consists of the files
|README.txt|, |childdoc.ins| and |childdoc.dtx|.
%
\begin{itemize}
\item
Run (pdf)\LaTeX{} on |childdoc.dtx|
to compile the manual |childdoc.pdf| (this file).
\item
Run \LaTeX{} on |childdoc.ins| to create the definitions file |childdoc.def|
and the sample |cdocsamp.tex| with include files
|cdocsch1.tex|, |cdocsch2.tex|, |cdocspt3.tex|, |cdocspt4.tex|,
|cdocsdrf.tex|, |cdocsfn1.tex|, |cdocsfn2.tex|.
Then copy the file |childdoc.def| to an appropriate directory of your \LaTeX{}
distribution, e.g.\ \textit{texmf-root}|/tex/latex/childdoc|.
\end{itemize}

%%%%%%%%%%%%%%%%%%%%%%%%%%%%%%%%%%%%%%%%%%%%%%%%%%%%%%%%%%%%%%%%%%%%%%%%%%%%%%%%
\subsection{Related CTAN Packages}

There are several other packages which offer a similar functionality:
%
\begin{itemize}
\item
The packages
\href{http://ctan.org/pkg/docmute}{\textsf{docmute}},
\href{http://ctan.org/pkg/includex}{\textsf{includex}} and
\href{http://ctan.org/pkg/standalone}{\textsf{standalone}}
provide commands to include only the document body of
a child file thus allowing both files to be compiled individually.
\item
The packages \href{http://ctan.org/pkg/subdocs}{\textsf{subdocs}}
and \href{http://ctan.org/pkg/subfiles}{\textsf{subfiles}}
provide structures in which the main and child documents can be
encapsulated and allowing them to be compiled individually.
The inclusion mechanism is different from the conventional |\include|.
\item
The package \href{http://ctan.org/pkg/combine}{\textsf{combine}}
is an elaborate solution to combine several documents into one.
\end{itemize}
%
See also the CTAN topic \href{http://ctan.org/topic/subdocs}{\textsf{subdocs}}
for further related packages.
The present package differs from the above solutions in that
a document structure constructed with the conventional |\include| mechanism
just needs two extra commands at the top of every file
such that all constituent files can be compiled individually.

%%%%%%%%%%%%%%%%%%%%%%%%%%%%%%%%%%%%%%%%%%%%%%%%%%%%%%%%%%%%%%%%%%%%%%%%%%%%%%%%
%\subsection{Feature Suggestions}
%
%The following is a list of features which may be useful for future
%versions of this package:
%%
%\begin{itemize}
%\item
%\ldots
%\end{itemize}

%%%%%%%%%%%%%%%%%%%%%%%%%%%%%%%%%%%%%%%%%%%%%%%%%%%%%%%%%%%%%%%%%%%%%%%%%%%%%%%%
\subsection{Revision History}

%%%%%%%%%%%%%%%%%%%%%%%%%%%%%%%%%%%%%%%%
\paragraph{v2.0:} 2018/12/30

\begin{itemize}
\item
immediate forward processing
\item
added |\childdocby| mechanism
\item
manual restructured
\end{itemize}

%%%%%%%%%%%%%%%%%%%%%%%%%%%%%%%%%%%%%%%%
\paragraph{v1.6:} 2018/01/17

\begin{itemize}
\item
application for development of include files
\item
corrections to manual
\end{itemize}

%%%%%%%%%%%%%%%%%%%%%%%%%%%%%%%%%%%%%%%%
\paragraph{v1.5:} 2017/05/21

\begin{itemize}
\item
more complete structuring introduced
\item
|\childdocof| introduced
\item
|\childdoc| renamed to |\childdocmain|
\item
|\childredirect| renamed to |\childdocforward| and |\childdocforwardprefix|
and functionality expanded
\end{itemize}

%%%%%%%%%%%%%%%%%%%%%%%%%%%%%%%%%%%%%%%%
\paragraph{v1.0:} 2017/04/27

\begin{itemize}
\item
manual and install package
\item
first version published on CTAN
\end{itemize}

%%%%%%%%%%%%%%%%%%%%%%%%%%%%%%%%%%%%%%%%
\paragraph{v0.6:} 2017/04/26

\begin{itemize}
\item
redirection mechanism added
\end{itemize}

%%%%%%%%%%%%%%%%%%%%%%%%%%%%%%%%%%%%%%%%
\paragraph{v0.5:} 2017/04/26

\begin{itemize}
\item
functionality in definition file
\end{itemize}


%%%%%%%%%%%%%%%%%%%%%%%%%%%%%%%%%%%%%%%%%%%%%%%%%%%%%%%%%%%%%%%%%%%%%%%%%%%%%%%%
%%%%%%%%%%%%%%%%%%%%%%%%%%%%%%%%%%%%%%%%%%%%%%%%%%%%%%%%%%%%%%%%%%%%%%%%%%%%%%%%
%%%%%%%%%%%%%%%%%%%%%%%%%%%%%%%%%%%%%%%%%%%%%%%%%%%%%%%%%%%%%%%%%%%%%%%%%%%%%%%%
\appendix

\settowidth\MacroIndent{\rmfamily\scriptsize 000\ }

 \DocInput{childdoc.dtx}

\end{document}
%</driver>
% \fi
%
% %%%%%%%%%%%%%%%%%%%%%%%%%%%%%%%%%%%%%%%%%%%%%%%%%%%%%%%%%%%%%%%%%%%%%%%%%%%%%%
% %%%%%%%%%%%%%%%%%%%%%%%%%%%%%%%%%%%%%%%%%%%%%%%%%%%%%%%%%%%%%%%%%%%%%%%%%%%%%%
% \section{Sample}
%\iffalse
%<*samplemain>
%\fi
%
% The following presents a sample document
% with two chapters, two parts, a title page,
% a compile flag as well as three forwarding files to set the flag.
% It consists of eight |.tex| files:
% \begin{center}
% \begin{tabular}{ll}
% |cdocsamp.tex|&main file\\
% |cdocsch1.tex|&include file for chapter 1\\
% |cdocsch2.tex|&include file for chapter 2\\
% |cdocspt3.tex|&include file for part 3\\
% |cdocspt4.tex|&include file for part 4\\
% |cdocsdrf.tex|&forwarding file for main file in draft mode\\
% |cdocsfi1.tex|&forwarding file for final version of chapter 1\\
% |cdocsfi2.tex|&forwarding file for final version of chapter 2\\
% \end{tabular}
% \end{center}
% Each of the eight files can be compiled directly by the \LaTeX{} compiler.
%
% %%%%%%%%%%%%%%%%%%%%%%%%%%%%%%%%%%%%%%
% \paragraph{Main File.}
%
% The main file is called |cdocsamp.tex|.
%
% Load the \textsf{childdoc} definitions and
% declare the filename for the main document:
%    \begin{macrocode}
\input{childdoc.def}
\childdocmain{}
%    \end{macrocode}

% Optional override for |\version| flag:
%    \begin{macrocode}
%%\ifchilddoc\else\providecommand{\version}{draft}\fi
%    \end{macrocode}

% Define the default values for the |\version| flag
% (|final| for the main file and |draft| for childs):
%    \begin{macrocode}
\ifchilddoc
\providecommand{\version}{draft}
\else
\providecommand{\version}{final}
\fi
%    \end{macrocode}

% Load the standard document class:
%    \begin{macrocode}
\documentclass[12pt]{article}
%    \end{macrocode}

% Start the document body:
%    \begin{macrocode}
\begin{document}
%    \end{macrocode}

% Declare a title page.
% Print title, part of document being processed and version flag:
%    \begin{macrocode}
\addtocounter{page}{-1}
\begin{center}
{\LARGE\bfseries{}childdoc example\par}
\vspace{1cm}
\ifchilddoc
\ifchilddocmanual part\else chapter\fi:
`\childdocname' of `\childdocjob'\par
\else
main document: `\childdocjob'\par
\fi
version: \version\par
\end{center}
\newpage
%    \end{macrocode}

% Manually include selected file,
% otherwise process as usual:
%    \begin{macrocode}
\ifchilddocmanual
\section*{part `\childdocname'}
\input{\childdocname}
\else
%    \end{macrocode}

% Include the two chapters:
%    \begin{macrocode}
\include{cdocsch1}
\include{cdocsch2}
%    \end{macrocode}

% Include the two parts unless only chapters should be displayed:
%    \begin{macrocode}
\ifchilddoc\else
\section{part three}
\input{cdocspt3}
\section{part four}
\input{cdocspt4}
\fi
%    \end{macrocode}

% Process as usual until here:
%    \begin{macrocode}
\fi
%    \end{macrocode}

% End of document body:
%    \begin{macrocode}
\end{document}
%    \end{macrocode}
%\iffalse
%</samplemain>
%\fi
%
% %%%%%%%%%%%%%%%%%%%%%%%%%%%%%%%%%%%%%%
% \paragraph{Chapter Include Files.}
%
% The include files are called |cdocsch1.tex| and |cdocsch2.tex|.
%
%\iffalse
%<*samplechap1|samplechap2>
%\fi

% Optional override for |\version| flag:
%    \begin{macrocode}
%%\providecommand{\version}{final}
%    \end{macrocode}

% Include the main document:
%    \begin{macrocode}
\input{childdoc.def}
\childdocof{cdocsamp}
%    \end{macrocode}

%\iffalse
%</samplechap1|samplechap2>
%\fi
%
%\iffalse
%<*samplechap1>
%\fi
% Some text for chapter 1:
%    \begin{macrocode}
\section{one}
some text in chapter one
%    \end{macrocode}

%\iffalse
%</samplechap1>
%\fi
% Some text for chapter 2:
%\iffalse
%<*samplechap2>
%\fi
%    \begin{macrocode}
\section{two}
more text in chapter two
%    \end{macrocode}

%\iffalse
%</samplechap2>
%\fi
%
% %%%%%%%%%%%%%%%%%%%%%%%%%%%%%%%%%%%%%%
% \paragraph{Part Include Files.}
%
% The include files are called |cdocspt3.tex| and |cdocspt4.tex|.
%
%\iffalse
%<*samplepart3|samplepart4>
%\fi

% Optional override for |\version| flag:
%    \begin{macrocode}
%%\providecommand{\version}{final}
%    \end{macrocode}

% Include the main document:
%    \begin{macrocode}
\input{childdoc.def}
\childdocby{cdocsamp}
%    \end{macrocode}

%\iffalse
%</samplepart3|samplepart4>
%\fi
%
%\iffalse
%<*samplepart3>
%\fi
% Some text for part 3:
%    \begin{macrocode}
some text in part three
%    \end{macrocode}

%\iffalse
%</samplepart3>
%\fi
% Some text for part 4:
%\iffalse
%<*samplepart4>
%\fi
%    \begin{macrocode}
more text in part four
%    \end{macrocode}

%\iffalse
%</samplepart4>
%\fi
%
% %%%%%%%%%%%%%%%%%%%%%%%%%%%%%%%%%%%%%%
% \paragraph{Forwarding for a Complete Draft.}
%
% The following forwarding file |cdocsdrf.tex|
% compiles the main document in draft mode:
%\iffalse
%<*sampledraft>
%\fi
%    \begin{macrocode}
\def\version{draft}
\input{childdoc.def}
\childdocforward{cdocsamp}
%    \end{macrocode}

%\iffalse
%</sampledraft>
%\fi
%
% %%%%%%%%%%%%%%%%%%%%%%%%%%%%%%%%%%%%%%
% \paragraph{Forwarding for Final Version of the Chapters.}
%
% The following forwarding files |cdocsfn1.tex| and |cdocsfn2.tex|
% (with identical content)
% compile the final versions of the child documents
% |cdocsch1.tex| and |cdocsch2.tex|, respectively:
%\iffalse
%<*samplefinal>
%\fi
%    \begin{macrocode}
\def\version{final}
\input{childdoc.def}
\childdocforwardprefix[cdocsamp]{cdocsfn}{cdocsch}
%    \end{macrocode}

%\iffalse
%</samplefinal>
%\fi
%
% %%%%%%%%%%%%%%%%%%%%%%%%%%%%%%%%%%%%%%
% \paragraph{Command Line Processing.}
%
% The following three command lines generate the output files
% |cdocscld|, |cdocscl1| and |cdocscl2|
% which should be identical to
% |cdocsdrf|, |cdocsch1| and |cdocsfn2|, respectively:
% \begin{center}
% \begin{tabular}{l}
% |latex -jobname cdocscld \|\\
% |  "\def\version{draft}\input{childdoc.def}\childdocforward{cdocsamp}"|\\
% |latex -jobname cdocscl1 \|\\
% |  "\input{childdoc.def}\childdocforward[cdocsamp]{cdocsch1}"|\\
% |latex -jobname cdocscl2 \|\\
% |  "\def\version{final}\input{childdoc.def}\childdocforward{cdocsch2}"|
% \end{tabular}
% \end{center}
% Note that the trailing backslash on each first line
% merely continues the input to the second line
% (for convenient cut ant paste).
% Furthermore, the command |latex| can be replaced by any
% of its alternative versions such as |pdflatex|.
%
% %%%%%%%%%%%%%%%%%%%%%%%%%%%%%%%%%%%%%%%%%%%%%%%%%%%%%%%%%%%%%%%%%%%%%%%%%%%%%%
% %%%%%%%%%%%%%%%%%%%%%%%%%%%%%%%%%%%%%%%%%%%%%%%%%%%%%%%%%%%%%%%%%%%%%%%%%%%%%%
% \section{Implementation}
%\iffalse
%<*package>
%\fi
%
% This section describes the definitions file |childdoc.def|.

% The definitions cannot be loaded using |\usepackage| or |\RequirePackage|
% which has a mechanism to prevent loading a style file more than once.
% When loading the definitions by means of |\input|
% multiple instances have to be prevented manually:
%\iffalse
%This code needs to be before the `\ProvidesFile' directive
%which is defined at the beginning of this file.
%Therefore it is also placed there and commented out here.
%</package>
%<*discard>
%\fi
%    \begin{macrocode}
\ifdefined\childdocmain\endinput\fi
%    \end{macrocode}
%\iffalse
%</discard>
%<*package>
%\fi
%
% \macro{\ifchilddoc}
% \macro{\ifchilddocmanual}
% The conditional |\ifchilddoc| tells whether a
% child (true) or main (false) document is being compiled.
% The conditional |\ifchilddocmanual| tells whether
% the |\includeonly| mechanism is used (false) or
% the selection of child files must be performed manually (true).
% The definitions initialise to false:
%    \begin{macrocode}
\newif\ifchilddoc
\newif\ifchilddocmanual
%    \end{macrocode}

% \macro{\childdocname}
% \macro{\childdocjob}
% The macro |\childdocname| stores the name of the main document
% to be compiled. The macro |\childdocjob| stores the name of
% the document on which the \LaTeX{} compiler was originally invoked.
% The content of |\jobname| cannot be compared
% to filenames specified in the source due to different catcodes.
% The following code rescans |\jobname|, stores the result
% in |\childdocname| and saves a copy in |\childdocjob|:
%    \begin{macrocode}
\edef\childdocname{\scantokens\expandafter{\jobname\noexpand}}
\let\childdocjob\childdocname
%    \end{macrocode}

% \macro{\childdocdisable}
% The macro |\childdocdisable| prevents the main file
% from being processed more than once.
% At this stage, the main document command |\childdocmain|
% is assumed to be called once again where it should do nothing.
% Any subsequent call to it should prevent
% a secondary processing of the main document
% It overwrites the forwarding commands
% |\childdocof| and |\childdocforward|
% with empty macros to prevent further inclusions of the main document:
%    \begin{macrocode}
\newcommand{\childdocdisable}
{
  \renewcommand{\childdocmain}[1]{\renewcommand{\childdocmain}[1]{\endinput}}
  \renewcommand{\childdocof}[1]{}
  \renewcommand{\childdocby}[2][]{}
  \renewcommand{\childdocforward}[2][]{}
  \renewcommand{\childdocdisable}{}
}
%    \end{macrocode}

% \macro{\childdocmain}
% The macro |\childdocmain| is to be called at the top of the main file
% with nothing or the main filename (without extension) as argument.
% First, it breaks loops.
% If the argument is not empty and does not match |\childdocname|
% (which is set by the first inclusion of |childdoc.def|),
% |\ifchilddoc| is set to true, |\includeonly| is applied to the child file
% and |\jobname| is set to the main file
% (for proper handling of |.aux| files):
%    \begin{macrocode}
\newcommand{\childdocmain}[1]
{
  \childdocdisable\childdocmain{}
  \if?#1?\else
    \begingroup
      \def\childdoctmp{#1}
      \ifx\childdoctmp\childdocname
        \def\childdoctmp{}
      \else
        \def\childdoctmp
        {
          \childdoctrue
          \includeonly{\childdocname}
          \def\childdocjob{#1}
          \def\jobname{#1}
        }
      \fi
      \expandafter
    \endgroup
    \childdoctmp
  \fi
}
%    \end{macrocode}

% \macro{\childdocof}
% The command |\childdocof| redirects
% compilation to the main file |#1|.
%    \begin{macrocode}
\newcommand{\childdocof}[1]
{
  \childdocdisable
  \childdoctrue
  \includeonly{\childdocname}
  \def\jobname{#1}
  \def\childdocjob{#1}
  \input{#1}
}
%    \end{macrocode}

% \macro{\childdocby}
% The command |\childdocby| ....
%    \begin{macrocode}
\newcommand{\childdocby}[2][]
{
  \childdocdisable
  \childdoctrue
  \childdocmanualtrue
  \if?#1?\else
    \def\jobname{#2}
  \fi
  \def\childdocjob{#2}
  \input{#2}
  \endinput
}
%    \end{macrocode}

% \macro{\childdocforward}
% The command |\childdocforward| redirects
% compilation to the main file or
% (if the optional argument is given) a child file.
% Parameters are set as if the main file
% or a child file starting with |\childdocof| was compiled.
% Then compilation is handed over to the main file:
%    \begin{macrocode}
\newcommand{\childdocforward}[2][]
{
  \begingroup
    \if?#1?
      \def\childdoctmp
      {
        \def\childdocname{#2}
        \def\childdocjob{#2}
        \def\jobname{#2}
        \input{#2}
        \endinput
      }
    \else
      \def\childdoctmp
      {
        \childdocdisable
        \def\childdocname{#2}
        \childdoctrue
        \includeonly{#2}
        \def\childdocjob{#1}
        \def\jobname{#1}
        \input{#1}
        \endinput
      }
    \fi
    \expandafter
  \endgroup
  \childdoctmp
}
%    \end{macrocode}

% \macro{\childdocforwardprefix}
% The command |\childdocforwardprefix| redirects
% compilation to the main or a child file by means of a pattern.
% The prefix |#1| in the current filename is replaced by |#2|
% and the suffix of the current filename is kept
% (it is assumed that the filename does not contain the substring `|~~~|'
% which is used as a delimiter).
% Compilation is handed over to the new file by |\childdocforward|:
%    \begin{macrocode}
\newcommand{\childdocforwardprefix}[3][]
{
  \begingroup
    \def\childdocextract #2##1~~~{\def\childdoctmp{\childdocforward[#1]{#3##1}}}
    \expandafter\childdocextract\childdocname~~~
    \expandafter
  \endgroup
  \childdoctmp
}
%    \end{macrocode}

% \macro{\childdoc}
% The deprecated macro |\childdoc| is a legacy version of |\childdocmain|:
%    \begin{macrocode}
\newcommand{\childdoc}{\childdocmain}
%    \end{macrocode}

% \macro{\childdocredirect}
% The deprecated macro |\childdocredirect| is a legacy version
% of |\childdocforward| and |\childdocforwardprefix|:
%    \begin{macrocode}
\newcommand{\childdocredirect}[2][]
{
  \begingroup
    \if?#1?
      \def\childdoctmp{\childdocforward{#2}}
    \else
      \def\childdoctmp{\childdocforwardprefix{#1}{#2}}
    \fi
    \expandafter
  \endgroup
  \childdoctmp
}
%    \end{macrocode}

%\iffalse
%</package>
%\fi
%
\endinput

\childdocmain{}
%    \end{macrocode}

% Optional override for |\version| flag:
%    \begin{macrocode}
%%\ifchilddoc\else\providecommand{\version}{draft}\fi
%    \end{macrocode}

% Define the default values for the |\version| flag
% (|final| for the main file and |draft| for childs):
%    \begin{macrocode}
\ifchilddoc
\providecommand{\version}{draft}
\else
\providecommand{\version}{final}
\fi
%    \end{macrocode}

% Load the standard document class:
%    \begin{macrocode}
\documentclass[12pt]{article}
%    \end{macrocode}

% Start the document body:
%    \begin{macrocode}
\begin{document}
%    \end{macrocode}

% Declare a title page.
% Print title, part of document being processed and version flag:
%    \begin{macrocode}
\addtocounter{page}{-1}
\begin{center}
{\LARGE\bfseries{}childdoc example\par}
\vspace{1cm}
\ifchilddoc
\ifchilddocmanual part\else chapter\fi:
`\childdocname' of `\childdocjob'\par
\else
main document: `\childdocjob'\par
\fi
version: \version\par
\end{center}
\newpage
%    \end{macrocode}

% Manually include selected file,
% otherwise process as usual:
%    \begin{macrocode}
\ifchilddocmanual
\section*{part `\childdocname'}
\input{\childdocname}
\else
%    \end{macrocode}

% Include the two chapters:
%    \begin{macrocode}
\include{cdocsch1}
\include{cdocsch2}
%    \end{macrocode}

% Include the two parts unless only chapters should be displayed:
%    \begin{macrocode}
\ifchilddoc\else
\section{part three}
\input{cdocspt3}
\section{part four}
\input{cdocspt4}
\fi
%    \end{macrocode}

% Process as usual until here:
%    \begin{macrocode}
\fi
%    \end{macrocode}

% End of document body:
%    \begin{macrocode}
\end{document}
%    \end{macrocode}
%\iffalse
%</samplemain>
%\fi
%
% %%%%%%%%%%%%%%%%%%%%%%%%%%%%%%%%%%%%%%
% \paragraph{Chapter Include Files.}
%
% The include files are called |cdocsch1.tex| and |cdocsch2.tex|.
%
%\iffalse
%<*samplechap1|samplechap2>
%\fi

% Optional override for |\version| flag:
%    \begin{macrocode}
%%\providecommand{\version}{final}
%    \end{macrocode}

% Include the main document:
%    \begin{macrocode}
% \iffalse
%
% childdoc.dtx Copyright (C) 2017-2018 Niklas Beisert
%
% This work may be distributed and/or modified under the
% conditions of the LaTeX Project Public License, either version 1.3
% of this license or (at your option) any later version.
% The latest version of this license is in
%   http://www.latex-project.org/lppl.txt
% and version 1.3 or later is part of all distributions of LaTeX
% version 2005/12/01 or later.
%
% This work has the LPPL maintenance status `maintained'.
%
% The Current Maintainer of this work is Niklas Beisert.
%
% This work consists of the files childdoc.dtx and childdoc.ins
% and the derived files childdoc.def and cdocsamp.tex with
% cdocsch1.tex, cdocsch2.tex, cdocsdrf.tex, cdocsfn1.tex, cdocsfn2.tex.
%
%<package>\ifdefined\childdocmain\endinput\fi
%<package>\ProvidesFile{childdoc.def}[2018/12/30 v2.0 child document driver]
%<samplemain>\ProvidesFile{cdocsamp.tex}[2018/12/30 v2.0 sample for childdoc]
%<*driver>
%\ProvidesFile{childdoc.drv}[2018/12/30 v2.0 childdoc reference manual file]
\PassOptionsToClass{10pt,a4paper}{article}
\documentclass{ltxdoc}

\usepackage[margin=35mm]{geometry}
\usepackage{hyperref}
\usepackage{hyperxmp}
\usepackage[usenames]{color}

\hypersetup{colorlinks=true}
\hypersetup{pdfstartview=FitH}
\hypersetup{pdfpagemode=UseNone}
\hypersetup{pdfsource={}}
\hypersetup{pdflang={en-UK}}
\hypersetup{pdfcopyright={Copyright 2017-2018 Niklas Beisert.
  This work may be distributed and/or modified under the
  conditions of the LaTeX Project Public License, either version 1.3
  of this license or (at your option) any later version.}}
\hypersetup{pdflicenseurl={http://www.latex-project.org/lppl.txt}}
\hypersetup{pdfcontactaddress={ETH Zurich, ITP, HIT K,
  Wolfgang-Pauli-Strasse 27}}
\hypersetup{pdfcontactpostcode={8093}}
\hypersetup{pdfcontactcity={Zurich}}
\hypersetup{pdfcontactcountry={Switzerland}}
\hypersetup{pdfcontactemail={nbeisert@itp.phys.ethz.ch}}
\hypersetup{pdfcontacturl={http://people.phys.ethz.ch/\xmptilde nbeisert/}}

\newcommand{\secref}[1]{\hyperref[#1]{section \ref*{#1}}}

\parskip1ex
\parindent0pt
\let\olditemize\itemize
\def\itemize{\olditemize\parskip0pt}

\begin{document}

\title{The \textsf{childdoc} Package}
\hypersetup{pdftitle={The childdoc Package}}
\author{Niklas Beisert\\[2ex]
  Institut f\"ur Theoretische Physik\\
  Eidgen\"ossische Technische Hochschule Z\"urich\\
  Wolfgang-Pauli-Strasse 27, 8093 Z\"urich, Switzerland\\[1ex]
  \href{mailto:nbeisert@itp.phys.ethz.ch}
  {\texttt{nbeisert@itp.phys.ethz.ch}}}
\hypersetup{pdfauthor={Niklas Beisert}}
\hypersetup{pdfsubject={Manual for the LaTeX2e Package childdoc}}
\date{30 December 2018, \textsf{v2.0}}
\maketitle

\begin{abstract}\noindent
\textsf{childdoc} is a \LaTeXe{} package
that enables the direct compilation
of document sections included by |\include|
to individual files.
\end{abstract}

\begingroup
\parskip0ex
\tableofcontents
\endgroup

%%%%%%%%%%%%%%%%%%%%%%%%%%%%%%%%%%%%%%%%%%%%%%%%%%%%%%%%%%%%%%%%%%%%%%%%%%%%%%%%
%%%%%%%%%%%%%%%%%%%%%%%%%%%%%%%%%%%%%%%%%%%%%%%%%%%%%%%%%%%%%%%%%%%%%%%%%%%%%%%%
\section{Introduction}

\LaTeX{} provides a mechanism to structure a large document (such as a book)
into a main file and several child files (containing the chapters)
using the |\include| command.
This mechanism is beneficial for documents
which span hundreds of pages in order to
make the source file(s) more manageable.
Moreover, compilation can be restricted to
selected child files by means of the |\includeonly| command.
The latter feature can be used to reduce the compilation time while editing
(this was significantly more useful in the earlier days of \LaTeX{})
or to generate a smaller document which is easier to navigate.
Another application of |\includeonly| is to generate
documents consisting of selected parts of the complete document.

However, there are a few drawbacks of the plain |\include| mechanism:
\begin{itemize}
\item
The child files cannot be compiled on their own,
they can only be compiled via the main file.
A naive editing environment
(such as a text editor with an option
to have the current file processed by \LaTeX)
may require one to switch to the main file before compiling;
attempting to compile the child file produces errors.
\item
The main file must be modified (each time)
to adjust the |\includeonly| command
to the present needs. This easily leaves the main file in a messy state.
\item
The generated document will always carry the filename
of the main document. This is inconvenient if
several child files are to be compiled and
to be kept for distribution.
\end{itemize}

The present package provides a simple interface
to make child files individually compilable by \LaTeX{}.
Compiling a child file then has the same effect as compiling
the main file with an |\includeonly| command
to select the appropriate child.
Moreover the generated document will carry the name of the child
rather than the main file.
This resolves all three above issues.

This feature is meant to make the editing of books,
thesis documents and lecture notes somewhat more convenient.
However, the package can also be used efficiently for
composing a series of documents (such as exercise sheets)
which are typically distributed individually.
It then assists the author in generating the individual documents
(potentially in different versions)
as well as a document containing the collected series.
Another application is in developing style files
or other kinds of included material
where compilation of the style file could redirect
to a sample or test file.

%%%%%%%%%%%%%%%%%%%%%%%%%%%%%%%%%%%%%%%%%%%%%%%%%%%%%%%%%%%%%%%%%%%%%%%%%%%%%%%%
%%%%%%%%%%%%%%%%%%%%%%%%%%%%%%%%%%%%%%%%%%%%%%%%%%%%%%%%%%%%%%%%%%%%%%%%%%%%%%%%
\section{Usage}

First of all, the package \textsf{childdoc} is \emph{not} a standard
\LaTeXe{} |.sty| style file! Therefore it needs to be invoked in
a non-standard way.

%%%%%%%%%%%%%%%%%%%%%%%%%%%%%%%%%%%%%%%%%%%%%%%%%%%%%%%%%%%%%%%%%%%%%%%%%%%%%%%%
\subsection{Included Files}
\label{sec:include}

%%%%%%%%%%%%%%%%%%%%%%%%%%%%%%%%%%%%%%%%
\DescribeMacro{\childdocmain}
To use the package, add the commands
\begin{center}
\begin{tabular}{l}
|\input{childdoc.def}|\\
|\childdocmain{}|\\
\end{tabular}
\end{center}
at the very top of the main \LaTeX{} file,
in particular \emph{before} the |\documentclass| statement!
The argument of |\childdocmain| should be left empty
(but it must be present).

%%%%%%%%%%%%%%%%%%%%%%%%%%%%%%%%%%%%%%%%
\DescribeMacro{\childdocof}
Furthermore, add the commands
\begin{center}
\begin{tabular}{l}
|\input{childdoc.def}|\\
|\childdocof{|\textit{main}|}|\\
\end{tabular}
\end{center}
at the top of every child file \textit{child}
which is included by |\include{|\textit{child}|}|
from within the main file
(or at least for those files to be compiled individually).
The argument \textit{main} must be the filename of the main file.

There are a couple of
considerations in setting up the main and child documents:

%%%%%%%%%%%%%%%%%%%%%%%%%%%%%%%%%%%%%%%%
\paragraph{Restrictions.}

Please note the following restrictions:
\begin{itemize}
\item
|\childdocmain| must be called with one argument \textit{main}
to ensure compatibility with earlier version of the package.
It must either be empty (|\childdocmain{}|)
or precisely match the filename of the main file in which it is specified.
See \secref{sec:detection} for further information.
\item
The filename \textit{main} must be specified without the |.tex| extension.
\item
The filename \textit{main} is case sensitive
(even in case-insensitive file systems)
due to internal string comparison.
\item
The argument \textit{main} should be fully expanded, it cannot be a macro.
\item
Subdirectories and special characters should be avoided in filenames.
\item
The command |\childdocmain{|\textit{main}|}| must be followed by a whitespace.
It should not be followed immediately by another command
or by a comment mark `|%|'.
This is because the \TeX{} parser reads the token immediately following
the argument of |\childdocmain| and puts it
at the beginning of every child section;
however, a white\-space is ignored.
\end{itemize}

%%%%%%%%%%%%%%%%%%%%%%%%%%%%%%%%%%%%%%%%
\paragraph{Content of Main File.}

It is advisable to place all content in the child files included by |\include|.
Any output contained in the main file will appear in all child documents
unless suppressed manually;
it cannot be suppressed automatically by the |\includeonly| directive
and thus should normally be avoided.
A method to include some content in the main file
by means of conditional processing is described in \secref{sec:conditional}.

%%%%%%%%%%%%%%%%%%%%%%%%%%%%%%%%%%%%%%%%
\paragraph{Page Numbering.}

When only a part of the document is compiled,
the appropriate numbering of pages
(as well as other status parameters)
is determined from the |.aux| files.
The latter contain information from previous passes.
However this information needs to propagate through
all intermediate child documents.
Therefore the page numbering in child documents may well
be inconsistent until the complete document is compiled at least once.

A useful (if unconventional) way to always ensure a consistent
page numbering is to restart the numbering in each child document
and denote the pages by `\textit{child}|.|\textit{page}'
where \textit{child} represents the chapter/section number of the child file.
This can be achieved by the command
|\numberwithin{page}{|\textit{child}|}|
of the \textsf{amsmath} package
where \textit{child} can be |chapter| or |section|
depending on the chosen structuring.
Alternatively, one can modify the macro |\thepage| appropriately
and reset the counter |page| at the start of each child file.

%%%%%%%%%%%%%%%%%%%%%%%%%%%%%%%%%%%%%%%%%%%%%%%%%%%%%%%%%%%%%%%%%%%%%%%%%%%%%%%%
\subsection{Conditional Processing}
\label{sec:conditional}

The package provides a mechanism to compile different versions
of a document. To customise the versions further some conditional processing
can come in handy to distinguish which version is being compiled.
The package provides two macros to describe the compilation context:

%%%%%%%%%%%%%%%%%%%%%%%%%%%%%%%%%%%%%%%%
\DescribeMacro{\ifchilddoc}
The conditional |\ifchilddoc| distinguishes between the compilation of
child documents and the main document:
%
\begin{center}
|\ifchilddoc |\textit{child-code}| |[|\||else |\textit{main-code}]| \||fi|
\end{center}

%%%%%%%%%%%%%%%%%%%%%%%%%%%%%%%%%%%%%%%%
\DescribeMacro{\childdocname}
\DescribeMacro{\childdocjob}
The macro |\childdocname| contains the filename (without extension)
of the main or child file being processed.
Note that |\childdocjob| will always contain the name of the main file.

%%%%%%%%%%%%%%%%%%%%%%%%%%%%%%%%%%%%%%%%
\paragraph{Title Page.}

Conditional processing can be used to include a title or banner page
in the main document when proper precautions are taken.
Importantly, the code in the main file should ensure that the page counter
(as well as other status parameters which are stored in the |.aux| files)
takes the same value after the conditional processing.
Otherwise the page numbers may take divergent values
depending on which part is compiled.

For example, a title page could be declared by:
%
\begin{center}
\begin{tabular}{l}
|\ifchilddoc\||else|\\
|\addtocounter{page}{-1}|\\
\textit{code for title page}\\
|\newpage|\\
|\||fi|
\end{tabular}
\end{center}
%
A banner page for the child documents can be generated by:
%
\begin{center}
\begin{tabular}{l}
|\ifchilddoc|\\
|\addtocounter{page}{-1}|\\
\textit{code for banner page}\\
|\newpage|\\
|\||fi|
\end{tabular}
\end{center}
%
Here one could write a message such as:
\begin{center}
|This is the part \childdocname{} of \childdocjob{}.|
\end{center}

%%%%%%%%%%%%%%%%%%%%%%%%%%%%%%%%%%%%%%%%%%%%%%%%%%%%%%%%%%%%%%%%%%%%%%%%%%%%%%%%
\subsection{Flags}
\label{sec:flags}

The package makes it easy to generate different versions
of the main or child documents.
To this end compilation flags can be defined
and assigned different default values.
They will be particularly useful in conjunction
with the forwarding mechanism described in \secref{sec:forward}.

For example, it may be useful to have a flag |\version|
which can be set to |draft| or |final|.
The document source will contain some conditional code
depending on the value of |\version|.
Suppose further, the flag should default to |final| for the main file
and to |draft| for child files
which is a natural assignment for editing the document.
This is achieved by placing the following code
in the preamble of the main document
(below the |\childdocmain| directive):
%
\begin{center}
\begin{tabular}{l}
|\ifchilddoc|\\
|\providecommand{\version}{draft}|\\
|\||else|\\
|\providecommand{\version}{final}|\\
|\||fi|
\end{tabular}
\end{center}
%
The definition by |\providecommand| makes sure
that previous definitions are not overwritten.
Further statements |\providecommand{\version}{...}|
can thus be added before the above code to override it.

For the main file, one might add a line
(between |\childdocmain| and the above block)
%
\begin{center}
|%\ifchilddoc\||else\providecommand{\version}{draft}\||fi|
\end{center}
%
which can be uncommented to produce a draft version.
Likewise one can add a line to the very top of a child file
(above the |\childdocof{|\textit{main}|}| directive)
%
\begin{center}
|%\providecommand{\version}{final}|
\end{center}
%
which can be uncommented to produce the final version of this child document.

%%%%%%%%%%%%%%%%%%%%%%%%%%%%%%%%%%%%%%%%%%%%%%%%%%%%%%%%%%%%%%%%%%%%%%%%%%%%%%%%
\subsection{Forwarding}
\label{sec:forward}

Different versions of the main or child documents
using compilation flags as described in \secref{sec:flags}
can be (permanently) stored in different files
for convenient compilation, viewing and distribution.
To this end, the package defines a command
to pass on compilation to a different file:

%%%%%%%%%%%%%%%%%%%%%%%%%%%%%%%%%%%%%%%%
\DescribeMacro{\childdocforward}
The command |\childdocforward| redirects processing to
another source file:
%
\begin{center}
\begin{tabular}{l}
|\input{childdoc.def}|\\
|\childdocforward[|\textit{main}|]{|\textit{dest}|}|\\
\end{tabular}
\end{center}
%
The argument \textit{dest} is the destination file
(without extension).
It should be the main file or one of the child files.
Note that further \textsf{childdoc} directives
such as |\childdocof| and |\childdocforward|
in the indicated file will be processed in this form.
The optional argument \textit{main}
passes on directly to the main file \textit{main}
while pretending to compile the child \textit{dest}.
This form behaves as if \textit{dest}
issues |\childdocof{|\textit{main}|}| right away,
and no further \textsf{childdoc} directives will be processed.

%%%%%%%%%%%%%%%%%%%%%%%%%%%%%%%%%%%%%%%%
\DescribeMacro{\...prefix}
In the alternative form |\childdocforwardprefix|,
%
\begin{center}
\begin{tabular}{l}
|\input{childdoc.def}|\\
|\childdocforwardprefix[|\textit{main}|]{|\textit{prefix}|}{|\textit{dest}|}|
\end{tabular}
\end{center}
%
the destination file is determined by a pattern
depending on the current file:
To make this work, the current file must be called
`{\textit{prefix}\hspace{0.2em}\textit{suffix}}'
with \textit{prefix} matching precisely the argument.
Processing is then passed on to the file
`{\textit{dest}\hspace{0.2em}\textit{suffix}}'.
Surely, the same effect is achieved by
directly specifying the
argument `{\textit{dest}\hspace{0.2em}\textit{suffix}}'
in the first form.
However, that requires to set up a different file
for each child. With the alternative form of the command
all these files can have exactly the same content
which simplifies setting them up and maintaining them.

For example, the following file |draft.tex|
with a compilation flag |\version| as described in \secref{sec:flags}
compiles the main document as a draft:
%
\begin{center}
\begin{tabular}{l}
|\def\version{draft}|\\
|\input{childdoc.def}|\\
|\childdocforward{|\textit{main}|}|
\end{tabular}
\end{center}
%
Likewise, the following files |final|\textit{nn}|.tex|
compile the final version of the child document
|child|\textit{nn}|.tex|:
%
\begin{center}
\begin{tabular}{l}
|\def\version{final}|\\
|\input{childdoc.def}|\\
|\childdocforwardprefix{final}{child}|
\end{tabular}
\end{center}
%

Note that when several versions of a main file and/or of each child file
are to be generated, it may be convenient to set up a |Makefile| or
shell script to automatise the process.

%%%%%%%%%%%%%%%%%%%%%%%%%%%%%%%%%%%%%%%%%%%%%%%%%%%%%%%%%%%%%%%%%%%%%%%%%%%%%%%%
\subsection{Command Line Processing}
\label{sec:commandline}

The effect of redirection files can also be achieved by invoking
the \LaTeX{} compiler with a more elaborate command line.
Most conveniently this should be done as part
of a shell script or a |Makefile|.

When using \textsf{childdoc} in the main file, the following
command lines effectively perform a redirection
(note that depending on the shell being used,
backslashes may have to be doubled: `|\|' $\to$ `|\\|'):
%
\begin{center}
|... -jobname "|\textit{target}|" |\\|"|[\textit{flags}]%
|\input{childdoc.def}\childdocforward[|\textit{main}|]{|\textit{dest}|}"|
\end{center}
%
Here \textit{target} is the name of the output file,
\textit{main} is the name of the main file
and \textit{dest} is the name of the main or child file to be processed
(all filenames without extensions).
The optional argument \textit{main} can be omitted
if \textit{main} matches \textit{dest}.
Optionally, compilation \textit{flags} can be defined via |\def| commands.
This command line makes the \TeX{} engine believe
it is compiling the file \textit{target}
whose content is specified as the latter parameter.
The provided code then forwards the processing to
\textit{main} or \textit{dest} as described in \secref{sec:forward}.

%%%%%%%%%%%%%%%%%%%%%%%%%%%%%%%%%%%%%%%%%%%%%%%%%%%%%%%%%%%%%%%%%%%%%%%%%%%%%%%%
\subsection{Include by Input}
\label{sec:input}

Including child documents by |\include| has some restrictions by design.
Most notably, the content of a child document always occupies
its own set of pages; pages cannot be shared between child documents.
Usually, this behaviour makes perfect sense
because each child document contain an essential part of the document.
However, in some situations it may be desirable to compose
a document from a collection of parts
without having mandatory page breaks between then.
For this case, the package
provides a mechanism to include parts
by |\input| which can also be processed individually.
However, by construction this mechanism
requires manual handling of the content to be output.

%%%%%%%%%%%%%%%%%%%%%%%%%%%%%%%%%%%%%%%%
\DescribeMacro{\ifchilddocmanual}
The main file should be prepared as usual, see \secref{sec:include}.
However, the document body must make a distinction
between processing of an individual part and of the main document, e.g.:
%
\begin{center}
\begin{tabular}{l}
|\ifchilddocmanual|\\
|\input{\childdocname}|\\
|\||else|\\
\textit{document body with }|\input{|\textit{part}|}|\\
|\||fi|
\end{tabular}
\end{center}
%
The conditional |\ifchilddocmanual| is true whenever
a part to be included by |\input| is being compiled,
and the name of the part is stored in |\childdocname|.

%%%%%%%%%%%%%%%%%%%%%%%%%%%%%%%%%%%%%%%%
\DescribeMacro{\childdocby}
Each part to be included by |\input| should start with:
%
\begin{center}
\begin{tabular}{l}
|\input{childdoc.def}|\\
|\childdocby{|\textit{main}|}|\\
\end{tabular}
\end{center}
%
The directive |\childdocby| is similar to |\childdocof|
described in \secref{sec:include},
but the subsequent selection of content must be done manually.
To that end, both |\ifchilddoc| and |\ifchilddocmanual|
will be true upon processing of a part,
and the name of the part is stored in |\childdocname|.
Note that |\jobname| will be set to the filename of the current part
so that each part receives an individual |.aux| file
that does not interfere with the |.aux| file(s) of the main document.
This behaviour can be altered by the alternative form
|\childdocby[*]{|\textit{main}|}| (with a non-empty optional argument)
which uses the |.aux| file of the main document
by setting |\jobname| to \textit{main}.

%%%%%%%%%%%%%%%%%%%%%%%%%%%%%%%%%%%%%%%%%%%%%%%%%%%%%%%%%%%%%%%%%%%%%%%%%%%%%%%%
\subsection{Driver Development}
\label{sec:driver}

The \textsf{childdoc} mechanism can also be use for the development
of definition files such as \LaTeX{} styles or classes.
This case differs from the above setup with multiple parts
included by |\include| in that no |\includeonly| should be invoked.
This can be achieved by starting the include file
(before |\ProvidesPackage|) with:
%
\begin{center}
\begin{tabular}{l}
|\input{childdoc.def}|\\
|\childdocforward{|\textit{main}|}|\\
\end{tabular}
\end{center}
%
or alternatively with:
%
\begin{center}
\begin{tabular}{l}
|\input{childdoc.def}|\\
|\childdocby{|\textit{main}|}|\\
\end{tabular}
\end{center}
%
Both forms have slightly different effects as described above.
The main file is prepared as usual, see \secref{sec:include}.

%%%%%%%%%%%%%%%%%%%%%%%%%%%%%%%%%%%%%%%%%%%%%%%%%%%%%%%%%%%%%%%%%%%%%%%%%%%%%%%%
\subsection{Legacy Detection}
\label{sec:detection}

The directive |\childdocmain| in the main file can detect
whether the complete document or merely a child is to be compiled
even without using the directive |\childdocof|.
This method is deprecated because it is less robust
and there is no compelling reason to use it;
it is merely provided for backward compatibility
and it may be removed in future versions.

If the detection mechanism is to be used,
it is mandatory to correctly specify
the filename of the main file as the argument of |\childdocmain|:
%
\begin{center}
\begin{tabular}{l}
|\input{childdoc.def}|\\
|\childdocmain{|\textit{main}|}|\\
\end{tabular}
\end{center}
%
If |\jobname| does not match the argument \textit{main} of |\childdocmain|,
it is assumed that |\jobname| points to the child file to be compiled.
When using |\childdocmain| with the main file specified as argument,
it suffices to start a child file
with just |\input{|\textit{main}|}|
without loading of the package and using |\childdocof|.
If instead all processing is done
with the appropriate \textsf{childdoc} directives,
the argument of \textit{main} of |\childdocmain| can be empty.

An alternative version of the command line processing described
in \secref{sec:commandline} using the detection mechanism reads:
%
\begin{center}
|... -jobname "|\textit{target}|" "|[\textit{flags}]%
[|\def\jobname{|\textit{dest}|}|]|\input{|\textit{main}|}"|
\end{center}

%%%%%%%%%%%%%%%%%%%%%%%%%%%%%%%%%%%%%%%%%%%%%%%%%%%%%%%%%%%%%%%%%%%%%%%%%%%%%%%%
\subsection{Manual Code}
\label{sec:manual}

In case one cannot be certain whether the definitions file |childdoc.def|
is installed on the target \TeX{} distribution
and one prefers not to ship it,
it is conceivable to paste a few relevant commands into the sources.

To that end, drop all statements |\input{childdoc.def}|
and perform the replacements as outlined below.
Instead of |\childdocmain{|\textit{main}|}| add the following code
to the top of the main file:
%
\begin{center}
\begin{tabular}{l}
|\||ifdefined\childdocname\endinput\||fi\newif\ifchilddoc|\\
|\edef\childdocname{\scantokens\expandafter{\jobname\noexpand}}|\\
|\def\childdocmain{|\textit{main}|}\||ifx\childdocmain\childdocname\||else|\\
|\childdoctrue\includeonly{\childdocname}\let\jobname\childdocmain\||fi|\\
\end{tabular}
\end{center}
%
Instead of |\childdocof{|\textit{main}|}| just include the main file
at the top of each child file:
%
\begin{center}
|\input{|\textit{main}|}|
\end{center}
%
A simple redirection |\childdocforward{|\textit{dest}|}| is achieved by:
%
\begin{center}
|\def\jobname{|\textit{dest}|}\input{\jobname}|
\end{center}
%
The redirection with prefix
|\childdocforwardprefix[|\textit{prefix}|]{|\textit{dest}|}|
is accomplished by:
%
\begin{center}
\begin{tabular}{l}
|{\edef\jobname{\scantokens\expandafter{\jobname\noexpand}}|\\
|\def\redirectjob |\textit{prefix}|#1~~~{\gdef\jobname{|\textit{dest}|#1}}|\\
|\expandafter\redirectjob\jobname~~~}\input{\jobname}|
\end{tabular}
\end{center}

In an alternative approach,
child documents can be compiled by a specific command line
without additional code or specific definitions:
%
\begin{center}
|... -jobname "|\textit{target}|" "|[\textit{flags}]%
|\includeonly{|\textit{dest}|}\input{|\textit{main}|}"|
\end{center}
%

%%%%%%%%%%%%%%%%%%%%%%%%%%%%%%%%%%%%%%%%%%%%%%%%%%%%%%%%%%%%%%%%%%%%%%%%%%%%%%%%
%%%%%%%%%%%%%%%%%%%%%%%%%%%%%%%%%%%%%%%%%%%%%%%%%%%%%%%%%%%%%%%%%%%%%%%%%%%%%%%%
\section{Information}

%%%%%%%%%%%%%%%%%%%%%%%%%%%%%%%%%%%%%%%%%%%%%%%%%%%%%%%%%%%%%%%%%%%%%%%%%%%%%%%%
\subsection{Copyright}

Copyright \copyright{} 2017--2018 Niklas Beisert

This work may be distributed and/or modified under the
conditions of the \LaTeX{} Project Public License, either version 1.3
of this license or (at your option) any later version.
The latest version of this license is in
  \url{http://www.latex-project.org/lppl.txt}
and version 1.3 or later is part of all distributions of \LaTeX{}
version 2005/12/01 or later.

This work has the LPPL maintenance status `maintained'.

The Current Maintainer of this work is Niklas Beisert.

This work consists of the files |README.txt|, |childdoc.ins| and |childdoc.dtx|
as well as the derived files |childdoc.def|, |cdocsamp.tex|
with |cdocsch1.tex|, |cdocsch2.tex|, |cdocspt3.tex|, |cdocspt4.tex|,
|cdocsdrf.tex|, |cdocsfn1.tex|, |cdocsfn2.tex|
as well as |childdoc.pdf|.

%%%%%%%%%%%%%%%%%%%%%%%%%%%%%%%%%%%%%%%%%%%%%%%%%%%%%%%%%%%%%%%%%%%%%%%%%%%%%%%%
\subsection{Files and Installation}

The package consists of the files:
%
\begin{center}
\begin{tabular}{ll}
    |README.txt|   & readme file \\
    |childdoc.ins| & installation file \\
    |childdoc.dtx| & source file \\
    |childdoc.def| & definition file \\
    |cdocsamp.tex| & sample main file \\
    |cdocsch1.tex| & sample include file \\
    |cdocsch2.tex| & sample include file \\
    |cdocspt3.tex| & sample part file \\
    |cdocspt4.tex| & sample part file \\
    |cdocsdrf.tex| & sample redirection file \\
    |cdocsfn1.tex| & sample redirection file \\
    |cdocsfn2.tex| & sample redirection file \\
    |childdoc.pdf| & manual
\end{tabular}
\end{center}
%
The distribution consists of the files
|README.txt|, |childdoc.ins| and |childdoc.dtx|.
%
\begin{itemize}
\item
Run (pdf)\LaTeX{} on |childdoc.dtx|
to compile the manual |childdoc.pdf| (this file).
\item
Run \LaTeX{} on |childdoc.ins| to create the definitions file |childdoc.def|
and the sample |cdocsamp.tex| with include files
|cdocsch1.tex|, |cdocsch2.tex|, |cdocspt3.tex|, |cdocspt4.tex|,
|cdocsdrf.tex|, |cdocsfn1.tex|, |cdocsfn2.tex|.
Then copy the file |childdoc.def| to an appropriate directory of your \LaTeX{}
distribution, e.g.\ \textit{texmf-root}|/tex/latex/childdoc|.
\end{itemize}

%%%%%%%%%%%%%%%%%%%%%%%%%%%%%%%%%%%%%%%%%%%%%%%%%%%%%%%%%%%%%%%%%%%%%%%%%%%%%%%%
\subsection{Related CTAN Packages}

There are several other packages which offer a similar functionality:
%
\begin{itemize}
\item
The packages
\href{http://ctan.org/pkg/docmute}{\textsf{docmute}},
\href{http://ctan.org/pkg/includex}{\textsf{includex}} and
\href{http://ctan.org/pkg/standalone}{\textsf{standalone}}
provide commands to include only the document body of
a child file thus allowing both files to be compiled individually.
\item
The packages \href{http://ctan.org/pkg/subdocs}{\textsf{subdocs}}
and \href{http://ctan.org/pkg/subfiles}{\textsf{subfiles}}
provide structures in which the main and child documents can be
encapsulated and allowing them to be compiled individually.
The inclusion mechanism is different from the conventional |\include|.
\item
The package \href{http://ctan.org/pkg/combine}{\textsf{combine}}
is an elaborate solution to combine several documents into one.
\end{itemize}
%
See also the CTAN topic \href{http://ctan.org/topic/subdocs}{\textsf{subdocs}}
for further related packages.
The present package differs from the above solutions in that
a document structure constructed with the conventional |\include| mechanism
just needs two extra commands at the top of every file
such that all constituent files can be compiled individually.

%%%%%%%%%%%%%%%%%%%%%%%%%%%%%%%%%%%%%%%%%%%%%%%%%%%%%%%%%%%%%%%%%%%%%%%%%%%%%%%%
%\subsection{Feature Suggestions}
%
%The following is a list of features which may be useful for future
%versions of this package:
%%
%\begin{itemize}
%\item
%\ldots
%\end{itemize}

%%%%%%%%%%%%%%%%%%%%%%%%%%%%%%%%%%%%%%%%%%%%%%%%%%%%%%%%%%%%%%%%%%%%%%%%%%%%%%%%
\subsection{Revision History}

%%%%%%%%%%%%%%%%%%%%%%%%%%%%%%%%%%%%%%%%
\paragraph{v2.0:} 2018/12/30

\begin{itemize}
\item
immediate forward processing
\item
added |\childdocby| mechanism
\item
manual restructured
\end{itemize}

%%%%%%%%%%%%%%%%%%%%%%%%%%%%%%%%%%%%%%%%
\paragraph{v1.6:} 2018/01/17

\begin{itemize}
\item
application for development of include files
\item
corrections to manual
\end{itemize}

%%%%%%%%%%%%%%%%%%%%%%%%%%%%%%%%%%%%%%%%
\paragraph{v1.5:} 2017/05/21

\begin{itemize}
\item
more complete structuring introduced
\item
|\childdocof| introduced
\item
|\childdoc| renamed to |\childdocmain|
\item
|\childredirect| renamed to |\childdocforward| and |\childdocforwardprefix|
and functionality expanded
\end{itemize}

%%%%%%%%%%%%%%%%%%%%%%%%%%%%%%%%%%%%%%%%
\paragraph{v1.0:} 2017/04/27

\begin{itemize}
\item
manual and install package
\item
first version published on CTAN
\end{itemize}

%%%%%%%%%%%%%%%%%%%%%%%%%%%%%%%%%%%%%%%%
\paragraph{v0.6:} 2017/04/26

\begin{itemize}
\item
redirection mechanism added
\end{itemize}

%%%%%%%%%%%%%%%%%%%%%%%%%%%%%%%%%%%%%%%%
\paragraph{v0.5:} 2017/04/26

\begin{itemize}
\item
functionality in definition file
\end{itemize}


%%%%%%%%%%%%%%%%%%%%%%%%%%%%%%%%%%%%%%%%%%%%%%%%%%%%%%%%%%%%%%%%%%%%%%%%%%%%%%%%
%%%%%%%%%%%%%%%%%%%%%%%%%%%%%%%%%%%%%%%%%%%%%%%%%%%%%%%%%%%%%%%%%%%%%%%%%%%%%%%%
%%%%%%%%%%%%%%%%%%%%%%%%%%%%%%%%%%%%%%%%%%%%%%%%%%%%%%%%%%%%%%%%%%%%%%%%%%%%%%%%
\appendix

\settowidth\MacroIndent{\rmfamily\scriptsize 000\ }

 \DocInput{childdoc.dtx}

\end{document}
%</driver>
% \fi
%
% %%%%%%%%%%%%%%%%%%%%%%%%%%%%%%%%%%%%%%%%%%%%%%%%%%%%%%%%%%%%%%%%%%%%%%%%%%%%%%
% %%%%%%%%%%%%%%%%%%%%%%%%%%%%%%%%%%%%%%%%%%%%%%%%%%%%%%%%%%%%%%%%%%%%%%%%%%%%%%
% \section{Sample}
%\iffalse
%<*samplemain>
%\fi
%
% The following presents a sample document
% with two chapters, two parts, a title page,
% a compile flag as well as three forwarding files to set the flag.
% It consists of eight |.tex| files:
% \begin{center}
% \begin{tabular}{ll}
% |cdocsamp.tex|&main file\\
% |cdocsch1.tex|&include file for chapter 1\\
% |cdocsch2.tex|&include file for chapter 2\\
% |cdocspt3.tex|&include file for part 3\\
% |cdocspt4.tex|&include file for part 4\\
% |cdocsdrf.tex|&forwarding file for main file in draft mode\\
% |cdocsfi1.tex|&forwarding file for final version of chapter 1\\
% |cdocsfi2.tex|&forwarding file for final version of chapter 2\\
% \end{tabular}
% \end{center}
% Each of the eight files can be compiled directly by the \LaTeX{} compiler.
%
% %%%%%%%%%%%%%%%%%%%%%%%%%%%%%%%%%%%%%%
% \paragraph{Main File.}
%
% The main file is called |cdocsamp.tex|.
%
% Load the \textsf{childdoc} definitions and
% declare the filename for the main document:
%    \begin{macrocode}
\input{childdoc.def}
\childdocmain{}
%    \end{macrocode}

% Optional override for |\version| flag:
%    \begin{macrocode}
%%\ifchilddoc\else\providecommand{\version}{draft}\fi
%    \end{macrocode}

% Define the default values for the |\version| flag
% (|final| for the main file and |draft| for childs):
%    \begin{macrocode}
\ifchilddoc
\providecommand{\version}{draft}
\else
\providecommand{\version}{final}
\fi
%    \end{macrocode}

% Load the standard document class:
%    \begin{macrocode}
\documentclass[12pt]{article}
%    \end{macrocode}

% Start the document body:
%    \begin{macrocode}
\begin{document}
%    \end{macrocode}

% Declare a title page.
% Print title, part of document being processed and version flag:
%    \begin{macrocode}
\addtocounter{page}{-1}
\begin{center}
{\LARGE\bfseries{}childdoc example\par}
\vspace{1cm}
\ifchilddoc
\ifchilddocmanual part\else chapter\fi:
`\childdocname' of `\childdocjob'\par
\else
main document: `\childdocjob'\par
\fi
version: \version\par
\end{center}
\newpage
%    \end{macrocode}

% Manually include selected file,
% otherwise process as usual:
%    \begin{macrocode}
\ifchilddocmanual
\section*{part `\childdocname'}
\input{\childdocname}
\else
%    \end{macrocode}

% Include the two chapters:
%    \begin{macrocode}
\include{cdocsch1}
\include{cdocsch2}
%    \end{macrocode}

% Include the two parts unless only chapters should be displayed:
%    \begin{macrocode}
\ifchilddoc\else
\section{part three}
\input{cdocspt3}
\section{part four}
\input{cdocspt4}
\fi
%    \end{macrocode}

% Process as usual until here:
%    \begin{macrocode}
\fi
%    \end{macrocode}

% End of document body:
%    \begin{macrocode}
\end{document}
%    \end{macrocode}
%\iffalse
%</samplemain>
%\fi
%
% %%%%%%%%%%%%%%%%%%%%%%%%%%%%%%%%%%%%%%
% \paragraph{Chapter Include Files.}
%
% The include files are called |cdocsch1.tex| and |cdocsch2.tex|.
%
%\iffalse
%<*samplechap1|samplechap2>
%\fi

% Optional override for |\version| flag:
%    \begin{macrocode}
%%\providecommand{\version}{final}
%    \end{macrocode}

% Include the main document:
%    \begin{macrocode}
\input{childdoc.def}
\childdocof{cdocsamp}
%    \end{macrocode}

%\iffalse
%</samplechap1|samplechap2>
%\fi
%
%\iffalse
%<*samplechap1>
%\fi
% Some text for chapter 1:
%    \begin{macrocode}
\section{one}
some text in chapter one
%    \end{macrocode}

%\iffalse
%</samplechap1>
%\fi
% Some text for chapter 2:
%\iffalse
%<*samplechap2>
%\fi
%    \begin{macrocode}
\section{two}
more text in chapter two
%    \end{macrocode}

%\iffalse
%</samplechap2>
%\fi
%
% %%%%%%%%%%%%%%%%%%%%%%%%%%%%%%%%%%%%%%
% \paragraph{Part Include Files.}
%
% The include files are called |cdocspt3.tex| and |cdocspt4.tex|.
%
%\iffalse
%<*samplepart3|samplepart4>
%\fi

% Optional override for |\version| flag:
%    \begin{macrocode}
%%\providecommand{\version}{final}
%    \end{macrocode}

% Include the main document:
%    \begin{macrocode}
\input{childdoc.def}
\childdocby{cdocsamp}
%    \end{macrocode}

%\iffalse
%</samplepart3|samplepart4>
%\fi
%
%\iffalse
%<*samplepart3>
%\fi
% Some text for part 3:
%    \begin{macrocode}
some text in part three
%    \end{macrocode}

%\iffalse
%</samplepart3>
%\fi
% Some text for part 4:
%\iffalse
%<*samplepart4>
%\fi
%    \begin{macrocode}
more text in part four
%    \end{macrocode}

%\iffalse
%</samplepart4>
%\fi
%
% %%%%%%%%%%%%%%%%%%%%%%%%%%%%%%%%%%%%%%
% \paragraph{Forwarding for a Complete Draft.}
%
% The following forwarding file |cdocsdrf.tex|
% compiles the main document in draft mode:
%\iffalse
%<*sampledraft>
%\fi
%    \begin{macrocode}
\def\version{draft}
\input{childdoc.def}
\childdocforward{cdocsamp}
%    \end{macrocode}

%\iffalse
%</sampledraft>
%\fi
%
% %%%%%%%%%%%%%%%%%%%%%%%%%%%%%%%%%%%%%%
% \paragraph{Forwarding for Final Version of the Chapters.}
%
% The following forwarding files |cdocsfn1.tex| and |cdocsfn2.tex|
% (with identical content)
% compile the final versions of the child documents
% |cdocsch1.tex| and |cdocsch2.tex|, respectively:
%\iffalse
%<*samplefinal>
%\fi
%    \begin{macrocode}
\def\version{final}
\input{childdoc.def}
\childdocforwardprefix[cdocsamp]{cdocsfn}{cdocsch}
%    \end{macrocode}

%\iffalse
%</samplefinal>
%\fi
%
% %%%%%%%%%%%%%%%%%%%%%%%%%%%%%%%%%%%%%%
% \paragraph{Command Line Processing.}
%
% The following three command lines generate the output files
% |cdocscld|, |cdocscl1| and |cdocscl2|
% which should be identical to
% |cdocsdrf|, |cdocsch1| and |cdocsfn2|, respectively:
% \begin{center}
% \begin{tabular}{l}
% |latex -jobname cdocscld \|\\
% |  "\def\version{draft}\input{childdoc.def}\childdocforward{cdocsamp}"|\\
% |latex -jobname cdocscl1 \|\\
% |  "\input{childdoc.def}\childdocforward[cdocsamp]{cdocsch1}"|\\
% |latex -jobname cdocscl2 \|\\
% |  "\def\version{final}\input{childdoc.def}\childdocforward{cdocsch2}"|
% \end{tabular}
% \end{center}
% Note that the trailing backslash on each first line
% merely continues the input to the second line
% (for convenient cut ant paste).
% Furthermore, the command |latex| can be replaced by any
% of its alternative versions such as |pdflatex|.
%
% %%%%%%%%%%%%%%%%%%%%%%%%%%%%%%%%%%%%%%%%%%%%%%%%%%%%%%%%%%%%%%%%%%%%%%%%%%%%%%
% %%%%%%%%%%%%%%%%%%%%%%%%%%%%%%%%%%%%%%%%%%%%%%%%%%%%%%%%%%%%%%%%%%%%%%%%%%%%%%
% \section{Implementation}
%\iffalse
%<*package>
%\fi
%
% This section describes the definitions file |childdoc.def|.

% The definitions cannot be loaded using |\usepackage| or |\RequirePackage|
% which has a mechanism to prevent loading a style file more than once.
% When loading the definitions by means of |\input|
% multiple instances have to be prevented manually:
%\iffalse
%This code needs to be before the `\ProvidesFile' directive
%which is defined at the beginning of this file.
%Therefore it is also placed there and commented out here.
%</package>
%<*discard>
%\fi
%    \begin{macrocode}
\ifdefined\childdocmain\endinput\fi
%    \end{macrocode}
%\iffalse
%</discard>
%<*package>
%\fi
%
% \macro{\ifchilddoc}
% \macro{\ifchilddocmanual}
% The conditional |\ifchilddoc| tells whether a
% child (true) or main (false) document is being compiled.
% The conditional |\ifchilddocmanual| tells whether
% the |\includeonly| mechanism is used (false) or
% the selection of child files must be performed manually (true).
% The definitions initialise to false:
%    \begin{macrocode}
\newif\ifchilddoc
\newif\ifchilddocmanual
%    \end{macrocode}

% \macro{\childdocname}
% \macro{\childdocjob}
% The macro |\childdocname| stores the name of the main document
% to be compiled. The macro |\childdocjob| stores the name of
% the document on which the \LaTeX{} compiler was originally invoked.
% The content of |\jobname| cannot be compared
% to filenames specified in the source due to different catcodes.
% The following code rescans |\jobname|, stores the result
% in |\childdocname| and saves a copy in |\childdocjob|:
%    \begin{macrocode}
\edef\childdocname{\scantokens\expandafter{\jobname\noexpand}}
\let\childdocjob\childdocname
%    \end{macrocode}

% \macro{\childdocdisable}
% The macro |\childdocdisable| prevents the main file
% from being processed more than once.
% At this stage, the main document command |\childdocmain|
% is assumed to be called once again where it should do nothing.
% Any subsequent call to it should prevent
% a secondary processing of the main document
% It overwrites the forwarding commands
% |\childdocof| and |\childdocforward|
% with empty macros to prevent further inclusions of the main document:
%    \begin{macrocode}
\newcommand{\childdocdisable}
{
  \renewcommand{\childdocmain}[1]{\renewcommand{\childdocmain}[1]{\endinput}}
  \renewcommand{\childdocof}[1]{}
  \renewcommand{\childdocby}[2][]{}
  \renewcommand{\childdocforward}[2][]{}
  \renewcommand{\childdocdisable}{}
}
%    \end{macrocode}

% \macro{\childdocmain}
% The macro |\childdocmain| is to be called at the top of the main file
% with nothing or the main filename (without extension) as argument.
% First, it breaks loops.
% If the argument is not empty and does not match |\childdocname|
% (which is set by the first inclusion of |childdoc.def|),
% |\ifchilddoc| is set to true, |\includeonly| is applied to the child file
% and |\jobname| is set to the main file
% (for proper handling of |.aux| files):
%    \begin{macrocode}
\newcommand{\childdocmain}[1]
{
  \childdocdisable\childdocmain{}
  \if?#1?\else
    \begingroup
      \def\childdoctmp{#1}
      \ifx\childdoctmp\childdocname
        \def\childdoctmp{}
      \else
        \def\childdoctmp
        {
          \childdoctrue
          \includeonly{\childdocname}
          \def\childdocjob{#1}
          \def\jobname{#1}
        }
      \fi
      \expandafter
    \endgroup
    \childdoctmp
  \fi
}
%    \end{macrocode}

% \macro{\childdocof}
% The command |\childdocof| redirects
% compilation to the main file |#1|.
%    \begin{macrocode}
\newcommand{\childdocof}[1]
{
  \childdocdisable
  \childdoctrue
  \includeonly{\childdocname}
  \def\jobname{#1}
  \def\childdocjob{#1}
  \input{#1}
}
%    \end{macrocode}

% \macro{\childdocby}
% The command |\childdocby| ....
%    \begin{macrocode}
\newcommand{\childdocby}[2][]
{
  \childdocdisable
  \childdoctrue
  \childdocmanualtrue
  \if?#1?\else
    \def\jobname{#2}
  \fi
  \def\childdocjob{#2}
  \input{#2}
  \endinput
}
%    \end{macrocode}

% \macro{\childdocforward}
% The command |\childdocforward| redirects
% compilation to the main file or
% (if the optional argument is given) a child file.
% Parameters are set as if the main file
% or a child file starting with |\childdocof| was compiled.
% Then compilation is handed over to the main file:
%    \begin{macrocode}
\newcommand{\childdocforward}[2][]
{
  \begingroup
    \if?#1?
      \def\childdoctmp
      {
        \def\childdocname{#2}
        \def\childdocjob{#2}
        \def\jobname{#2}
        \input{#2}
        \endinput
      }
    \else
      \def\childdoctmp
      {
        \childdocdisable
        \def\childdocname{#2}
        \childdoctrue
        \includeonly{#2}
        \def\childdocjob{#1}
        \def\jobname{#1}
        \input{#1}
        \endinput
      }
    \fi
    \expandafter
  \endgroup
  \childdoctmp
}
%    \end{macrocode}

% \macro{\childdocforwardprefix}
% The command |\childdocforwardprefix| redirects
% compilation to the main or a child file by means of a pattern.
% The prefix |#1| in the current filename is replaced by |#2|
% and the suffix of the current filename is kept
% (it is assumed that the filename does not contain the substring `|~~~|'
% which is used as a delimiter).
% Compilation is handed over to the new file by |\childdocforward|:
%    \begin{macrocode}
\newcommand{\childdocforwardprefix}[3][]
{
  \begingroup
    \def\childdocextract #2##1~~~{\def\childdoctmp{\childdocforward[#1]{#3##1}}}
    \expandafter\childdocextract\childdocname~~~
    \expandafter
  \endgroup
  \childdoctmp
}
%    \end{macrocode}

% \macro{\childdoc}
% The deprecated macro |\childdoc| is a legacy version of |\childdocmain|:
%    \begin{macrocode}
\newcommand{\childdoc}{\childdocmain}
%    \end{macrocode}

% \macro{\childdocredirect}
% The deprecated macro |\childdocredirect| is a legacy version
% of |\childdocforward| and |\childdocforwardprefix|:
%    \begin{macrocode}
\newcommand{\childdocredirect}[2][]
{
  \begingroup
    \if?#1?
      \def\childdoctmp{\childdocforward{#2}}
    \else
      \def\childdoctmp{\childdocforwardprefix{#1}{#2}}
    \fi
    \expandafter
  \endgroup
  \childdoctmp
}
%    \end{macrocode}

%\iffalse
%</package>
%\fi
%
\endinput

\childdocof{cdocsamp}
%    \end{macrocode}

%\iffalse
%</samplechap1|samplechap2>
%\fi
%
%\iffalse
%<*samplechap1>
%\fi
% Some text for chapter 1:
%    \begin{macrocode}
\section{one}
some text in chapter one
%    \end{macrocode}

%\iffalse
%</samplechap1>
%\fi
% Some text for chapter 2:
%\iffalse
%<*samplechap2>
%\fi
%    \begin{macrocode}
\section{two}
more text in chapter two
%    \end{macrocode}

%\iffalse
%</samplechap2>
%\fi
%
% %%%%%%%%%%%%%%%%%%%%%%%%%%%%%%%%%%%%%%
% \paragraph{Part Include Files.}
%
% The include files are called |cdocspt3.tex| and |cdocspt4.tex|.
%
%\iffalse
%<*samplepart3|samplepart4>
%\fi

% Optional override for |\version| flag:
%    \begin{macrocode}
%%\providecommand{\version}{final}
%    \end{macrocode}

% Include the main document:
%    \begin{macrocode}
% \iffalse
%
% childdoc.dtx Copyright (C) 2017-2018 Niklas Beisert
%
% This work may be distributed and/or modified under the
% conditions of the LaTeX Project Public License, either version 1.3
% of this license or (at your option) any later version.
% The latest version of this license is in
%   http://www.latex-project.org/lppl.txt
% and version 1.3 or later is part of all distributions of LaTeX
% version 2005/12/01 or later.
%
% This work has the LPPL maintenance status `maintained'.
%
% The Current Maintainer of this work is Niklas Beisert.
%
% This work consists of the files childdoc.dtx and childdoc.ins
% and the derived files childdoc.def and cdocsamp.tex with
% cdocsch1.tex, cdocsch2.tex, cdocsdrf.tex, cdocsfn1.tex, cdocsfn2.tex.
%
%<package>\ifdefined\childdocmain\endinput\fi
%<package>\ProvidesFile{childdoc.def}[2018/12/30 v2.0 child document driver]
%<samplemain>\ProvidesFile{cdocsamp.tex}[2018/12/30 v2.0 sample for childdoc]
%<*driver>
%\ProvidesFile{childdoc.drv}[2018/12/30 v2.0 childdoc reference manual file]
\PassOptionsToClass{10pt,a4paper}{article}
\documentclass{ltxdoc}

\usepackage[margin=35mm]{geometry}
\usepackage{hyperref}
\usepackage{hyperxmp}
\usepackage[usenames]{color}

\hypersetup{colorlinks=true}
\hypersetup{pdfstartview=FitH}
\hypersetup{pdfpagemode=UseNone}
\hypersetup{pdfsource={}}
\hypersetup{pdflang={en-UK}}
\hypersetup{pdfcopyright={Copyright 2017-2018 Niklas Beisert.
  This work may be distributed and/or modified under the
  conditions of the LaTeX Project Public License, either version 1.3
  of this license or (at your option) any later version.}}
\hypersetup{pdflicenseurl={http://www.latex-project.org/lppl.txt}}
\hypersetup{pdfcontactaddress={ETH Zurich, ITP, HIT K,
  Wolfgang-Pauli-Strasse 27}}
\hypersetup{pdfcontactpostcode={8093}}
\hypersetup{pdfcontactcity={Zurich}}
\hypersetup{pdfcontactcountry={Switzerland}}
\hypersetup{pdfcontactemail={nbeisert@itp.phys.ethz.ch}}
\hypersetup{pdfcontacturl={http://people.phys.ethz.ch/\xmptilde nbeisert/}}

\newcommand{\secref}[1]{\hyperref[#1]{section \ref*{#1}}}

\parskip1ex
\parindent0pt
\let\olditemize\itemize
\def\itemize{\olditemize\parskip0pt}

\begin{document}

\title{The \textsf{childdoc} Package}
\hypersetup{pdftitle={The childdoc Package}}
\author{Niklas Beisert\\[2ex]
  Institut f\"ur Theoretische Physik\\
  Eidgen\"ossische Technische Hochschule Z\"urich\\
  Wolfgang-Pauli-Strasse 27, 8093 Z\"urich, Switzerland\\[1ex]
  \href{mailto:nbeisert@itp.phys.ethz.ch}
  {\texttt{nbeisert@itp.phys.ethz.ch}}}
\hypersetup{pdfauthor={Niklas Beisert}}
\hypersetup{pdfsubject={Manual for the LaTeX2e Package childdoc}}
\date{30 December 2018, \textsf{v2.0}}
\maketitle

\begin{abstract}\noindent
\textsf{childdoc} is a \LaTeXe{} package
that enables the direct compilation
of document sections included by |\include|
to individual files.
\end{abstract}

\begingroup
\parskip0ex
\tableofcontents
\endgroup

%%%%%%%%%%%%%%%%%%%%%%%%%%%%%%%%%%%%%%%%%%%%%%%%%%%%%%%%%%%%%%%%%%%%%%%%%%%%%%%%
%%%%%%%%%%%%%%%%%%%%%%%%%%%%%%%%%%%%%%%%%%%%%%%%%%%%%%%%%%%%%%%%%%%%%%%%%%%%%%%%
\section{Introduction}

\LaTeX{} provides a mechanism to structure a large document (such as a book)
into a main file and several child files (containing the chapters)
using the |\include| command.
This mechanism is beneficial for documents
which span hundreds of pages in order to
make the source file(s) more manageable.
Moreover, compilation can be restricted to
selected child files by means of the |\includeonly| command.
The latter feature can be used to reduce the compilation time while editing
(this was significantly more useful in the earlier days of \LaTeX{})
or to generate a smaller document which is easier to navigate.
Another application of |\includeonly| is to generate
documents consisting of selected parts of the complete document.

However, there are a few drawbacks of the plain |\include| mechanism:
\begin{itemize}
\item
The child files cannot be compiled on their own,
they can only be compiled via the main file.
A naive editing environment
(such as a text editor with an option
to have the current file processed by \LaTeX)
may require one to switch to the main file before compiling;
attempting to compile the child file produces errors.
\item
The main file must be modified (each time)
to adjust the |\includeonly| command
to the present needs. This easily leaves the main file in a messy state.
\item
The generated document will always carry the filename
of the main document. This is inconvenient if
several child files are to be compiled and
to be kept for distribution.
\end{itemize}

The present package provides a simple interface
to make child files individually compilable by \LaTeX{}.
Compiling a child file then has the same effect as compiling
the main file with an |\includeonly| command
to select the appropriate child.
Moreover the generated document will carry the name of the child
rather than the main file.
This resolves all three above issues.

This feature is meant to make the editing of books,
thesis documents and lecture notes somewhat more convenient.
However, the package can also be used efficiently for
composing a series of documents (such as exercise sheets)
which are typically distributed individually.
It then assists the author in generating the individual documents
(potentially in different versions)
as well as a document containing the collected series.
Another application is in developing style files
or other kinds of included material
where compilation of the style file could redirect
to a sample or test file.

%%%%%%%%%%%%%%%%%%%%%%%%%%%%%%%%%%%%%%%%%%%%%%%%%%%%%%%%%%%%%%%%%%%%%%%%%%%%%%%%
%%%%%%%%%%%%%%%%%%%%%%%%%%%%%%%%%%%%%%%%%%%%%%%%%%%%%%%%%%%%%%%%%%%%%%%%%%%%%%%%
\section{Usage}

First of all, the package \textsf{childdoc} is \emph{not} a standard
\LaTeXe{} |.sty| style file! Therefore it needs to be invoked in
a non-standard way.

%%%%%%%%%%%%%%%%%%%%%%%%%%%%%%%%%%%%%%%%%%%%%%%%%%%%%%%%%%%%%%%%%%%%%%%%%%%%%%%%
\subsection{Included Files}
\label{sec:include}

%%%%%%%%%%%%%%%%%%%%%%%%%%%%%%%%%%%%%%%%
\DescribeMacro{\childdocmain}
To use the package, add the commands
\begin{center}
\begin{tabular}{l}
|\input{childdoc.def}|\\
|\childdocmain{}|\\
\end{tabular}
\end{center}
at the very top of the main \LaTeX{} file,
in particular \emph{before} the |\documentclass| statement!
The argument of |\childdocmain| should be left empty
(but it must be present).

%%%%%%%%%%%%%%%%%%%%%%%%%%%%%%%%%%%%%%%%
\DescribeMacro{\childdocof}
Furthermore, add the commands
\begin{center}
\begin{tabular}{l}
|\input{childdoc.def}|\\
|\childdocof{|\textit{main}|}|\\
\end{tabular}
\end{center}
at the top of every child file \textit{child}
which is included by |\include{|\textit{child}|}|
from within the main file
(or at least for those files to be compiled individually).
The argument \textit{main} must be the filename of the main file.

There are a couple of
considerations in setting up the main and child documents:

%%%%%%%%%%%%%%%%%%%%%%%%%%%%%%%%%%%%%%%%
\paragraph{Restrictions.}

Please note the following restrictions:
\begin{itemize}
\item
|\childdocmain| must be called with one argument \textit{main}
to ensure compatibility with earlier version of the package.
It must either be empty (|\childdocmain{}|)
or precisely match the filename of the main file in which it is specified.
See \secref{sec:detection} for further information.
\item
The filename \textit{main} must be specified without the |.tex| extension.
\item
The filename \textit{main} is case sensitive
(even in case-insensitive file systems)
due to internal string comparison.
\item
The argument \textit{main} should be fully expanded, it cannot be a macro.
\item
Subdirectories and special characters should be avoided in filenames.
\item
The command |\childdocmain{|\textit{main}|}| must be followed by a whitespace.
It should not be followed immediately by another command
or by a comment mark `|%|'.
This is because the \TeX{} parser reads the token immediately following
the argument of |\childdocmain| and puts it
at the beginning of every child section;
however, a white\-space is ignored.
\end{itemize}

%%%%%%%%%%%%%%%%%%%%%%%%%%%%%%%%%%%%%%%%
\paragraph{Content of Main File.}

It is advisable to place all content in the child files included by |\include|.
Any output contained in the main file will appear in all child documents
unless suppressed manually;
it cannot be suppressed automatically by the |\includeonly| directive
and thus should normally be avoided.
A method to include some content in the main file
by means of conditional processing is described in \secref{sec:conditional}.

%%%%%%%%%%%%%%%%%%%%%%%%%%%%%%%%%%%%%%%%
\paragraph{Page Numbering.}

When only a part of the document is compiled,
the appropriate numbering of pages
(as well as other status parameters)
is determined from the |.aux| files.
The latter contain information from previous passes.
However this information needs to propagate through
all intermediate child documents.
Therefore the page numbering in child documents may well
be inconsistent until the complete document is compiled at least once.

A useful (if unconventional) way to always ensure a consistent
page numbering is to restart the numbering in each child document
and denote the pages by `\textit{child}|.|\textit{page}'
where \textit{child} represents the chapter/section number of the child file.
This can be achieved by the command
|\numberwithin{page}{|\textit{child}|}|
of the \textsf{amsmath} package
where \textit{child} can be |chapter| or |section|
depending on the chosen structuring.
Alternatively, one can modify the macro |\thepage| appropriately
and reset the counter |page| at the start of each child file.

%%%%%%%%%%%%%%%%%%%%%%%%%%%%%%%%%%%%%%%%%%%%%%%%%%%%%%%%%%%%%%%%%%%%%%%%%%%%%%%%
\subsection{Conditional Processing}
\label{sec:conditional}

The package provides a mechanism to compile different versions
of a document. To customise the versions further some conditional processing
can come in handy to distinguish which version is being compiled.
The package provides two macros to describe the compilation context:

%%%%%%%%%%%%%%%%%%%%%%%%%%%%%%%%%%%%%%%%
\DescribeMacro{\ifchilddoc}
The conditional |\ifchilddoc| distinguishes between the compilation of
child documents and the main document:
%
\begin{center}
|\ifchilddoc |\textit{child-code}| |[|\||else |\textit{main-code}]| \||fi|
\end{center}

%%%%%%%%%%%%%%%%%%%%%%%%%%%%%%%%%%%%%%%%
\DescribeMacro{\childdocname}
\DescribeMacro{\childdocjob}
The macro |\childdocname| contains the filename (without extension)
of the main or child file being processed.
Note that |\childdocjob| will always contain the name of the main file.

%%%%%%%%%%%%%%%%%%%%%%%%%%%%%%%%%%%%%%%%
\paragraph{Title Page.}

Conditional processing can be used to include a title or banner page
in the main document when proper precautions are taken.
Importantly, the code in the main file should ensure that the page counter
(as well as other status parameters which are stored in the |.aux| files)
takes the same value after the conditional processing.
Otherwise the page numbers may take divergent values
depending on which part is compiled.

For example, a title page could be declared by:
%
\begin{center}
\begin{tabular}{l}
|\ifchilddoc\||else|\\
|\addtocounter{page}{-1}|\\
\textit{code for title page}\\
|\newpage|\\
|\||fi|
\end{tabular}
\end{center}
%
A banner page for the child documents can be generated by:
%
\begin{center}
\begin{tabular}{l}
|\ifchilddoc|\\
|\addtocounter{page}{-1}|\\
\textit{code for banner page}\\
|\newpage|\\
|\||fi|
\end{tabular}
\end{center}
%
Here one could write a message such as:
\begin{center}
|This is the part \childdocname{} of \childdocjob{}.|
\end{center}

%%%%%%%%%%%%%%%%%%%%%%%%%%%%%%%%%%%%%%%%%%%%%%%%%%%%%%%%%%%%%%%%%%%%%%%%%%%%%%%%
\subsection{Flags}
\label{sec:flags}

The package makes it easy to generate different versions
of the main or child documents.
To this end compilation flags can be defined
and assigned different default values.
They will be particularly useful in conjunction
with the forwarding mechanism described in \secref{sec:forward}.

For example, it may be useful to have a flag |\version|
which can be set to |draft| or |final|.
The document source will contain some conditional code
depending on the value of |\version|.
Suppose further, the flag should default to |final| for the main file
and to |draft| for child files
which is a natural assignment for editing the document.
This is achieved by placing the following code
in the preamble of the main document
(below the |\childdocmain| directive):
%
\begin{center}
\begin{tabular}{l}
|\ifchilddoc|\\
|\providecommand{\version}{draft}|\\
|\||else|\\
|\providecommand{\version}{final}|\\
|\||fi|
\end{tabular}
\end{center}
%
The definition by |\providecommand| makes sure
that previous definitions are not overwritten.
Further statements |\providecommand{\version}{...}|
can thus be added before the above code to override it.

For the main file, one might add a line
(between |\childdocmain| and the above block)
%
\begin{center}
|%\ifchilddoc\||else\providecommand{\version}{draft}\||fi|
\end{center}
%
which can be uncommented to produce a draft version.
Likewise one can add a line to the very top of a child file
(above the |\childdocof{|\textit{main}|}| directive)
%
\begin{center}
|%\providecommand{\version}{final}|
\end{center}
%
which can be uncommented to produce the final version of this child document.

%%%%%%%%%%%%%%%%%%%%%%%%%%%%%%%%%%%%%%%%%%%%%%%%%%%%%%%%%%%%%%%%%%%%%%%%%%%%%%%%
\subsection{Forwarding}
\label{sec:forward}

Different versions of the main or child documents
using compilation flags as described in \secref{sec:flags}
can be (permanently) stored in different files
for convenient compilation, viewing and distribution.
To this end, the package defines a command
to pass on compilation to a different file:

%%%%%%%%%%%%%%%%%%%%%%%%%%%%%%%%%%%%%%%%
\DescribeMacro{\childdocforward}
The command |\childdocforward| redirects processing to
another source file:
%
\begin{center}
\begin{tabular}{l}
|\input{childdoc.def}|\\
|\childdocforward[|\textit{main}|]{|\textit{dest}|}|\\
\end{tabular}
\end{center}
%
The argument \textit{dest} is the destination file
(without extension).
It should be the main file or one of the child files.
Note that further \textsf{childdoc} directives
such as |\childdocof| and |\childdocforward|
in the indicated file will be processed in this form.
The optional argument \textit{main}
passes on directly to the main file \textit{main}
while pretending to compile the child \textit{dest}.
This form behaves as if \textit{dest}
issues |\childdocof{|\textit{main}|}| right away,
and no further \textsf{childdoc} directives will be processed.

%%%%%%%%%%%%%%%%%%%%%%%%%%%%%%%%%%%%%%%%
\DescribeMacro{\...prefix}
In the alternative form |\childdocforwardprefix|,
%
\begin{center}
\begin{tabular}{l}
|\input{childdoc.def}|\\
|\childdocforwardprefix[|\textit{main}|]{|\textit{prefix}|}{|\textit{dest}|}|
\end{tabular}
\end{center}
%
the destination file is determined by a pattern
depending on the current file:
To make this work, the current file must be called
`{\textit{prefix}\hspace{0.2em}\textit{suffix}}'
with \textit{prefix} matching precisely the argument.
Processing is then passed on to the file
`{\textit{dest}\hspace{0.2em}\textit{suffix}}'.
Surely, the same effect is achieved by
directly specifying the
argument `{\textit{dest}\hspace{0.2em}\textit{suffix}}'
in the first form.
However, that requires to set up a different file
for each child. With the alternative form of the command
all these files can have exactly the same content
which simplifies setting them up and maintaining them.

For example, the following file |draft.tex|
with a compilation flag |\version| as described in \secref{sec:flags}
compiles the main document as a draft:
%
\begin{center}
\begin{tabular}{l}
|\def\version{draft}|\\
|\input{childdoc.def}|\\
|\childdocforward{|\textit{main}|}|
\end{tabular}
\end{center}
%
Likewise, the following files |final|\textit{nn}|.tex|
compile the final version of the child document
|child|\textit{nn}|.tex|:
%
\begin{center}
\begin{tabular}{l}
|\def\version{final}|\\
|\input{childdoc.def}|\\
|\childdocforwardprefix{final}{child}|
\end{tabular}
\end{center}
%

Note that when several versions of a main file and/or of each child file
are to be generated, it may be convenient to set up a |Makefile| or
shell script to automatise the process.

%%%%%%%%%%%%%%%%%%%%%%%%%%%%%%%%%%%%%%%%%%%%%%%%%%%%%%%%%%%%%%%%%%%%%%%%%%%%%%%%
\subsection{Command Line Processing}
\label{sec:commandline}

The effect of redirection files can also be achieved by invoking
the \LaTeX{} compiler with a more elaborate command line.
Most conveniently this should be done as part
of a shell script or a |Makefile|.

When using \textsf{childdoc} in the main file, the following
command lines effectively perform a redirection
(note that depending on the shell being used,
backslashes may have to be doubled: `|\|' $\to$ `|\\|'):
%
\begin{center}
|... -jobname "|\textit{target}|" |\\|"|[\textit{flags}]%
|\input{childdoc.def}\childdocforward[|\textit{main}|]{|\textit{dest}|}"|
\end{center}
%
Here \textit{target} is the name of the output file,
\textit{main} is the name of the main file
and \textit{dest} is the name of the main or child file to be processed
(all filenames without extensions).
The optional argument \textit{main} can be omitted
if \textit{main} matches \textit{dest}.
Optionally, compilation \textit{flags} can be defined via |\def| commands.
This command line makes the \TeX{} engine believe
it is compiling the file \textit{target}
whose content is specified as the latter parameter.
The provided code then forwards the processing to
\textit{main} or \textit{dest} as described in \secref{sec:forward}.

%%%%%%%%%%%%%%%%%%%%%%%%%%%%%%%%%%%%%%%%%%%%%%%%%%%%%%%%%%%%%%%%%%%%%%%%%%%%%%%%
\subsection{Include by Input}
\label{sec:input}

Including child documents by |\include| has some restrictions by design.
Most notably, the content of a child document always occupies
its own set of pages; pages cannot be shared between child documents.
Usually, this behaviour makes perfect sense
because each child document contain an essential part of the document.
However, in some situations it may be desirable to compose
a document from a collection of parts
without having mandatory page breaks between then.
For this case, the package
provides a mechanism to include parts
by |\input| which can also be processed individually.
However, by construction this mechanism
requires manual handling of the content to be output.

%%%%%%%%%%%%%%%%%%%%%%%%%%%%%%%%%%%%%%%%
\DescribeMacro{\ifchilddocmanual}
The main file should be prepared as usual, see \secref{sec:include}.
However, the document body must make a distinction
between processing of an individual part and of the main document, e.g.:
%
\begin{center}
\begin{tabular}{l}
|\ifchilddocmanual|\\
|\input{\childdocname}|\\
|\||else|\\
\textit{document body with }|\input{|\textit{part}|}|\\
|\||fi|
\end{tabular}
\end{center}
%
The conditional |\ifchilddocmanual| is true whenever
a part to be included by |\input| is being compiled,
and the name of the part is stored in |\childdocname|.

%%%%%%%%%%%%%%%%%%%%%%%%%%%%%%%%%%%%%%%%
\DescribeMacro{\childdocby}
Each part to be included by |\input| should start with:
%
\begin{center}
\begin{tabular}{l}
|\input{childdoc.def}|\\
|\childdocby{|\textit{main}|}|\\
\end{tabular}
\end{center}
%
The directive |\childdocby| is similar to |\childdocof|
described in \secref{sec:include},
but the subsequent selection of content must be done manually.
To that end, both |\ifchilddoc| and |\ifchilddocmanual|
will be true upon processing of a part,
and the name of the part is stored in |\childdocname|.
Note that |\jobname| will be set to the filename of the current part
so that each part receives an individual |.aux| file
that does not interfere with the |.aux| file(s) of the main document.
This behaviour can be altered by the alternative form
|\childdocby[*]{|\textit{main}|}| (with a non-empty optional argument)
which uses the |.aux| file of the main document
by setting |\jobname| to \textit{main}.

%%%%%%%%%%%%%%%%%%%%%%%%%%%%%%%%%%%%%%%%%%%%%%%%%%%%%%%%%%%%%%%%%%%%%%%%%%%%%%%%
\subsection{Driver Development}
\label{sec:driver}

The \textsf{childdoc} mechanism can also be use for the development
of definition files such as \LaTeX{} styles or classes.
This case differs from the above setup with multiple parts
included by |\include| in that no |\includeonly| should be invoked.
This can be achieved by starting the include file
(before |\ProvidesPackage|) with:
%
\begin{center}
\begin{tabular}{l}
|\input{childdoc.def}|\\
|\childdocforward{|\textit{main}|}|\\
\end{tabular}
\end{center}
%
or alternatively with:
%
\begin{center}
\begin{tabular}{l}
|\input{childdoc.def}|\\
|\childdocby{|\textit{main}|}|\\
\end{tabular}
\end{center}
%
Both forms have slightly different effects as described above.
The main file is prepared as usual, see \secref{sec:include}.

%%%%%%%%%%%%%%%%%%%%%%%%%%%%%%%%%%%%%%%%%%%%%%%%%%%%%%%%%%%%%%%%%%%%%%%%%%%%%%%%
\subsection{Legacy Detection}
\label{sec:detection}

The directive |\childdocmain| in the main file can detect
whether the complete document or merely a child is to be compiled
even without using the directive |\childdocof|.
This method is deprecated because it is less robust
and there is no compelling reason to use it;
it is merely provided for backward compatibility
and it may be removed in future versions.

If the detection mechanism is to be used,
it is mandatory to correctly specify
the filename of the main file as the argument of |\childdocmain|:
%
\begin{center}
\begin{tabular}{l}
|\input{childdoc.def}|\\
|\childdocmain{|\textit{main}|}|\\
\end{tabular}
\end{center}
%
If |\jobname| does not match the argument \textit{main} of |\childdocmain|,
it is assumed that |\jobname| points to the child file to be compiled.
When using |\childdocmain| with the main file specified as argument,
it suffices to start a child file
with just |\input{|\textit{main}|}|
without loading of the package and using |\childdocof|.
If instead all processing is done
with the appropriate \textsf{childdoc} directives,
the argument of \textit{main} of |\childdocmain| can be empty.

An alternative version of the command line processing described
in \secref{sec:commandline} using the detection mechanism reads:
%
\begin{center}
|... -jobname "|\textit{target}|" "|[\textit{flags}]%
[|\def\jobname{|\textit{dest}|}|]|\input{|\textit{main}|}"|
\end{center}

%%%%%%%%%%%%%%%%%%%%%%%%%%%%%%%%%%%%%%%%%%%%%%%%%%%%%%%%%%%%%%%%%%%%%%%%%%%%%%%%
\subsection{Manual Code}
\label{sec:manual}

In case one cannot be certain whether the definitions file |childdoc.def|
is installed on the target \TeX{} distribution
and one prefers not to ship it,
it is conceivable to paste a few relevant commands into the sources.

To that end, drop all statements |\input{childdoc.def}|
and perform the replacements as outlined below.
Instead of |\childdocmain{|\textit{main}|}| add the following code
to the top of the main file:
%
\begin{center}
\begin{tabular}{l}
|\||ifdefined\childdocname\endinput\||fi\newif\ifchilddoc|\\
|\edef\childdocname{\scantokens\expandafter{\jobname\noexpand}}|\\
|\def\childdocmain{|\textit{main}|}\||ifx\childdocmain\childdocname\||else|\\
|\childdoctrue\includeonly{\childdocname}\let\jobname\childdocmain\||fi|\\
\end{tabular}
\end{center}
%
Instead of |\childdocof{|\textit{main}|}| just include the main file
at the top of each child file:
%
\begin{center}
|\input{|\textit{main}|}|
\end{center}
%
A simple redirection |\childdocforward{|\textit{dest}|}| is achieved by:
%
\begin{center}
|\def\jobname{|\textit{dest}|}\input{\jobname}|
\end{center}
%
The redirection with prefix
|\childdocforwardprefix[|\textit{prefix}|]{|\textit{dest}|}|
is accomplished by:
%
\begin{center}
\begin{tabular}{l}
|{\edef\jobname{\scantokens\expandafter{\jobname\noexpand}}|\\
|\def\redirectjob |\textit{prefix}|#1~~~{\gdef\jobname{|\textit{dest}|#1}}|\\
|\expandafter\redirectjob\jobname~~~}\input{\jobname}|
\end{tabular}
\end{center}

In an alternative approach,
child documents can be compiled by a specific command line
without additional code or specific definitions:
%
\begin{center}
|... -jobname "|\textit{target}|" "|[\textit{flags}]%
|\includeonly{|\textit{dest}|}\input{|\textit{main}|}"|
\end{center}
%

%%%%%%%%%%%%%%%%%%%%%%%%%%%%%%%%%%%%%%%%%%%%%%%%%%%%%%%%%%%%%%%%%%%%%%%%%%%%%%%%
%%%%%%%%%%%%%%%%%%%%%%%%%%%%%%%%%%%%%%%%%%%%%%%%%%%%%%%%%%%%%%%%%%%%%%%%%%%%%%%%
\section{Information}

%%%%%%%%%%%%%%%%%%%%%%%%%%%%%%%%%%%%%%%%%%%%%%%%%%%%%%%%%%%%%%%%%%%%%%%%%%%%%%%%
\subsection{Copyright}

Copyright \copyright{} 2017--2018 Niklas Beisert

This work may be distributed and/or modified under the
conditions of the \LaTeX{} Project Public License, either version 1.3
of this license or (at your option) any later version.
The latest version of this license is in
  \url{http://www.latex-project.org/lppl.txt}
and version 1.3 or later is part of all distributions of \LaTeX{}
version 2005/12/01 or later.

This work has the LPPL maintenance status `maintained'.

The Current Maintainer of this work is Niklas Beisert.

This work consists of the files |README.txt|, |childdoc.ins| and |childdoc.dtx|
as well as the derived files |childdoc.def|, |cdocsamp.tex|
with |cdocsch1.tex|, |cdocsch2.tex|, |cdocspt3.tex|, |cdocspt4.tex|,
|cdocsdrf.tex|, |cdocsfn1.tex|, |cdocsfn2.tex|
as well as |childdoc.pdf|.

%%%%%%%%%%%%%%%%%%%%%%%%%%%%%%%%%%%%%%%%%%%%%%%%%%%%%%%%%%%%%%%%%%%%%%%%%%%%%%%%
\subsection{Files and Installation}

The package consists of the files:
%
\begin{center}
\begin{tabular}{ll}
    |README.txt|   & readme file \\
    |childdoc.ins| & installation file \\
    |childdoc.dtx| & source file \\
    |childdoc.def| & definition file \\
    |cdocsamp.tex| & sample main file \\
    |cdocsch1.tex| & sample include file \\
    |cdocsch2.tex| & sample include file \\
    |cdocspt3.tex| & sample part file \\
    |cdocspt4.tex| & sample part file \\
    |cdocsdrf.tex| & sample redirection file \\
    |cdocsfn1.tex| & sample redirection file \\
    |cdocsfn2.tex| & sample redirection file \\
    |childdoc.pdf| & manual
\end{tabular}
\end{center}
%
The distribution consists of the files
|README.txt|, |childdoc.ins| and |childdoc.dtx|.
%
\begin{itemize}
\item
Run (pdf)\LaTeX{} on |childdoc.dtx|
to compile the manual |childdoc.pdf| (this file).
\item
Run \LaTeX{} on |childdoc.ins| to create the definitions file |childdoc.def|
and the sample |cdocsamp.tex| with include files
|cdocsch1.tex|, |cdocsch2.tex|, |cdocspt3.tex|, |cdocspt4.tex|,
|cdocsdrf.tex|, |cdocsfn1.tex|, |cdocsfn2.tex|.
Then copy the file |childdoc.def| to an appropriate directory of your \LaTeX{}
distribution, e.g.\ \textit{texmf-root}|/tex/latex/childdoc|.
\end{itemize}

%%%%%%%%%%%%%%%%%%%%%%%%%%%%%%%%%%%%%%%%%%%%%%%%%%%%%%%%%%%%%%%%%%%%%%%%%%%%%%%%
\subsection{Related CTAN Packages}

There are several other packages which offer a similar functionality:
%
\begin{itemize}
\item
The packages
\href{http://ctan.org/pkg/docmute}{\textsf{docmute}},
\href{http://ctan.org/pkg/includex}{\textsf{includex}} and
\href{http://ctan.org/pkg/standalone}{\textsf{standalone}}
provide commands to include only the document body of
a child file thus allowing both files to be compiled individually.
\item
The packages \href{http://ctan.org/pkg/subdocs}{\textsf{subdocs}}
and \href{http://ctan.org/pkg/subfiles}{\textsf{subfiles}}
provide structures in which the main and child documents can be
encapsulated and allowing them to be compiled individually.
The inclusion mechanism is different from the conventional |\include|.
\item
The package \href{http://ctan.org/pkg/combine}{\textsf{combine}}
is an elaborate solution to combine several documents into one.
\end{itemize}
%
See also the CTAN topic \href{http://ctan.org/topic/subdocs}{\textsf{subdocs}}
for further related packages.
The present package differs from the above solutions in that
a document structure constructed with the conventional |\include| mechanism
just needs two extra commands at the top of every file
such that all constituent files can be compiled individually.

%%%%%%%%%%%%%%%%%%%%%%%%%%%%%%%%%%%%%%%%%%%%%%%%%%%%%%%%%%%%%%%%%%%%%%%%%%%%%%%%
%\subsection{Feature Suggestions}
%
%The following is a list of features which may be useful for future
%versions of this package:
%%
%\begin{itemize}
%\item
%\ldots
%\end{itemize}

%%%%%%%%%%%%%%%%%%%%%%%%%%%%%%%%%%%%%%%%%%%%%%%%%%%%%%%%%%%%%%%%%%%%%%%%%%%%%%%%
\subsection{Revision History}

%%%%%%%%%%%%%%%%%%%%%%%%%%%%%%%%%%%%%%%%
\paragraph{v2.0:} 2018/12/30

\begin{itemize}
\item
immediate forward processing
\item
added |\childdocby| mechanism
\item
manual restructured
\end{itemize}

%%%%%%%%%%%%%%%%%%%%%%%%%%%%%%%%%%%%%%%%
\paragraph{v1.6:} 2018/01/17

\begin{itemize}
\item
application for development of include files
\item
corrections to manual
\end{itemize}

%%%%%%%%%%%%%%%%%%%%%%%%%%%%%%%%%%%%%%%%
\paragraph{v1.5:} 2017/05/21

\begin{itemize}
\item
more complete structuring introduced
\item
|\childdocof| introduced
\item
|\childdoc| renamed to |\childdocmain|
\item
|\childredirect| renamed to |\childdocforward| and |\childdocforwardprefix|
and functionality expanded
\end{itemize}

%%%%%%%%%%%%%%%%%%%%%%%%%%%%%%%%%%%%%%%%
\paragraph{v1.0:} 2017/04/27

\begin{itemize}
\item
manual and install package
\item
first version published on CTAN
\end{itemize}

%%%%%%%%%%%%%%%%%%%%%%%%%%%%%%%%%%%%%%%%
\paragraph{v0.6:} 2017/04/26

\begin{itemize}
\item
redirection mechanism added
\end{itemize}

%%%%%%%%%%%%%%%%%%%%%%%%%%%%%%%%%%%%%%%%
\paragraph{v0.5:} 2017/04/26

\begin{itemize}
\item
functionality in definition file
\end{itemize}


%%%%%%%%%%%%%%%%%%%%%%%%%%%%%%%%%%%%%%%%%%%%%%%%%%%%%%%%%%%%%%%%%%%%%%%%%%%%%%%%
%%%%%%%%%%%%%%%%%%%%%%%%%%%%%%%%%%%%%%%%%%%%%%%%%%%%%%%%%%%%%%%%%%%%%%%%%%%%%%%%
%%%%%%%%%%%%%%%%%%%%%%%%%%%%%%%%%%%%%%%%%%%%%%%%%%%%%%%%%%%%%%%%%%%%%%%%%%%%%%%%
\appendix

\settowidth\MacroIndent{\rmfamily\scriptsize 000\ }

 \DocInput{childdoc.dtx}

\end{document}
%</driver>
% \fi
%
% %%%%%%%%%%%%%%%%%%%%%%%%%%%%%%%%%%%%%%%%%%%%%%%%%%%%%%%%%%%%%%%%%%%%%%%%%%%%%%
% %%%%%%%%%%%%%%%%%%%%%%%%%%%%%%%%%%%%%%%%%%%%%%%%%%%%%%%%%%%%%%%%%%%%%%%%%%%%%%
% \section{Sample}
%\iffalse
%<*samplemain>
%\fi
%
% The following presents a sample document
% with two chapters, two parts, a title page,
% a compile flag as well as three forwarding files to set the flag.
% It consists of eight |.tex| files:
% \begin{center}
% \begin{tabular}{ll}
% |cdocsamp.tex|&main file\\
% |cdocsch1.tex|&include file for chapter 1\\
% |cdocsch2.tex|&include file for chapter 2\\
% |cdocspt3.tex|&include file for part 3\\
% |cdocspt4.tex|&include file for part 4\\
% |cdocsdrf.tex|&forwarding file for main file in draft mode\\
% |cdocsfi1.tex|&forwarding file for final version of chapter 1\\
% |cdocsfi2.tex|&forwarding file for final version of chapter 2\\
% \end{tabular}
% \end{center}
% Each of the eight files can be compiled directly by the \LaTeX{} compiler.
%
% %%%%%%%%%%%%%%%%%%%%%%%%%%%%%%%%%%%%%%
% \paragraph{Main File.}
%
% The main file is called |cdocsamp.tex|.
%
% Load the \textsf{childdoc} definitions and
% declare the filename for the main document:
%    \begin{macrocode}
\input{childdoc.def}
\childdocmain{}
%    \end{macrocode}

% Optional override for |\version| flag:
%    \begin{macrocode}
%%\ifchilddoc\else\providecommand{\version}{draft}\fi
%    \end{macrocode}

% Define the default values for the |\version| flag
% (|final| for the main file and |draft| for childs):
%    \begin{macrocode}
\ifchilddoc
\providecommand{\version}{draft}
\else
\providecommand{\version}{final}
\fi
%    \end{macrocode}

% Load the standard document class:
%    \begin{macrocode}
\documentclass[12pt]{article}
%    \end{macrocode}

% Start the document body:
%    \begin{macrocode}
\begin{document}
%    \end{macrocode}

% Declare a title page.
% Print title, part of document being processed and version flag:
%    \begin{macrocode}
\addtocounter{page}{-1}
\begin{center}
{\LARGE\bfseries{}childdoc example\par}
\vspace{1cm}
\ifchilddoc
\ifchilddocmanual part\else chapter\fi:
`\childdocname' of `\childdocjob'\par
\else
main document: `\childdocjob'\par
\fi
version: \version\par
\end{center}
\newpage
%    \end{macrocode}

% Manually include selected file,
% otherwise process as usual:
%    \begin{macrocode}
\ifchilddocmanual
\section*{part `\childdocname'}
\input{\childdocname}
\else
%    \end{macrocode}

% Include the two chapters:
%    \begin{macrocode}
\include{cdocsch1}
\include{cdocsch2}
%    \end{macrocode}

% Include the two parts unless only chapters should be displayed:
%    \begin{macrocode}
\ifchilddoc\else
\section{part three}
\input{cdocspt3}
\section{part four}
\input{cdocspt4}
\fi
%    \end{macrocode}

% Process as usual until here:
%    \begin{macrocode}
\fi
%    \end{macrocode}

% End of document body:
%    \begin{macrocode}
\end{document}
%    \end{macrocode}
%\iffalse
%</samplemain>
%\fi
%
% %%%%%%%%%%%%%%%%%%%%%%%%%%%%%%%%%%%%%%
% \paragraph{Chapter Include Files.}
%
% The include files are called |cdocsch1.tex| and |cdocsch2.tex|.
%
%\iffalse
%<*samplechap1|samplechap2>
%\fi

% Optional override for |\version| flag:
%    \begin{macrocode}
%%\providecommand{\version}{final}
%    \end{macrocode}

% Include the main document:
%    \begin{macrocode}
\input{childdoc.def}
\childdocof{cdocsamp}
%    \end{macrocode}

%\iffalse
%</samplechap1|samplechap2>
%\fi
%
%\iffalse
%<*samplechap1>
%\fi
% Some text for chapter 1:
%    \begin{macrocode}
\section{one}
some text in chapter one
%    \end{macrocode}

%\iffalse
%</samplechap1>
%\fi
% Some text for chapter 2:
%\iffalse
%<*samplechap2>
%\fi
%    \begin{macrocode}
\section{two}
more text in chapter two
%    \end{macrocode}

%\iffalse
%</samplechap2>
%\fi
%
% %%%%%%%%%%%%%%%%%%%%%%%%%%%%%%%%%%%%%%
% \paragraph{Part Include Files.}
%
% The include files are called |cdocspt3.tex| and |cdocspt4.tex|.
%
%\iffalse
%<*samplepart3|samplepart4>
%\fi

% Optional override for |\version| flag:
%    \begin{macrocode}
%%\providecommand{\version}{final}
%    \end{macrocode}

% Include the main document:
%    \begin{macrocode}
\input{childdoc.def}
\childdocby{cdocsamp}
%    \end{macrocode}

%\iffalse
%</samplepart3|samplepart4>
%\fi
%
%\iffalse
%<*samplepart3>
%\fi
% Some text for part 3:
%    \begin{macrocode}
some text in part three
%    \end{macrocode}

%\iffalse
%</samplepart3>
%\fi
% Some text for part 4:
%\iffalse
%<*samplepart4>
%\fi
%    \begin{macrocode}
more text in part four
%    \end{macrocode}

%\iffalse
%</samplepart4>
%\fi
%
% %%%%%%%%%%%%%%%%%%%%%%%%%%%%%%%%%%%%%%
% \paragraph{Forwarding for a Complete Draft.}
%
% The following forwarding file |cdocsdrf.tex|
% compiles the main document in draft mode:
%\iffalse
%<*sampledraft>
%\fi
%    \begin{macrocode}
\def\version{draft}
\input{childdoc.def}
\childdocforward{cdocsamp}
%    \end{macrocode}

%\iffalse
%</sampledraft>
%\fi
%
% %%%%%%%%%%%%%%%%%%%%%%%%%%%%%%%%%%%%%%
% \paragraph{Forwarding for Final Version of the Chapters.}
%
% The following forwarding files |cdocsfn1.tex| and |cdocsfn2.tex|
% (with identical content)
% compile the final versions of the child documents
% |cdocsch1.tex| and |cdocsch2.tex|, respectively:
%\iffalse
%<*samplefinal>
%\fi
%    \begin{macrocode}
\def\version{final}
\input{childdoc.def}
\childdocforwardprefix[cdocsamp]{cdocsfn}{cdocsch}
%    \end{macrocode}

%\iffalse
%</samplefinal>
%\fi
%
% %%%%%%%%%%%%%%%%%%%%%%%%%%%%%%%%%%%%%%
% \paragraph{Command Line Processing.}
%
% The following three command lines generate the output files
% |cdocscld|, |cdocscl1| and |cdocscl2|
% which should be identical to
% |cdocsdrf|, |cdocsch1| and |cdocsfn2|, respectively:
% \begin{center}
% \begin{tabular}{l}
% |latex -jobname cdocscld \|\\
% |  "\def\version{draft}\input{childdoc.def}\childdocforward{cdocsamp}"|\\
% |latex -jobname cdocscl1 \|\\
% |  "\input{childdoc.def}\childdocforward[cdocsamp]{cdocsch1}"|\\
% |latex -jobname cdocscl2 \|\\
% |  "\def\version{final}\input{childdoc.def}\childdocforward{cdocsch2}"|
% \end{tabular}
% \end{center}
% Note that the trailing backslash on each first line
% merely continues the input to the second line
% (for convenient cut ant paste).
% Furthermore, the command |latex| can be replaced by any
% of its alternative versions such as |pdflatex|.
%
% %%%%%%%%%%%%%%%%%%%%%%%%%%%%%%%%%%%%%%%%%%%%%%%%%%%%%%%%%%%%%%%%%%%%%%%%%%%%%%
% %%%%%%%%%%%%%%%%%%%%%%%%%%%%%%%%%%%%%%%%%%%%%%%%%%%%%%%%%%%%%%%%%%%%%%%%%%%%%%
% \section{Implementation}
%\iffalse
%<*package>
%\fi
%
% This section describes the definitions file |childdoc.def|.

% The definitions cannot be loaded using |\usepackage| or |\RequirePackage|
% which has a mechanism to prevent loading a style file more than once.
% When loading the definitions by means of |\input|
% multiple instances have to be prevented manually:
%\iffalse
%This code needs to be before the `\ProvidesFile' directive
%which is defined at the beginning of this file.
%Therefore it is also placed there and commented out here.
%</package>
%<*discard>
%\fi
%    \begin{macrocode}
\ifdefined\childdocmain\endinput\fi
%    \end{macrocode}
%\iffalse
%</discard>
%<*package>
%\fi
%
% \macro{\ifchilddoc}
% \macro{\ifchilddocmanual}
% The conditional |\ifchilddoc| tells whether a
% child (true) or main (false) document is being compiled.
% The conditional |\ifchilddocmanual| tells whether
% the |\includeonly| mechanism is used (false) or
% the selection of child files must be performed manually (true).
% The definitions initialise to false:
%    \begin{macrocode}
\newif\ifchilddoc
\newif\ifchilddocmanual
%    \end{macrocode}

% \macro{\childdocname}
% \macro{\childdocjob}
% The macro |\childdocname| stores the name of the main document
% to be compiled. The macro |\childdocjob| stores the name of
% the document on which the \LaTeX{} compiler was originally invoked.
% The content of |\jobname| cannot be compared
% to filenames specified in the source due to different catcodes.
% The following code rescans |\jobname|, stores the result
% in |\childdocname| and saves a copy in |\childdocjob|:
%    \begin{macrocode}
\edef\childdocname{\scantokens\expandafter{\jobname\noexpand}}
\let\childdocjob\childdocname
%    \end{macrocode}

% \macro{\childdocdisable}
% The macro |\childdocdisable| prevents the main file
% from being processed more than once.
% At this stage, the main document command |\childdocmain|
% is assumed to be called once again where it should do nothing.
% Any subsequent call to it should prevent
% a secondary processing of the main document
% It overwrites the forwarding commands
% |\childdocof| and |\childdocforward|
% with empty macros to prevent further inclusions of the main document:
%    \begin{macrocode}
\newcommand{\childdocdisable}
{
  \renewcommand{\childdocmain}[1]{\renewcommand{\childdocmain}[1]{\endinput}}
  \renewcommand{\childdocof}[1]{}
  \renewcommand{\childdocby}[2][]{}
  \renewcommand{\childdocforward}[2][]{}
  \renewcommand{\childdocdisable}{}
}
%    \end{macrocode}

% \macro{\childdocmain}
% The macro |\childdocmain| is to be called at the top of the main file
% with nothing or the main filename (without extension) as argument.
% First, it breaks loops.
% If the argument is not empty and does not match |\childdocname|
% (which is set by the first inclusion of |childdoc.def|),
% |\ifchilddoc| is set to true, |\includeonly| is applied to the child file
% and |\jobname| is set to the main file
% (for proper handling of |.aux| files):
%    \begin{macrocode}
\newcommand{\childdocmain}[1]
{
  \childdocdisable\childdocmain{}
  \if?#1?\else
    \begingroup
      \def\childdoctmp{#1}
      \ifx\childdoctmp\childdocname
        \def\childdoctmp{}
      \else
        \def\childdoctmp
        {
          \childdoctrue
          \includeonly{\childdocname}
          \def\childdocjob{#1}
          \def\jobname{#1}
        }
      \fi
      \expandafter
    \endgroup
    \childdoctmp
  \fi
}
%    \end{macrocode}

% \macro{\childdocof}
% The command |\childdocof| redirects
% compilation to the main file |#1|.
%    \begin{macrocode}
\newcommand{\childdocof}[1]
{
  \childdocdisable
  \childdoctrue
  \includeonly{\childdocname}
  \def\jobname{#1}
  \def\childdocjob{#1}
  \input{#1}
}
%    \end{macrocode}

% \macro{\childdocby}
% The command |\childdocby| ....
%    \begin{macrocode}
\newcommand{\childdocby}[2][]
{
  \childdocdisable
  \childdoctrue
  \childdocmanualtrue
  \if?#1?\else
    \def\jobname{#2}
  \fi
  \def\childdocjob{#2}
  \input{#2}
  \endinput
}
%    \end{macrocode}

% \macro{\childdocforward}
% The command |\childdocforward| redirects
% compilation to the main file or
% (if the optional argument is given) a child file.
% Parameters are set as if the main file
% or a child file starting with |\childdocof| was compiled.
% Then compilation is handed over to the main file:
%    \begin{macrocode}
\newcommand{\childdocforward}[2][]
{
  \begingroup
    \if?#1?
      \def\childdoctmp
      {
        \def\childdocname{#2}
        \def\childdocjob{#2}
        \def\jobname{#2}
        \input{#2}
        \endinput
      }
    \else
      \def\childdoctmp
      {
        \childdocdisable
        \def\childdocname{#2}
        \childdoctrue
        \includeonly{#2}
        \def\childdocjob{#1}
        \def\jobname{#1}
        \input{#1}
        \endinput
      }
    \fi
    \expandafter
  \endgroup
  \childdoctmp
}
%    \end{macrocode}

% \macro{\childdocforwardprefix}
% The command |\childdocforwardprefix| redirects
% compilation to the main or a child file by means of a pattern.
% The prefix |#1| in the current filename is replaced by |#2|
% and the suffix of the current filename is kept
% (it is assumed that the filename does not contain the substring `|~~~|'
% which is used as a delimiter).
% Compilation is handed over to the new file by |\childdocforward|:
%    \begin{macrocode}
\newcommand{\childdocforwardprefix}[3][]
{
  \begingroup
    \def\childdocextract #2##1~~~{\def\childdoctmp{\childdocforward[#1]{#3##1}}}
    \expandafter\childdocextract\childdocname~~~
    \expandafter
  \endgroup
  \childdoctmp
}
%    \end{macrocode}

% \macro{\childdoc}
% The deprecated macro |\childdoc| is a legacy version of |\childdocmain|:
%    \begin{macrocode}
\newcommand{\childdoc}{\childdocmain}
%    \end{macrocode}

% \macro{\childdocredirect}
% The deprecated macro |\childdocredirect| is a legacy version
% of |\childdocforward| and |\childdocforwardprefix|:
%    \begin{macrocode}
\newcommand{\childdocredirect}[2][]
{
  \begingroup
    \if?#1?
      \def\childdoctmp{\childdocforward{#2}}
    \else
      \def\childdoctmp{\childdocforwardprefix{#1}{#2}}
    \fi
    \expandafter
  \endgroup
  \childdoctmp
}
%    \end{macrocode}

%\iffalse
%</package>
%\fi
%
\endinput

\childdocby{cdocsamp}
%    \end{macrocode}

%\iffalse
%</samplepart3|samplepart4>
%\fi
%
%\iffalse
%<*samplepart3>
%\fi
% Some text for part 3:
%    \begin{macrocode}
some text in part three
%    \end{macrocode}

%\iffalse
%</samplepart3>
%\fi
% Some text for part 4:
%\iffalse
%<*samplepart4>
%\fi
%    \begin{macrocode}
more text in part four
%    \end{macrocode}

%\iffalse
%</samplepart4>
%\fi
%
% %%%%%%%%%%%%%%%%%%%%%%%%%%%%%%%%%%%%%%
% \paragraph{Forwarding for a Complete Draft.}
%
% The following forwarding file |cdocsdrf.tex|
% compiles the main document in draft mode:
%\iffalse
%<*sampledraft>
%\fi
%    \begin{macrocode}
\def\version{draft}
% \iffalse
%
% childdoc.dtx Copyright (C) 2017-2018 Niklas Beisert
%
% This work may be distributed and/or modified under the
% conditions of the LaTeX Project Public License, either version 1.3
% of this license or (at your option) any later version.
% The latest version of this license is in
%   http://www.latex-project.org/lppl.txt
% and version 1.3 or later is part of all distributions of LaTeX
% version 2005/12/01 or later.
%
% This work has the LPPL maintenance status `maintained'.
%
% The Current Maintainer of this work is Niklas Beisert.
%
% This work consists of the files childdoc.dtx and childdoc.ins
% and the derived files childdoc.def and cdocsamp.tex with
% cdocsch1.tex, cdocsch2.tex, cdocsdrf.tex, cdocsfn1.tex, cdocsfn2.tex.
%
%<package>\ifdefined\childdocmain\endinput\fi
%<package>\ProvidesFile{childdoc.def}[2018/12/30 v2.0 child document driver]
%<samplemain>\ProvidesFile{cdocsamp.tex}[2018/12/30 v2.0 sample for childdoc]
%<*driver>
%\ProvidesFile{childdoc.drv}[2018/12/30 v2.0 childdoc reference manual file]
\PassOptionsToClass{10pt,a4paper}{article}
\documentclass{ltxdoc}

\usepackage[margin=35mm]{geometry}
\usepackage{hyperref}
\usepackage{hyperxmp}
\usepackage[usenames]{color}

\hypersetup{colorlinks=true}
\hypersetup{pdfstartview=FitH}
\hypersetup{pdfpagemode=UseNone}
\hypersetup{pdfsource={}}
\hypersetup{pdflang={en-UK}}
\hypersetup{pdfcopyright={Copyright 2017-2018 Niklas Beisert.
  This work may be distributed and/or modified under the
  conditions of the LaTeX Project Public License, either version 1.3
  of this license or (at your option) any later version.}}
\hypersetup{pdflicenseurl={http://www.latex-project.org/lppl.txt}}
\hypersetup{pdfcontactaddress={ETH Zurich, ITP, HIT K,
  Wolfgang-Pauli-Strasse 27}}
\hypersetup{pdfcontactpostcode={8093}}
\hypersetup{pdfcontactcity={Zurich}}
\hypersetup{pdfcontactcountry={Switzerland}}
\hypersetup{pdfcontactemail={nbeisert@itp.phys.ethz.ch}}
\hypersetup{pdfcontacturl={http://people.phys.ethz.ch/\xmptilde nbeisert/}}

\newcommand{\secref}[1]{\hyperref[#1]{section \ref*{#1}}}

\parskip1ex
\parindent0pt
\let\olditemize\itemize
\def\itemize{\olditemize\parskip0pt}

\begin{document}

\title{The \textsf{childdoc} Package}
\hypersetup{pdftitle={The childdoc Package}}
\author{Niklas Beisert\\[2ex]
  Institut f\"ur Theoretische Physik\\
  Eidgen\"ossische Technische Hochschule Z\"urich\\
  Wolfgang-Pauli-Strasse 27, 8093 Z\"urich, Switzerland\\[1ex]
  \href{mailto:nbeisert@itp.phys.ethz.ch}
  {\texttt{nbeisert@itp.phys.ethz.ch}}}
\hypersetup{pdfauthor={Niklas Beisert}}
\hypersetup{pdfsubject={Manual for the LaTeX2e Package childdoc}}
\date{30 December 2018, \textsf{v2.0}}
\maketitle

\begin{abstract}\noindent
\textsf{childdoc} is a \LaTeXe{} package
that enables the direct compilation
of document sections included by |\include|
to individual files.
\end{abstract}

\begingroup
\parskip0ex
\tableofcontents
\endgroup

%%%%%%%%%%%%%%%%%%%%%%%%%%%%%%%%%%%%%%%%%%%%%%%%%%%%%%%%%%%%%%%%%%%%%%%%%%%%%%%%
%%%%%%%%%%%%%%%%%%%%%%%%%%%%%%%%%%%%%%%%%%%%%%%%%%%%%%%%%%%%%%%%%%%%%%%%%%%%%%%%
\section{Introduction}

\LaTeX{} provides a mechanism to structure a large document (such as a book)
into a main file and several child files (containing the chapters)
using the |\include| command.
This mechanism is beneficial for documents
which span hundreds of pages in order to
make the source file(s) more manageable.
Moreover, compilation can be restricted to
selected child files by means of the |\includeonly| command.
The latter feature can be used to reduce the compilation time while editing
(this was significantly more useful in the earlier days of \LaTeX{})
or to generate a smaller document which is easier to navigate.
Another application of |\includeonly| is to generate
documents consisting of selected parts of the complete document.

However, there are a few drawbacks of the plain |\include| mechanism:
\begin{itemize}
\item
The child files cannot be compiled on their own,
they can only be compiled via the main file.
A naive editing environment
(such as a text editor with an option
to have the current file processed by \LaTeX)
may require one to switch to the main file before compiling;
attempting to compile the child file produces errors.
\item
The main file must be modified (each time)
to adjust the |\includeonly| command
to the present needs. This easily leaves the main file in a messy state.
\item
The generated document will always carry the filename
of the main document. This is inconvenient if
several child files are to be compiled and
to be kept for distribution.
\end{itemize}

The present package provides a simple interface
to make child files individually compilable by \LaTeX{}.
Compiling a child file then has the same effect as compiling
the main file with an |\includeonly| command
to select the appropriate child.
Moreover the generated document will carry the name of the child
rather than the main file.
This resolves all three above issues.

This feature is meant to make the editing of books,
thesis documents and lecture notes somewhat more convenient.
However, the package can also be used efficiently for
composing a series of documents (such as exercise sheets)
which are typically distributed individually.
It then assists the author in generating the individual documents
(potentially in different versions)
as well as a document containing the collected series.
Another application is in developing style files
or other kinds of included material
where compilation of the style file could redirect
to a sample or test file.

%%%%%%%%%%%%%%%%%%%%%%%%%%%%%%%%%%%%%%%%%%%%%%%%%%%%%%%%%%%%%%%%%%%%%%%%%%%%%%%%
%%%%%%%%%%%%%%%%%%%%%%%%%%%%%%%%%%%%%%%%%%%%%%%%%%%%%%%%%%%%%%%%%%%%%%%%%%%%%%%%
\section{Usage}

First of all, the package \textsf{childdoc} is \emph{not} a standard
\LaTeXe{} |.sty| style file! Therefore it needs to be invoked in
a non-standard way.

%%%%%%%%%%%%%%%%%%%%%%%%%%%%%%%%%%%%%%%%%%%%%%%%%%%%%%%%%%%%%%%%%%%%%%%%%%%%%%%%
\subsection{Included Files}
\label{sec:include}

%%%%%%%%%%%%%%%%%%%%%%%%%%%%%%%%%%%%%%%%
\DescribeMacro{\childdocmain}
To use the package, add the commands
\begin{center}
\begin{tabular}{l}
|\input{childdoc.def}|\\
|\childdocmain{}|\\
\end{tabular}
\end{center}
at the very top of the main \LaTeX{} file,
in particular \emph{before} the |\documentclass| statement!
The argument of |\childdocmain| should be left empty
(but it must be present).

%%%%%%%%%%%%%%%%%%%%%%%%%%%%%%%%%%%%%%%%
\DescribeMacro{\childdocof}
Furthermore, add the commands
\begin{center}
\begin{tabular}{l}
|\input{childdoc.def}|\\
|\childdocof{|\textit{main}|}|\\
\end{tabular}
\end{center}
at the top of every child file \textit{child}
which is included by |\include{|\textit{child}|}|
from within the main file
(or at least for those files to be compiled individually).
The argument \textit{main} must be the filename of the main file.

There are a couple of
considerations in setting up the main and child documents:

%%%%%%%%%%%%%%%%%%%%%%%%%%%%%%%%%%%%%%%%
\paragraph{Restrictions.}

Please note the following restrictions:
\begin{itemize}
\item
|\childdocmain| must be called with one argument \textit{main}
to ensure compatibility with earlier version of the package.
It must either be empty (|\childdocmain{}|)
or precisely match the filename of the main file in which it is specified.
See \secref{sec:detection} for further information.
\item
The filename \textit{main} must be specified without the |.tex| extension.
\item
The filename \textit{main} is case sensitive
(even in case-insensitive file systems)
due to internal string comparison.
\item
The argument \textit{main} should be fully expanded, it cannot be a macro.
\item
Subdirectories and special characters should be avoided in filenames.
\item
The command |\childdocmain{|\textit{main}|}| must be followed by a whitespace.
It should not be followed immediately by another command
or by a comment mark `|%|'.
This is because the \TeX{} parser reads the token immediately following
the argument of |\childdocmain| and puts it
at the beginning of every child section;
however, a white\-space is ignored.
\end{itemize}

%%%%%%%%%%%%%%%%%%%%%%%%%%%%%%%%%%%%%%%%
\paragraph{Content of Main File.}

It is advisable to place all content in the child files included by |\include|.
Any output contained in the main file will appear in all child documents
unless suppressed manually;
it cannot be suppressed automatically by the |\includeonly| directive
and thus should normally be avoided.
A method to include some content in the main file
by means of conditional processing is described in \secref{sec:conditional}.

%%%%%%%%%%%%%%%%%%%%%%%%%%%%%%%%%%%%%%%%
\paragraph{Page Numbering.}

When only a part of the document is compiled,
the appropriate numbering of pages
(as well as other status parameters)
is determined from the |.aux| files.
The latter contain information from previous passes.
However this information needs to propagate through
all intermediate child documents.
Therefore the page numbering in child documents may well
be inconsistent until the complete document is compiled at least once.

A useful (if unconventional) way to always ensure a consistent
page numbering is to restart the numbering in each child document
and denote the pages by `\textit{child}|.|\textit{page}'
where \textit{child} represents the chapter/section number of the child file.
This can be achieved by the command
|\numberwithin{page}{|\textit{child}|}|
of the \textsf{amsmath} package
where \textit{child} can be |chapter| or |section|
depending on the chosen structuring.
Alternatively, one can modify the macro |\thepage| appropriately
and reset the counter |page| at the start of each child file.

%%%%%%%%%%%%%%%%%%%%%%%%%%%%%%%%%%%%%%%%%%%%%%%%%%%%%%%%%%%%%%%%%%%%%%%%%%%%%%%%
\subsection{Conditional Processing}
\label{sec:conditional}

The package provides a mechanism to compile different versions
of a document. To customise the versions further some conditional processing
can come in handy to distinguish which version is being compiled.
The package provides two macros to describe the compilation context:

%%%%%%%%%%%%%%%%%%%%%%%%%%%%%%%%%%%%%%%%
\DescribeMacro{\ifchilddoc}
The conditional |\ifchilddoc| distinguishes between the compilation of
child documents and the main document:
%
\begin{center}
|\ifchilddoc |\textit{child-code}| |[|\||else |\textit{main-code}]| \||fi|
\end{center}

%%%%%%%%%%%%%%%%%%%%%%%%%%%%%%%%%%%%%%%%
\DescribeMacro{\childdocname}
\DescribeMacro{\childdocjob}
The macro |\childdocname| contains the filename (without extension)
of the main or child file being processed.
Note that |\childdocjob| will always contain the name of the main file.

%%%%%%%%%%%%%%%%%%%%%%%%%%%%%%%%%%%%%%%%
\paragraph{Title Page.}

Conditional processing can be used to include a title or banner page
in the main document when proper precautions are taken.
Importantly, the code in the main file should ensure that the page counter
(as well as other status parameters which are stored in the |.aux| files)
takes the same value after the conditional processing.
Otherwise the page numbers may take divergent values
depending on which part is compiled.

For example, a title page could be declared by:
%
\begin{center}
\begin{tabular}{l}
|\ifchilddoc\||else|\\
|\addtocounter{page}{-1}|\\
\textit{code for title page}\\
|\newpage|\\
|\||fi|
\end{tabular}
\end{center}
%
A banner page for the child documents can be generated by:
%
\begin{center}
\begin{tabular}{l}
|\ifchilddoc|\\
|\addtocounter{page}{-1}|\\
\textit{code for banner page}\\
|\newpage|\\
|\||fi|
\end{tabular}
\end{center}
%
Here one could write a message such as:
\begin{center}
|This is the part \childdocname{} of \childdocjob{}.|
\end{center}

%%%%%%%%%%%%%%%%%%%%%%%%%%%%%%%%%%%%%%%%%%%%%%%%%%%%%%%%%%%%%%%%%%%%%%%%%%%%%%%%
\subsection{Flags}
\label{sec:flags}

The package makes it easy to generate different versions
of the main or child documents.
To this end compilation flags can be defined
and assigned different default values.
They will be particularly useful in conjunction
with the forwarding mechanism described in \secref{sec:forward}.

For example, it may be useful to have a flag |\version|
which can be set to |draft| or |final|.
The document source will contain some conditional code
depending on the value of |\version|.
Suppose further, the flag should default to |final| for the main file
and to |draft| for child files
which is a natural assignment for editing the document.
This is achieved by placing the following code
in the preamble of the main document
(below the |\childdocmain| directive):
%
\begin{center}
\begin{tabular}{l}
|\ifchilddoc|\\
|\providecommand{\version}{draft}|\\
|\||else|\\
|\providecommand{\version}{final}|\\
|\||fi|
\end{tabular}
\end{center}
%
The definition by |\providecommand| makes sure
that previous definitions are not overwritten.
Further statements |\providecommand{\version}{...}|
can thus be added before the above code to override it.

For the main file, one might add a line
(between |\childdocmain| and the above block)
%
\begin{center}
|%\ifchilddoc\||else\providecommand{\version}{draft}\||fi|
\end{center}
%
which can be uncommented to produce a draft version.
Likewise one can add a line to the very top of a child file
(above the |\childdocof{|\textit{main}|}| directive)
%
\begin{center}
|%\providecommand{\version}{final}|
\end{center}
%
which can be uncommented to produce the final version of this child document.

%%%%%%%%%%%%%%%%%%%%%%%%%%%%%%%%%%%%%%%%%%%%%%%%%%%%%%%%%%%%%%%%%%%%%%%%%%%%%%%%
\subsection{Forwarding}
\label{sec:forward}

Different versions of the main or child documents
using compilation flags as described in \secref{sec:flags}
can be (permanently) stored in different files
for convenient compilation, viewing and distribution.
To this end, the package defines a command
to pass on compilation to a different file:

%%%%%%%%%%%%%%%%%%%%%%%%%%%%%%%%%%%%%%%%
\DescribeMacro{\childdocforward}
The command |\childdocforward| redirects processing to
another source file:
%
\begin{center}
\begin{tabular}{l}
|\input{childdoc.def}|\\
|\childdocforward[|\textit{main}|]{|\textit{dest}|}|\\
\end{tabular}
\end{center}
%
The argument \textit{dest} is the destination file
(without extension).
It should be the main file or one of the child files.
Note that further \textsf{childdoc} directives
such as |\childdocof| and |\childdocforward|
in the indicated file will be processed in this form.
The optional argument \textit{main}
passes on directly to the main file \textit{main}
while pretending to compile the child \textit{dest}.
This form behaves as if \textit{dest}
issues |\childdocof{|\textit{main}|}| right away,
and no further \textsf{childdoc} directives will be processed.

%%%%%%%%%%%%%%%%%%%%%%%%%%%%%%%%%%%%%%%%
\DescribeMacro{\...prefix}
In the alternative form |\childdocforwardprefix|,
%
\begin{center}
\begin{tabular}{l}
|\input{childdoc.def}|\\
|\childdocforwardprefix[|\textit{main}|]{|\textit{prefix}|}{|\textit{dest}|}|
\end{tabular}
\end{center}
%
the destination file is determined by a pattern
depending on the current file:
To make this work, the current file must be called
`{\textit{prefix}\hspace{0.2em}\textit{suffix}}'
with \textit{prefix} matching precisely the argument.
Processing is then passed on to the file
`{\textit{dest}\hspace{0.2em}\textit{suffix}}'.
Surely, the same effect is achieved by
directly specifying the
argument `{\textit{dest}\hspace{0.2em}\textit{suffix}}'
in the first form.
However, that requires to set up a different file
for each child. With the alternative form of the command
all these files can have exactly the same content
which simplifies setting them up and maintaining them.

For example, the following file |draft.tex|
with a compilation flag |\version| as described in \secref{sec:flags}
compiles the main document as a draft:
%
\begin{center}
\begin{tabular}{l}
|\def\version{draft}|\\
|\input{childdoc.def}|\\
|\childdocforward{|\textit{main}|}|
\end{tabular}
\end{center}
%
Likewise, the following files |final|\textit{nn}|.tex|
compile the final version of the child document
|child|\textit{nn}|.tex|:
%
\begin{center}
\begin{tabular}{l}
|\def\version{final}|\\
|\input{childdoc.def}|\\
|\childdocforwardprefix{final}{child}|
\end{tabular}
\end{center}
%

Note that when several versions of a main file and/or of each child file
are to be generated, it may be convenient to set up a |Makefile| or
shell script to automatise the process.

%%%%%%%%%%%%%%%%%%%%%%%%%%%%%%%%%%%%%%%%%%%%%%%%%%%%%%%%%%%%%%%%%%%%%%%%%%%%%%%%
\subsection{Command Line Processing}
\label{sec:commandline}

The effect of redirection files can also be achieved by invoking
the \LaTeX{} compiler with a more elaborate command line.
Most conveniently this should be done as part
of a shell script or a |Makefile|.

When using \textsf{childdoc} in the main file, the following
command lines effectively perform a redirection
(note that depending on the shell being used,
backslashes may have to be doubled: `|\|' $\to$ `|\\|'):
%
\begin{center}
|... -jobname "|\textit{target}|" |\\|"|[\textit{flags}]%
|\input{childdoc.def}\childdocforward[|\textit{main}|]{|\textit{dest}|}"|
\end{center}
%
Here \textit{target} is the name of the output file,
\textit{main} is the name of the main file
and \textit{dest} is the name of the main or child file to be processed
(all filenames without extensions).
The optional argument \textit{main} can be omitted
if \textit{main} matches \textit{dest}.
Optionally, compilation \textit{flags} can be defined via |\def| commands.
This command line makes the \TeX{} engine believe
it is compiling the file \textit{target}
whose content is specified as the latter parameter.
The provided code then forwards the processing to
\textit{main} or \textit{dest} as described in \secref{sec:forward}.

%%%%%%%%%%%%%%%%%%%%%%%%%%%%%%%%%%%%%%%%%%%%%%%%%%%%%%%%%%%%%%%%%%%%%%%%%%%%%%%%
\subsection{Include by Input}
\label{sec:input}

Including child documents by |\include| has some restrictions by design.
Most notably, the content of a child document always occupies
its own set of pages; pages cannot be shared between child documents.
Usually, this behaviour makes perfect sense
because each child document contain an essential part of the document.
However, in some situations it may be desirable to compose
a document from a collection of parts
without having mandatory page breaks between then.
For this case, the package
provides a mechanism to include parts
by |\input| which can also be processed individually.
However, by construction this mechanism
requires manual handling of the content to be output.

%%%%%%%%%%%%%%%%%%%%%%%%%%%%%%%%%%%%%%%%
\DescribeMacro{\ifchilddocmanual}
The main file should be prepared as usual, see \secref{sec:include}.
However, the document body must make a distinction
between processing of an individual part and of the main document, e.g.:
%
\begin{center}
\begin{tabular}{l}
|\ifchilddocmanual|\\
|\input{\childdocname}|\\
|\||else|\\
\textit{document body with }|\input{|\textit{part}|}|\\
|\||fi|
\end{tabular}
\end{center}
%
The conditional |\ifchilddocmanual| is true whenever
a part to be included by |\input| is being compiled,
and the name of the part is stored in |\childdocname|.

%%%%%%%%%%%%%%%%%%%%%%%%%%%%%%%%%%%%%%%%
\DescribeMacro{\childdocby}
Each part to be included by |\input| should start with:
%
\begin{center}
\begin{tabular}{l}
|\input{childdoc.def}|\\
|\childdocby{|\textit{main}|}|\\
\end{tabular}
\end{center}
%
The directive |\childdocby| is similar to |\childdocof|
described in \secref{sec:include},
but the subsequent selection of content must be done manually.
To that end, both |\ifchilddoc| and |\ifchilddocmanual|
will be true upon processing of a part,
and the name of the part is stored in |\childdocname|.
Note that |\jobname| will be set to the filename of the current part
so that each part receives an individual |.aux| file
that does not interfere with the |.aux| file(s) of the main document.
This behaviour can be altered by the alternative form
|\childdocby[*]{|\textit{main}|}| (with a non-empty optional argument)
which uses the |.aux| file of the main document
by setting |\jobname| to \textit{main}.

%%%%%%%%%%%%%%%%%%%%%%%%%%%%%%%%%%%%%%%%%%%%%%%%%%%%%%%%%%%%%%%%%%%%%%%%%%%%%%%%
\subsection{Driver Development}
\label{sec:driver}

The \textsf{childdoc} mechanism can also be use for the development
of definition files such as \LaTeX{} styles or classes.
This case differs from the above setup with multiple parts
included by |\include| in that no |\includeonly| should be invoked.
This can be achieved by starting the include file
(before |\ProvidesPackage|) with:
%
\begin{center}
\begin{tabular}{l}
|\input{childdoc.def}|\\
|\childdocforward{|\textit{main}|}|\\
\end{tabular}
\end{center}
%
or alternatively with:
%
\begin{center}
\begin{tabular}{l}
|\input{childdoc.def}|\\
|\childdocby{|\textit{main}|}|\\
\end{tabular}
\end{center}
%
Both forms have slightly different effects as described above.
The main file is prepared as usual, see \secref{sec:include}.

%%%%%%%%%%%%%%%%%%%%%%%%%%%%%%%%%%%%%%%%%%%%%%%%%%%%%%%%%%%%%%%%%%%%%%%%%%%%%%%%
\subsection{Legacy Detection}
\label{sec:detection}

The directive |\childdocmain| in the main file can detect
whether the complete document or merely a child is to be compiled
even without using the directive |\childdocof|.
This method is deprecated because it is less robust
and there is no compelling reason to use it;
it is merely provided for backward compatibility
and it may be removed in future versions.

If the detection mechanism is to be used,
it is mandatory to correctly specify
the filename of the main file as the argument of |\childdocmain|:
%
\begin{center}
\begin{tabular}{l}
|\input{childdoc.def}|\\
|\childdocmain{|\textit{main}|}|\\
\end{tabular}
\end{center}
%
If |\jobname| does not match the argument \textit{main} of |\childdocmain|,
it is assumed that |\jobname| points to the child file to be compiled.
When using |\childdocmain| with the main file specified as argument,
it suffices to start a child file
with just |\input{|\textit{main}|}|
without loading of the package and using |\childdocof|.
If instead all processing is done
with the appropriate \textsf{childdoc} directives,
the argument of \textit{main} of |\childdocmain| can be empty.

An alternative version of the command line processing described
in \secref{sec:commandline} using the detection mechanism reads:
%
\begin{center}
|... -jobname "|\textit{target}|" "|[\textit{flags}]%
[|\def\jobname{|\textit{dest}|}|]|\input{|\textit{main}|}"|
\end{center}

%%%%%%%%%%%%%%%%%%%%%%%%%%%%%%%%%%%%%%%%%%%%%%%%%%%%%%%%%%%%%%%%%%%%%%%%%%%%%%%%
\subsection{Manual Code}
\label{sec:manual}

In case one cannot be certain whether the definitions file |childdoc.def|
is installed on the target \TeX{} distribution
and one prefers not to ship it,
it is conceivable to paste a few relevant commands into the sources.

To that end, drop all statements |\input{childdoc.def}|
and perform the replacements as outlined below.
Instead of |\childdocmain{|\textit{main}|}| add the following code
to the top of the main file:
%
\begin{center}
\begin{tabular}{l}
|\||ifdefined\childdocname\endinput\||fi\newif\ifchilddoc|\\
|\edef\childdocname{\scantokens\expandafter{\jobname\noexpand}}|\\
|\def\childdocmain{|\textit{main}|}\||ifx\childdocmain\childdocname\||else|\\
|\childdoctrue\includeonly{\childdocname}\let\jobname\childdocmain\||fi|\\
\end{tabular}
\end{center}
%
Instead of |\childdocof{|\textit{main}|}| just include the main file
at the top of each child file:
%
\begin{center}
|\input{|\textit{main}|}|
\end{center}
%
A simple redirection |\childdocforward{|\textit{dest}|}| is achieved by:
%
\begin{center}
|\def\jobname{|\textit{dest}|}\input{\jobname}|
\end{center}
%
The redirection with prefix
|\childdocforwardprefix[|\textit{prefix}|]{|\textit{dest}|}|
is accomplished by:
%
\begin{center}
\begin{tabular}{l}
|{\edef\jobname{\scantokens\expandafter{\jobname\noexpand}}|\\
|\def\redirectjob |\textit{prefix}|#1~~~{\gdef\jobname{|\textit{dest}|#1}}|\\
|\expandafter\redirectjob\jobname~~~}\input{\jobname}|
\end{tabular}
\end{center}

In an alternative approach,
child documents can be compiled by a specific command line
without additional code or specific definitions:
%
\begin{center}
|... -jobname "|\textit{target}|" "|[\textit{flags}]%
|\includeonly{|\textit{dest}|}\input{|\textit{main}|}"|
\end{center}
%

%%%%%%%%%%%%%%%%%%%%%%%%%%%%%%%%%%%%%%%%%%%%%%%%%%%%%%%%%%%%%%%%%%%%%%%%%%%%%%%%
%%%%%%%%%%%%%%%%%%%%%%%%%%%%%%%%%%%%%%%%%%%%%%%%%%%%%%%%%%%%%%%%%%%%%%%%%%%%%%%%
\section{Information}

%%%%%%%%%%%%%%%%%%%%%%%%%%%%%%%%%%%%%%%%%%%%%%%%%%%%%%%%%%%%%%%%%%%%%%%%%%%%%%%%
\subsection{Copyright}

Copyright \copyright{} 2017--2018 Niklas Beisert

This work may be distributed and/or modified under the
conditions of the \LaTeX{} Project Public License, either version 1.3
of this license or (at your option) any later version.
The latest version of this license is in
  \url{http://www.latex-project.org/lppl.txt}
and version 1.3 or later is part of all distributions of \LaTeX{}
version 2005/12/01 or later.

This work has the LPPL maintenance status `maintained'.

The Current Maintainer of this work is Niklas Beisert.

This work consists of the files |README.txt|, |childdoc.ins| and |childdoc.dtx|
as well as the derived files |childdoc.def|, |cdocsamp.tex|
with |cdocsch1.tex|, |cdocsch2.tex|, |cdocspt3.tex|, |cdocspt4.tex|,
|cdocsdrf.tex|, |cdocsfn1.tex|, |cdocsfn2.tex|
as well as |childdoc.pdf|.

%%%%%%%%%%%%%%%%%%%%%%%%%%%%%%%%%%%%%%%%%%%%%%%%%%%%%%%%%%%%%%%%%%%%%%%%%%%%%%%%
\subsection{Files and Installation}

The package consists of the files:
%
\begin{center}
\begin{tabular}{ll}
    |README.txt|   & readme file \\
    |childdoc.ins| & installation file \\
    |childdoc.dtx| & source file \\
    |childdoc.def| & definition file \\
    |cdocsamp.tex| & sample main file \\
    |cdocsch1.tex| & sample include file \\
    |cdocsch2.tex| & sample include file \\
    |cdocspt3.tex| & sample part file \\
    |cdocspt4.tex| & sample part file \\
    |cdocsdrf.tex| & sample redirection file \\
    |cdocsfn1.tex| & sample redirection file \\
    |cdocsfn2.tex| & sample redirection file \\
    |childdoc.pdf| & manual
\end{tabular}
\end{center}
%
The distribution consists of the files
|README.txt|, |childdoc.ins| and |childdoc.dtx|.
%
\begin{itemize}
\item
Run (pdf)\LaTeX{} on |childdoc.dtx|
to compile the manual |childdoc.pdf| (this file).
\item
Run \LaTeX{} on |childdoc.ins| to create the definitions file |childdoc.def|
and the sample |cdocsamp.tex| with include files
|cdocsch1.tex|, |cdocsch2.tex|, |cdocspt3.tex|, |cdocspt4.tex|,
|cdocsdrf.tex|, |cdocsfn1.tex|, |cdocsfn2.tex|.
Then copy the file |childdoc.def| to an appropriate directory of your \LaTeX{}
distribution, e.g.\ \textit{texmf-root}|/tex/latex/childdoc|.
\end{itemize}

%%%%%%%%%%%%%%%%%%%%%%%%%%%%%%%%%%%%%%%%%%%%%%%%%%%%%%%%%%%%%%%%%%%%%%%%%%%%%%%%
\subsection{Related CTAN Packages}

There are several other packages which offer a similar functionality:
%
\begin{itemize}
\item
The packages
\href{http://ctan.org/pkg/docmute}{\textsf{docmute}},
\href{http://ctan.org/pkg/includex}{\textsf{includex}} and
\href{http://ctan.org/pkg/standalone}{\textsf{standalone}}
provide commands to include only the document body of
a child file thus allowing both files to be compiled individually.
\item
The packages \href{http://ctan.org/pkg/subdocs}{\textsf{subdocs}}
and \href{http://ctan.org/pkg/subfiles}{\textsf{subfiles}}
provide structures in which the main and child documents can be
encapsulated and allowing them to be compiled individually.
The inclusion mechanism is different from the conventional |\include|.
\item
The package \href{http://ctan.org/pkg/combine}{\textsf{combine}}
is an elaborate solution to combine several documents into one.
\end{itemize}
%
See also the CTAN topic \href{http://ctan.org/topic/subdocs}{\textsf{subdocs}}
for further related packages.
The present package differs from the above solutions in that
a document structure constructed with the conventional |\include| mechanism
just needs two extra commands at the top of every file
such that all constituent files can be compiled individually.

%%%%%%%%%%%%%%%%%%%%%%%%%%%%%%%%%%%%%%%%%%%%%%%%%%%%%%%%%%%%%%%%%%%%%%%%%%%%%%%%
%\subsection{Feature Suggestions}
%
%The following is a list of features which may be useful for future
%versions of this package:
%%
%\begin{itemize}
%\item
%\ldots
%\end{itemize}

%%%%%%%%%%%%%%%%%%%%%%%%%%%%%%%%%%%%%%%%%%%%%%%%%%%%%%%%%%%%%%%%%%%%%%%%%%%%%%%%
\subsection{Revision History}

%%%%%%%%%%%%%%%%%%%%%%%%%%%%%%%%%%%%%%%%
\paragraph{v2.0:} 2018/12/30

\begin{itemize}
\item
immediate forward processing
\item
added |\childdocby| mechanism
\item
manual restructured
\end{itemize}

%%%%%%%%%%%%%%%%%%%%%%%%%%%%%%%%%%%%%%%%
\paragraph{v1.6:} 2018/01/17

\begin{itemize}
\item
application for development of include files
\item
corrections to manual
\end{itemize}

%%%%%%%%%%%%%%%%%%%%%%%%%%%%%%%%%%%%%%%%
\paragraph{v1.5:} 2017/05/21

\begin{itemize}
\item
more complete structuring introduced
\item
|\childdocof| introduced
\item
|\childdoc| renamed to |\childdocmain|
\item
|\childredirect| renamed to |\childdocforward| and |\childdocforwardprefix|
and functionality expanded
\end{itemize}

%%%%%%%%%%%%%%%%%%%%%%%%%%%%%%%%%%%%%%%%
\paragraph{v1.0:} 2017/04/27

\begin{itemize}
\item
manual and install package
\item
first version published on CTAN
\end{itemize}

%%%%%%%%%%%%%%%%%%%%%%%%%%%%%%%%%%%%%%%%
\paragraph{v0.6:} 2017/04/26

\begin{itemize}
\item
redirection mechanism added
\end{itemize}

%%%%%%%%%%%%%%%%%%%%%%%%%%%%%%%%%%%%%%%%
\paragraph{v0.5:} 2017/04/26

\begin{itemize}
\item
functionality in definition file
\end{itemize}


%%%%%%%%%%%%%%%%%%%%%%%%%%%%%%%%%%%%%%%%%%%%%%%%%%%%%%%%%%%%%%%%%%%%%%%%%%%%%%%%
%%%%%%%%%%%%%%%%%%%%%%%%%%%%%%%%%%%%%%%%%%%%%%%%%%%%%%%%%%%%%%%%%%%%%%%%%%%%%%%%
%%%%%%%%%%%%%%%%%%%%%%%%%%%%%%%%%%%%%%%%%%%%%%%%%%%%%%%%%%%%%%%%%%%%%%%%%%%%%%%%
\appendix

\settowidth\MacroIndent{\rmfamily\scriptsize 000\ }

 \DocInput{childdoc.dtx}

\end{document}
%</driver>
% \fi
%
% %%%%%%%%%%%%%%%%%%%%%%%%%%%%%%%%%%%%%%%%%%%%%%%%%%%%%%%%%%%%%%%%%%%%%%%%%%%%%%
% %%%%%%%%%%%%%%%%%%%%%%%%%%%%%%%%%%%%%%%%%%%%%%%%%%%%%%%%%%%%%%%%%%%%%%%%%%%%%%
% \section{Sample}
%\iffalse
%<*samplemain>
%\fi
%
% The following presents a sample document
% with two chapters, two parts, a title page,
% a compile flag as well as three forwarding files to set the flag.
% It consists of eight |.tex| files:
% \begin{center}
% \begin{tabular}{ll}
% |cdocsamp.tex|&main file\\
% |cdocsch1.tex|&include file for chapter 1\\
% |cdocsch2.tex|&include file for chapter 2\\
% |cdocspt3.tex|&include file for part 3\\
% |cdocspt4.tex|&include file for part 4\\
% |cdocsdrf.tex|&forwarding file for main file in draft mode\\
% |cdocsfi1.tex|&forwarding file for final version of chapter 1\\
% |cdocsfi2.tex|&forwarding file for final version of chapter 2\\
% \end{tabular}
% \end{center}
% Each of the eight files can be compiled directly by the \LaTeX{} compiler.
%
% %%%%%%%%%%%%%%%%%%%%%%%%%%%%%%%%%%%%%%
% \paragraph{Main File.}
%
% The main file is called |cdocsamp.tex|.
%
% Load the \textsf{childdoc} definitions and
% declare the filename for the main document:
%    \begin{macrocode}
\input{childdoc.def}
\childdocmain{}
%    \end{macrocode}

% Optional override for |\version| flag:
%    \begin{macrocode}
%%\ifchilddoc\else\providecommand{\version}{draft}\fi
%    \end{macrocode}

% Define the default values for the |\version| flag
% (|final| for the main file and |draft| for childs):
%    \begin{macrocode}
\ifchilddoc
\providecommand{\version}{draft}
\else
\providecommand{\version}{final}
\fi
%    \end{macrocode}

% Load the standard document class:
%    \begin{macrocode}
\documentclass[12pt]{article}
%    \end{macrocode}

% Start the document body:
%    \begin{macrocode}
\begin{document}
%    \end{macrocode}

% Declare a title page.
% Print title, part of document being processed and version flag:
%    \begin{macrocode}
\addtocounter{page}{-1}
\begin{center}
{\LARGE\bfseries{}childdoc example\par}
\vspace{1cm}
\ifchilddoc
\ifchilddocmanual part\else chapter\fi:
`\childdocname' of `\childdocjob'\par
\else
main document: `\childdocjob'\par
\fi
version: \version\par
\end{center}
\newpage
%    \end{macrocode}

% Manually include selected file,
% otherwise process as usual:
%    \begin{macrocode}
\ifchilddocmanual
\section*{part `\childdocname'}
\input{\childdocname}
\else
%    \end{macrocode}

% Include the two chapters:
%    \begin{macrocode}
\include{cdocsch1}
\include{cdocsch2}
%    \end{macrocode}

% Include the two parts unless only chapters should be displayed:
%    \begin{macrocode}
\ifchilddoc\else
\section{part three}
\input{cdocspt3}
\section{part four}
\input{cdocspt4}
\fi
%    \end{macrocode}

% Process as usual until here:
%    \begin{macrocode}
\fi
%    \end{macrocode}

% End of document body:
%    \begin{macrocode}
\end{document}
%    \end{macrocode}
%\iffalse
%</samplemain>
%\fi
%
% %%%%%%%%%%%%%%%%%%%%%%%%%%%%%%%%%%%%%%
% \paragraph{Chapter Include Files.}
%
% The include files are called |cdocsch1.tex| and |cdocsch2.tex|.
%
%\iffalse
%<*samplechap1|samplechap2>
%\fi

% Optional override for |\version| flag:
%    \begin{macrocode}
%%\providecommand{\version}{final}
%    \end{macrocode}

% Include the main document:
%    \begin{macrocode}
\input{childdoc.def}
\childdocof{cdocsamp}
%    \end{macrocode}

%\iffalse
%</samplechap1|samplechap2>
%\fi
%
%\iffalse
%<*samplechap1>
%\fi
% Some text for chapter 1:
%    \begin{macrocode}
\section{one}
some text in chapter one
%    \end{macrocode}

%\iffalse
%</samplechap1>
%\fi
% Some text for chapter 2:
%\iffalse
%<*samplechap2>
%\fi
%    \begin{macrocode}
\section{two}
more text in chapter two
%    \end{macrocode}

%\iffalse
%</samplechap2>
%\fi
%
% %%%%%%%%%%%%%%%%%%%%%%%%%%%%%%%%%%%%%%
% \paragraph{Part Include Files.}
%
% The include files are called |cdocspt3.tex| and |cdocspt4.tex|.
%
%\iffalse
%<*samplepart3|samplepart4>
%\fi

% Optional override for |\version| flag:
%    \begin{macrocode}
%%\providecommand{\version}{final}
%    \end{macrocode}

% Include the main document:
%    \begin{macrocode}
\input{childdoc.def}
\childdocby{cdocsamp}
%    \end{macrocode}

%\iffalse
%</samplepart3|samplepart4>
%\fi
%
%\iffalse
%<*samplepart3>
%\fi
% Some text for part 3:
%    \begin{macrocode}
some text in part three
%    \end{macrocode}

%\iffalse
%</samplepart3>
%\fi
% Some text for part 4:
%\iffalse
%<*samplepart4>
%\fi
%    \begin{macrocode}
more text in part four
%    \end{macrocode}

%\iffalse
%</samplepart4>
%\fi
%
% %%%%%%%%%%%%%%%%%%%%%%%%%%%%%%%%%%%%%%
% \paragraph{Forwarding for a Complete Draft.}
%
% The following forwarding file |cdocsdrf.tex|
% compiles the main document in draft mode:
%\iffalse
%<*sampledraft>
%\fi
%    \begin{macrocode}
\def\version{draft}
\input{childdoc.def}
\childdocforward{cdocsamp}
%    \end{macrocode}

%\iffalse
%</sampledraft>
%\fi
%
% %%%%%%%%%%%%%%%%%%%%%%%%%%%%%%%%%%%%%%
% \paragraph{Forwarding for Final Version of the Chapters.}
%
% The following forwarding files |cdocsfn1.tex| and |cdocsfn2.tex|
% (with identical content)
% compile the final versions of the child documents
% |cdocsch1.tex| and |cdocsch2.tex|, respectively:
%\iffalse
%<*samplefinal>
%\fi
%    \begin{macrocode}
\def\version{final}
\input{childdoc.def}
\childdocforwardprefix[cdocsamp]{cdocsfn}{cdocsch}
%    \end{macrocode}

%\iffalse
%</samplefinal>
%\fi
%
% %%%%%%%%%%%%%%%%%%%%%%%%%%%%%%%%%%%%%%
% \paragraph{Command Line Processing.}
%
% The following three command lines generate the output files
% |cdocscld|, |cdocscl1| and |cdocscl2|
% which should be identical to
% |cdocsdrf|, |cdocsch1| and |cdocsfn2|, respectively:
% \begin{center}
% \begin{tabular}{l}
% |latex -jobname cdocscld \|\\
% |  "\def\version{draft}\input{childdoc.def}\childdocforward{cdocsamp}"|\\
% |latex -jobname cdocscl1 \|\\
% |  "\input{childdoc.def}\childdocforward[cdocsamp]{cdocsch1}"|\\
% |latex -jobname cdocscl2 \|\\
% |  "\def\version{final}\input{childdoc.def}\childdocforward{cdocsch2}"|
% \end{tabular}
% \end{center}
% Note that the trailing backslash on each first line
% merely continues the input to the second line
% (for convenient cut ant paste).
% Furthermore, the command |latex| can be replaced by any
% of its alternative versions such as |pdflatex|.
%
% %%%%%%%%%%%%%%%%%%%%%%%%%%%%%%%%%%%%%%%%%%%%%%%%%%%%%%%%%%%%%%%%%%%%%%%%%%%%%%
% %%%%%%%%%%%%%%%%%%%%%%%%%%%%%%%%%%%%%%%%%%%%%%%%%%%%%%%%%%%%%%%%%%%%%%%%%%%%%%
% \section{Implementation}
%\iffalse
%<*package>
%\fi
%
% This section describes the definitions file |childdoc.def|.

% The definitions cannot be loaded using |\usepackage| or |\RequirePackage|
% which has a mechanism to prevent loading a style file more than once.
% When loading the definitions by means of |\input|
% multiple instances have to be prevented manually:
%\iffalse
%This code needs to be before the `\ProvidesFile' directive
%which is defined at the beginning of this file.
%Therefore it is also placed there and commented out here.
%</package>
%<*discard>
%\fi
%    \begin{macrocode}
\ifdefined\childdocmain\endinput\fi
%    \end{macrocode}
%\iffalse
%</discard>
%<*package>
%\fi
%
% \macro{\ifchilddoc}
% \macro{\ifchilddocmanual}
% The conditional |\ifchilddoc| tells whether a
% child (true) or main (false) document is being compiled.
% The conditional |\ifchilddocmanual| tells whether
% the |\includeonly| mechanism is used (false) or
% the selection of child files must be performed manually (true).
% The definitions initialise to false:
%    \begin{macrocode}
\newif\ifchilddoc
\newif\ifchilddocmanual
%    \end{macrocode}

% \macro{\childdocname}
% \macro{\childdocjob}
% The macro |\childdocname| stores the name of the main document
% to be compiled. The macro |\childdocjob| stores the name of
% the document on which the \LaTeX{} compiler was originally invoked.
% The content of |\jobname| cannot be compared
% to filenames specified in the source due to different catcodes.
% The following code rescans |\jobname|, stores the result
% in |\childdocname| and saves a copy in |\childdocjob|:
%    \begin{macrocode}
\edef\childdocname{\scantokens\expandafter{\jobname\noexpand}}
\let\childdocjob\childdocname
%    \end{macrocode}

% \macro{\childdocdisable}
% The macro |\childdocdisable| prevents the main file
% from being processed more than once.
% At this stage, the main document command |\childdocmain|
% is assumed to be called once again where it should do nothing.
% Any subsequent call to it should prevent
% a secondary processing of the main document
% It overwrites the forwarding commands
% |\childdocof| and |\childdocforward|
% with empty macros to prevent further inclusions of the main document:
%    \begin{macrocode}
\newcommand{\childdocdisable}
{
  \renewcommand{\childdocmain}[1]{\renewcommand{\childdocmain}[1]{\endinput}}
  \renewcommand{\childdocof}[1]{}
  \renewcommand{\childdocby}[2][]{}
  \renewcommand{\childdocforward}[2][]{}
  \renewcommand{\childdocdisable}{}
}
%    \end{macrocode}

% \macro{\childdocmain}
% The macro |\childdocmain| is to be called at the top of the main file
% with nothing or the main filename (without extension) as argument.
% First, it breaks loops.
% If the argument is not empty and does not match |\childdocname|
% (which is set by the first inclusion of |childdoc.def|),
% |\ifchilddoc| is set to true, |\includeonly| is applied to the child file
% and |\jobname| is set to the main file
% (for proper handling of |.aux| files):
%    \begin{macrocode}
\newcommand{\childdocmain}[1]
{
  \childdocdisable\childdocmain{}
  \if?#1?\else
    \begingroup
      \def\childdoctmp{#1}
      \ifx\childdoctmp\childdocname
        \def\childdoctmp{}
      \else
        \def\childdoctmp
        {
          \childdoctrue
          \includeonly{\childdocname}
          \def\childdocjob{#1}
          \def\jobname{#1}
        }
      \fi
      \expandafter
    \endgroup
    \childdoctmp
  \fi
}
%    \end{macrocode}

% \macro{\childdocof}
% The command |\childdocof| redirects
% compilation to the main file |#1|.
%    \begin{macrocode}
\newcommand{\childdocof}[1]
{
  \childdocdisable
  \childdoctrue
  \includeonly{\childdocname}
  \def\jobname{#1}
  \def\childdocjob{#1}
  \input{#1}
}
%    \end{macrocode}

% \macro{\childdocby}
% The command |\childdocby| ....
%    \begin{macrocode}
\newcommand{\childdocby}[2][]
{
  \childdocdisable
  \childdoctrue
  \childdocmanualtrue
  \if?#1?\else
    \def\jobname{#2}
  \fi
  \def\childdocjob{#2}
  \input{#2}
  \endinput
}
%    \end{macrocode}

% \macro{\childdocforward}
% The command |\childdocforward| redirects
% compilation to the main file or
% (if the optional argument is given) a child file.
% Parameters are set as if the main file
% or a child file starting with |\childdocof| was compiled.
% Then compilation is handed over to the main file:
%    \begin{macrocode}
\newcommand{\childdocforward}[2][]
{
  \begingroup
    \if?#1?
      \def\childdoctmp
      {
        \def\childdocname{#2}
        \def\childdocjob{#2}
        \def\jobname{#2}
        \input{#2}
        \endinput
      }
    \else
      \def\childdoctmp
      {
        \childdocdisable
        \def\childdocname{#2}
        \childdoctrue
        \includeonly{#2}
        \def\childdocjob{#1}
        \def\jobname{#1}
        \input{#1}
        \endinput
      }
    \fi
    \expandafter
  \endgroup
  \childdoctmp
}
%    \end{macrocode}

% \macro{\childdocforwardprefix}
% The command |\childdocforwardprefix| redirects
% compilation to the main or a child file by means of a pattern.
% The prefix |#1| in the current filename is replaced by |#2|
% and the suffix of the current filename is kept
% (it is assumed that the filename does not contain the substring `|~~~|'
% which is used as a delimiter).
% Compilation is handed over to the new file by |\childdocforward|:
%    \begin{macrocode}
\newcommand{\childdocforwardprefix}[3][]
{
  \begingroup
    \def\childdocextract #2##1~~~{\def\childdoctmp{\childdocforward[#1]{#3##1}}}
    \expandafter\childdocextract\childdocname~~~
    \expandafter
  \endgroup
  \childdoctmp
}
%    \end{macrocode}

% \macro{\childdoc}
% The deprecated macro |\childdoc| is a legacy version of |\childdocmain|:
%    \begin{macrocode}
\newcommand{\childdoc}{\childdocmain}
%    \end{macrocode}

% \macro{\childdocredirect}
% The deprecated macro |\childdocredirect| is a legacy version
% of |\childdocforward| and |\childdocforwardprefix|:
%    \begin{macrocode}
\newcommand{\childdocredirect}[2][]
{
  \begingroup
    \if?#1?
      \def\childdoctmp{\childdocforward{#2}}
    \else
      \def\childdoctmp{\childdocforwardprefix{#1}{#2}}
    \fi
    \expandafter
  \endgroup
  \childdoctmp
}
%    \end{macrocode}

%\iffalse
%</package>
%\fi
%
\endinput

\childdocforward{cdocsamp}
%    \end{macrocode}

%\iffalse
%</sampledraft>
%\fi
%
% %%%%%%%%%%%%%%%%%%%%%%%%%%%%%%%%%%%%%%
% \paragraph{Forwarding for Final Version of the Chapters.}
%
% The following forwarding files |cdocsfn1.tex| and |cdocsfn2.tex|
% (with identical content)
% compile the final versions of the child documents
% |cdocsch1.tex| and |cdocsch2.tex|, respectively:
%\iffalse
%<*samplefinal>
%\fi
%    \begin{macrocode}
\def\version{final}
% \iffalse
%
% childdoc.dtx Copyright (C) 2017-2018 Niklas Beisert
%
% This work may be distributed and/or modified under the
% conditions of the LaTeX Project Public License, either version 1.3
% of this license or (at your option) any later version.
% The latest version of this license is in
%   http://www.latex-project.org/lppl.txt
% and version 1.3 or later is part of all distributions of LaTeX
% version 2005/12/01 or later.
%
% This work has the LPPL maintenance status `maintained'.
%
% The Current Maintainer of this work is Niklas Beisert.
%
% This work consists of the files childdoc.dtx and childdoc.ins
% and the derived files childdoc.def and cdocsamp.tex with
% cdocsch1.tex, cdocsch2.tex, cdocsdrf.tex, cdocsfn1.tex, cdocsfn2.tex.
%
%<package>\ifdefined\childdocmain\endinput\fi
%<package>\ProvidesFile{childdoc.def}[2018/12/30 v2.0 child document driver]
%<samplemain>\ProvidesFile{cdocsamp.tex}[2018/12/30 v2.0 sample for childdoc]
%<*driver>
%\ProvidesFile{childdoc.drv}[2018/12/30 v2.0 childdoc reference manual file]
\PassOptionsToClass{10pt,a4paper}{article}
\documentclass{ltxdoc}

\usepackage[margin=35mm]{geometry}
\usepackage{hyperref}
\usepackage{hyperxmp}
\usepackage[usenames]{color}

\hypersetup{colorlinks=true}
\hypersetup{pdfstartview=FitH}
\hypersetup{pdfpagemode=UseNone}
\hypersetup{pdfsource={}}
\hypersetup{pdflang={en-UK}}
\hypersetup{pdfcopyright={Copyright 2017-2018 Niklas Beisert.
  This work may be distributed and/or modified under the
  conditions of the LaTeX Project Public License, either version 1.3
  of this license or (at your option) any later version.}}
\hypersetup{pdflicenseurl={http://www.latex-project.org/lppl.txt}}
\hypersetup{pdfcontactaddress={ETH Zurich, ITP, HIT K,
  Wolfgang-Pauli-Strasse 27}}
\hypersetup{pdfcontactpostcode={8093}}
\hypersetup{pdfcontactcity={Zurich}}
\hypersetup{pdfcontactcountry={Switzerland}}
\hypersetup{pdfcontactemail={nbeisert@itp.phys.ethz.ch}}
\hypersetup{pdfcontacturl={http://people.phys.ethz.ch/\xmptilde nbeisert/}}

\newcommand{\secref}[1]{\hyperref[#1]{section \ref*{#1}}}

\parskip1ex
\parindent0pt
\let\olditemize\itemize
\def\itemize{\olditemize\parskip0pt}

\begin{document}

\title{The \textsf{childdoc} Package}
\hypersetup{pdftitle={The childdoc Package}}
\author{Niklas Beisert\\[2ex]
  Institut f\"ur Theoretische Physik\\
  Eidgen\"ossische Technische Hochschule Z\"urich\\
  Wolfgang-Pauli-Strasse 27, 8093 Z\"urich, Switzerland\\[1ex]
  \href{mailto:nbeisert@itp.phys.ethz.ch}
  {\texttt{nbeisert@itp.phys.ethz.ch}}}
\hypersetup{pdfauthor={Niklas Beisert}}
\hypersetup{pdfsubject={Manual for the LaTeX2e Package childdoc}}
\date{30 December 2018, \textsf{v2.0}}
\maketitle

\begin{abstract}\noindent
\textsf{childdoc} is a \LaTeXe{} package
that enables the direct compilation
of document sections included by |\include|
to individual files.
\end{abstract}

\begingroup
\parskip0ex
\tableofcontents
\endgroup

%%%%%%%%%%%%%%%%%%%%%%%%%%%%%%%%%%%%%%%%%%%%%%%%%%%%%%%%%%%%%%%%%%%%%%%%%%%%%%%%
%%%%%%%%%%%%%%%%%%%%%%%%%%%%%%%%%%%%%%%%%%%%%%%%%%%%%%%%%%%%%%%%%%%%%%%%%%%%%%%%
\section{Introduction}

\LaTeX{} provides a mechanism to structure a large document (such as a book)
into a main file and several child files (containing the chapters)
using the |\include| command.
This mechanism is beneficial for documents
which span hundreds of pages in order to
make the source file(s) more manageable.
Moreover, compilation can be restricted to
selected child files by means of the |\includeonly| command.
The latter feature can be used to reduce the compilation time while editing
(this was significantly more useful in the earlier days of \LaTeX{})
or to generate a smaller document which is easier to navigate.
Another application of |\includeonly| is to generate
documents consisting of selected parts of the complete document.

However, there are a few drawbacks of the plain |\include| mechanism:
\begin{itemize}
\item
The child files cannot be compiled on their own,
they can only be compiled via the main file.
A naive editing environment
(such as a text editor with an option
to have the current file processed by \LaTeX)
may require one to switch to the main file before compiling;
attempting to compile the child file produces errors.
\item
The main file must be modified (each time)
to adjust the |\includeonly| command
to the present needs. This easily leaves the main file in a messy state.
\item
The generated document will always carry the filename
of the main document. This is inconvenient if
several child files are to be compiled and
to be kept for distribution.
\end{itemize}

The present package provides a simple interface
to make child files individually compilable by \LaTeX{}.
Compiling a child file then has the same effect as compiling
the main file with an |\includeonly| command
to select the appropriate child.
Moreover the generated document will carry the name of the child
rather than the main file.
This resolves all three above issues.

This feature is meant to make the editing of books,
thesis documents and lecture notes somewhat more convenient.
However, the package can also be used efficiently for
composing a series of documents (such as exercise sheets)
which are typically distributed individually.
It then assists the author in generating the individual documents
(potentially in different versions)
as well as a document containing the collected series.
Another application is in developing style files
or other kinds of included material
where compilation of the style file could redirect
to a sample or test file.

%%%%%%%%%%%%%%%%%%%%%%%%%%%%%%%%%%%%%%%%%%%%%%%%%%%%%%%%%%%%%%%%%%%%%%%%%%%%%%%%
%%%%%%%%%%%%%%%%%%%%%%%%%%%%%%%%%%%%%%%%%%%%%%%%%%%%%%%%%%%%%%%%%%%%%%%%%%%%%%%%
\section{Usage}

First of all, the package \textsf{childdoc} is \emph{not} a standard
\LaTeXe{} |.sty| style file! Therefore it needs to be invoked in
a non-standard way.

%%%%%%%%%%%%%%%%%%%%%%%%%%%%%%%%%%%%%%%%%%%%%%%%%%%%%%%%%%%%%%%%%%%%%%%%%%%%%%%%
\subsection{Included Files}
\label{sec:include}

%%%%%%%%%%%%%%%%%%%%%%%%%%%%%%%%%%%%%%%%
\DescribeMacro{\childdocmain}
To use the package, add the commands
\begin{center}
\begin{tabular}{l}
|\input{childdoc.def}|\\
|\childdocmain{}|\\
\end{tabular}
\end{center}
at the very top of the main \LaTeX{} file,
in particular \emph{before} the |\documentclass| statement!
The argument of |\childdocmain| should be left empty
(but it must be present).

%%%%%%%%%%%%%%%%%%%%%%%%%%%%%%%%%%%%%%%%
\DescribeMacro{\childdocof}
Furthermore, add the commands
\begin{center}
\begin{tabular}{l}
|\input{childdoc.def}|\\
|\childdocof{|\textit{main}|}|\\
\end{tabular}
\end{center}
at the top of every child file \textit{child}
which is included by |\include{|\textit{child}|}|
from within the main file
(or at least for those files to be compiled individually).
The argument \textit{main} must be the filename of the main file.

There are a couple of
considerations in setting up the main and child documents:

%%%%%%%%%%%%%%%%%%%%%%%%%%%%%%%%%%%%%%%%
\paragraph{Restrictions.}

Please note the following restrictions:
\begin{itemize}
\item
|\childdocmain| must be called with one argument \textit{main}
to ensure compatibility with earlier version of the package.
It must either be empty (|\childdocmain{}|)
or precisely match the filename of the main file in which it is specified.
See \secref{sec:detection} for further information.
\item
The filename \textit{main} must be specified without the |.tex| extension.
\item
The filename \textit{main} is case sensitive
(even in case-insensitive file systems)
due to internal string comparison.
\item
The argument \textit{main} should be fully expanded, it cannot be a macro.
\item
Subdirectories and special characters should be avoided in filenames.
\item
The command |\childdocmain{|\textit{main}|}| must be followed by a whitespace.
It should not be followed immediately by another command
or by a comment mark `|%|'.
This is because the \TeX{} parser reads the token immediately following
the argument of |\childdocmain| and puts it
at the beginning of every child section;
however, a white\-space is ignored.
\end{itemize}

%%%%%%%%%%%%%%%%%%%%%%%%%%%%%%%%%%%%%%%%
\paragraph{Content of Main File.}

It is advisable to place all content in the child files included by |\include|.
Any output contained in the main file will appear in all child documents
unless suppressed manually;
it cannot be suppressed automatically by the |\includeonly| directive
and thus should normally be avoided.
A method to include some content in the main file
by means of conditional processing is described in \secref{sec:conditional}.

%%%%%%%%%%%%%%%%%%%%%%%%%%%%%%%%%%%%%%%%
\paragraph{Page Numbering.}

When only a part of the document is compiled,
the appropriate numbering of pages
(as well as other status parameters)
is determined from the |.aux| files.
The latter contain information from previous passes.
However this information needs to propagate through
all intermediate child documents.
Therefore the page numbering in child documents may well
be inconsistent until the complete document is compiled at least once.

A useful (if unconventional) way to always ensure a consistent
page numbering is to restart the numbering in each child document
and denote the pages by `\textit{child}|.|\textit{page}'
where \textit{child} represents the chapter/section number of the child file.
This can be achieved by the command
|\numberwithin{page}{|\textit{child}|}|
of the \textsf{amsmath} package
where \textit{child} can be |chapter| or |section|
depending on the chosen structuring.
Alternatively, one can modify the macro |\thepage| appropriately
and reset the counter |page| at the start of each child file.

%%%%%%%%%%%%%%%%%%%%%%%%%%%%%%%%%%%%%%%%%%%%%%%%%%%%%%%%%%%%%%%%%%%%%%%%%%%%%%%%
\subsection{Conditional Processing}
\label{sec:conditional}

The package provides a mechanism to compile different versions
of a document. To customise the versions further some conditional processing
can come in handy to distinguish which version is being compiled.
The package provides two macros to describe the compilation context:

%%%%%%%%%%%%%%%%%%%%%%%%%%%%%%%%%%%%%%%%
\DescribeMacro{\ifchilddoc}
The conditional |\ifchilddoc| distinguishes between the compilation of
child documents and the main document:
%
\begin{center}
|\ifchilddoc |\textit{child-code}| |[|\||else |\textit{main-code}]| \||fi|
\end{center}

%%%%%%%%%%%%%%%%%%%%%%%%%%%%%%%%%%%%%%%%
\DescribeMacro{\childdocname}
\DescribeMacro{\childdocjob}
The macro |\childdocname| contains the filename (without extension)
of the main or child file being processed.
Note that |\childdocjob| will always contain the name of the main file.

%%%%%%%%%%%%%%%%%%%%%%%%%%%%%%%%%%%%%%%%
\paragraph{Title Page.}

Conditional processing can be used to include a title or banner page
in the main document when proper precautions are taken.
Importantly, the code in the main file should ensure that the page counter
(as well as other status parameters which are stored in the |.aux| files)
takes the same value after the conditional processing.
Otherwise the page numbers may take divergent values
depending on which part is compiled.

For example, a title page could be declared by:
%
\begin{center}
\begin{tabular}{l}
|\ifchilddoc\||else|\\
|\addtocounter{page}{-1}|\\
\textit{code for title page}\\
|\newpage|\\
|\||fi|
\end{tabular}
\end{center}
%
A banner page for the child documents can be generated by:
%
\begin{center}
\begin{tabular}{l}
|\ifchilddoc|\\
|\addtocounter{page}{-1}|\\
\textit{code for banner page}\\
|\newpage|\\
|\||fi|
\end{tabular}
\end{center}
%
Here one could write a message such as:
\begin{center}
|This is the part \childdocname{} of \childdocjob{}.|
\end{center}

%%%%%%%%%%%%%%%%%%%%%%%%%%%%%%%%%%%%%%%%%%%%%%%%%%%%%%%%%%%%%%%%%%%%%%%%%%%%%%%%
\subsection{Flags}
\label{sec:flags}

The package makes it easy to generate different versions
of the main or child documents.
To this end compilation flags can be defined
and assigned different default values.
They will be particularly useful in conjunction
with the forwarding mechanism described in \secref{sec:forward}.

For example, it may be useful to have a flag |\version|
which can be set to |draft| or |final|.
The document source will contain some conditional code
depending on the value of |\version|.
Suppose further, the flag should default to |final| for the main file
and to |draft| for child files
which is a natural assignment for editing the document.
This is achieved by placing the following code
in the preamble of the main document
(below the |\childdocmain| directive):
%
\begin{center}
\begin{tabular}{l}
|\ifchilddoc|\\
|\providecommand{\version}{draft}|\\
|\||else|\\
|\providecommand{\version}{final}|\\
|\||fi|
\end{tabular}
\end{center}
%
The definition by |\providecommand| makes sure
that previous definitions are not overwritten.
Further statements |\providecommand{\version}{...}|
can thus be added before the above code to override it.

For the main file, one might add a line
(between |\childdocmain| and the above block)
%
\begin{center}
|%\ifchilddoc\||else\providecommand{\version}{draft}\||fi|
\end{center}
%
which can be uncommented to produce a draft version.
Likewise one can add a line to the very top of a child file
(above the |\childdocof{|\textit{main}|}| directive)
%
\begin{center}
|%\providecommand{\version}{final}|
\end{center}
%
which can be uncommented to produce the final version of this child document.

%%%%%%%%%%%%%%%%%%%%%%%%%%%%%%%%%%%%%%%%%%%%%%%%%%%%%%%%%%%%%%%%%%%%%%%%%%%%%%%%
\subsection{Forwarding}
\label{sec:forward}

Different versions of the main or child documents
using compilation flags as described in \secref{sec:flags}
can be (permanently) stored in different files
for convenient compilation, viewing and distribution.
To this end, the package defines a command
to pass on compilation to a different file:

%%%%%%%%%%%%%%%%%%%%%%%%%%%%%%%%%%%%%%%%
\DescribeMacro{\childdocforward}
The command |\childdocforward| redirects processing to
another source file:
%
\begin{center}
\begin{tabular}{l}
|\input{childdoc.def}|\\
|\childdocforward[|\textit{main}|]{|\textit{dest}|}|\\
\end{tabular}
\end{center}
%
The argument \textit{dest} is the destination file
(without extension).
It should be the main file or one of the child files.
Note that further \textsf{childdoc} directives
such as |\childdocof| and |\childdocforward|
in the indicated file will be processed in this form.
The optional argument \textit{main}
passes on directly to the main file \textit{main}
while pretending to compile the child \textit{dest}.
This form behaves as if \textit{dest}
issues |\childdocof{|\textit{main}|}| right away,
and no further \textsf{childdoc} directives will be processed.

%%%%%%%%%%%%%%%%%%%%%%%%%%%%%%%%%%%%%%%%
\DescribeMacro{\...prefix}
In the alternative form |\childdocforwardprefix|,
%
\begin{center}
\begin{tabular}{l}
|\input{childdoc.def}|\\
|\childdocforwardprefix[|\textit{main}|]{|\textit{prefix}|}{|\textit{dest}|}|
\end{tabular}
\end{center}
%
the destination file is determined by a pattern
depending on the current file:
To make this work, the current file must be called
`{\textit{prefix}\hspace{0.2em}\textit{suffix}}'
with \textit{prefix} matching precisely the argument.
Processing is then passed on to the file
`{\textit{dest}\hspace{0.2em}\textit{suffix}}'.
Surely, the same effect is achieved by
directly specifying the
argument `{\textit{dest}\hspace{0.2em}\textit{suffix}}'
in the first form.
However, that requires to set up a different file
for each child. With the alternative form of the command
all these files can have exactly the same content
which simplifies setting them up and maintaining them.

For example, the following file |draft.tex|
with a compilation flag |\version| as described in \secref{sec:flags}
compiles the main document as a draft:
%
\begin{center}
\begin{tabular}{l}
|\def\version{draft}|\\
|\input{childdoc.def}|\\
|\childdocforward{|\textit{main}|}|
\end{tabular}
\end{center}
%
Likewise, the following files |final|\textit{nn}|.tex|
compile the final version of the child document
|child|\textit{nn}|.tex|:
%
\begin{center}
\begin{tabular}{l}
|\def\version{final}|\\
|\input{childdoc.def}|\\
|\childdocforwardprefix{final}{child}|
\end{tabular}
\end{center}
%

Note that when several versions of a main file and/or of each child file
are to be generated, it may be convenient to set up a |Makefile| or
shell script to automatise the process.

%%%%%%%%%%%%%%%%%%%%%%%%%%%%%%%%%%%%%%%%%%%%%%%%%%%%%%%%%%%%%%%%%%%%%%%%%%%%%%%%
\subsection{Command Line Processing}
\label{sec:commandline}

The effect of redirection files can also be achieved by invoking
the \LaTeX{} compiler with a more elaborate command line.
Most conveniently this should be done as part
of a shell script or a |Makefile|.

When using \textsf{childdoc} in the main file, the following
command lines effectively perform a redirection
(note that depending on the shell being used,
backslashes may have to be doubled: `|\|' $\to$ `|\\|'):
%
\begin{center}
|... -jobname "|\textit{target}|" |\\|"|[\textit{flags}]%
|\input{childdoc.def}\childdocforward[|\textit{main}|]{|\textit{dest}|}"|
\end{center}
%
Here \textit{target} is the name of the output file,
\textit{main} is the name of the main file
and \textit{dest} is the name of the main or child file to be processed
(all filenames without extensions).
The optional argument \textit{main} can be omitted
if \textit{main} matches \textit{dest}.
Optionally, compilation \textit{flags} can be defined via |\def| commands.
This command line makes the \TeX{} engine believe
it is compiling the file \textit{target}
whose content is specified as the latter parameter.
The provided code then forwards the processing to
\textit{main} or \textit{dest} as described in \secref{sec:forward}.

%%%%%%%%%%%%%%%%%%%%%%%%%%%%%%%%%%%%%%%%%%%%%%%%%%%%%%%%%%%%%%%%%%%%%%%%%%%%%%%%
\subsection{Include by Input}
\label{sec:input}

Including child documents by |\include| has some restrictions by design.
Most notably, the content of a child document always occupies
its own set of pages; pages cannot be shared between child documents.
Usually, this behaviour makes perfect sense
because each child document contain an essential part of the document.
However, in some situations it may be desirable to compose
a document from a collection of parts
without having mandatory page breaks between then.
For this case, the package
provides a mechanism to include parts
by |\input| which can also be processed individually.
However, by construction this mechanism
requires manual handling of the content to be output.

%%%%%%%%%%%%%%%%%%%%%%%%%%%%%%%%%%%%%%%%
\DescribeMacro{\ifchilddocmanual}
The main file should be prepared as usual, see \secref{sec:include}.
However, the document body must make a distinction
between processing of an individual part and of the main document, e.g.:
%
\begin{center}
\begin{tabular}{l}
|\ifchilddocmanual|\\
|\input{\childdocname}|\\
|\||else|\\
\textit{document body with }|\input{|\textit{part}|}|\\
|\||fi|
\end{tabular}
\end{center}
%
The conditional |\ifchilddocmanual| is true whenever
a part to be included by |\input| is being compiled,
and the name of the part is stored in |\childdocname|.

%%%%%%%%%%%%%%%%%%%%%%%%%%%%%%%%%%%%%%%%
\DescribeMacro{\childdocby}
Each part to be included by |\input| should start with:
%
\begin{center}
\begin{tabular}{l}
|\input{childdoc.def}|\\
|\childdocby{|\textit{main}|}|\\
\end{tabular}
\end{center}
%
The directive |\childdocby| is similar to |\childdocof|
described in \secref{sec:include},
but the subsequent selection of content must be done manually.
To that end, both |\ifchilddoc| and |\ifchilddocmanual|
will be true upon processing of a part,
and the name of the part is stored in |\childdocname|.
Note that |\jobname| will be set to the filename of the current part
so that each part receives an individual |.aux| file
that does not interfere with the |.aux| file(s) of the main document.
This behaviour can be altered by the alternative form
|\childdocby[*]{|\textit{main}|}| (with a non-empty optional argument)
which uses the |.aux| file of the main document
by setting |\jobname| to \textit{main}.

%%%%%%%%%%%%%%%%%%%%%%%%%%%%%%%%%%%%%%%%%%%%%%%%%%%%%%%%%%%%%%%%%%%%%%%%%%%%%%%%
\subsection{Driver Development}
\label{sec:driver}

The \textsf{childdoc} mechanism can also be use for the development
of definition files such as \LaTeX{} styles or classes.
This case differs from the above setup with multiple parts
included by |\include| in that no |\includeonly| should be invoked.
This can be achieved by starting the include file
(before |\ProvidesPackage|) with:
%
\begin{center}
\begin{tabular}{l}
|\input{childdoc.def}|\\
|\childdocforward{|\textit{main}|}|\\
\end{tabular}
\end{center}
%
or alternatively with:
%
\begin{center}
\begin{tabular}{l}
|\input{childdoc.def}|\\
|\childdocby{|\textit{main}|}|\\
\end{tabular}
\end{center}
%
Both forms have slightly different effects as described above.
The main file is prepared as usual, see \secref{sec:include}.

%%%%%%%%%%%%%%%%%%%%%%%%%%%%%%%%%%%%%%%%%%%%%%%%%%%%%%%%%%%%%%%%%%%%%%%%%%%%%%%%
\subsection{Legacy Detection}
\label{sec:detection}

The directive |\childdocmain| in the main file can detect
whether the complete document or merely a child is to be compiled
even without using the directive |\childdocof|.
This method is deprecated because it is less robust
and there is no compelling reason to use it;
it is merely provided for backward compatibility
and it may be removed in future versions.

If the detection mechanism is to be used,
it is mandatory to correctly specify
the filename of the main file as the argument of |\childdocmain|:
%
\begin{center}
\begin{tabular}{l}
|\input{childdoc.def}|\\
|\childdocmain{|\textit{main}|}|\\
\end{tabular}
\end{center}
%
If |\jobname| does not match the argument \textit{main} of |\childdocmain|,
it is assumed that |\jobname| points to the child file to be compiled.
When using |\childdocmain| with the main file specified as argument,
it suffices to start a child file
with just |\input{|\textit{main}|}|
without loading of the package and using |\childdocof|.
If instead all processing is done
with the appropriate \textsf{childdoc} directives,
the argument of \textit{main} of |\childdocmain| can be empty.

An alternative version of the command line processing described
in \secref{sec:commandline} using the detection mechanism reads:
%
\begin{center}
|... -jobname "|\textit{target}|" "|[\textit{flags}]%
[|\def\jobname{|\textit{dest}|}|]|\input{|\textit{main}|}"|
\end{center}

%%%%%%%%%%%%%%%%%%%%%%%%%%%%%%%%%%%%%%%%%%%%%%%%%%%%%%%%%%%%%%%%%%%%%%%%%%%%%%%%
\subsection{Manual Code}
\label{sec:manual}

In case one cannot be certain whether the definitions file |childdoc.def|
is installed on the target \TeX{} distribution
and one prefers not to ship it,
it is conceivable to paste a few relevant commands into the sources.

To that end, drop all statements |\input{childdoc.def}|
and perform the replacements as outlined below.
Instead of |\childdocmain{|\textit{main}|}| add the following code
to the top of the main file:
%
\begin{center}
\begin{tabular}{l}
|\||ifdefined\childdocname\endinput\||fi\newif\ifchilddoc|\\
|\edef\childdocname{\scantokens\expandafter{\jobname\noexpand}}|\\
|\def\childdocmain{|\textit{main}|}\||ifx\childdocmain\childdocname\||else|\\
|\childdoctrue\includeonly{\childdocname}\let\jobname\childdocmain\||fi|\\
\end{tabular}
\end{center}
%
Instead of |\childdocof{|\textit{main}|}| just include the main file
at the top of each child file:
%
\begin{center}
|\input{|\textit{main}|}|
\end{center}
%
A simple redirection |\childdocforward{|\textit{dest}|}| is achieved by:
%
\begin{center}
|\def\jobname{|\textit{dest}|}\input{\jobname}|
\end{center}
%
The redirection with prefix
|\childdocforwardprefix[|\textit{prefix}|]{|\textit{dest}|}|
is accomplished by:
%
\begin{center}
\begin{tabular}{l}
|{\edef\jobname{\scantokens\expandafter{\jobname\noexpand}}|\\
|\def\redirectjob |\textit{prefix}|#1~~~{\gdef\jobname{|\textit{dest}|#1}}|\\
|\expandafter\redirectjob\jobname~~~}\input{\jobname}|
\end{tabular}
\end{center}

In an alternative approach,
child documents can be compiled by a specific command line
without additional code or specific definitions:
%
\begin{center}
|... -jobname "|\textit{target}|" "|[\textit{flags}]%
|\includeonly{|\textit{dest}|}\input{|\textit{main}|}"|
\end{center}
%

%%%%%%%%%%%%%%%%%%%%%%%%%%%%%%%%%%%%%%%%%%%%%%%%%%%%%%%%%%%%%%%%%%%%%%%%%%%%%%%%
%%%%%%%%%%%%%%%%%%%%%%%%%%%%%%%%%%%%%%%%%%%%%%%%%%%%%%%%%%%%%%%%%%%%%%%%%%%%%%%%
\section{Information}

%%%%%%%%%%%%%%%%%%%%%%%%%%%%%%%%%%%%%%%%%%%%%%%%%%%%%%%%%%%%%%%%%%%%%%%%%%%%%%%%
\subsection{Copyright}

Copyright \copyright{} 2017--2018 Niklas Beisert

This work may be distributed and/or modified under the
conditions of the \LaTeX{} Project Public License, either version 1.3
of this license or (at your option) any later version.
The latest version of this license is in
  \url{http://www.latex-project.org/lppl.txt}
and version 1.3 or later is part of all distributions of \LaTeX{}
version 2005/12/01 or later.

This work has the LPPL maintenance status `maintained'.

The Current Maintainer of this work is Niklas Beisert.

This work consists of the files |README.txt|, |childdoc.ins| and |childdoc.dtx|
as well as the derived files |childdoc.def|, |cdocsamp.tex|
with |cdocsch1.tex|, |cdocsch2.tex|, |cdocspt3.tex|, |cdocspt4.tex|,
|cdocsdrf.tex|, |cdocsfn1.tex|, |cdocsfn2.tex|
as well as |childdoc.pdf|.

%%%%%%%%%%%%%%%%%%%%%%%%%%%%%%%%%%%%%%%%%%%%%%%%%%%%%%%%%%%%%%%%%%%%%%%%%%%%%%%%
\subsection{Files and Installation}

The package consists of the files:
%
\begin{center}
\begin{tabular}{ll}
    |README.txt|   & readme file \\
    |childdoc.ins| & installation file \\
    |childdoc.dtx| & source file \\
    |childdoc.def| & definition file \\
    |cdocsamp.tex| & sample main file \\
    |cdocsch1.tex| & sample include file \\
    |cdocsch2.tex| & sample include file \\
    |cdocspt3.tex| & sample part file \\
    |cdocspt4.tex| & sample part file \\
    |cdocsdrf.tex| & sample redirection file \\
    |cdocsfn1.tex| & sample redirection file \\
    |cdocsfn2.tex| & sample redirection file \\
    |childdoc.pdf| & manual
\end{tabular}
\end{center}
%
The distribution consists of the files
|README.txt|, |childdoc.ins| and |childdoc.dtx|.
%
\begin{itemize}
\item
Run (pdf)\LaTeX{} on |childdoc.dtx|
to compile the manual |childdoc.pdf| (this file).
\item
Run \LaTeX{} on |childdoc.ins| to create the definitions file |childdoc.def|
and the sample |cdocsamp.tex| with include files
|cdocsch1.tex|, |cdocsch2.tex|, |cdocspt3.tex|, |cdocspt4.tex|,
|cdocsdrf.tex|, |cdocsfn1.tex|, |cdocsfn2.tex|.
Then copy the file |childdoc.def| to an appropriate directory of your \LaTeX{}
distribution, e.g.\ \textit{texmf-root}|/tex/latex/childdoc|.
\end{itemize}

%%%%%%%%%%%%%%%%%%%%%%%%%%%%%%%%%%%%%%%%%%%%%%%%%%%%%%%%%%%%%%%%%%%%%%%%%%%%%%%%
\subsection{Related CTAN Packages}

There are several other packages which offer a similar functionality:
%
\begin{itemize}
\item
The packages
\href{http://ctan.org/pkg/docmute}{\textsf{docmute}},
\href{http://ctan.org/pkg/includex}{\textsf{includex}} and
\href{http://ctan.org/pkg/standalone}{\textsf{standalone}}
provide commands to include only the document body of
a child file thus allowing both files to be compiled individually.
\item
The packages \href{http://ctan.org/pkg/subdocs}{\textsf{subdocs}}
and \href{http://ctan.org/pkg/subfiles}{\textsf{subfiles}}
provide structures in which the main and child documents can be
encapsulated and allowing them to be compiled individually.
The inclusion mechanism is different from the conventional |\include|.
\item
The package \href{http://ctan.org/pkg/combine}{\textsf{combine}}
is an elaborate solution to combine several documents into one.
\end{itemize}
%
See also the CTAN topic \href{http://ctan.org/topic/subdocs}{\textsf{subdocs}}
for further related packages.
The present package differs from the above solutions in that
a document structure constructed with the conventional |\include| mechanism
just needs two extra commands at the top of every file
such that all constituent files can be compiled individually.

%%%%%%%%%%%%%%%%%%%%%%%%%%%%%%%%%%%%%%%%%%%%%%%%%%%%%%%%%%%%%%%%%%%%%%%%%%%%%%%%
%\subsection{Feature Suggestions}
%
%The following is a list of features which may be useful for future
%versions of this package:
%%
%\begin{itemize}
%\item
%\ldots
%\end{itemize}

%%%%%%%%%%%%%%%%%%%%%%%%%%%%%%%%%%%%%%%%%%%%%%%%%%%%%%%%%%%%%%%%%%%%%%%%%%%%%%%%
\subsection{Revision History}

%%%%%%%%%%%%%%%%%%%%%%%%%%%%%%%%%%%%%%%%
\paragraph{v2.0:} 2018/12/30

\begin{itemize}
\item
immediate forward processing
\item
added |\childdocby| mechanism
\item
manual restructured
\end{itemize}

%%%%%%%%%%%%%%%%%%%%%%%%%%%%%%%%%%%%%%%%
\paragraph{v1.6:} 2018/01/17

\begin{itemize}
\item
application for development of include files
\item
corrections to manual
\end{itemize}

%%%%%%%%%%%%%%%%%%%%%%%%%%%%%%%%%%%%%%%%
\paragraph{v1.5:} 2017/05/21

\begin{itemize}
\item
more complete structuring introduced
\item
|\childdocof| introduced
\item
|\childdoc| renamed to |\childdocmain|
\item
|\childredirect| renamed to |\childdocforward| and |\childdocforwardprefix|
and functionality expanded
\end{itemize}

%%%%%%%%%%%%%%%%%%%%%%%%%%%%%%%%%%%%%%%%
\paragraph{v1.0:} 2017/04/27

\begin{itemize}
\item
manual and install package
\item
first version published on CTAN
\end{itemize}

%%%%%%%%%%%%%%%%%%%%%%%%%%%%%%%%%%%%%%%%
\paragraph{v0.6:} 2017/04/26

\begin{itemize}
\item
redirection mechanism added
\end{itemize}

%%%%%%%%%%%%%%%%%%%%%%%%%%%%%%%%%%%%%%%%
\paragraph{v0.5:} 2017/04/26

\begin{itemize}
\item
functionality in definition file
\end{itemize}


%%%%%%%%%%%%%%%%%%%%%%%%%%%%%%%%%%%%%%%%%%%%%%%%%%%%%%%%%%%%%%%%%%%%%%%%%%%%%%%%
%%%%%%%%%%%%%%%%%%%%%%%%%%%%%%%%%%%%%%%%%%%%%%%%%%%%%%%%%%%%%%%%%%%%%%%%%%%%%%%%
%%%%%%%%%%%%%%%%%%%%%%%%%%%%%%%%%%%%%%%%%%%%%%%%%%%%%%%%%%%%%%%%%%%%%%%%%%%%%%%%
\appendix

\settowidth\MacroIndent{\rmfamily\scriptsize 000\ }

 \DocInput{childdoc.dtx}

\end{document}
%</driver>
% \fi
%
% %%%%%%%%%%%%%%%%%%%%%%%%%%%%%%%%%%%%%%%%%%%%%%%%%%%%%%%%%%%%%%%%%%%%%%%%%%%%%%
% %%%%%%%%%%%%%%%%%%%%%%%%%%%%%%%%%%%%%%%%%%%%%%%%%%%%%%%%%%%%%%%%%%%%%%%%%%%%%%
% \section{Sample}
%\iffalse
%<*samplemain>
%\fi
%
% The following presents a sample document
% with two chapters, two parts, a title page,
% a compile flag as well as three forwarding files to set the flag.
% It consists of eight |.tex| files:
% \begin{center}
% \begin{tabular}{ll}
% |cdocsamp.tex|&main file\\
% |cdocsch1.tex|&include file for chapter 1\\
% |cdocsch2.tex|&include file for chapter 2\\
% |cdocspt3.tex|&include file for part 3\\
% |cdocspt4.tex|&include file for part 4\\
% |cdocsdrf.tex|&forwarding file for main file in draft mode\\
% |cdocsfi1.tex|&forwarding file for final version of chapter 1\\
% |cdocsfi2.tex|&forwarding file for final version of chapter 2\\
% \end{tabular}
% \end{center}
% Each of the eight files can be compiled directly by the \LaTeX{} compiler.
%
% %%%%%%%%%%%%%%%%%%%%%%%%%%%%%%%%%%%%%%
% \paragraph{Main File.}
%
% The main file is called |cdocsamp.tex|.
%
% Load the \textsf{childdoc} definitions and
% declare the filename for the main document:
%    \begin{macrocode}
\input{childdoc.def}
\childdocmain{}
%    \end{macrocode}

% Optional override for |\version| flag:
%    \begin{macrocode}
%%\ifchilddoc\else\providecommand{\version}{draft}\fi
%    \end{macrocode}

% Define the default values for the |\version| flag
% (|final| for the main file and |draft| for childs):
%    \begin{macrocode}
\ifchilddoc
\providecommand{\version}{draft}
\else
\providecommand{\version}{final}
\fi
%    \end{macrocode}

% Load the standard document class:
%    \begin{macrocode}
\documentclass[12pt]{article}
%    \end{macrocode}

% Start the document body:
%    \begin{macrocode}
\begin{document}
%    \end{macrocode}

% Declare a title page.
% Print title, part of document being processed and version flag:
%    \begin{macrocode}
\addtocounter{page}{-1}
\begin{center}
{\LARGE\bfseries{}childdoc example\par}
\vspace{1cm}
\ifchilddoc
\ifchilddocmanual part\else chapter\fi:
`\childdocname' of `\childdocjob'\par
\else
main document: `\childdocjob'\par
\fi
version: \version\par
\end{center}
\newpage
%    \end{macrocode}

% Manually include selected file,
% otherwise process as usual:
%    \begin{macrocode}
\ifchilddocmanual
\section*{part `\childdocname'}
\input{\childdocname}
\else
%    \end{macrocode}

% Include the two chapters:
%    \begin{macrocode}
\include{cdocsch1}
\include{cdocsch2}
%    \end{macrocode}

% Include the two parts unless only chapters should be displayed:
%    \begin{macrocode}
\ifchilddoc\else
\section{part three}
\input{cdocspt3}
\section{part four}
\input{cdocspt4}
\fi
%    \end{macrocode}

% Process as usual until here:
%    \begin{macrocode}
\fi
%    \end{macrocode}

% End of document body:
%    \begin{macrocode}
\end{document}
%    \end{macrocode}
%\iffalse
%</samplemain>
%\fi
%
% %%%%%%%%%%%%%%%%%%%%%%%%%%%%%%%%%%%%%%
% \paragraph{Chapter Include Files.}
%
% The include files are called |cdocsch1.tex| and |cdocsch2.tex|.
%
%\iffalse
%<*samplechap1|samplechap2>
%\fi

% Optional override for |\version| flag:
%    \begin{macrocode}
%%\providecommand{\version}{final}
%    \end{macrocode}

% Include the main document:
%    \begin{macrocode}
\input{childdoc.def}
\childdocof{cdocsamp}
%    \end{macrocode}

%\iffalse
%</samplechap1|samplechap2>
%\fi
%
%\iffalse
%<*samplechap1>
%\fi
% Some text for chapter 1:
%    \begin{macrocode}
\section{one}
some text in chapter one
%    \end{macrocode}

%\iffalse
%</samplechap1>
%\fi
% Some text for chapter 2:
%\iffalse
%<*samplechap2>
%\fi
%    \begin{macrocode}
\section{two}
more text in chapter two
%    \end{macrocode}

%\iffalse
%</samplechap2>
%\fi
%
% %%%%%%%%%%%%%%%%%%%%%%%%%%%%%%%%%%%%%%
% \paragraph{Part Include Files.}
%
% The include files are called |cdocspt3.tex| and |cdocspt4.tex|.
%
%\iffalse
%<*samplepart3|samplepart4>
%\fi

% Optional override for |\version| flag:
%    \begin{macrocode}
%%\providecommand{\version}{final}
%    \end{macrocode}

% Include the main document:
%    \begin{macrocode}
\input{childdoc.def}
\childdocby{cdocsamp}
%    \end{macrocode}

%\iffalse
%</samplepart3|samplepart4>
%\fi
%
%\iffalse
%<*samplepart3>
%\fi
% Some text for part 3:
%    \begin{macrocode}
some text in part three
%    \end{macrocode}

%\iffalse
%</samplepart3>
%\fi
% Some text for part 4:
%\iffalse
%<*samplepart4>
%\fi
%    \begin{macrocode}
more text in part four
%    \end{macrocode}

%\iffalse
%</samplepart4>
%\fi
%
% %%%%%%%%%%%%%%%%%%%%%%%%%%%%%%%%%%%%%%
% \paragraph{Forwarding for a Complete Draft.}
%
% The following forwarding file |cdocsdrf.tex|
% compiles the main document in draft mode:
%\iffalse
%<*sampledraft>
%\fi
%    \begin{macrocode}
\def\version{draft}
\input{childdoc.def}
\childdocforward{cdocsamp}
%    \end{macrocode}

%\iffalse
%</sampledraft>
%\fi
%
% %%%%%%%%%%%%%%%%%%%%%%%%%%%%%%%%%%%%%%
% \paragraph{Forwarding for Final Version of the Chapters.}
%
% The following forwarding files |cdocsfn1.tex| and |cdocsfn2.tex|
% (with identical content)
% compile the final versions of the child documents
% |cdocsch1.tex| and |cdocsch2.tex|, respectively:
%\iffalse
%<*samplefinal>
%\fi
%    \begin{macrocode}
\def\version{final}
\input{childdoc.def}
\childdocforwardprefix[cdocsamp]{cdocsfn}{cdocsch}
%    \end{macrocode}

%\iffalse
%</samplefinal>
%\fi
%
% %%%%%%%%%%%%%%%%%%%%%%%%%%%%%%%%%%%%%%
% \paragraph{Command Line Processing.}
%
% The following three command lines generate the output files
% |cdocscld|, |cdocscl1| and |cdocscl2|
% which should be identical to
% |cdocsdrf|, |cdocsch1| and |cdocsfn2|, respectively:
% \begin{center}
% \begin{tabular}{l}
% |latex -jobname cdocscld \|\\
% |  "\def\version{draft}\input{childdoc.def}\childdocforward{cdocsamp}"|\\
% |latex -jobname cdocscl1 \|\\
% |  "\input{childdoc.def}\childdocforward[cdocsamp]{cdocsch1}"|\\
% |latex -jobname cdocscl2 \|\\
% |  "\def\version{final}\input{childdoc.def}\childdocforward{cdocsch2}"|
% \end{tabular}
% \end{center}
% Note that the trailing backslash on each first line
% merely continues the input to the second line
% (for convenient cut ant paste).
% Furthermore, the command |latex| can be replaced by any
% of its alternative versions such as |pdflatex|.
%
% %%%%%%%%%%%%%%%%%%%%%%%%%%%%%%%%%%%%%%%%%%%%%%%%%%%%%%%%%%%%%%%%%%%%%%%%%%%%%%
% %%%%%%%%%%%%%%%%%%%%%%%%%%%%%%%%%%%%%%%%%%%%%%%%%%%%%%%%%%%%%%%%%%%%%%%%%%%%%%
% \section{Implementation}
%\iffalse
%<*package>
%\fi
%
% This section describes the definitions file |childdoc.def|.

% The definitions cannot be loaded using |\usepackage| or |\RequirePackage|
% which has a mechanism to prevent loading a style file more than once.
% When loading the definitions by means of |\input|
% multiple instances have to be prevented manually:
%\iffalse
%This code needs to be before the `\ProvidesFile' directive
%which is defined at the beginning of this file.
%Therefore it is also placed there and commented out here.
%</package>
%<*discard>
%\fi
%    \begin{macrocode}
\ifdefined\childdocmain\endinput\fi
%    \end{macrocode}
%\iffalse
%</discard>
%<*package>
%\fi
%
% \macro{\ifchilddoc}
% \macro{\ifchilddocmanual}
% The conditional |\ifchilddoc| tells whether a
% child (true) or main (false) document is being compiled.
% The conditional |\ifchilddocmanual| tells whether
% the |\includeonly| mechanism is used (false) or
% the selection of child files must be performed manually (true).
% The definitions initialise to false:
%    \begin{macrocode}
\newif\ifchilddoc
\newif\ifchilddocmanual
%    \end{macrocode}

% \macro{\childdocname}
% \macro{\childdocjob}
% The macro |\childdocname| stores the name of the main document
% to be compiled. The macro |\childdocjob| stores the name of
% the document on which the \LaTeX{} compiler was originally invoked.
% The content of |\jobname| cannot be compared
% to filenames specified in the source due to different catcodes.
% The following code rescans |\jobname|, stores the result
% in |\childdocname| and saves a copy in |\childdocjob|:
%    \begin{macrocode}
\edef\childdocname{\scantokens\expandafter{\jobname\noexpand}}
\let\childdocjob\childdocname
%    \end{macrocode}

% \macro{\childdocdisable}
% The macro |\childdocdisable| prevents the main file
% from being processed more than once.
% At this stage, the main document command |\childdocmain|
% is assumed to be called once again where it should do nothing.
% Any subsequent call to it should prevent
% a secondary processing of the main document
% It overwrites the forwarding commands
% |\childdocof| and |\childdocforward|
% with empty macros to prevent further inclusions of the main document:
%    \begin{macrocode}
\newcommand{\childdocdisable}
{
  \renewcommand{\childdocmain}[1]{\renewcommand{\childdocmain}[1]{\endinput}}
  \renewcommand{\childdocof}[1]{}
  \renewcommand{\childdocby}[2][]{}
  \renewcommand{\childdocforward}[2][]{}
  \renewcommand{\childdocdisable}{}
}
%    \end{macrocode}

% \macro{\childdocmain}
% The macro |\childdocmain| is to be called at the top of the main file
% with nothing or the main filename (without extension) as argument.
% First, it breaks loops.
% If the argument is not empty and does not match |\childdocname|
% (which is set by the first inclusion of |childdoc.def|),
% |\ifchilddoc| is set to true, |\includeonly| is applied to the child file
% and |\jobname| is set to the main file
% (for proper handling of |.aux| files):
%    \begin{macrocode}
\newcommand{\childdocmain}[1]
{
  \childdocdisable\childdocmain{}
  \if?#1?\else
    \begingroup
      \def\childdoctmp{#1}
      \ifx\childdoctmp\childdocname
        \def\childdoctmp{}
      \else
        \def\childdoctmp
        {
          \childdoctrue
          \includeonly{\childdocname}
          \def\childdocjob{#1}
          \def\jobname{#1}
        }
      \fi
      \expandafter
    \endgroup
    \childdoctmp
  \fi
}
%    \end{macrocode}

% \macro{\childdocof}
% The command |\childdocof| redirects
% compilation to the main file |#1|.
%    \begin{macrocode}
\newcommand{\childdocof}[1]
{
  \childdocdisable
  \childdoctrue
  \includeonly{\childdocname}
  \def\jobname{#1}
  \def\childdocjob{#1}
  \input{#1}
}
%    \end{macrocode}

% \macro{\childdocby}
% The command |\childdocby| ....
%    \begin{macrocode}
\newcommand{\childdocby}[2][]
{
  \childdocdisable
  \childdoctrue
  \childdocmanualtrue
  \if?#1?\else
    \def\jobname{#2}
  \fi
  \def\childdocjob{#2}
  \input{#2}
  \endinput
}
%    \end{macrocode}

% \macro{\childdocforward}
% The command |\childdocforward| redirects
% compilation to the main file or
% (if the optional argument is given) a child file.
% Parameters are set as if the main file
% or a child file starting with |\childdocof| was compiled.
% Then compilation is handed over to the main file:
%    \begin{macrocode}
\newcommand{\childdocforward}[2][]
{
  \begingroup
    \if?#1?
      \def\childdoctmp
      {
        \def\childdocname{#2}
        \def\childdocjob{#2}
        \def\jobname{#2}
        \input{#2}
        \endinput
      }
    \else
      \def\childdoctmp
      {
        \childdocdisable
        \def\childdocname{#2}
        \childdoctrue
        \includeonly{#2}
        \def\childdocjob{#1}
        \def\jobname{#1}
        \input{#1}
        \endinput
      }
    \fi
    \expandafter
  \endgroup
  \childdoctmp
}
%    \end{macrocode}

% \macro{\childdocforwardprefix}
% The command |\childdocforwardprefix| redirects
% compilation to the main or a child file by means of a pattern.
% The prefix |#1| in the current filename is replaced by |#2|
% and the suffix of the current filename is kept
% (it is assumed that the filename does not contain the substring `|~~~|'
% which is used as a delimiter).
% Compilation is handed over to the new file by |\childdocforward|:
%    \begin{macrocode}
\newcommand{\childdocforwardprefix}[3][]
{
  \begingroup
    \def\childdocextract #2##1~~~{\def\childdoctmp{\childdocforward[#1]{#3##1}}}
    \expandafter\childdocextract\childdocname~~~
    \expandafter
  \endgroup
  \childdoctmp
}
%    \end{macrocode}

% \macro{\childdoc}
% The deprecated macro |\childdoc| is a legacy version of |\childdocmain|:
%    \begin{macrocode}
\newcommand{\childdoc}{\childdocmain}
%    \end{macrocode}

% \macro{\childdocredirect}
% The deprecated macro |\childdocredirect| is a legacy version
% of |\childdocforward| and |\childdocforwardprefix|:
%    \begin{macrocode}
\newcommand{\childdocredirect}[2][]
{
  \begingroup
    \if?#1?
      \def\childdoctmp{\childdocforward{#2}}
    \else
      \def\childdoctmp{\childdocforwardprefix{#1}{#2}}
    \fi
    \expandafter
  \endgroup
  \childdoctmp
}
%    \end{macrocode}

%\iffalse
%</package>
%\fi
%
\endinput

\childdocforwardprefix[cdocsamp]{cdocsfn}{cdocsch}
%    \end{macrocode}

%\iffalse
%</samplefinal>
%\fi
%
% %%%%%%%%%%%%%%%%%%%%%%%%%%%%%%%%%%%%%%
% \paragraph{Command Line Processing.}
%
% The following three command lines generate the output files
% |cdocscld|, |cdocscl1| and |cdocscl2|
% which should be identical to
% |cdocsdrf|, |cdocsch1| and |cdocsfn2|, respectively:
% \begin{center}
% \begin{tabular}{l}
% |latex -jobname cdocscld \|\\
% |  "\def\version{draft}% \iffalse
%
% childdoc.dtx Copyright (C) 2017-2018 Niklas Beisert
%
% This work may be distributed and/or modified under the
% conditions of the LaTeX Project Public License, either version 1.3
% of this license or (at your option) any later version.
% The latest version of this license is in
%   http://www.latex-project.org/lppl.txt
% and version 1.3 or later is part of all distributions of LaTeX
% version 2005/12/01 or later.
%
% This work has the LPPL maintenance status `maintained'.
%
% The Current Maintainer of this work is Niklas Beisert.
%
% This work consists of the files childdoc.dtx and childdoc.ins
% and the derived files childdoc.def and cdocsamp.tex with
% cdocsch1.tex, cdocsch2.tex, cdocsdrf.tex, cdocsfn1.tex, cdocsfn2.tex.
%
%<package>\ifdefined\childdocmain\endinput\fi
%<package>\ProvidesFile{childdoc.def}[2018/12/30 v2.0 child document driver]
%<samplemain>\ProvidesFile{cdocsamp.tex}[2018/12/30 v2.0 sample for childdoc]
%<*driver>
%\ProvidesFile{childdoc.drv}[2018/12/30 v2.0 childdoc reference manual file]
\PassOptionsToClass{10pt,a4paper}{article}
\documentclass{ltxdoc}

\usepackage[margin=35mm]{geometry}
\usepackage{hyperref}
\usepackage{hyperxmp}
\usepackage[usenames]{color}

\hypersetup{colorlinks=true}
\hypersetup{pdfstartview=FitH}
\hypersetup{pdfpagemode=UseNone}
\hypersetup{pdfsource={}}
\hypersetup{pdflang={en-UK}}
\hypersetup{pdfcopyright={Copyright 2017-2018 Niklas Beisert.
  This work may be distributed and/or modified under the
  conditions of the LaTeX Project Public License, either version 1.3
  of this license or (at your option) any later version.}}
\hypersetup{pdflicenseurl={http://www.latex-project.org/lppl.txt}}
\hypersetup{pdfcontactaddress={ETH Zurich, ITP, HIT K,
  Wolfgang-Pauli-Strasse 27}}
\hypersetup{pdfcontactpostcode={8093}}
\hypersetup{pdfcontactcity={Zurich}}
\hypersetup{pdfcontactcountry={Switzerland}}
\hypersetup{pdfcontactemail={nbeisert@itp.phys.ethz.ch}}
\hypersetup{pdfcontacturl={http://people.phys.ethz.ch/\xmptilde nbeisert/}}

\newcommand{\secref}[1]{\hyperref[#1]{section \ref*{#1}}}

\parskip1ex
\parindent0pt
\let\olditemize\itemize
\def\itemize{\olditemize\parskip0pt}

\begin{document}

\title{The \textsf{childdoc} Package}
\hypersetup{pdftitle={The childdoc Package}}
\author{Niklas Beisert\\[2ex]
  Institut f\"ur Theoretische Physik\\
  Eidgen\"ossische Technische Hochschule Z\"urich\\
  Wolfgang-Pauli-Strasse 27, 8093 Z\"urich, Switzerland\\[1ex]
  \href{mailto:nbeisert@itp.phys.ethz.ch}
  {\texttt{nbeisert@itp.phys.ethz.ch}}}
\hypersetup{pdfauthor={Niklas Beisert}}
\hypersetup{pdfsubject={Manual for the LaTeX2e Package childdoc}}
\date{30 December 2018, \textsf{v2.0}}
\maketitle

\begin{abstract}\noindent
\textsf{childdoc} is a \LaTeXe{} package
that enables the direct compilation
of document sections included by |\include|
to individual files.
\end{abstract}

\begingroup
\parskip0ex
\tableofcontents
\endgroup

%%%%%%%%%%%%%%%%%%%%%%%%%%%%%%%%%%%%%%%%%%%%%%%%%%%%%%%%%%%%%%%%%%%%%%%%%%%%%%%%
%%%%%%%%%%%%%%%%%%%%%%%%%%%%%%%%%%%%%%%%%%%%%%%%%%%%%%%%%%%%%%%%%%%%%%%%%%%%%%%%
\section{Introduction}

\LaTeX{} provides a mechanism to structure a large document (such as a book)
into a main file and several child files (containing the chapters)
using the |\include| command.
This mechanism is beneficial for documents
which span hundreds of pages in order to
make the source file(s) more manageable.
Moreover, compilation can be restricted to
selected child files by means of the |\includeonly| command.
The latter feature can be used to reduce the compilation time while editing
(this was significantly more useful in the earlier days of \LaTeX{})
or to generate a smaller document which is easier to navigate.
Another application of |\includeonly| is to generate
documents consisting of selected parts of the complete document.

However, there are a few drawbacks of the plain |\include| mechanism:
\begin{itemize}
\item
The child files cannot be compiled on their own,
they can only be compiled via the main file.
A naive editing environment
(such as a text editor with an option
to have the current file processed by \LaTeX)
may require one to switch to the main file before compiling;
attempting to compile the child file produces errors.
\item
The main file must be modified (each time)
to adjust the |\includeonly| command
to the present needs. This easily leaves the main file in a messy state.
\item
The generated document will always carry the filename
of the main document. This is inconvenient if
several child files are to be compiled and
to be kept for distribution.
\end{itemize}

The present package provides a simple interface
to make child files individually compilable by \LaTeX{}.
Compiling a child file then has the same effect as compiling
the main file with an |\includeonly| command
to select the appropriate child.
Moreover the generated document will carry the name of the child
rather than the main file.
This resolves all three above issues.

This feature is meant to make the editing of books,
thesis documents and lecture notes somewhat more convenient.
However, the package can also be used efficiently for
composing a series of documents (such as exercise sheets)
which are typically distributed individually.
It then assists the author in generating the individual documents
(potentially in different versions)
as well as a document containing the collected series.
Another application is in developing style files
or other kinds of included material
where compilation of the style file could redirect
to a sample or test file.

%%%%%%%%%%%%%%%%%%%%%%%%%%%%%%%%%%%%%%%%%%%%%%%%%%%%%%%%%%%%%%%%%%%%%%%%%%%%%%%%
%%%%%%%%%%%%%%%%%%%%%%%%%%%%%%%%%%%%%%%%%%%%%%%%%%%%%%%%%%%%%%%%%%%%%%%%%%%%%%%%
\section{Usage}

First of all, the package \textsf{childdoc} is \emph{not} a standard
\LaTeXe{} |.sty| style file! Therefore it needs to be invoked in
a non-standard way.

%%%%%%%%%%%%%%%%%%%%%%%%%%%%%%%%%%%%%%%%%%%%%%%%%%%%%%%%%%%%%%%%%%%%%%%%%%%%%%%%
\subsection{Included Files}
\label{sec:include}

%%%%%%%%%%%%%%%%%%%%%%%%%%%%%%%%%%%%%%%%
\DescribeMacro{\childdocmain}
To use the package, add the commands
\begin{center}
\begin{tabular}{l}
|\input{childdoc.def}|\\
|\childdocmain{}|\\
\end{tabular}
\end{center}
at the very top of the main \LaTeX{} file,
in particular \emph{before} the |\documentclass| statement!
The argument of |\childdocmain| should be left empty
(but it must be present).

%%%%%%%%%%%%%%%%%%%%%%%%%%%%%%%%%%%%%%%%
\DescribeMacro{\childdocof}
Furthermore, add the commands
\begin{center}
\begin{tabular}{l}
|\input{childdoc.def}|\\
|\childdocof{|\textit{main}|}|\\
\end{tabular}
\end{center}
at the top of every child file \textit{child}
which is included by |\include{|\textit{child}|}|
from within the main file
(or at least for those files to be compiled individually).
The argument \textit{main} must be the filename of the main file.

There are a couple of
considerations in setting up the main and child documents:

%%%%%%%%%%%%%%%%%%%%%%%%%%%%%%%%%%%%%%%%
\paragraph{Restrictions.}

Please note the following restrictions:
\begin{itemize}
\item
|\childdocmain| must be called with one argument \textit{main}
to ensure compatibility with earlier version of the package.
It must either be empty (|\childdocmain{}|)
or precisely match the filename of the main file in which it is specified.
See \secref{sec:detection} for further information.
\item
The filename \textit{main} must be specified without the |.tex| extension.
\item
The filename \textit{main} is case sensitive
(even in case-insensitive file systems)
due to internal string comparison.
\item
The argument \textit{main} should be fully expanded, it cannot be a macro.
\item
Subdirectories and special characters should be avoided in filenames.
\item
The command |\childdocmain{|\textit{main}|}| must be followed by a whitespace.
It should not be followed immediately by another command
or by a comment mark `|%|'.
This is because the \TeX{} parser reads the token immediately following
the argument of |\childdocmain| and puts it
at the beginning of every child section;
however, a white\-space is ignored.
\end{itemize}

%%%%%%%%%%%%%%%%%%%%%%%%%%%%%%%%%%%%%%%%
\paragraph{Content of Main File.}

It is advisable to place all content in the child files included by |\include|.
Any output contained in the main file will appear in all child documents
unless suppressed manually;
it cannot be suppressed automatically by the |\includeonly| directive
and thus should normally be avoided.
A method to include some content in the main file
by means of conditional processing is described in \secref{sec:conditional}.

%%%%%%%%%%%%%%%%%%%%%%%%%%%%%%%%%%%%%%%%
\paragraph{Page Numbering.}

When only a part of the document is compiled,
the appropriate numbering of pages
(as well as other status parameters)
is determined from the |.aux| files.
The latter contain information from previous passes.
However this information needs to propagate through
all intermediate child documents.
Therefore the page numbering in child documents may well
be inconsistent until the complete document is compiled at least once.

A useful (if unconventional) way to always ensure a consistent
page numbering is to restart the numbering in each child document
and denote the pages by `\textit{child}|.|\textit{page}'
where \textit{child} represents the chapter/section number of the child file.
This can be achieved by the command
|\numberwithin{page}{|\textit{child}|}|
of the \textsf{amsmath} package
where \textit{child} can be |chapter| or |section|
depending on the chosen structuring.
Alternatively, one can modify the macro |\thepage| appropriately
and reset the counter |page| at the start of each child file.

%%%%%%%%%%%%%%%%%%%%%%%%%%%%%%%%%%%%%%%%%%%%%%%%%%%%%%%%%%%%%%%%%%%%%%%%%%%%%%%%
\subsection{Conditional Processing}
\label{sec:conditional}

The package provides a mechanism to compile different versions
of a document. To customise the versions further some conditional processing
can come in handy to distinguish which version is being compiled.
The package provides two macros to describe the compilation context:

%%%%%%%%%%%%%%%%%%%%%%%%%%%%%%%%%%%%%%%%
\DescribeMacro{\ifchilddoc}
The conditional |\ifchilddoc| distinguishes between the compilation of
child documents and the main document:
%
\begin{center}
|\ifchilddoc |\textit{child-code}| |[|\||else |\textit{main-code}]| \||fi|
\end{center}

%%%%%%%%%%%%%%%%%%%%%%%%%%%%%%%%%%%%%%%%
\DescribeMacro{\childdocname}
\DescribeMacro{\childdocjob}
The macro |\childdocname| contains the filename (without extension)
of the main or child file being processed.
Note that |\childdocjob| will always contain the name of the main file.

%%%%%%%%%%%%%%%%%%%%%%%%%%%%%%%%%%%%%%%%
\paragraph{Title Page.}

Conditional processing can be used to include a title or banner page
in the main document when proper precautions are taken.
Importantly, the code in the main file should ensure that the page counter
(as well as other status parameters which are stored in the |.aux| files)
takes the same value after the conditional processing.
Otherwise the page numbers may take divergent values
depending on which part is compiled.

For example, a title page could be declared by:
%
\begin{center}
\begin{tabular}{l}
|\ifchilddoc\||else|\\
|\addtocounter{page}{-1}|\\
\textit{code for title page}\\
|\newpage|\\
|\||fi|
\end{tabular}
\end{center}
%
A banner page for the child documents can be generated by:
%
\begin{center}
\begin{tabular}{l}
|\ifchilddoc|\\
|\addtocounter{page}{-1}|\\
\textit{code for banner page}\\
|\newpage|\\
|\||fi|
\end{tabular}
\end{center}
%
Here one could write a message such as:
\begin{center}
|This is the part \childdocname{} of \childdocjob{}.|
\end{center}

%%%%%%%%%%%%%%%%%%%%%%%%%%%%%%%%%%%%%%%%%%%%%%%%%%%%%%%%%%%%%%%%%%%%%%%%%%%%%%%%
\subsection{Flags}
\label{sec:flags}

The package makes it easy to generate different versions
of the main or child documents.
To this end compilation flags can be defined
and assigned different default values.
They will be particularly useful in conjunction
with the forwarding mechanism described in \secref{sec:forward}.

For example, it may be useful to have a flag |\version|
which can be set to |draft| or |final|.
The document source will contain some conditional code
depending on the value of |\version|.
Suppose further, the flag should default to |final| for the main file
and to |draft| for child files
which is a natural assignment for editing the document.
This is achieved by placing the following code
in the preamble of the main document
(below the |\childdocmain| directive):
%
\begin{center}
\begin{tabular}{l}
|\ifchilddoc|\\
|\providecommand{\version}{draft}|\\
|\||else|\\
|\providecommand{\version}{final}|\\
|\||fi|
\end{tabular}
\end{center}
%
The definition by |\providecommand| makes sure
that previous definitions are not overwritten.
Further statements |\providecommand{\version}{...}|
can thus be added before the above code to override it.

For the main file, one might add a line
(between |\childdocmain| and the above block)
%
\begin{center}
|%\ifchilddoc\||else\providecommand{\version}{draft}\||fi|
\end{center}
%
which can be uncommented to produce a draft version.
Likewise one can add a line to the very top of a child file
(above the |\childdocof{|\textit{main}|}| directive)
%
\begin{center}
|%\providecommand{\version}{final}|
\end{center}
%
which can be uncommented to produce the final version of this child document.

%%%%%%%%%%%%%%%%%%%%%%%%%%%%%%%%%%%%%%%%%%%%%%%%%%%%%%%%%%%%%%%%%%%%%%%%%%%%%%%%
\subsection{Forwarding}
\label{sec:forward}

Different versions of the main or child documents
using compilation flags as described in \secref{sec:flags}
can be (permanently) stored in different files
for convenient compilation, viewing and distribution.
To this end, the package defines a command
to pass on compilation to a different file:

%%%%%%%%%%%%%%%%%%%%%%%%%%%%%%%%%%%%%%%%
\DescribeMacro{\childdocforward}
The command |\childdocforward| redirects processing to
another source file:
%
\begin{center}
\begin{tabular}{l}
|\input{childdoc.def}|\\
|\childdocforward[|\textit{main}|]{|\textit{dest}|}|\\
\end{tabular}
\end{center}
%
The argument \textit{dest} is the destination file
(without extension).
It should be the main file or one of the child files.
Note that further \textsf{childdoc} directives
such as |\childdocof| and |\childdocforward|
in the indicated file will be processed in this form.
The optional argument \textit{main}
passes on directly to the main file \textit{main}
while pretending to compile the child \textit{dest}.
This form behaves as if \textit{dest}
issues |\childdocof{|\textit{main}|}| right away,
and no further \textsf{childdoc} directives will be processed.

%%%%%%%%%%%%%%%%%%%%%%%%%%%%%%%%%%%%%%%%
\DescribeMacro{\...prefix}
In the alternative form |\childdocforwardprefix|,
%
\begin{center}
\begin{tabular}{l}
|\input{childdoc.def}|\\
|\childdocforwardprefix[|\textit{main}|]{|\textit{prefix}|}{|\textit{dest}|}|
\end{tabular}
\end{center}
%
the destination file is determined by a pattern
depending on the current file:
To make this work, the current file must be called
`{\textit{prefix}\hspace{0.2em}\textit{suffix}}'
with \textit{prefix} matching precisely the argument.
Processing is then passed on to the file
`{\textit{dest}\hspace{0.2em}\textit{suffix}}'.
Surely, the same effect is achieved by
directly specifying the
argument `{\textit{dest}\hspace{0.2em}\textit{suffix}}'
in the first form.
However, that requires to set up a different file
for each child. With the alternative form of the command
all these files can have exactly the same content
which simplifies setting them up and maintaining them.

For example, the following file |draft.tex|
with a compilation flag |\version| as described in \secref{sec:flags}
compiles the main document as a draft:
%
\begin{center}
\begin{tabular}{l}
|\def\version{draft}|\\
|\input{childdoc.def}|\\
|\childdocforward{|\textit{main}|}|
\end{tabular}
\end{center}
%
Likewise, the following files |final|\textit{nn}|.tex|
compile the final version of the child document
|child|\textit{nn}|.tex|:
%
\begin{center}
\begin{tabular}{l}
|\def\version{final}|\\
|\input{childdoc.def}|\\
|\childdocforwardprefix{final}{child}|
\end{tabular}
\end{center}
%

Note that when several versions of a main file and/or of each child file
are to be generated, it may be convenient to set up a |Makefile| or
shell script to automatise the process.

%%%%%%%%%%%%%%%%%%%%%%%%%%%%%%%%%%%%%%%%%%%%%%%%%%%%%%%%%%%%%%%%%%%%%%%%%%%%%%%%
\subsection{Command Line Processing}
\label{sec:commandline}

The effect of redirection files can also be achieved by invoking
the \LaTeX{} compiler with a more elaborate command line.
Most conveniently this should be done as part
of a shell script or a |Makefile|.

When using \textsf{childdoc} in the main file, the following
command lines effectively perform a redirection
(note that depending on the shell being used,
backslashes may have to be doubled: `|\|' $\to$ `|\\|'):
%
\begin{center}
|... -jobname "|\textit{target}|" |\\|"|[\textit{flags}]%
|\input{childdoc.def}\childdocforward[|\textit{main}|]{|\textit{dest}|}"|
\end{center}
%
Here \textit{target} is the name of the output file,
\textit{main} is the name of the main file
and \textit{dest} is the name of the main or child file to be processed
(all filenames without extensions).
The optional argument \textit{main} can be omitted
if \textit{main} matches \textit{dest}.
Optionally, compilation \textit{flags} can be defined via |\def| commands.
This command line makes the \TeX{} engine believe
it is compiling the file \textit{target}
whose content is specified as the latter parameter.
The provided code then forwards the processing to
\textit{main} or \textit{dest} as described in \secref{sec:forward}.

%%%%%%%%%%%%%%%%%%%%%%%%%%%%%%%%%%%%%%%%%%%%%%%%%%%%%%%%%%%%%%%%%%%%%%%%%%%%%%%%
\subsection{Include by Input}
\label{sec:input}

Including child documents by |\include| has some restrictions by design.
Most notably, the content of a child document always occupies
its own set of pages; pages cannot be shared between child documents.
Usually, this behaviour makes perfect sense
because each child document contain an essential part of the document.
However, in some situations it may be desirable to compose
a document from a collection of parts
without having mandatory page breaks between then.
For this case, the package
provides a mechanism to include parts
by |\input| which can also be processed individually.
However, by construction this mechanism
requires manual handling of the content to be output.

%%%%%%%%%%%%%%%%%%%%%%%%%%%%%%%%%%%%%%%%
\DescribeMacro{\ifchilddocmanual}
The main file should be prepared as usual, see \secref{sec:include}.
However, the document body must make a distinction
between processing of an individual part and of the main document, e.g.:
%
\begin{center}
\begin{tabular}{l}
|\ifchilddocmanual|\\
|\input{\childdocname}|\\
|\||else|\\
\textit{document body with }|\input{|\textit{part}|}|\\
|\||fi|
\end{tabular}
\end{center}
%
The conditional |\ifchilddocmanual| is true whenever
a part to be included by |\input| is being compiled,
and the name of the part is stored in |\childdocname|.

%%%%%%%%%%%%%%%%%%%%%%%%%%%%%%%%%%%%%%%%
\DescribeMacro{\childdocby}
Each part to be included by |\input| should start with:
%
\begin{center}
\begin{tabular}{l}
|\input{childdoc.def}|\\
|\childdocby{|\textit{main}|}|\\
\end{tabular}
\end{center}
%
The directive |\childdocby| is similar to |\childdocof|
described in \secref{sec:include},
but the subsequent selection of content must be done manually.
To that end, both |\ifchilddoc| and |\ifchilddocmanual|
will be true upon processing of a part,
and the name of the part is stored in |\childdocname|.
Note that |\jobname| will be set to the filename of the current part
so that each part receives an individual |.aux| file
that does not interfere with the |.aux| file(s) of the main document.
This behaviour can be altered by the alternative form
|\childdocby[*]{|\textit{main}|}| (with a non-empty optional argument)
which uses the |.aux| file of the main document
by setting |\jobname| to \textit{main}.

%%%%%%%%%%%%%%%%%%%%%%%%%%%%%%%%%%%%%%%%%%%%%%%%%%%%%%%%%%%%%%%%%%%%%%%%%%%%%%%%
\subsection{Driver Development}
\label{sec:driver}

The \textsf{childdoc} mechanism can also be use for the development
of definition files such as \LaTeX{} styles or classes.
This case differs from the above setup with multiple parts
included by |\include| in that no |\includeonly| should be invoked.
This can be achieved by starting the include file
(before |\ProvidesPackage|) with:
%
\begin{center}
\begin{tabular}{l}
|\input{childdoc.def}|\\
|\childdocforward{|\textit{main}|}|\\
\end{tabular}
\end{center}
%
or alternatively with:
%
\begin{center}
\begin{tabular}{l}
|\input{childdoc.def}|\\
|\childdocby{|\textit{main}|}|\\
\end{tabular}
\end{center}
%
Both forms have slightly different effects as described above.
The main file is prepared as usual, see \secref{sec:include}.

%%%%%%%%%%%%%%%%%%%%%%%%%%%%%%%%%%%%%%%%%%%%%%%%%%%%%%%%%%%%%%%%%%%%%%%%%%%%%%%%
\subsection{Legacy Detection}
\label{sec:detection}

The directive |\childdocmain| in the main file can detect
whether the complete document or merely a child is to be compiled
even without using the directive |\childdocof|.
This method is deprecated because it is less robust
and there is no compelling reason to use it;
it is merely provided for backward compatibility
and it may be removed in future versions.

If the detection mechanism is to be used,
it is mandatory to correctly specify
the filename of the main file as the argument of |\childdocmain|:
%
\begin{center}
\begin{tabular}{l}
|\input{childdoc.def}|\\
|\childdocmain{|\textit{main}|}|\\
\end{tabular}
\end{center}
%
If |\jobname| does not match the argument \textit{main} of |\childdocmain|,
it is assumed that |\jobname| points to the child file to be compiled.
When using |\childdocmain| with the main file specified as argument,
it suffices to start a child file
with just |\input{|\textit{main}|}|
without loading of the package and using |\childdocof|.
If instead all processing is done
with the appropriate \textsf{childdoc} directives,
the argument of \textit{main} of |\childdocmain| can be empty.

An alternative version of the command line processing described
in \secref{sec:commandline} using the detection mechanism reads:
%
\begin{center}
|... -jobname "|\textit{target}|" "|[\textit{flags}]%
[|\def\jobname{|\textit{dest}|}|]|\input{|\textit{main}|}"|
\end{center}

%%%%%%%%%%%%%%%%%%%%%%%%%%%%%%%%%%%%%%%%%%%%%%%%%%%%%%%%%%%%%%%%%%%%%%%%%%%%%%%%
\subsection{Manual Code}
\label{sec:manual}

In case one cannot be certain whether the definitions file |childdoc.def|
is installed on the target \TeX{} distribution
and one prefers not to ship it,
it is conceivable to paste a few relevant commands into the sources.

To that end, drop all statements |\input{childdoc.def}|
and perform the replacements as outlined below.
Instead of |\childdocmain{|\textit{main}|}| add the following code
to the top of the main file:
%
\begin{center}
\begin{tabular}{l}
|\||ifdefined\childdocname\endinput\||fi\newif\ifchilddoc|\\
|\edef\childdocname{\scantokens\expandafter{\jobname\noexpand}}|\\
|\def\childdocmain{|\textit{main}|}\||ifx\childdocmain\childdocname\||else|\\
|\childdoctrue\includeonly{\childdocname}\let\jobname\childdocmain\||fi|\\
\end{tabular}
\end{center}
%
Instead of |\childdocof{|\textit{main}|}| just include the main file
at the top of each child file:
%
\begin{center}
|\input{|\textit{main}|}|
\end{center}
%
A simple redirection |\childdocforward{|\textit{dest}|}| is achieved by:
%
\begin{center}
|\def\jobname{|\textit{dest}|}\input{\jobname}|
\end{center}
%
The redirection with prefix
|\childdocforwardprefix[|\textit{prefix}|]{|\textit{dest}|}|
is accomplished by:
%
\begin{center}
\begin{tabular}{l}
|{\edef\jobname{\scantokens\expandafter{\jobname\noexpand}}|\\
|\def\redirectjob |\textit{prefix}|#1~~~{\gdef\jobname{|\textit{dest}|#1}}|\\
|\expandafter\redirectjob\jobname~~~}\input{\jobname}|
\end{tabular}
\end{center}

In an alternative approach,
child documents can be compiled by a specific command line
without additional code or specific definitions:
%
\begin{center}
|... -jobname "|\textit{target}|" "|[\textit{flags}]%
|\includeonly{|\textit{dest}|}\input{|\textit{main}|}"|
\end{center}
%

%%%%%%%%%%%%%%%%%%%%%%%%%%%%%%%%%%%%%%%%%%%%%%%%%%%%%%%%%%%%%%%%%%%%%%%%%%%%%%%%
%%%%%%%%%%%%%%%%%%%%%%%%%%%%%%%%%%%%%%%%%%%%%%%%%%%%%%%%%%%%%%%%%%%%%%%%%%%%%%%%
\section{Information}

%%%%%%%%%%%%%%%%%%%%%%%%%%%%%%%%%%%%%%%%%%%%%%%%%%%%%%%%%%%%%%%%%%%%%%%%%%%%%%%%
\subsection{Copyright}

Copyright \copyright{} 2017--2018 Niklas Beisert

This work may be distributed and/or modified under the
conditions of the \LaTeX{} Project Public License, either version 1.3
of this license or (at your option) any later version.
The latest version of this license is in
  \url{http://www.latex-project.org/lppl.txt}
and version 1.3 or later is part of all distributions of \LaTeX{}
version 2005/12/01 or later.

This work has the LPPL maintenance status `maintained'.

The Current Maintainer of this work is Niklas Beisert.

This work consists of the files |README.txt|, |childdoc.ins| and |childdoc.dtx|
as well as the derived files |childdoc.def|, |cdocsamp.tex|
with |cdocsch1.tex|, |cdocsch2.tex|, |cdocspt3.tex|, |cdocspt4.tex|,
|cdocsdrf.tex|, |cdocsfn1.tex|, |cdocsfn2.tex|
as well as |childdoc.pdf|.

%%%%%%%%%%%%%%%%%%%%%%%%%%%%%%%%%%%%%%%%%%%%%%%%%%%%%%%%%%%%%%%%%%%%%%%%%%%%%%%%
\subsection{Files and Installation}

The package consists of the files:
%
\begin{center}
\begin{tabular}{ll}
    |README.txt|   & readme file \\
    |childdoc.ins| & installation file \\
    |childdoc.dtx| & source file \\
    |childdoc.def| & definition file \\
    |cdocsamp.tex| & sample main file \\
    |cdocsch1.tex| & sample include file \\
    |cdocsch2.tex| & sample include file \\
    |cdocspt3.tex| & sample part file \\
    |cdocspt4.tex| & sample part file \\
    |cdocsdrf.tex| & sample redirection file \\
    |cdocsfn1.tex| & sample redirection file \\
    |cdocsfn2.tex| & sample redirection file \\
    |childdoc.pdf| & manual
\end{tabular}
\end{center}
%
The distribution consists of the files
|README.txt|, |childdoc.ins| and |childdoc.dtx|.
%
\begin{itemize}
\item
Run (pdf)\LaTeX{} on |childdoc.dtx|
to compile the manual |childdoc.pdf| (this file).
\item
Run \LaTeX{} on |childdoc.ins| to create the definitions file |childdoc.def|
and the sample |cdocsamp.tex| with include files
|cdocsch1.tex|, |cdocsch2.tex|, |cdocspt3.tex|, |cdocspt4.tex|,
|cdocsdrf.tex|, |cdocsfn1.tex|, |cdocsfn2.tex|.
Then copy the file |childdoc.def| to an appropriate directory of your \LaTeX{}
distribution, e.g.\ \textit{texmf-root}|/tex/latex/childdoc|.
\end{itemize}

%%%%%%%%%%%%%%%%%%%%%%%%%%%%%%%%%%%%%%%%%%%%%%%%%%%%%%%%%%%%%%%%%%%%%%%%%%%%%%%%
\subsection{Related CTAN Packages}

There are several other packages which offer a similar functionality:
%
\begin{itemize}
\item
The packages
\href{http://ctan.org/pkg/docmute}{\textsf{docmute}},
\href{http://ctan.org/pkg/includex}{\textsf{includex}} and
\href{http://ctan.org/pkg/standalone}{\textsf{standalone}}
provide commands to include only the document body of
a child file thus allowing both files to be compiled individually.
\item
The packages \href{http://ctan.org/pkg/subdocs}{\textsf{subdocs}}
and \href{http://ctan.org/pkg/subfiles}{\textsf{subfiles}}
provide structures in which the main and child documents can be
encapsulated and allowing them to be compiled individually.
The inclusion mechanism is different from the conventional |\include|.
\item
The package \href{http://ctan.org/pkg/combine}{\textsf{combine}}
is an elaborate solution to combine several documents into one.
\end{itemize}
%
See also the CTAN topic \href{http://ctan.org/topic/subdocs}{\textsf{subdocs}}
for further related packages.
The present package differs from the above solutions in that
a document structure constructed with the conventional |\include| mechanism
just needs two extra commands at the top of every file
such that all constituent files can be compiled individually.

%%%%%%%%%%%%%%%%%%%%%%%%%%%%%%%%%%%%%%%%%%%%%%%%%%%%%%%%%%%%%%%%%%%%%%%%%%%%%%%%
%\subsection{Feature Suggestions}
%
%The following is a list of features which may be useful for future
%versions of this package:
%%
%\begin{itemize}
%\item
%\ldots
%\end{itemize}

%%%%%%%%%%%%%%%%%%%%%%%%%%%%%%%%%%%%%%%%%%%%%%%%%%%%%%%%%%%%%%%%%%%%%%%%%%%%%%%%
\subsection{Revision History}

%%%%%%%%%%%%%%%%%%%%%%%%%%%%%%%%%%%%%%%%
\paragraph{v2.0:} 2018/12/30

\begin{itemize}
\item
immediate forward processing
\item
added |\childdocby| mechanism
\item
manual restructured
\end{itemize}

%%%%%%%%%%%%%%%%%%%%%%%%%%%%%%%%%%%%%%%%
\paragraph{v1.6:} 2018/01/17

\begin{itemize}
\item
application for development of include files
\item
corrections to manual
\end{itemize}

%%%%%%%%%%%%%%%%%%%%%%%%%%%%%%%%%%%%%%%%
\paragraph{v1.5:} 2017/05/21

\begin{itemize}
\item
more complete structuring introduced
\item
|\childdocof| introduced
\item
|\childdoc| renamed to |\childdocmain|
\item
|\childredirect| renamed to |\childdocforward| and |\childdocforwardprefix|
and functionality expanded
\end{itemize}

%%%%%%%%%%%%%%%%%%%%%%%%%%%%%%%%%%%%%%%%
\paragraph{v1.0:} 2017/04/27

\begin{itemize}
\item
manual and install package
\item
first version published on CTAN
\end{itemize}

%%%%%%%%%%%%%%%%%%%%%%%%%%%%%%%%%%%%%%%%
\paragraph{v0.6:} 2017/04/26

\begin{itemize}
\item
redirection mechanism added
\end{itemize}

%%%%%%%%%%%%%%%%%%%%%%%%%%%%%%%%%%%%%%%%
\paragraph{v0.5:} 2017/04/26

\begin{itemize}
\item
functionality in definition file
\end{itemize}


%%%%%%%%%%%%%%%%%%%%%%%%%%%%%%%%%%%%%%%%%%%%%%%%%%%%%%%%%%%%%%%%%%%%%%%%%%%%%%%%
%%%%%%%%%%%%%%%%%%%%%%%%%%%%%%%%%%%%%%%%%%%%%%%%%%%%%%%%%%%%%%%%%%%%%%%%%%%%%%%%
%%%%%%%%%%%%%%%%%%%%%%%%%%%%%%%%%%%%%%%%%%%%%%%%%%%%%%%%%%%%%%%%%%%%%%%%%%%%%%%%
\appendix

\settowidth\MacroIndent{\rmfamily\scriptsize 000\ }

 \DocInput{childdoc.dtx}

\end{document}
%</driver>
% \fi
%
% %%%%%%%%%%%%%%%%%%%%%%%%%%%%%%%%%%%%%%%%%%%%%%%%%%%%%%%%%%%%%%%%%%%%%%%%%%%%%%
% %%%%%%%%%%%%%%%%%%%%%%%%%%%%%%%%%%%%%%%%%%%%%%%%%%%%%%%%%%%%%%%%%%%%%%%%%%%%%%
% \section{Sample}
%\iffalse
%<*samplemain>
%\fi
%
% The following presents a sample document
% with two chapters, two parts, a title page,
% a compile flag as well as three forwarding files to set the flag.
% It consists of eight |.tex| files:
% \begin{center}
% \begin{tabular}{ll}
% |cdocsamp.tex|&main file\\
% |cdocsch1.tex|&include file for chapter 1\\
% |cdocsch2.tex|&include file for chapter 2\\
% |cdocspt3.tex|&include file for part 3\\
% |cdocspt4.tex|&include file for part 4\\
% |cdocsdrf.tex|&forwarding file for main file in draft mode\\
% |cdocsfi1.tex|&forwarding file for final version of chapter 1\\
% |cdocsfi2.tex|&forwarding file for final version of chapter 2\\
% \end{tabular}
% \end{center}
% Each of the eight files can be compiled directly by the \LaTeX{} compiler.
%
% %%%%%%%%%%%%%%%%%%%%%%%%%%%%%%%%%%%%%%
% \paragraph{Main File.}
%
% The main file is called |cdocsamp.tex|.
%
% Load the \textsf{childdoc} definitions and
% declare the filename for the main document:
%    \begin{macrocode}
\input{childdoc.def}
\childdocmain{}
%    \end{macrocode}

% Optional override for |\version| flag:
%    \begin{macrocode}
%%\ifchilddoc\else\providecommand{\version}{draft}\fi
%    \end{macrocode}

% Define the default values for the |\version| flag
% (|final| for the main file and |draft| for childs):
%    \begin{macrocode}
\ifchilddoc
\providecommand{\version}{draft}
\else
\providecommand{\version}{final}
\fi
%    \end{macrocode}

% Load the standard document class:
%    \begin{macrocode}
\documentclass[12pt]{article}
%    \end{macrocode}

% Start the document body:
%    \begin{macrocode}
\begin{document}
%    \end{macrocode}

% Declare a title page.
% Print title, part of document being processed and version flag:
%    \begin{macrocode}
\addtocounter{page}{-1}
\begin{center}
{\LARGE\bfseries{}childdoc example\par}
\vspace{1cm}
\ifchilddoc
\ifchilddocmanual part\else chapter\fi:
`\childdocname' of `\childdocjob'\par
\else
main document: `\childdocjob'\par
\fi
version: \version\par
\end{center}
\newpage
%    \end{macrocode}

% Manually include selected file,
% otherwise process as usual:
%    \begin{macrocode}
\ifchilddocmanual
\section*{part `\childdocname'}
\input{\childdocname}
\else
%    \end{macrocode}

% Include the two chapters:
%    \begin{macrocode}
\include{cdocsch1}
\include{cdocsch2}
%    \end{macrocode}

% Include the two parts unless only chapters should be displayed:
%    \begin{macrocode}
\ifchilddoc\else
\section{part three}
\input{cdocspt3}
\section{part four}
\input{cdocspt4}
\fi
%    \end{macrocode}

% Process as usual until here:
%    \begin{macrocode}
\fi
%    \end{macrocode}

% End of document body:
%    \begin{macrocode}
\end{document}
%    \end{macrocode}
%\iffalse
%</samplemain>
%\fi
%
% %%%%%%%%%%%%%%%%%%%%%%%%%%%%%%%%%%%%%%
% \paragraph{Chapter Include Files.}
%
% The include files are called |cdocsch1.tex| and |cdocsch2.tex|.
%
%\iffalse
%<*samplechap1|samplechap2>
%\fi

% Optional override for |\version| flag:
%    \begin{macrocode}
%%\providecommand{\version}{final}
%    \end{macrocode}

% Include the main document:
%    \begin{macrocode}
\input{childdoc.def}
\childdocof{cdocsamp}
%    \end{macrocode}

%\iffalse
%</samplechap1|samplechap2>
%\fi
%
%\iffalse
%<*samplechap1>
%\fi
% Some text for chapter 1:
%    \begin{macrocode}
\section{one}
some text in chapter one
%    \end{macrocode}

%\iffalse
%</samplechap1>
%\fi
% Some text for chapter 2:
%\iffalse
%<*samplechap2>
%\fi
%    \begin{macrocode}
\section{two}
more text in chapter two
%    \end{macrocode}

%\iffalse
%</samplechap2>
%\fi
%
% %%%%%%%%%%%%%%%%%%%%%%%%%%%%%%%%%%%%%%
% \paragraph{Part Include Files.}
%
% The include files are called |cdocspt3.tex| and |cdocspt4.tex|.
%
%\iffalse
%<*samplepart3|samplepart4>
%\fi

% Optional override for |\version| flag:
%    \begin{macrocode}
%%\providecommand{\version}{final}
%    \end{macrocode}

% Include the main document:
%    \begin{macrocode}
\input{childdoc.def}
\childdocby{cdocsamp}
%    \end{macrocode}

%\iffalse
%</samplepart3|samplepart4>
%\fi
%
%\iffalse
%<*samplepart3>
%\fi
% Some text for part 3:
%    \begin{macrocode}
some text in part three
%    \end{macrocode}

%\iffalse
%</samplepart3>
%\fi
% Some text for part 4:
%\iffalse
%<*samplepart4>
%\fi
%    \begin{macrocode}
more text in part four
%    \end{macrocode}

%\iffalse
%</samplepart4>
%\fi
%
% %%%%%%%%%%%%%%%%%%%%%%%%%%%%%%%%%%%%%%
% \paragraph{Forwarding for a Complete Draft.}
%
% The following forwarding file |cdocsdrf.tex|
% compiles the main document in draft mode:
%\iffalse
%<*sampledraft>
%\fi
%    \begin{macrocode}
\def\version{draft}
\input{childdoc.def}
\childdocforward{cdocsamp}
%    \end{macrocode}

%\iffalse
%</sampledraft>
%\fi
%
% %%%%%%%%%%%%%%%%%%%%%%%%%%%%%%%%%%%%%%
% \paragraph{Forwarding for Final Version of the Chapters.}
%
% The following forwarding files |cdocsfn1.tex| and |cdocsfn2.tex|
% (with identical content)
% compile the final versions of the child documents
% |cdocsch1.tex| and |cdocsch2.tex|, respectively:
%\iffalse
%<*samplefinal>
%\fi
%    \begin{macrocode}
\def\version{final}
\input{childdoc.def}
\childdocforwardprefix[cdocsamp]{cdocsfn}{cdocsch}
%    \end{macrocode}

%\iffalse
%</samplefinal>
%\fi
%
% %%%%%%%%%%%%%%%%%%%%%%%%%%%%%%%%%%%%%%
% \paragraph{Command Line Processing.}
%
% The following three command lines generate the output files
% |cdocscld|, |cdocscl1| and |cdocscl2|
% which should be identical to
% |cdocsdrf|, |cdocsch1| and |cdocsfn2|, respectively:
% \begin{center}
% \begin{tabular}{l}
% |latex -jobname cdocscld \|\\
% |  "\def\version{draft}\input{childdoc.def}\childdocforward{cdocsamp}"|\\
% |latex -jobname cdocscl1 \|\\
% |  "\input{childdoc.def}\childdocforward[cdocsamp]{cdocsch1}"|\\
% |latex -jobname cdocscl2 \|\\
% |  "\def\version{final}\input{childdoc.def}\childdocforward{cdocsch2}"|
% \end{tabular}
% \end{center}
% Note that the trailing backslash on each first line
% merely continues the input to the second line
% (for convenient cut ant paste).
% Furthermore, the command |latex| can be replaced by any
% of its alternative versions such as |pdflatex|.
%
% %%%%%%%%%%%%%%%%%%%%%%%%%%%%%%%%%%%%%%%%%%%%%%%%%%%%%%%%%%%%%%%%%%%%%%%%%%%%%%
% %%%%%%%%%%%%%%%%%%%%%%%%%%%%%%%%%%%%%%%%%%%%%%%%%%%%%%%%%%%%%%%%%%%%%%%%%%%%%%
% \section{Implementation}
%\iffalse
%<*package>
%\fi
%
% This section describes the definitions file |childdoc.def|.

% The definitions cannot be loaded using |\usepackage| or |\RequirePackage|
% which has a mechanism to prevent loading a style file more than once.
% When loading the definitions by means of |\input|
% multiple instances have to be prevented manually:
%\iffalse
%This code needs to be before the `\ProvidesFile' directive
%which is defined at the beginning of this file.
%Therefore it is also placed there and commented out here.
%</package>
%<*discard>
%\fi
%    \begin{macrocode}
\ifdefined\childdocmain\endinput\fi
%    \end{macrocode}
%\iffalse
%</discard>
%<*package>
%\fi
%
% \macro{\ifchilddoc}
% \macro{\ifchilddocmanual}
% The conditional |\ifchilddoc| tells whether a
% child (true) or main (false) document is being compiled.
% The conditional |\ifchilddocmanual| tells whether
% the |\includeonly| mechanism is used (false) or
% the selection of child files must be performed manually (true).
% The definitions initialise to false:
%    \begin{macrocode}
\newif\ifchilddoc
\newif\ifchilddocmanual
%    \end{macrocode}

% \macro{\childdocname}
% \macro{\childdocjob}
% The macro |\childdocname| stores the name of the main document
% to be compiled. The macro |\childdocjob| stores the name of
% the document on which the \LaTeX{} compiler was originally invoked.
% The content of |\jobname| cannot be compared
% to filenames specified in the source due to different catcodes.
% The following code rescans |\jobname|, stores the result
% in |\childdocname| and saves a copy in |\childdocjob|:
%    \begin{macrocode}
\edef\childdocname{\scantokens\expandafter{\jobname\noexpand}}
\let\childdocjob\childdocname
%    \end{macrocode}

% \macro{\childdocdisable}
% The macro |\childdocdisable| prevents the main file
% from being processed more than once.
% At this stage, the main document command |\childdocmain|
% is assumed to be called once again where it should do nothing.
% Any subsequent call to it should prevent
% a secondary processing of the main document
% It overwrites the forwarding commands
% |\childdocof| and |\childdocforward|
% with empty macros to prevent further inclusions of the main document:
%    \begin{macrocode}
\newcommand{\childdocdisable}
{
  \renewcommand{\childdocmain}[1]{\renewcommand{\childdocmain}[1]{\endinput}}
  \renewcommand{\childdocof}[1]{}
  \renewcommand{\childdocby}[2][]{}
  \renewcommand{\childdocforward}[2][]{}
  \renewcommand{\childdocdisable}{}
}
%    \end{macrocode}

% \macro{\childdocmain}
% The macro |\childdocmain| is to be called at the top of the main file
% with nothing or the main filename (without extension) as argument.
% First, it breaks loops.
% If the argument is not empty and does not match |\childdocname|
% (which is set by the first inclusion of |childdoc.def|),
% |\ifchilddoc| is set to true, |\includeonly| is applied to the child file
% and |\jobname| is set to the main file
% (for proper handling of |.aux| files):
%    \begin{macrocode}
\newcommand{\childdocmain}[1]
{
  \childdocdisable\childdocmain{}
  \if?#1?\else
    \begingroup
      \def\childdoctmp{#1}
      \ifx\childdoctmp\childdocname
        \def\childdoctmp{}
      \else
        \def\childdoctmp
        {
          \childdoctrue
          \includeonly{\childdocname}
          \def\childdocjob{#1}
          \def\jobname{#1}
        }
      \fi
      \expandafter
    \endgroup
    \childdoctmp
  \fi
}
%    \end{macrocode}

% \macro{\childdocof}
% The command |\childdocof| redirects
% compilation to the main file |#1|.
%    \begin{macrocode}
\newcommand{\childdocof}[1]
{
  \childdocdisable
  \childdoctrue
  \includeonly{\childdocname}
  \def\jobname{#1}
  \def\childdocjob{#1}
  \input{#1}
}
%    \end{macrocode}

% \macro{\childdocby}
% The command |\childdocby| ....
%    \begin{macrocode}
\newcommand{\childdocby}[2][]
{
  \childdocdisable
  \childdoctrue
  \childdocmanualtrue
  \if?#1?\else
    \def\jobname{#2}
  \fi
  \def\childdocjob{#2}
  \input{#2}
  \endinput
}
%    \end{macrocode}

% \macro{\childdocforward}
% The command |\childdocforward| redirects
% compilation to the main file or
% (if the optional argument is given) a child file.
% Parameters are set as if the main file
% or a child file starting with |\childdocof| was compiled.
% Then compilation is handed over to the main file:
%    \begin{macrocode}
\newcommand{\childdocforward}[2][]
{
  \begingroup
    \if?#1?
      \def\childdoctmp
      {
        \def\childdocname{#2}
        \def\childdocjob{#2}
        \def\jobname{#2}
        \input{#2}
        \endinput
      }
    \else
      \def\childdoctmp
      {
        \childdocdisable
        \def\childdocname{#2}
        \childdoctrue
        \includeonly{#2}
        \def\childdocjob{#1}
        \def\jobname{#1}
        \input{#1}
        \endinput
      }
    \fi
    \expandafter
  \endgroup
  \childdoctmp
}
%    \end{macrocode}

% \macro{\childdocforwardprefix}
% The command |\childdocforwardprefix| redirects
% compilation to the main or a child file by means of a pattern.
% The prefix |#1| in the current filename is replaced by |#2|
% and the suffix of the current filename is kept
% (it is assumed that the filename does not contain the substring `|~~~|'
% which is used as a delimiter).
% Compilation is handed over to the new file by |\childdocforward|:
%    \begin{macrocode}
\newcommand{\childdocforwardprefix}[3][]
{
  \begingroup
    \def\childdocextract #2##1~~~{\def\childdoctmp{\childdocforward[#1]{#3##1}}}
    \expandafter\childdocextract\childdocname~~~
    \expandafter
  \endgroup
  \childdoctmp
}
%    \end{macrocode}

% \macro{\childdoc}
% The deprecated macro |\childdoc| is a legacy version of |\childdocmain|:
%    \begin{macrocode}
\newcommand{\childdoc}{\childdocmain}
%    \end{macrocode}

% \macro{\childdocredirect}
% The deprecated macro |\childdocredirect| is a legacy version
% of |\childdocforward| and |\childdocforwardprefix|:
%    \begin{macrocode}
\newcommand{\childdocredirect}[2][]
{
  \begingroup
    \if?#1?
      \def\childdoctmp{\childdocforward{#2}}
    \else
      \def\childdoctmp{\childdocforwardprefix{#1}{#2}}
    \fi
    \expandafter
  \endgroup
  \childdoctmp
}
%    \end{macrocode}

%\iffalse
%</package>
%\fi
%
\endinput
\childdocforward{cdocsamp}"|\\
% |latex -jobname cdocscl1 \|\\
% |  "% \iffalse
%
% childdoc.dtx Copyright (C) 2017-2018 Niklas Beisert
%
% This work may be distributed and/or modified under the
% conditions of the LaTeX Project Public License, either version 1.3
% of this license or (at your option) any later version.
% The latest version of this license is in
%   http://www.latex-project.org/lppl.txt
% and version 1.3 or later is part of all distributions of LaTeX
% version 2005/12/01 or later.
%
% This work has the LPPL maintenance status `maintained'.
%
% The Current Maintainer of this work is Niklas Beisert.
%
% This work consists of the files childdoc.dtx and childdoc.ins
% and the derived files childdoc.def and cdocsamp.tex with
% cdocsch1.tex, cdocsch2.tex, cdocsdrf.tex, cdocsfn1.tex, cdocsfn2.tex.
%
%<package>\ifdefined\childdocmain\endinput\fi
%<package>\ProvidesFile{childdoc.def}[2018/12/30 v2.0 child document driver]
%<samplemain>\ProvidesFile{cdocsamp.tex}[2018/12/30 v2.0 sample for childdoc]
%<*driver>
%\ProvidesFile{childdoc.drv}[2018/12/30 v2.0 childdoc reference manual file]
\PassOptionsToClass{10pt,a4paper}{article}
\documentclass{ltxdoc}

\usepackage[margin=35mm]{geometry}
\usepackage{hyperref}
\usepackage{hyperxmp}
\usepackage[usenames]{color}

\hypersetup{colorlinks=true}
\hypersetup{pdfstartview=FitH}
\hypersetup{pdfpagemode=UseNone}
\hypersetup{pdfsource={}}
\hypersetup{pdflang={en-UK}}
\hypersetup{pdfcopyright={Copyright 2017-2018 Niklas Beisert.
  This work may be distributed and/or modified under the
  conditions of the LaTeX Project Public License, either version 1.3
  of this license or (at your option) any later version.}}
\hypersetup{pdflicenseurl={http://www.latex-project.org/lppl.txt}}
\hypersetup{pdfcontactaddress={ETH Zurich, ITP, HIT K,
  Wolfgang-Pauli-Strasse 27}}
\hypersetup{pdfcontactpostcode={8093}}
\hypersetup{pdfcontactcity={Zurich}}
\hypersetup{pdfcontactcountry={Switzerland}}
\hypersetup{pdfcontactemail={nbeisert@itp.phys.ethz.ch}}
\hypersetup{pdfcontacturl={http://people.phys.ethz.ch/\xmptilde nbeisert/}}

\newcommand{\secref}[1]{\hyperref[#1]{section \ref*{#1}}}

\parskip1ex
\parindent0pt
\let\olditemize\itemize
\def\itemize{\olditemize\parskip0pt}

\begin{document}

\title{The \textsf{childdoc} Package}
\hypersetup{pdftitle={The childdoc Package}}
\author{Niklas Beisert\\[2ex]
  Institut f\"ur Theoretische Physik\\
  Eidgen\"ossische Technische Hochschule Z\"urich\\
  Wolfgang-Pauli-Strasse 27, 8093 Z\"urich, Switzerland\\[1ex]
  \href{mailto:nbeisert@itp.phys.ethz.ch}
  {\texttt{nbeisert@itp.phys.ethz.ch}}}
\hypersetup{pdfauthor={Niklas Beisert}}
\hypersetup{pdfsubject={Manual for the LaTeX2e Package childdoc}}
\date{30 December 2018, \textsf{v2.0}}
\maketitle

\begin{abstract}\noindent
\textsf{childdoc} is a \LaTeXe{} package
that enables the direct compilation
of document sections included by |\include|
to individual files.
\end{abstract}

\begingroup
\parskip0ex
\tableofcontents
\endgroup

%%%%%%%%%%%%%%%%%%%%%%%%%%%%%%%%%%%%%%%%%%%%%%%%%%%%%%%%%%%%%%%%%%%%%%%%%%%%%%%%
%%%%%%%%%%%%%%%%%%%%%%%%%%%%%%%%%%%%%%%%%%%%%%%%%%%%%%%%%%%%%%%%%%%%%%%%%%%%%%%%
\section{Introduction}

\LaTeX{} provides a mechanism to structure a large document (such as a book)
into a main file and several child files (containing the chapters)
using the |\include| command.
This mechanism is beneficial for documents
which span hundreds of pages in order to
make the source file(s) more manageable.
Moreover, compilation can be restricted to
selected child files by means of the |\includeonly| command.
The latter feature can be used to reduce the compilation time while editing
(this was significantly more useful in the earlier days of \LaTeX{})
or to generate a smaller document which is easier to navigate.
Another application of |\includeonly| is to generate
documents consisting of selected parts of the complete document.

However, there are a few drawbacks of the plain |\include| mechanism:
\begin{itemize}
\item
The child files cannot be compiled on their own,
they can only be compiled via the main file.
A naive editing environment
(such as a text editor with an option
to have the current file processed by \LaTeX)
may require one to switch to the main file before compiling;
attempting to compile the child file produces errors.
\item
The main file must be modified (each time)
to adjust the |\includeonly| command
to the present needs. This easily leaves the main file in a messy state.
\item
The generated document will always carry the filename
of the main document. This is inconvenient if
several child files are to be compiled and
to be kept for distribution.
\end{itemize}

The present package provides a simple interface
to make child files individually compilable by \LaTeX{}.
Compiling a child file then has the same effect as compiling
the main file with an |\includeonly| command
to select the appropriate child.
Moreover the generated document will carry the name of the child
rather than the main file.
This resolves all three above issues.

This feature is meant to make the editing of books,
thesis documents and lecture notes somewhat more convenient.
However, the package can also be used efficiently for
composing a series of documents (such as exercise sheets)
which are typically distributed individually.
It then assists the author in generating the individual documents
(potentially in different versions)
as well as a document containing the collected series.
Another application is in developing style files
or other kinds of included material
where compilation of the style file could redirect
to a sample or test file.

%%%%%%%%%%%%%%%%%%%%%%%%%%%%%%%%%%%%%%%%%%%%%%%%%%%%%%%%%%%%%%%%%%%%%%%%%%%%%%%%
%%%%%%%%%%%%%%%%%%%%%%%%%%%%%%%%%%%%%%%%%%%%%%%%%%%%%%%%%%%%%%%%%%%%%%%%%%%%%%%%
\section{Usage}

First of all, the package \textsf{childdoc} is \emph{not} a standard
\LaTeXe{} |.sty| style file! Therefore it needs to be invoked in
a non-standard way.

%%%%%%%%%%%%%%%%%%%%%%%%%%%%%%%%%%%%%%%%%%%%%%%%%%%%%%%%%%%%%%%%%%%%%%%%%%%%%%%%
\subsection{Included Files}
\label{sec:include}

%%%%%%%%%%%%%%%%%%%%%%%%%%%%%%%%%%%%%%%%
\DescribeMacro{\childdocmain}
To use the package, add the commands
\begin{center}
\begin{tabular}{l}
|\input{childdoc.def}|\\
|\childdocmain{}|\\
\end{tabular}
\end{center}
at the very top of the main \LaTeX{} file,
in particular \emph{before} the |\documentclass| statement!
The argument of |\childdocmain| should be left empty
(but it must be present).

%%%%%%%%%%%%%%%%%%%%%%%%%%%%%%%%%%%%%%%%
\DescribeMacro{\childdocof}
Furthermore, add the commands
\begin{center}
\begin{tabular}{l}
|\input{childdoc.def}|\\
|\childdocof{|\textit{main}|}|\\
\end{tabular}
\end{center}
at the top of every child file \textit{child}
which is included by |\include{|\textit{child}|}|
from within the main file
(or at least for those files to be compiled individually).
The argument \textit{main} must be the filename of the main file.

There are a couple of
considerations in setting up the main and child documents:

%%%%%%%%%%%%%%%%%%%%%%%%%%%%%%%%%%%%%%%%
\paragraph{Restrictions.}

Please note the following restrictions:
\begin{itemize}
\item
|\childdocmain| must be called with one argument \textit{main}
to ensure compatibility with earlier version of the package.
It must either be empty (|\childdocmain{}|)
or precisely match the filename of the main file in which it is specified.
See \secref{sec:detection} for further information.
\item
The filename \textit{main} must be specified without the |.tex| extension.
\item
The filename \textit{main} is case sensitive
(even in case-insensitive file systems)
due to internal string comparison.
\item
The argument \textit{main} should be fully expanded, it cannot be a macro.
\item
Subdirectories and special characters should be avoided in filenames.
\item
The command |\childdocmain{|\textit{main}|}| must be followed by a whitespace.
It should not be followed immediately by another command
or by a comment mark `|%|'.
This is because the \TeX{} parser reads the token immediately following
the argument of |\childdocmain| and puts it
at the beginning of every child section;
however, a white\-space is ignored.
\end{itemize}

%%%%%%%%%%%%%%%%%%%%%%%%%%%%%%%%%%%%%%%%
\paragraph{Content of Main File.}

It is advisable to place all content in the child files included by |\include|.
Any output contained in the main file will appear in all child documents
unless suppressed manually;
it cannot be suppressed automatically by the |\includeonly| directive
and thus should normally be avoided.
A method to include some content in the main file
by means of conditional processing is described in \secref{sec:conditional}.

%%%%%%%%%%%%%%%%%%%%%%%%%%%%%%%%%%%%%%%%
\paragraph{Page Numbering.}

When only a part of the document is compiled,
the appropriate numbering of pages
(as well as other status parameters)
is determined from the |.aux| files.
The latter contain information from previous passes.
However this information needs to propagate through
all intermediate child documents.
Therefore the page numbering in child documents may well
be inconsistent until the complete document is compiled at least once.

A useful (if unconventional) way to always ensure a consistent
page numbering is to restart the numbering in each child document
and denote the pages by `\textit{child}|.|\textit{page}'
where \textit{child} represents the chapter/section number of the child file.
This can be achieved by the command
|\numberwithin{page}{|\textit{child}|}|
of the \textsf{amsmath} package
where \textit{child} can be |chapter| or |section|
depending on the chosen structuring.
Alternatively, one can modify the macro |\thepage| appropriately
and reset the counter |page| at the start of each child file.

%%%%%%%%%%%%%%%%%%%%%%%%%%%%%%%%%%%%%%%%%%%%%%%%%%%%%%%%%%%%%%%%%%%%%%%%%%%%%%%%
\subsection{Conditional Processing}
\label{sec:conditional}

The package provides a mechanism to compile different versions
of a document. To customise the versions further some conditional processing
can come in handy to distinguish which version is being compiled.
The package provides two macros to describe the compilation context:

%%%%%%%%%%%%%%%%%%%%%%%%%%%%%%%%%%%%%%%%
\DescribeMacro{\ifchilddoc}
The conditional |\ifchilddoc| distinguishes between the compilation of
child documents and the main document:
%
\begin{center}
|\ifchilddoc |\textit{child-code}| |[|\||else |\textit{main-code}]| \||fi|
\end{center}

%%%%%%%%%%%%%%%%%%%%%%%%%%%%%%%%%%%%%%%%
\DescribeMacro{\childdocname}
\DescribeMacro{\childdocjob}
The macro |\childdocname| contains the filename (without extension)
of the main or child file being processed.
Note that |\childdocjob| will always contain the name of the main file.

%%%%%%%%%%%%%%%%%%%%%%%%%%%%%%%%%%%%%%%%
\paragraph{Title Page.}

Conditional processing can be used to include a title or banner page
in the main document when proper precautions are taken.
Importantly, the code in the main file should ensure that the page counter
(as well as other status parameters which are stored in the |.aux| files)
takes the same value after the conditional processing.
Otherwise the page numbers may take divergent values
depending on which part is compiled.

For example, a title page could be declared by:
%
\begin{center}
\begin{tabular}{l}
|\ifchilddoc\||else|\\
|\addtocounter{page}{-1}|\\
\textit{code for title page}\\
|\newpage|\\
|\||fi|
\end{tabular}
\end{center}
%
A banner page for the child documents can be generated by:
%
\begin{center}
\begin{tabular}{l}
|\ifchilddoc|\\
|\addtocounter{page}{-1}|\\
\textit{code for banner page}\\
|\newpage|\\
|\||fi|
\end{tabular}
\end{center}
%
Here one could write a message such as:
\begin{center}
|This is the part \childdocname{} of \childdocjob{}.|
\end{center}

%%%%%%%%%%%%%%%%%%%%%%%%%%%%%%%%%%%%%%%%%%%%%%%%%%%%%%%%%%%%%%%%%%%%%%%%%%%%%%%%
\subsection{Flags}
\label{sec:flags}

The package makes it easy to generate different versions
of the main or child documents.
To this end compilation flags can be defined
and assigned different default values.
They will be particularly useful in conjunction
with the forwarding mechanism described in \secref{sec:forward}.

For example, it may be useful to have a flag |\version|
which can be set to |draft| or |final|.
The document source will contain some conditional code
depending on the value of |\version|.
Suppose further, the flag should default to |final| for the main file
and to |draft| for child files
which is a natural assignment for editing the document.
This is achieved by placing the following code
in the preamble of the main document
(below the |\childdocmain| directive):
%
\begin{center}
\begin{tabular}{l}
|\ifchilddoc|\\
|\providecommand{\version}{draft}|\\
|\||else|\\
|\providecommand{\version}{final}|\\
|\||fi|
\end{tabular}
\end{center}
%
The definition by |\providecommand| makes sure
that previous definitions are not overwritten.
Further statements |\providecommand{\version}{...}|
can thus be added before the above code to override it.

For the main file, one might add a line
(between |\childdocmain| and the above block)
%
\begin{center}
|%\ifchilddoc\||else\providecommand{\version}{draft}\||fi|
\end{center}
%
which can be uncommented to produce a draft version.
Likewise one can add a line to the very top of a child file
(above the |\childdocof{|\textit{main}|}| directive)
%
\begin{center}
|%\providecommand{\version}{final}|
\end{center}
%
which can be uncommented to produce the final version of this child document.

%%%%%%%%%%%%%%%%%%%%%%%%%%%%%%%%%%%%%%%%%%%%%%%%%%%%%%%%%%%%%%%%%%%%%%%%%%%%%%%%
\subsection{Forwarding}
\label{sec:forward}

Different versions of the main or child documents
using compilation flags as described in \secref{sec:flags}
can be (permanently) stored in different files
for convenient compilation, viewing and distribution.
To this end, the package defines a command
to pass on compilation to a different file:

%%%%%%%%%%%%%%%%%%%%%%%%%%%%%%%%%%%%%%%%
\DescribeMacro{\childdocforward}
The command |\childdocforward| redirects processing to
another source file:
%
\begin{center}
\begin{tabular}{l}
|\input{childdoc.def}|\\
|\childdocforward[|\textit{main}|]{|\textit{dest}|}|\\
\end{tabular}
\end{center}
%
The argument \textit{dest} is the destination file
(without extension).
It should be the main file or one of the child files.
Note that further \textsf{childdoc} directives
such as |\childdocof| and |\childdocforward|
in the indicated file will be processed in this form.
The optional argument \textit{main}
passes on directly to the main file \textit{main}
while pretending to compile the child \textit{dest}.
This form behaves as if \textit{dest}
issues |\childdocof{|\textit{main}|}| right away,
and no further \textsf{childdoc} directives will be processed.

%%%%%%%%%%%%%%%%%%%%%%%%%%%%%%%%%%%%%%%%
\DescribeMacro{\...prefix}
In the alternative form |\childdocforwardprefix|,
%
\begin{center}
\begin{tabular}{l}
|\input{childdoc.def}|\\
|\childdocforwardprefix[|\textit{main}|]{|\textit{prefix}|}{|\textit{dest}|}|
\end{tabular}
\end{center}
%
the destination file is determined by a pattern
depending on the current file:
To make this work, the current file must be called
`{\textit{prefix}\hspace{0.2em}\textit{suffix}}'
with \textit{prefix} matching precisely the argument.
Processing is then passed on to the file
`{\textit{dest}\hspace{0.2em}\textit{suffix}}'.
Surely, the same effect is achieved by
directly specifying the
argument `{\textit{dest}\hspace{0.2em}\textit{suffix}}'
in the first form.
However, that requires to set up a different file
for each child. With the alternative form of the command
all these files can have exactly the same content
which simplifies setting them up and maintaining them.

For example, the following file |draft.tex|
with a compilation flag |\version| as described in \secref{sec:flags}
compiles the main document as a draft:
%
\begin{center}
\begin{tabular}{l}
|\def\version{draft}|\\
|\input{childdoc.def}|\\
|\childdocforward{|\textit{main}|}|
\end{tabular}
\end{center}
%
Likewise, the following files |final|\textit{nn}|.tex|
compile the final version of the child document
|child|\textit{nn}|.tex|:
%
\begin{center}
\begin{tabular}{l}
|\def\version{final}|\\
|\input{childdoc.def}|\\
|\childdocforwardprefix{final}{child}|
\end{tabular}
\end{center}
%

Note that when several versions of a main file and/or of each child file
are to be generated, it may be convenient to set up a |Makefile| or
shell script to automatise the process.

%%%%%%%%%%%%%%%%%%%%%%%%%%%%%%%%%%%%%%%%%%%%%%%%%%%%%%%%%%%%%%%%%%%%%%%%%%%%%%%%
\subsection{Command Line Processing}
\label{sec:commandline}

The effect of redirection files can also be achieved by invoking
the \LaTeX{} compiler with a more elaborate command line.
Most conveniently this should be done as part
of a shell script or a |Makefile|.

When using \textsf{childdoc} in the main file, the following
command lines effectively perform a redirection
(note that depending on the shell being used,
backslashes may have to be doubled: `|\|' $\to$ `|\\|'):
%
\begin{center}
|... -jobname "|\textit{target}|" |\\|"|[\textit{flags}]%
|\input{childdoc.def}\childdocforward[|\textit{main}|]{|\textit{dest}|}"|
\end{center}
%
Here \textit{target} is the name of the output file,
\textit{main} is the name of the main file
and \textit{dest} is the name of the main or child file to be processed
(all filenames without extensions).
The optional argument \textit{main} can be omitted
if \textit{main} matches \textit{dest}.
Optionally, compilation \textit{flags} can be defined via |\def| commands.
This command line makes the \TeX{} engine believe
it is compiling the file \textit{target}
whose content is specified as the latter parameter.
The provided code then forwards the processing to
\textit{main} or \textit{dest} as described in \secref{sec:forward}.

%%%%%%%%%%%%%%%%%%%%%%%%%%%%%%%%%%%%%%%%%%%%%%%%%%%%%%%%%%%%%%%%%%%%%%%%%%%%%%%%
\subsection{Include by Input}
\label{sec:input}

Including child documents by |\include| has some restrictions by design.
Most notably, the content of a child document always occupies
its own set of pages; pages cannot be shared between child documents.
Usually, this behaviour makes perfect sense
because each child document contain an essential part of the document.
However, in some situations it may be desirable to compose
a document from a collection of parts
without having mandatory page breaks between then.
For this case, the package
provides a mechanism to include parts
by |\input| which can also be processed individually.
However, by construction this mechanism
requires manual handling of the content to be output.

%%%%%%%%%%%%%%%%%%%%%%%%%%%%%%%%%%%%%%%%
\DescribeMacro{\ifchilddocmanual}
The main file should be prepared as usual, see \secref{sec:include}.
However, the document body must make a distinction
between processing of an individual part and of the main document, e.g.:
%
\begin{center}
\begin{tabular}{l}
|\ifchilddocmanual|\\
|\input{\childdocname}|\\
|\||else|\\
\textit{document body with }|\input{|\textit{part}|}|\\
|\||fi|
\end{tabular}
\end{center}
%
The conditional |\ifchilddocmanual| is true whenever
a part to be included by |\input| is being compiled,
and the name of the part is stored in |\childdocname|.

%%%%%%%%%%%%%%%%%%%%%%%%%%%%%%%%%%%%%%%%
\DescribeMacro{\childdocby}
Each part to be included by |\input| should start with:
%
\begin{center}
\begin{tabular}{l}
|\input{childdoc.def}|\\
|\childdocby{|\textit{main}|}|\\
\end{tabular}
\end{center}
%
The directive |\childdocby| is similar to |\childdocof|
described in \secref{sec:include},
but the subsequent selection of content must be done manually.
To that end, both |\ifchilddoc| and |\ifchilddocmanual|
will be true upon processing of a part,
and the name of the part is stored in |\childdocname|.
Note that |\jobname| will be set to the filename of the current part
so that each part receives an individual |.aux| file
that does not interfere with the |.aux| file(s) of the main document.
This behaviour can be altered by the alternative form
|\childdocby[*]{|\textit{main}|}| (with a non-empty optional argument)
which uses the |.aux| file of the main document
by setting |\jobname| to \textit{main}.

%%%%%%%%%%%%%%%%%%%%%%%%%%%%%%%%%%%%%%%%%%%%%%%%%%%%%%%%%%%%%%%%%%%%%%%%%%%%%%%%
\subsection{Driver Development}
\label{sec:driver}

The \textsf{childdoc} mechanism can also be use for the development
of definition files such as \LaTeX{} styles or classes.
This case differs from the above setup with multiple parts
included by |\include| in that no |\includeonly| should be invoked.
This can be achieved by starting the include file
(before |\ProvidesPackage|) with:
%
\begin{center}
\begin{tabular}{l}
|\input{childdoc.def}|\\
|\childdocforward{|\textit{main}|}|\\
\end{tabular}
\end{center}
%
or alternatively with:
%
\begin{center}
\begin{tabular}{l}
|\input{childdoc.def}|\\
|\childdocby{|\textit{main}|}|\\
\end{tabular}
\end{center}
%
Both forms have slightly different effects as described above.
The main file is prepared as usual, see \secref{sec:include}.

%%%%%%%%%%%%%%%%%%%%%%%%%%%%%%%%%%%%%%%%%%%%%%%%%%%%%%%%%%%%%%%%%%%%%%%%%%%%%%%%
\subsection{Legacy Detection}
\label{sec:detection}

The directive |\childdocmain| in the main file can detect
whether the complete document or merely a child is to be compiled
even without using the directive |\childdocof|.
This method is deprecated because it is less robust
and there is no compelling reason to use it;
it is merely provided for backward compatibility
and it may be removed in future versions.

If the detection mechanism is to be used,
it is mandatory to correctly specify
the filename of the main file as the argument of |\childdocmain|:
%
\begin{center}
\begin{tabular}{l}
|\input{childdoc.def}|\\
|\childdocmain{|\textit{main}|}|\\
\end{tabular}
\end{center}
%
If |\jobname| does not match the argument \textit{main} of |\childdocmain|,
it is assumed that |\jobname| points to the child file to be compiled.
When using |\childdocmain| with the main file specified as argument,
it suffices to start a child file
with just |\input{|\textit{main}|}|
without loading of the package and using |\childdocof|.
If instead all processing is done
with the appropriate \textsf{childdoc} directives,
the argument of \textit{main} of |\childdocmain| can be empty.

An alternative version of the command line processing described
in \secref{sec:commandline} using the detection mechanism reads:
%
\begin{center}
|... -jobname "|\textit{target}|" "|[\textit{flags}]%
[|\def\jobname{|\textit{dest}|}|]|\input{|\textit{main}|}"|
\end{center}

%%%%%%%%%%%%%%%%%%%%%%%%%%%%%%%%%%%%%%%%%%%%%%%%%%%%%%%%%%%%%%%%%%%%%%%%%%%%%%%%
\subsection{Manual Code}
\label{sec:manual}

In case one cannot be certain whether the definitions file |childdoc.def|
is installed on the target \TeX{} distribution
and one prefers not to ship it,
it is conceivable to paste a few relevant commands into the sources.

To that end, drop all statements |\input{childdoc.def}|
and perform the replacements as outlined below.
Instead of |\childdocmain{|\textit{main}|}| add the following code
to the top of the main file:
%
\begin{center}
\begin{tabular}{l}
|\||ifdefined\childdocname\endinput\||fi\newif\ifchilddoc|\\
|\edef\childdocname{\scantokens\expandafter{\jobname\noexpand}}|\\
|\def\childdocmain{|\textit{main}|}\||ifx\childdocmain\childdocname\||else|\\
|\childdoctrue\includeonly{\childdocname}\let\jobname\childdocmain\||fi|\\
\end{tabular}
\end{center}
%
Instead of |\childdocof{|\textit{main}|}| just include the main file
at the top of each child file:
%
\begin{center}
|\input{|\textit{main}|}|
\end{center}
%
A simple redirection |\childdocforward{|\textit{dest}|}| is achieved by:
%
\begin{center}
|\def\jobname{|\textit{dest}|}\input{\jobname}|
\end{center}
%
The redirection with prefix
|\childdocforwardprefix[|\textit{prefix}|]{|\textit{dest}|}|
is accomplished by:
%
\begin{center}
\begin{tabular}{l}
|{\edef\jobname{\scantokens\expandafter{\jobname\noexpand}}|\\
|\def\redirectjob |\textit{prefix}|#1~~~{\gdef\jobname{|\textit{dest}|#1}}|\\
|\expandafter\redirectjob\jobname~~~}\input{\jobname}|
\end{tabular}
\end{center}

In an alternative approach,
child documents can be compiled by a specific command line
without additional code or specific definitions:
%
\begin{center}
|... -jobname "|\textit{target}|" "|[\textit{flags}]%
|\includeonly{|\textit{dest}|}\input{|\textit{main}|}"|
\end{center}
%

%%%%%%%%%%%%%%%%%%%%%%%%%%%%%%%%%%%%%%%%%%%%%%%%%%%%%%%%%%%%%%%%%%%%%%%%%%%%%%%%
%%%%%%%%%%%%%%%%%%%%%%%%%%%%%%%%%%%%%%%%%%%%%%%%%%%%%%%%%%%%%%%%%%%%%%%%%%%%%%%%
\section{Information}

%%%%%%%%%%%%%%%%%%%%%%%%%%%%%%%%%%%%%%%%%%%%%%%%%%%%%%%%%%%%%%%%%%%%%%%%%%%%%%%%
\subsection{Copyright}

Copyright \copyright{} 2017--2018 Niklas Beisert

This work may be distributed and/or modified under the
conditions of the \LaTeX{} Project Public License, either version 1.3
of this license or (at your option) any later version.
The latest version of this license is in
  \url{http://www.latex-project.org/lppl.txt}
and version 1.3 or later is part of all distributions of \LaTeX{}
version 2005/12/01 or later.

This work has the LPPL maintenance status `maintained'.

The Current Maintainer of this work is Niklas Beisert.

This work consists of the files |README.txt|, |childdoc.ins| and |childdoc.dtx|
as well as the derived files |childdoc.def|, |cdocsamp.tex|
with |cdocsch1.tex|, |cdocsch2.tex|, |cdocspt3.tex|, |cdocspt4.tex|,
|cdocsdrf.tex|, |cdocsfn1.tex|, |cdocsfn2.tex|
as well as |childdoc.pdf|.

%%%%%%%%%%%%%%%%%%%%%%%%%%%%%%%%%%%%%%%%%%%%%%%%%%%%%%%%%%%%%%%%%%%%%%%%%%%%%%%%
\subsection{Files and Installation}

The package consists of the files:
%
\begin{center}
\begin{tabular}{ll}
    |README.txt|   & readme file \\
    |childdoc.ins| & installation file \\
    |childdoc.dtx| & source file \\
    |childdoc.def| & definition file \\
    |cdocsamp.tex| & sample main file \\
    |cdocsch1.tex| & sample include file \\
    |cdocsch2.tex| & sample include file \\
    |cdocspt3.tex| & sample part file \\
    |cdocspt4.tex| & sample part file \\
    |cdocsdrf.tex| & sample redirection file \\
    |cdocsfn1.tex| & sample redirection file \\
    |cdocsfn2.tex| & sample redirection file \\
    |childdoc.pdf| & manual
\end{tabular}
\end{center}
%
The distribution consists of the files
|README.txt|, |childdoc.ins| and |childdoc.dtx|.
%
\begin{itemize}
\item
Run (pdf)\LaTeX{} on |childdoc.dtx|
to compile the manual |childdoc.pdf| (this file).
\item
Run \LaTeX{} on |childdoc.ins| to create the definitions file |childdoc.def|
and the sample |cdocsamp.tex| with include files
|cdocsch1.tex|, |cdocsch2.tex|, |cdocspt3.tex|, |cdocspt4.tex|,
|cdocsdrf.tex|, |cdocsfn1.tex|, |cdocsfn2.tex|.
Then copy the file |childdoc.def| to an appropriate directory of your \LaTeX{}
distribution, e.g.\ \textit{texmf-root}|/tex/latex/childdoc|.
\end{itemize}

%%%%%%%%%%%%%%%%%%%%%%%%%%%%%%%%%%%%%%%%%%%%%%%%%%%%%%%%%%%%%%%%%%%%%%%%%%%%%%%%
\subsection{Related CTAN Packages}

There are several other packages which offer a similar functionality:
%
\begin{itemize}
\item
The packages
\href{http://ctan.org/pkg/docmute}{\textsf{docmute}},
\href{http://ctan.org/pkg/includex}{\textsf{includex}} and
\href{http://ctan.org/pkg/standalone}{\textsf{standalone}}
provide commands to include only the document body of
a child file thus allowing both files to be compiled individually.
\item
The packages \href{http://ctan.org/pkg/subdocs}{\textsf{subdocs}}
and \href{http://ctan.org/pkg/subfiles}{\textsf{subfiles}}
provide structures in which the main and child documents can be
encapsulated and allowing them to be compiled individually.
The inclusion mechanism is different from the conventional |\include|.
\item
The package \href{http://ctan.org/pkg/combine}{\textsf{combine}}
is an elaborate solution to combine several documents into one.
\end{itemize}
%
See also the CTAN topic \href{http://ctan.org/topic/subdocs}{\textsf{subdocs}}
for further related packages.
The present package differs from the above solutions in that
a document structure constructed with the conventional |\include| mechanism
just needs two extra commands at the top of every file
such that all constituent files can be compiled individually.

%%%%%%%%%%%%%%%%%%%%%%%%%%%%%%%%%%%%%%%%%%%%%%%%%%%%%%%%%%%%%%%%%%%%%%%%%%%%%%%%
%\subsection{Feature Suggestions}
%
%The following is a list of features which may be useful for future
%versions of this package:
%%
%\begin{itemize}
%\item
%\ldots
%\end{itemize}

%%%%%%%%%%%%%%%%%%%%%%%%%%%%%%%%%%%%%%%%%%%%%%%%%%%%%%%%%%%%%%%%%%%%%%%%%%%%%%%%
\subsection{Revision History}

%%%%%%%%%%%%%%%%%%%%%%%%%%%%%%%%%%%%%%%%
\paragraph{v2.0:} 2018/12/30

\begin{itemize}
\item
immediate forward processing
\item
added |\childdocby| mechanism
\item
manual restructured
\end{itemize}

%%%%%%%%%%%%%%%%%%%%%%%%%%%%%%%%%%%%%%%%
\paragraph{v1.6:} 2018/01/17

\begin{itemize}
\item
application for development of include files
\item
corrections to manual
\end{itemize}

%%%%%%%%%%%%%%%%%%%%%%%%%%%%%%%%%%%%%%%%
\paragraph{v1.5:} 2017/05/21

\begin{itemize}
\item
more complete structuring introduced
\item
|\childdocof| introduced
\item
|\childdoc| renamed to |\childdocmain|
\item
|\childredirect| renamed to |\childdocforward| and |\childdocforwardprefix|
and functionality expanded
\end{itemize}

%%%%%%%%%%%%%%%%%%%%%%%%%%%%%%%%%%%%%%%%
\paragraph{v1.0:} 2017/04/27

\begin{itemize}
\item
manual and install package
\item
first version published on CTAN
\end{itemize}

%%%%%%%%%%%%%%%%%%%%%%%%%%%%%%%%%%%%%%%%
\paragraph{v0.6:} 2017/04/26

\begin{itemize}
\item
redirection mechanism added
\end{itemize}

%%%%%%%%%%%%%%%%%%%%%%%%%%%%%%%%%%%%%%%%
\paragraph{v0.5:} 2017/04/26

\begin{itemize}
\item
functionality in definition file
\end{itemize}


%%%%%%%%%%%%%%%%%%%%%%%%%%%%%%%%%%%%%%%%%%%%%%%%%%%%%%%%%%%%%%%%%%%%%%%%%%%%%%%%
%%%%%%%%%%%%%%%%%%%%%%%%%%%%%%%%%%%%%%%%%%%%%%%%%%%%%%%%%%%%%%%%%%%%%%%%%%%%%%%%
%%%%%%%%%%%%%%%%%%%%%%%%%%%%%%%%%%%%%%%%%%%%%%%%%%%%%%%%%%%%%%%%%%%%%%%%%%%%%%%%
\appendix

\settowidth\MacroIndent{\rmfamily\scriptsize 000\ }

 \DocInput{childdoc.dtx}

\end{document}
%</driver>
% \fi
%
% %%%%%%%%%%%%%%%%%%%%%%%%%%%%%%%%%%%%%%%%%%%%%%%%%%%%%%%%%%%%%%%%%%%%%%%%%%%%%%
% %%%%%%%%%%%%%%%%%%%%%%%%%%%%%%%%%%%%%%%%%%%%%%%%%%%%%%%%%%%%%%%%%%%%%%%%%%%%%%
% \section{Sample}
%\iffalse
%<*samplemain>
%\fi
%
% The following presents a sample document
% with two chapters, two parts, a title page,
% a compile flag as well as three forwarding files to set the flag.
% It consists of eight |.tex| files:
% \begin{center}
% \begin{tabular}{ll}
% |cdocsamp.tex|&main file\\
% |cdocsch1.tex|&include file for chapter 1\\
% |cdocsch2.tex|&include file for chapter 2\\
% |cdocspt3.tex|&include file for part 3\\
% |cdocspt4.tex|&include file for part 4\\
% |cdocsdrf.tex|&forwarding file for main file in draft mode\\
% |cdocsfi1.tex|&forwarding file for final version of chapter 1\\
% |cdocsfi2.tex|&forwarding file for final version of chapter 2\\
% \end{tabular}
% \end{center}
% Each of the eight files can be compiled directly by the \LaTeX{} compiler.
%
% %%%%%%%%%%%%%%%%%%%%%%%%%%%%%%%%%%%%%%
% \paragraph{Main File.}
%
% The main file is called |cdocsamp.tex|.
%
% Load the \textsf{childdoc} definitions and
% declare the filename for the main document:
%    \begin{macrocode}
\input{childdoc.def}
\childdocmain{}
%    \end{macrocode}

% Optional override for |\version| flag:
%    \begin{macrocode}
%%\ifchilddoc\else\providecommand{\version}{draft}\fi
%    \end{macrocode}

% Define the default values for the |\version| flag
% (|final| for the main file and |draft| for childs):
%    \begin{macrocode}
\ifchilddoc
\providecommand{\version}{draft}
\else
\providecommand{\version}{final}
\fi
%    \end{macrocode}

% Load the standard document class:
%    \begin{macrocode}
\documentclass[12pt]{article}
%    \end{macrocode}

% Start the document body:
%    \begin{macrocode}
\begin{document}
%    \end{macrocode}

% Declare a title page.
% Print title, part of document being processed and version flag:
%    \begin{macrocode}
\addtocounter{page}{-1}
\begin{center}
{\LARGE\bfseries{}childdoc example\par}
\vspace{1cm}
\ifchilddoc
\ifchilddocmanual part\else chapter\fi:
`\childdocname' of `\childdocjob'\par
\else
main document: `\childdocjob'\par
\fi
version: \version\par
\end{center}
\newpage
%    \end{macrocode}

% Manually include selected file,
% otherwise process as usual:
%    \begin{macrocode}
\ifchilddocmanual
\section*{part `\childdocname'}
\input{\childdocname}
\else
%    \end{macrocode}

% Include the two chapters:
%    \begin{macrocode}
\include{cdocsch1}
\include{cdocsch2}
%    \end{macrocode}

% Include the two parts unless only chapters should be displayed:
%    \begin{macrocode}
\ifchilddoc\else
\section{part three}
\input{cdocspt3}
\section{part four}
\input{cdocspt4}
\fi
%    \end{macrocode}

% Process as usual until here:
%    \begin{macrocode}
\fi
%    \end{macrocode}

% End of document body:
%    \begin{macrocode}
\end{document}
%    \end{macrocode}
%\iffalse
%</samplemain>
%\fi
%
% %%%%%%%%%%%%%%%%%%%%%%%%%%%%%%%%%%%%%%
% \paragraph{Chapter Include Files.}
%
% The include files are called |cdocsch1.tex| and |cdocsch2.tex|.
%
%\iffalse
%<*samplechap1|samplechap2>
%\fi

% Optional override for |\version| flag:
%    \begin{macrocode}
%%\providecommand{\version}{final}
%    \end{macrocode}

% Include the main document:
%    \begin{macrocode}
\input{childdoc.def}
\childdocof{cdocsamp}
%    \end{macrocode}

%\iffalse
%</samplechap1|samplechap2>
%\fi
%
%\iffalse
%<*samplechap1>
%\fi
% Some text for chapter 1:
%    \begin{macrocode}
\section{one}
some text in chapter one
%    \end{macrocode}

%\iffalse
%</samplechap1>
%\fi
% Some text for chapter 2:
%\iffalse
%<*samplechap2>
%\fi
%    \begin{macrocode}
\section{two}
more text in chapter two
%    \end{macrocode}

%\iffalse
%</samplechap2>
%\fi
%
% %%%%%%%%%%%%%%%%%%%%%%%%%%%%%%%%%%%%%%
% \paragraph{Part Include Files.}
%
% The include files are called |cdocspt3.tex| and |cdocspt4.tex|.
%
%\iffalse
%<*samplepart3|samplepart4>
%\fi

% Optional override for |\version| flag:
%    \begin{macrocode}
%%\providecommand{\version}{final}
%    \end{macrocode}

% Include the main document:
%    \begin{macrocode}
\input{childdoc.def}
\childdocby{cdocsamp}
%    \end{macrocode}

%\iffalse
%</samplepart3|samplepart4>
%\fi
%
%\iffalse
%<*samplepart3>
%\fi
% Some text for part 3:
%    \begin{macrocode}
some text in part three
%    \end{macrocode}

%\iffalse
%</samplepart3>
%\fi
% Some text for part 4:
%\iffalse
%<*samplepart4>
%\fi
%    \begin{macrocode}
more text in part four
%    \end{macrocode}

%\iffalse
%</samplepart4>
%\fi
%
% %%%%%%%%%%%%%%%%%%%%%%%%%%%%%%%%%%%%%%
% \paragraph{Forwarding for a Complete Draft.}
%
% The following forwarding file |cdocsdrf.tex|
% compiles the main document in draft mode:
%\iffalse
%<*sampledraft>
%\fi
%    \begin{macrocode}
\def\version{draft}
\input{childdoc.def}
\childdocforward{cdocsamp}
%    \end{macrocode}

%\iffalse
%</sampledraft>
%\fi
%
% %%%%%%%%%%%%%%%%%%%%%%%%%%%%%%%%%%%%%%
% \paragraph{Forwarding for Final Version of the Chapters.}
%
% The following forwarding files |cdocsfn1.tex| and |cdocsfn2.tex|
% (with identical content)
% compile the final versions of the child documents
% |cdocsch1.tex| and |cdocsch2.tex|, respectively:
%\iffalse
%<*samplefinal>
%\fi
%    \begin{macrocode}
\def\version{final}
\input{childdoc.def}
\childdocforwardprefix[cdocsamp]{cdocsfn}{cdocsch}
%    \end{macrocode}

%\iffalse
%</samplefinal>
%\fi
%
% %%%%%%%%%%%%%%%%%%%%%%%%%%%%%%%%%%%%%%
% \paragraph{Command Line Processing.}
%
% The following three command lines generate the output files
% |cdocscld|, |cdocscl1| and |cdocscl2|
% which should be identical to
% |cdocsdrf|, |cdocsch1| and |cdocsfn2|, respectively:
% \begin{center}
% \begin{tabular}{l}
% |latex -jobname cdocscld \|\\
% |  "\def\version{draft}\input{childdoc.def}\childdocforward{cdocsamp}"|\\
% |latex -jobname cdocscl1 \|\\
% |  "\input{childdoc.def}\childdocforward[cdocsamp]{cdocsch1}"|\\
% |latex -jobname cdocscl2 \|\\
% |  "\def\version{final}\input{childdoc.def}\childdocforward{cdocsch2}"|
% \end{tabular}
% \end{center}
% Note that the trailing backslash on each first line
% merely continues the input to the second line
% (for convenient cut ant paste).
% Furthermore, the command |latex| can be replaced by any
% of its alternative versions such as |pdflatex|.
%
% %%%%%%%%%%%%%%%%%%%%%%%%%%%%%%%%%%%%%%%%%%%%%%%%%%%%%%%%%%%%%%%%%%%%%%%%%%%%%%
% %%%%%%%%%%%%%%%%%%%%%%%%%%%%%%%%%%%%%%%%%%%%%%%%%%%%%%%%%%%%%%%%%%%%%%%%%%%%%%
% \section{Implementation}
%\iffalse
%<*package>
%\fi
%
% This section describes the definitions file |childdoc.def|.

% The definitions cannot be loaded using |\usepackage| or |\RequirePackage|
% which has a mechanism to prevent loading a style file more than once.
% When loading the definitions by means of |\input|
% multiple instances have to be prevented manually:
%\iffalse
%This code needs to be before the `\ProvidesFile' directive
%which is defined at the beginning of this file.
%Therefore it is also placed there and commented out here.
%</package>
%<*discard>
%\fi
%    \begin{macrocode}
\ifdefined\childdocmain\endinput\fi
%    \end{macrocode}
%\iffalse
%</discard>
%<*package>
%\fi
%
% \macro{\ifchilddoc}
% \macro{\ifchilddocmanual}
% The conditional |\ifchilddoc| tells whether a
% child (true) or main (false) document is being compiled.
% The conditional |\ifchilddocmanual| tells whether
% the |\includeonly| mechanism is used (false) or
% the selection of child files must be performed manually (true).
% The definitions initialise to false:
%    \begin{macrocode}
\newif\ifchilddoc
\newif\ifchilddocmanual
%    \end{macrocode}

% \macro{\childdocname}
% \macro{\childdocjob}
% The macro |\childdocname| stores the name of the main document
% to be compiled. The macro |\childdocjob| stores the name of
% the document on which the \LaTeX{} compiler was originally invoked.
% The content of |\jobname| cannot be compared
% to filenames specified in the source due to different catcodes.
% The following code rescans |\jobname|, stores the result
% in |\childdocname| and saves a copy in |\childdocjob|:
%    \begin{macrocode}
\edef\childdocname{\scantokens\expandafter{\jobname\noexpand}}
\let\childdocjob\childdocname
%    \end{macrocode}

% \macro{\childdocdisable}
% The macro |\childdocdisable| prevents the main file
% from being processed more than once.
% At this stage, the main document command |\childdocmain|
% is assumed to be called once again where it should do nothing.
% Any subsequent call to it should prevent
% a secondary processing of the main document
% It overwrites the forwarding commands
% |\childdocof| and |\childdocforward|
% with empty macros to prevent further inclusions of the main document:
%    \begin{macrocode}
\newcommand{\childdocdisable}
{
  \renewcommand{\childdocmain}[1]{\renewcommand{\childdocmain}[1]{\endinput}}
  \renewcommand{\childdocof}[1]{}
  \renewcommand{\childdocby}[2][]{}
  \renewcommand{\childdocforward}[2][]{}
  \renewcommand{\childdocdisable}{}
}
%    \end{macrocode}

% \macro{\childdocmain}
% The macro |\childdocmain| is to be called at the top of the main file
% with nothing or the main filename (without extension) as argument.
% First, it breaks loops.
% If the argument is not empty and does not match |\childdocname|
% (which is set by the first inclusion of |childdoc.def|),
% |\ifchilddoc| is set to true, |\includeonly| is applied to the child file
% and |\jobname| is set to the main file
% (for proper handling of |.aux| files):
%    \begin{macrocode}
\newcommand{\childdocmain}[1]
{
  \childdocdisable\childdocmain{}
  \if?#1?\else
    \begingroup
      \def\childdoctmp{#1}
      \ifx\childdoctmp\childdocname
        \def\childdoctmp{}
      \else
        \def\childdoctmp
        {
          \childdoctrue
          \includeonly{\childdocname}
          \def\childdocjob{#1}
          \def\jobname{#1}
        }
      \fi
      \expandafter
    \endgroup
    \childdoctmp
  \fi
}
%    \end{macrocode}

% \macro{\childdocof}
% The command |\childdocof| redirects
% compilation to the main file |#1|.
%    \begin{macrocode}
\newcommand{\childdocof}[1]
{
  \childdocdisable
  \childdoctrue
  \includeonly{\childdocname}
  \def\jobname{#1}
  \def\childdocjob{#1}
  \input{#1}
}
%    \end{macrocode}

% \macro{\childdocby}
% The command |\childdocby| ....
%    \begin{macrocode}
\newcommand{\childdocby}[2][]
{
  \childdocdisable
  \childdoctrue
  \childdocmanualtrue
  \if?#1?\else
    \def\jobname{#2}
  \fi
  \def\childdocjob{#2}
  \input{#2}
  \endinput
}
%    \end{macrocode}

% \macro{\childdocforward}
% The command |\childdocforward| redirects
% compilation to the main file or
% (if the optional argument is given) a child file.
% Parameters are set as if the main file
% or a child file starting with |\childdocof| was compiled.
% Then compilation is handed over to the main file:
%    \begin{macrocode}
\newcommand{\childdocforward}[2][]
{
  \begingroup
    \if?#1?
      \def\childdoctmp
      {
        \def\childdocname{#2}
        \def\childdocjob{#2}
        \def\jobname{#2}
        \input{#2}
        \endinput
      }
    \else
      \def\childdoctmp
      {
        \childdocdisable
        \def\childdocname{#2}
        \childdoctrue
        \includeonly{#2}
        \def\childdocjob{#1}
        \def\jobname{#1}
        \input{#1}
        \endinput
      }
    \fi
    \expandafter
  \endgroup
  \childdoctmp
}
%    \end{macrocode}

% \macro{\childdocforwardprefix}
% The command |\childdocforwardprefix| redirects
% compilation to the main or a child file by means of a pattern.
% The prefix |#1| in the current filename is replaced by |#2|
% and the suffix of the current filename is kept
% (it is assumed that the filename does not contain the substring `|~~~|'
% which is used as a delimiter).
% Compilation is handed over to the new file by |\childdocforward|:
%    \begin{macrocode}
\newcommand{\childdocforwardprefix}[3][]
{
  \begingroup
    \def\childdocextract #2##1~~~{\def\childdoctmp{\childdocforward[#1]{#3##1}}}
    \expandafter\childdocextract\childdocname~~~
    \expandafter
  \endgroup
  \childdoctmp
}
%    \end{macrocode}

% \macro{\childdoc}
% The deprecated macro |\childdoc| is a legacy version of |\childdocmain|:
%    \begin{macrocode}
\newcommand{\childdoc}{\childdocmain}
%    \end{macrocode}

% \macro{\childdocredirect}
% The deprecated macro |\childdocredirect| is a legacy version
% of |\childdocforward| and |\childdocforwardprefix|:
%    \begin{macrocode}
\newcommand{\childdocredirect}[2][]
{
  \begingroup
    \if?#1?
      \def\childdoctmp{\childdocforward{#2}}
    \else
      \def\childdoctmp{\childdocforwardprefix{#1}{#2}}
    \fi
    \expandafter
  \endgroup
  \childdoctmp
}
%    \end{macrocode}

%\iffalse
%</package>
%\fi
%
\endinput
\childdocforward[cdocsamp]{cdocsch1}"|\\
% |latex -jobname cdocscl2 \|\\
% |  "\def\version{final}% \iffalse
%
% childdoc.dtx Copyright (C) 2017-2018 Niklas Beisert
%
% This work may be distributed and/or modified under the
% conditions of the LaTeX Project Public License, either version 1.3
% of this license or (at your option) any later version.
% The latest version of this license is in
%   http://www.latex-project.org/lppl.txt
% and version 1.3 or later is part of all distributions of LaTeX
% version 2005/12/01 or later.
%
% This work has the LPPL maintenance status `maintained'.
%
% The Current Maintainer of this work is Niklas Beisert.
%
% This work consists of the files childdoc.dtx and childdoc.ins
% and the derived files childdoc.def and cdocsamp.tex with
% cdocsch1.tex, cdocsch2.tex, cdocsdrf.tex, cdocsfn1.tex, cdocsfn2.tex.
%
%<package>\ifdefined\childdocmain\endinput\fi
%<package>\ProvidesFile{childdoc.def}[2018/12/30 v2.0 child document driver]
%<samplemain>\ProvidesFile{cdocsamp.tex}[2018/12/30 v2.0 sample for childdoc]
%<*driver>
%\ProvidesFile{childdoc.drv}[2018/12/30 v2.0 childdoc reference manual file]
\PassOptionsToClass{10pt,a4paper}{article}
\documentclass{ltxdoc}

\usepackage[margin=35mm]{geometry}
\usepackage{hyperref}
\usepackage{hyperxmp}
\usepackage[usenames]{color}

\hypersetup{colorlinks=true}
\hypersetup{pdfstartview=FitH}
\hypersetup{pdfpagemode=UseNone}
\hypersetup{pdfsource={}}
\hypersetup{pdflang={en-UK}}
\hypersetup{pdfcopyright={Copyright 2017-2018 Niklas Beisert.
  This work may be distributed and/or modified under the
  conditions of the LaTeX Project Public License, either version 1.3
  of this license or (at your option) any later version.}}
\hypersetup{pdflicenseurl={http://www.latex-project.org/lppl.txt}}
\hypersetup{pdfcontactaddress={ETH Zurich, ITP, HIT K,
  Wolfgang-Pauli-Strasse 27}}
\hypersetup{pdfcontactpostcode={8093}}
\hypersetup{pdfcontactcity={Zurich}}
\hypersetup{pdfcontactcountry={Switzerland}}
\hypersetup{pdfcontactemail={nbeisert@itp.phys.ethz.ch}}
\hypersetup{pdfcontacturl={http://people.phys.ethz.ch/\xmptilde nbeisert/}}

\newcommand{\secref}[1]{\hyperref[#1]{section \ref*{#1}}}

\parskip1ex
\parindent0pt
\let\olditemize\itemize
\def\itemize{\olditemize\parskip0pt}

\begin{document}

\title{The \textsf{childdoc} Package}
\hypersetup{pdftitle={The childdoc Package}}
\author{Niklas Beisert\\[2ex]
  Institut f\"ur Theoretische Physik\\
  Eidgen\"ossische Technische Hochschule Z\"urich\\
  Wolfgang-Pauli-Strasse 27, 8093 Z\"urich, Switzerland\\[1ex]
  \href{mailto:nbeisert@itp.phys.ethz.ch}
  {\texttt{nbeisert@itp.phys.ethz.ch}}}
\hypersetup{pdfauthor={Niklas Beisert}}
\hypersetup{pdfsubject={Manual for the LaTeX2e Package childdoc}}
\date{30 December 2018, \textsf{v2.0}}
\maketitle

\begin{abstract}\noindent
\textsf{childdoc} is a \LaTeXe{} package
that enables the direct compilation
of document sections included by |\include|
to individual files.
\end{abstract}

\begingroup
\parskip0ex
\tableofcontents
\endgroup

%%%%%%%%%%%%%%%%%%%%%%%%%%%%%%%%%%%%%%%%%%%%%%%%%%%%%%%%%%%%%%%%%%%%%%%%%%%%%%%%
%%%%%%%%%%%%%%%%%%%%%%%%%%%%%%%%%%%%%%%%%%%%%%%%%%%%%%%%%%%%%%%%%%%%%%%%%%%%%%%%
\section{Introduction}

\LaTeX{} provides a mechanism to structure a large document (such as a book)
into a main file and several child files (containing the chapters)
using the |\include| command.
This mechanism is beneficial for documents
which span hundreds of pages in order to
make the source file(s) more manageable.
Moreover, compilation can be restricted to
selected child files by means of the |\includeonly| command.
The latter feature can be used to reduce the compilation time while editing
(this was significantly more useful in the earlier days of \LaTeX{})
or to generate a smaller document which is easier to navigate.
Another application of |\includeonly| is to generate
documents consisting of selected parts of the complete document.

However, there are a few drawbacks of the plain |\include| mechanism:
\begin{itemize}
\item
The child files cannot be compiled on their own,
they can only be compiled via the main file.
A naive editing environment
(such as a text editor with an option
to have the current file processed by \LaTeX)
may require one to switch to the main file before compiling;
attempting to compile the child file produces errors.
\item
The main file must be modified (each time)
to adjust the |\includeonly| command
to the present needs. This easily leaves the main file in a messy state.
\item
The generated document will always carry the filename
of the main document. This is inconvenient if
several child files are to be compiled and
to be kept for distribution.
\end{itemize}

The present package provides a simple interface
to make child files individually compilable by \LaTeX{}.
Compiling a child file then has the same effect as compiling
the main file with an |\includeonly| command
to select the appropriate child.
Moreover the generated document will carry the name of the child
rather than the main file.
This resolves all three above issues.

This feature is meant to make the editing of books,
thesis documents and lecture notes somewhat more convenient.
However, the package can also be used efficiently for
composing a series of documents (such as exercise sheets)
which are typically distributed individually.
It then assists the author in generating the individual documents
(potentially in different versions)
as well as a document containing the collected series.
Another application is in developing style files
or other kinds of included material
where compilation of the style file could redirect
to a sample or test file.

%%%%%%%%%%%%%%%%%%%%%%%%%%%%%%%%%%%%%%%%%%%%%%%%%%%%%%%%%%%%%%%%%%%%%%%%%%%%%%%%
%%%%%%%%%%%%%%%%%%%%%%%%%%%%%%%%%%%%%%%%%%%%%%%%%%%%%%%%%%%%%%%%%%%%%%%%%%%%%%%%
\section{Usage}

First of all, the package \textsf{childdoc} is \emph{not} a standard
\LaTeXe{} |.sty| style file! Therefore it needs to be invoked in
a non-standard way.

%%%%%%%%%%%%%%%%%%%%%%%%%%%%%%%%%%%%%%%%%%%%%%%%%%%%%%%%%%%%%%%%%%%%%%%%%%%%%%%%
\subsection{Included Files}
\label{sec:include}

%%%%%%%%%%%%%%%%%%%%%%%%%%%%%%%%%%%%%%%%
\DescribeMacro{\childdocmain}
To use the package, add the commands
\begin{center}
\begin{tabular}{l}
|\input{childdoc.def}|\\
|\childdocmain{}|\\
\end{tabular}
\end{center}
at the very top of the main \LaTeX{} file,
in particular \emph{before} the |\documentclass| statement!
The argument of |\childdocmain| should be left empty
(but it must be present).

%%%%%%%%%%%%%%%%%%%%%%%%%%%%%%%%%%%%%%%%
\DescribeMacro{\childdocof}
Furthermore, add the commands
\begin{center}
\begin{tabular}{l}
|\input{childdoc.def}|\\
|\childdocof{|\textit{main}|}|\\
\end{tabular}
\end{center}
at the top of every child file \textit{child}
which is included by |\include{|\textit{child}|}|
from within the main file
(or at least for those files to be compiled individually).
The argument \textit{main} must be the filename of the main file.

There are a couple of
considerations in setting up the main and child documents:

%%%%%%%%%%%%%%%%%%%%%%%%%%%%%%%%%%%%%%%%
\paragraph{Restrictions.}

Please note the following restrictions:
\begin{itemize}
\item
|\childdocmain| must be called with one argument \textit{main}
to ensure compatibility with earlier version of the package.
It must either be empty (|\childdocmain{}|)
or precisely match the filename of the main file in which it is specified.
See \secref{sec:detection} for further information.
\item
The filename \textit{main} must be specified without the |.tex| extension.
\item
The filename \textit{main} is case sensitive
(even in case-insensitive file systems)
due to internal string comparison.
\item
The argument \textit{main} should be fully expanded, it cannot be a macro.
\item
Subdirectories and special characters should be avoided in filenames.
\item
The command |\childdocmain{|\textit{main}|}| must be followed by a whitespace.
It should not be followed immediately by another command
or by a comment mark `|%|'.
This is because the \TeX{} parser reads the token immediately following
the argument of |\childdocmain| and puts it
at the beginning of every child section;
however, a white\-space is ignored.
\end{itemize}

%%%%%%%%%%%%%%%%%%%%%%%%%%%%%%%%%%%%%%%%
\paragraph{Content of Main File.}

It is advisable to place all content in the child files included by |\include|.
Any output contained in the main file will appear in all child documents
unless suppressed manually;
it cannot be suppressed automatically by the |\includeonly| directive
and thus should normally be avoided.
A method to include some content in the main file
by means of conditional processing is described in \secref{sec:conditional}.

%%%%%%%%%%%%%%%%%%%%%%%%%%%%%%%%%%%%%%%%
\paragraph{Page Numbering.}

When only a part of the document is compiled,
the appropriate numbering of pages
(as well as other status parameters)
is determined from the |.aux| files.
The latter contain information from previous passes.
However this information needs to propagate through
all intermediate child documents.
Therefore the page numbering in child documents may well
be inconsistent until the complete document is compiled at least once.

A useful (if unconventional) way to always ensure a consistent
page numbering is to restart the numbering in each child document
and denote the pages by `\textit{child}|.|\textit{page}'
where \textit{child} represents the chapter/section number of the child file.
This can be achieved by the command
|\numberwithin{page}{|\textit{child}|}|
of the \textsf{amsmath} package
where \textit{child} can be |chapter| or |section|
depending on the chosen structuring.
Alternatively, one can modify the macro |\thepage| appropriately
and reset the counter |page| at the start of each child file.

%%%%%%%%%%%%%%%%%%%%%%%%%%%%%%%%%%%%%%%%%%%%%%%%%%%%%%%%%%%%%%%%%%%%%%%%%%%%%%%%
\subsection{Conditional Processing}
\label{sec:conditional}

The package provides a mechanism to compile different versions
of a document. To customise the versions further some conditional processing
can come in handy to distinguish which version is being compiled.
The package provides two macros to describe the compilation context:

%%%%%%%%%%%%%%%%%%%%%%%%%%%%%%%%%%%%%%%%
\DescribeMacro{\ifchilddoc}
The conditional |\ifchilddoc| distinguishes between the compilation of
child documents and the main document:
%
\begin{center}
|\ifchilddoc |\textit{child-code}| |[|\||else |\textit{main-code}]| \||fi|
\end{center}

%%%%%%%%%%%%%%%%%%%%%%%%%%%%%%%%%%%%%%%%
\DescribeMacro{\childdocname}
\DescribeMacro{\childdocjob}
The macro |\childdocname| contains the filename (without extension)
of the main or child file being processed.
Note that |\childdocjob| will always contain the name of the main file.

%%%%%%%%%%%%%%%%%%%%%%%%%%%%%%%%%%%%%%%%
\paragraph{Title Page.}

Conditional processing can be used to include a title or banner page
in the main document when proper precautions are taken.
Importantly, the code in the main file should ensure that the page counter
(as well as other status parameters which are stored in the |.aux| files)
takes the same value after the conditional processing.
Otherwise the page numbers may take divergent values
depending on which part is compiled.

For example, a title page could be declared by:
%
\begin{center}
\begin{tabular}{l}
|\ifchilddoc\||else|\\
|\addtocounter{page}{-1}|\\
\textit{code for title page}\\
|\newpage|\\
|\||fi|
\end{tabular}
\end{center}
%
A banner page for the child documents can be generated by:
%
\begin{center}
\begin{tabular}{l}
|\ifchilddoc|\\
|\addtocounter{page}{-1}|\\
\textit{code for banner page}\\
|\newpage|\\
|\||fi|
\end{tabular}
\end{center}
%
Here one could write a message such as:
\begin{center}
|This is the part \childdocname{} of \childdocjob{}.|
\end{center}

%%%%%%%%%%%%%%%%%%%%%%%%%%%%%%%%%%%%%%%%%%%%%%%%%%%%%%%%%%%%%%%%%%%%%%%%%%%%%%%%
\subsection{Flags}
\label{sec:flags}

The package makes it easy to generate different versions
of the main or child documents.
To this end compilation flags can be defined
and assigned different default values.
They will be particularly useful in conjunction
with the forwarding mechanism described in \secref{sec:forward}.

For example, it may be useful to have a flag |\version|
which can be set to |draft| or |final|.
The document source will contain some conditional code
depending on the value of |\version|.
Suppose further, the flag should default to |final| for the main file
and to |draft| for child files
which is a natural assignment for editing the document.
This is achieved by placing the following code
in the preamble of the main document
(below the |\childdocmain| directive):
%
\begin{center}
\begin{tabular}{l}
|\ifchilddoc|\\
|\providecommand{\version}{draft}|\\
|\||else|\\
|\providecommand{\version}{final}|\\
|\||fi|
\end{tabular}
\end{center}
%
The definition by |\providecommand| makes sure
that previous definitions are not overwritten.
Further statements |\providecommand{\version}{...}|
can thus be added before the above code to override it.

For the main file, one might add a line
(between |\childdocmain| and the above block)
%
\begin{center}
|%\ifchilddoc\||else\providecommand{\version}{draft}\||fi|
\end{center}
%
which can be uncommented to produce a draft version.
Likewise one can add a line to the very top of a child file
(above the |\childdocof{|\textit{main}|}| directive)
%
\begin{center}
|%\providecommand{\version}{final}|
\end{center}
%
which can be uncommented to produce the final version of this child document.

%%%%%%%%%%%%%%%%%%%%%%%%%%%%%%%%%%%%%%%%%%%%%%%%%%%%%%%%%%%%%%%%%%%%%%%%%%%%%%%%
\subsection{Forwarding}
\label{sec:forward}

Different versions of the main or child documents
using compilation flags as described in \secref{sec:flags}
can be (permanently) stored in different files
for convenient compilation, viewing and distribution.
To this end, the package defines a command
to pass on compilation to a different file:

%%%%%%%%%%%%%%%%%%%%%%%%%%%%%%%%%%%%%%%%
\DescribeMacro{\childdocforward}
The command |\childdocforward| redirects processing to
another source file:
%
\begin{center}
\begin{tabular}{l}
|\input{childdoc.def}|\\
|\childdocforward[|\textit{main}|]{|\textit{dest}|}|\\
\end{tabular}
\end{center}
%
The argument \textit{dest} is the destination file
(without extension).
It should be the main file or one of the child files.
Note that further \textsf{childdoc} directives
such as |\childdocof| and |\childdocforward|
in the indicated file will be processed in this form.
The optional argument \textit{main}
passes on directly to the main file \textit{main}
while pretending to compile the child \textit{dest}.
This form behaves as if \textit{dest}
issues |\childdocof{|\textit{main}|}| right away,
and no further \textsf{childdoc} directives will be processed.

%%%%%%%%%%%%%%%%%%%%%%%%%%%%%%%%%%%%%%%%
\DescribeMacro{\...prefix}
In the alternative form |\childdocforwardprefix|,
%
\begin{center}
\begin{tabular}{l}
|\input{childdoc.def}|\\
|\childdocforwardprefix[|\textit{main}|]{|\textit{prefix}|}{|\textit{dest}|}|
\end{tabular}
\end{center}
%
the destination file is determined by a pattern
depending on the current file:
To make this work, the current file must be called
`{\textit{prefix}\hspace{0.2em}\textit{suffix}}'
with \textit{prefix} matching precisely the argument.
Processing is then passed on to the file
`{\textit{dest}\hspace{0.2em}\textit{suffix}}'.
Surely, the same effect is achieved by
directly specifying the
argument `{\textit{dest}\hspace{0.2em}\textit{suffix}}'
in the first form.
However, that requires to set up a different file
for each child. With the alternative form of the command
all these files can have exactly the same content
which simplifies setting them up and maintaining them.

For example, the following file |draft.tex|
with a compilation flag |\version| as described in \secref{sec:flags}
compiles the main document as a draft:
%
\begin{center}
\begin{tabular}{l}
|\def\version{draft}|\\
|\input{childdoc.def}|\\
|\childdocforward{|\textit{main}|}|
\end{tabular}
\end{center}
%
Likewise, the following files |final|\textit{nn}|.tex|
compile the final version of the child document
|child|\textit{nn}|.tex|:
%
\begin{center}
\begin{tabular}{l}
|\def\version{final}|\\
|\input{childdoc.def}|\\
|\childdocforwardprefix{final}{child}|
\end{tabular}
\end{center}
%

Note that when several versions of a main file and/or of each child file
are to be generated, it may be convenient to set up a |Makefile| or
shell script to automatise the process.

%%%%%%%%%%%%%%%%%%%%%%%%%%%%%%%%%%%%%%%%%%%%%%%%%%%%%%%%%%%%%%%%%%%%%%%%%%%%%%%%
\subsection{Command Line Processing}
\label{sec:commandline}

The effect of redirection files can also be achieved by invoking
the \LaTeX{} compiler with a more elaborate command line.
Most conveniently this should be done as part
of a shell script or a |Makefile|.

When using \textsf{childdoc} in the main file, the following
command lines effectively perform a redirection
(note that depending on the shell being used,
backslashes may have to be doubled: `|\|' $\to$ `|\\|'):
%
\begin{center}
|... -jobname "|\textit{target}|" |\\|"|[\textit{flags}]%
|\input{childdoc.def}\childdocforward[|\textit{main}|]{|\textit{dest}|}"|
\end{center}
%
Here \textit{target} is the name of the output file,
\textit{main} is the name of the main file
and \textit{dest} is the name of the main or child file to be processed
(all filenames without extensions).
The optional argument \textit{main} can be omitted
if \textit{main} matches \textit{dest}.
Optionally, compilation \textit{flags} can be defined via |\def| commands.
This command line makes the \TeX{} engine believe
it is compiling the file \textit{target}
whose content is specified as the latter parameter.
The provided code then forwards the processing to
\textit{main} or \textit{dest} as described in \secref{sec:forward}.

%%%%%%%%%%%%%%%%%%%%%%%%%%%%%%%%%%%%%%%%%%%%%%%%%%%%%%%%%%%%%%%%%%%%%%%%%%%%%%%%
\subsection{Include by Input}
\label{sec:input}

Including child documents by |\include| has some restrictions by design.
Most notably, the content of a child document always occupies
its own set of pages; pages cannot be shared between child documents.
Usually, this behaviour makes perfect sense
because each child document contain an essential part of the document.
However, in some situations it may be desirable to compose
a document from a collection of parts
without having mandatory page breaks between then.
For this case, the package
provides a mechanism to include parts
by |\input| which can also be processed individually.
However, by construction this mechanism
requires manual handling of the content to be output.

%%%%%%%%%%%%%%%%%%%%%%%%%%%%%%%%%%%%%%%%
\DescribeMacro{\ifchilddocmanual}
The main file should be prepared as usual, see \secref{sec:include}.
However, the document body must make a distinction
between processing of an individual part and of the main document, e.g.:
%
\begin{center}
\begin{tabular}{l}
|\ifchilddocmanual|\\
|\input{\childdocname}|\\
|\||else|\\
\textit{document body with }|\input{|\textit{part}|}|\\
|\||fi|
\end{tabular}
\end{center}
%
The conditional |\ifchilddocmanual| is true whenever
a part to be included by |\input| is being compiled,
and the name of the part is stored in |\childdocname|.

%%%%%%%%%%%%%%%%%%%%%%%%%%%%%%%%%%%%%%%%
\DescribeMacro{\childdocby}
Each part to be included by |\input| should start with:
%
\begin{center}
\begin{tabular}{l}
|\input{childdoc.def}|\\
|\childdocby{|\textit{main}|}|\\
\end{tabular}
\end{center}
%
The directive |\childdocby| is similar to |\childdocof|
described in \secref{sec:include},
but the subsequent selection of content must be done manually.
To that end, both |\ifchilddoc| and |\ifchilddocmanual|
will be true upon processing of a part,
and the name of the part is stored in |\childdocname|.
Note that |\jobname| will be set to the filename of the current part
so that each part receives an individual |.aux| file
that does not interfere with the |.aux| file(s) of the main document.
This behaviour can be altered by the alternative form
|\childdocby[*]{|\textit{main}|}| (with a non-empty optional argument)
which uses the |.aux| file of the main document
by setting |\jobname| to \textit{main}.

%%%%%%%%%%%%%%%%%%%%%%%%%%%%%%%%%%%%%%%%%%%%%%%%%%%%%%%%%%%%%%%%%%%%%%%%%%%%%%%%
\subsection{Driver Development}
\label{sec:driver}

The \textsf{childdoc} mechanism can also be use for the development
of definition files such as \LaTeX{} styles or classes.
This case differs from the above setup with multiple parts
included by |\include| in that no |\includeonly| should be invoked.
This can be achieved by starting the include file
(before |\ProvidesPackage|) with:
%
\begin{center}
\begin{tabular}{l}
|\input{childdoc.def}|\\
|\childdocforward{|\textit{main}|}|\\
\end{tabular}
\end{center}
%
or alternatively with:
%
\begin{center}
\begin{tabular}{l}
|\input{childdoc.def}|\\
|\childdocby{|\textit{main}|}|\\
\end{tabular}
\end{center}
%
Both forms have slightly different effects as described above.
The main file is prepared as usual, see \secref{sec:include}.

%%%%%%%%%%%%%%%%%%%%%%%%%%%%%%%%%%%%%%%%%%%%%%%%%%%%%%%%%%%%%%%%%%%%%%%%%%%%%%%%
\subsection{Legacy Detection}
\label{sec:detection}

The directive |\childdocmain| in the main file can detect
whether the complete document or merely a child is to be compiled
even without using the directive |\childdocof|.
This method is deprecated because it is less robust
and there is no compelling reason to use it;
it is merely provided for backward compatibility
and it may be removed in future versions.

If the detection mechanism is to be used,
it is mandatory to correctly specify
the filename of the main file as the argument of |\childdocmain|:
%
\begin{center}
\begin{tabular}{l}
|\input{childdoc.def}|\\
|\childdocmain{|\textit{main}|}|\\
\end{tabular}
\end{center}
%
If |\jobname| does not match the argument \textit{main} of |\childdocmain|,
it is assumed that |\jobname| points to the child file to be compiled.
When using |\childdocmain| with the main file specified as argument,
it suffices to start a child file
with just |\input{|\textit{main}|}|
without loading of the package and using |\childdocof|.
If instead all processing is done
with the appropriate \textsf{childdoc} directives,
the argument of \textit{main} of |\childdocmain| can be empty.

An alternative version of the command line processing described
in \secref{sec:commandline} using the detection mechanism reads:
%
\begin{center}
|... -jobname "|\textit{target}|" "|[\textit{flags}]%
[|\def\jobname{|\textit{dest}|}|]|\input{|\textit{main}|}"|
\end{center}

%%%%%%%%%%%%%%%%%%%%%%%%%%%%%%%%%%%%%%%%%%%%%%%%%%%%%%%%%%%%%%%%%%%%%%%%%%%%%%%%
\subsection{Manual Code}
\label{sec:manual}

In case one cannot be certain whether the definitions file |childdoc.def|
is installed on the target \TeX{} distribution
and one prefers not to ship it,
it is conceivable to paste a few relevant commands into the sources.

To that end, drop all statements |\input{childdoc.def}|
and perform the replacements as outlined below.
Instead of |\childdocmain{|\textit{main}|}| add the following code
to the top of the main file:
%
\begin{center}
\begin{tabular}{l}
|\||ifdefined\childdocname\endinput\||fi\newif\ifchilddoc|\\
|\edef\childdocname{\scantokens\expandafter{\jobname\noexpand}}|\\
|\def\childdocmain{|\textit{main}|}\||ifx\childdocmain\childdocname\||else|\\
|\childdoctrue\includeonly{\childdocname}\let\jobname\childdocmain\||fi|\\
\end{tabular}
\end{center}
%
Instead of |\childdocof{|\textit{main}|}| just include the main file
at the top of each child file:
%
\begin{center}
|\input{|\textit{main}|}|
\end{center}
%
A simple redirection |\childdocforward{|\textit{dest}|}| is achieved by:
%
\begin{center}
|\def\jobname{|\textit{dest}|}\input{\jobname}|
\end{center}
%
The redirection with prefix
|\childdocforwardprefix[|\textit{prefix}|]{|\textit{dest}|}|
is accomplished by:
%
\begin{center}
\begin{tabular}{l}
|{\edef\jobname{\scantokens\expandafter{\jobname\noexpand}}|\\
|\def\redirectjob |\textit{prefix}|#1~~~{\gdef\jobname{|\textit{dest}|#1}}|\\
|\expandafter\redirectjob\jobname~~~}\input{\jobname}|
\end{tabular}
\end{center}

In an alternative approach,
child documents can be compiled by a specific command line
without additional code or specific definitions:
%
\begin{center}
|... -jobname "|\textit{target}|" "|[\textit{flags}]%
|\includeonly{|\textit{dest}|}\input{|\textit{main}|}"|
\end{center}
%

%%%%%%%%%%%%%%%%%%%%%%%%%%%%%%%%%%%%%%%%%%%%%%%%%%%%%%%%%%%%%%%%%%%%%%%%%%%%%%%%
%%%%%%%%%%%%%%%%%%%%%%%%%%%%%%%%%%%%%%%%%%%%%%%%%%%%%%%%%%%%%%%%%%%%%%%%%%%%%%%%
\section{Information}

%%%%%%%%%%%%%%%%%%%%%%%%%%%%%%%%%%%%%%%%%%%%%%%%%%%%%%%%%%%%%%%%%%%%%%%%%%%%%%%%
\subsection{Copyright}

Copyright \copyright{} 2017--2018 Niklas Beisert

This work may be distributed and/or modified under the
conditions of the \LaTeX{} Project Public License, either version 1.3
of this license or (at your option) any later version.
The latest version of this license is in
  \url{http://www.latex-project.org/lppl.txt}
and version 1.3 or later is part of all distributions of \LaTeX{}
version 2005/12/01 or later.

This work has the LPPL maintenance status `maintained'.

The Current Maintainer of this work is Niklas Beisert.

This work consists of the files |README.txt|, |childdoc.ins| and |childdoc.dtx|
as well as the derived files |childdoc.def|, |cdocsamp.tex|
with |cdocsch1.tex|, |cdocsch2.tex|, |cdocspt3.tex|, |cdocspt4.tex|,
|cdocsdrf.tex|, |cdocsfn1.tex|, |cdocsfn2.tex|
as well as |childdoc.pdf|.

%%%%%%%%%%%%%%%%%%%%%%%%%%%%%%%%%%%%%%%%%%%%%%%%%%%%%%%%%%%%%%%%%%%%%%%%%%%%%%%%
\subsection{Files and Installation}

The package consists of the files:
%
\begin{center}
\begin{tabular}{ll}
    |README.txt|   & readme file \\
    |childdoc.ins| & installation file \\
    |childdoc.dtx| & source file \\
    |childdoc.def| & definition file \\
    |cdocsamp.tex| & sample main file \\
    |cdocsch1.tex| & sample include file \\
    |cdocsch2.tex| & sample include file \\
    |cdocspt3.tex| & sample part file \\
    |cdocspt4.tex| & sample part file \\
    |cdocsdrf.tex| & sample redirection file \\
    |cdocsfn1.tex| & sample redirection file \\
    |cdocsfn2.tex| & sample redirection file \\
    |childdoc.pdf| & manual
\end{tabular}
\end{center}
%
The distribution consists of the files
|README.txt|, |childdoc.ins| and |childdoc.dtx|.
%
\begin{itemize}
\item
Run (pdf)\LaTeX{} on |childdoc.dtx|
to compile the manual |childdoc.pdf| (this file).
\item
Run \LaTeX{} on |childdoc.ins| to create the definitions file |childdoc.def|
and the sample |cdocsamp.tex| with include files
|cdocsch1.tex|, |cdocsch2.tex|, |cdocspt3.tex|, |cdocspt4.tex|,
|cdocsdrf.tex|, |cdocsfn1.tex|, |cdocsfn2.tex|.
Then copy the file |childdoc.def| to an appropriate directory of your \LaTeX{}
distribution, e.g.\ \textit{texmf-root}|/tex/latex/childdoc|.
\end{itemize}

%%%%%%%%%%%%%%%%%%%%%%%%%%%%%%%%%%%%%%%%%%%%%%%%%%%%%%%%%%%%%%%%%%%%%%%%%%%%%%%%
\subsection{Related CTAN Packages}

There are several other packages which offer a similar functionality:
%
\begin{itemize}
\item
The packages
\href{http://ctan.org/pkg/docmute}{\textsf{docmute}},
\href{http://ctan.org/pkg/includex}{\textsf{includex}} and
\href{http://ctan.org/pkg/standalone}{\textsf{standalone}}
provide commands to include only the document body of
a child file thus allowing both files to be compiled individually.
\item
The packages \href{http://ctan.org/pkg/subdocs}{\textsf{subdocs}}
and \href{http://ctan.org/pkg/subfiles}{\textsf{subfiles}}
provide structures in which the main and child documents can be
encapsulated and allowing them to be compiled individually.
The inclusion mechanism is different from the conventional |\include|.
\item
The package \href{http://ctan.org/pkg/combine}{\textsf{combine}}
is an elaborate solution to combine several documents into one.
\end{itemize}
%
See also the CTAN topic \href{http://ctan.org/topic/subdocs}{\textsf{subdocs}}
for further related packages.
The present package differs from the above solutions in that
a document structure constructed with the conventional |\include| mechanism
just needs two extra commands at the top of every file
such that all constituent files can be compiled individually.

%%%%%%%%%%%%%%%%%%%%%%%%%%%%%%%%%%%%%%%%%%%%%%%%%%%%%%%%%%%%%%%%%%%%%%%%%%%%%%%%
%\subsection{Feature Suggestions}
%
%The following is a list of features which may be useful for future
%versions of this package:
%%
%\begin{itemize}
%\item
%\ldots
%\end{itemize}

%%%%%%%%%%%%%%%%%%%%%%%%%%%%%%%%%%%%%%%%%%%%%%%%%%%%%%%%%%%%%%%%%%%%%%%%%%%%%%%%
\subsection{Revision History}

%%%%%%%%%%%%%%%%%%%%%%%%%%%%%%%%%%%%%%%%
\paragraph{v2.0:} 2018/12/30

\begin{itemize}
\item
immediate forward processing
\item
added |\childdocby| mechanism
\item
manual restructured
\end{itemize}

%%%%%%%%%%%%%%%%%%%%%%%%%%%%%%%%%%%%%%%%
\paragraph{v1.6:} 2018/01/17

\begin{itemize}
\item
application for development of include files
\item
corrections to manual
\end{itemize}

%%%%%%%%%%%%%%%%%%%%%%%%%%%%%%%%%%%%%%%%
\paragraph{v1.5:} 2017/05/21

\begin{itemize}
\item
more complete structuring introduced
\item
|\childdocof| introduced
\item
|\childdoc| renamed to |\childdocmain|
\item
|\childredirect| renamed to |\childdocforward| and |\childdocforwardprefix|
and functionality expanded
\end{itemize}

%%%%%%%%%%%%%%%%%%%%%%%%%%%%%%%%%%%%%%%%
\paragraph{v1.0:} 2017/04/27

\begin{itemize}
\item
manual and install package
\item
first version published on CTAN
\end{itemize}

%%%%%%%%%%%%%%%%%%%%%%%%%%%%%%%%%%%%%%%%
\paragraph{v0.6:} 2017/04/26

\begin{itemize}
\item
redirection mechanism added
\end{itemize}

%%%%%%%%%%%%%%%%%%%%%%%%%%%%%%%%%%%%%%%%
\paragraph{v0.5:} 2017/04/26

\begin{itemize}
\item
functionality in definition file
\end{itemize}


%%%%%%%%%%%%%%%%%%%%%%%%%%%%%%%%%%%%%%%%%%%%%%%%%%%%%%%%%%%%%%%%%%%%%%%%%%%%%%%%
%%%%%%%%%%%%%%%%%%%%%%%%%%%%%%%%%%%%%%%%%%%%%%%%%%%%%%%%%%%%%%%%%%%%%%%%%%%%%%%%
%%%%%%%%%%%%%%%%%%%%%%%%%%%%%%%%%%%%%%%%%%%%%%%%%%%%%%%%%%%%%%%%%%%%%%%%%%%%%%%%
\appendix

\settowidth\MacroIndent{\rmfamily\scriptsize 000\ }

 \DocInput{childdoc.dtx}

\end{document}
%</driver>
% \fi
%
% %%%%%%%%%%%%%%%%%%%%%%%%%%%%%%%%%%%%%%%%%%%%%%%%%%%%%%%%%%%%%%%%%%%%%%%%%%%%%%
% %%%%%%%%%%%%%%%%%%%%%%%%%%%%%%%%%%%%%%%%%%%%%%%%%%%%%%%%%%%%%%%%%%%%%%%%%%%%%%
% \section{Sample}
%\iffalse
%<*samplemain>
%\fi
%
% The following presents a sample document
% with two chapters, two parts, a title page,
% a compile flag as well as three forwarding files to set the flag.
% It consists of eight |.tex| files:
% \begin{center}
% \begin{tabular}{ll}
% |cdocsamp.tex|&main file\\
% |cdocsch1.tex|&include file for chapter 1\\
% |cdocsch2.tex|&include file for chapter 2\\
% |cdocspt3.tex|&include file for part 3\\
% |cdocspt4.tex|&include file for part 4\\
% |cdocsdrf.tex|&forwarding file for main file in draft mode\\
% |cdocsfi1.tex|&forwarding file for final version of chapter 1\\
% |cdocsfi2.tex|&forwarding file for final version of chapter 2\\
% \end{tabular}
% \end{center}
% Each of the eight files can be compiled directly by the \LaTeX{} compiler.
%
% %%%%%%%%%%%%%%%%%%%%%%%%%%%%%%%%%%%%%%
% \paragraph{Main File.}
%
% The main file is called |cdocsamp.tex|.
%
% Load the \textsf{childdoc} definitions and
% declare the filename for the main document:
%    \begin{macrocode}
\input{childdoc.def}
\childdocmain{}
%    \end{macrocode}

% Optional override for |\version| flag:
%    \begin{macrocode}
%%\ifchilddoc\else\providecommand{\version}{draft}\fi
%    \end{macrocode}

% Define the default values for the |\version| flag
% (|final| for the main file and |draft| for childs):
%    \begin{macrocode}
\ifchilddoc
\providecommand{\version}{draft}
\else
\providecommand{\version}{final}
\fi
%    \end{macrocode}

% Load the standard document class:
%    \begin{macrocode}
\documentclass[12pt]{article}
%    \end{macrocode}

% Start the document body:
%    \begin{macrocode}
\begin{document}
%    \end{macrocode}

% Declare a title page.
% Print title, part of document being processed and version flag:
%    \begin{macrocode}
\addtocounter{page}{-1}
\begin{center}
{\LARGE\bfseries{}childdoc example\par}
\vspace{1cm}
\ifchilddoc
\ifchilddocmanual part\else chapter\fi:
`\childdocname' of `\childdocjob'\par
\else
main document: `\childdocjob'\par
\fi
version: \version\par
\end{center}
\newpage
%    \end{macrocode}

% Manually include selected file,
% otherwise process as usual:
%    \begin{macrocode}
\ifchilddocmanual
\section*{part `\childdocname'}
\input{\childdocname}
\else
%    \end{macrocode}

% Include the two chapters:
%    \begin{macrocode}
\include{cdocsch1}
\include{cdocsch2}
%    \end{macrocode}

% Include the two parts unless only chapters should be displayed:
%    \begin{macrocode}
\ifchilddoc\else
\section{part three}
\input{cdocspt3}
\section{part four}
\input{cdocspt4}
\fi
%    \end{macrocode}

% Process as usual until here:
%    \begin{macrocode}
\fi
%    \end{macrocode}

% End of document body:
%    \begin{macrocode}
\end{document}
%    \end{macrocode}
%\iffalse
%</samplemain>
%\fi
%
% %%%%%%%%%%%%%%%%%%%%%%%%%%%%%%%%%%%%%%
% \paragraph{Chapter Include Files.}
%
% The include files are called |cdocsch1.tex| and |cdocsch2.tex|.
%
%\iffalse
%<*samplechap1|samplechap2>
%\fi

% Optional override for |\version| flag:
%    \begin{macrocode}
%%\providecommand{\version}{final}
%    \end{macrocode}

% Include the main document:
%    \begin{macrocode}
\input{childdoc.def}
\childdocof{cdocsamp}
%    \end{macrocode}

%\iffalse
%</samplechap1|samplechap2>
%\fi
%
%\iffalse
%<*samplechap1>
%\fi
% Some text for chapter 1:
%    \begin{macrocode}
\section{one}
some text in chapter one
%    \end{macrocode}

%\iffalse
%</samplechap1>
%\fi
% Some text for chapter 2:
%\iffalse
%<*samplechap2>
%\fi
%    \begin{macrocode}
\section{two}
more text in chapter two
%    \end{macrocode}

%\iffalse
%</samplechap2>
%\fi
%
% %%%%%%%%%%%%%%%%%%%%%%%%%%%%%%%%%%%%%%
% \paragraph{Part Include Files.}
%
% The include files are called |cdocspt3.tex| and |cdocspt4.tex|.
%
%\iffalse
%<*samplepart3|samplepart4>
%\fi

% Optional override for |\version| flag:
%    \begin{macrocode}
%%\providecommand{\version}{final}
%    \end{macrocode}

% Include the main document:
%    \begin{macrocode}
\input{childdoc.def}
\childdocby{cdocsamp}
%    \end{macrocode}

%\iffalse
%</samplepart3|samplepart4>
%\fi
%
%\iffalse
%<*samplepart3>
%\fi
% Some text for part 3:
%    \begin{macrocode}
some text in part three
%    \end{macrocode}

%\iffalse
%</samplepart3>
%\fi
% Some text for part 4:
%\iffalse
%<*samplepart4>
%\fi
%    \begin{macrocode}
more text in part four
%    \end{macrocode}

%\iffalse
%</samplepart4>
%\fi
%
% %%%%%%%%%%%%%%%%%%%%%%%%%%%%%%%%%%%%%%
% \paragraph{Forwarding for a Complete Draft.}
%
% The following forwarding file |cdocsdrf.tex|
% compiles the main document in draft mode:
%\iffalse
%<*sampledraft>
%\fi
%    \begin{macrocode}
\def\version{draft}
\input{childdoc.def}
\childdocforward{cdocsamp}
%    \end{macrocode}

%\iffalse
%</sampledraft>
%\fi
%
% %%%%%%%%%%%%%%%%%%%%%%%%%%%%%%%%%%%%%%
% \paragraph{Forwarding for Final Version of the Chapters.}
%
% The following forwarding files |cdocsfn1.tex| and |cdocsfn2.tex|
% (with identical content)
% compile the final versions of the child documents
% |cdocsch1.tex| and |cdocsch2.tex|, respectively:
%\iffalse
%<*samplefinal>
%\fi
%    \begin{macrocode}
\def\version{final}
\input{childdoc.def}
\childdocforwardprefix[cdocsamp]{cdocsfn}{cdocsch}
%    \end{macrocode}

%\iffalse
%</samplefinal>
%\fi
%
% %%%%%%%%%%%%%%%%%%%%%%%%%%%%%%%%%%%%%%
% \paragraph{Command Line Processing.}
%
% The following three command lines generate the output files
% |cdocscld|, |cdocscl1| and |cdocscl2|
% which should be identical to
% |cdocsdrf|, |cdocsch1| and |cdocsfn2|, respectively:
% \begin{center}
% \begin{tabular}{l}
% |latex -jobname cdocscld \|\\
% |  "\def\version{draft}\input{childdoc.def}\childdocforward{cdocsamp}"|\\
% |latex -jobname cdocscl1 \|\\
% |  "\input{childdoc.def}\childdocforward[cdocsamp]{cdocsch1}"|\\
% |latex -jobname cdocscl2 \|\\
% |  "\def\version{final}\input{childdoc.def}\childdocforward{cdocsch2}"|
% \end{tabular}
% \end{center}
% Note that the trailing backslash on each first line
% merely continues the input to the second line
% (for convenient cut ant paste).
% Furthermore, the command |latex| can be replaced by any
% of its alternative versions such as |pdflatex|.
%
% %%%%%%%%%%%%%%%%%%%%%%%%%%%%%%%%%%%%%%%%%%%%%%%%%%%%%%%%%%%%%%%%%%%%%%%%%%%%%%
% %%%%%%%%%%%%%%%%%%%%%%%%%%%%%%%%%%%%%%%%%%%%%%%%%%%%%%%%%%%%%%%%%%%%%%%%%%%%%%
% \section{Implementation}
%\iffalse
%<*package>
%\fi
%
% This section describes the definitions file |childdoc.def|.

% The definitions cannot be loaded using |\usepackage| or |\RequirePackage|
% which has a mechanism to prevent loading a style file more than once.
% When loading the definitions by means of |\input|
% multiple instances have to be prevented manually:
%\iffalse
%This code needs to be before the `\ProvidesFile' directive
%which is defined at the beginning of this file.
%Therefore it is also placed there and commented out here.
%</package>
%<*discard>
%\fi
%    \begin{macrocode}
\ifdefined\childdocmain\endinput\fi
%    \end{macrocode}
%\iffalse
%</discard>
%<*package>
%\fi
%
% \macro{\ifchilddoc}
% \macro{\ifchilddocmanual}
% The conditional |\ifchilddoc| tells whether a
% child (true) or main (false) document is being compiled.
% The conditional |\ifchilddocmanual| tells whether
% the |\includeonly| mechanism is used (false) or
% the selection of child files must be performed manually (true).
% The definitions initialise to false:
%    \begin{macrocode}
\newif\ifchilddoc
\newif\ifchilddocmanual
%    \end{macrocode}

% \macro{\childdocname}
% \macro{\childdocjob}
% The macro |\childdocname| stores the name of the main document
% to be compiled. The macro |\childdocjob| stores the name of
% the document on which the \LaTeX{} compiler was originally invoked.
% The content of |\jobname| cannot be compared
% to filenames specified in the source due to different catcodes.
% The following code rescans |\jobname|, stores the result
% in |\childdocname| and saves a copy in |\childdocjob|:
%    \begin{macrocode}
\edef\childdocname{\scantokens\expandafter{\jobname\noexpand}}
\let\childdocjob\childdocname
%    \end{macrocode}

% \macro{\childdocdisable}
% The macro |\childdocdisable| prevents the main file
% from being processed more than once.
% At this stage, the main document command |\childdocmain|
% is assumed to be called once again where it should do nothing.
% Any subsequent call to it should prevent
% a secondary processing of the main document
% It overwrites the forwarding commands
% |\childdocof| and |\childdocforward|
% with empty macros to prevent further inclusions of the main document:
%    \begin{macrocode}
\newcommand{\childdocdisable}
{
  \renewcommand{\childdocmain}[1]{\renewcommand{\childdocmain}[1]{\endinput}}
  \renewcommand{\childdocof}[1]{}
  \renewcommand{\childdocby}[2][]{}
  \renewcommand{\childdocforward}[2][]{}
  \renewcommand{\childdocdisable}{}
}
%    \end{macrocode}

% \macro{\childdocmain}
% The macro |\childdocmain| is to be called at the top of the main file
% with nothing or the main filename (without extension) as argument.
% First, it breaks loops.
% If the argument is not empty and does not match |\childdocname|
% (which is set by the first inclusion of |childdoc.def|),
% |\ifchilddoc| is set to true, |\includeonly| is applied to the child file
% and |\jobname| is set to the main file
% (for proper handling of |.aux| files):
%    \begin{macrocode}
\newcommand{\childdocmain}[1]
{
  \childdocdisable\childdocmain{}
  \if?#1?\else
    \begingroup
      \def\childdoctmp{#1}
      \ifx\childdoctmp\childdocname
        \def\childdoctmp{}
      \else
        \def\childdoctmp
        {
          \childdoctrue
          \includeonly{\childdocname}
          \def\childdocjob{#1}
          \def\jobname{#1}
        }
      \fi
      \expandafter
    \endgroup
    \childdoctmp
  \fi
}
%    \end{macrocode}

% \macro{\childdocof}
% The command |\childdocof| redirects
% compilation to the main file |#1|.
%    \begin{macrocode}
\newcommand{\childdocof}[1]
{
  \childdocdisable
  \childdoctrue
  \includeonly{\childdocname}
  \def\jobname{#1}
  \def\childdocjob{#1}
  \input{#1}
}
%    \end{macrocode}

% \macro{\childdocby}
% The command |\childdocby| ....
%    \begin{macrocode}
\newcommand{\childdocby}[2][]
{
  \childdocdisable
  \childdoctrue
  \childdocmanualtrue
  \if?#1?\else
    \def\jobname{#2}
  \fi
  \def\childdocjob{#2}
  \input{#2}
  \endinput
}
%    \end{macrocode}

% \macro{\childdocforward}
% The command |\childdocforward| redirects
% compilation to the main file or
% (if the optional argument is given) a child file.
% Parameters are set as if the main file
% or a child file starting with |\childdocof| was compiled.
% Then compilation is handed over to the main file:
%    \begin{macrocode}
\newcommand{\childdocforward}[2][]
{
  \begingroup
    \if?#1?
      \def\childdoctmp
      {
        \def\childdocname{#2}
        \def\childdocjob{#2}
        \def\jobname{#2}
        \input{#2}
        \endinput
      }
    \else
      \def\childdoctmp
      {
        \childdocdisable
        \def\childdocname{#2}
        \childdoctrue
        \includeonly{#2}
        \def\childdocjob{#1}
        \def\jobname{#1}
        \input{#1}
        \endinput
      }
    \fi
    \expandafter
  \endgroup
  \childdoctmp
}
%    \end{macrocode}

% \macro{\childdocforwardprefix}
% The command |\childdocforwardprefix| redirects
% compilation to the main or a child file by means of a pattern.
% The prefix |#1| in the current filename is replaced by |#2|
% and the suffix of the current filename is kept
% (it is assumed that the filename does not contain the substring `|~~~|'
% which is used as a delimiter).
% Compilation is handed over to the new file by |\childdocforward|:
%    \begin{macrocode}
\newcommand{\childdocforwardprefix}[3][]
{
  \begingroup
    \def\childdocextract #2##1~~~{\def\childdoctmp{\childdocforward[#1]{#3##1}}}
    \expandafter\childdocextract\childdocname~~~
    \expandafter
  \endgroup
  \childdoctmp
}
%    \end{macrocode}

% \macro{\childdoc}
% The deprecated macro |\childdoc| is a legacy version of |\childdocmain|:
%    \begin{macrocode}
\newcommand{\childdoc}{\childdocmain}
%    \end{macrocode}

% \macro{\childdocredirect}
% The deprecated macro |\childdocredirect| is a legacy version
% of |\childdocforward| and |\childdocforwardprefix|:
%    \begin{macrocode}
\newcommand{\childdocredirect}[2][]
{
  \begingroup
    \if?#1?
      \def\childdoctmp{\childdocforward{#2}}
    \else
      \def\childdoctmp{\childdocforwardprefix{#1}{#2}}
    \fi
    \expandafter
  \endgroup
  \childdoctmp
}
%    \end{macrocode}

%\iffalse
%</package>
%\fi
%
\endinput
\childdocforward{cdocsch2}"|
% \end{tabular}
% \end{center}
% Note that the trailing backslash on each first line
% merely continues the input to the second line
% (for convenient cut ant paste).
% Furthermore, the command |latex| can be replaced by any
% of its alternative versions such as |pdflatex|.
%
% %%%%%%%%%%%%%%%%%%%%%%%%%%%%%%%%%%%%%%%%%%%%%%%%%%%%%%%%%%%%%%%%%%%%%%%%%%%%%%
% %%%%%%%%%%%%%%%%%%%%%%%%%%%%%%%%%%%%%%%%%%%%%%%%%%%%%%%%%%%%%%%%%%%%%%%%%%%%%%
% \section{Implementation}
%\iffalse
%<*package>
%\fi
%
% This section describes the definitions file |childdoc.def|.

% The definitions cannot be loaded using |\usepackage| or |\RequirePackage|
% which has a mechanism to prevent loading a style file more than once.
% When loading the definitions by means of |\input|
% multiple instances have to be prevented manually:
%\iffalse
%This code needs to be before the `\ProvidesFile' directive
%which is defined at the beginning of this file.
%Therefore it is also placed there and commented out here.
%</package>
%<*discard>
%\fi
%    \begin{macrocode}
\ifdefined\childdocmain\endinput\fi
%    \end{macrocode}
%\iffalse
%</discard>
%<*package>
%\fi
%
% \macro{\ifchilddoc}
% \macro{\ifchilddocmanual}
% The conditional |\ifchilddoc| tells whether a
% child (true) or main (false) document is being compiled.
% The conditional |\ifchilddocmanual| tells whether
% the |\includeonly| mechanism is used (false) or
% the selection of child files must be performed manually (true).
% The definitions initialise to false:
%    \begin{macrocode}
\newif\ifchilddoc
\newif\ifchilddocmanual
%    \end{macrocode}

% \macro{\childdocname}
% \macro{\childdocjob}
% The macro |\childdocname| stores the name of the main document
% to be compiled. The macro |\childdocjob| stores the name of
% the document on which the \LaTeX{} compiler was originally invoked.
% The content of |\jobname| cannot be compared
% to filenames specified in the source due to different catcodes.
% The following code rescans |\jobname|, stores the result
% in |\childdocname| and saves a copy in |\childdocjob|:
%    \begin{macrocode}
\edef\childdocname{\scantokens\expandafter{\jobname\noexpand}}
\let\childdocjob\childdocname
%    \end{macrocode}

% \macro{\childdocdisable}
% The macro |\childdocdisable| prevents the main file
% from being processed more than once.
% At this stage, the main document command |\childdocmain|
% is assumed to be called once again where it should do nothing.
% Any subsequent call to it should prevent
% a secondary processing of the main document
% It overwrites the forwarding commands
% |\childdocof| and |\childdocforward|
% with empty macros to prevent further inclusions of the main document:
%    \begin{macrocode}
\newcommand{\childdocdisable}
{
  \renewcommand{\childdocmain}[1]{\renewcommand{\childdocmain}[1]{\endinput}}
  \renewcommand{\childdocof}[1]{}
  \renewcommand{\childdocby}[2][]{}
  \renewcommand{\childdocforward}[2][]{}
  \renewcommand{\childdocdisable}{}
}
%    \end{macrocode}

% \macro{\childdocmain}
% The macro |\childdocmain| is to be called at the top of the main file
% with nothing or the main filename (without extension) as argument.
% First, it breaks loops.
% If the argument is not empty and does not match |\childdocname|
% (which is set by the first inclusion of |childdoc.def|),
% |\ifchilddoc| is set to true, |\includeonly| is applied to the child file
% and |\jobname| is set to the main file
% (for proper handling of |.aux| files):
%    \begin{macrocode}
\newcommand{\childdocmain}[1]
{
  \childdocdisable\childdocmain{}
  \if?#1?\else
    \begingroup
      \def\childdoctmp{#1}
      \ifx\childdoctmp\childdocname
        \def\childdoctmp{}
      \else
        \def\childdoctmp
        {
          \childdoctrue
          \includeonly{\childdocname}
          \def\childdocjob{#1}
          \def\jobname{#1}
        }
      \fi
      \expandafter
    \endgroup
    \childdoctmp
  \fi
}
%    \end{macrocode}

% \macro{\childdocof}
% The command |\childdocof| redirects
% compilation to the main file |#1|.
%    \begin{macrocode}
\newcommand{\childdocof}[1]
{
  \childdocdisable
  \childdoctrue
  \includeonly{\childdocname}
  \def\jobname{#1}
  \def\childdocjob{#1}
  \input{#1}
}
%    \end{macrocode}

% \macro{\childdocby}
% The command |\childdocby| ....
%    \begin{macrocode}
\newcommand{\childdocby}[2][]
{
  \childdocdisable
  \childdoctrue
  \childdocmanualtrue
  \if?#1?\else
    \def\jobname{#2}
  \fi
  \def\childdocjob{#2}
  \input{#2}
  \endinput
}
%    \end{macrocode}

% \macro{\childdocforward}
% The command |\childdocforward| redirects
% compilation to the main file or
% (if the optional argument is given) a child file.
% Parameters are set as if the main file
% or a child file starting with |\childdocof| was compiled.
% Then compilation is handed over to the main file:
%    \begin{macrocode}
\newcommand{\childdocforward}[2][]
{
  \begingroup
    \if?#1?
      \def\childdoctmp
      {
        \def\childdocname{#2}
        \def\childdocjob{#2}
        \def\jobname{#2}
        \input{#2}
        \endinput
      }
    \else
      \def\childdoctmp
      {
        \childdocdisable
        \def\childdocname{#2}
        \childdoctrue
        \includeonly{#2}
        \def\childdocjob{#1}
        \def\jobname{#1}
        \input{#1}
        \endinput
      }
    \fi
    \expandafter
  \endgroup
  \childdoctmp
}
%    \end{macrocode}

% \macro{\childdocforwardprefix}
% The command |\childdocforwardprefix| redirects
% compilation to the main or a child file by means of a pattern.
% The prefix |#1| in the current filename is replaced by |#2|
% and the suffix of the current filename is kept
% (it is assumed that the filename does not contain the substring `|~~~|'
% which is used as a delimiter).
% Compilation is handed over to the new file by |\childdocforward|:
%    \begin{macrocode}
\newcommand{\childdocforwardprefix}[3][]
{
  \begingroup
    \def\childdocextract #2##1~~~{\def\childdoctmp{\childdocforward[#1]{#3##1}}}
    \expandafter\childdocextract\childdocname~~~
    \expandafter
  \endgroup
  \childdoctmp
}
%    \end{macrocode}

% \macro{\childdoc}
% The deprecated macro |\childdoc| is a legacy version of |\childdocmain|:
%    \begin{macrocode}
\newcommand{\childdoc}{\childdocmain}
%    \end{macrocode}

% \macro{\childdocredirect}
% The deprecated macro |\childdocredirect| is a legacy version
% of |\childdocforward| and |\childdocforwardprefix|:
%    \begin{macrocode}
\newcommand{\childdocredirect}[2][]
{
  \begingroup
    \if?#1?
      \def\childdoctmp{\childdocforward{#2}}
    \else
      \def\childdoctmp{\childdocforwardprefix{#1}{#2}}
    \fi
    \expandafter
  \endgroup
  \childdoctmp
}
%    \end{macrocode}

%\iffalse
%</package>
%\fi
%
\endinput
|\\
|\childdocby{|\textit{main}|}|\\
\end{tabular}
\end{center}
%
The directive |\childdocby| is similar to |\childdocof|
described in \secref{sec:include},
but the subsequent selection of content must be done manually.
To that end, both |\ifchilddoc| and |\ifchilddocmanual|
will be true upon processing of a part,
and the name of the part is stored in |\childdocname|.
Note that |\jobname| will be set to the filename of the current part
so that each part receives an individual |.aux| file
that does not interfere with the |.aux| file(s) of the main document.
This behaviour can be altered by the alternative form
|\childdocby[*]{|\textit{main}|}| (with a non-empty optional argument)
which uses the |.aux| file of the main document
by setting |\jobname| to \textit{main}.

%%%%%%%%%%%%%%%%%%%%%%%%%%%%%%%%%%%%%%%%%%%%%%%%%%%%%%%%%%%%%%%%%%%%%%%%%%%%%%%%
\subsection{Driver Development}
\label{sec:driver}

The \textsf{childdoc} mechanism can also be use for the development
of definition files such as \LaTeX{} styles or classes.
This case differs from the above setup with multiple parts
included by |\include| in that no |\includeonly| should be invoked.
This can be achieved by starting the include file
(before |\ProvidesPackage|) with:
%
\begin{center}
\begin{tabular}{l}
|% \iffalse
%
% childdoc.dtx Copyright (C) 2017-2018 Niklas Beisert
%
% This work may be distributed and/or modified under the
% conditions of the LaTeX Project Public License, either version 1.3
% of this license or (at your option) any later version.
% The latest version of this license is in
%   http://www.latex-project.org/lppl.txt
% and version 1.3 or later is part of all distributions of LaTeX
% version 2005/12/01 or later.
%
% This work has the LPPL maintenance status `maintained'.
%
% The Current Maintainer of this work is Niklas Beisert.
%
% This work consists of the files childdoc.dtx and childdoc.ins
% and the derived files childdoc.def and cdocsamp.tex with
% cdocsch1.tex, cdocsch2.tex, cdocsdrf.tex, cdocsfn1.tex, cdocsfn2.tex.
%
%<package>\ifdefined\childdocmain\endinput\fi
%<package>\ProvidesFile{childdoc.def}[2018/12/30 v2.0 child document driver]
%<samplemain>\ProvidesFile{cdocsamp.tex}[2018/12/30 v2.0 sample for childdoc]
%<*driver>
%\ProvidesFile{childdoc.drv}[2018/12/30 v2.0 childdoc reference manual file]
\PassOptionsToClass{10pt,a4paper}{article}
\documentclass{ltxdoc}

\usepackage[margin=35mm]{geometry}
\usepackage{hyperref}
\usepackage{hyperxmp}
\usepackage[usenames]{color}

\hypersetup{colorlinks=true}
\hypersetup{pdfstartview=FitH}
\hypersetup{pdfpagemode=UseNone}
\hypersetup{pdfsource={}}
\hypersetup{pdflang={en-UK}}
\hypersetup{pdfcopyright={Copyright 2017-2018 Niklas Beisert.
  This work may be distributed and/or modified under the
  conditions of the LaTeX Project Public License, either version 1.3
  of this license or (at your option) any later version.}}
\hypersetup{pdflicenseurl={http://www.latex-project.org/lppl.txt}}
\hypersetup{pdfcontactaddress={ETH Zurich, ITP, HIT K,
  Wolfgang-Pauli-Strasse 27}}
\hypersetup{pdfcontactpostcode={8093}}
\hypersetup{pdfcontactcity={Zurich}}
\hypersetup{pdfcontactcountry={Switzerland}}
\hypersetup{pdfcontactemail={nbeisert@itp.phys.ethz.ch}}
\hypersetup{pdfcontacturl={http://people.phys.ethz.ch/\xmptilde nbeisert/}}

\newcommand{\secref}[1]{\hyperref[#1]{section \ref*{#1}}}

\parskip1ex
\parindent0pt
\let\olditemize\itemize
\def\itemize{\olditemize\parskip0pt}

\begin{document}

\title{The \textsf{childdoc} Package}
\hypersetup{pdftitle={The childdoc Package}}
\author{Niklas Beisert\\[2ex]
  Institut f\"ur Theoretische Physik\\
  Eidgen\"ossische Technische Hochschule Z\"urich\\
  Wolfgang-Pauli-Strasse 27, 8093 Z\"urich, Switzerland\\[1ex]
  \href{mailto:nbeisert@itp.phys.ethz.ch}
  {\texttt{nbeisert@itp.phys.ethz.ch}}}
\hypersetup{pdfauthor={Niklas Beisert}}
\hypersetup{pdfsubject={Manual for the LaTeX2e Package childdoc}}
\date{30 December 2018, \textsf{v2.0}}
\maketitle

\begin{abstract}\noindent
\textsf{childdoc} is a \LaTeXe{} package
that enables the direct compilation
of document sections included by |\include|
to individual files.
\end{abstract}

\begingroup
\parskip0ex
\tableofcontents
\endgroup

%%%%%%%%%%%%%%%%%%%%%%%%%%%%%%%%%%%%%%%%%%%%%%%%%%%%%%%%%%%%%%%%%%%%%%%%%%%%%%%%
%%%%%%%%%%%%%%%%%%%%%%%%%%%%%%%%%%%%%%%%%%%%%%%%%%%%%%%%%%%%%%%%%%%%%%%%%%%%%%%%
\section{Introduction}

\LaTeX{} provides a mechanism to structure a large document (such as a book)
into a main file and several child files (containing the chapters)
using the |\include| command.
This mechanism is beneficial for documents
which span hundreds of pages in order to
make the source file(s) more manageable.
Moreover, compilation can be restricted to
selected child files by means of the |\includeonly| command.
The latter feature can be used to reduce the compilation time while editing
(this was significantly more useful in the earlier days of \LaTeX{})
or to generate a smaller document which is easier to navigate.
Another application of |\includeonly| is to generate
documents consisting of selected parts of the complete document.

However, there are a few drawbacks of the plain |\include| mechanism:
\begin{itemize}
\item
The child files cannot be compiled on their own,
they can only be compiled via the main file.
A naive editing environment
(such as a text editor with an option
to have the current file processed by \LaTeX)
may require one to switch to the main file before compiling;
attempting to compile the child file produces errors.
\item
The main file must be modified (each time)
to adjust the |\includeonly| command
to the present needs. This easily leaves the main file in a messy state.
\item
The generated document will always carry the filename
of the main document. This is inconvenient if
several child files are to be compiled and
to be kept for distribution.
\end{itemize}

The present package provides a simple interface
to make child files individually compilable by \LaTeX{}.
Compiling a child file then has the same effect as compiling
the main file with an |\includeonly| command
to select the appropriate child.
Moreover the generated document will carry the name of the child
rather than the main file.
This resolves all three above issues.

This feature is meant to make the editing of books,
thesis documents and lecture notes somewhat more convenient.
However, the package can also be used efficiently for
composing a series of documents (such as exercise sheets)
which are typically distributed individually.
It then assists the author in generating the individual documents
(potentially in different versions)
as well as a document containing the collected series.
Another application is in developing style files
or other kinds of included material
where compilation of the style file could redirect
to a sample or test file.

%%%%%%%%%%%%%%%%%%%%%%%%%%%%%%%%%%%%%%%%%%%%%%%%%%%%%%%%%%%%%%%%%%%%%%%%%%%%%%%%
%%%%%%%%%%%%%%%%%%%%%%%%%%%%%%%%%%%%%%%%%%%%%%%%%%%%%%%%%%%%%%%%%%%%%%%%%%%%%%%%
\section{Usage}

First of all, the package \textsf{childdoc} is \emph{not} a standard
\LaTeXe{} |.sty| style file! Therefore it needs to be invoked in
a non-standard way.

%%%%%%%%%%%%%%%%%%%%%%%%%%%%%%%%%%%%%%%%%%%%%%%%%%%%%%%%%%%%%%%%%%%%%%%%%%%%%%%%
\subsection{Included Files}
\label{sec:include}

%%%%%%%%%%%%%%%%%%%%%%%%%%%%%%%%%%%%%%%%
\DescribeMacro{\childdocmain}
To use the package, add the commands
\begin{center}
\begin{tabular}{l}
|% \iffalse
%
% childdoc.dtx Copyright (C) 2017-2018 Niklas Beisert
%
% This work may be distributed and/or modified under the
% conditions of the LaTeX Project Public License, either version 1.3
% of this license or (at your option) any later version.
% The latest version of this license is in
%   http://www.latex-project.org/lppl.txt
% and version 1.3 or later is part of all distributions of LaTeX
% version 2005/12/01 or later.
%
% This work has the LPPL maintenance status `maintained'.
%
% The Current Maintainer of this work is Niklas Beisert.
%
% This work consists of the files childdoc.dtx and childdoc.ins
% and the derived files childdoc.def and cdocsamp.tex with
% cdocsch1.tex, cdocsch2.tex, cdocsdrf.tex, cdocsfn1.tex, cdocsfn2.tex.
%
%<package>\ifdefined\childdocmain\endinput\fi
%<package>\ProvidesFile{childdoc.def}[2018/12/30 v2.0 child document driver]
%<samplemain>\ProvidesFile{cdocsamp.tex}[2018/12/30 v2.0 sample for childdoc]
%<*driver>
%\ProvidesFile{childdoc.drv}[2018/12/30 v2.0 childdoc reference manual file]
\PassOptionsToClass{10pt,a4paper}{article}
\documentclass{ltxdoc}

\usepackage[margin=35mm]{geometry}
\usepackage{hyperref}
\usepackage{hyperxmp}
\usepackage[usenames]{color}

\hypersetup{colorlinks=true}
\hypersetup{pdfstartview=FitH}
\hypersetup{pdfpagemode=UseNone}
\hypersetup{pdfsource={}}
\hypersetup{pdflang={en-UK}}
\hypersetup{pdfcopyright={Copyright 2017-2018 Niklas Beisert.
  This work may be distributed and/or modified under the
  conditions of the LaTeX Project Public License, either version 1.3
  of this license or (at your option) any later version.}}
\hypersetup{pdflicenseurl={http://www.latex-project.org/lppl.txt}}
\hypersetup{pdfcontactaddress={ETH Zurich, ITP, HIT K,
  Wolfgang-Pauli-Strasse 27}}
\hypersetup{pdfcontactpostcode={8093}}
\hypersetup{pdfcontactcity={Zurich}}
\hypersetup{pdfcontactcountry={Switzerland}}
\hypersetup{pdfcontactemail={nbeisert@itp.phys.ethz.ch}}
\hypersetup{pdfcontacturl={http://people.phys.ethz.ch/\xmptilde nbeisert/}}

\newcommand{\secref}[1]{\hyperref[#1]{section \ref*{#1}}}

\parskip1ex
\parindent0pt
\let\olditemize\itemize
\def\itemize{\olditemize\parskip0pt}

\begin{document}

\title{The \textsf{childdoc} Package}
\hypersetup{pdftitle={The childdoc Package}}
\author{Niklas Beisert\\[2ex]
  Institut f\"ur Theoretische Physik\\
  Eidgen\"ossische Technische Hochschule Z\"urich\\
  Wolfgang-Pauli-Strasse 27, 8093 Z\"urich, Switzerland\\[1ex]
  \href{mailto:nbeisert@itp.phys.ethz.ch}
  {\texttt{nbeisert@itp.phys.ethz.ch}}}
\hypersetup{pdfauthor={Niklas Beisert}}
\hypersetup{pdfsubject={Manual for the LaTeX2e Package childdoc}}
\date{30 December 2018, \textsf{v2.0}}
\maketitle

\begin{abstract}\noindent
\textsf{childdoc} is a \LaTeXe{} package
that enables the direct compilation
of document sections included by |\include|
to individual files.
\end{abstract}

\begingroup
\parskip0ex
\tableofcontents
\endgroup

%%%%%%%%%%%%%%%%%%%%%%%%%%%%%%%%%%%%%%%%%%%%%%%%%%%%%%%%%%%%%%%%%%%%%%%%%%%%%%%%
%%%%%%%%%%%%%%%%%%%%%%%%%%%%%%%%%%%%%%%%%%%%%%%%%%%%%%%%%%%%%%%%%%%%%%%%%%%%%%%%
\section{Introduction}

\LaTeX{} provides a mechanism to structure a large document (such as a book)
into a main file and several child files (containing the chapters)
using the |\include| command.
This mechanism is beneficial for documents
which span hundreds of pages in order to
make the source file(s) more manageable.
Moreover, compilation can be restricted to
selected child files by means of the |\includeonly| command.
The latter feature can be used to reduce the compilation time while editing
(this was significantly more useful in the earlier days of \LaTeX{})
or to generate a smaller document which is easier to navigate.
Another application of |\includeonly| is to generate
documents consisting of selected parts of the complete document.

However, there are a few drawbacks of the plain |\include| mechanism:
\begin{itemize}
\item
The child files cannot be compiled on their own,
they can only be compiled via the main file.
A naive editing environment
(such as a text editor with an option
to have the current file processed by \LaTeX)
may require one to switch to the main file before compiling;
attempting to compile the child file produces errors.
\item
The main file must be modified (each time)
to adjust the |\includeonly| command
to the present needs. This easily leaves the main file in a messy state.
\item
The generated document will always carry the filename
of the main document. This is inconvenient if
several child files are to be compiled and
to be kept for distribution.
\end{itemize}

The present package provides a simple interface
to make child files individually compilable by \LaTeX{}.
Compiling a child file then has the same effect as compiling
the main file with an |\includeonly| command
to select the appropriate child.
Moreover the generated document will carry the name of the child
rather than the main file.
This resolves all three above issues.

This feature is meant to make the editing of books,
thesis documents and lecture notes somewhat more convenient.
However, the package can also be used efficiently for
composing a series of documents (such as exercise sheets)
which are typically distributed individually.
It then assists the author in generating the individual documents
(potentially in different versions)
as well as a document containing the collected series.
Another application is in developing style files
or other kinds of included material
where compilation of the style file could redirect
to a sample or test file.

%%%%%%%%%%%%%%%%%%%%%%%%%%%%%%%%%%%%%%%%%%%%%%%%%%%%%%%%%%%%%%%%%%%%%%%%%%%%%%%%
%%%%%%%%%%%%%%%%%%%%%%%%%%%%%%%%%%%%%%%%%%%%%%%%%%%%%%%%%%%%%%%%%%%%%%%%%%%%%%%%
\section{Usage}

First of all, the package \textsf{childdoc} is \emph{not} a standard
\LaTeXe{} |.sty| style file! Therefore it needs to be invoked in
a non-standard way.

%%%%%%%%%%%%%%%%%%%%%%%%%%%%%%%%%%%%%%%%%%%%%%%%%%%%%%%%%%%%%%%%%%%%%%%%%%%%%%%%
\subsection{Included Files}
\label{sec:include}

%%%%%%%%%%%%%%%%%%%%%%%%%%%%%%%%%%%%%%%%
\DescribeMacro{\childdocmain}
To use the package, add the commands
\begin{center}
\begin{tabular}{l}
|\input{childdoc.def}|\\
|\childdocmain{}|\\
\end{tabular}
\end{center}
at the very top of the main \LaTeX{} file,
in particular \emph{before} the |\documentclass| statement!
The argument of |\childdocmain| should be left empty
(but it must be present).

%%%%%%%%%%%%%%%%%%%%%%%%%%%%%%%%%%%%%%%%
\DescribeMacro{\childdocof}
Furthermore, add the commands
\begin{center}
\begin{tabular}{l}
|\input{childdoc.def}|\\
|\childdocof{|\textit{main}|}|\\
\end{tabular}
\end{center}
at the top of every child file \textit{child}
which is included by |\include{|\textit{child}|}|
from within the main file
(or at least for those files to be compiled individually).
The argument \textit{main} must be the filename of the main file.

There are a couple of
considerations in setting up the main and child documents:

%%%%%%%%%%%%%%%%%%%%%%%%%%%%%%%%%%%%%%%%
\paragraph{Restrictions.}

Please note the following restrictions:
\begin{itemize}
\item
|\childdocmain| must be called with one argument \textit{main}
to ensure compatibility with earlier version of the package.
It must either be empty (|\childdocmain{}|)
or precisely match the filename of the main file in which it is specified.
See \secref{sec:detection} for further information.
\item
The filename \textit{main} must be specified without the |.tex| extension.
\item
The filename \textit{main} is case sensitive
(even in case-insensitive file systems)
due to internal string comparison.
\item
The argument \textit{main} should be fully expanded, it cannot be a macro.
\item
Subdirectories and special characters should be avoided in filenames.
\item
The command |\childdocmain{|\textit{main}|}| must be followed by a whitespace.
It should not be followed immediately by another command
or by a comment mark `|%|'.
This is because the \TeX{} parser reads the token immediately following
the argument of |\childdocmain| and puts it
at the beginning of every child section;
however, a white\-space is ignored.
\end{itemize}

%%%%%%%%%%%%%%%%%%%%%%%%%%%%%%%%%%%%%%%%
\paragraph{Content of Main File.}

It is advisable to place all content in the child files included by |\include|.
Any output contained in the main file will appear in all child documents
unless suppressed manually;
it cannot be suppressed automatically by the |\includeonly| directive
and thus should normally be avoided.
A method to include some content in the main file
by means of conditional processing is described in \secref{sec:conditional}.

%%%%%%%%%%%%%%%%%%%%%%%%%%%%%%%%%%%%%%%%
\paragraph{Page Numbering.}

When only a part of the document is compiled,
the appropriate numbering of pages
(as well as other status parameters)
is determined from the |.aux| files.
The latter contain information from previous passes.
However this information needs to propagate through
all intermediate child documents.
Therefore the page numbering in child documents may well
be inconsistent until the complete document is compiled at least once.

A useful (if unconventional) way to always ensure a consistent
page numbering is to restart the numbering in each child document
and denote the pages by `\textit{child}|.|\textit{page}'
where \textit{child} represents the chapter/section number of the child file.
This can be achieved by the command
|\numberwithin{page}{|\textit{child}|}|
of the \textsf{amsmath} package
where \textit{child} can be |chapter| or |section|
depending on the chosen structuring.
Alternatively, one can modify the macro |\thepage| appropriately
and reset the counter |page| at the start of each child file.

%%%%%%%%%%%%%%%%%%%%%%%%%%%%%%%%%%%%%%%%%%%%%%%%%%%%%%%%%%%%%%%%%%%%%%%%%%%%%%%%
\subsection{Conditional Processing}
\label{sec:conditional}

The package provides a mechanism to compile different versions
of a document. To customise the versions further some conditional processing
can come in handy to distinguish which version is being compiled.
The package provides two macros to describe the compilation context:

%%%%%%%%%%%%%%%%%%%%%%%%%%%%%%%%%%%%%%%%
\DescribeMacro{\ifchilddoc}
The conditional |\ifchilddoc| distinguishes between the compilation of
child documents and the main document:
%
\begin{center}
|\ifchilddoc |\textit{child-code}| |[|\||else |\textit{main-code}]| \||fi|
\end{center}

%%%%%%%%%%%%%%%%%%%%%%%%%%%%%%%%%%%%%%%%
\DescribeMacro{\childdocname}
\DescribeMacro{\childdocjob}
The macro |\childdocname| contains the filename (without extension)
of the main or child file being processed.
Note that |\childdocjob| will always contain the name of the main file.

%%%%%%%%%%%%%%%%%%%%%%%%%%%%%%%%%%%%%%%%
\paragraph{Title Page.}

Conditional processing can be used to include a title or banner page
in the main document when proper precautions are taken.
Importantly, the code in the main file should ensure that the page counter
(as well as other status parameters which are stored in the |.aux| files)
takes the same value after the conditional processing.
Otherwise the page numbers may take divergent values
depending on which part is compiled.

For example, a title page could be declared by:
%
\begin{center}
\begin{tabular}{l}
|\ifchilddoc\||else|\\
|\addtocounter{page}{-1}|\\
\textit{code for title page}\\
|\newpage|\\
|\||fi|
\end{tabular}
\end{center}
%
A banner page for the child documents can be generated by:
%
\begin{center}
\begin{tabular}{l}
|\ifchilddoc|\\
|\addtocounter{page}{-1}|\\
\textit{code for banner page}\\
|\newpage|\\
|\||fi|
\end{tabular}
\end{center}
%
Here one could write a message such as:
\begin{center}
|This is the part \childdocname{} of \childdocjob{}.|
\end{center}

%%%%%%%%%%%%%%%%%%%%%%%%%%%%%%%%%%%%%%%%%%%%%%%%%%%%%%%%%%%%%%%%%%%%%%%%%%%%%%%%
\subsection{Flags}
\label{sec:flags}

The package makes it easy to generate different versions
of the main or child documents.
To this end compilation flags can be defined
and assigned different default values.
They will be particularly useful in conjunction
with the forwarding mechanism described in \secref{sec:forward}.

For example, it may be useful to have a flag |\version|
which can be set to |draft| or |final|.
The document source will contain some conditional code
depending on the value of |\version|.
Suppose further, the flag should default to |final| for the main file
and to |draft| for child files
which is a natural assignment for editing the document.
This is achieved by placing the following code
in the preamble of the main document
(below the |\childdocmain| directive):
%
\begin{center}
\begin{tabular}{l}
|\ifchilddoc|\\
|\providecommand{\version}{draft}|\\
|\||else|\\
|\providecommand{\version}{final}|\\
|\||fi|
\end{tabular}
\end{center}
%
The definition by |\providecommand| makes sure
that previous definitions are not overwritten.
Further statements |\providecommand{\version}{...}|
can thus be added before the above code to override it.

For the main file, one might add a line
(between |\childdocmain| and the above block)
%
\begin{center}
|%\ifchilddoc\||else\providecommand{\version}{draft}\||fi|
\end{center}
%
which can be uncommented to produce a draft version.
Likewise one can add a line to the very top of a child file
(above the |\childdocof{|\textit{main}|}| directive)
%
\begin{center}
|%\providecommand{\version}{final}|
\end{center}
%
which can be uncommented to produce the final version of this child document.

%%%%%%%%%%%%%%%%%%%%%%%%%%%%%%%%%%%%%%%%%%%%%%%%%%%%%%%%%%%%%%%%%%%%%%%%%%%%%%%%
\subsection{Forwarding}
\label{sec:forward}

Different versions of the main or child documents
using compilation flags as described in \secref{sec:flags}
can be (permanently) stored in different files
for convenient compilation, viewing and distribution.
To this end, the package defines a command
to pass on compilation to a different file:

%%%%%%%%%%%%%%%%%%%%%%%%%%%%%%%%%%%%%%%%
\DescribeMacro{\childdocforward}
The command |\childdocforward| redirects processing to
another source file:
%
\begin{center}
\begin{tabular}{l}
|\input{childdoc.def}|\\
|\childdocforward[|\textit{main}|]{|\textit{dest}|}|\\
\end{tabular}
\end{center}
%
The argument \textit{dest} is the destination file
(without extension).
It should be the main file or one of the child files.
Note that further \textsf{childdoc} directives
such as |\childdocof| and |\childdocforward|
in the indicated file will be processed in this form.
The optional argument \textit{main}
passes on directly to the main file \textit{main}
while pretending to compile the child \textit{dest}.
This form behaves as if \textit{dest}
issues |\childdocof{|\textit{main}|}| right away,
and no further \textsf{childdoc} directives will be processed.

%%%%%%%%%%%%%%%%%%%%%%%%%%%%%%%%%%%%%%%%
\DescribeMacro{\...prefix}
In the alternative form |\childdocforwardprefix|,
%
\begin{center}
\begin{tabular}{l}
|\input{childdoc.def}|\\
|\childdocforwardprefix[|\textit{main}|]{|\textit{prefix}|}{|\textit{dest}|}|
\end{tabular}
\end{center}
%
the destination file is determined by a pattern
depending on the current file:
To make this work, the current file must be called
`{\textit{prefix}\hspace{0.2em}\textit{suffix}}'
with \textit{prefix} matching precisely the argument.
Processing is then passed on to the file
`{\textit{dest}\hspace{0.2em}\textit{suffix}}'.
Surely, the same effect is achieved by
directly specifying the
argument `{\textit{dest}\hspace{0.2em}\textit{suffix}}'
in the first form.
However, that requires to set up a different file
for each child. With the alternative form of the command
all these files can have exactly the same content
which simplifies setting them up and maintaining them.

For example, the following file |draft.tex|
with a compilation flag |\version| as described in \secref{sec:flags}
compiles the main document as a draft:
%
\begin{center}
\begin{tabular}{l}
|\def\version{draft}|\\
|\input{childdoc.def}|\\
|\childdocforward{|\textit{main}|}|
\end{tabular}
\end{center}
%
Likewise, the following files |final|\textit{nn}|.tex|
compile the final version of the child document
|child|\textit{nn}|.tex|:
%
\begin{center}
\begin{tabular}{l}
|\def\version{final}|\\
|\input{childdoc.def}|\\
|\childdocforwardprefix{final}{child}|
\end{tabular}
\end{center}
%

Note that when several versions of a main file and/or of each child file
are to be generated, it may be convenient to set up a |Makefile| or
shell script to automatise the process.

%%%%%%%%%%%%%%%%%%%%%%%%%%%%%%%%%%%%%%%%%%%%%%%%%%%%%%%%%%%%%%%%%%%%%%%%%%%%%%%%
\subsection{Command Line Processing}
\label{sec:commandline}

The effect of redirection files can also be achieved by invoking
the \LaTeX{} compiler with a more elaborate command line.
Most conveniently this should be done as part
of a shell script or a |Makefile|.

When using \textsf{childdoc} in the main file, the following
command lines effectively perform a redirection
(note that depending on the shell being used,
backslashes may have to be doubled: `|\|' $\to$ `|\\|'):
%
\begin{center}
|... -jobname "|\textit{target}|" |\\|"|[\textit{flags}]%
|\input{childdoc.def}\childdocforward[|\textit{main}|]{|\textit{dest}|}"|
\end{center}
%
Here \textit{target} is the name of the output file,
\textit{main} is the name of the main file
and \textit{dest} is the name of the main or child file to be processed
(all filenames without extensions).
The optional argument \textit{main} can be omitted
if \textit{main} matches \textit{dest}.
Optionally, compilation \textit{flags} can be defined via |\def| commands.
This command line makes the \TeX{} engine believe
it is compiling the file \textit{target}
whose content is specified as the latter parameter.
The provided code then forwards the processing to
\textit{main} or \textit{dest} as described in \secref{sec:forward}.

%%%%%%%%%%%%%%%%%%%%%%%%%%%%%%%%%%%%%%%%%%%%%%%%%%%%%%%%%%%%%%%%%%%%%%%%%%%%%%%%
\subsection{Include by Input}
\label{sec:input}

Including child documents by |\include| has some restrictions by design.
Most notably, the content of a child document always occupies
its own set of pages; pages cannot be shared between child documents.
Usually, this behaviour makes perfect sense
because each child document contain an essential part of the document.
However, in some situations it may be desirable to compose
a document from a collection of parts
without having mandatory page breaks between then.
For this case, the package
provides a mechanism to include parts
by |\input| which can also be processed individually.
However, by construction this mechanism
requires manual handling of the content to be output.

%%%%%%%%%%%%%%%%%%%%%%%%%%%%%%%%%%%%%%%%
\DescribeMacro{\ifchilddocmanual}
The main file should be prepared as usual, see \secref{sec:include}.
However, the document body must make a distinction
between processing of an individual part and of the main document, e.g.:
%
\begin{center}
\begin{tabular}{l}
|\ifchilddocmanual|\\
|\input{\childdocname}|\\
|\||else|\\
\textit{document body with }|\input{|\textit{part}|}|\\
|\||fi|
\end{tabular}
\end{center}
%
The conditional |\ifchilddocmanual| is true whenever
a part to be included by |\input| is being compiled,
and the name of the part is stored in |\childdocname|.

%%%%%%%%%%%%%%%%%%%%%%%%%%%%%%%%%%%%%%%%
\DescribeMacro{\childdocby}
Each part to be included by |\input| should start with:
%
\begin{center}
\begin{tabular}{l}
|\input{childdoc.def}|\\
|\childdocby{|\textit{main}|}|\\
\end{tabular}
\end{center}
%
The directive |\childdocby| is similar to |\childdocof|
described in \secref{sec:include},
but the subsequent selection of content must be done manually.
To that end, both |\ifchilddoc| and |\ifchilddocmanual|
will be true upon processing of a part,
and the name of the part is stored in |\childdocname|.
Note that |\jobname| will be set to the filename of the current part
so that each part receives an individual |.aux| file
that does not interfere with the |.aux| file(s) of the main document.
This behaviour can be altered by the alternative form
|\childdocby[*]{|\textit{main}|}| (with a non-empty optional argument)
which uses the |.aux| file of the main document
by setting |\jobname| to \textit{main}.

%%%%%%%%%%%%%%%%%%%%%%%%%%%%%%%%%%%%%%%%%%%%%%%%%%%%%%%%%%%%%%%%%%%%%%%%%%%%%%%%
\subsection{Driver Development}
\label{sec:driver}

The \textsf{childdoc} mechanism can also be use for the development
of definition files such as \LaTeX{} styles or classes.
This case differs from the above setup with multiple parts
included by |\include| in that no |\includeonly| should be invoked.
This can be achieved by starting the include file
(before |\ProvidesPackage|) with:
%
\begin{center}
\begin{tabular}{l}
|\input{childdoc.def}|\\
|\childdocforward{|\textit{main}|}|\\
\end{tabular}
\end{center}
%
or alternatively with:
%
\begin{center}
\begin{tabular}{l}
|\input{childdoc.def}|\\
|\childdocby{|\textit{main}|}|\\
\end{tabular}
\end{center}
%
Both forms have slightly different effects as described above.
The main file is prepared as usual, see \secref{sec:include}.

%%%%%%%%%%%%%%%%%%%%%%%%%%%%%%%%%%%%%%%%%%%%%%%%%%%%%%%%%%%%%%%%%%%%%%%%%%%%%%%%
\subsection{Legacy Detection}
\label{sec:detection}

The directive |\childdocmain| in the main file can detect
whether the complete document or merely a child is to be compiled
even without using the directive |\childdocof|.
This method is deprecated because it is less robust
and there is no compelling reason to use it;
it is merely provided for backward compatibility
and it may be removed in future versions.

If the detection mechanism is to be used,
it is mandatory to correctly specify
the filename of the main file as the argument of |\childdocmain|:
%
\begin{center}
\begin{tabular}{l}
|\input{childdoc.def}|\\
|\childdocmain{|\textit{main}|}|\\
\end{tabular}
\end{center}
%
If |\jobname| does not match the argument \textit{main} of |\childdocmain|,
it is assumed that |\jobname| points to the child file to be compiled.
When using |\childdocmain| with the main file specified as argument,
it suffices to start a child file
with just |\input{|\textit{main}|}|
without loading of the package and using |\childdocof|.
If instead all processing is done
with the appropriate \textsf{childdoc} directives,
the argument of \textit{main} of |\childdocmain| can be empty.

An alternative version of the command line processing described
in \secref{sec:commandline} using the detection mechanism reads:
%
\begin{center}
|... -jobname "|\textit{target}|" "|[\textit{flags}]%
[|\def\jobname{|\textit{dest}|}|]|\input{|\textit{main}|}"|
\end{center}

%%%%%%%%%%%%%%%%%%%%%%%%%%%%%%%%%%%%%%%%%%%%%%%%%%%%%%%%%%%%%%%%%%%%%%%%%%%%%%%%
\subsection{Manual Code}
\label{sec:manual}

In case one cannot be certain whether the definitions file |childdoc.def|
is installed on the target \TeX{} distribution
and one prefers not to ship it,
it is conceivable to paste a few relevant commands into the sources.

To that end, drop all statements |\input{childdoc.def}|
and perform the replacements as outlined below.
Instead of |\childdocmain{|\textit{main}|}| add the following code
to the top of the main file:
%
\begin{center}
\begin{tabular}{l}
|\||ifdefined\childdocname\endinput\||fi\newif\ifchilddoc|\\
|\edef\childdocname{\scantokens\expandafter{\jobname\noexpand}}|\\
|\def\childdocmain{|\textit{main}|}\||ifx\childdocmain\childdocname\||else|\\
|\childdoctrue\includeonly{\childdocname}\let\jobname\childdocmain\||fi|\\
\end{tabular}
\end{center}
%
Instead of |\childdocof{|\textit{main}|}| just include the main file
at the top of each child file:
%
\begin{center}
|\input{|\textit{main}|}|
\end{center}
%
A simple redirection |\childdocforward{|\textit{dest}|}| is achieved by:
%
\begin{center}
|\def\jobname{|\textit{dest}|}\input{\jobname}|
\end{center}
%
The redirection with prefix
|\childdocforwardprefix[|\textit{prefix}|]{|\textit{dest}|}|
is accomplished by:
%
\begin{center}
\begin{tabular}{l}
|{\edef\jobname{\scantokens\expandafter{\jobname\noexpand}}|\\
|\def\redirectjob |\textit{prefix}|#1~~~{\gdef\jobname{|\textit{dest}|#1}}|\\
|\expandafter\redirectjob\jobname~~~}\input{\jobname}|
\end{tabular}
\end{center}

In an alternative approach,
child documents can be compiled by a specific command line
without additional code or specific definitions:
%
\begin{center}
|... -jobname "|\textit{target}|" "|[\textit{flags}]%
|\includeonly{|\textit{dest}|}\input{|\textit{main}|}"|
\end{center}
%

%%%%%%%%%%%%%%%%%%%%%%%%%%%%%%%%%%%%%%%%%%%%%%%%%%%%%%%%%%%%%%%%%%%%%%%%%%%%%%%%
%%%%%%%%%%%%%%%%%%%%%%%%%%%%%%%%%%%%%%%%%%%%%%%%%%%%%%%%%%%%%%%%%%%%%%%%%%%%%%%%
\section{Information}

%%%%%%%%%%%%%%%%%%%%%%%%%%%%%%%%%%%%%%%%%%%%%%%%%%%%%%%%%%%%%%%%%%%%%%%%%%%%%%%%
\subsection{Copyright}

Copyright \copyright{} 2017--2018 Niklas Beisert

This work may be distributed and/or modified under the
conditions of the \LaTeX{} Project Public License, either version 1.3
of this license or (at your option) any later version.
The latest version of this license is in
  \url{http://www.latex-project.org/lppl.txt}
and version 1.3 or later is part of all distributions of \LaTeX{}
version 2005/12/01 or later.

This work has the LPPL maintenance status `maintained'.

The Current Maintainer of this work is Niklas Beisert.

This work consists of the files |README.txt|, |childdoc.ins| and |childdoc.dtx|
as well as the derived files |childdoc.def|, |cdocsamp.tex|
with |cdocsch1.tex|, |cdocsch2.tex|, |cdocspt3.tex|, |cdocspt4.tex|,
|cdocsdrf.tex|, |cdocsfn1.tex|, |cdocsfn2.tex|
as well as |childdoc.pdf|.

%%%%%%%%%%%%%%%%%%%%%%%%%%%%%%%%%%%%%%%%%%%%%%%%%%%%%%%%%%%%%%%%%%%%%%%%%%%%%%%%
\subsection{Files and Installation}

The package consists of the files:
%
\begin{center}
\begin{tabular}{ll}
    |README.txt|   & readme file \\
    |childdoc.ins| & installation file \\
    |childdoc.dtx| & source file \\
    |childdoc.def| & definition file \\
    |cdocsamp.tex| & sample main file \\
    |cdocsch1.tex| & sample include file \\
    |cdocsch2.tex| & sample include file \\
    |cdocspt3.tex| & sample part file \\
    |cdocspt4.tex| & sample part file \\
    |cdocsdrf.tex| & sample redirection file \\
    |cdocsfn1.tex| & sample redirection file \\
    |cdocsfn2.tex| & sample redirection file \\
    |childdoc.pdf| & manual
\end{tabular}
\end{center}
%
The distribution consists of the files
|README.txt|, |childdoc.ins| and |childdoc.dtx|.
%
\begin{itemize}
\item
Run (pdf)\LaTeX{} on |childdoc.dtx|
to compile the manual |childdoc.pdf| (this file).
\item
Run \LaTeX{} on |childdoc.ins| to create the definitions file |childdoc.def|
and the sample |cdocsamp.tex| with include files
|cdocsch1.tex|, |cdocsch2.tex|, |cdocspt3.tex|, |cdocspt4.tex|,
|cdocsdrf.tex|, |cdocsfn1.tex|, |cdocsfn2.tex|.
Then copy the file |childdoc.def| to an appropriate directory of your \LaTeX{}
distribution, e.g.\ \textit{texmf-root}|/tex/latex/childdoc|.
\end{itemize}

%%%%%%%%%%%%%%%%%%%%%%%%%%%%%%%%%%%%%%%%%%%%%%%%%%%%%%%%%%%%%%%%%%%%%%%%%%%%%%%%
\subsection{Related CTAN Packages}

There are several other packages which offer a similar functionality:
%
\begin{itemize}
\item
The packages
\href{http://ctan.org/pkg/docmute}{\textsf{docmute}},
\href{http://ctan.org/pkg/includex}{\textsf{includex}} and
\href{http://ctan.org/pkg/standalone}{\textsf{standalone}}
provide commands to include only the document body of
a child file thus allowing both files to be compiled individually.
\item
The packages \href{http://ctan.org/pkg/subdocs}{\textsf{subdocs}}
and \href{http://ctan.org/pkg/subfiles}{\textsf{subfiles}}
provide structures in which the main and child documents can be
encapsulated and allowing them to be compiled individually.
The inclusion mechanism is different from the conventional |\include|.
\item
The package \href{http://ctan.org/pkg/combine}{\textsf{combine}}
is an elaborate solution to combine several documents into one.
\end{itemize}
%
See also the CTAN topic \href{http://ctan.org/topic/subdocs}{\textsf{subdocs}}
for further related packages.
The present package differs from the above solutions in that
a document structure constructed with the conventional |\include| mechanism
just needs two extra commands at the top of every file
such that all constituent files can be compiled individually.

%%%%%%%%%%%%%%%%%%%%%%%%%%%%%%%%%%%%%%%%%%%%%%%%%%%%%%%%%%%%%%%%%%%%%%%%%%%%%%%%
%\subsection{Feature Suggestions}
%
%The following is a list of features which may be useful for future
%versions of this package:
%%
%\begin{itemize}
%\item
%\ldots
%\end{itemize}

%%%%%%%%%%%%%%%%%%%%%%%%%%%%%%%%%%%%%%%%%%%%%%%%%%%%%%%%%%%%%%%%%%%%%%%%%%%%%%%%
\subsection{Revision History}

%%%%%%%%%%%%%%%%%%%%%%%%%%%%%%%%%%%%%%%%
\paragraph{v2.0:} 2018/12/30

\begin{itemize}
\item
immediate forward processing
\item
added |\childdocby| mechanism
\item
manual restructured
\end{itemize}

%%%%%%%%%%%%%%%%%%%%%%%%%%%%%%%%%%%%%%%%
\paragraph{v1.6:} 2018/01/17

\begin{itemize}
\item
application for development of include files
\item
corrections to manual
\end{itemize}

%%%%%%%%%%%%%%%%%%%%%%%%%%%%%%%%%%%%%%%%
\paragraph{v1.5:} 2017/05/21

\begin{itemize}
\item
more complete structuring introduced
\item
|\childdocof| introduced
\item
|\childdoc| renamed to |\childdocmain|
\item
|\childredirect| renamed to |\childdocforward| and |\childdocforwardprefix|
and functionality expanded
\end{itemize}

%%%%%%%%%%%%%%%%%%%%%%%%%%%%%%%%%%%%%%%%
\paragraph{v1.0:} 2017/04/27

\begin{itemize}
\item
manual and install package
\item
first version published on CTAN
\end{itemize}

%%%%%%%%%%%%%%%%%%%%%%%%%%%%%%%%%%%%%%%%
\paragraph{v0.6:} 2017/04/26

\begin{itemize}
\item
redirection mechanism added
\end{itemize}

%%%%%%%%%%%%%%%%%%%%%%%%%%%%%%%%%%%%%%%%
\paragraph{v0.5:} 2017/04/26

\begin{itemize}
\item
functionality in definition file
\end{itemize}


%%%%%%%%%%%%%%%%%%%%%%%%%%%%%%%%%%%%%%%%%%%%%%%%%%%%%%%%%%%%%%%%%%%%%%%%%%%%%%%%
%%%%%%%%%%%%%%%%%%%%%%%%%%%%%%%%%%%%%%%%%%%%%%%%%%%%%%%%%%%%%%%%%%%%%%%%%%%%%%%%
%%%%%%%%%%%%%%%%%%%%%%%%%%%%%%%%%%%%%%%%%%%%%%%%%%%%%%%%%%%%%%%%%%%%%%%%%%%%%%%%
\appendix

\settowidth\MacroIndent{\rmfamily\scriptsize 000\ }

 \DocInput{childdoc.dtx}

\end{document}
%</driver>
% \fi
%
% %%%%%%%%%%%%%%%%%%%%%%%%%%%%%%%%%%%%%%%%%%%%%%%%%%%%%%%%%%%%%%%%%%%%%%%%%%%%%%
% %%%%%%%%%%%%%%%%%%%%%%%%%%%%%%%%%%%%%%%%%%%%%%%%%%%%%%%%%%%%%%%%%%%%%%%%%%%%%%
% \section{Sample}
%\iffalse
%<*samplemain>
%\fi
%
% The following presents a sample document
% with two chapters, two parts, a title page,
% a compile flag as well as three forwarding files to set the flag.
% It consists of eight |.tex| files:
% \begin{center}
% \begin{tabular}{ll}
% |cdocsamp.tex|&main file\\
% |cdocsch1.tex|&include file for chapter 1\\
% |cdocsch2.tex|&include file for chapter 2\\
% |cdocspt3.tex|&include file for part 3\\
% |cdocspt4.tex|&include file for part 4\\
% |cdocsdrf.tex|&forwarding file for main file in draft mode\\
% |cdocsfi1.tex|&forwarding file for final version of chapter 1\\
% |cdocsfi2.tex|&forwarding file for final version of chapter 2\\
% \end{tabular}
% \end{center}
% Each of the eight files can be compiled directly by the \LaTeX{} compiler.
%
% %%%%%%%%%%%%%%%%%%%%%%%%%%%%%%%%%%%%%%
% \paragraph{Main File.}
%
% The main file is called |cdocsamp.tex|.
%
% Load the \textsf{childdoc} definitions and
% declare the filename for the main document:
%    \begin{macrocode}
\input{childdoc.def}
\childdocmain{}
%    \end{macrocode}

% Optional override for |\version| flag:
%    \begin{macrocode}
%%\ifchilddoc\else\providecommand{\version}{draft}\fi
%    \end{macrocode}

% Define the default values for the |\version| flag
% (|final| for the main file and |draft| for childs):
%    \begin{macrocode}
\ifchilddoc
\providecommand{\version}{draft}
\else
\providecommand{\version}{final}
\fi
%    \end{macrocode}

% Load the standard document class:
%    \begin{macrocode}
\documentclass[12pt]{article}
%    \end{macrocode}

% Start the document body:
%    \begin{macrocode}
\begin{document}
%    \end{macrocode}

% Declare a title page.
% Print title, part of document being processed and version flag:
%    \begin{macrocode}
\addtocounter{page}{-1}
\begin{center}
{\LARGE\bfseries{}childdoc example\par}
\vspace{1cm}
\ifchilddoc
\ifchilddocmanual part\else chapter\fi:
`\childdocname' of `\childdocjob'\par
\else
main document: `\childdocjob'\par
\fi
version: \version\par
\end{center}
\newpage
%    \end{macrocode}

% Manually include selected file,
% otherwise process as usual:
%    \begin{macrocode}
\ifchilddocmanual
\section*{part `\childdocname'}
\input{\childdocname}
\else
%    \end{macrocode}

% Include the two chapters:
%    \begin{macrocode}
\include{cdocsch1}
\include{cdocsch2}
%    \end{macrocode}

% Include the two parts unless only chapters should be displayed:
%    \begin{macrocode}
\ifchilddoc\else
\section{part three}
\input{cdocspt3}
\section{part four}
\input{cdocspt4}
\fi
%    \end{macrocode}

% Process as usual until here:
%    \begin{macrocode}
\fi
%    \end{macrocode}

% End of document body:
%    \begin{macrocode}
\end{document}
%    \end{macrocode}
%\iffalse
%</samplemain>
%\fi
%
% %%%%%%%%%%%%%%%%%%%%%%%%%%%%%%%%%%%%%%
% \paragraph{Chapter Include Files.}
%
% The include files are called |cdocsch1.tex| and |cdocsch2.tex|.
%
%\iffalse
%<*samplechap1|samplechap2>
%\fi

% Optional override for |\version| flag:
%    \begin{macrocode}
%%\providecommand{\version}{final}
%    \end{macrocode}

% Include the main document:
%    \begin{macrocode}
\input{childdoc.def}
\childdocof{cdocsamp}
%    \end{macrocode}

%\iffalse
%</samplechap1|samplechap2>
%\fi
%
%\iffalse
%<*samplechap1>
%\fi
% Some text for chapter 1:
%    \begin{macrocode}
\section{one}
some text in chapter one
%    \end{macrocode}

%\iffalse
%</samplechap1>
%\fi
% Some text for chapter 2:
%\iffalse
%<*samplechap2>
%\fi
%    \begin{macrocode}
\section{two}
more text in chapter two
%    \end{macrocode}

%\iffalse
%</samplechap2>
%\fi
%
% %%%%%%%%%%%%%%%%%%%%%%%%%%%%%%%%%%%%%%
% \paragraph{Part Include Files.}
%
% The include files are called |cdocspt3.tex| and |cdocspt4.tex|.
%
%\iffalse
%<*samplepart3|samplepart4>
%\fi

% Optional override for |\version| flag:
%    \begin{macrocode}
%%\providecommand{\version}{final}
%    \end{macrocode}

% Include the main document:
%    \begin{macrocode}
\input{childdoc.def}
\childdocby{cdocsamp}
%    \end{macrocode}

%\iffalse
%</samplepart3|samplepart4>
%\fi
%
%\iffalse
%<*samplepart3>
%\fi
% Some text for part 3:
%    \begin{macrocode}
some text in part three
%    \end{macrocode}

%\iffalse
%</samplepart3>
%\fi
% Some text for part 4:
%\iffalse
%<*samplepart4>
%\fi
%    \begin{macrocode}
more text in part four
%    \end{macrocode}

%\iffalse
%</samplepart4>
%\fi
%
% %%%%%%%%%%%%%%%%%%%%%%%%%%%%%%%%%%%%%%
% \paragraph{Forwarding for a Complete Draft.}
%
% The following forwarding file |cdocsdrf.tex|
% compiles the main document in draft mode:
%\iffalse
%<*sampledraft>
%\fi
%    \begin{macrocode}
\def\version{draft}
\input{childdoc.def}
\childdocforward{cdocsamp}
%    \end{macrocode}

%\iffalse
%</sampledraft>
%\fi
%
% %%%%%%%%%%%%%%%%%%%%%%%%%%%%%%%%%%%%%%
% \paragraph{Forwarding for Final Version of the Chapters.}
%
% The following forwarding files |cdocsfn1.tex| and |cdocsfn2.tex|
% (with identical content)
% compile the final versions of the child documents
% |cdocsch1.tex| and |cdocsch2.tex|, respectively:
%\iffalse
%<*samplefinal>
%\fi
%    \begin{macrocode}
\def\version{final}
\input{childdoc.def}
\childdocforwardprefix[cdocsamp]{cdocsfn}{cdocsch}
%    \end{macrocode}

%\iffalse
%</samplefinal>
%\fi
%
% %%%%%%%%%%%%%%%%%%%%%%%%%%%%%%%%%%%%%%
% \paragraph{Command Line Processing.}
%
% The following three command lines generate the output files
% |cdocscld|, |cdocscl1| and |cdocscl2|
% which should be identical to
% |cdocsdrf|, |cdocsch1| and |cdocsfn2|, respectively:
% \begin{center}
% \begin{tabular}{l}
% |latex -jobname cdocscld \|\\
% |  "\def\version{draft}\input{childdoc.def}\childdocforward{cdocsamp}"|\\
% |latex -jobname cdocscl1 \|\\
% |  "\input{childdoc.def}\childdocforward[cdocsamp]{cdocsch1}"|\\
% |latex -jobname cdocscl2 \|\\
% |  "\def\version{final}\input{childdoc.def}\childdocforward{cdocsch2}"|
% \end{tabular}
% \end{center}
% Note that the trailing backslash on each first line
% merely continues the input to the second line
% (for convenient cut ant paste).
% Furthermore, the command |latex| can be replaced by any
% of its alternative versions such as |pdflatex|.
%
% %%%%%%%%%%%%%%%%%%%%%%%%%%%%%%%%%%%%%%%%%%%%%%%%%%%%%%%%%%%%%%%%%%%%%%%%%%%%%%
% %%%%%%%%%%%%%%%%%%%%%%%%%%%%%%%%%%%%%%%%%%%%%%%%%%%%%%%%%%%%%%%%%%%%%%%%%%%%%%
% \section{Implementation}
%\iffalse
%<*package>
%\fi
%
% This section describes the definitions file |childdoc.def|.

% The definitions cannot be loaded using |\usepackage| or |\RequirePackage|
% which has a mechanism to prevent loading a style file more than once.
% When loading the definitions by means of |\input|
% multiple instances have to be prevented manually:
%\iffalse
%This code needs to be before the `\ProvidesFile' directive
%which is defined at the beginning of this file.
%Therefore it is also placed there and commented out here.
%</package>
%<*discard>
%\fi
%    \begin{macrocode}
\ifdefined\childdocmain\endinput\fi
%    \end{macrocode}
%\iffalse
%</discard>
%<*package>
%\fi
%
% \macro{\ifchilddoc}
% \macro{\ifchilddocmanual}
% The conditional |\ifchilddoc| tells whether a
% child (true) or main (false) document is being compiled.
% The conditional |\ifchilddocmanual| tells whether
% the |\includeonly| mechanism is used (false) or
% the selection of child files must be performed manually (true).
% The definitions initialise to false:
%    \begin{macrocode}
\newif\ifchilddoc
\newif\ifchilddocmanual
%    \end{macrocode}

% \macro{\childdocname}
% \macro{\childdocjob}
% The macro |\childdocname| stores the name of the main document
% to be compiled. The macro |\childdocjob| stores the name of
% the document on which the \LaTeX{} compiler was originally invoked.
% The content of |\jobname| cannot be compared
% to filenames specified in the source due to different catcodes.
% The following code rescans |\jobname|, stores the result
% in |\childdocname| and saves a copy in |\childdocjob|:
%    \begin{macrocode}
\edef\childdocname{\scantokens\expandafter{\jobname\noexpand}}
\let\childdocjob\childdocname
%    \end{macrocode}

% \macro{\childdocdisable}
% The macro |\childdocdisable| prevents the main file
% from being processed more than once.
% At this stage, the main document command |\childdocmain|
% is assumed to be called once again where it should do nothing.
% Any subsequent call to it should prevent
% a secondary processing of the main document
% It overwrites the forwarding commands
% |\childdocof| and |\childdocforward|
% with empty macros to prevent further inclusions of the main document:
%    \begin{macrocode}
\newcommand{\childdocdisable}
{
  \renewcommand{\childdocmain}[1]{\renewcommand{\childdocmain}[1]{\endinput}}
  \renewcommand{\childdocof}[1]{}
  \renewcommand{\childdocby}[2][]{}
  \renewcommand{\childdocforward}[2][]{}
  \renewcommand{\childdocdisable}{}
}
%    \end{macrocode}

% \macro{\childdocmain}
% The macro |\childdocmain| is to be called at the top of the main file
% with nothing or the main filename (without extension) as argument.
% First, it breaks loops.
% If the argument is not empty and does not match |\childdocname|
% (which is set by the first inclusion of |childdoc.def|),
% |\ifchilddoc| is set to true, |\includeonly| is applied to the child file
% and |\jobname| is set to the main file
% (for proper handling of |.aux| files):
%    \begin{macrocode}
\newcommand{\childdocmain}[1]
{
  \childdocdisable\childdocmain{}
  \if?#1?\else
    \begingroup
      \def\childdoctmp{#1}
      \ifx\childdoctmp\childdocname
        \def\childdoctmp{}
      \else
        \def\childdoctmp
        {
          \childdoctrue
          \includeonly{\childdocname}
          \def\childdocjob{#1}
          \def\jobname{#1}
        }
      \fi
      \expandafter
    \endgroup
    \childdoctmp
  \fi
}
%    \end{macrocode}

% \macro{\childdocof}
% The command |\childdocof| redirects
% compilation to the main file |#1|.
%    \begin{macrocode}
\newcommand{\childdocof}[1]
{
  \childdocdisable
  \childdoctrue
  \includeonly{\childdocname}
  \def\jobname{#1}
  \def\childdocjob{#1}
  \input{#1}
}
%    \end{macrocode}

% \macro{\childdocby}
% The command |\childdocby| ....
%    \begin{macrocode}
\newcommand{\childdocby}[2][]
{
  \childdocdisable
  \childdoctrue
  \childdocmanualtrue
  \if?#1?\else
    \def\jobname{#2}
  \fi
  \def\childdocjob{#2}
  \input{#2}
  \endinput
}
%    \end{macrocode}

% \macro{\childdocforward}
% The command |\childdocforward| redirects
% compilation to the main file or
% (if the optional argument is given) a child file.
% Parameters are set as if the main file
% or a child file starting with |\childdocof| was compiled.
% Then compilation is handed over to the main file:
%    \begin{macrocode}
\newcommand{\childdocforward}[2][]
{
  \begingroup
    \if?#1?
      \def\childdoctmp
      {
        \def\childdocname{#2}
        \def\childdocjob{#2}
        \def\jobname{#2}
        \input{#2}
        \endinput
      }
    \else
      \def\childdoctmp
      {
        \childdocdisable
        \def\childdocname{#2}
        \childdoctrue
        \includeonly{#2}
        \def\childdocjob{#1}
        \def\jobname{#1}
        \input{#1}
        \endinput
      }
    \fi
    \expandafter
  \endgroup
  \childdoctmp
}
%    \end{macrocode}

% \macro{\childdocforwardprefix}
% The command |\childdocforwardprefix| redirects
% compilation to the main or a child file by means of a pattern.
% The prefix |#1| in the current filename is replaced by |#2|
% and the suffix of the current filename is kept
% (it is assumed that the filename does not contain the substring `|~~~|'
% which is used as a delimiter).
% Compilation is handed over to the new file by |\childdocforward|:
%    \begin{macrocode}
\newcommand{\childdocforwardprefix}[3][]
{
  \begingroup
    \def\childdocextract #2##1~~~{\def\childdoctmp{\childdocforward[#1]{#3##1}}}
    \expandafter\childdocextract\childdocname~~~
    \expandafter
  \endgroup
  \childdoctmp
}
%    \end{macrocode}

% \macro{\childdoc}
% The deprecated macro |\childdoc| is a legacy version of |\childdocmain|:
%    \begin{macrocode}
\newcommand{\childdoc}{\childdocmain}
%    \end{macrocode}

% \macro{\childdocredirect}
% The deprecated macro |\childdocredirect| is a legacy version
% of |\childdocforward| and |\childdocforwardprefix|:
%    \begin{macrocode}
\newcommand{\childdocredirect}[2][]
{
  \begingroup
    \if?#1?
      \def\childdoctmp{\childdocforward{#2}}
    \else
      \def\childdoctmp{\childdocforwardprefix{#1}{#2}}
    \fi
    \expandafter
  \endgroup
  \childdoctmp
}
%    \end{macrocode}

%\iffalse
%</package>
%\fi
%
\endinput
|\\
|\childdocmain{}|\\
\end{tabular}
\end{center}
at the very top of the main \LaTeX{} file,
in particular \emph{before} the |\documentclass| statement!
The argument of |\childdocmain| should be left empty
(but it must be present).

%%%%%%%%%%%%%%%%%%%%%%%%%%%%%%%%%%%%%%%%
\DescribeMacro{\childdocof}
Furthermore, add the commands
\begin{center}
\begin{tabular}{l}
|% \iffalse
%
% childdoc.dtx Copyright (C) 2017-2018 Niklas Beisert
%
% This work may be distributed and/or modified under the
% conditions of the LaTeX Project Public License, either version 1.3
% of this license or (at your option) any later version.
% The latest version of this license is in
%   http://www.latex-project.org/lppl.txt
% and version 1.3 or later is part of all distributions of LaTeX
% version 2005/12/01 or later.
%
% This work has the LPPL maintenance status `maintained'.
%
% The Current Maintainer of this work is Niklas Beisert.
%
% This work consists of the files childdoc.dtx and childdoc.ins
% and the derived files childdoc.def and cdocsamp.tex with
% cdocsch1.tex, cdocsch2.tex, cdocsdrf.tex, cdocsfn1.tex, cdocsfn2.tex.
%
%<package>\ifdefined\childdocmain\endinput\fi
%<package>\ProvidesFile{childdoc.def}[2018/12/30 v2.0 child document driver]
%<samplemain>\ProvidesFile{cdocsamp.tex}[2018/12/30 v2.0 sample for childdoc]
%<*driver>
%\ProvidesFile{childdoc.drv}[2018/12/30 v2.0 childdoc reference manual file]
\PassOptionsToClass{10pt,a4paper}{article}
\documentclass{ltxdoc}

\usepackage[margin=35mm]{geometry}
\usepackage{hyperref}
\usepackage{hyperxmp}
\usepackage[usenames]{color}

\hypersetup{colorlinks=true}
\hypersetup{pdfstartview=FitH}
\hypersetup{pdfpagemode=UseNone}
\hypersetup{pdfsource={}}
\hypersetup{pdflang={en-UK}}
\hypersetup{pdfcopyright={Copyright 2017-2018 Niklas Beisert.
  This work may be distributed and/or modified under the
  conditions of the LaTeX Project Public License, either version 1.3
  of this license or (at your option) any later version.}}
\hypersetup{pdflicenseurl={http://www.latex-project.org/lppl.txt}}
\hypersetup{pdfcontactaddress={ETH Zurich, ITP, HIT K,
  Wolfgang-Pauli-Strasse 27}}
\hypersetup{pdfcontactpostcode={8093}}
\hypersetup{pdfcontactcity={Zurich}}
\hypersetup{pdfcontactcountry={Switzerland}}
\hypersetup{pdfcontactemail={nbeisert@itp.phys.ethz.ch}}
\hypersetup{pdfcontacturl={http://people.phys.ethz.ch/\xmptilde nbeisert/}}

\newcommand{\secref}[1]{\hyperref[#1]{section \ref*{#1}}}

\parskip1ex
\parindent0pt
\let\olditemize\itemize
\def\itemize{\olditemize\parskip0pt}

\begin{document}

\title{The \textsf{childdoc} Package}
\hypersetup{pdftitle={The childdoc Package}}
\author{Niklas Beisert\\[2ex]
  Institut f\"ur Theoretische Physik\\
  Eidgen\"ossische Technische Hochschule Z\"urich\\
  Wolfgang-Pauli-Strasse 27, 8093 Z\"urich, Switzerland\\[1ex]
  \href{mailto:nbeisert@itp.phys.ethz.ch}
  {\texttt{nbeisert@itp.phys.ethz.ch}}}
\hypersetup{pdfauthor={Niklas Beisert}}
\hypersetup{pdfsubject={Manual for the LaTeX2e Package childdoc}}
\date{30 December 2018, \textsf{v2.0}}
\maketitle

\begin{abstract}\noindent
\textsf{childdoc} is a \LaTeXe{} package
that enables the direct compilation
of document sections included by |\include|
to individual files.
\end{abstract}

\begingroup
\parskip0ex
\tableofcontents
\endgroup

%%%%%%%%%%%%%%%%%%%%%%%%%%%%%%%%%%%%%%%%%%%%%%%%%%%%%%%%%%%%%%%%%%%%%%%%%%%%%%%%
%%%%%%%%%%%%%%%%%%%%%%%%%%%%%%%%%%%%%%%%%%%%%%%%%%%%%%%%%%%%%%%%%%%%%%%%%%%%%%%%
\section{Introduction}

\LaTeX{} provides a mechanism to structure a large document (such as a book)
into a main file and several child files (containing the chapters)
using the |\include| command.
This mechanism is beneficial for documents
which span hundreds of pages in order to
make the source file(s) more manageable.
Moreover, compilation can be restricted to
selected child files by means of the |\includeonly| command.
The latter feature can be used to reduce the compilation time while editing
(this was significantly more useful in the earlier days of \LaTeX{})
or to generate a smaller document which is easier to navigate.
Another application of |\includeonly| is to generate
documents consisting of selected parts of the complete document.

However, there are a few drawbacks of the plain |\include| mechanism:
\begin{itemize}
\item
The child files cannot be compiled on their own,
they can only be compiled via the main file.
A naive editing environment
(such as a text editor with an option
to have the current file processed by \LaTeX)
may require one to switch to the main file before compiling;
attempting to compile the child file produces errors.
\item
The main file must be modified (each time)
to adjust the |\includeonly| command
to the present needs. This easily leaves the main file in a messy state.
\item
The generated document will always carry the filename
of the main document. This is inconvenient if
several child files are to be compiled and
to be kept for distribution.
\end{itemize}

The present package provides a simple interface
to make child files individually compilable by \LaTeX{}.
Compiling a child file then has the same effect as compiling
the main file with an |\includeonly| command
to select the appropriate child.
Moreover the generated document will carry the name of the child
rather than the main file.
This resolves all three above issues.

This feature is meant to make the editing of books,
thesis documents and lecture notes somewhat more convenient.
However, the package can also be used efficiently for
composing a series of documents (such as exercise sheets)
which are typically distributed individually.
It then assists the author in generating the individual documents
(potentially in different versions)
as well as a document containing the collected series.
Another application is in developing style files
or other kinds of included material
where compilation of the style file could redirect
to a sample or test file.

%%%%%%%%%%%%%%%%%%%%%%%%%%%%%%%%%%%%%%%%%%%%%%%%%%%%%%%%%%%%%%%%%%%%%%%%%%%%%%%%
%%%%%%%%%%%%%%%%%%%%%%%%%%%%%%%%%%%%%%%%%%%%%%%%%%%%%%%%%%%%%%%%%%%%%%%%%%%%%%%%
\section{Usage}

First of all, the package \textsf{childdoc} is \emph{not} a standard
\LaTeXe{} |.sty| style file! Therefore it needs to be invoked in
a non-standard way.

%%%%%%%%%%%%%%%%%%%%%%%%%%%%%%%%%%%%%%%%%%%%%%%%%%%%%%%%%%%%%%%%%%%%%%%%%%%%%%%%
\subsection{Included Files}
\label{sec:include}

%%%%%%%%%%%%%%%%%%%%%%%%%%%%%%%%%%%%%%%%
\DescribeMacro{\childdocmain}
To use the package, add the commands
\begin{center}
\begin{tabular}{l}
|\input{childdoc.def}|\\
|\childdocmain{}|\\
\end{tabular}
\end{center}
at the very top of the main \LaTeX{} file,
in particular \emph{before} the |\documentclass| statement!
The argument of |\childdocmain| should be left empty
(but it must be present).

%%%%%%%%%%%%%%%%%%%%%%%%%%%%%%%%%%%%%%%%
\DescribeMacro{\childdocof}
Furthermore, add the commands
\begin{center}
\begin{tabular}{l}
|\input{childdoc.def}|\\
|\childdocof{|\textit{main}|}|\\
\end{tabular}
\end{center}
at the top of every child file \textit{child}
which is included by |\include{|\textit{child}|}|
from within the main file
(or at least for those files to be compiled individually).
The argument \textit{main} must be the filename of the main file.

There are a couple of
considerations in setting up the main and child documents:

%%%%%%%%%%%%%%%%%%%%%%%%%%%%%%%%%%%%%%%%
\paragraph{Restrictions.}

Please note the following restrictions:
\begin{itemize}
\item
|\childdocmain| must be called with one argument \textit{main}
to ensure compatibility with earlier version of the package.
It must either be empty (|\childdocmain{}|)
or precisely match the filename of the main file in which it is specified.
See \secref{sec:detection} for further information.
\item
The filename \textit{main} must be specified without the |.tex| extension.
\item
The filename \textit{main} is case sensitive
(even in case-insensitive file systems)
due to internal string comparison.
\item
The argument \textit{main} should be fully expanded, it cannot be a macro.
\item
Subdirectories and special characters should be avoided in filenames.
\item
The command |\childdocmain{|\textit{main}|}| must be followed by a whitespace.
It should not be followed immediately by another command
or by a comment mark `|%|'.
This is because the \TeX{} parser reads the token immediately following
the argument of |\childdocmain| and puts it
at the beginning of every child section;
however, a white\-space is ignored.
\end{itemize}

%%%%%%%%%%%%%%%%%%%%%%%%%%%%%%%%%%%%%%%%
\paragraph{Content of Main File.}

It is advisable to place all content in the child files included by |\include|.
Any output contained in the main file will appear in all child documents
unless suppressed manually;
it cannot be suppressed automatically by the |\includeonly| directive
and thus should normally be avoided.
A method to include some content in the main file
by means of conditional processing is described in \secref{sec:conditional}.

%%%%%%%%%%%%%%%%%%%%%%%%%%%%%%%%%%%%%%%%
\paragraph{Page Numbering.}

When only a part of the document is compiled,
the appropriate numbering of pages
(as well as other status parameters)
is determined from the |.aux| files.
The latter contain information from previous passes.
However this information needs to propagate through
all intermediate child documents.
Therefore the page numbering in child documents may well
be inconsistent until the complete document is compiled at least once.

A useful (if unconventional) way to always ensure a consistent
page numbering is to restart the numbering in each child document
and denote the pages by `\textit{child}|.|\textit{page}'
where \textit{child} represents the chapter/section number of the child file.
This can be achieved by the command
|\numberwithin{page}{|\textit{child}|}|
of the \textsf{amsmath} package
where \textit{child} can be |chapter| or |section|
depending on the chosen structuring.
Alternatively, one can modify the macro |\thepage| appropriately
and reset the counter |page| at the start of each child file.

%%%%%%%%%%%%%%%%%%%%%%%%%%%%%%%%%%%%%%%%%%%%%%%%%%%%%%%%%%%%%%%%%%%%%%%%%%%%%%%%
\subsection{Conditional Processing}
\label{sec:conditional}

The package provides a mechanism to compile different versions
of a document. To customise the versions further some conditional processing
can come in handy to distinguish which version is being compiled.
The package provides two macros to describe the compilation context:

%%%%%%%%%%%%%%%%%%%%%%%%%%%%%%%%%%%%%%%%
\DescribeMacro{\ifchilddoc}
The conditional |\ifchilddoc| distinguishes between the compilation of
child documents and the main document:
%
\begin{center}
|\ifchilddoc |\textit{child-code}| |[|\||else |\textit{main-code}]| \||fi|
\end{center}

%%%%%%%%%%%%%%%%%%%%%%%%%%%%%%%%%%%%%%%%
\DescribeMacro{\childdocname}
\DescribeMacro{\childdocjob}
The macro |\childdocname| contains the filename (without extension)
of the main or child file being processed.
Note that |\childdocjob| will always contain the name of the main file.

%%%%%%%%%%%%%%%%%%%%%%%%%%%%%%%%%%%%%%%%
\paragraph{Title Page.}

Conditional processing can be used to include a title or banner page
in the main document when proper precautions are taken.
Importantly, the code in the main file should ensure that the page counter
(as well as other status parameters which are stored in the |.aux| files)
takes the same value after the conditional processing.
Otherwise the page numbers may take divergent values
depending on which part is compiled.

For example, a title page could be declared by:
%
\begin{center}
\begin{tabular}{l}
|\ifchilddoc\||else|\\
|\addtocounter{page}{-1}|\\
\textit{code for title page}\\
|\newpage|\\
|\||fi|
\end{tabular}
\end{center}
%
A banner page for the child documents can be generated by:
%
\begin{center}
\begin{tabular}{l}
|\ifchilddoc|\\
|\addtocounter{page}{-1}|\\
\textit{code for banner page}\\
|\newpage|\\
|\||fi|
\end{tabular}
\end{center}
%
Here one could write a message such as:
\begin{center}
|This is the part \childdocname{} of \childdocjob{}.|
\end{center}

%%%%%%%%%%%%%%%%%%%%%%%%%%%%%%%%%%%%%%%%%%%%%%%%%%%%%%%%%%%%%%%%%%%%%%%%%%%%%%%%
\subsection{Flags}
\label{sec:flags}

The package makes it easy to generate different versions
of the main or child documents.
To this end compilation flags can be defined
and assigned different default values.
They will be particularly useful in conjunction
with the forwarding mechanism described in \secref{sec:forward}.

For example, it may be useful to have a flag |\version|
which can be set to |draft| or |final|.
The document source will contain some conditional code
depending on the value of |\version|.
Suppose further, the flag should default to |final| for the main file
and to |draft| for child files
which is a natural assignment for editing the document.
This is achieved by placing the following code
in the preamble of the main document
(below the |\childdocmain| directive):
%
\begin{center}
\begin{tabular}{l}
|\ifchilddoc|\\
|\providecommand{\version}{draft}|\\
|\||else|\\
|\providecommand{\version}{final}|\\
|\||fi|
\end{tabular}
\end{center}
%
The definition by |\providecommand| makes sure
that previous definitions are not overwritten.
Further statements |\providecommand{\version}{...}|
can thus be added before the above code to override it.

For the main file, one might add a line
(between |\childdocmain| and the above block)
%
\begin{center}
|%\ifchilddoc\||else\providecommand{\version}{draft}\||fi|
\end{center}
%
which can be uncommented to produce a draft version.
Likewise one can add a line to the very top of a child file
(above the |\childdocof{|\textit{main}|}| directive)
%
\begin{center}
|%\providecommand{\version}{final}|
\end{center}
%
which can be uncommented to produce the final version of this child document.

%%%%%%%%%%%%%%%%%%%%%%%%%%%%%%%%%%%%%%%%%%%%%%%%%%%%%%%%%%%%%%%%%%%%%%%%%%%%%%%%
\subsection{Forwarding}
\label{sec:forward}

Different versions of the main or child documents
using compilation flags as described in \secref{sec:flags}
can be (permanently) stored in different files
for convenient compilation, viewing and distribution.
To this end, the package defines a command
to pass on compilation to a different file:

%%%%%%%%%%%%%%%%%%%%%%%%%%%%%%%%%%%%%%%%
\DescribeMacro{\childdocforward}
The command |\childdocforward| redirects processing to
another source file:
%
\begin{center}
\begin{tabular}{l}
|\input{childdoc.def}|\\
|\childdocforward[|\textit{main}|]{|\textit{dest}|}|\\
\end{tabular}
\end{center}
%
The argument \textit{dest} is the destination file
(without extension).
It should be the main file or one of the child files.
Note that further \textsf{childdoc} directives
such as |\childdocof| and |\childdocforward|
in the indicated file will be processed in this form.
The optional argument \textit{main}
passes on directly to the main file \textit{main}
while pretending to compile the child \textit{dest}.
This form behaves as if \textit{dest}
issues |\childdocof{|\textit{main}|}| right away,
and no further \textsf{childdoc} directives will be processed.

%%%%%%%%%%%%%%%%%%%%%%%%%%%%%%%%%%%%%%%%
\DescribeMacro{\...prefix}
In the alternative form |\childdocforwardprefix|,
%
\begin{center}
\begin{tabular}{l}
|\input{childdoc.def}|\\
|\childdocforwardprefix[|\textit{main}|]{|\textit{prefix}|}{|\textit{dest}|}|
\end{tabular}
\end{center}
%
the destination file is determined by a pattern
depending on the current file:
To make this work, the current file must be called
`{\textit{prefix}\hspace{0.2em}\textit{suffix}}'
with \textit{prefix} matching precisely the argument.
Processing is then passed on to the file
`{\textit{dest}\hspace{0.2em}\textit{suffix}}'.
Surely, the same effect is achieved by
directly specifying the
argument `{\textit{dest}\hspace{0.2em}\textit{suffix}}'
in the first form.
However, that requires to set up a different file
for each child. With the alternative form of the command
all these files can have exactly the same content
which simplifies setting them up and maintaining them.

For example, the following file |draft.tex|
with a compilation flag |\version| as described in \secref{sec:flags}
compiles the main document as a draft:
%
\begin{center}
\begin{tabular}{l}
|\def\version{draft}|\\
|\input{childdoc.def}|\\
|\childdocforward{|\textit{main}|}|
\end{tabular}
\end{center}
%
Likewise, the following files |final|\textit{nn}|.tex|
compile the final version of the child document
|child|\textit{nn}|.tex|:
%
\begin{center}
\begin{tabular}{l}
|\def\version{final}|\\
|\input{childdoc.def}|\\
|\childdocforwardprefix{final}{child}|
\end{tabular}
\end{center}
%

Note that when several versions of a main file and/or of each child file
are to be generated, it may be convenient to set up a |Makefile| or
shell script to automatise the process.

%%%%%%%%%%%%%%%%%%%%%%%%%%%%%%%%%%%%%%%%%%%%%%%%%%%%%%%%%%%%%%%%%%%%%%%%%%%%%%%%
\subsection{Command Line Processing}
\label{sec:commandline}

The effect of redirection files can also be achieved by invoking
the \LaTeX{} compiler with a more elaborate command line.
Most conveniently this should be done as part
of a shell script or a |Makefile|.

When using \textsf{childdoc} in the main file, the following
command lines effectively perform a redirection
(note that depending on the shell being used,
backslashes may have to be doubled: `|\|' $\to$ `|\\|'):
%
\begin{center}
|... -jobname "|\textit{target}|" |\\|"|[\textit{flags}]%
|\input{childdoc.def}\childdocforward[|\textit{main}|]{|\textit{dest}|}"|
\end{center}
%
Here \textit{target} is the name of the output file,
\textit{main} is the name of the main file
and \textit{dest} is the name of the main or child file to be processed
(all filenames without extensions).
The optional argument \textit{main} can be omitted
if \textit{main} matches \textit{dest}.
Optionally, compilation \textit{flags} can be defined via |\def| commands.
This command line makes the \TeX{} engine believe
it is compiling the file \textit{target}
whose content is specified as the latter parameter.
The provided code then forwards the processing to
\textit{main} or \textit{dest} as described in \secref{sec:forward}.

%%%%%%%%%%%%%%%%%%%%%%%%%%%%%%%%%%%%%%%%%%%%%%%%%%%%%%%%%%%%%%%%%%%%%%%%%%%%%%%%
\subsection{Include by Input}
\label{sec:input}

Including child documents by |\include| has some restrictions by design.
Most notably, the content of a child document always occupies
its own set of pages; pages cannot be shared between child documents.
Usually, this behaviour makes perfect sense
because each child document contain an essential part of the document.
However, in some situations it may be desirable to compose
a document from a collection of parts
without having mandatory page breaks between then.
For this case, the package
provides a mechanism to include parts
by |\input| which can also be processed individually.
However, by construction this mechanism
requires manual handling of the content to be output.

%%%%%%%%%%%%%%%%%%%%%%%%%%%%%%%%%%%%%%%%
\DescribeMacro{\ifchilddocmanual}
The main file should be prepared as usual, see \secref{sec:include}.
However, the document body must make a distinction
between processing of an individual part and of the main document, e.g.:
%
\begin{center}
\begin{tabular}{l}
|\ifchilddocmanual|\\
|\input{\childdocname}|\\
|\||else|\\
\textit{document body with }|\input{|\textit{part}|}|\\
|\||fi|
\end{tabular}
\end{center}
%
The conditional |\ifchilddocmanual| is true whenever
a part to be included by |\input| is being compiled,
and the name of the part is stored in |\childdocname|.

%%%%%%%%%%%%%%%%%%%%%%%%%%%%%%%%%%%%%%%%
\DescribeMacro{\childdocby}
Each part to be included by |\input| should start with:
%
\begin{center}
\begin{tabular}{l}
|\input{childdoc.def}|\\
|\childdocby{|\textit{main}|}|\\
\end{tabular}
\end{center}
%
The directive |\childdocby| is similar to |\childdocof|
described in \secref{sec:include},
but the subsequent selection of content must be done manually.
To that end, both |\ifchilddoc| and |\ifchilddocmanual|
will be true upon processing of a part,
and the name of the part is stored in |\childdocname|.
Note that |\jobname| will be set to the filename of the current part
so that each part receives an individual |.aux| file
that does not interfere with the |.aux| file(s) of the main document.
This behaviour can be altered by the alternative form
|\childdocby[*]{|\textit{main}|}| (with a non-empty optional argument)
which uses the |.aux| file of the main document
by setting |\jobname| to \textit{main}.

%%%%%%%%%%%%%%%%%%%%%%%%%%%%%%%%%%%%%%%%%%%%%%%%%%%%%%%%%%%%%%%%%%%%%%%%%%%%%%%%
\subsection{Driver Development}
\label{sec:driver}

The \textsf{childdoc} mechanism can also be use for the development
of definition files such as \LaTeX{} styles or classes.
This case differs from the above setup with multiple parts
included by |\include| in that no |\includeonly| should be invoked.
This can be achieved by starting the include file
(before |\ProvidesPackage|) with:
%
\begin{center}
\begin{tabular}{l}
|\input{childdoc.def}|\\
|\childdocforward{|\textit{main}|}|\\
\end{tabular}
\end{center}
%
or alternatively with:
%
\begin{center}
\begin{tabular}{l}
|\input{childdoc.def}|\\
|\childdocby{|\textit{main}|}|\\
\end{tabular}
\end{center}
%
Both forms have slightly different effects as described above.
The main file is prepared as usual, see \secref{sec:include}.

%%%%%%%%%%%%%%%%%%%%%%%%%%%%%%%%%%%%%%%%%%%%%%%%%%%%%%%%%%%%%%%%%%%%%%%%%%%%%%%%
\subsection{Legacy Detection}
\label{sec:detection}

The directive |\childdocmain| in the main file can detect
whether the complete document or merely a child is to be compiled
even without using the directive |\childdocof|.
This method is deprecated because it is less robust
and there is no compelling reason to use it;
it is merely provided for backward compatibility
and it may be removed in future versions.

If the detection mechanism is to be used,
it is mandatory to correctly specify
the filename of the main file as the argument of |\childdocmain|:
%
\begin{center}
\begin{tabular}{l}
|\input{childdoc.def}|\\
|\childdocmain{|\textit{main}|}|\\
\end{tabular}
\end{center}
%
If |\jobname| does not match the argument \textit{main} of |\childdocmain|,
it is assumed that |\jobname| points to the child file to be compiled.
When using |\childdocmain| with the main file specified as argument,
it suffices to start a child file
with just |\input{|\textit{main}|}|
without loading of the package and using |\childdocof|.
If instead all processing is done
with the appropriate \textsf{childdoc} directives,
the argument of \textit{main} of |\childdocmain| can be empty.

An alternative version of the command line processing described
in \secref{sec:commandline} using the detection mechanism reads:
%
\begin{center}
|... -jobname "|\textit{target}|" "|[\textit{flags}]%
[|\def\jobname{|\textit{dest}|}|]|\input{|\textit{main}|}"|
\end{center}

%%%%%%%%%%%%%%%%%%%%%%%%%%%%%%%%%%%%%%%%%%%%%%%%%%%%%%%%%%%%%%%%%%%%%%%%%%%%%%%%
\subsection{Manual Code}
\label{sec:manual}

In case one cannot be certain whether the definitions file |childdoc.def|
is installed on the target \TeX{} distribution
and one prefers not to ship it,
it is conceivable to paste a few relevant commands into the sources.

To that end, drop all statements |\input{childdoc.def}|
and perform the replacements as outlined below.
Instead of |\childdocmain{|\textit{main}|}| add the following code
to the top of the main file:
%
\begin{center}
\begin{tabular}{l}
|\||ifdefined\childdocname\endinput\||fi\newif\ifchilddoc|\\
|\edef\childdocname{\scantokens\expandafter{\jobname\noexpand}}|\\
|\def\childdocmain{|\textit{main}|}\||ifx\childdocmain\childdocname\||else|\\
|\childdoctrue\includeonly{\childdocname}\let\jobname\childdocmain\||fi|\\
\end{tabular}
\end{center}
%
Instead of |\childdocof{|\textit{main}|}| just include the main file
at the top of each child file:
%
\begin{center}
|\input{|\textit{main}|}|
\end{center}
%
A simple redirection |\childdocforward{|\textit{dest}|}| is achieved by:
%
\begin{center}
|\def\jobname{|\textit{dest}|}\input{\jobname}|
\end{center}
%
The redirection with prefix
|\childdocforwardprefix[|\textit{prefix}|]{|\textit{dest}|}|
is accomplished by:
%
\begin{center}
\begin{tabular}{l}
|{\edef\jobname{\scantokens\expandafter{\jobname\noexpand}}|\\
|\def\redirectjob |\textit{prefix}|#1~~~{\gdef\jobname{|\textit{dest}|#1}}|\\
|\expandafter\redirectjob\jobname~~~}\input{\jobname}|
\end{tabular}
\end{center}

In an alternative approach,
child documents can be compiled by a specific command line
without additional code or specific definitions:
%
\begin{center}
|... -jobname "|\textit{target}|" "|[\textit{flags}]%
|\includeonly{|\textit{dest}|}\input{|\textit{main}|}"|
\end{center}
%

%%%%%%%%%%%%%%%%%%%%%%%%%%%%%%%%%%%%%%%%%%%%%%%%%%%%%%%%%%%%%%%%%%%%%%%%%%%%%%%%
%%%%%%%%%%%%%%%%%%%%%%%%%%%%%%%%%%%%%%%%%%%%%%%%%%%%%%%%%%%%%%%%%%%%%%%%%%%%%%%%
\section{Information}

%%%%%%%%%%%%%%%%%%%%%%%%%%%%%%%%%%%%%%%%%%%%%%%%%%%%%%%%%%%%%%%%%%%%%%%%%%%%%%%%
\subsection{Copyright}

Copyright \copyright{} 2017--2018 Niklas Beisert

This work may be distributed and/or modified under the
conditions of the \LaTeX{} Project Public License, either version 1.3
of this license or (at your option) any later version.
The latest version of this license is in
  \url{http://www.latex-project.org/lppl.txt}
and version 1.3 or later is part of all distributions of \LaTeX{}
version 2005/12/01 or later.

This work has the LPPL maintenance status `maintained'.

The Current Maintainer of this work is Niklas Beisert.

This work consists of the files |README.txt|, |childdoc.ins| and |childdoc.dtx|
as well as the derived files |childdoc.def|, |cdocsamp.tex|
with |cdocsch1.tex|, |cdocsch2.tex|, |cdocspt3.tex|, |cdocspt4.tex|,
|cdocsdrf.tex|, |cdocsfn1.tex|, |cdocsfn2.tex|
as well as |childdoc.pdf|.

%%%%%%%%%%%%%%%%%%%%%%%%%%%%%%%%%%%%%%%%%%%%%%%%%%%%%%%%%%%%%%%%%%%%%%%%%%%%%%%%
\subsection{Files and Installation}

The package consists of the files:
%
\begin{center}
\begin{tabular}{ll}
    |README.txt|   & readme file \\
    |childdoc.ins| & installation file \\
    |childdoc.dtx| & source file \\
    |childdoc.def| & definition file \\
    |cdocsamp.tex| & sample main file \\
    |cdocsch1.tex| & sample include file \\
    |cdocsch2.tex| & sample include file \\
    |cdocspt3.tex| & sample part file \\
    |cdocspt4.tex| & sample part file \\
    |cdocsdrf.tex| & sample redirection file \\
    |cdocsfn1.tex| & sample redirection file \\
    |cdocsfn2.tex| & sample redirection file \\
    |childdoc.pdf| & manual
\end{tabular}
\end{center}
%
The distribution consists of the files
|README.txt|, |childdoc.ins| and |childdoc.dtx|.
%
\begin{itemize}
\item
Run (pdf)\LaTeX{} on |childdoc.dtx|
to compile the manual |childdoc.pdf| (this file).
\item
Run \LaTeX{} on |childdoc.ins| to create the definitions file |childdoc.def|
and the sample |cdocsamp.tex| with include files
|cdocsch1.tex|, |cdocsch2.tex|, |cdocspt3.tex|, |cdocspt4.tex|,
|cdocsdrf.tex|, |cdocsfn1.tex|, |cdocsfn2.tex|.
Then copy the file |childdoc.def| to an appropriate directory of your \LaTeX{}
distribution, e.g.\ \textit{texmf-root}|/tex/latex/childdoc|.
\end{itemize}

%%%%%%%%%%%%%%%%%%%%%%%%%%%%%%%%%%%%%%%%%%%%%%%%%%%%%%%%%%%%%%%%%%%%%%%%%%%%%%%%
\subsection{Related CTAN Packages}

There are several other packages which offer a similar functionality:
%
\begin{itemize}
\item
The packages
\href{http://ctan.org/pkg/docmute}{\textsf{docmute}},
\href{http://ctan.org/pkg/includex}{\textsf{includex}} and
\href{http://ctan.org/pkg/standalone}{\textsf{standalone}}
provide commands to include only the document body of
a child file thus allowing both files to be compiled individually.
\item
The packages \href{http://ctan.org/pkg/subdocs}{\textsf{subdocs}}
and \href{http://ctan.org/pkg/subfiles}{\textsf{subfiles}}
provide structures in which the main and child documents can be
encapsulated and allowing them to be compiled individually.
The inclusion mechanism is different from the conventional |\include|.
\item
The package \href{http://ctan.org/pkg/combine}{\textsf{combine}}
is an elaborate solution to combine several documents into one.
\end{itemize}
%
See also the CTAN topic \href{http://ctan.org/topic/subdocs}{\textsf{subdocs}}
for further related packages.
The present package differs from the above solutions in that
a document structure constructed with the conventional |\include| mechanism
just needs two extra commands at the top of every file
such that all constituent files can be compiled individually.

%%%%%%%%%%%%%%%%%%%%%%%%%%%%%%%%%%%%%%%%%%%%%%%%%%%%%%%%%%%%%%%%%%%%%%%%%%%%%%%%
%\subsection{Feature Suggestions}
%
%The following is a list of features which may be useful for future
%versions of this package:
%%
%\begin{itemize}
%\item
%\ldots
%\end{itemize}

%%%%%%%%%%%%%%%%%%%%%%%%%%%%%%%%%%%%%%%%%%%%%%%%%%%%%%%%%%%%%%%%%%%%%%%%%%%%%%%%
\subsection{Revision History}

%%%%%%%%%%%%%%%%%%%%%%%%%%%%%%%%%%%%%%%%
\paragraph{v2.0:} 2018/12/30

\begin{itemize}
\item
immediate forward processing
\item
added |\childdocby| mechanism
\item
manual restructured
\end{itemize}

%%%%%%%%%%%%%%%%%%%%%%%%%%%%%%%%%%%%%%%%
\paragraph{v1.6:} 2018/01/17

\begin{itemize}
\item
application for development of include files
\item
corrections to manual
\end{itemize}

%%%%%%%%%%%%%%%%%%%%%%%%%%%%%%%%%%%%%%%%
\paragraph{v1.5:} 2017/05/21

\begin{itemize}
\item
more complete structuring introduced
\item
|\childdocof| introduced
\item
|\childdoc| renamed to |\childdocmain|
\item
|\childredirect| renamed to |\childdocforward| and |\childdocforwardprefix|
and functionality expanded
\end{itemize}

%%%%%%%%%%%%%%%%%%%%%%%%%%%%%%%%%%%%%%%%
\paragraph{v1.0:} 2017/04/27

\begin{itemize}
\item
manual and install package
\item
first version published on CTAN
\end{itemize}

%%%%%%%%%%%%%%%%%%%%%%%%%%%%%%%%%%%%%%%%
\paragraph{v0.6:} 2017/04/26

\begin{itemize}
\item
redirection mechanism added
\end{itemize}

%%%%%%%%%%%%%%%%%%%%%%%%%%%%%%%%%%%%%%%%
\paragraph{v0.5:} 2017/04/26

\begin{itemize}
\item
functionality in definition file
\end{itemize}


%%%%%%%%%%%%%%%%%%%%%%%%%%%%%%%%%%%%%%%%%%%%%%%%%%%%%%%%%%%%%%%%%%%%%%%%%%%%%%%%
%%%%%%%%%%%%%%%%%%%%%%%%%%%%%%%%%%%%%%%%%%%%%%%%%%%%%%%%%%%%%%%%%%%%%%%%%%%%%%%%
%%%%%%%%%%%%%%%%%%%%%%%%%%%%%%%%%%%%%%%%%%%%%%%%%%%%%%%%%%%%%%%%%%%%%%%%%%%%%%%%
\appendix

\settowidth\MacroIndent{\rmfamily\scriptsize 000\ }

 \DocInput{childdoc.dtx}

\end{document}
%</driver>
% \fi
%
% %%%%%%%%%%%%%%%%%%%%%%%%%%%%%%%%%%%%%%%%%%%%%%%%%%%%%%%%%%%%%%%%%%%%%%%%%%%%%%
% %%%%%%%%%%%%%%%%%%%%%%%%%%%%%%%%%%%%%%%%%%%%%%%%%%%%%%%%%%%%%%%%%%%%%%%%%%%%%%
% \section{Sample}
%\iffalse
%<*samplemain>
%\fi
%
% The following presents a sample document
% with two chapters, two parts, a title page,
% a compile flag as well as three forwarding files to set the flag.
% It consists of eight |.tex| files:
% \begin{center}
% \begin{tabular}{ll}
% |cdocsamp.tex|&main file\\
% |cdocsch1.tex|&include file for chapter 1\\
% |cdocsch2.tex|&include file for chapter 2\\
% |cdocspt3.tex|&include file for part 3\\
% |cdocspt4.tex|&include file for part 4\\
% |cdocsdrf.tex|&forwarding file for main file in draft mode\\
% |cdocsfi1.tex|&forwarding file for final version of chapter 1\\
% |cdocsfi2.tex|&forwarding file for final version of chapter 2\\
% \end{tabular}
% \end{center}
% Each of the eight files can be compiled directly by the \LaTeX{} compiler.
%
% %%%%%%%%%%%%%%%%%%%%%%%%%%%%%%%%%%%%%%
% \paragraph{Main File.}
%
% The main file is called |cdocsamp.tex|.
%
% Load the \textsf{childdoc} definitions and
% declare the filename for the main document:
%    \begin{macrocode}
\input{childdoc.def}
\childdocmain{}
%    \end{macrocode}

% Optional override for |\version| flag:
%    \begin{macrocode}
%%\ifchilddoc\else\providecommand{\version}{draft}\fi
%    \end{macrocode}

% Define the default values for the |\version| flag
% (|final| for the main file and |draft| for childs):
%    \begin{macrocode}
\ifchilddoc
\providecommand{\version}{draft}
\else
\providecommand{\version}{final}
\fi
%    \end{macrocode}

% Load the standard document class:
%    \begin{macrocode}
\documentclass[12pt]{article}
%    \end{macrocode}

% Start the document body:
%    \begin{macrocode}
\begin{document}
%    \end{macrocode}

% Declare a title page.
% Print title, part of document being processed and version flag:
%    \begin{macrocode}
\addtocounter{page}{-1}
\begin{center}
{\LARGE\bfseries{}childdoc example\par}
\vspace{1cm}
\ifchilddoc
\ifchilddocmanual part\else chapter\fi:
`\childdocname' of `\childdocjob'\par
\else
main document: `\childdocjob'\par
\fi
version: \version\par
\end{center}
\newpage
%    \end{macrocode}

% Manually include selected file,
% otherwise process as usual:
%    \begin{macrocode}
\ifchilddocmanual
\section*{part `\childdocname'}
\input{\childdocname}
\else
%    \end{macrocode}

% Include the two chapters:
%    \begin{macrocode}
\include{cdocsch1}
\include{cdocsch2}
%    \end{macrocode}

% Include the two parts unless only chapters should be displayed:
%    \begin{macrocode}
\ifchilddoc\else
\section{part three}
\input{cdocspt3}
\section{part four}
\input{cdocspt4}
\fi
%    \end{macrocode}

% Process as usual until here:
%    \begin{macrocode}
\fi
%    \end{macrocode}

% End of document body:
%    \begin{macrocode}
\end{document}
%    \end{macrocode}
%\iffalse
%</samplemain>
%\fi
%
% %%%%%%%%%%%%%%%%%%%%%%%%%%%%%%%%%%%%%%
% \paragraph{Chapter Include Files.}
%
% The include files are called |cdocsch1.tex| and |cdocsch2.tex|.
%
%\iffalse
%<*samplechap1|samplechap2>
%\fi

% Optional override for |\version| flag:
%    \begin{macrocode}
%%\providecommand{\version}{final}
%    \end{macrocode}

% Include the main document:
%    \begin{macrocode}
\input{childdoc.def}
\childdocof{cdocsamp}
%    \end{macrocode}

%\iffalse
%</samplechap1|samplechap2>
%\fi
%
%\iffalse
%<*samplechap1>
%\fi
% Some text for chapter 1:
%    \begin{macrocode}
\section{one}
some text in chapter one
%    \end{macrocode}

%\iffalse
%</samplechap1>
%\fi
% Some text for chapter 2:
%\iffalse
%<*samplechap2>
%\fi
%    \begin{macrocode}
\section{two}
more text in chapter two
%    \end{macrocode}

%\iffalse
%</samplechap2>
%\fi
%
% %%%%%%%%%%%%%%%%%%%%%%%%%%%%%%%%%%%%%%
% \paragraph{Part Include Files.}
%
% The include files are called |cdocspt3.tex| and |cdocspt4.tex|.
%
%\iffalse
%<*samplepart3|samplepart4>
%\fi

% Optional override for |\version| flag:
%    \begin{macrocode}
%%\providecommand{\version}{final}
%    \end{macrocode}

% Include the main document:
%    \begin{macrocode}
\input{childdoc.def}
\childdocby{cdocsamp}
%    \end{macrocode}

%\iffalse
%</samplepart3|samplepart4>
%\fi
%
%\iffalse
%<*samplepart3>
%\fi
% Some text for part 3:
%    \begin{macrocode}
some text in part three
%    \end{macrocode}

%\iffalse
%</samplepart3>
%\fi
% Some text for part 4:
%\iffalse
%<*samplepart4>
%\fi
%    \begin{macrocode}
more text in part four
%    \end{macrocode}

%\iffalse
%</samplepart4>
%\fi
%
% %%%%%%%%%%%%%%%%%%%%%%%%%%%%%%%%%%%%%%
% \paragraph{Forwarding for a Complete Draft.}
%
% The following forwarding file |cdocsdrf.tex|
% compiles the main document in draft mode:
%\iffalse
%<*sampledraft>
%\fi
%    \begin{macrocode}
\def\version{draft}
\input{childdoc.def}
\childdocforward{cdocsamp}
%    \end{macrocode}

%\iffalse
%</sampledraft>
%\fi
%
% %%%%%%%%%%%%%%%%%%%%%%%%%%%%%%%%%%%%%%
% \paragraph{Forwarding for Final Version of the Chapters.}
%
% The following forwarding files |cdocsfn1.tex| and |cdocsfn2.tex|
% (with identical content)
% compile the final versions of the child documents
% |cdocsch1.tex| and |cdocsch2.tex|, respectively:
%\iffalse
%<*samplefinal>
%\fi
%    \begin{macrocode}
\def\version{final}
\input{childdoc.def}
\childdocforwardprefix[cdocsamp]{cdocsfn}{cdocsch}
%    \end{macrocode}

%\iffalse
%</samplefinal>
%\fi
%
% %%%%%%%%%%%%%%%%%%%%%%%%%%%%%%%%%%%%%%
% \paragraph{Command Line Processing.}
%
% The following three command lines generate the output files
% |cdocscld|, |cdocscl1| and |cdocscl2|
% which should be identical to
% |cdocsdrf|, |cdocsch1| and |cdocsfn2|, respectively:
% \begin{center}
% \begin{tabular}{l}
% |latex -jobname cdocscld \|\\
% |  "\def\version{draft}\input{childdoc.def}\childdocforward{cdocsamp}"|\\
% |latex -jobname cdocscl1 \|\\
% |  "\input{childdoc.def}\childdocforward[cdocsamp]{cdocsch1}"|\\
% |latex -jobname cdocscl2 \|\\
% |  "\def\version{final}\input{childdoc.def}\childdocforward{cdocsch2}"|
% \end{tabular}
% \end{center}
% Note that the trailing backslash on each first line
% merely continues the input to the second line
% (for convenient cut ant paste).
% Furthermore, the command |latex| can be replaced by any
% of its alternative versions such as |pdflatex|.
%
% %%%%%%%%%%%%%%%%%%%%%%%%%%%%%%%%%%%%%%%%%%%%%%%%%%%%%%%%%%%%%%%%%%%%%%%%%%%%%%
% %%%%%%%%%%%%%%%%%%%%%%%%%%%%%%%%%%%%%%%%%%%%%%%%%%%%%%%%%%%%%%%%%%%%%%%%%%%%%%
% \section{Implementation}
%\iffalse
%<*package>
%\fi
%
% This section describes the definitions file |childdoc.def|.

% The definitions cannot be loaded using |\usepackage| or |\RequirePackage|
% which has a mechanism to prevent loading a style file more than once.
% When loading the definitions by means of |\input|
% multiple instances have to be prevented manually:
%\iffalse
%This code needs to be before the `\ProvidesFile' directive
%which is defined at the beginning of this file.
%Therefore it is also placed there and commented out here.
%</package>
%<*discard>
%\fi
%    \begin{macrocode}
\ifdefined\childdocmain\endinput\fi
%    \end{macrocode}
%\iffalse
%</discard>
%<*package>
%\fi
%
% \macro{\ifchilddoc}
% \macro{\ifchilddocmanual}
% The conditional |\ifchilddoc| tells whether a
% child (true) or main (false) document is being compiled.
% The conditional |\ifchilddocmanual| tells whether
% the |\includeonly| mechanism is used (false) or
% the selection of child files must be performed manually (true).
% The definitions initialise to false:
%    \begin{macrocode}
\newif\ifchilddoc
\newif\ifchilddocmanual
%    \end{macrocode}

% \macro{\childdocname}
% \macro{\childdocjob}
% The macro |\childdocname| stores the name of the main document
% to be compiled. The macro |\childdocjob| stores the name of
% the document on which the \LaTeX{} compiler was originally invoked.
% The content of |\jobname| cannot be compared
% to filenames specified in the source due to different catcodes.
% The following code rescans |\jobname|, stores the result
% in |\childdocname| and saves a copy in |\childdocjob|:
%    \begin{macrocode}
\edef\childdocname{\scantokens\expandafter{\jobname\noexpand}}
\let\childdocjob\childdocname
%    \end{macrocode}

% \macro{\childdocdisable}
% The macro |\childdocdisable| prevents the main file
% from being processed more than once.
% At this stage, the main document command |\childdocmain|
% is assumed to be called once again where it should do nothing.
% Any subsequent call to it should prevent
% a secondary processing of the main document
% It overwrites the forwarding commands
% |\childdocof| and |\childdocforward|
% with empty macros to prevent further inclusions of the main document:
%    \begin{macrocode}
\newcommand{\childdocdisable}
{
  \renewcommand{\childdocmain}[1]{\renewcommand{\childdocmain}[1]{\endinput}}
  \renewcommand{\childdocof}[1]{}
  \renewcommand{\childdocby}[2][]{}
  \renewcommand{\childdocforward}[2][]{}
  \renewcommand{\childdocdisable}{}
}
%    \end{macrocode}

% \macro{\childdocmain}
% The macro |\childdocmain| is to be called at the top of the main file
% with nothing or the main filename (without extension) as argument.
% First, it breaks loops.
% If the argument is not empty and does not match |\childdocname|
% (which is set by the first inclusion of |childdoc.def|),
% |\ifchilddoc| is set to true, |\includeonly| is applied to the child file
% and |\jobname| is set to the main file
% (for proper handling of |.aux| files):
%    \begin{macrocode}
\newcommand{\childdocmain}[1]
{
  \childdocdisable\childdocmain{}
  \if?#1?\else
    \begingroup
      \def\childdoctmp{#1}
      \ifx\childdoctmp\childdocname
        \def\childdoctmp{}
      \else
        \def\childdoctmp
        {
          \childdoctrue
          \includeonly{\childdocname}
          \def\childdocjob{#1}
          \def\jobname{#1}
        }
      \fi
      \expandafter
    \endgroup
    \childdoctmp
  \fi
}
%    \end{macrocode}

% \macro{\childdocof}
% The command |\childdocof| redirects
% compilation to the main file |#1|.
%    \begin{macrocode}
\newcommand{\childdocof}[1]
{
  \childdocdisable
  \childdoctrue
  \includeonly{\childdocname}
  \def\jobname{#1}
  \def\childdocjob{#1}
  \input{#1}
}
%    \end{macrocode}

% \macro{\childdocby}
% The command |\childdocby| ....
%    \begin{macrocode}
\newcommand{\childdocby}[2][]
{
  \childdocdisable
  \childdoctrue
  \childdocmanualtrue
  \if?#1?\else
    \def\jobname{#2}
  \fi
  \def\childdocjob{#2}
  \input{#2}
  \endinput
}
%    \end{macrocode}

% \macro{\childdocforward}
% The command |\childdocforward| redirects
% compilation to the main file or
% (if the optional argument is given) a child file.
% Parameters are set as if the main file
% or a child file starting with |\childdocof| was compiled.
% Then compilation is handed over to the main file:
%    \begin{macrocode}
\newcommand{\childdocforward}[2][]
{
  \begingroup
    \if?#1?
      \def\childdoctmp
      {
        \def\childdocname{#2}
        \def\childdocjob{#2}
        \def\jobname{#2}
        \input{#2}
        \endinput
      }
    \else
      \def\childdoctmp
      {
        \childdocdisable
        \def\childdocname{#2}
        \childdoctrue
        \includeonly{#2}
        \def\childdocjob{#1}
        \def\jobname{#1}
        \input{#1}
        \endinput
      }
    \fi
    \expandafter
  \endgroup
  \childdoctmp
}
%    \end{macrocode}

% \macro{\childdocforwardprefix}
% The command |\childdocforwardprefix| redirects
% compilation to the main or a child file by means of a pattern.
% The prefix |#1| in the current filename is replaced by |#2|
% and the suffix of the current filename is kept
% (it is assumed that the filename does not contain the substring `|~~~|'
% which is used as a delimiter).
% Compilation is handed over to the new file by |\childdocforward|:
%    \begin{macrocode}
\newcommand{\childdocforwardprefix}[3][]
{
  \begingroup
    \def\childdocextract #2##1~~~{\def\childdoctmp{\childdocforward[#1]{#3##1}}}
    \expandafter\childdocextract\childdocname~~~
    \expandafter
  \endgroup
  \childdoctmp
}
%    \end{macrocode}

% \macro{\childdoc}
% The deprecated macro |\childdoc| is a legacy version of |\childdocmain|:
%    \begin{macrocode}
\newcommand{\childdoc}{\childdocmain}
%    \end{macrocode}

% \macro{\childdocredirect}
% The deprecated macro |\childdocredirect| is a legacy version
% of |\childdocforward| and |\childdocforwardprefix|:
%    \begin{macrocode}
\newcommand{\childdocredirect}[2][]
{
  \begingroup
    \if?#1?
      \def\childdoctmp{\childdocforward{#2}}
    \else
      \def\childdoctmp{\childdocforwardprefix{#1}{#2}}
    \fi
    \expandafter
  \endgroup
  \childdoctmp
}
%    \end{macrocode}

%\iffalse
%</package>
%\fi
%
\endinput
|\\
|\childdocof{|\textit{main}|}|\\
\end{tabular}
\end{center}
at the top of every child file \textit{child}
which is included by |\include{|\textit{child}|}|
from within the main file
(or at least for those files to be compiled individually).
The argument \textit{main} must be the filename of the main file.

There are a couple of
considerations in setting up the main and child documents:

%%%%%%%%%%%%%%%%%%%%%%%%%%%%%%%%%%%%%%%%
\paragraph{Restrictions.}

Please note the following restrictions:
\begin{itemize}
\item
|\childdocmain| must be called with one argument \textit{main}
to ensure compatibility with earlier version of the package.
It must either be empty (|\childdocmain{}|)
or precisely match the filename of the main file in which it is specified.
See \secref{sec:detection} for further information.
\item
The filename \textit{main} must be specified without the |.tex| extension.
\item
The filename \textit{main} is case sensitive
(even in case-insensitive file systems)
due to internal string comparison.
\item
The argument \textit{main} should be fully expanded, it cannot be a macro.
\item
Subdirectories and special characters should be avoided in filenames.
\item
The command |\childdocmain{|\textit{main}|}| must be followed by a whitespace.
It should not be followed immediately by another command
or by a comment mark `|%|'.
This is because the \TeX{} parser reads the token immediately following
the argument of |\childdocmain| and puts it
at the beginning of every child section;
however, a white\-space is ignored.
\end{itemize}

%%%%%%%%%%%%%%%%%%%%%%%%%%%%%%%%%%%%%%%%
\paragraph{Content of Main File.}

It is advisable to place all content in the child files included by |\include|.
Any output contained in the main file will appear in all child documents
unless suppressed manually;
it cannot be suppressed automatically by the |\includeonly| directive
and thus should normally be avoided.
A method to include some content in the main file
by means of conditional processing is described in \secref{sec:conditional}.

%%%%%%%%%%%%%%%%%%%%%%%%%%%%%%%%%%%%%%%%
\paragraph{Page Numbering.}

When only a part of the document is compiled,
the appropriate numbering of pages
(as well as other status parameters)
is determined from the |.aux| files.
The latter contain information from previous passes.
However this information needs to propagate through
all intermediate child documents.
Therefore the page numbering in child documents may well
be inconsistent until the complete document is compiled at least once.

A useful (if unconventional) way to always ensure a consistent
page numbering is to restart the numbering in each child document
and denote the pages by `\textit{child}|.|\textit{page}'
where \textit{child} represents the chapter/section number of the child file.
This can be achieved by the command
|\numberwithin{page}{|\textit{child}|}|
of the \textsf{amsmath} package
where \textit{child} can be |chapter| or |section|
depending on the chosen structuring.
Alternatively, one can modify the macro |\thepage| appropriately
and reset the counter |page| at the start of each child file.

%%%%%%%%%%%%%%%%%%%%%%%%%%%%%%%%%%%%%%%%%%%%%%%%%%%%%%%%%%%%%%%%%%%%%%%%%%%%%%%%
\subsection{Conditional Processing}
\label{sec:conditional}

The package provides a mechanism to compile different versions
of a document. To customise the versions further some conditional processing
can come in handy to distinguish which version is being compiled.
The package provides two macros to describe the compilation context:

%%%%%%%%%%%%%%%%%%%%%%%%%%%%%%%%%%%%%%%%
\DescribeMacro{\ifchilddoc}
The conditional |\ifchilddoc| distinguishes between the compilation of
child documents and the main document:
%
\begin{center}
|\ifchilddoc |\textit{child-code}| |[|\||else |\textit{main-code}]| \||fi|
\end{center}

%%%%%%%%%%%%%%%%%%%%%%%%%%%%%%%%%%%%%%%%
\DescribeMacro{\childdocname}
\DescribeMacro{\childdocjob}
The macro |\childdocname| contains the filename (without extension)
of the main or child file being processed.
Note that |\childdocjob| will always contain the name of the main file.

%%%%%%%%%%%%%%%%%%%%%%%%%%%%%%%%%%%%%%%%
\paragraph{Title Page.}

Conditional processing can be used to include a title or banner page
in the main document when proper precautions are taken.
Importantly, the code in the main file should ensure that the page counter
(as well as other status parameters which are stored in the |.aux| files)
takes the same value after the conditional processing.
Otherwise the page numbers may take divergent values
depending on which part is compiled.

For example, a title page could be declared by:
%
\begin{center}
\begin{tabular}{l}
|\ifchilddoc\||else|\\
|\addtocounter{page}{-1}|\\
\textit{code for title page}\\
|\newpage|\\
|\||fi|
\end{tabular}
\end{center}
%
A banner page for the child documents can be generated by:
%
\begin{center}
\begin{tabular}{l}
|\ifchilddoc|\\
|\addtocounter{page}{-1}|\\
\textit{code for banner page}\\
|\newpage|\\
|\||fi|
\end{tabular}
\end{center}
%
Here one could write a message such as:
\begin{center}
|This is the part \childdocname{} of \childdocjob{}.|
\end{center}

%%%%%%%%%%%%%%%%%%%%%%%%%%%%%%%%%%%%%%%%%%%%%%%%%%%%%%%%%%%%%%%%%%%%%%%%%%%%%%%%
\subsection{Flags}
\label{sec:flags}

The package makes it easy to generate different versions
of the main or child documents.
To this end compilation flags can be defined
and assigned different default values.
They will be particularly useful in conjunction
with the forwarding mechanism described in \secref{sec:forward}.

For example, it may be useful to have a flag |\version|
which can be set to |draft| or |final|.
The document source will contain some conditional code
depending on the value of |\version|.
Suppose further, the flag should default to |final| for the main file
and to |draft| for child files
which is a natural assignment for editing the document.
This is achieved by placing the following code
in the preamble of the main document
(below the |\childdocmain| directive):
%
\begin{center}
\begin{tabular}{l}
|\ifchilddoc|\\
|\providecommand{\version}{draft}|\\
|\||else|\\
|\providecommand{\version}{final}|\\
|\||fi|
\end{tabular}
\end{center}
%
The definition by |\providecommand| makes sure
that previous definitions are not overwritten.
Further statements |\providecommand{\version}{...}|
can thus be added before the above code to override it.

For the main file, one might add a line
(between |\childdocmain| and the above block)
%
\begin{center}
|%\ifchilddoc\||else\providecommand{\version}{draft}\||fi|
\end{center}
%
which can be uncommented to produce a draft version.
Likewise one can add a line to the very top of a child file
(above the |\childdocof{|\textit{main}|}| directive)
%
\begin{center}
|%\providecommand{\version}{final}|
\end{center}
%
which can be uncommented to produce the final version of this child document.

%%%%%%%%%%%%%%%%%%%%%%%%%%%%%%%%%%%%%%%%%%%%%%%%%%%%%%%%%%%%%%%%%%%%%%%%%%%%%%%%
\subsection{Forwarding}
\label{sec:forward}

Different versions of the main or child documents
using compilation flags as described in \secref{sec:flags}
can be (permanently) stored in different files
for convenient compilation, viewing and distribution.
To this end, the package defines a command
to pass on compilation to a different file:

%%%%%%%%%%%%%%%%%%%%%%%%%%%%%%%%%%%%%%%%
\DescribeMacro{\childdocforward}
The command |\childdocforward| redirects processing to
another source file:
%
\begin{center}
\begin{tabular}{l}
|% \iffalse
%
% childdoc.dtx Copyright (C) 2017-2018 Niklas Beisert
%
% This work may be distributed and/or modified under the
% conditions of the LaTeX Project Public License, either version 1.3
% of this license or (at your option) any later version.
% The latest version of this license is in
%   http://www.latex-project.org/lppl.txt
% and version 1.3 or later is part of all distributions of LaTeX
% version 2005/12/01 or later.
%
% This work has the LPPL maintenance status `maintained'.
%
% The Current Maintainer of this work is Niklas Beisert.
%
% This work consists of the files childdoc.dtx and childdoc.ins
% and the derived files childdoc.def and cdocsamp.tex with
% cdocsch1.tex, cdocsch2.tex, cdocsdrf.tex, cdocsfn1.tex, cdocsfn2.tex.
%
%<package>\ifdefined\childdocmain\endinput\fi
%<package>\ProvidesFile{childdoc.def}[2018/12/30 v2.0 child document driver]
%<samplemain>\ProvidesFile{cdocsamp.tex}[2018/12/30 v2.0 sample for childdoc]
%<*driver>
%\ProvidesFile{childdoc.drv}[2018/12/30 v2.0 childdoc reference manual file]
\PassOptionsToClass{10pt,a4paper}{article}
\documentclass{ltxdoc}

\usepackage[margin=35mm]{geometry}
\usepackage{hyperref}
\usepackage{hyperxmp}
\usepackage[usenames]{color}

\hypersetup{colorlinks=true}
\hypersetup{pdfstartview=FitH}
\hypersetup{pdfpagemode=UseNone}
\hypersetup{pdfsource={}}
\hypersetup{pdflang={en-UK}}
\hypersetup{pdfcopyright={Copyright 2017-2018 Niklas Beisert.
  This work may be distributed and/or modified under the
  conditions of the LaTeX Project Public License, either version 1.3
  of this license or (at your option) any later version.}}
\hypersetup{pdflicenseurl={http://www.latex-project.org/lppl.txt}}
\hypersetup{pdfcontactaddress={ETH Zurich, ITP, HIT K,
  Wolfgang-Pauli-Strasse 27}}
\hypersetup{pdfcontactpostcode={8093}}
\hypersetup{pdfcontactcity={Zurich}}
\hypersetup{pdfcontactcountry={Switzerland}}
\hypersetup{pdfcontactemail={nbeisert@itp.phys.ethz.ch}}
\hypersetup{pdfcontacturl={http://people.phys.ethz.ch/\xmptilde nbeisert/}}

\newcommand{\secref}[1]{\hyperref[#1]{section \ref*{#1}}}

\parskip1ex
\parindent0pt
\let\olditemize\itemize
\def\itemize{\olditemize\parskip0pt}

\begin{document}

\title{The \textsf{childdoc} Package}
\hypersetup{pdftitle={The childdoc Package}}
\author{Niklas Beisert\\[2ex]
  Institut f\"ur Theoretische Physik\\
  Eidgen\"ossische Technische Hochschule Z\"urich\\
  Wolfgang-Pauli-Strasse 27, 8093 Z\"urich, Switzerland\\[1ex]
  \href{mailto:nbeisert@itp.phys.ethz.ch}
  {\texttt{nbeisert@itp.phys.ethz.ch}}}
\hypersetup{pdfauthor={Niklas Beisert}}
\hypersetup{pdfsubject={Manual for the LaTeX2e Package childdoc}}
\date{30 December 2018, \textsf{v2.0}}
\maketitle

\begin{abstract}\noindent
\textsf{childdoc} is a \LaTeXe{} package
that enables the direct compilation
of document sections included by |\include|
to individual files.
\end{abstract}

\begingroup
\parskip0ex
\tableofcontents
\endgroup

%%%%%%%%%%%%%%%%%%%%%%%%%%%%%%%%%%%%%%%%%%%%%%%%%%%%%%%%%%%%%%%%%%%%%%%%%%%%%%%%
%%%%%%%%%%%%%%%%%%%%%%%%%%%%%%%%%%%%%%%%%%%%%%%%%%%%%%%%%%%%%%%%%%%%%%%%%%%%%%%%
\section{Introduction}

\LaTeX{} provides a mechanism to structure a large document (such as a book)
into a main file and several child files (containing the chapters)
using the |\include| command.
This mechanism is beneficial for documents
which span hundreds of pages in order to
make the source file(s) more manageable.
Moreover, compilation can be restricted to
selected child files by means of the |\includeonly| command.
The latter feature can be used to reduce the compilation time while editing
(this was significantly more useful in the earlier days of \LaTeX{})
or to generate a smaller document which is easier to navigate.
Another application of |\includeonly| is to generate
documents consisting of selected parts of the complete document.

However, there are a few drawbacks of the plain |\include| mechanism:
\begin{itemize}
\item
The child files cannot be compiled on their own,
they can only be compiled via the main file.
A naive editing environment
(such as a text editor with an option
to have the current file processed by \LaTeX)
may require one to switch to the main file before compiling;
attempting to compile the child file produces errors.
\item
The main file must be modified (each time)
to adjust the |\includeonly| command
to the present needs. This easily leaves the main file in a messy state.
\item
The generated document will always carry the filename
of the main document. This is inconvenient if
several child files are to be compiled and
to be kept for distribution.
\end{itemize}

The present package provides a simple interface
to make child files individually compilable by \LaTeX{}.
Compiling a child file then has the same effect as compiling
the main file with an |\includeonly| command
to select the appropriate child.
Moreover the generated document will carry the name of the child
rather than the main file.
This resolves all three above issues.

This feature is meant to make the editing of books,
thesis documents and lecture notes somewhat more convenient.
However, the package can also be used efficiently for
composing a series of documents (such as exercise sheets)
which are typically distributed individually.
It then assists the author in generating the individual documents
(potentially in different versions)
as well as a document containing the collected series.
Another application is in developing style files
or other kinds of included material
where compilation of the style file could redirect
to a sample or test file.

%%%%%%%%%%%%%%%%%%%%%%%%%%%%%%%%%%%%%%%%%%%%%%%%%%%%%%%%%%%%%%%%%%%%%%%%%%%%%%%%
%%%%%%%%%%%%%%%%%%%%%%%%%%%%%%%%%%%%%%%%%%%%%%%%%%%%%%%%%%%%%%%%%%%%%%%%%%%%%%%%
\section{Usage}

First of all, the package \textsf{childdoc} is \emph{not} a standard
\LaTeXe{} |.sty| style file! Therefore it needs to be invoked in
a non-standard way.

%%%%%%%%%%%%%%%%%%%%%%%%%%%%%%%%%%%%%%%%%%%%%%%%%%%%%%%%%%%%%%%%%%%%%%%%%%%%%%%%
\subsection{Included Files}
\label{sec:include}

%%%%%%%%%%%%%%%%%%%%%%%%%%%%%%%%%%%%%%%%
\DescribeMacro{\childdocmain}
To use the package, add the commands
\begin{center}
\begin{tabular}{l}
|\input{childdoc.def}|\\
|\childdocmain{}|\\
\end{tabular}
\end{center}
at the very top of the main \LaTeX{} file,
in particular \emph{before} the |\documentclass| statement!
The argument of |\childdocmain| should be left empty
(but it must be present).

%%%%%%%%%%%%%%%%%%%%%%%%%%%%%%%%%%%%%%%%
\DescribeMacro{\childdocof}
Furthermore, add the commands
\begin{center}
\begin{tabular}{l}
|\input{childdoc.def}|\\
|\childdocof{|\textit{main}|}|\\
\end{tabular}
\end{center}
at the top of every child file \textit{child}
which is included by |\include{|\textit{child}|}|
from within the main file
(or at least for those files to be compiled individually).
The argument \textit{main} must be the filename of the main file.

There are a couple of
considerations in setting up the main and child documents:

%%%%%%%%%%%%%%%%%%%%%%%%%%%%%%%%%%%%%%%%
\paragraph{Restrictions.}

Please note the following restrictions:
\begin{itemize}
\item
|\childdocmain| must be called with one argument \textit{main}
to ensure compatibility with earlier version of the package.
It must either be empty (|\childdocmain{}|)
or precisely match the filename of the main file in which it is specified.
See \secref{sec:detection} for further information.
\item
The filename \textit{main} must be specified without the |.tex| extension.
\item
The filename \textit{main} is case sensitive
(even in case-insensitive file systems)
due to internal string comparison.
\item
The argument \textit{main} should be fully expanded, it cannot be a macro.
\item
Subdirectories and special characters should be avoided in filenames.
\item
The command |\childdocmain{|\textit{main}|}| must be followed by a whitespace.
It should not be followed immediately by another command
or by a comment mark `|%|'.
This is because the \TeX{} parser reads the token immediately following
the argument of |\childdocmain| and puts it
at the beginning of every child section;
however, a white\-space is ignored.
\end{itemize}

%%%%%%%%%%%%%%%%%%%%%%%%%%%%%%%%%%%%%%%%
\paragraph{Content of Main File.}

It is advisable to place all content in the child files included by |\include|.
Any output contained in the main file will appear in all child documents
unless suppressed manually;
it cannot be suppressed automatically by the |\includeonly| directive
and thus should normally be avoided.
A method to include some content in the main file
by means of conditional processing is described in \secref{sec:conditional}.

%%%%%%%%%%%%%%%%%%%%%%%%%%%%%%%%%%%%%%%%
\paragraph{Page Numbering.}

When only a part of the document is compiled,
the appropriate numbering of pages
(as well as other status parameters)
is determined from the |.aux| files.
The latter contain information from previous passes.
However this information needs to propagate through
all intermediate child documents.
Therefore the page numbering in child documents may well
be inconsistent until the complete document is compiled at least once.

A useful (if unconventional) way to always ensure a consistent
page numbering is to restart the numbering in each child document
and denote the pages by `\textit{child}|.|\textit{page}'
where \textit{child} represents the chapter/section number of the child file.
This can be achieved by the command
|\numberwithin{page}{|\textit{child}|}|
of the \textsf{amsmath} package
where \textit{child} can be |chapter| or |section|
depending on the chosen structuring.
Alternatively, one can modify the macro |\thepage| appropriately
and reset the counter |page| at the start of each child file.

%%%%%%%%%%%%%%%%%%%%%%%%%%%%%%%%%%%%%%%%%%%%%%%%%%%%%%%%%%%%%%%%%%%%%%%%%%%%%%%%
\subsection{Conditional Processing}
\label{sec:conditional}

The package provides a mechanism to compile different versions
of a document. To customise the versions further some conditional processing
can come in handy to distinguish which version is being compiled.
The package provides two macros to describe the compilation context:

%%%%%%%%%%%%%%%%%%%%%%%%%%%%%%%%%%%%%%%%
\DescribeMacro{\ifchilddoc}
The conditional |\ifchilddoc| distinguishes between the compilation of
child documents and the main document:
%
\begin{center}
|\ifchilddoc |\textit{child-code}| |[|\||else |\textit{main-code}]| \||fi|
\end{center}

%%%%%%%%%%%%%%%%%%%%%%%%%%%%%%%%%%%%%%%%
\DescribeMacro{\childdocname}
\DescribeMacro{\childdocjob}
The macro |\childdocname| contains the filename (without extension)
of the main or child file being processed.
Note that |\childdocjob| will always contain the name of the main file.

%%%%%%%%%%%%%%%%%%%%%%%%%%%%%%%%%%%%%%%%
\paragraph{Title Page.}

Conditional processing can be used to include a title or banner page
in the main document when proper precautions are taken.
Importantly, the code in the main file should ensure that the page counter
(as well as other status parameters which are stored in the |.aux| files)
takes the same value after the conditional processing.
Otherwise the page numbers may take divergent values
depending on which part is compiled.

For example, a title page could be declared by:
%
\begin{center}
\begin{tabular}{l}
|\ifchilddoc\||else|\\
|\addtocounter{page}{-1}|\\
\textit{code for title page}\\
|\newpage|\\
|\||fi|
\end{tabular}
\end{center}
%
A banner page for the child documents can be generated by:
%
\begin{center}
\begin{tabular}{l}
|\ifchilddoc|\\
|\addtocounter{page}{-1}|\\
\textit{code for banner page}\\
|\newpage|\\
|\||fi|
\end{tabular}
\end{center}
%
Here one could write a message such as:
\begin{center}
|This is the part \childdocname{} of \childdocjob{}.|
\end{center}

%%%%%%%%%%%%%%%%%%%%%%%%%%%%%%%%%%%%%%%%%%%%%%%%%%%%%%%%%%%%%%%%%%%%%%%%%%%%%%%%
\subsection{Flags}
\label{sec:flags}

The package makes it easy to generate different versions
of the main or child documents.
To this end compilation flags can be defined
and assigned different default values.
They will be particularly useful in conjunction
with the forwarding mechanism described in \secref{sec:forward}.

For example, it may be useful to have a flag |\version|
which can be set to |draft| or |final|.
The document source will contain some conditional code
depending on the value of |\version|.
Suppose further, the flag should default to |final| for the main file
and to |draft| for child files
which is a natural assignment for editing the document.
This is achieved by placing the following code
in the preamble of the main document
(below the |\childdocmain| directive):
%
\begin{center}
\begin{tabular}{l}
|\ifchilddoc|\\
|\providecommand{\version}{draft}|\\
|\||else|\\
|\providecommand{\version}{final}|\\
|\||fi|
\end{tabular}
\end{center}
%
The definition by |\providecommand| makes sure
that previous definitions are not overwritten.
Further statements |\providecommand{\version}{...}|
can thus be added before the above code to override it.

For the main file, one might add a line
(between |\childdocmain| and the above block)
%
\begin{center}
|%\ifchilddoc\||else\providecommand{\version}{draft}\||fi|
\end{center}
%
which can be uncommented to produce a draft version.
Likewise one can add a line to the very top of a child file
(above the |\childdocof{|\textit{main}|}| directive)
%
\begin{center}
|%\providecommand{\version}{final}|
\end{center}
%
which can be uncommented to produce the final version of this child document.

%%%%%%%%%%%%%%%%%%%%%%%%%%%%%%%%%%%%%%%%%%%%%%%%%%%%%%%%%%%%%%%%%%%%%%%%%%%%%%%%
\subsection{Forwarding}
\label{sec:forward}

Different versions of the main or child documents
using compilation flags as described in \secref{sec:flags}
can be (permanently) stored in different files
for convenient compilation, viewing and distribution.
To this end, the package defines a command
to pass on compilation to a different file:

%%%%%%%%%%%%%%%%%%%%%%%%%%%%%%%%%%%%%%%%
\DescribeMacro{\childdocforward}
The command |\childdocforward| redirects processing to
another source file:
%
\begin{center}
\begin{tabular}{l}
|\input{childdoc.def}|\\
|\childdocforward[|\textit{main}|]{|\textit{dest}|}|\\
\end{tabular}
\end{center}
%
The argument \textit{dest} is the destination file
(without extension).
It should be the main file or one of the child files.
Note that further \textsf{childdoc} directives
such as |\childdocof| and |\childdocforward|
in the indicated file will be processed in this form.
The optional argument \textit{main}
passes on directly to the main file \textit{main}
while pretending to compile the child \textit{dest}.
This form behaves as if \textit{dest}
issues |\childdocof{|\textit{main}|}| right away,
and no further \textsf{childdoc} directives will be processed.

%%%%%%%%%%%%%%%%%%%%%%%%%%%%%%%%%%%%%%%%
\DescribeMacro{\...prefix}
In the alternative form |\childdocforwardprefix|,
%
\begin{center}
\begin{tabular}{l}
|\input{childdoc.def}|\\
|\childdocforwardprefix[|\textit{main}|]{|\textit{prefix}|}{|\textit{dest}|}|
\end{tabular}
\end{center}
%
the destination file is determined by a pattern
depending on the current file:
To make this work, the current file must be called
`{\textit{prefix}\hspace{0.2em}\textit{suffix}}'
with \textit{prefix} matching precisely the argument.
Processing is then passed on to the file
`{\textit{dest}\hspace{0.2em}\textit{suffix}}'.
Surely, the same effect is achieved by
directly specifying the
argument `{\textit{dest}\hspace{0.2em}\textit{suffix}}'
in the first form.
However, that requires to set up a different file
for each child. With the alternative form of the command
all these files can have exactly the same content
which simplifies setting them up and maintaining them.

For example, the following file |draft.tex|
with a compilation flag |\version| as described in \secref{sec:flags}
compiles the main document as a draft:
%
\begin{center}
\begin{tabular}{l}
|\def\version{draft}|\\
|\input{childdoc.def}|\\
|\childdocforward{|\textit{main}|}|
\end{tabular}
\end{center}
%
Likewise, the following files |final|\textit{nn}|.tex|
compile the final version of the child document
|child|\textit{nn}|.tex|:
%
\begin{center}
\begin{tabular}{l}
|\def\version{final}|\\
|\input{childdoc.def}|\\
|\childdocforwardprefix{final}{child}|
\end{tabular}
\end{center}
%

Note that when several versions of a main file and/or of each child file
are to be generated, it may be convenient to set up a |Makefile| or
shell script to automatise the process.

%%%%%%%%%%%%%%%%%%%%%%%%%%%%%%%%%%%%%%%%%%%%%%%%%%%%%%%%%%%%%%%%%%%%%%%%%%%%%%%%
\subsection{Command Line Processing}
\label{sec:commandline}

The effect of redirection files can also be achieved by invoking
the \LaTeX{} compiler with a more elaborate command line.
Most conveniently this should be done as part
of a shell script or a |Makefile|.

When using \textsf{childdoc} in the main file, the following
command lines effectively perform a redirection
(note that depending on the shell being used,
backslashes may have to be doubled: `|\|' $\to$ `|\\|'):
%
\begin{center}
|... -jobname "|\textit{target}|" |\\|"|[\textit{flags}]%
|\input{childdoc.def}\childdocforward[|\textit{main}|]{|\textit{dest}|}"|
\end{center}
%
Here \textit{target} is the name of the output file,
\textit{main} is the name of the main file
and \textit{dest} is the name of the main or child file to be processed
(all filenames without extensions).
The optional argument \textit{main} can be omitted
if \textit{main} matches \textit{dest}.
Optionally, compilation \textit{flags} can be defined via |\def| commands.
This command line makes the \TeX{} engine believe
it is compiling the file \textit{target}
whose content is specified as the latter parameter.
The provided code then forwards the processing to
\textit{main} or \textit{dest} as described in \secref{sec:forward}.

%%%%%%%%%%%%%%%%%%%%%%%%%%%%%%%%%%%%%%%%%%%%%%%%%%%%%%%%%%%%%%%%%%%%%%%%%%%%%%%%
\subsection{Include by Input}
\label{sec:input}

Including child documents by |\include| has some restrictions by design.
Most notably, the content of a child document always occupies
its own set of pages; pages cannot be shared between child documents.
Usually, this behaviour makes perfect sense
because each child document contain an essential part of the document.
However, in some situations it may be desirable to compose
a document from a collection of parts
without having mandatory page breaks between then.
For this case, the package
provides a mechanism to include parts
by |\input| which can also be processed individually.
However, by construction this mechanism
requires manual handling of the content to be output.

%%%%%%%%%%%%%%%%%%%%%%%%%%%%%%%%%%%%%%%%
\DescribeMacro{\ifchilddocmanual}
The main file should be prepared as usual, see \secref{sec:include}.
However, the document body must make a distinction
between processing of an individual part and of the main document, e.g.:
%
\begin{center}
\begin{tabular}{l}
|\ifchilddocmanual|\\
|\input{\childdocname}|\\
|\||else|\\
\textit{document body with }|\input{|\textit{part}|}|\\
|\||fi|
\end{tabular}
\end{center}
%
The conditional |\ifchilddocmanual| is true whenever
a part to be included by |\input| is being compiled,
and the name of the part is stored in |\childdocname|.

%%%%%%%%%%%%%%%%%%%%%%%%%%%%%%%%%%%%%%%%
\DescribeMacro{\childdocby}
Each part to be included by |\input| should start with:
%
\begin{center}
\begin{tabular}{l}
|\input{childdoc.def}|\\
|\childdocby{|\textit{main}|}|\\
\end{tabular}
\end{center}
%
The directive |\childdocby| is similar to |\childdocof|
described in \secref{sec:include},
but the subsequent selection of content must be done manually.
To that end, both |\ifchilddoc| and |\ifchilddocmanual|
will be true upon processing of a part,
and the name of the part is stored in |\childdocname|.
Note that |\jobname| will be set to the filename of the current part
so that each part receives an individual |.aux| file
that does not interfere with the |.aux| file(s) of the main document.
This behaviour can be altered by the alternative form
|\childdocby[*]{|\textit{main}|}| (with a non-empty optional argument)
which uses the |.aux| file of the main document
by setting |\jobname| to \textit{main}.

%%%%%%%%%%%%%%%%%%%%%%%%%%%%%%%%%%%%%%%%%%%%%%%%%%%%%%%%%%%%%%%%%%%%%%%%%%%%%%%%
\subsection{Driver Development}
\label{sec:driver}

The \textsf{childdoc} mechanism can also be use for the development
of definition files such as \LaTeX{} styles or classes.
This case differs from the above setup with multiple parts
included by |\include| in that no |\includeonly| should be invoked.
This can be achieved by starting the include file
(before |\ProvidesPackage|) with:
%
\begin{center}
\begin{tabular}{l}
|\input{childdoc.def}|\\
|\childdocforward{|\textit{main}|}|\\
\end{tabular}
\end{center}
%
or alternatively with:
%
\begin{center}
\begin{tabular}{l}
|\input{childdoc.def}|\\
|\childdocby{|\textit{main}|}|\\
\end{tabular}
\end{center}
%
Both forms have slightly different effects as described above.
The main file is prepared as usual, see \secref{sec:include}.

%%%%%%%%%%%%%%%%%%%%%%%%%%%%%%%%%%%%%%%%%%%%%%%%%%%%%%%%%%%%%%%%%%%%%%%%%%%%%%%%
\subsection{Legacy Detection}
\label{sec:detection}

The directive |\childdocmain| in the main file can detect
whether the complete document or merely a child is to be compiled
even without using the directive |\childdocof|.
This method is deprecated because it is less robust
and there is no compelling reason to use it;
it is merely provided for backward compatibility
and it may be removed in future versions.

If the detection mechanism is to be used,
it is mandatory to correctly specify
the filename of the main file as the argument of |\childdocmain|:
%
\begin{center}
\begin{tabular}{l}
|\input{childdoc.def}|\\
|\childdocmain{|\textit{main}|}|\\
\end{tabular}
\end{center}
%
If |\jobname| does not match the argument \textit{main} of |\childdocmain|,
it is assumed that |\jobname| points to the child file to be compiled.
When using |\childdocmain| with the main file specified as argument,
it suffices to start a child file
with just |\input{|\textit{main}|}|
without loading of the package and using |\childdocof|.
If instead all processing is done
with the appropriate \textsf{childdoc} directives,
the argument of \textit{main} of |\childdocmain| can be empty.

An alternative version of the command line processing described
in \secref{sec:commandline} using the detection mechanism reads:
%
\begin{center}
|... -jobname "|\textit{target}|" "|[\textit{flags}]%
[|\def\jobname{|\textit{dest}|}|]|\input{|\textit{main}|}"|
\end{center}

%%%%%%%%%%%%%%%%%%%%%%%%%%%%%%%%%%%%%%%%%%%%%%%%%%%%%%%%%%%%%%%%%%%%%%%%%%%%%%%%
\subsection{Manual Code}
\label{sec:manual}

In case one cannot be certain whether the definitions file |childdoc.def|
is installed on the target \TeX{} distribution
and one prefers not to ship it,
it is conceivable to paste a few relevant commands into the sources.

To that end, drop all statements |\input{childdoc.def}|
and perform the replacements as outlined below.
Instead of |\childdocmain{|\textit{main}|}| add the following code
to the top of the main file:
%
\begin{center}
\begin{tabular}{l}
|\||ifdefined\childdocname\endinput\||fi\newif\ifchilddoc|\\
|\edef\childdocname{\scantokens\expandafter{\jobname\noexpand}}|\\
|\def\childdocmain{|\textit{main}|}\||ifx\childdocmain\childdocname\||else|\\
|\childdoctrue\includeonly{\childdocname}\let\jobname\childdocmain\||fi|\\
\end{tabular}
\end{center}
%
Instead of |\childdocof{|\textit{main}|}| just include the main file
at the top of each child file:
%
\begin{center}
|\input{|\textit{main}|}|
\end{center}
%
A simple redirection |\childdocforward{|\textit{dest}|}| is achieved by:
%
\begin{center}
|\def\jobname{|\textit{dest}|}\input{\jobname}|
\end{center}
%
The redirection with prefix
|\childdocforwardprefix[|\textit{prefix}|]{|\textit{dest}|}|
is accomplished by:
%
\begin{center}
\begin{tabular}{l}
|{\edef\jobname{\scantokens\expandafter{\jobname\noexpand}}|\\
|\def\redirectjob |\textit{prefix}|#1~~~{\gdef\jobname{|\textit{dest}|#1}}|\\
|\expandafter\redirectjob\jobname~~~}\input{\jobname}|
\end{tabular}
\end{center}

In an alternative approach,
child documents can be compiled by a specific command line
without additional code or specific definitions:
%
\begin{center}
|... -jobname "|\textit{target}|" "|[\textit{flags}]%
|\includeonly{|\textit{dest}|}\input{|\textit{main}|}"|
\end{center}
%

%%%%%%%%%%%%%%%%%%%%%%%%%%%%%%%%%%%%%%%%%%%%%%%%%%%%%%%%%%%%%%%%%%%%%%%%%%%%%%%%
%%%%%%%%%%%%%%%%%%%%%%%%%%%%%%%%%%%%%%%%%%%%%%%%%%%%%%%%%%%%%%%%%%%%%%%%%%%%%%%%
\section{Information}

%%%%%%%%%%%%%%%%%%%%%%%%%%%%%%%%%%%%%%%%%%%%%%%%%%%%%%%%%%%%%%%%%%%%%%%%%%%%%%%%
\subsection{Copyright}

Copyright \copyright{} 2017--2018 Niklas Beisert

This work may be distributed and/or modified under the
conditions of the \LaTeX{} Project Public License, either version 1.3
of this license or (at your option) any later version.
The latest version of this license is in
  \url{http://www.latex-project.org/lppl.txt}
and version 1.3 or later is part of all distributions of \LaTeX{}
version 2005/12/01 or later.

This work has the LPPL maintenance status `maintained'.

The Current Maintainer of this work is Niklas Beisert.

This work consists of the files |README.txt|, |childdoc.ins| and |childdoc.dtx|
as well as the derived files |childdoc.def|, |cdocsamp.tex|
with |cdocsch1.tex|, |cdocsch2.tex|, |cdocspt3.tex|, |cdocspt4.tex|,
|cdocsdrf.tex|, |cdocsfn1.tex|, |cdocsfn2.tex|
as well as |childdoc.pdf|.

%%%%%%%%%%%%%%%%%%%%%%%%%%%%%%%%%%%%%%%%%%%%%%%%%%%%%%%%%%%%%%%%%%%%%%%%%%%%%%%%
\subsection{Files and Installation}

The package consists of the files:
%
\begin{center}
\begin{tabular}{ll}
    |README.txt|   & readme file \\
    |childdoc.ins| & installation file \\
    |childdoc.dtx| & source file \\
    |childdoc.def| & definition file \\
    |cdocsamp.tex| & sample main file \\
    |cdocsch1.tex| & sample include file \\
    |cdocsch2.tex| & sample include file \\
    |cdocspt3.tex| & sample part file \\
    |cdocspt4.tex| & sample part file \\
    |cdocsdrf.tex| & sample redirection file \\
    |cdocsfn1.tex| & sample redirection file \\
    |cdocsfn2.tex| & sample redirection file \\
    |childdoc.pdf| & manual
\end{tabular}
\end{center}
%
The distribution consists of the files
|README.txt|, |childdoc.ins| and |childdoc.dtx|.
%
\begin{itemize}
\item
Run (pdf)\LaTeX{} on |childdoc.dtx|
to compile the manual |childdoc.pdf| (this file).
\item
Run \LaTeX{} on |childdoc.ins| to create the definitions file |childdoc.def|
and the sample |cdocsamp.tex| with include files
|cdocsch1.tex|, |cdocsch2.tex|, |cdocspt3.tex|, |cdocspt4.tex|,
|cdocsdrf.tex|, |cdocsfn1.tex|, |cdocsfn2.tex|.
Then copy the file |childdoc.def| to an appropriate directory of your \LaTeX{}
distribution, e.g.\ \textit{texmf-root}|/tex/latex/childdoc|.
\end{itemize}

%%%%%%%%%%%%%%%%%%%%%%%%%%%%%%%%%%%%%%%%%%%%%%%%%%%%%%%%%%%%%%%%%%%%%%%%%%%%%%%%
\subsection{Related CTAN Packages}

There are several other packages which offer a similar functionality:
%
\begin{itemize}
\item
The packages
\href{http://ctan.org/pkg/docmute}{\textsf{docmute}},
\href{http://ctan.org/pkg/includex}{\textsf{includex}} and
\href{http://ctan.org/pkg/standalone}{\textsf{standalone}}
provide commands to include only the document body of
a child file thus allowing both files to be compiled individually.
\item
The packages \href{http://ctan.org/pkg/subdocs}{\textsf{subdocs}}
and \href{http://ctan.org/pkg/subfiles}{\textsf{subfiles}}
provide structures in which the main and child documents can be
encapsulated and allowing them to be compiled individually.
The inclusion mechanism is different from the conventional |\include|.
\item
The package \href{http://ctan.org/pkg/combine}{\textsf{combine}}
is an elaborate solution to combine several documents into one.
\end{itemize}
%
See also the CTAN topic \href{http://ctan.org/topic/subdocs}{\textsf{subdocs}}
for further related packages.
The present package differs from the above solutions in that
a document structure constructed with the conventional |\include| mechanism
just needs two extra commands at the top of every file
such that all constituent files can be compiled individually.

%%%%%%%%%%%%%%%%%%%%%%%%%%%%%%%%%%%%%%%%%%%%%%%%%%%%%%%%%%%%%%%%%%%%%%%%%%%%%%%%
%\subsection{Feature Suggestions}
%
%The following is a list of features which may be useful for future
%versions of this package:
%%
%\begin{itemize}
%\item
%\ldots
%\end{itemize}

%%%%%%%%%%%%%%%%%%%%%%%%%%%%%%%%%%%%%%%%%%%%%%%%%%%%%%%%%%%%%%%%%%%%%%%%%%%%%%%%
\subsection{Revision History}

%%%%%%%%%%%%%%%%%%%%%%%%%%%%%%%%%%%%%%%%
\paragraph{v2.0:} 2018/12/30

\begin{itemize}
\item
immediate forward processing
\item
added |\childdocby| mechanism
\item
manual restructured
\end{itemize}

%%%%%%%%%%%%%%%%%%%%%%%%%%%%%%%%%%%%%%%%
\paragraph{v1.6:} 2018/01/17

\begin{itemize}
\item
application for development of include files
\item
corrections to manual
\end{itemize}

%%%%%%%%%%%%%%%%%%%%%%%%%%%%%%%%%%%%%%%%
\paragraph{v1.5:} 2017/05/21

\begin{itemize}
\item
more complete structuring introduced
\item
|\childdocof| introduced
\item
|\childdoc| renamed to |\childdocmain|
\item
|\childredirect| renamed to |\childdocforward| and |\childdocforwardprefix|
and functionality expanded
\end{itemize}

%%%%%%%%%%%%%%%%%%%%%%%%%%%%%%%%%%%%%%%%
\paragraph{v1.0:} 2017/04/27

\begin{itemize}
\item
manual and install package
\item
first version published on CTAN
\end{itemize}

%%%%%%%%%%%%%%%%%%%%%%%%%%%%%%%%%%%%%%%%
\paragraph{v0.6:} 2017/04/26

\begin{itemize}
\item
redirection mechanism added
\end{itemize}

%%%%%%%%%%%%%%%%%%%%%%%%%%%%%%%%%%%%%%%%
\paragraph{v0.5:} 2017/04/26

\begin{itemize}
\item
functionality in definition file
\end{itemize}


%%%%%%%%%%%%%%%%%%%%%%%%%%%%%%%%%%%%%%%%%%%%%%%%%%%%%%%%%%%%%%%%%%%%%%%%%%%%%%%%
%%%%%%%%%%%%%%%%%%%%%%%%%%%%%%%%%%%%%%%%%%%%%%%%%%%%%%%%%%%%%%%%%%%%%%%%%%%%%%%%
%%%%%%%%%%%%%%%%%%%%%%%%%%%%%%%%%%%%%%%%%%%%%%%%%%%%%%%%%%%%%%%%%%%%%%%%%%%%%%%%
\appendix

\settowidth\MacroIndent{\rmfamily\scriptsize 000\ }

 \DocInput{childdoc.dtx}

\end{document}
%</driver>
% \fi
%
% %%%%%%%%%%%%%%%%%%%%%%%%%%%%%%%%%%%%%%%%%%%%%%%%%%%%%%%%%%%%%%%%%%%%%%%%%%%%%%
% %%%%%%%%%%%%%%%%%%%%%%%%%%%%%%%%%%%%%%%%%%%%%%%%%%%%%%%%%%%%%%%%%%%%%%%%%%%%%%
% \section{Sample}
%\iffalse
%<*samplemain>
%\fi
%
% The following presents a sample document
% with two chapters, two parts, a title page,
% a compile flag as well as three forwarding files to set the flag.
% It consists of eight |.tex| files:
% \begin{center}
% \begin{tabular}{ll}
% |cdocsamp.tex|&main file\\
% |cdocsch1.tex|&include file for chapter 1\\
% |cdocsch2.tex|&include file for chapter 2\\
% |cdocspt3.tex|&include file for part 3\\
% |cdocspt4.tex|&include file for part 4\\
% |cdocsdrf.tex|&forwarding file for main file in draft mode\\
% |cdocsfi1.tex|&forwarding file for final version of chapter 1\\
% |cdocsfi2.tex|&forwarding file for final version of chapter 2\\
% \end{tabular}
% \end{center}
% Each of the eight files can be compiled directly by the \LaTeX{} compiler.
%
% %%%%%%%%%%%%%%%%%%%%%%%%%%%%%%%%%%%%%%
% \paragraph{Main File.}
%
% The main file is called |cdocsamp.tex|.
%
% Load the \textsf{childdoc} definitions and
% declare the filename for the main document:
%    \begin{macrocode}
\input{childdoc.def}
\childdocmain{}
%    \end{macrocode}

% Optional override for |\version| flag:
%    \begin{macrocode}
%%\ifchilddoc\else\providecommand{\version}{draft}\fi
%    \end{macrocode}

% Define the default values for the |\version| flag
% (|final| for the main file and |draft| for childs):
%    \begin{macrocode}
\ifchilddoc
\providecommand{\version}{draft}
\else
\providecommand{\version}{final}
\fi
%    \end{macrocode}

% Load the standard document class:
%    \begin{macrocode}
\documentclass[12pt]{article}
%    \end{macrocode}

% Start the document body:
%    \begin{macrocode}
\begin{document}
%    \end{macrocode}

% Declare a title page.
% Print title, part of document being processed and version flag:
%    \begin{macrocode}
\addtocounter{page}{-1}
\begin{center}
{\LARGE\bfseries{}childdoc example\par}
\vspace{1cm}
\ifchilddoc
\ifchilddocmanual part\else chapter\fi:
`\childdocname' of `\childdocjob'\par
\else
main document: `\childdocjob'\par
\fi
version: \version\par
\end{center}
\newpage
%    \end{macrocode}

% Manually include selected file,
% otherwise process as usual:
%    \begin{macrocode}
\ifchilddocmanual
\section*{part `\childdocname'}
\input{\childdocname}
\else
%    \end{macrocode}

% Include the two chapters:
%    \begin{macrocode}
\include{cdocsch1}
\include{cdocsch2}
%    \end{macrocode}

% Include the two parts unless only chapters should be displayed:
%    \begin{macrocode}
\ifchilddoc\else
\section{part three}
\input{cdocspt3}
\section{part four}
\input{cdocspt4}
\fi
%    \end{macrocode}

% Process as usual until here:
%    \begin{macrocode}
\fi
%    \end{macrocode}

% End of document body:
%    \begin{macrocode}
\end{document}
%    \end{macrocode}
%\iffalse
%</samplemain>
%\fi
%
% %%%%%%%%%%%%%%%%%%%%%%%%%%%%%%%%%%%%%%
% \paragraph{Chapter Include Files.}
%
% The include files are called |cdocsch1.tex| and |cdocsch2.tex|.
%
%\iffalse
%<*samplechap1|samplechap2>
%\fi

% Optional override for |\version| flag:
%    \begin{macrocode}
%%\providecommand{\version}{final}
%    \end{macrocode}

% Include the main document:
%    \begin{macrocode}
\input{childdoc.def}
\childdocof{cdocsamp}
%    \end{macrocode}

%\iffalse
%</samplechap1|samplechap2>
%\fi
%
%\iffalse
%<*samplechap1>
%\fi
% Some text for chapter 1:
%    \begin{macrocode}
\section{one}
some text in chapter one
%    \end{macrocode}

%\iffalse
%</samplechap1>
%\fi
% Some text for chapter 2:
%\iffalse
%<*samplechap2>
%\fi
%    \begin{macrocode}
\section{two}
more text in chapter two
%    \end{macrocode}

%\iffalse
%</samplechap2>
%\fi
%
% %%%%%%%%%%%%%%%%%%%%%%%%%%%%%%%%%%%%%%
% \paragraph{Part Include Files.}
%
% The include files are called |cdocspt3.tex| and |cdocspt4.tex|.
%
%\iffalse
%<*samplepart3|samplepart4>
%\fi

% Optional override for |\version| flag:
%    \begin{macrocode}
%%\providecommand{\version}{final}
%    \end{macrocode}

% Include the main document:
%    \begin{macrocode}
\input{childdoc.def}
\childdocby{cdocsamp}
%    \end{macrocode}

%\iffalse
%</samplepart3|samplepart4>
%\fi
%
%\iffalse
%<*samplepart3>
%\fi
% Some text for part 3:
%    \begin{macrocode}
some text in part three
%    \end{macrocode}

%\iffalse
%</samplepart3>
%\fi
% Some text for part 4:
%\iffalse
%<*samplepart4>
%\fi
%    \begin{macrocode}
more text in part four
%    \end{macrocode}

%\iffalse
%</samplepart4>
%\fi
%
% %%%%%%%%%%%%%%%%%%%%%%%%%%%%%%%%%%%%%%
% \paragraph{Forwarding for a Complete Draft.}
%
% The following forwarding file |cdocsdrf.tex|
% compiles the main document in draft mode:
%\iffalse
%<*sampledraft>
%\fi
%    \begin{macrocode}
\def\version{draft}
\input{childdoc.def}
\childdocforward{cdocsamp}
%    \end{macrocode}

%\iffalse
%</sampledraft>
%\fi
%
% %%%%%%%%%%%%%%%%%%%%%%%%%%%%%%%%%%%%%%
% \paragraph{Forwarding for Final Version of the Chapters.}
%
% The following forwarding files |cdocsfn1.tex| and |cdocsfn2.tex|
% (with identical content)
% compile the final versions of the child documents
% |cdocsch1.tex| and |cdocsch2.tex|, respectively:
%\iffalse
%<*samplefinal>
%\fi
%    \begin{macrocode}
\def\version{final}
\input{childdoc.def}
\childdocforwardprefix[cdocsamp]{cdocsfn}{cdocsch}
%    \end{macrocode}

%\iffalse
%</samplefinal>
%\fi
%
% %%%%%%%%%%%%%%%%%%%%%%%%%%%%%%%%%%%%%%
% \paragraph{Command Line Processing.}
%
% The following three command lines generate the output files
% |cdocscld|, |cdocscl1| and |cdocscl2|
% which should be identical to
% |cdocsdrf|, |cdocsch1| and |cdocsfn2|, respectively:
% \begin{center}
% \begin{tabular}{l}
% |latex -jobname cdocscld \|\\
% |  "\def\version{draft}\input{childdoc.def}\childdocforward{cdocsamp}"|\\
% |latex -jobname cdocscl1 \|\\
% |  "\input{childdoc.def}\childdocforward[cdocsamp]{cdocsch1}"|\\
% |latex -jobname cdocscl2 \|\\
% |  "\def\version{final}\input{childdoc.def}\childdocforward{cdocsch2}"|
% \end{tabular}
% \end{center}
% Note that the trailing backslash on each first line
% merely continues the input to the second line
% (for convenient cut ant paste).
% Furthermore, the command |latex| can be replaced by any
% of its alternative versions such as |pdflatex|.
%
% %%%%%%%%%%%%%%%%%%%%%%%%%%%%%%%%%%%%%%%%%%%%%%%%%%%%%%%%%%%%%%%%%%%%%%%%%%%%%%
% %%%%%%%%%%%%%%%%%%%%%%%%%%%%%%%%%%%%%%%%%%%%%%%%%%%%%%%%%%%%%%%%%%%%%%%%%%%%%%
% \section{Implementation}
%\iffalse
%<*package>
%\fi
%
% This section describes the definitions file |childdoc.def|.

% The definitions cannot be loaded using |\usepackage| or |\RequirePackage|
% which has a mechanism to prevent loading a style file more than once.
% When loading the definitions by means of |\input|
% multiple instances have to be prevented manually:
%\iffalse
%This code needs to be before the `\ProvidesFile' directive
%which is defined at the beginning of this file.
%Therefore it is also placed there and commented out here.
%</package>
%<*discard>
%\fi
%    \begin{macrocode}
\ifdefined\childdocmain\endinput\fi
%    \end{macrocode}
%\iffalse
%</discard>
%<*package>
%\fi
%
% \macro{\ifchilddoc}
% \macro{\ifchilddocmanual}
% The conditional |\ifchilddoc| tells whether a
% child (true) or main (false) document is being compiled.
% The conditional |\ifchilddocmanual| tells whether
% the |\includeonly| mechanism is used (false) or
% the selection of child files must be performed manually (true).
% The definitions initialise to false:
%    \begin{macrocode}
\newif\ifchilddoc
\newif\ifchilddocmanual
%    \end{macrocode}

% \macro{\childdocname}
% \macro{\childdocjob}
% The macro |\childdocname| stores the name of the main document
% to be compiled. The macro |\childdocjob| stores the name of
% the document on which the \LaTeX{} compiler was originally invoked.
% The content of |\jobname| cannot be compared
% to filenames specified in the source due to different catcodes.
% The following code rescans |\jobname|, stores the result
% in |\childdocname| and saves a copy in |\childdocjob|:
%    \begin{macrocode}
\edef\childdocname{\scantokens\expandafter{\jobname\noexpand}}
\let\childdocjob\childdocname
%    \end{macrocode}

% \macro{\childdocdisable}
% The macro |\childdocdisable| prevents the main file
% from being processed more than once.
% At this stage, the main document command |\childdocmain|
% is assumed to be called once again where it should do nothing.
% Any subsequent call to it should prevent
% a secondary processing of the main document
% It overwrites the forwarding commands
% |\childdocof| and |\childdocforward|
% with empty macros to prevent further inclusions of the main document:
%    \begin{macrocode}
\newcommand{\childdocdisable}
{
  \renewcommand{\childdocmain}[1]{\renewcommand{\childdocmain}[1]{\endinput}}
  \renewcommand{\childdocof}[1]{}
  \renewcommand{\childdocby}[2][]{}
  \renewcommand{\childdocforward}[2][]{}
  \renewcommand{\childdocdisable}{}
}
%    \end{macrocode}

% \macro{\childdocmain}
% The macro |\childdocmain| is to be called at the top of the main file
% with nothing or the main filename (without extension) as argument.
% First, it breaks loops.
% If the argument is not empty and does not match |\childdocname|
% (which is set by the first inclusion of |childdoc.def|),
% |\ifchilddoc| is set to true, |\includeonly| is applied to the child file
% and |\jobname| is set to the main file
% (for proper handling of |.aux| files):
%    \begin{macrocode}
\newcommand{\childdocmain}[1]
{
  \childdocdisable\childdocmain{}
  \if?#1?\else
    \begingroup
      \def\childdoctmp{#1}
      \ifx\childdoctmp\childdocname
        \def\childdoctmp{}
      \else
        \def\childdoctmp
        {
          \childdoctrue
          \includeonly{\childdocname}
          \def\childdocjob{#1}
          \def\jobname{#1}
        }
      \fi
      \expandafter
    \endgroup
    \childdoctmp
  \fi
}
%    \end{macrocode}

% \macro{\childdocof}
% The command |\childdocof| redirects
% compilation to the main file |#1|.
%    \begin{macrocode}
\newcommand{\childdocof}[1]
{
  \childdocdisable
  \childdoctrue
  \includeonly{\childdocname}
  \def\jobname{#1}
  \def\childdocjob{#1}
  \input{#1}
}
%    \end{macrocode}

% \macro{\childdocby}
% The command |\childdocby| ....
%    \begin{macrocode}
\newcommand{\childdocby}[2][]
{
  \childdocdisable
  \childdoctrue
  \childdocmanualtrue
  \if?#1?\else
    \def\jobname{#2}
  \fi
  \def\childdocjob{#2}
  \input{#2}
  \endinput
}
%    \end{macrocode}

% \macro{\childdocforward}
% The command |\childdocforward| redirects
% compilation to the main file or
% (if the optional argument is given) a child file.
% Parameters are set as if the main file
% or a child file starting with |\childdocof| was compiled.
% Then compilation is handed over to the main file:
%    \begin{macrocode}
\newcommand{\childdocforward}[2][]
{
  \begingroup
    \if?#1?
      \def\childdoctmp
      {
        \def\childdocname{#2}
        \def\childdocjob{#2}
        \def\jobname{#2}
        \input{#2}
        \endinput
      }
    \else
      \def\childdoctmp
      {
        \childdocdisable
        \def\childdocname{#2}
        \childdoctrue
        \includeonly{#2}
        \def\childdocjob{#1}
        \def\jobname{#1}
        \input{#1}
        \endinput
      }
    \fi
    \expandafter
  \endgroup
  \childdoctmp
}
%    \end{macrocode}

% \macro{\childdocforwardprefix}
% The command |\childdocforwardprefix| redirects
% compilation to the main or a child file by means of a pattern.
% The prefix |#1| in the current filename is replaced by |#2|
% and the suffix of the current filename is kept
% (it is assumed that the filename does not contain the substring `|~~~|'
% which is used as a delimiter).
% Compilation is handed over to the new file by |\childdocforward|:
%    \begin{macrocode}
\newcommand{\childdocforwardprefix}[3][]
{
  \begingroup
    \def\childdocextract #2##1~~~{\def\childdoctmp{\childdocforward[#1]{#3##1}}}
    \expandafter\childdocextract\childdocname~~~
    \expandafter
  \endgroup
  \childdoctmp
}
%    \end{macrocode}

% \macro{\childdoc}
% The deprecated macro |\childdoc| is a legacy version of |\childdocmain|:
%    \begin{macrocode}
\newcommand{\childdoc}{\childdocmain}
%    \end{macrocode}

% \macro{\childdocredirect}
% The deprecated macro |\childdocredirect| is a legacy version
% of |\childdocforward| and |\childdocforwardprefix|:
%    \begin{macrocode}
\newcommand{\childdocredirect}[2][]
{
  \begingroup
    \if?#1?
      \def\childdoctmp{\childdocforward{#2}}
    \else
      \def\childdoctmp{\childdocforwardprefix{#1}{#2}}
    \fi
    \expandafter
  \endgroup
  \childdoctmp
}
%    \end{macrocode}

%\iffalse
%</package>
%\fi
%
\endinput
|\\
|\childdocforward[|\textit{main}|]{|\textit{dest}|}|\\
\end{tabular}
\end{center}
%
The argument \textit{dest} is the destination file
(without extension).
It should be the main file or one of the child files.
Note that further \textsf{childdoc} directives
such as |\childdocof| and |\childdocforward|
in the indicated file will be processed in this form.
The optional argument \textit{main}
passes on directly to the main file \textit{main}
while pretending to compile the child \textit{dest}.
This form behaves as if \textit{dest}
issues |\childdocof{|\textit{main}|}| right away,
and no further \textsf{childdoc} directives will be processed.

%%%%%%%%%%%%%%%%%%%%%%%%%%%%%%%%%%%%%%%%
\DescribeMacro{\...prefix}
In the alternative form |\childdocforwardprefix|,
%
\begin{center}
\begin{tabular}{l}
|% \iffalse
%
% childdoc.dtx Copyright (C) 2017-2018 Niklas Beisert
%
% This work may be distributed and/or modified under the
% conditions of the LaTeX Project Public License, either version 1.3
% of this license or (at your option) any later version.
% The latest version of this license is in
%   http://www.latex-project.org/lppl.txt
% and version 1.3 or later is part of all distributions of LaTeX
% version 2005/12/01 or later.
%
% This work has the LPPL maintenance status `maintained'.
%
% The Current Maintainer of this work is Niklas Beisert.
%
% This work consists of the files childdoc.dtx and childdoc.ins
% and the derived files childdoc.def and cdocsamp.tex with
% cdocsch1.tex, cdocsch2.tex, cdocsdrf.tex, cdocsfn1.tex, cdocsfn2.tex.
%
%<package>\ifdefined\childdocmain\endinput\fi
%<package>\ProvidesFile{childdoc.def}[2018/12/30 v2.0 child document driver]
%<samplemain>\ProvidesFile{cdocsamp.tex}[2018/12/30 v2.0 sample for childdoc]
%<*driver>
%\ProvidesFile{childdoc.drv}[2018/12/30 v2.0 childdoc reference manual file]
\PassOptionsToClass{10pt,a4paper}{article}
\documentclass{ltxdoc}

\usepackage[margin=35mm]{geometry}
\usepackage{hyperref}
\usepackage{hyperxmp}
\usepackage[usenames]{color}

\hypersetup{colorlinks=true}
\hypersetup{pdfstartview=FitH}
\hypersetup{pdfpagemode=UseNone}
\hypersetup{pdfsource={}}
\hypersetup{pdflang={en-UK}}
\hypersetup{pdfcopyright={Copyright 2017-2018 Niklas Beisert.
  This work may be distributed and/or modified under the
  conditions of the LaTeX Project Public License, either version 1.3
  of this license or (at your option) any later version.}}
\hypersetup{pdflicenseurl={http://www.latex-project.org/lppl.txt}}
\hypersetup{pdfcontactaddress={ETH Zurich, ITP, HIT K,
  Wolfgang-Pauli-Strasse 27}}
\hypersetup{pdfcontactpostcode={8093}}
\hypersetup{pdfcontactcity={Zurich}}
\hypersetup{pdfcontactcountry={Switzerland}}
\hypersetup{pdfcontactemail={nbeisert@itp.phys.ethz.ch}}
\hypersetup{pdfcontacturl={http://people.phys.ethz.ch/\xmptilde nbeisert/}}

\newcommand{\secref}[1]{\hyperref[#1]{section \ref*{#1}}}

\parskip1ex
\parindent0pt
\let\olditemize\itemize
\def\itemize{\olditemize\parskip0pt}

\begin{document}

\title{The \textsf{childdoc} Package}
\hypersetup{pdftitle={The childdoc Package}}
\author{Niklas Beisert\\[2ex]
  Institut f\"ur Theoretische Physik\\
  Eidgen\"ossische Technische Hochschule Z\"urich\\
  Wolfgang-Pauli-Strasse 27, 8093 Z\"urich, Switzerland\\[1ex]
  \href{mailto:nbeisert@itp.phys.ethz.ch}
  {\texttt{nbeisert@itp.phys.ethz.ch}}}
\hypersetup{pdfauthor={Niklas Beisert}}
\hypersetup{pdfsubject={Manual for the LaTeX2e Package childdoc}}
\date{30 December 2018, \textsf{v2.0}}
\maketitle

\begin{abstract}\noindent
\textsf{childdoc} is a \LaTeXe{} package
that enables the direct compilation
of document sections included by |\include|
to individual files.
\end{abstract}

\begingroup
\parskip0ex
\tableofcontents
\endgroup

%%%%%%%%%%%%%%%%%%%%%%%%%%%%%%%%%%%%%%%%%%%%%%%%%%%%%%%%%%%%%%%%%%%%%%%%%%%%%%%%
%%%%%%%%%%%%%%%%%%%%%%%%%%%%%%%%%%%%%%%%%%%%%%%%%%%%%%%%%%%%%%%%%%%%%%%%%%%%%%%%
\section{Introduction}

\LaTeX{} provides a mechanism to structure a large document (such as a book)
into a main file and several child files (containing the chapters)
using the |\include| command.
This mechanism is beneficial for documents
which span hundreds of pages in order to
make the source file(s) more manageable.
Moreover, compilation can be restricted to
selected child files by means of the |\includeonly| command.
The latter feature can be used to reduce the compilation time while editing
(this was significantly more useful in the earlier days of \LaTeX{})
or to generate a smaller document which is easier to navigate.
Another application of |\includeonly| is to generate
documents consisting of selected parts of the complete document.

However, there are a few drawbacks of the plain |\include| mechanism:
\begin{itemize}
\item
The child files cannot be compiled on their own,
they can only be compiled via the main file.
A naive editing environment
(such as a text editor with an option
to have the current file processed by \LaTeX)
may require one to switch to the main file before compiling;
attempting to compile the child file produces errors.
\item
The main file must be modified (each time)
to adjust the |\includeonly| command
to the present needs. This easily leaves the main file in a messy state.
\item
The generated document will always carry the filename
of the main document. This is inconvenient if
several child files are to be compiled and
to be kept for distribution.
\end{itemize}

The present package provides a simple interface
to make child files individually compilable by \LaTeX{}.
Compiling a child file then has the same effect as compiling
the main file with an |\includeonly| command
to select the appropriate child.
Moreover the generated document will carry the name of the child
rather than the main file.
This resolves all three above issues.

This feature is meant to make the editing of books,
thesis documents and lecture notes somewhat more convenient.
However, the package can also be used efficiently for
composing a series of documents (such as exercise sheets)
which are typically distributed individually.
It then assists the author in generating the individual documents
(potentially in different versions)
as well as a document containing the collected series.
Another application is in developing style files
or other kinds of included material
where compilation of the style file could redirect
to a sample or test file.

%%%%%%%%%%%%%%%%%%%%%%%%%%%%%%%%%%%%%%%%%%%%%%%%%%%%%%%%%%%%%%%%%%%%%%%%%%%%%%%%
%%%%%%%%%%%%%%%%%%%%%%%%%%%%%%%%%%%%%%%%%%%%%%%%%%%%%%%%%%%%%%%%%%%%%%%%%%%%%%%%
\section{Usage}

First of all, the package \textsf{childdoc} is \emph{not} a standard
\LaTeXe{} |.sty| style file! Therefore it needs to be invoked in
a non-standard way.

%%%%%%%%%%%%%%%%%%%%%%%%%%%%%%%%%%%%%%%%%%%%%%%%%%%%%%%%%%%%%%%%%%%%%%%%%%%%%%%%
\subsection{Included Files}
\label{sec:include}

%%%%%%%%%%%%%%%%%%%%%%%%%%%%%%%%%%%%%%%%
\DescribeMacro{\childdocmain}
To use the package, add the commands
\begin{center}
\begin{tabular}{l}
|\input{childdoc.def}|\\
|\childdocmain{}|\\
\end{tabular}
\end{center}
at the very top of the main \LaTeX{} file,
in particular \emph{before} the |\documentclass| statement!
The argument of |\childdocmain| should be left empty
(but it must be present).

%%%%%%%%%%%%%%%%%%%%%%%%%%%%%%%%%%%%%%%%
\DescribeMacro{\childdocof}
Furthermore, add the commands
\begin{center}
\begin{tabular}{l}
|\input{childdoc.def}|\\
|\childdocof{|\textit{main}|}|\\
\end{tabular}
\end{center}
at the top of every child file \textit{child}
which is included by |\include{|\textit{child}|}|
from within the main file
(or at least for those files to be compiled individually).
The argument \textit{main} must be the filename of the main file.

There are a couple of
considerations in setting up the main and child documents:

%%%%%%%%%%%%%%%%%%%%%%%%%%%%%%%%%%%%%%%%
\paragraph{Restrictions.}

Please note the following restrictions:
\begin{itemize}
\item
|\childdocmain| must be called with one argument \textit{main}
to ensure compatibility with earlier version of the package.
It must either be empty (|\childdocmain{}|)
or precisely match the filename of the main file in which it is specified.
See \secref{sec:detection} for further information.
\item
The filename \textit{main} must be specified without the |.tex| extension.
\item
The filename \textit{main} is case sensitive
(even in case-insensitive file systems)
due to internal string comparison.
\item
The argument \textit{main} should be fully expanded, it cannot be a macro.
\item
Subdirectories and special characters should be avoided in filenames.
\item
The command |\childdocmain{|\textit{main}|}| must be followed by a whitespace.
It should not be followed immediately by another command
or by a comment mark `|%|'.
This is because the \TeX{} parser reads the token immediately following
the argument of |\childdocmain| and puts it
at the beginning of every child section;
however, a white\-space is ignored.
\end{itemize}

%%%%%%%%%%%%%%%%%%%%%%%%%%%%%%%%%%%%%%%%
\paragraph{Content of Main File.}

It is advisable to place all content in the child files included by |\include|.
Any output contained in the main file will appear in all child documents
unless suppressed manually;
it cannot be suppressed automatically by the |\includeonly| directive
and thus should normally be avoided.
A method to include some content in the main file
by means of conditional processing is described in \secref{sec:conditional}.

%%%%%%%%%%%%%%%%%%%%%%%%%%%%%%%%%%%%%%%%
\paragraph{Page Numbering.}

When only a part of the document is compiled,
the appropriate numbering of pages
(as well as other status parameters)
is determined from the |.aux| files.
The latter contain information from previous passes.
However this information needs to propagate through
all intermediate child documents.
Therefore the page numbering in child documents may well
be inconsistent until the complete document is compiled at least once.

A useful (if unconventional) way to always ensure a consistent
page numbering is to restart the numbering in each child document
and denote the pages by `\textit{child}|.|\textit{page}'
where \textit{child} represents the chapter/section number of the child file.
This can be achieved by the command
|\numberwithin{page}{|\textit{child}|}|
of the \textsf{amsmath} package
where \textit{child} can be |chapter| or |section|
depending on the chosen structuring.
Alternatively, one can modify the macro |\thepage| appropriately
and reset the counter |page| at the start of each child file.

%%%%%%%%%%%%%%%%%%%%%%%%%%%%%%%%%%%%%%%%%%%%%%%%%%%%%%%%%%%%%%%%%%%%%%%%%%%%%%%%
\subsection{Conditional Processing}
\label{sec:conditional}

The package provides a mechanism to compile different versions
of a document. To customise the versions further some conditional processing
can come in handy to distinguish which version is being compiled.
The package provides two macros to describe the compilation context:

%%%%%%%%%%%%%%%%%%%%%%%%%%%%%%%%%%%%%%%%
\DescribeMacro{\ifchilddoc}
The conditional |\ifchilddoc| distinguishes between the compilation of
child documents and the main document:
%
\begin{center}
|\ifchilddoc |\textit{child-code}| |[|\||else |\textit{main-code}]| \||fi|
\end{center}

%%%%%%%%%%%%%%%%%%%%%%%%%%%%%%%%%%%%%%%%
\DescribeMacro{\childdocname}
\DescribeMacro{\childdocjob}
The macro |\childdocname| contains the filename (without extension)
of the main or child file being processed.
Note that |\childdocjob| will always contain the name of the main file.

%%%%%%%%%%%%%%%%%%%%%%%%%%%%%%%%%%%%%%%%
\paragraph{Title Page.}

Conditional processing can be used to include a title or banner page
in the main document when proper precautions are taken.
Importantly, the code in the main file should ensure that the page counter
(as well as other status parameters which are stored in the |.aux| files)
takes the same value after the conditional processing.
Otherwise the page numbers may take divergent values
depending on which part is compiled.

For example, a title page could be declared by:
%
\begin{center}
\begin{tabular}{l}
|\ifchilddoc\||else|\\
|\addtocounter{page}{-1}|\\
\textit{code for title page}\\
|\newpage|\\
|\||fi|
\end{tabular}
\end{center}
%
A banner page for the child documents can be generated by:
%
\begin{center}
\begin{tabular}{l}
|\ifchilddoc|\\
|\addtocounter{page}{-1}|\\
\textit{code for banner page}\\
|\newpage|\\
|\||fi|
\end{tabular}
\end{center}
%
Here one could write a message such as:
\begin{center}
|This is the part \childdocname{} of \childdocjob{}.|
\end{center}

%%%%%%%%%%%%%%%%%%%%%%%%%%%%%%%%%%%%%%%%%%%%%%%%%%%%%%%%%%%%%%%%%%%%%%%%%%%%%%%%
\subsection{Flags}
\label{sec:flags}

The package makes it easy to generate different versions
of the main or child documents.
To this end compilation flags can be defined
and assigned different default values.
They will be particularly useful in conjunction
with the forwarding mechanism described in \secref{sec:forward}.

For example, it may be useful to have a flag |\version|
which can be set to |draft| or |final|.
The document source will contain some conditional code
depending on the value of |\version|.
Suppose further, the flag should default to |final| for the main file
and to |draft| for child files
which is a natural assignment for editing the document.
This is achieved by placing the following code
in the preamble of the main document
(below the |\childdocmain| directive):
%
\begin{center}
\begin{tabular}{l}
|\ifchilddoc|\\
|\providecommand{\version}{draft}|\\
|\||else|\\
|\providecommand{\version}{final}|\\
|\||fi|
\end{tabular}
\end{center}
%
The definition by |\providecommand| makes sure
that previous definitions are not overwritten.
Further statements |\providecommand{\version}{...}|
can thus be added before the above code to override it.

For the main file, one might add a line
(between |\childdocmain| and the above block)
%
\begin{center}
|%\ifchilddoc\||else\providecommand{\version}{draft}\||fi|
\end{center}
%
which can be uncommented to produce a draft version.
Likewise one can add a line to the very top of a child file
(above the |\childdocof{|\textit{main}|}| directive)
%
\begin{center}
|%\providecommand{\version}{final}|
\end{center}
%
which can be uncommented to produce the final version of this child document.

%%%%%%%%%%%%%%%%%%%%%%%%%%%%%%%%%%%%%%%%%%%%%%%%%%%%%%%%%%%%%%%%%%%%%%%%%%%%%%%%
\subsection{Forwarding}
\label{sec:forward}

Different versions of the main or child documents
using compilation flags as described in \secref{sec:flags}
can be (permanently) stored in different files
for convenient compilation, viewing and distribution.
To this end, the package defines a command
to pass on compilation to a different file:

%%%%%%%%%%%%%%%%%%%%%%%%%%%%%%%%%%%%%%%%
\DescribeMacro{\childdocforward}
The command |\childdocforward| redirects processing to
another source file:
%
\begin{center}
\begin{tabular}{l}
|\input{childdoc.def}|\\
|\childdocforward[|\textit{main}|]{|\textit{dest}|}|\\
\end{tabular}
\end{center}
%
The argument \textit{dest} is the destination file
(without extension).
It should be the main file or one of the child files.
Note that further \textsf{childdoc} directives
such as |\childdocof| and |\childdocforward|
in the indicated file will be processed in this form.
The optional argument \textit{main}
passes on directly to the main file \textit{main}
while pretending to compile the child \textit{dest}.
This form behaves as if \textit{dest}
issues |\childdocof{|\textit{main}|}| right away,
and no further \textsf{childdoc} directives will be processed.

%%%%%%%%%%%%%%%%%%%%%%%%%%%%%%%%%%%%%%%%
\DescribeMacro{\...prefix}
In the alternative form |\childdocforwardprefix|,
%
\begin{center}
\begin{tabular}{l}
|\input{childdoc.def}|\\
|\childdocforwardprefix[|\textit{main}|]{|\textit{prefix}|}{|\textit{dest}|}|
\end{tabular}
\end{center}
%
the destination file is determined by a pattern
depending on the current file:
To make this work, the current file must be called
`{\textit{prefix}\hspace{0.2em}\textit{suffix}}'
with \textit{prefix} matching precisely the argument.
Processing is then passed on to the file
`{\textit{dest}\hspace{0.2em}\textit{suffix}}'.
Surely, the same effect is achieved by
directly specifying the
argument `{\textit{dest}\hspace{0.2em}\textit{suffix}}'
in the first form.
However, that requires to set up a different file
for each child. With the alternative form of the command
all these files can have exactly the same content
which simplifies setting them up and maintaining them.

For example, the following file |draft.tex|
with a compilation flag |\version| as described in \secref{sec:flags}
compiles the main document as a draft:
%
\begin{center}
\begin{tabular}{l}
|\def\version{draft}|\\
|\input{childdoc.def}|\\
|\childdocforward{|\textit{main}|}|
\end{tabular}
\end{center}
%
Likewise, the following files |final|\textit{nn}|.tex|
compile the final version of the child document
|child|\textit{nn}|.tex|:
%
\begin{center}
\begin{tabular}{l}
|\def\version{final}|\\
|\input{childdoc.def}|\\
|\childdocforwardprefix{final}{child}|
\end{tabular}
\end{center}
%

Note that when several versions of a main file and/or of each child file
are to be generated, it may be convenient to set up a |Makefile| or
shell script to automatise the process.

%%%%%%%%%%%%%%%%%%%%%%%%%%%%%%%%%%%%%%%%%%%%%%%%%%%%%%%%%%%%%%%%%%%%%%%%%%%%%%%%
\subsection{Command Line Processing}
\label{sec:commandline}

The effect of redirection files can also be achieved by invoking
the \LaTeX{} compiler with a more elaborate command line.
Most conveniently this should be done as part
of a shell script or a |Makefile|.

When using \textsf{childdoc} in the main file, the following
command lines effectively perform a redirection
(note that depending on the shell being used,
backslashes may have to be doubled: `|\|' $\to$ `|\\|'):
%
\begin{center}
|... -jobname "|\textit{target}|" |\\|"|[\textit{flags}]%
|\input{childdoc.def}\childdocforward[|\textit{main}|]{|\textit{dest}|}"|
\end{center}
%
Here \textit{target} is the name of the output file,
\textit{main} is the name of the main file
and \textit{dest} is the name of the main or child file to be processed
(all filenames without extensions).
The optional argument \textit{main} can be omitted
if \textit{main} matches \textit{dest}.
Optionally, compilation \textit{flags} can be defined via |\def| commands.
This command line makes the \TeX{} engine believe
it is compiling the file \textit{target}
whose content is specified as the latter parameter.
The provided code then forwards the processing to
\textit{main} or \textit{dest} as described in \secref{sec:forward}.

%%%%%%%%%%%%%%%%%%%%%%%%%%%%%%%%%%%%%%%%%%%%%%%%%%%%%%%%%%%%%%%%%%%%%%%%%%%%%%%%
\subsection{Include by Input}
\label{sec:input}

Including child documents by |\include| has some restrictions by design.
Most notably, the content of a child document always occupies
its own set of pages; pages cannot be shared between child documents.
Usually, this behaviour makes perfect sense
because each child document contain an essential part of the document.
However, in some situations it may be desirable to compose
a document from a collection of parts
without having mandatory page breaks between then.
For this case, the package
provides a mechanism to include parts
by |\input| which can also be processed individually.
However, by construction this mechanism
requires manual handling of the content to be output.

%%%%%%%%%%%%%%%%%%%%%%%%%%%%%%%%%%%%%%%%
\DescribeMacro{\ifchilddocmanual}
The main file should be prepared as usual, see \secref{sec:include}.
However, the document body must make a distinction
between processing of an individual part and of the main document, e.g.:
%
\begin{center}
\begin{tabular}{l}
|\ifchilddocmanual|\\
|\input{\childdocname}|\\
|\||else|\\
\textit{document body with }|\input{|\textit{part}|}|\\
|\||fi|
\end{tabular}
\end{center}
%
The conditional |\ifchilddocmanual| is true whenever
a part to be included by |\input| is being compiled,
and the name of the part is stored in |\childdocname|.

%%%%%%%%%%%%%%%%%%%%%%%%%%%%%%%%%%%%%%%%
\DescribeMacro{\childdocby}
Each part to be included by |\input| should start with:
%
\begin{center}
\begin{tabular}{l}
|\input{childdoc.def}|\\
|\childdocby{|\textit{main}|}|\\
\end{tabular}
\end{center}
%
The directive |\childdocby| is similar to |\childdocof|
described in \secref{sec:include},
but the subsequent selection of content must be done manually.
To that end, both |\ifchilddoc| and |\ifchilddocmanual|
will be true upon processing of a part,
and the name of the part is stored in |\childdocname|.
Note that |\jobname| will be set to the filename of the current part
so that each part receives an individual |.aux| file
that does not interfere with the |.aux| file(s) of the main document.
This behaviour can be altered by the alternative form
|\childdocby[*]{|\textit{main}|}| (with a non-empty optional argument)
which uses the |.aux| file of the main document
by setting |\jobname| to \textit{main}.

%%%%%%%%%%%%%%%%%%%%%%%%%%%%%%%%%%%%%%%%%%%%%%%%%%%%%%%%%%%%%%%%%%%%%%%%%%%%%%%%
\subsection{Driver Development}
\label{sec:driver}

The \textsf{childdoc} mechanism can also be use for the development
of definition files such as \LaTeX{} styles or classes.
This case differs from the above setup with multiple parts
included by |\include| in that no |\includeonly| should be invoked.
This can be achieved by starting the include file
(before |\ProvidesPackage|) with:
%
\begin{center}
\begin{tabular}{l}
|\input{childdoc.def}|\\
|\childdocforward{|\textit{main}|}|\\
\end{tabular}
\end{center}
%
or alternatively with:
%
\begin{center}
\begin{tabular}{l}
|\input{childdoc.def}|\\
|\childdocby{|\textit{main}|}|\\
\end{tabular}
\end{center}
%
Both forms have slightly different effects as described above.
The main file is prepared as usual, see \secref{sec:include}.

%%%%%%%%%%%%%%%%%%%%%%%%%%%%%%%%%%%%%%%%%%%%%%%%%%%%%%%%%%%%%%%%%%%%%%%%%%%%%%%%
\subsection{Legacy Detection}
\label{sec:detection}

The directive |\childdocmain| in the main file can detect
whether the complete document or merely a child is to be compiled
even without using the directive |\childdocof|.
This method is deprecated because it is less robust
and there is no compelling reason to use it;
it is merely provided for backward compatibility
and it may be removed in future versions.

If the detection mechanism is to be used,
it is mandatory to correctly specify
the filename of the main file as the argument of |\childdocmain|:
%
\begin{center}
\begin{tabular}{l}
|\input{childdoc.def}|\\
|\childdocmain{|\textit{main}|}|\\
\end{tabular}
\end{center}
%
If |\jobname| does not match the argument \textit{main} of |\childdocmain|,
it is assumed that |\jobname| points to the child file to be compiled.
When using |\childdocmain| with the main file specified as argument,
it suffices to start a child file
with just |\input{|\textit{main}|}|
without loading of the package and using |\childdocof|.
If instead all processing is done
with the appropriate \textsf{childdoc} directives,
the argument of \textit{main} of |\childdocmain| can be empty.

An alternative version of the command line processing described
in \secref{sec:commandline} using the detection mechanism reads:
%
\begin{center}
|... -jobname "|\textit{target}|" "|[\textit{flags}]%
[|\def\jobname{|\textit{dest}|}|]|\input{|\textit{main}|}"|
\end{center}

%%%%%%%%%%%%%%%%%%%%%%%%%%%%%%%%%%%%%%%%%%%%%%%%%%%%%%%%%%%%%%%%%%%%%%%%%%%%%%%%
\subsection{Manual Code}
\label{sec:manual}

In case one cannot be certain whether the definitions file |childdoc.def|
is installed on the target \TeX{} distribution
and one prefers not to ship it,
it is conceivable to paste a few relevant commands into the sources.

To that end, drop all statements |\input{childdoc.def}|
and perform the replacements as outlined below.
Instead of |\childdocmain{|\textit{main}|}| add the following code
to the top of the main file:
%
\begin{center}
\begin{tabular}{l}
|\||ifdefined\childdocname\endinput\||fi\newif\ifchilddoc|\\
|\edef\childdocname{\scantokens\expandafter{\jobname\noexpand}}|\\
|\def\childdocmain{|\textit{main}|}\||ifx\childdocmain\childdocname\||else|\\
|\childdoctrue\includeonly{\childdocname}\let\jobname\childdocmain\||fi|\\
\end{tabular}
\end{center}
%
Instead of |\childdocof{|\textit{main}|}| just include the main file
at the top of each child file:
%
\begin{center}
|\input{|\textit{main}|}|
\end{center}
%
A simple redirection |\childdocforward{|\textit{dest}|}| is achieved by:
%
\begin{center}
|\def\jobname{|\textit{dest}|}\input{\jobname}|
\end{center}
%
The redirection with prefix
|\childdocforwardprefix[|\textit{prefix}|]{|\textit{dest}|}|
is accomplished by:
%
\begin{center}
\begin{tabular}{l}
|{\edef\jobname{\scantokens\expandafter{\jobname\noexpand}}|\\
|\def\redirectjob |\textit{prefix}|#1~~~{\gdef\jobname{|\textit{dest}|#1}}|\\
|\expandafter\redirectjob\jobname~~~}\input{\jobname}|
\end{tabular}
\end{center}

In an alternative approach,
child documents can be compiled by a specific command line
without additional code or specific definitions:
%
\begin{center}
|... -jobname "|\textit{target}|" "|[\textit{flags}]%
|\includeonly{|\textit{dest}|}\input{|\textit{main}|}"|
\end{center}
%

%%%%%%%%%%%%%%%%%%%%%%%%%%%%%%%%%%%%%%%%%%%%%%%%%%%%%%%%%%%%%%%%%%%%%%%%%%%%%%%%
%%%%%%%%%%%%%%%%%%%%%%%%%%%%%%%%%%%%%%%%%%%%%%%%%%%%%%%%%%%%%%%%%%%%%%%%%%%%%%%%
\section{Information}

%%%%%%%%%%%%%%%%%%%%%%%%%%%%%%%%%%%%%%%%%%%%%%%%%%%%%%%%%%%%%%%%%%%%%%%%%%%%%%%%
\subsection{Copyright}

Copyright \copyright{} 2017--2018 Niklas Beisert

This work may be distributed and/or modified under the
conditions of the \LaTeX{} Project Public License, either version 1.3
of this license or (at your option) any later version.
The latest version of this license is in
  \url{http://www.latex-project.org/lppl.txt}
and version 1.3 or later is part of all distributions of \LaTeX{}
version 2005/12/01 or later.

This work has the LPPL maintenance status `maintained'.

The Current Maintainer of this work is Niklas Beisert.

This work consists of the files |README.txt|, |childdoc.ins| and |childdoc.dtx|
as well as the derived files |childdoc.def|, |cdocsamp.tex|
with |cdocsch1.tex|, |cdocsch2.tex|, |cdocspt3.tex|, |cdocspt4.tex|,
|cdocsdrf.tex|, |cdocsfn1.tex|, |cdocsfn2.tex|
as well as |childdoc.pdf|.

%%%%%%%%%%%%%%%%%%%%%%%%%%%%%%%%%%%%%%%%%%%%%%%%%%%%%%%%%%%%%%%%%%%%%%%%%%%%%%%%
\subsection{Files and Installation}

The package consists of the files:
%
\begin{center}
\begin{tabular}{ll}
    |README.txt|   & readme file \\
    |childdoc.ins| & installation file \\
    |childdoc.dtx| & source file \\
    |childdoc.def| & definition file \\
    |cdocsamp.tex| & sample main file \\
    |cdocsch1.tex| & sample include file \\
    |cdocsch2.tex| & sample include file \\
    |cdocspt3.tex| & sample part file \\
    |cdocspt4.tex| & sample part file \\
    |cdocsdrf.tex| & sample redirection file \\
    |cdocsfn1.tex| & sample redirection file \\
    |cdocsfn2.tex| & sample redirection file \\
    |childdoc.pdf| & manual
\end{tabular}
\end{center}
%
The distribution consists of the files
|README.txt|, |childdoc.ins| and |childdoc.dtx|.
%
\begin{itemize}
\item
Run (pdf)\LaTeX{} on |childdoc.dtx|
to compile the manual |childdoc.pdf| (this file).
\item
Run \LaTeX{} on |childdoc.ins| to create the definitions file |childdoc.def|
and the sample |cdocsamp.tex| with include files
|cdocsch1.tex|, |cdocsch2.tex|, |cdocspt3.tex|, |cdocspt4.tex|,
|cdocsdrf.tex|, |cdocsfn1.tex|, |cdocsfn2.tex|.
Then copy the file |childdoc.def| to an appropriate directory of your \LaTeX{}
distribution, e.g.\ \textit{texmf-root}|/tex/latex/childdoc|.
\end{itemize}

%%%%%%%%%%%%%%%%%%%%%%%%%%%%%%%%%%%%%%%%%%%%%%%%%%%%%%%%%%%%%%%%%%%%%%%%%%%%%%%%
\subsection{Related CTAN Packages}

There are several other packages which offer a similar functionality:
%
\begin{itemize}
\item
The packages
\href{http://ctan.org/pkg/docmute}{\textsf{docmute}},
\href{http://ctan.org/pkg/includex}{\textsf{includex}} and
\href{http://ctan.org/pkg/standalone}{\textsf{standalone}}
provide commands to include only the document body of
a child file thus allowing both files to be compiled individually.
\item
The packages \href{http://ctan.org/pkg/subdocs}{\textsf{subdocs}}
and \href{http://ctan.org/pkg/subfiles}{\textsf{subfiles}}
provide structures in which the main and child documents can be
encapsulated and allowing them to be compiled individually.
The inclusion mechanism is different from the conventional |\include|.
\item
The package \href{http://ctan.org/pkg/combine}{\textsf{combine}}
is an elaborate solution to combine several documents into one.
\end{itemize}
%
See also the CTAN topic \href{http://ctan.org/topic/subdocs}{\textsf{subdocs}}
for further related packages.
The present package differs from the above solutions in that
a document structure constructed with the conventional |\include| mechanism
just needs two extra commands at the top of every file
such that all constituent files can be compiled individually.

%%%%%%%%%%%%%%%%%%%%%%%%%%%%%%%%%%%%%%%%%%%%%%%%%%%%%%%%%%%%%%%%%%%%%%%%%%%%%%%%
%\subsection{Feature Suggestions}
%
%The following is a list of features which may be useful for future
%versions of this package:
%%
%\begin{itemize}
%\item
%\ldots
%\end{itemize}

%%%%%%%%%%%%%%%%%%%%%%%%%%%%%%%%%%%%%%%%%%%%%%%%%%%%%%%%%%%%%%%%%%%%%%%%%%%%%%%%
\subsection{Revision History}

%%%%%%%%%%%%%%%%%%%%%%%%%%%%%%%%%%%%%%%%
\paragraph{v2.0:} 2018/12/30

\begin{itemize}
\item
immediate forward processing
\item
added |\childdocby| mechanism
\item
manual restructured
\end{itemize}

%%%%%%%%%%%%%%%%%%%%%%%%%%%%%%%%%%%%%%%%
\paragraph{v1.6:} 2018/01/17

\begin{itemize}
\item
application for development of include files
\item
corrections to manual
\end{itemize}

%%%%%%%%%%%%%%%%%%%%%%%%%%%%%%%%%%%%%%%%
\paragraph{v1.5:} 2017/05/21

\begin{itemize}
\item
more complete structuring introduced
\item
|\childdocof| introduced
\item
|\childdoc| renamed to |\childdocmain|
\item
|\childredirect| renamed to |\childdocforward| and |\childdocforwardprefix|
and functionality expanded
\end{itemize}

%%%%%%%%%%%%%%%%%%%%%%%%%%%%%%%%%%%%%%%%
\paragraph{v1.0:} 2017/04/27

\begin{itemize}
\item
manual and install package
\item
first version published on CTAN
\end{itemize}

%%%%%%%%%%%%%%%%%%%%%%%%%%%%%%%%%%%%%%%%
\paragraph{v0.6:} 2017/04/26

\begin{itemize}
\item
redirection mechanism added
\end{itemize}

%%%%%%%%%%%%%%%%%%%%%%%%%%%%%%%%%%%%%%%%
\paragraph{v0.5:} 2017/04/26

\begin{itemize}
\item
functionality in definition file
\end{itemize}


%%%%%%%%%%%%%%%%%%%%%%%%%%%%%%%%%%%%%%%%%%%%%%%%%%%%%%%%%%%%%%%%%%%%%%%%%%%%%%%%
%%%%%%%%%%%%%%%%%%%%%%%%%%%%%%%%%%%%%%%%%%%%%%%%%%%%%%%%%%%%%%%%%%%%%%%%%%%%%%%%
%%%%%%%%%%%%%%%%%%%%%%%%%%%%%%%%%%%%%%%%%%%%%%%%%%%%%%%%%%%%%%%%%%%%%%%%%%%%%%%%
\appendix

\settowidth\MacroIndent{\rmfamily\scriptsize 000\ }

 \DocInput{childdoc.dtx}

\end{document}
%</driver>
% \fi
%
% %%%%%%%%%%%%%%%%%%%%%%%%%%%%%%%%%%%%%%%%%%%%%%%%%%%%%%%%%%%%%%%%%%%%%%%%%%%%%%
% %%%%%%%%%%%%%%%%%%%%%%%%%%%%%%%%%%%%%%%%%%%%%%%%%%%%%%%%%%%%%%%%%%%%%%%%%%%%%%
% \section{Sample}
%\iffalse
%<*samplemain>
%\fi
%
% The following presents a sample document
% with two chapters, two parts, a title page,
% a compile flag as well as three forwarding files to set the flag.
% It consists of eight |.tex| files:
% \begin{center}
% \begin{tabular}{ll}
% |cdocsamp.tex|&main file\\
% |cdocsch1.tex|&include file for chapter 1\\
% |cdocsch2.tex|&include file for chapter 2\\
% |cdocspt3.tex|&include file for part 3\\
% |cdocspt4.tex|&include file for part 4\\
% |cdocsdrf.tex|&forwarding file for main file in draft mode\\
% |cdocsfi1.tex|&forwarding file for final version of chapter 1\\
% |cdocsfi2.tex|&forwarding file for final version of chapter 2\\
% \end{tabular}
% \end{center}
% Each of the eight files can be compiled directly by the \LaTeX{} compiler.
%
% %%%%%%%%%%%%%%%%%%%%%%%%%%%%%%%%%%%%%%
% \paragraph{Main File.}
%
% The main file is called |cdocsamp.tex|.
%
% Load the \textsf{childdoc} definitions and
% declare the filename for the main document:
%    \begin{macrocode}
\input{childdoc.def}
\childdocmain{}
%    \end{macrocode}

% Optional override for |\version| flag:
%    \begin{macrocode}
%%\ifchilddoc\else\providecommand{\version}{draft}\fi
%    \end{macrocode}

% Define the default values for the |\version| flag
% (|final| for the main file and |draft| for childs):
%    \begin{macrocode}
\ifchilddoc
\providecommand{\version}{draft}
\else
\providecommand{\version}{final}
\fi
%    \end{macrocode}

% Load the standard document class:
%    \begin{macrocode}
\documentclass[12pt]{article}
%    \end{macrocode}

% Start the document body:
%    \begin{macrocode}
\begin{document}
%    \end{macrocode}

% Declare a title page.
% Print title, part of document being processed and version flag:
%    \begin{macrocode}
\addtocounter{page}{-1}
\begin{center}
{\LARGE\bfseries{}childdoc example\par}
\vspace{1cm}
\ifchilddoc
\ifchilddocmanual part\else chapter\fi:
`\childdocname' of `\childdocjob'\par
\else
main document: `\childdocjob'\par
\fi
version: \version\par
\end{center}
\newpage
%    \end{macrocode}

% Manually include selected file,
% otherwise process as usual:
%    \begin{macrocode}
\ifchilddocmanual
\section*{part `\childdocname'}
\input{\childdocname}
\else
%    \end{macrocode}

% Include the two chapters:
%    \begin{macrocode}
\include{cdocsch1}
\include{cdocsch2}
%    \end{macrocode}

% Include the two parts unless only chapters should be displayed:
%    \begin{macrocode}
\ifchilddoc\else
\section{part three}
\input{cdocspt3}
\section{part four}
\input{cdocspt4}
\fi
%    \end{macrocode}

% Process as usual until here:
%    \begin{macrocode}
\fi
%    \end{macrocode}

% End of document body:
%    \begin{macrocode}
\end{document}
%    \end{macrocode}
%\iffalse
%</samplemain>
%\fi
%
% %%%%%%%%%%%%%%%%%%%%%%%%%%%%%%%%%%%%%%
% \paragraph{Chapter Include Files.}
%
% The include files are called |cdocsch1.tex| and |cdocsch2.tex|.
%
%\iffalse
%<*samplechap1|samplechap2>
%\fi

% Optional override for |\version| flag:
%    \begin{macrocode}
%%\providecommand{\version}{final}
%    \end{macrocode}

% Include the main document:
%    \begin{macrocode}
\input{childdoc.def}
\childdocof{cdocsamp}
%    \end{macrocode}

%\iffalse
%</samplechap1|samplechap2>
%\fi
%
%\iffalse
%<*samplechap1>
%\fi
% Some text for chapter 1:
%    \begin{macrocode}
\section{one}
some text in chapter one
%    \end{macrocode}

%\iffalse
%</samplechap1>
%\fi
% Some text for chapter 2:
%\iffalse
%<*samplechap2>
%\fi
%    \begin{macrocode}
\section{two}
more text in chapter two
%    \end{macrocode}

%\iffalse
%</samplechap2>
%\fi
%
% %%%%%%%%%%%%%%%%%%%%%%%%%%%%%%%%%%%%%%
% \paragraph{Part Include Files.}
%
% The include files are called |cdocspt3.tex| and |cdocspt4.tex|.
%
%\iffalse
%<*samplepart3|samplepart4>
%\fi

% Optional override for |\version| flag:
%    \begin{macrocode}
%%\providecommand{\version}{final}
%    \end{macrocode}

% Include the main document:
%    \begin{macrocode}
\input{childdoc.def}
\childdocby{cdocsamp}
%    \end{macrocode}

%\iffalse
%</samplepart3|samplepart4>
%\fi
%
%\iffalse
%<*samplepart3>
%\fi
% Some text for part 3:
%    \begin{macrocode}
some text in part three
%    \end{macrocode}

%\iffalse
%</samplepart3>
%\fi
% Some text for part 4:
%\iffalse
%<*samplepart4>
%\fi
%    \begin{macrocode}
more text in part four
%    \end{macrocode}

%\iffalse
%</samplepart4>
%\fi
%
% %%%%%%%%%%%%%%%%%%%%%%%%%%%%%%%%%%%%%%
% \paragraph{Forwarding for a Complete Draft.}
%
% The following forwarding file |cdocsdrf.tex|
% compiles the main document in draft mode:
%\iffalse
%<*sampledraft>
%\fi
%    \begin{macrocode}
\def\version{draft}
\input{childdoc.def}
\childdocforward{cdocsamp}
%    \end{macrocode}

%\iffalse
%</sampledraft>
%\fi
%
% %%%%%%%%%%%%%%%%%%%%%%%%%%%%%%%%%%%%%%
% \paragraph{Forwarding for Final Version of the Chapters.}
%
% The following forwarding files |cdocsfn1.tex| and |cdocsfn2.tex|
% (with identical content)
% compile the final versions of the child documents
% |cdocsch1.tex| and |cdocsch2.tex|, respectively:
%\iffalse
%<*samplefinal>
%\fi
%    \begin{macrocode}
\def\version{final}
\input{childdoc.def}
\childdocforwardprefix[cdocsamp]{cdocsfn}{cdocsch}
%    \end{macrocode}

%\iffalse
%</samplefinal>
%\fi
%
% %%%%%%%%%%%%%%%%%%%%%%%%%%%%%%%%%%%%%%
% \paragraph{Command Line Processing.}
%
% The following three command lines generate the output files
% |cdocscld|, |cdocscl1| and |cdocscl2|
% which should be identical to
% |cdocsdrf|, |cdocsch1| and |cdocsfn2|, respectively:
% \begin{center}
% \begin{tabular}{l}
% |latex -jobname cdocscld \|\\
% |  "\def\version{draft}\input{childdoc.def}\childdocforward{cdocsamp}"|\\
% |latex -jobname cdocscl1 \|\\
% |  "\input{childdoc.def}\childdocforward[cdocsamp]{cdocsch1}"|\\
% |latex -jobname cdocscl2 \|\\
% |  "\def\version{final}\input{childdoc.def}\childdocforward{cdocsch2}"|
% \end{tabular}
% \end{center}
% Note that the trailing backslash on each first line
% merely continues the input to the second line
% (for convenient cut ant paste).
% Furthermore, the command |latex| can be replaced by any
% of its alternative versions such as |pdflatex|.
%
% %%%%%%%%%%%%%%%%%%%%%%%%%%%%%%%%%%%%%%%%%%%%%%%%%%%%%%%%%%%%%%%%%%%%%%%%%%%%%%
% %%%%%%%%%%%%%%%%%%%%%%%%%%%%%%%%%%%%%%%%%%%%%%%%%%%%%%%%%%%%%%%%%%%%%%%%%%%%%%
% \section{Implementation}
%\iffalse
%<*package>
%\fi
%
% This section describes the definitions file |childdoc.def|.

% The definitions cannot be loaded using |\usepackage| or |\RequirePackage|
% which has a mechanism to prevent loading a style file more than once.
% When loading the definitions by means of |\input|
% multiple instances have to be prevented manually:
%\iffalse
%This code needs to be before the `\ProvidesFile' directive
%which is defined at the beginning of this file.
%Therefore it is also placed there and commented out here.
%</package>
%<*discard>
%\fi
%    \begin{macrocode}
\ifdefined\childdocmain\endinput\fi
%    \end{macrocode}
%\iffalse
%</discard>
%<*package>
%\fi
%
% \macro{\ifchilddoc}
% \macro{\ifchilddocmanual}
% The conditional |\ifchilddoc| tells whether a
% child (true) or main (false) document is being compiled.
% The conditional |\ifchilddocmanual| tells whether
% the |\includeonly| mechanism is used (false) or
% the selection of child files must be performed manually (true).
% The definitions initialise to false:
%    \begin{macrocode}
\newif\ifchilddoc
\newif\ifchilddocmanual
%    \end{macrocode}

% \macro{\childdocname}
% \macro{\childdocjob}
% The macro |\childdocname| stores the name of the main document
% to be compiled. The macro |\childdocjob| stores the name of
% the document on which the \LaTeX{} compiler was originally invoked.
% The content of |\jobname| cannot be compared
% to filenames specified in the source due to different catcodes.
% The following code rescans |\jobname|, stores the result
% in |\childdocname| and saves a copy in |\childdocjob|:
%    \begin{macrocode}
\edef\childdocname{\scantokens\expandafter{\jobname\noexpand}}
\let\childdocjob\childdocname
%    \end{macrocode}

% \macro{\childdocdisable}
% The macro |\childdocdisable| prevents the main file
% from being processed more than once.
% At this stage, the main document command |\childdocmain|
% is assumed to be called once again where it should do nothing.
% Any subsequent call to it should prevent
% a secondary processing of the main document
% It overwrites the forwarding commands
% |\childdocof| and |\childdocforward|
% with empty macros to prevent further inclusions of the main document:
%    \begin{macrocode}
\newcommand{\childdocdisable}
{
  \renewcommand{\childdocmain}[1]{\renewcommand{\childdocmain}[1]{\endinput}}
  \renewcommand{\childdocof}[1]{}
  \renewcommand{\childdocby}[2][]{}
  \renewcommand{\childdocforward}[2][]{}
  \renewcommand{\childdocdisable}{}
}
%    \end{macrocode}

% \macro{\childdocmain}
% The macro |\childdocmain| is to be called at the top of the main file
% with nothing or the main filename (without extension) as argument.
% First, it breaks loops.
% If the argument is not empty and does not match |\childdocname|
% (which is set by the first inclusion of |childdoc.def|),
% |\ifchilddoc| is set to true, |\includeonly| is applied to the child file
% and |\jobname| is set to the main file
% (for proper handling of |.aux| files):
%    \begin{macrocode}
\newcommand{\childdocmain}[1]
{
  \childdocdisable\childdocmain{}
  \if?#1?\else
    \begingroup
      \def\childdoctmp{#1}
      \ifx\childdoctmp\childdocname
        \def\childdoctmp{}
      \else
        \def\childdoctmp
        {
          \childdoctrue
          \includeonly{\childdocname}
          \def\childdocjob{#1}
          \def\jobname{#1}
        }
      \fi
      \expandafter
    \endgroup
    \childdoctmp
  \fi
}
%    \end{macrocode}

% \macro{\childdocof}
% The command |\childdocof| redirects
% compilation to the main file |#1|.
%    \begin{macrocode}
\newcommand{\childdocof}[1]
{
  \childdocdisable
  \childdoctrue
  \includeonly{\childdocname}
  \def\jobname{#1}
  \def\childdocjob{#1}
  \input{#1}
}
%    \end{macrocode}

% \macro{\childdocby}
% The command |\childdocby| ....
%    \begin{macrocode}
\newcommand{\childdocby}[2][]
{
  \childdocdisable
  \childdoctrue
  \childdocmanualtrue
  \if?#1?\else
    \def\jobname{#2}
  \fi
  \def\childdocjob{#2}
  \input{#2}
  \endinput
}
%    \end{macrocode}

% \macro{\childdocforward}
% The command |\childdocforward| redirects
% compilation to the main file or
% (if the optional argument is given) a child file.
% Parameters are set as if the main file
% or a child file starting with |\childdocof| was compiled.
% Then compilation is handed over to the main file:
%    \begin{macrocode}
\newcommand{\childdocforward}[2][]
{
  \begingroup
    \if?#1?
      \def\childdoctmp
      {
        \def\childdocname{#2}
        \def\childdocjob{#2}
        \def\jobname{#2}
        \input{#2}
        \endinput
      }
    \else
      \def\childdoctmp
      {
        \childdocdisable
        \def\childdocname{#2}
        \childdoctrue
        \includeonly{#2}
        \def\childdocjob{#1}
        \def\jobname{#1}
        \input{#1}
        \endinput
      }
    \fi
    \expandafter
  \endgroup
  \childdoctmp
}
%    \end{macrocode}

% \macro{\childdocforwardprefix}
% The command |\childdocforwardprefix| redirects
% compilation to the main or a child file by means of a pattern.
% The prefix |#1| in the current filename is replaced by |#2|
% and the suffix of the current filename is kept
% (it is assumed that the filename does not contain the substring `|~~~|'
% which is used as a delimiter).
% Compilation is handed over to the new file by |\childdocforward|:
%    \begin{macrocode}
\newcommand{\childdocforwardprefix}[3][]
{
  \begingroup
    \def\childdocextract #2##1~~~{\def\childdoctmp{\childdocforward[#1]{#3##1}}}
    \expandafter\childdocextract\childdocname~~~
    \expandafter
  \endgroup
  \childdoctmp
}
%    \end{macrocode}

% \macro{\childdoc}
% The deprecated macro |\childdoc| is a legacy version of |\childdocmain|:
%    \begin{macrocode}
\newcommand{\childdoc}{\childdocmain}
%    \end{macrocode}

% \macro{\childdocredirect}
% The deprecated macro |\childdocredirect| is a legacy version
% of |\childdocforward| and |\childdocforwardprefix|:
%    \begin{macrocode}
\newcommand{\childdocredirect}[2][]
{
  \begingroup
    \if?#1?
      \def\childdoctmp{\childdocforward{#2}}
    \else
      \def\childdoctmp{\childdocforwardprefix{#1}{#2}}
    \fi
    \expandafter
  \endgroup
  \childdoctmp
}
%    \end{macrocode}

%\iffalse
%</package>
%\fi
%
\endinput
|\\
|\childdocforwardprefix[|\textit{main}|]{|\textit{prefix}|}{|\textit{dest}|}|
\end{tabular}
\end{center}
%
the destination file is determined by a pattern
depending on the current file:
To make this work, the current file must be called
`{\textit{prefix}\hspace{0.2em}\textit{suffix}}'
with \textit{prefix} matching precisely the argument.
Processing is then passed on to the file
`{\textit{dest}\hspace{0.2em}\textit{suffix}}'.
Surely, the same effect is achieved by
directly specifying the
argument `{\textit{dest}\hspace{0.2em}\textit{suffix}}'
in the first form.
However, that requires to set up a different file
for each child. With the alternative form of the command
all these files can have exactly the same content
which simplifies setting them up and maintaining them.

For example, the following file |draft.tex|
with a compilation flag |\version| as described in \secref{sec:flags}
compiles the main document as a draft:
%
\begin{center}
\begin{tabular}{l}
|\def\version{draft}|\\
|% \iffalse
%
% childdoc.dtx Copyright (C) 2017-2018 Niklas Beisert
%
% This work may be distributed and/or modified under the
% conditions of the LaTeX Project Public License, either version 1.3
% of this license or (at your option) any later version.
% The latest version of this license is in
%   http://www.latex-project.org/lppl.txt
% and version 1.3 or later is part of all distributions of LaTeX
% version 2005/12/01 or later.
%
% This work has the LPPL maintenance status `maintained'.
%
% The Current Maintainer of this work is Niklas Beisert.
%
% This work consists of the files childdoc.dtx and childdoc.ins
% and the derived files childdoc.def and cdocsamp.tex with
% cdocsch1.tex, cdocsch2.tex, cdocsdrf.tex, cdocsfn1.tex, cdocsfn2.tex.
%
%<package>\ifdefined\childdocmain\endinput\fi
%<package>\ProvidesFile{childdoc.def}[2018/12/30 v2.0 child document driver]
%<samplemain>\ProvidesFile{cdocsamp.tex}[2018/12/30 v2.0 sample for childdoc]
%<*driver>
%\ProvidesFile{childdoc.drv}[2018/12/30 v2.0 childdoc reference manual file]
\PassOptionsToClass{10pt,a4paper}{article}
\documentclass{ltxdoc}

\usepackage[margin=35mm]{geometry}
\usepackage{hyperref}
\usepackage{hyperxmp}
\usepackage[usenames]{color}

\hypersetup{colorlinks=true}
\hypersetup{pdfstartview=FitH}
\hypersetup{pdfpagemode=UseNone}
\hypersetup{pdfsource={}}
\hypersetup{pdflang={en-UK}}
\hypersetup{pdfcopyright={Copyright 2017-2018 Niklas Beisert.
  This work may be distributed and/or modified under the
  conditions of the LaTeX Project Public License, either version 1.3
  of this license or (at your option) any later version.}}
\hypersetup{pdflicenseurl={http://www.latex-project.org/lppl.txt}}
\hypersetup{pdfcontactaddress={ETH Zurich, ITP, HIT K,
  Wolfgang-Pauli-Strasse 27}}
\hypersetup{pdfcontactpostcode={8093}}
\hypersetup{pdfcontactcity={Zurich}}
\hypersetup{pdfcontactcountry={Switzerland}}
\hypersetup{pdfcontactemail={nbeisert@itp.phys.ethz.ch}}
\hypersetup{pdfcontacturl={http://people.phys.ethz.ch/\xmptilde nbeisert/}}

\newcommand{\secref}[1]{\hyperref[#1]{section \ref*{#1}}}

\parskip1ex
\parindent0pt
\let\olditemize\itemize
\def\itemize{\olditemize\parskip0pt}

\begin{document}

\title{The \textsf{childdoc} Package}
\hypersetup{pdftitle={The childdoc Package}}
\author{Niklas Beisert\\[2ex]
  Institut f\"ur Theoretische Physik\\
  Eidgen\"ossische Technische Hochschule Z\"urich\\
  Wolfgang-Pauli-Strasse 27, 8093 Z\"urich, Switzerland\\[1ex]
  \href{mailto:nbeisert@itp.phys.ethz.ch}
  {\texttt{nbeisert@itp.phys.ethz.ch}}}
\hypersetup{pdfauthor={Niklas Beisert}}
\hypersetup{pdfsubject={Manual for the LaTeX2e Package childdoc}}
\date{30 December 2018, \textsf{v2.0}}
\maketitle

\begin{abstract}\noindent
\textsf{childdoc} is a \LaTeXe{} package
that enables the direct compilation
of document sections included by |\include|
to individual files.
\end{abstract}

\begingroup
\parskip0ex
\tableofcontents
\endgroup

%%%%%%%%%%%%%%%%%%%%%%%%%%%%%%%%%%%%%%%%%%%%%%%%%%%%%%%%%%%%%%%%%%%%%%%%%%%%%%%%
%%%%%%%%%%%%%%%%%%%%%%%%%%%%%%%%%%%%%%%%%%%%%%%%%%%%%%%%%%%%%%%%%%%%%%%%%%%%%%%%
\section{Introduction}

\LaTeX{} provides a mechanism to structure a large document (such as a book)
into a main file and several child files (containing the chapters)
using the |\include| command.
This mechanism is beneficial for documents
which span hundreds of pages in order to
make the source file(s) more manageable.
Moreover, compilation can be restricted to
selected child files by means of the |\includeonly| command.
The latter feature can be used to reduce the compilation time while editing
(this was significantly more useful in the earlier days of \LaTeX{})
or to generate a smaller document which is easier to navigate.
Another application of |\includeonly| is to generate
documents consisting of selected parts of the complete document.

However, there are a few drawbacks of the plain |\include| mechanism:
\begin{itemize}
\item
The child files cannot be compiled on their own,
they can only be compiled via the main file.
A naive editing environment
(such as a text editor with an option
to have the current file processed by \LaTeX)
may require one to switch to the main file before compiling;
attempting to compile the child file produces errors.
\item
The main file must be modified (each time)
to adjust the |\includeonly| command
to the present needs. This easily leaves the main file in a messy state.
\item
The generated document will always carry the filename
of the main document. This is inconvenient if
several child files are to be compiled and
to be kept for distribution.
\end{itemize}

The present package provides a simple interface
to make child files individually compilable by \LaTeX{}.
Compiling a child file then has the same effect as compiling
the main file with an |\includeonly| command
to select the appropriate child.
Moreover the generated document will carry the name of the child
rather than the main file.
This resolves all three above issues.

This feature is meant to make the editing of books,
thesis documents and lecture notes somewhat more convenient.
However, the package can also be used efficiently for
composing a series of documents (such as exercise sheets)
which are typically distributed individually.
It then assists the author in generating the individual documents
(potentially in different versions)
as well as a document containing the collected series.
Another application is in developing style files
or other kinds of included material
where compilation of the style file could redirect
to a sample or test file.

%%%%%%%%%%%%%%%%%%%%%%%%%%%%%%%%%%%%%%%%%%%%%%%%%%%%%%%%%%%%%%%%%%%%%%%%%%%%%%%%
%%%%%%%%%%%%%%%%%%%%%%%%%%%%%%%%%%%%%%%%%%%%%%%%%%%%%%%%%%%%%%%%%%%%%%%%%%%%%%%%
\section{Usage}

First of all, the package \textsf{childdoc} is \emph{not} a standard
\LaTeXe{} |.sty| style file! Therefore it needs to be invoked in
a non-standard way.

%%%%%%%%%%%%%%%%%%%%%%%%%%%%%%%%%%%%%%%%%%%%%%%%%%%%%%%%%%%%%%%%%%%%%%%%%%%%%%%%
\subsection{Included Files}
\label{sec:include}

%%%%%%%%%%%%%%%%%%%%%%%%%%%%%%%%%%%%%%%%
\DescribeMacro{\childdocmain}
To use the package, add the commands
\begin{center}
\begin{tabular}{l}
|\input{childdoc.def}|\\
|\childdocmain{}|\\
\end{tabular}
\end{center}
at the very top of the main \LaTeX{} file,
in particular \emph{before} the |\documentclass| statement!
The argument of |\childdocmain| should be left empty
(but it must be present).

%%%%%%%%%%%%%%%%%%%%%%%%%%%%%%%%%%%%%%%%
\DescribeMacro{\childdocof}
Furthermore, add the commands
\begin{center}
\begin{tabular}{l}
|\input{childdoc.def}|\\
|\childdocof{|\textit{main}|}|\\
\end{tabular}
\end{center}
at the top of every child file \textit{child}
which is included by |\include{|\textit{child}|}|
from within the main file
(or at least for those files to be compiled individually).
The argument \textit{main} must be the filename of the main file.

There are a couple of
considerations in setting up the main and child documents:

%%%%%%%%%%%%%%%%%%%%%%%%%%%%%%%%%%%%%%%%
\paragraph{Restrictions.}

Please note the following restrictions:
\begin{itemize}
\item
|\childdocmain| must be called with one argument \textit{main}
to ensure compatibility with earlier version of the package.
It must either be empty (|\childdocmain{}|)
or precisely match the filename of the main file in which it is specified.
See \secref{sec:detection} for further information.
\item
The filename \textit{main} must be specified without the |.tex| extension.
\item
The filename \textit{main} is case sensitive
(even in case-insensitive file systems)
due to internal string comparison.
\item
The argument \textit{main} should be fully expanded, it cannot be a macro.
\item
Subdirectories and special characters should be avoided in filenames.
\item
The command |\childdocmain{|\textit{main}|}| must be followed by a whitespace.
It should not be followed immediately by another command
or by a comment mark `|%|'.
This is because the \TeX{} parser reads the token immediately following
the argument of |\childdocmain| and puts it
at the beginning of every child section;
however, a white\-space is ignored.
\end{itemize}

%%%%%%%%%%%%%%%%%%%%%%%%%%%%%%%%%%%%%%%%
\paragraph{Content of Main File.}

It is advisable to place all content in the child files included by |\include|.
Any output contained in the main file will appear in all child documents
unless suppressed manually;
it cannot be suppressed automatically by the |\includeonly| directive
and thus should normally be avoided.
A method to include some content in the main file
by means of conditional processing is described in \secref{sec:conditional}.

%%%%%%%%%%%%%%%%%%%%%%%%%%%%%%%%%%%%%%%%
\paragraph{Page Numbering.}

When only a part of the document is compiled,
the appropriate numbering of pages
(as well as other status parameters)
is determined from the |.aux| files.
The latter contain information from previous passes.
However this information needs to propagate through
all intermediate child documents.
Therefore the page numbering in child documents may well
be inconsistent until the complete document is compiled at least once.

A useful (if unconventional) way to always ensure a consistent
page numbering is to restart the numbering in each child document
and denote the pages by `\textit{child}|.|\textit{page}'
where \textit{child} represents the chapter/section number of the child file.
This can be achieved by the command
|\numberwithin{page}{|\textit{child}|}|
of the \textsf{amsmath} package
where \textit{child} can be |chapter| or |section|
depending on the chosen structuring.
Alternatively, one can modify the macro |\thepage| appropriately
and reset the counter |page| at the start of each child file.

%%%%%%%%%%%%%%%%%%%%%%%%%%%%%%%%%%%%%%%%%%%%%%%%%%%%%%%%%%%%%%%%%%%%%%%%%%%%%%%%
\subsection{Conditional Processing}
\label{sec:conditional}

The package provides a mechanism to compile different versions
of a document. To customise the versions further some conditional processing
can come in handy to distinguish which version is being compiled.
The package provides two macros to describe the compilation context:

%%%%%%%%%%%%%%%%%%%%%%%%%%%%%%%%%%%%%%%%
\DescribeMacro{\ifchilddoc}
The conditional |\ifchilddoc| distinguishes between the compilation of
child documents and the main document:
%
\begin{center}
|\ifchilddoc |\textit{child-code}| |[|\||else |\textit{main-code}]| \||fi|
\end{center}

%%%%%%%%%%%%%%%%%%%%%%%%%%%%%%%%%%%%%%%%
\DescribeMacro{\childdocname}
\DescribeMacro{\childdocjob}
The macro |\childdocname| contains the filename (without extension)
of the main or child file being processed.
Note that |\childdocjob| will always contain the name of the main file.

%%%%%%%%%%%%%%%%%%%%%%%%%%%%%%%%%%%%%%%%
\paragraph{Title Page.}

Conditional processing can be used to include a title or banner page
in the main document when proper precautions are taken.
Importantly, the code in the main file should ensure that the page counter
(as well as other status parameters which are stored in the |.aux| files)
takes the same value after the conditional processing.
Otherwise the page numbers may take divergent values
depending on which part is compiled.

For example, a title page could be declared by:
%
\begin{center}
\begin{tabular}{l}
|\ifchilddoc\||else|\\
|\addtocounter{page}{-1}|\\
\textit{code for title page}\\
|\newpage|\\
|\||fi|
\end{tabular}
\end{center}
%
A banner page for the child documents can be generated by:
%
\begin{center}
\begin{tabular}{l}
|\ifchilddoc|\\
|\addtocounter{page}{-1}|\\
\textit{code for banner page}\\
|\newpage|\\
|\||fi|
\end{tabular}
\end{center}
%
Here one could write a message such as:
\begin{center}
|This is the part \childdocname{} of \childdocjob{}.|
\end{center}

%%%%%%%%%%%%%%%%%%%%%%%%%%%%%%%%%%%%%%%%%%%%%%%%%%%%%%%%%%%%%%%%%%%%%%%%%%%%%%%%
\subsection{Flags}
\label{sec:flags}

The package makes it easy to generate different versions
of the main or child documents.
To this end compilation flags can be defined
and assigned different default values.
They will be particularly useful in conjunction
with the forwarding mechanism described in \secref{sec:forward}.

For example, it may be useful to have a flag |\version|
which can be set to |draft| or |final|.
The document source will contain some conditional code
depending on the value of |\version|.
Suppose further, the flag should default to |final| for the main file
and to |draft| for child files
which is a natural assignment for editing the document.
This is achieved by placing the following code
in the preamble of the main document
(below the |\childdocmain| directive):
%
\begin{center}
\begin{tabular}{l}
|\ifchilddoc|\\
|\providecommand{\version}{draft}|\\
|\||else|\\
|\providecommand{\version}{final}|\\
|\||fi|
\end{tabular}
\end{center}
%
The definition by |\providecommand| makes sure
that previous definitions are not overwritten.
Further statements |\providecommand{\version}{...}|
can thus be added before the above code to override it.

For the main file, one might add a line
(between |\childdocmain| and the above block)
%
\begin{center}
|%\ifchilddoc\||else\providecommand{\version}{draft}\||fi|
\end{center}
%
which can be uncommented to produce a draft version.
Likewise one can add a line to the very top of a child file
(above the |\childdocof{|\textit{main}|}| directive)
%
\begin{center}
|%\providecommand{\version}{final}|
\end{center}
%
which can be uncommented to produce the final version of this child document.

%%%%%%%%%%%%%%%%%%%%%%%%%%%%%%%%%%%%%%%%%%%%%%%%%%%%%%%%%%%%%%%%%%%%%%%%%%%%%%%%
\subsection{Forwarding}
\label{sec:forward}

Different versions of the main or child documents
using compilation flags as described in \secref{sec:flags}
can be (permanently) stored in different files
for convenient compilation, viewing and distribution.
To this end, the package defines a command
to pass on compilation to a different file:

%%%%%%%%%%%%%%%%%%%%%%%%%%%%%%%%%%%%%%%%
\DescribeMacro{\childdocforward}
The command |\childdocforward| redirects processing to
another source file:
%
\begin{center}
\begin{tabular}{l}
|\input{childdoc.def}|\\
|\childdocforward[|\textit{main}|]{|\textit{dest}|}|\\
\end{tabular}
\end{center}
%
The argument \textit{dest} is the destination file
(without extension).
It should be the main file or one of the child files.
Note that further \textsf{childdoc} directives
such as |\childdocof| and |\childdocforward|
in the indicated file will be processed in this form.
The optional argument \textit{main}
passes on directly to the main file \textit{main}
while pretending to compile the child \textit{dest}.
This form behaves as if \textit{dest}
issues |\childdocof{|\textit{main}|}| right away,
and no further \textsf{childdoc} directives will be processed.

%%%%%%%%%%%%%%%%%%%%%%%%%%%%%%%%%%%%%%%%
\DescribeMacro{\...prefix}
In the alternative form |\childdocforwardprefix|,
%
\begin{center}
\begin{tabular}{l}
|\input{childdoc.def}|\\
|\childdocforwardprefix[|\textit{main}|]{|\textit{prefix}|}{|\textit{dest}|}|
\end{tabular}
\end{center}
%
the destination file is determined by a pattern
depending on the current file:
To make this work, the current file must be called
`{\textit{prefix}\hspace{0.2em}\textit{suffix}}'
with \textit{prefix} matching precisely the argument.
Processing is then passed on to the file
`{\textit{dest}\hspace{0.2em}\textit{suffix}}'.
Surely, the same effect is achieved by
directly specifying the
argument `{\textit{dest}\hspace{0.2em}\textit{suffix}}'
in the first form.
However, that requires to set up a different file
for each child. With the alternative form of the command
all these files can have exactly the same content
which simplifies setting them up and maintaining them.

For example, the following file |draft.tex|
with a compilation flag |\version| as described in \secref{sec:flags}
compiles the main document as a draft:
%
\begin{center}
\begin{tabular}{l}
|\def\version{draft}|\\
|\input{childdoc.def}|\\
|\childdocforward{|\textit{main}|}|
\end{tabular}
\end{center}
%
Likewise, the following files |final|\textit{nn}|.tex|
compile the final version of the child document
|child|\textit{nn}|.tex|:
%
\begin{center}
\begin{tabular}{l}
|\def\version{final}|\\
|\input{childdoc.def}|\\
|\childdocforwardprefix{final}{child}|
\end{tabular}
\end{center}
%

Note that when several versions of a main file and/or of each child file
are to be generated, it may be convenient to set up a |Makefile| or
shell script to automatise the process.

%%%%%%%%%%%%%%%%%%%%%%%%%%%%%%%%%%%%%%%%%%%%%%%%%%%%%%%%%%%%%%%%%%%%%%%%%%%%%%%%
\subsection{Command Line Processing}
\label{sec:commandline}

The effect of redirection files can also be achieved by invoking
the \LaTeX{} compiler with a more elaborate command line.
Most conveniently this should be done as part
of a shell script or a |Makefile|.

When using \textsf{childdoc} in the main file, the following
command lines effectively perform a redirection
(note that depending on the shell being used,
backslashes may have to be doubled: `|\|' $\to$ `|\\|'):
%
\begin{center}
|... -jobname "|\textit{target}|" |\\|"|[\textit{flags}]%
|\input{childdoc.def}\childdocforward[|\textit{main}|]{|\textit{dest}|}"|
\end{center}
%
Here \textit{target} is the name of the output file,
\textit{main} is the name of the main file
and \textit{dest} is the name of the main or child file to be processed
(all filenames without extensions).
The optional argument \textit{main} can be omitted
if \textit{main} matches \textit{dest}.
Optionally, compilation \textit{flags} can be defined via |\def| commands.
This command line makes the \TeX{} engine believe
it is compiling the file \textit{target}
whose content is specified as the latter parameter.
The provided code then forwards the processing to
\textit{main} or \textit{dest} as described in \secref{sec:forward}.

%%%%%%%%%%%%%%%%%%%%%%%%%%%%%%%%%%%%%%%%%%%%%%%%%%%%%%%%%%%%%%%%%%%%%%%%%%%%%%%%
\subsection{Include by Input}
\label{sec:input}

Including child documents by |\include| has some restrictions by design.
Most notably, the content of a child document always occupies
its own set of pages; pages cannot be shared between child documents.
Usually, this behaviour makes perfect sense
because each child document contain an essential part of the document.
However, in some situations it may be desirable to compose
a document from a collection of parts
without having mandatory page breaks between then.
For this case, the package
provides a mechanism to include parts
by |\input| which can also be processed individually.
However, by construction this mechanism
requires manual handling of the content to be output.

%%%%%%%%%%%%%%%%%%%%%%%%%%%%%%%%%%%%%%%%
\DescribeMacro{\ifchilddocmanual}
The main file should be prepared as usual, see \secref{sec:include}.
However, the document body must make a distinction
between processing of an individual part and of the main document, e.g.:
%
\begin{center}
\begin{tabular}{l}
|\ifchilddocmanual|\\
|\input{\childdocname}|\\
|\||else|\\
\textit{document body with }|\input{|\textit{part}|}|\\
|\||fi|
\end{tabular}
\end{center}
%
The conditional |\ifchilddocmanual| is true whenever
a part to be included by |\input| is being compiled,
and the name of the part is stored in |\childdocname|.

%%%%%%%%%%%%%%%%%%%%%%%%%%%%%%%%%%%%%%%%
\DescribeMacro{\childdocby}
Each part to be included by |\input| should start with:
%
\begin{center}
\begin{tabular}{l}
|\input{childdoc.def}|\\
|\childdocby{|\textit{main}|}|\\
\end{tabular}
\end{center}
%
The directive |\childdocby| is similar to |\childdocof|
described in \secref{sec:include},
but the subsequent selection of content must be done manually.
To that end, both |\ifchilddoc| and |\ifchilddocmanual|
will be true upon processing of a part,
and the name of the part is stored in |\childdocname|.
Note that |\jobname| will be set to the filename of the current part
so that each part receives an individual |.aux| file
that does not interfere with the |.aux| file(s) of the main document.
This behaviour can be altered by the alternative form
|\childdocby[*]{|\textit{main}|}| (with a non-empty optional argument)
which uses the |.aux| file of the main document
by setting |\jobname| to \textit{main}.

%%%%%%%%%%%%%%%%%%%%%%%%%%%%%%%%%%%%%%%%%%%%%%%%%%%%%%%%%%%%%%%%%%%%%%%%%%%%%%%%
\subsection{Driver Development}
\label{sec:driver}

The \textsf{childdoc} mechanism can also be use for the development
of definition files such as \LaTeX{} styles or classes.
This case differs from the above setup with multiple parts
included by |\include| in that no |\includeonly| should be invoked.
This can be achieved by starting the include file
(before |\ProvidesPackage|) with:
%
\begin{center}
\begin{tabular}{l}
|\input{childdoc.def}|\\
|\childdocforward{|\textit{main}|}|\\
\end{tabular}
\end{center}
%
or alternatively with:
%
\begin{center}
\begin{tabular}{l}
|\input{childdoc.def}|\\
|\childdocby{|\textit{main}|}|\\
\end{tabular}
\end{center}
%
Both forms have slightly different effects as described above.
The main file is prepared as usual, see \secref{sec:include}.

%%%%%%%%%%%%%%%%%%%%%%%%%%%%%%%%%%%%%%%%%%%%%%%%%%%%%%%%%%%%%%%%%%%%%%%%%%%%%%%%
\subsection{Legacy Detection}
\label{sec:detection}

The directive |\childdocmain| in the main file can detect
whether the complete document or merely a child is to be compiled
even without using the directive |\childdocof|.
This method is deprecated because it is less robust
and there is no compelling reason to use it;
it is merely provided for backward compatibility
and it may be removed in future versions.

If the detection mechanism is to be used,
it is mandatory to correctly specify
the filename of the main file as the argument of |\childdocmain|:
%
\begin{center}
\begin{tabular}{l}
|\input{childdoc.def}|\\
|\childdocmain{|\textit{main}|}|\\
\end{tabular}
\end{center}
%
If |\jobname| does not match the argument \textit{main} of |\childdocmain|,
it is assumed that |\jobname| points to the child file to be compiled.
When using |\childdocmain| with the main file specified as argument,
it suffices to start a child file
with just |\input{|\textit{main}|}|
without loading of the package and using |\childdocof|.
If instead all processing is done
with the appropriate \textsf{childdoc} directives,
the argument of \textit{main} of |\childdocmain| can be empty.

An alternative version of the command line processing described
in \secref{sec:commandline} using the detection mechanism reads:
%
\begin{center}
|... -jobname "|\textit{target}|" "|[\textit{flags}]%
[|\def\jobname{|\textit{dest}|}|]|\input{|\textit{main}|}"|
\end{center}

%%%%%%%%%%%%%%%%%%%%%%%%%%%%%%%%%%%%%%%%%%%%%%%%%%%%%%%%%%%%%%%%%%%%%%%%%%%%%%%%
\subsection{Manual Code}
\label{sec:manual}

In case one cannot be certain whether the definitions file |childdoc.def|
is installed on the target \TeX{} distribution
and one prefers not to ship it,
it is conceivable to paste a few relevant commands into the sources.

To that end, drop all statements |\input{childdoc.def}|
and perform the replacements as outlined below.
Instead of |\childdocmain{|\textit{main}|}| add the following code
to the top of the main file:
%
\begin{center}
\begin{tabular}{l}
|\||ifdefined\childdocname\endinput\||fi\newif\ifchilddoc|\\
|\edef\childdocname{\scantokens\expandafter{\jobname\noexpand}}|\\
|\def\childdocmain{|\textit{main}|}\||ifx\childdocmain\childdocname\||else|\\
|\childdoctrue\includeonly{\childdocname}\let\jobname\childdocmain\||fi|\\
\end{tabular}
\end{center}
%
Instead of |\childdocof{|\textit{main}|}| just include the main file
at the top of each child file:
%
\begin{center}
|\input{|\textit{main}|}|
\end{center}
%
A simple redirection |\childdocforward{|\textit{dest}|}| is achieved by:
%
\begin{center}
|\def\jobname{|\textit{dest}|}\input{\jobname}|
\end{center}
%
The redirection with prefix
|\childdocforwardprefix[|\textit{prefix}|]{|\textit{dest}|}|
is accomplished by:
%
\begin{center}
\begin{tabular}{l}
|{\edef\jobname{\scantokens\expandafter{\jobname\noexpand}}|\\
|\def\redirectjob |\textit{prefix}|#1~~~{\gdef\jobname{|\textit{dest}|#1}}|\\
|\expandafter\redirectjob\jobname~~~}\input{\jobname}|
\end{tabular}
\end{center}

In an alternative approach,
child documents can be compiled by a specific command line
without additional code or specific definitions:
%
\begin{center}
|... -jobname "|\textit{target}|" "|[\textit{flags}]%
|\includeonly{|\textit{dest}|}\input{|\textit{main}|}"|
\end{center}
%

%%%%%%%%%%%%%%%%%%%%%%%%%%%%%%%%%%%%%%%%%%%%%%%%%%%%%%%%%%%%%%%%%%%%%%%%%%%%%%%%
%%%%%%%%%%%%%%%%%%%%%%%%%%%%%%%%%%%%%%%%%%%%%%%%%%%%%%%%%%%%%%%%%%%%%%%%%%%%%%%%
\section{Information}

%%%%%%%%%%%%%%%%%%%%%%%%%%%%%%%%%%%%%%%%%%%%%%%%%%%%%%%%%%%%%%%%%%%%%%%%%%%%%%%%
\subsection{Copyright}

Copyright \copyright{} 2017--2018 Niklas Beisert

This work may be distributed and/or modified under the
conditions of the \LaTeX{} Project Public License, either version 1.3
of this license or (at your option) any later version.
The latest version of this license is in
  \url{http://www.latex-project.org/lppl.txt}
and version 1.3 or later is part of all distributions of \LaTeX{}
version 2005/12/01 or later.

This work has the LPPL maintenance status `maintained'.

The Current Maintainer of this work is Niklas Beisert.

This work consists of the files |README.txt|, |childdoc.ins| and |childdoc.dtx|
as well as the derived files |childdoc.def|, |cdocsamp.tex|
with |cdocsch1.tex|, |cdocsch2.tex|, |cdocspt3.tex|, |cdocspt4.tex|,
|cdocsdrf.tex|, |cdocsfn1.tex|, |cdocsfn2.tex|
as well as |childdoc.pdf|.

%%%%%%%%%%%%%%%%%%%%%%%%%%%%%%%%%%%%%%%%%%%%%%%%%%%%%%%%%%%%%%%%%%%%%%%%%%%%%%%%
\subsection{Files and Installation}

The package consists of the files:
%
\begin{center}
\begin{tabular}{ll}
    |README.txt|   & readme file \\
    |childdoc.ins| & installation file \\
    |childdoc.dtx| & source file \\
    |childdoc.def| & definition file \\
    |cdocsamp.tex| & sample main file \\
    |cdocsch1.tex| & sample include file \\
    |cdocsch2.tex| & sample include file \\
    |cdocspt3.tex| & sample part file \\
    |cdocspt4.tex| & sample part file \\
    |cdocsdrf.tex| & sample redirection file \\
    |cdocsfn1.tex| & sample redirection file \\
    |cdocsfn2.tex| & sample redirection file \\
    |childdoc.pdf| & manual
\end{tabular}
\end{center}
%
The distribution consists of the files
|README.txt|, |childdoc.ins| and |childdoc.dtx|.
%
\begin{itemize}
\item
Run (pdf)\LaTeX{} on |childdoc.dtx|
to compile the manual |childdoc.pdf| (this file).
\item
Run \LaTeX{} on |childdoc.ins| to create the definitions file |childdoc.def|
and the sample |cdocsamp.tex| with include files
|cdocsch1.tex|, |cdocsch2.tex|, |cdocspt3.tex|, |cdocspt4.tex|,
|cdocsdrf.tex|, |cdocsfn1.tex|, |cdocsfn2.tex|.
Then copy the file |childdoc.def| to an appropriate directory of your \LaTeX{}
distribution, e.g.\ \textit{texmf-root}|/tex/latex/childdoc|.
\end{itemize}

%%%%%%%%%%%%%%%%%%%%%%%%%%%%%%%%%%%%%%%%%%%%%%%%%%%%%%%%%%%%%%%%%%%%%%%%%%%%%%%%
\subsection{Related CTAN Packages}

There are several other packages which offer a similar functionality:
%
\begin{itemize}
\item
The packages
\href{http://ctan.org/pkg/docmute}{\textsf{docmute}},
\href{http://ctan.org/pkg/includex}{\textsf{includex}} and
\href{http://ctan.org/pkg/standalone}{\textsf{standalone}}
provide commands to include only the document body of
a child file thus allowing both files to be compiled individually.
\item
The packages \href{http://ctan.org/pkg/subdocs}{\textsf{subdocs}}
and \href{http://ctan.org/pkg/subfiles}{\textsf{subfiles}}
provide structures in which the main and child documents can be
encapsulated and allowing them to be compiled individually.
The inclusion mechanism is different from the conventional |\include|.
\item
The package \href{http://ctan.org/pkg/combine}{\textsf{combine}}
is an elaborate solution to combine several documents into one.
\end{itemize}
%
See also the CTAN topic \href{http://ctan.org/topic/subdocs}{\textsf{subdocs}}
for further related packages.
The present package differs from the above solutions in that
a document structure constructed with the conventional |\include| mechanism
just needs two extra commands at the top of every file
such that all constituent files can be compiled individually.

%%%%%%%%%%%%%%%%%%%%%%%%%%%%%%%%%%%%%%%%%%%%%%%%%%%%%%%%%%%%%%%%%%%%%%%%%%%%%%%%
%\subsection{Feature Suggestions}
%
%The following is a list of features which may be useful for future
%versions of this package:
%%
%\begin{itemize}
%\item
%\ldots
%\end{itemize}

%%%%%%%%%%%%%%%%%%%%%%%%%%%%%%%%%%%%%%%%%%%%%%%%%%%%%%%%%%%%%%%%%%%%%%%%%%%%%%%%
\subsection{Revision History}

%%%%%%%%%%%%%%%%%%%%%%%%%%%%%%%%%%%%%%%%
\paragraph{v2.0:} 2018/12/30

\begin{itemize}
\item
immediate forward processing
\item
added |\childdocby| mechanism
\item
manual restructured
\end{itemize}

%%%%%%%%%%%%%%%%%%%%%%%%%%%%%%%%%%%%%%%%
\paragraph{v1.6:} 2018/01/17

\begin{itemize}
\item
application for development of include files
\item
corrections to manual
\end{itemize}

%%%%%%%%%%%%%%%%%%%%%%%%%%%%%%%%%%%%%%%%
\paragraph{v1.5:} 2017/05/21

\begin{itemize}
\item
more complete structuring introduced
\item
|\childdocof| introduced
\item
|\childdoc| renamed to |\childdocmain|
\item
|\childredirect| renamed to |\childdocforward| and |\childdocforwardprefix|
and functionality expanded
\end{itemize}

%%%%%%%%%%%%%%%%%%%%%%%%%%%%%%%%%%%%%%%%
\paragraph{v1.0:} 2017/04/27

\begin{itemize}
\item
manual and install package
\item
first version published on CTAN
\end{itemize}

%%%%%%%%%%%%%%%%%%%%%%%%%%%%%%%%%%%%%%%%
\paragraph{v0.6:} 2017/04/26

\begin{itemize}
\item
redirection mechanism added
\end{itemize}

%%%%%%%%%%%%%%%%%%%%%%%%%%%%%%%%%%%%%%%%
\paragraph{v0.5:} 2017/04/26

\begin{itemize}
\item
functionality in definition file
\end{itemize}


%%%%%%%%%%%%%%%%%%%%%%%%%%%%%%%%%%%%%%%%%%%%%%%%%%%%%%%%%%%%%%%%%%%%%%%%%%%%%%%%
%%%%%%%%%%%%%%%%%%%%%%%%%%%%%%%%%%%%%%%%%%%%%%%%%%%%%%%%%%%%%%%%%%%%%%%%%%%%%%%%
%%%%%%%%%%%%%%%%%%%%%%%%%%%%%%%%%%%%%%%%%%%%%%%%%%%%%%%%%%%%%%%%%%%%%%%%%%%%%%%%
\appendix

\settowidth\MacroIndent{\rmfamily\scriptsize 000\ }

 \DocInput{childdoc.dtx}

\end{document}
%</driver>
% \fi
%
% %%%%%%%%%%%%%%%%%%%%%%%%%%%%%%%%%%%%%%%%%%%%%%%%%%%%%%%%%%%%%%%%%%%%%%%%%%%%%%
% %%%%%%%%%%%%%%%%%%%%%%%%%%%%%%%%%%%%%%%%%%%%%%%%%%%%%%%%%%%%%%%%%%%%%%%%%%%%%%
% \section{Sample}
%\iffalse
%<*samplemain>
%\fi
%
% The following presents a sample document
% with two chapters, two parts, a title page,
% a compile flag as well as three forwarding files to set the flag.
% It consists of eight |.tex| files:
% \begin{center}
% \begin{tabular}{ll}
% |cdocsamp.tex|&main file\\
% |cdocsch1.tex|&include file for chapter 1\\
% |cdocsch2.tex|&include file for chapter 2\\
% |cdocspt3.tex|&include file for part 3\\
% |cdocspt4.tex|&include file for part 4\\
% |cdocsdrf.tex|&forwarding file for main file in draft mode\\
% |cdocsfi1.tex|&forwarding file for final version of chapter 1\\
% |cdocsfi2.tex|&forwarding file for final version of chapter 2\\
% \end{tabular}
% \end{center}
% Each of the eight files can be compiled directly by the \LaTeX{} compiler.
%
% %%%%%%%%%%%%%%%%%%%%%%%%%%%%%%%%%%%%%%
% \paragraph{Main File.}
%
% The main file is called |cdocsamp.tex|.
%
% Load the \textsf{childdoc} definitions and
% declare the filename for the main document:
%    \begin{macrocode}
\input{childdoc.def}
\childdocmain{}
%    \end{macrocode}

% Optional override for |\version| flag:
%    \begin{macrocode}
%%\ifchilddoc\else\providecommand{\version}{draft}\fi
%    \end{macrocode}

% Define the default values for the |\version| flag
% (|final| for the main file and |draft| for childs):
%    \begin{macrocode}
\ifchilddoc
\providecommand{\version}{draft}
\else
\providecommand{\version}{final}
\fi
%    \end{macrocode}

% Load the standard document class:
%    \begin{macrocode}
\documentclass[12pt]{article}
%    \end{macrocode}

% Start the document body:
%    \begin{macrocode}
\begin{document}
%    \end{macrocode}

% Declare a title page.
% Print title, part of document being processed and version flag:
%    \begin{macrocode}
\addtocounter{page}{-1}
\begin{center}
{\LARGE\bfseries{}childdoc example\par}
\vspace{1cm}
\ifchilddoc
\ifchilddocmanual part\else chapter\fi:
`\childdocname' of `\childdocjob'\par
\else
main document: `\childdocjob'\par
\fi
version: \version\par
\end{center}
\newpage
%    \end{macrocode}

% Manually include selected file,
% otherwise process as usual:
%    \begin{macrocode}
\ifchilddocmanual
\section*{part `\childdocname'}
\input{\childdocname}
\else
%    \end{macrocode}

% Include the two chapters:
%    \begin{macrocode}
\include{cdocsch1}
\include{cdocsch2}
%    \end{macrocode}

% Include the two parts unless only chapters should be displayed:
%    \begin{macrocode}
\ifchilddoc\else
\section{part three}
\input{cdocspt3}
\section{part four}
\input{cdocspt4}
\fi
%    \end{macrocode}

% Process as usual until here:
%    \begin{macrocode}
\fi
%    \end{macrocode}

% End of document body:
%    \begin{macrocode}
\end{document}
%    \end{macrocode}
%\iffalse
%</samplemain>
%\fi
%
% %%%%%%%%%%%%%%%%%%%%%%%%%%%%%%%%%%%%%%
% \paragraph{Chapter Include Files.}
%
% The include files are called |cdocsch1.tex| and |cdocsch2.tex|.
%
%\iffalse
%<*samplechap1|samplechap2>
%\fi

% Optional override for |\version| flag:
%    \begin{macrocode}
%%\providecommand{\version}{final}
%    \end{macrocode}

% Include the main document:
%    \begin{macrocode}
\input{childdoc.def}
\childdocof{cdocsamp}
%    \end{macrocode}

%\iffalse
%</samplechap1|samplechap2>
%\fi
%
%\iffalse
%<*samplechap1>
%\fi
% Some text for chapter 1:
%    \begin{macrocode}
\section{one}
some text in chapter one
%    \end{macrocode}

%\iffalse
%</samplechap1>
%\fi
% Some text for chapter 2:
%\iffalse
%<*samplechap2>
%\fi
%    \begin{macrocode}
\section{two}
more text in chapter two
%    \end{macrocode}

%\iffalse
%</samplechap2>
%\fi
%
% %%%%%%%%%%%%%%%%%%%%%%%%%%%%%%%%%%%%%%
% \paragraph{Part Include Files.}
%
% The include files are called |cdocspt3.tex| and |cdocspt4.tex|.
%
%\iffalse
%<*samplepart3|samplepart4>
%\fi

% Optional override for |\version| flag:
%    \begin{macrocode}
%%\providecommand{\version}{final}
%    \end{macrocode}

% Include the main document:
%    \begin{macrocode}
\input{childdoc.def}
\childdocby{cdocsamp}
%    \end{macrocode}

%\iffalse
%</samplepart3|samplepart4>
%\fi
%
%\iffalse
%<*samplepart3>
%\fi
% Some text for part 3:
%    \begin{macrocode}
some text in part three
%    \end{macrocode}

%\iffalse
%</samplepart3>
%\fi
% Some text for part 4:
%\iffalse
%<*samplepart4>
%\fi
%    \begin{macrocode}
more text in part four
%    \end{macrocode}

%\iffalse
%</samplepart4>
%\fi
%
% %%%%%%%%%%%%%%%%%%%%%%%%%%%%%%%%%%%%%%
% \paragraph{Forwarding for a Complete Draft.}
%
% The following forwarding file |cdocsdrf.tex|
% compiles the main document in draft mode:
%\iffalse
%<*sampledraft>
%\fi
%    \begin{macrocode}
\def\version{draft}
\input{childdoc.def}
\childdocforward{cdocsamp}
%    \end{macrocode}

%\iffalse
%</sampledraft>
%\fi
%
% %%%%%%%%%%%%%%%%%%%%%%%%%%%%%%%%%%%%%%
% \paragraph{Forwarding for Final Version of the Chapters.}
%
% The following forwarding files |cdocsfn1.tex| and |cdocsfn2.tex|
% (with identical content)
% compile the final versions of the child documents
% |cdocsch1.tex| and |cdocsch2.tex|, respectively:
%\iffalse
%<*samplefinal>
%\fi
%    \begin{macrocode}
\def\version{final}
\input{childdoc.def}
\childdocforwardprefix[cdocsamp]{cdocsfn}{cdocsch}
%    \end{macrocode}

%\iffalse
%</samplefinal>
%\fi
%
% %%%%%%%%%%%%%%%%%%%%%%%%%%%%%%%%%%%%%%
% \paragraph{Command Line Processing.}
%
% The following three command lines generate the output files
% |cdocscld|, |cdocscl1| and |cdocscl2|
% which should be identical to
% |cdocsdrf|, |cdocsch1| and |cdocsfn2|, respectively:
% \begin{center}
% \begin{tabular}{l}
% |latex -jobname cdocscld \|\\
% |  "\def\version{draft}\input{childdoc.def}\childdocforward{cdocsamp}"|\\
% |latex -jobname cdocscl1 \|\\
% |  "\input{childdoc.def}\childdocforward[cdocsamp]{cdocsch1}"|\\
% |latex -jobname cdocscl2 \|\\
% |  "\def\version{final}\input{childdoc.def}\childdocforward{cdocsch2}"|
% \end{tabular}
% \end{center}
% Note that the trailing backslash on each first line
% merely continues the input to the second line
% (for convenient cut ant paste).
% Furthermore, the command |latex| can be replaced by any
% of its alternative versions such as |pdflatex|.
%
% %%%%%%%%%%%%%%%%%%%%%%%%%%%%%%%%%%%%%%%%%%%%%%%%%%%%%%%%%%%%%%%%%%%%%%%%%%%%%%
% %%%%%%%%%%%%%%%%%%%%%%%%%%%%%%%%%%%%%%%%%%%%%%%%%%%%%%%%%%%%%%%%%%%%%%%%%%%%%%
% \section{Implementation}
%\iffalse
%<*package>
%\fi
%
% This section describes the definitions file |childdoc.def|.

% The definitions cannot be loaded using |\usepackage| or |\RequirePackage|
% which has a mechanism to prevent loading a style file more than once.
% When loading the definitions by means of |\input|
% multiple instances have to be prevented manually:
%\iffalse
%This code needs to be before the `\ProvidesFile' directive
%which is defined at the beginning of this file.
%Therefore it is also placed there and commented out here.
%</package>
%<*discard>
%\fi
%    \begin{macrocode}
\ifdefined\childdocmain\endinput\fi
%    \end{macrocode}
%\iffalse
%</discard>
%<*package>
%\fi
%
% \macro{\ifchilddoc}
% \macro{\ifchilddocmanual}
% The conditional |\ifchilddoc| tells whether a
% child (true) or main (false) document is being compiled.
% The conditional |\ifchilddocmanual| tells whether
% the |\includeonly| mechanism is used (false) or
% the selection of child files must be performed manually (true).
% The definitions initialise to false:
%    \begin{macrocode}
\newif\ifchilddoc
\newif\ifchilddocmanual
%    \end{macrocode}

% \macro{\childdocname}
% \macro{\childdocjob}
% The macro |\childdocname| stores the name of the main document
% to be compiled. The macro |\childdocjob| stores the name of
% the document on which the \LaTeX{} compiler was originally invoked.
% The content of |\jobname| cannot be compared
% to filenames specified in the source due to different catcodes.
% The following code rescans |\jobname|, stores the result
% in |\childdocname| and saves a copy in |\childdocjob|:
%    \begin{macrocode}
\edef\childdocname{\scantokens\expandafter{\jobname\noexpand}}
\let\childdocjob\childdocname
%    \end{macrocode}

% \macro{\childdocdisable}
% The macro |\childdocdisable| prevents the main file
% from being processed more than once.
% At this stage, the main document command |\childdocmain|
% is assumed to be called once again where it should do nothing.
% Any subsequent call to it should prevent
% a secondary processing of the main document
% It overwrites the forwarding commands
% |\childdocof| and |\childdocforward|
% with empty macros to prevent further inclusions of the main document:
%    \begin{macrocode}
\newcommand{\childdocdisable}
{
  \renewcommand{\childdocmain}[1]{\renewcommand{\childdocmain}[1]{\endinput}}
  \renewcommand{\childdocof}[1]{}
  \renewcommand{\childdocby}[2][]{}
  \renewcommand{\childdocforward}[2][]{}
  \renewcommand{\childdocdisable}{}
}
%    \end{macrocode}

% \macro{\childdocmain}
% The macro |\childdocmain| is to be called at the top of the main file
% with nothing or the main filename (without extension) as argument.
% First, it breaks loops.
% If the argument is not empty and does not match |\childdocname|
% (which is set by the first inclusion of |childdoc.def|),
% |\ifchilddoc| is set to true, |\includeonly| is applied to the child file
% and |\jobname| is set to the main file
% (for proper handling of |.aux| files):
%    \begin{macrocode}
\newcommand{\childdocmain}[1]
{
  \childdocdisable\childdocmain{}
  \if?#1?\else
    \begingroup
      \def\childdoctmp{#1}
      \ifx\childdoctmp\childdocname
        \def\childdoctmp{}
      \else
        \def\childdoctmp
        {
          \childdoctrue
          \includeonly{\childdocname}
          \def\childdocjob{#1}
          \def\jobname{#1}
        }
      \fi
      \expandafter
    \endgroup
    \childdoctmp
  \fi
}
%    \end{macrocode}

% \macro{\childdocof}
% The command |\childdocof| redirects
% compilation to the main file |#1|.
%    \begin{macrocode}
\newcommand{\childdocof}[1]
{
  \childdocdisable
  \childdoctrue
  \includeonly{\childdocname}
  \def\jobname{#1}
  \def\childdocjob{#1}
  \input{#1}
}
%    \end{macrocode}

% \macro{\childdocby}
% The command |\childdocby| ....
%    \begin{macrocode}
\newcommand{\childdocby}[2][]
{
  \childdocdisable
  \childdoctrue
  \childdocmanualtrue
  \if?#1?\else
    \def\jobname{#2}
  \fi
  \def\childdocjob{#2}
  \input{#2}
  \endinput
}
%    \end{macrocode}

% \macro{\childdocforward}
% The command |\childdocforward| redirects
% compilation to the main file or
% (if the optional argument is given) a child file.
% Parameters are set as if the main file
% or a child file starting with |\childdocof| was compiled.
% Then compilation is handed over to the main file:
%    \begin{macrocode}
\newcommand{\childdocforward}[2][]
{
  \begingroup
    \if?#1?
      \def\childdoctmp
      {
        \def\childdocname{#2}
        \def\childdocjob{#2}
        \def\jobname{#2}
        \input{#2}
        \endinput
      }
    \else
      \def\childdoctmp
      {
        \childdocdisable
        \def\childdocname{#2}
        \childdoctrue
        \includeonly{#2}
        \def\childdocjob{#1}
        \def\jobname{#1}
        \input{#1}
        \endinput
      }
    \fi
    \expandafter
  \endgroup
  \childdoctmp
}
%    \end{macrocode}

% \macro{\childdocforwardprefix}
% The command |\childdocforwardprefix| redirects
% compilation to the main or a child file by means of a pattern.
% The prefix |#1| in the current filename is replaced by |#2|
% and the suffix of the current filename is kept
% (it is assumed that the filename does not contain the substring `|~~~|'
% which is used as a delimiter).
% Compilation is handed over to the new file by |\childdocforward|:
%    \begin{macrocode}
\newcommand{\childdocforwardprefix}[3][]
{
  \begingroup
    \def\childdocextract #2##1~~~{\def\childdoctmp{\childdocforward[#1]{#3##1}}}
    \expandafter\childdocextract\childdocname~~~
    \expandafter
  \endgroup
  \childdoctmp
}
%    \end{macrocode}

% \macro{\childdoc}
% The deprecated macro |\childdoc| is a legacy version of |\childdocmain|:
%    \begin{macrocode}
\newcommand{\childdoc}{\childdocmain}
%    \end{macrocode}

% \macro{\childdocredirect}
% The deprecated macro |\childdocredirect| is a legacy version
% of |\childdocforward| and |\childdocforwardprefix|:
%    \begin{macrocode}
\newcommand{\childdocredirect}[2][]
{
  \begingroup
    \if?#1?
      \def\childdoctmp{\childdocforward{#2}}
    \else
      \def\childdoctmp{\childdocforwardprefix{#1}{#2}}
    \fi
    \expandafter
  \endgroup
  \childdoctmp
}
%    \end{macrocode}

%\iffalse
%</package>
%\fi
%
\endinput
|\\
|\childdocforward{|\textit{main}|}|
\end{tabular}
\end{center}
%
Likewise, the following files |final|\textit{nn}|.tex|
compile the final version of the child document
|child|\textit{nn}|.tex|:
%
\begin{center}
\begin{tabular}{l}
|\def\version{final}|\\
|% \iffalse
%
% childdoc.dtx Copyright (C) 2017-2018 Niklas Beisert
%
% This work may be distributed and/or modified under the
% conditions of the LaTeX Project Public License, either version 1.3
% of this license or (at your option) any later version.
% The latest version of this license is in
%   http://www.latex-project.org/lppl.txt
% and version 1.3 or later is part of all distributions of LaTeX
% version 2005/12/01 or later.
%
% This work has the LPPL maintenance status `maintained'.
%
% The Current Maintainer of this work is Niklas Beisert.
%
% This work consists of the files childdoc.dtx and childdoc.ins
% and the derived files childdoc.def and cdocsamp.tex with
% cdocsch1.tex, cdocsch2.tex, cdocsdrf.tex, cdocsfn1.tex, cdocsfn2.tex.
%
%<package>\ifdefined\childdocmain\endinput\fi
%<package>\ProvidesFile{childdoc.def}[2018/12/30 v2.0 child document driver]
%<samplemain>\ProvidesFile{cdocsamp.tex}[2018/12/30 v2.0 sample for childdoc]
%<*driver>
%\ProvidesFile{childdoc.drv}[2018/12/30 v2.0 childdoc reference manual file]
\PassOptionsToClass{10pt,a4paper}{article}
\documentclass{ltxdoc}

\usepackage[margin=35mm]{geometry}
\usepackage{hyperref}
\usepackage{hyperxmp}
\usepackage[usenames]{color}

\hypersetup{colorlinks=true}
\hypersetup{pdfstartview=FitH}
\hypersetup{pdfpagemode=UseNone}
\hypersetup{pdfsource={}}
\hypersetup{pdflang={en-UK}}
\hypersetup{pdfcopyright={Copyright 2017-2018 Niklas Beisert.
  This work may be distributed and/or modified under the
  conditions of the LaTeX Project Public License, either version 1.3
  of this license or (at your option) any later version.}}
\hypersetup{pdflicenseurl={http://www.latex-project.org/lppl.txt}}
\hypersetup{pdfcontactaddress={ETH Zurich, ITP, HIT K,
  Wolfgang-Pauli-Strasse 27}}
\hypersetup{pdfcontactpostcode={8093}}
\hypersetup{pdfcontactcity={Zurich}}
\hypersetup{pdfcontactcountry={Switzerland}}
\hypersetup{pdfcontactemail={nbeisert@itp.phys.ethz.ch}}
\hypersetup{pdfcontacturl={http://people.phys.ethz.ch/\xmptilde nbeisert/}}

\newcommand{\secref}[1]{\hyperref[#1]{section \ref*{#1}}}

\parskip1ex
\parindent0pt
\let\olditemize\itemize
\def\itemize{\olditemize\parskip0pt}

\begin{document}

\title{The \textsf{childdoc} Package}
\hypersetup{pdftitle={The childdoc Package}}
\author{Niklas Beisert\\[2ex]
  Institut f\"ur Theoretische Physik\\
  Eidgen\"ossische Technische Hochschule Z\"urich\\
  Wolfgang-Pauli-Strasse 27, 8093 Z\"urich, Switzerland\\[1ex]
  \href{mailto:nbeisert@itp.phys.ethz.ch}
  {\texttt{nbeisert@itp.phys.ethz.ch}}}
\hypersetup{pdfauthor={Niklas Beisert}}
\hypersetup{pdfsubject={Manual for the LaTeX2e Package childdoc}}
\date{30 December 2018, \textsf{v2.0}}
\maketitle

\begin{abstract}\noindent
\textsf{childdoc} is a \LaTeXe{} package
that enables the direct compilation
of document sections included by |\include|
to individual files.
\end{abstract}

\begingroup
\parskip0ex
\tableofcontents
\endgroup

%%%%%%%%%%%%%%%%%%%%%%%%%%%%%%%%%%%%%%%%%%%%%%%%%%%%%%%%%%%%%%%%%%%%%%%%%%%%%%%%
%%%%%%%%%%%%%%%%%%%%%%%%%%%%%%%%%%%%%%%%%%%%%%%%%%%%%%%%%%%%%%%%%%%%%%%%%%%%%%%%
\section{Introduction}

\LaTeX{} provides a mechanism to structure a large document (such as a book)
into a main file and several child files (containing the chapters)
using the |\include| command.
This mechanism is beneficial for documents
which span hundreds of pages in order to
make the source file(s) more manageable.
Moreover, compilation can be restricted to
selected child files by means of the |\includeonly| command.
The latter feature can be used to reduce the compilation time while editing
(this was significantly more useful in the earlier days of \LaTeX{})
or to generate a smaller document which is easier to navigate.
Another application of |\includeonly| is to generate
documents consisting of selected parts of the complete document.

However, there are a few drawbacks of the plain |\include| mechanism:
\begin{itemize}
\item
The child files cannot be compiled on their own,
they can only be compiled via the main file.
A naive editing environment
(such as a text editor with an option
to have the current file processed by \LaTeX)
may require one to switch to the main file before compiling;
attempting to compile the child file produces errors.
\item
The main file must be modified (each time)
to adjust the |\includeonly| command
to the present needs. This easily leaves the main file in a messy state.
\item
The generated document will always carry the filename
of the main document. This is inconvenient if
several child files are to be compiled and
to be kept for distribution.
\end{itemize}

The present package provides a simple interface
to make child files individually compilable by \LaTeX{}.
Compiling a child file then has the same effect as compiling
the main file with an |\includeonly| command
to select the appropriate child.
Moreover the generated document will carry the name of the child
rather than the main file.
This resolves all three above issues.

This feature is meant to make the editing of books,
thesis documents and lecture notes somewhat more convenient.
However, the package can also be used efficiently for
composing a series of documents (such as exercise sheets)
which are typically distributed individually.
It then assists the author in generating the individual documents
(potentially in different versions)
as well as a document containing the collected series.
Another application is in developing style files
or other kinds of included material
where compilation of the style file could redirect
to a sample or test file.

%%%%%%%%%%%%%%%%%%%%%%%%%%%%%%%%%%%%%%%%%%%%%%%%%%%%%%%%%%%%%%%%%%%%%%%%%%%%%%%%
%%%%%%%%%%%%%%%%%%%%%%%%%%%%%%%%%%%%%%%%%%%%%%%%%%%%%%%%%%%%%%%%%%%%%%%%%%%%%%%%
\section{Usage}

First of all, the package \textsf{childdoc} is \emph{not} a standard
\LaTeXe{} |.sty| style file! Therefore it needs to be invoked in
a non-standard way.

%%%%%%%%%%%%%%%%%%%%%%%%%%%%%%%%%%%%%%%%%%%%%%%%%%%%%%%%%%%%%%%%%%%%%%%%%%%%%%%%
\subsection{Included Files}
\label{sec:include}

%%%%%%%%%%%%%%%%%%%%%%%%%%%%%%%%%%%%%%%%
\DescribeMacro{\childdocmain}
To use the package, add the commands
\begin{center}
\begin{tabular}{l}
|\input{childdoc.def}|\\
|\childdocmain{}|\\
\end{tabular}
\end{center}
at the very top of the main \LaTeX{} file,
in particular \emph{before} the |\documentclass| statement!
The argument of |\childdocmain| should be left empty
(but it must be present).

%%%%%%%%%%%%%%%%%%%%%%%%%%%%%%%%%%%%%%%%
\DescribeMacro{\childdocof}
Furthermore, add the commands
\begin{center}
\begin{tabular}{l}
|\input{childdoc.def}|\\
|\childdocof{|\textit{main}|}|\\
\end{tabular}
\end{center}
at the top of every child file \textit{child}
which is included by |\include{|\textit{child}|}|
from within the main file
(or at least for those files to be compiled individually).
The argument \textit{main} must be the filename of the main file.

There are a couple of
considerations in setting up the main and child documents:

%%%%%%%%%%%%%%%%%%%%%%%%%%%%%%%%%%%%%%%%
\paragraph{Restrictions.}

Please note the following restrictions:
\begin{itemize}
\item
|\childdocmain| must be called with one argument \textit{main}
to ensure compatibility with earlier version of the package.
It must either be empty (|\childdocmain{}|)
or precisely match the filename of the main file in which it is specified.
See \secref{sec:detection} for further information.
\item
The filename \textit{main} must be specified without the |.tex| extension.
\item
The filename \textit{main} is case sensitive
(even in case-insensitive file systems)
due to internal string comparison.
\item
The argument \textit{main} should be fully expanded, it cannot be a macro.
\item
Subdirectories and special characters should be avoided in filenames.
\item
The command |\childdocmain{|\textit{main}|}| must be followed by a whitespace.
It should not be followed immediately by another command
or by a comment mark `|%|'.
This is because the \TeX{} parser reads the token immediately following
the argument of |\childdocmain| and puts it
at the beginning of every child section;
however, a white\-space is ignored.
\end{itemize}

%%%%%%%%%%%%%%%%%%%%%%%%%%%%%%%%%%%%%%%%
\paragraph{Content of Main File.}

It is advisable to place all content in the child files included by |\include|.
Any output contained in the main file will appear in all child documents
unless suppressed manually;
it cannot be suppressed automatically by the |\includeonly| directive
and thus should normally be avoided.
A method to include some content in the main file
by means of conditional processing is described in \secref{sec:conditional}.

%%%%%%%%%%%%%%%%%%%%%%%%%%%%%%%%%%%%%%%%
\paragraph{Page Numbering.}

When only a part of the document is compiled,
the appropriate numbering of pages
(as well as other status parameters)
is determined from the |.aux| files.
The latter contain information from previous passes.
However this information needs to propagate through
all intermediate child documents.
Therefore the page numbering in child documents may well
be inconsistent until the complete document is compiled at least once.

A useful (if unconventional) way to always ensure a consistent
page numbering is to restart the numbering in each child document
and denote the pages by `\textit{child}|.|\textit{page}'
where \textit{child} represents the chapter/section number of the child file.
This can be achieved by the command
|\numberwithin{page}{|\textit{child}|}|
of the \textsf{amsmath} package
where \textit{child} can be |chapter| or |section|
depending on the chosen structuring.
Alternatively, one can modify the macro |\thepage| appropriately
and reset the counter |page| at the start of each child file.

%%%%%%%%%%%%%%%%%%%%%%%%%%%%%%%%%%%%%%%%%%%%%%%%%%%%%%%%%%%%%%%%%%%%%%%%%%%%%%%%
\subsection{Conditional Processing}
\label{sec:conditional}

The package provides a mechanism to compile different versions
of a document. To customise the versions further some conditional processing
can come in handy to distinguish which version is being compiled.
The package provides two macros to describe the compilation context:

%%%%%%%%%%%%%%%%%%%%%%%%%%%%%%%%%%%%%%%%
\DescribeMacro{\ifchilddoc}
The conditional |\ifchilddoc| distinguishes between the compilation of
child documents and the main document:
%
\begin{center}
|\ifchilddoc |\textit{child-code}| |[|\||else |\textit{main-code}]| \||fi|
\end{center}

%%%%%%%%%%%%%%%%%%%%%%%%%%%%%%%%%%%%%%%%
\DescribeMacro{\childdocname}
\DescribeMacro{\childdocjob}
The macro |\childdocname| contains the filename (without extension)
of the main or child file being processed.
Note that |\childdocjob| will always contain the name of the main file.

%%%%%%%%%%%%%%%%%%%%%%%%%%%%%%%%%%%%%%%%
\paragraph{Title Page.}

Conditional processing can be used to include a title or banner page
in the main document when proper precautions are taken.
Importantly, the code in the main file should ensure that the page counter
(as well as other status parameters which are stored in the |.aux| files)
takes the same value after the conditional processing.
Otherwise the page numbers may take divergent values
depending on which part is compiled.

For example, a title page could be declared by:
%
\begin{center}
\begin{tabular}{l}
|\ifchilddoc\||else|\\
|\addtocounter{page}{-1}|\\
\textit{code for title page}\\
|\newpage|\\
|\||fi|
\end{tabular}
\end{center}
%
A banner page for the child documents can be generated by:
%
\begin{center}
\begin{tabular}{l}
|\ifchilddoc|\\
|\addtocounter{page}{-1}|\\
\textit{code for banner page}\\
|\newpage|\\
|\||fi|
\end{tabular}
\end{center}
%
Here one could write a message such as:
\begin{center}
|This is the part \childdocname{} of \childdocjob{}.|
\end{center}

%%%%%%%%%%%%%%%%%%%%%%%%%%%%%%%%%%%%%%%%%%%%%%%%%%%%%%%%%%%%%%%%%%%%%%%%%%%%%%%%
\subsection{Flags}
\label{sec:flags}

The package makes it easy to generate different versions
of the main or child documents.
To this end compilation flags can be defined
and assigned different default values.
They will be particularly useful in conjunction
with the forwarding mechanism described in \secref{sec:forward}.

For example, it may be useful to have a flag |\version|
which can be set to |draft| or |final|.
The document source will contain some conditional code
depending on the value of |\version|.
Suppose further, the flag should default to |final| for the main file
and to |draft| for child files
which is a natural assignment for editing the document.
This is achieved by placing the following code
in the preamble of the main document
(below the |\childdocmain| directive):
%
\begin{center}
\begin{tabular}{l}
|\ifchilddoc|\\
|\providecommand{\version}{draft}|\\
|\||else|\\
|\providecommand{\version}{final}|\\
|\||fi|
\end{tabular}
\end{center}
%
The definition by |\providecommand| makes sure
that previous definitions are not overwritten.
Further statements |\providecommand{\version}{...}|
can thus be added before the above code to override it.

For the main file, one might add a line
(between |\childdocmain| and the above block)
%
\begin{center}
|%\ifchilddoc\||else\providecommand{\version}{draft}\||fi|
\end{center}
%
which can be uncommented to produce a draft version.
Likewise one can add a line to the very top of a child file
(above the |\childdocof{|\textit{main}|}| directive)
%
\begin{center}
|%\providecommand{\version}{final}|
\end{center}
%
which can be uncommented to produce the final version of this child document.

%%%%%%%%%%%%%%%%%%%%%%%%%%%%%%%%%%%%%%%%%%%%%%%%%%%%%%%%%%%%%%%%%%%%%%%%%%%%%%%%
\subsection{Forwarding}
\label{sec:forward}

Different versions of the main or child documents
using compilation flags as described in \secref{sec:flags}
can be (permanently) stored in different files
for convenient compilation, viewing and distribution.
To this end, the package defines a command
to pass on compilation to a different file:

%%%%%%%%%%%%%%%%%%%%%%%%%%%%%%%%%%%%%%%%
\DescribeMacro{\childdocforward}
The command |\childdocforward| redirects processing to
another source file:
%
\begin{center}
\begin{tabular}{l}
|\input{childdoc.def}|\\
|\childdocforward[|\textit{main}|]{|\textit{dest}|}|\\
\end{tabular}
\end{center}
%
The argument \textit{dest} is the destination file
(without extension).
It should be the main file or one of the child files.
Note that further \textsf{childdoc} directives
such as |\childdocof| and |\childdocforward|
in the indicated file will be processed in this form.
The optional argument \textit{main}
passes on directly to the main file \textit{main}
while pretending to compile the child \textit{dest}.
This form behaves as if \textit{dest}
issues |\childdocof{|\textit{main}|}| right away,
and no further \textsf{childdoc} directives will be processed.

%%%%%%%%%%%%%%%%%%%%%%%%%%%%%%%%%%%%%%%%
\DescribeMacro{\...prefix}
In the alternative form |\childdocforwardprefix|,
%
\begin{center}
\begin{tabular}{l}
|\input{childdoc.def}|\\
|\childdocforwardprefix[|\textit{main}|]{|\textit{prefix}|}{|\textit{dest}|}|
\end{tabular}
\end{center}
%
the destination file is determined by a pattern
depending on the current file:
To make this work, the current file must be called
`{\textit{prefix}\hspace{0.2em}\textit{suffix}}'
with \textit{prefix} matching precisely the argument.
Processing is then passed on to the file
`{\textit{dest}\hspace{0.2em}\textit{suffix}}'.
Surely, the same effect is achieved by
directly specifying the
argument `{\textit{dest}\hspace{0.2em}\textit{suffix}}'
in the first form.
However, that requires to set up a different file
for each child. With the alternative form of the command
all these files can have exactly the same content
which simplifies setting them up and maintaining them.

For example, the following file |draft.tex|
with a compilation flag |\version| as described in \secref{sec:flags}
compiles the main document as a draft:
%
\begin{center}
\begin{tabular}{l}
|\def\version{draft}|\\
|\input{childdoc.def}|\\
|\childdocforward{|\textit{main}|}|
\end{tabular}
\end{center}
%
Likewise, the following files |final|\textit{nn}|.tex|
compile the final version of the child document
|child|\textit{nn}|.tex|:
%
\begin{center}
\begin{tabular}{l}
|\def\version{final}|\\
|\input{childdoc.def}|\\
|\childdocforwardprefix{final}{child}|
\end{tabular}
\end{center}
%

Note that when several versions of a main file and/or of each child file
are to be generated, it may be convenient to set up a |Makefile| or
shell script to automatise the process.

%%%%%%%%%%%%%%%%%%%%%%%%%%%%%%%%%%%%%%%%%%%%%%%%%%%%%%%%%%%%%%%%%%%%%%%%%%%%%%%%
\subsection{Command Line Processing}
\label{sec:commandline}

The effect of redirection files can also be achieved by invoking
the \LaTeX{} compiler with a more elaborate command line.
Most conveniently this should be done as part
of a shell script or a |Makefile|.

When using \textsf{childdoc} in the main file, the following
command lines effectively perform a redirection
(note that depending on the shell being used,
backslashes may have to be doubled: `|\|' $\to$ `|\\|'):
%
\begin{center}
|... -jobname "|\textit{target}|" |\\|"|[\textit{flags}]%
|\input{childdoc.def}\childdocforward[|\textit{main}|]{|\textit{dest}|}"|
\end{center}
%
Here \textit{target} is the name of the output file,
\textit{main} is the name of the main file
and \textit{dest} is the name of the main or child file to be processed
(all filenames without extensions).
The optional argument \textit{main} can be omitted
if \textit{main} matches \textit{dest}.
Optionally, compilation \textit{flags} can be defined via |\def| commands.
This command line makes the \TeX{} engine believe
it is compiling the file \textit{target}
whose content is specified as the latter parameter.
The provided code then forwards the processing to
\textit{main} or \textit{dest} as described in \secref{sec:forward}.

%%%%%%%%%%%%%%%%%%%%%%%%%%%%%%%%%%%%%%%%%%%%%%%%%%%%%%%%%%%%%%%%%%%%%%%%%%%%%%%%
\subsection{Include by Input}
\label{sec:input}

Including child documents by |\include| has some restrictions by design.
Most notably, the content of a child document always occupies
its own set of pages; pages cannot be shared between child documents.
Usually, this behaviour makes perfect sense
because each child document contain an essential part of the document.
However, in some situations it may be desirable to compose
a document from a collection of parts
without having mandatory page breaks between then.
For this case, the package
provides a mechanism to include parts
by |\input| which can also be processed individually.
However, by construction this mechanism
requires manual handling of the content to be output.

%%%%%%%%%%%%%%%%%%%%%%%%%%%%%%%%%%%%%%%%
\DescribeMacro{\ifchilddocmanual}
The main file should be prepared as usual, see \secref{sec:include}.
However, the document body must make a distinction
between processing of an individual part and of the main document, e.g.:
%
\begin{center}
\begin{tabular}{l}
|\ifchilddocmanual|\\
|\input{\childdocname}|\\
|\||else|\\
\textit{document body with }|\input{|\textit{part}|}|\\
|\||fi|
\end{tabular}
\end{center}
%
The conditional |\ifchilddocmanual| is true whenever
a part to be included by |\input| is being compiled,
and the name of the part is stored in |\childdocname|.

%%%%%%%%%%%%%%%%%%%%%%%%%%%%%%%%%%%%%%%%
\DescribeMacro{\childdocby}
Each part to be included by |\input| should start with:
%
\begin{center}
\begin{tabular}{l}
|\input{childdoc.def}|\\
|\childdocby{|\textit{main}|}|\\
\end{tabular}
\end{center}
%
The directive |\childdocby| is similar to |\childdocof|
described in \secref{sec:include},
but the subsequent selection of content must be done manually.
To that end, both |\ifchilddoc| and |\ifchilddocmanual|
will be true upon processing of a part,
and the name of the part is stored in |\childdocname|.
Note that |\jobname| will be set to the filename of the current part
so that each part receives an individual |.aux| file
that does not interfere with the |.aux| file(s) of the main document.
This behaviour can be altered by the alternative form
|\childdocby[*]{|\textit{main}|}| (with a non-empty optional argument)
which uses the |.aux| file of the main document
by setting |\jobname| to \textit{main}.

%%%%%%%%%%%%%%%%%%%%%%%%%%%%%%%%%%%%%%%%%%%%%%%%%%%%%%%%%%%%%%%%%%%%%%%%%%%%%%%%
\subsection{Driver Development}
\label{sec:driver}

The \textsf{childdoc} mechanism can also be use for the development
of definition files such as \LaTeX{} styles or classes.
This case differs from the above setup with multiple parts
included by |\include| in that no |\includeonly| should be invoked.
This can be achieved by starting the include file
(before |\ProvidesPackage|) with:
%
\begin{center}
\begin{tabular}{l}
|\input{childdoc.def}|\\
|\childdocforward{|\textit{main}|}|\\
\end{tabular}
\end{center}
%
or alternatively with:
%
\begin{center}
\begin{tabular}{l}
|\input{childdoc.def}|\\
|\childdocby{|\textit{main}|}|\\
\end{tabular}
\end{center}
%
Both forms have slightly different effects as described above.
The main file is prepared as usual, see \secref{sec:include}.

%%%%%%%%%%%%%%%%%%%%%%%%%%%%%%%%%%%%%%%%%%%%%%%%%%%%%%%%%%%%%%%%%%%%%%%%%%%%%%%%
\subsection{Legacy Detection}
\label{sec:detection}

The directive |\childdocmain| in the main file can detect
whether the complete document or merely a child is to be compiled
even without using the directive |\childdocof|.
This method is deprecated because it is less robust
and there is no compelling reason to use it;
it is merely provided for backward compatibility
and it may be removed in future versions.

If the detection mechanism is to be used,
it is mandatory to correctly specify
the filename of the main file as the argument of |\childdocmain|:
%
\begin{center}
\begin{tabular}{l}
|\input{childdoc.def}|\\
|\childdocmain{|\textit{main}|}|\\
\end{tabular}
\end{center}
%
If |\jobname| does not match the argument \textit{main} of |\childdocmain|,
it is assumed that |\jobname| points to the child file to be compiled.
When using |\childdocmain| with the main file specified as argument,
it suffices to start a child file
with just |\input{|\textit{main}|}|
without loading of the package and using |\childdocof|.
If instead all processing is done
with the appropriate \textsf{childdoc} directives,
the argument of \textit{main} of |\childdocmain| can be empty.

An alternative version of the command line processing described
in \secref{sec:commandline} using the detection mechanism reads:
%
\begin{center}
|... -jobname "|\textit{target}|" "|[\textit{flags}]%
[|\def\jobname{|\textit{dest}|}|]|\input{|\textit{main}|}"|
\end{center}

%%%%%%%%%%%%%%%%%%%%%%%%%%%%%%%%%%%%%%%%%%%%%%%%%%%%%%%%%%%%%%%%%%%%%%%%%%%%%%%%
\subsection{Manual Code}
\label{sec:manual}

In case one cannot be certain whether the definitions file |childdoc.def|
is installed on the target \TeX{} distribution
and one prefers not to ship it,
it is conceivable to paste a few relevant commands into the sources.

To that end, drop all statements |\input{childdoc.def}|
and perform the replacements as outlined below.
Instead of |\childdocmain{|\textit{main}|}| add the following code
to the top of the main file:
%
\begin{center}
\begin{tabular}{l}
|\||ifdefined\childdocname\endinput\||fi\newif\ifchilddoc|\\
|\edef\childdocname{\scantokens\expandafter{\jobname\noexpand}}|\\
|\def\childdocmain{|\textit{main}|}\||ifx\childdocmain\childdocname\||else|\\
|\childdoctrue\includeonly{\childdocname}\let\jobname\childdocmain\||fi|\\
\end{tabular}
\end{center}
%
Instead of |\childdocof{|\textit{main}|}| just include the main file
at the top of each child file:
%
\begin{center}
|\input{|\textit{main}|}|
\end{center}
%
A simple redirection |\childdocforward{|\textit{dest}|}| is achieved by:
%
\begin{center}
|\def\jobname{|\textit{dest}|}\input{\jobname}|
\end{center}
%
The redirection with prefix
|\childdocforwardprefix[|\textit{prefix}|]{|\textit{dest}|}|
is accomplished by:
%
\begin{center}
\begin{tabular}{l}
|{\edef\jobname{\scantokens\expandafter{\jobname\noexpand}}|\\
|\def\redirectjob |\textit{prefix}|#1~~~{\gdef\jobname{|\textit{dest}|#1}}|\\
|\expandafter\redirectjob\jobname~~~}\input{\jobname}|
\end{tabular}
\end{center}

In an alternative approach,
child documents can be compiled by a specific command line
without additional code or specific definitions:
%
\begin{center}
|... -jobname "|\textit{target}|" "|[\textit{flags}]%
|\includeonly{|\textit{dest}|}\input{|\textit{main}|}"|
\end{center}
%

%%%%%%%%%%%%%%%%%%%%%%%%%%%%%%%%%%%%%%%%%%%%%%%%%%%%%%%%%%%%%%%%%%%%%%%%%%%%%%%%
%%%%%%%%%%%%%%%%%%%%%%%%%%%%%%%%%%%%%%%%%%%%%%%%%%%%%%%%%%%%%%%%%%%%%%%%%%%%%%%%
\section{Information}

%%%%%%%%%%%%%%%%%%%%%%%%%%%%%%%%%%%%%%%%%%%%%%%%%%%%%%%%%%%%%%%%%%%%%%%%%%%%%%%%
\subsection{Copyright}

Copyright \copyright{} 2017--2018 Niklas Beisert

This work may be distributed and/or modified under the
conditions of the \LaTeX{} Project Public License, either version 1.3
of this license or (at your option) any later version.
The latest version of this license is in
  \url{http://www.latex-project.org/lppl.txt}
and version 1.3 or later is part of all distributions of \LaTeX{}
version 2005/12/01 or later.

This work has the LPPL maintenance status `maintained'.

The Current Maintainer of this work is Niklas Beisert.

This work consists of the files |README.txt|, |childdoc.ins| and |childdoc.dtx|
as well as the derived files |childdoc.def|, |cdocsamp.tex|
with |cdocsch1.tex|, |cdocsch2.tex|, |cdocspt3.tex|, |cdocspt4.tex|,
|cdocsdrf.tex|, |cdocsfn1.tex|, |cdocsfn2.tex|
as well as |childdoc.pdf|.

%%%%%%%%%%%%%%%%%%%%%%%%%%%%%%%%%%%%%%%%%%%%%%%%%%%%%%%%%%%%%%%%%%%%%%%%%%%%%%%%
\subsection{Files and Installation}

The package consists of the files:
%
\begin{center}
\begin{tabular}{ll}
    |README.txt|   & readme file \\
    |childdoc.ins| & installation file \\
    |childdoc.dtx| & source file \\
    |childdoc.def| & definition file \\
    |cdocsamp.tex| & sample main file \\
    |cdocsch1.tex| & sample include file \\
    |cdocsch2.tex| & sample include file \\
    |cdocspt3.tex| & sample part file \\
    |cdocspt4.tex| & sample part file \\
    |cdocsdrf.tex| & sample redirection file \\
    |cdocsfn1.tex| & sample redirection file \\
    |cdocsfn2.tex| & sample redirection file \\
    |childdoc.pdf| & manual
\end{tabular}
\end{center}
%
The distribution consists of the files
|README.txt|, |childdoc.ins| and |childdoc.dtx|.
%
\begin{itemize}
\item
Run (pdf)\LaTeX{} on |childdoc.dtx|
to compile the manual |childdoc.pdf| (this file).
\item
Run \LaTeX{} on |childdoc.ins| to create the definitions file |childdoc.def|
and the sample |cdocsamp.tex| with include files
|cdocsch1.tex|, |cdocsch2.tex|, |cdocspt3.tex|, |cdocspt4.tex|,
|cdocsdrf.tex|, |cdocsfn1.tex|, |cdocsfn2.tex|.
Then copy the file |childdoc.def| to an appropriate directory of your \LaTeX{}
distribution, e.g.\ \textit{texmf-root}|/tex/latex/childdoc|.
\end{itemize}

%%%%%%%%%%%%%%%%%%%%%%%%%%%%%%%%%%%%%%%%%%%%%%%%%%%%%%%%%%%%%%%%%%%%%%%%%%%%%%%%
\subsection{Related CTAN Packages}

There are several other packages which offer a similar functionality:
%
\begin{itemize}
\item
The packages
\href{http://ctan.org/pkg/docmute}{\textsf{docmute}},
\href{http://ctan.org/pkg/includex}{\textsf{includex}} and
\href{http://ctan.org/pkg/standalone}{\textsf{standalone}}
provide commands to include only the document body of
a child file thus allowing both files to be compiled individually.
\item
The packages \href{http://ctan.org/pkg/subdocs}{\textsf{subdocs}}
and \href{http://ctan.org/pkg/subfiles}{\textsf{subfiles}}
provide structures in which the main and child documents can be
encapsulated and allowing them to be compiled individually.
The inclusion mechanism is different from the conventional |\include|.
\item
The package \href{http://ctan.org/pkg/combine}{\textsf{combine}}
is an elaborate solution to combine several documents into one.
\end{itemize}
%
See also the CTAN topic \href{http://ctan.org/topic/subdocs}{\textsf{subdocs}}
for further related packages.
The present package differs from the above solutions in that
a document structure constructed with the conventional |\include| mechanism
just needs two extra commands at the top of every file
such that all constituent files can be compiled individually.

%%%%%%%%%%%%%%%%%%%%%%%%%%%%%%%%%%%%%%%%%%%%%%%%%%%%%%%%%%%%%%%%%%%%%%%%%%%%%%%%
%\subsection{Feature Suggestions}
%
%The following is a list of features which may be useful for future
%versions of this package:
%%
%\begin{itemize}
%\item
%\ldots
%\end{itemize}

%%%%%%%%%%%%%%%%%%%%%%%%%%%%%%%%%%%%%%%%%%%%%%%%%%%%%%%%%%%%%%%%%%%%%%%%%%%%%%%%
\subsection{Revision History}

%%%%%%%%%%%%%%%%%%%%%%%%%%%%%%%%%%%%%%%%
\paragraph{v2.0:} 2018/12/30

\begin{itemize}
\item
immediate forward processing
\item
added |\childdocby| mechanism
\item
manual restructured
\end{itemize}

%%%%%%%%%%%%%%%%%%%%%%%%%%%%%%%%%%%%%%%%
\paragraph{v1.6:} 2018/01/17

\begin{itemize}
\item
application for development of include files
\item
corrections to manual
\end{itemize}

%%%%%%%%%%%%%%%%%%%%%%%%%%%%%%%%%%%%%%%%
\paragraph{v1.5:} 2017/05/21

\begin{itemize}
\item
more complete structuring introduced
\item
|\childdocof| introduced
\item
|\childdoc| renamed to |\childdocmain|
\item
|\childredirect| renamed to |\childdocforward| and |\childdocforwardprefix|
and functionality expanded
\end{itemize}

%%%%%%%%%%%%%%%%%%%%%%%%%%%%%%%%%%%%%%%%
\paragraph{v1.0:} 2017/04/27

\begin{itemize}
\item
manual and install package
\item
first version published on CTAN
\end{itemize}

%%%%%%%%%%%%%%%%%%%%%%%%%%%%%%%%%%%%%%%%
\paragraph{v0.6:} 2017/04/26

\begin{itemize}
\item
redirection mechanism added
\end{itemize}

%%%%%%%%%%%%%%%%%%%%%%%%%%%%%%%%%%%%%%%%
\paragraph{v0.5:} 2017/04/26

\begin{itemize}
\item
functionality in definition file
\end{itemize}


%%%%%%%%%%%%%%%%%%%%%%%%%%%%%%%%%%%%%%%%%%%%%%%%%%%%%%%%%%%%%%%%%%%%%%%%%%%%%%%%
%%%%%%%%%%%%%%%%%%%%%%%%%%%%%%%%%%%%%%%%%%%%%%%%%%%%%%%%%%%%%%%%%%%%%%%%%%%%%%%%
%%%%%%%%%%%%%%%%%%%%%%%%%%%%%%%%%%%%%%%%%%%%%%%%%%%%%%%%%%%%%%%%%%%%%%%%%%%%%%%%
\appendix

\settowidth\MacroIndent{\rmfamily\scriptsize 000\ }

 \DocInput{childdoc.dtx}

\end{document}
%</driver>
% \fi
%
% %%%%%%%%%%%%%%%%%%%%%%%%%%%%%%%%%%%%%%%%%%%%%%%%%%%%%%%%%%%%%%%%%%%%%%%%%%%%%%
% %%%%%%%%%%%%%%%%%%%%%%%%%%%%%%%%%%%%%%%%%%%%%%%%%%%%%%%%%%%%%%%%%%%%%%%%%%%%%%
% \section{Sample}
%\iffalse
%<*samplemain>
%\fi
%
% The following presents a sample document
% with two chapters, two parts, a title page,
% a compile flag as well as three forwarding files to set the flag.
% It consists of eight |.tex| files:
% \begin{center}
% \begin{tabular}{ll}
% |cdocsamp.tex|&main file\\
% |cdocsch1.tex|&include file for chapter 1\\
% |cdocsch2.tex|&include file for chapter 2\\
% |cdocspt3.tex|&include file for part 3\\
% |cdocspt4.tex|&include file for part 4\\
% |cdocsdrf.tex|&forwarding file for main file in draft mode\\
% |cdocsfi1.tex|&forwarding file for final version of chapter 1\\
% |cdocsfi2.tex|&forwarding file for final version of chapter 2\\
% \end{tabular}
% \end{center}
% Each of the eight files can be compiled directly by the \LaTeX{} compiler.
%
% %%%%%%%%%%%%%%%%%%%%%%%%%%%%%%%%%%%%%%
% \paragraph{Main File.}
%
% The main file is called |cdocsamp.tex|.
%
% Load the \textsf{childdoc} definitions and
% declare the filename for the main document:
%    \begin{macrocode}
\input{childdoc.def}
\childdocmain{}
%    \end{macrocode}

% Optional override for |\version| flag:
%    \begin{macrocode}
%%\ifchilddoc\else\providecommand{\version}{draft}\fi
%    \end{macrocode}

% Define the default values for the |\version| flag
% (|final| for the main file and |draft| for childs):
%    \begin{macrocode}
\ifchilddoc
\providecommand{\version}{draft}
\else
\providecommand{\version}{final}
\fi
%    \end{macrocode}

% Load the standard document class:
%    \begin{macrocode}
\documentclass[12pt]{article}
%    \end{macrocode}

% Start the document body:
%    \begin{macrocode}
\begin{document}
%    \end{macrocode}

% Declare a title page.
% Print title, part of document being processed and version flag:
%    \begin{macrocode}
\addtocounter{page}{-1}
\begin{center}
{\LARGE\bfseries{}childdoc example\par}
\vspace{1cm}
\ifchilddoc
\ifchilddocmanual part\else chapter\fi:
`\childdocname' of `\childdocjob'\par
\else
main document: `\childdocjob'\par
\fi
version: \version\par
\end{center}
\newpage
%    \end{macrocode}

% Manually include selected file,
% otherwise process as usual:
%    \begin{macrocode}
\ifchilddocmanual
\section*{part `\childdocname'}
\input{\childdocname}
\else
%    \end{macrocode}

% Include the two chapters:
%    \begin{macrocode}
\include{cdocsch1}
\include{cdocsch2}
%    \end{macrocode}

% Include the two parts unless only chapters should be displayed:
%    \begin{macrocode}
\ifchilddoc\else
\section{part three}
\input{cdocspt3}
\section{part four}
\input{cdocspt4}
\fi
%    \end{macrocode}

% Process as usual until here:
%    \begin{macrocode}
\fi
%    \end{macrocode}

% End of document body:
%    \begin{macrocode}
\end{document}
%    \end{macrocode}
%\iffalse
%</samplemain>
%\fi
%
% %%%%%%%%%%%%%%%%%%%%%%%%%%%%%%%%%%%%%%
% \paragraph{Chapter Include Files.}
%
% The include files are called |cdocsch1.tex| and |cdocsch2.tex|.
%
%\iffalse
%<*samplechap1|samplechap2>
%\fi

% Optional override for |\version| flag:
%    \begin{macrocode}
%%\providecommand{\version}{final}
%    \end{macrocode}

% Include the main document:
%    \begin{macrocode}
\input{childdoc.def}
\childdocof{cdocsamp}
%    \end{macrocode}

%\iffalse
%</samplechap1|samplechap2>
%\fi
%
%\iffalse
%<*samplechap1>
%\fi
% Some text for chapter 1:
%    \begin{macrocode}
\section{one}
some text in chapter one
%    \end{macrocode}

%\iffalse
%</samplechap1>
%\fi
% Some text for chapter 2:
%\iffalse
%<*samplechap2>
%\fi
%    \begin{macrocode}
\section{two}
more text in chapter two
%    \end{macrocode}

%\iffalse
%</samplechap2>
%\fi
%
% %%%%%%%%%%%%%%%%%%%%%%%%%%%%%%%%%%%%%%
% \paragraph{Part Include Files.}
%
% The include files are called |cdocspt3.tex| and |cdocspt4.tex|.
%
%\iffalse
%<*samplepart3|samplepart4>
%\fi

% Optional override for |\version| flag:
%    \begin{macrocode}
%%\providecommand{\version}{final}
%    \end{macrocode}

% Include the main document:
%    \begin{macrocode}
\input{childdoc.def}
\childdocby{cdocsamp}
%    \end{macrocode}

%\iffalse
%</samplepart3|samplepart4>
%\fi
%
%\iffalse
%<*samplepart3>
%\fi
% Some text for part 3:
%    \begin{macrocode}
some text in part three
%    \end{macrocode}

%\iffalse
%</samplepart3>
%\fi
% Some text for part 4:
%\iffalse
%<*samplepart4>
%\fi
%    \begin{macrocode}
more text in part four
%    \end{macrocode}

%\iffalse
%</samplepart4>
%\fi
%
% %%%%%%%%%%%%%%%%%%%%%%%%%%%%%%%%%%%%%%
% \paragraph{Forwarding for a Complete Draft.}
%
% The following forwarding file |cdocsdrf.tex|
% compiles the main document in draft mode:
%\iffalse
%<*sampledraft>
%\fi
%    \begin{macrocode}
\def\version{draft}
\input{childdoc.def}
\childdocforward{cdocsamp}
%    \end{macrocode}

%\iffalse
%</sampledraft>
%\fi
%
% %%%%%%%%%%%%%%%%%%%%%%%%%%%%%%%%%%%%%%
% \paragraph{Forwarding for Final Version of the Chapters.}
%
% The following forwarding files |cdocsfn1.tex| and |cdocsfn2.tex|
% (with identical content)
% compile the final versions of the child documents
% |cdocsch1.tex| and |cdocsch2.tex|, respectively:
%\iffalse
%<*samplefinal>
%\fi
%    \begin{macrocode}
\def\version{final}
\input{childdoc.def}
\childdocforwardprefix[cdocsamp]{cdocsfn}{cdocsch}
%    \end{macrocode}

%\iffalse
%</samplefinal>
%\fi
%
% %%%%%%%%%%%%%%%%%%%%%%%%%%%%%%%%%%%%%%
% \paragraph{Command Line Processing.}
%
% The following three command lines generate the output files
% |cdocscld|, |cdocscl1| and |cdocscl2|
% which should be identical to
% |cdocsdrf|, |cdocsch1| and |cdocsfn2|, respectively:
% \begin{center}
% \begin{tabular}{l}
% |latex -jobname cdocscld \|\\
% |  "\def\version{draft}\input{childdoc.def}\childdocforward{cdocsamp}"|\\
% |latex -jobname cdocscl1 \|\\
% |  "\input{childdoc.def}\childdocforward[cdocsamp]{cdocsch1}"|\\
% |latex -jobname cdocscl2 \|\\
% |  "\def\version{final}\input{childdoc.def}\childdocforward{cdocsch2}"|
% \end{tabular}
% \end{center}
% Note that the trailing backslash on each first line
% merely continues the input to the second line
% (for convenient cut ant paste).
% Furthermore, the command |latex| can be replaced by any
% of its alternative versions such as |pdflatex|.
%
% %%%%%%%%%%%%%%%%%%%%%%%%%%%%%%%%%%%%%%%%%%%%%%%%%%%%%%%%%%%%%%%%%%%%%%%%%%%%%%
% %%%%%%%%%%%%%%%%%%%%%%%%%%%%%%%%%%%%%%%%%%%%%%%%%%%%%%%%%%%%%%%%%%%%%%%%%%%%%%
% \section{Implementation}
%\iffalse
%<*package>
%\fi
%
% This section describes the definitions file |childdoc.def|.

% The definitions cannot be loaded using |\usepackage| or |\RequirePackage|
% which has a mechanism to prevent loading a style file more than once.
% When loading the definitions by means of |\input|
% multiple instances have to be prevented manually:
%\iffalse
%This code needs to be before the `\ProvidesFile' directive
%which is defined at the beginning of this file.
%Therefore it is also placed there and commented out here.
%</package>
%<*discard>
%\fi
%    \begin{macrocode}
\ifdefined\childdocmain\endinput\fi
%    \end{macrocode}
%\iffalse
%</discard>
%<*package>
%\fi
%
% \macro{\ifchilddoc}
% \macro{\ifchilddocmanual}
% The conditional |\ifchilddoc| tells whether a
% child (true) or main (false) document is being compiled.
% The conditional |\ifchilddocmanual| tells whether
% the |\includeonly| mechanism is used (false) or
% the selection of child files must be performed manually (true).
% The definitions initialise to false:
%    \begin{macrocode}
\newif\ifchilddoc
\newif\ifchilddocmanual
%    \end{macrocode}

% \macro{\childdocname}
% \macro{\childdocjob}
% The macro |\childdocname| stores the name of the main document
% to be compiled. The macro |\childdocjob| stores the name of
% the document on which the \LaTeX{} compiler was originally invoked.
% The content of |\jobname| cannot be compared
% to filenames specified in the source due to different catcodes.
% The following code rescans |\jobname|, stores the result
% in |\childdocname| and saves a copy in |\childdocjob|:
%    \begin{macrocode}
\edef\childdocname{\scantokens\expandafter{\jobname\noexpand}}
\let\childdocjob\childdocname
%    \end{macrocode}

% \macro{\childdocdisable}
% The macro |\childdocdisable| prevents the main file
% from being processed more than once.
% At this stage, the main document command |\childdocmain|
% is assumed to be called once again where it should do nothing.
% Any subsequent call to it should prevent
% a secondary processing of the main document
% It overwrites the forwarding commands
% |\childdocof| and |\childdocforward|
% with empty macros to prevent further inclusions of the main document:
%    \begin{macrocode}
\newcommand{\childdocdisable}
{
  \renewcommand{\childdocmain}[1]{\renewcommand{\childdocmain}[1]{\endinput}}
  \renewcommand{\childdocof}[1]{}
  \renewcommand{\childdocby}[2][]{}
  \renewcommand{\childdocforward}[2][]{}
  \renewcommand{\childdocdisable}{}
}
%    \end{macrocode}

% \macro{\childdocmain}
% The macro |\childdocmain| is to be called at the top of the main file
% with nothing or the main filename (without extension) as argument.
% First, it breaks loops.
% If the argument is not empty and does not match |\childdocname|
% (which is set by the first inclusion of |childdoc.def|),
% |\ifchilddoc| is set to true, |\includeonly| is applied to the child file
% and |\jobname| is set to the main file
% (for proper handling of |.aux| files):
%    \begin{macrocode}
\newcommand{\childdocmain}[1]
{
  \childdocdisable\childdocmain{}
  \if?#1?\else
    \begingroup
      \def\childdoctmp{#1}
      \ifx\childdoctmp\childdocname
        \def\childdoctmp{}
      \else
        \def\childdoctmp
        {
          \childdoctrue
          \includeonly{\childdocname}
          \def\childdocjob{#1}
          \def\jobname{#1}
        }
      \fi
      \expandafter
    \endgroup
    \childdoctmp
  \fi
}
%    \end{macrocode}

% \macro{\childdocof}
% The command |\childdocof| redirects
% compilation to the main file |#1|.
%    \begin{macrocode}
\newcommand{\childdocof}[1]
{
  \childdocdisable
  \childdoctrue
  \includeonly{\childdocname}
  \def\jobname{#1}
  \def\childdocjob{#1}
  \input{#1}
}
%    \end{macrocode}

% \macro{\childdocby}
% The command |\childdocby| ....
%    \begin{macrocode}
\newcommand{\childdocby}[2][]
{
  \childdocdisable
  \childdoctrue
  \childdocmanualtrue
  \if?#1?\else
    \def\jobname{#2}
  \fi
  \def\childdocjob{#2}
  \input{#2}
  \endinput
}
%    \end{macrocode}

% \macro{\childdocforward}
% The command |\childdocforward| redirects
% compilation to the main file or
% (if the optional argument is given) a child file.
% Parameters are set as if the main file
% or a child file starting with |\childdocof| was compiled.
% Then compilation is handed over to the main file:
%    \begin{macrocode}
\newcommand{\childdocforward}[2][]
{
  \begingroup
    \if?#1?
      \def\childdoctmp
      {
        \def\childdocname{#2}
        \def\childdocjob{#2}
        \def\jobname{#2}
        \input{#2}
        \endinput
      }
    \else
      \def\childdoctmp
      {
        \childdocdisable
        \def\childdocname{#2}
        \childdoctrue
        \includeonly{#2}
        \def\childdocjob{#1}
        \def\jobname{#1}
        \input{#1}
        \endinput
      }
    \fi
    \expandafter
  \endgroup
  \childdoctmp
}
%    \end{macrocode}

% \macro{\childdocforwardprefix}
% The command |\childdocforwardprefix| redirects
% compilation to the main or a child file by means of a pattern.
% The prefix |#1| in the current filename is replaced by |#2|
% and the suffix of the current filename is kept
% (it is assumed that the filename does not contain the substring `|~~~|'
% which is used as a delimiter).
% Compilation is handed over to the new file by |\childdocforward|:
%    \begin{macrocode}
\newcommand{\childdocforwardprefix}[3][]
{
  \begingroup
    \def\childdocextract #2##1~~~{\def\childdoctmp{\childdocforward[#1]{#3##1}}}
    \expandafter\childdocextract\childdocname~~~
    \expandafter
  \endgroup
  \childdoctmp
}
%    \end{macrocode}

% \macro{\childdoc}
% The deprecated macro |\childdoc| is a legacy version of |\childdocmain|:
%    \begin{macrocode}
\newcommand{\childdoc}{\childdocmain}
%    \end{macrocode}

% \macro{\childdocredirect}
% The deprecated macro |\childdocredirect| is a legacy version
% of |\childdocforward| and |\childdocforwardprefix|:
%    \begin{macrocode}
\newcommand{\childdocredirect}[2][]
{
  \begingroup
    \if?#1?
      \def\childdoctmp{\childdocforward{#2}}
    \else
      \def\childdoctmp{\childdocforwardprefix{#1}{#2}}
    \fi
    \expandafter
  \endgroup
  \childdoctmp
}
%    \end{macrocode}

%\iffalse
%</package>
%\fi
%
\endinput
|\\
|\childdocforwardprefix{final}{child}|
\end{tabular}
\end{center}
%

Note that when several versions of a main file and/or of each child file
are to be generated, it may be convenient to set up a |Makefile| or
shell script to automatise the process.

%%%%%%%%%%%%%%%%%%%%%%%%%%%%%%%%%%%%%%%%%%%%%%%%%%%%%%%%%%%%%%%%%%%%%%%%%%%%%%%%
\subsection{Command Line Processing}
\label{sec:commandline}

The effect of redirection files can also be achieved by invoking
the \LaTeX{} compiler with a more elaborate command line.
Most conveniently this should be done as part
of a shell script or a |Makefile|.

When using \textsf{childdoc} in the main file, the following
command lines effectively perform a redirection
(note that depending on the shell being used,
backslashes may have to be doubled: `|\|' $\to$ `|\\|'):
%
\begin{center}
|... -jobname "|\textit{target}|" |\\|"|[\textit{flags}]%
|% \iffalse
%
% childdoc.dtx Copyright (C) 2017-2018 Niklas Beisert
%
% This work may be distributed and/or modified under the
% conditions of the LaTeX Project Public License, either version 1.3
% of this license or (at your option) any later version.
% The latest version of this license is in
%   http://www.latex-project.org/lppl.txt
% and version 1.3 or later is part of all distributions of LaTeX
% version 2005/12/01 or later.
%
% This work has the LPPL maintenance status `maintained'.
%
% The Current Maintainer of this work is Niklas Beisert.
%
% This work consists of the files childdoc.dtx and childdoc.ins
% and the derived files childdoc.def and cdocsamp.tex with
% cdocsch1.tex, cdocsch2.tex, cdocsdrf.tex, cdocsfn1.tex, cdocsfn2.tex.
%
%<package>\ifdefined\childdocmain\endinput\fi
%<package>\ProvidesFile{childdoc.def}[2018/12/30 v2.0 child document driver]
%<samplemain>\ProvidesFile{cdocsamp.tex}[2018/12/30 v2.0 sample for childdoc]
%<*driver>
%\ProvidesFile{childdoc.drv}[2018/12/30 v2.0 childdoc reference manual file]
\PassOptionsToClass{10pt,a4paper}{article}
\documentclass{ltxdoc}

\usepackage[margin=35mm]{geometry}
\usepackage{hyperref}
\usepackage{hyperxmp}
\usepackage[usenames]{color}

\hypersetup{colorlinks=true}
\hypersetup{pdfstartview=FitH}
\hypersetup{pdfpagemode=UseNone}
\hypersetup{pdfsource={}}
\hypersetup{pdflang={en-UK}}
\hypersetup{pdfcopyright={Copyright 2017-2018 Niklas Beisert.
  This work may be distributed and/or modified under the
  conditions of the LaTeX Project Public License, either version 1.3
  of this license or (at your option) any later version.}}
\hypersetup{pdflicenseurl={http://www.latex-project.org/lppl.txt}}
\hypersetup{pdfcontactaddress={ETH Zurich, ITP, HIT K,
  Wolfgang-Pauli-Strasse 27}}
\hypersetup{pdfcontactpostcode={8093}}
\hypersetup{pdfcontactcity={Zurich}}
\hypersetup{pdfcontactcountry={Switzerland}}
\hypersetup{pdfcontactemail={nbeisert@itp.phys.ethz.ch}}
\hypersetup{pdfcontacturl={http://people.phys.ethz.ch/\xmptilde nbeisert/}}

\newcommand{\secref}[1]{\hyperref[#1]{section \ref*{#1}}}

\parskip1ex
\parindent0pt
\let\olditemize\itemize
\def\itemize{\olditemize\parskip0pt}

\begin{document}

\title{The \textsf{childdoc} Package}
\hypersetup{pdftitle={The childdoc Package}}
\author{Niklas Beisert\\[2ex]
  Institut f\"ur Theoretische Physik\\
  Eidgen\"ossische Technische Hochschule Z\"urich\\
  Wolfgang-Pauli-Strasse 27, 8093 Z\"urich, Switzerland\\[1ex]
  \href{mailto:nbeisert@itp.phys.ethz.ch}
  {\texttt{nbeisert@itp.phys.ethz.ch}}}
\hypersetup{pdfauthor={Niklas Beisert}}
\hypersetup{pdfsubject={Manual for the LaTeX2e Package childdoc}}
\date{30 December 2018, \textsf{v2.0}}
\maketitle

\begin{abstract}\noindent
\textsf{childdoc} is a \LaTeXe{} package
that enables the direct compilation
of document sections included by |\include|
to individual files.
\end{abstract}

\begingroup
\parskip0ex
\tableofcontents
\endgroup

%%%%%%%%%%%%%%%%%%%%%%%%%%%%%%%%%%%%%%%%%%%%%%%%%%%%%%%%%%%%%%%%%%%%%%%%%%%%%%%%
%%%%%%%%%%%%%%%%%%%%%%%%%%%%%%%%%%%%%%%%%%%%%%%%%%%%%%%%%%%%%%%%%%%%%%%%%%%%%%%%
\section{Introduction}

\LaTeX{} provides a mechanism to structure a large document (such as a book)
into a main file and several child files (containing the chapters)
using the |\include| command.
This mechanism is beneficial for documents
which span hundreds of pages in order to
make the source file(s) more manageable.
Moreover, compilation can be restricted to
selected child files by means of the |\includeonly| command.
The latter feature can be used to reduce the compilation time while editing
(this was significantly more useful in the earlier days of \LaTeX{})
or to generate a smaller document which is easier to navigate.
Another application of |\includeonly| is to generate
documents consisting of selected parts of the complete document.

However, there are a few drawbacks of the plain |\include| mechanism:
\begin{itemize}
\item
The child files cannot be compiled on their own,
they can only be compiled via the main file.
A naive editing environment
(such as a text editor with an option
to have the current file processed by \LaTeX)
may require one to switch to the main file before compiling;
attempting to compile the child file produces errors.
\item
The main file must be modified (each time)
to adjust the |\includeonly| command
to the present needs. This easily leaves the main file in a messy state.
\item
The generated document will always carry the filename
of the main document. This is inconvenient if
several child files are to be compiled and
to be kept for distribution.
\end{itemize}

The present package provides a simple interface
to make child files individually compilable by \LaTeX{}.
Compiling a child file then has the same effect as compiling
the main file with an |\includeonly| command
to select the appropriate child.
Moreover the generated document will carry the name of the child
rather than the main file.
This resolves all three above issues.

This feature is meant to make the editing of books,
thesis documents and lecture notes somewhat more convenient.
However, the package can also be used efficiently for
composing a series of documents (such as exercise sheets)
which are typically distributed individually.
It then assists the author in generating the individual documents
(potentially in different versions)
as well as a document containing the collected series.
Another application is in developing style files
or other kinds of included material
where compilation of the style file could redirect
to a sample or test file.

%%%%%%%%%%%%%%%%%%%%%%%%%%%%%%%%%%%%%%%%%%%%%%%%%%%%%%%%%%%%%%%%%%%%%%%%%%%%%%%%
%%%%%%%%%%%%%%%%%%%%%%%%%%%%%%%%%%%%%%%%%%%%%%%%%%%%%%%%%%%%%%%%%%%%%%%%%%%%%%%%
\section{Usage}

First of all, the package \textsf{childdoc} is \emph{not} a standard
\LaTeXe{} |.sty| style file! Therefore it needs to be invoked in
a non-standard way.

%%%%%%%%%%%%%%%%%%%%%%%%%%%%%%%%%%%%%%%%%%%%%%%%%%%%%%%%%%%%%%%%%%%%%%%%%%%%%%%%
\subsection{Included Files}
\label{sec:include}

%%%%%%%%%%%%%%%%%%%%%%%%%%%%%%%%%%%%%%%%
\DescribeMacro{\childdocmain}
To use the package, add the commands
\begin{center}
\begin{tabular}{l}
|\input{childdoc.def}|\\
|\childdocmain{}|\\
\end{tabular}
\end{center}
at the very top of the main \LaTeX{} file,
in particular \emph{before} the |\documentclass| statement!
The argument of |\childdocmain| should be left empty
(but it must be present).

%%%%%%%%%%%%%%%%%%%%%%%%%%%%%%%%%%%%%%%%
\DescribeMacro{\childdocof}
Furthermore, add the commands
\begin{center}
\begin{tabular}{l}
|\input{childdoc.def}|\\
|\childdocof{|\textit{main}|}|\\
\end{tabular}
\end{center}
at the top of every child file \textit{child}
which is included by |\include{|\textit{child}|}|
from within the main file
(or at least for those files to be compiled individually).
The argument \textit{main} must be the filename of the main file.

There are a couple of
considerations in setting up the main and child documents:

%%%%%%%%%%%%%%%%%%%%%%%%%%%%%%%%%%%%%%%%
\paragraph{Restrictions.}

Please note the following restrictions:
\begin{itemize}
\item
|\childdocmain| must be called with one argument \textit{main}
to ensure compatibility with earlier version of the package.
It must either be empty (|\childdocmain{}|)
or precisely match the filename of the main file in which it is specified.
See \secref{sec:detection} for further information.
\item
The filename \textit{main} must be specified without the |.tex| extension.
\item
The filename \textit{main} is case sensitive
(even in case-insensitive file systems)
due to internal string comparison.
\item
The argument \textit{main} should be fully expanded, it cannot be a macro.
\item
Subdirectories and special characters should be avoided in filenames.
\item
The command |\childdocmain{|\textit{main}|}| must be followed by a whitespace.
It should not be followed immediately by another command
or by a comment mark `|%|'.
This is because the \TeX{} parser reads the token immediately following
the argument of |\childdocmain| and puts it
at the beginning of every child section;
however, a white\-space is ignored.
\end{itemize}

%%%%%%%%%%%%%%%%%%%%%%%%%%%%%%%%%%%%%%%%
\paragraph{Content of Main File.}

It is advisable to place all content in the child files included by |\include|.
Any output contained in the main file will appear in all child documents
unless suppressed manually;
it cannot be suppressed automatically by the |\includeonly| directive
and thus should normally be avoided.
A method to include some content in the main file
by means of conditional processing is described in \secref{sec:conditional}.

%%%%%%%%%%%%%%%%%%%%%%%%%%%%%%%%%%%%%%%%
\paragraph{Page Numbering.}

When only a part of the document is compiled,
the appropriate numbering of pages
(as well as other status parameters)
is determined from the |.aux| files.
The latter contain information from previous passes.
However this information needs to propagate through
all intermediate child documents.
Therefore the page numbering in child documents may well
be inconsistent until the complete document is compiled at least once.

A useful (if unconventional) way to always ensure a consistent
page numbering is to restart the numbering in each child document
and denote the pages by `\textit{child}|.|\textit{page}'
where \textit{child} represents the chapter/section number of the child file.
This can be achieved by the command
|\numberwithin{page}{|\textit{child}|}|
of the \textsf{amsmath} package
where \textit{child} can be |chapter| or |section|
depending on the chosen structuring.
Alternatively, one can modify the macro |\thepage| appropriately
and reset the counter |page| at the start of each child file.

%%%%%%%%%%%%%%%%%%%%%%%%%%%%%%%%%%%%%%%%%%%%%%%%%%%%%%%%%%%%%%%%%%%%%%%%%%%%%%%%
\subsection{Conditional Processing}
\label{sec:conditional}

The package provides a mechanism to compile different versions
of a document. To customise the versions further some conditional processing
can come in handy to distinguish which version is being compiled.
The package provides two macros to describe the compilation context:

%%%%%%%%%%%%%%%%%%%%%%%%%%%%%%%%%%%%%%%%
\DescribeMacro{\ifchilddoc}
The conditional |\ifchilddoc| distinguishes between the compilation of
child documents and the main document:
%
\begin{center}
|\ifchilddoc |\textit{child-code}| |[|\||else |\textit{main-code}]| \||fi|
\end{center}

%%%%%%%%%%%%%%%%%%%%%%%%%%%%%%%%%%%%%%%%
\DescribeMacro{\childdocname}
\DescribeMacro{\childdocjob}
The macro |\childdocname| contains the filename (without extension)
of the main or child file being processed.
Note that |\childdocjob| will always contain the name of the main file.

%%%%%%%%%%%%%%%%%%%%%%%%%%%%%%%%%%%%%%%%
\paragraph{Title Page.}

Conditional processing can be used to include a title or banner page
in the main document when proper precautions are taken.
Importantly, the code in the main file should ensure that the page counter
(as well as other status parameters which are stored in the |.aux| files)
takes the same value after the conditional processing.
Otherwise the page numbers may take divergent values
depending on which part is compiled.

For example, a title page could be declared by:
%
\begin{center}
\begin{tabular}{l}
|\ifchilddoc\||else|\\
|\addtocounter{page}{-1}|\\
\textit{code for title page}\\
|\newpage|\\
|\||fi|
\end{tabular}
\end{center}
%
A banner page for the child documents can be generated by:
%
\begin{center}
\begin{tabular}{l}
|\ifchilddoc|\\
|\addtocounter{page}{-1}|\\
\textit{code for banner page}\\
|\newpage|\\
|\||fi|
\end{tabular}
\end{center}
%
Here one could write a message such as:
\begin{center}
|This is the part \childdocname{} of \childdocjob{}.|
\end{center}

%%%%%%%%%%%%%%%%%%%%%%%%%%%%%%%%%%%%%%%%%%%%%%%%%%%%%%%%%%%%%%%%%%%%%%%%%%%%%%%%
\subsection{Flags}
\label{sec:flags}

The package makes it easy to generate different versions
of the main or child documents.
To this end compilation flags can be defined
and assigned different default values.
They will be particularly useful in conjunction
with the forwarding mechanism described in \secref{sec:forward}.

For example, it may be useful to have a flag |\version|
which can be set to |draft| or |final|.
The document source will contain some conditional code
depending on the value of |\version|.
Suppose further, the flag should default to |final| for the main file
and to |draft| for child files
which is a natural assignment for editing the document.
This is achieved by placing the following code
in the preamble of the main document
(below the |\childdocmain| directive):
%
\begin{center}
\begin{tabular}{l}
|\ifchilddoc|\\
|\providecommand{\version}{draft}|\\
|\||else|\\
|\providecommand{\version}{final}|\\
|\||fi|
\end{tabular}
\end{center}
%
The definition by |\providecommand| makes sure
that previous definitions are not overwritten.
Further statements |\providecommand{\version}{...}|
can thus be added before the above code to override it.

For the main file, one might add a line
(between |\childdocmain| and the above block)
%
\begin{center}
|%\ifchilddoc\||else\providecommand{\version}{draft}\||fi|
\end{center}
%
which can be uncommented to produce a draft version.
Likewise one can add a line to the very top of a child file
(above the |\childdocof{|\textit{main}|}| directive)
%
\begin{center}
|%\providecommand{\version}{final}|
\end{center}
%
which can be uncommented to produce the final version of this child document.

%%%%%%%%%%%%%%%%%%%%%%%%%%%%%%%%%%%%%%%%%%%%%%%%%%%%%%%%%%%%%%%%%%%%%%%%%%%%%%%%
\subsection{Forwarding}
\label{sec:forward}

Different versions of the main or child documents
using compilation flags as described in \secref{sec:flags}
can be (permanently) stored in different files
for convenient compilation, viewing and distribution.
To this end, the package defines a command
to pass on compilation to a different file:

%%%%%%%%%%%%%%%%%%%%%%%%%%%%%%%%%%%%%%%%
\DescribeMacro{\childdocforward}
The command |\childdocforward| redirects processing to
another source file:
%
\begin{center}
\begin{tabular}{l}
|\input{childdoc.def}|\\
|\childdocforward[|\textit{main}|]{|\textit{dest}|}|\\
\end{tabular}
\end{center}
%
The argument \textit{dest} is the destination file
(without extension).
It should be the main file or one of the child files.
Note that further \textsf{childdoc} directives
such as |\childdocof| and |\childdocforward|
in the indicated file will be processed in this form.
The optional argument \textit{main}
passes on directly to the main file \textit{main}
while pretending to compile the child \textit{dest}.
This form behaves as if \textit{dest}
issues |\childdocof{|\textit{main}|}| right away,
and no further \textsf{childdoc} directives will be processed.

%%%%%%%%%%%%%%%%%%%%%%%%%%%%%%%%%%%%%%%%
\DescribeMacro{\...prefix}
In the alternative form |\childdocforwardprefix|,
%
\begin{center}
\begin{tabular}{l}
|\input{childdoc.def}|\\
|\childdocforwardprefix[|\textit{main}|]{|\textit{prefix}|}{|\textit{dest}|}|
\end{tabular}
\end{center}
%
the destination file is determined by a pattern
depending on the current file:
To make this work, the current file must be called
`{\textit{prefix}\hspace{0.2em}\textit{suffix}}'
with \textit{prefix} matching precisely the argument.
Processing is then passed on to the file
`{\textit{dest}\hspace{0.2em}\textit{suffix}}'.
Surely, the same effect is achieved by
directly specifying the
argument `{\textit{dest}\hspace{0.2em}\textit{suffix}}'
in the first form.
However, that requires to set up a different file
for each child. With the alternative form of the command
all these files can have exactly the same content
which simplifies setting them up and maintaining them.

For example, the following file |draft.tex|
with a compilation flag |\version| as described in \secref{sec:flags}
compiles the main document as a draft:
%
\begin{center}
\begin{tabular}{l}
|\def\version{draft}|\\
|\input{childdoc.def}|\\
|\childdocforward{|\textit{main}|}|
\end{tabular}
\end{center}
%
Likewise, the following files |final|\textit{nn}|.tex|
compile the final version of the child document
|child|\textit{nn}|.tex|:
%
\begin{center}
\begin{tabular}{l}
|\def\version{final}|\\
|\input{childdoc.def}|\\
|\childdocforwardprefix{final}{child}|
\end{tabular}
\end{center}
%

Note that when several versions of a main file and/or of each child file
are to be generated, it may be convenient to set up a |Makefile| or
shell script to automatise the process.

%%%%%%%%%%%%%%%%%%%%%%%%%%%%%%%%%%%%%%%%%%%%%%%%%%%%%%%%%%%%%%%%%%%%%%%%%%%%%%%%
\subsection{Command Line Processing}
\label{sec:commandline}

The effect of redirection files can also be achieved by invoking
the \LaTeX{} compiler with a more elaborate command line.
Most conveniently this should be done as part
of a shell script or a |Makefile|.

When using \textsf{childdoc} in the main file, the following
command lines effectively perform a redirection
(note that depending on the shell being used,
backslashes may have to be doubled: `|\|' $\to$ `|\\|'):
%
\begin{center}
|... -jobname "|\textit{target}|" |\\|"|[\textit{flags}]%
|\input{childdoc.def}\childdocforward[|\textit{main}|]{|\textit{dest}|}"|
\end{center}
%
Here \textit{target} is the name of the output file,
\textit{main} is the name of the main file
and \textit{dest} is the name of the main or child file to be processed
(all filenames without extensions).
The optional argument \textit{main} can be omitted
if \textit{main} matches \textit{dest}.
Optionally, compilation \textit{flags} can be defined via |\def| commands.
This command line makes the \TeX{} engine believe
it is compiling the file \textit{target}
whose content is specified as the latter parameter.
The provided code then forwards the processing to
\textit{main} or \textit{dest} as described in \secref{sec:forward}.

%%%%%%%%%%%%%%%%%%%%%%%%%%%%%%%%%%%%%%%%%%%%%%%%%%%%%%%%%%%%%%%%%%%%%%%%%%%%%%%%
\subsection{Include by Input}
\label{sec:input}

Including child documents by |\include| has some restrictions by design.
Most notably, the content of a child document always occupies
its own set of pages; pages cannot be shared between child documents.
Usually, this behaviour makes perfect sense
because each child document contain an essential part of the document.
However, in some situations it may be desirable to compose
a document from a collection of parts
without having mandatory page breaks between then.
For this case, the package
provides a mechanism to include parts
by |\input| which can also be processed individually.
However, by construction this mechanism
requires manual handling of the content to be output.

%%%%%%%%%%%%%%%%%%%%%%%%%%%%%%%%%%%%%%%%
\DescribeMacro{\ifchilddocmanual}
The main file should be prepared as usual, see \secref{sec:include}.
However, the document body must make a distinction
between processing of an individual part and of the main document, e.g.:
%
\begin{center}
\begin{tabular}{l}
|\ifchilddocmanual|\\
|\input{\childdocname}|\\
|\||else|\\
\textit{document body with }|\input{|\textit{part}|}|\\
|\||fi|
\end{tabular}
\end{center}
%
The conditional |\ifchilddocmanual| is true whenever
a part to be included by |\input| is being compiled,
and the name of the part is stored in |\childdocname|.

%%%%%%%%%%%%%%%%%%%%%%%%%%%%%%%%%%%%%%%%
\DescribeMacro{\childdocby}
Each part to be included by |\input| should start with:
%
\begin{center}
\begin{tabular}{l}
|\input{childdoc.def}|\\
|\childdocby{|\textit{main}|}|\\
\end{tabular}
\end{center}
%
The directive |\childdocby| is similar to |\childdocof|
described in \secref{sec:include},
but the subsequent selection of content must be done manually.
To that end, both |\ifchilddoc| and |\ifchilddocmanual|
will be true upon processing of a part,
and the name of the part is stored in |\childdocname|.
Note that |\jobname| will be set to the filename of the current part
so that each part receives an individual |.aux| file
that does not interfere with the |.aux| file(s) of the main document.
This behaviour can be altered by the alternative form
|\childdocby[*]{|\textit{main}|}| (with a non-empty optional argument)
which uses the |.aux| file of the main document
by setting |\jobname| to \textit{main}.

%%%%%%%%%%%%%%%%%%%%%%%%%%%%%%%%%%%%%%%%%%%%%%%%%%%%%%%%%%%%%%%%%%%%%%%%%%%%%%%%
\subsection{Driver Development}
\label{sec:driver}

The \textsf{childdoc} mechanism can also be use for the development
of definition files such as \LaTeX{} styles or classes.
This case differs from the above setup with multiple parts
included by |\include| in that no |\includeonly| should be invoked.
This can be achieved by starting the include file
(before |\ProvidesPackage|) with:
%
\begin{center}
\begin{tabular}{l}
|\input{childdoc.def}|\\
|\childdocforward{|\textit{main}|}|\\
\end{tabular}
\end{center}
%
or alternatively with:
%
\begin{center}
\begin{tabular}{l}
|\input{childdoc.def}|\\
|\childdocby{|\textit{main}|}|\\
\end{tabular}
\end{center}
%
Both forms have slightly different effects as described above.
The main file is prepared as usual, see \secref{sec:include}.

%%%%%%%%%%%%%%%%%%%%%%%%%%%%%%%%%%%%%%%%%%%%%%%%%%%%%%%%%%%%%%%%%%%%%%%%%%%%%%%%
\subsection{Legacy Detection}
\label{sec:detection}

The directive |\childdocmain| in the main file can detect
whether the complete document or merely a child is to be compiled
even without using the directive |\childdocof|.
This method is deprecated because it is less robust
and there is no compelling reason to use it;
it is merely provided for backward compatibility
and it may be removed in future versions.

If the detection mechanism is to be used,
it is mandatory to correctly specify
the filename of the main file as the argument of |\childdocmain|:
%
\begin{center}
\begin{tabular}{l}
|\input{childdoc.def}|\\
|\childdocmain{|\textit{main}|}|\\
\end{tabular}
\end{center}
%
If |\jobname| does not match the argument \textit{main} of |\childdocmain|,
it is assumed that |\jobname| points to the child file to be compiled.
When using |\childdocmain| with the main file specified as argument,
it suffices to start a child file
with just |\input{|\textit{main}|}|
without loading of the package and using |\childdocof|.
If instead all processing is done
with the appropriate \textsf{childdoc} directives,
the argument of \textit{main} of |\childdocmain| can be empty.

An alternative version of the command line processing described
in \secref{sec:commandline} using the detection mechanism reads:
%
\begin{center}
|... -jobname "|\textit{target}|" "|[\textit{flags}]%
[|\def\jobname{|\textit{dest}|}|]|\input{|\textit{main}|}"|
\end{center}

%%%%%%%%%%%%%%%%%%%%%%%%%%%%%%%%%%%%%%%%%%%%%%%%%%%%%%%%%%%%%%%%%%%%%%%%%%%%%%%%
\subsection{Manual Code}
\label{sec:manual}

In case one cannot be certain whether the definitions file |childdoc.def|
is installed on the target \TeX{} distribution
and one prefers not to ship it,
it is conceivable to paste a few relevant commands into the sources.

To that end, drop all statements |\input{childdoc.def}|
and perform the replacements as outlined below.
Instead of |\childdocmain{|\textit{main}|}| add the following code
to the top of the main file:
%
\begin{center}
\begin{tabular}{l}
|\||ifdefined\childdocname\endinput\||fi\newif\ifchilddoc|\\
|\edef\childdocname{\scantokens\expandafter{\jobname\noexpand}}|\\
|\def\childdocmain{|\textit{main}|}\||ifx\childdocmain\childdocname\||else|\\
|\childdoctrue\includeonly{\childdocname}\let\jobname\childdocmain\||fi|\\
\end{tabular}
\end{center}
%
Instead of |\childdocof{|\textit{main}|}| just include the main file
at the top of each child file:
%
\begin{center}
|\input{|\textit{main}|}|
\end{center}
%
A simple redirection |\childdocforward{|\textit{dest}|}| is achieved by:
%
\begin{center}
|\def\jobname{|\textit{dest}|}\input{\jobname}|
\end{center}
%
The redirection with prefix
|\childdocforwardprefix[|\textit{prefix}|]{|\textit{dest}|}|
is accomplished by:
%
\begin{center}
\begin{tabular}{l}
|{\edef\jobname{\scantokens\expandafter{\jobname\noexpand}}|\\
|\def\redirectjob |\textit{prefix}|#1~~~{\gdef\jobname{|\textit{dest}|#1}}|\\
|\expandafter\redirectjob\jobname~~~}\input{\jobname}|
\end{tabular}
\end{center}

In an alternative approach,
child documents can be compiled by a specific command line
without additional code or specific definitions:
%
\begin{center}
|... -jobname "|\textit{target}|" "|[\textit{flags}]%
|\includeonly{|\textit{dest}|}\input{|\textit{main}|}"|
\end{center}
%

%%%%%%%%%%%%%%%%%%%%%%%%%%%%%%%%%%%%%%%%%%%%%%%%%%%%%%%%%%%%%%%%%%%%%%%%%%%%%%%%
%%%%%%%%%%%%%%%%%%%%%%%%%%%%%%%%%%%%%%%%%%%%%%%%%%%%%%%%%%%%%%%%%%%%%%%%%%%%%%%%
\section{Information}

%%%%%%%%%%%%%%%%%%%%%%%%%%%%%%%%%%%%%%%%%%%%%%%%%%%%%%%%%%%%%%%%%%%%%%%%%%%%%%%%
\subsection{Copyright}

Copyright \copyright{} 2017--2018 Niklas Beisert

This work may be distributed and/or modified under the
conditions of the \LaTeX{} Project Public License, either version 1.3
of this license or (at your option) any later version.
The latest version of this license is in
  \url{http://www.latex-project.org/lppl.txt}
and version 1.3 or later is part of all distributions of \LaTeX{}
version 2005/12/01 or later.

This work has the LPPL maintenance status `maintained'.

The Current Maintainer of this work is Niklas Beisert.

This work consists of the files |README.txt|, |childdoc.ins| and |childdoc.dtx|
as well as the derived files |childdoc.def|, |cdocsamp.tex|
with |cdocsch1.tex|, |cdocsch2.tex|, |cdocspt3.tex|, |cdocspt4.tex|,
|cdocsdrf.tex|, |cdocsfn1.tex|, |cdocsfn2.tex|
as well as |childdoc.pdf|.

%%%%%%%%%%%%%%%%%%%%%%%%%%%%%%%%%%%%%%%%%%%%%%%%%%%%%%%%%%%%%%%%%%%%%%%%%%%%%%%%
\subsection{Files and Installation}

The package consists of the files:
%
\begin{center}
\begin{tabular}{ll}
    |README.txt|   & readme file \\
    |childdoc.ins| & installation file \\
    |childdoc.dtx| & source file \\
    |childdoc.def| & definition file \\
    |cdocsamp.tex| & sample main file \\
    |cdocsch1.tex| & sample include file \\
    |cdocsch2.tex| & sample include file \\
    |cdocspt3.tex| & sample part file \\
    |cdocspt4.tex| & sample part file \\
    |cdocsdrf.tex| & sample redirection file \\
    |cdocsfn1.tex| & sample redirection file \\
    |cdocsfn2.tex| & sample redirection file \\
    |childdoc.pdf| & manual
\end{tabular}
\end{center}
%
The distribution consists of the files
|README.txt|, |childdoc.ins| and |childdoc.dtx|.
%
\begin{itemize}
\item
Run (pdf)\LaTeX{} on |childdoc.dtx|
to compile the manual |childdoc.pdf| (this file).
\item
Run \LaTeX{} on |childdoc.ins| to create the definitions file |childdoc.def|
and the sample |cdocsamp.tex| with include files
|cdocsch1.tex|, |cdocsch2.tex|, |cdocspt3.tex|, |cdocspt4.tex|,
|cdocsdrf.tex|, |cdocsfn1.tex|, |cdocsfn2.tex|.
Then copy the file |childdoc.def| to an appropriate directory of your \LaTeX{}
distribution, e.g.\ \textit{texmf-root}|/tex/latex/childdoc|.
\end{itemize}

%%%%%%%%%%%%%%%%%%%%%%%%%%%%%%%%%%%%%%%%%%%%%%%%%%%%%%%%%%%%%%%%%%%%%%%%%%%%%%%%
\subsection{Related CTAN Packages}

There are several other packages which offer a similar functionality:
%
\begin{itemize}
\item
The packages
\href{http://ctan.org/pkg/docmute}{\textsf{docmute}},
\href{http://ctan.org/pkg/includex}{\textsf{includex}} and
\href{http://ctan.org/pkg/standalone}{\textsf{standalone}}
provide commands to include only the document body of
a child file thus allowing both files to be compiled individually.
\item
The packages \href{http://ctan.org/pkg/subdocs}{\textsf{subdocs}}
and \href{http://ctan.org/pkg/subfiles}{\textsf{subfiles}}
provide structures in which the main and child documents can be
encapsulated and allowing them to be compiled individually.
The inclusion mechanism is different from the conventional |\include|.
\item
The package \href{http://ctan.org/pkg/combine}{\textsf{combine}}
is an elaborate solution to combine several documents into one.
\end{itemize}
%
See also the CTAN topic \href{http://ctan.org/topic/subdocs}{\textsf{subdocs}}
for further related packages.
The present package differs from the above solutions in that
a document structure constructed with the conventional |\include| mechanism
just needs two extra commands at the top of every file
such that all constituent files can be compiled individually.

%%%%%%%%%%%%%%%%%%%%%%%%%%%%%%%%%%%%%%%%%%%%%%%%%%%%%%%%%%%%%%%%%%%%%%%%%%%%%%%%
%\subsection{Feature Suggestions}
%
%The following is a list of features which may be useful for future
%versions of this package:
%%
%\begin{itemize}
%\item
%\ldots
%\end{itemize}

%%%%%%%%%%%%%%%%%%%%%%%%%%%%%%%%%%%%%%%%%%%%%%%%%%%%%%%%%%%%%%%%%%%%%%%%%%%%%%%%
\subsection{Revision History}

%%%%%%%%%%%%%%%%%%%%%%%%%%%%%%%%%%%%%%%%
\paragraph{v2.0:} 2018/12/30

\begin{itemize}
\item
immediate forward processing
\item
added |\childdocby| mechanism
\item
manual restructured
\end{itemize}

%%%%%%%%%%%%%%%%%%%%%%%%%%%%%%%%%%%%%%%%
\paragraph{v1.6:} 2018/01/17

\begin{itemize}
\item
application for development of include files
\item
corrections to manual
\end{itemize}

%%%%%%%%%%%%%%%%%%%%%%%%%%%%%%%%%%%%%%%%
\paragraph{v1.5:} 2017/05/21

\begin{itemize}
\item
more complete structuring introduced
\item
|\childdocof| introduced
\item
|\childdoc| renamed to |\childdocmain|
\item
|\childredirect| renamed to |\childdocforward| and |\childdocforwardprefix|
and functionality expanded
\end{itemize}

%%%%%%%%%%%%%%%%%%%%%%%%%%%%%%%%%%%%%%%%
\paragraph{v1.0:} 2017/04/27

\begin{itemize}
\item
manual and install package
\item
first version published on CTAN
\end{itemize}

%%%%%%%%%%%%%%%%%%%%%%%%%%%%%%%%%%%%%%%%
\paragraph{v0.6:} 2017/04/26

\begin{itemize}
\item
redirection mechanism added
\end{itemize}

%%%%%%%%%%%%%%%%%%%%%%%%%%%%%%%%%%%%%%%%
\paragraph{v0.5:} 2017/04/26

\begin{itemize}
\item
functionality in definition file
\end{itemize}


%%%%%%%%%%%%%%%%%%%%%%%%%%%%%%%%%%%%%%%%%%%%%%%%%%%%%%%%%%%%%%%%%%%%%%%%%%%%%%%%
%%%%%%%%%%%%%%%%%%%%%%%%%%%%%%%%%%%%%%%%%%%%%%%%%%%%%%%%%%%%%%%%%%%%%%%%%%%%%%%%
%%%%%%%%%%%%%%%%%%%%%%%%%%%%%%%%%%%%%%%%%%%%%%%%%%%%%%%%%%%%%%%%%%%%%%%%%%%%%%%%
\appendix

\settowidth\MacroIndent{\rmfamily\scriptsize 000\ }

 \DocInput{childdoc.dtx}

\end{document}
%</driver>
% \fi
%
% %%%%%%%%%%%%%%%%%%%%%%%%%%%%%%%%%%%%%%%%%%%%%%%%%%%%%%%%%%%%%%%%%%%%%%%%%%%%%%
% %%%%%%%%%%%%%%%%%%%%%%%%%%%%%%%%%%%%%%%%%%%%%%%%%%%%%%%%%%%%%%%%%%%%%%%%%%%%%%
% \section{Sample}
%\iffalse
%<*samplemain>
%\fi
%
% The following presents a sample document
% with two chapters, two parts, a title page,
% a compile flag as well as three forwarding files to set the flag.
% It consists of eight |.tex| files:
% \begin{center}
% \begin{tabular}{ll}
% |cdocsamp.tex|&main file\\
% |cdocsch1.tex|&include file for chapter 1\\
% |cdocsch2.tex|&include file for chapter 2\\
% |cdocspt3.tex|&include file for part 3\\
% |cdocspt4.tex|&include file for part 4\\
% |cdocsdrf.tex|&forwarding file for main file in draft mode\\
% |cdocsfi1.tex|&forwarding file for final version of chapter 1\\
% |cdocsfi2.tex|&forwarding file for final version of chapter 2\\
% \end{tabular}
% \end{center}
% Each of the eight files can be compiled directly by the \LaTeX{} compiler.
%
% %%%%%%%%%%%%%%%%%%%%%%%%%%%%%%%%%%%%%%
% \paragraph{Main File.}
%
% The main file is called |cdocsamp.tex|.
%
% Load the \textsf{childdoc} definitions and
% declare the filename for the main document:
%    \begin{macrocode}
\input{childdoc.def}
\childdocmain{}
%    \end{macrocode}

% Optional override for |\version| flag:
%    \begin{macrocode}
%%\ifchilddoc\else\providecommand{\version}{draft}\fi
%    \end{macrocode}

% Define the default values for the |\version| flag
% (|final| for the main file and |draft| for childs):
%    \begin{macrocode}
\ifchilddoc
\providecommand{\version}{draft}
\else
\providecommand{\version}{final}
\fi
%    \end{macrocode}

% Load the standard document class:
%    \begin{macrocode}
\documentclass[12pt]{article}
%    \end{macrocode}

% Start the document body:
%    \begin{macrocode}
\begin{document}
%    \end{macrocode}

% Declare a title page.
% Print title, part of document being processed and version flag:
%    \begin{macrocode}
\addtocounter{page}{-1}
\begin{center}
{\LARGE\bfseries{}childdoc example\par}
\vspace{1cm}
\ifchilddoc
\ifchilddocmanual part\else chapter\fi:
`\childdocname' of `\childdocjob'\par
\else
main document: `\childdocjob'\par
\fi
version: \version\par
\end{center}
\newpage
%    \end{macrocode}

% Manually include selected file,
% otherwise process as usual:
%    \begin{macrocode}
\ifchilddocmanual
\section*{part `\childdocname'}
\input{\childdocname}
\else
%    \end{macrocode}

% Include the two chapters:
%    \begin{macrocode}
\include{cdocsch1}
\include{cdocsch2}
%    \end{macrocode}

% Include the two parts unless only chapters should be displayed:
%    \begin{macrocode}
\ifchilddoc\else
\section{part three}
\input{cdocspt3}
\section{part four}
\input{cdocspt4}
\fi
%    \end{macrocode}

% Process as usual until here:
%    \begin{macrocode}
\fi
%    \end{macrocode}

% End of document body:
%    \begin{macrocode}
\end{document}
%    \end{macrocode}
%\iffalse
%</samplemain>
%\fi
%
% %%%%%%%%%%%%%%%%%%%%%%%%%%%%%%%%%%%%%%
% \paragraph{Chapter Include Files.}
%
% The include files are called |cdocsch1.tex| and |cdocsch2.tex|.
%
%\iffalse
%<*samplechap1|samplechap2>
%\fi

% Optional override for |\version| flag:
%    \begin{macrocode}
%%\providecommand{\version}{final}
%    \end{macrocode}

% Include the main document:
%    \begin{macrocode}
\input{childdoc.def}
\childdocof{cdocsamp}
%    \end{macrocode}

%\iffalse
%</samplechap1|samplechap2>
%\fi
%
%\iffalse
%<*samplechap1>
%\fi
% Some text for chapter 1:
%    \begin{macrocode}
\section{one}
some text in chapter one
%    \end{macrocode}

%\iffalse
%</samplechap1>
%\fi
% Some text for chapter 2:
%\iffalse
%<*samplechap2>
%\fi
%    \begin{macrocode}
\section{two}
more text in chapter two
%    \end{macrocode}

%\iffalse
%</samplechap2>
%\fi
%
% %%%%%%%%%%%%%%%%%%%%%%%%%%%%%%%%%%%%%%
% \paragraph{Part Include Files.}
%
% The include files are called |cdocspt3.tex| and |cdocspt4.tex|.
%
%\iffalse
%<*samplepart3|samplepart4>
%\fi

% Optional override for |\version| flag:
%    \begin{macrocode}
%%\providecommand{\version}{final}
%    \end{macrocode}

% Include the main document:
%    \begin{macrocode}
\input{childdoc.def}
\childdocby{cdocsamp}
%    \end{macrocode}

%\iffalse
%</samplepart3|samplepart4>
%\fi
%
%\iffalse
%<*samplepart3>
%\fi
% Some text for part 3:
%    \begin{macrocode}
some text in part three
%    \end{macrocode}

%\iffalse
%</samplepart3>
%\fi
% Some text for part 4:
%\iffalse
%<*samplepart4>
%\fi
%    \begin{macrocode}
more text in part four
%    \end{macrocode}

%\iffalse
%</samplepart4>
%\fi
%
% %%%%%%%%%%%%%%%%%%%%%%%%%%%%%%%%%%%%%%
% \paragraph{Forwarding for a Complete Draft.}
%
% The following forwarding file |cdocsdrf.tex|
% compiles the main document in draft mode:
%\iffalse
%<*sampledraft>
%\fi
%    \begin{macrocode}
\def\version{draft}
\input{childdoc.def}
\childdocforward{cdocsamp}
%    \end{macrocode}

%\iffalse
%</sampledraft>
%\fi
%
% %%%%%%%%%%%%%%%%%%%%%%%%%%%%%%%%%%%%%%
% \paragraph{Forwarding for Final Version of the Chapters.}
%
% The following forwarding files |cdocsfn1.tex| and |cdocsfn2.tex|
% (with identical content)
% compile the final versions of the child documents
% |cdocsch1.tex| and |cdocsch2.tex|, respectively:
%\iffalse
%<*samplefinal>
%\fi
%    \begin{macrocode}
\def\version{final}
\input{childdoc.def}
\childdocforwardprefix[cdocsamp]{cdocsfn}{cdocsch}
%    \end{macrocode}

%\iffalse
%</samplefinal>
%\fi
%
% %%%%%%%%%%%%%%%%%%%%%%%%%%%%%%%%%%%%%%
% \paragraph{Command Line Processing.}
%
% The following three command lines generate the output files
% |cdocscld|, |cdocscl1| and |cdocscl2|
% which should be identical to
% |cdocsdrf|, |cdocsch1| and |cdocsfn2|, respectively:
% \begin{center}
% \begin{tabular}{l}
% |latex -jobname cdocscld \|\\
% |  "\def\version{draft}\input{childdoc.def}\childdocforward{cdocsamp}"|\\
% |latex -jobname cdocscl1 \|\\
% |  "\input{childdoc.def}\childdocforward[cdocsamp]{cdocsch1}"|\\
% |latex -jobname cdocscl2 \|\\
% |  "\def\version{final}\input{childdoc.def}\childdocforward{cdocsch2}"|
% \end{tabular}
% \end{center}
% Note that the trailing backslash on each first line
% merely continues the input to the second line
% (for convenient cut ant paste).
% Furthermore, the command |latex| can be replaced by any
% of its alternative versions such as |pdflatex|.
%
% %%%%%%%%%%%%%%%%%%%%%%%%%%%%%%%%%%%%%%%%%%%%%%%%%%%%%%%%%%%%%%%%%%%%%%%%%%%%%%
% %%%%%%%%%%%%%%%%%%%%%%%%%%%%%%%%%%%%%%%%%%%%%%%%%%%%%%%%%%%%%%%%%%%%%%%%%%%%%%
% \section{Implementation}
%\iffalse
%<*package>
%\fi
%
% This section describes the definitions file |childdoc.def|.

% The definitions cannot be loaded using |\usepackage| or |\RequirePackage|
% which has a mechanism to prevent loading a style file more than once.
% When loading the definitions by means of |\input|
% multiple instances have to be prevented manually:
%\iffalse
%This code needs to be before the `\ProvidesFile' directive
%which is defined at the beginning of this file.
%Therefore it is also placed there and commented out here.
%</package>
%<*discard>
%\fi
%    \begin{macrocode}
\ifdefined\childdocmain\endinput\fi
%    \end{macrocode}
%\iffalse
%</discard>
%<*package>
%\fi
%
% \macro{\ifchilddoc}
% \macro{\ifchilddocmanual}
% The conditional |\ifchilddoc| tells whether a
% child (true) or main (false) document is being compiled.
% The conditional |\ifchilddocmanual| tells whether
% the |\includeonly| mechanism is used (false) or
% the selection of child files must be performed manually (true).
% The definitions initialise to false:
%    \begin{macrocode}
\newif\ifchilddoc
\newif\ifchilddocmanual
%    \end{macrocode}

% \macro{\childdocname}
% \macro{\childdocjob}
% The macro |\childdocname| stores the name of the main document
% to be compiled. The macro |\childdocjob| stores the name of
% the document on which the \LaTeX{} compiler was originally invoked.
% The content of |\jobname| cannot be compared
% to filenames specified in the source due to different catcodes.
% The following code rescans |\jobname|, stores the result
% in |\childdocname| and saves a copy in |\childdocjob|:
%    \begin{macrocode}
\edef\childdocname{\scantokens\expandafter{\jobname\noexpand}}
\let\childdocjob\childdocname
%    \end{macrocode}

% \macro{\childdocdisable}
% The macro |\childdocdisable| prevents the main file
% from being processed more than once.
% At this stage, the main document command |\childdocmain|
% is assumed to be called once again where it should do nothing.
% Any subsequent call to it should prevent
% a secondary processing of the main document
% It overwrites the forwarding commands
% |\childdocof| and |\childdocforward|
% with empty macros to prevent further inclusions of the main document:
%    \begin{macrocode}
\newcommand{\childdocdisable}
{
  \renewcommand{\childdocmain}[1]{\renewcommand{\childdocmain}[1]{\endinput}}
  \renewcommand{\childdocof}[1]{}
  \renewcommand{\childdocby}[2][]{}
  \renewcommand{\childdocforward}[2][]{}
  \renewcommand{\childdocdisable}{}
}
%    \end{macrocode}

% \macro{\childdocmain}
% The macro |\childdocmain| is to be called at the top of the main file
% with nothing or the main filename (without extension) as argument.
% First, it breaks loops.
% If the argument is not empty and does not match |\childdocname|
% (which is set by the first inclusion of |childdoc.def|),
% |\ifchilddoc| is set to true, |\includeonly| is applied to the child file
% and |\jobname| is set to the main file
% (for proper handling of |.aux| files):
%    \begin{macrocode}
\newcommand{\childdocmain}[1]
{
  \childdocdisable\childdocmain{}
  \if?#1?\else
    \begingroup
      \def\childdoctmp{#1}
      \ifx\childdoctmp\childdocname
        \def\childdoctmp{}
      \else
        \def\childdoctmp
        {
          \childdoctrue
          \includeonly{\childdocname}
          \def\childdocjob{#1}
          \def\jobname{#1}
        }
      \fi
      \expandafter
    \endgroup
    \childdoctmp
  \fi
}
%    \end{macrocode}

% \macro{\childdocof}
% The command |\childdocof| redirects
% compilation to the main file |#1|.
%    \begin{macrocode}
\newcommand{\childdocof}[1]
{
  \childdocdisable
  \childdoctrue
  \includeonly{\childdocname}
  \def\jobname{#1}
  \def\childdocjob{#1}
  \input{#1}
}
%    \end{macrocode}

% \macro{\childdocby}
% The command |\childdocby| ....
%    \begin{macrocode}
\newcommand{\childdocby}[2][]
{
  \childdocdisable
  \childdoctrue
  \childdocmanualtrue
  \if?#1?\else
    \def\jobname{#2}
  \fi
  \def\childdocjob{#2}
  \input{#2}
  \endinput
}
%    \end{macrocode}

% \macro{\childdocforward}
% The command |\childdocforward| redirects
% compilation to the main file or
% (if the optional argument is given) a child file.
% Parameters are set as if the main file
% or a child file starting with |\childdocof| was compiled.
% Then compilation is handed over to the main file:
%    \begin{macrocode}
\newcommand{\childdocforward}[2][]
{
  \begingroup
    \if?#1?
      \def\childdoctmp
      {
        \def\childdocname{#2}
        \def\childdocjob{#2}
        \def\jobname{#2}
        \input{#2}
        \endinput
      }
    \else
      \def\childdoctmp
      {
        \childdocdisable
        \def\childdocname{#2}
        \childdoctrue
        \includeonly{#2}
        \def\childdocjob{#1}
        \def\jobname{#1}
        \input{#1}
        \endinput
      }
    \fi
    \expandafter
  \endgroup
  \childdoctmp
}
%    \end{macrocode}

% \macro{\childdocforwardprefix}
% The command |\childdocforwardprefix| redirects
% compilation to the main or a child file by means of a pattern.
% The prefix |#1| in the current filename is replaced by |#2|
% and the suffix of the current filename is kept
% (it is assumed that the filename does not contain the substring `|~~~|'
% which is used as a delimiter).
% Compilation is handed over to the new file by |\childdocforward|:
%    \begin{macrocode}
\newcommand{\childdocforwardprefix}[3][]
{
  \begingroup
    \def\childdocextract #2##1~~~{\def\childdoctmp{\childdocforward[#1]{#3##1}}}
    \expandafter\childdocextract\childdocname~~~
    \expandafter
  \endgroup
  \childdoctmp
}
%    \end{macrocode}

% \macro{\childdoc}
% The deprecated macro |\childdoc| is a legacy version of |\childdocmain|:
%    \begin{macrocode}
\newcommand{\childdoc}{\childdocmain}
%    \end{macrocode}

% \macro{\childdocredirect}
% The deprecated macro |\childdocredirect| is a legacy version
% of |\childdocforward| and |\childdocforwardprefix|:
%    \begin{macrocode}
\newcommand{\childdocredirect}[2][]
{
  \begingroup
    \if?#1?
      \def\childdoctmp{\childdocforward{#2}}
    \else
      \def\childdoctmp{\childdocforwardprefix{#1}{#2}}
    \fi
    \expandafter
  \endgroup
  \childdoctmp
}
%    \end{macrocode}

%\iffalse
%</package>
%\fi
%
\endinput
\childdocforward[|\textit{main}|]{|\textit{dest}|}"|
\end{center}
%
Here \textit{target} is the name of the output file,
\textit{main} is the name of the main file
and \textit{dest} is the name of the main or child file to be processed
(all filenames without extensions).
The optional argument \textit{main} can be omitted
if \textit{main} matches \textit{dest}.
Optionally, compilation \textit{flags} can be defined via |\def| commands.
This command line makes the \TeX{} engine believe
it is compiling the file \textit{target}
whose content is specified as the latter parameter.
The provided code then forwards the processing to
\textit{main} or \textit{dest} as described in \secref{sec:forward}.

%%%%%%%%%%%%%%%%%%%%%%%%%%%%%%%%%%%%%%%%%%%%%%%%%%%%%%%%%%%%%%%%%%%%%%%%%%%%%%%%
\subsection{Include by Input}
\label{sec:input}

Including child documents by |\include| has some restrictions by design.
Most notably, the content of a child document always occupies
its own set of pages; pages cannot be shared between child documents.
Usually, this behaviour makes perfect sense
because each child document contain an essential part of the document.
However, in some situations it may be desirable to compose
a document from a collection of parts
without having mandatory page breaks between then.
For this case, the package
provides a mechanism to include parts
by |\input| which can also be processed individually.
However, by construction this mechanism
requires manual handling of the content to be output.

%%%%%%%%%%%%%%%%%%%%%%%%%%%%%%%%%%%%%%%%
\DescribeMacro{\ifchilddocmanual}
The main file should be prepared as usual, see \secref{sec:include}.
However, the document body must make a distinction
between processing of an individual part and of the main document, e.g.:
%
\begin{center}
\begin{tabular}{l}
|\ifchilddocmanual|\\
|\input{\childdocname}|\\
|\||else|\\
\textit{document body with }|\input{|\textit{part}|}|\\
|\||fi|
\end{tabular}
\end{center}
%
The conditional |\ifchilddocmanual| is true whenever
a part to be included by |\input| is being compiled,
and the name of the part is stored in |\childdocname|.

%%%%%%%%%%%%%%%%%%%%%%%%%%%%%%%%%%%%%%%%
\DescribeMacro{\childdocby}
Each part to be included by |\input| should start with:
%
\begin{center}
\begin{tabular}{l}
|% \iffalse
%
% childdoc.dtx Copyright (C) 2017-2018 Niklas Beisert
%
% This work may be distributed and/or modified under the
% conditions of the LaTeX Project Public License, either version 1.3
% of this license or (at your option) any later version.
% The latest version of this license is in
%   http://www.latex-project.org/lppl.txt
% and version 1.3 or later is part of all distributions of LaTeX
% version 2005/12/01 or later.
%
% This work has the LPPL maintenance status `maintained'.
%
% The Current Maintainer of this work is Niklas Beisert.
%
% This work consists of the files childdoc.dtx and childdoc.ins
% and the derived files childdoc.def and cdocsamp.tex with
% cdocsch1.tex, cdocsch2.tex, cdocsdrf.tex, cdocsfn1.tex, cdocsfn2.tex.
%
%<package>\ifdefined\childdocmain\endinput\fi
%<package>\ProvidesFile{childdoc.def}[2018/12/30 v2.0 child document driver]
%<samplemain>\ProvidesFile{cdocsamp.tex}[2018/12/30 v2.0 sample for childdoc]
%<*driver>
%\ProvidesFile{childdoc.drv}[2018/12/30 v2.0 childdoc reference manual file]
\PassOptionsToClass{10pt,a4paper}{article}
\documentclass{ltxdoc}

\usepackage[margin=35mm]{geometry}
\usepackage{hyperref}
\usepackage{hyperxmp}
\usepackage[usenames]{color}

\hypersetup{colorlinks=true}
\hypersetup{pdfstartview=FitH}
\hypersetup{pdfpagemode=UseNone}
\hypersetup{pdfsource={}}
\hypersetup{pdflang={en-UK}}
\hypersetup{pdfcopyright={Copyright 2017-2018 Niklas Beisert.
  This work may be distributed and/or modified under the
  conditions of the LaTeX Project Public License, either version 1.3
  of this license or (at your option) any later version.}}
\hypersetup{pdflicenseurl={http://www.latex-project.org/lppl.txt}}
\hypersetup{pdfcontactaddress={ETH Zurich, ITP, HIT K,
  Wolfgang-Pauli-Strasse 27}}
\hypersetup{pdfcontactpostcode={8093}}
\hypersetup{pdfcontactcity={Zurich}}
\hypersetup{pdfcontactcountry={Switzerland}}
\hypersetup{pdfcontactemail={nbeisert@itp.phys.ethz.ch}}
\hypersetup{pdfcontacturl={http://people.phys.ethz.ch/\xmptilde nbeisert/}}

\newcommand{\secref}[1]{\hyperref[#1]{section \ref*{#1}}}

\parskip1ex
\parindent0pt
\let\olditemize\itemize
\def\itemize{\olditemize\parskip0pt}

\begin{document}

\title{The \textsf{childdoc} Package}
\hypersetup{pdftitle={The childdoc Package}}
\author{Niklas Beisert\\[2ex]
  Institut f\"ur Theoretische Physik\\
  Eidgen\"ossische Technische Hochschule Z\"urich\\
  Wolfgang-Pauli-Strasse 27, 8093 Z\"urich, Switzerland\\[1ex]
  \href{mailto:nbeisert@itp.phys.ethz.ch}
  {\texttt{nbeisert@itp.phys.ethz.ch}}}
\hypersetup{pdfauthor={Niklas Beisert}}
\hypersetup{pdfsubject={Manual for the LaTeX2e Package childdoc}}
\date{30 December 2018, \textsf{v2.0}}
\maketitle

\begin{abstract}\noindent
\textsf{childdoc} is a \LaTeXe{} package
that enables the direct compilation
of document sections included by |\include|
to individual files.
\end{abstract}

\begingroup
\parskip0ex
\tableofcontents
\endgroup

%%%%%%%%%%%%%%%%%%%%%%%%%%%%%%%%%%%%%%%%%%%%%%%%%%%%%%%%%%%%%%%%%%%%%%%%%%%%%%%%
%%%%%%%%%%%%%%%%%%%%%%%%%%%%%%%%%%%%%%%%%%%%%%%%%%%%%%%%%%%%%%%%%%%%%%%%%%%%%%%%
\section{Introduction}

\LaTeX{} provides a mechanism to structure a large document (such as a book)
into a main file and several child files (containing the chapters)
using the |\include| command.
This mechanism is beneficial for documents
which span hundreds of pages in order to
make the source file(s) more manageable.
Moreover, compilation can be restricted to
selected child files by means of the |\includeonly| command.
The latter feature can be used to reduce the compilation time while editing
(this was significantly more useful in the earlier days of \LaTeX{})
or to generate a smaller document which is easier to navigate.
Another application of |\includeonly| is to generate
documents consisting of selected parts of the complete document.

However, there are a few drawbacks of the plain |\include| mechanism:
\begin{itemize}
\item
The child files cannot be compiled on their own,
they can only be compiled via the main file.
A naive editing environment
(such as a text editor with an option
to have the current file processed by \LaTeX)
may require one to switch to the main file before compiling;
attempting to compile the child file produces errors.
\item
The main file must be modified (each time)
to adjust the |\includeonly| command
to the present needs. This easily leaves the main file in a messy state.
\item
The generated document will always carry the filename
of the main document. This is inconvenient if
several child files are to be compiled and
to be kept for distribution.
\end{itemize}

The present package provides a simple interface
to make child files individually compilable by \LaTeX{}.
Compiling a child file then has the same effect as compiling
the main file with an |\includeonly| command
to select the appropriate child.
Moreover the generated document will carry the name of the child
rather than the main file.
This resolves all three above issues.

This feature is meant to make the editing of books,
thesis documents and lecture notes somewhat more convenient.
However, the package can also be used efficiently for
composing a series of documents (such as exercise sheets)
which are typically distributed individually.
It then assists the author in generating the individual documents
(potentially in different versions)
as well as a document containing the collected series.
Another application is in developing style files
or other kinds of included material
where compilation of the style file could redirect
to a sample or test file.

%%%%%%%%%%%%%%%%%%%%%%%%%%%%%%%%%%%%%%%%%%%%%%%%%%%%%%%%%%%%%%%%%%%%%%%%%%%%%%%%
%%%%%%%%%%%%%%%%%%%%%%%%%%%%%%%%%%%%%%%%%%%%%%%%%%%%%%%%%%%%%%%%%%%%%%%%%%%%%%%%
\section{Usage}

First of all, the package \textsf{childdoc} is \emph{not} a standard
\LaTeXe{} |.sty| style file! Therefore it needs to be invoked in
a non-standard way.

%%%%%%%%%%%%%%%%%%%%%%%%%%%%%%%%%%%%%%%%%%%%%%%%%%%%%%%%%%%%%%%%%%%%%%%%%%%%%%%%
\subsection{Included Files}
\label{sec:include}

%%%%%%%%%%%%%%%%%%%%%%%%%%%%%%%%%%%%%%%%
\DescribeMacro{\childdocmain}
To use the package, add the commands
\begin{center}
\begin{tabular}{l}
|\input{childdoc.def}|\\
|\childdocmain{}|\\
\end{tabular}
\end{center}
at the very top of the main \LaTeX{} file,
in particular \emph{before} the |\documentclass| statement!
The argument of |\childdocmain| should be left empty
(but it must be present).

%%%%%%%%%%%%%%%%%%%%%%%%%%%%%%%%%%%%%%%%
\DescribeMacro{\childdocof}
Furthermore, add the commands
\begin{center}
\begin{tabular}{l}
|\input{childdoc.def}|\\
|\childdocof{|\textit{main}|}|\\
\end{tabular}
\end{center}
at the top of every child file \textit{child}
which is included by |\include{|\textit{child}|}|
from within the main file
(or at least for those files to be compiled individually).
The argument \textit{main} must be the filename of the main file.

There are a couple of
considerations in setting up the main and child documents:

%%%%%%%%%%%%%%%%%%%%%%%%%%%%%%%%%%%%%%%%
\paragraph{Restrictions.}

Please note the following restrictions:
\begin{itemize}
\item
|\childdocmain| must be called with one argument \textit{main}
to ensure compatibility with earlier version of the package.
It must either be empty (|\childdocmain{}|)
or precisely match the filename of the main file in which it is specified.
See \secref{sec:detection} for further information.
\item
The filename \textit{main} must be specified without the |.tex| extension.
\item
The filename \textit{main} is case sensitive
(even in case-insensitive file systems)
due to internal string comparison.
\item
The argument \textit{main} should be fully expanded, it cannot be a macro.
\item
Subdirectories and special characters should be avoided in filenames.
\item
The command |\childdocmain{|\textit{main}|}| must be followed by a whitespace.
It should not be followed immediately by another command
or by a comment mark `|%|'.
This is because the \TeX{} parser reads the token immediately following
the argument of |\childdocmain| and puts it
at the beginning of every child section;
however, a white\-space is ignored.
\end{itemize}

%%%%%%%%%%%%%%%%%%%%%%%%%%%%%%%%%%%%%%%%
\paragraph{Content of Main File.}

It is advisable to place all content in the child files included by |\include|.
Any output contained in the main file will appear in all child documents
unless suppressed manually;
it cannot be suppressed automatically by the |\includeonly| directive
and thus should normally be avoided.
A method to include some content in the main file
by means of conditional processing is described in \secref{sec:conditional}.

%%%%%%%%%%%%%%%%%%%%%%%%%%%%%%%%%%%%%%%%
\paragraph{Page Numbering.}

When only a part of the document is compiled,
the appropriate numbering of pages
(as well as other status parameters)
is determined from the |.aux| files.
The latter contain information from previous passes.
However this information needs to propagate through
all intermediate child documents.
Therefore the page numbering in child documents may well
be inconsistent until the complete document is compiled at least once.

A useful (if unconventional) way to always ensure a consistent
page numbering is to restart the numbering in each child document
and denote the pages by `\textit{child}|.|\textit{page}'
where \textit{child} represents the chapter/section number of the child file.
This can be achieved by the command
|\numberwithin{page}{|\textit{child}|}|
of the \textsf{amsmath} package
where \textit{child} can be |chapter| or |section|
depending on the chosen structuring.
Alternatively, one can modify the macro |\thepage| appropriately
and reset the counter |page| at the start of each child file.

%%%%%%%%%%%%%%%%%%%%%%%%%%%%%%%%%%%%%%%%%%%%%%%%%%%%%%%%%%%%%%%%%%%%%%%%%%%%%%%%
\subsection{Conditional Processing}
\label{sec:conditional}

The package provides a mechanism to compile different versions
of a document. To customise the versions further some conditional processing
can come in handy to distinguish which version is being compiled.
The package provides two macros to describe the compilation context:

%%%%%%%%%%%%%%%%%%%%%%%%%%%%%%%%%%%%%%%%
\DescribeMacro{\ifchilddoc}
The conditional |\ifchilddoc| distinguishes between the compilation of
child documents and the main document:
%
\begin{center}
|\ifchilddoc |\textit{child-code}| |[|\||else |\textit{main-code}]| \||fi|
\end{center}

%%%%%%%%%%%%%%%%%%%%%%%%%%%%%%%%%%%%%%%%
\DescribeMacro{\childdocname}
\DescribeMacro{\childdocjob}
The macro |\childdocname| contains the filename (without extension)
of the main or child file being processed.
Note that |\childdocjob| will always contain the name of the main file.

%%%%%%%%%%%%%%%%%%%%%%%%%%%%%%%%%%%%%%%%
\paragraph{Title Page.}

Conditional processing can be used to include a title or banner page
in the main document when proper precautions are taken.
Importantly, the code in the main file should ensure that the page counter
(as well as other status parameters which are stored in the |.aux| files)
takes the same value after the conditional processing.
Otherwise the page numbers may take divergent values
depending on which part is compiled.

For example, a title page could be declared by:
%
\begin{center}
\begin{tabular}{l}
|\ifchilddoc\||else|\\
|\addtocounter{page}{-1}|\\
\textit{code for title page}\\
|\newpage|\\
|\||fi|
\end{tabular}
\end{center}
%
A banner page for the child documents can be generated by:
%
\begin{center}
\begin{tabular}{l}
|\ifchilddoc|\\
|\addtocounter{page}{-1}|\\
\textit{code for banner page}\\
|\newpage|\\
|\||fi|
\end{tabular}
\end{center}
%
Here one could write a message such as:
\begin{center}
|This is the part \childdocname{} of \childdocjob{}.|
\end{center}

%%%%%%%%%%%%%%%%%%%%%%%%%%%%%%%%%%%%%%%%%%%%%%%%%%%%%%%%%%%%%%%%%%%%%%%%%%%%%%%%
\subsection{Flags}
\label{sec:flags}

The package makes it easy to generate different versions
of the main or child documents.
To this end compilation flags can be defined
and assigned different default values.
They will be particularly useful in conjunction
with the forwarding mechanism described in \secref{sec:forward}.

For example, it may be useful to have a flag |\version|
which can be set to |draft| or |final|.
The document source will contain some conditional code
depending on the value of |\version|.
Suppose further, the flag should default to |final| for the main file
and to |draft| for child files
which is a natural assignment for editing the document.
This is achieved by placing the following code
in the preamble of the main document
(below the |\childdocmain| directive):
%
\begin{center}
\begin{tabular}{l}
|\ifchilddoc|\\
|\providecommand{\version}{draft}|\\
|\||else|\\
|\providecommand{\version}{final}|\\
|\||fi|
\end{tabular}
\end{center}
%
The definition by |\providecommand| makes sure
that previous definitions are not overwritten.
Further statements |\providecommand{\version}{...}|
can thus be added before the above code to override it.

For the main file, one might add a line
(between |\childdocmain| and the above block)
%
\begin{center}
|%\ifchilddoc\||else\providecommand{\version}{draft}\||fi|
\end{center}
%
which can be uncommented to produce a draft version.
Likewise one can add a line to the very top of a child file
(above the |\childdocof{|\textit{main}|}| directive)
%
\begin{center}
|%\providecommand{\version}{final}|
\end{center}
%
which can be uncommented to produce the final version of this child document.

%%%%%%%%%%%%%%%%%%%%%%%%%%%%%%%%%%%%%%%%%%%%%%%%%%%%%%%%%%%%%%%%%%%%%%%%%%%%%%%%
\subsection{Forwarding}
\label{sec:forward}

Different versions of the main or child documents
using compilation flags as described in \secref{sec:flags}
can be (permanently) stored in different files
for convenient compilation, viewing and distribution.
To this end, the package defines a command
to pass on compilation to a different file:

%%%%%%%%%%%%%%%%%%%%%%%%%%%%%%%%%%%%%%%%
\DescribeMacro{\childdocforward}
The command |\childdocforward| redirects processing to
another source file:
%
\begin{center}
\begin{tabular}{l}
|\input{childdoc.def}|\\
|\childdocforward[|\textit{main}|]{|\textit{dest}|}|\\
\end{tabular}
\end{center}
%
The argument \textit{dest} is the destination file
(without extension).
It should be the main file or one of the child files.
Note that further \textsf{childdoc} directives
such as |\childdocof| and |\childdocforward|
in the indicated file will be processed in this form.
The optional argument \textit{main}
passes on directly to the main file \textit{main}
while pretending to compile the child \textit{dest}.
This form behaves as if \textit{dest}
issues |\childdocof{|\textit{main}|}| right away,
and no further \textsf{childdoc} directives will be processed.

%%%%%%%%%%%%%%%%%%%%%%%%%%%%%%%%%%%%%%%%
\DescribeMacro{\...prefix}
In the alternative form |\childdocforwardprefix|,
%
\begin{center}
\begin{tabular}{l}
|\input{childdoc.def}|\\
|\childdocforwardprefix[|\textit{main}|]{|\textit{prefix}|}{|\textit{dest}|}|
\end{tabular}
\end{center}
%
the destination file is determined by a pattern
depending on the current file:
To make this work, the current file must be called
`{\textit{prefix}\hspace{0.2em}\textit{suffix}}'
with \textit{prefix} matching precisely the argument.
Processing is then passed on to the file
`{\textit{dest}\hspace{0.2em}\textit{suffix}}'.
Surely, the same effect is achieved by
directly specifying the
argument `{\textit{dest}\hspace{0.2em}\textit{suffix}}'
in the first form.
However, that requires to set up a different file
for each child. With the alternative form of the command
all these files can have exactly the same content
which simplifies setting them up and maintaining them.

For example, the following file |draft.tex|
with a compilation flag |\version| as described in \secref{sec:flags}
compiles the main document as a draft:
%
\begin{center}
\begin{tabular}{l}
|\def\version{draft}|\\
|\input{childdoc.def}|\\
|\childdocforward{|\textit{main}|}|
\end{tabular}
\end{center}
%
Likewise, the following files |final|\textit{nn}|.tex|
compile the final version of the child document
|child|\textit{nn}|.tex|:
%
\begin{center}
\begin{tabular}{l}
|\def\version{final}|\\
|\input{childdoc.def}|\\
|\childdocforwardprefix{final}{child}|
\end{tabular}
\end{center}
%

Note that when several versions of a main file and/or of each child file
are to be generated, it may be convenient to set up a |Makefile| or
shell script to automatise the process.

%%%%%%%%%%%%%%%%%%%%%%%%%%%%%%%%%%%%%%%%%%%%%%%%%%%%%%%%%%%%%%%%%%%%%%%%%%%%%%%%
\subsection{Command Line Processing}
\label{sec:commandline}

The effect of redirection files can also be achieved by invoking
the \LaTeX{} compiler with a more elaborate command line.
Most conveniently this should be done as part
of a shell script or a |Makefile|.

When using \textsf{childdoc} in the main file, the following
command lines effectively perform a redirection
(note that depending on the shell being used,
backslashes may have to be doubled: `|\|' $\to$ `|\\|'):
%
\begin{center}
|... -jobname "|\textit{target}|" |\\|"|[\textit{flags}]%
|\input{childdoc.def}\childdocforward[|\textit{main}|]{|\textit{dest}|}"|
\end{center}
%
Here \textit{target} is the name of the output file,
\textit{main} is the name of the main file
and \textit{dest} is the name of the main or child file to be processed
(all filenames without extensions).
The optional argument \textit{main} can be omitted
if \textit{main} matches \textit{dest}.
Optionally, compilation \textit{flags} can be defined via |\def| commands.
This command line makes the \TeX{} engine believe
it is compiling the file \textit{target}
whose content is specified as the latter parameter.
The provided code then forwards the processing to
\textit{main} or \textit{dest} as described in \secref{sec:forward}.

%%%%%%%%%%%%%%%%%%%%%%%%%%%%%%%%%%%%%%%%%%%%%%%%%%%%%%%%%%%%%%%%%%%%%%%%%%%%%%%%
\subsection{Include by Input}
\label{sec:input}

Including child documents by |\include| has some restrictions by design.
Most notably, the content of a child document always occupies
its own set of pages; pages cannot be shared between child documents.
Usually, this behaviour makes perfect sense
because each child document contain an essential part of the document.
However, in some situations it may be desirable to compose
a document from a collection of parts
without having mandatory page breaks between then.
For this case, the package
provides a mechanism to include parts
by |\input| which can also be processed individually.
However, by construction this mechanism
requires manual handling of the content to be output.

%%%%%%%%%%%%%%%%%%%%%%%%%%%%%%%%%%%%%%%%
\DescribeMacro{\ifchilddocmanual}
The main file should be prepared as usual, see \secref{sec:include}.
However, the document body must make a distinction
between processing of an individual part and of the main document, e.g.:
%
\begin{center}
\begin{tabular}{l}
|\ifchilddocmanual|\\
|\input{\childdocname}|\\
|\||else|\\
\textit{document body with }|\input{|\textit{part}|}|\\
|\||fi|
\end{tabular}
\end{center}
%
The conditional |\ifchilddocmanual| is true whenever
a part to be included by |\input| is being compiled,
and the name of the part is stored in |\childdocname|.

%%%%%%%%%%%%%%%%%%%%%%%%%%%%%%%%%%%%%%%%
\DescribeMacro{\childdocby}
Each part to be included by |\input| should start with:
%
\begin{center}
\begin{tabular}{l}
|\input{childdoc.def}|\\
|\childdocby{|\textit{main}|}|\\
\end{tabular}
\end{center}
%
The directive |\childdocby| is similar to |\childdocof|
described in \secref{sec:include},
but the subsequent selection of content must be done manually.
To that end, both |\ifchilddoc| and |\ifchilddocmanual|
will be true upon processing of a part,
and the name of the part is stored in |\childdocname|.
Note that |\jobname| will be set to the filename of the current part
so that each part receives an individual |.aux| file
that does not interfere with the |.aux| file(s) of the main document.
This behaviour can be altered by the alternative form
|\childdocby[*]{|\textit{main}|}| (with a non-empty optional argument)
which uses the |.aux| file of the main document
by setting |\jobname| to \textit{main}.

%%%%%%%%%%%%%%%%%%%%%%%%%%%%%%%%%%%%%%%%%%%%%%%%%%%%%%%%%%%%%%%%%%%%%%%%%%%%%%%%
\subsection{Driver Development}
\label{sec:driver}

The \textsf{childdoc} mechanism can also be use for the development
of definition files such as \LaTeX{} styles or classes.
This case differs from the above setup with multiple parts
included by |\include| in that no |\includeonly| should be invoked.
This can be achieved by starting the include file
(before |\ProvidesPackage|) with:
%
\begin{center}
\begin{tabular}{l}
|\input{childdoc.def}|\\
|\childdocforward{|\textit{main}|}|\\
\end{tabular}
\end{center}
%
or alternatively with:
%
\begin{center}
\begin{tabular}{l}
|\input{childdoc.def}|\\
|\childdocby{|\textit{main}|}|\\
\end{tabular}
\end{center}
%
Both forms have slightly different effects as described above.
The main file is prepared as usual, see \secref{sec:include}.

%%%%%%%%%%%%%%%%%%%%%%%%%%%%%%%%%%%%%%%%%%%%%%%%%%%%%%%%%%%%%%%%%%%%%%%%%%%%%%%%
\subsection{Legacy Detection}
\label{sec:detection}

The directive |\childdocmain| in the main file can detect
whether the complete document or merely a child is to be compiled
even without using the directive |\childdocof|.
This method is deprecated because it is less robust
and there is no compelling reason to use it;
it is merely provided for backward compatibility
and it may be removed in future versions.

If the detection mechanism is to be used,
it is mandatory to correctly specify
the filename of the main file as the argument of |\childdocmain|:
%
\begin{center}
\begin{tabular}{l}
|\input{childdoc.def}|\\
|\childdocmain{|\textit{main}|}|\\
\end{tabular}
\end{center}
%
If |\jobname| does not match the argument \textit{main} of |\childdocmain|,
it is assumed that |\jobname| points to the child file to be compiled.
When using |\childdocmain| with the main file specified as argument,
it suffices to start a child file
with just |\input{|\textit{main}|}|
without loading of the package and using |\childdocof|.
If instead all processing is done
with the appropriate \textsf{childdoc} directives,
the argument of \textit{main} of |\childdocmain| can be empty.

An alternative version of the command line processing described
in \secref{sec:commandline} using the detection mechanism reads:
%
\begin{center}
|... -jobname "|\textit{target}|" "|[\textit{flags}]%
[|\def\jobname{|\textit{dest}|}|]|\input{|\textit{main}|}"|
\end{center}

%%%%%%%%%%%%%%%%%%%%%%%%%%%%%%%%%%%%%%%%%%%%%%%%%%%%%%%%%%%%%%%%%%%%%%%%%%%%%%%%
\subsection{Manual Code}
\label{sec:manual}

In case one cannot be certain whether the definitions file |childdoc.def|
is installed on the target \TeX{} distribution
and one prefers not to ship it,
it is conceivable to paste a few relevant commands into the sources.

To that end, drop all statements |\input{childdoc.def}|
and perform the replacements as outlined below.
Instead of |\childdocmain{|\textit{main}|}| add the following code
to the top of the main file:
%
\begin{center}
\begin{tabular}{l}
|\||ifdefined\childdocname\endinput\||fi\newif\ifchilddoc|\\
|\edef\childdocname{\scantokens\expandafter{\jobname\noexpand}}|\\
|\def\childdocmain{|\textit{main}|}\||ifx\childdocmain\childdocname\||else|\\
|\childdoctrue\includeonly{\childdocname}\let\jobname\childdocmain\||fi|\\
\end{tabular}
\end{center}
%
Instead of |\childdocof{|\textit{main}|}| just include the main file
at the top of each child file:
%
\begin{center}
|\input{|\textit{main}|}|
\end{center}
%
A simple redirection |\childdocforward{|\textit{dest}|}| is achieved by:
%
\begin{center}
|\def\jobname{|\textit{dest}|}\input{\jobname}|
\end{center}
%
The redirection with prefix
|\childdocforwardprefix[|\textit{prefix}|]{|\textit{dest}|}|
is accomplished by:
%
\begin{center}
\begin{tabular}{l}
|{\edef\jobname{\scantokens\expandafter{\jobname\noexpand}}|\\
|\def\redirectjob |\textit{prefix}|#1~~~{\gdef\jobname{|\textit{dest}|#1}}|\\
|\expandafter\redirectjob\jobname~~~}\input{\jobname}|
\end{tabular}
\end{center}

In an alternative approach,
child documents can be compiled by a specific command line
without additional code or specific definitions:
%
\begin{center}
|... -jobname "|\textit{target}|" "|[\textit{flags}]%
|\includeonly{|\textit{dest}|}\input{|\textit{main}|}"|
\end{center}
%

%%%%%%%%%%%%%%%%%%%%%%%%%%%%%%%%%%%%%%%%%%%%%%%%%%%%%%%%%%%%%%%%%%%%%%%%%%%%%%%%
%%%%%%%%%%%%%%%%%%%%%%%%%%%%%%%%%%%%%%%%%%%%%%%%%%%%%%%%%%%%%%%%%%%%%%%%%%%%%%%%
\section{Information}

%%%%%%%%%%%%%%%%%%%%%%%%%%%%%%%%%%%%%%%%%%%%%%%%%%%%%%%%%%%%%%%%%%%%%%%%%%%%%%%%
\subsection{Copyright}

Copyright \copyright{} 2017--2018 Niklas Beisert

This work may be distributed and/or modified under the
conditions of the \LaTeX{} Project Public License, either version 1.3
of this license or (at your option) any later version.
The latest version of this license is in
  \url{http://www.latex-project.org/lppl.txt}
and version 1.3 or later is part of all distributions of \LaTeX{}
version 2005/12/01 or later.

This work has the LPPL maintenance status `maintained'.

The Current Maintainer of this work is Niklas Beisert.

This work consists of the files |README.txt|, |childdoc.ins| and |childdoc.dtx|
as well as the derived files |childdoc.def|, |cdocsamp.tex|
with |cdocsch1.tex|, |cdocsch2.tex|, |cdocspt3.tex|, |cdocspt4.tex|,
|cdocsdrf.tex|, |cdocsfn1.tex|, |cdocsfn2.tex|
as well as |childdoc.pdf|.

%%%%%%%%%%%%%%%%%%%%%%%%%%%%%%%%%%%%%%%%%%%%%%%%%%%%%%%%%%%%%%%%%%%%%%%%%%%%%%%%
\subsection{Files and Installation}

The package consists of the files:
%
\begin{center}
\begin{tabular}{ll}
    |README.txt|   & readme file \\
    |childdoc.ins| & installation file \\
    |childdoc.dtx| & source file \\
    |childdoc.def| & definition file \\
    |cdocsamp.tex| & sample main file \\
    |cdocsch1.tex| & sample include file \\
    |cdocsch2.tex| & sample include file \\
    |cdocspt3.tex| & sample part file \\
    |cdocspt4.tex| & sample part file \\
    |cdocsdrf.tex| & sample redirection file \\
    |cdocsfn1.tex| & sample redirection file \\
    |cdocsfn2.tex| & sample redirection file \\
    |childdoc.pdf| & manual
\end{tabular}
\end{center}
%
The distribution consists of the files
|README.txt|, |childdoc.ins| and |childdoc.dtx|.
%
\begin{itemize}
\item
Run (pdf)\LaTeX{} on |childdoc.dtx|
to compile the manual |childdoc.pdf| (this file).
\item
Run \LaTeX{} on |childdoc.ins| to create the definitions file |childdoc.def|
and the sample |cdocsamp.tex| with include files
|cdocsch1.tex|, |cdocsch2.tex|, |cdocspt3.tex|, |cdocspt4.tex|,
|cdocsdrf.tex|, |cdocsfn1.tex|, |cdocsfn2.tex|.
Then copy the file |childdoc.def| to an appropriate directory of your \LaTeX{}
distribution, e.g.\ \textit{texmf-root}|/tex/latex/childdoc|.
\end{itemize}

%%%%%%%%%%%%%%%%%%%%%%%%%%%%%%%%%%%%%%%%%%%%%%%%%%%%%%%%%%%%%%%%%%%%%%%%%%%%%%%%
\subsection{Related CTAN Packages}

There are several other packages which offer a similar functionality:
%
\begin{itemize}
\item
The packages
\href{http://ctan.org/pkg/docmute}{\textsf{docmute}},
\href{http://ctan.org/pkg/includex}{\textsf{includex}} and
\href{http://ctan.org/pkg/standalone}{\textsf{standalone}}
provide commands to include only the document body of
a child file thus allowing both files to be compiled individually.
\item
The packages \href{http://ctan.org/pkg/subdocs}{\textsf{subdocs}}
and \href{http://ctan.org/pkg/subfiles}{\textsf{subfiles}}
provide structures in which the main and child documents can be
encapsulated and allowing them to be compiled individually.
The inclusion mechanism is different from the conventional |\include|.
\item
The package \href{http://ctan.org/pkg/combine}{\textsf{combine}}
is an elaborate solution to combine several documents into one.
\end{itemize}
%
See also the CTAN topic \href{http://ctan.org/topic/subdocs}{\textsf{subdocs}}
for further related packages.
The present package differs from the above solutions in that
a document structure constructed with the conventional |\include| mechanism
just needs two extra commands at the top of every file
such that all constituent files can be compiled individually.

%%%%%%%%%%%%%%%%%%%%%%%%%%%%%%%%%%%%%%%%%%%%%%%%%%%%%%%%%%%%%%%%%%%%%%%%%%%%%%%%
%\subsection{Feature Suggestions}
%
%The following is a list of features which may be useful for future
%versions of this package:
%%
%\begin{itemize}
%\item
%\ldots
%\end{itemize}

%%%%%%%%%%%%%%%%%%%%%%%%%%%%%%%%%%%%%%%%%%%%%%%%%%%%%%%%%%%%%%%%%%%%%%%%%%%%%%%%
\subsection{Revision History}

%%%%%%%%%%%%%%%%%%%%%%%%%%%%%%%%%%%%%%%%
\paragraph{v2.0:} 2018/12/30

\begin{itemize}
\item
immediate forward processing
\item
added |\childdocby| mechanism
\item
manual restructured
\end{itemize}

%%%%%%%%%%%%%%%%%%%%%%%%%%%%%%%%%%%%%%%%
\paragraph{v1.6:} 2018/01/17

\begin{itemize}
\item
application for development of include files
\item
corrections to manual
\end{itemize}

%%%%%%%%%%%%%%%%%%%%%%%%%%%%%%%%%%%%%%%%
\paragraph{v1.5:} 2017/05/21

\begin{itemize}
\item
more complete structuring introduced
\item
|\childdocof| introduced
\item
|\childdoc| renamed to |\childdocmain|
\item
|\childredirect| renamed to |\childdocforward| and |\childdocforwardprefix|
and functionality expanded
\end{itemize}

%%%%%%%%%%%%%%%%%%%%%%%%%%%%%%%%%%%%%%%%
\paragraph{v1.0:} 2017/04/27

\begin{itemize}
\item
manual and install package
\item
first version published on CTAN
\end{itemize}

%%%%%%%%%%%%%%%%%%%%%%%%%%%%%%%%%%%%%%%%
\paragraph{v0.6:} 2017/04/26

\begin{itemize}
\item
redirection mechanism added
\end{itemize}

%%%%%%%%%%%%%%%%%%%%%%%%%%%%%%%%%%%%%%%%
\paragraph{v0.5:} 2017/04/26

\begin{itemize}
\item
functionality in definition file
\end{itemize}


%%%%%%%%%%%%%%%%%%%%%%%%%%%%%%%%%%%%%%%%%%%%%%%%%%%%%%%%%%%%%%%%%%%%%%%%%%%%%%%%
%%%%%%%%%%%%%%%%%%%%%%%%%%%%%%%%%%%%%%%%%%%%%%%%%%%%%%%%%%%%%%%%%%%%%%%%%%%%%%%%
%%%%%%%%%%%%%%%%%%%%%%%%%%%%%%%%%%%%%%%%%%%%%%%%%%%%%%%%%%%%%%%%%%%%%%%%%%%%%%%%
\appendix

\settowidth\MacroIndent{\rmfamily\scriptsize 000\ }

 \DocInput{childdoc.dtx}

\end{document}
%</driver>
% \fi
%
% %%%%%%%%%%%%%%%%%%%%%%%%%%%%%%%%%%%%%%%%%%%%%%%%%%%%%%%%%%%%%%%%%%%%%%%%%%%%%%
% %%%%%%%%%%%%%%%%%%%%%%%%%%%%%%%%%%%%%%%%%%%%%%%%%%%%%%%%%%%%%%%%%%%%%%%%%%%%%%
% \section{Sample}
%\iffalse
%<*samplemain>
%\fi
%
% The following presents a sample document
% with two chapters, two parts, a title page,
% a compile flag as well as three forwarding files to set the flag.
% It consists of eight |.tex| files:
% \begin{center}
% \begin{tabular}{ll}
% |cdocsamp.tex|&main file\\
% |cdocsch1.tex|&include file for chapter 1\\
% |cdocsch2.tex|&include file for chapter 2\\
% |cdocspt3.tex|&include file for part 3\\
% |cdocspt4.tex|&include file for part 4\\
% |cdocsdrf.tex|&forwarding file for main file in draft mode\\
% |cdocsfi1.tex|&forwarding file for final version of chapter 1\\
% |cdocsfi2.tex|&forwarding file for final version of chapter 2\\
% \end{tabular}
% \end{center}
% Each of the eight files can be compiled directly by the \LaTeX{} compiler.
%
% %%%%%%%%%%%%%%%%%%%%%%%%%%%%%%%%%%%%%%
% \paragraph{Main File.}
%
% The main file is called |cdocsamp.tex|.
%
% Load the \textsf{childdoc} definitions and
% declare the filename for the main document:
%    \begin{macrocode}
\input{childdoc.def}
\childdocmain{}
%    \end{macrocode}

% Optional override for |\version| flag:
%    \begin{macrocode}
%%\ifchilddoc\else\providecommand{\version}{draft}\fi
%    \end{macrocode}

% Define the default values for the |\version| flag
% (|final| for the main file and |draft| for childs):
%    \begin{macrocode}
\ifchilddoc
\providecommand{\version}{draft}
\else
\providecommand{\version}{final}
\fi
%    \end{macrocode}

% Load the standard document class:
%    \begin{macrocode}
\documentclass[12pt]{article}
%    \end{macrocode}

% Start the document body:
%    \begin{macrocode}
\begin{document}
%    \end{macrocode}

% Declare a title page.
% Print title, part of document being processed and version flag:
%    \begin{macrocode}
\addtocounter{page}{-1}
\begin{center}
{\LARGE\bfseries{}childdoc example\par}
\vspace{1cm}
\ifchilddoc
\ifchilddocmanual part\else chapter\fi:
`\childdocname' of `\childdocjob'\par
\else
main document: `\childdocjob'\par
\fi
version: \version\par
\end{center}
\newpage
%    \end{macrocode}

% Manually include selected file,
% otherwise process as usual:
%    \begin{macrocode}
\ifchilddocmanual
\section*{part `\childdocname'}
\input{\childdocname}
\else
%    \end{macrocode}

% Include the two chapters:
%    \begin{macrocode}
\include{cdocsch1}
\include{cdocsch2}
%    \end{macrocode}

% Include the two parts unless only chapters should be displayed:
%    \begin{macrocode}
\ifchilddoc\else
\section{part three}
\input{cdocspt3}
\section{part four}
\input{cdocspt4}
\fi
%    \end{macrocode}

% Process as usual until here:
%    \begin{macrocode}
\fi
%    \end{macrocode}

% End of document body:
%    \begin{macrocode}
\end{document}
%    \end{macrocode}
%\iffalse
%</samplemain>
%\fi
%
% %%%%%%%%%%%%%%%%%%%%%%%%%%%%%%%%%%%%%%
% \paragraph{Chapter Include Files.}
%
% The include files are called |cdocsch1.tex| and |cdocsch2.tex|.
%
%\iffalse
%<*samplechap1|samplechap2>
%\fi

% Optional override for |\version| flag:
%    \begin{macrocode}
%%\providecommand{\version}{final}
%    \end{macrocode}

% Include the main document:
%    \begin{macrocode}
\input{childdoc.def}
\childdocof{cdocsamp}
%    \end{macrocode}

%\iffalse
%</samplechap1|samplechap2>
%\fi
%
%\iffalse
%<*samplechap1>
%\fi
% Some text for chapter 1:
%    \begin{macrocode}
\section{one}
some text in chapter one
%    \end{macrocode}

%\iffalse
%</samplechap1>
%\fi
% Some text for chapter 2:
%\iffalse
%<*samplechap2>
%\fi
%    \begin{macrocode}
\section{two}
more text in chapter two
%    \end{macrocode}

%\iffalse
%</samplechap2>
%\fi
%
% %%%%%%%%%%%%%%%%%%%%%%%%%%%%%%%%%%%%%%
% \paragraph{Part Include Files.}
%
% The include files are called |cdocspt3.tex| and |cdocspt4.tex|.
%
%\iffalse
%<*samplepart3|samplepart4>
%\fi

% Optional override for |\version| flag:
%    \begin{macrocode}
%%\providecommand{\version}{final}
%    \end{macrocode}

% Include the main document:
%    \begin{macrocode}
\input{childdoc.def}
\childdocby{cdocsamp}
%    \end{macrocode}

%\iffalse
%</samplepart3|samplepart4>
%\fi
%
%\iffalse
%<*samplepart3>
%\fi
% Some text for part 3:
%    \begin{macrocode}
some text in part three
%    \end{macrocode}

%\iffalse
%</samplepart3>
%\fi
% Some text for part 4:
%\iffalse
%<*samplepart4>
%\fi
%    \begin{macrocode}
more text in part four
%    \end{macrocode}

%\iffalse
%</samplepart4>
%\fi
%
% %%%%%%%%%%%%%%%%%%%%%%%%%%%%%%%%%%%%%%
% \paragraph{Forwarding for a Complete Draft.}
%
% The following forwarding file |cdocsdrf.tex|
% compiles the main document in draft mode:
%\iffalse
%<*sampledraft>
%\fi
%    \begin{macrocode}
\def\version{draft}
\input{childdoc.def}
\childdocforward{cdocsamp}
%    \end{macrocode}

%\iffalse
%</sampledraft>
%\fi
%
% %%%%%%%%%%%%%%%%%%%%%%%%%%%%%%%%%%%%%%
% \paragraph{Forwarding for Final Version of the Chapters.}
%
% The following forwarding files |cdocsfn1.tex| and |cdocsfn2.tex|
% (with identical content)
% compile the final versions of the child documents
% |cdocsch1.tex| and |cdocsch2.tex|, respectively:
%\iffalse
%<*samplefinal>
%\fi
%    \begin{macrocode}
\def\version{final}
\input{childdoc.def}
\childdocforwardprefix[cdocsamp]{cdocsfn}{cdocsch}
%    \end{macrocode}

%\iffalse
%</samplefinal>
%\fi
%
% %%%%%%%%%%%%%%%%%%%%%%%%%%%%%%%%%%%%%%
% \paragraph{Command Line Processing.}
%
% The following three command lines generate the output files
% |cdocscld|, |cdocscl1| and |cdocscl2|
% which should be identical to
% |cdocsdrf|, |cdocsch1| and |cdocsfn2|, respectively:
% \begin{center}
% \begin{tabular}{l}
% |latex -jobname cdocscld \|\\
% |  "\def\version{draft}\input{childdoc.def}\childdocforward{cdocsamp}"|\\
% |latex -jobname cdocscl1 \|\\
% |  "\input{childdoc.def}\childdocforward[cdocsamp]{cdocsch1}"|\\
% |latex -jobname cdocscl2 \|\\
% |  "\def\version{final}\input{childdoc.def}\childdocforward{cdocsch2}"|
% \end{tabular}
% \end{center}
% Note that the trailing backslash on each first line
% merely continues the input to the second line
% (for convenient cut ant paste).
% Furthermore, the command |latex| can be replaced by any
% of its alternative versions such as |pdflatex|.
%
% %%%%%%%%%%%%%%%%%%%%%%%%%%%%%%%%%%%%%%%%%%%%%%%%%%%%%%%%%%%%%%%%%%%%%%%%%%%%%%
% %%%%%%%%%%%%%%%%%%%%%%%%%%%%%%%%%%%%%%%%%%%%%%%%%%%%%%%%%%%%%%%%%%%%%%%%%%%%%%
% \section{Implementation}
%\iffalse
%<*package>
%\fi
%
% This section describes the definitions file |childdoc.def|.

% The definitions cannot be loaded using |\usepackage| or |\RequirePackage|
% which has a mechanism to prevent loading a style file more than once.
% When loading the definitions by means of |\input|
% multiple instances have to be prevented manually:
%\iffalse
%This code needs to be before the `\ProvidesFile' directive
%which is defined at the beginning of this file.
%Therefore it is also placed there and commented out here.
%</package>
%<*discard>
%\fi
%    \begin{macrocode}
\ifdefined\childdocmain\endinput\fi
%    \end{macrocode}
%\iffalse
%</discard>
%<*package>
%\fi
%
% \macro{\ifchilddoc}
% \macro{\ifchilddocmanual}
% The conditional |\ifchilddoc| tells whether a
% child (true) or main (false) document is being compiled.
% The conditional |\ifchilddocmanual| tells whether
% the |\includeonly| mechanism is used (false) or
% the selection of child files must be performed manually (true).
% The definitions initialise to false:
%    \begin{macrocode}
\newif\ifchilddoc
\newif\ifchilddocmanual
%    \end{macrocode}

% \macro{\childdocname}
% \macro{\childdocjob}
% The macro |\childdocname| stores the name of the main document
% to be compiled. The macro |\childdocjob| stores the name of
% the document on which the \LaTeX{} compiler was originally invoked.
% The content of |\jobname| cannot be compared
% to filenames specified in the source due to different catcodes.
% The following code rescans |\jobname|, stores the result
% in |\childdocname| and saves a copy in |\childdocjob|:
%    \begin{macrocode}
\edef\childdocname{\scantokens\expandafter{\jobname\noexpand}}
\let\childdocjob\childdocname
%    \end{macrocode}

% \macro{\childdocdisable}
% The macro |\childdocdisable| prevents the main file
% from being processed more than once.
% At this stage, the main document command |\childdocmain|
% is assumed to be called once again where it should do nothing.
% Any subsequent call to it should prevent
% a secondary processing of the main document
% It overwrites the forwarding commands
% |\childdocof| and |\childdocforward|
% with empty macros to prevent further inclusions of the main document:
%    \begin{macrocode}
\newcommand{\childdocdisable}
{
  \renewcommand{\childdocmain}[1]{\renewcommand{\childdocmain}[1]{\endinput}}
  \renewcommand{\childdocof}[1]{}
  \renewcommand{\childdocby}[2][]{}
  \renewcommand{\childdocforward}[2][]{}
  \renewcommand{\childdocdisable}{}
}
%    \end{macrocode}

% \macro{\childdocmain}
% The macro |\childdocmain| is to be called at the top of the main file
% with nothing or the main filename (without extension) as argument.
% First, it breaks loops.
% If the argument is not empty and does not match |\childdocname|
% (which is set by the first inclusion of |childdoc.def|),
% |\ifchilddoc| is set to true, |\includeonly| is applied to the child file
% and |\jobname| is set to the main file
% (for proper handling of |.aux| files):
%    \begin{macrocode}
\newcommand{\childdocmain}[1]
{
  \childdocdisable\childdocmain{}
  \if?#1?\else
    \begingroup
      \def\childdoctmp{#1}
      \ifx\childdoctmp\childdocname
        \def\childdoctmp{}
      \else
        \def\childdoctmp
        {
          \childdoctrue
          \includeonly{\childdocname}
          \def\childdocjob{#1}
          \def\jobname{#1}
        }
      \fi
      \expandafter
    \endgroup
    \childdoctmp
  \fi
}
%    \end{macrocode}

% \macro{\childdocof}
% The command |\childdocof| redirects
% compilation to the main file |#1|.
%    \begin{macrocode}
\newcommand{\childdocof}[1]
{
  \childdocdisable
  \childdoctrue
  \includeonly{\childdocname}
  \def\jobname{#1}
  \def\childdocjob{#1}
  \input{#1}
}
%    \end{macrocode}

% \macro{\childdocby}
% The command |\childdocby| ....
%    \begin{macrocode}
\newcommand{\childdocby}[2][]
{
  \childdocdisable
  \childdoctrue
  \childdocmanualtrue
  \if?#1?\else
    \def\jobname{#2}
  \fi
  \def\childdocjob{#2}
  \input{#2}
  \endinput
}
%    \end{macrocode}

% \macro{\childdocforward}
% The command |\childdocforward| redirects
% compilation to the main file or
% (if the optional argument is given) a child file.
% Parameters are set as if the main file
% or a child file starting with |\childdocof| was compiled.
% Then compilation is handed over to the main file:
%    \begin{macrocode}
\newcommand{\childdocforward}[2][]
{
  \begingroup
    \if?#1?
      \def\childdoctmp
      {
        \def\childdocname{#2}
        \def\childdocjob{#2}
        \def\jobname{#2}
        \input{#2}
        \endinput
      }
    \else
      \def\childdoctmp
      {
        \childdocdisable
        \def\childdocname{#2}
        \childdoctrue
        \includeonly{#2}
        \def\childdocjob{#1}
        \def\jobname{#1}
        \input{#1}
        \endinput
      }
    \fi
    \expandafter
  \endgroup
  \childdoctmp
}
%    \end{macrocode}

% \macro{\childdocforwardprefix}
% The command |\childdocforwardprefix| redirects
% compilation to the main or a child file by means of a pattern.
% The prefix |#1| in the current filename is replaced by |#2|
% and the suffix of the current filename is kept
% (it is assumed that the filename does not contain the substring `|~~~|'
% which is used as a delimiter).
% Compilation is handed over to the new file by |\childdocforward|:
%    \begin{macrocode}
\newcommand{\childdocforwardprefix}[3][]
{
  \begingroup
    \def\childdocextract #2##1~~~{\def\childdoctmp{\childdocforward[#1]{#3##1}}}
    \expandafter\childdocextract\childdocname~~~
    \expandafter
  \endgroup
  \childdoctmp
}
%    \end{macrocode}

% \macro{\childdoc}
% The deprecated macro |\childdoc| is a legacy version of |\childdocmain|:
%    \begin{macrocode}
\newcommand{\childdoc}{\childdocmain}
%    \end{macrocode}

% \macro{\childdocredirect}
% The deprecated macro |\childdocredirect| is a legacy version
% of |\childdocforward| and |\childdocforwardprefix|:
%    \begin{macrocode}
\newcommand{\childdocredirect}[2][]
{
  \begingroup
    \if?#1?
      \def\childdoctmp{\childdocforward{#2}}
    \else
      \def\childdoctmp{\childdocforwardprefix{#1}{#2}}
    \fi
    \expandafter
  \endgroup
  \childdoctmp
}
%    \end{macrocode}

%\iffalse
%</package>
%\fi
%
\endinput
|\\
|\childdocby{|\textit{main}|}|\\
\end{tabular}
\end{center}
%
The directive |\childdocby| is similar to |\childdocof|
described in \secref{sec:include},
but the subsequent selection of content must be done manually.
To that end, both |\ifchilddoc| and |\ifchilddocmanual|
will be true upon processing of a part,
and the name of the part is stored in |\childdocname|.
Note that |\jobname| will be set to the filename of the current part
so that each part receives an individual |.aux| file
that does not interfere with the |.aux| file(s) of the main document.
This behaviour can be altered by the alternative form
|\childdocby[*]{|\textit{main}|}| (with a non-empty optional argument)
which uses the |.aux| file of the main document
by setting |\jobname| to \textit{main}.

%%%%%%%%%%%%%%%%%%%%%%%%%%%%%%%%%%%%%%%%%%%%%%%%%%%%%%%%%%%%%%%%%%%%%%%%%%%%%%%%
\subsection{Driver Development}
\label{sec:driver}

The \textsf{childdoc} mechanism can also be use for the development
of definition files such as \LaTeX{} styles or classes.
This case differs from the above setup with multiple parts
included by |\include| in that no |\includeonly| should be invoked.
This can be achieved by starting the include file
(before |\ProvidesPackage|) with:
%
\begin{center}
\begin{tabular}{l}
|% \iffalse
%
% childdoc.dtx Copyright (C) 2017-2018 Niklas Beisert
%
% This work may be distributed and/or modified under the
% conditions of the LaTeX Project Public License, either version 1.3
% of this license or (at your option) any later version.
% The latest version of this license is in
%   http://www.latex-project.org/lppl.txt
% and version 1.3 or later is part of all distributions of LaTeX
% version 2005/12/01 or later.
%
% This work has the LPPL maintenance status `maintained'.
%
% The Current Maintainer of this work is Niklas Beisert.
%
% This work consists of the files childdoc.dtx and childdoc.ins
% and the derived files childdoc.def and cdocsamp.tex with
% cdocsch1.tex, cdocsch2.tex, cdocsdrf.tex, cdocsfn1.tex, cdocsfn2.tex.
%
%<package>\ifdefined\childdocmain\endinput\fi
%<package>\ProvidesFile{childdoc.def}[2018/12/30 v2.0 child document driver]
%<samplemain>\ProvidesFile{cdocsamp.tex}[2018/12/30 v2.0 sample for childdoc]
%<*driver>
%\ProvidesFile{childdoc.drv}[2018/12/30 v2.0 childdoc reference manual file]
\PassOptionsToClass{10pt,a4paper}{article}
\documentclass{ltxdoc}

\usepackage[margin=35mm]{geometry}
\usepackage{hyperref}
\usepackage{hyperxmp}
\usepackage[usenames]{color}

\hypersetup{colorlinks=true}
\hypersetup{pdfstartview=FitH}
\hypersetup{pdfpagemode=UseNone}
\hypersetup{pdfsource={}}
\hypersetup{pdflang={en-UK}}
\hypersetup{pdfcopyright={Copyright 2017-2018 Niklas Beisert.
  This work may be distributed and/or modified under the
  conditions of the LaTeX Project Public License, either version 1.3
  of this license or (at your option) any later version.}}
\hypersetup{pdflicenseurl={http://www.latex-project.org/lppl.txt}}
\hypersetup{pdfcontactaddress={ETH Zurich, ITP, HIT K,
  Wolfgang-Pauli-Strasse 27}}
\hypersetup{pdfcontactpostcode={8093}}
\hypersetup{pdfcontactcity={Zurich}}
\hypersetup{pdfcontactcountry={Switzerland}}
\hypersetup{pdfcontactemail={nbeisert@itp.phys.ethz.ch}}
\hypersetup{pdfcontacturl={http://people.phys.ethz.ch/\xmptilde nbeisert/}}

\newcommand{\secref}[1]{\hyperref[#1]{section \ref*{#1}}}

\parskip1ex
\parindent0pt
\let\olditemize\itemize
\def\itemize{\olditemize\parskip0pt}

\begin{document}

\title{The \textsf{childdoc} Package}
\hypersetup{pdftitle={The childdoc Package}}
\author{Niklas Beisert\\[2ex]
  Institut f\"ur Theoretische Physik\\
  Eidgen\"ossische Technische Hochschule Z\"urich\\
  Wolfgang-Pauli-Strasse 27, 8093 Z\"urich, Switzerland\\[1ex]
  \href{mailto:nbeisert@itp.phys.ethz.ch}
  {\texttt{nbeisert@itp.phys.ethz.ch}}}
\hypersetup{pdfauthor={Niklas Beisert}}
\hypersetup{pdfsubject={Manual for the LaTeX2e Package childdoc}}
\date{30 December 2018, \textsf{v2.0}}
\maketitle

\begin{abstract}\noindent
\textsf{childdoc} is a \LaTeXe{} package
that enables the direct compilation
of document sections included by |\include|
to individual files.
\end{abstract}

\begingroup
\parskip0ex
\tableofcontents
\endgroup

%%%%%%%%%%%%%%%%%%%%%%%%%%%%%%%%%%%%%%%%%%%%%%%%%%%%%%%%%%%%%%%%%%%%%%%%%%%%%%%%
%%%%%%%%%%%%%%%%%%%%%%%%%%%%%%%%%%%%%%%%%%%%%%%%%%%%%%%%%%%%%%%%%%%%%%%%%%%%%%%%
\section{Introduction}

\LaTeX{} provides a mechanism to structure a large document (such as a book)
into a main file and several child files (containing the chapters)
using the |\include| command.
This mechanism is beneficial for documents
which span hundreds of pages in order to
make the source file(s) more manageable.
Moreover, compilation can be restricted to
selected child files by means of the |\includeonly| command.
The latter feature can be used to reduce the compilation time while editing
(this was significantly more useful in the earlier days of \LaTeX{})
or to generate a smaller document which is easier to navigate.
Another application of |\includeonly| is to generate
documents consisting of selected parts of the complete document.

However, there are a few drawbacks of the plain |\include| mechanism:
\begin{itemize}
\item
The child files cannot be compiled on their own,
they can only be compiled via the main file.
A naive editing environment
(such as a text editor with an option
to have the current file processed by \LaTeX)
may require one to switch to the main file before compiling;
attempting to compile the child file produces errors.
\item
The main file must be modified (each time)
to adjust the |\includeonly| command
to the present needs. This easily leaves the main file in a messy state.
\item
The generated document will always carry the filename
of the main document. This is inconvenient if
several child files are to be compiled and
to be kept for distribution.
\end{itemize}

The present package provides a simple interface
to make child files individually compilable by \LaTeX{}.
Compiling a child file then has the same effect as compiling
the main file with an |\includeonly| command
to select the appropriate child.
Moreover the generated document will carry the name of the child
rather than the main file.
This resolves all three above issues.

This feature is meant to make the editing of books,
thesis documents and lecture notes somewhat more convenient.
However, the package can also be used efficiently for
composing a series of documents (such as exercise sheets)
which are typically distributed individually.
It then assists the author in generating the individual documents
(potentially in different versions)
as well as a document containing the collected series.
Another application is in developing style files
or other kinds of included material
where compilation of the style file could redirect
to a sample or test file.

%%%%%%%%%%%%%%%%%%%%%%%%%%%%%%%%%%%%%%%%%%%%%%%%%%%%%%%%%%%%%%%%%%%%%%%%%%%%%%%%
%%%%%%%%%%%%%%%%%%%%%%%%%%%%%%%%%%%%%%%%%%%%%%%%%%%%%%%%%%%%%%%%%%%%%%%%%%%%%%%%
\section{Usage}

First of all, the package \textsf{childdoc} is \emph{not} a standard
\LaTeXe{} |.sty| style file! Therefore it needs to be invoked in
a non-standard way.

%%%%%%%%%%%%%%%%%%%%%%%%%%%%%%%%%%%%%%%%%%%%%%%%%%%%%%%%%%%%%%%%%%%%%%%%%%%%%%%%
\subsection{Included Files}
\label{sec:include}

%%%%%%%%%%%%%%%%%%%%%%%%%%%%%%%%%%%%%%%%
\DescribeMacro{\childdocmain}
To use the package, add the commands
\begin{center}
\begin{tabular}{l}
|\input{childdoc.def}|\\
|\childdocmain{}|\\
\end{tabular}
\end{center}
at the very top of the main \LaTeX{} file,
in particular \emph{before} the |\documentclass| statement!
The argument of |\childdocmain| should be left empty
(but it must be present).

%%%%%%%%%%%%%%%%%%%%%%%%%%%%%%%%%%%%%%%%
\DescribeMacro{\childdocof}
Furthermore, add the commands
\begin{center}
\begin{tabular}{l}
|\input{childdoc.def}|\\
|\childdocof{|\textit{main}|}|\\
\end{tabular}
\end{center}
at the top of every child file \textit{child}
which is included by |\include{|\textit{child}|}|
from within the main file
(or at least for those files to be compiled individually).
The argument \textit{main} must be the filename of the main file.

There are a couple of
considerations in setting up the main and child documents:

%%%%%%%%%%%%%%%%%%%%%%%%%%%%%%%%%%%%%%%%
\paragraph{Restrictions.}

Please note the following restrictions:
\begin{itemize}
\item
|\childdocmain| must be called with one argument \textit{main}
to ensure compatibility with earlier version of the package.
It must either be empty (|\childdocmain{}|)
or precisely match the filename of the main file in which it is specified.
See \secref{sec:detection} for further information.
\item
The filename \textit{main} must be specified without the |.tex| extension.
\item
The filename \textit{main} is case sensitive
(even in case-insensitive file systems)
due to internal string comparison.
\item
The argument \textit{main} should be fully expanded, it cannot be a macro.
\item
Subdirectories and special characters should be avoided in filenames.
\item
The command |\childdocmain{|\textit{main}|}| must be followed by a whitespace.
It should not be followed immediately by another command
or by a comment mark `|%|'.
This is because the \TeX{} parser reads the token immediately following
the argument of |\childdocmain| and puts it
at the beginning of every child section;
however, a white\-space is ignored.
\end{itemize}

%%%%%%%%%%%%%%%%%%%%%%%%%%%%%%%%%%%%%%%%
\paragraph{Content of Main File.}

It is advisable to place all content in the child files included by |\include|.
Any output contained in the main file will appear in all child documents
unless suppressed manually;
it cannot be suppressed automatically by the |\includeonly| directive
and thus should normally be avoided.
A method to include some content in the main file
by means of conditional processing is described in \secref{sec:conditional}.

%%%%%%%%%%%%%%%%%%%%%%%%%%%%%%%%%%%%%%%%
\paragraph{Page Numbering.}

When only a part of the document is compiled,
the appropriate numbering of pages
(as well as other status parameters)
is determined from the |.aux| files.
The latter contain information from previous passes.
However this information needs to propagate through
all intermediate child documents.
Therefore the page numbering in child documents may well
be inconsistent until the complete document is compiled at least once.

A useful (if unconventional) way to always ensure a consistent
page numbering is to restart the numbering in each child document
and denote the pages by `\textit{child}|.|\textit{page}'
where \textit{child} represents the chapter/section number of the child file.
This can be achieved by the command
|\numberwithin{page}{|\textit{child}|}|
of the \textsf{amsmath} package
where \textit{child} can be |chapter| or |section|
depending on the chosen structuring.
Alternatively, one can modify the macro |\thepage| appropriately
and reset the counter |page| at the start of each child file.

%%%%%%%%%%%%%%%%%%%%%%%%%%%%%%%%%%%%%%%%%%%%%%%%%%%%%%%%%%%%%%%%%%%%%%%%%%%%%%%%
\subsection{Conditional Processing}
\label{sec:conditional}

The package provides a mechanism to compile different versions
of a document. To customise the versions further some conditional processing
can come in handy to distinguish which version is being compiled.
The package provides two macros to describe the compilation context:

%%%%%%%%%%%%%%%%%%%%%%%%%%%%%%%%%%%%%%%%
\DescribeMacro{\ifchilddoc}
The conditional |\ifchilddoc| distinguishes between the compilation of
child documents and the main document:
%
\begin{center}
|\ifchilddoc |\textit{child-code}| |[|\||else |\textit{main-code}]| \||fi|
\end{center}

%%%%%%%%%%%%%%%%%%%%%%%%%%%%%%%%%%%%%%%%
\DescribeMacro{\childdocname}
\DescribeMacro{\childdocjob}
The macro |\childdocname| contains the filename (without extension)
of the main or child file being processed.
Note that |\childdocjob| will always contain the name of the main file.

%%%%%%%%%%%%%%%%%%%%%%%%%%%%%%%%%%%%%%%%
\paragraph{Title Page.}

Conditional processing can be used to include a title or banner page
in the main document when proper precautions are taken.
Importantly, the code in the main file should ensure that the page counter
(as well as other status parameters which are stored in the |.aux| files)
takes the same value after the conditional processing.
Otherwise the page numbers may take divergent values
depending on which part is compiled.

For example, a title page could be declared by:
%
\begin{center}
\begin{tabular}{l}
|\ifchilddoc\||else|\\
|\addtocounter{page}{-1}|\\
\textit{code for title page}\\
|\newpage|\\
|\||fi|
\end{tabular}
\end{center}
%
A banner page for the child documents can be generated by:
%
\begin{center}
\begin{tabular}{l}
|\ifchilddoc|\\
|\addtocounter{page}{-1}|\\
\textit{code for banner page}\\
|\newpage|\\
|\||fi|
\end{tabular}
\end{center}
%
Here one could write a message such as:
\begin{center}
|This is the part \childdocname{} of \childdocjob{}.|
\end{center}

%%%%%%%%%%%%%%%%%%%%%%%%%%%%%%%%%%%%%%%%%%%%%%%%%%%%%%%%%%%%%%%%%%%%%%%%%%%%%%%%
\subsection{Flags}
\label{sec:flags}

The package makes it easy to generate different versions
of the main or child documents.
To this end compilation flags can be defined
and assigned different default values.
They will be particularly useful in conjunction
with the forwarding mechanism described in \secref{sec:forward}.

For example, it may be useful to have a flag |\version|
which can be set to |draft| or |final|.
The document source will contain some conditional code
depending on the value of |\version|.
Suppose further, the flag should default to |final| for the main file
and to |draft| for child files
which is a natural assignment for editing the document.
This is achieved by placing the following code
in the preamble of the main document
(below the |\childdocmain| directive):
%
\begin{center}
\begin{tabular}{l}
|\ifchilddoc|\\
|\providecommand{\version}{draft}|\\
|\||else|\\
|\providecommand{\version}{final}|\\
|\||fi|
\end{tabular}
\end{center}
%
The definition by |\providecommand| makes sure
that previous definitions are not overwritten.
Further statements |\providecommand{\version}{...}|
can thus be added before the above code to override it.

For the main file, one might add a line
(between |\childdocmain| and the above block)
%
\begin{center}
|%\ifchilddoc\||else\providecommand{\version}{draft}\||fi|
\end{center}
%
which can be uncommented to produce a draft version.
Likewise one can add a line to the very top of a child file
(above the |\childdocof{|\textit{main}|}| directive)
%
\begin{center}
|%\providecommand{\version}{final}|
\end{center}
%
which can be uncommented to produce the final version of this child document.

%%%%%%%%%%%%%%%%%%%%%%%%%%%%%%%%%%%%%%%%%%%%%%%%%%%%%%%%%%%%%%%%%%%%%%%%%%%%%%%%
\subsection{Forwarding}
\label{sec:forward}

Different versions of the main or child documents
using compilation flags as described in \secref{sec:flags}
can be (permanently) stored in different files
for convenient compilation, viewing and distribution.
To this end, the package defines a command
to pass on compilation to a different file:

%%%%%%%%%%%%%%%%%%%%%%%%%%%%%%%%%%%%%%%%
\DescribeMacro{\childdocforward}
The command |\childdocforward| redirects processing to
another source file:
%
\begin{center}
\begin{tabular}{l}
|\input{childdoc.def}|\\
|\childdocforward[|\textit{main}|]{|\textit{dest}|}|\\
\end{tabular}
\end{center}
%
The argument \textit{dest} is the destination file
(without extension).
It should be the main file or one of the child files.
Note that further \textsf{childdoc} directives
such as |\childdocof| and |\childdocforward|
in the indicated file will be processed in this form.
The optional argument \textit{main}
passes on directly to the main file \textit{main}
while pretending to compile the child \textit{dest}.
This form behaves as if \textit{dest}
issues |\childdocof{|\textit{main}|}| right away,
and no further \textsf{childdoc} directives will be processed.

%%%%%%%%%%%%%%%%%%%%%%%%%%%%%%%%%%%%%%%%
\DescribeMacro{\...prefix}
In the alternative form |\childdocforwardprefix|,
%
\begin{center}
\begin{tabular}{l}
|\input{childdoc.def}|\\
|\childdocforwardprefix[|\textit{main}|]{|\textit{prefix}|}{|\textit{dest}|}|
\end{tabular}
\end{center}
%
the destination file is determined by a pattern
depending on the current file:
To make this work, the current file must be called
`{\textit{prefix}\hspace{0.2em}\textit{suffix}}'
with \textit{prefix} matching precisely the argument.
Processing is then passed on to the file
`{\textit{dest}\hspace{0.2em}\textit{suffix}}'.
Surely, the same effect is achieved by
directly specifying the
argument `{\textit{dest}\hspace{0.2em}\textit{suffix}}'
in the first form.
However, that requires to set up a different file
for each child. With the alternative form of the command
all these files can have exactly the same content
which simplifies setting them up and maintaining them.

For example, the following file |draft.tex|
with a compilation flag |\version| as described in \secref{sec:flags}
compiles the main document as a draft:
%
\begin{center}
\begin{tabular}{l}
|\def\version{draft}|\\
|\input{childdoc.def}|\\
|\childdocforward{|\textit{main}|}|
\end{tabular}
\end{center}
%
Likewise, the following files |final|\textit{nn}|.tex|
compile the final version of the child document
|child|\textit{nn}|.tex|:
%
\begin{center}
\begin{tabular}{l}
|\def\version{final}|\\
|\input{childdoc.def}|\\
|\childdocforwardprefix{final}{child}|
\end{tabular}
\end{center}
%

Note that when several versions of a main file and/or of each child file
are to be generated, it may be convenient to set up a |Makefile| or
shell script to automatise the process.

%%%%%%%%%%%%%%%%%%%%%%%%%%%%%%%%%%%%%%%%%%%%%%%%%%%%%%%%%%%%%%%%%%%%%%%%%%%%%%%%
\subsection{Command Line Processing}
\label{sec:commandline}

The effect of redirection files can also be achieved by invoking
the \LaTeX{} compiler with a more elaborate command line.
Most conveniently this should be done as part
of a shell script or a |Makefile|.

When using \textsf{childdoc} in the main file, the following
command lines effectively perform a redirection
(note that depending on the shell being used,
backslashes may have to be doubled: `|\|' $\to$ `|\\|'):
%
\begin{center}
|... -jobname "|\textit{target}|" |\\|"|[\textit{flags}]%
|\input{childdoc.def}\childdocforward[|\textit{main}|]{|\textit{dest}|}"|
\end{center}
%
Here \textit{target} is the name of the output file,
\textit{main} is the name of the main file
and \textit{dest} is the name of the main or child file to be processed
(all filenames without extensions).
The optional argument \textit{main} can be omitted
if \textit{main} matches \textit{dest}.
Optionally, compilation \textit{flags} can be defined via |\def| commands.
This command line makes the \TeX{} engine believe
it is compiling the file \textit{target}
whose content is specified as the latter parameter.
The provided code then forwards the processing to
\textit{main} or \textit{dest} as described in \secref{sec:forward}.

%%%%%%%%%%%%%%%%%%%%%%%%%%%%%%%%%%%%%%%%%%%%%%%%%%%%%%%%%%%%%%%%%%%%%%%%%%%%%%%%
\subsection{Include by Input}
\label{sec:input}

Including child documents by |\include| has some restrictions by design.
Most notably, the content of a child document always occupies
its own set of pages; pages cannot be shared between child documents.
Usually, this behaviour makes perfect sense
because each child document contain an essential part of the document.
However, in some situations it may be desirable to compose
a document from a collection of parts
without having mandatory page breaks between then.
For this case, the package
provides a mechanism to include parts
by |\input| which can also be processed individually.
However, by construction this mechanism
requires manual handling of the content to be output.

%%%%%%%%%%%%%%%%%%%%%%%%%%%%%%%%%%%%%%%%
\DescribeMacro{\ifchilddocmanual}
The main file should be prepared as usual, see \secref{sec:include}.
However, the document body must make a distinction
between processing of an individual part and of the main document, e.g.:
%
\begin{center}
\begin{tabular}{l}
|\ifchilddocmanual|\\
|\input{\childdocname}|\\
|\||else|\\
\textit{document body with }|\input{|\textit{part}|}|\\
|\||fi|
\end{tabular}
\end{center}
%
The conditional |\ifchilddocmanual| is true whenever
a part to be included by |\input| is being compiled,
and the name of the part is stored in |\childdocname|.

%%%%%%%%%%%%%%%%%%%%%%%%%%%%%%%%%%%%%%%%
\DescribeMacro{\childdocby}
Each part to be included by |\input| should start with:
%
\begin{center}
\begin{tabular}{l}
|\input{childdoc.def}|\\
|\childdocby{|\textit{main}|}|\\
\end{tabular}
\end{center}
%
The directive |\childdocby| is similar to |\childdocof|
described in \secref{sec:include},
but the subsequent selection of content must be done manually.
To that end, both |\ifchilddoc| and |\ifchilddocmanual|
will be true upon processing of a part,
and the name of the part is stored in |\childdocname|.
Note that |\jobname| will be set to the filename of the current part
so that each part receives an individual |.aux| file
that does not interfere with the |.aux| file(s) of the main document.
This behaviour can be altered by the alternative form
|\childdocby[*]{|\textit{main}|}| (with a non-empty optional argument)
which uses the |.aux| file of the main document
by setting |\jobname| to \textit{main}.

%%%%%%%%%%%%%%%%%%%%%%%%%%%%%%%%%%%%%%%%%%%%%%%%%%%%%%%%%%%%%%%%%%%%%%%%%%%%%%%%
\subsection{Driver Development}
\label{sec:driver}

The \textsf{childdoc} mechanism can also be use for the development
of definition files such as \LaTeX{} styles or classes.
This case differs from the above setup with multiple parts
included by |\include| in that no |\includeonly| should be invoked.
This can be achieved by starting the include file
(before |\ProvidesPackage|) with:
%
\begin{center}
\begin{tabular}{l}
|\input{childdoc.def}|\\
|\childdocforward{|\textit{main}|}|\\
\end{tabular}
\end{center}
%
or alternatively with:
%
\begin{center}
\begin{tabular}{l}
|\input{childdoc.def}|\\
|\childdocby{|\textit{main}|}|\\
\end{tabular}
\end{center}
%
Both forms have slightly different effects as described above.
The main file is prepared as usual, see \secref{sec:include}.

%%%%%%%%%%%%%%%%%%%%%%%%%%%%%%%%%%%%%%%%%%%%%%%%%%%%%%%%%%%%%%%%%%%%%%%%%%%%%%%%
\subsection{Legacy Detection}
\label{sec:detection}

The directive |\childdocmain| in the main file can detect
whether the complete document or merely a child is to be compiled
even without using the directive |\childdocof|.
This method is deprecated because it is less robust
and there is no compelling reason to use it;
it is merely provided for backward compatibility
and it may be removed in future versions.

If the detection mechanism is to be used,
it is mandatory to correctly specify
the filename of the main file as the argument of |\childdocmain|:
%
\begin{center}
\begin{tabular}{l}
|\input{childdoc.def}|\\
|\childdocmain{|\textit{main}|}|\\
\end{tabular}
\end{center}
%
If |\jobname| does not match the argument \textit{main} of |\childdocmain|,
it is assumed that |\jobname| points to the child file to be compiled.
When using |\childdocmain| with the main file specified as argument,
it suffices to start a child file
with just |\input{|\textit{main}|}|
without loading of the package and using |\childdocof|.
If instead all processing is done
with the appropriate \textsf{childdoc} directives,
the argument of \textit{main} of |\childdocmain| can be empty.

An alternative version of the command line processing described
in \secref{sec:commandline} using the detection mechanism reads:
%
\begin{center}
|... -jobname "|\textit{target}|" "|[\textit{flags}]%
[|\def\jobname{|\textit{dest}|}|]|\input{|\textit{main}|}"|
\end{center}

%%%%%%%%%%%%%%%%%%%%%%%%%%%%%%%%%%%%%%%%%%%%%%%%%%%%%%%%%%%%%%%%%%%%%%%%%%%%%%%%
\subsection{Manual Code}
\label{sec:manual}

In case one cannot be certain whether the definitions file |childdoc.def|
is installed on the target \TeX{} distribution
and one prefers not to ship it,
it is conceivable to paste a few relevant commands into the sources.

To that end, drop all statements |\input{childdoc.def}|
and perform the replacements as outlined below.
Instead of |\childdocmain{|\textit{main}|}| add the following code
to the top of the main file:
%
\begin{center}
\begin{tabular}{l}
|\||ifdefined\childdocname\endinput\||fi\newif\ifchilddoc|\\
|\edef\childdocname{\scantokens\expandafter{\jobname\noexpand}}|\\
|\def\childdocmain{|\textit{main}|}\||ifx\childdocmain\childdocname\||else|\\
|\childdoctrue\includeonly{\childdocname}\let\jobname\childdocmain\||fi|\\
\end{tabular}
\end{center}
%
Instead of |\childdocof{|\textit{main}|}| just include the main file
at the top of each child file:
%
\begin{center}
|\input{|\textit{main}|}|
\end{center}
%
A simple redirection |\childdocforward{|\textit{dest}|}| is achieved by:
%
\begin{center}
|\def\jobname{|\textit{dest}|}\input{\jobname}|
\end{center}
%
The redirection with prefix
|\childdocforwardprefix[|\textit{prefix}|]{|\textit{dest}|}|
is accomplished by:
%
\begin{center}
\begin{tabular}{l}
|{\edef\jobname{\scantokens\expandafter{\jobname\noexpand}}|\\
|\def\redirectjob |\textit{prefix}|#1~~~{\gdef\jobname{|\textit{dest}|#1}}|\\
|\expandafter\redirectjob\jobname~~~}\input{\jobname}|
\end{tabular}
\end{center}

In an alternative approach,
child documents can be compiled by a specific command line
without additional code or specific definitions:
%
\begin{center}
|... -jobname "|\textit{target}|" "|[\textit{flags}]%
|\includeonly{|\textit{dest}|}\input{|\textit{main}|}"|
\end{center}
%

%%%%%%%%%%%%%%%%%%%%%%%%%%%%%%%%%%%%%%%%%%%%%%%%%%%%%%%%%%%%%%%%%%%%%%%%%%%%%%%%
%%%%%%%%%%%%%%%%%%%%%%%%%%%%%%%%%%%%%%%%%%%%%%%%%%%%%%%%%%%%%%%%%%%%%%%%%%%%%%%%
\section{Information}

%%%%%%%%%%%%%%%%%%%%%%%%%%%%%%%%%%%%%%%%%%%%%%%%%%%%%%%%%%%%%%%%%%%%%%%%%%%%%%%%
\subsection{Copyright}

Copyright \copyright{} 2017--2018 Niklas Beisert

This work may be distributed and/or modified under the
conditions of the \LaTeX{} Project Public License, either version 1.3
of this license or (at your option) any later version.
The latest version of this license is in
  \url{http://www.latex-project.org/lppl.txt}
and version 1.3 or later is part of all distributions of \LaTeX{}
version 2005/12/01 or later.

This work has the LPPL maintenance status `maintained'.

The Current Maintainer of this work is Niklas Beisert.

This work consists of the files |README.txt|, |childdoc.ins| and |childdoc.dtx|
as well as the derived files |childdoc.def|, |cdocsamp.tex|
with |cdocsch1.tex|, |cdocsch2.tex|, |cdocspt3.tex|, |cdocspt4.tex|,
|cdocsdrf.tex|, |cdocsfn1.tex|, |cdocsfn2.tex|
as well as |childdoc.pdf|.

%%%%%%%%%%%%%%%%%%%%%%%%%%%%%%%%%%%%%%%%%%%%%%%%%%%%%%%%%%%%%%%%%%%%%%%%%%%%%%%%
\subsection{Files and Installation}

The package consists of the files:
%
\begin{center}
\begin{tabular}{ll}
    |README.txt|   & readme file \\
    |childdoc.ins| & installation file \\
    |childdoc.dtx| & source file \\
    |childdoc.def| & definition file \\
    |cdocsamp.tex| & sample main file \\
    |cdocsch1.tex| & sample include file \\
    |cdocsch2.tex| & sample include file \\
    |cdocspt3.tex| & sample part file \\
    |cdocspt4.tex| & sample part file \\
    |cdocsdrf.tex| & sample redirection file \\
    |cdocsfn1.tex| & sample redirection file \\
    |cdocsfn2.tex| & sample redirection file \\
    |childdoc.pdf| & manual
\end{tabular}
\end{center}
%
The distribution consists of the files
|README.txt|, |childdoc.ins| and |childdoc.dtx|.
%
\begin{itemize}
\item
Run (pdf)\LaTeX{} on |childdoc.dtx|
to compile the manual |childdoc.pdf| (this file).
\item
Run \LaTeX{} on |childdoc.ins| to create the definitions file |childdoc.def|
and the sample |cdocsamp.tex| with include files
|cdocsch1.tex|, |cdocsch2.tex|, |cdocspt3.tex|, |cdocspt4.tex|,
|cdocsdrf.tex|, |cdocsfn1.tex|, |cdocsfn2.tex|.
Then copy the file |childdoc.def| to an appropriate directory of your \LaTeX{}
distribution, e.g.\ \textit{texmf-root}|/tex/latex/childdoc|.
\end{itemize}

%%%%%%%%%%%%%%%%%%%%%%%%%%%%%%%%%%%%%%%%%%%%%%%%%%%%%%%%%%%%%%%%%%%%%%%%%%%%%%%%
\subsection{Related CTAN Packages}

There are several other packages which offer a similar functionality:
%
\begin{itemize}
\item
The packages
\href{http://ctan.org/pkg/docmute}{\textsf{docmute}},
\href{http://ctan.org/pkg/includex}{\textsf{includex}} and
\href{http://ctan.org/pkg/standalone}{\textsf{standalone}}
provide commands to include only the document body of
a child file thus allowing both files to be compiled individually.
\item
The packages \href{http://ctan.org/pkg/subdocs}{\textsf{subdocs}}
and \href{http://ctan.org/pkg/subfiles}{\textsf{subfiles}}
provide structures in which the main and child documents can be
encapsulated and allowing them to be compiled individually.
The inclusion mechanism is different from the conventional |\include|.
\item
The package \href{http://ctan.org/pkg/combine}{\textsf{combine}}
is an elaborate solution to combine several documents into one.
\end{itemize}
%
See also the CTAN topic \href{http://ctan.org/topic/subdocs}{\textsf{subdocs}}
for further related packages.
The present package differs from the above solutions in that
a document structure constructed with the conventional |\include| mechanism
just needs two extra commands at the top of every file
such that all constituent files can be compiled individually.

%%%%%%%%%%%%%%%%%%%%%%%%%%%%%%%%%%%%%%%%%%%%%%%%%%%%%%%%%%%%%%%%%%%%%%%%%%%%%%%%
%\subsection{Feature Suggestions}
%
%The following is a list of features which may be useful for future
%versions of this package:
%%
%\begin{itemize}
%\item
%\ldots
%\end{itemize}

%%%%%%%%%%%%%%%%%%%%%%%%%%%%%%%%%%%%%%%%%%%%%%%%%%%%%%%%%%%%%%%%%%%%%%%%%%%%%%%%
\subsection{Revision History}

%%%%%%%%%%%%%%%%%%%%%%%%%%%%%%%%%%%%%%%%
\paragraph{v2.0:} 2018/12/30

\begin{itemize}
\item
immediate forward processing
\item
added |\childdocby| mechanism
\item
manual restructured
\end{itemize}

%%%%%%%%%%%%%%%%%%%%%%%%%%%%%%%%%%%%%%%%
\paragraph{v1.6:} 2018/01/17

\begin{itemize}
\item
application for development of include files
\item
corrections to manual
\end{itemize}

%%%%%%%%%%%%%%%%%%%%%%%%%%%%%%%%%%%%%%%%
\paragraph{v1.5:} 2017/05/21

\begin{itemize}
\item
more complete structuring introduced
\item
|\childdocof| introduced
\item
|\childdoc| renamed to |\childdocmain|
\item
|\childredirect| renamed to |\childdocforward| and |\childdocforwardprefix|
and functionality expanded
\end{itemize}

%%%%%%%%%%%%%%%%%%%%%%%%%%%%%%%%%%%%%%%%
\paragraph{v1.0:} 2017/04/27

\begin{itemize}
\item
manual and install package
\item
first version published on CTAN
\end{itemize}

%%%%%%%%%%%%%%%%%%%%%%%%%%%%%%%%%%%%%%%%
\paragraph{v0.6:} 2017/04/26

\begin{itemize}
\item
redirection mechanism added
\end{itemize}

%%%%%%%%%%%%%%%%%%%%%%%%%%%%%%%%%%%%%%%%
\paragraph{v0.5:} 2017/04/26

\begin{itemize}
\item
functionality in definition file
\end{itemize}


%%%%%%%%%%%%%%%%%%%%%%%%%%%%%%%%%%%%%%%%%%%%%%%%%%%%%%%%%%%%%%%%%%%%%%%%%%%%%%%%
%%%%%%%%%%%%%%%%%%%%%%%%%%%%%%%%%%%%%%%%%%%%%%%%%%%%%%%%%%%%%%%%%%%%%%%%%%%%%%%%
%%%%%%%%%%%%%%%%%%%%%%%%%%%%%%%%%%%%%%%%%%%%%%%%%%%%%%%%%%%%%%%%%%%%%%%%%%%%%%%%
\appendix

\settowidth\MacroIndent{\rmfamily\scriptsize 000\ }

 \DocInput{childdoc.dtx}

\end{document}
%</driver>
% \fi
%
% %%%%%%%%%%%%%%%%%%%%%%%%%%%%%%%%%%%%%%%%%%%%%%%%%%%%%%%%%%%%%%%%%%%%%%%%%%%%%%
% %%%%%%%%%%%%%%%%%%%%%%%%%%%%%%%%%%%%%%%%%%%%%%%%%%%%%%%%%%%%%%%%%%%%%%%%%%%%%%
% \section{Sample}
%\iffalse
%<*samplemain>
%\fi
%
% The following presents a sample document
% with two chapters, two parts, a title page,
% a compile flag as well as three forwarding files to set the flag.
% It consists of eight |.tex| files:
% \begin{center}
% \begin{tabular}{ll}
% |cdocsamp.tex|&main file\\
% |cdocsch1.tex|&include file for chapter 1\\
% |cdocsch2.tex|&include file for chapter 2\\
% |cdocspt3.tex|&include file for part 3\\
% |cdocspt4.tex|&include file for part 4\\
% |cdocsdrf.tex|&forwarding file for main file in draft mode\\
% |cdocsfi1.tex|&forwarding file for final version of chapter 1\\
% |cdocsfi2.tex|&forwarding file for final version of chapter 2\\
% \end{tabular}
% \end{center}
% Each of the eight files can be compiled directly by the \LaTeX{} compiler.
%
% %%%%%%%%%%%%%%%%%%%%%%%%%%%%%%%%%%%%%%
% \paragraph{Main File.}
%
% The main file is called |cdocsamp.tex|.
%
% Load the \textsf{childdoc} definitions and
% declare the filename for the main document:
%    \begin{macrocode}
\input{childdoc.def}
\childdocmain{}
%    \end{macrocode}

% Optional override for |\version| flag:
%    \begin{macrocode}
%%\ifchilddoc\else\providecommand{\version}{draft}\fi
%    \end{macrocode}

% Define the default values for the |\version| flag
% (|final| for the main file and |draft| for childs):
%    \begin{macrocode}
\ifchilddoc
\providecommand{\version}{draft}
\else
\providecommand{\version}{final}
\fi
%    \end{macrocode}

% Load the standard document class:
%    \begin{macrocode}
\documentclass[12pt]{article}
%    \end{macrocode}

% Start the document body:
%    \begin{macrocode}
\begin{document}
%    \end{macrocode}

% Declare a title page.
% Print title, part of document being processed and version flag:
%    \begin{macrocode}
\addtocounter{page}{-1}
\begin{center}
{\LARGE\bfseries{}childdoc example\par}
\vspace{1cm}
\ifchilddoc
\ifchilddocmanual part\else chapter\fi:
`\childdocname' of `\childdocjob'\par
\else
main document: `\childdocjob'\par
\fi
version: \version\par
\end{center}
\newpage
%    \end{macrocode}

% Manually include selected file,
% otherwise process as usual:
%    \begin{macrocode}
\ifchilddocmanual
\section*{part `\childdocname'}
\input{\childdocname}
\else
%    \end{macrocode}

% Include the two chapters:
%    \begin{macrocode}
\include{cdocsch1}
\include{cdocsch2}
%    \end{macrocode}

% Include the two parts unless only chapters should be displayed:
%    \begin{macrocode}
\ifchilddoc\else
\section{part three}
\input{cdocspt3}
\section{part four}
\input{cdocspt4}
\fi
%    \end{macrocode}

% Process as usual until here:
%    \begin{macrocode}
\fi
%    \end{macrocode}

% End of document body:
%    \begin{macrocode}
\end{document}
%    \end{macrocode}
%\iffalse
%</samplemain>
%\fi
%
% %%%%%%%%%%%%%%%%%%%%%%%%%%%%%%%%%%%%%%
% \paragraph{Chapter Include Files.}
%
% The include files are called |cdocsch1.tex| and |cdocsch2.tex|.
%
%\iffalse
%<*samplechap1|samplechap2>
%\fi

% Optional override for |\version| flag:
%    \begin{macrocode}
%%\providecommand{\version}{final}
%    \end{macrocode}

% Include the main document:
%    \begin{macrocode}
\input{childdoc.def}
\childdocof{cdocsamp}
%    \end{macrocode}

%\iffalse
%</samplechap1|samplechap2>
%\fi
%
%\iffalse
%<*samplechap1>
%\fi
% Some text for chapter 1:
%    \begin{macrocode}
\section{one}
some text in chapter one
%    \end{macrocode}

%\iffalse
%</samplechap1>
%\fi
% Some text for chapter 2:
%\iffalse
%<*samplechap2>
%\fi
%    \begin{macrocode}
\section{two}
more text in chapter two
%    \end{macrocode}

%\iffalse
%</samplechap2>
%\fi
%
% %%%%%%%%%%%%%%%%%%%%%%%%%%%%%%%%%%%%%%
% \paragraph{Part Include Files.}
%
% The include files are called |cdocspt3.tex| and |cdocspt4.tex|.
%
%\iffalse
%<*samplepart3|samplepart4>
%\fi

% Optional override for |\version| flag:
%    \begin{macrocode}
%%\providecommand{\version}{final}
%    \end{macrocode}

% Include the main document:
%    \begin{macrocode}
\input{childdoc.def}
\childdocby{cdocsamp}
%    \end{macrocode}

%\iffalse
%</samplepart3|samplepart4>
%\fi
%
%\iffalse
%<*samplepart3>
%\fi
% Some text for part 3:
%    \begin{macrocode}
some text in part three
%    \end{macrocode}

%\iffalse
%</samplepart3>
%\fi
% Some text for part 4:
%\iffalse
%<*samplepart4>
%\fi
%    \begin{macrocode}
more text in part four
%    \end{macrocode}

%\iffalse
%</samplepart4>
%\fi
%
% %%%%%%%%%%%%%%%%%%%%%%%%%%%%%%%%%%%%%%
% \paragraph{Forwarding for a Complete Draft.}
%
% The following forwarding file |cdocsdrf.tex|
% compiles the main document in draft mode:
%\iffalse
%<*sampledraft>
%\fi
%    \begin{macrocode}
\def\version{draft}
\input{childdoc.def}
\childdocforward{cdocsamp}
%    \end{macrocode}

%\iffalse
%</sampledraft>
%\fi
%
% %%%%%%%%%%%%%%%%%%%%%%%%%%%%%%%%%%%%%%
% \paragraph{Forwarding for Final Version of the Chapters.}
%
% The following forwarding files |cdocsfn1.tex| and |cdocsfn2.tex|
% (with identical content)
% compile the final versions of the child documents
% |cdocsch1.tex| and |cdocsch2.tex|, respectively:
%\iffalse
%<*samplefinal>
%\fi
%    \begin{macrocode}
\def\version{final}
\input{childdoc.def}
\childdocforwardprefix[cdocsamp]{cdocsfn}{cdocsch}
%    \end{macrocode}

%\iffalse
%</samplefinal>
%\fi
%
% %%%%%%%%%%%%%%%%%%%%%%%%%%%%%%%%%%%%%%
% \paragraph{Command Line Processing.}
%
% The following three command lines generate the output files
% |cdocscld|, |cdocscl1| and |cdocscl2|
% which should be identical to
% |cdocsdrf|, |cdocsch1| and |cdocsfn2|, respectively:
% \begin{center}
% \begin{tabular}{l}
% |latex -jobname cdocscld \|\\
% |  "\def\version{draft}\input{childdoc.def}\childdocforward{cdocsamp}"|\\
% |latex -jobname cdocscl1 \|\\
% |  "\input{childdoc.def}\childdocforward[cdocsamp]{cdocsch1}"|\\
% |latex -jobname cdocscl2 \|\\
% |  "\def\version{final}\input{childdoc.def}\childdocforward{cdocsch2}"|
% \end{tabular}
% \end{center}
% Note that the trailing backslash on each first line
% merely continues the input to the second line
% (for convenient cut ant paste).
% Furthermore, the command |latex| can be replaced by any
% of its alternative versions such as |pdflatex|.
%
% %%%%%%%%%%%%%%%%%%%%%%%%%%%%%%%%%%%%%%%%%%%%%%%%%%%%%%%%%%%%%%%%%%%%%%%%%%%%%%
% %%%%%%%%%%%%%%%%%%%%%%%%%%%%%%%%%%%%%%%%%%%%%%%%%%%%%%%%%%%%%%%%%%%%%%%%%%%%%%
% \section{Implementation}
%\iffalse
%<*package>
%\fi
%
% This section describes the definitions file |childdoc.def|.

% The definitions cannot be loaded using |\usepackage| or |\RequirePackage|
% which has a mechanism to prevent loading a style file more than once.
% When loading the definitions by means of |\input|
% multiple instances have to be prevented manually:
%\iffalse
%This code needs to be before the `\ProvidesFile' directive
%which is defined at the beginning of this file.
%Therefore it is also placed there and commented out here.
%</package>
%<*discard>
%\fi
%    \begin{macrocode}
\ifdefined\childdocmain\endinput\fi
%    \end{macrocode}
%\iffalse
%</discard>
%<*package>
%\fi
%
% \macro{\ifchilddoc}
% \macro{\ifchilddocmanual}
% The conditional |\ifchilddoc| tells whether a
% child (true) or main (false) document is being compiled.
% The conditional |\ifchilddocmanual| tells whether
% the |\includeonly| mechanism is used (false) or
% the selection of child files must be performed manually (true).
% The definitions initialise to false:
%    \begin{macrocode}
\newif\ifchilddoc
\newif\ifchilddocmanual
%    \end{macrocode}

% \macro{\childdocname}
% \macro{\childdocjob}
% The macro |\childdocname| stores the name of the main document
% to be compiled. The macro |\childdocjob| stores the name of
% the document on which the \LaTeX{} compiler was originally invoked.
% The content of |\jobname| cannot be compared
% to filenames specified in the source due to different catcodes.
% The following code rescans |\jobname|, stores the result
% in |\childdocname| and saves a copy in |\childdocjob|:
%    \begin{macrocode}
\edef\childdocname{\scantokens\expandafter{\jobname\noexpand}}
\let\childdocjob\childdocname
%    \end{macrocode}

% \macro{\childdocdisable}
% The macro |\childdocdisable| prevents the main file
% from being processed more than once.
% At this stage, the main document command |\childdocmain|
% is assumed to be called once again where it should do nothing.
% Any subsequent call to it should prevent
% a secondary processing of the main document
% It overwrites the forwarding commands
% |\childdocof| and |\childdocforward|
% with empty macros to prevent further inclusions of the main document:
%    \begin{macrocode}
\newcommand{\childdocdisable}
{
  \renewcommand{\childdocmain}[1]{\renewcommand{\childdocmain}[1]{\endinput}}
  \renewcommand{\childdocof}[1]{}
  \renewcommand{\childdocby}[2][]{}
  \renewcommand{\childdocforward}[2][]{}
  \renewcommand{\childdocdisable}{}
}
%    \end{macrocode}

% \macro{\childdocmain}
% The macro |\childdocmain| is to be called at the top of the main file
% with nothing or the main filename (without extension) as argument.
% First, it breaks loops.
% If the argument is not empty and does not match |\childdocname|
% (which is set by the first inclusion of |childdoc.def|),
% |\ifchilddoc| is set to true, |\includeonly| is applied to the child file
% and |\jobname| is set to the main file
% (for proper handling of |.aux| files):
%    \begin{macrocode}
\newcommand{\childdocmain}[1]
{
  \childdocdisable\childdocmain{}
  \if?#1?\else
    \begingroup
      \def\childdoctmp{#1}
      \ifx\childdoctmp\childdocname
        \def\childdoctmp{}
      \else
        \def\childdoctmp
        {
          \childdoctrue
          \includeonly{\childdocname}
          \def\childdocjob{#1}
          \def\jobname{#1}
        }
      \fi
      \expandafter
    \endgroup
    \childdoctmp
  \fi
}
%    \end{macrocode}

% \macro{\childdocof}
% The command |\childdocof| redirects
% compilation to the main file |#1|.
%    \begin{macrocode}
\newcommand{\childdocof}[1]
{
  \childdocdisable
  \childdoctrue
  \includeonly{\childdocname}
  \def\jobname{#1}
  \def\childdocjob{#1}
  \input{#1}
}
%    \end{macrocode}

% \macro{\childdocby}
% The command |\childdocby| ....
%    \begin{macrocode}
\newcommand{\childdocby}[2][]
{
  \childdocdisable
  \childdoctrue
  \childdocmanualtrue
  \if?#1?\else
    \def\jobname{#2}
  \fi
  \def\childdocjob{#2}
  \input{#2}
  \endinput
}
%    \end{macrocode}

% \macro{\childdocforward}
% The command |\childdocforward| redirects
% compilation to the main file or
% (if the optional argument is given) a child file.
% Parameters are set as if the main file
% or a child file starting with |\childdocof| was compiled.
% Then compilation is handed over to the main file:
%    \begin{macrocode}
\newcommand{\childdocforward}[2][]
{
  \begingroup
    \if?#1?
      \def\childdoctmp
      {
        \def\childdocname{#2}
        \def\childdocjob{#2}
        \def\jobname{#2}
        \input{#2}
        \endinput
      }
    \else
      \def\childdoctmp
      {
        \childdocdisable
        \def\childdocname{#2}
        \childdoctrue
        \includeonly{#2}
        \def\childdocjob{#1}
        \def\jobname{#1}
        \input{#1}
        \endinput
      }
    \fi
    \expandafter
  \endgroup
  \childdoctmp
}
%    \end{macrocode}

% \macro{\childdocforwardprefix}
% The command |\childdocforwardprefix| redirects
% compilation to the main or a child file by means of a pattern.
% The prefix |#1| in the current filename is replaced by |#2|
% and the suffix of the current filename is kept
% (it is assumed that the filename does not contain the substring `|~~~|'
% which is used as a delimiter).
% Compilation is handed over to the new file by |\childdocforward|:
%    \begin{macrocode}
\newcommand{\childdocforwardprefix}[3][]
{
  \begingroup
    \def\childdocextract #2##1~~~{\def\childdoctmp{\childdocforward[#1]{#3##1}}}
    \expandafter\childdocextract\childdocname~~~
    \expandafter
  \endgroup
  \childdoctmp
}
%    \end{macrocode}

% \macro{\childdoc}
% The deprecated macro |\childdoc| is a legacy version of |\childdocmain|:
%    \begin{macrocode}
\newcommand{\childdoc}{\childdocmain}
%    \end{macrocode}

% \macro{\childdocredirect}
% The deprecated macro |\childdocredirect| is a legacy version
% of |\childdocforward| and |\childdocforwardprefix|:
%    \begin{macrocode}
\newcommand{\childdocredirect}[2][]
{
  \begingroup
    \if?#1?
      \def\childdoctmp{\childdocforward{#2}}
    \else
      \def\childdoctmp{\childdocforwardprefix{#1}{#2}}
    \fi
    \expandafter
  \endgroup
  \childdoctmp
}
%    \end{macrocode}

%\iffalse
%</package>
%\fi
%
\endinput
|\\
|\childdocforward{|\textit{main}|}|\\
\end{tabular}
\end{center}
%
or alternatively with:
%
\begin{center}
\begin{tabular}{l}
|% \iffalse
%
% childdoc.dtx Copyright (C) 2017-2018 Niklas Beisert
%
% This work may be distributed and/or modified under the
% conditions of the LaTeX Project Public License, either version 1.3
% of this license or (at your option) any later version.
% The latest version of this license is in
%   http://www.latex-project.org/lppl.txt
% and version 1.3 or later is part of all distributions of LaTeX
% version 2005/12/01 or later.
%
% This work has the LPPL maintenance status `maintained'.
%
% The Current Maintainer of this work is Niklas Beisert.
%
% This work consists of the files childdoc.dtx and childdoc.ins
% and the derived files childdoc.def and cdocsamp.tex with
% cdocsch1.tex, cdocsch2.tex, cdocsdrf.tex, cdocsfn1.tex, cdocsfn2.tex.
%
%<package>\ifdefined\childdocmain\endinput\fi
%<package>\ProvidesFile{childdoc.def}[2018/12/30 v2.0 child document driver]
%<samplemain>\ProvidesFile{cdocsamp.tex}[2018/12/30 v2.0 sample for childdoc]
%<*driver>
%\ProvidesFile{childdoc.drv}[2018/12/30 v2.0 childdoc reference manual file]
\PassOptionsToClass{10pt,a4paper}{article}
\documentclass{ltxdoc}

\usepackage[margin=35mm]{geometry}
\usepackage{hyperref}
\usepackage{hyperxmp}
\usepackage[usenames]{color}

\hypersetup{colorlinks=true}
\hypersetup{pdfstartview=FitH}
\hypersetup{pdfpagemode=UseNone}
\hypersetup{pdfsource={}}
\hypersetup{pdflang={en-UK}}
\hypersetup{pdfcopyright={Copyright 2017-2018 Niklas Beisert.
  This work may be distributed and/or modified under the
  conditions of the LaTeX Project Public License, either version 1.3
  of this license or (at your option) any later version.}}
\hypersetup{pdflicenseurl={http://www.latex-project.org/lppl.txt}}
\hypersetup{pdfcontactaddress={ETH Zurich, ITP, HIT K,
  Wolfgang-Pauli-Strasse 27}}
\hypersetup{pdfcontactpostcode={8093}}
\hypersetup{pdfcontactcity={Zurich}}
\hypersetup{pdfcontactcountry={Switzerland}}
\hypersetup{pdfcontactemail={nbeisert@itp.phys.ethz.ch}}
\hypersetup{pdfcontacturl={http://people.phys.ethz.ch/\xmptilde nbeisert/}}

\newcommand{\secref}[1]{\hyperref[#1]{section \ref*{#1}}}

\parskip1ex
\parindent0pt
\let\olditemize\itemize
\def\itemize{\olditemize\parskip0pt}

\begin{document}

\title{The \textsf{childdoc} Package}
\hypersetup{pdftitle={The childdoc Package}}
\author{Niklas Beisert\\[2ex]
  Institut f\"ur Theoretische Physik\\
  Eidgen\"ossische Technische Hochschule Z\"urich\\
  Wolfgang-Pauli-Strasse 27, 8093 Z\"urich, Switzerland\\[1ex]
  \href{mailto:nbeisert@itp.phys.ethz.ch}
  {\texttt{nbeisert@itp.phys.ethz.ch}}}
\hypersetup{pdfauthor={Niklas Beisert}}
\hypersetup{pdfsubject={Manual for the LaTeX2e Package childdoc}}
\date{30 December 2018, \textsf{v2.0}}
\maketitle

\begin{abstract}\noindent
\textsf{childdoc} is a \LaTeXe{} package
that enables the direct compilation
of document sections included by |\include|
to individual files.
\end{abstract}

\begingroup
\parskip0ex
\tableofcontents
\endgroup

%%%%%%%%%%%%%%%%%%%%%%%%%%%%%%%%%%%%%%%%%%%%%%%%%%%%%%%%%%%%%%%%%%%%%%%%%%%%%%%%
%%%%%%%%%%%%%%%%%%%%%%%%%%%%%%%%%%%%%%%%%%%%%%%%%%%%%%%%%%%%%%%%%%%%%%%%%%%%%%%%
\section{Introduction}

\LaTeX{} provides a mechanism to structure a large document (such as a book)
into a main file and several child files (containing the chapters)
using the |\include| command.
This mechanism is beneficial for documents
which span hundreds of pages in order to
make the source file(s) more manageable.
Moreover, compilation can be restricted to
selected child files by means of the |\includeonly| command.
The latter feature can be used to reduce the compilation time while editing
(this was significantly more useful in the earlier days of \LaTeX{})
or to generate a smaller document which is easier to navigate.
Another application of |\includeonly| is to generate
documents consisting of selected parts of the complete document.

However, there are a few drawbacks of the plain |\include| mechanism:
\begin{itemize}
\item
The child files cannot be compiled on their own,
they can only be compiled via the main file.
A naive editing environment
(such as a text editor with an option
to have the current file processed by \LaTeX)
may require one to switch to the main file before compiling;
attempting to compile the child file produces errors.
\item
The main file must be modified (each time)
to adjust the |\includeonly| command
to the present needs. This easily leaves the main file in a messy state.
\item
The generated document will always carry the filename
of the main document. This is inconvenient if
several child files are to be compiled and
to be kept for distribution.
\end{itemize}

The present package provides a simple interface
to make child files individually compilable by \LaTeX{}.
Compiling a child file then has the same effect as compiling
the main file with an |\includeonly| command
to select the appropriate child.
Moreover the generated document will carry the name of the child
rather than the main file.
This resolves all three above issues.

This feature is meant to make the editing of books,
thesis documents and lecture notes somewhat more convenient.
However, the package can also be used efficiently for
composing a series of documents (such as exercise sheets)
which are typically distributed individually.
It then assists the author in generating the individual documents
(potentially in different versions)
as well as a document containing the collected series.
Another application is in developing style files
or other kinds of included material
where compilation of the style file could redirect
to a sample or test file.

%%%%%%%%%%%%%%%%%%%%%%%%%%%%%%%%%%%%%%%%%%%%%%%%%%%%%%%%%%%%%%%%%%%%%%%%%%%%%%%%
%%%%%%%%%%%%%%%%%%%%%%%%%%%%%%%%%%%%%%%%%%%%%%%%%%%%%%%%%%%%%%%%%%%%%%%%%%%%%%%%
\section{Usage}

First of all, the package \textsf{childdoc} is \emph{not} a standard
\LaTeXe{} |.sty| style file! Therefore it needs to be invoked in
a non-standard way.

%%%%%%%%%%%%%%%%%%%%%%%%%%%%%%%%%%%%%%%%%%%%%%%%%%%%%%%%%%%%%%%%%%%%%%%%%%%%%%%%
\subsection{Included Files}
\label{sec:include}

%%%%%%%%%%%%%%%%%%%%%%%%%%%%%%%%%%%%%%%%
\DescribeMacro{\childdocmain}
To use the package, add the commands
\begin{center}
\begin{tabular}{l}
|\input{childdoc.def}|\\
|\childdocmain{}|\\
\end{tabular}
\end{center}
at the very top of the main \LaTeX{} file,
in particular \emph{before} the |\documentclass| statement!
The argument of |\childdocmain| should be left empty
(but it must be present).

%%%%%%%%%%%%%%%%%%%%%%%%%%%%%%%%%%%%%%%%
\DescribeMacro{\childdocof}
Furthermore, add the commands
\begin{center}
\begin{tabular}{l}
|\input{childdoc.def}|\\
|\childdocof{|\textit{main}|}|\\
\end{tabular}
\end{center}
at the top of every child file \textit{child}
which is included by |\include{|\textit{child}|}|
from within the main file
(or at least for those files to be compiled individually).
The argument \textit{main} must be the filename of the main file.

There are a couple of
considerations in setting up the main and child documents:

%%%%%%%%%%%%%%%%%%%%%%%%%%%%%%%%%%%%%%%%
\paragraph{Restrictions.}

Please note the following restrictions:
\begin{itemize}
\item
|\childdocmain| must be called with one argument \textit{main}
to ensure compatibility with earlier version of the package.
It must either be empty (|\childdocmain{}|)
or precisely match the filename of the main file in which it is specified.
See \secref{sec:detection} for further information.
\item
The filename \textit{main} must be specified without the |.tex| extension.
\item
The filename \textit{main} is case sensitive
(even in case-insensitive file systems)
due to internal string comparison.
\item
The argument \textit{main} should be fully expanded, it cannot be a macro.
\item
Subdirectories and special characters should be avoided in filenames.
\item
The command |\childdocmain{|\textit{main}|}| must be followed by a whitespace.
It should not be followed immediately by another command
or by a comment mark `|%|'.
This is because the \TeX{} parser reads the token immediately following
the argument of |\childdocmain| and puts it
at the beginning of every child section;
however, a white\-space is ignored.
\end{itemize}

%%%%%%%%%%%%%%%%%%%%%%%%%%%%%%%%%%%%%%%%
\paragraph{Content of Main File.}

It is advisable to place all content in the child files included by |\include|.
Any output contained in the main file will appear in all child documents
unless suppressed manually;
it cannot be suppressed automatically by the |\includeonly| directive
and thus should normally be avoided.
A method to include some content in the main file
by means of conditional processing is described in \secref{sec:conditional}.

%%%%%%%%%%%%%%%%%%%%%%%%%%%%%%%%%%%%%%%%
\paragraph{Page Numbering.}

When only a part of the document is compiled,
the appropriate numbering of pages
(as well as other status parameters)
is determined from the |.aux| files.
The latter contain information from previous passes.
However this information needs to propagate through
all intermediate child documents.
Therefore the page numbering in child documents may well
be inconsistent until the complete document is compiled at least once.

A useful (if unconventional) way to always ensure a consistent
page numbering is to restart the numbering in each child document
and denote the pages by `\textit{child}|.|\textit{page}'
where \textit{child} represents the chapter/section number of the child file.
This can be achieved by the command
|\numberwithin{page}{|\textit{child}|}|
of the \textsf{amsmath} package
where \textit{child} can be |chapter| or |section|
depending on the chosen structuring.
Alternatively, one can modify the macro |\thepage| appropriately
and reset the counter |page| at the start of each child file.

%%%%%%%%%%%%%%%%%%%%%%%%%%%%%%%%%%%%%%%%%%%%%%%%%%%%%%%%%%%%%%%%%%%%%%%%%%%%%%%%
\subsection{Conditional Processing}
\label{sec:conditional}

The package provides a mechanism to compile different versions
of a document. To customise the versions further some conditional processing
can come in handy to distinguish which version is being compiled.
The package provides two macros to describe the compilation context:

%%%%%%%%%%%%%%%%%%%%%%%%%%%%%%%%%%%%%%%%
\DescribeMacro{\ifchilddoc}
The conditional |\ifchilddoc| distinguishes between the compilation of
child documents and the main document:
%
\begin{center}
|\ifchilddoc |\textit{child-code}| |[|\||else |\textit{main-code}]| \||fi|
\end{center}

%%%%%%%%%%%%%%%%%%%%%%%%%%%%%%%%%%%%%%%%
\DescribeMacro{\childdocname}
\DescribeMacro{\childdocjob}
The macro |\childdocname| contains the filename (without extension)
of the main or child file being processed.
Note that |\childdocjob| will always contain the name of the main file.

%%%%%%%%%%%%%%%%%%%%%%%%%%%%%%%%%%%%%%%%
\paragraph{Title Page.}

Conditional processing can be used to include a title or banner page
in the main document when proper precautions are taken.
Importantly, the code in the main file should ensure that the page counter
(as well as other status parameters which are stored in the |.aux| files)
takes the same value after the conditional processing.
Otherwise the page numbers may take divergent values
depending on which part is compiled.

For example, a title page could be declared by:
%
\begin{center}
\begin{tabular}{l}
|\ifchilddoc\||else|\\
|\addtocounter{page}{-1}|\\
\textit{code for title page}\\
|\newpage|\\
|\||fi|
\end{tabular}
\end{center}
%
A banner page for the child documents can be generated by:
%
\begin{center}
\begin{tabular}{l}
|\ifchilddoc|\\
|\addtocounter{page}{-1}|\\
\textit{code for banner page}\\
|\newpage|\\
|\||fi|
\end{tabular}
\end{center}
%
Here one could write a message such as:
\begin{center}
|This is the part \childdocname{} of \childdocjob{}.|
\end{center}

%%%%%%%%%%%%%%%%%%%%%%%%%%%%%%%%%%%%%%%%%%%%%%%%%%%%%%%%%%%%%%%%%%%%%%%%%%%%%%%%
\subsection{Flags}
\label{sec:flags}

The package makes it easy to generate different versions
of the main or child documents.
To this end compilation flags can be defined
and assigned different default values.
They will be particularly useful in conjunction
with the forwarding mechanism described in \secref{sec:forward}.

For example, it may be useful to have a flag |\version|
which can be set to |draft| or |final|.
The document source will contain some conditional code
depending on the value of |\version|.
Suppose further, the flag should default to |final| for the main file
and to |draft| for child files
which is a natural assignment for editing the document.
This is achieved by placing the following code
in the preamble of the main document
(below the |\childdocmain| directive):
%
\begin{center}
\begin{tabular}{l}
|\ifchilddoc|\\
|\providecommand{\version}{draft}|\\
|\||else|\\
|\providecommand{\version}{final}|\\
|\||fi|
\end{tabular}
\end{center}
%
The definition by |\providecommand| makes sure
that previous definitions are not overwritten.
Further statements |\providecommand{\version}{...}|
can thus be added before the above code to override it.

For the main file, one might add a line
(between |\childdocmain| and the above block)
%
\begin{center}
|%\ifchilddoc\||else\providecommand{\version}{draft}\||fi|
\end{center}
%
which can be uncommented to produce a draft version.
Likewise one can add a line to the very top of a child file
(above the |\childdocof{|\textit{main}|}| directive)
%
\begin{center}
|%\providecommand{\version}{final}|
\end{center}
%
which can be uncommented to produce the final version of this child document.

%%%%%%%%%%%%%%%%%%%%%%%%%%%%%%%%%%%%%%%%%%%%%%%%%%%%%%%%%%%%%%%%%%%%%%%%%%%%%%%%
\subsection{Forwarding}
\label{sec:forward}

Different versions of the main or child documents
using compilation flags as described in \secref{sec:flags}
can be (permanently) stored in different files
for convenient compilation, viewing and distribution.
To this end, the package defines a command
to pass on compilation to a different file:

%%%%%%%%%%%%%%%%%%%%%%%%%%%%%%%%%%%%%%%%
\DescribeMacro{\childdocforward}
The command |\childdocforward| redirects processing to
another source file:
%
\begin{center}
\begin{tabular}{l}
|\input{childdoc.def}|\\
|\childdocforward[|\textit{main}|]{|\textit{dest}|}|\\
\end{tabular}
\end{center}
%
The argument \textit{dest} is the destination file
(without extension).
It should be the main file or one of the child files.
Note that further \textsf{childdoc} directives
such as |\childdocof| and |\childdocforward|
in the indicated file will be processed in this form.
The optional argument \textit{main}
passes on directly to the main file \textit{main}
while pretending to compile the child \textit{dest}.
This form behaves as if \textit{dest}
issues |\childdocof{|\textit{main}|}| right away,
and no further \textsf{childdoc} directives will be processed.

%%%%%%%%%%%%%%%%%%%%%%%%%%%%%%%%%%%%%%%%
\DescribeMacro{\...prefix}
In the alternative form |\childdocforwardprefix|,
%
\begin{center}
\begin{tabular}{l}
|\input{childdoc.def}|\\
|\childdocforwardprefix[|\textit{main}|]{|\textit{prefix}|}{|\textit{dest}|}|
\end{tabular}
\end{center}
%
the destination file is determined by a pattern
depending on the current file:
To make this work, the current file must be called
`{\textit{prefix}\hspace{0.2em}\textit{suffix}}'
with \textit{prefix} matching precisely the argument.
Processing is then passed on to the file
`{\textit{dest}\hspace{0.2em}\textit{suffix}}'.
Surely, the same effect is achieved by
directly specifying the
argument `{\textit{dest}\hspace{0.2em}\textit{suffix}}'
in the first form.
However, that requires to set up a different file
for each child. With the alternative form of the command
all these files can have exactly the same content
which simplifies setting them up and maintaining them.

For example, the following file |draft.tex|
with a compilation flag |\version| as described in \secref{sec:flags}
compiles the main document as a draft:
%
\begin{center}
\begin{tabular}{l}
|\def\version{draft}|\\
|\input{childdoc.def}|\\
|\childdocforward{|\textit{main}|}|
\end{tabular}
\end{center}
%
Likewise, the following files |final|\textit{nn}|.tex|
compile the final version of the child document
|child|\textit{nn}|.tex|:
%
\begin{center}
\begin{tabular}{l}
|\def\version{final}|\\
|\input{childdoc.def}|\\
|\childdocforwardprefix{final}{child}|
\end{tabular}
\end{center}
%

Note that when several versions of a main file and/or of each child file
are to be generated, it may be convenient to set up a |Makefile| or
shell script to automatise the process.

%%%%%%%%%%%%%%%%%%%%%%%%%%%%%%%%%%%%%%%%%%%%%%%%%%%%%%%%%%%%%%%%%%%%%%%%%%%%%%%%
\subsection{Command Line Processing}
\label{sec:commandline}

The effect of redirection files can also be achieved by invoking
the \LaTeX{} compiler with a more elaborate command line.
Most conveniently this should be done as part
of a shell script or a |Makefile|.

When using \textsf{childdoc} in the main file, the following
command lines effectively perform a redirection
(note that depending on the shell being used,
backslashes may have to be doubled: `|\|' $\to$ `|\\|'):
%
\begin{center}
|... -jobname "|\textit{target}|" |\\|"|[\textit{flags}]%
|\input{childdoc.def}\childdocforward[|\textit{main}|]{|\textit{dest}|}"|
\end{center}
%
Here \textit{target} is the name of the output file,
\textit{main} is the name of the main file
and \textit{dest} is the name of the main or child file to be processed
(all filenames without extensions).
The optional argument \textit{main} can be omitted
if \textit{main} matches \textit{dest}.
Optionally, compilation \textit{flags} can be defined via |\def| commands.
This command line makes the \TeX{} engine believe
it is compiling the file \textit{target}
whose content is specified as the latter parameter.
The provided code then forwards the processing to
\textit{main} or \textit{dest} as described in \secref{sec:forward}.

%%%%%%%%%%%%%%%%%%%%%%%%%%%%%%%%%%%%%%%%%%%%%%%%%%%%%%%%%%%%%%%%%%%%%%%%%%%%%%%%
\subsection{Include by Input}
\label{sec:input}

Including child documents by |\include| has some restrictions by design.
Most notably, the content of a child document always occupies
its own set of pages; pages cannot be shared between child documents.
Usually, this behaviour makes perfect sense
because each child document contain an essential part of the document.
However, in some situations it may be desirable to compose
a document from a collection of parts
without having mandatory page breaks between then.
For this case, the package
provides a mechanism to include parts
by |\input| which can also be processed individually.
However, by construction this mechanism
requires manual handling of the content to be output.

%%%%%%%%%%%%%%%%%%%%%%%%%%%%%%%%%%%%%%%%
\DescribeMacro{\ifchilddocmanual}
The main file should be prepared as usual, see \secref{sec:include}.
However, the document body must make a distinction
between processing of an individual part and of the main document, e.g.:
%
\begin{center}
\begin{tabular}{l}
|\ifchilddocmanual|\\
|\input{\childdocname}|\\
|\||else|\\
\textit{document body with }|\input{|\textit{part}|}|\\
|\||fi|
\end{tabular}
\end{center}
%
The conditional |\ifchilddocmanual| is true whenever
a part to be included by |\input| is being compiled,
and the name of the part is stored in |\childdocname|.

%%%%%%%%%%%%%%%%%%%%%%%%%%%%%%%%%%%%%%%%
\DescribeMacro{\childdocby}
Each part to be included by |\input| should start with:
%
\begin{center}
\begin{tabular}{l}
|\input{childdoc.def}|\\
|\childdocby{|\textit{main}|}|\\
\end{tabular}
\end{center}
%
The directive |\childdocby| is similar to |\childdocof|
described in \secref{sec:include},
but the subsequent selection of content must be done manually.
To that end, both |\ifchilddoc| and |\ifchilddocmanual|
will be true upon processing of a part,
and the name of the part is stored in |\childdocname|.
Note that |\jobname| will be set to the filename of the current part
so that each part receives an individual |.aux| file
that does not interfere with the |.aux| file(s) of the main document.
This behaviour can be altered by the alternative form
|\childdocby[*]{|\textit{main}|}| (with a non-empty optional argument)
which uses the |.aux| file of the main document
by setting |\jobname| to \textit{main}.

%%%%%%%%%%%%%%%%%%%%%%%%%%%%%%%%%%%%%%%%%%%%%%%%%%%%%%%%%%%%%%%%%%%%%%%%%%%%%%%%
\subsection{Driver Development}
\label{sec:driver}

The \textsf{childdoc} mechanism can also be use for the development
of definition files such as \LaTeX{} styles or classes.
This case differs from the above setup with multiple parts
included by |\include| in that no |\includeonly| should be invoked.
This can be achieved by starting the include file
(before |\ProvidesPackage|) with:
%
\begin{center}
\begin{tabular}{l}
|\input{childdoc.def}|\\
|\childdocforward{|\textit{main}|}|\\
\end{tabular}
\end{center}
%
or alternatively with:
%
\begin{center}
\begin{tabular}{l}
|\input{childdoc.def}|\\
|\childdocby{|\textit{main}|}|\\
\end{tabular}
\end{center}
%
Both forms have slightly different effects as described above.
The main file is prepared as usual, see \secref{sec:include}.

%%%%%%%%%%%%%%%%%%%%%%%%%%%%%%%%%%%%%%%%%%%%%%%%%%%%%%%%%%%%%%%%%%%%%%%%%%%%%%%%
\subsection{Legacy Detection}
\label{sec:detection}

The directive |\childdocmain| in the main file can detect
whether the complete document or merely a child is to be compiled
even without using the directive |\childdocof|.
This method is deprecated because it is less robust
and there is no compelling reason to use it;
it is merely provided for backward compatibility
and it may be removed in future versions.

If the detection mechanism is to be used,
it is mandatory to correctly specify
the filename of the main file as the argument of |\childdocmain|:
%
\begin{center}
\begin{tabular}{l}
|\input{childdoc.def}|\\
|\childdocmain{|\textit{main}|}|\\
\end{tabular}
\end{center}
%
If |\jobname| does not match the argument \textit{main} of |\childdocmain|,
it is assumed that |\jobname| points to the child file to be compiled.
When using |\childdocmain| with the main file specified as argument,
it suffices to start a child file
with just |\input{|\textit{main}|}|
without loading of the package and using |\childdocof|.
If instead all processing is done
with the appropriate \textsf{childdoc} directives,
the argument of \textit{main} of |\childdocmain| can be empty.

An alternative version of the command line processing described
in \secref{sec:commandline} using the detection mechanism reads:
%
\begin{center}
|... -jobname "|\textit{target}|" "|[\textit{flags}]%
[|\def\jobname{|\textit{dest}|}|]|\input{|\textit{main}|}"|
\end{center}

%%%%%%%%%%%%%%%%%%%%%%%%%%%%%%%%%%%%%%%%%%%%%%%%%%%%%%%%%%%%%%%%%%%%%%%%%%%%%%%%
\subsection{Manual Code}
\label{sec:manual}

In case one cannot be certain whether the definitions file |childdoc.def|
is installed on the target \TeX{} distribution
and one prefers not to ship it,
it is conceivable to paste a few relevant commands into the sources.

To that end, drop all statements |\input{childdoc.def}|
and perform the replacements as outlined below.
Instead of |\childdocmain{|\textit{main}|}| add the following code
to the top of the main file:
%
\begin{center}
\begin{tabular}{l}
|\||ifdefined\childdocname\endinput\||fi\newif\ifchilddoc|\\
|\edef\childdocname{\scantokens\expandafter{\jobname\noexpand}}|\\
|\def\childdocmain{|\textit{main}|}\||ifx\childdocmain\childdocname\||else|\\
|\childdoctrue\includeonly{\childdocname}\let\jobname\childdocmain\||fi|\\
\end{tabular}
\end{center}
%
Instead of |\childdocof{|\textit{main}|}| just include the main file
at the top of each child file:
%
\begin{center}
|\input{|\textit{main}|}|
\end{center}
%
A simple redirection |\childdocforward{|\textit{dest}|}| is achieved by:
%
\begin{center}
|\def\jobname{|\textit{dest}|}\input{\jobname}|
\end{center}
%
The redirection with prefix
|\childdocforwardprefix[|\textit{prefix}|]{|\textit{dest}|}|
is accomplished by:
%
\begin{center}
\begin{tabular}{l}
|{\edef\jobname{\scantokens\expandafter{\jobname\noexpand}}|\\
|\def\redirectjob |\textit{prefix}|#1~~~{\gdef\jobname{|\textit{dest}|#1}}|\\
|\expandafter\redirectjob\jobname~~~}\input{\jobname}|
\end{tabular}
\end{center}

In an alternative approach,
child documents can be compiled by a specific command line
without additional code or specific definitions:
%
\begin{center}
|... -jobname "|\textit{target}|" "|[\textit{flags}]%
|\includeonly{|\textit{dest}|}\input{|\textit{main}|}"|
\end{center}
%

%%%%%%%%%%%%%%%%%%%%%%%%%%%%%%%%%%%%%%%%%%%%%%%%%%%%%%%%%%%%%%%%%%%%%%%%%%%%%%%%
%%%%%%%%%%%%%%%%%%%%%%%%%%%%%%%%%%%%%%%%%%%%%%%%%%%%%%%%%%%%%%%%%%%%%%%%%%%%%%%%
\section{Information}

%%%%%%%%%%%%%%%%%%%%%%%%%%%%%%%%%%%%%%%%%%%%%%%%%%%%%%%%%%%%%%%%%%%%%%%%%%%%%%%%
\subsection{Copyright}

Copyright \copyright{} 2017--2018 Niklas Beisert

This work may be distributed and/or modified under the
conditions of the \LaTeX{} Project Public License, either version 1.3
of this license or (at your option) any later version.
The latest version of this license is in
  \url{http://www.latex-project.org/lppl.txt}
and version 1.3 or later is part of all distributions of \LaTeX{}
version 2005/12/01 or later.

This work has the LPPL maintenance status `maintained'.

The Current Maintainer of this work is Niklas Beisert.

This work consists of the files |README.txt|, |childdoc.ins| and |childdoc.dtx|
as well as the derived files |childdoc.def|, |cdocsamp.tex|
with |cdocsch1.tex|, |cdocsch2.tex|, |cdocspt3.tex|, |cdocspt4.tex|,
|cdocsdrf.tex|, |cdocsfn1.tex|, |cdocsfn2.tex|
as well as |childdoc.pdf|.

%%%%%%%%%%%%%%%%%%%%%%%%%%%%%%%%%%%%%%%%%%%%%%%%%%%%%%%%%%%%%%%%%%%%%%%%%%%%%%%%
\subsection{Files and Installation}

The package consists of the files:
%
\begin{center}
\begin{tabular}{ll}
    |README.txt|   & readme file \\
    |childdoc.ins| & installation file \\
    |childdoc.dtx| & source file \\
    |childdoc.def| & definition file \\
    |cdocsamp.tex| & sample main file \\
    |cdocsch1.tex| & sample include file \\
    |cdocsch2.tex| & sample include file \\
    |cdocspt3.tex| & sample part file \\
    |cdocspt4.tex| & sample part file \\
    |cdocsdrf.tex| & sample redirection file \\
    |cdocsfn1.tex| & sample redirection file \\
    |cdocsfn2.tex| & sample redirection file \\
    |childdoc.pdf| & manual
\end{tabular}
\end{center}
%
The distribution consists of the files
|README.txt|, |childdoc.ins| and |childdoc.dtx|.
%
\begin{itemize}
\item
Run (pdf)\LaTeX{} on |childdoc.dtx|
to compile the manual |childdoc.pdf| (this file).
\item
Run \LaTeX{} on |childdoc.ins| to create the definitions file |childdoc.def|
and the sample |cdocsamp.tex| with include files
|cdocsch1.tex|, |cdocsch2.tex|, |cdocspt3.tex|, |cdocspt4.tex|,
|cdocsdrf.tex|, |cdocsfn1.tex|, |cdocsfn2.tex|.
Then copy the file |childdoc.def| to an appropriate directory of your \LaTeX{}
distribution, e.g.\ \textit{texmf-root}|/tex/latex/childdoc|.
\end{itemize}

%%%%%%%%%%%%%%%%%%%%%%%%%%%%%%%%%%%%%%%%%%%%%%%%%%%%%%%%%%%%%%%%%%%%%%%%%%%%%%%%
\subsection{Related CTAN Packages}

There are several other packages which offer a similar functionality:
%
\begin{itemize}
\item
The packages
\href{http://ctan.org/pkg/docmute}{\textsf{docmute}},
\href{http://ctan.org/pkg/includex}{\textsf{includex}} and
\href{http://ctan.org/pkg/standalone}{\textsf{standalone}}
provide commands to include only the document body of
a child file thus allowing both files to be compiled individually.
\item
The packages \href{http://ctan.org/pkg/subdocs}{\textsf{subdocs}}
and \href{http://ctan.org/pkg/subfiles}{\textsf{subfiles}}
provide structures in which the main and child documents can be
encapsulated and allowing them to be compiled individually.
The inclusion mechanism is different from the conventional |\include|.
\item
The package \href{http://ctan.org/pkg/combine}{\textsf{combine}}
is an elaborate solution to combine several documents into one.
\end{itemize}
%
See also the CTAN topic \href{http://ctan.org/topic/subdocs}{\textsf{subdocs}}
for further related packages.
The present package differs from the above solutions in that
a document structure constructed with the conventional |\include| mechanism
just needs two extra commands at the top of every file
such that all constituent files can be compiled individually.

%%%%%%%%%%%%%%%%%%%%%%%%%%%%%%%%%%%%%%%%%%%%%%%%%%%%%%%%%%%%%%%%%%%%%%%%%%%%%%%%
%\subsection{Feature Suggestions}
%
%The following is a list of features which may be useful for future
%versions of this package:
%%
%\begin{itemize}
%\item
%\ldots
%\end{itemize}

%%%%%%%%%%%%%%%%%%%%%%%%%%%%%%%%%%%%%%%%%%%%%%%%%%%%%%%%%%%%%%%%%%%%%%%%%%%%%%%%
\subsection{Revision History}

%%%%%%%%%%%%%%%%%%%%%%%%%%%%%%%%%%%%%%%%
\paragraph{v2.0:} 2018/12/30

\begin{itemize}
\item
immediate forward processing
\item
added |\childdocby| mechanism
\item
manual restructured
\end{itemize}

%%%%%%%%%%%%%%%%%%%%%%%%%%%%%%%%%%%%%%%%
\paragraph{v1.6:} 2018/01/17

\begin{itemize}
\item
application for development of include files
\item
corrections to manual
\end{itemize}

%%%%%%%%%%%%%%%%%%%%%%%%%%%%%%%%%%%%%%%%
\paragraph{v1.5:} 2017/05/21

\begin{itemize}
\item
more complete structuring introduced
\item
|\childdocof| introduced
\item
|\childdoc| renamed to |\childdocmain|
\item
|\childredirect| renamed to |\childdocforward| and |\childdocforwardprefix|
and functionality expanded
\end{itemize}

%%%%%%%%%%%%%%%%%%%%%%%%%%%%%%%%%%%%%%%%
\paragraph{v1.0:} 2017/04/27

\begin{itemize}
\item
manual and install package
\item
first version published on CTAN
\end{itemize}

%%%%%%%%%%%%%%%%%%%%%%%%%%%%%%%%%%%%%%%%
\paragraph{v0.6:} 2017/04/26

\begin{itemize}
\item
redirection mechanism added
\end{itemize}

%%%%%%%%%%%%%%%%%%%%%%%%%%%%%%%%%%%%%%%%
\paragraph{v0.5:} 2017/04/26

\begin{itemize}
\item
functionality in definition file
\end{itemize}


%%%%%%%%%%%%%%%%%%%%%%%%%%%%%%%%%%%%%%%%%%%%%%%%%%%%%%%%%%%%%%%%%%%%%%%%%%%%%%%%
%%%%%%%%%%%%%%%%%%%%%%%%%%%%%%%%%%%%%%%%%%%%%%%%%%%%%%%%%%%%%%%%%%%%%%%%%%%%%%%%
%%%%%%%%%%%%%%%%%%%%%%%%%%%%%%%%%%%%%%%%%%%%%%%%%%%%%%%%%%%%%%%%%%%%%%%%%%%%%%%%
\appendix

\settowidth\MacroIndent{\rmfamily\scriptsize 000\ }

 \DocInput{childdoc.dtx}

\end{document}
%</driver>
% \fi
%
% %%%%%%%%%%%%%%%%%%%%%%%%%%%%%%%%%%%%%%%%%%%%%%%%%%%%%%%%%%%%%%%%%%%%%%%%%%%%%%
% %%%%%%%%%%%%%%%%%%%%%%%%%%%%%%%%%%%%%%%%%%%%%%%%%%%%%%%%%%%%%%%%%%%%%%%%%%%%%%
% \section{Sample}
%\iffalse
%<*samplemain>
%\fi
%
% The following presents a sample document
% with two chapters, two parts, a title page,
% a compile flag as well as three forwarding files to set the flag.
% It consists of eight |.tex| files:
% \begin{center}
% \begin{tabular}{ll}
% |cdocsamp.tex|&main file\\
% |cdocsch1.tex|&include file for chapter 1\\
% |cdocsch2.tex|&include file for chapter 2\\
% |cdocspt3.tex|&include file for part 3\\
% |cdocspt4.tex|&include file for part 4\\
% |cdocsdrf.tex|&forwarding file for main file in draft mode\\
% |cdocsfi1.tex|&forwarding file for final version of chapter 1\\
% |cdocsfi2.tex|&forwarding file for final version of chapter 2\\
% \end{tabular}
% \end{center}
% Each of the eight files can be compiled directly by the \LaTeX{} compiler.
%
% %%%%%%%%%%%%%%%%%%%%%%%%%%%%%%%%%%%%%%
% \paragraph{Main File.}
%
% The main file is called |cdocsamp.tex|.
%
% Load the \textsf{childdoc} definitions and
% declare the filename for the main document:
%    \begin{macrocode}
\input{childdoc.def}
\childdocmain{}
%    \end{macrocode}

% Optional override for |\version| flag:
%    \begin{macrocode}
%%\ifchilddoc\else\providecommand{\version}{draft}\fi
%    \end{macrocode}

% Define the default values for the |\version| flag
% (|final| for the main file and |draft| for childs):
%    \begin{macrocode}
\ifchilddoc
\providecommand{\version}{draft}
\else
\providecommand{\version}{final}
\fi
%    \end{macrocode}

% Load the standard document class:
%    \begin{macrocode}
\documentclass[12pt]{article}
%    \end{macrocode}

% Start the document body:
%    \begin{macrocode}
\begin{document}
%    \end{macrocode}

% Declare a title page.
% Print title, part of document being processed and version flag:
%    \begin{macrocode}
\addtocounter{page}{-1}
\begin{center}
{\LARGE\bfseries{}childdoc example\par}
\vspace{1cm}
\ifchilddoc
\ifchilddocmanual part\else chapter\fi:
`\childdocname' of `\childdocjob'\par
\else
main document: `\childdocjob'\par
\fi
version: \version\par
\end{center}
\newpage
%    \end{macrocode}

% Manually include selected file,
% otherwise process as usual:
%    \begin{macrocode}
\ifchilddocmanual
\section*{part `\childdocname'}
\input{\childdocname}
\else
%    \end{macrocode}

% Include the two chapters:
%    \begin{macrocode}
\include{cdocsch1}
\include{cdocsch2}
%    \end{macrocode}

% Include the two parts unless only chapters should be displayed:
%    \begin{macrocode}
\ifchilddoc\else
\section{part three}
\input{cdocspt3}
\section{part four}
\input{cdocspt4}
\fi
%    \end{macrocode}

% Process as usual until here:
%    \begin{macrocode}
\fi
%    \end{macrocode}

% End of document body:
%    \begin{macrocode}
\end{document}
%    \end{macrocode}
%\iffalse
%</samplemain>
%\fi
%
% %%%%%%%%%%%%%%%%%%%%%%%%%%%%%%%%%%%%%%
% \paragraph{Chapter Include Files.}
%
% The include files are called |cdocsch1.tex| and |cdocsch2.tex|.
%
%\iffalse
%<*samplechap1|samplechap2>
%\fi

% Optional override for |\version| flag:
%    \begin{macrocode}
%%\providecommand{\version}{final}
%    \end{macrocode}

% Include the main document:
%    \begin{macrocode}
\input{childdoc.def}
\childdocof{cdocsamp}
%    \end{macrocode}

%\iffalse
%</samplechap1|samplechap2>
%\fi
%
%\iffalse
%<*samplechap1>
%\fi
% Some text for chapter 1:
%    \begin{macrocode}
\section{one}
some text in chapter one
%    \end{macrocode}

%\iffalse
%</samplechap1>
%\fi
% Some text for chapter 2:
%\iffalse
%<*samplechap2>
%\fi
%    \begin{macrocode}
\section{two}
more text in chapter two
%    \end{macrocode}

%\iffalse
%</samplechap2>
%\fi
%
% %%%%%%%%%%%%%%%%%%%%%%%%%%%%%%%%%%%%%%
% \paragraph{Part Include Files.}
%
% The include files are called |cdocspt3.tex| and |cdocspt4.tex|.
%
%\iffalse
%<*samplepart3|samplepart4>
%\fi

% Optional override for |\version| flag:
%    \begin{macrocode}
%%\providecommand{\version}{final}
%    \end{macrocode}

% Include the main document:
%    \begin{macrocode}
\input{childdoc.def}
\childdocby{cdocsamp}
%    \end{macrocode}

%\iffalse
%</samplepart3|samplepart4>
%\fi
%
%\iffalse
%<*samplepart3>
%\fi
% Some text for part 3:
%    \begin{macrocode}
some text in part three
%    \end{macrocode}

%\iffalse
%</samplepart3>
%\fi
% Some text for part 4:
%\iffalse
%<*samplepart4>
%\fi
%    \begin{macrocode}
more text in part four
%    \end{macrocode}

%\iffalse
%</samplepart4>
%\fi
%
% %%%%%%%%%%%%%%%%%%%%%%%%%%%%%%%%%%%%%%
% \paragraph{Forwarding for a Complete Draft.}
%
% The following forwarding file |cdocsdrf.tex|
% compiles the main document in draft mode:
%\iffalse
%<*sampledraft>
%\fi
%    \begin{macrocode}
\def\version{draft}
\input{childdoc.def}
\childdocforward{cdocsamp}
%    \end{macrocode}

%\iffalse
%</sampledraft>
%\fi
%
% %%%%%%%%%%%%%%%%%%%%%%%%%%%%%%%%%%%%%%
% \paragraph{Forwarding for Final Version of the Chapters.}
%
% The following forwarding files |cdocsfn1.tex| and |cdocsfn2.tex|
% (with identical content)
% compile the final versions of the child documents
% |cdocsch1.tex| and |cdocsch2.tex|, respectively:
%\iffalse
%<*samplefinal>
%\fi
%    \begin{macrocode}
\def\version{final}
\input{childdoc.def}
\childdocforwardprefix[cdocsamp]{cdocsfn}{cdocsch}
%    \end{macrocode}

%\iffalse
%</samplefinal>
%\fi
%
% %%%%%%%%%%%%%%%%%%%%%%%%%%%%%%%%%%%%%%
% \paragraph{Command Line Processing.}
%
% The following three command lines generate the output files
% |cdocscld|, |cdocscl1| and |cdocscl2|
% which should be identical to
% |cdocsdrf|, |cdocsch1| and |cdocsfn2|, respectively:
% \begin{center}
% \begin{tabular}{l}
% |latex -jobname cdocscld \|\\
% |  "\def\version{draft}\input{childdoc.def}\childdocforward{cdocsamp}"|\\
% |latex -jobname cdocscl1 \|\\
% |  "\input{childdoc.def}\childdocforward[cdocsamp]{cdocsch1}"|\\
% |latex -jobname cdocscl2 \|\\
% |  "\def\version{final}\input{childdoc.def}\childdocforward{cdocsch2}"|
% \end{tabular}
% \end{center}
% Note that the trailing backslash on each first line
% merely continues the input to the second line
% (for convenient cut ant paste).
% Furthermore, the command |latex| can be replaced by any
% of its alternative versions such as |pdflatex|.
%
% %%%%%%%%%%%%%%%%%%%%%%%%%%%%%%%%%%%%%%%%%%%%%%%%%%%%%%%%%%%%%%%%%%%%%%%%%%%%%%
% %%%%%%%%%%%%%%%%%%%%%%%%%%%%%%%%%%%%%%%%%%%%%%%%%%%%%%%%%%%%%%%%%%%%%%%%%%%%%%
% \section{Implementation}
%\iffalse
%<*package>
%\fi
%
% This section describes the definitions file |childdoc.def|.

% The definitions cannot be loaded using |\usepackage| or |\RequirePackage|
% which has a mechanism to prevent loading a style file more than once.
% When loading the definitions by means of |\input|
% multiple instances have to be prevented manually:
%\iffalse
%This code needs to be before the `\ProvidesFile' directive
%which is defined at the beginning of this file.
%Therefore it is also placed there and commented out here.
%</package>
%<*discard>
%\fi
%    \begin{macrocode}
\ifdefined\childdocmain\endinput\fi
%    \end{macrocode}
%\iffalse
%</discard>
%<*package>
%\fi
%
% \macro{\ifchilddoc}
% \macro{\ifchilddocmanual}
% The conditional |\ifchilddoc| tells whether a
% child (true) or main (false) document is being compiled.
% The conditional |\ifchilddocmanual| tells whether
% the |\includeonly| mechanism is used (false) or
% the selection of child files must be performed manually (true).
% The definitions initialise to false:
%    \begin{macrocode}
\newif\ifchilddoc
\newif\ifchilddocmanual
%    \end{macrocode}

% \macro{\childdocname}
% \macro{\childdocjob}
% The macro |\childdocname| stores the name of the main document
% to be compiled. The macro |\childdocjob| stores the name of
% the document on which the \LaTeX{} compiler was originally invoked.
% The content of |\jobname| cannot be compared
% to filenames specified in the source due to different catcodes.
% The following code rescans |\jobname|, stores the result
% in |\childdocname| and saves a copy in |\childdocjob|:
%    \begin{macrocode}
\edef\childdocname{\scantokens\expandafter{\jobname\noexpand}}
\let\childdocjob\childdocname
%    \end{macrocode}

% \macro{\childdocdisable}
% The macro |\childdocdisable| prevents the main file
% from being processed more than once.
% At this stage, the main document command |\childdocmain|
% is assumed to be called once again where it should do nothing.
% Any subsequent call to it should prevent
% a secondary processing of the main document
% It overwrites the forwarding commands
% |\childdocof| and |\childdocforward|
% with empty macros to prevent further inclusions of the main document:
%    \begin{macrocode}
\newcommand{\childdocdisable}
{
  \renewcommand{\childdocmain}[1]{\renewcommand{\childdocmain}[1]{\endinput}}
  \renewcommand{\childdocof}[1]{}
  \renewcommand{\childdocby}[2][]{}
  \renewcommand{\childdocforward}[2][]{}
  \renewcommand{\childdocdisable}{}
}
%    \end{macrocode}

% \macro{\childdocmain}
% The macro |\childdocmain| is to be called at the top of the main file
% with nothing or the main filename (without extension) as argument.
% First, it breaks loops.
% If the argument is not empty and does not match |\childdocname|
% (which is set by the first inclusion of |childdoc.def|),
% |\ifchilddoc| is set to true, |\includeonly| is applied to the child file
% and |\jobname| is set to the main file
% (for proper handling of |.aux| files):
%    \begin{macrocode}
\newcommand{\childdocmain}[1]
{
  \childdocdisable\childdocmain{}
  \if?#1?\else
    \begingroup
      \def\childdoctmp{#1}
      \ifx\childdoctmp\childdocname
        \def\childdoctmp{}
      \else
        \def\childdoctmp
        {
          \childdoctrue
          \includeonly{\childdocname}
          \def\childdocjob{#1}
          \def\jobname{#1}
        }
      \fi
      \expandafter
    \endgroup
    \childdoctmp
  \fi
}
%    \end{macrocode}

% \macro{\childdocof}
% The command |\childdocof| redirects
% compilation to the main file |#1|.
%    \begin{macrocode}
\newcommand{\childdocof}[1]
{
  \childdocdisable
  \childdoctrue
  \includeonly{\childdocname}
  \def\jobname{#1}
  \def\childdocjob{#1}
  \input{#1}
}
%    \end{macrocode}

% \macro{\childdocby}
% The command |\childdocby| ....
%    \begin{macrocode}
\newcommand{\childdocby}[2][]
{
  \childdocdisable
  \childdoctrue
  \childdocmanualtrue
  \if?#1?\else
    \def\jobname{#2}
  \fi
  \def\childdocjob{#2}
  \input{#2}
  \endinput
}
%    \end{macrocode}

% \macro{\childdocforward}
% The command |\childdocforward| redirects
% compilation to the main file or
% (if the optional argument is given) a child file.
% Parameters are set as if the main file
% or a child file starting with |\childdocof| was compiled.
% Then compilation is handed over to the main file:
%    \begin{macrocode}
\newcommand{\childdocforward}[2][]
{
  \begingroup
    \if?#1?
      \def\childdoctmp
      {
        \def\childdocname{#2}
        \def\childdocjob{#2}
        \def\jobname{#2}
        \input{#2}
        \endinput
      }
    \else
      \def\childdoctmp
      {
        \childdocdisable
        \def\childdocname{#2}
        \childdoctrue
        \includeonly{#2}
        \def\childdocjob{#1}
        \def\jobname{#1}
        \input{#1}
        \endinput
      }
    \fi
    \expandafter
  \endgroup
  \childdoctmp
}
%    \end{macrocode}

% \macro{\childdocforwardprefix}
% The command |\childdocforwardprefix| redirects
% compilation to the main or a child file by means of a pattern.
% The prefix |#1| in the current filename is replaced by |#2|
% and the suffix of the current filename is kept
% (it is assumed that the filename does not contain the substring `|~~~|'
% which is used as a delimiter).
% Compilation is handed over to the new file by |\childdocforward|:
%    \begin{macrocode}
\newcommand{\childdocforwardprefix}[3][]
{
  \begingroup
    \def\childdocextract #2##1~~~{\def\childdoctmp{\childdocforward[#1]{#3##1}}}
    \expandafter\childdocextract\childdocname~~~
    \expandafter
  \endgroup
  \childdoctmp
}
%    \end{macrocode}

% \macro{\childdoc}
% The deprecated macro |\childdoc| is a legacy version of |\childdocmain|:
%    \begin{macrocode}
\newcommand{\childdoc}{\childdocmain}
%    \end{macrocode}

% \macro{\childdocredirect}
% The deprecated macro |\childdocredirect| is a legacy version
% of |\childdocforward| and |\childdocforwardprefix|:
%    \begin{macrocode}
\newcommand{\childdocredirect}[2][]
{
  \begingroup
    \if?#1?
      \def\childdoctmp{\childdocforward{#2}}
    \else
      \def\childdoctmp{\childdocforwardprefix{#1}{#2}}
    \fi
    \expandafter
  \endgroup
  \childdoctmp
}
%    \end{macrocode}

%\iffalse
%</package>
%\fi
%
\endinput
|\\
|\childdocby{|\textit{main}|}|\\
\end{tabular}
\end{center}
%
Both forms have slightly different effects as described above.
The main file is prepared as usual, see \secref{sec:include}.

%%%%%%%%%%%%%%%%%%%%%%%%%%%%%%%%%%%%%%%%%%%%%%%%%%%%%%%%%%%%%%%%%%%%%%%%%%%%%%%%
\subsection{Legacy Detection}
\label{sec:detection}

The directive |\childdocmain| in the main file can detect
whether the complete document or merely a child is to be compiled
even without using the directive |\childdocof|.
This method is deprecated because it is less robust
and there is no compelling reason to use it;
it is merely provided for backward compatibility
and it may be removed in future versions.

If the detection mechanism is to be used,
it is mandatory to correctly specify
the filename of the main file as the argument of |\childdocmain|:
%
\begin{center}
\begin{tabular}{l}
|% \iffalse
%
% childdoc.dtx Copyright (C) 2017-2018 Niklas Beisert
%
% This work may be distributed and/or modified under the
% conditions of the LaTeX Project Public License, either version 1.3
% of this license or (at your option) any later version.
% The latest version of this license is in
%   http://www.latex-project.org/lppl.txt
% and version 1.3 or later is part of all distributions of LaTeX
% version 2005/12/01 or later.
%
% This work has the LPPL maintenance status `maintained'.
%
% The Current Maintainer of this work is Niklas Beisert.
%
% This work consists of the files childdoc.dtx and childdoc.ins
% and the derived files childdoc.def and cdocsamp.tex with
% cdocsch1.tex, cdocsch2.tex, cdocsdrf.tex, cdocsfn1.tex, cdocsfn2.tex.
%
%<package>\ifdefined\childdocmain\endinput\fi
%<package>\ProvidesFile{childdoc.def}[2018/12/30 v2.0 child document driver]
%<samplemain>\ProvidesFile{cdocsamp.tex}[2018/12/30 v2.0 sample for childdoc]
%<*driver>
%\ProvidesFile{childdoc.drv}[2018/12/30 v2.0 childdoc reference manual file]
\PassOptionsToClass{10pt,a4paper}{article}
\documentclass{ltxdoc}

\usepackage[margin=35mm]{geometry}
\usepackage{hyperref}
\usepackage{hyperxmp}
\usepackage[usenames]{color}

\hypersetup{colorlinks=true}
\hypersetup{pdfstartview=FitH}
\hypersetup{pdfpagemode=UseNone}
\hypersetup{pdfsource={}}
\hypersetup{pdflang={en-UK}}
\hypersetup{pdfcopyright={Copyright 2017-2018 Niklas Beisert.
  This work may be distributed and/or modified under the
  conditions of the LaTeX Project Public License, either version 1.3
  of this license or (at your option) any later version.}}
\hypersetup{pdflicenseurl={http://www.latex-project.org/lppl.txt}}
\hypersetup{pdfcontactaddress={ETH Zurich, ITP, HIT K,
  Wolfgang-Pauli-Strasse 27}}
\hypersetup{pdfcontactpostcode={8093}}
\hypersetup{pdfcontactcity={Zurich}}
\hypersetup{pdfcontactcountry={Switzerland}}
\hypersetup{pdfcontactemail={nbeisert@itp.phys.ethz.ch}}
\hypersetup{pdfcontacturl={http://people.phys.ethz.ch/\xmptilde nbeisert/}}

\newcommand{\secref}[1]{\hyperref[#1]{section \ref*{#1}}}

\parskip1ex
\parindent0pt
\let\olditemize\itemize
\def\itemize{\olditemize\parskip0pt}

\begin{document}

\title{The \textsf{childdoc} Package}
\hypersetup{pdftitle={The childdoc Package}}
\author{Niklas Beisert\\[2ex]
  Institut f\"ur Theoretische Physik\\
  Eidgen\"ossische Technische Hochschule Z\"urich\\
  Wolfgang-Pauli-Strasse 27, 8093 Z\"urich, Switzerland\\[1ex]
  \href{mailto:nbeisert@itp.phys.ethz.ch}
  {\texttt{nbeisert@itp.phys.ethz.ch}}}
\hypersetup{pdfauthor={Niklas Beisert}}
\hypersetup{pdfsubject={Manual for the LaTeX2e Package childdoc}}
\date{30 December 2018, \textsf{v2.0}}
\maketitle

\begin{abstract}\noindent
\textsf{childdoc} is a \LaTeXe{} package
that enables the direct compilation
of document sections included by |\include|
to individual files.
\end{abstract}

\begingroup
\parskip0ex
\tableofcontents
\endgroup

%%%%%%%%%%%%%%%%%%%%%%%%%%%%%%%%%%%%%%%%%%%%%%%%%%%%%%%%%%%%%%%%%%%%%%%%%%%%%%%%
%%%%%%%%%%%%%%%%%%%%%%%%%%%%%%%%%%%%%%%%%%%%%%%%%%%%%%%%%%%%%%%%%%%%%%%%%%%%%%%%
\section{Introduction}

\LaTeX{} provides a mechanism to structure a large document (such as a book)
into a main file and several child files (containing the chapters)
using the |\include| command.
This mechanism is beneficial for documents
which span hundreds of pages in order to
make the source file(s) more manageable.
Moreover, compilation can be restricted to
selected child files by means of the |\includeonly| command.
The latter feature can be used to reduce the compilation time while editing
(this was significantly more useful in the earlier days of \LaTeX{})
or to generate a smaller document which is easier to navigate.
Another application of |\includeonly| is to generate
documents consisting of selected parts of the complete document.

However, there are a few drawbacks of the plain |\include| mechanism:
\begin{itemize}
\item
The child files cannot be compiled on their own,
they can only be compiled via the main file.
A naive editing environment
(such as a text editor with an option
to have the current file processed by \LaTeX)
may require one to switch to the main file before compiling;
attempting to compile the child file produces errors.
\item
The main file must be modified (each time)
to adjust the |\includeonly| command
to the present needs. This easily leaves the main file in a messy state.
\item
The generated document will always carry the filename
of the main document. This is inconvenient if
several child files are to be compiled and
to be kept for distribution.
\end{itemize}

The present package provides a simple interface
to make child files individually compilable by \LaTeX{}.
Compiling a child file then has the same effect as compiling
the main file with an |\includeonly| command
to select the appropriate child.
Moreover the generated document will carry the name of the child
rather than the main file.
This resolves all three above issues.

This feature is meant to make the editing of books,
thesis documents and lecture notes somewhat more convenient.
However, the package can also be used efficiently for
composing a series of documents (such as exercise sheets)
which are typically distributed individually.
It then assists the author in generating the individual documents
(potentially in different versions)
as well as a document containing the collected series.
Another application is in developing style files
or other kinds of included material
where compilation of the style file could redirect
to a sample or test file.

%%%%%%%%%%%%%%%%%%%%%%%%%%%%%%%%%%%%%%%%%%%%%%%%%%%%%%%%%%%%%%%%%%%%%%%%%%%%%%%%
%%%%%%%%%%%%%%%%%%%%%%%%%%%%%%%%%%%%%%%%%%%%%%%%%%%%%%%%%%%%%%%%%%%%%%%%%%%%%%%%
\section{Usage}

First of all, the package \textsf{childdoc} is \emph{not} a standard
\LaTeXe{} |.sty| style file! Therefore it needs to be invoked in
a non-standard way.

%%%%%%%%%%%%%%%%%%%%%%%%%%%%%%%%%%%%%%%%%%%%%%%%%%%%%%%%%%%%%%%%%%%%%%%%%%%%%%%%
\subsection{Included Files}
\label{sec:include}

%%%%%%%%%%%%%%%%%%%%%%%%%%%%%%%%%%%%%%%%
\DescribeMacro{\childdocmain}
To use the package, add the commands
\begin{center}
\begin{tabular}{l}
|\input{childdoc.def}|\\
|\childdocmain{}|\\
\end{tabular}
\end{center}
at the very top of the main \LaTeX{} file,
in particular \emph{before} the |\documentclass| statement!
The argument of |\childdocmain| should be left empty
(but it must be present).

%%%%%%%%%%%%%%%%%%%%%%%%%%%%%%%%%%%%%%%%
\DescribeMacro{\childdocof}
Furthermore, add the commands
\begin{center}
\begin{tabular}{l}
|\input{childdoc.def}|\\
|\childdocof{|\textit{main}|}|\\
\end{tabular}
\end{center}
at the top of every child file \textit{child}
which is included by |\include{|\textit{child}|}|
from within the main file
(or at least for those files to be compiled individually).
The argument \textit{main} must be the filename of the main file.

There are a couple of
considerations in setting up the main and child documents:

%%%%%%%%%%%%%%%%%%%%%%%%%%%%%%%%%%%%%%%%
\paragraph{Restrictions.}

Please note the following restrictions:
\begin{itemize}
\item
|\childdocmain| must be called with one argument \textit{main}
to ensure compatibility with earlier version of the package.
It must either be empty (|\childdocmain{}|)
or precisely match the filename of the main file in which it is specified.
See \secref{sec:detection} for further information.
\item
The filename \textit{main} must be specified without the |.tex| extension.
\item
The filename \textit{main} is case sensitive
(even in case-insensitive file systems)
due to internal string comparison.
\item
The argument \textit{main} should be fully expanded, it cannot be a macro.
\item
Subdirectories and special characters should be avoided in filenames.
\item
The command |\childdocmain{|\textit{main}|}| must be followed by a whitespace.
It should not be followed immediately by another command
or by a comment mark `|%|'.
This is because the \TeX{} parser reads the token immediately following
the argument of |\childdocmain| and puts it
at the beginning of every child section;
however, a white\-space is ignored.
\end{itemize}

%%%%%%%%%%%%%%%%%%%%%%%%%%%%%%%%%%%%%%%%
\paragraph{Content of Main File.}

It is advisable to place all content in the child files included by |\include|.
Any output contained in the main file will appear in all child documents
unless suppressed manually;
it cannot be suppressed automatically by the |\includeonly| directive
and thus should normally be avoided.
A method to include some content in the main file
by means of conditional processing is described in \secref{sec:conditional}.

%%%%%%%%%%%%%%%%%%%%%%%%%%%%%%%%%%%%%%%%
\paragraph{Page Numbering.}

When only a part of the document is compiled,
the appropriate numbering of pages
(as well as other status parameters)
is determined from the |.aux| files.
The latter contain information from previous passes.
However this information needs to propagate through
all intermediate child documents.
Therefore the page numbering in child documents may well
be inconsistent until the complete document is compiled at least once.

A useful (if unconventional) way to always ensure a consistent
page numbering is to restart the numbering in each child document
and denote the pages by `\textit{child}|.|\textit{page}'
where \textit{child} represents the chapter/section number of the child file.
This can be achieved by the command
|\numberwithin{page}{|\textit{child}|}|
of the \textsf{amsmath} package
where \textit{child} can be |chapter| or |section|
depending on the chosen structuring.
Alternatively, one can modify the macro |\thepage| appropriately
and reset the counter |page| at the start of each child file.

%%%%%%%%%%%%%%%%%%%%%%%%%%%%%%%%%%%%%%%%%%%%%%%%%%%%%%%%%%%%%%%%%%%%%%%%%%%%%%%%
\subsection{Conditional Processing}
\label{sec:conditional}

The package provides a mechanism to compile different versions
of a document. To customise the versions further some conditional processing
can come in handy to distinguish which version is being compiled.
The package provides two macros to describe the compilation context:

%%%%%%%%%%%%%%%%%%%%%%%%%%%%%%%%%%%%%%%%
\DescribeMacro{\ifchilddoc}
The conditional |\ifchilddoc| distinguishes between the compilation of
child documents and the main document:
%
\begin{center}
|\ifchilddoc |\textit{child-code}| |[|\||else |\textit{main-code}]| \||fi|
\end{center}

%%%%%%%%%%%%%%%%%%%%%%%%%%%%%%%%%%%%%%%%
\DescribeMacro{\childdocname}
\DescribeMacro{\childdocjob}
The macro |\childdocname| contains the filename (without extension)
of the main or child file being processed.
Note that |\childdocjob| will always contain the name of the main file.

%%%%%%%%%%%%%%%%%%%%%%%%%%%%%%%%%%%%%%%%
\paragraph{Title Page.}

Conditional processing can be used to include a title or banner page
in the main document when proper precautions are taken.
Importantly, the code in the main file should ensure that the page counter
(as well as other status parameters which are stored in the |.aux| files)
takes the same value after the conditional processing.
Otherwise the page numbers may take divergent values
depending on which part is compiled.

For example, a title page could be declared by:
%
\begin{center}
\begin{tabular}{l}
|\ifchilddoc\||else|\\
|\addtocounter{page}{-1}|\\
\textit{code for title page}\\
|\newpage|\\
|\||fi|
\end{tabular}
\end{center}
%
A banner page for the child documents can be generated by:
%
\begin{center}
\begin{tabular}{l}
|\ifchilddoc|\\
|\addtocounter{page}{-1}|\\
\textit{code for banner page}\\
|\newpage|\\
|\||fi|
\end{tabular}
\end{center}
%
Here one could write a message such as:
\begin{center}
|This is the part \childdocname{} of \childdocjob{}.|
\end{center}

%%%%%%%%%%%%%%%%%%%%%%%%%%%%%%%%%%%%%%%%%%%%%%%%%%%%%%%%%%%%%%%%%%%%%%%%%%%%%%%%
\subsection{Flags}
\label{sec:flags}

The package makes it easy to generate different versions
of the main or child documents.
To this end compilation flags can be defined
and assigned different default values.
They will be particularly useful in conjunction
with the forwarding mechanism described in \secref{sec:forward}.

For example, it may be useful to have a flag |\version|
which can be set to |draft| or |final|.
The document source will contain some conditional code
depending on the value of |\version|.
Suppose further, the flag should default to |final| for the main file
and to |draft| for child files
which is a natural assignment for editing the document.
This is achieved by placing the following code
in the preamble of the main document
(below the |\childdocmain| directive):
%
\begin{center}
\begin{tabular}{l}
|\ifchilddoc|\\
|\providecommand{\version}{draft}|\\
|\||else|\\
|\providecommand{\version}{final}|\\
|\||fi|
\end{tabular}
\end{center}
%
The definition by |\providecommand| makes sure
that previous definitions are not overwritten.
Further statements |\providecommand{\version}{...}|
can thus be added before the above code to override it.

For the main file, one might add a line
(between |\childdocmain| and the above block)
%
\begin{center}
|%\ifchilddoc\||else\providecommand{\version}{draft}\||fi|
\end{center}
%
which can be uncommented to produce a draft version.
Likewise one can add a line to the very top of a child file
(above the |\childdocof{|\textit{main}|}| directive)
%
\begin{center}
|%\providecommand{\version}{final}|
\end{center}
%
which can be uncommented to produce the final version of this child document.

%%%%%%%%%%%%%%%%%%%%%%%%%%%%%%%%%%%%%%%%%%%%%%%%%%%%%%%%%%%%%%%%%%%%%%%%%%%%%%%%
\subsection{Forwarding}
\label{sec:forward}

Different versions of the main or child documents
using compilation flags as described in \secref{sec:flags}
can be (permanently) stored in different files
for convenient compilation, viewing and distribution.
To this end, the package defines a command
to pass on compilation to a different file:

%%%%%%%%%%%%%%%%%%%%%%%%%%%%%%%%%%%%%%%%
\DescribeMacro{\childdocforward}
The command |\childdocforward| redirects processing to
another source file:
%
\begin{center}
\begin{tabular}{l}
|\input{childdoc.def}|\\
|\childdocforward[|\textit{main}|]{|\textit{dest}|}|\\
\end{tabular}
\end{center}
%
The argument \textit{dest} is the destination file
(without extension).
It should be the main file or one of the child files.
Note that further \textsf{childdoc} directives
such as |\childdocof| and |\childdocforward|
in the indicated file will be processed in this form.
The optional argument \textit{main}
passes on directly to the main file \textit{main}
while pretending to compile the child \textit{dest}.
This form behaves as if \textit{dest}
issues |\childdocof{|\textit{main}|}| right away,
and no further \textsf{childdoc} directives will be processed.

%%%%%%%%%%%%%%%%%%%%%%%%%%%%%%%%%%%%%%%%
\DescribeMacro{\...prefix}
In the alternative form |\childdocforwardprefix|,
%
\begin{center}
\begin{tabular}{l}
|\input{childdoc.def}|\\
|\childdocforwardprefix[|\textit{main}|]{|\textit{prefix}|}{|\textit{dest}|}|
\end{tabular}
\end{center}
%
the destination file is determined by a pattern
depending on the current file:
To make this work, the current file must be called
`{\textit{prefix}\hspace{0.2em}\textit{suffix}}'
with \textit{prefix} matching precisely the argument.
Processing is then passed on to the file
`{\textit{dest}\hspace{0.2em}\textit{suffix}}'.
Surely, the same effect is achieved by
directly specifying the
argument `{\textit{dest}\hspace{0.2em}\textit{suffix}}'
in the first form.
However, that requires to set up a different file
for each child. With the alternative form of the command
all these files can have exactly the same content
which simplifies setting them up and maintaining them.

For example, the following file |draft.tex|
with a compilation flag |\version| as described in \secref{sec:flags}
compiles the main document as a draft:
%
\begin{center}
\begin{tabular}{l}
|\def\version{draft}|\\
|\input{childdoc.def}|\\
|\childdocforward{|\textit{main}|}|
\end{tabular}
\end{center}
%
Likewise, the following files |final|\textit{nn}|.tex|
compile the final version of the child document
|child|\textit{nn}|.tex|:
%
\begin{center}
\begin{tabular}{l}
|\def\version{final}|\\
|\input{childdoc.def}|\\
|\childdocforwardprefix{final}{child}|
\end{tabular}
\end{center}
%

Note that when several versions of a main file and/or of each child file
are to be generated, it may be convenient to set up a |Makefile| or
shell script to automatise the process.

%%%%%%%%%%%%%%%%%%%%%%%%%%%%%%%%%%%%%%%%%%%%%%%%%%%%%%%%%%%%%%%%%%%%%%%%%%%%%%%%
\subsection{Command Line Processing}
\label{sec:commandline}

The effect of redirection files can also be achieved by invoking
the \LaTeX{} compiler with a more elaborate command line.
Most conveniently this should be done as part
of a shell script or a |Makefile|.

When using \textsf{childdoc} in the main file, the following
command lines effectively perform a redirection
(note that depending on the shell being used,
backslashes may have to be doubled: `|\|' $\to$ `|\\|'):
%
\begin{center}
|... -jobname "|\textit{target}|" |\\|"|[\textit{flags}]%
|\input{childdoc.def}\childdocforward[|\textit{main}|]{|\textit{dest}|}"|
\end{center}
%
Here \textit{target} is the name of the output file,
\textit{main} is the name of the main file
and \textit{dest} is the name of the main or child file to be processed
(all filenames without extensions).
The optional argument \textit{main} can be omitted
if \textit{main} matches \textit{dest}.
Optionally, compilation \textit{flags} can be defined via |\def| commands.
This command line makes the \TeX{} engine believe
it is compiling the file \textit{target}
whose content is specified as the latter parameter.
The provided code then forwards the processing to
\textit{main} or \textit{dest} as described in \secref{sec:forward}.

%%%%%%%%%%%%%%%%%%%%%%%%%%%%%%%%%%%%%%%%%%%%%%%%%%%%%%%%%%%%%%%%%%%%%%%%%%%%%%%%
\subsection{Include by Input}
\label{sec:input}

Including child documents by |\include| has some restrictions by design.
Most notably, the content of a child document always occupies
its own set of pages; pages cannot be shared between child documents.
Usually, this behaviour makes perfect sense
because each child document contain an essential part of the document.
However, in some situations it may be desirable to compose
a document from a collection of parts
without having mandatory page breaks between then.
For this case, the package
provides a mechanism to include parts
by |\input| which can also be processed individually.
However, by construction this mechanism
requires manual handling of the content to be output.

%%%%%%%%%%%%%%%%%%%%%%%%%%%%%%%%%%%%%%%%
\DescribeMacro{\ifchilddocmanual}
The main file should be prepared as usual, see \secref{sec:include}.
However, the document body must make a distinction
between processing of an individual part and of the main document, e.g.:
%
\begin{center}
\begin{tabular}{l}
|\ifchilddocmanual|\\
|\input{\childdocname}|\\
|\||else|\\
\textit{document body with }|\input{|\textit{part}|}|\\
|\||fi|
\end{tabular}
\end{center}
%
The conditional |\ifchilddocmanual| is true whenever
a part to be included by |\input| is being compiled,
and the name of the part is stored in |\childdocname|.

%%%%%%%%%%%%%%%%%%%%%%%%%%%%%%%%%%%%%%%%
\DescribeMacro{\childdocby}
Each part to be included by |\input| should start with:
%
\begin{center}
\begin{tabular}{l}
|\input{childdoc.def}|\\
|\childdocby{|\textit{main}|}|\\
\end{tabular}
\end{center}
%
The directive |\childdocby| is similar to |\childdocof|
described in \secref{sec:include},
but the subsequent selection of content must be done manually.
To that end, both |\ifchilddoc| and |\ifchilddocmanual|
will be true upon processing of a part,
and the name of the part is stored in |\childdocname|.
Note that |\jobname| will be set to the filename of the current part
so that each part receives an individual |.aux| file
that does not interfere with the |.aux| file(s) of the main document.
This behaviour can be altered by the alternative form
|\childdocby[*]{|\textit{main}|}| (with a non-empty optional argument)
which uses the |.aux| file of the main document
by setting |\jobname| to \textit{main}.

%%%%%%%%%%%%%%%%%%%%%%%%%%%%%%%%%%%%%%%%%%%%%%%%%%%%%%%%%%%%%%%%%%%%%%%%%%%%%%%%
\subsection{Driver Development}
\label{sec:driver}

The \textsf{childdoc} mechanism can also be use for the development
of definition files such as \LaTeX{} styles or classes.
This case differs from the above setup with multiple parts
included by |\include| in that no |\includeonly| should be invoked.
This can be achieved by starting the include file
(before |\ProvidesPackage|) with:
%
\begin{center}
\begin{tabular}{l}
|\input{childdoc.def}|\\
|\childdocforward{|\textit{main}|}|\\
\end{tabular}
\end{center}
%
or alternatively with:
%
\begin{center}
\begin{tabular}{l}
|\input{childdoc.def}|\\
|\childdocby{|\textit{main}|}|\\
\end{tabular}
\end{center}
%
Both forms have slightly different effects as described above.
The main file is prepared as usual, see \secref{sec:include}.

%%%%%%%%%%%%%%%%%%%%%%%%%%%%%%%%%%%%%%%%%%%%%%%%%%%%%%%%%%%%%%%%%%%%%%%%%%%%%%%%
\subsection{Legacy Detection}
\label{sec:detection}

The directive |\childdocmain| in the main file can detect
whether the complete document or merely a child is to be compiled
even without using the directive |\childdocof|.
This method is deprecated because it is less robust
and there is no compelling reason to use it;
it is merely provided for backward compatibility
and it may be removed in future versions.

If the detection mechanism is to be used,
it is mandatory to correctly specify
the filename of the main file as the argument of |\childdocmain|:
%
\begin{center}
\begin{tabular}{l}
|\input{childdoc.def}|\\
|\childdocmain{|\textit{main}|}|\\
\end{tabular}
\end{center}
%
If |\jobname| does not match the argument \textit{main} of |\childdocmain|,
it is assumed that |\jobname| points to the child file to be compiled.
When using |\childdocmain| with the main file specified as argument,
it suffices to start a child file
with just |\input{|\textit{main}|}|
without loading of the package and using |\childdocof|.
If instead all processing is done
with the appropriate \textsf{childdoc} directives,
the argument of \textit{main} of |\childdocmain| can be empty.

An alternative version of the command line processing described
in \secref{sec:commandline} using the detection mechanism reads:
%
\begin{center}
|... -jobname "|\textit{target}|" "|[\textit{flags}]%
[|\def\jobname{|\textit{dest}|}|]|\input{|\textit{main}|}"|
\end{center}

%%%%%%%%%%%%%%%%%%%%%%%%%%%%%%%%%%%%%%%%%%%%%%%%%%%%%%%%%%%%%%%%%%%%%%%%%%%%%%%%
\subsection{Manual Code}
\label{sec:manual}

In case one cannot be certain whether the definitions file |childdoc.def|
is installed on the target \TeX{} distribution
and one prefers not to ship it,
it is conceivable to paste a few relevant commands into the sources.

To that end, drop all statements |\input{childdoc.def}|
and perform the replacements as outlined below.
Instead of |\childdocmain{|\textit{main}|}| add the following code
to the top of the main file:
%
\begin{center}
\begin{tabular}{l}
|\||ifdefined\childdocname\endinput\||fi\newif\ifchilddoc|\\
|\edef\childdocname{\scantokens\expandafter{\jobname\noexpand}}|\\
|\def\childdocmain{|\textit{main}|}\||ifx\childdocmain\childdocname\||else|\\
|\childdoctrue\includeonly{\childdocname}\let\jobname\childdocmain\||fi|\\
\end{tabular}
\end{center}
%
Instead of |\childdocof{|\textit{main}|}| just include the main file
at the top of each child file:
%
\begin{center}
|\input{|\textit{main}|}|
\end{center}
%
A simple redirection |\childdocforward{|\textit{dest}|}| is achieved by:
%
\begin{center}
|\def\jobname{|\textit{dest}|}\input{\jobname}|
\end{center}
%
The redirection with prefix
|\childdocforwardprefix[|\textit{prefix}|]{|\textit{dest}|}|
is accomplished by:
%
\begin{center}
\begin{tabular}{l}
|{\edef\jobname{\scantokens\expandafter{\jobname\noexpand}}|\\
|\def\redirectjob |\textit{prefix}|#1~~~{\gdef\jobname{|\textit{dest}|#1}}|\\
|\expandafter\redirectjob\jobname~~~}\input{\jobname}|
\end{tabular}
\end{center}

In an alternative approach,
child documents can be compiled by a specific command line
without additional code or specific definitions:
%
\begin{center}
|... -jobname "|\textit{target}|" "|[\textit{flags}]%
|\includeonly{|\textit{dest}|}\input{|\textit{main}|}"|
\end{center}
%

%%%%%%%%%%%%%%%%%%%%%%%%%%%%%%%%%%%%%%%%%%%%%%%%%%%%%%%%%%%%%%%%%%%%%%%%%%%%%%%%
%%%%%%%%%%%%%%%%%%%%%%%%%%%%%%%%%%%%%%%%%%%%%%%%%%%%%%%%%%%%%%%%%%%%%%%%%%%%%%%%
\section{Information}

%%%%%%%%%%%%%%%%%%%%%%%%%%%%%%%%%%%%%%%%%%%%%%%%%%%%%%%%%%%%%%%%%%%%%%%%%%%%%%%%
\subsection{Copyright}

Copyright \copyright{} 2017--2018 Niklas Beisert

This work may be distributed and/or modified under the
conditions of the \LaTeX{} Project Public License, either version 1.3
of this license or (at your option) any later version.
The latest version of this license is in
  \url{http://www.latex-project.org/lppl.txt}
and version 1.3 or later is part of all distributions of \LaTeX{}
version 2005/12/01 or later.

This work has the LPPL maintenance status `maintained'.

The Current Maintainer of this work is Niklas Beisert.

This work consists of the files |README.txt|, |childdoc.ins| and |childdoc.dtx|
as well as the derived files |childdoc.def|, |cdocsamp.tex|
with |cdocsch1.tex|, |cdocsch2.tex|, |cdocspt3.tex|, |cdocspt4.tex|,
|cdocsdrf.tex|, |cdocsfn1.tex|, |cdocsfn2.tex|
as well as |childdoc.pdf|.

%%%%%%%%%%%%%%%%%%%%%%%%%%%%%%%%%%%%%%%%%%%%%%%%%%%%%%%%%%%%%%%%%%%%%%%%%%%%%%%%
\subsection{Files and Installation}

The package consists of the files:
%
\begin{center}
\begin{tabular}{ll}
    |README.txt|   & readme file \\
    |childdoc.ins| & installation file \\
    |childdoc.dtx| & source file \\
    |childdoc.def| & definition file \\
    |cdocsamp.tex| & sample main file \\
    |cdocsch1.tex| & sample include file \\
    |cdocsch2.tex| & sample include file \\
    |cdocspt3.tex| & sample part file \\
    |cdocspt4.tex| & sample part file \\
    |cdocsdrf.tex| & sample redirection file \\
    |cdocsfn1.tex| & sample redirection file \\
    |cdocsfn2.tex| & sample redirection file \\
    |childdoc.pdf| & manual
\end{tabular}
\end{center}
%
The distribution consists of the files
|README.txt|, |childdoc.ins| and |childdoc.dtx|.
%
\begin{itemize}
\item
Run (pdf)\LaTeX{} on |childdoc.dtx|
to compile the manual |childdoc.pdf| (this file).
\item
Run \LaTeX{} on |childdoc.ins| to create the definitions file |childdoc.def|
and the sample |cdocsamp.tex| with include files
|cdocsch1.tex|, |cdocsch2.tex|, |cdocspt3.tex|, |cdocspt4.tex|,
|cdocsdrf.tex|, |cdocsfn1.tex|, |cdocsfn2.tex|.
Then copy the file |childdoc.def| to an appropriate directory of your \LaTeX{}
distribution, e.g.\ \textit{texmf-root}|/tex/latex/childdoc|.
\end{itemize}

%%%%%%%%%%%%%%%%%%%%%%%%%%%%%%%%%%%%%%%%%%%%%%%%%%%%%%%%%%%%%%%%%%%%%%%%%%%%%%%%
\subsection{Related CTAN Packages}

There are several other packages which offer a similar functionality:
%
\begin{itemize}
\item
The packages
\href{http://ctan.org/pkg/docmute}{\textsf{docmute}},
\href{http://ctan.org/pkg/includex}{\textsf{includex}} and
\href{http://ctan.org/pkg/standalone}{\textsf{standalone}}
provide commands to include only the document body of
a child file thus allowing both files to be compiled individually.
\item
The packages \href{http://ctan.org/pkg/subdocs}{\textsf{subdocs}}
and \href{http://ctan.org/pkg/subfiles}{\textsf{subfiles}}
provide structures in which the main and child documents can be
encapsulated and allowing them to be compiled individually.
The inclusion mechanism is different from the conventional |\include|.
\item
The package \href{http://ctan.org/pkg/combine}{\textsf{combine}}
is an elaborate solution to combine several documents into one.
\end{itemize}
%
See also the CTAN topic \href{http://ctan.org/topic/subdocs}{\textsf{subdocs}}
for further related packages.
The present package differs from the above solutions in that
a document structure constructed with the conventional |\include| mechanism
just needs two extra commands at the top of every file
such that all constituent files can be compiled individually.

%%%%%%%%%%%%%%%%%%%%%%%%%%%%%%%%%%%%%%%%%%%%%%%%%%%%%%%%%%%%%%%%%%%%%%%%%%%%%%%%
%\subsection{Feature Suggestions}
%
%The following is a list of features which may be useful for future
%versions of this package:
%%
%\begin{itemize}
%\item
%\ldots
%\end{itemize}

%%%%%%%%%%%%%%%%%%%%%%%%%%%%%%%%%%%%%%%%%%%%%%%%%%%%%%%%%%%%%%%%%%%%%%%%%%%%%%%%
\subsection{Revision History}

%%%%%%%%%%%%%%%%%%%%%%%%%%%%%%%%%%%%%%%%
\paragraph{v2.0:} 2018/12/30

\begin{itemize}
\item
immediate forward processing
\item
added |\childdocby| mechanism
\item
manual restructured
\end{itemize}

%%%%%%%%%%%%%%%%%%%%%%%%%%%%%%%%%%%%%%%%
\paragraph{v1.6:} 2018/01/17

\begin{itemize}
\item
application for development of include files
\item
corrections to manual
\end{itemize}

%%%%%%%%%%%%%%%%%%%%%%%%%%%%%%%%%%%%%%%%
\paragraph{v1.5:} 2017/05/21

\begin{itemize}
\item
more complete structuring introduced
\item
|\childdocof| introduced
\item
|\childdoc| renamed to |\childdocmain|
\item
|\childredirect| renamed to |\childdocforward| and |\childdocforwardprefix|
and functionality expanded
\end{itemize}

%%%%%%%%%%%%%%%%%%%%%%%%%%%%%%%%%%%%%%%%
\paragraph{v1.0:} 2017/04/27

\begin{itemize}
\item
manual and install package
\item
first version published on CTAN
\end{itemize}

%%%%%%%%%%%%%%%%%%%%%%%%%%%%%%%%%%%%%%%%
\paragraph{v0.6:} 2017/04/26

\begin{itemize}
\item
redirection mechanism added
\end{itemize}

%%%%%%%%%%%%%%%%%%%%%%%%%%%%%%%%%%%%%%%%
\paragraph{v0.5:} 2017/04/26

\begin{itemize}
\item
functionality in definition file
\end{itemize}


%%%%%%%%%%%%%%%%%%%%%%%%%%%%%%%%%%%%%%%%%%%%%%%%%%%%%%%%%%%%%%%%%%%%%%%%%%%%%%%%
%%%%%%%%%%%%%%%%%%%%%%%%%%%%%%%%%%%%%%%%%%%%%%%%%%%%%%%%%%%%%%%%%%%%%%%%%%%%%%%%
%%%%%%%%%%%%%%%%%%%%%%%%%%%%%%%%%%%%%%%%%%%%%%%%%%%%%%%%%%%%%%%%%%%%%%%%%%%%%%%%
\appendix

\settowidth\MacroIndent{\rmfamily\scriptsize 000\ }

 \DocInput{childdoc.dtx}

\end{document}
%</driver>
% \fi
%
% %%%%%%%%%%%%%%%%%%%%%%%%%%%%%%%%%%%%%%%%%%%%%%%%%%%%%%%%%%%%%%%%%%%%%%%%%%%%%%
% %%%%%%%%%%%%%%%%%%%%%%%%%%%%%%%%%%%%%%%%%%%%%%%%%%%%%%%%%%%%%%%%%%%%%%%%%%%%%%
% \section{Sample}
%\iffalse
%<*samplemain>
%\fi
%
% The following presents a sample document
% with two chapters, two parts, a title page,
% a compile flag as well as three forwarding files to set the flag.
% It consists of eight |.tex| files:
% \begin{center}
% \begin{tabular}{ll}
% |cdocsamp.tex|&main file\\
% |cdocsch1.tex|&include file for chapter 1\\
% |cdocsch2.tex|&include file for chapter 2\\
% |cdocspt3.tex|&include file for part 3\\
% |cdocspt4.tex|&include file for part 4\\
% |cdocsdrf.tex|&forwarding file for main file in draft mode\\
% |cdocsfi1.tex|&forwarding file for final version of chapter 1\\
% |cdocsfi2.tex|&forwarding file for final version of chapter 2\\
% \end{tabular}
% \end{center}
% Each of the eight files can be compiled directly by the \LaTeX{} compiler.
%
% %%%%%%%%%%%%%%%%%%%%%%%%%%%%%%%%%%%%%%
% \paragraph{Main File.}
%
% The main file is called |cdocsamp.tex|.
%
% Load the \textsf{childdoc} definitions and
% declare the filename for the main document:
%    \begin{macrocode}
\input{childdoc.def}
\childdocmain{}
%    \end{macrocode}

% Optional override for |\version| flag:
%    \begin{macrocode}
%%\ifchilddoc\else\providecommand{\version}{draft}\fi
%    \end{macrocode}

% Define the default values for the |\version| flag
% (|final| for the main file and |draft| for childs):
%    \begin{macrocode}
\ifchilddoc
\providecommand{\version}{draft}
\else
\providecommand{\version}{final}
\fi
%    \end{macrocode}

% Load the standard document class:
%    \begin{macrocode}
\documentclass[12pt]{article}
%    \end{macrocode}

% Start the document body:
%    \begin{macrocode}
\begin{document}
%    \end{macrocode}

% Declare a title page.
% Print title, part of document being processed and version flag:
%    \begin{macrocode}
\addtocounter{page}{-1}
\begin{center}
{\LARGE\bfseries{}childdoc example\par}
\vspace{1cm}
\ifchilddoc
\ifchilddocmanual part\else chapter\fi:
`\childdocname' of `\childdocjob'\par
\else
main document: `\childdocjob'\par
\fi
version: \version\par
\end{center}
\newpage
%    \end{macrocode}

% Manually include selected file,
% otherwise process as usual:
%    \begin{macrocode}
\ifchilddocmanual
\section*{part `\childdocname'}
\input{\childdocname}
\else
%    \end{macrocode}

% Include the two chapters:
%    \begin{macrocode}
\include{cdocsch1}
\include{cdocsch2}
%    \end{macrocode}

% Include the two parts unless only chapters should be displayed:
%    \begin{macrocode}
\ifchilddoc\else
\section{part three}
\input{cdocspt3}
\section{part four}
\input{cdocspt4}
\fi
%    \end{macrocode}

% Process as usual until here:
%    \begin{macrocode}
\fi
%    \end{macrocode}

% End of document body:
%    \begin{macrocode}
\end{document}
%    \end{macrocode}
%\iffalse
%</samplemain>
%\fi
%
% %%%%%%%%%%%%%%%%%%%%%%%%%%%%%%%%%%%%%%
% \paragraph{Chapter Include Files.}
%
% The include files are called |cdocsch1.tex| and |cdocsch2.tex|.
%
%\iffalse
%<*samplechap1|samplechap2>
%\fi

% Optional override for |\version| flag:
%    \begin{macrocode}
%%\providecommand{\version}{final}
%    \end{macrocode}

% Include the main document:
%    \begin{macrocode}
\input{childdoc.def}
\childdocof{cdocsamp}
%    \end{macrocode}

%\iffalse
%</samplechap1|samplechap2>
%\fi
%
%\iffalse
%<*samplechap1>
%\fi
% Some text for chapter 1:
%    \begin{macrocode}
\section{one}
some text in chapter one
%    \end{macrocode}

%\iffalse
%</samplechap1>
%\fi
% Some text for chapter 2:
%\iffalse
%<*samplechap2>
%\fi
%    \begin{macrocode}
\section{two}
more text in chapter two
%    \end{macrocode}

%\iffalse
%</samplechap2>
%\fi
%
% %%%%%%%%%%%%%%%%%%%%%%%%%%%%%%%%%%%%%%
% \paragraph{Part Include Files.}
%
% The include files are called |cdocspt3.tex| and |cdocspt4.tex|.
%
%\iffalse
%<*samplepart3|samplepart4>
%\fi

% Optional override for |\version| flag:
%    \begin{macrocode}
%%\providecommand{\version}{final}
%    \end{macrocode}

% Include the main document:
%    \begin{macrocode}
\input{childdoc.def}
\childdocby{cdocsamp}
%    \end{macrocode}

%\iffalse
%</samplepart3|samplepart4>
%\fi
%
%\iffalse
%<*samplepart3>
%\fi
% Some text for part 3:
%    \begin{macrocode}
some text in part three
%    \end{macrocode}

%\iffalse
%</samplepart3>
%\fi
% Some text for part 4:
%\iffalse
%<*samplepart4>
%\fi
%    \begin{macrocode}
more text in part four
%    \end{macrocode}

%\iffalse
%</samplepart4>
%\fi
%
% %%%%%%%%%%%%%%%%%%%%%%%%%%%%%%%%%%%%%%
% \paragraph{Forwarding for a Complete Draft.}
%
% The following forwarding file |cdocsdrf.tex|
% compiles the main document in draft mode:
%\iffalse
%<*sampledraft>
%\fi
%    \begin{macrocode}
\def\version{draft}
\input{childdoc.def}
\childdocforward{cdocsamp}
%    \end{macrocode}

%\iffalse
%</sampledraft>
%\fi
%
% %%%%%%%%%%%%%%%%%%%%%%%%%%%%%%%%%%%%%%
% \paragraph{Forwarding for Final Version of the Chapters.}
%
% The following forwarding files |cdocsfn1.tex| and |cdocsfn2.tex|
% (with identical content)
% compile the final versions of the child documents
% |cdocsch1.tex| and |cdocsch2.tex|, respectively:
%\iffalse
%<*samplefinal>
%\fi
%    \begin{macrocode}
\def\version{final}
\input{childdoc.def}
\childdocforwardprefix[cdocsamp]{cdocsfn}{cdocsch}
%    \end{macrocode}

%\iffalse
%</samplefinal>
%\fi
%
% %%%%%%%%%%%%%%%%%%%%%%%%%%%%%%%%%%%%%%
% \paragraph{Command Line Processing.}
%
% The following three command lines generate the output files
% |cdocscld|, |cdocscl1| and |cdocscl2|
% which should be identical to
% |cdocsdrf|, |cdocsch1| and |cdocsfn2|, respectively:
% \begin{center}
% \begin{tabular}{l}
% |latex -jobname cdocscld \|\\
% |  "\def\version{draft}\input{childdoc.def}\childdocforward{cdocsamp}"|\\
% |latex -jobname cdocscl1 \|\\
% |  "\input{childdoc.def}\childdocforward[cdocsamp]{cdocsch1}"|\\
% |latex -jobname cdocscl2 \|\\
% |  "\def\version{final}\input{childdoc.def}\childdocforward{cdocsch2}"|
% \end{tabular}
% \end{center}
% Note that the trailing backslash on each first line
% merely continues the input to the second line
% (for convenient cut ant paste).
% Furthermore, the command |latex| can be replaced by any
% of its alternative versions such as |pdflatex|.
%
% %%%%%%%%%%%%%%%%%%%%%%%%%%%%%%%%%%%%%%%%%%%%%%%%%%%%%%%%%%%%%%%%%%%%%%%%%%%%%%
% %%%%%%%%%%%%%%%%%%%%%%%%%%%%%%%%%%%%%%%%%%%%%%%%%%%%%%%%%%%%%%%%%%%%%%%%%%%%%%
% \section{Implementation}
%\iffalse
%<*package>
%\fi
%
% This section describes the definitions file |childdoc.def|.

% The definitions cannot be loaded using |\usepackage| or |\RequirePackage|
% which has a mechanism to prevent loading a style file more than once.
% When loading the definitions by means of |\input|
% multiple instances have to be prevented manually:
%\iffalse
%This code needs to be before the `\ProvidesFile' directive
%which is defined at the beginning of this file.
%Therefore it is also placed there and commented out here.
%</package>
%<*discard>
%\fi
%    \begin{macrocode}
\ifdefined\childdocmain\endinput\fi
%    \end{macrocode}
%\iffalse
%</discard>
%<*package>
%\fi
%
% \macro{\ifchilddoc}
% \macro{\ifchilddocmanual}
% The conditional |\ifchilddoc| tells whether a
% child (true) or main (false) document is being compiled.
% The conditional |\ifchilddocmanual| tells whether
% the |\includeonly| mechanism is used (false) or
% the selection of child files must be performed manually (true).
% The definitions initialise to false:
%    \begin{macrocode}
\newif\ifchilddoc
\newif\ifchilddocmanual
%    \end{macrocode}

% \macro{\childdocname}
% \macro{\childdocjob}
% The macro |\childdocname| stores the name of the main document
% to be compiled. The macro |\childdocjob| stores the name of
% the document on which the \LaTeX{} compiler was originally invoked.
% The content of |\jobname| cannot be compared
% to filenames specified in the source due to different catcodes.
% The following code rescans |\jobname|, stores the result
% in |\childdocname| and saves a copy in |\childdocjob|:
%    \begin{macrocode}
\edef\childdocname{\scantokens\expandafter{\jobname\noexpand}}
\let\childdocjob\childdocname
%    \end{macrocode}

% \macro{\childdocdisable}
% The macro |\childdocdisable| prevents the main file
% from being processed more than once.
% At this stage, the main document command |\childdocmain|
% is assumed to be called once again where it should do nothing.
% Any subsequent call to it should prevent
% a secondary processing of the main document
% It overwrites the forwarding commands
% |\childdocof| and |\childdocforward|
% with empty macros to prevent further inclusions of the main document:
%    \begin{macrocode}
\newcommand{\childdocdisable}
{
  \renewcommand{\childdocmain}[1]{\renewcommand{\childdocmain}[1]{\endinput}}
  \renewcommand{\childdocof}[1]{}
  \renewcommand{\childdocby}[2][]{}
  \renewcommand{\childdocforward}[2][]{}
  \renewcommand{\childdocdisable}{}
}
%    \end{macrocode}

% \macro{\childdocmain}
% The macro |\childdocmain| is to be called at the top of the main file
% with nothing or the main filename (without extension) as argument.
% First, it breaks loops.
% If the argument is not empty and does not match |\childdocname|
% (which is set by the first inclusion of |childdoc.def|),
% |\ifchilddoc| is set to true, |\includeonly| is applied to the child file
% and |\jobname| is set to the main file
% (for proper handling of |.aux| files):
%    \begin{macrocode}
\newcommand{\childdocmain}[1]
{
  \childdocdisable\childdocmain{}
  \if?#1?\else
    \begingroup
      \def\childdoctmp{#1}
      \ifx\childdoctmp\childdocname
        \def\childdoctmp{}
      \else
        \def\childdoctmp
        {
          \childdoctrue
          \includeonly{\childdocname}
          \def\childdocjob{#1}
          \def\jobname{#1}
        }
      \fi
      \expandafter
    \endgroup
    \childdoctmp
  \fi
}
%    \end{macrocode}

% \macro{\childdocof}
% The command |\childdocof| redirects
% compilation to the main file |#1|.
%    \begin{macrocode}
\newcommand{\childdocof}[1]
{
  \childdocdisable
  \childdoctrue
  \includeonly{\childdocname}
  \def\jobname{#1}
  \def\childdocjob{#1}
  \input{#1}
}
%    \end{macrocode}

% \macro{\childdocby}
% The command |\childdocby| ....
%    \begin{macrocode}
\newcommand{\childdocby}[2][]
{
  \childdocdisable
  \childdoctrue
  \childdocmanualtrue
  \if?#1?\else
    \def\jobname{#2}
  \fi
  \def\childdocjob{#2}
  \input{#2}
  \endinput
}
%    \end{macrocode}

% \macro{\childdocforward}
% The command |\childdocforward| redirects
% compilation to the main file or
% (if the optional argument is given) a child file.
% Parameters are set as if the main file
% or a child file starting with |\childdocof| was compiled.
% Then compilation is handed over to the main file:
%    \begin{macrocode}
\newcommand{\childdocforward}[2][]
{
  \begingroup
    \if?#1?
      \def\childdoctmp
      {
        \def\childdocname{#2}
        \def\childdocjob{#2}
        \def\jobname{#2}
        \input{#2}
        \endinput
      }
    \else
      \def\childdoctmp
      {
        \childdocdisable
        \def\childdocname{#2}
        \childdoctrue
        \includeonly{#2}
        \def\childdocjob{#1}
        \def\jobname{#1}
        \input{#1}
        \endinput
      }
    \fi
    \expandafter
  \endgroup
  \childdoctmp
}
%    \end{macrocode}

% \macro{\childdocforwardprefix}
% The command |\childdocforwardprefix| redirects
% compilation to the main or a child file by means of a pattern.
% The prefix |#1| in the current filename is replaced by |#2|
% and the suffix of the current filename is kept
% (it is assumed that the filename does not contain the substring `|~~~|'
% which is used as a delimiter).
% Compilation is handed over to the new file by |\childdocforward|:
%    \begin{macrocode}
\newcommand{\childdocforwardprefix}[3][]
{
  \begingroup
    \def\childdocextract #2##1~~~{\def\childdoctmp{\childdocforward[#1]{#3##1}}}
    \expandafter\childdocextract\childdocname~~~
    \expandafter
  \endgroup
  \childdoctmp
}
%    \end{macrocode}

% \macro{\childdoc}
% The deprecated macro |\childdoc| is a legacy version of |\childdocmain|:
%    \begin{macrocode}
\newcommand{\childdoc}{\childdocmain}
%    \end{macrocode}

% \macro{\childdocredirect}
% The deprecated macro |\childdocredirect| is a legacy version
% of |\childdocforward| and |\childdocforwardprefix|:
%    \begin{macrocode}
\newcommand{\childdocredirect}[2][]
{
  \begingroup
    \if?#1?
      \def\childdoctmp{\childdocforward{#2}}
    \else
      \def\childdoctmp{\childdocforwardprefix{#1}{#2}}
    \fi
    \expandafter
  \endgroup
  \childdoctmp
}
%    \end{macrocode}

%\iffalse
%</package>
%\fi
%
\endinput
|\\
|\childdocmain{|\textit{main}|}|\\
\end{tabular}
\end{center}
%
If |\jobname| does not match the argument \textit{main} of |\childdocmain|,
it is assumed that |\jobname| points to the child file to be compiled.
When using |\childdocmain| with the main file specified as argument,
it suffices to start a child file
with just |\input{|\textit{main}|}|
without loading of the package and using |\childdocof|.
If instead all processing is done
with the appropriate \textsf{childdoc} directives,
the argument of \textit{main} of |\childdocmain| can be empty.

An alternative version of the command line processing described
in \secref{sec:commandline} using the detection mechanism reads:
%
\begin{center}
|... -jobname "|\textit{target}|" "|[\textit{flags}]%
[|\def\jobname{|\textit{dest}|}|]|\input{|\textit{main}|}"|
\end{center}

%%%%%%%%%%%%%%%%%%%%%%%%%%%%%%%%%%%%%%%%%%%%%%%%%%%%%%%%%%%%%%%%%%%%%%%%%%%%%%%%
\subsection{Manual Code}
\label{sec:manual}

In case one cannot be certain whether the definitions file |childdoc.def|
is installed on the target \TeX{} distribution
and one prefers not to ship it,
it is conceivable to paste a few relevant commands into the sources.

To that end, drop all statements |% \iffalse
%
% childdoc.dtx Copyright (C) 2017-2018 Niklas Beisert
%
% This work may be distributed and/or modified under the
% conditions of the LaTeX Project Public License, either version 1.3
% of this license or (at your option) any later version.
% The latest version of this license is in
%   http://www.latex-project.org/lppl.txt
% and version 1.3 or later is part of all distributions of LaTeX
% version 2005/12/01 or later.
%
% This work has the LPPL maintenance status `maintained'.
%
% The Current Maintainer of this work is Niklas Beisert.
%
% This work consists of the files childdoc.dtx and childdoc.ins
% and the derived files childdoc.def and cdocsamp.tex with
% cdocsch1.tex, cdocsch2.tex, cdocsdrf.tex, cdocsfn1.tex, cdocsfn2.tex.
%
%<package>\ifdefined\childdocmain\endinput\fi
%<package>\ProvidesFile{childdoc.def}[2018/12/30 v2.0 child document driver]
%<samplemain>\ProvidesFile{cdocsamp.tex}[2018/12/30 v2.0 sample for childdoc]
%<*driver>
%\ProvidesFile{childdoc.drv}[2018/12/30 v2.0 childdoc reference manual file]
\PassOptionsToClass{10pt,a4paper}{article}
\documentclass{ltxdoc}

\usepackage[margin=35mm]{geometry}
\usepackage{hyperref}
\usepackage{hyperxmp}
\usepackage[usenames]{color}

\hypersetup{colorlinks=true}
\hypersetup{pdfstartview=FitH}
\hypersetup{pdfpagemode=UseNone}
\hypersetup{pdfsource={}}
\hypersetup{pdflang={en-UK}}
\hypersetup{pdfcopyright={Copyright 2017-2018 Niklas Beisert.
  This work may be distributed and/or modified under the
  conditions of the LaTeX Project Public License, either version 1.3
  of this license or (at your option) any later version.}}
\hypersetup{pdflicenseurl={http://www.latex-project.org/lppl.txt}}
\hypersetup{pdfcontactaddress={ETH Zurich, ITP, HIT K,
  Wolfgang-Pauli-Strasse 27}}
\hypersetup{pdfcontactpostcode={8093}}
\hypersetup{pdfcontactcity={Zurich}}
\hypersetup{pdfcontactcountry={Switzerland}}
\hypersetup{pdfcontactemail={nbeisert@itp.phys.ethz.ch}}
\hypersetup{pdfcontacturl={http://people.phys.ethz.ch/\xmptilde nbeisert/}}

\newcommand{\secref}[1]{\hyperref[#1]{section \ref*{#1}}}

\parskip1ex
\parindent0pt
\let\olditemize\itemize
\def\itemize{\olditemize\parskip0pt}

\begin{document}

\title{The \textsf{childdoc} Package}
\hypersetup{pdftitle={The childdoc Package}}
\author{Niklas Beisert\\[2ex]
  Institut f\"ur Theoretische Physik\\
  Eidgen\"ossische Technische Hochschule Z\"urich\\
  Wolfgang-Pauli-Strasse 27, 8093 Z\"urich, Switzerland\\[1ex]
  \href{mailto:nbeisert@itp.phys.ethz.ch}
  {\texttt{nbeisert@itp.phys.ethz.ch}}}
\hypersetup{pdfauthor={Niklas Beisert}}
\hypersetup{pdfsubject={Manual for the LaTeX2e Package childdoc}}
\date{30 December 2018, \textsf{v2.0}}
\maketitle

\begin{abstract}\noindent
\textsf{childdoc} is a \LaTeXe{} package
that enables the direct compilation
of document sections included by |\include|
to individual files.
\end{abstract}

\begingroup
\parskip0ex
\tableofcontents
\endgroup

%%%%%%%%%%%%%%%%%%%%%%%%%%%%%%%%%%%%%%%%%%%%%%%%%%%%%%%%%%%%%%%%%%%%%%%%%%%%%%%%
%%%%%%%%%%%%%%%%%%%%%%%%%%%%%%%%%%%%%%%%%%%%%%%%%%%%%%%%%%%%%%%%%%%%%%%%%%%%%%%%
\section{Introduction}

\LaTeX{} provides a mechanism to structure a large document (such as a book)
into a main file and several child files (containing the chapters)
using the |\include| command.
This mechanism is beneficial for documents
which span hundreds of pages in order to
make the source file(s) more manageable.
Moreover, compilation can be restricted to
selected child files by means of the |\includeonly| command.
The latter feature can be used to reduce the compilation time while editing
(this was significantly more useful in the earlier days of \LaTeX{})
or to generate a smaller document which is easier to navigate.
Another application of |\includeonly| is to generate
documents consisting of selected parts of the complete document.

However, there are a few drawbacks of the plain |\include| mechanism:
\begin{itemize}
\item
The child files cannot be compiled on their own,
they can only be compiled via the main file.
A naive editing environment
(such as a text editor with an option
to have the current file processed by \LaTeX)
may require one to switch to the main file before compiling;
attempting to compile the child file produces errors.
\item
The main file must be modified (each time)
to adjust the |\includeonly| command
to the present needs. This easily leaves the main file in a messy state.
\item
The generated document will always carry the filename
of the main document. This is inconvenient if
several child files are to be compiled and
to be kept for distribution.
\end{itemize}

The present package provides a simple interface
to make child files individually compilable by \LaTeX{}.
Compiling a child file then has the same effect as compiling
the main file with an |\includeonly| command
to select the appropriate child.
Moreover the generated document will carry the name of the child
rather than the main file.
This resolves all three above issues.

This feature is meant to make the editing of books,
thesis documents and lecture notes somewhat more convenient.
However, the package can also be used efficiently for
composing a series of documents (such as exercise sheets)
which are typically distributed individually.
It then assists the author in generating the individual documents
(potentially in different versions)
as well as a document containing the collected series.
Another application is in developing style files
or other kinds of included material
where compilation of the style file could redirect
to a sample or test file.

%%%%%%%%%%%%%%%%%%%%%%%%%%%%%%%%%%%%%%%%%%%%%%%%%%%%%%%%%%%%%%%%%%%%%%%%%%%%%%%%
%%%%%%%%%%%%%%%%%%%%%%%%%%%%%%%%%%%%%%%%%%%%%%%%%%%%%%%%%%%%%%%%%%%%%%%%%%%%%%%%
\section{Usage}

First of all, the package \textsf{childdoc} is \emph{not} a standard
\LaTeXe{} |.sty| style file! Therefore it needs to be invoked in
a non-standard way.

%%%%%%%%%%%%%%%%%%%%%%%%%%%%%%%%%%%%%%%%%%%%%%%%%%%%%%%%%%%%%%%%%%%%%%%%%%%%%%%%
\subsection{Included Files}
\label{sec:include}

%%%%%%%%%%%%%%%%%%%%%%%%%%%%%%%%%%%%%%%%
\DescribeMacro{\childdocmain}
To use the package, add the commands
\begin{center}
\begin{tabular}{l}
|\input{childdoc.def}|\\
|\childdocmain{}|\\
\end{tabular}
\end{center}
at the very top of the main \LaTeX{} file,
in particular \emph{before} the |\documentclass| statement!
The argument of |\childdocmain| should be left empty
(but it must be present).

%%%%%%%%%%%%%%%%%%%%%%%%%%%%%%%%%%%%%%%%
\DescribeMacro{\childdocof}
Furthermore, add the commands
\begin{center}
\begin{tabular}{l}
|\input{childdoc.def}|\\
|\childdocof{|\textit{main}|}|\\
\end{tabular}
\end{center}
at the top of every child file \textit{child}
which is included by |\include{|\textit{child}|}|
from within the main file
(or at least for those files to be compiled individually).
The argument \textit{main} must be the filename of the main file.

There are a couple of
considerations in setting up the main and child documents:

%%%%%%%%%%%%%%%%%%%%%%%%%%%%%%%%%%%%%%%%
\paragraph{Restrictions.}

Please note the following restrictions:
\begin{itemize}
\item
|\childdocmain| must be called with one argument \textit{main}
to ensure compatibility with earlier version of the package.
It must either be empty (|\childdocmain{}|)
or precisely match the filename of the main file in which it is specified.
See \secref{sec:detection} for further information.
\item
The filename \textit{main} must be specified without the |.tex| extension.
\item
The filename \textit{main} is case sensitive
(even in case-insensitive file systems)
due to internal string comparison.
\item
The argument \textit{main} should be fully expanded, it cannot be a macro.
\item
Subdirectories and special characters should be avoided in filenames.
\item
The command |\childdocmain{|\textit{main}|}| must be followed by a whitespace.
It should not be followed immediately by another command
or by a comment mark `|%|'.
This is because the \TeX{} parser reads the token immediately following
the argument of |\childdocmain| and puts it
at the beginning of every child section;
however, a white\-space is ignored.
\end{itemize}

%%%%%%%%%%%%%%%%%%%%%%%%%%%%%%%%%%%%%%%%
\paragraph{Content of Main File.}

It is advisable to place all content in the child files included by |\include|.
Any output contained in the main file will appear in all child documents
unless suppressed manually;
it cannot be suppressed automatically by the |\includeonly| directive
and thus should normally be avoided.
A method to include some content in the main file
by means of conditional processing is described in \secref{sec:conditional}.

%%%%%%%%%%%%%%%%%%%%%%%%%%%%%%%%%%%%%%%%
\paragraph{Page Numbering.}

When only a part of the document is compiled,
the appropriate numbering of pages
(as well as other status parameters)
is determined from the |.aux| files.
The latter contain information from previous passes.
However this information needs to propagate through
all intermediate child documents.
Therefore the page numbering in child documents may well
be inconsistent until the complete document is compiled at least once.

A useful (if unconventional) way to always ensure a consistent
page numbering is to restart the numbering in each child document
and denote the pages by `\textit{child}|.|\textit{page}'
where \textit{child} represents the chapter/section number of the child file.
This can be achieved by the command
|\numberwithin{page}{|\textit{child}|}|
of the \textsf{amsmath} package
where \textit{child} can be |chapter| or |section|
depending on the chosen structuring.
Alternatively, one can modify the macro |\thepage| appropriately
and reset the counter |page| at the start of each child file.

%%%%%%%%%%%%%%%%%%%%%%%%%%%%%%%%%%%%%%%%%%%%%%%%%%%%%%%%%%%%%%%%%%%%%%%%%%%%%%%%
\subsection{Conditional Processing}
\label{sec:conditional}

The package provides a mechanism to compile different versions
of a document. To customise the versions further some conditional processing
can come in handy to distinguish which version is being compiled.
The package provides two macros to describe the compilation context:

%%%%%%%%%%%%%%%%%%%%%%%%%%%%%%%%%%%%%%%%
\DescribeMacro{\ifchilddoc}
The conditional |\ifchilddoc| distinguishes between the compilation of
child documents and the main document:
%
\begin{center}
|\ifchilddoc |\textit{child-code}| |[|\||else |\textit{main-code}]| \||fi|
\end{center}

%%%%%%%%%%%%%%%%%%%%%%%%%%%%%%%%%%%%%%%%
\DescribeMacro{\childdocname}
\DescribeMacro{\childdocjob}
The macro |\childdocname| contains the filename (without extension)
of the main or child file being processed.
Note that |\childdocjob| will always contain the name of the main file.

%%%%%%%%%%%%%%%%%%%%%%%%%%%%%%%%%%%%%%%%
\paragraph{Title Page.}

Conditional processing can be used to include a title or banner page
in the main document when proper precautions are taken.
Importantly, the code in the main file should ensure that the page counter
(as well as other status parameters which are stored in the |.aux| files)
takes the same value after the conditional processing.
Otherwise the page numbers may take divergent values
depending on which part is compiled.

For example, a title page could be declared by:
%
\begin{center}
\begin{tabular}{l}
|\ifchilddoc\||else|\\
|\addtocounter{page}{-1}|\\
\textit{code for title page}\\
|\newpage|\\
|\||fi|
\end{tabular}
\end{center}
%
A banner page for the child documents can be generated by:
%
\begin{center}
\begin{tabular}{l}
|\ifchilddoc|\\
|\addtocounter{page}{-1}|\\
\textit{code for banner page}\\
|\newpage|\\
|\||fi|
\end{tabular}
\end{center}
%
Here one could write a message such as:
\begin{center}
|This is the part \childdocname{} of \childdocjob{}.|
\end{center}

%%%%%%%%%%%%%%%%%%%%%%%%%%%%%%%%%%%%%%%%%%%%%%%%%%%%%%%%%%%%%%%%%%%%%%%%%%%%%%%%
\subsection{Flags}
\label{sec:flags}

The package makes it easy to generate different versions
of the main or child documents.
To this end compilation flags can be defined
and assigned different default values.
They will be particularly useful in conjunction
with the forwarding mechanism described in \secref{sec:forward}.

For example, it may be useful to have a flag |\version|
which can be set to |draft| or |final|.
The document source will contain some conditional code
depending on the value of |\version|.
Suppose further, the flag should default to |final| for the main file
and to |draft| for child files
which is a natural assignment for editing the document.
This is achieved by placing the following code
in the preamble of the main document
(below the |\childdocmain| directive):
%
\begin{center}
\begin{tabular}{l}
|\ifchilddoc|\\
|\providecommand{\version}{draft}|\\
|\||else|\\
|\providecommand{\version}{final}|\\
|\||fi|
\end{tabular}
\end{center}
%
The definition by |\providecommand| makes sure
that previous definitions are not overwritten.
Further statements |\providecommand{\version}{...}|
can thus be added before the above code to override it.

For the main file, one might add a line
(between |\childdocmain| and the above block)
%
\begin{center}
|%\ifchilddoc\||else\providecommand{\version}{draft}\||fi|
\end{center}
%
which can be uncommented to produce a draft version.
Likewise one can add a line to the very top of a child file
(above the |\childdocof{|\textit{main}|}| directive)
%
\begin{center}
|%\providecommand{\version}{final}|
\end{center}
%
which can be uncommented to produce the final version of this child document.

%%%%%%%%%%%%%%%%%%%%%%%%%%%%%%%%%%%%%%%%%%%%%%%%%%%%%%%%%%%%%%%%%%%%%%%%%%%%%%%%
\subsection{Forwarding}
\label{sec:forward}

Different versions of the main or child documents
using compilation flags as described in \secref{sec:flags}
can be (permanently) stored in different files
for convenient compilation, viewing and distribution.
To this end, the package defines a command
to pass on compilation to a different file:

%%%%%%%%%%%%%%%%%%%%%%%%%%%%%%%%%%%%%%%%
\DescribeMacro{\childdocforward}
The command |\childdocforward| redirects processing to
another source file:
%
\begin{center}
\begin{tabular}{l}
|\input{childdoc.def}|\\
|\childdocforward[|\textit{main}|]{|\textit{dest}|}|\\
\end{tabular}
\end{center}
%
The argument \textit{dest} is the destination file
(without extension).
It should be the main file or one of the child files.
Note that further \textsf{childdoc} directives
such as |\childdocof| and |\childdocforward|
in the indicated file will be processed in this form.
The optional argument \textit{main}
passes on directly to the main file \textit{main}
while pretending to compile the child \textit{dest}.
This form behaves as if \textit{dest}
issues |\childdocof{|\textit{main}|}| right away,
and no further \textsf{childdoc} directives will be processed.

%%%%%%%%%%%%%%%%%%%%%%%%%%%%%%%%%%%%%%%%
\DescribeMacro{\...prefix}
In the alternative form |\childdocforwardprefix|,
%
\begin{center}
\begin{tabular}{l}
|\input{childdoc.def}|\\
|\childdocforwardprefix[|\textit{main}|]{|\textit{prefix}|}{|\textit{dest}|}|
\end{tabular}
\end{center}
%
the destination file is determined by a pattern
depending on the current file:
To make this work, the current file must be called
`{\textit{prefix}\hspace{0.2em}\textit{suffix}}'
with \textit{prefix} matching precisely the argument.
Processing is then passed on to the file
`{\textit{dest}\hspace{0.2em}\textit{suffix}}'.
Surely, the same effect is achieved by
directly specifying the
argument `{\textit{dest}\hspace{0.2em}\textit{suffix}}'
in the first form.
However, that requires to set up a different file
for each child. With the alternative form of the command
all these files can have exactly the same content
which simplifies setting them up and maintaining them.

For example, the following file |draft.tex|
with a compilation flag |\version| as described in \secref{sec:flags}
compiles the main document as a draft:
%
\begin{center}
\begin{tabular}{l}
|\def\version{draft}|\\
|\input{childdoc.def}|\\
|\childdocforward{|\textit{main}|}|
\end{tabular}
\end{center}
%
Likewise, the following files |final|\textit{nn}|.tex|
compile the final version of the child document
|child|\textit{nn}|.tex|:
%
\begin{center}
\begin{tabular}{l}
|\def\version{final}|\\
|\input{childdoc.def}|\\
|\childdocforwardprefix{final}{child}|
\end{tabular}
\end{center}
%

Note that when several versions of a main file and/or of each child file
are to be generated, it may be convenient to set up a |Makefile| or
shell script to automatise the process.

%%%%%%%%%%%%%%%%%%%%%%%%%%%%%%%%%%%%%%%%%%%%%%%%%%%%%%%%%%%%%%%%%%%%%%%%%%%%%%%%
\subsection{Command Line Processing}
\label{sec:commandline}

The effect of redirection files can also be achieved by invoking
the \LaTeX{} compiler with a more elaborate command line.
Most conveniently this should be done as part
of a shell script or a |Makefile|.

When using \textsf{childdoc} in the main file, the following
command lines effectively perform a redirection
(note that depending on the shell being used,
backslashes may have to be doubled: `|\|' $\to$ `|\\|'):
%
\begin{center}
|... -jobname "|\textit{target}|" |\\|"|[\textit{flags}]%
|\input{childdoc.def}\childdocforward[|\textit{main}|]{|\textit{dest}|}"|
\end{center}
%
Here \textit{target} is the name of the output file,
\textit{main} is the name of the main file
and \textit{dest} is the name of the main or child file to be processed
(all filenames without extensions).
The optional argument \textit{main} can be omitted
if \textit{main} matches \textit{dest}.
Optionally, compilation \textit{flags} can be defined via |\def| commands.
This command line makes the \TeX{} engine believe
it is compiling the file \textit{target}
whose content is specified as the latter parameter.
The provided code then forwards the processing to
\textit{main} or \textit{dest} as described in \secref{sec:forward}.

%%%%%%%%%%%%%%%%%%%%%%%%%%%%%%%%%%%%%%%%%%%%%%%%%%%%%%%%%%%%%%%%%%%%%%%%%%%%%%%%
\subsection{Include by Input}
\label{sec:input}

Including child documents by |\include| has some restrictions by design.
Most notably, the content of a child document always occupies
its own set of pages; pages cannot be shared between child documents.
Usually, this behaviour makes perfect sense
because each child document contain an essential part of the document.
However, in some situations it may be desirable to compose
a document from a collection of parts
without having mandatory page breaks between then.
For this case, the package
provides a mechanism to include parts
by |\input| which can also be processed individually.
However, by construction this mechanism
requires manual handling of the content to be output.

%%%%%%%%%%%%%%%%%%%%%%%%%%%%%%%%%%%%%%%%
\DescribeMacro{\ifchilddocmanual}
The main file should be prepared as usual, see \secref{sec:include}.
However, the document body must make a distinction
between processing of an individual part and of the main document, e.g.:
%
\begin{center}
\begin{tabular}{l}
|\ifchilddocmanual|\\
|\input{\childdocname}|\\
|\||else|\\
\textit{document body with }|\input{|\textit{part}|}|\\
|\||fi|
\end{tabular}
\end{center}
%
The conditional |\ifchilddocmanual| is true whenever
a part to be included by |\input| is being compiled,
and the name of the part is stored in |\childdocname|.

%%%%%%%%%%%%%%%%%%%%%%%%%%%%%%%%%%%%%%%%
\DescribeMacro{\childdocby}
Each part to be included by |\input| should start with:
%
\begin{center}
\begin{tabular}{l}
|\input{childdoc.def}|\\
|\childdocby{|\textit{main}|}|\\
\end{tabular}
\end{center}
%
The directive |\childdocby| is similar to |\childdocof|
described in \secref{sec:include},
but the subsequent selection of content must be done manually.
To that end, both |\ifchilddoc| and |\ifchilddocmanual|
will be true upon processing of a part,
and the name of the part is stored in |\childdocname|.
Note that |\jobname| will be set to the filename of the current part
so that each part receives an individual |.aux| file
that does not interfere with the |.aux| file(s) of the main document.
This behaviour can be altered by the alternative form
|\childdocby[*]{|\textit{main}|}| (with a non-empty optional argument)
which uses the |.aux| file of the main document
by setting |\jobname| to \textit{main}.

%%%%%%%%%%%%%%%%%%%%%%%%%%%%%%%%%%%%%%%%%%%%%%%%%%%%%%%%%%%%%%%%%%%%%%%%%%%%%%%%
\subsection{Driver Development}
\label{sec:driver}

The \textsf{childdoc} mechanism can also be use for the development
of definition files such as \LaTeX{} styles or classes.
This case differs from the above setup with multiple parts
included by |\include| in that no |\includeonly| should be invoked.
This can be achieved by starting the include file
(before |\ProvidesPackage|) with:
%
\begin{center}
\begin{tabular}{l}
|\input{childdoc.def}|\\
|\childdocforward{|\textit{main}|}|\\
\end{tabular}
\end{center}
%
or alternatively with:
%
\begin{center}
\begin{tabular}{l}
|\input{childdoc.def}|\\
|\childdocby{|\textit{main}|}|\\
\end{tabular}
\end{center}
%
Both forms have slightly different effects as described above.
The main file is prepared as usual, see \secref{sec:include}.

%%%%%%%%%%%%%%%%%%%%%%%%%%%%%%%%%%%%%%%%%%%%%%%%%%%%%%%%%%%%%%%%%%%%%%%%%%%%%%%%
\subsection{Legacy Detection}
\label{sec:detection}

The directive |\childdocmain| in the main file can detect
whether the complete document or merely a child is to be compiled
even without using the directive |\childdocof|.
This method is deprecated because it is less robust
and there is no compelling reason to use it;
it is merely provided for backward compatibility
and it may be removed in future versions.

If the detection mechanism is to be used,
it is mandatory to correctly specify
the filename of the main file as the argument of |\childdocmain|:
%
\begin{center}
\begin{tabular}{l}
|\input{childdoc.def}|\\
|\childdocmain{|\textit{main}|}|\\
\end{tabular}
\end{center}
%
If |\jobname| does not match the argument \textit{main} of |\childdocmain|,
it is assumed that |\jobname| points to the child file to be compiled.
When using |\childdocmain| with the main file specified as argument,
it suffices to start a child file
with just |\input{|\textit{main}|}|
without loading of the package and using |\childdocof|.
If instead all processing is done
with the appropriate \textsf{childdoc} directives,
the argument of \textit{main} of |\childdocmain| can be empty.

An alternative version of the command line processing described
in \secref{sec:commandline} using the detection mechanism reads:
%
\begin{center}
|... -jobname "|\textit{target}|" "|[\textit{flags}]%
[|\def\jobname{|\textit{dest}|}|]|\input{|\textit{main}|}"|
\end{center}

%%%%%%%%%%%%%%%%%%%%%%%%%%%%%%%%%%%%%%%%%%%%%%%%%%%%%%%%%%%%%%%%%%%%%%%%%%%%%%%%
\subsection{Manual Code}
\label{sec:manual}

In case one cannot be certain whether the definitions file |childdoc.def|
is installed on the target \TeX{} distribution
and one prefers not to ship it,
it is conceivable to paste a few relevant commands into the sources.

To that end, drop all statements |\input{childdoc.def}|
and perform the replacements as outlined below.
Instead of |\childdocmain{|\textit{main}|}| add the following code
to the top of the main file:
%
\begin{center}
\begin{tabular}{l}
|\||ifdefined\childdocname\endinput\||fi\newif\ifchilddoc|\\
|\edef\childdocname{\scantokens\expandafter{\jobname\noexpand}}|\\
|\def\childdocmain{|\textit{main}|}\||ifx\childdocmain\childdocname\||else|\\
|\childdoctrue\includeonly{\childdocname}\let\jobname\childdocmain\||fi|\\
\end{tabular}
\end{center}
%
Instead of |\childdocof{|\textit{main}|}| just include the main file
at the top of each child file:
%
\begin{center}
|\input{|\textit{main}|}|
\end{center}
%
A simple redirection |\childdocforward{|\textit{dest}|}| is achieved by:
%
\begin{center}
|\def\jobname{|\textit{dest}|}\input{\jobname}|
\end{center}
%
The redirection with prefix
|\childdocforwardprefix[|\textit{prefix}|]{|\textit{dest}|}|
is accomplished by:
%
\begin{center}
\begin{tabular}{l}
|{\edef\jobname{\scantokens\expandafter{\jobname\noexpand}}|\\
|\def\redirectjob |\textit{prefix}|#1~~~{\gdef\jobname{|\textit{dest}|#1}}|\\
|\expandafter\redirectjob\jobname~~~}\input{\jobname}|
\end{tabular}
\end{center}

In an alternative approach,
child documents can be compiled by a specific command line
without additional code or specific definitions:
%
\begin{center}
|... -jobname "|\textit{target}|" "|[\textit{flags}]%
|\includeonly{|\textit{dest}|}\input{|\textit{main}|}"|
\end{center}
%

%%%%%%%%%%%%%%%%%%%%%%%%%%%%%%%%%%%%%%%%%%%%%%%%%%%%%%%%%%%%%%%%%%%%%%%%%%%%%%%%
%%%%%%%%%%%%%%%%%%%%%%%%%%%%%%%%%%%%%%%%%%%%%%%%%%%%%%%%%%%%%%%%%%%%%%%%%%%%%%%%
\section{Information}

%%%%%%%%%%%%%%%%%%%%%%%%%%%%%%%%%%%%%%%%%%%%%%%%%%%%%%%%%%%%%%%%%%%%%%%%%%%%%%%%
\subsection{Copyright}

Copyright \copyright{} 2017--2018 Niklas Beisert

This work may be distributed and/or modified under the
conditions of the \LaTeX{} Project Public License, either version 1.3
of this license or (at your option) any later version.
The latest version of this license is in
  \url{http://www.latex-project.org/lppl.txt}
and version 1.3 or later is part of all distributions of \LaTeX{}
version 2005/12/01 or later.

This work has the LPPL maintenance status `maintained'.

The Current Maintainer of this work is Niklas Beisert.

This work consists of the files |README.txt|, |childdoc.ins| and |childdoc.dtx|
as well as the derived files |childdoc.def|, |cdocsamp.tex|
with |cdocsch1.tex|, |cdocsch2.tex|, |cdocspt3.tex|, |cdocspt4.tex|,
|cdocsdrf.tex|, |cdocsfn1.tex|, |cdocsfn2.tex|
as well as |childdoc.pdf|.

%%%%%%%%%%%%%%%%%%%%%%%%%%%%%%%%%%%%%%%%%%%%%%%%%%%%%%%%%%%%%%%%%%%%%%%%%%%%%%%%
\subsection{Files and Installation}

The package consists of the files:
%
\begin{center}
\begin{tabular}{ll}
    |README.txt|   & readme file \\
    |childdoc.ins| & installation file \\
    |childdoc.dtx| & source file \\
    |childdoc.def| & definition file \\
    |cdocsamp.tex| & sample main file \\
    |cdocsch1.tex| & sample include file \\
    |cdocsch2.tex| & sample include file \\
    |cdocspt3.tex| & sample part file \\
    |cdocspt4.tex| & sample part file \\
    |cdocsdrf.tex| & sample redirection file \\
    |cdocsfn1.tex| & sample redirection file \\
    |cdocsfn2.tex| & sample redirection file \\
    |childdoc.pdf| & manual
\end{tabular}
\end{center}
%
The distribution consists of the files
|README.txt|, |childdoc.ins| and |childdoc.dtx|.
%
\begin{itemize}
\item
Run (pdf)\LaTeX{} on |childdoc.dtx|
to compile the manual |childdoc.pdf| (this file).
\item
Run \LaTeX{} on |childdoc.ins| to create the definitions file |childdoc.def|
and the sample |cdocsamp.tex| with include files
|cdocsch1.tex|, |cdocsch2.tex|, |cdocspt3.tex|, |cdocspt4.tex|,
|cdocsdrf.tex|, |cdocsfn1.tex|, |cdocsfn2.tex|.
Then copy the file |childdoc.def| to an appropriate directory of your \LaTeX{}
distribution, e.g.\ \textit{texmf-root}|/tex/latex/childdoc|.
\end{itemize}

%%%%%%%%%%%%%%%%%%%%%%%%%%%%%%%%%%%%%%%%%%%%%%%%%%%%%%%%%%%%%%%%%%%%%%%%%%%%%%%%
\subsection{Related CTAN Packages}

There are several other packages which offer a similar functionality:
%
\begin{itemize}
\item
The packages
\href{http://ctan.org/pkg/docmute}{\textsf{docmute}},
\href{http://ctan.org/pkg/includex}{\textsf{includex}} and
\href{http://ctan.org/pkg/standalone}{\textsf{standalone}}
provide commands to include only the document body of
a child file thus allowing both files to be compiled individually.
\item
The packages \href{http://ctan.org/pkg/subdocs}{\textsf{subdocs}}
and \href{http://ctan.org/pkg/subfiles}{\textsf{subfiles}}
provide structures in which the main and child documents can be
encapsulated and allowing them to be compiled individually.
The inclusion mechanism is different from the conventional |\include|.
\item
The package \href{http://ctan.org/pkg/combine}{\textsf{combine}}
is an elaborate solution to combine several documents into one.
\end{itemize}
%
See also the CTAN topic \href{http://ctan.org/topic/subdocs}{\textsf{subdocs}}
for further related packages.
The present package differs from the above solutions in that
a document structure constructed with the conventional |\include| mechanism
just needs two extra commands at the top of every file
such that all constituent files can be compiled individually.

%%%%%%%%%%%%%%%%%%%%%%%%%%%%%%%%%%%%%%%%%%%%%%%%%%%%%%%%%%%%%%%%%%%%%%%%%%%%%%%%
%\subsection{Feature Suggestions}
%
%The following is a list of features which may be useful for future
%versions of this package:
%%
%\begin{itemize}
%\item
%\ldots
%\end{itemize}

%%%%%%%%%%%%%%%%%%%%%%%%%%%%%%%%%%%%%%%%%%%%%%%%%%%%%%%%%%%%%%%%%%%%%%%%%%%%%%%%
\subsection{Revision History}

%%%%%%%%%%%%%%%%%%%%%%%%%%%%%%%%%%%%%%%%
\paragraph{v2.0:} 2018/12/30

\begin{itemize}
\item
immediate forward processing
\item
added |\childdocby| mechanism
\item
manual restructured
\end{itemize}

%%%%%%%%%%%%%%%%%%%%%%%%%%%%%%%%%%%%%%%%
\paragraph{v1.6:} 2018/01/17

\begin{itemize}
\item
application for development of include files
\item
corrections to manual
\end{itemize}

%%%%%%%%%%%%%%%%%%%%%%%%%%%%%%%%%%%%%%%%
\paragraph{v1.5:} 2017/05/21

\begin{itemize}
\item
more complete structuring introduced
\item
|\childdocof| introduced
\item
|\childdoc| renamed to |\childdocmain|
\item
|\childredirect| renamed to |\childdocforward| and |\childdocforwardprefix|
and functionality expanded
\end{itemize}

%%%%%%%%%%%%%%%%%%%%%%%%%%%%%%%%%%%%%%%%
\paragraph{v1.0:} 2017/04/27

\begin{itemize}
\item
manual and install package
\item
first version published on CTAN
\end{itemize}

%%%%%%%%%%%%%%%%%%%%%%%%%%%%%%%%%%%%%%%%
\paragraph{v0.6:} 2017/04/26

\begin{itemize}
\item
redirection mechanism added
\end{itemize}

%%%%%%%%%%%%%%%%%%%%%%%%%%%%%%%%%%%%%%%%
\paragraph{v0.5:} 2017/04/26

\begin{itemize}
\item
functionality in definition file
\end{itemize}


%%%%%%%%%%%%%%%%%%%%%%%%%%%%%%%%%%%%%%%%%%%%%%%%%%%%%%%%%%%%%%%%%%%%%%%%%%%%%%%%
%%%%%%%%%%%%%%%%%%%%%%%%%%%%%%%%%%%%%%%%%%%%%%%%%%%%%%%%%%%%%%%%%%%%%%%%%%%%%%%%
%%%%%%%%%%%%%%%%%%%%%%%%%%%%%%%%%%%%%%%%%%%%%%%%%%%%%%%%%%%%%%%%%%%%%%%%%%%%%%%%
\appendix

\settowidth\MacroIndent{\rmfamily\scriptsize 000\ }

 \DocInput{childdoc.dtx}

\end{document}
%</driver>
% \fi
%
% %%%%%%%%%%%%%%%%%%%%%%%%%%%%%%%%%%%%%%%%%%%%%%%%%%%%%%%%%%%%%%%%%%%%%%%%%%%%%%
% %%%%%%%%%%%%%%%%%%%%%%%%%%%%%%%%%%%%%%%%%%%%%%%%%%%%%%%%%%%%%%%%%%%%%%%%%%%%%%
% \section{Sample}
%\iffalse
%<*samplemain>
%\fi
%
% The following presents a sample document
% with two chapters, two parts, a title page,
% a compile flag as well as three forwarding files to set the flag.
% It consists of eight |.tex| files:
% \begin{center}
% \begin{tabular}{ll}
% |cdocsamp.tex|&main file\\
% |cdocsch1.tex|&include file for chapter 1\\
% |cdocsch2.tex|&include file for chapter 2\\
% |cdocspt3.tex|&include file for part 3\\
% |cdocspt4.tex|&include file for part 4\\
% |cdocsdrf.tex|&forwarding file for main file in draft mode\\
% |cdocsfi1.tex|&forwarding file for final version of chapter 1\\
% |cdocsfi2.tex|&forwarding file for final version of chapter 2\\
% \end{tabular}
% \end{center}
% Each of the eight files can be compiled directly by the \LaTeX{} compiler.
%
% %%%%%%%%%%%%%%%%%%%%%%%%%%%%%%%%%%%%%%
% \paragraph{Main File.}
%
% The main file is called |cdocsamp.tex|.
%
% Load the \textsf{childdoc} definitions and
% declare the filename for the main document:
%    \begin{macrocode}
\input{childdoc.def}
\childdocmain{}
%    \end{macrocode}

% Optional override for |\version| flag:
%    \begin{macrocode}
%%\ifchilddoc\else\providecommand{\version}{draft}\fi
%    \end{macrocode}

% Define the default values for the |\version| flag
% (|final| for the main file and |draft| for childs):
%    \begin{macrocode}
\ifchilddoc
\providecommand{\version}{draft}
\else
\providecommand{\version}{final}
\fi
%    \end{macrocode}

% Load the standard document class:
%    \begin{macrocode}
\documentclass[12pt]{article}
%    \end{macrocode}

% Start the document body:
%    \begin{macrocode}
\begin{document}
%    \end{macrocode}

% Declare a title page.
% Print title, part of document being processed and version flag:
%    \begin{macrocode}
\addtocounter{page}{-1}
\begin{center}
{\LARGE\bfseries{}childdoc example\par}
\vspace{1cm}
\ifchilddoc
\ifchilddocmanual part\else chapter\fi:
`\childdocname' of `\childdocjob'\par
\else
main document: `\childdocjob'\par
\fi
version: \version\par
\end{center}
\newpage
%    \end{macrocode}

% Manually include selected file,
% otherwise process as usual:
%    \begin{macrocode}
\ifchilddocmanual
\section*{part `\childdocname'}
\input{\childdocname}
\else
%    \end{macrocode}

% Include the two chapters:
%    \begin{macrocode}
\include{cdocsch1}
\include{cdocsch2}
%    \end{macrocode}

% Include the two parts unless only chapters should be displayed:
%    \begin{macrocode}
\ifchilddoc\else
\section{part three}
\input{cdocspt3}
\section{part four}
\input{cdocspt4}
\fi
%    \end{macrocode}

% Process as usual until here:
%    \begin{macrocode}
\fi
%    \end{macrocode}

% End of document body:
%    \begin{macrocode}
\end{document}
%    \end{macrocode}
%\iffalse
%</samplemain>
%\fi
%
% %%%%%%%%%%%%%%%%%%%%%%%%%%%%%%%%%%%%%%
% \paragraph{Chapter Include Files.}
%
% The include files are called |cdocsch1.tex| and |cdocsch2.tex|.
%
%\iffalse
%<*samplechap1|samplechap2>
%\fi

% Optional override for |\version| flag:
%    \begin{macrocode}
%%\providecommand{\version}{final}
%    \end{macrocode}

% Include the main document:
%    \begin{macrocode}
\input{childdoc.def}
\childdocof{cdocsamp}
%    \end{macrocode}

%\iffalse
%</samplechap1|samplechap2>
%\fi
%
%\iffalse
%<*samplechap1>
%\fi
% Some text for chapter 1:
%    \begin{macrocode}
\section{one}
some text in chapter one
%    \end{macrocode}

%\iffalse
%</samplechap1>
%\fi
% Some text for chapter 2:
%\iffalse
%<*samplechap2>
%\fi
%    \begin{macrocode}
\section{two}
more text in chapter two
%    \end{macrocode}

%\iffalse
%</samplechap2>
%\fi
%
% %%%%%%%%%%%%%%%%%%%%%%%%%%%%%%%%%%%%%%
% \paragraph{Part Include Files.}
%
% The include files are called |cdocspt3.tex| and |cdocspt4.tex|.
%
%\iffalse
%<*samplepart3|samplepart4>
%\fi

% Optional override for |\version| flag:
%    \begin{macrocode}
%%\providecommand{\version}{final}
%    \end{macrocode}

% Include the main document:
%    \begin{macrocode}
\input{childdoc.def}
\childdocby{cdocsamp}
%    \end{macrocode}

%\iffalse
%</samplepart3|samplepart4>
%\fi
%
%\iffalse
%<*samplepart3>
%\fi
% Some text for part 3:
%    \begin{macrocode}
some text in part three
%    \end{macrocode}

%\iffalse
%</samplepart3>
%\fi
% Some text for part 4:
%\iffalse
%<*samplepart4>
%\fi
%    \begin{macrocode}
more text in part four
%    \end{macrocode}

%\iffalse
%</samplepart4>
%\fi
%
% %%%%%%%%%%%%%%%%%%%%%%%%%%%%%%%%%%%%%%
% \paragraph{Forwarding for a Complete Draft.}
%
% The following forwarding file |cdocsdrf.tex|
% compiles the main document in draft mode:
%\iffalse
%<*sampledraft>
%\fi
%    \begin{macrocode}
\def\version{draft}
\input{childdoc.def}
\childdocforward{cdocsamp}
%    \end{macrocode}

%\iffalse
%</sampledraft>
%\fi
%
% %%%%%%%%%%%%%%%%%%%%%%%%%%%%%%%%%%%%%%
% \paragraph{Forwarding for Final Version of the Chapters.}
%
% The following forwarding files |cdocsfn1.tex| and |cdocsfn2.tex|
% (with identical content)
% compile the final versions of the child documents
% |cdocsch1.tex| and |cdocsch2.tex|, respectively:
%\iffalse
%<*samplefinal>
%\fi
%    \begin{macrocode}
\def\version{final}
\input{childdoc.def}
\childdocforwardprefix[cdocsamp]{cdocsfn}{cdocsch}
%    \end{macrocode}

%\iffalse
%</samplefinal>
%\fi
%
% %%%%%%%%%%%%%%%%%%%%%%%%%%%%%%%%%%%%%%
% \paragraph{Command Line Processing.}
%
% The following three command lines generate the output files
% |cdocscld|, |cdocscl1| and |cdocscl2|
% which should be identical to
% |cdocsdrf|, |cdocsch1| and |cdocsfn2|, respectively:
% \begin{center}
% \begin{tabular}{l}
% |latex -jobname cdocscld \|\\
% |  "\def\version{draft}\input{childdoc.def}\childdocforward{cdocsamp}"|\\
% |latex -jobname cdocscl1 \|\\
% |  "\input{childdoc.def}\childdocforward[cdocsamp]{cdocsch1}"|\\
% |latex -jobname cdocscl2 \|\\
% |  "\def\version{final}\input{childdoc.def}\childdocforward{cdocsch2}"|
% \end{tabular}
% \end{center}
% Note that the trailing backslash on each first line
% merely continues the input to the second line
% (for convenient cut ant paste).
% Furthermore, the command |latex| can be replaced by any
% of its alternative versions such as |pdflatex|.
%
% %%%%%%%%%%%%%%%%%%%%%%%%%%%%%%%%%%%%%%%%%%%%%%%%%%%%%%%%%%%%%%%%%%%%%%%%%%%%%%
% %%%%%%%%%%%%%%%%%%%%%%%%%%%%%%%%%%%%%%%%%%%%%%%%%%%%%%%%%%%%%%%%%%%%%%%%%%%%%%
% \section{Implementation}
%\iffalse
%<*package>
%\fi
%
% This section describes the definitions file |childdoc.def|.

% The definitions cannot be loaded using |\usepackage| or |\RequirePackage|
% which has a mechanism to prevent loading a style file more than once.
% When loading the definitions by means of |\input|
% multiple instances have to be prevented manually:
%\iffalse
%This code needs to be before the `\ProvidesFile' directive
%which is defined at the beginning of this file.
%Therefore it is also placed there and commented out here.
%</package>
%<*discard>
%\fi
%    \begin{macrocode}
\ifdefined\childdocmain\endinput\fi
%    \end{macrocode}
%\iffalse
%</discard>
%<*package>
%\fi
%
% \macro{\ifchilddoc}
% \macro{\ifchilddocmanual}
% The conditional |\ifchilddoc| tells whether a
% child (true) or main (false) document is being compiled.
% The conditional |\ifchilddocmanual| tells whether
% the |\includeonly| mechanism is used (false) or
% the selection of child files must be performed manually (true).
% The definitions initialise to false:
%    \begin{macrocode}
\newif\ifchilddoc
\newif\ifchilddocmanual
%    \end{macrocode}

% \macro{\childdocname}
% \macro{\childdocjob}
% The macro |\childdocname| stores the name of the main document
% to be compiled. The macro |\childdocjob| stores the name of
% the document on which the \LaTeX{} compiler was originally invoked.
% The content of |\jobname| cannot be compared
% to filenames specified in the source due to different catcodes.
% The following code rescans |\jobname|, stores the result
% in |\childdocname| and saves a copy in |\childdocjob|:
%    \begin{macrocode}
\edef\childdocname{\scantokens\expandafter{\jobname\noexpand}}
\let\childdocjob\childdocname
%    \end{macrocode}

% \macro{\childdocdisable}
% The macro |\childdocdisable| prevents the main file
% from being processed more than once.
% At this stage, the main document command |\childdocmain|
% is assumed to be called once again where it should do nothing.
% Any subsequent call to it should prevent
% a secondary processing of the main document
% It overwrites the forwarding commands
% |\childdocof| and |\childdocforward|
% with empty macros to prevent further inclusions of the main document:
%    \begin{macrocode}
\newcommand{\childdocdisable}
{
  \renewcommand{\childdocmain}[1]{\renewcommand{\childdocmain}[1]{\endinput}}
  \renewcommand{\childdocof}[1]{}
  \renewcommand{\childdocby}[2][]{}
  \renewcommand{\childdocforward}[2][]{}
  \renewcommand{\childdocdisable}{}
}
%    \end{macrocode}

% \macro{\childdocmain}
% The macro |\childdocmain| is to be called at the top of the main file
% with nothing or the main filename (without extension) as argument.
% First, it breaks loops.
% If the argument is not empty and does not match |\childdocname|
% (which is set by the first inclusion of |childdoc.def|),
% |\ifchilddoc| is set to true, |\includeonly| is applied to the child file
% and |\jobname| is set to the main file
% (for proper handling of |.aux| files):
%    \begin{macrocode}
\newcommand{\childdocmain}[1]
{
  \childdocdisable\childdocmain{}
  \if?#1?\else
    \begingroup
      \def\childdoctmp{#1}
      \ifx\childdoctmp\childdocname
        \def\childdoctmp{}
      \else
        \def\childdoctmp
        {
          \childdoctrue
          \includeonly{\childdocname}
          \def\childdocjob{#1}
          \def\jobname{#1}
        }
      \fi
      \expandafter
    \endgroup
    \childdoctmp
  \fi
}
%    \end{macrocode}

% \macro{\childdocof}
% The command |\childdocof| redirects
% compilation to the main file |#1|.
%    \begin{macrocode}
\newcommand{\childdocof}[1]
{
  \childdocdisable
  \childdoctrue
  \includeonly{\childdocname}
  \def\jobname{#1}
  \def\childdocjob{#1}
  \input{#1}
}
%    \end{macrocode}

% \macro{\childdocby}
% The command |\childdocby| ....
%    \begin{macrocode}
\newcommand{\childdocby}[2][]
{
  \childdocdisable
  \childdoctrue
  \childdocmanualtrue
  \if?#1?\else
    \def\jobname{#2}
  \fi
  \def\childdocjob{#2}
  \input{#2}
  \endinput
}
%    \end{macrocode}

% \macro{\childdocforward}
% The command |\childdocforward| redirects
% compilation to the main file or
% (if the optional argument is given) a child file.
% Parameters are set as if the main file
% or a child file starting with |\childdocof| was compiled.
% Then compilation is handed over to the main file:
%    \begin{macrocode}
\newcommand{\childdocforward}[2][]
{
  \begingroup
    \if?#1?
      \def\childdoctmp
      {
        \def\childdocname{#2}
        \def\childdocjob{#2}
        \def\jobname{#2}
        \input{#2}
        \endinput
      }
    \else
      \def\childdoctmp
      {
        \childdocdisable
        \def\childdocname{#2}
        \childdoctrue
        \includeonly{#2}
        \def\childdocjob{#1}
        \def\jobname{#1}
        \input{#1}
        \endinput
      }
    \fi
    \expandafter
  \endgroup
  \childdoctmp
}
%    \end{macrocode}

% \macro{\childdocforwardprefix}
% The command |\childdocforwardprefix| redirects
% compilation to the main or a child file by means of a pattern.
% The prefix |#1| in the current filename is replaced by |#2|
% and the suffix of the current filename is kept
% (it is assumed that the filename does not contain the substring `|~~~|'
% which is used as a delimiter).
% Compilation is handed over to the new file by |\childdocforward|:
%    \begin{macrocode}
\newcommand{\childdocforwardprefix}[3][]
{
  \begingroup
    \def\childdocextract #2##1~~~{\def\childdoctmp{\childdocforward[#1]{#3##1}}}
    \expandafter\childdocextract\childdocname~~~
    \expandafter
  \endgroup
  \childdoctmp
}
%    \end{macrocode}

% \macro{\childdoc}
% The deprecated macro |\childdoc| is a legacy version of |\childdocmain|:
%    \begin{macrocode}
\newcommand{\childdoc}{\childdocmain}
%    \end{macrocode}

% \macro{\childdocredirect}
% The deprecated macro |\childdocredirect| is a legacy version
% of |\childdocforward| and |\childdocforwardprefix|:
%    \begin{macrocode}
\newcommand{\childdocredirect}[2][]
{
  \begingroup
    \if?#1?
      \def\childdoctmp{\childdocforward{#2}}
    \else
      \def\childdoctmp{\childdocforwardprefix{#1}{#2}}
    \fi
    \expandafter
  \endgroup
  \childdoctmp
}
%    \end{macrocode}

%\iffalse
%</package>
%\fi
%
\endinput
|
and perform the replacements as outlined below.
Instead of |\childdocmain{|\textit{main}|}| add the following code
to the top of the main file:
%
\begin{center}
\begin{tabular}{l}
|\||ifdefined\childdocname\endinput\||fi\newif\ifchilddoc|\\
|\edef\childdocname{\scantokens\expandafter{\jobname\noexpand}}|\\
|\def\childdocmain{|\textit{main}|}\||ifx\childdocmain\childdocname\||else|\\
|\childdoctrue\includeonly{\childdocname}\let\jobname\childdocmain\||fi|\\
\end{tabular}
\end{center}
%
Instead of |\childdocof{|\textit{main}|}| just include the main file
at the top of each child file:
%
\begin{center}
|\input{|\textit{main}|}|
\end{center}
%
A simple redirection |\childdocforward{|\textit{dest}|}| is achieved by:
%
\begin{center}
|\def\jobname{|\textit{dest}|}\input{\jobname}|
\end{center}
%
The redirection with prefix
|\childdocforwardprefix[|\textit{prefix}|]{|\textit{dest}|}|
is accomplished by:
%
\begin{center}
\begin{tabular}{l}
|{\edef\jobname{\scantokens\expandafter{\jobname\noexpand}}|\\
|\def\redirectjob |\textit{prefix}|#1~~~{\gdef\jobname{|\textit{dest}|#1}}|\\
|\expandafter\redirectjob\jobname~~~}\input{\jobname}|
\end{tabular}
\end{center}

In an alternative approach,
child documents can be compiled by a specific command line
without additional code or specific definitions:
%
\begin{center}
|... -jobname "|\textit{target}|" "|[\textit{flags}]%
|\includeonly{|\textit{dest}|}\input{|\textit{main}|}"|
\end{center}
%

%%%%%%%%%%%%%%%%%%%%%%%%%%%%%%%%%%%%%%%%%%%%%%%%%%%%%%%%%%%%%%%%%%%%%%%%%%%%%%%%
%%%%%%%%%%%%%%%%%%%%%%%%%%%%%%%%%%%%%%%%%%%%%%%%%%%%%%%%%%%%%%%%%%%%%%%%%%%%%%%%
\section{Information}

%%%%%%%%%%%%%%%%%%%%%%%%%%%%%%%%%%%%%%%%%%%%%%%%%%%%%%%%%%%%%%%%%%%%%%%%%%%%%%%%
\subsection{Copyright}

Copyright \copyright{} 2017--2018 Niklas Beisert

This work may be distributed and/or modified under the
conditions of the \LaTeX{} Project Public License, either version 1.3
of this license or (at your option) any later version.
The latest version of this license is in
  \url{http://www.latex-project.org/lppl.txt}
and version 1.3 or later is part of all distributions of \LaTeX{}
version 2005/12/01 or later.

This work has the LPPL maintenance status `maintained'.

The Current Maintainer of this work is Niklas Beisert.

This work consists of the files |README.txt|, |childdoc.ins| and |childdoc.dtx|
as well as the derived files |childdoc.def|, |cdocsamp.tex|
with |cdocsch1.tex|, |cdocsch2.tex|, |cdocspt3.tex|, |cdocspt4.tex|,
|cdocsdrf.tex|, |cdocsfn1.tex|, |cdocsfn2.tex|
as well as |childdoc.pdf|.

%%%%%%%%%%%%%%%%%%%%%%%%%%%%%%%%%%%%%%%%%%%%%%%%%%%%%%%%%%%%%%%%%%%%%%%%%%%%%%%%
\subsection{Files and Installation}

The package consists of the files:
%
\begin{center}
\begin{tabular}{ll}
    |README.txt|   & readme file \\
    |childdoc.ins| & installation file \\
    |childdoc.dtx| & source file \\
    |childdoc.def| & definition file \\
    |cdocsamp.tex| & sample main file \\
    |cdocsch1.tex| & sample include file \\
    |cdocsch2.tex| & sample include file \\
    |cdocspt3.tex| & sample part file \\
    |cdocspt4.tex| & sample part file \\
    |cdocsdrf.tex| & sample redirection file \\
    |cdocsfn1.tex| & sample redirection file \\
    |cdocsfn2.tex| & sample redirection file \\
    |childdoc.pdf| & manual
\end{tabular}
\end{center}
%
The distribution consists of the files
|README.txt|, |childdoc.ins| and |childdoc.dtx|.
%
\begin{itemize}
\item
Run (pdf)\LaTeX{} on |childdoc.dtx|
to compile the manual |childdoc.pdf| (this file).
\item
Run \LaTeX{} on |childdoc.ins| to create the definitions file |childdoc.def|
and the sample |cdocsamp.tex| with include files
|cdocsch1.tex|, |cdocsch2.tex|, |cdocspt3.tex|, |cdocspt4.tex|,
|cdocsdrf.tex|, |cdocsfn1.tex|, |cdocsfn2.tex|.
Then copy the file |childdoc.def| to an appropriate directory of your \LaTeX{}
distribution, e.g.\ \textit{texmf-root}|/tex/latex/childdoc|.
\end{itemize}

%%%%%%%%%%%%%%%%%%%%%%%%%%%%%%%%%%%%%%%%%%%%%%%%%%%%%%%%%%%%%%%%%%%%%%%%%%%%%%%%
\subsection{Related CTAN Packages}

There are several other packages which offer a similar functionality:
%
\begin{itemize}
\item
The packages
\href{http://ctan.org/pkg/docmute}{\textsf{docmute}},
\href{http://ctan.org/pkg/includex}{\textsf{includex}} and
\href{http://ctan.org/pkg/standalone}{\textsf{standalone}}
provide commands to include only the document body of
a child file thus allowing both files to be compiled individually.
\item
The packages \href{http://ctan.org/pkg/subdocs}{\textsf{subdocs}}
and \href{http://ctan.org/pkg/subfiles}{\textsf{subfiles}}
provide structures in which the main and child documents can be
encapsulated and allowing them to be compiled individually.
The inclusion mechanism is different from the conventional |\include|.
\item
The package \href{http://ctan.org/pkg/combine}{\textsf{combine}}
is an elaborate solution to combine several documents into one.
\end{itemize}
%
See also the CTAN topic \href{http://ctan.org/topic/subdocs}{\textsf{subdocs}}
for further related packages.
The present package differs from the above solutions in that
a document structure constructed with the conventional |\include| mechanism
just needs two extra commands at the top of every file
such that all constituent files can be compiled individually.

%%%%%%%%%%%%%%%%%%%%%%%%%%%%%%%%%%%%%%%%%%%%%%%%%%%%%%%%%%%%%%%%%%%%%%%%%%%%%%%%
%\subsection{Feature Suggestions}
%
%The following is a list of features which may be useful for future
%versions of this package:
%%
%\begin{itemize}
%\item
%\ldots
%\end{itemize}

%%%%%%%%%%%%%%%%%%%%%%%%%%%%%%%%%%%%%%%%%%%%%%%%%%%%%%%%%%%%%%%%%%%%%%%%%%%%%%%%
\subsection{Revision History}

%%%%%%%%%%%%%%%%%%%%%%%%%%%%%%%%%%%%%%%%
\paragraph{v2.0:} 2018/12/30

\begin{itemize}
\item
immediate forward processing
\item
added |\childdocby| mechanism
\item
manual restructured
\end{itemize}

%%%%%%%%%%%%%%%%%%%%%%%%%%%%%%%%%%%%%%%%
\paragraph{v1.6:} 2018/01/17

\begin{itemize}
\item
application for development of include files
\item
corrections to manual
\end{itemize}

%%%%%%%%%%%%%%%%%%%%%%%%%%%%%%%%%%%%%%%%
\paragraph{v1.5:} 2017/05/21

\begin{itemize}
\item
more complete structuring introduced
\item
|\childdocof| introduced
\item
|\childdoc| renamed to |\childdocmain|
\item
|\childredirect| renamed to |\childdocforward| and |\childdocforwardprefix|
and functionality expanded
\end{itemize}

%%%%%%%%%%%%%%%%%%%%%%%%%%%%%%%%%%%%%%%%
\paragraph{v1.0:} 2017/04/27

\begin{itemize}
\item
manual and install package
\item
first version published on CTAN
\end{itemize}

%%%%%%%%%%%%%%%%%%%%%%%%%%%%%%%%%%%%%%%%
\paragraph{v0.6:} 2017/04/26

\begin{itemize}
\item
redirection mechanism added
\end{itemize}

%%%%%%%%%%%%%%%%%%%%%%%%%%%%%%%%%%%%%%%%
\paragraph{v0.5:} 2017/04/26

\begin{itemize}
\item
functionality in definition file
\end{itemize}


%%%%%%%%%%%%%%%%%%%%%%%%%%%%%%%%%%%%%%%%%%%%%%%%%%%%%%%%%%%%%%%%%%%%%%%%%%%%%%%%
%%%%%%%%%%%%%%%%%%%%%%%%%%%%%%%%%%%%%%%%%%%%%%%%%%%%%%%%%%%%%%%%%%%%%%%%%%%%%%%%
%%%%%%%%%%%%%%%%%%%%%%%%%%%%%%%%%%%%%%%%%%%%%%%%%%%%%%%%%%%%%%%%%%%%%%%%%%%%%%%%
\appendix

\settowidth\MacroIndent{\rmfamily\scriptsize 000\ }

 \DocInput{childdoc.dtx}

\end{document}
%</driver>
% \fi
%
% %%%%%%%%%%%%%%%%%%%%%%%%%%%%%%%%%%%%%%%%%%%%%%%%%%%%%%%%%%%%%%%%%%%%%%%%%%%%%%
% %%%%%%%%%%%%%%%%%%%%%%%%%%%%%%%%%%%%%%%%%%%%%%%%%%%%%%%%%%%%%%%%%%%%%%%%%%%%%%
% \section{Sample}
%\iffalse
%<*samplemain>
%\fi
%
% The following presents a sample document
% with two chapters, two parts, a title page,
% a compile flag as well as three forwarding files to set the flag.
% It consists of eight |.tex| files:
% \begin{center}
% \begin{tabular}{ll}
% |cdocsamp.tex|&main file\\
% |cdocsch1.tex|&include file for chapter 1\\
% |cdocsch2.tex|&include file for chapter 2\\
% |cdocspt3.tex|&include file for part 3\\
% |cdocspt4.tex|&include file for part 4\\
% |cdocsdrf.tex|&forwarding file for main file in draft mode\\
% |cdocsfi1.tex|&forwarding file for final version of chapter 1\\
% |cdocsfi2.tex|&forwarding file for final version of chapter 2\\
% \end{tabular}
% \end{center}
% Each of the eight files can be compiled directly by the \LaTeX{} compiler.
%
% %%%%%%%%%%%%%%%%%%%%%%%%%%%%%%%%%%%%%%
% \paragraph{Main File.}
%
% The main file is called |cdocsamp.tex|.
%
% Load the \textsf{childdoc} definitions and
% declare the filename for the main document:
%    \begin{macrocode}
% \iffalse
%
% childdoc.dtx Copyright (C) 2017-2018 Niklas Beisert
%
% This work may be distributed and/or modified under the
% conditions of the LaTeX Project Public License, either version 1.3
% of this license or (at your option) any later version.
% The latest version of this license is in
%   http://www.latex-project.org/lppl.txt
% and version 1.3 or later is part of all distributions of LaTeX
% version 2005/12/01 or later.
%
% This work has the LPPL maintenance status `maintained'.
%
% The Current Maintainer of this work is Niklas Beisert.
%
% This work consists of the files childdoc.dtx and childdoc.ins
% and the derived files childdoc.def and cdocsamp.tex with
% cdocsch1.tex, cdocsch2.tex, cdocsdrf.tex, cdocsfn1.tex, cdocsfn2.tex.
%
%<package>\ifdefined\childdocmain\endinput\fi
%<package>\ProvidesFile{childdoc.def}[2018/12/30 v2.0 child document driver]
%<samplemain>\ProvidesFile{cdocsamp.tex}[2018/12/30 v2.0 sample for childdoc]
%<*driver>
%\ProvidesFile{childdoc.drv}[2018/12/30 v2.0 childdoc reference manual file]
\PassOptionsToClass{10pt,a4paper}{article}
\documentclass{ltxdoc}

\usepackage[margin=35mm]{geometry}
\usepackage{hyperref}
\usepackage{hyperxmp}
\usepackage[usenames]{color}

\hypersetup{colorlinks=true}
\hypersetup{pdfstartview=FitH}
\hypersetup{pdfpagemode=UseNone}
\hypersetup{pdfsource={}}
\hypersetup{pdflang={en-UK}}
\hypersetup{pdfcopyright={Copyright 2017-2018 Niklas Beisert.
  This work may be distributed and/or modified under the
  conditions of the LaTeX Project Public License, either version 1.3
  of this license or (at your option) any later version.}}
\hypersetup{pdflicenseurl={http://www.latex-project.org/lppl.txt}}
\hypersetup{pdfcontactaddress={ETH Zurich, ITP, HIT K,
  Wolfgang-Pauli-Strasse 27}}
\hypersetup{pdfcontactpostcode={8093}}
\hypersetup{pdfcontactcity={Zurich}}
\hypersetup{pdfcontactcountry={Switzerland}}
\hypersetup{pdfcontactemail={nbeisert@itp.phys.ethz.ch}}
\hypersetup{pdfcontacturl={http://people.phys.ethz.ch/\xmptilde nbeisert/}}

\newcommand{\secref}[1]{\hyperref[#1]{section \ref*{#1}}}

\parskip1ex
\parindent0pt
\let\olditemize\itemize
\def\itemize{\olditemize\parskip0pt}

\begin{document}

\title{The \textsf{childdoc} Package}
\hypersetup{pdftitle={The childdoc Package}}
\author{Niklas Beisert\\[2ex]
  Institut f\"ur Theoretische Physik\\
  Eidgen\"ossische Technische Hochschule Z\"urich\\
  Wolfgang-Pauli-Strasse 27, 8093 Z\"urich, Switzerland\\[1ex]
  \href{mailto:nbeisert@itp.phys.ethz.ch}
  {\texttt{nbeisert@itp.phys.ethz.ch}}}
\hypersetup{pdfauthor={Niklas Beisert}}
\hypersetup{pdfsubject={Manual for the LaTeX2e Package childdoc}}
\date{30 December 2018, \textsf{v2.0}}
\maketitle

\begin{abstract}\noindent
\textsf{childdoc} is a \LaTeXe{} package
that enables the direct compilation
of document sections included by |\include|
to individual files.
\end{abstract}

\begingroup
\parskip0ex
\tableofcontents
\endgroup

%%%%%%%%%%%%%%%%%%%%%%%%%%%%%%%%%%%%%%%%%%%%%%%%%%%%%%%%%%%%%%%%%%%%%%%%%%%%%%%%
%%%%%%%%%%%%%%%%%%%%%%%%%%%%%%%%%%%%%%%%%%%%%%%%%%%%%%%%%%%%%%%%%%%%%%%%%%%%%%%%
\section{Introduction}

\LaTeX{} provides a mechanism to structure a large document (such as a book)
into a main file and several child files (containing the chapters)
using the |\include| command.
This mechanism is beneficial for documents
which span hundreds of pages in order to
make the source file(s) more manageable.
Moreover, compilation can be restricted to
selected child files by means of the |\includeonly| command.
The latter feature can be used to reduce the compilation time while editing
(this was significantly more useful in the earlier days of \LaTeX{})
or to generate a smaller document which is easier to navigate.
Another application of |\includeonly| is to generate
documents consisting of selected parts of the complete document.

However, there are a few drawbacks of the plain |\include| mechanism:
\begin{itemize}
\item
The child files cannot be compiled on their own,
they can only be compiled via the main file.
A naive editing environment
(such as a text editor with an option
to have the current file processed by \LaTeX)
may require one to switch to the main file before compiling;
attempting to compile the child file produces errors.
\item
The main file must be modified (each time)
to adjust the |\includeonly| command
to the present needs. This easily leaves the main file in a messy state.
\item
The generated document will always carry the filename
of the main document. This is inconvenient if
several child files are to be compiled and
to be kept for distribution.
\end{itemize}

The present package provides a simple interface
to make child files individually compilable by \LaTeX{}.
Compiling a child file then has the same effect as compiling
the main file with an |\includeonly| command
to select the appropriate child.
Moreover the generated document will carry the name of the child
rather than the main file.
This resolves all three above issues.

This feature is meant to make the editing of books,
thesis documents and lecture notes somewhat more convenient.
However, the package can also be used efficiently for
composing a series of documents (such as exercise sheets)
which are typically distributed individually.
It then assists the author in generating the individual documents
(potentially in different versions)
as well as a document containing the collected series.
Another application is in developing style files
or other kinds of included material
where compilation of the style file could redirect
to a sample or test file.

%%%%%%%%%%%%%%%%%%%%%%%%%%%%%%%%%%%%%%%%%%%%%%%%%%%%%%%%%%%%%%%%%%%%%%%%%%%%%%%%
%%%%%%%%%%%%%%%%%%%%%%%%%%%%%%%%%%%%%%%%%%%%%%%%%%%%%%%%%%%%%%%%%%%%%%%%%%%%%%%%
\section{Usage}

First of all, the package \textsf{childdoc} is \emph{not} a standard
\LaTeXe{} |.sty| style file! Therefore it needs to be invoked in
a non-standard way.

%%%%%%%%%%%%%%%%%%%%%%%%%%%%%%%%%%%%%%%%%%%%%%%%%%%%%%%%%%%%%%%%%%%%%%%%%%%%%%%%
\subsection{Included Files}
\label{sec:include}

%%%%%%%%%%%%%%%%%%%%%%%%%%%%%%%%%%%%%%%%
\DescribeMacro{\childdocmain}
To use the package, add the commands
\begin{center}
\begin{tabular}{l}
|\input{childdoc.def}|\\
|\childdocmain{}|\\
\end{tabular}
\end{center}
at the very top of the main \LaTeX{} file,
in particular \emph{before} the |\documentclass| statement!
The argument of |\childdocmain| should be left empty
(but it must be present).

%%%%%%%%%%%%%%%%%%%%%%%%%%%%%%%%%%%%%%%%
\DescribeMacro{\childdocof}
Furthermore, add the commands
\begin{center}
\begin{tabular}{l}
|\input{childdoc.def}|\\
|\childdocof{|\textit{main}|}|\\
\end{tabular}
\end{center}
at the top of every child file \textit{child}
which is included by |\include{|\textit{child}|}|
from within the main file
(or at least for those files to be compiled individually).
The argument \textit{main} must be the filename of the main file.

There are a couple of
considerations in setting up the main and child documents:

%%%%%%%%%%%%%%%%%%%%%%%%%%%%%%%%%%%%%%%%
\paragraph{Restrictions.}

Please note the following restrictions:
\begin{itemize}
\item
|\childdocmain| must be called with one argument \textit{main}
to ensure compatibility with earlier version of the package.
It must either be empty (|\childdocmain{}|)
or precisely match the filename of the main file in which it is specified.
See \secref{sec:detection} for further information.
\item
The filename \textit{main} must be specified without the |.tex| extension.
\item
The filename \textit{main} is case sensitive
(even in case-insensitive file systems)
due to internal string comparison.
\item
The argument \textit{main} should be fully expanded, it cannot be a macro.
\item
Subdirectories and special characters should be avoided in filenames.
\item
The command |\childdocmain{|\textit{main}|}| must be followed by a whitespace.
It should not be followed immediately by another command
or by a comment mark `|%|'.
This is because the \TeX{} parser reads the token immediately following
the argument of |\childdocmain| and puts it
at the beginning of every child section;
however, a white\-space is ignored.
\end{itemize}

%%%%%%%%%%%%%%%%%%%%%%%%%%%%%%%%%%%%%%%%
\paragraph{Content of Main File.}

It is advisable to place all content in the child files included by |\include|.
Any output contained in the main file will appear in all child documents
unless suppressed manually;
it cannot be suppressed automatically by the |\includeonly| directive
and thus should normally be avoided.
A method to include some content in the main file
by means of conditional processing is described in \secref{sec:conditional}.

%%%%%%%%%%%%%%%%%%%%%%%%%%%%%%%%%%%%%%%%
\paragraph{Page Numbering.}

When only a part of the document is compiled,
the appropriate numbering of pages
(as well as other status parameters)
is determined from the |.aux| files.
The latter contain information from previous passes.
However this information needs to propagate through
all intermediate child documents.
Therefore the page numbering in child documents may well
be inconsistent until the complete document is compiled at least once.

A useful (if unconventional) way to always ensure a consistent
page numbering is to restart the numbering in each child document
and denote the pages by `\textit{child}|.|\textit{page}'
where \textit{child} represents the chapter/section number of the child file.
This can be achieved by the command
|\numberwithin{page}{|\textit{child}|}|
of the \textsf{amsmath} package
where \textit{child} can be |chapter| or |section|
depending on the chosen structuring.
Alternatively, one can modify the macro |\thepage| appropriately
and reset the counter |page| at the start of each child file.

%%%%%%%%%%%%%%%%%%%%%%%%%%%%%%%%%%%%%%%%%%%%%%%%%%%%%%%%%%%%%%%%%%%%%%%%%%%%%%%%
\subsection{Conditional Processing}
\label{sec:conditional}

The package provides a mechanism to compile different versions
of a document. To customise the versions further some conditional processing
can come in handy to distinguish which version is being compiled.
The package provides two macros to describe the compilation context:

%%%%%%%%%%%%%%%%%%%%%%%%%%%%%%%%%%%%%%%%
\DescribeMacro{\ifchilddoc}
The conditional |\ifchilddoc| distinguishes between the compilation of
child documents and the main document:
%
\begin{center}
|\ifchilddoc |\textit{child-code}| |[|\||else |\textit{main-code}]| \||fi|
\end{center}

%%%%%%%%%%%%%%%%%%%%%%%%%%%%%%%%%%%%%%%%
\DescribeMacro{\childdocname}
\DescribeMacro{\childdocjob}
The macro |\childdocname| contains the filename (without extension)
of the main or child file being processed.
Note that |\childdocjob| will always contain the name of the main file.

%%%%%%%%%%%%%%%%%%%%%%%%%%%%%%%%%%%%%%%%
\paragraph{Title Page.}

Conditional processing can be used to include a title or banner page
in the main document when proper precautions are taken.
Importantly, the code in the main file should ensure that the page counter
(as well as other status parameters which are stored in the |.aux| files)
takes the same value after the conditional processing.
Otherwise the page numbers may take divergent values
depending on which part is compiled.

For example, a title page could be declared by:
%
\begin{center}
\begin{tabular}{l}
|\ifchilddoc\||else|\\
|\addtocounter{page}{-1}|\\
\textit{code for title page}\\
|\newpage|\\
|\||fi|
\end{tabular}
\end{center}
%
A banner page for the child documents can be generated by:
%
\begin{center}
\begin{tabular}{l}
|\ifchilddoc|\\
|\addtocounter{page}{-1}|\\
\textit{code for banner page}\\
|\newpage|\\
|\||fi|
\end{tabular}
\end{center}
%
Here one could write a message such as:
\begin{center}
|This is the part \childdocname{} of \childdocjob{}.|
\end{center}

%%%%%%%%%%%%%%%%%%%%%%%%%%%%%%%%%%%%%%%%%%%%%%%%%%%%%%%%%%%%%%%%%%%%%%%%%%%%%%%%
\subsection{Flags}
\label{sec:flags}

The package makes it easy to generate different versions
of the main or child documents.
To this end compilation flags can be defined
and assigned different default values.
They will be particularly useful in conjunction
with the forwarding mechanism described in \secref{sec:forward}.

For example, it may be useful to have a flag |\version|
which can be set to |draft| or |final|.
The document source will contain some conditional code
depending on the value of |\version|.
Suppose further, the flag should default to |final| for the main file
and to |draft| for child files
which is a natural assignment for editing the document.
This is achieved by placing the following code
in the preamble of the main document
(below the |\childdocmain| directive):
%
\begin{center}
\begin{tabular}{l}
|\ifchilddoc|\\
|\providecommand{\version}{draft}|\\
|\||else|\\
|\providecommand{\version}{final}|\\
|\||fi|
\end{tabular}
\end{center}
%
The definition by |\providecommand| makes sure
that previous definitions are not overwritten.
Further statements |\providecommand{\version}{...}|
can thus be added before the above code to override it.

For the main file, one might add a line
(between |\childdocmain| and the above block)
%
\begin{center}
|%\ifchilddoc\||else\providecommand{\version}{draft}\||fi|
\end{center}
%
which can be uncommented to produce a draft version.
Likewise one can add a line to the very top of a child file
(above the |\childdocof{|\textit{main}|}| directive)
%
\begin{center}
|%\providecommand{\version}{final}|
\end{center}
%
which can be uncommented to produce the final version of this child document.

%%%%%%%%%%%%%%%%%%%%%%%%%%%%%%%%%%%%%%%%%%%%%%%%%%%%%%%%%%%%%%%%%%%%%%%%%%%%%%%%
\subsection{Forwarding}
\label{sec:forward}

Different versions of the main or child documents
using compilation flags as described in \secref{sec:flags}
can be (permanently) stored in different files
for convenient compilation, viewing and distribution.
To this end, the package defines a command
to pass on compilation to a different file:

%%%%%%%%%%%%%%%%%%%%%%%%%%%%%%%%%%%%%%%%
\DescribeMacro{\childdocforward}
The command |\childdocforward| redirects processing to
another source file:
%
\begin{center}
\begin{tabular}{l}
|\input{childdoc.def}|\\
|\childdocforward[|\textit{main}|]{|\textit{dest}|}|\\
\end{tabular}
\end{center}
%
The argument \textit{dest} is the destination file
(without extension).
It should be the main file or one of the child files.
Note that further \textsf{childdoc} directives
such as |\childdocof| and |\childdocforward|
in the indicated file will be processed in this form.
The optional argument \textit{main}
passes on directly to the main file \textit{main}
while pretending to compile the child \textit{dest}.
This form behaves as if \textit{dest}
issues |\childdocof{|\textit{main}|}| right away,
and no further \textsf{childdoc} directives will be processed.

%%%%%%%%%%%%%%%%%%%%%%%%%%%%%%%%%%%%%%%%
\DescribeMacro{\...prefix}
In the alternative form |\childdocforwardprefix|,
%
\begin{center}
\begin{tabular}{l}
|\input{childdoc.def}|\\
|\childdocforwardprefix[|\textit{main}|]{|\textit{prefix}|}{|\textit{dest}|}|
\end{tabular}
\end{center}
%
the destination file is determined by a pattern
depending on the current file:
To make this work, the current file must be called
`{\textit{prefix}\hspace{0.2em}\textit{suffix}}'
with \textit{prefix} matching precisely the argument.
Processing is then passed on to the file
`{\textit{dest}\hspace{0.2em}\textit{suffix}}'.
Surely, the same effect is achieved by
directly specifying the
argument `{\textit{dest}\hspace{0.2em}\textit{suffix}}'
in the first form.
However, that requires to set up a different file
for each child. With the alternative form of the command
all these files can have exactly the same content
which simplifies setting them up and maintaining them.

For example, the following file |draft.tex|
with a compilation flag |\version| as described in \secref{sec:flags}
compiles the main document as a draft:
%
\begin{center}
\begin{tabular}{l}
|\def\version{draft}|\\
|\input{childdoc.def}|\\
|\childdocforward{|\textit{main}|}|
\end{tabular}
\end{center}
%
Likewise, the following files |final|\textit{nn}|.tex|
compile the final version of the child document
|child|\textit{nn}|.tex|:
%
\begin{center}
\begin{tabular}{l}
|\def\version{final}|\\
|\input{childdoc.def}|\\
|\childdocforwardprefix{final}{child}|
\end{tabular}
\end{center}
%

Note that when several versions of a main file and/or of each child file
are to be generated, it may be convenient to set up a |Makefile| or
shell script to automatise the process.

%%%%%%%%%%%%%%%%%%%%%%%%%%%%%%%%%%%%%%%%%%%%%%%%%%%%%%%%%%%%%%%%%%%%%%%%%%%%%%%%
\subsection{Command Line Processing}
\label{sec:commandline}

The effect of redirection files can also be achieved by invoking
the \LaTeX{} compiler with a more elaborate command line.
Most conveniently this should be done as part
of a shell script or a |Makefile|.

When using \textsf{childdoc} in the main file, the following
command lines effectively perform a redirection
(note that depending on the shell being used,
backslashes may have to be doubled: `|\|' $\to$ `|\\|'):
%
\begin{center}
|... -jobname "|\textit{target}|" |\\|"|[\textit{flags}]%
|\input{childdoc.def}\childdocforward[|\textit{main}|]{|\textit{dest}|}"|
\end{center}
%
Here \textit{target} is the name of the output file,
\textit{main} is the name of the main file
and \textit{dest} is the name of the main or child file to be processed
(all filenames without extensions).
The optional argument \textit{main} can be omitted
if \textit{main} matches \textit{dest}.
Optionally, compilation \textit{flags} can be defined via |\def| commands.
This command line makes the \TeX{} engine believe
it is compiling the file \textit{target}
whose content is specified as the latter parameter.
The provided code then forwards the processing to
\textit{main} or \textit{dest} as described in \secref{sec:forward}.

%%%%%%%%%%%%%%%%%%%%%%%%%%%%%%%%%%%%%%%%%%%%%%%%%%%%%%%%%%%%%%%%%%%%%%%%%%%%%%%%
\subsection{Include by Input}
\label{sec:input}

Including child documents by |\include| has some restrictions by design.
Most notably, the content of a child document always occupies
its own set of pages; pages cannot be shared between child documents.
Usually, this behaviour makes perfect sense
because each child document contain an essential part of the document.
However, in some situations it may be desirable to compose
a document from a collection of parts
without having mandatory page breaks between then.
For this case, the package
provides a mechanism to include parts
by |\input| which can also be processed individually.
However, by construction this mechanism
requires manual handling of the content to be output.

%%%%%%%%%%%%%%%%%%%%%%%%%%%%%%%%%%%%%%%%
\DescribeMacro{\ifchilddocmanual}
The main file should be prepared as usual, see \secref{sec:include}.
However, the document body must make a distinction
between processing of an individual part and of the main document, e.g.:
%
\begin{center}
\begin{tabular}{l}
|\ifchilddocmanual|\\
|\input{\childdocname}|\\
|\||else|\\
\textit{document body with }|\input{|\textit{part}|}|\\
|\||fi|
\end{tabular}
\end{center}
%
The conditional |\ifchilddocmanual| is true whenever
a part to be included by |\input| is being compiled,
and the name of the part is stored in |\childdocname|.

%%%%%%%%%%%%%%%%%%%%%%%%%%%%%%%%%%%%%%%%
\DescribeMacro{\childdocby}
Each part to be included by |\input| should start with:
%
\begin{center}
\begin{tabular}{l}
|\input{childdoc.def}|\\
|\childdocby{|\textit{main}|}|\\
\end{tabular}
\end{center}
%
The directive |\childdocby| is similar to |\childdocof|
described in \secref{sec:include},
but the subsequent selection of content must be done manually.
To that end, both |\ifchilddoc| and |\ifchilddocmanual|
will be true upon processing of a part,
and the name of the part is stored in |\childdocname|.
Note that |\jobname| will be set to the filename of the current part
so that each part receives an individual |.aux| file
that does not interfere with the |.aux| file(s) of the main document.
This behaviour can be altered by the alternative form
|\childdocby[*]{|\textit{main}|}| (with a non-empty optional argument)
which uses the |.aux| file of the main document
by setting |\jobname| to \textit{main}.

%%%%%%%%%%%%%%%%%%%%%%%%%%%%%%%%%%%%%%%%%%%%%%%%%%%%%%%%%%%%%%%%%%%%%%%%%%%%%%%%
\subsection{Driver Development}
\label{sec:driver}

The \textsf{childdoc} mechanism can also be use for the development
of definition files such as \LaTeX{} styles or classes.
This case differs from the above setup with multiple parts
included by |\include| in that no |\includeonly| should be invoked.
This can be achieved by starting the include file
(before |\ProvidesPackage|) with:
%
\begin{center}
\begin{tabular}{l}
|\input{childdoc.def}|\\
|\childdocforward{|\textit{main}|}|\\
\end{tabular}
\end{center}
%
or alternatively with:
%
\begin{center}
\begin{tabular}{l}
|\input{childdoc.def}|\\
|\childdocby{|\textit{main}|}|\\
\end{tabular}
\end{center}
%
Both forms have slightly different effects as described above.
The main file is prepared as usual, see \secref{sec:include}.

%%%%%%%%%%%%%%%%%%%%%%%%%%%%%%%%%%%%%%%%%%%%%%%%%%%%%%%%%%%%%%%%%%%%%%%%%%%%%%%%
\subsection{Legacy Detection}
\label{sec:detection}

The directive |\childdocmain| in the main file can detect
whether the complete document or merely a child is to be compiled
even without using the directive |\childdocof|.
This method is deprecated because it is less robust
and there is no compelling reason to use it;
it is merely provided for backward compatibility
and it may be removed in future versions.

If the detection mechanism is to be used,
it is mandatory to correctly specify
the filename of the main file as the argument of |\childdocmain|:
%
\begin{center}
\begin{tabular}{l}
|\input{childdoc.def}|\\
|\childdocmain{|\textit{main}|}|\\
\end{tabular}
\end{center}
%
If |\jobname| does not match the argument \textit{main} of |\childdocmain|,
it is assumed that |\jobname| points to the child file to be compiled.
When using |\childdocmain| with the main file specified as argument,
it suffices to start a child file
with just |\input{|\textit{main}|}|
without loading of the package and using |\childdocof|.
If instead all processing is done
with the appropriate \textsf{childdoc} directives,
the argument of \textit{main} of |\childdocmain| can be empty.

An alternative version of the command line processing described
in \secref{sec:commandline} using the detection mechanism reads:
%
\begin{center}
|... -jobname "|\textit{target}|" "|[\textit{flags}]%
[|\def\jobname{|\textit{dest}|}|]|\input{|\textit{main}|}"|
\end{center}

%%%%%%%%%%%%%%%%%%%%%%%%%%%%%%%%%%%%%%%%%%%%%%%%%%%%%%%%%%%%%%%%%%%%%%%%%%%%%%%%
\subsection{Manual Code}
\label{sec:manual}

In case one cannot be certain whether the definitions file |childdoc.def|
is installed on the target \TeX{} distribution
and one prefers not to ship it,
it is conceivable to paste a few relevant commands into the sources.

To that end, drop all statements |\input{childdoc.def}|
and perform the replacements as outlined below.
Instead of |\childdocmain{|\textit{main}|}| add the following code
to the top of the main file:
%
\begin{center}
\begin{tabular}{l}
|\||ifdefined\childdocname\endinput\||fi\newif\ifchilddoc|\\
|\edef\childdocname{\scantokens\expandafter{\jobname\noexpand}}|\\
|\def\childdocmain{|\textit{main}|}\||ifx\childdocmain\childdocname\||else|\\
|\childdoctrue\includeonly{\childdocname}\let\jobname\childdocmain\||fi|\\
\end{tabular}
\end{center}
%
Instead of |\childdocof{|\textit{main}|}| just include the main file
at the top of each child file:
%
\begin{center}
|\input{|\textit{main}|}|
\end{center}
%
A simple redirection |\childdocforward{|\textit{dest}|}| is achieved by:
%
\begin{center}
|\def\jobname{|\textit{dest}|}\input{\jobname}|
\end{center}
%
The redirection with prefix
|\childdocforwardprefix[|\textit{prefix}|]{|\textit{dest}|}|
is accomplished by:
%
\begin{center}
\begin{tabular}{l}
|{\edef\jobname{\scantokens\expandafter{\jobname\noexpand}}|\\
|\def\redirectjob |\textit{prefix}|#1~~~{\gdef\jobname{|\textit{dest}|#1}}|\\
|\expandafter\redirectjob\jobname~~~}\input{\jobname}|
\end{tabular}
\end{center}

In an alternative approach,
child documents can be compiled by a specific command line
without additional code or specific definitions:
%
\begin{center}
|... -jobname "|\textit{target}|" "|[\textit{flags}]%
|\includeonly{|\textit{dest}|}\input{|\textit{main}|}"|
\end{center}
%

%%%%%%%%%%%%%%%%%%%%%%%%%%%%%%%%%%%%%%%%%%%%%%%%%%%%%%%%%%%%%%%%%%%%%%%%%%%%%%%%
%%%%%%%%%%%%%%%%%%%%%%%%%%%%%%%%%%%%%%%%%%%%%%%%%%%%%%%%%%%%%%%%%%%%%%%%%%%%%%%%
\section{Information}

%%%%%%%%%%%%%%%%%%%%%%%%%%%%%%%%%%%%%%%%%%%%%%%%%%%%%%%%%%%%%%%%%%%%%%%%%%%%%%%%
\subsection{Copyright}

Copyright \copyright{} 2017--2018 Niklas Beisert

This work may be distributed and/or modified under the
conditions of the \LaTeX{} Project Public License, either version 1.3
of this license or (at your option) any later version.
The latest version of this license is in
  \url{http://www.latex-project.org/lppl.txt}
and version 1.3 or later is part of all distributions of \LaTeX{}
version 2005/12/01 or later.

This work has the LPPL maintenance status `maintained'.

The Current Maintainer of this work is Niklas Beisert.

This work consists of the files |README.txt|, |childdoc.ins| and |childdoc.dtx|
as well as the derived files |childdoc.def|, |cdocsamp.tex|
with |cdocsch1.tex|, |cdocsch2.tex|, |cdocspt3.tex|, |cdocspt4.tex|,
|cdocsdrf.tex|, |cdocsfn1.tex|, |cdocsfn2.tex|
as well as |childdoc.pdf|.

%%%%%%%%%%%%%%%%%%%%%%%%%%%%%%%%%%%%%%%%%%%%%%%%%%%%%%%%%%%%%%%%%%%%%%%%%%%%%%%%
\subsection{Files and Installation}

The package consists of the files:
%
\begin{center}
\begin{tabular}{ll}
    |README.txt|   & readme file \\
    |childdoc.ins| & installation file \\
    |childdoc.dtx| & source file \\
    |childdoc.def| & definition file \\
    |cdocsamp.tex| & sample main file \\
    |cdocsch1.tex| & sample include file \\
    |cdocsch2.tex| & sample include file \\
    |cdocspt3.tex| & sample part file \\
    |cdocspt4.tex| & sample part file \\
    |cdocsdrf.tex| & sample redirection file \\
    |cdocsfn1.tex| & sample redirection file \\
    |cdocsfn2.tex| & sample redirection file \\
    |childdoc.pdf| & manual
\end{tabular}
\end{center}
%
The distribution consists of the files
|README.txt|, |childdoc.ins| and |childdoc.dtx|.
%
\begin{itemize}
\item
Run (pdf)\LaTeX{} on |childdoc.dtx|
to compile the manual |childdoc.pdf| (this file).
\item
Run \LaTeX{} on |childdoc.ins| to create the definitions file |childdoc.def|
and the sample |cdocsamp.tex| with include files
|cdocsch1.tex|, |cdocsch2.tex|, |cdocspt3.tex|, |cdocspt4.tex|,
|cdocsdrf.tex|, |cdocsfn1.tex|, |cdocsfn2.tex|.
Then copy the file |childdoc.def| to an appropriate directory of your \LaTeX{}
distribution, e.g.\ \textit{texmf-root}|/tex/latex/childdoc|.
\end{itemize}

%%%%%%%%%%%%%%%%%%%%%%%%%%%%%%%%%%%%%%%%%%%%%%%%%%%%%%%%%%%%%%%%%%%%%%%%%%%%%%%%
\subsection{Related CTAN Packages}

There are several other packages which offer a similar functionality:
%
\begin{itemize}
\item
The packages
\href{http://ctan.org/pkg/docmute}{\textsf{docmute}},
\href{http://ctan.org/pkg/includex}{\textsf{includex}} and
\href{http://ctan.org/pkg/standalone}{\textsf{standalone}}
provide commands to include only the document body of
a child file thus allowing both files to be compiled individually.
\item
The packages \href{http://ctan.org/pkg/subdocs}{\textsf{subdocs}}
and \href{http://ctan.org/pkg/subfiles}{\textsf{subfiles}}
provide structures in which the main and child documents can be
encapsulated and allowing them to be compiled individually.
The inclusion mechanism is different from the conventional |\include|.
\item
The package \href{http://ctan.org/pkg/combine}{\textsf{combine}}
is an elaborate solution to combine several documents into one.
\end{itemize}
%
See also the CTAN topic \href{http://ctan.org/topic/subdocs}{\textsf{subdocs}}
for further related packages.
The present package differs from the above solutions in that
a document structure constructed with the conventional |\include| mechanism
just needs two extra commands at the top of every file
such that all constituent files can be compiled individually.

%%%%%%%%%%%%%%%%%%%%%%%%%%%%%%%%%%%%%%%%%%%%%%%%%%%%%%%%%%%%%%%%%%%%%%%%%%%%%%%%
%\subsection{Feature Suggestions}
%
%The following is a list of features which may be useful for future
%versions of this package:
%%
%\begin{itemize}
%\item
%\ldots
%\end{itemize}

%%%%%%%%%%%%%%%%%%%%%%%%%%%%%%%%%%%%%%%%%%%%%%%%%%%%%%%%%%%%%%%%%%%%%%%%%%%%%%%%
\subsection{Revision History}

%%%%%%%%%%%%%%%%%%%%%%%%%%%%%%%%%%%%%%%%
\paragraph{v2.0:} 2018/12/30

\begin{itemize}
\item
immediate forward processing
\item
added |\childdocby| mechanism
\item
manual restructured
\end{itemize}

%%%%%%%%%%%%%%%%%%%%%%%%%%%%%%%%%%%%%%%%
\paragraph{v1.6:} 2018/01/17

\begin{itemize}
\item
application for development of include files
\item
corrections to manual
\end{itemize}

%%%%%%%%%%%%%%%%%%%%%%%%%%%%%%%%%%%%%%%%
\paragraph{v1.5:} 2017/05/21

\begin{itemize}
\item
more complete structuring introduced
\item
|\childdocof| introduced
\item
|\childdoc| renamed to |\childdocmain|
\item
|\childredirect| renamed to |\childdocforward| and |\childdocforwardprefix|
and functionality expanded
\end{itemize}

%%%%%%%%%%%%%%%%%%%%%%%%%%%%%%%%%%%%%%%%
\paragraph{v1.0:} 2017/04/27

\begin{itemize}
\item
manual and install package
\item
first version published on CTAN
\end{itemize}

%%%%%%%%%%%%%%%%%%%%%%%%%%%%%%%%%%%%%%%%
\paragraph{v0.6:} 2017/04/26

\begin{itemize}
\item
redirection mechanism added
\end{itemize}

%%%%%%%%%%%%%%%%%%%%%%%%%%%%%%%%%%%%%%%%
\paragraph{v0.5:} 2017/04/26

\begin{itemize}
\item
functionality in definition file
\end{itemize}


%%%%%%%%%%%%%%%%%%%%%%%%%%%%%%%%%%%%%%%%%%%%%%%%%%%%%%%%%%%%%%%%%%%%%%%%%%%%%%%%
%%%%%%%%%%%%%%%%%%%%%%%%%%%%%%%%%%%%%%%%%%%%%%%%%%%%%%%%%%%%%%%%%%%%%%%%%%%%%%%%
%%%%%%%%%%%%%%%%%%%%%%%%%%%%%%%%%%%%%%%%%%%%%%%%%%%%%%%%%%%%%%%%%%%%%%%%%%%%%%%%
\appendix

\settowidth\MacroIndent{\rmfamily\scriptsize 000\ }

 \DocInput{childdoc.dtx}

\end{document}
%</driver>
% \fi
%
% %%%%%%%%%%%%%%%%%%%%%%%%%%%%%%%%%%%%%%%%%%%%%%%%%%%%%%%%%%%%%%%%%%%%%%%%%%%%%%
% %%%%%%%%%%%%%%%%%%%%%%%%%%%%%%%%%%%%%%%%%%%%%%%%%%%%%%%%%%%%%%%%%%%%%%%%%%%%%%
% \section{Sample}
%\iffalse
%<*samplemain>
%\fi
%
% The following presents a sample document
% with two chapters, two parts, a title page,
% a compile flag as well as three forwarding files to set the flag.
% It consists of eight |.tex| files:
% \begin{center}
% \begin{tabular}{ll}
% |cdocsamp.tex|&main file\\
% |cdocsch1.tex|&include file for chapter 1\\
% |cdocsch2.tex|&include file for chapter 2\\
% |cdocspt3.tex|&include file for part 3\\
% |cdocspt4.tex|&include file for part 4\\
% |cdocsdrf.tex|&forwarding file for main file in draft mode\\
% |cdocsfi1.tex|&forwarding file for final version of chapter 1\\
% |cdocsfi2.tex|&forwarding file for final version of chapter 2\\
% \end{tabular}
% \end{center}
% Each of the eight files can be compiled directly by the \LaTeX{} compiler.
%
% %%%%%%%%%%%%%%%%%%%%%%%%%%%%%%%%%%%%%%
% \paragraph{Main File.}
%
% The main file is called |cdocsamp.tex|.
%
% Load the \textsf{childdoc} definitions and
% declare the filename for the main document:
%    \begin{macrocode}
\input{childdoc.def}
\childdocmain{}
%    \end{macrocode}

% Optional override for |\version| flag:
%    \begin{macrocode}
%%\ifchilddoc\else\providecommand{\version}{draft}\fi
%    \end{macrocode}

% Define the default values for the |\version| flag
% (|final| for the main file and |draft| for childs):
%    \begin{macrocode}
\ifchilddoc
\providecommand{\version}{draft}
\else
\providecommand{\version}{final}
\fi
%    \end{macrocode}

% Load the standard document class:
%    \begin{macrocode}
\documentclass[12pt]{article}
%    \end{macrocode}

% Start the document body:
%    \begin{macrocode}
\begin{document}
%    \end{macrocode}

% Declare a title page.
% Print title, part of document being processed and version flag:
%    \begin{macrocode}
\addtocounter{page}{-1}
\begin{center}
{\LARGE\bfseries{}childdoc example\par}
\vspace{1cm}
\ifchilddoc
\ifchilddocmanual part\else chapter\fi:
`\childdocname' of `\childdocjob'\par
\else
main document: `\childdocjob'\par
\fi
version: \version\par
\end{center}
\newpage
%    \end{macrocode}

% Manually include selected file,
% otherwise process as usual:
%    \begin{macrocode}
\ifchilddocmanual
\section*{part `\childdocname'}
\input{\childdocname}
\else
%    \end{macrocode}

% Include the two chapters:
%    \begin{macrocode}
\include{cdocsch1}
\include{cdocsch2}
%    \end{macrocode}

% Include the two parts unless only chapters should be displayed:
%    \begin{macrocode}
\ifchilddoc\else
\section{part three}
\input{cdocspt3}
\section{part four}
\input{cdocspt4}
\fi
%    \end{macrocode}

% Process as usual until here:
%    \begin{macrocode}
\fi
%    \end{macrocode}

% End of document body:
%    \begin{macrocode}
\end{document}
%    \end{macrocode}
%\iffalse
%</samplemain>
%\fi
%
% %%%%%%%%%%%%%%%%%%%%%%%%%%%%%%%%%%%%%%
% \paragraph{Chapter Include Files.}
%
% The include files are called |cdocsch1.tex| and |cdocsch2.tex|.
%
%\iffalse
%<*samplechap1|samplechap2>
%\fi

% Optional override for |\version| flag:
%    \begin{macrocode}
%%\providecommand{\version}{final}
%    \end{macrocode}

% Include the main document:
%    \begin{macrocode}
\input{childdoc.def}
\childdocof{cdocsamp}
%    \end{macrocode}

%\iffalse
%</samplechap1|samplechap2>
%\fi
%
%\iffalse
%<*samplechap1>
%\fi
% Some text for chapter 1:
%    \begin{macrocode}
\section{one}
some text in chapter one
%    \end{macrocode}

%\iffalse
%</samplechap1>
%\fi
% Some text for chapter 2:
%\iffalse
%<*samplechap2>
%\fi
%    \begin{macrocode}
\section{two}
more text in chapter two
%    \end{macrocode}

%\iffalse
%</samplechap2>
%\fi
%
% %%%%%%%%%%%%%%%%%%%%%%%%%%%%%%%%%%%%%%
% \paragraph{Part Include Files.}
%
% The include files are called |cdocspt3.tex| and |cdocspt4.tex|.
%
%\iffalse
%<*samplepart3|samplepart4>
%\fi

% Optional override for |\version| flag:
%    \begin{macrocode}
%%\providecommand{\version}{final}
%    \end{macrocode}

% Include the main document:
%    \begin{macrocode}
\input{childdoc.def}
\childdocby{cdocsamp}
%    \end{macrocode}

%\iffalse
%</samplepart3|samplepart4>
%\fi
%
%\iffalse
%<*samplepart3>
%\fi
% Some text for part 3:
%    \begin{macrocode}
some text in part three
%    \end{macrocode}

%\iffalse
%</samplepart3>
%\fi
% Some text for part 4:
%\iffalse
%<*samplepart4>
%\fi
%    \begin{macrocode}
more text in part four
%    \end{macrocode}

%\iffalse
%</samplepart4>
%\fi
%
% %%%%%%%%%%%%%%%%%%%%%%%%%%%%%%%%%%%%%%
% \paragraph{Forwarding for a Complete Draft.}
%
% The following forwarding file |cdocsdrf.tex|
% compiles the main document in draft mode:
%\iffalse
%<*sampledraft>
%\fi
%    \begin{macrocode}
\def\version{draft}
\input{childdoc.def}
\childdocforward{cdocsamp}
%    \end{macrocode}

%\iffalse
%</sampledraft>
%\fi
%
% %%%%%%%%%%%%%%%%%%%%%%%%%%%%%%%%%%%%%%
% \paragraph{Forwarding for Final Version of the Chapters.}
%
% The following forwarding files |cdocsfn1.tex| and |cdocsfn2.tex|
% (with identical content)
% compile the final versions of the child documents
% |cdocsch1.tex| and |cdocsch2.tex|, respectively:
%\iffalse
%<*samplefinal>
%\fi
%    \begin{macrocode}
\def\version{final}
\input{childdoc.def}
\childdocforwardprefix[cdocsamp]{cdocsfn}{cdocsch}
%    \end{macrocode}

%\iffalse
%</samplefinal>
%\fi
%
% %%%%%%%%%%%%%%%%%%%%%%%%%%%%%%%%%%%%%%
% \paragraph{Command Line Processing.}
%
% The following three command lines generate the output files
% |cdocscld|, |cdocscl1| and |cdocscl2|
% which should be identical to
% |cdocsdrf|, |cdocsch1| and |cdocsfn2|, respectively:
% \begin{center}
% \begin{tabular}{l}
% |latex -jobname cdocscld \|\\
% |  "\def\version{draft}\input{childdoc.def}\childdocforward{cdocsamp}"|\\
% |latex -jobname cdocscl1 \|\\
% |  "\input{childdoc.def}\childdocforward[cdocsamp]{cdocsch1}"|\\
% |latex -jobname cdocscl2 \|\\
% |  "\def\version{final}\input{childdoc.def}\childdocforward{cdocsch2}"|
% \end{tabular}
% \end{center}
% Note that the trailing backslash on each first line
% merely continues the input to the second line
% (for convenient cut ant paste).
% Furthermore, the command |latex| can be replaced by any
% of its alternative versions such as |pdflatex|.
%
% %%%%%%%%%%%%%%%%%%%%%%%%%%%%%%%%%%%%%%%%%%%%%%%%%%%%%%%%%%%%%%%%%%%%%%%%%%%%%%
% %%%%%%%%%%%%%%%%%%%%%%%%%%%%%%%%%%%%%%%%%%%%%%%%%%%%%%%%%%%%%%%%%%%%%%%%%%%%%%
% \section{Implementation}
%\iffalse
%<*package>
%\fi
%
% This section describes the definitions file |childdoc.def|.

% The definitions cannot be loaded using |\usepackage| or |\RequirePackage|
% which has a mechanism to prevent loading a style file more than once.
% When loading the definitions by means of |\input|
% multiple instances have to be prevented manually:
%\iffalse
%This code needs to be before the `\ProvidesFile' directive
%which is defined at the beginning of this file.
%Therefore it is also placed there and commented out here.
%</package>
%<*discard>
%\fi
%    \begin{macrocode}
\ifdefined\childdocmain\endinput\fi
%    \end{macrocode}
%\iffalse
%</discard>
%<*package>
%\fi
%
% \macro{\ifchilddoc}
% \macro{\ifchilddocmanual}
% The conditional |\ifchilddoc| tells whether a
% child (true) or main (false) document is being compiled.
% The conditional |\ifchilddocmanual| tells whether
% the |\includeonly| mechanism is used (false) or
% the selection of child files must be performed manually (true).
% The definitions initialise to false:
%    \begin{macrocode}
\newif\ifchilddoc
\newif\ifchilddocmanual
%    \end{macrocode}

% \macro{\childdocname}
% \macro{\childdocjob}
% The macro |\childdocname| stores the name of the main document
% to be compiled. The macro |\childdocjob| stores the name of
% the document on which the \LaTeX{} compiler was originally invoked.
% The content of |\jobname| cannot be compared
% to filenames specified in the source due to different catcodes.
% The following code rescans |\jobname|, stores the result
% in |\childdocname| and saves a copy in |\childdocjob|:
%    \begin{macrocode}
\edef\childdocname{\scantokens\expandafter{\jobname\noexpand}}
\let\childdocjob\childdocname
%    \end{macrocode}

% \macro{\childdocdisable}
% The macro |\childdocdisable| prevents the main file
% from being processed more than once.
% At this stage, the main document command |\childdocmain|
% is assumed to be called once again where it should do nothing.
% Any subsequent call to it should prevent
% a secondary processing of the main document
% It overwrites the forwarding commands
% |\childdocof| and |\childdocforward|
% with empty macros to prevent further inclusions of the main document:
%    \begin{macrocode}
\newcommand{\childdocdisable}
{
  \renewcommand{\childdocmain}[1]{\renewcommand{\childdocmain}[1]{\endinput}}
  \renewcommand{\childdocof}[1]{}
  \renewcommand{\childdocby}[2][]{}
  \renewcommand{\childdocforward}[2][]{}
  \renewcommand{\childdocdisable}{}
}
%    \end{macrocode}

% \macro{\childdocmain}
% The macro |\childdocmain| is to be called at the top of the main file
% with nothing or the main filename (without extension) as argument.
% First, it breaks loops.
% If the argument is not empty and does not match |\childdocname|
% (which is set by the first inclusion of |childdoc.def|),
% |\ifchilddoc| is set to true, |\includeonly| is applied to the child file
% and |\jobname| is set to the main file
% (for proper handling of |.aux| files):
%    \begin{macrocode}
\newcommand{\childdocmain}[1]
{
  \childdocdisable\childdocmain{}
  \if?#1?\else
    \begingroup
      \def\childdoctmp{#1}
      \ifx\childdoctmp\childdocname
        \def\childdoctmp{}
      \else
        \def\childdoctmp
        {
          \childdoctrue
          \includeonly{\childdocname}
          \def\childdocjob{#1}
          \def\jobname{#1}
        }
      \fi
      \expandafter
    \endgroup
    \childdoctmp
  \fi
}
%    \end{macrocode}

% \macro{\childdocof}
% The command |\childdocof| redirects
% compilation to the main file |#1|.
%    \begin{macrocode}
\newcommand{\childdocof}[1]
{
  \childdocdisable
  \childdoctrue
  \includeonly{\childdocname}
  \def\jobname{#1}
  \def\childdocjob{#1}
  \input{#1}
}
%    \end{macrocode}

% \macro{\childdocby}
% The command |\childdocby| ....
%    \begin{macrocode}
\newcommand{\childdocby}[2][]
{
  \childdocdisable
  \childdoctrue
  \childdocmanualtrue
  \if?#1?\else
    \def\jobname{#2}
  \fi
  \def\childdocjob{#2}
  \input{#2}
  \endinput
}
%    \end{macrocode}

% \macro{\childdocforward}
% The command |\childdocforward| redirects
% compilation to the main file or
% (if the optional argument is given) a child file.
% Parameters are set as if the main file
% or a child file starting with |\childdocof| was compiled.
% Then compilation is handed over to the main file:
%    \begin{macrocode}
\newcommand{\childdocforward}[2][]
{
  \begingroup
    \if?#1?
      \def\childdoctmp
      {
        \def\childdocname{#2}
        \def\childdocjob{#2}
        \def\jobname{#2}
        \input{#2}
        \endinput
      }
    \else
      \def\childdoctmp
      {
        \childdocdisable
        \def\childdocname{#2}
        \childdoctrue
        \includeonly{#2}
        \def\childdocjob{#1}
        \def\jobname{#1}
        \input{#1}
        \endinput
      }
    \fi
    \expandafter
  \endgroup
  \childdoctmp
}
%    \end{macrocode}

% \macro{\childdocforwardprefix}
% The command |\childdocforwardprefix| redirects
% compilation to the main or a child file by means of a pattern.
% The prefix |#1| in the current filename is replaced by |#2|
% and the suffix of the current filename is kept
% (it is assumed that the filename does not contain the substring `|~~~|'
% which is used as a delimiter).
% Compilation is handed over to the new file by |\childdocforward|:
%    \begin{macrocode}
\newcommand{\childdocforwardprefix}[3][]
{
  \begingroup
    \def\childdocextract #2##1~~~{\def\childdoctmp{\childdocforward[#1]{#3##1}}}
    \expandafter\childdocextract\childdocname~~~
    \expandafter
  \endgroup
  \childdoctmp
}
%    \end{macrocode}

% \macro{\childdoc}
% The deprecated macro |\childdoc| is a legacy version of |\childdocmain|:
%    \begin{macrocode}
\newcommand{\childdoc}{\childdocmain}
%    \end{macrocode}

% \macro{\childdocredirect}
% The deprecated macro |\childdocredirect| is a legacy version
% of |\childdocforward| and |\childdocforwardprefix|:
%    \begin{macrocode}
\newcommand{\childdocredirect}[2][]
{
  \begingroup
    \if?#1?
      \def\childdoctmp{\childdocforward{#2}}
    \else
      \def\childdoctmp{\childdocforwardprefix{#1}{#2}}
    \fi
    \expandafter
  \endgroup
  \childdoctmp
}
%    \end{macrocode}

%\iffalse
%</package>
%\fi
%
\endinput

\childdocmain{}
%    \end{macrocode}

% Optional override for |\version| flag:
%    \begin{macrocode}
%%\ifchilddoc\else\providecommand{\version}{draft}\fi
%    \end{macrocode}

% Define the default values for the |\version| flag
% (|final| for the main file and |draft| for childs):
%    \begin{macrocode}
\ifchilddoc
\providecommand{\version}{draft}
\else
\providecommand{\version}{final}
\fi
%    \end{macrocode}

% Load the standard document class:
%    \begin{macrocode}
\documentclass[12pt]{article}
%    \end{macrocode}

% Start the document body:
%    \begin{macrocode}
\begin{document}
%    \end{macrocode}

% Declare a title page.
% Print title, part of document being processed and version flag:
%    \begin{macrocode}
\addtocounter{page}{-1}
\begin{center}
{\LARGE\bfseries{}childdoc example\par}
\vspace{1cm}
\ifchilddoc
\ifchilddocmanual part\else chapter\fi:
`\childdocname' of `\childdocjob'\par
\else
main document: `\childdocjob'\par
\fi
version: \version\par
\end{center}
\newpage
%    \end{macrocode}

% Manually include selected file,
% otherwise process as usual:
%    \begin{macrocode}
\ifchilddocmanual
\section*{part `\childdocname'}
\input{\childdocname}
\else
%    \end{macrocode}

% Include the two chapters:
%    \begin{macrocode}
\include{cdocsch1}
\include{cdocsch2}
%    \end{macrocode}

% Include the two parts unless only chapters should be displayed:
%    \begin{macrocode}
\ifchilddoc\else
\section{part three}
\input{cdocspt3}
\section{part four}
\input{cdocspt4}
\fi
%    \end{macrocode}

% Process as usual until here:
%    \begin{macrocode}
\fi
%    \end{macrocode}

% End of document body:
%    \begin{macrocode}
\end{document}
%    \end{macrocode}
%\iffalse
%</samplemain>
%\fi
%
% %%%%%%%%%%%%%%%%%%%%%%%%%%%%%%%%%%%%%%
% \paragraph{Chapter Include Files.}
%
% The include files are called |cdocsch1.tex| and |cdocsch2.tex|.
%
%\iffalse
%<*samplechap1|samplechap2>
%\fi

% Optional override for |\version| flag:
%    \begin{macrocode}
%%\providecommand{\version}{final}
%    \end{macrocode}

% Include the main document:
%    \begin{macrocode}
% \iffalse
%
% childdoc.dtx Copyright (C) 2017-2018 Niklas Beisert
%
% This work may be distributed and/or modified under the
% conditions of the LaTeX Project Public License, either version 1.3
% of this license or (at your option) any later version.
% The latest version of this license is in
%   http://www.latex-project.org/lppl.txt
% and version 1.3 or later is part of all distributions of LaTeX
% version 2005/12/01 or later.
%
% This work has the LPPL maintenance status `maintained'.
%
% The Current Maintainer of this work is Niklas Beisert.
%
% This work consists of the files childdoc.dtx and childdoc.ins
% and the derived files childdoc.def and cdocsamp.tex with
% cdocsch1.tex, cdocsch2.tex, cdocsdrf.tex, cdocsfn1.tex, cdocsfn2.tex.
%
%<package>\ifdefined\childdocmain\endinput\fi
%<package>\ProvidesFile{childdoc.def}[2018/12/30 v2.0 child document driver]
%<samplemain>\ProvidesFile{cdocsamp.tex}[2018/12/30 v2.0 sample for childdoc]
%<*driver>
%\ProvidesFile{childdoc.drv}[2018/12/30 v2.0 childdoc reference manual file]
\PassOptionsToClass{10pt,a4paper}{article}
\documentclass{ltxdoc}

\usepackage[margin=35mm]{geometry}
\usepackage{hyperref}
\usepackage{hyperxmp}
\usepackage[usenames]{color}

\hypersetup{colorlinks=true}
\hypersetup{pdfstartview=FitH}
\hypersetup{pdfpagemode=UseNone}
\hypersetup{pdfsource={}}
\hypersetup{pdflang={en-UK}}
\hypersetup{pdfcopyright={Copyright 2017-2018 Niklas Beisert.
  This work may be distributed and/or modified under the
  conditions of the LaTeX Project Public License, either version 1.3
  of this license or (at your option) any later version.}}
\hypersetup{pdflicenseurl={http://www.latex-project.org/lppl.txt}}
\hypersetup{pdfcontactaddress={ETH Zurich, ITP, HIT K,
  Wolfgang-Pauli-Strasse 27}}
\hypersetup{pdfcontactpostcode={8093}}
\hypersetup{pdfcontactcity={Zurich}}
\hypersetup{pdfcontactcountry={Switzerland}}
\hypersetup{pdfcontactemail={nbeisert@itp.phys.ethz.ch}}
\hypersetup{pdfcontacturl={http://people.phys.ethz.ch/\xmptilde nbeisert/}}

\newcommand{\secref}[1]{\hyperref[#1]{section \ref*{#1}}}

\parskip1ex
\parindent0pt
\let\olditemize\itemize
\def\itemize{\olditemize\parskip0pt}

\begin{document}

\title{The \textsf{childdoc} Package}
\hypersetup{pdftitle={The childdoc Package}}
\author{Niklas Beisert\\[2ex]
  Institut f\"ur Theoretische Physik\\
  Eidgen\"ossische Technische Hochschule Z\"urich\\
  Wolfgang-Pauli-Strasse 27, 8093 Z\"urich, Switzerland\\[1ex]
  \href{mailto:nbeisert@itp.phys.ethz.ch}
  {\texttt{nbeisert@itp.phys.ethz.ch}}}
\hypersetup{pdfauthor={Niklas Beisert}}
\hypersetup{pdfsubject={Manual for the LaTeX2e Package childdoc}}
\date{30 December 2018, \textsf{v2.0}}
\maketitle

\begin{abstract}\noindent
\textsf{childdoc} is a \LaTeXe{} package
that enables the direct compilation
of document sections included by |\include|
to individual files.
\end{abstract}

\begingroup
\parskip0ex
\tableofcontents
\endgroup

%%%%%%%%%%%%%%%%%%%%%%%%%%%%%%%%%%%%%%%%%%%%%%%%%%%%%%%%%%%%%%%%%%%%%%%%%%%%%%%%
%%%%%%%%%%%%%%%%%%%%%%%%%%%%%%%%%%%%%%%%%%%%%%%%%%%%%%%%%%%%%%%%%%%%%%%%%%%%%%%%
\section{Introduction}

\LaTeX{} provides a mechanism to structure a large document (such as a book)
into a main file and several child files (containing the chapters)
using the |\include| command.
This mechanism is beneficial for documents
which span hundreds of pages in order to
make the source file(s) more manageable.
Moreover, compilation can be restricted to
selected child files by means of the |\includeonly| command.
The latter feature can be used to reduce the compilation time while editing
(this was significantly more useful in the earlier days of \LaTeX{})
or to generate a smaller document which is easier to navigate.
Another application of |\includeonly| is to generate
documents consisting of selected parts of the complete document.

However, there are a few drawbacks of the plain |\include| mechanism:
\begin{itemize}
\item
The child files cannot be compiled on their own,
they can only be compiled via the main file.
A naive editing environment
(such as a text editor with an option
to have the current file processed by \LaTeX)
may require one to switch to the main file before compiling;
attempting to compile the child file produces errors.
\item
The main file must be modified (each time)
to adjust the |\includeonly| command
to the present needs. This easily leaves the main file in a messy state.
\item
The generated document will always carry the filename
of the main document. This is inconvenient if
several child files are to be compiled and
to be kept for distribution.
\end{itemize}

The present package provides a simple interface
to make child files individually compilable by \LaTeX{}.
Compiling a child file then has the same effect as compiling
the main file with an |\includeonly| command
to select the appropriate child.
Moreover the generated document will carry the name of the child
rather than the main file.
This resolves all three above issues.

This feature is meant to make the editing of books,
thesis documents and lecture notes somewhat more convenient.
However, the package can also be used efficiently for
composing a series of documents (such as exercise sheets)
which are typically distributed individually.
It then assists the author in generating the individual documents
(potentially in different versions)
as well as a document containing the collected series.
Another application is in developing style files
or other kinds of included material
where compilation of the style file could redirect
to a sample or test file.

%%%%%%%%%%%%%%%%%%%%%%%%%%%%%%%%%%%%%%%%%%%%%%%%%%%%%%%%%%%%%%%%%%%%%%%%%%%%%%%%
%%%%%%%%%%%%%%%%%%%%%%%%%%%%%%%%%%%%%%%%%%%%%%%%%%%%%%%%%%%%%%%%%%%%%%%%%%%%%%%%
\section{Usage}

First of all, the package \textsf{childdoc} is \emph{not} a standard
\LaTeXe{} |.sty| style file! Therefore it needs to be invoked in
a non-standard way.

%%%%%%%%%%%%%%%%%%%%%%%%%%%%%%%%%%%%%%%%%%%%%%%%%%%%%%%%%%%%%%%%%%%%%%%%%%%%%%%%
\subsection{Included Files}
\label{sec:include}

%%%%%%%%%%%%%%%%%%%%%%%%%%%%%%%%%%%%%%%%
\DescribeMacro{\childdocmain}
To use the package, add the commands
\begin{center}
\begin{tabular}{l}
|\input{childdoc.def}|\\
|\childdocmain{}|\\
\end{tabular}
\end{center}
at the very top of the main \LaTeX{} file,
in particular \emph{before} the |\documentclass| statement!
The argument of |\childdocmain| should be left empty
(but it must be present).

%%%%%%%%%%%%%%%%%%%%%%%%%%%%%%%%%%%%%%%%
\DescribeMacro{\childdocof}
Furthermore, add the commands
\begin{center}
\begin{tabular}{l}
|\input{childdoc.def}|\\
|\childdocof{|\textit{main}|}|\\
\end{tabular}
\end{center}
at the top of every child file \textit{child}
which is included by |\include{|\textit{child}|}|
from within the main file
(or at least for those files to be compiled individually).
The argument \textit{main} must be the filename of the main file.

There are a couple of
considerations in setting up the main and child documents:

%%%%%%%%%%%%%%%%%%%%%%%%%%%%%%%%%%%%%%%%
\paragraph{Restrictions.}

Please note the following restrictions:
\begin{itemize}
\item
|\childdocmain| must be called with one argument \textit{main}
to ensure compatibility with earlier version of the package.
It must either be empty (|\childdocmain{}|)
or precisely match the filename of the main file in which it is specified.
See \secref{sec:detection} for further information.
\item
The filename \textit{main} must be specified without the |.tex| extension.
\item
The filename \textit{main} is case sensitive
(even in case-insensitive file systems)
due to internal string comparison.
\item
The argument \textit{main} should be fully expanded, it cannot be a macro.
\item
Subdirectories and special characters should be avoided in filenames.
\item
The command |\childdocmain{|\textit{main}|}| must be followed by a whitespace.
It should not be followed immediately by another command
or by a comment mark `|%|'.
This is because the \TeX{} parser reads the token immediately following
the argument of |\childdocmain| and puts it
at the beginning of every child section;
however, a white\-space is ignored.
\end{itemize}

%%%%%%%%%%%%%%%%%%%%%%%%%%%%%%%%%%%%%%%%
\paragraph{Content of Main File.}

It is advisable to place all content in the child files included by |\include|.
Any output contained in the main file will appear in all child documents
unless suppressed manually;
it cannot be suppressed automatically by the |\includeonly| directive
and thus should normally be avoided.
A method to include some content in the main file
by means of conditional processing is described in \secref{sec:conditional}.

%%%%%%%%%%%%%%%%%%%%%%%%%%%%%%%%%%%%%%%%
\paragraph{Page Numbering.}

When only a part of the document is compiled,
the appropriate numbering of pages
(as well as other status parameters)
is determined from the |.aux| files.
The latter contain information from previous passes.
However this information needs to propagate through
all intermediate child documents.
Therefore the page numbering in child documents may well
be inconsistent until the complete document is compiled at least once.

A useful (if unconventional) way to always ensure a consistent
page numbering is to restart the numbering in each child document
and denote the pages by `\textit{child}|.|\textit{page}'
where \textit{child} represents the chapter/section number of the child file.
This can be achieved by the command
|\numberwithin{page}{|\textit{child}|}|
of the \textsf{amsmath} package
where \textit{child} can be |chapter| or |section|
depending on the chosen structuring.
Alternatively, one can modify the macro |\thepage| appropriately
and reset the counter |page| at the start of each child file.

%%%%%%%%%%%%%%%%%%%%%%%%%%%%%%%%%%%%%%%%%%%%%%%%%%%%%%%%%%%%%%%%%%%%%%%%%%%%%%%%
\subsection{Conditional Processing}
\label{sec:conditional}

The package provides a mechanism to compile different versions
of a document. To customise the versions further some conditional processing
can come in handy to distinguish which version is being compiled.
The package provides two macros to describe the compilation context:

%%%%%%%%%%%%%%%%%%%%%%%%%%%%%%%%%%%%%%%%
\DescribeMacro{\ifchilddoc}
The conditional |\ifchilddoc| distinguishes between the compilation of
child documents and the main document:
%
\begin{center}
|\ifchilddoc |\textit{child-code}| |[|\||else |\textit{main-code}]| \||fi|
\end{center}

%%%%%%%%%%%%%%%%%%%%%%%%%%%%%%%%%%%%%%%%
\DescribeMacro{\childdocname}
\DescribeMacro{\childdocjob}
The macro |\childdocname| contains the filename (without extension)
of the main or child file being processed.
Note that |\childdocjob| will always contain the name of the main file.

%%%%%%%%%%%%%%%%%%%%%%%%%%%%%%%%%%%%%%%%
\paragraph{Title Page.}

Conditional processing can be used to include a title or banner page
in the main document when proper precautions are taken.
Importantly, the code in the main file should ensure that the page counter
(as well as other status parameters which are stored in the |.aux| files)
takes the same value after the conditional processing.
Otherwise the page numbers may take divergent values
depending on which part is compiled.

For example, a title page could be declared by:
%
\begin{center}
\begin{tabular}{l}
|\ifchilddoc\||else|\\
|\addtocounter{page}{-1}|\\
\textit{code for title page}\\
|\newpage|\\
|\||fi|
\end{tabular}
\end{center}
%
A banner page for the child documents can be generated by:
%
\begin{center}
\begin{tabular}{l}
|\ifchilddoc|\\
|\addtocounter{page}{-1}|\\
\textit{code for banner page}\\
|\newpage|\\
|\||fi|
\end{tabular}
\end{center}
%
Here one could write a message such as:
\begin{center}
|This is the part \childdocname{} of \childdocjob{}.|
\end{center}

%%%%%%%%%%%%%%%%%%%%%%%%%%%%%%%%%%%%%%%%%%%%%%%%%%%%%%%%%%%%%%%%%%%%%%%%%%%%%%%%
\subsection{Flags}
\label{sec:flags}

The package makes it easy to generate different versions
of the main or child documents.
To this end compilation flags can be defined
and assigned different default values.
They will be particularly useful in conjunction
with the forwarding mechanism described in \secref{sec:forward}.

For example, it may be useful to have a flag |\version|
which can be set to |draft| or |final|.
The document source will contain some conditional code
depending on the value of |\version|.
Suppose further, the flag should default to |final| for the main file
and to |draft| for child files
which is a natural assignment for editing the document.
This is achieved by placing the following code
in the preamble of the main document
(below the |\childdocmain| directive):
%
\begin{center}
\begin{tabular}{l}
|\ifchilddoc|\\
|\providecommand{\version}{draft}|\\
|\||else|\\
|\providecommand{\version}{final}|\\
|\||fi|
\end{tabular}
\end{center}
%
The definition by |\providecommand| makes sure
that previous definitions are not overwritten.
Further statements |\providecommand{\version}{...}|
can thus be added before the above code to override it.

For the main file, one might add a line
(between |\childdocmain| and the above block)
%
\begin{center}
|%\ifchilddoc\||else\providecommand{\version}{draft}\||fi|
\end{center}
%
which can be uncommented to produce a draft version.
Likewise one can add a line to the very top of a child file
(above the |\childdocof{|\textit{main}|}| directive)
%
\begin{center}
|%\providecommand{\version}{final}|
\end{center}
%
which can be uncommented to produce the final version of this child document.

%%%%%%%%%%%%%%%%%%%%%%%%%%%%%%%%%%%%%%%%%%%%%%%%%%%%%%%%%%%%%%%%%%%%%%%%%%%%%%%%
\subsection{Forwarding}
\label{sec:forward}

Different versions of the main or child documents
using compilation flags as described in \secref{sec:flags}
can be (permanently) stored in different files
for convenient compilation, viewing and distribution.
To this end, the package defines a command
to pass on compilation to a different file:

%%%%%%%%%%%%%%%%%%%%%%%%%%%%%%%%%%%%%%%%
\DescribeMacro{\childdocforward}
The command |\childdocforward| redirects processing to
another source file:
%
\begin{center}
\begin{tabular}{l}
|\input{childdoc.def}|\\
|\childdocforward[|\textit{main}|]{|\textit{dest}|}|\\
\end{tabular}
\end{center}
%
The argument \textit{dest} is the destination file
(without extension).
It should be the main file or one of the child files.
Note that further \textsf{childdoc} directives
such as |\childdocof| and |\childdocforward|
in the indicated file will be processed in this form.
The optional argument \textit{main}
passes on directly to the main file \textit{main}
while pretending to compile the child \textit{dest}.
This form behaves as if \textit{dest}
issues |\childdocof{|\textit{main}|}| right away,
and no further \textsf{childdoc} directives will be processed.

%%%%%%%%%%%%%%%%%%%%%%%%%%%%%%%%%%%%%%%%
\DescribeMacro{\...prefix}
In the alternative form |\childdocforwardprefix|,
%
\begin{center}
\begin{tabular}{l}
|\input{childdoc.def}|\\
|\childdocforwardprefix[|\textit{main}|]{|\textit{prefix}|}{|\textit{dest}|}|
\end{tabular}
\end{center}
%
the destination file is determined by a pattern
depending on the current file:
To make this work, the current file must be called
`{\textit{prefix}\hspace{0.2em}\textit{suffix}}'
with \textit{prefix} matching precisely the argument.
Processing is then passed on to the file
`{\textit{dest}\hspace{0.2em}\textit{suffix}}'.
Surely, the same effect is achieved by
directly specifying the
argument `{\textit{dest}\hspace{0.2em}\textit{suffix}}'
in the first form.
However, that requires to set up a different file
for each child. With the alternative form of the command
all these files can have exactly the same content
which simplifies setting them up and maintaining them.

For example, the following file |draft.tex|
with a compilation flag |\version| as described in \secref{sec:flags}
compiles the main document as a draft:
%
\begin{center}
\begin{tabular}{l}
|\def\version{draft}|\\
|\input{childdoc.def}|\\
|\childdocforward{|\textit{main}|}|
\end{tabular}
\end{center}
%
Likewise, the following files |final|\textit{nn}|.tex|
compile the final version of the child document
|child|\textit{nn}|.tex|:
%
\begin{center}
\begin{tabular}{l}
|\def\version{final}|\\
|\input{childdoc.def}|\\
|\childdocforwardprefix{final}{child}|
\end{tabular}
\end{center}
%

Note that when several versions of a main file and/or of each child file
are to be generated, it may be convenient to set up a |Makefile| or
shell script to automatise the process.

%%%%%%%%%%%%%%%%%%%%%%%%%%%%%%%%%%%%%%%%%%%%%%%%%%%%%%%%%%%%%%%%%%%%%%%%%%%%%%%%
\subsection{Command Line Processing}
\label{sec:commandline}

The effect of redirection files can also be achieved by invoking
the \LaTeX{} compiler with a more elaborate command line.
Most conveniently this should be done as part
of a shell script or a |Makefile|.

When using \textsf{childdoc} in the main file, the following
command lines effectively perform a redirection
(note that depending on the shell being used,
backslashes may have to be doubled: `|\|' $\to$ `|\\|'):
%
\begin{center}
|... -jobname "|\textit{target}|" |\\|"|[\textit{flags}]%
|\input{childdoc.def}\childdocforward[|\textit{main}|]{|\textit{dest}|}"|
\end{center}
%
Here \textit{target} is the name of the output file,
\textit{main} is the name of the main file
and \textit{dest} is the name of the main or child file to be processed
(all filenames without extensions).
The optional argument \textit{main} can be omitted
if \textit{main} matches \textit{dest}.
Optionally, compilation \textit{flags} can be defined via |\def| commands.
This command line makes the \TeX{} engine believe
it is compiling the file \textit{target}
whose content is specified as the latter parameter.
The provided code then forwards the processing to
\textit{main} or \textit{dest} as described in \secref{sec:forward}.

%%%%%%%%%%%%%%%%%%%%%%%%%%%%%%%%%%%%%%%%%%%%%%%%%%%%%%%%%%%%%%%%%%%%%%%%%%%%%%%%
\subsection{Include by Input}
\label{sec:input}

Including child documents by |\include| has some restrictions by design.
Most notably, the content of a child document always occupies
its own set of pages; pages cannot be shared between child documents.
Usually, this behaviour makes perfect sense
because each child document contain an essential part of the document.
However, in some situations it may be desirable to compose
a document from a collection of parts
without having mandatory page breaks between then.
For this case, the package
provides a mechanism to include parts
by |\input| which can also be processed individually.
However, by construction this mechanism
requires manual handling of the content to be output.

%%%%%%%%%%%%%%%%%%%%%%%%%%%%%%%%%%%%%%%%
\DescribeMacro{\ifchilddocmanual}
The main file should be prepared as usual, see \secref{sec:include}.
However, the document body must make a distinction
between processing of an individual part and of the main document, e.g.:
%
\begin{center}
\begin{tabular}{l}
|\ifchilddocmanual|\\
|\input{\childdocname}|\\
|\||else|\\
\textit{document body with }|\input{|\textit{part}|}|\\
|\||fi|
\end{tabular}
\end{center}
%
The conditional |\ifchilddocmanual| is true whenever
a part to be included by |\input| is being compiled,
and the name of the part is stored in |\childdocname|.

%%%%%%%%%%%%%%%%%%%%%%%%%%%%%%%%%%%%%%%%
\DescribeMacro{\childdocby}
Each part to be included by |\input| should start with:
%
\begin{center}
\begin{tabular}{l}
|\input{childdoc.def}|\\
|\childdocby{|\textit{main}|}|\\
\end{tabular}
\end{center}
%
The directive |\childdocby| is similar to |\childdocof|
described in \secref{sec:include},
but the subsequent selection of content must be done manually.
To that end, both |\ifchilddoc| and |\ifchilddocmanual|
will be true upon processing of a part,
and the name of the part is stored in |\childdocname|.
Note that |\jobname| will be set to the filename of the current part
so that each part receives an individual |.aux| file
that does not interfere with the |.aux| file(s) of the main document.
This behaviour can be altered by the alternative form
|\childdocby[*]{|\textit{main}|}| (with a non-empty optional argument)
which uses the |.aux| file of the main document
by setting |\jobname| to \textit{main}.

%%%%%%%%%%%%%%%%%%%%%%%%%%%%%%%%%%%%%%%%%%%%%%%%%%%%%%%%%%%%%%%%%%%%%%%%%%%%%%%%
\subsection{Driver Development}
\label{sec:driver}

The \textsf{childdoc} mechanism can also be use for the development
of definition files such as \LaTeX{} styles or classes.
This case differs from the above setup with multiple parts
included by |\include| in that no |\includeonly| should be invoked.
This can be achieved by starting the include file
(before |\ProvidesPackage|) with:
%
\begin{center}
\begin{tabular}{l}
|\input{childdoc.def}|\\
|\childdocforward{|\textit{main}|}|\\
\end{tabular}
\end{center}
%
or alternatively with:
%
\begin{center}
\begin{tabular}{l}
|\input{childdoc.def}|\\
|\childdocby{|\textit{main}|}|\\
\end{tabular}
\end{center}
%
Both forms have slightly different effects as described above.
The main file is prepared as usual, see \secref{sec:include}.

%%%%%%%%%%%%%%%%%%%%%%%%%%%%%%%%%%%%%%%%%%%%%%%%%%%%%%%%%%%%%%%%%%%%%%%%%%%%%%%%
\subsection{Legacy Detection}
\label{sec:detection}

The directive |\childdocmain| in the main file can detect
whether the complete document or merely a child is to be compiled
even without using the directive |\childdocof|.
This method is deprecated because it is less robust
and there is no compelling reason to use it;
it is merely provided for backward compatibility
and it may be removed in future versions.

If the detection mechanism is to be used,
it is mandatory to correctly specify
the filename of the main file as the argument of |\childdocmain|:
%
\begin{center}
\begin{tabular}{l}
|\input{childdoc.def}|\\
|\childdocmain{|\textit{main}|}|\\
\end{tabular}
\end{center}
%
If |\jobname| does not match the argument \textit{main} of |\childdocmain|,
it is assumed that |\jobname| points to the child file to be compiled.
When using |\childdocmain| with the main file specified as argument,
it suffices to start a child file
with just |\input{|\textit{main}|}|
without loading of the package and using |\childdocof|.
If instead all processing is done
with the appropriate \textsf{childdoc} directives,
the argument of \textit{main} of |\childdocmain| can be empty.

An alternative version of the command line processing described
in \secref{sec:commandline} using the detection mechanism reads:
%
\begin{center}
|... -jobname "|\textit{target}|" "|[\textit{flags}]%
[|\def\jobname{|\textit{dest}|}|]|\input{|\textit{main}|}"|
\end{center}

%%%%%%%%%%%%%%%%%%%%%%%%%%%%%%%%%%%%%%%%%%%%%%%%%%%%%%%%%%%%%%%%%%%%%%%%%%%%%%%%
\subsection{Manual Code}
\label{sec:manual}

In case one cannot be certain whether the definitions file |childdoc.def|
is installed on the target \TeX{} distribution
and one prefers not to ship it,
it is conceivable to paste a few relevant commands into the sources.

To that end, drop all statements |\input{childdoc.def}|
and perform the replacements as outlined below.
Instead of |\childdocmain{|\textit{main}|}| add the following code
to the top of the main file:
%
\begin{center}
\begin{tabular}{l}
|\||ifdefined\childdocname\endinput\||fi\newif\ifchilddoc|\\
|\edef\childdocname{\scantokens\expandafter{\jobname\noexpand}}|\\
|\def\childdocmain{|\textit{main}|}\||ifx\childdocmain\childdocname\||else|\\
|\childdoctrue\includeonly{\childdocname}\let\jobname\childdocmain\||fi|\\
\end{tabular}
\end{center}
%
Instead of |\childdocof{|\textit{main}|}| just include the main file
at the top of each child file:
%
\begin{center}
|\input{|\textit{main}|}|
\end{center}
%
A simple redirection |\childdocforward{|\textit{dest}|}| is achieved by:
%
\begin{center}
|\def\jobname{|\textit{dest}|}\input{\jobname}|
\end{center}
%
The redirection with prefix
|\childdocforwardprefix[|\textit{prefix}|]{|\textit{dest}|}|
is accomplished by:
%
\begin{center}
\begin{tabular}{l}
|{\edef\jobname{\scantokens\expandafter{\jobname\noexpand}}|\\
|\def\redirectjob |\textit{prefix}|#1~~~{\gdef\jobname{|\textit{dest}|#1}}|\\
|\expandafter\redirectjob\jobname~~~}\input{\jobname}|
\end{tabular}
\end{center}

In an alternative approach,
child documents can be compiled by a specific command line
without additional code or specific definitions:
%
\begin{center}
|... -jobname "|\textit{target}|" "|[\textit{flags}]%
|\includeonly{|\textit{dest}|}\input{|\textit{main}|}"|
\end{center}
%

%%%%%%%%%%%%%%%%%%%%%%%%%%%%%%%%%%%%%%%%%%%%%%%%%%%%%%%%%%%%%%%%%%%%%%%%%%%%%%%%
%%%%%%%%%%%%%%%%%%%%%%%%%%%%%%%%%%%%%%%%%%%%%%%%%%%%%%%%%%%%%%%%%%%%%%%%%%%%%%%%
\section{Information}

%%%%%%%%%%%%%%%%%%%%%%%%%%%%%%%%%%%%%%%%%%%%%%%%%%%%%%%%%%%%%%%%%%%%%%%%%%%%%%%%
\subsection{Copyright}

Copyright \copyright{} 2017--2018 Niklas Beisert

This work may be distributed and/or modified under the
conditions of the \LaTeX{} Project Public License, either version 1.3
of this license or (at your option) any later version.
The latest version of this license is in
  \url{http://www.latex-project.org/lppl.txt}
and version 1.3 or later is part of all distributions of \LaTeX{}
version 2005/12/01 or later.

This work has the LPPL maintenance status `maintained'.

The Current Maintainer of this work is Niklas Beisert.

This work consists of the files |README.txt|, |childdoc.ins| and |childdoc.dtx|
as well as the derived files |childdoc.def|, |cdocsamp.tex|
with |cdocsch1.tex|, |cdocsch2.tex|, |cdocspt3.tex|, |cdocspt4.tex|,
|cdocsdrf.tex|, |cdocsfn1.tex|, |cdocsfn2.tex|
as well as |childdoc.pdf|.

%%%%%%%%%%%%%%%%%%%%%%%%%%%%%%%%%%%%%%%%%%%%%%%%%%%%%%%%%%%%%%%%%%%%%%%%%%%%%%%%
\subsection{Files and Installation}

The package consists of the files:
%
\begin{center}
\begin{tabular}{ll}
    |README.txt|   & readme file \\
    |childdoc.ins| & installation file \\
    |childdoc.dtx| & source file \\
    |childdoc.def| & definition file \\
    |cdocsamp.tex| & sample main file \\
    |cdocsch1.tex| & sample include file \\
    |cdocsch2.tex| & sample include file \\
    |cdocspt3.tex| & sample part file \\
    |cdocspt4.tex| & sample part file \\
    |cdocsdrf.tex| & sample redirection file \\
    |cdocsfn1.tex| & sample redirection file \\
    |cdocsfn2.tex| & sample redirection file \\
    |childdoc.pdf| & manual
\end{tabular}
\end{center}
%
The distribution consists of the files
|README.txt|, |childdoc.ins| and |childdoc.dtx|.
%
\begin{itemize}
\item
Run (pdf)\LaTeX{} on |childdoc.dtx|
to compile the manual |childdoc.pdf| (this file).
\item
Run \LaTeX{} on |childdoc.ins| to create the definitions file |childdoc.def|
and the sample |cdocsamp.tex| with include files
|cdocsch1.tex|, |cdocsch2.tex|, |cdocspt3.tex|, |cdocspt4.tex|,
|cdocsdrf.tex|, |cdocsfn1.tex|, |cdocsfn2.tex|.
Then copy the file |childdoc.def| to an appropriate directory of your \LaTeX{}
distribution, e.g.\ \textit{texmf-root}|/tex/latex/childdoc|.
\end{itemize}

%%%%%%%%%%%%%%%%%%%%%%%%%%%%%%%%%%%%%%%%%%%%%%%%%%%%%%%%%%%%%%%%%%%%%%%%%%%%%%%%
\subsection{Related CTAN Packages}

There are several other packages which offer a similar functionality:
%
\begin{itemize}
\item
The packages
\href{http://ctan.org/pkg/docmute}{\textsf{docmute}},
\href{http://ctan.org/pkg/includex}{\textsf{includex}} and
\href{http://ctan.org/pkg/standalone}{\textsf{standalone}}
provide commands to include only the document body of
a child file thus allowing both files to be compiled individually.
\item
The packages \href{http://ctan.org/pkg/subdocs}{\textsf{subdocs}}
and \href{http://ctan.org/pkg/subfiles}{\textsf{subfiles}}
provide structures in which the main and child documents can be
encapsulated and allowing them to be compiled individually.
The inclusion mechanism is different from the conventional |\include|.
\item
The package \href{http://ctan.org/pkg/combine}{\textsf{combine}}
is an elaborate solution to combine several documents into one.
\end{itemize}
%
See also the CTAN topic \href{http://ctan.org/topic/subdocs}{\textsf{subdocs}}
for further related packages.
The present package differs from the above solutions in that
a document structure constructed with the conventional |\include| mechanism
just needs two extra commands at the top of every file
such that all constituent files can be compiled individually.

%%%%%%%%%%%%%%%%%%%%%%%%%%%%%%%%%%%%%%%%%%%%%%%%%%%%%%%%%%%%%%%%%%%%%%%%%%%%%%%%
%\subsection{Feature Suggestions}
%
%The following is a list of features which may be useful for future
%versions of this package:
%%
%\begin{itemize}
%\item
%\ldots
%\end{itemize}

%%%%%%%%%%%%%%%%%%%%%%%%%%%%%%%%%%%%%%%%%%%%%%%%%%%%%%%%%%%%%%%%%%%%%%%%%%%%%%%%
\subsection{Revision History}

%%%%%%%%%%%%%%%%%%%%%%%%%%%%%%%%%%%%%%%%
\paragraph{v2.0:} 2018/12/30

\begin{itemize}
\item
immediate forward processing
\item
added |\childdocby| mechanism
\item
manual restructured
\end{itemize}

%%%%%%%%%%%%%%%%%%%%%%%%%%%%%%%%%%%%%%%%
\paragraph{v1.6:} 2018/01/17

\begin{itemize}
\item
application for development of include files
\item
corrections to manual
\end{itemize}

%%%%%%%%%%%%%%%%%%%%%%%%%%%%%%%%%%%%%%%%
\paragraph{v1.5:} 2017/05/21

\begin{itemize}
\item
more complete structuring introduced
\item
|\childdocof| introduced
\item
|\childdoc| renamed to |\childdocmain|
\item
|\childredirect| renamed to |\childdocforward| and |\childdocforwardprefix|
and functionality expanded
\end{itemize}

%%%%%%%%%%%%%%%%%%%%%%%%%%%%%%%%%%%%%%%%
\paragraph{v1.0:} 2017/04/27

\begin{itemize}
\item
manual and install package
\item
first version published on CTAN
\end{itemize}

%%%%%%%%%%%%%%%%%%%%%%%%%%%%%%%%%%%%%%%%
\paragraph{v0.6:} 2017/04/26

\begin{itemize}
\item
redirection mechanism added
\end{itemize}

%%%%%%%%%%%%%%%%%%%%%%%%%%%%%%%%%%%%%%%%
\paragraph{v0.5:} 2017/04/26

\begin{itemize}
\item
functionality in definition file
\end{itemize}


%%%%%%%%%%%%%%%%%%%%%%%%%%%%%%%%%%%%%%%%%%%%%%%%%%%%%%%%%%%%%%%%%%%%%%%%%%%%%%%%
%%%%%%%%%%%%%%%%%%%%%%%%%%%%%%%%%%%%%%%%%%%%%%%%%%%%%%%%%%%%%%%%%%%%%%%%%%%%%%%%
%%%%%%%%%%%%%%%%%%%%%%%%%%%%%%%%%%%%%%%%%%%%%%%%%%%%%%%%%%%%%%%%%%%%%%%%%%%%%%%%
\appendix

\settowidth\MacroIndent{\rmfamily\scriptsize 000\ }

 \DocInput{childdoc.dtx}

\end{document}
%</driver>
% \fi
%
% %%%%%%%%%%%%%%%%%%%%%%%%%%%%%%%%%%%%%%%%%%%%%%%%%%%%%%%%%%%%%%%%%%%%%%%%%%%%%%
% %%%%%%%%%%%%%%%%%%%%%%%%%%%%%%%%%%%%%%%%%%%%%%%%%%%%%%%%%%%%%%%%%%%%%%%%%%%%%%
% \section{Sample}
%\iffalse
%<*samplemain>
%\fi
%
% The following presents a sample document
% with two chapters, two parts, a title page,
% a compile flag as well as three forwarding files to set the flag.
% It consists of eight |.tex| files:
% \begin{center}
% \begin{tabular}{ll}
% |cdocsamp.tex|&main file\\
% |cdocsch1.tex|&include file for chapter 1\\
% |cdocsch2.tex|&include file for chapter 2\\
% |cdocspt3.tex|&include file for part 3\\
% |cdocspt4.tex|&include file for part 4\\
% |cdocsdrf.tex|&forwarding file for main file in draft mode\\
% |cdocsfi1.tex|&forwarding file for final version of chapter 1\\
% |cdocsfi2.tex|&forwarding file for final version of chapter 2\\
% \end{tabular}
% \end{center}
% Each of the eight files can be compiled directly by the \LaTeX{} compiler.
%
% %%%%%%%%%%%%%%%%%%%%%%%%%%%%%%%%%%%%%%
% \paragraph{Main File.}
%
% The main file is called |cdocsamp.tex|.
%
% Load the \textsf{childdoc} definitions and
% declare the filename for the main document:
%    \begin{macrocode}
\input{childdoc.def}
\childdocmain{}
%    \end{macrocode}

% Optional override for |\version| flag:
%    \begin{macrocode}
%%\ifchilddoc\else\providecommand{\version}{draft}\fi
%    \end{macrocode}

% Define the default values for the |\version| flag
% (|final| for the main file and |draft| for childs):
%    \begin{macrocode}
\ifchilddoc
\providecommand{\version}{draft}
\else
\providecommand{\version}{final}
\fi
%    \end{macrocode}

% Load the standard document class:
%    \begin{macrocode}
\documentclass[12pt]{article}
%    \end{macrocode}

% Start the document body:
%    \begin{macrocode}
\begin{document}
%    \end{macrocode}

% Declare a title page.
% Print title, part of document being processed and version flag:
%    \begin{macrocode}
\addtocounter{page}{-1}
\begin{center}
{\LARGE\bfseries{}childdoc example\par}
\vspace{1cm}
\ifchilddoc
\ifchilddocmanual part\else chapter\fi:
`\childdocname' of `\childdocjob'\par
\else
main document: `\childdocjob'\par
\fi
version: \version\par
\end{center}
\newpage
%    \end{macrocode}

% Manually include selected file,
% otherwise process as usual:
%    \begin{macrocode}
\ifchilddocmanual
\section*{part `\childdocname'}
\input{\childdocname}
\else
%    \end{macrocode}

% Include the two chapters:
%    \begin{macrocode}
\include{cdocsch1}
\include{cdocsch2}
%    \end{macrocode}

% Include the two parts unless only chapters should be displayed:
%    \begin{macrocode}
\ifchilddoc\else
\section{part three}
\input{cdocspt3}
\section{part four}
\input{cdocspt4}
\fi
%    \end{macrocode}

% Process as usual until here:
%    \begin{macrocode}
\fi
%    \end{macrocode}

% End of document body:
%    \begin{macrocode}
\end{document}
%    \end{macrocode}
%\iffalse
%</samplemain>
%\fi
%
% %%%%%%%%%%%%%%%%%%%%%%%%%%%%%%%%%%%%%%
% \paragraph{Chapter Include Files.}
%
% The include files are called |cdocsch1.tex| and |cdocsch2.tex|.
%
%\iffalse
%<*samplechap1|samplechap2>
%\fi

% Optional override for |\version| flag:
%    \begin{macrocode}
%%\providecommand{\version}{final}
%    \end{macrocode}

% Include the main document:
%    \begin{macrocode}
\input{childdoc.def}
\childdocof{cdocsamp}
%    \end{macrocode}

%\iffalse
%</samplechap1|samplechap2>
%\fi
%
%\iffalse
%<*samplechap1>
%\fi
% Some text for chapter 1:
%    \begin{macrocode}
\section{one}
some text in chapter one
%    \end{macrocode}

%\iffalse
%</samplechap1>
%\fi
% Some text for chapter 2:
%\iffalse
%<*samplechap2>
%\fi
%    \begin{macrocode}
\section{two}
more text in chapter two
%    \end{macrocode}

%\iffalse
%</samplechap2>
%\fi
%
% %%%%%%%%%%%%%%%%%%%%%%%%%%%%%%%%%%%%%%
% \paragraph{Part Include Files.}
%
% The include files are called |cdocspt3.tex| and |cdocspt4.tex|.
%
%\iffalse
%<*samplepart3|samplepart4>
%\fi

% Optional override for |\version| flag:
%    \begin{macrocode}
%%\providecommand{\version}{final}
%    \end{macrocode}

% Include the main document:
%    \begin{macrocode}
\input{childdoc.def}
\childdocby{cdocsamp}
%    \end{macrocode}

%\iffalse
%</samplepart3|samplepart4>
%\fi
%
%\iffalse
%<*samplepart3>
%\fi
% Some text for part 3:
%    \begin{macrocode}
some text in part three
%    \end{macrocode}

%\iffalse
%</samplepart3>
%\fi
% Some text for part 4:
%\iffalse
%<*samplepart4>
%\fi
%    \begin{macrocode}
more text in part four
%    \end{macrocode}

%\iffalse
%</samplepart4>
%\fi
%
% %%%%%%%%%%%%%%%%%%%%%%%%%%%%%%%%%%%%%%
% \paragraph{Forwarding for a Complete Draft.}
%
% The following forwarding file |cdocsdrf.tex|
% compiles the main document in draft mode:
%\iffalse
%<*sampledraft>
%\fi
%    \begin{macrocode}
\def\version{draft}
\input{childdoc.def}
\childdocforward{cdocsamp}
%    \end{macrocode}

%\iffalse
%</sampledraft>
%\fi
%
% %%%%%%%%%%%%%%%%%%%%%%%%%%%%%%%%%%%%%%
% \paragraph{Forwarding for Final Version of the Chapters.}
%
% The following forwarding files |cdocsfn1.tex| and |cdocsfn2.tex|
% (with identical content)
% compile the final versions of the child documents
% |cdocsch1.tex| and |cdocsch2.tex|, respectively:
%\iffalse
%<*samplefinal>
%\fi
%    \begin{macrocode}
\def\version{final}
\input{childdoc.def}
\childdocforwardprefix[cdocsamp]{cdocsfn}{cdocsch}
%    \end{macrocode}

%\iffalse
%</samplefinal>
%\fi
%
% %%%%%%%%%%%%%%%%%%%%%%%%%%%%%%%%%%%%%%
% \paragraph{Command Line Processing.}
%
% The following three command lines generate the output files
% |cdocscld|, |cdocscl1| and |cdocscl2|
% which should be identical to
% |cdocsdrf|, |cdocsch1| and |cdocsfn2|, respectively:
% \begin{center}
% \begin{tabular}{l}
% |latex -jobname cdocscld \|\\
% |  "\def\version{draft}\input{childdoc.def}\childdocforward{cdocsamp}"|\\
% |latex -jobname cdocscl1 \|\\
% |  "\input{childdoc.def}\childdocforward[cdocsamp]{cdocsch1}"|\\
% |latex -jobname cdocscl2 \|\\
% |  "\def\version{final}\input{childdoc.def}\childdocforward{cdocsch2}"|
% \end{tabular}
% \end{center}
% Note that the trailing backslash on each first line
% merely continues the input to the second line
% (for convenient cut ant paste).
% Furthermore, the command |latex| can be replaced by any
% of its alternative versions such as |pdflatex|.
%
% %%%%%%%%%%%%%%%%%%%%%%%%%%%%%%%%%%%%%%%%%%%%%%%%%%%%%%%%%%%%%%%%%%%%%%%%%%%%%%
% %%%%%%%%%%%%%%%%%%%%%%%%%%%%%%%%%%%%%%%%%%%%%%%%%%%%%%%%%%%%%%%%%%%%%%%%%%%%%%
% \section{Implementation}
%\iffalse
%<*package>
%\fi
%
% This section describes the definitions file |childdoc.def|.

% The definitions cannot be loaded using |\usepackage| or |\RequirePackage|
% which has a mechanism to prevent loading a style file more than once.
% When loading the definitions by means of |\input|
% multiple instances have to be prevented manually:
%\iffalse
%This code needs to be before the `\ProvidesFile' directive
%which is defined at the beginning of this file.
%Therefore it is also placed there and commented out here.
%</package>
%<*discard>
%\fi
%    \begin{macrocode}
\ifdefined\childdocmain\endinput\fi
%    \end{macrocode}
%\iffalse
%</discard>
%<*package>
%\fi
%
% \macro{\ifchilddoc}
% \macro{\ifchilddocmanual}
% The conditional |\ifchilddoc| tells whether a
% child (true) or main (false) document is being compiled.
% The conditional |\ifchilddocmanual| tells whether
% the |\includeonly| mechanism is used (false) or
% the selection of child files must be performed manually (true).
% The definitions initialise to false:
%    \begin{macrocode}
\newif\ifchilddoc
\newif\ifchilddocmanual
%    \end{macrocode}

% \macro{\childdocname}
% \macro{\childdocjob}
% The macro |\childdocname| stores the name of the main document
% to be compiled. The macro |\childdocjob| stores the name of
% the document on which the \LaTeX{} compiler was originally invoked.
% The content of |\jobname| cannot be compared
% to filenames specified in the source due to different catcodes.
% The following code rescans |\jobname|, stores the result
% in |\childdocname| and saves a copy in |\childdocjob|:
%    \begin{macrocode}
\edef\childdocname{\scantokens\expandafter{\jobname\noexpand}}
\let\childdocjob\childdocname
%    \end{macrocode}

% \macro{\childdocdisable}
% The macro |\childdocdisable| prevents the main file
% from being processed more than once.
% At this stage, the main document command |\childdocmain|
% is assumed to be called once again where it should do nothing.
% Any subsequent call to it should prevent
% a secondary processing of the main document
% It overwrites the forwarding commands
% |\childdocof| and |\childdocforward|
% with empty macros to prevent further inclusions of the main document:
%    \begin{macrocode}
\newcommand{\childdocdisable}
{
  \renewcommand{\childdocmain}[1]{\renewcommand{\childdocmain}[1]{\endinput}}
  \renewcommand{\childdocof}[1]{}
  \renewcommand{\childdocby}[2][]{}
  \renewcommand{\childdocforward}[2][]{}
  \renewcommand{\childdocdisable}{}
}
%    \end{macrocode}

% \macro{\childdocmain}
% The macro |\childdocmain| is to be called at the top of the main file
% with nothing or the main filename (without extension) as argument.
% First, it breaks loops.
% If the argument is not empty and does not match |\childdocname|
% (which is set by the first inclusion of |childdoc.def|),
% |\ifchilddoc| is set to true, |\includeonly| is applied to the child file
% and |\jobname| is set to the main file
% (for proper handling of |.aux| files):
%    \begin{macrocode}
\newcommand{\childdocmain}[1]
{
  \childdocdisable\childdocmain{}
  \if?#1?\else
    \begingroup
      \def\childdoctmp{#1}
      \ifx\childdoctmp\childdocname
        \def\childdoctmp{}
      \else
        \def\childdoctmp
        {
          \childdoctrue
          \includeonly{\childdocname}
          \def\childdocjob{#1}
          \def\jobname{#1}
        }
      \fi
      \expandafter
    \endgroup
    \childdoctmp
  \fi
}
%    \end{macrocode}

% \macro{\childdocof}
% The command |\childdocof| redirects
% compilation to the main file |#1|.
%    \begin{macrocode}
\newcommand{\childdocof}[1]
{
  \childdocdisable
  \childdoctrue
  \includeonly{\childdocname}
  \def\jobname{#1}
  \def\childdocjob{#1}
  \input{#1}
}
%    \end{macrocode}

% \macro{\childdocby}
% The command |\childdocby| ....
%    \begin{macrocode}
\newcommand{\childdocby}[2][]
{
  \childdocdisable
  \childdoctrue
  \childdocmanualtrue
  \if?#1?\else
    \def\jobname{#2}
  \fi
  \def\childdocjob{#2}
  \input{#2}
  \endinput
}
%    \end{macrocode}

% \macro{\childdocforward}
% The command |\childdocforward| redirects
% compilation to the main file or
% (if the optional argument is given) a child file.
% Parameters are set as if the main file
% or a child file starting with |\childdocof| was compiled.
% Then compilation is handed over to the main file:
%    \begin{macrocode}
\newcommand{\childdocforward}[2][]
{
  \begingroup
    \if?#1?
      \def\childdoctmp
      {
        \def\childdocname{#2}
        \def\childdocjob{#2}
        \def\jobname{#2}
        \input{#2}
        \endinput
      }
    \else
      \def\childdoctmp
      {
        \childdocdisable
        \def\childdocname{#2}
        \childdoctrue
        \includeonly{#2}
        \def\childdocjob{#1}
        \def\jobname{#1}
        \input{#1}
        \endinput
      }
    \fi
    \expandafter
  \endgroup
  \childdoctmp
}
%    \end{macrocode}

% \macro{\childdocforwardprefix}
% The command |\childdocforwardprefix| redirects
% compilation to the main or a child file by means of a pattern.
% The prefix |#1| in the current filename is replaced by |#2|
% and the suffix of the current filename is kept
% (it is assumed that the filename does not contain the substring `|~~~|'
% which is used as a delimiter).
% Compilation is handed over to the new file by |\childdocforward|:
%    \begin{macrocode}
\newcommand{\childdocforwardprefix}[3][]
{
  \begingroup
    \def\childdocextract #2##1~~~{\def\childdoctmp{\childdocforward[#1]{#3##1}}}
    \expandafter\childdocextract\childdocname~~~
    \expandafter
  \endgroup
  \childdoctmp
}
%    \end{macrocode}

% \macro{\childdoc}
% The deprecated macro |\childdoc| is a legacy version of |\childdocmain|:
%    \begin{macrocode}
\newcommand{\childdoc}{\childdocmain}
%    \end{macrocode}

% \macro{\childdocredirect}
% The deprecated macro |\childdocredirect| is a legacy version
% of |\childdocforward| and |\childdocforwardprefix|:
%    \begin{macrocode}
\newcommand{\childdocredirect}[2][]
{
  \begingroup
    \if?#1?
      \def\childdoctmp{\childdocforward{#2}}
    \else
      \def\childdoctmp{\childdocforwardprefix{#1}{#2}}
    \fi
    \expandafter
  \endgroup
  \childdoctmp
}
%    \end{macrocode}

%\iffalse
%</package>
%\fi
%
\endinput

\childdocof{cdocsamp}
%    \end{macrocode}

%\iffalse
%</samplechap1|samplechap2>
%\fi
%
%\iffalse
%<*samplechap1>
%\fi
% Some text for chapter 1:
%    \begin{macrocode}
\section{one}
some text in chapter one
%    \end{macrocode}

%\iffalse
%</samplechap1>
%\fi
% Some text for chapter 2:
%\iffalse
%<*samplechap2>
%\fi
%    \begin{macrocode}
\section{two}
more text in chapter two
%    \end{macrocode}

%\iffalse
%</samplechap2>
%\fi
%
% %%%%%%%%%%%%%%%%%%%%%%%%%%%%%%%%%%%%%%
% \paragraph{Part Include Files.}
%
% The include files are called |cdocspt3.tex| and |cdocspt4.tex|.
%
%\iffalse
%<*samplepart3|samplepart4>
%\fi

% Optional override for |\version| flag:
%    \begin{macrocode}
%%\providecommand{\version}{final}
%    \end{macrocode}

% Include the main document:
%    \begin{macrocode}
% \iffalse
%
% childdoc.dtx Copyright (C) 2017-2018 Niklas Beisert
%
% This work may be distributed and/or modified under the
% conditions of the LaTeX Project Public License, either version 1.3
% of this license or (at your option) any later version.
% The latest version of this license is in
%   http://www.latex-project.org/lppl.txt
% and version 1.3 or later is part of all distributions of LaTeX
% version 2005/12/01 or later.
%
% This work has the LPPL maintenance status `maintained'.
%
% The Current Maintainer of this work is Niklas Beisert.
%
% This work consists of the files childdoc.dtx and childdoc.ins
% and the derived files childdoc.def and cdocsamp.tex with
% cdocsch1.tex, cdocsch2.tex, cdocsdrf.tex, cdocsfn1.tex, cdocsfn2.tex.
%
%<package>\ifdefined\childdocmain\endinput\fi
%<package>\ProvidesFile{childdoc.def}[2018/12/30 v2.0 child document driver]
%<samplemain>\ProvidesFile{cdocsamp.tex}[2018/12/30 v2.0 sample for childdoc]
%<*driver>
%\ProvidesFile{childdoc.drv}[2018/12/30 v2.0 childdoc reference manual file]
\PassOptionsToClass{10pt,a4paper}{article}
\documentclass{ltxdoc}

\usepackage[margin=35mm]{geometry}
\usepackage{hyperref}
\usepackage{hyperxmp}
\usepackage[usenames]{color}

\hypersetup{colorlinks=true}
\hypersetup{pdfstartview=FitH}
\hypersetup{pdfpagemode=UseNone}
\hypersetup{pdfsource={}}
\hypersetup{pdflang={en-UK}}
\hypersetup{pdfcopyright={Copyright 2017-2018 Niklas Beisert.
  This work may be distributed and/or modified under the
  conditions of the LaTeX Project Public License, either version 1.3
  of this license or (at your option) any later version.}}
\hypersetup{pdflicenseurl={http://www.latex-project.org/lppl.txt}}
\hypersetup{pdfcontactaddress={ETH Zurich, ITP, HIT K,
  Wolfgang-Pauli-Strasse 27}}
\hypersetup{pdfcontactpostcode={8093}}
\hypersetup{pdfcontactcity={Zurich}}
\hypersetup{pdfcontactcountry={Switzerland}}
\hypersetup{pdfcontactemail={nbeisert@itp.phys.ethz.ch}}
\hypersetup{pdfcontacturl={http://people.phys.ethz.ch/\xmptilde nbeisert/}}

\newcommand{\secref}[1]{\hyperref[#1]{section \ref*{#1}}}

\parskip1ex
\parindent0pt
\let\olditemize\itemize
\def\itemize{\olditemize\parskip0pt}

\begin{document}

\title{The \textsf{childdoc} Package}
\hypersetup{pdftitle={The childdoc Package}}
\author{Niklas Beisert\\[2ex]
  Institut f\"ur Theoretische Physik\\
  Eidgen\"ossische Technische Hochschule Z\"urich\\
  Wolfgang-Pauli-Strasse 27, 8093 Z\"urich, Switzerland\\[1ex]
  \href{mailto:nbeisert@itp.phys.ethz.ch}
  {\texttt{nbeisert@itp.phys.ethz.ch}}}
\hypersetup{pdfauthor={Niklas Beisert}}
\hypersetup{pdfsubject={Manual for the LaTeX2e Package childdoc}}
\date{30 December 2018, \textsf{v2.0}}
\maketitle

\begin{abstract}\noindent
\textsf{childdoc} is a \LaTeXe{} package
that enables the direct compilation
of document sections included by |\include|
to individual files.
\end{abstract}

\begingroup
\parskip0ex
\tableofcontents
\endgroup

%%%%%%%%%%%%%%%%%%%%%%%%%%%%%%%%%%%%%%%%%%%%%%%%%%%%%%%%%%%%%%%%%%%%%%%%%%%%%%%%
%%%%%%%%%%%%%%%%%%%%%%%%%%%%%%%%%%%%%%%%%%%%%%%%%%%%%%%%%%%%%%%%%%%%%%%%%%%%%%%%
\section{Introduction}

\LaTeX{} provides a mechanism to structure a large document (such as a book)
into a main file and several child files (containing the chapters)
using the |\include| command.
This mechanism is beneficial for documents
which span hundreds of pages in order to
make the source file(s) more manageable.
Moreover, compilation can be restricted to
selected child files by means of the |\includeonly| command.
The latter feature can be used to reduce the compilation time while editing
(this was significantly more useful in the earlier days of \LaTeX{})
or to generate a smaller document which is easier to navigate.
Another application of |\includeonly| is to generate
documents consisting of selected parts of the complete document.

However, there are a few drawbacks of the plain |\include| mechanism:
\begin{itemize}
\item
The child files cannot be compiled on their own,
they can only be compiled via the main file.
A naive editing environment
(such as a text editor with an option
to have the current file processed by \LaTeX)
may require one to switch to the main file before compiling;
attempting to compile the child file produces errors.
\item
The main file must be modified (each time)
to adjust the |\includeonly| command
to the present needs. This easily leaves the main file in a messy state.
\item
The generated document will always carry the filename
of the main document. This is inconvenient if
several child files are to be compiled and
to be kept for distribution.
\end{itemize}

The present package provides a simple interface
to make child files individually compilable by \LaTeX{}.
Compiling a child file then has the same effect as compiling
the main file with an |\includeonly| command
to select the appropriate child.
Moreover the generated document will carry the name of the child
rather than the main file.
This resolves all three above issues.

This feature is meant to make the editing of books,
thesis documents and lecture notes somewhat more convenient.
However, the package can also be used efficiently for
composing a series of documents (such as exercise sheets)
which are typically distributed individually.
It then assists the author in generating the individual documents
(potentially in different versions)
as well as a document containing the collected series.
Another application is in developing style files
or other kinds of included material
where compilation of the style file could redirect
to a sample or test file.

%%%%%%%%%%%%%%%%%%%%%%%%%%%%%%%%%%%%%%%%%%%%%%%%%%%%%%%%%%%%%%%%%%%%%%%%%%%%%%%%
%%%%%%%%%%%%%%%%%%%%%%%%%%%%%%%%%%%%%%%%%%%%%%%%%%%%%%%%%%%%%%%%%%%%%%%%%%%%%%%%
\section{Usage}

First of all, the package \textsf{childdoc} is \emph{not} a standard
\LaTeXe{} |.sty| style file! Therefore it needs to be invoked in
a non-standard way.

%%%%%%%%%%%%%%%%%%%%%%%%%%%%%%%%%%%%%%%%%%%%%%%%%%%%%%%%%%%%%%%%%%%%%%%%%%%%%%%%
\subsection{Included Files}
\label{sec:include}

%%%%%%%%%%%%%%%%%%%%%%%%%%%%%%%%%%%%%%%%
\DescribeMacro{\childdocmain}
To use the package, add the commands
\begin{center}
\begin{tabular}{l}
|\input{childdoc.def}|\\
|\childdocmain{}|\\
\end{tabular}
\end{center}
at the very top of the main \LaTeX{} file,
in particular \emph{before} the |\documentclass| statement!
The argument of |\childdocmain| should be left empty
(but it must be present).

%%%%%%%%%%%%%%%%%%%%%%%%%%%%%%%%%%%%%%%%
\DescribeMacro{\childdocof}
Furthermore, add the commands
\begin{center}
\begin{tabular}{l}
|\input{childdoc.def}|\\
|\childdocof{|\textit{main}|}|\\
\end{tabular}
\end{center}
at the top of every child file \textit{child}
which is included by |\include{|\textit{child}|}|
from within the main file
(or at least for those files to be compiled individually).
The argument \textit{main} must be the filename of the main file.

There are a couple of
considerations in setting up the main and child documents:

%%%%%%%%%%%%%%%%%%%%%%%%%%%%%%%%%%%%%%%%
\paragraph{Restrictions.}

Please note the following restrictions:
\begin{itemize}
\item
|\childdocmain| must be called with one argument \textit{main}
to ensure compatibility with earlier version of the package.
It must either be empty (|\childdocmain{}|)
or precisely match the filename of the main file in which it is specified.
See \secref{sec:detection} for further information.
\item
The filename \textit{main} must be specified without the |.tex| extension.
\item
The filename \textit{main} is case sensitive
(even in case-insensitive file systems)
due to internal string comparison.
\item
The argument \textit{main} should be fully expanded, it cannot be a macro.
\item
Subdirectories and special characters should be avoided in filenames.
\item
The command |\childdocmain{|\textit{main}|}| must be followed by a whitespace.
It should not be followed immediately by another command
or by a comment mark `|%|'.
This is because the \TeX{} parser reads the token immediately following
the argument of |\childdocmain| and puts it
at the beginning of every child section;
however, a white\-space is ignored.
\end{itemize}

%%%%%%%%%%%%%%%%%%%%%%%%%%%%%%%%%%%%%%%%
\paragraph{Content of Main File.}

It is advisable to place all content in the child files included by |\include|.
Any output contained in the main file will appear in all child documents
unless suppressed manually;
it cannot be suppressed automatically by the |\includeonly| directive
and thus should normally be avoided.
A method to include some content in the main file
by means of conditional processing is described in \secref{sec:conditional}.

%%%%%%%%%%%%%%%%%%%%%%%%%%%%%%%%%%%%%%%%
\paragraph{Page Numbering.}

When only a part of the document is compiled,
the appropriate numbering of pages
(as well as other status parameters)
is determined from the |.aux| files.
The latter contain information from previous passes.
However this information needs to propagate through
all intermediate child documents.
Therefore the page numbering in child documents may well
be inconsistent until the complete document is compiled at least once.

A useful (if unconventional) way to always ensure a consistent
page numbering is to restart the numbering in each child document
and denote the pages by `\textit{child}|.|\textit{page}'
where \textit{child} represents the chapter/section number of the child file.
This can be achieved by the command
|\numberwithin{page}{|\textit{child}|}|
of the \textsf{amsmath} package
where \textit{child} can be |chapter| or |section|
depending on the chosen structuring.
Alternatively, one can modify the macro |\thepage| appropriately
and reset the counter |page| at the start of each child file.

%%%%%%%%%%%%%%%%%%%%%%%%%%%%%%%%%%%%%%%%%%%%%%%%%%%%%%%%%%%%%%%%%%%%%%%%%%%%%%%%
\subsection{Conditional Processing}
\label{sec:conditional}

The package provides a mechanism to compile different versions
of a document. To customise the versions further some conditional processing
can come in handy to distinguish which version is being compiled.
The package provides two macros to describe the compilation context:

%%%%%%%%%%%%%%%%%%%%%%%%%%%%%%%%%%%%%%%%
\DescribeMacro{\ifchilddoc}
The conditional |\ifchilddoc| distinguishes between the compilation of
child documents and the main document:
%
\begin{center}
|\ifchilddoc |\textit{child-code}| |[|\||else |\textit{main-code}]| \||fi|
\end{center}

%%%%%%%%%%%%%%%%%%%%%%%%%%%%%%%%%%%%%%%%
\DescribeMacro{\childdocname}
\DescribeMacro{\childdocjob}
The macro |\childdocname| contains the filename (without extension)
of the main or child file being processed.
Note that |\childdocjob| will always contain the name of the main file.

%%%%%%%%%%%%%%%%%%%%%%%%%%%%%%%%%%%%%%%%
\paragraph{Title Page.}

Conditional processing can be used to include a title or banner page
in the main document when proper precautions are taken.
Importantly, the code in the main file should ensure that the page counter
(as well as other status parameters which are stored in the |.aux| files)
takes the same value after the conditional processing.
Otherwise the page numbers may take divergent values
depending on which part is compiled.

For example, a title page could be declared by:
%
\begin{center}
\begin{tabular}{l}
|\ifchilddoc\||else|\\
|\addtocounter{page}{-1}|\\
\textit{code for title page}\\
|\newpage|\\
|\||fi|
\end{tabular}
\end{center}
%
A banner page for the child documents can be generated by:
%
\begin{center}
\begin{tabular}{l}
|\ifchilddoc|\\
|\addtocounter{page}{-1}|\\
\textit{code for banner page}\\
|\newpage|\\
|\||fi|
\end{tabular}
\end{center}
%
Here one could write a message such as:
\begin{center}
|This is the part \childdocname{} of \childdocjob{}.|
\end{center}

%%%%%%%%%%%%%%%%%%%%%%%%%%%%%%%%%%%%%%%%%%%%%%%%%%%%%%%%%%%%%%%%%%%%%%%%%%%%%%%%
\subsection{Flags}
\label{sec:flags}

The package makes it easy to generate different versions
of the main or child documents.
To this end compilation flags can be defined
and assigned different default values.
They will be particularly useful in conjunction
with the forwarding mechanism described in \secref{sec:forward}.

For example, it may be useful to have a flag |\version|
which can be set to |draft| or |final|.
The document source will contain some conditional code
depending on the value of |\version|.
Suppose further, the flag should default to |final| for the main file
and to |draft| for child files
which is a natural assignment for editing the document.
This is achieved by placing the following code
in the preamble of the main document
(below the |\childdocmain| directive):
%
\begin{center}
\begin{tabular}{l}
|\ifchilddoc|\\
|\providecommand{\version}{draft}|\\
|\||else|\\
|\providecommand{\version}{final}|\\
|\||fi|
\end{tabular}
\end{center}
%
The definition by |\providecommand| makes sure
that previous definitions are not overwritten.
Further statements |\providecommand{\version}{...}|
can thus be added before the above code to override it.

For the main file, one might add a line
(between |\childdocmain| and the above block)
%
\begin{center}
|%\ifchilddoc\||else\providecommand{\version}{draft}\||fi|
\end{center}
%
which can be uncommented to produce a draft version.
Likewise one can add a line to the very top of a child file
(above the |\childdocof{|\textit{main}|}| directive)
%
\begin{center}
|%\providecommand{\version}{final}|
\end{center}
%
which can be uncommented to produce the final version of this child document.

%%%%%%%%%%%%%%%%%%%%%%%%%%%%%%%%%%%%%%%%%%%%%%%%%%%%%%%%%%%%%%%%%%%%%%%%%%%%%%%%
\subsection{Forwarding}
\label{sec:forward}

Different versions of the main or child documents
using compilation flags as described in \secref{sec:flags}
can be (permanently) stored in different files
for convenient compilation, viewing and distribution.
To this end, the package defines a command
to pass on compilation to a different file:

%%%%%%%%%%%%%%%%%%%%%%%%%%%%%%%%%%%%%%%%
\DescribeMacro{\childdocforward}
The command |\childdocforward| redirects processing to
another source file:
%
\begin{center}
\begin{tabular}{l}
|\input{childdoc.def}|\\
|\childdocforward[|\textit{main}|]{|\textit{dest}|}|\\
\end{tabular}
\end{center}
%
The argument \textit{dest} is the destination file
(without extension).
It should be the main file or one of the child files.
Note that further \textsf{childdoc} directives
such as |\childdocof| and |\childdocforward|
in the indicated file will be processed in this form.
The optional argument \textit{main}
passes on directly to the main file \textit{main}
while pretending to compile the child \textit{dest}.
This form behaves as if \textit{dest}
issues |\childdocof{|\textit{main}|}| right away,
and no further \textsf{childdoc} directives will be processed.

%%%%%%%%%%%%%%%%%%%%%%%%%%%%%%%%%%%%%%%%
\DescribeMacro{\...prefix}
In the alternative form |\childdocforwardprefix|,
%
\begin{center}
\begin{tabular}{l}
|\input{childdoc.def}|\\
|\childdocforwardprefix[|\textit{main}|]{|\textit{prefix}|}{|\textit{dest}|}|
\end{tabular}
\end{center}
%
the destination file is determined by a pattern
depending on the current file:
To make this work, the current file must be called
`{\textit{prefix}\hspace{0.2em}\textit{suffix}}'
with \textit{prefix} matching precisely the argument.
Processing is then passed on to the file
`{\textit{dest}\hspace{0.2em}\textit{suffix}}'.
Surely, the same effect is achieved by
directly specifying the
argument `{\textit{dest}\hspace{0.2em}\textit{suffix}}'
in the first form.
However, that requires to set up a different file
for each child. With the alternative form of the command
all these files can have exactly the same content
which simplifies setting them up and maintaining them.

For example, the following file |draft.tex|
with a compilation flag |\version| as described in \secref{sec:flags}
compiles the main document as a draft:
%
\begin{center}
\begin{tabular}{l}
|\def\version{draft}|\\
|\input{childdoc.def}|\\
|\childdocforward{|\textit{main}|}|
\end{tabular}
\end{center}
%
Likewise, the following files |final|\textit{nn}|.tex|
compile the final version of the child document
|child|\textit{nn}|.tex|:
%
\begin{center}
\begin{tabular}{l}
|\def\version{final}|\\
|\input{childdoc.def}|\\
|\childdocforwardprefix{final}{child}|
\end{tabular}
\end{center}
%

Note that when several versions of a main file and/or of each child file
are to be generated, it may be convenient to set up a |Makefile| or
shell script to automatise the process.

%%%%%%%%%%%%%%%%%%%%%%%%%%%%%%%%%%%%%%%%%%%%%%%%%%%%%%%%%%%%%%%%%%%%%%%%%%%%%%%%
\subsection{Command Line Processing}
\label{sec:commandline}

The effect of redirection files can also be achieved by invoking
the \LaTeX{} compiler with a more elaborate command line.
Most conveniently this should be done as part
of a shell script or a |Makefile|.

When using \textsf{childdoc} in the main file, the following
command lines effectively perform a redirection
(note that depending on the shell being used,
backslashes may have to be doubled: `|\|' $\to$ `|\\|'):
%
\begin{center}
|... -jobname "|\textit{target}|" |\\|"|[\textit{flags}]%
|\input{childdoc.def}\childdocforward[|\textit{main}|]{|\textit{dest}|}"|
\end{center}
%
Here \textit{target} is the name of the output file,
\textit{main} is the name of the main file
and \textit{dest} is the name of the main or child file to be processed
(all filenames without extensions).
The optional argument \textit{main} can be omitted
if \textit{main} matches \textit{dest}.
Optionally, compilation \textit{flags} can be defined via |\def| commands.
This command line makes the \TeX{} engine believe
it is compiling the file \textit{target}
whose content is specified as the latter parameter.
The provided code then forwards the processing to
\textit{main} or \textit{dest} as described in \secref{sec:forward}.

%%%%%%%%%%%%%%%%%%%%%%%%%%%%%%%%%%%%%%%%%%%%%%%%%%%%%%%%%%%%%%%%%%%%%%%%%%%%%%%%
\subsection{Include by Input}
\label{sec:input}

Including child documents by |\include| has some restrictions by design.
Most notably, the content of a child document always occupies
its own set of pages; pages cannot be shared between child documents.
Usually, this behaviour makes perfect sense
because each child document contain an essential part of the document.
However, in some situations it may be desirable to compose
a document from a collection of parts
without having mandatory page breaks between then.
For this case, the package
provides a mechanism to include parts
by |\input| which can also be processed individually.
However, by construction this mechanism
requires manual handling of the content to be output.

%%%%%%%%%%%%%%%%%%%%%%%%%%%%%%%%%%%%%%%%
\DescribeMacro{\ifchilddocmanual}
The main file should be prepared as usual, see \secref{sec:include}.
However, the document body must make a distinction
between processing of an individual part and of the main document, e.g.:
%
\begin{center}
\begin{tabular}{l}
|\ifchilddocmanual|\\
|\input{\childdocname}|\\
|\||else|\\
\textit{document body with }|\input{|\textit{part}|}|\\
|\||fi|
\end{tabular}
\end{center}
%
The conditional |\ifchilddocmanual| is true whenever
a part to be included by |\input| is being compiled,
and the name of the part is stored in |\childdocname|.

%%%%%%%%%%%%%%%%%%%%%%%%%%%%%%%%%%%%%%%%
\DescribeMacro{\childdocby}
Each part to be included by |\input| should start with:
%
\begin{center}
\begin{tabular}{l}
|\input{childdoc.def}|\\
|\childdocby{|\textit{main}|}|\\
\end{tabular}
\end{center}
%
The directive |\childdocby| is similar to |\childdocof|
described in \secref{sec:include},
but the subsequent selection of content must be done manually.
To that end, both |\ifchilddoc| and |\ifchilddocmanual|
will be true upon processing of a part,
and the name of the part is stored in |\childdocname|.
Note that |\jobname| will be set to the filename of the current part
so that each part receives an individual |.aux| file
that does not interfere with the |.aux| file(s) of the main document.
This behaviour can be altered by the alternative form
|\childdocby[*]{|\textit{main}|}| (with a non-empty optional argument)
which uses the |.aux| file of the main document
by setting |\jobname| to \textit{main}.

%%%%%%%%%%%%%%%%%%%%%%%%%%%%%%%%%%%%%%%%%%%%%%%%%%%%%%%%%%%%%%%%%%%%%%%%%%%%%%%%
\subsection{Driver Development}
\label{sec:driver}

The \textsf{childdoc} mechanism can also be use for the development
of definition files such as \LaTeX{} styles or classes.
This case differs from the above setup with multiple parts
included by |\include| in that no |\includeonly| should be invoked.
This can be achieved by starting the include file
(before |\ProvidesPackage|) with:
%
\begin{center}
\begin{tabular}{l}
|\input{childdoc.def}|\\
|\childdocforward{|\textit{main}|}|\\
\end{tabular}
\end{center}
%
or alternatively with:
%
\begin{center}
\begin{tabular}{l}
|\input{childdoc.def}|\\
|\childdocby{|\textit{main}|}|\\
\end{tabular}
\end{center}
%
Both forms have slightly different effects as described above.
The main file is prepared as usual, see \secref{sec:include}.

%%%%%%%%%%%%%%%%%%%%%%%%%%%%%%%%%%%%%%%%%%%%%%%%%%%%%%%%%%%%%%%%%%%%%%%%%%%%%%%%
\subsection{Legacy Detection}
\label{sec:detection}

The directive |\childdocmain| in the main file can detect
whether the complete document or merely a child is to be compiled
even without using the directive |\childdocof|.
This method is deprecated because it is less robust
and there is no compelling reason to use it;
it is merely provided for backward compatibility
and it may be removed in future versions.

If the detection mechanism is to be used,
it is mandatory to correctly specify
the filename of the main file as the argument of |\childdocmain|:
%
\begin{center}
\begin{tabular}{l}
|\input{childdoc.def}|\\
|\childdocmain{|\textit{main}|}|\\
\end{tabular}
\end{center}
%
If |\jobname| does not match the argument \textit{main} of |\childdocmain|,
it is assumed that |\jobname| points to the child file to be compiled.
When using |\childdocmain| with the main file specified as argument,
it suffices to start a child file
with just |\input{|\textit{main}|}|
without loading of the package and using |\childdocof|.
If instead all processing is done
with the appropriate \textsf{childdoc} directives,
the argument of \textit{main} of |\childdocmain| can be empty.

An alternative version of the command line processing described
in \secref{sec:commandline} using the detection mechanism reads:
%
\begin{center}
|... -jobname "|\textit{target}|" "|[\textit{flags}]%
[|\def\jobname{|\textit{dest}|}|]|\input{|\textit{main}|}"|
\end{center}

%%%%%%%%%%%%%%%%%%%%%%%%%%%%%%%%%%%%%%%%%%%%%%%%%%%%%%%%%%%%%%%%%%%%%%%%%%%%%%%%
\subsection{Manual Code}
\label{sec:manual}

In case one cannot be certain whether the definitions file |childdoc.def|
is installed on the target \TeX{} distribution
and one prefers not to ship it,
it is conceivable to paste a few relevant commands into the sources.

To that end, drop all statements |\input{childdoc.def}|
and perform the replacements as outlined below.
Instead of |\childdocmain{|\textit{main}|}| add the following code
to the top of the main file:
%
\begin{center}
\begin{tabular}{l}
|\||ifdefined\childdocname\endinput\||fi\newif\ifchilddoc|\\
|\edef\childdocname{\scantokens\expandafter{\jobname\noexpand}}|\\
|\def\childdocmain{|\textit{main}|}\||ifx\childdocmain\childdocname\||else|\\
|\childdoctrue\includeonly{\childdocname}\let\jobname\childdocmain\||fi|\\
\end{tabular}
\end{center}
%
Instead of |\childdocof{|\textit{main}|}| just include the main file
at the top of each child file:
%
\begin{center}
|\input{|\textit{main}|}|
\end{center}
%
A simple redirection |\childdocforward{|\textit{dest}|}| is achieved by:
%
\begin{center}
|\def\jobname{|\textit{dest}|}\input{\jobname}|
\end{center}
%
The redirection with prefix
|\childdocforwardprefix[|\textit{prefix}|]{|\textit{dest}|}|
is accomplished by:
%
\begin{center}
\begin{tabular}{l}
|{\edef\jobname{\scantokens\expandafter{\jobname\noexpand}}|\\
|\def\redirectjob |\textit{prefix}|#1~~~{\gdef\jobname{|\textit{dest}|#1}}|\\
|\expandafter\redirectjob\jobname~~~}\input{\jobname}|
\end{tabular}
\end{center}

In an alternative approach,
child documents can be compiled by a specific command line
without additional code or specific definitions:
%
\begin{center}
|... -jobname "|\textit{target}|" "|[\textit{flags}]%
|\includeonly{|\textit{dest}|}\input{|\textit{main}|}"|
\end{center}
%

%%%%%%%%%%%%%%%%%%%%%%%%%%%%%%%%%%%%%%%%%%%%%%%%%%%%%%%%%%%%%%%%%%%%%%%%%%%%%%%%
%%%%%%%%%%%%%%%%%%%%%%%%%%%%%%%%%%%%%%%%%%%%%%%%%%%%%%%%%%%%%%%%%%%%%%%%%%%%%%%%
\section{Information}

%%%%%%%%%%%%%%%%%%%%%%%%%%%%%%%%%%%%%%%%%%%%%%%%%%%%%%%%%%%%%%%%%%%%%%%%%%%%%%%%
\subsection{Copyright}

Copyright \copyright{} 2017--2018 Niklas Beisert

This work may be distributed and/or modified under the
conditions of the \LaTeX{} Project Public License, either version 1.3
of this license or (at your option) any later version.
The latest version of this license is in
  \url{http://www.latex-project.org/lppl.txt}
and version 1.3 or later is part of all distributions of \LaTeX{}
version 2005/12/01 or later.

This work has the LPPL maintenance status `maintained'.

The Current Maintainer of this work is Niklas Beisert.

This work consists of the files |README.txt|, |childdoc.ins| and |childdoc.dtx|
as well as the derived files |childdoc.def|, |cdocsamp.tex|
with |cdocsch1.tex|, |cdocsch2.tex|, |cdocspt3.tex|, |cdocspt4.tex|,
|cdocsdrf.tex|, |cdocsfn1.tex|, |cdocsfn2.tex|
as well as |childdoc.pdf|.

%%%%%%%%%%%%%%%%%%%%%%%%%%%%%%%%%%%%%%%%%%%%%%%%%%%%%%%%%%%%%%%%%%%%%%%%%%%%%%%%
\subsection{Files and Installation}

The package consists of the files:
%
\begin{center}
\begin{tabular}{ll}
    |README.txt|   & readme file \\
    |childdoc.ins| & installation file \\
    |childdoc.dtx| & source file \\
    |childdoc.def| & definition file \\
    |cdocsamp.tex| & sample main file \\
    |cdocsch1.tex| & sample include file \\
    |cdocsch2.tex| & sample include file \\
    |cdocspt3.tex| & sample part file \\
    |cdocspt4.tex| & sample part file \\
    |cdocsdrf.tex| & sample redirection file \\
    |cdocsfn1.tex| & sample redirection file \\
    |cdocsfn2.tex| & sample redirection file \\
    |childdoc.pdf| & manual
\end{tabular}
\end{center}
%
The distribution consists of the files
|README.txt|, |childdoc.ins| and |childdoc.dtx|.
%
\begin{itemize}
\item
Run (pdf)\LaTeX{} on |childdoc.dtx|
to compile the manual |childdoc.pdf| (this file).
\item
Run \LaTeX{} on |childdoc.ins| to create the definitions file |childdoc.def|
and the sample |cdocsamp.tex| with include files
|cdocsch1.tex|, |cdocsch2.tex|, |cdocspt3.tex|, |cdocspt4.tex|,
|cdocsdrf.tex|, |cdocsfn1.tex|, |cdocsfn2.tex|.
Then copy the file |childdoc.def| to an appropriate directory of your \LaTeX{}
distribution, e.g.\ \textit{texmf-root}|/tex/latex/childdoc|.
\end{itemize}

%%%%%%%%%%%%%%%%%%%%%%%%%%%%%%%%%%%%%%%%%%%%%%%%%%%%%%%%%%%%%%%%%%%%%%%%%%%%%%%%
\subsection{Related CTAN Packages}

There are several other packages which offer a similar functionality:
%
\begin{itemize}
\item
The packages
\href{http://ctan.org/pkg/docmute}{\textsf{docmute}},
\href{http://ctan.org/pkg/includex}{\textsf{includex}} and
\href{http://ctan.org/pkg/standalone}{\textsf{standalone}}
provide commands to include only the document body of
a child file thus allowing both files to be compiled individually.
\item
The packages \href{http://ctan.org/pkg/subdocs}{\textsf{subdocs}}
and \href{http://ctan.org/pkg/subfiles}{\textsf{subfiles}}
provide structures in which the main and child documents can be
encapsulated and allowing them to be compiled individually.
The inclusion mechanism is different from the conventional |\include|.
\item
The package \href{http://ctan.org/pkg/combine}{\textsf{combine}}
is an elaborate solution to combine several documents into one.
\end{itemize}
%
See also the CTAN topic \href{http://ctan.org/topic/subdocs}{\textsf{subdocs}}
for further related packages.
The present package differs from the above solutions in that
a document structure constructed with the conventional |\include| mechanism
just needs two extra commands at the top of every file
such that all constituent files can be compiled individually.

%%%%%%%%%%%%%%%%%%%%%%%%%%%%%%%%%%%%%%%%%%%%%%%%%%%%%%%%%%%%%%%%%%%%%%%%%%%%%%%%
%\subsection{Feature Suggestions}
%
%The following is a list of features which may be useful for future
%versions of this package:
%%
%\begin{itemize}
%\item
%\ldots
%\end{itemize}

%%%%%%%%%%%%%%%%%%%%%%%%%%%%%%%%%%%%%%%%%%%%%%%%%%%%%%%%%%%%%%%%%%%%%%%%%%%%%%%%
\subsection{Revision History}

%%%%%%%%%%%%%%%%%%%%%%%%%%%%%%%%%%%%%%%%
\paragraph{v2.0:} 2018/12/30

\begin{itemize}
\item
immediate forward processing
\item
added |\childdocby| mechanism
\item
manual restructured
\end{itemize}

%%%%%%%%%%%%%%%%%%%%%%%%%%%%%%%%%%%%%%%%
\paragraph{v1.6:} 2018/01/17

\begin{itemize}
\item
application for development of include files
\item
corrections to manual
\end{itemize}

%%%%%%%%%%%%%%%%%%%%%%%%%%%%%%%%%%%%%%%%
\paragraph{v1.5:} 2017/05/21

\begin{itemize}
\item
more complete structuring introduced
\item
|\childdocof| introduced
\item
|\childdoc| renamed to |\childdocmain|
\item
|\childredirect| renamed to |\childdocforward| and |\childdocforwardprefix|
and functionality expanded
\end{itemize}

%%%%%%%%%%%%%%%%%%%%%%%%%%%%%%%%%%%%%%%%
\paragraph{v1.0:} 2017/04/27

\begin{itemize}
\item
manual and install package
\item
first version published on CTAN
\end{itemize}

%%%%%%%%%%%%%%%%%%%%%%%%%%%%%%%%%%%%%%%%
\paragraph{v0.6:} 2017/04/26

\begin{itemize}
\item
redirection mechanism added
\end{itemize}

%%%%%%%%%%%%%%%%%%%%%%%%%%%%%%%%%%%%%%%%
\paragraph{v0.5:} 2017/04/26

\begin{itemize}
\item
functionality in definition file
\end{itemize}


%%%%%%%%%%%%%%%%%%%%%%%%%%%%%%%%%%%%%%%%%%%%%%%%%%%%%%%%%%%%%%%%%%%%%%%%%%%%%%%%
%%%%%%%%%%%%%%%%%%%%%%%%%%%%%%%%%%%%%%%%%%%%%%%%%%%%%%%%%%%%%%%%%%%%%%%%%%%%%%%%
%%%%%%%%%%%%%%%%%%%%%%%%%%%%%%%%%%%%%%%%%%%%%%%%%%%%%%%%%%%%%%%%%%%%%%%%%%%%%%%%
\appendix

\settowidth\MacroIndent{\rmfamily\scriptsize 000\ }

 \DocInput{childdoc.dtx}

\end{document}
%</driver>
% \fi
%
% %%%%%%%%%%%%%%%%%%%%%%%%%%%%%%%%%%%%%%%%%%%%%%%%%%%%%%%%%%%%%%%%%%%%%%%%%%%%%%
% %%%%%%%%%%%%%%%%%%%%%%%%%%%%%%%%%%%%%%%%%%%%%%%%%%%%%%%%%%%%%%%%%%%%%%%%%%%%%%
% \section{Sample}
%\iffalse
%<*samplemain>
%\fi
%
% The following presents a sample document
% with two chapters, two parts, a title page,
% a compile flag as well as three forwarding files to set the flag.
% It consists of eight |.tex| files:
% \begin{center}
% \begin{tabular}{ll}
% |cdocsamp.tex|&main file\\
% |cdocsch1.tex|&include file for chapter 1\\
% |cdocsch2.tex|&include file for chapter 2\\
% |cdocspt3.tex|&include file for part 3\\
% |cdocspt4.tex|&include file for part 4\\
% |cdocsdrf.tex|&forwarding file for main file in draft mode\\
% |cdocsfi1.tex|&forwarding file for final version of chapter 1\\
% |cdocsfi2.tex|&forwarding file for final version of chapter 2\\
% \end{tabular}
% \end{center}
% Each of the eight files can be compiled directly by the \LaTeX{} compiler.
%
% %%%%%%%%%%%%%%%%%%%%%%%%%%%%%%%%%%%%%%
% \paragraph{Main File.}
%
% The main file is called |cdocsamp.tex|.
%
% Load the \textsf{childdoc} definitions and
% declare the filename for the main document:
%    \begin{macrocode}
\input{childdoc.def}
\childdocmain{}
%    \end{macrocode}

% Optional override for |\version| flag:
%    \begin{macrocode}
%%\ifchilddoc\else\providecommand{\version}{draft}\fi
%    \end{macrocode}

% Define the default values for the |\version| flag
% (|final| for the main file and |draft| for childs):
%    \begin{macrocode}
\ifchilddoc
\providecommand{\version}{draft}
\else
\providecommand{\version}{final}
\fi
%    \end{macrocode}

% Load the standard document class:
%    \begin{macrocode}
\documentclass[12pt]{article}
%    \end{macrocode}

% Start the document body:
%    \begin{macrocode}
\begin{document}
%    \end{macrocode}

% Declare a title page.
% Print title, part of document being processed and version flag:
%    \begin{macrocode}
\addtocounter{page}{-1}
\begin{center}
{\LARGE\bfseries{}childdoc example\par}
\vspace{1cm}
\ifchilddoc
\ifchilddocmanual part\else chapter\fi:
`\childdocname' of `\childdocjob'\par
\else
main document: `\childdocjob'\par
\fi
version: \version\par
\end{center}
\newpage
%    \end{macrocode}

% Manually include selected file,
% otherwise process as usual:
%    \begin{macrocode}
\ifchilddocmanual
\section*{part `\childdocname'}
\input{\childdocname}
\else
%    \end{macrocode}

% Include the two chapters:
%    \begin{macrocode}
\include{cdocsch1}
\include{cdocsch2}
%    \end{macrocode}

% Include the two parts unless only chapters should be displayed:
%    \begin{macrocode}
\ifchilddoc\else
\section{part three}
\input{cdocspt3}
\section{part four}
\input{cdocspt4}
\fi
%    \end{macrocode}

% Process as usual until here:
%    \begin{macrocode}
\fi
%    \end{macrocode}

% End of document body:
%    \begin{macrocode}
\end{document}
%    \end{macrocode}
%\iffalse
%</samplemain>
%\fi
%
% %%%%%%%%%%%%%%%%%%%%%%%%%%%%%%%%%%%%%%
% \paragraph{Chapter Include Files.}
%
% The include files are called |cdocsch1.tex| and |cdocsch2.tex|.
%
%\iffalse
%<*samplechap1|samplechap2>
%\fi

% Optional override for |\version| flag:
%    \begin{macrocode}
%%\providecommand{\version}{final}
%    \end{macrocode}

% Include the main document:
%    \begin{macrocode}
\input{childdoc.def}
\childdocof{cdocsamp}
%    \end{macrocode}

%\iffalse
%</samplechap1|samplechap2>
%\fi
%
%\iffalse
%<*samplechap1>
%\fi
% Some text for chapter 1:
%    \begin{macrocode}
\section{one}
some text in chapter one
%    \end{macrocode}

%\iffalse
%</samplechap1>
%\fi
% Some text for chapter 2:
%\iffalse
%<*samplechap2>
%\fi
%    \begin{macrocode}
\section{two}
more text in chapter two
%    \end{macrocode}

%\iffalse
%</samplechap2>
%\fi
%
% %%%%%%%%%%%%%%%%%%%%%%%%%%%%%%%%%%%%%%
% \paragraph{Part Include Files.}
%
% The include files are called |cdocspt3.tex| and |cdocspt4.tex|.
%
%\iffalse
%<*samplepart3|samplepart4>
%\fi

% Optional override for |\version| flag:
%    \begin{macrocode}
%%\providecommand{\version}{final}
%    \end{macrocode}

% Include the main document:
%    \begin{macrocode}
\input{childdoc.def}
\childdocby{cdocsamp}
%    \end{macrocode}

%\iffalse
%</samplepart3|samplepart4>
%\fi
%
%\iffalse
%<*samplepart3>
%\fi
% Some text for part 3:
%    \begin{macrocode}
some text in part three
%    \end{macrocode}

%\iffalse
%</samplepart3>
%\fi
% Some text for part 4:
%\iffalse
%<*samplepart4>
%\fi
%    \begin{macrocode}
more text in part four
%    \end{macrocode}

%\iffalse
%</samplepart4>
%\fi
%
% %%%%%%%%%%%%%%%%%%%%%%%%%%%%%%%%%%%%%%
% \paragraph{Forwarding for a Complete Draft.}
%
% The following forwarding file |cdocsdrf.tex|
% compiles the main document in draft mode:
%\iffalse
%<*sampledraft>
%\fi
%    \begin{macrocode}
\def\version{draft}
\input{childdoc.def}
\childdocforward{cdocsamp}
%    \end{macrocode}

%\iffalse
%</sampledraft>
%\fi
%
% %%%%%%%%%%%%%%%%%%%%%%%%%%%%%%%%%%%%%%
% \paragraph{Forwarding for Final Version of the Chapters.}
%
% The following forwarding files |cdocsfn1.tex| and |cdocsfn2.tex|
% (with identical content)
% compile the final versions of the child documents
% |cdocsch1.tex| and |cdocsch2.tex|, respectively:
%\iffalse
%<*samplefinal>
%\fi
%    \begin{macrocode}
\def\version{final}
\input{childdoc.def}
\childdocforwardprefix[cdocsamp]{cdocsfn}{cdocsch}
%    \end{macrocode}

%\iffalse
%</samplefinal>
%\fi
%
% %%%%%%%%%%%%%%%%%%%%%%%%%%%%%%%%%%%%%%
% \paragraph{Command Line Processing.}
%
% The following three command lines generate the output files
% |cdocscld|, |cdocscl1| and |cdocscl2|
% which should be identical to
% |cdocsdrf|, |cdocsch1| and |cdocsfn2|, respectively:
% \begin{center}
% \begin{tabular}{l}
% |latex -jobname cdocscld \|\\
% |  "\def\version{draft}\input{childdoc.def}\childdocforward{cdocsamp}"|\\
% |latex -jobname cdocscl1 \|\\
% |  "\input{childdoc.def}\childdocforward[cdocsamp]{cdocsch1}"|\\
% |latex -jobname cdocscl2 \|\\
% |  "\def\version{final}\input{childdoc.def}\childdocforward{cdocsch2}"|
% \end{tabular}
% \end{center}
% Note that the trailing backslash on each first line
% merely continues the input to the second line
% (for convenient cut ant paste).
% Furthermore, the command |latex| can be replaced by any
% of its alternative versions such as |pdflatex|.
%
% %%%%%%%%%%%%%%%%%%%%%%%%%%%%%%%%%%%%%%%%%%%%%%%%%%%%%%%%%%%%%%%%%%%%%%%%%%%%%%
% %%%%%%%%%%%%%%%%%%%%%%%%%%%%%%%%%%%%%%%%%%%%%%%%%%%%%%%%%%%%%%%%%%%%%%%%%%%%%%
% \section{Implementation}
%\iffalse
%<*package>
%\fi
%
% This section describes the definitions file |childdoc.def|.

% The definitions cannot be loaded using |\usepackage| or |\RequirePackage|
% which has a mechanism to prevent loading a style file more than once.
% When loading the definitions by means of |\input|
% multiple instances have to be prevented manually:
%\iffalse
%This code needs to be before the `\ProvidesFile' directive
%which is defined at the beginning of this file.
%Therefore it is also placed there and commented out here.
%</package>
%<*discard>
%\fi
%    \begin{macrocode}
\ifdefined\childdocmain\endinput\fi
%    \end{macrocode}
%\iffalse
%</discard>
%<*package>
%\fi
%
% \macro{\ifchilddoc}
% \macro{\ifchilddocmanual}
% The conditional |\ifchilddoc| tells whether a
% child (true) or main (false) document is being compiled.
% The conditional |\ifchilddocmanual| tells whether
% the |\includeonly| mechanism is used (false) or
% the selection of child files must be performed manually (true).
% The definitions initialise to false:
%    \begin{macrocode}
\newif\ifchilddoc
\newif\ifchilddocmanual
%    \end{macrocode}

% \macro{\childdocname}
% \macro{\childdocjob}
% The macro |\childdocname| stores the name of the main document
% to be compiled. The macro |\childdocjob| stores the name of
% the document on which the \LaTeX{} compiler was originally invoked.
% The content of |\jobname| cannot be compared
% to filenames specified in the source due to different catcodes.
% The following code rescans |\jobname|, stores the result
% in |\childdocname| and saves a copy in |\childdocjob|:
%    \begin{macrocode}
\edef\childdocname{\scantokens\expandafter{\jobname\noexpand}}
\let\childdocjob\childdocname
%    \end{macrocode}

% \macro{\childdocdisable}
% The macro |\childdocdisable| prevents the main file
% from being processed more than once.
% At this stage, the main document command |\childdocmain|
% is assumed to be called once again where it should do nothing.
% Any subsequent call to it should prevent
% a secondary processing of the main document
% It overwrites the forwarding commands
% |\childdocof| and |\childdocforward|
% with empty macros to prevent further inclusions of the main document:
%    \begin{macrocode}
\newcommand{\childdocdisable}
{
  \renewcommand{\childdocmain}[1]{\renewcommand{\childdocmain}[1]{\endinput}}
  \renewcommand{\childdocof}[1]{}
  \renewcommand{\childdocby}[2][]{}
  \renewcommand{\childdocforward}[2][]{}
  \renewcommand{\childdocdisable}{}
}
%    \end{macrocode}

% \macro{\childdocmain}
% The macro |\childdocmain| is to be called at the top of the main file
% with nothing or the main filename (without extension) as argument.
% First, it breaks loops.
% If the argument is not empty and does not match |\childdocname|
% (which is set by the first inclusion of |childdoc.def|),
% |\ifchilddoc| is set to true, |\includeonly| is applied to the child file
% and |\jobname| is set to the main file
% (for proper handling of |.aux| files):
%    \begin{macrocode}
\newcommand{\childdocmain}[1]
{
  \childdocdisable\childdocmain{}
  \if?#1?\else
    \begingroup
      \def\childdoctmp{#1}
      \ifx\childdoctmp\childdocname
        \def\childdoctmp{}
      \else
        \def\childdoctmp
        {
          \childdoctrue
          \includeonly{\childdocname}
          \def\childdocjob{#1}
          \def\jobname{#1}
        }
      \fi
      \expandafter
    \endgroup
    \childdoctmp
  \fi
}
%    \end{macrocode}

% \macro{\childdocof}
% The command |\childdocof| redirects
% compilation to the main file |#1|.
%    \begin{macrocode}
\newcommand{\childdocof}[1]
{
  \childdocdisable
  \childdoctrue
  \includeonly{\childdocname}
  \def\jobname{#1}
  \def\childdocjob{#1}
  \input{#1}
}
%    \end{macrocode}

% \macro{\childdocby}
% The command |\childdocby| ....
%    \begin{macrocode}
\newcommand{\childdocby}[2][]
{
  \childdocdisable
  \childdoctrue
  \childdocmanualtrue
  \if?#1?\else
    \def\jobname{#2}
  \fi
  \def\childdocjob{#2}
  \input{#2}
  \endinput
}
%    \end{macrocode}

% \macro{\childdocforward}
% The command |\childdocforward| redirects
% compilation to the main file or
% (if the optional argument is given) a child file.
% Parameters are set as if the main file
% or a child file starting with |\childdocof| was compiled.
% Then compilation is handed over to the main file:
%    \begin{macrocode}
\newcommand{\childdocforward}[2][]
{
  \begingroup
    \if?#1?
      \def\childdoctmp
      {
        \def\childdocname{#2}
        \def\childdocjob{#2}
        \def\jobname{#2}
        \input{#2}
        \endinput
      }
    \else
      \def\childdoctmp
      {
        \childdocdisable
        \def\childdocname{#2}
        \childdoctrue
        \includeonly{#2}
        \def\childdocjob{#1}
        \def\jobname{#1}
        \input{#1}
        \endinput
      }
    \fi
    \expandafter
  \endgroup
  \childdoctmp
}
%    \end{macrocode}

% \macro{\childdocforwardprefix}
% The command |\childdocforwardprefix| redirects
% compilation to the main or a child file by means of a pattern.
% The prefix |#1| in the current filename is replaced by |#2|
% and the suffix of the current filename is kept
% (it is assumed that the filename does not contain the substring `|~~~|'
% which is used as a delimiter).
% Compilation is handed over to the new file by |\childdocforward|:
%    \begin{macrocode}
\newcommand{\childdocforwardprefix}[3][]
{
  \begingroup
    \def\childdocextract #2##1~~~{\def\childdoctmp{\childdocforward[#1]{#3##1}}}
    \expandafter\childdocextract\childdocname~~~
    \expandafter
  \endgroup
  \childdoctmp
}
%    \end{macrocode}

% \macro{\childdoc}
% The deprecated macro |\childdoc| is a legacy version of |\childdocmain|:
%    \begin{macrocode}
\newcommand{\childdoc}{\childdocmain}
%    \end{macrocode}

% \macro{\childdocredirect}
% The deprecated macro |\childdocredirect| is a legacy version
% of |\childdocforward| and |\childdocforwardprefix|:
%    \begin{macrocode}
\newcommand{\childdocredirect}[2][]
{
  \begingroup
    \if?#1?
      \def\childdoctmp{\childdocforward{#2}}
    \else
      \def\childdoctmp{\childdocforwardprefix{#1}{#2}}
    \fi
    \expandafter
  \endgroup
  \childdoctmp
}
%    \end{macrocode}

%\iffalse
%</package>
%\fi
%
\endinput

\childdocby{cdocsamp}
%    \end{macrocode}

%\iffalse
%</samplepart3|samplepart4>
%\fi
%
%\iffalse
%<*samplepart3>
%\fi
% Some text for part 3:
%    \begin{macrocode}
some text in part three
%    \end{macrocode}

%\iffalse
%</samplepart3>
%\fi
% Some text for part 4:
%\iffalse
%<*samplepart4>
%\fi
%    \begin{macrocode}
more text in part four
%    \end{macrocode}

%\iffalse
%</samplepart4>
%\fi
%
% %%%%%%%%%%%%%%%%%%%%%%%%%%%%%%%%%%%%%%
% \paragraph{Forwarding for a Complete Draft.}
%
% The following forwarding file |cdocsdrf.tex|
% compiles the main document in draft mode:
%\iffalse
%<*sampledraft>
%\fi
%    \begin{macrocode}
\def\version{draft}
% \iffalse
%
% childdoc.dtx Copyright (C) 2017-2018 Niklas Beisert
%
% This work may be distributed and/or modified under the
% conditions of the LaTeX Project Public License, either version 1.3
% of this license or (at your option) any later version.
% The latest version of this license is in
%   http://www.latex-project.org/lppl.txt
% and version 1.3 or later is part of all distributions of LaTeX
% version 2005/12/01 or later.
%
% This work has the LPPL maintenance status `maintained'.
%
% The Current Maintainer of this work is Niklas Beisert.
%
% This work consists of the files childdoc.dtx and childdoc.ins
% and the derived files childdoc.def and cdocsamp.tex with
% cdocsch1.tex, cdocsch2.tex, cdocsdrf.tex, cdocsfn1.tex, cdocsfn2.tex.
%
%<package>\ifdefined\childdocmain\endinput\fi
%<package>\ProvidesFile{childdoc.def}[2018/12/30 v2.0 child document driver]
%<samplemain>\ProvidesFile{cdocsamp.tex}[2018/12/30 v2.0 sample for childdoc]
%<*driver>
%\ProvidesFile{childdoc.drv}[2018/12/30 v2.0 childdoc reference manual file]
\PassOptionsToClass{10pt,a4paper}{article}
\documentclass{ltxdoc}

\usepackage[margin=35mm]{geometry}
\usepackage{hyperref}
\usepackage{hyperxmp}
\usepackage[usenames]{color}

\hypersetup{colorlinks=true}
\hypersetup{pdfstartview=FitH}
\hypersetup{pdfpagemode=UseNone}
\hypersetup{pdfsource={}}
\hypersetup{pdflang={en-UK}}
\hypersetup{pdfcopyright={Copyright 2017-2018 Niklas Beisert.
  This work may be distributed and/or modified under the
  conditions of the LaTeX Project Public License, either version 1.3
  of this license or (at your option) any later version.}}
\hypersetup{pdflicenseurl={http://www.latex-project.org/lppl.txt}}
\hypersetup{pdfcontactaddress={ETH Zurich, ITP, HIT K,
  Wolfgang-Pauli-Strasse 27}}
\hypersetup{pdfcontactpostcode={8093}}
\hypersetup{pdfcontactcity={Zurich}}
\hypersetup{pdfcontactcountry={Switzerland}}
\hypersetup{pdfcontactemail={nbeisert@itp.phys.ethz.ch}}
\hypersetup{pdfcontacturl={http://people.phys.ethz.ch/\xmptilde nbeisert/}}

\newcommand{\secref}[1]{\hyperref[#1]{section \ref*{#1}}}

\parskip1ex
\parindent0pt
\let\olditemize\itemize
\def\itemize{\olditemize\parskip0pt}

\begin{document}

\title{The \textsf{childdoc} Package}
\hypersetup{pdftitle={The childdoc Package}}
\author{Niklas Beisert\\[2ex]
  Institut f\"ur Theoretische Physik\\
  Eidgen\"ossische Technische Hochschule Z\"urich\\
  Wolfgang-Pauli-Strasse 27, 8093 Z\"urich, Switzerland\\[1ex]
  \href{mailto:nbeisert@itp.phys.ethz.ch}
  {\texttt{nbeisert@itp.phys.ethz.ch}}}
\hypersetup{pdfauthor={Niklas Beisert}}
\hypersetup{pdfsubject={Manual for the LaTeX2e Package childdoc}}
\date{30 December 2018, \textsf{v2.0}}
\maketitle

\begin{abstract}\noindent
\textsf{childdoc} is a \LaTeXe{} package
that enables the direct compilation
of document sections included by |\include|
to individual files.
\end{abstract}

\begingroup
\parskip0ex
\tableofcontents
\endgroup

%%%%%%%%%%%%%%%%%%%%%%%%%%%%%%%%%%%%%%%%%%%%%%%%%%%%%%%%%%%%%%%%%%%%%%%%%%%%%%%%
%%%%%%%%%%%%%%%%%%%%%%%%%%%%%%%%%%%%%%%%%%%%%%%%%%%%%%%%%%%%%%%%%%%%%%%%%%%%%%%%
\section{Introduction}

\LaTeX{} provides a mechanism to structure a large document (such as a book)
into a main file and several child files (containing the chapters)
using the |\include| command.
This mechanism is beneficial for documents
which span hundreds of pages in order to
make the source file(s) more manageable.
Moreover, compilation can be restricted to
selected child files by means of the |\includeonly| command.
The latter feature can be used to reduce the compilation time while editing
(this was significantly more useful in the earlier days of \LaTeX{})
or to generate a smaller document which is easier to navigate.
Another application of |\includeonly| is to generate
documents consisting of selected parts of the complete document.

However, there are a few drawbacks of the plain |\include| mechanism:
\begin{itemize}
\item
The child files cannot be compiled on their own,
they can only be compiled via the main file.
A naive editing environment
(such as a text editor with an option
to have the current file processed by \LaTeX)
may require one to switch to the main file before compiling;
attempting to compile the child file produces errors.
\item
The main file must be modified (each time)
to adjust the |\includeonly| command
to the present needs. This easily leaves the main file in a messy state.
\item
The generated document will always carry the filename
of the main document. This is inconvenient if
several child files are to be compiled and
to be kept for distribution.
\end{itemize}

The present package provides a simple interface
to make child files individually compilable by \LaTeX{}.
Compiling a child file then has the same effect as compiling
the main file with an |\includeonly| command
to select the appropriate child.
Moreover the generated document will carry the name of the child
rather than the main file.
This resolves all three above issues.

This feature is meant to make the editing of books,
thesis documents and lecture notes somewhat more convenient.
However, the package can also be used efficiently for
composing a series of documents (such as exercise sheets)
which are typically distributed individually.
It then assists the author in generating the individual documents
(potentially in different versions)
as well as a document containing the collected series.
Another application is in developing style files
or other kinds of included material
where compilation of the style file could redirect
to a sample or test file.

%%%%%%%%%%%%%%%%%%%%%%%%%%%%%%%%%%%%%%%%%%%%%%%%%%%%%%%%%%%%%%%%%%%%%%%%%%%%%%%%
%%%%%%%%%%%%%%%%%%%%%%%%%%%%%%%%%%%%%%%%%%%%%%%%%%%%%%%%%%%%%%%%%%%%%%%%%%%%%%%%
\section{Usage}

First of all, the package \textsf{childdoc} is \emph{not} a standard
\LaTeXe{} |.sty| style file! Therefore it needs to be invoked in
a non-standard way.

%%%%%%%%%%%%%%%%%%%%%%%%%%%%%%%%%%%%%%%%%%%%%%%%%%%%%%%%%%%%%%%%%%%%%%%%%%%%%%%%
\subsection{Included Files}
\label{sec:include}

%%%%%%%%%%%%%%%%%%%%%%%%%%%%%%%%%%%%%%%%
\DescribeMacro{\childdocmain}
To use the package, add the commands
\begin{center}
\begin{tabular}{l}
|\input{childdoc.def}|\\
|\childdocmain{}|\\
\end{tabular}
\end{center}
at the very top of the main \LaTeX{} file,
in particular \emph{before} the |\documentclass| statement!
The argument of |\childdocmain| should be left empty
(but it must be present).

%%%%%%%%%%%%%%%%%%%%%%%%%%%%%%%%%%%%%%%%
\DescribeMacro{\childdocof}
Furthermore, add the commands
\begin{center}
\begin{tabular}{l}
|\input{childdoc.def}|\\
|\childdocof{|\textit{main}|}|\\
\end{tabular}
\end{center}
at the top of every child file \textit{child}
which is included by |\include{|\textit{child}|}|
from within the main file
(or at least for those files to be compiled individually).
The argument \textit{main} must be the filename of the main file.

There are a couple of
considerations in setting up the main and child documents:

%%%%%%%%%%%%%%%%%%%%%%%%%%%%%%%%%%%%%%%%
\paragraph{Restrictions.}

Please note the following restrictions:
\begin{itemize}
\item
|\childdocmain| must be called with one argument \textit{main}
to ensure compatibility with earlier version of the package.
It must either be empty (|\childdocmain{}|)
or precisely match the filename of the main file in which it is specified.
See \secref{sec:detection} for further information.
\item
The filename \textit{main} must be specified without the |.tex| extension.
\item
The filename \textit{main} is case sensitive
(even in case-insensitive file systems)
due to internal string comparison.
\item
The argument \textit{main} should be fully expanded, it cannot be a macro.
\item
Subdirectories and special characters should be avoided in filenames.
\item
The command |\childdocmain{|\textit{main}|}| must be followed by a whitespace.
It should not be followed immediately by another command
or by a comment mark `|%|'.
This is because the \TeX{} parser reads the token immediately following
the argument of |\childdocmain| and puts it
at the beginning of every child section;
however, a white\-space is ignored.
\end{itemize}

%%%%%%%%%%%%%%%%%%%%%%%%%%%%%%%%%%%%%%%%
\paragraph{Content of Main File.}

It is advisable to place all content in the child files included by |\include|.
Any output contained in the main file will appear in all child documents
unless suppressed manually;
it cannot be suppressed automatically by the |\includeonly| directive
and thus should normally be avoided.
A method to include some content in the main file
by means of conditional processing is described in \secref{sec:conditional}.

%%%%%%%%%%%%%%%%%%%%%%%%%%%%%%%%%%%%%%%%
\paragraph{Page Numbering.}

When only a part of the document is compiled,
the appropriate numbering of pages
(as well as other status parameters)
is determined from the |.aux| files.
The latter contain information from previous passes.
However this information needs to propagate through
all intermediate child documents.
Therefore the page numbering in child documents may well
be inconsistent until the complete document is compiled at least once.

A useful (if unconventional) way to always ensure a consistent
page numbering is to restart the numbering in each child document
and denote the pages by `\textit{child}|.|\textit{page}'
where \textit{child} represents the chapter/section number of the child file.
This can be achieved by the command
|\numberwithin{page}{|\textit{child}|}|
of the \textsf{amsmath} package
where \textit{child} can be |chapter| or |section|
depending on the chosen structuring.
Alternatively, one can modify the macro |\thepage| appropriately
and reset the counter |page| at the start of each child file.

%%%%%%%%%%%%%%%%%%%%%%%%%%%%%%%%%%%%%%%%%%%%%%%%%%%%%%%%%%%%%%%%%%%%%%%%%%%%%%%%
\subsection{Conditional Processing}
\label{sec:conditional}

The package provides a mechanism to compile different versions
of a document. To customise the versions further some conditional processing
can come in handy to distinguish which version is being compiled.
The package provides two macros to describe the compilation context:

%%%%%%%%%%%%%%%%%%%%%%%%%%%%%%%%%%%%%%%%
\DescribeMacro{\ifchilddoc}
The conditional |\ifchilddoc| distinguishes between the compilation of
child documents and the main document:
%
\begin{center}
|\ifchilddoc |\textit{child-code}| |[|\||else |\textit{main-code}]| \||fi|
\end{center}

%%%%%%%%%%%%%%%%%%%%%%%%%%%%%%%%%%%%%%%%
\DescribeMacro{\childdocname}
\DescribeMacro{\childdocjob}
The macro |\childdocname| contains the filename (without extension)
of the main or child file being processed.
Note that |\childdocjob| will always contain the name of the main file.

%%%%%%%%%%%%%%%%%%%%%%%%%%%%%%%%%%%%%%%%
\paragraph{Title Page.}

Conditional processing can be used to include a title or banner page
in the main document when proper precautions are taken.
Importantly, the code in the main file should ensure that the page counter
(as well as other status parameters which are stored in the |.aux| files)
takes the same value after the conditional processing.
Otherwise the page numbers may take divergent values
depending on which part is compiled.

For example, a title page could be declared by:
%
\begin{center}
\begin{tabular}{l}
|\ifchilddoc\||else|\\
|\addtocounter{page}{-1}|\\
\textit{code for title page}\\
|\newpage|\\
|\||fi|
\end{tabular}
\end{center}
%
A banner page for the child documents can be generated by:
%
\begin{center}
\begin{tabular}{l}
|\ifchilddoc|\\
|\addtocounter{page}{-1}|\\
\textit{code for banner page}\\
|\newpage|\\
|\||fi|
\end{tabular}
\end{center}
%
Here one could write a message such as:
\begin{center}
|This is the part \childdocname{} of \childdocjob{}.|
\end{center}

%%%%%%%%%%%%%%%%%%%%%%%%%%%%%%%%%%%%%%%%%%%%%%%%%%%%%%%%%%%%%%%%%%%%%%%%%%%%%%%%
\subsection{Flags}
\label{sec:flags}

The package makes it easy to generate different versions
of the main or child documents.
To this end compilation flags can be defined
and assigned different default values.
They will be particularly useful in conjunction
with the forwarding mechanism described in \secref{sec:forward}.

For example, it may be useful to have a flag |\version|
which can be set to |draft| or |final|.
The document source will contain some conditional code
depending on the value of |\version|.
Suppose further, the flag should default to |final| for the main file
and to |draft| for child files
which is a natural assignment for editing the document.
This is achieved by placing the following code
in the preamble of the main document
(below the |\childdocmain| directive):
%
\begin{center}
\begin{tabular}{l}
|\ifchilddoc|\\
|\providecommand{\version}{draft}|\\
|\||else|\\
|\providecommand{\version}{final}|\\
|\||fi|
\end{tabular}
\end{center}
%
The definition by |\providecommand| makes sure
that previous definitions are not overwritten.
Further statements |\providecommand{\version}{...}|
can thus be added before the above code to override it.

For the main file, one might add a line
(between |\childdocmain| and the above block)
%
\begin{center}
|%\ifchilddoc\||else\providecommand{\version}{draft}\||fi|
\end{center}
%
which can be uncommented to produce a draft version.
Likewise one can add a line to the very top of a child file
(above the |\childdocof{|\textit{main}|}| directive)
%
\begin{center}
|%\providecommand{\version}{final}|
\end{center}
%
which can be uncommented to produce the final version of this child document.

%%%%%%%%%%%%%%%%%%%%%%%%%%%%%%%%%%%%%%%%%%%%%%%%%%%%%%%%%%%%%%%%%%%%%%%%%%%%%%%%
\subsection{Forwarding}
\label{sec:forward}

Different versions of the main or child documents
using compilation flags as described in \secref{sec:flags}
can be (permanently) stored in different files
for convenient compilation, viewing and distribution.
To this end, the package defines a command
to pass on compilation to a different file:

%%%%%%%%%%%%%%%%%%%%%%%%%%%%%%%%%%%%%%%%
\DescribeMacro{\childdocforward}
The command |\childdocforward| redirects processing to
another source file:
%
\begin{center}
\begin{tabular}{l}
|\input{childdoc.def}|\\
|\childdocforward[|\textit{main}|]{|\textit{dest}|}|\\
\end{tabular}
\end{center}
%
The argument \textit{dest} is the destination file
(without extension).
It should be the main file or one of the child files.
Note that further \textsf{childdoc} directives
such as |\childdocof| and |\childdocforward|
in the indicated file will be processed in this form.
The optional argument \textit{main}
passes on directly to the main file \textit{main}
while pretending to compile the child \textit{dest}.
This form behaves as if \textit{dest}
issues |\childdocof{|\textit{main}|}| right away,
and no further \textsf{childdoc} directives will be processed.

%%%%%%%%%%%%%%%%%%%%%%%%%%%%%%%%%%%%%%%%
\DescribeMacro{\...prefix}
In the alternative form |\childdocforwardprefix|,
%
\begin{center}
\begin{tabular}{l}
|\input{childdoc.def}|\\
|\childdocforwardprefix[|\textit{main}|]{|\textit{prefix}|}{|\textit{dest}|}|
\end{tabular}
\end{center}
%
the destination file is determined by a pattern
depending on the current file:
To make this work, the current file must be called
`{\textit{prefix}\hspace{0.2em}\textit{suffix}}'
with \textit{prefix} matching precisely the argument.
Processing is then passed on to the file
`{\textit{dest}\hspace{0.2em}\textit{suffix}}'.
Surely, the same effect is achieved by
directly specifying the
argument `{\textit{dest}\hspace{0.2em}\textit{suffix}}'
in the first form.
However, that requires to set up a different file
for each child. With the alternative form of the command
all these files can have exactly the same content
which simplifies setting them up and maintaining them.

For example, the following file |draft.tex|
with a compilation flag |\version| as described in \secref{sec:flags}
compiles the main document as a draft:
%
\begin{center}
\begin{tabular}{l}
|\def\version{draft}|\\
|\input{childdoc.def}|\\
|\childdocforward{|\textit{main}|}|
\end{tabular}
\end{center}
%
Likewise, the following files |final|\textit{nn}|.tex|
compile the final version of the child document
|child|\textit{nn}|.tex|:
%
\begin{center}
\begin{tabular}{l}
|\def\version{final}|\\
|\input{childdoc.def}|\\
|\childdocforwardprefix{final}{child}|
\end{tabular}
\end{center}
%

Note that when several versions of a main file and/or of each child file
are to be generated, it may be convenient to set up a |Makefile| or
shell script to automatise the process.

%%%%%%%%%%%%%%%%%%%%%%%%%%%%%%%%%%%%%%%%%%%%%%%%%%%%%%%%%%%%%%%%%%%%%%%%%%%%%%%%
\subsection{Command Line Processing}
\label{sec:commandline}

The effect of redirection files can also be achieved by invoking
the \LaTeX{} compiler with a more elaborate command line.
Most conveniently this should be done as part
of a shell script or a |Makefile|.

When using \textsf{childdoc} in the main file, the following
command lines effectively perform a redirection
(note that depending on the shell being used,
backslashes may have to be doubled: `|\|' $\to$ `|\\|'):
%
\begin{center}
|... -jobname "|\textit{target}|" |\\|"|[\textit{flags}]%
|\input{childdoc.def}\childdocforward[|\textit{main}|]{|\textit{dest}|}"|
\end{center}
%
Here \textit{target} is the name of the output file,
\textit{main} is the name of the main file
and \textit{dest} is the name of the main or child file to be processed
(all filenames without extensions).
The optional argument \textit{main} can be omitted
if \textit{main} matches \textit{dest}.
Optionally, compilation \textit{flags} can be defined via |\def| commands.
This command line makes the \TeX{} engine believe
it is compiling the file \textit{target}
whose content is specified as the latter parameter.
The provided code then forwards the processing to
\textit{main} or \textit{dest} as described in \secref{sec:forward}.

%%%%%%%%%%%%%%%%%%%%%%%%%%%%%%%%%%%%%%%%%%%%%%%%%%%%%%%%%%%%%%%%%%%%%%%%%%%%%%%%
\subsection{Include by Input}
\label{sec:input}

Including child documents by |\include| has some restrictions by design.
Most notably, the content of a child document always occupies
its own set of pages; pages cannot be shared between child documents.
Usually, this behaviour makes perfect sense
because each child document contain an essential part of the document.
However, in some situations it may be desirable to compose
a document from a collection of parts
without having mandatory page breaks between then.
For this case, the package
provides a mechanism to include parts
by |\input| which can also be processed individually.
However, by construction this mechanism
requires manual handling of the content to be output.

%%%%%%%%%%%%%%%%%%%%%%%%%%%%%%%%%%%%%%%%
\DescribeMacro{\ifchilddocmanual}
The main file should be prepared as usual, see \secref{sec:include}.
However, the document body must make a distinction
between processing of an individual part and of the main document, e.g.:
%
\begin{center}
\begin{tabular}{l}
|\ifchilddocmanual|\\
|\input{\childdocname}|\\
|\||else|\\
\textit{document body with }|\input{|\textit{part}|}|\\
|\||fi|
\end{tabular}
\end{center}
%
The conditional |\ifchilddocmanual| is true whenever
a part to be included by |\input| is being compiled,
and the name of the part is stored in |\childdocname|.

%%%%%%%%%%%%%%%%%%%%%%%%%%%%%%%%%%%%%%%%
\DescribeMacro{\childdocby}
Each part to be included by |\input| should start with:
%
\begin{center}
\begin{tabular}{l}
|\input{childdoc.def}|\\
|\childdocby{|\textit{main}|}|\\
\end{tabular}
\end{center}
%
The directive |\childdocby| is similar to |\childdocof|
described in \secref{sec:include},
but the subsequent selection of content must be done manually.
To that end, both |\ifchilddoc| and |\ifchilddocmanual|
will be true upon processing of a part,
and the name of the part is stored in |\childdocname|.
Note that |\jobname| will be set to the filename of the current part
so that each part receives an individual |.aux| file
that does not interfere with the |.aux| file(s) of the main document.
This behaviour can be altered by the alternative form
|\childdocby[*]{|\textit{main}|}| (with a non-empty optional argument)
which uses the |.aux| file of the main document
by setting |\jobname| to \textit{main}.

%%%%%%%%%%%%%%%%%%%%%%%%%%%%%%%%%%%%%%%%%%%%%%%%%%%%%%%%%%%%%%%%%%%%%%%%%%%%%%%%
\subsection{Driver Development}
\label{sec:driver}

The \textsf{childdoc} mechanism can also be use for the development
of definition files such as \LaTeX{} styles or classes.
This case differs from the above setup with multiple parts
included by |\include| in that no |\includeonly| should be invoked.
This can be achieved by starting the include file
(before |\ProvidesPackage|) with:
%
\begin{center}
\begin{tabular}{l}
|\input{childdoc.def}|\\
|\childdocforward{|\textit{main}|}|\\
\end{tabular}
\end{center}
%
or alternatively with:
%
\begin{center}
\begin{tabular}{l}
|\input{childdoc.def}|\\
|\childdocby{|\textit{main}|}|\\
\end{tabular}
\end{center}
%
Both forms have slightly different effects as described above.
The main file is prepared as usual, see \secref{sec:include}.

%%%%%%%%%%%%%%%%%%%%%%%%%%%%%%%%%%%%%%%%%%%%%%%%%%%%%%%%%%%%%%%%%%%%%%%%%%%%%%%%
\subsection{Legacy Detection}
\label{sec:detection}

The directive |\childdocmain| in the main file can detect
whether the complete document or merely a child is to be compiled
even without using the directive |\childdocof|.
This method is deprecated because it is less robust
and there is no compelling reason to use it;
it is merely provided for backward compatibility
and it may be removed in future versions.

If the detection mechanism is to be used,
it is mandatory to correctly specify
the filename of the main file as the argument of |\childdocmain|:
%
\begin{center}
\begin{tabular}{l}
|\input{childdoc.def}|\\
|\childdocmain{|\textit{main}|}|\\
\end{tabular}
\end{center}
%
If |\jobname| does not match the argument \textit{main} of |\childdocmain|,
it is assumed that |\jobname| points to the child file to be compiled.
When using |\childdocmain| with the main file specified as argument,
it suffices to start a child file
with just |\input{|\textit{main}|}|
without loading of the package and using |\childdocof|.
If instead all processing is done
with the appropriate \textsf{childdoc} directives,
the argument of \textit{main} of |\childdocmain| can be empty.

An alternative version of the command line processing described
in \secref{sec:commandline} using the detection mechanism reads:
%
\begin{center}
|... -jobname "|\textit{target}|" "|[\textit{flags}]%
[|\def\jobname{|\textit{dest}|}|]|\input{|\textit{main}|}"|
\end{center}

%%%%%%%%%%%%%%%%%%%%%%%%%%%%%%%%%%%%%%%%%%%%%%%%%%%%%%%%%%%%%%%%%%%%%%%%%%%%%%%%
\subsection{Manual Code}
\label{sec:manual}

In case one cannot be certain whether the definitions file |childdoc.def|
is installed on the target \TeX{} distribution
and one prefers not to ship it,
it is conceivable to paste a few relevant commands into the sources.

To that end, drop all statements |\input{childdoc.def}|
and perform the replacements as outlined below.
Instead of |\childdocmain{|\textit{main}|}| add the following code
to the top of the main file:
%
\begin{center}
\begin{tabular}{l}
|\||ifdefined\childdocname\endinput\||fi\newif\ifchilddoc|\\
|\edef\childdocname{\scantokens\expandafter{\jobname\noexpand}}|\\
|\def\childdocmain{|\textit{main}|}\||ifx\childdocmain\childdocname\||else|\\
|\childdoctrue\includeonly{\childdocname}\let\jobname\childdocmain\||fi|\\
\end{tabular}
\end{center}
%
Instead of |\childdocof{|\textit{main}|}| just include the main file
at the top of each child file:
%
\begin{center}
|\input{|\textit{main}|}|
\end{center}
%
A simple redirection |\childdocforward{|\textit{dest}|}| is achieved by:
%
\begin{center}
|\def\jobname{|\textit{dest}|}\input{\jobname}|
\end{center}
%
The redirection with prefix
|\childdocforwardprefix[|\textit{prefix}|]{|\textit{dest}|}|
is accomplished by:
%
\begin{center}
\begin{tabular}{l}
|{\edef\jobname{\scantokens\expandafter{\jobname\noexpand}}|\\
|\def\redirectjob |\textit{prefix}|#1~~~{\gdef\jobname{|\textit{dest}|#1}}|\\
|\expandafter\redirectjob\jobname~~~}\input{\jobname}|
\end{tabular}
\end{center}

In an alternative approach,
child documents can be compiled by a specific command line
without additional code or specific definitions:
%
\begin{center}
|... -jobname "|\textit{target}|" "|[\textit{flags}]%
|\includeonly{|\textit{dest}|}\input{|\textit{main}|}"|
\end{center}
%

%%%%%%%%%%%%%%%%%%%%%%%%%%%%%%%%%%%%%%%%%%%%%%%%%%%%%%%%%%%%%%%%%%%%%%%%%%%%%%%%
%%%%%%%%%%%%%%%%%%%%%%%%%%%%%%%%%%%%%%%%%%%%%%%%%%%%%%%%%%%%%%%%%%%%%%%%%%%%%%%%
\section{Information}

%%%%%%%%%%%%%%%%%%%%%%%%%%%%%%%%%%%%%%%%%%%%%%%%%%%%%%%%%%%%%%%%%%%%%%%%%%%%%%%%
\subsection{Copyright}

Copyright \copyright{} 2017--2018 Niklas Beisert

This work may be distributed and/or modified under the
conditions of the \LaTeX{} Project Public License, either version 1.3
of this license or (at your option) any later version.
The latest version of this license is in
  \url{http://www.latex-project.org/lppl.txt}
and version 1.3 or later is part of all distributions of \LaTeX{}
version 2005/12/01 or later.

This work has the LPPL maintenance status `maintained'.

The Current Maintainer of this work is Niklas Beisert.

This work consists of the files |README.txt|, |childdoc.ins| and |childdoc.dtx|
as well as the derived files |childdoc.def|, |cdocsamp.tex|
with |cdocsch1.tex|, |cdocsch2.tex|, |cdocspt3.tex|, |cdocspt4.tex|,
|cdocsdrf.tex|, |cdocsfn1.tex|, |cdocsfn2.tex|
as well as |childdoc.pdf|.

%%%%%%%%%%%%%%%%%%%%%%%%%%%%%%%%%%%%%%%%%%%%%%%%%%%%%%%%%%%%%%%%%%%%%%%%%%%%%%%%
\subsection{Files and Installation}

The package consists of the files:
%
\begin{center}
\begin{tabular}{ll}
    |README.txt|   & readme file \\
    |childdoc.ins| & installation file \\
    |childdoc.dtx| & source file \\
    |childdoc.def| & definition file \\
    |cdocsamp.tex| & sample main file \\
    |cdocsch1.tex| & sample include file \\
    |cdocsch2.tex| & sample include file \\
    |cdocspt3.tex| & sample part file \\
    |cdocspt4.tex| & sample part file \\
    |cdocsdrf.tex| & sample redirection file \\
    |cdocsfn1.tex| & sample redirection file \\
    |cdocsfn2.tex| & sample redirection file \\
    |childdoc.pdf| & manual
\end{tabular}
\end{center}
%
The distribution consists of the files
|README.txt|, |childdoc.ins| and |childdoc.dtx|.
%
\begin{itemize}
\item
Run (pdf)\LaTeX{} on |childdoc.dtx|
to compile the manual |childdoc.pdf| (this file).
\item
Run \LaTeX{} on |childdoc.ins| to create the definitions file |childdoc.def|
and the sample |cdocsamp.tex| with include files
|cdocsch1.tex|, |cdocsch2.tex|, |cdocspt3.tex|, |cdocspt4.tex|,
|cdocsdrf.tex|, |cdocsfn1.tex|, |cdocsfn2.tex|.
Then copy the file |childdoc.def| to an appropriate directory of your \LaTeX{}
distribution, e.g.\ \textit{texmf-root}|/tex/latex/childdoc|.
\end{itemize}

%%%%%%%%%%%%%%%%%%%%%%%%%%%%%%%%%%%%%%%%%%%%%%%%%%%%%%%%%%%%%%%%%%%%%%%%%%%%%%%%
\subsection{Related CTAN Packages}

There are several other packages which offer a similar functionality:
%
\begin{itemize}
\item
The packages
\href{http://ctan.org/pkg/docmute}{\textsf{docmute}},
\href{http://ctan.org/pkg/includex}{\textsf{includex}} and
\href{http://ctan.org/pkg/standalone}{\textsf{standalone}}
provide commands to include only the document body of
a child file thus allowing both files to be compiled individually.
\item
The packages \href{http://ctan.org/pkg/subdocs}{\textsf{subdocs}}
and \href{http://ctan.org/pkg/subfiles}{\textsf{subfiles}}
provide structures in which the main and child documents can be
encapsulated and allowing them to be compiled individually.
The inclusion mechanism is different from the conventional |\include|.
\item
The package \href{http://ctan.org/pkg/combine}{\textsf{combine}}
is an elaborate solution to combine several documents into one.
\end{itemize}
%
See also the CTAN topic \href{http://ctan.org/topic/subdocs}{\textsf{subdocs}}
for further related packages.
The present package differs from the above solutions in that
a document structure constructed with the conventional |\include| mechanism
just needs two extra commands at the top of every file
such that all constituent files can be compiled individually.

%%%%%%%%%%%%%%%%%%%%%%%%%%%%%%%%%%%%%%%%%%%%%%%%%%%%%%%%%%%%%%%%%%%%%%%%%%%%%%%%
%\subsection{Feature Suggestions}
%
%The following is a list of features which may be useful for future
%versions of this package:
%%
%\begin{itemize}
%\item
%\ldots
%\end{itemize}

%%%%%%%%%%%%%%%%%%%%%%%%%%%%%%%%%%%%%%%%%%%%%%%%%%%%%%%%%%%%%%%%%%%%%%%%%%%%%%%%
\subsection{Revision History}

%%%%%%%%%%%%%%%%%%%%%%%%%%%%%%%%%%%%%%%%
\paragraph{v2.0:} 2018/12/30

\begin{itemize}
\item
immediate forward processing
\item
added |\childdocby| mechanism
\item
manual restructured
\end{itemize}

%%%%%%%%%%%%%%%%%%%%%%%%%%%%%%%%%%%%%%%%
\paragraph{v1.6:} 2018/01/17

\begin{itemize}
\item
application for development of include files
\item
corrections to manual
\end{itemize}

%%%%%%%%%%%%%%%%%%%%%%%%%%%%%%%%%%%%%%%%
\paragraph{v1.5:} 2017/05/21

\begin{itemize}
\item
more complete structuring introduced
\item
|\childdocof| introduced
\item
|\childdoc| renamed to |\childdocmain|
\item
|\childredirect| renamed to |\childdocforward| and |\childdocforwardprefix|
and functionality expanded
\end{itemize}

%%%%%%%%%%%%%%%%%%%%%%%%%%%%%%%%%%%%%%%%
\paragraph{v1.0:} 2017/04/27

\begin{itemize}
\item
manual and install package
\item
first version published on CTAN
\end{itemize}

%%%%%%%%%%%%%%%%%%%%%%%%%%%%%%%%%%%%%%%%
\paragraph{v0.6:} 2017/04/26

\begin{itemize}
\item
redirection mechanism added
\end{itemize}

%%%%%%%%%%%%%%%%%%%%%%%%%%%%%%%%%%%%%%%%
\paragraph{v0.5:} 2017/04/26

\begin{itemize}
\item
functionality in definition file
\end{itemize}


%%%%%%%%%%%%%%%%%%%%%%%%%%%%%%%%%%%%%%%%%%%%%%%%%%%%%%%%%%%%%%%%%%%%%%%%%%%%%%%%
%%%%%%%%%%%%%%%%%%%%%%%%%%%%%%%%%%%%%%%%%%%%%%%%%%%%%%%%%%%%%%%%%%%%%%%%%%%%%%%%
%%%%%%%%%%%%%%%%%%%%%%%%%%%%%%%%%%%%%%%%%%%%%%%%%%%%%%%%%%%%%%%%%%%%%%%%%%%%%%%%
\appendix

\settowidth\MacroIndent{\rmfamily\scriptsize 000\ }

 \DocInput{childdoc.dtx}

\end{document}
%</driver>
% \fi
%
% %%%%%%%%%%%%%%%%%%%%%%%%%%%%%%%%%%%%%%%%%%%%%%%%%%%%%%%%%%%%%%%%%%%%%%%%%%%%%%
% %%%%%%%%%%%%%%%%%%%%%%%%%%%%%%%%%%%%%%%%%%%%%%%%%%%%%%%%%%%%%%%%%%%%%%%%%%%%%%
% \section{Sample}
%\iffalse
%<*samplemain>
%\fi
%
% The following presents a sample document
% with two chapters, two parts, a title page,
% a compile flag as well as three forwarding files to set the flag.
% It consists of eight |.tex| files:
% \begin{center}
% \begin{tabular}{ll}
% |cdocsamp.tex|&main file\\
% |cdocsch1.tex|&include file for chapter 1\\
% |cdocsch2.tex|&include file for chapter 2\\
% |cdocspt3.tex|&include file for part 3\\
% |cdocspt4.tex|&include file for part 4\\
% |cdocsdrf.tex|&forwarding file for main file in draft mode\\
% |cdocsfi1.tex|&forwarding file for final version of chapter 1\\
% |cdocsfi2.tex|&forwarding file for final version of chapter 2\\
% \end{tabular}
% \end{center}
% Each of the eight files can be compiled directly by the \LaTeX{} compiler.
%
% %%%%%%%%%%%%%%%%%%%%%%%%%%%%%%%%%%%%%%
% \paragraph{Main File.}
%
% The main file is called |cdocsamp.tex|.
%
% Load the \textsf{childdoc} definitions and
% declare the filename for the main document:
%    \begin{macrocode}
\input{childdoc.def}
\childdocmain{}
%    \end{macrocode}

% Optional override for |\version| flag:
%    \begin{macrocode}
%%\ifchilddoc\else\providecommand{\version}{draft}\fi
%    \end{macrocode}

% Define the default values for the |\version| flag
% (|final| for the main file and |draft| for childs):
%    \begin{macrocode}
\ifchilddoc
\providecommand{\version}{draft}
\else
\providecommand{\version}{final}
\fi
%    \end{macrocode}

% Load the standard document class:
%    \begin{macrocode}
\documentclass[12pt]{article}
%    \end{macrocode}

% Start the document body:
%    \begin{macrocode}
\begin{document}
%    \end{macrocode}

% Declare a title page.
% Print title, part of document being processed and version flag:
%    \begin{macrocode}
\addtocounter{page}{-1}
\begin{center}
{\LARGE\bfseries{}childdoc example\par}
\vspace{1cm}
\ifchilddoc
\ifchilddocmanual part\else chapter\fi:
`\childdocname' of `\childdocjob'\par
\else
main document: `\childdocjob'\par
\fi
version: \version\par
\end{center}
\newpage
%    \end{macrocode}

% Manually include selected file,
% otherwise process as usual:
%    \begin{macrocode}
\ifchilddocmanual
\section*{part `\childdocname'}
\input{\childdocname}
\else
%    \end{macrocode}

% Include the two chapters:
%    \begin{macrocode}
\include{cdocsch1}
\include{cdocsch2}
%    \end{macrocode}

% Include the two parts unless only chapters should be displayed:
%    \begin{macrocode}
\ifchilddoc\else
\section{part three}
\input{cdocspt3}
\section{part four}
\input{cdocspt4}
\fi
%    \end{macrocode}

% Process as usual until here:
%    \begin{macrocode}
\fi
%    \end{macrocode}

% End of document body:
%    \begin{macrocode}
\end{document}
%    \end{macrocode}
%\iffalse
%</samplemain>
%\fi
%
% %%%%%%%%%%%%%%%%%%%%%%%%%%%%%%%%%%%%%%
% \paragraph{Chapter Include Files.}
%
% The include files are called |cdocsch1.tex| and |cdocsch2.tex|.
%
%\iffalse
%<*samplechap1|samplechap2>
%\fi

% Optional override for |\version| flag:
%    \begin{macrocode}
%%\providecommand{\version}{final}
%    \end{macrocode}

% Include the main document:
%    \begin{macrocode}
\input{childdoc.def}
\childdocof{cdocsamp}
%    \end{macrocode}

%\iffalse
%</samplechap1|samplechap2>
%\fi
%
%\iffalse
%<*samplechap1>
%\fi
% Some text for chapter 1:
%    \begin{macrocode}
\section{one}
some text in chapter one
%    \end{macrocode}

%\iffalse
%</samplechap1>
%\fi
% Some text for chapter 2:
%\iffalse
%<*samplechap2>
%\fi
%    \begin{macrocode}
\section{two}
more text in chapter two
%    \end{macrocode}

%\iffalse
%</samplechap2>
%\fi
%
% %%%%%%%%%%%%%%%%%%%%%%%%%%%%%%%%%%%%%%
% \paragraph{Part Include Files.}
%
% The include files are called |cdocspt3.tex| and |cdocspt4.tex|.
%
%\iffalse
%<*samplepart3|samplepart4>
%\fi

% Optional override for |\version| flag:
%    \begin{macrocode}
%%\providecommand{\version}{final}
%    \end{macrocode}

% Include the main document:
%    \begin{macrocode}
\input{childdoc.def}
\childdocby{cdocsamp}
%    \end{macrocode}

%\iffalse
%</samplepart3|samplepart4>
%\fi
%
%\iffalse
%<*samplepart3>
%\fi
% Some text for part 3:
%    \begin{macrocode}
some text in part three
%    \end{macrocode}

%\iffalse
%</samplepart3>
%\fi
% Some text for part 4:
%\iffalse
%<*samplepart4>
%\fi
%    \begin{macrocode}
more text in part four
%    \end{macrocode}

%\iffalse
%</samplepart4>
%\fi
%
% %%%%%%%%%%%%%%%%%%%%%%%%%%%%%%%%%%%%%%
% \paragraph{Forwarding for a Complete Draft.}
%
% The following forwarding file |cdocsdrf.tex|
% compiles the main document in draft mode:
%\iffalse
%<*sampledraft>
%\fi
%    \begin{macrocode}
\def\version{draft}
\input{childdoc.def}
\childdocforward{cdocsamp}
%    \end{macrocode}

%\iffalse
%</sampledraft>
%\fi
%
% %%%%%%%%%%%%%%%%%%%%%%%%%%%%%%%%%%%%%%
% \paragraph{Forwarding for Final Version of the Chapters.}
%
% The following forwarding files |cdocsfn1.tex| and |cdocsfn2.tex|
% (with identical content)
% compile the final versions of the child documents
% |cdocsch1.tex| and |cdocsch2.tex|, respectively:
%\iffalse
%<*samplefinal>
%\fi
%    \begin{macrocode}
\def\version{final}
\input{childdoc.def}
\childdocforwardprefix[cdocsamp]{cdocsfn}{cdocsch}
%    \end{macrocode}

%\iffalse
%</samplefinal>
%\fi
%
% %%%%%%%%%%%%%%%%%%%%%%%%%%%%%%%%%%%%%%
% \paragraph{Command Line Processing.}
%
% The following three command lines generate the output files
% |cdocscld|, |cdocscl1| and |cdocscl2|
% which should be identical to
% |cdocsdrf|, |cdocsch1| and |cdocsfn2|, respectively:
% \begin{center}
% \begin{tabular}{l}
% |latex -jobname cdocscld \|\\
% |  "\def\version{draft}\input{childdoc.def}\childdocforward{cdocsamp}"|\\
% |latex -jobname cdocscl1 \|\\
% |  "\input{childdoc.def}\childdocforward[cdocsamp]{cdocsch1}"|\\
% |latex -jobname cdocscl2 \|\\
% |  "\def\version{final}\input{childdoc.def}\childdocforward{cdocsch2}"|
% \end{tabular}
% \end{center}
% Note that the trailing backslash on each first line
% merely continues the input to the second line
% (for convenient cut ant paste).
% Furthermore, the command |latex| can be replaced by any
% of its alternative versions such as |pdflatex|.
%
% %%%%%%%%%%%%%%%%%%%%%%%%%%%%%%%%%%%%%%%%%%%%%%%%%%%%%%%%%%%%%%%%%%%%%%%%%%%%%%
% %%%%%%%%%%%%%%%%%%%%%%%%%%%%%%%%%%%%%%%%%%%%%%%%%%%%%%%%%%%%%%%%%%%%%%%%%%%%%%
% \section{Implementation}
%\iffalse
%<*package>
%\fi
%
% This section describes the definitions file |childdoc.def|.

% The definitions cannot be loaded using |\usepackage| or |\RequirePackage|
% which has a mechanism to prevent loading a style file more than once.
% When loading the definitions by means of |\input|
% multiple instances have to be prevented manually:
%\iffalse
%This code needs to be before the `\ProvidesFile' directive
%which is defined at the beginning of this file.
%Therefore it is also placed there and commented out here.
%</package>
%<*discard>
%\fi
%    \begin{macrocode}
\ifdefined\childdocmain\endinput\fi
%    \end{macrocode}
%\iffalse
%</discard>
%<*package>
%\fi
%
% \macro{\ifchilddoc}
% \macro{\ifchilddocmanual}
% The conditional |\ifchilddoc| tells whether a
% child (true) or main (false) document is being compiled.
% The conditional |\ifchilddocmanual| tells whether
% the |\includeonly| mechanism is used (false) or
% the selection of child files must be performed manually (true).
% The definitions initialise to false:
%    \begin{macrocode}
\newif\ifchilddoc
\newif\ifchilddocmanual
%    \end{macrocode}

% \macro{\childdocname}
% \macro{\childdocjob}
% The macro |\childdocname| stores the name of the main document
% to be compiled. The macro |\childdocjob| stores the name of
% the document on which the \LaTeX{} compiler was originally invoked.
% The content of |\jobname| cannot be compared
% to filenames specified in the source due to different catcodes.
% The following code rescans |\jobname|, stores the result
% in |\childdocname| and saves a copy in |\childdocjob|:
%    \begin{macrocode}
\edef\childdocname{\scantokens\expandafter{\jobname\noexpand}}
\let\childdocjob\childdocname
%    \end{macrocode}

% \macro{\childdocdisable}
% The macro |\childdocdisable| prevents the main file
% from being processed more than once.
% At this stage, the main document command |\childdocmain|
% is assumed to be called once again where it should do nothing.
% Any subsequent call to it should prevent
% a secondary processing of the main document
% It overwrites the forwarding commands
% |\childdocof| and |\childdocforward|
% with empty macros to prevent further inclusions of the main document:
%    \begin{macrocode}
\newcommand{\childdocdisable}
{
  \renewcommand{\childdocmain}[1]{\renewcommand{\childdocmain}[1]{\endinput}}
  \renewcommand{\childdocof}[1]{}
  \renewcommand{\childdocby}[2][]{}
  \renewcommand{\childdocforward}[2][]{}
  \renewcommand{\childdocdisable}{}
}
%    \end{macrocode}

% \macro{\childdocmain}
% The macro |\childdocmain| is to be called at the top of the main file
% with nothing or the main filename (without extension) as argument.
% First, it breaks loops.
% If the argument is not empty and does not match |\childdocname|
% (which is set by the first inclusion of |childdoc.def|),
% |\ifchilddoc| is set to true, |\includeonly| is applied to the child file
% and |\jobname| is set to the main file
% (for proper handling of |.aux| files):
%    \begin{macrocode}
\newcommand{\childdocmain}[1]
{
  \childdocdisable\childdocmain{}
  \if?#1?\else
    \begingroup
      \def\childdoctmp{#1}
      \ifx\childdoctmp\childdocname
        \def\childdoctmp{}
      \else
        \def\childdoctmp
        {
          \childdoctrue
          \includeonly{\childdocname}
          \def\childdocjob{#1}
          \def\jobname{#1}
        }
      \fi
      \expandafter
    \endgroup
    \childdoctmp
  \fi
}
%    \end{macrocode}

% \macro{\childdocof}
% The command |\childdocof| redirects
% compilation to the main file |#1|.
%    \begin{macrocode}
\newcommand{\childdocof}[1]
{
  \childdocdisable
  \childdoctrue
  \includeonly{\childdocname}
  \def\jobname{#1}
  \def\childdocjob{#1}
  \input{#1}
}
%    \end{macrocode}

% \macro{\childdocby}
% The command |\childdocby| ....
%    \begin{macrocode}
\newcommand{\childdocby}[2][]
{
  \childdocdisable
  \childdoctrue
  \childdocmanualtrue
  \if?#1?\else
    \def\jobname{#2}
  \fi
  \def\childdocjob{#2}
  \input{#2}
  \endinput
}
%    \end{macrocode}

% \macro{\childdocforward}
% The command |\childdocforward| redirects
% compilation to the main file or
% (if the optional argument is given) a child file.
% Parameters are set as if the main file
% or a child file starting with |\childdocof| was compiled.
% Then compilation is handed over to the main file:
%    \begin{macrocode}
\newcommand{\childdocforward}[2][]
{
  \begingroup
    \if?#1?
      \def\childdoctmp
      {
        \def\childdocname{#2}
        \def\childdocjob{#2}
        \def\jobname{#2}
        \input{#2}
        \endinput
      }
    \else
      \def\childdoctmp
      {
        \childdocdisable
        \def\childdocname{#2}
        \childdoctrue
        \includeonly{#2}
        \def\childdocjob{#1}
        \def\jobname{#1}
        \input{#1}
        \endinput
      }
    \fi
    \expandafter
  \endgroup
  \childdoctmp
}
%    \end{macrocode}

% \macro{\childdocforwardprefix}
% The command |\childdocforwardprefix| redirects
% compilation to the main or a child file by means of a pattern.
% The prefix |#1| in the current filename is replaced by |#2|
% and the suffix of the current filename is kept
% (it is assumed that the filename does not contain the substring `|~~~|'
% which is used as a delimiter).
% Compilation is handed over to the new file by |\childdocforward|:
%    \begin{macrocode}
\newcommand{\childdocforwardprefix}[3][]
{
  \begingroup
    \def\childdocextract #2##1~~~{\def\childdoctmp{\childdocforward[#1]{#3##1}}}
    \expandafter\childdocextract\childdocname~~~
    \expandafter
  \endgroup
  \childdoctmp
}
%    \end{macrocode}

% \macro{\childdoc}
% The deprecated macro |\childdoc| is a legacy version of |\childdocmain|:
%    \begin{macrocode}
\newcommand{\childdoc}{\childdocmain}
%    \end{macrocode}

% \macro{\childdocredirect}
% The deprecated macro |\childdocredirect| is a legacy version
% of |\childdocforward| and |\childdocforwardprefix|:
%    \begin{macrocode}
\newcommand{\childdocredirect}[2][]
{
  \begingroup
    \if?#1?
      \def\childdoctmp{\childdocforward{#2}}
    \else
      \def\childdoctmp{\childdocforwardprefix{#1}{#2}}
    \fi
    \expandafter
  \endgroup
  \childdoctmp
}
%    \end{macrocode}

%\iffalse
%</package>
%\fi
%
\endinput

\childdocforward{cdocsamp}
%    \end{macrocode}

%\iffalse
%</sampledraft>
%\fi
%
% %%%%%%%%%%%%%%%%%%%%%%%%%%%%%%%%%%%%%%
% \paragraph{Forwarding for Final Version of the Chapters.}
%
% The following forwarding files |cdocsfn1.tex| and |cdocsfn2.tex|
% (with identical content)
% compile the final versions of the child documents
% |cdocsch1.tex| and |cdocsch2.tex|, respectively:
%\iffalse
%<*samplefinal>
%\fi
%    \begin{macrocode}
\def\version{final}
% \iffalse
%
% childdoc.dtx Copyright (C) 2017-2018 Niklas Beisert
%
% This work may be distributed and/or modified under the
% conditions of the LaTeX Project Public License, either version 1.3
% of this license or (at your option) any later version.
% The latest version of this license is in
%   http://www.latex-project.org/lppl.txt
% and version 1.3 or later is part of all distributions of LaTeX
% version 2005/12/01 or later.
%
% This work has the LPPL maintenance status `maintained'.
%
% The Current Maintainer of this work is Niklas Beisert.
%
% This work consists of the files childdoc.dtx and childdoc.ins
% and the derived files childdoc.def and cdocsamp.tex with
% cdocsch1.tex, cdocsch2.tex, cdocsdrf.tex, cdocsfn1.tex, cdocsfn2.tex.
%
%<package>\ifdefined\childdocmain\endinput\fi
%<package>\ProvidesFile{childdoc.def}[2018/12/30 v2.0 child document driver]
%<samplemain>\ProvidesFile{cdocsamp.tex}[2018/12/30 v2.0 sample for childdoc]
%<*driver>
%\ProvidesFile{childdoc.drv}[2018/12/30 v2.0 childdoc reference manual file]
\PassOptionsToClass{10pt,a4paper}{article}
\documentclass{ltxdoc}

\usepackage[margin=35mm]{geometry}
\usepackage{hyperref}
\usepackage{hyperxmp}
\usepackage[usenames]{color}

\hypersetup{colorlinks=true}
\hypersetup{pdfstartview=FitH}
\hypersetup{pdfpagemode=UseNone}
\hypersetup{pdfsource={}}
\hypersetup{pdflang={en-UK}}
\hypersetup{pdfcopyright={Copyright 2017-2018 Niklas Beisert.
  This work may be distributed and/or modified under the
  conditions of the LaTeX Project Public License, either version 1.3
  of this license or (at your option) any later version.}}
\hypersetup{pdflicenseurl={http://www.latex-project.org/lppl.txt}}
\hypersetup{pdfcontactaddress={ETH Zurich, ITP, HIT K,
  Wolfgang-Pauli-Strasse 27}}
\hypersetup{pdfcontactpostcode={8093}}
\hypersetup{pdfcontactcity={Zurich}}
\hypersetup{pdfcontactcountry={Switzerland}}
\hypersetup{pdfcontactemail={nbeisert@itp.phys.ethz.ch}}
\hypersetup{pdfcontacturl={http://people.phys.ethz.ch/\xmptilde nbeisert/}}

\newcommand{\secref}[1]{\hyperref[#1]{section \ref*{#1}}}

\parskip1ex
\parindent0pt
\let\olditemize\itemize
\def\itemize{\olditemize\parskip0pt}

\begin{document}

\title{The \textsf{childdoc} Package}
\hypersetup{pdftitle={The childdoc Package}}
\author{Niklas Beisert\\[2ex]
  Institut f\"ur Theoretische Physik\\
  Eidgen\"ossische Technische Hochschule Z\"urich\\
  Wolfgang-Pauli-Strasse 27, 8093 Z\"urich, Switzerland\\[1ex]
  \href{mailto:nbeisert@itp.phys.ethz.ch}
  {\texttt{nbeisert@itp.phys.ethz.ch}}}
\hypersetup{pdfauthor={Niklas Beisert}}
\hypersetup{pdfsubject={Manual for the LaTeX2e Package childdoc}}
\date{30 December 2018, \textsf{v2.0}}
\maketitle

\begin{abstract}\noindent
\textsf{childdoc} is a \LaTeXe{} package
that enables the direct compilation
of document sections included by |\include|
to individual files.
\end{abstract}

\begingroup
\parskip0ex
\tableofcontents
\endgroup

%%%%%%%%%%%%%%%%%%%%%%%%%%%%%%%%%%%%%%%%%%%%%%%%%%%%%%%%%%%%%%%%%%%%%%%%%%%%%%%%
%%%%%%%%%%%%%%%%%%%%%%%%%%%%%%%%%%%%%%%%%%%%%%%%%%%%%%%%%%%%%%%%%%%%%%%%%%%%%%%%
\section{Introduction}

\LaTeX{} provides a mechanism to structure a large document (such as a book)
into a main file and several child files (containing the chapters)
using the |\include| command.
This mechanism is beneficial for documents
which span hundreds of pages in order to
make the source file(s) more manageable.
Moreover, compilation can be restricted to
selected child files by means of the |\includeonly| command.
The latter feature can be used to reduce the compilation time while editing
(this was significantly more useful in the earlier days of \LaTeX{})
or to generate a smaller document which is easier to navigate.
Another application of |\includeonly| is to generate
documents consisting of selected parts of the complete document.

However, there are a few drawbacks of the plain |\include| mechanism:
\begin{itemize}
\item
The child files cannot be compiled on their own,
they can only be compiled via the main file.
A naive editing environment
(such as a text editor with an option
to have the current file processed by \LaTeX)
may require one to switch to the main file before compiling;
attempting to compile the child file produces errors.
\item
The main file must be modified (each time)
to adjust the |\includeonly| command
to the present needs. This easily leaves the main file in a messy state.
\item
The generated document will always carry the filename
of the main document. This is inconvenient if
several child files are to be compiled and
to be kept for distribution.
\end{itemize}

The present package provides a simple interface
to make child files individually compilable by \LaTeX{}.
Compiling a child file then has the same effect as compiling
the main file with an |\includeonly| command
to select the appropriate child.
Moreover the generated document will carry the name of the child
rather than the main file.
This resolves all three above issues.

This feature is meant to make the editing of books,
thesis documents and lecture notes somewhat more convenient.
However, the package can also be used efficiently for
composing a series of documents (such as exercise sheets)
which are typically distributed individually.
It then assists the author in generating the individual documents
(potentially in different versions)
as well as a document containing the collected series.
Another application is in developing style files
or other kinds of included material
where compilation of the style file could redirect
to a sample or test file.

%%%%%%%%%%%%%%%%%%%%%%%%%%%%%%%%%%%%%%%%%%%%%%%%%%%%%%%%%%%%%%%%%%%%%%%%%%%%%%%%
%%%%%%%%%%%%%%%%%%%%%%%%%%%%%%%%%%%%%%%%%%%%%%%%%%%%%%%%%%%%%%%%%%%%%%%%%%%%%%%%
\section{Usage}

First of all, the package \textsf{childdoc} is \emph{not} a standard
\LaTeXe{} |.sty| style file! Therefore it needs to be invoked in
a non-standard way.

%%%%%%%%%%%%%%%%%%%%%%%%%%%%%%%%%%%%%%%%%%%%%%%%%%%%%%%%%%%%%%%%%%%%%%%%%%%%%%%%
\subsection{Included Files}
\label{sec:include}

%%%%%%%%%%%%%%%%%%%%%%%%%%%%%%%%%%%%%%%%
\DescribeMacro{\childdocmain}
To use the package, add the commands
\begin{center}
\begin{tabular}{l}
|\input{childdoc.def}|\\
|\childdocmain{}|\\
\end{tabular}
\end{center}
at the very top of the main \LaTeX{} file,
in particular \emph{before} the |\documentclass| statement!
The argument of |\childdocmain| should be left empty
(but it must be present).

%%%%%%%%%%%%%%%%%%%%%%%%%%%%%%%%%%%%%%%%
\DescribeMacro{\childdocof}
Furthermore, add the commands
\begin{center}
\begin{tabular}{l}
|\input{childdoc.def}|\\
|\childdocof{|\textit{main}|}|\\
\end{tabular}
\end{center}
at the top of every child file \textit{child}
which is included by |\include{|\textit{child}|}|
from within the main file
(or at least for those files to be compiled individually).
The argument \textit{main} must be the filename of the main file.

There are a couple of
considerations in setting up the main and child documents:

%%%%%%%%%%%%%%%%%%%%%%%%%%%%%%%%%%%%%%%%
\paragraph{Restrictions.}

Please note the following restrictions:
\begin{itemize}
\item
|\childdocmain| must be called with one argument \textit{main}
to ensure compatibility with earlier version of the package.
It must either be empty (|\childdocmain{}|)
or precisely match the filename of the main file in which it is specified.
See \secref{sec:detection} for further information.
\item
The filename \textit{main} must be specified without the |.tex| extension.
\item
The filename \textit{main} is case sensitive
(even in case-insensitive file systems)
due to internal string comparison.
\item
The argument \textit{main} should be fully expanded, it cannot be a macro.
\item
Subdirectories and special characters should be avoided in filenames.
\item
The command |\childdocmain{|\textit{main}|}| must be followed by a whitespace.
It should not be followed immediately by another command
or by a comment mark `|%|'.
This is because the \TeX{} parser reads the token immediately following
the argument of |\childdocmain| and puts it
at the beginning of every child section;
however, a white\-space is ignored.
\end{itemize}

%%%%%%%%%%%%%%%%%%%%%%%%%%%%%%%%%%%%%%%%
\paragraph{Content of Main File.}

It is advisable to place all content in the child files included by |\include|.
Any output contained in the main file will appear in all child documents
unless suppressed manually;
it cannot be suppressed automatically by the |\includeonly| directive
and thus should normally be avoided.
A method to include some content in the main file
by means of conditional processing is described in \secref{sec:conditional}.

%%%%%%%%%%%%%%%%%%%%%%%%%%%%%%%%%%%%%%%%
\paragraph{Page Numbering.}

When only a part of the document is compiled,
the appropriate numbering of pages
(as well as other status parameters)
is determined from the |.aux| files.
The latter contain information from previous passes.
However this information needs to propagate through
all intermediate child documents.
Therefore the page numbering in child documents may well
be inconsistent until the complete document is compiled at least once.

A useful (if unconventional) way to always ensure a consistent
page numbering is to restart the numbering in each child document
and denote the pages by `\textit{child}|.|\textit{page}'
where \textit{child} represents the chapter/section number of the child file.
This can be achieved by the command
|\numberwithin{page}{|\textit{child}|}|
of the \textsf{amsmath} package
where \textit{child} can be |chapter| or |section|
depending on the chosen structuring.
Alternatively, one can modify the macro |\thepage| appropriately
and reset the counter |page| at the start of each child file.

%%%%%%%%%%%%%%%%%%%%%%%%%%%%%%%%%%%%%%%%%%%%%%%%%%%%%%%%%%%%%%%%%%%%%%%%%%%%%%%%
\subsection{Conditional Processing}
\label{sec:conditional}

The package provides a mechanism to compile different versions
of a document. To customise the versions further some conditional processing
can come in handy to distinguish which version is being compiled.
The package provides two macros to describe the compilation context:

%%%%%%%%%%%%%%%%%%%%%%%%%%%%%%%%%%%%%%%%
\DescribeMacro{\ifchilddoc}
The conditional |\ifchilddoc| distinguishes between the compilation of
child documents and the main document:
%
\begin{center}
|\ifchilddoc |\textit{child-code}| |[|\||else |\textit{main-code}]| \||fi|
\end{center}

%%%%%%%%%%%%%%%%%%%%%%%%%%%%%%%%%%%%%%%%
\DescribeMacro{\childdocname}
\DescribeMacro{\childdocjob}
The macro |\childdocname| contains the filename (without extension)
of the main or child file being processed.
Note that |\childdocjob| will always contain the name of the main file.

%%%%%%%%%%%%%%%%%%%%%%%%%%%%%%%%%%%%%%%%
\paragraph{Title Page.}

Conditional processing can be used to include a title or banner page
in the main document when proper precautions are taken.
Importantly, the code in the main file should ensure that the page counter
(as well as other status parameters which are stored in the |.aux| files)
takes the same value after the conditional processing.
Otherwise the page numbers may take divergent values
depending on which part is compiled.

For example, a title page could be declared by:
%
\begin{center}
\begin{tabular}{l}
|\ifchilddoc\||else|\\
|\addtocounter{page}{-1}|\\
\textit{code for title page}\\
|\newpage|\\
|\||fi|
\end{tabular}
\end{center}
%
A banner page for the child documents can be generated by:
%
\begin{center}
\begin{tabular}{l}
|\ifchilddoc|\\
|\addtocounter{page}{-1}|\\
\textit{code for banner page}\\
|\newpage|\\
|\||fi|
\end{tabular}
\end{center}
%
Here one could write a message such as:
\begin{center}
|This is the part \childdocname{} of \childdocjob{}.|
\end{center}

%%%%%%%%%%%%%%%%%%%%%%%%%%%%%%%%%%%%%%%%%%%%%%%%%%%%%%%%%%%%%%%%%%%%%%%%%%%%%%%%
\subsection{Flags}
\label{sec:flags}

The package makes it easy to generate different versions
of the main or child documents.
To this end compilation flags can be defined
and assigned different default values.
They will be particularly useful in conjunction
with the forwarding mechanism described in \secref{sec:forward}.

For example, it may be useful to have a flag |\version|
which can be set to |draft| or |final|.
The document source will contain some conditional code
depending on the value of |\version|.
Suppose further, the flag should default to |final| for the main file
and to |draft| for child files
which is a natural assignment for editing the document.
This is achieved by placing the following code
in the preamble of the main document
(below the |\childdocmain| directive):
%
\begin{center}
\begin{tabular}{l}
|\ifchilddoc|\\
|\providecommand{\version}{draft}|\\
|\||else|\\
|\providecommand{\version}{final}|\\
|\||fi|
\end{tabular}
\end{center}
%
The definition by |\providecommand| makes sure
that previous definitions are not overwritten.
Further statements |\providecommand{\version}{...}|
can thus be added before the above code to override it.

For the main file, one might add a line
(between |\childdocmain| and the above block)
%
\begin{center}
|%\ifchilddoc\||else\providecommand{\version}{draft}\||fi|
\end{center}
%
which can be uncommented to produce a draft version.
Likewise one can add a line to the very top of a child file
(above the |\childdocof{|\textit{main}|}| directive)
%
\begin{center}
|%\providecommand{\version}{final}|
\end{center}
%
which can be uncommented to produce the final version of this child document.

%%%%%%%%%%%%%%%%%%%%%%%%%%%%%%%%%%%%%%%%%%%%%%%%%%%%%%%%%%%%%%%%%%%%%%%%%%%%%%%%
\subsection{Forwarding}
\label{sec:forward}

Different versions of the main or child documents
using compilation flags as described in \secref{sec:flags}
can be (permanently) stored in different files
for convenient compilation, viewing and distribution.
To this end, the package defines a command
to pass on compilation to a different file:

%%%%%%%%%%%%%%%%%%%%%%%%%%%%%%%%%%%%%%%%
\DescribeMacro{\childdocforward}
The command |\childdocforward| redirects processing to
another source file:
%
\begin{center}
\begin{tabular}{l}
|\input{childdoc.def}|\\
|\childdocforward[|\textit{main}|]{|\textit{dest}|}|\\
\end{tabular}
\end{center}
%
The argument \textit{dest} is the destination file
(without extension).
It should be the main file or one of the child files.
Note that further \textsf{childdoc} directives
such as |\childdocof| and |\childdocforward|
in the indicated file will be processed in this form.
The optional argument \textit{main}
passes on directly to the main file \textit{main}
while pretending to compile the child \textit{dest}.
This form behaves as if \textit{dest}
issues |\childdocof{|\textit{main}|}| right away,
and no further \textsf{childdoc} directives will be processed.

%%%%%%%%%%%%%%%%%%%%%%%%%%%%%%%%%%%%%%%%
\DescribeMacro{\...prefix}
In the alternative form |\childdocforwardprefix|,
%
\begin{center}
\begin{tabular}{l}
|\input{childdoc.def}|\\
|\childdocforwardprefix[|\textit{main}|]{|\textit{prefix}|}{|\textit{dest}|}|
\end{tabular}
\end{center}
%
the destination file is determined by a pattern
depending on the current file:
To make this work, the current file must be called
`{\textit{prefix}\hspace{0.2em}\textit{suffix}}'
with \textit{prefix} matching precisely the argument.
Processing is then passed on to the file
`{\textit{dest}\hspace{0.2em}\textit{suffix}}'.
Surely, the same effect is achieved by
directly specifying the
argument `{\textit{dest}\hspace{0.2em}\textit{suffix}}'
in the first form.
However, that requires to set up a different file
for each child. With the alternative form of the command
all these files can have exactly the same content
which simplifies setting them up and maintaining them.

For example, the following file |draft.tex|
with a compilation flag |\version| as described in \secref{sec:flags}
compiles the main document as a draft:
%
\begin{center}
\begin{tabular}{l}
|\def\version{draft}|\\
|\input{childdoc.def}|\\
|\childdocforward{|\textit{main}|}|
\end{tabular}
\end{center}
%
Likewise, the following files |final|\textit{nn}|.tex|
compile the final version of the child document
|child|\textit{nn}|.tex|:
%
\begin{center}
\begin{tabular}{l}
|\def\version{final}|\\
|\input{childdoc.def}|\\
|\childdocforwardprefix{final}{child}|
\end{tabular}
\end{center}
%

Note that when several versions of a main file and/or of each child file
are to be generated, it may be convenient to set up a |Makefile| or
shell script to automatise the process.

%%%%%%%%%%%%%%%%%%%%%%%%%%%%%%%%%%%%%%%%%%%%%%%%%%%%%%%%%%%%%%%%%%%%%%%%%%%%%%%%
\subsection{Command Line Processing}
\label{sec:commandline}

The effect of redirection files can also be achieved by invoking
the \LaTeX{} compiler with a more elaborate command line.
Most conveniently this should be done as part
of a shell script or a |Makefile|.

When using \textsf{childdoc} in the main file, the following
command lines effectively perform a redirection
(note that depending on the shell being used,
backslashes may have to be doubled: `|\|' $\to$ `|\\|'):
%
\begin{center}
|... -jobname "|\textit{target}|" |\\|"|[\textit{flags}]%
|\input{childdoc.def}\childdocforward[|\textit{main}|]{|\textit{dest}|}"|
\end{center}
%
Here \textit{target} is the name of the output file,
\textit{main} is the name of the main file
and \textit{dest} is the name of the main or child file to be processed
(all filenames without extensions).
The optional argument \textit{main} can be omitted
if \textit{main} matches \textit{dest}.
Optionally, compilation \textit{flags} can be defined via |\def| commands.
This command line makes the \TeX{} engine believe
it is compiling the file \textit{target}
whose content is specified as the latter parameter.
The provided code then forwards the processing to
\textit{main} or \textit{dest} as described in \secref{sec:forward}.

%%%%%%%%%%%%%%%%%%%%%%%%%%%%%%%%%%%%%%%%%%%%%%%%%%%%%%%%%%%%%%%%%%%%%%%%%%%%%%%%
\subsection{Include by Input}
\label{sec:input}

Including child documents by |\include| has some restrictions by design.
Most notably, the content of a child document always occupies
its own set of pages; pages cannot be shared between child documents.
Usually, this behaviour makes perfect sense
because each child document contain an essential part of the document.
However, in some situations it may be desirable to compose
a document from a collection of parts
without having mandatory page breaks between then.
For this case, the package
provides a mechanism to include parts
by |\input| which can also be processed individually.
However, by construction this mechanism
requires manual handling of the content to be output.

%%%%%%%%%%%%%%%%%%%%%%%%%%%%%%%%%%%%%%%%
\DescribeMacro{\ifchilddocmanual}
The main file should be prepared as usual, see \secref{sec:include}.
However, the document body must make a distinction
between processing of an individual part and of the main document, e.g.:
%
\begin{center}
\begin{tabular}{l}
|\ifchilddocmanual|\\
|\input{\childdocname}|\\
|\||else|\\
\textit{document body with }|\input{|\textit{part}|}|\\
|\||fi|
\end{tabular}
\end{center}
%
The conditional |\ifchilddocmanual| is true whenever
a part to be included by |\input| is being compiled,
and the name of the part is stored in |\childdocname|.

%%%%%%%%%%%%%%%%%%%%%%%%%%%%%%%%%%%%%%%%
\DescribeMacro{\childdocby}
Each part to be included by |\input| should start with:
%
\begin{center}
\begin{tabular}{l}
|\input{childdoc.def}|\\
|\childdocby{|\textit{main}|}|\\
\end{tabular}
\end{center}
%
The directive |\childdocby| is similar to |\childdocof|
described in \secref{sec:include},
but the subsequent selection of content must be done manually.
To that end, both |\ifchilddoc| and |\ifchilddocmanual|
will be true upon processing of a part,
and the name of the part is stored in |\childdocname|.
Note that |\jobname| will be set to the filename of the current part
so that each part receives an individual |.aux| file
that does not interfere with the |.aux| file(s) of the main document.
This behaviour can be altered by the alternative form
|\childdocby[*]{|\textit{main}|}| (with a non-empty optional argument)
which uses the |.aux| file of the main document
by setting |\jobname| to \textit{main}.

%%%%%%%%%%%%%%%%%%%%%%%%%%%%%%%%%%%%%%%%%%%%%%%%%%%%%%%%%%%%%%%%%%%%%%%%%%%%%%%%
\subsection{Driver Development}
\label{sec:driver}

The \textsf{childdoc} mechanism can also be use for the development
of definition files such as \LaTeX{} styles or classes.
This case differs from the above setup with multiple parts
included by |\include| in that no |\includeonly| should be invoked.
This can be achieved by starting the include file
(before |\ProvidesPackage|) with:
%
\begin{center}
\begin{tabular}{l}
|\input{childdoc.def}|\\
|\childdocforward{|\textit{main}|}|\\
\end{tabular}
\end{center}
%
or alternatively with:
%
\begin{center}
\begin{tabular}{l}
|\input{childdoc.def}|\\
|\childdocby{|\textit{main}|}|\\
\end{tabular}
\end{center}
%
Both forms have slightly different effects as described above.
The main file is prepared as usual, see \secref{sec:include}.

%%%%%%%%%%%%%%%%%%%%%%%%%%%%%%%%%%%%%%%%%%%%%%%%%%%%%%%%%%%%%%%%%%%%%%%%%%%%%%%%
\subsection{Legacy Detection}
\label{sec:detection}

The directive |\childdocmain| in the main file can detect
whether the complete document or merely a child is to be compiled
even without using the directive |\childdocof|.
This method is deprecated because it is less robust
and there is no compelling reason to use it;
it is merely provided for backward compatibility
and it may be removed in future versions.

If the detection mechanism is to be used,
it is mandatory to correctly specify
the filename of the main file as the argument of |\childdocmain|:
%
\begin{center}
\begin{tabular}{l}
|\input{childdoc.def}|\\
|\childdocmain{|\textit{main}|}|\\
\end{tabular}
\end{center}
%
If |\jobname| does not match the argument \textit{main} of |\childdocmain|,
it is assumed that |\jobname| points to the child file to be compiled.
When using |\childdocmain| with the main file specified as argument,
it suffices to start a child file
with just |\input{|\textit{main}|}|
without loading of the package and using |\childdocof|.
If instead all processing is done
with the appropriate \textsf{childdoc} directives,
the argument of \textit{main} of |\childdocmain| can be empty.

An alternative version of the command line processing described
in \secref{sec:commandline} using the detection mechanism reads:
%
\begin{center}
|... -jobname "|\textit{target}|" "|[\textit{flags}]%
[|\def\jobname{|\textit{dest}|}|]|\input{|\textit{main}|}"|
\end{center}

%%%%%%%%%%%%%%%%%%%%%%%%%%%%%%%%%%%%%%%%%%%%%%%%%%%%%%%%%%%%%%%%%%%%%%%%%%%%%%%%
\subsection{Manual Code}
\label{sec:manual}

In case one cannot be certain whether the definitions file |childdoc.def|
is installed on the target \TeX{} distribution
and one prefers not to ship it,
it is conceivable to paste a few relevant commands into the sources.

To that end, drop all statements |\input{childdoc.def}|
and perform the replacements as outlined below.
Instead of |\childdocmain{|\textit{main}|}| add the following code
to the top of the main file:
%
\begin{center}
\begin{tabular}{l}
|\||ifdefined\childdocname\endinput\||fi\newif\ifchilddoc|\\
|\edef\childdocname{\scantokens\expandafter{\jobname\noexpand}}|\\
|\def\childdocmain{|\textit{main}|}\||ifx\childdocmain\childdocname\||else|\\
|\childdoctrue\includeonly{\childdocname}\let\jobname\childdocmain\||fi|\\
\end{tabular}
\end{center}
%
Instead of |\childdocof{|\textit{main}|}| just include the main file
at the top of each child file:
%
\begin{center}
|\input{|\textit{main}|}|
\end{center}
%
A simple redirection |\childdocforward{|\textit{dest}|}| is achieved by:
%
\begin{center}
|\def\jobname{|\textit{dest}|}\input{\jobname}|
\end{center}
%
The redirection with prefix
|\childdocforwardprefix[|\textit{prefix}|]{|\textit{dest}|}|
is accomplished by:
%
\begin{center}
\begin{tabular}{l}
|{\edef\jobname{\scantokens\expandafter{\jobname\noexpand}}|\\
|\def\redirectjob |\textit{prefix}|#1~~~{\gdef\jobname{|\textit{dest}|#1}}|\\
|\expandafter\redirectjob\jobname~~~}\input{\jobname}|
\end{tabular}
\end{center}

In an alternative approach,
child documents can be compiled by a specific command line
without additional code or specific definitions:
%
\begin{center}
|... -jobname "|\textit{target}|" "|[\textit{flags}]%
|\includeonly{|\textit{dest}|}\input{|\textit{main}|}"|
\end{center}
%

%%%%%%%%%%%%%%%%%%%%%%%%%%%%%%%%%%%%%%%%%%%%%%%%%%%%%%%%%%%%%%%%%%%%%%%%%%%%%%%%
%%%%%%%%%%%%%%%%%%%%%%%%%%%%%%%%%%%%%%%%%%%%%%%%%%%%%%%%%%%%%%%%%%%%%%%%%%%%%%%%
\section{Information}

%%%%%%%%%%%%%%%%%%%%%%%%%%%%%%%%%%%%%%%%%%%%%%%%%%%%%%%%%%%%%%%%%%%%%%%%%%%%%%%%
\subsection{Copyright}

Copyright \copyright{} 2017--2018 Niklas Beisert

This work may be distributed and/or modified under the
conditions of the \LaTeX{} Project Public License, either version 1.3
of this license or (at your option) any later version.
The latest version of this license is in
  \url{http://www.latex-project.org/lppl.txt}
and version 1.3 or later is part of all distributions of \LaTeX{}
version 2005/12/01 or later.

This work has the LPPL maintenance status `maintained'.

The Current Maintainer of this work is Niklas Beisert.

This work consists of the files |README.txt|, |childdoc.ins| and |childdoc.dtx|
as well as the derived files |childdoc.def|, |cdocsamp.tex|
with |cdocsch1.tex|, |cdocsch2.tex|, |cdocspt3.tex|, |cdocspt4.tex|,
|cdocsdrf.tex|, |cdocsfn1.tex|, |cdocsfn2.tex|
as well as |childdoc.pdf|.

%%%%%%%%%%%%%%%%%%%%%%%%%%%%%%%%%%%%%%%%%%%%%%%%%%%%%%%%%%%%%%%%%%%%%%%%%%%%%%%%
\subsection{Files and Installation}

The package consists of the files:
%
\begin{center}
\begin{tabular}{ll}
    |README.txt|   & readme file \\
    |childdoc.ins| & installation file \\
    |childdoc.dtx| & source file \\
    |childdoc.def| & definition file \\
    |cdocsamp.tex| & sample main file \\
    |cdocsch1.tex| & sample include file \\
    |cdocsch2.tex| & sample include file \\
    |cdocspt3.tex| & sample part file \\
    |cdocspt4.tex| & sample part file \\
    |cdocsdrf.tex| & sample redirection file \\
    |cdocsfn1.tex| & sample redirection file \\
    |cdocsfn2.tex| & sample redirection file \\
    |childdoc.pdf| & manual
\end{tabular}
\end{center}
%
The distribution consists of the files
|README.txt|, |childdoc.ins| and |childdoc.dtx|.
%
\begin{itemize}
\item
Run (pdf)\LaTeX{} on |childdoc.dtx|
to compile the manual |childdoc.pdf| (this file).
\item
Run \LaTeX{} on |childdoc.ins| to create the definitions file |childdoc.def|
and the sample |cdocsamp.tex| with include files
|cdocsch1.tex|, |cdocsch2.tex|, |cdocspt3.tex|, |cdocspt4.tex|,
|cdocsdrf.tex|, |cdocsfn1.tex|, |cdocsfn2.tex|.
Then copy the file |childdoc.def| to an appropriate directory of your \LaTeX{}
distribution, e.g.\ \textit{texmf-root}|/tex/latex/childdoc|.
\end{itemize}

%%%%%%%%%%%%%%%%%%%%%%%%%%%%%%%%%%%%%%%%%%%%%%%%%%%%%%%%%%%%%%%%%%%%%%%%%%%%%%%%
\subsection{Related CTAN Packages}

There are several other packages which offer a similar functionality:
%
\begin{itemize}
\item
The packages
\href{http://ctan.org/pkg/docmute}{\textsf{docmute}},
\href{http://ctan.org/pkg/includex}{\textsf{includex}} and
\href{http://ctan.org/pkg/standalone}{\textsf{standalone}}
provide commands to include only the document body of
a child file thus allowing both files to be compiled individually.
\item
The packages \href{http://ctan.org/pkg/subdocs}{\textsf{subdocs}}
and \href{http://ctan.org/pkg/subfiles}{\textsf{subfiles}}
provide structures in which the main and child documents can be
encapsulated and allowing them to be compiled individually.
The inclusion mechanism is different from the conventional |\include|.
\item
The package \href{http://ctan.org/pkg/combine}{\textsf{combine}}
is an elaborate solution to combine several documents into one.
\end{itemize}
%
See also the CTAN topic \href{http://ctan.org/topic/subdocs}{\textsf{subdocs}}
for further related packages.
The present package differs from the above solutions in that
a document structure constructed with the conventional |\include| mechanism
just needs two extra commands at the top of every file
such that all constituent files can be compiled individually.

%%%%%%%%%%%%%%%%%%%%%%%%%%%%%%%%%%%%%%%%%%%%%%%%%%%%%%%%%%%%%%%%%%%%%%%%%%%%%%%%
%\subsection{Feature Suggestions}
%
%The following is a list of features which may be useful for future
%versions of this package:
%%
%\begin{itemize}
%\item
%\ldots
%\end{itemize}

%%%%%%%%%%%%%%%%%%%%%%%%%%%%%%%%%%%%%%%%%%%%%%%%%%%%%%%%%%%%%%%%%%%%%%%%%%%%%%%%
\subsection{Revision History}

%%%%%%%%%%%%%%%%%%%%%%%%%%%%%%%%%%%%%%%%
\paragraph{v2.0:} 2018/12/30

\begin{itemize}
\item
immediate forward processing
\item
added |\childdocby| mechanism
\item
manual restructured
\end{itemize}

%%%%%%%%%%%%%%%%%%%%%%%%%%%%%%%%%%%%%%%%
\paragraph{v1.6:} 2018/01/17

\begin{itemize}
\item
application for development of include files
\item
corrections to manual
\end{itemize}

%%%%%%%%%%%%%%%%%%%%%%%%%%%%%%%%%%%%%%%%
\paragraph{v1.5:} 2017/05/21

\begin{itemize}
\item
more complete structuring introduced
\item
|\childdocof| introduced
\item
|\childdoc| renamed to |\childdocmain|
\item
|\childredirect| renamed to |\childdocforward| and |\childdocforwardprefix|
and functionality expanded
\end{itemize}

%%%%%%%%%%%%%%%%%%%%%%%%%%%%%%%%%%%%%%%%
\paragraph{v1.0:} 2017/04/27

\begin{itemize}
\item
manual and install package
\item
first version published on CTAN
\end{itemize}

%%%%%%%%%%%%%%%%%%%%%%%%%%%%%%%%%%%%%%%%
\paragraph{v0.6:} 2017/04/26

\begin{itemize}
\item
redirection mechanism added
\end{itemize}

%%%%%%%%%%%%%%%%%%%%%%%%%%%%%%%%%%%%%%%%
\paragraph{v0.5:} 2017/04/26

\begin{itemize}
\item
functionality in definition file
\end{itemize}


%%%%%%%%%%%%%%%%%%%%%%%%%%%%%%%%%%%%%%%%%%%%%%%%%%%%%%%%%%%%%%%%%%%%%%%%%%%%%%%%
%%%%%%%%%%%%%%%%%%%%%%%%%%%%%%%%%%%%%%%%%%%%%%%%%%%%%%%%%%%%%%%%%%%%%%%%%%%%%%%%
%%%%%%%%%%%%%%%%%%%%%%%%%%%%%%%%%%%%%%%%%%%%%%%%%%%%%%%%%%%%%%%%%%%%%%%%%%%%%%%%
\appendix

\settowidth\MacroIndent{\rmfamily\scriptsize 000\ }

 \DocInput{childdoc.dtx}

\end{document}
%</driver>
% \fi
%
% %%%%%%%%%%%%%%%%%%%%%%%%%%%%%%%%%%%%%%%%%%%%%%%%%%%%%%%%%%%%%%%%%%%%%%%%%%%%%%
% %%%%%%%%%%%%%%%%%%%%%%%%%%%%%%%%%%%%%%%%%%%%%%%%%%%%%%%%%%%%%%%%%%%%%%%%%%%%%%
% \section{Sample}
%\iffalse
%<*samplemain>
%\fi
%
% The following presents a sample document
% with two chapters, two parts, a title page,
% a compile flag as well as three forwarding files to set the flag.
% It consists of eight |.tex| files:
% \begin{center}
% \begin{tabular}{ll}
% |cdocsamp.tex|&main file\\
% |cdocsch1.tex|&include file for chapter 1\\
% |cdocsch2.tex|&include file for chapter 2\\
% |cdocspt3.tex|&include file for part 3\\
% |cdocspt4.tex|&include file for part 4\\
% |cdocsdrf.tex|&forwarding file for main file in draft mode\\
% |cdocsfi1.tex|&forwarding file for final version of chapter 1\\
% |cdocsfi2.tex|&forwarding file for final version of chapter 2\\
% \end{tabular}
% \end{center}
% Each of the eight files can be compiled directly by the \LaTeX{} compiler.
%
% %%%%%%%%%%%%%%%%%%%%%%%%%%%%%%%%%%%%%%
% \paragraph{Main File.}
%
% The main file is called |cdocsamp.tex|.
%
% Load the \textsf{childdoc} definitions and
% declare the filename for the main document:
%    \begin{macrocode}
\input{childdoc.def}
\childdocmain{}
%    \end{macrocode}

% Optional override for |\version| flag:
%    \begin{macrocode}
%%\ifchilddoc\else\providecommand{\version}{draft}\fi
%    \end{macrocode}

% Define the default values for the |\version| flag
% (|final| for the main file and |draft| for childs):
%    \begin{macrocode}
\ifchilddoc
\providecommand{\version}{draft}
\else
\providecommand{\version}{final}
\fi
%    \end{macrocode}

% Load the standard document class:
%    \begin{macrocode}
\documentclass[12pt]{article}
%    \end{macrocode}

% Start the document body:
%    \begin{macrocode}
\begin{document}
%    \end{macrocode}

% Declare a title page.
% Print title, part of document being processed and version flag:
%    \begin{macrocode}
\addtocounter{page}{-1}
\begin{center}
{\LARGE\bfseries{}childdoc example\par}
\vspace{1cm}
\ifchilddoc
\ifchilddocmanual part\else chapter\fi:
`\childdocname' of `\childdocjob'\par
\else
main document: `\childdocjob'\par
\fi
version: \version\par
\end{center}
\newpage
%    \end{macrocode}

% Manually include selected file,
% otherwise process as usual:
%    \begin{macrocode}
\ifchilddocmanual
\section*{part `\childdocname'}
\input{\childdocname}
\else
%    \end{macrocode}

% Include the two chapters:
%    \begin{macrocode}
\include{cdocsch1}
\include{cdocsch2}
%    \end{macrocode}

% Include the two parts unless only chapters should be displayed:
%    \begin{macrocode}
\ifchilddoc\else
\section{part three}
\input{cdocspt3}
\section{part four}
\input{cdocspt4}
\fi
%    \end{macrocode}

% Process as usual until here:
%    \begin{macrocode}
\fi
%    \end{macrocode}

% End of document body:
%    \begin{macrocode}
\end{document}
%    \end{macrocode}
%\iffalse
%</samplemain>
%\fi
%
% %%%%%%%%%%%%%%%%%%%%%%%%%%%%%%%%%%%%%%
% \paragraph{Chapter Include Files.}
%
% The include files are called |cdocsch1.tex| and |cdocsch2.tex|.
%
%\iffalse
%<*samplechap1|samplechap2>
%\fi

% Optional override for |\version| flag:
%    \begin{macrocode}
%%\providecommand{\version}{final}
%    \end{macrocode}

% Include the main document:
%    \begin{macrocode}
\input{childdoc.def}
\childdocof{cdocsamp}
%    \end{macrocode}

%\iffalse
%</samplechap1|samplechap2>
%\fi
%
%\iffalse
%<*samplechap1>
%\fi
% Some text for chapter 1:
%    \begin{macrocode}
\section{one}
some text in chapter one
%    \end{macrocode}

%\iffalse
%</samplechap1>
%\fi
% Some text for chapter 2:
%\iffalse
%<*samplechap2>
%\fi
%    \begin{macrocode}
\section{two}
more text in chapter two
%    \end{macrocode}

%\iffalse
%</samplechap2>
%\fi
%
% %%%%%%%%%%%%%%%%%%%%%%%%%%%%%%%%%%%%%%
% \paragraph{Part Include Files.}
%
% The include files are called |cdocspt3.tex| and |cdocspt4.tex|.
%
%\iffalse
%<*samplepart3|samplepart4>
%\fi

% Optional override for |\version| flag:
%    \begin{macrocode}
%%\providecommand{\version}{final}
%    \end{macrocode}

% Include the main document:
%    \begin{macrocode}
\input{childdoc.def}
\childdocby{cdocsamp}
%    \end{macrocode}

%\iffalse
%</samplepart3|samplepart4>
%\fi
%
%\iffalse
%<*samplepart3>
%\fi
% Some text for part 3:
%    \begin{macrocode}
some text in part three
%    \end{macrocode}

%\iffalse
%</samplepart3>
%\fi
% Some text for part 4:
%\iffalse
%<*samplepart4>
%\fi
%    \begin{macrocode}
more text in part four
%    \end{macrocode}

%\iffalse
%</samplepart4>
%\fi
%
% %%%%%%%%%%%%%%%%%%%%%%%%%%%%%%%%%%%%%%
% \paragraph{Forwarding for a Complete Draft.}
%
% The following forwarding file |cdocsdrf.tex|
% compiles the main document in draft mode:
%\iffalse
%<*sampledraft>
%\fi
%    \begin{macrocode}
\def\version{draft}
\input{childdoc.def}
\childdocforward{cdocsamp}
%    \end{macrocode}

%\iffalse
%</sampledraft>
%\fi
%
% %%%%%%%%%%%%%%%%%%%%%%%%%%%%%%%%%%%%%%
% \paragraph{Forwarding for Final Version of the Chapters.}
%
% The following forwarding files |cdocsfn1.tex| and |cdocsfn2.tex|
% (with identical content)
% compile the final versions of the child documents
% |cdocsch1.tex| and |cdocsch2.tex|, respectively:
%\iffalse
%<*samplefinal>
%\fi
%    \begin{macrocode}
\def\version{final}
\input{childdoc.def}
\childdocforwardprefix[cdocsamp]{cdocsfn}{cdocsch}
%    \end{macrocode}

%\iffalse
%</samplefinal>
%\fi
%
% %%%%%%%%%%%%%%%%%%%%%%%%%%%%%%%%%%%%%%
% \paragraph{Command Line Processing.}
%
% The following three command lines generate the output files
% |cdocscld|, |cdocscl1| and |cdocscl2|
% which should be identical to
% |cdocsdrf|, |cdocsch1| and |cdocsfn2|, respectively:
% \begin{center}
% \begin{tabular}{l}
% |latex -jobname cdocscld \|\\
% |  "\def\version{draft}\input{childdoc.def}\childdocforward{cdocsamp}"|\\
% |latex -jobname cdocscl1 \|\\
% |  "\input{childdoc.def}\childdocforward[cdocsamp]{cdocsch1}"|\\
% |latex -jobname cdocscl2 \|\\
% |  "\def\version{final}\input{childdoc.def}\childdocforward{cdocsch2}"|
% \end{tabular}
% \end{center}
% Note that the trailing backslash on each first line
% merely continues the input to the second line
% (for convenient cut ant paste).
% Furthermore, the command |latex| can be replaced by any
% of its alternative versions such as |pdflatex|.
%
% %%%%%%%%%%%%%%%%%%%%%%%%%%%%%%%%%%%%%%%%%%%%%%%%%%%%%%%%%%%%%%%%%%%%%%%%%%%%%%
% %%%%%%%%%%%%%%%%%%%%%%%%%%%%%%%%%%%%%%%%%%%%%%%%%%%%%%%%%%%%%%%%%%%%%%%%%%%%%%
% \section{Implementation}
%\iffalse
%<*package>
%\fi
%
% This section describes the definitions file |childdoc.def|.

% The definitions cannot be loaded using |\usepackage| or |\RequirePackage|
% which has a mechanism to prevent loading a style file more than once.
% When loading the definitions by means of |\input|
% multiple instances have to be prevented manually:
%\iffalse
%This code needs to be before the `\ProvidesFile' directive
%which is defined at the beginning of this file.
%Therefore it is also placed there and commented out here.
%</package>
%<*discard>
%\fi
%    \begin{macrocode}
\ifdefined\childdocmain\endinput\fi
%    \end{macrocode}
%\iffalse
%</discard>
%<*package>
%\fi
%
% \macro{\ifchilddoc}
% \macro{\ifchilddocmanual}
% The conditional |\ifchilddoc| tells whether a
% child (true) or main (false) document is being compiled.
% The conditional |\ifchilddocmanual| tells whether
% the |\includeonly| mechanism is used (false) or
% the selection of child files must be performed manually (true).
% The definitions initialise to false:
%    \begin{macrocode}
\newif\ifchilddoc
\newif\ifchilddocmanual
%    \end{macrocode}

% \macro{\childdocname}
% \macro{\childdocjob}
% The macro |\childdocname| stores the name of the main document
% to be compiled. The macro |\childdocjob| stores the name of
% the document on which the \LaTeX{} compiler was originally invoked.
% The content of |\jobname| cannot be compared
% to filenames specified in the source due to different catcodes.
% The following code rescans |\jobname|, stores the result
% in |\childdocname| and saves a copy in |\childdocjob|:
%    \begin{macrocode}
\edef\childdocname{\scantokens\expandafter{\jobname\noexpand}}
\let\childdocjob\childdocname
%    \end{macrocode}

% \macro{\childdocdisable}
% The macro |\childdocdisable| prevents the main file
% from being processed more than once.
% At this stage, the main document command |\childdocmain|
% is assumed to be called once again where it should do nothing.
% Any subsequent call to it should prevent
% a secondary processing of the main document
% It overwrites the forwarding commands
% |\childdocof| and |\childdocforward|
% with empty macros to prevent further inclusions of the main document:
%    \begin{macrocode}
\newcommand{\childdocdisable}
{
  \renewcommand{\childdocmain}[1]{\renewcommand{\childdocmain}[1]{\endinput}}
  \renewcommand{\childdocof}[1]{}
  \renewcommand{\childdocby}[2][]{}
  \renewcommand{\childdocforward}[2][]{}
  \renewcommand{\childdocdisable}{}
}
%    \end{macrocode}

% \macro{\childdocmain}
% The macro |\childdocmain| is to be called at the top of the main file
% with nothing or the main filename (without extension) as argument.
% First, it breaks loops.
% If the argument is not empty and does not match |\childdocname|
% (which is set by the first inclusion of |childdoc.def|),
% |\ifchilddoc| is set to true, |\includeonly| is applied to the child file
% and |\jobname| is set to the main file
% (for proper handling of |.aux| files):
%    \begin{macrocode}
\newcommand{\childdocmain}[1]
{
  \childdocdisable\childdocmain{}
  \if?#1?\else
    \begingroup
      \def\childdoctmp{#1}
      \ifx\childdoctmp\childdocname
        \def\childdoctmp{}
      \else
        \def\childdoctmp
        {
          \childdoctrue
          \includeonly{\childdocname}
          \def\childdocjob{#1}
          \def\jobname{#1}
        }
      \fi
      \expandafter
    \endgroup
    \childdoctmp
  \fi
}
%    \end{macrocode}

% \macro{\childdocof}
% The command |\childdocof| redirects
% compilation to the main file |#1|.
%    \begin{macrocode}
\newcommand{\childdocof}[1]
{
  \childdocdisable
  \childdoctrue
  \includeonly{\childdocname}
  \def\jobname{#1}
  \def\childdocjob{#1}
  \input{#1}
}
%    \end{macrocode}

% \macro{\childdocby}
% The command |\childdocby| ....
%    \begin{macrocode}
\newcommand{\childdocby}[2][]
{
  \childdocdisable
  \childdoctrue
  \childdocmanualtrue
  \if?#1?\else
    \def\jobname{#2}
  \fi
  \def\childdocjob{#2}
  \input{#2}
  \endinput
}
%    \end{macrocode}

% \macro{\childdocforward}
% The command |\childdocforward| redirects
% compilation to the main file or
% (if the optional argument is given) a child file.
% Parameters are set as if the main file
% or a child file starting with |\childdocof| was compiled.
% Then compilation is handed over to the main file:
%    \begin{macrocode}
\newcommand{\childdocforward}[2][]
{
  \begingroup
    \if?#1?
      \def\childdoctmp
      {
        \def\childdocname{#2}
        \def\childdocjob{#2}
        \def\jobname{#2}
        \input{#2}
        \endinput
      }
    \else
      \def\childdoctmp
      {
        \childdocdisable
        \def\childdocname{#2}
        \childdoctrue
        \includeonly{#2}
        \def\childdocjob{#1}
        \def\jobname{#1}
        \input{#1}
        \endinput
      }
    \fi
    \expandafter
  \endgroup
  \childdoctmp
}
%    \end{macrocode}

% \macro{\childdocforwardprefix}
% The command |\childdocforwardprefix| redirects
% compilation to the main or a child file by means of a pattern.
% The prefix |#1| in the current filename is replaced by |#2|
% and the suffix of the current filename is kept
% (it is assumed that the filename does not contain the substring `|~~~|'
% which is used as a delimiter).
% Compilation is handed over to the new file by |\childdocforward|:
%    \begin{macrocode}
\newcommand{\childdocforwardprefix}[3][]
{
  \begingroup
    \def\childdocextract #2##1~~~{\def\childdoctmp{\childdocforward[#1]{#3##1}}}
    \expandafter\childdocextract\childdocname~~~
    \expandafter
  \endgroup
  \childdoctmp
}
%    \end{macrocode}

% \macro{\childdoc}
% The deprecated macro |\childdoc| is a legacy version of |\childdocmain|:
%    \begin{macrocode}
\newcommand{\childdoc}{\childdocmain}
%    \end{macrocode}

% \macro{\childdocredirect}
% The deprecated macro |\childdocredirect| is a legacy version
% of |\childdocforward| and |\childdocforwardprefix|:
%    \begin{macrocode}
\newcommand{\childdocredirect}[2][]
{
  \begingroup
    \if?#1?
      \def\childdoctmp{\childdocforward{#2}}
    \else
      \def\childdoctmp{\childdocforwardprefix{#1}{#2}}
    \fi
    \expandafter
  \endgroup
  \childdoctmp
}
%    \end{macrocode}

%\iffalse
%</package>
%\fi
%
\endinput

\childdocforwardprefix[cdocsamp]{cdocsfn}{cdocsch}
%    \end{macrocode}

%\iffalse
%</samplefinal>
%\fi
%
% %%%%%%%%%%%%%%%%%%%%%%%%%%%%%%%%%%%%%%
% \paragraph{Command Line Processing.}
%
% The following three command lines generate the output files
% |cdocscld|, |cdocscl1| and |cdocscl2|
% which should be identical to
% |cdocsdrf|, |cdocsch1| and |cdocsfn2|, respectively:
% \begin{center}
% \begin{tabular}{l}
% |latex -jobname cdocscld \|\\
% |  "\def\version{draft}% \iffalse
%
% childdoc.dtx Copyright (C) 2017-2018 Niklas Beisert
%
% This work may be distributed and/or modified under the
% conditions of the LaTeX Project Public License, either version 1.3
% of this license or (at your option) any later version.
% The latest version of this license is in
%   http://www.latex-project.org/lppl.txt
% and version 1.3 or later is part of all distributions of LaTeX
% version 2005/12/01 or later.
%
% This work has the LPPL maintenance status `maintained'.
%
% The Current Maintainer of this work is Niklas Beisert.
%
% This work consists of the files childdoc.dtx and childdoc.ins
% and the derived files childdoc.def and cdocsamp.tex with
% cdocsch1.tex, cdocsch2.tex, cdocsdrf.tex, cdocsfn1.tex, cdocsfn2.tex.
%
%<package>\ifdefined\childdocmain\endinput\fi
%<package>\ProvidesFile{childdoc.def}[2018/12/30 v2.0 child document driver]
%<samplemain>\ProvidesFile{cdocsamp.tex}[2018/12/30 v2.0 sample for childdoc]
%<*driver>
%\ProvidesFile{childdoc.drv}[2018/12/30 v2.0 childdoc reference manual file]
\PassOptionsToClass{10pt,a4paper}{article}
\documentclass{ltxdoc}

\usepackage[margin=35mm]{geometry}
\usepackage{hyperref}
\usepackage{hyperxmp}
\usepackage[usenames]{color}

\hypersetup{colorlinks=true}
\hypersetup{pdfstartview=FitH}
\hypersetup{pdfpagemode=UseNone}
\hypersetup{pdfsource={}}
\hypersetup{pdflang={en-UK}}
\hypersetup{pdfcopyright={Copyright 2017-2018 Niklas Beisert.
  This work may be distributed and/or modified under the
  conditions of the LaTeX Project Public License, either version 1.3
  of this license or (at your option) any later version.}}
\hypersetup{pdflicenseurl={http://www.latex-project.org/lppl.txt}}
\hypersetup{pdfcontactaddress={ETH Zurich, ITP, HIT K,
  Wolfgang-Pauli-Strasse 27}}
\hypersetup{pdfcontactpostcode={8093}}
\hypersetup{pdfcontactcity={Zurich}}
\hypersetup{pdfcontactcountry={Switzerland}}
\hypersetup{pdfcontactemail={nbeisert@itp.phys.ethz.ch}}
\hypersetup{pdfcontacturl={http://people.phys.ethz.ch/\xmptilde nbeisert/}}

\newcommand{\secref}[1]{\hyperref[#1]{section \ref*{#1}}}

\parskip1ex
\parindent0pt
\let\olditemize\itemize
\def\itemize{\olditemize\parskip0pt}

\begin{document}

\title{The \textsf{childdoc} Package}
\hypersetup{pdftitle={The childdoc Package}}
\author{Niklas Beisert\\[2ex]
  Institut f\"ur Theoretische Physik\\
  Eidgen\"ossische Technische Hochschule Z\"urich\\
  Wolfgang-Pauli-Strasse 27, 8093 Z\"urich, Switzerland\\[1ex]
  \href{mailto:nbeisert@itp.phys.ethz.ch}
  {\texttt{nbeisert@itp.phys.ethz.ch}}}
\hypersetup{pdfauthor={Niklas Beisert}}
\hypersetup{pdfsubject={Manual for the LaTeX2e Package childdoc}}
\date{30 December 2018, \textsf{v2.0}}
\maketitle

\begin{abstract}\noindent
\textsf{childdoc} is a \LaTeXe{} package
that enables the direct compilation
of document sections included by |\include|
to individual files.
\end{abstract}

\begingroup
\parskip0ex
\tableofcontents
\endgroup

%%%%%%%%%%%%%%%%%%%%%%%%%%%%%%%%%%%%%%%%%%%%%%%%%%%%%%%%%%%%%%%%%%%%%%%%%%%%%%%%
%%%%%%%%%%%%%%%%%%%%%%%%%%%%%%%%%%%%%%%%%%%%%%%%%%%%%%%%%%%%%%%%%%%%%%%%%%%%%%%%
\section{Introduction}

\LaTeX{} provides a mechanism to structure a large document (such as a book)
into a main file and several child files (containing the chapters)
using the |\include| command.
This mechanism is beneficial for documents
which span hundreds of pages in order to
make the source file(s) more manageable.
Moreover, compilation can be restricted to
selected child files by means of the |\includeonly| command.
The latter feature can be used to reduce the compilation time while editing
(this was significantly more useful in the earlier days of \LaTeX{})
or to generate a smaller document which is easier to navigate.
Another application of |\includeonly| is to generate
documents consisting of selected parts of the complete document.

However, there are a few drawbacks of the plain |\include| mechanism:
\begin{itemize}
\item
The child files cannot be compiled on their own,
they can only be compiled via the main file.
A naive editing environment
(such as a text editor with an option
to have the current file processed by \LaTeX)
may require one to switch to the main file before compiling;
attempting to compile the child file produces errors.
\item
The main file must be modified (each time)
to adjust the |\includeonly| command
to the present needs. This easily leaves the main file in a messy state.
\item
The generated document will always carry the filename
of the main document. This is inconvenient if
several child files are to be compiled and
to be kept for distribution.
\end{itemize}

The present package provides a simple interface
to make child files individually compilable by \LaTeX{}.
Compiling a child file then has the same effect as compiling
the main file with an |\includeonly| command
to select the appropriate child.
Moreover the generated document will carry the name of the child
rather than the main file.
This resolves all three above issues.

This feature is meant to make the editing of books,
thesis documents and lecture notes somewhat more convenient.
However, the package can also be used efficiently for
composing a series of documents (such as exercise sheets)
which are typically distributed individually.
It then assists the author in generating the individual documents
(potentially in different versions)
as well as a document containing the collected series.
Another application is in developing style files
or other kinds of included material
where compilation of the style file could redirect
to a sample or test file.

%%%%%%%%%%%%%%%%%%%%%%%%%%%%%%%%%%%%%%%%%%%%%%%%%%%%%%%%%%%%%%%%%%%%%%%%%%%%%%%%
%%%%%%%%%%%%%%%%%%%%%%%%%%%%%%%%%%%%%%%%%%%%%%%%%%%%%%%%%%%%%%%%%%%%%%%%%%%%%%%%
\section{Usage}

First of all, the package \textsf{childdoc} is \emph{not} a standard
\LaTeXe{} |.sty| style file! Therefore it needs to be invoked in
a non-standard way.

%%%%%%%%%%%%%%%%%%%%%%%%%%%%%%%%%%%%%%%%%%%%%%%%%%%%%%%%%%%%%%%%%%%%%%%%%%%%%%%%
\subsection{Included Files}
\label{sec:include}

%%%%%%%%%%%%%%%%%%%%%%%%%%%%%%%%%%%%%%%%
\DescribeMacro{\childdocmain}
To use the package, add the commands
\begin{center}
\begin{tabular}{l}
|\input{childdoc.def}|\\
|\childdocmain{}|\\
\end{tabular}
\end{center}
at the very top of the main \LaTeX{} file,
in particular \emph{before} the |\documentclass| statement!
The argument of |\childdocmain| should be left empty
(but it must be present).

%%%%%%%%%%%%%%%%%%%%%%%%%%%%%%%%%%%%%%%%
\DescribeMacro{\childdocof}
Furthermore, add the commands
\begin{center}
\begin{tabular}{l}
|\input{childdoc.def}|\\
|\childdocof{|\textit{main}|}|\\
\end{tabular}
\end{center}
at the top of every child file \textit{child}
which is included by |\include{|\textit{child}|}|
from within the main file
(or at least for those files to be compiled individually).
The argument \textit{main} must be the filename of the main file.

There are a couple of
considerations in setting up the main and child documents:

%%%%%%%%%%%%%%%%%%%%%%%%%%%%%%%%%%%%%%%%
\paragraph{Restrictions.}

Please note the following restrictions:
\begin{itemize}
\item
|\childdocmain| must be called with one argument \textit{main}
to ensure compatibility with earlier version of the package.
It must either be empty (|\childdocmain{}|)
or precisely match the filename of the main file in which it is specified.
See \secref{sec:detection} for further information.
\item
The filename \textit{main} must be specified without the |.tex| extension.
\item
The filename \textit{main} is case sensitive
(even in case-insensitive file systems)
due to internal string comparison.
\item
The argument \textit{main} should be fully expanded, it cannot be a macro.
\item
Subdirectories and special characters should be avoided in filenames.
\item
The command |\childdocmain{|\textit{main}|}| must be followed by a whitespace.
It should not be followed immediately by another command
or by a comment mark `|%|'.
This is because the \TeX{} parser reads the token immediately following
the argument of |\childdocmain| and puts it
at the beginning of every child section;
however, a white\-space is ignored.
\end{itemize}

%%%%%%%%%%%%%%%%%%%%%%%%%%%%%%%%%%%%%%%%
\paragraph{Content of Main File.}

It is advisable to place all content in the child files included by |\include|.
Any output contained in the main file will appear in all child documents
unless suppressed manually;
it cannot be suppressed automatically by the |\includeonly| directive
and thus should normally be avoided.
A method to include some content in the main file
by means of conditional processing is described in \secref{sec:conditional}.

%%%%%%%%%%%%%%%%%%%%%%%%%%%%%%%%%%%%%%%%
\paragraph{Page Numbering.}

When only a part of the document is compiled,
the appropriate numbering of pages
(as well as other status parameters)
is determined from the |.aux| files.
The latter contain information from previous passes.
However this information needs to propagate through
all intermediate child documents.
Therefore the page numbering in child documents may well
be inconsistent until the complete document is compiled at least once.

A useful (if unconventional) way to always ensure a consistent
page numbering is to restart the numbering in each child document
and denote the pages by `\textit{child}|.|\textit{page}'
where \textit{child} represents the chapter/section number of the child file.
This can be achieved by the command
|\numberwithin{page}{|\textit{child}|}|
of the \textsf{amsmath} package
where \textit{child} can be |chapter| or |section|
depending on the chosen structuring.
Alternatively, one can modify the macro |\thepage| appropriately
and reset the counter |page| at the start of each child file.

%%%%%%%%%%%%%%%%%%%%%%%%%%%%%%%%%%%%%%%%%%%%%%%%%%%%%%%%%%%%%%%%%%%%%%%%%%%%%%%%
\subsection{Conditional Processing}
\label{sec:conditional}

The package provides a mechanism to compile different versions
of a document. To customise the versions further some conditional processing
can come in handy to distinguish which version is being compiled.
The package provides two macros to describe the compilation context:

%%%%%%%%%%%%%%%%%%%%%%%%%%%%%%%%%%%%%%%%
\DescribeMacro{\ifchilddoc}
The conditional |\ifchilddoc| distinguishes between the compilation of
child documents and the main document:
%
\begin{center}
|\ifchilddoc |\textit{child-code}| |[|\||else |\textit{main-code}]| \||fi|
\end{center}

%%%%%%%%%%%%%%%%%%%%%%%%%%%%%%%%%%%%%%%%
\DescribeMacro{\childdocname}
\DescribeMacro{\childdocjob}
The macro |\childdocname| contains the filename (without extension)
of the main or child file being processed.
Note that |\childdocjob| will always contain the name of the main file.

%%%%%%%%%%%%%%%%%%%%%%%%%%%%%%%%%%%%%%%%
\paragraph{Title Page.}

Conditional processing can be used to include a title or banner page
in the main document when proper precautions are taken.
Importantly, the code in the main file should ensure that the page counter
(as well as other status parameters which are stored in the |.aux| files)
takes the same value after the conditional processing.
Otherwise the page numbers may take divergent values
depending on which part is compiled.

For example, a title page could be declared by:
%
\begin{center}
\begin{tabular}{l}
|\ifchilddoc\||else|\\
|\addtocounter{page}{-1}|\\
\textit{code for title page}\\
|\newpage|\\
|\||fi|
\end{tabular}
\end{center}
%
A banner page for the child documents can be generated by:
%
\begin{center}
\begin{tabular}{l}
|\ifchilddoc|\\
|\addtocounter{page}{-1}|\\
\textit{code for banner page}\\
|\newpage|\\
|\||fi|
\end{tabular}
\end{center}
%
Here one could write a message such as:
\begin{center}
|This is the part \childdocname{} of \childdocjob{}.|
\end{center}

%%%%%%%%%%%%%%%%%%%%%%%%%%%%%%%%%%%%%%%%%%%%%%%%%%%%%%%%%%%%%%%%%%%%%%%%%%%%%%%%
\subsection{Flags}
\label{sec:flags}

The package makes it easy to generate different versions
of the main or child documents.
To this end compilation flags can be defined
and assigned different default values.
They will be particularly useful in conjunction
with the forwarding mechanism described in \secref{sec:forward}.

For example, it may be useful to have a flag |\version|
which can be set to |draft| or |final|.
The document source will contain some conditional code
depending on the value of |\version|.
Suppose further, the flag should default to |final| for the main file
and to |draft| for child files
which is a natural assignment for editing the document.
This is achieved by placing the following code
in the preamble of the main document
(below the |\childdocmain| directive):
%
\begin{center}
\begin{tabular}{l}
|\ifchilddoc|\\
|\providecommand{\version}{draft}|\\
|\||else|\\
|\providecommand{\version}{final}|\\
|\||fi|
\end{tabular}
\end{center}
%
The definition by |\providecommand| makes sure
that previous definitions are not overwritten.
Further statements |\providecommand{\version}{...}|
can thus be added before the above code to override it.

For the main file, one might add a line
(between |\childdocmain| and the above block)
%
\begin{center}
|%\ifchilddoc\||else\providecommand{\version}{draft}\||fi|
\end{center}
%
which can be uncommented to produce a draft version.
Likewise one can add a line to the very top of a child file
(above the |\childdocof{|\textit{main}|}| directive)
%
\begin{center}
|%\providecommand{\version}{final}|
\end{center}
%
which can be uncommented to produce the final version of this child document.

%%%%%%%%%%%%%%%%%%%%%%%%%%%%%%%%%%%%%%%%%%%%%%%%%%%%%%%%%%%%%%%%%%%%%%%%%%%%%%%%
\subsection{Forwarding}
\label{sec:forward}

Different versions of the main or child documents
using compilation flags as described in \secref{sec:flags}
can be (permanently) stored in different files
for convenient compilation, viewing and distribution.
To this end, the package defines a command
to pass on compilation to a different file:

%%%%%%%%%%%%%%%%%%%%%%%%%%%%%%%%%%%%%%%%
\DescribeMacro{\childdocforward}
The command |\childdocforward| redirects processing to
another source file:
%
\begin{center}
\begin{tabular}{l}
|\input{childdoc.def}|\\
|\childdocforward[|\textit{main}|]{|\textit{dest}|}|\\
\end{tabular}
\end{center}
%
The argument \textit{dest} is the destination file
(without extension).
It should be the main file or one of the child files.
Note that further \textsf{childdoc} directives
such as |\childdocof| and |\childdocforward|
in the indicated file will be processed in this form.
The optional argument \textit{main}
passes on directly to the main file \textit{main}
while pretending to compile the child \textit{dest}.
This form behaves as if \textit{dest}
issues |\childdocof{|\textit{main}|}| right away,
and no further \textsf{childdoc} directives will be processed.

%%%%%%%%%%%%%%%%%%%%%%%%%%%%%%%%%%%%%%%%
\DescribeMacro{\...prefix}
In the alternative form |\childdocforwardprefix|,
%
\begin{center}
\begin{tabular}{l}
|\input{childdoc.def}|\\
|\childdocforwardprefix[|\textit{main}|]{|\textit{prefix}|}{|\textit{dest}|}|
\end{tabular}
\end{center}
%
the destination file is determined by a pattern
depending on the current file:
To make this work, the current file must be called
`{\textit{prefix}\hspace{0.2em}\textit{suffix}}'
with \textit{prefix} matching precisely the argument.
Processing is then passed on to the file
`{\textit{dest}\hspace{0.2em}\textit{suffix}}'.
Surely, the same effect is achieved by
directly specifying the
argument `{\textit{dest}\hspace{0.2em}\textit{suffix}}'
in the first form.
However, that requires to set up a different file
for each child. With the alternative form of the command
all these files can have exactly the same content
which simplifies setting them up and maintaining them.

For example, the following file |draft.tex|
with a compilation flag |\version| as described in \secref{sec:flags}
compiles the main document as a draft:
%
\begin{center}
\begin{tabular}{l}
|\def\version{draft}|\\
|\input{childdoc.def}|\\
|\childdocforward{|\textit{main}|}|
\end{tabular}
\end{center}
%
Likewise, the following files |final|\textit{nn}|.tex|
compile the final version of the child document
|child|\textit{nn}|.tex|:
%
\begin{center}
\begin{tabular}{l}
|\def\version{final}|\\
|\input{childdoc.def}|\\
|\childdocforwardprefix{final}{child}|
\end{tabular}
\end{center}
%

Note that when several versions of a main file and/or of each child file
are to be generated, it may be convenient to set up a |Makefile| or
shell script to automatise the process.

%%%%%%%%%%%%%%%%%%%%%%%%%%%%%%%%%%%%%%%%%%%%%%%%%%%%%%%%%%%%%%%%%%%%%%%%%%%%%%%%
\subsection{Command Line Processing}
\label{sec:commandline}

The effect of redirection files can also be achieved by invoking
the \LaTeX{} compiler with a more elaborate command line.
Most conveniently this should be done as part
of a shell script or a |Makefile|.

When using \textsf{childdoc} in the main file, the following
command lines effectively perform a redirection
(note that depending on the shell being used,
backslashes may have to be doubled: `|\|' $\to$ `|\\|'):
%
\begin{center}
|... -jobname "|\textit{target}|" |\\|"|[\textit{flags}]%
|\input{childdoc.def}\childdocforward[|\textit{main}|]{|\textit{dest}|}"|
\end{center}
%
Here \textit{target} is the name of the output file,
\textit{main} is the name of the main file
and \textit{dest} is the name of the main or child file to be processed
(all filenames without extensions).
The optional argument \textit{main} can be omitted
if \textit{main} matches \textit{dest}.
Optionally, compilation \textit{flags} can be defined via |\def| commands.
This command line makes the \TeX{} engine believe
it is compiling the file \textit{target}
whose content is specified as the latter parameter.
The provided code then forwards the processing to
\textit{main} or \textit{dest} as described in \secref{sec:forward}.

%%%%%%%%%%%%%%%%%%%%%%%%%%%%%%%%%%%%%%%%%%%%%%%%%%%%%%%%%%%%%%%%%%%%%%%%%%%%%%%%
\subsection{Include by Input}
\label{sec:input}

Including child documents by |\include| has some restrictions by design.
Most notably, the content of a child document always occupies
its own set of pages; pages cannot be shared between child documents.
Usually, this behaviour makes perfect sense
because each child document contain an essential part of the document.
However, in some situations it may be desirable to compose
a document from a collection of parts
without having mandatory page breaks between then.
For this case, the package
provides a mechanism to include parts
by |\input| which can also be processed individually.
However, by construction this mechanism
requires manual handling of the content to be output.

%%%%%%%%%%%%%%%%%%%%%%%%%%%%%%%%%%%%%%%%
\DescribeMacro{\ifchilddocmanual}
The main file should be prepared as usual, see \secref{sec:include}.
However, the document body must make a distinction
between processing of an individual part and of the main document, e.g.:
%
\begin{center}
\begin{tabular}{l}
|\ifchilddocmanual|\\
|\input{\childdocname}|\\
|\||else|\\
\textit{document body with }|\input{|\textit{part}|}|\\
|\||fi|
\end{tabular}
\end{center}
%
The conditional |\ifchilddocmanual| is true whenever
a part to be included by |\input| is being compiled,
and the name of the part is stored in |\childdocname|.

%%%%%%%%%%%%%%%%%%%%%%%%%%%%%%%%%%%%%%%%
\DescribeMacro{\childdocby}
Each part to be included by |\input| should start with:
%
\begin{center}
\begin{tabular}{l}
|\input{childdoc.def}|\\
|\childdocby{|\textit{main}|}|\\
\end{tabular}
\end{center}
%
The directive |\childdocby| is similar to |\childdocof|
described in \secref{sec:include},
but the subsequent selection of content must be done manually.
To that end, both |\ifchilddoc| and |\ifchilddocmanual|
will be true upon processing of a part,
and the name of the part is stored in |\childdocname|.
Note that |\jobname| will be set to the filename of the current part
so that each part receives an individual |.aux| file
that does not interfere with the |.aux| file(s) of the main document.
This behaviour can be altered by the alternative form
|\childdocby[*]{|\textit{main}|}| (with a non-empty optional argument)
which uses the |.aux| file of the main document
by setting |\jobname| to \textit{main}.

%%%%%%%%%%%%%%%%%%%%%%%%%%%%%%%%%%%%%%%%%%%%%%%%%%%%%%%%%%%%%%%%%%%%%%%%%%%%%%%%
\subsection{Driver Development}
\label{sec:driver}

The \textsf{childdoc} mechanism can also be use for the development
of definition files such as \LaTeX{} styles or classes.
This case differs from the above setup with multiple parts
included by |\include| in that no |\includeonly| should be invoked.
This can be achieved by starting the include file
(before |\ProvidesPackage|) with:
%
\begin{center}
\begin{tabular}{l}
|\input{childdoc.def}|\\
|\childdocforward{|\textit{main}|}|\\
\end{tabular}
\end{center}
%
or alternatively with:
%
\begin{center}
\begin{tabular}{l}
|\input{childdoc.def}|\\
|\childdocby{|\textit{main}|}|\\
\end{tabular}
\end{center}
%
Both forms have slightly different effects as described above.
The main file is prepared as usual, see \secref{sec:include}.

%%%%%%%%%%%%%%%%%%%%%%%%%%%%%%%%%%%%%%%%%%%%%%%%%%%%%%%%%%%%%%%%%%%%%%%%%%%%%%%%
\subsection{Legacy Detection}
\label{sec:detection}

The directive |\childdocmain| in the main file can detect
whether the complete document or merely a child is to be compiled
even without using the directive |\childdocof|.
This method is deprecated because it is less robust
and there is no compelling reason to use it;
it is merely provided for backward compatibility
and it may be removed in future versions.

If the detection mechanism is to be used,
it is mandatory to correctly specify
the filename of the main file as the argument of |\childdocmain|:
%
\begin{center}
\begin{tabular}{l}
|\input{childdoc.def}|\\
|\childdocmain{|\textit{main}|}|\\
\end{tabular}
\end{center}
%
If |\jobname| does not match the argument \textit{main} of |\childdocmain|,
it is assumed that |\jobname| points to the child file to be compiled.
When using |\childdocmain| with the main file specified as argument,
it suffices to start a child file
with just |\input{|\textit{main}|}|
without loading of the package and using |\childdocof|.
If instead all processing is done
with the appropriate \textsf{childdoc} directives,
the argument of \textit{main} of |\childdocmain| can be empty.

An alternative version of the command line processing described
in \secref{sec:commandline} using the detection mechanism reads:
%
\begin{center}
|... -jobname "|\textit{target}|" "|[\textit{flags}]%
[|\def\jobname{|\textit{dest}|}|]|\input{|\textit{main}|}"|
\end{center}

%%%%%%%%%%%%%%%%%%%%%%%%%%%%%%%%%%%%%%%%%%%%%%%%%%%%%%%%%%%%%%%%%%%%%%%%%%%%%%%%
\subsection{Manual Code}
\label{sec:manual}

In case one cannot be certain whether the definitions file |childdoc.def|
is installed on the target \TeX{} distribution
and one prefers not to ship it,
it is conceivable to paste a few relevant commands into the sources.

To that end, drop all statements |\input{childdoc.def}|
and perform the replacements as outlined below.
Instead of |\childdocmain{|\textit{main}|}| add the following code
to the top of the main file:
%
\begin{center}
\begin{tabular}{l}
|\||ifdefined\childdocname\endinput\||fi\newif\ifchilddoc|\\
|\edef\childdocname{\scantokens\expandafter{\jobname\noexpand}}|\\
|\def\childdocmain{|\textit{main}|}\||ifx\childdocmain\childdocname\||else|\\
|\childdoctrue\includeonly{\childdocname}\let\jobname\childdocmain\||fi|\\
\end{tabular}
\end{center}
%
Instead of |\childdocof{|\textit{main}|}| just include the main file
at the top of each child file:
%
\begin{center}
|\input{|\textit{main}|}|
\end{center}
%
A simple redirection |\childdocforward{|\textit{dest}|}| is achieved by:
%
\begin{center}
|\def\jobname{|\textit{dest}|}\input{\jobname}|
\end{center}
%
The redirection with prefix
|\childdocforwardprefix[|\textit{prefix}|]{|\textit{dest}|}|
is accomplished by:
%
\begin{center}
\begin{tabular}{l}
|{\edef\jobname{\scantokens\expandafter{\jobname\noexpand}}|\\
|\def\redirectjob |\textit{prefix}|#1~~~{\gdef\jobname{|\textit{dest}|#1}}|\\
|\expandafter\redirectjob\jobname~~~}\input{\jobname}|
\end{tabular}
\end{center}

In an alternative approach,
child documents can be compiled by a specific command line
without additional code or specific definitions:
%
\begin{center}
|... -jobname "|\textit{target}|" "|[\textit{flags}]%
|\includeonly{|\textit{dest}|}\input{|\textit{main}|}"|
\end{center}
%

%%%%%%%%%%%%%%%%%%%%%%%%%%%%%%%%%%%%%%%%%%%%%%%%%%%%%%%%%%%%%%%%%%%%%%%%%%%%%%%%
%%%%%%%%%%%%%%%%%%%%%%%%%%%%%%%%%%%%%%%%%%%%%%%%%%%%%%%%%%%%%%%%%%%%%%%%%%%%%%%%
\section{Information}

%%%%%%%%%%%%%%%%%%%%%%%%%%%%%%%%%%%%%%%%%%%%%%%%%%%%%%%%%%%%%%%%%%%%%%%%%%%%%%%%
\subsection{Copyright}

Copyright \copyright{} 2017--2018 Niklas Beisert

This work may be distributed and/or modified under the
conditions of the \LaTeX{} Project Public License, either version 1.3
of this license or (at your option) any later version.
The latest version of this license is in
  \url{http://www.latex-project.org/lppl.txt}
and version 1.3 or later is part of all distributions of \LaTeX{}
version 2005/12/01 or later.

This work has the LPPL maintenance status `maintained'.

The Current Maintainer of this work is Niklas Beisert.

This work consists of the files |README.txt|, |childdoc.ins| and |childdoc.dtx|
as well as the derived files |childdoc.def|, |cdocsamp.tex|
with |cdocsch1.tex|, |cdocsch2.tex|, |cdocspt3.tex|, |cdocspt4.tex|,
|cdocsdrf.tex|, |cdocsfn1.tex|, |cdocsfn2.tex|
as well as |childdoc.pdf|.

%%%%%%%%%%%%%%%%%%%%%%%%%%%%%%%%%%%%%%%%%%%%%%%%%%%%%%%%%%%%%%%%%%%%%%%%%%%%%%%%
\subsection{Files and Installation}

The package consists of the files:
%
\begin{center}
\begin{tabular}{ll}
    |README.txt|   & readme file \\
    |childdoc.ins| & installation file \\
    |childdoc.dtx| & source file \\
    |childdoc.def| & definition file \\
    |cdocsamp.tex| & sample main file \\
    |cdocsch1.tex| & sample include file \\
    |cdocsch2.tex| & sample include file \\
    |cdocspt3.tex| & sample part file \\
    |cdocspt4.tex| & sample part file \\
    |cdocsdrf.tex| & sample redirection file \\
    |cdocsfn1.tex| & sample redirection file \\
    |cdocsfn2.tex| & sample redirection file \\
    |childdoc.pdf| & manual
\end{tabular}
\end{center}
%
The distribution consists of the files
|README.txt|, |childdoc.ins| and |childdoc.dtx|.
%
\begin{itemize}
\item
Run (pdf)\LaTeX{} on |childdoc.dtx|
to compile the manual |childdoc.pdf| (this file).
\item
Run \LaTeX{} on |childdoc.ins| to create the definitions file |childdoc.def|
and the sample |cdocsamp.tex| with include files
|cdocsch1.tex|, |cdocsch2.tex|, |cdocspt3.tex|, |cdocspt4.tex|,
|cdocsdrf.tex|, |cdocsfn1.tex|, |cdocsfn2.tex|.
Then copy the file |childdoc.def| to an appropriate directory of your \LaTeX{}
distribution, e.g.\ \textit{texmf-root}|/tex/latex/childdoc|.
\end{itemize}

%%%%%%%%%%%%%%%%%%%%%%%%%%%%%%%%%%%%%%%%%%%%%%%%%%%%%%%%%%%%%%%%%%%%%%%%%%%%%%%%
\subsection{Related CTAN Packages}

There are several other packages which offer a similar functionality:
%
\begin{itemize}
\item
The packages
\href{http://ctan.org/pkg/docmute}{\textsf{docmute}},
\href{http://ctan.org/pkg/includex}{\textsf{includex}} and
\href{http://ctan.org/pkg/standalone}{\textsf{standalone}}
provide commands to include only the document body of
a child file thus allowing both files to be compiled individually.
\item
The packages \href{http://ctan.org/pkg/subdocs}{\textsf{subdocs}}
and \href{http://ctan.org/pkg/subfiles}{\textsf{subfiles}}
provide structures in which the main and child documents can be
encapsulated and allowing them to be compiled individually.
The inclusion mechanism is different from the conventional |\include|.
\item
The package \href{http://ctan.org/pkg/combine}{\textsf{combine}}
is an elaborate solution to combine several documents into one.
\end{itemize}
%
See also the CTAN topic \href{http://ctan.org/topic/subdocs}{\textsf{subdocs}}
for further related packages.
The present package differs from the above solutions in that
a document structure constructed with the conventional |\include| mechanism
just needs two extra commands at the top of every file
such that all constituent files can be compiled individually.

%%%%%%%%%%%%%%%%%%%%%%%%%%%%%%%%%%%%%%%%%%%%%%%%%%%%%%%%%%%%%%%%%%%%%%%%%%%%%%%%
%\subsection{Feature Suggestions}
%
%The following is a list of features which may be useful for future
%versions of this package:
%%
%\begin{itemize}
%\item
%\ldots
%\end{itemize}

%%%%%%%%%%%%%%%%%%%%%%%%%%%%%%%%%%%%%%%%%%%%%%%%%%%%%%%%%%%%%%%%%%%%%%%%%%%%%%%%
\subsection{Revision History}

%%%%%%%%%%%%%%%%%%%%%%%%%%%%%%%%%%%%%%%%
\paragraph{v2.0:} 2018/12/30

\begin{itemize}
\item
immediate forward processing
\item
added |\childdocby| mechanism
\item
manual restructured
\end{itemize}

%%%%%%%%%%%%%%%%%%%%%%%%%%%%%%%%%%%%%%%%
\paragraph{v1.6:} 2018/01/17

\begin{itemize}
\item
application for development of include files
\item
corrections to manual
\end{itemize}

%%%%%%%%%%%%%%%%%%%%%%%%%%%%%%%%%%%%%%%%
\paragraph{v1.5:} 2017/05/21

\begin{itemize}
\item
more complete structuring introduced
\item
|\childdocof| introduced
\item
|\childdoc| renamed to |\childdocmain|
\item
|\childredirect| renamed to |\childdocforward| and |\childdocforwardprefix|
and functionality expanded
\end{itemize}

%%%%%%%%%%%%%%%%%%%%%%%%%%%%%%%%%%%%%%%%
\paragraph{v1.0:} 2017/04/27

\begin{itemize}
\item
manual and install package
\item
first version published on CTAN
\end{itemize}

%%%%%%%%%%%%%%%%%%%%%%%%%%%%%%%%%%%%%%%%
\paragraph{v0.6:} 2017/04/26

\begin{itemize}
\item
redirection mechanism added
\end{itemize}

%%%%%%%%%%%%%%%%%%%%%%%%%%%%%%%%%%%%%%%%
\paragraph{v0.5:} 2017/04/26

\begin{itemize}
\item
functionality in definition file
\end{itemize}


%%%%%%%%%%%%%%%%%%%%%%%%%%%%%%%%%%%%%%%%%%%%%%%%%%%%%%%%%%%%%%%%%%%%%%%%%%%%%%%%
%%%%%%%%%%%%%%%%%%%%%%%%%%%%%%%%%%%%%%%%%%%%%%%%%%%%%%%%%%%%%%%%%%%%%%%%%%%%%%%%
%%%%%%%%%%%%%%%%%%%%%%%%%%%%%%%%%%%%%%%%%%%%%%%%%%%%%%%%%%%%%%%%%%%%%%%%%%%%%%%%
\appendix

\settowidth\MacroIndent{\rmfamily\scriptsize 000\ }

 \DocInput{childdoc.dtx}

\end{document}
%</driver>
% \fi
%
% %%%%%%%%%%%%%%%%%%%%%%%%%%%%%%%%%%%%%%%%%%%%%%%%%%%%%%%%%%%%%%%%%%%%%%%%%%%%%%
% %%%%%%%%%%%%%%%%%%%%%%%%%%%%%%%%%%%%%%%%%%%%%%%%%%%%%%%%%%%%%%%%%%%%%%%%%%%%%%
% \section{Sample}
%\iffalse
%<*samplemain>
%\fi
%
% The following presents a sample document
% with two chapters, two parts, a title page,
% a compile flag as well as three forwarding files to set the flag.
% It consists of eight |.tex| files:
% \begin{center}
% \begin{tabular}{ll}
% |cdocsamp.tex|&main file\\
% |cdocsch1.tex|&include file for chapter 1\\
% |cdocsch2.tex|&include file for chapter 2\\
% |cdocspt3.tex|&include file for part 3\\
% |cdocspt4.tex|&include file for part 4\\
% |cdocsdrf.tex|&forwarding file for main file in draft mode\\
% |cdocsfi1.tex|&forwarding file for final version of chapter 1\\
% |cdocsfi2.tex|&forwarding file for final version of chapter 2\\
% \end{tabular}
% \end{center}
% Each of the eight files can be compiled directly by the \LaTeX{} compiler.
%
% %%%%%%%%%%%%%%%%%%%%%%%%%%%%%%%%%%%%%%
% \paragraph{Main File.}
%
% The main file is called |cdocsamp.tex|.
%
% Load the \textsf{childdoc} definitions and
% declare the filename for the main document:
%    \begin{macrocode}
\input{childdoc.def}
\childdocmain{}
%    \end{macrocode}

% Optional override for |\version| flag:
%    \begin{macrocode}
%%\ifchilddoc\else\providecommand{\version}{draft}\fi
%    \end{macrocode}

% Define the default values for the |\version| flag
% (|final| for the main file and |draft| for childs):
%    \begin{macrocode}
\ifchilddoc
\providecommand{\version}{draft}
\else
\providecommand{\version}{final}
\fi
%    \end{macrocode}

% Load the standard document class:
%    \begin{macrocode}
\documentclass[12pt]{article}
%    \end{macrocode}

% Start the document body:
%    \begin{macrocode}
\begin{document}
%    \end{macrocode}

% Declare a title page.
% Print title, part of document being processed and version flag:
%    \begin{macrocode}
\addtocounter{page}{-1}
\begin{center}
{\LARGE\bfseries{}childdoc example\par}
\vspace{1cm}
\ifchilddoc
\ifchilddocmanual part\else chapter\fi:
`\childdocname' of `\childdocjob'\par
\else
main document: `\childdocjob'\par
\fi
version: \version\par
\end{center}
\newpage
%    \end{macrocode}

% Manually include selected file,
% otherwise process as usual:
%    \begin{macrocode}
\ifchilddocmanual
\section*{part `\childdocname'}
\input{\childdocname}
\else
%    \end{macrocode}

% Include the two chapters:
%    \begin{macrocode}
\include{cdocsch1}
\include{cdocsch2}
%    \end{macrocode}

% Include the two parts unless only chapters should be displayed:
%    \begin{macrocode}
\ifchilddoc\else
\section{part three}
\input{cdocspt3}
\section{part four}
\input{cdocspt4}
\fi
%    \end{macrocode}

% Process as usual until here:
%    \begin{macrocode}
\fi
%    \end{macrocode}

% End of document body:
%    \begin{macrocode}
\end{document}
%    \end{macrocode}
%\iffalse
%</samplemain>
%\fi
%
% %%%%%%%%%%%%%%%%%%%%%%%%%%%%%%%%%%%%%%
% \paragraph{Chapter Include Files.}
%
% The include files are called |cdocsch1.tex| and |cdocsch2.tex|.
%
%\iffalse
%<*samplechap1|samplechap2>
%\fi

% Optional override for |\version| flag:
%    \begin{macrocode}
%%\providecommand{\version}{final}
%    \end{macrocode}

% Include the main document:
%    \begin{macrocode}
\input{childdoc.def}
\childdocof{cdocsamp}
%    \end{macrocode}

%\iffalse
%</samplechap1|samplechap2>
%\fi
%
%\iffalse
%<*samplechap1>
%\fi
% Some text for chapter 1:
%    \begin{macrocode}
\section{one}
some text in chapter one
%    \end{macrocode}

%\iffalse
%</samplechap1>
%\fi
% Some text for chapter 2:
%\iffalse
%<*samplechap2>
%\fi
%    \begin{macrocode}
\section{two}
more text in chapter two
%    \end{macrocode}

%\iffalse
%</samplechap2>
%\fi
%
% %%%%%%%%%%%%%%%%%%%%%%%%%%%%%%%%%%%%%%
% \paragraph{Part Include Files.}
%
% The include files are called |cdocspt3.tex| and |cdocspt4.tex|.
%
%\iffalse
%<*samplepart3|samplepart4>
%\fi

% Optional override for |\version| flag:
%    \begin{macrocode}
%%\providecommand{\version}{final}
%    \end{macrocode}

% Include the main document:
%    \begin{macrocode}
\input{childdoc.def}
\childdocby{cdocsamp}
%    \end{macrocode}

%\iffalse
%</samplepart3|samplepart4>
%\fi
%
%\iffalse
%<*samplepart3>
%\fi
% Some text for part 3:
%    \begin{macrocode}
some text in part three
%    \end{macrocode}

%\iffalse
%</samplepart3>
%\fi
% Some text for part 4:
%\iffalse
%<*samplepart4>
%\fi
%    \begin{macrocode}
more text in part four
%    \end{macrocode}

%\iffalse
%</samplepart4>
%\fi
%
% %%%%%%%%%%%%%%%%%%%%%%%%%%%%%%%%%%%%%%
% \paragraph{Forwarding for a Complete Draft.}
%
% The following forwarding file |cdocsdrf.tex|
% compiles the main document in draft mode:
%\iffalse
%<*sampledraft>
%\fi
%    \begin{macrocode}
\def\version{draft}
\input{childdoc.def}
\childdocforward{cdocsamp}
%    \end{macrocode}

%\iffalse
%</sampledraft>
%\fi
%
% %%%%%%%%%%%%%%%%%%%%%%%%%%%%%%%%%%%%%%
% \paragraph{Forwarding for Final Version of the Chapters.}
%
% The following forwarding files |cdocsfn1.tex| and |cdocsfn2.tex|
% (with identical content)
% compile the final versions of the child documents
% |cdocsch1.tex| and |cdocsch2.tex|, respectively:
%\iffalse
%<*samplefinal>
%\fi
%    \begin{macrocode}
\def\version{final}
\input{childdoc.def}
\childdocforwardprefix[cdocsamp]{cdocsfn}{cdocsch}
%    \end{macrocode}

%\iffalse
%</samplefinal>
%\fi
%
% %%%%%%%%%%%%%%%%%%%%%%%%%%%%%%%%%%%%%%
% \paragraph{Command Line Processing.}
%
% The following three command lines generate the output files
% |cdocscld|, |cdocscl1| and |cdocscl2|
% which should be identical to
% |cdocsdrf|, |cdocsch1| and |cdocsfn2|, respectively:
% \begin{center}
% \begin{tabular}{l}
% |latex -jobname cdocscld \|\\
% |  "\def\version{draft}\input{childdoc.def}\childdocforward{cdocsamp}"|\\
% |latex -jobname cdocscl1 \|\\
% |  "\input{childdoc.def}\childdocforward[cdocsamp]{cdocsch1}"|\\
% |latex -jobname cdocscl2 \|\\
% |  "\def\version{final}\input{childdoc.def}\childdocforward{cdocsch2}"|
% \end{tabular}
% \end{center}
% Note that the trailing backslash on each first line
% merely continues the input to the second line
% (for convenient cut ant paste).
% Furthermore, the command |latex| can be replaced by any
% of its alternative versions such as |pdflatex|.
%
% %%%%%%%%%%%%%%%%%%%%%%%%%%%%%%%%%%%%%%%%%%%%%%%%%%%%%%%%%%%%%%%%%%%%%%%%%%%%%%
% %%%%%%%%%%%%%%%%%%%%%%%%%%%%%%%%%%%%%%%%%%%%%%%%%%%%%%%%%%%%%%%%%%%%%%%%%%%%%%
% \section{Implementation}
%\iffalse
%<*package>
%\fi
%
% This section describes the definitions file |childdoc.def|.

% The definitions cannot be loaded using |\usepackage| or |\RequirePackage|
% which has a mechanism to prevent loading a style file more than once.
% When loading the definitions by means of |\input|
% multiple instances have to be prevented manually:
%\iffalse
%This code needs to be before the `\ProvidesFile' directive
%which is defined at the beginning of this file.
%Therefore it is also placed there and commented out here.
%</package>
%<*discard>
%\fi
%    \begin{macrocode}
\ifdefined\childdocmain\endinput\fi
%    \end{macrocode}
%\iffalse
%</discard>
%<*package>
%\fi
%
% \macro{\ifchilddoc}
% \macro{\ifchilddocmanual}
% The conditional |\ifchilddoc| tells whether a
% child (true) or main (false) document is being compiled.
% The conditional |\ifchilddocmanual| tells whether
% the |\includeonly| mechanism is used (false) or
% the selection of child files must be performed manually (true).
% The definitions initialise to false:
%    \begin{macrocode}
\newif\ifchilddoc
\newif\ifchilddocmanual
%    \end{macrocode}

% \macro{\childdocname}
% \macro{\childdocjob}
% The macro |\childdocname| stores the name of the main document
% to be compiled. The macro |\childdocjob| stores the name of
% the document on which the \LaTeX{} compiler was originally invoked.
% The content of |\jobname| cannot be compared
% to filenames specified in the source due to different catcodes.
% The following code rescans |\jobname|, stores the result
% in |\childdocname| and saves a copy in |\childdocjob|:
%    \begin{macrocode}
\edef\childdocname{\scantokens\expandafter{\jobname\noexpand}}
\let\childdocjob\childdocname
%    \end{macrocode}

% \macro{\childdocdisable}
% The macro |\childdocdisable| prevents the main file
% from being processed more than once.
% At this stage, the main document command |\childdocmain|
% is assumed to be called once again where it should do nothing.
% Any subsequent call to it should prevent
% a secondary processing of the main document
% It overwrites the forwarding commands
% |\childdocof| and |\childdocforward|
% with empty macros to prevent further inclusions of the main document:
%    \begin{macrocode}
\newcommand{\childdocdisable}
{
  \renewcommand{\childdocmain}[1]{\renewcommand{\childdocmain}[1]{\endinput}}
  \renewcommand{\childdocof}[1]{}
  \renewcommand{\childdocby}[2][]{}
  \renewcommand{\childdocforward}[2][]{}
  \renewcommand{\childdocdisable}{}
}
%    \end{macrocode}

% \macro{\childdocmain}
% The macro |\childdocmain| is to be called at the top of the main file
% with nothing or the main filename (without extension) as argument.
% First, it breaks loops.
% If the argument is not empty and does not match |\childdocname|
% (which is set by the first inclusion of |childdoc.def|),
% |\ifchilddoc| is set to true, |\includeonly| is applied to the child file
% and |\jobname| is set to the main file
% (for proper handling of |.aux| files):
%    \begin{macrocode}
\newcommand{\childdocmain}[1]
{
  \childdocdisable\childdocmain{}
  \if?#1?\else
    \begingroup
      \def\childdoctmp{#1}
      \ifx\childdoctmp\childdocname
        \def\childdoctmp{}
      \else
        \def\childdoctmp
        {
          \childdoctrue
          \includeonly{\childdocname}
          \def\childdocjob{#1}
          \def\jobname{#1}
        }
      \fi
      \expandafter
    \endgroup
    \childdoctmp
  \fi
}
%    \end{macrocode}

% \macro{\childdocof}
% The command |\childdocof| redirects
% compilation to the main file |#1|.
%    \begin{macrocode}
\newcommand{\childdocof}[1]
{
  \childdocdisable
  \childdoctrue
  \includeonly{\childdocname}
  \def\jobname{#1}
  \def\childdocjob{#1}
  \input{#1}
}
%    \end{macrocode}

% \macro{\childdocby}
% The command |\childdocby| ....
%    \begin{macrocode}
\newcommand{\childdocby}[2][]
{
  \childdocdisable
  \childdoctrue
  \childdocmanualtrue
  \if?#1?\else
    \def\jobname{#2}
  \fi
  \def\childdocjob{#2}
  \input{#2}
  \endinput
}
%    \end{macrocode}

% \macro{\childdocforward}
% The command |\childdocforward| redirects
% compilation to the main file or
% (if the optional argument is given) a child file.
% Parameters are set as if the main file
% or a child file starting with |\childdocof| was compiled.
% Then compilation is handed over to the main file:
%    \begin{macrocode}
\newcommand{\childdocforward}[2][]
{
  \begingroup
    \if?#1?
      \def\childdoctmp
      {
        \def\childdocname{#2}
        \def\childdocjob{#2}
        \def\jobname{#2}
        \input{#2}
        \endinput
      }
    \else
      \def\childdoctmp
      {
        \childdocdisable
        \def\childdocname{#2}
        \childdoctrue
        \includeonly{#2}
        \def\childdocjob{#1}
        \def\jobname{#1}
        \input{#1}
        \endinput
      }
    \fi
    \expandafter
  \endgroup
  \childdoctmp
}
%    \end{macrocode}

% \macro{\childdocforwardprefix}
% The command |\childdocforwardprefix| redirects
% compilation to the main or a child file by means of a pattern.
% The prefix |#1| in the current filename is replaced by |#2|
% and the suffix of the current filename is kept
% (it is assumed that the filename does not contain the substring `|~~~|'
% which is used as a delimiter).
% Compilation is handed over to the new file by |\childdocforward|:
%    \begin{macrocode}
\newcommand{\childdocforwardprefix}[3][]
{
  \begingroup
    \def\childdocextract #2##1~~~{\def\childdoctmp{\childdocforward[#1]{#3##1}}}
    \expandafter\childdocextract\childdocname~~~
    \expandafter
  \endgroup
  \childdoctmp
}
%    \end{macrocode}

% \macro{\childdoc}
% The deprecated macro |\childdoc| is a legacy version of |\childdocmain|:
%    \begin{macrocode}
\newcommand{\childdoc}{\childdocmain}
%    \end{macrocode}

% \macro{\childdocredirect}
% The deprecated macro |\childdocredirect| is a legacy version
% of |\childdocforward| and |\childdocforwardprefix|:
%    \begin{macrocode}
\newcommand{\childdocredirect}[2][]
{
  \begingroup
    \if?#1?
      \def\childdoctmp{\childdocforward{#2}}
    \else
      \def\childdoctmp{\childdocforwardprefix{#1}{#2}}
    \fi
    \expandafter
  \endgroup
  \childdoctmp
}
%    \end{macrocode}

%\iffalse
%</package>
%\fi
%
\endinput
\childdocforward{cdocsamp}"|\\
% |latex -jobname cdocscl1 \|\\
% |  "% \iffalse
%
% childdoc.dtx Copyright (C) 2017-2018 Niklas Beisert
%
% This work may be distributed and/or modified under the
% conditions of the LaTeX Project Public License, either version 1.3
% of this license or (at your option) any later version.
% The latest version of this license is in
%   http://www.latex-project.org/lppl.txt
% and version 1.3 or later is part of all distributions of LaTeX
% version 2005/12/01 or later.
%
% This work has the LPPL maintenance status `maintained'.
%
% The Current Maintainer of this work is Niklas Beisert.
%
% This work consists of the files childdoc.dtx and childdoc.ins
% and the derived files childdoc.def and cdocsamp.tex with
% cdocsch1.tex, cdocsch2.tex, cdocsdrf.tex, cdocsfn1.tex, cdocsfn2.tex.
%
%<package>\ifdefined\childdocmain\endinput\fi
%<package>\ProvidesFile{childdoc.def}[2018/12/30 v2.0 child document driver]
%<samplemain>\ProvidesFile{cdocsamp.tex}[2018/12/30 v2.0 sample for childdoc]
%<*driver>
%\ProvidesFile{childdoc.drv}[2018/12/30 v2.0 childdoc reference manual file]
\PassOptionsToClass{10pt,a4paper}{article}
\documentclass{ltxdoc}

\usepackage[margin=35mm]{geometry}
\usepackage{hyperref}
\usepackage{hyperxmp}
\usepackage[usenames]{color}

\hypersetup{colorlinks=true}
\hypersetup{pdfstartview=FitH}
\hypersetup{pdfpagemode=UseNone}
\hypersetup{pdfsource={}}
\hypersetup{pdflang={en-UK}}
\hypersetup{pdfcopyright={Copyright 2017-2018 Niklas Beisert.
  This work may be distributed and/or modified under the
  conditions of the LaTeX Project Public License, either version 1.3
  of this license or (at your option) any later version.}}
\hypersetup{pdflicenseurl={http://www.latex-project.org/lppl.txt}}
\hypersetup{pdfcontactaddress={ETH Zurich, ITP, HIT K,
  Wolfgang-Pauli-Strasse 27}}
\hypersetup{pdfcontactpostcode={8093}}
\hypersetup{pdfcontactcity={Zurich}}
\hypersetup{pdfcontactcountry={Switzerland}}
\hypersetup{pdfcontactemail={nbeisert@itp.phys.ethz.ch}}
\hypersetup{pdfcontacturl={http://people.phys.ethz.ch/\xmptilde nbeisert/}}

\newcommand{\secref}[1]{\hyperref[#1]{section \ref*{#1}}}

\parskip1ex
\parindent0pt
\let\olditemize\itemize
\def\itemize{\olditemize\parskip0pt}

\begin{document}

\title{The \textsf{childdoc} Package}
\hypersetup{pdftitle={The childdoc Package}}
\author{Niklas Beisert\\[2ex]
  Institut f\"ur Theoretische Physik\\
  Eidgen\"ossische Technische Hochschule Z\"urich\\
  Wolfgang-Pauli-Strasse 27, 8093 Z\"urich, Switzerland\\[1ex]
  \href{mailto:nbeisert@itp.phys.ethz.ch}
  {\texttt{nbeisert@itp.phys.ethz.ch}}}
\hypersetup{pdfauthor={Niklas Beisert}}
\hypersetup{pdfsubject={Manual for the LaTeX2e Package childdoc}}
\date{30 December 2018, \textsf{v2.0}}
\maketitle

\begin{abstract}\noindent
\textsf{childdoc} is a \LaTeXe{} package
that enables the direct compilation
of document sections included by |\include|
to individual files.
\end{abstract}

\begingroup
\parskip0ex
\tableofcontents
\endgroup

%%%%%%%%%%%%%%%%%%%%%%%%%%%%%%%%%%%%%%%%%%%%%%%%%%%%%%%%%%%%%%%%%%%%%%%%%%%%%%%%
%%%%%%%%%%%%%%%%%%%%%%%%%%%%%%%%%%%%%%%%%%%%%%%%%%%%%%%%%%%%%%%%%%%%%%%%%%%%%%%%
\section{Introduction}

\LaTeX{} provides a mechanism to structure a large document (such as a book)
into a main file and several child files (containing the chapters)
using the |\include| command.
This mechanism is beneficial for documents
which span hundreds of pages in order to
make the source file(s) more manageable.
Moreover, compilation can be restricted to
selected child files by means of the |\includeonly| command.
The latter feature can be used to reduce the compilation time while editing
(this was significantly more useful in the earlier days of \LaTeX{})
or to generate a smaller document which is easier to navigate.
Another application of |\includeonly| is to generate
documents consisting of selected parts of the complete document.

However, there are a few drawbacks of the plain |\include| mechanism:
\begin{itemize}
\item
The child files cannot be compiled on their own,
they can only be compiled via the main file.
A naive editing environment
(such as a text editor with an option
to have the current file processed by \LaTeX)
may require one to switch to the main file before compiling;
attempting to compile the child file produces errors.
\item
The main file must be modified (each time)
to adjust the |\includeonly| command
to the present needs. This easily leaves the main file in a messy state.
\item
The generated document will always carry the filename
of the main document. This is inconvenient if
several child files are to be compiled and
to be kept for distribution.
\end{itemize}

The present package provides a simple interface
to make child files individually compilable by \LaTeX{}.
Compiling a child file then has the same effect as compiling
the main file with an |\includeonly| command
to select the appropriate child.
Moreover the generated document will carry the name of the child
rather than the main file.
This resolves all three above issues.

This feature is meant to make the editing of books,
thesis documents and lecture notes somewhat more convenient.
However, the package can also be used efficiently for
composing a series of documents (such as exercise sheets)
which are typically distributed individually.
It then assists the author in generating the individual documents
(potentially in different versions)
as well as a document containing the collected series.
Another application is in developing style files
or other kinds of included material
where compilation of the style file could redirect
to a sample or test file.

%%%%%%%%%%%%%%%%%%%%%%%%%%%%%%%%%%%%%%%%%%%%%%%%%%%%%%%%%%%%%%%%%%%%%%%%%%%%%%%%
%%%%%%%%%%%%%%%%%%%%%%%%%%%%%%%%%%%%%%%%%%%%%%%%%%%%%%%%%%%%%%%%%%%%%%%%%%%%%%%%
\section{Usage}

First of all, the package \textsf{childdoc} is \emph{not} a standard
\LaTeXe{} |.sty| style file! Therefore it needs to be invoked in
a non-standard way.

%%%%%%%%%%%%%%%%%%%%%%%%%%%%%%%%%%%%%%%%%%%%%%%%%%%%%%%%%%%%%%%%%%%%%%%%%%%%%%%%
\subsection{Included Files}
\label{sec:include}

%%%%%%%%%%%%%%%%%%%%%%%%%%%%%%%%%%%%%%%%
\DescribeMacro{\childdocmain}
To use the package, add the commands
\begin{center}
\begin{tabular}{l}
|\input{childdoc.def}|\\
|\childdocmain{}|\\
\end{tabular}
\end{center}
at the very top of the main \LaTeX{} file,
in particular \emph{before} the |\documentclass| statement!
The argument of |\childdocmain| should be left empty
(but it must be present).

%%%%%%%%%%%%%%%%%%%%%%%%%%%%%%%%%%%%%%%%
\DescribeMacro{\childdocof}
Furthermore, add the commands
\begin{center}
\begin{tabular}{l}
|\input{childdoc.def}|\\
|\childdocof{|\textit{main}|}|\\
\end{tabular}
\end{center}
at the top of every child file \textit{child}
which is included by |\include{|\textit{child}|}|
from within the main file
(or at least for those files to be compiled individually).
The argument \textit{main} must be the filename of the main file.

There are a couple of
considerations in setting up the main and child documents:

%%%%%%%%%%%%%%%%%%%%%%%%%%%%%%%%%%%%%%%%
\paragraph{Restrictions.}

Please note the following restrictions:
\begin{itemize}
\item
|\childdocmain| must be called with one argument \textit{main}
to ensure compatibility with earlier version of the package.
It must either be empty (|\childdocmain{}|)
or precisely match the filename of the main file in which it is specified.
See \secref{sec:detection} for further information.
\item
The filename \textit{main} must be specified without the |.tex| extension.
\item
The filename \textit{main} is case sensitive
(even in case-insensitive file systems)
due to internal string comparison.
\item
The argument \textit{main} should be fully expanded, it cannot be a macro.
\item
Subdirectories and special characters should be avoided in filenames.
\item
The command |\childdocmain{|\textit{main}|}| must be followed by a whitespace.
It should not be followed immediately by another command
or by a comment mark `|%|'.
This is because the \TeX{} parser reads the token immediately following
the argument of |\childdocmain| and puts it
at the beginning of every child section;
however, a white\-space is ignored.
\end{itemize}

%%%%%%%%%%%%%%%%%%%%%%%%%%%%%%%%%%%%%%%%
\paragraph{Content of Main File.}

It is advisable to place all content in the child files included by |\include|.
Any output contained in the main file will appear in all child documents
unless suppressed manually;
it cannot be suppressed automatically by the |\includeonly| directive
and thus should normally be avoided.
A method to include some content in the main file
by means of conditional processing is described in \secref{sec:conditional}.

%%%%%%%%%%%%%%%%%%%%%%%%%%%%%%%%%%%%%%%%
\paragraph{Page Numbering.}

When only a part of the document is compiled,
the appropriate numbering of pages
(as well as other status parameters)
is determined from the |.aux| files.
The latter contain information from previous passes.
However this information needs to propagate through
all intermediate child documents.
Therefore the page numbering in child documents may well
be inconsistent until the complete document is compiled at least once.

A useful (if unconventional) way to always ensure a consistent
page numbering is to restart the numbering in each child document
and denote the pages by `\textit{child}|.|\textit{page}'
where \textit{child} represents the chapter/section number of the child file.
This can be achieved by the command
|\numberwithin{page}{|\textit{child}|}|
of the \textsf{amsmath} package
where \textit{child} can be |chapter| or |section|
depending on the chosen structuring.
Alternatively, one can modify the macro |\thepage| appropriately
and reset the counter |page| at the start of each child file.

%%%%%%%%%%%%%%%%%%%%%%%%%%%%%%%%%%%%%%%%%%%%%%%%%%%%%%%%%%%%%%%%%%%%%%%%%%%%%%%%
\subsection{Conditional Processing}
\label{sec:conditional}

The package provides a mechanism to compile different versions
of a document. To customise the versions further some conditional processing
can come in handy to distinguish which version is being compiled.
The package provides two macros to describe the compilation context:

%%%%%%%%%%%%%%%%%%%%%%%%%%%%%%%%%%%%%%%%
\DescribeMacro{\ifchilddoc}
The conditional |\ifchilddoc| distinguishes between the compilation of
child documents and the main document:
%
\begin{center}
|\ifchilddoc |\textit{child-code}| |[|\||else |\textit{main-code}]| \||fi|
\end{center}

%%%%%%%%%%%%%%%%%%%%%%%%%%%%%%%%%%%%%%%%
\DescribeMacro{\childdocname}
\DescribeMacro{\childdocjob}
The macro |\childdocname| contains the filename (without extension)
of the main or child file being processed.
Note that |\childdocjob| will always contain the name of the main file.

%%%%%%%%%%%%%%%%%%%%%%%%%%%%%%%%%%%%%%%%
\paragraph{Title Page.}

Conditional processing can be used to include a title or banner page
in the main document when proper precautions are taken.
Importantly, the code in the main file should ensure that the page counter
(as well as other status parameters which are stored in the |.aux| files)
takes the same value after the conditional processing.
Otherwise the page numbers may take divergent values
depending on which part is compiled.

For example, a title page could be declared by:
%
\begin{center}
\begin{tabular}{l}
|\ifchilddoc\||else|\\
|\addtocounter{page}{-1}|\\
\textit{code for title page}\\
|\newpage|\\
|\||fi|
\end{tabular}
\end{center}
%
A banner page for the child documents can be generated by:
%
\begin{center}
\begin{tabular}{l}
|\ifchilddoc|\\
|\addtocounter{page}{-1}|\\
\textit{code for banner page}\\
|\newpage|\\
|\||fi|
\end{tabular}
\end{center}
%
Here one could write a message such as:
\begin{center}
|This is the part \childdocname{} of \childdocjob{}.|
\end{center}

%%%%%%%%%%%%%%%%%%%%%%%%%%%%%%%%%%%%%%%%%%%%%%%%%%%%%%%%%%%%%%%%%%%%%%%%%%%%%%%%
\subsection{Flags}
\label{sec:flags}

The package makes it easy to generate different versions
of the main or child documents.
To this end compilation flags can be defined
and assigned different default values.
They will be particularly useful in conjunction
with the forwarding mechanism described in \secref{sec:forward}.

For example, it may be useful to have a flag |\version|
which can be set to |draft| or |final|.
The document source will contain some conditional code
depending on the value of |\version|.
Suppose further, the flag should default to |final| for the main file
and to |draft| for child files
which is a natural assignment for editing the document.
This is achieved by placing the following code
in the preamble of the main document
(below the |\childdocmain| directive):
%
\begin{center}
\begin{tabular}{l}
|\ifchilddoc|\\
|\providecommand{\version}{draft}|\\
|\||else|\\
|\providecommand{\version}{final}|\\
|\||fi|
\end{tabular}
\end{center}
%
The definition by |\providecommand| makes sure
that previous definitions are not overwritten.
Further statements |\providecommand{\version}{...}|
can thus be added before the above code to override it.

For the main file, one might add a line
(between |\childdocmain| and the above block)
%
\begin{center}
|%\ifchilddoc\||else\providecommand{\version}{draft}\||fi|
\end{center}
%
which can be uncommented to produce a draft version.
Likewise one can add a line to the very top of a child file
(above the |\childdocof{|\textit{main}|}| directive)
%
\begin{center}
|%\providecommand{\version}{final}|
\end{center}
%
which can be uncommented to produce the final version of this child document.

%%%%%%%%%%%%%%%%%%%%%%%%%%%%%%%%%%%%%%%%%%%%%%%%%%%%%%%%%%%%%%%%%%%%%%%%%%%%%%%%
\subsection{Forwarding}
\label{sec:forward}

Different versions of the main or child documents
using compilation flags as described in \secref{sec:flags}
can be (permanently) stored in different files
for convenient compilation, viewing and distribution.
To this end, the package defines a command
to pass on compilation to a different file:

%%%%%%%%%%%%%%%%%%%%%%%%%%%%%%%%%%%%%%%%
\DescribeMacro{\childdocforward}
The command |\childdocforward| redirects processing to
another source file:
%
\begin{center}
\begin{tabular}{l}
|\input{childdoc.def}|\\
|\childdocforward[|\textit{main}|]{|\textit{dest}|}|\\
\end{tabular}
\end{center}
%
The argument \textit{dest} is the destination file
(without extension).
It should be the main file or one of the child files.
Note that further \textsf{childdoc} directives
such as |\childdocof| and |\childdocforward|
in the indicated file will be processed in this form.
The optional argument \textit{main}
passes on directly to the main file \textit{main}
while pretending to compile the child \textit{dest}.
This form behaves as if \textit{dest}
issues |\childdocof{|\textit{main}|}| right away,
and no further \textsf{childdoc} directives will be processed.

%%%%%%%%%%%%%%%%%%%%%%%%%%%%%%%%%%%%%%%%
\DescribeMacro{\...prefix}
In the alternative form |\childdocforwardprefix|,
%
\begin{center}
\begin{tabular}{l}
|\input{childdoc.def}|\\
|\childdocforwardprefix[|\textit{main}|]{|\textit{prefix}|}{|\textit{dest}|}|
\end{tabular}
\end{center}
%
the destination file is determined by a pattern
depending on the current file:
To make this work, the current file must be called
`{\textit{prefix}\hspace{0.2em}\textit{suffix}}'
with \textit{prefix} matching precisely the argument.
Processing is then passed on to the file
`{\textit{dest}\hspace{0.2em}\textit{suffix}}'.
Surely, the same effect is achieved by
directly specifying the
argument `{\textit{dest}\hspace{0.2em}\textit{suffix}}'
in the first form.
However, that requires to set up a different file
for each child. With the alternative form of the command
all these files can have exactly the same content
which simplifies setting them up and maintaining them.

For example, the following file |draft.tex|
with a compilation flag |\version| as described in \secref{sec:flags}
compiles the main document as a draft:
%
\begin{center}
\begin{tabular}{l}
|\def\version{draft}|\\
|\input{childdoc.def}|\\
|\childdocforward{|\textit{main}|}|
\end{tabular}
\end{center}
%
Likewise, the following files |final|\textit{nn}|.tex|
compile the final version of the child document
|child|\textit{nn}|.tex|:
%
\begin{center}
\begin{tabular}{l}
|\def\version{final}|\\
|\input{childdoc.def}|\\
|\childdocforwardprefix{final}{child}|
\end{tabular}
\end{center}
%

Note that when several versions of a main file and/or of each child file
are to be generated, it may be convenient to set up a |Makefile| or
shell script to automatise the process.

%%%%%%%%%%%%%%%%%%%%%%%%%%%%%%%%%%%%%%%%%%%%%%%%%%%%%%%%%%%%%%%%%%%%%%%%%%%%%%%%
\subsection{Command Line Processing}
\label{sec:commandline}

The effect of redirection files can also be achieved by invoking
the \LaTeX{} compiler with a more elaborate command line.
Most conveniently this should be done as part
of a shell script or a |Makefile|.

When using \textsf{childdoc} in the main file, the following
command lines effectively perform a redirection
(note that depending on the shell being used,
backslashes may have to be doubled: `|\|' $\to$ `|\\|'):
%
\begin{center}
|... -jobname "|\textit{target}|" |\\|"|[\textit{flags}]%
|\input{childdoc.def}\childdocforward[|\textit{main}|]{|\textit{dest}|}"|
\end{center}
%
Here \textit{target} is the name of the output file,
\textit{main} is the name of the main file
and \textit{dest} is the name of the main or child file to be processed
(all filenames without extensions).
The optional argument \textit{main} can be omitted
if \textit{main} matches \textit{dest}.
Optionally, compilation \textit{flags} can be defined via |\def| commands.
This command line makes the \TeX{} engine believe
it is compiling the file \textit{target}
whose content is specified as the latter parameter.
The provided code then forwards the processing to
\textit{main} or \textit{dest} as described in \secref{sec:forward}.

%%%%%%%%%%%%%%%%%%%%%%%%%%%%%%%%%%%%%%%%%%%%%%%%%%%%%%%%%%%%%%%%%%%%%%%%%%%%%%%%
\subsection{Include by Input}
\label{sec:input}

Including child documents by |\include| has some restrictions by design.
Most notably, the content of a child document always occupies
its own set of pages; pages cannot be shared between child documents.
Usually, this behaviour makes perfect sense
because each child document contain an essential part of the document.
However, in some situations it may be desirable to compose
a document from a collection of parts
without having mandatory page breaks between then.
For this case, the package
provides a mechanism to include parts
by |\input| which can also be processed individually.
However, by construction this mechanism
requires manual handling of the content to be output.

%%%%%%%%%%%%%%%%%%%%%%%%%%%%%%%%%%%%%%%%
\DescribeMacro{\ifchilddocmanual}
The main file should be prepared as usual, see \secref{sec:include}.
However, the document body must make a distinction
between processing of an individual part and of the main document, e.g.:
%
\begin{center}
\begin{tabular}{l}
|\ifchilddocmanual|\\
|\input{\childdocname}|\\
|\||else|\\
\textit{document body with }|\input{|\textit{part}|}|\\
|\||fi|
\end{tabular}
\end{center}
%
The conditional |\ifchilddocmanual| is true whenever
a part to be included by |\input| is being compiled,
and the name of the part is stored in |\childdocname|.

%%%%%%%%%%%%%%%%%%%%%%%%%%%%%%%%%%%%%%%%
\DescribeMacro{\childdocby}
Each part to be included by |\input| should start with:
%
\begin{center}
\begin{tabular}{l}
|\input{childdoc.def}|\\
|\childdocby{|\textit{main}|}|\\
\end{tabular}
\end{center}
%
The directive |\childdocby| is similar to |\childdocof|
described in \secref{sec:include},
but the subsequent selection of content must be done manually.
To that end, both |\ifchilddoc| and |\ifchilddocmanual|
will be true upon processing of a part,
and the name of the part is stored in |\childdocname|.
Note that |\jobname| will be set to the filename of the current part
so that each part receives an individual |.aux| file
that does not interfere with the |.aux| file(s) of the main document.
This behaviour can be altered by the alternative form
|\childdocby[*]{|\textit{main}|}| (with a non-empty optional argument)
which uses the |.aux| file of the main document
by setting |\jobname| to \textit{main}.

%%%%%%%%%%%%%%%%%%%%%%%%%%%%%%%%%%%%%%%%%%%%%%%%%%%%%%%%%%%%%%%%%%%%%%%%%%%%%%%%
\subsection{Driver Development}
\label{sec:driver}

The \textsf{childdoc} mechanism can also be use for the development
of definition files such as \LaTeX{} styles or classes.
This case differs from the above setup with multiple parts
included by |\include| in that no |\includeonly| should be invoked.
This can be achieved by starting the include file
(before |\ProvidesPackage|) with:
%
\begin{center}
\begin{tabular}{l}
|\input{childdoc.def}|\\
|\childdocforward{|\textit{main}|}|\\
\end{tabular}
\end{center}
%
or alternatively with:
%
\begin{center}
\begin{tabular}{l}
|\input{childdoc.def}|\\
|\childdocby{|\textit{main}|}|\\
\end{tabular}
\end{center}
%
Both forms have slightly different effects as described above.
The main file is prepared as usual, see \secref{sec:include}.

%%%%%%%%%%%%%%%%%%%%%%%%%%%%%%%%%%%%%%%%%%%%%%%%%%%%%%%%%%%%%%%%%%%%%%%%%%%%%%%%
\subsection{Legacy Detection}
\label{sec:detection}

The directive |\childdocmain| in the main file can detect
whether the complete document or merely a child is to be compiled
even without using the directive |\childdocof|.
This method is deprecated because it is less robust
and there is no compelling reason to use it;
it is merely provided for backward compatibility
and it may be removed in future versions.

If the detection mechanism is to be used,
it is mandatory to correctly specify
the filename of the main file as the argument of |\childdocmain|:
%
\begin{center}
\begin{tabular}{l}
|\input{childdoc.def}|\\
|\childdocmain{|\textit{main}|}|\\
\end{tabular}
\end{center}
%
If |\jobname| does not match the argument \textit{main} of |\childdocmain|,
it is assumed that |\jobname| points to the child file to be compiled.
When using |\childdocmain| with the main file specified as argument,
it suffices to start a child file
with just |\input{|\textit{main}|}|
without loading of the package and using |\childdocof|.
If instead all processing is done
with the appropriate \textsf{childdoc} directives,
the argument of \textit{main} of |\childdocmain| can be empty.

An alternative version of the command line processing described
in \secref{sec:commandline} using the detection mechanism reads:
%
\begin{center}
|... -jobname "|\textit{target}|" "|[\textit{flags}]%
[|\def\jobname{|\textit{dest}|}|]|\input{|\textit{main}|}"|
\end{center}

%%%%%%%%%%%%%%%%%%%%%%%%%%%%%%%%%%%%%%%%%%%%%%%%%%%%%%%%%%%%%%%%%%%%%%%%%%%%%%%%
\subsection{Manual Code}
\label{sec:manual}

In case one cannot be certain whether the definitions file |childdoc.def|
is installed on the target \TeX{} distribution
and one prefers not to ship it,
it is conceivable to paste a few relevant commands into the sources.

To that end, drop all statements |\input{childdoc.def}|
and perform the replacements as outlined below.
Instead of |\childdocmain{|\textit{main}|}| add the following code
to the top of the main file:
%
\begin{center}
\begin{tabular}{l}
|\||ifdefined\childdocname\endinput\||fi\newif\ifchilddoc|\\
|\edef\childdocname{\scantokens\expandafter{\jobname\noexpand}}|\\
|\def\childdocmain{|\textit{main}|}\||ifx\childdocmain\childdocname\||else|\\
|\childdoctrue\includeonly{\childdocname}\let\jobname\childdocmain\||fi|\\
\end{tabular}
\end{center}
%
Instead of |\childdocof{|\textit{main}|}| just include the main file
at the top of each child file:
%
\begin{center}
|\input{|\textit{main}|}|
\end{center}
%
A simple redirection |\childdocforward{|\textit{dest}|}| is achieved by:
%
\begin{center}
|\def\jobname{|\textit{dest}|}\input{\jobname}|
\end{center}
%
The redirection with prefix
|\childdocforwardprefix[|\textit{prefix}|]{|\textit{dest}|}|
is accomplished by:
%
\begin{center}
\begin{tabular}{l}
|{\edef\jobname{\scantokens\expandafter{\jobname\noexpand}}|\\
|\def\redirectjob |\textit{prefix}|#1~~~{\gdef\jobname{|\textit{dest}|#1}}|\\
|\expandafter\redirectjob\jobname~~~}\input{\jobname}|
\end{tabular}
\end{center}

In an alternative approach,
child documents can be compiled by a specific command line
without additional code or specific definitions:
%
\begin{center}
|... -jobname "|\textit{target}|" "|[\textit{flags}]%
|\includeonly{|\textit{dest}|}\input{|\textit{main}|}"|
\end{center}
%

%%%%%%%%%%%%%%%%%%%%%%%%%%%%%%%%%%%%%%%%%%%%%%%%%%%%%%%%%%%%%%%%%%%%%%%%%%%%%%%%
%%%%%%%%%%%%%%%%%%%%%%%%%%%%%%%%%%%%%%%%%%%%%%%%%%%%%%%%%%%%%%%%%%%%%%%%%%%%%%%%
\section{Information}

%%%%%%%%%%%%%%%%%%%%%%%%%%%%%%%%%%%%%%%%%%%%%%%%%%%%%%%%%%%%%%%%%%%%%%%%%%%%%%%%
\subsection{Copyright}

Copyright \copyright{} 2017--2018 Niklas Beisert

This work may be distributed and/or modified under the
conditions of the \LaTeX{} Project Public License, either version 1.3
of this license or (at your option) any later version.
The latest version of this license is in
  \url{http://www.latex-project.org/lppl.txt}
and version 1.3 or later is part of all distributions of \LaTeX{}
version 2005/12/01 or later.

This work has the LPPL maintenance status `maintained'.

The Current Maintainer of this work is Niklas Beisert.

This work consists of the files |README.txt|, |childdoc.ins| and |childdoc.dtx|
as well as the derived files |childdoc.def|, |cdocsamp.tex|
with |cdocsch1.tex|, |cdocsch2.tex|, |cdocspt3.tex|, |cdocspt4.tex|,
|cdocsdrf.tex|, |cdocsfn1.tex|, |cdocsfn2.tex|
as well as |childdoc.pdf|.

%%%%%%%%%%%%%%%%%%%%%%%%%%%%%%%%%%%%%%%%%%%%%%%%%%%%%%%%%%%%%%%%%%%%%%%%%%%%%%%%
\subsection{Files and Installation}

The package consists of the files:
%
\begin{center}
\begin{tabular}{ll}
    |README.txt|   & readme file \\
    |childdoc.ins| & installation file \\
    |childdoc.dtx| & source file \\
    |childdoc.def| & definition file \\
    |cdocsamp.tex| & sample main file \\
    |cdocsch1.tex| & sample include file \\
    |cdocsch2.tex| & sample include file \\
    |cdocspt3.tex| & sample part file \\
    |cdocspt4.tex| & sample part file \\
    |cdocsdrf.tex| & sample redirection file \\
    |cdocsfn1.tex| & sample redirection file \\
    |cdocsfn2.tex| & sample redirection file \\
    |childdoc.pdf| & manual
\end{tabular}
\end{center}
%
The distribution consists of the files
|README.txt|, |childdoc.ins| and |childdoc.dtx|.
%
\begin{itemize}
\item
Run (pdf)\LaTeX{} on |childdoc.dtx|
to compile the manual |childdoc.pdf| (this file).
\item
Run \LaTeX{} on |childdoc.ins| to create the definitions file |childdoc.def|
and the sample |cdocsamp.tex| with include files
|cdocsch1.tex|, |cdocsch2.tex|, |cdocspt3.tex|, |cdocspt4.tex|,
|cdocsdrf.tex|, |cdocsfn1.tex|, |cdocsfn2.tex|.
Then copy the file |childdoc.def| to an appropriate directory of your \LaTeX{}
distribution, e.g.\ \textit{texmf-root}|/tex/latex/childdoc|.
\end{itemize}

%%%%%%%%%%%%%%%%%%%%%%%%%%%%%%%%%%%%%%%%%%%%%%%%%%%%%%%%%%%%%%%%%%%%%%%%%%%%%%%%
\subsection{Related CTAN Packages}

There are several other packages which offer a similar functionality:
%
\begin{itemize}
\item
The packages
\href{http://ctan.org/pkg/docmute}{\textsf{docmute}},
\href{http://ctan.org/pkg/includex}{\textsf{includex}} and
\href{http://ctan.org/pkg/standalone}{\textsf{standalone}}
provide commands to include only the document body of
a child file thus allowing both files to be compiled individually.
\item
The packages \href{http://ctan.org/pkg/subdocs}{\textsf{subdocs}}
and \href{http://ctan.org/pkg/subfiles}{\textsf{subfiles}}
provide structures in which the main and child documents can be
encapsulated and allowing them to be compiled individually.
The inclusion mechanism is different from the conventional |\include|.
\item
The package \href{http://ctan.org/pkg/combine}{\textsf{combine}}
is an elaborate solution to combine several documents into one.
\end{itemize}
%
See also the CTAN topic \href{http://ctan.org/topic/subdocs}{\textsf{subdocs}}
for further related packages.
The present package differs from the above solutions in that
a document structure constructed with the conventional |\include| mechanism
just needs two extra commands at the top of every file
such that all constituent files can be compiled individually.

%%%%%%%%%%%%%%%%%%%%%%%%%%%%%%%%%%%%%%%%%%%%%%%%%%%%%%%%%%%%%%%%%%%%%%%%%%%%%%%%
%\subsection{Feature Suggestions}
%
%The following is a list of features which may be useful for future
%versions of this package:
%%
%\begin{itemize}
%\item
%\ldots
%\end{itemize}

%%%%%%%%%%%%%%%%%%%%%%%%%%%%%%%%%%%%%%%%%%%%%%%%%%%%%%%%%%%%%%%%%%%%%%%%%%%%%%%%
\subsection{Revision History}

%%%%%%%%%%%%%%%%%%%%%%%%%%%%%%%%%%%%%%%%
\paragraph{v2.0:} 2018/12/30

\begin{itemize}
\item
immediate forward processing
\item
added |\childdocby| mechanism
\item
manual restructured
\end{itemize}

%%%%%%%%%%%%%%%%%%%%%%%%%%%%%%%%%%%%%%%%
\paragraph{v1.6:} 2018/01/17

\begin{itemize}
\item
application for development of include files
\item
corrections to manual
\end{itemize}

%%%%%%%%%%%%%%%%%%%%%%%%%%%%%%%%%%%%%%%%
\paragraph{v1.5:} 2017/05/21

\begin{itemize}
\item
more complete structuring introduced
\item
|\childdocof| introduced
\item
|\childdoc| renamed to |\childdocmain|
\item
|\childredirect| renamed to |\childdocforward| and |\childdocforwardprefix|
and functionality expanded
\end{itemize}

%%%%%%%%%%%%%%%%%%%%%%%%%%%%%%%%%%%%%%%%
\paragraph{v1.0:} 2017/04/27

\begin{itemize}
\item
manual and install package
\item
first version published on CTAN
\end{itemize}

%%%%%%%%%%%%%%%%%%%%%%%%%%%%%%%%%%%%%%%%
\paragraph{v0.6:} 2017/04/26

\begin{itemize}
\item
redirection mechanism added
\end{itemize}

%%%%%%%%%%%%%%%%%%%%%%%%%%%%%%%%%%%%%%%%
\paragraph{v0.5:} 2017/04/26

\begin{itemize}
\item
functionality in definition file
\end{itemize}


%%%%%%%%%%%%%%%%%%%%%%%%%%%%%%%%%%%%%%%%%%%%%%%%%%%%%%%%%%%%%%%%%%%%%%%%%%%%%%%%
%%%%%%%%%%%%%%%%%%%%%%%%%%%%%%%%%%%%%%%%%%%%%%%%%%%%%%%%%%%%%%%%%%%%%%%%%%%%%%%%
%%%%%%%%%%%%%%%%%%%%%%%%%%%%%%%%%%%%%%%%%%%%%%%%%%%%%%%%%%%%%%%%%%%%%%%%%%%%%%%%
\appendix

\settowidth\MacroIndent{\rmfamily\scriptsize 000\ }

 \DocInput{childdoc.dtx}

\end{document}
%</driver>
% \fi
%
% %%%%%%%%%%%%%%%%%%%%%%%%%%%%%%%%%%%%%%%%%%%%%%%%%%%%%%%%%%%%%%%%%%%%%%%%%%%%%%
% %%%%%%%%%%%%%%%%%%%%%%%%%%%%%%%%%%%%%%%%%%%%%%%%%%%%%%%%%%%%%%%%%%%%%%%%%%%%%%
% \section{Sample}
%\iffalse
%<*samplemain>
%\fi
%
% The following presents a sample document
% with two chapters, two parts, a title page,
% a compile flag as well as three forwarding files to set the flag.
% It consists of eight |.tex| files:
% \begin{center}
% \begin{tabular}{ll}
% |cdocsamp.tex|&main file\\
% |cdocsch1.tex|&include file for chapter 1\\
% |cdocsch2.tex|&include file for chapter 2\\
% |cdocspt3.tex|&include file for part 3\\
% |cdocspt4.tex|&include file for part 4\\
% |cdocsdrf.tex|&forwarding file for main file in draft mode\\
% |cdocsfi1.tex|&forwarding file for final version of chapter 1\\
% |cdocsfi2.tex|&forwarding file for final version of chapter 2\\
% \end{tabular}
% \end{center}
% Each of the eight files can be compiled directly by the \LaTeX{} compiler.
%
% %%%%%%%%%%%%%%%%%%%%%%%%%%%%%%%%%%%%%%
% \paragraph{Main File.}
%
% The main file is called |cdocsamp.tex|.
%
% Load the \textsf{childdoc} definitions and
% declare the filename for the main document:
%    \begin{macrocode}
\input{childdoc.def}
\childdocmain{}
%    \end{macrocode}

% Optional override for |\version| flag:
%    \begin{macrocode}
%%\ifchilddoc\else\providecommand{\version}{draft}\fi
%    \end{macrocode}

% Define the default values for the |\version| flag
% (|final| for the main file and |draft| for childs):
%    \begin{macrocode}
\ifchilddoc
\providecommand{\version}{draft}
\else
\providecommand{\version}{final}
\fi
%    \end{macrocode}

% Load the standard document class:
%    \begin{macrocode}
\documentclass[12pt]{article}
%    \end{macrocode}

% Start the document body:
%    \begin{macrocode}
\begin{document}
%    \end{macrocode}

% Declare a title page.
% Print title, part of document being processed and version flag:
%    \begin{macrocode}
\addtocounter{page}{-1}
\begin{center}
{\LARGE\bfseries{}childdoc example\par}
\vspace{1cm}
\ifchilddoc
\ifchilddocmanual part\else chapter\fi:
`\childdocname' of `\childdocjob'\par
\else
main document: `\childdocjob'\par
\fi
version: \version\par
\end{center}
\newpage
%    \end{macrocode}

% Manually include selected file,
% otherwise process as usual:
%    \begin{macrocode}
\ifchilddocmanual
\section*{part `\childdocname'}
\input{\childdocname}
\else
%    \end{macrocode}

% Include the two chapters:
%    \begin{macrocode}
\include{cdocsch1}
\include{cdocsch2}
%    \end{macrocode}

% Include the two parts unless only chapters should be displayed:
%    \begin{macrocode}
\ifchilddoc\else
\section{part three}
\input{cdocspt3}
\section{part four}
\input{cdocspt4}
\fi
%    \end{macrocode}

% Process as usual until here:
%    \begin{macrocode}
\fi
%    \end{macrocode}

% End of document body:
%    \begin{macrocode}
\end{document}
%    \end{macrocode}
%\iffalse
%</samplemain>
%\fi
%
% %%%%%%%%%%%%%%%%%%%%%%%%%%%%%%%%%%%%%%
% \paragraph{Chapter Include Files.}
%
% The include files are called |cdocsch1.tex| and |cdocsch2.tex|.
%
%\iffalse
%<*samplechap1|samplechap2>
%\fi

% Optional override for |\version| flag:
%    \begin{macrocode}
%%\providecommand{\version}{final}
%    \end{macrocode}

% Include the main document:
%    \begin{macrocode}
\input{childdoc.def}
\childdocof{cdocsamp}
%    \end{macrocode}

%\iffalse
%</samplechap1|samplechap2>
%\fi
%
%\iffalse
%<*samplechap1>
%\fi
% Some text for chapter 1:
%    \begin{macrocode}
\section{one}
some text in chapter one
%    \end{macrocode}

%\iffalse
%</samplechap1>
%\fi
% Some text for chapter 2:
%\iffalse
%<*samplechap2>
%\fi
%    \begin{macrocode}
\section{two}
more text in chapter two
%    \end{macrocode}

%\iffalse
%</samplechap2>
%\fi
%
% %%%%%%%%%%%%%%%%%%%%%%%%%%%%%%%%%%%%%%
% \paragraph{Part Include Files.}
%
% The include files are called |cdocspt3.tex| and |cdocspt4.tex|.
%
%\iffalse
%<*samplepart3|samplepart4>
%\fi

% Optional override for |\version| flag:
%    \begin{macrocode}
%%\providecommand{\version}{final}
%    \end{macrocode}

% Include the main document:
%    \begin{macrocode}
\input{childdoc.def}
\childdocby{cdocsamp}
%    \end{macrocode}

%\iffalse
%</samplepart3|samplepart4>
%\fi
%
%\iffalse
%<*samplepart3>
%\fi
% Some text for part 3:
%    \begin{macrocode}
some text in part three
%    \end{macrocode}

%\iffalse
%</samplepart3>
%\fi
% Some text for part 4:
%\iffalse
%<*samplepart4>
%\fi
%    \begin{macrocode}
more text in part four
%    \end{macrocode}

%\iffalse
%</samplepart4>
%\fi
%
% %%%%%%%%%%%%%%%%%%%%%%%%%%%%%%%%%%%%%%
% \paragraph{Forwarding for a Complete Draft.}
%
% The following forwarding file |cdocsdrf.tex|
% compiles the main document in draft mode:
%\iffalse
%<*sampledraft>
%\fi
%    \begin{macrocode}
\def\version{draft}
\input{childdoc.def}
\childdocforward{cdocsamp}
%    \end{macrocode}

%\iffalse
%</sampledraft>
%\fi
%
% %%%%%%%%%%%%%%%%%%%%%%%%%%%%%%%%%%%%%%
% \paragraph{Forwarding for Final Version of the Chapters.}
%
% The following forwarding files |cdocsfn1.tex| and |cdocsfn2.tex|
% (with identical content)
% compile the final versions of the child documents
% |cdocsch1.tex| and |cdocsch2.tex|, respectively:
%\iffalse
%<*samplefinal>
%\fi
%    \begin{macrocode}
\def\version{final}
\input{childdoc.def}
\childdocforwardprefix[cdocsamp]{cdocsfn}{cdocsch}
%    \end{macrocode}

%\iffalse
%</samplefinal>
%\fi
%
% %%%%%%%%%%%%%%%%%%%%%%%%%%%%%%%%%%%%%%
% \paragraph{Command Line Processing.}
%
% The following three command lines generate the output files
% |cdocscld|, |cdocscl1| and |cdocscl2|
% which should be identical to
% |cdocsdrf|, |cdocsch1| and |cdocsfn2|, respectively:
% \begin{center}
% \begin{tabular}{l}
% |latex -jobname cdocscld \|\\
% |  "\def\version{draft}\input{childdoc.def}\childdocforward{cdocsamp}"|\\
% |latex -jobname cdocscl1 \|\\
% |  "\input{childdoc.def}\childdocforward[cdocsamp]{cdocsch1}"|\\
% |latex -jobname cdocscl2 \|\\
% |  "\def\version{final}\input{childdoc.def}\childdocforward{cdocsch2}"|
% \end{tabular}
% \end{center}
% Note that the trailing backslash on each first line
% merely continues the input to the second line
% (for convenient cut ant paste).
% Furthermore, the command |latex| can be replaced by any
% of its alternative versions such as |pdflatex|.
%
% %%%%%%%%%%%%%%%%%%%%%%%%%%%%%%%%%%%%%%%%%%%%%%%%%%%%%%%%%%%%%%%%%%%%%%%%%%%%%%
% %%%%%%%%%%%%%%%%%%%%%%%%%%%%%%%%%%%%%%%%%%%%%%%%%%%%%%%%%%%%%%%%%%%%%%%%%%%%%%
% \section{Implementation}
%\iffalse
%<*package>
%\fi
%
% This section describes the definitions file |childdoc.def|.

% The definitions cannot be loaded using |\usepackage| or |\RequirePackage|
% which has a mechanism to prevent loading a style file more than once.
% When loading the definitions by means of |\input|
% multiple instances have to be prevented manually:
%\iffalse
%This code needs to be before the `\ProvidesFile' directive
%which is defined at the beginning of this file.
%Therefore it is also placed there and commented out here.
%</package>
%<*discard>
%\fi
%    \begin{macrocode}
\ifdefined\childdocmain\endinput\fi
%    \end{macrocode}
%\iffalse
%</discard>
%<*package>
%\fi
%
% \macro{\ifchilddoc}
% \macro{\ifchilddocmanual}
% The conditional |\ifchilddoc| tells whether a
% child (true) or main (false) document is being compiled.
% The conditional |\ifchilddocmanual| tells whether
% the |\includeonly| mechanism is used (false) or
% the selection of child files must be performed manually (true).
% The definitions initialise to false:
%    \begin{macrocode}
\newif\ifchilddoc
\newif\ifchilddocmanual
%    \end{macrocode}

% \macro{\childdocname}
% \macro{\childdocjob}
% The macro |\childdocname| stores the name of the main document
% to be compiled. The macro |\childdocjob| stores the name of
% the document on which the \LaTeX{} compiler was originally invoked.
% The content of |\jobname| cannot be compared
% to filenames specified in the source due to different catcodes.
% The following code rescans |\jobname|, stores the result
% in |\childdocname| and saves a copy in |\childdocjob|:
%    \begin{macrocode}
\edef\childdocname{\scantokens\expandafter{\jobname\noexpand}}
\let\childdocjob\childdocname
%    \end{macrocode}

% \macro{\childdocdisable}
% The macro |\childdocdisable| prevents the main file
% from being processed more than once.
% At this stage, the main document command |\childdocmain|
% is assumed to be called once again where it should do nothing.
% Any subsequent call to it should prevent
% a secondary processing of the main document
% It overwrites the forwarding commands
% |\childdocof| and |\childdocforward|
% with empty macros to prevent further inclusions of the main document:
%    \begin{macrocode}
\newcommand{\childdocdisable}
{
  \renewcommand{\childdocmain}[1]{\renewcommand{\childdocmain}[1]{\endinput}}
  \renewcommand{\childdocof}[1]{}
  \renewcommand{\childdocby}[2][]{}
  \renewcommand{\childdocforward}[2][]{}
  \renewcommand{\childdocdisable}{}
}
%    \end{macrocode}

% \macro{\childdocmain}
% The macro |\childdocmain| is to be called at the top of the main file
% with nothing or the main filename (without extension) as argument.
% First, it breaks loops.
% If the argument is not empty and does not match |\childdocname|
% (which is set by the first inclusion of |childdoc.def|),
% |\ifchilddoc| is set to true, |\includeonly| is applied to the child file
% and |\jobname| is set to the main file
% (for proper handling of |.aux| files):
%    \begin{macrocode}
\newcommand{\childdocmain}[1]
{
  \childdocdisable\childdocmain{}
  \if?#1?\else
    \begingroup
      \def\childdoctmp{#1}
      \ifx\childdoctmp\childdocname
        \def\childdoctmp{}
      \else
        \def\childdoctmp
        {
          \childdoctrue
          \includeonly{\childdocname}
          \def\childdocjob{#1}
          \def\jobname{#1}
        }
      \fi
      \expandafter
    \endgroup
    \childdoctmp
  \fi
}
%    \end{macrocode}

% \macro{\childdocof}
% The command |\childdocof| redirects
% compilation to the main file |#1|.
%    \begin{macrocode}
\newcommand{\childdocof}[1]
{
  \childdocdisable
  \childdoctrue
  \includeonly{\childdocname}
  \def\jobname{#1}
  \def\childdocjob{#1}
  \input{#1}
}
%    \end{macrocode}

% \macro{\childdocby}
% The command |\childdocby| ....
%    \begin{macrocode}
\newcommand{\childdocby}[2][]
{
  \childdocdisable
  \childdoctrue
  \childdocmanualtrue
  \if?#1?\else
    \def\jobname{#2}
  \fi
  \def\childdocjob{#2}
  \input{#2}
  \endinput
}
%    \end{macrocode}

% \macro{\childdocforward}
% The command |\childdocforward| redirects
% compilation to the main file or
% (if the optional argument is given) a child file.
% Parameters are set as if the main file
% or a child file starting with |\childdocof| was compiled.
% Then compilation is handed over to the main file:
%    \begin{macrocode}
\newcommand{\childdocforward}[2][]
{
  \begingroup
    \if?#1?
      \def\childdoctmp
      {
        \def\childdocname{#2}
        \def\childdocjob{#2}
        \def\jobname{#2}
        \input{#2}
        \endinput
      }
    \else
      \def\childdoctmp
      {
        \childdocdisable
        \def\childdocname{#2}
        \childdoctrue
        \includeonly{#2}
        \def\childdocjob{#1}
        \def\jobname{#1}
        \input{#1}
        \endinput
      }
    \fi
    \expandafter
  \endgroup
  \childdoctmp
}
%    \end{macrocode}

% \macro{\childdocforwardprefix}
% The command |\childdocforwardprefix| redirects
% compilation to the main or a child file by means of a pattern.
% The prefix |#1| in the current filename is replaced by |#2|
% and the suffix of the current filename is kept
% (it is assumed that the filename does not contain the substring `|~~~|'
% which is used as a delimiter).
% Compilation is handed over to the new file by |\childdocforward|:
%    \begin{macrocode}
\newcommand{\childdocforwardprefix}[3][]
{
  \begingroup
    \def\childdocextract #2##1~~~{\def\childdoctmp{\childdocforward[#1]{#3##1}}}
    \expandafter\childdocextract\childdocname~~~
    \expandafter
  \endgroup
  \childdoctmp
}
%    \end{macrocode}

% \macro{\childdoc}
% The deprecated macro |\childdoc| is a legacy version of |\childdocmain|:
%    \begin{macrocode}
\newcommand{\childdoc}{\childdocmain}
%    \end{macrocode}

% \macro{\childdocredirect}
% The deprecated macro |\childdocredirect| is a legacy version
% of |\childdocforward| and |\childdocforwardprefix|:
%    \begin{macrocode}
\newcommand{\childdocredirect}[2][]
{
  \begingroup
    \if?#1?
      \def\childdoctmp{\childdocforward{#2}}
    \else
      \def\childdoctmp{\childdocforwardprefix{#1}{#2}}
    \fi
    \expandafter
  \endgroup
  \childdoctmp
}
%    \end{macrocode}

%\iffalse
%</package>
%\fi
%
\endinput
\childdocforward[cdocsamp]{cdocsch1}"|\\
% |latex -jobname cdocscl2 \|\\
% |  "\def\version{final}% \iffalse
%
% childdoc.dtx Copyright (C) 2017-2018 Niklas Beisert
%
% This work may be distributed and/or modified under the
% conditions of the LaTeX Project Public License, either version 1.3
% of this license or (at your option) any later version.
% The latest version of this license is in
%   http://www.latex-project.org/lppl.txt
% and version 1.3 or later is part of all distributions of LaTeX
% version 2005/12/01 or later.
%
% This work has the LPPL maintenance status `maintained'.
%
% The Current Maintainer of this work is Niklas Beisert.
%
% This work consists of the files childdoc.dtx and childdoc.ins
% and the derived files childdoc.def and cdocsamp.tex with
% cdocsch1.tex, cdocsch2.tex, cdocsdrf.tex, cdocsfn1.tex, cdocsfn2.tex.
%
%<package>\ifdefined\childdocmain\endinput\fi
%<package>\ProvidesFile{childdoc.def}[2018/12/30 v2.0 child document driver]
%<samplemain>\ProvidesFile{cdocsamp.tex}[2018/12/30 v2.0 sample for childdoc]
%<*driver>
%\ProvidesFile{childdoc.drv}[2018/12/30 v2.0 childdoc reference manual file]
\PassOptionsToClass{10pt,a4paper}{article}
\documentclass{ltxdoc}

\usepackage[margin=35mm]{geometry}
\usepackage{hyperref}
\usepackage{hyperxmp}
\usepackage[usenames]{color}

\hypersetup{colorlinks=true}
\hypersetup{pdfstartview=FitH}
\hypersetup{pdfpagemode=UseNone}
\hypersetup{pdfsource={}}
\hypersetup{pdflang={en-UK}}
\hypersetup{pdfcopyright={Copyright 2017-2018 Niklas Beisert.
  This work may be distributed and/or modified under the
  conditions of the LaTeX Project Public License, either version 1.3
  of this license or (at your option) any later version.}}
\hypersetup{pdflicenseurl={http://www.latex-project.org/lppl.txt}}
\hypersetup{pdfcontactaddress={ETH Zurich, ITP, HIT K,
  Wolfgang-Pauli-Strasse 27}}
\hypersetup{pdfcontactpostcode={8093}}
\hypersetup{pdfcontactcity={Zurich}}
\hypersetup{pdfcontactcountry={Switzerland}}
\hypersetup{pdfcontactemail={nbeisert@itp.phys.ethz.ch}}
\hypersetup{pdfcontacturl={http://people.phys.ethz.ch/\xmptilde nbeisert/}}

\newcommand{\secref}[1]{\hyperref[#1]{section \ref*{#1}}}

\parskip1ex
\parindent0pt
\let\olditemize\itemize
\def\itemize{\olditemize\parskip0pt}

\begin{document}

\title{The \textsf{childdoc} Package}
\hypersetup{pdftitle={The childdoc Package}}
\author{Niklas Beisert\\[2ex]
  Institut f\"ur Theoretische Physik\\
  Eidgen\"ossische Technische Hochschule Z\"urich\\
  Wolfgang-Pauli-Strasse 27, 8093 Z\"urich, Switzerland\\[1ex]
  \href{mailto:nbeisert@itp.phys.ethz.ch}
  {\texttt{nbeisert@itp.phys.ethz.ch}}}
\hypersetup{pdfauthor={Niklas Beisert}}
\hypersetup{pdfsubject={Manual for the LaTeX2e Package childdoc}}
\date{30 December 2018, \textsf{v2.0}}
\maketitle

\begin{abstract}\noindent
\textsf{childdoc} is a \LaTeXe{} package
that enables the direct compilation
of document sections included by |\include|
to individual files.
\end{abstract}

\begingroup
\parskip0ex
\tableofcontents
\endgroup

%%%%%%%%%%%%%%%%%%%%%%%%%%%%%%%%%%%%%%%%%%%%%%%%%%%%%%%%%%%%%%%%%%%%%%%%%%%%%%%%
%%%%%%%%%%%%%%%%%%%%%%%%%%%%%%%%%%%%%%%%%%%%%%%%%%%%%%%%%%%%%%%%%%%%%%%%%%%%%%%%
\section{Introduction}

\LaTeX{} provides a mechanism to structure a large document (such as a book)
into a main file and several child files (containing the chapters)
using the |\include| command.
This mechanism is beneficial for documents
which span hundreds of pages in order to
make the source file(s) more manageable.
Moreover, compilation can be restricted to
selected child files by means of the |\includeonly| command.
The latter feature can be used to reduce the compilation time while editing
(this was significantly more useful in the earlier days of \LaTeX{})
or to generate a smaller document which is easier to navigate.
Another application of |\includeonly| is to generate
documents consisting of selected parts of the complete document.

However, there are a few drawbacks of the plain |\include| mechanism:
\begin{itemize}
\item
The child files cannot be compiled on their own,
they can only be compiled via the main file.
A naive editing environment
(such as a text editor with an option
to have the current file processed by \LaTeX)
may require one to switch to the main file before compiling;
attempting to compile the child file produces errors.
\item
The main file must be modified (each time)
to adjust the |\includeonly| command
to the present needs. This easily leaves the main file in a messy state.
\item
The generated document will always carry the filename
of the main document. This is inconvenient if
several child files are to be compiled and
to be kept for distribution.
\end{itemize}

The present package provides a simple interface
to make child files individually compilable by \LaTeX{}.
Compiling a child file then has the same effect as compiling
the main file with an |\includeonly| command
to select the appropriate child.
Moreover the generated document will carry the name of the child
rather than the main file.
This resolves all three above issues.

This feature is meant to make the editing of books,
thesis documents and lecture notes somewhat more convenient.
However, the package can also be used efficiently for
composing a series of documents (such as exercise sheets)
which are typically distributed individually.
It then assists the author in generating the individual documents
(potentially in different versions)
as well as a document containing the collected series.
Another application is in developing style files
or other kinds of included material
where compilation of the style file could redirect
to a sample or test file.

%%%%%%%%%%%%%%%%%%%%%%%%%%%%%%%%%%%%%%%%%%%%%%%%%%%%%%%%%%%%%%%%%%%%%%%%%%%%%%%%
%%%%%%%%%%%%%%%%%%%%%%%%%%%%%%%%%%%%%%%%%%%%%%%%%%%%%%%%%%%%%%%%%%%%%%%%%%%%%%%%
\section{Usage}

First of all, the package \textsf{childdoc} is \emph{not} a standard
\LaTeXe{} |.sty| style file! Therefore it needs to be invoked in
a non-standard way.

%%%%%%%%%%%%%%%%%%%%%%%%%%%%%%%%%%%%%%%%%%%%%%%%%%%%%%%%%%%%%%%%%%%%%%%%%%%%%%%%
\subsection{Included Files}
\label{sec:include}

%%%%%%%%%%%%%%%%%%%%%%%%%%%%%%%%%%%%%%%%
\DescribeMacro{\childdocmain}
To use the package, add the commands
\begin{center}
\begin{tabular}{l}
|\input{childdoc.def}|\\
|\childdocmain{}|\\
\end{tabular}
\end{center}
at the very top of the main \LaTeX{} file,
in particular \emph{before} the |\documentclass| statement!
The argument of |\childdocmain| should be left empty
(but it must be present).

%%%%%%%%%%%%%%%%%%%%%%%%%%%%%%%%%%%%%%%%
\DescribeMacro{\childdocof}
Furthermore, add the commands
\begin{center}
\begin{tabular}{l}
|\input{childdoc.def}|\\
|\childdocof{|\textit{main}|}|\\
\end{tabular}
\end{center}
at the top of every child file \textit{child}
which is included by |\include{|\textit{child}|}|
from within the main file
(or at least for those files to be compiled individually).
The argument \textit{main} must be the filename of the main file.

There are a couple of
considerations in setting up the main and child documents:

%%%%%%%%%%%%%%%%%%%%%%%%%%%%%%%%%%%%%%%%
\paragraph{Restrictions.}

Please note the following restrictions:
\begin{itemize}
\item
|\childdocmain| must be called with one argument \textit{main}
to ensure compatibility with earlier version of the package.
It must either be empty (|\childdocmain{}|)
or precisely match the filename of the main file in which it is specified.
See \secref{sec:detection} for further information.
\item
The filename \textit{main} must be specified without the |.tex| extension.
\item
The filename \textit{main} is case sensitive
(even in case-insensitive file systems)
due to internal string comparison.
\item
The argument \textit{main} should be fully expanded, it cannot be a macro.
\item
Subdirectories and special characters should be avoided in filenames.
\item
The command |\childdocmain{|\textit{main}|}| must be followed by a whitespace.
It should not be followed immediately by another command
or by a comment mark `|%|'.
This is because the \TeX{} parser reads the token immediately following
the argument of |\childdocmain| and puts it
at the beginning of every child section;
however, a white\-space is ignored.
\end{itemize}

%%%%%%%%%%%%%%%%%%%%%%%%%%%%%%%%%%%%%%%%
\paragraph{Content of Main File.}

It is advisable to place all content in the child files included by |\include|.
Any output contained in the main file will appear in all child documents
unless suppressed manually;
it cannot be suppressed automatically by the |\includeonly| directive
and thus should normally be avoided.
A method to include some content in the main file
by means of conditional processing is described in \secref{sec:conditional}.

%%%%%%%%%%%%%%%%%%%%%%%%%%%%%%%%%%%%%%%%
\paragraph{Page Numbering.}

When only a part of the document is compiled,
the appropriate numbering of pages
(as well as other status parameters)
is determined from the |.aux| files.
The latter contain information from previous passes.
However this information needs to propagate through
all intermediate child documents.
Therefore the page numbering in child documents may well
be inconsistent until the complete document is compiled at least once.

A useful (if unconventional) way to always ensure a consistent
page numbering is to restart the numbering in each child document
and denote the pages by `\textit{child}|.|\textit{page}'
where \textit{child} represents the chapter/section number of the child file.
This can be achieved by the command
|\numberwithin{page}{|\textit{child}|}|
of the \textsf{amsmath} package
where \textit{child} can be |chapter| or |section|
depending on the chosen structuring.
Alternatively, one can modify the macro |\thepage| appropriately
and reset the counter |page| at the start of each child file.

%%%%%%%%%%%%%%%%%%%%%%%%%%%%%%%%%%%%%%%%%%%%%%%%%%%%%%%%%%%%%%%%%%%%%%%%%%%%%%%%
\subsection{Conditional Processing}
\label{sec:conditional}

The package provides a mechanism to compile different versions
of a document. To customise the versions further some conditional processing
can come in handy to distinguish which version is being compiled.
The package provides two macros to describe the compilation context:

%%%%%%%%%%%%%%%%%%%%%%%%%%%%%%%%%%%%%%%%
\DescribeMacro{\ifchilddoc}
The conditional |\ifchilddoc| distinguishes between the compilation of
child documents and the main document:
%
\begin{center}
|\ifchilddoc |\textit{child-code}| |[|\||else |\textit{main-code}]| \||fi|
\end{center}

%%%%%%%%%%%%%%%%%%%%%%%%%%%%%%%%%%%%%%%%
\DescribeMacro{\childdocname}
\DescribeMacro{\childdocjob}
The macro |\childdocname| contains the filename (without extension)
of the main or child file being processed.
Note that |\childdocjob| will always contain the name of the main file.

%%%%%%%%%%%%%%%%%%%%%%%%%%%%%%%%%%%%%%%%
\paragraph{Title Page.}

Conditional processing can be used to include a title or banner page
in the main document when proper precautions are taken.
Importantly, the code in the main file should ensure that the page counter
(as well as other status parameters which are stored in the |.aux| files)
takes the same value after the conditional processing.
Otherwise the page numbers may take divergent values
depending on which part is compiled.

For example, a title page could be declared by:
%
\begin{center}
\begin{tabular}{l}
|\ifchilddoc\||else|\\
|\addtocounter{page}{-1}|\\
\textit{code for title page}\\
|\newpage|\\
|\||fi|
\end{tabular}
\end{center}
%
A banner page for the child documents can be generated by:
%
\begin{center}
\begin{tabular}{l}
|\ifchilddoc|\\
|\addtocounter{page}{-1}|\\
\textit{code for banner page}\\
|\newpage|\\
|\||fi|
\end{tabular}
\end{center}
%
Here one could write a message such as:
\begin{center}
|This is the part \childdocname{} of \childdocjob{}.|
\end{center}

%%%%%%%%%%%%%%%%%%%%%%%%%%%%%%%%%%%%%%%%%%%%%%%%%%%%%%%%%%%%%%%%%%%%%%%%%%%%%%%%
\subsection{Flags}
\label{sec:flags}

The package makes it easy to generate different versions
of the main or child documents.
To this end compilation flags can be defined
and assigned different default values.
They will be particularly useful in conjunction
with the forwarding mechanism described in \secref{sec:forward}.

For example, it may be useful to have a flag |\version|
which can be set to |draft| or |final|.
The document source will contain some conditional code
depending on the value of |\version|.
Suppose further, the flag should default to |final| for the main file
and to |draft| for child files
which is a natural assignment for editing the document.
This is achieved by placing the following code
in the preamble of the main document
(below the |\childdocmain| directive):
%
\begin{center}
\begin{tabular}{l}
|\ifchilddoc|\\
|\providecommand{\version}{draft}|\\
|\||else|\\
|\providecommand{\version}{final}|\\
|\||fi|
\end{tabular}
\end{center}
%
The definition by |\providecommand| makes sure
that previous definitions are not overwritten.
Further statements |\providecommand{\version}{...}|
can thus be added before the above code to override it.

For the main file, one might add a line
(between |\childdocmain| and the above block)
%
\begin{center}
|%\ifchilddoc\||else\providecommand{\version}{draft}\||fi|
\end{center}
%
which can be uncommented to produce a draft version.
Likewise one can add a line to the very top of a child file
(above the |\childdocof{|\textit{main}|}| directive)
%
\begin{center}
|%\providecommand{\version}{final}|
\end{center}
%
which can be uncommented to produce the final version of this child document.

%%%%%%%%%%%%%%%%%%%%%%%%%%%%%%%%%%%%%%%%%%%%%%%%%%%%%%%%%%%%%%%%%%%%%%%%%%%%%%%%
\subsection{Forwarding}
\label{sec:forward}

Different versions of the main or child documents
using compilation flags as described in \secref{sec:flags}
can be (permanently) stored in different files
for convenient compilation, viewing and distribution.
To this end, the package defines a command
to pass on compilation to a different file:

%%%%%%%%%%%%%%%%%%%%%%%%%%%%%%%%%%%%%%%%
\DescribeMacro{\childdocforward}
The command |\childdocforward| redirects processing to
another source file:
%
\begin{center}
\begin{tabular}{l}
|\input{childdoc.def}|\\
|\childdocforward[|\textit{main}|]{|\textit{dest}|}|\\
\end{tabular}
\end{center}
%
The argument \textit{dest} is the destination file
(without extension).
It should be the main file or one of the child files.
Note that further \textsf{childdoc} directives
such as |\childdocof| and |\childdocforward|
in the indicated file will be processed in this form.
The optional argument \textit{main}
passes on directly to the main file \textit{main}
while pretending to compile the child \textit{dest}.
This form behaves as if \textit{dest}
issues |\childdocof{|\textit{main}|}| right away,
and no further \textsf{childdoc} directives will be processed.

%%%%%%%%%%%%%%%%%%%%%%%%%%%%%%%%%%%%%%%%
\DescribeMacro{\...prefix}
In the alternative form |\childdocforwardprefix|,
%
\begin{center}
\begin{tabular}{l}
|\input{childdoc.def}|\\
|\childdocforwardprefix[|\textit{main}|]{|\textit{prefix}|}{|\textit{dest}|}|
\end{tabular}
\end{center}
%
the destination file is determined by a pattern
depending on the current file:
To make this work, the current file must be called
`{\textit{prefix}\hspace{0.2em}\textit{suffix}}'
with \textit{prefix} matching precisely the argument.
Processing is then passed on to the file
`{\textit{dest}\hspace{0.2em}\textit{suffix}}'.
Surely, the same effect is achieved by
directly specifying the
argument `{\textit{dest}\hspace{0.2em}\textit{suffix}}'
in the first form.
However, that requires to set up a different file
for each child. With the alternative form of the command
all these files can have exactly the same content
which simplifies setting them up and maintaining them.

For example, the following file |draft.tex|
with a compilation flag |\version| as described in \secref{sec:flags}
compiles the main document as a draft:
%
\begin{center}
\begin{tabular}{l}
|\def\version{draft}|\\
|\input{childdoc.def}|\\
|\childdocforward{|\textit{main}|}|
\end{tabular}
\end{center}
%
Likewise, the following files |final|\textit{nn}|.tex|
compile the final version of the child document
|child|\textit{nn}|.tex|:
%
\begin{center}
\begin{tabular}{l}
|\def\version{final}|\\
|\input{childdoc.def}|\\
|\childdocforwardprefix{final}{child}|
\end{tabular}
\end{center}
%

Note that when several versions of a main file and/or of each child file
are to be generated, it may be convenient to set up a |Makefile| or
shell script to automatise the process.

%%%%%%%%%%%%%%%%%%%%%%%%%%%%%%%%%%%%%%%%%%%%%%%%%%%%%%%%%%%%%%%%%%%%%%%%%%%%%%%%
\subsection{Command Line Processing}
\label{sec:commandline}

The effect of redirection files can also be achieved by invoking
the \LaTeX{} compiler with a more elaborate command line.
Most conveniently this should be done as part
of a shell script or a |Makefile|.

When using \textsf{childdoc} in the main file, the following
command lines effectively perform a redirection
(note that depending on the shell being used,
backslashes may have to be doubled: `|\|' $\to$ `|\\|'):
%
\begin{center}
|... -jobname "|\textit{target}|" |\\|"|[\textit{flags}]%
|\input{childdoc.def}\childdocforward[|\textit{main}|]{|\textit{dest}|}"|
\end{center}
%
Here \textit{target} is the name of the output file,
\textit{main} is the name of the main file
and \textit{dest} is the name of the main or child file to be processed
(all filenames without extensions).
The optional argument \textit{main} can be omitted
if \textit{main} matches \textit{dest}.
Optionally, compilation \textit{flags} can be defined via |\def| commands.
This command line makes the \TeX{} engine believe
it is compiling the file \textit{target}
whose content is specified as the latter parameter.
The provided code then forwards the processing to
\textit{main} or \textit{dest} as described in \secref{sec:forward}.

%%%%%%%%%%%%%%%%%%%%%%%%%%%%%%%%%%%%%%%%%%%%%%%%%%%%%%%%%%%%%%%%%%%%%%%%%%%%%%%%
\subsection{Include by Input}
\label{sec:input}

Including child documents by |\include| has some restrictions by design.
Most notably, the content of a child document always occupies
its own set of pages; pages cannot be shared between child documents.
Usually, this behaviour makes perfect sense
because each child document contain an essential part of the document.
However, in some situations it may be desirable to compose
a document from a collection of parts
without having mandatory page breaks between then.
For this case, the package
provides a mechanism to include parts
by |\input| which can also be processed individually.
However, by construction this mechanism
requires manual handling of the content to be output.

%%%%%%%%%%%%%%%%%%%%%%%%%%%%%%%%%%%%%%%%
\DescribeMacro{\ifchilddocmanual}
The main file should be prepared as usual, see \secref{sec:include}.
However, the document body must make a distinction
between processing of an individual part and of the main document, e.g.:
%
\begin{center}
\begin{tabular}{l}
|\ifchilddocmanual|\\
|\input{\childdocname}|\\
|\||else|\\
\textit{document body with }|\input{|\textit{part}|}|\\
|\||fi|
\end{tabular}
\end{center}
%
The conditional |\ifchilddocmanual| is true whenever
a part to be included by |\input| is being compiled,
and the name of the part is stored in |\childdocname|.

%%%%%%%%%%%%%%%%%%%%%%%%%%%%%%%%%%%%%%%%
\DescribeMacro{\childdocby}
Each part to be included by |\input| should start with:
%
\begin{center}
\begin{tabular}{l}
|\input{childdoc.def}|\\
|\childdocby{|\textit{main}|}|\\
\end{tabular}
\end{center}
%
The directive |\childdocby| is similar to |\childdocof|
described in \secref{sec:include},
but the subsequent selection of content must be done manually.
To that end, both |\ifchilddoc| and |\ifchilddocmanual|
will be true upon processing of a part,
and the name of the part is stored in |\childdocname|.
Note that |\jobname| will be set to the filename of the current part
so that each part receives an individual |.aux| file
that does not interfere with the |.aux| file(s) of the main document.
This behaviour can be altered by the alternative form
|\childdocby[*]{|\textit{main}|}| (with a non-empty optional argument)
which uses the |.aux| file of the main document
by setting |\jobname| to \textit{main}.

%%%%%%%%%%%%%%%%%%%%%%%%%%%%%%%%%%%%%%%%%%%%%%%%%%%%%%%%%%%%%%%%%%%%%%%%%%%%%%%%
\subsection{Driver Development}
\label{sec:driver}

The \textsf{childdoc} mechanism can also be use for the development
of definition files such as \LaTeX{} styles or classes.
This case differs from the above setup with multiple parts
included by |\include| in that no |\includeonly| should be invoked.
This can be achieved by starting the include file
(before |\ProvidesPackage|) with:
%
\begin{center}
\begin{tabular}{l}
|\input{childdoc.def}|\\
|\childdocforward{|\textit{main}|}|\\
\end{tabular}
\end{center}
%
or alternatively with:
%
\begin{center}
\begin{tabular}{l}
|\input{childdoc.def}|\\
|\childdocby{|\textit{main}|}|\\
\end{tabular}
\end{center}
%
Both forms have slightly different effects as described above.
The main file is prepared as usual, see \secref{sec:include}.

%%%%%%%%%%%%%%%%%%%%%%%%%%%%%%%%%%%%%%%%%%%%%%%%%%%%%%%%%%%%%%%%%%%%%%%%%%%%%%%%
\subsection{Legacy Detection}
\label{sec:detection}

The directive |\childdocmain| in the main file can detect
whether the complete document or merely a child is to be compiled
even without using the directive |\childdocof|.
This method is deprecated because it is less robust
and there is no compelling reason to use it;
it is merely provided for backward compatibility
and it may be removed in future versions.

If the detection mechanism is to be used,
it is mandatory to correctly specify
the filename of the main file as the argument of |\childdocmain|:
%
\begin{center}
\begin{tabular}{l}
|\input{childdoc.def}|\\
|\childdocmain{|\textit{main}|}|\\
\end{tabular}
\end{center}
%
If |\jobname| does not match the argument \textit{main} of |\childdocmain|,
it is assumed that |\jobname| points to the child file to be compiled.
When using |\childdocmain| with the main file specified as argument,
it suffices to start a child file
with just |\input{|\textit{main}|}|
without loading of the package and using |\childdocof|.
If instead all processing is done
with the appropriate \textsf{childdoc} directives,
the argument of \textit{main} of |\childdocmain| can be empty.

An alternative version of the command line processing described
in \secref{sec:commandline} using the detection mechanism reads:
%
\begin{center}
|... -jobname "|\textit{target}|" "|[\textit{flags}]%
[|\def\jobname{|\textit{dest}|}|]|\input{|\textit{main}|}"|
\end{center}

%%%%%%%%%%%%%%%%%%%%%%%%%%%%%%%%%%%%%%%%%%%%%%%%%%%%%%%%%%%%%%%%%%%%%%%%%%%%%%%%
\subsection{Manual Code}
\label{sec:manual}

In case one cannot be certain whether the definitions file |childdoc.def|
is installed on the target \TeX{} distribution
and one prefers not to ship it,
it is conceivable to paste a few relevant commands into the sources.

To that end, drop all statements |\input{childdoc.def}|
and perform the replacements as outlined below.
Instead of |\childdocmain{|\textit{main}|}| add the following code
to the top of the main file:
%
\begin{center}
\begin{tabular}{l}
|\||ifdefined\childdocname\endinput\||fi\newif\ifchilddoc|\\
|\edef\childdocname{\scantokens\expandafter{\jobname\noexpand}}|\\
|\def\childdocmain{|\textit{main}|}\||ifx\childdocmain\childdocname\||else|\\
|\childdoctrue\includeonly{\childdocname}\let\jobname\childdocmain\||fi|\\
\end{tabular}
\end{center}
%
Instead of |\childdocof{|\textit{main}|}| just include the main file
at the top of each child file:
%
\begin{center}
|\input{|\textit{main}|}|
\end{center}
%
A simple redirection |\childdocforward{|\textit{dest}|}| is achieved by:
%
\begin{center}
|\def\jobname{|\textit{dest}|}\input{\jobname}|
\end{center}
%
The redirection with prefix
|\childdocforwardprefix[|\textit{prefix}|]{|\textit{dest}|}|
is accomplished by:
%
\begin{center}
\begin{tabular}{l}
|{\edef\jobname{\scantokens\expandafter{\jobname\noexpand}}|\\
|\def\redirectjob |\textit{prefix}|#1~~~{\gdef\jobname{|\textit{dest}|#1}}|\\
|\expandafter\redirectjob\jobname~~~}\input{\jobname}|
\end{tabular}
\end{center}

In an alternative approach,
child documents can be compiled by a specific command line
without additional code or specific definitions:
%
\begin{center}
|... -jobname "|\textit{target}|" "|[\textit{flags}]%
|\includeonly{|\textit{dest}|}\input{|\textit{main}|}"|
\end{center}
%

%%%%%%%%%%%%%%%%%%%%%%%%%%%%%%%%%%%%%%%%%%%%%%%%%%%%%%%%%%%%%%%%%%%%%%%%%%%%%%%%
%%%%%%%%%%%%%%%%%%%%%%%%%%%%%%%%%%%%%%%%%%%%%%%%%%%%%%%%%%%%%%%%%%%%%%%%%%%%%%%%
\section{Information}

%%%%%%%%%%%%%%%%%%%%%%%%%%%%%%%%%%%%%%%%%%%%%%%%%%%%%%%%%%%%%%%%%%%%%%%%%%%%%%%%
\subsection{Copyright}

Copyright \copyright{} 2017--2018 Niklas Beisert

This work may be distributed and/or modified under the
conditions of the \LaTeX{} Project Public License, either version 1.3
of this license or (at your option) any later version.
The latest version of this license is in
  \url{http://www.latex-project.org/lppl.txt}
and version 1.3 or later is part of all distributions of \LaTeX{}
version 2005/12/01 or later.

This work has the LPPL maintenance status `maintained'.

The Current Maintainer of this work is Niklas Beisert.

This work consists of the files |README.txt|, |childdoc.ins| and |childdoc.dtx|
as well as the derived files |childdoc.def|, |cdocsamp.tex|
with |cdocsch1.tex|, |cdocsch2.tex|, |cdocspt3.tex|, |cdocspt4.tex|,
|cdocsdrf.tex|, |cdocsfn1.tex|, |cdocsfn2.tex|
as well as |childdoc.pdf|.

%%%%%%%%%%%%%%%%%%%%%%%%%%%%%%%%%%%%%%%%%%%%%%%%%%%%%%%%%%%%%%%%%%%%%%%%%%%%%%%%
\subsection{Files and Installation}

The package consists of the files:
%
\begin{center}
\begin{tabular}{ll}
    |README.txt|   & readme file \\
    |childdoc.ins| & installation file \\
    |childdoc.dtx| & source file \\
    |childdoc.def| & definition file \\
    |cdocsamp.tex| & sample main file \\
    |cdocsch1.tex| & sample include file \\
    |cdocsch2.tex| & sample include file \\
    |cdocspt3.tex| & sample part file \\
    |cdocspt4.tex| & sample part file \\
    |cdocsdrf.tex| & sample redirection file \\
    |cdocsfn1.tex| & sample redirection file \\
    |cdocsfn2.tex| & sample redirection file \\
    |childdoc.pdf| & manual
\end{tabular}
\end{center}
%
The distribution consists of the files
|README.txt|, |childdoc.ins| and |childdoc.dtx|.
%
\begin{itemize}
\item
Run (pdf)\LaTeX{} on |childdoc.dtx|
to compile the manual |childdoc.pdf| (this file).
\item
Run \LaTeX{} on |childdoc.ins| to create the definitions file |childdoc.def|
and the sample |cdocsamp.tex| with include files
|cdocsch1.tex|, |cdocsch2.tex|, |cdocspt3.tex|, |cdocspt4.tex|,
|cdocsdrf.tex|, |cdocsfn1.tex|, |cdocsfn2.tex|.
Then copy the file |childdoc.def| to an appropriate directory of your \LaTeX{}
distribution, e.g.\ \textit{texmf-root}|/tex/latex/childdoc|.
\end{itemize}

%%%%%%%%%%%%%%%%%%%%%%%%%%%%%%%%%%%%%%%%%%%%%%%%%%%%%%%%%%%%%%%%%%%%%%%%%%%%%%%%
\subsection{Related CTAN Packages}

There are several other packages which offer a similar functionality:
%
\begin{itemize}
\item
The packages
\href{http://ctan.org/pkg/docmute}{\textsf{docmute}},
\href{http://ctan.org/pkg/includex}{\textsf{includex}} and
\href{http://ctan.org/pkg/standalone}{\textsf{standalone}}
provide commands to include only the document body of
a child file thus allowing both files to be compiled individually.
\item
The packages \href{http://ctan.org/pkg/subdocs}{\textsf{subdocs}}
and \href{http://ctan.org/pkg/subfiles}{\textsf{subfiles}}
provide structures in which the main and child documents can be
encapsulated and allowing them to be compiled individually.
The inclusion mechanism is different from the conventional |\include|.
\item
The package \href{http://ctan.org/pkg/combine}{\textsf{combine}}
is an elaborate solution to combine several documents into one.
\end{itemize}
%
See also the CTAN topic \href{http://ctan.org/topic/subdocs}{\textsf{subdocs}}
for further related packages.
The present package differs from the above solutions in that
a document structure constructed with the conventional |\include| mechanism
just needs two extra commands at the top of every file
such that all constituent files can be compiled individually.

%%%%%%%%%%%%%%%%%%%%%%%%%%%%%%%%%%%%%%%%%%%%%%%%%%%%%%%%%%%%%%%%%%%%%%%%%%%%%%%%
%\subsection{Feature Suggestions}
%
%The following is a list of features which may be useful for future
%versions of this package:
%%
%\begin{itemize}
%\item
%\ldots
%\end{itemize}

%%%%%%%%%%%%%%%%%%%%%%%%%%%%%%%%%%%%%%%%%%%%%%%%%%%%%%%%%%%%%%%%%%%%%%%%%%%%%%%%
\subsection{Revision History}

%%%%%%%%%%%%%%%%%%%%%%%%%%%%%%%%%%%%%%%%
\paragraph{v2.0:} 2018/12/30

\begin{itemize}
\item
immediate forward processing
\item
added |\childdocby| mechanism
\item
manual restructured
\end{itemize}

%%%%%%%%%%%%%%%%%%%%%%%%%%%%%%%%%%%%%%%%
\paragraph{v1.6:} 2018/01/17

\begin{itemize}
\item
application for development of include files
\item
corrections to manual
\end{itemize}

%%%%%%%%%%%%%%%%%%%%%%%%%%%%%%%%%%%%%%%%
\paragraph{v1.5:} 2017/05/21

\begin{itemize}
\item
more complete structuring introduced
\item
|\childdocof| introduced
\item
|\childdoc| renamed to |\childdocmain|
\item
|\childredirect| renamed to |\childdocforward| and |\childdocforwardprefix|
and functionality expanded
\end{itemize}

%%%%%%%%%%%%%%%%%%%%%%%%%%%%%%%%%%%%%%%%
\paragraph{v1.0:} 2017/04/27

\begin{itemize}
\item
manual and install package
\item
first version published on CTAN
\end{itemize}

%%%%%%%%%%%%%%%%%%%%%%%%%%%%%%%%%%%%%%%%
\paragraph{v0.6:} 2017/04/26

\begin{itemize}
\item
redirection mechanism added
\end{itemize}

%%%%%%%%%%%%%%%%%%%%%%%%%%%%%%%%%%%%%%%%
\paragraph{v0.5:} 2017/04/26

\begin{itemize}
\item
functionality in definition file
\end{itemize}


%%%%%%%%%%%%%%%%%%%%%%%%%%%%%%%%%%%%%%%%%%%%%%%%%%%%%%%%%%%%%%%%%%%%%%%%%%%%%%%%
%%%%%%%%%%%%%%%%%%%%%%%%%%%%%%%%%%%%%%%%%%%%%%%%%%%%%%%%%%%%%%%%%%%%%%%%%%%%%%%%
%%%%%%%%%%%%%%%%%%%%%%%%%%%%%%%%%%%%%%%%%%%%%%%%%%%%%%%%%%%%%%%%%%%%%%%%%%%%%%%%
\appendix

\settowidth\MacroIndent{\rmfamily\scriptsize 000\ }

 \DocInput{childdoc.dtx}

\end{document}
%</driver>
% \fi
%
% %%%%%%%%%%%%%%%%%%%%%%%%%%%%%%%%%%%%%%%%%%%%%%%%%%%%%%%%%%%%%%%%%%%%%%%%%%%%%%
% %%%%%%%%%%%%%%%%%%%%%%%%%%%%%%%%%%%%%%%%%%%%%%%%%%%%%%%%%%%%%%%%%%%%%%%%%%%%%%
% \section{Sample}
%\iffalse
%<*samplemain>
%\fi
%
% The following presents a sample document
% with two chapters, two parts, a title page,
% a compile flag as well as three forwarding files to set the flag.
% It consists of eight |.tex| files:
% \begin{center}
% \begin{tabular}{ll}
% |cdocsamp.tex|&main file\\
% |cdocsch1.tex|&include file for chapter 1\\
% |cdocsch2.tex|&include file for chapter 2\\
% |cdocspt3.tex|&include file for part 3\\
% |cdocspt4.tex|&include file for part 4\\
% |cdocsdrf.tex|&forwarding file for main file in draft mode\\
% |cdocsfi1.tex|&forwarding file for final version of chapter 1\\
% |cdocsfi2.tex|&forwarding file for final version of chapter 2\\
% \end{tabular}
% \end{center}
% Each of the eight files can be compiled directly by the \LaTeX{} compiler.
%
% %%%%%%%%%%%%%%%%%%%%%%%%%%%%%%%%%%%%%%
% \paragraph{Main File.}
%
% The main file is called |cdocsamp.tex|.
%
% Load the \textsf{childdoc} definitions and
% declare the filename for the main document:
%    \begin{macrocode}
\input{childdoc.def}
\childdocmain{}
%    \end{macrocode}

% Optional override for |\version| flag:
%    \begin{macrocode}
%%\ifchilddoc\else\providecommand{\version}{draft}\fi
%    \end{macrocode}

% Define the default values for the |\version| flag
% (|final| for the main file and |draft| for childs):
%    \begin{macrocode}
\ifchilddoc
\providecommand{\version}{draft}
\else
\providecommand{\version}{final}
\fi
%    \end{macrocode}

% Load the standard document class:
%    \begin{macrocode}
\documentclass[12pt]{article}
%    \end{macrocode}

% Start the document body:
%    \begin{macrocode}
\begin{document}
%    \end{macrocode}

% Declare a title page.
% Print title, part of document being processed and version flag:
%    \begin{macrocode}
\addtocounter{page}{-1}
\begin{center}
{\LARGE\bfseries{}childdoc example\par}
\vspace{1cm}
\ifchilddoc
\ifchilddocmanual part\else chapter\fi:
`\childdocname' of `\childdocjob'\par
\else
main document: `\childdocjob'\par
\fi
version: \version\par
\end{center}
\newpage
%    \end{macrocode}

% Manually include selected file,
% otherwise process as usual:
%    \begin{macrocode}
\ifchilddocmanual
\section*{part `\childdocname'}
\input{\childdocname}
\else
%    \end{macrocode}

% Include the two chapters:
%    \begin{macrocode}
\include{cdocsch1}
\include{cdocsch2}
%    \end{macrocode}

% Include the two parts unless only chapters should be displayed:
%    \begin{macrocode}
\ifchilddoc\else
\section{part three}
\input{cdocspt3}
\section{part four}
\input{cdocspt4}
\fi
%    \end{macrocode}

% Process as usual until here:
%    \begin{macrocode}
\fi
%    \end{macrocode}

% End of document body:
%    \begin{macrocode}
\end{document}
%    \end{macrocode}
%\iffalse
%</samplemain>
%\fi
%
% %%%%%%%%%%%%%%%%%%%%%%%%%%%%%%%%%%%%%%
% \paragraph{Chapter Include Files.}
%
% The include files are called |cdocsch1.tex| and |cdocsch2.tex|.
%
%\iffalse
%<*samplechap1|samplechap2>
%\fi

% Optional override for |\version| flag:
%    \begin{macrocode}
%%\providecommand{\version}{final}
%    \end{macrocode}

% Include the main document:
%    \begin{macrocode}
\input{childdoc.def}
\childdocof{cdocsamp}
%    \end{macrocode}

%\iffalse
%</samplechap1|samplechap2>
%\fi
%
%\iffalse
%<*samplechap1>
%\fi
% Some text for chapter 1:
%    \begin{macrocode}
\section{one}
some text in chapter one
%    \end{macrocode}

%\iffalse
%</samplechap1>
%\fi
% Some text for chapter 2:
%\iffalse
%<*samplechap2>
%\fi
%    \begin{macrocode}
\section{two}
more text in chapter two
%    \end{macrocode}

%\iffalse
%</samplechap2>
%\fi
%
% %%%%%%%%%%%%%%%%%%%%%%%%%%%%%%%%%%%%%%
% \paragraph{Part Include Files.}
%
% The include files are called |cdocspt3.tex| and |cdocspt4.tex|.
%
%\iffalse
%<*samplepart3|samplepart4>
%\fi

% Optional override for |\version| flag:
%    \begin{macrocode}
%%\providecommand{\version}{final}
%    \end{macrocode}

% Include the main document:
%    \begin{macrocode}
\input{childdoc.def}
\childdocby{cdocsamp}
%    \end{macrocode}

%\iffalse
%</samplepart3|samplepart4>
%\fi
%
%\iffalse
%<*samplepart3>
%\fi
% Some text for part 3:
%    \begin{macrocode}
some text in part three
%    \end{macrocode}

%\iffalse
%</samplepart3>
%\fi
% Some text for part 4:
%\iffalse
%<*samplepart4>
%\fi
%    \begin{macrocode}
more text in part four
%    \end{macrocode}

%\iffalse
%</samplepart4>
%\fi
%
% %%%%%%%%%%%%%%%%%%%%%%%%%%%%%%%%%%%%%%
% \paragraph{Forwarding for a Complete Draft.}
%
% The following forwarding file |cdocsdrf.tex|
% compiles the main document in draft mode:
%\iffalse
%<*sampledraft>
%\fi
%    \begin{macrocode}
\def\version{draft}
\input{childdoc.def}
\childdocforward{cdocsamp}
%    \end{macrocode}

%\iffalse
%</sampledraft>
%\fi
%
% %%%%%%%%%%%%%%%%%%%%%%%%%%%%%%%%%%%%%%
% \paragraph{Forwarding for Final Version of the Chapters.}
%
% The following forwarding files |cdocsfn1.tex| and |cdocsfn2.tex|
% (with identical content)
% compile the final versions of the child documents
% |cdocsch1.tex| and |cdocsch2.tex|, respectively:
%\iffalse
%<*samplefinal>
%\fi
%    \begin{macrocode}
\def\version{final}
\input{childdoc.def}
\childdocforwardprefix[cdocsamp]{cdocsfn}{cdocsch}
%    \end{macrocode}

%\iffalse
%</samplefinal>
%\fi
%
% %%%%%%%%%%%%%%%%%%%%%%%%%%%%%%%%%%%%%%
% \paragraph{Command Line Processing.}
%
% The following three command lines generate the output files
% |cdocscld|, |cdocscl1| and |cdocscl2|
% which should be identical to
% |cdocsdrf|, |cdocsch1| and |cdocsfn2|, respectively:
% \begin{center}
% \begin{tabular}{l}
% |latex -jobname cdocscld \|\\
% |  "\def\version{draft}\input{childdoc.def}\childdocforward{cdocsamp}"|\\
% |latex -jobname cdocscl1 \|\\
% |  "\input{childdoc.def}\childdocforward[cdocsamp]{cdocsch1}"|\\
% |latex -jobname cdocscl2 \|\\
% |  "\def\version{final}\input{childdoc.def}\childdocforward{cdocsch2}"|
% \end{tabular}
% \end{center}
% Note that the trailing backslash on each first line
% merely continues the input to the second line
% (for convenient cut ant paste).
% Furthermore, the command |latex| can be replaced by any
% of its alternative versions such as |pdflatex|.
%
% %%%%%%%%%%%%%%%%%%%%%%%%%%%%%%%%%%%%%%%%%%%%%%%%%%%%%%%%%%%%%%%%%%%%%%%%%%%%%%
% %%%%%%%%%%%%%%%%%%%%%%%%%%%%%%%%%%%%%%%%%%%%%%%%%%%%%%%%%%%%%%%%%%%%%%%%%%%%%%
% \section{Implementation}
%\iffalse
%<*package>
%\fi
%
% This section describes the definitions file |childdoc.def|.

% The definitions cannot be loaded using |\usepackage| or |\RequirePackage|
% which has a mechanism to prevent loading a style file more than once.
% When loading the definitions by means of |\input|
% multiple instances have to be prevented manually:
%\iffalse
%This code needs to be before the `\ProvidesFile' directive
%which is defined at the beginning of this file.
%Therefore it is also placed there and commented out here.
%</package>
%<*discard>
%\fi
%    \begin{macrocode}
\ifdefined\childdocmain\endinput\fi
%    \end{macrocode}
%\iffalse
%</discard>
%<*package>
%\fi
%
% \macro{\ifchilddoc}
% \macro{\ifchilddocmanual}
% The conditional |\ifchilddoc| tells whether a
% child (true) or main (false) document is being compiled.
% The conditional |\ifchilddocmanual| tells whether
% the |\includeonly| mechanism is used (false) or
% the selection of child files must be performed manually (true).
% The definitions initialise to false:
%    \begin{macrocode}
\newif\ifchilddoc
\newif\ifchilddocmanual
%    \end{macrocode}

% \macro{\childdocname}
% \macro{\childdocjob}
% The macro |\childdocname| stores the name of the main document
% to be compiled. The macro |\childdocjob| stores the name of
% the document on which the \LaTeX{} compiler was originally invoked.
% The content of |\jobname| cannot be compared
% to filenames specified in the source due to different catcodes.
% The following code rescans |\jobname|, stores the result
% in |\childdocname| and saves a copy in |\childdocjob|:
%    \begin{macrocode}
\edef\childdocname{\scantokens\expandafter{\jobname\noexpand}}
\let\childdocjob\childdocname
%    \end{macrocode}

% \macro{\childdocdisable}
% The macro |\childdocdisable| prevents the main file
% from being processed more than once.
% At this stage, the main document command |\childdocmain|
% is assumed to be called once again where it should do nothing.
% Any subsequent call to it should prevent
% a secondary processing of the main document
% It overwrites the forwarding commands
% |\childdocof| and |\childdocforward|
% with empty macros to prevent further inclusions of the main document:
%    \begin{macrocode}
\newcommand{\childdocdisable}
{
  \renewcommand{\childdocmain}[1]{\renewcommand{\childdocmain}[1]{\endinput}}
  \renewcommand{\childdocof}[1]{}
  \renewcommand{\childdocby}[2][]{}
  \renewcommand{\childdocforward}[2][]{}
  \renewcommand{\childdocdisable}{}
}
%    \end{macrocode}

% \macro{\childdocmain}
% The macro |\childdocmain| is to be called at the top of the main file
% with nothing or the main filename (without extension) as argument.
% First, it breaks loops.
% If the argument is not empty and does not match |\childdocname|
% (which is set by the first inclusion of |childdoc.def|),
% |\ifchilddoc| is set to true, |\includeonly| is applied to the child file
% and |\jobname| is set to the main file
% (for proper handling of |.aux| files):
%    \begin{macrocode}
\newcommand{\childdocmain}[1]
{
  \childdocdisable\childdocmain{}
  \if?#1?\else
    \begingroup
      \def\childdoctmp{#1}
      \ifx\childdoctmp\childdocname
        \def\childdoctmp{}
      \else
        \def\childdoctmp
        {
          \childdoctrue
          \includeonly{\childdocname}
          \def\childdocjob{#1}
          \def\jobname{#1}
        }
      \fi
      \expandafter
    \endgroup
    \childdoctmp
  \fi
}
%    \end{macrocode}

% \macro{\childdocof}
% The command |\childdocof| redirects
% compilation to the main file |#1|.
%    \begin{macrocode}
\newcommand{\childdocof}[1]
{
  \childdocdisable
  \childdoctrue
  \includeonly{\childdocname}
  \def\jobname{#1}
  \def\childdocjob{#1}
  \input{#1}
}
%    \end{macrocode}

% \macro{\childdocby}
% The command |\childdocby| ....
%    \begin{macrocode}
\newcommand{\childdocby}[2][]
{
  \childdocdisable
  \childdoctrue
  \childdocmanualtrue
  \if?#1?\else
    \def\jobname{#2}
  \fi
  \def\childdocjob{#2}
  \input{#2}
  \endinput
}
%    \end{macrocode}

% \macro{\childdocforward}
% The command |\childdocforward| redirects
% compilation to the main file or
% (if the optional argument is given) a child file.
% Parameters are set as if the main file
% or a child file starting with |\childdocof| was compiled.
% Then compilation is handed over to the main file:
%    \begin{macrocode}
\newcommand{\childdocforward}[2][]
{
  \begingroup
    \if?#1?
      \def\childdoctmp
      {
        \def\childdocname{#2}
        \def\childdocjob{#2}
        \def\jobname{#2}
        \input{#2}
        \endinput
      }
    \else
      \def\childdoctmp
      {
        \childdocdisable
        \def\childdocname{#2}
        \childdoctrue
        \includeonly{#2}
        \def\childdocjob{#1}
        \def\jobname{#1}
        \input{#1}
        \endinput
      }
    \fi
    \expandafter
  \endgroup
  \childdoctmp
}
%    \end{macrocode}

% \macro{\childdocforwardprefix}
% The command |\childdocforwardprefix| redirects
% compilation to the main or a child file by means of a pattern.
% The prefix |#1| in the current filename is replaced by |#2|
% and the suffix of the current filename is kept
% (it is assumed that the filename does not contain the substring `|~~~|'
% which is used as a delimiter).
% Compilation is handed over to the new file by |\childdocforward|:
%    \begin{macrocode}
\newcommand{\childdocforwardprefix}[3][]
{
  \begingroup
    \def\childdocextract #2##1~~~{\def\childdoctmp{\childdocforward[#1]{#3##1}}}
    \expandafter\childdocextract\childdocname~~~
    \expandafter
  \endgroup
  \childdoctmp
}
%    \end{macrocode}

% \macro{\childdoc}
% The deprecated macro |\childdoc| is a legacy version of |\childdocmain|:
%    \begin{macrocode}
\newcommand{\childdoc}{\childdocmain}
%    \end{macrocode}

% \macro{\childdocredirect}
% The deprecated macro |\childdocredirect| is a legacy version
% of |\childdocforward| and |\childdocforwardprefix|:
%    \begin{macrocode}
\newcommand{\childdocredirect}[2][]
{
  \begingroup
    \if?#1?
      \def\childdoctmp{\childdocforward{#2}}
    \else
      \def\childdoctmp{\childdocforwardprefix{#1}{#2}}
    \fi
    \expandafter
  \endgroup
  \childdoctmp
}
%    \end{macrocode}

%\iffalse
%</package>
%\fi
%
\endinput
\childdocforward{cdocsch2}"|
% \end{tabular}
% \end{center}
% Note that the trailing backslash on each first line
% merely continues the input to the second line
% (for convenient cut ant paste).
% Furthermore, the command |latex| can be replaced by any
% of its alternative versions such as |pdflatex|.
%
% %%%%%%%%%%%%%%%%%%%%%%%%%%%%%%%%%%%%%%%%%%%%%%%%%%%%%%%%%%%%%%%%%%%%%%%%%%%%%%
% %%%%%%%%%%%%%%%%%%%%%%%%%%%%%%%%%%%%%%%%%%%%%%%%%%%%%%%%%%%%%%%%%%%%%%%%%%%%%%
% \section{Implementation}
%\iffalse
%<*package>
%\fi
%
% This section describes the definitions file |childdoc.def|.

% The definitions cannot be loaded using |\usepackage| or |\RequirePackage|
% which has a mechanism to prevent loading a style file more than once.
% When loading the definitions by means of |\input|
% multiple instances have to be prevented manually:
%\iffalse
%This code needs to be before the `\ProvidesFile' directive
%which is defined at the beginning of this file.
%Therefore it is also placed there and commented out here.
%</package>
%<*discard>
%\fi
%    \begin{macrocode}
\ifdefined\childdocmain\endinput\fi
%    \end{macrocode}
%\iffalse
%</discard>
%<*package>
%\fi
%
% \macro{\ifchilddoc}
% \macro{\ifchilddocmanual}
% The conditional |\ifchilddoc| tells whether a
% child (true) or main (false) document is being compiled.
% The conditional |\ifchilddocmanual| tells whether
% the |\includeonly| mechanism is used (false) or
% the selection of child files must be performed manually (true).
% The definitions initialise to false:
%    \begin{macrocode}
\newif\ifchilddoc
\newif\ifchilddocmanual
%    \end{macrocode}

% \macro{\childdocname}
% \macro{\childdocjob}
% The macro |\childdocname| stores the name of the main document
% to be compiled. The macro |\childdocjob| stores the name of
% the document on which the \LaTeX{} compiler was originally invoked.
% The content of |\jobname| cannot be compared
% to filenames specified in the source due to different catcodes.
% The following code rescans |\jobname|, stores the result
% in |\childdocname| and saves a copy in |\childdocjob|:
%    \begin{macrocode}
\edef\childdocname{\scantokens\expandafter{\jobname\noexpand}}
\let\childdocjob\childdocname
%    \end{macrocode}

% \macro{\childdocdisable}
% The macro |\childdocdisable| prevents the main file
% from being processed more than once.
% At this stage, the main document command |\childdocmain|
% is assumed to be called once again where it should do nothing.
% Any subsequent call to it should prevent
% a secondary processing of the main document
% It overwrites the forwarding commands
% |\childdocof| and |\childdocforward|
% with empty macros to prevent further inclusions of the main document:
%    \begin{macrocode}
\newcommand{\childdocdisable}
{
  \renewcommand{\childdocmain}[1]{\renewcommand{\childdocmain}[1]{\endinput}}
  \renewcommand{\childdocof}[1]{}
  \renewcommand{\childdocby}[2][]{}
  \renewcommand{\childdocforward}[2][]{}
  \renewcommand{\childdocdisable}{}
}
%    \end{macrocode}

% \macro{\childdocmain}
% The macro |\childdocmain| is to be called at the top of the main file
% with nothing or the main filename (without extension) as argument.
% First, it breaks loops.
% If the argument is not empty and does not match |\childdocname|
% (which is set by the first inclusion of |childdoc.def|),
% |\ifchilddoc| is set to true, |\includeonly| is applied to the child file
% and |\jobname| is set to the main file
% (for proper handling of |.aux| files):
%    \begin{macrocode}
\newcommand{\childdocmain}[1]
{
  \childdocdisable\childdocmain{}
  \if?#1?\else
    \begingroup
      \def\childdoctmp{#1}
      \ifx\childdoctmp\childdocname
        \def\childdoctmp{}
      \else
        \def\childdoctmp
        {
          \childdoctrue
          \includeonly{\childdocname}
          \def\childdocjob{#1}
          \def\jobname{#1}
        }
      \fi
      \expandafter
    \endgroup
    \childdoctmp
  \fi
}
%    \end{macrocode}

% \macro{\childdocof}
% The command |\childdocof| redirects
% compilation to the main file |#1|.
%    \begin{macrocode}
\newcommand{\childdocof}[1]
{
  \childdocdisable
  \childdoctrue
  \includeonly{\childdocname}
  \def\jobname{#1}
  \def\childdocjob{#1}
  \input{#1}
}
%    \end{macrocode}

% \macro{\childdocby}
% The command |\childdocby| ....
%    \begin{macrocode}
\newcommand{\childdocby}[2][]
{
  \childdocdisable
  \childdoctrue
  \childdocmanualtrue
  \if?#1?\else
    \def\jobname{#2}
  \fi
  \def\childdocjob{#2}
  \input{#2}
  \endinput
}
%    \end{macrocode}

% \macro{\childdocforward}
% The command |\childdocforward| redirects
% compilation to the main file or
% (if the optional argument is given) a child file.
% Parameters are set as if the main file
% or a child file starting with |\childdocof| was compiled.
% Then compilation is handed over to the main file:
%    \begin{macrocode}
\newcommand{\childdocforward}[2][]
{
  \begingroup
    \if?#1?
      \def\childdoctmp
      {
        \def\childdocname{#2}
        \def\childdocjob{#2}
        \def\jobname{#2}
        \input{#2}
        \endinput
      }
    \else
      \def\childdoctmp
      {
        \childdocdisable
        \def\childdocname{#2}
        \childdoctrue
        \includeonly{#2}
        \def\childdocjob{#1}
        \def\jobname{#1}
        \input{#1}
        \endinput
      }
    \fi
    \expandafter
  \endgroup
  \childdoctmp
}
%    \end{macrocode}

% \macro{\childdocforwardprefix}
% The command |\childdocforwardprefix| redirects
% compilation to the main or a child file by means of a pattern.
% The prefix |#1| in the current filename is replaced by |#2|
% and the suffix of the current filename is kept
% (it is assumed that the filename does not contain the substring `|~~~|'
% which is used as a delimiter).
% Compilation is handed over to the new file by |\childdocforward|:
%    \begin{macrocode}
\newcommand{\childdocforwardprefix}[3][]
{
  \begingroup
    \def\childdocextract #2##1~~~{\def\childdoctmp{\childdocforward[#1]{#3##1}}}
    \expandafter\childdocextract\childdocname~~~
    \expandafter
  \endgroup
  \childdoctmp
}
%    \end{macrocode}

% \macro{\childdoc}
% The deprecated macro |\childdoc| is a legacy version of |\childdocmain|:
%    \begin{macrocode}
\newcommand{\childdoc}{\childdocmain}
%    \end{macrocode}

% \macro{\childdocredirect}
% The deprecated macro |\childdocredirect| is a legacy version
% of |\childdocforward| and |\childdocforwardprefix|:
%    \begin{macrocode}
\newcommand{\childdocredirect}[2][]
{
  \begingroup
    \if?#1?
      \def\childdoctmp{\childdocforward{#2}}
    \else
      \def\childdoctmp{\childdocforwardprefix{#1}{#2}}
    \fi
    \expandafter
  \endgroup
  \childdoctmp
}
%    \end{macrocode}

%\iffalse
%</package>
%\fi
%
\endinput
|\\
|\childdocforward{|\textit{main}|}|\\
\end{tabular}
\end{center}
%
or alternatively with:
%
\begin{center}
\begin{tabular}{l}
|% \iffalse
%
% childdoc.dtx Copyright (C) 2017-2018 Niklas Beisert
%
% This work may be distributed and/or modified under the
% conditions of the LaTeX Project Public License, either version 1.3
% of this license or (at your option) any later version.
% The latest version of this license is in
%   http://www.latex-project.org/lppl.txt
% and version 1.3 or later is part of all distributions of LaTeX
% version 2005/12/01 or later.
%
% This work has the LPPL maintenance status `maintained'.
%
% The Current Maintainer of this work is Niklas Beisert.
%
% This work consists of the files childdoc.dtx and childdoc.ins
% and the derived files childdoc.def and cdocsamp.tex with
% cdocsch1.tex, cdocsch2.tex, cdocsdrf.tex, cdocsfn1.tex, cdocsfn2.tex.
%
%<package>\ifdefined\childdocmain\endinput\fi
%<package>\ProvidesFile{childdoc.def}[2018/12/30 v2.0 child document driver]
%<samplemain>\ProvidesFile{cdocsamp.tex}[2018/12/30 v2.0 sample for childdoc]
%<*driver>
%\ProvidesFile{childdoc.drv}[2018/12/30 v2.0 childdoc reference manual file]
\PassOptionsToClass{10pt,a4paper}{article}
\documentclass{ltxdoc}

\usepackage[margin=35mm]{geometry}
\usepackage{hyperref}
\usepackage{hyperxmp}
\usepackage[usenames]{color}

\hypersetup{colorlinks=true}
\hypersetup{pdfstartview=FitH}
\hypersetup{pdfpagemode=UseNone}
\hypersetup{pdfsource={}}
\hypersetup{pdflang={en-UK}}
\hypersetup{pdfcopyright={Copyright 2017-2018 Niklas Beisert.
  This work may be distributed and/or modified under the
  conditions of the LaTeX Project Public License, either version 1.3
  of this license or (at your option) any later version.}}
\hypersetup{pdflicenseurl={http://www.latex-project.org/lppl.txt}}
\hypersetup{pdfcontactaddress={ETH Zurich, ITP, HIT K,
  Wolfgang-Pauli-Strasse 27}}
\hypersetup{pdfcontactpostcode={8093}}
\hypersetup{pdfcontactcity={Zurich}}
\hypersetup{pdfcontactcountry={Switzerland}}
\hypersetup{pdfcontactemail={nbeisert@itp.phys.ethz.ch}}
\hypersetup{pdfcontacturl={http://people.phys.ethz.ch/\xmptilde nbeisert/}}

\newcommand{\secref}[1]{\hyperref[#1]{section \ref*{#1}}}

\parskip1ex
\parindent0pt
\let\olditemize\itemize
\def\itemize{\olditemize\parskip0pt}

\begin{document}

\title{The \textsf{childdoc} Package}
\hypersetup{pdftitle={The childdoc Package}}
\author{Niklas Beisert\\[2ex]
  Institut f\"ur Theoretische Physik\\
  Eidgen\"ossische Technische Hochschule Z\"urich\\
  Wolfgang-Pauli-Strasse 27, 8093 Z\"urich, Switzerland\\[1ex]
  \href{mailto:nbeisert@itp.phys.ethz.ch}
  {\texttt{nbeisert@itp.phys.ethz.ch}}}
\hypersetup{pdfauthor={Niklas Beisert}}
\hypersetup{pdfsubject={Manual for the LaTeX2e Package childdoc}}
\date{30 December 2018, \textsf{v2.0}}
\maketitle

\begin{abstract}\noindent
\textsf{childdoc} is a \LaTeXe{} package
that enables the direct compilation
of document sections included by |\include|
to individual files.
\end{abstract}

\begingroup
\parskip0ex
\tableofcontents
\endgroup

%%%%%%%%%%%%%%%%%%%%%%%%%%%%%%%%%%%%%%%%%%%%%%%%%%%%%%%%%%%%%%%%%%%%%%%%%%%%%%%%
%%%%%%%%%%%%%%%%%%%%%%%%%%%%%%%%%%%%%%%%%%%%%%%%%%%%%%%%%%%%%%%%%%%%%%%%%%%%%%%%
\section{Introduction}

\LaTeX{} provides a mechanism to structure a large document (such as a book)
into a main file and several child files (containing the chapters)
using the |\include| command.
This mechanism is beneficial for documents
which span hundreds of pages in order to
make the source file(s) more manageable.
Moreover, compilation can be restricted to
selected child files by means of the |\includeonly| command.
The latter feature can be used to reduce the compilation time while editing
(this was significantly more useful in the earlier days of \LaTeX{})
or to generate a smaller document which is easier to navigate.
Another application of |\includeonly| is to generate
documents consisting of selected parts of the complete document.

However, there are a few drawbacks of the plain |\include| mechanism:
\begin{itemize}
\item
The child files cannot be compiled on their own,
they can only be compiled via the main file.
A naive editing environment
(such as a text editor with an option
to have the current file processed by \LaTeX)
may require one to switch to the main file before compiling;
attempting to compile the child file produces errors.
\item
The main file must be modified (each time)
to adjust the |\includeonly| command
to the present needs. This easily leaves the main file in a messy state.
\item
The generated document will always carry the filename
of the main document. This is inconvenient if
several child files are to be compiled and
to be kept for distribution.
\end{itemize}

The present package provides a simple interface
to make child files individually compilable by \LaTeX{}.
Compiling a child file then has the same effect as compiling
the main file with an |\includeonly| command
to select the appropriate child.
Moreover the generated document will carry the name of the child
rather than the main file.
This resolves all three above issues.

This feature is meant to make the editing of books,
thesis documents and lecture notes somewhat more convenient.
However, the package can also be used efficiently for
composing a series of documents (such as exercise sheets)
which are typically distributed individually.
It then assists the author in generating the individual documents
(potentially in different versions)
as well as a document containing the collected series.
Another application is in developing style files
or other kinds of included material
where compilation of the style file could redirect
to a sample or test file.

%%%%%%%%%%%%%%%%%%%%%%%%%%%%%%%%%%%%%%%%%%%%%%%%%%%%%%%%%%%%%%%%%%%%%%%%%%%%%%%%
%%%%%%%%%%%%%%%%%%%%%%%%%%%%%%%%%%%%%%%%%%%%%%%%%%%%%%%%%%%%%%%%%%%%%%%%%%%%%%%%
\section{Usage}

First of all, the package \textsf{childdoc} is \emph{not} a standard
\LaTeXe{} |.sty| style file! Therefore it needs to be invoked in
a non-standard way.

%%%%%%%%%%%%%%%%%%%%%%%%%%%%%%%%%%%%%%%%%%%%%%%%%%%%%%%%%%%%%%%%%%%%%%%%%%%%%%%%
\subsection{Included Files}
\label{sec:include}

%%%%%%%%%%%%%%%%%%%%%%%%%%%%%%%%%%%%%%%%
\DescribeMacro{\childdocmain}
To use the package, add the commands
\begin{center}
\begin{tabular}{l}
|% \iffalse
%
% childdoc.dtx Copyright (C) 2017-2018 Niklas Beisert
%
% This work may be distributed and/or modified under the
% conditions of the LaTeX Project Public License, either version 1.3
% of this license or (at your option) any later version.
% The latest version of this license is in
%   http://www.latex-project.org/lppl.txt
% and version 1.3 or later is part of all distributions of LaTeX
% version 2005/12/01 or later.
%
% This work has the LPPL maintenance status `maintained'.
%
% The Current Maintainer of this work is Niklas Beisert.
%
% This work consists of the files childdoc.dtx and childdoc.ins
% and the derived files childdoc.def and cdocsamp.tex with
% cdocsch1.tex, cdocsch2.tex, cdocsdrf.tex, cdocsfn1.tex, cdocsfn2.tex.
%
%<package>\ifdefined\childdocmain\endinput\fi
%<package>\ProvidesFile{childdoc.def}[2018/12/30 v2.0 child document driver]
%<samplemain>\ProvidesFile{cdocsamp.tex}[2018/12/30 v2.0 sample for childdoc]
%<*driver>
%\ProvidesFile{childdoc.drv}[2018/12/30 v2.0 childdoc reference manual file]
\PassOptionsToClass{10pt,a4paper}{article}
\documentclass{ltxdoc}

\usepackage[margin=35mm]{geometry}
\usepackage{hyperref}
\usepackage{hyperxmp}
\usepackage[usenames]{color}

\hypersetup{colorlinks=true}
\hypersetup{pdfstartview=FitH}
\hypersetup{pdfpagemode=UseNone}
\hypersetup{pdfsource={}}
\hypersetup{pdflang={en-UK}}
\hypersetup{pdfcopyright={Copyright 2017-2018 Niklas Beisert.
  This work may be distributed and/or modified under the
  conditions of the LaTeX Project Public License, either version 1.3
  of this license or (at your option) any later version.}}
\hypersetup{pdflicenseurl={http://www.latex-project.org/lppl.txt}}
\hypersetup{pdfcontactaddress={ETH Zurich, ITP, HIT K,
  Wolfgang-Pauli-Strasse 27}}
\hypersetup{pdfcontactpostcode={8093}}
\hypersetup{pdfcontactcity={Zurich}}
\hypersetup{pdfcontactcountry={Switzerland}}
\hypersetup{pdfcontactemail={nbeisert@itp.phys.ethz.ch}}
\hypersetup{pdfcontacturl={http://people.phys.ethz.ch/\xmptilde nbeisert/}}

\newcommand{\secref}[1]{\hyperref[#1]{section \ref*{#1}}}

\parskip1ex
\parindent0pt
\let\olditemize\itemize
\def\itemize{\olditemize\parskip0pt}

\begin{document}

\title{The \textsf{childdoc} Package}
\hypersetup{pdftitle={The childdoc Package}}
\author{Niklas Beisert\\[2ex]
  Institut f\"ur Theoretische Physik\\
  Eidgen\"ossische Technische Hochschule Z\"urich\\
  Wolfgang-Pauli-Strasse 27, 8093 Z\"urich, Switzerland\\[1ex]
  \href{mailto:nbeisert@itp.phys.ethz.ch}
  {\texttt{nbeisert@itp.phys.ethz.ch}}}
\hypersetup{pdfauthor={Niklas Beisert}}
\hypersetup{pdfsubject={Manual for the LaTeX2e Package childdoc}}
\date{30 December 2018, \textsf{v2.0}}
\maketitle

\begin{abstract}\noindent
\textsf{childdoc} is a \LaTeXe{} package
that enables the direct compilation
of document sections included by |\include|
to individual files.
\end{abstract}

\begingroup
\parskip0ex
\tableofcontents
\endgroup

%%%%%%%%%%%%%%%%%%%%%%%%%%%%%%%%%%%%%%%%%%%%%%%%%%%%%%%%%%%%%%%%%%%%%%%%%%%%%%%%
%%%%%%%%%%%%%%%%%%%%%%%%%%%%%%%%%%%%%%%%%%%%%%%%%%%%%%%%%%%%%%%%%%%%%%%%%%%%%%%%
\section{Introduction}

\LaTeX{} provides a mechanism to structure a large document (such as a book)
into a main file and several child files (containing the chapters)
using the |\include| command.
This mechanism is beneficial for documents
which span hundreds of pages in order to
make the source file(s) more manageable.
Moreover, compilation can be restricted to
selected child files by means of the |\includeonly| command.
The latter feature can be used to reduce the compilation time while editing
(this was significantly more useful in the earlier days of \LaTeX{})
or to generate a smaller document which is easier to navigate.
Another application of |\includeonly| is to generate
documents consisting of selected parts of the complete document.

However, there are a few drawbacks of the plain |\include| mechanism:
\begin{itemize}
\item
The child files cannot be compiled on their own,
they can only be compiled via the main file.
A naive editing environment
(such as a text editor with an option
to have the current file processed by \LaTeX)
may require one to switch to the main file before compiling;
attempting to compile the child file produces errors.
\item
The main file must be modified (each time)
to adjust the |\includeonly| command
to the present needs. This easily leaves the main file in a messy state.
\item
The generated document will always carry the filename
of the main document. This is inconvenient if
several child files are to be compiled and
to be kept for distribution.
\end{itemize}

The present package provides a simple interface
to make child files individually compilable by \LaTeX{}.
Compiling a child file then has the same effect as compiling
the main file with an |\includeonly| command
to select the appropriate child.
Moreover the generated document will carry the name of the child
rather than the main file.
This resolves all three above issues.

This feature is meant to make the editing of books,
thesis documents and lecture notes somewhat more convenient.
However, the package can also be used efficiently for
composing a series of documents (such as exercise sheets)
which are typically distributed individually.
It then assists the author in generating the individual documents
(potentially in different versions)
as well as a document containing the collected series.
Another application is in developing style files
or other kinds of included material
where compilation of the style file could redirect
to a sample or test file.

%%%%%%%%%%%%%%%%%%%%%%%%%%%%%%%%%%%%%%%%%%%%%%%%%%%%%%%%%%%%%%%%%%%%%%%%%%%%%%%%
%%%%%%%%%%%%%%%%%%%%%%%%%%%%%%%%%%%%%%%%%%%%%%%%%%%%%%%%%%%%%%%%%%%%%%%%%%%%%%%%
\section{Usage}

First of all, the package \textsf{childdoc} is \emph{not} a standard
\LaTeXe{} |.sty| style file! Therefore it needs to be invoked in
a non-standard way.

%%%%%%%%%%%%%%%%%%%%%%%%%%%%%%%%%%%%%%%%%%%%%%%%%%%%%%%%%%%%%%%%%%%%%%%%%%%%%%%%
\subsection{Included Files}
\label{sec:include}

%%%%%%%%%%%%%%%%%%%%%%%%%%%%%%%%%%%%%%%%
\DescribeMacro{\childdocmain}
To use the package, add the commands
\begin{center}
\begin{tabular}{l}
|\input{childdoc.def}|\\
|\childdocmain{}|\\
\end{tabular}
\end{center}
at the very top of the main \LaTeX{} file,
in particular \emph{before} the |\documentclass| statement!
The argument of |\childdocmain| should be left empty
(but it must be present).

%%%%%%%%%%%%%%%%%%%%%%%%%%%%%%%%%%%%%%%%
\DescribeMacro{\childdocof}
Furthermore, add the commands
\begin{center}
\begin{tabular}{l}
|\input{childdoc.def}|\\
|\childdocof{|\textit{main}|}|\\
\end{tabular}
\end{center}
at the top of every child file \textit{child}
which is included by |\include{|\textit{child}|}|
from within the main file
(or at least for those files to be compiled individually).
The argument \textit{main} must be the filename of the main file.

There are a couple of
considerations in setting up the main and child documents:

%%%%%%%%%%%%%%%%%%%%%%%%%%%%%%%%%%%%%%%%
\paragraph{Restrictions.}

Please note the following restrictions:
\begin{itemize}
\item
|\childdocmain| must be called with one argument \textit{main}
to ensure compatibility with earlier version of the package.
It must either be empty (|\childdocmain{}|)
or precisely match the filename of the main file in which it is specified.
See \secref{sec:detection} for further information.
\item
The filename \textit{main} must be specified without the |.tex| extension.
\item
The filename \textit{main} is case sensitive
(even in case-insensitive file systems)
due to internal string comparison.
\item
The argument \textit{main} should be fully expanded, it cannot be a macro.
\item
Subdirectories and special characters should be avoided in filenames.
\item
The command |\childdocmain{|\textit{main}|}| must be followed by a whitespace.
It should not be followed immediately by another command
or by a comment mark `|%|'.
This is because the \TeX{} parser reads the token immediately following
the argument of |\childdocmain| and puts it
at the beginning of every child section;
however, a white\-space is ignored.
\end{itemize}

%%%%%%%%%%%%%%%%%%%%%%%%%%%%%%%%%%%%%%%%
\paragraph{Content of Main File.}

It is advisable to place all content in the child files included by |\include|.
Any output contained in the main file will appear in all child documents
unless suppressed manually;
it cannot be suppressed automatically by the |\includeonly| directive
and thus should normally be avoided.
A method to include some content in the main file
by means of conditional processing is described in \secref{sec:conditional}.

%%%%%%%%%%%%%%%%%%%%%%%%%%%%%%%%%%%%%%%%
\paragraph{Page Numbering.}

When only a part of the document is compiled,
the appropriate numbering of pages
(as well as other status parameters)
is determined from the |.aux| files.
The latter contain information from previous passes.
However this information needs to propagate through
all intermediate child documents.
Therefore the page numbering in child documents may well
be inconsistent until the complete document is compiled at least once.

A useful (if unconventional) way to always ensure a consistent
page numbering is to restart the numbering in each child document
and denote the pages by `\textit{child}|.|\textit{page}'
where \textit{child} represents the chapter/section number of the child file.
This can be achieved by the command
|\numberwithin{page}{|\textit{child}|}|
of the \textsf{amsmath} package
where \textit{child} can be |chapter| or |section|
depending on the chosen structuring.
Alternatively, one can modify the macro |\thepage| appropriately
and reset the counter |page| at the start of each child file.

%%%%%%%%%%%%%%%%%%%%%%%%%%%%%%%%%%%%%%%%%%%%%%%%%%%%%%%%%%%%%%%%%%%%%%%%%%%%%%%%
\subsection{Conditional Processing}
\label{sec:conditional}

The package provides a mechanism to compile different versions
of a document. To customise the versions further some conditional processing
can come in handy to distinguish which version is being compiled.
The package provides two macros to describe the compilation context:

%%%%%%%%%%%%%%%%%%%%%%%%%%%%%%%%%%%%%%%%
\DescribeMacro{\ifchilddoc}
The conditional |\ifchilddoc| distinguishes between the compilation of
child documents and the main document:
%
\begin{center}
|\ifchilddoc |\textit{child-code}| |[|\||else |\textit{main-code}]| \||fi|
\end{center}

%%%%%%%%%%%%%%%%%%%%%%%%%%%%%%%%%%%%%%%%
\DescribeMacro{\childdocname}
\DescribeMacro{\childdocjob}
The macro |\childdocname| contains the filename (without extension)
of the main or child file being processed.
Note that |\childdocjob| will always contain the name of the main file.

%%%%%%%%%%%%%%%%%%%%%%%%%%%%%%%%%%%%%%%%
\paragraph{Title Page.}

Conditional processing can be used to include a title or banner page
in the main document when proper precautions are taken.
Importantly, the code in the main file should ensure that the page counter
(as well as other status parameters which are stored in the |.aux| files)
takes the same value after the conditional processing.
Otherwise the page numbers may take divergent values
depending on which part is compiled.

For example, a title page could be declared by:
%
\begin{center}
\begin{tabular}{l}
|\ifchilddoc\||else|\\
|\addtocounter{page}{-1}|\\
\textit{code for title page}\\
|\newpage|\\
|\||fi|
\end{tabular}
\end{center}
%
A banner page for the child documents can be generated by:
%
\begin{center}
\begin{tabular}{l}
|\ifchilddoc|\\
|\addtocounter{page}{-1}|\\
\textit{code for banner page}\\
|\newpage|\\
|\||fi|
\end{tabular}
\end{center}
%
Here one could write a message such as:
\begin{center}
|This is the part \childdocname{} of \childdocjob{}.|
\end{center}

%%%%%%%%%%%%%%%%%%%%%%%%%%%%%%%%%%%%%%%%%%%%%%%%%%%%%%%%%%%%%%%%%%%%%%%%%%%%%%%%
\subsection{Flags}
\label{sec:flags}

The package makes it easy to generate different versions
of the main or child documents.
To this end compilation flags can be defined
and assigned different default values.
They will be particularly useful in conjunction
with the forwarding mechanism described in \secref{sec:forward}.

For example, it may be useful to have a flag |\version|
which can be set to |draft| or |final|.
The document source will contain some conditional code
depending on the value of |\version|.
Suppose further, the flag should default to |final| for the main file
and to |draft| for child files
which is a natural assignment for editing the document.
This is achieved by placing the following code
in the preamble of the main document
(below the |\childdocmain| directive):
%
\begin{center}
\begin{tabular}{l}
|\ifchilddoc|\\
|\providecommand{\version}{draft}|\\
|\||else|\\
|\providecommand{\version}{final}|\\
|\||fi|
\end{tabular}
\end{center}
%
The definition by |\providecommand| makes sure
that previous definitions are not overwritten.
Further statements |\providecommand{\version}{...}|
can thus be added before the above code to override it.

For the main file, one might add a line
(between |\childdocmain| and the above block)
%
\begin{center}
|%\ifchilddoc\||else\providecommand{\version}{draft}\||fi|
\end{center}
%
which can be uncommented to produce a draft version.
Likewise one can add a line to the very top of a child file
(above the |\childdocof{|\textit{main}|}| directive)
%
\begin{center}
|%\providecommand{\version}{final}|
\end{center}
%
which can be uncommented to produce the final version of this child document.

%%%%%%%%%%%%%%%%%%%%%%%%%%%%%%%%%%%%%%%%%%%%%%%%%%%%%%%%%%%%%%%%%%%%%%%%%%%%%%%%
\subsection{Forwarding}
\label{sec:forward}

Different versions of the main or child documents
using compilation flags as described in \secref{sec:flags}
can be (permanently) stored in different files
for convenient compilation, viewing and distribution.
To this end, the package defines a command
to pass on compilation to a different file:

%%%%%%%%%%%%%%%%%%%%%%%%%%%%%%%%%%%%%%%%
\DescribeMacro{\childdocforward}
The command |\childdocforward| redirects processing to
another source file:
%
\begin{center}
\begin{tabular}{l}
|\input{childdoc.def}|\\
|\childdocforward[|\textit{main}|]{|\textit{dest}|}|\\
\end{tabular}
\end{center}
%
The argument \textit{dest} is the destination file
(without extension).
It should be the main file or one of the child files.
Note that further \textsf{childdoc} directives
such as |\childdocof| and |\childdocforward|
in the indicated file will be processed in this form.
The optional argument \textit{main}
passes on directly to the main file \textit{main}
while pretending to compile the child \textit{dest}.
This form behaves as if \textit{dest}
issues |\childdocof{|\textit{main}|}| right away,
and no further \textsf{childdoc} directives will be processed.

%%%%%%%%%%%%%%%%%%%%%%%%%%%%%%%%%%%%%%%%
\DescribeMacro{\...prefix}
In the alternative form |\childdocforwardprefix|,
%
\begin{center}
\begin{tabular}{l}
|\input{childdoc.def}|\\
|\childdocforwardprefix[|\textit{main}|]{|\textit{prefix}|}{|\textit{dest}|}|
\end{tabular}
\end{center}
%
the destination file is determined by a pattern
depending on the current file:
To make this work, the current file must be called
`{\textit{prefix}\hspace{0.2em}\textit{suffix}}'
with \textit{prefix} matching precisely the argument.
Processing is then passed on to the file
`{\textit{dest}\hspace{0.2em}\textit{suffix}}'.
Surely, the same effect is achieved by
directly specifying the
argument `{\textit{dest}\hspace{0.2em}\textit{suffix}}'
in the first form.
However, that requires to set up a different file
for each child. With the alternative form of the command
all these files can have exactly the same content
which simplifies setting them up and maintaining them.

For example, the following file |draft.tex|
with a compilation flag |\version| as described in \secref{sec:flags}
compiles the main document as a draft:
%
\begin{center}
\begin{tabular}{l}
|\def\version{draft}|\\
|\input{childdoc.def}|\\
|\childdocforward{|\textit{main}|}|
\end{tabular}
\end{center}
%
Likewise, the following files |final|\textit{nn}|.tex|
compile the final version of the child document
|child|\textit{nn}|.tex|:
%
\begin{center}
\begin{tabular}{l}
|\def\version{final}|\\
|\input{childdoc.def}|\\
|\childdocforwardprefix{final}{child}|
\end{tabular}
\end{center}
%

Note that when several versions of a main file and/or of each child file
are to be generated, it may be convenient to set up a |Makefile| or
shell script to automatise the process.

%%%%%%%%%%%%%%%%%%%%%%%%%%%%%%%%%%%%%%%%%%%%%%%%%%%%%%%%%%%%%%%%%%%%%%%%%%%%%%%%
\subsection{Command Line Processing}
\label{sec:commandline}

The effect of redirection files can also be achieved by invoking
the \LaTeX{} compiler with a more elaborate command line.
Most conveniently this should be done as part
of a shell script or a |Makefile|.

When using \textsf{childdoc} in the main file, the following
command lines effectively perform a redirection
(note that depending on the shell being used,
backslashes may have to be doubled: `|\|' $\to$ `|\\|'):
%
\begin{center}
|... -jobname "|\textit{target}|" |\\|"|[\textit{flags}]%
|\input{childdoc.def}\childdocforward[|\textit{main}|]{|\textit{dest}|}"|
\end{center}
%
Here \textit{target} is the name of the output file,
\textit{main} is the name of the main file
and \textit{dest} is the name of the main or child file to be processed
(all filenames without extensions).
The optional argument \textit{main} can be omitted
if \textit{main} matches \textit{dest}.
Optionally, compilation \textit{flags} can be defined via |\def| commands.
This command line makes the \TeX{} engine believe
it is compiling the file \textit{target}
whose content is specified as the latter parameter.
The provided code then forwards the processing to
\textit{main} or \textit{dest} as described in \secref{sec:forward}.

%%%%%%%%%%%%%%%%%%%%%%%%%%%%%%%%%%%%%%%%%%%%%%%%%%%%%%%%%%%%%%%%%%%%%%%%%%%%%%%%
\subsection{Include by Input}
\label{sec:input}

Including child documents by |\include| has some restrictions by design.
Most notably, the content of a child document always occupies
its own set of pages; pages cannot be shared between child documents.
Usually, this behaviour makes perfect sense
because each child document contain an essential part of the document.
However, in some situations it may be desirable to compose
a document from a collection of parts
without having mandatory page breaks between then.
For this case, the package
provides a mechanism to include parts
by |\input| which can also be processed individually.
However, by construction this mechanism
requires manual handling of the content to be output.

%%%%%%%%%%%%%%%%%%%%%%%%%%%%%%%%%%%%%%%%
\DescribeMacro{\ifchilddocmanual}
The main file should be prepared as usual, see \secref{sec:include}.
However, the document body must make a distinction
between processing of an individual part and of the main document, e.g.:
%
\begin{center}
\begin{tabular}{l}
|\ifchilddocmanual|\\
|\input{\childdocname}|\\
|\||else|\\
\textit{document body with }|\input{|\textit{part}|}|\\
|\||fi|
\end{tabular}
\end{center}
%
The conditional |\ifchilddocmanual| is true whenever
a part to be included by |\input| is being compiled,
and the name of the part is stored in |\childdocname|.

%%%%%%%%%%%%%%%%%%%%%%%%%%%%%%%%%%%%%%%%
\DescribeMacro{\childdocby}
Each part to be included by |\input| should start with:
%
\begin{center}
\begin{tabular}{l}
|\input{childdoc.def}|\\
|\childdocby{|\textit{main}|}|\\
\end{tabular}
\end{center}
%
The directive |\childdocby| is similar to |\childdocof|
described in \secref{sec:include},
but the subsequent selection of content must be done manually.
To that end, both |\ifchilddoc| and |\ifchilddocmanual|
will be true upon processing of a part,
and the name of the part is stored in |\childdocname|.
Note that |\jobname| will be set to the filename of the current part
so that each part receives an individual |.aux| file
that does not interfere with the |.aux| file(s) of the main document.
This behaviour can be altered by the alternative form
|\childdocby[*]{|\textit{main}|}| (with a non-empty optional argument)
which uses the |.aux| file of the main document
by setting |\jobname| to \textit{main}.

%%%%%%%%%%%%%%%%%%%%%%%%%%%%%%%%%%%%%%%%%%%%%%%%%%%%%%%%%%%%%%%%%%%%%%%%%%%%%%%%
\subsection{Driver Development}
\label{sec:driver}

The \textsf{childdoc} mechanism can also be use for the development
of definition files such as \LaTeX{} styles or classes.
This case differs from the above setup with multiple parts
included by |\include| in that no |\includeonly| should be invoked.
This can be achieved by starting the include file
(before |\ProvidesPackage|) with:
%
\begin{center}
\begin{tabular}{l}
|\input{childdoc.def}|\\
|\childdocforward{|\textit{main}|}|\\
\end{tabular}
\end{center}
%
or alternatively with:
%
\begin{center}
\begin{tabular}{l}
|\input{childdoc.def}|\\
|\childdocby{|\textit{main}|}|\\
\end{tabular}
\end{center}
%
Both forms have slightly different effects as described above.
The main file is prepared as usual, see \secref{sec:include}.

%%%%%%%%%%%%%%%%%%%%%%%%%%%%%%%%%%%%%%%%%%%%%%%%%%%%%%%%%%%%%%%%%%%%%%%%%%%%%%%%
\subsection{Legacy Detection}
\label{sec:detection}

The directive |\childdocmain| in the main file can detect
whether the complete document or merely a child is to be compiled
even without using the directive |\childdocof|.
This method is deprecated because it is less robust
and there is no compelling reason to use it;
it is merely provided for backward compatibility
and it may be removed in future versions.

If the detection mechanism is to be used,
it is mandatory to correctly specify
the filename of the main file as the argument of |\childdocmain|:
%
\begin{center}
\begin{tabular}{l}
|\input{childdoc.def}|\\
|\childdocmain{|\textit{main}|}|\\
\end{tabular}
\end{center}
%
If |\jobname| does not match the argument \textit{main} of |\childdocmain|,
it is assumed that |\jobname| points to the child file to be compiled.
When using |\childdocmain| with the main file specified as argument,
it suffices to start a child file
with just |\input{|\textit{main}|}|
without loading of the package and using |\childdocof|.
If instead all processing is done
with the appropriate \textsf{childdoc} directives,
the argument of \textit{main} of |\childdocmain| can be empty.

An alternative version of the command line processing described
in \secref{sec:commandline} using the detection mechanism reads:
%
\begin{center}
|... -jobname "|\textit{target}|" "|[\textit{flags}]%
[|\def\jobname{|\textit{dest}|}|]|\input{|\textit{main}|}"|
\end{center}

%%%%%%%%%%%%%%%%%%%%%%%%%%%%%%%%%%%%%%%%%%%%%%%%%%%%%%%%%%%%%%%%%%%%%%%%%%%%%%%%
\subsection{Manual Code}
\label{sec:manual}

In case one cannot be certain whether the definitions file |childdoc.def|
is installed on the target \TeX{} distribution
and one prefers not to ship it,
it is conceivable to paste a few relevant commands into the sources.

To that end, drop all statements |\input{childdoc.def}|
and perform the replacements as outlined below.
Instead of |\childdocmain{|\textit{main}|}| add the following code
to the top of the main file:
%
\begin{center}
\begin{tabular}{l}
|\||ifdefined\childdocname\endinput\||fi\newif\ifchilddoc|\\
|\edef\childdocname{\scantokens\expandafter{\jobname\noexpand}}|\\
|\def\childdocmain{|\textit{main}|}\||ifx\childdocmain\childdocname\||else|\\
|\childdoctrue\includeonly{\childdocname}\let\jobname\childdocmain\||fi|\\
\end{tabular}
\end{center}
%
Instead of |\childdocof{|\textit{main}|}| just include the main file
at the top of each child file:
%
\begin{center}
|\input{|\textit{main}|}|
\end{center}
%
A simple redirection |\childdocforward{|\textit{dest}|}| is achieved by:
%
\begin{center}
|\def\jobname{|\textit{dest}|}\input{\jobname}|
\end{center}
%
The redirection with prefix
|\childdocforwardprefix[|\textit{prefix}|]{|\textit{dest}|}|
is accomplished by:
%
\begin{center}
\begin{tabular}{l}
|{\edef\jobname{\scantokens\expandafter{\jobname\noexpand}}|\\
|\def\redirectjob |\textit{prefix}|#1~~~{\gdef\jobname{|\textit{dest}|#1}}|\\
|\expandafter\redirectjob\jobname~~~}\input{\jobname}|
\end{tabular}
\end{center}

In an alternative approach,
child documents can be compiled by a specific command line
without additional code or specific definitions:
%
\begin{center}
|... -jobname "|\textit{target}|" "|[\textit{flags}]%
|\includeonly{|\textit{dest}|}\input{|\textit{main}|}"|
\end{center}
%

%%%%%%%%%%%%%%%%%%%%%%%%%%%%%%%%%%%%%%%%%%%%%%%%%%%%%%%%%%%%%%%%%%%%%%%%%%%%%%%%
%%%%%%%%%%%%%%%%%%%%%%%%%%%%%%%%%%%%%%%%%%%%%%%%%%%%%%%%%%%%%%%%%%%%%%%%%%%%%%%%
\section{Information}

%%%%%%%%%%%%%%%%%%%%%%%%%%%%%%%%%%%%%%%%%%%%%%%%%%%%%%%%%%%%%%%%%%%%%%%%%%%%%%%%
\subsection{Copyright}

Copyright \copyright{} 2017--2018 Niklas Beisert

This work may be distributed and/or modified under the
conditions of the \LaTeX{} Project Public License, either version 1.3
of this license or (at your option) any later version.
The latest version of this license is in
  \url{http://www.latex-project.org/lppl.txt}
and version 1.3 or later is part of all distributions of \LaTeX{}
version 2005/12/01 or later.

This work has the LPPL maintenance status `maintained'.

The Current Maintainer of this work is Niklas Beisert.

This work consists of the files |README.txt|, |childdoc.ins| and |childdoc.dtx|
as well as the derived files |childdoc.def|, |cdocsamp.tex|
with |cdocsch1.tex|, |cdocsch2.tex|, |cdocspt3.tex|, |cdocspt4.tex|,
|cdocsdrf.tex|, |cdocsfn1.tex|, |cdocsfn2.tex|
as well as |childdoc.pdf|.

%%%%%%%%%%%%%%%%%%%%%%%%%%%%%%%%%%%%%%%%%%%%%%%%%%%%%%%%%%%%%%%%%%%%%%%%%%%%%%%%
\subsection{Files and Installation}

The package consists of the files:
%
\begin{center}
\begin{tabular}{ll}
    |README.txt|   & readme file \\
    |childdoc.ins| & installation file \\
    |childdoc.dtx| & source file \\
    |childdoc.def| & definition file \\
    |cdocsamp.tex| & sample main file \\
    |cdocsch1.tex| & sample include file \\
    |cdocsch2.tex| & sample include file \\
    |cdocspt3.tex| & sample part file \\
    |cdocspt4.tex| & sample part file \\
    |cdocsdrf.tex| & sample redirection file \\
    |cdocsfn1.tex| & sample redirection file \\
    |cdocsfn2.tex| & sample redirection file \\
    |childdoc.pdf| & manual
\end{tabular}
\end{center}
%
The distribution consists of the files
|README.txt|, |childdoc.ins| and |childdoc.dtx|.
%
\begin{itemize}
\item
Run (pdf)\LaTeX{} on |childdoc.dtx|
to compile the manual |childdoc.pdf| (this file).
\item
Run \LaTeX{} on |childdoc.ins| to create the definitions file |childdoc.def|
and the sample |cdocsamp.tex| with include files
|cdocsch1.tex|, |cdocsch2.tex|, |cdocspt3.tex|, |cdocspt4.tex|,
|cdocsdrf.tex|, |cdocsfn1.tex|, |cdocsfn2.tex|.
Then copy the file |childdoc.def| to an appropriate directory of your \LaTeX{}
distribution, e.g.\ \textit{texmf-root}|/tex/latex/childdoc|.
\end{itemize}

%%%%%%%%%%%%%%%%%%%%%%%%%%%%%%%%%%%%%%%%%%%%%%%%%%%%%%%%%%%%%%%%%%%%%%%%%%%%%%%%
\subsection{Related CTAN Packages}

There are several other packages which offer a similar functionality:
%
\begin{itemize}
\item
The packages
\href{http://ctan.org/pkg/docmute}{\textsf{docmute}},
\href{http://ctan.org/pkg/includex}{\textsf{includex}} and
\href{http://ctan.org/pkg/standalone}{\textsf{standalone}}
provide commands to include only the document body of
a child file thus allowing both files to be compiled individually.
\item
The packages \href{http://ctan.org/pkg/subdocs}{\textsf{subdocs}}
and \href{http://ctan.org/pkg/subfiles}{\textsf{subfiles}}
provide structures in which the main and child documents can be
encapsulated and allowing them to be compiled individually.
The inclusion mechanism is different from the conventional |\include|.
\item
The package \href{http://ctan.org/pkg/combine}{\textsf{combine}}
is an elaborate solution to combine several documents into one.
\end{itemize}
%
See also the CTAN topic \href{http://ctan.org/topic/subdocs}{\textsf{subdocs}}
for further related packages.
The present package differs from the above solutions in that
a document structure constructed with the conventional |\include| mechanism
just needs two extra commands at the top of every file
such that all constituent files can be compiled individually.

%%%%%%%%%%%%%%%%%%%%%%%%%%%%%%%%%%%%%%%%%%%%%%%%%%%%%%%%%%%%%%%%%%%%%%%%%%%%%%%%
%\subsection{Feature Suggestions}
%
%The following is a list of features which may be useful for future
%versions of this package:
%%
%\begin{itemize}
%\item
%\ldots
%\end{itemize}

%%%%%%%%%%%%%%%%%%%%%%%%%%%%%%%%%%%%%%%%%%%%%%%%%%%%%%%%%%%%%%%%%%%%%%%%%%%%%%%%
\subsection{Revision History}

%%%%%%%%%%%%%%%%%%%%%%%%%%%%%%%%%%%%%%%%
\paragraph{v2.0:} 2018/12/30

\begin{itemize}
\item
immediate forward processing
\item
added |\childdocby| mechanism
\item
manual restructured
\end{itemize}

%%%%%%%%%%%%%%%%%%%%%%%%%%%%%%%%%%%%%%%%
\paragraph{v1.6:} 2018/01/17

\begin{itemize}
\item
application for development of include files
\item
corrections to manual
\end{itemize}

%%%%%%%%%%%%%%%%%%%%%%%%%%%%%%%%%%%%%%%%
\paragraph{v1.5:} 2017/05/21

\begin{itemize}
\item
more complete structuring introduced
\item
|\childdocof| introduced
\item
|\childdoc| renamed to |\childdocmain|
\item
|\childredirect| renamed to |\childdocforward| and |\childdocforwardprefix|
and functionality expanded
\end{itemize}

%%%%%%%%%%%%%%%%%%%%%%%%%%%%%%%%%%%%%%%%
\paragraph{v1.0:} 2017/04/27

\begin{itemize}
\item
manual and install package
\item
first version published on CTAN
\end{itemize}

%%%%%%%%%%%%%%%%%%%%%%%%%%%%%%%%%%%%%%%%
\paragraph{v0.6:} 2017/04/26

\begin{itemize}
\item
redirection mechanism added
\end{itemize}

%%%%%%%%%%%%%%%%%%%%%%%%%%%%%%%%%%%%%%%%
\paragraph{v0.5:} 2017/04/26

\begin{itemize}
\item
functionality in definition file
\end{itemize}


%%%%%%%%%%%%%%%%%%%%%%%%%%%%%%%%%%%%%%%%%%%%%%%%%%%%%%%%%%%%%%%%%%%%%%%%%%%%%%%%
%%%%%%%%%%%%%%%%%%%%%%%%%%%%%%%%%%%%%%%%%%%%%%%%%%%%%%%%%%%%%%%%%%%%%%%%%%%%%%%%
%%%%%%%%%%%%%%%%%%%%%%%%%%%%%%%%%%%%%%%%%%%%%%%%%%%%%%%%%%%%%%%%%%%%%%%%%%%%%%%%
\appendix

\settowidth\MacroIndent{\rmfamily\scriptsize 000\ }

 \DocInput{childdoc.dtx}

\end{document}
%</driver>
% \fi
%
% %%%%%%%%%%%%%%%%%%%%%%%%%%%%%%%%%%%%%%%%%%%%%%%%%%%%%%%%%%%%%%%%%%%%%%%%%%%%%%
% %%%%%%%%%%%%%%%%%%%%%%%%%%%%%%%%%%%%%%%%%%%%%%%%%%%%%%%%%%%%%%%%%%%%%%%%%%%%%%
% \section{Sample}
%\iffalse
%<*samplemain>
%\fi
%
% The following presents a sample document
% with two chapters, two parts, a title page,
% a compile flag as well as three forwarding files to set the flag.
% It consists of eight |.tex| files:
% \begin{center}
% \begin{tabular}{ll}
% |cdocsamp.tex|&main file\\
% |cdocsch1.tex|&include file for chapter 1\\
% |cdocsch2.tex|&include file for chapter 2\\
% |cdocspt3.tex|&include file for part 3\\
% |cdocspt4.tex|&include file for part 4\\
% |cdocsdrf.tex|&forwarding file for main file in draft mode\\
% |cdocsfi1.tex|&forwarding file for final version of chapter 1\\
% |cdocsfi2.tex|&forwarding file for final version of chapter 2\\
% \end{tabular}
% \end{center}
% Each of the eight files can be compiled directly by the \LaTeX{} compiler.
%
% %%%%%%%%%%%%%%%%%%%%%%%%%%%%%%%%%%%%%%
% \paragraph{Main File.}
%
% The main file is called |cdocsamp.tex|.
%
% Load the \textsf{childdoc} definitions and
% declare the filename for the main document:
%    \begin{macrocode}
\input{childdoc.def}
\childdocmain{}
%    \end{macrocode}

% Optional override for |\version| flag:
%    \begin{macrocode}
%%\ifchilddoc\else\providecommand{\version}{draft}\fi
%    \end{macrocode}

% Define the default values for the |\version| flag
% (|final| for the main file and |draft| for childs):
%    \begin{macrocode}
\ifchilddoc
\providecommand{\version}{draft}
\else
\providecommand{\version}{final}
\fi
%    \end{macrocode}

% Load the standard document class:
%    \begin{macrocode}
\documentclass[12pt]{article}
%    \end{macrocode}

% Start the document body:
%    \begin{macrocode}
\begin{document}
%    \end{macrocode}

% Declare a title page.
% Print title, part of document being processed and version flag:
%    \begin{macrocode}
\addtocounter{page}{-1}
\begin{center}
{\LARGE\bfseries{}childdoc example\par}
\vspace{1cm}
\ifchilddoc
\ifchilddocmanual part\else chapter\fi:
`\childdocname' of `\childdocjob'\par
\else
main document: `\childdocjob'\par
\fi
version: \version\par
\end{center}
\newpage
%    \end{macrocode}

% Manually include selected file,
% otherwise process as usual:
%    \begin{macrocode}
\ifchilddocmanual
\section*{part `\childdocname'}
\input{\childdocname}
\else
%    \end{macrocode}

% Include the two chapters:
%    \begin{macrocode}
\include{cdocsch1}
\include{cdocsch2}
%    \end{macrocode}

% Include the two parts unless only chapters should be displayed:
%    \begin{macrocode}
\ifchilddoc\else
\section{part three}
\input{cdocspt3}
\section{part four}
\input{cdocspt4}
\fi
%    \end{macrocode}

% Process as usual until here:
%    \begin{macrocode}
\fi
%    \end{macrocode}

% End of document body:
%    \begin{macrocode}
\end{document}
%    \end{macrocode}
%\iffalse
%</samplemain>
%\fi
%
% %%%%%%%%%%%%%%%%%%%%%%%%%%%%%%%%%%%%%%
% \paragraph{Chapter Include Files.}
%
% The include files are called |cdocsch1.tex| and |cdocsch2.tex|.
%
%\iffalse
%<*samplechap1|samplechap2>
%\fi

% Optional override for |\version| flag:
%    \begin{macrocode}
%%\providecommand{\version}{final}
%    \end{macrocode}

% Include the main document:
%    \begin{macrocode}
\input{childdoc.def}
\childdocof{cdocsamp}
%    \end{macrocode}

%\iffalse
%</samplechap1|samplechap2>
%\fi
%
%\iffalse
%<*samplechap1>
%\fi
% Some text for chapter 1:
%    \begin{macrocode}
\section{one}
some text in chapter one
%    \end{macrocode}

%\iffalse
%</samplechap1>
%\fi
% Some text for chapter 2:
%\iffalse
%<*samplechap2>
%\fi
%    \begin{macrocode}
\section{two}
more text in chapter two
%    \end{macrocode}

%\iffalse
%</samplechap2>
%\fi
%
% %%%%%%%%%%%%%%%%%%%%%%%%%%%%%%%%%%%%%%
% \paragraph{Part Include Files.}
%
% The include files are called |cdocspt3.tex| and |cdocspt4.tex|.
%
%\iffalse
%<*samplepart3|samplepart4>
%\fi

% Optional override for |\version| flag:
%    \begin{macrocode}
%%\providecommand{\version}{final}
%    \end{macrocode}

% Include the main document:
%    \begin{macrocode}
\input{childdoc.def}
\childdocby{cdocsamp}
%    \end{macrocode}

%\iffalse
%</samplepart3|samplepart4>
%\fi
%
%\iffalse
%<*samplepart3>
%\fi
% Some text for part 3:
%    \begin{macrocode}
some text in part three
%    \end{macrocode}

%\iffalse
%</samplepart3>
%\fi
% Some text for part 4:
%\iffalse
%<*samplepart4>
%\fi
%    \begin{macrocode}
more text in part four
%    \end{macrocode}

%\iffalse
%</samplepart4>
%\fi
%
% %%%%%%%%%%%%%%%%%%%%%%%%%%%%%%%%%%%%%%
% \paragraph{Forwarding for a Complete Draft.}
%
% The following forwarding file |cdocsdrf.tex|
% compiles the main document in draft mode:
%\iffalse
%<*sampledraft>
%\fi
%    \begin{macrocode}
\def\version{draft}
\input{childdoc.def}
\childdocforward{cdocsamp}
%    \end{macrocode}

%\iffalse
%</sampledraft>
%\fi
%
% %%%%%%%%%%%%%%%%%%%%%%%%%%%%%%%%%%%%%%
% \paragraph{Forwarding for Final Version of the Chapters.}
%
% The following forwarding files |cdocsfn1.tex| and |cdocsfn2.tex|
% (with identical content)
% compile the final versions of the child documents
% |cdocsch1.tex| and |cdocsch2.tex|, respectively:
%\iffalse
%<*samplefinal>
%\fi
%    \begin{macrocode}
\def\version{final}
\input{childdoc.def}
\childdocforwardprefix[cdocsamp]{cdocsfn}{cdocsch}
%    \end{macrocode}

%\iffalse
%</samplefinal>
%\fi
%
% %%%%%%%%%%%%%%%%%%%%%%%%%%%%%%%%%%%%%%
% \paragraph{Command Line Processing.}
%
% The following three command lines generate the output files
% |cdocscld|, |cdocscl1| and |cdocscl2|
% which should be identical to
% |cdocsdrf|, |cdocsch1| and |cdocsfn2|, respectively:
% \begin{center}
% \begin{tabular}{l}
% |latex -jobname cdocscld \|\\
% |  "\def\version{draft}\input{childdoc.def}\childdocforward{cdocsamp}"|\\
% |latex -jobname cdocscl1 \|\\
% |  "\input{childdoc.def}\childdocforward[cdocsamp]{cdocsch1}"|\\
% |latex -jobname cdocscl2 \|\\
% |  "\def\version{final}\input{childdoc.def}\childdocforward{cdocsch2}"|
% \end{tabular}
% \end{center}
% Note that the trailing backslash on each first line
% merely continues the input to the second line
% (for convenient cut ant paste).
% Furthermore, the command |latex| can be replaced by any
% of its alternative versions such as |pdflatex|.
%
% %%%%%%%%%%%%%%%%%%%%%%%%%%%%%%%%%%%%%%%%%%%%%%%%%%%%%%%%%%%%%%%%%%%%%%%%%%%%%%
% %%%%%%%%%%%%%%%%%%%%%%%%%%%%%%%%%%%%%%%%%%%%%%%%%%%%%%%%%%%%%%%%%%%%%%%%%%%%%%
% \section{Implementation}
%\iffalse
%<*package>
%\fi
%
% This section describes the definitions file |childdoc.def|.

% The definitions cannot be loaded using |\usepackage| or |\RequirePackage|
% which has a mechanism to prevent loading a style file more than once.
% When loading the definitions by means of |\input|
% multiple instances have to be prevented manually:
%\iffalse
%This code needs to be before the `\ProvidesFile' directive
%which is defined at the beginning of this file.
%Therefore it is also placed there and commented out here.
%</package>
%<*discard>
%\fi
%    \begin{macrocode}
\ifdefined\childdocmain\endinput\fi
%    \end{macrocode}
%\iffalse
%</discard>
%<*package>
%\fi
%
% \macro{\ifchilddoc}
% \macro{\ifchilddocmanual}
% The conditional |\ifchilddoc| tells whether a
% child (true) or main (false) document is being compiled.
% The conditional |\ifchilddocmanual| tells whether
% the |\includeonly| mechanism is used (false) or
% the selection of child files must be performed manually (true).
% The definitions initialise to false:
%    \begin{macrocode}
\newif\ifchilddoc
\newif\ifchilddocmanual
%    \end{macrocode}

% \macro{\childdocname}
% \macro{\childdocjob}
% The macro |\childdocname| stores the name of the main document
% to be compiled. The macro |\childdocjob| stores the name of
% the document on which the \LaTeX{} compiler was originally invoked.
% The content of |\jobname| cannot be compared
% to filenames specified in the source due to different catcodes.
% The following code rescans |\jobname|, stores the result
% in |\childdocname| and saves a copy in |\childdocjob|:
%    \begin{macrocode}
\edef\childdocname{\scantokens\expandafter{\jobname\noexpand}}
\let\childdocjob\childdocname
%    \end{macrocode}

% \macro{\childdocdisable}
% The macro |\childdocdisable| prevents the main file
% from being processed more than once.
% At this stage, the main document command |\childdocmain|
% is assumed to be called once again where it should do nothing.
% Any subsequent call to it should prevent
% a secondary processing of the main document
% It overwrites the forwarding commands
% |\childdocof| and |\childdocforward|
% with empty macros to prevent further inclusions of the main document:
%    \begin{macrocode}
\newcommand{\childdocdisable}
{
  \renewcommand{\childdocmain}[1]{\renewcommand{\childdocmain}[1]{\endinput}}
  \renewcommand{\childdocof}[1]{}
  \renewcommand{\childdocby}[2][]{}
  \renewcommand{\childdocforward}[2][]{}
  \renewcommand{\childdocdisable}{}
}
%    \end{macrocode}

% \macro{\childdocmain}
% The macro |\childdocmain| is to be called at the top of the main file
% with nothing or the main filename (without extension) as argument.
% First, it breaks loops.
% If the argument is not empty and does not match |\childdocname|
% (which is set by the first inclusion of |childdoc.def|),
% |\ifchilddoc| is set to true, |\includeonly| is applied to the child file
% and |\jobname| is set to the main file
% (for proper handling of |.aux| files):
%    \begin{macrocode}
\newcommand{\childdocmain}[1]
{
  \childdocdisable\childdocmain{}
  \if?#1?\else
    \begingroup
      \def\childdoctmp{#1}
      \ifx\childdoctmp\childdocname
        \def\childdoctmp{}
      \else
        \def\childdoctmp
        {
          \childdoctrue
          \includeonly{\childdocname}
          \def\childdocjob{#1}
          \def\jobname{#1}
        }
      \fi
      \expandafter
    \endgroup
    \childdoctmp
  \fi
}
%    \end{macrocode}

% \macro{\childdocof}
% The command |\childdocof| redirects
% compilation to the main file |#1|.
%    \begin{macrocode}
\newcommand{\childdocof}[1]
{
  \childdocdisable
  \childdoctrue
  \includeonly{\childdocname}
  \def\jobname{#1}
  \def\childdocjob{#1}
  \input{#1}
}
%    \end{macrocode}

% \macro{\childdocby}
% The command |\childdocby| ....
%    \begin{macrocode}
\newcommand{\childdocby}[2][]
{
  \childdocdisable
  \childdoctrue
  \childdocmanualtrue
  \if?#1?\else
    \def\jobname{#2}
  \fi
  \def\childdocjob{#2}
  \input{#2}
  \endinput
}
%    \end{macrocode}

% \macro{\childdocforward}
% The command |\childdocforward| redirects
% compilation to the main file or
% (if the optional argument is given) a child file.
% Parameters are set as if the main file
% or a child file starting with |\childdocof| was compiled.
% Then compilation is handed over to the main file:
%    \begin{macrocode}
\newcommand{\childdocforward}[2][]
{
  \begingroup
    \if?#1?
      \def\childdoctmp
      {
        \def\childdocname{#2}
        \def\childdocjob{#2}
        \def\jobname{#2}
        \input{#2}
        \endinput
      }
    \else
      \def\childdoctmp
      {
        \childdocdisable
        \def\childdocname{#2}
        \childdoctrue
        \includeonly{#2}
        \def\childdocjob{#1}
        \def\jobname{#1}
        \input{#1}
        \endinput
      }
    \fi
    \expandafter
  \endgroup
  \childdoctmp
}
%    \end{macrocode}

% \macro{\childdocforwardprefix}
% The command |\childdocforwardprefix| redirects
% compilation to the main or a child file by means of a pattern.
% The prefix |#1| in the current filename is replaced by |#2|
% and the suffix of the current filename is kept
% (it is assumed that the filename does not contain the substring `|~~~|'
% which is used as a delimiter).
% Compilation is handed over to the new file by |\childdocforward|:
%    \begin{macrocode}
\newcommand{\childdocforwardprefix}[3][]
{
  \begingroup
    \def\childdocextract #2##1~~~{\def\childdoctmp{\childdocforward[#1]{#3##1}}}
    \expandafter\childdocextract\childdocname~~~
    \expandafter
  \endgroup
  \childdoctmp
}
%    \end{macrocode}

% \macro{\childdoc}
% The deprecated macro |\childdoc| is a legacy version of |\childdocmain|:
%    \begin{macrocode}
\newcommand{\childdoc}{\childdocmain}
%    \end{macrocode}

% \macro{\childdocredirect}
% The deprecated macro |\childdocredirect| is a legacy version
% of |\childdocforward| and |\childdocforwardprefix|:
%    \begin{macrocode}
\newcommand{\childdocredirect}[2][]
{
  \begingroup
    \if?#1?
      \def\childdoctmp{\childdocforward{#2}}
    \else
      \def\childdoctmp{\childdocforwardprefix{#1}{#2}}
    \fi
    \expandafter
  \endgroup
  \childdoctmp
}
%    \end{macrocode}

%\iffalse
%</package>
%\fi
%
\endinput
|\\
|\childdocmain{}|\\
\end{tabular}
\end{center}
at the very top of the main \LaTeX{} file,
in particular \emph{before} the |\documentclass| statement!
The argument of |\childdocmain| should be left empty
(but it must be present).

%%%%%%%%%%%%%%%%%%%%%%%%%%%%%%%%%%%%%%%%
\DescribeMacro{\childdocof}
Furthermore, add the commands
\begin{center}
\begin{tabular}{l}
|% \iffalse
%
% childdoc.dtx Copyright (C) 2017-2018 Niklas Beisert
%
% This work may be distributed and/or modified under the
% conditions of the LaTeX Project Public License, either version 1.3
% of this license or (at your option) any later version.
% The latest version of this license is in
%   http://www.latex-project.org/lppl.txt
% and version 1.3 or later is part of all distributions of LaTeX
% version 2005/12/01 or later.
%
% This work has the LPPL maintenance status `maintained'.
%
% The Current Maintainer of this work is Niklas Beisert.
%
% This work consists of the files childdoc.dtx and childdoc.ins
% and the derived files childdoc.def and cdocsamp.tex with
% cdocsch1.tex, cdocsch2.tex, cdocsdrf.tex, cdocsfn1.tex, cdocsfn2.tex.
%
%<package>\ifdefined\childdocmain\endinput\fi
%<package>\ProvidesFile{childdoc.def}[2018/12/30 v2.0 child document driver]
%<samplemain>\ProvidesFile{cdocsamp.tex}[2018/12/30 v2.0 sample for childdoc]
%<*driver>
%\ProvidesFile{childdoc.drv}[2018/12/30 v2.0 childdoc reference manual file]
\PassOptionsToClass{10pt,a4paper}{article}
\documentclass{ltxdoc}

\usepackage[margin=35mm]{geometry}
\usepackage{hyperref}
\usepackage{hyperxmp}
\usepackage[usenames]{color}

\hypersetup{colorlinks=true}
\hypersetup{pdfstartview=FitH}
\hypersetup{pdfpagemode=UseNone}
\hypersetup{pdfsource={}}
\hypersetup{pdflang={en-UK}}
\hypersetup{pdfcopyright={Copyright 2017-2018 Niklas Beisert.
  This work may be distributed and/or modified under the
  conditions of the LaTeX Project Public License, either version 1.3
  of this license or (at your option) any later version.}}
\hypersetup{pdflicenseurl={http://www.latex-project.org/lppl.txt}}
\hypersetup{pdfcontactaddress={ETH Zurich, ITP, HIT K,
  Wolfgang-Pauli-Strasse 27}}
\hypersetup{pdfcontactpostcode={8093}}
\hypersetup{pdfcontactcity={Zurich}}
\hypersetup{pdfcontactcountry={Switzerland}}
\hypersetup{pdfcontactemail={nbeisert@itp.phys.ethz.ch}}
\hypersetup{pdfcontacturl={http://people.phys.ethz.ch/\xmptilde nbeisert/}}

\newcommand{\secref}[1]{\hyperref[#1]{section \ref*{#1}}}

\parskip1ex
\parindent0pt
\let\olditemize\itemize
\def\itemize{\olditemize\parskip0pt}

\begin{document}

\title{The \textsf{childdoc} Package}
\hypersetup{pdftitle={The childdoc Package}}
\author{Niklas Beisert\\[2ex]
  Institut f\"ur Theoretische Physik\\
  Eidgen\"ossische Technische Hochschule Z\"urich\\
  Wolfgang-Pauli-Strasse 27, 8093 Z\"urich, Switzerland\\[1ex]
  \href{mailto:nbeisert@itp.phys.ethz.ch}
  {\texttt{nbeisert@itp.phys.ethz.ch}}}
\hypersetup{pdfauthor={Niklas Beisert}}
\hypersetup{pdfsubject={Manual for the LaTeX2e Package childdoc}}
\date{30 December 2018, \textsf{v2.0}}
\maketitle

\begin{abstract}\noindent
\textsf{childdoc} is a \LaTeXe{} package
that enables the direct compilation
of document sections included by |\include|
to individual files.
\end{abstract}

\begingroup
\parskip0ex
\tableofcontents
\endgroup

%%%%%%%%%%%%%%%%%%%%%%%%%%%%%%%%%%%%%%%%%%%%%%%%%%%%%%%%%%%%%%%%%%%%%%%%%%%%%%%%
%%%%%%%%%%%%%%%%%%%%%%%%%%%%%%%%%%%%%%%%%%%%%%%%%%%%%%%%%%%%%%%%%%%%%%%%%%%%%%%%
\section{Introduction}

\LaTeX{} provides a mechanism to structure a large document (such as a book)
into a main file and several child files (containing the chapters)
using the |\include| command.
This mechanism is beneficial for documents
which span hundreds of pages in order to
make the source file(s) more manageable.
Moreover, compilation can be restricted to
selected child files by means of the |\includeonly| command.
The latter feature can be used to reduce the compilation time while editing
(this was significantly more useful in the earlier days of \LaTeX{})
or to generate a smaller document which is easier to navigate.
Another application of |\includeonly| is to generate
documents consisting of selected parts of the complete document.

However, there are a few drawbacks of the plain |\include| mechanism:
\begin{itemize}
\item
The child files cannot be compiled on their own,
they can only be compiled via the main file.
A naive editing environment
(such as a text editor with an option
to have the current file processed by \LaTeX)
may require one to switch to the main file before compiling;
attempting to compile the child file produces errors.
\item
The main file must be modified (each time)
to adjust the |\includeonly| command
to the present needs. This easily leaves the main file in a messy state.
\item
The generated document will always carry the filename
of the main document. This is inconvenient if
several child files are to be compiled and
to be kept for distribution.
\end{itemize}

The present package provides a simple interface
to make child files individually compilable by \LaTeX{}.
Compiling a child file then has the same effect as compiling
the main file with an |\includeonly| command
to select the appropriate child.
Moreover the generated document will carry the name of the child
rather than the main file.
This resolves all three above issues.

This feature is meant to make the editing of books,
thesis documents and lecture notes somewhat more convenient.
However, the package can also be used efficiently for
composing a series of documents (such as exercise sheets)
which are typically distributed individually.
It then assists the author in generating the individual documents
(potentially in different versions)
as well as a document containing the collected series.
Another application is in developing style files
or other kinds of included material
where compilation of the style file could redirect
to a sample or test file.

%%%%%%%%%%%%%%%%%%%%%%%%%%%%%%%%%%%%%%%%%%%%%%%%%%%%%%%%%%%%%%%%%%%%%%%%%%%%%%%%
%%%%%%%%%%%%%%%%%%%%%%%%%%%%%%%%%%%%%%%%%%%%%%%%%%%%%%%%%%%%%%%%%%%%%%%%%%%%%%%%
\section{Usage}

First of all, the package \textsf{childdoc} is \emph{not} a standard
\LaTeXe{} |.sty| style file! Therefore it needs to be invoked in
a non-standard way.

%%%%%%%%%%%%%%%%%%%%%%%%%%%%%%%%%%%%%%%%%%%%%%%%%%%%%%%%%%%%%%%%%%%%%%%%%%%%%%%%
\subsection{Included Files}
\label{sec:include}

%%%%%%%%%%%%%%%%%%%%%%%%%%%%%%%%%%%%%%%%
\DescribeMacro{\childdocmain}
To use the package, add the commands
\begin{center}
\begin{tabular}{l}
|\input{childdoc.def}|\\
|\childdocmain{}|\\
\end{tabular}
\end{center}
at the very top of the main \LaTeX{} file,
in particular \emph{before} the |\documentclass| statement!
The argument of |\childdocmain| should be left empty
(but it must be present).

%%%%%%%%%%%%%%%%%%%%%%%%%%%%%%%%%%%%%%%%
\DescribeMacro{\childdocof}
Furthermore, add the commands
\begin{center}
\begin{tabular}{l}
|\input{childdoc.def}|\\
|\childdocof{|\textit{main}|}|\\
\end{tabular}
\end{center}
at the top of every child file \textit{child}
which is included by |\include{|\textit{child}|}|
from within the main file
(or at least for those files to be compiled individually).
The argument \textit{main} must be the filename of the main file.

There are a couple of
considerations in setting up the main and child documents:

%%%%%%%%%%%%%%%%%%%%%%%%%%%%%%%%%%%%%%%%
\paragraph{Restrictions.}

Please note the following restrictions:
\begin{itemize}
\item
|\childdocmain| must be called with one argument \textit{main}
to ensure compatibility with earlier version of the package.
It must either be empty (|\childdocmain{}|)
or precisely match the filename of the main file in which it is specified.
See \secref{sec:detection} for further information.
\item
The filename \textit{main} must be specified without the |.tex| extension.
\item
The filename \textit{main} is case sensitive
(even in case-insensitive file systems)
due to internal string comparison.
\item
The argument \textit{main} should be fully expanded, it cannot be a macro.
\item
Subdirectories and special characters should be avoided in filenames.
\item
The command |\childdocmain{|\textit{main}|}| must be followed by a whitespace.
It should not be followed immediately by another command
or by a comment mark `|%|'.
This is because the \TeX{} parser reads the token immediately following
the argument of |\childdocmain| and puts it
at the beginning of every child section;
however, a white\-space is ignored.
\end{itemize}

%%%%%%%%%%%%%%%%%%%%%%%%%%%%%%%%%%%%%%%%
\paragraph{Content of Main File.}

It is advisable to place all content in the child files included by |\include|.
Any output contained in the main file will appear in all child documents
unless suppressed manually;
it cannot be suppressed automatically by the |\includeonly| directive
and thus should normally be avoided.
A method to include some content in the main file
by means of conditional processing is described in \secref{sec:conditional}.

%%%%%%%%%%%%%%%%%%%%%%%%%%%%%%%%%%%%%%%%
\paragraph{Page Numbering.}

When only a part of the document is compiled,
the appropriate numbering of pages
(as well as other status parameters)
is determined from the |.aux| files.
The latter contain information from previous passes.
However this information needs to propagate through
all intermediate child documents.
Therefore the page numbering in child documents may well
be inconsistent until the complete document is compiled at least once.

A useful (if unconventional) way to always ensure a consistent
page numbering is to restart the numbering in each child document
and denote the pages by `\textit{child}|.|\textit{page}'
where \textit{child} represents the chapter/section number of the child file.
This can be achieved by the command
|\numberwithin{page}{|\textit{child}|}|
of the \textsf{amsmath} package
where \textit{child} can be |chapter| or |section|
depending on the chosen structuring.
Alternatively, one can modify the macro |\thepage| appropriately
and reset the counter |page| at the start of each child file.

%%%%%%%%%%%%%%%%%%%%%%%%%%%%%%%%%%%%%%%%%%%%%%%%%%%%%%%%%%%%%%%%%%%%%%%%%%%%%%%%
\subsection{Conditional Processing}
\label{sec:conditional}

The package provides a mechanism to compile different versions
of a document. To customise the versions further some conditional processing
can come in handy to distinguish which version is being compiled.
The package provides two macros to describe the compilation context:

%%%%%%%%%%%%%%%%%%%%%%%%%%%%%%%%%%%%%%%%
\DescribeMacro{\ifchilddoc}
The conditional |\ifchilddoc| distinguishes between the compilation of
child documents and the main document:
%
\begin{center}
|\ifchilddoc |\textit{child-code}| |[|\||else |\textit{main-code}]| \||fi|
\end{center}

%%%%%%%%%%%%%%%%%%%%%%%%%%%%%%%%%%%%%%%%
\DescribeMacro{\childdocname}
\DescribeMacro{\childdocjob}
The macro |\childdocname| contains the filename (without extension)
of the main or child file being processed.
Note that |\childdocjob| will always contain the name of the main file.

%%%%%%%%%%%%%%%%%%%%%%%%%%%%%%%%%%%%%%%%
\paragraph{Title Page.}

Conditional processing can be used to include a title or banner page
in the main document when proper precautions are taken.
Importantly, the code in the main file should ensure that the page counter
(as well as other status parameters which are stored in the |.aux| files)
takes the same value after the conditional processing.
Otherwise the page numbers may take divergent values
depending on which part is compiled.

For example, a title page could be declared by:
%
\begin{center}
\begin{tabular}{l}
|\ifchilddoc\||else|\\
|\addtocounter{page}{-1}|\\
\textit{code for title page}\\
|\newpage|\\
|\||fi|
\end{tabular}
\end{center}
%
A banner page for the child documents can be generated by:
%
\begin{center}
\begin{tabular}{l}
|\ifchilddoc|\\
|\addtocounter{page}{-1}|\\
\textit{code for banner page}\\
|\newpage|\\
|\||fi|
\end{tabular}
\end{center}
%
Here one could write a message such as:
\begin{center}
|This is the part \childdocname{} of \childdocjob{}.|
\end{center}

%%%%%%%%%%%%%%%%%%%%%%%%%%%%%%%%%%%%%%%%%%%%%%%%%%%%%%%%%%%%%%%%%%%%%%%%%%%%%%%%
\subsection{Flags}
\label{sec:flags}

The package makes it easy to generate different versions
of the main or child documents.
To this end compilation flags can be defined
and assigned different default values.
They will be particularly useful in conjunction
with the forwarding mechanism described in \secref{sec:forward}.

For example, it may be useful to have a flag |\version|
which can be set to |draft| or |final|.
The document source will contain some conditional code
depending on the value of |\version|.
Suppose further, the flag should default to |final| for the main file
and to |draft| for child files
which is a natural assignment for editing the document.
This is achieved by placing the following code
in the preamble of the main document
(below the |\childdocmain| directive):
%
\begin{center}
\begin{tabular}{l}
|\ifchilddoc|\\
|\providecommand{\version}{draft}|\\
|\||else|\\
|\providecommand{\version}{final}|\\
|\||fi|
\end{tabular}
\end{center}
%
The definition by |\providecommand| makes sure
that previous definitions are not overwritten.
Further statements |\providecommand{\version}{...}|
can thus be added before the above code to override it.

For the main file, one might add a line
(between |\childdocmain| and the above block)
%
\begin{center}
|%\ifchilddoc\||else\providecommand{\version}{draft}\||fi|
\end{center}
%
which can be uncommented to produce a draft version.
Likewise one can add a line to the very top of a child file
(above the |\childdocof{|\textit{main}|}| directive)
%
\begin{center}
|%\providecommand{\version}{final}|
\end{center}
%
which can be uncommented to produce the final version of this child document.

%%%%%%%%%%%%%%%%%%%%%%%%%%%%%%%%%%%%%%%%%%%%%%%%%%%%%%%%%%%%%%%%%%%%%%%%%%%%%%%%
\subsection{Forwarding}
\label{sec:forward}

Different versions of the main or child documents
using compilation flags as described in \secref{sec:flags}
can be (permanently) stored in different files
for convenient compilation, viewing and distribution.
To this end, the package defines a command
to pass on compilation to a different file:

%%%%%%%%%%%%%%%%%%%%%%%%%%%%%%%%%%%%%%%%
\DescribeMacro{\childdocforward}
The command |\childdocforward| redirects processing to
another source file:
%
\begin{center}
\begin{tabular}{l}
|\input{childdoc.def}|\\
|\childdocforward[|\textit{main}|]{|\textit{dest}|}|\\
\end{tabular}
\end{center}
%
The argument \textit{dest} is the destination file
(without extension).
It should be the main file or one of the child files.
Note that further \textsf{childdoc} directives
such as |\childdocof| and |\childdocforward|
in the indicated file will be processed in this form.
The optional argument \textit{main}
passes on directly to the main file \textit{main}
while pretending to compile the child \textit{dest}.
This form behaves as if \textit{dest}
issues |\childdocof{|\textit{main}|}| right away,
and no further \textsf{childdoc} directives will be processed.

%%%%%%%%%%%%%%%%%%%%%%%%%%%%%%%%%%%%%%%%
\DescribeMacro{\...prefix}
In the alternative form |\childdocforwardprefix|,
%
\begin{center}
\begin{tabular}{l}
|\input{childdoc.def}|\\
|\childdocforwardprefix[|\textit{main}|]{|\textit{prefix}|}{|\textit{dest}|}|
\end{tabular}
\end{center}
%
the destination file is determined by a pattern
depending on the current file:
To make this work, the current file must be called
`{\textit{prefix}\hspace{0.2em}\textit{suffix}}'
with \textit{prefix} matching precisely the argument.
Processing is then passed on to the file
`{\textit{dest}\hspace{0.2em}\textit{suffix}}'.
Surely, the same effect is achieved by
directly specifying the
argument `{\textit{dest}\hspace{0.2em}\textit{suffix}}'
in the first form.
However, that requires to set up a different file
for each child. With the alternative form of the command
all these files can have exactly the same content
which simplifies setting them up and maintaining them.

For example, the following file |draft.tex|
with a compilation flag |\version| as described in \secref{sec:flags}
compiles the main document as a draft:
%
\begin{center}
\begin{tabular}{l}
|\def\version{draft}|\\
|\input{childdoc.def}|\\
|\childdocforward{|\textit{main}|}|
\end{tabular}
\end{center}
%
Likewise, the following files |final|\textit{nn}|.tex|
compile the final version of the child document
|child|\textit{nn}|.tex|:
%
\begin{center}
\begin{tabular}{l}
|\def\version{final}|\\
|\input{childdoc.def}|\\
|\childdocforwardprefix{final}{child}|
\end{tabular}
\end{center}
%

Note that when several versions of a main file and/or of each child file
are to be generated, it may be convenient to set up a |Makefile| or
shell script to automatise the process.

%%%%%%%%%%%%%%%%%%%%%%%%%%%%%%%%%%%%%%%%%%%%%%%%%%%%%%%%%%%%%%%%%%%%%%%%%%%%%%%%
\subsection{Command Line Processing}
\label{sec:commandline}

The effect of redirection files can also be achieved by invoking
the \LaTeX{} compiler with a more elaborate command line.
Most conveniently this should be done as part
of a shell script or a |Makefile|.

When using \textsf{childdoc} in the main file, the following
command lines effectively perform a redirection
(note that depending on the shell being used,
backslashes may have to be doubled: `|\|' $\to$ `|\\|'):
%
\begin{center}
|... -jobname "|\textit{target}|" |\\|"|[\textit{flags}]%
|\input{childdoc.def}\childdocforward[|\textit{main}|]{|\textit{dest}|}"|
\end{center}
%
Here \textit{target} is the name of the output file,
\textit{main} is the name of the main file
and \textit{dest} is the name of the main or child file to be processed
(all filenames without extensions).
The optional argument \textit{main} can be omitted
if \textit{main} matches \textit{dest}.
Optionally, compilation \textit{flags} can be defined via |\def| commands.
This command line makes the \TeX{} engine believe
it is compiling the file \textit{target}
whose content is specified as the latter parameter.
The provided code then forwards the processing to
\textit{main} or \textit{dest} as described in \secref{sec:forward}.

%%%%%%%%%%%%%%%%%%%%%%%%%%%%%%%%%%%%%%%%%%%%%%%%%%%%%%%%%%%%%%%%%%%%%%%%%%%%%%%%
\subsection{Include by Input}
\label{sec:input}

Including child documents by |\include| has some restrictions by design.
Most notably, the content of a child document always occupies
its own set of pages; pages cannot be shared between child documents.
Usually, this behaviour makes perfect sense
because each child document contain an essential part of the document.
However, in some situations it may be desirable to compose
a document from a collection of parts
without having mandatory page breaks between then.
For this case, the package
provides a mechanism to include parts
by |\input| which can also be processed individually.
However, by construction this mechanism
requires manual handling of the content to be output.

%%%%%%%%%%%%%%%%%%%%%%%%%%%%%%%%%%%%%%%%
\DescribeMacro{\ifchilddocmanual}
The main file should be prepared as usual, see \secref{sec:include}.
However, the document body must make a distinction
between processing of an individual part and of the main document, e.g.:
%
\begin{center}
\begin{tabular}{l}
|\ifchilddocmanual|\\
|\input{\childdocname}|\\
|\||else|\\
\textit{document body with }|\input{|\textit{part}|}|\\
|\||fi|
\end{tabular}
\end{center}
%
The conditional |\ifchilddocmanual| is true whenever
a part to be included by |\input| is being compiled,
and the name of the part is stored in |\childdocname|.

%%%%%%%%%%%%%%%%%%%%%%%%%%%%%%%%%%%%%%%%
\DescribeMacro{\childdocby}
Each part to be included by |\input| should start with:
%
\begin{center}
\begin{tabular}{l}
|\input{childdoc.def}|\\
|\childdocby{|\textit{main}|}|\\
\end{tabular}
\end{center}
%
The directive |\childdocby| is similar to |\childdocof|
described in \secref{sec:include},
but the subsequent selection of content must be done manually.
To that end, both |\ifchilddoc| and |\ifchilddocmanual|
will be true upon processing of a part,
and the name of the part is stored in |\childdocname|.
Note that |\jobname| will be set to the filename of the current part
so that each part receives an individual |.aux| file
that does not interfere with the |.aux| file(s) of the main document.
This behaviour can be altered by the alternative form
|\childdocby[*]{|\textit{main}|}| (with a non-empty optional argument)
which uses the |.aux| file of the main document
by setting |\jobname| to \textit{main}.

%%%%%%%%%%%%%%%%%%%%%%%%%%%%%%%%%%%%%%%%%%%%%%%%%%%%%%%%%%%%%%%%%%%%%%%%%%%%%%%%
\subsection{Driver Development}
\label{sec:driver}

The \textsf{childdoc} mechanism can also be use for the development
of definition files such as \LaTeX{} styles or classes.
This case differs from the above setup with multiple parts
included by |\include| in that no |\includeonly| should be invoked.
This can be achieved by starting the include file
(before |\ProvidesPackage|) with:
%
\begin{center}
\begin{tabular}{l}
|\input{childdoc.def}|\\
|\childdocforward{|\textit{main}|}|\\
\end{tabular}
\end{center}
%
or alternatively with:
%
\begin{center}
\begin{tabular}{l}
|\input{childdoc.def}|\\
|\childdocby{|\textit{main}|}|\\
\end{tabular}
\end{center}
%
Both forms have slightly different effects as described above.
The main file is prepared as usual, see \secref{sec:include}.

%%%%%%%%%%%%%%%%%%%%%%%%%%%%%%%%%%%%%%%%%%%%%%%%%%%%%%%%%%%%%%%%%%%%%%%%%%%%%%%%
\subsection{Legacy Detection}
\label{sec:detection}

The directive |\childdocmain| in the main file can detect
whether the complete document or merely a child is to be compiled
even without using the directive |\childdocof|.
This method is deprecated because it is less robust
and there is no compelling reason to use it;
it is merely provided for backward compatibility
and it may be removed in future versions.

If the detection mechanism is to be used,
it is mandatory to correctly specify
the filename of the main file as the argument of |\childdocmain|:
%
\begin{center}
\begin{tabular}{l}
|\input{childdoc.def}|\\
|\childdocmain{|\textit{main}|}|\\
\end{tabular}
\end{center}
%
If |\jobname| does not match the argument \textit{main} of |\childdocmain|,
it is assumed that |\jobname| points to the child file to be compiled.
When using |\childdocmain| with the main file specified as argument,
it suffices to start a child file
with just |\input{|\textit{main}|}|
without loading of the package and using |\childdocof|.
If instead all processing is done
with the appropriate \textsf{childdoc} directives,
the argument of \textit{main} of |\childdocmain| can be empty.

An alternative version of the command line processing described
in \secref{sec:commandline} using the detection mechanism reads:
%
\begin{center}
|... -jobname "|\textit{target}|" "|[\textit{flags}]%
[|\def\jobname{|\textit{dest}|}|]|\input{|\textit{main}|}"|
\end{center}

%%%%%%%%%%%%%%%%%%%%%%%%%%%%%%%%%%%%%%%%%%%%%%%%%%%%%%%%%%%%%%%%%%%%%%%%%%%%%%%%
\subsection{Manual Code}
\label{sec:manual}

In case one cannot be certain whether the definitions file |childdoc.def|
is installed on the target \TeX{} distribution
and one prefers not to ship it,
it is conceivable to paste a few relevant commands into the sources.

To that end, drop all statements |\input{childdoc.def}|
and perform the replacements as outlined below.
Instead of |\childdocmain{|\textit{main}|}| add the following code
to the top of the main file:
%
\begin{center}
\begin{tabular}{l}
|\||ifdefined\childdocname\endinput\||fi\newif\ifchilddoc|\\
|\edef\childdocname{\scantokens\expandafter{\jobname\noexpand}}|\\
|\def\childdocmain{|\textit{main}|}\||ifx\childdocmain\childdocname\||else|\\
|\childdoctrue\includeonly{\childdocname}\let\jobname\childdocmain\||fi|\\
\end{tabular}
\end{center}
%
Instead of |\childdocof{|\textit{main}|}| just include the main file
at the top of each child file:
%
\begin{center}
|\input{|\textit{main}|}|
\end{center}
%
A simple redirection |\childdocforward{|\textit{dest}|}| is achieved by:
%
\begin{center}
|\def\jobname{|\textit{dest}|}\input{\jobname}|
\end{center}
%
The redirection with prefix
|\childdocforwardprefix[|\textit{prefix}|]{|\textit{dest}|}|
is accomplished by:
%
\begin{center}
\begin{tabular}{l}
|{\edef\jobname{\scantokens\expandafter{\jobname\noexpand}}|\\
|\def\redirectjob |\textit{prefix}|#1~~~{\gdef\jobname{|\textit{dest}|#1}}|\\
|\expandafter\redirectjob\jobname~~~}\input{\jobname}|
\end{tabular}
\end{center}

In an alternative approach,
child documents can be compiled by a specific command line
without additional code or specific definitions:
%
\begin{center}
|... -jobname "|\textit{target}|" "|[\textit{flags}]%
|\includeonly{|\textit{dest}|}\input{|\textit{main}|}"|
\end{center}
%

%%%%%%%%%%%%%%%%%%%%%%%%%%%%%%%%%%%%%%%%%%%%%%%%%%%%%%%%%%%%%%%%%%%%%%%%%%%%%%%%
%%%%%%%%%%%%%%%%%%%%%%%%%%%%%%%%%%%%%%%%%%%%%%%%%%%%%%%%%%%%%%%%%%%%%%%%%%%%%%%%
\section{Information}

%%%%%%%%%%%%%%%%%%%%%%%%%%%%%%%%%%%%%%%%%%%%%%%%%%%%%%%%%%%%%%%%%%%%%%%%%%%%%%%%
\subsection{Copyright}

Copyright \copyright{} 2017--2018 Niklas Beisert

This work may be distributed and/or modified under the
conditions of the \LaTeX{} Project Public License, either version 1.3
of this license or (at your option) any later version.
The latest version of this license is in
  \url{http://www.latex-project.org/lppl.txt}
and version 1.3 or later is part of all distributions of \LaTeX{}
version 2005/12/01 or later.

This work has the LPPL maintenance status `maintained'.

The Current Maintainer of this work is Niklas Beisert.

This work consists of the files |README.txt|, |childdoc.ins| and |childdoc.dtx|
as well as the derived files |childdoc.def|, |cdocsamp.tex|
with |cdocsch1.tex|, |cdocsch2.tex|, |cdocspt3.tex|, |cdocspt4.tex|,
|cdocsdrf.tex|, |cdocsfn1.tex|, |cdocsfn2.tex|
as well as |childdoc.pdf|.

%%%%%%%%%%%%%%%%%%%%%%%%%%%%%%%%%%%%%%%%%%%%%%%%%%%%%%%%%%%%%%%%%%%%%%%%%%%%%%%%
\subsection{Files and Installation}

The package consists of the files:
%
\begin{center}
\begin{tabular}{ll}
    |README.txt|   & readme file \\
    |childdoc.ins| & installation file \\
    |childdoc.dtx| & source file \\
    |childdoc.def| & definition file \\
    |cdocsamp.tex| & sample main file \\
    |cdocsch1.tex| & sample include file \\
    |cdocsch2.tex| & sample include file \\
    |cdocspt3.tex| & sample part file \\
    |cdocspt4.tex| & sample part file \\
    |cdocsdrf.tex| & sample redirection file \\
    |cdocsfn1.tex| & sample redirection file \\
    |cdocsfn2.tex| & sample redirection file \\
    |childdoc.pdf| & manual
\end{tabular}
\end{center}
%
The distribution consists of the files
|README.txt|, |childdoc.ins| and |childdoc.dtx|.
%
\begin{itemize}
\item
Run (pdf)\LaTeX{} on |childdoc.dtx|
to compile the manual |childdoc.pdf| (this file).
\item
Run \LaTeX{} on |childdoc.ins| to create the definitions file |childdoc.def|
and the sample |cdocsamp.tex| with include files
|cdocsch1.tex|, |cdocsch2.tex|, |cdocspt3.tex|, |cdocspt4.tex|,
|cdocsdrf.tex|, |cdocsfn1.tex|, |cdocsfn2.tex|.
Then copy the file |childdoc.def| to an appropriate directory of your \LaTeX{}
distribution, e.g.\ \textit{texmf-root}|/tex/latex/childdoc|.
\end{itemize}

%%%%%%%%%%%%%%%%%%%%%%%%%%%%%%%%%%%%%%%%%%%%%%%%%%%%%%%%%%%%%%%%%%%%%%%%%%%%%%%%
\subsection{Related CTAN Packages}

There are several other packages which offer a similar functionality:
%
\begin{itemize}
\item
The packages
\href{http://ctan.org/pkg/docmute}{\textsf{docmute}},
\href{http://ctan.org/pkg/includex}{\textsf{includex}} and
\href{http://ctan.org/pkg/standalone}{\textsf{standalone}}
provide commands to include only the document body of
a child file thus allowing both files to be compiled individually.
\item
The packages \href{http://ctan.org/pkg/subdocs}{\textsf{subdocs}}
and \href{http://ctan.org/pkg/subfiles}{\textsf{subfiles}}
provide structures in which the main and child documents can be
encapsulated and allowing them to be compiled individually.
The inclusion mechanism is different from the conventional |\include|.
\item
The package \href{http://ctan.org/pkg/combine}{\textsf{combine}}
is an elaborate solution to combine several documents into one.
\end{itemize}
%
See also the CTAN topic \href{http://ctan.org/topic/subdocs}{\textsf{subdocs}}
for further related packages.
The present package differs from the above solutions in that
a document structure constructed with the conventional |\include| mechanism
just needs two extra commands at the top of every file
such that all constituent files can be compiled individually.

%%%%%%%%%%%%%%%%%%%%%%%%%%%%%%%%%%%%%%%%%%%%%%%%%%%%%%%%%%%%%%%%%%%%%%%%%%%%%%%%
%\subsection{Feature Suggestions}
%
%The following is a list of features which may be useful for future
%versions of this package:
%%
%\begin{itemize}
%\item
%\ldots
%\end{itemize}

%%%%%%%%%%%%%%%%%%%%%%%%%%%%%%%%%%%%%%%%%%%%%%%%%%%%%%%%%%%%%%%%%%%%%%%%%%%%%%%%
\subsection{Revision History}

%%%%%%%%%%%%%%%%%%%%%%%%%%%%%%%%%%%%%%%%
\paragraph{v2.0:} 2018/12/30

\begin{itemize}
\item
immediate forward processing
\item
added |\childdocby| mechanism
\item
manual restructured
\end{itemize}

%%%%%%%%%%%%%%%%%%%%%%%%%%%%%%%%%%%%%%%%
\paragraph{v1.6:} 2018/01/17

\begin{itemize}
\item
application for development of include files
\item
corrections to manual
\end{itemize}

%%%%%%%%%%%%%%%%%%%%%%%%%%%%%%%%%%%%%%%%
\paragraph{v1.5:} 2017/05/21

\begin{itemize}
\item
more complete structuring introduced
\item
|\childdocof| introduced
\item
|\childdoc| renamed to |\childdocmain|
\item
|\childredirect| renamed to |\childdocforward| and |\childdocforwardprefix|
and functionality expanded
\end{itemize}

%%%%%%%%%%%%%%%%%%%%%%%%%%%%%%%%%%%%%%%%
\paragraph{v1.0:} 2017/04/27

\begin{itemize}
\item
manual and install package
\item
first version published on CTAN
\end{itemize}

%%%%%%%%%%%%%%%%%%%%%%%%%%%%%%%%%%%%%%%%
\paragraph{v0.6:} 2017/04/26

\begin{itemize}
\item
redirection mechanism added
\end{itemize}

%%%%%%%%%%%%%%%%%%%%%%%%%%%%%%%%%%%%%%%%
\paragraph{v0.5:} 2017/04/26

\begin{itemize}
\item
functionality in definition file
\end{itemize}


%%%%%%%%%%%%%%%%%%%%%%%%%%%%%%%%%%%%%%%%%%%%%%%%%%%%%%%%%%%%%%%%%%%%%%%%%%%%%%%%
%%%%%%%%%%%%%%%%%%%%%%%%%%%%%%%%%%%%%%%%%%%%%%%%%%%%%%%%%%%%%%%%%%%%%%%%%%%%%%%%
%%%%%%%%%%%%%%%%%%%%%%%%%%%%%%%%%%%%%%%%%%%%%%%%%%%%%%%%%%%%%%%%%%%%%%%%%%%%%%%%
\appendix

\settowidth\MacroIndent{\rmfamily\scriptsize 000\ }

 \DocInput{childdoc.dtx}

\end{document}
%</driver>
% \fi
%
% %%%%%%%%%%%%%%%%%%%%%%%%%%%%%%%%%%%%%%%%%%%%%%%%%%%%%%%%%%%%%%%%%%%%%%%%%%%%%%
% %%%%%%%%%%%%%%%%%%%%%%%%%%%%%%%%%%%%%%%%%%%%%%%%%%%%%%%%%%%%%%%%%%%%%%%%%%%%%%
% \section{Sample}
%\iffalse
%<*samplemain>
%\fi
%
% The following presents a sample document
% with two chapters, two parts, a title page,
% a compile flag as well as three forwarding files to set the flag.
% It consists of eight |.tex| files:
% \begin{center}
% \begin{tabular}{ll}
% |cdocsamp.tex|&main file\\
% |cdocsch1.tex|&include file for chapter 1\\
% |cdocsch2.tex|&include file for chapter 2\\
% |cdocspt3.tex|&include file for part 3\\
% |cdocspt4.tex|&include file for part 4\\
% |cdocsdrf.tex|&forwarding file for main file in draft mode\\
% |cdocsfi1.tex|&forwarding file for final version of chapter 1\\
% |cdocsfi2.tex|&forwarding file for final version of chapter 2\\
% \end{tabular}
% \end{center}
% Each of the eight files can be compiled directly by the \LaTeX{} compiler.
%
% %%%%%%%%%%%%%%%%%%%%%%%%%%%%%%%%%%%%%%
% \paragraph{Main File.}
%
% The main file is called |cdocsamp.tex|.
%
% Load the \textsf{childdoc} definitions and
% declare the filename for the main document:
%    \begin{macrocode}
\input{childdoc.def}
\childdocmain{}
%    \end{macrocode}

% Optional override for |\version| flag:
%    \begin{macrocode}
%%\ifchilddoc\else\providecommand{\version}{draft}\fi
%    \end{macrocode}

% Define the default values for the |\version| flag
% (|final| for the main file and |draft| for childs):
%    \begin{macrocode}
\ifchilddoc
\providecommand{\version}{draft}
\else
\providecommand{\version}{final}
\fi
%    \end{macrocode}

% Load the standard document class:
%    \begin{macrocode}
\documentclass[12pt]{article}
%    \end{macrocode}

% Start the document body:
%    \begin{macrocode}
\begin{document}
%    \end{macrocode}

% Declare a title page.
% Print title, part of document being processed and version flag:
%    \begin{macrocode}
\addtocounter{page}{-1}
\begin{center}
{\LARGE\bfseries{}childdoc example\par}
\vspace{1cm}
\ifchilddoc
\ifchilddocmanual part\else chapter\fi:
`\childdocname' of `\childdocjob'\par
\else
main document: `\childdocjob'\par
\fi
version: \version\par
\end{center}
\newpage
%    \end{macrocode}

% Manually include selected file,
% otherwise process as usual:
%    \begin{macrocode}
\ifchilddocmanual
\section*{part `\childdocname'}
\input{\childdocname}
\else
%    \end{macrocode}

% Include the two chapters:
%    \begin{macrocode}
\include{cdocsch1}
\include{cdocsch2}
%    \end{macrocode}

% Include the two parts unless only chapters should be displayed:
%    \begin{macrocode}
\ifchilddoc\else
\section{part three}
\input{cdocspt3}
\section{part four}
\input{cdocspt4}
\fi
%    \end{macrocode}

% Process as usual until here:
%    \begin{macrocode}
\fi
%    \end{macrocode}

% End of document body:
%    \begin{macrocode}
\end{document}
%    \end{macrocode}
%\iffalse
%</samplemain>
%\fi
%
% %%%%%%%%%%%%%%%%%%%%%%%%%%%%%%%%%%%%%%
% \paragraph{Chapter Include Files.}
%
% The include files are called |cdocsch1.tex| and |cdocsch2.tex|.
%
%\iffalse
%<*samplechap1|samplechap2>
%\fi

% Optional override for |\version| flag:
%    \begin{macrocode}
%%\providecommand{\version}{final}
%    \end{macrocode}

% Include the main document:
%    \begin{macrocode}
\input{childdoc.def}
\childdocof{cdocsamp}
%    \end{macrocode}

%\iffalse
%</samplechap1|samplechap2>
%\fi
%
%\iffalse
%<*samplechap1>
%\fi
% Some text for chapter 1:
%    \begin{macrocode}
\section{one}
some text in chapter one
%    \end{macrocode}

%\iffalse
%</samplechap1>
%\fi
% Some text for chapter 2:
%\iffalse
%<*samplechap2>
%\fi
%    \begin{macrocode}
\section{two}
more text in chapter two
%    \end{macrocode}

%\iffalse
%</samplechap2>
%\fi
%
% %%%%%%%%%%%%%%%%%%%%%%%%%%%%%%%%%%%%%%
% \paragraph{Part Include Files.}
%
% The include files are called |cdocspt3.tex| and |cdocspt4.tex|.
%
%\iffalse
%<*samplepart3|samplepart4>
%\fi

% Optional override for |\version| flag:
%    \begin{macrocode}
%%\providecommand{\version}{final}
%    \end{macrocode}

% Include the main document:
%    \begin{macrocode}
\input{childdoc.def}
\childdocby{cdocsamp}
%    \end{macrocode}

%\iffalse
%</samplepart3|samplepart4>
%\fi
%
%\iffalse
%<*samplepart3>
%\fi
% Some text for part 3:
%    \begin{macrocode}
some text in part three
%    \end{macrocode}

%\iffalse
%</samplepart3>
%\fi
% Some text for part 4:
%\iffalse
%<*samplepart4>
%\fi
%    \begin{macrocode}
more text in part four
%    \end{macrocode}

%\iffalse
%</samplepart4>
%\fi
%
% %%%%%%%%%%%%%%%%%%%%%%%%%%%%%%%%%%%%%%
% \paragraph{Forwarding for a Complete Draft.}
%
% The following forwarding file |cdocsdrf.tex|
% compiles the main document in draft mode:
%\iffalse
%<*sampledraft>
%\fi
%    \begin{macrocode}
\def\version{draft}
\input{childdoc.def}
\childdocforward{cdocsamp}
%    \end{macrocode}

%\iffalse
%</sampledraft>
%\fi
%
% %%%%%%%%%%%%%%%%%%%%%%%%%%%%%%%%%%%%%%
% \paragraph{Forwarding for Final Version of the Chapters.}
%
% The following forwarding files |cdocsfn1.tex| and |cdocsfn2.tex|
% (with identical content)
% compile the final versions of the child documents
% |cdocsch1.tex| and |cdocsch2.tex|, respectively:
%\iffalse
%<*samplefinal>
%\fi
%    \begin{macrocode}
\def\version{final}
\input{childdoc.def}
\childdocforwardprefix[cdocsamp]{cdocsfn}{cdocsch}
%    \end{macrocode}

%\iffalse
%</samplefinal>
%\fi
%
% %%%%%%%%%%%%%%%%%%%%%%%%%%%%%%%%%%%%%%
% \paragraph{Command Line Processing.}
%
% The following three command lines generate the output files
% |cdocscld|, |cdocscl1| and |cdocscl2|
% which should be identical to
% |cdocsdrf|, |cdocsch1| and |cdocsfn2|, respectively:
% \begin{center}
% \begin{tabular}{l}
% |latex -jobname cdocscld \|\\
% |  "\def\version{draft}\input{childdoc.def}\childdocforward{cdocsamp}"|\\
% |latex -jobname cdocscl1 \|\\
% |  "\input{childdoc.def}\childdocforward[cdocsamp]{cdocsch1}"|\\
% |latex -jobname cdocscl2 \|\\
% |  "\def\version{final}\input{childdoc.def}\childdocforward{cdocsch2}"|
% \end{tabular}
% \end{center}
% Note that the trailing backslash on each first line
% merely continues the input to the second line
% (for convenient cut ant paste).
% Furthermore, the command |latex| can be replaced by any
% of its alternative versions such as |pdflatex|.
%
% %%%%%%%%%%%%%%%%%%%%%%%%%%%%%%%%%%%%%%%%%%%%%%%%%%%%%%%%%%%%%%%%%%%%%%%%%%%%%%
% %%%%%%%%%%%%%%%%%%%%%%%%%%%%%%%%%%%%%%%%%%%%%%%%%%%%%%%%%%%%%%%%%%%%%%%%%%%%%%
% \section{Implementation}
%\iffalse
%<*package>
%\fi
%
% This section describes the definitions file |childdoc.def|.

% The definitions cannot be loaded using |\usepackage| or |\RequirePackage|
% which has a mechanism to prevent loading a style file more than once.
% When loading the definitions by means of |\input|
% multiple instances have to be prevented manually:
%\iffalse
%This code needs to be before the `\ProvidesFile' directive
%which is defined at the beginning of this file.
%Therefore it is also placed there and commented out here.
%</package>
%<*discard>
%\fi
%    \begin{macrocode}
\ifdefined\childdocmain\endinput\fi
%    \end{macrocode}
%\iffalse
%</discard>
%<*package>
%\fi
%
% \macro{\ifchilddoc}
% \macro{\ifchilddocmanual}
% The conditional |\ifchilddoc| tells whether a
% child (true) or main (false) document is being compiled.
% The conditional |\ifchilddocmanual| tells whether
% the |\includeonly| mechanism is used (false) or
% the selection of child files must be performed manually (true).
% The definitions initialise to false:
%    \begin{macrocode}
\newif\ifchilddoc
\newif\ifchilddocmanual
%    \end{macrocode}

% \macro{\childdocname}
% \macro{\childdocjob}
% The macro |\childdocname| stores the name of the main document
% to be compiled. The macro |\childdocjob| stores the name of
% the document on which the \LaTeX{} compiler was originally invoked.
% The content of |\jobname| cannot be compared
% to filenames specified in the source due to different catcodes.
% The following code rescans |\jobname|, stores the result
% in |\childdocname| and saves a copy in |\childdocjob|:
%    \begin{macrocode}
\edef\childdocname{\scantokens\expandafter{\jobname\noexpand}}
\let\childdocjob\childdocname
%    \end{macrocode}

% \macro{\childdocdisable}
% The macro |\childdocdisable| prevents the main file
% from being processed more than once.
% At this stage, the main document command |\childdocmain|
% is assumed to be called once again where it should do nothing.
% Any subsequent call to it should prevent
% a secondary processing of the main document
% It overwrites the forwarding commands
% |\childdocof| and |\childdocforward|
% with empty macros to prevent further inclusions of the main document:
%    \begin{macrocode}
\newcommand{\childdocdisable}
{
  \renewcommand{\childdocmain}[1]{\renewcommand{\childdocmain}[1]{\endinput}}
  \renewcommand{\childdocof}[1]{}
  \renewcommand{\childdocby}[2][]{}
  \renewcommand{\childdocforward}[2][]{}
  \renewcommand{\childdocdisable}{}
}
%    \end{macrocode}

% \macro{\childdocmain}
% The macro |\childdocmain| is to be called at the top of the main file
% with nothing or the main filename (without extension) as argument.
% First, it breaks loops.
% If the argument is not empty and does not match |\childdocname|
% (which is set by the first inclusion of |childdoc.def|),
% |\ifchilddoc| is set to true, |\includeonly| is applied to the child file
% and |\jobname| is set to the main file
% (for proper handling of |.aux| files):
%    \begin{macrocode}
\newcommand{\childdocmain}[1]
{
  \childdocdisable\childdocmain{}
  \if?#1?\else
    \begingroup
      \def\childdoctmp{#1}
      \ifx\childdoctmp\childdocname
        \def\childdoctmp{}
      \else
        \def\childdoctmp
        {
          \childdoctrue
          \includeonly{\childdocname}
          \def\childdocjob{#1}
          \def\jobname{#1}
        }
      \fi
      \expandafter
    \endgroup
    \childdoctmp
  \fi
}
%    \end{macrocode}

% \macro{\childdocof}
% The command |\childdocof| redirects
% compilation to the main file |#1|.
%    \begin{macrocode}
\newcommand{\childdocof}[1]
{
  \childdocdisable
  \childdoctrue
  \includeonly{\childdocname}
  \def\jobname{#1}
  \def\childdocjob{#1}
  \input{#1}
}
%    \end{macrocode}

% \macro{\childdocby}
% The command |\childdocby| ....
%    \begin{macrocode}
\newcommand{\childdocby}[2][]
{
  \childdocdisable
  \childdoctrue
  \childdocmanualtrue
  \if?#1?\else
    \def\jobname{#2}
  \fi
  \def\childdocjob{#2}
  \input{#2}
  \endinput
}
%    \end{macrocode}

% \macro{\childdocforward}
% The command |\childdocforward| redirects
% compilation to the main file or
% (if the optional argument is given) a child file.
% Parameters are set as if the main file
% or a child file starting with |\childdocof| was compiled.
% Then compilation is handed over to the main file:
%    \begin{macrocode}
\newcommand{\childdocforward}[2][]
{
  \begingroup
    \if?#1?
      \def\childdoctmp
      {
        \def\childdocname{#2}
        \def\childdocjob{#2}
        \def\jobname{#2}
        \input{#2}
        \endinput
      }
    \else
      \def\childdoctmp
      {
        \childdocdisable
        \def\childdocname{#2}
        \childdoctrue
        \includeonly{#2}
        \def\childdocjob{#1}
        \def\jobname{#1}
        \input{#1}
        \endinput
      }
    \fi
    \expandafter
  \endgroup
  \childdoctmp
}
%    \end{macrocode}

% \macro{\childdocforwardprefix}
% The command |\childdocforwardprefix| redirects
% compilation to the main or a child file by means of a pattern.
% The prefix |#1| in the current filename is replaced by |#2|
% and the suffix of the current filename is kept
% (it is assumed that the filename does not contain the substring `|~~~|'
% which is used as a delimiter).
% Compilation is handed over to the new file by |\childdocforward|:
%    \begin{macrocode}
\newcommand{\childdocforwardprefix}[3][]
{
  \begingroup
    \def\childdocextract #2##1~~~{\def\childdoctmp{\childdocforward[#1]{#3##1}}}
    \expandafter\childdocextract\childdocname~~~
    \expandafter
  \endgroup
  \childdoctmp
}
%    \end{macrocode}

% \macro{\childdoc}
% The deprecated macro |\childdoc| is a legacy version of |\childdocmain|:
%    \begin{macrocode}
\newcommand{\childdoc}{\childdocmain}
%    \end{macrocode}

% \macro{\childdocredirect}
% The deprecated macro |\childdocredirect| is a legacy version
% of |\childdocforward| and |\childdocforwardprefix|:
%    \begin{macrocode}
\newcommand{\childdocredirect}[2][]
{
  \begingroup
    \if?#1?
      \def\childdoctmp{\childdocforward{#2}}
    \else
      \def\childdoctmp{\childdocforwardprefix{#1}{#2}}
    \fi
    \expandafter
  \endgroup
  \childdoctmp
}
%    \end{macrocode}

%\iffalse
%</package>
%\fi
%
\endinput
|\\
|\childdocof{|\textit{main}|}|\\
\end{tabular}
\end{center}
at the top of every child file \textit{child}
which is included by |\include{|\textit{child}|}|
from within the main file
(or at least for those files to be compiled individually).
The argument \textit{main} must be the filename of the main file.

There are a couple of
considerations in setting up the main and child documents:

%%%%%%%%%%%%%%%%%%%%%%%%%%%%%%%%%%%%%%%%
\paragraph{Restrictions.}

Please note the following restrictions:
\begin{itemize}
\item
|\childdocmain| must be called with one argument \textit{main}
to ensure compatibility with earlier version of the package.
It must either be empty (|\childdocmain{}|)
or precisely match the filename of the main file in which it is specified.
See \secref{sec:detection} for further information.
\item
The filename \textit{main} must be specified without the |.tex| extension.
\item
The filename \textit{main} is case sensitive
(even in case-insensitive file systems)
due to internal string comparison.
\item
The argument \textit{main} should be fully expanded, it cannot be a macro.
\item
Subdirectories and special characters should be avoided in filenames.
\item
The command |\childdocmain{|\textit{main}|}| must be followed by a whitespace.
It should not be followed immediately by another command
or by a comment mark `|%|'.
This is because the \TeX{} parser reads the token immediately following
the argument of |\childdocmain| and puts it
at the beginning of every child section;
however, a white\-space is ignored.
\end{itemize}

%%%%%%%%%%%%%%%%%%%%%%%%%%%%%%%%%%%%%%%%
\paragraph{Content of Main File.}

It is advisable to place all content in the child files included by |\include|.
Any output contained in the main file will appear in all child documents
unless suppressed manually;
it cannot be suppressed automatically by the |\includeonly| directive
and thus should normally be avoided.
A method to include some content in the main file
by means of conditional processing is described in \secref{sec:conditional}.

%%%%%%%%%%%%%%%%%%%%%%%%%%%%%%%%%%%%%%%%
\paragraph{Page Numbering.}

When only a part of the document is compiled,
the appropriate numbering of pages
(as well as other status parameters)
is determined from the |.aux| files.
The latter contain information from previous passes.
However this information needs to propagate through
all intermediate child documents.
Therefore the page numbering in child documents may well
be inconsistent until the complete document is compiled at least once.

A useful (if unconventional) way to always ensure a consistent
page numbering is to restart the numbering in each child document
and denote the pages by `\textit{child}|.|\textit{page}'
where \textit{child} represents the chapter/section number of the child file.
This can be achieved by the command
|\numberwithin{page}{|\textit{child}|}|
of the \textsf{amsmath} package
where \textit{child} can be |chapter| or |section|
depending on the chosen structuring.
Alternatively, one can modify the macro |\thepage| appropriately
and reset the counter |page| at the start of each child file.

%%%%%%%%%%%%%%%%%%%%%%%%%%%%%%%%%%%%%%%%%%%%%%%%%%%%%%%%%%%%%%%%%%%%%%%%%%%%%%%%
\subsection{Conditional Processing}
\label{sec:conditional}

The package provides a mechanism to compile different versions
of a document. To customise the versions further some conditional processing
can come in handy to distinguish which version is being compiled.
The package provides two macros to describe the compilation context:

%%%%%%%%%%%%%%%%%%%%%%%%%%%%%%%%%%%%%%%%
\DescribeMacro{\ifchilddoc}
The conditional |\ifchilddoc| distinguishes between the compilation of
child documents and the main document:
%
\begin{center}
|\ifchilddoc |\textit{child-code}| |[|\||else |\textit{main-code}]| \||fi|
\end{center}

%%%%%%%%%%%%%%%%%%%%%%%%%%%%%%%%%%%%%%%%
\DescribeMacro{\childdocname}
\DescribeMacro{\childdocjob}
The macro |\childdocname| contains the filename (without extension)
of the main or child file being processed.
Note that |\childdocjob| will always contain the name of the main file.

%%%%%%%%%%%%%%%%%%%%%%%%%%%%%%%%%%%%%%%%
\paragraph{Title Page.}

Conditional processing can be used to include a title or banner page
in the main document when proper precautions are taken.
Importantly, the code in the main file should ensure that the page counter
(as well as other status parameters which are stored in the |.aux| files)
takes the same value after the conditional processing.
Otherwise the page numbers may take divergent values
depending on which part is compiled.

For example, a title page could be declared by:
%
\begin{center}
\begin{tabular}{l}
|\ifchilddoc\||else|\\
|\addtocounter{page}{-1}|\\
\textit{code for title page}\\
|\newpage|\\
|\||fi|
\end{tabular}
\end{center}
%
A banner page for the child documents can be generated by:
%
\begin{center}
\begin{tabular}{l}
|\ifchilddoc|\\
|\addtocounter{page}{-1}|\\
\textit{code for banner page}\\
|\newpage|\\
|\||fi|
\end{tabular}
\end{center}
%
Here one could write a message such as:
\begin{center}
|This is the part \childdocname{} of \childdocjob{}.|
\end{center}

%%%%%%%%%%%%%%%%%%%%%%%%%%%%%%%%%%%%%%%%%%%%%%%%%%%%%%%%%%%%%%%%%%%%%%%%%%%%%%%%
\subsection{Flags}
\label{sec:flags}

The package makes it easy to generate different versions
of the main or child documents.
To this end compilation flags can be defined
and assigned different default values.
They will be particularly useful in conjunction
with the forwarding mechanism described in \secref{sec:forward}.

For example, it may be useful to have a flag |\version|
which can be set to |draft| or |final|.
The document source will contain some conditional code
depending on the value of |\version|.
Suppose further, the flag should default to |final| for the main file
and to |draft| for child files
which is a natural assignment for editing the document.
This is achieved by placing the following code
in the preamble of the main document
(below the |\childdocmain| directive):
%
\begin{center}
\begin{tabular}{l}
|\ifchilddoc|\\
|\providecommand{\version}{draft}|\\
|\||else|\\
|\providecommand{\version}{final}|\\
|\||fi|
\end{tabular}
\end{center}
%
The definition by |\providecommand| makes sure
that previous definitions are not overwritten.
Further statements |\providecommand{\version}{...}|
can thus be added before the above code to override it.

For the main file, one might add a line
(between |\childdocmain| and the above block)
%
\begin{center}
|%\ifchilddoc\||else\providecommand{\version}{draft}\||fi|
\end{center}
%
which can be uncommented to produce a draft version.
Likewise one can add a line to the very top of a child file
(above the |\childdocof{|\textit{main}|}| directive)
%
\begin{center}
|%\providecommand{\version}{final}|
\end{center}
%
which can be uncommented to produce the final version of this child document.

%%%%%%%%%%%%%%%%%%%%%%%%%%%%%%%%%%%%%%%%%%%%%%%%%%%%%%%%%%%%%%%%%%%%%%%%%%%%%%%%
\subsection{Forwarding}
\label{sec:forward}

Different versions of the main or child documents
using compilation flags as described in \secref{sec:flags}
can be (permanently) stored in different files
for convenient compilation, viewing and distribution.
To this end, the package defines a command
to pass on compilation to a different file:

%%%%%%%%%%%%%%%%%%%%%%%%%%%%%%%%%%%%%%%%
\DescribeMacro{\childdocforward}
The command |\childdocforward| redirects processing to
another source file:
%
\begin{center}
\begin{tabular}{l}
|% \iffalse
%
% childdoc.dtx Copyright (C) 2017-2018 Niklas Beisert
%
% This work may be distributed and/or modified under the
% conditions of the LaTeX Project Public License, either version 1.3
% of this license or (at your option) any later version.
% The latest version of this license is in
%   http://www.latex-project.org/lppl.txt
% and version 1.3 or later is part of all distributions of LaTeX
% version 2005/12/01 or later.
%
% This work has the LPPL maintenance status `maintained'.
%
% The Current Maintainer of this work is Niklas Beisert.
%
% This work consists of the files childdoc.dtx and childdoc.ins
% and the derived files childdoc.def and cdocsamp.tex with
% cdocsch1.tex, cdocsch2.tex, cdocsdrf.tex, cdocsfn1.tex, cdocsfn2.tex.
%
%<package>\ifdefined\childdocmain\endinput\fi
%<package>\ProvidesFile{childdoc.def}[2018/12/30 v2.0 child document driver]
%<samplemain>\ProvidesFile{cdocsamp.tex}[2018/12/30 v2.0 sample for childdoc]
%<*driver>
%\ProvidesFile{childdoc.drv}[2018/12/30 v2.0 childdoc reference manual file]
\PassOptionsToClass{10pt,a4paper}{article}
\documentclass{ltxdoc}

\usepackage[margin=35mm]{geometry}
\usepackage{hyperref}
\usepackage{hyperxmp}
\usepackage[usenames]{color}

\hypersetup{colorlinks=true}
\hypersetup{pdfstartview=FitH}
\hypersetup{pdfpagemode=UseNone}
\hypersetup{pdfsource={}}
\hypersetup{pdflang={en-UK}}
\hypersetup{pdfcopyright={Copyright 2017-2018 Niklas Beisert.
  This work may be distributed and/or modified under the
  conditions of the LaTeX Project Public License, either version 1.3
  of this license or (at your option) any later version.}}
\hypersetup{pdflicenseurl={http://www.latex-project.org/lppl.txt}}
\hypersetup{pdfcontactaddress={ETH Zurich, ITP, HIT K,
  Wolfgang-Pauli-Strasse 27}}
\hypersetup{pdfcontactpostcode={8093}}
\hypersetup{pdfcontactcity={Zurich}}
\hypersetup{pdfcontactcountry={Switzerland}}
\hypersetup{pdfcontactemail={nbeisert@itp.phys.ethz.ch}}
\hypersetup{pdfcontacturl={http://people.phys.ethz.ch/\xmptilde nbeisert/}}

\newcommand{\secref}[1]{\hyperref[#1]{section \ref*{#1}}}

\parskip1ex
\parindent0pt
\let\olditemize\itemize
\def\itemize{\olditemize\parskip0pt}

\begin{document}

\title{The \textsf{childdoc} Package}
\hypersetup{pdftitle={The childdoc Package}}
\author{Niklas Beisert\\[2ex]
  Institut f\"ur Theoretische Physik\\
  Eidgen\"ossische Technische Hochschule Z\"urich\\
  Wolfgang-Pauli-Strasse 27, 8093 Z\"urich, Switzerland\\[1ex]
  \href{mailto:nbeisert@itp.phys.ethz.ch}
  {\texttt{nbeisert@itp.phys.ethz.ch}}}
\hypersetup{pdfauthor={Niklas Beisert}}
\hypersetup{pdfsubject={Manual for the LaTeX2e Package childdoc}}
\date{30 December 2018, \textsf{v2.0}}
\maketitle

\begin{abstract}\noindent
\textsf{childdoc} is a \LaTeXe{} package
that enables the direct compilation
of document sections included by |\include|
to individual files.
\end{abstract}

\begingroup
\parskip0ex
\tableofcontents
\endgroup

%%%%%%%%%%%%%%%%%%%%%%%%%%%%%%%%%%%%%%%%%%%%%%%%%%%%%%%%%%%%%%%%%%%%%%%%%%%%%%%%
%%%%%%%%%%%%%%%%%%%%%%%%%%%%%%%%%%%%%%%%%%%%%%%%%%%%%%%%%%%%%%%%%%%%%%%%%%%%%%%%
\section{Introduction}

\LaTeX{} provides a mechanism to structure a large document (such as a book)
into a main file and several child files (containing the chapters)
using the |\include| command.
This mechanism is beneficial for documents
which span hundreds of pages in order to
make the source file(s) more manageable.
Moreover, compilation can be restricted to
selected child files by means of the |\includeonly| command.
The latter feature can be used to reduce the compilation time while editing
(this was significantly more useful in the earlier days of \LaTeX{})
or to generate a smaller document which is easier to navigate.
Another application of |\includeonly| is to generate
documents consisting of selected parts of the complete document.

However, there are a few drawbacks of the plain |\include| mechanism:
\begin{itemize}
\item
The child files cannot be compiled on their own,
they can only be compiled via the main file.
A naive editing environment
(such as a text editor with an option
to have the current file processed by \LaTeX)
may require one to switch to the main file before compiling;
attempting to compile the child file produces errors.
\item
The main file must be modified (each time)
to adjust the |\includeonly| command
to the present needs. This easily leaves the main file in a messy state.
\item
The generated document will always carry the filename
of the main document. This is inconvenient if
several child files are to be compiled and
to be kept for distribution.
\end{itemize}

The present package provides a simple interface
to make child files individually compilable by \LaTeX{}.
Compiling a child file then has the same effect as compiling
the main file with an |\includeonly| command
to select the appropriate child.
Moreover the generated document will carry the name of the child
rather than the main file.
This resolves all three above issues.

This feature is meant to make the editing of books,
thesis documents and lecture notes somewhat more convenient.
However, the package can also be used efficiently for
composing a series of documents (such as exercise sheets)
which are typically distributed individually.
It then assists the author in generating the individual documents
(potentially in different versions)
as well as a document containing the collected series.
Another application is in developing style files
or other kinds of included material
where compilation of the style file could redirect
to a sample or test file.

%%%%%%%%%%%%%%%%%%%%%%%%%%%%%%%%%%%%%%%%%%%%%%%%%%%%%%%%%%%%%%%%%%%%%%%%%%%%%%%%
%%%%%%%%%%%%%%%%%%%%%%%%%%%%%%%%%%%%%%%%%%%%%%%%%%%%%%%%%%%%%%%%%%%%%%%%%%%%%%%%
\section{Usage}

First of all, the package \textsf{childdoc} is \emph{not} a standard
\LaTeXe{} |.sty| style file! Therefore it needs to be invoked in
a non-standard way.

%%%%%%%%%%%%%%%%%%%%%%%%%%%%%%%%%%%%%%%%%%%%%%%%%%%%%%%%%%%%%%%%%%%%%%%%%%%%%%%%
\subsection{Included Files}
\label{sec:include}

%%%%%%%%%%%%%%%%%%%%%%%%%%%%%%%%%%%%%%%%
\DescribeMacro{\childdocmain}
To use the package, add the commands
\begin{center}
\begin{tabular}{l}
|\input{childdoc.def}|\\
|\childdocmain{}|\\
\end{tabular}
\end{center}
at the very top of the main \LaTeX{} file,
in particular \emph{before} the |\documentclass| statement!
The argument of |\childdocmain| should be left empty
(but it must be present).

%%%%%%%%%%%%%%%%%%%%%%%%%%%%%%%%%%%%%%%%
\DescribeMacro{\childdocof}
Furthermore, add the commands
\begin{center}
\begin{tabular}{l}
|\input{childdoc.def}|\\
|\childdocof{|\textit{main}|}|\\
\end{tabular}
\end{center}
at the top of every child file \textit{child}
which is included by |\include{|\textit{child}|}|
from within the main file
(or at least for those files to be compiled individually).
The argument \textit{main} must be the filename of the main file.

There are a couple of
considerations in setting up the main and child documents:

%%%%%%%%%%%%%%%%%%%%%%%%%%%%%%%%%%%%%%%%
\paragraph{Restrictions.}

Please note the following restrictions:
\begin{itemize}
\item
|\childdocmain| must be called with one argument \textit{main}
to ensure compatibility with earlier version of the package.
It must either be empty (|\childdocmain{}|)
or precisely match the filename of the main file in which it is specified.
See \secref{sec:detection} for further information.
\item
The filename \textit{main} must be specified without the |.tex| extension.
\item
The filename \textit{main} is case sensitive
(even in case-insensitive file systems)
due to internal string comparison.
\item
The argument \textit{main} should be fully expanded, it cannot be a macro.
\item
Subdirectories and special characters should be avoided in filenames.
\item
The command |\childdocmain{|\textit{main}|}| must be followed by a whitespace.
It should not be followed immediately by another command
or by a comment mark `|%|'.
This is because the \TeX{} parser reads the token immediately following
the argument of |\childdocmain| and puts it
at the beginning of every child section;
however, a white\-space is ignored.
\end{itemize}

%%%%%%%%%%%%%%%%%%%%%%%%%%%%%%%%%%%%%%%%
\paragraph{Content of Main File.}

It is advisable to place all content in the child files included by |\include|.
Any output contained in the main file will appear in all child documents
unless suppressed manually;
it cannot be suppressed automatically by the |\includeonly| directive
and thus should normally be avoided.
A method to include some content in the main file
by means of conditional processing is described in \secref{sec:conditional}.

%%%%%%%%%%%%%%%%%%%%%%%%%%%%%%%%%%%%%%%%
\paragraph{Page Numbering.}

When only a part of the document is compiled,
the appropriate numbering of pages
(as well as other status parameters)
is determined from the |.aux| files.
The latter contain information from previous passes.
However this information needs to propagate through
all intermediate child documents.
Therefore the page numbering in child documents may well
be inconsistent until the complete document is compiled at least once.

A useful (if unconventional) way to always ensure a consistent
page numbering is to restart the numbering in each child document
and denote the pages by `\textit{child}|.|\textit{page}'
where \textit{child} represents the chapter/section number of the child file.
This can be achieved by the command
|\numberwithin{page}{|\textit{child}|}|
of the \textsf{amsmath} package
where \textit{child} can be |chapter| or |section|
depending on the chosen structuring.
Alternatively, one can modify the macro |\thepage| appropriately
and reset the counter |page| at the start of each child file.

%%%%%%%%%%%%%%%%%%%%%%%%%%%%%%%%%%%%%%%%%%%%%%%%%%%%%%%%%%%%%%%%%%%%%%%%%%%%%%%%
\subsection{Conditional Processing}
\label{sec:conditional}

The package provides a mechanism to compile different versions
of a document. To customise the versions further some conditional processing
can come in handy to distinguish which version is being compiled.
The package provides two macros to describe the compilation context:

%%%%%%%%%%%%%%%%%%%%%%%%%%%%%%%%%%%%%%%%
\DescribeMacro{\ifchilddoc}
The conditional |\ifchilddoc| distinguishes between the compilation of
child documents and the main document:
%
\begin{center}
|\ifchilddoc |\textit{child-code}| |[|\||else |\textit{main-code}]| \||fi|
\end{center}

%%%%%%%%%%%%%%%%%%%%%%%%%%%%%%%%%%%%%%%%
\DescribeMacro{\childdocname}
\DescribeMacro{\childdocjob}
The macro |\childdocname| contains the filename (without extension)
of the main or child file being processed.
Note that |\childdocjob| will always contain the name of the main file.

%%%%%%%%%%%%%%%%%%%%%%%%%%%%%%%%%%%%%%%%
\paragraph{Title Page.}

Conditional processing can be used to include a title or banner page
in the main document when proper precautions are taken.
Importantly, the code in the main file should ensure that the page counter
(as well as other status parameters which are stored in the |.aux| files)
takes the same value after the conditional processing.
Otherwise the page numbers may take divergent values
depending on which part is compiled.

For example, a title page could be declared by:
%
\begin{center}
\begin{tabular}{l}
|\ifchilddoc\||else|\\
|\addtocounter{page}{-1}|\\
\textit{code for title page}\\
|\newpage|\\
|\||fi|
\end{tabular}
\end{center}
%
A banner page for the child documents can be generated by:
%
\begin{center}
\begin{tabular}{l}
|\ifchilddoc|\\
|\addtocounter{page}{-1}|\\
\textit{code for banner page}\\
|\newpage|\\
|\||fi|
\end{tabular}
\end{center}
%
Here one could write a message such as:
\begin{center}
|This is the part \childdocname{} of \childdocjob{}.|
\end{center}

%%%%%%%%%%%%%%%%%%%%%%%%%%%%%%%%%%%%%%%%%%%%%%%%%%%%%%%%%%%%%%%%%%%%%%%%%%%%%%%%
\subsection{Flags}
\label{sec:flags}

The package makes it easy to generate different versions
of the main or child documents.
To this end compilation flags can be defined
and assigned different default values.
They will be particularly useful in conjunction
with the forwarding mechanism described in \secref{sec:forward}.

For example, it may be useful to have a flag |\version|
which can be set to |draft| or |final|.
The document source will contain some conditional code
depending on the value of |\version|.
Suppose further, the flag should default to |final| for the main file
and to |draft| for child files
which is a natural assignment for editing the document.
This is achieved by placing the following code
in the preamble of the main document
(below the |\childdocmain| directive):
%
\begin{center}
\begin{tabular}{l}
|\ifchilddoc|\\
|\providecommand{\version}{draft}|\\
|\||else|\\
|\providecommand{\version}{final}|\\
|\||fi|
\end{tabular}
\end{center}
%
The definition by |\providecommand| makes sure
that previous definitions are not overwritten.
Further statements |\providecommand{\version}{...}|
can thus be added before the above code to override it.

For the main file, one might add a line
(between |\childdocmain| and the above block)
%
\begin{center}
|%\ifchilddoc\||else\providecommand{\version}{draft}\||fi|
\end{center}
%
which can be uncommented to produce a draft version.
Likewise one can add a line to the very top of a child file
(above the |\childdocof{|\textit{main}|}| directive)
%
\begin{center}
|%\providecommand{\version}{final}|
\end{center}
%
which can be uncommented to produce the final version of this child document.

%%%%%%%%%%%%%%%%%%%%%%%%%%%%%%%%%%%%%%%%%%%%%%%%%%%%%%%%%%%%%%%%%%%%%%%%%%%%%%%%
\subsection{Forwarding}
\label{sec:forward}

Different versions of the main or child documents
using compilation flags as described in \secref{sec:flags}
can be (permanently) stored in different files
for convenient compilation, viewing and distribution.
To this end, the package defines a command
to pass on compilation to a different file:

%%%%%%%%%%%%%%%%%%%%%%%%%%%%%%%%%%%%%%%%
\DescribeMacro{\childdocforward}
The command |\childdocforward| redirects processing to
another source file:
%
\begin{center}
\begin{tabular}{l}
|\input{childdoc.def}|\\
|\childdocforward[|\textit{main}|]{|\textit{dest}|}|\\
\end{tabular}
\end{center}
%
The argument \textit{dest} is the destination file
(without extension).
It should be the main file or one of the child files.
Note that further \textsf{childdoc} directives
such as |\childdocof| and |\childdocforward|
in the indicated file will be processed in this form.
The optional argument \textit{main}
passes on directly to the main file \textit{main}
while pretending to compile the child \textit{dest}.
This form behaves as if \textit{dest}
issues |\childdocof{|\textit{main}|}| right away,
and no further \textsf{childdoc} directives will be processed.

%%%%%%%%%%%%%%%%%%%%%%%%%%%%%%%%%%%%%%%%
\DescribeMacro{\...prefix}
In the alternative form |\childdocforwardprefix|,
%
\begin{center}
\begin{tabular}{l}
|\input{childdoc.def}|\\
|\childdocforwardprefix[|\textit{main}|]{|\textit{prefix}|}{|\textit{dest}|}|
\end{tabular}
\end{center}
%
the destination file is determined by a pattern
depending on the current file:
To make this work, the current file must be called
`{\textit{prefix}\hspace{0.2em}\textit{suffix}}'
with \textit{prefix} matching precisely the argument.
Processing is then passed on to the file
`{\textit{dest}\hspace{0.2em}\textit{suffix}}'.
Surely, the same effect is achieved by
directly specifying the
argument `{\textit{dest}\hspace{0.2em}\textit{suffix}}'
in the first form.
However, that requires to set up a different file
for each child. With the alternative form of the command
all these files can have exactly the same content
which simplifies setting them up and maintaining them.

For example, the following file |draft.tex|
with a compilation flag |\version| as described in \secref{sec:flags}
compiles the main document as a draft:
%
\begin{center}
\begin{tabular}{l}
|\def\version{draft}|\\
|\input{childdoc.def}|\\
|\childdocforward{|\textit{main}|}|
\end{tabular}
\end{center}
%
Likewise, the following files |final|\textit{nn}|.tex|
compile the final version of the child document
|child|\textit{nn}|.tex|:
%
\begin{center}
\begin{tabular}{l}
|\def\version{final}|\\
|\input{childdoc.def}|\\
|\childdocforwardprefix{final}{child}|
\end{tabular}
\end{center}
%

Note that when several versions of a main file and/or of each child file
are to be generated, it may be convenient to set up a |Makefile| or
shell script to automatise the process.

%%%%%%%%%%%%%%%%%%%%%%%%%%%%%%%%%%%%%%%%%%%%%%%%%%%%%%%%%%%%%%%%%%%%%%%%%%%%%%%%
\subsection{Command Line Processing}
\label{sec:commandline}

The effect of redirection files can also be achieved by invoking
the \LaTeX{} compiler with a more elaborate command line.
Most conveniently this should be done as part
of a shell script or a |Makefile|.

When using \textsf{childdoc} in the main file, the following
command lines effectively perform a redirection
(note that depending on the shell being used,
backslashes may have to be doubled: `|\|' $\to$ `|\\|'):
%
\begin{center}
|... -jobname "|\textit{target}|" |\\|"|[\textit{flags}]%
|\input{childdoc.def}\childdocforward[|\textit{main}|]{|\textit{dest}|}"|
\end{center}
%
Here \textit{target} is the name of the output file,
\textit{main} is the name of the main file
and \textit{dest} is the name of the main or child file to be processed
(all filenames without extensions).
The optional argument \textit{main} can be omitted
if \textit{main} matches \textit{dest}.
Optionally, compilation \textit{flags} can be defined via |\def| commands.
This command line makes the \TeX{} engine believe
it is compiling the file \textit{target}
whose content is specified as the latter parameter.
The provided code then forwards the processing to
\textit{main} or \textit{dest} as described in \secref{sec:forward}.

%%%%%%%%%%%%%%%%%%%%%%%%%%%%%%%%%%%%%%%%%%%%%%%%%%%%%%%%%%%%%%%%%%%%%%%%%%%%%%%%
\subsection{Include by Input}
\label{sec:input}

Including child documents by |\include| has some restrictions by design.
Most notably, the content of a child document always occupies
its own set of pages; pages cannot be shared between child documents.
Usually, this behaviour makes perfect sense
because each child document contain an essential part of the document.
However, in some situations it may be desirable to compose
a document from a collection of parts
without having mandatory page breaks between then.
For this case, the package
provides a mechanism to include parts
by |\input| which can also be processed individually.
However, by construction this mechanism
requires manual handling of the content to be output.

%%%%%%%%%%%%%%%%%%%%%%%%%%%%%%%%%%%%%%%%
\DescribeMacro{\ifchilddocmanual}
The main file should be prepared as usual, see \secref{sec:include}.
However, the document body must make a distinction
between processing of an individual part and of the main document, e.g.:
%
\begin{center}
\begin{tabular}{l}
|\ifchilddocmanual|\\
|\input{\childdocname}|\\
|\||else|\\
\textit{document body with }|\input{|\textit{part}|}|\\
|\||fi|
\end{tabular}
\end{center}
%
The conditional |\ifchilddocmanual| is true whenever
a part to be included by |\input| is being compiled,
and the name of the part is stored in |\childdocname|.

%%%%%%%%%%%%%%%%%%%%%%%%%%%%%%%%%%%%%%%%
\DescribeMacro{\childdocby}
Each part to be included by |\input| should start with:
%
\begin{center}
\begin{tabular}{l}
|\input{childdoc.def}|\\
|\childdocby{|\textit{main}|}|\\
\end{tabular}
\end{center}
%
The directive |\childdocby| is similar to |\childdocof|
described in \secref{sec:include},
but the subsequent selection of content must be done manually.
To that end, both |\ifchilddoc| and |\ifchilddocmanual|
will be true upon processing of a part,
and the name of the part is stored in |\childdocname|.
Note that |\jobname| will be set to the filename of the current part
so that each part receives an individual |.aux| file
that does not interfere with the |.aux| file(s) of the main document.
This behaviour can be altered by the alternative form
|\childdocby[*]{|\textit{main}|}| (with a non-empty optional argument)
which uses the |.aux| file of the main document
by setting |\jobname| to \textit{main}.

%%%%%%%%%%%%%%%%%%%%%%%%%%%%%%%%%%%%%%%%%%%%%%%%%%%%%%%%%%%%%%%%%%%%%%%%%%%%%%%%
\subsection{Driver Development}
\label{sec:driver}

The \textsf{childdoc} mechanism can also be use for the development
of definition files such as \LaTeX{} styles or classes.
This case differs from the above setup with multiple parts
included by |\include| in that no |\includeonly| should be invoked.
This can be achieved by starting the include file
(before |\ProvidesPackage|) with:
%
\begin{center}
\begin{tabular}{l}
|\input{childdoc.def}|\\
|\childdocforward{|\textit{main}|}|\\
\end{tabular}
\end{center}
%
or alternatively with:
%
\begin{center}
\begin{tabular}{l}
|\input{childdoc.def}|\\
|\childdocby{|\textit{main}|}|\\
\end{tabular}
\end{center}
%
Both forms have slightly different effects as described above.
The main file is prepared as usual, see \secref{sec:include}.

%%%%%%%%%%%%%%%%%%%%%%%%%%%%%%%%%%%%%%%%%%%%%%%%%%%%%%%%%%%%%%%%%%%%%%%%%%%%%%%%
\subsection{Legacy Detection}
\label{sec:detection}

The directive |\childdocmain| in the main file can detect
whether the complete document or merely a child is to be compiled
even without using the directive |\childdocof|.
This method is deprecated because it is less robust
and there is no compelling reason to use it;
it is merely provided for backward compatibility
and it may be removed in future versions.

If the detection mechanism is to be used,
it is mandatory to correctly specify
the filename of the main file as the argument of |\childdocmain|:
%
\begin{center}
\begin{tabular}{l}
|\input{childdoc.def}|\\
|\childdocmain{|\textit{main}|}|\\
\end{tabular}
\end{center}
%
If |\jobname| does not match the argument \textit{main} of |\childdocmain|,
it is assumed that |\jobname| points to the child file to be compiled.
When using |\childdocmain| with the main file specified as argument,
it suffices to start a child file
with just |\input{|\textit{main}|}|
without loading of the package and using |\childdocof|.
If instead all processing is done
with the appropriate \textsf{childdoc} directives,
the argument of \textit{main} of |\childdocmain| can be empty.

An alternative version of the command line processing described
in \secref{sec:commandline} using the detection mechanism reads:
%
\begin{center}
|... -jobname "|\textit{target}|" "|[\textit{flags}]%
[|\def\jobname{|\textit{dest}|}|]|\input{|\textit{main}|}"|
\end{center}

%%%%%%%%%%%%%%%%%%%%%%%%%%%%%%%%%%%%%%%%%%%%%%%%%%%%%%%%%%%%%%%%%%%%%%%%%%%%%%%%
\subsection{Manual Code}
\label{sec:manual}

In case one cannot be certain whether the definitions file |childdoc.def|
is installed on the target \TeX{} distribution
and one prefers not to ship it,
it is conceivable to paste a few relevant commands into the sources.

To that end, drop all statements |\input{childdoc.def}|
and perform the replacements as outlined below.
Instead of |\childdocmain{|\textit{main}|}| add the following code
to the top of the main file:
%
\begin{center}
\begin{tabular}{l}
|\||ifdefined\childdocname\endinput\||fi\newif\ifchilddoc|\\
|\edef\childdocname{\scantokens\expandafter{\jobname\noexpand}}|\\
|\def\childdocmain{|\textit{main}|}\||ifx\childdocmain\childdocname\||else|\\
|\childdoctrue\includeonly{\childdocname}\let\jobname\childdocmain\||fi|\\
\end{tabular}
\end{center}
%
Instead of |\childdocof{|\textit{main}|}| just include the main file
at the top of each child file:
%
\begin{center}
|\input{|\textit{main}|}|
\end{center}
%
A simple redirection |\childdocforward{|\textit{dest}|}| is achieved by:
%
\begin{center}
|\def\jobname{|\textit{dest}|}\input{\jobname}|
\end{center}
%
The redirection with prefix
|\childdocforwardprefix[|\textit{prefix}|]{|\textit{dest}|}|
is accomplished by:
%
\begin{center}
\begin{tabular}{l}
|{\edef\jobname{\scantokens\expandafter{\jobname\noexpand}}|\\
|\def\redirectjob |\textit{prefix}|#1~~~{\gdef\jobname{|\textit{dest}|#1}}|\\
|\expandafter\redirectjob\jobname~~~}\input{\jobname}|
\end{tabular}
\end{center}

In an alternative approach,
child documents can be compiled by a specific command line
without additional code or specific definitions:
%
\begin{center}
|... -jobname "|\textit{target}|" "|[\textit{flags}]%
|\includeonly{|\textit{dest}|}\input{|\textit{main}|}"|
\end{center}
%

%%%%%%%%%%%%%%%%%%%%%%%%%%%%%%%%%%%%%%%%%%%%%%%%%%%%%%%%%%%%%%%%%%%%%%%%%%%%%%%%
%%%%%%%%%%%%%%%%%%%%%%%%%%%%%%%%%%%%%%%%%%%%%%%%%%%%%%%%%%%%%%%%%%%%%%%%%%%%%%%%
\section{Information}

%%%%%%%%%%%%%%%%%%%%%%%%%%%%%%%%%%%%%%%%%%%%%%%%%%%%%%%%%%%%%%%%%%%%%%%%%%%%%%%%
\subsection{Copyright}

Copyright \copyright{} 2017--2018 Niklas Beisert

This work may be distributed and/or modified under the
conditions of the \LaTeX{} Project Public License, either version 1.3
of this license or (at your option) any later version.
The latest version of this license is in
  \url{http://www.latex-project.org/lppl.txt}
and version 1.3 or later is part of all distributions of \LaTeX{}
version 2005/12/01 or later.

This work has the LPPL maintenance status `maintained'.

The Current Maintainer of this work is Niklas Beisert.

This work consists of the files |README.txt|, |childdoc.ins| and |childdoc.dtx|
as well as the derived files |childdoc.def|, |cdocsamp.tex|
with |cdocsch1.tex|, |cdocsch2.tex|, |cdocspt3.tex|, |cdocspt4.tex|,
|cdocsdrf.tex|, |cdocsfn1.tex|, |cdocsfn2.tex|
as well as |childdoc.pdf|.

%%%%%%%%%%%%%%%%%%%%%%%%%%%%%%%%%%%%%%%%%%%%%%%%%%%%%%%%%%%%%%%%%%%%%%%%%%%%%%%%
\subsection{Files and Installation}

The package consists of the files:
%
\begin{center}
\begin{tabular}{ll}
    |README.txt|   & readme file \\
    |childdoc.ins| & installation file \\
    |childdoc.dtx| & source file \\
    |childdoc.def| & definition file \\
    |cdocsamp.tex| & sample main file \\
    |cdocsch1.tex| & sample include file \\
    |cdocsch2.tex| & sample include file \\
    |cdocspt3.tex| & sample part file \\
    |cdocspt4.tex| & sample part file \\
    |cdocsdrf.tex| & sample redirection file \\
    |cdocsfn1.tex| & sample redirection file \\
    |cdocsfn2.tex| & sample redirection file \\
    |childdoc.pdf| & manual
\end{tabular}
\end{center}
%
The distribution consists of the files
|README.txt|, |childdoc.ins| and |childdoc.dtx|.
%
\begin{itemize}
\item
Run (pdf)\LaTeX{} on |childdoc.dtx|
to compile the manual |childdoc.pdf| (this file).
\item
Run \LaTeX{} on |childdoc.ins| to create the definitions file |childdoc.def|
and the sample |cdocsamp.tex| with include files
|cdocsch1.tex|, |cdocsch2.tex|, |cdocspt3.tex|, |cdocspt4.tex|,
|cdocsdrf.tex|, |cdocsfn1.tex|, |cdocsfn2.tex|.
Then copy the file |childdoc.def| to an appropriate directory of your \LaTeX{}
distribution, e.g.\ \textit{texmf-root}|/tex/latex/childdoc|.
\end{itemize}

%%%%%%%%%%%%%%%%%%%%%%%%%%%%%%%%%%%%%%%%%%%%%%%%%%%%%%%%%%%%%%%%%%%%%%%%%%%%%%%%
\subsection{Related CTAN Packages}

There are several other packages which offer a similar functionality:
%
\begin{itemize}
\item
The packages
\href{http://ctan.org/pkg/docmute}{\textsf{docmute}},
\href{http://ctan.org/pkg/includex}{\textsf{includex}} and
\href{http://ctan.org/pkg/standalone}{\textsf{standalone}}
provide commands to include only the document body of
a child file thus allowing both files to be compiled individually.
\item
The packages \href{http://ctan.org/pkg/subdocs}{\textsf{subdocs}}
and \href{http://ctan.org/pkg/subfiles}{\textsf{subfiles}}
provide structures in which the main and child documents can be
encapsulated and allowing them to be compiled individually.
The inclusion mechanism is different from the conventional |\include|.
\item
The package \href{http://ctan.org/pkg/combine}{\textsf{combine}}
is an elaborate solution to combine several documents into one.
\end{itemize}
%
See also the CTAN topic \href{http://ctan.org/topic/subdocs}{\textsf{subdocs}}
for further related packages.
The present package differs from the above solutions in that
a document structure constructed with the conventional |\include| mechanism
just needs two extra commands at the top of every file
such that all constituent files can be compiled individually.

%%%%%%%%%%%%%%%%%%%%%%%%%%%%%%%%%%%%%%%%%%%%%%%%%%%%%%%%%%%%%%%%%%%%%%%%%%%%%%%%
%\subsection{Feature Suggestions}
%
%The following is a list of features which may be useful for future
%versions of this package:
%%
%\begin{itemize}
%\item
%\ldots
%\end{itemize}

%%%%%%%%%%%%%%%%%%%%%%%%%%%%%%%%%%%%%%%%%%%%%%%%%%%%%%%%%%%%%%%%%%%%%%%%%%%%%%%%
\subsection{Revision History}

%%%%%%%%%%%%%%%%%%%%%%%%%%%%%%%%%%%%%%%%
\paragraph{v2.0:} 2018/12/30

\begin{itemize}
\item
immediate forward processing
\item
added |\childdocby| mechanism
\item
manual restructured
\end{itemize}

%%%%%%%%%%%%%%%%%%%%%%%%%%%%%%%%%%%%%%%%
\paragraph{v1.6:} 2018/01/17

\begin{itemize}
\item
application for development of include files
\item
corrections to manual
\end{itemize}

%%%%%%%%%%%%%%%%%%%%%%%%%%%%%%%%%%%%%%%%
\paragraph{v1.5:} 2017/05/21

\begin{itemize}
\item
more complete structuring introduced
\item
|\childdocof| introduced
\item
|\childdoc| renamed to |\childdocmain|
\item
|\childredirect| renamed to |\childdocforward| and |\childdocforwardprefix|
and functionality expanded
\end{itemize}

%%%%%%%%%%%%%%%%%%%%%%%%%%%%%%%%%%%%%%%%
\paragraph{v1.0:} 2017/04/27

\begin{itemize}
\item
manual and install package
\item
first version published on CTAN
\end{itemize}

%%%%%%%%%%%%%%%%%%%%%%%%%%%%%%%%%%%%%%%%
\paragraph{v0.6:} 2017/04/26

\begin{itemize}
\item
redirection mechanism added
\end{itemize}

%%%%%%%%%%%%%%%%%%%%%%%%%%%%%%%%%%%%%%%%
\paragraph{v0.5:} 2017/04/26

\begin{itemize}
\item
functionality in definition file
\end{itemize}


%%%%%%%%%%%%%%%%%%%%%%%%%%%%%%%%%%%%%%%%%%%%%%%%%%%%%%%%%%%%%%%%%%%%%%%%%%%%%%%%
%%%%%%%%%%%%%%%%%%%%%%%%%%%%%%%%%%%%%%%%%%%%%%%%%%%%%%%%%%%%%%%%%%%%%%%%%%%%%%%%
%%%%%%%%%%%%%%%%%%%%%%%%%%%%%%%%%%%%%%%%%%%%%%%%%%%%%%%%%%%%%%%%%%%%%%%%%%%%%%%%
\appendix

\settowidth\MacroIndent{\rmfamily\scriptsize 000\ }

 \DocInput{childdoc.dtx}

\end{document}
%</driver>
% \fi
%
% %%%%%%%%%%%%%%%%%%%%%%%%%%%%%%%%%%%%%%%%%%%%%%%%%%%%%%%%%%%%%%%%%%%%%%%%%%%%%%
% %%%%%%%%%%%%%%%%%%%%%%%%%%%%%%%%%%%%%%%%%%%%%%%%%%%%%%%%%%%%%%%%%%%%%%%%%%%%%%
% \section{Sample}
%\iffalse
%<*samplemain>
%\fi
%
% The following presents a sample document
% with two chapters, two parts, a title page,
% a compile flag as well as three forwarding files to set the flag.
% It consists of eight |.tex| files:
% \begin{center}
% \begin{tabular}{ll}
% |cdocsamp.tex|&main file\\
% |cdocsch1.tex|&include file for chapter 1\\
% |cdocsch2.tex|&include file for chapter 2\\
% |cdocspt3.tex|&include file for part 3\\
% |cdocspt4.tex|&include file for part 4\\
% |cdocsdrf.tex|&forwarding file for main file in draft mode\\
% |cdocsfi1.tex|&forwarding file for final version of chapter 1\\
% |cdocsfi2.tex|&forwarding file for final version of chapter 2\\
% \end{tabular}
% \end{center}
% Each of the eight files can be compiled directly by the \LaTeX{} compiler.
%
% %%%%%%%%%%%%%%%%%%%%%%%%%%%%%%%%%%%%%%
% \paragraph{Main File.}
%
% The main file is called |cdocsamp.tex|.
%
% Load the \textsf{childdoc} definitions and
% declare the filename for the main document:
%    \begin{macrocode}
\input{childdoc.def}
\childdocmain{}
%    \end{macrocode}

% Optional override for |\version| flag:
%    \begin{macrocode}
%%\ifchilddoc\else\providecommand{\version}{draft}\fi
%    \end{macrocode}

% Define the default values for the |\version| flag
% (|final| for the main file and |draft| for childs):
%    \begin{macrocode}
\ifchilddoc
\providecommand{\version}{draft}
\else
\providecommand{\version}{final}
\fi
%    \end{macrocode}

% Load the standard document class:
%    \begin{macrocode}
\documentclass[12pt]{article}
%    \end{macrocode}

% Start the document body:
%    \begin{macrocode}
\begin{document}
%    \end{macrocode}

% Declare a title page.
% Print title, part of document being processed and version flag:
%    \begin{macrocode}
\addtocounter{page}{-1}
\begin{center}
{\LARGE\bfseries{}childdoc example\par}
\vspace{1cm}
\ifchilddoc
\ifchilddocmanual part\else chapter\fi:
`\childdocname' of `\childdocjob'\par
\else
main document: `\childdocjob'\par
\fi
version: \version\par
\end{center}
\newpage
%    \end{macrocode}

% Manually include selected file,
% otherwise process as usual:
%    \begin{macrocode}
\ifchilddocmanual
\section*{part `\childdocname'}
\input{\childdocname}
\else
%    \end{macrocode}

% Include the two chapters:
%    \begin{macrocode}
\include{cdocsch1}
\include{cdocsch2}
%    \end{macrocode}

% Include the two parts unless only chapters should be displayed:
%    \begin{macrocode}
\ifchilddoc\else
\section{part three}
\input{cdocspt3}
\section{part four}
\input{cdocspt4}
\fi
%    \end{macrocode}

% Process as usual until here:
%    \begin{macrocode}
\fi
%    \end{macrocode}

% End of document body:
%    \begin{macrocode}
\end{document}
%    \end{macrocode}
%\iffalse
%</samplemain>
%\fi
%
% %%%%%%%%%%%%%%%%%%%%%%%%%%%%%%%%%%%%%%
% \paragraph{Chapter Include Files.}
%
% The include files are called |cdocsch1.tex| and |cdocsch2.tex|.
%
%\iffalse
%<*samplechap1|samplechap2>
%\fi

% Optional override for |\version| flag:
%    \begin{macrocode}
%%\providecommand{\version}{final}
%    \end{macrocode}

% Include the main document:
%    \begin{macrocode}
\input{childdoc.def}
\childdocof{cdocsamp}
%    \end{macrocode}

%\iffalse
%</samplechap1|samplechap2>
%\fi
%
%\iffalse
%<*samplechap1>
%\fi
% Some text for chapter 1:
%    \begin{macrocode}
\section{one}
some text in chapter one
%    \end{macrocode}

%\iffalse
%</samplechap1>
%\fi
% Some text for chapter 2:
%\iffalse
%<*samplechap2>
%\fi
%    \begin{macrocode}
\section{two}
more text in chapter two
%    \end{macrocode}

%\iffalse
%</samplechap2>
%\fi
%
% %%%%%%%%%%%%%%%%%%%%%%%%%%%%%%%%%%%%%%
% \paragraph{Part Include Files.}
%
% The include files are called |cdocspt3.tex| and |cdocspt4.tex|.
%
%\iffalse
%<*samplepart3|samplepart4>
%\fi

% Optional override for |\version| flag:
%    \begin{macrocode}
%%\providecommand{\version}{final}
%    \end{macrocode}

% Include the main document:
%    \begin{macrocode}
\input{childdoc.def}
\childdocby{cdocsamp}
%    \end{macrocode}

%\iffalse
%</samplepart3|samplepart4>
%\fi
%
%\iffalse
%<*samplepart3>
%\fi
% Some text for part 3:
%    \begin{macrocode}
some text in part three
%    \end{macrocode}

%\iffalse
%</samplepart3>
%\fi
% Some text for part 4:
%\iffalse
%<*samplepart4>
%\fi
%    \begin{macrocode}
more text in part four
%    \end{macrocode}

%\iffalse
%</samplepart4>
%\fi
%
% %%%%%%%%%%%%%%%%%%%%%%%%%%%%%%%%%%%%%%
% \paragraph{Forwarding for a Complete Draft.}
%
% The following forwarding file |cdocsdrf.tex|
% compiles the main document in draft mode:
%\iffalse
%<*sampledraft>
%\fi
%    \begin{macrocode}
\def\version{draft}
\input{childdoc.def}
\childdocforward{cdocsamp}
%    \end{macrocode}

%\iffalse
%</sampledraft>
%\fi
%
% %%%%%%%%%%%%%%%%%%%%%%%%%%%%%%%%%%%%%%
% \paragraph{Forwarding for Final Version of the Chapters.}
%
% The following forwarding files |cdocsfn1.tex| and |cdocsfn2.tex|
% (with identical content)
% compile the final versions of the child documents
% |cdocsch1.tex| and |cdocsch2.tex|, respectively:
%\iffalse
%<*samplefinal>
%\fi
%    \begin{macrocode}
\def\version{final}
\input{childdoc.def}
\childdocforwardprefix[cdocsamp]{cdocsfn}{cdocsch}
%    \end{macrocode}

%\iffalse
%</samplefinal>
%\fi
%
% %%%%%%%%%%%%%%%%%%%%%%%%%%%%%%%%%%%%%%
% \paragraph{Command Line Processing.}
%
% The following three command lines generate the output files
% |cdocscld|, |cdocscl1| and |cdocscl2|
% which should be identical to
% |cdocsdrf|, |cdocsch1| and |cdocsfn2|, respectively:
% \begin{center}
% \begin{tabular}{l}
% |latex -jobname cdocscld \|\\
% |  "\def\version{draft}\input{childdoc.def}\childdocforward{cdocsamp}"|\\
% |latex -jobname cdocscl1 \|\\
% |  "\input{childdoc.def}\childdocforward[cdocsamp]{cdocsch1}"|\\
% |latex -jobname cdocscl2 \|\\
% |  "\def\version{final}\input{childdoc.def}\childdocforward{cdocsch2}"|
% \end{tabular}
% \end{center}
% Note that the trailing backslash on each first line
% merely continues the input to the second line
% (for convenient cut ant paste).
% Furthermore, the command |latex| can be replaced by any
% of its alternative versions such as |pdflatex|.
%
% %%%%%%%%%%%%%%%%%%%%%%%%%%%%%%%%%%%%%%%%%%%%%%%%%%%%%%%%%%%%%%%%%%%%%%%%%%%%%%
% %%%%%%%%%%%%%%%%%%%%%%%%%%%%%%%%%%%%%%%%%%%%%%%%%%%%%%%%%%%%%%%%%%%%%%%%%%%%%%
% \section{Implementation}
%\iffalse
%<*package>
%\fi
%
% This section describes the definitions file |childdoc.def|.

% The definitions cannot be loaded using |\usepackage| or |\RequirePackage|
% which has a mechanism to prevent loading a style file more than once.
% When loading the definitions by means of |\input|
% multiple instances have to be prevented manually:
%\iffalse
%This code needs to be before the `\ProvidesFile' directive
%which is defined at the beginning of this file.
%Therefore it is also placed there and commented out here.
%</package>
%<*discard>
%\fi
%    \begin{macrocode}
\ifdefined\childdocmain\endinput\fi
%    \end{macrocode}
%\iffalse
%</discard>
%<*package>
%\fi
%
% \macro{\ifchilddoc}
% \macro{\ifchilddocmanual}
% The conditional |\ifchilddoc| tells whether a
% child (true) or main (false) document is being compiled.
% The conditional |\ifchilddocmanual| tells whether
% the |\includeonly| mechanism is used (false) or
% the selection of child files must be performed manually (true).
% The definitions initialise to false:
%    \begin{macrocode}
\newif\ifchilddoc
\newif\ifchilddocmanual
%    \end{macrocode}

% \macro{\childdocname}
% \macro{\childdocjob}
% The macro |\childdocname| stores the name of the main document
% to be compiled. The macro |\childdocjob| stores the name of
% the document on which the \LaTeX{} compiler was originally invoked.
% The content of |\jobname| cannot be compared
% to filenames specified in the source due to different catcodes.
% The following code rescans |\jobname|, stores the result
% in |\childdocname| and saves a copy in |\childdocjob|:
%    \begin{macrocode}
\edef\childdocname{\scantokens\expandafter{\jobname\noexpand}}
\let\childdocjob\childdocname
%    \end{macrocode}

% \macro{\childdocdisable}
% The macro |\childdocdisable| prevents the main file
% from being processed more than once.
% At this stage, the main document command |\childdocmain|
% is assumed to be called once again where it should do nothing.
% Any subsequent call to it should prevent
% a secondary processing of the main document
% It overwrites the forwarding commands
% |\childdocof| and |\childdocforward|
% with empty macros to prevent further inclusions of the main document:
%    \begin{macrocode}
\newcommand{\childdocdisable}
{
  \renewcommand{\childdocmain}[1]{\renewcommand{\childdocmain}[1]{\endinput}}
  \renewcommand{\childdocof}[1]{}
  \renewcommand{\childdocby}[2][]{}
  \renewcommand{\childdocforward}[2][]{}
  \renewcommand{\childdocdisable}{}
}
%    \end{macrocode}

% \macro{\childdocmain}
% The macro |\childdocmain| is to be called at the top of the main file
% with nothing or the main filename (without extension) as argument.
% First, it breaks loops.
% If the argument is not empty and does not match |\childdocname|
% (which is set by the first inclusion of |childdoc.def|),
% |\ifchilddoc| is set to true, |\includeonly| is applied to the child file
% and |\jobname| is set to the main file
% (for proper handling of |.aux| files):
%    \begin{macrocode}
\newcommand{\childdocmain}[1]
{
  \childdocdisable\childdocmain{}
  \if?#1?\else
    \begingroup
      \def\childdoctmp{#1}
      \ifx\childdoctmp\childdocname
        \def\childdoctmp{}
      \else
        \def\childdoctmp
        {
          \childdoctrue
          \includeonly{\childdocname}
          \def\childdocjob{#1}
          \def\jobname{#1}
        }
      \fi
      \expandafter
    \endgroup
    \childdoctmp
  \fi
}
%    \end{macrocode}

% \macro{\childdocof}
% The command |\childdocof| redirects
% compilation to the main file |#1|.
%    \begin{macrocode}
\newcommand{\childdocof}[1]
{
  \childdocdisable
  \childdoctrue
  \includeonly{\childdocname}
  \def\jobname{#1}
  \def\childdocjob{#1}
  \input{#1}
}
%    \end{macrocode}

% \macro{\childdocby}
% The command |\childdocby| ....
%    \begin{macrocode}
\newcommand{\childdocby}[2][]
{
  \childdocdisable
  \childdoctrue
  \childdocmanualtrue
  \if?#1?\else
    \def\jobname{#2}
  \fi
  \def\childdocjob{#2}
  \input{#2}
  \endinput
}
%    \end{macrocode}

% \macro{\childdocforward}
% The command |\childdocforward| redirects
% compilation to the main file or
% (if the optional argument is given) a child file.
% Parameters are set as if the main file
% or a child file starting with |\childdocof| was compiled.
% Then compilation is handed over to the main file:
%    \begin{macrocode}
\newcommand{\childdocforward}[2][]
{
  \begingroup
    \if?#1?
      \def\childdoctmp
      {
        \def\childdocname{#2}
        \def\childdocjob{#2}
        \def\jobname{#2}
        \input{#2}
        \endinput
      }
    \else
      \def\childdoctmp
      {
        \childdocdisable
        \def\childdocname{#2}
        \childdoctrue
        \includeonly{#2}
        \def\childdocjob{#1}
        \def\jobname{#1}
        \input{#1}
        \endinput
      }
    \fi
    \expandafter
  \endgroup
  \childdoctmp
}
%    \end{macrocode}

% \macro{\childdocforwardprefix}
% The command |\childdocforwardprefix| redirects
% compilation to the main or a child file by means of a pattern.
% The prefix |#1| in the current filename is replaced by |#2|
% and the suffix of the current filename is kept
% (it is assumed that the filename does not contain the substring `|~~~|'
% which is used as a delimiter).
% Compilation is handed over to the new file by |\childdocforward|:
%    \begin{macrocode}
\newcommand{\childdocforwardprefix}[3][]
{
  \begingroup
    \def\childdocextract #2##1~~~{\def\childdoctmp{\childdocforward[#1]{#3##1}}}
    \expandafter\childdocextract\childdocname~~~
    \expandafter
  \endgroup
  \childdoctmp
}
%    \end{macrocode}

% \macro{\childdoc}
% The deprecated macro |\childdoc| is a legacy version of |\childdocmain|:
%    \begin{macrocode}
\newcommand{\childdoc}{\childdocmain}
%    \end{macrocode}

% \macro{\childdocredirect}
% The deprecated macro |\childdocredirect| is a legacy version
% of |\childdocforward| and |\childdocforwardprefix|:
%    \begin{macrocode}
\newcommand{\childdocredirect}[2][]
{
  \begingroup
    \if?#1?
      \def\childdoctmp{\childdocforward{#2}}
    \else
      \def\childdoctmp{\childdocforwardprefix{#1}{#2}}
    \fi
    \expandafter
  \endgroup
  \childdoctmp
}
%    \end{macrocode}

%\iffalse
%</package>
%\fi
%
\endinput
|\\
|\childdocforward[|\textit{main}|]{|\textit{dest}|}|\\
\end{tabular}
\end{center}
%
The argument \textit{dest} is the destination file
(without extension).
It should be the main file or one of the child files.
Note that further \textsf{childdoc} directives
such as |\childdocof| and |\childdocforward|
in the indicated file will be processed in this form.
The optional argument \textit{main}
passes on directly to the main file \textit{main}
while pretending to compile the child \textit{dest}.
This form behaves as if \textit{dest}
issues |\childdocof{|\textit{main}|}| right away,
and no further \textsf{childdoc} directives will be processed.

%%%%%%%%%%%%%%%%%%%%%%%%%%%%%%%%%%%%%%%%
\DescribeMacro{\...prefix}
In the alternative form |\childdocforwardprefix|,
%
\begin{center}
\begin{tabular}{l}
|% \iffalse
%
% childdoc.dtx Copyright (C) 2017-2018 Niklas Beisert
%
% This work may be distributed and/or modified under the
% conditions of the LaTeX Project Public License, either version 1.3
% of this license or (at your option) any later version.
% The latest version of this license is in
%   http://www.latex-project.org/lppl.txt
% and version 1.3 or later is part of all distributions of LaTeX
% version 2005/12/01 or later.
%
% This work has the LPPL maintenance status `maintained'.
%
% The Current Maintainer of this work is Niklas Beisert.
%
% This work consists of the files childdoc.dtx and childdoc.ins
% and the derived files childdoc.def and cdocsamp.tex with
% cdocsch1.tex, cdocsch2.tex, cdocsdrf.tex, cdocsfn1.tex, cdocsfn2.tex.
%
%<package>\ifdefined\childdocmain\endinput\fi
%<package>\ProvidesFile{childdoc.def}[2018/12/30 v2.0 child document driver]
%<samplemain>\ProvidesFile{cdocsamp.tex}[2018/12/30 v2.0 sample for childdoc]
%<*driver>
%\ProvidesFile{childdoc.drv}[2018/12/30 v2.0 childdoc reference manual file]
\PassOptionsToClass{10pt,a4paper}{article}
\documentclass{ltxdoc}

\usepackage[margin=35mm]{geometry}
\usepackage{hyperref}
\usepackage{hyperxmp}
\usepackage[usenames]{color}

\hypersetup{colorlinks=true}
\hypersetup{pdfstartview=FitH}
\hypersetup{pdfpagemode=UseNone}
\hypersetup{pdfsource={}}
\hypersetup{pdflang={en-UK}}
\hypersetup{pdfcopyright={Copyright 2017-2018 Niklas Beisert.
  This work may be distributed and/or modified under the
  conditions of the LaTeX Project Public License, either version 1.3
  of this license or (at your option) any later version.}}
\hypersetup{pdflicenseurl={http://www.latex-project.org/lppl.txt}}
\hypersetup{pdfcontactaddress={ETH Zurich, ITP, HIT K,
  Wolfgang-Pauli-Strasse 27}}
\hypersetup{pdfcontactpostcode={8093}}
\hypersetup{pdfcontactcity={Zurich}}
\hypersetup{pdfcontactcountry={Switzerland}}
\hypersetup{pdfcontactemail={nbeisert@itp.phys.ethz.ch}}
\hypersetup{pdfcontacturl={http://people.phys.ethz.ch/\xmptilde nbeisert/}}

\newcommand{\secref}[1]{\hyperref[#1]{section \ref*{#1}}}

\parskip1ex
\parindent0pt
\let\olditemize\itemize
\def\itemize{\olditemize\parskip0pt}

\begin{document}

\title{The \textsf{childdoc} Package}
\hypersetup{pdftitle={The childdoc Package}}
\author{Niklas Beisert\\[2ex]
  Institut f\"ur Theoretische Physik\\
  Eidgen\"ossische Technische Hochschule Z\"urich\\
  Wolfgang-Pauli-Strasse 27, 8093 Z\"urich, Switzerland\\[1ex]
  \href{mailto:nbeisert@itp.phys.ethz.ch}
  {\texttt{nbeisert@itp.phys.ethz.ch}}}
\hypersetup{pdfauthor={Niklas Beisert}}
\hypersetup{pdfsubject={Manual for the LaTeX2e Package childdoc}}
\date{30 December 2018, \textsf{v2.0}}
\maketitle

\begin{abstract}\noindent
\textsf{childdoc} is a \LaTeXe{} package
that enables the direct compilation
of document sections included by |\include|
to individual files.
\end{abstract}

\begingroup
\parskip0ex
\tableofcontents
\endgroup

%%%%%%%%%%%%%%%%%%%%%%%%%%%%%%%%%%%%%%%%%%%%%%%%%%%%%%%%%%%%%%%%%%%%%%%%%%%%%%%%
%%%%%%%%%%%%%%%%%%%%%%%%%%%%%%%%%%%%%%%%%%%%%%%%%%%%%%%%%%%%%%%%%%%%%%%%%%%%%%%%
\section{Introduction}

\LaTeX{} provides a mechanism to structure a large document (such as a book)
into a main file and several child files (containing the chapters)
using the |\include| command.
This mechanism is beneficial for documents
which span hundreds of pages in order to
make the source file(s) more manageable.
Moreover, compilation can be restricted to
selected child files by means of the |\includeonly| command.
The latter feature can be used to reduce the compilation time while editing
(this was significantly more useful in the earlier days of \LaTeX{})
or to generate a smaller document which is easier to navigate.
Another application of |\includeonly| is to generate
documents consisting of selected parts of the complete document.

However, there are a few drawbacks of the plain |\include| mechanism:
\begin{itemize}
\item
The child files cannot be compiled on their own,
they can only be compiled via the main file.
A naive editing environment
(such as a text editor with an option
to have the current file processed by \LaTeX)
may require one to switch to the main file before compiling;
attempting to compile the child file produces errors.
\item
The main file must be modified (each time)
to adjust the |\includeonly| command
to the present needs. This easily leaves the main file in a messy state.
\item
The generated document will always carry the filename
of the main document. This is inconvenient if
several child files are to be compiled and
to be kept for distribution.
\end{itemize}

The present package provides a simple interface
to make child files individually compilable by \LaTeX{}.
Compiling a child file then has the same effect as compiling
the main file with an |\includeonly| command
to select the appropriate child.
Moreover the generated document will carry the name of the child
rather than the main file.
This resolves all three above issues.

This feature is meant to make the editing of books,
thesis documents and lecture notes somewhat more convenient.
However, the package can also be used efficiently for
composing a series of documents (such as exercise sheets)
which are typically distributed individually.
It then assists the author in generating the individual documents
(potentially in different versions)
as well as a document containing the collected series.
Another application is in developing style files
or other kinds of included material
where compilation of the style file could redirect
to a sample or test file.

%%%%%%%%%%%%%%%%%%%%%%%%%%%%%%%%%%%%%%%%%%%%%%%%%%%%%%%%%%%%%%%%%%%%%%%%%%%%%%%%
%%%%%%%%%%%%%%%%%%%%%%%%%%%%%%%%%%%%%%%%%%%%%%%%%%%%%%%%%%%%%%%%%%%%%%%%%%%%%%%%
\section{Usage}

First of all, the package \textsf{childdoc} is \emph{not} a standard
\LaTeXe{} |.sty| style file! Therefore it needs to be invoked in
a non-standard way.

%%%%%%%%%%%%%%%%%%%%%%%%%%%%%%%%%%%%%%%%%%%%%%%%%%%%%%%%%%%%%%%%%%%%%%%%%%%%%%%%
\subsection{Included Files}
\label{sec:include}

%%%%%%%%%%%%%%%%%%%%%%%%%%%%%%%%%%%%%%%%
\DescribeMacro{\childdocmain}
To use the package, add the commands
\begin{center}
\begin{tabular}{l}
|\input{childdoc.def}|\\
|\childdocmain{}|\\
\end{tabular}
\end{center}
at the very top of the main \LaTeX{} file,
in particular \emph{before} the |\documentclass| statement!
The argument of |\childdocmain| should be left empty
(but it must be present).

%%%%%%%%%%%%%%%%%%%%%%%%%%%%%%%%%%%%%%%%
\DescribeMacro{\childdocof}
Furthermore, add the commands
\begin{center}
\begin{tabular}{l}
|\input{childdoc.def}|\\
|\childdocof{|\textit{main}|}|\\
\end{tabular}
\end{center}
at the top of every child file \textit{child}
which is included by |\include{|\textit{child}|}|
from within the main file
(or at least for those files to be compiled individually).
The argument \textit{main} must be the filename of the main file.

There are a couple of
considerations in setting up the main and child documents:

%%%%%%%%%%%%%%%%%%%%%%%%%%%%%%%%%%%%%%%%
\paragraph{Restrictions.}

Please note the following restrictions:
\begin{itemize}
\item
|\childdocmain| must be called with one argument \textit{main}
to ensure compatibility with earlier version of the package.
It must either be empty (|\childdocmain{}|)
or precisely match the filename of the main file in which it is specified.
See \secref{sec:detection} for further information.
\item
The filename \textit{main} must be specified without the |.tex| extension.
\item
The filename \textit{main} is case sensitive
(even in case-insensitive file systems)
due to internal string comparison.
\item
The argument \textit{main} should be fully expanded, it cannot be a macro.
\item
Subdirectories and special characters should be avoided in filenames.
\item
The command |\childdocmain{|\textit{main}|}| must be followed by a whitespace.
It should not be followed immediately by another command
or by a comment mark `|%|'.
This is because the \TeX{} parser reads the token immediately following
the argument of |\childdocmain| and puts it
at the beginning of every child section;
however, a white\-space is ignored.
\end{itemize}

%%%%%%%%%%%%%%%%%%%%%%%%%%%%%%%%%%%%%%%%
\paragraph{Content of Main File.}

It is advisable to place all content in the child files included by |\include|.
Any output contained in the main file will appear in all child documents
unless suppressed manually;
it cannot be suppressed automatically by the |\includeonly| directive
and thus should normally be avoided.
A method to include some content in the main file
by means of conditional processing is described in \secref{sec:conditional}.

%%%%%%%%%%%%%%%%%%%%%%%%%%%%%%%%%%%%%%%%
\paragraph{Page Numbering.}

When only a part of the document is compiled,
the appropriate numbering of pages
(as well as other status parameters)
is determined from the |.aux| files.
The latter contain information from previous passes.
However this information needs to propagate through
all intermediate child documents.
Therefore the page numbering in child documents may well
be inconsistent until the complete document is compiled at least once.

A useful (if unconventional) way to always ensure a consistent
page numbering is to restart the numbering in each child document
and denote the pages by `\textit{child}|.|\textit{page}'
where \textit{child} represents the chapter/section number of the child file.
This can be achieved by the command
|\numberwithin{page}{|\textit{child}|}|
of the \textsf{amsmath} package
where \textit{child} can be |chapter| or |section|
depending on the chosen structuring.
Alternatively, one can modify the macro |\thepage| appropriately
and reset the counter |page| at the start of each child file.

%%%%%%%%%%%%%%%%%%%%%%%%%%%%%%%%%%%%%%%%%%%%%%%%%%%%%%%%%%%%%%%%%%%%%%%%%%%%%%%%
\subsection{Conditional Processing}
\label{sec:conditional}

The package provides a mechanism to compile different versions
of a document. To customise the versions further some conditional processing
can come in handy to distinguish which version is being compiled.
The package provides two macros to describe the compilation context:

%%%%%%%%%%%%%%%%%%%%%%%%%%%%%%%%%%%%%%%%
\DescribeMacro{\ifchilddoc}
The conditional |\ifchilddoc| distinguishes between the compilation of
child documents and the main document:
%
\begin{center}
|\ifchilddoc |\textit{child-code}| |[|\||else |\textit{main-code}]| \||fi|
\end{center}

%%%%%%%%%%%%%%%%%%%%%%%%%%%%%%%%%%%%%%%%
\DescribeMacro{\childdocname}
\DescribeMacro{\childdocjob}
The macro |\childdocname| contains the filename (without extension)
of the main or child file being processed.
Note that |\childdocjob| will always contain the name of the main file.

%%%%%%%%%%%%%%%%%%%%%%%%%%%%%%%%%%%%%%%%
\paragraph{Title Page.}

Conditional processing can be used to include a title or banner page
in the main document when proper precautions are taken.
Importantly, the code in the main file should ensure that the page counter
(as well as other status parameters which are stored in the |.aux| files)
takes the same value after the conditional processing.
Otherwise the page numbers may take divergent values
depending on which part is compiled.

For example, a title page could be declared by:
%
\begin{center}
\begin{tabular}{l}
|\ifchilddoc\||else|\\
|\addtocounter{page}{-1}|\\
\textit{code for title page}\\
|\newpage|\\
|\||fi|
\end{tabular}
\end{center}
%
A banner page for the child documents can be generated by:
%
\begin{center}
\begin{tabular}{l}
|\ifchilddoc|\\
|\addtocounter{page}{-1}|\\
\textit{code for banner page}\\
|\newpage|\\
|\||fi|
\end{tabular}
\end{center}
%
Here one could write a message such as:
\begin{center}
|This is the part \childdocname{} of \childdocjob{}.|
\end{center}

%%%%%%%%%%%%%%%%%%%%%%%%%%%%%%%%%%%%%%%%%%%%%%%%%%%%%%%%%%%%%%%%%%%%%%%%%%%%%%%%
\subsection{Flags}
\label{sec:flags}

The package makes it easy to generate different versions
of the main or child documents.
To this end compilation flags can be defined
and assigned different default values.
They will be particularly useful in conjunction
with the forwarding mechanism described in \secref{sec:forward}.

For example, it may be useful to have a flag |\version|
which can be set to |draft| or |final|.
The document source will contain some conditional code
depending on the value of |\version|.
Suppose further, the flag should default to |final| for the main file
and to |draft| for child files
which is a natural assignment for editing the document.
This is achieved by placing the following code
in the preamble of the main document
(below the |\childdocmain| directive):
%
\begin{center}
\begin{tabular}{l}
|\ifchilddoc|\\
|\providecommand{\version}{draft}|\\
|\||else|\\
|\providecommand{\version}{final}|\\
|\||fi|
\end{tabular}
\end{center}
%
The definition by |\providecommand| makes sure
that previous definitions are not overwritten.
Further statements |\providecommand{\version}{...}|
can thus be added before the above code to override it.

For the main file, one might add a line
(between |\childdocmain| and the above block)
%
\begin{center}
|%\ifchilddoc\||else\providecommand{\version}{draft}\||fi|
\end{center}
%
which can be uncommented to produce a draft version.
Likewise one can add a line to the very top of a child file
(above the |\childdocof{|\textit{main}|}| directive)
%
\begin{center}
|%\providecommand{\version}{final}|
\end{center}
%
which can be uncommented to produce the final version of this child document.

%%%%%%%%%%%%%%%%%%%%%%%%%%%%%%%%%%%%%%%%%%%%%%%%%%%%%%%%%%%%%%%%%%%%%%%%%%%%%%%%
\subsection{Forwarding}
\label{sec:forward}

Different versions of the main or child documents
using compilation flags as described in \secref{sec:flags}
can be (permanently) stored in different files
for convenient compilation, viewing and distribution.
To this end, the package defines a command
to pass on compilation to a different file:

%%%%%%%%%%%%%%%%%%%%%%%%%%%%%%%%%%%%%%%%
\DescribeMacro{\childdocforward}
The command |\childdocforward| redirects processing to
another source file:
%
\begin{center}
\begin{tabular}{l}
|\input{childdoc.def}|\\
|\childdocforward[|\textit{main}|]{|\textit{dest}|}|\\
\end{tabular}
\end{center}
%
The argument \textit{dest} is the destination file
(without extension).
It should be the main file or one of the child files.
Note that further \textsf{childdoc} directives
such as |\childdocof| and |\childdocforward|
in the indicated file will be processed in this form.
The optional argument \textit{main}
passes on directly to the main file \textit{main}
while pretending to compile the child \textit{dest}.
This form behaves as if \textit{dest}
issues |\childdocof{|\textit{main}|}| right away,
and no further \textsf{childdoc} directives will be processed.

%%%%%%%%%%%%%%%%%%%%%%%%%%%%%%%%%%%%%%%%
\DescribeMacro{\...prefix}
In the alternative form |\childdocforwardprefix|,
%
\begin{center}
\begin{tabular}{l}
|\input{childdoc.def}|\\
|\childdocforwardprefix[|\textit{main}|]{|\textit{prefix}|}{|\textit{dest}|}|
\end{tabular}
\end{center}
%
the destination file is determined by a pattern
depending on the current file:
To make this work, the current file must be called
`{\textit{prefix}\hspace{0.2em}\textit{suffix}}'
with \textit{prefix} matching precisely the argument.
Processing is then passed on to the file
`{\textit{dest}\hspace{0.2em}\textit{suffix}}'.
Surely, the same effect is achieved by
directly specifying the
argument `{\textit{dest}\hspace{0.2em}\textit{suffix}}'
in the first form.
However, that requires to set up a different file
for each child. With the alternative form of the command
all these files can have exactly the same content
which simplifies setting them up and maintaining them.

For example, the following file |draft.tex|
with a compilation flag |\version| as described in \secref{sec:flags}
compiles the main document as a draft:
%
\begin{center}
\begin{tabular}{l}
|\def\version{draft}|\\
|\input{childdoc.def}|\\
|\childdocforward{|\textit{main}|}|
\end{tabular}
\end{center}
%
Likewise, the following files |final|\textit{nn}|.tex|
compile the final version of the child document
|child|\textit{nn}|.tex|:
%
\begin{center}
\begin{tabular}{l}
|\def\version{final}|\\
|\input{childdoc.def}|\\
|\childdocforwardprefix{final}{child}|
\end{tabular}
\end{center}
%

Note that when several versions of a main file and/or of each child file
are to be generated, it may be convenient to set up a |Makefile| or
shell script to automatise the process.

%%%%%%%%%%%%%%%%%%%%%%%%%%%%%%%%%%%%%%%%%%%%%%%%%%%%%%%%%%%%%%%%%%%%%%%%%%%%%%%%
\subsection{Command Line Processing}
\label{sec:commandline}

The effect of redirection files can also be achieved by invoking
the \LaTeX{} compiler with a more elaborate command line.
Most conveniently this should be done as part
of a shell script or a |Makefile|.

When using \textsf{childdoc} in the main file, the following
command lines effectively perform a redirection
(note that depending on the shell being used,
backslashes may have to be doubled: `|\|' $\to$ `|\\|'):
%
\begin{center}
|... -jobname "|\textit{target}|" |\\|"|[\textit{flags}]%
|\input{childdoc.def}\childdocforward[|\textit{main}|]{|\textit{dest}|}"|
\end{center}
%
Here \textit{target} is the name of the output file,
\textit{main} is the name of the main file
and \textit{dest} is the name of the main or child file to be processed
(all filenames without extensions).
The optional argument \textit{main} can be omitted
if \textit{main} matches \textit{dest}.
Optionally, compilation \textit{flags} can be defined via |\def| commands.
This command line makes the \TeX{} engine believe
it is compiling the file \textit{target}
whose content is specified as the latter parameter.
The provided code then forwards the processing to
\textit{main} or \textit{dest} as described in \secref{sec:forward}.

%%%%%%%%%%%%%%%%%%%%%%%%%%%%%%%%%%%%%%%%%%%%%%%%%%%%%%%%%%%%%%%%%%%%%%%%%%%%%%%%
\subsection{Include by Input}
\label{sec:input}

Including child documents by |\include| has some restrictions by design.
Most notably, the content of a child document always occupies
its own set of pages; pages cannot be shared between child documents.
Usually, this behaviour makes perfect sense
because each child document contain an essential part of the document.
However, in some situations it may be desirable to compose
a document from a collection of parts
without having mandatory page breaks between then.
For this case, the package
provides a mechanism to include parts
by |\input| which can also be processed individually.
However, by construction this mechanism
requires manual handling of the content to be output.

%%%%%%%%%%%%%%%%%%%%%%%%%%%%%%%%%%%%%%%%
\DescribeMacro{\ifchilddocmanual}
The main file should be prepared as usual, see \secref{sec:include}.
However, the document body must make a distinction
between processing of an individual part and of the main document, e.g.:
%
\begin{center}
\begin{tabular}{l}
|\ifchilddocmanual|\\
|\input{\childdocname}|\\
|\||else|\\
\textit{document body with }|\input{|\textit{part}|}|\\
|\||fi|
\end{tabular}
\end{center}
%
The conditional |\ifchilddocmanual| is true whenever
a part to be included by |\input| is being compiled,
and the name of the part is stored in |\childdocname|.

%%%%%%%%%%%%%%%%%%%%%%%%%%%%%%%%%%%%%%%%
\DescribeMacro{\childdocby}
Each part to be included by |\input| should start with:
%
\begin{center}
\begin{tabular}{l}
|\input{childdoc.def}|\\
|\childdocby{|\textit{main}|}|\\
\end{tabular}
\end{center}
%
The directive |\childdocby| is similar to |\childdocof|
described in \secref{sec:include},
but the subsequent selection of content must be done manually.
To that end, both |\ifchilddoc| and |\ifchilddocmanual|
will be true upon processing of a part,
and the name of the part is stored in |\childdocname|.
Note that |\jobname| will be set to the filename of the current part
so that each part receives an individual |.aux| file
that does not interfere with the |.aux| file(s) of the main document.
This behaviour can be altered by the alternative form
|\childdocby[*]{|\textit{main}|}| (with a non-empty optional argument)
which uses the |.aux| file of the main document
by setting |\jobname| to \textit{main}.

%%%%%%%%%%%%%%%%%%%%%%%%%%%%%%%%%%%%%%%%%%%%%%%%%%%%%%%%%%%%%%%%%%%%%%%%%%%%%%%%
\subsection{Driver Development}
\label{sec:driver}

The \textsf{childdoc} mechanism can also be use for the development
of definition files such as \LaTeX{} styles or classes.
This case differs from the above setup with multiple parts
included by |\include| in that no |\includeonly| should be invoked.
This can be achieved by starting the include file
(before |\ProvidesPackage|) with:
%
\begin{center}
\begin{tabular}{l}
|\input{childdoc.def}|\\
|\childdocforward{|\textit{main}|}|\\
\end{tabular}
\end{center}
%
or alternatively with:
%
\begin{center}
\begin{tabular}{l}
|\input{childdoc.def}|\\
|\childdocby{|\textit{main}|}|\\
\end{tabular}
\end{center}
%
Both forms have slightly different effects as described above.
The main file is prepared as usual, see \secref{sec:include}.

%%%%%%%%%%%%%%%%%%%%%%%%%%%%%%%%%%%%%%%%%%%%%%%%%%%%%%%%%%%%%%%%%%%%%%%%%%%%%%%%
\subsection{Legacy Detection}
\label{sec:detection}

The directive |\childdocmain| in the main file can detect
whether the complete document or merely a child is to be compiled
even without using the directive |\childdocof|.
This method is deprecated because it is less robust
and there is no compelling reason to use it;
it is merely provided for backward compatibility
and it may be removed in future versions.

If the detection mechanism is to be used,
it is mandatory to correctly specify
the filename of the main file as the argument of |\childdocmain|:
%
\begin{center}
\begin{tabular}{l}
|\input{childdoc.def}|\\
|\childdocmain{|\textit{main}|}|\\
\end{tabular}
\end{center}
%
If |\jobname| does not match the argument \textit{main} of |\childdocmain|,
it is assumed that |\jobname| points to the child file to be compiled.
When using |\childdocmain| with the main file specified as argument,
it suffices to start a child file
with just |\input{|\textit{main}|}|
without loading of the package and using |\childdocof|.
If instead all processing is done
with the appropriate \textsf{childdoc} directives,
the argument of \textit{main} of |\childdocmain| can be empty.

An alternative version of the command line processing described
in \secref{sec:commandline} using the detection mechanism reads:
%
\begin{center}
|... -jobname "|\textit{target}|" "|[\textit{flags}]%
[|\def\jobname{|\textit{dest}|}|]|\input{|\textit{main}|}"|
\end{center}

%%%%%%%%%%%%%%%%%%%%%%%%%%%%%%%%%%%%%%%%%%%%%%%%%%%%%%%%%%%%%%%%%%%%%%%%%%%%%%%%
\subsection{Manual Code}
\label{sec:manual}

In case one cannot be certain whether the definitions file |childdoc.def|
is installed on the target \TeX{} distribution
and one prefers not to ship it,
it is conceivable to paste a few relevant commands into the sources.

To that end, drop all statements |\input{childdoc.def}|
and perform the replacements as outlined below.
Instead of |\childdocmain{|\textit{main}|}| add the following code
to the top of the main file:
%
\begin{center}
\begin{tabular}{l}
|\||ifdefined\childdocname\endinput\||fi\newif\ifchilddoc|\\
|\edef\childdocname{\scantokens\expandafter{\jobname\noexpand}}|\\
|\def\childdocmain{|\textit{main}|}\||ifx\childdocmain\childdocname\||else|\\
|\childdoctrue\includeonly{\childdocname}\let\jobname\childdocmain\||fi|\\
\end{tabular}
\end{center}
%
Instead of |\childdocof{|\textit{main}|}| just include the main file
at the top of each child file:
%
\begin{center}
|\input{|\textit{main}|}|
\end{center}
%
A simple redirection |\childdocforward{|\textit{dest}|}| is achieved by:
%
\begin{center}
|\def\jobname{|\textit{dest}|}\input{\jobname}|
\end{center}
%
The redirection with prefix
|\childdocforwardprefix[|\textit{prefix}|]{|\textit{dest}|}|
is accomplished by:
%
\begin{center}
\begin{tabular}{l}
|{\edef\jobname{\scantokens\expandafter{\jobname\noexpand}}|\\
|\def\redirectjob |\textit{prefix}|#1~~~{\gdef\jobname{|\textit{dest}|#1}}|\\
|\expandafter\redirectjob\jobname~~~}\input{\jobname}|
\end{tabular}
\end{center}

In an alternative approach,
child documents can be compiled by a specific command line
without additional code or specific definitions:
%
\begin{center}
|... -jobname "|\textit{target}|" "|[\textit{flags}]%
|\includeonly{|\textit{dest}|}\input{|\textit{main}|}"|
\end{center}
%

%%%%%%%%%%%%%%%%%%%%%%%%%%%%%%%%%%%%%%%%%%%%%%%%%%%%%%%%%%%%%%%%%%%%%%%%%%%%%%%%
%%%%%%%%%%%%%%%%%%%%%%%%%%%%%%%%%%%%%%%%%%%%%%%%%%%%%%%%%%%%%%%%%%%%%%%%%%%%%%%%
\section{Information}

%%%%%%%%%%%%%%%%%%%%%%%%%%%%%%%%%%%%%%%%%%%%%%%%%%%%%%%%%%%%%%%%%%%%%%%%%%%%%%%%
\subsection{Copyright}

Copyright \copyright{} 2017--2018 Niklas Beisert

This work may be distributed and/or modified under the
conditions of the \LaTeX{} Project Public License, either version 1.3
of this license or (at your option) any later version.
The latest version of this license is in
  \url{http://www.latex-project.org/lppl.txt}
and version 1.3 or later is part of all distributions of \LaTeX{}
version 2005/12/01 or later.

This work has the LPPL maintenance status `maintained'.

The Current Maintainer of this work is Niklas Beisert.

This work consists of the files |README.txt|, |childdoc.ins| and |childdoc.dtx|
as well as the derived files |childdoc.def|, |cdocsamp.tex|
with |cdocsch1.tex|, |cdocsch2.tex|, |cdocspt3.tex|, |cdocspt4.tex|,
|cdocsdrf.tex|, |cdocsfn1.tex|, |cdocsfn2.tex|
as well as |childdoc.pdf|.

%%%%%%%%%%%%%%%%%%%%%%%%%%%%%%%%%%%%%%%%%%%%%%%%%%%%%%%%%%%%%%%%%%%%%%%%%%%%%%%%
\subsection{Files and Installation}

The package consists of the files:
%
\begin{center}
\begin{tabular}{ll}
    |README.txt|   & readme file \\
    |childdoc.ins| & installation file \\
    |childdoc.dtx| & source file \\
    |childdoc.def| & definition file \\
    |cdocsamp.tex| & sample main file \\
    |cdocsch1.tex| & sample include file \\
    |cdocsch2.tex| & sample include file \\
    |cdocspt3.tex| & sample part file \\
    |cdocspt4.tex| & sample part file \\
    |cdocsdrf.tex| & sample redirection file \\
    |cdocsfn1.tex| & sample redirection file \\
    |cdocsfn2.tex| & sample redirection file \\
    |childdoc.pdf| & manual
\end{tabular}
\end{center}
%
The distribution consists of the files
|README.txt|, |childdoc.ins| and |childdoc.dtx|.
%
\begin{itemize}
\item
Run (pdf)\LaTeX{} on |childdoc.dtx|
to compile the manual |childdoc.pdf| (this file).
\item
Run \LaTeX{} on |childdoc.ins| to create the definitions file |childdoc.def|
and the sample |cdocsamp.tex| with include files
|cdocsch1.tex|, |cdocsch2.tex|, |cdocspt3.tex|, |cdocspt4.tex|,
|cdocsdrf.tex|, |cdocsfn1.tex|, |cdocsfn2.tex|.
Then copy the file |childdoc.def| to an appropriate directory of your \LaTeX{}
distribution, e.g.\ \textit{texmf-root}|/tex/latex/childdoc|.
\end{itemize}

%%%%%%%%%%%%%%%%%%%%%%%%%%%%%%%%%%%%%%%%%%%%%%%%%%%%%%%%%%%%%%%%%%%%%%%%%%%%%%%%
\subsection{Related CTAN Packages}

There are several other packages which offer a similar functionality:
%
\begin{itemize}
\item
The packages
\href{http://ctan.org/pkg/docmute}{\textsf{docmute}},
\href{http://ctan.org/pkg/includex}{\textsf{includex}} and
\href{http://ctan.org/pkg/standalone}{\textsf{standalone}}
provide commands to include only the document body of
a child file thus allowing both files to be compiled individually.
\item
The packages \href{http://ctan.org/pkg/subdocs}{\textsf{subdocs}}
and \href{http://ctan.org/pkg/subfiles}{\textsf{subfiles}}
provide structures in which the main and child documents can be
encapsulated and allowing them to be compiled individually.
The inclusion mechanism is different from the conventional |\include|.
\item
The package \href{http://ctan.org/pkg/combine}{\textsf{combine}}
is an elaborate solution to combine several documents into one.
\end{itemize}
%
See also the CTAN topic \href{http://ctan.org/topic/subdocs}{\textsf{subdocs}}
for further related packages.
The present package differs from the above solutions in that
a document structure constructed with the conventional |\include| mechanism
just needs two extra commands at the top of every file
such that all constituent files can be compiled individually.

%%%%%%%%%%%%%%%%%%%%%%%%%%%%%%%%%%%%%%%%%%%%%%%%%%%%%%%%%%%%%%%%%%%%%%%%%%%%%%%%
%\subsection{Feature Suggestions}
%
%The following is a list of features which may be useful for future
%versions of this package:
%%
%\begin{itemize}
%\item
%\ldots
%\end{itemize}

%%%%%%%%%%%%%%%%%%%%%%%%%%%%%%%%%%%%%%%%%%%%%%%%%%%%%%%%%%%%%%%%%%%%%%%%%%%%%%%%
\subsection{Revision History}

%%%%%%%%%%%%%%%%%%%%%%%%%%%%%%%%%%%%%%%%
\paragraph{v2.0:} 2018/12/30

\begin{itemize}
\item
immediate forward processing
\item
added |\childdocby| mechanism
\item
manual restructured
\end{itemize}

%%%%%%%%%%%%%%%%%%%%%%%%%%%%%%%%%%%%%%%%
\paragraph{v1.6:} 2018/01/17

\begin{itemize}
\item
application for development of include files
\item
corrections to manual
\end{itemize}

%%%%%%%%%%%%%%%%%%%%%%%%%%%%%%%%%%%%%%%%
\paragraph{v1.5:} 2017/05/21

\begin{itemize}
\item
more complete structuring introduced
\item
|\childdocof| introduced
\item
|\childdoc| renamed to |\childdocmain|
\item
|\childredirect| renamed to |\childdocforward| and |\childdocforwardprefix|
and functionality expanded
\end{itemize}

%%%%%%%%%%%%%%%%%%%%%%%%%%%%%%%%%%%%%%%%
\paragraph{v1.0:} 2017/04/27

\begin{itemize}
\item
manual and install package
\item
first version published on CTAN
\end{itemize}

%%%%%%%%%%%%%%%%%%%%%%%%%%%%%%%%%%%%%%%%
\paragraph{v0.6:} 2017/04/26

\begin{itemize}
\item
redirection mechanism added
\end{itemize}

%%%%%%%%%%%%%%%%%%%%%%%%%%%%%%%%%%%%%%%%
\paragraph{v0.5:} 2017/04/26

\begin{itemize}
\item
functionality in definition file
\end{itemize}


%%%%%%%%%%%%%%%%%%%%%%%%%%%%%%%%%%%%%%%%%%%%%%%%%%%%%%%%%%%%%%%%%%%%%%%%%%%%%%%%
%%%%%%%%%%%%%%%%%%%%%%%%%%%%%%%%%%%%%%%%%%%%%%%%%%%%%%%%%%%%%%%%%%%%%%%%%%%%%%%%
%%%%%%%%%%%%%%%%%%%%%%%%%%%%%%%%%%%%%%%%%%%%%%%%%%%%%%%%%%%%%%%%%%%%%%%%%%%%%%%%
\appendix

\settowidth\MacroIndent{\rmfamily\scriptsize 000\ }

 \DocInput{childdoc.dtx}

\end{document}
%</driver>
% \fi
%
% %%%%%%%%%%%%%%%%%%%%%%%%%%%%%%%%%%%%%%%%%%%%%%%%%%%%%%%%%%%%%%%%%%%%%%%%%%%%%%
% %%%%%%%%%%%%%%%%%%%%%%%%%%%%%%%%%%%%%%%%%%%%%%%%%%%%%%%%%%%%%%%%%%%%%%%%%%%%%%
% \section{Sample}
%\iffalse
%<*samplemain>
%\fi
%
% The following presents a sample document
% with two chapters, two parts, a title page,
% a compile flag as well as three forwarding files to set the flag.
% It consists of eight |.tex| files:
% \begin{center}
% \begin{tabular}{ll}
% |cdocsamp.tex|&main file\\
% |cdocsch1.tex|&include file for chapter 1\\
% |cdocsch2.tex|&include file for chapter 2\\
% |cdocspt3.tex|&include file for part 3\\
% |cdocspt4.tex|&include file for part 4\\
% |cdocsdrf.tex|&forwarding file for main file in draft mode\\
% |cdocsfi1.tex|&forwarding file for final version of chapter 1\\
% |cdocsfi2.tex|&forwarding file for final version of chapter 2\\
% \end{tabular}
% \end{center}
% Each of the eight files can be compiled directly by the \LaTeX{} compiler.
%
% %%%%%%%%%%%%%%%%%%%%%%%%%%%%%%%%%%%%%%
% \paragraph{Main File.}
%
% The main file is called |cdocsamp.tex|.
%
% Load the \textsf{childdoc} definitions and
% declare the filename for the main document:
%    \begin{macrocode}
\input{childdoc.def}
\childdocmain{}
%    \end{macrocode}

% Optional override for |\version| flag:
%    \begin{macrocode}
%%\ifchilddoc\else\providecommand{\version}{draft}\fi
%    \end{macrocode}

% Define the default values for the |\version| flag
% (|final| for the main file and |draft| for childs):
%    \begin{macrocode}
\ifchilddoc
\providecommand{\version}{draft}
\else
\providecommand{\version}{final}
\fi
%    \end{macrocode}

% Load the standard document class:
%    \begin{macrocode}
\documentclass[12pt]{article}
%    \end{macrocode}

% Start the document body:
%    \begin{macrocode}
\begin{document}
%    \end{macrocode}

% Declare a title page.
% Print title, part of document being processed and version flag:
%    \begin{macrocode}
\addtocounter{page}{-1}
\begin{center}
{\LARGE\bfseries{}childdoc example\par}
\vspace{1cm}
\ifchilddoc
\ifchilddocmanual part\else chapter\fi:
`\childdocname' of `\childdocjob'\par
\else
main document: `\childdocjob'\par
\fi
version: \version\par
\end{center}
\newpage
%    \end{macrocode}

% Manually include selected file,
% otherwise process as usual:
%    \begin{macrocode}
\ifchilddocmanual
\section*{part `\childdocname'}
\input{\childdocname}
\else
%    \end{macrocode}

% Include the two chapters:
%    \begin{macrocode}
\include{cdocsch1}
\include{cdocsch2}
%    \end{macrocode}

% Include the two parts unless only chapters should be displayed:
%    \begin{macrocode}
\ifchilddoc\else
\section{part three}
\input{cdocspt3}
\section{part four}
\input{cdocspt4}
\fi
%    \end{macrocode}

% Process as usual until here:
%    \begin{macrocode}
\fi
%    \end{macrocode}

% End of document body:
%    \begin{macrocode}
\end{document}
%    \end{macrocode}
%\iffalse
%</samplemain>
%\fi
%
% %%%%%%%%%%%%%%%%%%%%%%%%%%%%%%%%%%%%%%
% \paragraph{Chapter Include Files.}
%
% The include files are called |cdocsch1.tex| and |cdocsch2.tex|.
%
%\iffalse
%<*samplechap1|samplechap2>
%\fi

% Optional override for |\version| flag:
%    \begin{macrocode}
%%\providecommand{\version}{final}
%    \end{macrocode}

% Include the main document:
%    \begin{macrocode}
\input{childdoc.def}
\childdocof{cdocsamp}
%    \end{macrocode}

%\iffalse
%</samplechap1|samplechap2>
%\fi
%
%\iffalse
%<*samplechap1>
%\fi
% Some text for chapter 1:
%    \begin{macrocode}
\section{one}
some text in chapter one
%    \end{macrocode}

%\iffalse
%</samplechap1>
%\fi
% Some text for chapter 2:
%\iffalse
%<*samplechap2>
%\fi
%    \begin{macrocode}
\section{two}
more text in chapter two
%    \end{macrocode}

%\iffalse
%</samplechap2>
%\fi
%
% %%%%%%%%%%%%%%%%%%%%%%%%%%%%%%%%%%%%%%
% \paragraph{Part Include Files.}
%
% The include files are called |cdocspt3.tex| and |cdocspt4.tex|.
%
%\iffalse
%<*samplepart3|samplepart4>
%\fi

% Optional override for |\version| flag:
%    \begin{macrocode}
%%\providecommand{\version}{final}
%    \end{macrocode}

% Include the main document:
%    \begin{macrocode}
\input{childdoc.def}
\childdocby{cdocsamp}
%    \end{macrocode}

%\iffalse
%</samplepart3|samplepart4>
%\fi
%
%\iffalse
%<*samplepart3>
%\fi
% Some text for part 3:
%    \begin{macrocode}
some text in part three
%    \end{macrocode}

%\iffalse
%</samplepart3>
%\fi
% Some text for part 4:
%\iffalse
%<*samplepart4>
%\fi
%    \begin{macrocode}
more text in part four
%    \end{macrocode}

%\iffalse
%</samplepart4>
%\fi
%
% %%%%%%%%%%%%%%%%%%%%%%%%%%%%%%%%%%%%%%
% \paragraph{Forwarding for a Complete Draft.}
%
% The following forwarding file |cdocsdrf.tex|
% compiles the main document in draft mode:
%\iffalse
%<*sampledraft>
%\fi
%    \begin{macrocode}
\def\version{draft}
\input{childdoc.def}
\childdocforward{cdocsamp}
%    \end{macrocode}

%\iffalse
%</sampledraft>
%\fi
%
% %%%%%%%%%%%%%%%%%%%%%%%%%%%%%%%%%%%%%%
% \paragraph{Forwarding for Final Version of the Chapters.}
%
% The following forwarding files |cdocsfn1.tex| and |cdocsfn2.tex|
% (with identical content)
% compile the final versions of the child documents
% |cdocsch1.tex| and |cdocsch2.tex|, respectively:
%\iffalse
%<*samplefinal>
%\fi
%    \begin{macrocode}
\def\version{final}
\input{childdoc.def}
\childdocforwardprefix[cdocsamp]{cdocsfn}{cdocsch}
%    \end{macrocode}

%\iffalse
%</samplefinal>
%\fi
%
% %%%%%%%%%%%%%%%%%%%%%%%%%%%%%%%%%%%%%%
% \paragraph{Command Line Processing.}
%
% The following three command lines generate the output files
% |cdocscld|, |cdocscl1| and |cdocscl2|
% which should be identical to
% |cdocsdrf|, |cdocsch1| and |cdocsfn2|, respectively:
% \begin{center}
% \begin{tabular}{l}
% |latex -jobname cdocscld \|\\
% |  "\def\version{draft}\input{childdoc.def}\childdocforward{cdocsamp}"|\\
% |latex -jobname cdocscl1 \|\\
% |  "\input{childdoc.def}\childdocforward[cdocsamp]{cdocsch1}"|\\
% |latex -jobname cdocscl2 \|\\
% |  "\def\version{final}\input{childdoc.def}\childdocforward{cdocsch2}"|
% \end{tabular}
% \end{center}
% Note that the trailing backslash on each first line
% merely continues the input to the second line
% (for convenient cut ant paste).
% Furthermore, the command |latex| can be replaced by any
% of its alternative versions such as |pdflatex|.
%
% %%%%%%%%%%%%%%%%%%%%%%%%%%%%%%%%%%%%%%%%%%%%%%%%%%%%%%%%%%%%%%%%%%%%%%%%%%%%%%
% %%%%%%%%%%%%%%%%%%%%%%%%%%%%%%%%%%%%%%%%%%%%%%%%%%%%%%%%%%%%%%%%%%%%%%%%%%%%%%
% \section{Implementation}
%\iffalse
%<*package>
%\fi
%
% This section describes the definitions file |childdoc.def|.

% The definitions cannot be loaded using |\usepackage| or |\RequirePackage|
% which has a mechanism to prevent loading a style file more than once.
% When loading the definitions by means of |\input|
% multiple instances have to be prevented manually:
%\iffalse
%This code needs to be before the `\ProvidesFile' directive
%which is defined at the beginning of this file.
%Therefore it is also placed there and commented out here.
%</package>
%<*discard>
%\fi
%    \begin{macrocode}
\ifdefined\childdocmain\endinput\fi
%    \end{macrocode}
%\iffalse
%</discard>
%<*package>
%\fi
%
% \macro{\ifchilddoc}
% \macro{\ifchilddocmanual}
% The conditional |\ifchilddoc| tells whether a
% child (true) or main (false) document is being compiled.
% The conditional |\ifchilddocmanual| tells whether
% the |\includeonly| mechanism is used (false) or
% the selection of child files must be performed manually (true).
% The definitions initialise to false:
%    \begin{macrocode}
\newif\ifchilddoc
\newif\ifchilddocmanual
%    \end{macrocode}

% \macro{\childdocname}
% \macro{\childdocjob}
% The macro |\childdocname| stores the name of the main document
% to be compiled. The macro |\childdocjob| stores the name of
% the document on which the \LaTeX{} compiler was originally invoked.
% The content of |\jobname| cannot be compared
% to filenames specified in the source due to different catcodes.
% The following code rescans |\jobname|, stores the result
% in |\childdocname| and saves a copy in |\childdocjob|:
%    \begin{macrocode}
\edef\childdocname{\scantokens\expandafter{\jobname\noexpand}}
\let\childdocjob\childdocname
%    \end{macrocode}

% \macro{\childdocdisable}
% The macro |\childdocdisable| prevents the main file
% from being processed more than once.
% At this stage, the main document command |\childdocmain|
% is assumed to be called once again where it should do nothing.
% Any subsequent call to it should prevent
% a secondary processing of the main document
% It overwrites the forwarding commands
% |\childdocof| and |\childdocforward|
% with empty macros to prevent further inclusions of the main document:
%    \begin{macrocode}
\newcommand{\childdocdisable}
{
  \renewcommand{\childdocmain}[1]{\renewcommand{\childdocmain}[1]{\endinput}}
  \renewcommand{\childdocof}[1]{}
  \renewcommand{\childdocby}[2][]{}
  \renewcommand{\childdocforward}[2][]{}
  \renewcommand{\childdocdisable}{}
}
%    \end{macrocode}

% \macro{\childdocmain}
% The macro |\childdocmain| is to be called at the top of the main file
% with nothing or the main filename (without extension) as argument.
% First, it breaks loops.
% If the argument is not empty and does not match |\childdocname|
% (which is set by the first inclusion of |childdoc.def|),
% |\ifchilddoc| is set to true, |\includeonly| is applied to the child file
% and |\jobname| is set to the main file
% (for proper handling of |.aux| files):
%    \begin{macrocode}
\newcommand{\childdocmain}[1]
{
  \childdocdisable\childdocmain{}
  \if?#1?\else
    \begingroup
      \def\childdoctmp{#1}
      \ifx\childdoctmp\childdocname
        \def\childdoctmp{}
      \else
        \def\childdoctmp
        {
          \childdoctrue
          \includeonly{\childdocname}
          \def\childdocjob{#1}
          \def\jobname{#1}
        }
      \fi
      \expandafter
    \endgroup
    \childdoctmp
  \fi
}
%    \end{macrocode}

% \macro{\childdocof}
% The command |\childdocof| redirects
% compilation to the main file |#1|.
%    \begin{macrocode}
\newcommand{\childdocof}[1]
{
  \childdocdisable
  \childdoctrue
  \includeonly{\childdocname}
  \def\jobname{#1}
  \def\childdocjob{#1}
  \input{#1}
}
%    \end{macrocode}

% \macro{\childdocby}
% The command |\childdocby| ....
%    \begin{macrocode}
\newcommand{\childdocby}[2][]
{
  \childdocdisable
  \childdoctrue
  \childdocmanualtrue
  \if?#1?\else
    \def\jobname{#2}
  \fi
  \def\childdocjob{#2}
  \input{#2}
  \endinput
}
%    \end{macrocode}

% \macro{\childdocforward}
% The command |\childdocforward| redirects
% compilation to the main file or
% (if the optional argument is given) a child file.
% Parameters are set as if the main file
% or a child file starting with |\childdocof| was compiled.
% Then compilation is handed over to the main file:
%    \begin{macrocode}
\newcommand{\childdocforward}[2][]
{
  \begingroup
    \if?#1?
      \def\childdoctmp
      {
        \def\childdocname{#2}
        \def\childdocjob{#2}
        \def\jobname{#2}
        \input{#2}
        \endinput
      }
    \else
      \def\childdoctmp
      {
        \childdocdisable
        \def\childdocname{#2}
        \childdoctrue
        \includeonly{#2}
        \def\childdocjob{#1}
        \def\jobname{#1}
        \input{#1}
        \endinput
      }
    \fi
    \expandafter
  \endgroup
  \childdoctmp
}
%    \end{macrocode}

% \macro{\childdocforwardprefix}
% The command |\childdocforwardprefix| redirects
% compilation to the main or a child file by means of a pattern.
% The prefix |#1| in the current filename is replaced by |#2|
% and the suffix of the current filename is kept
% (it is assumed that the filename does not contain the substring `|~~~|'
% which is used as a delimiter).
% Compilation is handed over to the new file by |\childdocforward|:
%    \begin{macrocode}
\newcommand{\childdocforwardprefix}[3][]
{
  \begingroup
    \def\childdocextract #2##1~~~{\def\childdoctmp{\childdocforward[#1]{#3##1}}}
    \expandafter\childdocextract\childdocname~~~
    \expandafter
  \endgroup
  \childdoctmp
}
%    \end{macrocode}

% \macro{\childdoc}
% The deprecated macro |\childdoc| is a legacy version of |\childdocmain|:
%    \begin{macrocode}
\newcommand{\childdoc}{\childdocmain}
%    \end{macrocode}

% \macro{\childdocredirect}
% The deprecated macro |\childdocredirect| is a legacy version
% of |\childdocforward| and |\childdocforwardprefix|:
%    \begin{macrocode}
\newcommand{\childdocredirect}[2][]
{
  \begingroup
    \if?#1?
      \def\childdoctmp{\childdocforward{#2}}
    \else
      \def\childdoctmp{\childdocforwardprefix{#1}{#2}}
    \fi
    \expandafter
  \endgroup
  \childdoctmp
}
%    \end{macrocode}

%\iffalse
%</package>
%\fi
%
\endinput
|\\
|\childdocforwardprefix[|\textit{main}|]{|\textit{prefix}|}{|\textit{dest}|}|
\end{tabular}
\end{center}
%
the destination file is determined by a pattern
depending on the current file:
To make this work, the current file must be called
`{\textit{prefix}\hspace{0.2em}\textit{suffix}}'
with \textit{prefix} matching precisely the argument.
Processing is then passed on to the file
`{\textit{dest}\hspace{0.2em}\textit{suffix}}'.
Surely, the same effect is achieved by
directly specifying the
argument `{\textit{dest}\hspace{0.2em}\textit{suffix}}'
in the first form.
However, that requires to set up a different file
for each child. With the alternative form of the command
all these files can have exactly the same content
which simplifies setting them up and maintaining them.

For example, the following file |draft.tex|
with a compilation flag |\version| as described in \secref{sec:flags}
compiles the main document as a draft:
%
\begin{center}
\begin{tabular}{l}
|\def\version{draft}|\\
|% \iffalse
%
% childdoc.dtx Copyright (C) 2017-2018 Niklas Beisert
%
% This work may be distributed and/or modified under the
% conditions of the LaTeX Project Public License, either version 1.3
% of this license or (at your option) any later version.
% The latest version of this license is in
%   http://www.latex-project.org/lppl.txt
% and version 1.3 or later is part of all distributions of LaTeX
% version 2005/12/01 or later.
%
% This work has the LPPL maintenance status `maintained'.
%
% The Current Maintainer of this work is Niklas Beisert.
%
% This work consists of the files childdoc.dtx and childdoc.ins
% and the derived files childdoc.def and cdocsamp.tex with
% cdocsch1.tex, cdocsch2.tex, cdocsdrf.tex, cdocsfn1.tex, cdocsfn2.tex.
%
%<package>\ifdefined\childdocmain\endinput\fi
%<package>\ProvidesFile{childdoc.def}[2018/12/30 v2.0 child document driver]
%<samplemain>\ProvidesFile{cdocsamp.tex}[2018/12/30 v2.0 sample for childdoc]
%<*driver>
%\ProvidesFile{childdoc.drv}[2018/12/30 v2.0 childdoc reference manual file]
\PassOptionsToClass{10pt,a4paper}{article}
\documentclass{ltxdoc}

\usepackage[margin=35mm]{geometry}
\usepackage{hyperref}
\usepackage{hyperxmp}
\usepackage[usenames]{color}

\hypersetup{colorlinks=true}
\hypersetup{pdfstartview=FitH}
\hypersetup{pdfpagemode=UseNone}
\hypersetup{pdfsource={}}
\hypersetup{pdflang={en-UK}}
\hypersetup{pdfcopyright={Copyright 2017-2018 Niklas Beisert.
  This work may be distributed and/or modified under the
  conditions of the LaTeX Project Public License, either version 1.3
  of this license or (at your option) any later version.}}
\hypersetup{pdflicenseurl={http://www.latex-project.org/lppl.txt}}
\hypersetup{pdfcontactaddress={ETH Zurich, ITP, HIT K,
  Wolfgang-Pauli-Strasse 27}}
\hypersetup{pdfcontactpostcode={8093}}
\hypersetup{pdfcontactcity={Zurich}}
\hypersetup{pdfcontactcountry={Switzerland}}
\hypersetup{pdfcontactemail={nbeisert@itp.phys.ethz.ch}}
\hypersetup{pdfcontacturl={http://people.phys.ethz.ch/\xmptilde nbeisert/}}

\newcommand{\secref}[1]{\hyperref[#1]{section \ref*{#1}}}

\parskip1ex
\parindent0pt
\let\olditemize\itemize
\def\itemize{\olditemize\parskip0pt}

\begin{document}

\title{The \textsf{childdoc} Package}
\hypersetup{pdftitle={The childdoc Package}}
\author{Niklas Beisert\\[2ex]
  Institut f\"ur Theoretische Physik\\
  Eidgen\"ossische Technische Hochschule Z\"urich\\
  Wolfgang-Pauli-Strasse 27, 8093 Z\"urich, Switzerland\\[1ex]
  \href{mailto:nbeisert@itp.phys.ethz.ch}
  {\texttt{nbeisert@itp.phys.ethz.ch}}}
\hypersetup{pdfauthor={Niklas Beisert}}
\hypersetup{pdfsubject={Manual for the LaTeX2e Package childdoc}}
\date{30 December 2018, \textsf{v2.0}}
\maketitle

\begin{abstract}\noindent
\textsf{childdoc} is a \LaTeXe{} package
that enables the direct compilation
of document sections included by |\include|
to individual files.
\end{abstract}

\begingroup
\parskip0ex
\tableofcontents
\endgroup

%%%%%%%%%%%%%%%%%%%%%%%%%%%%%%%%%%%%%%%%%%%%%%%%%%%%%%%%%%%%%%%%%%%%%%%%%%%%%%%%
%%%%%%%%%%%%%%%%%%%%%%%%%%%%%%%%%%%%%%%%%%%%%%%%%%%%%%%%%%%%%%%%%%%%%%%%%%%%%%%%
\section{Introduction}

\LaTeX{} provides a mechanism to structure a large document (such as a book)
into a main file and several child files (containing the chapters)
using the |\include| command.
This mechanism is beneficial for documents
which span hundreds of pages in order to
make the source file(s) more manageable.
Moreover, compilation can be restricted to
selected child files by means of the |\includeonly| command.
The latter feature can be used to reduce the compilation time while editing
(this was significantly more useful in the earlier days of \LaTeX{})
or to generate a smaller document which is easier to navigate.
Another application of |\includeonly| is to generate
documents consisting of selected parts of the complete document.

However, there are a few drawbacks of the plain |\include| mechanism:
\begin{itemize}
\item
The child files cannot be compiled on their own,
they can only be compiled via the main file.
A naive editing environment
(such as a text editor with an option
to have the current file processed by \LaTeX)
may require one to switch to the main file before compiling;
attempting to compile the child file produces errors.
\item
The main file must be modified (each time)
to adjust the |\includeonly| command
to the present needs. This easily leaves the main file in a messy state.
\item
The generated document will always carry the filename
of the main document. This is inconvenient if
several child files are to be compiled and
to be kept for distribution.
\end{itemize}

The present package provides a simple interface
to make child files individually compilable by \LaTeX{}.
Compiling a child file then has the same effect as compiling
the main file with an |\includeonly| command
to select the appropriate child.
Moreover the generated document will carry the name of the child
rather than the main file.
This resolves all three above issues.

This feature is meant to make the editing of books,
thesis documents and lecture notes somewhat more convenient.
However, the package can also be used efficiently for
composing a series of documents (such as exercise sheets)
which are typically distributed individually.
It then assists the author in generating the individual documents
(potentially in different versions)
as well as a document containing the collected series.
Another application is in developing style files
or other kinds of included material
where compilation of the style file could redirect
to a sample or test file.

%%%%%%%%%%%%%%%%%%%%%%%%%%%%%%%%%%%%%%%%%%%%%%%%%%%%%%%%%%%%%%%%%%%%%%%%%%%%%%%%
%%%%%%%%%%%%%%%%%%%%%%%%%%%%%%%%%%%%%%%%%%%%%%%%%%%%%%%%%%%%%%%%%%%%%%%%%%%%%%%%
\section{Usage}

First of all, the package \textsf{childdoc} is \emph{not} a standard
\LaTeXe{} |.sty| style file! Therefore it needs to be invoked in
a non-standard way.

%%%%%%%%%%%%%%%%%%%%%%%%%%%%%%%%%%%%%%%%%%%%%%%%%%%%%%%%%%%%%%%%%%%%%%%%%%%%%%%%
\subsection{Included Files}
\label{sec:include}

%%%%%%%%%%%%%%%%%%%%%%%%%%%%%%%%%%%%%%%%
\DescribeMacro{\childdocmain}
To use the package, add the commands
\begin{center}
\begin{tabular}{l}
|\input{childdoc.def}|\\
|\childdocmain{}|\\
\end{tabular}
\end{center}
at the very top of the main \LaTeX{} file,
in particular \emph{before} the |\documentclass| statement!
The argument of |\childdocmain| should be left empty
(but it must be present).

%%%%%%%%%%%%%%%%%%%%%%%%%%%%%%%%%%%%%%%%
\DescribeMacro{\childdocof}
Furthermore, add the commands
\begin{center}
\begin{tabular}{l}
|\input{childdoc.def}|\\
|\childdocof{|\textit{main}|}|\\
\end{tabular}
\end{center}
at the top of every child file \textit{child}
which is included by |\include{|\textit{child}|}|
from within the main file
(or at least for those files to be compiled individually).
The argument \textit{main} must be the filename of the main file.

There are a couple of
considerations in setting up the main and child documents:

%%%%%%%%%%%%%%%%%%%%%%%%%%%%%%%%%%%%%%%%
\paragraph{Restrictions.}

Please note the following restrictions:
\begin{itemize}
\item
|\childdocmain| must be called with one argument \textit{main}
to ensure compatibility with earlier version of the package.
It must either be empty (|\childdocmain{}|)
or precisely match the filename of the main file in which it is specified.
See \secref{sec:detection} for further information.
\item
The filename \textit{main} must be specified without the |.tex| extension.
\item
The filename \textit{main} is case sensitive
(even in case-insensitive file systems)
due to internal string comparison.
\item
The argument \textit{main} should be fully expanded, it cannot be a macro.
\item
Subdirectories and special characters should be avoided in filenames.
\item
The command |\childdocmain{|\textit{main}|}| must be followed by a whitespace.
It should not be followed immediately by another command
or by a comment mark `|%|'.
This is because the \TeX{} parser reads the token immediately following
the argument of |\childdocmain| and puts it
at the beginning of every child section;
however, a white\-space is ignored.
\end{itemize}

%%%%%%%%%%%%%%%%%%%%%%%%%%%%%%%%%%%%%%%%
\paragraph{Content of Main File.}

It is advisable to place all content in the child files included by |\include|.
Any output contained in the main file will appear in all child documents
unless suppressed manually;
it cannot be suppressed automatically by the |\includeonly| directive
and thus should normally be avoided.
A method to include some content in the main file
by means of conditional processing is described in \secref{sec:conditional}.

%%%%%%%%%%%%%%%%%%%%%%%%%%%%%%%%%%%%%%%%
\paragraph{Page Numbering.}

When only a part of the document is compiled,
the appropriate numbering of pages
(as well as other status parameters)
is determined from the |.aux| files.
The latter contain information from previous passes.
However this information needs to propagate through
all intermediate child documents.
Therefore the page numbering in child documents may well
be inconsistent until the complete document is compiled at least once.

A useful (if unconventional) way to always ensure a consistent
page numbering is to restart the numbering in each child document
and denote the pages by `\textit{child}|.|\textit{page}'
where \textit{child} represents the chapter/section number of the child file.
This can be achieved by the command
|\numberwithin{page}{|\textit{child}|}|
of the \textsf{amsmath} package
where \textit{child} can be |chapter| or |section|
depending on the chosen structuring.
Alternatively, one can modify the macro |\thepage| appropriately
and reset the counter |page| at the start of each child file.

%%%%%%%%%%%%%%%%%%%%%%%%%%%%%%%%%%%%%%%%%%%%%%%%%%%%%%%%%%%%%%%%%%%%%%%%%%%%%%%%
\subsection{Conditional Processing}
\label{sec:conditional}

The package provides a mechanism to compile different versions
of a document. To customise the versions further some conditional processing
can come in handy to distinguish which version is being compiled.
The package provides two macros to describe the compilation context:

%%%%%%%%%%%%%%%%%%%%%%%%%%%%%%%%%%%%%%%%
\DescribeMacro{\ifchilddoc}
The conditional |\ifchilddoc| distinguishes between the compilation of
child documents and the main document:
%
\begin{center}
|\ifchilddoc |\textit{child-code}| |[|\||else |\textit{main-code}]| \||fi|
\end{center}

%%%%%%%%%%%%%%%%%%%%%%%%%%%%%%%%%%%%%%%%
\DescribeMacro{\childdocname}
\DescribeMacro{\childdocjob}
The macro |\childdocname| contains the filename (without extension)
of the main or child file being processed.
Note that |\childdocjob| will always contain the name of the main file.

%%%%%%%%%%%%%%%%%%%%%%%%%%%%%%%%%%%%%%%%
\paragraph{Title Page.}

Conditional processing can be used to include a title or banner page
in the main document when proper precautions are taken.
Importantly, the code in the main file should ensure that the page counter
(as well as other status parameters which are stored in the |.aux| files)
takes the same value after the conditional processing.
Otherwise the page numbers may take divergent values
depending on which part is compiled.

For example, a title page could be declared by:
%
\begin{center}
\begin{tabular}{l}
|\ifchilddoc\||else|\\
|\addtocounter{page}{-1}|\\
\textit{code for title page}\\
|\newpage|\\
|\||fi|
\end{tabular}
\end{center}
%
A banner page for the child documents can be generated by:
%
\begin{center}
\begin{tabular}{l}
|\ifchilddoc|\\
|\addtocounter{page}{-1}|\\
\textit{code for banner page}\\
|\newpage|\\
|\||fi|
\end{tabular}
\end{center}
%
Here one could write a message such as:
\begin{center}
|This is the part \childdocname{} of \childdocjob{}.|
\end{center}

%%%%%%%%%%%%%%%%%%%%%%%%%%%%%%%%%%%%%%%%%%%%%%%%%%%%%%%%%%%%%%%%%%%%%%%%%%%%%%%%
\subsection{Flags}
\label{sec:flags}

The package makes it easy to generate different versions
of the main or child documents.
To this end compilation flags can be defined
and assigned different default values.
They will be particularly useful in conjunction
with the forwarding mechanism described in \secref{sec:forward}.

For example, it may be useful to have a flag |\version|
which can be set to |draft| or |final|.
The document source will contain some conditional code
depending on the value of |\version|.
Suppose further, the flag should default to |final| for the main file
and to |draft| for child files
which is a natural assignment for editing the document.
This is achieved by placing the following code
in the preamble of the main document
(below the |\childdocmain| directive):
%
\begin{center}
\begin{tabular}{l}
|\ifchilddoc|\\
|\providecommand{\version}{draft}|\\
|\||else|\\
|\providecommand{\version}{final}|\\
|\||fi|
\end{tabular}
\end{center}
%
The definition by |\providecommand| makes sure
that previous definitions are not overwritten.
Further statements |\providecommand{\version}{...}|
can thus be added before the above code to override it.

For the main file, one might add a line
(between |\childdocmain| and the above block)
%
\begin{center}
|%\ifchilddoc\||else\providecommand{\version}{draft}\||fi|
\end{center}
%
which can be uncommented to produce a draft version.
Likewise one can add a line to the very top of a child file
(above the |\childdocof{|\textit{main}|}| directive)
%
\begin{center}
|%\providecommand{\version}{final}|
\end{center}
%
which can be uncommented to produce the final version of this child document.

%%%%%%%%%%%%%%%%%%%%%%%%%%%%%%%%%%%%%%%%%%%%%%%%%%%%%%%%%%%%%%%%%%%%%%%%%%%%%%%%
\subsection{Forwarding}
\label{sec:forward}

Different versions of the main or child documents
using compilation flags as described in \secref{sec:flags}
can be (permanently) stored in different files
for convenient compilation, viewing and distribution.
To this end, the package defines a command
to pass on compilation to a different file:

%%%%%%%%%%%%%%%%%%%%%%%%%%%%%%%%%%%%%%%%
\DescribeMacro{\childdocforward}
The command |\childdocforward| redirects processing to
another source file:
%
\begin{center}
\begin{tabular}{l}
|\input{childdoc.def}|\\
|\childdocforward[|\textit{main}|]{|\textit{dest}|}|\\
\end{tabular}
\end{center}
%
The argument \textit{dest} is the destination file
(without extension).
It should be the main file or one of the child files.
Note that further \textsf{childdoc} directives
such as |\childdocof| and |\childdocforward|
in the indicated file will be processed in this form.
The optional argument \textit{main}
passes on directly to the main file \textit{main}
while pretending to compile the child \textit{dest}.
This form behaves as if \textit{dest}
issues |\childdocof{|\textit{main}|}| right away,
and no further \textsf{childdoc} directives will be processed.

%%%%%%%%%%%%%%%%%%%%%%%%%%%%%%%%%%%%%%%%
\DescribeMacro{\...prefix}
In the alternative form |\childdocforwardprefix|,
%
\begin{center}
\begin{tabular}{l}
|\input{childdoc.def}|\\
|\childdocforwardprefix[|\textit{main}|]{|\textit{prefix}|}{|\textit{dest}|}|
\end{tabular}
\end{center}
%
the destination file is determined by a pattern
depending on the current file:
To make this work, the current file must be called
`{\textit{prefix}\hspace{0.2em}\textit{suffix}}'
with \textit{prefix} matching precisely the argument.
Processing is then passed on to the file
`{\textit{dest}\hspace{0.2em}\textit{suffix}}'.
Surely, the same effect is achieved by
directly specifying the
argument `{\textit{dest}\hspace{0.2em}\textit{suffix}}'
in the first form.
However, that requires to set up a different file
for each child. With the alternative form of the command
all these files can have exactly the same content
which simplifies setting them up and maintaining them.

For example, the following file |draft.tex|
with a compilation flag |\version| as described in \secref{sec:flags}
compiles the main document as a draft:
%
\begin{center}
\begin{tabular}{l}
|\def\version{draft}|\\
|\input{childdoc.def}|\\
|\childdocforward{|\textit{main}|}|
\end{tabular}
\end{center}
%
Likewise, the following files |final|\textit{nn}|.tex|
compile the final version of the child document
|child|\textit{nn}|.tex|:
%
\begin{center}
\begin{tabular}{l}
|\def\version{final}|\\
|\input{childdoc.def}|\\
|\childdocforwardprefix{final}{child}|
\end{tabular}
\end{center}
%

Note that when several versions of a main file and/or of each child file
are to be generated, it may be convenient to set up a |Makefile| or
shell script to automatise the process.

%%%%%%%%%%%%%%%%%%%%%%%%%%%%%%%%%%%%%%%%%%%%%%%%%%%%%%%%%%%%%%%%%%%%%%%%%%%%%%%%
\subsection{Command Line Processing}
\label{sec:commandline}

The effect of redirection files can also be achieved by invoking
the \LaTeX{} compiler with a more elaborate command line.
Most conveniently this should be done as part
of a shell script or a |Makefile|.

When using \textsf{childdoc} in the main file, the following
command lines effectively perform a redirection
(note that depending on the shell being used,
backslashes may have to be doubled: `|\|' $\to$ `|\\|'):
%
\begin{center}
|... -jobname "|\textit{target}|" |\\|"|[\textit{flags}]%
|\input{childdoc.def}\childdocforward[|\textit{main}|]{|\textit{dest}|}"|
\end{center}
%
Here \textit{target} is the name of the output file,
\textit{main} is the name of the main file
and \textit{dest} is the name of the main or child file to be processed
(all filenames without extensions).
The optional argument \textit{main} can be omitted
if \textit{main} matches \textit{dest}.
Optionally, compilation \textit{flags} can be defined via |\def| commands.
This command line makes the \TeX{} engine believe
it is compiling the file \textit{target}
whose content is specified as the latter parameter.
The provided code then forwards the processing to
\textit{main} or \textit{dest} as described in \secref{sec:forward}.

%%%%%%%%%%%%%%%%%%%%%%%%%%%%%%%%%%%%%%%%%%%%%%%%%%%%%%%%%%%%%%%%%%%%%%%%%%%%%%%%
\subsection{Include by Input}
\label{sec:input}

Including child documents by |\include| has some restrictions by design.
Most notably, the content of a child document always occupies
its own set of pages; pages cannot be shared between child documents.
Usually, this behaviour makes perfect sense
because each child document contain an essential part of the document.
However, in some situations it may be desirable to compose
a document from a collection of parts
without having mandatory page breaks between then.
For this case, the package
provides a mechanism to include parts
by |\input| which can also be processed individually.
However, by construction this mechanism
requires manual handling of the content to be output.

%%%%%%%%%%%%%%%%%%%%%%%%%%%%%%%%%%%%%%%%
\DescribeMacro{\ifchilddocmanual}
The main file should be prepared as usual, see \secref{sec:include}.
However, the document body must make a distinction
between processing of an individual part and of the main document, e.g.:
%
\begin{center}
\begin{tabular}{l}
|\ifchilddocmanual|\\
|\input{\childdocname}|\\
|\||else|\\
\textit{document body with }|\input{|\textit{part}|}|\\
|\||fi|
\end{tabular}
\end{center}
%
The conditional |\ifchilddocmanual| is true whenever
a part to be included by |\input| is being compiled,
and the name of the part is stored in |\childdocname|.

%%%%%%%%%%%%%%%%%%%%%%%%%%%%%%%%%%%%%%%%
\DescribeMacro{\childdocby}
Each part to be included by |\input| should start with:
%
\begin{center}
\begin{tabular}{l}
|\input{childdoc.def}|\\
|\childdocby{|\textit{main}|}|\\
\end{tabular}
\end{center}
%
The directive |\childdocby| is similar to |\childdocof|
described in \secref{sec:include},
but the subsequent selection of content must be done manually.
To that end, both |\ifchilddoc| and |\ifchilddocmanual|
will be true upon processing of a part,
and the name of the part is stored in |\childdocname|.
Note that |\jobname| will be set to the filename of the current part
so that each part receives an individual |.aux| file
that does not interfere with the |.aux| file(s) of the main document.
This behaviour can be altered by the alternative form
|\childdocby[*]{|\textit{main}|}| (with a non-empty optional argument)
which uses the |.aux| file of the main document
by setting |\jobname| to \textit{main}.

%%%%%%%%%%%%%%%%%%%%%%%%%%%%%%%%%%%%%%%%%%%%%%%%%%%%%%%%%%%%%%%%%%%%%%%%%%%%%%%%
\subsection{Driver Development}
\label{sec:driver}

The \textsf{childdoc} mechanism can also be use for the development
of definition files such as \LaTeX{} styles or classes.
This case differs from the above setup with multiple parts
included by |\include| in that no |\includeonly| should be invoked.
This can be achieved by starting the include file
(before |\ProvidesPackage|) with:
%
\begin{center}
\begin{tabular}{l}
|\input{childdoc.def}|\\
|\childdocforward{|\textit{main}|}|\\
\end{tabular}
\end{center}
%
or alternatively with:
%
\begin{center}
\begin{tabular}{l}
|\input{childdoc.def}|\\
|\childdocby{|\textit{main}|}|\\
\end{tabular}
\end{center}
%
Both forms have slightly different effects as described above.
The main file is prepared as usual, see \secref{sec:include}.

%%%%%%%%%%%%%%%%%%%%%%%%%%%%%%%%%%%%%%%%%%%%%%%%%%%%%%%%%%%%%%%%%%%%%%%%%%%%%%%%
\subsection{Legacy Detection}
\label{sec:detection}

The directive |\childdocmain| in the main file can detect
whether the complete document or merely a child is to be compiled
even without using the directive |\childdocof|.
This method is deprecated because it is less robust
and there is no compelling reason to use it;
it is merely provided for backward compatibility
and it may be removed in future versions.

If the detection mechanism is to be used,
it is mandatory to correctly specify
the filename of the main file as the argument of |\childdocmain|:
%
\begin{center}
\begin{tabular}{l}
|\input{childdoc.def}|\\
|\childdocmain{|\textit{main}|}|\\
\end{tabular}
\end{center}
%
If |\jobname| does not match the argument \textit{main} of |\childdocmain|,
it is assumed that |\jobname| points to the child file to be compiled.
When using |\childdocmain| with the main file specified as argument,
it suffices to start a child file
with just |\input{|\textit{main}|}|
without loading of the package and using |\childdocof|.
If instead all processing is done
with the appropriate \textsf{childdoc} directives,
the argument of \textit{main} of |\childdocmain| can be empty.

An alternative version of the command line processing described
in \secref{sec:commandline} using the detection mechanism reads:
%
\begin{center}
|... -jobname "|\textit{target}|" "|[\textit{flags}]%
[|\def\jobname{|\textit{dest}|}|]|\input{|\textit{main}|}"|
\end{center}

%%%%%%%%%%%%%%%%%%%%%%%%%%%%%%%%%%%%%%%%%%%%%%%%%%%%%%%%%%%%%%%%%%%%%%%%%%%%%%%%
\subsection{Manual Code}
\label{sec:manual}

In case one cannot be certain whether the definitions file |childdoc.def|
is installed on the target \TeX{} distribution
and one prefers not to ship it,
it is conceivable to paste a few relevant commands into the sources.

To that end, drop all statements |\input{childdoc.def}|
and perform the replacements as outlined below.
Instead of |\childdocmain{|\textit{main}|}| add the following code
to the top of the main file:
%
\begin{center}
\begin{tabular}{l}
|\||ifdefined\childdocname\endinput\||fi\newif\ifchilddoc|\\
|\edef\childdocname{\scantokens\expandafter{\jobname\noexpand}}|\\
|\def\childdocmain{|\textit{main}|}\||ifx\childdocmain\childdocname\||else|\\
|\childdoctrue\includeonly{\childdocname}\let\jobname\childdocmain\||fi|\\
\end{tabular}
\end{center}
%
Instead of |\childdocof{|\textit{main}|}| just include the main file
at the top of each child file:
%
\begin{center}
|\input{|\textit{main}|}|
\end{center}
%
A simple redirection |\childdocforward{|\textit{dest}|}| is achieved by:
%
\begin{center}
|\def\jobname{|\textit{dest}|}\input{\jobname}|
\end{center}
%
The redirection with prefix
|\childdocforwardprefix[|\textit{prefix}|]{|\textit{dest}|}|
is accomplished by:
%
\begin{center}
\begin{tabular}{l}
|{\edef\jobname{\scantokens\expandafter{\jobname\noexpand}}|\\
|\def\redirectjob |\textit{prefix}|#1~~~{\gdef\jobname{|\textit{dest}|#1}}|\\
|\expandafter\redirectjob\jobname~~~}\input{\jobname}|
\end{tabular}
\end{center}

In an alternative approach,
child documents can be compiled by a specific command line
without additional code or specific definitions:
%
\begin{center}
|... -jobname "|\textit{target}|" "|[\textit{flags}]%
|\includeonly{|\textit{dest}|}\input{|\textit{main}|}"|
\end{center}
%

%%%%%%%%%%%%%%%%%%%%%%%%%%%%%%%%%%%%%%%%%%%%%%%%%%%%%%%%%%%%%%%%%%%%%%%%%%%%%%%%
%%%%%%%%%%%%%%%%%%%%%%%%%%%%%%%%%%%%%%%%%%%%%%%%%%%%%%%%%%%%%%%%%%%%%%%%%%%%%%%%
\section{Information}

%%%%%%%%%%%%%%%%%%%%%%%%%%%%%%%%%%%%%%%%%%%%%%%%%%%%%%%%%%%%%%%%%%%%%%%%%%%%%%%%
\subsection{Copyright}

Copyright \copyright{} 2017--2018 Niklas Beisert

This work may be distributed and/or modified under the
conditions of the \LaTeX{} Project Public License, either version 1.3
of this license or (at your option) any later version.
The latest version of this license is in
  \url{http://www.latex-project.org/lppl.txt}
and version 1.3 or later is part of all distributions of \LaTeX{}
version 2005/12/01 or later.

This work has the LPPL maintenance status `maintained'.

The Current Maintainer of this work is Niklas Beisert.

This work consists of the files |README.txt|, |childdoc.ins| and |childdoc.dtx|
as well as the derived files |childdoc.def|, |cdocsamp.tex|
with |cdocsch1.tex|, |cdocsch2.tex|, |cdocspt3.tex|, |cdocspt4.tex|,
|cdocsdrf.tex|, |cdocsfn1.tex|, |cdocsfn2.tex|
as well as |childdoc.pdf|.

%%%%%%%%%%%%%%%%%%%%%%%%%%%%%%%%%%%%%%%%%%%%%%%%%%%%%%%%%%%%%%%%%%%%%%%%%%%%%%%%
\subsection{Files and Installation}

The package consists of the files:
%
\begin{center}
\begin{tabular}{ll}
    |README.txt|   & readme file \\
    |childdoc.ins| & installation file \\
    |childdoc.dtx| & source file \\
    |childdoc.def| & definition file \\
    |cdocsamp.tex| & sample main file \\
    |cdocsch1.tex| & sample include file \\
    |cdocsch2.tex| & sample include file \\
    |cdocspt3.tex| & sample part file \\
    |cdocspt4.tex| & sample part file \\
    |cdocsdrf.tex| & sample redirection file \\
    |cdocsfn1.tex| & sample redirection file \\
    |cdocsfn2.tex| & sample redirection file \\
    |childdoc.pdf| & manual
\end{tabular}
\end{center}
%
The distribution consists of the files
|README.txt|, |childdoc.ins| and |childdoc.dtx|.
%
\begin{itemize}
\item
Run (pdf)\LaTeX{} on |childdoc.dtx|
to compile the manual |childdoc.pdf| (this file).
\item
Run \LaTeX{} on |childdoc.ins| to create the definitions file |childdoc.def|
and the sample |cdocsamp.tex| with include files
|cdocsch1.tex|, |cdocsch2.tex|, |cdocspt3.tex|, |cdocspt4.tex|,
|cdocsdrf.tex|, |cdocsfn1.tex|, |cdocsfn2.tex|.
Then copy the file |childdoc.def| to an appropriate directory of your \LaTeX{}
distribution, e.g.\ \textit{texmf-root}|/tex/latex/childdoc|.
\end{itemize}

%%%%%%%%%%%%%%%%%%%%%%%%%%%%%%%%%%%%%%%%%%%%%%%%%%%%%%%%%%%%%%%%%%%%%%%%%%%%%%%%
\subsection{Related CTAN Packages}

There are several other packages which offer a similar functionality:
%
\begin{itemize}
\item
The packages
\href{http://ctan.org/pkg/docmute}{\textsf{docmute}},
\href{http://ctan.org/pkg/includex}{\textsf{includex}} and
\href{http://ctan.org/pkg/standalone}{\textsf{standalone}}
provide commands to include only the document body of
a child file thus allowing both files to be compiled individually.
\item
The packages \href{http://ctan.org/pkg/subdocs}{\textsf{subdocs}}
and \href{http://ctan.org/pkg/subfiles}{\textsf{subfiles}}
provide structures in which the main and child documents can be
encapsulated and allowing them to be compiled individually.
The inclusion mechanism is different from the conventional |\include|.
\item
The package \href{http://ctan.org/pkg/combine}{\textsf{combine}}
is an elaborate solution to combine several documents into one.
\end{itemize}
%
See also the CTAN topic \href{http://ctan.org/topic/subdocs}{\textsf{subdocs}}
for further related packages.
The present package differs from the above solutions in that
a document structure constructed with the conventional |\include| mechanism
just needs two extra commands at the top of every file
such that all constituent files can be compiled individually.

%%%%%%%%%%%%%%%%%%%%%%%%%%%%%%%%%%%%%%%%%%%%%%%%%%%%%%%%%%%%%%%%%%%%%%%%%%%%%%%%
%\subsection{Feature Suggestions}
%
%The following is a list of features which may be useful for future
%versions of this package:
%%
%\begin{itemize}
%\item
%\ldots
%\end{itemize}

%%%%%%%%%%%%%%%%%%%%%%%%%%%%%%%%%%%%%%%%%%%%%%%%%%%%%%%%%%%%%%%%%%%%%%%%%%%%%%%%
\subsection{Revision History}

%%%%%%%%%%%%%%%%%%%%%%%%%%%%%%%%%%%%%%%%
\paragraph{v2.0:} 2018/12/30

\begin{itemize}
\item
immediate forward processing
\item
added |\childdocby| mechanism
\item
manual restructured
\end{itemize}

%%%%%%%%%%%%%%%%%%%%%%%%%%%%%%%%%%%%%%%%
\paragraph{v1.6:} 2018/01/17

\begin{itemize}
\item
application for development of include files
\item
corrections to manual
\end{itemize}

%%%%%%%%%%%%%%%%%%%%%%%%%%%%%%%%%%%%%%%%
\paragraph{v1.5:} 2017/05/21

\begin{itemize}
\item
more complete structuring introduced
\item
|\childdocof| introduced
\item
|\childdoc| renamed to |\childdocmain|
\item
|\childredirect| renamed to |\childdocforward| and |\childdocforwardprefix|
and functionality expanded
\end{itemize}

%%%%%%%%%%%%%%%%%%%%%%%%%%%%%%%%%%%%%%%%
\paragraph{v1.0:} 2017/04/27

\begin{itemize}
\item
manual and install package
\item
first version published on CTAN
\end{itemize}

%%%%%%%%%%%%%%%%%%%%%%%%%%%%%%%%%%%%%%%%
\paragraph{v0.6:} 2017/04/26

\begin{itemize}
\item
redirection mechanism added
\end{itemize}

%%%%%%%%%%%%%%%%%%%%%%%%%%%%%%%%%%%%%%%%
\paragraph{v0.5:} 2017/04/26

\begin{itemize}
\item
functionality in definition file
\end{itemize}


%%%%%%%%%%%%%%%%%%%%%%%%%%%%%%%%%%%%%%%%%%%%%%%%%%%%%%%%%%%%%%%%%%%%%%%%%%%%%%%%
%%%%%%%%%%%%%%%%%%%%%%%%%%%%%%%%%%%%%%%%%%%%%%%%%%%%%%%%%%%%%%%%%%%%%%%%%%%%%%%%
%%%%%%%%%%%%%%%%%%%%%%%%%%%%%%%%%%%%%%%%%%%%%%%%%%%%%%%%%%%%%%%%%%%%%%%%%%%%%%%%
\appendix

\settowidth\MacroIndent{\rmfamily\scriptsize 000\ }

 \DocInput{childdoc.dtx}

\end{document}
%</driver>
% \fi
%
% %%%%%%%%%%%%%%%%%%%%%%%%%%%%%%%%%%%%%%%%%%%%%%%%%%%%%%%%%%%%%%%%%%%%%%%%%%%%%%
% %%%%%%%%%%%%%%%%%%%%%%%%%%%%%%%%%%%%%%%%%%%%%%%%%%%%%%%%%%%%%%%%%%%%%%%%%%%%%%
% \section{Sample}
%\iffalse
%<*samplemain>
%\fi
%
% The following presents a sample document
% with two chapters, two parts, a title page,
% a compile flag as well as three forwarding files to set the flag.
% It consists of eight |.tex| files:
% \begin{center}
% \begin{tabular}{ll}
% |cdocsamp.tex|&main file\\
% |cdocsch1.tex|&include file for chapter 1\\
% |cdocsch2.tex|&include file for chapter 2\\
% |cdocspt3.tex|&include file for part 3\\
% |cdocspt4.tex|&include file for part 4\\
% |cdocsdrf.tex|&forwarding file for main file in draft mode\\
% |cdocsfi1.tex|&forwarding file for final version of chapter 1\\
% |cdocsfi2.tex|&forwarding file for final version of chapter 2\\
% \end{tabular}
% \end{center}
% Each of the eight files can be compiled directly by the \LaTeX{} compiler.
%
% %%%%%%%%%%%%%%%%%%%%%%%%%%%%%%%%%%%%%%
% \paragraph{Main File.}
%
% The main file is called |cdocsamp.tex|.
%
% Load the \textsf{childdoc} definitions and
% declare the filename for the main document:
%    \begin{macrocode}
\input{childdoc.def}
\childdocmain{}
%    \end{macrocode}

% Optional override for |\version| flag:
%    \begin{macrocode}
%%\ifchilddoc\else\providecommand{\version}{draft}\fi
%    \end{macrocode}

% Define the default values for the |\version| flag
% (|final| for the main file and |draft| for childs):
%    \begin{macrocode}
\ifchilddoc
\providecommand{\version}{draft}
\else
\providecommand{\version}{final}
\fi
%    \end{macrocode}

% Load the standard document class:
%    \begin{macrocode}
\documentclass[12pt]{article}
%    \end{macrocode}

% Start the document body:
%    \begin{macrocode}
\begin{document}
%    \end{macrocode}

% Declare a title page.
% Print title, part of document being processed and version flag:
%    \begin{macrocode}
\addtocounter{page}{-1}
\begin{center}
{\LARGE\bfseries{}childdoc example\par}
\vspace{1cm}
\ifchilddoc
\ifchilddocmanual part\else chapter\fi:
`\childdocname' of `\childdocjob'\par
\else
main document: `\childdocjob'\par
\fi
version: \version\par
\end{center}
\newpage
%    \end{macrocode}

% Manually include selected file,
% otherwise process as usual:
%    \begin{macrocode}
\ifchilddocmanual
\section*{part `\childdocname'}
\input{\childdocname}
\else
%    \end{macrocode}

% Include the two chapters:
%    \begin{macrocode}
\include{cdocsch1}
\include{cdocsch2}
%    \end{macrocode}

% Include the two parts unless only chapters should be displayed:
%    \begin{macrocode}
\ifchilddoc\else
\section{part three}
\input{cdocspt3}
\section{part four}
\input{cdocspt4}
\fi
%    \end{macrocode}

% Process as usual until here:
%    \begin{macrocode}
\fi
%    \end{macrocode}

% End of document body:
%    \begin{macrocode}
\end{document}
%    \end{macrocode}
%\iffalse
%</samplemain>
%\fi
%
% %%%%%%%%%%%%%%%%%%%%%%%%%%%%%%%%%%%%%%
% \paragraph{Chapter Include Files.}
%
% The include files are called |cdocsch1.tex| and |cdocsch2.tex|.
%
%\iffalse
%<*samplechap1|samplechap2>
%\fi

% Optional override for |\version| flag:
%    \begin{macrocode}
%%\providecommand{\version}{final}
%    \end{macrocode}

% Include the main document:
%    \begin{macrocode}
\input{childdoc.def}
\childdocof{cdocsamp}
%    \end{macrocode}

%\iffalse
%</samplechap1|samplechap2>
%\fi
%
%\iffalse
%<*samplechap1>
%\fi
% Some text for chapter 1:
%    \begin{macrocode}
\section{one}
some text in chapter one
%    \end{macrocode}

%\iffalse
%</samplechap1>
%\fi
% Some text for chapter 2:
%\iffalse
%<*samplechap2>
%\fi
%    \begin{macrocode}
\section{two}
more text in chapter two
%    \end{macrocode}

%\iffalse
%</samplechap2>
%\fi
%
% %%%%%%%%%%%%%%%%%%%%%%%%%%%%%%%%%%%%%%
% \paragraph{Part Include Files.}
%
% The include files are called |cdocspt3.tex| and |cdocspt4.tex|.
%
%\iffalse
%<*samplepart3|samplepart4>
%\fi

% Optional override for |\version| flag:
%    \begin{macrocode}
%%\providecommand{\version}{final}
%    \end{macrocode}

% Include the main document:
%    \begin{macrocode}
\input{childdoc.def}
\childdocby{cdocsamp}
%    \end{macrocode}

%\iffalse
%</samplepart3|samplepart4>
%\fi
%
%\iffalse
%<*samplepart3>
%\fi
% Some text for part 3:
%    \begin{macrocode}
some text in part three
%    \end{macrocode}

%\iffalse
%</samplepart3>
%\fi
% Some text for part 4:
%\iffalse
%<*samplepart4>
%\fi
%    \begin{macrocode}
more text in part four
%    \end{macrocode}

%\iffalse
%</samplepart4>
%\fi
%
% %%%%%%%%%%%%%%%%%%%%%%%%%%%%%%%%%%%%%%
% \paragraph{Forwarding for a Complete Draft.}
%
% The following forwarding file |cdocsdrf.tex|
% compiles the main document in draft mode:
%\iffalse
%<*sampledraft>
%\fi
%    \begin{macrocode}
\def\version{draft}
\input{childdoc.def}
\childdocforward{cdocsamp}
%    \end{macrocode}

%\iffalse
%</sampledraft>
%\fi
%
% %%%%%%%%%%%%%%%%%%%%%%%%%%%%%%%%%%%%%%
% \paragraph{Forwarding for Final Version of the Chapters.}
%
% The following forwarding files |cdocsfn1.tex| and |cdocsfn2.tex|
% (with identical content)
% compile the final versions of the child documents
% |cdocsch1.tex| and |cdocsch2.tex|, respectively:
%\iffalse
%<*samplefinal>
%\fi
%    \begin{macrocode}
\def\version{final}
\input{childdoc.def}
\childdocforwardprefix[cdocsamp]{cdocsfn}{cdocsch}
%    \end{macrocode}

%\iffalse
%</samplefinal>
%\fi
%
% %%%%%%%%%%%%%%%%%%%%%%%%%%%%%%%%%%%%%%
% \paragraph{Command Line Processing.}
%
% The following three command lines generate the output files
% |cdocscld|, |cdocscl1| and |cdocscl2|
% which should be identical to
% |cdocsdrf|, |cdocsch1| and |cdocsfn2|, respectively:
% \begin{center}
% \begin{tabular}{l}
% |latex -jobname cdocscld \|\\
% |  "\def\version{draft}\input{childdoc.def}\childdocforward{cdocsamp}"|\\
% |latex -jobname cdocscl1 \|\\
% |  "\input{childdoc.def}\childdocforward[cdocsamp]{cdocsch1}"|\\
% |latex -jobname cdocscl2 \|\\
% |  "\def\version{final}\input{childdoc.def}\childdocforward{cdocsch2}"|
% \end{tabular}
% \end{center}
% Note that the trailing backslash on each first line
% merely continues the input to the second line
% (for convenient cut ant paste).
% Furthermore, the command |latex| can be replaced by any
% of its alternative versions such as |pdflatex|.
%
% %%%%%%%%%%%%%%%%%%%%%%%%%%%%%%%%%%%%%%%%%%%%%%%%%%%%%%%%%%%%%%%%%%%%%%%%%%%%%%
% %%%%%%%%%%%%%%%%%%%%%%%%%%%%%%%%%%%%%%%%%%%%%%%%%%%%%%%%%%%%%%%%%%%%%%%%%%%%%%
% \section{Implementation}
%\iffalse
%<*package>
%\fi
%
% This section describes the definitions file |childdoc.def|.

% The definitions cannot be loaded using |\usepackage| or |\RequirePackage|
% which has a mechanism to prevent loading a style file more than once.
% When loading the definitions by means of |\input|
% multiple instances have to be prevented manually:
%\iffalse
%This code needs to be before the `\ProvidesFile' directive
%which is defined at the beginning of this file.
%Therefore it is also placed there and commented out here.
%</package>
%<*discard>
%\fi
%    \begin{macrocode}
\ifdefined\childdocmain\endinput\fi
%    \end{macrocode}
%\iffalse
%</discard>
%<*package>
%\fi
%
% \macro{\ifchilddoc}
% \macro{\ifchilddocmanual}
% The conditional |\ifchilddoc| tells whether a
% child (true) or main (false) document is being compiled.
% The conditional |\ifchilddocmanual| tells whether
% the |\includeonly| mechanism is used (false) or
% the selection of child files must be performed manually (true).
% The definitions initialise to false:
%    \begin{macrocode}
\newif\ifchilddoc
\newif\ifchilddocmanual
%    \end{macrocode}

% \macro{\childdocname}
% \macro{\childdocjob}
% The macro |\childdocname| stores the name of the main document
% to be compiled. The macro |\childdocjob| stores the name of
% the document on which the \LaTeX{} compiler was originally invoked.
% The content of |\jobname| cannot be compared
% to filenames specified in the source due to different catcodes.
% The following code rescans |\jobname|, stores the result
% in |\childdocname| and saves a copy in |\childdocjob|:
%    \begin{macrocode}
\edef\childdocname{\scantokens\expandafter{\jobname\noexpand}}
\let\childdocjob\childdocname
%    \end{macrocode}

% \macro{\childdocdisable}
% The macro |\childdocdisable| prevents the main file
% from being processed more than once.
% At this stage, the main document command |\childdocmain|
% is assumed to be called once again where it should do nothing.
% Any subsequent call to it should prevent
% a secondary processing of the main document
% It overwrites the forwarding commands
% |\childdocof| and |\childdocforward|
% with empty macros to prevent further inclusions of the main document:
%    \begin{macrocode}
\newcommand{\childdocdisable}
{
  \renewcommand{\childdocmain}[1]{\renewcommand{\childdocmain}[1]{\endinput}}
  \renewcommand{\childdocof}[1]{}
  \renewcommand{\childdocby}[2][]{}
  \renewcommand{\childdocforward}[2][]{}
  \renewcommand{\childdocdisable}{}
}
%    \end{macrocode}

% \macro{\childdocmain}
% The macro |\childdocmain| is to be called at the top of the main file
% with nothing or the main filename (without extension) as argument.
% First, it breaks loops.
% If the argument is not empty and does not match |\childdocname|
% (which is set by the first inclusion of |childdoc.def|),
% |\ifchilddoc| is set to true, |\includeonly| is applied to the child file
% and |\jobname| is set to the main file
% (for proper handling of |.aux| files):
%    \begin{macrocode}
\newcommand{\childdocmain}[1]
{
  \childdocdisable\childdocmain{}
  \if?#1?\else
    \begingroup
      \def\childdoctmp{#1}
      \ifx\childdoctmp\childdocname
        \def\childdoctmp{}
      \else
        \def\childdoctmp
        {
          \childdoctrue
          \includeonly{\childdocname}
          \def\childdocjob{#1}
          \def\jobname{#1}
        }
      \fi
      \expandafter
    \endgroup
    \childdoctmp
  \fi
}
%    \end{macrocode}

% \macro{\childdocof}
% The command |\childdocof| redirects
% compilation to the main file |#1|.
%    \begin{macrocode}
\newcommand{\childdocof}[1]
{
  \childdocdisable
  \childdoctrue
  \includeonly{\childdocname}
  \def\jobname{#1}
  \def\childdocjob{#1}
  \input{#1}
}
%    \end{macrocode}

% \macro{\childdocby}
% The command |\childdocby| ....
%    \begin{macrocode}
\newcommand{\childdocby}[2][]
{
  \childdocdisable
  \childdoctrue
  \childdocmanualtrue
  \if?#1?\else
    \def\jobname{#2}
  \fi
  \def\childdocjob{#2}
  \input{#2}
  \endinput
}
%    \end{macrocode}

% \macro{\childdocforward}
% The command |\childdocforward| redirects
% compilation to the main file or
% (if the optional argument is given) a child file.
% Parameters are set as if the main file
% or a child file starting with |\childdocof| was compiled.
% Then compilation is handed over to the main file:
%    \begin{macrocode}
\newcommand{\childdocforward}[2][]
{
  \begingroup
    \if?#1?
      \def\childdoctmp
      {
        \def\childdocname{#2}
        \def\childdocjob{#2}
        \def\jobname{#2}
        \input{#2}
        \endinput
      }
    \else
      \def\childdoctmp
      {
        \childdocdisable
        \def\childdocname{#2}
        \childdoctrue
        \includeonly{#2}
        \def\childdocjob{#1}
        \def\jobname{#1}
        \input{#1}
        \endinput
      }
    \fi
    \expandafter
  \endgroup
  \childdoctmp
}
%    \end{macrocode}

% \macro{\childdocforwardprefix}
% The command |\childdocforwardprefix| redirects
% compilation to the main or a child file by means of a pattern.
% The prefix |#1| in the current filename is replaced by |#2|
% and the suffix of the current filename is kept
% (it is assumed that the filename does not contain the substring `|~~~|'
% which is used as a delimiter).
% Compilation is handed over to the new file by |\childdocforward|:
%    \begin{macrocode}
\newcommand{\childdocforwardprefix}[3][]
{
  \begingroup
    \def\childdocextract #2##1~~~{\def\childdoctmp{\childdocforward[#1]{#3##1}}}
    \expandafter\childdocextract\childdocname~~~
    \expandafter
  \endgroup
  \childdoctmp
}
%    \end{macrocode}

% \macro{\childdoc}
% The deprecated macro |\childdoc| is a legacy version of |\childdocmain|:
%    \begin{macrocode}
\newcommand{\childdoc}{\childdocmain}
%    \end{macrocode}

% \macro{\childdocredirect}
% The deprecated macro |\childdocredirect| is a legacy version
% of |\childdocforward| and |\childdocforwardprefix|:
%    \begin{macrocode}
\newcommand{\childdocredirect}[2][]
{
  \begingroup
    \if?#1?
      \def\childdoctmp{\childdocforward{#2}}
    \else
      \def\childdoctmp{\childdocforwardprefix{#1}{#2}}
    \fi
    \expandafter
  \endgroup
  \childdoctmp
}
%    \end{macrocode}

%\iffalse
%</package>
%\fi
%
\endinput
|\\
|\childdocforward{|\textit{main}|}|
\end{tabular}
\end{center}
%
Likewise, the following files |final|\textit{nn}|.tex|
compile the final version of the child document
|child|\textit{nn}|.tex|:
%
\begin{center}
\begin{tabular}{l}
|\def\version{final}|\\
|% \iffalse
%
% childdoc.dtx Copyright (C) 2017-2018 Niklas Beisert
%
% This work may be distributed and/or modified under the
% conditions of the LaTeX Project Public License, either version 1.3
% of this license or (at your option) any later version.
% The latest version of this license is in
%   http://www.latex-project.org/lppl.txt
% and version 1.3 or later is part of all distributions of LaTeX
% version 2005/12/01 or later.
%
% This work has the LPPL maintenance status `maintained'.
%
% The Current Maintainer of this work is Niklas Beisert.
%
% This work consists of the files childdoc.dtx and childdoc.ins
% and the derived files childdoc.def and cdocsamp.tex with
% cdocsch1.tex, cdocsch2.tex, cdocsdrf.tex, cdocsfn1.tex, cdocsfn2.tex.
%
%<package>\ifdefined\childdocmain\endinput\fi
%<package>\ProvidesFile{childdoc.def}[2018/12/30 v2.0 child document driver]
%<samplemain>\ProvidesFile{cdocsamp.tex}[2018/12/30 v2.0 sample for childdoc]
%<*driver>
%\ProvidesFile{childdoc.drv}[2018/12/30 v2.0 childdoc reference manual file]
\PassOptionsToClass{10pt,a4paper}{article}
\documentclass{ltxdoc}

\usepackage[margin=35mm]{geometry}
\usepackage{hyperref}
\usepackage{hyperxmp}
\usepackage[usenames]{color}

\hypersetup{colorlinks=true}
\hypersetup{pdfstartview=FitH}
\hypersetup{pdfpagemode=UseNone}
\hypersetup{pdfsource={}}
\hypersetup{pdflang={en-UK}}
\hypersetup{pdfcopyright={Copyright 2017-2018 Niklas Beisert.
  This work may be distributed and/or modified under the
  conditions of the LaTeX Project Public License, either version 1.3
  of this license or (at your option) any later version.}}
\hypersetup{pdflicenseurl={http://www.latex-project.org/lppl.txt}}
\hypersetup{pdfcontactaddress={ETH Zurich, ITP, HIT K,
  Wolfgang-Pauli-Strasse 27}}
\hypersetup{pdfcontactpostcode={8093}}
\hypersetup{pdfcontactcity={Zurich}}
\hypersetup{pdfcontactcountry={Switzerland}}
\hypersetup{pdfcontactemail={nbeisert@itp.phys.ethz.ch}}
\hypersetup{pdfcontacturl={http://people.phys.ethz.ch/\xmptilde nbeisert/}}

\newcommand{\secref}[1]{\hyperref[#1]{section \ref*{#1}}}

\parskip1ex
\parindent0pt
\let\olditemize\itemize
\def\itemize{\olditemize\parskip0pt}

\begin{document}

\title{The \textsf{childdoc} Package}
\hypersetup{pdftitle={The childdoc Package}}
\author{Niklas Beisert\\[2ex]
  Institut f\"ur Theoretische Physik\\
  Eidgen\"ossische Technische Hochschule Z\"urich\\
  Wolfgang-Pauli-Strasse 27, 8093 Z\"urich, Switzerland\\[1ex]
  \href{mailto:nbeisert@itp.phys.ethz.ch}
  {\texttt{nbeisert@itp.phys.ethz.ch}}}
\hypersetup{pdfauthor={Niklas Beisert}}
\hypersetup{pdfsubject={Manual for the LaTeX2e Package childdoc}}
\date{30 December 2018, \textsf{v2.0}}
\maketitle

\begin{abstract}\noindent
\textsf{childdoc} is a \LaTeXe{} package
that enables the direct compilation
of document sections included by |\include|
to individual files.
\end{abstract}

\begingroup
\parskip0ex
\tableofcontents
\endgroup

%%%%%%%%%%%%%%%%%%%%%%%%%%%%%%%%%%%%%%%%%%%%%%%%%%%%%%%%%%%%%%%%%%%%%%%%%%%%%%%%
%%%%%%%%%%%%%%%%%%%%%%%%%%%%%%%%%%%%%%%%%%%%%%%%%%%%%%%%%%%%%%%%%%%%%%%%%%%%%%%%
\section{Introduction}

\LaTeX{} provides a mechanism to structure a large document (such as a book)
into a main file and several child files (containing the chapters)
using the |\include| command.
This mechanism is beneficial for documents
which span hundreds of pages in order to
make the source file(s) more manageable.
Moreover, compilation can be restricted to
selected child files by means of the |\includeonly| command.
The latter feature can be used to reduce the compilation time while editing
(this was significantly more useful in the earlier days of \LaTeX{})
or to generate a smaller document which is easier to navigate.
Another application of |\includeonly| is to generate
documents consisting of selected parts of the complete document.

However, there are a few drawbacks of the plain |\include| mechanism:
\begin{itemize}
\item
The child files cannot be compiled on their own,
they can only be compiled via the main file.
A naive editing environment
(such as a text editor with an option
to have the current file processed by \LaTeX)
may require one to switch to the main file before compiling;
attempting to compile the child file produces errors.
\item
The main file must be modified (each time)
to adjust the |\includeonly| command
to the present needs. This easily leaves the main file in a messy state.
\item
The generated document will always carry the filename
of the main document. This is inconvenient if
several child files are to be compiled and
to be kept for distribution.
\end{itemize}

The present package provides a simple interface
to make child files individually compilable by \LaTeX{}.
Compiling a child file then has the same effect as compiling
the main file with an |\includeonly| command
to select the appropriate child.
Moreover the generated document will carry the name of the child
rather than the main file.
This resolves all three above issues.

This feature is meant to make the editing of books,
thesis documents and lecture notes somewhat more convenient.
However, the package can also be used efficiently for
composing a series of documents (such as exercise sheets)
which are typically distributed individually.
It then assists the author in generating the individual documents
(potentially in different versions)
as well as a document containing the collected series.
Another application is in developing style files
or other kinds of included material
where compilation of the style file could redirect
to a sample or test file.

%%%%%%%%%%%%%%%%%%%%%%%%%%%%%%%%%%%%%%%%%%%%%%%%%%%%%%%%%%%%%%%%%%%%%%%%%%%%%%%%
%%%%%%%%%%%%%%%%%%%%%%%%%%%%%%%%%%%%%%%%%%%%%%%%%%%%%%%%%%%%%%%%%%%%%%%%%%%%%%%%
\section{Usage}

First of all, the package \textsf{childdoc} is \emph{not} a standard
\LaTeXe{} |.sty| style file! Therefore it needs to be invoked in
a non-standard way.

%%%%%%%%%%%%%%%%%%%%%%%%%%%%%%%%%%%%%%%%%%%%%%%%%%%%%%%%%%%%%%%%%%%%%%%%%%%%%%%%
\subsection{Included Files}
\label{sec:include}

%%%%%%%%%%%%%%%%%%%%%%%%%%%%%%%%%%%%%%%%
\DescribeMacro{\childdocmain}
To use the package, add the commands
\begin{center}
\begin{tabular}{l}
|\input{childdoc.def}|\\
|\childdocmain{}|\\
\end{tabular}
\end{center}
at the very top of the main \LaTeX{} file,
in particular \emph{before} the |\documentclass| statement!
The argument of |\childdocmain| should be left empty
(but it must be present).

%%%%%%%%%%%%%%%%%%%%%%%%%%%%%%%%%%%%%%%%
\DescribeMacro{\childdocof}
Furthermore, add the commands
\begin{center}
\begin{tabular}{l}
|\input{childdoc.def}|\\
|\childdocof{|\textit{main}|}|\\
\end{tabular}
\end{center}
at the top of every child file \textit{child}
which is included by |\include{|\textit{child}|}|
from within the main file
(or at least for those files to be compiled individually).
The argument \textit{main} must be the filename of the main file.

There are a couple of
considerations in setting up the main and child documents:

%%%%%%%%%%%%%%%%%%%%%%%%%%%%%%%%%%%%%%%%
\paragraph{Restrictions.}

Please note the following restrictions:
\begin{itemize}
\item
|\childdocmain| must be called with one argument \textit{main}
to ensure compatibility with earlier version of the package.
It must either be empty (|\childdocmain{}|)
or precisely match the filename of the main file in which it is specified.
See \secref{sec:detection} for further information.
\item
The filename \textit{main} must be specified without the |.tex| extension.
\item
The filename \textit{main} is case sensitive
(even in case-insensitive file systems)
due to internal string comparison.
\item
The argument \textit{main} should be fully expanded, it cannot be a macro.
\item
Subdirectories and special characters should be avoided in filenames.
\item
The command |\childdocmain{|\textit{main}|}| must be followed by a whitespace.
It should not be followed immediately by another command
or by a comment mark `|%|'.
This is because the \TeX{} parser reads the token immediately following
the argument of |\childdocmain| and puts it
at the beginning of every child section;
however, a white\-space is ignored.
\end{itemize}

%%%%%%%%%%%%%%%%%%%%%%%%%%%%%%%%%%%%%%%%
\paragraph{Content of Main File.}

It is advisable to place all content in the child files included by |\include|.
Any output contained in the main file will appear in all child documents
unless suppressed manually;
it cannot be suppressed automatically by the |\includeonly| directive
and thus should normally be avoided.
A method to include some content in the main file
by means of conditional processing is described in \secref{sec:conditional}.

%%%%%%%%%%%%%%%%%%%%%%%%%%%%%%%%%%%%%%%%
\paragraph{Page Numbering.}

When only a part of the document is compiled,
the appropriate numbering of pages
(as well as other status parameters)
is determined from the |.aux| files.
The latter contain information from previous passes.
However this information needs to propagate through
all intermediate child documents.
Therefore the page numbering in child documents may well
be inconsistent until the complete document is compiled at least once.

A useful (if unconventional) way to always ensure a consistent
page numbering is to restart the numbering in each child document
and denote the pages by `\textit{child}|.|\textit{page}'
where \textit{child} represents the chapter/section number of the child file.
This can be achieved by the command
|\numberwithin{page}{|\textit{child}|}|
of the \textsf{amsmath} package
where \textit{child} can be |chapter| or |section|
depending on the chosen structuring.
Alternatively, one can modify the macro |\thepage| appropriately
and reset the counter |page| at the start of each child file.

%%%%%%%%%%%%%%%%%%%%%%%%%%%%%%%%%%%%%%%%%%%%%%%%%%%%%%%%%%%%%%%%%%%%%%%%%%%%%%%%
\subsection{Conditional Processing}
\label{sec:conditional}

The package provides a mechanism to compile different versions
of a document. To customise the versions further some conditional processing
can come in handy to distinguish which version is being compiled.
The package provides two macros to describe the compilation context:

%%%%%%%%%%%%%%%%%%%%%%%%%%%%%%%%%%%%%%%%
\DescribeMacro{\ifchilddoc}
The conditional |\ifchilddoc| distinguishes between the compilation of
child documents and the main document:
%
\begin{center}
|\ifchilddoc |\textit{child-code}| |[|\||else |\textit{main-code}]| \||fi|
\end{center}

%%%%%%%%%%%%%%%%%%%%%%%%%%%%%%%%%%%%%%%%
\DescribeMacro{\childdocname}
\DescribeMacro{\childdocjob}
The macro |\childdocname| contains the filename (without extension)
of the main or child file being processed.
Note that |\childdocjob| will always contain the name of the main file.

%%%%%%%%%%%%%%%%%%%%%%%%%%%%%%%%%%%%%%%%
\paragraph{Title Page.}

Conditional processing can be used to include a title or banner page
in the main document when proper precautions are taken.
Importantly, the code in the main file should ensure that the page counter
(as well as other status parameters which are stored in the |.aux| files)
takes the same value after the conditional processing.
Otherwise the page numbers may take divergent values
depending on which part is compiled.

For example, a title page could be declared by:
%
\begin{center}
\begin{tabular}{l}
|\ifchilddoc\||else|\\
|\addtocounter{page}{-1}|\\
\textit{code for title page}\\
|\newpage|\\
|\||fi|
\end{tabular}
\end{center}
%
A banner page for the child documents can be generated by:
%
\begin{center}
\begin{tabular}{l}
|\ifchilddoc|\\
|\addtocounter{page}{-1}|\\
\textit{code for banner page}\\
|\newpage|\\
|\||fi|
\end{tabular}
\end{center}
%
Here one could write a message such as:
\begin{center}
|This is the part \childdocname{} of \childdocjob{}.|
\end{center}

%%%%%%%%%%%%%%%%%%%%%%%%%%%%%%%%%%%%%%%%%%%%%%%%%%%%%%%%%%%%%%%%%%%%%%%%%%%%%%%%
\subsection{Flags}
\label{sec:flags}

The package makes it easy to generate different versions
of the main or child documents.
To this end compilation flags can be defined
and assigned different default values.
They will be particularly useful in conjunction
with the forwarding mechanism described in \secref{sec:forward}.

For example, it may be useful to have a flag |\version|
which can be set to |draft| or |final|.
The document source will contain some conditional code
depending on the value of |\version|.
Suppose further, the flag should default to |final| for the main file
and to |draft| for child files
which is a natural assignment for editing the document.
This is achieved by placing the following code
in the preamble of the main document
(below the |\childdocmain| directive):
%
\begin{center}
\begin{tabular}{l}
|\ifchilddoc|\\
|\providecommand{\version}{draft}|\\
|\||else|\\
|\providecommand{\version}{final}|\\
|\||fi|
\end{tabular}
\end{center}
%
The definition by |\providecommand| makes sure
that previous definitions are not overwritten.
Further statements |\providecommand{\version}{...}|
can thus be added before the above code to override it.

For the main file, one might add a line
(between |\childdocmain| and the above block)
%
\begin{center}
|%\ifchilddoc\||else\providecommand{\version}{draft}\||fi|
\end{center}
%
which can be uncommented to produce a draft version.
Likewise one can add a line to the very top of a child file
(above the |\childdocof{|\textit{main}|}| directive)
%
\begin{center}
|%\providecommand{\version}{final}|
\end{center}
%
which can be uncommented to produce the final version of this child document.

%%%%%%%%%%%%%%%%%%%%%%%%%%%%%%%%%%%%%%%%%%%%%%%%%%%%%%%%%%%%%%%%%%%%%%%%%%%%%%%%
\subsection{Forwarding}
\label{sec:forward}

Different versions of the main or child documents
using compilation flags as described in \secref{sec:flags}
can be (permanently) stored in different files
for convenient compilation, viewing and distribution.
To this end, the package defines a command
to pass on compilation to a different file:

%%%%%%%%%%%%%%%%%%%%%%%%%%%%%%%%%%%%%%%%
\DescribeMacro{\childdocforward}
The command |\childdocforward| redirects processing to
another source file:
%
\begin{center}
\begin{tabular}{l}
|\input{childdoc.def}|\\
|\childdocforward[|\textit{main}|]{|\textit{dest}|}|\\
\end{tabular}
\end{center}
%
The argument \textit{dest} is the destination file
(without extension).
It should be the main file or one of the child files.
Note that further \textsf{childdoc} directives
such as |\childdocof| and |\childdocforward|
in the indicated file will be processed in this form.
The optional argument \textit{main}
passes on directly to the main file \textit{main}
while pretending to compile the child \textit{dest}.
This form behaves as if \textit{dest}
issues |\childdocof{|\textit{main}|}| right away,
and no further \textsf{childdoc} directives will be processed.

%%%%%%%%%%%%%%%%%%%%%%%%%%%%%%%%%%%%%%%%
\DescribeMacro{\...prefix}
In the alternative form |\childdocforwardprefix|,
%
\begin{center}
\begin{tabular}{l}
|\input{childdoc.def}|\\
|\childdocforwardprefix[|\textit{main}|]{|\textit{prefix}|}{|\textit{dest}|}|
\end{tabular}
\end{center}
%
the destination file is determined by a pattern
depending on the current file:
To make this work, the current file must be called
`{\textit{prefix}\hspace{0.2em}\textit{suffix}}'
with \textit{prefix} matching precisely the argument.
Processing is then passed on to the file
`{\textit{dest}\hspace{0.2em}\textit{suffix}}'.
Surely, the same effect is achieved by
directly specifying the
argument `{\textit{dest}\hspace{0.2em}\textit{suffix}}'
in the first form.
However, that requires to set up a different file
for each child. With the alternative form of the command
all these files can have exactly the same content
which simplifies setting them up and maintaining them.

For example, the following file |draft.tex|
with a compilation flag |\version| as described in \secref{sec:flags}
compiles the main document as a draft:
%
\begin{center}
\begin{tabular}{l}
|\def\version{draft}|\\
|\input{childdoc.def}|\\
|\childdocforward{|\textit{main}|}|
\end{tabular}
\end{center}
%
Likewise, the following files |final|\textit{nn}|.tex|
compile the final version of the child document
|child|\textit{nn}|.tex|:
%
\begin{center}
\begin{tabular}{l}
|\def\version{final}|\\
|\input{childdoc.def}|\\
|\childdocforwardprefix{final}{child}|
\end{tabular}
\end{center}
%

Note that when several versions of a main file and/or of each child file
are to be generated, it may be convenient to set up a |Makefile| or
shell script to automatise the process.

%%%%%%%%%%%%%%%%%%%%%%%%%%%%%%%%%%%%%%%%%%%%%%%%%%%%%%%%%%%%%%%%%%%%%%%%%%%%%%%%
\subsection{Command Line Processing}
\label{sec:commandline}

The effect of redirection files can also be achieved by invoking
the \LaTeX{} compiler with a more elaborate command line.
Most conveniently this should be done as part
of a shell script or a |Makefile|.

When using \textsf{childdoc} in the main file, the following
command lines effectively perform a redirection
(note that depending on the shell being used,
backslashes may have to be doubled: `|\|' $\to$ `|\\|'):
%
\begin{center}
|... -jobname "|\textit{target}|" |\\|"|[\textit{flags}]%
|\input{childdoc.def}\childdocforward[|\textit{main}|]{|\textit{dest}|}"|
\end{center}
%
Here \textit{target} is the name of the output file,
\textit{main} is the name of the main file
and \textit{dest} is the name of the main or child file to be processed
(all filenames without extensions).
The optional argument \textit{main} can be omitted
if \textit{main} matches \textit{dest}.
Optionally, compilation \textit{flags} can be defined via |\def| commands.
This command line makes the \TeX{} engine believe
it is compiling the file \textit{target}
whose content is specified as the latter parameter.
The provided code then forwards the processing to
\textit{main} or \textit{dest} as described in \secref{sec:forward}.

%%%%%%%%%%%%%%%%%%%%%%%%%%%%%%%%%%%%%%%%%%%%%%%%%%%%%%%%%%%%%%%%%%%%%%%%%%%%%%%%
\subsection{Include by Input}
\label{sec:input}

Including child documents by |\include| has some restrictions by design.
Most notably, the content of a child document always occupies
its own set of pages; pages cannot be shared between child documents.
Usually, this behaviour makes perfect sense
because each child document contain an essential part of the document.
However, in some situations it may be desirable to compose
a document from a collection of parts
without having mandatory page breaks between then.
For this case, the package
provides a mechanism to include parts
by |\input| which can also be processed individually.
However, by construction this mechanism
requires manual handling of the content to be output.

%%%%%%%%%%%%%%%%%%%%%%%%%%%%%%%%%%%%%%%%
\DescribeMacro{\ifchilddocmanual}
The main file should be prepared as usual, see \secref{sec:include}.
However, the document body must make a distinction
between processing of an individual part and of the main document, e.g.:
%
\begin{center}
\begin{tabular}{l}
|\ifchilddocmanual|\\
|\input{\childdocname}|\\
|\||else|\\
\textit{document body with }|\input{|\textit{part}|}|\\
|\||fi|
\end{tabular}
\end{center}
%
The conditional |\ifchilddocmanual| is true whenever
a part to be included by |\input| is being compiled,
and the name of the part is stored in |\childdocname|.

%%%%%%%%%%%%%%%%%%%%%%%%%%%%%%%%%%%%%%%%
\DescribeMacro{\childdocby}
Each part to be included by |\input| should start with:
%
\begin{center}
\begin{tabular}{l}
|\input{childdoc.def}|\\
|\childdocby{|\textit{main}|}|\\
\end{tabular}
\end{center}
%
The directive |\childdocby| is similar to |\childdocof|
described in \secref{sec:include},
but the subsequent selection of content must be done manually.
To that end, both |\ifchilddoc| and |\ifchilddocmanual|
will be true upon processing of a part,
and the name of the part is stored in |\childdocname|.
Note that |\jobname| will be set to the filename of the current part
so that each part receives an individual |.aux| file
that does not interfere with the |.aux| file(s) of the main document.
This behaviour can be altered by the alternative form
|\childdocby[*]{|\textit{main}|}| (with a non-empty optional argument)
which uses the |.aux| file of the main document
by setting |\jobname| to \textit{main}.

%%%%%%%%%%%%%%%%%%%%%%%%%%%%%%%%%%%%%%%%%%%%%%%%%%%%%%%%%%%%%%%%%%%%%%%%%%%%%%%%
\subsection{Driver Development}
\label{sec:driver}

The \textsf{childdoc} mechanism can also be use for the development
of definition files such as \LaTeX{} styles or classes.
This case differs from the above setup with multiple parts
included by |\include| in that no |\includeonly| should be invoked.
This can be achieved by starting the include file
(before |\ProvidesPackage|) with:
%
\begin{center}
\begin{tabular}{l}
|\input{childdoc.def}|\\
|\childdocforward{|\textit{main}|}|\\
\end{tabular}
\end{center}
%
or alternatively with:
%
\begin{center}
\begin{tabular}{l}
|\input{childdoc.def}|\\
|\childdocby{|\textit{main}|}|\\
\end{tabular}
\end{center}
%
Both forms have slightly different effects as described above.
The main file is prepared as usual, see \secref{sec:include}.

%%%%%%%%%%%%%%%%%%%%%%%%%%%%%%%%%%%%%%%%%%%%%%%%%%%%%%%%%%%%%%%%%%%%%%%%%%%%%%%%
\subsection{Legacy Detection}
\label{sec:detection}

The directive |\childdocmain| in the main file can detect
whether the complete document or merely a child is to be compiled
even without using the directive |\childdocof|.
This method is deprecated because it is less robust
and there is no compelling reason to use it;
it is merely provided for backward compatibility
and it may be removed in future versions.

If the detection mechanism is to be used,
it is mandatory to correctly specify
the filename of the main file as the argument of |\childdocmain|:
%
\begin{center}
\begin{tabular}{l}
|\input{childdoc.def}|\\
|\childdocmain{|\textit{main}|}|\\
\end{tabular}
\end{center}
%
If |\jobname| does not match the argument \textit{main} of |\childdocmain|,
it is assumed that |\jobname| points to the child file to be compiled.
When using |\childdocmain| with the main file specified as argument,
it suffices to start a child file
with just |\input{|\textit{main}|}|
without loading of the package and using |\childdocof|.
If instead all processing is done
with the appropriate \textsf{childdoc} directives,
the argument of \textit{main} of |\childdocmain| can be empty.

An alternative version of the command line processing described
in \secref{sec:commandline} using the detection mechanism reads:
%
\begin{center}
|... -jobname "|\textit{target}|" "|[\textit{flags}]%
[|\def\jobname{|\textit{dest}|}|]|\input{|\textit{main}|}"|
\end{center}

%%%%%%%%%%%%%%%%%%%%%%%%%%%%%%%%%%%%%%%%%%%%%%%%%%%%%%%%%%%%%%%%%%%%%%%%%%%%%%%%
\subsection{Manual Code}
\label{sec:manual}

In case one cannot be certain whether the definitions file |childdoc.def|
is installed on the target \TeX{} distribution
and one prefers not to ship it,
it is conceivable to paste a few relevant commands into the sources.

To that end, drop all statements |\input{childdoc.def}|
and perform the replacements as outlined below.
Instead of |\childdocmain{|\textit{main}|}| add the following code
to the top of the main file:
%
\begin{center}
\begin{tabular}{l}
|\||ifdefined\childdocname\endinput\||fi\newif\ifchilddoc|\\
|\edef\childdocname{\scantokens\expandafter{\jobname\noexpand}}|\\
|\def\childdocmain{|\textit{main}|}\||ifx\childdocmain\childdocname\||else|\\
|\childdoctrue\includeonly{\childdocname}\let\jobname\childdocmain\||fi|\\
\end{tabular}
\end{center}
%
Instead of |\childdocof{|\textit{main}|}| just include the main file
at the top of each child file:
%
\begin{center}
|\input{|\textit{main}|}|
\end{center}
%
A simple redirection |\childdocforward{|\textit{dest}|}| is achieved by:
%
\begin{center}
|\def\jobname{|\textit{dest}|}\input{\jobname}|
\end{center}
%
The redirection with prefix
|\childdocforwardprefix[|\textit{prefix}|]{|\textit{dest}|}|
is accomplished by:
%
\begin{center}
\begin{tabular}{l}
|{\edef\jobname{\scantokens\expandafter{\jobname\noexpand}}|\\
|\def\redirectjob |\textit{prefix}|#1~~~{\gdef\jobname{|\textit{dest}|#1}}|\\
|\expandafter\redirectjob\jobname~~~}\input{\jobname}|
\end{tabular}
\end{center}

In an alternative approach,
child documents can be compiled by a specific command line
without additional code or specific definitions:
%
\begin{center}
|... -jobname "|\textit{target}|" "|[\textit{flags}]%
|\includeonly{|\textit{dest}|}\input{|\textit{main}|}"|
\end{center}
%

%%%%%%%%%%%%%%%%%%%%%%%%%%%%%%%%%%%%%%%%%%%%%%%%%%%%%%%%%%%%%%%%%%%%%%%%%%%%%%%%
%%%%%%%%%%%%%%%%%%%%%%%%%%%%%%%%%%%%%%%%%%%%%%%%%%%%%%%%%%%%%%%%%%%%%%%%%%%%%%%%
\section{Information}

%%%%%%%%%%%%%%%%%%%%%%%%%%%%%%%%%%%%%%%%%%%%%%%%%%%%%%%%%%%%%%%%%%%%%%%%%%%%%%%%
\subsection{Copyright}

Copyright \copyright{} 2017--2018 Niklas Beisert

This work may be distributed and/or modified under the
conditions of the \LaTeX{} Project Public License, either version 1.3
of this license or (at your option) any later version.
The latest version of this license is in
  \url{http://www.latex-project.org/lppl.txt}
and version 1.3 or later is part of all distributions of \LaTeX{}
version 2005/12/01 or later.

This work has the LPPL maintenance status `maintained'.

The Current Maintainer of this work is Niklas Beisert.

This work consists of the files |README.txt|, |childdoc.ins| and |childdoc.dtx|
as well as the derived files |childdoc.def|, |cdocsamp.tex|
with |cdocsch1.tex|, |cdocsch2.tex|, |cdocspt3.tex|, |cdocspt4.tex|,
|cdocsdrf.tex|, |cdocsfn1.tex|, |cdocsfn2.tex|
as well as |childdoc.pdf|.

%%%%%%%%%%%%%%%%%%%%%%%%%%%%%%%%%%%%%%%%%%%%%%%%%%%%%%%%%%%%%%%%%%%%%%%%%%%%%%%%
\subsection{Files and Installation}

The package consists of the files:
%
\begin{center}
\begin{tabular}{ll}
    |README.txt|   & readme file \\
    |childdoc.ins| & installation file \\
    |childdoc.dtx| & source file \\
    |childdoc.def| & definition file \\
    |cdocsamp.tex| & sample main file \\
    |cdocsch1.tex| & sample include file \\
    |cdocsch2.tex| & sample include file \\
    |cdocspt3.tex| & sample part file \\
    |cdocspt4.tex| & sample part file \\
    |cdocsdrf.tex| & sample redirection file \\
    |cdocsfn1.tex| & sample redirection file \\
    |cdocsfn2.tex| & sample redirection file \\
    |childdoc.pdf| & manual
\end{tabular}
\end{center}
%
The distribution consists of the files
|README.txt|, |childdoc.ins| and |childdoc.dtx|.
%
\begin{itemize}
\item
Run (pdf)\LaTeX{} on |childdoc.dtx|
to compile the manual |childdoc.pdf| (this file).
\item
Run \LaTeX{} on |childdoc.ins| to create the definitions file |childdoc.def|
and the sample |cdocsamp.tex| with include files
|cdocsch1.tex|, |cdocsch2.tex|, |cdocspt3.tex|, |cdocspt4.tex|,
|cdocsdrf.tex|, |cdocsfn1.tex|, |cdocsfn2.tex|.
Then copy the file |childdoc.def| to an appropriate directory of your \LaTeX{}
distribution, e.g.\ \textit{texmf-root}|/tex/latex/childdoc|.
\end{itemize}

%%%%%%%%%%%%%%%%%%%%%%%%%%%%%%%%%%%%%%%%%%%%%%%%%%%%%%%%%%%%%%%%%%%%%%%%%%%%%%%%
\subsection{Related CTAN Packages}

There are several other packages which offer a similar functionality:
%
\begin{itemize}
\item
The packages
\href{http://ctan.org/pkg/docmute}{\textsf{docmute}},
\href{http://ctan.org/pkg/includex}{\textsf{includex}} and
\href{http://ctan.org/pkg/standalone}{\textsf{standalone}}
provide commands to include only the document body of
a child file thus allowing both files to be compiled individually.
\item
The packages \href{http://ctan.org/pkg/subdocs}{\textsf{subdocs}}
and \href{http://ctan.org/pkg/subfiles}{\textsf{subfiles}}
provide structures in which the main and child documents can be
encapsulated and allowing them to be compiled individually.
The inclusion mechanism is different from the conventional |\include|.
\item
The package \href{http://ctan.org/pkg/combine}{\textsf{combine}}
is an elaborate solution to combine several documents into one.
\end{itemize}
%
See also the CTAN topic \href{http://ctan.org/topic/subdocs}{\textsf{subdocs}}
for further related packages.
The present package differs from the above solutions in that
a document structure constructed with the conventional |\include| mechanism
just needs two extra commands at the top of every file
such that all constituent files can be compiled individually.

%%%%%%%%%%%%%%%%%%%%%%%%%%%%%%%%%%%%%%%%%%%%%%%%%%%%%%%%%%%%%%%%%%%%%%%%%%%%%%%%
%\subsection{Feature Suggestions}
%
%The following is a list of features which may be useful for future
%versions of this package:
%%
%\begin{itemize}
%\item
%\ldots
%\end{itemize}

%%%%%%%%%%%%%%%%%%%%%%%%%%%%%%%%%%%%%%%%%%%%%%%%%%%%%%%%%%%%%%%%%%%%%%%%%%%%%%%%
\subsection{Revision History}

%%%%%%%%%%%%%%%%%%%%%%%%%%%%%%%%%%%%%%%%
\paragraph{v2.0:} 2018/12/30

\begin{itemize}
\item
immediate forward processing
\item
added |\childdocby| mechanism
\item
manual restructured
\end{itemize}

%%%%%%%%%%%%%%%%%%%%%%%%%%%%%%%%%%%%%%%%
\paragraph{v1.6:} 2018/01/17

\begin{itemize}
\item
application for development of include files
\item
corrections to manual
\end{itemize}

%%%%%%%%%%%%%%%%%%%%%%%%%%%%%%%%%%%%%%%%
\paragraph{v1.5:} 2017/05/21

\begin{itemize}
\item
more complete structuring introduced
\item
|\childdocof| introduced
\item
|\childdoc| renamed to |\childdocmain|
\item
|\childredirect| renamed to |\childdocforward| and |\childdocforwardprefix|
and functionality expanded
\end{itemize}

%%%%%%%%%%%%%%%%%%%%%%%%%%%%%%%%%%%%%%%%
\paragraph{v1.0:} 2017/04/27

\begin{itemize}
\item
manual and install package
\item
first version published on CTAN
\end{itemize}

%%%%%%%%%%%%%%%%%%%%%%%%%%%%%%%%%%%%%%%%
\paragraph{v0.6:} 2017/04/26

\begin{itemize}
\item
redirection mechanism added
\end{itemize}

%%%%%%%%%%%%%%%%%%%%%%%%%%%%%%%%%%%%%%%%
\paragraph{v0.5:} 2017/04/26

\begin{itemize}
\item
functionality in definition file
\end{itemize}


%%%%%%%%%%%%%%%%%%%%%%%%%%%%%%%%%%%%%%%%%%%%%%%%%%%%%%%%%%%%%%%%%%%%%%%%%%%%%%%%
%%%%%%%%%%%%%%%%%%%%%%%%%%%%%%%%%%%%%%%%%%%%%%%%%%%%%%%%%%%%%%%%%%%%%%%%%%%%%%%%
%%%%%%%%%%%%%%%%%%%%%%%%%%%%%%%%%%%%%%%%%%%%%%%%%%%%%%%%%%%%%%%%%%%%%%%%%%%%%%%%
\appendix

\settowidth\MacroIndent{\rmfamily\scriptsize 000\ }

 \DocInput{childdoc.dtx}

\end{document}
%</driver>
% \fi
%
% %%%%%%%%%%%%%%%%%%%%%%%%%%%%%%%%%%%%%%%%%%%%%%%%%%%%%%%%%%%%%%%%%%%%%%%%%%%%%%
% %%%%%%%%%%%%%%%%%%%%%%%%%%%%%%%%%%%%%%%%%%%%%%%%%%%%%%%%%%%%%%%%%%%%%%%%%%%%%%
% \section{Sample}
%\iffalse
%<*samplemain>
%\fi
%
% The following presents a sample document
% with two chapters, two parts, a title page,
% a compile flag as well as three forwarding files to set the flag.
% It consists of eight |.tex| files:
% \begin{center}
% \begin{tabular}{ll}
% |cdocsamp.tex|&main file\\
% |cdocsch1.tex|&include file for chapter 1\\
% |cdocsch2.tex|&include file for chapter 2\\
% |cdocspt3.tex|&include file for part 3\\
% |cdocspt4.tex|&include file for part 4\\
% |cdocsdrf.tex|&forwarding file for main file in draft mode\\
% |cdocsfi1.tex|&forwarding file for final version of chapter 1\\
% |cdocsfi2.tex|&forwarding file for final version of chapter 2\\
% \end{tabular}
% \end{center}
% Each of the eight files can be compiled directly by the \LaTeX{} compiler.
%
% %%%%%%%%%%%%%%%%%%%%%%%%%%%%%%%%%%%%%%
% \paragraph{Main File.}
%
% The main file is called |cdocsamp.tex|.
%
% Load the \textsf{childdoc} definitions and
% declare the filename for the main document:
%    \begin{macrocode}
\input{childdoc.def}
\childdocmain{}
%    \end{macrocode}

% Optional override for |\version| flag:
%    \begin{macrocode}
%%\ifchilddoc\else\providecommand{\version}{draft}\fi
%    \end{macrocode}

% Define the default values for the |\version| flag
% (|final| for the main file and |draft| for childs):
%    \begin{macrocode}
\ifchilddoc
\providecommand{\version}{draft}
\else
\providecommand{\version}{final}
\fi
%    \end{macrocode}

% Load the standard document class:
%    \begin{macrocode}
\documentclass[12pt]{article}
%    \end{macrocode}

% Start the document body:
%    \begin{macrocode}
\begin{document}
%    \end{macrocode}

% Declare a title page.
% Print title, part of document being processed and version flag:
%    \begin{macrocode}
\addtocounter{page}{-1}
\begin{center}
{\LARGE\bfseries{}childdoc example\par}
\vspace{1cm}
\ifchilddoc
\ifchilddocmanual part\else chapter\fi:
`\childdocname' of `\childdocjob'\par
\else
main document: `\childdocjob'\par
\fi
version: \version\par
\end{center}
\newpage
%    \end{macrocode}

% Manually include selected file,
% otherwise process as usual:
%    \begin{macrocode}
\ifchilddocmanual
\section*{part `\childdocname'}
\input{\childdocname}
\else
%    \end{macrocode}

% Include the two chapters:
%    \begin{macrocode}
\include{cdocsch1}
\include{cdocsch2}
%    \end{macrocode}

% Include the two parts unless only chapters should be displayed:
%    \begin{macrocode}
\ifchilddoc\else
\section{part three}
\input{cdocspt3}
\section{part four}
\input{cdocspt4}
\fi
%    \end{macrocode}

% Process as usual until here:
%    \begin{macrocode}
\fi
%    \end{macrocode}

% End of document body:
%    \begin{macrocode}
\end{document}
%    \end{macrocode}
%\iffalse
%</samplemain>
%\fi
%
% %%%%%%%%%%%%%%%%%%%%%%%%%%%%%%%%%%%%%%
% \paragraph{Chapter Include Files.}
%
% The include files are called |cdocsch1.tex| and |cdocsch2.tex|.
%
%\iffalse
%<*samplechap1|samplechap2>
%\fi

% Optional override for |\version| flag:
%    \begin{macrocode}
%%\providecommand{\version}{final}
%    \end{macrocode}

% Include the main document:
%    \begin{macrocode}
\input{childdoc.def}
\childdocof{cdocsamp}
%    \end{macrocode}

%\iffalse
%</samplechap1|samplechap2>
%\fi
%
%\iffalse
%<*samplechap1>
%\fi
% Some text for chapter 1:
%    \begin{macrocode}
\section{one}
some text in chapter one
%    \end{macrocode}

%\iffalse
%</samplechap1>
%\fi
% Some text for chapter 2:
%\iffalse
%<*samplechap2>
%\fi
%    \begin{macrocode}
\section{two}
more text in chapter two
%    \end{macrocode}

%\iffalse
%</samplechap2>
%\fi
%
% %%%%%%%%%%%%%%%%%%%%%%%%%%%%%%%%%%%%%%
% \paragraph{Part Include Files.}
%
% The include files are called |cdocspt3.tex| and |cdocspt4.tex|.
%
%\iffalse
%<*samplepart3|samplepart4>
%\fi

% Optional override for |\version| flag:
%    \begin{macrocode}
%%\providecommand{\version}{final}
%    \end{macrocode}

% Include the main document:
%    \begin{macrocode}
\input{childdoc.def}
\childdocby{cdocsamp}
%    \end{macrocode}

%\iffalse
%</samplepart3|samplepart4>
%\fi
%
%\iffalse
%<*samplepart3>
%\fi
% Some text for part 3:
%    \begin{macrocode}
some text in part three
%    \end{macrocode}

%\iffalse
%</samplepart3>
%\fi
% Some text for part 4:
%\iffalse
%<*samplepart4>
%\fi
%    \begin{macrocode}
more text in part four
%    \end{macrocode}

%\iffalse
%</samplepart4>
%\fi
%
% %%%%%%%%%%%%%%%%%%%%%%%%%%%%%%%%%%%%%%
% \paragraph{Forwarding for a Complete Draft.}
%
% The following forwarding file |cdocsdrf.tex|
% compiles the main document in draft mode:
%\iffalse
%<*sampledraft>
%\fi
%    \begin{macrocode}
\def\version{draft}
\input{childdoc.def}
\childdocforward{cdocsamp}
%    \end{macrocode}

%\iffalse
%</sampledraft>
%\fi
%
% %%%%%%%%%%%%%%%%%%%%%%%%%%%%%%%%%%%%%%
% \paragraph{Forwarding for Final Version of the Chapters.}
%
% The following forwarding files |cdocsfn1.tex| and |cdocsfn2.tex|
% (with identical content)
% compile the final versions of the child documents
% |cdocsch1.tex| and |cdocsch2.tex|, respectively:
%\iffalse
%<*samplefinal>
%\fi
%    \begin{macrocode}
\def\version{final}
\input{childdoc.def}
\childdocforwardprefix[cdocsamp]{cdocsfn}{cdocsch}
%    \end{macrocode}

%\iffalse
%</samplefinal>
%\fi
%
% %%%%%%%%%%%%%%%%%%%%%%%%%%%%%%%%%%%%%%
% \paragraph{Command Line Processing.}
%
% The following three command lines generate the output files
% |cdocscld|, |cdocscl1| and |cdocscl2|
% which should be identical to
% |cdocsdrf|, |cdocsch1| and |cdocsfn2|, respectively:
% \begin{center}
% \begin{tabular}{l}
% |latex -jobname cdocscld \|\\
% |  "\def\version{draft}\input{childdoc.def}\childdocforward{cdocsamp}"|\\
% |latex -jobname cdocscl1 \|\\
% |  "\input{childdoc.def}\childdocforward[cdocsamp]{cdocsch1}"|\\
% |latex -jobname cdocscl2 \|\\
% |  "\def\version{final}\input{childdoc.def}\childdocforward{cdocsch2}"|
% \end{tabular}
% \end{center}
% Note that the trailing backslash on each first line
% merely continues the input to the second line
% (for convenient cut ant paste).
% Furthermore, the command |latex| can be replaced by any
% of its alternative versions such as |pdflatex|.
%
% %%%%%%%%%%%%%%%%%%%%%%%%%%%%%%%%%%%%%%%%%%%%%%%%%%%%%%%%%%%%%%%%%%%%%%%%%%%%%%
% %%%%%%%%%%%%%%%%%%%%%%%%%%%%%%%%%%%%%%%%%%%%%%%%%%%%%%%%%%%%%%%%%%%%%%%%%%%%%%
% \section{Implementation}
%\iffalse
%<*package>
%\fi
%
% This section describes the definitions file |childdoc.def|.

% The definitions cannot be loaded using |\usepackage| or |\RequirePackage|
% which has a mechanism to prevent loading a style file more than once.
% When loading the definitions by means of |\input|
% multiple instances have to be prevented manually:
%\iffalse
%This code needs to be before the `\ProvidesFile' directive
%which is defined at the beginning of this file.
%Therefore it is also placed there and commented out here.
%</package>
%<*discard>
%\fi
%    \begin{macrocode}
\ifdefined\childdocmain\endinput\fi
%    \end{macrocode}
%\iffalse
%</discard>
%<*package>
%\fi
%
% \macro{\ifchilddoc}
% \macro{\ifchilddocmanual}
% The conditional |\ifchilddoc| tells whether a
% child (true) or main (false) document is being compiled.
% The conditional |\ifchilddocmanual| tells whether
% the |\includeonly| mechanism is used (false) or
% the selection of child files must be performed manually (true).
% The definitions initialise to false:
%    \begin{macrocode}
\newif\ifchilddoc
\newif\ifchilddocmanual
%    \end{macrocode}

% \macro{\childdocname}
% \macro{\childdocjob}
% The macro |\childdocname| stores the name of the main document
% to be compiled. The macro |\childdocjob| stores the name of
% the document on which the \LaTeX{} compiler was originally invoked.
% The content of |\jobname| cannot be compared
% to filenames specified in the source due to different catcodes.
% The following code rescans |\jobname|, stores the result
% in |\childdocname| and saves a copy in |\childdocjob|:
%    \begin{macrocode}
\edef\childdocname{\scantokens\expandafter{\jobname\noexpand}}
\let\childdocjob\childdocname
%    \end{macrocode}

% \macro{\childdocdisable}
% The macro |\childdocdisable| prevents the main file
% from being processed more than once.
% At this stage, the main document command |\childdocmain|
% is assumed to be called once again where it should do nothing.
% Any subsequent call to it should prevent
% a secondary processing of the main document
% It overwrites the forwarding commands
% |\childdocof| and |\childdocforward|
% with empty macros to prevent further inclusions of the main document:
%    \begin{macrocode}
\newcommand{\childdocdisable}
{
  \renewcommand{\childdocmain}[1]{\renewcommand{\childdocmain}[1]{\endinput}}
  \renewcommand{\childdocof}[1]{}
  \renewcommand{\childdocby}[2][]{}
  \renewcommand{\childdocforward}[2][]{}
  \renewcommand{\childdocdisable}{}
}
%    \end{macrocode}

% \macro{\childdocmain}
% The macro |\childdocmain| is to be called at the top of the main file
% with nothing or the main filename (without extension) as argument.
% First, it breaks loops.
% If the argument is not empty and does not match |\childdocname|
% (which is set by the first inclusion of |childdoc.def|),
% |\ifchilddoc| is set to true, |\includeonly| is applied to the child file
% and |\jobname| is set to the main file
% (for proper handling of |.aux| files):
%    \begin{macrocode}
\newcommand{\childdocmain}[1]
{
  \childdocdisable\childdocmain{}
  \if?#1?\else
    \begingroup
      \def\childdoctmp{#1}
      \ifx\childdoctmp\childdocname
        \def\childdoctmp{}
      \else
        \def\childdoctmp
        {
          \childdoctrue
          \includeonly{\childdocname}
          \def\childdocjob{#1}
          \def\jobname{#1}
        }
      \fi
      \expandafter
    \endgroup
    \childdoctmp
  \fi
}
%    \end{macrocode}

% \macro{\childdocof}
% The command |\childdocof| redirects
% compilation to the main file |#1|.
%    \begin{macrocode}
\newcommand{\childdocof}[1]
{
  \childdocdisable
  \childdoctrue
  \includeonly{\childdocname}
  \def\jobname{#1}
  \def\childdocjob{#1}
  \input{#1}
}
%    \end{macrocode}

% \macro{\childdocby}
% The command |\childdocby| ....
%    \begin{macrocode}
\newcommand{\childdocby}[2][]
{
  \childdocdisable
  \childdoctrue
  \childdocmanualtrue
  \if?#1?\else
    \def\jobname{#2}
  \fi
  \def\childdocjob{#2}
  \input{#2}
  \endinput
}
%    \end{macrocode}

% \macro{\childdocforward}
% The command |\childdocforward| redirects
% compilation to the main file or
% (if the optional argument is given) a child file.
% Parameters are set as if the main file
% or a child file starting with |\childdocof| was compiled.
% Then compilation is handed over to the main file:
%    \begin{macrocode}
\newcommand{\childdocforward}[2][]
{
  \begingroup
    \if?#1?
      \def\childdoctmp
      {
        \def\childdocname{#2}
        \def\childdocjob{#2}
        \def\jobname{#2}
        \input{#2}
        \endinput
      }
    \else
      \def\childdoctmp
      {
        \childdocdisable
        \def\childdocname{#2}
        \childdoctrue
        \includeonly{#2}
        \def\childdocjob{#1}
        \def\jobname{#1}
        \input{#1}
        \endinput
      }
    \fi
    \expandafter
  \endgroup
  \childdoctmp
}
%    \end{macrocode}

% \macro{\childdocforwardprefix}
% The command |\childdocforwardprefix| redirects
% compilation to the main or a child file by means of a pattern.
% The prefix |#1| in the current filename is replaced by |#2|
% and the suffix of the current filename is kept
% (it is assumed that the filename does not contain the substring `|~~~|'
% which is used as a delimiter).
% Compilation is handed over to the new file by |\childdocforward|:
%    \begin{macrocode}
\newcommand{\childdocforwardprefix}[3][]
{
  \begingroup
    \def\childdocextract #2##1~~~{\def\childdoctmp{\childdocforward[#1]{#3##1}}}
    \expandafter\childdocextract\childdocname~~~
    \expandafter
  \endgroup
  \childdoctmp
}
%    \end{macrocode}

% \macro{\childdoc}
% The deprecated macro |\childdoc| is a legacy version of |\childdocmain|:
%    \begin{macrocode}
\newcommand{\childdoc}{\childdocmain}
%    \end{macrocode}

% \macro{\childdocredirect}
% The deprecated macro |\childdocredirect| is a legacy version
% of |\childdocforward| and |\childdocforwardprefix|:
%    \begin{macrocode}
\newcommand{\childdocredirect}[2][]
{
  \begingroup
    \if?#1?
      \def\childdoctmp{\childdocforward{#2}}
    \else
      \def\childdoctmp{\childdocforwardprefix{#1}{#2}}
    \fi
    \expandafter
  \endgroup
  \childdoctmp
}
%    \end{macrocode}

%\iffalse
%</package>
%\fi
%
\endinput
|\\
|\childdocforwardprefix{final}{child}|
\end{tabular}
\end{center}
%

Note that when several versions of a main file and/or of each child file
are to be generated, it may be convenient to set up a |Makefile| or
shell script to automatise the process.

%%%%%%%%%%%%%%%%%%%%%%%%%%%%%%%%%%%%%%%%%%%%%%%%%%%%%%%%%%%%%%%%%%%%%%%%%%%%%%%%
\subsection{Command Line Processing}
\label{sec:commandline}

The effect of redirection files can also be achieved by invoking
the \LaTeX{} compiler with a more elaborate command line.
Most conveniently this should be done as part
of a shell script or a |Makefile|.

When using \textsf{childdoc} in the main file, the following
command lines effectively perform a redirection
(note that depending on the shell being used,
backslashes may have to be doubled: `|\|' $\to$ `|\\|'):
%
\begin{center}
|... -jobname "|\textit{target}|" |\\|"|[\textit{flags}]%
|% \iffalse
%
% childdoc.dtx Copyright (C) 2017-2018 Niklas Beisert
%
% This work may be distributed and/or modified under the
% conditions of the LaTeX Project Public License, either version 1.3
% of this license or (at your option) any later version.
% The latest version of this license is in
%   http://www.latex-project.org/lppl.txt
% and version 1.3 or later is part of all distributions of LaTeX
% version 2005/12/01 or later.
%
% This work has the LPPL maintenance status `maintained'.
%
% The Current Maintainer of this work is Niklas Beisert.
%
% This work consists of the files childdoc.dtx and childdoc.ins
% and the derived files childdoc.def and cdocsamp.tex with
% cdocsch1.tex, cdocsch2.tex, cdocsdrf.tex, cdocsfn1.tex, cdocsfn2.tex.
%
%<package>\ifdefined\childdocmain\endinput\fi
%<package>\ProvidesFile{childdoc.def}[2018/12/30 v2.0 child document driver]
%<samplemain>\ProvidesFile{cdocsamp.tex}[2018/12/30 v2.0 sample for childdoc]
%<*driver>
%\ProvidesFile{childdoc.drv}[2018/12/30 v2.0 childdoc reference manual file]
\PassOptionsToClass{10pt,a4paper}{article}
\documentclass{ltxdoc}

\usepackage[margin=35mm]{geometry}
\usepackage{hyperref}
\usepackage{hyperxmp}
\usepackage[usenames]{color}

\hypersetup{colorlinks=true}
\hypersetup{pdfstartview=FitH}
\hypersetup{pdfpagemode=UseNone}
\hypersetup{pdfsource={}}
\hypersetup{pdflang={en-UK}}
\hypersetup{pdfcopyright={Copyright 2017-2018 Niklas Beisert.
  This work may be distributed and/or modified under the
  conditions of the LaTeX Project Public License, either version 1.3
  of this license or (at your option) any later version.}}
\hypersetup{pdflicenseurl={http://www.latex-project.org/lppl.txt}}
\hypersetup{pdfcontactaddress={ETH Zurich, ITP, HIT K,
  Wolfgang-Pauli-Strasse 27}}
\hypersetup{pdfcontactpostcode={8093}}
\hypersetup{pdfcontactcity={Zurich}}
\hypersetup{pdfcontactcountry={Switzerland}}
\hypersetup{pdfcontactemail={nbeisert@itp.phys.ethz.ch}}
\hypersetup{pdfcontacturl={http://people.phys.ethz.ch/\xmptilde nbeisert/}}

\newcommand{\secref}[1]{\hyperref[#1]{section \ref*{#1}}}

\parskip1ex
\parindent0pt
\let\olditemize\itemize
\def\itemize{\olditemize\parskip0pt}

\begin{document}

\title{The \textsf{childdoc} Package}
\hypersetup{pdftitle={The childdoc Package}}
\author{Niklas Beisert\\[2ex]
  Institut f\"ur Theoretische Physik\\
  Eidgen\"ossische Technische Hochschule Z\"urich\\
  Wolfgang-Pauli-Strasse 27, 8093 Z\"urich, Switzerland\\[1ex]
  \href{mailto:nbeisert@itp.phys.ethz.ch}
  {\texttt{nbeisert@itp.phys.ethz.ch}}}
\hypersetup{pdfauthor={Niklas Beisert}}
\hypersetup{pdfsubject={Manual for the LaTeX2e Package childdoc}}
\date{30 December 2018, \textsf{v2.0}}
\maketitle

\begin{abstract}\noindent
\textsf{childdoc} is a \LaTeXe{} package
that enables the direct compilation
of document sections included by |\include|
to individual files.
\end{abstract}

\begingroup
\parskip0ex
\tableofcontents
\endgroup

%%%%%%%%%%%%%%%%%%%%%%%%%%%%%%%%%%%%%%%%%%%%%%%%%%%%%%%%%%%%%%%%%%%%%%%%%%%%%%%%
%%%%%%%%%%%%%%%%%%%%%%%%%%%%%%%%%%%%%%%%%%%%%%%%%%%%%%%%%%%%%%%%%%%%%%%%%%%%%%%%
\section{Introduction}

\LaTeX{} provides a mechanism to structure a large document (such as a book)
into a main file and several child files (containing the chapters)
using the |\include| command.
This mechanism is beneficial for documents
which span hundreds of pages in order to
make the source file(s) more manageable.
Moreover, compilation can be restricted to
selected child files by means of the |\includeonly| command.
The latter feature can be used to reduce the compilation time while editing
(this was significantly more useful in the earlier days of \LaTeX{})
or to generate a smaller document which is easier to navigate.
Another application of |\includeonly| is to generate
documents consisting of selected parts of the complete document.

However, there are a few drawbacks of the plain |\include| mechanism:
\begin{itemize}
\item
The child files cannot be compiled on their own,
they can only be compiled via the main file.
A naive editing environment
(such as a text editor with an option
to have the current file processed by \LaTeX)
may require one to switch to the main file before compiling;
attempting to compile the child file produces errors.
\item
The main file must be modified (each time)
to adjust the |\includeonly| command
to the present needs. This easily leaves the main file in a messy state.
\item
The generated document will always carry the filename
of the main document. This is inconvenient if
several child files are to be compiled and
to be kept for distribution.
\end{itemize}

The present package provides a simple interface
to make child files individually compilable by \LaTeX{}.
Compiling a child file then has the same effect as compiling
the main file with an |\includeonly| command
to select the appropriate child.
Moreover the generated document will carry the name of the child
rather than the main file.
This resolves all three above issues.

This feature is meant to make the editing of books,
thesis documents and lecture notes somewhat more convenient.
However, the package can also be used efficiently for
composing a series of documents (such as exercise sheets)
which are typically distributed individually.
It then assists the author in generating the individual documents
(potentially in different versions)
as well as a document containing the collected series.
Another application is in developing style files
or other kinds of included material
where compilation of the style file could redirect
to a sample or test file.

%%%%%%%%%%%%%%%%%%%%%%%%%%%%%%%%%%%%%%%%%%%%%%%%%%%%%%%%%%%%%%%%%%%%%%%%%%%%%%%%
%%%%%%%%%%%%%%%%%%%%%%%%%%%%%%%%%%%%%%%%%%%%%%%%%%%%%%%%%%%%%%%%%%%%%%%%%%%%%%%%
\section{Usage}

First of all, the package \textsf{childdoc} is \emph{not} a standard
\LaTeXe{} |.sty| style file! Therefore it needs to be invoked in
a non-standard way.

%%%%%%%%%%%%%%%%%%%%%%%%%%%%%%%%%%%%%%%%%%%%%%%%%%%%%%%%%%%%%%%%%%%%%%%%%%%%%%%%
\subsection{Included Files}
\label{sec:include}

%%%%%%%%%%%%%%%%%%%%%%%%%%%%%%%%%%%%%%%%
\DescribeMacro{\childdocmain}
To use the package, add the commands
\begin{center}
\begin{tabular}{l}
|\input{childdoc.def}|\\
|\childdocmain{}|\\
\end{tabular}
\end{center}
at the very top of the main \LaTeX{} file,
in particular \emph{before} the |\documentclass| statement!
The argument of |\childdocmain| should be left empty
(but it must be present).

%%%%%%%%%%%%%%%%%%%%%%%%%%%%%%%%%%%%%%%%
\DescribeMacro{\childdocof}
Furthermore, add the commands
\begin{center}
\begin{tabular}{l}
|\input{childdoc.def}|\\
|\childdocof{|\textit{main}|}|\\
\end{tabular}
\end{center}
at the top of every child file \textit{child}
which is included by |\include{|\textit{child}|}|
from within the main file
(or at least for those files to be compiled individually).
The argument \textit{main} must be the filename of the main file.

There are a couple of
considerations in setting up the main and child documents:

%%%%%%%%%%%%%%%%%%%%%%%%%%%%%%%%%%%%%%%%
\paragraph{Restrictions.}

Please note the following restrictions:
\begin{itemize}
\item
|\childdocmain| must be called with one argument \textit{main}
to ensure compatibility with earlier version of the package.
It must either be empty (|\childdocmain{}|)
or precisely match the filename of the main file in which it is specified.
See \secref{sec:detection} for further information.
\item
The filename \textit{main} must be specified without the |.tex| extension.
\item
The filename \textit{main} is case sensitive
(even in case-insensitive file systems)
due to internal string comparison.
\item
The argument \textit{main} should be fully expanded, it cannot be a macro.
\item
Subdirectories and special characters should be avoided in filenames.
\item
The command |\childdocmain{|\textit{main}|}| must be followed by a whitespace.
It should not be followed immediately by another command
or by a comment mark `|%|'.
This is because the \TeX{} parser reads the token immediately following
the argument of |\childdocmain| and puts it
at the beginning of every child section;
however, a white\-space is ignored.
\end{itemize}

%%%%%%%%%%%%%%%%%%%%%%%%%%%%%%%%%%%%%%%%
\paragraph{Content of Main File.}

It is advisable to place all content in the child files included by |\include|.
Any output contained in the main file will appear in all child documents
unless suppressed manually;
it cannot be suppressed automatically by the |\includeonly| directive
and thus should normally be avoided.
A method to include some content in the main file
by means of conditional processing is described in \secref{sec:conditional}.

%%%%%%%%%%%%%%%%%%%%%%%%%%%%%%%%%%%%%%%%
\paragraph{Page Numbering.}

When only a part of the document is compiled,
the appropriate numbering of pages
(as well as other status parameters)
is determined from the |.aux| files.
The latter contain information from previous passes.
However this information needs to propagate through
all intermediate child documents.
Therefore the page numbering in child documents may well
be inconsistent until the complete document is compiled at least once.

A useful (if unconventional) way to always ensure a consistent
page numbering is to restart the numbering in each child document
and denote the pages by `\textit{child}|.|\textit{page}'
where \textit{child} represents the chapter/section number of the child file.
This can be achieved by the command
|\numberwithin{page}{|\textit{child}|}|
of the \textsf{amsmath} package
where \textit{child} can be |chapter| or |section|
depending on the chosen structuring.
Alternatively, one can modify the macro |\thepage| appropriately
and reset the counter |page| at the start of each child file.

%%%%%%%%%%%%%%%%%%%%%%%%%%%%%%%%%%%%%%%%%%%%%%%%%%%%%%%%%%%%%%%%%%%%%%%%%%%%%%%%
\subsection{Conditional Processing}
\label{sec:conditional}

The package provides a mechanism to compile different versions
of a document. To customise the versions further some conditional processing
can come in handy to distinguish which version is being compiled.
The package provides two macros to describe the compilation context:

%%%%%%%%%%%%%%%%%%%%%%%%%%%%%%%%%%%%%%%%
\DescribeMacro{\ifchilddoc}
The conditional |\ifchilddoc| distinguishes between the compilation of
child documents and the main document:
%
\begin{center}
|\ifchilddoc |\textit{child-code}| |[|\||else |\textit{main-code}]| \||fi|
\end{center}

%%%%%%%%%%%%%%%%%%%%%%%%%%%%%%%%%%%%%%%%
\DescribeMacro{\childdocname}
\DescribeMacro{\childdocjob}
The macro |\childdocname| contains the filename (without extension)
of the main or child file being processed.
Note that |\childdocjob| will always contain the name of the main file.

%%%%%%%%%%%%%%%%%%%%%%%%%%%%%%%%%%%%%%%%
\paragraph{Title Page.}

Conditional processing can be used to include a title or banner page
in the main document when proper precautions are taken.
Importantly, the code in the main file should ensure that the page counter
(as well as other status parameters which are stored in the |.aux| files)
takes the same value after the conditional processing.
Otherwise the page numbers may take divergent values
depending on which part is compiled.

For example, a title page could be declared by:
%
\begin{center}
\begin{tabular}{l}
|\ifchilddoc\||else|\\
|\addtocounter{page}{-1}|\\
\textit{code for title page}\\
|\newpage|\\
|\||fi|
\end{tabular}
\end{center}
%
A banner page for the child documents can be generated by:
%
\begin{center}
\begin{tabular}{l}
|\ifchilddoc|\\
|\addtocounter{page}{-1}|\\
\textit{code for banner page}\\
|\newpage|\\
|\||fi|
\end{tabular}
\end{center}
%
Here one could write a message such as:
\begin{center}
|This is the part \childdocname{} of \childdocjob{}.|
\end{center}

%%%%%%%%%%%%%%%%%%%%%%%%%%%%%%%%%%%%%%%%%%%%%%%%%%%%%%%%%%%%%%%%%%%%%%%%%%%%%%%%
\subsection{Flags}
\label{sec:flags}

The package makes it easy to generate different versions
of the main or child documents.
To this end compilation flags can be defined
and assigned different default values.
They will be particularly useful in conjunction
with the forwarding mechanism described in \secref{sec:forward}.

For example, it may be useful to have a flag |\version|
which can be set to |draft| or |final|.
The document source will contain some conditional code
depending on the value of |\version|.
Suppose further, the flag should default to |final| for the main file
and to |draft| for child files
which is a natural assignment for editing the document.
This is achieved by placing the following code
in the preamble of the main document
(below the |\childdocmain| directive):
%
\begin{center}
\begin{tabular}{l}
|\ifchilddoc|\\
|\providecommand{\version}{draft}|\\
|\||else|\\
|\providecommand{\version}{final}|\\
|\||fi|
\end{tabular}
\end{center}
%
The definition by |\providecommand| makes sure
that previous definitions are not overwritten.
Further statements |\providecommand{\version}{...}|
can thus be added before the above code to override it.

For the main file, one might add a line
(between |\childdocmain| and the above block)
%
\begin{center}
|%\ifchilddoc\||else\providecommand{\version}{draft}\||fi|
\end{center}
%
which can be uncommented to produce a draft version.
Likewise one can add a line to the very top of a child file
(above the |\childdocof{|\textit{main}|}| directive)
%
\begin{center}
|%\providecommand{\version}{final}|
\end{center}
%
which can be uncommented to produce the final version of this child document.

%%%%%%%%%%%%%%%%%%%%%%%%%%%%%%%%%%%%%%%%%%%%%%%%%%%%%%%%%%%%%%%%%%%%%%%%%%%%%%%%
\subsection{Forwarding}
\label{sec:forward}

Different versions of the main or child documents
using compilation flags as described in \secref{sec:flags}
can be (permanently) stored in different files
for convenient compilation, viewing and distribution.
To this end, the package defines a command
to pass on compilation to a different file:

%%%%%%%%%%%%%%%%%%%%%%%%%%%%%%%%%%%%%%%%
\DescribeMacro{\childdocforward}
The command |\childdocforward| redirects processing to
another source file:
%
\begin{center}
\begin{tabular}{l}
|\input{childdoc.def}|\\
|\childdocforward[|\textit{main}|]{|\textit{dest}|}|\\
\end{tabular}
\end{center}
%
The argument \textit{dest} is the destination file
(without extension).
It should be the main file or one of the child files.
Note that further \textsf{childdoc} directives
such as |\childdocof| and |\childdocforward|
in the indicated file will be processed in this form.
The optional argument \textit{main}
passes on directly to the main file \textit{main}
while pretending to compile the child \textit{dest}.
This form behaves as if \textit{dest}
issues |\childdocof{|\textit{main}|}| right away,
and no further \textsf{childdoc} directives will be processed.

%%%%%%%%%%%%%%%%%%%%%%%%%%%%%%%%%%%%%%%%
\DescribeMacro{\...prefix}
In the alternative form |\childdocforwardprefix|,
%
\begin{center}
\begin{tabular}{l}
|\input{childdoc.def}|\\
|\childdocforwardprefix[|\textit{main}|]{|\textit{prefix}|}{|\textit{dest}|}|
\end{tabular}
\end{center}
%
the destination file is determined by a pattern
depending on the current file:
To make this work, the current file must be called
`{\textit{prefix}\hspace{0.2em}\textit{suffix}}'
with \textit{prefix} matching precisely the argument.
Processing is then passed on to the file
`{\textit{dest}\hspace{0.2em}\textit{suffix}}'.
Surely, the same effect is achieved by
directly specifying the
argument `{\textit{dest}\hspace{0.2em}\textit{suffix}}'
in the first form.
However, that requires to set up a different file
for each child. With the alternative form of the command
all these files can have exactly the same content
which simplifies setting them up and maintaining them.

For example, the following file |draft.tex|
with a compilation flag |\version| as described in \secref{sec:flags}
compiles the main document as a draft:
%
\begin{center}
\begin{tabular}{l}
|\def\version{draft}|\\
|\input{childdoc.def}|\\
|\childdocforward{|\textit{main}|}|
\end{tabular}
\end{center}
%
Likewise, the following files |final|\textit{nn}|.tex|
compile the final version of the child document
|child|\textit{nn}|.tex|:
%
\begin{center}
\begin{tabular}{l}
|\def\version{final}|\\
|\input{childdoc.def}|\\
|\childdocforwardprefix{final}{child}|
\end{tabular}
\end{center}
%

Note that when several versions of a main file and/or of each child file
are to be generated, it may be convenient to set up a |Makefile| or
shell script to automatise the process.

%%%%%%%%%%%%%%%%%%%%%%%%%%%%%%%%%%%%%%%%%%%%%%%%%%%%%%%%%%%%%%%%%%%%%%%%%%%%%%%%
\subsection{Command Line Processing}
\label{sec:commandline}

The effect of redirection files can also be achieved by invoking
the \LaTeX{} compiler with a more elaborate command line.
Most conveniently this should be done as part
of a shell script or a |Makefile|.

When using \textsf{childdoc} in the main file, the following
command lines effectively perform a redirection
(note that depending on the shell being used,
backslashes may have to be doubled: `|\|' $\to$ `|\\|'):
%
\begin{center}
|... -jobname "|\textit{target}|" |\\|"|[\textit{flags}]%
|\input{childdoc.def}\childdocforward[|\textit{main}|]{|\textit{dest}|}"|
\end{center}
%
Here \textit{target} is the name of the output file,
\textit{main} is the name of the main file
and \textit{dest} is the name of the main or child file to be processed
(all filenames without extensions).
The optional argument \textit{main} can be omitted
if \textit{main} matches \textit{dest}.
Optionally, compilation \textit{flags} can be defined via |\def| commands.
This command line makes the \TeX{} engine believe
it is compiling the file \textit{target}
whose content is specified as the latter parameter.
The provided code then forwards the processing to
\textit{main} or \textit{dest} as described in \secref{sec:forward}.

%%%%%%%%%%%%%%%%%%%%%%%%%%%%%%%%%%%%%%%%%%%%%%%%%%%%%%%%%%%%%%%%%%%%%%%%%%%%%%%%
\subsection{Include by Input}
\label{sec:input}

Including child documents by |\include| has some restrictions by design.
Most notably, the content of a child document always occupies
its own set of pages; pages cannot be shared between child documents.
Usually, this behaviour makes perfect sense
because each child document contain an essential part of the document.
However, in some situations it may be desirable to compose
a document from a collection of parts
without having mandatory page breaks between then.
For this case, the package
provides a mechanism to include parts
by |\input| which can also be processed individually.
However, by construction this mechanism
requires manual handling of the content to be output.

%%%%%%%%%%%%%%%%%%%%%%%%%%%%%%%%%%%%%%%%
\DescribeMacro{\ifchilddocmanual}
The main file should be prepared as usual, see \secref{sec:include}.
However, the document body must make a distinction
between processing of an individual part and of the main document, e.g.:
%
\begin{center}
\begin{tabular}{l}
|\ifchilddocmanual|\\
|\input{\childdocname}|\\
|\||else|\\
\textit{document body with }|\input{|\textit{part}|}|\\
|\||fi|
\end{tabular}
\end{center}
%
The conditional |\ifchilddocmanual| is true whenever
a part to be included by |\input| is being compiled,
and the name of the part is stored in |\childdocname|.

%%%%%%%%%%%%%%%%%%%%%%%%%%%%%%%%%%%%%%%%
\DescribeMacro{\childdocby}
Each part to be included by |\input| should start with:
%
\begin{center}
\begin{tabular}{l}
|\input{childdoc.def}|\\
|\childdocby{|\textit{main}|}|\\
\end{tabular}
\end{center}
%
The directive |\childdocby| is similar to |\childdocof|
described in \secref{sec:include},
but the subsequent selection of content must be done manually.
To that end, both |\ifchilddoc| and |\ifchilddocmanual|
will be true upon processing of a part,
and the name of the part is stored in |\childdocname|.
Note that |\jobname| will be set to the filename of the current part
so that each part receives an individual |.aux| file
that does not interfere with the |.aux| file(s) of the main document.
This behaviour can be altered by the alternative form
|\childdocby[*]{|\textit{main}|}| (with a non-empty optional argument)
which uses the |.aux| file of the main document
by setting |\jobname| to \textit{main}.

%%%%%%%%%%%%%%%%%%%%%%%%%%%%%%%%%%%%%%%%%%%%%%%%%%%%%%%%%%%%%%%%%%%%%%%%%%%%%%%%
\subsection{Driver Development}
\label{sec:driver}

The \textsf{childdoc} mechanism can also be use for the development
of definition files such as \LaTeX{} styles or classes.
This case differs from the above setup with multiple parts
included by |\include| in that no |\includeonly| should be invoked.
This can be achieved by starting the include file
(before |\ProvidesPackage|) with:
%
\begin{center}
\begin{tabular}{l}
|\input{childdoc.def}|\\
|\childdocforward{|\textit{main}|}|\\
\end{tabular}
\end{center}
%
or alternatively with:
%
\begin{center}
\begin{tabular}{l}
|\input{childdoc.def}|\\
|\childdocby{|\textit{main}|}|\\
\end{tabular}
\end{center}
%
Both forms have slightly different effects as described above.
The main file is prepared as usual, see \secref{sec:include}.

%%%%%%%%%%%%%%%%%%%%%%%%%%%%%%%%%%%%%%%%%%%%%%%%%%%%%%%%%%%%%%%%%%%%%%%%%%%%%%%%
\subsection{Legacy Detection}
\label{sec:detection}

The directive |\childdocmain| in the main file can detect
whether the complete document or merely a child is to be compiled
even without using the directive |\childdocof|.
This method is deprecated because it is less robust
and there is no compelling reason to use it;
it is merely provided for backward compatibility
and it may be removed in future versions.

If the detection mechanism is to be used,
it is mandatory to correctly specify
the filename of the main file as the argument of |\childdocmain|:
%
\begin{center}
\begin{tabular}{l}
|\input{childdoc.def}|\\
|\childdocmain{|\textit{main}|}|\\
\end{tabular}
\end{center}
%
If |\jobname| does not match the argument \textit{main} of |\childdocmain|,
it is assumed that |\jobname| points to the child file to be compiled.
When using |\childdocmain| with the main file specified as argument,
it suffices to start a child file
with just |\input{|\textit{main}|}|
without loading of the package and using |\childdocof|.
If instead all processing is done
with the appropriate \textsf{childdoc} directives,
the argument of \textit{main} of |\childdocmain| can be empty.

An alternative version of the command line processing described
in \secref{sec:commandline} using the detection mechanism reads:
%
\begin{center}
|... -jobname "|\textit{target}|" "|[\textit{flags}]%
[|\def\jobname{|\textit{dest}|}|]|\input{|\textit{main}|}"|
\end{center}

%%%%%%%%%%%%%%%%%%%%%%%%%%%%%%%%%%%%%%%%%%%%%%%%%%%%%%%%%%%%%%%%%%%%%%%%%%%%%%%%
\subsection{Manual Code}
\label{sec:manual}

In case one cannot be certain whether the definitions file |childdoc.def|
is installed on the target \TeX{} distribution
and one prefers not to ship it,
it is conceivable to paste a few relevant commands into the sources.

To that end, drop all statements |\input{childdoc.def}|
and perform the replacements as outlined below.
Instead of |\childdocmain{|\textit{main}|}| add the following code
to the top of the main file:
%
\begin{center}
\begin{tabular}{l}
|\||ifdefined\childdocname\endinput\||fi\newif\ifchilddoc|\\
|\edef\childdocname{\scantokens\expandafter{\jobname\noexpand}}|\\
|\def\childdocmain{|\textit{main}|}\||ifx\childdocmain\childdocname\||else|\\
|\childdoctrue\includeonly{\childdocname}\let\jobname\childdocmain\||fi|\\
\end{tabular}
\end{center}
%
Instead of |\childdocof{|\textit{main}|}| just include the main file
at the top of each child file:
%
\begin{center}
|\input{|\textit{main}|}|
\end{center}
%
A simple redirection |\childdocforward{|\textit{dest}|}| is achieved by:
%
\begin{center}
|\def\jobname{|\textit{dest}|}\input{\jobname}|
\end{center}
%
The redirection with prefix
|\childdocforwardprefix[|\textit{prefix}|]{|\textit{dest}|}|
is accomplished by:
%
\begin{center}
\begin{tabular}{l}
|{\edef\jobname{\scantokens\expandafter{\jobname\noexpand}}|\\
|\def\redirectjob |\textit{prefix}|#1~~~{\gdef\jobname{|\textit{dest}|#1}}|\\
|\expandafter\redirectjob\jobname~~~}\input{\jobname}|
\end{tabular}
\end{center}

In an alternative approach,
child documents can be compiled by a specific command line
without additional code or specific definitions:
%
\begin{center}
|... -jobname "|\textit{target}|" "|[\textit{flags}]%
|\includeonly{|\textit{dest}|}\input{|\textit{main}|}"|
\end{center}
%

%%%%%%%%%%%%%%%%%%%%%%%%%%%%%%%%%%%%%%%%%%%%%%%%%%%%%%%%%%%%%%%%%%%%%%%%%%%%%%%%
%%%%%%%%%%%%%%%%%%%%%%%%%%%%%%%%%%%%%%%%%%%%%%%%%%%%%%%%%%%%%%%%%%%%%%%%%%%%%%%%
\section{Information}

%%%%%%%%%%%%%%%%%%%%%%%%%%%%%%%%%%%%%%%%%%%%%%%%%%%%%%%%%%%%%%%%%%%%%%%%%%%%%%%%
\subsection{Copyright}

Copyright \copyright{} 2017--2018 Niklas Beisert

This work may be distributed and/or modified under the
conditions of the \LaTeX{} Project Public License, either version 1.3
of this license or (at your option) any later version.
The latest version of this license is in
  \url{http://www.latex-project.org/lppl.txt}
and version 1.3 or later is part of all distributions of \LaTeX{}
version 2005/12/01 or later.

This work has the LPPL maintenance status `maintained'.

The Current Maintainer of this work is Niklas Beisert.

This work consists of the files |README.txt|, |childdoc.ins| and |childdoc.dtx|
as well as the derived files |childdoc.def|, |cdocsamp.tex|
with |cdocsch1.tex|, |cdocsch2.tex|, |cdocspt3.tex|, |cdocspt4.tex|,
|cdocsdrf.tex|, |cdocsfn1.tex|, |cdocsfn2.tex|
as well as |childdoc.pdf|.

%%%%%%%%%%%%%%%%%%%%%%%%%%%%%%%%%%%%%%%%%%%%%%%%%%%%%%%%%%%%%%%%%%%%%%%%%%%%%%%%
\subsection{Files and Installation}

The package consists of the files:
%
\begin{center}
\begin{tabular}{ll}
    |README.txt|   & readme file \\
    |childdoc.ins| & installation file \\
    |childdoc.dtx| & source file \\
    |childdoc.def| & definition file \\
    |cdocsamp.tex| & sample main file \\
    |cdocsch1.tex| & sample include file \\
    |cdocsch2.tex| & sample include file \\
    |cdocspt3.tex| & sample part file \\
    |cdocspt4.tex| & sample part file \\
    |cdocsdrf.tex| & sample redirection file \\
    |cdocsfn1.tex| & sample redirection file \\
    |cdocsfn2.tex| & sample redirection file \\
    |childdoc.pdf| & manual
\end{tabular}
\end{center}
%
The distribution consists of the files
|README.txt|, |childdoc.ins| and |childdoc.dtx|.
%
\begin{itemize}
\item
Run (pdf)\LaTeX{} on |childdoc.dtx|
to compile the manual |childdoc.pdf| (this file).
\item
Run \LaTeX{} on |childdoc.ins| to create the definitions file |childdoc.def|
and the sample |cdocsamp.tex| with include files
|cdocsch1.tex|, |cdocsch2.tex|, |cdocspt3.tex|, |cdocspt4.tex|,
|cdocsdrf.tex|, |cdocsfn1.tex|, |cdocsfn2.tex|.
Then copy the file |childdoc.def| to an appropriate directory of your \LaTeX{}
distribution, e.g.\ \textit{texmf-root}|/tex/latex/childdoc|.
\end{itemize}

%%%%%%%%%%%%%%%%%%%%%%%%%%%%%%%%%%%%%%%%%%%%%%%%%%%%%%%%%%%%%%%%%%%%%%%%%%%%%%%%
\subsection{Related CTAN Packages}

There are several other packages which offer a similar functionality:
%
\begin{itemize}
\item
The packages
\href{http://ctan.org/pkg/docmute}{\textsf{docmute}},
\href{http://ctan.org/pkg/includex}{\textsf{includex}} and
\href{http://ctan.org/pkg/standalone}{\textsf{standalone}}
provide commands to include only the document body of
a child file thus allowing both files to be compiled individually.
\item
The packages \href{http://ctan.org/pkg/subdocs}{\textsf{subdocs}}
and \href{http://ctan.org/pkg/subfiles}{\textsf{subfiles}}
provide structures in which the main and child documents can be
encapsulated and allowing them to be compiled individually.
The inclusion mechanism is different from the conventional |\include|.
\item
The package \href{http://ctan.org/pkg/combine}{\textsf{combine}}
is an elaborate solution to combine several documents into one.
\end{itemize}
%
See also the CTAN topic \href{http://ctan.org/topic/subdocs}{\textsf{subdocs}}
for further related packages.
The present package differs from the above solutions in that
a document structure constructed with the conventional |\include| mechanism
just needs two extra commands at the top of every file
such that all constituent files can be compiled individually.

%%%%%%%%%%%%%%%%%%%%%%%%%%%%%%%%%%%%%%%%%%%%%%%%%%%%%%%%%%%%%%%%%%%%%%%%%%%%%%%%
%\subsection{Feature Suggestions}
%
%The following is a list of features which may be useful for future
%versions of this package:
%%
%\begin{itemize}
%\item
%\ldots
%\end{itemize}

%%%%%%%%%%%%%%%%%%%%%%%%%%%%%%%%%%%%%%%%%%%%%%%%%%%%%%%%%%%%%%%%%%%%%%%%%%%%%%%%
\subsection{Revision History}

%%%%%%%%%%%%%%%%%%%%%%%%%%%%%%%%%%%%%%%%
\paragraph{v2.0:} 2018/12/30

\begin{itemize}
\item
immediate forward processing
\item
added |\childdocby| mechanism
\item
manual restructured
\end{itemize}

%%%%%%%%%%%%%%%%%%%%%%%%%%%%%%%%%%%%%%%%
\paragraph{v1.6:} 2018/01/17

\begin{itemize}
\item
application for development of include files
\item
corrections to manual
\end{itemize}

%%%%%%%%%%%%%%%%%%%%%%%%%%%%%%%%%%%%%%%%
\paragraph{v1.5:} 2017/05/21

\begin{itemize}
\item
more complete structuring introduced
\item
|\childdocof| introduced
\item
|\childdoc| renamed to |\childdocmain|
\item
|\childredirect| renamed to |\childdocforward| and |\childdocforwardprefix|
and functionality expanded
\end{itemize}

%%%%%%%%%%%%%%%%%%%%%%%%%%%%%%%%%%%%%%%%
\paragraph{v1.0:} 2017/04/27

\begin{itemize}
\item
manual and install package
\item
first version published on CTAN
\end{itemize}

%%%%%%%%%%%%%%%%%%%%%%%%%%%%%%%%%%%%%%%%
\paragraph{v0.6:} 2017/04/26

\begin{itemize}
\item
redirection mechanism added
\end{itemize}

%%%%%%%%%%%%%%%%%%%%%%%%%%%%%%%%%%%%%%%%
\paragraph{v0.5:} 2017/04/26

\begin{itemize}
\item
functionality in definition file
\end{itemize}


%%%%%%%%%%%%%%%%%%%%%%%%%%%%%%%%%%%%%%%%%%%%%%%%%%%%%%%%%%%%%%%%%%%%%%%%%%%%%%%%
%%%%%%%%%%%%%%%%%%%%%%%%%%%%%%%%%%%%%%%%%%%%%%%%%%%%%%%%%%%%%%%%%%%%%%%%%%%%%%%%
%%%%%%%%%%%%%%%%%%%%%%%%%%%%%%%%%%%%%%%%%%%%%%%%%%%%%%%%%%%%%%%%%%%%%%%%%%%%%%%%
\appendix

\settowidth\MacroIndent{\rmfamily\scriptsize 000\ }

 \DocInput{childdoc.dtx}

\end{document}
%</driver>
% \fi
%
% %%%%%%%%%%%%%%%%%%%%%%%%%%%%%%%%%%%%%%%%%%%%%%%%%%%%%%%%%%%%%%%%%%%%%%%%%%%%%%
% %%%%%%%%%%%%%%%%%%%%%%%%%%%%%%%%%%%%%%%%%%%%%%%%%%%%%%%%%%%%%%%%%%%%%%%%%%%%%%
% \section{Sample}
%\iffalse
%<*samplemain>
%\fi
%
% The following presents a sample document
% with two chapters, two parts, a title page,
% a compile flag as well as three forwarding files to set the flag.
% It consists of eight |.tex| files:
% \begin{center}
% \begin{tabular}{ll}
% |cdocsamp.tex|&main file\\
% |cdocsch1.tex|&include file for chapter 1\\
% |cdocsch2.tex|&include file for chapter 2\\
% |cdocspt3.tex|&include file for part 3\\
% |cdocspt4.tex|&include file for part 4\\
% |cdocsdrf.tex|&forwarding file for main file in draft mode\\
% |cdocsfi1.tex|&forwarding file for final version of chapter 1\\
% |cdocsfi2.tex|&forwarding file for final version of chapter 2\\
% \end{tabular}
% \end{center}
% Each of the eight files can be compiled directly by the \LaTeX{} compiler.
%
% %%%%%%%%%%%%%%%%%%%%%%%%%%%%%%%%%%%%%%
% \paragraph{Main File.}
%
% The main file is called |cdocsamp.tex|.
%
% Load the \textsf{childdoc} definitions and
% declare the filename for the main document:
%    \begin{macrocode}
\input{childdoc.def}
\childdocmain{}
%    \end{macrocode}

% Optional override for |\version| flag:
%    \begin{macrocode}
%%\ifchilddoc\else\providecommand{\version}{draft}\fi
%    \end{macrocode}

% Define the default values for the |\version| flag
% (|final| for the main file and |draft| for childs):
%    \begin{macrocode}
\ifchilddoc
\providecommand{\version}{draft}
\else
\providecommand{\version}{final}
\fi
%    \end{macrocode}

% Load the standard document class:
%    \begin{macrocode}
\documentclass[12pt]{article}
%    \end{macrocode}

% Start the document body:
%    \begin{macrocode}
\begin{document}
%    \end{macrocode}

% Declare a title page.
% Print title, part of document being processed and version flag:
%    \begin{macrocode}
\addtocounter{page}{-1}
\begin{center}
{\LARGE\bfseries{}childdoc example\par}
\vspace{1cm}
\ifchilddoc
\ifchilddocmanual part\else chapter\fi:
`\childdocname' of `\childdocjob'\par
\else
main document: `\childdocjob'\par
\fi
version: \version\par
\end{center}
\newpage
%    \end{macrocode}

% Manually include selected file,
% otherwise process as usual:
%    \begin{macrocode}
\ifchilddocmanual
\section*{part `\childdocname'}
\input{\childdocname}
\else
%    \end{macrocode}

% Include the two chapters:
%    \begin{macrocode}
\include{cdocsch1}
\include{cdocsch2}
%    \end{macrocode}

% Include the two parts unless only chapters should be displayed:
%    \begin{macrocode}
\ifchilddoc\else
\section{part three}
\input{cdocspt3}
\section{part four}
\input{cdocspt4}
\fi
%    \end{macrocode}

% Process as usual until here:
%    \begin{macrocode}
\fi
%    \end{macrocode}

% End of document body:
%    \begin{macrocode}
\end{document}
%    \end{macrocode}
%\iffalse
%</samplemain>
%\fi
%
% %%%%%%%%%%%%%%%%%%%%%%%%%%%%%%%%%%%%%%
% \paragraph{Chapter Include Files.}
%
% The include files are called |cdocsch1.tex| and |cdocsch2.tex|.
%
%\iffalse
%<*samplechap1|samplechap2>
%\fi

% Optional override for |\version| flag:
%    \begin{macrocode}
%%\providecommand{\version}{final}
%    \end{macrocode}

% Include the main document:
%    \begin{macrocode}
\input{childdoc.def}
\childdocof{cdocsamp}
%    \end{macrocode}

%\iffalse
%</samplechap1|samplechap2>
%\fi
%
%\iffalse
%<*samplechap1>
%\fi
% Some text for chapter 1:
%    \begin{macrocode}
\section{one}
some text in chapter one
%    \end{macrocode}

%\iffalse
%</samplechap1>
%\fi
% Some text for chapter 2:
%\iffalse
%<*samplechap2>
%\fi
%    \begin{macrocode}
\section{two}
more text in chapter two
%    \end{macrocode}

%\iffalse
%</samplechap2>
%\fi
%
% %%%%%%%%%%%%%%%%%%%%%%%%%%%%%%%%%%%%%%
% \paragraph{Part Include Files.}
%
% The include files are called |cdocspt3.tex| and |cdocspt4.tex|.
%
%\iffalse
%<*samplepart3|samplepart4>
%\fi

% Optional override for |\version| flag:
%    \begin{macrocode}
%%\providecommand{\version}{final}
%    \end{macrocode}

% Include the main document:
%    \begin{macrocode}
\input{childdoc.def}
\childdocby{cdocsamp}
%    \end{macrocode}

%\iffalse
%</samplepart3|samplepart4>
%\fi
%
%\iffalse
%<*samplepart3>
%\fi
% Some text for part 3:
%    \begin{macrocode}
some text in part three
%    \end{macrocode}

%\iffalse
%</samplepart3>
%\fi
% Some text for part 4:
%\iffalse
%<*samplepart4>
%\fi
%    \begin{macrocode}
more text in part four
%    \end{macrocode}

%\iffalse
%</samplepart4>
%\fi
%
% %%%%%%%%%%%%%%%%%%%%%%%%%%%%%%%%%%%%%%
% \paragraph{Forwarding for a Complete Draft.}
%
% The following forwarding file |cdocsdrf.tex|
% compiles the main document in draft mode:
%\iffalse
%<*sampledraft>
%\fi
%    \begin{macrocode}
\def\version{draft}
\input{childdoc.def}
\childdocforward{cdocsamp}
%    \end{macrocode}

%\iffalse
%</sampledraft>
%\fi
%
% %%%%%%%%%%%%%%%%%%%%%%%%%%%%%%%%%%%%%%
% \paragraph{Forwarding for Final Version of the Chapters.}
%
% The following forwarding files |cdocsfn1.tex| and |cdocsfn2.tex|
% (with identical content)
% compile the final versions of the child documents
% |cdocsch1.tex| and |cdocsch2.tex|, respectively:
%\iffalse
%<*samplefinal>
%\fi
%    \begin{macrocode}
\def\version{final}
\input{childdoc.def}
\childdocforwardprefix[cdocsamp]{cdocsfn}{cdocsch}
%    \end{macrocode}

%\iffalse
%</samplefinal>
%\fi
%
% %%%%%%%%%%%%%%%%%%%%%%%%%%%%%%%%%%%%%%
% \paragraph{Command Line Processing.}
%
% The following three command lines generate the output files
% |cdocscld|, |cdocscl1| and |cdocscl2|
% which should be identical to
% |cdocsdrf|, |cdocsch1| and |cdocsfn2|, respectively:
% \begin{center}
% \begin{tabular}{l}
% |latex -jobname cdocscld \|\\
% |  "\def\version{draft}\input{childdoc.def}\childdocforward{cdocsamp}"|\\
% |latex -jobname cdocscl1 \|\\
% |  "\input{childdoc.def}\childdocforward[cdocsamp]{cdocsch1}"|\\
% |latex -jobname cdocscl2 \|\\
% |  "\def\version{final}\input{childdoc.def}\childdocforward{cdocsch2}"|
% \end{tabular}
% \end{center}
% Note that the trailing backslash on each first line
% merely continues the input to the second line
% (for convenient cut ant paste).
% Furthermore, the command |latex| can be replaced by any
% of its alternative versions such as |pdflatex|.
%
% %%%%%%%%%%%%%%%%%%%%%%%%%%%%%%%%%%%%%%%%%%%%%%%%%%%%%%%%%%%%%%%%%%%%%%%%%%%%%%
% %%%%%%%%%%%%%%%%%%%%%%%%%%%%%%%%%%%%%%%%%%%%%%%%%%%%%%%%%%%%%%%%%%%%%%%%%%%%%%
% \section{Implementation}
%\iffalse
%<*package>
%\fi
%
% This section describes the definitions file |childdoc.def|.

% The definitions cannot be loaded using |\usepackage| or |\RequirePackage|
% which has a mechanism to prevent loading a style file more than once.
% When loading the definitions by means of |\input|
% multiple instances have to be prevented manually:
%\iffalse
%This code needs to be before the `\ProvidesFile' directive
%which is defined at the beginning of this file.
%Therefore it is also placed there and commented out here.
%</package>
%<*discard>
%\fi
%    \begin{macrocode}
\ifdefined\childdocmain\endinput\fi
%    \end{macrocode}
%\iffalse
%</discard>
%<*package>
%\fi
%
% \macro{\ifchilddoc}
% \macro{\ifchilddocmanual}
% The conditional |\ifchilddoc| tells whether a
% child (true) or main (false) document is being compiled.
% The conditional |\ifchilddocmanual| tells whether
% the |\includeonly| mechanism is used (false) or
% the selection of child files must be performed manually (true).
% The definitions initialise to false:
%    \begin{macrocode}
\newif\ifchilddoc
\newif\ifchilddocmanual
%    \end{macrocode}

% \macro{\childdocname}
% \macro{\childdocjob}
% The macro |\childdocname| stores the name of the main document
% to be compiled. The macro |\childdocjob| stores the name of
% the document on which the \LaTeX{} compiler was originally invoked.
% The content of |\jobname| cannot be compared
% to filenames specified in the source due to different catcodes.
% The following code rescans |\jobname|, stores the result
% in |\childdocname| and saves a copy in |\childdocjob|:
%    \begin{macrocode}
\edef\childdocname{\scantokens\expandafter{\jobname\noexpand}}
\let\childdocjob\childdocname
%    \end{macrocode}

% \macro{\childdocdisable}
% The macro |\childdocdisable| prevents the main file
% from being processed more than once.
% At this stage, the main document command |\childdocmain|
% is assumed to be called once again where it should do nothing.
% Any subsequent call to it should prevent
% a secondary processing of the main document
% It overwrites the forwarding commands
% |\childdocof| and |\childdocforward|
% with empty macros to prevent further inclusions of the main document:
%    \begin{macrocode}
\newcommand{\childdocdisable}
{
  \renewcommand{\childdocmain}[1]{\renewcommand{\childdocmain}[1]{\endinput}}
  \renewcommand{\childdocof}[1]{}
  \renewcommand{\childdocby}[2][]{}
  \renewcommand{\childdocforward}[2][]{}
  \renewcommand{\childdocdisable}{}
}
%    \end{macrocode}

% \macro{\childdocmain}
% The macro |\childdocmain| is to be called at the top of the main file
% with nothing or the main filename (without extension) as argument.
% First, it breaks loops.
% If the argument is not empty and does not match |\childdocname|
% (which is set by the first inclusion of |childdoc.def|),
% |\ifchilddoc| is set to true, |\includeonly| is applied to the child file
% and |\jobname| is set to the main file
% (for proper handling of |.aux| files):
%    \begin{macrocode}
\newcommand{\childdocmain}[1]
{
  \childdocdisable\childdocmain{}
  \if?#1?\else
    \begingroup
      \def\childdoctmp{#1}
      \ifx\childdoctmp\childdocname
        \def\childdoctmp{}
      \else
        \def\childdoctmp
        {
          \childdoctrue
          \includeonly{\childdocname}
          \def\childdocjob{#1}
          \def\jobname{#1}
        }
      \fi
      \expandafter
    \endgroup
    \childdoctmp
  \fi
}
%    \end{macrocode}

% \macro{\childdocof}
% The command |\childdocof| redirects
% compilation to the main file |#1|.
%    \begin{macrocode}
\newcommand{\childdocof}[1]
{
  \childdocdisable
  \childdoctrue
  \includeonly{\childdocname}
  \def\jobname{#1}
  \def\childdocjob{#1}
  \input{#1}
}
%    \end{macrocode}

% \macro{\childdocby}
% The command |\childdocby| ....
%    \begin{macrocode}
\newcommand{\childdocby}[2][]
{
  \childdocdisable
  \childdoctrue
  \childdocmanualtrue
  \if?#1?\else
    \def\jobname{#2}
  \fi
  \def\childdocjob{#2}
  \input{#2}
  \endinput
}
%    \end{macrocode}

% \macro{\childdocforward}
% The command |\childdocforward| redirects
% compilation to the main file or
% (if the optional argument is given) a child file.
% Parameters are set as if the main file
% or a child file starting with |\childdocof| was compiled.
% Then compilation is handed over to the main file:
%    \begin{macrocode}
\newcommand{\childdocforward}[2][]
{
  \begingroup
    \if?#1?
      \def\childdoctmp
      {
        \def\childdocname{#2}
        \def\childdocjob{#2}
        \def\jobname{#2}
        \input{#2}
        \endinput
      }
    \else
      \def\childdoctmp
      {
        \childdocdisable
        \def\childdocname{#2}
        \childdoctrue
        \includeonly{#2}
        \def\childdocjob{#1}
        \def\jobname{#1}
        \input{#1}
        \endinput
      }
    \fi
    \expandafter
  \endgroup
  \childdoctmp
}
%    \end{macrocode}

% \macro{\childdocforwardprefix}
% The command |\childdocforwardprefix| redirects
% compilation to the main or a child file by means of a pattern.
% The prefix |#1| in the current filename is replaced by |#2|
% and the suffix of the current filename is kept
% (it is assumed that the filename does not contain the substring `|~~~|'
% which is used as a delimiter).
% Compilation is handed over to the new file by |\childdocforward|:
%    \begin{macrocode}
\newcommand{\childdocforwardprefix}[3][]
{
  \begingroup
    \def\childdocextract #2##1~~~{\def\childdoctmp{\childdocforward[#1]{#3##1}}}
    \expandafter\childdocextract\childdocname~~~
    \expandafter
  \endgroup
  \childdoctmp
}
%    \end{macrocode}

% \macro{\childdoc}
% The deprecated macro |\childdoc| is a legacy version of |\childdocmain|:
%    \begin{macrocode}
\newcommand{\childdoc}{\childdocmain}
%    \end{macrocode}

% \macro{\childdocredirect}
% The deprecated macro |\childdocredirect| is a legacy version
% of |\childdocforward| and |\childdocforwardprefix|:
%    \begin{macrocode}
\newcommand{\childdocredirect}[2][]
{
  \begingroup
    \if?#1?
      \def\childdoctmp{\childdocforward{#2}}
    \else
      \def\childdoctmp{\childdocforwardprefix{#1}{#2}}
    \fi
    \expandafter
  \endgroup
  \childdoctmp
}
%    \end{macrocode}

%\iffalse
%</package>
%\fi
%
\endinput
\childdocforward[|\textit{main}|]{|\textit{dest}|}"|
\end{center}
%
Here \textit{target} is the name of the output file,
\textit{main} is the name of the main file
and \textit{dest} is the name of the main or child file to be processed
(all filenames without extensions).
The optional argument \textit{main} can be omitted
if \textit{main} matches \textit{dest}.
Optionally, compilation \textit{flags} can be defined via |\def| commands.
This command line makes the \TeX{} engine believe
it is compiling the file \textit{target}
whose content is specified as the latter parameter.
The provided code then forwards the processing to
\textit{main} or \textit{dest} as described in \secref{sec:forward}.

%%%%%%%%%%%%%%%%%%%%%%%%%%%%%%%%%%%%%%%%%%%%%%%%%%%%%%%%%%%%%%%%%%%%%%%%%%%%%%%%
\subsection{Include by Input}
\label{sec:input}

Including child documents by |\include| has some restrictions by design.
Most notably, the content of a child document always occupies
its own set of pages; pages cannot be shared between child documents.
Usually, this behaviour makes perfect sense
because each child document contain an essential part of the document.
However, in some situations it may be desirable to compose
a document from a collection of parts
without having mandatory page breaks between then.
For this case, the package
provides a mechanism to include parts
by |\input| which can also be processed individually.
However, by construction this mechanism
requires manual handling of the content to be output.

%%%%%%%%%%%%%%%%%%%%%%%%%%%%%%%%%%%%%%%%
\DescribeMacro{\ifchilddocmanual}
The main file should be prepared as usual, see \secref{sec:include}.
However, the document body must make a distinction
between processing of an individual part and of the main document, e.g.:
%
\begin{center}
\begin{tabular}{l}
|\ifchilddocmanual|\\
|\input{\childdocname}|\\
|\||else|\\
\textit{document body with }|\input{|\textit{part}|}|\\
|\||fi|
\end{tabular}
\end{center}
%
The conditional |\ifchilddocmanual| is true whenever
a part to be included by |\input| is being compiled,
and the name of the part is stored in |\childdocname|.

%%%%%%%%%%%%%%%%%%%%%%%%%%%%%%%%%%%%%%%%
\DescribeMacro{\childdocby}
Each part to be included by |\input| should start with:
%
\begin{center}
\begin{tabular}{l}
|% \iffalse
%
% childdoc.dtx Copyright (C) 2017-2018 Niklas Beisert
%
% This work may be distributed and/or modified under the
% conditions of the LaTeX Project Public License, either version 1.3
% of this license or (at your option) any later version.
% The latest version of this license is in
%   http://www.latex-project.org/lppl.txt
% and version 1.3 or later is part of all distributions of LaTeX
% version 2005/12/01 or later.
%
% This work has the LPPL maintenance status `maintained'.
%
% The Current Maintainer of this work is Niklas Beisert.
%
% This work consists of the files childdoc.dtx and childdoc.ins
% and the derived files childdoc.def and cdocsamp.tex with
% cdocsch1.tex, cdocsch2.tex, cdocsdrf.tex, cdocsfn1.tex, cdocsfn2.tex.
%
%<package>\ifdefined\childdocmain\endinput\fi
%<package>\ProvidesFile{childdoc.def}[2018/12/30 v2.0 child document driver]
%<samplemain>\ProvidesFile{cdocsamp.tex}[2018/12/30 v2.0 sample for childdoc]
%<*driver>
%\ProvidesFile{childdoc.drv}[2018/12/30 v2.0 childdoc reference manual file]
\PassOptionsToClass{10pt,a4paper}{article}
\documentclass{ltxdoc}

\usepackage[margin=35mm]{geometry}
\usepackage{hyperref}
\usepackage{hyperxmp}
\usepackage[usenames]{color}

\hypersetup{colorlinks=true}
\hypersetup{pdfstartview=FitH}
\hypersetup{pdfpagemode=UseNone}
\hypersetup{pdfsource={}}
\hypersetup{pdflang={en-UK}}
\hypersetup{pdfcopyright={Copyright 2017-2018 Niklas Beisert.
  This work may be distributed and/or modified under the
  conditions of the LaTeX Project Public License, either version 1.3
  of this license or (at your option) any later version.}}
\hypersetup{pdflicenseurl={http://www.latex-project.org/lppl.txt}}
\hypersetup{pdfcontactaddress={ETH Zurich, ITP, HIT K,
  Wolfgang-Pauli-Strasse 27}}
\hypersetup{pdfcontactpostcode={8093}}
\hypersetup{pdfcontactcity={Zurich}}
\hypersetup{pdfcontactcountry={Switzerland}}
\hypersetup{pdfcontactemail={nbeisert@itp.phys.ethz.ch}}
\hypersetup{pdfcontacturl={http://people.phys.ethz.ch/\xmptilde nbeisert/}}

\newcommand{\secref}[1]{\hyperref[#1]{section \ref*{#1}}}

\parskip1ex
\parindent0pt
\let\olditemize\itemize
\def\itemize{\olditemize\parskip0pt}

\begin{document}

\title{The \textsf{childdoc} Package}
\hypersetup{pdftitle={The childdoc Package}}
\author{Niklas Beisert\\[2ex]
  Institut f\"ur Theoretische Physik\\
  Eidgen\"ossische Technische Hochschule Z\"urich\\
  Wolfgang-Pauli-Strasse 27, 8093 Z\"urich, Switzerland\\[1ex]
  \href{mailto:nbeisert@itp.phys.ethz.ch}
  {\texttt{nbeisert@itp.phys.ethz.ch}}}
\hypersetup{pdfauthor={Niklas Beisert}}
\hypersetup{pdfsubject={Manual for the LaTeX2e Package childdoc}}
\date{30 December 2018, \textsf{v2.0}}
\maketitle

\begin{abstract}\noindent
\textsf{childdoc} is a \LaTeXe{} package
that enables the direct compilation
of document sections included by |\include|
to individual files.
\end{abstract}

\begingroup
\parskip0ex
\tableofcontents
\endgroup

%%%%%%%%%%%%%%%%%%%%%%%%%%%%%%%%%%%%%%%%%%%%%%%%%%%%%%%%%%%%%%%%%%%%%%%%%%%%%%%%
%%%%%%%%%%%%%%%%%%%%%%%%%%%%%%%%%%%%%%%%%%%%%%%%%%%%%%%%%%%%%%%%%%%%%%%%%%%%%%%%
\section{Introduction}

\LaTeX{} provides a mechanism to structure a large document (such as a book)
into a main file and several child files (containing the chapters)
using the |\include| command.
This mechanism is beneficial for documents
which span hundreds of pages in order to
make the source file(s) more manageable.
Moreover, compilation can be restricted to
selected child files by means of the |\includeonly| command.
The latter feature can be used to reduce the compilation time while editing
(this was significantly more useful in the earlier days of \LaTeX{})
or to generate a smaller document which is easier to navigate.
Another application of |\includeonly| is to generate
documents consisting of selected parts of the complete document.

However, there are a few drawbacks of the plain |\include| mechanism:
\begin{itemize}
\item
The child files cannot be compiled on their own,
they can only be compiled via the main file.
A naive editing environment
(such as a text editor with an option
to have the current file processed by \LaTeX)
may require one to switch to the main file before compiling;
attempting to compile the child file produces errors.
\item
The main file must be modified (each time)
to adjust the |\includeonly| command
to the present needs. This easily leaves the main file in a messy state.
\item
The generated document will always carry the filename
of the main document. This is inconvenient if
several child files are to be compiled and
to be kept for distribution.
\end{itemize}

The present package provides a simple interface
to make child files individually compilable by \LaTeX{}.
Compiling a child file then has the same effect as compiling
the main file with an |\includeonly| command
to select the appropriate child.
Moreover the generated document will carry the name of the child
rather than the main file.
This resolves all three above issues.

This feature is meant to make the editing of books,
thesis documents and lecture notes somewhat more convenient.
However, the package can also be used efficiently for
composing a series of documents (such as exercise sheets)
which are typically distributed individually.
It then assists the author in generating the individual documents
(potentially in different versions)
as well as a document containing the collected series.
Another application is in developing style files
or other kinds of included material
where compilation of the style file could redirect
to a sample or test file.

%%%%%%%%%%%%%%%%%%%%%%%%%%%%%%%%%%%%%%%%%%%%%%%%%%%%%%%%%%%%%%%%%%%%%%%%%%%%%%%%
%%%%%%%%%%%%%%%%%%%%%%%%%%%%%%%%%%%%%%%%%%%%%%%%%%%%%%%%%%%%%%%%%%%%%%%%%%%%%%%%
\section{Usage}

First of all, the package \textsf{childdoc} is \emph{not} a standard
\LaTeXe{} |.sty| style file! Therefore it needs to be invoked in
a non-standard way.

%%%%%%%%%%%%%%%%%%%%%%%%%%%%%%%%%%%%%%%%%%%%%%%%%%%%%%%%%%%%%%%%%%%%%%%%%%%%%%%%
\subsection{Included Files}
\label{sec:include}

%%%%%%%%%%%%%%%%%%%%%%%%%%%%%%%%%%%%%%%%
\DescribeMacro{\childdocmain}
To use the package, add the commands
\begin{center}
\begin{tabular}{l}
|\input{childdoc.def}|\\
|\childdocmain{}|\\
\end{tabular}
\end{center}
at the very top of the main \LaTeX{} file,
in particular \emph{before} the |\documentclass| statement!
The argument of |\childdocmain| should be left empty
(but it must be present).

%%%%%%%%%%%%%%%%%%%%%%%%%%%%%%%%%%%%%%%%
\DescribeMacro{\childdocof}
Furthermore, add the commands
\begin{center}
\begin{tabular}{l}
|\input{childdoc.def}|\\
|\childdocof{|\textit{main}|}|\\
\end{tabular}
\end{center}
at the top of every child file \textit{child}
which is included by |\include{|\textit{child}|}|
from within the main file
(or at least for those files to be compiled individually).
The argument \textit{main} must be the filename of the main file.

There are a couple of
considerations in setting up the main and child documents:

%%%%%%%%%%%%%%%%%%%%%%%%%%%%%%%%%%%%%%%%
\paragraph{Restrictions.}

Please note the following restrictions:
\begin{itemize}
\item
|\childdocmain| must be called with one argument \textit{main}
to ensure compatibility with earlier version of the package.
It must either be empty (|\childdocmain{}|)
or precisely match the filename of the main file in which it is specified.
See \secref{sec:detection} for further information.
\item
The filename \textit{main} must be specified without the |.tex| extension.
\item
The filename \textit{main} is case sensitive
(even in case-insensitive file systems)
due to internal string comparison.
\item
The argument \textit{main} should be fully expanded, it cannot be a macro.
\item
Subdirectories and special characters should be avoided in filenames.
\item
The command |\childdocmain{|\textit{main}|}| must be followed by a whitespace.
It should not be followed immediately by another command
or by a comment mark `|%|'.
This is because the \TeX{} parser reads the token immediately following
the argument of |\childdocmain| and puts it
at the beginning of every child section;
however, a white\-space is ignored.
\end{itemize}

%%%%%%%%%%%%%%%%%%%%%%%%%%%%%%%%%%%%%%%%
\paragraph{Content of Main File.}

It is advisable to place all content in the child files included by |\include|.
Any output contained in the main file will appear in all child documents
unless suppressed manually;
it cannot be suppressed automatically by the |\includeonly| directive
and thus should normally be avoided.
A method to include some content in the main file
by means of conditional processing is described in \secref{sec:conditional}.

%%%%%%%%%%%%%%%%%%%%%%%%%%%%%%%%%%%%%%%%
\paragraph{Page Numbering.}

When only a part of the document is compiled,
the appropriate numbering of pages
(as well as other status parameters)
is determined from the |.aux| files.
The latter contain information from previous passes.
However this information needs to propagate through
all intermediate child documents.
Therefore the page numbering in child documents may well
be inconsistent until the complete document is compiled at least once.

A useful (if unconventional) way to always ensure a consistent
page numbering is to restart the numbering in each child document
and denote the pages by `\textit{child}|.|\textit{page}'
where \textit{child} represents the chapter/section number of the child file.
This can be achieved by the command
|\numberwithin{page}{|\textit{child}|}|
of the \textsf{amsmath} package
where \textit{child} can be |chapter| or |section|
depending on the chosen structuring.
Alternatively, one can modify the macro |\thepage| appropriately
and reset the counter |page| at the start of each child file.

%%%%%%%%%%%%%%%%%%%%%%%%%%%%%%%%%%%%%%%%%%%%%%%%%%%%%%%%%%%%%%%%%%%%%%%%%%%%%%%%
\subsection{Conditional Processing}
\label{sec:conditional}

The package provides a mechanism to compile different versions
of a document. To customise the versions further some conditional processing
can come in handy to distinguish which version is being compiled.
The package provides two macros to describe the compilation context:

%%%%%%%%%%%%%%%%%%%%%%%%%%%%%%%%%%%%%%%%
\DescribeMacro{\ifchilddoc}
The conditional |\ifchilddoc| distinguishes between the compilation of
child documents and the main document:
%
\begin{center}
|\ifchilddoc |\textit{child-code}| |[|\||else |\textit{main-code}]| \||fi|
\end{center}

%%%%%%%%%%%%%%%%%%%%%%%%%%%%%%%%%%%%%%%%
\DescribeMacro{\childdocname}
\DescribeMacro{\childdocjob}
The macro |\childdocname| contains the filename (without extension)
of the main or child file being processed.
Note that |\childdocjob| will always contain the name of the main file.

%%%%%%%%%%%%%%%%%%%%%%%%%%%%%%%%%%%%%%%%
\paragraph{Title Page.}

Conditional processing can be used to include a title or banner page
in the main document when proper precautions are taken.
Importantly, the code in the main file should ensure that the page counter
(as well as other status parameters which are stored in the |.aux| files)
takes the same value after the conditional processing.
Otherwise the page numbers may take divergent values
depending on which part is compiled.

For example, a title page could be declared by:
%
\begin{center}
\begin{tabular}{l}
|\ifchilddoc\||else|\\
|\addtocounter{page}{-1}|\\
\textit{code for title page}\\
|\newpage|\\
|\||fi|
\end{tabular}
\end{center}
%
A banner page for the child documents can be generated by:
%
\begin{center}
\begin{tabular}{l}
|\ifchilddoc|\\
|\addtocounter{page}{-1}|\\
\textit{code for banner page}\\
|\newpage|\\
|\||fi|
\end{tabular}
\end{center}
%
Here one could write a message such as:
\begin{center}
|This is the part \childdocname{} of \childdocjob{}.|
\end{center}

%%%%%%%%%%%%%%%%%%%%%%%%%%%%%%%%%%%%%%%%%%%%%%%%%%%%%%%%%%%%%%%%%%%%%%%%%%%%%%%%
\subsection{Flags}
\label{sec:flags}

The package makes it easy to generate different versions
of the main or child documents.
To this end compilation flags can be defined
and assigned different default values.
They will be particularly useful in conjunction
with the forwarding mechanism described in \secref{sec:forward}.

For example, it may be useful to have a flag |\version|
which can be set to |draft| or |final|.
The document source will contain some conditional code
depending on the value of |\version|.
Suppose further, the flag should default to |final| for the main file
and to |draft| for child files
which is a natural assignment for editing the document.
This is achieved by placing the following code
in the preamble of the main document
(below the |\childdocmain| directive):
%
\begin{center}
\begin{tabular}{l}
|\ifchilddoc|\\
|\providecommand{\version}{draft}|\\
|\||else|\\
|\providecommand{\version}{final}|\\
|\||fi|
\end{tabular}
\end{center}
%
The definition by |\providecommand| makes sure
that previous definitions are not overwritten.
Further statements |\providecommand{\version}{...}|
can thus be added before the above code to override it.

For the main file, one might add a line
(between |\childdocmain| and the above block)
%
\begin{center}
|%\ifchilddoc\||else\providecommand{\version}{draft}\||fi|
\end{center}
%
which can be uncommented to produce a draft version.
Likewise one can add a line to the very top of a child file
(above the |\childdocof{|\textit{main}|}| directive)
%
\begin{center}
|%\providecommand{\version}{final}|
\end{center}
%
which can be uncommented to produce the final version of this child document.

%%%%%%%%%%%%%%%%%%%%%%%%%%%%%%%%%%%%%%%%%%%%%%%%%%%%%%%%%%%%%%%%%%%%%%%%%%%%%%%%
\subsection{Forwarding}
\label{sec:forward}

Different versions of the main or child documents
using compilation flags as described in \secref{sec:flags}
can be (permanently) stored in different files
for convenient compilation, viewing and distribution.
To this end, the package defines a command
to pass on compilation to a different file:

%%%%%%%%%%%%%%%%%%%%%%%%%%%%%%%%%%%%%%%%
\DescribeMacro{\childdocforward}
The command |\childdocforward| redirects processing to
another source file:
%
\begin{center}
\begin{tabular}{l}
|\input{childdoc.def}|\\
|\childdocforward[|\textit{main}|]{|\textit{dest}|}|\\
\end{tabular}
\end{center}
%
The argument \textit{dest} is the destination file
(without extension).
It should be the main file or one of the child files.
Note that further \textsf{childdoc} directives
such as |\childdocof| and |\childdocforward|
in the indicated file will be processed in this form.
The optional argument \textit{main}
passes on directly to the main file \textit{main}
while pretending to compile the child \textit{dest}.
This form behaves as if \textit{dest}
issues |\childdocof{|\textit{main}|}| right away,
and no further \textsf{childdoc} directives will be processed.

%%%%%%%%%%%%%%%%%%%%%%%%%%%%%%%%%%%%%%%%
\DescribeMacro{\...prefix}
In the alternative form |\childdocforwardprefix|,
%
\begin{center}
\begin{tabular}{l}
|\input{childdoc.def}|\\
|\childdocforwardprefix[|\textit{main}|]{|\textit{prefix}|}{|\textit{dest}|}|
\end{tabular}
\end{center}
%
the destination file is determined by a pattern
depending on the current file:
To make this work, the current file must be called
`{\textit{prefix}\hspace{0.2em}\textit{suffix}}'
with \textit{prefix} matching precisely the argument.
Processing is then passed on to the file
`{\textit{dest}\hspace{0.2em}\textit{suffix}}'.
Surely, the same effect is achieved by
directly specifying the
argument `{\textit{dest}\hspace{0.2em}\textit{suffix}}'
in the first form.
However, that requires to set up a different file
for each child. With the alternative form of the command
all these files can have exactly the same content
which simplifies setting them up and maintaining them.

For example, the following file |draft.tex|
with a compilation flag |\version| as described in \secref{sec:flags}
compiles the main document as a draft:
%
\begin{center}
\begin{tabular}{l}
|\def\version{draft}|\\
|\input{childdoc.def}|\\
|\childdocforward{|\textit{main}|}|
\end{tabular}
\end{center}
%
Likewise, the following files |final|\textit{nn}|.tex|
compile the final version of the child document
|child|\textit{nn}|.tex|:
%
\begin{center}
\begin{tabular}{l}
|\def\version{final}|\\
|\input{childdoc.def}|\\
|\childdocforwardprefix{final}{child}|
\end{tabular}
\end{center}
%

Note that when several versions of a main file and/or of each child file
are to be generated, it may be convenient to set up a |Makefile| or
shell script to automatise the process.

%%%%%%%%%%%%%%%%%%%%%%%%%%%%%%%%%%%%%%%%%%%%%%%%%%%%%%%%%%%%%%%%%%%%%%%%%%%%%%%%
\subsection{Command Line Processing}
\label{sec:commandline}

The effect of redirection files can also be achieved by invoking
the \LaTeX{} compiler with a more elaborate command line.
Most conveniently this should be done as part
of a shell script or a |Makefile|.

When using \textsf{childdoc} in the main file, the following
command lines effectively perform a redirection
(note that depending on the shell being used,
backslashes may have to be doubled: `|\|' $\to$ `|\\|'):
%
\begin{center}
|... -jobname "|\textit{target}|" |\\|"|[\textit{flags}]%
|\input{childdoc.def}\childdocforward[|\textit{main}|]{|\textit{dest}|}"|
\end{center}
%
Here \textit{target} is the name of the output file,
\textit{main} is the name of the main file
and \textit{dest} is the name of the main or child file to be processed
(all filenames without extensions).
The optional argument \textit{main} can be omitted
if \textit{main} matches \textit{dest}.
Optionally, compilation \textit{flags} can be defined via |\def| commands.
This command line makes the \TeX{} engine believe
it is compiling the file \textit{target}
whose content is specified as the latter parameter.
The provided code then forwards the processing to
\textit{main} or \textit{dest} as described in \secref{sec:forward}.

%%%%%%%%%%%%%%%%%%%%%%%%%%%%%%%%%%%%%%%%%%%%%%%%%%%%%%%%%%%%%%%%%%%%%%%%%%%%%%%%
\subsection{Include by Input}
\label{sec:input}

Including child documents by |\include| has some restrictions by design.
Most notably, the content of a child document always occupies
its own set of pages; pages cannot be shared between child documents.
Usually, this behaviour makes perfect sense
because each child document contain an essential part of the document.
However, in some situations it may be desirable to compose
a document from a collection of parts
without having mandatory page breaks between then.
For this case, the package
provides a mechanism to include parts
by |\input| which can also be processed individually.
However, by construction this mechanism
requires manual handling of the content to be output.

%%%%%%%%%%%%%%%%%%%%%%%%%%%%%%%%%%%%%%%%
\DescribeMacro{\ifchilddocmanual}
The main file should be prepared as usual, see \secref{sec:include}.
However, the document body must make a distinction
between processing of an individual part and of the main document, e.g.:
%
\begin{center}
\begin{tabular}{l}
|\ifchilddocmanual|\\
|\input{\childdocname}|\\
|\||else|\\
\textit{document body with }|\input{|\textit{part}|}|\\
|\||fi|
\end{tabular}
\end{center}
%
The conditional |\ifchilddocmanual| is true whenever
a part to be included by |\input| is being compiled,
and the name of the part is stored in |\childdocname|.

%%%%%%%%%%%%%%%%%%%%%%%%%%%%%%%%%%%%%%%%
\DescribeMacro{\childdocby}
Each part to be included by |\input| should start with:
%
\begin{center}
\begin{tabular}{l}
|\input{childdoc.def}|\\
|\childdocby{|\textit{main}|}|\\
\end{tabular}
\end{center}
%
The directive |\childdocby| is similar to |\childdocof|
described in \secref{sec:include},
but the subsequent selection of content must be done manually.
To that end, both |\ifchilddoc| and |\ifchilddocmanual|
will be true upon processing of a part,
and the name of the part is stored in |\childdocname|.
Note that |\jobname| will be set to the filename of the current part
so that each part receives an individual |.aux| file
that does not interfere with the |.aux| file(s) of the main document.
This behaviour can be altered by the alternative form
|\childdocby[*]{|\textit{main}|}| (with a non-empty optional argument)
which uses the |.aux| file of the main document
by setting |\jobname| to \textit{main}.

%%%%%%%%%%%%%%%%%%%%%%%%%%%%%%%%%%%%%%%%%%%%%%%%%%%%%%%%%%%%%%%%%%%%%%%%%%%%%%%%
\subsection{Driver Development}
\label{sec:driver}

The \textsf{childdoc} mechanism can also be use for the development
of definition files such as \LaTeX{} styles or classes.
This case differs from the above setup with multiple parts
included by |\include| in that no |\includeonly| should be invoked.
This can be achieved by starting the include file
(before |\ProvidesPackage|) with:
%
\begin{center}
\begin{tabular}{l}
|\input{childdoc.def}|\\
|\childdocforward{|\textit{main}|}|\\
\end{tabular}
\end{center}
%
or alternatively with:
%
\begin{center}
\begin{tabular}{l}
|\input{childdoc.def}|\\
|\childdocby{|\textit{main}|}|\\
\end{tabular}
\end{center}
%
Both forms have slightly different effects as described above.
The main file is prepared as usual, see \secref{sec:include}.

%%%%%%%%%%%%%%%%%%%%%%%%%%%%%%%%%%%%%%%%%%%%%%%%%%%%%%%%%%%%%%%%%%%%%%%%%%%%%%%%
\subsection{Legacy Detection}
\label{sec:detection}

The directive |\childdocmain| in the main file can detect
whether the complete document or merely a child is to be compiled
even without using the directive |\childdocof|.
This method is deprecated because it is less robust
and there is no compelling reason to use it;
it is merely provided for backward compatibility
and it may be removed in future versions.

If the detection mechanism is to be used,
it is mandatory to correctly specify
the filename of the main file as the argument of |\childdocmain|:
%
\begin{center}
\begin{tabular}{l}
|\input{childdoc.def}|\\
|\childdocmain{|\textit{main}|}|\\
\end{tabular}
\end{center}
%
If |\jobname| does not match the argument \textit{main} of |\childdocmain|,
it is assumed that |\jobname| points to the child file to be compiled.
When using |\childdocmain| with the main file specified as argument,
it suffices to start a child file
with just |\input{|\textit{main}|}|
without loading of the package and using |\childdocof|.
If instead all processing is done
with the appropriate \textsf{childdoc} directives,
the argument of \textit{main} of |\childdocmain| can be empty.

An alternative version of the command line processing described
in \secref{sec:commandline} using the detection mechanism reads:
%
\begin{center}
|... -jobname "|\textit{target}|" "|[\textit{flags}]%
[|\def\jobname{|\textit{dest}|}|]|\input{|\textit{main}|}"|
\end{center}

%%%%%%%%%%%%%%%%%%%%%%%%%%%%%%%%%%%%%%%%%%%%%%%%%%%%%%%%%%%%%%%%%%%%%%%%%%%%%%%%
\subsection{Manual Code}
\label{sec:manual}

In case one cannot be certain whether the definitions file |childdoc.def|
is installed on the target \TeX{} distribution
and one prefers not to ship it,
it is conceivable to paste a few relevant commands into the sources.

To that end, drop all statements |\input{childdoc.def}|
and perform the replacements as outlined below.
Instead of |\childdocmain{|\textit{main}|}| add the following code
to the top of the main file:
%
\begin{center}
\begin{tabular}{l}
|\||ifdefined\childdocname\endinput\||fi\newif\ifchilddoc|\\
|\edef\childdocname{\scantokens\expandafter{\jobname\noexpand}}|\\
|\def\childdocmain{|\textit{main}|}\||ifx\childdocmain\childdocname\||else|\\
|\childdoctrue\includeonly{\childdocname}\let\jobname\childdocmain\||fi|\\
\end{tabular}
\end{center}
%
Instead of |\childdocof{|\textit{main}|}| just include the main file
at the top of each child file:
%
\begin{center}
|\input{|\textit{main}|}|
\end{center}
%
A simple redirection |\childdocforward{|\textit{dest}|}| is achieved by:
%
\begin{center}
|\def\jobname{|\textit{dest}|}\input{\jobname}|
\end{center}
%
The redirection with prefix
|\childdocforwardprefix[|\textit{prefix}|]{|\textit{dest}|}|
is accomplished by:
%
\begin{center}
\begin{tabular}{l}
|{\edef\jobname{\scantokens\expandafter{\jobname\noexpand}}|\\
|\def\redirectjob |\textit{prefix}|#1~~~{\gdef\jobname{|\textit{dest}|#1}}|\\
|\expandafter\redirectjob\jobname~~~}\input{\jobname}|
\end{tabular}
\end{center}

In an alternative approach,
child documents can be compiled by a specific command line
without additional code or specific definitions:
%
\begin{center}
|... -jobname "|\textit{target}|" "|[\textit{flags}]%
|\includeonly{|\textit{dest}|}\input{|\textit{main}|}"|
\end{center}
%

%%%%%%%%%%%%%%%%%%%%%%%%%%%%%%%%%%%%%%%%%%%%%%%%%%%%%%%%%%%%%%%%%%%%%%%%%%%%%%%%
%%%%%%%%%%%%%%%%%%%%%%%%%%%%%%%%%%%%%%%%%%%%%%%%%%%%%%%%%%%%%%%%%%%%%%%%%%%%%%%%
\section{Information}

%%%%%%%%%%%%%%%%%%%%%%%%%%%%%%%%%%%%%%%%%%%%%%%%%%%%%%%%%%%%%%%%%%%%%%%%%%%%%%%%
\subsection{Copyright}

Copyright \copyright{} 2017--2018 Niklas Beisert

This work may be distributed and/or modified under the
conditions of the \LaTeX{} Project Public License, either version 1.3
of this license or (at your option) any later version.
The latest version of this license is in
  \url{http://www.latex-project.org/lppl.txt}
and version 1.3 or later is part of all distributions of \LaTeX{}
version 2005/12/01 or later.

This work has the LPPL maintenance status `maintained'.

The Current Maintainer of this work is Niklas Beisert.

This work consists of the files |README.txt|, |childdoc.ins| and |childdoc.dtx|
as well as the derived files |childdoc.def|, |cdocsamp.tex|
with |cdocsch1.tex|, |cdocsch2.tex|, |cdocspt3.tex|, |cdocspt4.tex|,
|cdocsdrf.tex|, |cdocsfn1.tex|, |cdocsfn2.tex|
as well as |childdoc.pdf|.

%%%%%%%%%%%%%%%%%%%%%%%%%%%%%%%%%%%%%%%%%%%%%%%%%%%%%%%%%%%%%%%%%%%%%%%%%%%%%%%%
\subsection{Files and Installation}

The package consists of the files:
%
\begin{center}
\begin{tabular}{ll}
    |README.txt|   & readme file \\
    |childdoc.ins| & installation file \\
    |childdoc.dtx| & source file \\
    |childdoc.def| & definition file \\
    |cdocsamp.tex| & sample main file \\
    |cdocsch1.tex| & sample include file \\
    |cdocsch2.tex| & sample include file \\
    |cdocspt3.tex| & sample part file \\
    |cdocspt4.tex| & sample part file \\
    |cdocsdrf.tex| & sample redirection file \\
    |cdocsfn1.tex| & sample redirection file \\
    |cdocsfn2.tex| & sample redirection file \\
    |childdoc.pdf| & manual
\end{tabular}
\end{center}
%
The distribution consists of the files
|README.txt|, |childdoc.ins| and |childdoc.dtx|.
%
\begin{itemize}
\item
Run (pdf)\LaTeX{} on |childdoc.dtx|
to compile the manual |childdoc.pdf| (this file).
\item
Run \LaTeX{} on |childdoc.ins| to create the definitions file |childdoc.def|
and the sample |cdocsamp.tex| with include files
|cdocsch1.tex|, |cdocsch2.tex|, |cdocspt3.tex|, |cdocspt4.tex|,
|cdocsdrf.tex|, |cdocsfn1.tex|, |cdocsfn2.tex|.
Then copy the file |childdoc.def| to an appropriate directory of your \LaTeX{}
distribution, e.g.\ \textit{texmf-root}|/tex/latex/childdoc|.
\end{itemize}

%%%%%%%%%%%%%%%%%%%%%%%%%%%%%%%%%%%%%%%%%%%%%%%%%%%%%%%%%%%%%%%%%%%%%%%%%%%%%%%%
\subsection{Related CTAN Packages}

There are several other packages which offer a similar functionality:
%
\begin{itemize}
\item
The packages
\href{http://ctan.org/pkg/docmute}{\textsf{docmute}},
\href{http://ctan.org/pkg/includex}{\textsf{includex}} and
\href{http://ctan.org/pkg/standalone}{\textsf{standalone}}
provide commands to include only the document body of
a child file thus allowing both files to be compiled individually.
\item
The packages \href{http://ctan.org/pkg/subdocs}{\textsf{subdocs}}
and \href{http://ctan.org/pkg/subfiles}{\textsf{subfiles}}
provide structures in which the main and child documents can be
encapsulated and allowing them to be compiled individually.
The inclusion mechanism is different from the conventional |\include|.
\item
The package \href{http://ctan.org/pkg/combine}{\textsf{combine}}
is an elaborate solution to combine several documents into one.
\end{itemize}
%
See also the CTAN topic \href{http://ctan.org/topic/subdocs}{\textsf{subdocs}}
for further related packages.
The present package differs from the above solutions in that
a document structure constructed with the conventional |\include| mechanism
just needs two extra commands at the top of every file
such that all constituent files can be compiled individually.

%%%%%%%%%%%%%%%%%%%%%%%%%%%%%%%%%%%%%%%%%%%%%%%%%%%%%%%%%%%%%%%%%%%%%%%%%%%%%%%%
%\subsection{Feature Suggestions}
%
%The following is a list of features which may be useful for future
%versions of this package:
%%
%\begin{itemize}
%\item
%\ldots
%\end{itemize}

%%%%%%%%%%%%%%%%%%%%%%%%%%%%%%%%%%%%%%%%%%%%%%%%%%%%%%%%%%%%%%%%%%%%%%%%%%%%%%%%
\subsection{Revision History}

%%%%%%%%%%%%%%%%%%%%%%%%%%%%%%%%%%%%%%%%
\paragraph{v2.0:} 2018/12/30

\begin{itemize}
\item
immediate forward processing
\item
added |\childdocby| mechanism
\item
manual restructured
\end{itemize}

%%%%%%%%%%%%%%%%%%%%%%%%%%%%%%%%%%%%%%%%
\paragraph{v1.6:} 2018/01/17

\begin{itemize}
\item
application for development of include files
\item
corrections to manual
\end{itemize}

%%%%%%%%%%%%%%%%%%%%%%%%%%%%%%%%%%%%%%%%
\paragraph{v1.5:} 2017/05/21

\begin{itemize}
\item
more complete structuring introduced
\item
|\childdocof| introduced
\item
|\childdoc| renamed to |\childdocmain|
\item
|\childredirect| renamed to |\childdocforward| and |\childdocforwardprefix|
and functionality expanded
\end{itemize}

%%%%%%%%%%%%%%%%%%%%%%%%%%%%%%%%%%%%%%%%
\paragraph{v1.0:} 2017/04/27

\begin{itemize}
\item
manual and install package
\item
first version published on CTAN
\end{itemize}

%%%%%%%%%%%%%%%%%%%%%%%%%%%%%%%%%%%%%%%%
\paragraph{v0.6:} 2017/04/26

\begin{itemize}
\item
redirection mechanism added
\end{itemize}

%%%%%%%%%%%%%%%%%%%%%%%%%%%%%%%%%%%%%%%%
\paragraph{v0.5:} 2017/04/26

\begin{itemize}
\item
functionality in definition file
\end{itemize}


%%%%%%%%%%%%%%%%%%%%%%%%%%%%%%%%%%%%%%%%%%%%%%%%%%%%%%%%%%%%%%%%%%%%%%%%%%%%%%%%
%%%%%%%%%%%%%%%%%%%%%%%%%%%%%%%%%%%%%%%%%%%%%%%%%%%%%%%%%%%%%%%%%%%%%%%%%%%%%%%%
%%%%%%%%%%%%%%%%%%%%%%%%%%%%%%%%%%%%%%%%%%%%%%%%%%%%%%%%%%%%%%%%%%%%%%%%%%%%%%%%
\appendix

\settowidth\MacroIndent{\rmfamily\scriptsize 000\ }

 \DocInput{childdoc.dtx}

\end{document}
%</driver>
% \fi
%
% %%%%%%%%%%%%%%%%%%%%%%%%%%%%%%%%%%%%%%%%%%%%%%%%%%%%%%%%%%%%%%%%%%%%%%%%%%%%%%
% %%%%%%%%%%%%%%%%%%%%%%%%%%%%%%%%%%%%%%%%%%%%%%%%%%%%%%%%%%%%%%%%%%%%%%%%%%%%%%
% \section{Sample}
%\iffalse
%<*samplemain>
%\fi
%
% The following presents a sample document
% with two chapters, two parts, a title page,
% a compile flag as well as three forwarding files to set the flag.
% It consists of eight |.tex| files:
% \begin{center}
% \begin{tabular}{ll}
% |cdocsamp.tex|&main file\\
% |cdocsch1.tex|&include file for chapter 1\\
% |cdocsch2.tex|&include file for chapter 2\\
% |cdocspt3.tex|&include file for part 3\\
% |cdocspt4.tex|&include file for part 4\\
% |cdocsdrf.tex|&forwarding file for main file in draft mode\\
% |cdocsfi1.tex|&forwarding file for final version of chapter 1\\
% |cdocsfi2.tex|&forwarding file for final version of chapter 2\\
% \end{tabular}
% \end{center}
% Each of the eight files can be compiled directly by the \LaTeX{} compiler.
%
% %%%%%%%%%%%%%%%%%%%%%%%%%%%%%%%%%%%%%%
% \paragraph{Main File.}
%
% The main file is called |cdocsamp.tex|.
%
% Load the \textsf{childdoc} definitions and
% declare the filename for the main document:
%    \begin{macrocode}
\input{childdoc.def}
\childdocmain{}
%    \end{macrocode}

% Optional override for |\version| flag:
%    \begin{macrocode}
%%\ifchilddoc\else\providecommand{\version}{draft}\fi
%    \end{macrocode}

% Define the default values for the |\version| flag
% (|final| for the main file and |draft| for childs):
%    \begin{macrocode}
\ifchilddoc
\providecommand{\version}{draft}
\else
\providecommand{\version}{final}
\fi
%    \end{macrocode}

% Load the standard document class:
%    \begin{macrocode}
\documentclass[12pt]{article}
%    \end{macrocode}

% Start the document body:
%    \begin{macrocode}
\begin{document}
%    \end{macrocode}

% Declare a title page.
% Print title, part of document being processed and version flag:
%    \begin{macrocode}
\addtocounter{page}{-1}
\begin{center}
{\LARGE\bfseries{}childdoc example\par}
\vspace{1cm}
\ifchilddoc
\ifchilddocmanual part\else chapter\fi:
`\childdocname' of `\childdocjob'\par
\else
main document: `\childdocjob'\par
\fi
version: \version\par
\end{center}
\newpage
%    \end{macrocode}

% Manually include selected file,
% otherwise process as usual:
%    \begin{macrocode}
\ifchilddocmanual
\section*{part `\childdocname'}
\input{\childdocname}
\else
%    \end{macrocode}

% Include the two chapters:
%    \begin{macrocode}
\include{cdocsch1}
\include{cdocsch2}
%    \end{macrocode}

% Include the two parts unless only chapters should be displayed:
%    \begin{macrocode}
\ifchilddoc\else
\section{part three}
\input{cdocspt3}
\section{part four}
\input{cdocspt4}
\fi
%    \end{macrocode}

% Process as usual until here:
%    \begin{macrocode}
\fi
%    \end{macrocode}

% End of document body:
%    \begin{macrocode}
\end{document}
%    \end{macrocode}
%\iffalse
%</samplemain>
%\fi
%
% %%%%%%%%%%%%%%%%%%%%%%%%%%%%%%%%%%%%%%
% \paragraph{Chapter Include Files.}
%
% The include files are called |cdocsch1.tex| and |cdocsch2.tex|.
%
%\iffalse
%<*samplechap1|samplechap2>
%\fi

% Optional override for |\version| flag:
%    \begin{macrocode}
%%\providecommand{\version}{final}
%    \end{macrocode}

% Include the main document:
%    \begin{macrocode}
\input{childdoc.def}
\childdocof{cdocsamp}
%    \end{macrocode}

%\iffalse
%</samplechap1|samplechap2>
%\fi
%
%\iffalse
%<*samplechap1>
%\fi
% Some text for chapter 1:
%    \begin{macrocode}
\section{one}
some text in chapter one
%    \end{macrocode}

%\iffalse
%</samplechap1>
%\fi
% Some text for chapter 2:
%\iffalse
%<*samplechap2>
%\fi
%    \begin{macrocode}
\section{two}
more text in chapter two
%    \end{macrocode}

%\iffalse
%</samplechap2>
%\fi
%
% %%%%%%%%%%%%%%%%%%%%%%%%%%%%%%%%%%%%%%
% \paragraph{Part Include Files.}
%
% The include files are called |cdocspt3.tex| and |cdocspt4.tex|.
%
%\iffalse
%<*samplepart3|samplepart4>
%\fi

% Optional override for |\version| flag:
%    \begin{macrocode}
%%\providecommand{\version}{final}
%    \end{macrocode}

% Include the main document:
%    \begin{macrocode}
\input{childdoc.def}
\childdocby{cdocsamp}
%    \end{macrocode}

%\iffalse
%</samplepart3|samplepart4>
%\fi
%
%\iffalse
%<*samplepart3>
%\fi
% Some text for part 3:
%    \begin{macrocode}
some text in part three
%    \end{macrocode}

%\iffalse
%</samplepart3>
%\fi
% Some text for part 4:
%\iffalse
%<*samplepart4>
%\fi
%    \begin{macrocode}
more text in part four
%    \end{macrocode}

%\iffalse
%</samplepart4>
%\fi
%
% %%%%%%%%%%%%%%%%%%%%%%%%%%%%%%%%%%%%%%
% \paragraph{Forwarding for a Complete Draft.}
%
% The following forwarding file |cdocsdrf.tex|
% compiles the main document in draft mode:
%\iffalse
%<*sampledraft>
%\fi
%    \begin{macrocode}
\def\version{draft}
\input{childdoc.def}
\childdocforward{cdocsamp}
%    \end{macrocode}

%\iffalse
%</sampledraft>
%\fi
%
% %%%%%%%%%%%%%%%%%%%%%%%%%%%%%%%%%%%%%%
% \paragraph{Forwarding for Final Version of the Chapters.}
%
% The following forwarding files |cdocsfn1.tex| and |cdocsfn2.tex|
% (with identical content)
% compile the final versions of the child documents
% |cdocsch1.tex| and |cdocsch2.tex|, respectively:
%\iffalse
%<*samplefinal>
%\fi
%    \begin{macrocode}
\def\version{final}
\input{childdoc.def}
\childdocforwardprefix[cdocsamp]{cdocsfn}{cdocsch}
%    \end{macrocode}

%\iffalse
%</samplefinal>
%\fi
%
% %%%%%%%%%%%%%%%%%%%%%%%%%%%%%%%%%%%%%%
% \paragraph{Command Line Processing.}
%
% The following three command lines generate the output files
% |cdocscld|, |cdocscl1| and |cdocscl2|
% which should be identical to
% |cdocsdrf|, |cdocsch1| and |cdocsfn2|, respectively:
% \begin{center}
% \begin{tabular}{l}
% |latex -jobname cdocscld \|\\
% |  "\def\version{draft}\input{childdoc.def}\childdocforward{cdocsamp}"|\\
% |latex -jobname cdocscl1 \|\\
% |  "\input{childdoc.def}\childdocforward[cdocsamp]{cdocsch1}"|\\
% |latex -jobname cdocscl2 \|\\
% |  "\def\version{final}\input{childdoc.def}\childdocforward{cdocsch2}"|
% \end{tabular}
% \end{center}
% Note that the trailing backslash on each first line
% merely continues the input to the second line
% (for convenient cut ant paste).
% Furthermore, the command |latex| can be replaced by any
% of its alternative versions such as |pdflatex|.
%
% %%%%%%%%%%%%%%%%%%%%%%%%%%%%%%%%%%%%%%%%%%%%%%%%%%%%%%%%%%%%%%%%%%%%%%%%%%%%%%
% %%%%%%%%%%%%%%%%%%%%%%%%%%%%%%%%%%%%%%%%%%%%%%%%%%%%%%%%%%%%%%%%%%%%%%%%%%%%%%
% \section{Implementation}
%\iffalse
%<*package>
%\fi
%
% This section describes the definitions file |childdoc.def|.

% The definitions cannot be loaded using |\usepackage| or |\RequirePackage|
% which has a mechanism to prevent loading a style file more than once.
% When loading the definitions by means of |\input|
% multiple instances have to be prevented manually:
%\iffalse
%This code needs to be before the `\ProvidesFile' directive
%which is defined at the beginning of this file.
%Therefore it is also placed there and commented out here.
%</package>
%<*discard>
%\fi
%    \begin{macrocode}
\ifdefined\childdocmain\endinput\fi
%    \end{macrocode}
%\iffalse
%</discard>
%<*package>
%\fi
%
% \macro{\ifchilddoc}
% \macro{\ifchilddocmanual}
% The conditional |\ifchilddoc| tells whether a
% child (true) or main (false) document is being compiled.
% The conditional |\ifchilddocmanual| tells whether
% the |\includeonly| mechanism is used (false) or
% the selection of child files must be performed manually (true).
% The definitions initialise to false:
%    \begin{macrocode}
\newif\ifchilddoc
\newif\ifchilddocmanual
%    \end{macrocode}

% \macro{\childdocname}
% \macro{\childdocjob}
% The macro |\childdocname| stores the name of the main document
% to be compiled. The macro |\childdocjob| stores the name of
% the document on which the \LaTeX{} compiler was originally invoked.
% The content of |\jobname| cannot be compared
% to filenames specified in the source due to different catcodes.
% The following code rescans |\jobname|, stores the result
% in |\childdocname| and saves a copy in |\childdocjob|:
%    \begin{macrocode}
\edef\childdocname{\scantokens\expandafter{\jobname\noexpand}}
\let\childdocjob\childdocname
%    \end{macrocode}

% \macro{\childdocdisable}
% The macro |\childdocdisable| prevents the main file
% from being processed more than once.
% At this stage, the main document command |\childdocmain|
% is assumed to be called once again where it should do nothing.
% Any subsequent call to it should prevent
% a secondary processing of the main document
% It overwrites the forwarding commands
% |\childdocof| and |\childdocforward|
% with empty macros to prevent further inclusions of the main document:
%    \begin{macrocode}
\newcommand{\childdocdisable}
{
  \renewcommand{\childdocmain}[1]{\renewcommand{\childdocmain}[1]{\endinput}}
  \renewcommand{\childdocof}[1]{}
  \renewcommand{\childdocby}[2][]{}
  \renewcommand{\childdocforward}[2][]{}
  \renewcommand{\childdocdisable}{}
}
%    \end{macrocode}

% \macro{\childdocmain}
% The macro |\childdocmain| is to be called at the top of the main file
% with nothing or the main filename (without extension) as argument.
% First, it breaks loops.
% If the argument is not empty and does not match |\childdocname|
% (which is set by the first inclusion of |childdoc.def|),
% |\ifchilddoc| is set to true, |\includeonly| is applied to the child file
% and |\jobname| is set to the main file
% (for proper handling of |.aux| files):
%    \begin{macrocode}
\newcommand{\childdocmain}[1]
{
  \childdocdisable\childdocmain{}
  \if?#1?\else
    \begingroup
      \def\childdoctmp{#1}
      \ifx\childdoctmp\childdocname
        \def\childdoctmp{}
      \else
        \def\childdoctmp
        {
          \childdoctrue
          \includeonly{\childdocname}
          \def\childdocjob{#1}
          \def\jobname{#1}
        }
      \fi
      \expandafter
    \endgroup
    \childdoctmp
  \fi
}
%    \end{macrocode}

% \macro{\childdocof}
% The command |\childdocof| redirects
% compilation to the main file |#1|.
%    \begin{macrocode}
\newcommand{\childdocof}[1]
{
  \childdocdisable
  \childdoctrue
  \includeonly{\childdocname}
  \def\jobname{#1}
  \def\childdocjob{#1}
  \input{#1}
}
%    \end{macrocode}

% \macro{\childdocby}
% The command |\childdocby| ....
%    \begin{macrocode}
\newcommand{\childdocby}[2][]
{
  \childdocdisable
  \childdoctrue
  \childdocmanualtrue
  \if?#1?\else
    \def\jobname{#2}
  \fi
  \def\childdocjob{#2}
  \input{#2}
  \endinput
}
%    \end{macrocode}

% \macro{\childdocforward}
% The command |\childdocforward| redirects
% compilation to the main file or
% (if the optional argument is given) a child file.
% Parameters are set as if the main file
% or a child file starting with |\childdocof| was compiled.
% Then compilation is handed over to the main file:
%    \begin{macrocode}
\newcommand{\childdocforward}[2][]
{
  \begingroup
    \if?#1?
      \def\childdoctmp
      {
        \def\childdocname{#2}
        \def\childdocjob{#2}
        \def\jobname{#2}
        \input{#2}
        \endinput
      }
    \else
      \def\childdoctmp
      {
        \childdocdisable
        \def\childdocname{#2}
        \childdoctrue
        \includeonly{#2}
        \def\childdocjob{#1}
        \def\jobname{#1}
        \input{#1}
        \endinput
      }
    \fi
    \expandafter
  \endgroup
  \childdoctmp
}
%    \end{macrocode}

% \macro{\childdocforwardprefix}
% The command |\childdocforwardprefix| redirects
% compilation to the main or a child file by means of a pattern.
% The prefix |#1| in the current filename is replaced by |#2|
% and the suffix of the current filename is kept
% (it is assumed that the filename does not contain the substring `|~~~|'
% which is used as a delimiter).
% Compilation is handed over to the new file by |\childdocforward|:
%    \begin{macrocode}
\newcommand{\childdocforwardprefix}[3][]
{
  \begingroup
    \def\childdocextract #2##1~~~{\def\childdoctmp{\childdocforward[#1]{#3##1}}}
    \expandafter\childdocextract\childdocname~~~
    \expandafter
  \endgroup
  \childdoctmp
}
%    \end{macrocode}

% \macro{\childdoc}
% The deprecated macro |\childdoc| is a legacy version of |\childdocmain|:
%    \begin{macrocode}
\newcommand{\childdoc}{\childdocmain}
%    \end{macrocode}

% \macro{\childdocredirect}
% The deprecated macro |\childdocredirect| is a legacy version
% of |\childdocforward| and |\childdocforwardprefix|:
%    \begin{macrocode}
\newcommand{\childdocredirect}[2][]
{
  \begingroup
    \if?#1?
      \def\childdoctmp{\childdocforward{#2}}
    \else
      \def\childdoctmp{\childdocforwardprefix{#1}{#2}}
    \fi
    \expandafter
  \endgroup
  \childdoctmp
}
%    \end{macrocode}

%\iffalse
%</package>
%\fi
%
\endinput
|\\
|\childdocby{|\textit{main}|}|\\
\end{tabular}
\end{center}
%
The directive |\childdocby| is similar to |\childdocof|
described in \secref{sec:include},
but the subsequent selection of content must be done manually.
To that end, both |\ifchilddoc| and |\ifchilddocmanual|
will be true upon processing of a part,
and the name of the part is stored in |\childdocname|.
Note that |\jobname| will be set to the filename of the current part
so that each part receives an individual |.aux| file
that does not interfere with the |.aux| file(s) of the main document.
This behaviour can be altered by the alternative form
|\childdocby[*]{|\textit{main}|}| (with a non-empty optional argument)
which uses the |.aux| file of the main document
by setting |\jobname| to \textit{main}.

%%%%%%%%%%%%%%%%%%%%%%%%%%%%%%%%%%%%%%%%%%%%%%%%%%%%%%%%%%%%%%%%%%%%%%%%%%%%%%%%
\subsection{Driver Development}
\label{sec:driver}

The \textsf{childdoc} mechanism can also be use for the development
of definition files such as \LaTeX{} styles or classes.
This case differs from the above setup with multiple parts
included by |\include| in that no |\includeonly| should be invoked.
This can be achieved by starting the include file
(before |\ProvidesPackage|) with:
%
\begin{center}
\begin{tabular}{l}
|% \iffalse
%
% childdoc.dtx Copyright (C) 2017-2018 Niklas Beisert
%
% This work may be distributed and/or modified under the
% conditions of the LaTeX Project Public License, either version 1.3
% of this license or (at your option) any later version.
% The latest version of this license is in
%   http://www.latex-project.org/lppl.txt
% and version 1.3 or later is part of all distributions of LaTeX
% version 2005/12/01 or later.
%
% This work has the LPPL maintenance status `maintained'.
%
% The Current Maintainer of this work is Niklas Beisert.
%
% This work consists of the files childdoc.dtx and childdoc.ins
% and the derived files childdoc.def and cdocsamp.tex with
% cdocsch1.tex, cdocsch2.tex, cdocsdrf.tex, cdocsfn1.tex, cdocsfn2.tex.
%
%<package>\ifdefined\childdocmain\endinput\fi
%<package>\ProvidesFile{childdoc.def}[2018/12/30 v2.0 child document driver]
%<samplemain>\ProvidesFile{cdocsamp.tex}[2018/12/30 v2.0 sample for childdoc]
%<*driver>
%\ProvidesFile{childdoc.drv}[2018/12/30 v2.0 childdoc reference manual file]
\PassOptionsToClass{10pt,a4paper}{article}
\documentclass{ltxdoc}

\usepackage[margin=35mm]{geometry}
\usepackage{hyperref}
\usepackage{hyperxmp}
\usepackage[usenames]{color}

\hypersetup{colorlinks=true}
\hypersetup{pdfstartview=FitH}
\hypersetup{pdfpagemode=UseNone}
\hypersetup{pdfsource={}}
\hypersetup{pdflang={en-UK}}
\hypersetup{pdfcopyright={Copyright 2017-2018 Niklas Beisert.
  This work may be distributed and/or modified under the
  conditions of the LaTeX Project Public License, either version 1.3
  of this license or (at your option) any later version.}}
\hypersetup{pdflicenseurl={http://www.latex-project.org/lppl.txt}}
\hypersetup{pdfcontactaddress={ETH Zurich, ITP, HIT K,
  Wolfgang-Pauli-Strasse 27}}
\hypersetup{pdfcontactpostcode={8093}}
\hypersetup{pdfcontactcity={Zurich}}
\hypersetup{pdfcontactcountry={Switzerland}}
\hypersetup{pdfcontactemail={nbeisert@itp.phys.ethz.ch}}
\hypersetup{pdfcontacturl={http://people.phys.ethz.ch/\xmptilde nbeisert/}}

\newcommand{\secref}[1]{\hyperref[#1]{section \ref*{#1}}}

\parskip1ex
\parindent0pt
\let\olditemize\itemize
\def\itemize{\olditemize\parskip0pt}

\begin{document}

\title{The \textsf{childdoc} Package}
\hypersetup{pdftitle={The childdoc Package}}
\author{Niklas Beisert\\[2ex]
  Institut f\"ur Theoretische Physik\\
  Eidgen\"ossische Technische Hochschule Z\"urich\\
  Wolfgang-Pauli-Strasse 27, 8093 Z\"urich, Switzerland\\[1ex]
  \href{mailto:nbeisert@itp.phys.ethz.ch}
  {\texttt{nbeisert@itp.phys.ethz.ch}}}
\hypersetup{pdfauthor={Niklas Beisert}}
\hypersetup{pdfsubject={Manual for the LaTeX2e Package childdoc}}
\date{30 December 2018, \textsf{v2.0}}
\maketitle

\begin{abstract}\noindent
\textsf{childdoc} is a \LaTeXe{} package
that enables the direct compilation
of document sections included by |\include|
to individual files.
\end{abstract}

\begingroup
\parskip0ex
\tableofcontents
\endgroup

%%%%%%%%%%%%%%%%%%%%%%%%%%%%%%%%%%%%%%%%%%%%%%%%%%%%%%%%%%%%%%%%%%%%%%%%%%%%%%%%
%%%%%%%%%%%%%%%%%%%%%%%%%%%%%%%%%%%%%%%%%%%%%%%%%%%%%%%%%%%%%%%%%%%%%%%%%%%%%%%%
\section{Introduction}

\LaTeX{} provides a mechanism to structure a large document (such as a book)
into a main file and several child files (containing the chapters)
using the |\include| command.
This mechanism is beneficial for documents
which span hundreds of pages in order to
make the source file(s) more manageable.
Moreover, compilation can be restricted to
selected child files by means of the |\includeonly| command.
The latter feature can be used to reduce the compilation time while editing
(this was significantly more useful in the earlier days of \LaTeX{})
or to generate a smaller document which is easier to navigate.
Another application of |\includeonly| is to generate
documents consisting of selected parts of the complete document.

However, there are a few drawbacks of the plain |\include| mechanism:
\begin{itemize}
\item
The child files cannot be compiled on their own,
they can only be compiled via the main file.
A naive editing environment
(such as a text editor with an option
to have the current file processed by \LaTeX)
may require one to switch to the main file before compiling;
attempting to compile the child file produces errors.
\item
The main file must be modified (each time)
to adjust the |\includeonly| command
to the present needs. This easily leaves the main file in a messy state.
\item
The generated document will always carry the filename
of the main document. This is inconvenient if
several child files are to be compiled and
to be kept for distribution.
\end{itemize}

The present package provides a simple interface
to make child files individually compilable by \LaTeX{}.
Compiling a child file then has the same effect as compiling
the main file with an |\includeonly| command
to select the appropriate child.
Moreover the generated document will carry the name of the child
rather than the main file.
This resolves all three above issues.

This feature is meant to make the editing of books,
thesis documents and lecture notes somewhat more convenient.
However, the package can also be used efficiently for
composing a series of documents (such as exercise sheets)
which are typically distributed individually.
It then assists the author in generating the individual documents
(potentially in different versions)
as well as a document containing the collected series.
Another application is in developing style files
or other kinds of included material
where compilation of the style file could redirect
to a sample or test file.

%%%%%%%%%%%%%%%%%%%%%%%%%%%%%%%%%%%%%%%%%%%%%%%%%%%%%%%%%%%%%%%%%%%%%%%%%%%%%%%%
%%%%%%%%%%%%%%%%%%%%%%%%%%%%%%%%%%%%%%%%%%%%%%%%%%%%%%%%%%%%%%%%%%%%%%%%%%%%%%%%
\section{Usage}

First of all, the package \textsf{childdoc} is \emph{not} a standard
\LaTeXe{} |.sty| style file! Therefore it needs to be invoked in
a non-standard way.

%%%%%%%%%%%%%%%%%%%%%%%%%%%%%%%%%%%%%%%%%%%%%%%%%%%%%%%%%%%%%%%%%%%%%%%%%%%%%%%%
\subsection{Included Files}
\label{sec:include}

%%%%%%%%%%%%%%%%%%%%%%%%%%%%%%%%%%%%%%%%
\DescribeMacro{\childdocmain}
To use the package, add the commands
\begin{center}
\begin{tabular}{l}
|\input{childdoc.def}|\\
|\childdocmain{}|\\
\end{tabular}
\end{center}
at the very top of the main \LaTeX{} file,
in particular \emph{before} the |\documentclass| statement!
The argument of |\childdocmain| should be left empty
(but it must be present).

%%%%%%%%%%%%%%%%%%%%%%%%%%%%%%%%%%%%%%%%
\DescribeMacro{\childdocof}
Furthermore, add the commands
\begin{center}
\begin{tabular}{l}
|\input{childdoc.def}|\\
|\childdocof{|\textit{main}|}|\\
\end{tabular}
\end{center}
at the top of every child file \textit{child}
which is included by |\include{|\textit{child}|}|
from within the main file
(or at least for those files to be compiled individually).
The argument \textit{main} must be the filename of the main file.

There are a couple of
considerations in setting up the main and child documents:

%%%%%%%%%%%%%%%%%%%%%%%%%%%%%%%%%%%%%%%%
\paragraph{Restrictions.}

Please note the following restrictions:
\begin{itemize}
\item
|\childdocmain| must be called with one argument \textit{main}
to ensure compatibility with earlier version of the package.
It must either be empty (|\childdocmain{}|)
or precisely match the filename of the main file in which it is specified.
See \secref{sec:detection} for further information.
\item
The filename \textit{main} must be specified without the |.tex| extension.
\item
The filename \textit{main} is case sensitive
(even in case-insensitive file systems)
due to internal string comparison.
\item
The argument \textit{main} should be fully expanded, it cannot be a macro.
\item
Subdirectories and special characters should be avoided in filenames.
\item
The command |\childdocmain{|\textit{main}|}| must be followed by a whitespace.
It should not be followed immediately by another command
or by a comment mark `|%|'.
This is because the \TeX{} parser reads the token immediately following
the argument of |\childdocmain| and puts it
at the beginning of every child section;
however, a white\-space is ignored.
\end{itemize}

%%%%%%%%%%%%%%%%%%%%%%%%%%%%%%%%%%%%%%%%
\paragraph{Content of Main File.}

It is advisable to place all content in the child files included by |\include|.
Any output contained in the main file will appear in all child documents
unless suppressed manually;
it cannot be suppressed automatically by the |\includeonly| directive
and thus should normally be avoided.
A method to include some content in the main file
by means of conditional processing is described in \secref{sec:conditional}.

%%%%%%%%%%%%%%%%%%%%%%%%%%%%%%%%%%%%%%%%
\paragraph{Page Numbering.}

When only a part of the document is compiled,
the appropriate numbering of pages
(as well as other status parameters)
is determined from the |.aux| files.
The latter contain information from previous passes.
However this information needs to propagate through
all intermediate child documents.
Therefore the page numbering in child documents may well
be inconsistent until the complete document is compiled at least once.

A useful (if unconventional) way to always ensure a consistent
page numbering is to restart the numbering in each child document
and denote the pages by `\textit{child}|.|\textit{page}'
where \textit{child} represents the chapter/section number of the child file.
This can be achieved by the command
|\numberwithin{page}{|\textit{child}|}|
of the \textsf{amsmath} package
where \textit{child} can be |chapter| or |section|
depending on the chosen structuring.
Alternatively, one can modify the macro |\thepage| appropriately
and reset the counter |page| at the start of each child file.

%%%%%%%%%%%%%%%%%%%%%%%%%%%%%%%%%%%%%%%%%%%%%%%%%%%%%%%%%%%%%%%%%%%%%%%%%%%%%%%%
\subsection{Conditional Processing}
\label{sec:conditional}

The package provides a mechanism to compile different versions
of a document. To customise the versions further some conditional processing
can come in handy to distinguish which version is being compiled.
The package provides two macros to describe the compilation context:

%%%%%%%%%%%%%%%%%%%%%%%%%%%%%%%%%%%%%%%%
\DescribeMacro{\ifchilddoc}
The conditional |\ifchilddoc| distinguishes between the compilation of
child documents and the main document:
%
\begin{center}
|\ifchilddoc |\textit{child-code}| |[|\||else |\textit{main-code}]| \||fi|
\end{center}

%%%%%%%%%%%%%%%%%%%%%%%%%%%%%%%%%%%%%%%%
\DescribeMacro{\childdocname}
\DescribeMacro{\childdocjob}
The macro |\childdocname| contains the filename (without extension)
of the main or child file being processed.
Note that |\childdocjob| will always contain the name of the main file.

%%%%%%%%%%%%%%%%%%%%%%%%%%%%%%%%%%%%%%%%
\paragraph{Title Page.}

Conditional processing can be used to include a title or banner page
in the main document when proper precautions are taken.
Importantly, the code in the main file should ensure that the page counter
(as well as other status parameters which are stored in the |.aux| files)
takes the same value after the conditional processing.
Otherwise the page numbers may take divergent values
depending on which part is compiled.

For example, a title page could be declared by:
%
\begin{center}
\begin{tabular}{l}
|\ifchilddoc\||else|\\
|\addtocounter{page}{-1}|\\
\textit{code for title page}\\
|\newpage|\\
|\||fi|
\end{tabular}
\end{center}
%
A banner page for the child documents can be generated by:
%
\begin{center}
\begin{tabular}{l}
|\ifchilddoc|\\
|\addtocounter{page}{-1}|\\
\textit{code for banner page}\\
|\newpage|\\
|\||fi|
\end{tabular}
\end{center}
%
Here one could write a message such as:
\begin{center}
|This is the part \childdocname{} of \childdocjob{}.|
\end{center}

%%%%%%%%%%%%%%%%%%%%%%%%%%%%%%%%%%%%%%%%%%%%%%%%%%%%%%%%%%%%%%%%%%%%%%%%%%%%%%%%
\subsection{Flags}
\label{sec:flags}

The package makes it easy to generate different versions
of the main or child documents.
To this end compilation flags can be defined
and assigned different default values.
They will be particularly useful in conjunction
with the forwarding mechanism described in \secref{sec:forward}.

For example, it may be useful to have a flag |\version|
which can be set to |draft| or |final|.
The document source will contain some conditional code
depending on the value of |\version|.
Suppose further, the flag should default to |final| for the main file
and to |draft| for child files
which is a natural assignment for editing the document.
This is achieved by placing the following code
in the preamble of the main document
(below the |\childdocmain| directive):
%
\begin{center}
\begin{tabular}{l}
|\ifchilddoc|\\
|\providecommand{\version}{draft}|\\
|\||else|\\
|\providecommand{\version}{final}|\\
|\||fi|
\end{tabular}
\end{center}
%
The definition by |\providecommand| makes sure
that previous definitions are not overwritten.
Further statements |\providecommand{\version}{...}|
can thus be added before the above code to override it.

For the main file, one might add a line
(between |\childdocmain| and the above block)
%
\begin{center}
|%\ifchilddoc\||else\providecommand{\version}{draft}\||fi|
\end{center}
%
which can be uncommented to produce a draft version.
Likewise one can add a line to the very top of a child file
(above the |\childdocof{|\textit{main}|}| directive)
%
\begin{center}
|%\providecommand{\version}{final}|
\end{center}
%
which can be uncommented to produce the final version of this child document.

%%%%%%%%%%%%%%%%%%%%%%%%%%%%%%%%%%%%%%%%%%%%%%%%%%%%%%%%%%%%%%%%%%%%%%%%%%%%%%%%
\subsection{Forwarding}
\label{sec:forward}

Different versions of the main or child documents
using compilation flags as described in \secref{sec:flags}
can be (permanently) stored in different files
for convenient compilation, viewing and distribution.
To this end, the package defines a command
to pass on compilation to a different file:

%%%%%%%%%%%%%%%%%%%%%%%%%%%%%%%%%%%%%%%%
\DescribeMacro{\childdocforward}
The command |\childdocforward| redirects processing to
another source file:
%
\begin{center}
\begin{tabular}{l}
|\input{childdoc.def}|\\
|\childdocforward[|\textit{main}|]{|\textit{dest}|}|\\
\end{tabular}
\end{center}
%
The argument \textit{dest} is the destination file
(without extension).
It should be the main file or one of the child files.
Note that further \textsf{childdoc} directives
such as |\childdocof| and |\childdocforward|
in the indicated file will be processed in this form.
The optional argument \textit{main}
passes on directly to the main file \textit{main}
while pretending to compile the child \textit{dest}.
This form behaves as if \textit{dest}
issues |\childdocof{|\textit{main}|}| right away,
and no further \textsf{childdoc} directives will be processed.

%%%%%%%%%%%%%%%%%%%%%%%%%%%%%%%%%%%%%%%%
\DescribeMacro{\...prefix}
In the alternative form |\childdocforwardprefix|,
%
\begin{center}
\begin{tabular}{l}
|\input{childdoc.def}|\\
|\childdocforwardprefix[|\textit{main}|]{|\textit{prefix}|}{|\textit{dest}|}|
\end{tabular}
\end{center}
%
the destination file is determined by a pattern
depending on the current file:
To make this work, the current file must be called
`{\textit{prefix}\hspace{0.2em}\textit{suffix}}'
with \textit{prefix} matching precisely the argument.
Processing is then passed on to the file
`{\textit{dest}\hspace{0.2em}\textit{suffix}}'.
Surely, the same effect is achieved by
directly specifying the
argument `{\textit{dest}\hspace{0.2em}\textit{suffix}}'
in the first form.
However, that requires to set up a different file
for each child. With the alternative form of the command
all these files can have exactly the same content
which simplifies setting them up and maintaining them.

For example, the following file |draft.tex|
with a compilation flag |\version| as described in \secref{sec:flags}
compiles the main document as a draft:
%
\begin{center}
\begin{tabular}{l}
|\def\version{draft}|\\
|\input{childdoc.def}|\\
|\childdocforward{|\textit{main}|}|
\end{tabular}
\end{center}
%
Likewise, the following files |final|\textit{nn}|.tex|
compile the final version of the child document
|child|\textit{nn}|.tex|:
%
\begin{center}
\begin{tabular}{l}
|\def\version{final}|\\
|\input{childdoc.def}|\\
|\childdocforwardprefix{final}{child}|
\end{tabular}
\end{center}
%

Note that when several versions of a main file and/or of each child file
are to be generated, it may be convenient to set up a |Makefile| or
shell script to automatise the process.

%%%%%%%%%%%%%%%%%%%%%%%%%%%%%%%%%%%%%%%%%%%%%%%%%%%%%%%%%%%%%%%%%%%%%%%%%%%%%%%%
\subsection{Command Line Processing}
\label{sec:commandline}

The effect of redirection files can also be achieved by invoking
the \LaTeX{} compiler with a more elaborate command line.
Most conveniently this should be done as part
of a shell script or a |Makefile|.

When using \textsf{childdoc} in the main file, the following
command lines effectively perform a redirection
(note that depending on the shell being used,
backslashes may have to be doubled: `|\|' $\to$ `|\\|'):
%
\begin{center}
|... -jobname "|\textit{target}|" |\\|"|[\textit{flags}]%
|\input{childdoc.def}\childdocforward[|\textit{main}|]{|\textit{dest}|}"|
\end{center}
%
Here \textit{target} is the name of the output file,
\textit{main} is the name of the main file
and \textit{dest} is the name of the main or child file to be processed
(all filenames without extensions).
The optional argument \textit{main} can be omitted
if \textit{main} matches \textit{dest}.
Optionally, compilation \textit{flags} can be defined via |\def| commands.
This command line makes the \TeX{} engine believe
it is compiling the file \textit{target}
whose content is specified as the latter parameter.
The provided code then forwards the processing to
\textit{main} or \textit{dest} as described in \secref{sec:forward}.

%%%%%%%%%%%%%%%%%%%%%%%%%%%%%%%%%%%%%%%%%%%%%%%%%%%%%%%%%%%%%%%%%%%%%%%%%%%%%%%%
\subsection{Include by Input}
\label{sec:input}

Including child documents by |\include| has some restrictions by design.
Most notably, the content of a child document always occupies
its own set of pages; pages cannot be shared between child documents.
Usually, this behaviour makes perfect sense
because each child document contain an essential part of the document.
However, in some situations it may be desirable to compose
a document from a collection of parts
without having mandatory page breaks between then.
For this case, the package
provides a mechanism to include parts
by |\input| which can also be processed individually.
However, by construction this mechanism
requires manual handling of the content to be output.

%%%%%%%%%%%%%%%%%%%%%%%%%%%%%%%%%%%%%%%%
\DescribeMacro{\ifchilddocmanual}
The main file should be prepared as usual, see \secref{sec:include}.
However, the document body must make a distinction
between processing of an individual part and of the main document, e.g.:
%
\begin{center}
\begin{tabular}{l}
|\ifchilddocmanual|\\
|\input{\childdocname}|\\
|\||else|\\
\textit{document body with }|\input{|\textit{part}|}|\\
|\||fi|
\end{tabular}
\end{center}
%
The conditional |\ifchilddocmanual| is true whenever
a part to be included by |\input| is being compiled,
and the name of the part is stored in |\childdocname|.

%%%%%%%%%%%%%%%%%%%%%%%%%%%%%%%%%%%%%%%%
\DescribeMacro{\childdocby}
Each part to be included by |\input| should start with:
%
\begin{center}
\begin{tabular}{l}
|\input{childdoc.def}|\\
|\childdocby{|\textit{main}|}|\\
\end{tabular}
\end{center}
%
The directive |\childdocby| is similar to |\childdocof|
described in \secref{sec:include},
but the subsequent selection of content must be done manually.
To that end, both |\ifchilddoc| and |\ifchilddocmanual|
will be true upon processing of a part,
and the name of the part is stored in |\childdocname|.
Note that |\jobname| will be set to the filename of the current part
so that each part receives an individual |.aux| file
that does not interfere with the |.aux| file(s) of the main document.
This behaviour can be altered by the alternative form
|\childdocby[*]{|\textit{main}|}| (with a non-empty optional argument)
which uses the |.aux| file of the main document
by setting |\jobname| to \textit{main}.

%%%%%%%%%%%%%%%%%%%%%%%%%%%%%%%%%%%%%%%%%%%%%%%%%%%%%%%%%%%%%%%%%%%%%%%%%%%%%%%%
\subsection{Driver Development}
\label{sec:driver}

The \textsf{childdoc} mechanism can also be use for the development
of definition files such as \LaTeX{} styles or classes.
This case differs from the above setup with multiple parts
included by |\include| in that no |\includeonly| should be invoked.
This can be achieved by starting the include file
(before |\ProvidesPackage|) with:
%
\begin{center}
\begin{tabular}{l}
|\input{childdoc.def}|\\
|\childdocforward{|\textit{main}|}|\\
\end{tabular}
\end{center}
%
or alternatively with:
%
\begin{center}
\begin{tabular}{l}
|\input{childdoc.def}|\\
|\childdocby{|\textit{main}|}|\\
\end{tabular}
\end{center}
%
Both forms have slightly different effects as described above.
The main file is prepared as usual, see \secref{sec:include}.

%%%%%%%%%%%%%%%%%%%%%%%%%%%%%%%%%%%%%%%%%%%%%%%%%%%%%%%%%%%%%%%%%%%%%%%%%%%%%%%%
\subsection{Legacy Detection}
\label{sec:detection}

The directive |\childdocmain| in the main file can detect
whether the complete document or merely a child is to be compiled
even without using the directive |\childdocof|.
This method is deprecated because it is less robust
and there is no compelling reason to use it;
it is merely provided for backward compatibility
and it may be removed in future versions.

If the detection mechanism is to be used,
it is mandatory to correctly specify
the filename of the main file as the argument of |\childdocmain|:
%
\begin{center}
\begin{tabular}{l}
|\input{childdoc.def}|\\
|\childdocmain{|\textit{main}|}|\\
\end{tabular}
\end{center}
%
If |\jobname| does not match the argument \textit{main} of |\childdocmain|,
it is assumed that |\jobname| points to the child file to be compiled.
When using |\childdocmain| with the main file specified as argument,
it suffices to start a child file
with just |\input{|\textit{main}|}|
without loading of the package and using |\childdocof|.
If instead all processing is done
with the appropriate \textsf{childdoc} directives,
the argument of \textit{main} of |\childdocmain| can be empty.

An alternative version of the command line processing described
in \secref{sec:commandline} using the detection mechanism reads:
%
\begin{center}
|... -jobname "|\textit{target}|" "|[\textit{flags}]%
[|\def\jobname{|\textit{dest}|}|]|\input{|\textit{main}|}"|
\end{center}

%%%%%%%%%%%%%%%%%%%%%%%%%%%%%%%%%%%%%%%%%%%%%%%%%%%%%%%%%%%%%%%%%%%%%%%%%%%%%%%%
\subsection{Manual Code}
\label{sec:manual}

In case one cannot be certain whether the definitions file |childdoc.def|
is installed on the target \TeX{} distribution
and one prefers not to ship it,
it is conceivable to paste a few relevant commands into the sources.

To that end, drop all statements |\input{childdoc.def}|
and perform the replacements as outlined below.
Instead of |\childdocmain{|\textit{main}|}| add the following code
to the top of the main file:
%
\begin{center}
\begin{tabular}{l}
|\||ifdefined\childdocname\endinput\||fi\newif\ifchilddoc|\\
|\edef\childdocname{\scantokens\expandafter{\jobname\noexpand}}|\\
|\def\childdocmain{|\textit{main}|}\||ifx\childdocmain\childdocname\||else|\\
|\childdoctrue\includeonly{\childdocname}\let\jobname\childdocmain\||fi|\\
\end{tabular}
\end{center}
%
Instead of |\childdocof{|\textit{main}|}| just include the main file
at the top of each child file:
%
\begin{center}
|\input{|\textit{main}|}|
\end{center}
%
A simple redirection |\childdocforward{|\textit{dest}|}| is achieved by:
%
\begin{center}
|\def\jobname{|\textit{dest}|}\input{\jobname}|
\end{center}
%
The redirection with prefix
|\childdocforwardprefix[|\textit{prefix}|]{|\textit{dest}|}|
is accomplished by:
%
\begin{center}
\begin{tabular}{l}
|{\edef\jobname{\scantokens\expandafter{\jobname\noexpand}}|\\
|\def\redirectjob |\textit{prefix}|#1~~~{\gdef\jobname{|\textit{dest}|#1}}|\\
|\expandafter\redirectjob\jobname~~~}\input{\jobname}|
\end{tabular}
\end{center}

In an alternative approach,
child documents can be compiled by a specific command line
without additional code or specific definitions:
%
\begin{center}
|... -jobname "|\textit{target}|" "|[\textit{flags}]%
|\includeonly{|\textit{dest}|}\input{|\textit{main}|}"|
\end{center}
%

%%%%%%%%%%%%%%%%%%%%%%%%%%%%%%%%%%%%%%%%%%%%%%%%%%%%%%%%%%%%%%%%%%%%%%%%%%%%%%%%
%%%%%%%%%%%%%%%%%%%%%%%%%%%%%%%%%%%%%%%%%%%%%%%%%%%%%%%%%%%%%%%%%%%%%%%%%%%%%%%%
\section{Information}

%%%%%%%%%%%%%%%%%%%%%%%%%%%%%%%%%%%%%%%%%%%%%%%%%%%%%%%%%%%%%%%%%%%%%%%%%%%%%%%%
\subsection{Copyright}

Copyright \copyright{} 2017--2018 Niklas Beisert

This work may be distributed and/or modified under the
conditions of the \LaTeX{} Project Public License, either version 1.3
of this license or (at your option) any later version.
The latest version of this license is in
  \url{http://www.latex-project.org/lppl.txt}
and version 1.3 or later is part of all distributions of \LaTeX{}
version 2005/12/01 or later.

This work has the LPPL maintenance status `maintained'.

The Current Maintainer of this work is Niklas Beisert.

This work consists of the files |README.txt|, |childdoc.ins| and |childdoc.dtx|
as well as the derived files |childdoc.def|, |cdocsamp.tex|
with |cdocsch1.tex|, |cdocsch2.tex|, |cdocspt3.tex|, |cdocspt4.tex|,
|cdocsdrf.tex|, |cdocsfn1.tex|, |cdocsfn2.tex|
as well as |childdoc.pdf|.

%%%%%%%%%%%%%%%%%%%%%%%%%%%%%%%%%%%%%%%%%%%%%%%%%%%%%%%%%%%%%%%%%%%%%%%%%%%%%%%%
\subsection{Files and Installation}

The package consists of the files:
%
\begin{center}
\begin{tabular}{ll}
    |README.txt|   & readme file \\
    |childdoc.ins| & installation file \\
    |childdoc.dtx| & source file \\
    |childdoc.def| & definition file \\
    |cdocsamp.tex| & sample main file \\
    |cdocsch1.tex| & sample include file \\
    |cdocsch2.tex| & sample include file \\
    |cdocspt3.tex| & sample part file \\
    |cdocspt4.tex| & sample part file \\
    |cdocsdrf.tex| & sample redirection file \\
    |cdocsfn1.tex| & sample redirection file \\
    |cdocsfn2.tex| & sample redirection file \\
    |childdoc.pdf| & manual
\end{tabular}
\end{center}
%
The distribution consists of the files
|README.txt|, |childdoc.ins| and |childdoc.dtx|.
%
\begin{itemize}
\item
Run (pdf)\LaTeX{} on |childdoc.dtx|
to compile the manual |childdoc.pdf| (this file).
\item
Run \LaTeX{} on |childdoc.ins| to create the definitions file |childdoc.def|
and the sample |cdocsamp.tex| with include files
|cdocsch1.tex|, |cdocsch2.tex|, |cdocspt3.tex|, |cdocspt4.tex|,
|cdocsdrf.tex|, |cdocsfn1.tex|, |cdocsfn2.tex|.
Then copy the file |childdoc.def| to an appropriate directory of your \LaTeX{}
distribution, e.g.\ \textit{texmf-root}|/tex/latex/childdoc|.
\end{itemize}

%%%%%%%%%%%%%%%%%%%%%%%%%%%%%%%%%%%%%%%%%%%%%%%%%%%%%%%%%%%%%%%%%%%%%%%%%%%%%%%%
\subsection{Related CTAN Packages}

There are several other packages which offer a similar functionality:
%
\begin{itemize}
\item
The packages
\href{http://ctan.org/pkg/docmute}{\textsf{docmute}},
\href{http://ctan.org/pkg/includex}{\textsf{includex}} and
\href{http://ctan.org/pkg/standalone}{\textsf{standalone}}
provide commands to include only the document body of
a child file thus allowing both files to be compiled individually.
\item
The packages \href{http://ctan.org/pkg/subdocs}{\textsf{subdocs}}
and \href{http://ctan.org/pkg/subfiles}{\textsf{subfiles}}
provide structures in which the main and child documents can be
encapsulated and allowing them to be compiled individually.
The inclusion mechanism is different from the conventional |\include|.
\item
The package \href{http://ctan.org/pkg/combine}{\textsf{combine}}
is an elaborate solution to combine several documents into one.
\end{itemize}
%
See also the CTAN topic \href{http://ctan.org/topic/subdocs}{\textsf{subdocs}}
for further related packages.
The present package differs from the above solutions in that
a document structure constructed with the conventional |\include| mechanism
just needs two extra commands at the top of every file
such that all constituent files can be compiled individually.

%%%%%%%%%%%%%%%%%%%%%%%%%%%%%%%%%%%%%%%%%%%%%%%%%%%%%%%%%%%%%%%%%%%%%%%%%%%%%%%%
%\subsection{Feature Suggestions}
%
%The following is a list of features which may be useful for future
%versions of this package:
%%
%\begin{itemize}
%\item
%\ldots
%\end{itemize}

%%%%%%%%%%%%%%%%%%%%%%%%%%%%%%%%%%%%%%%%%%%%%%%%%%%%%%%%%%%%%%%%%%%%%%%%%%%%%%%%
\subsection{Revision History}

%%%%%%%%%%%%%%%%%%%%%%%%%%%%%%%%%%%%%%%%
\paragraph{v2.0:} 2018/12/30

\begin{itemize}
\item
immediate forward processing
\item
added |\childdocby| mechanism
\item
manual restructured
\end{itemize}

%%%%%%%%%%%%%%%%%%%%%%%%%%%%%%%%%%%%%%%%
\paragraph{v1.6:} 2018/01/17

\begin{itemize}
\item
application for development of include files
\item
corrections to manual
\end{itemize}

%%%%%%%%%%%%%%%%%%%%%%%%%%%%%%%%%%%%%%%%
\paragraph{v1.5:} 2017/05/21

\begin{itemize}
\item
more complete structuring introduced
\item
|\childdocof| introduced
\item
|\childdoc| renamed to |\childdocmain|
\item
|\childredirect| renamed to |\childdocforward| and |\childdocforwardprefix|
and functionality expanded
\end{itemize}

%%%%%%%%%%%%%%%%%%%%%%%%%%%%%%%%%%%%%%%%
\paragraph{v1.0:} 2017/04/27

\begin{itemize}
\item
manual and install package
\item
first version published on CTAN
\end{itemize}

%%%%%%%%%%%%%%%%%%%%%%%%%%%%%%%%%%%%%%%%
\paragraph{v0.6:} 2017/04/26

\begin{itemize}
\item
redirection mechanism added
\end{itemize}

%%%%%%%%%%%%%%%%%%%%%%%%%%%%%%%%%%%%%%%%
\paragraph{v0.5:} 2017/04/26

\begin{itemize}
\item
functionality in definition file
\end{itemize}


%%%%%%%%%%%%%%%%%%%%%%%%%%%%%%%%%%%%%%%%%%%%%%%%%%%%%%%%%%%%%%%%%%%%%%%%%%%%%%%%
%%%%%%%%%%%%%%%%%%%%%%%%%%%%%%%%%%%%%%%%%%%%%%%%%%%%%%%%%%%%%%%%%%%%%%%%%%%%%%%%
%%%%%%%%%%%%%%%%%%%%%%%%%%%%%%%%%%%%%%%%%%%%%%%%%%%%%%%%%%%%%%%%%%%%%%%%%%%%%%%%
\appendix

\settowidth\MacroIndent{\rmfamily\scriptsize 000\ }

 \DocInput{childdoc.dtx}

\end{document}
%</driver>
% \fi
%
% %%%%%%%%%%%%%%%%%%%%%%%%%%%%%%%%%%%%%%%%%%%%%%%%%%%%%%%%%%%%%%%%%%%%%%%%%%%%%%
% %%%%%%%%%%%%%%%%%%%%%%%%%%%%%%%%%%%%%%%%%%%%%%%%%%%%%%%%%%%%%%%%%%%%%%%%%%%%%%
% \section{Sample}
%\iffalse
%<*samplemain>
%\fi
%
% The following presents a sample document
% with two chapters, two parts, a title page,
% a compile flag as well as three forwarding files to set the flag.
% It consists of eight |.tex| files:
% \begin{center}
% \begin{tabular}{ll}
% |cdocsamp.tex|&main file\\
% |cdocsch1.tex|&include file for chapter 1\\
% |cdocsch2.tex|&include file for chapter 2\\
% |cdocspt3.tex|&include file for part 3\\
% |cdocspt4.tex|&include file for part 4\\
% |cdocsdrf.tex|&forwarding file for main file in draft mode\\
% |cdocsfi1.tex|&forwarding file for final version of chapter 1\\
% |cdocsfi2.tex|&forwarding file for final version of chapter 2\\
% \end{tabular}
% \end{center}
% Each of the eight files can be compiled directly by the \LaTeX{} compiler.
%
% %%%%%%%%%%%%%%%%%%%%%%%%%%%%%%%%%%%%%%
% \paragraph{Main File.}
%
% The main file is called |cdocsamp.tex|.
%
% Load the \textsf{childdoc} definitions and
% declare the filename for the main document:
%    \begin{macrocode}
\input{childdoc.def}
\childdocmain{}
%    \end{macrocode}

% Optional override for |\version| flag:
%    \begin{macrocode}
%%\ifchilddoc\else\providecommand{\version}{draft}\fi
%    \end{macrocode}

% Define the default values for the |\version| flag
% (|final| for the main file and |draft| for childs):
%    \begin{macrocode}
\ifchilddoc
\providecommand{\version}{draft}
\else
\providecommand{\version}{final}
\fi
%    \end{macrocode}

% Load the standard document class:
%    \begin{macrocode}
\documentclass[12pt]{article}
%    \end{macrocode}

% Start the document body:
%    \begin{macrocode}
\begin{document}
%    \end{macrocode}

% Declare a title page.
% Print title, part of document being processed and version flag:
%    \begin{macrocode}
\addtocounter{page}{-1}
\begin{center}
{\LARGE\bfseries{}childdoc example\par}
\vspace{1cm}
\ifchilddoc
\ifchilddocmanual part\else chapter\fi:
`\childdocname' of `\childdocjob'\par
\else
main document: `\childdocjob'\par
\fi
version: \version\par
\end{center}
\newpage
%    \end{macrocode}

% Manually include selected file,
% otherwise process as usual:
%    \begin{macrocode}
\ifchilddocmanual
\section*{part `\childdocname'}
\input{\childdocname}
\else
%    \end{macrocode}

% Include the two chapters:
%    \begin{macrocode}
\include{cdocsch1}
\include{cdocsch2}
%    \end{macrocode}

% Include the two parts unless only chapters should be displayed:
%    \begin{macrocode}
\ifchilddoc\else
\section{part three}
\input{cdocspt3}
\section{part four}
\input{cdocspt4}
\fi
%    \end{macrocode}

% Process as usual until here:
%    \begin{macrocode}
\fi
%    \end{macrocode}

% End of document body:
%    \begin{macrocode}
\end{document}
%    \end{macrocode}
%\iffalse
%</samplemain>
%\fi
%
% %%%%%%%%%%%%%%%%%%%%%%%%%%%%%%%%%%%%%%
% \paragraph{Chapter Include Files.}
%
% The include files are called |cdocsch1.tex| and |cdocsch2.tex|.
%
%\iffalse
%<*samplechap1|samplechap2>
%\fi

% Optional override for |\version| flag:
%    \begin{macrocode}
%%\providecommand{\version}{final}
%    \end{macrocode}

% Include the main document:
%    \begin{macrocode}
\input{childdoc.def}
\childdocof{cdocsamp}
%    \end{macrocode}

%\iffalse
%</samplechap1|samplechap2>
%\fi
%
%\iffalse
%<*samplechap1>
%\fi
% Some text for chapter 1:
%    \begin{macrocode}
\section{one}
some text in chapter one
%    \end{macrocode}

%\iffalse
%</samplechap1>
%\fi
% Some text for chapter 2:
%\iffalse
%<*samplechap2>
%\fi
%    \begin{macrocode}
\section{two}
more text in chapter two
%    \end{macrocode}

%\iffalse
%</samplechap2>
%\fi
%
% %%%%%%%%%%%%%%%%%%%%%%%%%%%%%%%%%%%%%%
% \paragraph{Part Include Files.}
%
% The include files are called |cdocspt3.tex| and |cdocspt4.tex|.
%
%\iffalse
%<*samplepart3|samplepart4>
%\fi

% Optional override for |\version| flag:
%    \begin{macrocode}
%%\providecommand{\version}{final}
%    \end{macrocode}

% Include the main document:
%    \begin{macrocode}
\input{childdoc.def}
\childdocby{cdocsamp}
%    \end{macrocode}

%\iffalse
%</samplepart3|samplepart4>
%\fi
%
%\iffalse
%<*samplepart3>
%\fi
% Some text for part 3:
%    \begin{macrocode}
some text in part three
%    \end{macrocode}

%\iffalse
%</samplepart3>
%\fi
% Some text for part 4:
%\iffalse
%<*samplepart4>
%\fi
%    \begin{macrocode}
more text in part four
%    \end{macrocode}

%\iffalse
%</samplepart4>
%\fi
%
% %%%%%%%%%%%%%%%%%%%%%%%%%%%%%%%%%%%%%%
% \paragraph{Forwarding for a Complete Draft.}
%
% The following forwarding file |cdocsdrf.tex|
% compiles the main document in draft mode:
%\iffalse
%<*sampledraft>
%\fi
%    \begin{macrocode}
\def\version{draft}
\input{childdoc.def}
\childdocforward{cdocsamp}
%    \end{macrocode}

%\iffalse
%</sampledraft>
%\fi
%
% %%%%%%%%%%%%%%%%%%%%%%%%%%%%%%%%%%%%%%
% \paragraph{Forwarding for Final Version of the Chapters.}
%
% The following forwarding files |cdocsfn1.tex| and |cdocsfn2.tex|
% (with identical content)
% compile the final versions of the child documents
% |cdocsch1.tex| and |cdocsch2.tex|, respectively:
%\iffalse
%<*samplefinal>
%\fi
%    \begin{macrocode}
\def\version{final}
\input{childdoc.def}
\childdocforwardprefix[cdocsamp]{cdocsfn}{cdocsch}
%    \end{macrocode}

%\iffalse
%</samplefinal>
%\fi
%
% %%%%%%%%%%%%%%%%%%%%%%%%%%%%%%%%%%%%%%
% \paragraph{Command Line Processing.}
%
% The following three command lines generate the output files
% |cdocscld|, |cdocscl1| and |cdocscl2|
% which should be identical to
% |cdocsdrf|, |cdocsch1| and |cdocsfn2|, respectively:
% \begin{center}
% \begin{tabular}{l}
% |latex -jobname cdocscld \|\\
% |  "\def\version{draft}\input{childdoc.def}\childdocforward{cdocsamp}"|\\
% |latex -jobname cdocscl1 \|\\
% |  "\input{childdoc.def}\childdocforward[cdocsamp]{cdocsch1}"|\\
% |latex -jobname cdocscl2 \|\\
% |  "\def\version{final}\input{childdoc.def}\childdocforward{cdocsch2}"|
% \end{tabular}
% \end{center}
% Note that the trailing backslash on each first line
% merely continues the input to the second line
% (for convenient cut ant paste).
% Furthermore, the command |latex| can be replaced by any
% of its alternative versions such as |pdflatex|.
%
% %%%%%%%%%%%%%%%%%%%%%%%%%%%%%%%%%%%%%%%%%%%%%%%%%%%%%%%%%%%%%%%%%%%%%%%%%%%%%%
% %%%%%%%%%%%%%%%%%%%%%%%%%%%%%%%%%%%%%%%%%%%%%%%%%%%%%%%%%%%%%%%%%%%%%%%%%%%%%%
% \section{Implementation}
%\iffalse
%<*package>
%\fi
%
% This section describes the definitions file |childdoc.def|.

% The definitions cannot be loaded using |\usepackage| or |\RequirePackage|
% which has a mechanism to prevent loading a style file more than once.
% When loading the definitions by means of |\input|
% multiple instances have to be prevented manually:
%\iffalse
%This code needs to be before the `\ProvidesFile' directive
%which is defined at the beginning of this file.
%Therefore it is also placed there and commented out here.
%</package>
%<*discard>
%\fi
%    \begin{macrocode}
\ifdefined\childdocmain\endinput\fi
%    \end{macrocode}
%\iffalse
%</discard>
%<*package>
%\fi
%
% \macro{\ifchilddoc}
% \macro{\ifchilddocmanual}
% The conditional |\ifchilddoc| tells whether a
% child (true) or main (false) document is being compiled.
% The conditional |\ifchilddocmanual| tells whether
% the |\includeonly| mechanism is used (false) or
% the selection of child files must be performed manually (true).
% The definitions initialise to false:
%    \begin{macrocode}
\newif\ifchilddoc
\newif\ifchilddocmanual
%    \end{macrocode}

% \macro{\childdocname}
% \macro{\childdocjob}
% The macro |\childdocname| stores the name of the main document
% to be compiled. The macro |\childdocjob| stores the name of
% the document on which the \LaTeX{} compiler was originally invoked.
% The content of |\jobname| cannot be compared
% to filenames specified in the source due to different catcodes.
% The following code rescans |\jobname|, stores the result
% in |\childdocname| and saves a copy in |\childdocjob|:
%    \begin{macrocode}
\edef\childdocname{\scantokens\expandafter{\jobname\noexpand}}
\let\childdocjob\childdocname
%    \end{macrocode}

% \macro{\childdocdisable}
% The macro |\childdocdisable| prevents the main file
% from being processed more than once.
% At this stage, the main document command |\childdocmain|
% is assumed to be called once again where it should do nothing.
% Any subsequent call to it should prevent
% a secondary processing of the main document
% It overwrites the forwarding commands
% |\childdocof| and |\childdocforward|
% with empty macros to prevent further inclusions of the main document:
%    \begin{macrocode}
\newcommand{\childdocdisable}
{
  \renewcommand{\childdocmain}[1]{\renewcommand{\childdocmain}[1]{\endinput}}
  \renewcommand{\childdocof}[1]{}
  \renewcommand{\childdocby}[2][]{}
  \renewcommand{\childdocforward}[2][]{}
  \renewcommand{\childdocdisable}{}
}
%    \end{macrocode}

% \macro{\childdocmain}
% The macro |\childdocmain| is to be called at the top of the main file
% with nothing or the main filename (without extension) as argument.
% First, it breaks loops.
% If the argument is not empty and does not match |\childdocname|
% (which is set by the first inclusion of |childdoc.def|),
% |\ifchilddoc| is set to true, |\includeonly| is applied to the child file
% and |\jobname| is set to the main file
% (for proper handling of |.aux| files):
%    \begin{macrocode}
\newcommand{\childdocmain}[1]
{
  \childdocdisable\childdocmain{}
  \if?#1?\else
    \begingroup
      \def\childdoctmp{#1}
      \ifx\childdoctmp\childdocname
        \def\childdoctmp{}
      \else
        \def\childdoctmp
        {
          \childdoctrue
          \includeonly{\childdocname}
          \def\childdocjob{#1}
          \def\jobname{#1}
        }
      \fi
      \expandafter
    \endgroup
    \childdoctmp
  \fi
}
%    \end{macrocode}

% \macro{\childdocof}
% The command |\childdocof| redirects
% compilation to the main file |#1|.
%    \begin{macrocode}
\newcommand{\childdocof}[1]
{
  \childdocdisable
  \childdoctrue
  \includeonly{\childdocname}
  \def\jobname{#1}
  \def\childdocjob{#1}
  \input{#1}
}
%    \end{macrocode}

% \macro{\childdocby}
% The command |\childdocby| ....
%    \begin{macrocode}
\newcommand{\childdocby}[2][]
{
  \childdocdisable
  \childdoctrue
  \childdocmanualtrue
  \if?#1?\else
    \def\jobname{#2}
  \fi
  \def\childdocjob{#2}
  \input{#2}
  \endinput
}
%    \end{macrocode}

% \macro{\childdocforward}
% The command |\childdocforward| redirects
% compilation to the main file or
% (if the optional argument is given) a child file.
% Parameters are set as if the main file
% or a child file starting with |\childdocof| was compiled.
% Then compilation is handed over to the main file:
%    \begin{macrocode}
\newcommand{\childdocforward}[2][]
{
  \begingroup
    \if?#1?
      \def\childdoctmp
      {
        \def\childdocname{#2}
        \def\childdocjob{#2}
        \def\jobname{#2}
        \input{#2}
        \endinput
      }
    \else
      \def\childdoctmp
      {
        \childdocdisable
        \def\childdocname{#2}
        \childdoctrue
        \includeonly{#2}
        \def\childdocjob{#1}
        \def\jobname{#1}
        \input{#1}
        \endinput
      }
    \fi
    \expandafter
  \endgroup
  \childdoctmp
}
%    \end{macrocode}

% \macro{\childdocforwardprefix}
% The command |\childdocforwardprefix| redirects
% compilation to the main or a child file by means of a pattern.
% The prefix |#1| in the current filename is replaced by |#2|
% and the suffix of the current filename is kept
% (it is assumed that the filename does not contain the substring `|~~~|'
% which is used as a delimiter).
% Compilation is handed over to the new file by |\childdocforward|:
%    \begin{macrocode}
\newcommand{\childdocforwardprefix}[3][]
{
  \begingroup
    \def\childdocextract #2##1~~~{\def\childdoctmp{\childdocforward[#1]{#3##1}}}
    \expandafter\childdocextract\childdocname~~~
    \expandafter
  \endgroup
  \childdoctmp
}
%    \end{macrocode}

% \macro{\childdoc}
% The deprecated macro |\childdoc| is a legacy version of |\childdocmain|:
%    \begin{macrocode}
\newcommand{\childdoc}{\childdocmain}
%    \end{macrocode}

% \macro{\childdocredirect}
% The deprecated macro |\childdocredirect| is a legacy version
% of |\childdocforward| and |\childdocforwardprefix|:
%    \begin{macrocode}
\newcommand{\childdocredirect}[2][]
{
  \begingroup
    \if?#1?
      \def\childdoctmp{\childdocforward{#2}}
    \else
      \def\childdoctmp{\childdocforwardprefix{#1}{#2}}
    \fi
    \expandafter
  \endgroup
  \childdoctmp
}
%    \end{macrocode}

%\iffalse
%</package>
%\fi
%
\endinput
|\\
|\childdocforward{|\textit{main}|}|\\
\end{tabular}
\end{center}
%
or alternatively with:
%
\begin{center}
\begin{tabular}{l}
|% \iffalse
%
% childdoc.dtx Copyright (C) 2017-2018 Niklas Beisert
%
% This work may be distributed and/or modified under the
% conditions of the LaTeX Project Public License, either version 1.3
% of this license or (at your option) any later version.
% The latest version of this license is in
%   http://www.latex-project.org/lppl.txt
% and version 1.3 or later is part of all distributions of LaTeX
% version 2005/12/01 or later.
%
% This work has the LPPL maintenance status `maintained'.
%
% The Current Maintainer of this work is Niklas Beisert.
%
% This work consists of the files childdoc.dtx and childdoc.ins
% and the derived files childdoc.def and cdocsamp.tex with
% cdocsch1.tex, cdocsch2.tex, cdocsdrf.tex, cdocsfn1.tex, cdocsfn2.tex.
%
%<package>\ifdefined\childdocmain\endinput\fi
%<package>\ProvidesFile{childdoc.def}[2018/12/30 v2.0 child document driver]
%<samplemain>\ProvidesFile{cdocsamp.tex}[2018/12/30 v2.0 sample for childdoc]
%<*driver>
%\ProvidesFile{childdoc.drv}[2018/12/30 v2.0 childdoc reference manual file]
\PassOptionsToClass{10pt,a4paper}{article}
\documentclass{ltxdoc}

\usepackage[margin=35mm]{geometry}
\usepackage{hyperref}
\usepackage{hyperxmp}
\usepackage[usenames]{color}

\hypersetup{colorlinks=true}
\hypersetup{pdfstartview=FitH}
\hypersetup{pdfpagemode=UseNone}
\hypersetup{pdfsource={}}
\hypersetup{pdflang={en-UK}}
\hypersetup{pdfcopyright={Copyright 2017-2018 Niklas Beisert.
  This work may be distributed and/or modified under the
  conditions of the LaTeX Project Public License, either version 1.3
  of this license or (at your option) any later version.}}
\hypersetup{pdflicenseurl={http://www.latex-project.org/lppl.txt}}
\hypersetup{pdfcontactaddress={ETH Zurich, ITP, HIT K,
  Wolfgang-Pauli-Strasse 27}}
\hypersetup{pdfcontactpostcode={8093}}
\hypersetup{pdfcontactcity={Zurich}}
\hypersetup{pdfcontactcountry={Switzerland}}
\hypersetup{pdfcontactemail={nbeisert@itp.phys.ethz.ch}}
\hypersetup{pdfcontacturl={http://people.phys.ethz.ch/\xmptilde nbeisert/}}

\newcommand{\secref}[1]{\hyperref[#1]{section \ref*{#1}}}

\parskip1ex
\parindent0pt
\let\olditemize\itemize
\def\itemize{\olditemize\parskip0pt}

\begin{document}

\title{The \textsf{childdoc} Package}
\hypersetup{pdftitle={The childdoc Package}}
\author{Niklas Beisert\\[2ex]
  Institut f\"ur Theoretische Physik\\
  Eidgen\"ossische Technische Hochschule Z\"urich\\
  Wolfgang-Pauli-Strasse 27, 8093 Z\"urich, Switzerland\\[1ex]
  \href{mailto:nbeisert@itp.phys.ethz.ch}
  {\texttt{nbeisert@itp.phys.ethz.ch}}}
\hypersetup{pdfauthor={Niklas Beisert}}
\hypersetup{pdfsubject={Manual for the LaTeX2e Package childdoc}}
\date{30 December 2018, \textsf{v2.0}}
\maketitle

\begin{abstract}\noindent
\textsf{childdoc} is a \LaTeXe{} package
that enables the direct compilation
of document sections included by |\include|
to individual files.
\end{abstract}

\begingroup
\parskip0ex
\tableofcontents
\endgroup

%%%%%%%%%%%%%%%%%%%%%%%%%%%%%%%%%%%%%%%%%%%%%%%%%%%%%%%%%%%%%%%%%%%%%%%%%%%%%%%%
%%%%%%%%%%%%%%%%%%%%%%%%%%%%%%%%%%%%%%%%%%%%%%%%%%%%%%%%%%%%%%%%%%%%%%%%%%%%%%%%
\section{Introduction}

\LaTeX{} provides a mechanism to structure a large document (such as a book)
into a main file and several child files (containing the chapters)
using the |\include| command.
This mechanism is beneficial for documents
which span hundreds of pages in order to
make the source file(s) more manageable.
Moreover, compilation can be restricted to
selected child files by means of the |\includeonly| command.
The latter feature can be used to reduce the compilation time while editing
(this was significantly more useful in the earlier days of \LaTeX{})
or to generate a smaller document which is easier to navigate.
Another application of |\includeonly| is to generate
documents consisting of selected parts of the complete document.

However, there are a few drawbacks of the plain |\include| mechanism:
\begin{itemize}
\item
The child files cannot be compiled on their own,
they can only be compiled via the main file.
A naive editing environment
(such as a text editor with an option
to have the current file processed by \LaTeX)
may require one to switch to the main file before compiling;
attempting to compile the child file produces errors.
\item
The main file must be modified (each time)
to adjust the |\includeonly| command
to the present needs. This easily leaves the main file in a messy state.
\item
The generated document will always carry the filename
of the main document. This is inconvenient if
several child files are to be compiled and
to be kept for distribution.
\end{itemize}

The present package provides a simple interface
to make child files individually compilable by \LaTeX{}.
Compiling a child file then has the same effect as compiling
the main file with an |\includeonly| command
to select the appropriate child.
Moreover the generated document will carry the name of the child
rather than the main file.
This resolves all three above issues.

This feature is meant to make the editing of books,
thesis documents and lecture notes somewhat more convenient.
However, the package can also be used efficiently for
composing a series of documents (such as exercise sheets)
which are typically distributed individually.
It then assists the author in generating the individual documents
(potentially in different versions)
as well as a document containing the collected series.
Another application is in developing style files
or other kinds of included material
where compilation of the style file could redirect
to a sample or test file.

%%%%%%%%%%%%%%%%%%%%%%%%%%%%%%%%%%%%%%%%%%%%%%%%%%%%%%%%%%%%%%%%%%%%%%%%%%%%%%%%
%%%%%%%%%%%%%%%%%%%%%%%%%%%%%%%%%%%%%%%%%%%%%%%%%%%%%%%%%%%%%%%%%%%%%%%%%%%%%%%%
\section{Usage}

First of all, the package \textsf{childdoc} is \emph{not} a standard
\LaTeXe{} |.sty| style file! Therefore it needs to be invoked in
a non-standard way.

%%%%%%%%%%%%%%%%%%%%%%%%%%%%%%%%%%%%%%%%%%%%%%%%%%%%%%%%%%%%%%%%%%%%%%%%%%%%%%%%
\subsection{Included Files}
\label{sec:include}

%%%%%%%%%%%%%%%%%%%%%%%%%%%%%%%%%%%%%%%%
\DescribeMacro{\childdocmain}
To use the package, add the commands
\begin{center}
\begin{tabular}{l}
|\input{childdoc.def}|\\
|\childdocmain{}|\\
\end{tabular}
\end{center}
at the very top of the main \LaTeX{} file,
in particular \emph{before} the |\documentclass| statement!
The argument of |\childdocmain| should be left empty
(but it must be present).

%%%%%%%%%%%%%%%%%%%%%%%%%%%%%%%%%%%%%%%%
\DescribeMacro{\childdocof}
Furthermore, add the commands
\begin{center}
\begin{tabular}{l}
|\input{childdoc.def}|\\
|\childdocof{|\textit{main}|}|\\
\end{tabular}
\end{center}
at the top of every child file \textit{child}
which is included by |\include{|\textit{child}|}|
from within the main file
(or at least for those files to be compiled individually).
The argument \textit{main} must be the filename of the main file.

There are a couple of
considerations in setting up the main and child documents:

%%%%%%%%%%%%%%%%%%%%%%%%%%%%%%%%%%%%%%%%
\paragraph{Restrictions.}

Please note the following restrictions:
\begin{itemize}
\item
|\childdocmain| must be called with one argument \textit{main}
to ensure compatibility with earlier version of the package.
It must either be empty (|\childdocmain{}|)
or precisely match the filename of the main file in which it is specified.
See \secref{sec:detection} for further information.
\item
The filename \textit{main} must be specified without the |.tex| extension.
\item
The filename \textit{main} is case sensitive
(even in case-insensitive file systems)
due to internal string comparison.
\item
The argument \textit{main} should be fully expanded, it cannot be a macro.
\item
Subdirectories and special characters should be avoided in filenames.
\item
The command |\childdocmain{|\textit{main}|}| must be followed by a whitespace.
It should not be followed immediately by another command
or by a comment mark `|%|'.
This is because the \TeX{} parser reads the token immediately following
the argument of |\childdocmain| and puts it
at the beginning of every child section;
however, a white\-space is ignored.
\end{itemize}

%%%%%%%%%%%%%%%%%%%%%%%%%%%%%%%%%%%%%%%%
\paragraph{Content of Main File.}

It is advisable to place all content in the child files included by |\include|.
Any output contained in the main file will appear in all child documents
unless suppressed manually;
it cannot be suppressed automatically by the |\includeonly| directive
and thus should normally be avoided.
A method to include some content in the main file
by means of conditional processing is described in \secref{sec:conditional}.

%%%%%%%%%%%%%%%%%%%%%%%%%%%%%%%%%%%%%%%%
\paragraph{Page Numbering.}

When only a part of the document is compiled,
the appropriate numbering of pages
(as well as other status parameters)
is determined from the |.aux| files.
The latter contain information from previous passes.
However this information needs to propagate through
all intermediate child documents.
Therefore the page numbering in child documents may well
be inconsistent until the complete document is compiled at least once.

A useful (if unconventional) way to always ensure a consistent
page numbering is to restart the numbering in each child document
and denote the pages by `\textit{child}|.|\textit{page}'
where \textit{child} represents the chapter/section number of the child file.
This can be achieved by the command
|\numberwithin{page}{|\textit{child}|}|
of the \textsf{amsmath} package
where \textit{child} can be |chapter| or |section|
depending on the chosen structuring.
Alternatively, one can modify the macro |\thepage| appropriately
and reset the counter |page| at the start of each child file.

%%%%%%%%%%%%%%%%%%%%%%%%%%%%%%%%%%%%%%%%%%%%%%%%%%%%%%%%%%%%%%%%%%%%%%%%%%%%%%%%
\subsection{Conditional Processing}
\label{sec:conditional}

The package provides a mechanism to compile different versions
of a document. To customise the versions further some conditional processing
can come in handy to distinguish which version is being compiled.
The package provides two macros to describe the compilation context:

%%%%%%%%%%%%%%%%%%%%%%%%%%%%%%%%%%%%%%%%
\DescribeMacro{\ifchilddoc}
The conditional |\ifchilddoc| distinguishes between the compilation of
child documents and the main document:
%
\begin{center}
|\ifchilddoc |\textit{child-code}| |[|\||else |\textit{main-code}]| \||fi|
\end{center}

%%%%%%%%%%%%%%%%%%%%%%%%%%%%%%%%%%%%%%%%
\DescribeMacro{\childdocname}
\DescribeMacro{\childdocjob}
The macro |\childdocname| contains the filename (without extension)
of the main or child file being processed.
Note that |\childdocjob| will always contain the name of the main file.

%%%%%%%%%%%%%%%%%%%%%%%%%%%%%%%%%%%%%%%%
\paragraph{Title Page.}

Conditional processing can be used to include a title or banner page
in the main document when proper precautions are taken.
Importantly, the code in the main file should ensure that the page counter
(as well as other status parameters which are stored in the |.aux| files)
takes the same value after the conditional processing.
Otherwise the page numbers may take divergent values
depending on which part is compiled.

For example, a title page could be declared by:
%
\begin{center}
\begin{tabular}{l}
|\ifchilddoc\||else|\\
|\addtocounter{page}{-1}|\\
\textit{code for title page}\\
|\newpage|\\
|\||fi|
\end{tabular}
\end{center}
%
A banner page for the child documents can be generated by:
%
\begin{center}
\begin{tabular}{l}
|\ifchilddoc|\\
|\addtocounter{page}{-1}|\\
\textit{code for banner page}\\
|\newpage|\\
|\||fi|
\end{tabular}
\end{center}
%
Here one could write a message such as:
\begin{center}
|This is the part \childdocname{} of \childdocjob{}.|
\end{center}

%%%%%%%%%%%%%%%%%%%%%%%%%%%%%%%%%%%%%%%%%%%%%%%%%%%%%%%%%%%%%%%%%%%%%%%%%%%%%%%%
\subsection{Flags}
\label{sec:flags}

The package makes it easy to generate different versions
of the main or child documents.
To this end compilation flags can be defined
and assigned different default values.
They will be particularly useful in conjunction
with the forwarding mechanism described in \secref{sec:forward}.

For example, it may be useful to have a flag |\version|
which can be set to |draft| or |final|.
The document source will contain some conditional code
depending on the value of |\version|.
Suppose further, the flag should default to |final| for the main file
and to |draft| for child files
which is a natural assignment for editing the document.
This is achieved by placing the following code
in the preamble of the main document
(below the |\childdocmain| directive):
%
\begin{center}
\begin{tabular}{l}
|\ifchilddoc|\\
|\providecommand{\version}{draft}|\\
|\||else|\\
|\providecommand{\version}{final}|\\
|\||fi|
\end{tabular}
\end{center}
%
The definition by |\providecommand| makes sure
that previous definitions are not overwritten.
Further statements |\providecommand{\version}{...}|
can thus be added before the above code to override it.

For the main file, one might add a line
(between |\childdocmain| and the above block)
%
\begin{center}
|%\ifchilddoc\||else\providecommand{\version}{draft}\||fi|
\end{center}
%
which can be uncommented to produce a draft version.
Likewise one can add a line to the very top of a child file
(above the |\childdocof{|\textit{main}|}| directive)
%
\begin{center}
|%\providecommand{\version}{final}|
\end{center}
%
which can be uncommented to produce the final version of this child document.

%%%%%%%%%%%%%%%%%%%%%%%%%%%%%%%%%%%%%%%%%%%%%%%%%%%%%%%%%%%%%%%%%%%%%%%%%%%%%%%%
\subsection{Forwarding}
\label{sec:forward}

Different versions of the main or child documents
using compilation flags as described in \secref{sec:flags}
can be (permanently) stored in different files
for convenient compilation, viewing and distribution.
To this end, the package defines a command
to pass on compilation to a different file:

%%%%%%%%%%%%%%%%%%%%%%%%%%%%%%%%%%%%%%%%
\DescribeMacro{\childdocforward}
The command |\childdocforward| redirects processing to
another source file:
%
\begin{center}
\begin{tabular}{l}
|\input{childdoc.def}|\\
|\childdocforward[|\textit{main}|]{|\textit{dest}|}|\\
\end{tabular}
\end{center}
%
The argument \textit{dest} is the destination file
(without extension).
It should be the main file or one of the child files.
Note that further \textsf{childdoc} directives
such as |\childdocof| and |\childdocforward|
in the indicated file will be processed in this form.
The optional argument \textit{main}
passes on directly to the main file \textit{main}
while pretending to compile the child \textit{dest}.
This form behaves as if \textit{dest}
issues |\childdocof{|\textit{main}|}| right away,
and no further \textsf{childdoc} directives will be processed.

%%%%%%%%%%%%%%%%%%%%%%%%%%%%%%%%%%%%%%%%
\DescribeMacro{\...prefix}
In the alternative form |\childdocforwardprefix|,
%
\begin{center}
\begin{tabular}{l}
|\input{childdoc.def}|\\
|\childdocforwardprefix[|\textit{main}|]{|\textit{prefix}|}{|\textit{dest}|}|
\end{tabular}
\end{center}
%
the destination file is determined by a pattern
depending on the current file:
To make this work, the current file must be called
`{\textit{prefix}\hspace{0.2em}\textit{suffix}}'
with \textit{prefix} matching precisely the argument.
Processing is then passed on to the file
`{\textit{dest}\hspace{0.2em}\textit{suffix}}'.
Surely, the same effect is achieved by
directly specifying the
argument `{\textit{dest}\hspace{0.2em}\textit{suffix}}'
in the first form.
However, that requires to set up a different file
for each child. With the alternative form of the command
all these files can have exactly the same content
which simplifies setting them up and maintaining them.

For example, the following file |draft.tex|
with a compilation flag |\version| as described in \secref{sec:flags}
compiles the main document as a draft:
%
\begin{center}
\begin{tabular}{l}
|\def\version{draft}|\\
|\input{childdoc.def}|\\
|\childdocforward{|\textit{main}|}|
\end{tabular}
\end{center}
%
Likewise, the following files |final|\textit{nn}|.tex|
compile the final version of the child document
|child|\textit{nn}|.tex|:
%
\begin{center}
\begin{tabular}{l}
|\def\version{final}|\\
|\input{childdoc.def}|\\
|\childdocforwardprefix{final}{child}|
\end{tabular}
\end{center}
%

Note that when several versions of a main file and/or of each child file
are to be generated, it may be convenient to set up a |Makefile| or
shell script to automatise the process.

%%%%%%%%%%%%%%%%%%%%%%%%%%%%%%%%%%%%%%%%%%%%%%%%%%%%%%%%%%%%%%%%%%%%%%%%%%%%%%%%
\subsection{Command Line Processing}
\label{sec:commandline}

The effect of redirection files can also be achieved by invoking
the \LaTeX{} compiler with a more elaborate command line.
Most conveniently this should be done as part
of a shell script or a |Makefile|.

When using \textsf{childdoc} in the main file, the following
command lines effectively perform a redirection
(note that depending on the shell being used,
backslashes may have to be doubled: `|\|' $\to$ `|\\|'):
%
\begin{center}
|... -jobname "|\textit{target}|" |\\|"|[\textit{flags}]%
|\input{childdoc.def}\childdocforward[|\textit{main}|]{|\textit{dest}|}"|
\end{center}
%
Here \textit{target} is the name of the output file,
\textit{main} is the name of the main file
and \textit{dest} is the name of the main or child file to be processed
(all filenames without extensions).
The optional argument \textit{main} can be omitted
if \textit{main} matches \textit{dest}.
Optionally, compilation \textit{flags} can be defined via |\def| commands.
This command line makes the \TeX{} engine believe
it is compiling the file \textit{target}
whose content is specified as the latter parameter.
The provided code then forwards the processing to
\textit{main} or \textit{dest} as described in \secref{sec:forward}.

%%%%%%%%%%%%%%%%%%%%%%%%%%%%%%%%%%%%%%%%%%%%%%%%%%%%%%%%%%%%%%%%%%%%%%%%%%%%%%%%
\subsection{Include by Input}
\label{sec:input}

Including child documents by |\include| has some restrictions by design.
Most notably, the content of a child document always occupies
its own set of pages; pages cannot be shared between child documents.
Usually, this behaviour makes perfect sense
because each child document contain an essential part of the document.
However, in some situations it may be desirable to compose
a document from a collection of parts
without having mandatory page breaks between then.
For this case, the package
provides a mechanism to include parts
by |\input| which can also be processed individually.
However, by construction this mechanism
requires manual handling of the content to be output.

%%%%%%%%%%%%%%%%%%%%%%%%%%%%%%%%%%%%%%%%
\DescribeMacro{\ifchilddocmanual}
The main file should be prepared as usual, see \secref{sec:include}.
However, the document body must make a distinction
between processing of an individual part and of the main document, e.g.:
%
\begin{center}
\begin{tabular}{l}
|\ifchilddocmanual|\\
|\input{\childdocname}|\\
|\||else|\\
\textit{document body with }|\input{|\textit{part}|}|\\
|\||fi|
\end{tabular}
\end{center}
%
The conditional |\ifchilddocmanual| is true whenever
a part to be included by |\input| is being compiled,
and the name of the part is stored in |\childdocname|.

%%%%%%%%%%%%%%%%%%%%%%%%%%%%%%%%%%%%%%%%
\DescribeMacro{\childdocby}
Each part to be included by |\input| should start with:
%
\begin{center}
\begin{tabular}{l}
|\input{childdoc.def}|\\
|\childdocby{|\textit{main}|}|\\
\end{tabular}
\end{center}
%
The directive |\childdocby| is similar to |\childdocof|
described in \secref{sec:include},
but the subsequent selection of content must be done manually.
To that end, both |\ifchilddoc| and |\ifchilddocmanual|
will be true upon processing of a part,
and the name of the part is stored in |\childdocname|.
Note that |\jobname| will be set to the filename of the current part
so that each part receives an individual |.aux| file
that does not interfere with the |.aux| file(s) of the main document.
This behaviour can be altered by the alternative form
|\childdocby[*]{|\textit{main}|}| (with a non-empty optional argument)
which uses the |.aux| file of the main document
by setting |\jobname| to \textit{main}.

%%%%%%%%%%%%%%%%%%%%%%%%%%%%%%%%%%%%%%%%%%%%%%%%%%%%%%%%%%%%%%%%%%%%%%%%%%%%%%%%
\subsection{Driver Development}
\label{sec:driver}

The \textsf{childdoc} mechanism can also be use for the development
of definition files such as \LaTeX{} styles or classes.
This case differs from the above setup with multiple parts
included by |\include| in that no |\includeonly| should be invoked.
This can be achieved by starting the include file
(before |\ProvidesPackage|) with:
%
\begin{center}
\begin{tabular}{l}
|\input{childdoc.def}|\\
|\childdocforward{|\textit{main}|}|\\
\end{tabular}
\end{center}
%
or alternatively with:
%
\begin{center}
\begin{tabular}{l}
|\input{childdoc.def}|\\
|\childdocby{|\textit{main}|}|\\
\end{tabular}
\end{center}
%
Both forms have slightly different effects as described above.
The main file is prepared as usual, see \secref{sec:include}.

%%%%%%%%%%%%%%%%%%%%%%%%%%%%%%%%%%%%%%%%%%%%%%%%%%%%%%%%%%%%%%%%%%%%%%%%%%%%%%%%
\subsection{Legacy Detection}
\label{sec:detection}

The directive |\childdocmain| in the main file can detect
whether the complete document or merely a child is to be compiled
even without using the directive |\childdocof|.
This method is deprecated because it is less robust
and there is no compelling reason to use it;
it is merely provided for backward compatibility
and it may be removed in future versions.

If the detection mechanism is to be used,
it is mandatory to correctly specify
the filename of the main file as the argument of |\childdocmain|:
%
\begin{center}
\begin{tabular}{l}
|\input{childdoc.def}|\\
|\childdocmain{|\textit{main}|}|\\
\end{tabular}
\end{center}
%
If |\jobname| does not match the argument \textit{main} of |\childdocmain|,
it is assumed that |\jobname| points to the child file to be compiled.
When using |\childdocmain| with the main file specified as argument,
it suffices to start a child file
with just |\input{|\textit{main}|}|
without loading of the package and using |\childdocof|.
If instead all processing is done
with the appropriate \textsf{childdoc} directives,
the argument of \textit{main} of |\childdocmain| can be empty.

An alternative version of the command line processing described
in \secref{sec:commandline} using the detection mechanism reads:
%
\begin{center}
|... -jobname "|\textit{target}|" "|[\textit{flags}]%
[|\def\jobname{|\textit{dest}|}|]|\input{|\textit{main}|}"|
\end{center}

%%%%%%%%%%%%%%%%%%%%%%%%%%%%%%%%%%%%%%%%%%%%%%%%%%%%%%%%%%%%%%%%%%%%%%%%%%%%%%%%
\subsection{Manual Code}
\label{sec:manual}

In case one cannot be certain whether the definitions file |childdoc.def|
is installed on the target \TeX{} distribution
and one prefers not to ship it,
it is conceivable to paste a few relevant commands into the sources.

To that end, drop all statements |\input{childdoc.def}|
and perform the replacements as outlined below.
Instead of |\childdocmain{|\textit{main}|}| add the following code
to the top of the main file:
%
\begin{center}
\begin{tabular}{l}
|\||ifdefined\childdocname\endinput\||fi\newif\ifchilddoc|\\
|\edef\childdocname{\scantokens\expandafter{\jobname\noexpand}}|\\
|\def\childdocmain{|\textit{main}|}\||ifx\childdocmain\childdocname\||else|\\
|\childdoctrue\includeonly{\childdocname}\let\jobname\childdocmain\||fi|\\
\end{tabular}
\end{center}
%
Instead of |\childdocof{|\textit{main}|}| just include the main file
at the top of each child file:
%
\begin{center}
|\input{|\textit{main}|}|
\end{center}
%
A simple redirection |\childdocforward{|\textit{dest}|}| is achieved by:
%
\begin{center}
|\def\jobname{|\textit{dest}|}\input{\jobname}|
\end{center}
%
The redirection with prefix
|\childdocforwardprefix[|\textit{prefix}|]{|\textit{dest}|}|
is accomplished by:
%
\begin{center}
\begin{tabular}{l}
|{\edef\jobname{\scantokens\expandafter{\jobname\noexpand}}|\\
|\def\redirectjob |\textit{prefix}|#1~~~{\gdef\jobname{|\textit{dest}|#1}}|\\
|\expandafter\redirectjob\jobname~~~}\input{\jobname}|
\end{tabular}
\end{center}

In an alternative approach,
child documents can be compiled by a specific command line
without additional code or specific definitions:
%
\begin{center}
|... -jobname "|\textit{target}|" "|[\textit{flags}]%
|\includeonly{|\textit{dest}|}\input{|\textit{main}|}"|
\end{center}
%

%%%%%%%%%%%%%%%%%%%%%%%%%%%%%%%%%%%%%%%%%%%%%%%%%%%%%%%%%%%%%%%%%%%%%%%%%%%%%%%%
%%%%%%%%%%%%%%%%%%%%%%%%%%%%%%%%%%%%%%%%%%%%%%%%%%%%%%%%%%%%%%%%%%%%%%%%%%%%%%%%
\section{Information}

%%%%%%%%%%%%%%%%%%%%%%%%%%%%%%%%%%%%%%%%%%%%%%%%%%%%%%%%%%%%%%%%%%%%%%%%%%%%%%%%
\subsection{Copyright}

Copyright \copyright{} 2017--2018 Niklas Beisert

This work may be distributed and/or modified under the
conditions of the \LaTeX{} Project Public License, either version 1.3
of this license or (at your option) any later version.
The latest version of this license is in
  \url{http://www.latex-project.org/lppl.txt}
and version 1.3 or later is part of all distributions of \LaTeX{}
version 2005/12/01 or later.

This work has the LPPL maintenance status `maintained'.

The Current Maintainer of this work is Niklas Beisert.

This work consists of the files |README.txt|, |childdoc.ins| and |childdoc.dtx|
as well as the derived files |childdoc.def|, |cdocsamp.tex|
with |cdocsch1.tex|, |cdocsch2.tex|, |cdocspt3.tex|, |cdocspt4.tex|,
|cdocsdrf.tex|, |cdocsfn1.tex|, |cdocsfn2.tex|
as well as |childdoc.pdf|.

%%%%%%%%%%%%%%%%%%%%%%%%%%%%%%%%%%%%%%%%%%%%%%%%%%%%%%%%%%%%%%%%%%%%%%%%%%%%%%%%
\subsection{Files and Installation}

The package consists of the files:
%
\begin{center}
\begin{tabular}{ll}
    |README.txt|   & readme file \\
    |childdoc.ins| & installation file \\
    |childdoc.dtx| & source file \\
    |childdoc.def| & definition file \\
    |cdocsamp.tex| & sample main file \\
    |cdocsch1.tex| & sample include file \\
    |cdocsch2.tex| & sample include file \\
    |cdocspt3.tex| & sample part file \\
    |cdocspt4.tex| & sample part file \\
    |cdocsdrf.tex| & sample redirection file \\
    |cdocsfn1.tex| & sample redirection file \\
    |cdocsfn2.tex| & sample redirection file \\
    |childdoc.pdf| & manual
\end{tabular}
\end{center}
%
The distribution consists of the files
|README.txt|, |childdoc.ins| and |childdoc.dtx|.
%
\begin{itemize}
\item
Run (pdf)\LaTeX{} on |childdoc.dtx|
to compile the manual |childdoc.pdf| (this file).
\item
Run \LaTeX{} on |childdoc.ins| to create the definitions file |childdoc.def|
and the sample |cdocsamp.tex| with include files
|cdocsch1.tex|, |cdocsch2.tex|, |cdocspt3.tex|, |cdocspt4.tex|,
|cdocsdrf.tex|, |cdocsfn1.tex|, |cdocsfn2.tex|.
Then copy the file |childdoc.def| to an appropriate directory of your \LaTeX{}
distribution, e.g.\ \textit{texmf-root}|/tex/latex/childdoc|.
\end{itemize}

%%%%%%%%%%%%%%%%%%%%%%%%%%%%%%%%%%%%%%%%%%%%%%%%%%%%%%%%%%%%%%%%%%%%%%%%%%%%%%%%
\subsection{Related CTAN Packages}

There are several other packages which offer a similar functionality:
%
\begin{itemize}
\item
The packages
\href{http://ctan.org/pkg/docmute}{\textsf{docmute}},
\href{http://ctan.org/pkg/includex}{\textsf{includex}} and
\href{http://ctan.org/pkg/standalone}{\textsf{standalone}}
provide commands to include only the document body of
a child file thus allowing both files to be compiled individually.
\item
The packages \href{http://ctan.org/pkg/subdocs}{\textsf{subdocs}}
and \href{http://ctan.org/pkg/subfiles}{\textsf{subfiles}}
provide structures in which the main and child documents can be
encapsulated and allowing them to be compiled individually.
The inclusion mechanism is different from the conventional |\include|.
\item
The package \href{http://ctan.org/pkg/combine}{\textsf{combine}}
is an elaborate solution to combine several documents into one.
\end{itemize}
%
See also the CTAN topic \href{http://ctan.org/topic/subdocs}{\textsf{subdocs}}
for further related packages.
The present package differs from the above solutions in that
a document structure constructed with the conventional |\include| mechanism
just needs two extra commands at the top of every file
such that all constituent files can be compiled individually.

%%%%%%%%%%%%%%%%%%%%%%%%%%%%%%%%%%%%%%%%%%%%%%%%%%%%%%%%%%%%%%%%%%%%%%%%%%%%%%%%
%\subsection{Feature Suggestions}
%
%The following is a list of features which may be useful for future
%versions of this package:
%%
%\begin{itemize}
%\item
%\ldots
%\end{itemize}

%%%%%%%%%%%%%%%%%%%%%%%%%%%%%%%%%%%%%%%%%%%%%%%%%%%%%%%%%%%%%%%%%%%%%%%%%%%%%%%%
\subsection{Revision History}

%%%%%%%%%%%%%%%%%%%%%%%%%%%%%%%%%%%%%%%%
\paragraph{v2.0:} 2018/12/30

\begin{itemize}
\item
immediate forward processing
\item
added |\childdocby| mechanism
\item
manual restructured
\end{itemize}

%%%%%%%%%%%%%%%%%%%%%%%%%%%%%%%%%%%%%%%%
\paragraph{v1.6:} 2018/01/17

\begin{itemize}
\item
application for development of include files
\item
corrections to manual
\end{itemize}

%%%%%%%%%%%%%%%%%%%%%%%%%%%%%%%%%%%%%%%%
\paragraph{v1.5:} 2017/05/21

\begin{itemize}
\item
more complete structuring introduced
\item
|\childdocof| introduced
\item
|\childdoc| renamed to |\childdocmain|
\item
|\childredirect| renamed to |\childdocforward| and |\childdocforwardprefix|
and functionality expanded
\end{itemize}

%%%%%%%%%%%%%%%%%%%%%%%%%%%%%%%%%%%%%%%%
\paragraph{v1.0:} 2017/04/27

\begin{itemize}
\item
manual and install package
\item
first version published on CTAN
\end{itemize}

%%%%%%%%%%%%%%%%%%%%%%%%%%%%%%%%%%%%%%%%
\paragraph{v0.6:} 2017/04/26

\begin{itemize}
\item
redirection mechanism added
\end{itemize}

%%%%%%%%%%%%%%%%%%%%%%%%%%%%%%%%%%%%%%%%
\paragraph{v0.5:} 2017/04/26

\begin{itemize}
\item
functionality in definition file
\end{itemize}


%%%%%%%%%%%%%%%%%%%%%%%%%%%%%%%%%%%%%%%%%%%%%%%%%%%%%%%%%%%%%%%%%%%%%%%%%%%%%%%%
%%%%%%%%%%%%%%%%%%%%%%%%%%%%%%%%%%%%%%%%%%%%%%%%%%%%%%%%%%%%%%%%%%%%%%%%%%%%%%%%
%%%%%%%%%%%%%%%%%%%%%%%%%%%%%%%%%%%%%%%%%%%%%%%%%%%%%%%%%%%%%%%%%%%%%%%%%%%%%%%%
\appendix

\settowidth\MacroIndent{\rmfamily\scriptsize 000\ }

 \DocInput{childdoc.dtx}

\end{document}
%</driver>
% \fi
%
% %%%%%%%%%%%%%%%%%%%%%%%%%%%%%%%%%%%%%%%%%%%%%%%%%%%%%%%%%%%%%%%%%%%%%%%%%%%%%%
% %%%%%%%%%%%%%%%%%%%%%%%%%%%%%%%%%%%%%%%%%%%%%%%%%%%%%%%%%%%%%%%%%%%%%%%%%%%%%%
% \section{Sample}
%\iffalse
%<*samplemain>
%\fi
%
% The following presents a sample document
% with two chapters, two parts, a title page,
% a compile flag as well as three forwarding files to set the flag.
% It consists of eight |.tex| files:
% \begin{center}
% \begin{tabular}{ll}
% |cdocsamp.tex|&main file\\
% |cdocsch1.tex|&include file for chapter 1\\
% |cdocsch2.tex|&include file for chapter 2\\
% |cdocspt3.tex|&include file for part 3\\
% |cdocspt4.tex|&include file for part 4\\
% |cdocsdrf.tex|&forwarding file for main file in draft mode\\
% |cdocsfi1.tex|&forwarding file for final version of chapter 1\\
% |cdocsfi2.tex|&forwarding file for final version of chapter 2\\
% \end{tabular}
% \end{center}
% Each of the eight files can be compiled directly by the \LaTeX{} compiler.
%
% %%%%%%%%%%%%%%%%%%%%%%%%%%%%%%%%%%%%%%
% \paragraph{Main File.}
%
% The main file is called |cdocsamp.tex|.
%
% Load the \textsf{childdoc} definitions and
% declare the filename for the main document:
%    \begin{macrocode}
\input{childdoc.def}
\childdocmain{}
%    \end{macrocode}

% Optional override for |\version| flag:
%    \begin{macrocode}
%%\ifchilddoc\else\providecommand{\version}{draft}\fi
%    \end{macrocode}

% Define the default values for the |\version| flag
% (|final| for the main file and |draft| for childs):
%    \begin{macrocode}
\ifchilddoc
\providecommand{\version}{draft}
\else
\providecommand{\version}{final}
\fi
%    \end{macrocode}

% Load the standard document class:
%    \begin{macrocode}
\documentclass[12pt]{article}
%    \end{macrocode}

% Start the document body:
%    \begin{macrocode}
\begin{document}
%    \end{macrocode}

% Declare a title page.
% Print title, part of document being processed and version flag:
%    \begin{macrocode}
\addtocounter{page}{-1}
\begin{center}
{\LARGE\bfseries{}childdoc example\par}
\vspace{1cm}
\ifchilddoc
\ifchilddocmanual part\else chapter\fi:
`\childdocname' of `\childdocjob'\par
\else
main document: `\childdocjob'\par
\fi
version: \version\par
\end{center}
\newpage
%    \end{macrocode}

% Manually include selected file,
% otherwise process as usual:
%    \begin{macrocode}
\ifchilddocmanual
\section*{part `\childdocname'}
\input{\childdocname}
\else
%    \end{macrocode}

% Include the two chapters:
%    \begin{macrocode}
\include{cdocsch1}
\include{cdocsch2}
%    \end{macrocode}

% Include the two parts unless only chapters should be displayed:
%    \begin{macrocode}
\ifchilddoc\else
\section{part three}
\input{cdocspt3}
\section{part four}
\input{cdocspt4}
\fi
%    \end{macrocode}

% Process as usual until here:
%    \begin{macrocode}
\fi
%    \end{macrocode}

% End of document body:
%    \begin{macrocode}
\end{document}
%    \end{macrocode}
%\iffalse
%</samplemain>
%\fi
%
% %%%%%%%%%%%%%%%%%%%%%%%%%%%%%%%%%%%%%%
% \paragraph{Chapter Include Files.}
%
% The include files are called |cdocsch1.tex| and |cdocsch2.tex|.
%
%\iffalse
%<*samplechap1|samplechap2>
%\fi

% Optional override for |\version| flag:
%    \begin{macrocode}
%%\providecommand{\version}{final}
%    \end{macrocode}

% Include the main document:
%    \begin{macrocode}
\input{childdoc.def}
\childdocof{cdocsamp}
%    \end{macrocode}

%\iffalse
%</samplechap1|samplechap2>
%\fi
%
%\iffalse
%<*samplechap1>
%\fi
% Some text for chapter 1:
%    \begin{macrocode}
\section{one}
some text in chapter one
%    \end{macrocode}

%\iffalse
%</samplechap1>
%\fi
% Some text for chapter 2:
%\iffalse
%<*samplechap2>
%\fi
%    \begin{macrocode}
\section{two}
more text in chapter two
%    \end{macrocode}

%\iffalse
%</samplechap2>
%\fi
%
% %%%%%%%%%%%%%%%%%%%%%%%%%%%%%%%%%%%%%%
% \paragraph{Part Include Files.}
%
% The include files are called |cdocspt3.tex| and |cdocspt4.tex|.
%
%\iffalse
%<*samplepart3|samplepart4>
%\fi

% Optional override for |\version| flag:
%    \begin{macrocode}
%%\providecommand{\version}{final}
%    \end{macrocode}

% Include the main document:
%    \begin{macrocode}
\input{childdoc.def}
\childdocby{cdocsamp}
%    \end{macrocode}

%\iffalse
%</samplepart3|samplepart4>
%\fi
%
%\iffalse
%<*samplepart3>
%\fi
% Some text for part 3:
%    \begin{macrocode}
some text in part three
%    \end{macrocode}

%\iffalse
%</samplepart3>
%\fi
% Some text for part 4:
%\iffalse
%<*samplepart4>
%\fi
%    \begin{macrocode}
more text in part four
%    \end{macrocode}

%\iffalse
%</samplepart4>
%\fi
%
% %%%%%%%%%%%%%%%%%%%%%%%%%%%%%%%%%%%%%%
% \paragraph{Forwarding for a Complete Draft.}
%
% The following forwarding file |cdocsdrf.tex|
% compiles the main document in draft mode:
%\iffalse
%<*sampledraft>
%\fi
%    \begin{macrocode}
\def\version{draft}
\input{childdoc.def}
\childdocforward{cdocsamp}
%    \end{macrocode}

%\iffalse
%</sampledraft>
%\fi
%
% %%%%%%%%%%%%%%%%%%%%%%%%%%%%%%%%%%%%%%
% \paragraph{Forwarding for Final Version of the Chapters.}
%
% The following forwarding files |cdocsfn1.tex| and |cdocsfn2.tex|
% (with identical content)
% compile the final versions of the child documents
% |cdocsch1.tex| and |cdocsch2.tex|, respectively:
%\iffalse
%<*samplefinal>
%\fi
%    \begin{macrocode}
\def\version{final}
\input{childdoc.def}
\childdocforwardprefix[cdocsamp]{cdocsfn}{cdocsch}
%    \end{macrocode}

%\iffalse
%</samplefinal>
%\fi
%
% %%%%%%%%%%%%%%%%%%%%%%%%%%%%%%%%%%%%%%
% \paragraph{Command Line Processing.}
%
% The following three command lines generate the output files
% |cdocscld|, |cdocscl1| and |cdocscl2|
% which should be identical to
% |cdocsdrf|, |cdocsch1| and |cdocsfn2|, respectively:
% \begin{center}
% \begin{tabular}{l}
% |latex -jobname cdocscld \|\\
% |  "\def\version{draft}\input{childdoc.def}\childdocforward{cdocsamp}"|\\
% |latex -jobname cdocscl1 \|\\
% |  "\input{childdoc.def}\childdocforward[cdocsamp]{cdocsch1}"|\\
% |latex -jobname cdocscl2 \|\\
% |  "\def\version{final}\input{childdoc.def}\childdocforward{cdocsch2}"|
% \end{tabular}
% \end{center}
% Note that the trailing backslash on each first line
% merely continues the input to the second line
% (for convenient cut ant paste).
% Furthermore, the command |latex| can be replaced by any
% of its alternative versions such as |pdflatex|.
%
% %%%%%%%%%%%%%%%%%%%%%%%%%%%%%%%%%%%%%%%%%%%%%%%%%%%%%%%%%%%%%%%%%%%%%%%%%%%%%%
% %%%%%%%%%%%%%%%%%%%%%%%%%%%%%%%%%%%%%%%%%%%%%%%%%%%%%%%%%%%%%%%%%%%%%%%%%%%%%%
% \section{Implementation}
%\iffalse
%<*package>
%\fi
%
% This section describes the definitions file |childdoc.def|.

% The definitions cannot be loaded using |\usepackage| or |\RequirePackage|
% which has a mechanism to prevent loading a style file more than once.
% When loading the definitions by means of |\input|
% multiple instances have to be prevented manually:
%\iffalse
%This code needs to be before the `\ProvidesFile' directive
%which is defined at the beginning of this file.
%Therefore it is also placed there and commented out here.
%</package>
%<*discard>
%\fi
%    \begin{macrocode}
\ifdefined\childdocmain\endinput\fi
%    \end{macrocode}
%\iffalse
%</discard>
%<*package>
%\fi
%
% \macro{\ifchilddoc}
% \macro{\ifchilddocmanual}
% The conditional |\ifchilddoc| tells whether a
% child (true) or main (false) document is being compiled.
% The conditional |\ifchilddocmanual| tells whether
% the |\includeonly| mechanism is used (false) or
% the selection of child files must be performed manually (true).
% The definitions initialise to false:
%    \begin{macrocode}
\newif\ifchilddoc
\newif\ifchilddocmanual
%    \end{macrocode}

% \macro{\childdocname}
% \macro{\childdocjob}
% The macro |\childdocname| stores the name of the main document
% to be compiled. The macro |\childdocjob| stores the name of
% the document on which the \LaTeX{} compiler was originally invoked.
% The content of |\jobname| cannot be compared
% to filenames specified in the source due to different catcodes.
% The following code rescans |\jobname|, stores the result
% in |\childdocname| and saves a copy in |\childdocjob|:
%    \begin{macrocode}
\edef\childdocname{\scantokens\expandafter{\jobname\noexpand}}
\let\childdocjob\childdocname
%    \end{macrocode}

% \macro{\childdocdisable}
% The macro |\childdocdisable| prevents the main file
% from being processed more than once.
% At this stage, the main document command |\childdocmain|
% is assumed to be called once again where it should do nothing.
% Any subsequent call to it should prevent
% a secondary processing of the main document
% It overwrites the forwarding commands
% |\childdocof| and |\childdocforward|
% with empty macros to prevent further inclusions of the main document:
%    \begin{macrocode}
\newcommand{\childdocdisable}
{
  \renewcommand{\childdocmain}[1]{\renewcommand{\childdocmain}[1]{\endinput}}
  \renewcommand{\childdocof}[1]{}
  \renewcommand{\childdocby}[2][]{}
  \renewcommand{\childdocforward}[2][]{}
  \renewcommand{\childdocdisable}{}
}
%    \end{macrocode}

% \macro{\childdocmain}
% The macro |\childdocmain| is to be called at the top of the main file
% with nothing or the main filename (without extension) as argument.
% First, it breaks loops.
% If the argument is not empty and does not match |\childdocname|
% (which is set by the first inclusion of |childdoc.def|),
% |\ifchilddoc| is set to true, |\includeonly| is applied to the child file
% and |\jobname| is set to the main file
% (for proper handling of |.aux| files):
%    \begin{macrocode}
\newcommand{\childdocmain}[1]
{
  \childdocdisable\childdocmain{}
  \if?#1?\else
    \begingroup
      \def\childdoctmp{#1}
      \ifx\childdoctmp\childdocname
        \def\childdoctmp{}
      \else
        \def\childdoctmp
        {
          \childdoctrue
          \includeonly{\childdocname}
          \def\childdocjob{#1}
          \def\jobname{#1}
        }
      \fi
      \expandafter
    \endgroup
    \childdoctmp
  \fi
}
%    \end{macrocode}

% \macro{\childdocof}
% The command |\childdocof| redirects
% compilation to the main file |#1|.
%    \begin{macrocode}
\newcommand{\childdocof}[1]
{
  \childdocdisable
  \childdoctrue
  \includeonly{\childdocname}
  \def\jobname{#1}
  \def\childdocjob{#1}
  \input{#1}
}
%    \end{macrocode}

% \macro{\childdocby}
% The command |\childdocby| ....
%    \begin{macrocode}
\newcommand{\childdocby}[2][]
{
  \childdocdisable
  \childdoctrue
  \childdocmanualtrue
  \if?#1?\else
    \def\jobname{#2}
  \fi
  \def\childdocjob{#2}
  \input{#2}
  \endinput
}
%    \end{macrocode}

% \macro{\childdocforward}
% The command |\childdocforward| redirects
% compilation to the main file or
% (if the optional argument is given) a child file.
% Parameters are set as if the main file
% or a child file starting with |\childdocof| was compiled.
% Then compilation is handed over to the main file:
%    \begin{macrocode}
\newcommand{\childdocforward}[2][]
{
  \begingroup
    \if?#1?
      \def\childdoctmp
      {
        \def\childdocname{#2}
        \def\childdocjob{#2}
        \def\jobname{#2}
        \input{#2}
        \endinput
      }
    \else
      \def\childdoctmp
      {
        \childdocdisable
        \def\childdocname{#2}
        \childdoctrue
        \includeonly{#2}
        \def\childdocjob{#1}
        \def\jobname{#1}
        \input{#1}
        \endinput
      }
    \fi
    \expandafter
  \endgroup
  \childdoctmp
}
%    \end{macrocode}

% \macro{\childdocforwardprefix}
% The command |\childdocforwardprefix| redirects
% compilation to the main or a child file by means of a pattern.
% The prefix |#1| in the current filename is replaced by |#2|
% and the suffix of the current filename is kept
% (it is assumed that the filename does not contain the substring `|~~~|'
% which is used as a delimiter).
% Compilation is handed over to the new file by |\childdocforward|:
%    \begin{macrocode}
\newcommand{\childdocforwardprefix}[3][]
{
  \begingroup
    \def\childdocextract #2##1~~~{\def\childdoctmp{\childdocforward[#1]{#3##1}}}
    \expandafter\childdocextract\childdocname~~~
    \expandafter
  \endgroup
  \childdoctmp
}
%    \end{macrocode}

% \macro{\childdoc}
% The deprecated macro |\childdoc| is a legacy version of |\childdocmain|:
%    \begin{macrocode}
\newcommand{\childdoc}{\childdocmain}
%    \end{macrocode}

% \macro{\childdocredirect}
% The deprecated macro |\childdocredirect| is a legacy version
% of |\childdocforward| and |\childdocforwardprefix|:
%    \begin{macrocode}
\newcommand{\childdocredirect}[2][]
{
  \begingroup
    \if?#1?
      \def\childdoctmp{\childdocforward{#2}}
    \else
      \def\childdoctmp{\childdocforwardprefix{#1}{#2}}
    \fi
    \expandafter
  \endgroup
  \childdoctmp
}
%    \end{macrocode}

%\iffalse
%</package>
%\fi
%
\endinput
|\\
|\childdocby{|\textit{main}|}|\\
\end{tabular}
\end{center}
%
Both forms have slightly different effects as described above.
The main file is prepared as usual, see \secref{sec:include}.

%%%%%%%%%%%%%%%%%%%%%%%%%%%%%%%%%%%%%%%%%%%%%%%%%%%%%%%%%%%%%%%%%%%%%%%%%%%%%%%%
\subsection{Legacy Detection}
\label{sec:detection}

The directive |\childdocmain| in the main file can detect
whether the complete document or merely a child is to be compiled
even without using the directive |\childdocof|.
This method is deprecated because it is less robust
and there is no compelling reason to use it;
it is merely provided for backward compatibility
and it may be removed in future versions.

If the detection mechanism is to be used,
it is mandatory to correctly specify
the filename of the main file as the argument of |\childdocmain|:
%
\begin{center}
\begin{tabular}{l}
|% \iffalse
%
% childdoc.dtx Copyright (C) 2017-2018 Niklas Beisert
%
% This work may be distributed and/or modified under the
% conditions of the LaTeX Project Public License, either version 1.3
% of this license or (at your option) any later version.
% The latest version of this license is in
%   http://www.latex-project.org/lppl.txt
% and version 1.3 or later is part of all distributions of LaTeX
% version 2005/12/01 or later.
%
% This work has the LPPL maintenance status `maintained'.
%
% The Current Maintainer of this work is Niklas Beisert.
%
% This work consists of the files childdoc.dtx and childdoc.ins
% and the derived files childdoc.def and cdocsamp.tex with
% cdocsch1.tex, cdocsch2.tex, cdocsdrf.tex, cdocsfn1.tex, cdocsfn2.tex.
%
%<package>\ifdefined\childdocmain\endinput\fi
%<package>\ProvidesFile{childdoc.def}[2018/12/30 v2.0 child document driver]
%<samplemain>\ProvidesFile{cdocsamp.tex}[2018/12/30 v2.0 sample for childdoc]
%<*driver>
%\ProvidesFile{childdoc.drv}[2018/12/30 v2.0 childdoc reference manual file]
\PassOptionsToClass{10pt,a4paper}{article}
\documentclass{ltxdoc}

\usepackage[margin=35mm]{geometry}
\usepackage{hyperref}
\usepackage{hyperxmp}
\usepackage[usenames]{color}

\hypersetup{colorlinks=true}
\hypersetup{pdfstartview=FitH}
\hypersetup{pdfpagemode=UseNone}
\hypersetup{pdfsource={}}
\hypersetup{pdflang={en-UK}}
\hypersetup{pdfcopyright={Copyright 2017-2018 Niklas Beisert.
  This work may be distributed and/or modified under the
  conditions of the LaTeX Project Public License, either version 1.3
  of this license or (at your option) any later version.}}
\hypersetup{pdflicenseurl={http://www.latex-project.org/lppl.txt}}
\hypersetup{pdfcontactaddress={ETH Zurich, ITP, HIT K,
  Wolfgang-Pauli-Strasse 27}}
\hypersetup{pdfcontactpostcode={8093}}
\hypersetup{pdfcontactcity={Zurich}}
\hypersetup{pdfcontactcountry={Switzerland}}
\hypersetup{pdfcontactemail={nbeisert@itp.phys.ethz.ch}}
\hypersetup{pdfcontacturl={http://people.phys.ethz.ch/\xmptilde nbeisert/}}

\newcommand{\secref}[1]{\hyperref[#1]{section \ref*{#1}}}

\parskip1ex
\parindent0pt
\let\olditemize\itemize
\def\itemize{\olditemize\parskip0pt}

\begin{document}

\title{The \textsf{childdoc} Package}
\hypersetup{pdftitle={The childdoc Package}}
\author{Niklas Beisert\\[2ex]
  Institut f\"ur Theoretische Physik\\
  Eidgen\"ossische Technische Hochschule Z\"urich\\
  Wolfgang-Pauli-Strasse 27, 8093 Z\"urich, Switzerland\\[1ex]
  \href{mailto:nbeisert@itp.phys.ethz.ch}
  {\texttt{nbeisert@itp.phys.ethz.ch}}}
\hypersetup{pdfauthor={Niklas Beisert}}
\hypersetup{pdfsubject={Manual for the LaTeX2e Package childdoc}}
\date{30 December 2018, \textsf{v2.0}}
\maketitle

\begin{abstract}\noindent
\textsf{childdoc} is a \LaTeXe{} package
that enables the direct compilation
of document sections included by |\include|
to individual files.
\end{abstract}

\begingroup
\parskip0ex
\tableofcontents
\endgroup

%%%%%%%%%%%%%%%%%%%%%%%%%%%%%%%%%%%%%%%%%%%%%%%%%%%%%%%%%%%%%%%%%%%%%%%%%%%%%%%%
%%%%%%%%%%%%%%%%%%%%%%%%%%%%%%%%%%%%%%%%%%%%%%%%%%%%%%%%%%%%%%%%%%%%%%%%%%%%%%%%
\section{Introduction}

\LaTeX{} provides a mechanism to structure a large document (such as a book)
into a main file and several child files (containing the chapters)
using the |\include| command.
This mechanism is beneficial for documents
which span hundreds of pages in order to
make the source file(s) more manageable.
Moreover, compilation can be restricted to
selected child files by means of the |\includeonly| command.
The latter feature can be used to reduce the compilation time while editing
(this was significantly more useful in the earlier days of \LaTeX{})
or to generate a smaller document which is easier to navigate.
Another application of |\includeonly| is to generate
documents consisting of selected parts of the complete document.

However, there are a few drawbacks of the plain |\include| mechanism:
\begin{itemize}
\item
The child files cannot be compiled on their own,
they can only be compiled via the main file.
A naive editing environment
(such as a text editor with an option
to have the current file processed by \LaTeX)
may require one to switch to the main file before compiling;
attempting to compile the child file produces errors.
\item
The main file must be modified (each time)
to adjust the |\includeonly| command
to the present needs. This easily leaves the main file in a messy state.
\item
The generated document will always carry the filename
of the main document. This is inconvenient if
several child files are to be compiled and
to be kept for distribution.
\end{itemize}

The present package provides a simple interface
to make child files individually compilable by \LaTeX{}.
Compiling a child file then has the same effect as compiling
the main file with an |\includeonly| command
to select the appropriate child.
Moreover the generated document will carry the name of the child
rather than the main file.
This resolves all three above issues.

This feature is meant to make the editing of books,
thesis documents and lecture notes somewhat more convenient.
However, the package can also be used efficiently for
composing a series of documents (such as exercise sheets)
which are typically distributed individually.
It then assists the author in generating the individual documents
(potentially in different versions)
as well as a document containing the collected series.
Another application is in developing style files
or other kinds of included material
where compilation of the style file could redirect
to a sample or test file.

%%%%%%%%%%%%%%%%%%%%%%%%%%%%%%%%%%%%%%%%%%%%%%%%%%%%%%%%%%%%%%%%%%%%%%%%%%%%%%%%
%%%%%%%%%%%%%%%%%%%%%%%%%%%%%%%%%%%%%%%%%%%%%%%%%%%%%%%%%%%%%%%%%%%%%%%%%%%%%%%%
\section{Usage}

First of all, the package \textsf{childdoc} is \emph{not} a standard
\LaTeXe{} |.sty| style file! Therefore it needs to be invoked in
a non-standard way.

%%%%%%%%%%%%%%%%%%%%%%%%%%%%%%%%%%%%%%%%%%%%%%%%%%%%%%%%%%%%%%%%%%%%%%%%%%%%%%%%
\subsection{Included Files}
\label{sec:include}

%%%%%%%%%%%%%%%%%%%%%%%%%%%%%%%%%%%%%%%%
\DescribeMacro{\childdocmain}
To use the package, add the commands
\begin{center}
\begin{tabular}{l}
|\input{childdoc.def}|\\
|\childdocmain{}|\\
\end{tabular}
\end{center}
at the very top of the main \LaTeX{} file,
in particular \emph{before} the |\documentclass| statement!
The argument of |\childdocmain| should be left empty
(but it must be present).

%%%%%%%%%%%%%%%%%%%%%%%%%%%%%%%%%%%%%%%%
\DescribeMacro{\childdocof}
Furthermore, add the commands
\begin{center}
\begin{tabular}{l}
|\input{childdoc.def}|\\
|\childdocof{|\textit{main}|}|\\
\end{tabular}
\end{center}
at the top of every child file \textit{child}
which is included by |\include{|\textit{child}|}|
from within the main file
(or at least for those files to be compiled individually).
The argument \textit{main} must be the filename of the main file.

There are a couple of
considerations in setting up the main and child documents:

%%%%%%%%%%%%%%%%%%%%%%%%%%%%%%%%%%%%%%%%
\paragraph{Restrictions.}

Please note the following restrictions:
\begin{itemize}
\item
|\childdocmain| must be called with one argument \textit{main}
to ensure compatibility with earlier version of the package.
It must either be empty (|\childdocmain{}|)
or precisely match the filename of the main file in which it is specified.
See \secref{sec:detection} for further information.
\item
The filename \textit{main} must be specified without the |.tex| extension.
\item
The filename \textit{main} is case sensitive
(even in case-insensitive file systems)
due to internal string comparison.
\item
The argument \textit{main} should be fully expanded, it cannot be a macro.
\item
Subdirectories and special characters should be avoided in filenames.
\item
The command |\childdocmain{|\textit{main}|}| must be followed by a whitespace.
It should not be followed immediately by another command
or by a comment mark `|%|'.
This is because the \TeX{} parser reads the token immediately following
the argument of |\childdocmain| and puts it
at the beginning of every child section;
however, a white\-space is ignored.
\end{itemize}

%%%%%%%%%%%%%%%%%%%%%%%%%%%%%%%%%%%%%%%%
\paragraph{Content of Main File.}

It is advisable to place all content in the child files included by |\include|.
Any output contained in the main file will appear in all child documents
unless suppressed manually;
it cannot be suppressed automatically by the |\includeonly| directive
and thus should normally be avoided.
A method to include some content in the main file
by means of conditional processing is described in \secref{sec:conditional}.

%%%%%%%%%%%%%%%%%%%%%%%%%%%%%%%%%%%%%%%%
\paragraph{Page Numbering.}

When only a part of the document is compiled,
the appropriate numbering of pages
(as well as other status parameters)
is determined from the |.aux| files.
The latter contain information from previous passes.
However this information needs to propagate through
all intermediate child documents.
Therefore the page numbering in child documents may well
be inconsistent until the complete document is compiled at least once.

A useful (if unconventional) way to always ensure a consistent
page numbering is to restart the numbering in each child document
and denote the pages by `\textit{child}|.|\textit{page}'
where \textit{child} represents the chapter/section number of the child file.
This can be achieved by the command
|\numberwithin{page}{|\textit{child}|}|
of the \textsf{amsmath} package
where \textit{child} can be |chapter| or |section|
depending on the chosen structuring.
Alternatively, one can modify the macro |\thepage| appropriately
and reset the counter |page| at the start of each child file.

%%%%%%%%%%%%%%%%%%%%%%%%%%%%%%%%%%%%%%%%%%%%%%%%%%%%%%%%%%%%%%%%%%%%%%%%%%%%%%%%
\subsection{Conditional Processing}
\label{sec:conditional}

The package provides a mechanism to compile different versions
of a document. To customise the versions further some conditional processing
can come in handy to distinguish which version is being compiled.
The package provides two macros to describe the compilation context:

%%%%%%%%%%%%%%%%%%%%%%%%%%%%%%%%%%%%%%%%
\DescribeMacro{\ifchilddoc}
The conditional |\ifchilddoc| distinguishes between the compilation of
child documents and the main document:
%
\begin{center}
|\ifchilddoc |\textit{child-code}| |[|\||else |\textit{main-code}]| \||fi|
\end{center}

%%%%%%%%%%%%%%%%%%%%%%%%%%%%%%%%%%%%%%%%
\DescribeMacro{\childdocname}
\DescribeMacro{\childdocjob}
The macro |\childdocname| contains the filename (without extension)
of the main or child file being processed.
Note that |\childdocjob| will always contain the name of the main file.

%%%%%%%%%%%%%%%%%%%%%%%%%%%%%%%%%%%%%%%%
\paragraph{Title Page.}

Conditional processing can be used to include a title or banner page
in the main document when proper precautions are taken.
Importantly, the code in the main file should ensure that the page counter
(as well as other status parameters which are stored in the |.aux| files)
takes the same value after the conditional processing.
Otherwise the page numbers may take divergent values
depending on which part is compiled.

For example, a title page could be declared by:
%
\begin{center}
\begin{tabular}{l}
|\ifchilddoc\||else|\\
|\addtocounter{page}{-1}|\\
\textit{code for title page}\\
|\newpage|\\
|\||fi|
\end{tabular}
\end{center}
%
A banner page for the child documents can be generated by:
%
\begin{center}
\begin{tabular}{l}
|\ifchilddoc|\\
|\addtocounter{page}{-1}|\\
\textit{code for banner page}\\
|\newpage|\\
|\||fi|
\end{tabular}
\end{center}
%
Here one could write a message such as:
\begin{center}
|This is the part \childdocname{} of \childdocjob{}.|
\end{center}

%%%%%%%%%%%%%%%%%%%%%%%%%%%%%%%%%%%%%%%%%%%%%%%%%%%%%%%%%%%%%%%%%%%%%%%%%%%%%%%%
\subsection{Flags}
\label{sec:flags}

The package makes it easy to generate different versions
of the main or child documents.
To this end compilation flags can be defined
and assigned different default values.
They will be particularly useful in conjunction
with the forwarding mechanism described in \secref{sec:forward}.

For example, it may be useful to have a flag |\version|
which can be set to |draft| or |final|.
The document source will contain some conditional code
depending on the value of |\version|.
Suppose further, the flag should default to |final| for the main file
and to |draft| for child files
which is a natural assignment for editing the document.
This is achieved by placing the following code
in the preamble of the main document
(below the |\childdocmain| directive):
%
\begin{center}
\begin{tabular}{l}
|\ifchilddoc|\\
|\providecommand{\version}{draft}|\\
|\||else|\\
|\providecommand{\version}{final}|\\
|\||fi|
\end{tabular}
\end{center}
%
The definition by |\providecommand| makes sure
that previous definitions are not overwritten.
Further statements |\providecommand{\version}{...}|
can thus be added before the above code to override it.

For the main file, one might add a line
(between |\childdocmain| and the above block)
%
\begin{center}
|%\ifchilddoc\||else\providecommand{\version}{draft}\||fi|
\end{center}
%
which can be uncommented to produce a draft version.
Likewise one can add a line to the very top of a child file
(above the |\childdocof{|\textit{main}|}| directive)
%
\begin{center}
|%\providecommand{\version}{final}|
\end{center}
%
which can be uncommented to produce the final version of this child document.

%%%%%%%%%%%%%%%%%%%%%%%%%%%%%%%%%%%%%%%%%%%%%%%%%%%%%%%%%%%%%%%%%%%%%%%%%%%%%%%%
\subsection{Forwarding}
\label{sec:forward}

Different versions of the main or child documents
using compilation flags as described in \secref{sec:flags}
can be (permanently) stored in different files
for convenient compilation, viewing and distribution.
To this end, the package defines a command
to pass on compilation to a different file:

%%%%%%%%%%%%%%%%%%%%%%%%%%%%%%%%%%%%%%%%
\DescribeMacro{\childdocforward}
The command |\childdocforward| redirects processing to
another source file:
%
\begin{center}
\begin{tabular}{l}
|\input{childdoc.def}|\\
|\childdocforward[|\textit{main}|]{|\textit{dest}|}|\\
\end{tabular}
\end{center}
%
The argument \textit{dest} is the destination file
(without extension).
It should be the main file or one of the child files.
Note that further \textsf{childdoc} directives
such as |\childdocof| and |\childdocforward|
in the indicated file will be processed in this form.
The optional argument \textit{main}
passes on directly to the main file \textit{main}
while pretending to compile the child \textit{dest}.
This form behaves as if \textit{dest}
issues |\childdocof{|\textit{main}|}| right away,
and no further \textsf{childdoc} directives will be processed.

%%%%%%%%%%%%%%%%%%%%%%%%%%%%%%%%%%%%%%%%
\DescribeMacro{\...prefix}
In the alternative form |\childdocforwardprefix|,
%
\begin{center}
\begin{tabular}{l}
|\input{childdoc.def}|\\
|\childdocforwardprefix[|\textit{main}|]{|\textit{prefix}|}{|\textit{dest}|}|
\end{tabular}
\end{center}
%
the destination file is determined by a pattern
depending on the current file:
To make this work, the current file must be called
`{\textit{prefix}\hspace{0.2em}\textit{suffix}}'
with \textit{prefix} matching precisely the argument.
Processing is then passed on to the file
`{\textit{dest}\hspace{0.2em}\textit{suffix}}'.
Surely, the same effect is achieved by
directly specifying the
argument `{\textit{dest}\hspace{0.2em}\textit{suffix}}'
in the first form.
However, that requires to set up a different file
for each child. With the alternative form of the command
all these files can have exactly the same content
which simplifies setting them up and maintaining them.

For example, the following file |draft.tex|
with a compilation flag |\version| as described in \secref{sec:flags}
compiles the main document as a draft:
%
\begin{center}
\begin{tabular}{l}
|\def\version{draft}|\\
|\input{childdoc.def}|\\
|\childdocforward{|\textit{main}|}|
\end{tabular}
\end{center}
%
Likewise, the following files |final|\textit{nn}|.tex|
compile the final version of the child document
|child|\textit{nn}|.tex|:
%
\begin{center}
\begin{tabular}{l}
|\def\version{final}|\\
|\input{childdoc.def}|\\
|\childdocforwardprefix{final}{child}|
\end{tabular}
\end{center}
%

Note that when several versions of a main file and/or of each child file
are to be generated, it may be convenient to set up a |Makefile| or
shell script to automatise the process.

%%%%%%%%%%%%%%%%%%%%%%%%%%%%%%%%%%%%%%%%%%%%%%%%%%%%%%%%%%%%%%%%%%%%%%%%%%%%%%%%
\subsection{Command Line Processing}
\label{sec:commandline}

The effect of redirection files can also be achieved by invoking
the \LaTeX{} compiler with a more elaborate command line.
Most conveniently this should be done as part
of a shell script or a |Makefile|.

When using \textsf{childdoc} in the main file, the following
command lines effectively perform a redirection
(note that depending on the shell being used,
backslashes may have to be doubled: `|\|' $\to$ `|\\|'):
%
\begin{center}
|... -jobname "|\textit{target}|" |\\|"|[\textit{flags}]%
|\input{childdoc.def}\childdocforward[|\textit{main}|]{|\textit{dest}|}"|
\end{center}
%
Here \textit{target} is the name of the output file,
\textit{main} is the name of the main file
and \textit{dest} is the name of the main or child file to be processed
(all filenames without extensions).
The optional argument \textit{main} can be omitted
if \textit{main} matches \textit{dest}.
Optionally, compilation \textit{flags} can be defined via |\def| commands.
This command line makes the \TeX{} engine believe
it is compiling the file \textit{target}
whose content is specified as the latter parameter.
The provided code then forwards the processing to
\textit{main} or \textit{dest} as described in \secref{sec:forward}.

%%%%%%%%%%%%%%%%%%%%%%%%%%%%%%%%%%%%%%%%%%%%%%%%%%%%%%%%%%%%%%%%%%%%%%%%%%%%%%%%
\subsection{Include by Input}
\label{sec:input}

Including child documents by |\include| has some restrictions by design.
Most notably, the content of a child document always occupies
its own set of pages; pages cannot be shared between child documents.
Usually, this behaviour makes perfect sense
because each child document contain an essential part of the document.
However, in some situations it may be desirable to compose
a document from a collection of parts
without having mandatory page breaks between then.
For this case, the package
provides a mechanism to include parts
by |\input| which can also be processed individually.
However, by construction this mechanism
requires manual handling of the content to be output.

%%%%%%%%%%%%%%%%%%%%%%%%%%%%%%%%%%%%%%%%
\DescribeMacro{\ifchilddocmanual}
The main file should be prepared as usual, see \secref{sec:include}.
However, the document body must make a distinction
between processing of an individual part and of the main document, e.g.:
%
\begin{center}
\begin{tabular}{l}
|\ifchilddocmanual|\\
|\input{\childdocname}|\\
|\||else|\\
\textit{document body with }|\input{|\textit{part}|}|\\
|\||fi|
\end{tabular}
\end{center}
%
The conditional |\ifchilddocmanual| is true whenever
a part to be included by |\input| is being compiled,
and the name of the part is stored in |\childdocname|.

%%%%%%%%%%%%%%%%%%%%%%%%%%%%%%%%%%%%%%%%
\DescribeMacro{\childdocby}
Each part to be included by |\input| should start with:
%
\begin{center}
\begin{tabular}{l}
|\input{childdoc.def}|\\
|\childdocby{|\textit{main}|}|\\
\end{tabular}
\end{center}
%
The directive |\childdocby| is similar to |\childdocof|
described in \secref{sec:include},
but the subsequent selection of content must be done manually.
To that end, both |\ifchilddoc| and |\ifchilddocmanual|
will be true upon processing of a part,
and the name of the part is stored in |\childdocname|.
Note that |\jobname| will be set to the filename of the current part
so that each part receives an individual |.aux| file
that does not interfere with the |.aux| file(s) of the main document.
This behaviour can be altered by the alternative form
|\childdocby[*]{|\textit{main}|}| (with a non-empty optional argument)
which uses the |.aux| file of the main document
by setting |\jobname| to \textit{main}.

%%%%%%%%%%%%%%%%%%%%%%%%%%%%%%%%%%%%%%%%%%%%%%%%%%%%%%%%%%%%%%%%%%%%%%%%%%%%%%%%
\subsection{Driver Development}
\label{sec:driver}

The \textsf{childdoc} mechanism can also be use for the development
of definition files such as \LaTeX{} styles or classes.
This case differs from the above setup with multiple parts
included by |\include| in that no |\includeonly| should be invoked.
This can be achieved by starting the include file
(before |\ProvidesPackage|) with:
%
\begin{center}
\begin{tabular}{l}
|\input{childdoc.def}|\\
|\childdocforward{|\textit{main}|}|\\
\end{tabular}
\end{center}
%
or alternatively with:
%
\begin{center}
\begin{tabular}{l}
|\input{childdoc.def}|\\
|\childdocby{|\textit{main}|}|\\
\end{tabular}
\end{center}
%
Both forms have slightly different effects as described above.
The main file is prepared as usual, see \secref{sec:include}.

%%%%%%%%%%%%%%%%%%%%%%%%%%%%%%%%%%%%%%%%%%%%%%%%%%%%%%%%%%%%%%%%%%%%%%%%%%%%%%%%
\subsection{Legacy Detection}
\label{sec:detection}

The directive |\childdocmain| in the main file can detect
whether the complete document or merely a child is to be compiled
even without using the directive |\childdocof|.
This method is deprecated because it is less robust
and there is no compelling reason to use it;
it is merely provided for backward compatibility
and it may be removed in future versions.

If the detection mechanism is to be used,
it is mandatory to correctly specify
the filename of the main file as the argument of |\childdocmain|:
%
\begin{center}
\begin{tabular}{l}
|\input{childdoc.def}|\\
|\childdocmain{|\textit{main}|}|\\
\end{tabular}
\end{center}
%
If |\jobname| does not match the argument \textit{main} of |\childdocmain|,
it is assumed that |\jobname| points to the child file to be compiled.
When using |\childdocmain| with the main file specified as argument,
it suffices to start a child file
with just |\input{|\textit{main}|}|
without loading of the package and using |\childdocof|.
If instead all processing is done
with the appropriate \textsf{childdoc} directives,
the argument of \textit{main} of |\childdocmain| can be empty.

An alternative version of the command line processing described
in \secref{sec:commandline} using the detection mechanism reads:
%
\begin{center}
|... -jobname "|\textit{target}|" "|[\textit{flags}]%
[|\def\jobname{|\textit{dest}|}|]|\input{|\textit{main}|}"|
\end{center}

%%%%%%%%%%%%%%%%%%%%%%%%%%%%%%%%%%%%%%%%%%%%%%%%%%%%%%%%%%%%%%%%%%%%%%%%%%%%%%%%
\subsection{Manual Code}
\label{sec:manual}

In case one cannot be certain whether the definitions file |childdoc.def|
is installed on the target \TeX{} distribution
and one prefers not to ship it,
it is conceivable to paste a few relevant commands into the sources.

To that end, drop all statements |\input{childdoc.def}|
and perform the replacements as outlined below.
Instead of |\childdocmain{|\textit{main}|}| add the following code
to the top of the main file:
%
\begin{center}
\begin{tabular}{l}
|\||ifdefined\childdocname\endinput\||fi\newif\ifchilddoc|\\
|\edef\childdocname{\scantokens\expandafter{\jobname\noexpand}}|\\
|\def\childdocmain{|\textit{main}|}\||ifx\childdocmain\childdocname\||else|\\
|\childdoctrue\includeonly{\childdocname}\let\jobname\childdocmain\||fi|\\
\end{tabular}
\end{center}
%
Instead of |\childdocof{|\textit{main}|}| just include the main file
at the top of each child file:
%
\begin{center}
|\input{|\textit{main}|}|
\end{center}
%
A simple redirection |\childdocforward{|\textit{dest}|}| is achieved by:
%
\begin{center}
|\def\jobname{|\textit{dest}|}\input{\jobname}|
\end{center}
%
The redirection with prefix
|\childdocforwardprefix[|\textit{prefix}|]{|\textit{dest}|}|
is accomplished by:
%
\begin{center}
\begin{tabular}{l}
|{\edef\jobname{\scantokens\expandafter{\jobname\noexpand}}|\\
|\def\redirectjob |\textit{prefix}|#1~~~{\gdef\jobname{|\textit{dest}|#1}}|\\
|\expandafter\redirectjob\jobname~~~}\input{\jobname}|
\end{tabular}
\end{center}

In an alternative approach,
child documents can be compiled by a specific command line
without additional code or specific definitions:
%
\begin{center}
|... -jobname "|\textit{target}|" "|[\textit{flags}]%
|\includeonly{|\textit{dest}|}\input{|\textit{main}|}"|
\end{center}
%

%%%%%%%%%%%%%%%%%%%%%%%%%%%%%%%%%%%%%%%%%%%%%%%%%%%%%%%%%%%%%%%%%%%%%%%%%%%%%%%%
%%%%%%%%%%%%%%%%%%%%%%%%%%%%%%%%%%%%%%%%%%%%%%%%%%%%%%%%%%%%%%%%%%%%%%%%%%%%%%%%
\section{Information}

%%%%%%%%%%%%%%%%%%%%%%%%%%%%%%%%%%%%%%%%%%%%%%%%%%%%%%%%%%%%%%%%%%%%%%%%%%%%%%%%
\subsection{Copyright}

Copyright \copyright{} 2017--2018 Niklas Beisert

This work may be distributed and/or modified under the
conditions of the \LaTeX{} Project Public License, either version 1.3
of this license or (at your option) any later version.
The latest version of this license is in
  \url{http://www.latex-project.org/lppl.txt}
and version 1.3 or later is part of all distributions of \LaTeX{}
version 2005/12/01 or later.

This work has the LPPL maintenance status `maintained'.

The Current Maintainer of this work is Niklas Beisert.

This work consists of the files |README.txt|, |childdoc.ins| and |childdoc.dtx|
as well as the derived files |childdoc.def|, |cdocsamp.tex|
with |cdocsch1.tex|, |cdocsch2.tex|, |cdocspt3.tex|, |cdocspt4.tex|,
|cdocsdrf.tex|, |cdocsfn1.tex|, |cdocsfn2.tex|
as well as |childdoc.pdf|.

%%%%%%%%%%%%%%%%%%%%%%%%%%%%%%%%%%%%%%%%%%%%%%%%%%%%%%%%%%%%%%%%%%%%%%%%%%%%%%%%
\subsection{Files and Installation}

The package consists of the files:
%
\begin{center}
\begin{tabular}{ll}
    |README.txt|   & readme file \\
    |childdoc.ins| & installation file \\
    |childdoc.dtx| & source file \\
    |childdoc.def| & definition file \\
    |cdocsamp.tex| & sample main file \\
    |cdocsch1.tex| & sample include file \\
    |cdocsch2.tex| & sample include file \\
    |cdocspt3.tex| & sample part file \\
    |cdocspt4.tex| & sample part file \\
    |cdocsdrf.tex| & sample redirection file \\
    |cdocsfn1.tex| & sample redirection file \\
    |cdocsfn2.tex| & sample redirection file \\
    |childdoc.pdf| & manual
\end{tabular}
\end{center}
%
The distribution consists of the files
|README.txt|, |childdoc.ins| and |childdoc.dtx|.
%
\begin{itemize}
\item
Run (pdf)\LaTeX{} on |childdoc.dtx|
to compile the manual |childdoc.pdf| (this file).
\item
Run \LaTeX{} on |childdoc.ins| to create the definitions file |childdoc.def|
and the sample |cdocsamp.tex| with include files
|cdocsch1.tex|, |cdocsch2.tex|, |cdocspt3.tex|, |cdocspt4.tex|,
|cdocsdrf.tex|, |cdocsfn1.tex|, |cdocsfn2.tex|.
Then copy the file |childdoc.def| to an appropriate directory of your \LaTeX{}
distribution, e.g.\ \textit{texmf-root}|/tex/latex/childdoc|.
\end{itemize}

%%%%%%%%%%%%%%%%%%%%%%%%%%%%%%%%%%%%%%%%%%%%%%%%%%%%%%%%%%%%%%%%%%%%%%%%%%%%%%%%
\subsection{Related CTAN Packages}

There are several other packages which offer a similar functionality:
%
\begin{itemize}
\item
The packages
\href{http://ctan.org/pkg/docmute}{\textsf{docmute}},
\href{http://ctan.org/pkg/includex}{\textsf{includex}} and
\href{http://ctan.org/pkg/standalone}{\textsf{standalone}}
provide commands to include only the document body of
a child file thus allowing both files to be compiled individually.
\item
The packages \href{http://ctan.org/pkg/subdocs}{\textsf{subdocs}}
and \href{http://ctan.org/pkg/subfiles}{\textsf{subfiles}}
provide structures in which the main and child documents can be
encapsulated and allowing them to be compiled individually.
The inclusion mechanism is different from the conventional |\include|.
\item
The package \href{http://ctan.org/pkg/combine}{\textsf{combine}}
is an elaborate solution to combine several documents into one.
\end{itemize}
%
See also the CTAN topic \href{http://ctan.org/topic/subdocs}{\textsf{subdocs}}
for further related packages.
The present package differs from the above solutions in that
a document structure constructed with the conventional |\include| mechanism
just needs two extra commands at the top of every file
such that all constituent files can be compiled individually.

%%%%%%%%%%%%%%%%%%%%%%%%%%%%%%%%%%%%%%%%%%%%%%%%%%%%%%%%%%%%%%%%%%%%%%%%%%%%%%%%
%\subsection{Feature Suggestions}
%
%The following is a list of features which may be useful for future
%versions of this package:
%%
%\begin{itemize}
%\item
%\ldots
%\end{itemize}

%%%%%%%%%%%%%%%%%%%%%%%%%%%%%%%%%%%%%%%%%%%%%%%%%%%%%%%%%%%%%%%%%%%%%%%%%%%%%%%%
\subsection{Revision History}

%%%%%%%%%%%%%%%%%%%%%%%%%%%%%%%%%%%%%%%%
\paragraph{v2.0:} 2018/12/30

\begin{itemize}
\item
immediate forward processing
\item
added |\childdocby| mechanism
\item
manual restructured
\end{itemize}

%%%%%%%%%%%%%%%%%%%%%%%%%%%%%%%%%%%%%%%%
\paragraph{v1.6:} 2018/01/17

\begin{itemize}
\item
application for development of include files
\item
corrections to manual
\end{itemize}

%%%%%%%%%%%%%%%%%%%%%%%%%%%%%%%%%%%%%%%%
\paragraph{v1.5:} 2017/05/21

\begin{itemize}
\item
more complete structuring introduced
\item
|\childdocof| introduced
\item
|\childdoc| renamed to |\childdocmain|
\item
|\childredirect| renamed to |\childdocforward| and |\childdocforwardprefix|
and functionality expanded
\end{itemize}

%%%%%%%%%%%%%%%%%%%%%%%%%%%%%%%%%%%%%%%%
\paragraph{v1.0:} 2017/04/27

\begin{itemize}
\item
manual and install package
\item
first version published on CTAN
\end{itemize}

%%%%%%%%%%%%%%%%%%%%%%%%%%%%%%%%%%%%%%%%
\paragraph{v0.6:} 2017/04/26

\begin{itemize}
\item
redirection mechanism added
\end{itemize}

%%%%%%%%%%%%%%%%%%%%%%%%%%%%%%%%%%%%%%%%
\paragraph{v0.5:} 2017/04/26

\begin{itemize}
\item
functionality in definition file
\end{itemize}


%%%%%%%%%%%%%%%%%%%%%%%%%%%%%%%%%%%%%%%%%%%%%%%%%%%%%%%%%%%%%%%%%%%%%%%%%%%%%%%%
%%%%%%%%%%%%%%%%%%%%%%%%%%%%%%%%%%%%%%%%%%%%%%%%%%%%%%%%%%%%%%%%%%%%%%%%%%%%%%%%
%%%%%%%%%%%%%%%%%%%%%%%%%%%%%%%%%%%%%%%%%%%%%%%%%%%%%%%%%%%%%%%%%%%%%%%%%%%%%%%%
\appendix

\settowidth\MacroIndent{\rmfamily\scriptsize 000\ }

 \DocInput{childdoc.dtx}

\end{document}
%</driver>
% \fi
%
% %%%%%%%%%%%%%%%%%%%%%%%%%%%%%%%%%%%%%%%%%%%%%%%%%%%%%%%%%%%%%%%%%%%%%%%%%%%%%%
% %%%%%%%%%%%%%%%%%%%%%%%%%%%%%%%%%%%%%%%%%%%%%%%%%%%%%%%%%%%%%%%%%%%%%%%%%%%%%%
% \section{Sample}
%\iffalse
%<*samplemain>
%\fi
%
% The following presents a sample document
% with two chapters, two parts, a title page,
% a compile flag as well as three forwarding files to set the flag.
% It consists of eight |.tex| files:
% \begin{center}
% \begin{tabular}{ll}
% |cdocsamp.tex|&main file\\
% |cdocsch1.tex|&include file for chapter 1\\
% |cdocsch2.tex|&include file for chapter 2\\
% |cdocspt3.tex|&include file for part 3\\
% |cdocspt4.tex|&include file for part 4\\
% |cdocsdrf.tex|&forwarding file for main file in draft mode\\
% |cdocsfi1.tex|&forwarding file for final version of chapter 1\\
% |cdocsfi2.tex|&forwarding file for final version of chapter 2\\
% \end{tabular}
% \end{center}
% Each of the eight files can be compiled directly by the \LaTeX{} compiler.
%
% %%%%%%%%%%%%%%%%%%%%%%%%%%%%%%%%%%%%%%
% \paragraph{Main File.}
%
% The main file is called |cdocsamp.tex|.
%
% Load the \textsf{childdoc} definitions and
% declare the filename for the main document:
%    \begin{macrocode}
\input{childdoc.def}
\childdocmain{}
%    \end{macrocode}

% Optional override for |\version| flag:
%    \begin{macrocode}
%%\ifchilddoc\else\providecommand{\version}{draft}\fi
%    \end{macrocode}

% Define the default values for the |\version| flag
% (|final| for the main file and |draft| for childs):
%    \begin{macrocode}
\ifchilddoc
\providecommand{\version}{draft}
\else
\providecommand{\version}{final}
\fi
%    \end{macrocode}

% Load the standard document class:
%    \begin{macrocode}
\documentclass[12pt]{article}
%    \end{macrocode}

% Start the document body:
%    \begin{macrocode}
\begin{document}
%    \end{macrocode}

% Declare a title page.
% Print title, part of document being processed and version flag:
%    \begin{macrocode}
\addtocounter{page}{-1}
\begin{center}
{\LARGE\bfseries{}childdoc example\par}
\vspace{1cm}
\ifchilddoc
\ifchilddocmanual part\else chapter\fi:
`\childdocname' of `\childdocjob'\par
\else
main document: `\childdocjob'\par
\fi
version: \version\par
\end{center}
\newpage
%    \end{macrocode}

% Manually include selected file,
% otherwise process as usual:
%    \begin{macrocode}
\ifchilddocmanual
\section*{part `\childdocname'}
\input{\childdocname}
\else
%    \end{macrocode}

% Include the two chapters:
%    \begin{macrocode}
\include{cdocsch1}
\include{cdocsch2}
%    \end{macrocode}

% Include the two parts unless only chapters should be displayed:
%    \begin{macrocode}
\ifchilddoc\else
\section{part three}
\input{cdocspt3}
\section{part four}
\input{cdocspt4}
\fi
%    \end{macrocode}

% Process as usual until here:
%    \begin{macrocode}
\fi
%    \end{macrocode}

% End of document body:
%    \begin{macrocode}
\end{document}
%    \end{macrocode}
%\iffalse
%</samplemain>
%\fi
%
% %%%%%%%%%%%%%%%%%%%%%%%%%%%%%%%%%%%%%%
% \paragraph{Chapter Include Files.}
%
% The include files are called |cdocsch1.tex| and |cdocsch2.tex|.
%
%\iffalse
%<*samplechap1|samplechap2>
%\fi

% Optional override for |\version| flag:
%    \begin{macrocode}
%%\providecommand{\version}{final}
%    \end{macrocode}

% Include the main document:
%    \begin{macrocode}
\input{childdoc.def}
\childdocof{cdocsamp}
%    \end{macrocode}

%\iffalse
%</samplechap1|samplechap2>
%\fi
%
%\iffalse
%<*samplechap1>
%\fi
% Some text for chapter 1:
%    \begin{macrocode}
\section{one}
some text in chapter one
%    \end{macrocode}

%\iffalse
%</samplechap1>
%\fi
% Some text for chapter 2:
%\iffalse
%<*samplechap2>
%\fi
%    \begin{macrocode}
\section{two}
more text in chapter two
%    \end{macrocode}

%\iffalse
%</samplechap2>
%\fi
%
% %%%%%%%%%%%%%%%%%%%%%%%%%%%%%%%%%%%%%%
% \paragraph{Part Include Files.}
%
% The include files are called |cdocspt3.tex| and |cdocspt4.tex|.
%
%\iffalse
%<*samplepart3|samplepart4>
%\fi

% Optional override for |\version| flag:
%    \begin{macrocode}
%%\providecommand{\version}{final}
%    \end{macrocode}

% Include the main document:
%    \begin{macrocode}
\input{childdoc.def}
\childdocby{cdocsamp}
%    \end{macrocode}

%\iffalse
%</samplepart3|samplepart4>
%\fi
%
%\iffalse
%<*samplepart3>
%\fi
% Some text for part 3:
%    \begin{macrocode}
some text in part three
%    \end{macrocode}

%\iffalse
%</samplepart3>
%\fi
% Some text for part 4:
%\iffalse
%<*samplepart4>
%\fi
%    \begin{macrocode}
more text in part four
%    \end{macrocode}

%\iffalse
%</samplepart4>
%\fi
%
% %%%%%%%%%%%%%%%%%%%%%%%%%%%%%%%%%%%%%%
% \paragraph{Forwarding for a Complete Draft.}
%
% The following forwarding file |cdocsdrf.tex|
% compiles the main document in draft mode:
%\iffalse
%<*sampledraft>
%\fi
%    \begin{macrocode}
\def\version{draft}
\input{childdoc.def}
\childdocforward{cdocsamp}
%    \end{macrocode}

%\iffalse
%</sampledraft>
%\fi
%
% %%%%%%%%%%%%%%%%%%%%%%%%%%%%%%%%%%%%%%
% \paragraph{Forwarding for Final Version of the Chapters.}
%
% The following forwarding files |cdocsfn1.tex| and |cdocsfn2.tex|
% (with identical content)
% compile the final versions of the child documents
% |cdocsch1.tex| and |cdocsch2.tex|, respectively:
%\iffalse
%<*samplefinal>
%\fi
%    \begin{macrocode}
\def\version{final}
\input{childdoc.def}
\childdocforwardprefix[cdocsamp]{cdocsfn}{cdocsch}
%    \end{macrocode}

%\iffalse
%</samplefinal>
%\fi
%
% %%%%%%%%%%%%%%%%%%%%%%%%%%%%%%%%%%%%%%
% \paragraph{Command Line Processing.}
%
% The following three command lines generate the output files
% |cdocscld|, |cdocscl1| and |cdocscl2|
% which should be identical to
% |cdocsdrf|, |cdocsch1| and |cdocsfn2|, respectively:
% \begin{center}
% \begin{tabular}{l}
% |latex -jobname cdocscld \|\\
% |  "\def\version{draft}\input{childdoc.def}\childdocforward{cdocsamp}"|\\
% |latex -jobname cdocscl1 \|\\
% |  "\input{childdoc.def}\childdocforward[cdocsamp]{cdocsch1}"|\\
% |latex -jobname cdocscl2 \|\\
% |  "\def\version{final}\input{childdoc.def}\childdocforward{cdocsch2}"|
% \end{tabular}
% \end{center}
% Note that the trailing backslash on each first line
% merely continues the input to the second line
% (for convenient cut ant paste).
% Furthermore, the command |latex| can be replaced by any
% of its alternative versions such as |pdflatex|.
%
% %%%%%%%%%%%%%%%%%%%%%%%%%%%%%%%%%%%%%%%%%%%%%%%%%%%%%%%%%%%%%%%%%%%%%%%%%%%%%%
% %%%%%%%%%%%%%%%%%%%%%%%%%%%%%%%%%%%%%%%%%%%%%%%%%%%%%%%%%%%%%%%%%%%%%%%%%%%%%%
% \section{Implementation}
%\iffalse
%<*package>
%\fi
%
% This section describes the definitions file |childdoc.def|.

% The definitions cannot be loaded using |\usepackage| or |\RequirePackage|
% which has a mechanism to prevent loading a style file more than once.
% When loading the definitions by means of |\input|
% multiple instances have to be prevented manually:
%\iffalse
%This code needs to be before the `\ProvidesFile' directive
%which is defined at the beginning of this file.
%Therefore it is also placed there and commented out here.
%</package>
%<*discard>
%\fi
%    \begin{macrocode}
\ifdefined\childdocmain\endinput\fi
%    \end{macrocode}
%\iffalse
%</discard>
%<*package>
%\fi
%
% \macro{\ifchilddoc}
% \macro{\ifchilddocmanual}
% The conditional |\ifchilddoc| tells whether a
% child (true) or main (false) document is being compiled.
% The conditional |\ifchilddocmanual| tells whether
% the |\includeonly| mechanism is used (false) or
% the selection of child files must be performed manually (true).
% The definitions initialise to false:
%    \begin{macrocode}
\newif\ifchilddoc
\newif\ifchilddocmanual
%    \end{macrocode}

% \macro{\childdocname}
% \macro{\childdocjob}
% The macro |\childdocname| stores the name of the main document
% to be compiled. The macro |\childdocjob| stores the name of
% the document on which the \LaTeX{} compiler was originally invoked.
% The content of |\jobname| cannot be compared
% to filenames specified in the source due to different catcodes.
% The following code rescans |\jobname|, stores the result
% in |\childdocname| and saves a copy in |\childdocjob|:
%    \begin{macrocode}
\edef\childdocname{\scantokens\expandafter{\jobname\noexpand}}
\let\childdocjob\childdocname
%    \end{macrocode}

% \macro{\childdocdisable}
% The macro |\childdocdisable| prevents the main file
% from being processed more than once.
% At this stage, the main document command |\childdocmain|
% is assumed to be called once again where it should do nothing.
% Any subsequent call to it should prevent
% a secondary processing of the main document
% It overwrites the forwarding commands
% |\childdocof| and |\childdocforward|
% with empty macros to prevent further inclusions of the main document:
%    \begin{macrocode}
\newcommand{\childdocdisable}
{
  \renewcommand{\childdocmain}[1]{\renewcommand{\childdocmain}[1]{\endinput}}
  \renewcommand{\childdocof}[1]{}
  \renewcommand{\childdocby}[2][]{}
  \renewcommand{\childdocforward}[2][]{}
  \renewcommand{\childdocdisable}{}
}
%    \end{macrocode}

% \macro{\childdocmain}
% The macro |\childdocmain| is to be called at the top of the main file
% with nothing or the main filename (without extension) as argument.
% First, it breaks loops.
% If the argument is not empty and does not match |\childdocname|
% (which is set by the first inclusion of |childdoc.def|),
% |\ifchilddoc| is set to true, |\includeonly| is applied to the child file
% and |\jobname| is set to the main file
% (for proper handling of |.aux| files):
%    \begin{macrocode}
\newcommand{\childdocmain}[1]
{
  \childdocdisable\childdocmain{}
  \if?#1?\else
    \begingroup
      \def\childdoctmp{#1}
      \ifx\childdoctmp\childdocname
        \def\childdoctmp{}
      \else
        \def\childdoctmp
        {
          \childdoctrue
          \includeonly{\childdocname}
          \def\childdocjob{#1}
          \def\jobname{#1}
        }
      \fi
      \expandafter
    \endgroup
    \childdoctmp
  \fi
}
%    \end{macrocode}

% \macro{\childdocof}
% The command |\childdocof| redirects
% compilation to the main file |#1|.
%    \begin{macrocode}
\newcommand{\childdocof}[1]
{
  \childdocdisable
  \childdoctrue
  \includeonly{\childdocname}
  \def\jobname{#1}
  \def\childdocjob{#1}
  \input{#1}
}
%    \end{macrocode}

% \macro{\childdocby}
% The command |\childdocby| ....
%    \begin{macrocode}
\newcommand{\childdocby}[2][]
{
  \childdocdisable
  \childdoctrue
  \childdocmanualtrue
  \if?#1?\else
    \def\jobname{#2}
  \fi
  \def\childdocjob{#2}
  \input{#2}
  \endinput
}
%    \end{macrocode}

% \macro{\childdocforward}
% The command |\childdocforward| redirects
% compilation to the main file or
% (if the optional argument is given) a child file.
% Parameters are set as if the main file
% or a child file starting with |\childdocof| was compiled.
% Then compilation is handed over to the main file:
%    \begin{macrocode}
\newcommand{\childdocforward}[2][]
{
  \begingroup
    \if?#1?
      \def\childdoctmp
      {
        \def\childdocname{#2}
        \def\childdocjob{#2}
        \def\jobname{#2}
        \input{#2}
        \endinput
      }
    \else
      \def\childdoctmp
      {
        \childdocdisable
        \def\childdocname{#2}
        \childdoctrue
        \includeonly{#2}
        \def\childdocjob{#1}
        \def\jobname{#1}
        \input{#1}
        \endinput
      }
    \fi
    \expandafter
  \endgroup
  \childdoctmp
}
%    \end{macrocode}

% \macro{\childdocforwardprefix}
% The command |\childdocforwardprefix| redirects
% compilation to the main or a child file by means of a pattern.
% The prefix |#1| in the current filename is replaced by |#2|
% and the suffix of the current filename is kept
% (it is assumed that the filename does not contain the substring `|~~~|'
% which is used as a delimiter).
% Compilation is handed over to the new file by |\childdocforward|:
%    \begin{macrocode}
\newcommand{\childdocforwardprefix}[3][]
{
  \begingroup
    \def\childdocextract #2##1~~~{\def\childdoctmp{\childdocforward[#1]{#3##1}}}
    \expandafter\childdocextract\childdocname~~~
    \expandafter
  \endgroup
  \childdoctmp
}
%    \end{macrocode}

% \macro{\childdoc}
% The deprecated macro |\childdoc| is a legacy version of |\childdocmain|:
%    \begin{macrocode}
\newcommand{\childdoc}{\childdocmain}
%    \end{macrocode}

% \macro{\childdocredirect}
% The deprecated macro |\childdocredirect| is a legacy version
% of |\childdocforward| and |\childdocforwardprefix|:
%    \begin{macrocode}
\newcommand{\childdocredirect}[2][]
{
  \begingroup
    \if?#1?
      \def\childdoctmp{\childdocforward{#2}}
    \else
      \def\childdoctmp{\childdocforwardprefix{#1}{#2}}
    \fi
    \expandafter
  \endgroup
  \childdoctmp
}
%    \end{macrocode}

%\iffalse
%</package>
%\fi
%
\endinput
|\\
|\childdocmain{|\textit{main}|}|\\
\end{tabular}
\end{center}
%
If |\jobname| does not match the argument \textit{main} of |\childdocmain|,
it is assumed that |\jobname| points to the child file to be compiled.
When using |\childdocmain| with the main file specified as argument,
it suffices to start a child file
with just |\input{|\textit{main}|}|
without loading of the package and using |\childdocof|.
If instead all processing is done
with the appropriate \textsf{childdoc} directives,
the argument of \textit{main} of |\childdocmain| can be empty.

An alternative version of the command line processing described
in \secref{sec:commandline} using the detection mechanism reads:
%
\begin{center}
|... -jobname "|\textit{target}|" "|[\textit{flags}]%
[|\def\jobname{|\textit{dest}|}|]|\input{|\textit{main}|}"|
\end{center}

%%%%%%%%%%%%%%%%%%%%%%%%%%%%%%%%%%%%%%%%%%%%%%%%%%%%%%%%%%%%%%%%%%%%%%%%%%%%%%%%
\subsection{Manual Code}
\label{sec:manual}

In case one cannot be certain whether the definitions file |childdoc.def|
is installed on the target \TeX{} distribution
and one prefers not to ship it,
it is conceivable to paste a few relevant commands into the sources.

To that end, drop all statements |% \iffalse
%
% childdoc.dtx Copyright (C) 2017-2018 Niklas Beisert
%
% This work may be distributed and/or modified under the
% conditions of the LaTeX Project Public License, either version 1.3
% of this license or (at your option) any later version.
% The latest version of this license is in
%   http://www.latex-project.org/lppl.txt
% and version 1.3 or later is part of all distributions of LaTeX
% version 2005/12/01 or later.
%
% This work has the LPPL maintenance status `maintained'.
%
% The Current Maintainer of this work is Niklas Beisert.
%
% This work consists of the files childdoc.dtx and childdoc.ins
% and the derived files childdoc.def and cdocsamp.tex with
% cdocsch1.tex, cdocsch2.tex, cdocsdrf.tex, cdocsfn1.tex, cdocsfn2.tex.
%
%<package>\ifdefined\childdocmain\endinput\fi
%<package>\ProvidesFile{childdoc.def}[2018/12/30 v2.0 child document driver]
%<samplemain>\ProvidesFile{cdocsamp.tex}[2018/12/30 v2.0 sample for childdoc]
%<*driver>
%\ProvidesFile{childdoc.drv}[2018/12/30 v2.0 childdoc reference manual file]
\PassOptionsToClass{10pt,a4paper}{article}
\documentclass{ltxdoc}

\usepackage[margin=35mm]{geometry}
\usepackage{hyperref}
\usepackage{hyperxmp}
\usepackage[usenames]{color}

\hypersetup{colorlinks=true}
\hypersetup{pdfstartview=FitH}
\hypersetup{pdfpagemode=UseNone}
\hypersetup{pdfsource={}}
\hypersetup{pdflang={en-UK}}
\hypersetup{pdfcopyright={Copyright 2017-2018 Niklas Beisert.
  This work may be distributed and/or modified under the
  conditions of the LaTeX Project Public License, either version 1.3
  of this license or (at your option) any later version.}}
\hypersetup{pdflicenseurl={http://www.latex-project.org/lppl.txt}}
\hypersetup{pdfcontactaddress={ETH Zurich, ITP, HIT K,
  Wolfgang-Pauli-Strasse 27}}
\hypersetup{pdfcontactpostcode={8093}}
\hypersetup{pdfcontactcity={Zurich}}
\hypersetup{pdfcontactcountry={Switzerland}}
\hypersetup{pdfcontactemail={nbeisert@itp.phys.ethz.ch}}
\hypersetup{pdfcontacturl={http://people.phys.ethz.ch/\xmptilde nbeisert/}}

\newcommand{\secref}[1]{\hyperref[#1]{section \ref*{#1}}}

\parskip1ex
\parindent0pt
\let\olditemize\itemize
\def\itemize{\olditemize\parskip0pt}

\begin{document}

\title{The \textsf{childdoc} Package}
\hypersetup{pdftitle={The childdoc Package}}
\author{Niklas Beisert\\[2ex]
  Institut f\"ur Theoretische Physik\\
  Eidgen\"ossische Technische Hochschule Z\"urich\\
  Wolfgang-Pauli-Strasse 27, 8093 Z\"urich, Switzerland\\[1ex]
  \href{mailto:nbeisert@itp.phys.ethz.ch}
  {\texttt{nbeisert@itp.phys.ethz.ch}}}
\hypersetup{pdfauthor={Niklas Beisert}}
\hypersetup{pdfsubject={Manual for the LaTeX2e Package childdoc}}
\date{30 December 2018, \textsf{v2.0}}
\maketitle

\begin{abstract}\noindent
\textsf{childdoc} is a \LaTeXe{} package
that enables the direct compilation
of document sections included by |\include|
to individual files.
\end{abstract}

\begingroup
\parskip0ex
\tableofcontents
\endgroup

%%%%%%%%%%%%%%%%%%%%%%%%%%%%%%%%%%%%%%%%%%%%%%%%%%%%%%%%%%%%%%%%%%%%%%%%%%%%%%%%
%%%%%%%%%%%%%%%%%%%%%%%%%%%%%%%%%%%%%%%%%%%%%%%%%%%%%%%%%%%%%%%%%%%%%%%%%%%%%%%%
\section{Introduction}

\LaTeX{} provides a mechanism to structure a large document (such as a book)
into a main file and several child files (containing the chapters)
using the |\include| command.
This mechanism is beneficial for documents
which span hundreds of pages in order to
make the source file(s) more manageable.
Moreover, compilation can be restricted to
selected child files by means of the |\includeonly| command.
The latter feature can be used to reduce the compilation time while editing
(this was significantly more useful in the earlier days of \LaTeX{})
or to generate a smaller document which is easier to navigate.
Another application of |\includeonly| is to generate
documents consisting of selected parts of the complete document.

However, there are a few drawbacks of the plain |\include| mechanism:
\begin{itemize}
\item
The child files cannot be compiled on their own,
they can only be compiled via the main file.
A naive editing environment
(such as a text editor with an option
to have the current file processed by \LaTeX)
may require one to switch to the main file before compiling;
attempting to compile the child file produces errors.
\item
The main file must be modified (each time)
to adjust the |\includeonly| command
to the present needs. This easily leaves the main file in a messy state.
\item
The generated document will always carry the filename
of the main document. This is inconvenient if
several child files are to be compiled and
to be kept for distribution.
\end{itemize}

The present package provides a simple interface
to make child files individually compilable by \LaTeX{}.
Compiling a child file then has the same effect as compiling
the main file with an |\includeonly| command
to select the appropriate child.
Moreover the generated document will carry the name of the child
rather than the main file.
This resolves all three above issues.

This feature is meant to make the editing of books,
thesis documents and lecture notes somewhat more convenient.
However, the package can also be used efficiently for
composing a series of documents (such as exercise sheets)
which are typically distributed individually.
It then assists the author in generating the individual documents
(potentially in different versions)
as well as a document containing the collected series.
Another application is in developing style files
or other kinds of included material
where compilation of the style file could redirect
to a sample or test file.

%%%%%%%%%%%%%%%%%%%%%%%%%%%%%%%%%%%%%%%%%%%%%%%%%%%%%%%%%%%%%%%%%%%%%%%%%%%%%%%%
%%%%%%%%%%%%%%%%%%%%%%%%%%%%%%%%%%%%%%%%%%%%%%%%%%%%%%%%%%%%%%%%%%%%%%%%%%%%%%%%
\section{Usage}

First of all, the package \textsf{childdoc} is \emph{not} a standard
\LaTeXe{} |.sty| style file! Therefore it needs to be invoked in
a non-standard way.

%%%%%%%%%%%%%%%%%%%%%%%%%%%%%%%%%%%%%%%%%%%%%%%%%%%%%%%%%%%%%%%%%%%%%%%%%%%%%%%%
\subsection{Included Files}
\label{sec:include}

%%%%%%%%%%%%%%%%%%%%%%%%%%%%%%%%%%%%%%%%
\DescribeMacro{\childdocmain}
To use the package, add the commands
\begin{center}
\begin{tabular}{l}
|\input{childdoc.def}|\\
|\childdocmain{}|\\
\end{tabular}
\end{center}
at the very top of the main \LaTeX{} file,
in particular \emph{before} the |\documentclass| statement!
The argument of |\childdocmain| should be left empty
(but it must be present).

%%%%%%%%%%%%%%%%%%%%%%%%%%%%%%%%%%%%%%%%
\DescribeMacro{\childdocof}
Furthermore, add the commands
\begin{center}
\begin{tabular}{l}
|\input{childdoc.def}|\\
|\childdocof{|\textit{main}|}|\\
\end{tabular}
\end{center}
at the top of every child file \textit{child}
which is included by |\include{|\textit{child}|}|
from within the main file
(or at least for those files to be compiled individually).
The argument \textit{main} must be the filename of the main file.

There are a couple of
considerations in setting up the main and child documents:

%%%%%%%%%%%%%%%%%%%%%%%%%%%%%%%%%%%%%%%%
\paragraph{Restrictions.}

Please note the following restrictions:
\begin{itemize}
\item
|\childdocmain| must be called with one argument \textit{main}
to ensure compatibility with earlier version of the package.
It must either be empty (|\childdocmain{}|)
or precisely match the filename of the main file in which it is specified.
See \secref{sec:detection} for further information.
\item
The filename \textit{main} must be specified without the |.tex| extension.
\item
The filename \textit{main} is case sensitive
(even in case-insensitive file systems)
due to internal string comparison.
\item
The argument \textit{main} should be fully expanded, it cannot be a macro.
\item
Subdirectories and special characters should be avoided in filenames.
\item
The command |\childdocmain{|\textit{main}|}| must be followed by a whitespace.
It should not be followed immediately by another command
or by a comment mark `|%|'.
This is because the \TeX{} parser reads the token immediately following
the argument of |\childdocmain| and puts it
at the beginning of every child section;
however, a white\-space is ignored.
\end{itemize}

%%%%%%%%%%%%%%%%%%%%%%%%%%%%%%%%%%%%%%%%
\paragraph{Content of Main File.}

It is advisable to place all content in the child files included by |\include|.
Any output contained in the main file will appear in all child documents
unless suppressed manually;
it cannot be suppressed automatically by the |\includeonly| directive
and thus should normally be avoided.
A method to include some content in the main file
by means of conditional processing is described in \secref{sec:conditional}.

%%%%%%%%%%%%%%%%%%%%%%%%%%%%%%%%%%%%%%%%
\paragraph{Page Numbering.}

When only a part of the document is compiled,
the appropriate numbering of pages
(as well as other status parameters)
is determined from the |.aux| files.
The latter contain information from previous passes.
However this information needs to propagate through
all intermediate child documents.
Therefore the page numbering in child documents may well
be inconsistent until the complete document is compiled at least once.

A useful (if unconventional) way to always ensure a consistent
page numbering is to restart the numbering in each child document
and denote the pages by `\textit{child}|.|\textit{page}'
where \textit{child} represents the chapter/section number of the child file.
This can be achieved by the command
|\numberwithin{page}{|\textit{child}|}|
of the \textsf{amsmath} package
where \textit{child} can be |chapter| or |section|
depending on the chosen structuring.
Alternatively, one can modify the macro |\thepage| appropriately
and reset the counter |page| at the start of each child file.

%%%%%%%%%%%%%%%%%%%%%%%%%%%%%%%%%%%%%%%%%%%%%%%%%%%%%%%%%%%%%%%%%%%%%%%%%%%%%%%%
\subsection{Conditional Processing}
\label{sec:conditional}

The package provides a mechanism to compile different versions
of a document. To customise the versions further some conditional processing
can come in handy to distinguish which version is being compiled.
The package provides two macros to describe the compilation context:

%%%%%%%%%%%%%%%%%%%%%%%%%%%%%%%%%%%%%%%%
\DescribeMacro{\ifchilddoc}
The conditional |\ifchilddoc| distinguishes between the compilation of
child documents and the main document:
%
\begin{center}
|\ifchilddoc |\textit{child-code}| |[|\||else |\textit{main-code}]| \||fi|
\end{center}

%%%%%%%%%%%%%%%%%%%%%%%%%%%%%%%%%%%%%%%%
\DescribeMacro{\childdocname}
\DescribeMacro{\childdocjob}
The macro |\childdocname| contains the filename (without extension)
of the main or child file being processed.
Note that |\childdocjob| will always contain the name of the main file.

%%%%%%%%%%%%%%%%%%%%%%%%%%%%%%%%%%%%%%%%
\paragraph{Title Page.}

Conditional processing can be used to include a title or banner page
in the main document when proper precautions are taken.
Importantly, the code in the main file should ensure that the page counter
(as well as other status parameters which are stored in the |.aux| files)
takes the same value after the conditional processing.
Otherwise the page numbers may take divergent values
depending on which part is compiled.

For example, a title page could be declared by:
%
\begin{center}
\begin{tabular}{l}
|\ifchilddoc\||else|\\
|\addtocounter{page}{-1}|\\
\textit{code for title page}\\
|\newpage|\\
|\||fi|
\end{tabular}
\end{center}
%
A banner page for the child documents can be generated by:
%
\begin{center}
\begin{tabular}{l}
|\ifchilddoc|\\
|\addtocounter{page}{-1}|\\
\textit{code for banner page}\\
|\newpage|\\
|\||fi|
\end{tabular}
\end{center}
%
Here one could write a message such as:
\begin{center}
|This is the part \childdocname{} of \childdocjob{}.|
\end{center}

%%%%%%%%%%%%%%%%%%%%%%%%%%%%%%%%%%%%%%%%%%%%%%%%%%%%%%%%%%%%%%%%%%%%%%%%%%%%%%%%
\subsection{Flags}
\label{sec:flags}

The package makes it easy to generate different versions
of the main or child documents.
To this end compilation flags can be defined
and assigned different default values.
They will be particularly useful in conjunction
with the forwarding mechanism described in \secref{sec:forward}.

For example, it may be useful to have a flag |\version|
which can be set to |draft| or |final|.
The document source will contain some conditional code
depending on the value of |\version|.
Suppose further, the flag should default to |final| for the main file
and to |draft| for child files
which is a natural assignment for editing the document.
This is achieved by placing the following code
in the preamble of the main document
(below the |\childdocmain| directive):
%
\begin{center}
\begin{tabular}{l}
|\ifchilddoc|\\
|\providecommand{\version}{draft}|\\
|\||else|\\
|\providecommand{\version}{final}|\\
|\||fi|
\end{tabular}
\end{center}
%
The definition by |\providecommand| makes sure
that previous definitions are not overwritten.
Further statements |\providecommand{\version}{...}|
can thus be added before the above code to override it.

For the main file, one might add a line
(between |\childdocmain| and the above block)
%
\begin{center}
|%\ifchilddoc\||else\providecommand{\version}{draft}\||fi|
\end{center}
%
which can be uncommented to produce a draft version.
Likewise one can add a line to the very top of a child file
(above the |\childdocof{|\textit{main}|}| directive)
%
\begin{center}
|%\providecommand{\version}{final}|
\end{center}
%
which can be uncommented to produce the final version of this child document.

%%%%%%%%%%%%%%%%%%%%%%%%%%%%%%%%%%%%%%%%%%%%%%%%%%%%%%%%%%%%%%%%%%%%%%%%%%%%%%%%
\subsection{Forwarding}
\label{sec:forward}

Different versions of the main or child documents
using compilation flags as described in \secref{sec:flags}
can be (permanently) stored in different files
for convenient compilation, viewing and distribution.
To this end, the package defines a command
to pass on compilation to a different file:

%%%%%%%%%%%%%%%%%%%%%%%%%%%%%%%%%%%%%%%%
\DescribeMacro{\childdocforward}
The command |\childdocforward| redirects processing to
another source file:
%
\begin{center}
\begin{tabular}{l}
|\input{childdoc.def}|\\
|\childdocforward[|\textit{main}|]{|\textit{dest}|}|\\
\end{tabular}
\end{center}
%
The argument \textit{dest} is the destination file
(without extension).
It should be the main file or one of the child files.
Note that further \textsf{childdoc} directives
such as |\childdocof| and |\childdocforward|
in the indicated file will be processed in this form.
The optional argument \textit{main}
passes on directly to the main file \textit{main}
while pretending to compile the child \textit{dest}.
This form behaves as if \textit{dest}
issues |\childdocof{|\textit{main}|}| right away,
and no further \textsf{childdoc} directives will be processed.

%%%%%%%%%%%%%%%%%%%%%%%%%%%%%%%%%%%%%%%%
\DescribeMacro{\...prefix}
In the alternative form |\childdocforwardprefix|,
%
\begin{center}
\begin{tabular}{l}
|\input{childdoc.def}|\\
|\childdocforwardprefix[|\textit{main}|]{|\textit{prefix}|}{|\textit{dest}|}|
\end{tabular}
\end{center}
%
the destination file is determined by a pattern
depending on the current file:
To make this work, the current file must be called
`{\textit{prefix}\hspace{0.2em}\textit{suffix}}'
with \textit{prefix} matching precisely the argument.
Processing is then passed on to the file
`{\textit{dest}\hspace{0.2em}\textit{suffix}}'.
Surely, the same effect is achieved by
directly specifying the
argument `{\textit{dest}\hspace{0.2em}\textit{suffix}}'
in the first form.
However, that requires to set up a different file
for each child. With the alternative form of the command
all these files can have exactly the same content
which simplifies setting them up and maintaining them.

For example, the following file |draft.tex|
with a compilation flag |\version| as described in \secref{sec:flags}
compiles the main document as a draft:
%
\begin{center}
\begin{tabular}{l}
|\def\version{draft}|\\
|\input{childdoc.def}|\\
|\childdocforward{|\textit{main}|}|
\end{tabular}
\end{center}
%
Likewise, the following files |final|\textit{nn}|.tex|
compile the final version of the child document
|child|\textit{nn}|.tex|:
%
\begin{center}
\begin{tabular}{l}
|\def\version{final}|\\
|\input{childdoc.def}|\\
|\childdocforwardprefix{final}{child}|
\end{tabular}
\end{center}
%

Note that when several versions of a main file and/or of each child file
are to be generated, it may be convenient to set up a |Makefile| or
shell script to automatise the process.

%%%%%%%%%%%%%%%%%%%%%%%%%%%%%%%%%%%%%%%%%%%%%%%%%%%%%%%%%%%%%%%%%%%%%%%%%%%%%%%%
\subsection{Command Line Processing}
\label{sec:commandline}

The effect of redirection files can also be achieved by invoking
the \LaTeX{} compiler with a more elaborate command line.
Most conveniently this should be done as part
of a shell script or a |Makefile|.

When using \textsf{childdoc} in the main file, the following
command lines effectively perform a redirection
(note that depending on the shell being used,
backslashes may have to be doubled: `|\|' $\to$ `|\\|'):
%
\begin{center}
|... -jobname "|\textit{target}|" |\\|"|[\textit{flags}]%
|\input{childdoc.def}\childdocforward[|\textit{main}|]{|\textit{dest}|}"|
\end{center}
%
Here \textit{target} is the name of the output file,
\textit{main} is the name of the main file
and \textit{dest} is the name of the main or child file to be processed
(all filenames without extensions).
The optional argument \textit{main} can be omitted
if \textit{main} matches \textit{dest}.
Optionally, compilation \textit{flags} can be defined via |\def| commands.
This command line makes the \TeX{} engine believe
it is compiling the file \textit{target}
whose content is specified as the latter parameter.
The provided code then forwards the processing to
\textit{main} or \textit{dest} as described in \secref{sec:forward}.

%%%%%%%%%%%%%%%%%%%%%%%%%%%%%%%%%%%%%%%%%%%%%%%%%%%%%%%%%%%%%%%%%%%%%%%%%%%%%%%%
\subsection{Include by Input}
\label{sec:input}

Including child documents by |\include| has some restrictions by design.
Most notably, the content of a child document always occupies
its own set of pages; pages cannot be shared between child documents.
Usually, this behaviour makes perfect sense
because each child document contain an essential part of the document.
However, in some situations it may be desirable to compose
a document from a collection of parts
without having mandatory page breaks between then.
For this case, the package
provides a mechanism to include parts
by |\input| which can also be processed individually.
However, by construction this mechanism
requires manual handling of the content to be output.

%%%%%%%%%%%%%%%%%%%%%%%%%%%%%%%%%%%%%%%%
\DescribeMacro{\ifchilddocmanual}
The main file should be prepared as usual, see \secref{sec:include}.
However, the document body must make a distinction
between processing of an individual part and of the main document, e.g.:
%
\begin{center}
\begin{tabular}{l}
|\ifchilddocmanual|\\
|\input{\childdocname}|\\
|\||else|\\
\textit{document body with }|\input{|\textit{part}|}|\\
|\||fi|
\end{tabular}
\end{center}
%
The conditional |\ifchilddocmanual| is true whenever
a part to be included by |\input| is being compiled,
and the name of the part is stored in |\childdocname|.

%%%%%%%%%%%%%%%%%%%%%%%%%%%%%%%%%%%%%%%%
\DescribeMacro{\childdocby}
Each part to be included by |\input| should start with:
%
\begin{center}
\begin{tabular}{l}
|\input{childdoc.def}|\\
|\childdocby{|\textit{main}|}|\\
\end{tabular}
\end{center}
%
The directive |\childdocby| is similar to |\childdocof|
described in \secref{sec:include},
but the subsequent selection of content must be done manually.
To that end, both |\ifchilddoc| and |\ifchilddocmanual|
will be true upon processing of a part,
and the name of the part is stored in |\childdocname|.
Note that |\jobname| will be set to the filename of the current part
so that each part receives an individual |.aux| file
that does not interfere with the |.aux| file(s) of the main document.
This behaviour can be altered by the alternative form
|\childdocby[*]{|\textit{main}|}| (with a non-empty optional argument)
which uses the |.aux| file of the main document
by setting |\jobname| to \textit{main}.

%%%%%%%%%%%%%%%%%%%%%%%%%%%%%%%%%%%%%%%%%%%%%%%%%%%%%%%%%%%%%%%%%%%%%%%%%%%%%%%%
\subsection{Driver Development}
\label{sec:driver}

The \textsf{childdoc} mechanism can also be use for the development
of definition files such as \LaTeX{} styles or classes.
This case differs from the above setup with multiple parts
included by |\include| in that no |\includeonly| should be invoked.
This can be achieved by starting the include file
(before |\ProvidesPackage|) with:
%
\begin{center}
\begin{tabular}{l}
|\input{childdoc.def}|\\
|\childdocforward{|\textit{main}|}|\\
\end{tabular}
\end{center}
%
or alternatively with:
%
\begin{center}
\begin{tabular}{l}
|\input{childdoc.def}|\\
|\childdocby{|\textit{main}|}|\\
\end{tabular}
\end{center}
%
Both forms have slightly different effects as described above.
The main file is prepared as usual, see \secref{sec:include}.

%%%%%%%%%%%%%%%%%%%%%%%%%%%%%%%%%%%%%%%%%%%%%%%%%%%%%%%%%%%%%%%%%%%%%%%%%%%%%%%%
\subsection{Legacy Detection}
\label{sec:detection}

The directive |\childdocmain| in the main file can detect
whether the complete document or merely a child is to be compiled
even without using the directive |\childdocof|.
This method is deprecated because it is less robust
and there is no compelling reason to use it;
it is merely provided for backward compatibility
and it may be removed in future versions.

If the detection mechanism is to be used,
it is mandatory to correctly specify
the filename of the main file as the argument of |\childdocmain|:
%
\begin{center}
\begin{tabular}{l}
|\input{childdoc.def}|\\
|\childdocmain{|\textit{main}|}|\\
\end{tabular}
\end{center}
%
If |\jobname| does not match the argument \textit{main} of |\childdocmain|,
it is assumed that |\jobname| points to the child file to be compiled.
When using |\childdocmain| with the main file specified as argument,
it suffices to start a child file
with just |\input{|\textit{main}|}|
without loading of the package and using |\childdocof|.
If instead all processing is done
with the appropriate \textsf{childdoc} directives,
the argument of \textit{main} of |\childdocmain| can be empty.

An alternative version of the command line processing described
in \secref{sec:commandline} using the detection mechanism reads:
%
\begin{center}
|... -jobname "|\textit{target}|" "|[\textit{flags}]%
[|\def\jobname{|\textit{dest}|}|]|\input{|\textit{main}|}"|
\end{center}

%%%%%%%%%%%%%%%%%%%%%%%%%%%%%%%%%%%%%%%%%%%%%%%%%%%%%%%%%%%%%%%%%%%%%%%%%%%%%%%%
\subsection{Manual Code}
\label{sec:manual}

In case one cannot be certain whether the definitions file |childdoc.def|
is installed on the target \TeX{} distribution
and one prefers not to ship it,
it is conceivable to paste a few relevant commands into the sources.

To that end, drop all statements |\input{childdoc.def}|
and perform the replacements as outlined below.
Instead of |\childdocmain{|\textit{main}|}| add the following code
to the top of the main file:
%
\begin{center}
\begin{tabular}{l}
|\||ifdefined\childdocname\endinput\||fi\newif\ifchilddoc|\\
|\edef\childdocname{\scantokens\expandafter{\jobname\noexpand}}|\\
|\def\childdocmain{|\textit{main}|}\||ifx\childdocmain\childdocname\||else|\\
|\childdoctrue\includeonly{\childdocname}\let\jobname\childdocmain\||fi|\\
\end{tabular}
\end{center}
%
Instead of |\childdocof{|\textit{main}|}| just include the main file
at the top of each child file:
%
\begin{center}
|\input{|\textit{main}|}|
\end{center}
%
A simple redirection |\childdocforward{|\textit{dest}|}| is achieved by:
%
\begin{center}
|\def\jobname{|\textit{dest}|}\input{\jobname}|
\end{center}
%
The redirection with prefix
|\childdocforwardprefix[|\textit{prefix}|]{|\textit{dest}|}|
is accomplished by:
%
\begin{center}
\begin{tabular}{l}
|{\edef\jobname{\scantokens\expandafter{\jobname\noexpand}}|\\
|\def\redirectjob |\textit{prefix}|#1~~~{\gdef\jobname{|\textit{dest}|#1}}|\\
|\expandafter\redirectjob\jobname~~~}\input{\jobname}|
\end{tabular}
\end{center}

In an alternative approach,
child documents can be compiled by a specific command line
without additional code or specific definitions:
%
\begin{center}
|... -jobname "|\textit{target}|" "|[\textit{flags}]%
|\includeonly{|\textit{dest}|}\input{|\textit{main}|}"|
\end{center}
%

%%%%%%%%%%%%%%%%%%%%%%%%%%%%%%%%%%%%%%%%%%%%%%%%%%%%%%%%%%%%%%%%%%%%%%%%%%%%%%%%
%%%%%%%%%%%%%%%%%%%%%%%%%%%%%%%%%%%%%%%%%%%%%%%%%%%%%%%%%%%%%%%%%%%%%%%%%%%%%%%%
\section{Information}

%%%%%%%%%%%%%%%%%%%%%%%%%%%%%%%%%%%%%%%%%%%%%%%%%%%%%%%%%%%%%%%%%%%%%%%%%%%%%%%%
\subsection{Copyright}

Copyright \copyright{} 2017--2018 Niklas Beisert

This work may be distributed and/or modified under the
conditions of the \LaTeX{} Project Public License, either version 1.3
of this license or (at your option) any later version.
The latest version of this license is in
  \url{http://www.latex-project.org/lppl.txt}
and version 1.3 or later is part of all distributions of \LaTeX{}
version 2005/12/01 or later.

This work has the LPPL maintenance status `maintained'.

The Current Maintainer of this work is Niklas Beisert.

This work consists of the files |README.txt|, |childdoc.ins| and |childdoc.dtx|
as well as the derived files |childdoc.def|, |cdocsamp.tex|
with |cdocsch1.tex|, |cdocsch2.tex|, |cdocspt3.tex|, |cdocspt4.tex|,
|cdocsdrf.tex|, |cdocsfn1.tex|, |cdocsfn2.tex|
as well as |childdoc.pdf|.

%%%%%%%%%%%%%%%%%%%%%%%%%%%%%%%%%%%%%%%%%%%%%%%%%%%%%%%%%%%%%%%%%%%%%%%%%%%%%%%%
\subsection{Files and Installation}

The package consists of the files:
%
\begin{center}
\begin{tabular}{ll}
    |README.txt|   & readme file \\
    |childdoc.ins| & installation file \\
    |childdoc.dtx| & source file \\
    |childdoc.def| & definition file \\
    |cdocsamp.tex| & sample main file \\
    |cdocsch1.tex| & sample include file \\
    |cdocsch2.tex| & sample include file \\
    |cdocspt3.tex| & sample part file \\
    |cdocspt4.tex| & sample part file \\
    |cdocsdrf.tex| & sample redirection file \\
    |cdocsfn1.tex| & sample redirection file \\
    |cdocsfn2.tex| & sample redirection file \\
    |childdoc.pdf| & manual
\end{tabular}
\end{center}
%
The distribution consists of the files
|README.txt|, |childdoc.ins| and |childdoc.dtx|.
%
\begin{itemize}
\item
Run (pdf)\LaTeX{} on |childdoc.dtx|
to compile the manual |childdoc.pdf| (this file).
\item
Run \LaTeX{} on |childdoc.ins| to create the definitions file |childdoc.def|
and the sample |cdocsamp.tex| with include files
|cdocsch1.tex|, |cdocsch2.tex|, |cdocspt3.tex|, |cdocspt4.tex|,
|cdocsdrf.tex|, |cdocsfn1.tex|, |cdocsfn2.tex|.
Then copy the file |childdoc.def| to an appropriate directory of your \LaTeX{}
distribution, e.g.\ \textit{texmf-root}|/tex/latex/childdoc|.
\end{itemize}

%%%%%%%%%%%%%%%%%%%%%%%%%%%%%%%%%%%%%%%%%%%%%%%%%%%%%%%%%%%%%%%%%%%%%%%%%%%%%%%%
\subsection{Related CTAN Packages}

There are several other packages which offer a similar functionality:
%
\begin{itemize}
\item
The packages
\href{http://ctan.org/pkg/docmute}{\textsf{docmute}},
\href{http://ctan.org/pkg/includex}{\textsf{includex}} and
\href{http://ctan.org/pkg/standalone}{\textsf{standalone}}
provide commands to include only the document body of
a child file thus allowing both files to be compiled individually.
\item
The packages \href{http://ctan.org/pkg/subdocs}{\textsf{subdocs}}
and \href{http://ctan.org/pkg/subfiles}{\textsf{subfiles}}
provide structures in which the main and child documents can be
encapsulated and allowing them to be compiled individually.
The inclusion mechanism is different from the conventional |\include|.
\item
The package \href{http://ctan.org/pkg/combine}{\textsf{combine}}
is an elaborate solution to combine several documents into one.
\end{itemize}
%
See also the CTAN topic \href{http://ctan.org/topic/subdocs}{\textsf{subdocs}}
for further related packages.
The present package differs from the above solutions in that
a document structure constructed with the conventional |\include| mechanism
just needs two extra commands at the top of every file
such that all constituent files can be compiled individually.

%%%%%%%%%%%%%%%%%%%%%%%%%%%%%%%%%%%%%%%%%%%%%%%%%%%%%%%%%%%%%%%%%%%%%%%%%%%%%%%%
%\subsection{Feature Suggestions}
%
%The following is a list of features which may be useful for future
%versions of this package:
%%
%\begin{itemize}
%\item
%\ldots
%\end{itemize}

%%%%%%%%%%%%%%%%%%%%%%%%%%%%%%%%%%%%%%%%%%%%%%%%%%%%%%%%%%%%%%%%%%%%%%%%%%%%%%%%
\subsection{Revision History}

%%%%%%%%%%%%%%%%%%%%%%%%%%%%%%%%%%%%%%%%
\paragraph{v2.0:} 2018/12/30

\begin{itemize}
\item
immediate forward processing
\item
added |\childdocby| mechanism
\item
manual restructured
\end{itemize}

%%%%%%%%%%%%%%%%%%%%%%%%%%%%%%%%%%%%%%%%
\paragraph{v1.6:} 2018/01/17

\begin{itemize}
\item
application for development of include files
\item
corrections to manual
\end{itemize}

%%%%%%%%%%%%%%%%%%%%%%%%%%%%%%%%%%%%%%%%
\paragraph{v1.5:} 2017/05/21

\begin{itemize}
\item
more complete structuring introduced
\item
|\childdocof| introduced
\item
|\childdoc| renamed to |\childdocmain|
\item
|\childredirect| renamed to |\childdocforward| and |\childdocforwardprefix|
and functionality expanded
\end{itemize}

%%%%%%%%%%%%%%%%%%%%%%%%%%%%%%%%%%%%%%%%
\paragraph{v1.0:} 2017/04/27

\begin{itemize}
\item
manual and install package
\item
first version published on CTAN
\end{itemize}

%%%%%%%%%%%%%%%%%%%%%%%%%%%%%%%%%%%%%%%%
\paragraph{v0.6:} 2017/04/26

\begin{itemize}
\item
redirection mechanism added
\end{itemize}

%%%%%%%%%%%%%%%%%%%%%%%%%%%%%%%%%%%%%%%%
\paragraph{v0.5:} 2017/04/26

\begin{itemize}
\item
functionality in definition file
\end{itemize}


%%%%%%%%%%%%%%%%%%%%%%%%%%%%%%%%%%%%%%%%%%%%%%%%%%%%%%%%%%%%%%%%%%%%%%%%%%%%%%%%
%%%%%%%%%%%%%%%%%%%%%%%%%%%%%%%%%%%%%%%%%%%%%%%%%%%%%%%%%%%%%%%%%%%%%%%%%%%%%%%%
%%%%%%%%%%%%%%%%%%%%%%%%%%%%%%%%%%%%%%%%%%%%%%%%%%%%%%%%%%%%%%%%%%%%%%%%%%%%%%%%
\appendix

\settowidth\MacroIndent{\rmfamily\scriptsize 000\ }

 \DocInput{childdoc.dtx}

\end{document}
%</driver>
% \fi
%
% %%%%%%%%%%%%%%%%%%%%%%%%%%%%%%%%%%%%%%%%%%%%%%%%%%%%%%%%%%%%%%%%%%%%%%%%%%%%%%
% %%%%%%%%%%%%%%%%%%%%%%%%%%%%%%%%%%%%%%%%%%%%%%%%%%%%%%%%%%%%%%%%%%%%%%%%%%%%%%
% \section{Sample}
%\iffalse
%<*samplemain>
%\fi
%
% The following presents a sample document
% with two chapters, two parts, a title page,
% a compile flag as well as three forwarding files to set the flag.
% It consists of eight |.tex| files:
% \begin{center}
% \begin{tabular}{ll}
% |cdocsamp.tex|&main file\\
% |cdocsch1.tex|&include file for chapter 1\\
% |cdocsch2.tex|&include file for chapter 2\\
% |cdocspt3.tex|&include file for part 3\\
% |cdocspt4.tex|&include file for part 4\\
% |cdocsdrf.tex|&forwarding file for main file in draft mode\\
% |cdocsfi1.tex|&forwarding file for final version of chapter 1\\
% |cdocsfi2.tex|&forwarding file for final version of chapter 2\\
% \end{tabular}
% \end{center}
% Each of the eight files can be compiled directly by the \LaTeX{} compiler.
%
% %%%%%%%%%%%%%%%%%%%%%%%%%%%%%%%%%%%%%%
% \paragraph{Main File.}
%
% The main file is called |cdocsamp.tex|.
%
% Load the \textsf{childdoc} definitions and
% declare the filename for the main document:
%    \begin{macrocode}
\input{childdoc.def}
\childdocmain{}
%    \end{macrocode}

% Optional override for |\version| flag:
%    \begin{macrocode}
%%\ifchilddoc\else\providecommand{\version}{draft}\fi
%    \end{macrocode}

% Define the default values for the |\version| flag
% (|final| for the main file and |draft| for childs):
%    \begin{macrocode}
\ifchilddoc
\providecommand{\version}{draft}
\else
\providecommand{\version}{final}
\fi
%    \end{macrocode}

% Load the standard document class:
%    \begin{macrocode}
\documentclass[12pt]{article}
%    \end{macrocode}

% Start the document body:
%    \begin{macrocode}
\begin{document}
%    \end{macrocode}

% Declare a title page.
% Print title, part of document being processed and version flag:
%    \begin{macrocode}
\addtocounter{page}{-1}
\begin{center}
{\LARGE\bfseries{}childdoc example\par}
\vspace{1cm}
\ifchilddoc
\ifchilddocmanual part\else chapter\fi:
`\childdocname' of `\childdocjob'\par
\else
main document: `\childdocjob'\par
\fi
version: \version\par
\end{center}
\newpage
%    \end{macrocode}

% Manually include selected file,
% otherwise process as usual:
%    \begin{macrocode}
\ifchilddocmanual
\section*{part `\childdocname'}
\input{\childdocname}
\else
%    \end{macrocode}

% Include the two chapters:
%    \begin{macrocode}
\include{cdocsch1}
\include{cdocsch2}
%    \end{macrocode}

% Include the two parts unless only chapters should be displayed:
%    \begin{macrocode}
\ifchilddoc\else
\section{part three}
\input{cdocspt3}
\section{part four}
\input{cdocspt4}
\fi
%    \end{macrocode}

% Process as usual until here:
%    \begin{macrocode}
\fi
%    \end{macrocode}

% End of document body:
%    \begin{macrocode}
\end{document}
%    \end{macrocode}
%\iffalse
%</samplemain>
%\fi
%
% %%%%%%%%%%%%%%%%%%%%%%%%%%%%%%%%%%%%%%
% \paragraph{Chapter Include Files.}
%
% The include files are called |cdocsch1.tex| and |cdocsch2.tex|.
%
%\iffalse
%<*samplechap1|samplechap2>
%\fi

% Optional override for |\version| flag:
%    \begin{macrocode}
%%\providecommand{\version}{final}
%    \end{macrocode}

% Include the main document:
%    \begin{macrocode}
\input{childdoc.def}
\childdocof{cdocsamp}
%    \end{macrocode}

%\iffalse
%</samplechap1|samplechap2>
%\fi
%
%\iffalse
%<*samplechap1>
%\fi
% Some text for chapter 1:
%    \begin{macrocode}
\section{one}
some text in chapter one
%    \end{macrocode}

%\iffalse
%</samplechap1>
%\fi
% Some text for chapter 2:
%\iffalse
%<*samplechap2>
%\fi
%    \begin{macrocode}
\section{two}
more text in chapter two
%    \end{macrocode}

%\iffalse
%</samplechap2>
%\fi
%
% %%%%%%%%%%%%%%%%%%%%%%%%%%%%%%%%%%%%%%
% \paragraph{Part Include Files.}
%
% The include files are called |cdocspt3.tex| and |cdocspt4.tex|.
%
%\iffalse
%<*samplepart3|samplepart4>
%\fi

% Optional override for |\version| flag:
%    \begin{macrocode}
%%\providecommand{\version}{final}
%    \end{macrocode}

% Include the main document:
%    \begin{macrocode}
\input{childdoc.def}
\childdocby{cdocsamp}
%    \end{macrocode}

%\iffalse
%</samplepart3|samplepart4>
%\fi
%
%\iffalse
%<*samplepart3>
%\fi
% Some text for part 3:
%    \begin{macrocode}
some text in part three
%    \end{macrocode}

%\iffalse
%</samplepart3>
%\fi
% Some text for part 4:
%\iffalse
%<*samplepart4>
%\fi
%    \begin{macrocode}
more text in part four
%    \end{macrocode}

%\iffalse
%</samplepart4>
%\fi
%
% %%%%%%%%%%%%%%%%%%%%%%%%%%%%%%%%%%%%%%
% \paragraph{Forwarding for a Complete Draft.}
%
% The following forwarding file |cdocsdrf.tex|
% compiles the main document in draft mode:
%\iffalse
%<*sampledraft>
%\fi
%    \begin{macrocode}
\def\version{draft}
\input{childdoc.def}
\childdocforward{cdocsamp}
%    \end{macrocode}

%\iffalse
%</sampledraft>
%\fi
%
% %%%%%%%%%%%%%%%%%%%%%%%%%%%%%%%%%%%%%%
% \paragraph{Forwarding for Final Version of the Chapters.}
%
% The following forwarding files |cdocsfn1.tex| and |cdocsfn2.tex|
% (with identical content)
% compile the final versions of the child documents
% |cdocsch1.tex| and |cdocsch2.tex|, respectively:
%\iffalse
%<*samplefinal>
%\fi
%    \begin{macrocode}
\def\version{final}
\input{childdoc.def}
\childdocforwardprefix[cdocsamp]{cdocsfn}{cdocsch}
%    \end{macrocode}

%\iffalse
%</samplefinal>
%\fi
%
% %%%%%%%%%%%%%%%%%%%%%%%%%%%%%%%%%%%%%%
% \paragraph{Command Line Processing.}
%
% The following three command lines generate the output files
% |cdocscld|, |cdocscl1| and |cdocscl2|
% which should be identical to
% |cdocsdrf|, |cdocsch1| and |cdocsfn2|, respectively:
% \begin{center}
% \begin{tabular}{l}
% |latex -jobname cdocscld \|\\
% |  "\def\version{draft}\input{childdoc.def}\childdocforward{cdocsamp}"|\\
% |latex -jobname cdocscl1 \|\\
% |  "\input{childdoc.def}\childdocforward[cdocsamp]{cdocsch1}"|\\
% |latex -jobname cdocscl2 \|\\
% |  "\def\version{final}\input{childdoc.def}\childdocforward{cdocsch2}"|
% \end{tabular}
% \end{center}
% Note that the trailing backslash on each first line
% merely continues the input to the second line
% (for convenient cut ant paste).
% Furthermore, the command |latex| can be replaced by any
% of its alternative versions such as |pdflatex|.
%
% %%%%%%%%%%%%%%%%%%%%%%%%%%%%%%%%%%%%%%%%%%%%%%%%%%%%%%%%%%%%%%%%%%%%%%%%%%%%%%
% %%%%%%%%%%%%%%%%%%%%%%%%%%%%%%%%%%%%%%%%%%%%%%%%%%%%%%%%%%%%%%%%%%%%%%%%%%%%%%
% \section{Implementation}
%\iffalse
%<*package>
%\fi
%
% This section describes the definitions file |childdoc.def|.

% The definitions cannot be loaded using |\usepackage| or |\RequirePackage|
% which has a mechanism to prevent loading a style file more than once.
% When loading the definitions by means of |\input|
% multiple instances have to be prevented manually:
%\iffalse
%This code needs to be before the `\ProvidesFile' directive
%which is defined at the beginning of this file.
%Therefore it is also placed there and commented out here.
%</package>
%<*discard>
%\fi
%    \begin{macrocode}
\ifdefined\childdocmain\endinput\fi
%    \end{macrocode}
%\iffalse
%</discard>
%<*package>
%\fi
%
% \macro{\ifchilddoc}
% \macro{\ifchilddocmanual}
% The conditional |\ifchilddoc| tells whether a
% child (true) or main (false) document is being compiled.
% The conditional |\ifchilddocmanual| tells whether
% the |\includeonly| mechanism is used (false) or
% the selection of child files must be performed manually (true).
% The definitions initialise to false:
%    \begin{macrocode}
\newif\ifchilddoc
\newif\ifchilddocmanual
%    \end{macrocode}

% \macro{\childdocname}
% \macro{\childdocjob}
% The macro |\childdocname| stores the name of the main document
% to be compiled. The macro |\childdocjob| stores the name of
% the document on which the \LaTeX{} compiler was originally invoked.
% The content of |\jobname| cannot be compared
% to filenames specified in the source due to different catcodes.
% The following code rescans |\jobname|, stores the result
% in |\childdocname| and saves a copy in |\childdocjob|:
%    \begin{macrocode}
\edef\childdocname{\scantokens\expandafter{\jobname\noexpand}}
\let\childdocjob\childdocname
%    \end{macrocode}

% \macro{\childdocdisable}
% The macro |\childdocdisable| prevents the main file
% from being processed more than once.
% At this stage, the main document command |\childdocmain|
% is assumed to be called once again where it should do nothing.
% Any subsequent call to it should prevent
% a secondary processing of the main document
% It overwrites the forwarding commands
% |\childdocof| and |\childdocforward|
% with empty macros to prevent further inclusions of the main document:
%    \begin{macrocode}
\newcommand{\childdocdisable}
{
  \renewcommand{\childdocmain}[1]{\renewcommand{\childdocmain}[1]{\endinput}}
  \renewcommand{\childdocof}[1]{}
  \renewcommand{\childdocby}[2][]{}
  \renewcommand{\childdocforward}[2][]{}
  \renewcommand{\childdocdisable}{}
}
%    \end{macrocode}

% \macro{\childdocmain}
% The macro |\childdocmain| is to be called at the top of the main file
% with nothing or the main filename (without extension) as argument.
% First, it breaks loops.
% If the argument is not empty and does not match |\childdocname|
% (which is set by the first inclusion of |childdoc.def|),
% |\ifchilddoc| is set to true, |\includeonly| is applied to the child file
% and |\jobname| is set to the main file
% (for proper handling of |.aux| files):
%    \begin{macrocode}
\newcommand{\childdocmain}[1]
{
  \childdocdisable\childdocmain{}
  \if?#1?\else
    \begingroup
      \def\childdoctmp{#1}
      \ifx\childdoctmp\childdocname
        \def\childdoctmp{}
      \else
        \def\childdoctmp
        {
          \childdoctrue
          \includeonly{\childdocname}
          \def\childdocjob{#1}
          \def\jobname{#1}
        }
      \fi
      \expandafter
    \endgroup
    \childdoctmp
  \fi
}
%    \end{macrocode}

% \macro{\childdocof}
% The command |\childdocof| redirects
% compilation to the main file |#1|.
%    \begin{macrocode}
\newcommand{\childdocof}[1]
{
  \childdocdisable
  \childdoctrue
  \includeonly{\childdocname}
  \def\jobname{#1}
  \def\childdocjob{#1}
  \input{#1}
}
%    \end{macrocode}

% \macro{\childdocby}
% The command |\childdocby| ....
%    \begin{macrocode}
\newcommand{\childdocby}[2][]
{
  \childdocdisable
  \childdoctrue
  \childdocmanualtrue
  \if?#1?\else
    \def\jobname{#2}
  \fi
  \def\childdocjob{#2}
  \input{#2}
  \endinput
}
%    \end{macrocode}

% \macro{\childdocforward}
% The command |\childdocforward| redirects
% compilation to the main file or
% (if the optional argument is given) a child file.
% Parameters are set as if the main file
% or a child file starting with |\childdocof| was compiled.
% Then compilation is handed over to the main file:
%    \begin{macrocode}
\newcommand{\childdocforward}[2][]
{
  \begingroup
    \if?#1?
      \def\childdoctmp
      {
        \def\childdocname{#2}
        \def\childdocjob{#2}
        \def\jobname{#2}
        \input{#2}
        \endinput
      }
    \else
      \def\childdoctmp
      {
        \childdocdisable
        \def\childdocname{#2}
        \childdoctrue
        \includeonly{#2}
        \def\childdocjob{#1}
        \def\jobname{#1}
        \input{#1}
        \endinput
      }
    \fi
    \expandafter
  \endgroup
  \childdoctmp
}
%    \end{macrocode}

% \macro{\childdocforwardprefix}
% The command |\childdocforwardprefix| redirects
% compilation to the main or a child file by means of a pattern.
% The prefix |#1| in the current filename is replaced by |#2|
% and the suffix of the current filename is kept
% (it is assumed that the filename does not contain the substring `|~~~|'
% which is used as a delimiter).
% Compilation is handed over to the new file by |\childdocforward|:
%    \begin{macrocode}
\newcommand{\childdocforwardprefix}[3][]
{
  \begingroup
    \def\childdocextract #2##1~~~{\def\childdoctmp{\childdocforward[#1]{#3##1}}}
    \expandafter\childdocextract\childdocname~~~
    \expandafter
  \endgroup
  \childdoctmp
}
%    \end{macrocode}

% \macro{\childdoc}
% The deprecated macro |\childdoc| is a legacy version of |\childdocmain|:
%    \begin{macrocode}
\newcommand{\childdoc}{\childdocmain}
%    \end{macrocode}

% \macro{\childdocredirect}
% The deprecated macro |\childdocredirect| is a legacy version
% of |\childdocforward| and |\childdocforwardprefix|:
%    \begin{macrocode}
\newcommand{\childdocredirect}[2][]
{
  \begingroup
    \if?#1?
      \def\childdoctmp{\childdocforward{#2}}
    \else
      \def\childdoctmp{\childdocforwardprefix{#1}{#2}}
    \fi
    \expandafter
  \endgroup
  \childdoctmp
}
%    \end{macrocode}

%\iffalse
%</package>
%\fi
%
\endinput
|
and perform the replacements as outlined below.
Instead of |\childdocmain{|\textit{main}|}| add the following code
to the top of the main file:
%
\begin{center}
\begin{tabular}{l}
|\||ifdefined\childdocname\endinput\||fi\newif\ifchilddoc|\\
|\edef\childdocname{\scantokens\expandafter{\jobname\noexpand}}|\\
|\def\childdocmain{|\textit{main}|}\||ifx\childdocmain\childdocname\||else|\\
|\childdoctrue\includeonly{\childdocname}\let\jobname\childdocmain\||fi|\\
\end{tabular}
\end{center}
%
Instead of |\childdocof{|\textit{main}|}| just include the main file
at the top of each child file:
%
\begin{center}
|\input{|\textit{main}|}|
\end{center}
%
A simple redirection |\childdocforward{|\textit{dest}|}| is achieved by:
%
\begin{center}
|\def\jobname{|\textit{dest}|}\input{\jobname}|
\end{center}
%
The redirection with prefix
|\childdocforwardprefix[|\textit{prefix}|]{|\textit{dest}|}|
is accomplished by:
%
\begin{center}
\begin{tabular}{l}
|{\edef\jobname{\scantokens\expandafter{\jobname\noexpand}}|\\
|\def\redirectjob |\textit{prefix}|#1~~~{\gdef\jobname{|\textit{dest}|#1}}|\\
|\expandafter\redirectjob\jobname~~~}\input{\jobname}|
\end{tabular}
\end{center}

In an alternative approach,
child documents can be compiled by a specific command line
without additional code or specific definitions:
%
\begin{center}
|... -jobname "|\textit{target}|" "|[\textit{flags}]%
|\includeonly{|\textit{dest}|}\input{|\textit{main}|}"|
\end{center}
%

%%%%%%%%%%%%%%%%%%%%%%%%%%%%%%%%%%%%%%%%%%%%%%%%%%%%%%%%%%%%%%%%%%%%%%%%%%%%%%%%
%%%%%%%%%%%%%%%%%%%%%%%%%%%%%%%%%%%%%%%%%%%%%%%%%%%%%%%%%%%%%%%%%%%%%%%%%%%%%%%%
\section{Information}

%%%%%%%%%%%%%%%%%%%%%%%%%%%%%%%%%%%%%%%%%%%%%%%%%%%%%%%%%%%%%%%%%%%%%%%%%%%%%%%%
\subsection{Copyright}

Copyright \copyright{} 2017--2018 Niklas Beisert

This work may be distributed and/or modified under the
conditions of the \LaTeX{} Project Public License, either version 1.3
of this license or (at your option) any later version.
The latest version of this license is in
  \url{http://www.latex-project.org/lppl.txt}
and version 1.3 or later is part of all distributions of \LaTeX{}
version 2005/12/01 or later.

This work has the LPPL maintenance status `maintained'.

The Current Maintainer of this work is Niklas Beisert.

This work consists of the files |README.txt|, |childdoc.ins| and |childdoc.dtx|
as well as the derived files |childdoc.def|, |cdocsamp.tex|
with |cdocsch1.tex|, |cdocsch2.tex|, |cdocspt3.tex|, |cdocspt4.tex|,
|cdocsdrf.tex|, |cdocsfn1.tex|, |cdocsfn2.tex|
as well as |childdoc.pdf|.

%%%%%%%%%%%%%%%%%%%%%%%%%%%%%%%%%%%%%%%%%%%%%%%%%%%%%%%%%%%%%%%%%%%%%%%%%%%%%%%%
\subsection{Files and Installation}

The package consists of the files:
%
\begin{center}
\begin{tabular}{ll}
    |README.txt|   & readme file \\
    |childdoc.ins| & installation file \\
    |childdoc.dtx| & source file \\
    |childdoc.def| & definition file \\
    |cdocsamp.tex| & sample main file \\
    |cdocsch1.tex| & sample include file \\
    |cdocsch2.tex| & sample include file \\
    |cdocspt3.tex| & sample part file \\
    |cdocspt4.tex| & sample part file \\
    |cdocsdrf.tex| & sample redirection file \\
    |cdocsfn1.tex| & sample redirection file \\
    |cdocsfn2.tex| & sample redirection file \\
    |childdoc.pdf| & manual
\end{tabular}
\end{center}
%
The distribution consists of the files
|README.txt|, |childdoc.ins| and |childdoc.dtx|.
%
\begin{itemize}
\item
Run (pdf)\LaTeX{} on |childdoc.dtx|
to compile the manual |childdoc.pdf| (this file).
\item
Run \LaTeX{} on |childdoc.ins| to create the definitions file |childdoc.def|
and the sample |cdocsamp.tex| with include files
|cdocsch1.tex|, |cdocsch2.tex|, |cdocspt3.tex|, |cdocspt4.tex|,
|cdocsdrf.tex|, |cdocsfn1.tex|, |cdocsfn2.tex|.
Then copy the file |childdoc.def| to an appropriate directory of your \LaTeX{}
distribution, e.g.\ \textit{texmf-root}|/tex/latex/childdoc|.
\end{itemize}

%%%%%%%%%%%%%%%%%%%%%%%%%%%%%%%%%%%%%%%%%%%%%%%%%%%%%%%%%%%%%%%%%%%%%%%%%%%%%%%%
\subsection{Related CTAN Packages}

There are several other packages which offer a similar functionality:
%
\begin{itemize}
\item
The packages
\href{http://ctan.org/pkg/docmute}{\textsf{docmute}},
\href{http://ctan.org/pkg/includex}{\textsf{includex}} and
\href{http://ctan.org/pkg/standalone}{\textsf{standalone}}
provide commands to include only the document body of
a child file thus allowing both files to be compiled individually.
\item
The packages \href{http://ctan.org/pkg/subdocs}{\textsf{subdocs}}
and \href{http://ctan.org/pkg/subfiles}{\textsf{subfiles}}
provide structures in which the main and child documents can be
encapsulated and allowing them to be compiled individually.
The inclusion mechanism is different from the conventional |\include|.
\item
The package \href{http://ctan.org/pkg/combine}{\textsf{combine}}
is an elaborate solution to combine several documents into one.
\end{itemize}
%
See also the CTAN topic \href{http://ctan.org/topic/subdocs}{\textsf{subdocs}}
for further related packages.
The present package differs from the above solutions in that
a document structure constructed with the conventional |\include| mechanism
just needs two extra commands at the top of every file
such that all constituent files can be compiled individually.

%%%%%%%%%%%%%%%%%%%%%%%%%%%%%%%%%%%%%%%%%%%%%%%%%%%%%%%%%%%%%%%%%%%%%%%%%%%%%%%%
%\subsection{Feature Suggestions}
%
%The following is a list of features which may be useful for future
%versions of this package:
%%
%\begin{itemize}
%\item
%\ldots
%\end{itemize}

%%%%%%%%%%%%%%%%%%%%%%%%%%%%%%%%%%%%%%%%%%%%%%%%%%%%%%%%%%%%%%%%%%%%%%%%%%%%%%%%
\subsection{Revision History}

%%%%%%%%%%%%%%%%%%%%%%%%%%%%%%%%%%%%%%%%
\paragraph{v2.0:} 2018/12/30

\begin{itemize}
\item
immediate forward processing
\item
added |\childdocby| mechanism
\item
manual restructured
\end{itemize}

%%%%%%%%%%%%%%%%%%%%%%%%%%%%%%%%%%%%%%%%
\paragraph{v1.6:} 2018/01/17

\begin{itemize}
\item
application for development of include files
\item
corrections to manual
\end{itemize}

%%%%%%%%%%%%%%%%%%%%%%%%%%%%%%%%%%%%%%%%
\paragraph{v1.5:} 2017/05/21

\begin{itemize}
\item
more complete structuring introduced
\item
|\childdocof| introduced
\item
|\childdoc| renamed to |\childdocmain|
\item
|\childredirect| renamed to |\childdocforward| and |\childdocforwardprefix|
and functionality expanded
\end{itemize}

%%%%%%%%%%%%%%%%%%%%%%%%%%%%%%%%%%%%%%%%
\paragraph{v1.0:} 2017/04/27

\begin{itemize}
\item
manual and install package
\item
first version published on CTAN
\end{itemize}

%%%%%%%%%%%%%%%%%%%%%%%%%%%%%%%%%%%%%%%%
\paragraph{v0.6:} 2017/04/26

\begin{itemize}
\item
redirection mechanism added
\end{itemize}

%%%%%%%%%%%%%%%%%%%%%%%%%%%%%%%%%%%%%%%%
\paragraph{v0.5:} 2017/04/26

\begin{itemize}
\item
functionality in definition file
\end{itemize}


%%%%%%%%%%%%%%%%%%%%%%%%%%%%%%%%%%%%%%%%%%%%%%%%%%%%%%%%%%%%%%%%%%%%%%%%%%%%%%%%
%%%%%%%%%%%%%%%%%%%%%%%%%%%%%%%%%%%%%%%%%%%%%%%%%%%%%%%%%%%%%%%%%%%%%%%%%%%%%%%%
%%%%%%%%%%%%%%%%%%%%%%%%%%%%%%%%%%%%%%%%%%%%%%%%%%%%%%%%%%%%%%%%%%%%%%%%%%%%%%%%
\appendix

\settowidth\MacroIndent{\rmfamily\scriptsize 000\ }

 \DocInput{childdoc.dtx}

\end{document}
%</driver>
% \fi
%
% %%%%%%%%%%%%%%%%%%%%%%%%%%%%%%%%%%%%%%%%%%%%%%%%%%%%%%%%%%%%%%%%%%%%%%%%%%%%%%
% %%%%%%%%%%%%%%%%%%%%%%%%%%%%%%%%%%%%%%%%%%%%%%%%%%%%%%%%%%%%%%%%%%%%%%%%%%%%%%
% \section{Sample}
%\iffalse
%<*samplemain>
%\fi
%
% The following presents a sample document
% with two chapters, two parts, a title page,
% a compile flag as well as three forwarding files to set the flag.
% It consists of eight |.tex| files:
% \begin{center}
% \begin{tabular}{ll}
% |cdocsamp.tex|&main file\\
% |cdocsch1.tex|&include file for chapter 1\\
% |cdocsch2.tex|&include file for chapter 2\\
% |cdocspt3.tex|&include file for part 3\\
% |cdocspt4.tex|&include file for part 4\\
% |cdocsdrf.tex|&forwarding file for main file in draft mode\\
% |cdocsfi1.tex|&forwarding file for final version of chapter 1\\
% |cdocsfi2.tex|&forwarding file for final version of chapter 2\\
% \end{tabular}
% \end{center}
% Each of the eight files can be compiled directly by the \LaTeX{} compiler.
%
% %%%%%%%%%%%%%%%%%%%%%%%%%%%%%%%%%%%%%%
% \paragraph{Main File.}
%
% The main file is called |cdocsamp.tex|.
%
% Load the \textsf{childdoc} definitions and
% declare the filename for the main document:
%    \begin{macrocode}
% \iffalse
%
% childdoc.dtx Copyright (C) 2017-2018 Niklas Beisert
%
% This work may be distributed and/or modified under the
% conditions of the LaTeX Project Public License, either version 1.3
% of this license or (at your option) any later version.
% The latest version of this license is in
%   http://www.latex-project.org/lppl.txt
% and version 1.3 or later is part of all distributions of LaTeX
% version 2005/12/01 or later.
%
% This work has the LPPL maintenance status `maintained'.
%
% The Current Maintainer of this work is Niklas Beisert.
%
% This work consists of the files childdoc.dtx and childdoc.ins
% and the derived files childdoc.def and cdocsamp.tex with
% cdocsch1.tex, cdocsch2.tex, cdocsdrf.tex, cdocsfn1.tex, cdocsfn2.tex.
%
%<package>\ifdefined\childdocmain\endinput\fi
%<package>\ProvidesFile{childdoc.def}[2018/12/30 v2.0 child document driver]
%<samplemain>\ProvidesFile{cdocsamp.tex}[2018/12/30 v2.0 sample for childdoc]
%<*driver>
%\ProvidesFile{childdoc.drv}[2018/12/30 v2.0 childdoc reference manual file]
\PassOptionsToClass{10pt,a4paper}{article}
\documentclass{ltxdoc}

\usepackage[margin=35mm]{geometry}
\usepackage{hyperref}
\usepackage{hyperxmp}
\usepackage[usenames]{color}

\hypersetup{colorlinks=true}
\hypersetup{pdfstartview=FitH}
\hypersetup{pdfpagemode=UseNone}
\hypersetup{pdfsource={}}
\hypersetup{pdflang={en-UK}}
\hypersetup{pdfcopyright={Copyright 2017-2018 Niklas Beisert.
  This work may be distributed and/or modified under the
  conditions of the LaTeX Project Public License, either version 1.3
  of this license or (at your option) any later version.}}
\hypersetup{pdflicenseurl={http://www.latex-project.org/lppl.txt}}
\hypersetup{pdfcontactaddress={ETH Zurich, ITP, HIT K,
  Wolfgang-Pauli-Strasse 27}}
\hypersetup{pdfcontactpostcode={8093}}
\hypersetup{pdfcontactcity={Zurich}}
\hypersetup{pdfcontactcountry={Switzerland}}
\hypersetup{pdfcontactemail={nbeisert@itp.phys.ethz.ch}}
\hypersetup{pdfcontacturl={http://people.phys.ethz.ch/\xmptilde nbeisert/}}

\newcommand{\secref}[1]{\hyperref[#1]{section \ref*{#1}}}

\parskip1ex
\parindent0pt
\let\olditemize\itemize
\def\itemize{\olditemize\parskip0pt}

\begin{document}

\title{The \textsf{childdoc} Package}
\hypersetup{pdftitle={The childdoc Package}}
\author{Niklas Beisert\\[2ex]
  Institut f\"ur Theoretische Physik\\
  Eidgen\"ossische Technische Hochschule Z\"urich\\
  Wolfgang-Pauli-Strasse 27, 8093 Z\"urich, Switzerland\\[1ex]
  \href{mailto:nbeisert@itp.phys.ethz.ch}
  {\texttt{nbeisert@itp.phys.ethz.ch}}}
\hypersetup{pdfauthor={Niklas Beisert}}
\hypersetup{pdfsubject={Manual for the LaTeX2e Package childdoc}}
\date{30 December 2018, \textsf{v2.0}}
\maketitle

\begin{abstract}\noindent
\textsf{childdoc} is a \LaTeXe{} package
that enables the direct compilation
of document sections included by |\include|
to individual files.
\end{abstract}

\begingroup
\parskip0ex
\tableofcontents
\endgroup

%%%%%%%%%%%%%%%%%%%%%%%%%%%%%%%%%%%%%%%%%%%%%%%%%%%%%%%%%%%%%%%%%%%%%%%%%%%%%%%%
%%%%%%%%%%%%%%%%%%%%%%%%%%%%%%%%%%%%%%%%%%%%%%%%%%%%%%%%%%%%%%%%%%%%%%%%%%%%%%%%
\section{Introduction}

\LaTeX{} provides a mechanism to structure a large document (such as a book)
into a main file and several child files (containing the chapters)
using the |\include| command.
This mechanism is beneficial for documents
which span hundreds of pages in order to
make the source file(s) more manageable.
Moreover, compilation can be restricted to
selected child files by means of the |\includeonly| command.
The latter feature can be used to reduce the compilation time while editing
(this was significantly more useful in the earlier days of \LaTeX{})
or to generate a smaller document which is easier to navigate.
Another application of |\includeonly| is to generate
documents consisting of selected parts of the complete document.

However, there are a few drawbacks of the plain |\include| mechanism:
\begin{itemize}
\item
The child files cannot be compiled on their own,
they can only be compiled via the main file.
A naive editing environment
(such as a text editor with an option
to have the current file processed by \LaTeX)
may require one to switch to the main file before compiling;
attempting to compile the child file produces errors.
\item
The main file must be modified (each time)
to adjust the |\includeonly| command
to the present needs. This easily leaves the main file in a messy state.
\item
The generated document will always carry the filename
of the main document. This is inconvenient if
several child files are to be compiled and
to be kept for distribution.
\end{itemize}

The present package provides a simple interface
to make child files individually compilable by \LaTeX{}.
Compiling a child file then has the same effect as compiling
the main file with an |\includeonly| command
to select the appropriate child.
Moreover the generated document will carry the name of the child
rather than the main file.
This resolves all three above issues.

This feature is meant to make the editing of books,
thesis documents and lecture notes somewhat more convenient.
However, the package can also be used efficiently for
composing a series of documents (such as exercise sheets)
which are typically distributed individually.
It then assists the author in generating the individual documents
(potentially in different versions)
as well as a document containing the collected series.
Another application is in developing style files
or other kinds of included material
where compilation of the style file could redirect
to a sample or test file.

%%%%%%%%%%%%%%%%%%%%%%%%%%%%%%%%%%%%%%%%%%%%%%%%%%%%%%%%%%%%%%%%%%%%%%%%%%%%%%%%
%%%%%%%%%%%%%%%%%%%%%%%%%%%%%%%%%%%%%%%%%%%%%%%%%%%%%%%%%%%%%%%%%%%%%%%%%%%%%%%%
\section{Usage}

First of all, the package \textsf{childdoc} is \emph{not} a standard
\LaTeXe{} |.sty| style file! Therefore it needs to be invoked in
a non-standard way.

%%%%%%%%%%%%%%%%%%%%%%%%%%%%%%%%%%%%%%%%%%%%%%%%%%%%%%%%%%%%%%%%%%%%%%%%%%%%%%%%
\subsection{Included Files}
\label{sec:include}

%%%%%%%%%%%%%%%%%%%%%%%%%%%%%%%%%%%%%%%%
\DescribeMacro{\childdocmain}
To use the package, add the commands
\begin{center}
\begin{tabular}{l}
|\input{childdoc.def}|\\
|\childdocmain{}|\\
\end{tabular}
\end{center}
at the very top of the main \LaTeX{} file,
in particular \emph{before} the |\documentclass| statement!
The argument of |\childdocmain| should be left empty
(but it must be present).

%%%%%%%%%%%%%%%%%%%%%%%%%%%%%%%%%%%%%%%%
\DescribeMacro{\childdocof}
Furthermore, add the commands
\begin{center}
\begin{tabular}{l}
|\input{childdoc.def}|\\
|\childdocof{|\textit{main}|}|\\
\end{tabular}
\end{center}
at the top of every child file \textit{child}
which is included by |\include{|\textit{child}|}|
from within the main file
(or at least for those files to be compiled individually).
The argument \textit{main} must be the filename of the main file.

There are a couple of
considerations in setting up the main and child documents:

%%%%%%%%%%%%%%%%%%%%%%%%%%%%%%%%%%%%%%%%
\paragraph{Restrictions.}

Please note the following restrictions:
\begin{itemize}
\item
|\childdocmain| must be called with one argument \textit{main}
to ensure compatibility with earlier version of the package.
It must either be empty (|\childdocmain{}|)
or precisely match the filename of the main file in which it is specified.
See \secref{sec:detection} for further information.
\item
The filename \textit{main} must be specified without the |.tex| extension.
\item
The filename \textit{main} is case sensitive
(even in case-insensitive file systems)
due to internal string comparison.
\item
The argument \textit{main} should be fully expanded, it cannot be a macro.
\item
Subdirectories and special characters should be avoided in filenames.
\item
The command |\childdocmain{|\textit{main}|}| must be followed by a whitespace.
It should not be followed immediately by another command
or by a comment mark `|%|'.
This is because the \TeX{} parser reads the token immediately following
the argument of |\childdocmain| and puts it
at the beginning of every child section;
however, a white\-space is ignored.
\end{itemize}

%%%%%%%%%%%%%%%%%%%%%%%%%%%%%%%%%%%%%%%%
\paragraph{Content of Main File.}

It is advisable to place all content in the child files included by |\include|.
Any output contained in the main file will appear in all child documents
unless suppressed manually;
it cannot be suppressed automatically by the |\includeonly| directive
and thus should normally be avoided.
A method to include some content in the main file
by means of conditional processing is described in \secref{sec:conditional}.

%%%%%%%%%%%%%%%%%%%%%%%%%%%%%%%%%%%%%%%%
\paragraph{Page Numbering.}

When only a part of the document is compiled,
the appropriate numbering of pages
(as well as other status parameters)
is determined from the |.aux| files.
The latter contain information from previous passes.
However this information needs to propagate through
all intermediate child documents.
Therefore the page numbering in child documents may well
be inconsistent until the complete document is compiled at least once.

A useful (if unconventional) way to always ensure a consistent
page numbering is to restart the numbering in each child document
and denote the pages by `\textit{child}|.|\textit{page}'
where \textit{child} represents the chapter/section number of the child file.
This can be achieved by the command
|\numberwithin{page}{|\textit{child}|}|
of the \textsf{amsmath} package
where \textit{child} can be |chapter| or |section|
depending on the chosen structuring.
Alternatively, one can modify the macro |\thepage| appropriately
and reset the counter |page| at the start of each child file.

%%%%%%%%%%%%%%%%%%%%%%%%%%%%%%%%%%%%%%%%%%%%%%%%%%%%%%%%%%%%%%%%%%%%%%%%%%%%%%%%
\subsection{Conditional Processing}
\label{sec:conditional}

The package provides a mechanism to compile different versions
of a document. To customise the versions further some conditional processing
can come in handy to distinguish which version is being compiled.
The package provides two macros to describe the compilation context:

%%%%%%%%%%%%%%%%%%%%%%%%%%%%%%%%%%%%%%%%
\DescribeMacro{\ifchilddoc}
The conditional |\ifchilddoc| distinguishes between the compilation of
child documents and the main document:
%
\begin{center}
|\ifchilddoc |\textit{child-code}| |[|\||else |\textit{main-code}]| \||fi|
\end{center}

%%%%%%%%%%%%%%%%%%%%%%%%%%%%%%%%%%%%%%%%
\DescribeMacro{\childdocname}
\DescribeMacro{\childdocjob}
The macro |\childdocname| contains the filename (without extension)
of the main or child file being processed.
Note that |\childdocjob| will always contain the name of the main file.

%%%%%%%%%%%%%%%%%%%%%%%%%%%%%%%%%%%%%%%%
\paragraph{Title Page.}

Conditional processing can be used to include a title or banner page
in the main document when proper precautions are taken.
Importantly, the code in the main file should ensure that the page counter
(as well as other status parameters which are stored in the |.aux| files)
takes the same value after the conditional processing.
Otherwise the page numbers may take divergent values
depending on which part is compiled.

For example, a title page could be declared by:
%
\begin{center}
\begin{tabular}{l}
|\ifchilddoc\||else|\\
|\addtocounter{page}{-1}|\\
\textit{code for title page}\\
|\newpage|\\
|\||fi|
\end{tabular}
\end{center}
%
A banner page for the child documents can be generated by:
%
\begin{center}
\begin{tabular}{l}
|\ifchilddoc|\\
|\addtocounter{page}{-1}|\\
\textit{code for banner page}\\
|\newpage|\\
|\||fi|
\end{tabular}
\end{center}
%
Here one could write a message such as:
\begin{center}
|This is the part \childdocname{} of \childdocjob{}.|
\end{center}

%%%%%%%%%%%%%%%%%%%%%%%%%%%%%%%%%%%%%%%%%%%%%%%%%%%%%%%%%%%%%%%%%%%%%%%%%%%%%%%%
\subsection{Flags}
\label{sec:flags}

The package makes it easy to generate different versions
of the main or child documents.
To this end compilation flags can be defined
and assigned different default values.
They will be particularly useful in conjunction
with the forwarding mechanism described in \secref{sec:forward}.

For example, it may be useful to have a flag |\version|
which can be set to |draft| or |final|.
The document source will contain some conditional code
depending on the value of |\version|.
Suppose further, the flag should default to |final| for the main file
and to |draft| for child files
which is a natural assignment for editing the document.
This is achieved by placing the following code
in the preamble of the main document
(below the |\childdocmain| directive):
%
\begin{center}
\begin{tabular}{l}
|\ifchilddoc|\\
|\providecommand{\version}{draft}|\\
|\||else|\\
|\providecommand{\version}{final}|\\
|\||fi|
\end{tabular}
\end{center}
%
The definition by |\providecommand| makes sure
that previous definitions are not overwritten.
Further statements |\providecommand{\version}{...}|
can thus be added before the above code to override it.

For the main file, one might add a line
(between |\childdocmain| and the above block)
%
\begin{center}
|%\ifchilddoc\||else\providecommand{\version}{draft}\||fi|
\end{center}
%
which can be uncommented to produce a draft version.
Likewise one can add a line to the very top of a child file
(above the |\childdocof{|\textit{main}|}| directive)
%
\begin{center}
|%\providecommand{\version}{final}|
\end{center}
%
which can be uncommented to produce the final version of this child document.

%%%%%%%%%%%%%%%%%%%%%%%%%%%%%%%%%%%%%%%%%%%%%%%%%%%%%%%%%%%%%%%%%%%%%%%%%%%%%%%%
\subsection{Forwarding}
\label{sec:forward}

Different versions of the main or child documents
using compilation flags as described in \secref{sec:flags}
can be (permanently) stored in different files
for convenient compilation, viewing and distribution.
To this end, the package defines a command
to pass on compilation to a different file:

%%%%%%%%%%%%%%%%%%%%%%%%%%%%%%%%%%%%%%%%
\DescribeMacro{\childdocforward}
The command |\childdocforward| redirects processing to
another source file:
%
\begin{center}
\begin{tabular}{l}
|\input{childdoc.def}|\\
|\childdocforward[|\textit{main}|]{|\textit{dest}|}|\\
\end{tabular}
\end{center}
%
The argument \textit{dest} is the destination file
(without extension).
It should be the main file or one of the child files.
Note that further \textsf{childdoc} directives
such as |\childdocof| and |\childdocforward|
in the indicated file will be processed in this form.
The optional argument \textit{main}
passes on directly to the main file \textit{main}
while pretending to compile the child \textit{dest}.
This form behaves as if \textit{dest}
issues |\childdocof{|\textit{main}|}| right away,
and no further \textsf{childdoc} directives will be processed.

%%%%%%%%%%%%%%%%%%%%%%%%%%%%%%%%%%%%%%%%
\DescribeMacro{\...prefix}
In the alternative form |\childdocforwardprefix|,
%
\begin{center}
\begin{tabular}{l}
|\input{childdoc.def}|\\
|\childdocforwardprefix[|\textit{main}|]{|\textit{prefix}|}{|\textit{dest}|}|
\end{tabular}
\end{center}
%
the destination file is determined by a pattern
depending on the current file:
To make this work, the current file must be called
`{\textit{prefix}\hspace{0.2em}\textit{suffix}}'
with \textit{prefix} matching precisely the argument.
Processing is then passed on to the file
`{\textit{dest}\hspace{0.2em}\textit{suffix}}'.
Surely, the same effect is achieved by
directly specifying the
argument `{\textit{dest}\hspace{0.2em}\textit{suffix}}'
in the first form.
However, that requires to set up a different file
for each child. With the alternative form of the command
all these files can have exactly the same content
which simplifies setting them up and maintaining them.

For example, the following file |draft.tex|
with a compilation flag |\version| as described in \secref{sec:flags}
compiles the main document as a draft:
%
\begin{center}
\begin{tabular}{l}
|\def\version{draft}|\\
|\input{childdoc.def}|\\
|\childdocforward{|\textit{main}|}|
\end{tabular}
\end{center}
%
Likewise, the following files |final|\textit{nn}|.tex|
compile the final version of the child document
|child|\textit{nn}|.tex|:
%
\begin{center}
\begin{tabular}{l}
|\def\version{final}|\\
|\input{childdoc.def}|\\
|\childdocforwardprefix{final}{child}|
\end{tabular}
\end{center}
%

Note that when several versions of a main file and/or of each child file
are to be generated, it may be convenient to set up a |Makefile| or
shell script to automatise the process.

%%%%%%%%%%%%%%%%%%%%%%%%%%%%%%%%%%%%%%%%%%%%%%%%%%%%%%%%%%%%%%%%%%%%%%%%%%%%%%%%
\subsection{Command Line Processing}
\label{sec:commandline}

The effect of redirection files can also be achieved by invoking
the \LaTeX{} compiler with a more elaborate command line.
Most conveniently this should be done as part
of a shell script or a |Makefile|.

When using \textsf{childdoc} in the main file, the following
command lines effectively perform a redirection
(note that depending on the shell being used,
backslashes may have to be doubled: `|\|' $\to$ `|\\|'):
%
\begin{center}
|... -jobname "|\textit{target}|" |\\|"|[\textit{flags}]%
|\input{childdoc.def}\childdocforward[|\textit{main}|]{|\textit{dest}|}"|
\end{center}
%
Here \textit{target} is the name of the output file,
\textit{main} is the name of the main file
and \textit{dest} is the name of the main or child file to be processed
(all filenames without extensions).
The optional argument \textit{main} can be omitted
if \textit{main} matches \textit{dest}.
Optionally, compilation \textit{flags} can be defined via |\def| commands.
This command line makes the \TeX{} engine believe
it is compiling the file \textit{target}
whose content is specified as the latter parameter.
The provided code then forwards the processing to
\textit{main} or \textit{dest} as described in \secref{sec:forward}.

%%%%%%%%%%%%%%%%%%%%%%%%%%%%%%%%%%%%%%%%%%%%%%%%%%%%%%%%%%%%%%%%%%%%%%%%%%%%%%%%
\subsection{Include by Input}
\label{sec:input}

Including child documents by |\include| has some restrictions by design.
Most notably, the content of a child document always occupies
its own set of pages; pages cannot be shared between child documents.
Usually, this behaviour makes perfect sense
because each child document contain an essential part of the document.
However, in some situations it may be desirable to compose
a document from a collection of parts
without having mandatory page breaks between then.
For this case, the package
provides a mechanism to include parts
by |\input| which can also be processed individually.
However, by construction this mechanism
requires manual handling of the content to be output.

%%%%%%%%%%%%%%%%%%%%%%%%%%%%%%%%%%%%%%%%
\DescribeMacro{\ifchilddocmanual}
The main file should be prepared as usual, see \secref{sec:include}.
However, the document body must make a distinction
between processing of an individual part and of the main document, e.g.:
%
\begin{center}
\begin{tabular}{l}
|\ifchilddocmanual|\\
|\input{\childdocname}|\\
|\||else|\\
\textit{document body with }|\input{|\textit{part}|}|\\
|\||fi|
\end{tabular}
\end{center}
%
The conditional |\ifchilddocmanual| is true whenever
a part to be included by |\input| is being compiled,
and the name of the part is stored in |\childdocname|.

%%%%%%%%%%%%%%%%%%%%%%%%%%%%%%%%%%%%%%%%
\DescribeMacro{\childdocby}
Each part to be included by |\input| should start with:
%
\begin{center}
\begin{tabular}{l}
|\input{childdoc.def}|\\
|\childdocby{|\textit{main}|}|\\
\end{tabular}
\end{center}
%
The directive |\childdocby| is similar to |\childdocof|
described in \secref{sec:include},
but the subsequent selection of content must be done manually.
To that end, both |\ifchilddoc| and |\ifchilddocmanual|
will be true upon processing of a part,
and the name of the part is stored in |\childdocname|.
Note that |\jobname| will be set to the filename of the current part
so that each part receives an individual |.aux| file
that does not interfere with the |.aux| file(s) of the main document.
This behaviour can be altered by the alternative form
|\childdocby[*]{|\textit{main}|}| (with a non-empty optional argument)
which uses the |.aux| file of the main document
by setting |\jobname| to \textit{main}.

%%%%%%%%%%%%%%%%%%%%%%%%%%%%%%%%%%%%%%%%%%%%%%%%%%%%%%%%%%%%%%%%%%%%%%%%%%%%%%%%
\subsection{Driver Development}
\label{sec:driver}

The \textsf{childdoc} mechanism can also be use for the development
of definition files such as \LaTeX{} styles or classes.
This case differs from the above setup with multiple parts
included by |\include| in that no |\includeonly| should be invoked.
This can be achieved by starting the include file
(before |\ProvidesPackage|) with:
%
\begin{center}
\begin{tabular}{l}
|\input{childdoc.def}|\\
|\childdocforward{|\textit{main}|}|\\
\end{tabular}
\end{center}
%
or alternatively with:
%
\begin{center}
\begin{tabular}{l}
|\input{childdoc.def}|\\
|\childdocby{|\textit{main}|}|\\
\end{tabular}
\end{center}
%
Both forms have slightly different effects as described above.
The main file is prepared as usual, see \secref{sec:include}.

%%%%%%%%%%%%%%%%%%%%%%%%%%%%%%%%%%%%%%%%%%%%%%%%%%%%%%%%%%%%%%%%%%%%%%%%%%%%%%%%
\subsection{Legacy Detection}
\label{sec:detection}

The directive |\childdocmain| in the main file can detect
whether the complete document or merely a child is to be compiled
even without using the directive |\childdocof|.
This method is deprecated because it is less robust
and there is no compelling reason to use it;
it is merely provided for backward compatibility
and it may be removed in future versions.

If the detection mechanism is to be used,
it is mandatory to correctly specify
the filename of the main file as the argument of |\childdocmain|:
%
\begin{center}
\begin{tabular}{l}
|\input{childdoc.def}|\\
|\childdocmain{|\textit{main}|}|\\
\end{tabular}
\end{center}
%
If |\jobname| does not match the argument \textit{main} of |\childdocmain|,
it is assumed that |\jobname| points to the child file to be compiled.
When using |\childdocmain| with the main file specified as argument,
it suffices to start a child file
with just |\input{|\textit{main}|}|
without loading of the package and using |\childdocof|.
If instead all processing is done
with the appropriate \textsf{childdoc} directives,
the argument of \textit{main} of |\childdocmain| can be empty.

An alternative version of the command line processing described
in \secref{sec:commandline} using the detection mechanism reads:
%
\begin{center}
|... -jobname "|\textit{target}|" "|[\textit{flags}]%
[|\def\jobname{|\textit{dest}|}|]|\input{|\textit{main}|}"|
\end{center}

%%%%%%%%%%%%%%%%%%%%%%%%%%%%%%%%%%%%%%%%%%%%%%%%%%%%%%%%%%%%%%%%%%%%%%%%%%%%%%%%
\subsection{Manual Code}
\label{sec:manual}

In case one cannot be certain whether the definitions file |childdoc.def|
is installed on the target \TeX{} distribution
and one prefers not to ship it,
it is conceivable to paste a few relevant commands into the sources.

To that end, drop all statements |\input{childdoc.def}|
and perform the replacements as outlined below.
Instead of |\childdocmain{|\textit{main}|}| add the following code
to the top of the main file:
%
\begin{center}
\begin{tabular}{l}
|\||ifdefined\childdocname\endinput\||fi\newif\ifchilddoc|\\
|\edef\childdocname{\scantokens\expandafter{\jobname\noexpand}}|\\
|\def\childdocmain{|\textit{main}|}\||ifx\childdocmain\childdocname\||else|\\
|\childdoctrue\includeonly{\childdocname}\let\jobname\childdocmain\||fi|\\
\end{tabular}
\end{center}
%
Instead of |\childdocof{|\textit{main}|}| just include the main file
at the top of each child file:
%
\begin{center}
|\input{|\textit{main}|}|
\end{center}
%
A simple redirection |\childdocforward{|\textit{dest}|}| is achieved by:
%
\begin{center}
|\def\jobname{|\textit{dest}|}\input{\jobname}|
\end{center}
%
The redirection with prefix
|\childdocforwardprefix[|\textit{prefix}|]{|\textit{dest}|}|
is accomplished by:
%
\begin{center}
\begin{tabular}{l}
|{\edef\jobname{\scantokens\expandafter{\jobname\noexpand}}|\\
|\def\redirectjob |\textit{prefix}|#1~~~{\gdef\jobname{|\textit{dest}|#1}}|\\
|\expandafter\redirectjob\jobname~~~}\input{\jobname}|
\end{tabular}
\end{center}

In an alternative approach,
child documents can be compiled by a specific command line
without additional code or specific definitions:
%
\begin{center}
|... -jobname "|\textit{target}|" "|[\textit{flags}]%
|\includeonly{|\textit{dest}|}\input{|\textit{main}|}"|
\end{center}
%

%%%%%%%%%%%%%%%%%%%%%%%%%%%%%%%%%%%%%%%%%%%%%%%%%%%%%%%%%%%%%%%%%%%%%%%%%%%%%%%%
%%%%%%%%%%%%%%%%%%%%%%%%%%%%%%%%%%%%%%%%%%%%%%%%%%%%%%%%%%%%%%%%%%%%%%%%%%%%%%%%
\section{Information}

%%%%%%%%%%%%%%%%%%%%%%%%%%%%%%%%%%%%%%%%%%%%%%%%%%%%%%%%%%%%%%%%%%%%%%%%%%%%%%%%
\subsection{Copyright}

Copyright \copyright{} 2017--2018 Niklas Beisert

This work may be distributed and/or modified under the
conditions of the \LaTeX{} Project Public License, either version 1.3
of this license or (at your option) any later version.
The latest version of this license is in
  \url{http://www.latex-project.org/lppl.txt}
and version 1.3 or later is part of all distributions of \LaTeX{}
version 2005/12/01 or later.

This work has the LPPL maintenance status `maintained'.

The Current Maintainer of this work is Niklas Beisert.

This work consists of the files |README.txt|, |childdoc.ins| and |childdoc.dtx|
as well as the derived files |childdoc.def|, |cdocsamp.tex|
with |cdocsch1.tex|, |cdocsch2.tex|, |cdocspt3.tex|, |cdocspt4.tex|,
|cdocsdrf.tex|, |cdocsfn1.tex|, |cdocsfn2.tex|
as well as |childdoc.pdf|.

%%%%%%%%%%%%%%%%%%%%%%%%%%%%%%%%%%%%%%%%%%%%%%%%%%%%%%%%%%%%%%%%%%%%%%%%%%%%%%%%
\subsection{Files and Installation}

The package consists of the files:
%
\begin{center}
\begin{tabular}{ll}
    |README.txt|   & readme file \\
    |childdoc.ins| & installation file \\
    |childdoc.dtx| & source file \\
    |childdoc.def| & definition file \\
    |cdocsamp.tex| & sample main file \\
    |cdocsch1.tex| & sample include file \\
    |cdocsch2.tex| & sample include file \\
    |cdocspt3.tex| & sample part file \\
    |cdocspt4.tex| & sample part file \\
    |cdocsdrf.tex| & sample redirection file \\
    |cdocsfn1.tex| & sample redirection file \\
    |cdocsfn2.tex| & sample redirection file \\
    |childdoc.pdf| & manual
\end{tabular}
\end{center}
%
The distribution consists of the files
|README.txt|, |childdoc.ins| and |childdoc.dtx|.
%
\begin{itemize}
\item
Run (pdf)\LaTeX{} on |childdoc.dtx|
to compile the manual |childdoc.pdf| (this file).
\item
Run \LaTeX{} on |childdoc.ins| to create the definitions file |childdoc.def|
and the sample |cdocsamp.tex| with include files
|cdocsch1.tex|, |cdocsch2.tex|, |cdocspt3.tex|, |cdocspt4.tex|,
|cdocsdrf.tex|, |cdocsfn1.tex|, |cdocsfn2.tex|.
Then copy the file |childdoc.def| to an appropriate directory of your \LaTeX{}
distribution, e.g.\ \textit{texmf-root}|/tex/latex/childdoc|.
\end{itemize}

%%%%%%%%%%%%%%%%%%%%%%%%%%%%%%%%%%%%%%%%%%%%%%%%%%%%%%%%%%%%%%%%%%%%%%%%%%%%%%%%
\subsection{Related CTAN Packages}

There are several other packages which offer a similar functionality:
%
\begin{itemize}
\item
The packages
\href{http://ctan.org/pkg/docmute}{\textsf{docmute}},
\href{http://ctan.org/pkg/includex}{\textsf{includex}} and
\href{http://ctan.org/pkg/standalone}{\textsf{standalone}}
provide commands to include only the document body of
a child file thus allowing both files to be compiled individually.
\item
The packages \href{http://ctan.org/pkg/subdocs}{\textsf{subdocs}}
and \href{http://ctan.org/pkg/subfiles}{\textsf{subfiles}}
provide structures in which the main and child documents can be
encapsulated and allowing them to be compiled individually.
The inclusion mechanism is different from the conventional |\include|.
\item
The package \href{http://ctan.org/pkg/combine}{\textsf{combine}}
is an elaborate solution to combine several documents into one.
\end{itemize}
%
See also the CTAN topic \href{http://ctan.org/topic/subdocs}{\textsf{subdocs}}
for further related packages.
The present package differs from the above solutions in that
a document structure constructed with the conventional |\include| mechanism
just needs two extra commands at the top of every file
such that all constituent files can be compiled individually.

%%%%%%%%%%%%%%%%%%%%%%%%%%%%%%%%%%%%%%%%%%%%%%%%%%%%%%%%%%%%%%%%%%%%%%%%%%%%%%%%
%\subsection{Feature Suggestions}
%
%The following is a list of features which may be useful for future
%versions of this package:
%%
%\begin{itemize}
%\item
%\ldots
%\end{itemize}

%%%%%%%%%%%%%%%%%%%%%%%%%%%%%%%%%%%%%%%%%%%%%%%%%%%%%%%%%%%%%%%%%%%%%%%%%%%%%%%%
\subsection{Revision History}

%%%%%%%%%%%%%%%%%%%%%%%%%%%%%%%%%%%%%%%%
\paragraph{v2.0:} 2018/12/30

\begin{itemize}
\item
immediate forward processing
\item
added |\childdocby| mechanism
\item
manual restructured
\end{itemize}

%%%%%%%%%%%%%%%%%%%%%%%%%%%%%%%%%%%%%%%%
\paragraph{v1.6:} 2018/01/17

\begin{itemize}
\item
application for development of include files
\item
corrections to manual
\end{itemize}

%%%%%%%%%%%%%%%%%%%%%%%%%%%%%%%%%%%%%%%%
\paragraph{v1.5:} 2017/05/21

\begin{itemize}
\item
more complete structuring introduced
\item
|\childdocof| introduced
\item
|\childdoc| renamed to |\childdocmain|
\item
|\childredirect| renamed to |\childdocforward| and |\childdocforwardprefix|
and functionality expanded
\end{itemize}

%%%%%%%%%%%%%%%%%%%%%%%%%%%%%%%%%%%%%%%%
\paragraph{v1.0:} 2017/04/27

\begin{itemize}
\item
manual and install package
\item
first version published on CTAN
\end{itemize}

%%%%%%%%%%%%%%%%%%%%%%%%%%%%%%%%%%%%%%%%
\paragraph{v0.6:} 2017/04/26

\begin{itemize}
\item
redirection mechanism added
\end{itemize}

%%%%%%%%%%%%%%%%%%%%%%%%%%%%%%%%%%%%%%%%
\paragraph{v0.5:} 2017/04/26

\begin{itemize}
\item
functionality in definition file
\end{itemize}


%%%%%%%%%%%%%%%%%%%%%%%%%%%%%%%%%%%%%%%%%%%%%%%%%%%%%%%%%%%%%%%%%%%%%%%%%%%%%%%%
%%%%%%%%%%%%%%%%%%%%%%%%%%%%%%%%%%%%%%%%%%%%%%%%%%%%%%%%%%%%%%%%%%%%%%%%%%%%%%%%
%%%%%%%%%%%%%%%%%%%%%%%%%%%%%%%%%%%%%%%%%%%%%%%%%%%%%%%%%%%%%%%%%%%%%%%%%%%%%%%%
\appendix

\settowidth\MacroIndent{\rmfamily\scriptsize 000\ }

 \DocInput{childdoc.dtx}

\end{document}
%</driver>
% \fi
%
% %%%%%%%%%%%%%%%%%%%%%%%%%%%%%%%%%%%%%%%%%%%%%%%%%%%%%%%%%%%%%%%%%%%%%%%%%%%%%%
% %%%%%%%%%%%%%%%%%%%%%%%%%%%%%%%%%%%%%%%%%%%%%%%%%%%%%%%%%%%%%%%%%%%%%%%%%%%%%%
% \section{Sample}
%\iffalse
%<*samplemain>
%\fi
%
% The following presents a sample document
% with two chapters, two parts, a title page,
% a compile flag as well as three forwarding files to set the flag.
% It consists of eight |.tex| files:
% \begin{center}
% \begin{tabular}{ll}
% |cdocsamp.tex|&main file\\
% |cdocsch1.tex|&include file for chapter 1\\
% |cdocsch2.tex|&include file for chapter 2\\
% |cdocspt3.tex|&include file for part 3\\
% |cdocspt4.tex|&include file for part 4\\
% |cdocsdrf.tex|&forwarding file for main file in draft mode\\
% |cdocsfi1.tex|&forwarding file for final version of chapter 1\\
% |cdocsfi2.tex|&forwarding file for final version of chapter 2\\
% \end{tabular}
% \end{center}
% Each of the eight files can be compiled directly by the \LaTeX{} compiler.
%
% %%%%%%%%%%%%%%%%%%%%%%%%%%%%%%%%%%%%%%
% \paragraph{Main File.}
%
% The main file is called |cdocsamp.tex|.
%
% Load the \textsf{childdoc} definitions and
% declare the filename for the main document:
%    \begin{macrocode}
\input{childdoc.def}
\childdocmain{}
%    \end{macrocode}

% Optional override for |\version| flag:
%    \begin{macrocode}
%%\ifchilddoc\else\providecommand{\version}{draft}\fi
%    \end{macrocode}

% Define the default values for the |\version| flag
% (|final| for the main file and |draft| for childs):
%    \begin{macrocode}
\ifchilddoc
\providecommand{\version}{draft}
\else
\providecommand{\version}{final}
\fi
%    \end{macrocode}

% Load the standard document class:
%    \begin{macrocode}
\documentclass[12pt]{article}
%    \end{macrocode}

% Start the document body:
%    \begin{macrocode}
\begin{document}
%    \end{macrocode}

% Declare a title page.
% Print title, part of document being processed and version flag:
%    \begin{macrocode}
\addtocounter{page}{-1}
\begin{center}
{\LARGE\bfseries{}childdoc example\par}
\vspace{1cm}
\ifchilddoc
\ifchilddocmanual part\else chapter\fi:
`\childdocname' of `\childdocjob'\par
\else
main document: `\childdocjob'\par
\fi
version: \version\par
\end{center}
\newpage
%    \end{macrocode}

% Manually include selected file,
% otherwise process as usual:
%    \begin{macrocode}
\ifchilddocmanual
\section*{part `\childdocname'}
\input{\childdocname}
\else
%    \end{macrocode}

% Include the two chapters:
%    \begin{macrocode}
\include{cdocsch1}
\include{cdocsch2}
%    \end{macrocode}

% Include the two parts unless only chapters should be displayed:
%    \begin{macrocode}
\ifchilddoc\else
\section{part three}
\input{cdocspt3}
\section{part four}
\input{cdocspt4}
\fi
%    \end{macrocode}

% Process as usual until here:
%    \begin{macrocode}
\fi
%    \end{macrocode}

% End of document body:
%    \begin{macrocode}
\end{document}
%    \end{macrocode}
%\iffalse
%</samplemain>
%\fi
%
% %%%%%%%%%%%%%%%%%%%%%%%%%%%%%%%%%%%%%%
% \paragraph{Chapter Include Files.}
%
% The include files are called |cdocsch1.tex| and |cdocsch2.tex|.
%
%\iffalse
%<*samplechap1|samplechap2>
%\fi

% Optional override for |\version| flag:
%    \begin{macrocode}
%%\providecommand{\version}{final}
%    \end{macrocode}

% Include the main document:
%    \begin{macrocode}
\input{childdoc.def}
\childdocof{cdocsamp}
%    \end{macrocode}

%\iffalse
%</samplechap1|samplechap2>
%\fi
%
%\iffalse
%<*samplechap1>
%\fi
% Some text for chapter 1:
%    \begin{macrocode}
\section{one}
some text in chapter one
%    \end{macrocode}

%\iffalse
%</samplechap1>
%\fi
% Some text for chapter 2:
%\iffalse
%<*samplechap2>
%\fi
%    \begin{macrocode}
\section{two}
more text in chapter two
%    \end{macrocode}

%\iffalse
%</samplechap2>
%\fi
%
% %%%%%%%%%%%%%%%%%%%%%%%%%%%%%%%%%%%%%%
% \paragraph{Part Include Files.}
%
% The include files are called |cdocspt3.tex| and |cdocspt4.tex|.
%
%\iffalse
%<*samplepart3|samplepart4>
%\fi

% Optional override for |\version| flag:
%    \begin{macrocode}
%%\providecommand{\version}{final}
%    \end{macrocode}

% Include the main document:
%    \begin{macrocode}
\input{childdoc.def}
\childdocby{cdocsamp}
%    \end{macrocode}

%\iffalse
%</samplepart3|samplepart4>
%\fi
%
%\iffalse
%<*samplepart3>
%\fi
% Some text for part 3:
%    \begin{macrocode}
some text in part three
%    \end{macrocode}

%\iffalse
%</samplepart3>
%\fi
% Some text for part 4:
%\iffalse
%<*samplepart4>
%\fi
%    \begin{macrocode}
more text in part four
%    \end{macrocode}

%\iffalse
%</samplepart4>
%\fi
%
% %%%%%%%%%%%%%%%%%%%%%%%%%%%%%%%%%%%%%%
% \paragraph{Forwarding for a Complete Draft.}
%
% The following forwarding file |cdocsdrf.tex|
% compiles the main document in draft mode:
%\iffalse
%<*sampledraft>
%\fi
%    \begin{macrocode}
\def\version{draft}
\input{childdoc.def}
\childdocforward{cdocsamp}
%    \end{macrocode}

%\iffalse
%</sampledraft>
%\fi
%
% %%%%%%%%%%%%%%%%%%%%%%%%%%%%%%%%%%%%%%
% \paragraph{Forwarding for Final Version of the Chapters.}
%
% The following forwarding files |cdocsfn1.tex| and |cdocsfn2.tex|
% (with identical content)
% compile the final versions of the child documents
% |cdocsch1.tex| and |cdocsch2.tex|, respectively:
%\iffalse
%<*samplefinal>
%\fi
%    \begin{macrocode}
\def\version{final}
\input{childdoc.def}
\childdocforwardprefix[cdocsamp]{cdocsfn}{cdocsch}
%    \end{macrocode}

%\iffalse
%</samplefinal>
%\fi
%
% %%%%%%%%%%%%%%%%%%%%%%%%%%%%%%%%%%%%%%
% \paragraph{Command Line Processing.}
%
% The following three command lines generate the output files
% |cdocscld|, |cdocscl1| and |cdocscl2|
% which should be identical to
% |cdocsdrf|, |cdocsch1| and |cdocsfn2|, respectively:
% \begin{center}
% \begin{tabular}{l}
% |latex -jobname cdocscld \|\\
% |  "\def\version{draft}\input{childdoc.def}\childdocforward{cdocsamp}"|\\
% |latex -jobname cdocscl1 \|\\
% |  "\input{childdoc.def}\childdocforward[cdocsamp]{cdocsch1}"|\\
% |latex -jobname cdocscl2 \|\\
% |  "\def\version{final}\input{childdoc.def}\childdocforward{cdocsch2}"|
% \end{tabular}
% \end{center}
% Note that the trailing backslash on each first line
% merely continues the input to the second line
% (for convenient cut ant paste).
% Furthermore, the command |latex| can be replaced by any
% of its alternative versions such as |pdflatex|.
%
% %%%%%%%%%%%%%%%%%%%%%%%%%%%%%%%%%%%%%%%%%%%%%%%%%%%%%%%%%%%%%%%%%%%%%%%%%%%%%%
% %%%%%%%%%%%%%%%%%%%%%%%%%%%%%%%%%%%%%%%%%%%%%%%%%%%%%%%%%%%%%%%%%%%%%%%%%%%%%%
% \section{Implementation}
%\iffalse
%<*package>
%\fi
%
% This section describes the definitions file |childdoc.def|.

% The definitions cannot be loaded using |\usepackage| or |\RequirePackage|
% which has a mechanism to prevent loading a style file more than once.
% When loading the definitions by means of |\input|
% multiple instances have to be prevented manually:
%\iffalse
%This code needs to be before the `\ProvidesFile' directive
%which is defined at the beginning of this file.
%Therefore it is also placed there and commented out here.
%</package>
%<*discard>
%\fi
%    \begin{macrocode}
\ifdefined\childdocmain\endinput\fi
%    \end{macrocode}
%\iffalse
%</discard>
%<*package>
%\fi
%
% \macro{\ifchilddoc}
% \macro{\ifchilddocmanual}
% The conditional |\ifchilddoc| tells whether a
% child (true) or main (false) document is being compiled.
% The conditional |\ifchilddocmanual| tells whether
% the |\includeonly| mechanism is used (false) or
% the selection of child files must be performed manually (true).
% The definitions initialise to false:
%    \begin{macrocode}
\newif\ifchilddoc
\newif\ifchilddocmanual
%    \end{macrocode}

% \macro{\childdocname}
% \macro{\childdocjob}
% The macro |\childdocname| stores the name of the main document
% to be compiled. The macro |\childdocjob| stores the name of
% the document on which the \LaTeX{} compiler was originally invoked.
% The content of |\jobname| cannot be compared
% to filenames specified in the source due to different catcodes.
% The following code rescans |\jobname|, stores the result
% in |\childdocname| and saves a copy in |\childdocjob|:
%    \begin{macrocode}
\edef\childdocname{\scantokens\expandafter{\jobname\noexpand}}
\let\childdocjob\childdocname
%    \end{macrocode}

% \macro{\childdocdisable}
% The macro |\childdocdisable| prevents the main file
% from being processed more than once.
% At this stage, the main document command |\childdocmain|
% is assumed to be called once again where it should do nothing.
% Any subsequent call to it should prevent
% a secondary processing of the main document
% It overwrites the forwarding commands
% |\childdocof| and |\childdocforward|
% with empty macros to prevent further inclusions of the main document:
%    \begin{macrocode}
\newcommand{\childdocdisable}
{
  \renewcommand{\childdocmain}[1]{\renewcommand{\childdocmain}[1]{\endinput}}
  \renewcommand{\childdocof}[1]{}
  \renewcommand{\childdocby}[2][]{}
  \renewcommand{\childdocforward}[2][]{}
  \renewcommand{\childdocdisable}{}
}
%    \end{macrocode}

% \macro{\childdocmain}
% The macro |\childdocmain| is to be called at the top of the main file
% with nothing or the main filename (without extension) as argument.
% First, it breaks loops.
% If the argument is not empty and does not match |\childdocname|
% (which is set by the first inclusion of |childdoc.def|),
% |\ifchilddoc| is set to true, |\includeonly| is applied to the child file
% and |\jobname| is set to the main file
% (for proper handling of |.aux| files):
%    \begin{macrocode}
\newcommand{\childdocmain}[1]
{
  \childdocdisable\childdocmain{}
  \if?#1?\else
    \begingroup
      \def\childdoctmp{#1}
      \ifx\childdoctmp\childdocname
        \def\childdoctmp{}
      \else
        \def\childdoctmp
        {
          \childdoctrue
          \includeonly{\childdocname}
          \def\childdocjob{#1}
          \def\jobname{#1}
        }
      \fi
      \expandafter
    \endgroup
    \childdoctmp
  \fi
}
%    \end{macrocode}

% \macro{\childdocof}
% The command |\childdocof| redirects
% compilation to the main file |#1|.
%    \begin{macrocode}
\newcommand{\childdocof}[1]
{
  \childdocdisable
  \childdoctrue
  \includeonly{\childdocname}
  \def\jobname{#1}
  \def\childdocjob{#1}
  \input{#1}
}
%    \end{macrocode}

% \macro{\childdocby}
% The command |\childdocby| ....
%    \begin{macrocode}
\newcommand{\childdocby}[2][]
{
  \childdocdisable
  \childdoctrue
  \childdocmanualtrue
  \if?#1?\else
    \def\jobname{#2}
  \fi
  \def\childdocjob{#2}
  \input{#2}
  \endinput
}
%    \end{macrocode}

% \macro{\childdocforward}
% The command |\childdocforward| redirects
% compilation to the main file or
% (if the optional argument is given) a child file.
% Parameters are set as if the main file
% or a child file starting with |\childdocof| was compiled.
% Then compilation is handed over to the main file:
%    \begin{macrocode}
\newcommand{\childdocforward}[2][]
{
  \begingroup
    \if?#1?
      \def\childdoctmp
      {
        \def\childdocname{#2}
        \def\childdocjob{#2}
        \def\jobname{#2}
        \input{#2}
        \endinput
      }
    \else
      \def\childdoctmp
      {
        \childdocdisable
        \def\childdocname{#2}
        \childdoctrue
        \includeonly{#2}
        \def\childdocjob{#1}
        \def\jobname{#1}
        \input{#1}
        \endinput
      }
    \fi
    \expandafter
  \endgroup
  \childdoctmp
}
%    \end{macrocode}

% \macro{\childdocforwardprefix}
% The command |\childdocforwardprefix| redirects
% compilation to the main or a child file by means of a pattern.
% The prefix |#1| in the current filename is replaced by |#2|
% and the suffix of the current filename is kept
% (it is assumed that the filename does not contain the substring `|~~~|'
% which is used as a delimiter).
% Compilation is handed over to the new file by |\childdocforward|:
%    \begin{macrocode}
\newcommand{\childdocforwardprefix}[3][]
{
  \begingroup
    \def\childdocextract #2##1~~~{\def\childdoctmp{\childdocforward[#1]{#3##1}}}
    \expandafter\childdocextract\childdocname~~~
    \expandafter
  \endgroup
  \childdoctmp
}
%    \end{macrocode}

% \macro{\childdoc}
% The deprecated macro |\childdoc| is a legacy version of |\childdocmain|:
%    \begin{macrocode}
\newcommand{\childdoc}{\childdocmain}
%    \end{macrocode}

% \macro{\childdocredirect}
% The deprecated macro |\childdocredirect| is a legacy version
% of |\childdocforward| and |\childdocforwardprefix|:
%    \begin{macrocode}
\newcommand{\childdocredirect}[2][]
{
  \begingroup
    \if?#1?
      \def\childdoctmp{\childdocforward{#2}}
    \else
      \def\childdoctmp{\childdocforwardprefix{#1}{#2}}
    \fi
    \expandafter
  \endgroup
  \childdoctmp
}
%    \end{macrocode}

%\iffalse
%</package>
%\fi
%
\endinput

\childdocmain{}
%    \end{macrocode}

% Optional override for |\version| flag:
%    \begin{macrocode}
%%\ifchilddoc\else\providecommand{\version}{draft}\fi
%    \end{macrocode}

% Define the default values for the |\version| flag
% (|final| for the main file and |draft| for childs):
%    \begin{macrocode}
\ifchilddoc
\providecommand{\version}{draft}
\else
\providecommand{\version}{final}
\fi
%    \end{macrocode}

% Load the standard document class:
%    \begin{macrocode}
\documentclass[12pt]{article}
%    \end{macrocode}

% Start the document body:
%    \begin{macrocode}
\begin{document}
%    \end{macrocode}

% Declare a title page.
% Print title, part of document being processed and version flag:
%    \begin{macrocode}
\addtocounter{page}{-1}
\begin{center}
{\LARGE\bfseries{}childdoc example\par}
\vspace{1cm}
\ifchilddoc
\ifchilddocmanual part\else chapter\fi:
`\childdocname' of `\childdocjob'\par
\else
main document: `\childdocjob'\par
\fi
version: \version\par
\end{center}
\newpage
%    \end{macrocode}

% Manually include selected file,
% otherwise process as usual:
%    \begin{macrocode}
\ifchilddocmanual
\section*{part `\childdocname'}
\input{\childdocname}
\else
%    \end{macrocode}

% Include the two chapters:
%    \begin{macrocode}
\include{cdocsch1}
\include{cdocsch2}
%    \end{macrocode}

% Include the two parts unless only chapters should be displayed:
%    \begin{macrocode}
\ifchilddoc\else
\section{part three}
\input{cdocspt3}
\section{part four}
\input{cdocspt4}
\fi
%    \end{macrocode}

% Process as usual until here:
%    \begin{macrocode}
\fi
%    \end{macrocode}

% End of document body:
%    \begin{macrocode}
\end{document}
%    \end{macrocode}
%\iffalse
%</samplemain>
%\fi
%
% %%%%%%%%%%%%%%%%%%%%%%%%%%%%%%%%%%%%%%
% \paragraph{Chapter Include Files.}
%
% The include files are called |cdocsch1.tex| and |cdocsch2.tex|.
%
%\iffalse
%<*samplechap1|samplechap2>
%\fi

% Optional override for |\version| flag:
%    \begin{macrocode}
%%\providecommand{\version}{final}
%    \end{macrocode}

% Include the main document:
%    \begin{macrocode}
% \iffalse
%
% childdoc.dtx Copyright (C) 2017-2018 Niklas Beisert
%
% This work may be distributed and/or modified under the
% conditions of the LaTeX Project Public License, either version 1.3
% of this license or (at your option) any later version.
% The latest version of this license is in
%   http://www.latex-project.org/lppl.txt
% and version 1.3 or later is part of all distributions of LaTeX
% version 2005/12/01 or later.
%
% This work has the LPPL maintenance status `maintained'.
%
% The Current Maintainer of this work is Niklas Beisert.
%
% This work consists of the files childdoc.dtx and childdoc.ins
% and the derived files childdoc.def and cdocsamp.tex with
% cdocsch1.tex, cdocsch2.tex, cdocsdrf.tex, cdocsfn1.tex, cdocsfn2.tex.
%
%<package>\ifdefined\childdocmain\endinput\fi
%<package>\ProvidesFile{childdoc.def}[2018/12/30 v2.0 child document driver]
%<samplemain>\ProvidesFile{cdocsamp.tex}[2018/12/30 v2.0 sample for childdoc]
%<*driver>
%\ProvidesFile{childdoc.drv}[2018/12/30 v2.0 childdoc reference manual file]
\PassOptionsToClass{10pt,a4paper}{article}
\documentclass{ltxdoc}

\usepackage[margin=35mm]{geometry}
\usepackage{hyperref}
\usepackage{hyperxmp}
\usepackage[usenames]{color}

\hypersetup{colorlinks=true}
\hypersetup{pdfstartview=FitH}
\hypersetup{pdfpagemode=UseNone}
\hypersetup{pdfsource={}}
\hypersetup{pdflang={en-UK}}
\hypersetup{pdfcopyright={Copyright 2017-2018 Niklas Beisert.
  This work may be distributed and/or modified under the
  conditions of the LaTeX Project Public License, either version 1.3
  of this license or (at your option) any later version.}}
\hypersetup{pdflicenseurl={http://www.latex-project.org/lppl.txt}}
\hypersetup{pdfcontactaddress={ETH Zurich, ITP, HIT K,
  Wolfgang-Pauli-Strasse 27}}
\hypersetup{pdfcontactpostcode={8093}}
\hypersetup{pdfcontactcity={Zurich}}
\hypersetup{pdfcontactcountry={Switzerland}}
\hypersetup{pdfcontactemail={nbeisert@itp.phys.ethz.ch}}
\hypersetup{pdfcontacturl={http://people.phys.ethz.ch/\xmptilde nbeisert/}}

\newcommand{\secref}[1]{\hyperref[#1]{section \ref*{#1}}}

\parskip1ex
\parindent0pt
\let\olditemize\itemize
\def\itemize{\olditemize\parskip0pt}

\begin{document}

\title{The \textsf{childdoc} Package}
\hypersetup{pdftitle={The childdoc Package}}
\author{Niklas Beisert\\[2ex]
  Institut f\"ur Theoretische Physik\\
  Eidgen\"ossische Technische Hochschule Z\"urich\\
  Wolfgang-Pauli-Strasse 27, 8093 Z\"urich, Switzerland\\[1ex]
  \href{mailto:nbeisert@itp.phys.ethz.ch}
  {\texttt{nbeisert@itp.phys.ethz.ch}}}
\hypersetup{pdfauthor={Niklas Beisert}}
\hypersetup{pdfsubject={Manual for the LaTeX2e Package childdoc}}
\date{30 December 2018, \textsf{v2.0}}
\maketitle

\begin{abstract}\noindent
\textsf{childdoc} is a \LaTeXe{} package
that enables the direct compilation
of document sections included by |\include|
to individual files.
\end{abstract}

\begingroup
\parskip0ex
\tableofcontents
\endgroup

%%%%%%%%%%%%%%%%%%%%%%%%%%%%%%%%%%%%%%%%%%%%%%%%%%%%%%%%%%%%%%%%%%%%%%%%%%%%%%%%
%%%%%%%%%%%%%%%%%%%%%%%%%%%%%%%%%%%%%%%%%%%%%%%%%%%%%%%%%%%%%%%%%%%%%%%%%%%%%%%%
\section{Introduction}

\LaTeX{} provides a mechanism to structure a large document (such as a book)
into a main file and several child files (containing the chapters)
using the |\include| command.
This mechanism is beneficial for documents
which span hundreds of pages in order to
make the source file(s) more manageable.
Moreover, compilation can be restricted to
selected child files by means of the |\includeonly| command.
The latter feature can be used to reduce the compilation time while editing
(this was significantly more useful in the earlier days of \LaTeX{})
or to generate a smaller document which is easier to navigate.
Another application of |\includeonly| is to generate
documents consisting of selected parts of the complete document.

However, there are a few drawbacks of the plain |\include| mechanism:
\begin{itemize}
\item
The child files cannot be compiled on their own,
they can only be compiled via the main file.
A naive editing environment
(such as a text editor with an option
to have the current file processed by \LaTeX)
may require one to switch to the main file before compiling;
attempting to compile the child file produces errors.
\item
The main file must be modified (each time)
to adjust the |\includeonly| command
to the present needs. This easily leaves the main file in a messy state.
\item
The generated document will always carry the filename
of the main document. This is inconvenient if
several child files are to be compiled and
to be kept for distribution.
\end{itemize}

The present package provides a simple interface
to make child files individually compilable by \LaTeX{}.
Compiling a child file then has the same effect as compiling
the main file with an |\includeonly| command
to select the appropriate child.
Moreover the generated document will carry the name of the child
rather than the main file.
This resolves all three above issues.

This feature is meant to make the editing of books,
thesis documents and lecture notes somewhat more convenient.
However, the package can also be used efficiently for
composing a series of documents (such as exercise sheets)
which are typically distributed individually.
It then assists the author in generating the individual documents
(potentially in different versions)
as well as a document containing the collected series.
Another application is in developing style files
or other kinds of included material
where compilation of the style file could redirect
to a sample or test file.

%%%%%%%%%%%%%%%%%%%%%%%%%%%%%%%%%%%%%%%%%%%%%%%%%%%%%%%%%%%%%%%%%%%%%%%%%%%%%%%%
%%%%%%%%%%%%%%%%%%%%%%%%%%%%%%%%%%%%%%%%%%%%%%%%%%%%%%%%%%%%%%%%%%%%%%%%%%%%%%%%
\section{Usage}

First of all, the package \textsf{childdoc} is \emph{not} a standard
\LaTeXe{} |.sty| style file! Therefore it needs to be invoked in
a non-standard way.

%%%%%%%%%%%%%%%%%%%%%%%%%%%%%%%%%%%%%%%%%%%%%%%%%%%%%%%%%%%%%%%%%%%%%%%%%%%%%%%%
\subsection{Included Files}
\label{sec:include}

%%%%%%%%%%%%%%%%%%%%%%%%%%%%%%%%%%%%%%%%
\DescribeMacro{\childdocmain}
To use the package, add the commands
\begin{center}
\begin{tabular}{l}
|\input{childdoc.def}|\\
|\childdocmain{}|\\
\end{tabular}
\end{center}
at the very top of the main \LaTeX{} file,
in particular \emph{before} the |\documentclass| statement!
The argument of |\childdocmain| should be left empty
(but it must be present).

%%%%%%%%%%%%%%%%%%%%%%%%%%%%%%%%%%%%%%%%
\DescribeMacro{\childdocof}
Furthermore, add the commands
\begin{center}
\begin{tabular}{l}
|\input{childdoc.def}|\\
|\childdocof{|\textit{main}|}|\\
\end{tabular}
\end{center}
at the top of every child file \textit{child}
which is included by |\include{|\textit{child}|}|
from within the main file
(or at least for those files to be compiled individually).
The argument \textit{main} must be the filename of the main file.

There are a couple of
considerations in setting up the main and child documents:

%%%%%%%%%%%%%%%%%%%%%%%%%%%%%%%%%%%%%%%%
\paragraph{Restrictions.}

Please note the following restrictions:
\begin{itemize}
\item
|\childdocmain| must be called with one argument \textit{main}
to ensure compatibility with earlier version of the package.
It must either be empty (|\childdocmain{}|)
or precisely match the filename of the main file in which it is specified.
See \secref{sec:detection} for further information.
\item
The filename \textit{main} must be specified without the |.tex| extension.
\item
The filename \textit{main} is case sensitive
(even in case-insensitive file systems)
due to internal string comparison.
\item
The argument \textit{main} should be fully expanded, it cannot be a macro.
\item
Subdirectories and special characters should be avoided in filenames.
\item
The command |\childdocmain{|\textit{main}|}| must be followed by a whitespace.
It should not be followed immediately by another command
or by a comment mark `|%|'.
This is because the \TeX{} parser reads the token immediately following
the argument of |\childdocmain| and puts it
at the beginning of every child section;
however, a white\-space is ignored.
\end{itemize}

%%%%%%%%%%%%%%%%%%%%%%%%%%%%%%%%%%%%%%%%
\paragraph{Content of Main File.}

It is advisable to place all content in the child files included by |\include|.
Any output contained in the main file will appear in all child documents
unless suppressed manually;
it cannot be suppressed automatically by the |\includeonly| directive
and thus should normally be avoided.
A method to include some content in the main file
by means of conditional processing is described in \secref{sec:conditional}.

%%%%%%%%%%%%%%%%%%%%%%%%%%%%%%%%%%%%%%%%
\paragraph{Page Numbering.}

When only a part of the document is compiled,
the appropriate numbering of pages
(as well as other status parameters)
is determined from the |.aux| files.
The latter contain information from previous passes.
However this information needs to propagate through
all intermediate child documents.
Therefore the page numbering in child documents may well
be inconsistent until the complete document is compiled at least once.

A useful (if unconventional) way to always ensure a consistent
page numbering is to restart the numbering in each child document
and denote the pages by `\textit{child}|.|\textit{page}'
where \textit{child} represents the chapter/section number of the child file.
This can be achieved by the command
|\numberwithin{page}{|\textit{child}|}|
of the \textsf{amsmath} package
where \textit{child} can be |chapter| or |section|
depending on the chosen structuring.
Alternatively, one can modify the macro |\thepage| appropriately
and reset the counter |page| at the start of each child file.

%%%%%%%%%%%%%%%%%%%%%%%%%%%%%%%%%%%%%%%%%%%%%%%%%%%%%%%%%%%%%%%%%%%%%%%%%%%%%%%%
\subsection{Conditional Processing}
\label{sec:conditional}

The package provides a mechanism to compile different versions
of a document. To customise the versions further some conditional processing
can come in handy to distinguish which version is being compiled.
The package provides two macros to describe the compilation context:

%%%%%%%%%%%%%%%%%%%%%%%%%%%%%%%%%%%%%%%%
\DescribeMacro{\ifchilddoc}
The conditional |\ifchilddoc| distinguishes between the compilation of
child documents and the main document:
%
\begin{center}
|\ifchilddoc |\textit{child-code}| |[|\||else |\textit{main-code}]| \||fi|
\end{center}

%%%%%%%%%%%%%%%%%%%%%%%%%%%%%%%%%%%%%%%%
\DescribeMacro{\childdocname}
\DescribeMacro{\childdocjob}
The macro |\childdocname| contains the filename (without extension)
of the main or child file being processed.
Note that |\childdocjob| will always contain the name of the main file.

%%%%%%%%%%%%%%%%%%%%%%%%%%%%%%%%%%%%%%%%
\paragraph{Title Page.}

Conditional processing can be used to include a title or banner page
in the main document when proper precautions are taken.
Importantly, the code in the main file should ensure that the page counter
(as well as other status parameters which are stored in the |.aux| files)
takes the same value after the conditional processing.
Otherwise the page numbers may take divergent values
depending on which part is compiled.

For example, a title page could be declared by:
%
\begin{center}
\begin{tabular}{l}
|\ifchilddoc\||else|\\
|\addtocounter{page}{-1}|\\
\textit{code for title page}\\
|\newpage|\\
|\||fi|
\end{tabular}
\end{center}
%
A banner page for the child documents can be generated by:
%
\begin{center}
\begin{tabular}{l}
|\ifchilddoc|\\
|\addtocounter{page}{-1}|\\
\textit{code for banner page}\\
|\newpage|\\
|\||fi|
\end{tabular}
\end{center}
%
Here one could write a message such as:
\begin{center}
|This is the part \childdocname{} of \childdocjob{}.|
\end{center}

%%%%%%%%%%%%%%%%%%%%%%%%%%%%%%%%%%%%%%%%%%%%%%%%%%%%%%%%%%%%%%%%%%%%%%%%%%%%%%%%
\subsection{Flags}
\label{sec:flags}

The package makes it easy to generate different versions
of the main or child documents.
To this end compilation flags can be defined
and assigned different default values.
They will be particularly useful in conjunction
with the forwarding mechanism described in \secref{sec:forward}.

For example, it may be useful to have a flag |\version|
which can be set to |draft| or |final|.
The document source will contain some conditional code
depending on the value of |\version|.
Suppose further, the flag should default to |final| for the main file
and to |draft| for child files
which is a natural assignment for editing the document.
This is achieved by placing the following code
in the preamble of the main document
(below the |\childdocmain| directive):
%
\begin{center}
\begin{tabular}{l}
|\ifchilddoc|\\
|\providecommand{\version}{draft}|\\
|\||else|\\
|\providecommand{\version}{final}|\\
|\||fi|
\end{tabular}
\end{center}
%
The definition by |\providecommand| makes sure
that previous definitions are not overwritten.
Further statements |\providecommand{\version}{...}|
can thus be added before the above code to override it.

For the main file, one might add a line
(between |\childdocmain| and the above block)
%
\begin{center}
|%\ifchilddoc\||else\providecommand{\version}{draft}\||fi|
\end{center}
%
which can be uncommented to produce a draft version.
Likewise one can add a line to the very top of a child file
(above the |\childdocof{|\textit{main}|}| directive)
%
\begin{center}
|%\providecommand{\version}{final}|
\end{center}
%
which can be uncommented to produce the final version of this child document.

%%%%%%%%%%%%%%%%%%%%%%%%%%%%%%%%%%%%%%%%%%%%%%%%%%%%%%%%%%%%%%%%%%%%%%%%%%%%%%%%
\subsection{Forwarding}
\label{sec:forward}

Different versions of the main or child documents
using compilation flags as described in \secref{sec:flags}
can be (permanently) stored in different files
for convenient compilation, viewing and distribution.
To this end, the package defines a command
to pass on compilation to a different file:

%%%%%%%%%%%%%%%%%%%%%%%%%%%%%%%%%%%%%%%%
\DescribeMacro{\childdocforward}
The command |\childdocforward| redirects processing to
another source file:
%
\begin{center}
\begin{tabular}{l}
|\input{childdoc.def}|\\
|\childdocforward[|\textit{main}|]{|\textit{dest}|}|\\
\end{tabular}
\end{center}
%
The argument \textit{dest} is the destination file
(without extension).
It should be the main file or one of the child files.
Note that further \textsf{childdoc} directives
such as |\childdocof| and |\childdocforward|
in the indicated file will be processed in this form.
The optional argument \textit{main}
passes on directly to the main file \textit{main}
while pretending to compile the child \textit{dest}.
This form behaves as if \textit{dest}
issues |\childdocof{|\textit{main}|}| right away,
and no further \textsf{childdoc} directives will be processed.

%%%%%%%%%%%%%%%%%%%%%%%%%%%%%%%%%%%%%%%%
\DescribeMacro{\...prefix}
In the alternative form |\childdocforwardprefix|,
%
\begin{center}
\begin{tabular}{l}
|\input{childdoc.def}|\\
|\childdocforwardprefix[|\textit{main}|]{|\textit{prefix}|}{|\textit{dest}|}|
\end{tabular}
\end{center}
%
the destination file is determined by a pattern
depending on the current file:
To make this work, the current file must be called
`{\textit{prefix}\hspace{0.2em}\textit{suffix}}'
with \textit{prefix} matching precisely the argument.
Processing is then passed on to the file
`{\textit{dest}\hspace{0.2em}\textit{suffix}}'.
Surely, the same effect is achieved by
directly specifying the
argument `{\textit{dest}\hspace{0.2em}\textit{suffix}}'
in the first form.
However, that requires to set up a different file
for each child. With the alternative form of the command
all these files can have exactly the same content
which simplifies setting them up and maintaining them.

For example, the following file |draft.tex|
with a compilation flag |\version| as described in \secref{sec:flags}
compiles the main document as a draft:
%
\begin{center}
\begin{tabular}{l}
|\def\version{draft}|\\
|\input{childdoc.def}|\\
|\childdocforward{|\textit{main}|}|
\end{tabular}
\end{center}
%
Likewise, the following files |final|\textit{nn}|.tex|
compile the final version of the child document
|child|\textit{nn}|.tex|:
%
\begin{center}
\begin{tabular}{l}
|\def\version{final}|\\
|\input{childdoc.def}|\\
|\childdocforwardprefix{final}{child}|
\end{tabular}
\end{center}
%

Note that when several versions of a main file and/or of each child file
are to be generated, it may be convenient to set up a |Makefile| or
shell script to automatise the process.

%%%%%%%%%%%%%%%%%%%%%%%%%%%%%%%%%%%%%%%%%%%%%%%%%%%%%%%%%%%%%%%%%%%%%%%%%%%%%%%%
\subsection{Command Line Processing}
\label{sec:commandline}

The effect of redirection files can also be achieved by invoking
the \LaTeX{} compiler with a more elaborate command line.
Most conveniently this should be done as part
of a shell script or a |Makefile|.

When using \textsf{childdoc} in the main file, the following
command lines effectively perform a redirection
(note that depending on the shell being used,
backslashes may have to be doubled: `|\|' $\to$ `|\\|'):
%
\begin{center}
|... -jobname "|\textit{target}|" |\\|"|[\textit{flags}]%
|\input{childdoc.def}\childdocforward[|\textit{main}|]{|\textit{dest}|}"|
\end{center}
%
Here \textit{target} is the name of the output file,
\textit{main} is the name of the main file
and \textit{dest} is the name of the main or child file to be processed
(all filenames without extensions).
The optional argument \textit{main} can be omitted
if \textit{main} matches \textit{dest}.
Optionally, compilation \textit{flags} can be defined via |\def| commands.
This command line makes the \TeX{} engine believe
it is compiling the file \textit{target}
whose content is specified as the latter parameter.
The provided code then forwards the processing to
\textit{main} or \textit{dest} as described in \secref{sec:forward}.

%%%%%%%%%%%%%%%%%%%%%%%%%%%%%%%%%%%%%%%%%%%%%%%%%%%%%%%%%%%%%%%%%%%%%%%%%%%%%%%%
\subsection{Include by Input}
\label{sec:input}

Including child documents by |\include| has some restrictions by design.
Most notably, the content of a child document always occupies
its own set of pages; pages cannot be shared between child documents.
Usually, this behaviour makes perfect sense
because each child document contain an essential part of the document.
However, in some situations it may be desirable to compose
a document from a collection of parts
without having mandatory page breaks between then.
For this case, the package
provides a mechanism to include parts
by |\input| which can also be processed individually.
However, by construction this mechanism
requires manual handling of the content to be output.

%%%%%%%%%%%%%%%%%%%%%%%%%%%%%%%%%%%%%%%%
\DescribeMacro{\ifchilddocmanual}
The main file should be prepared as usual, see \secref{sec:include}.
However, the document body must make a distinction
between processing of an individual part and of the main document, e.g.:
%
\begin{center}
\begin{tabular}{l}
|\ifchilddocmanual|\\
|\input{\childdocname}|\\
|\||else|\\
\textit{document body with }|\input{|\textit{part}|}|\\
|\||fi|
\end{tabular}
\end{center}
%
The conditional |\ifchilddocmanual| is true whenever
a part to be included by |\input| is being compiled,
and the name of the part is stored in |\childdocname|.

%%%%%%%%%%%%%%%%%%%%%%%%%%%%%%%%%%%%%%%%
\DescribeMacro{\childdocby}
Each part to be included by |\input| should start with:
%
\begin{center}
\begin{tabular}{l}
|\input{childdoc.def}|\\
|\childdocby{|\textit{main}|}|\\
\end{tabular}
\end{center}
%
The directive |\childdocby| is similar to |\childdocof|
described in \secref{sec:include},
but the subsequent selection of content must be done manually.
To that end, both |\ifchilddoc| and |\ifchilddocmanual|
will be true upon processing of a part,
and the name of the part is stored in |\childdocname|.
Note that |\jobname| will be set to the filename of the current part
so that each part receives an individual |.aux| file
that does not interfere with the |.aux| file(s) of the main document.
This behaviour can be altered by the alternative form
|\childdocby[*]{|\textit{main}|}| (with a non-empty optional argument)
which uses the |.aux| file of the main document
by setting |\jobname| to \textit{main}.

%%%%%%%%%%%%%%%%%%%%%%%%%%%%%%%%%%%%%%%%%%%%%%%%%%%%%%%%%%%%%%%%%%%%%%%%%%%%%%%%
\subsection{Driver Development}
\label{sec:driver}

The \textsf{childdoc} mechanism can also be use for the development
of definition files such as \LaTeX{} styles or classes.
This case differs from the above setup with multiple parts
included by |\include| in that no |\includeonly| should be invoked.
This can be achieved by starting the include file
(before |\ProvidesPackage|) with:
%
\begin{center}
\begin{tabular}{l}
|\input{childdoc.def}|\\
|\childdocforward{|\textit{main}|}|\\
\end{tabular}
\end{center}
%
or alternatively with:
%
\begin{center}
\begin{tabular}{l}
|\input{childdoc.def}|\\
|\childdocby{|\textit{main}|}|\\
\end{tabular}
\end{center}
%
Both forms have slightly different effects as described above.
The main file is prepared as usual, see \secref{sec:include}.

%%%%%%%%%%%%%%%%%%%%%%%%%%%%%%%%%%%%%%%%%%%%%%%%%%%%%%%%%%%%%%%%%%%%%%%%%%%%%%%%
\subsection{Legacy Detection}
\label{sec:detection}

The directive |\childdocmain| in the main file can detect
whether the complete document or merely a child is to be compiled
even without using the directive |\childdocof|.
This method is deprecated because it is less robust
and there is no compelling reason to use it;
it is merely provided for backward compatibility
and it may be removed in future versions.

If the detection mechanism is to be used,
it is mandatory to correctly specify
the filename of the main file as the argument of |\childdocmain|:
%
\begin{center}
\begin{tabular}{l}
|\input{childdoc.def}|\\
|\childdocmain{|\textit{main}|}|\\
\end{tabular}
\end{center}
%
If |\jobname| does not match the argument \textit{main} of |\childdocmain|,
it is assumed that |\jobname| points to the child file to be compiled.
When using |\childdocmain| with the main file specified as argument,
it suffices to start a child file
with just |\input{|\textit{main}|}|
without loading of the package and using |\childdocof|.
If instead all processing is done
with the appropriate \textsf{childdoc} directives,
the argument of \textit{main} of |\childdocmain| can be empty.

An alternative version of the command line processing described
in \secref{sec:commandline} using the detection mechanism reads:
%
\begin{center}
|... -jobname "|\textit{target}|" "|[\textit{flags}]%
[|\def\jobname{|\textit{dest}|}|]|\input{|\textit{main}|}"|
\end{center}

%%%%%%%%%%%%%%%%%%%%%%%%%%%%%%%%%%%%%%%%%%%%%%%%%%%%%%%%%%%%%%%%%%%%%%%%%%%%%%%%
\subsection{Manual Code}
\label{sec:manual}

In case one cannot be certain whether the definitions file |childdoc.def|
is installed on the target \TeX{} distribution
and one prefers not to ship it,
it is conceivable to paste a few relevant commands into the sources.

To that end, drop all statements |\input{childdoc.def}|
and perform the replacements as outlined below.
Instead of |\childdocmain{|\textit{main}|}| add the following code
to the top of the main file:
%
\begin{center}
\begin{tabular}{l}
|\||ifdefined\childdocname\endinput\||fi\newif\ifchilddoc|\\
|\edef\childdocname{\scantokens\expandafter{\jobname\noexpand}}|\\
|\def\childdocmain{|\textit{main}|}\||ifx\childdocmain\childdocname\||else|\\
|\childdoctrue\includeonly{\childdocname}\let\jobname\childdocmain\||fi|\\
\end{tabular}
\end{center}
%
Instead of |\childdocof{|\textit{main}|}| just include the main file
at the top of each child file:
%
\begin{center}
|\input{|\textit{main}|}|
\end{center}
%
A simple redirection |\childdocforward{|\textit{dest}|}| is achieved by:
%
\begin{center}
|\def\jobname{|\textit{dest}|}\input{\jobname}|
\end{center}
%
The redirection with prefix
|\childdocforwardprefix[|\textit{prefix}|]{|\textit{dest}|}|
is accomplished by:
%
\begin{center}
\begin{tabular}{l}
|{\edef\jobname{\scantokens\expandafter{\jobname\noexpand}}|\\
|\def\redirectjob |\textit{prefix}|#1~~~{\gdef\jobname{|\textit{dest}|#1}}|\\
|\expandafter\redirectjob\jobname~~~}\input{\jobname}|
\end{tabular}
\end{center}

In an alternative approach,
child documents can be compiled by a specific command line
without additional code or specific definitions:
%
\begin{center}
|... -jobname "|\textit{target}|" "|[\textit{flags}]%
|\includeonly{|\textit{dest}|}\input{|\textit{main}|}"|
\end{center}
%

%%%%%%%%%%%%%%%%%%%%%%%%%%%%%%%%%%%%%%%%%%%%%%%%%%%%%%%%%%%%%%%%%%%%%%%%%%%%%%%%
%%%%%%%%%%%%%%%%%%%%%%%%%%%%%%%%%%%%%%%%%%%%%%%%%%%%%%%%%%%%%%%%%%%%%%%%%%%%%%%%
\section{Information}

%%%%%%%%%%%%%%%%%%%%%%%%%%%%%%%%%%%%%%%%%%%%%%%%%%%%%%%%%%%%%%%%%%%%%%%%%%%%%%%%
\subsection{Copyright}

Copyright \copyright{} 2017--2018 Niklas Beisert

This work may be distributed and/or modified under the
conditions of the \LaTeX{} Project Public License, either version 1.3
of this license or (at your option) any later version.
The latest version of this license is in
  \url{http://www.latex-project.org/lppl.txt}
and version 1.3 or later is part of all distributions of \LaTeX{}
version 2005/12/01 or later.

This work has the LPPL maintenance status `maintained'.

The Current Maintainer of this work is Niklas Beisert.

This work consists of the files |README.txt|, |childdoc.ins| and |childdoc.dtx|
as well as the derived files |childdoc.def|, |cdocsamp.tex|
with |cdocsch1.tex|, |cdocsch2.tex|, |cdocspt3.tex|, |cdocspt4.tex|,
|cdocsdrf.tex|, |cdocsfn1.tex|, |cdocsfn2.tex|
as well as |childdoc.pdf|.

%%%%%%%%%%%%%%%%%%%%%%%%%%%%%%%%%%%%%%%%%%%%%%%%%%%%%%%%%%%%%%%%%%%%%%%%%%%%%%%%
\subsection{Files and Installation}

The package consists of the files:
%
\begin{center}
\begin{tabular}{ll}
    |README.txt|   & readme file \\
    |childdoc.ins| & installation file \\
    |childdoc.dtx| & source file \\
    |childdoc.def| & definition file \\
    |cdocsamp.tex| & sample main file \\
    |cdocsch1.tex| & sample include file \\
    |cdocsch2.tex| & sample include file \\
    |cdocspt3.tex| & sample part file \\
    |cdocspt4.tex| & sample part file \\
    |cdocsdrf.tex| & sample redirection file \\
    |cdocsfn1.tex| & sample redirection file \\
    |cdocsfn2.tex| & sample redirection file \\
    |childdoc.pdf| & manual
\end{tabular}
\end{center}
%
The distribution consists of the files
|README.txt|, |childdoc.ins| and |childdoc.dtx|.
%
\begin{itemize}
\item
Run (pdf)\LaTeX{} on |childdoc.dtx|
to compile the manual |childdoc.pdf| (this file).
\item
Run \LaTeX{} on |childdoc.ins| to create the definitions file |childdoc.def|
and the sample |cdocsamp.tex| with include files
|cdocsch1.tex|, |cdocsch2.tex|, |cdocspt3.tex|, |cdocspt4.tex|,
|cdocsdrf.tex|, |cdocsfn1.tex|, |cdocsfn2.tex|.
Then copy the file |childdoc.def| to an appropriate directory of your \LaTeX{}
distribution, e.g.\ \textit{texmf-root}|/tex/latex/childdoc|.
\end{itemize}

%%%%%%%%%%%%%%%%%%%%%%%%%%%%%%%%%%%%%%%%%%%%%%%%%%%%%%%%%%%%%%%%%%%%%%%%%%%%%%%%
\subsection{Related CTAN Packages}

There are several other packages which offer a similar functionality:
%
\begin{itemize}
\item
The packages
\href{http://ctan.org/pkg/docmute}{\textsf{docmute}},
\href{http://ctan.org/pkg/includex}{\textsf{includex}} and
\href{http://ctan.org/pkg/standalone}{\textsf{standalone}}
provide commands to include only the document body of
a child file thus allowing both files to be compiled individually.
\item
The packages \href{http://ctan.org/pkg/subdocs}{\textsf{subdocs}}
and \href{http://ctan.org/pkg/subfiles}{\textsf{subfiles}}
provide structures in which the main and child documents can be
encapsulated and allowing them to be compiled individually.
The inclusion mechanism is different from the conventional |\include|.
\item
The package \href{http://ctan.org/pkg/combine}{\textsf{combine}}
is an elaborate solution to combine several documents into one.
\end{itemize}
%
See also the CTAN topic \href{http://ctan.org/topic/subdocs}{\textsf{subdocs}}
for further related packages.
The present package differs from the above solutions in that
a document structure constructed with the conventional |\include| mechanism
just needs two extra commands at the top of every file
such that all constituent files can be compiled individually.

%%%%%%%%%%%%%%%%%%%%%%%%%%%%%%%%%%%%%%%%%%%%%%%%%%%%%%%%%%%%%%%%%%%%%%%%%%%%%%%%
%\subsection{Feature Suggestions}
%
%The following is a list of features which may be useful for future
%versions of this package:
%%
%\begin{itemize}
%\item
%\ldots
%\end{itemize}

%%%%%%%%%%%%%%%%%%%%%%%%%%%%%%%%%%%%%%%%%%%%%%%%%%%%%%%%%%%%%%%%%%%%%%%%%%%%%%%%
\subsection{Revision History}

%%%%%%%%%%%%%%%%%%%%%%%%%%%%%%%%%%%%%%%%
\paragraph{v2.0:} 2018/12/30

\begin{itemize}
\item
immediate forward processing
\item
added |\childdocby| mechanism
\item
manual restructured
\end{itemize}

%%%%%%%%%%%%%%%%%%%%%%%%%%%%%%%%%%%%%%%%
\paragraph{v1.6:} 2018/01/17

\begin{itemize}
\item
application for development of include files
\item
corrections to manual
\end{itemize}

%%%%%%%%%%%%%%%%%%%%%%%%%%%%%%%%%%%%%%%%
\paragraph{v1.5:} 2017/05/21

\begin{itemize}
\item
more complete structuring introduced
\item
|\childdocof| introduced
\item
|\childdoc| renamed to |\childdocmain|
\item
|\childredirect| renamed to |\childdocforward| and |\childdocforwardprefix|
and functionality expanded
\end{itemize}

%%%%%%%%%%%%%%%%%%%%%%%%%%%%%%%%%%%%%%%%
\paragraph{v1.0:} 2017/04/27

\begin{itemize}
\item
manual and install package
\item
first version published on CTAN
\end{itemize}

%%%%%%%%%%%%%%%%%%%%%%%%%%%%%%%%%%%%%%%%
\paragraph{v0.6:} 2017/04/26

\begin{itemize}
\item
redirection mechanism added
\end{itemize}

%%%%%%%%%%%%%%%%%%%%%%%%%%%%%%%%%%%%%%%%
\paragraph{v0.5:} 2017/04/26

\begin{itemize}
\item
functionality in definition file
\end{itemize}


%%%%%%%%%%%%%%%%%%%%%%%%%%%%%%%%%%%%%%%%%%%%%%%%%%%%%%%%%%%%%%%%%%%%%%%%%%%%%%%%
%%%%%%%%%%%%%%%%%%%%%%%%%%%%%%%%%%%%%%%%%%%%%%%%%%%%%%%%%%%%%%%%%%%%%%%%%%%%%%%%
%%%%%%%%%%%%%%%%%%%%%%%%%%%%%%%%%%%%%%%%%%%%%%%%%%%%%%%%%%%%%%%%%%%%%%%%%%%%%%%%
\appendix

\settowidth\MacroIndent{\rmfamily\scriptsize 000\ }

 \DocInput{childdoc.dtx}

\end{document}
%</driver>
% \fi
%
% %%%%%%%%%%%%%%%%%%%%%%%%%%%%%%%%%%%%%%%%%%%%%%%%%%%%%%%%%%%%%%%%%%%%%%%%%%%%%%
% %%%%%%%%%%%%%%%%%%%%%%%%%%%%%%%%%%%%%%%%%%%%%%%%%%%%%%%%%%%%%%%%%%%%%%%%%%%%%%
% \section{Sample}
%\iffalse
%<*samplemain>
%\fi
%
% The following presents a sample document
% with two chapters, two parts, a title page,
% a compile flag as well as three forwarding files to set the flag.
% It consists of eight |.tex| files:
% \begin{center}
% \begin{tabular}{ll}
% |cdocsamp.tex|&main file\\
% |cdocsch1.tex|&include file for chapter 1\\
% |cdocsch2.tex|&include file for chapter 2\\
% |cdocspt3.tex|&include file for part 3\\
% |cdocspt4.tex|&include file for part 4\\
% |cdocsdrf.tex|&forwarding file for main file in draft mode\\
% |cdocsfi1.tex|&forwarding file for final version of chapter 1\\
% |cdocsfi2.tex|&forwarding file for final version of chapter 2\\
% \end{tabular}
% \end{center}
% Each of the eight files can be compiled directly by the \LaTeX{} compiler.
%
% %%%%%%%%%%%%%%%%%%%%%%%%%%%%%%%%%%%%%%
% \paragraph{Main File.}
%
% The main file is called |cdocsamp.tex|.
%
% Load the \textsf{childdoc} definitions and
% declare the filename for the main document:
%    \begin{macrocode}
\input{childdoc.def}
\childdocmain{}
%    \end{macrocode}

% Optional override for |\version| flag:
%    \begin{macrocode}
%%\ifchilddoc\else\providecommand{\version}{draft}\fi
%    \end{macrocode}

% Define the default values for the |\version| flag
% (|final| for the main file and |draft| for childs):
%    \begin{macrocode}
\ifchilddoc
\providecommand{\version}{draft}
\else
\providecommand{\version}{final}
\fi
%    \end{macrocode}

% Load the standard document class:
%    \begin{macrocode}
\documentclass[12pt]{article}
%    \end{macrocode}

% Start the document body:
%    \begin{macrocode}
\begin{document}
%    \end{macrocode}

% Declare a title page.
% Print title, part of document being processed and version flag:
%    \begin{macrocode}
\addtocounter{page}{-1}
\begin{center}
{\LARGE\bfseries{}childdoc example\par}
\vspace{1cm}
\ifchilddoc
\ifchilddocmanual part\else chapter\fi:
`\childdocname' of `\childdocjob'\par
\else
main document: `\childdocjob'\par
\fi
version: \version\par
\end{center}
\newpage
%    \end{macrocode}

% Manually include selected file,
% otherwise process as usual:
%    \begin{macrocode}
\ifchilddocmanual
\section*{part `\childdocname'}
\input{\childdocname}
\else
%    \end{macrocode}

% Include the two chapters:
%    \begin{macrocode}
\include{cdocsch1}
\include{cdocsch2}
%    \end{macrocode}

% Include the two parts unless only chapters should be displayed:
%    \begin{macrocode}
\ifchilddoc\else
\section{part three}
\input{cdocspt3}
\section{part four}
\input{cdocspt4}
\fi
%    \end{macrocode}

% Process as usual until here:
%    \begin{macrocode}
\fi
%    \end{macrocode}

% End of document body:
%    \begin{macrocode}
\end{document}
%    \end{macrocode}
%\iffalse
%</samplemain>
%\fi
%
% %%%%%%%%%%%%%%%%%%%%%%%%%%%%%%%%%%%%%%
% \paragraph{Chapter Include Files.}
%
% The include files are called |cdocsch1.tex| and |cdocsch2.tex|.
%
%\iffalse
%<*samplechap1|samplechap2>
%\fi

% Optional override for |\version| flag:
%    \begin{macrocode}
%%\providecommand{\version}{final}
%    \end{macrocode}

% Include the main document:
%    \begin{macrocode}
\input{childdoc.def}
\childdocof{cdocsamp}
%    \end{macrocode}

%\iffalse
%</samplechap1|samplechap2>
%\fi
%
%\iffalse
%<*samplechap1>
%\fi
% Some text for chapter 1:
%    \begin{macrocode}
\section{one}
some text in chapter one
%    \end{macrocode}

%\iffalse
%</samplechap1>
%\fi
% Some text for chapter 2:
%\iffalse
%<*samplechap2>
%\fi
%    \begin{macrocode}
\section{two}
more text in chapter two
%    \end{macrocode}

%\iffalse
%</samplechap2>
%\fi
%
% %%%%%%%%%%%%%%%%%%%%%%%%%%%%%%%%%%%%%%
% \paragraph{Part Include Files.}
%
% The include files are called |cdocspt3.tex| and |cdocspt4.tex|.
%
%\iffalse
%<*samplepart3|samplepart4>
%\fi

% Optional override for |\version| flag:
%    \begin{macrocode}
%%\providecommand{\version}{final}
%    \end{macrocode}

% Include the main document:
%    \begin{macrocode}
\input{childdoc.def}
\childdocby{cdocsamp}
%    \end{macrocode}

%\iffalse
%</samplepart3|samplepart4>
%\fi
%
%\iffalse
%<*samplepart3>
%\fi
% Some text for part 3:
%    \begin{macrocode}
some text in part three
%    \end{macrocode}

%\iffalse
%</samplepart3>
%\fi
% Some text for part 4:
%\iffalse
%<*samplepart4>
%\fi
%    \begin{macrocode}
more text in part four
%    \end{macrocode}

%\iffalse
%</samplepart4>
%\fi
%
% %%%%%%%%%%%%%%%%%%%%%%%%%%%%%%%%%%%%%%
% \paragraph{Forwarding for a Complete Draft.}
%
% The following forwarding file |cdocsdrf.tex|
% compiles the main document in draft mode:
%\iffalse
%<*sampledraft>
%\fi
%    \begin{macrocode}
\def\version{draft}
\input{childdoc.def}
\childdocforward{cdocsamp}
%    \end{macrocode}

%\iffalse
%</sampledraft>
%\fi
%
% %%%%%%%%%%%%%%%%%%%%%%%%%%%%%%%%%%%%%%
% \paragraph{Forwarding for Final Version of the Chapters.}
%
% The following forwarding files |cdocsfn1.tex| and |cdocsfn2.tex|
% (with identical content)
% compile the final versions of the child documents
% |cdocsch1.tex| and |cdocsch2.tex|, respectively:
%\iffalse
%<*samplefinal>
%\fi
%    \begin{macrocode}
\def\version{final}
\input{childdoc.def}
\childdocforwardprefix[cdocsamp]{cdocsfn}{cdocsch}
%    \end{macrocode}

%\iffalse
%</samplefinal>
%\fi
%
% %%%%%%%%%%%%%%%%%%%%%%%%%%%%%%%%%%%%%%
% \paragraph{Command Line Processing.}
%
% The following three command lines generate the output files
% |cdocscld|, |cdocscl1| and |cdocscl2|
% which should be identical to
% |cdocsdrf|, |cdocsch1| and |cdocsfn2|, respectively:
% \begin{center}
% \begin{tabular}{l}
% |latex -jobname cdocscld \|\\
% |  "\def\version{draft}\input{childdoc.def}\childdocforward{cdocsamp}"|\\
% |latex -jobname cdocscl1 \|\\
% |  "\input{childdoc.def}\childdocforward[cdocsamp]{cdocsch1}"|\\
% |latex -jobname cdocscl2 \|\\
% |  "\def\version{final}\input{childdoc.def}\childdocforward{cdocsch2}"|
% \end{tabular}
% \end{center}
% Note that the trailing backslash on each first line
% merely continues the input to the second line
% (for convenient cut ant paste).
% Furthermore, the command |latex| can be replaced by any
% of its alternative versions such as |pdflatex|.
%
% %%%%%%%%%%%%%%%%%%%%%%%%%%%%%%%%%%%%%%%%%%%%%%%%%%%%%%%%%%%%%%%%%%%%%%%%%%%%%%
% %%%%%%%%%%%%%%%%%%%%%%%%%%%%%%%%%%%%%%%%%%%%%%%%%%%%%%%%%%%%%%%%%%%%%%%%%%%%%%
% \section{Implementation}
%\iffalse
%<*package>
%\fi
%
% This section describes the definitions file |childdoc.def|.

% The definitions cannot be loaded using |\usepackage| or |\RequirePackage|
% which has a mechanism to prevent loading a style file more than once.
% When loading the definitions by means of |\input|
% multiple instances have to be prevented manually:
%\iffalse
%This code needs to be before the `\ProvidesFile' directive
%which is defined at the beginning of this file.
%Therefore it is also placed there and commented out here.
%</package>
%<*discard>
%\fi
%    \begin{macrocode}
\ifdefined\childdocmain\endinput\fi
%    \end{macrocode}
%\iffalse
%</discard>
%<*package>
%\fi
%
% \macro{\ifchilddoc}
% \macro{\ifchilddocmanual}
% The conditional |\ifchilddoc| tells whether a
% child (true) or main (false) document is being compiled.
% The conditional |\ifchilddocmanual| tells whether
% the |\includeonly| mechanism is used (false) or
% the selection of child files must be performed manually (true).
% The definitions initialise to false:
%    \begin{macrocode}
\newif\ifchilddoc
\newif\ifchilddocmanual
%    \end{macrocode}

% \macro{\childdocname}
% \macro{\childdocjob}
% The macro |\childdocname| stores the name of the main document
% to be compiled. The macro |\childdocjob| stores the name of
% the document on which the \LaTeX{} compiler was originally invoked.
% The content of |\jobname| cannot be compared
% to filenames specified in the source due to different catcodes.
% The following code rescans |\jobname|, stores the result
% in |\childdocname| and saves a copy in |\childdocjob|:
%    \begin{macrocode}
\edef\childdocname{\scantokens\expandafter{\jobname\noexpand}}
\let\childdocjob\childdocname
%    \end{macrocode}

% \macro{\childdocdisable}
% The macro |\childdocdisable| prevents the main file
% from being processed more than once.
% At this stage, the main document command |\childdocmain|
% is assumed to be called once again where it should do nothing.
% Any subsequent call to it should prevent
% a secondary processing of the main document
% It overwrites the forwarding commands
% |\childdocof| and |\childdocforward|
% with empty macros to prevent further inclusions of the main document:
%    \begin{macrocode}
\newcommand{\childdocdisable}
{
  \renewcommand{\childdocmain}[1]{\renewcommand{\childdocmain}[1]{\endinput}}
  \renewcommand{\childdocof}[1]{}
  \renewcommand{\childdocby}[2][]{}
  \renewcommand{\childdocforward}[2][]{}
  \renewcommand{\childdocdisable}{}
}
%    \end{macrocode}

% \macro{\childdocmain}
% The macro |\childdocmain| is to be called at the top of the main file
% with nothing or the main filename (without extension) as argument.
% First, it breaks loops.
% If the argument is not empty and does not match |\childdocname|
% (which is set by the first inclusion of |childdoc.def|),
% |\ifchilddoc| is set to true, |\includeonly| is applied to the child file
% and |\jobname| is set to the main file
% (for proper handling of |.aux| files):
%    \begin{macrocode}
\newcommand{\childdocmain}[1]
{
  \childdocdisable\childdocmain{}
  \if?#1?\else
    \begingroup
      \def\childdoctmp{#1}
      \ifx\childdoctmp\childdocname
        \def\childdoctmp{}
      \else
        \def\childdoctmp
        {
          \childdoctrue
          \includeonly{\childdocname}
          \def\childdocjob{#1}
          \def\jobname{#1}
        }
      \fi
      \expandafter
    \endgroup
    \childdoctmp
  \fi
}
%    \end{macrocode}

% \macro{\childdocof}
% The command |\childdocof| redirects
% compilation to the main file |#1|.
%    \begin{macrocode}
\newcommand{\childdocof}[1]
{
  \childdocdisable
  \childdoctrue
  \includeonly{\childdocname}
  \def\jobname{#1}
  \def\childdocjob{#1}
  \input{#1}
}
%    \end{macrocode}

% \macro{\childdocby}
% The command |\childdocby| ....
%    \begin{macrocode}
\newcommand{\childdocby}[2][]
{
  \childdocdisable
  \childdoctrue
  \childdocmanualtrue
  \if?#1?\else
    \def\jobname{#2}
  \fi
  \def\childdocjob{#2}
  \input{#2}
  \endinput
}
%    \end{macrocode}

% \macro{\childdocforward}
% The command |\childdocforward| redirects
% compilation to the main file or
% (if the optional argument is given) a child file.
% Parameters are set as if the main file
% or a child file starting with |\childdocof| was compiled.
% Then compilation is handed over to the main file:
%    \begin{macrocode}
\newcommand{\childdocforward}[2][]
{
  \begingroup
    \if?#1?
      \def\childdoctmp
      {
        \def\childdocname{#2}
        \def\childdocjob{#2}
        \def\jobname{#2}
        \input{#2}
        \endinput
      }
    \else
      \def\childdoctmp
      {
        \childdocdisable
        \def\childdocname{#2}
        \childdoctrue
        \includeonly{#2}
        \def\childdocjob{#1}
        \def\jobname{#1}
        \input{#1}
        \endinput
      }
    \fi
    \expandafter
  \endgroup
  \childdoctmp
}
%    \end{macrocode}

% \macro{\childdocforwardprefix}
% The command |\childdocforwardprefix| redirects
% compilation to the main or a child file by means of a pattern.
% The prefix |#1| in the current filename is replaced by |#2|
% and the suffix of the current filename is kept
% (it is assumed that the filename does not contain the substring `|~~~|'
% which is used as a delimiter).
% Compilation is handed over to the new file by |\childdocforward|:
%    \begin{macrocode}
\newcommand{\childdocforwardprefix}[3][]
{
  \begingroup
    \def\childdocextract #2##1~~~{\def\childdoctmp{\childdocforward[#1]{#3##1}}}
    \expandafter\childdocextract\childdocname~~~
    \expandafter
  \endgroup
  \childdoctmp
}
%    \end{macrocode}

% \macro{\childdoc}
% The deprecated macro |\childdoc| is a legacy version of |\childdocmain|:
%    \begin{macrocode}
\newcommand{\childdoc}{\childdocmain}
%    \end{macrocode}

% \macro{\childdocredirect}
% The deprecated macro |\childdocredirect| is a legacy version
% of |\childdocforward| and |\childdocforwardprefix|:
%    \begin{macrocode}
\newcommand{\childdocredirect}[2][]
{
  \begingroup
    \if?#1?
      \def\childdoctmp{\childdocforward{#2}}
    \else
      \def\childdoctmp{\childdocforwardprefix{#1}{#2}}
    \fi
    \expandafter
  \endgroup
  \childdoctmp
}
%    \end{macrocode}

%\iffalse
%</package>
%\fi
%
\endinput

\childdocof{cdocsamp}
%    \end{macrocode}

%\iffalse
%</samplechap1|samplechap2>
%\fi
%
%\iffalse
%<*samplechap1>
%\fi
% Some text for chapter 1:
%    \begin{macrocode}
\section{one}
some text in chapter one
%    \end{macrocode}

%\iffalse
%</samplechap1>
%\fi
% Some text for chapter 2:
%\iffalse
%<*samplechap2>
%\fi
%    \begin{macrocode}
\section{two}
more text in chapter two
%    \end{macrocode}

%\iffalse
%</samplechap2>
%\fi
%
% %%%%%%%%%%%%%%%%%%%%%%%%%%%%%%%%%%%%%%
% \paragraph{Part Include Files.}
%
% The include files are called |cdocspt3.tex| and |cdocspt4.tex|.
%
%\iffalse
%<*samplepart3|samplepart4>
%\fi

% Optional override for |\version| flag:
%    \begin{macrocode}
%%\providecommand{\version}{final}
%    \end{macrocode}

% Include the main document:
%    \begin{macrocode}
% \iffalse
%
% childdoc.dtx Copyright (C) 2017-2018 Niklas Beisert
%
% This work may be distributed and/or modified under the
% conditions of the LaTeX Project Public License, either version 1.3
% of this license or (at your option) any later version.
% The latest version of this license is in
%   http://www.latex-project.org/lppl.txt
% and version 1.3 or later is part of all distributions of LaTeX
% version 2005/12/01 or later.
%
% This work has the LPPL maintenance status `maintained'.
%
% The Current Maintainer of this work is Niklas Beisert.
%
% This work consists of the files childdoc.dtx and childdoc.ins
% and the derived files childdoc.def and cdocsamp.tex with
% cdocsch1.tex, cdocsch2.tex, cdocsdrf.tex, cdocsfn1.tex, cdocsfn2.tex.
%
%<package>\ifdefined\childdocmain\endinput\fi
%<package>\ProvidesFile{childdoc.def}[2018/12/30 v2.0 child document driver]
%<samplemain>\ProvidesFile{cdocsamp.tex}[2018/12/30 v2.0 sample for childdoc]
%<*driver>
%\ProvidesFile{childdoc.drv}[2018/12/30 v2.0 childdoc reference manual file]
\PassOptionsToClass{10pt,a4paper}{article}
\documentclass{ltxdoc}

\usepackage[margin=35mm]{geometry}
\usepackage{hyperref}
\usepackage{hyperxmp}
\usepackage[usenames]{color}

\hypersetup{colorlinks=true}
\hypersetup{pdfstartview=FitH}
\hypersetup{pdfpagemode=UseNone}
\hypersetup{pdfsource={}}
\hypersetup{pdflang={en-UK}}
\hypersetup{pdfcopyright={Copyright 2017-2018 Niklas Beisert.
  This work may be distributed and/or modified under the
  conditions of the LaTeX Project Public License, either version 1.3
  of this license or (at your option) any later version.}}
\hypersetup{pdflicenseurl={http://www.latex-project.org/lppl.txt}}
\hypersetup{pdfcontactaddress={ETH Zurich, ITP, HIT K,
  Wolfgang-Pauli-Strasse 27}}
\hypersetup{pdfcontactpostcode={8093}}
\hypersetup{pdfcontactcity={Zurich}}
\hypersetup{pdfcontactcountry={Switzerland}}
\hypersetup{pdfcontactemail={nbeisert@itp.phys.ethz.ch}}
\hypersetup{pdfcontacturl={http://people.phys.ethz.ch/\xmptilde nbeisert/}}

\newcommand{\secref}[1]{\hyperref[#1]{section \ref*{#1}}}

\parskip1ex
\parindent0pt
\let\olditemize\itemize
\def\itemize{\olditemize\parskip0pt}

\begin{document}

\title{The \textsf{childdoc} Package}
\hypersetup{pdftitle={The childdoc Package}}
\author{Niklas Beisert\\[2ex]
  Institut f\"ur Theoretische Physik\\
  Eidgen\"ossische Technische Hochschule Z\"urich\\
  Wolfgang-Pauli-Strasse 27, 8093 Z\"urich, Switzerland\\[1ex]
  \href{mailto:nbeisert@itp.phys.ethz.ch}
  {\texttt{nbeisert@itp.phys.ethz.ch}}}
\hypersetup{pdfauthor={Niklas Beisert}}
\hypersetup{pdfsubject={Manual for the LaTeX2e Package childdoc}}
\date{30 December 2018, \textsf{v2.0}}
\maketitle

\begin{abstract}\noindent
\textsf{childdoc} is a \LaTeXe{} package
that enables the direct compilation
of document sections included by |\include|
to individual files.
\end{abstract}

\begingroup
\parskip0ex
\tableofcontents
\endgroup

%%%%%%%%%%%%%%%%%%%%%%%%%%%%%%%%%%%%%%%%%%%%%%%%%%%%%%%%%%%%%%%%%%%%%%%%%%%%%%%%
%%%%%%%%%%%%%%%%%%%%%%%%%%%%%%%%%%%%%%%%%%%%%%%%%%%%%%%%%%%%%%%%%%%%%%%%%%%%%%%%
\section{Introduction}

\LaTeX{} provides a mechanism to structure a large document (such as a book)
into a main file and several child files (containing the chapters)
using the |\include| command.
This mechanism is beneficial for documents
which span hundreds of pages in order to
make the source file(s) more manageable.
Moreover, compilation can be restricted to
selected child files by means of the |\includeonly| command.
The latter feature can be used to reduce the compilation time while editing
(this was significantly more useful in the earlier days of \LaTeX{})
or to generate a smaller document which is easier to navigate.
Another application of |\includeonly| is to generate
documents consisting of selected parts of the complete document.

However, there are a few drawbacks of the plain |\include| mechanism:
\begin{itemize}
\item
The child files cannot be compiled on their own,
they can only be compiled via the main file.
A naive editing environment
(such as a text editor with an option
to have the current file processed by \LaTeX)
may require one to switch to the main file before compiling;
attempting to compile the child file produces errors.
\item
The main file must be modified (each time)
to adjust the |\includeonly| command
to the present needs. This easily leaves the main file in a messy state.
\item
The generated document will always carry the filename
of the main document. This is inconvenient if
several child files are to be compiled and
to be kept for distribution.
\end{itemize}

The present package provides a simple interface
to make child files individually compilable by \LaTeX{}.
Compiling a child file then has the same effect as compiling
the main file with an |\includeonly| command
to select the appropriate child.
Moreover the generated document will carry the name of the child
rather than the main file.
This resolves all three above issues.

This feature is meant to make the editing of books,
thesis documents and lecture notes somewhat more convenient.
However, the package can also be used efficiently for
composing a series of documents (such as exercise sheets)
which are typically distributed individually.
It then assists the author in generating the individual documents
(potentially in different versions)
as well as a document containing the collected series.
Another application is in developing style files
or other kinds of included material
where compilation of the style file could redirect
to a sample or test file.

%%%%%%%%%%%%%%%%%%%%%%%%%%%%%%%%%%%%%%%%%%%%%%%%%%%%%%%%%%%%%%%%%%%%%%%%%%%%%%%%
%%%%%%%%%%%%%%%%%%%%%%%%%%%%%%%%%%%%%%%%%%%%%%%%%%%%%%%%%%%%%%%%%%%%%%%%%%%%%%%%
\section{Usage}

First of all, the package \textsf{childdoc} is \emph{not} a standard
\LaTeXe{} |.sty| style file! Therefore it needs to be invoked in
a non-standard way.

%%%%%%%%%%%%%%%%%%%%%%%%%%%%%%%%%%%%%%%%%%%%%%%%%%%%%%%%%%%%%%%%%%%%%%%%%%%%%%%%
\subsection{Included Files}
\label{sec:include}

%%%%%%%%%%%%%%%%%%%%%%%%%%%%%%%%%%%%%%%%
\DescribeMacro{\childdocmain}
To use the package, add the commands
\begin{center}
\begin{tabular}{l}
|\input{childdoc.def}|\\
|\childdocmain{}|\\
\end{tabular}
\end{center}
at the very top of the main \LaTeX{} file,
in particular \emph{before} the |\documentclass| statement!
The argument of |\childdocmain| should be left empty
(but it must be present).

%%%%%%%%%%%%%%%%%%%%%%%%%%%%%%%%%%%%%%%%
\DescribeMacro{\childdocof}
Furthermore, add the commands
\begin{center}
\begin{tabular}{l}
|\input{childdoc.def}|\\
|\childdocof{|\textit{main}|}|\\
\end{tabular}
\end{center}
at the top of every child file \textit{child}
which is included by |\include{|\textit{child}|}|
from within the main file
(or at least for those files to be compiled individually).
The argument \textit{main} must be the filename of the main file.

There are a couple of
considerations in setting up the main and child documents:

%%%%%%%%%%%%%%%%%%%%%%%%%%%%%%%%%%%%%%%%
\paragraph{Restrictions.}

Please note the following restrictions:
\begin{itemize}
\item
|\childdocmain| must be called with one argument \textit{main}
to ensure compatibility with earlier version of the package.
It must either be empty (|\childdocmain{}|)
or precisely match the filename of the main file in which it is specified.
See \secref{sec:detection} for further information.
\item
The filename \textit{main} must be specified without the |.tex| extension.
\item
The filename \textit{main} is case sensitive
(even in case-insensitive file systems)
due to internal string comparison.
\item
The argument \textit{main} should be fully expanded, it cannot be a macro.
\item
Subdirectories and special characters should be avoided in filenames.
\item
The command |\childdocmain{|\textit{main}|}| must be followed by a whitespace.
It should not be followed immediately by another command
or by a comment mark `|%|'.
This is because the \TeX{} parser reads the token immediately following
the argument of |\childdocmain| and puts it
at the beginning of every child section;
however, a white\-space is ignored.
\end{itemize}

%%%%%%%%%%%%%%%%%%%%%%%%%%%%%%%%%%%%%%%%
\paragraph{Content of Main File.}

It is advisable to place all content in the child files included by |\include|.
Any output contained in the main file will appear in all child documents
unless suppressed manually;
it cannot be suppressed automatically by the |\includeonly| directive
and thus should normally be avoided.
A method to include some content in the main file
by means of conditional processing is described in \secref{sec:conditional}.

%%%%%%%%%%%%%%%%%%%%%%%%%%%%%%%%%%%%%%%%
\paragraph{Page Numbering.}

When only a part of the document is compiled,
the appropriate numbering of pages
(as well as other status parameters)
is determined from the |.aux| files.
The latter contain information from previous passes.
However this information needs to propagate through
all intermediate child documents.
Therefore the page numbering in child documents may well
be inconsistent until the complete document is compiled at least once.

A useful (if unconventional) way to always ensure a consistent
page numbering is to restart the numbering in each child document
and denote the pages by `\textit{child}|.|\textit{page}'
where \textit{child} represents the chapter/section number of the child file.
This can be achieved by the command
|\numberwithin{page}{|\textit{child}|}|
of the \textsf{amsmath} package
where \textit{child} can be |chapter| or |section|
depending on the chosen structuring.
Alternatively, one can modify the macro |\thepage| appropriately
and reset the counter |page| at the start of each child file.

%%%%%%%%%%%%%%%%%%%%%%%%%%%%%%%%%%%%%%%%%%%%%%%%%%%%%%%%%%%%%%%%%%%%%%%%%%%%%%%%
\subsection{Conditional Processing}
\label{sec:conditional}

The package provides a mechanism to compile different versions
of a document. To customise the versions further some conditional processing
can come in handy to distinguish which version is being compiled.
The package provides two macros to describe the compilation context:

%%%%%%%%%%%%%%%%%%%%%%%%%%%%%%%%%%%%%%%%
\DescribeMacro{\ifchilddoc}
The conditional |\ifchilddoc| distinguishes between the compilation of
child documents and the main document:
%
\begin{center}
|\ifchilddoc |\textit{child-code}| |[|\||else |\textit{main-code}]| \||fi|
\end{center}

%%%%%%%%%%%%%%%%%%%%%%%%%%%%%%%%%%%%%%%%
\DescribeMacro{\childdocname}
\DescribeMacro{\childdocjob}
The macro |\childdocname| contains the filename (without extension)
of the main or child file being processed.
Note that |\childdocjob| will always contain the name of the main file.

%%%%%%%%%%%%%%%%%%%%%%%%%%%%%%%%%%%%%%%%
\paragraph{Title Page.}

Conditional processing can be used to include a title or banner page
in the main document when proper precautions are taken.
Importantly, the code in the main file should ensure that the page counter
(as well as other status parameters which are stored in the |.aux| files)
takes the same value after the conditional processing.
Otherwise the page numbers may take divergent values
depending on which part is compiled.

For example, a title page could be declared by:
%
\begin{center}
\begin{tabular}{l}
|\ifchilddoc\||else|\\
|\addtocounter{page}{-1}|\\
\textit{code for title page}\\
|\newpage|\\
|\||fi|
\end{tabular}
\end{center}
%
A banner page for the child documents can be generated by:
%
\begin{center}
\begin{tabular}{l}
|\ifchilddoc|\\
|\addtocounter{page}{-1}|\\
\textit{code for banner page}\\
|\newpage|\\
|\||fi|
\end{tabular}
\end{center}
%
Here one could write a message such as:
\begin{center}
|This is the part \childdocname{} of \childdocjob{}.|
\end{center}

%%%%%%%%%%%%%%%%%%%%%%%%%%%%%%%%%%%%%%%%%%%%%%%%%%%%%%%%%%%%%%%%%%%%%%%%%%%%%%%%
\subsection{Flags}
\label{sec:flags}

The package makes it easy to generate different versions
of the main or child documents.
To this end compilation flags can be defined
and assigned different default values.
They will be particularly useful in conjunction
with the forwarding mechanism described in \secref{sec:forward}.

For example, it may be useful to have a flag |\version|
which can be set to |draft| or |final|.
The document source will contain some conditional code
depending on the value of |\version|.
Suppose further, the flag should default to |final| for the main file
and to |draft| for child files
which is a natural assignment for editing the document.
This is achieved by placing the following code
in the preamble of the main document
(below the |\childdocmain| directive):
%
\begin{center}
\begin{tabular}{l}
|\ifchilddoc|\\
|\providecommand{\version}{draft}|\\
|\||else|\\
|\providecommand{\version}{final}|\\
|\||fi|
\end{tabular}
\end{center}
%
The definition by |\providecommand| makes sure
that previous definitions are not overwritten.
Further statements |\providecommand{\version}{...}|
can thus be added before the above code to override it.

For the main file, one might add a line
(between |\childdocmain| and the above block)
%
\begin{center}
|%\ifchilddoc\||else\providecommand{\version}{draft}\||fi|
\end{center}
%
which can be uncommented to produce a draft version.
Likewise one can add a line to the very top of a child file
(above the |\childdocof{|\textit{main}|}| directive)
%
\begin{center}
|%\providecommand{\version}{final}|
\end{center}
%
which can be uncommented to produce the final version of this child document.

%%%%%%%%%%%%%%%%%%%%%%%%%%%%%%%%%%%%%%%%%%%%%%%%%%%%%%%%%%%%%%%%%%%%%%%%%%%%%%%%
\subsection{Forwarding}
\label{sec:forward}

Different versions of the main or child documents
using compilation flags as described in \secref{sec:flags}
can be (permanently) stored in different files
for convenient compilation, viewing and distribution.
To this end, the package defines a command
to pass on compilation to a different file:

%%%%%%%%%%%%%%%%%%%%%%%%%%%%%%%%%%%%%%%%
\DescribeMacro{\childdocforward}
The command |\childdocforward| redirects processing to
another source file:
%
\begin{center}
\begin{tabular}{l}
|\input{childdoc.def}|\\
|\childdocforward[|\textit{main}|]{|\textit{dest}|}|\\
\end{tabular}
\end{center}
%
The argument \textit{dest} is the destination file
(without extension).
It should be the main file or one of the child files.
Note that further \textsf{childdoc} directives
such as |\childdocof| and |\childdocforward|
in the indicated file will be processed in this form.
The optional argument \textit{main}
passes on directly to the main file \textit{main}
while pretending to compile the child \textit{dest}.
This form behaves as if \textit{dest}
issues |\childdocof{|\textit{main}|}| right away,
and no further \textsf{childdoc} directives will be processed.

%%%%%%%%%%%%%%%%%%%%%%%%%%%%%%%%%%%%%%%%
\DescribeMacro{\...prefix}
In the alternative form |\childdocforwardprefix|,
%
\begin{center}
\begin{tabular}{l}
|\input{childdoc.def}|\\
|\childdocforwardprefix[|\textit{main}|]{|\textit{prefix}|}{|\textit{dest}|}|
\end{tabular}
\end{center}
%
the destination file is determined by a pattern
depending on the current file:
To make this work, the current file must be called
`{\textit{prefix}\hspace{0.2em}\textit{suffix}}'
with \textit{prefix} matching precisely the argument.
Processing is then passed on to the file
`{\textit{dest}\hspace{0.2em}\textit{suffix}}'.
Surely, the same effect is achieved by
directly specifying the
argument `{\textit{dest}\hspace{0.2em}\textit{suffix}}'
in the first form.
However, that requires to set up a different file
for each child. With the alternative form of the command
all these files can have exactly the same content
which simplifies setting them up and maintaining them.

For example, the following file |draft.tex|
with a compilation flag |\version| as described in \secref{sec:flags}
compiles the main document as a draft:
%
\begin{center}
\begin{tabular}{l}
|\def\version{draft}|\\
|\input{childdoc.def}|\\
|\childdocforward{|\textit{main}|}|
\end{tabular}
\end{center}
%
Likewise, the following files |final|\textit{nn}|.tex|
compile the final version of the child document
|child|\textit{nn}|.tex|:
%
\begin{center}
\begin{tabular}{l}
|\def\version{final}|\\
|\input{childdoc.def}|\\
|\childdocforwardprefix{final}{child}|
\end{tabular}
\end{center}
%

Note that when several versions of a main file and/or of each child file
are to be generated, it may be convenient to set up a |Makefile| or
shell script to automatise the process.

%%%%%%%%%%%%%%%%%%%%%%%%%%%%%%%%%%%%%%%%%%%%%%%%%%%%%%%%%%%%%%%%%%%%%%%%%%%%%%%%
\subsection{Command Line Processing}
\label{sec:commandline}

The effect of redirection files can also be achieved by invoking
the \LaTeX{} compiler with a more elaborate command line.
Most conveniently this should be done as part
of a shell script or a |Makefile|.

When using \textsf{childdoc} in the main file, the following
command lines effectively perform a redirection
(note that depending on the shell being used,
backslashes may have to be doubled: `|\|' $\to$ `|\\|'):
%
\begin{center}
|... -jobname "|\textit{target}|" |\\|"|[\textit{flags}]%
|\input{childdoc.def}\childdocforward[|\textit{main}|]{|\textit{dest}|}"|
\end{center}
%
Here \textit{target} is the name of the output file,
\textit{main} is the name of the main file
and \textit{dest} is the name of the main or child file to be processed
(all filenames without extensions).
The optional argument \textit{main} can be omitted
if \textit{main} matches \textit{dest}.
Optionally, compilation \textit{flags} can be defined via |\def| commands.
This command line makes the \TeX{} engine believe
it is compiling the file \textit{target}
whose content is specified as the latter parameter.
The provided code then forwards the processing to
\textit{main} or \textit{dest} as described in \secref{sec:forward}.

%%%%%%%%%%%%%%%%%%%%%%%%%%%%%%%%%%%%%%%%%%%%%%%%%%%%%%%%%%%%%%%%%%%%%%%%%%%%%%%%
\subsection{Include by Input}
\label{sec:input}

Including child documents by |\include| has some restrictions by design.
Most notably, the content of a child document always occupies
its own set of pages; pages cannot be shared between child documents.
Usually, this behaviour makes perfect sense
because each child document contain an essential part of the document.
However, in some situations it may be desirable to compose
a document from a collection of parts
without having mandatory page breaks between then.
For this case, the package
provides a mechanism to include parts
by |\input| which can also be processed individually.
However, by construction this mechanism
requires manual handling of the content to be output.

%%%%%%%%%%%%%%%%%%%%%%%%%%%%%%%%%%%%%%%%
\DescribeMacro{\ifchilddocmanual}
The main file should be prepared as usual, see \secref{sec:include}.
However, the document body must make a distinction
between processing of an individual part and of the main document, e.g.:
%
\begin{center}
\begin{tabular}{l}
|\ifchilddocmanual|\\
|\input{\childdocname}|\\
|\||else|\\
\textit{document body with }|\input{|\textit{part}|}|\\
|\||fi|
\end{tabular}
\end{center}
%
The conditional |\ifchilddocmanual| is true whenever
a part to be included by |\input| is being compiled,
and the name of the part is stored in |\childdocname|.

%%%%%%%%%%%%%%%%%%%%%%%%%%%%%%%%%%%%%%%%
\DescribeMacro{\childdocby}
Each part to be included by |\input| should start with:
%
\begin{center}
\begin{tabular}{l}
|\input{childdoc.def}|\\
|\childdocby{|\textit{main}|}|\\
\end{tabular}
\end{center}
%
The directive |\childdocby| is similar to |\childdocof|
described in \secref{sec:include},
but the subsequent selection of content must be done manually.
To that end, both |\ifchilddoc| and |\ifchilddocmanual|
will be true upon processing of a part,
and the name of the part is stored in |\childdocname|.
Note that |\jobname| will be set to the filename of the current part
so that each part receives an individual |.aux| file
that does not interfere with the |.aux| file(s) of the main document.
This behaviour can be altered by the alternative form
|\childdocby[*]{|\textit{main}|}| (with a non-empty optional argument)
which uses the |.aux| file of the main document
by setting |\jobname| to \textit{main}.

%%%%%%%%%%%%%%%%%%%%%%%%%%%%%%%%%%%%%%%%%%%%%%%%%%%%%%%%%%%%%%%%%%%%%%%%%%%%%%%%
\subsection{Driver Development}
\label{sec:driver}

The \textsf{childdoc} mechanism can also be use for the development
of definition files such as \LaTeX{} styles or classes.
This case differs from the above setup with multiple parts
included by |\include| in that no |\includeonly| should be invoked.
This can be achieved by starting the include file
(before |\ProvidesPackage|) with:
%
\begin{center}
\begin{tabular}{l}
|\input{childdoc.def}|\\
|\childdocforward{|\textit{main}|}|\\
\end{tabular}
\end{center}
%
or alternatively with:
%
\begin{center}
\begin{tabular}{l}
|\input{childdoc.def}|\\
|\childdocby{|\textit{main}|}|\\
\end{tabular}
\end{center}
%
Both forms have slightly different effects as described above.
The main file is prepared as usual, see \secref{sec:include}.

%%%%%%%%%%%%%%%%%%%%%%%%%%%%%%%%%%%%%%%%%%%%%%%%%%%%%%%%%%%%%%%%%%%%%%%%%%%%%%%%
\subsection{Legacy Detection}
\label{sec:detection}

The directive |\childdocmain| in the main file can detect
whether the complete document or merely a child is to be compiled
even without using the directive |\childdocof|.
This method is deprecated because it is less robust
and there is no compelling reason to use it;
it is merely provided for backward compatibility
and it may be removed in future versions.

If the detection mechanism is to be used,
it is mandatory to correctly specify
the filename of the main file as the argument of |\childdocmain|:
%
\begin{center}
\begin{tabular}{l}
|\input{childdoc.def}|\\
|\childdocmain{|\textit{main}|}|\\
\end{tabular}
\end{center}
%
If |\jobname| does not match the argument \textit{main} of |\childdocmain|,
it is assumed that |\jobname| points to the child file to be compiled.
When using |\childdocmain| with the main file specified as argument,
it suffices to start a child file
with just |\input{|\textit{main}|}|
without loading of the package and using |\childdocof|.
If instead all processing is done
with the appropriate \textsf{childdoc} directives,
the argument of \textit{main} of |\childdocmain| can be empty.

An alternative version of the command line processing described
in \secref{sec:commandline} using the detection mechanism reads:
%
\begin{center}
|... -jobname "|\textit{target}|" "|[\textit{flags}]%
[|\def\jobname{|\textit{dest}|}|]|\input{|\textit{main}|}"|
\end{center}

%%%%%%%%%%%%%%%%%%%%%%%%%%%%%%%%%%%%%%%%%%%%%%%%%%%%%%%%%%%%%%%%%%%%%%%%%%%%%%%%
\subsection{Manual Code}
\label{sec:manual}

In case one cannot be certain whether the definitions file |childdoc.def|
is installed on the target \TeX{} distribution
and one prefers not to ship it,
it is conceivable to paste a few relevant commands into the sources.

To that end, drop all statements |\input{childdoc.def}|
and perform the replacements as outlined below.
Instead of |\childdocmain{|\textit{main}|}| add the following code
to the top of the main file:
%
\begin{center}
\begin{tabular}{l}
|\||ifdefined\childdocname\endinput\||fi\newif\ifchilddoc|\\
|\edef\childdocname{\scantokens\expandafter{\jobname\noexpand}}|\\
|\def\childdocmain{|\textit{main}|}\||ifx\childdocmain\childdocname\||else|\\
|\childdoctrue\includeonly{\childdocname}\let\jobname\childdocmain\||fi|\\
\end{tabular}
\end{center}
%
Instead of |\childdocof{|\textit{main}|}| just include the main file
at the top of each child file:
%
\begin{center}
|\input{|\textit{main}|}|
\end{center}
%
A simple redirection |\childdocforward{|\textit{dest}|}| is achieved by:
%
\begin{center}
|\def\jobname{|\textit{dest}|}\input{\jobname}|
\end{center}
%
The redirection with prefix
|\childdocforwardprefix[|\textit{prefix}|]{|\textit{dest}|}|
is accomplished by:
%
\begin{center}
\begin{tabular}{l}
|{\edef\jobname{\scantokens\expandafter{\jobname\noexpand}}|\\
|\def\redirectjob |\textit{prefix}|#1~~~{\gdef\jobname{|\textit{dest}|#1}}|\\
|\expandafter\redirectjob\jobname~~~}\input{\jobname}|
\end{tabular}
\end{center}

In an alternative approach,
child documents can be compiled by a specific command line
without additional code or specific definitions:
%
\begin{center}
|... -jobname "|\textit{target}|" "|[\textit{flags}]%
|\includeonly{|\textit{dest}|}\input{|\textit{main}|}"|
\end{center}
%

%%%%%%%%%%%%%%%%%%%%%%%%%%%%%%%%%%%%%%%%%%%%%%%%%%%%%%%%%%%%%%%%%%%%%%%%%%%%%%%%
%%%%%%%%%%%%%%%%%%%%%%%%%%%%%%%%%%%%%%%%%%%%%%%%%%%%%%%%%%%%%%%%%%%%%%%%%%%%%%%%
\section{Information}

%%%%%%%%%%%%%%%%%%%%%%%%%%%%%%%%%%%%%%%%%%%%%%%%%%%%%%%%%%%%%%%%%%%%%%%%%%%%%%%%
\subsection{Copyright}

Copyright \copyright{} 2017--2018 Niklas Beisert

This work may be distributed and/or modified under the
conditions of the \LaTeX{} Project Public License, either version 1.3
of this license or (at your option) any later version.
The latest version of this license is in
  \url{http://www.latex-project.org/lppl.txt}
and version 1.3 or later is part of all distributions of \LaTeX{}
version 2005/12/01 or later.

This work has the LPPL maintenance status `maintained'.

The Current Maintainer of this work is Niklas Beisert.

This work consists of the files |README.txt|, |childdoc.ins| and |childdoc.dtx|
as well as the derived files |childdoc.def|, |cdocsamp.tex|
with |cdocsch1.tex|, |cdocsch2.tex|, |cdocspt3.tex|, |cdocspt4.tex|,
|cdocsdrf.tex|, |cdocsfn1.tex|, |cdocsfn2.tex|
as well as |childdoc.pdf|.

%%%%%%%%%%%%%%%%%%%%%%%%%%%%%%%%%%%%%%%%%%%%%%%%%%%%%%%%%%%%%%%%%%%%%%%%%%%%%%%%
\subsection{Files and Installation}

The package consists of the files:
%
\begin{center}
\begin{tabular}{ll}
    |README.txt|   & readme file \\
    |childdoc.ins| & installation file \\
    |childdoc.dtx| & source file \\
    |childdoc.def| & definition file \\
    |cdocsamp.tex| & sample main file \\
    |cdocsch1.tex| & sample include file \\
    |cdocsch2.tex| & sample include file \\
    |cdocspt3.tex| & sample part file \\
    |cdocspt4.tex| & sample part file \\
    |cdocsdrf.tex| & sample redirection file \\
    |cdocsfn1.tex| & sample redirection file \\
    |cdocsfn2.tex| & sample redirection file \\
    |childdoc.pdf| & manual
\end{tabular}
\end{center}
%
The distribution consists of the files
|README.txt|, |childdoc.ins| and |childdoc.dtx|.
%
\begin{itemize}
\item
Run (pdf)\LaTeX{} on |childdoc.dtx|
to compile the manual |childdoc.pdf| (this file).
\item
Run \LaTeX{} on |childdoc.ins| to create the definitions file |childdoc.def|
and the sample |cdocsamp.tex| with include files
|cdocsch1.tex|, |cdocsch2.tex|, |cdocspt3.tex|, |cdocspt4.tex|,
|cdocsdrf.tex|, |cdocsfn1.tex|, |cdocsfn2.tex|.
Then copy the file |childdoc.def| to an appropriate directory of your \LaTeX{}
distribution, e.g.\ \textit{texmf-root}|/tex/latex/childdoc|.
\end{itemize}

%%%%%%%%%%%%%%%%%%%%%%%%%%%%%%%%%%%%%%%%%%%%%%%%%%%%%%%%%%%%%%%%%%%%%%%%%%%%%%%%
\subsection{Related CTAN Packages}

There are several other packages which offer a similar functionality:
%
\begin{itemize}
\item
The packages
\href{http://ctan.org/pkg/docmute}{\textsf{docmute}},
\href{http://ctan.org/pkg/includex}{\textsf{includex}} and
\href{http://ctan.org/pkg/standalone}{\textsf{standalone}}
provide commands to include only the document body of
a child file thus allowing both files to be compiled individually.
\item
The packages \href{http://ctan.org/pkg/subdocs}{\textsf{subdocs}}
and \href{http://ctan.org/pkg/subfiles}{\textsf{subfiles}}
provide structures in which the main and child documents can be
encapsulated and allowing them to be compiled individually.
The inclusion mechanism is different from the conventional |\include|.
\item
The package \href{http://ctan.org/pkg/combine}{\textsf{combine}}
is an elaborate solution to combine several documents into one.
\end{itemize}
%
See also the CTAN topic \href{http://ctan.org/topic/subdocs}{\textsf{subdocs}}
for further related packages.
The present package differs from the above solutions in that
a document structure constructed with the conventional |\include| mechanism
just needs two extra commands at the top of every file
such that all constituent files can be compiled individually.

%%%%%%%%%%%%%%%%%%%%%%%%%%%%%%%%%%%%%%%%%%%%%%%%%%%%%%%%%%%%%%%%%%%%%%%%%%%%%%%%
%\subsection{Feature Suggestions}
%
%The following is a list of features which may be useful for future
%versions of this package:
%%
%\begin{itemize}
%\item
%\ldots
%\end{itemize}

%%%%%%%%%%%%%%%%%%%%%%%%%%%%%%%%%%%%%%%%%%%%%%%%%%%%%%%%%%%%%%%%%%%%%%%%%%%%%%%%
\subsection{Revision History}

%%%%%%%%%%%%%%%%%%%%%%%%%%%%%%%%%%%%%%%%
\paragraph{v2.0:} 2018/12/30

\begin{itemize}
\item
immediate forward processing
\item
added |\childdocby| mechanism
\item
manual restructured
\end{itemize}

%%%%%%%%%%%%%%%%%%%%%%%%%%%%%%%%%%%%%%%%
\paragraph{v1.6:} 2018/01/17

\begin{itemize}
\item
application for development of include files
\item
corrections to manual
\end{itemize}

%%%%%%%%%%%%%%%%%%%%%%%%%%%%%%%%%%%%%%%%
\paragraph{v1.5:} 2017/05/21

\begin{itemize}
\item
more complete structuring introduced
\item
|\childdocof| introduced
\item
|\childdoc| renamed to |\childdocmain|
\item
|\childredirect| renamed to |\childdocforward| and |\childdocforwardprefix|
and functionality expanded
\end{itemize}

%%%%%%%%%%%%%%%%%%%%%%%%%%%%%%%%%%%%%%%%
\paragraph{v1.0:} 2017/04/27

\begin{itemize}
\item
manual and install package
\item
first version published on CTAN
\end{itemize}

%%%%%%%%%%%%%%%%%%%%%%%%%%%%%%%%%%%%%%%%
\paragraph{v0.6:} 2017/04/26

\begin{itemize}
\item
redirection mechanism added
\end{itemize}

%%%%%%%%%%%%%%%%%%%%%%%%%%%%%%%%%%%%%%%%
\paragraph{v0.5:} 2017/04/26

\begin{itemize}
\item
functionality in definition file
\end{itemize}


%%%%%%%%%%%%%%%%%%%%%%%%%%%%%%%%%%%%%%%%%%%%%%%%%%%%%%%%%%%%%%%%%%%%%%%%%%%%%%%%
%%%%%%%%%%%%%%%%%%%%%%%%%%%%%%%%%%%%%%%%%%%%%%%%%%%%%%%%%%%%%%%%%%%%%%%%%%%%%%%%
%%%%%%%%%%%%%%%%%%%%%%%%%%%%%%%%%%%%%%%%%%%%%%%%%%%%%%%%%%%%%%%%%%%%%%%%%%%%%%%%
\appendix

\settowidth\MacroIndent{\rmfamily\scriptsize 000\ }

 \DocInput{childdoc.dtx}

\end{document}
%</driver>
% \fi
%
% %%%%%%%%%%%%%%%%%%%%%%%%%%%%%%%%%%%%%%%%%%%%%%%%%%%%%%%%%%%%%%%%%%%%%%%%%%%%%%
% %%%%%%%%%%%%%%%%%%%%%%%%%%%%%%%%%%%%%%%%%%%%%%%%%%%%%%%%%%%%%%%%%%%%%%%%%%%%%%
% \section{Sample}
%\iffalse
%<*samplemain>
%\fi
%
% The following presents a sample document
% with two chapters, two parts, a title page,
% a compile flag as well as three forwarding files to set the flag.
% It consists of eight |.tex| files:
% \begin{center}
% \begin{tabular}{ll}
% |cdocsamp.tex|&main file\\
% |cdocsch1.tex|&include file for chapter 1\\
% |cdocsch2.tex|&include file for chapter 2\\
% |cdocspt3.tex|&include file for part 3\\
% |cdocspt4.tex|&include file for part 4\\
% |cdocsdrf.tex|&forwarding file for main file in draft mode\\
% |cdocsfi1.tex|&forwarding file for final version of chapter 1\\
% |cdocsfi2.tex|&forwarding file for final version of chapter 2\\
% \end{tabular}
% \end{center}
% Each of the eight files can be compiled directly by the \LaTeX{} compiler.
%
% %%%%%%%%%%%%%%%%%%%%%%%%%%%%%%%%%%%%%%
% \paragraph{Main File.}
%
% The main file is called |cdocsamp.tex|.
%
% Load the \textsf{childdoc} definitions and
% declare the filename for the main document:
%    \begin{macrocode}
\input{childdoc.def}
\childdocmain{}
%    \end{macrocode}

% Optional override for |\version| flag:
%    \begin{macrocode}
%%\ifchilddoc\else\providecommand{\version}{draft}\fi
%    \end{macrocode}

% Define the default values for the |\version| flag
% (|final| for the main file and |draft| for childs):
%    \begin{macrocode}
\ifchilddoc
\providecommand{\version}{draft}
\else
\providecommand{\version}{final}
\fi
%    \end{macrocode}

% Load the standard document class:
%    \begin{macrocode}
\documentclass[12pt]{article}
%    \end{macrocode}

% Start the document body:
%    \begin{macrocode}
\begin{document}
%    \end{macrocode}

% Declare a title page.
% Print title, part of document being processed and version flag:
%    \begin{macrocode}
\addtocounter{page}{-1}
\begin{center}
{\LARGE\bfseries{}childdoc example\par}
\vspace{1cm}
\ifchilddoc
\ifchilddocmanual part\else chapter\fi:
`\childdocname' of `\childdocjob'\par
\else
main document: `\childdocjob'\par
\fi
version: \version\par
\end{center}
\newpage
%    \end{macrocode}

% Manually include selected file,
% otherwise process as usual:
%    \begin{macrocode}
\ifchilddocmanual
\section*{part `\childdocname'}
\input{\childdocname}
\else
%    \end{macrocode}

% Include the two chapters:
%    \begin{macrocode}
\include{cdocsch1}
\include{cdocsch2}
%    \end{macrocode}

% Include the two parts unless only chapters should be displayed:
%    \begin{macrocode}
\ifchilddoc\else
\section{part three}
\input{cdocspt3}
\section{part four}
\input{cdocspt4}
\fi
%    \end{macrocode}

% Process as usual until here:
%    \begin{macrocode}
\fi
%    \end{macrocode}

% End of document body:
%    \begin{macrocode}
\end{document}
%    \end{macrocode}
%\iffalse
%</samplemain>
%\fi
%
% %%%%%%%%%%%%%%%%%%%%%%%%%%%%%%%%%%%%%%
% \paragraph{Chapter Include Files.}
%
% The include files are called |cdocsch1.tex| and |cdocsch2.tex|.
%
%\iffalse
%<*samplechap1|samplechap2>
%\fi

% Optional override for |\version| flag:
%    \begin{macrocode}
%%\providecommand{\version}{final}
%    \end{macrocode}

% Include the main document:
%    \begin{macrocode}
\input{childdoc.def}
\childdocof{cdocsamp}
%    \end{macrocode}

%\iffalse
%</samplechap1|samplechap2>
%\fi
%
%\iffalse
%<*samplechap1>
%\fi
% Some text for chapter 1:
%    \begin{macrocode}
\section{one}
some text in chapter one
%    \end{macrocode}

%\iffalse
%</samplechap1>
%\fi
% Some text for chapter 2:
%\iffalse
%<*samplechap2>
%\fi
%    \begin{macrocode}
\section{two}
more text in chapter two
%    \end{macrocode}

%\iffalse
%</samplechap2>
%\fi
%
% %%%%%%%%%%%%%%%%%%%%%%%%%%%%%%%%%%%%%%
% \paragraph{Part Include Files.}
%
% The include files are called |cdocspt3.tex| and |cdocspt4.tex|.
%
%\iffalse
%<*samplepart3|samplepart4>
%\fi

% Optional override for |\version| flag:
%    \begin{macrocode}
%%\providecommand{\version}{final}
%    \end{macrocode}

% Include the main document:
%    \begin{macrocode}
\input{childdoc.def}
\childdocby{cdocsamp}
%    \end{macrocode}

%\iffalse
%</samplepart3|samplepart4>
%\fi
%
%\iffalse
%<*samplepart3>
%\fi
% Some text for part 3:
%    \begin{macrocode}
some text in part three
%    \end{macrocode}

%\iffalse
%</samplepart3>
%\fi
% Some text for part 4:
%\iffalse
%<*samplepart4>
%\fi
%    \begin{macrocode}
more text in part four
%    \end{macrocode}

%\iffalse
%</samplepart4>
%\fi
%
% %%%%%%%%%%%%%%%%%%%%%%%%%%%%%%%%%%%%%%
% \paragraph{Forwarding for a Complete Draft.}
%
% The following forwarding file |cdocsdrf.tex|
% compiles the main document in draft mode:
%\iffalse
%<*sampledraft>
%\fi
%    \begin{macrocode}
\def\version{draft}
\input{childdoc.def}
\childdocforward{cdocsamp}
%    \end{macrocode}

%\iffalse
%</sampledraft>
%\fi
%
% %%%%%%%%%%%%%%%%%%%%%%%%%%%%%%%%%%%%%%
% \paragraph{Forwarding for Final Version of the Chapters.}
%
% The following forwarding files |cdocsfn1.tex| and |cdocsfn2.tex|
% (with identical content)
% compile the final versions of the child documents
% |cdocsch1.tex| and |cdocsch2.tex|, respectively:
%\iffalse
%<*samplefinal>
%\fi
%    \begin{macrocode}
\def\version{final}
\input{childdoc.def}
\childdocforwardprefix[cdocsamp]{cdocsfn}{cdocsch}
%    \end{macrocode}

%\iffalse
%</samplefinal>
%\fi
%
% %%%%%%%%%%%%%%%%%%%%%%%%%%%%%%%%%%%%%%
% \paragraph{Command Line Processing.}
%
% The following three command lines generate the output files
% |cdocscld|, |cdocscl1| and |cdocscl2|
% which should be identical to
% |cdocsdrf|, |cdocsch1| and |cdocsfn2|, respectively:
% \begin{center}
% \begin{tabular}{l}
% |latex -jobname cdocscld \|\\
% |  "\def\version{draft}\input{childdoc.def}\childdocforward{cdocsamp}"|\\
% |latex -jobname cdocscl1 \|\\
% |  "\input{childdoc.def}\childdocforward[cdocsamp]{cdocsch1}"|\\
% |latex -jobname cdocscl2 \|\\
% |  "\def\version{final}\input{childdoc.def}\childdocforward{cdocsch2}"|
% \end{tabular}
% \end{center}
% Note that the trailing backslash on each first line
% merely continues the input to the second line
% (for convenient cut ant paste).
% Furthermore, the command |latex| can be replaced by any
% of its alternative versions such as |pdflatex|.
%
% %%%%%%%%%%%%%%%%%%%%%%%%%%%%%%%%%%%%%%%%%%%%%%%%%%%%%%%%%%%%%%%%%%%%%%%%%%%%%%
% %%%%%%%%%%%%%%%%%%%%%%%%%%%%%%%%%%%%%%%%%%%%%%%%%%%%%%%%%%%%%%%%%%%%%%%%%%%%%%
% \section{Implementation}
%\iffalse
%<*package>
%\fi
%
% This section describes the definitions file |childdoc.def|.

% The definitions cannot be loaded using |\usepackage| or |\RequirePackage|
% which has a mechanism to prevent loading a style file more than once.
% When loading the definitions by means of |\input|
% multiple instances have to be prevented manually:
%\iffalse
%This code needs to be before the `\ProvidesFile' directive
%which is defined at the beginning of this file.
%Therefore it is also placed there and commented out here.
%</package>
%<*discard>
%\fi
%    \begin{macrocode}
\ifdefined\childdocmain\endinput\fi
%    \end{macrocode}
%\iffalse
%</discard>
%<*package>
%\fi
%
% \macro{\ifchilddoc}
% \macro{\ifchilddocmanual}
% The conditional |\ifchilddoc| tells whether a
% child (true) or main (false) document is being compiled.
% The conditional |\ifchilddocmanual| tells whether
% the |\includeonly| mechanism is used (false) or
% the selection of child files must be performed manually (true).
% The definitions initialise to false:
%    \begin{macrocode}
\newif\ifchilddoc
\newif\ifchilddocmanual
%    \end{macrocode}

% \macro{\childdocname}
% \macro{\childdocjob}
% The macro |\childdocname| stores the name of the main document
% to be compiled. The macro |\childdocjob| stores the name of
% the document on which the \LaTeX{} compiler was originally invoked.
% The content of |\jobname| cannot be compared
% to filenames specified in the source due to different catcodes.
% The following code rescans |\jobname|, stores the result
% in |\childdocname| and saves a copy in |\childdocjob|:
%    \begin{macrocode}
\edef\childdocname{\scantokens\expandafter{\jobname\noexpand}}
\let\childdocjob\childdocname
%    \end{macrocode}

% \macro{\childdocdisable}
% The macro |\childdocdisable| prevents the main file
% from being processed more than once.
% At this stage, the main document command |\childdocmain|
% is assumed to be called once again where it should do nothing.
% Any subsequent call to it should prevent
% a secondary processing of the main document
% It overwrites the forwarding commands
% |\childdocof| and |\childdocforward|
% with empty macros to prevent further inclusions of the main document:
%    \begin{macrocode}
\newcommand{\childdocdisable}
{
  \renewcommand{\childdocmain}[1]{\renewcommand{\childdocmain}[1]{\endinput}}
  \renewcommand{\childdocof}[1]{}
  \renewcommand{\childdocby}[2][]{}
  \renewcommand{\childdocforward}[2][]{}
  \renewcommand{\childdocdisable}{}
}
%    \end{macrocode}

% \macro{\childdocmain}
% The macro |\childdocmain| is to be called at the top of the main file
% with nothing or the main filename (without extension) as argument.
% First, it breaks loops.
% If the argument is not empty and does not match |\childdocname|
% (which is set by the first inclusion of |childdoc.def|),
% |\ifchilddoc| is set to true, |\includeonly| is applied to the child file
% and |\jobname| is set to the main file
% (for proper handling of |.aux| files):
%    \begin{macrocode}
\newcommand{\childdocmain}[1]
{
  \childdocdisable\childdocmain{}
  \if?#1?\else
    \begingroup
      \def\childdoctmp{#1}
      \ifx\childdoctmp\childdocname
        \def\childdoctmp{}
      \else
        \def\childdoctmp
        {
          \childdoctrue
          \includeonly{\childdocname}
          \def\childdocjob{#1}
          \def\jobname{#1}
        }
      \fi
      \expandafter
    \endgroup
    \childdoctmp
  \fi
}
%    \end{macrocode}

% \macro{\childdocof}
% The command |\childdocof| redirects
% compilation to the main file |#1|.
%    \begin{macrocode}
\newcommand{\childdocof}[1]
{
  \childdocdisable
  \childdoctrue
  \includeonly{\childdocname}
  \def\jobname{#1}
  \def\childdocjob{#1}
  \input{#1}
}
%    \end{macrocode}

% \macro{\childdocby}
% The command |\childdocby| ....
%    \begin{macrocode}
\newcommand{\childdocby}[2][]
{
  \childdocdisable
  \childdoctrue
  \childdocmanualtrue
  \if?#1?\else
    \def\jobname{#2}
  \fi
  \def\childdocjob{#2}
  \input{#2}
  \endinput
}
%    \end{macrocode}

% \macro{\childdocforward}
% The command |\childdocforward| redirects
% compilation to the main file or
% (if the optional argument is given) a child file.
% Parameters are set as if the main file
% or a child file starting with |\childdocof| was compiled.
% Then compilation is handed over to the main file:
%    \begin{macrocode}
\newcommand{\childdocforward}[2][]
{
  \begingroup
    \if?#1?
      \def\childdoctmp
      {
        \def\childdocname{#2}
        \def\childdocjob{#2}
        \def\jobname{#2}
        \input{#2}
        \endinput
      }
    \else
      \def\childdoctmp
      {
        \childdocdisable
        \def\childdocname{#2}
        \childdoctrue
        \includeonly{#2}
        \def\childdocjob{#1}
        \def\jobname{#1}
        \input{#1}
        \endinput
      }
    \fi
    \expandafter
  \endgroup
  \childdoctmp
}
%    \end{macrocode}

% \macro{\childdocforwardprefix}
% The command |\childdocforwardprefix| redirects
% compilation to the main or a child file by means of a pattern.
% The prefix |#1| in the current filename is replaced by |#2|
% and the suffix of the current filename is kept
% (it is assumed that the filename does not contain the substring `|~~~|'
% which is used as a delimiter).
% Compilation is handed over to the new file by |\childdocforward|:
%    \begin{macrocode}
\newcommand{\childdocforwardprefix}[3][]
{
  \begingroup
    \def\childdocextract #2##1~~~{\def\childdoctmp{\childdocforward[#1]{#3##1}}}
    \expandafter\childdocextract\childdocname~~~
    \expandafter
  \endgroup
  \childdoctmp
}
%    \end{macrocode}

% \macro{\childdoc}
% The deprecated macro |\childdoc| is a legacy version of |\childdocmain|:
%    \begin{macrocode}
\newcommand{\childdoc}{\childdocmain}
%    \end{macrocode}

% \macro{\childdocredirect}
% The deprecated macro |\childdocredirect| is a legacy version
% of |\childdocforward| and |\childdocforwardprefix|:
%    \begin{macrocode}
\newcommand{\childdocredirect}[2][]
{
  \begingroup
    \if?#1?
      \def\childdoctmp{\childdocforward{#2}}
    \else
      \def\childdoctmp{\childdocforwardprefix{#1}{#2}}
    \fi
    \expandafter
  \endgroup
  \childdoctmp
}
%    \end{macrocode}

%\iffalse
%</package>
%\fi
%
\endinput

\childdocby{cdocsamp}
%    \end{macrocode}

%\iffalse
%</samplepart3|samplepart4>
%\fi
%
%\iffalse
%<*samplepart3>
%\fi
% Some text for part 3:
%    \begin{macrocode}
some text in part three
%    \end{macrocode}

%\iffalse
%</samplepart3>
%\fi
% Some text for part 4:
%\iffalse
%<*samplepart4>
%\fi
%    \begin{macrocode}
more text in part four
%    \end{macrocode}

%\iffalse
%</samplepart4>
%\fi
%
% %%%%%%%%%%%%%%%%%%%%%%%%%%%%%%%%%%%%%%
% \paragraph{Forwarding for a Complete Draft.}
%
% The following forwarding file |cdocsdrf.tex|
% compiles the main document in draft mode:
%\iffalse
%<*sampledraft>
%\fi
%    \begin{macrocode}
\def\version{draft}
% \iffalse
%
% childdoc.dtx Copyright (C) 2017-2018 Niklas Beisert
%
% This work may be distributed and/or modified under the
% conditions of the LaTeX Project Public License, either version 1.3
% of this license or (at your option) any later version.
% The latest version of this license is in
%   http://www.latex-project.org/lppl.txt
% and version 1.3 or later is part of all distributions of LaTeX
% version 2005/12/01 or later.
%
% This work has the LPPL maintenance status `maintained'.
%
% The Current Maintainer of this work is Niklas Beisert.
%
% This work consists of the files childdoc.dtx and childdoc.ins
% and the derived files childdoc.def and cdocsamp.tex with
% cdocsch1.tex, cdocsch2.tex, cdocsdrf.tex, cdocsfn1.tex, cdocsfn2.tex.
%
%<package>\ifdefined\childdocmain\endinput\fi
%<package>\ProvidesFile{childdoc.def}[2018/12/30 v2.0 child document driver]
%<samplemain>\ProvidesFile{cdocsamp.tex}[2018/12/30 v2.0 sample for childdoc]
%<*driver>
%\ProvidesFile{childdoc.drv}[2018/12/30 v2.0 childdoc reference manual file]
\PassOptionsToClass{10pt,a4paper}{article}
\documentclass{ltxdoc}

\usepackage[margin=35mm]{geometry}
\usepackage{hyperref}
\usepackage{hyperxmp}
\usepackage[usenames]{color}

\hypersetup{colorlinks=true}
\hypersetup{pdfstartview=FitH}
\hypersetup{pdfpagemode=UseNone}
\hypersetup{pdfsource={}}
\hypersetup{pdflang={en-UK}}
\hypersetup{pdfcopyright={Copyright 2017-2018 Niklas Beisert.
  This work may be distributed and/or modified under the
  conditions of the LaTeX Project Public License, either version 1.3
  of this license or (at your option) any later version.}}
\hypersetup{pdflicenseurl={http://www.latex-project.org/lppl.txt}}
\hypersetup{pdfcontactaddress={ETH Zurich, ITP, HIT K,
  Wolfgang-Pauli-Strasse 27}}
\hypersetup{pdfcontactpostcode={8093}}
\hypersetup{pdfcontactcity={Zurich}}
\hypersetup{pdfcontactcountry={Switzerland}}
\hypersetup{pdfcontactemail={nbeisert@itp.phys.ethz.ch}}
\hypersetup{pdfcontacturl={http://people.phys.ethz.ch/\xmptilde nbeisert/}}

\newcommand{\secref}[1]{\hyperref[#1]{section \ref*{#1}}}

\parskip1ex
\parindent0pt
\let\olditemize\itemize
\def\itemize{\olditemize\parskip0pt}

\begin{document}

\title{The \textsf{childdoc} Package}
\hypersetup{pdftitle={The childdoc Package}}
\author{Niklas Beisert\\[2ex]
  Institut f\"ur Theoretische Physik\\
  Eidgen\"ossische Technische Hochschule Z\"urich\\
  Wolfgang-Pauli-Strasse 27, 8093 Z\"urich, Switzerland\\[1ex]
  \href{mailto:nbeisert@itp.phys.ethz.ch}
  {\texttt{nbeisert@itp.phys.ethz.ch}}}
\hypersetup{pdfauthor={Niklas Beisert}}
\hypersetup{pdfsubject={Manual for the LaTeX2e Package childdoc}}
\date{30 December 2018, \textsf{v2.0}}
\maketitle

\begin{abstract}\noindent
\textsf{childdoc} is a \LaTeXe{} package
that enables the direct compilation
of document sections included by |\include|
to individual files.
\end{abstract}

\begingroup
\parskip0ex
\tableofcontents
\endgroup

%%%%%%%%%%%%%%%%%%%%%%%%%%%%%%%%%%%%%%%%%%%%%%%%%%%%%%%%%%%%%%%%%%%%%%%%%%%%%%%%
%%%%%%%%%%%%%%%%%%%%%%%%%%%%%%%%%%%%%%%%%%%%%%%%%%%%%%%%%%%%%%%%%%%%%%%%%%%%%%%%
\section{Introduction}

\LaTeX{} provides a mechanism to structure a large document (such as a book)
into a main file and several child files (containing the chapters)
using the |\include| command.
This mechanism is beneficial for documents
which span hundreds of pages in order to
make the source file(s) more manageable.
Moreover, compilation can be restricted to
selected child files by means of the |\includeonly| command.
The latter feature can be used to reduce the compilation time while editing
(this was significantly more useful in the earlier days of \LaTeX{})
or to generate a smaller document which is easier to navigate.
Another application of |\includeonly| is to generate
documents consisting of selected parts of the complete document.

However, there are a few drawbacks of the plain |\include| mechanism:
\begin{itemize}
\item
The child files cannot be compiled on their own,
they can only be compiled via the main file.
A naive editing environment
(such as a text editor with an option
to have the current file processed by \LaTeX)
may require one to switch to the main file before compiling;
attempting to compile the child file produces errors.
\item
The main file must be modified (each time)
to adjust the |\includeonly| command
to the present needs. This easily leaves the main file in a messy state.
\item
The generated document will always carry the filename
of the main document. This is inconvenient if
several child files are to be compiled and
to be kept for distribution.
\end{itemize}

The present package provides a simple interface
to make child files individually compilable by \LaTeX{}.
Compiling a child file then has the same effect as compiling
the main file with an |\includeonly| command
to select the appropriate child.
Moreover the generated document will carry the name of the child
rather than the main file.
This resolves all three above issues.

This feature is meant to make the editing of books,
thesis documents and lecture notes somewhat more convenient.
However, the package can also be used efficiently for
composing a series of documents (such as exercise sheets)
which are typically distributed individually.
It then assists the author in generating the individual documents
(potentially in different versions)
as well as a document containing the collected series.
Another application is in developing style files
or other kinds of included material
where compilation of the style file could redirect
to a sample or test file.

%%%%%%%%%%%%%%%%%%%%%%%%%%%%%%%%%%%%%%%%%%%%%%%%%%%%%%%%%%%%%%%%%%%%%%%%%%%%%%%%
%%%%%%%%%%%%%%%%%%%%%%%%%%%%%%%%%%%%%%%%%%%%%%%%%%%%%%%%%%%%%%%%%%%%%%%%%%%%%%%%
\section{Usage}

First of all, the package \textsf{childdoc} is \emph{not} a standard
\LaTeXe{} |.sty| style file! Therefore it needs to be invoked in
a non-standard way.

%%%%%%%%%%%%%%%%%%%%%%%%%%%%%%%%%%%%%%%%%%%%%%%%%%%%%%%%%%%%%%%%%%%%%%%%%%%%%%%%
\subsection{Included Files}
\label{sec:include}

%%%%%%%%%%%%%%%%%%%%%%%%%%%%%%%%%%%%%%%%
\DescribeMacro{\childdocmain}
To use the package, add the commands
\begin{center}
\begin{tabular}{l}
|\input{childdoc.def}|\\
|\childdocmain{}|\\
\end{tabular}
\end{center}
at the very top of the main \LaTeX{} file,
in particular \emph{before} the |\documentclass| statement!
The argument of |\childdocmain| should be left empty
(but it must be present).

%%%%%%%%%%%%%%%%%%%%%%%%%%%%%%%%%%%%%%%%
\DescribeMacro{\childdocof}
Furthermore, add the commands
\begin{center}
\begin{tabular}{l}
|\input{childdoc.def}|\\
|\childdocof{|\textit{main}|}|\\
\end{tabular}
\end{center}
at the top of every child file \textit{child}
which is included by |\include{|\textit{child}|}|
from within the main file
(or at least for those files to be compiled individually).
The argument \textit{main} must be the filename of the main file.

There are a couple of
considerations in setting up the main and child documents:

%%%%%%%%%%%%%%%%%%%%%%%%%%%%%%%%%%%%%%%%
\paragraph{Restrictions.}

Please note the following restrictions:
\begin{itemize}
\item
|\childdocmain| must be called with one argument \textit{main}
to ensure compatibility with earlier version of the package.
It must either be empty (|\childdocmain{}|)
or precisely match the filename of the main file in which it is specified.
See \secref{sec:detection} for further information.
\item
The filename \textit{main} must be specified without the |.tex| extension.
\item
The filename \textit{main} is case sensitive
(even in case-insensitive file systems)
due to internal string comparison.
\item
The argument \textit{main} should be fully expanded, it cannot be a macro.
\item
Subdirectories and special characters should be avoided in filenames.
\item
The command |\childdocmain{|\textit{main}|}| must be followed by a whitespace.
It should not be followed immediately by another command
or by a comment mark `|%|'.
This is because the \TeX{} parser reads the token immediately following
the argument of |\childdocmain| and puts it
at the beginning of every child section;
however, a white\-space is ignored.
\end{itemize}

%%%%%%%%%%%%%%%%%%%%%%%%%%%%%%%%%%%%%%%%
\paragraph{Content of Main File.}

It is advisable to place all content in the child files included by |\include|.
Any output contained in the main file will appear in all child documents
unless suppressed manually;
it cannot be suppressed automatically by the |\includeonly| directive
and thus should normally be avoided.
A method to include some content in the main file
by means of conditional processing is described in \secref{sec:conditional}.

%%%%%%%%%%%%%%%%%%%%%%%%%%%%%%%%%%%%%%%%
\paragraph{Page Numbering.}

When only a part of the document is compiled,
the appropriate numbering of pages
(as well as other status parameters)
is determined from the |.aux| files.
The latter contain information from previous passes.
However this information needs to propagate through
all intermediate child documents.
Therefore the page numbering in child documents may well
be inconsistent until the complete document is compiled at least once.

A useful (if unconventional) way to always ensure a consistent
page numbering is to restart the numbering in each child document
and denote the pages by `\textit{child}|.|\textit{page}'
where \textit{child} represents the chapter/section number of the child file.
This can be achieved by the command
|\numberwithin{page}{|\textit{child}|}|
of the \textsf{amsmath} package
where \textit{child} can be |chapter| or |section|
depending on the chosen structuring.
Alternatively, one can modify the macro |\thepage| appropriately
and reset the counter |page| at the start of each child file.

%%%%%%%%%%%%%%%%%%%%%%%%%%%%%%%%%%%%%%%%%%%%%%%%%%%%%%%%%%%%%%%%%%%%%%%%%%%%%%%%
\subsection{Conditional Processing}
\label{sec:conditional}

The package provides a mechanism to compile different versions
of a document. To customise the versions further some conditional processing
can come in handy to distinguish which version is being compiled.
The package provides two macros to describe the compilation context:

%%%%%%%%%%%%%%%%%%%%%%%%%%%%%%%%%%%%%%%%
\DescribeMacro{\ifchilddoc}
The conditional |\ifchilddoc| distinguishes between the compilation of
child documents and the main document:
%
\begin{center}
|\ifchilddoc |\textit{child-code}| |[|\||else |\textit{main-code}]| \||fi|
\end{center}

%%%%%%%%%%%%%%%%%%%%%%%%%%%%%%%%%%%%%%%%
\DescribeMacro{\childdocname}
\DescribeMacro{\childdocjob}
The macro |\childdocname| contains the filename (without extension)
of the main or child file being processed.
Note that |\childdocjob| will always contain the name of the main file.

%%%%%%%%%%%%%%%%%%%%%%%%%%%%%%%%%%%%%%%%
\paragraph{Title Page.}

Conditional processing can be used to include a title or banner page
in the main document when proper precautions are taken.
Importantly, the code in the main file should ensure that the page counter
(as well as other status parameters which are stored in the |.aux| files)
takes the same value after the conditional processing.
Otherwise the page numbers may take divergent values
depending on which part is compiled.

For example, a title page could be declared by:
%
\begin{center}
\begin{tabular}{l}
|\ifchilddoc\||else|\\
|\addtocounter{page}{-1}|\\
\textit{code for title page}\\
|\newpage|\\
|\||fi|
\end{tabular}
\end{center}
%
A banner page for the child documents can be generated by:
%
\begin{center}
\begin{tabular}{l}
|\ifchilddoc|\\
|\addtocounter{page}{-1}|\\
\textit{code for banner page}\\
|\newpage|\\
|\||fi|
\end{tabular}
\end{center}
%
Here one could write a message such as:
\begin{center}
|This is the part \childdocname{} of \childdocjob{}.|
\end{center}

%%%%%%%%%%%%%%%%%%%%%%%%%%%%%%%%%%%%%%%%%%%%%%%%%%%%%%%%%%%%%%%%%%%%%%%%%%%%%%%%
\subsection{Flags}
\label{sec:flags}

The package makes it easy to generate different versions
of the main or child documents.
To this end compilation flags can be defined
and assigned different default values.
They will be particularly useful in conjunction
with the forwarding mechanism described in \secref{sec:forward}.

For example, it may be useful to have a flag |\version|
which can be set to |draft| or |final|.
The document source will contain some conditional code
depending on the value of |\version|.
Suppose further, the flag should default to |final| for the main file
and to |draft| for child files
which is a natural assignment for editing the document.
This is achieved by placing the following code
in the preamble of the main document
(below the |\childdocmain| directive):
%
\begin{center}
\begin{tabular}{l}
|\ifchilddoc|\\
|\providecommand{\version}{draft}|\\
|\||else|\\
|\providecommand{\version}{final}|\\
|\||fi|
\end{tabular}
\end{center}
%
The definition by |\providecommand| makes sure
that previous definitions are not overwritten.
Further statements |\providecommand{\version}{...}|
can thus be added before the above code to override it.

For the main file, one might add a line
(between |\childdocmain| and the above block)
%
\begin{center}
|%\ifchilddoc\||else\providecommand{\version}{draft}\||fi|
\end{center}
%
which can be uncommented to produce a draft version.
Likewise one can add a line to the very top of a child file
(above the |\childdocof{|\textit{main}|}| directive)
%
\begin{center}
|%\providecommand{\version}{final}|
\end{center}
%
which can be uncommented to produce the final version of this child document.

%%%%%%%%%%%%%%%%%%%%%%%%%%%%%%%%%%%%%%%%%%%%%%%%%%%%%%%%%%%%%%%%%%%%%%%%%%%%%%%%
\subsection{Forwarding}
\label{sec:forward}

Different versions of the main or child documents
using compilation flags as described in \secref{sec:flags}
can be (permanently) stored in different files
for convenient compilation, viewing and distribution.
To this end, the package defines a command
to pass on compilation to a different file:

%%%%%%%%%%%%%%%%%%%%%%%%%%%%%%%%%%%%%%%%
\DescribeMacro{\childdocforward}
The command |\childdocforward| redirects processing to
another source file:
%
\begin{center}
\begin{tabular}{l}
|\input{childdoc.def}|\\
|\childdocforward[|\textit{main}|]{|\textit{dest}|}|\\
\end{tabular}
\end{center}
%
The argument \textit{dest} is the destination file
(without extension).
It should be the main file or one of the child files.
Note that further \textsf{childdoc} directives
such as |\childdocof| and |\childdocforward|
in the indicated file will be processed in this form.
The optional argument \textit{main}
passes on directly to the main file \textit{main}
while pretending to compile the child \textit{dest}.
This form behaves as if \textit{dest}
issues |\childdocof{|\textit{main}|}| right away,
and no further \textsf{childdoc} directives will be processed.

%%%%%%%%%%%%%%%%%%%%%%%%%%%%%%%%%%%%%%%%
\DescribeMacro{\...prefix}
In the alternative form |\childdocforwardprefix|,
%
\begin{center}
\begin{tabular}{l}
|\input{childdoc.def}|\\
|\childdocforwardprefix[|\textit{main}|]{|\textit{prefix}|}{|\textit{dest}|}|
\end{tabular}
\end{center}
%
the destination file is determined by a pattern
depending on the current file:
To make this work, the current file must be called
`{\textit{prefix}\hspace{0.2em}\textit{suffix}}'
with \textit{prefix} matching precisely the argument.
Processing is then passed on to the file
`{\textit{dest}\hspace{0.2em}\textit{suffix}}'.
Surely, the same effect is achieved by
directly specifying the
argument `{\textit{dest}\hspace{0.2em}\textit{suffix}}'
in the first form.
However, that requires to set up a different file
for each child. With the alternative form of the command
all these files can have exactly the same content
which simplifies setting them up and maintaining them.

For example, the following file |draft.tex|
with a compilation flag |\version| as described in \secref{sec:flags}
compiles the main document as a draft:
%
\begin{center}
\begin{tabular}{l}
|\def\version{draft}|\\
|\input{childdoc.def}|\\
|\childdocforward{|\textit{main}|}|
\end{tabular}
\end{center}
%
Likewise, the following files |final|\textit{nn}|.tex|
compile the final version of the child document
|child|\textit{nn}|.tex|:
%
\begin{center}
\begin{tabular}{l}
|\def\version{final}|\\
|\input{childdoc.def}|\\
|\childdocforwardprefix{final}{child}|
\end{tabular}
\end{center}
%

Note that when several versions of a main file and/or of each child file
are to be generated, it may be convenient to set up a |Makefile| or
shell script to automatise the process.

%%%%%%%%%%%%%%%%%%%%%%%%%%%%%%%%%%%%%%%%%%%%%%%%%%%%%%%%%%%%%%%%%%%%%%%%%%%%%%%%
\subsection{Command Line Processing}
\label{sec:commandline}

The effect of redirection files can also be achieved by invoking
the \LaTeX{} compiler with a more elaborate command line.
Most conveniently this should be done as part
of a shell script or a |Makefile|.

When using \textsf{childdoc} in the main file, the following
command lines effectively perform a redirection
(note that depending on the shell being used,
backslashes may have to be doubled: `|\|' $\to$ `|\\|'):
%
\begin{center}
|... -jobname "|\textit{target}|" |\\|"|[\textit{flags}]%
|\input{childdoc.def}\childdocforward[|\textit{main}|]{|\textit{dest}|}"|
\end{center}
%
Here \textit{target} is the name of the output file,
\textit{main} is the name of the main file
and \textit{dest} is the name of the main or child file to be processed
(all filenames without extensions).
The optional argument \textit{main} can be omitted
if \textit{main} matches \textit{dest}.
Optionally, compilation \textit{flags} can be defined via |\def| commands.
This command line makes the \TeX{} engine believe
it is compiling the file \textit{target}
whose content is specified as the latter parameter.
The provided code then forwards the processing to
\textit{main} or \textit{dest} as described in \secref{sec:forward}.

%%%%%%%%%%%%%%%%%%%%%%%%%%%%%%%%%%%%%%%%%%%%%%%%%%%%%%%%%%%%%%%%%%%%%%%%%%%%%%%%
\subsection{Include by Input}
\label{sec:input}

Including child documents by |\include| has some restrictions by design.
Most notably, the content of a child document always occupies
its own set of pages; pages cannot be shared between child documents.
Usually, this behaviour makes perfect sense
because each child document contain an essential part of the document.
However, in some situations it may be desirable to compose
a document from a collection of parts
without having mandatory page breaks between then.
For this case, the package
provides a mechanism to include parts
by |\input| which can also be processed individually.
However, by construction this mechanism
requires manual handling of the content to be output.

%%%%%%%%%%%%%%%%%%%%%%%%%%%%%%%%%%%%%%%%
\DescribeMacro{\ifchilddocmanual}
The main file should be prepared as usual, see \secref{sec:include}.
However, the document body must make a distinction
between processing of an individual part and of the main document, e.g.:
%
\begin{center}
\begin{tabular}{l}
|\ifchilddocmanual|\\
|\input{\childdocname}|\\
|\||else|\\
\textit{document body with }|\input{|\textit{part}|}|\\
|\||fi|
\end{tabular}
\end{center}
%
The conditional |\ifchilddocmanual| is true whenever
a part to be included by |\input| is being compiled,
and the name of the part is stored in |\childdocname|.

%%%%%%%%%%%%%%%%%%%%%%%%%%%%%%%%%%%%%%%%
\DescribeMacro{\childdocby}
Each part to be included by |\input| should start with:
%
\begin{center}
\begin{tabular}{l}
|\input{childdoc.def}|\\
|\childdocby{|\textit{main}|}|\\
\end{tabular}
\end{center}
%
The directive |\childdocby| is similar to |\childdocof|
described in \secref{sec:include},
but the subsequent selection of content must be done manually.
To that end, both |\ifchilddoc| and |\ifchilddocmanual|
will be true upon processing of a part,
and the name of the part is stored in |\childdocname|.
Note that |\jobname| will be set to the filename of the current part
so that each part receives an individual |.aux| file
that does not interfere with the |.aux| file(s) of the main document.
This behaviour can be altered by the alternative form
|\childdocby[*]{|\textit{main}|}| (with a non-empty optional argument)
which uses the |.aux| file of the main document
by setting |\jobname| to \textit{main}.

%%%%%%%%%%%%%%%%%%%%%%%%%%%%%%%%%%%%%%%%%%%%%%%%%%%%%%%%%%%%%%%%%%%%%%%%%%%%%%%%
\subsection{Driver Development}
\label{sec:driver}

The \textsf{childdoc} mechanism can also be use for the development
of definition files such as \LaTeX{} styles or classes.
This case differs from the above setup with multiple parts
included by |\include| in that no |\includeonly| should be invoked.
This can be achieved by starting the include file
(before |\ProvidesPackage|) with:
%
\begin{center}
\begin{tabular}{l}
|\input{childdoc.def}|\\
|\childdocforward{|\textit{main}|}|\\
\end{tabular}
\end{center}
%
or alternatively with:
%
\begin{center}
\begin{tabular}{l}
|\input{childdoc.def}|\\
|\childdocby{|\textit{main}|}|\\
\end{tabular}
\end{center}
%
Both forms have slightly different effects as described above.
The main file is prepared as usual, see \secref{sec:include}.

%%%%%%%%%%%%%%%%%%%%%%%%%%%%%%%%%%%%%%%%%%%%%%%%%%%%%%%%%%%%%%%%%%%%%%%%%%%%%%%%
\subsection{Legacy Detection}
\label{sec:detection}

The directive |\childdocmain| in the main file can detect
whether the complete document or merely a child is to be compiled
even without using the directive |\childdocof|.
This method is deprecated because it is less robust
and there is no compelling reason to use it;
it is merely provided for backward compatibility
and it may be removed in future versions.

If the detection mechanism is to be used,
it is mandatory to correctly specify
the filename of the main file as the argument of |\childdocmain|:
%
\begin{center}
\begin{tabular}{l}
|\input{childdoc.def}|\\
|\childdocmain{|\textit{main}|}|\\
\end{tabular}
\end{center}
%
If |\jobname| does not match the argument \textit{main} of |\childdocmain|,
it is assumed that |\jobname| points to the child file to be compiled.
When using |\childdocmain| with the main file specified as argument,
it suffices to start a child file
with just |\input{|\textit{main}|}|
without loading of the package and using |\childdocof|.
If instead all processing is done
with the appropriate \textsf{childdoc} directives,
the argument of \textit{main} of |\childdocmain| can be empty.

An alternative version of the command line processing described
in \secref{sec:commandline} using the detection mechanism reads:
%
\begin{center}
|... -jobname "|\textit{target}|" "|[\textit{flags}]%
[|\def\jobname{|\textit{dest}|}|]|\input{|\textit{main}|}"|
\end{center}

%%%%%%%%%%%%%%%%%%%%%%%%%%%%%%%%%%%%%%%%%%%%%%%%%%%%%%%%%%%%%%%%%%%%%%%%%%%%%%%%
\subsection{Manual Code}
\label{sec:manual}

In case one cannot be certain whether the definitions file |childdoc.def|
is installed on the target \TeX{} distribution
and one prefers not to ship it,
it is conceivable to paste a few relevant commands into the sources.

To that end, drop all statements |\input{childdoc.def}|
and perform the replacements as outlined below.
Instead of |\childdocmain{|\textit{main}|}| add the following code
to the top of the main file:
%
\begin{center}
\begin{tabular}{l}
|\||ifdefined\childdocname\endinput\||fi\newif\ifchilddoc|\\
|\edef\childdocname{\scantokens\expandafter{\jobname\noexpand}}|\\
|\def\childdocmain{|\textit{main}|}\||ifx\childdocmain\childdocname\||else|\\
|\childdoctrue\includeonly{\childdocname}\let\jobname\childdocmain\||fi|\\
\end{tabular}
\end{center}
%
Instead of |\childdocof{|\textit{main}|}| just include the main file
at the top of each child file:
%
\begin{center}
|\input{|\textit{main}|}|
\end{center}
%
A simple redirection |\childdocforward{|\textit{dest}|}| is achieved by:
%
\begin{center}
|\def\jobname{|\textit{dest}|}\input{\jobname}|
\end{center}
%
The redirection with prefix
|\childdocforwardprefix[|\textit{prefix}|]{|\textit{dest}|}|
is accomplished by:
%
\begin{center}
\begin{tabular}{l}
|{\edef\jobname{\scantokens\expandafter{\jobname\noexpand}}|\\
|\def\redirectjob |\textit{prefix}|#1~~~{\gdef\jobname{|\textit{dest}|#1}}|\\
|\expandafter\redirectjob\jobname~~~}\input{\jobname}|
\end{tabular}
\end{center}

In an alternative approach,
child documents can be compiled by a specific command line
without additional code or specific definitions:
%
\begin{center}
|... -jobname "|\textit{target}|" "|[\textit{flags}]%
|\includeonly{|\textit{dest}|}\input{|\textit{main}|}"|
\end{center}
%

%%%%%%%%%%%%%%%%%%%%%%%%%%%%%%%%%%%%%%%%%%%%%%%%%%%%%%%%%%%%%%%%%%%%%%%%%%%%%%%%
%%%%%%%%%%%%%%%%%%%%%%%%%%%%%%%%%%%%%%%%%%%%%%%%%%%%%%%%%%%%%%%%%%%%%%%%%%%%%%%%
\section{Information}

%%%%%%%%%%%%%%%%%%%%%%%%%%%%%%%%%%%%%%%%%%%%%%%%%%%%%%%%%%%%%%%%%%%%%%%%%%%%%%%%
\subsection{Copyright}

Copyright \copyright{} 2017--2018 Niklas Beisert

This work may be distributed and/or modified under the
conditions of the \LaTeX{} Project Public License, either version 1.3
of this license or (at your option) any later version.
The latest version of this license is in
  \url{http://www.latex-project.org/lppl.txt}
and version 1.3 or later is part of all distributions of \LaTeX{}
version 2005/12/01 or later.

This work has the LPPL maintenance status `maintained'.

The Current Maintainer of this work is Niklas Beisert.

This work consists of the files |README.txt|, |childdoc.ins| and |childdoc.dtx|
as well as the derived files |childdoc.def|, |cdocsamp.tex|
with |cdocsch1.tex|, |cdocsch2.tex|, |cdocspt3.tex|, |cdocspt4.tex|,
|cdocsdrf.tex|, |cdocsfn1.tex|, |cdocsfn2.tex|
as well as |childdoc.pdf|.

%%%%%%%%%%%%%%%%%%%%%%%%%%%%%%%%%%%%%%%%%%%%%%%%%%%%%%%%%%%%%%%%%%%%%%%%%%%%%%%%
\subsection{Files and Installation}

The package consists of the files:
%
\begin{center}
\begin{tabular}{ll}
    |README.txt|   & readme file \\
    |childdoc.ins| & installation file \\
    |childdoc.dtx| & source file \\
    |childdoc.def| & definition file \\
    |cdocsamp.tex| & sample main file \\
    |cdocsch1.tex| & sample include file \\
    |cdocsch2.tex| & sample include file \\
    |cdocspt3.tex| & sample part file \\
    |cdocspt4.tex| & sample part file \\
    |cdocsdrf.tex| & sample redirection file \\
    |cdocsfn1.tex| & sample redirection file \\
    |cdocsfn2.tex| & sample redirection file \\
    |childdoc.pdf| & manual
\end{tabular}
\end{center}
%
The distribution consists of the files
|README.txt|, |childdoc.ins| and |childdoc.dtx|.
%
\begin{itemize}
\item
Run (pdf)\LaTeX{} on |childdoc.dtx|
to compile the manual |childdoc.pdf| (this file).
\item
Run \LaTeX{} on |childdoc.ins| to create the definitions file |childdoc.def|
and the sample |cdocsamp.tex| with include files
|cdocsch1.tex|, |cdocsch2.tex|, |cdocspt3.tex|, |cdocspt4.tex|,
|cdocsdrf.tex|, |cdocsfn1.tex|, |cdocsfn2.tex|.
Then copy the file |childdoc.def| to an appropriate directory of your \LaTeX{}
distribution, e.g.\ \textit{texmf-root}|/tex/latex/childdoc|.
\end{itemize}

%%%%%%%%%%%%%%%%%%%%%%%%%%%%%%%%%%%%%%%%%%%%%%%%%%%%%%%%%%%%%%%%%%%%%%%%%%%%%%%%
\subsection{Related CTAN Packages}

There are several other packages which offer a similar functionality:
%
\begin{itemize}
\item
The packages
\href{http://ctan.org/pkg/docmute}{\textsf{docmute}},
\href{http://ctan.org/pkg/includex}{\textsf{includex}} and
\href{http://ctan.org/pkg/standalone}{\textsf{standalone}}
provide commands to include only the document body of
a child file thus allowing both files to be compiled individually.
\item
The packages \href{http://ctan.org/pkg/subdocs}{\textsf{subdocs}}
and \href{http://ctan.org/pkg/subfiles}{\textsf{subfiles}}
provide structures in which the main and child documents can be
encapsulated and allowing them to be compiled individually.
The inclusion mechanism is different from the conventional |\include|.
\item
The package \href{http://ctan.org/pkg/combine}{\textsf{combine}}
is an elaborate solution to combine several documents into one.
\end{itemize}
%
See also the CTAN topic \href{http://ctan.org/topic/subdocs}{\textsf{subdocs}}
for further related packages.
The present package differs from the above solutions in that
a document structure constructed with the conventional |\include| mechanism
just needs two extra commands at the top of every file
such that all constituent files can be compiled individually.

%%%%%%%%%%%%%%%%%%%%%%%%%%%%%%%%%%%%%%%%%%%%%%%%%%%%%%%%%%%%%%%%%%%%%%%%%%%%%%%%
%\subsection{Feature Suggestions}
%
%The following is a list of features which may be useful for future
%versions of this package:
%%
%\begin{itemize}
%\item
%\ldots
%\end{itemize}

%%%%%%%%%%%%%%%%%%%%%%%%%%%%%%%%%%%%%%%%%%%%%%%%%%%%%%%%%%%%%%%%%%%%%%%%%%%%%%%%
\subsection{Revision History}

%%%%%%%%%%%%%%%%%%%%%%%%%%%%%%%%%%%%%%%%
\paragraph{v2.0:} 2018/12/30

\begin{itemize}
\item
immediate forward processing
\item
added |\childdocby| mechanism
\item
manual restructured
\end{itemize}

%%%%%%%%%%%%%%%%%%%%%%%%%%%%%%%%%%%%%%%%
\paragraph{v1.6:} 2018/01/17

\begin{itemize}
\item
application for development of include files
\item
corrections to manual
\end{itemize}

%%%%%%%%%%%%%%%%%%%%%%%%%%%%%%%%%%%%%%%%
\paragraph{v1.5:} 2017/05/21

\begin{itemize}
\item
more complete structuring introduced
\item
|\childdocof| introduced
\item
|\childdoc| renamed to |\childdocmain|
\item
|\childredirect| renamed to |\childdocforward| and |\childdocforwardprefix|
and functionality expanded
\end{itemize}

%%%%%%%%%%%%%%%%%%%%%%%%%%%%%%%%%%%%%%%%
\paragraph{v1.0:} 2017/04/27

\begin{itemize}
\item
manual and install package
\item
first version published on CTAN
\end{itemize}

%%%%%%%%%%%%%%%%%%%%%%%%%%%%%%%%%%%%%%%%
\paragraph{v0.6:} 2017/04/26

\begin{itemize}
\item
redirection mechanism added
\end{itemize}

%%%%%%%%%%%%%%%%%%%%%%%%%%%%%%%%%%%%%%%%
\paragraph{v0.5:} 2017/04/26

\begin{itemize}
\item
functionality in definition file
\end{itemize}


%%%%%%%%%%%%%%%%%%%%%%%%%%%%%%%%%%%%%%%%%%%%%%%%%%%%%%%%%%%%%%%%%%%%%%%%%%%%%%%%
%%%%%%%%%%%%%%%%%%%%%%%%%%%%%%%%%%%%%%%%%%%%%%%%%%%%%%%%%%%%%%%%%%%%%%%%%%%%%%%%
%%%%%%%%%%%%%%%%%%%%%%%%%%%%%%%%%%%%%%%%%%%%%%%%%%%%%%%%%%%%%%%%%%%%%%%%%%%%%%%%
\appendix

\settowidth\MacroIndent{\rmfamily\scriptsize 000\ }

 \DocInput{childdoc.dtx}

\end{document}
%</driver>
% \fi
%
% %%%%%%%%%%%%%%%%%%%%%%%%%%%%%%%%%%%%%%%%%%%%%%%%%%%%%%%%%%%%%%%%%%%%%%%%%%%%%%
% %%%%%%%%%%%%%%%%%%%%%%%%%%%%%%%%%%%%%%%%%%%%%%%%%%%%%%%%%%%%%%%%%%%%%%%%%%%%%%
% \section{Sample}
%\iffalse
%<*samplemain>
%\fi
%
% The following presents a sample document
% with two chapters, two parts, a title page,
% a compile flag as well as three forwarding files to set the flag.
% It consists of eight |.tex| files:
% \begin{center}
% \begin{tabular}{ll}
% |cdocsamp.tex|&main file\\
% |cdocsch1.tex|&include file for chapter 1\\
% |cdocsch2.tex|&include file for chapter 2\\
% |cdocspt3.tex|&include file for part 3\\
% |cdocspt4.tex|&include file for part 4\\
% |cdocsdrf.tex|&forwarding file for main file in draft mode\\
% |cdocsfi1.tex|&forwarding file for final version of chapter 1\\
% |cdocsfi2.tex|&forwarding file for final version of chapter 2\\
% \end{tabular}
% \end{center}
% Each of the eight files can be compiled directly by the \LaTeX{} compiler.
%
% %%%%%%%%%%%%%%%%%%%%%%%%%%%%%%%%%%%%%%
% \paragraph{Main File.}
%
% The main file is called |cdocsamp.tex|.
%
% Load the \textsf{childdoc} definitions and
% declare the filename for the main document:
%    \begin{macrocode}
\input{childdoc.def}
\childdocmain{}
%    \end{macrocode}

% Optional override for |\version| flag:
%    \begin{macrocode}
%%\ifchilddoc\else\providecommand{\version}{draft}\fi
%    \end{macrocode}

% Define the default values for the |\version| flag
% (|final| for the main file and |draft| for childs):
%    \begin{macrocode}
\ifchilddoc
\providecommand{\version}{draft}
\else
\providecommand{\version}{final}
\fi
%    \end{macrocode}

% Load the standard document class:
%    \begin{macrocode}
\documentclass[12pt]{article}
%    \end{macrocode}

% Start the document body:
%    \begin{macrocode}
\begin{document}
%    \end{macrocode}

% Declare a title page.
% Print title, part of document being processed and version flag:
%    \begin{macrocode}
\addtocounter{page}{-1}
\begin{center}
{\LARGE\bfseries{}childdoc example\par}
\vspace{1cm}
\ifchilddoc
\ifchilddocmanual part\else chapter\fi:
`\childdocname' of `\childdocjob'\par
\else
main document: `\childdocjob'\par
\fi
version: \version\par
\end{center}
\newpage
%    \end{macrocode}

% Manually include selected file,
% otherwise process as usual:
%    \begin{macrocode}
\ifchilddocmanual
\section*{part `\childdocname'}
\input{\childdocname}
\else
%    \end{macrocode}

% Include the two chapters:
%    \begin{macrocode}
\include{cdocsch1}
\include{cdocsch2}
%    \end{macrocode}

% Include the two parts unless only chapters should be displayed:
%    \begin{macrocode}
\ifchilddoc\else
\section{part three}
\input{cdocspt3}
\section{part four}
\input{cdocspt4}
\fi
%    \end{macrocode}

% Process as usual until here:
%    \begin{macrocode}
\fi
%    \end{macrocode}

% End of document body:
%    \begin{macrocode}
\end{document}
%    \end{macrocode}
%\iffalse
%</samplemain>
%\fi
%
% %%%%%%%%%%%%%%%%%%%%%%%%%%%%%%%%%%%%%%
% \paragraph{Chapter Include Files.}
%
% The include files are called |cdocsch1.tex| and |cdocsch2.tex|.
%
%\iffalse
%<*samplechap1|samplechap2>
%\fi

% Optional override for |\version| flag:
%    \begin{macrocode}
%%\providecommand{\version}{final}
%    \end{macrocode}

% Include the main document:
%    \begin{macrocode}
\input{childdoc.def}
\childdocof{cdocsamp}
%    \end{macrocode}

%\iffalse
%</samplechap1|samplechap2>
%\fi
%
%\iffalse
%<*samplechap1>
%\fi
% Some text for chapter 1:
%    \begin{macrocode}
\section{one}
some text in chapter one
%    \end{macrocode}

%\iffalse
%</samplechap1>
%\fi
% Some text for chapter 2:
%\iffalse
%<*samplechap2>
%\fi
%    \begin{macrocode}
\section{two}
more text in chapter two
%    \end{macrocode}

%\iffalse
%</samplechap2>
%\fi
%
% %%%%%%%%%%%%%%%%%%%%%%%%%%%%%%%%%%%%%%
% \paragraph{Part Include Files.}
%
% The include files are called |cdocspt3.tex| and |cdocspt4.tex|.
%
%\iffalse
%<*samplepart3|samplepart4>
%\fi

% Optional override for |\version| flag:
%    \begin{macrocode}
%%\providecommand{\version}{final}
%    \end{macrocode}

% Include the main document:
%    \begin{macrocode}
\input{childdoc.def}
\childdocby{cdocsamp}
%    \end{macrocode}

%\iffalse
%</samplepart3|samplepart4>
%\fi
%
%\iffalse
%<*samplepart3>
%\fi
% Some text for part 3:
%    \begin{macrocode}
some text in part three
%    \end{macrocode}

%\iffalse
%</samplepart3>
%\fi
% Some text for part 4:
%\iffalse
%<*samplepart4>
%\fi
%    \begin{macrocode}
more text in part four
%    \end{macrocode}

%\iffalse
%</samplepart4>
%\fi
%
% %%%%%%%%%%%%%%%%%%%%%%%%%%%%%%%%%%%%%%
% \paragraph{Forwarding for a Complete Draft.}
%
% The following forwarding file |cdocsdrf.tex|
% compiles the main document in draft mode:
%\iffalse
%<*sampledraft>
%\fi
%    \begin{macrocode}
\def\version{draft}
\input{childdoc.def}
\childdocforward{cdocsamp}
%    \end{macrocode}

%\iffalse
%</sampledraft>
%\fi
%
% %%%%%%%%%%%%%%%%%%%%%%%%%%%%%%%%%%%%%%
% \paragraph{Forwarding for Final Version of the Chapters.}
%
% The following forwarding files |cdocsfn1.tex| and |cdocsfn2.tex|
% (with identical content)
% compile the final versions of the child documents
% |cdocsch1.tex| and |cdocsch2.tex|, respectively:
%\iffalse
%<*samplefinal>
%\fi
%    \begin{macrocode}
\def\version{final}
\input{childdoc.def}
\childdocforwardprefix[cdocsamp]{cdocsfn}{cdocsch}
%    \end{macrocode}

%\iffalse
%</samplefinal>
%\fi
%
% %%%%%%%%%%%%%%%%%%%%%%%%%%%%%%%%%%%%%%
% \paragraph{Command Line Processing.}
%
% The following three command lines generate the output files
% |cdocscld|, |cdocscl1| and |cdocscl2|
% which should be identical to
% |cdocsdrf|, |cdocsch1| and |cdocsfn2|, respectively:
% \begin{center}
% \begin{tabular}{l}
% |latex -jobname cdocscld \|\\
% |  "\def\version{draft}\input{childdoc.def}\childdocforward{cdocsamp}"|\\
% |latex -jobname cdocscl1 \|\\
% |  "\input{childdoc.def}\childdocforward[cdocsamp]{cdocsch1}"|\\
% |latex -jobname cdocscl2 \|\\
% |  "\def\version{final}\input{childdoc.def}\childdocforward{cdocsch2}"|
% \end{tabular}
% \end{center}
% Note that the trailing backslash on each first line
% merely continues the input to the second line
% (for convenient cut ant paste).
% Furthermore, the command |latex| can be replaced by any
% of its alternative versions such as |pdflatex|.
%
% %%%%%%%%%%%%%%%%%%%%%%%%%%%%%%%%%%%%%%%%%%%%%%%%%%%%%%%%%%%%%%%%%%%%%%%%%%%%%%
% %%%%%%%%%%%%%%%%%%%%%%%%%%%%%%%%%%%%%%%%%%%%%%%%%%%%%%%%%%%%%%%%%%%%%%%%%%%%%%
% \section{Implementation}
%\iffalse
%<*package>
%\fi
%
% This section describes the definitions file |childdoc.def|.

% The definitions cannot be loaded using |\usepackage| or |\RequirePackage|
% which has a mechanism to prevent loading a style file more than once.
% When loading the definitions by means of |\input|
% multiple instances have to be prevented manually:
%\iffalse
%This code needs to be before the `\ProvidesFile' directive
%which is defined at the beginning of this file.
%Therefore it is also placed there and commented out here.
%</package>
%<*discard>
%\fi
%    \begin{macrocode}
\ifdefined\childdocmain\endinput\fi
%    \end{macrocode}
%\iffalse
%</discard>
%<*package>
%\fi
%
% \macro{\ifchilddoc}
% \macro{\ifchilddocmanual}
% The conditional |\ifchilddoc| tells whether a
% child (true) or main (false) document is being compiled.
% The conditional |\ifchilddocmanual| tells whether
% the |\includeonly| mechanism is used (false) or
% the selection of child files must be performed manually (true).
% The definitions initialise to false:
%    \begin{macrocode}
\newif\ifchilddoc
\newif\ifchilddocmanual
%    \end{macrocode}

% \macro{\childdocname}
% \macro{\childdocjob}
% The macro |\childdocname| stores the name of the main document
% to be compiled. The macro |\childdocjob| stores the name of
% the document on which the \LaTeX{} compiler was originally invoked.
% The content of |\jobname| cannot be compared
% to filenames specified in the source due to different catcodes.
% The following code rescans |\jobname|, stores the result
% in |\childdocname| and saves a copy in |\childdocjob|:
%    \begin{macrocode}
\edef\childdocname{\scantokens\expandafter{\jobname\noexpand}}
\let\childdocjob\childdocname
%    \end{macrocode}

% \macro{\childdocdisable}
% The macro |\childdocdisable| prevents the main file
% from being processed more than once.
% At this stage, the main document command |\childdocmain|
% is assumed to be called once again where it should do nothing.
% Any subsequent call to it should prevent
% a secondary processing of the main document
% It overwrites the forwarding commands
% |\childdocof| and |\childdocforward|
% with empty macros to prevent further inclusions of the main document:
%    \begin{macrocode}
\newcommand{\childdocdisable}
{
  \renewcommand{\childdocmain}[1]{\renewcommand{\childdocmain}[1]{\endinput}}
  \renewcommand{\childdocof}[1]{}
  \renewcommand{\childdocby}[2][]{}
  \renewcommand{\childdocforward}[2][]{}
  \renewcommand{\childdocdisable}{}
}
%    \end{macrocode}

% \macro{\childdocmain}
% The macro |\childdocmain| is to be called at the top of the main file
% with nothing or the main filename (without extension) as argument.
% First, it breaks loops.
% If the argument is not empty and does not match |\childdocname|
% (which is set by the first inclusion of |childdoc.def|),
% |\ifchilddoc| is set to true, |\includeonly| is applied to the child file
% and |\jobname| is set to the main file
% (for proper handling of |.aux| files):
%    \begin{macrocode}
\newcommand{\childdocmain}[1]
{
  \childdocdisable\childdocmain{}
  \if?#1?\else
    \begingroup
      \def\childdoctmp{#1}
      \ifx\childdoctmp\childdocname
        \def\childdoctmp{}
      \else
        \def\childdoctmp
        {
          \childdoctrue
          \includeonly{\childdocname}
          \def\childdocjob{#1}
          \def\jobname{#1}
        }
      \fi
      \expandafter
    \endgroup
    \childdoctmp
  \fi
}
%    \end{macrocode}

% \macro{\childdocof}
% The command |\childdocof| redirects
% compilation to the main file |#1|.
%    \begin{macrocode}
\newcommand{\childdocof}[1]
{
  \childdocdisable
  \childdoctrue
  \includeonly{\childdocname}
  \def\jobname{#1}
  \def\childdocjob{#1}
  \input{#1}
}
%    \end{macrocode}

% \macro{\childdocby}
% The command |\childdocby| ....
%    \begin{macrocode}
\newcommand{\childdocby}[2][]
{
  \childdocdisable
  \childdoctrue
  \childdocmanualtrue
  \if?#1?\else
    \def\jobname{#2}
  \fi
  \def\childdocjob{#2}
  \input{#2}
  \endinput
}
%    \end{macrocode}

% \macro{\childdocforward}
% The command |\childdocforward| redirects
% compilation to the main file or
% (if the optional argument is given) a child file.
% Parameters are set as if the main file
% or a child file starting with |\childdocof| was compiled.
% Then compilation is handed over to the main file:
%    \begin{macrocode}
\newcommand{\childdocforward}[2][]
{
  \begingroup
    \if?#1?
      \def\childdoctmp
      {
        \def\childdocname{#2}
        \def\childdocjob{#2}
        \def\jobname{#2}
        \input{#2}
        \endinput
      }
    \else
      \def\childdoctmp
      {
        \childdocdisable
        \def\childdocname{#2}
        \childdoctrue
        \includeonly{#2}
        \def\childdocjob{#1}
        \def\jobname{#1}
        \input{#1}
        \endinput
      }
    \fi
    \expandafter
  \endgroup
  \childdoctmp
}
%    \end{macrocode}

% \macro{\childdocforwardprefix}
% The command |\childdocforwardprefix| redirects
% compilation to the main or a child file by means of a pattern.
% The prefix |#1| in the current filename is replaced by |#2|
% and the suffix of the current filename is kept
% (it is assumed that the filename does not contain the substring `|~~~|'
% which is used as a delimiter).
% Compilation is handed over to the new file by |\childdocforward|:
%    \begin{macrocode}
\newcommand{\childdocforwardprefix}[3][]
{
  \begingroup
    \def\childdocextract #2##1~~~{\def\childdoctmp{\childdocforward[#1]{#3##1}}}
    \expandafter\childdocextract\childdocname~~~
    \expandafter
  \endgroup
  \childdoctmp
}
%    \end{macrocode}

% \macro{\childdoc}
% The deprecated macro |\childdoc| is a legacy version of |\childdocmain|:
%    \begin{macrocode}
\newcommand{\childdoc}{\childdocmain}
%    \end{macrocode}

% \macro{\childdocredirect}
% The deprecated macro |\childdocredirect| is a legacy version
% of |\childdocforward| and |\childdocforwardprefix|:
%    \begin{macrocode}
\newcommand{\childdocredirect}[2][]
{
  \begingroup
    \if?#1?
      \def\childdoctmp{\childdocforward{#2}}
    \else
      \def\childdoctmp{\childdocforwardprefix{#1}{#2}}
    \fi
    \expandafter
  \endgroup
  \childdoctmp
}
%    \end{macrocode}

%\iffalse
%</package>
%\fi
%
\endinput

\childdocforward{cdocsamp}
%    \end{macrocode}

%\iffalse
%</sampledraft>
%\fi
%
% %%%%%%%%%%%%%%%%%%%%%%%%%%%%%%%%%%%%%%
% \paragraph{Forwarding for Final Version of the Chapters.}
%
% The following forwarding files |cdocsfn1.tex| and |cdocsfn2.tex|
% (with identical content)
% compile the final versions of the child documents
% |cdocsch1.tex| and |cdocsch2.tex|, respectively:
%\iffalse
%<*samplefinal>
%\fi
%    \begin{macrocode}
\def\version{final}
% \iffalse
%
% childdoc.dtx Copyright (C) 2017-2018 Niklas Beisert
%
% This work may be distributed and/or modified under the
% conditions of the LaTeX Project Public License, either version 1.3
% of this license or (at your option) any later version.
% The latest version of this license is in
%   http://www.latex-project.org/lppl.txt
% and version 1.3 or later is part of all distributions of LaTeX
% version 2005/12/01 or later.
%
% This work has the LPPL maintenance status `maintained'.
%
% The Current Maintainer of this work is Niklas Beisert.
%
% This work consists of the files childdoc.dtx and childdoc.ins
% and the derived files childdoc.def and cdocsamp.tex with
% cdocsch1.tex, cdocsch2.tex, cdocsdrf.tex, cdocsfn1.tex, cdocsfn2.tex.
%
%<package>\ifdefined\childdocmain\endinput\fi
%<package>\ProvidesFile{childdoc.def}[2018/12/30 v2.0 child document driver]
%<samplemain>\ProvidesFile{cdocsamp.tex}[2018/12/30 v2.0 sample for childdoc]
%<*driver>
%\ProvidesFile{childdoc.drv}[2018/12/30 v2.0 childdoc reference manual file]
\PassOptionsToClass{10pt,a4paper}{article}
\documentclass{ltxdoc}

\usepackage[margin=35mm]{geometry}
\usepackage{hyperref}
\usepackage{hyperxmp}
\usepackage[usenames]{color}

\hypersetup{colorlinks=true}
\hypersetup{pdfstartview=FitH}
\hypersetup{pdfpagemode=UseNone}
\hypersetup{pdfsource={}}
\hypersetup{pdflang={en-UK}}
\hypersetup{pdfcopyright={Copyright 2017-2018 Niklas Beisert.
  This work may be distributed and/or modified under the
  conditions of the LaTeX Project Public License, either version 1.3
  of this license or (at your option) any later version.}}
\hypersetup{pdflicenseurl={http://www.latex-project.org/lppl.txt}}
\hypersetup{pdfcontactaddress={ETH Zurich, ITP, HIT K,
  Wolfgang-Pauli-Strasse 27}}
\hypersetup{pdfcontactpostcode={8093}}
\hypersetup{pdfcontactcity={Zurich}}
\hypersetup{pdfcontactcountry={Switzerland}}
\hypersetup{pdfcontactemail={nbeisert@itp.phys.ethz.ch}}
\hypersetup{pdfcontacturl={http://people.phys.ethz.ch/\xmptilde nbeisert/}}

\newcommand{\secref}[1]{\hyperref[#1]{section \ref*{#1}}}

\parskip1ex
\parindent0pt
\let\olditemize\itemize
\def\itemize{\olditemize\parskip0pt}

\begin{document}

\title{The \textsf{childdoc} Package}
\hypersetup{pdftitle={The childdoc Package}}
\author{Niklas Beisert\\[2ex]
  Institut f\"ur Theoretische Physik\\
  Eidgen\"ossische Technische Hochschule Z\"urich\\
  Wolfgang-Pauli-Strasse 27, 8093 Z\"urich, Switzerland\\[1ex]
  \href{mailto:nbeisert@itp.phys.ethz.ch}
  {\texttt{nbeisert@itp.phys.ethz.ch}}}
\hypersetup{pdfauthor={Niklas Beisert}}
\hypersetup{pdfsubject={Manual for the LaTeX2e Package childdoc}}
\date{30 December 2018, \textsf{v2.0}}
\maketitle

\begin{abstract}\noindent
\textsf{childdoc} is a \LaTeXe{} package
that enables the direct compilation
of document sections included by |\include|
to individual files.
\end{abstract}

\begingroup
\parskip0ex
\tableofcontents
\endgroup

%%%%%%%%%%%%%%%%%%%%%%%%%%%%%%%%%%%%%%%%%%%%%%%%%%%%%%%%%%%%%%%%%%%%%%%%%%%%%%%%
%%%%%%%%%%%%%%%%%%%%%%%%%%%%%%%%%%%%%%%%%%%%%%%%%%%%%%%%%%%%%%%%%%%%%%%%%%%%%%%%
\section{Introduction}

\LaTeX{} provides a mechanism to structure a large document (such as a book)
into a main file and several child files (containing the chapters)
using the |\include| command.
This mechanism is beneficial for documents
which span hundreds of pages in order to
make the source file(s) more manageable.
Moreover, compilation can be restricted to
selected child files by means of the |\includeonly| command.
The latter feature can be used to reduce the compilation time while editing
(this was significantly more useful in the earlier days of \LaTeX{})
or to generate a smaller document which is easier to navigate.
Another application of |\includeonly| is to generate
documents consisting of selected parts of the complete document.

However, there are a few drawbacks of the plain |\include| mechanism:
\begin{itemize}
\item
The child files cannot be compiled on their own,
they can only be compiled via the main file.
A naive editing environment
(such as a text editor with an option
to have the current file processed by \LaTeX)
may require one to switch to the main file before compiling;
attempting to compile the child file produces errors.
\item
The main file must be modified (each time)
to adjust the |\includeonly| command
to the present needs. This easily leaves the main file in a messy state.
\item
The generated document will always carry the filename
of the main document. This is inconvenient if
several child files are to be compiled and
to be kept for distribution.
\end{itemize}

The present package provides a simple interface
to make child files individually compilable by \LaTeX{}.
Compiling a child file then has the same effect as compiling
the main file with an |\includeonly| command
to select the appropriate child.
Moreover the generated document will carry the name of the child
rather than the main file.
This resolves all three above issues.

This feature is meant to make the editing of books,
thesis documents and lecture notes somewhat more convenient.
However, the package can also be used efficiently for
composing a series of documents (such as exercise sheets)
which are typically distributed individually.
It then assists the author in generating the individual documents
(potentially in different versions)
as well as a document containing the collected series.
Another application is in developing style files
or other kinds of included material
where compilation of the style file could redirect
to a sample or test file.

%%%%%%%%%%%%%%%%%%%%%%%%%%%%%%%%%%%%%%%%%%%%%%%%%%%%%%%%%%%%%%%%%%%%%%%%%%%%%%%%
%%%%%%%%%%%%%%%%%%%%%%%%%%%%%%%%%%%%%%%%%%%%%%%%%%%%%%%%%%%%%%%%%%%%%%%%%%%%%%%%
\section{Usage}

First of all, the package \textsf{childdoc} is \emph{not} a standard
\LaTeXe{} |.sty| style file! Therefore it needs to be invoked in
a non-standard way.

%%%%%%%%%%%%%%%%%%%%%%%%%%%%%%%%%%%%%%%%%%%%%%%%%%%%%%%%%%%%%%%%%%%%%%%%%%%%%%%%
\subsection{Included Files}
\label{sec:include}

%%%%%%%%%%%%%%%%%%%%%%%%%%%%%%%%%%%%%%%%
\DescribeMacro{\childdocmain}
To use the package, add the commands
\begin{center}
\begin{tabular}{l}
|\input{childdoc.def}|\\
|\childdocmain{}|\\
\end{tabular}
\end{center}
at the very top of the main \LaTeX{} file,
in particular \emph{before} the |\documentclass| statement!
The argument of |\childdocmain| should be left empty
(but it must be present).

%%%%%%%%%%%%%%%%%%%%%%%%%%%%%%%%%%%%%%%%
\DescribeMacro{\childdocof}
Furthermore, add the commands
\begin{center}
\begin{tabular}{l}
|\input{childdoc.def}|\\
|\childdocof{|\textit{main}|}|\\
\end{tabular}
\end{center}
at the top of every child file \textit{child}
which is included by |\include{|\textit{child}|}|
from within the main file
(or at least for those files to be compiled individually).
The argument \textit{main} must be the filename of the main file.

There are a couple of
considerations in setting up the main and child documents:

%%%%%%%%%%%%%%%%%%%%%%%%%%%%%%%%%%%%%%%%
\paragraph{Restrictions.}

Please note the following restrictions:
\begin{itemize}
\item
|\childdocmain| must be called with one argument \textit{main}
to ensure compatibility with earlier version of the package.
It must either be empty (|\childdocmain{}|)
or precisely match the filename of the main file in which it is specified.
See \secref{sec:detection} for further information.
\item
The filename \textit{main} must be specified without the |.tex| extension.
\item
The filename \textit{main} is case sensitive
(even in case-insensitive file systems)
due to internal string comparison.
\item
The argument \textit{main} should be fully expanded, it cannot be a macro.
\item
Subdirectories and special characters should be avoided in filenames.
\item
The command |\childdocmain{|\textit{main}|}| must be followed by a whitespace.
It should not be followed immediately by another command
or by a comment mark `|%|'.
This is because the \TeX{} parser reads the token immediately following
the argument of |\childdocmain| and puts it
at the beginning of every child section;
however, a white\-space is ignored.
\end{itemize}

%%%%%%%%%%%%%%%%%%%%%%%%%%%%%%%%%%%%%%%%
\paragraph{Content of Main File.}

It is advisable to place all content in the child files included by |\include|.
Any output contained in the main file will appear in all child documents
unless suppressed manually;
it cannot be suppressed automatically by the |\includeonly| directive
and thus should normally be avoided.
A method to include some content in the main file
by means of conditional processing is described in \secref{sec:conditional}.

%%%%%%%%%%%%%%%%%%%%%%%%%%%%%%%%%%%%%%%%
\paragraph{Page Numbering.}

When only a part of the document is compiled,
the appropriate numbering of pages
(as well as other status parameters)
is determined from the |.aux| files.
The latter contain information from previous passes.
However this information needs to propagate through
all intermediate child documents.
Therefore the page numbering in child documents may well
be inconsistent until the complete document is compiled at least once.

A useful (if unconventional) way to always ensure a consistent
page numbering is to restart the numbering in each child document
and denote the pages by `\textit{child}|.|\textit{page}'
where \textit{child} represents the chapter/section number of the child file.
This can be achieved by the command
|\numberwithin{page}{|\textit{child}|}|
of the \textsf{amsmath} package
where \textit{child} can be |chapter| or |section|
depending on the chosen structuring.
Alternatively, one can modify the macro |\thepage| appropriately
and reset the counter |page| at the start of each child file.

%%%%%%%%%%%%%%%%%%%%%%%%%%%%%%%%%%%%%%%%%%%%%%%%%%%%%%%%%%%%%%%%%%%%%%%%%%%%%%%%
\subsection{Conditional Processing}
\label{sec:conditional}

The package provides a mechanism to compile different versions
of a document. To customise the versions further some conditional processing
can come in handy to distinguish which version is being compiled.
The package provides two macros to describe the compilation context:

%%%%%%%%%%%%%%%%%%%%%%%%%%%%%%%%%%%%%%%%
\DescribeMacro{\ifchilddoc}
The conditional |\ifchilddoc| distinguishes between the compilation of
child documents and the main document:
%
\begin{center}
|\ifchilddoc |\textit{child-code}| |[|\||else |\textit{main-code}]| \||fi|
\end{center}

%%%%%%%%%%%%%%%%%%%%%%%%%%%%%%%%%%%%%%%%
\DescribeMacro{\childdocname}
\DescribeMacro{\childdocjob}
The macro |\childdocname| contains the filename (without extension)
of the main or child file being processed.
Note that |\childdocjob| will always contain the name of the main file.

%%%%%%%%%%%%%%%%%%%%%%%%%%%%%%%%%%%%%%%%
\paragraph{Title Page.}

Conditional processing can be used to include a title or banner page
in the main document when proper precautions are taken.
Importantly, the code in the main file should ensure that the page counter
(as well as other status parameters which are stored in the |.aux| files)
takes the same value after the conditional processing.
Otherwise the page numbers may take divergent values
depending on which part is compiled.

For example, a title page could be declared by:
%
\begin{center}
\begin{tabular}{l}
|\ifchilddoc\||else|\\
|\addtocounter{page}{-1}|\\
\textit{code for title page}\\
|\newpage|\\
|\||fi|
\end{tabular}
\end{center}
%
A banner page for the child documents can be generated by:
%
\begin{center}
\begin{tabular}{l}
|\ifchilddoc|\\
|\addtocounter{page}{-1}|\\
\textit{code for banner page}\\
|\newpage|\\
|\||fi|
\end{tabular}
\end{center}
%
Here one could write a message such as:
\begin{center}
|This is the part \childdocname{} of \childdocjob{}.|
\end{center}

%%%%%%%%%%%%%%%%%%%%%%%%%%%%%%%%%%%%%%%%%%%%%%%%%%%%%%%%%%%%%%%%%%%%%%%%%%%%%%%%
\subsection{Flags}
\label{sec:flags}

The package makes it easy to generate different versions
of the main or child documents.
To this end compilation flags can be defined
and assigned different default values.
They will be particularly useful in conjunction
with the forwarding mechanism described in \secref{sec:forward}.

For example, it may be useful to have a flag |\version|
which can be set to |draft| or |final|.
The document source will contain some conditional code
depending on the value of |\version|.
Suppose further, the flag should default to |final| for the main file
and to |draft| for child files
which is a natural assignment for editing the document.
This is achieved by placing the following code
in the preamble of the main document
(below the |\childdocmain| directive):
%
\begin{center}
\begin{tabular}{l}
|\ifchilddoc|\\
|\providecommand{\version}{draft}|\\
|\||else|\\
|\providecommand{\version}{final}|\\
|\||fi|
\end{tabular}
\end{center}
%
The definition by |\providecommand| makes sure
that previous definitions are not overwritten.
Further statements |\providecommand{\version}{...}|
can thus be added before the above code to override it.

For the main file, one might add a line
(between |\childdocmain| and the above block)
%
\begin{center}
|%\ifchilddoc\||else\providecommand{\version}{draft}\||fi|
\end{center}
%
which can be uncommented to produce a draft version.
Likewise one can add a line to the very top of a child file
(above the |\childdocof{|\textit{main}|}| directive)
%
\begin{center}
|%\providecommand{\version}{final}|
\end{center}
%
which can be uncommented to produce the final version of this child document.

%%%%%%%%%%%%%%%%%%%%%%%%%%%%%%%%%%%%%%%%%%%%%%%%%%%%%%%%%%%%%%%%%%%%%%%%%%%%%%%%
\subsection{Forwarding}
\label{sec:forward}

Different versions of the main or child documents
using compilation flags as described in \secref{sec:flags}
can be (permanently) stored in different files
for convenient compilation, viewing and distribution.
To this end, the package defines a command
to pass on compilation to a different file:

%%%%%%%%%%%%%%%%%%%%%%%%%%%%%%%%%%%%%%%%
\DescribeMacro{\childdocforward}
The command |\childdocforward| redirects processing to
another source file:
%
\begin{center}
\begin{tabular}{l}
|\input{childdoc.def}|\\
|\childdocforward[|\textit{main}|]{|\textit{dest}|}|\\
\end{tabular}
\end{center}
%
The argument \textit{dest} is the destination file
(without extension).
It should be the main file or one of the child files.
Note that further \textsf{childdoc} directives
such as |\childdocof| and |\childdocforward|
in the indicated file will be processed in this form.
The optional argument \textit{main}
passes on directly to the main file \textit{main}
while pretending to compile the child \textit{dest}.
This form behaves as if \textit{dest}
issues |\childdocof{|\textit{main}|}| right away,
and no further \textsf{childdoc} directives will be processed.

%%%%%%%%%%%%%%%%%%%%%%%%%%%%%%%%%%%%%%%%
\DescribeMacro{\...prefix}
In the alternative form |\childdocforwardprefix|,
%
\begin{center}
\begin{tabular}{l}
|\input{childdoc.def}|\\
|\childdocforwardprefix[|\textit{main}|]{|\textit{prefix}|}{|\textit{dest}|}|
\end{tabular}
\end{center}
%
the destination file is determined by a pattern
depending on the current file:
To make this work, the current file must be called
`{\textit{prefix}\hspace{0.2em}\textit{suffix}}'
with \textit{prefix} matching precisely the argument.
Processing is then passed on to the file
`{\textit{dest}\hspace{0.2em}\textit{suffix}}'.
Surely, the same effect is achieved by
directly specifying the
argument `{\textit{dest}\hspace{0.2em}\textit{suffix}}'
in the first form.
However, that requires to set up a different file
for each child. With the alternative form of the command
all these files can have exactly the same content
which simplifies setting them up and maintaining them.

For example, the following file |draft.tex|
with a compilation flag |\version| as described in \secref{sec:flags}
compiles the main document as a draft:
%
\begin{center}
\begin{tabular}{l}
|\def\version{draft}|\\
|\input{childdoc.def}|\\
|\childdocforward{|\textit{main}|}|
\end{tabular}
\end{center}
%
Likewise, the following files |final|\textit{nn}|.tex|
compile the final version of the child document
|child|\textit{nn}|.tex|:
%
\begin{center}
\begin{tabular}{l}
|\def\version{final}|\\
|\input{childdoc.def}|\\
|\childdocforwardprefix{final}{child}|
\end{tabular}
\end{center}
%

Note that when several versions of a main file and/or of each child file
are to be generated, it may be convenient to set up a |Makefile| or
shell script to automatise the process.

%%%%%%%%%%%%%%%%%%%%%%%%%%%%%%%%%%%%%%%%%%%%%%%%%%%%%%%%%%%%%%%%%%%%%%%%%%%%%%%%
\subsection{Command Line Processing}
\label{sec:commandline}

The effect of redirection files can also be achieved by invoking
the \LaTeX{} compiler with a more elaborate command line.
Most conveniently this should be done as part
of a shell script or a |Makefile|.

When using \textsf{childdoc} in the main file, the following
command lines effectively perform a redirection
(note that depending on the shell being used,
backslashes may have to be doubled: `|\|' $\to$ `|\\|'):
%
\begin{center}
|... -jobname "|\textit{target}|" |\\|"|[\textit{flags}]%
|\input{childdoc.def}\childdocforward[|\textit{main}|]{|\textit{dest}|}"|
\end{center}
%
Here \textit{target} is the name of the output file,
\textit{main} is the name of the main file
and \textit{dest} is the name of the main or child file to be processed
(all filenames without extensions).
The optional argument \textit{main} can be omitted
if \textit{main} matches \textit{dest}.
Optionally, compilation \textit{flags} can be defined via |\def| commands.
This command line makes the \TeX{} engine believe
it is compiling the file \textit{target}
whose content is specified as the latter parameter.
The provided code then forwards the processing to
\textit{main} or \textit{dest} as described in \secref{sec:forward}.

%%%%%%%%%%%%%%%%%%%%%%%%%%%%%%%%%%%%%%%%%%%%%%%%%%%%%%%%%%%%%%%%%%%%%%%%%%%%%%%%
\subsection{Include by Input}
\label{sec:input}

Including child documents by |\include| has some restrictions by design.
Most notably, the content of a child document always occupies
its own set of pages; pages cannot be shared between child documents.
Usually, this behaviour makes perfect sense
because each child document contain an essential part of the document.
However, in some situations it may be desirable to compose
a document from a collection of parts
without having mandatory page breaks between then.
For this case, the package
provides a mechanism to include parts
by |\input| which can also be processed individually.
However, by construction this mechanism
requires manual handling of the content to be output.

%%%%%%%%%%%%%%%%%%%%%%%%%%%%%%%%%%%%%%%%
\DescribeMacro{\ifchilddocmanual}
The main file should be prepared as usual, see \secref{sec:include}.
However, the document body must make a distinction
between processing of an individual part and of the main document, e.g.:
%
\begin{center}
\begin{tabular}{l}
|\ifchilddocmanual|\\
|\input{\childdocname}|\\
|\||else|\\
\textit{document body with }|\input{|\textit{part}|}|\\
|\||fi|
\end{tabular}
\end{center}
%
The conditional |\ifchilddocmanual| is true whenever
a part to be included by |\input| is being compiled,
and the name of the part is stored in |\childdocname|.

%%%%%%%%%%%%%%%%%%%%%%%%%%%%%%%%%%%%%%%%
\DescribeMacro{\childdocby}
Each part to be included by |\input| should start with:
%
\begin{center}
\begin{tabular}{l}
|\input{childdoc.def}|\\
|\childdocby{|\textit{main}|}|\\
\end{tabular}
\end{center}
%
The directive |\childdocby| is similar to |\childdocof|
described in \secref{sec:include},
but the subsequent selection of content must be done manually.
To that end, both |\ifchilddoc| and |\ifchilddocmanual|
will be true upon processing of a part,
and the name of the part is stored in |\childdocname|.
Note that |\jobname| will be set to the filename of the current part
so that each part receives an individual |.aux| file
that does not interfere with the |.aux| file(s) of the main document.
This behaviour can be altered by the alternative form
|\childdocby[*]{|\textit{main}|}| (with a non-empty optional argument)
which uses the |.aux| file of the main document
by setting |\jobname| to \textit{main}.

%%%%%%%%%%%%%%%%%%%%%%%%%%%%%%%%%%%%%%%%%%%%%%%%%%%%%%%%%%%%%%%%%%%%%%%%%%%%%%%%
\subsection{Driver Development}
\label{sec:driver}

The \textsf{childdoc} mechanism can also be use for the development
of definition files such as \LaTeX{} styles or classes.
This case differs from the above setup with multiple parts
included by |\include| in that no |\includeonly| should be invoked.
This can be achieved by starting the include file
(before |\ProvidesPackage|) with:
%
\begin{center}
\begin{tabular}{l}
|\input{childdoc.def}|\\
|\childdocforward{|\textit{main}|}|\\
\end{tabular}
\end{center}
%
or alternatively with:
%
\begin{center}
\begin{tabular}{l}
|\input{childdoc.def}|\\
|\childdocby{|\textit{main}|}|\\
\end{tabular}
\end{center}
%
Both forms have slightly different effects as described above.
The main file is prepared as usual, see \secref{sec:include}.

%%%%%%%%%%%%%%%%%%%%%%%%%%%%%%%%%%%%%%%%%%%%%%%%%%%%%%%%%%%%%%%%%%%%%%%%%%%%%%%%
\subsection{Legacy Detection}
\label{sec:detection}

The directive |\childdocmain| in the main file can detect
whether the complete document or merely a child is to be compiled
even without using the directive |\childdocof|.
This method is deprecated because it is less robust
and there is no compelling reason to use it;
it is merely provided for backward compatibility
and it may be removed in future versions.

If the detection mechanism is to be used,
it is mandatory to correctly specify
the filename of the main file as the argument of |\childdocmain|:
%
\begin{center}
\begin{tabular}{l}
|\input{childdoc.def}|\\
|\childdocmain{|\textit{main}|}|\\
\end{tabular}
\end{center}
%
If |\jobname| does not match the argument \textit{main} of |\childdocmain|,
it is assumed that |\jobname| points to the child file to be compiled.
When using |\childdocmain| with the main file specified as argument,
it suffices to start a child file
with just |\input{|\textit{main}|}|
without loading of the package and using |\childdocof|.
If instead all processing is done
with the appropriate \textsf{childdoc} directives,
the argument of \textit{main} of |\childdocmain| can be empty.

An alternative version of the command line processing described
in \secref{sec:commandline} using the detection mechanism reads:
%
\begin{center}
|... -jobname "|\textit{target}|" "|[\textit{flags}]%
[|\def\jobname{|\textit{dest}|}|]|\input{|\textit{main}|}"|
\end{center}

%%%%%%%%%%%%%%%%%%%%%%%%%%%%%%%%%%%%%%%%%%%%%%%%%%%%%%%%%%%%%%%%%%%%%%%%%%%%%%%%
\subsection{Manual Code}
\label{sec:manual}

In case one cannot be certain whether the definitions file |childdoc.def|
is installed on the target \TeX{} distribution
and one prefers not to ship it,
it is conceivable to paste a few relevant commands into the sources.

To that end, drop all statements |\input{childdoc.def}|
and perform the replacements as outlined below.
Instead of |\childdocmain{|\textit{main}|}| add the following code
to the top of the main file:
%
\begin{center}
\begin{tabular}{l}
|\||ifdefined\childdocname\endinput\||fi\newif\ifchilddoc|\\
|\edef\childdocname{\scantokens\expandafter{\jobname\noexpand}}|\\
|\def\childdocmain{|\textit{main}|}\||ifx\childdocmain\childdocname\||else|\\
|\childdoctrue\includeonly{\childdocname}\let\jobname\childdocmain\||fi|\\
\end{tabular}
\end{center}
%
Instead of |\childdocof{|\textit{main}|}| just include the main file
at the top of each child file:
%
\begin{center}
|\input{|\textit{main}|}|
\end{center}
%
A simple redirection |\childdocforward{|\textit{dest}|}| is achieved by:
%
\begin{center}
|\def\jobname{|\textit{dest}|}\input{\jobname}|
\end{center}
%
The redirection with prefix
|\childdocforwardprefix[|\textit{prefix}|]{|\textit{dest}|}|
is accomplished by:
%
\begin{center}
\begin{tabular}{l}
|{\edef\jobname{\scantokens\expandafter{\jobname\noexpand}}|\\
|\def\redirectjob |\textit{prefix}|#1~~~{\gdef\jobname{|\textit{dest}|#1}}|\\
|\expandafter\redirectjob\jobname~~~}\input{\jobname}|
\end{tabular}
\end{center}

In an alternative approach,
child documents can be compiled by a specific command line
without additional code or specific definitions:
%
\begin{center}
|... -jobname "|\textit{target}|" "|[\textit{flags}]%
|\includeonly{|\textit{dest}|}\input{|\textit{main}|}"|
\end{center}
%

%%%%%%%%%%%%%%%%%%%%%%%%%%%%%%%%%%%%%%%%%%%%%%%%%%%%%%%%%%%%%%%%%%%%%%%%%%%%%%%%
%%%%%%%%%%%%%%%%%%%%%%%%%%%%%%%%%%%%%%%%%%%%%%%%%%%%%%%%%%%%%%%%%%%%%%%%%%%%%%%%
\section{Information}

%%%%%%%%%%%%%%%%%%%%%%%%%%%%%%%%%%%%%%%%%%%%%%%%%%%%%%%%%%%%%%%%%%%%%%%%%%%%%%%%
\subsection{Copyright}

Copyright \copyright{} 2017--2018 Niklas Beisert

This work may be distributed and/or modified under the
conditions of the \LaTeX{} Project Public License, either version 1.3
of this license or (at your option) any later version.
The latest version of this license is in
  \url{http://www.latex-project.org/lppl.txt}
and version 1.3 or later is part of all distributions of \LaTeX{}
version 2005/12/01 or later.

This work has the LPPL maintenance status `maintained'.

The Current Maintainer of this work is Niklas Beisert.

This work consists of the files |README.txt|, |childdoc.ins| and |childdoc.dtx|
as well as the derived files |childdoc.def|, |cdocsamp.tex|
with |cdocsch1.tex|, |cdocsch2.tex|, |cdocspt3.tex|, |cdocspt4.tex|,
|cdocsdrf.tex|, |cdocsfn1.tex|, |cdocsfn2.tex|
as well as |childdoc.pdf|.

%%%%%%%%%%%%%%%%%%%%%%%%%%%%%%%%%%%%%%%%%%%%%%%%%%%%%%%%%%%%%%%%%%%%%%%%%%%%%%%%
\subsection{Files and Installation}

The package consists of the files:
%
\begin{center}
\begin{tabular}{ll}
    |README.txt|   & readme file \\
    |childdoc.ins| & installation file \\
    |childdoc.dtx| & source file \\
    |childdoc.def| & definition file \\
    |cdocsamp.tex| & sample main file \\
    |cdocsch1.tex| & sample include file \\
    |cdocsch2.tex| & sample include file \\
    |cdocspt3.tex| & sample part file \\
    |cdocspt4.tex| & sample part file \\
    |cdocsdrf.tex| & sample redirection file \\
    |cdocsfn1.tex| & sample redirection file \\
    |cdocsfn2.tex| & sample redirection file \\
    |childdoc.pdf| & manual
\end{tabular}
\end{center}
%
The distribution consists of the files
|README.txt|, |childdoc.ins| and |childdoc.dtx|.
%
\begin{itemize}
\item
Run (pdf)\LaTeX{} on |childdoc.dtx|
to compile the manual |childdoc.pdf| (this file).
\item
Run \LaTeX{} on |childdoc.ins| to create the definitions file |childdoc.def|
and the sample |cdocsamp.tex| with include files
|cdocsch1.tex|, |cdocsch2.tex|, |cdocspt3.tex|, |cdocspt4.tex|,
|cdocsdrf.tex|, |cdocsfn1.tex|, |cdocsfn2.tex|.
Then copy the file |childdoc.def| to an appropriate directory of your \LaTeX{}
distribution, e.g.\ \textit{texmf-root}|/tex/latex/childdoc|.
\end{itemize}

%%%%%%%%%%%%%%%%%%%%%%%%%%%%%%%%%%%%%%%%%%%%%%%%%%%%%%%%%%%%%%%%%%%%%%%%%%%%%%%%
\subsection{Related CTAN Packages}

There are several other packages which offer a similar functionality:
%
\begin{itemize}
\item
The packages
\href{http://ctan.org/pkg/docmute}{\textsf{docmute}},
\href{http://ctan.org/pkg/includex}{\textsf{includex}} and
\href{http://ctan.org/pkg/standalone}{\textsf{standalone}}
provide commands to include only the document body of
a child file thus allowing both files to be compiled individually.
\item
The packages \href{http://ctan.org/pkg/subdocs}{\textsf{subdocs}}
and \href{http://ctan.org/pkg/subfiles}{\textsf{subfiles}}
provide structures in which the main and child documents can be
encapsulated and allowing them to be compiled individually.
The inclusion mechanism is different from the conventional |\include|.
\item
The package \href{http://ctan.org/pkg/combine}{\textsf{combine}}
is an elaborate solution to combine several documents into one.
\end{itemize}
%
See also the CTAN topic \href{http://ctan.org/topic/subdocs}{\textsf{subdocs}}
for further related packages.
The present package differs from the above solutions in that
a document structure constructed with the conventional |\include| mechanism
just needs two extra commands at the top of every file
such that all constituent files can be compiled individually.

%%%%%%%%%%%%%%%%%%%%%%%%%%%%%%%%%%%%%%%%%%%%%%%%%%%%%%%%%%%%%%%%%%%%%%%%%%%%%%%%
%\subsection{Feature Suggestions}
%
%The following is a list of features which may be useful for future
%versions of this package:
%%
%\begin{itemize}
%\item
%\ldots
%\end{itemize}

%%%%%%%%%%%%%%%%%%%%%%%%%%%%%%%%%%%%%%%%%%%%%%%%%%%%%%%%%%%%%%%%%%%%%%%%%%%%%%%%
\subsection{Revision History}

%%%%%%%%%%%%%%%%%%%%%%%%%%%%%%%%%%%%%%%%
\paragraph{v2.0:} 2018/12/30

\begin{itemize}
\item
immediate forward processing
\item
added |\childdocby| mechanism
\item
manual restructured
\end{itemize}

%%%%%%%%%%%%%%%%%%%%%%%%%%%%%%%%%%%%%%%%
\paragraph{v1.6:} 2018/01/17

\begin{itemize}
\item
application for development of include files
\item
corrections to manual
\end{itemize}

%%%%%%%%%%%%%%%%%%%%%%%%%%%%%%%%%%%%%%%%
\paragraph{v1.5:} 2017/05/21

\begin{itemize}
\item
more complete structuring introduced
\item
|\childdocof| introduced
\item
|\childdoc| renamed to |\childdocmain|
\item
|\childredirect| renamed to |\childdocforward| and |\childdocforwardprefix|
and functionality expanded
\end{itemize}

%%%%%%%%%%%%%%%%%%%%%%%%%%%%%%%%%%%%%%%%
\paragraph{v1.0:} 2017/04/27

\begin{itemize}
\item
manual and install package
\item
first version published on CTAN
\end{itemize}

%%%%%%%%%%%%%%%%%%%%%%%%%%%%%%%%%%%%%%%%
\paragraph{v0.6:} 2017/04/26

\begin{itemize}
\item
redirection mechanism added
\end{itemize}

%%%%%%%%%%%%%%%%%%%%%%%%%%%%%%%%%%%%%%%%
\paragraph{v0.5:} 2017/04/26

\begin{itemize}
\item
functionality in definition file
\end{itemize}


%%%%%%%%%%%%%%%%%%%%%%%%%%%%%%%%%%%%%%%%%%%%%%%%%%%%%%%%%%%%%%%%%%%%%%%%%%%%%%%%
%%%%%%%%%%%%%%%%%%%%%%%%%%%%%%%%%%%%%%%%%%%%%%%%%%%%%%%%%%%%%%%%%%%%%%%%%%%%%%%%
%%%%%%%%%%%%%%%%%%%%%%%%%%%%%%%%%%%%%%%%%%%%%%%%%%%%%%%%%%%%%%%%%%%%%%%%%%%%%%%%
\appendix

\settowidth\MacroIndent{\rmfamily\scriptsize 000\ }

 \DocInput{childdoc.dtx}

\end{document}
%</driver>
% \fi
%
% %%%%%%%%%%%%%%%%%%%%%%%%%%%%%%%%%%%%%%%%%%%%%%%%%%%%%%%%%%%%%%%%%%%%%%%%%%%%%%
% %%%%%%%%%%%%%%%%%%%%%%%%%%%%%%%%%%%%%%%%%%%%%%%%%%%%%%%%%%%%%%%%%%%%%%%%%%%%%%
% \section{Sample}
%\iffalse
%<*samplemain>
%\fi
%
% The following presents a sample document
% with two chapters, two parts, a title page,
% a compile flag as well as three forwarding files to set the flag.
% It consists of eight |.tex| files:
% \begin{center}
% \begin{tabular}{ll}
% |cdocsamp.tex|&main file\\
% |cdocsch1.tex|&include file for chapter 1\\
% |cdocsch2.tex|&include file for chapter 2\\
% |cdocspt3.tex|&include file for part 3\\
% |cdocspt4.tex|&include file for part 4\\
% |cdocsdrf.tex|&forwarding file for main file in draft mode\\
% |cdocsfi1.tex|&forwarding file for final version of chapter 1\\
% |cdocsfi2.tex|&forwarding file for final version of chapter 2\\
% \end{tabular}
% \end{center}
% Each of the eight files can be compiled directly by the \LaTeX{} compiler.
%
% %%%%%%%%%%%%%%%%%%%%%%%%%%%%%%%%%%%%%%
% \paragraph{Main File.}
%
% The main file is called |cdocsamp.tex|.
%
% Load the \textsf{childdoc} definitions and
% declare the filename for the main document:
%    \begin{macrocode}
\input{childdoc.def}
\childdocmain{}
%    \end{macrocode}

% Optional override for |\version| flag:
%    \begin{macrocode}
%%\ifchilddoc\else\providecommand{\version}{draft}\fi
%    \end{macrocode}

% Define the default values for the |\version| flag
% (|final| for the main file and |draft| for childs):
%    \begin{macrocode}
\ifchilddoc
\providecommand{\version}{draft}
\else
\providecommand{\version}{final}
\fi
%    \end{macrocode}

% Load the standard document class:
%    \begin{macrocode}
\documentclass[12pt]{article}
%    \end{macrocode}

% Start the document body:
%    \begin{macrocode}
\begin{document}
%    \end{macrocode}

% Declare a title page.
% Print title, part of document being processed and version flag:
%    \begin{macrocode}
\addtocounter{page}{-1}
\begin{center}
{\LARGE\bfseries{}childdoc example\par}
\vspace{1cm}
\ifchilddoc
\ifchilddocmanual part\else chapter\fi:
`\childdocname' of `\childdocjob'\par
\else
main document: `\childdocjob'\par
\fi
version: \version\par
\end{center}
\newpage
%    \end{macrocode}

% Manually include selected file,
% otherwise process as usual:
%    \begin{macrocode}
\ifchilddocmanual
\section*{part `\childdocname'}
\input{\childdocname}
\else
%    \end{macrocode}

% Include the two chapters:
%    \begin{macrocode}
\include{cdocsch1}
\include{cdocsch2}
%    \end{macrocode}

% Include the two parts unless only chapters should be displayed:
%    \begin{macrocode}
\ifchilddoc\else
\section{part three}
\input{cdocspt3}
\section{part four}
\input{cdocspt4}
\fi
%    \end{macrocode}

% Process as usual until here:
%    \begin{macrocode}
\fi
%    \end{macrocode}

% End of document body:
%    \begin{macrocode}
\end{document}
%    \end{macrocode}
%\iffalse
%</samplemain>
%\fi
%
% %%%%%%%%%%%%%%%%%%%%%%%%%%%%%%%%%%%%%%
% \paragraph{Chapter Include Files.}
%
% The include files are called |cdocsch1.tex| and |cdocsch2.tex|.
%
%\iffalse
%<*samplechap1|samplechap2>
%\fi

% Optional override for |\version| flag:
%    \begin{macrocode}
%%\providecommand{\version}{final}
%    \end{macrocode}

% Include the main document:
%    \begin{macrocode}
\input{childdoc.def}
\childdocof{cdocsamp}
%    \end{macrocode}

%\iffalse
%</samplechap1|samplechap2>
%\fi
%
%\iffalse
%<*samplechap1>
%\fi
% Some text for chapter 1:
%    \begin{macrocode}
\section{one}
some text in chapter one
%    \end{macrocode}

%\iffalse
%</samplechap1>
%\fi
% Some text for chapter 2:
%\iffalse
%<*samplechap2>
%\fi
%    \begin{macrocode}
\section{two}
more text in chapter two
%    \end{macrocode}

%\iffalse
%</samplechap2>
%\fi
%
% %%%%%%%%%%%%%%%%%%%%%%%%%%%%%%%%%%%%%%
% \paragraph{Part Include Files.}
%
% The include files are called |cdocspt3.tex| and |cdocspt4.tex|.
%
%\iffalse
%<*samplepart3|samplepart4>
%\fi

% Optional override for |\version| flag:
%    \begin{macrocode}
%%\providecommand{\version}{final}
%    \end{macrocode}

% Include the main document:
%    \begin{macrocode}
\input{childdoc.def}
\childdocby{cdocsamp}
%    \end{macrocode}

%\iffalse
%</samplepart3|samplepart4>
%\fi
%
%\iffalse
%<*samplepart3>
%\fi
% Some text for part 3:
%    \begin{macrocode}
some text in part three
%    \end{macrocode}

%\iffalse
%</samplepart3>
%\fi
% Some text for part 4:
%\iffalse
%<*samplepart4>
%\fi
%    \begin{macrocode}
more text in part four
%    \end{macrocode}

%\iffalse
%</samplepart4>
%\fi
%
% %%%%%%%%%%%%%%%%%%%%%%%%%%%%%%%%%%%%%%
% \paragraph{Forwarding for a Complete Draft.}
%
% The following forwarding file |cdocsdrf.tex|
% compiles the main document in draft mode:
%\iffalse
%<*sampledraft>
%\fi
%    \begin{macrocode}
\def\version{draft}
\input{childdoc.def}
\childdocforward{cdocsamp}
%    \end{macrocode}

%\iffalse
%</sampledraft>
%\fi
%
% %%%%%%%%%%%%%%%%%%%%%%%%%%%%%%%%%%%%%%
% \paragraph{Forwarding for Final Version of the Chapters.}
%
% The following forwarding files |cdocsfn1.tex| and |cdocsfn2.tex|
% (with identical content)
% compile the final versions of the child documents
% |cdocsch1.tex| and |cdocsch2.tex|, respectively:
%\iffalse
%<*samplefinal>
%\fi
%    \begin{macrocode}
\def\version{final}
\input{childdoc.def}
\childdocforwardprefix[cdocsamp]{cdocsfn}{cdocsch}
%    \end{macrocode}

%\iffalse
%</samplefinal>
%\fi
%
% %%%%%%%%%%%%%%%%%%%%%%%%%%%%%%%%%%%%%%
% \paragraph{Command Line Processing.}
%
% The following three command lines generate the output files
% |cdocscld|, |cdocscl1| and |cdocscl2|
% which should be identical to
% |cdocsdrf|, |cdocsch1| and |cdocsfn2|, respectively:
% \begin{center}
% \begin{tabular}{l}
% |latex -jobname cdocscld \|\\
% |  "\def\version{draft}\input{childdoc.def}\childdocforward{cdocsamp}"|\\
% |latex -jobname cdocscl1 \|\\
% |  "\input{childdoc.def}\childdocforward[cdocsamp]{cdocsch1}"|\\
% |latex -jobname cdocscl2 \|\\
% |  "\def\version{final}\input{childdoc.def}\childdocforward{cdocsch2}"|
% \end{tabular}
% \end{center}
% Note that the trailing backslash on each first line
% merely continues the input to the second line
% (for convenient cut ant paste).
% Furthermore, the command |latex| can be replaced by any
% of its alternative versions such as |pdflatex|.
%
% %%%%%%%%%%%%%%%%%%%%%%%%%%%%%%%%%%%%%%%%%%%%%%%%%%%%%%%%%%%%%%%%%%%%%%%%%%%%%%
% %%%%%%%%%%%%%%%%%%%%%%%%%%%%%%%%%%%%%%%%%%%%%%%%%%%%%%%%%%%%%%%%%%%%%%%%%%%%%%
% \section{Implementation}
%\iffalse
%<*package>
%\fi
%
% This section describes the definitions file |childdoc.def|.

% The definitions cannot be loaded using |\usepackage| or |\RequirePackage|
% which has a mechanism to prevent loading a style file more than once.
% When loading the definitions by means of |\input|
% multiple instances have to be prevented manually:
%\iffalse
%This code needs to be before the `\ProvidesFile' directive
%which is defined at the beginning of this file.
%Therefore it is also placed there and commented out here.
%</package>
%<*discard>
%\fi
%    \begin{macrocode}
\ifdefined\childdocmain\endinput\fi
%    \end{macrocode}
%\iffalse
%</discard>
%<*package>
%\fi
%
% \macro{\ifchilddoc}
% \macro{\ifchilddocmanual}
% The conditional |\ifchilddoc| tells whether a
% child (true) or main (false) document is being compiled.
% The conditional |\ifchilddocmanual| tells whether
% the |\includeonly| mechanism is used (false) or
% the selection of child files must be performed manually (true).
% The definitions initialise to false:
%    \begin{macrocode}
\newif\ifchilddoc
\newif\ifchilddocmanual
%    \end{macrocode}

% \macro{\childdocname}
% \macro{\childdocjob}
% The macro |\childdocname| stores the name of the main document
% to be compiled. The macro |\childdocjob| stores the name of
% the document on which the \LaTeX{} compiler was originally invoked.
% The content of |\jobname| cannot be compared
% to filenames specified in the source due to different catcodes.
% The following code rescans |\jobname|, stores the result
% in |\childdocname| and saves a copy in |\childdocjob|:
%    \begin{macrocode}
\edef\childdocname{\scantokens\expandafter{\jobname\noexpand}}
\let\childdocjob\childdocname
%    \end{macrocode}

% \macro{\childdocdisable}
% The macro |\childdocdisable| prevents the main file
% from being processed more than once.
% At this stage, the main document command |\childdocmain|
% is assumed to be called once again where it should do nothing.
% Any subsequent call to it should prevent
% a secondary processing of the main document
% It overwrites the forwarding commands
% |\childdocof| and |\childdocforward|
% with empty macros to prevent further inclusions of the main document:
%    \begin{macrocode}
\newcommand{\childdocdisable}
{
  \renewcommand{\childdocmain}[1]{\renewcommand{\childdocmain}[1]{\endinput}}
  \renewcommand{\childdocof}[1]{}
  \renewcommand{\childdocby}[2][]{}
  \renewcommand{\childdocforward}[2][]{}
  \renewcommand{\childdocdisable}{}
}
%    \end{macrocode}

% \macro{\childdocmain}
% The macro |\childdocmain| is to be called at the top of the main file
% with nothing or the main filename (without extension) as argument.
% First, it breaks loops.
% If the argument is not empty and does not match |\childdocname|
% (which is set by the first inclusion of |childdoc.def|),
% |\ifchilddoc| is set to true, |\includeonly| is applied to the child file
% and |\jobname| is set to the main file
% (for proper handling of |.aux| files):
%    \begin{macrocode}
\newcommand{\childdocmain}[1]
{
  \childdocdisable\childdocmain{}
  \if?#1?\else
    \begingroup
      \def\childdoctmp{#1}
      \ifx\childdoctmp\childdocname
        \def\childdoctmp{}
      \else
        \def\childdoctmp
        {
          \childdoctrue
          \includeonly{\childdocname}
          \def\childdocjob{#1}
          \def\jobname{#1}
        }
      \fi
      \expandafter
    \endgroup
    \childdoctmp
  \fi
}
%    \end{macrocode}

% \macro{\childdocof}
% The command |\childdocof| redirects
% compilation to the main file |#1|.
%    \begin{macrocode}
\newcommand{\childdocof}[1]
{
  \childdocdisable
  \childdoctrue
  \includeonly{\childdocname}
  \def\jobname{#1}
  \def\childdocjob{#1}
  \input{#1}
}
%    \end{macrocode}

% \macro{\childdocby}
% The command |\childdocby| ....
%    \begin{macrocode}
\newcommand{\childdocby}[2][]
{
  \childdocdisable
  \childdoctrue
  \childdocmanualtrue
  \if?#1?\else
    \def\jobname{#2}
  \fi
  \def\childdocjob{#2}
  \input{#2}
  \endinput
}
%    \end{macrocode}

% \macro{\childdocforward}
% The command |\childdocforward| redirects
% compilation to the main file or
% (if the optional argument is given) a child file.
% Parameters are set as if the main file
% or a child file starting with |\childdocof| was compiled.
% Then compilation is handed over to the main file:
%    \begin{macrocode}
\newcommand{\childdocforward}[2][]
{
  \begingroup
    \if?#1?
      \def\childdoctmp
      {
        \def\childdocname{#2}
        \def\childdocjob{#2}
        \def\jobname{#2}
        \input{#2}
        \endinput
      }
    \else
      \def\childdoctmp
      {
        \childdocdisable
        \def\childdocname{#2}
        \childdoctrue
        \includeonly{#2}
        \def\childdocjob{#1}
        \def\jobname{#1}
        \input{#1}
        \endinput
      }
    \fi
    \expandafter
  \endgroup
  \childdoctmp
}
%    \end{macrocode}

% \macro{\childdocforwardprefix}
% The command |\childdocforwardprefix| redirects
% compilation to the main or a child file by means of a pattern.
% The prefix |#1| in the current filename is replaced by |#2|
% and the suffix of the current filename is kept
% (it is assumed that the filename does not contain the substring `|~~~|'
% which is used as a delimiter).
% Compilation is handed over to the new file by |\childdocforward|:
%    \begin{macrocode}
\newcommand{\childdocforwardprefix}[3][]
{
  \begingroup
    \def\childdocextract #2##1~~~{\def\childdoctmp{\childdocforward[#1]{#3##1}}}
    \expandafter\childdocextract\childdocname~~~
    \expandafter
  \endgroup
  \childdoctmp
}
%    \end{macrocode}

% \macro{\childdoc}
% The deprecated macro |\childdoc| is a legacy version of |\childdocmain|:
%    \begin{macrocode}
\newcommand{\childdoc}{\childdocmain}
%    \end{macrocode}

% \macro{\childdocredirect}
% The deprecated macro |\childdocredirect| is a legacy version
% of |\childdocforward| and |\childdocforwardprefix|:
%    \begin{macrocode}
\newcommand{\childdocredirect}[2][]
{
  \begingroup
    \if?#1?
      \def\childdoctmp{\childdocforward{#2}}
    \else
      \def\childdoctmp{\childdocforwardprefix{#1}{#2}}
    \fi
    \expandafter
  \endgroup
  \childdoctmp
}
%    \end{macrocode}

%\iffalse
%</package>
%\fi
%
\endinput

\childdocforwardprefix[cdocsamp]{cdocsfn}{cdocsch}
%    \end{macrocode}

%\iffalse
%</samplefinal>
%\fi
%
% %%%%%%%%%%%%%%%%%%%%%%%%%%%%%%%%%%%%%%
% \paragraph{Command Line Processing.}
%
% The following three command lines generate the output files
% |cdocscld|, |cdocscl1| and |cdocscl2|
% which should be identical to
% |cdocsdrf|, |cdocsch1| and |cdocsfn2|, respectively:
% \begin{center}
% \begin{tabular}{l}
% |latex -jobname cdocscld \|\\
% |  "\def\version{draft}% \iffalse
%
% childdoc.dtx Copyright (C) 2017-2018 Niklas Beisert
%
% This work may be distributed and/or modified under the
% conditions of the LaTeX Project Public License, either version 1.3
% of this license or (at your option) any later version.
% The latest version of this license is in
%   http://www.latex-project.org/lppl.txt
% and version 1.3 or later is part of all distributions of LaTeX
% version 2005/12/01 or later.
%
% This work has the LPPL maintenance status `maintained'.
%
% The Current Maintainer of this work is Niklas Beisert.
%
% This work consists of the files childdoc.dtx and childdoc.ins
% and the derived files childdoc.def and cdocsamp.tex with
% cdocsch1.tex, cdocsch2.tex, cdocsdrf.tex, cdocsfn1.tex, cdocsfn2.tex.
%
%<package>\ifdefined\childdocmain\endinput\fi
%<package>\ProvidesFile{childdoc.def}[2018/12/30 v2.0 child document driver]
%<samplemain>\ProvidesFile{cdocsamp.tex}[2018/12/30 v2.0 sample for childdoc]
%<*driver>
%\ProvidesFile{childdoc.drv}[2018/12/30 v2.0 childdoc reference manual file]
\PassOptionsToClass{10pt,a4paper}{article}
\documentclass{ltxdoc}

\usepackage[margin=35mm]{geometry}
\usepackage{hyperref}
\usepackage{hyperxmp}
\usepackage[usenames]{color}

\hypersetup{colorlinks=true}
\hypersetup{pdfstartview=FitH}
\hypersetup{pdfpagemode=UseNone}
\hypersetup{pdfsource={}}
\hypersetup{pdflang={en-UK}}
\hypersetup{pdfcopyright={Copyright 2017-2018 Niklas Beisert.
  This work may be distributed and/or modified under the
  conditions of the LaTeX Project Public License, either version 1.3
  of this license or (at your option) any later version.}}
\hypersetup{pdflicenseurl={http://www.latex-project.org/lppl.txt}}
\hypersetup{pdfcontactaddress={ETH Zurich, ITP, HIT K,
  Wolfgang-Pauli-Strasse 27}}
\hypersetup{pdfcontactpostcode={8093}}
\hypersetup{pdfcontactcity={Zurich}}
\hypersetup{pdfcontactcountry={Switzerland}}
\hypersetup{pdfcontactemail={nbeisert@itp.phys.ethz.ch}}
\hypersetup{pdfcontacturl={http://people.phys.ethz.ch/\xmptilde nbeisert/}}

\newcommand{\secref}[1]{\hyperref[#1]{section \ref*{#1}}}

\parskip1ex
\parindent0pt
\let\olditemize\itemize
\def\itemize{\olditemize\parskip0pt}

\begin{document}

\title{The \textsf{childdoc} Package}
\hypersetup{pdftitle={The childdoc Package}}
\author{Niklas Beisert\\[2ex]
  Institut f\"ur Theoretische Physik\\
  Eidgen\"ossische Technische Hochschule Z\"urich\\
  Wolfgang-Pauli-Strasse 27, 8093 Z\"urich, Switzerland\\[1ex]
  \href{mailto:nbeisert@itp.phys.ethz.ch}
  {\texttt{nbeisert@itp.phys.ethz.ch}}}
\hypersetup{pdfauthor={Niklas Beisert}}
\hypersetup{pdfsubject={Manual for the LaTeX2e Package childdoc}}
\date{30 December 2018, \textsf{v2.0}}
\maketitle

\begin{abstract}\noindent
\textsf{childdoc} is a \LaTeXe{} package
that enables the direct compilation
of document sections included by |\include|
to individual files.
\end{abstract}

\begingroup
\parskip0ex
\tableofcontents
\endgroup

%%%%%%%%%%%%%%%%%%%%%%%%%%%%%%%%%%%%%%%%%%%%%%%%%%%%%%%%%%%%%%%%%%%%%%%%%%%%%%%%
%%%%%%%%%%%%%%%%%%%%%%%%%%%%%%%%%%%%%%%%%%%%%%%%%%%%%%%%%%%%%%%%%%%%%%%%%%%%%%%%
\section{Introduction}

\LaTeX{} provides a mechanism to structure a large document (such as a book)
into a main file and several child files (containing the chapters)
using the |\include| command.
This mechanism is beneficial for documents
which span hundreds of pages in order to
make the source file(s) more manageable.
Moreover, compilation can be restricted to
selected child files by means of the |\includeonly| command.
The latter feature can be used to reduce the compilation time while editing
(this was significantly more useful in the earlier days of \LaTeX{})
or to generate a smaller document which is easier to navigate.
Another application of |\includeonly| is to generate
documents consisting of selected parts of the complete document.

However, there are a few drawbacks of the plain |\include| mechanism:
\begin{itemize}
\item
The child files cannot be compiled on their own,
they can only be compiled via the main file.
A naive editing environment
(such as a text editor with an option
to have the current file processed by \LaTeX)
may require one to switch to the main file before compiling;
attempting to compile the child file produces errors.
\item
The main file must be modified (each time)
to adjust the |\includeonly| command
to the present needs. This easily leaves the main file in a messy state.
\item
The generated document will always carry the filename
of the main document. This is inconvenient if
several child files are to be compiled and
to be kept for distribution.
\end{itemize}

The present package provides a simple interface
to make child files individually compilable by \LaTeX{}.
Compiling a child file then has the same effect as compiling
the main file with an |\includeonly| command
to select the appropriate child.
Moreover the generated document will carry the name of the child
rather than the main file.
This resolves all three above issues.

This feature is meant to make the editing of books,
thesis documents and lecture notes somewhat more convenient.
However, the package can also be used efficiently for
composing a series of documents (such as exercise sheets)
which are typically distributed individually.
It then assists the author in generating the individual documents
(potentially in different versions)
as well as a document containing the collected series.
Another application is in developing style files
or other kinds of included material
where compilation of the style file could redirect
to a sample or test file.

%%%%%%%%%%%%%%%%%%%%%%%%%%%%%%%%%%%%%%%%%%%%%%%%%%%%%%%%%%%%%%%%%%%%%%%%%%%%%%%%
%%%%%%%%%%%%%%%%%%%%%%%%%%%%%%%%%%%%%%%%%%%%%%%%%%%%%%%%%%%%%%%%%%%%%%%%%%%%%%%%
\section{Usage}

First of all, the package \textsf{childdoc} is \emph{not} a standard
\LaTeXe{} |.sty| style file! Therefore it needs to be invoked in
a non-standard way.

%%%%%%%%%%%%%%%%%%%%%%%%%%%%%%%%%%%%%%%%%%%%%%%%%%%%%%%%%%%%%%%%%%%%%%%%%%%%%%%%
\subsection{Included Files}
\label{sec:include}

%%%%%%%%%%%%%%%%%%%%%%%%%%%%%%%%%%%%%%%%
\DescribeMacro{\childdocmain}
To use the package, add the commands
\begin{center}
\begin{tabular}{l}
|\input{childdoc.def}|\\
|\childdocmain{}|\\
\end{tabular}
\end{center}
at the very top of the main \LaTeX{} file,
in particular \emph{before} the |\documentclass| statement!
The argument of |\childdocmain| should be left empty
(but it must be present).

%%%%%%%%%%%%%%%%%%%%%%%%%%%%%%%%%%%%%%%%
\DescribeMacro{\childdocof}
Furthermore, add the commands
\begin{center}
\begin{tabular}{l}
|\input{childdoc.def}|\\
|\childdocof{|\textit{main}|}|\\
\end{tabular}
\end{center}
at the top of every child file \textit{child}
which is included by |\include{|\textit{child}|}|
from within the main file
(or at least for those files to be compiled individually).
The argument \textit{main} must be the filename of the main file.

There are a couple of
considerations in setting up the main and child documents:

%%%%%%%%%%%%%%%%%%%%%%%%%%%%%%%%%%%%%%%%
\paragraph{Restrictions.}

Please note the following restrictions:
\begin{itemize}
\item
|\childdocmain| must be called with one argument \textit{main}
to ensure compatibility with earlier version of the package.
It must either be empty (|\childdocmain{}|)
or precisely match the filename of the main file in which it is specified.
See \secref{sec:detection} for further information.
\item
The filename \textit{main} must be specified without the |.tex| extension.
\item
The filename \textit{main} is case sensitive
(even in case-insensitive file systems)
due to internal string comparison.
\item
The argument \textit{main} should be fully expanded, it cannot be a macro.
\item
Subdirectories and special characters should be avoided in filenames.
\item
The command |\childdocmain{|\textit{main}|}| must be followed by a whitespace.
It should not be followed immediately by another command
or by a comment mark `|%|'.
This is because the \TeX{} parser reads the token immediately following
the argument of |\childdocmain| and puts it
at the beginning of every child section;
however, a white\-space is ignored.
\end{itemize}

%%%%%%%%%%%%%%%%%%%%%%%%%%%%%%%%%%%%%%%%
\paragraph{Content of Main File.}

It is advisable to place all content in the child files included by |\include|.
Any output contained in the main file will appear in all child documents
unless suppressed manually;
it cannot be suppressed automatically by the |\includeonly| directive
and thus should normally be avoided.
A method to include some content in the main file
by means of conditional processing is described in \secref{sec:conditional}.

%%%%%%%%%%%%%%%%%%%%%%%%%%%%%%%%%%%%%%%%
\paragraph{Page Numbering.}

When only a part of the document is compiled,
the appropriate numbering of pages
(as well as other status parameters)
is determined from the |.aux| files.
The latter contain information from previous passes.
However this information needs to propagate through
all intermediate child documents.
Therefore the page numbering in child documents may well
be inconsistent until the complete document is compiled at least once.

A useful (if unconventional) way to always ensure a consistent
page numbering is to restart the numbering in each child document
and denote the pages by `\textit{child}|.|\textit{page}'
where \textit{child} represents the chapter/section number of the child file.
This can be achieved by the command
|\numberwithin{page}{|\textit{child}|}|
of the \textsf{amsmath} package
where \textit{child} can be |chapter| or |section|
depending on the chosen structuring.
Alternatively, one can modify the macro |\thepage| appropriately
and reset the counter |page| at the start of each child file.

%%%%%%%%%%%%%%%%%%%%%%%%%%%%%%%%%%%%%%%%%%%%%%%%%%%%%%%%%%%%%%%%%%%%%%%%%%%%%%%%
\subsection{Conditional Processing}
\label{sec:conditional}

The package provides a mechanism to compile different versions
of a document. To customise the versions further some conditional processing
can come in handy to distinguish which version is being compiled.
The package provides two macros to describe the compilation context:

%%%%%%%%%%%%%%%%%%%%%%%%%%%%%%%%%%%%%%%%
\DescribeMacro{\ifchilddoc}
The conditional |\ifchilddoc| distinguishes between the compilation of
child documents and the main document:
%
\begin{center}
|\ifchilddoc |\textit{child-code}| |[|\||else |\textit{main-code}]| \||fi|
\end{center}

%%%%%%%%%%%%%%%%%%%%%%%%%%%%%%%%%%%%%%%%
\DescribeMacro{\childdocname}
\DescribeMacro{\childdocjob}
The macro |\childdocname| contains the filename (without extension)
of the main or child file being processed.
Note that |\childdocjob| will always contain the name of the main file.

%%%%%%%%%%%%%%%%%%%%%%%%%%%%%%%%%%%%%%%%
\paragraph{Title Page.}

Conditional processing can be used to include a title or banner page
in the main document when proper precautions are taken.
Importantly, the code in the main file should ensure that the page counter
(as well as other status parameters which are stored in the |.aux| files)
takes the same value after the conditional processing.
Otherwise the page numbers may take divergent values
depending on which part is compiled.

For example, a title page could be declared by:
%
\begin{center}
\begin{tabular}{l}
|\ifchilddoc\||else|\\
|\addtocounter{page}{-1}|\\
\textit{code for title page}\\
|\newpage|\\
|\||fi|
\end{tabular}
\end{center}
%
A banner page for the child documents can be generated by:
%
\begin{center}
\begin{tabular}{l}
|\ifchilddoc|\\
|\addtocounter{page}{-1}|\\
\textit{code for banner page}\\
|\newpage|\\
|\||fi|
\end{tabular}
\end{center}
%
Here one could write a message such as:
\begin{center}
|This is the part \childdocname{} of \childdocjob{}.|
\end{center}

%%%%%%%%%%%%%%%%%%%%%%%%%%%%%%%%%%%%%%%%%%%%%%%%%%%%%%%%%%%%%%%%%%%%%%%%%%%%%%%%
\subsection{Flags}
\label{sec:flags}

The package makes it easy to generate different versions
of the main or child documents.
To this end compilation flags can be defined
and assigned different default values.
They will be particularly useful in conjunction
with the forwarding mechanism described in \secref{sec:forward}.

For example, it may be useful to have a flag |\version|
which can be set to |draft| or |final|.
The document source will contain some conditional code
depending on the value of |\version|.
Suppose further, the flag should default to |final| for the main file
and to |draft| for child files
which is a natural assignment for editing the document.
This is achieved by placing the following code
in the preamble of the main document
(below the |\childdocmain| directive):
%
\begin{center}
\begin{tabular}{l}
|\ifchilddoc|\\
|\providecommand{\version}{draft}|\\
|\||else|\\
|\providecommand{\version}{final}|\\
|\||fi|
\end{tabular}
\end{center}
%
The definition by |\providecommand| makes sure
that previous definitions are not overwritten.
Further statements |\providecommand{\version}{...}|
can thus be added before the above code to override it.

For the main file, one might add a line
(between |\childdocmain| and the above block)
%
\begin{center}
|%\ifchilddoc\||else\providecommand{\version}{draft}\||fi|
\end{center}
%
which can be uncommented to produce a draft version.
Likewise one can add a line to the very top of a child file
(above the |\childdocof{|\textit{main}|}| directive)
%
\begin{center}
|%\providecommand{\version}{final}|
\end{center}
%
which can be uncommented to produce the final version of this child document.

%%%%%%%%%%%%%%%%%%%%%%%%%%%%%%%%%%%%%%%%%%%%%%%%%%%%%%%%%%%%%%%%%%%%%%%%%%%%%%%%
\subsection{Forwarding}
\label{sec:forward}

Different versions of the main or child documents
using compilation flags as described in \secref{sec:flags}
can be (permanently) stored in different files
for convenient compilation, viewing and distribution.
To this end, the package defines a command
to pass on compilation to a different file:

%%%%%%%%%%%%%%%%%%%%%%%%%%%%%%%%%%%%%%%%
\DescribeMacro{\childdocforward}
The command |\childdocforward| redirects processing to
another source file:
%
\begin{center}
\begin{tabular}{l}
|\input{childdoc.def}|\\
|\childdocforward[|\textit{main}|]{|\textit{dest}|}|\\
\end{tabular}
\end{center}
%
The argument \textit{dest} is the destination file
(without extension).
It should be the main file or one of the child files.
Note that further \textsf{childdoc} directives
such as |\childdocof| and |\childdocforward|
in the indicated file will be processed in this form.
The optional argument \textit{main}
passes on directly to the main file \textit{main}
while pretending to compile the child \textit{dest}.
This form behaves as if \textit{dest}
issues |\childdocof{|\textit{main}|}| right away,
and no further \textsf{childdoc} directives will be processed.

%%%%%%%%%%%%%%%%%%%%%%%%%%%%%%%%%%%%%%%%
\DescribeMacro{\...prefix}
In the alternative form |\childdocforwardprefix|,
%
\begin{center}
\begin{tabular}{l}
|\input{childdoc.def}|\\
|\childdocforwardprefix[|\textit{main}|]{|\textit{prefix}|}{|\textit{dest}|}|
\end{tabular}
\end{center}
%
the destination file is determined by a pattern
depending on the current file:
To make this work, the current file must be called
`{\textit{prefix}\hspace{0.2em}\textit{suffix}}'
with \textit{prefix} matching precisely the argument.
Processing is then passed on to the file
`{\textit{dest}\hspace{0.2em}\textit{suffix}}'.
Surely, the same effect is achieved by
directly specifying the
argument `{\textit{dest}\hspace{0.2em}\textit{suffix}}'
in the first form.
However, that requires to set up a different file
for each child. With the alternative form of the command
all these files can have exactly the same content
which simplifies setting them up and maintaining them.

For example, the following file |draft.tex|
with a compilation flag |\version| as described in \secref{sec:flags}
compiles the main document as a draft:
%
\begin{center}
\begin{tabular}{l}
|\def\version{draft}|\\
|\input{childdoc.def}|\\
|\childdocforward{|\textit{main}|}|
\end{tabular}
\end{center}
%
Likewise, the following files |final|\textit{nn}|.tex|
compile the final version of the child document
|child|\textit{nn}|.tex|:
%
\begin{center}
\begin{tabular}{l}
|\def\version{final}|\\
|\input{childdoc.def}|\\
|\childdocforwardprefix{final}{child}|
\end{tabular}
\end{center}
%

Note that when several versions of a main file and/or of each child file
are to be generated, it may be convenient to set up a |Makefile| or
shell script to automatise the process.

%%%%%%%%%%%%%%%%%%%%%%%%%%%%%%%%%%%%%%%%%%%%%%%%%%%%%%%%%%%%%%%%%%%%%%%%%%%%%%%%
\subsection{Command Line Processing}
\label{sec:commandline}

The effect of redirection files can also be achieved by invoking
the \LaTeX{} compiler with a more elaborate command line.
Most conveniently this should be done as part
of a shell script or a |Makefile|.

When using \textsf{childdoc} in the main file, the following
command lines effectively perform a redirection
(note that depending on the shell being used,
backslashes may have to be doubled: `|\|' $\to$ `|\\|'):
%
\begin{center}
|... -jobname "|\textit{target}|" |\\|"|[\textit{flags}]%
|\input{childdoc.def}\childdocforward[|\textit{main}|]{|\textit{dest}|}"|
\end{center}
%
Here \textit{target} is the name of the output file,
\textit{main} is the name of the main file
and \textit{dest} is the name of the main or child file to be processed
(all filenames without extensions).
The optional argument \textit{main} can be omitted
if \textit{main} matches \textit{dest}.
Optionally, compilation \textit{flags} can be defined via |\def| commands.
This command line makes the \TeX{} engine believe
it is compiling the file \textit{target}
whose content is specified as the latter parameter.
The provided code then forwards the processing to
\textit{main} or \textit{dest} as described in \secref{sec:forward}.

%%%%%%%%%%%%%%%%%%%%%%%%%%%%%%%%%%%%%%%%%%%%%%%%%%%%%%%%%%%%%%%%%%%%%%%%%%%%%%%%
\subsection{Include by Input}
\label{sec:input}

Including child documents by |\include| has some restrictions by design.
Most notably, the content of a child document always occupies
its own set of pages; pages cannot be shared between child documents.
Usually, this behaviour makes perfect sense
because each child document contain an essential part of the document.
However, in some situations it may be desirable to compose
a document from a collection of parts
without having mandatory page breaks between then.
For this case, the package
provides a mechanism to include parts
by |\input| which can also be processed individually.
However, by construction this mechanism
requires manual handling of the content to be output.

%%%%%%%%%%%%%%%%%%%%%%%%%%%%%%%%%%%%%%%%
\DescribeMacro{\ifchilddocmanual}
The main file should be prepared as usual, see \secref{sec:include}.
However, the document body must make a distinction
between processing of an individual part and of the main document, e.g.:
%
\begin{center}
\begin{tabular}{l}
|\ifchilddocmanual|\\
|\input{\childdocname}|\\
|\||else|\\
\textit{document body with }|\input{|\textit{part}|}|\\
|\||fi|
\end{tabular}
\end{center}
%
The conditional |\ifchilddocmanual| is true whenever
a part to be included by |\input| is being compiled,
and the name of the part is stored in |\childdocname|.

%%%%%%%%%%%%%%%%%%%%%%%%%%%%%%%%%%%%%%%%
\DescribeMacro{\childdocby}
Each part to be included by |\input| should start with:
%
\begin{center}
\begin{tabular}{l}
|\input{childdoc.def}|\\
|\childdocby{|\textit{main}|}|\\
\end{tabular}
\end{center}
%
The directive |\childdocby| is similar to |\childdocof|
described in \secref{sec:include},
but the subsequent selection of content must be done manually.
To that end, both |\ifchilddoc| and |\ifchilddocmanual|
will be true upon processing of a part,
and the name of the part is stored in |\childdocname|.
Note that |\jobname| will be set to the filename of the current part
so that each part receives an individual |.aux| file
that does not interfere with the |.aux| file(s) of the main document.
This behaviour can be altered by the alternative form
|\childdocby[*]{|\textit{main}|}| (with a non-empty optional argument)
which uses the |.aux| file of the main document
by setting |\jobname| to \textit{main}.

%%%%%%%%%%%%%%%%%%%%%%%%%%%%%%%%%%%%%%%%%%%%%%%%%%%%%%%%%%%%%%%%%%%%%%%%%%%%%%%%
\subsection{Driver Development}
\label{sec:driver}

The \textsf{childdoc} mechanism can also be use for the development
of definition files such as \LaTeX{} styles or classes.
This case differs from the above setup with multiple parts
included by |\include| in that no |\includeonly| should be invoked.
This can be achieved by starting the include file
(before |\ProvidesPackage|) with:
%
\begin{center}
\begin{tabular}{l}
|\input{childdoc.def}|\\
|\childdocforward{|\textit{main}|}|\\
\end{tabular}
\end{center}
%
or alternatively with:
%
\begin{center}
\begin{tabular}{l}
|\input{childdoc.def}|\\
|\childdocby{|\textit{main}|}|\\
\end{tabular}
\end{center}
%
Both forms have slightly different effects as described above.
The main file is prepared as usual, see \secref{sec:include}.

%%%%%%%%%%%%%%%%%%%%%%%%%%%%%%%%%%%%%%%%%%%%%%%%%%%%%%%%%%%%%%%%%%%%%%%%%%%%%%%%
\subsection{Legacy Detection}
\label{sec:detection}

The directive |\childdocmain| in the main file can detect
whether the complete document or merely a child is to be compiled
even without using the directive |\childdocof|.
This method is deprecated because it is less robust
and there is no compelling reason to use it;
it is merely provided for backward compatibility
and it may be removed in future versions.

If the detection mechanism is to be used,
it is mandatory to correctly specify
the filename of the main file as the argument of |\childdocmain|:
%
\begin{center}
\begin{tabular}{l}
|\input{childdoc.def}|\\
|\childdocmain{|\textit{main}|}|\\
\end{tabular}
\end{center}
%
If |\jobname| does not match the argument \textit{main} of |\childdocmain|,
it is assumed that |\jobname| points to the child file to be compiled.
When using |\childdocmain| with the main file specified as argument,
it suffices to start a child file
with just |\input{|\textit{main}|}|
without loading of the package and using |\childdocof|.
If instead all processing is done
with the appropriate \textsf{childdoc} directives,
the argument of \textit{main} of |\childdocmain| can be empty.

An alternative version of the command line processing described
in \secref{sec:commandline} using the detection mechanism reads:
%
\begin{center}
|... -jobname "|\textit{target}|" "|[\textit{flags}]%
[|\def\jobname{|\textit{dest}|}|]|\input{|\textit{main}|}"|
\end{center}

%%%%%%%%%%%%%%%%%%%%%%%%%%%%%%%%%%%%%%%%%%%%%%%%%%%%%%%%%%%%%%%%%%%%%%%%%%%%%%%%
\subsection{Manual Code}
\label{sec:manual}

In case one cannot be certain whether the definitions file |childdoc.def|
is installed on the target \TeX{} distribution
and one prefers not to ship it,
it is conceivable to paste a few relevant commands into the sources.

To that end, drop all statements |\input{childdoc.def}|
and perform the replacements as outlined below.
Instead of |\childdocmain{|\textit{main}|}| add the following code
to the top of the main file:
%
\begin{center}
\begin{tabular}{l}
|\||ifdefined\childdocname\endinput\||fi\newif\ifchilddoc|\\
|\edef\childdocname{\scantokens\expandafter{\jobname\noexpand}}|\\
|\def\childdocmain{|\textit{main}|}\||ifx\childdocmain\childdocname\||else|\\
|\childdoctrue\includeonly{\childdocname}\let\jobname\childdocmain\||fi|\\
\end{tabular}
\end{center}
%
Instead of |\childdocof{|\textit{main}|}| just include the main file
at the top of each child file:
%
\begin{center}
|\input{|\textit{main}|}|
\end{center}
%
A simple redirection |\childdocforward{|\textit{dest}|}| is achieved by:
%
\begin{center}
|\def\jobname{|\textit{dest}|}\input{\jobname}|
\end{center}
%
The redirection with prefix
|\childdocforwardprefix[|\textit{prefix}|]{|\textit{dest}|}|
is accomplished by:
%
\begin{center}
\begin{tabular}{l}
|{\edef\jobname{\scantokens\expandafter{\jobname\noexpand}}|\\
|\def\redirectjob |\textit{prefix}|#1~~~{\gdef\jobname{|\textit{dest}|#1}}|\\
|\expandafter\redirectjob\jobname~~~}\input{\jobname}|
\end{tabular}
\end{center}

In an alternative approach,
child documents can be compiled by a specific command line
without additional code or specific definitions:
%
\begin{center}
|... -jobname "|\textit{target}|" "|[\textit{flags}]%
|\includeonly{|\textit{dest}|}\input{|\textit{main}|}"|
\end{center}
%

%%%%%%%%%%%%%%%%%%%%%%%%%%%%%%%%%%%%%%%%%%%%%%%%%%%%%%%%%%%%%%%%%%%%%%%%%%%%%%%%
%%%%%%%%%%%%%%%%%%%%%%%%%%%%%%%%%%%%%%%%%%%%%%%%%%%%%%%%%%%%%%%%%%%%%%%%%%%%%%%%
\section{Information}

%%%%%%%%%%%%%%%%%%%%%%%%%%%%%%%%%%%%%%%%%%%%%%%%%%%%%%%%%%%%%%%%%%%%%%%%%%%%%%%%
\subsection{Copyright}

Copyright \copyright{} 2017--2018 Niklas Beisert

This work may be distributed and/or modified under the
conditions of the \LaTeX{} Project Public License, either version 1.3
of this license or (at your option) any later version.
The latest version of this license is in
  \url{http://www.latex-project.org/lppl.txt}
and version 1.3 or later is part of all distributions of \LaTeX{}
version 2005/12/01 or later.

This work has the LPPL maintenance status `maintained'.

The Current Maintainer of this work is Niklas Beisert.

This work consists of the files |README.txt|, |childdoc.ins| and |childdoc.dtx|
as well as the derived files |childdoc.def|, |cdocsamp.tex|
with |cdocsch1.tex|, |cdocsch2.tex|, |cdocspt3.tex|, |cdocspt4.tex|,
|cdocsdrf.tex|, |cdocsfn1.tex|, |cdocsfn2.tex|
as well as |childdoc.pdf|.

%%%%%%%%%%%%%%%%%%%%%%%%%%%%%%%%%%%%%%%%%%%%%%%%%%%%%%%%%%%%%%%%%%%%%%%%%%%%%%%%
\subsection{Files and Installation}

The package consists of the files:
%
\begin{center}
\begin{tabular}{ll}
    |README.txt|   & readme file \\
    |childdoc.ins| & installation file \\
    |childdoc.dtx| & source file \\
    |childdoc.def| & definition file \\
    |cdocsamp.tex| & sample main file \\
    |cdocsch1.tex| & sample include file \\
    |cdocsch2.tex| & sample include file \\
    |cdocspt3.tex| & sample part file \\
    |cdocspt4.tex| & sample part file \\
    |cdocsdrf.tex| & sample redirection file \\
    |cdocsfn1.tex| & sample redirection file \\
    |cdocsfn2.tex| & sample redirection file \\
    |childdoc.pdf| & manual
\end{tabular}
\end{center}
%
The distribution consists of the files
|README.txt|, |childdoc.ins| and |childdoc.dtx|.
%
\begin{itemize}
\item
Run (pdf)\LaTeX{} on |childdoc.dtx|
to compile the manual |childdoc.pdf| (this file).
\item
Run \LaTeX{} on |childdoc.ins| to create the definitions file |childdoc.def|
and the sample |cdocsamp.tex| with include files
|cdocsch1.tex|, |cdocsch2.tex|, |cdocspt3.tex|, |cdocspt4.tex|,
|cdocsdrf.tex|, |cdocsfn1.tex|, |cdocsfn2.tex|.
Then copy the file |childdoc.def| to an appropriate directory of your \LaTeX{}
distribution, e.g.\ \textit{texmf-root}|/tex/latex/childdoc|.
\end{itemize}

%%%%%%%%%%%%%%%%%%%%%%%%%%%%%%%%%%%%%%%%%%%%%%%%%%%%%%%%%%%%%%%%%%%%%%%%%%%%%%%%
\subsection{Related CTAN Packages}

There are several other packages which offer a similar functionality:
%
\begin{itemize}
\item
The packages
\href{http://ctan.org/pkg/docmute}{\textsf{docmute}},
\href{http://ctan.org/pkg/includex}{\textsf{includex}} and
\href{http://ctan.org/pkg/standalone}{\textsf{standalone}}
provide commands to include only the document body of
a child file thus allowing both files to be compiled individually.
\item
The packages \href{http://ctan.org/pkg/subdocs}{\textsf{subdocs}}
and \href{http://ctan.org/pkg/subfiles}{\textsf{subfiles}}
provide structures in which the main and child documents can be
encapsulated and allowing them to be compiled individually.
The inclusion mechanism is different from the conventional |\include|.
\item
The package \href{http://ctan.org/pkg/combine}{\textsf{combine}}
is an elaborate solution to combine several documents into one.
\end{itemize}
%
See also the CTAN topic \href{http://ctan.org/topic/subdocs}{\textsf{subdocs}}
for further related packages.
The present package differs from the above solutions in that
a document structure constructed with the conventional |\include| mechanism
just needs two extra commands at the top of every file
such that all constituent files can be compiled individually.

%%%%%%%%%%%%%%%%%%%%%%%%%%%%%%%%%%%%%%%%%%%%%%%%%%%%%%%%%%%%%%%%%%%%%%%%%%%%%%%%
%\subsection{Feature Suggestions}
%
%The following is a list of features which may be useful for future
%versions of this package:
%%
%\begin{itemize}
%\item
%\ldots
%\end{itemize}

%%%%%%%%%%%%%%%%%%%%%%%%%%%%%%%%%%%%%%%%%%%%%%%%%%%%%%%%%%%%%%%%%%%%%%%%%%%%%%%%
\subsection{Revision History}

%%%%%%%%%%%%%%%%%%%%%%%%%%%%%%%%%%%%%%%%
\paragraph{v2.0:} 2018/12/30

\begin{itemize}
\item
immediate forward processing
\item
added |\childdocby| mechanism
\item
manual restructured
\end{itemize}

%%%%%%%%%%%%%%%%%%%%%%%%%%%%%%%%%%%%%%%%
\paragraph{v1.6:} 2018/01/17

\begin{itemize}
\item
application for development of include files
\item
corrections to manual
\end{itemize}

%%%%%%%%%%%%%%%%%%%%%%%%%%%%%%%%%%%%%%%%
\paragraph{v1.5:} 2017/05/21

\begin{itemize}
\item
more complete structuring introduced
\item
|\childdocof| introduced
\item
|\childdoc| renamed to |\childdocmain|
\item
|\childredirect| renamed to |\childdocforward| and |\childdocforwardprefix|
and functionality expanded
\end{itemize}

%%%%%%%%%%%%%%%%%%%%%%%%%%%%%%%%%%%%%%%%
\paragraph{v1.0:} 2017/04/27

\begin{itemize}
\item
manual and install package
\item
first version published on CTAN
\end{itemize}

%%%%%%%%%%%%%%%%%%%%%%%%%%%%%%%%%%%%%%%%
\paragraph{v0.6:} 2017/04/26

\begin{itemize}
\item
redirection mechanism added
\end{itemize}

%%%%%%%%%%%%%%%%%%%%%%%%%%%%%%%%%%%%%%%%
\paragraph{v0.5:} 2017/04/26

\begin{itemize}
\item
functionality in definition file
\end{itemize}


%%%%%%%%%%%%%%%%%%%%%%%%%%%%%%%%%%%%%%%%%%%%%%%%%%%%%%%%%%%%%%%%%%%%%%%%%%%%%%%%
%%%%%%%%%%%%%%%%%%%%%%%%%%%%%%%%%%%%%%%%%%%%%%%%%%%%%%%%%%%%%%%%%%%%%%%%%%%%%%%%
%%%%%%%%%%%%%%%%%%%%%%%%%%%%%%%%%%%%%%%%%%%%%%%%%%%%%%%%%%%%%%%%%%%%%%%%%%%%%%%%
\appendix

\settowidth\MacroIndent{\rmfamily\scriptsize 000\ }

 \DocInput{childdoc.dtx}

\end{document}
%</driver>
% \fi
%
% %%%%%%%%%%%%%%%%%%%%%%%%%%%%%%%%%%%%%%%%%%%%%%%%%%%%%%%%%%%%%%%%%%%%%%%%%%%%%%
% %%%%%%%%%%%%%%%%%%%%%%%%%%%%%%%%%%%%%%%%%%%%%%%%%%%%%%%%%%%%%%%%%%%%%%%%%%%%%%
% \section{Sample}
%\iffalse
%<*samplemain>
%\fi
%
% The following presents a sample document
% with two chapters, two parts, a title page,
% a compile flag as well as three forwarding files to set the flag.
% It consists of eight |.tex| files:
% \begin{center}
% \begin{tabular}{ll}
% |cdocsamp.tex|&main file\\
% |cdocsch1.tex|&include file for chapter 1\\
% |cdocsch2.tex|&include file for chapter 2\\
% |cdocspt3.tex|&include file for part 3\\
% |cdocspt4.tex|&include file for part 4\\
% |cdocsdrf.tex|&forwarding file for main file in draft mode\\
% |cdocsfi1.tex|&forwarding file for final version of chapter 1\\
% |cdocsfi2.tex|&forwarding file for final version of chapter 2\\
% \end{tabular}
% \end{center}
% Each of the eight files can be compiled directly by the \LaTeX{} compiler.
%
% %%%%%%%%%%%%%%%%%%%%%%%%%%%%%%%%%%%%%%
% \paragraph{Main File.}
%
% The main file is called |cdocsamp.tex|.
%
% Load the \textsf{childdoc} definitions and
% declare the filename for the main document:
%    \begin{macrocode}
\input{childdoc.def}
\childdocmain{}
%    \end{macrocode}

% Optional override for |\version| flag:
%    \begin{macrocode}
%%\ifchilddoc\else\providecommand{\version}{draft}\fi
%    \end{macrocode}

% Define the default values for the |\version| flag
% (|final| for the main file and |draft| for childs):
%    \begin{macrocode}
\ifchilddoc
\providecommand{\version}{draft}
\else
\providecommand{\version}{final}
\fi
%    \end{macrocode}

% Load the standard document class:
%    \begin{macrocode}
\documentclass[12pt]{article}
%    \end{macrocode}

% Start the document body:
%    \begin{macrocode}
\begin{document}
%    \end{macrocode}

% Declare a title page.
% Print title, part of document being processed and version flag:
%    \begin{macrocode}
\addtocounter{page}{-1}
\begin{center}
{\LARGE\bfseries{}childdoc example\par}
\vspace{1cm}
\ifchilddoc
\ifchilddocmanual part\else chapter\fi:
`\childdocname' of `\childdocjob'\par
\else
main document: `\childdocjob'\par
\fi
version: \version\par
\end{center}
\newpage
%    \end{macrocode}

% Manually include selected file,
% otherwise process as usual:
%    \begin{macrocode}
\ifchilddocmanual
\section*{part `\childdocname'}
\input{\childdocname}
\else
%    \end{macrocode}

% Include the two chapters:
%    \begin{macrocode}
\include{cdocsch1}
\include{cdocsch2}
%    \end{macrocode}

% Include the two parts unless only chapters should be displayed:
%    \begin{macrocode}
\ifchilddoc\else
\section{part three}
\input{cdocspt3}
\section{part four}
\input{cdocspt4}
\fi
%    \end{macrocode}

% Process as usual until here:
%    \begin{macrocode}
\fi
%    \end{macrocode}

% End of document body:
%    \begin{macrocode}
\end{document}
%    \end{macrocode}
%\iffalse
%</samplemain>
%\fi
%
% %%%%%%%%%%%%%%%%%%%%%%%%%%%%%%%%%%%%%%
% \paragraph{Chapter Include Files.}
%
% The include files are called |cdocsch1.tex| and |cdocsch2.tex|.
%
%\iffalse
%<*samplechap1|samplechap2>
%\fi

% Optional override for |\version| flag:
%    \begin{macrocode}
%%\providecommand{\version}{final}
%    \end{macrocode}

% Include the main document:
%    \begin{macrocode}
\input{childdoc.def}
\childdocof{cdocsamp}
%    \end{macrocode}

%\iffalse
%</samplechap1|samplechap2>
%\fi
%
%\iffalse
%<*samplechap1>
%\fi
% Some text for chapter 1:
%    \begin{macrocode}
\section{one}
some text in chapter one
%    \end{macrocode}

%\iffalse
%</samplechap1>
%\fi
% Some text for chapter 2:
%\iffalse
%<*samplechap2>
%\fi
%    \begin{macrocode}
\section{two}
more text in chapter two
%    \end{macrocode}

%\iffalse
%</samplechap2>
%\fi
%
% %%%%%%%%%%%%%%%%%%%%%%%%%%%%%%%%%%%%%%
% \paragraph{Part Include Files.}
%
% The include files are called |cdocspt3.tex| and |cdocspt4.tex|.
%
%\iffalse
%<*samplepart3|samplepart4>
%\fi

% Optional override for |\version| flag:
%    \begin{macrocode}
%%\providecommand{\version}{final}
%    \end{macrocode}

% Include the main document:
%    \begin{macrocode}
\input{childdoc.def}
\childdocby{cdocsamp}
%    \end{macrocode}

%\iffalse
%</samplepart3|samplepart4>
%\fi
%
%\iffalse
%<*samplepart3>
%\fi
% Some text for part 3:
%    \begin{macrocode}
some text in part three
%    \end{macrocode}

%\iffalse
%</samplepart3>
%\fi
% Some text for part 4:
%\iffalse
%<*samplepart4>
%\fi
%    \begin{macrocode}
more text in part four
%    \end{macrocode}

%\iffalse
%</samplepart4>
%\fi
%
% %%%%%%%%%%%%%%%%%%%%%%%%%%%%%%%%%%%%%%
% \paragraph{Forwarding for a Complete Draft.}
%
% The following forwarding file |cdocsdrf.tex|
% compiles the main document in draft mode:
%\iffalse
%<*sampledraft>
%\fi
%    \begin{macrocode}
\def\version{draft}
\input{childdoc.def}
\childdocforward{cdocsamp}
%    \end{macrocode}

%\iffalse
%</sampledraft>
%\fi
%
% %%%%%%%%%%%%%%%%%%%%%%%%%%%%%%%%%%%%%%
% \paragraph{Forwarding for Final Version of the Chapters.}
%
% The following forwarding files |cdocsfn1.tex| and |cdocsfn2.tex|
% (with identical content)
% compile the final versions of the child documents
% |cdocsch1.tex| and |cdocsch2.tex|, respectively:
%\iffalse
%<*samplefinal>
%\fi
%    \begin{macrocode}
\def\version{final}
\input{childdoc.def}
\childdocforwardprefix[cdocsamp]{cdocsfn}{cdocsch}
%    \end{macrocode}

%\iffalse
%</samplefinal>
%\fi
%
% %%%%%%%%%%%%%%%%%%%%%%%%%%%%%%%%%%%%%%
% \paragraph{Command Line Processing.}
%
% The following three command lines generate the output files
% |cdocscld|, |cdocscl1| and |cdocscl2|
% which should be identical to
% |cdocsdrf|, |cdocsch1| and |cdocsfn2|, respectively:
% \begin{center}
% \begin{tabular}{l}
% |latex -jobname cdocscld \|\\
% |  "\def\version{draft}\input{childdoc.def}\childdocforward{cdocsamp}"|\\
% |latex -jobname cdocscl1 \|\\
% |  "\input{childdoc.def}\childdocforward[cdocsamp]{cdocsch1}"|\\
% |latex -jobname cdocscl2 \|\\
% |  "\def\version{final}\input{childdoc.def}\childdocforward{cdocsch2}"|
% \end{tabular}
% \end{center}
% Note that the trailing backslash on each first line
% merely continues the input to the second line
% (for convenient cut ant paste).
% Furthermore, the command |latex| can be replaced by any
% of its alternative versions such as |pdflatex|.
%
% %%%%%%%%%%%%%%%%%%%%%%%%%%%%%%%%%%%%%%%%%%%%%%%%%%%%%%%%%%%%%%%%%%%%%%%%%%%%%%
% %%%%%%%%%%%%%%%%%%%%%%%%%%%%%%%%%%%%%%%%%%%%%%%%%%%%%%%%%%%%%%%%%%%%%%%%%%%%%%
% \section{Implementation}
%\iffalse
%<*package>
%\fi
%
% This section describes the definitions file |childdoc.def|.

% The definitions cannot be loaded using |\usepackage| or |\RequirePackage|
% which has a mechanism to prevent loading a style file more than once.
% When loading the definitions by means of |\input|
% multiple instances have to be prevented manually:
%\iffalse
%This code needs to be before the `\ProvidesFile' directive
%which is defined at the beginning of this file.
%Therefore it is also placed there and commented out here.
%</package>
%<*discard>
%\fi
%    \begin{macrocode}
\ifdefined\childdocmain\endinput\fi
%    \end{macrocode}
%\iffalse
%</discard>
%<*package>
%\fi
%
% \macro{\ifchilddoc}
% \macro{\ifchilddocmanual}
% The conditional |\ifchilddoc| tells whether a
% child (true) or main (false) document is being compiled.
% The conditional |\ifchilddocmanual| tells whether
% the |\includeonly| mechanism is used (false) or
% the selection of child files must be performed manually (true).
% The definitions initialise to false:
%    \begin{macrocode}
\newif\ifchilddoc
\newif\ifchilddocmanual
%    \end{macrocode}

% \macro{\childdocname}
% \macro{\childdocjob}
% The macro |\childdocname| stores the name of the main document
% to be compiled. The macro |\childdocjob| stores the name of
% the document on which the \LaTeX{} compiler was originally invoked.
% The content of |\jobname| cannot be compared
% to filenames specified in the source due to different catcodes.
% The following code rescans |\jobname|, stores the result
% in |\childdocname| and saves a copy in |\childdocjob|:
%    \begin{macrocode}
\edef\childdocname{\scantokens\expandafter{\jobname\noexpand}}
\let\childdocjob\childdocname
%    \end{macrocode}

% \macro{\childdocdisable}
% The macro |\childdocdisable| prevents the main file
% from being processed more than once.
% At this stage, the main document command |\childdocmain|
% is assumed to be called once again where it should do nothing.
% Any subsequent call to it should prevent
% a secondary processing of the main document
% It overwrites the forwarding commands
% |\childdocof| and |\childdocforward|
% with empty macros to prevent further inclusions of the main document:
%    \begin{macrocode}
\newcommand{\childdocdisable}
{
  \renewcommand{\childdocmain}[1]{\renewcommand{\childdocmain}[1]{\endinput}}
  \renewcommand{\childdocof}[1]{}
  \renewcommand{\childdocby}[2][]{}
  \renewcommand{\childdocforward}[2][]{}
  \renewcommand{\childdocdisable}{}
}
%    \end{macrocode}

% \macro{\childdocmain}
% The macro |\childdocmain| is to be called at the top of the main file
% with nothing or the main filename (without extension) as argument.
% First, it breaks loops.
% If the argument is not empty and does not match |\childdocname|
% (which is set by the first inclusion of |childdoc.def|),
% |\ifchilddoc| is set to true, |\includeonly| is applied to the child file
% and |\jobname| is set to the main file
% (for proper handling of |.aux| files):
%    \begin{macrocode}
\newcommand{\childdocmain}[1]
{
  \childdocdisable\childdocmain{}
  \if?#1?\else
    \begingroup
      \def\childdoctmp{#1}
      \ifx\childdoctmp\childdocname
        \def\childdoctmp{}
      \else
        \def\childdoctmp
        {
          \childdoctrue
          \includeonly{\childdocname}
          \def\childdocjob{#1}
          \def\jobname{#1}
        }
      \fi
      \expandafter
    \endgroup
    \childdoctmp
  \fi
}
%    \end{macrocode}

% \macro{\childdocof}
% The command |\childdocof| redirects
% compilation to the main file |#1|.
%    \begin{macrocode}
\newcommand{\childdocof}[1]
{
  \childdocdisable
  \childdoctrue
  \includeonly{\childdocname}
  \def\jobname{#1}
  \def\childdocjob{#1}
  \input{#1}
}
%    \end{macrocode}

% \macro{\childdocby}
% The command |\childdocby| ....
%    \begin{macrocode}
\newcommand{\childdocby}[2][]
{
  \childdocdisable
  \childdoctrue
  \childdocmanualtrue
  \if?#1?\else
    \def\jobname{#2}
  \fi
  \def\childdocjob{#2}
  \input{#2}
  \endinput
}
%    \end{macrocode}

% \macro{\childdocforward}
% The command |\childdocforward| redirects
% compilation to the main file or
% (if the optional argument is given) a child file.
% Parameters are set as if the main file
% or a child file starting with |\childdocof| was compiled.
% Then compilation is handed over to the main file:
%    \begin{macrocode}
\newcommand{\childdocforward}[2][]
{
  \begingroup
    \if?#1?
      \def\childdoctmp
      {
        \def\childdocname{#2}
        \def\childdocjob{#2}
        \def\jobname{#2}
        \input{#2}
        \endinput
      }
    \else
      \def\childdoctmp
      {
        \childdocdisable
        \def\childdocname{#2}
        \childdoctrue
        \includeonly{#2}
        \def\childdocjob{#1}
        \def\jobname{#1}
        \input{#1}
        \endinput
      }
    \fi
    \expandafter
  \endgroup
  \childdoctmp
}
%    \end{macrocode}

% \macro{\childdocforwardprefix}
% The command |\childdocforwardprefix| redirects
% compilation to the main or a child file by means of a pattern.
% The prefix |#1| in the current filename is replaced by |#2|
% and the suffix of the current filename is kept
% (it is assumed that the filename does not contain the substring `|~~~|'
% which is used as a delimiter).
% Compilation is handed over to the new file by |\childdocforward|:
%    \begin{macrocode}
\newcommand{\childdocforwardprefix}[3][]
{
  \begingroup
    \def\childdocextract #2##1~~~{\def\childdoctmp{\childdocforward[#1]{#3##1}}}
    \expandafter\childdocextract\childdocname~~~
    \expandafter
  \endgroup
  \childdoctmp
}
%    \end{macrocode}

% \macro{\childdoc}
% The deprecated macro |\childdoc| is a legacy version of |\childdocmain|:
%    \begin{macrocode}
\newcommand{\childdoc}{\childdocmain}
%    \end{macrocode}

% \macro{\childdocredirect}
% The deprecated macro |\childdocredirect| is a legacy version
% of |\childdocforward| and |\childdocforwardprefix|:
%    \begin{macrocode}
\newcommand{\childdocredirect}[2][]
{
  \begingroup
    \if?#1?
      \def\childdoctmp{\childdocforward{#2}}
    \else
      \def\childdoctmp{\childdocforwardprefix{#1}{#2}}
    \fi
    \expandafter
  \endgroup
  \childdoctmp
}
%    \end{macrocode}

%\iffalse
%</package>
%\fi
%
\endinput
\childdocforward{cdocsamp}"|\\
% |latex -jobname cdocscl1 \|\\
% |  "% \iffalse
%
% childdoc.dtx Copyright (C) 2017-2018 Niklas Beisert
%
% This work may be distributed and/or modified under the
% conditions of the LaTeX Project Public License, either version 1.3
% of this license or (at your option) any later version.
% The latest version of this license is in
%   http://www.latex-project.org/lppl.txt
% and version 1.3 or later is part of all distributions of LaTeX
% version 2005/12/01 or later.
%
% This work has the LPPL maintenance status `maintained'.
%
% The Current Maintainer of this work is Niklas Beisert.
%
% This work consists of the files childdoc.dtx and childdoc.ins
% and the derived files childdoc.def and cdocsamp.tex with
% cdocsch1.tex, cdocsch2.tex, cdocsdrf.tex, cdocsfn1.tex, cdocsfn2.tex.
%
%<package>\ifdefined\childdocmain\endinput\fi
%<package>\ProvidesFile{childdoc.def}[2018/12/30 v2.0 child document driver]
%<samplemain>\ProvidesFile{cdocsamp.tex}[2018/12/30 v2.0 sample for childdoc]
%<*driver>
%\ProvidesFile{childdoc.drv}[2018/12/30 v2.0 childdoc reference manual file]
\PassOptionsToClass{10pt,a4paper}{article}
\documentclass{ltxdoc}

\usepackage[margin=35mm]{geometry}
\usepackage{hyperref}
\usepackage{hyperxmp}
\usepackage[usenames]{color}

\hypersetup{colorlinks=true}
\hypersetup{pdfstartview=FitH}
\hypersetup{pdfpagemode=UseNone}
\hypersetup{pdfsource={}}
\hypersetup{pdflang={en-UK}}
\hypersetup{pdfcopyright={Copyright 2017-2018 Niklas Beisert.
  This work may be distributed and/or modified under the
  conditions of the LaTeX Project Public License, either version 1.3
  of this license or (at your option) any later version.}}
\hypersetup{pdflicenseurl={http://www.latex-project.org/lppl.txt}}
\hypersetup{pdfcontactaddress={ETH Zurich, ITP, HIT K,
  Wolfgang-Pauli-Strasse 27}}
\hypersetup{pdfcontactpostcode={8093}}
\hypersetup{pdfcontactcity={Zurich}}
\hypersetup{pdfcontactcountry={Switzerland}}
\hypersetup{pdfcontactemail={nbeisert@itp.phys.ethz.ch}}
\hypersetup{pdfcontacturl={http://people.phys.ethz.ch/\xmptilde nbeisert/}}

\newcommand{\secref}[1]{\hyperref[#1]{section \ref*{#1}}}

\parskip1ex
\parindent0pt
\let\olditemize\itemize
\def\itemize{\olditemize\parskip0pt}

\begin{document}

\title{The \textsf{childdoc} Package}
\hypersetup{pdftitle={The childdoc Package}}
\author{Niklas Beisert\\[2ex]
  Institut f\"ur Theoretische Physik\\
  Eidgen\"ossische Technische Hochschule Z\"urich\\
  Wolfgang-Pauli-Strasse 27, 8093 Z\"urich, Switzerland\\[1ex]
  \href{mailto:nbeisert@itp.phys.ethz.ch}
  {\texttt{nbeisert@itp.phys.ethz.ch}}}
\hypersetup{pdfauthor={Niklas Beisert}}
\hypersetup{pdfsubject={Manual for the LaTeX2e Package childdoc}}
\date{30 December 2018, \textsf{v2.0}}
\maketitle

\begin{abstract}\noindent
\textsf{childdoc} is a \LaTeXe{} package
that enables the direct compilation
of document sections included by |\include|
to individual files.
\end{abstract}

\begingroup
\parskip0ex
\tableofcontents
\endgroup

%%%%%%%%%%%%%%%%%%%%%%%%%%%%%%%%%%%%%%%%%%%%%%%%%%%%%%%%%%%%%%%%%%%%%%%%%%%%%%%%
%%%%%%%%%%%%%%%%%%%%%%%%%%%%%%%%%%%%%%%%%%%%%%%%%%%%%%%%%%%%%%%%%%%%%%%%%%%%%%%%
\section{Introduction}

\LaTeX{} provides a mechanism to structure a large document (such as a book)
into a main file and several child files (containing the chapters)
using the |\include| command.
This mechanism is beneficial for documents
which span hundreds of pages in order to
make the source file(s) more manageable.
Moreover, compilation can be restricted to
selected child files by means of the |\includeonly| command.
The latter feature can be used to reduce the compilation time while editing
(this was significantly more useful in the earlier days of \LaTeX{})
or to generate a smaller document which is easier to navigate.
Another application of |\includeonly| is to generate
documents consisting of selected parts of the complete document.

However, there are a few drawbacks of the plain |\include| mechanism:
\begin{itemize}
\item
The child files cannot be compiled on their own,
they can only be compiled via the main file.
A naive editing environment
(such as a text editor with an option
to have the current file processed by \LaTeX)
may require one to switch to the main file before compiling;
attempting to compile the child file produces errors.
\item
The main file must be modified (each time)
to adjust the |\includeonly| command
to the present needs. This easily leaves the main file in a messy state.
\item
The generated document will always carry the filename
of the main document. This is inconvenient if
several child files are to be compiled and
to be kept for distribution.
\end{itemize}

The present package provides a simple interface
to make child files individually compilable by \LaTeX{}.
Compiling a child file then has the same effect as compiling
the main file with an |\includeonly| command
to select the appropriate child.
Moreover the generated document will carry the name of the child
rather than the main file.
This resolves all three above issues.

This feature is meant to make the editing of books,
thesis documents and lecture notes somewhat more convenient.
However, the package can also be used efficiently for
composing a series of documents (such as exercise sheets)
which are typically distributed individually.
It then assists the author in generating the individual documents
(potentially in different versions)
as well as a document containing the collected series.
Another application is in developing style files
or other kinds of included material
where compilation of the style file could redirect
to a sample or test file.

%%%%%%%%%%%%%%%%%%%%%%%%%%%%%%%%%%%%%%%%%%%%%%%%%%%%%%%%%%%%%%%%%%%%%%%%%%%%%%%%
%%%%%%%%%%%%%%%%%%%%%%%%%%%%%%%%%%%%%%%%%%%%%%%%%%%%%%%%%%%%%%%%%%%%%%%%%%%%%%%%
\section{Usage}

First of all, the package \textsf{childdoc} is \emph{not} a standard
\LaTeXe{} |.sty| style file! Therefore it needs to be invoked in
a non-standard way.

%%%%%%%%%%%%%%%%%%%%%%%%%%%%%%%%%%%%%%%%%%%%%%%%%%%%%%%%%%%%%%%%%%%%%%%%%%%%%%%%
\subsection{Included Files}
\label{sec:include}

%%%%%%%%%%%%%%%%%%%%%%%%%%%%%%%%%%%%%%%%
\DescribeMacro{\childdocmain}
To use the package, add the commands
\begin{center}
\begin{tabular}{l}
|\input{childdoc.def}|\\
|\childdocmain{}|\\
\end{tabular}
\end{center}
at the very top of the main \LaTeX{} file,
in particular \emph{before} the |\documentclass| statement!
The argument of |\childdocmain| should be left empty
(but it must be present).

%%%%%%%%%%%%%%%%%%%%%%%%%%%%%%%%%%%%%%%%
\DescribeMacro{\childdocof}
Furthermore, add the commands
\begin{center}
\begin{tabular}{l}
|\input{childdoc.def}|\\
|\childdocof{|\textit{main}|}|\\
\end{tabular}
\end{center}
at the top of every child file \textit{child}
which is included by |\include{|\textit{child}|}|
from within the main file
(or at least for those files to be compiled individually).
The argument \textit{main} must be the filename of the main file.

There are a couple of
considerations in setting up the main and child documents:

%%%%%%%%%%%%%%%%%%%%%%%%%%%%%%%%%%%%%%%%
\paragraph{Restrictions.}

Please note the following restrictions:
\begin{itemize}
\item
|\childdocmain| must be called with one argument \textit{main}
to ensure compatibility with earlier version of the package.
It must either be empty (|\childdocmain{}|)
or precisely match the filename of the main file in which it is specified.
See \secref{sec:detection} for further information.
\item
The filename \textit{main} must be specified without the |.tex| extension.
\item
The filename \textit{main} is case sensitive
(even in case-insensitive file systems)
due to internal string comparison.
\item
The argument \textit{main} should be fully expanded, it cannot be a macro.
\item
Subdirectories and special characters should be avoided in filenames.
\item
The command |\childdocmain{|\textit{main}|}| must be followed by a whitespace.
It should not be followed immediately by another command
or by a comment mark `|%|'.
This is because the \TeX{} parser reads the token immediately following
the argument of |\childdocmain| and puts it
at the beginning of every child section;
however, a white\-space is ignored.
\end{itemize}

%%%%%%%%%%%%%%%%%%%%%%%%%%%%%%%%%%%%%%%%
\paragraph{Content of Main File.}

It is advisable to place all content in the child files included by |\include|.
Any output contained in the main file will appear in all child documents
unless suppressed manually;
it cannot be suppressed automatically by the |\includeonly| directive
and thus should normally be avoided.
A method to include some content in the main file
by means of conditional processing is described in \secref{sec:conditional}.

%%%%%%%%%%%%%%%%%%%%%%%%%%%%%%%%%%%%%%%%
\paragraph{Page Numbering.}

When only a part of the document is compiled,
the appropriate numbering of pages
(as well as other status parameters)
is determined from the |.aux| files.
The latter contain information from previous passes.
However this information needs to propagate through
all intermediate child documents.
Therefore the page numbering in child documents may well
be inconsistent until the complete document is compiled at least once.

A useful (if unconventional) way to always ensure a consistent
page numbering is to restart the numbering in each child document
and denote the pages by `\textit{child}|.|\textit{page}'
where \textit{child} represents the chapter/section number of the child file.
This can be achieved by the command
|\numberwithin{page}{|\textit{child}|}|
of the \textsf{amsmath} package
where \textit{child} can be |chapter| or |section|
depending on the chosen structuring.
Alternatively, one can modify the macro |\thepage| appropriately
and reset the counter |page| at the start of each child file.

%%%%%%%%%%%%%%%%%%%%%%%%%%%%%%%%%%%%%%%%%%%%%%%%%%%%%%%%%%%%%%%%%%%%%%%%%%%%%%%%
\subsection{Conditional Processing}
\label{sec:conditional}

The package provides a mechanism to compile different versions
of a document. To customise the versions further some conditional processing
can come in handy to distinguish which version is being compiled.
The package provides two macros to describe the compilation context:

%%%%%%%%%%%%%%%%%%%%%%%%%%%%%%%%%%%%%%%%
\DescribeMacro{\ifchilddoc}
The conditional |\ifchilddoc| distinguishes between the compilation of
child documents and the main document:
%
\begin{center}
|\ifchilddoc |\textit{child-code}| |[|\||else |\textit{main-code}]| \||fi|
\end{center}

%%%%%%%%%%%%%%%%%%%%%%%%%%%%%%%%%%%%%%%%
\DescribeMacro{\childdocname}
\DescribeMacro{\childdocjob}
The macro |\childdocname| contains the filename (without extension)
of the main or child file being processed.
Note that |\childdocjob| will always contain the name of the main file.

%%%%%%%%%%%%%%%%%%%%%%%%%%%%%%%%%%%%%%%%
\paragraph{Title Page.}

Conditional processing can be used to include a title or banner page
in the main document when proper precautions are taken.
Importantly, the code in the main file should ensure that the page counter
(as well as other status parameters which are stored in the |.aux| files)
takes the same value after the conditional processing.
Otherwise the page numbers may take divergent values
depending on which part is compiled.

For example, a title page could be declared by:
%
\begin{center}
\begin{tabular}{l}
|\ifchilddoc\||else|\\
|\addtocounter{page}{-1}|\\
\textit{code for title page}\\
|\newpage|\\
|\||fi|
\end{tabular}
\end{center}
%
A banner page for the child documents can be generated by:
%
\begin{center}
\begin{tabular}{l}
|\ifchilddoc|\\
|\addtocounter{page}{-1}|\\
\textit{code for banner page}\\
|\newpage|\\
|\||fi|
\end{tabular}
\end{center}
%
Here one could write a message such as:
\begin{center}
|This is the part \childdocname{} of \childdocjob{}.|
\end{center}

%%%%%%%%%%%%%%%%%%%%%%%%%%%%%%%%%%%%%%%%%%%%%%%%%%%%%%%%%%%%%%%%%%%%%%%%%%%%%%%%
\subsection{Flags}
\label{sec:flags}

The package makes it easy to generate different versions
of the main or child documents.
To this end compilation flags can be defined
and assigned different default values.
They will be particularly useful in conjunction
with the forwarding mechanism described in \secref{sec:forward}.

For example, it may be useful to have a flag |\version|
which can be set to |draft| or |final|.
The document source will contain some conditional code
depending on the value of |\version|.
Suppose further, the flag should default to |final| for the main file
and to |draft| for child files
which is a natural assignment for editing the document.
This is achieved by placing the following code
in the preamble of the main document
(below the |\childdocmain| directive):
%
\begin{center}
\begin{tabular}{l}
|\ifchilddoc|\\
|\providecommand{\version}{draft}|\\
|\||else|\\
|\providecommand{\version}{final}|\\
|\||fi|
\end{tabular}
\end{center}
%
The definition by |\providecommand| makes sure
that previous definitions are not overwritten.
Further statements |\providecommand{\version}{...}|
can thus be added before the above code to override it.

For the main file, one might add a line
(between |\childdocmain| and the above block)
%
\begin{center}
|%\ifchilddoc\||else\providecommand{\version}{draft}\||fi|
\end{center}
%
which can be uncommented to produce a draft version.
Likewise one can add a line to the very top of a child file
(above the |\childdocof{|\textit{main}|}| directive)
%
\begin{center}
|%\providecommand{\version}{final}|
\end{center}
%
which can be uncommented to produce the final version of this child document.

%%%%%%%%%%%%%%%%%%%%%%%%%%%%%%%%%%%%%%%%%%%%%%%%%%%%%%%%%%%%%%%%%%%%%%%%%%%%%%%%
\subsection{Forwarding}
\label{sec:forward}

Different versions of the main or child documents
using compilation flags as described in \secref{sec:flags}
can be (permanently) stored in different files
for convenient compilation, viewing and distribution.
To this end, the package defines a command
to pass on compilation to a different file:

%%%%%%%%%%%%%%%%%%%%%%%%%%%%%%%%%%%%%%%%
\DescribeMacro{\childdocforward}
The command |\childdocforward| redirects processing to
another source file:
%
\begin{center}
\begin{tabular}{l}
|\input{childdoc.def}|\\
|\childdocforward[|\textit{main}|]{|\textit{dest}|}|\\
\end{tabular}
\end{center}
%
The argument \textit{dest} is the destination file
(without extension).
It should be the main file or one of the child files.
Note that further \textsf{childdoc} directives
such as |\childdocof| and |\childdocforward|
in the indicated file will be processed in this form.
The optional argument \textit{main}
passes on directly to the main file \textit{main}
while pretending to compile the child \textit{dest}.
This form behaves as if \textit{dest}
issues |\childdocof{|\textit{main}|}| right away,
and no further \textsf{childdoc} directives will be processed.

%%%%%%%%%%%%%%%%%%%%%%%%%%%%%%%%%%%%%%%%
\DescribeMacro{\...prefix}
In the alternative form |\childdocforwardprefix|,
%
\begin{center}
\begin{tabular}{l}
|\input{childdoc.def}|\\
|\childdocforwardprefix[|\textit{main}|]{|\textit{prefix}|}{|\textit{dest}|}|
\end{tabular}
\end{center}
%
the destination file is determined by a pattern
depending on the current file:
To make this work, the current file must be called
`{\textit{prefix}\hspace{0.2em}\textit{suffix}}'
with \textit{prefix} matching precisely the argument.
Processing is then passed on to the file
`{\textit{dest}\hspace{0.2em}\textit{suffix}}'.
Surely, the same effect is achieved by
directly specifying the
argument `{\textit{dest}\hspace{0.2em}\textit{suffix}}'
in the first form.
However, that requires to set up a different file
for each child. With the alternative form of the command
all these files can have exactly the same content
which simplifies setting them up and maintaining them.

For example, the following file |draft.tex|
with a compilation flag |\version| as described in \secref{sec:flags}
compiles the main document as a draft:
%
\begin{center}
\begin{tabular}{l}
|\def\version{draft}|\\
|\input{childdoc.def}|\\
|\childdocforward{|\textit{main}|}|
\end{tabular}
\end{center}
%
Likewise, the following files |final|\textit{nn}|.tex|
compile the final version of the child document
|child|\textit{nn}|.tex|:
%
\begin{center}
\begin{tabular}{l}
|\def\version{final}|\\
|\input{childdoc.def}|\\
|\childdocforwardprefix{final}{child}|
\end{tabular}
\end{center}
%

Note that when several versions of a main file and/or of each child file
are to be generated, it may be convenient to set up a |Makefile| or
shell script to automatise the process.

%%%%%%%%%%%%%%%%%%%%%%%%%%%%%%%%%%%%%%%%%%%%%%%%%%%%%%%%%%%%%%%%%%%%%%%%%%%%%%%%
\subsection{Command Line Processing}
\label{sec:commandline}

The effect of redirection files can also be achieved by invoking
the \LaTeX{} compiler with a more elaborate command line.
Most conveniently this should be done as part
of a shell script or a |Makefile|.

When using \textsf{childdoc} in the main file, the following
command lines effectively perform a redirection
(note that depending on the shell being used,
backslashes may have to be doubled: `|\|' $\to$ `|\\|'):
%
\begin{center}
|... -jobname "|\textit{target}|" |\\|"|[\textit{flags}]%
|\input{childdoc.def}\childdocforward[|\textit{main}|]{|\textit{dest}|}"|
\end{center}
%
Here \textit{target} is the name of the output file,
\textit{main} is the name of the main file
and \textit{dest} is the name of the main or child file to be processed
(all filenames without extensions).
The optional argument \textit{main} can be omitted
if \textit{main} matches \textit{dest}.
Optionally, compilation \textit{flags} can be defined via |\def| commands.
This command line makes the \TeX{} engine believe
it is compiling the file \textit{target}
whose content is specified as the latter parameter.
The provided code then forwards the processing to
\textit{main} or \textit{dest} as described in \secref{sec:forward}.

%%%%%%%%%%%%%%%%%%%%%%%%%%%%%%%%%%%%%%%%%%%%%%%%%%%%%%%%%%%%%%%%%%%%%%%%%%%%%%%%
\subsection{Include by Input}
\label{sec:input}

Including child documents by |\include| has some restrictions by design.
Most notably, the content of a child document always occupies
its own set of pages; pages cannot be shared between child documents.
Usually, this behaviour makes perfect sense
because each child document contain an essential part of the document.
However, in some situations it may be desirable to compose
a document from a collection of parts
without having mandatory page breaks between then.
For this case, the package
provides a mechanism to include parts
by |\input| which can also be processed individually.
However, by construction this mechanism
requires manual handling of the content to be output.

%%%%%%%%%%%%%%%%%%%%%%%%%%%%%%%%%%%%%%%%
\DescribeMacro{\ifchilddocmanual}
The main file should be prepared as usual, see \secref{sec:include}.
However, the document body must make a distinction
between processing of an individual part and of the main document, e.g.:
%
\begin{center}
\begin{tabular}{l}
|\ifchilddocmanual|\\
|\input{\childdocname}|\\
|\||else|\\
\textit{document body with }|\input{|\textit{part}|}|\\
|\||fi|
\end{tabular}
\end{center}
%
The conditional |\ifchilddocmanual| is true whenever
a part to be included by |\input| is being compiled,
and the name of the part is stored in |\childdocname|.

%%%%%%%%%%%%%%%%%%%%%%%%%%%%%%%%%%%%%%%%
\DescribeMacro{\childdocby}
Each part to be included by |\input| should start with:
%
\begin{center}
\begin{tabular}{l}
|\input{childdoc.def}|\\
|\childdocby{|\textit{main}|}|\\
\end{tabular}
\end{center}
%
The directive |\childdocby| is similar to |\childdocof|
described in \secref{sec:include},
but the subsequent selection of content must be done manually.
To that end, both |\ifchilddoc| and |\ifchilddocmanual|
will be true upon processing of a part,
and the name of the part is stored in |\childdocname|.
Note that |\jobname| will be set to the filename of the current part
so that each part receives an individual |.aux| file
that does not interfere with the |.aux| file(s) of the main document.
This behaviour can be altered by the alternative form
|\childdocby[*]{|\textit{main}|}| (with a non-empty optional argument)
which uses the |.aux| file of the main document
by setting |\jobname| to \textit{main}.

%%%%%%%%%%%%%%%%%%%%%%%%%%%%%%%%%%%%%%%%%%%%%%%%%%%%%%%%%%%%%%%%%%%%%%%%%%%%%%%%
\subsection{Driver Development}
\label{sec:driver}

The \textsf{childdoc} mechanism can also be use for the development
of definition files such as \LaTeX{} styles or classes.
This case differs from the above setup with multiple parts
included by |\include| in that no |\includeonly| should be invoked.
This can be achieved by starting the include file
(before |\ProvidesPackage|) with:
%
\begin{center}
\begin{tabular}{l}
|\input{childdoc.def}|\\
|\childdocforward{|\textit{main}|}|\\
\end{tabular}
\end{center}
%
or alternatively with:
%
\begin{center}
\begin{tabular}{l}
|\input{childdoc.def}|\\
|\childdocby{|\textit{main}|}|\\
\end{tabular}
\end{center}
%
Both forms have slightly different effects as described above.
The main file is prepared as usual, see \secref{sec:include}.

%%%%%%%%%%%%%%%%%%%%%%%%%%%%%%%%%%%%%%%%%%%%%%%%%%%%%%%%%%%%%%%%%%%%%%%%%%%%%%%%
\subsection{Legacy Detection}
\label{sec:detection}

The directive |\childdocmain| in the main file can detect
whether the complete document or merely a child is to be compiled
even without using the directive |\childdocof|.
This method is deprecated because it is less robust
and there is no compelling reason to use it;
it is merely provided for backward compatibility
and it may be removed in future versions.

If the detection mechanism is to be used,
it is mandatory to correctly specify
the filename of the main file as the argument of |\childdocmain|:
%
\begin{center}
\begin{tabular}{l}
|\input{childdoc.def}|\\
|\childdocmain{|\textit{main}|}|\\
\end{tabular}
\end{center}
%
If |\jobname| does not match the argument \textit{main} of |\childdocmain|,
it is assumed that |\jobname| points to the child file to be compiled.
When using |\childdocmain| with the main file specified as argument,
it suffices to start a child file
with just |\input{|\textit{main}|}|
without loading of the package and using |\childdocof|.
If instead all processing is done
with the appropriate \textsf{childdoc} directives,
the argument of \textit{main} of |\childdocmain| can be empty.

An alternative version of the command line processing described
in \secref{sec:commandline} using the detection mechanism reads:
%
\begin{center}
|... -jobname "|\textit{target}|" "|[\textit{flags}]%
[|\def\jobname{|\textit{dest}|}|]|\input{|\textit{main}|}"|
\end{center}

%%%%%%%%%%%%%%%%%%%%%%%%%%%%%%%%%%%%%%%%%%%%%%%%%%%%%%%%%%%%%%%%%%%%%%%%%%%%%%%%
\subsection{Manual Code}
\label{sec:manual}

In case one cannot be certain whether the definitions file |childdoc.def|
is installed on the target \TeX{} distribution
and one prefers not to ship it,
it is conceivable to paste a few relevant commands into the sources.

To that end, drop all statements |\input{childdoc.def}|
and perform the replacements as outlined below.
Instead of |\childdocmain{|\textit{main}|}| add the following code
to the top of the main file:
%
\begin{center}
\begin{tabular}{l}
|\||ifdefined\childdocname\endinput\||fi\newif\ifchilddoc|\\
|\edef\childdocname{\scantokens\expandafter{\jobname\noexpand}}|\\
|\def\childdocmain{|\textit{main}|}\||ifx\childdocmain\childdocname\||else|\\
|\childdoctrue\includeonly{\childdocname}\let\jobname\childdocmain\||fi|\\
\end{tabular}
\end{center}
%
Instead of |\childdocof{|\textit{main}|}| just include the main file
at the top of each child file:
%
\begin{center}
|\input{|\textit{main}|}|
\end{center}
%
A simple redirection |\childdocforward{|\textit{dest}|}| is achieved by:
%
\begin{center}
|\def\jobname{|\textit{dest}|}\input{\jobname}|
\end{center}
%
The redirection with prefix
|\childdocforwardprefix[|\textit{prefix}|]{|\textit{dest}|}|
is accomplished by:
%
\begin{center}
\begin{tabular}{l}
|{\edef\jobname{\scantokens\expandafter{\jobname\noexpand}}|\\
|\def\redirectjob |\textit{prefix}|#1~~~{\gdef\jobname{|\textit{dest}|#1}}|\\
|\expandafter\redirectjob\jobname~~~}\input{\jobname}|
\end{tabular}
\end{center}

In an alternative approach,
child documents can be compiled by a specific command line
without additional code or specific definitions:
%
\begin{center}
|... -jobname "|\textit{target}|" "|[\textit{flags}]%
|\includeonly{|\textit{dest}|}\input{|\textit{main}|}"|
\end{center}
%

%%%%%%%%%%%%%%%%%%%%%%%%%%%%%%%%%%%%%%%%%%%%%%%%%%%%%%%%%%%%%%%%%%%%%%%%%%%%%%%%
%%%%%%%%%%%%%%%%%%%%%%%%%%%%%%%%%%%%%%%%%%%%%%%%%%%%%%%%%%%%%%%%%%%%%%%%%%%%%%%%
\section{Information}

%%%%%%%%%%%%%%%%%%%%%%%%%%%%%%%%%%%%%%%%%%%%%%%%%%%%%%%%%%%%%%%%%%%%%%%%%%%%%%%%
\subsection{Copyright}

Copyright \copyright{} 2017--2018 Niklas Beisert

This work may be distributed and/or modified under the
conditions of the \LaTeX{} Project Public License, either version 1.3
of this license or (at your option) any later version.
The latest version of this license is in
  \url{http://www.latex-project.org/lppl.txt}
and version 1.3 or later is part of all distributions of \LaTeX{}
version 2005/12/01 or later.

This work has the LPPL maintenance status `maintained'.

The Current Maintainer of this work is Niklas Beisert.

This work consists of the files |README.txt|, |childdoc.ins| and |childdoc.dtx|
as well as the derived files |childdoc.def|, |cdocsamp.tex|
with |cdocsch1.tex|, |cdocsch2.tex|, |cdocspt3.tex|, |cdocspt4.tex|,
|cdocsdrf.tex|, |cdocsfn1.tex|, |cdocsfn2.tex|
as well as |childdoc.pdf|.

%%%%%%%%%%%%%%%%%%%%%%%%%%%%%%%%%%%%%%%%%%%%%%%%%%%%%%%%%%%%%%%%%%%%%%%%%%%%%%%%
\subsection{Files and Installation}

The package consists of the files:
%
\begin{center}
\begin{tabular}{ll}
    |README.txt|   & readme file \\
    |childdoc.ins| & installation file \\
    |childdoc.dtx| & source file \\
    |childdoc.def| & definition file \\
    |cdocsamp.tex| & sample main file \\
    |cdocsch1.tex| & sample include file \\
    |cdocsch2.tex| & sample include file \\
    |cdocspt3.tex| & sample part file \\
    |cdocspt4.tex| & sample part file \\
    |cdocsdrf.tex| & sample redirection file \\
    |cdocsfn1.tex| & sample redirection file \\
    |cdocsfn2.tex| & sample redirection file \\
    |childdoc.pdf| & manual
\end{tabular}
\end{center}
%
The distribution consists of the files
|README.txt|, |childdoc.ins| and |childdoc.dtx|.
%
\begin{itemize}
\item
Run (pdf)\LaTeX{} on |childdoc.dtx|
to compile the manual |childdoc.pdf| (this file).
\item
Run \LaTeX{} on |childdoc.ins| to create the definitions file |childdoc.def|
and the sample |cdocsamp.tex| with include files
|cdocsch1.tex|, |cdocsch2.tex|, |cdocspt3.tex|, |cdocspt4.tex|,
|cdocsdrf.tex|, |cdocsfn1.tex|, |cdocsfn2.tex|.
Then copy the file |childdoc.def| to an appropriate directory of your \LaTeX{}
distribution, e.g.\ \textit{texmf-root}|/tex/latex/childdoc|.
\end{itemize}

%%%%%%%%%%%%%%%%%%%%%%%%%%%%%%%%%%%%%%%%%%%%%%%%%%%%%%%%%%%%%%%%%%%%%%%%%%%%%%%%
\subsection{Related CTAN Packages}

There are several other packages which offer a similar functionality:
%
\begin{itemize}
\item
The packages
\href{http://ctan.org/pkg/docmute}{\textsf{docmute}},
\href{http://ctan.org/pkg/includex}{\textsf{includex}} and
\href{http://ctan.org/pkg/standalone}{\textsf{standalone}}
provide commands to include only the document body of
a child file thus allowing both files to be compiled individually.
\item
The packages \href{http://ctan.org/pkg/subdocs}{\textsf{subdocs}}
and \href{http://ctan.org/pkg/subfiles}{\textsf{subfiles}}
provide structures in which the main and child documents can be
encapsulated and allowing them to be compiled individually.
The inclusion mechanism is different from the conventional |\include|.
\item
The package \href{http://ctan.org/pkg/combine}{\textsf{combine}}
is an elaborate solution to combine several documents into one.
\end{itemize}
%
See also the CTAN topic \href{http://ctan.org/topic/subdocs}{\textsf{subdocs}}
for further related packages.
The present package differs from the above solutions in that
a document structure constructed with the conventional |\include| mechanism
just needs two extra commands at the top of every file
such that all constituent files can be compiled individually.

%%%%%%%%%%%%%%%%%%%%%%%%%%%%%%%%%%%%%%%%%%%%%%%%%%%%%%%%%%%%%%%%%%%%%%%%%%%%%%%%
%\subsection{Feature Suggestions}
%
%The following is a list of features which may be useful for future
%versions of this package:
%%
%\begin{itemize}
%\item
%\ldots
%\end{itemize}

%%%%%%%%%%%%%%%%%%%%%%%%%%%%%%%%%%%%%%%%%%%%%%%%%%%%%%%%%%%%%%%%%%%%%%%%%%%%%%%%
\subsection{Revision History}

%%%%%%%%%%%%%%%%%%%%%%%%%%%%%%%%%%%%%%%%
\paragraph{v2.0:} 2018/12/30

\begin{itemize}
\item
immediate forward processing
\item
added |\childdocby| mechanism
\item
manual restructured
\end{itemize}

%%%%%%%%%%%%%%%%%%%%%%%%%%%%%%%%%%%%%%%%
\paragraph{v1.6:} 2018/01/17

\begin{itemize}
\item
application for development of include files
\item
corrections to manual
\end{itemize}

%%%%%%%%%%%%%%%%%%%%%%%%%%%%%%%%%%%%%%%%
\paragraph{v1.5:} 2017/05/21

\begin{itemize}
\item
more complete structuring introduced
\item
|\childdocof| introduced
\item
|\childdoc| renamed to |\childdocmain|
\item
|\childredirect| renamed to |\childdocforward| and |\childdocforwardprefix|
and functionality expanded
\end{itemize}

%%%%%%%%%%%%%%%%%%%%%%%%%%%%%%%%%%%%%%%%
\paragraph{v1.0:} 2017/04/27

\begin{itemize}
\item
manual and install package
\item
first version published on CTAN
\end{itemize}

%%%%%%%%%%%%%%%%%%%%%%%%%%%%%%%%%%%%%%%%
\paragraph{v0.6:} 2017/04/26

\begin{itemize}
\item
redirection mechanism added
\end{itemize}

%%%%%%%%%%%%%%%%%%%%%%%%%%%%%%%%%%%%%%%%
\paragraph{v0.5:} 2017/04/26

\begin{itemize}
\item
functionality in definition file
\end{itemize}


%%%%%%%%%%%%%%%%%%%%%%%%%%%%%%%%%%%%%%%%%%%%%%%%%%%%%%%%%%%%%%%%%%%%%%%%%%%%%%%%
%%%%%%%%%%%%%%%%%%%%%%%%%%%%%%%%%%%%%%%%%%%%%%%%%%%%%%%%%%%%%%%%%%%%%%%%%%%%%%%%
%%%%%%%%%%%%%%%%%%%%%%%%%%%%%%%%%%%%%%%%%%%%%%%%%%%%%%%%%%%%%%%%%%%%%%%%%%%%%%%%
\appendix

\settowidth\MacroIndent{\rmfamily\scriptsize 000\ }

 \DocInput{childdoc.dtx}

\end{document}
%</driver>
% \fi
%
% %%%%%%%%%%%%%%%%%%%%%%%%%%%%%%%%%%%%%%%%%%%%%%%%%%%%%%%%%%%%%%%%%%%%%%%%%%%%%%
% %%%%%%%%%%%%%%%%%%%%%%%%%%%%%%%%%%%%%%%%%%%%%%%%%%%%%%%%%%%%%%%%%%%%%%%%%%%%%%
% \section{Sample}
%\iffalse
%<*samplemain>
%\fi
%
% The following presents a sample document
% with two chapters, two parts, a title page,
% a compile flag as well as three forwarding files to set the flag.
% It consists of eight |.tex| files:
% \begin{center}
% \begin{tabular}{ll}
% |cdocsamp.tex|&main file\\
% |cdocsch1.tex|&include file for chapter 1\\
% |cdocsch2.tex|&include file for chapter 2\\
% |cdocspt3.tex|&include file for part 3\\
% |cdocspt4.tex|&include file for part 4\\
% |cdocsdrf.tex|&forwarding file for main file in draft mode\\
% |cdocsfi1.tex|&forwarding file for final version of chapter 1\\
% |cdocsfi2.tex|&forwarding file for final version of chapter 2\\
% \end{tabular}
% \end{center}
% Each of the eight files can be compiled directly by the \LaTeX{} compiler.
%
% %%%%%%%%%%%%%%%%%%%%%%%%%%%%%%%%%%%%%%
% \paragraph{Main File.}
%
% The main file is called |cdocsamp.tex|.
%
% Load the \textsf{childdoc} definitions and
% declare the filename for the main document:
%    \begin{macrocode}
\input{childdoc.def}
\childdocmain{}
%    \end{macrocode}

% Optional override for |\version| flag:
%    \begin{macrocode}
%%\ifchilddoc\else\providecommand{\version}{draft}\fi
%    \end{macrocode}

% Define the default values for the |\version| flag
% (|final| for the main file and |draft| for childs):
%    \begin{macrocode}
\ifchilddoc
\providecommand{\version}{draft}
\else
\providecommand{\version}{final}
\fi
%    \end{macrocode}

% Load the standard document class:
%    \begin{macrocode}
\documentclass[12pt]{article}
%    \end{macrocode}

% Start the document body:
%    \begin{macrocode}
\begin{document}
%    \end{macrocode}

% Declare a title page.
% Print title, part of document being processed and version flag:
%    \begin{macrocode}
\addtocounter{page}{-1}
\begin{center}
{\LARGE\bfseries{}childdoc example\par}
\vspace{1cm}
\ifchilddoc
\ifchilddocmanual part\else chapter\fi:
`\childdocname' of `\childdocjob'\par
\else
main document: `\childdocjob'\par
\fi
version: \version\par
\end{center}
\newpage
%    \end{macrocode}

% Manually include selected file,
% otherwise process as usual:
%    \begin{macrocode}
\ifchilddocmanual
\section*{part `\childdocname'}
\input{\childdocname}
\else
%    \end{macrocode}

% Include the two chapters:
%    \begin{macrocode}
\include{cdocsch1}
\include{cdocsch2}
%    \end{macrocode}

% Include the two parts unless only chapters should be displayed:
%    \begin{macrocode}
\ifchilddoc\else
\section{part three}
\input{cdocspt3}
\section{part four}
\input{cdocspt4}
\fi
%    \end{macrocode}

% Process as usual until here:
%    \begin{macrocode}
\fi
%    \end{macrocode}

% End of document body:
%    \begin{macrocode}
\end{document}
%    \end{macrocode}
%\iffalse
%</samplemain>
%\fi
%
% %%%%%%%%%%%%%%%%%%%%%%%%%%%%%%%%%%%%%%
% \paragraph{Chapter Include Files.}
%
% The include files are called |cdocsch1.tex| and |cdocsch2.tex|.
%
%\iffalse
%<*samplechap1|samplechap2>
%\fi

% Optional override for |\version| flag:
%    \begin{macrocode}
%%\providecommand{\version}{final}
%    \end{macrocode}

% Include the main document:
%    \begin{macrocode}
\input{childdoc.def}
\childdocof{cdocsamp}
%    \end{macrocode}

%\iffalse
%</samplechap1|samplechap2>
%\fi
%
%\iffalse
%<*samplechap1>
%\fi
% Some text for chapter 1:
%    \begin{macrocode}
\section{one}
some text in chapter one
%    \end{macrocode}

%\iffalse
%</samplechap1>
%\fi
% Some text for chapter 2:
%\iffalse
%<*samplechap2>
%\fi
%    \begin{macrocode}
\section{two}
more text in chapter two
%    \end{macrocode}

%\iffalse
%</samplechap2>
%\fi
%
% %%%%%%%%%%%%%%%%%%%%%%%%%%%%%%%%%%%%%%
% \paragraph{Part Include Files.}
%
% The include files are called |cdocspt3.tex| and |cdocspt4.tex|.
%
%\iffalse
%<*samplepart3|samplepart4>
%\fi

% Optional override for |\version| flag:
%    \begin{macrocode}
%%\providecommand{\version}{final}
%    \end{macrocode}

% Include the main document:
%    \begin{macrocode}
\input{childdoc.def}
\childdocby{cdocsamp}
%    \end{macrocode}

%\iffalse
%</samplepart3|samplepart4>
%\fi
%
%\iffalse
%<*samplepart3>
%\fi
% Some text for part 3:
%    \begin{macrocode}
some text in part three
%    \end{macrocode}

%\iffalse
%</samplepart3>
%\fi
% Some text for part 4:
%\iffalse
%<*samplepart4>
%\fi
%    \begin{macrocode}
more text in part four
%    \end{macrocode}

%\iffalse
%</samplepart4>
%\fi
%
% %%%%%%%%%%%%%%%%%%%%%%%%%%%%%%%%%%%%%%
% \paragraph{Forwarding for a Complete Draft.}
%
% The following forwarding file |cdocsdrf.tex|
% compiles the main document in draft mode:
%\iffalse
%<*sampledraft>
%\fi
%    \begin{macrocode}
\def\version{draft}
\input{childdoc.def}
\childdocforward{cdocsamp}
%    \end{macrocode}

%\iffalse
%</sampledraft>
%\fi
%
% %%%%%%%%%%%%%%%%%%%%%%%%%%%%%%%%%%%%%%
% \paragraph{Forwarding for Final Version of the Chapters.}
%
% The following forwarding files |cdocsfn1.tex| and |cdocsfn2.tex|
% (with identical content)
% compile the final versions of the child documents
% |cdocsch1.tex| and |cdocsch2.tex|, respectively:
%\iffalse
%<*samplefinal>
%\fi
%    \begin{macrocode}
\def\version{final}
\input{childdoc.def}
\childdocforwardprefix[cdocsamp]{cdocsfn}{cdocsch}
%    \end{macrocode}

%\iffalse
%</samplefinal>
%\fi
%
% %%%%%%%%%%%%%%%%%%%%%%%%%%%%%%%%%%%%%%
% \paragraph{Command Line Processing.}
%
% The following three command lines generate the output files
% |cdocscld|, |cdocscl1| and |cdocscl2|
% which should be identical to
% |cdocsdrf|, |cdocsch1| and |cdocsfn2|, respectively:
% \begin{center}
% \begin{tabular}{l}
% |latex -jobname cdocscld \|\\
% |  "\def\version{draft}\input{childdoc.def}\childdocforward{cdocsamp}"|\\
% |latex -jobname cdocscl1 \|\\
% |  "\input{childdoc.def}\childdocforward[cdocsamp]{cdocsch1}"|\\
% |latex -jobname cdocscl2 \|\\
% |  "\def\version{final}\input{childdoc.def}\childdocforward{cdocsch2}"|
% \end{tabular}
% \end{center}
% Note that the trailing backslash on each first line
% merely continues the input to the second line
% (for convenient cut ant paste).
% Furthermore, the command |latex| can be replaced by any
% of its alternative versions such as |pdflatex|.
%
% %%%%%%%%%%%%%%%%%%%%%%%%%%%%%%%%%%%%%%%%%%%%%%%%%%%%%%%%%%%%%%%%%%%%%%%%%%%%%%
% %%%%%%%%%%%%%%%%%%%%%%%%%%%%%%%%%%%%%%%%%%%%%%%%%%%%%%%%%%%%%%%%%%%%%%%%%%%%%%
% \section{Implementation}
%\iffalse
%<*package>
%\fi
%
% This section describes the definitions file |childdoc.def|.

% The definitions cannot be loaded using |\usepackage| or |\RequirePackage|
% which has a mechanism to prevent loading a style file more than once.
% When loading the definitions by means of |\input|
% multiple instances have to be prevented manually:
%\iffalse
%This code needs to be before the `\ProvidesFile' directive
%which is defined at the beginning of this file.
%Therefore it is also placed there and commented out here.
%</package>
%<*discard>
%\fi
%    \begin{macrocode}
\ifdefined\childdocmain\endinput\fi
%    \end{macrocode}
%\iffalse
%</discard>
%<*package>
%\fi
%
% \macro{\ifchilddoc}
% \macro{\ifchilddocmanual}
% The conditional |\ifchilddoc| tells whether a
% child (true) or main (false) document is being compiled.
% The conditional |\ifchilddocmanual| tells whether
% the |\includeonly| mechanism is used (false) or
% the selection of child files must be performed manually (true).
% The definitions initialise to false:
%    \begin{macrocode}
\newif\ifchilddoc
\newif\ifchilddocmanual
%    \end{macrocode}

% \macro{\childdocname}
% \macro{\childdocjob}
% The macro |\childdocname| stores the name of the main document
% to be compiled. The macro |\childdocjob| stores the name of
% the document on which the \LaTeX{} compiler was originally invoked.
% The content of |\jobname| cannot be compared
% to filenames specified in the source due to different catcodes.
% The following code rescans |\jobname|, stores the result
% in |\childdocname| and saves a copy in |\childdocjob|:
%    \begin{macrocode}
\edef\childdocname{\scantokens\expandafter{\jobname\noexpand}}
\let\childdocjob\childdocname
%    \end{macrocode}

% \macro{\childdocdisable}
% The macro |\childdocdisable| prevents the main file
% from being processed more than once.
% At this stage, the main document command |\childdocmain|
% is assumed to be called once again where it should do nothing.
% Any subsequent call to it should prevent
% a secondary processing of the main document
% It overwrites the forwarding commands
% |\childdocof| and |\childdocforward|
% with empty macros to prevent further inclusions of the main document:
%    \begin{macrocode}
\newcommand{\childdocdisable}
{
  \renewcommand{\childdocmain}[1]{\renewcommand{\childdocmain}[1]{\endinput}}
  \renewcommand{\childdocof}[1]{}
  \renewcommand{\childdocby}[2][]{}
  \renewcommand{\childdocforward}[2][]{}
  \renewcommand{\childdocdisable}{}
}
%    \end{macrocode}

% \macro{\childdocmain}
% The macro |\childdocmain| is to be called at the top of the main file
% with nothing or the main filename (without extension) as argument.
% First, it breaks loops.
% If the argument is not empty and does not match |\childdocname|
% (which is set by the first inclusion of |childdoc.def|),
% |\ifchilddoc| is set to true, |\includeonly| is applied to the child file
% and |\jobname| is set to the main file
% (for proper handling of |.aux| files):
%    \begin{macrocode}
\newcommand{\childdocmain}[1]
{
  \childdocdisable\childdocmain{}
  \if?#1?\else
    \begingroup
      \def\childdoctmp{#1}
      \ifx\childdoctmp\childdocname
        \def\childdoctmp{}
      \else
        \def\childdoctmp
        {
          \childdoctrue
          \includeonly{\childdocname}
          \def\childdocjob{#1}
          \def\jobname{#1}
        }
      \fi
      \expandafter
    \endgroup
    \childdoctmp
  \fi
}
%    \end{macrocode}

% \macro{\childdocof}
% The command |\childdocof| redirects
% compilation to the main file |#1|.
%    \begin{macrocode}
\newcommand{\childdocof}[1]
{
  \childdocdisable
  \childdoctrue
  \includeonly{\childdocname}
  \def\jobname{#1}
  \def\childdocjob{#1}
  \input{#1}
}
%    \end{macrocode}

% \macro{\childdocby}
% The command |\childdocby| ....
%    \begin{macrocode}
\newcommand{\childdocby}[2][]
{
  \childdocdisable
  \childdoctrue
  \childdocmanualtrue
  \if?#1?\else
    \def\jobname{#2}
  \fi
  \def\childdocjob{#2}
  \input{#2}
  \endinput
}
%    \end{macrocode}

% \macro{\childdocforward}
% The command |\childdocforward| redirects
% compilation to the main file or
% (if the optional argument is given) a child file.
% Parameters are set as if the main file
% or a child file starting with |\childdocof| was compiled.
% Then compilation is handed over to the main file:
%    \begin{macrocode}
\newcommand{\childdocforward}[2][]
{
  \begingroup
    \if?#1?
      \def\childdoctmp
      {
        \def\childdocname{#2}
        \def\childdocjob{#2}
        \def\jobname{#2}
        \input{#2}
        \endinput
      }
    \else
      \def\childdoctmp
      {
        \childdocdisable
        \def\childdocname{#2}
        \childdoctrue
        \includeonly{#2}
        \def\childdocjob{#1}
        \def\jobname{#1}
        \input{#1}
        \endinput
      }
    \fi
    \expandafter
  \endgroup
  \childdoctmp
}
%    \end{macrocode}

% \macro{\childdocforwardprefix}
% The command |\childdocforwardprefix| redirects
% compilation to the main or a child file by means of a pattern.
% The prefix |#1| in the current filename is replaced by |#2|
% and the suffix of the current filename is kept
% (it is assumed that the filename does not contain the substring `|~~~|'
% which is used as a delimiter).
% Compilation is handed over to the new file by |\childdocforward|:
%    \begin{macrocode}
\newcommand{\childdocforwardprefix}[3][]
{
  \begingroup
    \def\childdocextract #2##1~~~{\def\childdoctmp{\childdocforward[#1]{#3##1}}}
    \expandafter\childdocextract\childdocname~~~
    \expandafter
  \endgroup
  \childdoctmp
}
%    \end{macrocode}

% \macro{\childdoc}
% The deprecated macro |\childdoc| is a legacy version of |\childdocmain|:
%    \begin{macrocode}
\newcommand{\childdoc}{\childdocmain}
%    \end{macrocode}

% \macro{\childdocredirect}
% The deprecated macro |\childdocredirect| is a legacy version
% of |\childdocforward| and |\childdocforwardprefix|:
%    \begin{macrocode}
\newcommand{\childdocredirect}[2][]
{
  \begingroup
    \if?#1?
      \def\childdoctmp{\childdocforward{#2}}
    \else
      \def\childdoctmp{\childdocforwardprefix{#1}{#2}}
    \fi
    \expandafter
  \endgroup
  \childdoctmp
}
%    \end{macrocode}

%\iffalse
%</package>
%\fi
%
\endinput
\childdocforward[cdocsamp]{cdocsch1}"|\\
% |latex -jobname cdocscl2 \|\\
% |  "\def\version{final}% \iffalse
%
% childdoc.dtx Copyright (C) 2017-2018 Niklas Beisert
%
% This work may be distributed and/or modified under the
% conditions of the LaTeX Project Public License, either version 1.3
% of this license or (at your option) any later version.
% The latest version of this license is in
%   http://www.latex-project.org/lppl.txt
% and version 1.3 or later is part of all distributions of LaTeX
% version 2005/12/01 or later.
%
% This work has the LPPL maintenance status `maintained'.
%
% The Current Maintainer of this work is Niklas Beisert.
%
% This work consists of the files childdoc.dtx and childdoc.ins
% and the derived files childdoc.def and cdocsamp.tex with
% cdocsch1.tex, cdocsch2.tex, cdocsdrf.tex, cdocsfn1.tex, cdocsfn2.tex.
%
%<package>\ifdefined\childdocmain\endinput\fi
%<package>\ProvidesFile{childdoc.def}[2018/12/30 v2.0 child document driver]
%<samplemain>\ProvidesFile{cdocsamp.tex}[2018/12/30 v2.0 sample for childdoc]
%<*driver>
%\ProvidesFile{childdoc.drv}[2018/12/30 v2.0 childdoc reference manual file]
\PassOptionsToClass{10pt,a4paper}{article}
\documentclass{ltxdoc}

\usepackage[margin=35mm]{geometry}
\usepackage{hyperref}
\usepackage{hyperxmp}
\usepackage[usenames]{color}

\hypersetup{colorlinks=true}
\hypersetup{pdfstartview=FitH}
\hypersetup{pdfpagemode=UseNone}
\hypersetup{pdfsource={}}
\hypersetup{pdflang={en-UK}}
\hypersetup{pdfcopyright={Copyright 2017-2018 Niklas Beisert.
  This work may be distributed and/or modified under the
  conditions of the LaTeX Project Public License, either version 1.3
  of this license or (at your option) any later version.}}
\hypersetup{pdflicenseurl={http://www.latex-project.org/lppl.txt}}
\hypersetup{pdfcontactaddress={ETH Zurich, ITP, HIT K,
  Wolfgang-Pauli-Strasse 27}}
\hypersetup{pdfcontactpostcode={8093}}
\hypersetup{pdfcontactcity={Zurich}}
\hypersetup{pdfcontactcountry={Switzerland}}
\hypersetup{pdfcontactemail={nbeisert@itp.phys.ethz.ch}}
\hypersetup{pdfcontacturl={http://people.phys.ethz.ch/\xmptilde nbeisert/}}

\newcommand{\secref}[1]{\hyperref[#1]{section \ref*{#1}}}

\parskip1ex
\parindent0pt
\let\olditemize\itemize
\def\itemize{\olditemize\parskip0pt}

\begin{document}

\title{The \textsf{childdoc} Package}
\hypersetup{pdftitle={The childdoc Package}}
\author{Niklas Beisert\\[2ex]
  Institut f\"ur Theoretische Physik\\
  Eidgen\"ossische Technische Hochschule Z\"urich\\
  Wolfgang-Pauli-Strasse 27, 8093 Z\"urich, Switzerland\\[1ex]
  \href{mailto:nbeisert@itp.phys.ethz.ch}
  {\texttt{nbeisert@itp.phys.ethz.ch}}}
\hypersetup{pdfauthor={Niklas Beisert}}
\hypersetup{pdfsubject={Manual for the LaTeX2e Package childdoc}}
\date{30 December 2018, \textsf{v2.0}}
\maketitle

\begin{abstract}\noindent
\textsf{childdoc} is a \LaTeXe{} package
that enables the direct compilation
of document sections included by |\include|
to individual files.
\end{abstract}

\begingroup
\parskip0ex
\tableofcontents
\endgroup

%%%%%%%%%%%%%%%%%%%%%%%%%%%%%%%%%%%%%%%%%%%%%%%%%%%%%%%%%%%%%%%%%%%%%%%%%%%%%%%%
%%%%%%%%%%%%%%%%%%%%%%%%%%%%%%%%%%%%%%%%%%%%%%%%%%%%%%%%%%%%%%%%%%%%%%%%%%%%%%%%
\section{Introduction}

\LaTeX{} provides a mechanism to structure a large document (such as a book)
into a main file and several child files (containing the chapters)
using the |\include| command.
This mechanism is beneficial for documents
which span hundreds of pages in order to
make the source file(s) more manageable.
Moreover, compilation can be restricted to
selected child files by means of the |\includeonly| command.
The latter feature can be used to reduce the compilation time while editing
(this was significantly more useful in the earlier days of \LaTeX{})
or to generate a smaller document which is easier to navigate.
Another application of |\includeonly| is to generate
documents consisting of selected parts of the complete document.

However, there are a few drawbacks of the plain |\include| mechanism:
\begin{itemize}
\item
The child files cannot be compiled on their own,
they can only be compiled via the main file.
A naive editing environment
(such as a text editor with an option
to have the current file processed by \LaTeX)
may require one to switch to the main file before compiling;
attempting to compile the child file produces errors.
\item
The main file must be modified (each time)
to adjust the |\includeonly| command
to the present needs. This easily leaves the main file in a messy state.
\item
The generated document will always carry the filename
of the main document. This is inconvenient if
several child files are to be compiled and
to be kept for distribution.
\end{itemize}

The present package provides a simple interface
to make child files individually compilable by \LaTeX{}.
Compiling a child file then has the same effect as compiling
the main file with an |\includeonly| command
to select the appropriate child.
Moreover the generated document will carry the name of the child
rather than the main file.
This resolves all three above issues.

This feature is meant to make the editing of books,
thesis documents and lecture notes somewhat more convenient.
However, the package can also be used efficiently for
composing a series of documents (such as exercise sheets)
which are typically distributed individually.
It then assists the author in generating the individual documents
(potentially in different versions)
as well as a document containing the collected series.
Another application is in developing style files
or other kinds of included material
where compilation of the style file could redirect
to a sample or test file.

%%%%%%%%%%%%%%%%%%%%%%%%%%%%%%%%%%%%%%%%%%%%%%%%%%%%%%%%%%%%%%%%%%%%%%%%%%%%%%%%
%%%%%%%%%%%%%%%%%%%%%%%%%%%%%%%%%%%%%%%%%%%%%%%%%%%%%%%%%%%%%%%%%%%%%%%%%%%%%%%%
\section{Usage}

First of all, the package \textsf{childdoc} is \emph{not} a standard
\LaTeXe{} |.sty| style file! Therefore it needs to be invoked in
a non-standard way.

%%%%%%%%%%%%%%%%%%%%%%%%%%%%%%%%%%%%%%%%%%%%%%%%%%%%%%%%%%%%%%%%%%%%%%%%%%%%%%%%
\subsection{Included Files}
\label{sec:include}

%%%%%%%%%%%%%%%%%%%%%%%%%%%%%%%%%%%%%%%%
\DescribeMacro{\childdocmain}
To use the package, add the commands
\begin{center}
\begin{tabular}{l}
|\input{childdoc.def}|\\
|\childdocmain{}|\\
\end{tabular}
\end{center}
at the very top of the main \LaTeX{} file,
in particular \emph{before} the |\documentclass| statement!
The argument of |\childdocmain| should be left empty
(but it must be present).

%%%%%%%%%%%%%%%%%%%%%%%%%%%%%%%%%%%%%%%%
\DescribeMacro{\childdocof}
Furthermore, add the commands
\begin{center}
\begin{tabular}{l}
|\input{childdoc.def}|\\
|\childdocof{|\textit{main}|}|\\
\end{tabular}
\end{center}
at the top of every child file \textit{child}
which is included by |\include{|\textit{child}|}|
from within the main file
(or at least for those files to be compiled individually).
The argument \textit{main} must be the filename of the main file.

There are a couple of
considerations in setting up the main and child documents:

%%%%%%%%%%%%%%%%%%%%%%%%%%%%%%%%%%%%%%%%
\paragraph{Restrictions.}

Please note the following restrictions:
\begin{itemize}
\item
|\childdocmain| must be called with one argument \textit{main}
to ensure compatibility with earlier version of the package.
It must either be empty (|\childdocmain{}|)
or precisely match the filename of the main file in which it is specified.
See \secref{sec:detection} for further information.
\item
The filename \textit{main} must be specified without the |.tex| extension.
\item
The filename \textit{main} is case sensitive
(even in case-insensitive file systems)
due to internal string comparison.
\item
The argument \textit{main} should be fully expanded, it cannot be a macro.
\item
Subdirectories and special characters should be avoided in filenames.
\item
The command |\childdocmain{|\textit{main}|}| must be followed by a whitespace.
It should not be followed immediately by another command
or by a comment mark `|%|'.
This is because the \TeX{} parser reads the token immediately following
the argument of |\childdocmain| and puts it
at the beginning of every child section;
however, a white\-space is ignored.
\end{itemize}

%%%%%%%%%%%%%%%%%%%%%%%%%%%%%%%%%%%%%%%%
\paragraph{Content of Main File.}

It is advisable to place all content in the child files included by |\include|.
Any output contained in the main file will appear in all child documents
unless suppressed manually;
it cannot be suppressed automatically by the |\includeonly| directive
and thus should normally be avoided.
A method to include some content in the main file
by means of conditional processing is described in \secref{sec:conditional}.

%%%%%%%%%%%%%%%%%%%%%%%%%%%%%%%%%%%%%%%%
\paragraph{Page Numbering.}

When only a part of the document is compiled,
the appropriate numbering of pages
(as well as other status parameters)
is determined from the |.aux| files.
The latter contain information from previous passes.
However this information needs to propagate through
all intermediate child documents.
Therefore the page numbering in child documents may well
be inconsistent until the complete document is compiled at least once.

A useful (if unconventional) way to always ensure a consistent
page numbering is to restart the numbering in each child document
and denote the pages by `\textit{child}|.|\textit{page}'
where \textit{child} represents the chapter/section number of the child file.
This can be achieved by the command
|\numberwithin{page}{|\textit{child}|}|
of the \textsf{amsmath} package
where \textit{child} can be |chapter| or |section|
depending on the chosen structuring.
Alternatively, one can modify the macro |\thepage| appropriately
and reset the counter |page| at the start of each child file.

%%%%%%%%%%%%%%%%%%%%%%%%%%%%%%%%%%%%%%%%%%%%%%%%%%%%%%%%%%%%%%%%%%%%%%%%%%%%%%%%
\subsection{Conditional Processing}
\label{sec:conditional}

The package provides a mechanism to compile different versions
of a document. To customise the versions further some conditional processing
can come in handy to distinguish which version is being compiled.
The package provides two macros to describe the compilation context:

%%%%%%%%%%%%%%%%%%%%%%%%%%%%%%%%%%%%%%%%
\DescribeMacro{\ifchilddoc}
The conditional |\ifchilddoc| distinguishes between the compilation of
child documents and the main document:
%
\begin{center}
|\ifchilddoc |\textit{child-code}| |[|\||else |\textit{main-code}]| \||fi|
\end{center}

%%%%%%%%%%%%%%%%%%%%%%%%%%%%%%%%%%%%%%%%
\DescribeMacro{\childdocname}
\DescribeMacro{\childdocjob}
The macro |\childdocname| contains the filename (without extension)
of the main or child file being processed.
Note that |\childdocjob| will always contain the name of the main file.

%%%%%%%%%%%%%%%%%%%%%%%%%%%%%%%%%%%%%%%%
\paragraph{Title Page.}

Conditional processing can be used to include a title or banner page
in the main document when proper precautions are taken.
Importantly, the code in the main file should ensure that the page counter
(as well as other status parameters which are stored in the |.aux| files)
takes the same value after the conditional processing.
Otherwise the page numbers may take divergent values
depending on which part is compiled.

For example, a title page could be declared by:
%
\begin{center}
\begin{tabular}{l}
|\ifchilddoc\||else|\\
|\addtocounter{page}{-1}|\\
\textit{code for title page}\\
|\newpage|\\
|\||fi|
\end{tabular}
\end{center}
%
A banner page for the child documents can be generated by:
%
\begin{center}
\begin{tabular}{l}
|\ifchilddoc|\\
|\addtocounter{page}{-1}|\\
\textit{code for banner page}\\
|\newpage|\\
|\||fi|
\end{tabular}
\end{center}
%
Here one could write a message such as:
\begin{center}
|This is the part \childdocname{} of \childdocjob{}.|
\end{center}

%%%%%%%%%%%%%%%%%%%%%%%%%%%%%%%%%%%%%%%%%%%%%%%%%%%%%%%%%%%%%%%%%%%%%%%%%%%%%%%%
\subsection{Flags}
\label{sec:flags}

The package makes it easy to generate different versions
of the main or child documents.
To this end compilation flags can be defined
and assigned different default values.
They will be particularly useful in conjunction
with the forwarding mechanism described in \secref{sec:forward}.

For example, it may be useful to have a flag |\version|
which can be set to |draft| or |final|.
The document source will contain some conditional code
depending on the value of |\version|.
Suppose further, the flag should default to |final| for the main file
and to |draft| for child files
which is a natural assignment for editing the document.
This is achieved by placing the following code
in the preamble of the main document
(below the |\childdocmain| directive):
%
\begin{center}
\begin{tabular}{l}
|\ifchilddoc|\\
|\providecommand{\version}{draft}|\\
|\||else|\\
|\providecommand{\version}{final}|\\
|\||fi|
\end{tabular}
\end{center}
%
The definition by |\providecommand| makes sure
that previous definitions are not overwritten.
Further statements |\providecommand{\version}{...}|
can thus be added before the above code to override it.

For the main file, one might add a line
(between |\childdocmain| and the above block)
%
\begin{center}
|%\ifchilddoc\||else\providecommand{\version}{draft}\||fi|
\end{center}
%
which can be uncommented to produce a draft version.
Likewise one can add a line to the very top of a child file
(above the |\childdocof{|\textit{main}|}| directive)
%
\begin{center}
|%\providecommand{\version}{final}|
\end{center}
%
which can be uncommented to produce the final version of this child document.

%%%%%%%%%%%%%%%%%%%%%%%%%%%%%%%%%%%%%%%%%%%%%%%%%%%%%%%%%%%%%%%%%%%%%%%%%%%%%%%%
\subsection{Forwarding}
\label{sec:forward}

Different versions of the main or child documents
using compilation flags as described in \secref{sec:flags}
can be (permanently) stored in different files
for convenient compilation, viewing and distribution.
To this end, the package defines a command
to pass on compilation to a different file:

%%%%%%%%%%%%%%%%%%%%%%%%%%%%%%%%%%%%%%%%
\DescribeMacro{\childdocforward}
The command |\childdocforward| redirects processing to
another source file:
%
\begin{center}
\begin{tabular}{l}
|\input{childdoc.def}|\\
|\childdocforward[|\textit{main}|]{|\textit{dest}|}|\\
\end{tabular}
\end{center}
%
The argument \textit{dest} is the destination file
(without extension).
It should be the main file or one of the child files.
Note that further \textsf{childdoc} directives
such as |\childdocof| and |\childdocforward|
in the indicated file will be processed in this form.
The optional argument \textit{main}
passes on directly to the main file \textit{main}
while pretending to compile the child \textit{dest}.
This form behaves as if \textit{dest}
issues |\childdocof{|\textit{main}|}| right away,
and no further \textsf{childdoc} directives will be processed.

%%%%%%%%%%%%%%%%%%%%%%%%%%%%%%%%%%%%%%%%
\DescribeMacro{\...prefix}
In the alternative form |\childdocforwardprefix|,
%
\begin{center}
\begin{tabular}{l}
|\input{childdoc.def}|\\
|\childdocforwardprefix[|\textit{main}|]{|\textit{prefix}|}{|\textit{dest}|}|
\end{tabular}
\end{center}
%
the destination file is determined by a pattern
depending on the current file:
To make this work, the current file must be called
`{\textit{prefix}\hspace{0.2em}\textit{suffix}}'
with \textit{prefix} matching precisely the argument.
Processing is then passed on to the file
`{\textit{dest}\hspace{0.2em}\textit{suffix}}'.
Surely, the same effect is achieved by
directly specifying the
argument `{\textit{dest}\hspace{0.2em}\textit{suffix}}'
in the first form.
However, that requires to set up a different file
for each child. With the alternative form of the command
all these files can have exactly the same content
which simplifies setting them up and maintaining them.

For example, the following file |draft.tex|
with a compilation flag |\version| as described in \secref{sec:flags}
compiles the main document as a draft:
%
\begin{center}
\begin{tabular}{l}
|\def\version{draft}|\\
|\input{childdoc.def}|\\
|\childdocforward{|\textit{main}|}|
\end{tabular}
\end{center}
%
Likewise, the following files |final|\textit{nn}|.tex|
compile the final version of the child document
|child|\textit{nn}|.tex|:
%
\begin{center}
\begin{tabular}{l}
|\def\version{final}|\\
|\input{childdoc.def}|\\
|\childdocforwardprefix{final}{child}|
\end{tabular}
\end{center}
%

Note that when several versions of a main file and/or of each child file
are to be generated, it may be convenient to set up a |Makefile| or
shell script to automatise the process.

%%%%%%%%%%%%%%%%%%%%%%%%%%%%%%%%%%%%%%%%%%%%%%%%%%%%%%%%%%%%%%%%%%%%%%%%%%%%%%%%
\subsection{Command Line Processing}
\label{sec:commandline}

The effect of redirection files can also be achieved by invoking
the \LaTeX{} compiler with a more elaborate command line.
Most conveniently this should be done as part
of a shell script or a |Makefile|.

When using \textsf{childdoc} in the main file, the following
command lines effectively perform a redirection
(note that depending on the shell being used,
backslashes may have to be doubled: `|\|' $\to$ `|\\|'):
%
\begin{center}
|... -jobname "|\textit{target}|" |\\|"|[\textit{flags}]%
|\input{childdoc.def}\childdocforward[|\textit{main}|]{|\textit{dest}|}"|
\end{center}
%
Here \textit{target} is the name of the output file,
\textit{main} is the name of the main file
and \textit{dest} is the name of the main or child file to be processed
(all filenames without extensions).
The optional argument \textit{main} can be omitted
if \textit{main} matches \textit{dest}.
Optionally, compilation \textit{flags} can be defined via |\def| commands.
This command line makes the \TeX{} engine believe
it is compiling the file \textit{target}
whose content is specified as the latter parameter.
The provided code then forwards the processing to
\textit{main} or \textit{dest} as described in \secref{sec:forward}.

%%%%%%%%%%%%%%%%%%%%%%%%%%%%%%%%%%%%%%%%%%%%%%%%%%%%%%%%%%%%%%%%%%%%%%%%%%%%%%%%
\subsection{Include by Input}
\label{sec:input}

Including child documents by |\include| has some restrictions by design.
Most notably, the content of a child document always occupies
its own set of pages; pages cannot be shared between child documents.
Usually, this behaviour makes perfect sense
because each child document contain an essential part of the document.
However, in some situations it may be desirable to compose
a document from a collection of parts
without having mandatory page breaks between then.
For this case, the package
provides a mechanism to include parts
by |\input| which can also be processed individually.
However, by construction this mechanism
requires manual handling of the content to be output.

%%%%%%%%%%%%%%%%%%%%%%%%%%%%%%%%%%%%%%%%
\DescribeMacro{\ifchilddocmanual}
The main file should be prepared as usual, see \secref{sec:include}.
However, the document body must make a distinction
between processing of an individual part and of the main document, e.g.:
%
\begin{center}
\begin{tabular}{l}
|\ifchilddocmanual|\\
|\input{\childdocname}|\\
|\||else|\\
\textit{document body with }|\input{|\textit{part}|}|\\
|\||fi|
\end{tabular}
\end{center}
%
The conditional |\ifchilddocmanual| is true whenever
a part to be included by |\input| is being compiled,
and the name of the part is stored in |\childdocname|.

%%%%%%%%%%%%%%%%%%%%%%%%%%%%%%%%%%%%%%%%
\DescribeMacro{\childdocby}
Each part to be included by |\input| should start with:
%
\begin{center}
\begin{tabular}{l}
|\input{childdoc.def}|\\
|\childdocby{|\textit{main}|}|\\
\end{tabular}
\end{center}
%
The directive |\childdocby| is similar to |\childdocof|
described in \secref{sec:include},
but the subsequent selection of content must be done manually.
To that end, both |\ifchilddoc| and |\ifchilddocmanual|
will be true upon processing of a part,
and the name of the part is stored in |\childdocname|.
Note that |\jobname| will be set to the filename of the current part
so that each part receives an individual |.aux| file
that does not interfere with the |.aux| file(s) of the main document.
This behaviour can be altered by the alternative form
|\childdocby[*]{|\textit{main}|}| (with a non-empty optional argument)
which uses the |.aux| file of the main document
by setting |\jobname| to \textit{main}.

%%%%%%%%%%%%%%%%%%%%%%%%%%%%%%%%%%%%%%%%%%%%%%%%%%%%%%%%%%%%%%%%%%%%%%%%%%%%%%%%
\subsection{Driver Development}
\label{sec:driver}

The \textsf{childdoc} mechanism can also be use for the development
of definition files such as \LaTeX{} styles or classes.
This case differs from the above setup with multiple parts
included by |\include| in that no |\includeonly| should be invoked.
This can be achieved by starting the include file
(before |\ProvidesPackage|) with:
%
\begin{center}
\begin{tabular}{l}
|\input{childdoc.def}|\\
|\childdocforward{|\textit{main}|}|\\
\end{tabular}
\end{center}
%
or alternatively with:
%
\begin{center}
\begin{tabular}{l}
|\input{childdoc.def}|\\
|\childdocby{|\textit{main}|}|\\
\end{tabular}
\end{center}
%
Both forms have slightly different effects as described above.
The main file is prepared as usual, see \secref{sec:include}.

%%%%%%%%%%%%%%%%%%%%%%%%%%%%%%%%%%%%%%%%%%%%%%%%%%%%%%%%%%%%%%%%%%%%%%%%%%%%%%%%
\subsection{Legacy Detection}
\label{sec:detection}

The directive |\childdocmain| in the main file can detect
whether the complete document or merely a child is to be compiled
even without using the directive |\childdocof|.
This method is deprecated because it is less robust
and there is no compelling reason to use it;
it is merely provided for backward compatibility
and it may be removed in future versions.

If the detection mechanism is to be used,
it is mandatory to correctly specify
the filename of the main file as the argument of |\childdocmain|:
%
\begin{center}
\begin{tabular}{l}
|\input{childdoc.def}|\\
|\childdocmain{|\textit{main}|}|\\
\end{tabular}
\end{center}
%
If |\jobname| does not match the argument \textit{main} of |\childdocmain|,
it is assumed that |\jobname| points to the child file to be compiled.
When using |\childdocmain| with the main file specified as argument,
it suffices to start a child file
with just |\input{|\textit{main}|}|
without loading of the package and using |\childdocof|.
If instead all processing is done
with the appropriate \textsf{childdoc} directives,
the argument of \textit{main} of |\childdocmain| can be empty.

An alternative version of the command line processing described
in \secref{sec:commandline} using the detection mechanism reads:
%
\begin{center}
|... -jobname "|\textit{target}|" "|[\textit{flags}]%
[|\def\jobname{|\textit{dest}|}|]|\input{|\textit{main}|}"|
\end{center}

%%%%%%%%%%%%%%%%%%%%%%%%%%%%%%%%%%%%%%%%%%%%%%%%%%%%%%%%%%%%%%%%%%%%%%%%%%%%%%%%
\subsection{Manual Code}
\label{sec:manual}

In case one cannot be certain whether the definitions file |childdoc.def|
is installed on the target \TeX{} distribution
and one prefers not to ship it,
it is conceivable to paste a few relevant commands into the sources.

To that end, drop all statements |\input{childdoc.def}|
and perform the replacements as outlined below.
Instead of |\childdocmain{|\textit{main}|}| add the following code
to the top of the main file:
%
\begin{center}
\begin{tabular}{l}
|\||ifdefined\childdocname\endinput\||fi\newif\ifchilddoc|\\
|\edef\childdocname{\scantokens\expandafter{\jobname\noexpand}}|\\
|\def\childdocmain{|\textit{main}|}\||ifx\childdocmain\childdocname\||else|\\
|\childdoctrue\includeonly{\childdocname}\let\jobname\childdocmain\||fi|\\
\end{tabular}
\end{center}
%
Instead of |\childdocof{|\textit{main}|}| just include the main file
at the top of each child file:
%
\begin{center}
|\input{|\textit{main}|}|
\end{center}
%
A simple redirection |\childdocforward{|\textit{dest}|}| is achieved by:
%
\begin{center}
|\def\jobname{|\textit{dest}|}\input{\jobname}|
\end{center}
%
The redirection with prefix
|\childdocforwardprefix[|\textit{prefix}|]{|\textit{dest}|}|
is accomplished by:
%
\begin{center}
\begin{tabular}{l}
|{\edef\jobname{\scantokens\expandafter{\jobname\noexpand}}|\\
|\def\redirectjob |\textit{prefix}|#1~~~{\gdef\jobname{|\textit{dest}|#1}}|\\
|\expandafter\redirectjob\jobname~~~}\input{\jobname}|
\end{tabular}
\end{center}

In an alternative approach,
child documents can be compiled by a specific command line
without additional code or specific definitions:
%
\begin{center}
|... -jobname "|\textit{target}|" "|[\textit{flags}]%
|\includeonly{|\textit{dest}|}\input{|\textit{main}|}"|
\end{center}
%

%%%%%%%%%%%%%%%%%%%%%%%%%%%%%%%%%%%%%%%%%%%%%%%%%%%%%%%%%%%%%%%%%%%%%%%%%%%%%%%%
%%%%%%%%%%%%%%%%%%%%%%%%%%%%%%%%%%%%%%%%%%%%%%%%%%%%%%%%%%%%%%%%%%%%%%%%%%%%%%%%
\section{Information}

%%%%%%%%%%%%%%%%%%%%%%%%%%%%%%%%%%%%%%%%%%%%%%%%%%%%%%%%%%%%%%%%%%%%%%%%%%%%%%%%
\subsection{Copyright}

Copyright \copyright{} 2017--2018 Niklas Beisert

This work may be distributed and/or modified under the
conditions of the \LaTeX{} Project Public License, either version 1.3
of this license or (at your option) any later version.
The latest version of this license is in
  \url{http://www.latex-project.org/lppl.txt}
and version 1.3 or later is part of all distributions of \LaTeX{}
version 2005/12/01 or later.

This work has the LPPL maintenance status `maintained'.

The Current Maintainer of this work is Niklas Beisert.

This work consists of the files |README.txt|, |childdoc.ins| and |childdoc.dtx|
as well as the derived files |childdoc.def|, |cdocsamp.tex|
with |cdocsch1.tex|, |cdocsch2.tex|, |cdocspt3.tex|, |cdocspt4.tex|,
|cdocsdrf.tex|, |cdocsfn1.tex|, |cdocsfn2.tex|
as well as |childdoc.pdf|.

%%%%%%%%%%%%%%%%%%%%%%%%%%%%%%%%%%%%%%%%%%%%%%%%%%%%%%%%%%%%%%%%%%%%%%%%%%%%%%%%
\subsection{Files and Installation}

The package consists of the files:
%
\begin{center}
\begin{tabular}{ll}
    |README.txt|   & readme file \\
    |childdoc.ins| & installation file \\
    |childdoc.dtx| & source file \\
    |childdoc.def| & definition file \\
    |cdocsamp.tex| & sample main file \\
    |cdocsch1.tex| & sample include file \\
    |cdocsch2.tex| & sample include file \\
    |cdocspt3.tex| & sample part file \\
    |cdocspt4.tex| & sample part file \\
    |cdocsdrf.tex| & sample redirection file \\
    |cdocsfn1.tex| & sample redirection file \\
    |cdocsfn2.tex| & sample redirection file \\
    |childdoc.pdf| & manual
\end{tabular}
\end{center}
%
The distribution consists of the files
|README.txt|, |childdoc.ins| and |childdoc.dtx|.
%
\begin{itemize}
\item
Run (pdf)\LaTeX{} on |childdoc.dtx|
to compile the manual |childdoc.pdf| (this file).
\item
Run \LaTeX{} on |childdoc.ins| to create the definitions file |childdoc.def|
and the sample |cdocsamp.tex| with include files
|cdocsch1.tex|, |cdocsch2.tex|, |cdocspt3.tex|, |cdocspt4.tex|,
|cdocsdrf.tex|, |cdocsfn1.tex|, |cdocsfn2.tex|.
Then copy the file |childdoc.def| to an appropriate directory of your \LaTeX{}
distribution, e.g.\ \textit{texmf-root}|/tex/latex/childdoc|.
\end{itemize}

%%%%%%%%%%%%%%%%%%%%%%%%%%%%%%%%%%%%%%%%%%%%%%%%%%%%%%%%%%%%%%%%%%%%%%%%%%%%%%%%
\subsection{Related CTAN Packages}

There are several other packages which offer a similar functionality:
%
\begin{itemize}
\item
The packages
\href{http://ctan.org/pkg/docmute}{\textsf{docmute}},
\href{http://ctan.org/pkg/includex}{\textsf{includex}} and
\href{http://ctan.org/pkg/standalone}{\textsf{standalone}}
provide commands to include only the document body of
a child file thus allowing both files to be compiled individually.
\item
The packages \href{http://ctan.org/pkg/subdocs}{\textsf{subdocs}}
and \href{http://ctan.org/pkg/subfiles}{\textsf{subfiles}}
provide structures in which the main and child documents can be
encapsulated and allowing them to be compiled individually.
The inclusion mechanism is different from the conventional |\include|.
\item
The package \href{http://ctan.org/pkg/combine}{\textsf{combine}}
is an elaborate solution to combine several documents into one.
\end{itemize}
%
See also the CTAN topic \href{http://ctan.org/topic/subdocs}{\textsf{subdocs}}
for further related packages.
The present package differs from the above solutions in that
a document structure constructed with the conventional |\include| mechanism
just needs two extra commands at the top of every file
such that all constituent files can be compiled individually.

%%%%%%%%%%%%%%%%%%%%%%%%%%%%%%%%%%%%%%%%%%%%%%%%%%%%%%%%%%%%%%%%%%%%%%%%%%%%%%%%
%\subsection{Feature Suggestions}
%
%The following is a list of features which may be useful for future
%versions of this package:
%%
%\begin{itemize}
%\item
%\ldots
%\end{itemize}

%%%%%%%%%%%%%%%%%%%%%%%%%%%%%%%%%%%%%%%%%%%%%%%%%%%%%%%%%%%%%%%%%%%%%%%%%%%%%%%%
\subsection{Revision History}

%%%%%%%%%%%%%%%%%%%%%%%%%%%%%%%%%%%%%%%%
\paragraph{v2.0:} 2018/12/30

\begin{itemize}
\item
immediate forward processing
\item
added |\childdocby| mechanism
\item
manual restructured
\end{itemize}

%%%%%%%%%%%%%%%%%%%%%%%%%%%%%%%%%%%%%%%%
\paragraph{v1.6:} 2018/01/17

\begin{itemize}
\item
application for development of include files
\item
corrections to manual
\end{itemize}

%%%%%%%%%%%%%%%%%%%%%%%%%%%%%%%%%%%%%%%%
\paragraph{v1.5:} 2017/05/21

\begin{itemize}
\item
more complete structuring introduced
\item
|\childdocof| introduced
\item
|\childdoc| renamed to |\childdocmain|
\item
|\childredirect| renamed to |\childdocforward| and |\childdocforwardprefix|
and functionality expanded
\end{itemize}

%%%%%%%%%%%%%%%%%%%%%%%%%%%%%%%%%%%%%%%%
\paragraph{v1.0:} 2017/04/27

\begin{itemize}
\item
manual and install package
\item
first version published on CTAN
\end{itemize}

%%%%%%%%%%%%%%%%%%%%%%%%%%%%%%%%%%%%%%%%
\paragraph{v0.6:} 2017/04/26

\begin{itemize}
\item
redirection mechanism added
\end{itemize}

%%%%%%%%%%%%%%%%%%%%%%%%%%%%%%%%%%%%%%%%
\paragraph{v0.5:} 2017/04/26

\begin{itemize}
\item
functionality in definition file
\end{itemize}


%%%%%%%%%%%%%%%%%%%%%%%%%%%%%%%%%%%%%%%%%%%%%%%%%%%%%%%%%%%%%%%%%%%%%%%%%%%%%%%%
%%%%%%%%%%%%%%%%%%%%%%%%%%%%%%%%%%%%%%%%%%%%%%%%%%%%%%%%%%%%%%%%%%%%%%%%%%%%%%%%
%%%%%%%%%%%%%%%%%%%%%%%%%%%%%%%%%%%%%%%%%%%%%%%%%%%%%%%%%%%%%%%%%%%%%%%%%%%%%%%%
\appendix

\settowidth\MacroIndent{\rmfamily\scriptsize 000\ }

 \DocInput{childdoc.dtx}

\end{document}
%</driver>
% \fi
%
% %%%%%%%%%%%%%%%%%%%%%%%%%%%%%%%%%%%%%%%%%%%%%%%%%%%%%%%%%%%%%%%%%%%%%%%%%%%%%%
% %%%%%%%%%%%%%%%%%%%%%%%%%%%%%%%%%%%%%%%%%%%%%%%%%%%%%%%%%%%%%%%%%%%%%%%%%%%%%%
% \section{Sample}
%\iffalse
%<*samplemain>
%\fi
%
% The following presents a sample document
% with two chapters, two parts, a title page,
% a compile flag as well as three forwarding files to set the flag.
% It consists of eight |.tex| files:
% \begin{center}
% \begin{tabular}{ll}
% |cdocsamp.tex|&main file\\
% |cdocsch1.tex|&include file for chapter 1\\
% |cdocsch2.tex|&include file for chapter 2\\
% |cdocspt3.tex|&include file for part 3\\
% |cdocspt4.tex|&include file for part 4\\
% |cdocsdrf.tex|&forwarding file for main file in draft mode\\
% |cdocsfi1.tex|&forwarding file for final version of chapter 1\\
% |cdocsfi2.tex|&forwarding file for final version of chapter 2\\
% \end{tabular}
% \end{center}
% Each of the eight files can be compiled directly by the \LaTeX{} compiler.
%
% %%%%%%%%%%%%%%%%%%%%%%%%%%%%%%%%%%%%%%
% \paragraph{Main File.}
%
% The main file is called |cdocsamp.tex|.
%
% Load the \textsf{childdoc} definitions and
% declare the filename for the main document:
%    \begin{macrocode}
\input{childdoc.def}
\childdocmain{}
%    \end{macrocode}

% Optional override for |\version| flag:
%    \begin{macrocode}
%%\ifchilddoc\else\providecommand{\version}{draft}\fi
%    \end{macrocode}

% Define the default values for the |\version| flag
% (|final| for the main file and |draft| for childs):
%    \begin{macrocode}
\ifchilddoc
\providecommand{\version}{draft}
\else
\providecommand{\version}{final}
\fi
%    \end{macrocode}

% Load the standard document class:
%    \begin{macrocode}
\documentclass[12pt]{article}
%    \end{macrocode}

% Start the document body:
%    \begin{macrocode}
\begin{document}
%    \end{macrocode}

% Declare a title page.
% Print title, part of document being processed and version flag:
%    \begin{macrocode}
\addtocounter{page}{-1}
\begin{center}
{\LARGE\bfseries{}childdoc example\par}
\vspace{1cm}
\ifchilddoc
\ifchilddocmanual part\else chapter\fi:
`\childdocname' of `\childdocjob'\par
\else
main document: `\childdocjob'\par
\fi
version: \version\par
\end{center}
\newpage
%    \end{macrocode}

% Manually include selected file,
% otherwise process as usual:
%    \begin{macrocode}
\ifchilddocmanual
\section*{part `\childdocname'}
\input{\childdocname}
\else
%    \end{macrocode}

% Include the two chapters:
%    \begin{macrocode}
\include{cdocsch1}
\include{cdocsch2}
%    \end{macrocode}

% Include the two parts unless only chapters should be displayed:
%    \begin{macrocode}
\ifchilddoc\else
\section{part three}
\input{cdocspt3}
\section{part four}
\input{cdocspt4}
\fi
%    \end{macrocode}

% Process as usual until here:
%    \begin{macrocode}
\fi
%    \end{macrocode}

% End of document body:
%    \begin{macrocode}
\end{document}
%    \end{macrocode}
%\iffalse
%</samplemain>
%\fi
%
% %%%%%%%%%%%%%%%%%%%%%%%%%%%%%%%%%%%%%%
% \paragraph{Chapter Include Files.}
%
% The include files are called |cdocsch1.tex| and |cdocsch2.tex|.
%
%\iffalse
%<*samplechap1|samplechap2>
%\fi

% Optional override for |\version| flag:
%    \begin{macrocode}
%%\providecommand{\version}{final}
%    \end{macrocode}

% Include the main document:
%    \begin{macrocode}
\input{childdoc.def}
\childdocof{cdocsamp}
%    \end{macrocode}

%\iffalse
%</samplechap1|samplechap2>
%\fi
%
%\iffalse
%<*samplechap1>
%\fi
% Some text for chapter 1:
%    \begin{macrocode}
\section{one}
some text in chapter one
%    \end{macrocode}

%\iffalse
%</samplechap1>
%\fi
% Some text for chapter 2:
%\iffalse
%<*samplechap2>
%\fi
%    \begin{macrocode}
\section{two}
more text in chapter two
%    \end{macrocode}

%\iffalse
%</samplechap2>
%\fi
%
% %%%%%%%%%%%%%%%%%%%%%%%%%%%%%%%%%%%%%%
% \paragraph{Part Include Files.}
%
% The include files are called |cdocspt3.tex| and |cdocspt4.tex|.
%
%\iffalse
%<*samplepart3|samplepart4>
%\fi

% Optional override for |\version| flag:
%    \begin{macrocode}
%%\providecommand{\version}{final}
%    \end{macrocode}

% Include the main document:
%    \begin{macrocode}
\input{childdoc.def}
\childdocby{cdocsamp}
%    \end{macrocode}

%\iffalse
%</samplepart3|samplepart4>
%\fi
%
%\iffalse
%<*samplepart3>
%\fi
% Some text for part 3:
%    \begin{macrocode}
some text in part three
%    \end{macrocode}

%\iffalse
%</samplepart3>
%\fi
% Some text for part 4:
%\iffalse
%<*samplepart4>
%\fi
%    \begin{macrocode}
more text in part four
%    \end{macrocode}

%\iffalse
%</samplepart4>
%\fi
%
% %%%%%%%%%%%%%%%%%%%%%%%%%%%%%%%%%%%%%%
% \paragraph{Forwarding for a Complete Draft.}
%
% The following forwarding file |cdocsdrf.tex|
% compiles the main document in draft mode:
%\iffalse
%<*sampledraft>
%\fi
%    \begin{macrocode}
\def\version{draft}
\input{childdoc.def}
\childdocforward{cdocsamp}
%    \end{macrocode}

%\iffalse
%</sampledraft>
%\fi
%
% %%%%%%%%%%%%%%%%%%%%%%%%%%%%%%%%%%%%%%
% \paragraph{Forwarding for Final Version of the Chapters.}
%
% The following forwarding files |cdocsfn1.tex| and |cdocsfn2.tex|
% (with identical content)
% compile the final versions of the child documents
% |cdocsch1.tex| and |cdocsch2.tex|, respectively:
%\iffalse
%<*samplefinal>
%\fi
%    \begin{macrocode}
\def\version{final}
\input{childdoc.def}
\childdocforwardprefix[cdocsamp]{cdocsfn}{cdocsch}
%    \end{macrocode}

%\iffalse
%</samplefinal>
%\fi
%
% %%%%%%%%%%%%%%%%%%%%%%%%%%%%%%%%%%%%%%
% \paragraph{Command Line Processing.}
%
% The following three command lines generate the output files
% |cdocscld|, |cdocscl1| and |cdocscl2|
% which should be identical to
% |cdocsdrf|, |cdocsch1| and |cdocsfn2|, respectively:
% \begin{center}
% \begin{tabular}{l}
% |latex -jobname cdocscld \|\\
% |  "\def\version{draft}\input{childdoc.def}\childdocforward{cdocsamp}"|\\
% |latex -jobname cdocscl1 \|\\
% |  "\input{childdoc.def}\childdocforward[cdocsamp]{cdocsch1}"|\\
% |latex -jobname cdocscl2 \|\\
% |  "\def\version{final}\input{childdoc.def}\childdocforward{cdocsch2}"|
% \end{tabular}
% \end{center}
% Note that the trailing backslash on each first line
% merely continues the input to the second line
% (for convenient cut ant paste).
% Furthermore, the command |latex| can be replaced by any
% of its alternative versions such as |pdflatex|.
%
% %%%%%%%%%%%%%%%%%%%%%%%%%%%%%%%%%%%%%%%%%%%%%%%%%%%%%%%%%%%%%%%%%%%%%%%%%%%%%%
% %%%%%%%%%%%%%%%%%%%%%%%%%%%%%%%%%%%%%%%%%%%%%%%%%%%%%%%%%%%%%%%%%%%%%%%%%%%%%%
% \section{Implementation}
%\iffalse
%<*package>
%\fi
%
% This section describes the definitions file |childdoc.def|.

% The definitions cannot be loaded using |\usepackage| or |\RequirePackage|
% which has a mechanism to prevent loading a style file more than once.
% When loading the definitions by means of |\input|
% multiple instances have to be prevented manually:
%\iffalse
%This code needs to be before the `\ProvidesFile' directive
%which is defined at the beginning of this file.
%Therefore it is also placed there and commented out here.
%</package>
%<*discard>
%\fi
%    \begin{macrocode}
\ifdefined\childdocmain\endinput\fi
%    \end{macrocode}
%\iffalse
%</discard>
%<*package>
%\fi
%
% \macro{\ifchilddoc}
% \macro{\ifchilddocmanual}
% The conditional |\ifchilddoc| tells whether a
% child (true) or main (false) document is being compiled.
% The conditional |\ifchilddocmanual| tells whether
% the |\includeonly| mechanism is used (false) or
% the selection of child files must be performed manually (true).
% The definitions initialise to false:
%    \begin{macrocode}
\newif\ifchilddoc
\newif\ifchilddocmanual
%    \end{macrocode}

% \macro{\childdocname}
% \macro{\childdocjob}
% The macro |\childdocname| stores the name of the main document
% to be compiled. The macro |\childdocjob| stores the name of
% the document on which the \LaTeX{} compiler was originally invoked.
% The content of |\jobname| cannot be compared
% to filenames specified in the source due to different catcodes.
% The following code rescans |\jobname|, stores the result
% in |\childdocname| and saves a copy in |\childdocjob|:
%    \begin{macrocode}
\edef\childdocname{\scantokens\expandafter{\jobname\noexpand}}
\let\childdocjob\childdocname
%    \end{macrocode}

% \macro{\childdocdisable}
% The macro |\childdocdisable| prevents the main file
% from being processed more than once.
% At this stage, the main document command |\childdocmain|
% is assumed to be called once again where it should do nothing.
% Any subsequent call to it should prevent
% a secondary processing of the main document
% It overwrites the forwarding commands
% |\childdocof| and |\childdocforward|
% with empty macros to prevent further inclusions of the main document:
%    \begin{macrocode}
\newcommand{\childdocdisable}
{
  \renewcommand{\childdocmain}[1]{\renewcommand{\childdocmain}[1]{\endinput}}
  \renewcommand{\childdocof}[1]{}
  \renewcommand{\childdocby}[2][]{}
  \renewcommand{\childdocforward}[2][]{}
  \renewcommand{\childdocdisable}{}
}
%    \end{macrocode}

% \macro{\childdocmain}
% The macro |\childdocmain| is to be called at the top of the main file
% with nothing or the main filename (without extension) as argument.
% First, it breaks loops.
% If the argument is not empty and does not match |\childdocname|
% (which is set by the first inclusion of |childdoc.def|),
% |\ifchilddoc| is set to true, |\includeonly| is applied to the child file
% and |\jobname| is set to the main file
% (for proper handling of |.aux| files):
%    \begin{macrocode}
\newcommand{\childdocmain}[1]
{
  \childdocdisable\childdocmain{}
  \if?#1?\else
    \begingroup
      \def\childdoctmp{#1}
      \ifx\childdoctmp\childdocname
        \def\childdoctmp{}
      \else
        \def\childdoctmp
        {
          \childdoctrue
          \includeonly{\childdocname}
          \def\childdocjob{#1}
          \def\jobname{#1}
        }
      \fi
      \expandafter
    \endgroup
    \childdoctmp
  \fi
}
%    \end{macrocode}

% \macro{\childdocof}
% The command |\childdocof| redirects
% compilation to the main file |#1|.
%    \begin{macrocode}
\newcommand{\childdocof}[1]
{
  \childdocdisable
  \childdoctrue
  \includeonly{\childdocname}
  \def\jobname{#1}
  \def\childdocjob{#1}
  \input{#1}
}
%    \end{macrocode}

% \macro{\childdocby}
% The command |\childdocby| ....
%    \begin{macrocode}
\newcommand{\childdocby}[2][]
{
  \childdocdisable
  \childdoctrue
  \childdocmanualtrue
  \if?#1?\else
    \def\jobname{#2}
  \fi
  \def\childdocjob{#2}
  \input{#2}
  \endinput
}
%    \end{macrocode}

% \macro{\childdocforward}
% The command |\childdocforward| redirects
% compilation to the main file or
% (if the optional argument is given) a child file.
% Parameters are set as if the main file
% or a child file starting with |\childdocof| was compiled.
% Then compilation is handed over to the main file:
%    \begin{macrocode}
\newcommand{\childdocforward}[2][]
{
  \begingroup
    \if?#1?
      \def\childdoctmp
      {
        \def\childdocname{#2}
        \def\childdocjob{#2}
        \def\jobname{#2}
        \input{#2}
        \endinput
      }
    \else
      \def\childdoctmp
      {
        \childdocdisable
        \def\childdocname{#2}
        \childdoctrue
        \includeonly{#2}
        \def\childdocjob{#1}
        \def\jobname{#1}
        \input{#1}
        \endinput
      }
    \fi
    \expandafter
  \endgroup
  \childdoctmp
}
%    \end{macrocode}

% \macro{\childdocforwardprefix}
% The command |\childdocforwardprefix| redirects
% compilation to the main or a child file by means of a pattern.
% The prefix |#1| in the current filename is replaced by |#2|
% and the suffix of the current filename is kept
% (it is assumed that the filename does not contain the substring `|~~~|'
% which is used as a delimiter).
% Compilation is handed over to the new file by |\childdocforward|:
%    \begin{macrocode}
\newcommand{\childdocforwardprefix}[3][]
{
  \begingroup
    \def\childdocextract #2##1~~~{\def\childdoctmp{\childdocforward[#1]{#3##1}}}
    \expandafter\childdocextract\childdocname~~~
    \expandafter
  \endgroup
  \childdoctmp
}
%    \end{macrocode}

% \macro{\childdoc}
% The deprecated macro |\childdoc| is a legacy version of |\childdocmain|:
%    \begin{macrocode}
\newcommand{\childdoc}{\childdocmain}
%    \end{macrocode}

% \macro{\childdocredirect}
% The deprecated macro |\childdocredirect| is a legacy version
% of |\childdocforward| and |\childdocforwardprefix|:
%    \begin{macrocode}
\newcommand{\childdocredirect}[2][]
{
  \begingroup
    \if?#1?
      \def\childdoctmp{\childdocforward{#2}}
    \else
      \def\childdoctmp{\childdocforwardprefix{#1}{#2}}
    \fi
    \expandafter
  \endgroup
  \childdoctmp
}
%    \end{macrocode}

%\iffalse
%</package>
%\fi
%
\endinput
\childdocforward{cdocsch2}"|
% \end{tabular}
% \end{center}
% Note that the trailing backslash on each first line
% merely continues the input to the second line
% (for convenient cut ant paste).
% Furthermore, the command |latex| can be replaced by any
% of its alternative versions such as |pdflatex|.
%
% %%%%%%%%%%%%%%%%%%%%%%%%%%%%%%%%%%%%%%%%%%%%%%%%%%%%%%%%%%%%%%%%%%%%%%%%%%%%%%
% %%%%%%%%%%%%%%%%%%%%%%%%%%%%%%%%%%%%%%%%%%%%%%%%%%%%%%%%%%%%%%%%%%%%%%%%%%%%%%
% \section{Implementation}
%\iffalse
%<*package>
%\fi
%
% This section describes the definitions file |childdoc.def|.

% The definitions cannot be loaded using |\usepackage| or |\RequirePackage|
% which has a mechanism to prevent loading a style file more than once.
% When loading the definitions by means of |\input|
% multiple instances have to be prevented manually:
%\iffalse
%This code needs to be before the `\ProvidesFile' directive
%which is defined at the beginning of this file.
%Therefore it is also placed there and commented out here.
%</package>
%<*discard>
%\fi
%    \begin{macrocode}
\ifdefined\childdocmain\endinput\fi
%    \end{macrocode}
%\iffalse
%</discard>
%<*package>
%\fi
%
% \macro{\ifchilddoc}
% \macro{\ifchilddocmanual}
% The conditional |\ifchilddoc| tells whether a
% child (true) or main (false) document is being compiled.
% The conditional |\ifchilddocmanual| tells whether
% the |\includeonly| mechanism is used (false) or
% the selection of child files must be performed manually (true).
% The definitions initialise to false:
%    \begin{macrocode}
\newif\ifchilddoc
\newif\ifchilddocmanual
%    \end{macrocode}

% \macro{\childdocname}
% \macro{\childdocjob}
% The macro |\childdocname| stores the name of the main document
% to be compiled. The macro |\childdocjob| stores the name of
% the document on which the \LaTeX{} compiler was originally invoked.
% The content of |\jobname| cannot be compared
% to filenames specified in the source due to different catcodes.
% The following code rescans |\jobname|, stores the result
% in |\childdocname| and saves a copy in |\childdocjob|:
%    \begin{macrocode}
\edef\childdocname{\scantokens\expandafter{\jobname\noexpand}}
\let\childdocjob\childdocname
%    \end{macrocode}

% \macro{\childdocdisable}
% The macro |\childdocdisable| prevents the main file
% from being processed more than once.
% At this stage, the main document command |\childdocmain|
% is assumed to be called once again where it should do nothing.
% Any subsequent call to it should prevent
% a secondary processing of the main document
% It overwrites the forwarding commands
% |\childdocof| and |\childdocforward|
% with empty macros to prevent further inclusions of the main document:
%    \begin{macrocode}
\newcommand{\childdocdisable}
{
  \renewcommand{\childdocmain}[1]{\renewcommand{\childdocmain}[1]{\endinput}}
  \renewcommand{\childdocof}[1]{}
  \renewcommand{\childdocby}[2][]{}
  \renewcommand{\childdocforward}[2][]{}
  \renewcommand{\childdocdisable}{}
}
%    \end{macrocode}

% \macro{\childdocmain}
% The macro |\childdocmain| is to be called at the top of the main file
% with nothing or the main filename (without extension) as argument.
% First, it breaks loops.
% If the argument is not empty and does not match |\childdocname|
% (which is set by the first inclusion of |childdoc.def|),
% |\ifchilddoc| is set to true, |\includeonly| is applied to the child file
% and |\jobname| is set to the main file
% (for proper handling of |.aux| files):
%    \begin{macrocode}
\newcommand{\childdocmain}[1]
{
  \childdocdisable\childdocmain{}
  \if?#1?\else
    \begingroup
      \def\childdoctmp{#1}
      \ifx\childdoctmp\childdocname
        \def\childdoctmp{}
      \else
        \def\childdoctmp
        {
          \childdoctrue
          \includeonly{\childdocname}
          \def\childdocjob{#1}
          \def\jobname{#1}
        }
      \fi
      \expandafter
    \endgroup
    \childdoctmp
  \fi
}
%    \end{macrocode}

% \macro{\childdocof}
% The command |\childdocof| redirects
% compilation to the main file |#1|.
%    \begin{macrocode}
\newcommand{\childdocof}[1]
{
  \childdocdisable
  \childdoctrue
  \includeonly{\childdocname}
  \def\jobname{#1}
  \def\childdocjob{#1}
  \input{#1}
}
%    \end{macrocode}

% \macro{\childdocby}
% The command |\childdocby| ....
%    \begin{macrocode}
\newcommand{\childdocby}[2][]
{
  \childdocdisable
  \childdoctrue
  \childdocmanualtrue
  \if?#1?\else
    \def\jobname{#2}
  \fi
  \def\childdocjob{#2}
  \input{#2}
  \endinput
}
%    \end{macrocode}

% \macro{\childdocforward}
% The command |\childdocforward| redirects
% compilation to the main file or
% (if the optional argument is given) a child file.
% Parameters are set as if the main file
% or a child file starting with |\childdocof| was compiled.
% Then compilation is handed over to the main file:
%    \begin{macrocode}
\newcommand{\childdocforward}[2][]
{
  \begingroup
    \if?#1?
      \def\childdoctmp
      {
        \def\childdocname{#2}
        \def\childdocjob{#2}
        \def\jobname{#2}
        \input{#2}
        \endinput
      }
    \else
      \def\childdoctmp
      {
        \childdocdisable
        \def\childdocname{#2}
        \childdoctrue
        \includeonly{#2}
        \def\childdocjob{#1}
        \def\jobname{#1}
        \input{#1}
        \endinput
      }
    \fi
    \expandafter
  \endgroup
  \childdoctmp
}
%    \end{macrocode}

% \macro{\childdocforwardprefix}
% The command |\childdocforwardprefix| redirects
% compilation to the main or a child file by means of a pattern.
% The prefix |#1| in the current filename is replaced by |#2|
% and the suffix of the current filename is kept
% (it is assumed that the filename does not contain the substring `|~~~|'
% which is used as a delimiter).
% Compilation is handed over to the new file by |\childdocforward|:
%    \begin{macrocode}
\newcommand{\childdocforwardprefix}[3][]
{
  \begingroup
    \def\childdocextract #2##1~~~{\def\childdoctmp{\childdocforward[#1]{#3##1}}}
    \expandafter\childdocextract\childdocname~~~
    \expandafter
  \endgroup
  \childdoctmp
}
%    \end{macrocode}

% \macro{\childdoc}
% The deprecated macro |\childdoc| is a legacy version of |\childdocmain|:
%    \begin{macrocode}
\newcommand{\childdoc}{\childdocmain}
%    \end{macrocode}

% \macro{\childdocredirect}
% The deprecated macro |\childdocredirect| is a legacy version
% of |\childdocforward| and |\childdocforwardprefix|:
%    \begin{macrocode}
\newcommand{\childdocredirect}[2][]
{
  \begingroup
    \if?#1?
      \def\childdoctmp{\childdocforward{#2}}
    \else
      \def\childdoctmp{\childdocforwardprefix{#1}{#2}}
    \fi
    \expandafter
  \endgroup
  \childdoctmp
}
%    \end{macrocode}

%\iffalse
%</package>
%\fi
%
\endinput
|\\
|\childdocby{|\textit{main}|}|\\
\end{tabular}
\end{center}
%
Both forms have slightly different effects as described above.
The main file is prepared as usual, see \secref{sec:include}.

%%%%%%%%%%%%%%%%%%%%%%%%%%%%%%%%%%%%%%%%%%%%%%%%%%%%%%%%%%%%%%%%%%%%%%%%%%%%%%%%
\subsection{Legacy Detection}
\label{sec:detection}

The directive |\childdocmain| in the main file can detect
whether the complete document or merely a child is to be compiled
even without using the directive |\childdocof|.
This method is deprecated because it is less robust
and there is no compelling reason to use it;
it is merely provided for backward compatibility
and it may be removed in future versions.

If the detection mechanism is to be used,
it is mandatory to correctly specify
the filename of the main file as the argument of |\childdocmain|:
%
\begin{center}
\begin{tabular}{l}
|% \iffalse
%
% childdoc.dtx Copyright (C) 2017-2018 Niklas Beisert
%
% This work may be distributed and/or modified under the
% conditions of the LaTeX Project Public License, either version 1.3
% of this license or (at your option) any later version.
% The latest version of this license is in
%   http://www.latex-project.org/lppl.txt
% and version 1.3 or later is part of all distributions of LaTeX
% version 2005/12/01 or later.
%
% This work has the LPPL maintenance status `maintained'.
%
% The Current Maintainer of this work is Niklas Beisert.
%
% This work consists of the files childdoc.dtx and childdoc.ins
% and the derived files childdoc.def and cdocsamp.tex with
% cdocsch1.tex, cdocsch2.tex, cdocsdrf.tex, cdocsfn1.tex, cdocsfn2.tex.
%
%<package>\ifdefined\childdocmain\endinput\fi
%<package>\ProvidesFile{childdoc.def}[2018/12/30 v2.0 child document driver]
%<samplemain>\ProvidesFile{cdocsamp.tex}[2018/12/30 v2.0 sample for childdoc]
%<*driver>
%\ProvidesFile{childdoc.drv}[2018/12/30 v2.0 childdoc reference manual file]
\PassOptionsToClass{10pt,a4paper}{article}
\documentclass{ltxdoc}

\usepackage[margin=35mm]{geometry}
\usepackage{hyperref}
\usepackage{hyperxmp}
\usepackage[usenames]{color}

\hypersetup{colorlinks=true}
\hypersetup{pdfstartview=FitH}
\hypersetup{pdfpagemode=UseNone}
\hypersetup{pdfsource={}}
\hypersetup{pdflang={en-UK}}
\hypersetup{pdfcopyright={Copyright 2017-2018 Niklas Beisert.
  This work may be distributed and/or modified under the
  conditions of the LaTeX Project Public License, either version 1.3
  of this license or (at your option) any later version.}}
\hypersetup{pdflicenseurl={http://www.latex-project.org/lppl.txt}}
\hypersetup{pdfcontactaddress={ETH Zurich, ITP, HIT K,
  Wolfgang-Pauli-Strasse 27}}
\hypersetup{pdfcontactpostcode={8093}}
\hypersetup{pdfcontactcity={Zurich}}
\hypersetup{pdfcontactcountry={Switzerland}}
\hypersetup{pdfcontactemail={nbeisert@itp.phys.ethz.ch}}
\hypersetup{pdfcontacturl={http://people.phys.ethz.ch/\xmptilde nbeisert/}}

\newcommand{\secref}[1]{\hyperref[#1]{section \ref*{#1}}}

\parskip1ex
\parindent0pt
\let\olditemize\itemize
\def\itemize{\olditemize\parskip0pt}

\begin{document}

\title{The \textsf{childdoc} Package}
\hypersetup{pdftitle={The childdoc Package}}
\author{Niklas Beisert\\[2ex]
  Institut f\"ur Theoretische Physik\\
  Eidgen\"ossische Technische Hochschule Z\"urich\\
  Wolfgang-Pauli-Strasse 27, 8093 Z\"urich, Switzerland\\[1ex]
  \href{mailto:nbeisert@itp.phys.ethz.ch}
  {\texttt{nbeisert@itp.phys.ethz.ch}}}
\hypersetup{pdfauthor={Niklas Beisert}}
\hypersetup{pdfsubject={Manual for the LaTeX2e Package childdoc}}
\date{30 December 2018, \textsf{v2.0}}
\maketitle

\begin{abstract}\noindent
\textsf{childdoc} is a \LaTeXe{} package
that enables the direct compilation
of document sections included by |\include|
to individual files.
\end{abstract}

\begingroup
\parskip0ex
\tableofcontents
\endgroup

%%%%%%%%%%%%%%%%%%%%%%%%%%%%%%%%%%%%%%%%%%%%%%%%%%%%%%%%%%%%%%%%%%%%%%%%%%%%%%%%
%%%%%%%%%%%%%%%%%%%%%%%%%%%%%%%%%%%%%%%%%%%%%%%%%%%%%%%%%%%%%%%%%%%%%%%%%%%%%%%%
\section{Introduction}

\LaTeX{} provides a mechanism to structure a large document (such as a book)
into a main file and several child files (containing the chapters)
using the |\include| command.
This mechanism is beneficial for documents
which span hundreds of pages in order to
make the source file(s) more manageable.
Moreover, compilation can be restricted to
selected child files by means of the |\includeonly| command.
The latter feature can be used to reduce the compilation time while editing
(this was significantly more useful in the earlier days of \LaTeX{})
or to generate a smaller document which is easier to navigate.
Another application of |\includeonly| is to generate
documents consisting of selected parts of the complete document.

However, there are a few drawbacks of the plain |\include| mechanism:
\begin{itemize}
\item
The child files cannot be compiled on their own,
they can only be compiled via the main file.
A naive editing environment
(such as a text editor with an option
to have the current file processed by \LaTeX)
may require one to switch to the main file before compiling;
attempting to compile the child file produces errors.
\item
The main file must be modified (each time)
to adjust the |\includeonly| command
to the present needs. This easily leaves the main file in a messy state.
\item
The generated document will always carry the filename
of the main document. This is inconvenient if
several child files are to be compiled and
to be kept for distribution.
\end{itemize}

The present package provides a simple interface
to make child files individually compilable by \LaTeX{}.
Compiling a child file then has the same effect as compiling
the main file with an |\includeonly| command
to select the appropriate child.
Moreover the generated document will carry the name of the child
rather than the main file.
This resolves all three above issues.

This feature is meant to make the editing of books,
thesis documents and lecture notes somewhat more convenient.
However, the package can also be used efficiently for
composing a series of documents (such as exercise sheets)
which are typically distributed individually.
It then assists the author in generating the individual documents
(potentially in different versions)
as well as a document containing the collected series.
Another application is in developing style files
or other kinds of included material
where compilation of the style file could redirect
to a sample or test file.

%%%%%%%%%%%%%%%%%%%%%%%%%%%%%%%%%%%%%%%%%%%%%%%%%%%%%%%%%%%%%%%%%%%%%%%%%%%%%%%%
%%%%%%%%%%%%%%%%%%%%%%%%%%%%%%%%%%%%%%%%%%%%%%%%%%%%%%%%%%%%%%%%%%%%%%%%%%%%%%%%
\section{Usage}

First of all, the package \textsf{childdoc} is \emph{not} a standard
\LaTeXe{} |.sty| style file! Therefore it needs to be invoked in
a non-standard way.

%%%%%%%%%%%%%%%%%%%%%%%%%%%%%%%%%%%%%%%%%%%%%%%%%%%%%%%%%%%%%%%%%%%%%%%%%%%%%%%%
\subsection{Included Files}
\label{sec:include}

%%%%%%%%%%%%%%%%%%%%%%%%%%%%%%%%%%%%%%%%
\DescribeMacro{\childdocmain}
To use the package, add the commands
\begin{center}
\begin{tabular}{l}
|% \iffalse
%
% childdoc.dtx Copyright (C) 2017-2018 Niklas Beisert
%
% This work may be distributed and/or modified under the
% conditions of the LaTeX Project Public License, either version 1.3
% of this license or (at your option) any later version.
% The latest version of this license is in
%   http://www.latex-project.org/lppl.txt
% and version 1.3 or later is part of all distributions of LaTeX
% version 2005/12/01 or later.
%
% This work has the LPPL maintenance status `maintained'.
%
% The Current Maintainer of this work is Niklas Beisert.
%
% This work consists of the files childdoc.dtx and childdoc.ins
% and the derived files childdoc.def and cdocsamp.tex with
% cdocsch1.tex, cdocsch2.tex, cdocsdrf.tex, cdocsfn1.tex, cdocsfn2.tex.
%
%<package>\ifdefined\childdocmain\endinput\fi
%<package>\ProvidesFile{childdoc.def}[2018/12/30 v2.0 child document driver]
%<samplemain>\ProvidesFile{cdocsamp.tex}[2018/12/30 v2.0 sample for childdoc]
%<*driver>
%\ProvidesFile{childdoc.drv}[2018/12/30 v2.0 childdoc reference manual file]
\PassOptionsToClass{10pt,a4paper}{article}
\documentclass{ltxdoc}

\usepackage[margin=35mm]{geometry}
\usepackage{hyperref}
\usepackage{hyperxmp}
\usepackage[usenames]{color}

\hypersetup{colorlinks=true}
\hypersetup{pdfstartview=FitH}
\hypersetup{pdfpagemode=UseNone}
\hypersetup{pdfsource={}}
\hypersetup{pdflang={en-UK}}
\hypersetup{pdfcopyright={Copyright 2017-2018 Niklas Beisert.
  This work may be distributed and/or modified under the
  conditions of the LaTeX Project Public License, either version 1.3
  of this license or (at your option) any later version.}}
\hypersetup{pdflicenseurl={http://www.latex-project.org/lppl.txt}}
\hypersetup{pdfcontactaddress={ETH Zurich, ITP, HIT K,
  Wolfgang-Pauli-Strasse 27}}
\hypersetup{pdfcontactpostcode={8093}}
\hypersetup{pdfcontactcity={Zurich}}
\hypersetup{pdfcontactcountry={Switzerland}}
\hypersetup{pdfcontactemail={nbeisert@itp.phys.ethz.ch}}
\hypersetup{pdfcontacturl={http://people.phys.ethz.ch/\xmptilde nbeisert/}}

\newcommand{\secref}[1]{\hyperref[#1]{section \ref*{#1}}}

\parskip1ex
\parindent0pt
\let\olditemize\itemize
\def\itemize{\olditemize\parskip0pt}

\begin{document}

\title{The \textsf{childdoc} Package}
\hypersetup{pdftitle={The childdoc Package}}
\author{Niklas Beisert\\[2ex]
  Institut f\"ur Theoretische Physik\\
  Eidgen\"ossische Technische Hochschule Z\"urich\\
  Wolfgang-Pauli-Strasse 27, 8093 Z\"urich, Switzerland\\[1ex]
  \href{mailto:nbeisert@itp.phys.ethz.ch}
  {\texttt{nbeisert@itp.phys.ethz.ch}}}
\hypersetup{pdfauthor={Niklas Beisert}}
\hypersetup{pdfsubject={Manual for the LaTeX2e Package childdoc}}
\date{30 December 2018, \textsf{v2.0}}
\maketitle

\begin{abstract}\noindent
\textsf{childdoc} is a \LaTeXe{} package
that enables the direct compilation
of document sections included by |\include|
to individual files.
\end{abstract}

\begingroup
\parskip0ex
\tableofcontents
\endgroup

%%%%%%%%%%%%%%%%%%%%%%%%%%%%%%%%%%%%%%%%%%%%%%%%%%%%%%%%%%%%%%%%%%%%%%%%%%%%%%%%
%%%%%%%%%%%%%%%%%%%%%%%%%%%%%%%%%%%%%%%%%%%%%%%%%%%%%%%%%%%%%%%%%%%%%%%%%%%%%%%%
\section{Introduction}

\LaTeX{} provides a mechanism to structure a large document (such as a book)
into a main file and several child files (containing the chapters)
using the |\include| command.
This mechanism is beneficial for documents
which span hundreds of pages in order to
make the source file(s) more manageable.
Moreover, compilation can be restricted to
selected child files by means of the |\includeonly| command.
The latter feature can be used to reduce the compilation time while editing
(this was significantly more useful in the earlier days of \LaTeX{})
or to generate a smaller document which is easier to navigate.
Another application of |\includeonly| is to generate
documents consisting of selected parts of the complete document.

However, there are a few drawbacks of the plain |\include| mechanism:
\begin{itemize}
\item
The child files cannot be compiled on their own,
they can only be compiled via the main file.
A naive editing environment
(such as a text editor with an option
to have the current file processed by \LaTeX)
may require one to switch to the main file before compiling;
attempting to compile the child file produces errors.
\item
The main file must be modified (each time)
to adjust the |\includeonly| command
to the present needs. This easily leaves the main file in a messy state.
\item
The generated document will always carry the filename
of the main document. This is inconvenient if
several child files are to be compiled and
to be kept for distribution.
\end{itemize}

The present package provides a simple interface
to make child files individually compilable by \LaTeX{}.
Compiling a child file then has the same effect as compiling
the main file with an |\includeonly| command
to select the appropriate child.
Moreover the generated document will carry the name of the child
rather than the main file.
This resolves all three above issues.

This feature is meant to make the editing of books,
thesis documents and lecture notes somewhat more convenient.
However, the package can also be used efficiently for
composing a series of documents (such as exercise sheets)
which are typically distributed individually.
It then assists the author in generating the individual documents
(potentially in different versions)
as well as a document containing the collected series.
Another application is in developing style files
or other kinds of included material
where compilation of the style file could redirect
to a sample or test file.

%%%%%%%%%%%%%%%%%%%%%%%%%%%%%%%%%%%%%%%%%%%%%%%%%%%%%%%%%%%%%%%%%%%%%%%%%%%%%%%%
%%%%%%%%%%%%%%%%%%%%%%%%%%%%%%%%%%%%%%%%%%%%%%%%%%%%%%%%%%%%%%%%%%%%%%%%%%%%%%%%
\section{Usage}

First of all, the package \textsf{childdoc} is \emph{not} a standard
\LaTeXe{} |.sty| style file! Therefore it needs to be invoked in
a non-standard way.

%%%%%%%%%%%%%%%%%%%%%%%%%%%%%%%%%%%%%%%%%%%%%%%%%%%%%%%%%%%%%%%%%%%%%%%%%%%%%%%%
\subsection{Included Files}
\label{sec:include}

%%%%%%%%%%%%%%%%%%%%%%%%%%%%%%%%%%%%%%%%
\DescribeMacro{\childdocmain}
To use the package, add the commands
\begin{center}
\begin{tabular}{l}
|\input{childdoc.def}|\\
|\childdocmain{}|\\
\end{tabular}
\end{center}
at the very top of the main \LaTeX{} file,
in particular \emph{before} the |\documentclass| statement!
The argument of |\childdocmain| should be left empty
(but it must be present).

%%%%%%%%%%%%%%%%%%%%%%%%%%%%%%%%%%%%%%%%
\DescribeMacro{\childdocof}
Furthermore, add the commands
\begin{center}
\begin{tabular}{l}
|\input{childdoc.def}|\\
|\childdocof{|\textit{main}|}|\\
\end{tabular}
\end{center}
at the top of every child file \textit{child}
which is included by |\include{|\textit{child}|}|
from within the main file
(or at least for those files to be compiled individually).
The argument \textit{main} must be the filename of the main file.

There are a couple of
considerations in setting up the main and child documents:

%%%%%%%%%%%%%%%%%%%%%%%%%%%%%%%%%%%%%%%%
\paragraph{Restrictions.}

Please note the following restrictions:
\begin{itemize}
\item
|\childdocmain| must be called with one argument \textit{main}
to ensure compatibility with earlier version of the package.
It must either be empty (|\childdocmain{}|)
or precisely match the filename of the main file in which it is specified.
See \secref{sec:detection} for further information.
\item
The filename \textit{main} must be specified without the |.tex| extension.
\item
The filename \textit{main} is case sensitive
(even in case-insensitive file systems)
due to internal string comparison.
\item
The argument \textit{main} should be fully expanded, it cannot be a macro.
\item
Subdirectories and special characters should be avoided in filenames.
\item
The command |\childdocmain{|\textit{main}|}| must be followed by a whitespace.
It should not be followed immediately by another command
or by a comment mark `|%|'.
This is because the \TeX{} parser reads the token immediately following
the argument of |\childdocmain| and puts it
at the beginning of every child section;
however, a white\-space is ignored.
\end{itemize}

%%%%%%%%%%%%%%%%%%%%%%%%%%%%%%%%%%%%%%%%
\paragraph{Content of Main File.}

It is advisable to place all content in the child files included by |\include|.
Any output contained in the main file will appear in all child documents
unless suppressed manually;
it cannot be suppressed automatically by the |\includeonly| directive
and thus should normally be avoided.
A method to include some content in the main file
by means of conditional processing is described in \secref{sec:conditional}.

%%%%%%%%%%%%%%%%%%%%%%%%%%%%%%%%%%%%%%%%
\paragraph{Page Numbering.}

When only a part of the document is compiled,
the appropriate numbering of pages
(as well as other status parameters)
is determined from the |.aux| files.
The latter contain information from previous passes.
However this information needs to propagate through
all intermediate child documents.
Therefore the page numbering in child documents may well
be inconsistent until the complete document is compiled at least once.

A useful (if unconventional) way to always ensure a consistent
page numbering is to restart the numbering in each child document
and denote the pages by `\textit{child}|.|\textit{page}'
where \textit{child} represents the chapter/section number of the child file.
This can be achieved by the command
|\numberwithin{page}{|\textit{child}|}|
of the \textsf{amsmath} package
where \textit{child} can be |chapter| or |section|
depending on the chosen structuring.
Alternatively, one can modify the macro |\thepage| appropriately
and reset the counter |page| at the start of each child file.

%%%%%%%%%%%%%%%%%%%%%%%%%%%%%%%%%%%%%%%%%%%%%%%%%%%%%%%%%%%%%%%%%%%%%%%%%%%%%%%%
\subsection{Conditional Processing}
\label{sec:conditional}

The package provides a mechanism to compile different versions
of a document. To customise the versions further some conditional processing
can come in handy to distinguish which version is being compiled.
The package provides two macros to describe the compilation context:

%%%%%%%%%%%%%%%%%%%%%%%%%%%%%%%%%%%%%%%%
\DescribeMacro{\ifchilddoc}
The conditional |\ifchilddoc| distinguishes between the compilation of
child documents and the main document:
%
\begin{center}
|\ifchilddoc |\textit{child-code}| |[|\||else |\textit{main-code}]| \||fi|
\end{center}

%%%%%%%%%%%%%%%%%%%%%%%%%%%%%%%%%%%%%%%%
\DescribeMacro{\childdocname}
\DescribeMacro{\childdocjob}
The macro |\childdocname| contains the filename (without extension)
of the main or child file being processed.
Note that |\childdocjob| will always contain the name of the main file.

%%%%%%%%%%%%%%%%%%%%%%%%%%%%%%%%%%%%%%%%
\paragraph{Title Page.}

Conditional processing can be used to include a title or banner page
in the main document when proper precautions are taken.
Importantly, the code in the main file should ensure that the page counter
(as well as other status parameters which are stored in the |.aux| files)
takes the same value after the conditional processing.
Otherwise the page numbers may take divergent values
depending on which part is compiled.

For example, a title page could be declared by:
%
\begin{center}
\begin{tabular}{l}
|\ifchilddoc\||else|\\
|\addtocounter{page}{-1}|\\
\textit{code for title page}\\
|\newpage|\\
|\||fi|
\end{tabular}
\end{center}
%
A banner page for the child documents can be generated by:
%
\begin{center}
\begin{tabular}{l}
|\ifchilddoc|\\
|\addtocounter{page}{-1}|\\
\textit{code for banner page}\\
|\newpage|\\
|\||fi|
\end{tabular}
\end{center}
%
Here one could write a message such as:
\begin{center}
|This is the part \childdocname{} of \childdocjob{}.|
\end{center}

%%%%%%%%%%%%%%%%%%%%%%%%%%%%%%%%%%%%%%%%%%%%%%%%%%%%%%%%%%%%%%%%%%%%%%%%%%%%%%%%
\subsection{Flags}
\label{sec:flags}

The package makes it easy to generate different versions
of the main or child documents.
To this end compilation flags can be defined
and assigned different default values.
They will be particularly useful in conjunction
with the forwarding mechanism described in \secref{sec:forward}.

For example, it may be useful to have a flag |\version|
which can be set to |draft| or |final|.
The document source will contain some conditional code
depending on the value of |\version|.
Suppose further, the flag should default to |final| for the main file
and to |draft| for child files
which is a natural assignment for editing the document.
This is achieved by placing the following code
in the preamble of the main document
(below the |\childdocmain| directive):
%
\begin{center}
\begin{tabular}{l}
|\ifchilddoc|\\
|\providecommand{\version}{draft}|\\
|\||else|\\
|\providecommand{\version}{final}|\\
|\||fi|
\end{tabular}
\end{center}
%
The definition by |\providecommand| makes sure
that previous definitions are not overwritten.
Further statements |\providecommand{\version}{...}|
can thus be added before the above code to override it.

For the main file, one might add a line
(between |\childdocmain| and the above block)
%
\begin{center}
|%\ifchilddoc\||else\providecommand{\version}{draft}\||fi|
\end{center}
%
which can be uncommented to produce a draft version.
Likewise one can add a line to the very top of a child file
(above the |\childdocof{|\textit{main}|}| directive)
%
\begin{center}
|%\providecommand{\version}{final}|
\end{center}
%
which can be uncommented to produce the final version of this child document.

%%%%%%%%%%%%%%%%%%%%%%%%%%%%%%%%%%%%%%%%%%%%%%%%%%%%%%%%%%%%%%%%%%%%%%%%%%%%%%%%
\subsection{Forwarding}
\label{sec:forward}

Different versions of the main or child documents
using compilation flags as described in \secref{sec:flags}
can be (permanently) stored in different files
for convenient compilation, viewing and distribution.
To this end, the package defines a command
to pass on compilation to a different file:

%%%%%%%%%%%%%%%%%%%%%%%%%%%%%%%%%%%%%%%%
\DescribeMacro{\childdocforward}
The command |\childdocforward| redirects processing to
another source file:
%
\begin{center}
\begin{tabular}{l}
|\input{childdoc.def}|\\
|\childdocforward[|\textit{main}|]{|\textit{dest}|}|\\
\end{tabular}
\end{center}
%
The argument \textit{dest} is the destination file
(without extension).
It should be the main file or one of the child files.
Note that further \textsf{childdoc} directives
such as |\childdocof| and |\childdocforward|
in the indicated file will be processed in this form.
The optional argument \textit{main}
passes on directly to the main file \textit{main}
while pretending to compile the child \textit{dest}.
This form behaves as if \textit{dest}
issues |\childdocof{|\textit{main}|}| right away,
and no further \textsf{childdoc} directives will be processed.

%%%%%%%%%%%%%%%%%%%%%%%%%%%%%%%%%%%%%%%%
\DescribeMacro{\...prefix}
In the alternative form |\childdocforwardprefix|,
%
\begin{center}
\begin{tabular}{l}
|\input{childdoc.def}|\\
|\childdocforwardprefix[|\textit{main}|]{|\textit{prefix}|}{|\textit{dest}|}|
\end{tabular}
\end{center}
%
the destination file is determined by a pattern
depending on the current file:
To make this work, the current file must be called
`{\textit{prefix}\hspace{0.2em}\textit{suffix}}'
with \textit{prefix} matching precisely the argument.
Processing is then passed on to the file
`{\textit{dest}\hspace{0.2em}\textit{suffix}}'.
Surely, the same effect is achieved by
directly specifying the
argument `{\textit{dest}\hspace{0.2em}\textit{suffix}}'
in the first form.
However, that requires to set up a different file
for each child. With the alternative form of the command
all these files can have exactly the same content
which simplifies setting them up and maintaining them.

For example, the following file |draft.tex|
with a compilation flag |\version| as described in \secref{sec:flags}
compiles the main document as a draft:
%
\begin{center}
\begin{tabular}{l}
|\def\version{draft}|\\
|\input{childdoc.def}|\\
|\childdocforward{|\textit{main}|}|
\end{tabular}
\end{center}
%
Likewise, the following files |final|\textit{nn}|.tex|
compile the final version of the child document
|child|\textit{nn}|.tex|:
%
\begin{center}
\begin{tabular}{l}
|\def\version{final}|\\
|\input{childdoc.def}|\\
|\childdocforwardprefix{final}{child}|
\end{tabular}
\end{center}
%

Note that when several versions of a main file and/or of each child file
are to be generated, it may be convenient to set up a |Makefile| or
shell script to automatise the process.

%%%%%%%%%%%%%%%%%%%%%%%%%%%%%%%%%%%%%%%%%%%%%%%%%%%%%%%%%%%%%%%%%%%%%%%%%%%%%%%%
\subsection{Command Line Processing}
\label{sec:commandline}

The effect of redirection files can also be achieved by invoking
the \LaTeX{} compiler with a more elaborate command line.
Most conveniently this should be done as part
of a shell script or a |Makefile|.

When using \textsf{childdoc} in the main file, the following
command lines effectively perform a redirection
(note that depending on the shell being used,
backslashes may have to be doubled: `|\|' $\to$ `|\\|'):
%
\begin{center}
|... -jobname "|\textit{target}|" |\\|"|[\textit{flags}]%
|\input{childdoc.def}\childdocforward[|\textit{main}|]{|\textit{dest}|}"|
\end{center}
%
Here \textit{target} is the name of the output file,
\textit{main} is the name of the main file
and \textit{dest} is the name of the main or child file to be processed
(all filenames without extensions).
The optional argument \textit{main} can be omitted
if \textit{main} matches \textit{dest}.
Optionally, compilation \textit{flags} can be defined via |\def| commands.
This command line makes the \TeX{} engine believe
it is compiling the file \textit{target}
whose content is specified as the latter parameter.
The provided code then forwards the processing to
\textit{main} or \textit{dest} as described in \secref{sec:forward}.

%%%%%%%%%%%%%%%%%%%%%%%%%%%%%%%%%%%%%%%%%%%%%%%%%%%%%%%%%%%%%%%%%%%%%%%%%%%%%%%%
\subsection{Include by Input}
\label{sec:input}

Including child documents by |\include| has some restrictions by design.
Most notably, the content of a child document always occupies
its own set of pages; pages cannot be shared between child documents.
Usually, this behaviour makes perfect sense
because each child document contain an essential part of the document.
However, in some situations it may be desirable to compose
a document from a collection of parts
without having mandatory page breaks between then.
For this case, the package
provides a mechanism to include parts
by |\input| which can also be processed individually.
However, by construction this mechanism
requires manual handling of the content to be output.

%%%%%%%%%%%%%%%%%%%%%%%%%%%%%%%%%%%%%%%%
\DescribeMacro{\ifchilddocmanual}
The main file should be prepared as usual, see \secref{sec:include}.
However, the document body must make a distinction
between processing of an individual part and of the main document, e.g.:
%
\begin{center}
\begin{tabular}{l}
|\ifchilddocmanual|\\
|\input{\childdocname}|\\
|\||else|\\
\textit{document body with }|\input{|\textit{part}|}|\\
|\||fi|
\end{tabular}
\end{center}
%
The conditional |\ifchilddocmanual| is true whenever
a part to be included by |\input| is being compiled,
and the name of the part is stored in |\childdocname|.

%%%%%%%%%%%%%%%%%%%%%%%%%%%%%%%%%%%%%%%%
\DescribeMacro{\childdocby}
Each part to be included by |\input| should start with:
%
\begin{center}
\begin{tabular}{l}
|\input{childdoc.def}|\\
|\childdocby{|\textit{main}|}|\\
\end{tabular}
\end{center}
%
The directive |\childdocby| is similar to |\childdocof|
described in \secref{sec:include},
but the subsequent selection of content must be done manually.
To that end, both |\ifchilddoc| and |\ifchilddocmanual|
will be true upon processing of a part,
and the name of the part is stored in |\childdocname|.
Note that |\jobname| will be set to the filename of the current part
so that each part receives an individual |.aux| file
that does not interfere with the |.aux| file(s) of the main document.
This behaviour can be altered by the alternative form
|\childdocby[*]{|\textit{main}|}| (with a non-empty optional argument)
which uses the |.aux| file of the main document
by setting |\jobname| to \textit{main}.

%%%%%%%%%%%%%%%%%%%%%%%%%%%%%%%%%%%%%%%%%%%%%%%%%%%%%%%%%%%%%%%%%%%%%%%%%%%%%%%%
\subsection{Driver Development}
\label{sec:driver}

The \textsf{childdoc} mechanism can also be use for the development
of definition files such as \LaTeX{} styles or classes.
This case differs from the above setup with multiple parts
included by |\include| in that no |\includeonly| should be invoked.
This can be achieved by starting the include file
(before |\ProvidesPackage|) with:
%
\begin{center}
\begin{tabular}{l}
|\input{childdoc.def}|\\
|\childdocforward{|\textit{main}|}|\\
\end{tabular}
\end{center}
%
or alternatively with:
%
\begin{center}
\begin{tabular}{l}
|\input{childdoc.def}|\\
|\childdocby{|\textit{main}|}|\\
\end{tabular}
\end{center}
%
Both forms have slightly different effects as described above.
The main file is prepared as usual, see \secref{sec:include}.

%%%%%%%%%%%%%%%%%%%%%%%%%%%%%%%%%%%%%%%%%%%%%%%%%%%%%%%%%%%%%%%%%%%%%%%%%%%%%%%%
\subsection{Legacy Detection}
\label{sec:detection}

The directive |\childdocmain| in the main file can detect
whether the complete document or merely a child is to be compiled
even without using the directive |\childdocof|.
This method is deprecated because it is less robust
and there is no compelling reason to use it;
it is merely provided for backward compatibility
and it may be removed in future versions.

If the detection mechanism is to be used,
it is mandatory to correctly specify
the filename of the main file as the argument of |\childdocmain|:
%
\begin{center}
\begin{tabular}{l}
|\input{childdoc.def}|\\
|\childdocmain{|\textit{main}|}|\\
\end{tabular}
\end{center}
%
If |\jobname| does not match the argument \textit{main} of |\childdocmain|,
it is assumed that |\jobname| points to the child file to be compiled.
When using |\childdocmain| with the main file specified as argument,
it suffices to start a child file
with just |\input{|\textit{main}|}|
without loading of the package and using |\childdocof|.
If instead all processing is done
with the appropriate \textsf{childdoc} directives,
the argument of \textit{main} of |\childdocmain| can be empty.

An alternative version of the command line processing described
in \secref{sec:commandline} using the detection mechanism reads:
%
\begin{center}
|... -jobname "|\textit{target}|" "|[\textit{flags}]%
[|\def\jobname{|\textit{dest}|}|]|\input{|\textit{main}|}"|
\end{center}

%%%%%%%%%%%%%%%%%%%%%%%%%%%%%%%%%%%%%%%%%%%%%%%%%%%%%%%%%%%%%%%%%%%%%%%%%%%%%%%%
\subsection{Manual Code}
\label{sec:manual}

In case one cannot be certain whether the definitions file |childdoc.def|
is installed on the target \TeX{} distribution
and one prefers not to ship it,
it is conceivable to paste a few relevant commands into the sources.

To that end, drop all statements |\input{childdoc.def}|
and perform the replacements as outlined below.
Instead of |\childdocmain{|\textit{main}|}| add the following code
to the top of the main file:
%
\begin{center}
\begin{tabular}{l}
|\||ifdefined\childdocname\endinput\||fi\newif\ifchilddoc|\\
|\edef\childdocname{\scantokens\expandafter{\jobname\noexpand}}|\\
|\def\childdocmain{|\textit{main}|}\||ifx\childdocmain\childdocname\||else|\\
|\childdoctrue\includeonly{\childdocname}\let\jobname\childdocmain\||fi|\\
\end{tabular}
\end{center}
%
Instead of |\childdocof{|\textit{main}|}| just include the main file
at the top of each child file:
%
\begin{center}
|\input{|\textit{main}|}|
\end{center}
%
A simple redirection |\childdocforward{|\textit{dest}|}| is achieved by:
%
\begin{center}
|\def\jobname{|\textit{dest}|}\input{\jobname}|
\end{center}
%
The redirection with prefix
|\childdocforwardprefix[|\textit{prefix}|]{|\textit{dest}|}|
is accomplished by:
%
\begin{center}
\begin{tabular}{l}
|{\edef\jobname{\scantokens\expandafter{\jobname\noexpand}}|\\
|\def\redirectjob |\textit{prefix}|#1~~~{\gdef\jobname{|\textit{dest}|#1}}|\\
|\expandafter\redirectjob\jobname~~~}\input{\jobname}|
\end{tabular}
\end{center}

In an alternative approach,
child documents can be compiled by a specific command line
without additional code or specific definitions:
%
\begin{center}
|... -jobname "|\textit{target}|" "|[\textit{flags}]%
|\includeonly{|\textit{dest}|}\input{|\textit{main}|}"|
\end{center}
%

%%%%%%%%%%%%%%%%%%%%%%%%%%%%%%%%%%%%%%%%%%%%%%%%%%%%%%%%%%%%%%%%%%%%%%%%%%%%%%%%
%%%%%%%%%%%%%%%%%%%%%%%%%%%%%%%%%%%%%%%%%%%%%%%%%%%%%%%%%%%%%%%%%%%%%%%%%%%%%%%%
\section{Information}

%%%%%%%%%%%%%%%%%%%%%%%%%%%%%%%%%%%%%%%%%%%%%%%%%%%%%%%%%%%%%%%%%%%%%%%%%%%%%%%%
\subsection{Copyright}

Copyright \copyright{} 2017--2018 Niklas Beisert

This work may be distributed and/or modified under the
conditions of the \LaTeX{} Project Public License, either version 1.3
of this license or (at your option) any later version.
The latest version of this license is in
  \url{http://www.latex-project.org/lppl.txt}
and version 1.3 or later is part of all distributions of \LaTeX{}
version 2005/12/01 or later.

This work has the LPPL maintenance status `maintained'.

The Current Maintainer of this work is Niklas Beisert.

This work consists of the files |README.txt|, |childdoc.ins| and |childdoc.dtx|
as well as the derived files |childdoc.def|, |cdocsamp.tex|
with |cdocsch1.tex|, |cdocsch2.tex|, |cdocspt3.tex|, |cdocspt4.tex|,
|cdocsdrf.tex|, |cdocsfn1.tex|, |cdocsfn2.tex|
as well as |childdoc.pdf|.

%%%%%%%%%%%%%%%%%%%%%%%%%%%%%%%%%%%%%%%%%%%%%%%%%%%%%%%%%%%%%%%%%%%%%%%%%%%%%%%%
\subsection{Files and Installation}

The package consists of the files:
%
\begin{center}
\begin{tabular}{ll}
    |README.txt|   & readme file \\
    |childdoc.ins| & installation file \\
    |childdoc.dtx| & source file \\
    |childdoc.def| & definition file \\
    |cdocsamp.tex| & sample main file \\
    |cdocsch1.tex| & sample include file \\
    |cdocsch2.tex| & sample include file \\
    |cdocspt3.tex| & sample part file \\
    |cdocspt4.tex| & sample part file \\
    |cdocsdrf.tex| & sample redirection file \\
    |cdocsfn1.tex| & sample redirection file \\
    |cdocsfn2.tex| & sample redirection file \\
    |childdoc.pdf| & manual
\end{tabular}
\end{center}
%
The distribution consists of the files
|README.txt|, |childdoc.ins| and |childdoc.dtx|.
%
\begin{itemize}
\item
Run (pdf)\LaTeX{} on |childdoc.dtx|
to compile the manual |childdoc.pdf| (this file).
\item
Run \LaTeX{} on |childdoc.ins| to create the definitions file |childdoc.def|
and the sample |cdocsamp.tex| with include files
|cdocsch1.tex|, |cdocsch2.tex|, |cdocspt3.tex|, |cdocspt4.tex|,
|cdocsdrf.tex|, |cdocsfn1.tex|, |cdocsfn2.tex|.
Then copy the file |childdoc.def| to an appropriate directory of your \LaTeX{}
distribution, e.g.\ \textit{texmf-root}|/tex/latex/childdoc|.
\end{itemize}

%%%%%%%%%%%%%%%%%%%%%%%%%%%%%%%%%%%%%%%%%%%%%%%%%%%%%%%%%%%%%%%%%%%%%%%%%%%%%%%%
\subsection{Related CTAN Packages}

There are several other packages which offer a similar functionality:
%
\begin{itemize}
\item
The packages
\href{http://ctan.org/pkg/docmute}{\textsf{docmute}},
\href{http://ctan.org/pkg/includex}{\textsf{includex}} and
\href{http://ctan.org/pkg/standalone}{\textsf{standalone}}
provide commands to include only the document body of
a child file thus allowing both files to be compiled individually.
\item
The packages \href{http://ctan.org/pkg/subdocs}{\textsf{subdocs}}
and \href{http://ctan.org/pkg/subfiles}{\textsf{subfiles}}
provide structures in which the main and child documents can be
encapsulated and allowing them to be compiled individually.
The inclusion mechanism is different from the conventional |\include|.
\item
The package \href{http://ctan.org/pkg/combine}{\textsf{combine}}
is an elaborate solution to combine several documents into one.
\end{itemize}
%
See also the CTAN topic \href{http://ctan.org/topic/subdocs}{\textsf{subdocs}}
for further related packages.
The present package differs from the above solutions in that
a document structure constructed with the conventional |\include| mechanism
just needs two extra commands at the top of every file
such that all constituent files can be compiled individually.

%%%%%%%%%%%%%%%%%%%%%%%%%%%%%%%%%%%%%%%%%%%%%%%%%%%%%%%%%%%%%%%%%%%%%%%%%%%%%%%%
%\subsection{Feature Suggestions}
%
%The following is a list of features which may be useful for future
%versions of this package:
%%
%\begin{itemize}
%\item
%\ldots
%\end{itemize}

%%%%%%%%%%%%%%%%%%%%%%%%%%%%%%%%%%%%%%%%%%%%%%%%%%%%%%%%%%%%%%%%%%%%%%%%%%%%%%%%
\subsection{Revision History}

%%%%%%%%%%%%%%%%%%%%%%%%%%%%%%%%%%%%%%%%
\paragraph{v2.0:} 2018/12/30

\begin{itemize}
\item
immediate forward processing
\item
added |\childdocby| mechanism
\item
manual restructured
\end{itemize}

%%%%%%%%%%%%%%%%%%%%%%%%%%%%%%%%%%%%%%%%
\paragraph{v1.6:} 2018/01/17

\begin{itemize}
\item
application for development of include files
\item
corrections to manual
\end{itemize}

%%%%%%%%%%%%%%%%%%%%%%%%%%%%%%%%%%%%%%%%
\paragraph{v1.5:} 2017/05/21

\begin{itemize}
\item
more complete structuring introduced
\item
|\childdocof| introduced
\item
|\childdoc| renamed to |\childdocmain|
\item
|\childredirect| renamed to |\childdocforward| and |\childdocforwardprefix|
and functionality expanded
\end{itemize}

%%%%%%%%%%%%%%%%%%%%%%%%%%%%%%%%%%%%%%%%
\paragraph{v1.0:} 2017/04/27

\begin{itemize}
\item
manual and install package
\item
first version published on CTAN
\end{itemize}

%%%%%%%%%%%%%%%%%%%%%%%%%%%%%%%%%%%%%%%%
\paragraph{v0.6:} 2017/04/26

\begin{itemize}
\item
redirection mechanism added
\end{itemize}

%%%%%%%%%%%%%%%%%%%%%%%%%%%%%%%%%%%%%%%%
\paragraph{v0.5:} 2017/04/26

\begin{itemize}
\item
functionality in definition file
\end{itemize}


%%%%%%%%%%%%%%%%%%%%%%%%%%%%%%%%%%%%%%%%%%%%%%%%%%%%%%%%%%%%%%%%%%%%%%%%%%%%%%%%
%%%%%%%%%%%%%%%%%%%%%%%%%%%%%%%%%%%%%%%%%%%%%%%%%%%%%%%%%%%%%%%%%%%%%%%%%%%%%%%%
%%%%%%%%%%%%%%%%%%%%%%%%%%%%%%%%%%%%%%%%%%%%%%%%%%%%%%%%%%%%%%%%%%%%%%%%%%%%%%%%
\appendix

\settowidth\MacroIndent{\rmfamily\scriptsize 000\ }

 \DocInput{childdoc.dtx}

\end{document}
%</driver>
% \fi
%
% %%%%%%%%%%%%%%%%%%%%%%%%%%%%%%%%%%%%%%%%%%%%%%%%%%%%%%%%%%%%%%%%%%%%%%%%%%%%%%
% %%%%%%%%%%%%%%%%%%%%%%%%%%%%%%%%%%%%%%%%%%%%%%%%%%%%%%%%%%%%%%%%%%%%%%%%%%%%%%
% \section{Sample}
%\iffalse
%<*samplemain>
%\fi
%
% The following presents a sample document
% with two chapters, two parts, a title page,
% a compile flag as well as three forwarding files to set the flag.
% It consists of eight |.tex| files:
% \begin{center}
% \begin{tabular}{ll}
% |cdocsamp.tex|&main file\\
% |cdocsch1.tex|&include file for chapter 1\\
% |cdocsch2.tex|&include file for chapter 2\\
% |cdocspt3.tex|&include file for part 3\\
% |cdocspt4.tex|&include file for part 4\\
% |cdocsdrf.tex|&forwarding file for main file in draft mode\\
% |cdocsfi1.tex|&forwarding file for final version of chapter 1\\
% |cdocsfi2.tex|&forwarding file for final version of chapter 2\\
% \end{tabular}
% \end{center}
% Each of the eight files can be compiled directly by the \LaTeX{} compiler.
%
% %%%%%%%%%%%%%%%%%%%%%%%%%%%%%%%%%%%%%%
% \paragraph{Main File.}
%
% The main file is called |cdocsamp.tex|.
%
% Load the \textsf{childdoc} definitions and
% declare the filename for the main document:
%    \begin{macrocode}
\input{childdoc.def}
\childdocmain{}
%    \end{macrocode}

% Optional override for |\version| flag:
%    \begin{macrocode}
%%\ifchilddoc\else\providecommand{\version}{draft}\fi
%    \end{macrocode}

% Define the default values for the |\version| flag
% (|final| for the main file and |draft| for childs):
%    \begin{macrocode}
\ifchilddoc
\providecommand{\version}{draft}
\else
\providecommand{\version}{final}
\fi
%    \end{macrocode}

% Load the standard document class:
%    \begin{macrocode}
\documentclass[12pt]{article}
%    \end{macrocode}

% Start the document body:
%    \begin{macrocode}
\begin{document}
%    \end{macrocode}

% Declare a title page.
% Print title, part of document being processed and version flag:
%    \begin{macrocode}
\addtocounter{page}{-1}
\begin{center}
{\LARGE\bfseries{}childdoc example\par}
\vspace{1cm}
\ifchilddoc
\ifchilddocmanual part\else chapter\fi:
`\childdocname' of `\childdocjob'\par
\else
main document: `\childdocjob'\par
\fi
version: \version\par
\end{center}
\newpage
%    \end{macrocode}

% Manually include selected file,
% otherwise process as usual:
%    \begin{macrocode}
\ifchilddocmanual
\section*{part `\childdocname'}
\input{\childdocname}
\else
%    \end{macrocode}

% Include the two chapters:
%    \begin{macrocode}
\include{cdocsch1}
\include{cdocsch2}
%    \end{macrocode}

% Include the two parts unless only chapters should be displayed:
%    \begin{macrocode}
\ifchilddoc\else
\section{part three}
\input{cdocspt3}
\section{part four}
\input{cdocspt4}
\fi
%    \end{macrocode}

% Process as usual until here:
%    \begin{macrocode}
\fi
%    \end{macrocode}

% End of document body:
%    \begin{macrocode}
\end{document}
%    \end{macrocode}
%\iffalse
%</samplemain>
%\fi
%
% %%%%%%%%%%%%%%%%%%%%%%%%%%%%%%%%%%%%%%
% \paragraph{Chapter Include Files.}
%
% The include files are called |cdocsch1.tex| and |cdocsch2.tex|.
%
%\iffalse
%<*samplechap1|samplechap2>
%\fi

% Optional override for |\version| flag:
%    \begin{macrocode}
%%\providecommand{\version}{final}
%    \end{macrocode}

% Include the main document:
%    \begin{macrocode}
\input{childdoc.def}
\childdocof{cdocsamp}
%    \end{macrocode}

%\iffalse
%</samplechap1|samplechap2>
%\fi
%
%\iffalse
%<*samplechap1>
%\fi
% Some text for chapter 1:
%    \begin{macrocode}
\section{one}
some text in chapter one
%    \end{macrocode}

%\iffalse
%</samplechap1>
%\fi
% Some text for chapter 2:
%\iffalse
%<*samplechap2>
%\fi
%    \begin{macrocode}
\section{two}
more text in chapter two
%    \end{macrocode}

%\iffalse
%</samplechap2>
%\fi
%
% %%%%%%%%%%%%%%%%%%%%%%%%%%%%%%%%%%%%%%
% \paragraph{Part Include Files.}
%
% The include files are called |cdocspt3.tex| and |cdocspt4.tex|.
%
%\iffalse
%<*samplepart3|samplepart4>
%\fi

% Optional override for |\version| flag:
%    \begin{macrocode}
%%\providecommand{\version}{final}
%    \end{macrocode}

% Include the main document:
%    \begin{macrocode}
\input{childdoc.def}
\childdocby{cdocsamp}
%    \end{macrocode}

%\iffalse
%</samplepart3|samplepart4>
%\fi
%
%\iffalse
%<*samplepart3>
%\fi
% Some text for part 3:
%    \begin{macrocode}
some text in part three
%    \end{macrocode}

%\iffalse
%</samplepart3>
%\fi
% Some text for part 4:
%\iffalse
%<*samplepart4>
%\fi
%    \begin{macrocode}
more text in part four
%    \end{macrocode}

%\iffalse
%</samplepart4>
%\fi
%
% %%%%%%%%%%%%%%%%%%%%%%%%%%%%%%%%%%%%%%
% \paragraph{Forwarding for a Complete Draft.}
%
% The following forwarding file |cdocsdrf.tex|
% compiles the main document in draft mode:
%\iffalse
%<*sampledraft>
%\fi
%    \begin{macrocode}
\def\version{draft}
\input{childdoc.def}
\childdocforward{cdocsamp}
%    \end{macrocode}

%\iffalse
%</sampledraft>
%\fi
%
% %%%%%%%%%%%%%%%%%%%%%%%%%%%%%%%%%%%%%%
% \paragraph{Forwarding for Final Version of the Chapters.}
%
% The following forwarding files |cdocsfn1.tex| and |cdocsfn2.tex|
% (with identical content)
% compile the final versions of the child documents
% |cdocsch1.tex| and |cdocsch2.tex|, respectively:
%\iffalse
%<*samplefinal>
%\fi
%    \begin{macrocode}
\def\version{final}
\input{childdoc.def}
\childdocforwardprefix[cdocsamp]{cdocsfn}{cdocsch}
%    \end{macrocode}

%\iffalse
%</samplefinal>
%\fi
%
% %%%%%%%%%%%%%%%%%%%%%%%%%%%%%%%%%%%%%%
% \paragraph{Command Line Processing.}
%
% The following three command lines generate the output files
% |cdocscld|, |cdocscl1| and |cdocscl2|
% which should be identical to
% |cdocsdrf|, |cdocsch1| and |cdocsfn2|, respectively:
% \begin{center}
% \begin{tabular}{l}
% |latex -jobname cdocscld \|\\
% |  "\def\version{draft}\input{childdoc.def}\childdocforward{cdocsamp}"|\\
% |latex -jobname cdocscl1 \|\\
% |  "\input{childdoc.def}\childdocforward[cdocsamp]{cdocsch1}"|\\
% |latex -jobname cdocscl2 \|\\
% |  "\def\version{final}\input{childdoc.def}\childdocforward{cdocsch2}"|
% \end{tabular}
% \end{center}
% Note that the trailing backslash on each first line
% merely continues the input to the second line
% (for convenient cut ant paste).
% Furthermore, the command |latex| can be replaced by any
% of its alternative versions such as |pdflatex|.
%
% %%%%%%%%%%%%%%%%%%%%%%%%%%%%%%%%%%%%%%%%%%%%%%%%%%%%%%%%%%%%%%%%%%%%%%%%%%%%%%
% %%%%%%%%%%%%%%%%%%%%%%%%%%%%%%%%%%%%%%%%%%%%%%%%%%%%%%%%%%%%%%%%%%%%%%%%%%%%%%
% \section{Implementation}
%\iffalse
%<*package>
%\fi
%
% This section describes the definitions file |childdoc.def|.

% The definitions cannot be loaded using |\usepackage| or |\RequirePackage|
% which has a mechanism to prevent loading a style file more than once.
% When loading the definitions by means of |\input|
% multiple instances have to be prevented manually:
%\iffalse
%This code needs to be before the `\ProvidesFile' directive
%which is defined at the beginning of this file.
%Therefore it is also placed there and commented out here.
%</package>
%<*discard>
%\fi
%    \begin{macrocode}
\ifdefined\childdocmain\endinput\fi
%    \end{macrocode}
%\iffalse
%</discard>
%<*package>
%\fi
%
% \macro{\ifchilddoc}
% \macro{\ifchilddocmanual}
% The conditional |\ifchilddoc| tells whether a
% child (true) or main (false) document is being compiled.
% The conditional |\ifchilddocmanual| tells whether
% the |\includeonly| mechanism is used (false) or
% the selection of child files must be performed manually (true).
% The definitions initialise to false:
%    \begin{macrocode}
\newif\ifchilddoc
\newif\ifchilddocmanual
%    \end{macrocode}

% \macro{\childdocname}
% \macro{\childdocjob}
% The macro |\childdocname| stores the name of the main document
% to be compiled. The macro |\childdocjob| stores the name of
% the document on which the \LaTeX{} compiler was originally invoked.
% The content of |\jobname| cannot be compared
% to filenames specified in the source due to different catcodes.
% The following code rescans |\jobname|, stores the result
% in |\childdocname| and saves a copy in |\childdocjob|:
%    \begin{macrocode}
\edef\childdocname{\scantokens\expandafter{\jobname\noexpand}}
\let\childdocjob\childdocname
%    \end{macrocode}

% \macro{\childdocdisable}
% The macro |\childdocdisable| prevents the main file
% from being processed more than once.
% At this stage, the main document command |\childdocmain|
% is assumed to be called once again where it should do nothing.
% Any subsequent call to it should prevent
% a secondary processing of the main document
% It overwrites the forwarding commands
% |\childdocof| and |\childdocforward|
% with empty macros to prevent further inclusions of the main document:
%    \begin{macrocode}
\newcommand{\childdocdisable}
{
  \renewcommand{\childdocmain}[1]{\renewcommand{\childdocmain}[1]{\endinput}}
  \renewcommand{\childdocof}[1]{}
  \renewcommand{\childdocby}[2][]{}
  \renewcommand{\childdocforward}[2][]{}
  \renewcommand{\childdocdisable}{}
}
%    \end{macrocode}

% \macro{\childdocmain}
% The macro |\childdocmain| is to be called at the top of the main file
% with nothing or the main filename (without extension) as argument.
% First, it breaks loops.
% If the argument is not empty and does not match |\childdocname|
% (which is set by the first inclusion of |childdoc.def|),
% |\ifchilddoc| is set to true, |\includeonly| is applied to the child file
% and |\jobname| is set to the main file
% (for proper handling of |.aux| files):
%    \begin{macrocode}
\newcommand{\childdocmain}[1]
{
  \childdocdisable\childdocmain{}
  \if?#1?\else
    \begingroup
      \def\childdoctmp{#1}
      \ifx\childdoctmp\childdocname
        \def\childdoctmp{}
      \else
        \def\childdoctmp
        {
          \childdoctrue
          \includeonly{\childdocname}
          \def\childdocjob{#1}
          \def\jobname{#1}
        }
      \fi
      \expandafter
    \endgroup
    \childdoctmp
  \fi
}
%    \end{macrocode}

% \macro{\childdocof}
% The command |\childdocof| redirects
% compilation to the main file |#1|.
%    \begin{macrocode}
\newcommand{\childdocof}[1]
{
  \childdocdisable
  \childdoctrue
  \includeonly{\childdocname}
  \def\jobname{#1}
  \def\childdocjob{#1}
  \input{#1}
}
%    \end{macrocode}

% \macro{\childdocby}
% The command |\childdocby| ....
%    \begin{macrocode}
\newcommand{\childdocby}[2][]
{
  \childdocdisable
  \childdoctrue
  \childdocmanualtrue
  \if?#1?\else
    \def\jobname{#2}
  \fi
  \def\childdocjob{#2}
  \input{#2}
  \endinput
}
%    \end{macrocode}

% \macro{\childdocforward}
% The command |\childdocforward| redirects
% compilation to the main file or
% (if the optional argument is given) a child file.
% Parameters are set as if the main file
% or a child file starting with |\childdocof| was compiled.
% Then compilation is handed over to the main file:
%    \begin{macrocode}
\newcommand{\childdocforward}[2][]
{
  \begingroup
    \if?#1?
      \def\childdoctmp
      {
        \def\childdocname{#2}
        \def\childdocjob{#2}
        \def\jobname{#2}
        \input{#2}
        \endinput
      }
    \else
      \def\childdoctmp
      {
        \childdocdisable
        \def\childdocname{#2}
        \childdoctrue
        \includeonly{#2}
        \def\childdocjob{#1}
        \def\jobname{#1}
        \input{#1}
        \endinput
      }
    \fi
    \expandafter
  \endgroup
  \childdoctmp
}
%    \end{macrocode}

% \macro{\childdocforwardprefix}
% The command |\childdocforwardprefix| redirects
% compilation to the main or a child file by means of a pattern.
% The prefix |#1| in the current filename is replaced by |#2|
% and the suffix of the current filename is kept
% (it is assumed that the filename does not contain the substring `|~~~|'
% which is used as a delimiter).
% Compilation is handed over to the new file by |\childdocforward|:
%    \begin{macrocode}
\newcommand{\childdocforwardprefix}[3][]
{
  \begingroup
    \def\childdocextract #2##1~~~{\def\childdoctmp{\childdocforward[#1]{#3##1}}}
    \expandafter\childdocextract\childdocname~~~
    \expandafter
  \endgroup
  \childdoctmp
}
%    \end{macrocode}

% \macro{\childdoc}
% The deprecated macro |\childdoc| is a legacy version of |\childdocmain|:
%    \begin{macrocode}
\newcommand{\childdoc}{\childdocmain}
%    \end{macrocode}

% \macro{\childdocredirect}
% The deprecated macro |\childdocredirect| is a legacy version
% of |\childdocforward| and |\childdocforwardprefix|:
%    \begin{macrocode}
\newcommand{\childdocredirect}[2][]
{
  \begingroup
    \if?#1?
      \def\childdoctmp{\childdocforward{#2}}
    \else
      \def\childdoctmp{\childdocforwardprefix{#1}{#2}}
    \fi
    \expandafter
  \endgroup
  \childdoctmp
}
%    \end{macrocode}

%\iffalse
%</package>
%\fi
%
\endinput
|\\
|\childdocmain{}|\\
\end{tabular}
\end{center}
at the very top of the main \LaTeX{} file,
in particular \emph{before} the |\documentclass| statement!
The argument of |\childdocmain| should be left empty
(but it must be present).

%%%%%%%%%%%%%%%%%%%%%%%%%%%%%%%%%%%%%%%%
\DescribeMacro{\childdocof}
Furthermore, add the commands
\begin{center}
\begin{tabular}{l}
|% \iffalse
%
% childdoc.dtx Copyright (C) 2017-2018 Niklas Beisert
%
% This work may be distributed and/or modified under the
% conditions of the LaTeX Project Public License, either version 1.3
% of this license or (at your option) any later version.
% The latest version of this license is in
%   http://www.latex-project.org/lppl.txt
% and version 1.3 or later is part of all distributions of LaTeX
% version 2005/12/01 or later.
%
% This work has the LPPL maintenance status `maintained'.
%
% The Current Maintainer of this work is Niklas Beisert.
%
% This work consists of the files childdoc.dtx and childdoc.ins
% and the derived files childdoc.def and cdocsamp.tex with
% cdocsch1.tex, cdocsch2.tex, cdocsdrf.tex, cdocsfn1.tex, cdocsfn2.tex.
%
%<package>\ifdefined\childdocmain\endinput\fi
%<package>\ProvidesFile{childdoc.def}[2018/12/30 v2.0 child document driver]
%<samplemain>\ProvidesFile{cdocsamp.tex}[2018/12/30 v2.0 sample for childdoc]
%<*driver>
%\ProvidesFile{childdoc.drv}[2018/12/30 v2.0 childdoc reference manual file]
\PassOptionsToClass{10pt,a4paper}{article}
\documentclass{ltxdoc}

\usepackage[margin=35mm]{geometry}
\usepackage{hyperref}
\usepackage{hyperxmp}
\usepackage[usenames]{color}

\hypersetup{colorlinks=true}
\hypersetup{pdfstartview=FitH}
\hypersetup{pdfpagemode=UseNone}
\hypersetup{pdfsource={}}
\hypersetup{pdflang={en-UK}}
\hypersetup{pdfcopyright={Copyright 2017-2018 Niklas Beisert.
  This work may be distributed and/or modified under the
  conditions of the LaTeX Project Public License, either version 1.3
  of this license or (at your option) any later version.}}
\hypersetup{pdflicenseurl={http://www.latex-project.org/lppl.txt}}
\hypersetup{pdfcontactaddress={ETH Zurich, ITP, HIT K,
  Wolfgang-Pauli-Strasse 27}}
\hypersetup{pdfcontactpostcode={8093}}
\hypersetup{pdfcontactcity={Zurich}}
\hypersetup{pdfcontactcountry={Switzerland}}
\hypersetup{pdfcontactemail={nbeisert@itp.phys.ethz.ch}}
\hypersetup{pdfcontacturl={http://people.phys.ethz.ch/\xmptilde nbeisert/}}

\newcommand{\secref}[1]{\hyperref[#1]{section \ref*{#1}}}

\parskip1ex
\parindent0pt
\let\olditemize\itemize
\def\itemize{\olditemize\parskip0pt}

\begin{document}

\title{The \textsf{childdoc} Package}
\hypersetup{pdftitle={The childdoc Package}}
\author{Niklas Beisert\\[2ex]
  Institut f\"ur Theoretische Physik\\
  Eidgen\"ossische Technische Hochschule Z\"urich\\
  Wolfgang-Pauli-Strasse 27, 8093 Z\"urich, Switzerland\\[1ex]
  \href{mailto:nbeisert@itp.phys.ethz.ch}
  {\texttt{nbeisert@itp.phys.ethz.ch}}}
\hypersetup{pdfauthor={Niklas Beisert}}
\hypersetup{pdfsubject={Manual for the LaTeX2e Package childdoc}}
\date{30 December 2018, \textsf{v2.0}}
\maketitle

\begin{abstract}\noindent
\textsf{childdoc} is a \LaTeXe{} package
that enables the direct compilation
of document sections included by |\include|
to individual files.
\end{abstract}

\begingroup
\parskip0ex
\tableofcontents
\endgroup

%%%%%%%%%%%%%%%%%%%%%%%%%%%%%%%%%%%%%%%%%%%%%%%%%%%%%%%%%%%%%%%%%%%%%%%%%%%%%%%%
%%%%%%%%%%%%%%%%%%%%%%%%%%%%%%%%%%%%%%%%%%%%%%%%%%%%%%%%%%%%%%%%%%%%%%%%%%%%%%%%
\section{Introduction}

\LaTeX{} provides a mechanism to structure a large document (such as a book)
into a main file and several child files (containing the chapters)
using the |\include| command.
This mechanism is beneficial for documents
which span hundreds of pages in order to
make the source file(s) more manageable.
Moreover, compilation can be restricted to
selected child files by means of the |\includeonly| command.
The latter feature can be used to reduce the compilation time while editing
(this was significantly more useful in the earlier days of \LaTeX{})
or to generate a smaller document which is easier to navigate.
Another application of |\includeonly| is to generate
documents consisting of selected parts of the complete document.

However, there are a few drawbacks of the plain |\include| mechanism:
\begin{itemize}
\item
The child files cannot be compiled on their own,
they can only be compiled via the main file.
A naive editing environment
(such as a text editor with an option
to have the current file processed by \LaTeX)
may require one to switch to the main file before compiling;
attempting to compile the child file produces errors.
\item
The main file must be modified (each time)
to adjust the |\includeonly| command
to the present needs. This easily leaves the main file in a messy state.
\item
The generated document will always carry the filename
of the main document. This is inconvenient if
several child files are to be compiled and
to be kept for distribution.
\end{itemize}

The present package provides a simple interface
to make child files individually compilable by \LaTeX{}.
Compiling a child file then has the same effect as compiling
the main file with an |\includeonly| command
to select the appropriate child.
Moreover the generated document will carry the name of the child
rather than the main file.
This resolves all three above issues.

This feature is meant to make the editing of books,
thesis documents and lecture notes somewhat more convenient.
However, the package can also be used efficiently for
composing a series of documents (such as exercise sheets)
which are typically distributed individually.
It then assists the author in generating the individual documents
(potentially in different versions)
as well as a document containing the collected series.
Another application is in developing style files
or other kinds of included material
where compilation of the style file could redirect
to a sample or test file.

%%%%%%%%%%%%%%%%%%%%%%%%%%%%%%%%%%%%%%%%%%%%%%%%%%%%%%%%%%%%%%%%%%%%%%%%%%%%%%%%
%%%%%%%%%%%%%%%%%%%%%%%%%%%%%%%%%%%%%%%%%%%%%%%%%%%%%%%%%%%%%%%%%%%%%%%%%%%%%%%%
\section{Usage}

First of all, the package \textsf{childdoc} is \emph{not} a standard
\LaTeXe{} |.sty| style file! Therefore it needs to be invoked in
a non-standard way.

%%%%%%%%%%%%%%%%%%%%%%%%%%%%%%%%%%%%%%%%%%%%%%%%%%%%%%%%%%%%%%%%%%%%%%%%%%%%%%%%
\subsection{Included Files}
\label{sec:include}

%%%%%%%%%%%%%%%%%%%%%%%%%%%%%%%%%%%%%%%%
\DescribeMacro{\childdocmain}
To use the package, add the commands
\begin{center}
\begin{tabular}{l}
|\input{childdoc.def}|\\
|\childdocmain{}|\\
\end{tabular}
\end{center}
at the very top of the main \LaTeX{} file,
in particular \emph{before} the |\documentclass| statement!
The argument of |\childdocmain| should be left empty
(but it must be present).

%%%%%%%%%%%%%%%%%%%%%%%%%%%%%%%%%%%%%%%%
\DescribeMacro{\childdocof}
Furthermore, add the commands
\begin{center}
\begin{tabular}{l}
|\input{childdoc.def}|\\
|\childdocof{|\textit{main}|}|\\
\end{tabular}
\end{center}
at the top of every child file \textit{child}
which is included by |\include{|\textit{child}|}|
from within the main file
(or at least for those files to be compiled individually).
The argument \textit{main} must be the filename of the main file.

There are a couple of
considerations in setting up the main and child documents:

%%%%%%%%%%%%%%%%%%%%%%%%%%%%%%%%%%%%%%%%
\paragraph{Restrictions.}

Please note the following restrictions:
\begin{itemize}
\item
|\childdocmain| must be called with one argument \textit{main}
to ensure compatibility with earlier version of the package.
It must either be empty (|\childdocmain{}|)
or precisely match the filename of the main file in which it is specified.
See \secref{sec:detection} for further information.
\item
The filename \textit{main} must be specified without the |.tex| extension.
\item
The filename \textit{main} is case sensitive
(even in case-insensitive file systems)
due to internal string comparison.
\item
The argument \textit{main} should be fully expanded, it cannot be a macro.
\item
Subdirectories and special characters should be avoided in filenames.
\item
The command |\childdocmain{|\textit{main}|}| must be followed by a whitespace.
It should not be followed immediately by another command
or by a comment mark `|%|'.
This is because the \TeX{} parser reads the token immediately following
the argument of |\childdocmain| and puts it
at the beginning of every child section;
however, a white\-space is ignored.
\end{itemize}

%%%%%%%%%%%%%%%%%%%%%%%%%%%%%%%%%%%%%%%%
\paragraph{Content of Main File.}

It is advisable to place all content in the child files included by |\include|.
Any output contained in the main file will appear in all child documents
unless suppressed manually;
it cannot be suppressed automatically by the |\includeonly| directive
and thus should normally be avoided.
A method to include some content in the main file
by means of conditional processing is described in \secref{sec:conditional}.

%%%%%%%%%%%%%%%%%%%%%%%%%%%%%%%%%%%%%%%%
\paragraph{Page Numbering.}

When only a part of the document is compiled,
the appropriate numbering of pages
(as well as other status parameters)
is determined from the |.aux| files.
The latter contain information from previous passes.
However this information needs to propagate through
all intermediate child documents.
Therefore the page numbering in child documents may well
be inconsistent until the complete document is compiled at least once.

A useful (if unconventional) way to always ensure a consistent
page numbering is to restart the numbering in each child document
and denote the pages by `\textit{child}|.|\textit{page}'
where \textit{child} represents the chapter/section number of the child file.
This can be achieved by the command
|\numberwithin{page}{|\textit{child}|}|
of the \textsf{amsmath} package
where \textit{child} can be |chapter| or |section|
depending on the chosen structuring.
Alternatively, one can modify the macro |\thepage| appropriately
and reset the counter |page| at the start of each child file.

%%%%%%%%%%%%%%%%%%%%%%%%%%%%%%%%%%%%%%%%%%%%%%%%%%%%%%%%%%%%%%%%%%%%%%%%%%%%%%%%
\subsection{Conditional Processing}
\label{sec:conditional}

The package provides a mechanism to compile different versions
of a document. To customise the versions further some conditional processing
can come in handy to distinguish which version is being compiled.
The package provides two macros to describe the compilation context:

%%%%%%%%%%%%%%%%%%%%%%%%%%%%%%%%%%%%%%%%
\DescribeMacro{\ifchilddoc}
The conditional |\ifchilddoc| distinguishes between the compilation of
child documents and the main document:
%
\begin{center}
|\ifchilddoc |\textit{child-code}| |[|\||else |\textit{main-code}]| \||fi|
\end{center}

%%%%%%%%%%%%%%%%%%%%%%%%%%%%%%%%%%%%%%%%
\DescribeMacro{\childdocname}
\DescribeMacro{\childdocjob}
The macro |\childdocname| contains the filename (without extension)
of the main or child file being processed.
Note that |\childdocjob| will always contain the name of the main file.

%%%%%%%%%%%%%%%%%%%%%%%%%%%%%%%%%%%%%%%%
\paragraph{Title Page.}

Conditional processing can be used to include a title or banner page
in the main document when proper precautions are taken.
Importantly, the code in the main file should ensure that the page counter
(as well as other status parameters which are stored in the |.aux| files)
takes the same value after the conditional processing.
Otherwise the page numbers may take divergent values
depending on which part is compiled.

For example, a title page could be declared by:
%
\begin{center}
\begin{tabular}{l}
|\ifchilddoc\||else|\\
|\addtocounter{page}{-1}|\\
\textit{code for title page}\\
|\newpage|\\
|\||fi|
\end{tabular}
\end{center}
%
A banner page for the child documents can be generated by:
%
\begin{center}
\begin{tabular}{l}
|\ifchilddoc|\\
|\addtocounter{page}{-1}|\\
\textit{code for banner page}\\
|\newpage|\\
|\||fi|
\end{tabular}
\end{center}
%
Here one could write a message such as:
\begin{center}
|This is the part \childdocname{} of \childdocjob{}.|
\end{center}

%%%%%%%%%%%%%%%%%%%%%%%%%%%%%%%%%%%%%%%%%%%%%%%%%%%%%%%%%%%%%%%%%%%%%%%%%%%%%%%%
\subsection{Flags}
\label{sec:flags}

The package makes it easy to generate different versions
of the main or child documents.
To this end compilation flags can be defined
and assigned different default values.
They will be particularly useful in conjunction
with the forwarding mechanism described in \secref{sec:forward}.

For example, it may be useful to have a flag |\version|
which can be set to |draft| or |final|.
The document source will contain some conditional code
depending on the value of |\version|.
Suppose further, the flag should default to |final| for the main file
and to |draft| for child files
which is a natural assignment for editing the document.
This is achieved by placing the following code
in the preamble of the main document
(below the |\childdocmain| directive):
%
\begin{center}
\begin{tabular}{l}
|\ifchilddoc|\\
|\providecommand{\version}{draft}|\\
|\||else|\\
|\providecommand{\version}{final}|\\
|\||fi|
\end{tabular}
\end{center}
%
The definition by |\providecommand| makes sure
that previous definitions are not overwritten.
Further statements |\providecommand{\version}{...}|
can thus be added before the above code to override it.

For the main file, one might add a line
(between |\childdocmain| and the above block)
%
\begin{center}
|%\ifchilddoc\||else\providecommand{\version}{draft}\||fi|
\end{center}
%
which can be uncommented to produce a draft version.
Likewise one can add a line to the very top of a child file
(above the |\childdocof{|\textit{main}|}| directive)
%
\begin{center}
|%\providecommand{\version}{final}|
\end{center}
%
which can be uncommented to produce the final version of this child document.

%%%%%%%%%%%%%%%%%%%%%%%%%%%%%%%%%%%%%%%%%%%%%%%%%%%%%%%%%%%%%%%%%%%%%%%%%%%%%%%%
\subsection{Forwarding}
\label{sec:forward}

Different versions of the main or child documents
using compilation flags as described in \secref{sec:flags}
can be (permanently) stored in different files
for convenient compilation, viewing and distribution.
To this end, the package defines a command
to pass on compilation to a different file:

%%%%%%%%%%%%%%%%%%%%%%%%%%%%%%%%%%%%%%%%
\DescribeMacro{\childdocforward}
The command |\childdocforward| redirects processing to
another source file:
%
\begin{center}
\begin{tabular}{l}
|\input{childdoc.def}|\\
|\childdocforward[|\textit{main}|]{|\textit{dest}|}|\\
\end{tabular}
\end{center}
%
The argument \textit{dest} is the destination file
(without extension).
It should be the main file or one of the child files.
Note that further \textsf{childdoc} directives
such as |\childdocof| and |\childdocforward|
in the indicated file will be processed in this form.
The optional argument \textit{main}
passes on directly to the main file \textit{main}
while pretending to compile the child \textit{dest}.
This form behaves as if \textit{dest}
issues |\childdocof{|\textit{main}|}| right away,
and no further \textsf{childdoc} directives will be processed.

%%%%%%%%%%%%%%%%%%%%%%%%%%%%%%%%%%%%%%%%
\DescribeMacro{\...prefix}
In the alternative form |\childdocforwardprefix|,
%
\begin{center}
\begin{tabular}{l}
|\input{childdoc.def}|\\
|\childdocforwardprefix[|\textit{main}|]{|\textit{prefix}|}{|\textit{dest}|}|
\end{tabular}
\end{center}
%
the destination file is determined by a pattern
depending on the current file:
To make this work, the current file must be called
`{\textit{prefix}\hspace{0.2em}\textit{suffix}}'
with \textit{prefix} matching precisely the argument.
Processing is then passed on to the file
`{\textit{dest}\hspace{0.2em}\textit{suffix}}'.
Surely, the same effect is achieved by
directly specifying the
argument `{\textit{dest}\hspace{0.2em}\textit{suffix}}'
in the first form.
However, that requires to set up a different file
for each child. With the alternative form of the command
all these files can have exactly the same content
which simplifies setting them up and maintaining them.

For example, the following file |draft.tex|
with a compilation flag |\version| as described in \secref{sec:flags}
compiles the main document as a draft:
%
\begin{center}
\begin{tabular}{l}
|\def\version{draft}|\\
|\input{childdoc.def}|\\
|\childdocforward{|\textit{main}|}|
\end{tabular}
\end{center}
%
Likewise, the following files |final|\textit{nn}|.tex|
compile the final version of the child document
|child|\textit{nn}|.tex|:
%
\begin{center}
\begin{tabular}{l}
|\def\version{final}|\\
|\input{childdoc.def}|\\
|\childdocforwardprefix{final}{child}|
\end{tabular}
\end{center}
%

Note that when several versions of a main file and/or of each child file
are to be generated, it may be convenient to set up a |Makefile| or
shell script to automatise the process.

%%%%%%%%%%%%%%%%%%%%%%%%%%%%%%%%%%%%%%%%%%%%%%%%%%%%%%%%%%%%%%%%%%%%%%%%%%%%%%%%
\subsection{Command Line Processing}
\label{sec:commandline}

The effect of redirection files can also be achieved by invoking
the \LaTeX{} compiler with a more elaborate command line.
Most conveniently this should be done as part
of a shell script or a |Makefile|.

When using \textsf{childdoc} in the main file, the following
command lines effectively perform a redirection
(note that depending on the shell being used,
backslashes may have to be doubled: `|\|' $\to$ `|\\|'):
%
\begin{center}
|... -jobname "|\textit{target}|" |\\|"|[\textit{flags}]%
|\input{childdoc.def}\childdocforward[|\textit{main}|]{|\textit{dest}|}"|
\end{center}
%
Here \textit{target} is the name of the output file,
\textit{main} is the name of the main file
and \textit{dest} is the name of the main or child file to be processed
(all filenames without extensions).
The optional argument \textit{main} can be omitted
if \textit{main} matches \textit{dest}.
Optionally, compilation \textit{flags} can be defined via |\def| commands.
This command line makes the \TeX{} engine believe
it is compiling the file \textit{target}
whose content is specified as the latter parameter.
The provided code then forwards the processing to
\textit{main} or \textit{dest} as described in \secref{sec:forward}.

%%%%%%%%%%%%%%%%%%%%%%%%%%%%%%%%%%%%%%%%%%%%%%%%%%%%%%%%%%%%%%%%%%%%%%%%%%%%%%%%
\subsection{Include by Input}
\label{sec:input}

Including child documents by |\include| has some restrictions by design.
Most notably, the content of a child document always occupies
its own set of pages; pages cannot be shared between child documents.
Usually, this behaviour makes perfect sense
because each child document contain an essential part of the document.
However, in some situations it may be desirable to compose
a document from a collection of parts
without having mandatory page breaks between then.
For this case, the package
provides a mechanism to include parts
by |\input| which can also be processed individually.
However, by construction this mechanism
requires manual handling of the content to be output.

%%%%%%%%%%%%%%%%%%%%%%%%%%%%%%%%%%%%%%%%
\DescribeMacro{\ifchilddocmanual}
The main file should be prepared as usual, see \secref{sec:include}.
However, the document body must make a distinction
between processing of an individual part and of the main document, e.g.:
%
\begin{center}
\begin{tabular}{l}
|\ifchilddocmanual|\\
|\input{\childdocname}|\\
|\||else|\\
\textit{document body with }|\input{|\textit{part}|}|\\
|\||fi|
\end{tabular}
\end{center}
%
The conditional |\ifchilddocmanual| is true whenever
a part to be included by |\input| is being compiled,
and the name of the part is stored in |\childdocname|.

%%%%%%%%%%%%%%%%%%%%%%%%%%%%%%%%%%%%%%%%
\DescribeMacro{\childdocby}
Each part to be included by |\input| should start with:
%
\begin{center}
\begin{tabular}{l}
|\input{childdoc.def}|\\
|\childdocby{|\textit{main}|}|\\
\end{tabular}
\end{center}
%
The directive |\childdocby| is similar to |\childdocof|
described in \secref{sec:include},
but the subsequent selection of content must be done manually.
To that end, both |\ifchilddoc| and |\ifchilddocmanual|
will be true upon processing of a part,
and the name of the part is stored in |\childdocname|.
Note that |\jobname| will be set to the filename of the current part
so that each part receives an individual |.aux| file
that does not interfere with the |.aux| file(s) of the main document.
This behaviour can be altered by the alternative form
|\childdocby[*]{|\textit{main}|}| (with a non-empty optional argument)
which uses the |.aux| file of the main document
by setting |\jobname| to \textit{main}.

%%%%%%%%%%%%%%%%%%%%%%%%%%%%%%%%%%%%%%%%%%%%%%%%%%%%%%%%%%%%%%%%%%%%%%%%%%%%%%%%
\subsection{Driver Development}
\label{sec:driver}

The \textsf{childdoc} mechanism can also be use for the development
of definition files such as \LaTeX{} styles or classes.
This case differs from the above setup with multiple parts
included by |\include| in that no |\includeonly| should be invoked.
This can be achieved by starting the include file
(before |\ProvidesPackage|) with:
%
\begin{center}
\begin{tabular}{l}
|\input{childdoc.def}|\\
|\childdocforward{|\textit{main}|}|\\
\end{tabular}
\end{center}
%
or alternatively with:
%
\begin{center}
\begin{tabular}{l}
|\input{childdoc.def}|\\
|\childdocby{|\textit{main}|}|\\
\end{tabular}
\end{center}
%
Both forms have slightly different effects as described above.
The main file is prepared as usual, see \secref{sec:include}.

%%%%%%%%%%%%%%%%%%%%%%%%%%%%%%%%%%%%%%%%%%%%%%%%%%%%%%%%%%%%%%%%%%%%%%%%%%%%%%%%
\subsection{Legacy Detection}
\label{sec:detection}

The directive |\childdocmain| in the main file can detect
whether the complete document or merely a child is to be compiled
even without using the directive |\childdocof|.
This method is deprecated because it is less robust
and there is no compelling reason to use it;
it is merely provided for backward compatibility
and it may be removed in future versions.

If the detection mechanism is to be used,
it is mandatory to correctly specify
the filename of the main file as the argument of |\childdocmain|:
%
\begin{center}
\begin{tabular}{l}
|\input{childdoc.def}|\\
|\childdocmain{|\textit{main}|}|\\
\end{tabular}
\end{center}
%
If |\jobname| does not match the argument \textit{main} of |\childdocmain|,
it is assumed that |\jobname| points to the child file to be compiled.
When using |\childdocmain| with the main file specified as argument,
it suffices to start a child file
with just |\input{|\textit{main}|}|
without loading of the package and using |\childdocof|.
If instead all processing is done
with the appropriate \textsf{childdoc} directives,
the argument of \textit{main} of |\childdocmain| can be empty.

An alternative version of the command line processing described
in \secref{sec:commandline} using the detection mechanism reads:
%
\begin{center}
|... -jobname "|\textit{target}|" "|[\textit{flags}]%
[|\def\jobname{|\textit{dest}|}|]|\input{|\textit{main}|}"|
\end{center}

%%%%%%%%%%%%%%%%%%%%%%%%%%%%%%%%%%%%%%%%%%%%%%%%%%%%%%%%%%%%%%%%%%%%%%%%%%%%%%%%
\subsection{Manual Code}
\label{sec:manual}

In case one cannot be certain whether the definitions file |childdoc.def|
is installed on the target \TeX{} distribution
and one prefers not to ship it,
it is conceivable to paste a few relevant commands into the sources.

To that end, drop all statements |\input{childdoc.def}|
and perform the replacements as outlined below.
Instead of |\childdocmain{|\textit{main}|}| add the following code
to the top of the main file:
%
\begin{center}
\begin{tabular}{l}
|\||ifdefined\childdocname\endinput\||fi\newif\ifchilddoc|\\
|\edef\childdocname{\scantokens\expandafter{\jobname\noexpand}}|\\
|\def\childdocmain{|\textit{main}|}\||ifx\childdocmain\childdocname\||else|\\
|\childdoctrue\includeonly{\childdocname}\let\jobname\childdocmain\||fi|\\
\end{tabular}
\end{center}
%
Instead of |\childdocof{|\textit{main}|}| just include the main file
at the top of each child file:
%
\begin{center}
|\input{|\textit{main}|}|
\end{center}
%
A simple redirection |\childdocforward{|\textit{dest}|}| is achieved by:
%
\begin{center}
|\def\jobname{|\textit{dest}|}\input{\jobname}|
\end{center}
%
The redirection with prefix
|\childdocforwardprefix[|\textit{prefix}|]{|\textit{dest}|}|
is accomplished by:
%
\begin{center}
\begin{tabular}{l}
|{\edef\jobname{\scantokens\expandafter{\jobname\noexpand}}|\\
|\def\redirectjob |\textit{prefix}|#1~~~{\gdef\jobname{|\textit{dest}|#1}}|\\
|\expandafter\redirectjob\jobname~~~}\input{\jobname}|
\end{tabular}
\end{center}

In an alternative approach,
child documents can be compiled by a specific command line
without additional code or specific definitions:
%
\begin{center}
|... -jobname "|\textit{target}|" "|[\textit{flags}]%
|\includeonly{|\textit{dest}|}\input{|\textit{main}|}"|
\end{center}
%

%%%%%%%%%%%%%%%%%%%%%%%%%%%%%%%%%%%%%%%%%%%%%%%%%%%%%%%%%%%%%%%%%%%%%%%%%%%%%%%%
%%%%%%%%%%%%%%%%%%%%%%%%%%%%%%%%%%%%%%%%%%%%%%%%%%%%%%%%%%%%%%%%%%%%%%%%%%%%%%%%
\section{Information}

%%%%%%%%%%%%%%%%%%%%%%%%%%%%%%%%%%%%%%%%%%%%%%%%%%%%%%%%%%%%%%%%%%%%%%%%%%%%%%%%
\subsection{Copyright}

Copyright \copyright{} 2017--2018 Niklas Beisert

This work may be distributed and/or modified under the
conditions of the \LaTeX{} Project Public License, either version 1.3
of this license or (at your option) any later version.
The latest version of this license is in
  \url{http://www.latex-project.org/lppl.txt}
and version 1.3 or later is part of all distributions of \LaTeX{}
version 2005/12/01 or later.

This work has the LPPL maintenance status `maintained'.

The Current Maintainer of this work is Niklas Beisert.

This work consists of the files |README.txt|, |childdoc.ins| and |childdoc.dtx|
as well as the derived files |childdoc.def|, |cdocsamp.tex|
with |cdocsch1.tex|, |cdocsch2.tex|, |cdocspt3.tex|, |cdocspt4.tex|,
|cdocsdrf.tex|, |cdocsfn1.tex|, |cdocsfn2.tex|
as well as |childdoc.pdf|.

%%%%%%%%%%%%%%%%%%%%%%%%%%%%%%%%%%%%%%%%%%%%%%%%%%%%%%%%%%%%%%%%%%%%%%%%%%%%%%%%
\subsection{Files and Installation}

The package consists of the files:
%
\begin{center}
\begin{tabular}{ll}
    |README.txt|   & readme file \\
    |childdoc.ins| & installation file \\
    |childdoc.dtx| & source file \\
    |childdoc.def| & definition file \\
    |cdocsamp.tex| & sample main file \\
    |cdocsch1.tex| & sample include file \\
    |cdocsch2.tex| & sample include file \\
    |cdocspt3.tex| & sample part file \\
    |cdocspt4.tex| & sample part file \\
    |cdocsdrf.tex| & sample redirection file \\
    |cdocsfn1.tex| & sample redirection file \\
    |cdocsfn2.tex| & sample redirection file \\
    |childdoc.pdf| & manual
\end{tabular}
\end{center}
%
The distribution consists of the files
|README.txt|, |childdoc.ins| and |childdoc.dtx|.
%
\begin{itemize}
\item
Run (pdf)\LaTeX{} on |childdoc.dtx|
to compile the manual |childdoc.pdf| (this file).
\item
Run \LaTeX{} on |childdoc.ins| to create the definitions file |childdoc.def|
and the sample |cdocsamp.tex| with include files
|cdocsch1.tex|, |cdocsch2.tex|, |cdocspt3.tex|, |cdocspt4.tex|,
|cdocsdrf.tex|, |cdocsfn1.tex|, |cdocsfn2.tex|.
Then copy the file |childdoc.def| to an appropriate directory of your \LaTeX{}
distribution, e.g.\ \textit{texmf-root}|/tex/latex/childdoc|.
\end{itemize}

%%%%%%%%%%%%%%%%%%%%%%%%%%%%%%%%%%%%%%%%%%%%%%%%%%%%%%%%%%%%%%%%%%%%%%%%%%%%%%%%
\subsection{Related CTAN Packages}

There are several other packages which offer a similar functionality:
%
\begin{itemize}
\item
The packages
\href{http://ctan.org/pkg/docmute}{\textsf{docmute}},
\href{http://ctan.org/pkg/includex}{\textsf{includex}} and
\href{http://ctan.org/pkg/standalone}{\textsf{standalone}}
provide commands to include only the document body of
a child file thus allowing both files to be compiled individually.
\item
The packages \href{http://ctan.org/pkg/subdocs}{\textsf{subdocs}}
and \href{http://ctan.org/pkg/subfiles}{\textsf{subfiles}}
provide structures in which the main and child documents can be
encapsulated and allowing them to be compiled individually.
The inclusion mechanism is different from the conventional |\include|.
\item
The package \href{http://ctan.org/pkg/combine}{\textsf{combine}}
is an elaborate solution to combine several documents into one.
\end{itemize}
%
See also the CTAN topic \href{http://ctan.org/topic/subdocs}{\textsf{subdocs}}
for further related packages.
The present package differs from the above solutions in that
a document structure constructed with the conventional |\include| mechanism
just needs two extra commands at the top of every file
such that all constituent files can be compiled individually.

%%%%%%%%%%%%%%%%%%%%%%%%%%%%%%%%%%%%%%%%%%%%%%%%%%%%%%%%%%%%%%%%%%%%%%%%%%%%%%%%
%\subsection{Feature Suggestions}
%
%The following is a list of features which may be useful for future
%versions of this package:
%%
%\begin{itemize}
%\item
%\ldots
%\end{itemize}

%%%%%%%%%%%%%%%%%%%%%%%%%%%%%%%%%%%%%%%%%%%%%%%%%%%%%%%%%%%%%%%%%%%%%%%%%%%%%%%%
\subsection{Revision History}

%%%%%%%%%%%%%%%%%%%%%%%%%%%%%%%%%%%%%%%%
\paragraph{v2.0:} 2018/12/30

\begin{itemize}
\item
immediate forward processing
\item
added |\childdocby| mechanism
\item
manual restructured
\end{itemize}

%%%%%%%%%%%%%%%%%%%%%%%%%%%%%%%%%%%%%%%%
\paragraph{v1.6:} 2018/01/17

\begin{itemize}
\item
application for development of include files
\item
corrections to manual
\end{itemize}

%%%%%%%%%%%%%%%%%%%%%%%%%%%%%%%%%%%%%%%%
\paragraph{v1.5:} 2017/05/21

\begin{itemize}
\item
more complete structuring introduced
\item
|\childdocof| introduced
\item
|\childdoc| renamed to |\childdocmain|
\item
|\childredirect| renamed to |\childdocforward| and |\childdocforwardprefix|
and functionality expanded
\end{itemize}

%%%%%%%%%%%%%%%%%%%%%%%%%%%%%%%%%%%%%%%%
\paragraph{v1.0:} 2017/04/27

\begin{itemize}
\item
manual and install package
\item
first version published on CTAN
\end{itemize}

%%%%%%%%%%%%%%%%%%%%%%%%%%%%%%%%%%%%%%%%
\paragraph{v0.6:} 2017/04/26

\begin{itemize}
\item
redirection mechanism added
\end{itemize}

%%%%%%%%%%%%%%%%%%%%%%%%%%%%%%%%%%%%%%%%
\paragraph{v0.5:} 2017/04/26

\begin{itemize}
\item
functionality in definition file
\end{itemize}


%%%%%%%%%%%%%%%%%%%%%%%%%%%%%%%%%%%%%%%%%%%%%%%%%%%%%%%%%%%%%%%%%%%%%%%%%%%%%%%%
%%%%%%%%%%%%%%%%%%%%%%%%%%%%%%%%%%%%%%%%%%%%%%%%%%%%%%%%%%%%%%%%%%%%%%%%%%%%%%%%
%%%%%%%%%%%%%%%%%%%%%%%%%%%%%%%%%%%%%%%%%%%%%%%%%%%%%%%%%%%%%%%%%%%%%%%%%%%%%%%%
\appendix

\settowidth\MacroIndent{\rmfamily\scriptsize 000\ }

 \DocInput{childdoc.dtx}

\end{document}
%</driver>
% \fi
%
% %%%%%%%%%%%%%%%%%%%%%%%%%%%%%%%%%%%%%%%%%%%%%%%%%%%%%%%%%%%%%%%%%%%%%%%%%%%%%%
% %%%%%%%%%%%%%%%%%%%%%%%%%%%%%%%%%%%%%%%%%%%%%%%%%%%%%%%%%%%%%%%%%%%%%%%%%%%%%%
% \section{Sample}
%\iffalse
%<*samplemain>
%\fi
%
% The following presents a sample document
% with two chapters, two parts, a title page,
% a compile flag as well as three forwarding files to set the flag.
% It consists of eight |.tex| files:
% \begin{center}
% \begin{tabular}{ll}
% |cdocsamp.tex|&main file\\
% |cdocsch1.tex|&include file for chapter 1\\
% |cdocsch2.tex|&include file for chapter 2\\
% |cdocspt3.tex|&include file for part 3\\
% |cdocspt4.tex|&include file for part 4\\
% |cdocsdrf.tex|&forwarding file for main file in draft mode\\
% |cdocsfi1.tex|&forwarding file for final version of chapter 1\\
% |cdocsfi2.tex|&forwarding file for final version of chapter 2\\
% \end{tabular}
% \end{center}
% Each of the eight files can be compiled directly by the \LaTeX{} compiler.
%
% %%%%%%%%%%%%%%%%%%%%%%%%%%%%%%%%%%%%%%
% \paragraph{Main File.}
%
% The main file is called |cdocsamp.tex|.
%
% Load the \textsf{childdoc} definitions and
% declare the filename for the main document:
%    \begin{macrocode}
\input{childdoc.def}
\childdocmain{}
%    \end{macrocode}

% Optional override for |\version| flag:
%    \begin{macrocode}
%%\ifchilddoc\else\providecommand{\version}{draft}\fi
%    \end{macrocode}

% Define the default values for the |\version| flag
% (|final| for the main file and |draft| for childs):
%    \begin{macrocode}
\ifchilddoc
\providecommand{\version}{draft}
\else
\providecommand{\version}{final}
\fi
%    \end{macrocode}

% Load the standard document class:
%    \begin{macrocode}
\documentclass[12pt]{article}
%    \end{macrocode}

% Start the document body:
%    \begin{macrocode}
\begin{document}
%    \end{macrocode}

% Declare a title page.
% Print title, part of document being processed and version flag:
%    \begin{macrocode}
\addtocounter{page}{-1}
\begin{center}
{\LARGE\bfseries{}childdoc example\par}
\vspace{1cm}
\ifchilddoc
\ifchilddocmanual part\else chapter\fi:
`\childdocname' of `\childdocjob'\par
\else
main document: `\childdocjob'\par
\fi
version: \version\par
\end{center}
\newpage
%    \end{macrocode}

% Manually include selected file,
% otherwise process as usual:
%    \begin{macrocode}
\ifchilddocmanual
\section*{part `\childdocname'}
\input{\childdocname}
\else
%    \end{macrocode}

% Include the two chapters:
%    \begin{macrocode}
\include{cdocsch1}
\include{cdocsch2}
%    \end{macrocode}

% Include the two parts unless only chapters should be displayed:
%    \begin{macrocode}
\ifchilddoc\else
\section{part three}
\input{cdocspt3}
\section{part four}
\input{cdocspt4}
\fi
%    \end{macrocode}

% Process as usual until here:
%    \begin{macrocode}
\fi
%    \end{macrocode}

% End of document body:
%    \begin{macrocode}
\end{document}
%    \end{macrocode}
%\iffalse
%</samplemain>
%\fi
%
% %%%%%%%%%%%%%%%%%%%%%%%%%%%%%%%%%%%%%%
% \paragraph{Chapter Include Files.}
%
% The include files are called |cdocsch1.tex| and |cdocsch2.tex|.
%
%\iffalse
%<*samplechap1|samplechap2>
%\fi

% Optional override for |\version| flag:
%    \begin{macrocode}
%%\providecommand{\version}{final}
%    \end{macrocode}

% Include the main document:
%    \begin{macrocode}
\input{childdoc.def}
\childdocof{cdocsamp}
%    \end{macrocode}

%\iffalse
%</samplechap1|samplechap2>
%\fi
%
%\iffalse
%<*samplechap1>
%\fi
% Some text for chapter 1:
%    \begin{macrocode}
\section{one}
some text in chapter one
%    \end{macrocode}

%\iffalse
%</samplechap1>
%\fi
% Some text for chapter 2:
%\iffalse
%<*samplechap2>
%\fi
%    \begin{macrocode}
\section{two}
more text in chapter two
%    \end{macrocode}

%\iffalse
%</samplechap2>
%\fi
%
% %%%%%%%%%%%%%%%%%%%%%%%%%%%%%%%%%%%%%%
% \paragraph{Part Include Files.}
%
% The include files are called |cdocspt3.tex| and |cdocspt4.tex|.
%
%\iffalse
%<*samplepart3|samplepart4>
%\fi

% Optional override for |\version| flag:
%    \begin{macrocode}
%%\providecommand{\version}{final}
%    \end{macrocode}

% Include the main document:
%    \begin{macrocode}
\input{childdoc.def}
\childdocby{cdocsamp}
%    \end{macrocode}

%\iffalse
%</samplepart3|samplepart4>
%\fi
%
%\iffalse
%<*samplepart3>
%\fi
% Some text for part 3:
%    \begin{macrocode}
some text in part three
%    \end{macrocode}

%\iffalse
%</samplepart3>
%\fi
% Some text for part 4:
%\iffalse
%<*samplepart4>
%\fi
%    \begin{macrocode}
more text in part four
%    \end{macrocode}

%\iffalse
%</samplepart4>
%\fi
%
% %%%%%%%%%%%%%%%%%%%%%%%%%%%%%%%%%%%%%%
% \paragraph{Forwarding for a Complete Draft.}
%
% The following forwarding file |cdocsdrf.tex|
% compiles the main document in draft mode:
%\iffalse
%<*sampledraft>
%\fi
%    \begin{macrocode}
\def\version{draft}
\input{childdoc.def}
\childdocforward{cdocsamp}
%    \end{macrocode}

%\iffalse
%</sampledraft>
%\fi
%
% %%%%%%%%%%%%%%%%%%%%%%%%%%%%%%%%%%%%%%
% \paragraph{Forwarding for Final Version of the Chapters.}
%
% The following forwarding files |cdocsfn1.tex| and |cdocsfn2.tex|
% (with identical content)
% compile the final versions of the child documents
% |cdocsch1.tex| and |cdocsch2.tex|, respectively:
%\iffalse
%<*samplefinal>
%\fi
%    \begin{macrocode}
\def\version{final}
\input{childdoc.def}
\childdocforwardprefix[cdocsamp]{cdocsfn}{cdocsch}
%    \end{macrocode}

%\iffalse
%</samplefinal>
%\fi
%
% %%%%%%%%%%%%%%%%%%%%%%%%%%%%%%%%%%%%%%
% \paragraph{Command Line Processing.}
%
% The following three command lines generate the output files
% |cdocscld|, |cdocscl1| and |cdocscl2|
% which should be identical to
% |cdocsdrf|, |cdocsch1| and |cdocsfn2|, respectively:
% \begin{center}
% \begin{tabular}{l}
% |latex -jobname cdocscld \|\\
% |  "\def\version{draft}\input{childdoc.def}\childdocforward{cdocsamp}"|\\
% |latex -jobname cdocscl1 \|\\
% |  "\input{childdoc.def}\childdocforward[cdocsamp]{cdocsch1}"|\\
% |latex -jobname cdocscl2 \|\\
% |  "\def\version{final}\input{childdoc.def}\childdocforward{cdocsch2}"|
% \end{tabular}
% \end{center}
% Note that the trailing backslash on each first line
% merely continues the input to the second line
% (for convenient cut ant paste).
% Furthermore, the command |latex| can be replaced by any
% of its alternative versions such as |pdflatex|.
%
% %%%%%%%%%%%%%%%%%%%%%%%%%%%%%%%%%%%%%%%%%%%%%%%%%%%%%%%%%%%%%%%%%%%%%%%%%%%%%%
% %%%%%%%%%%%%%%%%%%%%%%%%%%%%%%%%%%%%%%%%%%%%%%%%%%%%%%%%%%%%%%%%%%%%%%%%%%%%%%
% \section{Implementation}
%\iffalse
%<*package>
%\fi
%
% This section describes the definitions file |childdoc.def|.

% The definitions cannot be loaded using |\usepackage| or |\RequirePackage|
% which has a mechanism to prevent loading a style file more than once.
% When loading the definitions by means of |\input|
% multiple instances have to be prevented manually:
%\iffalse
%This code needs to be before the `\ProvidesFile' directive
%which is defined at the beginning of this file.
%Therefore it is also placed there and commented out here.
%</package>
%<*discard>
%\fi
%    \begin{macrocode}
\ifdefined\childdocmain\endinput\fi
%    \end{macrocode}
%\iffalse
%</discard>
%<*package>
%\fi
%
% \macro{\ifchilddoc}
% \macro{\ifchilddocmanual}
% The conditional |\ifchilddoc| tells whether a
% child (true) or main (false) document is being compiled.
% The conditional |\ifchilddocmanual| tells whether
% the |\includeonly| mechanism is used (false) or
% the selection of child files must be performed manually (true).
% The definitions initialise to false:
%    \begin{macrocode}
\newif\ifchilddoc
\newif\ifchilddocmanual
%    \end{macrocode}

% \macro{\childdocname}
% \macro{\childdocjob}
% The macro |\childdocname| stores the name of the main document
% to be compiled. The macro |\childdocjob| stores the name of
% the document on which the \LaTeX{} compiler was originally invoked.
% The content of |\jobname| cannot be compared
% to filenames specified in the source due to different catcodes.
% The following code rescans |\jobname|, stores the result
% in |\childdocname| and saves a copy in |\childdocjob|:
%    \begin{macrocode}
\edef\childdocname{\scantokens\expandafter{\jobname\noexpand}}
\let\childdocjob\childdocname
%    \end{macrocode}

% \macro{\childdocdisable}
% The macro |\childdocdisable| prevents the main file
% from being processed more than once.
% At this stage, the main document command |\childdocmain|
% is assumed to be called once again where it should do nothing.
% Any subsequent call to it should prevent
% a secondary processing of the main document
% It overwrites the forwarding commands
% |\childdocof| and |\childdocforward|
% with empty macros to prevent further inclusions of the main document:
%    \begin{macrocode}
\newcommand{\childdocdisable}
{
  \renewcommand{\childdocmain}[1]{\renewcommand{\childdocmain}[1]{\endinput}}
  \renewcommand{\childdocof}[1]{}
  \renewcommand{\childdocby}[2][]{}
  \renewcommand{\childdocforward}[2][]{}
  \renewcommand{\childdocdisable}{}
}
%    \end{macrocode}

% \macro{\childdocmain}
% The macro |\childdocmain| is to be called at the top of the main file
% with nothing or the main filename (without extension) as argument.
% First, it breaks loops.
% If the argument is not empty and does not match |\childdocname|
% (which is set by the first inclusion of |childdoc.def|),
% |\ifchilddoc| is set to true, |\includeonly| is applied to the child file
% and |\jobname| is set to the main file
% (for proper handling of |.aux| files):
%    \begin{macrocode}
\newcommand{\childdocmain}[1]
{
  \childdocdisable\childdocmain{}
  \if?#1?\else
    \begingroup
      \def\childdoctmp{#1}
      \ifx\childdoctmp\childdocname
        \def\childdoctmp{}
      \else
        \def\childdoctmp
        {
          \childdoctrue
          \includeonly{\childdocname}
          \def\childdocjob{#1}
          \def\jobname{#1}
        }
      \fi
      \expandafter
    \endgroup
    \childdoctmp
  \fi
}
%    \end{macrocode}

% \macro{\childdocof}
% The command |\childdocof| redirects
% compilation to the main file |#1|.
%    \begin{macrocode}
\newcommand{\childdocof}[1]
{
  \childdocdisable
  \childdoctrue
  \includeonly{\childdocname}
  \def\jobname{#1}
  \def\childdocjob{#1}
  \input{#1}
}
%    \end{macrocode}

% \macro{\childdocby}
% The command |\childdocby| ....
%    \begin{macrocode}
\newcommand{\childdocby}[2][]
{
  \childdocdisable
  \childdoctrue
  \childdocmanualtrue
  \if?#1?\else
    \def\jobname{#2}
  \fi
  \def\childdocjob{#2}
  \input{#2}
  \endinput
}
%    \end{macrocode}

% \macro{\childdocforward}
% The command |\childdocforward| redirects
% compilation to the main file or
% (if the optional argument is given) a child file.
% Parameters are set as if the main file
% or a child file starting with |\childdocof| was compiled.
% Then compilation is handed over to the main file:
%    \begin{macrocode}
\newcommand{\childdocforward}[2][]
{
  \begingroup
    \if?#1?
      \def\childdoctmp
      {
        \def\childdocname{#2}
        \def\childdocjob{#2}
        \def\jobname{#2}
        \input{#2}
        \endinput
      }
    \else
      \def\childdoctmp
      {
        \childdocdisable
        \def\childdocname{#2}
        \childdoctrue
        \includeonly{#2}
        \def\childdocjob{#1}
        \def\jobname{#1}
        \input{#1}
        \endinput
      }
    \fi
    \expandafter
  \endgroup
  \childdoctmp
}
%    \end{macrocode}

% \macro{\childdocforwardprefix}
% The command |\childdocforwardprefix| redirects
% compilation to the main or a child file by means of a pattern.
% The prefix |#1| in the current filename is replaced by |#2|
% and the suffix of the current filename is kept
% (it is assumed that the filename does not contain the substring `|~~~|'
% which is used as a delimiter).
% Compilation is handed over to the new file by |\childdocforward|:
%    \begin{macrocode}
\newcommand{\childdocforwardprefix}[3][]
{
  \begingroup
    \def\childdocextract #2##1~~~{\def\childdoctmp{\childdocforward[#1]{#3##1}}}
    \expandafter\childdocextract\childdocname~~~
    \expandafter
  \endgroup
  \childdoctmp
}
%    \end{macrocode}

% \macro{\childdoc}
% The deprecated macro |\childdoc| is a legacy version of |\childdocmain|:
%    \begin{macrocode}
\newcommand{\childdoc}{\childdocmain}
%    \end{macrocode}

% \macro{\childdocredirect}
% The deprecated macro |\childdocredirect| is a legacy version
% of |\childdocforward| and |\childdocforwardprefix|:
%    \begin{macrocode}
\newcommand{\childdocredirect}[2][]
{
  \begingroup
    \if?#1?
      \def\childdoctmp{\childdocforward{#2}}
    \else
      \def\childdoctmp{\childdocforwardprefix{#1}{#2}}
    \fi
    \expandafter
  \endgroup
  \childdoctmp
}
%    \end{macrocode}

%\iffalse
%</package>
%\fi
%
\endinput
|\\
|\childdocof{|\textit{main}|}|\\
\end{tabular}
\end{center}
at the top of every child file \textit{child}
which is included by |\include{|\textit{child}|}|
from within the main file
(or at least for those files to be compiled individually).
The argument \textit{main} must be the filename of the main file.

There are a couple of
considerations in setting up the main and child documents:

%%%%%%%%%%%%%%%%%%%%%%%%%%%%%%%%%%%%%%%%
\paragraph{Restrictions.}

Please note the following restrictions:
\begin{itemize}
\item
|\childdocmain| must be called with one argument \textit{main}
to ensure compatibility with earlier version of the package.
It must either be empty (|\childdocmain{}|)
or precisely match the filename of the main file in which it is specified.
See \secref{sec:detection} for further information.
\item
The filename \textit{main} must be specified without the |.tex| extension.
\item
The filename \textit{main} is case sensitive
(even in case-insensitive file systems)
due to internal string comparison.
\item
The argument \textit{main} should be fully expanded, it cannot be a macro.
\item
Subdirectories and special characters should be avoided in filenames.
\item
The command |\childdocmain{|\textit{main}|}| must be followed by a whitespace.
It should not be followed immediately by another command
or by a comment mark `|%|'.
This is because the \TeX{} parser reads the token immediately following
the argument of |\childdocmain| and puts it
at the beginning of every child section;
however, a white\-space is ignored.
\end{itemize}

%%%%%%%%%%%%%%%%%%%%%%%%%%%%%%%%%%%%%%%%
\paragraph{Content of Main File.}

It is advisable to place all content in the child files included by |\include|.
Any output contained in the main file will appear in all child documents
unless suppressed manually;
it cannot be suppressed automatically by the |\includeonly| directive
and thus should normally be avoided.
A method to include some content in the main file
by means of conditional processing is described in \secref{sec:conditional}.

%%%%%%%%%%%%%%%%%%%%%%%%%%%%%%%%%%%%%%%%
\paragraph{Page Numbering.}

When only a part of the document is compiled,
the appropriate numbering of pages
(as well as other status parameters)
is determined from the |.aux| files.
The latter contain information from previous passes.
However this information needs to propagate through
all intermediate child documents.
Therefore the page numbering in child documents may well
be inconsistent until the complete document is compiled at least once.

A useful (if unconventional) way to always ensure a consistent
page numbering is to restart the numbering in each child document
and denote the pages by `\textit{child}|.|\textit{page}'
where \textit{child} represents the chapter/section number of the child file.
This can be achieved by the command
|\numberwithin{page}{|\textit{child}|}|
of the \textsf{amsmath} package
where \textit{child} can be |chapter| or |section|
depending on the chosen structuring.
Alternatively, one can modify the macro |\thepage| appropriately
and reset the counter |page| at the start of each child file.

%%%%%%%%%%%%%%%%%%%%%%%%%%%%%%%%%%%%%%%%%%%%%%%%%%%%%%%%%%%%%%%%%%%%%%%%%%%%%%%%
\subsection{Conditional Processing}
\label{sec:conditional}

The package provides a mechanism to compile different versions
of a document. To customise the versions further some conditional processing
can come in handy to distinguish which version is being compiled.
The package provides two macros to describe the compilation context:

%%%%%%%%%%%%%%%%%%%%%%%%%%%%%%%%%%%%%%%%
\DescribeMacro{\ifchilddoc}
The conditional |\ifchilddoc| distinguishes between the compilation of
child documents and the main document:
%
\begin{center}
|\ifchilddoc |\textit{child-code}| |[|\||else |\textit{main-code}]| \||fi|
\end{center}

%%%%%%%%%%%%%%%%%%%%%%%%%%%%%%%%%%%%%%%%
\DescribeMacro{\childdocname}
\DescribeMacro{\childdocjob}
The macro |\childdocname| contains the filename (without extension)
of the main or child file being processed.
Note that |\childdocjob| will always contain the name of the main file.

%%%%%%%%%%%%%%%%%%%%%%%%%%%%%%%%%%%%%%%%
\paragraph{Title Page.}

Conditional processing can be used to include a title or banner page
in the main document when proper precautions are taken.
Importantly, the code in the main file should ensure that the page counter
(as well as other status parameters which are stored in the |.aux| files)
takes the same value after the conditional processing.
Otherwise the page numbers may take divergent values
depending on which part is compiled.

For example, a title page could be declared by:
%
\begin{center}
\begin{tabular}{l}
|\ifchilddoc\||else|\\
|\addtocounter{page}{-1}|\\
\textit{code for title page}\\
|\newpage|\\
|\||fi|
\end{tabular}
\end{center}
%
A banner page for the child documents can be generated by:
%
\begin{center}
\begin{tabular}{l}
|\ifchilddoc|\\
|\addtocounter{page}{-1}|\\
\textit{code for banner page}\\
|\newpage|\\
|\||fi|
\end{tabular}
\end{center}
%
Here one could write a message such as:
\begin{center}
|This is the part \childdocname{} of \childdocjob{}.|
\end{center}

%%%%%%%%%%%%%%%%%%%%%%%%%%%%%%%%%%%%%%%%%%%%%%%%%%%%%%%%%%%%%%%%%%%%%%%%%%%%%%%%
\subsection{Flags}
\label{sec:flags}

The package makes it easy to generate different versions
of the main or child documents.
To this end compilation flags can be defined
and assigned different default values.
They will be particularly useful in conjunction
with the forwarding mechanism described in \secref{sec:forward}.

For example, it may be useful to have a flag |\version|
which can be set to |draft| or |final|.
The document source will contain some conditional code
depending on the value of |\version|.
Suppose further, the flag should default to |final| for the main file
and to |draft| for child files
which is a natural assignment for editing the document.
This is achieved by placing the following code
in the preamble of the main document
(below the |\childdocmain| directive):
%
\begin{center}
\begin{tabular}{l}
|\ifchilddoc|\\
|\providecommand{\version}{draft}|\\
|\||else|\\
|\providecommand{\version}{final}|\\
|\||fi|
\end{tabular}
\end{center}
%
The definition by |\providecommand| makes sure
that previous definitions are not overwritten.
Further statements |\providecommand{\version}{...}|
can thus be added before the above code to override it.

For the main file, one might add a line
(between |\childdocmain| and the above block)
%
\begin{center}
|%\ifchilddoc\||else\providecommand{\version}{draft}\||fi|
\end{center}
%
which can be uncommented to produce a draft version.
Likewise one can add a line to the very top of a child file
(above the |\childdocof{|\textit{main}|}| directive)
%
\begin{center}
|%\providecommand{\version}{final}|
\end{center}
%
which can be uncommented to produce the final version of this child document.

%%%%%%%%%%%%%%%%%%%%%%%%%%%%%%%%%%%%%%%%%%%%%%%%%%%%%%%%%%%%%%%%%%%%%%%%%%%%%%%%
\subsection{Forwarding}
\label{sec:forward}

Different versions of the main or child documents
using compilation flags as described in \secref{sec:flags}
can be (permanently) stored in different files
for convenient compilation, viewing and distribution.
To this end, the package defines a command
to pass on compilation to a different file:

%%%%%%%%%%%%%%%%%%%%%%%%%%%%%%%%%%%%%%%%
\DescribeMacro{\childdocforward}
The command |\childdocforward| redirects processing to
another source file:
%
\begin{center}
\begin{tabular}{l}
|% \iffalse
%
% childdoc.dtx Copyright (C) 2017-2018 Niklas Beisert
%
% This work may be distributed and/or modified under the
% conditions of the LaTeX Project Public License, either version 1.3
% of this license or (at your option) any later version.
% The latest version of this license is in
%   http://www.latex-project.org/lppl.txt
% and version 1.3 or later is part of all distributions of LaTeX
% version 2005/12/01 or later.
%
% This work has the LPPL maintenance status `maintained'.
%
% The Current Maintainer of this work is Niklas Beisert.
%
% This work consists of the files childdoc.dtx and childdoc.ins
% and the derived files childdoc.def and cdocsamp.tex with
% cdocsch1.tex, cdocsch2.tex, cdocsdrf.tex, cdocsfn1.tex, cdocsfn2.tex.
%
%<package>\ifdefined\childdocmain\endinput\fi
%<package>\ProvidesFile{childdoc.def}[2018/12/30 v2.0 child document driver]
%<samplemain>\ProvidesFile{cdocsamp.tex}[2018/12/30 v2.0 sample for childdoc]
%<*driver>
%\ProvidesFile{childdoc.drv}[2018/12/30 v2.0 childdoc reference manual file]
\PassOptionsToClass{10pt,a4paper}{article}
\documentclass{ltxdoc}

\usepackage[margin=35mm]{geometry}
\usepackage{hyperref}
\usepackage{hyperxmp}
\usepackage[usenames]{color}

\hypersetup{colorlinks=true}
\hypersetup{pdfstartview=FitH}
\hypersetup{pdfpagemode=UseNone}
\hypersetup{pdfsource={}}
\hypersetup{pdflang={en-UK}}
\hypersetup{pdfcopyright={Copyright 2017-2018 Niklas Beisert.
  This work may be distributed and/or modified under the
  conditions of the LaTeX Project Public License, either version 1.3
  of this license or (at your option) any later version.}}
\hypersetup{pdflicenseurl={http://www.latex-project.org/lppl.txt}}
\hypersetup{pdfcontactaddress={ETH Zurich, ITP, HIT K,
  Wolfgang-Pauli-Strasse 27}}
\hypersetup{pdfcontactpostcode={8093}}
\hypersetup{pdfcontactcity={Zurich}}
\hypersetup{pdfcontactcountry={Switzerland}}
\hypersetup{pdfcontactemail={nbeisert@itp.phys.ethz.ch}}
\hypersetup{pdfcontacturl={http://people.phys.ethz.ch/\xmptilde nbeisert/}}

\newcommand{\secref}[1]{\hyperref[#1]{section \ref*{#1}}}

\parskip1ex
\parindent0pt
\let\olditemize\itemize
\def\itemize{\olditemize\parskip0pt}

\begin{document}

\title{The \textsf{childdoc} Package}
\hypersetup{pdftitle={The childdoc Package}}
\author{Niklas Beisert\\[2ex]
  Institut f\"ur Theoretische Physik\\
  Eidgen\"ossische Technische Hochschule Z\"urich\\
  Wolfgang-Pauli-Strasse 27, 8093 Z\"urich, Switzerland\\[1ex]
  \href{mailto:nbeisert@itp.phys.ethz.ch}
  {\texttt{nbeisert@itp.phys.ethz.ch}}}
\hypersetup{pdfauthor={Niklas Beisert}}
\hypersetup{pdfsubject={Manual for the LaTeX2e Package childdoc}}
\date{30 December 2018, \textsf{v2.0}}
\maketitle

\begin{abstract}\noindent
\textsf{childdoc} is a \LaTeXe{} package
that enables the direct compilation
of document sections included by |\include|
to individual files.
\end{abstract}

\begingroup
\parskip0ex
\tableofcontents
\endgroup

%%%%%%%%%%%%%%%%%%%%%%%%%%%%%%%%%%%%%%%%%%%%%%%%%%%%%%%%%%%%%%%%%%%%%%%%%%%%%%%%
%%%%%%%%%%%%%%%%%%%%%%%%%%%%%%%%%%%%%%%%%%%%%%%%%%%%%%%%%%%%%%%%%%%%%%%%%%%%%%%%
\section{Introduction}

\LaTeX{} provides a mechanism to structure a large document (such as a book)
into a main file and several child files (containing the chapters)
using the |\include| command.
This mechanism is beneficial for documents
which span hundreds of pages in order to
make the source file(s) more manageable.
Moreover, compilation can be restricted to
selected child files by means of the |\includeonly| command.
The latter feature can be used to reduce the compilation time while editing
(this was significantly more useful in the earlier days of \LaTeX{})
or to generate a smaller document which is easier to navigate.
Another application of |\includeonly| is to generate
documents consisting of selected parts of the complete document.

However, there are a few drawbacks of the plain |\include| mechanism:
\begin{itemize}
\item
The child files cannot be compiled on their own,
they can only be compiled via the main file.
A naive editing environment
(such as a text editor with an option
to have the current file processed by \LaTeX)
may require one to switch to the main file before compiling;
attempting to compile the child file produces errors.
\item
The main file must be modified (each time)
to adjust the |\includeonly| command
to the present needs. This easily leaves the main file in a messy state.
\item
The generated document will always carry the filename
of the main document. This is inconvenient if
several child files are to be compiled and
to be kept for distribution.
\end{itemize}

The present package provides a simple interface
to make child files individually compilable by \LaTeX{}.
Compiling a child file then has the same effect as compiling
the main file with an |\includeonly| command
to select the appropriate child.
Moreover the generated document will carry the name of the child
rather than the main file.
This resolves all three above issues.

This feature is meant to make the editing of books,
thesis documents and lecture notes somewhat more convenient.
However, the package can also be used efficiently for
composing a series of documents (such as exercise sheets)
which are typically distributed individually.
It then assists the author in generating the individual documents
(potentially in different versions)
as well as a document containing the collected series.
Another application is in developing style files
or other kinds of included material
where compilation of the style file could redirect
to a sample or test file.

%%%%%%%%%%%%%%%%%%%%%%%%%%%%%%%%%%%%%%%%%%%%%%%%%%%%%%%%%%%%%%%%%%%%%%%%%%%%%%%%
%%%%%%%%%%%%%%%%%%%%%%%%%%%%%%%%%%%%%%%%%%%%%%%%%%%%%%%%%%%%%%%%%%%%%%%%%%%%%%%%
\section{Usage}

First of all, the package \textsf{childdoc} is \emph{not} a standard
\LaTeXe{} |.sty| style file! Therefore it needs to be invoked in
a non-standard way.

%%%%%%%%%%%%%%%%%%%%%%%%%%%%%%%%%%%%%%%%%%%%%%%%%%%%%%%%%%%%%%%%%%%%%%%%%%%%%%%%
\subsection{Included Files}
\label{sec:include}

%%%%%%%%%%%%%%%%%%%%%%%%%%%%%%%%%%%%%%%%
\DescribeMacro{\childdocmain}
To use the package, add the commands
\begin{center}
\begin{tabular}{l}
|\input{childdoc.def}|\\
|\childdocmain{}|\\
\end{tabular}
\end{center}
at the very top of the main \LaTeX{} file,
in particular \emph{before} the |\documentclass| statement!
The argument of |\childdocmain| should be left empty
(but it must be present).

%%%%%%%%%%%%%%%%%%%%%%%%%%%%%%%%%%%%%%%%
\DescribeMacro{\childdocof}
Furthermore, add the commands
\begin{center}
\begin{tabular}{l}
|\input{childdoc.def}|\\
|\childdocof{|\textit{main}|}|\\
\end{tabular}
\end{center}
at the top of every child file \textit{child}
which is included by |\include{|\textit{child}|}|
from within the main file
(or at least for those files to be compiled individually).
The argument \textit{main} must be the filename of the main file.

There are a couple of
considerations in setting up the main and child documents:

%%%%%%%%%%%%%%%%%%%%%%%%%%%%%%%%%%%%%%%%
\paragraph{Restrictions.}

Please note the following restrictions:
\begin{itemize}
\item
|\childdocmain| must be called with one argument \textit{main}
to ensure compatibility with earlier version of the package.
It must either be empty (|\childdocmain{}|)
or precisely match the filename of the main file in which it is specified.
See \secref{sec:detection} for further information.
\item
The filename \textit{main} must be specified without the |.tex| extension.
\item
The filename \textit{main} is case sensitive
(even in case-insensitive file systems)
due to internal string comparison.
\item
The argument \textit{main} should be fully expanded, it cannot be a macro.
\item
Subdirectories and special characters should be avoided in filenames.
\item
The command |\childdocmain{|\textit{main}|}| must be followed by a whitespace.
It should not be followed immediately by another command
or by a comment mark `|%|'.
This is because the \TeX{} parser reads the token immediately following
the argument of |\childdocmain| and puts it
at the beginning of every child section;
however, a white\-space is ignored.
\end{itemize}

%%%%%%%%%%%%%%%%%%%%%%%%%%%%%%%%%%%%%%%%
\paragraph{Content of Main File.}

It is advisable to place all content in the child files included by |\include|.
Any output contained in the main file will appear in all child documents
unless suppressed manually;
it cannot be suppressed automatically by the |\includeonly| directive
and thus should normally be avoided.
A method to include some content in the main file
by means of conditional processing is described in \secref{sec:conditional}.

%%%%%%%%%%%%%%%%%%%%%%%%%%%%%%%%%%%%%%%%
\paragraph{Page Numbering.}

When only a part of the document is compiled,
the appropriate numbering of pages
(as well as other status parameters)
is determined from the |.aux| files.
The latter contain information from previous passes.
However this information needs to propagate through
all intermediate child documents.
Therefore the page numbering in child documents may well
be inconsistent until the complete document is compiled at least once.

A useful (if unconventional) way to always ensure a consistent
page numbering is to restart the numbering in each child document
and denote the pages by `\textit{child}|.|\textit{page}'
where \textit{child} represents the chapter/section number of the child file.
This can be achieved by the command
|\numberwithin{page}{|\textit{child}|}|
of the \textsf{amsmath} package
where \textit{child} can be |chapter| or |section|
depending on the chosen structuring.
Alternatively, one can modify the macro |\thepage| appropriately
and reset the counter |page| at the start of each child file.

%%%%%%%%%%%%%%%%%%%%%%%%%%%%%%%%%%%%%%%%%%%%%%%%%%%%%%%%%%%%%%%%%%%%%%%%%%%%%%%%
\subsection{Conditional Processing}
\label{sec:conditional}

The package provides a mechanism to compile different versions
of a document. To customise the versions further some conditional processing
can come in handy to distinguish which version is being compiled.
The package provides two macros to describe the compilation context:

%%%%%%%%%%%%%%%%%%%%%%%%%%%%%%%%%%%%%%%%
\DescribeMacro{\ifchilddoc}
The conditional |\ifchilddoc| distinguishes between the compilation of
child documents and the main document:
%
\begin{center}
|\ifchilddoc |\textit{child-code}| |[|\||else |\textit{main-code}]| \||fi|
\end{center}

%%%%%%%%%%%%%%%%%%%%%%%%%%%%%%%%%%%%%%%%
\DescribeMacro{\childdocname}
\DescribeMacro{\childdocjob}
The macro |\childdocname| contains the filename (without extension)
of the main or child file being processed.
Note that |\childdocjob| will always contain the name of the main file.

%%%%%%%%%%%%%%%%%%%%%%%%%%%%%%%%%%%%%%%%
\paragraph{Title Page.}

Conditional processing can be used to include a title or banner page
in the main document when proper precautions are taken.
Importantly, the code in the main file should ensure that the page counter
(as well as other status parameters which are stored in the |.aux| files)
takes the same value after the conditional processing.
Otherwise the page numbers may take divergent values
depending on which part is compiled.

For example, a title page could be declared by:
%
\begin{center}
\begin{tabular}{l}
|\ifchilddoc\||else|\\
|\addtocounter{page}{-1}|\\
\textit{code for title page}\\
|\newpage|\\
|\||fi|
\end{tabular}
\end{center}
%
A banner page for the child documents can be generated by:
%
\begin{center}
\begin{tabular}{l}
|\ifchilddoc|\\
|\addtocounter{page}{-1}|\\
\textit{code for banner page}\\
|\newpage|\\
|\||fi|
\end{tabular}
\end{center}
%
Here one could write a message such as:
\begin{center}
|This is the part \childdocname{} of \childdocjob{}.|
\end{center}

%%%%%%%%%%%%%%%%%%%%%%%%%%%%%%%%%%%%%%%%%%%%%%%%%%%%%%%%%%%%%%%%%%%%%%%%%%%%%%%%
\subsection{Flags}
\label{sec:flags}

The package makes it easy to generate different versions
of the main or child documents.
To this end compilation flags can be defined
and assigned different default values.
They will be particularly useful in conjunction
with the forwarding mechanism described in \secref{sec:forward}.

For example, it may be useful to have a flag |\version|
which can be set to |draft| or |final|.
The document source will contain some conditional code
depending on the value of |\version|.
Suppose further, the flag should default to |final| for the main file
and to |draft| for child files
which is a natural assignment for editing the document.
This is achieved by placing the following code
in the preamble of the main document
(below the |\childdocmain| directive):
%
\begin{center}
\begin{tabular}{l}
|\ifchilddoc|\\
|\providecommand{\version}{draft}|\\
|\||else|\\
|\providecommand{\version}{final}|\\
|\||fi|
\end{tabular}
\end{center}
%
The definition by |\providecommand| makes sure
that previous definitions are not overwritten.
Further statements |\providecommand{\version}{...}|
can thus be added before the above code to override it.

For the main file, one might add a line
(between |\childdocmain| and the above block)
%
\begin{center}
|%\ifchilddoc\||else\providecommand{\version}{draft}\||fi|
\end{center}
%
which can be uncommented to produce a draft version.
Likewise one can add a line to the very top of a child file
(above the |\childdocof{|\textit{main}|}| directive)
%
\begin{center}
|%\providecommand{\version}{final}|
\end{center}
%
which can be uncommented to produce the final version of this child document.

%%%%%%%%%%%%%%%%%%%%%%%%%%%%%%%%%%%%%%%%%%%%%%%%%%%%%%%%%%%%%%%%%%%%%%%%%%%%%%%%
\subsection{Forwarding}
\label{sec:forward}

Different versions of the main or child documents
using compilation flags as described in \secref{sec:flags}
can be (permanently) stored in different files
for convenient compilation, viewing and distribution.
To this end, the package defines a command
to pass on compilation to a different file:

%%%%%%%%%%%%%%%%%%%%%%%%%%%%%%%%%%%%%%%%
\DescribeMacro{\childdocforward}
The command |\childdocforward| redirects processing to
another source file:
%
\begin{center}
\begin{tabular}{l}
|\input{childdoc.def}|\\
|\childdocforward[|\textit{main}|]{|\textit{dest}|}|\\
\end{tabular}
\end{center}
%
The argument \textit{dest} is the destination file
(without extension).
It should be the main file or one of the child files.
Note that further \textsf{childdoc} directives
such as |\childdocof| and |\childdocforward|
in the indicated file will be processed in this form.
The optional argument \textit{main}
passes on directly to the main file \textit{main}
while pretending to compile the child \textit{dest}.
This form behaves as if \textit{dest}
issues |\childdocof{|\textit{main}|}| right away,
and no further \textsf{childdoc} directives will be processed.

%%%%%%%%%%%%%%%%%%%%%%%%%%%%%%%%%%%%%%%%
\DescribeMacro{\...prefix}
In the alternative form |\childdocforwardprefix|,
%
\begin{center}
\begin{tabular}{l}
|\input{childdoc.def}|\\
|\childdocforwardprefix[|\textit{main}|]{|\textit{prefix}|}{|\textit{dest}|}|
\end{tabular}
\end{center}
%
the destination file is determined by a pattern
depending on the current file:
To make this work, the current file must be called
`{\textit{prefix}\hspace{0.2em}\textit{suffix}}'
with \textit{prefix} matching precisely the argument.
Processing is then passed on to the file
`{\textit{dest}\hspace{0.2em}\textit{suffix}}'.
Surely, the same effect is achieved by
directly specifying the
argument `{\textit{dest}\hspace{0.2em}\textit{suffix}}'
in the first form.
However, that requires to set up a different file
for each child. With the alternative form of the command
all these files can have exactly the same content
which simplifies setting them up and maintaining them.

For example, the following file |draft.tex|
with a compilation flag |\version| as described in \secref{sec:flags}
compiles the main document as a draft:
%
\begin{center}
\begin{tabular}{l}
|\def\version{draft}|\\
|\input{childdoc.def}|\\
|\childdocforward{|\textit{main}|}|
\end{tabular}
\end{center}
%
Likewise, the following files |final|\textit{nn}|.tex|
compile the final version of the child document
|child|\textit{nn}|.tex|:
%
\begin{center}
\begin{tabular}{l}
|\def\version{final}|\\
|\input{childdoc.def}|\\
|\childdocforwardprefix{final}{child}|
\end{tabular}
\end{center}
%

Note that when several versions of a main file and/or of each child file
are to be generated, it may be convenient to set up a |Makefile| or
shell script to automatise the process.

%%%%%%%%%%%%%%%%%%%%%%%%%%%%%%%%%%%%%%%%%%%%%%%%%%%%%%%%%%%%%%%%%%%%%%%%%%%%%%%%
\subsection{Command Line Processing}
\label{sec:commandline}

The effect of redirection files can also be achieved by invoking
the \LaTeX{} compiler with a more elaborate command line.
Most conveniently this should be done as part
of a shell script or a |Makefile|.

When using \textsf{childdoc} in the main file, the following
command lines effectively perform a redirection
(note that depending on the shell being used,
backslashes may have to be doubled: `|\|' $\to$ `|\\|'):
%
\begin{center}
|... -jobname "|\textit{target}|" |\\|"|[\textit{flags}]%
|\input{childdoc.def}\childdocforward[|\textit{main}|]{|\textit{dest}|}"|
\end{center}
%
Here \textit{target} is the name of the output file,
\textit{main} is the name of the main file
and \textit{dest} is the name of the main or child file to be processed
(all filenames without extensions).
The optional argument \textit{main} can be omitted
if \textit{main} matches \textit{dest}.
Optionally, compilation \textit{flags} can be defined via |\def| commands.
This command line makes the \TeX{} engine believe
it is compiling the file \textit{target}
whose content is specified as the latter parameter.
The provided code then forwards the processing to
\textit{main} or \textit{dest} as described in \secref{sec:forward}.

%%%%%%%%%%%%%%%%%%%%%%%%%%%%%%%%%%%%%%%%%%%%%%%%%%%%%%%%%%%%%%%%%%%%%%%%%%%%%%%%
\subsection{Include by Input}
\label{sec:input}

Including child documents by |\include| has some restrictions by design.
Most notably, the content of a child document always occupies
its own set of pages; pages cannot be shared between child documents.
Usually, this behaviour makes perfect sense
because each child document contain an essential part of the document.
However, in some situations it may be desirable to compose
a document from a collection of parts
without having mandatory page breaks between then.
For this case, the package
provides a mechanism to include parts
by |\input| which can also be processed individually.
However, by construction this mechanism
requires manual handling of the content to be output.

%%%%%%%%%%%%%%%%%%%%%%%%%%%%%%%%%%%%%%%%
\DescribeMacro{\ifchilddocmanual}
The main file should be prepared as usual, see \secref{sec:include}.
However, the document body must make a distinction
between processing of an individual part and of the main document, e.g.:
%
\begin{center}
\begin{tabular}{l}
|\ifchilddocmanual|\\
|\input{\childdocname}|\\
|\||else|\\
\textit{document body with }|\input{|\textit{part}|}|\\
|\||fi|
\end{tabular}
\end{center}
%
The conditional |\ifchilddocmanual| is true whenever
a part to be included by |\input| is being compiled,
and the name of the part is stored in |\childdocname|.

%%%%%%%%%%%%%%%%%%%%%%%%%%%%%%%%%%%%%%%%
\DescribeMacro{\childdocby}
Each part to be included by |\input| should start with:
%
\begin{center}
\begin{tabular}{l}
|\input{childdoc.def}|\\
|\childdocby{|\textit{main}|}|\\
\end{tabular}
\end{center}
%
The directive |\childdocby| is similar to |\childdocof|
described in \secref{sec:include},
but the subsequent selection of content must be done manually.
To that end, both |\ifchilddoc| and |\ifchilddocmanual|
will be true upon processing of a part,
and the name of the part is stored in |\childdocname|.
Note that |\jobname| will be set to the filename of the current part
so that each part receives an individual |.aux| file
that does not interfere with the |.aux| file(s) of the main document.
This behaviour can be altered by the alternative form
|\childdocby[*]{|\textit{main}|}| (with a non-empty optional argument)
which uses the |.aux| file of the main document
by setting |\jobname| to \textit{main}.

%%%%%%%%%%%%%%%%%%%%%%%%%%%%%%%%%%%%%%%%%%%%%%%%%%%%%%%%%%%%%%%%%%%%%%%%%%%%%%%%
\subsection{Driver Development}
\label{sec:driver}

The \textsf{childdoc} mechanism can also be use for the development
of definition files such as \LaTeX{} styles or classes.
This case differs from the above setup with multiple parts
included by |\include| in that no |\includeonly| should be invoked.
This can be achieved by starting the include file
(before |\ProvidesPackage|) with:
%
\begin{center}
\begin{tabular}{l}
|\input{childdoc.def}|\\
|\childdocforward{|\textit{main}|}|\\
\end{tabular}
\end{center}
%
or alternatively with:
%
\begin{center}
\begin{tabular}{l}
|\input{childdoc.def}|\\
|\childdocby{|\textit{main}|}|\\
\end{tabular}
\end{center}
%
Both forms have slightly different effects as described above.
The main file is prepared as usual, see \secref{sec:include}.

%%%%%%%%%%%%%%%%%%%%%%%%%%%%%%%%%%%%%%%%%%%%%%%%%%%%%%%%%%%%%%%%%%%%%%%%%%%%%%%%
\subsection{Legacy Detection}
\label{sec:detection}

The directive |\childdocmain| in the main file can detect
whether the complete document or merely a child is to be compiled
even without using the directive |\childdocof|.
This method is deprecated because it is less robust
and there is no compelling reason to use it;
it is merely provided for backward compatibility
and it may be removed in future versions.

If the detection mechanism is to be used,
it is mandatory to correctly specify
the filename of the main file as the argument of |\childdocmain|:
%
\begin{center}
\begin{tabular}{l}
|\input{childdoc.def}|\\
|\childdocmain{|\textit{main}|}|\\
\end{tabular}
\end{center}
%
If |\jobname| does not match the argument \textit{main} of |\childdocmain|,
it is assumed that |\jobname| points to the child file to be compiled.
When using |\childdocmain| with the main file specified as argument,
it suffices to start a child file
with just |\input{|\textit{main}|}|
without loading of the package and using |\childdocof|.
If instead all processing is done
with the appropriate \textsf{childdoc} directives,
the argument of \textit{main} of |\childdocmain| can be empty.

An alternative version of the command line processing described
in \secref{sec:commandline} using the detection mechanism reads:
%
\begin{center}
|... -jobname "|\textit{target}|" "|[\textit{flags}]%
[|\def\jobname{|\textit{dest}|}|]|\input{|\textit{main}|}"|
\end{center}

%%%%%%%%%%%%%%%%%%%%%%%%%%%%%%%%%%%%%%%%%%%%%%%%%%%%%%%%%%%%%%%%%%%%%%%%%%%%%%%%
\subsection{Manual Code}
\label{sec:manual}

In case one cannot be certain whether the definitions file |childdoc.def|
is installed on the target \TeX{} distribution
and one prefers not to ship it,
it is conceivable to paste a few relevant commands into the sources.

To that end, drop all statements |\input{childdoc.def}|
and perform the replacements as outlined below.
Instead of |\childdocmain{|\textit{main}|}| add the following code
to the top of the main file:
%
\begin{center}
\begin{tabular}{l}
|\||ifdefined\childdocname\endinput\||fi\newif\ifchilddoc|\\
|\edef\childdocname{\scantokens\expandafter{\jobname\noexpand}}|\\
|\def\childdocmain{|\textit{main}|}\||ifx\childdocmain\childdocname\||else|\\
|\childdoctrue\includeonly{\childdocname}\let\jobname\childdocmain\||fi|\\
\end{tabular}
\end{center}
%
Instead of |\childdocof{|\textit{main}|}| just include the main file
at the top of each child file:
%
\begin{center}
|\input{|\textit{main}|}|
\end{center}
%
A simple redirection |\childdocforward{|\textit{dest}|}| is achieved by:
%
\begin{center}
|\def\jobname{|\textit{dest}|}\input{\jobname}|
\end{center}
%
The redirection with prefix
|\childdocforwardprefix[|\textit{prefix}|]{|\textit{dest}|}|
is accomplished by:
%
\begin{center}
\begin{tabular}{l}
|{\edef\jobname{\scantokens\expandafter{\jobname\noexpand}}|\\
|\def\redirectjob |\textit{prefix}|#1~~~{\gdef\jobname{|\textit{dest}|#1}}|\\
|\expandafter\redirectjob\jobname~~~}\input{\jobname}|
\end{tabular}
\end{center}

In an alternative approach,
child documents can be compiled by a specific command line
without additional code or specific definitions:
%
\begin{center}
|... -jobname "|\textit{target}|" "|[\textit{flags}]%
|\includeonly{|\textit{dest}|}\input{|\textit{main}|}"|
\end{center}
%

%%%%%%%%%%%%%%%%%%%%%%%%%%%%%%%%%%%%%%%%%%%%%%%%%%%%%%%%%%%%%%%%%%%%%%%%%%%%%%%%
%%%%%%%%%%%%%%%%%%%%%%%%%%%%%%%%%%%%%%%%%%%%%%%%%%%%%%%%%%%%%%%%%%%%%%%%%%%%%%%%
\section{Information}

%%%%%%%%%%%%%%%%%%%%%%%%%%%%%%%%%%%%%%%%%%%%%%%%%%%%%%%%%%%%%%%%%%%%%%%%%%%%%%%%
\subsection{Copyright}

Copyright \copyright{} 2017--2018 Niklas Beisert

This work may be distributed and/or modified under the
conditions of the \LaTeX{} Project Public License, either version 1.3
of this license or (at your option) any later version.
The latest version of this license is in
  \url{http://www.latex-project.org/lppl.txt}
and version 1.3 or later is part of all distributions of \LaTeX{}
version 2005/12/01 or later.

This work has the LPPL maintenance status `maintained'.

The Current Maintainer of this work is Niklas Beisert.

This work consists of the files |README.txt|, |childdoc.ins| and |childdoc.dtx|
as well as the derived files |childdoc.def|, |cdocsamp.tex|
with |cdocsch1.tex|, |cdocsch2.tex|, |cdocspt3.tex|, |cdocspt4.tex|,
|cdocsdrf.tex|, |cdocsfn1.tex|, |cdocsfn2.tex|
as well as |childdoc.pdf|.

%%%%%%%%%%%%%%%%%%%%%%%%%%%%%%%%%%%%%%%%%%%%%%%%%%%%%%%%%%%%%%%%%%%%%%%%%%%%%%%%
\subsection{Files and Installation}

The package consists of the files:
%
\begin{center}
\begin{tabular}{ll}
    |README.txt|   & readme file \\
    |childdoc.ins| & installation file \\
    |childdoc.dtx| & source file \\
    |childdoc.def| & definition file \\
    |cdocsamp.tex| & sample main file \\
    |cdocsch1.tex| & sample include file \\
    |cdocsch2.tex| & sample include file \\
    |cdocspt3.tex| & sample part file \\
    |cdocspt4.tex| & sample part file \\
    |cdocsdrf.tex| & sample redirection file \\
    |cdocsfn1.tex| & sample redirection file \\
    |cdocsfn2.tex| & sample redirection file \\
    |childdoc.pdf| & manual
\end{tabular}
\end{center}
%
The distribution consists of the files
|README.txt|, |childdoc.ins| and |childdoc.dtx|.
%
\begin{itemize}
\item
Run (pdf)\LaTeX{} on |childdoc.dtx|
to compile the manual |childdoc.pdf| (this file).
\item
Run \LaTeX{} on |childdoc.ins| to create the definitions file |childdoc.def|
and the sample |cdocsamp.tex| with include files
|cdocsch1.tex|, |cdocsch2.tex|, |cdocspt3.tex|, |cdocspt4.tex|,
|cdocsdrf.tex|, |cdocsfn1.tex|, |cdocsfn2.tex|.
Then copy the file |childdoc.def| to an appropriate directory of your \LaTeX{}
distribution, e.g.\ \textit{texmf-root}|/tex/latex/childdoc|.
\end{itemize}

%%%%%%%%%%%%%%%%%%%%%%%%%%%%%%%%%%%%%%%%%%%%%%%%%%%%%%%%%%%%%%%%%%%%%%%%%%%%%%%%
\subsection{Related CTAN Packages}

There are several other packages which offer a similar functionality:
%
\begin{itemize}
\item
The packages
\href{http://ctan.org/pkg/docmute}{\textsf{docmute}},
\href{http://ctan.org/pkg/includex}{\textsf{includex}} and
\href{http://ctan.org/pkg/standalone}{\textsf{standalone}}
provide commands to include only the document body of
a child file thus allowing both files to be compiled individually.
\item
The packages \href{http://ctan.org/pkg/subdocs}{\textsf{subdocs}}
and \href{http://ctan.org/pkg/subfiles}{\textsf{subfiles}}
provide structures in which the main and child documents can be
encapsulated and allowing them to be compiled individually.
The inclusion mechanism is different from the conventional |\include|.
\item
The package \href{http://ctan.org/pkg/combine}{\textsf{combine}}
is an elaborate solution to combine several documents into one.
\end{itemize}
%
See also the CTAN topic \href{http://ctan.org/topic/subdocs}{\textsf{subdocs}}
for further related packages.
The present package differs from the above solutions in that
a document structure constructed with the conventional |\include| mechanism
just needs two extra commands at the top of every file
such that all constituent files can be compiled individually.

%%%%%%%%%%%%%%%%%%%%%%%%%%%%%%%%%%%%%%%%%%%%%%%%%%%%%%%%%%%%%%%%%%%%%%%%%%%%%%%%
%\subsection{Feature Suggestions}
%
%The following is a list of features which may be useful for future
%versions of this package:
%%
%\begin{itemize}
%\item
%\ldots
%\end{itemize}

%%%%%%%%%%%%%%%%%%%%%%%%%%%%%%%%%%%%%%%%%%%%%%%%%%%%%%%%%%%%%%%%%%%%%%%%%%%%%%%%
\subsection{Revision History}

%%%%%%%%%%%%%%%%%%%%%%%%%%%%%%%%%%%%%%%%
\paragraph{v2.0:} 2018/12/30

\begin{itemize}
\item
immediate forward processing
\item
added |\childdocby| mechanism
\item
manual restructured
\end{itemize}

%%%%%%%%%%%%%%%%%%%%%%%%%%%%%%%%%%%%%%%%
\paragraph{v1.6:} 2018/01/17

\begin{itemize}
\item
application for development of include files
\item
corrections to manual
\end{itemize}

%%%%%%%%%%%%%%%%%%%%%%%%%%%%%%%%%%%%%%%%
\paragraph{v1.5:} 2017/05/21

\begin{itemize}
\item
more complete structuring introduced
\item
|\childdocof| introduced
\item
|\childdoc| renamed to |\childdocmain|
\item
|\childredirect| renamed to |\childdocforward| and |\childdocforwardprefix|
and functionality expanded
\end{itemize}

%%%%%%%%%%%%%%%%%%%%%%%%%%%%%%%%%%%%%%%%
\paragraph{v1.0:} 2017/04/27

\begin{itemize}
\item
manual and install package
\item
first version published on CTAN
\end{itemize}

%%%%%%%%%%%%%%%%%%%%%%%%%%%%%%%%%%%%%%%%
\paragraph{v0.6:} 2017/04/26

\begin{itemize}
\item
redirection mechanism added
\end{itemize}

%%%%%%%%%%%%%%%%%%%%%%%%%%%%%%%%%%%%%%%%
\paragraph{v0.5:} 2017/04/26

\begin{itemize}
\item
functionality in definition file
\end{itemize}


%%%%%%%%%%%%%%%%%%%%%%%%%%%%%%%%%%%%%%%%%%%%%%%%%%%%%%%%%%%%%%%%%%%%%%%%%%%%%%%%
%%%%%%%%%%%%%%%%%%%%%%%%%%%%%%%%%%%%%%%%%%%%%%%%%%%%%%%%%%%%%%%%%%%%%%%%%%%%%%%%
%%%%%%%%%%%%%%%%%%%%%%%%%%%%%%%%%%%%%%%%%%%%%%%%%%%%%%%%%%%%%%%%%%%%%%%%%%%%%%%%
\appendix

\settowidth\MacroIndent{\rmfamily\scriptsize 000\ }

 \DocInput{childdoc.dtx}

\end{document}
%</driver>
% \fi
%
% %%%%%%%%%%%%%%%%%%%%%%%%%%%%%%%%%%%%%%%%%%%%%%%%%%%%%%%%%%%%%%%%%%%%%%%%%%%%%%
% %%%%%%%%%%%%%%%%%%%%%%%%%%%%%%%%%%%%%%%%%%%%%%%%%%%%%%%%%%%%%%%%%%%%%%%%%%%%%%
% \section{Sample}
%\iffalse
%<*samplemain>
%\fi
%
% The following presents a sample document
% with two chapters, two parts, a title page,
% a compile flag as well as three forwarding files to set the flag.
% It consists of eight |.tex| files:
% \begin{center}
% \begin{tabular}{ll}
% |cdocsamp.tex|&main file\\
% |cdocsch1.tex|&include file for chapter 1\\
% |cdocsch2.tex|&include file for chapter 2\\
% |cdocspt3.tex|&include file for part 3\\
% |cdocspt4.tex|&include file for part 4\\
% |cdocsdrf.tex|&forwarding file for main file in draft mode\\
% |cdocsfi1.tex|&forwarding file for final version of chapter 1\\
% |cdocsfi2.tex|&forwarding file for final version of chapter 2\\
% \end{tabular}
% \end{center}
% Each of the eight files can be compiled directly by the \LaTeX{} compiler.
%
% %%%%%%%%%%%%%%%%%%%%%%%%%%%%%%%%%%%%%%
% \paragraph{Main File.}
%
% The main file is called |cdocsamp.tex|.
%
% Load the \textsf{childdoc} definitions and
% declare the filename for the main document:
%    \begin{macrocode}
\input{childdoc.def}
\childdocmain{}
%    \end{macrocode}

% Optional override for |\version| flag:
%    \begin{macrocode}
%%\ifchilddoc\else\providecommand{\version}{draft}\fi
%    \end{macrocode}

% Define the default values for the |\version| flag
% (|final| for the main file and |draft| for childs):
%    \begin{macrocode}
\ifchilddoc
\providecommand{\version}{draft}
\else
\providecommand{\version}{final}
\fi
%    \end{macrocode}

% Load the standard document class:
%    \begin{macrocode}
\documentclass[12pt]{article}
%    \end{macrocode}

% Start the document body:
%    \begin{macrocode}
\begin{document}
%    \end{macrocode}

% Declare a title page.
% Print title, part of document being processed and version flag:
%    \begin{macrocode}
\addtocounter{page}{-1}
\begin{center}
{\LARGE\bfseries{}childdoc example\par}
\vspace{1cm}
\ifchilddoc
\ifchilddocmanual part\else chapter\fi:
`\childdocname' of `\childdocjob'\par
\else
main document: `\childdocjob'\par
\fi
version: \version\par
\end{center}
\newpage
%    \end{macrocode}

% Manually include selected file,
% otherwise process as usual:
%    \begin{macrocode}
\ifchilddocmanual
\section*{part `\childdocname'}
\input{\childdocname}
\else
%    \end{macrocode}

% Include the two chapters:
%    \begin{macrocode}
\include{cdocsch1}
\include{cdocsch2}
%    \end{macrocode}

% Include the two parts unless only chapters should be displayed:
%    \begin{macrocode}
\ifchilddoc\else
\section{part three}
\input{cdocspt3}
\section{part four}
\input{cdocspt4}
\fi
%    \end{macrocode}

% Process as usual until here:
%    \begin{macrocode}
\fi
%    \end{macrocode}

% End of document body:
%    \begin{macrocode}
\end{document}
%    \end{macrocode}
%\iffalse
%</samplemain>
%\fi
%
% %%%%%%%%%%%%%%%%%%%%%%%%%%%%%%%%%%%%%%
% \paragraph{Chapter Include Files.}
%
% The include files are called |cdocsch1.tex| and |cdocsch2.tex|.
%
%\iffalse
%<*samplechap1|samplechap2>
%\fi

% Optional override for |\version| flag:
%    \begin{macrocode}
%%\providecommand{\version}{final}
%    \end{macrocode}

% Include the main document:
%    \begin{macrocode}
\input{childdoc.def}
\childdocof{cdocsamp}
%    \end{macrocode}

%\iffalse
%</samplechap1|samplechap2>
%\fi
%
%\iffalse
%<*samplechap1>
%\fi
% Some text for chapter 1:
%    \begin{macrocode}
\section{one}
some text in chapter one
%    \end{macrocode}

%\iffalse
%</samplechap1>
%\fi
% Some text for chapter 2:
%\iffalse
%<*samplechap2>
%\fi
%    \begin{macrocode}
\section{two}
more text in chapter two
%    \end{macrocode}

%\iffalse
%</samplechap2>
%\fi
%
% %%%%%%%%%%%%%%%%%%%%%%%%%%%%%%%%%%%%%%
% \paragraph{Part Include Files.}
%
% The include files are called |cdocspt3.tex| and |cdocspt4.tex|.
%
%\iffalse
%<*samplepart3|samplepart4>
%\fi

% Optional override for |\version| flag:
%    \begin{macrocode}
%%\providecommand{\version}{final}
%    \end{macrocode}

% Include the main document:
%    \begin{macrocode}
\input{childdoc.def}
\childdocby{cdocsamp}
%    \end{macrocode}

%\iffalse
%</samplepart3|samplepart4>
%\fi
%
%\iffalse
%<*samplepart3>
%\fi
% Some text for part 3:
%    \begin{macrocode}
some text in part three
%    \end{macrocode}

%\iffalse
%</samplepart3>
%\fi
% Some text for part 4:
%\iffalse
%<*samplepart4>
%\fi
%    \begin{macrocode}
more text in part four
%    \end{macrocode}

%\iffalse
%</samplepart4>
%\fi
%
% %%%%%%%%%%%%%%%%%%%%%%%%%%%%%%%%%%%%%%
% \paragraph{Forwarding for a Complete Draft.}
%
% The following forwarding file |cdocsdrf.tex|
% compiles the main document in draft mode:
%\iffalse
%<*sampledraft>
%\fi
%    \begin{macrocode}
\def\version{draft}
\input{childdoc.def}
\childdocforward{cdocsamp}
%    \end{macrocode}

%\iffalse
%</sampledraft>
%\fi
%
% %%%%%%%%%%%%%%%%%%%%%%%%%%%%%%%%%%%%%%
% \paragraph{Forwarding for Final Version of the Chapters.}
%
% The following forwarding files |cdocsfn1.tex| and |cdocsfn2.tex|
% (with identical content)
% compile the final versions of the child documents
% |cdocsch1.tex| and |cdocsch2.tex|, respectively:
%\iffalse
%<*samplefinal>
%\fi
%    \begin{macrocode}
\def\version{final}
\input{childdoc.def}
\childdocforwardprefix[cdocsamp]{cdocsfn}{cdocsch}
%    \end{macrocode}

%\iffalse
%</samplefinal>
%\fi
%
% %%%%%%%%%%%%%%%%%%%%%%%%%%%%%%%%%%%%%%
% \paragraph{Command Line Processing.}
%
% The following three command lines generate the output files
% |cdocscld|, |cdocscl1| and |cdocscl2|
% which should be identical to
% |cdocsdrf|, |cdocsch1| and |cdocsfn2|, respectively:
% \begin{center}
% \begin{tabular}{l}
% |latex -jobname cdocscld \|\\
% |  "\def\version{draft}\input{childdoc.def}\childdocforward{cdocsamp}"|\\
% |latex -jobname cdocscl1 \|\\
% |  "\input{childdoc.def}\childdocforward[cdocsamp]{cdocsch1}"|\\
% |latex -jobname cdocscl2 \|\\
% |  "\def\version{final}\input{childdoc.def}\childdocforward{cdocsch2}"|
% \end{tabular}
% \end{center}
% Note that the trailing backslash on each first line
% merely continues the input to the second line
% (for convenient cut ant paste).
% Furthermore, the command |latex| can be replaced by any
% of its alternative versions such as |pdflatex|.
%
% %%%%%%%%%%%%%%%%%%%%%%%%%%%%%%%%%%%%%%%%%%%%%%%%%%%%%%%%%%%%%%%%%%%%%%%%%%%%%%
% %%%%%%%%%%%%%%%%%%%%%%%%%%%%%%%%%%%%%%%%%%%%%%%%%%%%%%%%%%%%%%%%%%%%%%%%%%%%%%
% \section{Implementation}
%\iffalse
%<*package>
%\fi
%
% This section describes the definitions file |childdoc.def|.

% The definitions cannot be loaded using |\usepackage| or |\RequirePackage|
% which has a mechanism to prevent loading a style file more than once.
% When loading the definitions by means of |\input|
% multiple instances have to be prevented manually:
%\iffalse
%This code needs to be before the `\ProvidesFile' directive
%which is defined at the beginning of this file.
%Therefore it is also placed there and commented out here.
%</package>
%<*discard>
%\fi
%    \begin{macrocode}
\ifdefined\childdocmain\endinput\fi
%    \end{macrocode}
%\iffalse
%</discard>
%<*package>
%\fi
%
% \macro{\ifchilddoc}
% \macro{\ifchilddocmanual}
% The conditional |\ifchilddoc| tells whether a
% child (true) or main (false) document is being compiled.
% The conditional |\ifchilddocmanual| tells whether
% the |\includeonly| mechanism is used (false) or
% the selection of child files must be performed manually (true).
% The definitions initialise to false:
%    \begin{macrocode}
\newif\ifchilddoc
\newif\ifchilddocmanual
%    \end{macrocode}

% \macro{\childdocname}
% \macro{\childdocjob}
% The macro |\childdocname| stores the name of the main document
% to be compiled. The macro |\childdocjob| stores the name of
% the document on which the \LaTeX{} compiler was originally invoked.
% The content of |\jobname| cannot be compared
% to filenames specified in the source due to different catcodes.
% The following code rescans |\jobname|, stores the result
% in |\childdocname| and saves a copy in |\childdocjob|:
%    \begin{macrocode}
\edef\childdocname{\scantokens\expandafter{\jobname\noexpand}}
\let\childdocjob\childdocname
%    \end{macrocode}

% \macro{\childdocdisable}
% The macro |\childdocdisable| prevents the main file
% from being processed more than once.
% At this stage, the main document command |\childdocmain|
% is assumed to be called once again where it should do nothing.
% Any subsequent call to it should prevent
% a secondary processing of the main document
% It overwrites the forwarding commands
% |\childdocof| and |\childdocforward|
% with empty macros to prevent further inclusions of the main document:
%    \begin{macrocode}
\newcommand{\childdocdisable}
{
  \renewcommand{\childdocmain}[1]{\renewcommand{\childdocmain}[1]{\endinput}}
  \renewcommand{\childdocof}[1]{}
  \renewcommand{\childdocby}[2][]{}
  \renewcommand{\childdocforward}[2][]{}
  \renewcommand{\childdocdisable}{}
}
%    \end{macrocode}

% \macro{\childdocmain}
% The macro |\childdocmain| is to be called at the top of the main file
% with nothing or the main filename (without extension) as argument.
% First, it breaks loops.
% If the argument is not empty and does not match |\childdocname|
% (which is set by the first inclusion of |childdoc.def|),
% |\ifchilddoc| is set to true, |\includeonly| is applied to the child file
% and |\jobname| is set to the main file
% (for proper handling of |.aux| files):
%    \begin{macrocode}
\newcommand{\childdocmain}[1]
{
  \childdocdisable\childdocmain{}
  \if?#1?\else
    \begingroup
      \def\childdoctmp{#1}
      \ifx\childdoctmp\childdocname
        \def\childdoctmp{}
      \else
        \def\childdoctmp
        {
          \childdoctrue
          \includeonly{\childdocname}
          \def\childdocjob{#1}
          \def\jobname{#1}
        }
      \fi
      \expandafter
    \endgroup
    \childdoctmp
  \fi
}
%    \end{macrocode}

% \macro{\childdocof}
% The command |\childdocof| redirects
% compilation to the main file |#1|.
%    \begin{macrocode}
\newcommand{\childdocof}[1]
{
  \childdocdisable
  \childdoctrue
  \includeonly{\childdocname}
  \def\jobname{#1}
  \def\childdocjob{#1}
  \input{#1}
}
%    \end{macrocode}

% \macro{\childdocby}
% The command |\childdocby| ....
%    \begin{macrocode}
\newcommand{\childdocby}[2][]
{
  \childdocdisable
  \childdoctrue
  \childdocmanualtrue
  \if?#1?\else
    \def\jobname{#2}
  \fi
  \def\childdocjob{#2}
  \input{#2}
  \endinput
}
%    \end{macrocode}

% \macro{\childdocforward}
% The command |\childdocforward| redirects
% compilation to the main file or
% (if the optional argument is given) a child file.
% Parameters are set as if the main file
% or a child file starting with |\childdocof| was compiled.
% Then compilation is handed over to the main file:
%    \begin{macrocode}
\newcommand{\childdocforward}[2][]
{
  \begingroup
    \if?#1?
      \def\childdoctmp
      {
        \def\childdocname{#2}
        \def\childdocjob{#2}
        \def\jobname{#2}
        \input{#2}
        \endinput
      }
    \else
      \def\childdoctmp
      {
        \childdocdisable
        \def\childdocname{#2}
        \childdoctrue
        \includeonly{#2}
        \def\childdocjob{#1}
        \def\jobname{#1}
        \input{#1}
        \endinput
      }
    \fi
    \expandafter
  \endgroup
  \childdoctmp
}
%    \end{macrocode}

% \macro{\childdocforwardprefix}
% The command |\childdocforwardprefix| redirects
% compilation to the main or a child file by means of a pattern.
% The prefix |#1| in the current filename is replaced by |#2|
% and the suffix of the current filename is kept
% (it is assumed that the filename does not contain the substring `|~~~|'
% which is used as a delimiter).
% Compilation is handed over to the new file by |\childdocforward|:
%    \begin{macrocode}
\newcommand{\childdocforwardprefix}[3][]
{
  \begingroup
    \def\childdocextract #2##1~~~{\def\childdoctmp{\childdocforward[#1]{#3##1}}}
    \expandafter\childdocextract\childdocname~~~
    \expandafter
  \endgroup
  \childdoctmp
}
%    \end{macrocode}

% \macro{\childdoc}
% The deprecated macro |\childdoc| is a legacy version of |\childdocmain|:
%    \begin{macrocode}
\newcommand{\childdoc}{\childdocmain}
%    \end{macrocode}

% \macro{\childdocredirect}
% The deprecated macro |\childdocredirect| is a legacy version
% of |\childdocforward| and |\childdocforwardprefix|:
%    \begin{macrocode}
\newcommand{\childdocredirect}[2][]
{
  \begingroup
    \if?#1?
      \def\childdoctmp{\childdocforward{#2}}
    \else
      \def\childdoctmp{\childdocforwardprefix{#1}{#2}}
    \fi
    \expandafter
  \endgroup
  \childdoctmp
}
%    \end{macrocode}

%\iffalse
%</package>
%\fi
%
\endinput
|\\
|\childdocforward[|\textit{main}|]{|\textit{dest}|}|\\
\end{tabular}
\end{center}
%
The argument \textit{dest} is the destination file
(without extension).
It should be the main file or one of the child files.
Note that further \textsf{childdoc} directives
such as |\childdocof| and |\childdocforward|
in the indicated file will be processed in this form.
The optional argument \textit{main}
passes on directly to the main file \textit{main}
while pretending to compile the child \textit{dest}.
This form behaves as if \textit{dest}
issues |\childdocof{|\textit{main}|}| right away,
and no further \textsf{childdoc} directives will be processed.

%%%%%%%%%%%%%%%%%%%%%%%%%%%%%%%%%%%%%%%%
\DescribeMacro{\...prefix}
In the alternative form |\childdocforwardprefix|,
%
\begin{center}
\begin{tabular}{l}
|% \iffalse
%
% childdoc.dtx Copyright (C) 2017-2018 Niklas Beisert
%
% This work may be distributed and/or modified under the
% conditions of the LaTeX Project Public License, either version 1.3
% of this license or (at your option) any later version.
% The latest version of this license is in
%   http://www.latex-project.org/lppl.txt
% and version 1.3 or later is part of all distributions of LaTeX
% version 2005/12/01 or later.
%
% This work has the LPPL maintenance status `maintained'.
%
% The Current Maintainer of this work is Niklas Beisert.
%
% This work consists of the files childdoc.dtx and childdoc.ins
% and the derived files childdoc.def and cdocsamp.tex with
% cdocsch1.tex, cdocsch2.tex, cdocsdrf.tex, cdocsfn1.tex, cdocsfn2.tex.
%
%<package>\ifdefined\childdocmain\endinput\fi
%<package>\ProvidesFile{childdoc.def}[2018/12/30 v2.0 child document driver]
%<samplemain>\ProvidesFile{cdocsamp.tex}[2018/12/30 v2.0 sample for childdoc]
%<*driver>
%\ProvidesFile{childdoc.drv}[2018/12/30 v2.0 childdoc reference manual file]
\PassOptionsToClass{10pt,a4paper}{article}
\documentclass{ltxdoc}

\usepackage[margin=35mm]{geometry}
\usepackage{hyperref}
\usepackage{hyperxmp}
\usepackage[usenames]{color}

\hypersetup{colorlinks=true}
\hypersetup{pdfstartview=FitH}
\hypersetup{pdfpagemode=UseNone}
\hypersetup{pdfsource={}}
\hypersetup{pdflang={en-UK}}
\hypersetup{pdfcopyright={Copyright 2017-2018 Niklas Beisert.
  This work may be distributed and/or modified under the
  conditions of the LaTeX Project Public License, either version 1.3
  of this license or (at your option) any later version.}}
\hypersetup{pdflicenseurl={http://www.latex-project.org/lppl.txt}}
\hypersetup{pdfcontactaddress={ETH Zurich, ITP, HIT K,
  Wolfgang-Pauli-Strasse 27}}
\hypersetup{pdfcontactpostcode={8093}}
\hypersetup{pdfcontactcity={Zurich}}
\hypersetup{pdfcontactcountry={Switzerland}}
\hypersetup{pdfcontactemail={nbeisert@itp.phys.ethz.ch}}
\hypersetup{pdfcontacturl={http://people.phys.ethz.ch/\xmptilde nbeisert/}}

\newcommand{\secref}[1]{\hyperref[#1]{section \ref*{#1}}}

\parskip1ex
\parindent0pt
\let\olditemize\itemize
\def\itemize{\olditemize\parskip0pt}

\begin{document}

\title{The \textsf{childdoc} Package}
\hypersetup{pdftitle={The childdoc Package}}
\author{Niklas Beisert\\[2ex]
  Institut f\"ur Theoretische Physik\\
  Eidgen\"ossische Technische Hochschule Z\"urich\\
  Wolfgang-Pauli-Strasse 27, 8093 Z\"urich, Switzerland\\[1ex]
  \href{mailto:nbeisert@itp.phys.ethz.ch}
  {\texttt{nbeisert@itp.phys.ethz.ch}}}
\hypersetup{pdfauthor={Niklas Beisert}}
\hypersetup{pdfsubject={Manual for the LaTeX2e Package childdoc}}
\date{30 December 2018, \textsf{v2.0}}
\maketitle

\begin{abstract}\noindent
\textsf{childdoc} is a \LaTeXe{} package
that enables the direct compilation
of document sections included by |\include|
to individual files.
\end{abstract}

\begingroup
\parskip0ex
\tableofcontents
\endgroup

%%%%%%%%%%%%%%%%%%%%%%%%%%%%%%%%%%%%%%%%%%%%%%%%%%%%%%%%%%%%%%%%%%%%%%%%%%%%%%%%
%%%%%%%%%%%%%%%%%%%%%%%%%%%%%%%%%%%%%%%%%%%%%%%%%%%%%%%%%%%%%%%%%%%%%%%%%%%%%%%%
\section{Introduction}

\LaTeX{} provides a mechanism to structure a large document (such as a book)
into a main file and several child files (containing the chapters)
using the |\include| command.
This mechanism is beneficial for documents
which span hundreds of pages in order to
make the source file(s) more manageable.
Moreover, compilation can be restricted to
selected child files by means of the |\includeonly| command.
The latter feature can be used to reduce the compilation time while editing
(this was significantly more useful in the earlier days of \LaTeX{})
or to generate a smaller document which is easier to navigate.
Another application of |\includeonly| is to generate
documents consisting of selected parts of the complete document.

However, there are a few drawbacks of the plain |\include| mechanism:
\begin{itemize}
\item
The child files cannot be compiled on their own,
they can only be compiled via the main file.
A naive editing environment
(such as a text editor with an option
to have the current file processed by \LaTeX)
may require one to switch to the main file before compiling;
attempting to compile the child file produces errors.
\item
The main file must be modified (each time)
to adjust the |\includeonly| command
to the present needs. This easily leaves the main file in a messy state.
\item
The generated document will always carry the filename
of the main document. This is inconvenient if
several child files are to be compiled and
to be kept for distribution.
\end{itemize}

The present package provides a simple interface
to make child files individually compilable by \LaTeX{}.
Compiling a child file then has the same effect as compiling
the main file with an |\includeonly| command
to select the appropriate child.
Moreover the generated document will carry the name of the child
rather than the main file.
This resolves all three above issues.

This feature is meant to make the editing of books,
thesis documents and lecture notes somewhat more convenient.
However, the package can also be used efficiently for
composing a series of documents (such as exercise sheets)
which are typically distributed individually.
It then assists the author in generating the individual documents
(potentially in different versions)
as well as a document containing the collected series.
Another application is in developing style files
or other kinds of included material
where compilation of the style file could redirect
to a sample or test file.

%%%%%%%%%%%%%%%%%%%%%%%%%%%%%%%%%%%%%%%%%%%%%%%%%%%%%%%%%%%%%%%%%%%%%%%%%%%%%%%%
%%%%%%%%%%%%%%%%%%%%%%%%%%%%%%%%%%%%%%%%%%%%%%%%%%%%%%%%%%%%%%%%%%%%%%%%%%%%%%%%
\section{Usage}

First of all, the package \textsf{childdoc} is \emph{not} a standard
\LaTeXe{} |.sty| style file! Therefore it needs to be invoked in
a non-standard way.

%%%%%%%%%%%%%%%%%%%%%%%%%%%%%%%%%%%%%%%%%%%%%%%%%%%%%%%%%%%%%%%%%%%%%%%%%%%%%%%%
\subsection{Included Files}
\label{sec:include}

%%%%%%%%%%%%%%%%%%%%%%%%%%%%%%%%%%%%%%%%
\DescribeMacro{\childdocmain}
To use the package, add the commands
\begin{center}
\begin{tabular}{l}
|\input{childdoc.def}|\\
|\childdocmain{}|\\
\end{tabular}
\end{center}
at the very top of the main \LaTeX{} file,
in particular \emph{before} the |\documentclass| statement!
The argument of |\childdocmain| should be left empty
(but it must be present).

%%%%%%%%%%%%%%%%%%%%%%%%%%%%%%%%%%%%%%%%
\DescribeMacro{\childdocof}
Furthermore, add the commands
\begin{center}
\begin{tabular}{l}
|\input{childdoc.def}|\\
|\childdocof{|\textit{main}|}|\\
\end{tabular}
\end{center}
at the top of every child file \textit{child}
which is included by |\include{|\textit{child}|}|
from within the main file
(or at least for those files to be compiled individually).
The argument \textit{main} must be the filename of the main file.

There are a couple of
considerations in setting up the main and child documents:

%%%%%%%%%%%%%%%%%%%%%%%%%%%%%%%%%%%%%%%%
\paragraph{Restrictions.}

Please note the following restrictions:
\begin{itemize}
\item
|\childdocmain| must be called with one argument \textit{main}
to ensure compatibility with earlier version of the package.
It must either be empty (|\childdocmain{}|)
or precisely match the filename of the main file in which it is specified.
See \secref{sec:detection} for further information.
\item
The filename \textit{main} must be specified without the |.tex| extension.
\item
The filename \textit{main} is case sensitive
(even in case-insensitive file systems)
due to internal string comparison.
\item
The argument \textit{main} should be fully expanded, it cannot be a macro.
\item
Subdirectories and special characters should be avoided in filenames.
\item
The command |\childdocmain{|\textit{main}|}| must be followed by a whitespace.
It should not be followed immediately by another command
or by a comment mark `|%|'.
This is because the \TeX{} parser reads the token immediately following
the argument of |\childdocmain| and puts it
at the beginning of every child section;
however, a white\-space is ignored.
\end{itemize}

%%%%%%%%%%%%%%%%%%%%%%%%%%%%%%%%%%%%%%%%
\paragraph{Content of Main File.}

It is advisable to place all content in the child files included by |\include|.
Any output contained in the main file will appear in all child documents
unless suppressed manually;
it cannot be suppressed automatically by the |\includeonly| directive
and thus should normally be avoided.
A method to include some content in the main file
by means of conditional processing is described in \secref{sec:conditional}.

%%%%%%%%%%%%%%%%%%%%%%%%%%%%%%%%%%%%%%%%
\paragraph{Page Numbering.}

When only a part of the document is compiled,
the appropriate numbering of pages
(as well as other status parameters)
is determined from the |.aux| files.
The latter contain information from previous passes.
However this information needs to propagate through
all intermediate child documents.
Therefore the page numbering in child documents may well
be inconsistent until the complete document is compiled at least once.

A useful (if unconventional) way to always ensure a consistent
page numbering is to restart the numbering in each child document
and denote the pages by `\textit{child}|.|\textit{page}'
where \textit{child} represents the chapter/section number of the child file.
This can be achieved by the command
|\numberwithin{page}{|\textit{child}|}|
of the \textsf{amsmath} package
where \textit{child} can be |chapter| or |section|
depending on the chosen structuring.
Alternatively, one can modify the macro |\thepage| appropriately
and reset the counter |page| at the start of each child file.

%%%%%%%%%%%%%%%%%%%%%%%%%%%%%%%%%%%%%%%%%%%%%%%%%%%%%%%%%%%%%%%%%%%%%%%%%%%%%%%%
\subsection{Conditional Processing}
\label{sec:conditional}

The package provides a mechanism to compile different versions
of a document. To customise the versions further some conditional processing
can come in handy to distinguish which version is being compiled.
The package provides two macros to describe the compilation context:

%%%%%%%%%%%%%%%%%%%%%%%%%%%%%%%%%%%%%%%%
\DescribeMacro{\ifchilddoc}
The conditional |\ifchilddoc| distinguishes between the compilation of
child documents and the main document:
%
\begin{center}
|\ifchilddoc |\textit{child-code}| |[|\||else |\textit{main-code}]| \||fi|
\end{center}

%%%%%%%%%%%%%%%%%%%%%%%%%%%%%%%%%%%%%%%%
\DescribeMacro{\childdocname}
\DescribeMacro{\childdocjob}
The macro |\childdocname| contains the filename (without extension)
of the main or child file being processed.
Note that |\childdocjob| will always contain the name of the main file.

%%%%%%%%%%%%%%%%%%%%%%%%%%%%%%%%%%%%%%%%
\paragraph{Title Page.}

Conditional processing can be used to include a title or banner page
in the main document when proper precautions are taken.
Importantly, the code in the main file should ensure that the page counter
(as well as other status parameters which are stored in the |.aux| files)
takes the same value after the conditional processing.
Otherwise the page numbers may take divergent values
depending on which part is compiled.

For example, a title page could be declared by:
%
\begin{center}
\begin{tabular}{l}
|\ifchilddoc\||else|\\
|\addtocounter{page}{-1}|\\
\textit{code for title page}\\
|\newpage|\\
|\||fi|
\end{tabular}
\end{center}
%
A banner page for the child documents can be generated by:
%
\begin{center}
\begin{tabular}{l}
|\ifchilddoc|\\
|\addtocounter{page}{-1}|\\
\textit{code for banner page}\\
|\newpage|\\
|\||fi|
\end{tabular}
\end{center}
%
Here one could write a message such as:
\begin{center}
|This is the part \childdocname{} of \childdocjob{}.|
\end{center}

%%%%%%%%%%%%%%%%%%%%%%%%%%%%%%%%%%%%%%%%%%%%%%%%%%%%%%%%%%%%%%%%%%%%%%%%%%%%%%%%
\subsection{Flags}
\label{sec:flags}

The package makes it easy to generate different versions
of the main or child documents.
To this end compilation flags can be defined
and assigned different default values.
They will be particularly useful in conjunction
with the forwarding mechanism described in \secref{sec:forward}.

For example, it may be useful to have a flag |\version|
which can be set to |draft| or |final|.
The document source will contain some conditional code
depending on the value of |\version|.
Suppose further, the flag should default to |final| for the main file
and to |draft| for child files
which is a natural assignment for editing the document.
This is achieved by placing the following code
in the preamble of the main document
(below the |\childdocmain| directive):
%
\begin{center}
\begin{tabular}{l}
|\ifchilddoc|\\
|\providecommand{\version}{draft}|\\
|\||else|\\
|\providecommand{\version}{final}|\\
|\||fi|
\end{tabular}
\end{center}
%
The definition by |\providecommand| makes sure
that previous definitions are not overwritten.
Further statements |\providecommand{\version}{...}|
can thus be added before the above code to override it.

For the main file, one might add a line
(between |\childdocmain| and the above block)
%
\begin{center}
|%\ifchilddoc\||else\providecommand{\version}{draft}\||fi|
\end{center}
%
which can be uncommented to produce a draft version.
Likewise one can add a line to the very top of a child file
(above the |\childdocof{|\textit{main}|}| directive)
%
\begin{center}
|%\providecommand{\version}{final}|
\end{center}
%
which can be uncommented to produce the final version of this child document.

%%%%%%%%%%%%%%%%%%%%%%%%%%%%%%%%%%%%%%%%%%%%%%%%%%%%%%%%%%%%%%%%%%%%%%%%%%%%%%%%
\subsection{Forwarding}
\label{sec:forward}

Different versions of the main or child documents
using compilation flags as described in \secref{sec:flags}
can be (permanently) stored in different files
for convenient compilation, viewing and distribution.
To this end, the package defines a command
to pass on compilation to a different file:

%%%%%%%%%%%%%%%%%%%%%%%%%%%%%%%%%%%%%%%%
\DescribeMacro{\childdocforward}
The command |\childdocforward| redirects processing to
another source file:
%
\begin{center}
\begin{tabular}{l}
|\input{childdoc.def}|\\
|\childdocforward[|\textit{main}|]{|\textit{dest}|}|\\
\end{tabular}
\end{center}
%
The argument \textit{dest} is the destination file
(without extension).
It should be the main file or one of the child files.
Note that further \textsf{childdoc} directives
such as |\childdocof| and |\childdocforward|
in the indicated file will be processed in this form.
The optional argument \textit{main}
passes on directly to the main file \textit{main}
while pretending to compile the child \textit{dest}.
This form behaves as if \textit{dest}
issues |\childdocof{|\textit{main}|}| right away,
and no further \textsf{childdoc} directives will be processed.

%%%%%%%%%%%%%%%%%%%%%%%%%%%%%%%%%%%%%%%%
\DescribeMacro{\...prefix}
In the alternative form |\childdocforwardprefix|,
%
\begin{center}
\begin{tabular}{l}
|\input{childdoc.def}|\\
|\childdocforwardprefix[|\textit{main}|]{|\textit{prefix}|}{|\textit{dest}|}|
\end{tabular}
\end{center}
%
the destination file is determined by a pattern
depending on the current file:
To make this work, the current file must be called
`{\textit{prefix}\hspace{0.2em}\textit{suffix}}'
with \textit{prefix} matching precisely the argument.
Processing is then passed on to the file
`{\textit{dest}\hspace{0.2em}\textit{suffix}}'.
Surely, the same effect is achieved by
directly specifying the
argument `{\textit{dest}\hspace{0.2em}\textit{suffix}}'
in the first form.
However, that requires to set up a different file
for each child. With the alternative form of the command
all these files can have exactly the same content
which simplifies setting them up and maintaining them.

For example, the following file |draft.tex|
with a compilation flag |\version| as described in \secref{sec:flags}
compiles the main document as a draft:
%
\begin{center}
\begin{tabular}{l}
|\def\version{draft}|\\
|\input{childdoc.def}|\\
|\childdocforward{|\textit{main}|}|
\end{tabular}
\end{center}
%
Likewise, the following files |final|\textit{nn}|.tex|
compile the final version of the child document
|child|\textit{nn}|.tex|:
%
\begin{center}
\begin{tabular}{l}
|\def\version{final}|\\
|\input{childdoc.def}|\\
|\childdocforwardprefix{final}{child}|
\end{tabular}
\end{center}
%

Note that when several versions of a main file and/or of each child file
are to be generated, it may be convenient to set up a |Makefile| or
shell script to automatise the process.

%%%%%%%%%%%%%%%%%%%%%%%%%%%%%%%%%%%%%%%%%%%%%%%%%%%%%%%%%%%%%%%%%%%%%%%%%%%%%%%%
\subsection{Command Line Processing}
\label{sec:commandline}

The effect of redirection files can also be achieved by invoking
the \LaTeX{} compiler with a more elaborate command line.
Most conveniently this should be done as part
of a shell script or a |Makefile|.

When using \textsf{childdoc} in the main file, the following
command lines effectively perform a redirection
(note that depending on the shell being used,
backslashes may have to be doubled: `|\|' $\to$ `|\\|'):
%
\begin{center}
|... -jobname "|\textit{target}|" |\\|"|[\textit{flags}]%
|\input{childdoc.def}\childdocforward[|\textit{main}|]{|\textit{dest}|}"|
\end{center}
%
Here \textit{target} is the name of the output file,
\textit{main} is the name of the main file
and \textit{dest} is the name of the main or child file to be processed
(all filenames without extensions).
The optional argument \textit{main} can be omitted
if \textit{main} matches \textit{dest}.
Optionally, compilation \textit{flags} can be defined via |\def| commands.
This command line makes the \TeX{} engine believe
it is compiling the file \textit{target}
whose content is specified as the latter parameter.
The provided code then forwards the processing to
\textit{main} or \textit{dest} as described in \secref{sec:forward}.

%%%%%%%%%%%%%%%%%%%%%%%%%%%%%%%%%%%%%%%%%%%%%%%%%%%%%%%%%%%%%%%%%%%%%%%%%%%%%%%%
\subsection{Include by Input}
\label{sec:input}

Including child documents by |\include| has some restrictions by design.
Most notably, the content of a child document always occupies
its own set of pages; pages cannot be shared between child documents.
Usually, this behaviour makes perfect sense
because each child document contain an essential part of the document.
However, in some situations it may be desirable to compose
a document from a collection of parts
without having mandatory page breaks between then.
For this case, the package
provides a mechanism to include parts
by |\input| which can also be processed individually.
However, by construction this mechanism
requires manual handling of the content to be output.

%%%%%%%%%%%%%%%%%%%%%%%%%%%%%%%%%%%%%%%%
\DescribeMacro{\ifchilddocmanual}
The main file should be prepared as usual, see \secref{sec:include}.
However, the document body must make a distinction
between processing of an individual part and of the main document, e.g.:
%
\begin{center}
\begin{tabular}{l}
|\ifchilddocmanual|\\
|\input{\childdocname}|\\
|\||else|\\
\textit{document body with }|\input{|\textit{part}|}|\\
|\||fi|
\end{tabular}
\end{center}
%
The conditional |\ifchilddocmanual| is true whenever
a part to be included by |\input| is being compiled,
and the name of the part is stored in |\childdocname|.

%%%%%%%%%%%%%%%%%%%%%%%%%%%%%%%%%%%%%%%%
\DescribeMacro{\childdocby}
Each part to be included by |\input| should start with:
%
\begin{center}
\begin{tabular}{l}
|\input{childdoc.def}|\\
|\childdocby{|\textit{main}|}|\\
\end{tabular}
\end{center}
%
The directive |\childdocby| is similar to |\childdocof|
described in \secref{sec:include},
but the subsequent selection of content must be done manually.
To that end, both |\ifchilddoc| and |\ifchilddocmanual|
will be true upon processing of a part,
and the name of the part is stored in |\childdocname|.
Note that |\jobname| will be set to the filename of the current part
so that each part receives an individual |.aux| file
that does not interfere with the |.aux| file(s) of the main document.
This behaviour can be altered by the alternative form
|\childdocby[*]{|\textit{main}|}| (with a non-empty optional argument)
which uses the |.aux| file of the main document
by setting |\jobname| to \textit{main}.

%%%%%%%%%%%%%%%%%%%%%%%%%%%%%%%%%%%%%%%%%%%%%%%%%%%%%%%%%%%%%%%%%%%%%%%%%%%%%%%%
\subsection{Driver Development}
\label{sec:driver}

The \textsf{childdoc} mechanism can also be use for the development
of definition files such as \LaTeX{} styles or classes.
This case differs from the above setup with multiple parts
included by |\include| in that no |\includeonly| should be invoked.
This can be achieved by starting the include file
(before |\ProvidesPackage|) with:
%
\begin{center}
\begin{tabular}{l}
|\input{childdoc.def}|\\
|\childdocforward{|\textit{main}|}|\\
\end{tabular}
\end{center}
%
or alternatively with:
%
\begin{center}
\begin{tabular}{l}
|\input{childdoc.def}|\\
|\childdocby{|\textit{main}|}|\\
\end{tabular}
\end{center}
%
Both forms have slightly different effects as described above.
The main file is prepared as usual, see \secref{sec:include}.

%%%%%%%%%%%%%%%%%%%%%%%%%%%%%%%%%%%%%%%%%%%%%%%%%%%%%%%%%%%%%%%%%%%%%%%%%%%%%%%%
\subsection{Legacy Detection}
\label{sec:detection}

The directive |\childdocmain| in the main file can detect
whether the complete document or merely a child is to be compiled
even without using the directive |\childdocof|.
This method is deprecated because it is less robust
and there is no compelling reason to use it;
it is merely provided for backward compatibility
and it may be removed in future versions.

If the detection mechanism is to be used,
it is mandatory to correctly specify
the filename of the main file as the argument of |\childdocmain|:
%
\begin{center}
\begin{tabular}{l}
|\input{childdoc.def}|\\
|\childdocmain{|\textit{main}|}|\\
\end{tabular}
\end{center}
%
If |\jobname| does not match the argument \textit{main} of |\childdocmain|,
it is assumed that |\jobname| points to the child file to be compiled.
When using |\childdocmain| with the main file specified as argument,
it suffices to start a child file
with just |\input{|\textit{main}|}|
without loading of the package and using |\childdocof|.
If instead all processing is done
with the appropriate \textsf{childdoc} directives,
the argument of \textit{main} of |\childdocmain| can be empty.

An alternative version of the command line processing described
in \secref{sec:commandline} using the detection mechanism reads:
%
\begin{center}
|... -jobname "|\textit{target}|" "|[\textit{flags}]%
[|\def\jobname{|\textit{dest}|}|]|\input{|\textit{main}|}"|
\end{center}

%%%%%%%%%%%%%%%%%%%%%%%%%%%%%%%%%%%%%%%%%%%%%%%%%%%%%%%%%%%%%%%%%%%%%%%%%%%%%%%%
\subsection{Manual Code}
\label{sec:manual}

In case one cannot be certain whether the definitions file |childdoc.def|
is installed on the target \TeX{} distribution
and one prefers not to ship it,
it is conceivable to paste a few relevant commands into the sources.

To that end, drop all statements |\input{childdoc.def}|
and perform the replacements as outlined below.
Instead of |\childdocmain{|\textit{main}|}| add the following code
to the top of the main file:
%
\begin{center}
\begin{tabular}{l}
|\||ifdefined\childdocname\endinput\||fi\newif\ifchilddoc|\\
|\edef\childdocname{\scantokens\expandafter{\jobname\noexpand}}|\\
|\def\childdocmain{|\textit{main}|}\||ifx\childdocmain\childdocname\||else|\\
|\childdoctrue\includeonly{\childdocname}\let\jobname\childdocmain\||fi|\\
\end{tabular}
\end{center}
%
Instead of |\childdocof{|\textit{main}|}| just include the main file
at the top of each child file:
%
\begin{center}
|\input{|\textit{main}|}|
\end{center}
%
A simple redirection |\childdocforward{|\textit{dest}|}| is achieved by:
%
\begin{center}
|\def\jobname{|\textit{dest}|}\input{\jobname}|
\end{center}
%
The redirection with prefix
|\childdocforwardprefix[|\textit{prefix}|]{|\textit{dest}|}|
is accomplished by:
%
\begin{center}
\begin{tabular}{l}
|{\edef\jobname{\scantokens\expandafter{\jobname\noexpand}}|\\
|\def\redirectjob |\textit{prefix}|#1~~~{\gdef\jobname{|\textit{dest}|#1}}|\\
|\expandafter\redirectjob\jobname~~~}\input{\jobname}|
\end{tabular}
\end{center}

In an alternative approach,
child documents can be compiled by a specific command line
without additional code or specific definitions:
%
\begin{center}
|... -jobname "|\textit{target}|" "|[\textit{flags}]%
|\includeonly{|\textit{dest}|}\input{|\textit{main}|}"|
\end{center}
%

%%%%%%%%%%%%%%%%%%%%%%%%%%%%%%%%%%%%%%%%%%%%%%%%%%%%%%%%%%%%%%%%%%%%%%%%%%%%%%%%
%%%%%%%%%%%%%%%%%%%%%%%%%%%%%%%%%%%%%%%%%%%%%%%%%%%%%%%%%%%%%%%%%%%%%%%%%%%%%%%%
\section{Information}

%%%%%%%%%%%%%%%%%%%%%%%%%%%%%%%%%%%%%%%%%%%%%%%%%%%%%%%%%%%%%%%%%%%%%%%%%%%%%%%%
\subsection{Copyright}

Copyright \copyright{} 2017--2018 Niklas Beisert

This work may be distributed and/or modified under the
conditions of the \LaTeX{} Project Public License, either version 1.3
of this license or (at your option) any later version.
The latest version of this license is in
  \url{http://www.latex-project.org/lppl.txt}
and version 1.3 or later is part of all distributions of \LaTeX{}
version 2005/12/01 or later.

This work has the LPPL maintenance status `maintained'.

The Current Maintainer of this work is Niklas Beisert.

This work consists of the files |README.txt|, |childdoc.ins| and |childdoc.dtx|
as well as the derived files |childdoc.def|, |cdocsamp.tex|
with |cdocsch1.tex|, |cdocsch2.tex|, |cdocspt3.tex|, |cdocspt4.tex|,
|cdocsdrf.tex|, |cdocsfn1.tex|, |cdocsfn2.tex|
as well as |childdoc.pdf|.

%%%%%%%%%%%%%%%%%%%%%%%%%%%%%%%%%%%%%%%%%%%%%%%%%%%%%%%%%%%%%%%%%%%%%%%%%%%%%%%%
\subsection{Files and Installation}

The package consists of the files:
%
\begin{center}
\begin{tabular}{ll}
    |README.txt|   & readme file \\
    |childdoc.ins| & installation file \\
    |childdoc.dtx| & source file \\
    |childdoc.def| & definition file \\
    |cdocsamp.tex| & sample main file \\
    |cdocsch1.tex| & sample include file \\
    |cdocsch2.tex| & sample include file \\
    |cdocspt3.tex| & sample part file \\
    |cdocspt4.tex| & sample part file \\
    |cdocsdrf.tex| & sample redirection file \\
    |cdocsfn1.tex| & sample redirection file \\
    |cdocsfn2.tex| & sample redirection file \\
    |childdoc.pdf| & manual
\end{tabular}
\end{center}
%
The distribution consists of the files
|README.txt|, |childdoc.ins| and |childdoc.dtx|.
%
\begin{itemize}
\item
Run (pdf)\LaTeX{} on |childdoc.dtx|
to compile the manual |childdoc.pdf| (this file).
\item
Run \LaTeX{} on |childdoc.ins| to create the definitions file |childdoc.def|
and the sample |cdocsamp.tex| with include files
|cdocsch1.tex|, |cdocsch2.tex|, |cdocspt3.tex|, |cdocspt4.tex|,
|cdocsdrf.tex|, |cdocsfn1.tex|, |cdocsfn2.tex|.
Then copy the file |childdoc.def| to an appropriate directory of your \LaTeX{}
distribution, e.g.\ \textit{texmf-root}|/tex/latex/childdoc|.
\end{itemize}

%%%%%%%%%%%%%%%%%%%%%%%%%%%%%%%%%%%%%%%%%%%%%%%%%%%%%%%%%%%%%%%%%%%%%%%%%%%%%%%%
\subsection{Related CTAN Packages}

There are several other packages which offer a similar functionality:
%
\begin{itemize}
\item
The packages
\href{http://ctan.org/pkg/docmute}{\textsf{docmute}},
\href{http://ctan.org/pkg/includex}{\textsf{includex}} and
\href{http://ctan.org/pkg/standalone}{\textsf{standalone}}
provide commands to include only the document body of
a child file thus allowing both files to be compiled individually.
\item
The packages \href{http://ctan.org/pkg/subdocs}{\textsf{subdocs}}
and \href{http://ctan.org/pkg/subfiles}{\textsf{subfiles}}
provide structures in which the main and child documents can be
encapsulated and allowing them to be compiled individually.
The inclusion mechanism is different from the conventional |\include|.
\item
The package \href{http://ctan.org/pkg/combine}{\textsf{combine}}
is an elaborate solution to combine several documents into one.
\end{itemize}
%
See also the CTAN topic \href{http://ctan.org/topic/subdocs}{\textsf{subdocs}}
for further related packages.
The present package differs from the above solutions in that
a document structure constructed with the conventional |\include| mechanism
just needs two extra commands at the top of every file
such that all constituent files can be compiled individually.

%%%%%%%%%%%%%%%%%%%%%%%%%%%%%%%%%%%%%%%%%%%%%%%%%%%%%%%%%%%%%%%%%%%%%%%%%%%%%%%%
%\subsection{Feature Suggestions}
%
%The following is a list of features which may be useful for future
%versions of this package:
%%
%\begin{itemize}
%\item
%\ldots
%\end{itemize}

%%%%%%%%%%%%%%%%%%%%%%%%%%%%%%%%%%%%%%%%%%%%%%%%%%%%%%%%%%%%%%%%%%%%%%%%%%%%%%%%
\subsection{Revision History}

%%%%%%%%%%%%%%%%%%%%%%%%%%%%%%%%%%%%%%%%
\paragraph{v2.0:} 2018/12/30

\begin{itemize}
\item
immediate forward processing
\item
added |\childdocby| mechanism
\item
manual restructured
\end{itemize}

%%%%%%%%%%%%%%%%%%%%%%%%%%%%%%%%%%%%%%%%
\paragraph{v1.6:} 2018/01/17

\begin{itemize}
\item
application for development of include files
\item
corrections to manual
\end{itemize}

%%%%%%%%%%%%%%%%%%%%%%%%%%%%%%%%%%%%%%%%
\paragraph{v1.5:} 2017/05/21

\begin{itemize}
\item
more complete structuring introduced
\item
|\childdocof| introduced
\item
|\childdoc| renamed to |\childdocmain|
\item
|\childredirect| renamed to |\childdocforward| and |\childdocforwardprefix|
and functionality expanded
\end{itemize}

%%%%%%%%%%%%%%%%%%%%%%%%%%%%%%%%%%%%%%%%
\paragraph{v1.0:} 2017/04/27

\begin{itemize}
\item
manual and install package
\item
first version published on CTAN
\end{itemize}

%%%%%%%%%%%%%%%%%%%%%%%%%%%%%%%%%%%%%%%%
\paragraph{v0.6:} 2017/04/26

\begin{itemize}
\item
redirection mechanism added
\end{itemize}

%%%%%%%%%%%%%%%%%%%%%%%%%%%%%%%%%%%%%%%%
\paragraph{v0.5:} 2017/04/26

\begin{itemize}
\item
functionality in definition file
\end{itemize}


%%%%%%%%%%%%%%%%%%%%%%%%%%%%%%%%%%%%%%%%%%%%%%%%%%%%%%%%%%%%%%%%%%%%%%%%%%%%%%%%
%%%%%%%%%%%%%%%%%%%%%%%%%%%%%%%%%%%%%%%%%%%%%%%%%%%%%%%%%%%%%%%%%%%%%%%%%%%%%%%%
%%%%%%%%%%%%%%%%%%%%%%%%%%%%%%%%%%%%%%%%%%%%%%%%%%%%%%%%%%%%%%%%%%%%%%%%%%%%%%%%
\appendix

\settowidth\MacroIndent{\rmfamily\scriptsize 000\ }

 \DocInput{childdoc.dtx}

\end{document}
%</driver>
% \fi
%
% %%%%%%%%%%%%%%%%%%%%%%%%%%%%%%%%%%%%%%%%%%%%%%%%%%%%%%%%%%%%%%%%%%%%%%%%%%%%%%
% %%%%%%%%%%%%%%%%%%%%%%%%%%%%%%%%%%%%%%%%%%%%%%%%%%%%%%%%%%%%%%%%%%%%%%%%%%%%%%
% \section{Sample}
%\iffalse
%<*samplemain>
%\fi
%
% The following presents a sample document
% with two chapters, two parts, a title page,
% a compile flag as well as three forwarding files to set the flag.
% It consists of eight |.tex| files:
% \begin{center}
% \begin{tabular}{ll}
% |cdocsamp.tex|&main file\\
% |cdocsch1.tex|&include file for chapter 1\\
% |cdocsch2.tex|&include file for chapter 2\\
% |cdocspt3.tex|&include file for part 3\\
% |cdocspt4.tex|&include file for part 4\\
% |cdocsdrf.tex|&forwarding file for main file in draft mode\\
% |cdocsfi1.tex|&forwarding file for final version of chapter 1\\
% |cdocsfi2.tex|&forwarding file for final version of chapter 2\\
% \end{tabular}
% \end{center}
% Each of the eight files can be compiled directly by the \LaTeX{} compiler.
%
% %%%%%%%%%%%%%%%%%%%%%%%%%%%%%%%%%%%%%%
% \paragraph{Main File.}
%
% The main file is called |cdocsamp.tex|.
%
% Load the \textsf{childdoc} definitions and
% declare the filename for the main document:
%    \begin{macrocode}
\input{childdoc.def}
\childdocmain{}
%    \end{macrocode}

% Optional override for |\version| flag:
%    \begin{macrocode}
%%\ifchilddoc\else\providecommand{\version}{draft}\fi
%    \end{macrocode}

% Define the default values for the |\version| flag
% (|final| for the main file and |draft| for childs):
%    \begin{macrocode}
\ifchilddoc
\providecommand{\version}{draft}
\else
\providecommand{\version}{final}
\fi
%    \end{macrocode}

% Load the standard document class:
%    \begin{macrocode}
\documentclass[12pt]{article}
%    \end{macrocode}

% Start the document body:
%    \begin{macrocode}
\begin{document}
%    \end{macrocode}

% Declare a title page.
% Print title, part of document being processed and version flag:
%    \begin{macrocode}
\addtocounter{page}{-1}
\begin{center}
{\LARGE\bfseries{}childdoc example\par}
\vspace{1cm}
\ifchilddoc
\ifchilddocmanual part\else chapter\fi:
`\childdocname' of `\childdocjob'\par
\else
main document: `\childdocjob'\par
\fi
version: \version\par
\end{center}
\newpage
%    \end{macrocode}

% Manually include selected file,
% otherwise process as usual:
%    \begin{macrocode}
\ifchilddocmanual
\section*{part `\childdocname'}
\input{\childdocname}
\else
%    \end{macrocode}

% Include the two chapters:
%    \begin{macrocode}
\include{cdocsch1}
\include{cdocsch2}
%    \end{macrocode}

% Include the two parts unless only chapters should be displayed:
%    \begin{macrocode}
\ifchilddoc\else
\section{part three}
\input{cdocspt3}
\section{part four}
\input{cdocspt4}
\fi
%    \end{macrocode}

% Process as usual until here:
%    \begin{macrocode}
\fi
%    \end{macrocode}

% End of document body:
%    \begin{macrocode}
\end{document}
%    \end{macrocode}
%\iffalse
%</samplemain>
%\fi
%
% %%%%%%%%%%%%%%%%%%%%%%%%%%%%%%%%%%%%%%
% \paragraph{Chapter Include Files.}
%
% The include files are called |cdocsch1.tex| and |cdocsch2.tex|.
%
%\iffalse
%<*samplechap1|samplechap2>
%\fi

% Optional override for |\version| flag:
%    \begin{macrocode}
%%\providecommand{\version}{final}
%    \end{macrocode}

% Include the main document:
%    \begin{macrocode}
\input{childdoc.def}
\childdocof{cdocsamp}
%    \end{macrocode}

%\iffalse
%</samplechap1|samplechap2>
%\fi
%
%\iffalse
%<*samplechap1>
%\fi
% Some text for chapter 1:
%    \begin{macrocode}
\section{one}
some text in chapter one
%    \end{macrocode}

%\iffalse
%</samplechap1>
%\fi
% Some text for chapter 2:
%\iffalse
%<*samplechap2>
%\fi
%    \begin{macrocode}
\section{two}
more text in chapter two
%    \end{macrocode}

%\iffalse
%</samplechap2>
%\fi
%
% %%%%%%%%%%%%%%%%%%%%%%%%%%%%%%%%%%%%%%
% \paragraph{Part Include Files.}
%
% The include files are called |cdocspt3.tex| and |cdocspt4.tex|.
%
%\iffalse
%<*samplepart3|samplepart4>
%\fi

% Optional override for |\version| flag:
%    \begin{macrocode}
%%\providecommand{\version}{final}
%    \end{macrocode}

% Include the main document:
%    \begin{macrocode}
\input{childdoc.def}
\childdocby{cdocsamp}
%    \end{macrocode}

%\iffalse
%</samplepart3|samplepart4>
%\fi
%
%\iffalse
%<*samplepart3>
%\fi
% Some text for part 3:
%    \begin{macrocode}
some text in part three
%    \end{macrocode}

%\iffalse
%</samplepart3>
%\fi
% Some text for part 4:
%\iffalse
%<*samplepart4>
%\fi
%    \begin{macrocode}
more text in part four
%    \end{macrocode}

%\iffalse
%</samplepart4>
%\fi
%
% %%%%%%%%%%%%%%%%%%%%%%%%%%%%%%%%%%%%%%
% \paragraph{Forwarding for a Complete Draft.}
%
% The following forwarding file |cdocsdrf.tex|
% compiles the main document in draft mode:
%\iffalse
%<*sampledraft>
%\fi
%    \begin{macrocode}
\def\version{draft}
\input{childdoc.def}
\childdocforward{cdocsamp}
%    \end{macrocode}

%\iffalse
%</sampledraft>
%\fi
%
% %%%%%%%%%%%%%%%%%%%%%%%%%%%%%%%%%%%%%%
% \paragraph{Forwarding for Final Version of the Chapters.}
%
% The following forwarding files |cdocsfn1.tex| and |cdocsfn2.tex|
% (with identical content)
% compile the final versions of the child documents
% |cdocsch1.tex| and |cdocsch2.tex|, respectively:
%\iffalse
%<*samplefinal>
%\fi
%    \begin{macrocode}
\def\version{final}
\input{childdoc.def}
\childdocforwardprefix[cdocsamp]{cdocsfn}{cdocsch}
%    \end{macrocode}

%\iffalse
%</samplefinal>
%\fi
%
% %%%%%%%%%%%%%%%%%%%%%%%%%%%%%%%%%%%%%%
% \paragraph{Command Line Processing.}
%
% The following three command lines generate the output files
% |cdocscld|, |cdocscl1| and |cdocscl2|
% which should be identical to
% |cdocsdrf|, |cdocsch1| and |cdocsfn2|, respectively:
% \begin{center}
% \begin{tabular}{l}
% |latex -jobname cdocscld \|\\
% |  "\def\version{draft}\input{childdoc.def}\childdocforward{cdocsamp}"|\\
% |latex -jobname cdocscl1 \|\\
% |  "\input{childdoc.def}\childdocforward[cdocsamp]{cdocsch1}"|\\
% |latex -jobname cdocscl2 \|\\
% |  "\def\version{final}\input{childdoc.def}\childdocforward{cdocsch2}"|
% \end{tabular}
% \end{center}
% Note that the trailing backslash on each first line
% merely continues the input to the second line
% (for convenient cut ant paste).
% Furthermore, the command |latex| can be replaced by any
% of its alternative versions such as |pdflatex|.
%
% %%%%%%%%%%%%%%%%%%%%%%%%%%%%%%%%%%%%%%%%%%%%%%%%%%%%%%%%%%%%%%%%%%%%%%%%%%%%%%
% %%%%%%%%%%%%%%%%%%%%%%%%%%%%%%%%%%%%%%%%%%%%%%%%%%%%%%%%%%%%%%%%%%%%%%%%%%%%%%
% \section{Implementation}
%\iffalse
%<*package>
%\fi
%
% This section describes the definitions file |childdoc.def|.

% The definitions cannot be loaded using |\usepackage| or |\RequirePackage|
% which has a mechanism to prevent loading a style file more than once.
% When loading the definitions by means of |\input|
% multiple instances have to be prevented manually:
%\iffalse
%This code needs to be before the `\ProvidesFile' directive
%which is defined at the beginning of this file.
%Therefore it is also placed there and commented out here.
%</package>
%<*discard>
%\fi
%    \begin{macrocode}
\ifdefined\childdocmain\endinput\fi
%    \end{macrocode}
%\iffalse
%</discard>
%<*package>
%\fi
%
% \macro{\ifchilddoc}
% \macro{\ifchilddocmanual}
% The conditional |\ifchilddoc| tells whether a
% child (true) or main (false) document is being compiled.
% The conditional |\ifchilddocmanual| tells whether
% the |\includeonly| mechanism is used (false) or
% the selection of child files must be performed manually (true).
% The definitions initialise to false:
%    \begin{macrocode}
\newif\ifchilddoc
\newif\ifchilddocmanual
%    \end{macrocode}

% \macro{\childdocname}
% \macro{\childdocjob}
% The macro |\childdocname| stores the name of the main document
% to be compiled. The macro |\childdocjob| stores the name of
% the document on which the \LaTeX{} compiler was originally invoked.
% The content of |\jobname| cannot be compared
% to filenames specified in the source due to different catcodes.
% The following code rescans |\jobname|, stores the result
% in |\childdocname| and saves a copy in |\childdocjob|:
%    \begin{macrocode}
\edef\childdocname{\scantokens\expandafter{\jobname\noexpand}}
\let\childdocjob\childdocname
%    \end{macrocode}

% \macro{\childdocdisable}
% The macro |\childdocdisable| prevents the main file
% from being processed more than once.
% At this stage, the main document command |\childdocmain|
% is assumed to be called once again where it should do nothing.
% Any subsequent call to it should prevent
% a secondary processing of the main document
% It overwrites the forwarding commands
% |\childdocof| and |\childdocforward|
% with empty macros to prevent further inclusions of the main document:
%    \begin{macrocode}
\newcommand{\childdocdisable}
{
  \renewcommand{\childdocmain}[1]{\renewcommand{\childdocmain}[1]{\endinput}}
  \renewcommand{\childdocof}[1]{}
  \renewcommand{\childdocby}[2][]{}
  \renewcommand{\childdocforward}[2][]{}
  \renewcommand{\childdocdisable}{}
}
%    \end{macrocode}

% \macro{\childdocmain}
% The macro |\childdocmain| is to be called at the top of the main file
% with nothing or the main filename (without extension) as argument.
% First, it breaks loops.
% If the argument is not empty and does not match |\childdocname|
% (which is set by the first inclusion of |childdoc.def|),
% |\ifchilddoc| is set to true, |\includeonly| is applied to the child file
% and |\jobname| is set to the main file
% (for proper handling of |.aux| files):
%    \begin{macrocode}
\newcommand{\childdocmain}[1]
{
  \childdocdisable\childdocmain{}
  \if?#1?\else
    \begingroup
      \def\childdoctmp{#1}
      \ifx\childdoctmp\childdocname
        \def\childdoctmp{}
      \else
        \def\childdoctmp
        {
          \childdoctrue
          \includeonly{\childdocname}
          \def\childdocjob{#1}
          \def\jobname{#1}
        }
      \fi
      \expandafter
    \endgroup
    \childdoctmp
  \fi
}
%    \end{macrocode}

% \macro{\childdocof}
% The command |\childdocof| redirects
% compilation to the main file |#1|.
%    \begin{macrocode}
\newcommand{\childdocof}[1]
{
  \childdocdisable
  \childdoctrue
  \includeonly{\childdocname}
  \def\jobname{#1}
  \def\childdocjob{#1}
  \input{#1}
}
%    \end{macrocode}

% \macro{\childdocby}
% The command |\childdocby| ....
%    \begin{macrocode}
\newcommand{\childdocby}[2][]
{
  \childdocdisable
  \childdoctrue
  \childdocmanualtrue
  \if?#1?\else
    \def\jobname{#2}
  \fi
  \def\childdocjob{#2}
  \input{#2}
  \endinput
}
%    \end{macrocode}

% \macro{\childdocforward}
% The command |\childdocforward| redirects
% compilation to the main file or
% (if the optional argument is given) a child file.
% Parameters are set as if the main file
% or a child file starting with |\childdocof| was compiled.
% Then compilation is handed over to the main file:
%    \begin{macrocode}
\newcommand{\childdocforward}[2][]
{
  \begingroup
    \if?#1?
      \def\childdoctmp
      {
        \def\childdocname{#2}
        \def\childdocjob{#2}
        \def\jobname{#2}
        \input{#2}
        \endinput
      }
    \else
      \def\childdoctmp
      {
        \childdocdisable
        \def\childdocname{#2}
        \childdoctrue
        \includeonly{#2}
        \def\childdocjob{#1}
        \def\jobname{#1}
        \input{#1}
        \endinput
      }
    \fi
    \expandafter
  \endgroup
  \childdoctmp
}
%    \end{macrocode}

% \macro{\childdocforwardprefix}
% The command |\childdocforwardprefix| redirects
% compilation to the main or a child file by means of a pattern.
% The prefix |#1| in the current filename is replaced by |#2|
% and the suffix of the current filename is kept
% (it is assumed that the filename does not contain the substring `|~~~|'
% which is used as a delimiter).
% Compilation is handed over to the new file by |\childdocforward|:
%    \begin{macrocode}
\newcommand{\childdocforwardprefix}[3][]
{
  \begingroup
    \def\childdocextract #2##1~~~{\def\childdoctmp{\childdocforward[#1]{#3##1}}}
    \expandafter\childdocextract\childdocname~~~
    \expandafter
  \endgroup
  \childdoctmp
}
%    \end{macrocode}

% \macro{\childdoc}
% The deprecated macro |\childdoc| is a legacy version of |\childdocmain|:
%    \begin{macrocode}
\newcommand{\childdoc}{\childdocmain}
%    \end{macrocode}

% \macro{\childdocredirect}
% The deprecated macro |\childdocredirect| is a legacy version
% of |\childdocforward| and |\childdocforwardprefix|:
%    \begin{macrocode}
\newcommand{\childdocredirect}[2][]
{
  \begingroup
    \if?#1?
      \def\childdoctmp{\childdocforward{#2}}
    \else
      \def\childdoctmp{\childdocforwardprefix{#1}{#2}}
    \fi
    \expandafter
  \endgroup
  \childdoctmp
}
%    \end{macrocode}

%\iffalse
%</package>
%\fi
%
\endinput
|\\
|\childdocforwardprefix[|\textit{main}|]{|\textit{prefix}|}{|\textit{dest}|}|
\end{tabular}
\end{center}
%
the destination file is determined by a pattern
depending on the current file:
To make this work, the current file must be called
`{\textit{prefix}\hspace{0.2em}\textit{suffix}}'
with \textit{prefix} matching precisely the argument.
Processing is then passed on to the file
`{\textit{dest}\hspace{0.2em}\textit{suffix}}'.
Surely, the same effect is achieved by
directly specifying the
argument `{\textit{dest}\hspace{0.2em}\textit{suffix}}'
in the first form.
However, that requires to set up a different file
for each child. With the alternative form of the command
all these files can have exactly the same content
which simplifies setting them up and maintaining them.

For example, the following file |draft.tex|
with a compilation flag |\version| as described in \secref{sec:flags}
compiles the main document as a draft:
%
\begin{center}
\begin{tabular}{l}
|\def\version{draft}|\\
|% \iffalse
%
% childdoc.dtx Copyright (C) 2017-2018 Niklas Beisert
%
% This work may be distributed and/or modified under the
% conditions of the LaTeX Project Public License, either version 1.3
% of this license or (at your option) any later version.
% The latest version of this license is in
%   http://www.latex-project.org/lppl.txt
% and version 1.3 or later is part of all distributions of LaTeX
% version 2005/12/01 or later.
%
% This work has the LPPL maintenance status `maintained'.
%
% The Current Maintainer of this work is Niklas Beisert.
%
% This work consists of the files childdoc.dtx and childdoc.ins
% and the derived files childdoc.def and cdocsamp.tex with
% cdocsch1.tex, cdocsch2.tex, cdocsdrf.tex, cdocsfn1.tex, cdocsfn2.tex.
%
%<package>\ifdefined\childdocmain\endinput\fi
%<package>\ProvidesFile{childdoc.def}[2018/12/30 v2.0 child document driver]
%<samplemain>\ProvidesFile{cdocsamp.tex}[2018/12/30 v2.0 sample for childdoc]
%<*driver>
%\ProvidesFile{childdoc.drv}[2018/12/30 v2.0 childdoc reference manual file]
\PassOptionsToClass{10pt,a4paper}{article}
\documentclass{ltxdoc}

\usepackage[margin=35mm]{geometry}
\usepackage{hyperref}
\usepackage{hyperxmp}
\usepackage[usenames]{color}

\hypersetup{colorlinks=true}
\hypersetup{pdfstartview=FitH}
\hypersetup{pdfpagemode=UseNone}
\hypersetup{pdfsource={}}
\hypersetup{pdflang={en-UK}}
\hypersetup{pdfcopyright={Copyright 2017-2018 Niklas Beisert.
  This work may be distributed and/or modified under the
  conditions of the LaTeX Project Public License, either version 1.3
  of this license or (at your option) any later version.}}
\hypersetup{pdflicenseurl={http://www.latex-project.org/lppl.txt}}
\hypersetup{pdfcontactaddress={ETH Zurich, ITP, HIT K,
  Wolfgang-Pauli-Strasse 27}}
\hypersetup{pdfcontactpostcode={8093}}
\hypersetup{pdfcontactcity={Zurich}}
\hypersetup{pdfcontactcountry={Switzerland}}
\hypersetup{pdfcontactemail={nbeisert@itp.phys.ethz.ch}}
\hypersetup{pdfcontacturl={http://people.phys.ethz.ch/\xmptilde nbeisert/}}

\newcommand{\secref}[1]{\hyperref[#1]{section \ref*{#1}}}

\parskip1ex
\parindent0pt
\let\olditemize\itemize
\def\itemize{\olditemize\parskip0pt}

\begin{document}

\title{The \textsf{childdoc} Package}
\hypersetup{pdftitle={The childdoc Package}}
\author{Niklas Beisert\\[2ex]
  Institut f\"ur Theoretische Physik\\
  Eidgen\"ossische Technische Hochschule Z\"urich\\
  Wolfgang-Pauli-Strasse 27, 8093 Z\"urich, Switzerland\\[1ex]
  \href{mailto:nbeisert@itp.phys.ethz.ch}
  {\texttt{nbeisert@itp.phys.ethz.ch}}}
\hypersetup{pdfauthor={Niklas Beisert}}
\hypersetup{pdfsubject={Manual for the LaTeX2e Package childdoc}}
\date{30 December 2018, \textsf{v2.0}}
\maketitle

\begin{abstract}\noindent
\textsf{childdoc} is a \LaTeXe{} package
that enables the direct compilation
of document sections included by |\include|
to individual files.
\end{abstract}

\begingroup
\parskip0ex
\tableofcontents
\endgroup

%%%%%%%%%%%%%%%%%%%%%%%%%%%%%%%%%%%%%%%%%%%%%%%%%%%%%%%%%%%%%%%%%%%%%%%%%%%%%%%%
%%%%%%%%%%%%%%%%%%%%%%%%%%%%%%%%%%%%%%%%%%%%%%%%%%%%%%%%%%%%%%%%%%%%%%%%%%%%%%%%
\section{Introduction}

\LaTeX{} provides a mechanism to structure a large document (such as a book)
into a main file and several child files (containing the chapters)
using the |\include| command.
This mechanism is beneficial for documents
which span hundreds of pages in order to
make the source file(s) more manageable.
Moreover, compilation can be restricted to
selected child files by means of the |\includeonly| command.
The latter feature can be used to reduce the compilation time while editing
(this was significantly more useful in the earlier days of \LaTeX{})
or to generate a smaller document which is easier to navigate.
Another application of |\includeonly| is to generate
documents consisting of selected parts of the complete document.

However, there are a few drawbacks of the plain |\include| mechanism:
\begin{itemize}
\item
The child files cannot be compiled on their own,
they can only be compiled via the main file.
A naive editing environment
(such as a text editor with an option
to have the current file processed by \LaTeX)
may require one to switch to the main file before compiling;
attempting to compile the child file produces errors.
\item
The main file must be modified (each time)
to adjust the |\includeonly| command
to the present needs. This easily leaves the main file in a messy state.
\item
The generated document will always carry the filename
of the main document. This is inconvenient if
several child files are to be compiled and
to be kept for distribution.
\end{itemize}

The present package provides a simple interface
to make child files individually compilable by \LaTeX{}.
Compiling a child file then has the same effect as compiling
the main file with an |\includeonly| command
to select the appropriate child.
Moreover the generated document will carry the name of the child
rather than the main file.
This resolves all three above issues.

This feature is meant to make the editing of books,
thesis documents and lecture notes somewhat more convenient.
However, the package can also be used efficiently for
composing a series of documents (such as exercise sheets)
which are typically distributed individually.
It then assists the author in generating the individual documents
(potentially in different versions)
as well as a document containing the collected series.
Another application is in developing style files
or other kinds of included material
where compilation of the style file could redirect
to a sample or test file.

%%%%%%%%%%%%%%%%%%%%%%%%%%%%%%%%%%%%%%%%%%%%%%%%%%%%%%%%%%%%%%%%%%%%%%%%%%%%%%%%
%%%%%%%%%%%%%%%%%%%%%%%%%%%%%%%%%%%%%%%%%%%%%%%%%%%%%%%%%%%%%%%%%%%%%%%%%%%%%%%%
\section{Usage}

First of all, the package \textsf{childdoc} is \emph{not} a standard
\LaTeXe{} |.sty| style file! Therefore it needs to be invoked in
a non-standard way.

%%%%%%%%%%%%%%%%%%%%%%%%%%%%%%%%%%%%%%%%%%%%%%%%%%%%%%%%%%%%%%%%%%%%%%%%%%%%%%%%
\subsection{Included Files}
\label{sec:include}

%%%%%%%%%%%%%%%%%%%%%%%%%%%%%%%%%%%%%%%%
\DescribeMacro{\childdocmain}
To use the package, add the commands
\begin{center}
\begin{tabular}{l}
|\input{childdoc.def}|\\
|\childdocmain{}|\\
\end{tabular}
\end{center}
at the very top of the main \LaTeX{} file,
in particular \emph{before} the |\documentclass| statement!
The argument of |\childdocmain| should be left empty
(but it must be present).

%%%%%%%%%%%%%%%%%%%%%%%%%%%%%%%%%%%%%%%%
\DescribeMacro{\childdocof}
Furthermore, add the commands
\begin{center}
\begin{tabular}{l}
|\input{childdoc.def}|\\
|\childdocof{|\textit{main}|}|\\
\end{tabular}
\end{center}
at the top of every child file \textit{child}
which is included by |\include{|\textit{child}|}|
from within the main file
(or at least for those files to be compiled individually).
The argument \textit{main} must be the filename of the main file.

There are a couple of
considerations in setting up the main and child documents:

%%%%%%%%%%%%%%%%%%%%%%%%%%%%%%%%%%%%%%%%
\paragraph{Restrictions.}

Please note the following restrictions:
\begin{itemize}
\item
|\childdocmain| must be called with one argument \textit{main}
to ensure compatibility with earlier version of the package.
It must either be empty (|\childdocmain{}|)
or precisely match the filename of the main file in which it is specified.
See \secref{sec:detection} for further information.
\item
The filename \textit{main} must be specified without the |.tex| extension.
\item
The filename \textit{main} is case sensitive
(even in case-insensitive file systems)
due to internal string comparison.
\item
The argument \textit{main} should be fully expanded, it cannot be a macro.
\item
Subdirectories and special characters should be avoided in filenames.
\item
The command |\childdocmain{|\textit{main}|}| must be followed by a whitespace.
It should not be followed immediately by another command
or by a comment mark `|%|'.
This is because the \TeX{} parser reads the token immediately following
the argument of |\childdocmain| and puts it
at the beginning of every child section;
however, a white\-space is ignored.
\end{itemize}

%%%%%%%%%%%%%%%%%%%%%%%%%%%%%%%%%%%%%%%%
\paragraph{Content of Main File.}

It is advisable to place all content in the child files included by |\include|.
Any output contained in the main file will appear in all child documents
unless suppressed manually;
it cannot be suppressed automatically by the |\includeonly| directive
and thus should normally be avoided.
A method to include some content in the main file
by means of conditional processing is described in \secref{sec:conditional}.

%%%%%%%%%%%%%%%%%%%%%%%%%%%%%%%%%%%%%%%%
\paragraph{Page Numbering.}

When only a part of the document is compiled,
the appropriate numbering of pages
(as well as other status parameters)
is determined from the |.aux| files.
The latter contain information from previous passes.
However this information needs to propagate through
all intermediate child documents.
Therefore the page numbering in child documents may well
be inconsistent until the complete document is compiled at least once.

A useful (if unconventional) way to always ensure a consistent
page numbering is to restart the numbering in each child document
and denote the pages by `\textit{child}|.|\textit{page}'
where \textit{child} represents the chapter/section number of the child file.
This can be achieved by the command
|\numberwithin{page}{|\textit{child}|}|
of the \textsf{amsmath} package
where \textit{child} can be |chapter| or |section|
depending on the chosen structuring.
Alternatively, one can modify the macro |\thepage| appropriately
and reset the counter |page| at the start of each child file.

%%%%%%%%%%%%%%%%%%%%%%%%%%%%%%%%%%%%%%%%%%%%%%%%%%%%%%%%%%%%%%%%%%%%%%%%%%%%%%%%
\subsection{Conditional Processing}
\label{sec:conditional}

The package provides a mechanism to compile different versions
of a document. To customise the versions further some conditional processing
can come in handy to distinguish which version is being compiled.
The package provides two macros to describe the compilation context:

%%%%%%%%%%%%%%%%%%%%%%%%%%%%%%%%%%%%%%%%
\DescribeMacro{\ifchilddoc}
The conditional |\ifchilddoc| distinguishes between the compilation of
child documents and the main document:
%
\begin{center}
|\ifchilddoc |\textit{child-code}| |[|\||else |\textit{main-code}]| \||fi|
\end{center}

%%%%%%%%%%%%%%%%%%%%%%%%%%%%%%%%%%%%%%%%
\DescribeMacro{\childdocname}
\DescribeMacro{\childdocjob}
The macro |\childdocname| contains the filename (without extension)
of the main or child file being processed.
Note that |\childdocjob| will always contain the name of the main file.

%%%%%%%%%%%%%%%%%%%%%%%%%%%%%%%%%%%%%%%%
\paragraph{Title Page.}

Conditional processing can be used to include a title or banner page
in the main document when proper precautions are taken.
Importantly, the code in the main file should ensure that the page counter
(as well as other status parameters which are stored in the |.aux| files)
takes the same value after the conditional processing.
Otherwise the page numbers may take divergent values
depending on which part is compiled.

For example, a title page could be declared by:
%
\begin{center}
\begin{tabular}{l}
|\ifchilddoc\||else|\\
|\addtocounter{page}{-1}|\\
\textit{code for title page}\\
|\newpage|\\
|\||fi|
\end{tabular}
\end{center}
%
A banner page for the child documents can be generated by:
%
\begin{center}
\begin{tabular}{l}
|\ifchilddoc|\\
|\addtocounter{page}{-1}|\\
\textit{code for banner page}\\
|\newpage|\\
|\||fi|
\end{tabular}
\end{center}
%
Here one could write a message such as:
\begin{center}
|This is the part \childdocname{} of \childdocjob{}.|
\end{center}

%%%%%%%%%%%%%%%%%%%%%%%%%%%%%%%%%%%%%%%%%%%%%%%%%%%%%%%%%%%%%%%%%%%%%%%%%%%%%%%%
\subsection{Flags}
\label{sec:flags}

The package makes it easy to generate different versions
of the main or child documents.
To this end compilation flags can be defined
and assigned different default values.
They will be particularly useful in conjunction
with the forwarding mechanism described in \secref{sec:forward}.

For example, it may be useful to have a flag |\version|
which can be set to |draft| or |final|.
The document source will contain some conditional code
depending on the value of |\version|.
Suppose further, the flag should default to |final| for the main file
and to |draft| for child files
which is a natural assignment for editing the document.
This is achieved by placing the following code
in the preamble of the main document
(below the |\childdocmain| directive):
%
\begin{center}
\begin{tabular}{l}
|\ifchilddoc|\\
|\providecommand{\version}{draft}|\\
|\||else|\\
|\providecommand{\version}{final}|\\
|\||fi|
\end{tabular}
\end{center}
%
The definition by |\providecommand| makes sure
that previous definitions are not overwritten.
Further statements |\providecommand{\version}{...}|
can thus be added before the above code to override it.

For the main file, one might add a line
(between |\childdocmain| and the above block)
%
\begin{center}
|%\ifchilddoc\||else\providecommand{\version}{draft}\||fi|
\end{center}
%
which can be uncommented to produce a draft version.
Likewise one can add a line to the very top of a child file
(above the |\childdocof{|\textit{main}|}| directive)
%
\begin{center}
|%\providecommand{\version}{final}|
\end{center}
%
which can be uncommented to produce the final version of this child document.

%%%%%%%%%%%%%%%%%%%%%%%%%%%%%%%%%%%%%%%%%%%%%%%%%%%%%%%%%%%%%%%%%%%%%%%%%%%%%%%%
\subsection{Forwarding}
\label{sec:forward}

Different versions of the main or child documents
using compilation flags as described in \secref{sec:flags}
can be (permanently) stored in different files
for convenient compilation, viewing and distribution.
To this end, the package defines a command
to pass on compilation to a different file:

%%%%%%%%%%%%%%%%%%%%%%%%%%%%%%%%%%%%%%%%
\DescribeMacro{\childdocforward}
The command |\childdocforward| redirects processing to
another source file:
%
\begin{center}
\begin{tabular}{l}
|\input{childdoc.def}|\\
|\childdocforward[|\textit{main}|]{|\textit{dest}|}|\\
\end{tabular}
\end{center}
%
The argument \textit{dest} is the destination file
(without extension).
It should be the main file or one of the child files.
Note that further \textsf{childdoc} directives
such as |\childdocof| and |\childdocforward|
in the indicated file will be processed in this form.
The optional argument \textit{main}
passes on directly to the main file \textit{main}
while pretending to compile the child \textit{dest}.
This form behaves as if \textit{dest}
issues |\childdocof{|\textit{main}|}| right away,
and no further \textsf{childdoc} directives will be processed.

%%%%%%%%%%%%%%%%%%%%%%%%%%%%%%%%%%%%%%%%
\DescribeMacro{\...prefix}
In the alternative form |\childdocforwardprefix|,
%
\begin{center}
\begin{tabular}{l}
|\input{childdoc.def}|\\
|\childdocforwardprefix[|\textit{main}|]{|\textit{prefix}|}{|\textit{dest}|}|
\end{tabular}
\end{center}
%
the destination file is determined by a pattern
depending on the current file:
To make this work, the current file must be called
`{\textit{prefix}\hspace{0.2em}\textit{suffix}}'
with \textit{prefix} matching precisely the argument.
Processing is then passed on to the file
`{\textit{dest}\hspace{0.2em}\textit{suffix}}'.
Surely, the same effect is achieved by
directly specifying the
argument `{\textit{dest}\hspace{0.2em}\textit{suffix}}'
in the first form.
However, that requires to set up a different file
for each child. With the alternative form of the command
all these files can have exactly the same content
which simplifies setting them up and maintaining them.

For example, the following file |draft.tex|
with a compilation flag |\version| as described in \secref{sec:flags}
compiles the main document as a draft:
%
\begin{center}
\begin{tabular}{l}
|\def\version{draft}|\\
|\input{childdoc.def}|\\
|\childdocforward{|\textit{main}|}|
\end{tabular}
\end{center}
%
Likewise, the following files |final|\textit{nn}|.tex|
compile the final version of the child document
|child|\textit{nn}|.tex|:
%
\begin{center}
\begin{tabular}{l}
|\def\version{final}|\\
|\input{childdoc.def}|\\
|\childdocforwardprefix{final}{child}|
\end{tabular}
\end{center}
%

Note that when several versions of a main file and/or of each child file
are to be generated, it may be convenient to set up a |Makefile| or
shell script to automatise the process.

%%%%%%%%%%%%%%%%%%%%%%%%%%%%%%%%%%%%%%%%%%%%%%%%%%%%%%%%%%%%%%%%%%%%%%%%%%%%%%%%
\subsection{Command Line Processing}
\label{sec:commandline}

The effect of redirection files can also be achieved by invoking
the \LaTeX{} compiler with a more elaborate command line.
Most conveniently this should be done as part
of a shell script or a |Makefile|.

When using \textsf{childdoc} in the main file, the following
command lines effectively perform a redirection
(note that depending on the shell being used,
backslashes may have to be doubled: `|\|' $\to$ `|\\|'):
%
\begin{center}
|... -jobname "|\textit{target}|" |\\|"|[\textit{flags}]%
|\input{childdoc.def}\childdocforward[|\textit{main}|]{|\textit{dest}|}"|
\end{center}
%
Here \textit{target} is the name of the output file,
\textit{main} is the name of the main file
and \textit{dest} is the name of the main or child file to be processed
(all filenames without extensions).
The optional argument \textit{main} can be omitted
if \textit{main} matches \textit{dest}.
Optionally, compilation \textit{flags} can be defined via |\def| commands.
This command line makes the \TeX{} engine believe
it is compiling the file \textit{target}
whose content is specified as the latter parameter.
The provided code then forwards the processing to
\textit{main} or \textit{dest} as described in \secref{sec:forward}.

%%%%%%%%%%%%%%%%%%%%%%%%%%%%%%%%%%%%%%%%%%%%%%%%%%%%%%%%%%%%%%%%%%%%%%%%%%%%%%%%
\subsection{Include by Input}
\label{sec:input}

Including child documents by |\include| has some restrictions by design.
Most notably, the content of a child document always occupies
its own set of pages; pages cannot be shared between child documents.
Usually, this behaviour makes perfect sense
because each child document contain an essential part of the document.
However, in some situations it may be desirable to compose
a document from a collection of parts
without having mandatory page breaks between then.
For this case, the package
provides a mechanism to include parts
by |\input| which can also be processed individually.
However, by construction this mechanism
requires manual handling of the content to be output.

%%%%%%%%%%%%%%%%%%%%%%%%%%%%%%%%%%%%%%%%
\DescribeMacro{\ifchilddocmanual}
The main file should be prepared as usual, see \secref{sec:include}.
However, the document body must make a distinction
between processing of an individual part and of the main document, e.g.:
%
\begin{center}
\begin{tabular}{l}
|\ifchilddocmanual|\\
|\input{\childdocname}|\\
|\||else|\\
\textit{document body with }|\input{|\textit{part}|}|\\
|\||fi|
\end{tabular}
\end{center}
%
The conditional |\ifchilddocmanual| is true whenever
a part to be included by |\input| is being compiled,
and the name of the part is stored in |\childdocname|.

%%%%%%%%%%%%%%%%%%%%%%%%%%%%%%%%%%%%%%%%
\DescribeMacro{\childdocby}
Each part to be included by |\input| should start with:
%
\begin{center}
\begin{tabular}{l}
|\input{childdoc.def}|\\
|\childdocby{|\textit{main}|}|\\
\end{tabular}
\end{center}
%
The directive |\childdocby| is similar to |\childdocof|
described in \secref{sec:include},
but the subsequent selection of content must be done manually.
To that end, both |\ifchilddoc| and |\ifchilddocmanual|
will be true upon processing of a part,
and the name of the part is stored in |\childdocname|.
Note that |\jobname| will be set to the filename of the current part
so that each part receives an individual |.aux| file
that does not interfere with the |.aux| file(s) of the main document.
This behaviour can be altered by the alternative form
|\childdocby[*]{|\textit{main}|}| (with a non-empty optional argument)
which uses the |.aux| file of the main document
by setting |\jobname| to \textit{main}.

%%%%%%%%%%%%%%%%%%%%%%%%%%%%%%%%%%%%%%%%%%%%%%%%%%%%%%%%%%%%%%%%%%%%%%%%%%%%%%%%
\subsection{Driver Development}
\label{sec:driver}

The \textsf{childdoc} mechanism can also be use for the development
of definition files such as \LaTeX{} styles or classes.
This case differs from the above setup with multiple parts
included by |\include| in that no |\includeonly| should be invoked.
This can be achieved by starting the include file
(before |\ProvidesPackage|) with:
%
\begin{center}
\begin{tabular}{l}
|\input{childdoc.def}|\\
|\childdocforward{|\textit{main}|}|\\
\end{tabular}
\end{center}
%
or alternatively with:
%
\begin{center}
\begin{tabular}{l}
|\input{childdoc.def}|\\
|\childdocby{|\textit{main}|}|\\
\end{tabular}
\end{center}
%
Both forms have slightly different effects as described above.
The main file is prepared as usual, see \secref{sec:include}.

%%%%%%%%%%%%%%%%%%%%%%%%%%%%%%%%%%%%%%%%%%%%%%%%%%%%%%%%%%%%%%%%%%%%%%%%%%%%%%%%
\subsection{Legacy Detection}
\label{sec:detection}

The directive |\childdocmain| in the main file can detect
whether the complete document or merely a child is to be compiled
even without using the directive |\childdocof|.
This method is deprecated because it is less robust
and there is no compelling reason to use it;
it is merely provided for backward compatibility
and it may be removed in future versions.

If the detection mechanism is to be used,
it is mandatory to correctly specify
the filename of the main file as the argument of |\childdocmain|:
%
\begin{center}
\begin{tabular}{l}
|\input{childdoc.def}|\\
|\childdocmain{|\textit{main}|}|\\
\end{tabular}
\end{center}
%
If |\jobname| does not match the argument \textit{main} of |\childdocmain|,
it is assumed that |\jobname| points to the child file to be compiled.
When using |\childdocmain| with the main file specified as argument,
it suffices to start a child file
with just |\input{|\textit{main}|}|
without loading of the package and using |\childdocof|.
If instead all processing is done
with the appropriate \textsf{childdoc} directives,
the argument of \textit{main} of |\childdocmain| can be empty.

An alternative version of the command line processing described
in \secref{sec:commandline} using the detection mechanism reads:
%
\begin{center}
|... -jobname "|\textit{target}|" "|[\textit{flags}]%
[|\def\jobname{|\textit{dest}|}|]|\input{|\textit{main}|}"|
\end{center}

%%%%%%%%%%%%%%%%%%%%%%%%%%%%%%%%%%%%%%%%%%%%%%%%%%%%%%%%%%%%%%%%%%%%%%%%%%%%%%%%
\subsection{Manual Code}
\label{sec:manual}

In case one cannot be certain whether the definitions file |childdoc.def|
is installed on the target \TeX{} distribution
and one prefers not to ship it,
it is conceivable to paste a few relevant commands into the sources.

To that end, drop all statements |\input{childdoc.def}|
and perform the replacements as outlined below.
Instead of |\childdocmain{|\textit{main}|}| add the following code
to the top of the main file:
%
\begin{center}
\begin{tabular}{l}
|\||ifdefined\childdocname\endinput\||fi\newif\ifchilddoc|\\
|\edef\childdocname{\scantokens\expandafter{\jobname\noexpand}}|\\
|\def\childdocmain{|\textit{main}|}\||ifx\childdocmain\childdocname\||else|\\
|\childdoctrue\includeonly{\childdocname}\let\jobname\childdocmain\||fi|\\
\end{tabular}
\end{center}
%
Instead of |\childdocof{|\textit{main}|}| just include the main file
at the top of each child file:
%
\begin{center}
|\input{|\textit{main}|}|
\end{center}
%
A simple redirection |\childdocforward{|\textit{dest}|}| is achieved by:
%
\begin{center}
|\def\jobname{|\textit{dest}|}\input{\jobname}|
\end{center}
%
The redirection with prefix
|\childdocforwardprefix[|\textit{prefix}|]{|\textit{dest}|}|
is accomplished by:
%
\begin{center}
\begin{tabular}{l}
|{\edef\jobname{\scantokens\expandafter{\jobname\noexpand}}|\\
|\def\redirectjob |\textit{prefix}|#1~~~{\gdef\jobname{|\textit{dest}|#1}}|\\
|\expandafter\redirectjob\jobname~~~}\input{\jobname}|
\end{tabular}
\end{center}

In an alternative approach,
child documents can be compiled by a specific command line
without additional code or specific definitions:
%
\begin{center}
|... -jobname "|\textit{target}|" "|[\textit{flags}]%
|\includeonly{|\textit{dest}|}\input{|\textit{main}|}"|
\end{center}
%

%%%%%%%%%%%%%%%%%%%%%%%%%%%%%%%%%%%%%%%%%%%%%%%%%%%%%%%%%%%%%%%%%%%%%%%%%%%%%%%%
%%%%%%%%%%%%%%%%%%%%%%%%%%%%%%%%%%%%%%%%%%%%%%%%%%%%%%%%%%%%%%%%%%%%%%%%%%%%%%%%
\section{Information}

%%%%%%%%%%%%%%%%%%%%%%%%%%%%%%%%%%%%%%%%%%%%%%%%%%%%%%%%%%%%%%%%%%%%%%%%%%%%%%%%
\subsection{Copyright}

Copyright \copyright{} 2017--2018 Niklas Beisert

This work may be distributed and/or modified under the
conditions of the \LaTeX{} Project Public License, either version 1.3
of this license or (at your option) any later version.
The latest version of this license is in
  \url{http://www.latex-project.org/lppl.txt}
and version 1.3 or later is part of all distributions of \LaTeX{}
version 2005/12/01 or later.

This work has the LPPL maintenance status `maintained'.

The Current Maintainer of this work is Niklas Beisert.

This work consists of the files |README.txt|, |childdoc.ins| and |childdoc.dtx|
as well as the derived files |childdoc.def|, |cdocsamp.tex|
with |cdocsch1.tex|, |cdocsch2.tex|, |cdocspt3.tex|, |cdocspt4.tex|,
|cdocsdrf.tex|, |cdocsfn1.tex|, |cdocsfn2.tex|
as well as |childdoc.pdf|.

%%%%%%%%%%%%%%%%%%%%%%%%%%%%%%%%%%%%%%%%%%%%%%%%%%%%%%%%%%%%%%%%%%%%%%%%%%%%%%%%
\subsection{Files and Installation}

The package consists of the files:
%
\begin{center}
\begin{tabular}{ll}
    |README.txt|   & readme file \\
    |childdoc.ins| & installation file \\
    |childdoc.dtx| & source file \\
    |childdoc.def| & definition file \\
    |cdocsamp.tex| & sample main file \\
    |cdocsch1.tex| & sample include file \\
    |cdocsch2.tex| & sample include file \\
    |cdocspt3.tex| & sample part file \\
    |cdocspt4.tex| & sample part file \\
    |cdocsdrf.tex| & sample redirection file \\
    |cdocsfn1.tex| & sample redirection file \\
    |cdocsfn2.tex| & sample redirection file \\
    |childdoc.pdf| & manual
\end{tabular}
\end{center}
%
The distribution consists of the files
|README.txt|, |childdoc.ins| and |childdoc.dtx|.
%
\begin{itemize}
\item
Run (pdf)\LaTeX{} on |childdoc.dtx|
to compile the manual |childdoc.pdf| (this file).
\item
Run \LaTeX{} on |childdoc.ins| to create the definitions file |childdoc.def|
and the sample |cdocsamp.tex| with include files
|cdocsch1.tex|, |cdocsch2.tex|, |cdocspt3.tex|, |cdocspt4.tex|,
|cdocsdrf.tex|, |cdocsfn1.tex|, |cdocsfn2.tex|.
Then copy the file |childdoc.def| to an appropriate directory of your \LaTeX{}
distribution, e.g.\ \textit{texmf-root}|/tex/latex/childdoc|.
\end{itemize}

%%%%%%%%%%%%%%%%%%%%%%%%%%%%%%%%%%%%%%%%%%%%%%%%%%%%%%%%%%%%%%%%%%%%%%%%%%%%%%%%
\subsection{Related CTAN Packages}

There are several other packages which offer a similar functionality:
%
\begin{itemize}
\item
The packages
\href{http://ctan.org/pkg/docmute}{\textsf{docmute}},
\href{http://ctan.org/pkg/includex}{\textsf{includex}} and
\href{http://ctan.org/pkg/standalone}{\textsf{standalone}}
provide commands to include only the document body of
a child file thus allowing both files to be compiled individually.
\item
The packages \href{http://ctan.org/pkg/subdocs}{\textsf{subdocs}}
and \href{http://ctan.org/pkg/subfiles}{\textsf{subfiles}}
provide structures in which the main and child documents can be
encapsulated and allowing them to be compiled individually.
The inclusion mechanism is different from the conventional |\include|.
\item
The package \href{http://ctan.org/pkg/combine}{\textsf{combine}}
is an elaborate solution to combine several documents into one.
\end{itemize}
%
See also the CTAN topic \href{http://ctan.org/topic/subdocs}{\textsf{subdocs}}
for further related packages.
The present package differs from the above solutions in that
a document structure constructed with the conventional |\include| mechanism
just needs two extra commands at the top of every file
such that all constituent files can be compiled individually.

%%%%%%%%%%%%%%%%%%%%%%%%%%%%%%%%%%%%%%%%%%%%%%%%%%%%%%%%%%%%%%%%%%%%%%%%%%%%%%%%
%\subsection{Feature Suggestions}
%
%The following is a list of features which may be useful for future
%versions of this package:
%%
%\begin{itemize}
%\item
%\ldots
%\end{itemize}

%%%%%%%%%%%%%%%%%%%%%%%%%%%%%%%%%%%%%%%%%%%%%%%%%%%%%%%%%%%%%%%%%%%%%%%%%%%%%%%%
\subsection{Revision History}

%%%%%%%%%%%%%%%%%%%%%%%%%%%%%%%%%%%%%%%%
\paragraph{v2.0:} 2018/12/30

\begin{itemize}
\item
immediate forward processing
\item
added |\childdocby| mechanism
\item
manual restructured
\end{itemize}

%%%%%%%%%%%%%%%%%%%%%%%%%%%%%%%%%%%%%%%%
\paragraph{v1.6:} 2018/01/17

\begin{itemize}
\item
application for development of include files
\item
corrections to manual
\end{itemize}

%%%%%%%%%%%%%%%%%%%%%%%%%%%%%%%%%%%%%%%%
\paragraph{v1.5:} 2017/05/21

\begin{itemize}
\item
more complete structuring introduced
\item
|\childdocof| introduced
\item
|\childdoc| renamed to |\childdocmain|
\item
|\childredirect| renamed to |\childdocforward| and |\childdocforwardprefix|
and functionality expanded
\end{itemize}

%%%%%%%%%%%%%%%%%%%%%%%%%%%%%%%%%%%%%%%%
\paragraph{v1.0:} 2017/04/27

\begin{itemize}
\item
manual and install package
\item
first version published on CTAN
\end{itemize}

%%%%%%%%%%%%%%%%%%%%%%%%%%%%%%%%%%%%%%%%
\paragraph{v0.6:} 2017/04/26

\begin{itemize}
\item
redirection mechanism added
\end{itemize}

%%%%%%%%%%%%%%%%%%%%%%%%%%%%%%%%%%%%%%%%
\paragraph{v0.5:} 2017/04/26

\begin{itemize}
\item
functionality in definition file
\end{itemize}


%%%%%%%%%%%%%%%%%%%%%%%%%%%%%%%%%%%%%%%%%%%%%%%%%%%%%%%%%%%%%%%%%%%%%%%%%%%%%%%%
%%%%%%%%%%%%%%%%%%%%%%%%%%%%%%%%%%%%%%%%%%%%%%%%%%%%%%%%%%%%%%%%%%%%%%%%%%%%%%%%
%%%%%%%%%%%%%%%%%%%%%%%%%%%%%%%%%%%%%%%%%%%%%%%%%%%%%%%%%%%%%%%%%%%%%%%%%%%%%%%%
\appendix

\settowidth\MacroIndent{\rmfamily\scriptsize 000\ }

 \DocInput{childdoc.dtx}

\end{document}
%</driver>
% \fi
%
% %%%%%%%%%%%%%%%%%%%%%%%%%%%%%%%%%%%%%%%%%%%%%%%%%%%%%%%%%%%%%%%%%%%%%%%%%%%%%%
% %%%%%%%%%%%%%%%%%%%%%%%%%%%%%%%%%%%%%%%%%%%%%%%%%%%%%%%%%%%%%%%%%%%%%%%%%%%%%%
% \section{Sample}
%\iffalse
%<*samplemain>
%\fi
%
% The following presents a sample document
% with two chapters, two parts, a title page,
% a compile flag as well as three forwarding files to set the flag.
% It consists of eight |.tex| files:
% \begin{center}
% \begin{tabular}{ll}
% |cdocsamp.tex|&main file\\
% |cdocsch1.tex|&include file for chapter 1\\
% |cdocsch2.tex|&include file for chapter 2\\
% |cdocspt3.tex|&include file for part 3\\
% |cdocspt4.tex|&include file for part 4\\
% |cdocsdrf.tex|&forwarding file for main file in draft mode\\
% |cdocsfi1.tex|&forwarding file for final version of chapter 1\\
% |cdocsfi2.tex|&forwarding file for final version of chapter 2\\
% \end{tabular}
% \end{center}
% Each of the eight files can be compiled directly by the \LaTeX{} compiler.
%
% %%%%%%%%%%%%%%%%%%%%%%%%%%%%%%%%%%%%%%
% \paragraph{Main File.}
%
% The main file is called |cdocsamp.tex|.
%
% Load the \textsf{childdoc} definitions and
% declare the filename for the main document:
%    \begin{macrocode}
\input{childdoc.def}
\childdocmain{}
%    \end{macrocode}

% Optional override for |\version| flag:
%    \begin{macrocode}
%%\ifchilddoc\else\providecommand{\version}{draft}\fi
%    \end{macrocode}

% Define the default values for the |\version| flag
% (|final| for the main file and |draft| for childs):
%    \begin{macrocode}
\ifchilddoc
\providecommand{\version}{draft}
\else
\providecommand{\version}{final}
\fi
%    \end{macrocode}

% Load the standard document class:
%    \begin{macrocode}
\documentclass[12pt]{article}
%    \end{macrocode}

% Start the document body:
%    \begin{macrocode}
\begin{document}
%    \end{macrocode}

% Declare a title page.
% Print title, part of document being processed and version flag:
%    \begin{macrocode}
\addtocounter{page}{-1}
\begin{center}
{\LARGE\bfseries{}childdoc example\par}
\vspace{1cm}
\ifchilddoc
\ifchilddocmanual part\else chapter\fi:
`\childdocname' of `\childdocjob'\par
\else
main document: `\childdocjob'\par
\fi
version: \version\par
\end{center}
\newpage
%    \end{macrocode}

% Manually include selected file,
% otherwise process as usual:
%    \begin{macrocode}
\ifchilddocmanual
\section*{part `\childdocname'}
\input{\childdocname}
\else
%    \end{macrocode}

% Include the two chapters:
%    \begin{macrocode}
\include{cdocsch1}
\include{cdocsch2}
%    \end{macrocode}

% Include the two parts unless only chapters should be displayed:
%    \begin{macrocode}
\ifchilddoc\else
\section{part three}
\input{cdocspt3}
\section{part four}
\input{cdocspt4}
\fi
%    \end{macrocode}

% Process as usual until here:
%    \begin{macrocode}
\fi
%    \end{macrocode}

% End of document body:
%    \begin{macrocode}
\end{document}
%    \end{macrocode}
%\iffalse
%</samplemain>
%\fi
%
% %%%%%%%%%%%%%%%%%%%%%%%%%%%%%%%%%%%%%%
% \paragraph{Chapter Include Files.}
%
% The include files are called |cdocsch1.tex| and |cdocsch2.tex|.
%
%\iffalse
%<*samplechap1|samplechap2>
%\fi

% Optional override for |\version| flag:
%    \begin{macrocode}
%%\providecommand{\version}{final}
%    \end{macrocode}

% Include the main document:
%    \begin{macrocode}
\input{childdoc.def}
\childdocof{cdocsamp}
%    \end{macrocode}

%\iffalse
%</samplechap1|samplechap2>
%\fi
%
%\iffalse
%<*samplechap1>
%\fi
% Some text for chapter 1:
%    \begin{macrocode}
\section{one}
some text in chapter one
%    \end{macrocode}

%\iffalse
%</samplechap1>
%\fi
% Some text for chapter 2:
%\iffalse
%<*samplechap2>
%\fi
%    \begin{macrocode}
\section{two}
more text in chapter two
%    \end{macrocode}

%\iffalse
%</samplechap2>
%\fi
%
% %%%%%%%%%%%%%%%%%%%%%%%%%%%%%%%%%%%%%%
% \paragraph{Part Include Files.}
%
% The include files are called |cdocspt3.tex| and |cdocspt4.tex|.
%
%\iffalse
%<*samplepart3|samplepart4>
%\fi

% Optional override for |\version| flag:
%    \begin{macrocode}
%%\providecommand{\version}{final}
%    \end{macrocode}

% Include the main document:
%    \begin{macrocode}
\input{childdoc.def}
\childdocby{cdocsamp}
%    \end{macrocode}

%\iffalse
%</samplepart3|samplepart4>
%\fi
%
%\iffalse
%<*samplepart3>
%\fi
% Some text for part 3:
%    \begin{macrocode}
some text in part three
%    \end{macrocode}

%\iffalse
%</samplepart3>
%\fi
% Some text for part 4:
%\iffalse
%<*samplepart4>
%\fi
%    \begin{macrocode}
more text in part four
%    \end{macrocode}

%\iffalse
%</samplepart4>
%\fi
%
% %%%%%%%%%%%%%%%%%%%%%%%%%%%%%%%%%%%%%%
% \paragraph{Forwarding for a Complete Draft.}
%
% The following forwarding file |cdocsdrf.tex|
% compiles the main document in draft mode:
%\iffalse
%<*sampledraft>
%\fi
%    \begin{macrocode}
\def\version{draft}
\input{childdoc.def}
\childdocforward{cdocsamp}
%    \end{macrocode}

%\iffalse
%</sampledraft>
%\fi
%
% %%%%%%%%%%%%%%%%%%%%%%%%%%%%%%%%%%%%%%
% \paragraph{Forwarding for Final Version of the Chapters.}
%
% The following forwarding files |cdocsfn1.tex| and |cdocsfn2.tex|
% (with identical content)
% compile the final versions of the child documents
% |cdocsch1.tex| and |cdocsch2.tex|, respectively:
%\iffalse
%<*samplefinal>
%\fi
%    \begin{macrocode}
\def\version{final}
\input{childdoc.def}
\childdocforwardprefix[cdocsamp]{cdocsfn}{cdocsch}
%    \end{macrocode}

%\iffalse
%</samplefinal>
%\fi
%
% %%%%%%%%%%%%%%%%%%%%%%%%%%%%%%%%%%%%%%
% \paragraph{Command Line Processing.}
%
% The following three command lines generate the output files
% |cdocscld|, |cdocscl1| and |cdocscl2|
% which should be identical to
% |cdocsdrf|, |cdocsch1| and |cdocsfn2|, respectively:
% \begin{center}
% \begin{tabular}{l}
% |latex -jobname cdocscld \|\\
% |  "\def\version{draft}\input{childdoc.def}\childdocforward{cdocsamp}"|\\
% |latex -jobname cdocscl1 \|\\
% |  "\input{childdoc.def}\childdocforward[cdocsamp]{cdocsch1}"|\\
% |latex -jobname cdocscl2 \|\\
% |  "\def\version{final}\input{childdoc.def}\childdocforward{cdocsch2}"|
% \end{tabular}
% \end{center}
% Note that the trailing backslash on each first line
% merely continues the input to the second line
% (for convenient cut ant paste).
% Furthermore, the command |latex| can be replaced by any
% of its alternative versions such as |pdflatex|.
%
% %%%%%%%%%%%%%%%%%%%%%%%%%%%%%%%%%%%%%%%%%%%%%%%%%%%%%%%%%%%%%%%%%%%%%%%%%%%%%%
% %%%%%%%%%%%%%%%%%%%%%%%%%%%%%%%%%%%%%%%%%%%%%%%%%%%%%%%%%%%%%%%%%%%%%%%%%%%%%%
% \section{Implementation}
%\iffalse
%<*package>
%\fi
%
% This section describes the definitions file |childdoc.def|.

% The definitions cannot be loaded using |\usepackage| or |\RequirePackage|
% which has a mechanism to prevent loading a style file more than once.
% When loading the definitions by means of |\input|
% multiple instances have to be prevented manually:
%\iffalse
%This code needs to be before the `\ProvidesFile' directive
%which is defined at the beginning of this file.
%Therefore it is also placed there and commented out here.
%</package>
%<*discard>
%\fi
%    \begin{macrocode}
\ifdefined\childdocmain\endinput\fi
%    \end{macrocode}
%\iffalse
%</discard>
%<*package>
%\fi
%
% \macro{\ifchilddoc}
% \macro{\ifchilddocmanual}
% The conditional |\ifchilddoc| tells whether a
% child (true) or main (false) document is being compiled.
% The conditional |\ifchilddocmanual| tells whether
% the |\includeonly| mechanism is used (false) or
% the selection of child files must be performed manually (true).
% The definitions initialise to false:
%    \begin{macrocode}
\newif\ifchilddoc
\newif\ifchilddocmanual
%    \end{macrocode}

% \macro{\childdocname}
% \macro{\childdocjob}
% The macro |\childdocname| stores the name of the main document
% to be compiled. The macro |\childdocjob| stores the name of
% the document on which the \LaTeX{} compiler was originally invoked.
% The content of |\jobname| cannot be compared
% to filenames specified in the source due to different catcodes.
% The following code rescans |\jobname|, stores the result
% in |\childdocname| and saves a copy in |\childdocjob|:
%    \begin{macrocode}
\edef\childdocname{\scantokens\expandafter{\jobname\noexpand}}
\let\childdocjob\childdocname
%    \end{macrocode}

% \macro{\childdocdisable}
% The macro |\childdocdisable| prevents the main file
% from being processed more than once.
% At this stage, the main document command |\childdocmain|
% is assumed to be called once again where it should do nothing.
% Any subsequent call to it should prevent
% a secondary processing of the main document
% It overwrites the forwarding commands
% |\childdocof| and |\childdocforward|
% with empty macros to prevent further inclusions of the main document:
%    \begin{macrocode}
\newcommand{\childdocdisable}
{
  \renewcommand{\childdocmain}[1]{\renewcommand{\childdocmain}[1]{\endinput}}
  \renewcommand{\childdocof}[1]{}
  \renewcommand{\childdocby}[2][]{}
  \renewcommand{\childdocforward}[2][]{}
  \renewcommand{\childdocdisable}{}
}
%    \end{macrocode}

% \macro{\childdocmain}
% The macro |\childdocmain| is to be called at the top of the main file
% with nothing or the main filename (without extension) as argument.
% First, it breaks loops.
% If the argument is not empty and does not match |\childdocname|
% (which is set by the first inclusion of |childdoc.def|),
% |\ifchilddoc| is set to true, |\includeonly| is applied to the child file
% and |\jobname| is set to the main file
% (for proper handling of |.aux| files):
%    \begin{macrocode}
\newcommand{\childdocmain}[1]
{
  \childdocdisable\childdocmain{}
  \if?#1?\else
    \begingroup
      \def\childdoctmp{#1}
      \ifx\childdoctmp\childdocname
        \def\childdoctmp{}
      \else
        \def\childdoctmp
        {
          \childdoctrue
          \includeonly{\childdocname}
          \def\childdocjob{#1}
          \def\jobname{#1}
        }
      \fi
      \expandafter
    \endgroup
    \childdoctmp
  \fi
}
%    \end{macrocode}

% \macro{\childdocof}
% The command |\childdocof| redirects
% compilation to the main file |#1|.
%    \begin{macrocode}
\newcommand{\childdocof}[1]
{
  \childdocdisable
  \childdoctrue
  \includeonly{\childdocname}
  \def\jobname{#1}
  \def\childdocjob{#1}
  \input{#1}
}
%    \end{macrocode}

% \macro{\childdocby}
% The command |\childdocby| ....
%    \begin{macrocode}
\newcommand{\childdocby}[2][]
{
  \childdocdisable
  \childdoctrue
  \childdocmanualtrue
  \if?#1?\else
    \def\jobname{#2}
  \fi
  \def\childdocjob{#2}
  \input{#2}
  \endinput
}
%    \end{macrocode}

% \macro{\childdocforward}
% The command |\childdocforward| redirects
% compilation to the main file or
% (if the optional argument is given) a child file.
% Parameters are set as if the main file
% or a child file starting with |\childdocof| was compiled.
% Then compilation is handed over to the main file:
%    \begin{macrocode}
\newcommand{\childdocforward}[2][]
{
  \begingroup
    \if?#1?
      \def\childdoctmp
      {
        \def\childdocname{#2}
        \def\childdocjob{#2}
        \def\jobname{#2}
        \input{#2}
        \endinput
      }
    \else
      \def\childdoctmp
      {
        \childdocdisable
        \def\childdocname{#2}
        \childdoctrue
        \includeonly{#2}
        \def\childdocjob{#1}
        \def\jobname{#1}
        \input{#1}
        \endinput
      }
    \fi
    \expandafter
  \endgroup
  \childdoctmp
}
%    \end{macrocode}

% \macro{\childdocforwardprefix}
% The command |\childdocforwardprefix| redirects
% compilation to the main or a child file by means of a pattern.
% The prefix |#1| in the current filename is replaced by |#2|
% and the suffix of the current filename is kept
% (it is assumed that the filename does not contain the substring `|~~~|'
% which is used as a delimiter).
% Compilation is handed over to the new file by |\childdocforward|:
%    \begin{macrocode}
\newcommand{\childdocforwardprefix}[3][]
{
  \begingroup
    \def\childdocextract #2##1~~~{\def\childdoctmp{\childdocforward[#1]{#3##1}}}
    \expandafter\childdocextract\childdocname~~~
    \expandafter
  \endgroup
  \childdoctmp
}
%    \end{macrocode}

% \macro{\childdoc}
% The deprecated macro |\childdoc| is a legacy version of |\childdocmain|:
%    \begin{macrocode}
\newcommand{\childdoc}{\childdocmain}
%    \end{macrocode}

% \macro{\childdocredirect}
% The deprecated macro |\childdocredirect| is a legacy version
% of |\childdocforward| and |\childdocforwardprefix|:
%    \begin{macrocode}
\newcommand{\childdocredirect}[2][]
{
  \begingroup
    \if?#1?
      \def\childdoctmp{\childdocforward{#2}}
    \else
      \def\childdoctmp{\childdocforwardprefix{#1}{#2}}
    \fi
    \expandafter
  \endgroup
  \childdoctmp
}
%    \end{macrocode}

%\iffalse
%</package>
%\fi
%
\endinput
|\\
|\childdocforward{|\textit{main}|}|
\end{tabular}
\end{center}
%
Likewise, the following files |final|\textit{nn}|.tex|
compile the final version of the child document
|child|\textit{nn}|.tex|:
%
\begin{center}
\begin{tabular}{l}
|\def\version{final}|\\
|% \iffalse
%
% childdoc.dtx Copyright (C) 2017-2018 Niklas Beisert
%
% This work may be distributed and/or modified under the
% conditions of the LaTeX Project Public License, either version 1.3
% of this license or (at your option) any later version.
% The latest version of this license is in
%   http://www.latex-project.org/lppl.txt
% and version 1.3 or later is part of all distributions of LaTeX
% version 2005/12/01 or later.
%
% This work has the LPPL maintenance status `maintained'.
%
% The Current Maintainer of this work is Niklas Beisert.
%
% This work consists of the files childdoc.dtx and childdoc.ins
% and the derived files childdoc.def and cdocsamp.tex with
% cdocsch1.tex, cdocsch2.tex, cdocsdrf.tex, cdocsfn1.tex, cdocsfn2.tex.
%
%<package>\ifdefined\childdocmain\endinput\fi
%<package>\ProvidesFile{childdoc.def}[2018/12/30 v2.0 child document driver]
%<samplemain>\ProvidesFile{cdocsamp.tex}[2018/12/30 v2.0 sample for childdoc]
%<*driver>
%\ProvidesFile{childdoc.drv}[2018/12/30 v2.0 childdoc reference manual file]
\PassOptionsToClass{10pt,a4paper}{article}
\documentclass{ltxdoc}

\usepackage[margin=35mm]{geometry}
\usepackage{hyperref}
\usepackage{hyperxmp}
\usepackage[usenames]{color}

\hypersetup{colorlinks=true}
\hypersetup{pdfstartview=FitH}
\hypersetup{pdfpagemode=UseNone}
\hypersetup{pdfsource={}}
\hypersetup{pdflang={en-UK}}
\hypersetup{pdfcopyright={Copyright 2017-2018 Niklas Beisert.
  This work may be distributed and/or modified under the
  conditions of the LaTeX Project Public License, either version 1.3
  of this license or (at your option) any later version.}}
\hypersetup{pdflicenseurl={http://www.latex-project.org/lppl.txt}}
\hypersetup{pdfcontactaddress={ETH Zurich, ITP, HIT K,
  Wolfgang-Pauli-Strasse 27}}
\hypersetup{pdfcontactpostcode={8093}}
\hypersetup{pdfcontactcity={Zurich}}
\hypersetup{pdfcontactcountry={Switzerland}}
\hypersetup{pdfcontactemail={nbeisert@itp.phys.ethz.ch}}
\hypersetup{pdfcontacturl={http://people.phys.ethz.ch/\xmptilde nbeisert/}}

\newcommand{\secref}[1]{\hyperref[#1]{section \ref*{#1}}}

\parskip1ex
\parindent0pt
\let\olditemize\itemize
\def\itemize{\olditemize\parskip0pt}

\begin{document}

\title{The \textsf{childdoc} Package}
\hypersetup{pdftitle={The childdoc Package}}
\author{Niklas Beisert\\[2ex]
  Institut f\"ur Theoretische Physik\\
  Eidgen\"ossische Technische Hochschule Z\"urich\\
  Wolfgang-Pauli-Strasse 27, 8093 Z\"urich, Switzerland\\[1ex]
  \href{mailto:nbeisert@itp.phys.ethz.ch}
  {\texttt{nbeisert@itp.phys.ethz.ch}}}
\hypersetup{pdfauthor={Niklas Beisert}}
\hypersetup{pdfsubject={Manual for the LaTeX2e Package childdoc}}
\date{30 December 2018, \textsf{v2.0}}
\maketitle

\begin{abstract}\noindent
\textsf{childdoc} is a \LaTeXe{} package
that enables the direct compilation
of document sections included by |\include|
to individual files.
\end{abstract}

\begingroup
\parskip0ex
\tableofcontents
\endgroup

%%%%%%%%%%%%%%%%%%%%%%%%%%%%%%%%%%%%%%%%%%%%%%%%%%%%%%%%%%%%%%%%%%%%%%%%%%%%%%%%
%%%%%%%%%%%%%%%%%%%%%%%%%%%%%%%%%%%%%%%%%%%%%%%%%%%%%%%%%%%%%%%%%%%%%%%%%%%%%%%%
\section{Introduction}

\LaTeX{} provides a mechanism to structure a large document (such as a book)
into a main file and several child files (containing the chapters)
using the |\include| command.
This mechanism is beneficial for documents
which span hundreds of pages in order to
make the source file(s) more manageable.
Moreover, compilation can be restricted to
selected child files by means of the |\includeonly| command.
The latter feature can be used to reduce the compilation time while editing
(this was significantly more useful in the earlier days of \LaTeX{})
or to generate a smaller document which is easier to navigate.
Another application of |\includeonly| is to generate
documents consisting of selected parts of the complete document.

However, there are a few drawbacks of the plain |\include| mechanism:
\begin{itemize}
\item
The child files cannot be compiled on their own,
they can only be compiled via the main file.
A naive editing environment
(such as a text editor with an option
to have the current file processed by \LaTeX)
may require one to switch to the main file before compiling;
attempting to compile the child file produces errors.
\item
The main file must be modified (each time)
to adjust the |\includeonly| command
to the present needs. This easily leaves the main file in a messy state.
\item
The generated document will always carry the filename
of the main document. This is inconvenient if
several child files are to be compiled and
to be kept for distribution.
\end{itemize}

The present package provides a simple interface
to make child files individually compilable by \LaTeX{}.
Compiling a child file then has the same effect as compiling
the main file with an |\includeonly| command
to select the appropriate child.
Moreover the generated document will carry the name of the child
rather than the main file.
This resolves all three above issues.

This feature is meant to make the editing of books,
thesis documents and lecture notes somewhat more convenient.
However, the package can also be used efficiently for
composing a series of documents (such as exercise sheets)
which are typically distributed individually.
It then assists the author in generating the individual documents
(potentially in different versions)
as well as a document containing the collected series.
Another application is in developing style files
or other kinds of included material
where compilation of the style file could redirect
to a sample or test file.

%%%%%%%%%%%%%%%%%%%%%%%%%%%%%%%%%%%%%%%%%%%%%%%%%%%%%%%%%%%%%%%%%%%%%%%%%%%%%%%%
%%%%%%%%%%%%%%%%%%%%%%%%%%%%%%%%%%%%%%%%%%%%%%%%%%%%%%%%%%%%%%%%%%%%%%%%%%%%%%%%
\section{Usage}

First of all, the package \textsf{childdoc} is \emph{not} a standard
\LaTeXe{} |.sty| style file! Therefore it needs to be invoked in
a non-standard way.

%%%%%%%%%%%%%%%%%%%%%%%%%%%%%%%%%%%%%%%%%%%%%%%%%%%%%%%%%%%%%%%%%%%%%%%%%%%%%%%%
\subsection{Included Files}
\label{sec:include}

%%%%%%%%%%%%%%%%%%%%%%%%%%%%%%%%%%%%%%%%
\DescribeMacro{\childdocmain}
To use the package, add the commands
\begin{center}
\begin{tabular}{l}
|\input{childdoc.def}|\\
|\childdocmain{}|\\
\end{tabular}
\end{center}
at the very top of the main \LaTeX{} file,
in particular \emph{before} the |\documentclass| statement!
The argument of |\childdocmain| should be left empty
(but it must be present).

%%%%%%%%%%%%%%%%%%%%%%%%%%%%%%%%%%%%%%%%
\DescribeMacro{\childdocof}
Furthermore, add the commands
\begin{center}
\begin{tabular}{l}
|\input{childdoc.def}|\\
|\childdocof{|\textit{main}|}|\\
\end{tabular}
\end{center}
at the top of every child file \textit{child}
which is included by |\include{|\textit{child}|}|
from within the main file
(or at least for those files to be compiled individually).
The argument \textit{main} must be the filename of the main file.

There are a couple of
considerations in setting up the main and child documents:

%%%%%%%%%%%%%%%%%%%%%%%%%%%%%%%%%%%%%%%%
\paragraph{Restrictions.}

Please note the following restrictions:
\begin{itemize}
\item
|\childdocmain| must be called with one argument \textit{main}
to ensure compatibility with earlier version of the package.
It must either be empty (|\childdocmain{}|)
or precisely match the filename of the main file in which it is specified.
See \secref{sec:detection} for further information.
\item
The filename \textit{main} must be specified without the |.tex| extension.
\item
The filename \textit{main} is case sensitive
(even in case-insensitive file systems)
due to internal string comparison.
\item
The argument \textit{main} should be fully expanded, it cannot be a macro.
\item
Subdirectories and special characters should be avoided in filenames.
\item
The command |\childdocmain{|\textit{main}|}| must be followed by a whitespace.
It should not be followed immediately by another command
or by a comment mark `|%|'.
This is because the \TeX{} parser reads the token immediately following
the argument of |\childdocmain| and puts it
at the beginning of every child section;
however, a white\-space is ignored.
\end{itemize}

%%%%%%%%%%%%%%%%%%%%%%%%%%%%%%%%%%%%%%%%
\paragraph{Content of Main File.}

It is advisable to place all content in the child files included by |\include|.
Any output contained in the main file will appear in all child documents
unless suppressed manually;
it cannot be suppressed automatically by the |\includeonly| directive
and thus should normally be avoided.
A method to include some content in the main file
by means of conditional processing is described in \secref{sec:conditional}.

%%%%%%%%%%%%%%%%%%%%%%%%%%%%%%%%%%%%%%%%
\paragraph{Page Numbering.}

When only a part of the document is compiled,
the appropriate numbering of pages
(as well as other status parameters)
is determined from the |.aux| files.
The latter contain information from previous passes.
However this information needs to propagate through
all intermediate child documents.
Therefore the page numbering in child documents may well
be inconsistent until the complete document is compiled at least once.

A useful (if unconventional) way to always ensure a consistent
page numbering is to restart the numbering in each child document
and denote the pages by `\textit{child}|.|\textit{page}'
where \textit{child} represents the chapter/section number of the child file.
This can be achieved by the command
|\numberwithin{page}{|\textit{child}|}|
of the \textsf{amsmath} package
where \textit{child} can be |chapter| or |section|
depending on the chosen structuring.
Alternatively, one can modify the macro |\thepage| appropriately
and reset the counter |page| at the start of each child file.

%%%%%%%%%%%%%%%%%%%%%%%%%%%%%%%%%%%%%%%%%%%%%%%%%%%%%%%%%%%%%%%%%%%%%%%%%%%%%%%%
\subsection{Conditional Processing}
\label{sec:conditional}

The package provides a mechanism to compile different versions
of a document. To customise the versions further some conditional processing
can come in handy to distinguish which version is being compiled.
The package provides two macros to describe the compilation context:

%%%%%%%%%%%%%%%%%%%%%%%%%%%%%%%%%%%%%%%%
\DescribeMacro{\ifchilddoc}
The conditional |\ifchilddoc| distinguishes between the compilation of
child documents and the main document:
%
\begin{center}
|\ifchilddoc |\textit{child-code}| |[|\||else |\textit{main-code}]| \||fi|
\end{center}

%%%%%%%%%%%%%%%%%%%%%%%%%%%%%%%%%%%%%%%%
\DescribeMacro{\childdocname}
\DescribeMacro{\childdocjob}
The macro |\childdocname| contains the filename (without extension)
of the main or child file being processed.
Note that |\childdocjob| will always contain the name of the main file.

%%%%%%%%%%%%%%%%%%%%%%%%%%%%%%%%%%%%%%%%
\paragraph{Title Page.}

Conditional processing can be used to include a title or banner page
in the main document when proper precautions are taken.
Importantly, the code in the main file should ensure that the page counter
(as well as other status parameters which are stored in the |.aux| files)
takes the same value after the conditional processing.
Otherwise the page numbers may take divergent values
depending on which part is compiled.

For example, a title page could be declared by:
%
\begin{center}
\begin{tabular}{l}
|\ifchilddoc\||else|\\
|\addtocounter{page}{-1}|\\
\textit{code for title page}\\
|\newpage|\\
|\||fi|
\end{tabular}
\end{center}
%
A banner page for the child documents can be generated by:
%
\begin{center}
\begin{tabular}{l}
|\ifchilddoc|\\
|\addtocounter{page}{-1}|\\
\textit{code for banner page}\\
|\newpage|\\
|\||fi|
\end{tabular}
\end{center}
%
Here one could write a message such as:
\begin{center}
|This is the part \childdocname{} of \childdocjob{}.|
\end{center}

%%%%%%%%%%%%%%%%%%%%%%%%%%%%%%%%%%%%%%%%%%%%%%%%%%%%%%%%%%%%%%%%%%%%%%%%%%%%%%%%
\subsection{Flags}
\label{sec:flags}

The package makes it easy to generate different versions
of the main or child documents.
To this end compilation flags can be defined
and assigned different default values.
They will be particularly useful in conjunction
with the forwarding mechanism described in \secref{sec:forward}.

For example, it may be useful to have a flag |\version|
which can be set to |draft| or |final|.
The document source will contain some conditional code
depending on the value of |\version|.
Suppose further, the flag should default to |final| for the main file
and to |draft| for child files
which is a natural assignment for editing the document.
This is achieved by placing the following code
in the preamble of the main document
(below the |\childdocmain| directive):
%
\begin{center}
\begin{tabular}{l}
|\ifchilddoc|\\
|\providecommand{\version}{draft}|\\
|\||else|\\
|\providecommand{\version}{final}|\\
|\||fi|
\end{tabular}
\end{center}
%
The definition by |\providecommand| makes sure
that previous definitions are not overwritten.
Further statements |\providecommand{\version}{...}|
can thus be added before the above code to override it.

For the main file, one might add a line
(between |\childdocmain| and the above block)
%
\begin{center}
|%\ifchilddoc\||else\providecommand{\version}{draft}\||fi|
\end{center}
%
which can be uncommented to produce a draft version.
Likewise one can add a line to the very top of a child file
(above the |\childdocof{|\textit{main}|}| directive)
%
\begin{center}
|%\providecommand{\version}{final}|
\end{center}
%
which can be uncommented to produce the final version of this child document.

%%%%%%%%%%%%%%%%%%%%%%%%%%%%%%%%%%%%%%%%%%%%%%%%%%%%%%%%%%%%%%%%%%%%%%%%%%%%%%%%
\subsection{Forwarding}
\label{sec:forward}

Different versions of the main or child documents
using compilation flags as described in \secref{sec:flags}
can be (permanently) stored in different files
for convenient compilation, viewing and distribution.
To this end, the package defines a command
to pass on compilation to a different file:

%%%%%%%%%%%%%%%%%%%%%%%%%%%%%%%%%%%%%%%%
\DescribeMacro{\childdocforward}
The command |\childdocforward| redirects processing to
another source file:
%
\begin{center}
\begin{tabular}{l}
|\input{childdoc.def}|\\
|\childdocforward[|\textit{main}|]{|\textit{dest}|}|\\
\end{tabular}
\end{center}
%
The argument \textit{dest} is the destination file
(without extension).
It should be the main file or one of the child files.
Note that further \textsf{childdoc} directives
such as |\childdocof| and |\childdocforward|
in the indicated file will be processed in this form.
The optional argument \textit{main}
passes on directly to the main file \textit{main}
while pretending to compile the child \textit{dest}.
This form behaves as if \textit{dest}
issues |\childdocof{|\textit{main}|}| right away,
and no further \textsf{childdoc} directives will be processed.

%%%%%%%%%%%%%%%%%%%%%%%%%%%%%%%%%%%%%%%%
\DescribeMacro{\...prefix}
In the alternative form |\childdocforwardprefix|,
%
\begin{center}
\begin{tabular}{l}
|\input{childdoc.def}|\\
|\childdocforwardprefix[|\textit{main}|]{|\textit{prefix}|}{|\textit{dest}|}|
\end{tabular}
\end{center}
%
the destination file is determined by a pattern
depending on the current file:
To make this work, the current file must be called
`{\textit{prefix}\hspace{0.2em}\textit{suffix}}'
with \textit{prefix} matching precisely the argument.
Processing is then passed on to the file
`{\textit{dest}\hspace{0.2em}\textit{suffix}}'.
Surely, the same effect is achieved by
directly specifying the
argument `{\textit{dest}\hspace{0.2em}\textit{suffix}}'
in the first form.
However, that requires to set up a different file
for each child. With the alternative form of the command
all these files can have exactly the same content
which simplifies setting them up and maintaining them.

For example, the following file |draft.tex|
with a compilation flag |\version| as described in \secref{sec:flags}
compiles the main document as a draft:
%
\begin{center}
\begin{tabular}{l}
|\def\version{draft}|\\
|\input{childdoc.def}|\\
|\childdocforward{|\textit{main}|}|
\end{tabular}
\end{center}
%
Likewise, the following files |final|\textit{nn}|.tex|
compile the final version of the child document
|child|\textit{nn}|.tex|:
%
\begin{center}
\begin{tabular}{l}
|\def\version{final}|\\
|\input{childdoc.def}|\\
|\childdocforwardprefix{final}{child}|
\end{tabular}
\end{center}
%

Note that when several versions of a main file and/or of each child file
are to be generated, it may be convenient to set up a |Makefile| or
shell script to automatise the process.

%%%%%%%%%%%%%%%%%%%%%%%%%%%%%%%%%%%%%%%%%%%%%%%%%%%%%%%%%%%%%%%%%%%%%%%%%%%%%%%%
\subsection{Command Line Processing}
\label{sec:commandline}

The effect of redirection files can also be achieved by invoking
the \LaTeX{} compiler with a more elaborate command line.
Most conveniently this should be done as part
of a shell script or a |Makefile|.

When using \textsf{childdoc} in the main file, the following
command lines effectively perform a redirection
(note that depending on the shell being used,
backslashes may have to be doubled: `|\|' $\to$ `|\\|'):
%
\begin{center}
|... -jobname "|\textit{target}|" |\\|"|[\textit{flags}]%
|\input{childdoc.def}\childdocforward[|\textit{main}|]{|\textit{dest}|}"|
\end{center}
%
Here \textit{target} is the name of the output file,
\textit{main} is the name of the main file
and \textit{dest} is the name of the main or child file to be processed
(all filenames without extensions).
The optional argument \textit{main} can be omitted
if \textit{main} matches \textit{dest}.
Optionally, compilation \textit{flags} can be defined via |\def| commands.
This command line makes the \TeX{} engine believe
it is compiling the file \textit{target}
whose content is specified as the latter parameter.
The provided code then forwards the processing to
\textit{main} or \textit{dest} as described in \secref{sec:forward}.

%%%%%%%%%%%%%%%%%%%%%%%%%%%%%%%%%%%%%%%%%%%%%%%%%%%%%%%%%%%%%%%%%%%%%%%%%%%%%%%%
\subsection{Include by Input}
\label{sec:input}

Including child documents by |\include| has some restrictions by design.
Most notably, the content of a child document always occupies
its own set of pages; pages cannot be shared between child documents.
Usually, this behaviour makes perfect sense
because each child document contain an essential part of the document.
However, in some situations it may be desirable to compose
a document from a collection of parts
without having mandatory page breaks between then.
For this case, the package
provides a mechanism to include parts
by |\input| which can also be processed individually.
However, by construction this mechanism
requires manual handling of the content to be output.

%%%%%%%%%%%%%%%%%%%%%%%%%%%%%%%%%%%%%%%%
\DescribeMacro{\ifchilddocmanual}
The main file should be prepared as usual, see \secref{sec:include}.
However, the document body must make a distinction
between processing of an individual part and of the main document, e.g.:
%
\begin{center}
\begin{tabular}{l}
|\ifchilddocmanual|\\
|\input{\childdocname}|\\
|\||else|\\
\textit{document body with }|\input{|\textit{part}|}|\\
|\||fi|
\end{tabular}
\end{center}
%
The conditional |\ifchilddocmanual| is true whenever
a part to be included by |\input| is being compiled,
and the name of the part is stored in |\childdocname|.

%%%%%%%%%%%%%%%%%%%%%%%%%%%%%%%%%%%%%%%%
\DescribeMacro{\childdocby}
Each part to be included by |\input| should start with:
%
\begin{center}
\begin{tabular}{l}
|\input{childdoc.def}|\\
|\childdocby{|\textit{main}|}|\\
\end{tabular}
\end{center}
%
The directive |\childdocby| is similar to |\childdocof|
described in \secref{sec:include},
but the subsequent selection of content must be done manually.
To that end, both |\ifchilddoc| and |\ifchilddocmanual|
will be true upon processing of a part,
and the name of the part is stored in |\childdocname|.
Note that |\jobname| will be set to the filename of the current part
so that each part receives an individual |.aux| file
that does not interfere with the |.aux| file(s) of the main document.
This behaviour can be altered by the alternative form
|\childdocby[*]{|\textit{main}|}| (with a non-empty optional argument)
which uses the |.aux| file of the main document
by setting |\jobname| to \textit{main}.

%%%%%%%%%%%%%%%%%%%%%%%%%%%%%%%%%%%%%%%%%%%%%%%%%%%%%%%%%%%%%%%%%%%%%%%%%%%%%%%%
\subsection{Driver Development}
\label{sec:driver}

The \textsf{childdoc} mechanism can also be use for the development
of definition files such as \LaTeX{} styles or classes.
This case differs from the above setup with multiple parts
included by |\include| in that no |\includeonly| should be invoked.
This can be achieved by starting the include file
(before |\ProvidesPackage|) with:
%
\begin{center}
\begin{tabular}{l}
|\input{childdoc.def}|\\
|\childdocforward{|\textit{main}|}|\\
\end{tabular}
\end{center}
%
or alternatively with:
%
\begin{center}
\begin{tabular}{l}
|\input{childdoc.def}|\\
|\childdocby{|\textit{main}|}|\\
\end{tabular}
\end{center}
%
Both forms have slightly different effects as described above.
The main file is prepared as usual, see \secref{sec:include}.

%%%%%%%%%%%%%%%%%%%%%%%%%%%%%%%%%%%%%%%%%%%%%%%%%%%%%%%%%%%%%%%%%%%%%%%%%%%%%%%%
\subsection{Legacy Detection}
\label{sec:detection}

The directive |\childdocmain| in the main file can detect
whether the complete document or merely a child is to be compiled
even without using the directive |\childdocof|.
This method is deprecated because it is less robust
and there is no compelling reason to use it;
it is merely provided for backward compatibility
and it may be removed in future versions.

If the detection mechanism is to be used,
it is mandatory to correctly specify
the filename of the main file as the argument of |\childdocmain|:
%
\begin{center}
\begin{tabular}{l}
|\input{childdoc.def}|\\
|\childdocmain{|\textit{main}|}|\\
\end{tabular}
\end{center}
%
If |\jobname| does not match the argument \textit{main} of |\childdocmain|,
it is assumed that |\jobname| points to the child file to be compiled.
When using |\childdocmain| with the main file specified as argument,
it suffices to start a child file
with just |\input{|\textit{main}|}|
without loading of the package and using |\childdocof|.
If instead all processing is done
with the appropriate \textsf{childdoc} directives,
the argument of \textit{main} of |\childdocmain| can be empty.

An alternative version of the command line processing described
in \secref{sec:commandline} using the detection mechanism reads:
%
\begin{center}
|... -jobname "|\textit{target}|" "|[\textit{flags}]%
[|\def\jobname{|\textit{dest}|}|]|\input{|\textit{main}|}"|
\end{center}

%%%%%%%%%%%%%%%%%%%%%%%%%%%%%%%%%%%%%%%%%%%%%%%%%%%%%%%%%%%%%%%%%%%%%%%%%%%%%%%%
\subsection{Manual Code}
\label{sec:manual}

In case one cannot be certain whether the definitions file |childdoc.def|
is installed on the target \TeX{} distribution
and one prefers not to ship it,
it is conceivable to paste a few relevant commands into the sources.

To that end, drop all statements |\input{childdoc.def}|
and perform the replacements as outlined below.
Instead of |\childdocmain{|\textit{main}|}| add the following code
to the top of the main file:
%
\begin{center}
\begin{tabular}{l}
|\||ifdefined\childdocname\endinput\||fi\newif\ifchilddoc|\\
|\edef\childdocname{\scantokens\expandafter{\jobname\noexpand}}|\\
|\def\childdocmain{|\textit{main}|}\||ifx\childdocmain\childdocname\||else|\\
|\childdoctrue\includeonly{\childdocname}\let\jobname\childdocmain\||fi|\\
\end{tabular}
\end{center}
%
Instead of |\childdocof{|\textit{main}|}| just include the main file
at the top of each child file:
%
\begin{center}
|\input{|\textit{main}|}|
\end{center}
%
A simple redirection |\childdocforward{|\textit{dest}|}| is achieved by:
%
\begin{center}
|\def\jobname{|\textit{dest}|}\input{\jobname}|
\end{center}
%
The redirection with prefix
|\childdocforwardprefix[|\textit{prefix}|]{|\textit{dest}|}|
is accomplished by:
%
\begin{center}
\begin{tabular}{l}
|{\edef\jobname{\scantokens\expandafter{\jobname\noexpand}}|\\
|\def\redirectjob |\textit{prefix}|#1~~~{\gdef\jobname{|\textit{dest}|#1}}|\\
|\expandafter\redirectjob\jobname~~~}\input{\jobname}|
\end{tabular}
\end{center}

In an alternative approach,
child documents can be compiled by a specific command line
without additional code or specific definitions:
%
\begin{center}
|... -jobname "|\textit{target}|" "|[\textit{flags}]%
|\includeonly{|\textit{dest}|}\input{|\textit{main}|}"|
\end{center}
%

%%%%%%%%%%%%%%%%%%%%%%%%%%%%%%%%%%%%%%%%%%%%%%%%%%%%%%%%%%%%%%%%%%%%%%%%%%%%%%%%
%%%%%%%%%%%%%%%%%%%%%%%%%%%%%%%%%%%%%%%%%%%%%%%%%%%%%%%%%%%%%%%%%%%%%%%%%%%%%%%%
\section{Information}

%%%%%%%%%%%%%%%%%%%%%%%%%%%%%%%%%%%%%%%%%%%%%%%%%%%%%%%%%%%%%%%%%%%%%%%%%%%%%%%%
\subsection{Copyright}

Copyright \copyright{} 2017--2018 Niklas Beisert

This work may be distributed and/or modified under the
conditions of the \LaTeX{} Project Public License, either version 1.3
of this license or (at your option) any later version.
The latest version of this license is in
  \url{http://www.latex-project.org/lppl.txt}
and version 1.3 or later is part of all distributions of \LaTeX{}
version 2005/12/01 or later.

This work has the LPPL maintenance status `maintained'.

The Current Maintainer of this work is Niklas Beisert.

This work consists of the files |README.txt|, |childdoc.ins| and |childdoc.dtx|
as well as the derived files |childdoc.def|, |cdocsamp.tex|
with |cdocsch1.tex|, |cdocsch2.tex|, |cdocspt3.tex|, |cdocspt4.tex|,
|cdocsdrf.tex|, |cdocsfn1.tex|, |cdocsfn2.tex|
as well as |childdoc.pdf|.

%%%%%%%%%%%%%%%%%%%%%%%%%%%%%%%%%%%%%%%%%%%%%%%%%%%%%%%%%%%%%%%%%%%%%%%%%%%%%%%%
\subsection{Files and Installation}

The package consists of the files:
%
\begin{center}
\begin{tabular}{ll}
    |README.txt|   & readme file \\
    |childdoc.ins| & installation file \\
    |childdoc.dtx| & source file \\
    |childdoc.def| & definition file \\
    |cdocsamp.tex| & sample main file \\
    |cdocsch1.tex| & sample include file \\
    |cdocsch2.tex| & sample include file \\
    |cdocspt3.tex| & sample part file \\
    |cdocspt4.tex| & sample part file \\
    |cdocsdrf.tex| & sample redirection file \\
    |cdocsfn1.tex| & sample redirection file \\
    |cdocsfn2.tex| & sample redirection file \\
    |childdoc.pdf| & manual
\end{tabular}
\end{center}
%
The distribution consists of the files
|README.txt|, |childdoc.ins| and |childdoc.dtx|.
%
\begin{itemize}
\item
Run (pdf)\LaTeX{} on |childdoc.dtx|
to compile the manual |childdoc.pdf| (this file).
\item
Run \LaTeX{} on |childdoc.ins| to create the definitions file |childdoc.def|
and the sample |cdocsamp.tex| with include files
|cdocsch1.tex|, |cdocsch2.tex|, |cdocspt3.tex|, |cdocspt4.tex|,
|cdocsdrf.tex|, |cdocsfn1.tex|, |cdocsfn2.tex|.
Then copy the file |childdoc.def| to an appropriate directory of your \LaTeX{}
distribution, e.g.\ \textit{texmf-root}|/tex/latex/childdoc|.
\end{itemize}

%%%%%%%%%%%%%%%%%%%%%%%%%%%%%%%%%%%%%%%%%%%%%%%%%%%%%%%%%%%%%%%%%%%%%%%%%%%%%%%%
\subsection{Related CTAN Packages}

There are several other packages which offer a similar functionality:
%
\begin{itemize}
\item
The packages
\href{http://ctan.org/pkg/docmute}{\textsf{docmute}},
\href{http://ctan.org/pkg/includex}{\textsf{includex}} and
\href{http://ctan.org/pkg/standalone}{\textsf{standalone}}
provide commands to include only the document body of
a child file thus allowing both files to be compiled individually.
\item
The packages \href{http://ctan.org/pkg/subdocs}{\textsf{subdocs}}
and \href{http://ctan.org/pkg/subfiles}{\textsf{subfiles}}
provide structures in which the main and child documents can be
encapsulated and allowing them to be compiled individually.
The inclusion mechanism is different from the conventional |\include|.
\item
The package \href{http://ctan.org/pkg/combine}{\textsf{combine}}
is an elaborate solution to combine several documents into one.
\end{itemize}
%
See also the CTAN topic \href{http://ctan.org/topic/subdocs}{\textsf{subdocs}}
for further related packages.
The present package differs from the above solutions in that
a document structure constructed with the conventional |\include| mechanism
just needs two extra commands at the top of every file
such that all constituent files can be compiled individually.

%%%%%%%%%%%%%%%%%%%%%%%%%%%%%%%%%%%%%%%%%%%%%%%%%%%%%%%%%%%%%%%%%%%%%%%%%%%%%%%%
%\subsection{Feature Suggestions}
%
%The following is a list of features which may be useful for future
%versions of this package:
%%
%\begin{itemize}
%\item
%\ldots
%\end{itemize}

%%%%%%%%%%%%%%%%%%%%%%%%%%%%%%%%%%%%%%%%%%%%%%%%%%%%%%%%%%%%%%%%%%%%%%%%%%%%%%%%
\subsection{Revision History}

%%%%%%%%%%%%%%%%%%%%%%%%%%%%%%%%%%%%%%%%
\paragraph{v2.0:} 2018/12/30

\begin{itemize}
\item
immediate forward processing
\item
added |\childdocby| mechanism
\item
manual restructured
\end{itemize}

%%%%%%%%%%%%%%%%%%%%%%%%%%%%%%%%%%%%%%%%
\paragraph{v1.6:} 2018/01/17

\begin{itemize}
\item
application for development of include files
\item
corrections to manual
\end{itemize}

%%%%%%%%%%%%%%%%%%%%%%%%%%%%%%%%%%%%%%%%
\paragraph{v1.5:} 2017/05/21

\begin{itemize}
\item
more complete structuring introduced
\item
|\childdocof| introduced
\item
|\childdoc| renamed to |\childdocmain|
\item
|\childredirect| renamed to |\childdocforward| and |\childdocforwardprefix|
and functionality expanded
\end{itemize}

%%%%%%%%%%%%%%%%%%%%%%%%%%%%%%%%%%%%%%%%
\paragraph{v1.0:} 2017/04/27

\begin{itemize}
\item
manual and install package
\item
first version published on CTAN
\end{itemize}

%%%%%%%%%%%%%%%%%%%%%%%%%%%%%%%%%%%%%%%%
\paragraph{v0.6:} 2017/04/26

\begin{itemize}
\item
redirection mechanism added
\end{itemize}

%%%%%%%%%%%%%%%%%%%%%%%%%%%%%%%%%%%%%%%%
\paragraph{v0.5:} 2017/04/26

\begin{itemize}
\item
functionality in definition file
\end{itemize}


%%%%%%%%%%%%%%%%%%%%%%%%%%%%%%%%%%%%%%%%%%%%%%%%%%%%%%%%%%%%%%%%%%%%%%%%%%%%%%%%
%%%%%%%%%%%%%%%%%%%%%%%%%%%%%%%%%%%%%%%%%%%%%%%%%%%%%%%%%%%%%%%%%%%%%%%%%%%%%%%%
%%%%%%%%%%%%%%%%%%%%%%%%%%%%%%%%%%%%%%%%%%%%%%%%%%%%%%%%%%%%%%%%%%%%%%%%%%%%%%%%
\appendix

\settowidth\MacroIndent{\rmfamily\scriptsize 000\ }

 \DocInput{childdoc.dtx}

\end{document}
%</driver>
% \fi
%
% %%%%%%%%%%%%%%%%%%%%%%%%%%%%%%%%%%%%%%%%%%%%%%%%%%%%%%%%%%%%%%%%%%%%%%%%%%%%%%
% %%%%%%%%%%%%%%%%%%%%%%%%%%%%%%%%%%%%%%%%%%%%%%%%%%%%%%%%%%%%%%%%%%%%%%%%%%%%%%
% \section{Sample}
%\iffalse
%<*samplemain>
%\fi
%
% The following presents a sample document
% with two chapters, two parts, a title page,
% a compile flag as well as three forwarding files to set the flag.
% It consists of eight |.tex| files:
% \begin{center}
% \begin{tabular}{ll}
% |cdocsamp.tex|&main file\\
% |cdocsch1.tex|&include file for chapter 1\\
% |cdocsch2.tex|&include file for chapter 2\\
% |cdocspt3.tex|&include file for part 3\\
% |cdocspt4.tex|&include file for part 4\\
% |cdocsdrf.tex|&forwarding file for main file in draft mode\\
% |cdocsfi1.tex|&forwarding file for final version of chapter 1\\
% |cdocsfi2.tex|&forwarding file for final version of chapter 2\\
% \end{tabular}
% \end{center}
% Each of the eight files can be compiled directly by the \LaTeX{} compiler.
%
% %%%%%%%%%%%%%%%%%%%%%%%%%%%%%%%%%%%%%%
% \paragraph{Main File.}
%
% The main file is called |cdocsamp.tex|.
%
% Load the \textsf{childdoc} definitions and
% declare the filename for the main document:
%    \begin{macrocode}
\input{childdoc.def}
\childdocmain{}
%    \end{macrocode}

% Optional override for |\version| flag:
%    \begin{macrocode}
%%\ifchilddoc\else\providecommand{\version}{draft}\fi
%    \end{macrocode}

% Define the default values for the |\version| flag
% (|final| for the main file and |draft| for childs):
%    \begin{macrocode}
\ifchilddoc
\providecommand{\version}{draft}
\else
\providecommand{\version}{final}
\fi
%    \end{macrocode}

% Load the standard document class:
%    \begin{macrocode}
\documentclass[12pt]{article}
%    \end{macrocode}

% Start the document body:
%    \begin{macrocode}
\begin{document}
%    \end{macrocode}

% Declare a title page.
% Print title, part of document being processed and version flag:
%    \begin{macrocode}
\addtocounter{page}{-1}
\begin{center}
{\LARGE\bfseries{}childdoc example\par}
\vspace{1cm}
\ifchilddoc
\ifchilddocmanual part\else chapter\fi:
`\childdocname' of `\childdocjob'\par
\else
main document: `\childdocjob'\par
\fi
version: \version\par
\end{center}
\newpage
%    \end{macrocode}

% Manually include selected file,
% otherwise process as usual:
%    \begin{macrocode}
\ifchilddocmanual
\section*{part `\childdocname'}
\input{\childdocname}
\else
%    \end{macrocode}

% Include the two chapters:
%    \begin{macrocode}
\include{cdocsch1}
\include{cdocsch2}
%    \end{macrocode}

% Include the two parts unless only chapters should be displayed:
%    \begin{macrocode}
\ifchilddoc\else
\section{part three}
\input{cdocspt3}
\section{part four}
\input{cdocspt4}
\fi
%    \end{macrocode}

% Process as usual until here:
%    \begin{macrocode}
\fi
%    \end{macrocode}

% End of document body:
%    \begin{macrocode}
\end{document}
%    \end{macrocode}
%\iffalse
%</samplemain>
%\fi
%
% %%%%%%%%%%%%%%%%%%%%%%%%%%%%%%%%%%%%%%
% \paragraph{Chapter Include Files.}
%
% The include files are called |cdocsch1.tex| and |cdocsch2.tex|.
%
%\iffalse
%<*samplechap1|samplechap2>
%\fi

% Optional override for |\version| flag:
%    \begin{macrocode}
%%\providecommand{\version}{final}
%    \end{macrocode}

% Include the main document:
%    \begin{macrocode}
\input{childdoc.def}
\childdocof{cdocsamp}
%    \end{macrocode}

%\iffalse
%</samplechap1|samplechap2>
%\fi
%
%\iffalse
%<*samplechap1>
%\fi
% Some text for chapter 1:
%    \begin{macrocode}
\section{one}
some text in chapter one
%    \end{macrocode}

%\iffalse
%</samplechap1>
%\fi
% Some text for chapter 2:
%\iffalse
%<*samplechap2>
%\fi
%    \begin{macrocode}
\section{two}
more text in chapter two
%    \end{macrocode}

%\iffalse
%</samplechap2>
%\fi
%
% %%%%%%%%%%%%%%%%%%%%%%%%%%%%%%%%%%%%%%
% \paragraph{Part Include Files.}
%
% The include files are called |cdocspt3.tex| and |cdocspt4.tex|.
%
%\iffalse
%<*samplepart3|samplepart4>
%\fi

% Optional override for |\version| flag:
%    \begin{macrocode}
%%\providecommand{\version}{final}
%    \end{macrocode}

% Include the main document:
%    \begin{macrocode}
\input{childdoc.def}
\childdocby{cdocsamp}
%    \end{macrocode}

%\iffalse
%</samplepart3|samplepart4>
%\fi
%
%\iffalse
%<*samplepart3>
%\fi
% Some text for part 3:
%    \begin{macrocode}
some text in part three
%    \end{macrocode}

%\iffalse
%</samplepart3>
%\fi
% Some text for part 4:
%\iffalse
%<*samplepart4>
%\fi
%    \begin{macrocode}
more text in part four
%    \end{macrocode}

%\iffalse
%</samplepart4>
%\fi
%
% %%%%%%%%%%%%%%%%%%%%%%%%%%%%%%%%%%%%%%
% \paragraph{Forwarding for a Complete Draft.}
%
% The following forwarding file |cdocsdrf.tex|
% compiles the main document in draft mode:
%\iffalse
%<*sampledraft>
%\fi
%    \begin{macrocode}
\def\version{draft}
\input{childdoc.def}
\childdocforward{cdocsamp}
%    \end{macrocode}

%\iffalse
%</sampledraft>
%\fi
%
% %%%%%%%%%%%%%%%%%%%%%%%%%%%%%%%%%%%%%%
% \paragraph{Forwarding for Final Version of the Chapters.}
%
% The following forwarding files |cdocsfn1.tex| and |cdocsfn2.tex|
% (with identical content)
% compile the final versions of the child documents
% |cdocsch1.tex| and |cdocsch2.tex|, respectively:
%\iffalse
%<*samplefinal>
%\fi
%    \begin{macrocode}
\def\version{final}
\input{childdoc.def}
\childdocforwardprefix[cdocsamp]{cdocsfn}{cdocsch}
%    \end{macrocode}

%\iffalse
%</samplefinal>
%\fi
%
% %%%%%%%%%%%%%%%%%%%%%%%%%%%%%%%%%%%%%%
% \paragraph{Command Line Processing.}
%
% The following three command lines generate the output files
% |cdocscld|, |cdocscl1| and |cdocscl2|
% which should be identical to
% |cdocsdrf|, |cdocsch1| and |cdocsfn2|, respectively:
% \begin{center}
% \begin{tabular}{l}
% |latex -jobname cdocscld \|\\
% |  "\def\version{draft}\input{childdoc.def}\childdocforward{cdocsamp}"|\\
% |latex -jobname cdocscl1 \|\\
% |  "\input{childdoc.def}\childdocforward[cdocsamp]{cdocsch1}"|\\
% |latex -jobname cdocscl2 \|\\
% |  "\def\version{final}\input{childdoc.def}\childdocforward{cdocsch2}"|
% \end{tabular}
% \end{center}
% Note that the trailing backslash on each first line
% merely continues the input to the second line
% (for convenient cut ant paste).
% Furthermore, the command |latex| can be replaced by any
% of its alternative versions such as |pdflatex|.
%
% %%%%%%%%%%%%%%%%%%%%%%%%%%%%%%%%%%%%%%%%%%%%%%%%%%%%%%%%%%%%%%%%%%%%%%%%%%%%%%
% %%%%%%%%%%%%%%%%%%%%%%%%%%%%%%%%%%%%%%%%%%%%%%%%%%%%%%%%%%%%%%%%%%%%%%%%%%%%%%
% \section{Implementation}
%\iffalse
%<*package>
%\fi
%
% This section describes the definitions file |childdoc.def|.

% The definitions cannot be loaded using |\usepackage| or |\RequirePackage|
% which has a mechanism to prevent loading a style file more than once.
% When loading the definitions by means of |\input|
% multiple instances have to be prevented manually:
%\iffalse
%This code needs to be before the `\ProvidesFile' directive
%which is defined at the beginning of this file.
%Therefore it is also placed there and commented out here.
%</package>
%<*discard>
%\fi
%    \begin{macrocode}
\ifdefined\childdocmain\endinput\fi
%    \end{macrocode}
%\iffalse
%</discard>
%<*package>
%\fi
%
% \macro{\ifchilddoc}
% \macro{\ifchilddocmanual}
% The conditional |\ifchilddoc| tells whether a
% child (true) or main (false) document is being compiled.
% The conditional |\ifchilddocmanual| tells whether
% the |\includeonly| mechanism is used (false) or
% the selection of child files must be performed manually (true).
% The definitions initialise to false:
%    \begin{macrocode}
\newif\ifchilddoc
\newif\ifchilddocmanual
%    \end{macrocode}

% \macro{\childdocname}
% \macro{\childdocjob}
% The macro |\childdocname| stores the name of the main document
% to be compiled. The macro |\childdocjob| stores the name of
% the document on which the \LaTeX{} compiler was originally invoked.
% The content of |\jobname| cannot be compared
% to filenames specified in the source due to different catcodes.
% The following code rescans |\jobname|, stores the result
% in |\childdocname| and saves a copy in |\childdocjob|:
%    \begin{macrocode}
\edef\childdocname{\scantokens\expandafter{\jobname\noexpand}}
\let\childdocjob\childdocname
%    \end{macrocode}

% \macro{\childdocdisable}
% The macro |\childdocdisable| prevents the main file
% from being processed more than once.
% At this stage, the main document command |\childdocmain|
% is assumed to be called once again where it should do nothing.
% Any subsequent call to it should prevent
% a secondary processing of the main document
% It overwrites the forwarding commands
% |\childdocof| and |\childdocforward|
% with empty macros to prevent further inclusions of the main document:
%    \begin{macrocode}
\newcommand{\childdocdisable}
{
  \renewcommand{\childdocmain}[1]{\renewcommand{\childdocmain}[1]{\endinput}}
  \renewcommand{\childdocof}[1]{}
  \renewcommand{\childdocby}[2][]{}
  \renewcommand{\childdocforward}[2][]{}
  \renewcommand{\childdocdisable}{}
}
%    \end{macrocode}

% \macro{\childdocmain}
% The macro |\childdocmain| is to be called at the top of the main file
% with nothing or the main filename (without extension) as argument.
% First, it breaks loops.
% If the argument is not empty and does not match |\childdocname|
% (which is set by the first inclusion of |childdoc.def|),
% |\ifchilddoc| is set to true, |\includeonly| is applied to the child file
% and |\jobname| is set to the main file
% (for proper handling of |.aux| files):
%    \begin{macrocode}
\newcommand{\childdocmain}[1]
{
  \childdocdisable\childdocmain{}
  \if?#1?\else
    \begingroup
      \def\childdoctmp{#1}
      \ifx\childdoctmp\childdocname
        \def\childdoctmp{}
      \else
        \def\childdoctmp
        {
          \childdoctrue
          \includeonly{\childdocname}
          \def\childdocjob{#1}
          \def\jobname{#1}
        }
      \fi
      \expandafter
    \endgroup
    \childdoctmp
  \fi
}
%    \end{macrocode}

% \macro{\childdocof}
% The command |\childdocof| redirects
% compilation to the main file |#1|.
%    \begin{macrocode}
\newcommand{\childdocof}[1]
{
  \childdocdisable
  \childdoctrue
  \includeonly{\childdocname}
  \def\jobname{#1}
  \def\childdocjob{#1}
  \input{#1}
}
%    \end{macrocode}

% \macro{\childdocby}
% The command |\childdocby| ....
%    \begin{macrocode}
\newcommand{\childdocby}[2][]
{
  \childdocdisable
  \childdoctrue
  \childdocmanualtrue
  \if?#1?\else
    \def\jobname{#2}
  \fi
  \def\childdocjob{#2}
  \input{#2}
  \endinput
}
%    \end{macrocode}

% \macro{\childdocforward}
% The command |\childdocforward| redirects
% compilation to the main file or
% (if the optional argument is given) a child file.
% Parameters are set as if the main file
% or a child file starting with |\childdocof| was compiled.
% Then compilation is handed over to the main file:
%    \begin{macrocode}
\newcommand{\childdocforward}[2][]
{
  \begingroup
    \if?#1?
      \def\childdoctmp
      {
        \def\childdocname{#2}
        \def\childdocjob{#2}
        \def\jobname{#2}
        \input{#2}
        \endinput
      }
    \else
      \def\childdoctmp
      {
        \childdocdisable
        \def\childdocname{#2}
        \childdoctrue
        \includeonly{#2}
        \def\childdocjob{#1}
        \def\jobname{#1}
        \input{#1}
        \endinput
      }
    \fi
    \expandafter
  \endgroup
  \childdoctmp
}
%    \end{macrocode}

% \macro{\childdocforwardprefix}
% The command |\childdocforwardprefix| redirects
% compilation to the main or a child file by means of a pattern.
% The prefix |#1| in the current filename is replaced by |#2|
% and the suffix of the current filename is kept
% (it is assumed that the filename does not contain the substring `|~~~|'
% which is used as a delimiter).
% Compilation is handed over to the new file by |\childdocforward|:
%    \begin{macrocode}
\newcommand{\childdocforwardprefix}[3][]
{
  \begingroup
    \def\childdocextract #2##1~~~{\def\childdoctmp{\childdocforward[#1]{#3##1}}}
    \expandafter\childdocextract\childdocname~~~
    \expandafter
  \endgroup
  \childdoctmp
}
%    \end{macrocode}

% \macro{\childdoc}
% The deprecated macro |\childdoc| is a legacy version of |\childdocmain|:
%    \begin{macrocode}
\newcommand{\childdoc}{\childdocmain}
%    \end{macrocode}

% \macro{\childdocredirect}
% The deprecated macro |\childdocredirect| is a legacy version
% of |\childdocforward| and |\childdocforwardprefix|:
%    \begin{macrocode}
\newcommand{\childdocredirect}[2][]
{
  \begingroup
    \if?#1?
      \def\childdoctmp{\childdocforward{#2}}
    \else
      \def\childdoctmp{\childdocforwardprefix{#1}{#2}}
    \fi
    \expandafter
  \endgroup
  \childdoctmp
}
%    \end{macrocode}

%\iffalse
%</package>
%\fi
%
\endinput
|\\
|\childdocforwardprefix{final}{child}|
\end{tabular}
\end{center}
%

Note that when several versions of a main file and/or of each child file
are to be generated, it may be convenient to set up a |Makefile| or
shell script to automatise the process.

%%%%%%%%%%%%%%%%%%%%%%%%%%%%%%%%%%%%%%%%%%%%%%%%%%%%%%%%%%%%%%%%%%%%%%%%%%%%%%%%
\subsection{Command Line Processing}
\label{sec:commandline}

The effect of redirection files can also be achieved by invoking
the \LaTeX{} compiler with a more elaborate command line.
Most conveniently this should be done as part
of a shell script or a |Makefile|.

When using \textsf{childdoc} in the main file, the following
command lines effectively perform a redirection
(note that depending on the shell being used,
backslashes may have to be doubled: `|\|' $\to$ `|\\|'):
%
\begin{center}
|... -jobname "|\textit{target}|" |\\|"|[\textit{flags}]%
|% \iffalse
%
% childdoc.dtx Copyright (C) 2017-2018 Niklas Beisert
%
% This work may be distributed and/or modified under the
% conditions of the LaTeX Project Public License, either version 1.3
% of this license or (at your option) any later version.
% The latest version of this license is in
%   http://www.latex-project.org/lppl.txt
% and version 1.3 or later is part of all distributions of LaTeX
% version 2005/12/01 or later.
%
% This work has the LPPL maintenance status `maintained'.
%
% The Current Maintainer of this work is Niklas Beisert.
%
% This work consists of the files childdoc.dtx and childdoc.ins
% and the derived files childdoc.def and cdocsamp.tex with
% cdocsch1.tex, cdocsch2.tex, cdocsdrf.tex, cdocsfn1.tex, cdocsfn2.tex.
%
%<package>\ifdefined\childdocmain\endinput\fi
%<package>\ProvidesFile{childdoc.def}[2018/12/30 v2.0 child document driver]
%<samplemain>\ProvidesFile{cdocsamp.tex}[2018/12/30 v2.0 sample for childdoc]
%<*driver>
%\ProvidesFile{childdoc.drv}[2018/12/30 v2.0 childdoc reference manual file]
\PassOptionsToClass{10pt,a4paper}{article}
\documentclass{ltxdoc}

\usepackage[margin=35mm]{geometry}
\usepackage{hyperref}
\usepackage{hyperxmp}
\usepackage[usenames]{color}

\hypersetup{colorlinks=true}
\hypersetup{pdfstartview=FitH}
\hypersetup{pdfpagemode=UseNone}
\hypersetup{pdfsource={}}
\hypersetup{pdflang={en-UK}}
\hypersetup{pdfcopyright={Copyright 2017-2018 Niklas Beisert.
  This work may be distributed and/or modified under the
  conditions of the LaTeX Project Public License, either version 1.3
  of this license or (at your option) any later version.}}
\hypersetup{pdflicenseurl={http://www.latex-project.org/lppl.txt}}
\hypersetup{pdfcontactaddress={ETH Zurich, ITP, HIT K,
  Wolfgang-Pauli-Strasse 27}}
\hypersetup{pdfcontactpostcode={8093}}
\hypersetup{pdfcontactcity={Zurich}}
\hypersetup{pdfcontactcountry={Switzerland}}
\hypersetup{pdfcontactemail={nbeisert@itp.phys.ethz.ch}}
\hypersetup{pdfcontacturl={http://people.phys.ethz.ch/\xmptilde nbeisert/}}

\newcommand{\secref}[1]{\hyperref[#1]{section \ref*{#1}}}

\parskip1ex
\parindent0pt
\let\olditemize\itemize
\def\itemize{\olditemize\parskip0pt}

\begin{document}

\title{The \textsf{childdoc} Package}
\hypersetup{pdftitle={The childdoc Package}}
\author{Niklas Beisert\\[2ex]
  Institut f\"ur Theoretische Physik\\
  Eidgen\"ossische Technische Hochschule Z\"urich\\
  Wolfgang-Pauli-Strasse 27, 8093 Z\"urich, Switzerland\\[1ex]
  \href{mailto:nbeisert@itp.phys.ethz.ch}
  {\texttt{nbeisert@itp.phys.ethz.ch}}}
\hypersetup{pdfauthor={Niklas Beisert}}
\hypersetup{pdfsubject={Manual for the LaTeX2e Package childdoc}}
\date{30 December 2018, \textsf{v2.0}}
\maketitle

\begin{abstract}\noindent
\textsf{childdoc} is a \LaTeXe{} package
that enables the direct compilation
of document sections included by |\include|
to individual files.
\end{abstract}

\begingroup
\parskip0ex
\tableofcontents
\endgroup

%%%%%%%%%%%%%%%%%%%%%%%%%%%%%%%%%%%%%%%%%%%%%%%%%%%%%%%%%%%%%%%%%%%%%%%%%%%%%%%%
%%%%%%%%%%%%%%%%%%%%%%%%%%%%%%%%%%%%%%%%%%%%%%%%%%%%%%%%%%%%%%%%%%%%%%%%%%%%%%%%
\section{Introduction}

\LaTeX{} provides a mechanism to structure a large document (such as a book)
into a main file and several child files (containing the chapters)
using the |\include| command.
This mechanism is beneficial for documents
which span hundreds of pages in order to
make the source file(s) more manageable.
Moreover, compilation can be restricted to
selected child files by means of the |\includeonly| command.
The latter feature can be used to reduce the compilation time while editing
(this was significantly more useful in the earlier days of \LaTeX{})
or to generate a smaller document which is easier to navigate.
Another application of |\includeonly| is to generate
documents consisting of selected parts of the complete document.

However, there are a few drawbacks of the plain |\include| mechanism:
\begin{itemize}
\item
The child files cannot be compiled on their own,
they can only be compiled via the main file.
A naive editing environment
(such as a text editor with an option
to have the current file processed by \LaTeX)
may require one to switch to the main file before compiling;
attempting to compile the child file produces errors.
\item
The main file must be modified (each time)
to adjust the |\includeonly| command
to the present needs. This easily leaves the main file in a messy state.
\item
The generated document will always carry the filename
of the main document. This is inconvenient if
several child files are to be compiled and
to be kept for distribution.
\end{itemize}

The present package provides a simple interface
to make child files individually compilable by \LaTeX{}.
Compiling a child file then has the same effect as compiling
the main file with an |\includeonly| command
to select the appropriate child.
Moreover the generated document will carry the name of the child
rather than the main file.
This resolves all three above issues.

This feature is meant to make the editing of books,
thesis documents and lecture notes somewhat more convenient.
However, the package can also be used efficiently for
composing a series of documents (such as exercise sheets)
which are typically distributed individually.
It then assists the author in generating the individual documents
(potentially in different versions)
as well as a document containing the collected series.
Another application is in developing style files
or other kinds of included material
where compilation of the style file could redirect
to a sample or test file.

%%%%%%%%%%%%%%%%%%%%%%%%%%%%%%%%%%%%%%%%%%%%%%%%%%%%%%%%%%%%%%%%%%%%%%%%%%%%%%%%
%%%%%%%%%%%%%%%%%%%%%%%%%%%%%%%%%%%%%%%%%%%%%%%%%%%%%%%%%%%%%%%%%%%%%%%%%%%%%%%%
\section{Usage}

First of all, the package \textsf{childdoc} is \emph{not} a standard
\LaTeXe{} |.sty| style file! Therefore it needs to be invoked in
a non-standard way.

%%%%%%%%%%%%%%%%%%%%%%%%%%%%%%%%%%%%%%%%%%%%%%%%%%%%%%%%%%%%%%%%%%%%%%%%%%%%%%%%
\subsection{Included Files}
\label{sec:include}

%%%%%%%%%%%%%%%%%%%%%%%%%%%%%%%%%%%%%%%%
\DescribeMacro{\childdocmain}
To use the package, add the commands
\begin{center}
\begin{tabular}{l}
|\input{childdoc.def}|\\
|\childdocmain{}|\\
\end{tabular}
\end{center}
at the very top of the main \LaTeX{} file,
in particular \emph{before} the |\documentclass| statement!
The argument of |\childdocmain| should be left empty
(but it must be present).

%%%%%%%%%%%%%%%%%%%%%%%%%%%%%%%%%%%%%%%%
\DescribeMacro{\childdocof}
Furthermore, add the commands
\begin{center}
\begin{tabular}{l}
|\input{childdoc.def}|\\
|\childdocof{|\textit{main}|}|\\
\end{tabular}
\end{center}
at the top of every child file \textit{child}
which is included by |\include{|\textit{child}|}|
from within the main file
(or at least for those files to be compiled individually).
The argument \textit{main} must be the filename of the main file.

There are a couple of
considerations in setting up the main and child documents:

%%%%%%%%%%%%%%%%%%%%%%%%%%%%%%%%%%%%%%%%
\paragraph{Restrictions.}

Please note the following restrictions:
\begin{itemize}
\item
|\childdocmain| must be called with one argument \textit{main}
to ensure compatibility with earlier version of the package.
It must either be empty (|\childdocmain{}|)
or precisely match the filename of the main file in which it is specified.
See \secref{sec:detection} for further information.
\item
The filename \textit{main} must be specified without the |.tex| extension.
\item
The filename \textit{main} is case sensitive
(even in case-insensitive file systems)
due to internal string comparison.
\item
The argument \textit{main} should be fully expanded, it cannot be a macro.
\item
Subdirectories and special characters should be avoided in filenames.
\item
The command |\childdocmain{|\textit{main}|}| must be followed by a whitespace.
It should not be followed immediately by another command
or by a comment mark `|%|'.
This is because the \TeX{} parser reads the token immediately following
the argument of |\childdocmain| and puts it
at the beginning of every child section;
however, a white\-space is ignored.
\end{itemize}

%%%%%%%%%%%%%%%%%%%%%%%%%%%%%%%%%%%%%%%%
\paragraph{Content of Main File.}

It is advisable to place all content in the child files included by |\include|.
Any output contained in the main file will appear in all child documents
unless suppressed manually;
it cannot be suppressed automatically by the |\includeonly| directive
and thus should normally be avoided.
A method to include some content in the main file
by means of conditional processing is described in \secref{sec:conditional}.

%%%%%%%%%%%%%%%%%%%%%%%%%%%%%%%%%%%%%%%%
\paragraph{Page Numbering.}

When only a part of the document is compiled,
the appropriate numbering of pages
(as well as other status parameters)
is determined from the |.aux| files.
The latter contain information from previous passes.
However this information needs to propagate through
all intermediate child documents.
Therefore the page numbering in child documents may well
be inconsistent until the complete document is compiled at least once.

A useful (if unconventional) way to always ensure a consistent
page numbering is to restart the numbering in each child document
and denote the pages by `\textit{child}|.|\textit{page}'
where \textit{child} represents the chapter/section number of the child file.
This can be achieved by the command
|\numberwithin{page}{|\textit{child}|}|
of the \textsf{amsmath} package
where \textit{child} can be |chapter| or |section|
depending on the chosen structuring.
Alternatively, one can modify the macro |\thepage| appropriately
and reset the counter |page| at the start of each child file.

%%%%%%%%%%%%%%%%%%%%%%%%%%%%%%%%%%%%%%%%%%%%%%%%%%%%%%%%%%%%%%%%%%%%%%%%%%%%%%%%
\subsection{Conditional Processing}
\label{sec:conditional}

The package provides a mechanism to compile different versions
of a document. To customise the versions further some conditional processing
can come in handy to distinguish which version is being compiled.
The package provides two macros to describe the compilation context:

%%%%%%%%%%%%%%%%%%%%%%%%%%%%%%%%%%%%%%%%
\DescribeMacro{\ifchilddoc}
The conditional |\ifchilddoc| distinguishes between the compilation of
child documents and the main document:
%
\begin{center}
|\ifchilddoc |\textit{child-code}| |[|\||else |\textit{main-code}]| \||fi|
\end{center}

%%%%%%%%%%%%%%%%%%%%%%%%%%%%%%%%%%%%%%%%
\DescribeMacro{\childdocname}
\DescribeMacro{\childdocjob}
The macro |\childdocname| contains the filename (without extension)
of the main or child file being processed.
Note that |\childdocjob| will always contain the name of the main file.

%%%%%%%%%%%%%%%%%%%%%%%%%%%%%%%%%%%%%%%%
\paragraph{Title Page.}

Conditional processing can be used to include a title or banner page
in the main document when proper precautions are taken.
Importantly, the code in the main file should ensure that the page counter
(as well as other status parameters which are stored in the |.aux| files)
takes the same value after the conditional processing.
Otherwise the page numbers may take divergent values
depending on which part is compiled.

For example, a title page could be declared by:
%
\begin{center}
\begin{tabular}{l}
|\ifchilddoc\||else|\\
|\addtocounter{page}{-1}|\\
\textit{code for title page}\\
|\newpage|\\
|\||fi|
\end{tabular}
\end{center}
%
A banner page for the child documents can be generated by:
%
\begin{center}
\begin{tabular}{l}
|\ifchilddoc|\\
|\addtocounter{page}{-1}|\\
\textit{code for banner page}\\
|\newpage|\\
|\||fi|
\end{tabular}
\end{center}
%
Here one could write a message such as:
\begin{center}
|This is the part \childdocname{} of \childdocjob{}.|
\end{center}

%%%%%%%%%%%%%%%%%%%%%%%%%%%%%%%%%%%%%%%%%%%%%%%%%%%%%%%%%%%%%%%%%%%%%%%%%%%%%%%%
\subsection{Flags}
\label{sec:flags}

The package makes it easy to generate different versions
of the main or child documents.
To this end compilation flags can be defined
and assigned different default values.
They will be particularly useful in conjunction
with the forwarding mechanism described in \secref{sec:forward}.

For example, it may be useful to have a flag |\version|
which can be set to |draft| or |final|.
The document source will contain some conditional code
depending on the value of |\version|.
Suppose further, the flag should default to |final| for the main file
and to |draft| for child files
which is a natural assignment for editing the document.
This is achieved by placing the following code
in the preamble of the main document
(below the |\childdocmain| directive):
%
\begin{center}
\begin{tabular}{l}
|\ifchilddoc|\\
|\providecommand{\version}{draft}|\\
|\||else|\\
|\providecommand{\version}{final}|\\
|\||fi|
\end{tabular}
\end{center}
%
The definition by |\providecommand| makes sure
that previous definitions are not overwritten.
Further statements |\providecommand{\version}{...}|
can thus be added before the above code to override it.

For the main file, one might add a line
(between |\childdocmain| and the above block)
%
\begin{center}
|%\ifchilddoc\||else\providecommand{\version}{draft}\||fi|
\end{center}
%
which can be uncommented to produce a draft version.
Likewise one can add a line to the very top of a child file
(above the |\childdocof{|\textit{main}|}| directive)
%
\begin{center}
|%\providecommand{\version}{final}|
\end{center}
%
which can be uncommented to produce the final version of this child document.

%%%%%%%%%%%%%%%%%%%%%%%%%%%%%%%%%%%%%%%%%%%%%%%%%%%%%%%%%%%%%%%%%%%%%%%%%%%%%%%%
\subsection{Forwarding}
\label{sec:forward}

Different versions of the main or child documents
using compilation flags as described in \secref{sec:flags}
can be (permanently) stored in different files
for convenient compilation, viewing and distribution.
To this end, the package defines a command
to pass on compilation to a different file:

%%%%%%%%%%%%%%%%%%%%%%%%%%%%%%%%%%%%%%%%
\DescribeMacro{\childdocforward}
The command |\childdocforward| redirects processing to
another source file:
%
\begin{center}
\begin{tabular}{l}
|\input{childdoc.def}|\\
|\childdocforward[|\textit{main}|]{|\textit{dest}|}|\\
\end{tabular}
\end{center}
%
The argument \textit{dest} is the destination file
(without extension).
It should be the main file or one of the child files.
Note that further \textsf{childdoc} directives
such as |\childdocof| and |\childdocforward|
in the indicated file will be processed in this form.
The optional argument \textit{main}
passes on directly to the main file \textit{main}
while pretending to compile the child \textit{dest}.
This form behaves as if \textit{dest}
issues |\childdocof{|\textit{main}|}| right away,
and no further \textsf{childdoc} directives will be processed.

%%%%%%%%%%%%%%%%%%%%%%%%%%%%%%%%%%%%%%%%
\DescribeMacro{\...prefix}
In the alternative form |\childdocforwardprefix|,
%
\begin{center}
\begin{tabular}{l}
|\input{childdoc.def}|\\
|\childdocforwardprefix[|\textit{main}|]{|\textit{prefix}|}{|\textit{dest}|}|
\end{tabular}
\end{center}
%
the destination file is determined by a pattern
depending on the current file:
To make this work, the current file must be called
`{\textit{prefix}\hspace{0.2em}\textit{suffix}}'
with \textit{prefix} matching precisely the argument.
Processing is then passed on to the file
`{\textit{dest}\hspace{0.2em}\textit{suffix}}'.
Surely, the same effect is achieved by
directly specifying the
argument `{\textit{dest}\hspace{0.2em}\textit{suffix}}'
in the first form.
However, that requires to set up a different file
for each child. With the alternative form of the command
all these files can have exactly the same content
which simplifies setting them up and maintaining them.

For example, the following file |draft.tex|
with a compilation flag |\version| as described in \secref{sec:flags}
compiles the main document as a draft:
%
\begin{center}
\begin{tabular}{l}
|\def\version{draft}|\\
|\input{childdoc.def}|\\
|\childdocforward{|\textit{main}|}|
\end{tabular}
\end{center}
%
Likewise, the following files |final|\textit{nn}|.tex|
compile the final version of the child document
|child|\textit{nn}|.tex|:
%
\begin{center}
\begin{tabular}{l}
|\def\version{final}|\\
|\input{childdoc.def}|\\
|\childdocforwardprefix{final}{child}|
\end{tabular}
\end{center}
%

Note that when several versions of a main file and/or of each child file
are to be generated, it may be convenient to set up a |Makefile| or
shell script to automatise the process.

%%%%%%%%%%%%%%%%%%%%%%%%%%%%%%%%%%%%%%%%%%%%%%%%%%%%%%%%%%%%%%%%%%%%%%%%%%%%%%%%
\subsection{Command Line Processing}
\label{sec:commandline}

The effect of redirection files can also be achieved by invoking
the \LaTeX{} compiler with a more elaborate command line.
Most conveniently this should be done as part
of a shell script or a |Makefile|.

When using \textsf{childdoc} in the main file, the following
command lines effectively perform a redirection
(note that depending on the shell being used,
backslashes may have to be doubled: `|\|' $\to$ `|\\|'):
%
\begin{center}
|... -jobname "|\textit{target}|" |\\|"|[\textit{flags}]%
|\input{childdoc.def}\childdocforward[|\textit{main}|]{|\textit{dest}|}"|
\end{center}
%
Here \textit{target} is the name of the output file,
\textit{main} is the name of the main file
and \textit{dest} is the name of the main or child file to be processed
(all filenames without extensions).
The optional argument \textit{main} can be omitted
if \textit{main} matches \textit{dest}.
Optionally, compilation \textit{flags} can be defined via |\def| commands.
This command line makes the \TeX{} engine believe
it is compiling the file \textit{target}
whose content is specified as the latter parameter.
The provided code then forwards the processing to
\textit{main} or \textit{dest} as described in \secref{sec:forward}.

%%%%%%%%%%%%%%%%%%%%%%%%%%%%%%%%%%%%%%%%%%%%%%%%%%%%%%%%%%%%%%%%%%%%%%%%%%%%%%%%
\subsection{Include by Input}
\label{sec:input}

Including child documents by |\include| has some restrictions by design.
Most notably, the content of a child document always occupies
its own set of pages; pages cannot be shared between child documents.
Usually, this behaviour makes perfect sense
because each child document contain an essential part of the document.
However, in some situations it may be desirable to compose
a document from a collection of parts
without having mandatory page breaks between then.
For this case, the package
provides a mechanism to include parts
by |\input| which can also be processed individually.
However, by construction this mechanism
requires manual handling of the content to be output.

%%%%%%%%%%%%%%%%%%%%%%%%%%%%%%%%%%%%%%%%
\DescribeMacro{\ifchilddocmanual}
The main file should be prepared as usual, see \secref{sec:include}.
However, the document body must make a distinction
between processing of an individual part and of the main document, e.g.:
%
\begin{center}
\begin{tabular}{l}
|\ifchilddocmanual|\\
|\input{\childdocname}|\\
|\||else|\\
\textit{document body with }|\input{|\textit{part}|}|\\
|\||fi|
\end{tabular}
\end{center}
%
The conditional |\ifchilddocmanual| is true whenever
a part to be included by |\input| is being compiled,
and the name of the part is stored in |\childdocname|.

%%%%%%%%%%%%%%%%%%%%%%%%%%%%%%%%%%%%%%%%
\DescribeMacro{\childdocby}
Each part to be included by |\input| should start with:
%
\begin{center}
\begin{tabular}{l}
|\input{childdoc.def}|\\
|\childdocby{|\textit{main}|}|\\
\end{tabular}
\end{center}
%
The directive |\childdocby| is similar to |\childdocof|
described in \secref{sec:include},
but the subsequent selection of content must be done manually.
To that end, both |\ifchilddoc| and |\ifchilddocmanual|
will be true upon processing of a part,
and the name of the part is stored in |\childdocname|.
Note that |\jobname| will be set to the filename of the current part
so that each part receives an individual |.aux| file
that does not interfere with the |.aux| file(s) of the main document.
This behaviour can be altered by the alternative form
|\childdocby[*]{|\textit{main}|}| (with a non-empty optional argument)
which uses the |.aux| file of the main document
by setting |\jobname| to \textit{main}.

%%%%%%%%%%%%%%%%%%%%%%%%%%%%%%%%%%%%%%%%%%%%%%%%%%%%%%%%%%%%%%%%%%%%%%%%%%%%%%%%
\subsection{Driver Development}
\label{sec:driver}

The \textsf{childdoc} mechanism can also be use for the development
of definition files such as \LaTeX{} styles or classes.
This case differs from the above setup with multiple parts
included by |\include| in that no |\includeonly| should be invoked.
This can be achieved by starting the include file
(before |\ProvidesPackage|) with:
%
\begin{center}
\begin{tabular}{l}
|\input{childdoc.def}|\\
|\childdocforward{|\textit{main}|}|\\
\end{tabular}
\end{center}
%
or alternatively with:
%
\begin{center}
\begin{tabular}{l}
|\input{childdoc.def}|\\
|\childdocby{|\textit{main}|}|\\
\end{tabular}
\end{center}
%
Both forms have slightly different effects as described above.
The main file is prepared as usual, see \secref{sec:include}.

%%%%%%%%%%%%%%%%%%%%%%%%%%%%%%%%%%%%%%%%%%%%%%%%%%%%%%%%%%%%%%%%%%%%%%%%%%%%%%%%
\subsection{Legacy Detection}
\label{sec:detection}

The directive |\childdocmain| in the main file can detect
whether the complete document or merely a child is to be compiled
even without using the directive |\childdocof|.
This method is deprecated because it is less robust
and there is no compelling reason to use it;
it is merely provided for backward compatibility
and it may be removed in future versions.

If the detection mechanism is to be used,
it is mandatory to correctly specify
the filename of the main file as the argument of |\childdocmain|:
%
\begin{center}
\begin{tabular}{l}
|\input{childdoc.def}|\\
|\childdocmain{|\textit{main}|}|\\
\end{tabular}
\end{center}
%
If |\jobname| does not match the argument \textit{main} of |\childdocmain|,
it is assumed that |\jobname| points to the child file to be compiled.
When using |\childdocmain| with the main file specified as argument,
it suffices to start a child file
with just |\input{|\textit{main}|}|
without loading of the package and using |\childdocof|.
If instead all processing is done
with the appropriate \textsf{childdoc} directives,
the argument of \textit{main} of |\childdocmain| can be empty.

An alternative version of the command line processing described
in \secref{sec:commandline} using the detection mechanism reads:
%
\begin{center}
|... -jobname "|\textit{target}|" "|[\textit{flags}]%
[|\def\jobname{|\textit{dest}|}|]|\input{|\textit{main}|}"|
\end{center}

%%%%%%%%%%%%%%%%%%%%%%%%%%%%%%%%%%%%%%%%%%%%%%%%%%%%%%%%%%%%%%%%%%%%%%%%%%%%%%%%
\subsection{Manual Code}
\label{sec:manual}

In case one cannot be certain whether the definitions file |childdoc.def|
is installed on the target \TeX{} distribution
and one prefers not to ship it,
it is conceivable to paste a few relevant commands into the sources.

To that end, drop all statements |\input{childdoc.def}|
and perform the replacements as outlined below.
Instead of |\childdocmain{|\textit{main}|}| add the following code
to the top of the main file:
%
\begin{center}
\begin{tabular}{l}
|\||ifdefined\childdocname\endinput\||fi\newif\ifchilddoc|\\
|\edef\childdocname{\scantokens\expandafter{\jobname\noexpand}}|\\
|\def\childdocmain{|\textit{main}|}\||ifx\childdocmain\childdocname\||else|\\
|\childdoctrue\includeonly{\childdocname}\let\jobname\childdocmain\||fi|\\
\end{tabular}
\end{center}
%
Instead of |\childdocof{|\textit{main}|}| just include the main file
at the top of each child file:
%
\begin{center}
|\input{|\textit{main}|}|
\end{center}
%
A simple redirection |\childdocforward{|\textit{dest}|}| is achieved by:
%
\begin{center}
|\def\jobname{|\textit{dest}|}\input{\jobname}|
\end{center}
%
The redirection with prefix
|\childdocforwardprefix[|\textit{prefix}|]{|\textit{dest}|}|
is accomplished by:
%
\begin{center}
\begin{tabular}{l}
|{\edef\jobname{\scantokens\expandafter{\jobname\noexpand}}|\\
|\def\redirectjob |\textit{prefix}|#1~~~{\gdef\jobname{|\textit{dest}|#1}}|\\
|\expandafter\redirectjob\jobname~~~}\input{\jobname}|
\end{tabular}
\end{center}

In an alternative approach,
child documents can be compiled by a specific command line
without additional code or specific definitions:
%
\begin{center}
|... -jobname "|\textit{target}|" "|[\textit{flags}]%
|\includeonly{|\textit{dest}|}\input{|\textit{main}|}"|
\end{center}
%

%%%%%%%%%%%%%%%%%%%%%%%%%%%%%%%%%%%%%%%%%%%%%%%%%%%%%%%%%%%%%%%%%%%%%%%%%%%%%%%%
%%%%%%%%%%%%%%%%%%%%%%%%%%%%%%%%%%%%%%%%%%%%%%%%%%%%%%%%%%%%%%%%%%%%%%%%%%%%%%%%
\section{Information}

%%%%%%%%%%%%%%%%%%%%%%%%%%%%%%%%%%%%%%%%%%%%%%%%%%%%%%%%%%%%%%%%%%%%%%%%%%%%%%%%
\subsection{Copyright}

Copyright \copyright{} 2017--2018 Niklas Beisert

This work may be distributed and/or modified under the
conditions of the \LaTeX{} Project Public License, either version 1.3
of this license or (at your option) any later version.
The latest version of this license is in
  \url{http://www.latex-project.org/lppl.txt}
and version 1.3 or later is part of all distributions of \LaTeX{}
version 2005/12/01 or later.

This work has the LPPL maintenance status `maintained'.

The Current Maintainer of this work is Niklas Beisert.

This work consists of the files |README.txt|, |childdoc.ins| and |childdoc.dtx|
as well as the derived files |childdoc.def|, |cdocsamp.tex|
with |cdocsch1.tex|, |cdocsch2.tex|, |cdocspt3.tex|, |cdocspt4.tex|,
|cdocsdrf.tex|, |cdocsfn1.tex|, |cdocsfn2.tex|
as well as |childdoc.pdf|.

%%%%%%%%%%%%%%%%%%%%%%%%%%%%%%%%%%%%%%%%%%%%%%%%%%%%%%%%%%%%%%%%%%%%%%%%%%%%%%%%
\subsection{Files and Installation}

The package consists of the files:
%
\begin{center}
\begin{tabular}{ll}
    |README.txt|   & readme file \\
    |childdoc.ins| & installation file \\
    |childdoc.dtx| & source file \\
    |childdoc.def| & definition file \\
    |cdocsamp.tex| & sample main file \\
    |cdocsch1.tex| & sample include file \\
    |cdocsch2.tex| & sample include file \\
    |cdocspt3.tex| & sample part file \\
    |cdocspt4.tex| & sample part file \\
    |cdocsdrf.tex| & sample redirection file \\
    |cdocsfn1.tex| & sample redirection file \\
    |cdocsfn2.tex| & sample redirection file \\
    |childdoc.pdf| & manual
\end{tabular}
\end{center}
%
The distribution consists of the files
|README.txt|, |childdoc.ins| and |childdoc.dtx|.
%
\begin{itemize}
\item
Run (pdf)\LaTeX{} on |childdoc.dtx|
to compile the manual |childdoc.pdf| (this file).
\item
Run \LaTeX{} on |childdoc.ins| to create the definitions file |childdoc.def|
and the sample |cdocsamp.tex| with include files
|cdocsch1.tex|, |cdocsch2.tex|, |cdocspt3.tex|, |cdocspt4.tex|,
|cdocsdrf.tex|, |cdocsfn1.tex|, |cdocsfn2.tex|.
Then copy the file |childdoc.def| to an appropriate directory of your \LaTeX{}
distribution, e.g.\ \textit{texmf-root}|/tex/latex/childdoc|.
\end{itemize}

%%%%%%%%%%%%%%%%%%%%%%%%%%%%%%%%%%%%%%%%%%%%%%%%%%%%%%%%%%%%%%%%%%%%%%%%%%%%%%%%
\subsection{Related CTAN Packages}

There are several other packages which offer a similar functionality:
%
\begin{itemize}
\item
The packages
\href{http://ctan.org/pkg/docmute}{\textsf{docmute}},
\href{http://ctan.org/pkg/includex}{\textsf{includex}} and
\href{http://ctan.org/pkg/standalone}{\textsf{standalone}}
provide commands to include only the document body of
a child file thus allowing both files to be compiled individually.
\item
The packages \href{http://ctan.org/pkg/subdocs}{\textsf{subdocs}}
and \href{http://ctan.org/pkg/subfiles}{\textsf{subfiles}}
provide structures in which the main and child documents can be
encapsulated and allowing them to be compiled individually.
The inclusion mechanism is different from the conventional |\include|.
\item
The package \href{http://ctan.org/pkg/combine}{\textsf{combine}}
is an elaborate solution to combine several documents into one.
\end{itemize}
%
See also the CTAN topic \href{http://ctan.org/topic/subdocs}{\textsf{subdocs}}
for further related packages.
The present package differs from the above solutions in that
a document structure constructed with the conventional |\include| mechanism
just needs two extra commands at the top of every file
such that all constituent files can be compiled individually.

%%%%%%%%%%%%%%%%%%%%%%%%%%%%%%%%%%%%%%%%%%%%%%%%%%%%%%%%%%%%%%%%%%%%%%%%%%%%%%%%
%\subsection{Feature Suggestions}
%
%The following is a list of features which may be useful for future
%versions of this package:
%%
%\begin{itemize}
%\item
%\ldots
%\end{itemize}

%%%%%%%%%%%%%%%%%%%%%%%%%%%%%%%%%%%%%%%%%%%%%%%%%%%%%%%%%%%%%%%%%%%%%%%%%%%%%%%%
\subsection{Revision History}

%%%%%%%%%%%%%%%%%%%%%%%%%%%%%%%%%%%%%%%%
\paragraph{v2.0:} 2018/12/30

\begin{itemize}
\item
immediate forward processing
\item
added |\childdocby| mechanism
\item
manual restructured
\end{itemize}

%%%%%%%%%%%%%%%%%%%%%%%%%%%%%%%%%%%%%%%%
\paragraph{v1.6:} 2018/01/17

\begin{itemize}
\item
application for development of include files
\item
corrections to manual
\end{itemize}

%%%%%%%%%%%%%%%%%%%%%%%%%%%%%%%%%%%%%%%%
\paragraph{v1.5:} 2017/05/21

\begin{itemize}
\item
more complete structuring introduced
\item
|\childdocof| introduced
\item
|\childdoc| renamed to |\childdocmain|
\item
|\childredirect| renamed to |\childdocforward| and |\childdocforwardprefix|
and functionality expanded
\end{itemize}

%%%%%%%%%%%%%%%%%%%%%%%%%%%%%%%%%%%%%%%%
\paragraph{v1.0:} 2017/04/27

\begin{itemize}
\item
manual and install package
\item
first version published on CTAN
\end{itemize}

%%%%%%%%%%%%%%%%%%%%%%%%%%%%%%%%%%%%%%%%
\paragraph{v0.6:} 2017/04/26

\begin{itemize}
\item
redirection mechanism added
\end{itemize}

%%%%%%%%%%%%%%%%%%%%%%%%%%%%%%%%%%%%%%%%
\paragraph{v0.5:} 2017/04/26

\begin{itemize}
\item
functionality in definition file
\end{itemize}


%%%%%%%%%%%%%%%%%%%%%%%%%%%%%%%%%%%%%%%%%%%%%%%%%%%%%%%%%%%%%%%%%%%%%%%%%%%%%%%%
%%%%%%%%%%%%%%%%%%%%%%%%%%%%%%%%%%%%%%%%%%%%%%%%%%%%%%%%%%%%%%%%%%%%%%%%%%%%%%%%
%%%%%%%%%%%%%%%%%%%%%%%%%%%%%%%%%%%%%%%%%%%%%%%%%%%%%%%%%%%%%%%%%%%%%%%%%%%%%%%%
\appendix

\settowidth\MacroIndent{\rmfamily\scriptsize 000\ }

 \DocInput{childdoc.dtx}

\end{document}
%</driver>
% \fi
%
% %%%%%%%%%%%%%%%%%%%%%%%%%%%%%%%%%%%%%%%%%%%%%%%%%%%%%%%%%%%%%%%%%%%%%%%%%%%%%%
% %%%%%%%%%%%%%%%%%%%%%%%%%%%%%%%%%%%%%%%%%%%%%%%%%%%%%%%%%%%%%%%%%%%%%%%%%%%%%%
% \section{Sample}
%\iffalse
%<*samplemain>
%\fi
%
% The following presents a sample document
% with two chapters, two parts, a title page,
% a compile flag as well as three forwarding files to set the flag.
% It consists of eight |.tex| files:
% \begin{center}
% \begin{tabular}{ll}
% |cdocsamp.tex|&main file\\
% |cdocsch1.tex|&include file for chapter 1\\
% |cdocsch2.tex|&include file for chapter 2\\
% |cdocspt3.tex|&include file for part 3\\
% |cdocspt4.tex|&include file for part 4\\
% |cdocsdrf.tex|&forwarding file for main file in draft mode\\
% |cdocsfi1.tex|&forwarding file for final version of chapter 1\\
% |cdocsfi2.tex|&forwarding file for final version of chapter 2\\
% \end{tabular}
% \end{center}
% Each of the eight files can be compiled directly by the \LaTeX{} compiler.
%
% %%%%%%%%%%%%%%%%%%%%%%%%%%%%%%%%%%%%%%
% \paragraph{Main File.}
%
% The main file is called |cdocsamp.tex|.
%
% Load the \textsf{childdoc} definitions and
% declare the filename for the main document:
%    \begin{macrocode}
\input{childdoc.def}
\childdocmain{}
%    \end{macrocode}

% Optional override for |\version| flag:
%    \begin{macrocode}
%%\ifchilddoc\else\providecommand{\version}{draft}\fi
%    \end{macrocode}

% Define the default values for the |\version| flag
% (|final| for the main file and |draft| for childs):
%    \begin{macrocode}
\ifchilddoc
\providecommand{\version}{draft}
\else
\providecommand{\version}{final}
\fi
%    \end{macrocode}

% Load the standard document class:
%    \begin{macrocode}
\documentclass[12pt]{article}
%    \end{macrocode}

% Start the document body:
%    \begin{macrocode}
\begin{document}
%    \end{macrocode}

% Declare a title page.
% Print title, part of document being processed and version flag:
%    \begin{macrocode}
\addtocounter{page}{-1}
\begin{center}
{\LARGE\bfseries{}childdoc example\par}
\vspace{1cm}
\ifchilddoc
\ifchilddocmanual part\else chapter\fi:
`\childdocname' of `\childdocjob'\par
\else
main document: `\childdocjob'\par
\fi
version: \version\par
\end{center}
\newpage
%    \end{macrocode}

% Manually include selected file,
% otherwise process as usual:
%    \begin{macrocode}
\ifchilddocmanual
\section*{part `\childdocname'}
\input{\childdocname}
\else
%    \end{macrocode}

% Include the two chapters:
%    \begin{macrocode}
\include{cdocsch1}
\include{cdocsch2}
%    \end{macrocode}

% Include the two parts unless only chapters should be displayed:
%    \begin{macrocode}
\ifchilddoc\else
\section{part three}
\input{cdocspt3}
\section{part four}
\input{cdocspt4}
\fi
%    \end{macrocode}

% Process as usual until here:
%    \begin{macrocode}
\fi
%    \end{macrocode}

% End of document body:
%    \begin{macrocode}
\end{document}
%    \end{macrocode}
%\iffalse
%</samplemain>
%\fi
%
% %%%%%%%%%%%%%%%%%%%%%%%%%%%%%%%%%%%%%%
% \paragraph{Chapter Include Files.}
%
% The include files are called |cdocsch1.tex| and |cdocsch2.tex|.
%
%\iffalse
%<*samplechap1|samplechap2>
%\fi

% Optional override for |\version| flag:
%    \begin{macrocode}
%%\providecommand{\version}{final}
%    \end{macrocode}

% Include the main document:
%    \begin{macrocode}
\input{childdoc.def}
\childdocof{cdocsamp}
%    \end{macrocode}

%\iffalse
%</samplechap1|samplechap2>
%\fi
%
%\iffalse
%<*samplechap1>
%\fi
% Some text for chapter 1:
%    \begin{macrocode}
\section{one}
some text in chapter one
%    \end{macrocode}

%\iffalse
%</samplechap1>
%\fi
% Some text for chapter 2:
%\iffalse
%<*samplechap2>
%\fi
%    \begin{macrocode}
\section{two}
more text in chapter two
%    \end{macrocode}

%\iffalse
%</samplechap2>
%\fi
%
% %%%%%%%%%%%%%%%%%%%%%%%%%%%%%%%%%%%%%%
% \paragraph{Part Include Files.}
%
% The include files are called |cdocspt3.tex| and |cdocspt4.tex|.
%
%\iffalse
%<*samplepart3|samplepart4>
%\fi

% Optional override for |\version| flag:
%    \begin{macrocode}
%%\providecommand{\version}{final}
%    \end{macrocode}

% Include the main document:
%    \begin{macrocode}
\input{childdoc.def}
\childdocby{cdocsamp}
%    \end{macrocode}

%\iffalse
%</samplepart3|samplepart4>
%\fi
%
%\iffalse
%<*samplepart3>
%\fi
% Some text for part 3:
%    \begin{macrocode}
some text in part three
%    \end{macrocode}

%\iffalse
%</samplepart3>
%\fi
% Some text for part 4:
%\iffalse
%<*samplepart4>
%\fi
%    \begin{macrocode}
more text in part four
%    \end{macrocode}

%\iffalse
%</samplepart4>
%\fi
%
% %%%%%%%%%%%%%%%%%%%%%%%%%%%%%%%%%%%%%%
% \paragraph{Forwarding for a Complete Draft.}
%
% The following forwarding file |cdocsdrf.tex|
% compiles the main document in draft mode:
%\iffalse
%<*sampledraft>
%\fi
%    \begin{macrocode}
\def\version{draft}
\input{childdoc.def}
\childdocforward{cdocsamp}
%    \end{macrocode}

%\iffalse
%</sampledraft>
%\fi
%
% %%%%%%%%%%%%%%%%%%%%%%%%%%%%%%%%%%%%%%
% \paragraph{Forwarding for Final Version of the Chapters.}
%
% The following forwarding files |cdocsfn1.tex| and |cdocsfn2.tex|
% (with identical content)
% compile the final versions of the child documents
% |cdocsch1.tex| and |cdocsch2.tex|, respectively:
%\iffalse
%<*samplefinal>
%\fi
%    \begin{macrocode}
\def\version{final}
\input{childdoc.def}
\childdocforwardprefix[cdocsamp]{cdocsfn}{cdocsch}
%    \end{macrocode}

%\iffalse
%</samplefinal>
%\fi
%
% %%%%%%%%%%%%%%%%%%%%%%%%%%%%%%%%%%%%%%
% \paragraph{Command Line Processing.}
%
% The following three command lines generate the output files
% |cdocscld|, |cdocscl1| and |cdocscl2|
% which should be identical to
% |cdocsdrf|, |cdocsch1| and |cdocsfn2|, respectively:
% \begin{center}
% \begin{tabular}{l}
% |latex -jobname cdocscld \|\\
% |  "\def\version{draft}\input{childdoc.def}\childdocforward{cdocsamp}"|\\
% |latex -jobname cdocscl1 \|\\
% |  "\input{childdoc.def}\childdocforward[cdocsamp]{cdocsch1}"|\\
% |latex -jobname cdocscl2 \|\\
% |  "\def\version{final}\input{childdoc.def}\childdocforward{cdocsch2}"|
% \end{tabular}
% \end{center}
% Note that the trailing backslash on each first line
% merely continues the input to the second line
% (for convenient cut ant paste).
% Furthermore, the command |latex| can be replaced by any
% of its alternative versions such as |pdflatex|.
%
% %%%%%%%%%%%%%%%%%%%%%%%%%%%%%%%%%%%%%%%%%%%%%%%%%%%%%%%%%%%%%%%%%%%%%%%%%%%%%%
% %%%%%%%%%%%%%%%%%%%%%%%%%%%%%%%%%%%%%%%%%%%%%%%%%%%%%%%%%%%%%%%%%%%%%%%%%%%%%%
% \section{Implementation}
%\iffalse
%<*package>
%\fi
%
% This section describes the definitions file |childdoc.def|.

% The definitions cannot be loaded using |\usepackage| or |\RequirePackage|
% which has a mechanism to prevent loading a style file more than once.
% When loading the definitions by means of |\input|
% multiple instances have to be prevented manually:
%\iffalse
%This code needs to be before the `\ProvidesFile' directive
%which is defined at the beginning of this file.
%Therefore it is also placed there and commented out here.
%</package>
%<*discard>
%\fi
%    \begin{macrocode}
\ifdefined\childdocmain\endinput\fi
%    \end{macrocode}
%\iffalse
%</discard>
%<*package>
%\fi
%
% \macro{\ifchilddoc}
% \macro{\ifchilddocmanual}
% The conditional |\ifchilddoc| tells whether a
% child (true) or main (false) document is being compiled.
% The conditional |\ifchilddocmanual| tells whether
% the |\includeonly| mechanism is used (false) or
% the selection of child files must be performed manually (true).
% The definitions initialise to false:
%    \begin{macrocode}
\newif\ifchilddoc
\newif\ifchilddocmanual
%    \end{macrocode}

% \macro{\childdocname}
% \macro{\childdocjob}
% The macro |\childdocname| stores the name of the main document
% to be compiled. The macro |\childdocjob| stores the name of
% the document on which the \LaTeX{} compiler was originally invoked.
% The content of |\jobname| cannot be compared
% to filenames specified in the source due to different catcodes.
% The following code rescans |\jobname|, stores the result
% in |\childdocname| and saves a copy in |\childdocjob|:
%    \begin{macrocode}
\edef\childdocname{\scantokens\expandafter{\jobname\noexpand}}
\let\childdocjob\childdocname
%    \end{macrocode}

% \macro{\childdocdisable}
% The macro |\childdocdisable| prevents the main file
% from being processed more than once.
% At this stage, the main document command |\childdocmain|
% is assumed to be called once again where it should do nothing.
% Any subsequent call to it should prevent
% a secondary processing of the main document
% It overwrites the forwarding commands
% |\childdocof| and |\childdocforward|
% with empty macros to prevent further inclusions of the main document:
%    \begin{macrocode}
\newcommand{\childdocdisable}
{
  \renewcommand{\childdocmain}[1]{\renewcommand{\childdocmain}[1]{\endinput}}
  \renewcommand{\childdocof}[1]{}
  \renewcommand{\childdocby}[2][]{}
  \renewcommand{\childdocforward}[2][]{}
  \renewcommand{\childdocdisable}{}
}
%    \end{macrocode}

% \macro{\childdocmain}
% The macro |\childdocmain| is to be called at the top of the main file
% with nothing or the main filename (without extension) as argument.
% First, it breaks loops.
% If the argument is not empty and does not match |\childdocname|
% (which is set by the first inclusion of |childdoc.def|),
% |\ifchilddoc| is set to true, |\includeonly| is applied to the child file
% and |\jobname| is set to the main file
% (for proper handling of |.aux| files):
%    \begin{macrocode}
\newcommand{\childdocmain}[1]
{
  \childdocdisable\childdocmain{}
  \if?#1?\else
    \begingroup
      \def\childdoctmp{#1}
      \ifx\childdoctmp\childdocname
        \def\childdoctmp{}
      \else
        \def\childdoctmp
        {
          \childdoctrue
          \includeonly{\childdocname}
          \def\childdocjob{#1}
          \def\jobname{#1}
        }
      \fi
      \expandafter
    \endgroup
    \childdoctmp
  \fi
}
%    \end{macrocode}

% \macro{\childdocof}
% The command |\childdocof| redirects
% compilation to the main file |#1|.
%    \begin{macrocode}
\newcommand{\childdocof}[1]
{
  \childdocdisable
  \childdoctrue
  \includeonly{\childdocname}
  \def\jobname{#1}
  \def\childdocjob{#1}
  \input{#1}
}
%    \end{macrocode}

% \macro{\childdocby}
% The command |\childdocby| ....
%    \begin{macrocode}
\newcommand{\childdocby}[2][]
{
  \childdocdisable
  \childdoctrue
  \childdocmanualtrue
  \if?#1?\else
    \def\jobname{#2}
  \fi
  \def\childdocjob{#2}
  \input{#2}
  \endinput
}
%    \end{macrocode}

% \macro{\childdocforward}
% The command |\childdocforward| redirects
% compilation to the main file or
% (if the optional argument is given) a child file.
% Parameters are set as if the main file
% or a child file starting with |\childdocof| was compiled.
% Then compilation is handed over to the main file:
%    \begin{macrocode}
\newcommand{\childdocforward}[2][]
{
  \begingroup
    \if?#1?
      \def\childdoctmp
      {
        \def\childdocname{#2}
        \def\childdocjob{#2}
        \def\jobname{#2}
        \input{#2}
        \endinput
      }
    \else
      \def\childdoctmp
      {
        \childdocdisable
        \def\childdocname{#2}
        \childdoctrue
        \includeonly{#2}
        \def\childdocjob{#1}
        \def\jobname{#1}
        \input{#1}
        \endinput
      }
    \fi
    \expandafter
  \endgroup
  \childdoctmp
}
%    \end{macrocode}

% \macro{\childdocforwardprefix}
% The command |\childdocforwardprefix| redirects
% compilation to the main or a child file by means of a pattern.
% The prefix |#1| in the current filename is replaced by |#2|
% and the suffix of the current filename is kept
% (it is assumed that the filename does not contain the substring `|~~~|'
% which is used as a delimiter).
% Compilation is handed over to the new file by |\childdocforward|:
%    \begin{macrocode}
\newcommand{\childdocforwardprefix}[3][]
{
  \begingroup
    \def\childdocextract #2##1~~~{\def\childdoctmp{\childdocforward[#1]{#3##1}}}
    \expandafter\childdocextract\childdocname~~~
    \expandafter
  \endgroup
  \childdoctmp
}
%    \end{macrocode}

% \macro{\childdoc}
% The deprecated macro |\childdoc| is a legacy version of |\childdocmain|:
%    \begin{macrocode}
\newcommand{\childdoc}{\childdocmain}
%    \end{macrocode}

% \macro{\childdocredirect}
% The deprecated macro |\childdocredirect| is a legacy version
% of |\childdocforward| and |\childdocforwardprefix|:
%    \begin{macrocode}
\newcommand{\childdocredirect}[2][]
{
  \begingroup
    \if?#1?
      \def\childdoctmp{\childdocforward{#2}}
    \else
      \def\childdoctmp{\childdocforwardprefix{#1}{#2}}
    \fi
    \expandafter
  \endgroup
  \childdoctmp
}
%    \end{macrocode}

%\iffalse
%</package>
%\fi
%
\endinput
\childdocforward[|\textit{main}|]{|\textit{dest}|}"|
\end{center}
%
Here \textit{target} is the name of the output file,
\textit{main} is the name of the main file
and \textit{dest} is the name of the main or child file to be processed
(all filenames without extensions).
The optional argument \textit{main} can be omitted
if \textit{main} matches \textit{dest}.
Optionally, compilation \textit{flags} can be defined via |\def| commands.
This command line makes the \TeX{} engine believe
it is compiling the file \textit{target}
whose content is specified as the latter parameter.
The provided code then forwards the processing to
\textit{main} or \textit{dest} as described in \secref{sec:forward}.

%%%%%%%%%%%%%%%%%%%%%%%%%%%%%%%%%%%%%%%%%%%%%%%%%%%%%%%%%%%%%%%%%%%%%%%%%%%%%%%%
\subsection{Include by Input}
\label{sec:input}

Including child documents by |\include| has some restrictions by design.
Most notably, the content of a child document always occupies
its own set of pages; pages cannot be shared between child documents.
Usually, this behaviour makes perfect sense
because each child document contain an essential part of the document.
However, in some situations it may be desirable to compose
a document from a collection of parts
without having mandatory page breaks between then.
For this case, the package
provides a mechanism to include parts
by |\input| which can also be processed individually.
However, by construction this mechanism
requires manual handling of the content to be output.

%%%%%%%%%%%%%%%%%%%%%%%%%%%%%%%%%%%%%%%%
\DescribeMacro{\ifchilddocmanual}
The main file should be prepared as usual, see \secref{sec:include}.
However, the document body must make a distinction
between processing of an individual part and of the main document, e.g.:
%
\begin{center}
\begin{tabular}{l}
|\ifchilddocmanual|\\
|\input{\childdocname}|\\
|\||else|\\
\textit{document body with }|\input{|\textit{part}|}|\\
|\||fi|
\end{tabular}
\end{center}
%
The conditional |\ifchilddocmanual| is true whenever
a part to be included by |\input| is being compiled,
and the name of the part is stored in |\childdocname|.

%%%%%%%%%%%%%%%%%%%%%%%%%%%%%%%%%%%%%%%%
\DescribeMacro{\childdocby}
Each part to be included by |\input| should start with:
%
\begin{center}
\begin{tabular}{l}
|% \iffalse
%
% childdoc.dtx Copyright (C) 2017-2018 Niklas Beisert
%
% This work may be distributed and/or modified under the
% conditions of the LaTeX Project Public License, either version 1.3
% of this license or (at your option) any later version.
% The latest version of this license is in
%   http://www.latex-project.org/lppl.txt
% and version 1.3 or later is part of all distributions of LaTeX
% version 2005/12/01 or later.
%
% This work has the LPPL maintenance status `maintained'.
%
% The Current Maintainer of this work is Niklas Beisert.
%
% This work consists of the files childdoc.dtx and childdoc.ins
% and the derived files childdoc.def and cdocsamp.tex with
% cdocsch1.tex, cdocsch2.tex, cdocsdrf.tex, cdocsfn1.tex, cdocsfn2.tex.
%
%<package>\ifdefined\childdocmain\endinput\fi
%<package>\ProvidesFile{childdoc.def}[2018/12/30 v2.0 child document driver]
%<samplemain>\ProvidesFile{cdocsamp.tex}[2018/12/30 v2.0 sample for childdoc]
%<*driver>
%\ProvidesFile{childdoc.drv}[2018/12/30 v2.0 childdoc reference manual file]
\PassOptionsToClass{10pt,a4paper}{article}
\documentclass{ltxdoc}

\usepackage[margin=35mm]{geometry}
\usepackage{hyperref}
\usepackage{hyperxmp}
\usepackage[usenames]{color}

\hypersetup{colorlinks=true}
\hypersetup{pdfstartview=FitH}
\hypersetup{pdfpagemode=UseNone}
\hypersetup{pdfsource={}}
\hypersetup{pdflang={en-UK}}
\hypersetup{pdfcopyright={Copyright 2017-2018 Niklas Beisert.
  This work may be distributed and/or modified under the
  conditions of the LaTeX Project Public License, either version 1.3
  of this license or (at your option) any later version.}}
\hypersetup{pdflicenseurl={http://www.latex-project.org/lppl.txt}}
\hypersetup{pdfcontactaddress={ETH Zurich, ITP, HIT K,
  Wolfgang-Pauli-Strasse 27}}
\hypersetup{pdfcontactpostcode={8093}}
\hypersetup{pdfcontactcity={Zurich}}
\hypersetup{pdfcontactcountry={Switzerland}}
\hypersetup{pdfcontactemail={nbeisert@itp.phys.ethz.ch}}
\hypersetup{pdfcontacturl={http://people.phys.ethz.ch/\xmptilde nbeisert/}}

\newcommand{\secref}[1]{\hyperref[#1]{section \ref*{#1}}}

\parskip1ex
\parindent0pt
\let\olditemize\itemize
\def\itemize{\olditemize\parskip0pt}

\begin{document}

\title{The \textsf{childdoc} Package}
\hypersetup{pdftitle={The childdoc Package}}
\author{Niklas Beisert\\[2ex]
  Institut f\"ur Theoretische Physik\\
  Eidgen\"ossische Technische Hochschule Z\"urich\\
  Wolfgang-Pauli-Strasse 27, 8093 Z\"urich, Switzerland\\[1ex]
  \href{mailto:nbeisert@itp.phys.ethz.ch}
  {\texttt{nbeisert@itp.phys.ethz.ch}}}
\hypersetup{pdfauthor={Niklas Beisert}}
\hypersetup{pdfsubject={Manual for the LaTeX2e Package childdoc}}
\date{30 December 2018, \textsf{v2.0}}
\maketitle

\begin{abstract}\noindent
\textsf{childdoc} is a \LaTeXe{} package
that enables the direct compilation
of document sections included by |\include|
to individual files.
\end{abstract}

\begingroup
\parskip0ex
\tableofcontents
\endgroup

%%%%%%%%%%%%%%%%%%%%%%%%%%%%%%%%%%%%%%%%%%%%%%%%%%%%%%%%%%%%%%%%%%%%%%%%%%%%%%%%
%%%%%%%%%%%%%%%%%%%%%%%%%%%%%%%%%%%%%%%%%%%%%%%%%%%%%%%%%%%%%%%%%%%%%%%%%%%%%%%%
\section{Introduction}

\LaTeX{} provides a mechanism to structure a large document (such as a book)
into a main file and several child files (containing the chapters)
using the |\include| command.
This mechanism is beneficial for documents
which span hundreds of pages in order to
make the source file(s) more manageable.
Moreover, compilation can be restricted to
selected child files by means of the |\includeonly| command.
The latter feature can be used to reduce the compilation time while editing
(this was significantly more useful in the earlier days of \LaTeX{})
or to generate a smaller document which is easier to navigate.
Another application of |\includeonly| is to generate
documents consisting of selected parts of the complete document.

However, there are a few drawbacks of the plain |\include| mechanism:
\begin{itemize}
\item
The child files cannot be compiled on their own,
they can only be compiled via the main file.
A naive editing environment
(such as a text editor with an option
to have the current file processed by \LaTeX)
may require one to switch to the main file before compiling;
attempting to compile the child file produces errors.
\item
The main file must be modified (each time)
to adjust the |\includeonly| command
to the present needs. This easily leaves the main file in a messy state.
\item
The generated document will always carry the filename
of the main document. This is inconvenient if
several child files are to be compiled and
to be kept for distribution.
\end{itemize}

The present package provides a simple interface
to make child files individually compilable by \LaTeX{}.
Compiling a child file then has the same effect as compiling
the main file with an |\includeonly| command
to select the appropriate child.
Moreover the generated document will carry the name of the child
rather than the main file.
This resolves all three above issues.

This feature is meant to make the editing of books,
thesis documents and lecture notes somewhat more convenient.
However, the package can also be used efficiently for
composing a series of documents (such as exercise sheets)
which are typically distributed individually.
It then assists the author in generating the individual documents
(potentially in different versions)
as well as a document containing the collected series.
Another application is in developing style files
or other kinds of included material
where compilation of the style file could redirect
to a sample or test file.

%%%%%%%%%%%%%%%%%%%%%%%%%%%%%%%%%%%%%%%%%%%%%%%%%%%%%%%%%%%%%%%%%%%%%%%%%%%%%%%%
%%%%%%%%%%%%%%%%%%%%%%%%%%%%%%%%%%%%%%%%%%%%%%%%%%%%%%%%%%%%%%%%%%%%%%%%%%%%%%%%
\section{Usage}

First of all, the package \textsf{childdoc} is \emph{not} a standard
\LaTeXe{} |.sty| style file! Therefore it needs to be invoked in
a non-standard way.

%%%%%%%%%%%%%%%%%%%%%%%%%%%%%%%%%%%%%%%%%%%%%%%%%%%%%%%%%%%%%%%%%%%%%%%%%%%%%%%%
\subsection{Included Files}
\label{sec:include}

%%%%%%%%%%%%%%%%%%%%%%%%%%%%%%%%%%%%%%%%
\DescribeMacro{\childdocmain}
To use the package, add the commands
\begin{center}
\begin{tabular}{l}
|\input{childdoc.def}|\\
|\childdocmain{}|\\
\end{tabular}
\end{center}
at the very top of the main \LaTeX{} file,
in particular \emph{before} the |\documentclass| statement!
The argument of |\childdocmain| should be left empty
(but it must be present).

%%%%%%%%%%%%%%%%%%%%%%%%%%%%%%%%%%%%%%%%
\DescribeMacro{\childdocof}
Furthermore, add the commands
\begin{center}
\begin{tabular}{l}
|\input{childdoc.def}|\\
|\childdocof{|\textit{main}|}|\\
\end{tabular}
\end{center}
at the top of every child file \textit{child}
which is included by |\include{|\textit{child}|}|
from within the main file
(or at least for those files to be compiled individually).
The argument \textit{main} must be the filename of the main file.

There are a couple of
considerations in setting up the main and child documents:

%%%%%%%%%%%%%%%%%%%%%%%%%%%%%%%%%%%%%%%%
\paragraph{Restrictions.}

Please note the following restrictions:
\begin{itemize}
\item
|\childdocmain| must be called with one argument \textit{main}
to ensure compatibility with earlier version of the package.
It must either be empty (|\childdocmain{}|)
or precisely match the filename of the main file in which it is specified.
See \secref{sec:detection} for further information.
\item
The filename \textit{main} must be specified without the |.tex| extension.
\item
The filename \textit{main} is case sensitive
(even in case-insensitive file systems)
due to internal string comparison.
\item
The argument \textit{main} should be fully expanded, it cannot be a macro.
\item
Subdirectories and special characters should be avoided in filenames.
\item
The command |\childdocmain{|\textit{main}|}| must be followed by a whitespace.
It should not be followed immediately by another command
or by a comment mark `|%|'.
This is because the \TeX{} parser reads the token immediately following
the argument of |\childdocmain| and puts it
at the beginning of every child section;
however, a white\-space is ignored.
\end{itemize}

%%%%%%%%%%%%%%%%%%%%%%%%%%%%%%%%%%%%%%%%
\paragraph{Content of Main File.}

It is advisable to place all content in the child files included by |\include|.
Any output contained in the main file will appear in all child documents
unless suppressed manually;
it cannot be suppressed automatically by the |\includeonly| directive
and thus should normally be avoided.
A method to include some content in the main file
by means of conditional processing is described in \secref{sec:conditional}.

%%%%%%%%%%%%%%%%%%%%%%%%%%%%%%%%%%%%%%%%
\paragraph{Page Numbering.}

When only a part of the document is compiled,
the appropriate numbering of pages
(as well as other status parameters)
is determined from the |.aux| files.
The latter contain information from previous passes.
However this information needs to propagate through
all intermediate child documents.
Therefore the page numbering in child documents may well
be inconsistent until the complete document is compiled at least once.

A useful (if unconventional) way to always ensure a consistent
page numbering is to restart the numbering in each child document
and denote the pages by `\textit{child}|.|\textit{page}'
where \textit{child} represents the chapter/section number of the child file.
This can be achieved by the command
|\numberwithin{page}{|\textit{child}|}|
of the \textsf{amsmath} package
where \textit{child} can be |chapter| or |section|
depending on the chosen structuring.
Alternatively, one can modify the macro |\thepage| appropriately
and reset the counter |page| at the start of each child file.

%%%%%%%%%%%%%%%%%%%%%%%%%%%%%%%%%%%%%%%%%%%%%%%%%%%%%%%%%%%%%%%%%%%%%%%%%%%%%%%%
\subsection{Conditional Processing}
\label{sec:conditional}

The package provides a mechanism to compile different versions
of a document. To customise the versions further some conditional processing
can come in handy to distinguish which version is being compiled.
The package provides two macros to describe the compilation context:

%%%%%%%%%%%%%%%%%%%%%%%%%%%%%%%%%%%%%%%%
\DescribeMacro{\ifchilddoc}
The conditional |\ifchilddoc| distinguishes between the compilation of
child documents and the main document:
%
\begin{center}
|\ifchilddoc |\textit{child-code}| |[|\||else |\textit{main-code}]| \||fi|
\end{center}

%%%%%%%%%%%%%%%%%%%%%%%%%%%%%%%%%%%%%%%%
\DescribeMacro{\childdocname}
\DescribeMacro{\childdocjob}
The macro |\childdocname| contains the filename (without extension)
of the main or child file being processed.
Note that |\childdocjob| will always contain the name of the main file.

%%%%%%%%%%%%%%%%%%%%%%%%%%%%%%%%%%%%%%%%
\paragraph{Title Page.}

Conditional processing can be used to include a title or banner page
in the main document when proper precautions are taken.
Importantly, the code in the main file should ensure that the page counter
(as well as other status parameters which are stored in the |.aux| files)
takes the same value after the conditional processing.
Otherwise the page numbers may take divergent values
depending on which part is compiled.

For example, a title page could be declared by:
%
\begin{center}
\begin{tabular}{l}
|\ifchilddoc\||else|\\
|\addtocounter{page}{-1}|\\
\textit{code for title page}\\
|\newpage|\\
|\||fi|
\end{tabular}
\end{center}
%
A banner page for the child documents can be generated by:
%
\begin{center}
\begin{tabular}{l}
|\ifchilddoc|\\
|\addtocounter{page}{-1}|\\
\textit{code for banner page}\\
|\newpage|\\
|\||fi|
\end{tabular}
\end{center}
%
Here one could write a message such as:
\begin{center}
|This is the part \childdocname{} of \childdocjob{}.|
\end{center}

%%%%%%%%%%%%%%%%%%%%%%%%%%%%%%%%%%%%%%%%%%%%%%%%%%%%%%%%%%%%%%%%%%%%%%%%%%%%%%%%
\subsection{Flags}
\label{sec:flags}

The package makes it easy to generate different versions
of the main or child documents.
To this end compilation flags can be defined
and assigned different default values.
They will be particularly useful in conjunction
with the forwarding mechanism described in \secref{sec:forward}.

For example, it may be useful to have a flag |\version|
which can be set to |draft| or |final|.
The document source will contain some conditional code
depending on the value of |\version|.
Suppose further, the flag should default to |final| for the main file
and to |draft| for child files
which is a natural assignment for editing the document.
This is achieved by placing the following code
in the preamble of the main document
(below the |\childdocmain| directive):
%
\begin{center}
\begin{tabular}{l}
|\ifchilddoc|\\
|\providecommand{\version}{draft}|\\
|\||else|\\
|\providecommand{\version}{final}|\\
|\||fi|
\end{tabular}
\end{center}
%
The definition by |\providecommand| makes sure
that previous definitions are not overwritten.
Further statements |\providecommand{\version}{...}|
can thus be added before the above code to override it.

For the main file, one might add a line
(between |\childdocmain| and the above block)
%
\begin{center}
|%\ifchilddoc\||else\providecommand{\version}{draft}\||fi|
\end{center}
%
which can be uncommented to produce a draft version.
Likewise one can add a line to the very top of a child file
(above the |\childdocof{|\textit{main}|}| directive)
%
\begin{center}
|%\providecommand{\version}{final}|
\end{center}
%
which can be uncommented to produce the final version of this child document.

%%%%%%%%%%%%%%%%%%%%%%%%%%%%%%%%%%%%%%%%%%%%%%%%%%%%%%%%%%%%%%%%%%%%%%%%%%%%%%%%
\subsection{Forwarding}
\label{sec:forward}

Different versions of the main or child documents
using compilation flags as described in \secref{sec:flags}
can be (permanently) stored in different files
for convenient compilation, viewing and distribution.
To this end, the package defines a command
to pass on compilation to a different file:

%%%%%%%%%%%%%%%%%%%%%%%%%%%%%%%%%%%%%%%%
\DescribeMacro{\childdocforward}
The command |\childdocforward| redirects processing to
another source file:
%
\begin{center}
\begin{tabular}{l}
|\input{childdoc.def}|\\
|\childdocforward[|\textit{main}|]{|\textit{dest}|}|\\
\end{tabular}
\end{center}
%
The argument \textit{dest} is the destination file
(without extension).
It should be the main file or one of the child files.
Note that further \textsf{childdoc} directives
such as |\childdocof| and |\childdocforward|
in the indicated file will be processed in this form.
The optional argument \textit{main}
passes on directly to the main file \textit{main}
while pretending to compile the child \textit{dest}.
This form behaves as if \textit{dest}
issues |\childdocof{|\textit{main}|}| right away,
and no further \textsf{childdoc} directives will be processed.

%%%%%%%%%%%%%%%%%%%%%%%%%%%%%%%%%%%%%%%%
\DescribeMacro{\...prefix}
In the alternative form |\childdocforwardprefix|,
%
\begin{center}
\begin{tabular}{l}
|\input{childdoc.def}|\\
|\childdocforwardprefix[|\textit{main}|]{|\textit{prefix}|}{|\textit{dest}|}|
\end{tabular}
\end{center}
%
the destination file is determined by a pattern
depending on the current file:
To make this work, the current file must be called
`{\textit{prefix}\hspace{0.2em}\textit{suffix}}'
with \textit{prefix} matching precisely the argument.
Processing is then passed on to the file
`{\textit{dest}\hspace{0.2em}\textit{suffix}}'.
Surely, the same effect is achieved by
directly specifying the
argument `{\textit{dest}\hspace{0.2em}\textit{suffix}}'
in the first form.
However, that requires to set up a different file
for each child. With the alternative form of the command
all these files can have exactly the same content
which simplifies setting them up and maintaining them.

For example, the following file |draft.tex|
with a compilation flag |\version| as described in \secref{sec:flags}
compiles the main document as a draft:
%
\begin{center}
\begin{tabular}{l}
|\def\version{draft}|\\
|\input{childdoc.def}|\\
|\childdocforward{|\textit{main}|}|
\end{tabular}
\end{center}
%
Likewise, the following files |final|\textit{nn}|.tex|
compile the final version of the child document
|child|\textit{nn}|.tex|:
%
\begin{center}
\begin{tabular}{l}
|\def\version{final}|\\
|\input{childdoc.def}|\\
|\childdocforwardprefix{final}{child}|
\end{tabular}
\end{center}
%

Note that when several versions of a main file and/or of each child file
are to be generated, it may be convenient to set up a |Makefile| or
shell script to automatise the process.

%%%%%%%%%%%%%%%%%%%%%%%%%%%%%%%%%%%%%%%%%%%%%%%%%%%%%%%%%%%%%%%%%%%%%%%%%%%%%%%%
\subsection{Command Line Processing}
\label{sec:commandline}

The effect of redirection files can also be achieved by invoking
the \LaTeX{} compiler with a more elaborate command line.
Most conveniently this should be done as part
of a shell script or a |Makefile|.

When using \textsf{childdoc} in the main file, the following
command lines effectively perform a redirection
(note that depending on the shell being used,
backslashes may have to be doubled: `|\|' $\to$ `|\\|'):
%
\begin{center}
|... -jobname "|\textit{target}|" |\\|"|[\textit{flags}]%
|\input{childdoc.def}\childdocforward[|\textit{main}|]{|\textit{dest}|}"|
\end{center}
%
Here \textit{target} is the name of the output file,
\textit{main} is the name of the main file
and \textit{dest} is the name of the main or child file to be processed
(all filenames without extensions).
The optional argument \textit{main} can be omitted
if \textit{main} matches \textit{dest}.
Optionally, compilation \textit{flags} can be defined via |\def| commands.
This command line makes the \TeX{} engine believe
it is compiling the file \textit{target}
whose content is specified as the latter parameter.
The provided code then forwards the processing to
\textit{main} or \textit{dest} as described in \secref{sec:forward}.

%%%%%%%%%%%%%%%%%%%%%%%%%%%%%%%%%%%%%%%%%%%%%%%%%%%%%%%%%%%%%%%%%%%%%%%%%%%%%%%%
\subsection{Include by Input}
\label{sec:input}

Including child documents by |\include| has some restrictions by design.
Most notably, the content of a child document always occupies
its own set of pages; pages cannot be shared between child documents.
Usually, this behaviour makes perfect sense
because each child document contain an essential part of the document.
However, in some situations it may be desirable to compose
a document from a collection of parts
without having mandatory page breaks between then.
For this case, the package
provides a mechanism to include parts
by |\input| which can also be processed individually.
However, by construction this mechanism
requires manual handling of the content to be output.

%%%%%%%%%%%%%%%%%%%%%%%%%%%%%%%%%%%%%%%%
\DescribeMacro{\ifchilddocmanual}
The main file should be prepared as usual, see \secref{sec:include}.
However, the document body must make a distinction
between processing of an individual part and of the main document, e.g.:
%
\begin{center}
\begin{tabular}{l}
|\ifchilddocmanual|\\
|\input{\childdocname}|\\
|\||else|\\
\textit{document body with }|\input{|\textit{part}|}|\\
|\||fi|
\end{tabular}
\end{center}
%
The conditional |\ifchilddocmanual| is true whenever
a part to be included by |\input| is being compiled,
and the name of the part is stored in |\childdocname|.

%%%%%%%%%%%%%%%%%%%%%%%%%%%%%%%%%%%%%%%%
\DescribeMacro{\childdocby}
Each part to be included by |\input| should start with:
%
\begin{center}
\begin{tabular}{l}
|\input{childdoc.def}|\\
|\childdocby{|\textit{main}|}|\\
\end{tabular}
\end{center}
%
The directive |\childdocby| is similar to |\childdocof|
described in \secref{sec:include},
but the subsequent selection of content must be done manually.
To that end, both |\ifchilddoc| and |\ifchilddocmanual|
will be true upon processing of a part,
and the name of the part is stored in |\childdocname|.
Note that |\jobname| will be set to the filename of the current part
so that each part receives an individual |.aux| file
that does not interfere with the |.aux| file(s) of the main document.
This behaviour can be altered by the alternative form
|\childdocby[*]{|\textit{main}|}| (with a non-empty optional argument)
which uses the |.aux| file of the main document
by setting |\jobname| to \textit{main}.

%%%%%%%%%%%%%%%%%%%%%%%%%%%%%%%%%%%%%%%%%%%%%%%%%%%%%%%%%%%%%%%%%%%%%%%%%%%%%%%%
\subsection{Driver Development}
\label{sec:driver}

The \textsf{childdoc} mechanism can also be use for the development
of definition files such as \LaTeX{} styles or classes.
This case differs from the above setup with multiple parts
included by |\include| in that no |\includeonly| should be invoked.
This can be achieved by starting the include file
(before |\ProvidesPackage|) with:
%
\begin{center}
\begin{tabular}{l}
|\input{childdoc.def}|\\
|\childdocforward{|\textit{main}|}|\\
\end{tabular}
\end{center}
%
or alternatively with:
%
\begin{center}
\begin{tabular}{l}
|\input{childdoc.def}|\\
|\childdocby{|\textit{main}|}|\\
\end{tabular}
\end{center}
%
Both forms have slightly different effects as described above.
The main file is prepared as usual, see \secref{sec:include}.

%%%%%%%%%%%%%%%%%%%%%%%%%%%%%%%%%%%%%%%%%%%%%%%%%%%%%%%%%%%%%%%%%%%%%%%%%%%%%%%%
\subsection{Legacy Detection}
\label{sec:detection}

The directive |\childdocmain| in the main file can detect
whether the complete document or merely a child is to be compiled
even without using the directive |\childdocof|.
This method is deprecated because it is less robust
and there is no compelling reason to use it;
it is merely provided for backward compatibility
and it may be removed in future versions.

If the detection mechanism is to be used,
it is mandatory to correctly specify
the filename of the main file as the argument of |\childdocmain|:
%
\begin{center}
\begin{tabular}{l}
|\input{childdoc.def}|\\
|\childdocmain{|\textit{main}|}|\\
\end{tabular}
\end{center}
%
If |\jobname| does not match the argument \textit{main} of |\childdocmain|,
it is assumed that |\jobname| points to the child file to be compiled.
When using |\childdocmain| with the main file specified as argument,
it suffices to start a child file
with just |\input{|\textit{main}|}|
without loading of the package and using |\childdocof|.
If instead all processing is done
with the appropriate \textsf{childdoc} directives,
the argument of \textit{main} of |\childdocmain| can be empty.

An alternative version of the command line processing described
in \secref{sec:commandline} using the detection mechanism reads:
%
\begin{center}
|... -jobname "|\textit{target}|" "|[\textit{flags}]%
[|\def\jobname{|\textit{dest}|}|]|\input{|\textit{main}|}"|
\end{center}

%%%%%%%%%%%%%%%%%%%%%%%%%%%%%%%%%%%%%%%%%%%%%%%%%%%%%%%%%%%%%%%%%%%%%%%%%%%%%%%%
\subsection{Manual Code}
\label{sec:manual}

In case one cannot be certain whether the definitions file |childdoc.def|
is installed on the target \TeX{} distribution
and one prefers not to ship it,
it is conceivable to paste a few relevant commands into the sources.

To that end, drop all statements |\input{childdoc.def}|
and perform the replacements as outlined below.
Instead of |\childdocmain{|\textit{main}|}| add the following code
to the top of the main file:
%
\begin{center}
\begin{tabular}{l}
|\||ifdefined\childdocname\endinput\||fi\newif\ifchilddoc|\\
|\edef\childdocname{\scantokens\expandafter{\jobname\noexpand}}|\\
|\def\childdocmain{|\textit{main}|}\||ifx\childdocmain\childdocname\||else|\\
|\childdoctrue\includeonly{\childdocname}\let\jobname\childdocmain\||fi|\\
\end{tabular}
\end{center}
%
Instead of |\childdocof{|\textit{main}|}| just include the main file
at the top of each child file:
%
\begin{center}
|\input{|\textit{main}|}|
\end{center}
%
A simple redirection |\childdocforward{|\textit{dest}|}| is achieved by:
%
\begin{center}
|\def\jobname{|\textit{dest}|}\input{\jobname}|
\end{center}
%
The redirection with prefix
|\childdocforwardprefix[|\textit{prefix}|]{|\textit{dest}|}|
is accomplished by:
%
\begin{center}
\begin{tabular}{l}
|{\edef\jobname{\scantokens\expandafter{\jobname\noexpand}}|\\
|\def\redirectjob |\textit{prefix}|#1~~~{\gdef\jobname{|\textit{dest}|#1}}|\\
|\expandafter\redirectjob\jobname~~~}\input{\jobname}|
\end{tabular}
\end{center}

In an alternative approach,
child documents can be compiled by a specific command line
without additional code or specific definitions:
%
\begin{center}
|... -jobname "|\textit{target}|" "|[\textit{flags}]%
|\includeonly{|\textit{dest}|}\input{|\textit{main}|}"|
\end{center}
%

%%%%%%%%%%%%%%%%%%%%%%%%%%%%%%%%%%%%%%%%%%%%%%%%%%%%%%%%%%%%%%%%%%%%%%%%%%%%%%%%
%%%%%%%%%%%%%%%%%%%%%%%%%%%%%%%%%%%%%%%%%%%%%%%%%%%%%%%%%%%%%%%%%%%%%%%%%%%%%%%%
\section{Information}

%%%%%%%%%%%%%%%%%%%%%%%%%%%%%%%%%%%%%%%%%%%%%%%%%%%%%%%%%%%%%%%%%%%%%%%%%%%%%%%%
\subsection{Copyright}

Copyright \copyright{} 2017--2018 Niklas Beisert

This work may be distributed and/or modified under the
conditions of the \LaTeX{} Project Public License, either version 1.3
of this license or (at your option) any later version.
The latest version of this license is in
  \url{http://www.latex-project.org/lppl.txt}
and version 1.3 or later is part of all distributions of \LaTeX{}
version 2005/12/01 or later.

This work has the LPPL maintenance status `maintained'.

The Current Maintainer of this work is Niklas Beisert.

This work consists of the files |README.txt|, |childdoc.ins| and |childdoc.dtx|
as well as the derived files |childdoc.def|, |cdocsamp.tex|
with |cdocsch1.tex|, |cdocsch2.tex|, |cdocspt3.tex|, |cdocspt4.tex|,
|cdocsdrf.tex|, |cdocsfn1.tex|, |cdocsfn2.tex|
as well as |childdoc.pdf|.

%%%%%%%%%%%%%%%%%%%%%%%%%%%%%%%%%%%%%%%%%%%%%%%%%%%%%%%%%%%%%%%%%%%%%%%%%%%%%%%%
\subsection{Files and Installation}

The package consists of the files:
%
\begin{center}
\begin{tabular}{ll}
    |README.txt|   & readme file \\
    |childdoc.ins| & installation file \\
    |childdoc.dtx| & source file \\
    |childdoc.def| & definition file \\
    |cdocsamp.tex| & sample main file \\
    |cdocsch1.tex| & sample include file \\
    |cdocsch2.tex| & sample include file \\
    |cdocspt3.tex| & sample part file \\
    |cdocspt4.tex| & sample part file \\
    |cdocsdrf.tex| & sample redirection file \\
    |cdocsfn1.tex| & sample redirection file \\
    |cdocsfn2.tex| & sample redirection file \\
    |childdoc.pdf| & manual
\end{tabular}
\end{center}
%
The distribution consists of the files
|README.txt|, |childdoc.ins| and |childdoc.dtx|.
%
\begin{itemize}
\item
Run (pdf)\LaTeX{} on |childdoc.dtx|
to compile the manual |childdoc.pdf| (this file).
\item
Run \LaTeX{} on |childdoc.ins| to create the definitions file |childdoc.def|
and the sample |cdocsamp.tex| with include files
|cdocsch1.tex|, |cdocsch2.tex|, |cdocspt3.tex|, |cdocspt4.tex|,
|cdocsdrf.tex|, |cdocsfn1.tex|, |cdocsfn2.tex|.
Then copy the file |childdoc.def| to an appropriate directory of your \LaTeX{}
distribution, e.g.\ \textit{texmf-root}|/tex/latex/childdoc|.
\end{itemize}

%%%%%%%%%%%%%%%%%%%%%%%%%%%%%%%%%%%%%%%%%%%%%%%%%%%%%%%%%%%%%%%%%%%%%%%%%%%%%%%%
\subsection{Related CTAN Packages}

There are several other packages which offer a similar functionality:
%
\begin{itemize}
\item
The packages
\href{http://ctan.org/pkg/docmute}{\textsf{docmute}},
\href{http://ctan.org/pkg/includex}{\textsf{includex}} and
\href{http://ctan.org/pkg/standalone}{\textsf{standalone}}
provide commands to include only the document body of
a child file thus allowing both files to be compiled individually.
\item
The packages \href{http://ctan.org/pkg/subdocs}{\textsf{subdocs}}
and \href{http://ctan.org/pkg/subfiles}{\textsf{subfiles}}
provide structures in which the main and child documents can be
encapsulated and allowing them to be compiled individually.
The inclusion mechanism is different from the conventional |\include|.
\item
The package \href{http://ctan.org/pkg/combine}{\textsf{combine}}
is an elaborate solution to combine several documents into one.
\end{itemize}
%
See also the CTAN topic \href{http://ctan.org/topic/subdocs}{\textsf{subdocs}}
for further related packages.
The present package differs from the above solutions in that
a document structure constructed with the conventional |\include| mechanism
just needs two extra commands at the top of every file
such that all constituent files can be compiled individually.

%%%%%%%%%%%%%%%%%%%%%%%%%%%%%%%%%%%%%%%%%%%%%%%%%%%%%%%%%%%%%%%%%%%%%%%%%%%%%%%%
%\subsection{Feature Suggestions}
%
%The following is a list of features which may be useful for future
%versions of this package:
%%
%\begin{itemize}
%\item
%\ldots
%\end{itemize}

%%%%%%%%%%%%%%%%%%%%%%%%%%%%%%%%%%%%%%%%%%%%%%%%%%%%%%%%%%%%%%%%%%%%%%%%%%%%%%%%
\subsection{Revision History}

%%%%%%%%%%%%%%%%%%%%%%%%%%%%%%%%%%%%%%%%
\paragraph{v2.0:} 2018/12/30

\begin{itemize}
\item
immediate forward processing
\item
added |\childdocby| mechanism
\item
manual restructured
\end{itemize}

%%%%%%%%%%%%%%%%%%%%%%%%%%%%%%%%%%%%%%%%
\paragraph{v1.6:} 2018/01/17

\begin{itemize}
\item
application for development of include files
\item
corrections to manual
\end{itemize}

%%%%%%%%%%%%%%%%%%%%%%%%%%%%%%%%%%%%%%%%
\paragraph{v1.5:} 2017/05/21

\begin{itemize}
\item
more complete structuring introduced
\item
|\childdocof| introduced
\item
|\childdoc| renamed to |\childdocmain|
\item
|\childredirect| renamed to |\childdocforward| and |\childdocforwardprefix|
and functionality expanded
\end{itemize}

%%%%%%%%%%%%%%%%%%%%%%%%%%%%%%%%%%%%%%%%
\paragraph{v1.0:} 2017/04/27

\begin{itemize}
\item
manual and install package
\item
first version published on CTAN
\end{itemize}

%%%%%%%%%%%%%%%%%%%%%%%%%%%%%%%%%%%%%%%%
\paragraph{v0.6:} 2017/04/26

\begin{itemize}
\item
redirection mechanism added
\end{itemize}

%%%%%%%%%%%%%%%%%%%%%%%%%%%%%%%%%%%%%%%%
\paragraph{v0.5:} 2017/04/26

\begin{itemize}
\item
functionality in definition file
\end{itemize}


%%%%%%%%%%%%%%%%%%%%%%%%%%%%%%%%%%%%%%%%%%%%%%%%%%%%%%%%%%%%%%%%%%%%%%%%%%%%%%%%
%%%%%%%%%%%%%%%%%%%%%%%%%%%%%%%%%%%%%%%%%%%%%%%%%%%%%%%%%%%%%%%%%%%%%%%%%%%%%%%%
%%%%%%%%%%%%%%%%%%%%%%%%%%%%%%%%%%%%%%%%%%%%%%%%%%%%%%%%%%%%%%%%%%%%%%%%%%%%%%%%
\appendix

\settowidth\MacroIndent{\rmfamily\scriptsize 000\ }

 \DocInput{childdoc.dtx}

\end{document}
%</driver>
% \fi
%
% %%%%%%%%%%%%%%%%%%%%%%%%%%%%%%%%%%%%%%%%%%%%%%%%%%%%%%%%%%%%%%%%%%%%%%%%%%%%%%
% %%%%%%%%%%%%%%%%%%%%%%%%%%%%%%%%%%%%%%%%%%%%%%%%%%%%%%%%%%%%%%%%%%%%%%%%%%%%%%
% \section{Sample}
%\iffalse
%<*samplemain>
%\fi
%
% The following presents a sample document
% with two chapters, two parts, a title page,
% a compile flag as well as three forwarding files to set the flag.
% It consists of eight |.tex| files:
% \begin{center}
% \begin{tabular}{ll}
% |cdocsamp.tex|&main file\\
% |cdocsch1.tex|&include file for chapter 1\\
% |cdocsch2.tex|&include file for chapter 2\\
% |cdocspt3.tex|&include file for part 3\\
% |cdocspt4.tex|&include file for part 4\\
% |cdocsdrf.tex|&forwarding file for main file in draft mode\\
% |cdocsfi1.tex|&forwarding file for final version of chapter 1\\
% |cdocsfi2.tex|&forwarding file for final version of chapter 2\\
% \end{tabular}
% \end{center}
% Each of the eight files can be compiled directly by the \LaTeX{} compiler.
%
% %%%%%%%%%%%%%%%%%%%%%%%%%%%%%%%%%%%%%%
% \paragraph{Main File.}
%
% The main file is called |cdocsamp.tex|.
%
% Load the \textsf{childdoc} definitions and
% declare the filename for the main document:
%    \begin{macrocode}
\input{childdoc.def}
\childdocmain{}
%    \end{macrocode}

% Optional override for |\version| flag:
%    \begin{macrocode}
%%\ifchilddoc\else\providecommand{\version}{draft}\fi
%    \end{macrocode}

% Define the default values for the |\version| flag
% (|final| for the main file and |draft| for childs):
%    \begin{macrocode}
\ifchilddoc
\providecommand{\version}{draft}
\else
\providecommand{\version}{final}
\fi
%    \end{macrocode}

% Load the standard document class:
%    \begin{macrocode}
\documentclass[12pt]{article}
%    \end{macrocode}

% Start the document body:
%    \begin{macrocode}
\begin{document}
%    \end{macrocode}

% Declare a title page.
% Print title, part of document being processed and version flag:
%    \begin{macrocode}
\addtocounter{page}{-1}
\begin{center}
{\LARGE\bfseries{}childdoc example\par}
\vspace{1cm}
\ifchilddoc
\ifchilddocmanual part\else chapter\fi:
`\childdocname' of `\childdocjob'\par
\else
main document: `\childdocjob'\par
\fi
version: \version\par
\end{center}
\newpage
%    \end{macrocode}

% Manually include selected file,
% otherwise process as usual:
%    \begin{macrocode}
\ifchilddocmanual
\section*{part `\childdocname'}
\input{\childdocname}
\else
%    \end{macrocode}

% Include the two chapters:
%    \begin{macrocode}
\include{cdocsch1}
\include{cdocsch2}
%    \end{macrocode}

% Include the two parts unless only chapters should be displayed:
%    \begin{macrocode}
\ifchilddoc\else
\section{part three}
\input{cdocspt3}
\section{part four}
\input{cdocspt4}
\fi
%    \end{macrocode}

% Process as usual until here:
%    \begin{macrocode}
\fi
%    \end{macrocode}

% End of document body:
%    \begin{macrocode}
\end{document}
%    \end{macrocode}
%\iffalse
%</samplemain>
%\fi
%
% %%%%%%%%%%%%%%%%%%%%%%%%%%%%%%%%%%%%%%
% \paragraph{Chapter Include Files.}
%
% The include files are called |cdocsch1.tex| and |cdocsch2.tex|.
%
%\iffalse
%<*samplechap1|samplechap2>
%\fi

% Optional override for |\version| flag:
%    \begin{macrocode}
%%\providecommand{\version}{final}
%    \end{macrocode}

% Include the main document:
%    \begin{macrocode}
\input{childdoc.def}
\childdocof{cdocsamp}
%    \end{macrocode}

%\iffalse
%</samplechap1|samplechap2>
%\fi
%
%\iffalse
%<*samplechap1>
%\fi
% Some text for chapter 1:
%    \begin{macrocode}
\section{one}
some text in chapter one
%    \end{macrocode}

%\iffalse
%</samplechap1>
%\fi
% Some text for chapter 2:
%\iffalse
%<*samplechap2>
%\fi
%    \begin{macrocode}
\section{two}
more text in chapter two
%    \end{macrocode}

%\iffalse
%</samplechap2>
%\fi
%
% %%%%%%%%%%%%%%%%%%%%%%%%%%%%%%%%%%%%%%
% \paragraph{Part Include Files.}
%
% The include files are called |cdocspt3.tex| and |cdocspt4.tex|.
%
%\iffalse
%<*samplepart3|samplepart4>
%\fi

% Optional override for |\version| flag:
%    \begin{macrocode}
%%\providecommand{\version}{final}
%    \end{macrocode}

% Include the main document:
%    \begin{macrocode}
\input{childdoc.def}
\childdocby{cdocsamp}
%    \end{macrocode}

%\iffalse
%</samplepart3|samplepart4>
%\fi
%
%\iffalse
%<*samplepart3>
%\fi
% Some text for part 3:
%    \begin{macrocode}
some text in part three
%    \end{macrocode}

%\iffalse
%</samplepart3>
%\fi
% Some text for part 4:
%\iffalse
%<*samplepart4>
%\fi
%    \begin{macrocode}
more text in part four
%    \end{macrocode}

%\iffalse
%</samplepart4>
%\fi
%
% %%%%%%%%%%%%%%%%%%%%%%%%%%%%%%%%%%%%%%
% \paragraph{Forwarding for a Complete Draft.}
%
% The following forwarding file |cdocsdrf.tex|
% compiles the main document in draft mode:
%\iffalse
%<*sampledraft>
%\fi
%    \begin{macrocode}
\def\version{draft}
\input{childdoc.def}
\childdocforward{cdocsamp}
%    \end{macrocode}

%\iffalse
%</sampledraft>
%\fi
%
% %%%%%%%%%%%%%%%%%%%%%%%%%%%%%%%%%%%%%%
% \paragraph{Forwarding for Final Version of the Chapters.}
%
% The following forwarding files |cdocsfn1.tex| and |cdocsfn2.tex|
% (with identical content)
% compile the final versions of the child documents
% |cdocsch1.tex| and |cdocsch2.tex|, respectively:
%\iffalse
%<*samplefinal>
%\fi
%    \begin{macrocode}
\def\version{final}
\input{childdoc.def}
\childdocforwardprefix[cdocsamp]{cdocsfn}{cdocsch}
%    \end{macrocode}

%\iffalse
%</samplefinal>
%\fi
%
% %%%%%%%%%%%%%%%%%%%%%%%%%%%%%%%%%%%%%%
% \paragraph{Command Line Processing.}
%
% The following three command lines generate the output files
% |cdocscld|, |cdocscl1| and |cdocscl2|
% which should be identical to
% |cdocsdrf|, |cdocsch1| and |cdocsfn2|, respectively:
% \begin{center}
% \begin{tabular}{l}
% |latex -jobname cdocscld \|\\
% |  "\def\version{draft}\input{childdoc.def}\childdocforward{cdocsamp}"|\\
% |latex -jobname cdocscl1 \|\\
% |  "\input{childdoc.def}\childdocforward[cdocsamp]{cdocsch1}"|\\
% |latex -jobname cdocscl2 \|\\
% |  "\def\version{final}\input{childdoc.def}\childdocforward{cdocsch2}"|
% \end{tabular}
% \end{center}
% Note that the trailing backslash on each first line
% merely continues the input to the second line
% (for convenient cut ant paste).
% Furthermore, the command |latex| can be replaced by any
% of its alternative versions such as |pdflatex|.
%
% %%%%%%%%%%%%%%%%%%%%%%%%%%%%%%%%%%%%%%%%%%%%%%%%%%%%%%%%%%%%%%%%%%%%%%%%%%%%%%
% %%%%%%%%%%%%%%%%%%%%%%%%%%%%%%%%%%%%%%%%%%%%%%%%%%%%%%%%%%%%%%%%%%%%%%%%%%%%%%
% \section{Implementation}
%\iffalse
%<*package>
%\fi
%
% This section describes the definitions file |childdoc.def|.

% The definitions cannot be loaded using |\usepackage| or |\RequirePackage|
% which has a mechanism to prevent loading a style file more than once.
% When loading the definitions by means of |\input|
% multiple instances have to be prevented manually:
%\iffalse
%This code needs to be before the `\ProvidesFile' directive
%which is defined at the beginning of this file.
%Therefore it is also placed there and commented out here.
%</package>
%<*discard>
%\fi
%    \begin{macrocode}
\ifdefined\childdocmain\endinput\fi
%    \end{macrocode}
%\iffalse
%</discard>
%<*package>
%\fi
%
% \macro{\ifchilddoc}
% \macro{\ifchilddocmanual}
% The conditional |\ifchilddoc| tells whether a
% child (true) or main (false) document is being compiled.
% The conditional |\ifchilddocmanual| tells whether
% the |\includeonly| mechanism is used (false) or
% the selection of child files must be performed manually (true).
% The definitions initialise to false:
%    \begin{macrocode}
\newif\ifchilddoc
\newif\ifchilddocmanual
%    \end{macrocode}

% \macro{\childdocname}
% \macro{\childdocjob}
% The macro |\childdocname| stores the name of the main document
% to be compiled. The macro |\childdocjob| stores the name of
% the document on which the \LaTeX{} compiler was originally invoked.
% The content of |\jobname| cannot be compared
% to filenames specified in the source due to different catcodes.
% The following code rescans |\jobname|, stores the result
% in |\childdocname| and saves a copy in |\childdocjob|:
%    \begin{macrocode}
\edef\childdocname{\scantokens\expandafter{\jobname\noexpand}}
\let\childdocjob\childdocname
%    \end{macrocode}

% \macro{\childdocdisable}
% The macro |\childdocdisable| prevents the main file
% from being processed more than once.
% At this stage, the main document command |\childdocmain|
% is assumed to be called once again where it should do nothing.
% Any subsequent call to it should prevent
% a secondary processing of the main document
% It overwrites the forwarding commands
% |\childdocof| and |\childdocforward|
% with empty macros to prevent further inclusions of the main document:
%    \begin{macrocode}
\newcommand{\childdocdisable}
{
  \renewcommand{\childdocmain}[1]{\renewcommand{\childdocmain}[1]{\endinput}}
  \renewcommand{\childdocof}[1]{}
  \renewcommand{\childdocby}[2][]{}
  \renewcommand{\childdocforward}[2][]{}
  \renewcommand{\childdocdisable}{}
}
%    \end{macrocode}

% \macro{\childdocmain}
% The macro |\childdocmain| is to be called at the top of the main file
% with nothing or the main filename (without extension) as argument.
% First, it breaks loops.
% If the argument is not empty and does not match |\childdocname|
% (which is set by the first inclusion of |childdoc.def|),
% |\ifchilddoc| is set to true, |\includeonly| is applied to the child file
% and |\jobname| is set to the main file
% (for proper handling of |.aux| files):
%    \begin{macrocode}
\newcommand{\childdocmain}[1]
{
  \childdocdisable\childdocmain{}
  \if?#1?\else
    \begingroup
      \def\childdoctmp{#1}
      \ifx\childdoctmp\childdocname
        \def\childdoctmp{}
      \else
        \def\childdoctmp
        {
          \childdoctrue
          \includeonly{\childdocname}
          \def\childdocjob{#1}
          \def\jobname{#1}
        }
      \fi
      \expandafter
    \endgroup
    \childdoctmp
  \fi
}
%    \end{macrocode}

% \macro{\childdocof}
% The command |\childdocof| redirects
% compilation to the main file |#1|.
%    \begin{macrocode}
\newcommand{\childdocof}[1]
{
  \childdocdisable
  \childdoctrue
  \includeonly{\childdocname}
  \def\jobname{#1}
  \def\childdocjob{#1}
  \input{#1}
}
%    \end{macrocode}

% \macro{\childdocby}
% The command |\childdocby| ....
%    \begin{macrocode}
\newcommand{\childdocby}[2][]
{
  \childdocdisable
  \childdoctrue
  \childdocmanualtrue
  \if?#1?\else
    \def\jobname{#2}
  \fi
  \def\childdocjob{#2}
  \input{#2}
  \endinput
}
%    \end{macrocode}

% \macro{\childdocforward}
% The command |\childdocforward| redirects
% compilation to the main file or
% (if the optional argument is given) a child file.
% Parameters are set as if the main file
% or a child file starting with |\childdocof| was compiled.
% Then compilation is handed over to the main file:
%    \begin{macrocode}
\newcommand{\childdocforward}[2][]
{
  \begingroup
    \if?#1?
      \def\childdoctmp
      {
        \def\childdocname{#2}
        \def\childdocjob{#2}
        \def\jobname{#2}
        \input{#2}
        \endinput
      }
    \else
      \def\childdoctmp
      {
        \childdocdisable
        \def\childdocname{#2}
        \childdoctrue
        \includeonly{#2}
        \def\childdocjob{#1}
        \def\jobname{#1}
        \input{#1}
        \endinput
      }
    \fi
    \expandafter
  \endgroup
  \childdoctmp
}
%    \end{macrocode}

% \macro{\childdocforwardprefix}
% The command |\childdocforwardprefix| redirects
% compilation to the main or a child file by means of a pattern.
% The prefix |#1| in the current filename is replaced by |#2|
% and the suffix of the current filename is kept
% (it is assumed that the filename does not contain the substring `|~~~|'
% which is used as a delimiter).
% Compilation is handed over to the new file by |\childdocforward|:
%    \begin{macrocode}
\newcommand{\childdocforwardprefix}[3][]
{
  \begingroup
    \def\childdocextract #2##1~~~{\def\childdoctmp{\childdocforward[#1]{#3##1}}}
    \expandafter\childdocextract\childdocname~~~
    \expandafter
  \endgroup
  \childdoctmp
}
%    \end{macrocode}

% \macro{\childdoc}
% The deprecated macro |\childdoc| is a legacy version of |\childdocmain|:
%    \begin{macrocode}
\newcommand{\childdoc}{\childdocmain}
%    \end{macrocode}

% \macro{\childdocredirect}
% The deprecated macro |\childdocredirect| is a legacy version
% of |\childdocforward| and |\childdocforwardprefix|:
%    \begin{macrocode}
\newcommand{\childdocredirect}[2][]
{
  \begingroup
    \if?#1?
      \def\childdoctmp{\childdocforward{#2}}
    \else
      \def\childdoctmp{\childdocforwardprefix{#1}{#2}}
    \fi
    \expandafter
  \endgroup
  \childdoctmp
}
%    \end{macrocode}

%\iffalse
%</package>
%\fi
%
\endinput
|\\
|\childdocby{|\textit{main}|}|\\
\end{tabular}
\end{center}
%
The directive |\childdocby| is similar to |\childdocof|
described in \secref{sec:include},
but the subsequent selection of content must be done manually.
To that end, both |\ifchilddoc| and |\ifchilddocmanual|
will be true upon processing of a part,
and the name of the part is stored in |\childdocname|.
Note that |\jobname| will be set to the filename of the current part
so that each part receives an individual |.aux| file
that does not interfere with the |.aux| file(s) of the main document.
This behaviour can be altered by the alternative form
|\childdocby[*]{|\textit{main}|}| (with a non-empty optional argument)
which uses the |.aux| file of the main document
by setting |\jobname| to \textit{main}.

%%%%%%%%%%%%%%%%%%%%%%%%%%%%%%%%%%%%%%%%%%%%%%%%%%%%%%%%%%%%%%%%%%%%%%%%%%%%%%%%
\subsection{Driver Development}
\label{sec:driver}

The \textsf{childdoc} mechanism can also be use for the development
of definition files such as \LaTeX{} styles or classes.
This case differs from the above setup with multiple parts
included by |\include| in that no |\includeonly| should be invoked.
This can be achieved by starting the include file
(before |\ProvidesPackage|) with:
%
\begin{center}
\begin{tabular}{l}
|% \iffalse
%
% childdoc.dtx Copyright (C) 2017-2018 Niklas Beisert
%
% This work may be distributed and/or modified under the
% conditions of the LaTeX Project Public License, either version 1.3
% of this license or (at your option) any later version.
% The latest version of this license is in
%   http://www.latex-project.org/lppl.txt
% and version 1.3 or later is part of all distributions of LaTeX
% version 2005/12/01 or later.
%
% This work has the LPPL maintenance status `maintained'.
%
% The Current Maintainer of this work is Niklas Beisert.
%
% This work consists of the files childdoc.dtx and childdoc.ins
% and the derived files childdoc.def and cdocsamp.tex with
% cdocsch1.tex, cdocsch2.tex, cdocsdrf.tex, cdocsfn1.tex, cdocsfn2.tex.
%
%<package>\ifdefined\childdocmain\endinput\fi
%<package>\ProvidesFile{childdoc.def}[2018/12/30 v2.0 child document driver]
%<samplemain>\ProvidesFile{cdocsamp.tex}[2018/12/30 v2.0 sample for childdoc]
%<*driver>
%\ProvidesFile{childdoc.drv}[2018/12/30 v2.0 childdoc reference manual file]
\PassOptionsToClass{10pt,a4paper}{article}
\documentclass{ltxdoc}

\usepackage[margin=35mm]{geometry}
\usepackage{hyperref}
\usepackage{hyperxmp}
\usepackage[usenames]{color}

\hypersetup{colorlinks=true}
\hypersetup{pdfstartview=FitH}
\hypersetup{pdfpagemode=UseNone}
\hypersetup{pdfsource={}}
\hypersetup{pdflang={en-UK}}
\hypersetup{pdfcopyright={Copyright 2017-2018 Niklas Beisert.
  This work may be distributed and/or modified under the
  conditions of the LaTeX Project Public License, either version 1.3
  of this license or (at your option) any later version.}}
\hypersetup{pdflicenseurl={http://www.latex-project.org/lppl.txt}}
\hypersetup{pdfcontactaddress={ETH Zurich, ITP, HIT K,
  Wolfgang-Pauli-Strasse 27}}
\hypersetup{pdfcontactpostcode={8093}}
\hypersetup{pdfcontactcity={Zurich}}
\hypersetup{pdfcontactcountry={Switzerland}}
\hypersetup{pdfcontactemail={nbeisert@itp.phys.ethz.ch}}
\hypersetup{pdfcontacturl={http://people.phys.ethz.ch/\xmptilde nbeisert/}}

\newcommand{\secref}[1]{\hyperref[#1]{section \ref*{#1}}}

\parskip1ex
\parindent0pt
\let\olditemize\itemize
\def\itemize{\olditemize\parskip0pt}

\begin{document}

\title{The \textsf{childdoc} Package}
\hypersetup{pdftitle={The childdoc Package}}
\author{Niklas Beisert\\[2ex]
  Institut f\"ur Theoretische Physik\\
  Eidgen\"ossische Technische Hochschule Z\"urich\\
  Wolfgang-Pauli-Strasse 27, 8093 Z\"urich, Switzerland\\[1ex]
  \href{mailto:nbeisert@itp.phys.ethz.ch}
  {\texttt{nbeisert@itp.phys.ethz.ch}}}
\hypersetup{pdfauthor={Niklas Beisert}}
\hypersetup{pdfsubject={Manual for the LaTeX2e Package childdoc}}
\date{30 December 2018, \textsf{v2.0}}
\maketitle

\begin{abstract}\noindent
\textsf{childdoc} is a \LaTeXe{} package
that enables the direct compilation
of document sections included by |\include|
to individual files.
\end{abstract}

\begingroup
\parskip0ex
\tableofcontents
\endgroup

%%%%%%%%%%%%%%%%%%%%%%%%%%%%%%%%%%%%%%%%%%%%%%%%%%%%%%%%%%%%%%%%%%%%%%%%%%%%%%%%
%%%%%%%%%%%%%%%%%%%%%%%%%%%%%%%%%%%%%%%%%%%%%%%%%%%%%%%%%%%%%%%%%%%%%%%%%%%%%%%%
\section{Introduction}

\LaTeX{} provides a mechanism to structure a large document (such as a book)
into a main file and several child files (containing the chapters)
using the |\include| command.
This mechanism is beneficial for documents
which span hundreds of pages in order to
make the source file(s) more manageable.
Moreover, compilation can be restricted to
selected child files by means of the |\includeonly| command.
The latter feature can be used to reduce the compilation time while editing
(this was significantly more useful in the earlier days of \LaTeX{})
or to generate a smaller document which is easier to navigate.
Another application of |\includeonly| is to generate
documents consisting of selected parts of the complete document.

However, there are a few drawbacks of the plain |\include| mechanism:
\begin{itemize}
\item
The child files cannot be compiled on their own,
they can only be compiled via the main file.
A naive editing environment
(such as a text editor with an option
to have the current file processed by \LaTeX)
may require one to switch to the main file before compiling;
attempting to compile the child file produces errors.
\item
The main file must be modified (each time)
to adjust the |\includeonly| command
to the present needs. This easily leaves the main file in a messy state.
\item
The generated document will always carry the filename
of the main document. This is inconvenient if
several child files are to be compiled and
to be kept for distribution.
\end{itemize}

The present package provides a simple interface
to make child files individually compilable by \LaTeX{}.
Compiling a child file then has the same effect as compiling
the main file with an |\includeonly| command
to select the appropriate child.
Moreover the generated document will carry the name of the child
rather than the main file.
This resolves all three above issues.

This feature is meant to make the editing of books,
thesis documents and lecture notes somewhat more convenient.
However, the package can also be used efficiently for
composing a series of documents (such as exercise sheets)
which are typically distributed individually.
It then assists the author in generating the individual documents
(potentially in different versions)
as well as a document containing the collected series.
Another application is in developing style files
or other kinds of included material
where compilation of the style file could redirect
to a sample or test file.

%%%%%%%%%%%%%%%%%%%%%%%%%%%%%%%%%%%%%%%%%%%%%%%%%%%%%%%%%%%%%%%%%%%%%%%%%%%%%%%%
%%%%%%%%%%%%%%%%%%%%%%%%%%%%%%%%%%%%%%%%%%%%%%%%%%%%%%%%%%%%%%%%%%%%%%%%%%%%%%%%
\section{Usage}

First of all, the package \textsf{childdoc} is \emph{not} a standard
\LaTeXe{} |.sty| style file! Therefore it needs to be invoked in
a non-standard way.

%%%%%%%%%%%%%%%%%%%%%%%%%%%%%%%%%%%%%%%%%%%%%%%%%%%%%%%%%%%%%%%%%%%%%%%%%%%%%%%%
\subsection{Included Files}
\label{sec:include}

%%%%%%%%%%%%%%%%%%%%%%%%%%%%%%%%%%%%%%%%
\DescribeMacro{\childdocmain}
To use the package, add the commands
\begin{center}
\begin{tabular}{l}
|\input{childdoc.def}|\\
|\childdocmain{}|\\
\end{tabular}
\end{center}
at the very top of the main \LaTeX{} file,
in particular \emph{before} the |\documentclass| statement!
The argument of |\childdocmain| should be left empty
(but it must be present).

%%%%%%%%%%%%%%%%%%%%%%%%%%%%%%%%%%%%%%%%
\DescribeMacro{\childdocof}
Furthermore, add the commands
\begin{center}
\begin{tabular}{l}
|\input{childdoc.def}|\\
|\childdocof{|\textit{main}|}|\\
\end{tabular}
\end{center}
at the top of every child file \textit{child}
which is included by |\include{|\textit{child}|}|
from within the main file
(or at least for those files to be compiled individually).
The argument \textit{main} must be the filename of the main file.

There are a couple of
considerations in setting up the main and child documents:

%%%%%%%%%%%%%%%%%%%%%%%%%%%%%%%%%%%%%%%%
\paragraph{Restrictions.}

Please note the following restrictions:
\begin{itemize}
\item
|\childdocmain| must be called with one argument \textit{main}
to ensure compatibility with earlier version of the package.
It must either be empty (|\childdocmain{}|)
or precisely match the filename of the main file in which it is specified.
See \secref{sec:detection} for further information.
\item
The filename \textit{main} must be specified without the |.tex| extension.
\item
The filename \textit{main} is case sensitive
(even in case-insensitive file systems)
due to internal string comparison.
\item
The argument \textit{main} should be fully expanded, it cannot be a macro.
\item
Subdirectories and special characters should be avoided in filenames.
\item
The command |\childdocmain{|\textit{main}|}| must be followed by a whitespace.
It should not be followed immediately by another command
or by a comment mark `|%|'.
This is because the \TeX{} parser reads the token immediately following
the argument of |\childdocmain| and puts it
at the beginning of every child section;
however, a white\-space is ignored.
\end{itemize}

%%%%%%%%%%%%%%%%%%%%%%%%%%%%%%%%%%%%%%%%
\paragraph{Content of Main File.}

It is advisable to place all content in the child files included by |\include|.
Any output contained in the main file will appear in all child documents
unless suppressed manually;
it cannot be suppressed automatically by the |\includeonly| directive
and thus should normally be avoided.
A method to include some content in the main file
by means of conditional processing is described in \secref{sec:conditional}.

%%%%%%%%%%%%%%%%%%%%%%%%%%%%%%%%%%%%%%%%
\paragraph{Page Numbering.}

When only a part of the document is compiled,
the appropriate numbering of pages
(as well as other status parameters)
is determined from the |.aux| files.
The latter contain information from previous passes.
However this information needs to propagate through
all intermediate child documents.
Therefore the page numbering in child documents may well
be inconsistent until the complete document is compiled at least once.

A useful (if unconventional) way to always ensure a consistent
page numbering is to restart the numbering in each child document
and denote the pages by `\textit{child}|.|\textit{page}'
where \textit{child} represents the chapter/section number of the child file.
This can be achieved by the command
|\numberwithin{page}{|\textit{child}|}|
of the \textsf{amsmath} package
where \textit{child} can be |chapter| or |section|
depending on the chosen structuring.
Alternatively, one can modify the macro |\thepage| appropriately
and reset the counter |page| at the start of each child file.

%%%%%%%%%%%%%%%%%%%%%%%%%%%%%%%%%%%%%%%%%%%%%%%%%%%%%%%%%%%%%%%%%%%%%%%%%%%%%%%%
\subsection{Conditional Processing}
\label{sec:conditional}

The package provides a mechanism to compile different versions
of a document. To customise the versions further some conditional processing
can come in handy to distinguish which version is being compiled.
The package provides two macros to describe the compilation context:

%%%%%%%%%%%%%%%%%%%%%%%%%%%%%%%%%%%%%%%%
\DescribeMacro{\ifchilddoc}
The conditional |\ifchilddoc| distinguishes between the compilation of
child documents and the main document:
%
\begin{center}
|\ifchilddoc |\textit{child-code}| |[|\||else |\textit{main-code}]| \||fi|
\end{center}

%%%%%%%%%%%%%%%%%%%%%%%%%%%%%%%%%%%%%%%%
\DescribeMacro{\childdocname}
\DescribeMacro{\childdocjob}
The macro |\childdocname| contains the filename (without extension)
of the main or child file being processed.
Note that |\childdocjob| will always contain the name of the main file.

%%%%%%%%%%%%%%%%%%%%%%%%%%%%%%%%%%%%%%%%
\paragraph{Title Page.}

Conditional processing can be used to include a title or banner page
in the main document when proper precautions are taken.
Importantly, the code in the main file should ensure that the page counter
(as well as other status parameters which are stored in the |.aux| files)
takes the same value after the conditional processing.
Otherwise the page numbers may take divergent values
depending on which part is compiled.

For example, a title page could be declared by:
%
\begin{center}
\begin{tabular}{l}
|\ifchilddoc\||else|\\
|\addtocounter{page}{-1}|\\
\textit{code for title page}\\
|\newpage|\\
|\||fi|
\end{tabular}
\end{center}
%
A banner page for the child documents can be generated by:
%
\begin{center}
\begin{tabular}{l}
|\ifchilddoc|\\
|\addtocounter{page}{-1}|\\
\textit{code for banner page}\\
|\newpage|\\
|\||fi|
\end{tabular}
\end{center}
%
Here one could write a message such as:
\begin{center}
|This is the part \childdocname{} of \childdocjob{}.|
\end{center}

%%%%%%%%%%%%%%%%%%%%%%%%%%%%%%%%%%%%%%%%%%%%%%%%%%%%%%%%%%%%%%%%%%%%%%%%%%%%%%%%
\subsection{Flags}
\label{sec:flags}

The package makes it easy to generate different versions
of the main or child documents.
To this end compilation flags can be defined
and assigned different default values.
They will be particularly useful in conjunction
with the forwarding mechanism described in \secref{sec:forward}.

For example, it may be useful to have a flag |\version|
which can be set to |draft| or |final|.
The document source will contain some conditional code
depending on the value of |\version|.
Suppose further, the flag should default to |final| for the main file
and to |draft| for child files
which is a natural assignment for editing the document.
This is achieved by placing the following code
in the preamble of the main document
(below the |\childdocmain| directive):
%
\begin{center}
\begin{tabular}{l}
|\ifchilddoc|\\
|\providecommand{\version}{draft}|\\
|\||else|\\
|\providecommand{\version}{final}|\\
|\||fi|
\end{tabular}
\end{center}
%
The definition by |\providecommand| makes sure
that previous definitions are not overwritten.
Further statements |\providecommand{\version}{...}|
can thus be added before the above code to override it.

For the main file, one might add a line
(between |\childdocmain| and the above block)
%
\begin{center}
|%\ifchilddoc\||else\providecommand{\version}{draft}\||fi|
\end{center}
%
which can be uncommented to produce a draft version.
Likewise one can add a line to the very top of a child file
(above the |\childdocof{|\textit{main}|}| directive)
%
\begin{center}
|%\providecommand{\version}{final}|
\end{center}
%
which can be uncommented to produce the final version of this child document.

%%%%%%%%%%%%%%%%%%%%%%%%%%%%%%%%%%%%%%%%%%%%%%%%%%%%%%%%%%%%%%%%%%%%%%%%%%%%%%%%
\subsection{Forwarding}
\label{sec:forward}

Different versions of the main or child documents
using compilation flags as described in \secref{sec:flags}
can be (permanently) stored in different files
for convenient compilation, viewing and distribution.
To this end, the package defines a command
to pass on compilation to a different file:

%%%%%%%%%%%%%%%%%%%%%%%%%%%%%%%%%%%%%%%%
\DescribeMacro{\childdocforward}
The command |\childdocforward| redirects processing to
another source file:
%
\begin{center}
\begin{tabular}{l}
|\input{childdoc.def}|\\
|\childdocforward[|\textit{main}|]{|\textit{dest}|}|\\
\end{tabular}
\end{center}
%
The argument \textit{dest} is the destination file
(without extension).
It should be the main file or one of the child files.
Note that further \textsf{childdoc} directives
such as |\childdocof| and |\childdocforward|
in the indicated file will be processed in this form.
The optional argument \textit{main}
passes on directly to the main file \textit{main}
while pretending to compile the child \textit{dest}.
This form behaves as if \textit{dest}
issues |\childdocof{|\textit{main}|}| right away,
and no further \textsf{childdoc} directives will be processed.

%%%%%%%%%%%%%%%%%%%%%%%%%%%%%%%%%%%%%%%%
\DescribeMacro{\...prefix}
In the alternative form |\childdocforwardprefix|,
%
\begin{center}
\begin{tabular}{l}
|\input{childdoc.def}|\\
|\childdocforwardprefix[|\textit{main}|]{|\textit{prefix}|}{|\textit{dest}|}|
\end{tabular}
\end{center}
%
the destination file is determined by a pattern
depending on the current file:
To make this work, the current file must be called
`{\textit{prefix}\hspace{0.2em}\textit{suffix}}'
with \textit{prefix} matching precisely the argument.
Processing is then passed on to the file
`{\textit{dest}\hspace{0.2em}\textit{suffix}}'.
Surely, the same effect is achieved by
directly specifying the
argument `{\textit{dest}\hspace{0.2em}\textit{suffix}}'
in the first form.
However, that requires to set up a different file
for each child. With the alternative form of the command
all these files can have exactly the same content
which simplifies setting them up and maintaining them.

For example, the following file |draft.tex|
with a compilation flag |\version| as described in \secref{sec:flags}
compiles the main document as a draft:
%
\begin{center}
\begin{tabular}{l}
|\def\version{draft}|\\
|\input{childdoc.def}|\\
|\childdocforward{|\textit{main}|}|
\end{tabular}
\end{center}
%
Likewise, the following files |final|\textit{nn}|.tex|
compile the final version of the child document
|child|\textit{nn}|.tex|:
%
\begin{center}
\begin{tabular}{l}
|\def\version{final}|\\
|\input{childdoc.def}|\\
|\childdocforwardprefix{final}{child}|
\end{tabular}
\end{center}
%

Note that when several versions of a main file and/or of each child file
are to be generated, it may be convenient to set up a |Makefile| or
shell script to automatise the process.

%%%%%%%%%%%%%%%%%%%%%%%%%%%%%%%%%%%%%%%%%%%%%%%%%%%%%%%%%%%%%%%%%%%%%%%%%%%%%%%%
\subsection{Command Line Processing}
\label{sec:commandline}

The effect of redirection files can also be achieved by invoking
the \LaTeX{} compiler with a more elaborate command line.
Most conveniently this should be done as part
of a shell script or a |Makefile|.

When using \textsf{childdoc} in the main file, the following
command lines effectively perform a redirection
(note that depending on the shell being used,
backslashes may have to be doubled: `|\|' $\to$ `|\\|'):
%
\begin{center}
|... -jobname "|\textit{target}|" |\\|"|[\textit{flags}]%
|\input{childdoc.def}\childdocforward[|\textit{main}|]{|\textit{dest}|}"|
\end{center}
%
Here \textit{target} is the name of the output file,
\textit{main} is the name of the main file
and \textit{dest} is the name of the main or child file to be processed
(all filenames without extensions).
The optional argument \textit{main} can be omitted
if \textit{main} matches \textit{dest}.
Optionally, compilation \textit{flags} can be defined via |\def| commands.
This command line makes the \TeX{} engine believe
it is compiling the file \textit{target}
whose content is specified as the latter parameter.
The provided code then forwards the processing to
\textit{main} or \textit{dest} as described in \secref{sec:forward}.

%%%%%%%%%%%%%%%%%%%%%%%%%%%%%%%%%%%%%%%%%%%%%%%%%%%%%%%%%%%%%%%%%%%%%%%%%%%%%%%%
\subsection{Include by Input}
\label{sec:input}

Including child documents by |\include| has some restrictions by design.
Most notably, the content of a child document always occupies
its own set of pages; pages cannot be shared between child documents.
Usually, this behaviour makes perfect sense
because each child document contain an essential part of the document.
However, in some situations it may be desirable to compose
a document from a collection of parts
without having mandatory page breaks between then.
For this case, the package
provides a mechanism to include parts
by |\input| which can also be processed individually.
However, by construction this mechanism
requires manual handling of the content to be output.

%%%%%%%%%%%%%%%%%%%%%%%%%%%%%%%%%%%%%%%%
\DescribeMacro{\ifchilddocmanual}
The main file should be prepared as usual, see \secref{sec:include}.
However, the document body must make a distinction
between processing of an individual part and of the main document, e.g.:
%
\begin{center}
\begin{tabular}{l}
|\ifchilddocmanual|\\
|\input{\childdocname}|\\
|\||else|\\
\textit{document body with }|\input{|\textit{part}|}|\\
|\||fi|
\end{tabular}
\end{center}
%
The conditional |\ifchilddocmanual| is true whenever
a part to be included by |\input| is being compiled,
and the name of the part is stored in |\childdocname|.

%%%%%%%%%%%%%%%%%%%%%%%%%%%%%%%%%%%%%%%%
\DescribeMacro{\childdocby}
Each part to be included by |\input| should start with:
%
\begin{center}
\begin{tabular}{l}
|\input{childdoc.def}|\\
|\childdocby{|\textit{main}|}|\\
\end{tabular}
\end{center}
%
The directive |\childdocby| is similar to |\childdocof|
described in \secref{sec:include},
but the subsequent selection of content must be done manually.
To that end, both |\ifchilddoc| and |\ifchilddocmanual|
will be true upon processing of a part,
and the name of the part is stored in |\childdocname|.
Note that |\jobname| will be set to the filename of the current part
so that each part receives an individual |.aux| file
that does not interfere with the |.aux| file(s) of the main document.
This behaviour can be altered by the alternative form
|\childdocby[*]{|\textit{main}|}| (with a non-empty optional argument)
which uses the |.aux| file of the main document
by setting |\jobname| to \textit{main}.

%%%%%%%%%%%%%%%%%%%%%%%%%%%%%%%%%%%%%%%%%%%%%%%%%%%%%%%%%%%%%%%%%%%%%%%%%%%%%%%%
\subsection{Driver Development}
\label{sec:driver}

The \textsf{childdoc} mechanism can also be use for the development
of definition files such as \LaTeX{} styles or classes.
This case differs from the above setup with multiple parts
included by |\include| in that no |\includeonly| should be invoked.
This can be achieved by starting the include file
(before |\ProvidesPackage|) with:
%
\begin{center}
\begin{tabular}{l}
|\input{childdoc.def}|\\
|\childdocforward{|\textit{main}|}|\\
\end{tabular}
\end{center}
%
or alternatively with:
%
\begin{center}
\begin{tabular}{l}
|\input{childdoc.def}|\\
|\childdocby{|\textit{main}|}|\\
\end{tabular}
\end{center}
%
Both forms have slightly different effects as described above.
The main file is prepared as usual, see \secref{sec:include}.

%%%%%%%%%%%%%%%%%%%%%%%%%%%%%%%%%%%%%%%%%%%%%%%%%%%%%%%%%%%%%%%%%%%%%%%%%%%%%%%%
\subsection{Legacy Detection}
\label{sec:detection}

The directive |\childdocmain| in the main file can detect
whether the complete document or merely a child is to be compiled
even without using the directive |\childdocof|.
This method is deprecated because it is less robust
and there is no compelling reason to use it;
it is merely provided for backward compatibility
and it may be removed in future versions.

If the detection mechanism is to be used,
it is mandatory to correctly specify
the filename of the main file as the argument of |\childdocmain|:
%
\begin{center}
\begin{tabular}{l}
|\input{childdoc.def}|\\
|\childdocmain{|\textit{main}|}|\\
\end{tabular}
\end{center}
%
If |\jobname| does not match the argument \textit{main} of |\childdocmain|,
it is assumed that |\jobname| points to the child file to be compiled.
When using |\childdocmain| with the main file specified as argument,
it suffices to start a child file
with just |\input{|\textit{main}|}|
without loading of the package and using |\childdocof|.
If instead all processing is done
with the appropriate \textsf{childdoc} directives,
the argument of \textit{main} of |\childdocmain| can be empty.

An alternative version of the command line processing described
in \secref{sec:commandline} using the detection mechanism reads:
%
\begin{center}
|... -jobname "|\textit{target}|" "|[\textit{flags}]%
[|\def\jobname{|\textit{dest}|}|]|\input{|\textit{main}|}"|
\end{center}

%%%%%%%%%%%%%%%%%%%%%%%%%%%%%%%%%%%%%%%%%%%%%%%%%%%%%%%%%%%%%%%%%%%%%%%%%%%%%%%%
\subsection{Manual Code}
\label{sec:manual}

In case one cannot be certain whether the definitions file |childdoc.def|
is installed on the target \TeX{} distribution
and one prefers not to ship it,
it is conceivable to paste a few relevant commands into the sources.

To that end, drop all statements |\input{childdoc.def}|
and perform the replacements as outlined below.
Instead of |\childdocmain{|\textit{main}|}| add the following code
to the top of the main file:
%
\begin{center}
\begin{tabular}{l}
|\||ifdefined\childdocname\endinput\||fi\newif\ifchilddoc|\\
|\edef\childdocname{\scantokens\expandafter{\jobname\noexpand}}|\\
|\def\childdocmain{|\textit{main}|}\||ifx\childdocmain\childdocname\||else|\\
|\childdoctrue\includeonly{\childdocname}\let\jobname\childdocmain\||fi|\\
\end{tabular}
\end{center}
%
Instead of |\childdocof{|\textit{main}|}| just include the main file
at the top of each child file:
%
\begin{center}
|\input{|\textit{main}|}|
\end{center}
%
A simple redirection |\childdocforward{|\textit{dest}|}| is achieved by:
%
\begin{center}
|\def\jobname{|\textit{dest}|}\input{\jobname}|
\end{center}
%
The redirection with prefix
|\childdocforwardprefix[|\textit{prefix}|]{|\textit{dest}|}|
is accomplished by:
%
\begin{center}
\begin{tabular}{l}
|{\edef\jobname{\scantokens\expandafter{\jobname\noexpand}}|\\
|\def\redirectjob |\textit{prefix}|#1~~~{\gdef\jobname{|\textit{dest}|#1}}|\\
|\expandafter\redirectjob\jobname~~~}\input{\jobname}|
\end{tabular}
\end{center}

In an alternative approach,
child documents can be compiled by a specific command line
without additional code or specific definitions:
%
\begin{center}
|... -jobname "|\textit{target}|" "|[\textit{flags}]%
|\includeonly{|\textit{dest}|}\input{|\textit{main}|}"|
\end{center}
%

%%%%%%%%%%%%%%%%%%%%%%%%%%%%%%%%%%%%%%%%%%%%%%%%%%%%%%%%%%%%%%%%%%%%%%%%%%%%%%%%
%%%%%%%%%%%%%%%%%%%%%%%%%%%%%%%%%%%%%%%%%%%%%%%%%%%%%%%%%%%%%%%%%%%%%%%%%%%%%%%%
\section{Information}

%%%%%%%%%%%%%%%%%%%%%%%%%%%%%%%%%%%%%%%%%%%%%%%%%%%%%%%%%%%%%%%%%%%%%%%%%%%%%%%%
\subsection{Copyright}

Copyright \copyright{} 2017--2018 Niklas Beisert

This work may be distributed and/or modified under the
conditions of the \LaTeX{} Project Public License, either version 1.3
of this license or (at your option) any later version.
The latest version of this license is in
  \url{http://www.latex-project.org/lppl.txt}
and version 1.3 or later is part of all distributions of \LaTeX{}
version 2005/12/01 or later.

This work has the LPPL maintenance status `maintained'.

The Current Maintainer of this work is Niklas Beisert.

This work consists of the files |README.txt|, |childdoc.ins| and |childdoc.dtx|
as well as the derived files |childdoc.def|, |cdocsamp.tex|
with |cdocsch1.tex|, |cdocsch2.tex|, |cdocspt3.tex|, |cdocspt4.tex|,
|cdocsdrf.tex|, |cdocsfn1.tex|, |cdocsfn2.tex|
as well as |childdoc.pdf|.

%%%%%%%%%%%%%%%%%%%%%%%%%%%%%%%%%%%%%%%%%%%%%%%%%%%%%%%%%%%%%%%%%%%%%%%%%%%%%%%%
\subsection{Files and Installation}

The package consists of the files:
%
\begin{center}
\begin{tabular}{ll}
    |README.txt|   & readme file \\
    |childdoc.ins| & installation file \\
    |childdoc.dtx| & source file \\
    |childdoc.def| & definition file \\
    |cdocsamp.tex| & sample main file \\
    |cdocsch1.tex| & sample include file \\
    |cdocsch2.tex| & sample include file \\
    |cdocspt3.tex| & sample part file \\
    |cdocspt4.tex| & sample part file \\
    |cdocsdrf.tex| & sample redirection file \\
    |cdocsfn1.tex| & sample redirection file \\
    |cdocsfn2.tex| & sample redirection file \\
    |childdoc.pdf| & manual
\end{tabular}
\end{center}
%
The distribution consists of the files
|README.txt|, |childdoc.ins| and |childdoc.dtx|.
%
\begin{itemize}
\item
Run (pdf)\LaTeX{} on |childdoc.dtx|
to compile the manual |childdoc.pdf| (this file).
\item
Run \LaTeX{} on |childdoc.ins| to create the definitions file |childdoc.def|
and the sample |cdocsamp.tex| with include files
|cdocsch1.tex|, |cdocsch2.tex|, |cdocspt3.tex|, |cdocspt4.tex|,
|cdocsdrf.tex|, |cdocsfn1.tex|, |cdocsfn2.tex|.
Then copy the file |childdoc.def| to an appropriate directory of your \LaTeX{}
distribution, e.g.\ \textit{texmf-root}|/tex/latex/childdoc|.
\end{itemize}

%%%%%%%%%%%%%%%%%%%%%%%%%%%%%%%%%%%%%%%%%%%%%%%%%%%%%%%%%%%%%%%%%%%%%%%%%%%%%%%%
\subsection{Related CTAN Packages}

There are several other packages which offer a similar functionality:
%
\begin{itemize}
\item
The packages
\href{http://ctan.org/pkg/docmute}{\textsf{docmute}},
\href{http://ctan.org/pkg/includex}{\textsf{includex}} and
\href{http://ctan.org/pkg/standalone}{\textsf{standalone}}
provide commands to include only the document body of
a child file thus allowing both files to be compiled individually.
\item
The packages \href{http://ctan.org/pkg/subdocs}{\textsf{subdocs}}
and \href{http://ctan.org/pkg/subfiles}{\textsf{subfiles}}
provide structures in which the main and child documents can be
encapsulated and allowing them to be compiled individually.
The inclusion mechanism is different from the conventional |\include|.
\item
The package \href{http://ctan.org/pkg/combine}{\textsf{combine}}
is an elaborate solution to combine several documents into one.
\end{itemize}
%
See also the CTAN topic \href{http://ctan.org/topic/subdocs}{\textsf{subdocs}}
for further related packages.
The present package differs from the above solutions in that
a document structure constructed with the conventional |\include| mechanism
just needs two extra commands at the top of every file
such that all constituent files can be compiled individually.

%%%%%%%%%%%%%%%%%%%%%%%%%%%%%%%%%%%%%%%%%%%%%%%%%%%%%%%%%%%%%%%%%%%%%%%%%%%%%%%%
%\subsection{Feature Suggestions}
%
%The following is a list of features which may be useful for future
%versions of this package:
%%
%\begin{itemize}
%\item
%\ldots
%\end{itemize}

%%%%%%%%%%%%%%%%%%%%%%%%%%%%%%%%%%%%%%%%%%%%%%%%%%%%%%%%%%%%%%%%%%%%%%%%%%%%%%%%
\subsection{Revision History}

%%%%%%%%%%%%%%%%%%%%%%%%%%%%%%%%%%%%%%%%
\paragraph{v2.0:} 2018/12/30

\begin{itemize}
\item
immediate forward processing
\item
added |\childdocby| mechanism
\item
manual restructured
\end{itemize}

%%%%%%%%%%%%%%%%%%%%%%%%%%%%%%%%%%%%%%%%
\paragraph{v1.6:} 2018/01/17

\begin{itemize}
\item
application for development of include files
\item
corrections to manual
\end{itemize}

%%%%%%%%%%%%%%%%%%%%%%%%%%%%%%%%%%%%%%%%
\paragraph{v1.5:} 2017/05/21

\begin{itemize}
\item
more complete structuring introduced
\item
|\childdocof| introduced
\item
|\childdoc| renamed to |\childdocmain|
\item
|\childredirect| renamed to |\childdocforward| and |\childdocforwardprefix|
and functionality expanded
\end{itemize}

%%%%%%%%%%%%%%%%%%%%%%%%%%%%%%%%%%%%%%%%
\paragraph{v1.0:} 2017/04/27

\begin{itemize}
\item
manual and install package
\item
first version published on CTAN
\end{itemize}

%%%%%%%%%%%%%%%%%%%%%%%%%%%%%%%%%%%%%%%%
\paragraph{v0.6:} 2017/04/26

\begin{itemize}
\item
redirection mechanism added
\end{itemize}

%%%%%%%%%%%%%%%%%%%%%%%%%%%%%%%%%%%%%%%%
\paragraph{v0.5:} 2017/04/26

\begin{itemize}
\item
functionality in definition file
\end{itemize}


%%%%%%%%%%%%%%%%%%%%%%%%%%%%%%%%%%%%%%%%%%%%%%%%%%%%%%%%%%%%%%%%%%%%%%%%%%%%%%%%
%%%%%%%%%%%%%%%%%%%%%%%%%%%%%%%%%%%%%%%%%%%%%%%%%%%%%%%%%%%%%%%%%%%%%%%%%%%%%%%%
%%%%%%%%%%%%%%%%%%%%%%%%%%%%%%%%%%%%%%%%%%%%%%%%%%%%%%%%%%%%%%%%%%%%%%%%%%%%%%%%
\appendix

\settowidth\MacroIndent{\rmfamily\scriptsize 000\ }

 \DocInput{childdoc.dtx}

\end{document}
%</driver>
% \fi
%
% %%%%%%%%%%%%%%%%%%%%%%%%%%%%%%%%%%%%%%%%%%%%%%%%%%%%%%%%%%%%%%%%%%%%%%%%%%%%%%
% %%%%%%%%%%%%%%%%%%%%%%%%%%%%%%%%%%%%%%%%%%%%%%%%%%%%%%%%%%%%%%%%%%%%%%%%%%%%%%
% \section{Sample}
%\iffalse
%<*samplemain>
%\fi
%
% The following presents a sample document
% with two chapters, two parts, a title page,
% a compile flag as well as three forwarding files to set the flag.
% It consists of eight |.tex| files:
% \begin{center}
% \begin{tabular}{ll}
% |cdocsamp.tex|&main file\\
% |cdocsch1.tex|&include file for chapter 1\\
% |cdocsch2.tex|&include file for chapter 2\\
% |cdocspt3.tex|&include file for part 3\\
% |cdocspt4.tex|&include file for part 4\\
% |cdocsdrf.tex|&forwarding file for main file in draft mode\\
% |cdocsfi1.tex|&forwarding file for final version of chapter 1\\
% |cdocsfi2.tex|&forwarding file for final version of chapter 2\\
% \end{tabular}
% \end{center}
% Each of the eight files can be compiled directly by the \LaTeX{} compiler.
%
% %%%%%%%%%%%%%%%%%%%%%%%%%%%%%%%%%%%%%%
% \paragraph{Main File.}
%
% The main file is called |cdocsamp.tex|.
%
% Load the \textsf{childdoc} definitions and
% declare the filename for the main document:
%    \begin{macrocode}
\input{childdoc.def}
\childdocmain{}
%    \end{macrocode}

% Optional override for |\version| flag:
%    \begin{macrocode}
%%\ifchilddoc\else\providecommand{\version}{draft}\fi
%    \end{macrocode}

% Define the default values for the |\version| flag
% (|final| for the main file and |draft| for childs):
%    \begin{macrocode}
\ifchilddoc
\providecommand{\version}{draft}
\else
\providecommand{\version}{final}
\fi
%    \end{macrocode}

% Load the standard document class:
%    \begin{macrocode}
\documentclass[12pt]{article}
%    \end{macrocode}

% Start the document body:
%    \begin{macrocode}
\begin{document}
%    \end{macrocode}

% Declare a title page.
% Print title, part of document being processed and version flag:
%    \begin{macrocode}
\addtocounter{page}{-1}
\begin{center}
{\LARGE\bfseries{}childdoc example\par}
\vspace{1cm}
\ifchilddoc
\ifchilddocmanual part\else chapter\fi:
`\childdocname' of `\childdocjob'\par
\else
main document: `\childdocjob'\par
\fi
version: \version\par
\end{center}
\newpage
%    \end{macrocode}

% Manually include selected file,
% otherwise process as usual:
%    \begin{macrocode}
\ifchilddocmanual
\section*{part `\childdocname'}
\input{\childdocname}
\else
%    \end{macrocode}

% Include the two chapters:
%    \begin{macrocode}
\include{cdocsch1}
\include{cdocsch2}
%    \end{macrocode}

% Include the two parts unless only chapters should be displayed:
%    \begin{macrocode}
\ifchilddoc\else
\section{part three}
\input{cdocspt3}
\section{part four}
\input{cdocspt4}
\fi
%    \end{macrocode}

% Process as usual until here:
%    \begin{macrocode}
\fi
%    \end{macrocode}

% End of document body:
%    \begin{macrocode}
\end{document}
%    \end{macrocode}
%\iffalse
%</samplemain>
%\fi
%
% %%%%%%%%%%%%%%%%%%%%%%%%%%%%%%%%%%%%%%
% \paragraph{Chapter Include Files.}
%
% The include files are called |cdocsch1.tex| and |cdocsch2.tex|.
%
%\iffalse
%<*samplechap1|samplechap2>
%\fi

% Optional override for |\version| flag:
%    \begin{macrocode}
%%\providecommand{\version}{final}
%    \end{macrocode}

% Include the main document:
%    \begin{macrocode}
\input{childdoc.def}
\childdocof{cdocsamp}
%    \end{macrocode}

%\iffalse
%</samplechap1|samplechap2>
%\fi
%
%\iffalse
%<*samplechap1>
%\fi
% Some text for chapter 1:
%    \begin{macrocode}
\section{one}
some text in chapter one
%    \end{macrocode}

%\iffalse
%</samplechap1>
%\fi
% Some text for chapter 2:
%\iffalse
%<*samplechap2>
%\fi
%    \begin{macrocode}
\section{two}
more text in chapter two
%    \end{macrocode}

%\iffalse
%</samplechap2>
%\fi
%
% %%%%%%%%%%%%%%%%%%%%%%%%%%%%%%%%%%%%%%
% \paragraph{Part Include Files.}
%
% The include files are called |cdocspt3.tex| and |cdocspt4.tex|.
%
%\iffalse
%<*samplepart3|samplepart4>
%\fi

% Optional override for |\version| flag:
%    \begin{macrocode}
%%\providecommand{\version}{final}
%    \end{macrocode}

% Include the main document:
%    \begin{macrocode}
\input{childdoc.def}
\childdocby{cdocsamp}
%    \end{macrocode}

%\iffalse
%</samplepart3|samplepart4>
%\fi
%
%\iffalse
%<*samplepart3>
%\fi
% Some text for part 3:
%    \begin{macrocode}
some text in part three
%    \end{macrocode}

%\iffalse
%</samplepart3>
%\fi
% Some text for part 4:
%\iffalse
%<*samplepart4>
%\fi
%    \begin{macrocode}
more text in part four
%    \end{macrocode}

%\iffalse
%</samplepart4>
%\fi
%
% %%%%%%%%%%%%%%%%%%%%%%%%%%%%%%%%%%%%%%
% \paragraph{Forwarding for a Complete Draft.}
%
% The following forwarding file |cdocsdrf.tex|
% compiles the main document in draft mode:
%\iffalse
%<*sampledraft>
%\fi
%    \begin{macrocode}
\def\version{draft}
\input{childdoc.def}
\childdocforward{cdocsamp}
%    \end{macrocode}

%\iffalse
%</sampledraft>
%\fi
%
% %%%%%%%%%%%%%%%%%%%%%%%%%%%%%%%%%%%%%%
% \paragraph{Forwarding for Final Version of the Chapters.}
%
% The following forwarding files |cdocsfn1.tex| and |cdocsfn2.tex|
% (with identical content)
% compile the final versions of the child documents
% |cdocsch1.tex| and |cdocsch2.tex|, respectively:
%\iffalse
%<*samplefinal>
%\fi
%    \begin{macrocode}
\def\version{final}
\input{childdoc.def}
\childdocforwardprefix[cdocsamp]{cdocsfn}{cdocsch}
%    \end{macrocode}

%\iffalse
%</samplefinal>
%\fi
%
% %%%%%%%%%%%%%%%%%%%%%%%%%%%%%%%%%%%%%%
% \paragraph{Command Line Processing.}
%
% The following three command lines generate the output files
% |cdocscld|, |cdocscl1| and |cdocscl2|
% which should be identical to
% |cdocsdrf|, |cdocsch1| and |cdocsfn2|, respectively:
% \begin{center}
% \begin{tabular}{l}
% |latex -jobname cdocscld \|\\
% |  "\def\version{draft}\input{childdoc.def}\childdocforward{cdocsamp}"|\\
% |latex -jobname cdocscl1 \|\\
% |  "\input{childdoc.def}\childdocforward[cdocsamp]{cdocsch1}"|\\
% |latex -jobname cdocscl2 \|\\
% |  "\def\version{final}\input{childdoc.def}\childdocforward{cdocsch2}"|
% \end{tabular}
% \end{center}
% Note that the trailing backslash on each first line
% merely continues the input to the second line
% (for convenient cut ant paste).
% Furthermore, the command |latex| can be replaced by any
% of its alternative versions such as |pdflatex|.
%
% %%%%%%%%%%%%%%%%%%%%%%%%%%%%%%%%%%%%%%%%%%%%%%%%%%%%%%%%%%%%%%%%%%%%%%%%%%%%%%
% %%%%%%%%%%%%%%%%%%%%%%%%%%%%%%%%%%%%%%%%%%%%%%%%%%%%%%%%%%%%%%%%%%%%%%%%%%%%%%
% \section{Implementation}
%\iffalse
%<*package>
%\fi
%
% This section describes the definitions file |childdoc.def|.

% The definitions cannot be loaded using |\usepackage| or |\RequirePackage|
% which has a mechanism to prevent loading a style file more than once.
% When loading the definitions by means of |\input|
% multiple instances have to be prevented manually:
%\iffalse
%This code needs to be before the `\ProvidesFile' directive
%which is defined at the beginning of this file.
%Therefore it is also placed there and commented out here.
%</package>
%<*discard>
%\fi
%    \begin{macrocode}
\ifdefined\childdocmain\endinput\fi
%    \end{macrocode}
%\iffalse
%</discard>
%<*package>
%\fi
%
% \macro{\ifchilddoc}
% \macro{\ifchilddocmanual}
% The conditional |\ifchilddoc| tells whether a
% child (true) or main (false) document is being compiled.
% The conditional |\ifchilddocmanual| tells whether
% the |\includeonly| mechanism is used (false) or
% the selection of child files must be performed manually (true).
% The definitions initialise to false:
%    \begin{macrocode}
\newif\ifchilddoc
\newif\ifchilddocmanual
%    \end{macrocode}

% \macro{\childdocname}
% \macro{\childdocjob}
% The macro |\childdocname| stores the name of the main document
% to be compiled. The macro |\childdocjob| stores the name of
% the document on which the \LaTeX{} compiler was originally invoked.
% The content of |\jobname| cannot be compared
% to filenames specified in the source due to different catcodes.
% The following code rescans |\jobname|, stores the result
% in |\childdocname| and saves a copy in |\childdocjob|:
%    \begin{macrocode}
\edef\childdocname{\scantokens\expandafter{\jobname\noexpand}}
\let\childdocjob\childdocname
%    \end{macrocode}

% \macro{\childdocdisable}
% The macro |\childdocdisable| prevents the main file
% from being processed more than once.
% At this stage, the main document command |\childdocmain|
% is assumed to be called once again where it should do nothing.
% Any subsequent call to it should prevent
% a secondary processing of the main document
% It overwrites the forwarding commands
% |\childdocof| and |\childdocforward|
% with empty macros to prevent further inclusions of the main document:
%    \begin{macrocode}
\newcommand{\childdocdisable}
{
  \renewcommand{\childdocmain}[1]{\renewcommand{\childdocmain}[1]{\endinput}}
  \renewcommand{\childdocof}[1]{}
  \renewcommand{\childdocby}[2][]{}
  \renewcommand{\childdocforward}[2][]{}
  \renewcommand{\childdocdisable}{}
}
%    \end{macrocode}

% \macro{\childdocmain}
% The macro |\childdocmain| is to be called at the top of the main file
% with nothing or the main filename (without extension) as argument.
% First, it breaks loops.
% If the argument is not empty and does not match |\childdocname|
% (which is set by the first inclusion of |childdoc.def|),
% |\ifchilddoc| is set to true, |\includeonly| is applied to the child file
% and |\jobname| is set to the main file
% (for proper handling of |.aux| files):
%    \begin{macrocode}
\newcommand{\childdocmain}[1]
{
  \childdocdisable\childdocmain{}
  \if?#1?\else
    \begingroup
      \def\childdoctmp{#1}
      \ifx\childdoctmp\childdocname
        \def\childdoctmp{}
      \else
        \def\childdoctmp
        {
          \childdoctrue
          \includeonly{\childdocname}
          \def\childdocjob{#1}
          \def\jobname{#1}
        }
      \fi
      \expandafter
    \endgroup
    \childdoctmp
  \fi
}
%    \end{macrocode}

% \macro{\childdocof}
% The command |\childdocof| redirects
% compilation to the main file |#1|.
%    \begin{macrocode}
\newcommand{\childdocof}[1]
{
  \childdocdisable
  \childdoctrue
  \includeonly{\childdocname}
  \def\jobname{#1}
  \def\childdocjob{#1}
  \input{#1}
}
%    \end{macrocode}

% \macro{\childdocby}
% The command |\childdocby| ....
%    \begin{macrocode}
\newcommand{\childdocby}[2][]
{
  \childdocdisable
  \childdoctrue
  \childdocmanualtrue
  \if?#1?\else
    \def\jobname{#2}
  \fi
  \def\childdocjob{#2}
  \input{#2}
  \endinput
}
%    \end{macrocode}

% \macro{\childdocforward}
% The command |\childdocforward| redirects
% compilation to the main file or
% (if the optional argument is given) a child file.
% Parameters are set as if the main file
% or a child file starting with |\childdocof| was compiled.
% Then compilation is handed over to the main file:
%    \begin{macrocode}
\newcommand{\childdocforward}[2][]
{
  \begingroup
    \if?#1?
      \def\childdoctmp
      {
        \def\childdocname{#2}
        \def\childdocjob{#2}
        \def\jobname{#2}
        \input{#2}
        \endinput
      }
    \else
      \def\childdoctmp
      {
        \childdocdisable
        \def\childdocname{#2}
        \childdoctrue
        \includeonly{#2}
        \def\childdocjob{#1}
        \def\jobname{#1}
        \input{#1}
        \endinput
      }
    \fi
    \expandafter
  \endgroup
  \childdoctmp
}
%    \end{macrocode}

% \macro{\childdocforwardprefix}
% The command |\childdocforwardprefix| redirects
% compilation to the main or a child file by means of a pattern.
% The prefix |#1| in the current filename is replaced by |#2|
% and the suffix of the current filename is kept
% (it is assumed that the filename does not contain the substring `|~~~|'
% which is used as a delimiter).
% Compilation is handed over to the new file by |\childdocforward|:
%    \begin{macrocode}
\newcommand{\childdocforwardprefix}[3][]
{
  \begingroup
    \def\childdocextract #2##1~~~{\def\childdoctmp{\childdocforward[#1]{#3##1}}}
    \expandafter\childdocextract\childdocname~~~
    \expandafter
  \endgroup
  \childdoctmp
}
%    \end{macrocode}

% \macro{\childdoc}
% The deprecated macro |\childdoc| is a legacy version of |\childdocmain|:
%    \begin{macrocode}
\newcommand{\childdoc}{\childdocmain}
%    \end{macrocode}

% \macro{\childdocredirect}
% The deprecated macro |\childdocredirect| is a legacy version
% of |\childdocforward| and |\childdocforwardprefix|:
%    \begin{macrocode}
\newcommand{\childdocredirect}[2][]
{
  \begingroup
    \if?#1?
      \def\childdoctmp{\childdocforward{#2}}
    \else
      \def\childdoctmp{\childdocforwardprefix{#1}{#2}}
    \fi
    \expandafter
  \endgroup
  \childdoctmp
}
%    \end{macrocode}

%\iffalse
%</package>
%\fi
%
\endinput
|\\
|\childdocforward{|\textit{main}|}|\\
\end{tabular}
\end{center}
%
or alternatively with:
%
\begin{center}
\begin{tabular}{l}
|% \iffalse
%
% childdoc.dtx Copyright (C) 2017-2018 Niklas Beisert
%
% This work may be distributed and/or modified under the
% conditions of the LaTeX Project Public License, either version 1.3
% of this license or (at your option) any later version.
% The latest version of this license is in
%   http://www.latex-project.org/lppl.txt
% and version 1.3 or later is part of all distributions of LaTeX
% version 2005/12/01 or later.
%
% This work has the LPPL maintenance status `maintained'.
%
% The Current Maintainer of this work is Niklas Beisert.
%
% This work consists of the files childdoc.dtx and childdoc.ins
% and the derived files childdoc.def and cdocsamp.tex with
% cdocsch1.tex, cdocsch2.tex, cdocsdrf.tex, cdocsfn1.tex, cdocsfn2.tex.
%
%<package>\ifdefined\childdocmain\endinput\fi
%<package>\ProvidesFile{childdoc.def}[2018/12/30 v2.0 child document driver]
%<samplemain>\ProvidesFile{cdocsamp.tex}[2018/12/30 v2.0 sample for childdoc]
%<*driver>
%\ProvidesFile{childdoc.drv}[2018/12/30 v2.0 childdoc reference manual file]
\PassOptionsToClass{10pt,a4paper}{article}
\documentclass{ltxdoc}

\usepackage[margin=35mm]{geometry}
\usepackage{hyperref}
\usepackage{hyperxmp}
\usepackage[usenames]{color}

\hypersetup{colorlinks=true}
\hypersetup{pdfstartview=FitH}
\hypersetup{pdfpagemode=UseNone}
\hypersetup{pdfsource={}}
\hypersetup{pdflang={en-UK}}
\hypersetup{pdfcopyright={Copyright 2017-2018 Niklas Beisert.
  This work may be distributed and/or modified under the
  conditions of the LaTeX Project Public License, either version 1.3
  of this license or (at your option) any later version.}}
\hypersetup{pdflicenseurl={http://www.latex-project.org/lppl.txt}}
\hypersetup{pdfcontactaddress={ETH Zurich, ITP, HIT K,
  Wolfgang-Pauli-Strasse 27}}
\hypersetup{pdfcontactpostcode={8093}}
\hypersetup{pdfcontactcity={Zurich}}
\hypersetup{pdfcontactcountry={Switzerland}}
\hypersetup{pdfcontactemail={nbeisert@itp.phys.ethz.ch}}
\hypersetup{pdfcontacturl={http://people.phys.ethz.ch/\xmptilde nbeisert/}}

\newcommand{\secref}[1]{\hyperref[#1]{section \ref*{#1}}}

\parskip1ex
\parindent0pt
\let\olditemize\itemize
\def\itemize{\olditemize\parskip0pt}

\begin{document}

\title{The \textsf{childdoc} Package}
\hypersetup{pdftitle={The childdoc Package}}
\author{Niklas Beisert\\[2ex]
  Institut f\"ur Theoretische Physik\\
  Eidgen\"ossische Technische Hochschule Z\"urich\\
  Wolfgang-Pauli-Strasse 27, 8093 Z\"urich, Switzerland\\[1ex]
  \href{mailto:nbeisert@itp.phys.ethz.ch}
  {\texttt{nbeisert@itp.phys.ethz.ch}}}
\hypersetup{pdfauthor={Niklas Beisert}}
\hypersetup{pdfsubject={Manual for the LaTeX2e Package childdoc}}
\date{30 December 2018, \textsf{v2.0}}
\maketitle

\begin{abstract}\noindent
\textsf{childdoc} is a \LaTeXe{} package
that enables the direct compilation
of document sections included by |\include|
to individual files.
\end{abstract}

\begingroup
\parskip0ex
\tableofcontents
\endgroup

%%%%%%%%%%%%%%%%%%%%%%%%%%%%%%%%%%%%%%%%%%%%%%%%%%%%%%%%%%%%%%%%%%%%%%%%%%%%%%%%
%%%%%%%%%%%%%%%%%%%%%%%%%%%%%%%%%%%%%%%%%%%%%%%%%%%%%%%%%%%%%%%%%%%%%%%%%%%%%%%%
\section{Introduction}

\LaTeX{} provides a mechanism to structure a large document (such as a book)
into a main file and several child files (containing the chapters)
using the |\include| command.
This mechanism is beneficial for documents
which span hundreds of pages in order to
make the source file(s) more manageable.
Moreover, compilation can be restricted to
selected child files by means of the |\includeonly| command.
The latter feature can be used to reduce the compilation time while editing
(this was significantly more useful in the earlier days of \LaTeX{})
or to generate a smaller document which is easier to navigate.
Another application of |\includeonly| is to generate
documents consisting of selected parts of the complete document.

However, there are a few drawbacks of the plain |\include| mechanism:
\begin{itemize}
\item
The child files cannot be compiled on their own,
they can only be compiled via the main file.
A naive editing environment
(such as a text editor with an option
to have the current file processed by \LaTeX)
may require one to switch to the main file before compiling;
attempting to compile the child file produces errors.
\item
The main file must be modified (each time)
to adjust the |\includeonly| command
to the present needs. This easily leaves the main file in a messy state.
\item
The generated document will always carry the filename
of the main document. This is inconvenient if
several child files are to be compiled and
to be kept for distribution.
\end{itemize}

The present package provides a simple interface
to make child files individually compilable by \LaTeX{}.
Compiling a child file then has the same effect as compiling
the main file with an |\includeonly| command
to select the appropriate child.
Moreover the generated document will carry the name of the child
rather than the main file.
This resolves all three above issues.

This feature is meant to make the editing of books,
thesis documents and lecture notes somewhat more convenient.
However, the package can also be used efficiently for
composing a series of documents (such as exercise sheets)
which are typically distributed individually.
It then assists the author in generating the individual documents
(potentially in different versions)
as well as a document containing the collected series.
Another application is in developing style files
or other kinds of included material
where compilation of the style file could redirect
to a sample or test file.

%%%%%%%%%%%%%%%%%%%%%%%%%%%%%%%%%%%%%%%%%%%%%%%%%%%%%%%%%%%%%%%%%%%%%%%%%%%%%%%%
%%%%%%%%%%%%%%%%%%%%%%%%%%%%%%%%%%%%%%%%%%%%%%%%%%%%%%%%%%%%%%%%%%%%%%%%%%%%%%%%
\section{Usage}

First of all, the package \textsf{childdoc} is \emph{not} a standard
\LaTeXe{} |.sty| style file! Therefore it needs to be invoked in
a non-standard way.

%%%%%%%%%%%%%%%%%%%%%%%%%%%%%%%%%%%%%%%%%%%%%%%%%%%%%%%%%%%%%%%%%%%%%%%%%%%%%%%%
\subsection{Included Files}
\label{sec:include}

%%%%%%%%%%%%%%%%%%%%%%%%%%%%%%%%%%%%%%%%
\DescribeMacro{\childdocmain}
To use the package, add the commands
\begin{center}
\begin{tabular}{l}
|\input{childdoc.def}|\\
|\childdocmain{}|\\
\end{tabular}
\end{center}
at the very top of the main \LaTeX{} file,
in particular \emph{before} the |\documentclass| statement!
The argument of |\childdocmain| should be left empty
(but it must be present).

%%%%%%%%%%%%%%%%%%%%%%%%%%%%%%%%%%%%%%%%
\DescribeMacro{\childdocof}
Furthermore, add the commands
\begin{center}
\begin{tabular}{l}
|\input{childdoc.def}|\\
|\childdocof{|\textit{main}|}|\\
\end{tabular}
\end{center}
at the top of every child file \textit{child}
which is included by |\include{|\textit{child}|}|
from within the main file
(or at least for those files to be compiled individually).
The argument \textit{main} must be the filename of the main file.

There are a couple of
considerations in setting up the main and child documents:

%%%%%%%%%%%%%%%%%%%%%%%%%%%%%%%%%%%%%%%%
\paragraph{Restrictions.}

Please note the following restrictions:
\begin{itemize}
\item
|\childdocmain| must be called with one argument \textit{main}
to ensure compatibility with earlier version of the package.
It must either be empty (|\childdocmain{}|)
or precisely match the filename of the main file in which it is specified.
See \secref{sec:detection} for further information.
\item
The filename \textit{main} must be specified without the |.tex| extension.
\item
The filename \textit{main} is case sensitive
(even in case-insensitive file systems)
due to internal string comparison.
\item
The argument \textit{main} should be fully expanded, it cannot be a macro.
\item
Subdirectories and special characters should be avoided in filenames.
\item
The command |\childdocmain{|\textit{main}|}| must be followed by a whitespace.
It should not be followed immediately by another command
or by a comment mark `|%|'.
This is because the \TeX{} parser reads the token immediately following
the argument of |\childdocmain| and puts it
at the beginning of every child section;
however, a white\-space is ignored.
\end{itemize}

%%%%%%%%%%%%%%%%%%%%%%%%%%%%%%%%%%%%%%%%
\paragraph{Content of Main File.}

It is advisable to place all content in the child files included by |\include|.
Any output contained in the main file will appear in all child documents
unless suppressed manually;
it cannot be suppressed automatically by the |\includeonly| directive
and thus should normally be avoided.
A method to include some content in the main file
by means of conditional processing is described in \secref{sec:conditional}.

%%%%%%%%%%%%%%%%%%%%%%%%%%%%%%%%%%%%%%%%
\paragraph{Page Numbering.}

When only a part of the document is compiled,
the appropriate numbering of pages
(as well as other status parameters)
is determined from the |.aux| files.
The latter contain information from previous passes.
However this information needs to propagate through
all intermediate child documents.
Therefore the page numbering in child documents may well
be inconsistent until the complete document is compiled at least once.

A useful (if unconventional) way to always ensure a consistent
page numbering is to restart the numbering in each child document
and denote the pages by `\textit{child}|.|\textit{page}'
where \textit{child} represents the chapter/section number of the child file.
This can be achieved by the command
|\numberwithin{page}{|\textit{child}|}|
of the \textsf{amsmath} package
where \textit{child} can be |chapter| or |section|
depending on the chosen structuring.
Alternatively, one can modify the macro |\thepage| appropriately
and reset the counter |page| at the start of each child file.

%%%%%%%%%%%%%%%%%%%%%%%%%%%%%%%%%%%%%%%%%%%%%%%%%%%%%%%%%%%%%%%%%%%%%%%%%%%%%%%%
\subsection{Conditional Processing}
\label{sec:conditional}

The package provides a mechanism to compile different versions
of a document. To customise the versions further some conditional processing
can come in handy to distinguish which version is being compiled.
The package provides two macros to describe the compilation context:

%%%%%%%%%%%%%%%%%%%%%%%%%%%%%%%%%%%%%%%%
\DescribeMacro{\ifchilddoc}
The conditional |\ifchilddoc| distinguishes between the compilation of
child documents and the main document:
%
\begin{center}
|\ifchilddoc |\textit{child-code}| |[|\||else |\textit{main-code}]| \||fi|
\end{center}

%%%%%%%%%%%%%%%%%%%%%%%%%%%%%%%%%%%%%%%%
\DescribeMacro{\childdocname}
\DescribeMacro{\childdocjob}
The macro |\childdocname| contains the filename (without extension)
of the main or child file being processed.
Note that |\childdocjob| will always contain the name of the main file.

%%%%%%%%%%%%%%%%%%%%%%%%%%%%%%%%%%%%%%%%
\paragraph{Title Page.}

Conditional processing can be used to include a title or banner page
in the main document when proper precautions are taken.
Importantly, the code in the main file should ensure that the page counter
(as well as other status parameters which are stored in the |.aux| files)
takes the same value after the conditional processing.
Otherwise the page numbers may take divergent values
depending on which part is compiled.

For example, a title page could be declared by:
%
\begin{center}
\begin{tabular}{l}
|\ifchilddoc\||else|\\
|\addtocounter{page}{-1}|\\
\textit{code for title page}\\
|\newpage|\\
|\||fi|
\end{tabular}
\end{center}
%
A banner page for the child documents can be generated by:
%
\begin{center}
\begin{tabular}{l}
|\ifchilddoc|\\
|\addtocounter{page}{-1}|\\
\textit{code for banner page}\\
|\newpage|\\
|\||fi|
\end{tabular}
\end{center}
%
Here one could write a message such as:
\begin{center}
|This is the part \childdocname{} of \childdocjob{}.|
\end{center}

%%%%%%%%%%%%%%%%%%%%%%%%%%%%%%%%%%%%%%%%%%%%%%%%%%%%%%%%%%%%%%%%%%%%%%%%%%%%%%%%
\subsection{Flags}
\label{sec:flags}

The package makes it easy to generate different versions
of the main or child documents.
To this end compilation flags can be defined
and assigned different default values.
They will be particularly useful in conjunction
with the forwarding mechanism described in \secref{sec:forward}.

For example, it may be useful to have a flag |\version|
which can be set to |draft| or |final|.
The document source will contain some conditional code
depending on the value of |\version|.
Suppose further, the flag should default to |final| for the main file
and to |draft| for child files
which is a natural assignment for editing the document.
This is achieved by placing the following code
in the preamble of the main document
(below the |\childdocmain| directive):
%
\begin{center}
\begin{tabular}{l}
|\ifchilddoc|\\
|\providecommand{\version}{draft}|\\
|\||else|\\
|\providecommand{\version}{final}|\\
|\||fi|
\end{tabular}
\end{center}
%
The definition by |\providecommand| makes sure
that previous definitions are not overwritten.
Further statements |\providecommand{\version}{...}|
can thus be added before the above code to override it.

For the main file, one might add a line
(between |\childdocmain| and the above block)
%
\begin{center}
|%\ifchilddoc\||else\providecommand{\version}{draft}\||fi|
\end{center}
%
which can be uncommented to produce a draft version.
Likewise one can add a line to the very top of a child file
(above the |\childdocof{|\textit{main}|}| directive)
%
\begin{center}
|%\providecommand{\version}{final}|
\end{center}
%
which can be uncommented to produce the final version of this child document.

%%%%%%%%%%%%%%%%%%%%%%%%%%%%%%%%%%%%%%%%%%%%%%%%%%%%%%%%%%%%%%%%%%%%%%%%%%%%%%%%
\subsection{Forwarding}
\label{sec:forward}

Different versions of the main or child documents
using compilation flags as described in \secref{sec:flags}
can be (permanently) stored in different files
for convenient compilation, viewing and distribution.
To this end, the package defines a command
to pass on compilation to a different file:

%%%%%%%%%%%%%%%%%%%%%%%%%%%%%%%%%%%%%%%%
\DescribeMacro{\childdocforward}
The command |\childdocforward| redirects processing to
another source file:
%
\begin{center}
\begin{tabular}{l}
|\input{childdoc.def}|\\
|\childdocforward[|\textit{main}|]{|\textit{dest}|}|\\
\end{tabular}
\end{center}
%
The argument \textit{dest} is the destination file
(without extension).
It should be the main file or one of the child files.
Note that further \textsf{childdoc} directives
such as |\childdocof| and |\childdocforward|
in the indicated file will be processed in this form.
The optional argument \textit{main}
passes on directly to the main file \textit{main}
while pretending to compile the child \textit{dest}.
This form behaves as if \textit{dest}
issues |\childdocof{|\textit{main}|}| right away,
and no further \textsf{childdoc} directives will be processed.

%%%%%%%%%%%%%%%%%%%%%%%%%%%%%%%%%%%%%%%%
\DescribeMacro{\...prefix}
In the alternative form |\childdocforwardprefix|,
%
\begin{center}
\begin{tabular}{l}
|\input{childdoc.def}|\\
|\childdocforwardprefix[|\textit{main}|]{|\textit{prefix}|}{|\textit{dest}|}|
\end{tabular}
\end{center}
%
the destination file is determined by a pattern
depending on the current file:
To make this work, the current file must be called
`{\textit{prefix}\hspace{0.2em}\textit{suffix}}'
with \textit{prefix} matching precisely the argument.
Processing is then passed on to the file
`{\textit{dest}\hspace{0.2em}\textit{suffix}}'.
Surely, the same effect is achieved by
directly specifying the
argument `{\textit{dest}\hspace{0.2em}\textit{suffix}}'
in the first form.
However, that requires to set up a different file
for each child. With the alternative form of the command
all these files can have exactly the same content
which simplifies setting them up and maintaining them.

For example, the following file |draft.tex|
with a compilation flag |\version| as described in \secref{sec:flags}
compiles the main document as a draft:
%
\begin{center}
\begin{tabular}{l}
|\def\version{draft}|\\
|\input{childdoc.def}|\\
|\childdocforward{|\textit{main}|}|
\end{tabular}
\end{center}
%
Likewise, the following files |final|\textit{nn}|.tex|
compile the final version of the child document
|child|\textit{nn}|.tex|:
%
\begin{center}
\begin{tabular}{l}
|\def\version{final}|\\
|\input{childdoc.def}|\\
|\childdocforwardprefix{final}{child}|
\end{tabular}
\end{center}
%

Note that when several versions of a main file and/or of each child file
are to be generated, it may be convenient to set up a |Makefile| or
shell script to automatise the process.

%%%%%%%%%%%%%%%%%%%%%%%%%%%%%%%%%%%%%%%%%%%%%%%%%%%%%%%%%%%%%%%%%%%%%%%%%%%%%%%%
\subsection{Command Line Processing}
\label{sec:commandline}

The effect of redirection files can also be achieved by invoking
the \LaTeX{} compiler with a more elaborate command line.
Most conveniently this should be done as part
of a shell script or a |Makefile|.

When using \textsf{childdoc} in the main file, the following
command lines effectively perform a redirection
(note that depending on the shell being used,
backslashes may have to be doubled: `|\|' $\to$ `|\\|'):
%
\begin{center}
|... -jobname "|\textit{target}|" |\\|"|[\textit{flags}]%
|\input{childdoc.def}\childdocforward[|\textit{main}|]{|\textit{dest}|}"|
\end{center}
%
Here \textit{target} is the name of the output file,
\textit{main} is the name of the main file
and \textit{dest} is the name of the main or child file to be processed
(all filenames without extensions).
The optional argument \textit{main} can be omitted
if \textit{main} matches \textit{dest}.
Optionally, compilation \textit{flags} can be defined via |\def| commands.
This command line makes the \TeX{} engine believe
it is compiling the file \textit{target}
whose content is specified as the latter parameter.
The provided code then forwards the processing to
\textit{main} or \textit{dest} as described in \secref{sec:forward}.

%%%%%%%%%%%%%%%%%%%%%%%%%%%%%%%%%%%%%%%%%%%%%%%%%%%%%%%%%%%%%%%%%%%%%%%%%%%%%%%%
\subsection{Include by Input}
\label{sec:input}

Including child documents by |\include| has some restrictions by design.
Most notably, the content of a child document always occupies
its own set of pages; pages cannot be shared between child documents.
Usually, this behaviour makes perfect sense
because each child document contain an essential part of the document.
However, in some situations it may be desirable to compose
a document from a collection of parts
without having mandatory page breaks between then.
For this case, the package
provides a mechanism to include parts
by |\input| which can also be processed individually.
However, by construction this mechanism
requires manual handling of the content to be output.

%%%%%%%%%%%%%%%%%%%%%%%%%%%%%%%%%%%%%%%%
\DescribeMacro{\ifchilddocmanual}
The main file should be prepared as usual, see \secref{sec:include}.
However, the document body must make a distinction
between processing of an individual part and of the main document, e.g.:
%
\begin{center}
\begin{tabular}{l}
|\ifchilddocmanual|\\
|\input{\childdocname}|\\
|\||else|\\
\textit{document body with }|\input{|\textit{part}|}|\\
|\||fi|
\end{tabular}
\end{center}
%
The conditional |\ifchilddocmanual| is true whenever
a part to be included by |\input| is being compiled,
and the name of the part is stored in |\childdocname|.

%%%%%%%%%%%%%%%%%%%%%%%%%%%%%%%%%%%%%%%%
\DescribeMacro{\childdocby}
Each part to be included by |\input| should start with:
%
\begin{center}
\begin{tabular}{l}
|\input{childdoc.def}|\\
|\childdocby{|\textit{main}|}|\\
\end{tabular}
\end{center}
%
The directive |\childdocby| is similar to |\childdocof|
described in \secref{sec:include},
but the subsequent selection of content must be done manually.
To that end, both |\ifchilddoc| and |\ifchilddocmanual|
will be true upon processing of a part,
and the name of the part is stored in |\childdocname|.
Note that |\jobname| will be set to the filename of the current part
so that each part receives an individual |.aux| file
that does not interfere with the |.aux| file(s) of the main document.
This behaviour can be altered by the alternative form
|\childdocby[*]{|\textit{main}|}| (with a non-empty optional argument)
which uses the |.aux| file of the main document
by setting |\jobname| to \textit{main}.

%%%%%%%%%%%%%%%%%%%%%%%%%%%%%%%%%%%%%%%%%%%%%%%%%%%%%%%%%%%%%%%%%%%%%%%%%%%%%%%%
\subsection{Driver Development}
\label{sec:driver}

The \textsf{childdoc} mechanism can also be use for the development
of definition files such as \LaTeX{} styles or classes.
This case differs from the above setup with multiple parts
included by |\include| in that no |\includeonly| should be invoked.
This can be achieved by starting the include file
(before |\ProvidesPackage|) with:
%
\begin{center}
\begin{tabular}{l}
|\input{childdoc.def}|\\
|\childdocforward{|\textit{main}|}|\\
\end{tabular}
\end{center}
%
or alternatively with:
%
\begin{center}
\begin{tabular}{l}
|\input{childdoc.def}|\\
|\childdocby{|\textit{main}|}|\\
\end{tabular}
\end{center}
%
Both forms have slightly different effects as described above.
The main file is prepared as usual, see \secref{sec:include}.

%%%%%%%%%%%%%%%%%%%%%%%%%%%%%%%%%%%%%%%%%%%%%%%%%%%%%%%%%%%%%%%%%%%%%%%%%%%%%%%%
\subsection{Legacy Detection}
\label{sec:detection}

The directive |\childdocmain| in the main file can detect
whether the complete document or merely a child is to be compiled
even without using the directive |\childdocof|.
This method is deprecated because it is less robust
and there is no compelling reason to use it;
it is merely provided for backward compatibility
and it may be removed in future versions.

If the detection mechanism is to be used,
it is mandatory to correctly specify
the filename of the main file as the argument of |\childdocmain|:
%
\begin{center}
\begin{tabular}{l}
|\input{childdoc.def}|\\
|\childdocmain{|\textit{main}|}|\\
\end{tabular}
\end{center}
%
If |\jobname| does not match the argument \textit{main} of |\childdocmain|,
it is assumed that |\jobname| points to the child file to be compiled.
When using |\childdocmain| with the main file specified as argument,
it suffices to start a child file
with just |\input{|\textit{main}|}|
without loading of the package and using |\childdocof|.
If instead all processing is done
with the appropriate \textsf{childdoc} directives,
the argument of \textit{main} of |\childdocmain| can be empty.

An alternative version of the command line processing described
in \secref{sec:commandline} using the detection mechanism reads:
%
\begin{center}
|... -jobname "|\textit{target}|" "|[\textit{flags}]%
[|\def\jobname{|\textit{dest}|}|]|\input{|\textit{main}|}"|
\end{center}

%%%%%%%%%%%%%%%%%%%%%%%%%%%%%%%%%%%%%%%%%%%%%%%%%%%%%%%%%%%%%%%%%%%%%%%%%%%%%%%%
\subsection{Manual Code}
\label{sec:manual}

In case one cannot be certain whether the definitions file |childdoc.def|
is installed on the target \TeX{} distribution
and one prefers not to ship it,
it is conceivable to paste a few relevant commands into the sources.

To that end, drop all statements |\input{childdoc.def}|
and perform the replacements as outlined below.
Instead of |\childdocmain{|\textit{main}|}| add the following code
to the top of the main file:
%
\begin{center}
\begin{tabular}{l}
|\||ifdefined\childdocname\endinput\||fi\newif\ifchilddoc|\\
|\edef\childdocname{\scantokens\expandafter{\jobname\noexpand}}|\\
|\def\childdocmain{|\textit{main}|}\||ifx\childdocmain\childdocname\||else|\\
|\childdoctrue\includeonly{\childdocname}\let\jobname\childdocmain\||fi|\\
\end{tabular}
\end{center}
%
Instead of |\childdocof{|\textit{main}|}| just include the main file
at the top of each child file:
%
\begin{center}
|\input{|\textit{main}|}|
\end{center}
%
A simple redirection |\childdocforward{|\textit{dest}|}| is achieved by:
%
\begin{center}
|\def\jobname{|\textit{dest}|}\input{\jobname}|
\end{center}
%
The redirection with prefix
|\childdocforwardprefix[|\textit{prefix}|]{|\textit{dest}|}|
is accomplished by:
%
\begin{center}
\begin{tabular}{l}
|{\edef\jobname{\scantokens\expandafter{\jobname\noexpand}}|\\
|\def\redirectjob |\textit{prefix}|#1~~~{\gdef\jobname{|\textit{dest}|#1}}|\\
|\expandafter\redirectjob\jobname~~~}\input{\jobname}|
\end{tabular}
\end{center}

In an alternative approach,
child documents can be compiled by a specific command line
without additional code or specific definitions:
%
\begin{center}
|... -jobname "|\textit{target}|" "|[\textit{flags}]%
|\includeonly{|\textit{dest}|}\input{|\textit{main}|}"|
\end{center}
%

%%%%%%%%%%%%%%%%%%%%%%%%%%%%%%%%%%%%%%%%%%%%%%%%%%%%%%%%%%%%%%%%%%%%%%%%%%%%%%%%
%%%%%%%%%%%%%%%%%%%%%%%%%%%%%%%%%%%%%%%%%%%%%%%%%%%%%%%%%%%%%%%%%%%%%%%%%%%%%%%%
\section{Information}

%%%%%%%%%%%%%%%%%%%%%%%%%%%%%%%%%%%%%%%%%%%%%%%%%%%%%%%%%%%%%%%%%%%%%%%%%%%%%%%%
\subsection{Copyright}

Copyright \copyright{} 2017--2018 Niklas Beisert

This work may be distributed and/or modified under the
conditions of the \LaTeX{} Project Public License, either version 1.3
of this license or (at your option) any later version.
The latest version of this license is in
  \url{http://www.latex-project.org/lppl.txt}
and version 1.3 or later is part of all distributions of \LaTeX{}
version 2005/12/01 or later.

This work has the LPPL maintenance status `maintained'.

The Current Maintainer of this work is Niklas Beisert.

This work consists of the files |README.txt|, |childdoc.ins| and |childdoc.dtx|
as well as the derived files |childdoc.def|, |cdocsamp.tex|
with |cdocsch1.tex|, |cdocsch2.tex|, |cdocspt3.tex|, |cdocspt4.tex|,
|cdocsdrf.tex|, |cdocsfn1.tex|, |cdocsfn2.tex|
as well as |childdoc.pdf|.

%%%%%%%%%%%%%%%%%%%%%%%%%%%%%%%%%%%%%%%%%%%%%%%%%%%%%%%%%%%%%%%%%%%%%%%%%%%%%%%%
\subsection{Files and Installation}

The package consists of the files:
%
\begin{center}
\begin{tabular}{ll}
    |README.txt|   & readme file \\
    |childdoc.ins| & installation file \\
    |childdoc.dtx| & source file \\
    |childdoc.def| & definition file \\
    |cdocsamp.tex| & sample main file \\
    |cdocsch1.tex| & sample include file \\
    |cdocsch2.tex| & sample include file \\
    |cdocspt3.tex| & sample part file \\
    |cdocspt4.tex| & sample part file \\
    |cdocsdrf.tex| & sample redirection file \\
    |cdocsfn1.tex| & sample redirection file \\
    |cdocsfn2.tex| & sample redirection file \\
    |childdoc.pdf| & manual
\end{tabular}
\end{center}
%
The distribution consists of the files
|README.txt|, |childdoc.ins| and |childdoc.dtx|.
%
\begin{itemize}
\item
Run (pdf)\LaTeX{} on |childdoc.dtx|
to compile the manual |childdoc.pdf| (this file).
\item
Run \LaTeX{} on |childdoc.ins| to create the definitions file |childdoc.def|
and the sample |cdocsamp.tex| with include files
|cdocsch1.tex|, |cdocsch2.tex|, |cdocspt3.tex|, |cdocspt4.tex|,
|cdocsdrf.tex|, |cdocsfn1.tex|, |cdocsfn2.tex|.
Then copy the file |childdoc.def| to an appropriate directory of your \LaTeX{}
distribution, e.g.\ \textit{texmf-root}|/tex/latex/childdoc|.
\end{itemize}

%%%%%%%%%%%%%%%%%%%%%%%%%%%%%%%%%%%%%%%%%%%%%%%%%%%%%%%%%%%%%%%%%%%%%%%%%%%%%%%%
\subsection{Related CTAN Packages}

There are several other packages which offer a similar functionality:
%
\begin{itemize}
\item
The packages
\href{http://ctan.org/pkg/docmute}{\textsf{docmute}},
\href{http://ctan.org/pkg/includex}{\textsf{includex}} and
\href{http://ctan.org/pkg/standalone}{\textsf{standalone}}
provide commands to include only the document body of
a child file thus allowing both files to be compiled individually.
\item
The packages \href{http://ctan.org/pkg/subdocs}{\textsf{subdocs}}
and \href{http://ctan.org/pkg/subfiles}{\textsf{subfiles}}
provide structures in which the main and child documents can be
encapsulated and allowing them to be compiled individually.
The inclusion mechanism is different from the conventional |\include|.
\item
The package \href{http://ctan.org/pkg/combine}{\textsf{combine}}
is an elaborate solution to combine several documents into one.
\end{itemize}
%
See also the CTAN topic \href{http://ctan.org/topic/subdocs}{\textsf{subdocs}}
for further related packages.
The present package differs from the above solutions in that
a document structure constructed with the conventional |\include| mechanism
just needs two extra commands at the top of every file
such that all constituent files can be compiled individually.

%%%%%%%%%%%%%%%%%%%%%%%%%%%%%%%%%%%%%%%%%%%%%%%%%%%%%%%%%%%%%%%%%%%%%%%%%%%%%%%%
%\subsection{Feature Suggestions}
%
%The following is a list of features which may be useful for future
%versions of this package:
%%
%\begin{itemize}
%\item
%\ldots
%\end{itemize}

%%%%%%%%%%%%%%%%%%%%%%%%%%%%%%%%%%%%%%%%%%%%%%%%%%%%%%%%%%%%%%%%%%%%%%%%%%%%%%%%
\subsection{Revision History}

%%%%%%%%%%%%%%%%%%%%%%%%%%%%%%%%%%%%%%%%
\paragraph{v2.0:} 2018/12/30

\begin{itemize}
\item
immediate forward processing
\item
added |\childdocby| mechanism
\item
manual restructured
\end{itemize}

%%%%%%%%%%%%%%%%%%%%%%%%%%%%%%%%%%%%%%%%
\paragraph{v1.6:} 2018/01/17

\begin{itemize}
\item
application for development of include files
\item
corrections to manual
\end{itemize}

%%%%%%%%%%%%%%%%%%%%%%%%%%%%%%%%%%%%%%%%
\paragraph{v1.5:} 2017/05/21

\begin{itemize}
\item
more complete structuring introduced
\item
|\childdocof| introduced
\item
|\childdoc| renamed to |\childdocmain|
\item
|\childredirect| renamed to |\childdocforward| and |\childdocforwardprefix|
and functionality expanded
\end{itemize}

%%%%%%%%%%%%%%%%%%%%%%%%%%%%%%%%%%%%%%%%
\paragraph{v1.0:} 2017/04/27

\begin{itemize}
\item
manual and install package
\item
first version published on CTAN
\end{itemize}

%%%%%%%%%%%%%%%%%%%%%%%%%%%%%%%%%%%%%%%%
\paragraph{v0.6:} 2017/04/26

\begin{itemize}
\item
redirection mechanism added
\end{itemize}

%%%%%%%%%%%%%%%%%%%%%%%%%%%%%%%%%%%%%%%%
\paragraph{v0.5:} 2017/04/26

\begin{itemize}
\item
functionality in definition file
\end{itemize}


%%%%%%%%%%%%%%%%%%%%%%%%%%%%%%%%%%%%%%%%%%%%%%%%%%%%%%%%%%%%%%%%%%%%%%%%%%%%%%%%
%%%%%%%%%%%%%%%%%%%%%%%%%%%%%%%%%%%%%%%%%%%%%%%%%%%%%%%%%%%%%%%%%%%%%%%%%%%%%%%%
%%%%%%%%%%%%%%%%%%%%%%%%%%%%%%%%%%%%%%%%%%%%%%%%%%%%%%%%%%%%%%%%%%%%%%%%%%%%%%%%
\appendix

\settowidth\MacroIndent{\rmfamily\scriptsize 000\ }

 \DocInput{childdoc.dtx}

\end{document}
%</driver>
% \fi
%
% %%%%%%%%%%%%%%%%%%%%%%%%%%%%%%%%%%%%%%%%%%%%%%%%%%%%%%%%%%%%%%%%%%%%%%%%%%%%%%
% %%%%%%%%%%%%%%%%%%%%%%%%%%%%%%%%%%%%%%%%%%%%%%%%%%%%%%%%%%%%%%%%%%%%%%%%%%%%%%
% \section{Sample}
%\iffalse
%<*samplemain>
%\fi
%
% The following presents a sample document
% with two chapters, two parts, a title page,
% a compile flag as well as three forwarding files to set the flag.
% It consists of eight |.tex| files:
% \begin{center}
% \begin{tabular}{ll}
% |cdocsamp.tex|&main file\\
% |cdocsch1.tex|&include file for chapter 1\\
% |cdocsch2.tex|&include file for chapter 2\\
% |cdocspt3.tex|&include file for part 3\\
% |cdocspt4.tex|&include file for part 4\\
% |cdocsdrf.tex|&forwarding file for main file in draft mode\\
% |cdocsfi1.tex|&forwarding file for final version of chapter 1\\
% |cdocsfi2.tex|&forwarding file for final version of chapter 2\\
% \end{tabular}
% \end{center}
% Each of the eight files can be compiled directly by the \LaTeX{} compiler.
%
% %%%%%%%%%%%%%%%%%%%%%%%%%%%%%%%%%%%%%%
% \paragraph{Main File.}
%
% The main file is called |cdocsamp.tex|.
%
% Load the \textsf{childdoc} definitions and
% declare the filename for the main document:
%    \begin{macrocode}
\input{childdoc.def}
\childdocmain{}
%    \end{macrocode}

% Optional override for |\version| flag:
%    \begin{macrocode}
%%\ifchilddoc\else\providecommand{\version}{draft}\fi
%    \end{macrocode}

% Define the default values for the |\version| flag
% (|final| for the main file and |draft| for childs):
%    \begin{macrocode}
\ifchilddoc
\providecommand{\version}{draft}
\else
\providecommand{\version}{final}
\fi
%    \end{macrocode}

% Load the standard document class:
%    \begin{macrocode}
\documentclass[12pt]{article}
%    \end{macrocode}

% Start the document body:
%    \begin{macrocode}
\begin{document}
%    \end{macrocode}

% Declare a title page.
% Print title, part of document being processed and version flag:
%    \begin{macrocode}
\addtocounter{page}{-1}
\begin{center}
{\LARGE\bfseries{}childdoc example\par}
\vspace{1cm}
\ifchilddoc
\ifchilddocmanual part\else chapter\fi:
`\childdocname' of `\childdocjob'\par
\else
main document: `\childdocjob'\par
\fi
version: \version\par
\end{center}
\newpage
%    \end{macrocode}

% Manually include selected file,
% otherwise process as usual:
%    \begin{macrocode}
\ifchilddocmanual
\section*{part `\childdocname'}
\input{\childdocname}
\else
%    \end{macrocode}

% Include the two chapters:
%    \begin{macrocode}
\include{cdocsch1}
\include{cdocsch2}
%    \end{macrocode}

% Include the two parts unless only chapters should be displayed:
%    \begin{macrocode}
\ifchilddoc\else
\section{part three}
\input{cdocspt3}
\section{part four}
\input{cdocspt4}
\fi
%    \end{macrocode}

% Process as usual until here:
%    \begin{macrocode}
\fi
%    \end{macrocode}

% End of document body:
%    \begin{macrocode}
\end{document}
%    \end{macrocode}
%\iffalse
%</samplemain>
%\fi
%
% %%%%%%%%%%%%%%%%%%%%%%%%%%%%%%%%%%%%%%
% \paragraph{Chapter Include Files.}
%
% The include files are called |cdocsch1.tex| and |cdocsch2.tex|.
%
%\iffalse
%<*samplechap1|samplechap2>
%\fi

% Optional override for |\version| flag:
%    \begin{macrocode}
%%\providecommand{\version}{final}
%    \end{macrocode}

% Include the main document:
%    \begin{macrocode}
\input{childdoc.def}
\childdocof{cdocsamp}
%    \end{macrocode}

%\iffalse
%</samplechap1|samplechap2>
%\fi
%
%\iffalse
%<*samplechap1>
%\fi
% Some text for chapter 1:
%    \begin{macrocode}
\section{one}
some text in chapter one
%    \end{macrocode}

%\iffalse
%</samplechap1>
%\fi
% Some text for chapter 2:
%\iffalse
%<*samplechap2>
%\fi
%    \begin{macrocode}
\section{two}
more text in chapter two
%    \end{macrocode}

%\iffalse
%</samplechap2>
%\fi
%
% %%%%%%%%%%%%%%%%%%%%%%%%%%%%%%%%%%%%%%
% \paragraph{Part Include Files.}
%
% The include files are called |cdocspt3.tex| and |cdocspt4.tex|.
%
%\iffalse
%<*samplepart3|samplepart4>
%\fi

% Optional override for |\version| flag:
%    \begin{macrocode}
%%\providecommand{\version}{final}
%    \end{macrocode}

% Include the main document:
%    \begin{macrocode}
\input{childdoc.def}
\childdocby{cdocsamp}
%    \end{macrocode}

%\iffalse
%</samplepart3|samplepart4>
%\fi
%
%\iffalse
%<*samplepart3>
%\fi
% Some text for part 3:
%    \begin{macrocode}
some text in part three
%    \end{macrocode}

%\iffalse
%</samplepart3>
%\fi
% Some text for part 4:
%\iffalse
%<*samplepart4>
%\fi
%    \begin{macrocode}
more text in part four
%    \end{macrocode}

%\iffalse
%</samplepart4>
%\fi
%
% %%%%%%%%%%%%%%%%%%%%%%%%%%%%%%%%%%%%%%
% \paragraph{Forwarding for a Complete Draft.}
%
% The following forwarding file |cdocsdrf.tex|
% compiles the main document in draft mode:
%\iffalse
%<*sampledraft>
%\fi
%    \begin{macrocode}
\def\version{draft}
\input{childdoc.def}
\childdocforward{cdocsamp}
%    \end{macrocode}

%\iffalse
%</sampledraft>
%\fi
%
% %%%%%%%%%%%%%%%%%%%%%%%%%%%%%%%%%%%%%%
% \paragraph{Forwarding for Final Version of the Chapters.}
%
% The following forwarding files |cdocsfn1.tex| and |cdocsfn2.tex|
% (with identical content)
% compile the final versions of the child documents
% |cdocsch1.tex| and |cdocsch2.tex|, respectively:
%\iffalse
%<*samplefinal>
%\fi
%    \begin{macrocode}
\def\version{final}
\input{childdoc.def}
\childdocforwardprefix[cdocsamp]{cdocsfn}{cdocsch}
%    \end{macrocode}

%\iffalse
%</samplefinal>
%\fi
%
% %%%%%%%%%%%%%%%%%%%%%%%%%%%%%%%%%%%%%%
% \paragraph{Command Line Processing.}
%
% The following three command lines generate the output files
% |cdocscld|, |cdocscl1| and |cdocscl2|
% which should be identical to
% |cdocsdrf|, |cdocsch1| and |cdocsfn2|, respectively:
% \begin{center}
% \begin{tabular}{l}
% |latex -jobname cdocscld \|\\
% |  "\def\version{draft}\input{childdoc.def}\childdocforward{cdocsamp}"|\\
% |latex -jobname cdocscl1 \|\\
% |  "\input{childdoc.def}\childdocforward[cdocsamp]{cdocsch1}"|\\
% |latex -jobname cdocscl2 \|\\
% |  "\def\version{final}\input{childdoc.def}\childdocforward{cdocsch2}"|
% \end{tabular}
% \end{center}
% Note that the trailing backslash on each first line
% merely continues the input to the second line
% (for convenient cut ant paste).
% Furthermore, the command |latex| can be replaced by any
% of its alternative versions such as |pdflatex|.
%
% %%%%%%%%%%%%%%%%%%%%%%%%%%%%%%%%%%%%%%%%%%%%%%%%%%%%%%%%%%%%%%%%%%%%%%%%%%%%%%
% %%%%%%%%%%%%%%%%%%%%%%%%%%%%%%%%%%%%%%%%%%%%%%%%%%%%%%%%%%%%%%%%%%%%%%%%%%%%%%
% \section{Implementation}
%\iffalse
%<*package>
%\fi
%
% This section describes the definitions file |childdoc.def|.

% The definitions cannot be loaded using |\usepackage| or |\RequirePackage|
% which has a mechanism to prevent loading a style file more than once.
% When loading the definitions by means of |\input|
% multiple instances have to be prevented manually:
%\iffalse
%This code needs to be before the `\ProvidesFile' directive
%which is defined at the beginning of this file.
%Therefore it is also placed there and commented out here.
%</package>
%<*discard>
%\fi
%    \begin{macrocode}
\ifdefined\childdocmain\endinput\fi
%    \end{macrocode}
%\iffalse
%</discard>
%<*package>
%\fi
%
% \macro{\ifchilddoc}
% \macro{\ifchilddocmanual}
% The conditional |\ifchilddoc| tells whether a
% child (true) or main (false) document is being compiled.
% The conditional |\ifchilddocmanual| tells whether
% the |\includeonly| mechanism is used (false) or
% the selection of child files must be performed manually (true).
% The definitions initialise to false:
%    \begin{macrocode}
\newif\ifchilddoc
\newif\ifchilddocmanual
%    \end{macrocode}

% \macro{\childdocname}
% \macro{\childdocjob}
% The macro |\childdocname| stores the name of the main document
% to be compiled. The macro |\childdocjob| stores the name of
% the document on which the \LaTeX{} compiler was originally invoked.
% The content of |\jobname| cannot be compared
% to filenames specified in the source due to different catcodes.
% The following code rescans |\jobname|, stores the result
% in |\childdocname| and saves a copy in |\childdocjob|:
%    \begin{macrocode}
\edef\childdocname{\scantokens\expandafter{\jobname\noexpand}}
\let\childdocjob\childdocname
%    \end{macrocode}

% \macro{\childdocdisable}
% The macro |\childdocdisable| prevents the main file
% from being processed more than once.
% At this stage, the main document command |\childdocmain|
% is assumed to be called once again where it should do nothing.
% Any subsequent call to it should prevent
% a secondary processing of the main document
% It overwrites the forwarding commands
% |\childdocof| and |\childdocforward|
% with empty macros to prevent further inclusions of the main document:
%    \begin{macrocode}
\newcommand{\childdocdisable}
{
  \renewcommand{\childdocmain}[1]{\renewcommand{\childdocmain}[1]{\endinput}}
  \renewcommand{\childdocof}[1]{}
  \renewcommand{\childdocby}[2][]{}
  \renewcommand{\childdocforward}[2][]{}
  \renewcommand{\childdocdisable}{}
}
%    \end{macrocode}

% \macro{\childdocmain}
% The macro |\childdocmain| is to be called at the top of the main file
% with nothing or the main filename (without extension) as argument.
% First, it breaks loops.
% If the argument is not empty and does not match |\childdocname|
% (which is set by the first inclusion of |childdoc.def|),
% |\ifchilddoc| is set to true, |\includeonly| is applied to the child file
% and |\jobname| is set to the main file
% (for proper handling of |.aux| files):
%    \begin{macrocode}
\newcommand{\childdocmain}[1]
{
  \childdocdisable\childdocmain{}
  \if?#1?\else
    \begingroup
      \def\childdoctmp{#1}
      \ifx\childdoctmp\childdocname
        \def\childdoctmp{}
      \else
        \def\childdoctmp
        {
          \childdoctrue
          \includeonly{\childdocname}
          \def\childdocjob{#1}
          \def\jobname{#1}
        }
      \fi
      \expandafter
    \endgroup
    \childdoctmp
  \fi
}
%    \end{macrocode}

% \macro{\childdocof}
% The command |\childdocof| redirects
% compilation to the main file |#1|.
%    \begin{macrocode}
\newcommand{\childdocof}[1]
{
  \childdocdisable
  \childdoctrue
  \includeonly{\childdocname}
  \def\jobname{#1}
  \def\childdocjob{#1}
  \input{#1}
}
%    \end{macrocode}

% \macro{\childdocby}
% The command |\childdocby| ....
%    \begin{macrocode}
\newcommand{\childdocby}[2][]
{
  \childdocdisable
  \childdoctrue
  \childdocmanualtrue
  \if?#1?\else
    \def\jobname{#2}
  \fi
  \def\childdocjob{#2}
  \input{#2}
  \endinput
}
%    \end{macrocode}

% \macro{\childdocforward}
% The command |\childdocforward| redirects
% compilation to the main file or
% (if the optional argument is given) a child file.
% Parameters are set as if the main file
% or a child file starting with |\childdocof| was compiled.
% Then compilation is handed over to the main file:
%    \begin{macrocode}
\newcommand{\childdocforward}[2][]
{
  \begingroup
    \if?#1?
      \def\childdoctmp
      {
        \def\childdocname{#2}
        \def\childdocjob{#2}
        \def\jobname{#2}
        \input{#2}
        \endinput
      }
    \else
      \def\childdoctmp
      {
        \childdocdisable
        \def\childdocname{#2}
        \childdoctrue
        \includeonly{#2}
        \def\childdocjob{#1}
        \def\jobname{#1}
        \input{#1}
        \endinput
      }
    \fi
    \expandafter
  \endgroup
  \childdoctmp
}
%    \end{macrocode}

% \macro{\childdocforwardprefix}
% The command |\childdocforwardprefix| redirects
% compilation to the main or a child file by means of a pattern.
% The prefix |#1| in the current filename is replaced by |#2|
% and the suffix of the current filename is kept
% (it is assumed that the filename does not contain the substring `|~~~|'
% which is used as a delimiter).
% Compilation is handed over to the new file by |\childdocforward|:
%    \begin{macrocode}
\newcommand{\childdocforwardprefix}[3][]
{
  \begingroup
    \def\childdocextract #2##1~~~{\def\childdoctmp{\childdocforward[#1]{#3##1}}}
    \expandafter\childdocextract\childdocname~~~
    \expandafter
  \endgroup
  \childdoctmp
}
%    \end{macrocode}

% \macro{\childdoc}
% The deprecated macro |\childdoc| is a legacy version of |\childdocmain|:
%    \begin{macrocode}
\newcommand{\childdoc}{\childdocmain}
%    \end{macrocode}

% \macro{\childdocredirect}
% The deprecated macro |\childdocredirect| is a legacy version
% of |\childdocforward| and |\childdocforwardprefix|:
%    \begin{macrocode}
\newcommand{\childdocredirect}[2][]
{
  \begingroup
    \if?#1?
      \def\childdoctmp{\childdocforward{#2}}
    \else
      \def\childdoctmp{\childdocforwardprefix{#1}{#2}}
    \fi
    \expandafter
  \endgroup
  \childdoctmp
}
%    \end{macrocode}

%\iffalse
%</package>
%\fi
%
\endinput
|\\
|\childdocby{|\textit{main}|}|\\
\end{tabular}
\end{center}
%
Both forms have slightly different effects as described above.
The main file is prepared as usual, see \secref{sec:include}.

%%%%%%%%%%%%%%%%%%%%%%%%%%%%%%%%%%%%%%%%%%%%%%%%%%%%%%%%%%%%%%%%%%%%%%%%%%%%%%%%
\subsection{Legacy Detection}
\label{sec:detection}

The directive |\childdocmain| in the main file can detect
whether the complete document or merely a child is to be compiled
even without using the directive |\childdocof|.
This method is deprecated because it is less robust
and there is no compelling reason to use it;
it is merely provided for backward compatibility
and it may be removed in future versions.

If the detection mechanism is to be used,
it is mandatory to correctly specify
the filename of the main file as the argument of |\childdocmain|:
%
\begin{center}
\begin{tabular}{l}
|% \iffalse
%
% childdoc.dtx Copyright (C) 2017-2018 Niklas Beisert
%
% This work may be distributed and/or modified under the
% conditions of the LaTeX Project Public License, either version 1.3
% of this license or (at your option) any later version.
% The latest version of this license is in
%   http://www.latex-project.org/lppl.txt
% and version 1.3 or later is part of all distributions of LaTeX
% version 2005/12/01 or later.
%
% This work has the LPPL maintenance status `maintained'.
%
% The Current Maintainer of this work is Niklas Beisert.
%
% This work consists of the files childdoc.dtx and childdoc.ins
% and the derived files childdoc.def and cdocsamp.tex with
% cdocsch1.tex, cdocsch2.tex, cdocsdrf.tex, cdocsfn1.tex, cdocsfn2.tex.
%
%<package>\ifdefined\childdocmain\endinput\fi
%<package>\ProvidesFile{childdoc.def}[2018/12/30 v2.0 child document driver]
%<samplemain>\ProvidesFile{cdocsamp.tex}[2018/12/30 v2.0 sample for childdoc]
%<*driver>
%\ProvidesFile{childdoc.drv}[2018/12/30 v2.0 childdoc reference manual file]
\PassOptionsToClass{10pt,a4paper}{article}
\documentclass{ltxdoc}

\usepackage[margin=35mm]{geometry}
\usepackage{hyperref}
\usepackage{hyperxmp}
\usepackage[usenames]{color}

\hypersetup{colorlinks=true}
\hypersetup{pdfstartview=FitH}
\hypersetup{pdfpagemode=UseNone}
\hypersetup{pdfsource={}}
\hypersetup{pdflang={en-UK}}
\hypersetup{pdfcopyright={Copyright 2017-2018 Niklas Beisert.
  This work may be distributed and/or modified under the
  conditions of the LaTeX Project Public License, either version 1.3
  of this license or (at your option) any later version.}}
\hypersetup{pdflicenseurl={http://www.latex-project.org/lppl.txt}}
\hypersetup{pdfcontactaddress={ETH Zurich, ITP, HIT K,
  Wolfgang-Pauli-Strasse 27}}
\hypersetup{pdfcontactpostcode={8093}}
\hypersetup{pdfcontactcity={Zurich}}
\hypersetup{pdfcontactcountry={Switzerland}}
\hypersetup{pdfcontactemail={nbeisert@itp.phys.ethz.ch}}
\hypersetup{pdfcontacturl={http://people.phys.ethz.ch/\xmptilde nbeisert/}}

\newcommand{\secref}[1]{\hyperref[#1]{section \ref*{#1}}}

\parskip1ex
\parindent0pt
\let\olditemize\itemize
\def\itemize{\olditemize\parskip0pt}

\begin{document}

\title{The \textsf{childdoc} Package}
\hypersetup{pdftitle={The childdoc Package}}
\author{Niklas Beisert\\[2ex]
  Institut f\"ur Theoretische Physik\\
  Eidgen\"ossische Technische Hochschule Z\"urich\\
  Wolfgang-Pauli-Strasse 27, 8093 Z\"urich, Switzerland\\[1ex]
  \href{mailto:nbeisert@itp.phys.ethz.ch}
  {\texttt{nbeisert@itp.phys.ethz.ch}}}
\hypersetup{pdfauthor={Niklas Beisert}}
\hypersetup{pdfsubject={Manual for the LaTeX2e Package childdoc}}
\date{30 December 2018, \textsf{v2.0}}
\maketitle

\begin{abstract}\noindent
\textsf{childdoc} is a \LaTeXe{} package
that enables the direct compilation
of document sections included by |\include|
to individual files.
\end{abstract}

\begingroup
\parskip0ex
\tableofcontents
\endgroup

%%%%%%%%%%%%%%%%%%%%%%%%%%%%%%%%%%%%%%%%%%%%%%%%%%%%%%%%%%%%%%%%%%%%%%%%%%%%%%%%
%%%%%%%%%%%%%%%%%%%%%%%%%%%%%%%%%%%%%%%%%%%%%%%%%%%%%%%%%%%%%%%%%%%%%%%%%%%%%%%%
\section{Introduction}

\LaTeX{} provides a mechanism to structure a large document (such as a book)
into a main file and several child files (containing the chapters)
using the |\include| command.
This mechanism is beneficial for documents
which span hundreds of pages in order to
make the source file(s) more manageable.
Moreover, compilation can be restricted to
selected child files by means of the |\includeonly| command.
The latter feature can be used to reduce the compilation time while editing
(this was significantly more useful in the earlier days of \LaTeX{})
or to generate a smaller document which is easier to navigate.
Another application of |\includeonly| is to generate
documents consisting of selected parts of the complete document.

However, there are a few drawbacks of the plain |\include| mechanism:
\begin{itemize}
\item
The child files cannot be compiled on their own,
they can only be compiled via the main file.
A naive editing environment
(such as a text editor with an option
to have the current file processed by \LaTeX)
may require one to switch to the main file before compiling;
attempting to compile the child file produces errors.
\item
The main file must be modified (each time)
to adjust the |\includeonly| command
to the present needs. This easily leaves the main file in a messy state.
\item
The generated document will always carry the filename
of the main document. This is inconvenient if
several child files are to be compiled and
to be kept for distribution.
\end{itemize}

The present package provides a simple interface
to make child files individually compilable by \LaTeX{}.
Compiling a child file then has the same effect as compiling
the main file with an |\includeonly| command
to select the appropriate child.
Moreover the generated document will carry the name of the child
rather than the main file.
This resolves all three above issues.

This feature is meant to make the editing of books,
thesis documents and lecture notes somewhat more convenient.
However, the package can also be used efficiently for
composing a series of documents (such as exercise sheets)
which are typically distributed individually.
It then assists the author in generating the individual documents
(potentially in different versions)
as well as a document containing the collected series.
Another application is in developing style files
or other kinds of included material
where compilation of the style file could redirect
to a sample or test file.

%%%%%%%%%%%%%%%%%%%%%%%%%%%%%%%%%%%%%%%%%%%%%%%%%%%%%%%%%%%%%%%%%%%%%%%%%%%%%%%%
%%%%%%%%%%%%%%%%%%%%%%%%%%%%%%%%%%%%%%%%%%%%%%%%%%%%%%%%%%%%%%%%%%%%%%%%%%%%%%%%
\section{Usage}

First of all, the package \textsf{childdoc} is \emph{not} a standard
\LaTeXe{} |.sty| style file! Therefore it needs to be invoked in
a non-standard way.

%%%%%%%%%%%%%%%%%%%%%%%%%%%%%%%%%%%%%%%%%%%%%%%%%%%%%%%%%%%%%%%%%%%%%%%%%%%%%%%%
\subsection{Included Files}
\label{sec:include}

%%%%%%%%%%%%%%%%%%%%%%%%%%%%%%%%%%%%%%%%
\DescribeMacro{\childdocmain}
To use the package, add the commands
\begin{center}
\begin{tabular}{l}
|\input{childdoc.def}|\\
|\childdocmain{}|\\
\end{tabular}
\end{center}
at the very top of the main \LaTeX{} file,
in particular \emph{before} the |\documentclass| statement!
The argument of |\childdocmain| should be left empty
(but it must be present).

%%%%%%%%%%%%%%%%%%%%%%%%%%%%%%%%%%%%%%%%
\DescribeMacro{\childdocof}
Furthermore, add the commands
\begin{center}
\begin{tabular}{l}
|\input{childdoc.def}|\\
|\childdocof{|\textit{main}|}|\\
\end{tabular}
\end{center}
at the top of every child file \textit{child}
which is included by |\include{|\textit{child}|}|
from within the main file
(or at least for those files to be compiled individually).
The argument \textit{main} must be the filename of the main file.

There are a couple of
considerations in setting up the main and child documents:

%%%%%%%%%%%%%%%%%%%%%%%%%%%%%%%%%%%%%%%%
\paragraph{Restrictions.}

Please note the following restrictions:
\begin{itemize}
\item
|\childdocmain| must be called with one argument \textit{main}
to ensure compatibility with earlier version of the package.
It must either be empty (|\childdocmain{}|)
or precisely match the filename of the main file in which it is specified.
See \secref{sec:detection} for further information.
\item
The filename \textit{main} must be specified without the |.tex| extension.
\item
The filename \textit{main} is case sensitive
(even in case-insensitive file systems)
due to internal string comparison.
\item
The argument \textit{main} should be fully expanded, it cannot be a macro.
\item
Subdirectories and special characters should be avoided in filenames.
\item
The command |\childdocmain{|\textit{main}|}| must be followed by a whitespace.
It should not be followed immediately by another command
or by a comment mark `|%|'.
This is because the \TeX{} parser reads the token immediately following
the argument of |\childdocmain| and puts it
at the beginning of every child section;
however, a white\-space is ignored.
\end{itemize}

%%%%%%%%%%%%%%%%%%%%%%%%%%%%%%%%%%%%%%%%
\paragraph{Content of Main File.}

It is advisable to place all content in the child files included by |\include|.
Any output contained in the main file will appear in all child documents
unless suppressed manually;
it cannot be suppressed automatically by the |\includeonly| directive
and thus should normally be avoided.
A method to include some content in the main file
by means of conditional processing is described in \secref{sec:conditional}.

%%%%%%%%%%%%%%%%%%%%%%%%%%%%%%%%%%%%%%%%
\paragraph{Page Numbering.}

When only a part of the document is compiled,
the appropriate numbering of pages
(as well as other status parameters)
is determined from the |.aux| files.
The latter contain information from previous passes.
However this information needs to propagate through
all intermediate child documents.
Therefore the page numbering in child documents may well
be inconsistent until the complete document is compiled at least once.

A useful (if unconventional) way to always ensure a consistent
page numbering is to restart the numbering in each child document
and denote the pages by `\textit{child}|.|\textit{page}'
where \textit{child} represents the chapter/section number of the child file.
This can be achieved by the command
|\numberwithin{page}{|\textit{child}|}|
of the \textsf{amsmath} package
where \textit{child} can be |chapter| or |section|
depending on the chosen structuring.
Alternatively, one can modify the macro |\thepage| appropriately
and reset the counter |page| at the start of each child file.

%%%%%%%%%%%%%%%%%%%%%%%%%%%%%%%%%%%%%%%%%%%%%%%%%%%%%%%%%%%%%%%%%%%%%%%%%%%%%%%%
\subsection{Conditional Processing}
\label{sec:conditional}

The package provides a mechanism to compile different versions
of a document. To customise the versions further some conditional processing
can come in handy to distinguish which version is being compiled.
The package provides two macros to describe the compilation context:

%%%%%%%%%%%%%%%%%%%%%%%%%%%%%%%%%%%%%%%%
\DescribeMacro{\ifchilddoc}
The conditional |\ifchilddoc| distinguishes between the compilation of
child documents and the main document:
%
\begin{center}
|\ifchilddoc |\textit{child-code}| |[|\||else |\textit{main-code}]| \||fi|
\end{center}

%%%%%%%%%%%%%%%%%%%%%%%%%%%%%%%%%%%%%%%%
\DescribeMacro{\childdocname}
\DescribeMacro{\childdocjob}
The macro |\childdocname| contains the filename (without extension)
of the main or child file being processed.
Note that |\childdocjob| will always contain the name of the main file.

%%%%%%%%%%%%%%%%%%%%%%%%%%%%%%%%%%%%%%%%
\paragraph{Title Page.}

Conditional processing can be used to include a title or banner page
in the main document when proper precautions are taken.
Importantly, the code in the main file should ensure that the page counter
(as well as other status parameters which are stored in the |.aux| files)
takes the same value after the conditional processing.
Otherwise the page numbers may take divergent values
depending on which part is compiled.

For example, a title page could be declared by:
%
\begin{center}
\begin{tabular}{l}
|\ifchilddoc\||else|\\
|\addtocounter{page}{-1}|\\
\textit{code for title page}\\
|\newpage|\\
|\||fi|
\end{tabular}
\end{center}
%
A banner page for the child documents can be generated by:
%
\begin{center}
\begin{tabular}{l}
|\ifchilddoc|\\
|\addtocounter{page}{-1}|\\
\textit{code for banner page}\\
|\newpage|\\
|\||fi|
\end{tabular}
\end{center}
%
Here one could write a message such as:
\begin{center}
|This is the part \childdocname{} of \childdocjob{}.|
\end{center}

%%%%%%%%%%%%%%%%%%%%%%%%%%%%%%%%%%%%%%%%%%%%%%%%%%%%%%%%%%%%%%%%%%%%%%%%%%%%%%%%
\subsection{Flags}
\label{sec:flags}

The package makes it easy to generate different versions
of the main or child documents.
To this end compilation flags can be defined
and assigned different default values.
They will be particularly useful in conjunction
with the forwarding mechanism described in \secref{sec:forward}.

For example, it may be useful to have a flag |\version|
which can be set to |draft| or |final|.
The document source will contain some conditional code
depending on the value of |\version|.
Suppose further, the flag should default to |final| for the main file
and to |draft| for child files
which is a natural assignment for editing the document.
This is achieved by placing the following code
in the preamble of the main document
(below the |\childdocmain| directive):
%
\begin{center}
\begin{tabular}{l}
|\ifchilddoc|\\
|\providecommand{\version}{draft}|\\
|\||else|\\
|\providecommand{\version}{final}|\\
|\||fi|
\end{tabular}
\end{center}
%
The definition by |\providecommand| makes sure
that previous definitions are not overwritten.
Further statements |\providecommand{\version}{...}|
can thus be added before the above code to override it.

For the main file, one might add a line
(between |\childdocmain| and the above block)
%
\begin{center}
|%\ifchilddoc\||else\providecommand{\version}{draft}\||fi|
\end{center}
%
which can be uncommented to produce a draft version.
Likewise one can add a line to the very top of a child file
(above the |\childdocof{|\textit{main}|}| directive)
%
\begin{center}
|%\providecommand{\version}{final}|
\end{center}
%
which can be uncommented to produce the final version of this child document.

%%%%%%%%%%%%%%%%%%%%%%%%%%%%%%%%%%%%%%%%%%%%%%%%%%%%%%%%%%%%%%%%%%%%%%%%%%%%%%%%
\subsection{Forwarding}
\label{sec:forward}

Different versions of the main or child documents
using compilation flags as described in \secref{sec:flags}
can be (permanently) stored in different files
for convenient compilation, viewing and distribution.
To this end, the package defines a command
to pass on compilation to a different file:

%%%%%%%%%%%%%%%%%%%%%%%%%%%%%%%%%%%%%%%%
\DescribeMacro{\childdocforward}
The command |\childdocforward| redirects processing to
another source file:
%
\begin{center}
\begin{tabular}{l}
|\input{childdoc.def}|\\
|\childdocforward[|\textit{main}|]{|\textit{dest}|}|\\
\end{tabular}
\end{center}
%
The argument \textit{dest} is the destination file
(without extension).
It should be the main file or one of the child files.
Note that further \textsf{childdoc} directives
such as |\childdocof| and |\childdocforward|
in the indicated file will be processed in this form.
The optional argument \textit{main}
passes on directly to the main file \textit{main}
while pretending to compile the child \textit{dest}.
This form behaves as if \textit{dest}
issues |\childdocof{|\textit{main}|}| right away,
and no further \textsf{childdoc} directives will be processed.

%%%%%%%%%%%%%%%%%%%%%%%%%%%%%%%%%%%%%%%%
\DescribeMacro{\...prefix}
In the alternative form |\childdocforwardprefix|,
%
\begin{center}
\begin{tabular}{l}
|\input{childdoc.def}|\\
|\childdocforwardprefix[|\textit{main}|]{|\textit{prefix}|}{|\textit{dest}|}|
\end{tabular}
\end{center}
%
the destination file is determined by a pattern
depending on the current file:
To make this work, the current file must be called
`{\textit{prefix}\hspace{0.2em}\textit{suffix}}'
with \textit{prefix} matching precisely the argument.
Processing is then passed on to the file
`{\textit{dest}\hspace{0.2em}\textit{suffix}}'.
Surely, the same effect is achieved by
directly specifying the
argument `{\textit{dest}\hspace{0.2em}\textit{suffix}}'
in the first form.
However, that requires to set up a different file
for each child. With the alternative form of the command
all these files can have exactly the same content
which simplifies setting them up and maintaining them.

For example, the following file |draft.tex|
with a compilation flag |\version| as described in \secref{sec:flags}
compiles the main document as a draft:
%
\begin{center}
\begin{tabular}{l}
|\def\version{draft}|\\
|\input{childdoc.def}|\\
|\childdocforward{|\textit{main}|}|
\end{tabular}
\end{center}
%
Likewise, the following files |final|\textit{nn}|.tex|
compile the final version of the child document
|child|\textit{nn}|.tex|:
%
\begin{center}
\begin{tabular}{l}
|\def\version{final}|\\
|\input{childdoc.def}|\\
|\childdocforwardprefix{final}{child}|
\end{tabular}
\end{center}
%

Note that when several versions of a main file and/or of each child file
are to be generated, it may be convenient to set up a |Makefile| or
shell script to automatise the process.

%%%%%%%%%%%%%%%%%%%%%%%%%%%%%%%%%%%%%%%%%%%%%%%%%%%%%%%%%%%%%%%%%%%%%%%%%%%%%%%%
\subsection{Command Line Processing}
\label{sec:commandline}

The effect of redirection files can also be achieved by invoking
the \LaTeX{} compiler with a more elaborate command line.
Most conveniently this should be done as part
of a shell script or a |Makefile|.

When using \textsf{childdoc} in the main file, the following
command lines effectively perform a redirection
(note that depending on the shell being used,
backslashes may have to be doubled: `|\|' $\to$ `|\\|'):
%
\begin{center}
|... -jobname "|\textit{target}|" |\\|"|[\textit{flags}]%
|\input{childdoc.def}\childdocforward[|\textit{main}|]{|\textit{dest}|}"|
\end{center}
%
Here \textit{target} is the name of the output file,
\textit{main} is the name of the main file
and \textit{dest} is the name of the main or child file to be processed
(all filenames without extensions).
The optional argument \textit{main} can be omitted
if \textit{main} matches \textit{dest}.
Optionally, compilation \textit{flags} can be defined via |\def| commands.
This command line makes the \TeX{} engine believe
it is compiling the file \textit{target}
whose content is specified as the latter parameter.
The provided code then forwards the processing to
\textit{main} or \textit{dest} as described in \secref{sec:forward}.

%%%%%%%%%%%%%%%%%%%%%%%%%%%%%%%%%%%%%%%%%%%%%%%%%%%%%%%%%%%%%%%%%%%%%%%%%%%%%%%%
\subsection{Include by Input}
\label{sec:input}

Including child documents by |\include| has some restrictions by design.
Most notably, the content of a child document always occupies
its own set of pages; pages cannot be shared between child documents.
Usually, this behaviour makes perfect sense
because each child document contain an essential part of the document.
However, in some situations it may be desirable to compose
a document from a collection of parts
without having mandatory page breaks between then.
For this case, the package
provides a mechanism to include parts
by |\input| which can also be processed individually.
However, by construction this mechanism
requires manual handling of the content to be output.

%%%%%%%%%%%%%%%%%%%%%%%%%%%%%%%%%%%%%%%%
\DescribeMacro{\ifchilddocmanual}
The main file should be prepared as usual, see \secref{sec:include}.
However, the document body must make a distinction
between processing of an individual part and of the main document, e.g.:
%
\begin{center}
\begin{tabular}{l}
|\ifchilddocmanual|\\
|\input{\childdocname}|\\
|\||else|\\
\textit{document body with }|\input{|\textit{part}|}|\\
|\||fi|
\end{tabular}
\end{center}
%
The conditional |\ifchilddocmanual| is true whenever
a part to be included by |\input| is being compiled,
and the name of the part is stored in |\childdocname|.

%%%%%%%%%%%%%%%%%%%%%%%%%%%%%%%%%%%%%%%%
\DescribeMacro{\childdocby}
Each part to be included by |\input| should start with:
%
\begin{center}
\begin{tabular}{l}
|\input{childdoc.def}|\\
|\childdocby{|\textit{main}|}|\\
\end{tabular}
\end{center}
%
The directive |\childdocby| is similar to |\childdocof|
described in \secref{sec:include},
but the subsequent selection of content must be done manually.
To that end, both |\ifchilddoc| and |\ifchilddocmanual|
will be true upon processing of a part,
and the name of the part is stored in |\childdocname|.
Note that |\jobname| will be set to the filename of the current part
so that each part receives an individual |.aux| file
that does not interfere with the |.aux| file(s) of the main document.
This behaviour can be altered by the alternative form
|\childdocby[*]{|\textit{main}|}| (with a non-empty optional argument)
which uses the |.aux| file of the main document
by setting |\jobname| to \textit{main}.

%%%%%%%%%%%%%%%%%%%%%%%%%%%%%%%%%%%%%%%%%%%%%%%%%%%%%%%%%%%%%%%%%%%%%%%%%%%%%%%%
\subsection{Driver Development}
\label{sec:driver}

The \textsf{childdoc} mechanism can also be use for the development
of definition files such as \LaTeX{} styles or classes.
This case differs from the above setup with multiple parts
included by |\include| in that no |\includeonly| should be invoked.
This can be achieved by starting the include file
(before |\ProvidesPackage|) with:
%
\begin{center}
\begin{tabular}{l}
|\input{childdoc.def}|\\
|\childdocforward{|\textit{main}|}|\\
\end{tabular}
\end{center}
%
or alternatively with:
%
\begin{center}
\begin{tabular}{l}
|\input{childdoc.def}|\\
|\childdocby{|\textit{main}|}|\\
\end{tabular}
\end{center}
%
Both forms have slightly different effects as described above.
The main file is prepared as usual, see \secref{sec:include}.

%%%%%%%%%%%%%%%%%%%%%%%%%%%%%%%%%%%%%%%%%%%%%%%%%%%%%%%%%%%%%%%%%%%%%%%%%%%%%%%%
\subsection{Legacy Detection}
\label{sec:detection}

The directive |\childdocmain| in the main file can detect
whether the complete document or merely a child is to be compiled
even without using the directive |\childdocof|.
This method is deprecated because it is less robust
and there is no compelling reason to use it;
it is merely provided for backward compatibility
and it may be removed in future versions.

If the detection mechanism is to be used,
it is mandatory to correctly specify
the filename of the main file as the argument of |\childdocmain|:
%
\begin{center}
\begin{tabular}{l}
|\input{childdoc.def}|\\
|\childdocmain{|\textit{main}|}|\\
\end{tabular}
\end{center}
%
If |\jobname| does not match the argument \textit{main} of |\childdocmain|,
it is assumed that |\jobname| points to the child file to be compiled.
When using |\childdocmain| with the main file specified as argument,
it suffices to start a child file
with just |\input{|\textit{main}|}|
without loading of the package and using |\childdocof|.
If instead all processing is done
with the appropriate \textsf{childdoc} directives,
the argument of \textit{main} of |\childdocmain| can be empty.

An alternative version of the command line processing described
in \secref{sec:commandline} using the detection mechanism reads:
%
\begin{center}
|... -jobname "|\textit{target}|" "|[\textit{flags}]%
[|\def\jobname{|\textit{dest}|}|]|\input{|\textit{main}|}"|
\end{center}

%%%%%%%%%%%%%%%%%%%%%%%%%%%%%%%%%%%%%%%%%%%%%%%%%%%%%%%%%%%%%%%%%%%%%%%%%%%%%%%%
\subsection{Manual Code}
\label{sec:manual}

In case one cannot be certain whether the definitions file |childdoc.def|
is installed on the target \TeX{} distribution
and one prefers not to ship it,
it is conceivable to paste a few relevant commands into the sources.

To that end, drop all statements |\input{childdoc.def}|
and perform the replacements as outlined below.
Instead of |\childdocmain{|\textit{main}|}| add the following code
to the top of the main file:
%
\begin{center}
\begin{tabular}{l}
|\||ifdefined\childdocname\endinput\||fi\newif\ifchilddoc|\\
|\edef\childdocname{\scantokens\expandafter{\jobname\noexpand}}|\\
|\def\childdocmain{|\textit{main}|}\||ifx\childdocmain\childdocname\||else|\\
|\childdoctrue\includeonly{\childdocname}\let\jobname\childdocmain\||fi|\\
\end{tabular}
\end{center}
%
Instead of |\childdocof{|\textit{main}|}| just include the main file
at the top of each child file:
%
\begin{center}
|\input{|\textit{main}|}|
\end{center}
%
A simple redirection |\childdocforward{|\textit{dest}|}| is achieved by:
%
\begin{center}
|\def\jobname{|\textit{dest}|}\input{\jobname}|
\end{center}
%
The redirection with prefix
|\childdocforwardprefix[|\textit{prefix}|]{|\textit{dest}|}|
is accomplished by:
%
\begin{center}
\begin{tabular}{l}
|{\edef\jobname{\scantokens\expandafter{\jobname\noexpand}}|\\
|\def\redirectjob |\textit{prefix}|#1~~~{\gdef\jobname{|\textit{dest}|#1}}|\\
|\expandafter\redirectjob\jobname~~~}\input{\jobname}|
\end{tabular}
\end{center}

In an alternative approach,
child documents can be compiled by a specific command line
without additional code or specific definitions:
%
\begin{center}
|... -jobname "|\textit{target}|" "|[\textit{flags}]%
|\includeonly{|\textit{dest}|}\input{|\textit{main}|}"|
\end{center}
%

%%%%%%%%%%%%%%%%%%%%%%%%%%%%%%%%%%%%%%%%%%%%%%%%%%%%%%%%%%%%%%%%%%%%%%%%%%%%%%%%
%%%%%%%%%%%%%%%%%%%%%%%%%%%%%%%%%%%%%%%%%%%%%%%%%%%%%%%%%%%%%%%%%%%%%%%%%%%%%%%%
\section{Information}

%%%%%%%%%%%%%%%%%%%%%%%%%%%%%%%%%%%%%%%%%%%%%%%%%%%%%%%%%%%%%%%%%%%%%%%%%%%%%%%%
\subsection{Copyright}

Copyright \copyright{} 2017--2018 Niklas Beisert

This work may be distributed and/or modified under the
conditions of the \LaTeX{} Project Public License, either version 1.3
of this license or (at your option) any later version.
The latest version of this license is in
  \url{http://www.latex-project.org/lppl.txt}
and version 1.3 or later is part of all distributions of \LaTeX{}
version 2005/12/01 or later.

This work has the LPPL maintenance status `maintained'.

The Current Maintainer of this work is Niklas Beisert.

This work consists of the files |README.txt|, |childdoc.ins| and |childdoc.dtx|
as well as the derived files |childdoc.def|, |cdocsamp.tex|
with |cdocsch1.tex|, |cdocsch2.tex|, |cdocspt3.tex|, |cdocspt4.tex|,
|cdocsdrf.tex|, |cdocsfn1.tex|, |cdocsfn2.tex|
as well as |childdoc.pdf|.

%%%%%%%%%%%%%%%%%%%%%%%%%%%%%%%%%%%%%%%%%%%%%%%%%%%%%%%%%%%%%%%%%%%%%%%%%%%%%%%%
\subsection{Files and Installation}

The package consists of the files:
%
\begin{center}
\begin{tabular}{ll}
    |README.txt|   & readme file \\
    |childdoc.ins| & installation file \\
    |childdoc.dtx| & source file \\
    |childdoc.def| & definition file \\
    |cdocsamp.tex| & sample main file \\
    |cdocsch1.tex| & sample include file \\
    |cdocsch2.tex| & sample include file \\
    |cdocspt3.tex| & sample part file \\
    |cdocspt4.tex| & sample part file \\
    |cdocsdrf.tex| & sample redirection file \\
    |cdocsfn1.tex| & sample redirection file \\
    |cdocsfn2.tex| & sample redirection file \\
    |childdoc.pdf| & manual
\end{tabular}
\end{center}
%
The distribution consists of the files
|README.txt|, |childdoc.ins| and |childdoc.dtx|.
%
\begin{itemize}
\item
Run (pdf)\LaTeX{} on |childdoc.dtx|
to compile the manual |childdoc.pdf| (this file).
\item
Run \LaTeX{} on |childdoc.ins| to create the definitions file |childdoc.def|
and the sample |cdocsamp.tex| with include files
|cdocsch1.tex|, |cdocsch2.tex|, |cdocspt3.tex|, |cdocspt4.tex|,
|cdocsdrf.tex|, |cdocsfn1.tex|, |cdocsfn2.tex|.
Then copy the file |childdoc.def| to an appropriate directory of your \LaTeX{}
distribution, e.g.\ \textit{texmf-root}|/tex/latex/childdoc|.
\end{itemize}

%%%%%%%%%%%%%%%%%%%%%%%%%%%%%%%%%%%%%%%%%%%%%%%%%%%%%%%%%%%%%%%%%%%%%%%%%%%%%%%%
\subsection{Related CTAN Packages}

There are several other packages which offer a similar functionality:
%
\begin{itemize}
\item
The packages
\href{http://ctan.org/pkg/docmute}{\textsf{docmute}},
\href{http://ctan.org/pkg/includex}{\textsf{includex}} and
\href{http://ctan.org/pkg/standalone}{\textsf{standalone}}
provide commands to include only the document body of
a child file thus allowing both files to be compiled individually.
\item
The packages \href{http://ctan.org/pkg/subdocs}{\textsf{subdocs}}
and \href{http://ctan.org/pkg/subfiles}{\textsf{subfiles}}
provide structures in which the main and child documents can be
encapsulated and allowing them to be compiled individually.
The inclusion mechanism is different from the conventional |\include|.
\item
The package \href{http://ctan.org/pkg/combine}{\textsf{combine}}
is an elaborate solution to combine several documents into one.
\end{itemize}
%
See also the CTAN topic \href{http://ctan.org/topic/subdocs}{\textsf{subdocs}}
for further related packages.
The present package differs from the above solutions in that
a document structure constructed with the conventional |\include| mechanism
just needs two extra commands at the top of every file
such that all constituent files can be compiled individually.

%%%%%%%%%%%%%%%%%%%%%%%%%%%%%%%%%%%%%%%%%%%%%%%%%%%%%%%%%%%%%%%%%%%%%%%%%%%%%%%%
%\subsection{Feature Suggestions}
%
%The following is a list of features which may be useful for future
%versions of this package:
%%
%\begin{itemize}
%\item
%\ldots
%\end{itemize}

%%%%%%%%%%%%%%%%%%%%%%%%%%%%%%%%%%%%%%%%%%%%%%%%%%%%%%%%%%%%%%%%%%%%%%%%%%%%%%%%
\subsection{Revision History}

%%%%%%%%%%%%%%%%%%%%%%%%%%%%%%%%%%%%%%%%
\paragraph{v2.0:} 2018/12/30

\begin{itemize}
\item
immediate forward processing
\item
added |\childdocby| mechanism
\item
manual restructured
\end{itemize}

%%%%%%%%%%%%%%%%%%%%%%%%%%%%%%%%%%%%%%%%
\paragraph{v1.6:} 2018/01/17

\begin{itemize}
\item
application for development of include files
\item
corrections to manual
\end{itemize}

%%%%%%%%%%%%%%%%%%%%%%%%%%%%%%%%%%%%%%%%
\paragraph{v1.5:} 2017/05/21

\begin{itemize}
\item
more complete structuring introduced
\item
|\childdocof| introduced
\item
|\childdoc| renamed to |\childdocmain|
\item
|\childredirect| renamed to |\childdocforward| and |\childdocforwardprefix|
and functionality expanded
\end{itemize}

%%%%%%%%%%%%%%%%%%%%%%%%%%%%%%%%%%%%%%%%
\paragraph{v1.0:} 2017/04/27

\begin{itemize}
\item
manual and install package
\item
first version published on CTAN
\end{itemize}

%%%%%%%%%%%%%%%%%%%%%%%%%%%%%%%%%%%%%%%%
\paragraph{v0.6:} 2017/04/26

\begin{itemize}
\item
redirection mechanism added
\end{itemize}

%%%%%%%%%%%%%%%%%%%%%%%%%%%%%%%%%%%%%%%%
\paragraph{v0.5:} 2017/04/26

\begin{itemize}
\item
functionality in definition file
\end{itemize}


%%%%%%%%%%%%%%%%%%%%%%%%%%%%%%%%%%%%%%%%%%%%%%%%%%%%%%%%%%%%%%%%%%%%%%%%%%%%%%%%
%%%%%%%%%%%%%%%%%%%%%%%%%%%%%%%%%%%%%%%%%%%%%%%%%%%%%%%%%%%%%%%%%%%%%%%%%%%%%%%%
%%%%%%%%%%%%%%%%%%%%%%%%%%%%%%%%%%%%%%%%%%%%%%%%%%%%%%%%%%%%%%%%%%%%%%%%%%%%%%%%
\appendix

\settowidth\MacroIndent{\rmfamily\scriptsize 000\ }

 \DocInput{childdoc.dtx}

\end{document}
%</driver>
% \fi
%
% %%%%%%%%%%%%%%%%%%%%%%%%%%%%%%%%%%%%%%%%%%%%%%%%%%%%%%%%%%%%%%%%%%%%%%%%%%%%%%
% %%%%%%%%%%%%%%%%%%%%%%%%%%%%%%%%%%%%%%%%%%%%%%%%%%%%%%%%%%%%%%%%%%%%%%%%%%%%%%
% \section{Sample}
%\iffalse
%<*samplemain>
%\fi
%
% The following presents a sample document
% with two chapters, two parts, a title page,
% a compile flag as well as three forwarding files to set the flag.
% It consists of eight |.tex| files:
% \begin{center}
% \begin{tabular}{ll}
% |cdocsamp.tex|&main file\\
% |cdocsch1.tex|&include file for chapter 1\\
% |cdocsch2.tex|&include file for chapter 2\\
% |cdocspt3.tex|&include file for part 3\\
% |cdocspt4.tex|&include file for part 4\\
% |cdocsdrf.tex|&forwarding file for main file in draft mode\\
% |cdocsfi1.tex|&forwarding file for final version of chapter 1\\
% |cdocsfi2.tex|&forwarding file for final version of chapter 2\\
% \end{tabular}
% \end{center}
% Each of the eight files can be compiled directly by the \LaTeX{} compiler.
%
% %%%%%%%%%%%%%%%%%%%%%%%%%%%%%%%%%%%%%%
% \paragraph{Main File.}
%
% The main file is called |cdocsamp.tex|.
%
% Load the \textsf{childdoc} definitions and
% declare the filename for the main document:
%    \begin{macrocode}
\input{childdoc.def}
\childdocmain{}
%    \end{macrocode}

% Optional override for |\version| flag:
%    \begin{macrocode}
%%\ifchilddoc\else\providecommand{\version}{draft}\fi
%    \end{macrocode}

% Define the default values for the |\version| flag
% (|final| for the main file and |draft| for childs):
%    \begin{macrocode}
\ifchilddoc
\providecommand{\version}{draft}
\else
\providecommand{\version}{final}
\fi
%    \end{macrocode}

% Load the standard document class:
%    \begin{macrocode}
\documentclass[12pt]{article}
%    \end{macrocode}

% Start the document body:
%    \begin{macrocode}
\begin{document}
%    \end{macrocode}

% Declare a title page.
% Print title, part of document being processed and version flag:
%    \begin{macrocode}
\addtocounter{page}{-1}
\begin{center}
{\LARGE\bfseries{}childdoc example\par}
\vspace{1cm}
\ifchilddoc
\ifchilddocmanual part\else chapter\fi:
`\childdocname' of `\childdocjob'\par
\else
main document: `\childdocjob'\par
\fi
version: \version\par
\end{center}
\newpage
%    \end{macrocode}

% Manually include selected file,
% otherwise process as usual:
%    \begin{macrocode}
\ifchilddocmanual
\section*{part `\childdocname'}
\input{\childdocname}
\else
%    \end{macrocode}

% Include the two chapters:
%    \begin{macrocode}
\include{cdocsch1}
\include{cdocsch2}
%    \end{macrocode}

% Include the two parts unless only chapters should be displayed:
%    \begin{macrocode}
\ifchilddoc\else
\section{part three}
\input{cdocspt3}
\section{part four}
\input{cdocspt4}
\fi
%    \end{macrocode}

% Process as usual until here:
%    \begin{macrocode}
\fi
%    \end{macrocode}

% End of document body:
%    \begin{macrocode}
\end{document}
%    \end{macrocode}
%\iffalse
%</samplemain>
%\fi
%
% %%%%%%%%%%%%%%%%%%%%%%%%%%%%%%%%%%%%%%
% \paragraph{Chapter Include Files.}
%
% The include files are called |cdocsch1.tex| and |cdocsch2.tex|.
%
%\iffalse
%<*samplechap1|samplechap2>
%\fi

% Optional override for |\version| flag:
%    \begin{macrocode}
%%\providecommand{\version}{final}
%    \end{macrocode}

% Include the main document:
%    \begin{macrocode}
\input{childdoc.def}
\childdocof{cdocsamp}
%    \end{macrocode}

%\iffalse
%</samplechap1|samplechap2>
%\fi
%
%\iffalse
%<*samplechap1>
%\fi
% Some text for chapter 1:
%    \begin{macrocode}
\section{one}
some text in chapter one
%    \end{macrocode}

%\iffalse
%</samplechap1>
%\fi
% Some text for chapter 2:
%\iffalse
%<*samplechap2>
%\fi
%    \begin{macrocode}
\section{two}
more text in chapter two
%    \end{macrocode}

%\iffalse
%</samplechap2>
%\fi
%
% %%%%%%%%%%%%%%%%%%%%%%%%%%%%%%%%%%%%%%
% \paragraph{Part Include Files.}
%
% The include files are called |cdocspt3.tex| and |cdocspt4.tex|.
%
%\iffalse
%<*samplepart3|samplepart4>
%\fi

% Optional override for |\version| flag:
%    \begin{macrocode}
%%\providecommand{\version}{final}
%    \end{macrocode}

% Include the main document:
%    \begin{macrocode}
\input{childdoc.def}
\childdocby{cdocsamp}
%    \end{macrocode}

%\iffalse
%</samplepart3|samplepart4>
%\fi
%
%\iffalse
%<*samplepart3>
%\fi
% Some text for part 3:
%    \begin{macrocode}
some text in part three
%    \end{macrocode}

%\iffalse
%</samplepart3>
%\fi
% Some text for part 4:
%\iffalse
%<*samplepart4>
%\fi
%    \begin{macrocode}
more text in part four
%    \end{macrocode}

%\iffalse
%</samplepart4>
%\fi
%
% %%%%%%%%%%%%%%%%%%%%%%%%%%%%%%%%%%%%%%
% \paragraph{Forwarding for a Complete Draft.}
%
% The following forwarding file |cdocsdrf.tex|
% compiles the main document in draft mode:
%\iffalse
%<*sampledraft>
%\fi
%    \begin{macrocode}
\def\version{draft}
\input{childdoc.def}
\childdocforward{cdocsamp}
%    \end{macrocode}

%\iffalse
%</sampledraft>
%\fi
%
% %%%%%%%%%%%%%%%%%%%%%%%%%%%%%%%%%%%%%%
% \paragraph{Forwarding for Final Version of the Chapters.}
%
% The following forwarding files |cdocsfn1.tex| and |cdocsfn2.tex|
% (with identical content)
% compile the final versions of the child documents
% |cdocsch1.tex| and |cdocsch2.tex|, respectively:
%\iffalse
%<*samplefinal>
%\fi
%    \begin{macrocode}
\def\version{final}
\input{childdoc.def}
\childdocforwardprefix[cdocsamp]{cdocsfn}{cdocsch}
%    \end{macrocode}

%\iffalse
%</samplefinal>
%\fi
%
% %%%%%%%%%%%%%%%%%%%%%%%%%%%%%%%%%%%%%%
% \paragraph{Command Line Processing.}
%
% The following three command lines generate the output files
% |cdocscld|, |cdocscl1| and |cdocscl2|
% which should be identical to
% |cdocsdrf|, |cdocsch1| and |cdocsfn2|, respectively:
% \begin{center}
% \begin{tabular}{l}
% |latex -jobname cdocscld \|\\
% |  "\def\version{draft}\input{childdoc.def}\childdocforward{cdocsamp}"|\\
% |latex -jobname cdocscl1 \|\\
% |  "\input{childdoc.def}\childdocforward[cdocsamp]{cdocsch1}"|\\
% |latex -jobname cdocscl2 \|\\
% |  "\def\version{final}\input{childdoc.def}\childdocforward{cdocsch2}"|
% \end{tabular}
% \end{center}
% Note that the trailing backslash on each first line
% merely continues the input to the second line
% (for convenient cut ant paste).
% Furthermore, the command |latex| can be replaced by any
% of its alternative versions such as |pdflatex|.
%
% %%%%%%%%%%%%%%%%%%%%%%%%%%%%%%%%%%%%%%%%%%%%%%%%%%%%%%%%%%%%%%%%%%%%%%%%%%%%%%
% %%%%%%%%%%%%%%%%%%%%%%%%%%%%%%%%%%%%%%%%%%%%%%%%%%%%%%%%%%%%%%%%%%%%%%%%%%%%%%
% \section{Implementation}
%\iffalse
%<*package>
%\fi
%
% This section describes the definitions file |childdoc.def|.

% The definitions cannot be loaded using |\usepackage| or |\RequirePackage|
% which has a mechanism to prevent loading a style file more than once.
% When loading the definitions by means of |\input|
% multiple instances have to be prevented manually:
%\iffalse
%This code needs to be before the `\ProvidesFile' directive
%which is defined at the beginning of this file.
%Therefore it is also placed there and commented out here.
%</package>
%<*discard>
%\fi
%    \begin{macrocode}
\ifdefined\childdocmain\endinput\fi
%    \end{macrocode}
%\iffalse
%</discard>
%<*package>
%\fi
%
% \macro{\ifchilddoc}
% \macro{\ifchilddocmanual}
% The conditional |\ifchilddoc| tells whether a
% child (true) or main (false) document is being compiled.
% The conditional |\ifchilddocmanual| tells whether
% the |\includeonly| mechanism is used (false) or
% the selection of child files must be performed manually (true).
% The definitions initialise to false:
%    \begin{macrocode}
\newif\ifchilddoc
\newif\ifchilddocmanual
%    \end{macrocode}

% \macro{\childdocname}
% \macro{\childdocjob}
% The macro |\childdocname| stores the name of the main document
% to be compiled. The macro |\childdocjob| stores the name of
% the document on which the \LaTeX{} compiler was originally invoked.
% The content of |\jobname| cannot be compared
% to filenames specified in the source due to different catcodes.
% The following code rescans |\jobname|, stores the result
% in |\childdocname| and saves a copy in |\childdocjob|:
%    \begin{macrocode}
\edef\childdocname{\scantokens\expandafter{\jobname\noexpand}}
\let\childdocjob\childdocname
%    \end{macrocode}

% \macro{\childdocdisable}
% The macro |\childdocdisable| prevents the main file
% from being processed more than once.
% At this stage, the main document command |\childdocmain|
% is assumed to be called once again where it should do nothing.
% Any subsequent call to it should prevent
% a secondary processing of the main document
% It overwrites the forwarding commands
% |\childdocof| and |\childdocforward|
% with empty macros to prevent further inclusions of the main document:
%    \begin{macrocode}
\newcommand{\childdocdisable}
{
  \renewcommand{\childdocmain}[1]{\renewcommand{\childdocmain}[1]{\endinput}}
  \renewcommand{\childdocof}[1]{}
  \renewcommand{\childdocby}[2][]{}
  \renewcommand{\childdocforward}[2][]{}
  \renewcommand{\childdocdisable}{}
}
%    \end{macrocode}

% \macro{\childdocmain}
% The macro |\childdocmain| is to be called at the top of the main file
% with nothing or the main filename (without extension) as argument.
% First, it breaks loops.
% If the argument is not empty and does not match |\childdocname|
% (which is set by the first inclusion of |childdoc.def|),
% |\ifchilddoc| is set to true, |\includeonly| is applied to the child file
% and |\jobname| is set to the main file
% (for proper handling of |.aux| files):
%    \begin{macrocode}
\newcommand{\childdocmain}[1]
{
  \childdocdisable\childdocmain{}
  \if?#1?\else
    \begingroup
      \def\childdoctmp{#1}
      \ifx\childdoctmp\childdocname
        \def\childdoctmp{}
      \else
        \def\childdoctmp
        {
          \childdoctrue
          \includeonly{\childdocname}
          \def\childdocjob{#1}
          \def\jobname{#1}
        }
      \fi
      \expandafter
    \endgroup
    \childdoctmp
  \fi
}
%    \end{macrocode}

% \macro{\childdocof}
% The command |\childdocof| redirects
% compilation to the main file |#1|.
%    \begin{macrocode}
\newcommand{\childdocof}[1]
{
  \childdocdisable
  \childdoctrue
  \includeonly{\childdocname}
  \def\jobname{#1}
  \def\childdocjob{#1}
  \input{#1}
}
%    \end{macrocode}

% \macro{\childdocby}
% The command |\childdocby| ....
%    \begin{macrocode}
\newcommand{\childdocby}[2][]
{
  \childdocdisable
  \childdoctrue
  \childdocmanualtrue
  \if?#1?\else
    \def\jobname{#2}
  \fi
  \def\childdocjob{#2}
  \input{#2}
  \endinput
}
%    \end{macrocode}

% \macro{\childdocforward}
% The command |\childdocforward| redirects
% compilation to the main file or
% (if the optional argument is given) a child file.
% Parameters are set as if the main file
% or a child file starting with |\childdocof| was compiled.
% Then compilation is handed over to the main file:
%    \begin{macrocode}
\newcommand{\childdocforward}[2][]
{
  \begingroup
    \if?#1?
      \def\childdoctmp
      {
        \def\childdocname{#2}
        \def\childdocjob{#2}
        \def\jobname{#2}
        \input{#2}
        \endinput
      }
    \else
      \def\childdoctmp
      {
        \childdocdisable
        \def\childdocname{#2}
        \childdoctrue
        \includeonly{#2}
        \def\childdocjob{#1}
        \def\jobname{#1}
        \input{#1}
        \endinput
      }
    \fi
    \expandafter
  \endgroup
  \childdoctmp
}
%    \end{macrocode}

% \macro{\childdocforwardprefix}
% The command |\childdocforwardprefix| redirects
% compilation to the main or a child file by means of a pattern.
% The prefix |#1| in the current filename is replaced by |#2|
% and the suffix of the current filename is kept
% (it is assumed that the filename does not contain the substring `|~~~|'
% which is used as a delimiter).
% Compilation is handed over to the new file by |\childdocforward|:
%    \begin{macrocode}
\newcommand{\childdocforwardprefix}[3][]
{
  \begingroup
    \def\childdocextract #2##1~~~{\def\childdoctmp{\childdocforward[#1]{#3##1}}}
    \expandafter\childdocextract\childdocname~~~
    \expandafter
  \endgroup
  \childdoctmp
}
%    \end{macrocode}

% \macro{\childdoc}
% The deprecated macro |\childdoc| is a legacy version of |\childdocmain|:
%    \begin{macrocode}
\newcommand{\childdoc}{\childdocmain}
%    \end{macrocode}

% \macro{\childdocredirect}
% The deprecated macro |\childdocredirect| is a legacy version
% of |\childdocforward| and |\childdocforwardprefix|:
%    \begin{macrocode}
\newcommand{\childdocredirect}[2][]
{
  \begingroup
    \if?#1?
      \def\childdoctmp{\childdocforward{#2}}
    \else
      \def\childdoctmp{\childdocforwardprefix{#1}{#2}}
    \fi
    \expandafter
  \endgroup
  \childdoctmp
}
%    \end{macrocode}

%\iffalse
%</package>
%\fi
%
\endinput
|\\
|\childdocmain{|\textit{main}|}|\\
\end{tabular}
\end{center}
%
If |\jobname| does not match the argument \textit{main} of |\childdocmain|,
it is assumed that |\jobname| points to the child file to be compiled.
When using |\childdocmain| with the main file specified as argument,
it suffices to start a child file
with just |\input{|\textit{main}|}|
without loading of the package and using |\childdocof|.
If instead all processing is done
with the appropriate \textsf{childdoc} directives,
the argument of \textit{main} of |\childdocmain| can be empty.

An alternative version of the command line processing described
in \secref{sec:commandline} using the detection mechanism reads:
%
\begin{center}
|... -jobname "|\textit{target}|" "|[\textit{flags}]%
[|\def\jobname{|\textit{dest}|}|]|\input{|\textit{main}|}"|
\end{center}

%%%%%%%%%%%%%%%%%%%%%%%%%%%%%%%%%%%%%%%%%%%%%%%%%%%%%%%%%%%%%%%%%%%%%%%%%%%%%%%%
\subsection{Manual Code}
\label{sec:manual}

In case one cannot be certain whether the definitions file |childdoc.def|
is installed on the target \TeX{} distribution
and one prefers not to ship it,
it is conceivable to paste a few relevant commands into the sources.

To that end, drop all statements |% \iffalse
%
% childdoc.dtx Copyright (C) 2017-2018 Niklas Beisert
%
% This work may be distributed and/or modified under the
% conditions of the LaTeX Project Public License, either version 1.3
% of this license or (at your option) any later version.
% The latest version of this license is in
%   http://www.latex-project.org/lppl.txt
% and version 1.3 or later is part of all distributions of LaTeX
% version 2005/12/01 or later.
%
% This work has the LPPL maintenance status `maintained'.
%
% The Current Maintainer of this work is Niklas Beisert.
%
% This work consists of the files childdoc.dtx and childdoc.ins
% and the derived files childdoc.def and cdocsamp.tex with
% cdocsch1.tex, cdocsch2.tex, cdocsdrf.tex, cdocsfn1.tex, cdocsfn2.tex.
%
%<package>\ifdefined\childdocmain\endinput\fi
%<package>\ProvidesFile{childdoc.def}[2018/12/30 v2.0 child document driver]
%<samplemain>\ProvidesFile{cdocsamp.tex}[2018/12/30 v2.0 sample for childdoc]
%<*driver>
%\ProvidesFile{childdoc.drv}[2018/12/30 v2.0 childdoc reference manual file]
\PassOptionsToClass{10pt,a4paper}{article}
\documentclass{ltxdoc}

\usepackage[margin=35mm]{geometry}
\usepackage{hyperref}
\usepackage{hyperxmp}
\usepackage[usenames]{color}

\hypersetup{colorlinks=true}
\hypersetup{pdfstartview=FitH}
\hypersetup{pdfpagemode=UseNone}
\hypersetup{pdfsource={}}
\hypersetup{pdflang={en-UK}}
\hypersetup{pdfcopyright={Copyright 2017-2018 Niklas Beisert.
  This work may be distributed and/or modified under the
  conditions of the LaTeX Project Public License, either version 1.3
  of this license or (at your option) any later version.}}
\hypersetup{pdflicenseurl={http://www.latex-project.org/lppl.txt}}
\hypersetup{pdfcontactaddress={ETH Zurich, ITP, HIT K,
  Wolfgang-Pauli-Strasse 27}}
\hypersetup{pdfcontactpostcode={8093}}
\hypersetup{pdfcontactcity={Zurich}}
\hypersetup{pdfcontactcountry={Switzerland}}
\hypersetup{pdfcontactemail={nbeisert@itp.phys.ethz.ch}}
\hypersetup{pdfcontacturl={http://people.phys.ethz.ch/\xmptilde nbeisert/}}

\newcommand{\secref}[1]{\hyperref[#1]{section \ref*{#1}}}

\parskip1ex
\parindent0pt
\let\olditemize\itemize
\def\itemize{\olditemize\parskip0pt}

\begin{document}

\title{The \textsf{childdoc} Package}
\hypersetup{pdftitle={The childdoc Package}}
\author{Niklas Beisert\\[2ex]
  Institut f\"ur Theoretische Physik\\
  Eidgen\"ossische Technische Hochschule Z\"urich\\
  Wolfgang-Pauli-Strasse 27, 8093 Z\"urich, Switzerland\\[1ex]
  \href{mailto:nbeisert@itp.phys.ethz.ch}
  {\texttt{nbeisert@itp.phys.ethz.ch}}}
\hypersetup{pdfauthor={Niklas Beisert}}
\hypersetup{pdfsubject={Manual for the LaTeX2e Package childdoc}}
\date{30 December 2018, \textsf{v2.0}}
\maketitle

\begin{abstract}\noindent
\textsf{childdoc} is a \LaTeXe{} package
that enables the direct compilation
of document sections included by |\include|
to individual files.
\end{abstract}

\begingroup
\parskip0ex
\tableofcontents
\endgroup

%%%%%%%%%%%%%%%%%%%%%%%%%%%%%%%%%%%%%%%%%%%%%%%%%%%%%%%%%%%%%%%%%%%%%%%%%%%%%%%%
%%%%%%%%%%%%%%%%%%%%%%%%%%%%%%%%%%%%%%%%%%%%%%%%%%%%%%%%%%%%%%%%%%%%%%%%%%%%%%%%
\section{Introduction}

\LaTeX{} provides a mechanism to structure a large document (such as a book)
into a main file and several child files (containing the chapters)
using the |\include| command.
This mechanism is beneficial for documents
which span hundreds of pages in order to
make the source file(s) more manageable.
Moreover, compilation can be restricted to
selected child files by means of the |\includeonly| command.
The latter feature can be used to reduce the compilation time while editing
(this was significantly more useful in the earlier days of \LaTeX{})
or to generate a smaller document which is easier to navigate.
Another application of |\includeonly| is to generate
documents consisting of selected parts of the complete document.

However, there are a few drawbacks of the plain |\include| mechanism:
\begin{itemize}
\item
The child files cannot be compiled on their own,
they can only be compiled via the main file.
A naive editing environment
(such as a text editor with an option
to have the current file processed by \LaTeX)
may require one to switch to the main file before compiling;
attempting to compile the child file produces errors.
\item
The main file must be modified (each time)
to adjust the |\includeonly| command
to the present needs. This easily leaves the main file in a messy state.
\item
The generated document will always carry the filename
of the main document. This is inconvenient if
several child files are to be compiled and
to be kept for distribution.
\end{itemize}

The present package provides a simple interface
to make child files individually compilable by \LaTeX{}.
Compiling a child file then has the same effect as compiling
the main file with an |\includeonly| command
to select the appropriate child.
Moreover the generated document will carry the name of the child
rather than the main file.
This resolves all three above issues.

This feature is meant to make the editing of books,
thesis documents and lecture notes somewhat more convenient.
However, the package can also be used efficiently for
composing a series of documents (such as exercise sheets)
which are typically distributed individually.
It then assists the author in generating the individual documents
(potentially in different versions)
as well as a document containing the collected series.
Another application is in developing style files
or other kinds of included material
where compilation of the style file could redirect
to a sample or test file.

%%%%%%%%%%%%%%%%%%%%%%%%%%%%%%%%%%%%%%%%%%%%%%%%%%%%%%%%%%%%%%%%%%%%%%%%%%%%%%%%
%%%%%%%%%%%%%%%%%%%%%%%%%%%%%%%%%%%%%%%%%%%%%%%%%%%%%%%%%%%%%%%%%%%%%%%%%%%%%%%%
\section{Usage}

First of all, the package \textsf{childdoc} is \emph{not} a standard
\LaTeXe{} |.sty| style file! Therefore it needs to be invoked in
a non-standard way.

%%%%%%%%%%%%%%%%%%%%%%%%%%%%%%%%%%%%%%%%%%%%%%%%%%%%%%%%%%%%%%%%%%%%%%%%%%%%%%%%
\subsection{Included Files}
\label{sec:include}

%%%%%%%%%%%%%%%%%%%%%%%%%%%%%%%%%%%%%%%%
\DescribeMacro{\childdocmain}
To use the package, add the commands
\begin{center}
\begin{tabular}{l}
|\input{childdoc.def}|\\
|\childdocmain{}|\\
\end{tabular}
\end{center}
at the very top of the main \LaTeX{} file,
in particular \emph{before} the |\documentclass| statement!
The argument of |\childdocmain| should be left empty
(but it must be present).

%%%%%%%%%%%%%%%%%%%%%%%%%%%%%%%%%%%%%%%%
\DescribeMacro{\childdocof}
Furthermore, add the commands
\begin{center}
\begin{tabular}{l}
|\input{childdoc.def}|\\
|\childdocof{|\textit{main}|}|\\
\end{tabular}
\end{center}
at the top of every child file \textit{child}
which is included by |\include{|\textit{child}|}|
from within the main file
(or at least for those files to be compiled individually).
The argument \textit{main} must be the filename of the main file.

There are a couple of
considerations in setting up the main and child documents:

%%%%%%%%%%%%%%%%%%%%%%%%%%%%%%%%%%%%%%%%
\paragraph{Restrictions.}

Please note the following restrictions:
\begin{itemize}
\item
|\childdocmain| must be called with one argument \textit{main}
to ensure compatibility with earlier version of the package.
It must either be empty (|\childdocmain{}|)
or precisely match the filename of the main file in which it is specified.
See \secref{sec:detection} for further information.
\item
The filename \textit{main} must be specified without the |.tex| extension.
\item
The filename \textit{main} is case sensitive
(even in case-insensitive file systems)
due to internal string comparison.
\item
The argument \textit{main} should be fully expanded, it cannot be a macro.
\item
Subdirectories and special characters should be avoided in filenames.
\item
The command |\childdocmain{|\textit{main}|}| must be followed by a whitespace.
It should not be followed immediately by another command
or by a comment mark `|%|'.
This is because the \TeX{} parser reads the token immediately following
the argument of |\childdocmain| and puts it
at the beginning of every child section;
however, a white\-space is ignored.
\end{itemize}

%%%%%%%%%%%%%%%%%%%%%%%%%%%%%%%%%%%%%%%%
\paragraph{Content of Main File.}

It is advisable to place all content in the child files included by |\include|.
Any output contained in the main file will appear in all child documents
unless suppressed manually;
it cannot be suppressed automatically by the |\includeonly| directive
and thus should normally be avoided.
A method to include some content in the main file
by means of conditional processing is described in \secref{sec:conditional}.

%%%%%%%%%%%%%%%%%%%%%%%%%%%%%%%%%%%%%%%%
\paragraph{Page Numbering.}

When only a part of the document is compiled,
the appropriate numbering of pages
(as well as other status parameters)
is determined from the |.aux| files.
The latter contain information from previous passes.
However this information needs to propagate through
all intermediate child documents.
Therefore the page numbering in child documents may well
be inconsistent until the complete document is compiled at least once.

A useful (if unconventional) way to always ensure a consistent
page numbering is to restart the numbering in each child document
and denote the pages by `\textit{child}|.|\textit{page}'
where \textit{child} represents the chapter/section number of the child file.
This can be achieved by the command
|\numberwithin{page}{|\textit{child}|}|
of the \textsf{amsmath} package
where \textit{child} can be |chapter| or |section|
depending on the chosen structuring.
Alternatively, one can modify the macro |\thepage| appropriately
and reset the counter |page| at the start of each child file.

%%%%%%%%%%%%%%%%%%%%%%%%%%%%%%%%%%%%%%%%%%%%%%%%%%%%%%%%%%%%%%%%%%%%%%%%%%%%%%%%
\subsection{Conditional Processing}
\label{sec:conditional}

The package provides a mechanism to compile different versions
of a document. To customise the versions further some conditional processing
can come in handy to distinguish which version is being compiled.
The package provides two macros to describe the compilation context:

%%%%%%%%%%%%%%%%%%%%%%%%%%%%%%%%%%%%%%%%
\DescribeMacro{\ifchilddoc}
The conditional |\ifchilddoc| distinguishes between the compilation of
child documents and the main document:
%
\begin{center}
|\ifchilddoc |\textit{child-code}| |[|\||else |\textit{main-code}]| \||fi|
\end{center}

%%%%%%%%%%%%%%%%%%%%%%%%%%%%%%%%%%%%%%%%
\DescribeMacro{\childdocname}
\DescribeMacro{\childdocjob}
The macro |\childdocname| contains the filename (without extension)
of the main or child file being processed.
Note that |\childdocjob| will always contain the name of the main file.

%%%%%%%%%%%%%%%%%%%%%%%%%%%%%%%%%%%%%%%%
\paragraph{Title Page.}

Conditional processing can be used to include a title or banner page
in the main document when proper precautions are taken.
Importantly, the code in the main file should ensure that the page counter
(as well as other status parameters which are stored in the |.aux| files)
takes the same value after the conditional processing.
Otherwise the page numbers may take divergent values
depending on which part is compiled.

For example, a title page could be declared by:
%
\begin{center}
\begin{tabular}{l}
|\ifchilddoc\||else|\\
|\addtocounter{page}{-1}|\\
\textit{code for title page}\\
|\newpage|\\
|\||fi|
\end{tabular}
\end{center}
%
A banner page for the child documents can be generated by:
%
\begin{center}
\begin{tabular}{l}
|\ifchilddoc|\\
|\addtocounter{page}{-1}|\\
\textit{code for banner page}\\
|\newpage|\\
|\||fi|
\end{tabular}
\end{center}
%
Here one could write a message such as:
\begin{center}
|This is the part \childdocname{} of \childdocjob{}.|
\end{center}

%%%%%%%%%%%%%%%%%%%%%%%%%%%%%%%%%%%%%%%%%%%%%%%%%%%%%%%%%%%%%%%%%%%%%%%%%%%%%%%%
\subsection{Flags}
\label{sec:flags}

The package makes it easy to generate different versions
of the main or child documents.
To this end compilation flags can be defined
and assigned different default values.
They will be particularly useful in conjunction
with the forwarding mechanism described in \secref{sec:forward}.

For example, it may be useful to have a flag |\version|
which can be set to |draft| or |final|.
The document source will contain some conditional code
depending on the value of |\version|.
Suppose further, the flag should default to |final| for the main file
and to |draft| for child files
which is a natural assignment for editing the document.
This is achieved by placing the following code
in the preamble of the main document
(below the |\childdocmain| directive):
%
\begin{center}
\begin{tabular}{l}
|\ifchilddoc|\\
|\providecommand{\version}{draft}|\\
|\||else|\\
|\providecommand{\version}{final}|\\
|\||fi|
\end{tabular}
\end{center}
%
The definition by |\providecommand| makes sure
that previous definitions are not overwritten.
Further statements |\providecommand{\version}{...}|
can thus be added before the above code to override it.

For the main file, one might add a line
(between |\childdocmain| and the above block)
%
\begin{center}
|%\ifchilddoc\||else\providecommand{\version}{draft}\||fi|
\end{center}
%
which can be uncommented to produce a draft version.
Likewise one can add a line to the very top of a child file
(above the |\childdocof{|\textit{main}|}| directive)
%
\begin{center}
|%\providecommand{\version}{final}|
\end{center}
%
which can be uncommented to produce the final version of this child document.

%%%%%%%%%%%%%%%%%%%%%%%%%%%%%%%%%%%%%%%%%%%%%%%%%%%%%%%%%%%%%%%%%%%%%%%%%%%%%%%%
\subsection{Forwarding}
\label{sec:forward}

Different versions of the main or child documents
using compilation flags as described in \secref{sec:flags}
can be (permanently) stored in different files
for convenient compilation, viewing and distribution.
To this end, the package defines a command
to pass on compilation to a different file:

%%%%%%%%%%%%%%%%%%%%%%%%%%%%%%%%%%%%%%%%
\DescribeMacro{\childdocforward}
The command |\childdocforward| redirects processing to
another source file:
%
\begin{center}
\begin{tabular}{l}
|\input{childdoc.def}|\\
|\childdocforward[|\textit{main}|]{|\textit{dest}|}|\\
\end{tabular}
\end{center}
%
The argument \textit{dest} is the destination file
(without extension).
It should be the main file or one of the child files.
Note that further \textsf{childdoc} directives
such as |\childdocof| and |\childdocforward|
in the indicated file will be processed in this form.
The optional argument \textit{main}
passes on directly to the main file \textit{main}
while pretending to compile the child \textit{dest}.
This form behaves as if \textit{dest}
issues |\childdocof{|\textit{main}|}| right away,
and no further \textsf{childdoc} directives will be processed.

%%%%%%%%%%%%%%%%%%%%%%%%%%%%%%%%%%%%%%%%
\DescribeMacro{\...prefix}
In the alternative form |\childdocforwardprefix|,
%
\begin{center}
\begin{tabular}{l}
|\input{childdoc.def}|\\
|\childdocforwardprefix[|\textit{main}|]{|\textit{prefix}|}{|\textit{dest}|}|
\end{tabular}
\end{center}
%
the destination file is determined by a pattern
depending on the current file:
To make this work, the current file must be called
`{\textit{prefix}\hspace{0.2em}\textit{suffix}}'
with \textit{prefix} matching precisely the argument.
Processing is then passed on to the file
`{\textit{dest}\hspace{0.2em}\textit{suffix}}'.
Surely, the same effect is achieved by
directly specifying the
argument `{\textit{dest}\hspace{0.2em}\textit{suffix}}'
in the first form.
However, that requires to set up a different file
for each child. With the alternative form of the command
all these files can have exactly the same content
which simplifies setting them up and maintaining them.

For example, the following file |draft.tex|
with a compilation flag |\version| as described in \secref{sec:flags}
compiles the main document as a draft:
%
\begin{center}
\begin{tabular}{l}
|\def\version{draft}|\\
|\input{childdoc.def}|\\
|\childdocforward{|\textit{main}|}|
\end{tabular}
\end{center}
%
Likewise, the following files |final|\textit{nn}|.tex|
compile the final version of the child document
|child|\textit{nn}|.tex|:
%
\begin{center}
\begin{tabular}{l}
|\def\version{final}|\\
|\input{childdoc.def}|\\
|\childdocforwardprefix{final}{child}|
\end{tabular}
\end{center}
%

Note that when several versions of a main file and/or of each child file
are to be generated, it may be convenient to set up a |Makefile| or
shell script to automatise the process.

%%%%%%%%%%%%%%%%%%%%%%%%%%%%%%%%%%%%%%%%%%%%%%%%%%%%%%%%%%%%%%%%%%%%%%%%%%%%%%%%
\subsection{Command Line Processing}
\label{sec:commandline}

The effect of redirection files can also be achieved by invoking
the \LaTeX{} compiler with a more elaborate command line.
Most conveniently this should be done as part
of a shell script or a |Makefile|.

When using \textsf{childdoc} in the main file, the following
command lines effectively perform a redirection
(note that depending on the shell being used,
backslashes may have to be doubled: `|\|' $\to$ `|\\|'):
%
\begin{center}
|... -jobname "|\textit{target}|" |\\|"|[\textit{flags}]%
|\input{childdoc.def}\childdocforward[|\textit{main}|]{|\textit{dest}|}"|
\end{center}
%
Here \textit{target} is the name of the output file,
\textit{main} is the name of the main file
and \textit{dest} is the name of the main or child file to be processed
(all filenames without extensions).
The optional argument \textit{main} can be omitted
if \textit{main} matches \textit{dest}.
Optionally, compilation \textit{flags} can be defined via |\def| commands.
This command line makes the \TeX{} engine believe
it is compiling the file \textit{target}
whose content is specified as the latter parameter.
The provided code then forwards the processing to
\textit{main} or \textit{dest} as described in \secref{sec:forward}.

%%%%%%%%%%%%%%%%%%%%%%%%%%%%%%%%%%%%%%%%%%%%%%%%%%%%%%%%%%%%%%%%%%%%%%%%%%%%%%%%
\subsection{Include by Input}
\label{sec:input}

Including child documents by |\include| has some restrictions by design.
Most notably, the content of a child document always occupies
its own set of pages; pages cannot be shared between child documents.
Usually, this behaviour makes perfect sense
because each child document contain an essential part of the document.
However, in some situations it may be desirable to compose
a document from a collection of parts
without having mandatory page breaks between then.
For this case, the package
provides a mechanism to include parts
by |\input| which can also be processed individually.
However, by construction this mechanism
requires manual handling of the content to be output.

%%%%%%%%%%%%%%%%%%%%%%%%%%%%%%%%%%%%%%%%
\DescribeMacro{\ifchilddocmanual}
The main file should be prepared as usual, see \secref{sec:include}.
However, the document body must make a distinction
between processing of an individual part and of the main document, e.g.:
%
\begin{center}
\begin{tabular}{l}
|\ifchilddocmanual|\\
|\input{\childdocname}|\\
|\||else|\\
\textit{document body with }|\input{|\textit{part}|}|\\
|\||fi|
\end{tabular}
\end{center}
%
The conditional |\ifchilddocmanual| is true whenever
a part to be included by |\input| is being compiled,
and the name of the part is stored in |\childdocname|.

%%%%%%%%%%%%%%%%%%%%%%%%%%%%%%%%%%%%%%%%
\DescribeMacro{\childdocby}
Each part to be included by |\input| should start with:
%
\begin{center}
\begin{tabular}{l}
|\input{childdoc.def}|\\
|\childdocby{|\textit{main}|}|\\
\end{tabular}
\end{center}
%
The directive |\childdocby| is similar to |\childdocof|
described in \secref{sec:include},
but the subsequent selection of content must be done manually.
To that end, both |\ifchilddoc| and |\ifchilddocmanual|
will be true upon processing of a part,
and the name of the part is stored in |\childdocname|.
Note that |\jobname| will be set to the filename of the current part
so that each part receives an individual |.aux| file
that does not interfere with the |.aux| file(s) of the main document.
This behaviour can be altered by the alternative form
|\childdocby[*]{|\textit{main}|}| (with a non-empty optional argument)
which uses the |.aux| file of the main document
by setting |\jobname| to \textit{main}.

%%%%%%%%%%%%%%%%%%%%%%%%%%%%%%%%%%%%%%%%%%%%%%%%%%%%%%%%%%%%%%%%%%%%%%%%%%%%%%%%
\subsection{Driver Development}
\label{sec:driver}

The \textsf{childdoc} mechanism can also be use for the development
of definition files such as \LaTeX{} styles or classes.
This case differs from the above setup with multiple parts
included by |\include| in that no |\includeonly| should be invoked.
This can be achieved by starting the include file
(before |\ProvidesPackage|) with:
%
\begin{center}
\begin{tabular}{l}
|\input{childdoc.def}|\\
|\childdocforward{|\textit{main}|}|\\
\end{tabular}
\end{center}
%
or alternatively with:
%
\begin{center}
\begin{tabular}{l}
|\input{childdoc.def}|\\
|\childdocby{|\textit{main}|}|\\
\end{tabular}
\end{center}
%
Both forms have slightly different effects as described above.
The main file is prepared as usual, see \secref{sec:include}.

%%%%%%%%%%%%%%%%%%%%%%%%%%%%%%%%%%%%%%%%%%%%%%%%%%%%%%%%%%%%%%%%%%%%%%%%%%%%%%%%
\subsection{Legacy Detection}
\label{sec:detection}

The directive |\childdocmain| in the main file can detect
whether the complete document or merely a child is to be compiled
even without using the directive |\childdocof|.
This method is deprecated because it is less robust
and there is no compelling reason to use it;
it is merely provided for backward compatibility
and it may be removed in future versions.

If the detection mechanism is to be used,
it is mandatory to correctly specify
the filename of the main file as the argument of |\childdocmain|:
%
\begin{center}
\begin{tabular}{l}
|\input{childdoc.def}|\\
|\childdocmain{|\textit{main}|}|\\
\end{tabular}
\end{center}
%
If |\jobname| does not match the argument \textit{main} of |\childdocmain|,
it is assumed that |\jobname| points to the child file to be compiled.
When using |\childdocmain| with the main file specified as argument,
it suffices to start a child file
with just |\input{|\textit{main}|}|
without loading of the package and using |\childdocof|.
If instead all processing is done
with the appropriate \textsf{childdoc} directives,
the argument of \textit{main} of |\childdocmain| can be empty.

An alternative version of the command line processing described
in \secref{sec:commandline} using the detection mechanism reads:
%
\begin{center}
|... -jobname "|\textit{target}|" "|[\textit{flags}]%
[|\def\jobname{|\textit{dest}|}|]|\input{|\textit{main}|}"|
\end{center}

%%%%%%%%%%%%%%%%%%%%%%%%%%%%%%%%%%%%%%%%%%%%%%%%%%%%%%%%%%%%%%%%%%%%%%%%%%%%%%%%
\subsection{Manual Code}
\label{sec:manual}

In case one cannot be certain whether the definitions file |childdoc.def|
is installed on the target \TeX{} distribution
and one prefers not to ship it,
it is conceivable to paste a few relevant commands into the sources.

To that end, drop all statements |\input{childdoc.def}|
and perform the replacements as outlined below.
Instead of |\childdocmain{|\textit{main}|}| add the following code
to the top of the main file:
%
\begin{center}
\begin{tabular}{l}
|\||ifdefined\childdocname\endinput\||fi\newif\ifchilddoc|\\
|\edef\childdocname{\scantokens\expandafter{\jobname\noexpand}}|\\
|\def\childdocmain{|\textit{main}|}\||ifx\childdocmain\childdocname\||else|\\
|\childdoctrue\includeonly{\childdocname}\let\jobname\childdocmain\||fi|\\
\end{tabular}
\end{center}
%
Instead of |\childdocof{|\textit{main}|}| just include the main file
at the top of each child file:
%
\begin{center}
|\input{|\textit{main}|}|
\end{center}
%
A simple redirection |\childdocforward{|\textit{dest}|}| is achieved by:
%
\begin{center}
|\def\jobname{|\textit{dest}|}\input{\jobname}|
\end{center}
%
The redirection with prefix
|\childdocforwardprefix[|\textit{prefix}|]{|\textit{dest}|}|
is accomplished by:
%
\begin{center}
\begin{tabular}{l}
|{\edef\jobname{\scantokens\expandafter{\jobname\noexpand}}|\\
|\def\redirectjob |\textit{prefix}|#1~~~{\gdef\jobname{|\textit{dest}|#1}}|\\
|\expandafter\redirectjob\jobname~~~}\input{\jobname}|
\end{tabular}
\end{center}

In an alternative approach,
child documents can be compiled by a specific command line
without additional code or specific definitions:
%
\begin{center}
|... -jobname "|\textit{target}|" "|[\textit{flags}]%
|\includeonly{|\textit{dest}|}\input{|\textit{main}|}"|
\end{center}
%

%%%%%%%%%%%%%%%%%%%%%%%%%%%%%%%%%%%%%%%%%%%%%%%%%%%%%%%%%%%%%%%%%%%%%%%%%%%%%%%%
%%%%%%%%%%%%%%%%%%%%%%%%%%%%%%%%%%%%%%%%%%%%%%%%%%%%%%%%%%%%%%%%%%%%%%%%%%%%%%%%
\section{Information}

%%%%%%%%%%%%%%%%%%%%%%%%%%%%%%%%%%%%%%%%%%%%%%%%%%%%%%%%%%%%%%%%%%%%%%%%%%%%%%%%
\subsection{Copyright}

Copyright \copyright{} 2017--2018 Niklas Beisert

This work may be distributed and/or modified under the
conditions of the \LaTeX{} Project Public License, either version 1.3
of this license or (at your option) any later version.
The latest version of this license is in
  \url{http://www.latex-project.org/lppl.txt}
and version 1.3 or later is part of all distributions of \LaTeX{}
version 2005/12/01 or later.

This work has the LPPL maintenance status `maintained'.

The Current Maintainer of this work is Niklas Beisert.

This work consists of the files |README.txt|, |childdoc.ins| and |childdoc.dtx|
as well as the derived files |childdoc.def|, |cdocsamp.tex|
with |cdocsch1.tex|, |cdocsch2.tex|, |cdocspt3.tex|, |cdocspt4.tex|,
|cdocsdrf.tex|, |cdocsfn1.tex|, |cdocsfn2.tex|
as well as |childdoc.pdf|.

%%%%%%%%%%%%%%%%%%%%%%%%%%%%%%%%%%%%%%%%%%%%%%%%%%%%%%%%%%%%%%%%%%%%%%%%%%%%%%%%
\subsection{Files and Installation}

The package consists of the files:
%
\begin{center}
\begin{tabular}{ll}
    |README.txt|   & readme file \\
    |childdoc.ins| & installation file \\
    |childdoc.dtx| & source file \\
    |childdoc.def| & definition file \\
    |cdocsamp.tex| & sample main file \\
    |cdocsch1.tex| & sample include file \\
    |cdocsch2.tex| & sample include file \\
    |cdocspt3.tex| & sample part file \\
    |cdocspt4.tex| & sample part file \\
    |cdocsdrf.tex| & sample redirection file \\
    |cdocsfn1.tex| & sample redirection file \\
    |cdocsfn2.tex| & sample redirection file \\
    |childdoc.pdf| & manual
\end{tabular}
\end{center}
%
The distribution consists of the files
|README.txt|, |childdoc.ins| and |childdoc.dtx|.
%
\begin{itemize}
\item
Run (pdf)\LaTeX{} on |childdoc.dtx|
to compile the manual |childdoc.pdf| (this file).
\item
Run \LaTeX{} on |childdoc.ins| to create the definitions file |childdoc.def|
and the sample |cdocsamp.tex| with include files
|cdocsch1.tex|, |cdocsch2.tex|, |cdocspt3.tex|, |cdocspt4.tex|,
|cdocsdrf.tex|, |cdocsfn1.tex|, |cdocsfn2.tex|.
Then copy the file |childdoc.def| to an appropriate directory of your \LaTeX{}
distribution, e.g.\ \textit{texmf-root}|/tex/latex/childdoc|.
\end{itemize}

%%%%%%%%%%%%%%%%%%%%%%%%%%%%%%%%%%%%%%%%%%%%%%%%%%%%%%%%%%%%%%%%%%%%%%%%%%%%%%%%
\subsection{Related CTAN Packages}

There are several other packages which offer a similar functionality:
%
\begin{itemize}
\item
The packages
\href{http://ctan.org/pkg/docmute}{\textsf{docmute}},
\href{http://ctan.org/pkg/includex}{\textsf{includex}} and
\href{http://ctan.org/pkg/standalone}{\textsf{standalone}}
provide commands to include only the document body of
a child file thus allowing both files to be compiled individually.
\item
The packages \href{http://ctan.org/pkg/subdocs}{\textsf{subdocs}}
and \href{http://ctan.org/pkg/subfiles}{\textsf{subfiles}}
provide structures in which the main and child documents can be
encapsulated and allowing them to be compiled individually.
The inclusion mechanism is different from the conventional |\include|.
\item
The package \href{http://ctan.org/pkg/combine}{\textsf{combine}}
is an elaborate solution to combine several documents into one.
\end{itemize}
%
See also the CTAN topic \href{http://ctan.org/topic/subdocs}{\textsf{subdocs}}
for further related packages.
The present package differs from the above solutions in that
a document structure constructed with the conventional |\include| mechanism
just needs two extra commands at the top of every file
such that all constituent files can be compiled individually.

%%%%%%%%%%%%%%%%%%%%%%%%%%%%%%%%%%%%%%%%%%%%%%%%%%%%%%%%%%%%%%%%%%%%%%%%%%%%%%%%
%\subsection{Feature Suggestions}
%
%The following is a list of features which may be useful for future
%versions of this package:
%%
%\begin{itemize}
%\item
%\ldots
%\end{itemize}

%%%%%%%%%%%%%%%%%%%%%%%%%%%%%%%%%%%%%%%%%%%%%%%%%%%%%%%%%%%%%%%%%%%%%%%%%%%%%%%%
\subsection{Revision History}

%%%%%%%%%%%%%%%%%%%%%%%%%%%%%%%%%%%%%%%%
\paragraph{v2.0:} 2018/12/30

\begin{itemize}
\item
immediate forward processing
\item
added |\childdocby| mechanism
\item
manual restructured
\end{itemize}

%%%%%%%%%%%%%%%%%%%%%%%%%%%%%%%%%%%%%%%%
\paragraph{v1.6:} 2018/01/17

\begin{itemize}
\item
application for development of include files
\item
corrections to manual
\end{itemize}

%%%%%%%%%%%%%%%%%%%%%%%%%%%%%%%%%%%%%%%%
\paragraph{v1.5:} 2017/05/21

\begin{itemize}
\item
more complete structuring introduced
\item
|\childdocof| introduced
\item
|\childdoc| renamed to |\childdocmain|
\item
|\childredirect| renamed to |\childdocforward| and |\childdocforwardprefix|
and functionality expanded
\end{itemize}

%%%%%%%%%%%%%%%%%%%%%%%%%%%%%%%%%%%%%%%%
\paragraph{v1.0:} 2017/04/27

\begin{itemize}
\item
manual and install package
\item
first version published on CTAN
\end{itemize}

%%%%%%%%%%%%%%%%%%%%%%%%%%%%%%%%%%%%%%%%
\paragraph{v0.6:} 2017/04/26

\begin{itemize}
\item
redirection mechanism added
\end{itemize}

%%%%%%%%%%%%%%%%%%%%%%%%%%%%%%%%%%%%%%%%
\paragraph{v0.5:} 2017/04/26

\begin{itemize}
\item
functionality in definition file
\end{itemize}


%%%%%%%%%%%%%%%%%%%%%%%%%%%%%%%%%%%%%%%%%%%%%%%%%%%%%%%%%%%%%%%%%%%%%%%%%%%%%%%%
%%%%%%%%%%%%%%%%%%%%%%%%%%%%%%%%%%%%%%%%%%%%%%%%%%%%%%%%%%%%%%%%%%%%%%%%%%%%%%%%
%%%%%%%%%%%%%%%%%%%%%%%%%%%%%%%%%%%%%%%%%%%%%%%%%%%%%%%%%%%%%%%%%%%%%%%%%%%%%%%%
\appendix

\settowidth\MacroIndent{\rmfamily\scriptsize 000\ }

 \DocInput{childdoc.dtx}

\end{document}
%</driver>
% \fi
%
% %%%%%%%%%%%%%%%%%%%%%%%%%%%%%%%%%%%%%%%%%%%%%%%%%%%%%%%%%%%%%%%%%%%%%%%%%%%%%%
% %%%%%%%%%%%%%%%%%%%%%%%%%%%%%%%%%%%%%%%%%%%%%%%%%%%%%%%%%%%%%%%%%%%%%%%%%%%%%%
% \section{Sample}
%\iffalse
%<*samplemain>
%\fi
%
% The following presents a sample document
% with two chapters, two parts, a title page,
% a compile flag as well as three forwarding files to set the flag.
% It consists of eight |.tex| files:
% \begin{center}
% \begin{tabular}{ll}
% |cdocsamp.tex|&main file\\
% |cdocsch1.tex|&include file for chapter 1\\
% |cdocsch2.tex|&include file for chapter 2\\
% |cdocspt3.tex|&include file for part 3\\
% |cdocspt4.tex|&include file for part 4\\
% |cdocsdrf.tex|&forwarding file for main file in draft mode\\
% |cdocsfi1.tex|&forwarding file for final version of chapter 1\\
% |cdocsfi2.tex|&forwarding file for final version of chapter 2\\
% \end{tabular}
% \end{center}
% Each of the eight files can be compiled directly by the \LaTeX{} compiler.
%
% %%%%%%%%%%%%%%%%%%%%%%%%%%%%%%%%%%%%%%
% \paragraph{Main File.}
%
% The main file is called |cdocsamp.tex|.
%
% Load the \textsf{childdoc} definitions and
% declare the filename for the main document:
%    \begin{macrocode}
\input{childdoc.def}
\childdocmain{}
%    \end{macrocode}

% Optional override for |\version| flag:
%    \begin{macrocode}
%%\ifchilddoc\else\providecommand{\version}{draft}\fi
%    \end{macrocode}

% Define the default values for the |\version| flag
% (|final| for the main file and |draft| for childs):
%    \begin{macrocode}
\ifchilddoc
\providecommand{\version}{draft}
\else
\providecommand{\version}{final}
\fi
%    \end{macrocode}

% Load the standard document class:
%    \begin{macrocode}
\documentclass[12pt]{article}
%    \end{macrocode}

% Start the document body:
%    \begin{macrocode}
\begin{document}
%    \end{macrocode}

% Declare a title page.
% Print title, part of document being processed and version flag:
%    \begin{macrocode}
\addtocounter{page}{-1}
\begin{center}
{\LARGE\bfseries{}childdoc example\par}
\vspace{1cm}
\ifchilddoc
\ifchilddocmanual part\else chapter\fi:
`\childdocname' of `\childdocjob'\par
\else
main document: `\childdocjob'\par
\fi
version: \version\par
\end{center}
\newpage
%    \end{macrocode}

% Manually include selected file,
% otherwise process as usual:
%    \begin{macrocode}
\ifchilddocmanual
\section*{part `\childdocname'}
\input{\childdocname}
\else
%    \end{macrocode}

% Include the two chapters:
%    \begin{macrocode}
\include{cdocsch1}
\include{cdocsch2}
%    \end{macrocode}

% Include the two parts unless only chapters should be displayed:
%    \begin{macrocode}
\ifchilddoc\else
\section{part three}
\input{cdocspt3}
\section{part four}
\input{cdocspt4}
\fi
%    \end{macrocode}

% Process as usual until here:
%    \begin{macrocode}
\fi
%    \end{macrocode}

% End of document body:
%    \begin{macrocode}
\end{document}
%    \end{macrocode}
%\iffalse
%</samplemain>
%\fi
%
% %%%%%%%%%%%%%%%%%%%%%%%%%%%%%%%%%%%%%%
% \paragraph{Chapter Include Files.}
%
% The include files are called |cdocsch1.tex| and |cdocsch2.tex|.
%
%\iffalse
%<*samplechap1|samplechap2>
%\fi

% Optional override for |\version| flag:
%    \begin{macrocode}
%%\providecommand{\version}{final}
%    \end{macrocode}

% Include the main document:
%    \begin{macrocode}
\input{childdoc.def}
\childdocof{cdocsamp}
%    \end{macrocode}

%\iffalse
%</samplechap1|samplechap2>
%\fi
%
%\iffalse
%<*samplechap1>
%\fi
% Some text for chapter 1:
%    \begin{macrocode}
\section{one}
some text in chapter one
%    \end{macrocode}

%\iffalse
%</samplechap1>
%\fi
% Some text for chapter 2:
%\iffalse
%<*samplechap2>
%\fi
%    \begin{macrocode}
\section{two}
more text in chapter two
%    \end{macrocode}

%\iffalse
%</samplechap2>
%\fi
%
% %%%%%%%%%%%%%%%%%%%%%%%%%%%%%%%%%%%%%%
% \paragraph{Part Include Files.}
%
% The include files are called |cdocspt3.tex| and |cdocspt4.tex|.
%
%\iffalse
%<*samplepart3|samplepart4>
%\fi

% Optional override for |\version| flag:
%    \begin{macrocode}
%%\providecommand{\version}{final}
%    \end{macrocode}

% Include the main document:
%    \begin{macrocode}
\input{childdoc.def}
\childdocby{cdocsamp}
%    \end{macrocode}

%\iffalse
%</samplepart3|samplepart4>
%\fi
%
%\iffalse
%<*samplepart3>
%\fi
% Some text for part 3:
%    \begin{macrocode}
some text in part three
%    \end{macrocode}

%\iffalse
%</samplepart3>
%\fi
% Some text for part 4:
%\iffalse
%<*samplepart4>
%\fi
%    \begin{macrocode}
more text in part four
%    \end{macrocode}

%\iffalse
%</samplepart4>
%\fi
%
% %%%%%%%%%%%%%%%%%%%%%%%%%%%%%%%%%%%%%%
% \paragraph{Forwarding for a Complete Draft.}
%
% The following forwarding file |cdocsdrf.tex|
% compiles the main document in draft mode:
%\iffalse
%<*sampledraft>
%\fi
%    \begin{macrocode}
\def\version{draft}
\input{childdoc.def}
\childdocforward{cdocsamp}
%    \end{macrocode}

%\iffalse
%</sampledraft>
%\fi
%
% %%%%%%%%%%%%%%%%%%%%%%%%%%%%%%%%%%%%%%
% \paragraph{Forwarding for Final Version of the Chapters.}
%
% The following forwarding files |cdocsfn1.tex| and |cdocsfn2.tex|
% (with identical content)
% compile the final versions of the child documents
% |cdocsch1.tex| and |cdocsch2.tex|, respectively:
%\iffalse
%<*samplefinal>
%\fi
%    \begin{macrocode}
\def\version{final}
\input{childdoc.def}
\childdocforwardprefix[cdocsamp]{cdocsfn}{cdocsch}
%    \end{macrocode}

%\iffalse
%</samplefinal>
%\fi
%
% %%%%%%%%%%%%%%%%%%%%%%%%%%%%%%%%%%%%%%
% \paragraph{Command Line Processing.}
%
% The following three command lines generate the output files
% |cdocscld|, |cdocscl1| and |cdocscl2|
% which should be identical to
% |cdocsdrf|, |cdocsch1| and |cdocsfn2|, respectively:
% \begin{center}
% \begin{tabular}{l}
% |latex -jobname cdocscld \|\\
% |  "\def\version{draft}\input{childdoc.def}\childdocforward{cdocsamp}"|\\
% |latex -jobname cdocscl1 \|\\
% |  "\input{childdoc.def}\childdocforward[cdocsamp]{cdocsch1}"|\\
% |latex -jobname cdocscl2 \|\\
% |  "\def\version{final}\input{childdoc.def}\childdocforward{cdocsch2}"|
% \end{tabular}
% \end{center}
% Note that the trailing backslash on each first line
% merely continues the input to the second line
% (for convenient cut ant paste).
% Furthermore, the command |latex| can be replaced by any
% of its alternative versions such as |pdflatex|.
%
% %%%%%%%%%%%%%%%%%%%%%%%%%%%%%%%%%%%%%%%%%%%%%%%%%%%%%%%%%%%%%%%%%%%%%%%%%%%%%%
% %%%%%%%%%%%%%%%%%%%%%%%%%%%%%%%%%%%%%%%%%%%%%%%%%%%%%%%%%%%%%%%%%%%%%%%%%%%%%%
% \section{Implementation}
%\iffalse
%<*package>
%\fi
%
% This section describes the definitions file |childdoc.def|.

% The definitions cannot be loaded using |\usepackage| or |\RequirePackage|
% which has a mechanism to prevent loading a style file more than once.
% When loading the definitions by means of |\input|
% multiple instances have to be prevented manually:
%\iffalse
%This code needs to be before the `\ProvidesFile' directive
%which is defined at the beginning of this file.
%Therefore it is also placed there and commented out here.
%</package>
%<*discard>
%\fi
%    \begin{macrocode}
\ifdefined\childdocmain\endinput\fi
%    \end{macrocode}
%\iffalse
%</discard>
%<*package>
%\fi
%
% \macro{\ifchilddoc}
% \macro{\ifchilddocmanual}
% The conditional |\ifchilddoc| tells whether a
% child (true) or main (false) document is being compiled.
% The conditional |\ifchilddocmanual| tells whether
% the |\includeonly| mechanism is used (false) or
% the selection of child files must be performed manually (true).
% The definitions initialise to false:
%    \begin{macrocode}
\newif\ifchilddoc
\newif\ifchilddocmanual
%    \end{macrocode}

% \macro{\childdocname}
% \macro{\childdocjob}
% The macro |\childdocname| stores the name of the main document
% to be compiled. The macro |\childdocjob| stores the name of
% the document on which the \LaTeX{} compiler was originally invoked.
% The content of |\jobname| cannot be compared
% to filenames specified in the source due to different catcodes.
% The following code rescans |\jobname|, stores the result
% in |\childdocname| and saves a copy in |\childdocjob|:
%    \begin{macrocode}
\edef\childdocname{\scantokens\expandafter{\jobname\noexpand}}
\let\childdocjob\childdocname
%    \end{macrocode}

% \macro{\childdocdisable}
% The macro |\childdocdisable| prevents the main file
% from being processed more than once.
% At this stage, the main document command |\childdocmain|
% is assumed to be called once again where it should do nothing.
% Any subsequent call to it should prevent
% a secondary processing of the main document
% It overwrites the forwarding commands
% |\childdocof| and |\childdocforward|
% with empty macros to prevent further inclusions of the main document:
%    \begin{macrocode}
\newcommand{\childdocdisable}
{
  \renewcommand{\childdocmain}[1]{\renewcommand{\childdocmain}[1]{\endinput}}
  \renewcommand{\childdocof}[1]{}
  \renewcommand{\childdocby}[2][]{}
  \renewcommand{\childdocforward}[2][]{}
  \renewcommand{\childdocdisable}{}
}
%    \end{macrocode}

% \macro{\childdocmain}
% The macro |\childdocmain| is to be called at the top of the main file
% with nothing or the main filename (without extension) as argument.
% First, it breaks loops.
% If the argument is not empty and does not match |\childdocname|
% (which is set by the first inclusion of |childdoc.def|),
% |\ifchilddoc| is set to true, |\includeonly| is applied to the child file
% and |\jobname| is set to the main file
% (for proper handling of |.aux| files):
%    \begin{macrocode}
\newcommand{\childdocmain}[1]
{
  \childdocdisable\childdocmain{}
  \if?#1?\else
    \begingroup
      \def\childdoctmp{#1}
      \ifx\childdoctmp\childdocname
        \def\childdoctmp{}
      \else
        \def\childdoctmp
        {
          \childdoctrue
          \includeonly{\childdocname}
          \def\childdocjob{#1}
          \def\jobname{#1}
        }
      \fi
      \expandafter
    \endgroup
    \childdoctmp
  \fi
}
%    \end{macrocode}

% \macro{\childdocof}
% The command |\childdocof| redirects
% compilation to the main file |#1|.
%    \begin{macrocode}
\newcommand{\childdocof}[1]
{
  \childdocdisable
  \childdoctrue
  \includeonly{\childdocname}
  \def\jobname{#1}
  \def\childdocjob{#1}
  \input{#1}
}
%    \end{macrocode}

% \macro{\childdocby}
% The command |\childdocby| ....
%    \begin{macrocode}
\newcommand{\childdocby}[2][]
{
  \childdocdisable
  \childdoctrue
  \childdocmanualtrue
  \if?#1?\else
    \def\jobname{#2}
  \fi
  \def\childdocjob{#2}
  \input{#2}
  \endinput
}
%    \end{macrocode}

% \macro{\childdocforward}
% The command |\childdocforward| redirects
% compilation to the main file or
% (if the optional argument is given) a child file.
% Parameters are set as if the main file
% or a child file starting with |\childdocof| was compiled.
% Then compilation is handed over to the main file:
%    \begin{macrocode}
\newcommand{\childdocforward}[2][]
{
  \begingroup
    \if?#1?
      \def\childdoctmp
      {
        \def\childdocname{#2}
        \def\childdocjob{#2}
        \def\jobname{#2}
        \input{#2}
        \endinput
      }
    \else
      \def\childdoctmp
      {
        \childdocdisable
        \def\childdocname{#2}
        \childdoctrue
        \includeonly{#2}
        \def\childdocjob{#1}
        \def\jobname{#1}
        \input{#1}
        \endinput
      }
    \fi
    \expandafter
  \endgroup
  \childdoctmp
}
%    \end{macrocode}

% \macro{\childdocforwardprefix}
% The command |\childdocforwardprefix| redirects
% compilation to the main or a child file by means of a pattern.
% The prefix |#1| in the current filename is replaced by |#2|
% and the suffix of the current filename is kept
% (it is assumed that the filename does not contain the substring `|~~~|'
% which is used as a delimiter).
% Compilation is handed over to the new file by |\childdocforward|:
%    \begin{macrocode}
\newcommand{\childdocforwardprefix}[3][]
{
  \begingroup
    \def\childdocextract #2##1~~~{\def\childdoctmp{\childdocforward[#1]{#3##1}}}
    \expandafter\childdocextract\childdocname~~~
    \expandafter
  \endgroup
  \childdoctmp
}
%    \end{macrocode}

% \macro{\childdoc}
% The deprecated macro |\childdoc| is a legacy version of |\childdocmain|:
%    \begin{macrocode}
\newcommand{\childdoc}{\childdocmain}
%    \end{macrocode}

% \macro{\childdocredirect}
% The deprecated macro |\childdocredirect| is a legacy version
% of |\childdocforward| and |\childdocforwardprefix|:
%    \begin{macrocode}
\newcommand{\childdocredirect}[2][]
{
  \begingroup
    \if?#1?
      \def\childdoctmp{\childdocforward{#2}}
    \else
      \def\childdoctmp{\childdocforwardprefix{#1}{#2}}
    \fi
    \expandafter
  \endgroup
  \childdoctmp
}
%    \end{macrocode}

%\iffalse
%</package>
%\fi
%
\endinput
|
and perform the replacements as outlined below.
Instead of |\childdocmain{|\textit{main}|}| add the following code
to the top of the main file:
%
\begin{center}
\begin{tabular}{l}
|\||ifdefined\childdocname\endinput\||fi\newif\ifchilddoc|\\
|\edef\childdocname{\scantokens\expandafter{\jobname\noexpand}}|\\
|\def\childdocmain{|\textit{main}|}\||ifx\childdocmain\childdocname\||else|\\
|\childdoctrue\includeonly{\childdocname}\let\jobname\childdocmain\||fi|\\
\end{tabular}
\end{center}
%
Instead of |\childdocof{|\textit{main}|}| just include the main file
at the top of each child file:
%
\begin{center}
|\input{|\textit{main}|}|
\end{center}
%
A simple redirection |\childdocforward{|\textit{dest}|}| is achieved by:
%
\begin{center}
|\def\jobname{|\textit{dest}|}\input{\jobname}|
\end{center}
%
The redirection with prefix
|\childdocforwardprefix[|\textit{prefix}|]{|\textit{dest}|}|
is accomplished by:
%
\begin{center}
\begin{tabular}{l}
|{\edef\jobname{\scantokens\expandafter{\jobname\noexpand}}|\\
|\def\redirectjob |\textit{prefix}|#1~~~{\gdef\jobname{|\textit{dest}|#1}}|\\
|\expandafter\redirectjob\jobname~~~}\input{\jobname}|
\end{tabular}
\end{center}

In an alternative approach,
child documents can be compiled by a specific command line
without additional code or specific definitions:
%
\begin{center}
|... -jobname "|\textit{target}|" "|[\textit{flags}]%
|\includeonly{|\textit{dest}|}\input{|\textit{main}|}"|
\end{center}
%

%%%%%%%%%%%%%%%%%%%%%%%%%%%%%%%%%%%%%%%%%%%%%%%%%%%%%%%%%%%%%%%%%%%%%%%%%%%%%%%%
%%%%%%%%%%%%%%%%%%%%%%%%%%%%%%%%%%%%%%%%%%%%%%%%%%%%%%%%%%%%%%%%%%%%%%%%%%%%%%%%
\section{Information}

%%%%%%%%%%%%%%%%%%%%%%%%%%%%%%%%%%%%%%%%%%%%%%%%%%%%%%%%%%%%%%%%%%%%%%%%%%%%%%%%
\subsection{Copyright}

Copyright \copyright{} 2017--2018 Niklas Beisert

This work may be distributed and/or modified under the
conditions of the \LaTeX{} Project Public License, either version 1.3
of this license or (at your option) any later version.
The latest version of this license is in
  \url{http://www.latex-project.org/lppl.txt}
and version 1.3 or later is part of all distributions of \LaTeX{}
version 2005/12/01 or later.

This work has the LPPL maintenance status `maintained'.

The Current Maintainer of this work is Niklas Beisert.

This work consists of the files |README.txt|, |childdoc.ins| and |childdoc.dtx|
as well as the derived files |childdoc.def|, |cdocsamp.tex|
with |cdocsch1.tex|, |cdocsch2.tex|, |cdocspt3.tex|, |cdocspt4.tex|,
|cdocsdrf.tex|, |cdocsfn1.tex|, |cdocsfn2.tex|
as well as |childdoc.pdf|.

%%%%%%%%%%%%%%%%%%%%%%%%%%%%%%%%%%%%%%%%%%%%%%%%%%%%%%%%%%%%%%%%%%%%%%%%%%%%%%%%
\subsection{Files and Installation}

The package consists of the files:
%
\begin{center}
\begin{tabular}{ll}
    |README.txt|   & readme file \\
    |childdoc.ins| & installation file \\
    |childdoc.dtx| & source file \\
    |childdoc.def| & definition file \\
    |cdocsamp.tex| & sample main file \\
    |cdocsch1.tex| & sample include file \\
    |cdocsch2.tex| & sample include file \\
    |cdocspt3.tex| & sample part file \\
    |cdocspt4.tex| & sample part file \\
    |cdocsdrf.tex| & sample redirection file \\
    |cdocsfn1.tex| & sample redirection file \\
    |cdocsfn2.tex| & sample redirection file \\
    |childdoc.pdf| & manual
\end{tabular}
\end{center}
%
The distribution consists of the files
|README.txt|, |childdoc.ins| and |childdoc.dtx|.
%
\begin{itemize}
\item
Run (pdf)\LaTeX{} on |childdoc.dtx|
to compile the manual |childdoc.pdf| (this file).
\item
Run \LaTeX{} on |childdoc.ins| to create the definitions file |childdoc.def|
and the sample |cdocsamp.tex| with include files
|cdocsch1.tex|, |cdocsch2.tex|, |cdocspt3.tex|, |cdocspt4.tex|,
|cdocsdrf.tex|, |cdocsfn1.tex|, |cdocsfn2.tex|.
Then copy the file |childdoc.def| to an appropriate directory of your \LaTeX{}
distribution, e.g.\ \textit{texmf-root}|/tex/latex/childdoc|.
\end{itemize}

%%%%%%%%%%%%%%%%%%%%%%%%%%%%%%%%%%%%%%%%%%%%%%%%%%%%%%%%%%%%%%%%%%%%%%%%%%%%%%%%
\subsection{Related CTAN Packages}

There are several other packages which offer a similar functionality:
%
\begin{itemize}
\item
The packages
\href{http://ctan.org/pkg/docmute}{\textsf{docmute}},
\href{http://ctan.org/pkg/includex}{\textsf{includex}} and
\href{http://ctan.org/pkg/standalone}{\textsf{standalone}}
provide commands to include only the document body of
a child file thus allowing both files to be compiled individually.
\item
The packages \href{http://ctan.org/pkg/subdocs}{\textsf{subdocs}}
and \href{http://ctan.org/pkg/subfiles}{\textsf{subfiles}}
provide structures in which the main and child documents can be
encapsulated and allowing them to be compiled individually.
The inclusion mechanism is different from the conventional |\include|.
\item
The package \href{http://ctan.org/pkg/combine}{\textsf{combine}}
is an elaborate solution to combine several documents into one.
\end{itemize}
%
See also the CTAN topic \href{http://ctan.org/topic/subdocs}{\textsf{subdocs}}
for further related packages.
The present package differs from the above solutions in that
a document structure constructed with the conventional |\include| mechanism
just needs two extra commands at the top of every file
such that all constituent files can be compiled individually.

%%%%%%%%%%%%%%%%%%%%%%%%%%%%%%%%%%%%%%%%%%%%%%%%%%%%%%%%%%%%%%%%%%%%%%%%%%%%%%%%
%\subsection{Feature Suggestions}
%
%The following is a list of features which may be useful for future
%versions of this package:
%%
%\begin{itemize}
%\item
%\ldots
%\end{itemize}

%%%%%%%%%%%%%%%%%%%%%%%%%%%%%%%%%%%%%%%%%%%%%%%%%%%%%%%%%%%%%%%%%%%%%%%%%%%%%%%%
\subsection{Revision History}

%%%%%%%%%%%%%%%%%%%%%%%%%%%%%%%%%%%%%%%%
\paragraph{v2.0:} 2018/12/30

\begin{itemize}
\item
immediate forward processing
\item
added |\childdocby| mechanism
\item
manual restructured
\end{itemize}

%%%%%%%%%%%%%%%%%%%%%%%%%%%%%%%%%%%%%%%%
\paragraph{v1.6:} 2018/01/17

\begin{itemize}
\item
application for development of include files
\item
corrections to manual
\end{itemize}

%%%%%%%%%%%%%%%%%%%%%%%%%%%%%%%%%%%%%%%%
\paragraph{v1.5:} 2017/05/21

\begin{itemize}
\item
more complete structuring introduced
\item
|\childdocof| introduced
\item
|\childdoc| renamed to |\childdocmain|
\item
|\childredirect| renamed to |\childdocforward| and |\childdocforwardprefix|
and functionality expanded
\end{itemize}

%%%%%%%%%%%%%%%%%%%%%%%%%%%%%%%%%%%%%%%%
\paragraph{v1.0:} 2017/04/27

\begin{itemize}
\item
manual and install package
\item
first version published on CTAN
\end{itemize}

%%%%%%%%%%%%%%%%%%%%%%%%%%%%%%%%%%%%%%%%
\paragraph{v0.6:} 2017/04/26

\begin{itemize}
\item
redirection mechanism added
\end{itemize}

%%%%%%%%%%%%%%%%%%%%%%%%%%%%%%%%%%%%%%%%
\paragraph{v0.5:} 2017/04/26

\begin{itemize}
\item
functionality in definition file
\end{itemize}


%%%%%%%%%%%%%%%%%%%%%%%%%%%%%%%%%%%%%%%%%%%%%%%%%%%%%%%%%%%%%%%%%%%%%%%%%%%%%%%%
%%%%%%%%%%%%%%%%%%%%%%%%%%%%%%%%%%%%%%%%%%%%%%%%%%%%%%%%%%%%%%%%%%%%%%%%%%%%%%%%
%%%%%%%%%%%%%%%%%%%%%%%%%%%%%%%%%%%%%%%%%%%%%%%%%%%%%%%%%%%%%%%%%%%%%%%%%%%%%%%%
\appendix

\settowidth\MacroIndent{\rmfamily\scriptsize 000\ }

 \DocInput{childdoc.dtx}

\end{document}
%</driver>
% \fi
%
% %%%%%%%%%%%%%%%%%%%%%%%%%%%%%%%%%%%%%%%%%%%%%%%%%%%%%%%%%%%%%%%%%%%%%%%%%%%%%%
% %%%%%%%%%%%%%%%%%%%%%%%%%%%%%%%%%%%%%%%%%%%%%%%%%%%%%%%%%%%%%%%%%%%%%%%%%%%%%%
% \section{Sample}
%\iffalse
%<*samplemain>
%\fi
%
% The following presents a sample document
% with two chapters, two parts, a title page,
% a compile flag as well as three forwarding files to set the flag.
% It consists of eight |.tex| files:
% \begin{center}
% \begin{tabular}{ll}
% |cdocsamp.tex|&main file\\
% |cdocsch1.tex|&include file for chapter 1\\
% |cdocsch2.tex|&include file for chapter 2\\
% |cdocspt3.tex|&include file for part 3\\
% |cdocspt4.tex|&include file for part 4\\
% |cdocsdrf.tex|&forwarding file for main file in draft mode\\
% |cdocsfi1.tex|&forwarding file for final version of chapter 1\\
% |cdocsfi2.tex|&forwarding file for final version of chapter 2\\
% \end{tabular}
% \end{center}
% Each of the eight files can be compiled directly by the \LaTeX{} compiler.
%
% %%%%%%%%%%%%%%%%%%%%%%%%%%%%%%%%%%%%%%
% \paragraph{Main File.}
%
% The main file is called |cdocsamp.tex|.
%
% Load the \textsf{childdoc} definitions and
% declare the filename for the main document:
%    \begin{macrocode}
% \iffalse
%
% childdoc.dtx Copyright (C) 2017-2018 Niklas Beisert
%
% This work may be distributed and/or modified under the
% conditions of the LaTeX Project Public License, either version 1.3
% of this license or (at your option) any later version.
% The latest version of this license is in
%   http://www.latex-project.org/lppl.txt
% and version 1.3 or later is part of all distributions of LaTeX
% version 2005/12/01 or later.
%
% This work has the LPPL maintenance status `maintained'.
%
% The Current Maintainer of this work is Niklas Beisert.
%
% This work consists of the files childdoc.dtx and childdoc.ins
% and the derived files childdoc.def and cdocsamp.tex with
% cdocsch1.tex, cdocsch2.tex, cdocsdrf.tex, cdocsfn1.tex, cdocsfn2.tex.
%
%<package>\ifdefined\childdocmain\endinput\fi
%<package>\ProvidesFile{childdoc.def}[2018/12/30 v2.0 child document driver]
%<samplemain>\ProvidesFile{cdocsamp.tex}[2018/12/30 v2.0 sample for childdoc]
%<*driver>
%\ProvidesFile{childdoc.drv}[2018/12/30 v2.0 childdoc reference manual file]
\PassOptionsToClass{10pt,a4paper}{article}
\documentclass{ltxdoc}

\usepackage[margin=35mm]{geometry}
\usepackage{hyperref}
\usepackage{hyperxmp}
\usepackage[usenames]{color}

\hypersetup{colorlinks=true}
\hypersetup{pdfstartview=FitH}
\hypersetup{pdfpagemode=UseNone}
\hypersetup{pdfsource={}}
\hypersetup{pdflang={en-UK}}
\hypersetup{pdfcopyright={Copyright 2017-2018 Niklas Beisert.
  This work may be distributed and/or modified under the
  conditions of the LaTeX Project Public License, either version 1.3
  of this license or (at your option) any later version.}}
\hypersetup{pdflicenseurl={http://www.latex-project.org/lppl.txt}}
\hypersetup{pdfcontactaddress={ETH Zurich, ITP, HIT K,
  Wolfgang-Pauli-Strasse 27}}
\hypersetup{pdfcontactpostcode={8093}}
\hypersetup{pdfcontactcity={Zurich}}
\hypersetup{pdfcontactcountry={Switzerland}}
\hypersetup{pdfcontactemail={nbeisert@itp.phys.ethz.ch}}
\hypersetup{pdfcontacturl={http://people.phys.ethz.ch/\xmptilde nbeisert/}}

\newcommand{\secref}[1]{\hyperref[#1]{section \ref*{#1}}}

\parskip1ex
\parindent0pt
\let\olditemize\itemize
\def\itemize{\olditemize\parskip0pt}

\begin{document}

\title{The \textsf{childdoc} Package}
\hypersetup{pdftitle={The childdoc Package}}
\author{Niklas Beisert\\[2ex]
  Institut f\"ur Theoretische Physik\\
  Eidgen\"ossische Technische Hochschule Z\"urich\\
  Wolfgang-Pauli-Strasse 27, 8093 Z\"urich, Switzerland\\[1ex]
  \href{mailto:nbeisert@itp.phys.ethz.ch}
  {\texttt{nbeisert@itp.phys.ethz.ch}}}
\hypersetup{pdfauthor={Niklas Beisert}}
\hypersetup{pdfsubject={Manual for the LaTeX2e Package childdoc}}
\date{30 December 2018, \textsf{v2.0}}
\maketitle

\begin{abstract}\noindent
\textsf{childdoc} is a \LaTeXe{} package
that enables the direct compilation
of document sections included by |\include|
to individual files.
\end{abstract}

\begingroup
\parskip0ex
\tableofcontents
\endgroup

%%%%%%%%%%%%%%%%%%%%%%%%%%%%%%%%%%%%%%%%%%%%%%%%%%%%%%%%%%%%%%%%%%%%%%%%%%%%%%%%
%%%%%%%%%%%%%%%%%%%%%%%%%%%%%%%%%%%%%%%%%%%%%%%%%%%%%%%%%%%%%%%%%%%%%%%%%%%%%%%%
\section{Introduction}

\LaTeX{} provides a mechanism to structure a large document (such as a book)
into a main file and several child files (containing the chapters)
using the |\include| command.
This mechanism is beneficial for documents
which span hundreds of pages in order to
make the source file(s) more manageable.
Moreover, compilation can be restricted to
selected child files by means of the |\includeonly| command.
The latter feature can be used to reduce the compilation time while editing
(this was significantly more useful in the earlier days of \LaTeX{})
or to generate a smaller document which is easier to navigate.
Another application of |\includeonly| is to generate
documents consisting of selected parts of the complete document.

However, there are a few drawbacks of the plain |\include| mechanism:
\begin{itemize}
\item
The child files cannot be compiled on their own,
they can only be compiled via the main file.
A naive editing environment
(such as a text editor with an option
to have the current file processed by \LaTeX)
may require one to switch to the main file before compiling;
attempting to compile the child file produces errors.
\item
The main file must be modified (each time)
to adjust the |\includeonly| command
to the present needs. This easily leaves the main file in a messy state.
\item
The generated document will always carry the filename
of the main document. This is inconvenient if
several child files are to be compiled and
to be kept for distribution.
\end{itemize}

The present package provides a simple interface
to make child files individually compilable by \LaTeX{}.
Compiling a child file then has the same effect as compiling
the main file with an |\includeonly| command
to select the appropriate child.
Moreover the generated document will carry the name of the child
rather than the main file.
This resolves all three above issues.

This feature is meant to make the editing of books,
thesis documents and lecture notes somewhat more convenient.
However, the package can also be used efficiently for
composing a series of documents (such as exercise sheets)
which are typically distributed individually.
It then assists the author in generating the individual documents
(potentially in different versions)
as well as a document containing the collected series.
Another application is in developing style files
or other kinds of included material
where compilation of the style file could redirect
to a sample or test file.

%%%%%%%%%%%%%%%%%%%%%%%%%%%%%%%%%%%%%%%%%%%%%%%%%%%%%%%%%%%%%%%%%%%%%%%%%%%%%%%%
%%%%%%%%%%%%%%%%%%%%%%%%%%%%%%%%%%%%%%%%%%%%%%%%%%%%%%%%%%%%%%%%%%%%%%%%%%%%%%%%
\section{Usage}

First of all, the package \textsf{childdoc} is \emph{not} a standard
\LaTeXe{} |.sty| style file! Therefore it needs to be invoked in
a non-standard way.

%%%%%%%%%%%%%%%%%%%%%%%%%%%%%%%%%%%%%%%%%%%%%%%%%%%%%%%%%%%%%%%%%%%%%%%%%%%%%%%%
\subsection{Included Files}
\label{sec:include}

%%%%%%%%%%%%%%%%%%%%%%%%%%%%%%%%%%%%%%%%
\DescribeMacro{\childdocmain}
To use the package, add the commands
\begin{center}
\begin{tabular}{l}
|\input{childdoc.def}|\\
|\childdocmain{}|\\
\end{tabular}
\end{center}
at the very top of the main \LaTeX{} file,
in particular \emph{before} the |\documentclass| statement!
The argument of |\childdocmain| should be left empty
(but it must be present).

%%%%%%%%%%%%%%%%%%%%%%%%%%%%%%%%%%%%%%%%
\DescribeMacro{\childdocof}
Furthermore, add the commands
\begin{center}
\begin{tabular}{l}
|\input{childdoc.def}|\\
|\childdocof{|\textit{main}|}|\\
\end{tabular}
\end{center}
at the top of every child file \textit{child}
which is included by |\include{|\textit{child}|}|
from within the main file
(or at least for those files to be compiled individually).
The argument \textit{main} must be the filename of the main file.

There are a couple of
considerations in setting up the main and child documents:

%%%%%%%%%%%%%%%%%%%%%%%%%%%%%%%%%%%%%%%%
\paragraph{Restrictions.}

Please note the following restrictions:
\begin{itemize}
\item
|\childdocmain| must be called with one argument \textit{main}
to ensure compatibility with earlier version of the package.
It must either be empty (|\childdocmain{}|)
or precisely match the filename of the main file in which it is specified.
See \secref{sec:detection} for further information.
\item
The filename \textit{main} must be specified without the |.tex| extension.
\item
The filename \textit{main} is case sensitive
(even in case-insensitive file systems)
due to internal string comparison.
\item
The argument \textit{main} should be fully expanded, it cannot be a macro.
\item
Subdirectories and special characters should be avoided in filenames.
\item
The command |\childdocmain{|\textit{main}|}| must be followed by a whitespace.
It should not be followed immediately by another command
or by a comment mark `|%|'.
This is because the \TeX{} parser reads the token immediately following
the argument of |\childdocmain| and puts it
at the beginning of every child section;
however, a white\-space is ignored.
\end{itemize}

%%%%%%%%%%%%%%%%%%%%%%%%%%%%%%%%%%%%%%%%
\paragraph{Content of Main File.}

It is advisable to place all content in the child files included by |\include|.
Any output contained in the main file will appear in all child documents
unless suppressed manually;
it cannot be suppressed automatically by the |\includeonly| directive
and thus should normally be avoided.
A method to include some content in the main file
by means of conditional processing is described in \secref{sec:conditional}.

%%%%%%%%%%%%%%%%%%%%%%%%%%%%%%%%%%%%%%%%
\paragraph{Page Numbering.}

When only a part of the document is compiled,
the appropriate numbering of pages
(as well as other status parameters)
is determined from the |.aux| files.
The latter contain information from previous passes.
However this information needs to propagate through
all intermediate child documents.
Therefore the page numbering in child documents may well
be inconsistent until the complete document is compiled at least once.

A useful (if unconventional) way to always ensure a consistent
page numbering is to restart the numbering in each child document
and denote the pages by `\textit{child}|.|\textit{page}'
where \textit{child} represents the chapter/section number of the child file.
This can be achieved by the command
|\numberwithin{page}{|\textit{child}|}|
of the \textsf{amsmath} package
where \textit{child} can be |chapter| or |section|
depending on the chosen structuring.
Alternatively, one can modify the macro |\thepage| appropriately
and reset the counter |page| at the start of each child file.

%%%%%%%%%%%%%%%%%%%%%%%%%%%%%%%%%%%%%%%%%%%%%%%%%%%%%%%%%%%%%%%%%%%%%%%%%%%%%%%%
\subsection{Conditional Processing}
\label{sec:conditional}

The package provides a mechanism to compile different versions
of a document. To customise the versions further some conditional processing
can come in handy to distinguish which version is being compiled.
The package provides two macros to describe the compilation context:

%%%%%%%%%%%%%%%%%%%%%%%%%%%%%%%%%%%%%%%%
\DescribeMacro{\ifchilddoc}
The conditional |\ifchilddoc| distinguishes between the compilation of
child documents and the main document:
%
\begin{center}
|\ifchilddoc |\textit{child-code}| |[|\||else |\textit{main-code}]| \||fi|
\end{center}

%%%%%%%%%%%%%%%%%%%%%%%%%%%%%%%%%%%%%%%%
\DescribeMacro{\childdocname}
\DescribeMacro{\childdocjob}
The macro |\childdocname| contains the filename (without extension)
of the main or child file being processed.
Note that |\childdocjob| will always contain the name of the main file.

%%%%%%%%%%%%%%%%%%%%%%%%%%%%%%%%%%%%%%%%
\paragraph{Title Page.}

Conditional processing can be used to include a title or banner page
in the main document when proper precautions are taken.
Importantly, the code in the main file should ensure that the page counter
(as well as other status parameters which are stored in the |.aux| files)
takes the same value after the conditional processing.
Otherwise the page numbers may take divergent values
depending on which part is compiled.

For example, a title page could be declared by:
%
\begin{center}
\begin{tabular}{l}
|\ifchilddoc\||else|\\
|\addtocounter{page}{-1}|\\
\textit{code for title page}\\
|\newpage|\\
|\||fi|
\end{tabular}
\end{center}
%
A banner page for the child documents can be generated by:
%
\begin{center}
\begin{tabular}{l}
|\ifchilddoc|\\
|\addtocounter{page}{-1}|\\
\textit{code for banner page}\\
|\newpage|\\
|\||fi|
\end{tabular}
\end{center}
%
Here one could write a message such as:
\begin{center}
|This is the part \childdocname{} of \childdocjob{}.|
\end{center}

%%%%%%%%%%%%%%%%%%%%%%%%%%%%%%%%%%%%%%%%%%%%%%%%%%%%%%%%%%%%%%%%%%%%%%%%%%%%%%%%
\subsection{Flags}
\label{sec:flags}

The package makes it easy to generate different versions
of the main or child documents.
To this end compilation flags can be defined
and assigned different default values.
They will be particularly useful in conjunction
with the forwarding mechanism described in \secref{sec:forward}.

For example, it may be useful to have a flag |\version|
which can be set to |draft| or |final|.
The document source will contain some conditional code
depending on the value of |\version|.
Suppose further, the flag should default to |final| for the main file
and to |draft| for child files
which is a natural assignment for editing the document.
This is achieved by placing the following code
in the preamble of the main document
(below the |\childdocmain| directive):
%
\begin{center}
\begin{tabular}{l}
|\ifchilddoc|\\
|\providecommand{\version}{draft}|\\
|\||else|\\
|\providecommand{\version}{final}|\\
|\||fi|
\end{tabular}
\end{center}
%
The definition by |\providecommand| makes sure
that previous definitions are not overwritten.
Further statements |\providecommand{\version}{...}|
can thus be added before the above code to override it.

For the main file, one might add a line
(between |\childdocmain| and the above block)
%
\begin{center}
|%\ifchilddoc\||else\providecommand{\version}{draft}\||fi|
\end{center}
%
which can be uncommented to produce a draft version.
Likewise one can add a line to the very top of a child file
(above the |\childdocof{|\textit{main}|}| directive)
%
\begin{center}
|%\providecommand{\version}{final}|
\end{center}
%
which can be uncommented to produce the final version of this child document.

%%%%%%%%%%%%%%%%%%%%%%%%%%%%%%%%%%%%%%%%%%%%%%%%%%%%%%%%%%%%%%%%%%%%%%%%%%%%%%%%
\subsection{Forwarding}
\label{sec:forward}

Different versions of the main or child documents
using compilation flags as described in \secref{sec:flags}
can be (permanently) stored in different files
for convenient compilation, viewing and distribution.
To this end, the package defines a command
to pass on compilation to a different file:

%%%%%%%%%%%%%%%%%%%%%%%%%%%%%%%%%%%%%%%%
\DescribeMacro{\childdocforward}
The command |\childdocforward| redirects processing to
another source file:
%
\begin{center}
\begin{tabular}{l}
|\input{childdoc.def}|\\
|\childdocforward[|\textit{main}|]{|\textit{dest}|}|\\
\end{tabular}
\end{center}
%
The argument \textit{dest} is the destination file
(without extension).
It should be the main file or one of the child files.
Note that further \textsf{childdoc} directives
such as |\childdocof| and |\childdocforward|
in the indicated file will be processed in this form.
The optional argument \textit{main}
passes on directly to the main file \textit{main}
while pretending to compile the child \textit{dest}.
This form behaves as if \textit{dest}
issues |\childdocof{|\textit{main}|}| right away,
and no further \textsf{childdoc} directives will be processed.

%%%%%%%%%%%%%%%%%%%%%%%%%%%%%%%%%%%%%%%%
\DescribeMacro{\...prefix}
In the alternative form |\childdocforwardprefix|,
%
\begin{center}
\begin{tabular}{l}
|\input{childdoc.def}|\\
|\childdocforwardprefix[|\textit{main}|]{|\textit{prefix}|}{|\textit{dest}|}|
\end{tabular}
\end{center}
%
the destination file is determined by a pattern
depending on the current file:
To make this work, the current file must be called
`{\textit{prefix}\hspace{0.2em}\textit{suffix}}'
with \textit{prefix} matching precisely the argument.
Processing is then passed on to the file
`{\textit{dest}\hspace{0.2em}\textit{suffix}}'.
Surely, the same effect is achieved by
directly specifying the
argument `{\textit{dest}\hspace{0.2em}\textit{suffix}}'
in the first form.
However, that requires to set up a different file
for each child. With the alternative form of the command
all these files can have exactly the same content
which simplifies setting them up and maintaining them.

For example, the following file |draft.tex|
with a compilation flag |\version| as described in \secref{sec:flags}
compiles the main document as a draft:
%
\begin{center}
\begin{tabular}{l}
|\def\version{draft}|\\
|\input{childdoc.def}|\\
|\childdocforward{|\textit{main}|}|
\end{tabular}
\end{center}
%
Likewise, the following files |final|\textit{nn}|.tex|
compile the final version of the child document
|child|\textit{nn}|.tex|:
%
\begin{center}
\begin{tabular}{l}
|\def\version{final}|\\
|\input{childdoc.def}|\\
|\childdocforwardprefix{final}{child}|
\end{tabular}
\end{center}
%

Note that when several versions of a main file and/or of each child file
are to be generated, it may be convenient to set up a |Makefile| or
shell script to automatise the process.

%%%%%%%%%%%%%%%%%%%%%%%%%%%%%%%%%%%%%%%%%%%%%%%%%%%%%%%%%%%%%%%%%%%%%%%%%%%%%%%%
\subsection{Command Line Processing}
\label{sec:commandline}

The effect of redirection files can also be achieved by invoking
the \LaTeX{} compiler with a more elaborate command line.
Most conveniently this should be done as part
of a shell script or a |Makefile|.

When using \textsf{childdoc} in the main file, the following
command lines effectively perform a redirection
(note that depending on the shell being used,
backslashes may have to be doubled: `|\|' $\to$ `|\\|'):
%
\begin{center}
|... -jobname "|\textit{target}|" |\\|"|[\textit{flags}]%
|\input{childdoc.def}\childdocforward[|\textit{main}|]{|\textit{dest}|}"|
\end{center}
%
Here \textit{target} is the name of the output file,
\textit{main} is the name of the main file
and \textit{dest} is the name of the main or child file to be processed
(all filenames without extensions).
The optional argument \textit{main} can be omitted
if \textit{main} matches \textit{dest}.
Optionally, compilation \textit{flags} can be defined via |\def| commands.
This command line makes the \TeX{} engine believe
it is compiling the file \textit{target}
whose content is specified as the latter parameter.
The provided code then forwards the processing to
\textit{main} or \textit{dest} as described in \secref{sec:forward}.

%%%%%%%%%%%%%%%%%%%%%%%%%%%%%%%%%%%%%%%%%%%%%%%%%%%%%%%%%%%%%%%%%%%%%%%%%%%%%%%%
\subsection{Include by Input}
\label{sec:input}

Including child documents by |\include| has some restrictions by design.
Most notably, the content of a child document always occupies
its own set of pages; pages cannot be shared between child documents.
Usually, this behaviour makes perfect sense
because each child document contain an essential part of the document.
However, in some situations it may be desirable to compose
a document from a collection of parts
without having mandatory page breaks between then.
For this case, the package
provides a mechanism to include parts
by |\input| which can also be processed individually.
However, by construction this mechanism
requires manual handling of the content to be output.

%%%%%%%%%%%%%%%%%%%%%%%%%%%%%%%%%%%%%%%%
\DescribeMacro{\ifchilddocmanual}
The main file should be prepared as usual, see \secref{sec:include}.
However, the document body must make a distinction
between processing of an individual part and of the main document, e.g.:
%
\begin{center}
\begin{tabular}{l}
|\ifchilddocmanual|\\
|\input{\childdocname}|\\
|\||else|\\
\textit{document body with }|\input{|\textit{part}|}|\\
|\||fi|
\end{tabular}
\end{center}
%
The conditional |\ifchilddocmanual| is true whenever
a part to be included by |\input| is being compiled,
and the name of the part is stored in |\childdocname|.

%%%%%%%%%%%%%%%%%%%%%%%%%%%%%%%%%%%%%%%%
\DescribeMacro{\childdocby}
Each part to be included by |\input| should start with:
%
\begin{center}
\begin{tabular}{l}
|\input{childdoc.def}|\\
|\childdocby{|\textit{main}|}|\\
\end{tabular}
\end{center}
%
The directive |\childdocby| is similar to |\childdocof|
described in \secref{sec:include},
but the subsequent selection of content must be done manually.
To that end, both |\ifchilddoc| and |\ifchilddocmanual|
will be true upon processing of a part,
and the name of the part is stored in |\childdocname|.
Note that |\jobname| will be set to the filename of the current part
so that each part receives an individual |.aux| file
that does not interfere with the |.aux| file(s) of the main document.
This behaviour can be altered by the alternative form
|\childdocby[*]{|\textit{main}|}| (with a non-empty optional argument)
which uses the |.aux| file of the main document
by setting |\jobname| to \textit{main}.

%%%%%%%%%%%%%%%%%%%%%%%%%%%%%%%%%%%%%%%%%%%%%%%%%%%%%%%%%%%%%%%%%%%%%%%%%%%%%%%%
\subsection{Driver Development}
\label{sec:driver}

The \textsf{childdoc} mechanism can also be use for the development
of definition files such as \LaTeX{} styles or classes.
This case differs from the above setup with multiple parts
included by |\include| in that no |\includeonly| should be invoked.
This can be achieved by starting the include file
(before |\ProvidesPackage|) with:
%
\begin{center}
\begin{tabular}{l}
|\input{childdoc.def}|\\
|\childdocforward{|\textit{main}|}|\\
\end{tabular}
\end{center}
%
or alternatively with:
%
\begin{center}
\begin{tabular}{l}
|\input{childdoc.def}|\\
|\childdocby{|\textit{main}|}|\\
\end{tabular}
\end{center}
%
Both forms have slightly different effects as described above.
The main file is prepared as usual, see \secref{sec:include}.

%%%%%%%%%%%%%%%%%%%%%%%%%%%%%%%%%%%%%%%%%%%%%%%%%%%%%%%%%%%%%%%%%%%%%%%%%%%%%%%%
\subsection{Legacy Detection}
\label{sec:detection}

The directive |\childdocmain| in the main file can detect
whether the complete document or merely a child is to be compiled
even without using the directive |\childdocof|.
This method is deprecated because it is less robust
and there is no compelling reason to use it;
it is merely provided for backward compatibility
and it may be removed in future versions.

If the detection mechanism is to be used,
it is mandatory to correctly specify
the filename of the main file as the argument of |\childdocmain|:
%
\begin{center}
\begin{tabular}{l}
|\input{childdoc.def}|\\
|\childdocmain{|\textit{main}|}|\\
\end{tabular}
\end{center}
%
If |\jobname| does not match the argument \textit{main} of |\childdocmain|,
it is assumed that |\jobname| points to the child file to be compiled.
When using |\childdocmain| with the main file specified as argument,
it suffices to start a child file
with just |\input{|\textit{main}|}|
without loading of the package and using |\childdocof|.
If instead all processing is done
with the appropriate \textsf{childdoc} directives,
the argument of \textit{main} of |\childdocmain| can be empty.

An alternative version of the command line processing described
in \secref{sec:commandline} using the detection mechanism reads:
%
\begin{center}
|... -jobname "|\textit{target}|" "|[\textit{flags}]%
[|\def\jobname{|\textit{dest}|}|]|\input{|\textit{main}|}"|
\end{center}

%%%%%%%%%%%%%%%%%%%%%%%%%%%%%%%%%%%%%%%%%%%%%%%%%%%%%%%%%%%%%%%%%%%%%%%%%%%%%%%%
\subsection{Manual Code}
\label{sec:manual}

In case one cannot be certain whether the definitions file |childdoc.def|
is installed on the target \TeX{} distribution
and one prefers not to ship it,
it is conceivable to paste a few relevant commands into the sources.

To that end, drop all statements |\input{childdoc.def}|
and perform the replacements as outlined below.
Instead of |\childdocmain{|\textit{main}|}| add the following code
to the top of the main file:
%
\begin{center}
\begin{tabular}{l}
|\||ifdefined\childdocname\endinput\||fi\newif\ifchilddoc|\\
|\edef\childdocname{\scantokens\expandafter{\jobname\noexpand}}|\\
|\def\childdocmain{|\textit{main}|}\||ifx\childdocmain\childdocname\||else|\\
|\childdoctrue\includeonly{\childdocname}\let\jobname\childdocmain\||fi|\\
\end{tabular}
\end{center}
%
Instead of |\childdocof{|\textit{main}|}| just include the main file
at the top of each child file:
%
\begin{center}
|\input{|\textit{main}|}|
\end{center}
%
A simple redirection |\childdocforward{|\textit{dest}|}| is achieved by:
%
\begin{center}
|\def\jobname{|\textit{dest}|}\input{\jobname}|
\end{center}
%
The redirection with prefix
|\childdocforwardprefix[|\textit{prefix}|]{|\textit{dest}|}|
is accomplished by:
%
\begin{center}
\begin{tabular}{l}
|{\edef\jobname{\scantokens\expandafter{\jobname\noexpand}}|\\
|\def\redirectjob |\textit{prefix}|#1~~~{\gdef\jobname{|\textit{dest}|#1}}|\\
|\expandafter\redirectjob\jobname~~~}\input{\jobname}|
\end{tabular}
\end{center}

In an alternative approach,
child documents can be compiled by a specific command line
without additional code or specific definitions:
%
\begin{center}
|... -jobname "|\textit{target}|" "|[\textit{flags}]%
|\includeonly{|\textit{dest}|}\input{|\textit{main}|}"|
\end{center}
%

%%%%%%%%%%%%%%%%%%%%%%%%%%%%%%%%%%%%%%%%%%%%%%%%%%%%%%%%%%%%%%%%%%%%%%%%%%%%%%%%
%%%%%%%%%%%%%%%%%%%%%%%%%%%%%%%%%%%%%%%%%%%%%%%%%%%%%%%%%%%%%%%%%%%%%%%%%%%%%%%%
\section{Information}

%%%%%%%%%%%%%%%%%%%%%%%%%%%%%%%%%%%%%%%%%%%%%%%%%%%%%%%%%%%%%%%%%%%%%%%%%%%%%%%%
\subsection{Copyright}

Copyright \copyright{} 2017--2018 Niklas Beisert

This work may be distributed and/or modified under the
conditions of the \LaTeX{} Project Public License, either version 1.3
of this license or (at your option) any later version.
The latest version of this license is in
  \url{http://www.latex-project.org/lppl.txt}
and version 1.3 or later is part of all distributions of \LaTeX{}
version 2005/12/01 or later.

This work has the LPPL maintenance status `maintained'.

The Current Maintainer of this work is Niklas Beisert.

This work consists of the files |README.txt|, |childdoc.ins| and |childdoc.dtx|
as well as the derived files |childdoc.def|, |cdocsamp.tex|
with |cdocsch1.tex|, |cdocsch2.tex|, |cdocspt3.tex|, |cdocspt4.tex|,
|cdocsdrf.tex|, |cdocsfn1.tex|, |cdocsfn2.tex|
as well as |childdoc.pdf|.

%%%%%%%%%%%%%%%%%%%%%%%%%%%%%%%%%%%%%%%%%%%%%%%%%%%%%%%%%%%%%%%%%%%%%%%%%%%%%%%%
\subsection{Files and Installation}

The package consists of the files:
%
\begin{center}
\begin{tabular}{ll}
    |README.txt|   & readme file \\
    |childdoc.ins| & installation file \\
    |childdoc.dtx| & source file \\
    |childdoc.def| & definition file \\
    |cdocsamp.tex| & sample main file \\
    |cdocsch1.tex| & sample include file \\
    |cdocsch2.tex| & sample include file \\
    |cdocspt3.tex| & sample part file \\
    |cdocspt4.tex| & sample part file \\
    |cdocsdrf.tex| & sample redirection file \\
    |cdocsfn1.tex| & sample redirection file \\
    |cdocsfn2.tex| & sample redirection file \\
    |childdoc.pdf| & manual
\end{tabular}
\end{center}
%
The distribution consists of the files
|README.txt|, |childdoc.ins| and |childdoc.dtx|.
%
\begin{itemize}
\item
Run (pdf)\LaTeX{} on |childdoc.dtx|
to compile the manual |childdoc.pdf| (this file).
\item
Run \LaTeX{} on |childdoc.ins| to create the definitions file |childdoc.def|
and the sample |cdocsamp.tex| with include files
|cdocsch1.tex|, |cdocsch2.tex|, |cdocspt3.tex|, |cdocspt4.tex|,
|cdocsdrf.tex|, |cdocsfn1.tex|, |cdocsfn2.tex|.
Then copy the file |childdoc.def| to an appropriate directory of your \LaTeX{}
distribution, e.g.\ \textit{texmf-root}|/tex/latex/childdoc|.
\end{itemize}

%%%%%%%%%%%%%%%%%%%%%%%%%%%%%%%%%%%%%%%%%%%%%%%%%%%%%%%%%%%%%%%%%%%%%%%%%%%%%%%%
\subsection{Related CTAN Packages}

There are several other packages which offer a similar functionality:
%
\begin{itemize}
\item
The packages
\href{http://ctan.org/pkg/docmute}{\textsf{docmute}},
\href{http://ctan.org/pkg/includex}{\textsf{includex}} and
\href{http://ctan.org/pkg/standalone}{\textsf{standalone}}
provide commands to include only the document body of
a child file thus allowing both files to be compiled individually.
\item
The packages \href{http://ctan.org/pkg/subdocs}{\textsf{subdocs}}
and \href{http://ctan.org/pkg/subfiles}{\textsf{subfiles}}
provide structures in which the main and child documents can be
encapsulated and allowing them to be compiled individually.
The inclusion mechanism is different from the conventional |\include|.
\item
The package \href{http://ctan.org/pkg/combine}{\textsf{combine}}
is an elaborate solution to combine several documents into one.
\end{itemize}
%
See also the CTAN topic \href{http://ctan.org/topic/subdocs}{\textsf{subdocs}}
for further related packages.
The present package differs from the above solutions in that
a document structure constructed with the conventional |\include| mechanism
just needs two extra commands at the top of every file
such that all constituent files can be compiled individually.

%%%%%%%%%%%%%%%%%%%%%%%%%%%%%%%%%%%%%%%%%%%%%%%%%%%%%%%%%%%%%%%%%%%%%%%%%%%%%%%%
%\subsection{Feature Suggestions}
%
%The following is a list of features which may be useful for future
%versions of this package:
%%
%\begin{itemize}
%\item
%\ldots
%\end{itemize}

%%%%%%%%%%%%%%%%%%%%%%%%%%%%%%%%%%%%%%%%%%%%%%%%%%%%%%%%%%%%%%%%%%%%%%%%%%%%%%%%
\subsection{Revision History}

%%%%%%%%%%%%%%%%%%%%%%%%%%%%%%%%%%%%%%%%
\paragraph{v2.0:} 2018/12/30

\begin{itemize}
\item
immediate forward processing
\item
added |\childdocby| mechanism
\item
manual restructured
\end{itemize}

%%%%%%%%%%%%%%%%%%%%%%%%%%%%%%%%%%%%%%%%
\paragraph{v1.6:} 2018/01/17

\begin{itemize}
\item
application for development of include files
\item
corrections to manual
\end{itemize}

%%%%%%%%%%%%%%%%%%%%%%%%%%%%%%%%%%%%%%%%
\paragraph{v1.5:} 2017/05/21

\begin{itemize}
\item
more complete structuring introduced
\item
|\childdocof| introduced
\item
|\childdoc| renamed to |\childdocmain|
\item
|\childredirect| renamed to |\childdocforward| and |\childdocforwardprefix|
and functionality expanded
\end{itemize}

%%%%%%%%%%%%%%%%%%%%%%%%%%%%%%%%%%%%%%%%
\paragraph{v1.0:} 2017/04/27

\begin{itemize}
\item
manual and install package
\item
first version published on CTAN
\end{itemize}

%%%%%%%%%%%%%%%%%%%%%%%%%%%%%%%%%%%%%%%%
\paragraph{v0.6:} 2017/04/26

\begin{itemize}
\item
redirection mechanism added
\end{itemize}

%%%%%%%%%%%%%%%%%%%%%%%%%%%%%%%%%%%%%%%%
\paragraph{v0.5:} 2017/04/26

\begin{itemize}
\item
functionality in definition file
\end{itemize}


%%%%%%%%%%%%%%%%%%%%%%%%%%%%%%%%%%%%%%%%%%%%%%%%%%%%%%%%%%%%%%%%%%%%%%%%%%%%%%%%
%%%%%%%%%%%%%%%%%%%%%%%%%%%%%%%%%%%%%%%%%%%%%%%%%%%%%%%%%%%%%%%%%%%%%%%%%%%%%%%%
%%%%%%%%%%%%%%%%%%%%%%%%%%%%%%%%%%%%%%%%%%%%%%%%%%%%%%%%%%%%%%%%%%%%%%%%%%%%%%%%
\appendix

\settowidth\MacroIndent{\rmfamily\scriptsize 000\ }

 \DocInput{childdoc.dtx}

\end{document}
%</driver>
% \fi
%
% %%%%%%%%%%%%%%%%%%%%%%%%%%%%%%%%%%%%%%%%%%%%%%%%%%%%%%%%%%%%%%%%%%%%%%%%%%%%%%
% %%%%%%%%%%%%%%%%%%%%%%%%%%%%%%%%%%%%%%%%%%%%%%%%%%%%%%%%%%%%%%%%%%%%%%%%%%%%%%
% \section{Sample}
%\iffalse
%<*samplemain>
%\fi
%
% The following presents a sample document
% with two chapters, two parts, a title page,
% a compile flag as well as three forwarding files to set the flag.
% It consists of eight |.tex| files:
% \begin{center}
% \begin{tabular}{ll}
% |cdocsamp.tex|&main file\\
% |cdocsch1.tex|&include file for chapter 1\\
% |cdocsch2.tex|&include file for chapter 2\\
% |cdocspt3.tex|&include file for part 3\\
% |cdocspt4.tex|&include file for part 4\\
% |cdocsdrf.tex|&forwarding file for main file in draft mode\\
% |cdocsfi1.tex|&forwarding file for final version of chapter 1\\
% |cdocsfi2.tex|&forwarding file for final version of chapter 2\\
% \end{tabular}
% \end{center}
% Each of the eight files can be compiled directly by the \LaTeX{} compiler.
%
% %%%%%%%%%%%%%%%%%%%%%%%%%%%%%%%%%%%%%%
% \paragraph{Main File.}
%
% The main file is called |cdocsamp.tex|.
%
% Load the \textsf{childdoc} definitions and
% declare the filename for the main document:
%    \begin{macrocode}
\input{childdoc.def}
\childdocmain{}
%    \end{macrocode}

% Optional override for |\version| flag:
%    \begin{macrocode}
%%\ifchilddoc\else\providecommand{\version}{draft}\fi
%    \end{macrocode}

% Define the default values for the |\version| flag
% (|final| for the main file and |draft| for childs):
%    \begin{macrocode}
\ifchilddoc
\providecommand{\version}{draft}
\else
\providecommand{\version}{final}
\fi
%    \end{macrocode}

% Load the standard document class:
%    \begin{macrocode}
\documentclass[12pt]{article}
%    \end{macrocode}

% Start the document body:
%    \begin{macrocode}
\begin{document}
%    \end{macrocode}

% Declare a title page.
% Print title, part of document being processed and version flag:
%    \begin{macrocode}
\addtocounter{page}{-1}
\begin{center}
{\LARGE\bfseries{}childdoc example\par}
\vspace{1cm}
\ifchilddoc
\ifchilddocmanual part\else chapter\fi:
`\childdocname' of `\childdocjob'\par
\else
main document: `\childdocjob'\par
\fi
version: \version\par
\end{center}
\newpage
%    \end{macrocode}

% Manually include selected file,
% otherwise process as usual:
%    \begin{macrocode}
\ifchilddocmanual
\section*{part `\childdocname'}
\input{\childdocname}
\else
%    \end{macrocode}

% Include the two chapters:
%    \begin{macrocode}
\include{cdocsch1}
\include{cdocsch2}
%    \end{macrocode}

% Include the two parts unless only chapters should be displayed:
%    \begin{macrocode}
\ifchilddoc\else
\section{part three}
\input{cdocspt3}
\section{part four}
\input{cdocspt4}
\fi
%    \end{macrocode}

% Process as usual until here:
%    \begin{macrocode}
\fi
%    \end{macrocode}

% End of document body:
%    \begin{macrocode}
\end{document}
%    \end{macrocode}
%\iffalse
%</samplemain>
%\fi
%
% %%%%%%%%%%%%%%%%%%%%%%%%%%%%%%%%%%%%%%
% \paragraph{Chapter Include Files.}
%
% The include files are called |cdocsch1.tex| and |cdocsch2.tex|.
%
%\iffalse
%<*samplechap1|samplechap2>
%\fi

% Optional override for |\version| flag:
%    \begin{macrocode}
%%\providecommand{\version}{final}
%    \end{macrocode}

% Include the main document:
%    \begin{macrocode}
\input{childdoc.def}
\childdocof{cdocsamp}
%    \end{macrocode}

%\iffalse
%</samplechap1|samplechap2>
%\fi
%
%\iffalse
%<*samplechap1>
%\fi
% Some text for chapter 1:
%    \begin{macrocode}
\section{one}
some text in chapter one
%    \end{macrocode}

%\iffalse
%</samplechap1>
%\fi
% Some text for chapter 2:
%\iffalse
%<*samplechap2>
%\fi
%    \begin{macrocode}
\section{two}
more text in chapter two
%    \end{macrocode}

%\iffalse
%</samplechap2>
%\fi
%
% %%%%%%%%%%%%%%%%%%%%%%%%%%%%%%%%%%%%%%
% \paragraph{Part Include Files.}
%
% The include files are called |cdocspt3.tex| and |cdocspt4.tex|.
%
%\iffalse
%<*samplepart3|samplepart4>
%\fi

% Optional override for |\version| flag:
%    \begin{macrocode}
%%\providecommand{\version}{final}
%    \end{macrocode}

% Include the main document:
%    \begin{macrocode}
\input{childdoc.def}
\childdocby{cdocsamp}
%    \end{macrocode}

%\iffalse
%</samplepart3|samplepart4>
%\fi
%
%\iffalse
%<*samplepart3>
%\fi
% Some text for part 3:
%    \begin{macrocode}
some text in part three
%    \end{macrocode}

%\iffalse
%</samplepart3>
%\fi
% Some text for part 4:
%\iffalse
%<*samplepart4>
%\fi
%    \begin{macrocode}
more text in part four
%    \end{macrocode}

%\iffalse
%</samplepart4>
%\fi
%
% %%%%%%%%%%%%%%%%%%%%%%%%%%%%%%%%%%%%%%
% \paragraph{Forwarding for a Complete Draft.}
%
% The following forwarding file |cdocsdrf.tex|
% compiles the main document in draft mode:
%\iffalse
%<*sampledraft>
%\fi
%    \begin{macrocode}
\def\version{draft}
\input{childdoc.def}
\childdocforward{cdocsamp}
%    \end{macrocode}

%\iffalse
%</sampledraft>
%\fi
%
% %%%%%%%%%%%%%%%%%%%%%%%%%%%%%%%%%%%%%%
% \paragraph{Forwarding for Final Version of the Chapters.}
%
% The following forwarding files |cdocsfn1.tex| and |cdocsfn2.tex|
% (with identical content)
% compile the final versions of the child documents
% |cdocsch1.tex| and |cdocsch2.tex|, respectively:
%\iffalse
%<*samplefinal>
%\fi
%    \begin{macrocode}
\def\version{final}
\input{childdoc.def}
\childdocforwardprefix[cdocsamp]{cdocsfn}{cdocsch}
%    \end{macrocode}

%\iffalse
%</samplefinal>
%\fi
%
% %%%%%%%%%%%%%%%%%%%%%%%%%%%%%%%%%%%%%%
% \paragraph{Command Line Processing.}
%
% The following three command lines generate the output files
% |cdocscld|, |cdocscl1| and |cdocscl2|
% which should be identical to
% |cdocsdrf|, |cdocsch1| and |cdocsfn2|, respectively:
% \begin{center}
% \begin{tabular}{l}
% |latex -jobname cdocscld \|\\
% |  "\def\version{draft}\input{childdoc.def}\childdocforward{cdocsamp}"|\\
% |latex -jobname cdocscl1 \|\\
% |  "\input{childdoc.def}\childdocforward[cdocsamp]{cdocsch1}"|\\
% |latex -jobname cdocscl2 \|\\
% |  "\def\version{final}\input{childdoc.def}\childdocforward{cdocsch2}"|
% \end{tabular}
% \end{center}
% Note that the trailing backslash on each first line
% merely continues the input to the second line
% (for convenient cut ant paste).
% Furthermore, the command |latex| can be replaced by any
% of its alternative versions such as |pdflatex|.
%
% %%%%%%%%%%%%%%%%%%%%%%%%%%%%%%%%%%%%%%%%%%%%%%%%%%%%%%%%%%%%%%%%%%%%%%%%%%%%%%
% %%%%%%%%%%%%%%%%%%%%%%%%%%%%%%%%%%%%%%%%%%%%%%%%%%%%%%%%%%%%%%%%%%%%%%%%%%%%%%
% \section{Implementation}
%\iffalse
%<*package>
%\fi
%
% This section describes the definitions file |childdoc.def|.

% The definitions cannot be loaded using |\usepackage| or |\RequirePackage|
% which has a mechanism to prevent loading a style file more than once.
% When loading the definitions by means of |\input|
% multiple instances have to be prevented manually:
%\iffalse
%This code needs to be before the `\ProvidesFile' directive
%which is defined at the beginning of this file.
%Therefore it is also placed there and commented out here.
%</package>
%<*discard>
%\fi
%    \begin{macrocode}
\ifdefined\childdocmain\endinput\fi
%    \end{macrocode}
%\iffalse
%</discard>
%<*package>
%\fi
%
% \macro{\ifchilddoc}
% \macro{\ifchilddocmanual}
% The conditional |\ifchilddoc| tells whether a
% child (true) or main (false) document is being compiled.
% The conditional |\ifchilddocmanual| tells whether
% the |\includeonly| mechanism is used (false) or
% the selection of child files must be performed manually (true).
% The definitions initialise to false:
%    \begin{macrocode}
\newif\ifchilddoc
\newif\ifchilddocmanual
%    \end{macrocode}

% \macro{\childdocname}
% \macro{\childdocjob}
% The macro |\childdocname| stores the name of the main document
% to be compiled. The macro |\childdocjob| stores the name of
% the document on which the \LaTeX{} compiler was originally invoked.
% The content of |\jobname| cannot be compared
% to filenames specified in the source due to different catcodes.
% The following code rescans |\jobname|, stores the result
% in |\childdocname| and saves a copy in |\childdocjob|:
%    \begin{macrocode}
\edef\childdocname{\scantokens\expandafter{\jobname\noexpand}}
\let\childdocjob\childdocname
%    \end{macrocode}

% \macro{\childdocdisable}
% The macro |\childdocdisable| prevents the main file
% from being processed more than once.
% At this stage, the main document command |\childdocmain|
% is assumed to be called once again where it should do nothing.
% Any subsequent call to it should prevent
% a secondary processing of the main document
% It overwrites the forwarding commands
% |\childdocof| and |\childdocforward|
% with empty macros to prevent further inclusions of the main document:
%    \begin{macrocode}
\newcommand{\childdocdisable}
{
  \renewcommand{\childdocmain}[1]{\renewcommand{\childdocmain}[1]{\endinput}}
  \renewcommand{\childdocof}[1]{}
  \renewcommand{\childdocby}[2][]{}
  \renewcommand{\childdocforward}[2][]{}
  \renewcommand{\childdocdisable}{}
}
%    \end{macrocode}

% \macro{\childdocmain}
% The macro |\childdocmain| is to be called at the top of the main file
% with nothing or the main filename (without extension) as argument.
% First, it breaks loops.
% If the argument is not empty and does not match |\childdocname|
% (which is set by the first inclusion of |childdoc.def|),
% |\ifchilddoc| is set to true, |\includeonly| is applied to the child file
% and |\jobname| is set to the main file
% (for proper handling of |.aux| files):
%    \begin{macrocode}
\newcommand{\childdocmain}[1]
{
  \childdocdisable\childdocmain{}
  \if?#1?\else
    \begingroup
      \def\childdoctmp{#1}
      \ifx\childdoctmp\childdocname
        \def\childdoctmp{}
      \else
        \def\childdoctmp
        {
          \childdoctrue
          \includeonly{\childdocname}
          \def\childdocjob{#1}
          \def\jobname{#1}
        }
      \fi
      \expandafter
    \endgroup
    \childdoctmp
  \fi
}
%    \end{macrocode}

% \macro{\childdocof}
% The command |\childdocof| redirects
% compilation to the main file |#1|.
%    \begin{macrocode}
\newcommand{\childdocof}[1]
{
  \childdocdisable
  \childdoctrue
  \includeonly{\childdocname}
  \def\jobname{#1}
  \def\childdocjob{#1}
  \input{#1}
}
%    \end{macrocode}

% \macro{\childdocby}
% The command |\childdocby| ....
%    \begin{macrocode}
\newcommand{\childdocby}[2][]
{
  \childdocdisable
  \childdoctrue
  \childdocmanualtrue
  \if?#1?\else
    \def\jobname{#2}
  \fi
  \def\childdocjob{#2}
  \input{#2}
  \endinput
}
%    \end{macrocode}

% \macro{\childdocforward}
% The command |\childdocforward| redirects
% compilation to the main file or
% (if the optional argument is given) a child file.
% Parameters are set as if the main file
% or a child file starting with |\childdocof| was compiled.
% Then compilation is handed over to the main file:
%    \begin{macrocode}
\newcommand{\childdocforward}[2][]
{
  \begingroup
    \if?#1?
      \def\childdoctmp
      {
        \def\childdocname{#2}
        \def\childdocjob{#2}
        \def\jobname{#2}
        \input{#2}
        \endinput
      }
    \else
      \def\childdoctmp
      {
        \childdocdisable
        \def\childdocname{#2}
        \childdoctrue
        \includeonly{#2}
        \def\childdocjob{#1}
        \def\jobname{#1}
        \input{#1}
        \endinput
      }
    \fi
    \expandafter
  \endgroup
  \childdoctmp
}
%    \end{macrocode}

% \macro{\childdocforwardprefix}
% The command |\childdocforwardprefix| redirects
% compilation to the main or a child file by means of a pattern.
% The prefix |#1| in the current filename is replaced by |#2|
% and the suffix of the current filename is kept
% (it is assumed that the filename does not contain the substring `|~~~|'
% which is used as a delimiter).
% Compilation is handed over to the new file by |\childdocforward|:
%    \begin{macrocode}
\newcommand{\childdocforwardprefix}[3][]
{
  \begingroup
    \def\childdocextract #2##1~~~{\def\childdoctmp{\childdocforward[#1]{#3##1}}}
    \expandafter\childdocextract\childdocname~~~
    \expandafter
  \endgroup
  \childdoctmp
}
%    \end{macrocode}

% \macro{\childdoc}
% The deprecated macro |\childdoc| is a legacy version of |\childdocmain|:
%    \begin{macrocode}
\newcommand{\childdoc}{\childdocmain}
%    \end{macrocode}

% \macro{\childdocredirect}
% The deprecated macro |\childdocredirect| is a legacy version
% of |\childdocforward| and |\childdocforwardprefix|:
%    \begin{macrocode}
\newcommand{\childdocredirect}[2][]
{
  \begingroup
    \if?#1?
      \def\childdoctmp{\childdocforward{#2}}
    \else
      \def\childdoctmp{\childdocforwardprefix{#1}{#2}}
    \fi
    \expandafter
  \endgroup
  \childdoctmp
}
%    \end{macrocode}

%\iffalse
%</package>
%\fi
%
\endinput

\childdocmain{}
%    \end{macrocode}

% Optional override for |\version| flag:
%    \begin{macrocode}
%%\ifchilddoc\else\providecommand{\version}{draft}\fi
%    \end{macrocode}

% Define the default values for the |\version| flag
% (|final| for the main file and |draft| for childs):
%    \begin{macrocode}
\ifchilddoc
\providecommand{\version}{draft}
\else
\providecommand{\version}{final}
\fi
%    \end{macrocode}

% Load the standard document class:
%    \begin{macrocode}
\documentclass[12pt]{article}
%    \end{macrocode}

% Start the document body:
%    \begin{macrocode}
\begin{document}
%    \end{macrocode}

% Declare a title page.
% Print title, part of document being processed and version flag:
%    \begin{macrocode}
\addtocounter{page}{-1}
\begin{center}
{\LARGE\bfseries{}childdoc example\par}
\vspace{1cm}
\ifchilddoc
\ifchilddocmanual part\else chapter\fi:
`\childdocname' of `\childdocjob'\par
\else
main document: `\childdocjob'\par
\fi
version: \version\par
\end{center}
\newpage
%    \end{macrocode}

% Manually include selected file,
% otherwise process as usual:
%    \begin{macrocode}
\ifchilddocmanual
\section*{part `\childdocname'}
\input{\childdocname}
\else
%    \end{macrocode}

% Include the two chapters:
%    \begin{macrocode}
\include{cdocsch1}
\include{cdocsch2}
%    \end{macrocode}

% Include the two parts unless only chapters should be displayed:
%    \begin{macrocode}
\ifchilddoc\else
\section{part three}
\input{cdocspt3}
\section{part four}
\input{cdocspt4}
\fi
%    \end{macrocode}

% Process as usual until here:
%    \begin{macrocode}
\fi
%    \end{macrocode}

% End of document body:
%    \begin{macrocode}
\end{document}
%    \end{macrocode}
%\iffalse
%</samplemain>
%\fi
%
% %%%%%%%%%%%%%%%%%%%%%%%%%%%%%%%%%%%%%%
% \paragraph{Chapter Include Files.}
%
% The include files are called |cdocsch1.tex| and |cdocsch2.tex|.
%
%\iffalse
%<*samplechap1|samplechap2>
%\fi

% Optional override for |\version| flag:
%    \begin{macrocode}
%%\providecommand{\version}{final}
%    \end{macrocode}

% Include the main document:
%    \begin{macrocode}
% \iffalse
%
% childdoc.dtx Copyright (C) 2017-2018 Niklas Beisert
%
% This work may be distributed and/or modified under the
% conditions of the LaTeX Project Public License, either version 1.3
% of this license or (at your option) any later version.
% The latest version of this license is in
%   http://www.latex-project.org/lppl.txt
% and version 1.3 or later is part of all distributions of LaTeX
% version 2005/12/01 or later.
%
% This work has the LPPL maintenance status `maintained'.
%
% The Current Maintainer of this work is Niklas Beisert.
%
% This work consists of the files childdoc.dtx and childdoc.ins
% and the derived files childdoc.def and cdocsamp.tex with
% cdocsch1.tex, cdocsch2.tex, cdocsdrf.tex, cdocsfn1.tex, cdocsfn2.tex.
%
%<package>\ifdefined\childdocmain\endinput\fi
%<package>\ProvidesFile{childdoc.def}[2018/12/30 v2.0 child document driver]
%<samplemain>\ProvidesFile{cdocsamp.tex}[2018/12/30 v2.0 sample for childdoc]
%<*driver>
%\ProvidesFile{childdoc.drv}[2018/12/30 v2.0 childdoc reference manual file]
\PassOptionsToClass{10pt,a4paper}{article}
\documentclass{ltxdoc}

\usepackage[margin=35mm]{geometry}
\usepackage{hyperref}
\usepackage{hyperxmp}
\usepackage[usenames]{color}

\hypersetup{colorlinks=true}
\hypersetup{pdfstartview=FitH}
\hypersetup{pdfpagemode=UseNone}
\hypersetup{pdfsource={}}
\hypersetup{pdflang={en-UK}}
\hypersetup{pdfcopyright={Copyright 2017-2018 Niklas Beisert.
  This work may be distributed and/or modified under the
  conditions of the LaTeX Project Public License, either version 1.3
  of this license or (at your option) any later version.}}
\hypersetup{pdflicenseurl={http://www.latex-project.org/lppl.txt}}
\hypersetup{pdfcontactaddress={ETH Zurich, ITP, HIT K,
  Wolfgang-Pauli-Strasse 27}}
\hypersetup{pdfcontactpostcode={8093}}
\hypersetup{pdfcontactcity={Zurich}}
\hypersetup{pdfcontactcountry={Switzerland}}
\hypersetup{pdfcontactemail={nbeisert@itp.phys.ethz.ch}}
\hypersetup{pdfcontacturl={http://people.phys.ethz.ch/\xmptilde nbeisert/}}

\newcommand{\secref}[1]{\hyperref[#1]{section \ref*{#1}}}

\parskip1ex
\parindent0pt
\let\olditemize\itemize
\def\itemize{\olditemize\parskip0pt}

\begin{document}

\title{The \textsf{childdoc} Package}
\hypersetup{pdftitle={The childdoc Package}}
\author{Niklas Beisert\\[2ex]
  Institut f\"ur Theoretische Physik\\
  Eidgen\"ossische Technische Hochschule Z\"urich\\
  Wolfgang-Pauli-Strasse 27, 8093 Z\"urich, Switzerland\\[1ex]
  \href{mailto:nbeisert@itp.phys.ethz.ch}
  {\texttt{nbeisert@itp.phys.ethz.ch}}}
\hypersetup{pdfauthor={Niklas Beisert}}
\hypersetup{pdfsubject={Manual for the LaTeX2e Package childdoc}}
\date{30 December 2018, \textsf{v2.0}}
\maketitle

\begin{abstract}\noindent
\textsf{childdoc} is a \LaTeXe{} package
that enables the direct compilation
of document sections included by |\include|
to individual files.
\end{abstract}

\begingroup
\parskip0ex
\tableofcontents
\endgroup

%%%%%%%%%%%%%%%%%%%%%%%%%%%%%%%%%%%%%%%%%%%%%%%%%%%%%%%%%%%%%%%%%%%%%%%%%%%%%%%%
%%%%%%%%%%%%%%%%%%%%%%%%%%%%%%%%%%%%%%%%%%%%%%%%%%%%%%%%%%%%%%%%%%%%%%%%%%%%%%%%
\section{Introduction}

\LaTeX{} provides a mechanism to structure a large document (such as a book)
into a main file and several child files (containing the chapters)
using the |\include| command.
This mechanism is beneficial for documents
which span hundreds of pages in order to
make the source file(s) more manageable.
Moreover, compilation can be restricted to
selected child files by means of the |\includeonly| command.
The latter feature can be used to reduce the compilation time while editing
(this was significantly more useful in the earlier days of \LaTeX{})
or to generate a smaller document which is easier to navigate.
Another application of |\includeonly| is to generate
documents consisting of selected parts of the complete document.

However, there are a few drawbacks of the plain |\include| mechanism:
\begin{itemize}
\item
The child files cannot be compiled on their own,
they can only be compiled via the main file.
A naive editing environment
(such as a text editor with an option
to have the current file processed by \LaTeX)
may require one to switch to the main file before compiling;
attempting to compile the child file produces errors.
\item
The main file must be modified (each time)
to adjust the |\includeonly| command
to the present needs. This easily leaves the main file in a messy state.
\item
The generated document will always carry the filename
of the main document. This is inconvenient if
several child files are to be compiled and
to be kept for distribution.
\end{itemize}

The present package provides a simple interface
to make child files individually compilable by \LaTeX{}.
Compiling a child file then has the same effect as compiling
the main file with an |\includeonly| command
to select the appropriate child.
Moreover the generated document will carry the name of the child
rather than the main file.
This resolves all three above issues.

This feature is meant to make the editing of books,
thesis documents and lecture notes somewhat more convenient.
However, the package can also be used efficiently for
composing a series of documents (such as exercise sheets)
which are typically distributed individually.
It then assists the author in generating the individual documents
(potentially in different versions)
as well as a document containing the collected series.
Another application is in developing style files
or other kinds of included material
where compilation of the style file could redirect
to a sample or test file.

%%%%%%%%%%%%%%%%%%%%%%%%%%%%%%%%%%%%%%%%%%%%%%%%%%%%%%%%%%%%%%%%%%%%%%%%%%%%%%%%
%%%%%%%%%%%%%%%%%%%%%%%%%%%%%%%%%%%%%%%%%%%%%%%%%%%%%%%%%%%%%%%%%%%%%%%%%%%%%%%%
\section{Usage}

First of all, the package \textsf{childdoc} is \emph{not} a standard
\LaTeXe{} |.sty| style file! Therefore it needs to be invoked in
a non-standard way.

%%%%%%%%%%%%%%%%%%%%%%%%%%%%%%%%%%%%%%%%%%%%%%%%%%%%%%%%%%%%%%%%%%%%%%%%%%%%%%%%
\subsection{Included Files}
\label{sec:include}

%%%%%%%%%%%%%%%%%%%%%%%%%%%%%%%%%%%%%%%%
\DescribeMacro{\childdocmain}
To use the package, add the commands
\begin{center}
\begin{tabular}{l}
|\input{childdoc.def}|\\
|\childdocmain{}|\\
\end{tabular}
\end{center}
at the very top of the main \LaTeX{} file,
in particular \emph{before} the |\documentclass| statement!
The argument of |\childdocmain| should be left empty
(but it must be present).

%%%%%%%%%%%%%%%%%%%%%%%%%%%%%%%%%%%%%%%%
\DescribeMacro{\childdocof}
Furthermore, add the commands
\begin{center}
\begin{tabular}{l}
|\input{childdoc.def}|\\
|\childdocof{|\textit{main}|}|\\
\end{tabular}
\end{center}
at the top of every child file \textit{child}
which is included by |\include{|\textit{child}|}|
from within the main file
(or at least for those files to be compiled individually).
The argument \textit{main} must be the filename of the main file.

There are a couple of
considerations in setting up the main and child documents:

%%%%%%%%%%%%%%%%%%%%%%%%%%%%%%%%%%%%%%%%
\paragraph{Restrictions.}

Please note the following restrictions:
\begin{itemize}
\item
|\childdocmain| must be called with one argument \textit{main}
to ensure compatibility with earlier version of the package.
It must either be empty (|\childdocmain{}|)
or precisely match the filename of the main file in which it is specified.
See \secref{sec:detection} for further information.
\item
The filename \textit{main} must be specified without the |.tex| extension.
\item
The filename \textit{main} is case sensitive
(even in case-insensitive file systems)
due to internal string comparison.
\item
The argument \textit{main} should be fully expanded, it cannot be a macro.
\item
Subdirectories and special characters should be avoided in filenames.
\item
The command |\childdocmain{|\textit{main}|}| must be followed by a whitespace.
It should not be followed immediately by another command
or by a comment mark `|%|'.
This is because the \TeX{} parser reads the token immediately following
the argument of |\childdocmain| and puts it
at the beginning of every child section;
however, a white\-space is ignored.
\end{itemize}

%%%%%%%%%%%%%%%%%%%%%%%%%%%%%%%%%%%%%%%%
\paragraph{Content of Main File.}

It is advisable to place all content in the child files included by |\include|.
Any output contained in the main file will appear in all child documents
unless suppressed manually;
it cannot be suppressed automatically by the |\includeonly| directive
and thus should normally be avoided.
A method to include some content in the main file
by means of conditional processing is described in \secref{sec:conditional}.

%%%%%%%%%%%%%%%%%%%%%%%%%%%%%%%%%%%%%%%%
\paragraph{Page Numbering.}

When only a part of the document is compiled,
the appropriate numbering of pages
(as well as other status parameters)
is determined from the |.aux| files.
The latter contain information from previous passes.
However this information needs to propagate through
all intermediate child documents.
Therefore the page numbering in child documents may well
be inconsistent until the complete document is compiled at least once.

A useful (if unconventional) way to always ensure a consistent
page numbering is to restart the numbering in each child document
and denote the pages by `\textit{child}|.|\textit{page}'
where \textit{child} represents the chapter/section number of the child file.
This can be achieved by the command
|\numberwithin{page}{|\textit{child}|}|
of the \textsf{amsmath} package
where \textit{child} can be |chapter| or |section|
depending on the chosen structuring.
Alternatively, one can modify the macro |\thepage| appropriately
and reset the counter |page| at the start of each child file.

%%%%%%%%%%%%%%%%%%%%%%%%%%%%%%%%%%%%%%%%%%%%%%%%%%%%%%%%%%%%%%%%%%%%%%%%%%%%%%%%
\subsection{Conditional Processing}
\label{sec:conditional}

The package provides a mechanism to compile different versions
of a document. To customise the versions further some conditional processing
can come in handy to distinguish which version is being compiled.
The package provides two macros to describe the compilation context:

%%%%%%%%%%%%%%%%%%%%%%%%%%%%%%%%%%%%%%%%
\DescribeMacro{\ifchilddoc}
The conditional |\ifchilddoc| distinguishes between the compilation of
child documents and the main document:
%
\begin{center}
|\ifchilddoc |\textit{child-code}| |[|\||else |\textit{main-code}]| \||fi|
\end{center}

%%%%%%%%%%%%%%%%%%%%%%%%%%%%%%%%%%%%%%%%
\DescribeMacro{\childdocname}
\DescribeMacro{\childdocjob}
The macro |\childdocname| contains the filename (without extension)
of the main or child file being processed.
Note that |\childdocjob| will always contain the name of the main file.

%%%%%%%%%%%%%%%%%%%%%%%%%%%%%%%%%%%%%%%%
\paragraph{Title Page.}

Conditional processing can be used to include a title or banner page
in the main document when proper precautions are taken.
Importantly, the code in the main file should ensure that the page counter
(as well as other status parameters which are stored in the |.aux| files)
takes the same value after the conditional processing.
Otherwise the page numbers may take divergent values
depending on which part is compiled.

For example, a title page could be declared by:
%
\begin{center}
\begin{tabular}{l}
|\ifchilddoc\||else|\\
|\addtocounter{page}{-1}|\\
\textit{code for title page}\\
|\newpage|\\
|\||fi|
\end{tabular}
\end{center}
%
A banner page for the child documents can be generated by:
%
\begin{center}
\begin{tabular}{l}
|\ifchilddoc|\\
|\addtocounter{page}{-1}|\\
\textit{code for banner page}\\
|\newpage|\\
|\||fi|
\end{tabular}
\end{center}
%
Here one could write a message such as:
\begin{center}
|This is the part \childdocname{} of \childdocjob{}.|
\end{center}

%%%%%%%%%%%%%%%%%%%%%%%%%%%%%%%%%%%%%%%%%%%%%%%%%%%%%%%%%%%%%%%%%%%%%%%%%%%%%%%%
\subsection{Flags}
\label{sec:flags}

The package makes it easy to generate different versions
of the main or child documents.
To this end compilation flags can be defined
and assigned different default values.
They will be particularly useful in conjunction
with the forwarding mechanism described in \secref{sec:forward}.

For example, it may be useful to have a flag |\version|
which can be set to |draft| or |final|.
The document source will contain some conditional code
depending on the value of |\version|.
Suppose further, the flag should default to |final| for the main file
and to |draft| for child files
which is a natural assignment for editing the document.
This is achieved by placing the following code
in the preamble of the main document
(below the |\childdocmain| directive):
%
\begin{center}
\begin{tabular}{l}
|\ifchilddoc|\\
|\providecommand{\version}{draft}|\\
|\||else|\\
|\providecommand{\version}{final}|\\
|\||fi|
\end{tabular}
\end{center}
%
The definition by |\providecommand| makes sure
that previous definitions are not overwritten.
Further statements |\providecommand{\version}{...}|
can thus be added before the above code to override it.

For the main file, one might add a line
(between |\childdocmain| and the above block)
%
\begin{center}
|%\ifchilddoc\||else\providecommand{\version}{draft}\||fi|
\end{center}
%
which can be uncommented to produce a draft version.
Likewise one can add a line to the very top of a child file
(above the |\childdocof{|\textit{main}|}| directive)
%
\begin{center}
|%\providecommand{\version}{final}|
\end{center}
%
which can be uncommented to produce the final version of this child document.

%%%%%%%%%%%%%%%%%%%%%%%%%%%%%%%%%%%%%%%%%%%%%%%%%%%%%%%%%%%%%%%%%%%%%%%%%%%%%%%%
\subsection{Forwarding}
\label{sec:forward}

Different versions of the main or child documents
using compilation flags as described in \secref{sec:flags}
can be (permanently) stored in different files
for convenient compilation, viewing and distribution.
To this end, the package defines a command
to pass on compilation to a different file:

%%%%%%%%%%%%%%%%%%%%%%%%%%%%%%%%%%%%%%%%
\DescribeMacro{\childdocforward}
The command |\childdocforward| redirects processing to
another source file:
%
\begin{center}
\begin{tabular}{l}
|\input{childdoc.def}|\\
|\childdocforward[|\textit{main}|]{|\textit{dest}|}|\\
\end{tabular}
\end{center}
%
The argument \textit{dest} is the destination file
(without extension).
It should be the main file or one of the child files.
Note that further \textsf{childdoc} directives
such as |\childdocof| and |\childdocforward|
in the indicated file will be processed in this form.
The optional argument \textit{main}
passes on directly to the main file \textit{main}
while pretending to compile the child \textit{dest}.
This form behaves as if \textit{dest}
issues |\childdocof{|\textit{main}|}| right away,
and no further \textsf{childdoc} directives will be processed.

%%%%%%%%%%%%%%%%%%%%%%%%%%%%%%%%%%%%%%%%
\DescribeMacro{\...prefix}
In the alternative form |\childdocforwardprefix|,
%
\begin{center}
\begin{tabular}{l}
|\input{childdoc.def}|\\
|\childdocforwardprefix[|\textit{main}|]{|\textit{prefix}|}{|\textit{dest}|}|
\end{tabular}
\end{center}
%
the destination file is determined by a pattern
depending on the current file:
To make this work, the current file must be called
`{\textit{prefix}\hspace{0.2em}\textit{suffix}}'
with \textit{prefix} matching precisely the argument.
Processing is then passed on to the file
`{\textit{dest}\hspace{0.2em}\textit{suffix}}'.
Surely, the same effect is achieved by
directly specifying the
argument `{\textit{dest}\hspace{0.2em}\textit{suffix}}'
in the first form.
However, that requires to set up a different file
for each child. With the alternative form of the command
all these files can have exactly the same content
which simplifies setting them up and maintaining them.

For example, the following file |draft.tex|
with a compilation flag |\version| as described in \secref{sec:flags}
compiles the main document as a draft:
%
\begin{center}
\begin{tabular}{l}
|\def\version{draft}|\\
|\input{childdoc.def}|\\
|\childdocforward{|\textit{main}|}|
\end{tabular}
\end{center}
%
Likewise, the following files |final|\textit{nn}|.tex|
compile the final version of the child document
|child|\textit{nn}|.tex|:
%
\begin{center}
\begin{tabular}{l}
|\def\version{final}|\\
|\input{childdoc.def}|\\
|\childdocforwardprefix{final}{child}|
\end{tabular}
\end{center}
%

Note that when several versions of a main file and/or of each child file
are to be generated, it may be convenient to set up a |Makefile| or
shell script to automatise the process.

%%%%%%%%%%%%%%%%%%%%%%%%%%%%%%%%%%%%%%%%%%%%%%%%%%%%%%%%%%%%%%%%%%%%%%%%%%%%%%%%
\subsection{Command Line Processing}
\label{sec:commandline}

The effect of redirection files can also be achieved by invoking
the \LaTeX{} compiler with a more elaborate command line.
Most conveniently this should be done as part
of a shell script or a |Makefile|.

When using \textsf{childdoc} in the main file, the following
command lines effectively perform a redirection
(note that depending on the shell being used,
backslashes may have to be doubled: `|\|' $\to$ `|\\|'):
%
\begin{center}
|... -jobname "|\textit{target}|" |\\|"|[\textit{flags}]%
|\input{childdoc.def}\childdocforward[|\textit{main}|]{|\textit{dest}|}"|
\end{center}
%
Here \textit{target} is the name of the output file,
\textit{main} is the name of the main file
and \textit{dest} is the name of the main or child file to be processed
(all filenames without extensions).
The optional argument \textit{main} can be omitted
if \textit{main} matches \textit{dest}.
Optionally, compilation \textit{flags} can be defined via |\def| commands.
This command line makes the \TeX{} engine believe
it is compiling the file \textit{target}
whose content is specified as the latter parameter.
The provided code then forwards the processing to
\textit{main} or \textit{dest} as described in \secref{sec:forward}.

%%%%%%%%%%%%%%%%%%%%%%%%%%%%%%%%%%%%%%%%%%%%%%%%%%%%%%%%%%%%%%%%%%%%%%%%%%%%%%%%
\subsection{Include by Input}
\label{sec:input}

Including child documents by |\include| has some restrictions by design.
Most notably, the content of a child document always occupies
its own set of pages; pages cannot be shared between child documents.
Usually, this behaviour makes perfect sense
because each child document contain an essential part of the document.
However, in some situations it may be desirable to compose
a document from a collection of parts
without having mandatory page breaks between then.
For this case, the package
provides a mechanism to include parts
by |\input| which can also be processed individually.
However, by construction this mechanism
requires manual handling of the content to be output.

%%%%%%%%%%%%%%%%%%%%%%%%%%%%%%%%%%%%%%%%
\DescribeMacro{\ifchilddocmanual}
The main file should be prepared as usual, see \secref{sec:include}.
However, the document body must make a distinction
between processing of an individual part and of the main document, e.g.:
%
\begin{center}
\begin{tabular}{l}
|\ifchilddocmanual|\\
|\input{\childdocname}|\\
|\||else|\\
\textit{document body with }|\input{|\textit{part}|}|\\
|\||fi|
\end{tabular}
\end{center}
%
The conditional |\ifchilddocmanual| is true whenever
a part to be included by |\input| is being compiled,
and the name of the part is stored in |\childdocname|.

%%%%%%%%%%%%%%%%%%%%%%%%%%%%%%%%%%%%%%%%
\DescribeMacro{\childdocby}
Each part to be included by |\input| should start with:
%
\begin{center}
\begin{tabular}{l}
|\input{childdoc.def}|\\
|\childdocby{|\textit{main}|}|\\
\end{tabular}
\end{center}
%
The directive |\childdocby| is similar to |\childdocof|
described in \secref{sec:include},
but the subsequent selection of content must be done manually.
To that end, both |\ifchilddoc| and |\ifchilddocmanual|
will be true upon processing of a part,
and the name of the part is stored in |\childdocname|.
Note that |\jobname| will be set to the filename of the current part
so that each part receives an individual |.aux| file
that does not interfere with the |.aux| file(s) of the main document.
This behaviour can be altered by the alternative form
|\childdocby[*]{|\textit{main}|}| (with a non-empty optional argument)
which uses the |.aux| file of the main document
by setting |\jobname| to \textit{main}.

%%%%%%%%%%%%%%%%%%%%%%%%%%%%%%%%%%%%%%%%%%%%%%%%%%%%%%%%%%%%%%%%%%%%%%%%%%%%%%%%
\subsection{Driver Development}
\label{sec:driver}

The \textsf{childdoc} mechanism can also be use for the development
of definition files such as \LaTeX{} styles or classes.
This case differs from the above setup with multiple parts
included by |\include| in that no |\includeonly| should be invoked.
This can be achieved by starting the include file
(before |\ProvidesPackage|) with:
%
\begin{center}
\begin{tabular}{l}
|\input{childdoc.def}|\\
|\childdocforward{|\textit{main}|}|\\
\end{tabular}
\end{center}
%
or alternatively with:
%
\begin{center}
\begin{tabular}{l}
|\input{childdoc.def}|\\
|\childdocby{|\textit{main}|}|\\
\end{tabular}
\end{center}
%
Both forms have slightly different effects as described above.
The main file is prepared as usual, see \secref{sec:include}.

%%%%%%%%%%%%%%%%%%%%%%%%%%%%%%%%%%%%%%%%%%%%%%%%%%%%%%%%%%%%%%%%%%%%%%%%%%%%%%%%
\subsection{Legacy Detection}
\label{sec:detection}

The directive |\childdocmain| in the main file can detect
whether the complete document or merely a child is to be compiled
even without using the directive |\childdocof|.
This method is deprecated because it is less robust
and there is no compelling reason to use it;
it is merely provided for backward compatibility
and it may be removed in future versions.

If the detection mechanism is to be used,
it is mandatory to correctly specify
the filename of the main file as the argument of |\childdocmain|:
%
\begin{center}
\begin{tabular}{l}
|\input{childdoc.def}|\\
|\childdocmain{|\textit{main}|}|\\
\end{tabular}
\end{center}
%
If |\jobname| does not match the argument \textit{main} of |\childdocmain|,
it is assumed that |\jobname| points to the child file to be compiled.
When using |\childdocmain| with the main file specified as argument,
it suffices to start a child file
with just |\input{|\textit{main}|}|
without loading of the package and using |\childdocof|.
If instead all processing is done
with the appropriate \textsf{childdoc} directives,
the argument of \textit{main} of |\childdocmain| can be empty.

An alternative version of the command line processing described
in \secref{sec:commandline} using the detection mechanism reads:
%
\begin{center}
|... -jobname "|\textit{target}|" "|[\textit{flags}]%
[|\def\jobname{|\textit{dest}|}|]|\input{|\textit{main}|}"|
\end{center}

%%%%%%%%%%%%%%%%%%%%%%%%%%%%%%%%%%%%%%%%%%%%%%%%%%%%%%%%%%%%%%%%%%%%%%%%%%%%%%%%
\subsection{Manual Code}
\label{sec:manual}

In case one cannot be certain whether the definitions file |childdoc.def|
is installed on the target \TeX{} distribution
and one prefers not to ship it,
it is conceivable to paste a few relevant commands into the sources.

To that end, drop all statements |\input{childdoc.def}|
and perform the replacements as outlined below.
Instead of |\childdocmain{|\textit{main}|}| add the following code
to the top of the main file:
%
\begin{center}
\begin{tabular}{l}
|\||ifdefined\childdocname\endinput\||fi\newif\ifchilddoc|\\
|\edef\childdocname{\scantokens\expandafter{\jobname\noexpand}}|\\
|\def\childdocmain{|\textit{main}|}\||ifx\childdocmain\childdocname\||else|\\
|\childdoctrue\includeonly{\childdocname}\let\jobname\childdocmain\||fi|\\
\end{tabular}
\end{center}
%
Instead of |\childdocof{|\textit{main}|}| just include the main file
at the top of each child file:
%
\begin{center}
|\input{|\textit{main}|}|
\end{center}
%
A simple redirection |\childdocforward{|\textit{dest}|}| is achieved by:
%
\begin{center}
|\def\jobname{|\textit{dest}|}\input{\jobname}|
\end{center}
%
The redirection with prefix
|\childdocforwardprefix[|\textit{prefix}|]{|\textit{dest}|}|
is accomplished by:
%
\begin{center}
\begin{tabular}{l}
|{\edef\jobname{\scantokens\expandafter{\jobname\noexpand}}|\\
|\def\redirectjob |\textit{prefix}|#1~~~{\gdef\jobname{|\textit{dest}|#1}}|\\
|\expandafter\redirectjob\jobname~~~}\input{\jobname}|
\end{tabular}
\end{center}

In an alternative approach,
child documents can be compiled by a specific command line
without additional code or specific definitions:
%
\begin{center}
|... -jobname "|\textit{target}|" "|[\textit{flags}]%
|\includeonly{|\textit{dest}|}\input{|\textit{main}|}"|
\end{center}
%

%%%%%%%%%%%%%%%%%%%%%%%%%%%%%%%%%%%%%%%%%%%%%%%%%%%%%%%%%%%%%%%%%%%%%%%%%%%%%%%%
%%%%%%%%%%%%%%%%%%%%%%%%%%%%%%%%%%%%%%%%%%%%%%%%%%%%%%%%%%%%%%%%%%%%%%%%%%%%%%%%
\section{Information}

%%%%%%%%%%%%%%%%%%%%%%%%%%%%%%%%%%%%%%%%%%%%%%%%%%%%%%%%%%%%%%%%%%%%%%%%%%%%%%%%
\subsection{Copyright}

Copyright \copyright{} 2017--2018 Niklas Beisert

This work may be distributed and/or modified under the
conditions of the \LaTeX{} Project Public License, either version 1.3
of this license or (at your option) any later version.
The latest version of this license is in
  \url{http://www.latex-project.org/lppl.txt}
and version 1.3 or later is part of all distributions of \LaTeX{}
version 2005/12/01 or later.

This work has the LPPL maintenance status `maintained'.

The Current Maintainer of this work is Niklas Beisert.

This work consists of the files |README.txt|, |childdoc.ins| and |childdoc.dtx|
as well as the derived files |childdoc.def|, |cdocsamp.tex|
with |cdocsch1.tex|, |cdocsch2.tex|, |cdocspt3.tex|, |cdocspt4.tex|,
|cdocsdrf.tex|, |cdocsfn1.tex|, |cdocsfn2.tex|
as well as |childdoc.pdf|.

%%%%%%%%%%%%%%%%%%%%%%%%%%%%%%%%%%%%%%%%%%%%%%%%%%%%%%%%%%%%%%%%%%%%%%%%%%%%%%%%
\subsection{Files and Installation}

The package consists of the files:
%
\begin{center}
\begin{tabular}{ll}
    |README.txt|   & readme file \\
    |childdoc.ins| & installation file \\
    |childdoc.dtx| & source file \\
    |childdoc.def| & definition file \\
    |cdocsamp.tex| & sample main file \\
    |cdocsch1.tex| & sample include file \\
    |cdocsch2.tex| & sample include file \\
    |cdocspt3.tex| & sample part file \\
    |cdocspt4.tex| & sample part file \\
    |cdocsdrf.tex| & sample redirection file \\
    |cdocsfn1.tex| & sample redirection file \\
    |cdocsfn2.tex| & sample redirection file \\
    |childdoc.pdf| & manual
\end{tabular}
\end{center}
%
The distribution consists of the files
|README.txt|, |childdoc.ins| and |childdoc.dtx|.
%
\begin{itemize}
\item
Run (pdf)\LaTeX{} on |childdoc.dtx|
to compile the manual |childdoc.pdf| (this file).
\item
Run \LaTeX{} on |childdoc.ins| to create the definitions file |childdoc.def|
and the sample |cdocsamp.tex| with include files
|cdocsch1.tex|, |cdocsch2.tex|, |cdocspt3.tex|, |cdocspt4.tex|,
|cdocsdrf.tex|, |cdocsfn1.tex|, |cdocsfn2.tex|.
Then copy the file |childdoc.def| to an appropriate directory of your \LaTeX{}
distribution, e.g.\ \textit{texmf-root}|/tex/latex/childdoc|.
\end{itemize}

%%%%%%%%%%%%%%%%%%%%%%%%%%%%%%%%%%%%%%%%%%%%%%%%%%%%%%%%%%%%%%%%%%%%%%%%%%%%%%%%
\subsection{Related CTAN Packages}

There are several other packages which offer a similar functionality:
%
\begin{itemize}
\item
The packages
\href{http://ctan.org/pkg/docmute}{\textsf{docmute}},
\href{http://ctan.org/pkg/includex}{\textsf{includex}} and
\href{http://ctan.org/pkg/standalone}{\textsf{standalone}}
provide commands to include only the document body of
a child file thus allowing both files to be compiled individually.
\item
The packages \href{http://ctan.org/pkg/subdocs}{\textsf{subdocs}}
and \href{http://ctan.org/pkg/subfiles}{\textsf{subfiles}}
provide structures in which the main and child documents can be
encapsulated and allowing them to be compiled individually.
The inclusion mechanism is different from the conventional |\include|.
\item
The package \href{http://ctan.org/pkg/combine}{\textsf{combine}}
is an elaborate solution to combine several documents into one.
\end{itemize}
%
See also the CTAN topic \href{http://ctan.org/topic/subdocs}{\textsf{subdocs}}
for further related packages.
The present package differs from the above solutions in that
a document structure constructed with the conventional |\include| mechanism
just needs two extra commands at the top of every file
such that all constituent files can be compiled individually.

%%%%%%%%%%%%%%%%%%%%%%%%%%%%%%%%%%%%%%%%%%%%%%%%%%%%%%%%%%%%%%%%%%%%%%%%%%%%%%%%
%\subsection{Feature Suggestions}
%
%The following is a list of features which may be useful for future
%versions of this package:
%%
%\begin{itemize}
%\item
%\ldots
%\end{itemize}

%%%%%%%%%%%%%%%%%%%%%%%%%%%%%%%%%%%%%%%%%%%%%%%%%%%%%%%%%%%%%%%%%%%%%%%%%%%%%%%%
\subsection{Revision History}

%%%%%%%%%%%%%%%%%%%%%%%%%%%%%%%%%%%%%%%%
\paragraph{v2.0:} 2018/12/30

\begin{itemize}
\item
immediate forward processing
\item
added |\childdocby| mechanism
\item
manual restructured
\end{itemize}

%%%%%%%%%%%%%%%%%%%%%%%%%%%%%%%%%%%%%%%%
\paragraph{v1.6:} 2018/01/17

\begin{itemize}
\item
application for development of include files
\item
corrections to manual
\end{itemize}

%%%%%%%%%%%%%%%%%%%%%%%%%%%%%%%%%%%%%%%%
\paragraph{v1.5:} 2017/05/21

\begin{itemize}
\item
more complete structuring introduced
\item
|\childdocof| introduced
\item
|\childdoc| renamed to |\childdocmain|
\item
|\childredirect| renamed to |\childdocforward| and |\childdocforwardprefix|
and functionality expanded
\end{itemize}

%%%%%%%%%%%%%%%%%%%%%%%%%%%%%%%%%%%%%%%%
\paragraph{v1.0:} 2017/04/27

\begin{itemize}
\item
manual and install package
\item
first version published on CTAN
\end{itemize}

%%%%%%%%%%%%%%%%%%%%%%%%%%%%%%%%%%%%%%%%
\paragraph{v0.6:} 2017/04/26

\begin{itemize}
\item
redirection mechanism added
\end{itemize}

%%%%%%%%%%%%%%%%%%%%%%%%%%%%%%%%%%%%%%%%
\paragraph{v0.5:} 2017/04/26

\begin{itemize}
\item
functionality in definition file
\end{itemize}


%%%%%%%%%%%%%%%%%%%%%%%%%%%%%%%%%%%%%%%%%%%%%%%%%%%%%%%%%%%%%%%%%%%%%%%%%%%%%%%%
%%%%%%%%%%%%%%%%%%%%%%%%%%%%%%%%%%%%%%%%%%%%%%%%%%%%%%%%%%%%%%%%%%%%%%%%%%%%%%%%
%%%%%%%%%%%%%%%%%%%%%%%%%%%%%%%%%%%%%%%%%%%%%%%%%%%%%%%%%%%%%%%%%%%%%%%%%%%%%%%%
\appendix

\settowidth\MacroIndent{\rmfamily\scriptsize 000\ }

 \DocInput{childdoc.dtx}

\end{document}
%</driver>
% \fi
%
% %%%%%%%%%%%%%%%%%%%%%%%%%%%%%%%%%%%%%%%%%%%%%%%%%%%%%%%%%%%%%%%%%%%%%%%%%%%%%%
% %%%%%%%%%%%%%%%%%%%%%%%%%%%%%%%%%%%%%%%%%%%%%%%%%%%%%%%%%%%%%%%%%%%%%%%%%%%%%%
% \section{Sample}
%\iffalse
%<*samplemain>
%\fi
%
% The following presents a sample document
% with two chapters, two parts, a title page,
% a compile flag as well as three forwarding files to set the flag.
% It consists of eight |.tex| files:
% \begin{center}
% \begin{tabular}{ll}
% |cdocsamp.tex|&main file\\
% |cdocsch1.tex|&include file for chapter 1\\
% |cdocsch2.tex|&include file for chapter 2\\
% |cdocspt3.tex|&include file for part 3\\
% |cdocspt4.tex|&include file for part 4\\
% |cdocsdrf.tex|&forwarding file for main file in draft mode\\
% |cdocsfi1.tex|&forwarding file for final version of chapter 1\\
% |cdocsfi2.tex|&forwarding file for final version of chapter 2\\
% \end{tabular}
% \end{center}
% Each of the eight files can be compiled directly by the \LaTeX{} compiler.
%
% %%%%%%%%%%%%%%%%%%%%%%%%%%%%%%%%%%%%%%
% \paragraph{Main File.}
%
% The main file is called |cdocsamp.tex|.
%
% Load the \textsf{childdoc} definitions and
% declare the filename for the main document:
%    \begin{macrocode}
\input{childdoc.def}
\childdocmain{}
%    \end{macrocode}

% Optional override for |\version| flag:
%    \begin{macrocode}
%%\ifchilddoc\else\providecommand{\version}{draft}\fi
%    \end{macrocode}

% Define the default values for the |\version| flag
% (|final| for the main file and |draft| for childs):
%    \begin{macrocode}
\ifchilddoc
\providecommand{\version}{draft}
\else
\providecommand{\version}{final}
\fi
%    \end{macrocode}

% Load the standard document class:
%    \begin{macrocode}
\documentclass[12pt]{article}
%    \end{macrocode}

% Start the document body:
%    \begin{macrocode}
\begin{document}
%    \end{macrocode}

% Declare a title page.
% Print title, part of document being processed and version flag:
%    \begin{macrocode}
\addtocounter{page}{-1}
\begin{center}
{\LARGE\bfseries{}childdoc example\par}
\vspace{1cm}
\ifchilddoc
\ifchilddocmanual part\else chapter\fi:
`\childdocname' of `\childdocjob'\par
\else
main document: `\childdocjob'\par
\fi
version: \version\par
\end{center}
\newpage
%    \end{macrocode}

% Manually include selected file,
% otherwise process as usual:
%    \begin{macrocode}
\ifchilddocmanual
\section*{part `\childdocname'}
\input{\childdocname}
\else
%    \end{macrocode}

% Include the two chapters:
%    \begin{macrocode}
\include{cdocsch1}
\include{cdocsch2}
%    \end{macrocode}

% Include the two parts unless only chapters should be displayed:
%    \begin{macrocode}
\ifchilddoc\else
\section{part three}
\input{cdocspt3}
\section{part four}
\input{cdocspt4}
\fi
%    \end{macrocode}

% Process as usual until here:
%    \begin{macrocode}
\fi
%    \end{macrocode}

% End of document body:
%    \begin{macrocode}
\end{document}
%    \end{macrocode}
%\iffalse
%</samplemain>
%\fi
%
% %%%%%%%%%%%%%%%%%%%%%%%%%%%%%%%%%%%%%%
% \paragraph{Chapter Include Files.}
%
% The include files are called |cdocsch1.tex| and |cdocsch2.tex|.
%
%\iffalse
%<*samplechap1|samplechap2>
%\fi

% Optional override for |\version| flag:
%    \begin{macrocode}
%%\providecommand{\version}{final}
%    \end{macrocode}

% Include the main document:
%    \begin{macrocode}
\input{childdoc.def}
\childdocof{cdocsamp}
%    \end{macrocode}

%\iffalse
%</samplechap1|samplechap2>
%\fi
%
%\iffalse
%<*samplechap1>
%\fi
% Some text for chapter 1:
%    \begin{macrocode}
\section{one}
some text in chapter one
%    \end{macrocode}

%\iffalse
%</samplechap1>
%\fi
% Some text for chapter 2:
%\iffalse
%<*samplechap2>
%\fi
%    \begin{macrocode}
\section{two}
more text in chapter two
%    \end{macrocode}

%\iffalse
%</samplechap2>
%\fi
%
% %%%%%%%%%%%%%%%%%%%%%%%%%%%%%%%%%%%%%%
% \paragraph{Part Include Files.}
%
% The include files are called |cdocspt3.tex| and |cdocspt4.tex|.
%
%\iffalse
%<*samplepart3|samplepart4>
%\fi

% Optional override for |\version| flag:
%    \begin{macrocode}
%%\providecommand{\version}{final}
%    \end{macrocode}

% Include the main document:
%    \begin{macrocode}
\input{childdoc.def}
\childdocby{cdocsamp}
%    \end{macrocode}

%\iffalse
%</samplepart3|samplepart4>
%\fi
%
%\iffalse
%<*samplepart3>
%\fi
% Some text for part 3:
%    \begin{macrocode}
some text in part three
%    \end{macrocode}

%\iffalse
%</samplepart3>
%\fi
% Some text for part 4:
%\iffalse
%<*samplepart4>
%\fi
%    \begin{macrocode}
more text in part four
%    \end{macrocode}

%\iffalse
%</samplepart4>
%\fi
%
% %%%%%%%%%%%%%%%%%%%%%%%%%%%%%%%%%%%%%%
% \paragraph{Forwarding for a Complete Draft.}
%
% The following forwarding file |cdocsdrf.tex|
% compiles the main document in draft mode:
%\iffalse
%<*sampledraft>
%\fi
%    \begin{macrocode}
\def\version{draft}
\input{childdoc.def}
\childdocforward{cdocsamp}
%    \end{macrocode}

%\iffalse
%</sampledraft>
%\fi
%
% %%%%%%%%%%%%%%%%%%%%%%%%%%%%%%%%%%%%%%
% \paragraph{Forwarding for Final Version of the Chapters.}
%
% The following forwarding files |cdocsfn1.tex| and |cdocsfn2.tex|
% (with identical content)
% compile the final versions of the child documents
% |cdocsch1.tex| and |cdocsch2.tex|, respectively:
%\iffalse
%<*samplefinal>
%\fi
%    \begin{macrocode}
\def\version{final}
\input{childdoc.def}
\childdocforwardprefix[cdocsamp]{cdocsfn}{cdocsch}
%    \end{macrocode}

%\iffalse
%</samplefinal>
%\fi
%
% %%%%%%%%%%%%%%%%%%%%%%%%%%%%%%%%%%%%%%
% \paragraph{Command Line Processing.}
%
% The following three command lines generate the output files
% |cdocscld|, |cdocscl1| and |cdocscl2|
% which should be identical to
% |cdocsdrf|, |cdocsch1| and |cdocsfn2|, respectively:
% \begin{center}
% \begin{tabular}{l}
% |latex -jobname cdocscld \|\\
% |  "\def\version{draft}\input{childdoc.def}\childdocforward{cdocsamp}"|\\
% |latex -jobname cdocscl1 \|\\
% |  "\input{childdoc.def}\childdocforward[cdocsamp]{cdocsch1}"|\\
% |latex -jobname cdocscl2 \|\\
% |  "\def\version{final}\input{childdoc.def}\childdocforward{cdocsch2}"|
% \end{tabular}
% \end{center}
% Note that the trailing backslash on each first line
% merely continues the input to the second line
% (for convenient cut ant paste).
% Furthermore, the command |latex| can be replaced by any
% of its alternative versions such as |pdflatex|.
%
% %%%%%%%%%%%%%%%%%%%%%%%%%%%%%%%%%%%%%%%%%%%%%%%%%%%%%%%%%%%%%%%%%%%%%%%%%%%%%%
% %%%%%%%%%%%%%%%%%%%%%%%%%%%%%%%%%%%%%%%%%%%%%%%%%%%%%%%%%%%%%%%%%%%%%%%%%%%%%%
% \section{Implementation}
%\iffalse
%<*package>
%\fi
%
% This section describes the definitions file |childdoc.def|.

% The definitions cannot be loaded using |\usepackage| or |\RequirePackage|
% which has a mechanism to prevent loading a style file more than once.
% When loading the definitions by means of |\input|
% multiple instances have to be prevented manually:
%\iffalse
%This code needs to be before the `\ProvidesFile' directive
%which is defined at the beginning of this file.
%Therefore it is also placed there and commented out here.
%</package>
%<*discard>
%\fi
%    \begin{macrocode}
\ifdefined\childdocmain\endinput\fi
%    \end{macrocode}
%\iffalse
%</discard>
%<*package>
%\fi
%
% \macro{\ifchilddoc}
% \macro{\ifchilddocmanual}
% The conditional |\ifchilddoc| tells whether a
% child (true) or main (false) document is being compiled.
% The conditional |\ifchilddocmanual| tells whether
% the |\includeonly| mechanism is used (false) or
% the selection of child files must be performed manually (true).
% The definitions initialise to false:
%    \begin{macrocode}
\newif\ifchilddoc
\newif\ifchilddocmanual
%    \end{macrocode}

% \macro{\childdocname}
% \macro{\childdocjob}
% The macro |\childdocname| stores the name of the main document
% to be compiled. The macro |\childdocjob| stores the name of
% the document on which the \LaTeX{} compiler was originally invoked.
% The content of |\jobname| cannot be compared
% to filenames specified in the source due to different catcodes.
% The following code rescans |\jobname|, stores the result
% in |\childdocname| and saves a copy in |\childdocjob|:
%    \begin{macrocode}
\edef\childdocname{\scantokens\expandafter{\jobname\noexpand}}
\let\childdocjob\childdocname
%    \end{macrocode}

% \macro{\childdocdisable}
% The macro |\childdocdisable| prevents the main file
% from being processed more than once.
% At this stage, the main document command |\childdocmain|
% is assumed to be called once again where it should do nothing.
% Any subsequent call to it should prevent
% a secondary processing of the main document
% It overwrites the forwarding commands
% |\childdocof| and |\childdocforward|
% with empty macros to prevent further inclusions of the main document:
%    \begin{macrocode}
\newcommand{\childdocdisable}
{
  \renewcommand{\childdocmain}[1]{\renewcommand{\childdocmain}[1]{\endinput}}
  \renewcommand{\childdocof}[1]{}
  \renewcommand{\childdocby}[2][]{}
  \renewcommand{\childdocforward}[2][]{}
  \renewcommand{\childdocdisable}{}
}
%    \end{macrocode}

% \macro{\childdocmain}
% The macro |\childdocmain| is to be called at the top of the main file
% with nothing or the main filename (without extension) as argument.
% First, it breaks loops.
% If the argument is not empty and does not match |\childdocname|
% (which is set by the first inclusion of |childdoc.def|),
% |\ifchilddoc| is set to true, |\includeonly| is applied to the child file
% and |\jobname| is set to the main file
% (for proper handling of |.aux| files):
%    \begin{macrocode}
\newcommand{\childdocmain}[1]
{
  \childdocdisable\childdocmain{}
  \if?#1?\else
    \begingroup
      \def\childdoctmp{#1}
      \ifx\childdoctmp\childdocname
        \def\childdoctmp{}
      \else
        \def\childdoctmp
        {
          \childdoctrue
          \includeonly{\childdocname}
          \def\childdocjob{#1}
          \def\jobname{#1}
        }
      \fi
      \expandafter
    \endgroup
    \childdoctmp
  \fi
}
%    \end{macrocode}

% \macro{\childdocof}
% The command |\childdocof| redirects
% compilation to the main file |#1|.
%    \begin{macrocode}
\newcommand{\childdocof}[1]
{
  \childdocdisable
  \childdoctrue
  \includeonly{\childdocname}
  \def\jobname{#1}
  \def\childdocjob{#1}
  \input{#1}
}
%    \end{macrocode}

% \macro{\childdocby}
% The command |\childdocby| ....
%    \begin{macrocode}
\newcommand{\childdocby}[2][]
{
  \childdocdisable
  \childdoctrue
  \childdocmanualtrue
  \if?#1?\else
    \def\jobname{#2}
  \fi
  \def\childdocjob{#2}
  \input{#2}
  \endinput
}
%    \end{macrocode}

% \macro{\childdocforward}
% The command |\childdocforward| redirects
% compilation to the main file or
% (if the optional argument is given) a child file.
% Parameters are set as if the main file
% or a child file starting with |\childdocof| was compiled.
% Then compilation is handed over to the main file:
%    \begin{macrocode}
\newcommand{\childdocforward}[2][]
{
  \begingroup
    \if?#1?
      \def\childdoctmp
      {
        \def\childdocname{#2}
        \def\childdocjob{#2}
        \def\jobname{#2}
        \input{#2}
        \endinput
      }
    \else
      \def\childdoctmp
      {
        \childdocdisable
        \def\childdocname{#2}
        \childdoctrue
        \includeonly{#2}
        \def\childdocjob{#1}
        \def\jobname{#1}
        \input{#1}
        \endinput
      }
    \fi
    \expandafter
  \endgroup
  \childdoctmp
}
%    \end{macrocode}

% \macro{\childdocforwardprefix}
% The command |\childdocforwardprefix| redirects
% compilation to the main or a child file by means of a pattern.
% The prefix |#1| in the current filename is replaced by |#2|
% and the suffix of the current filename is kept
% (it is assumed that the filename does not contain the substring `|~~~|'
% which is used as a delimiter).
% Compilation is handed over to the new file by |\childdocforward|:
%    \begin{macrocode}
\newcommand{\childdocforwardprefix}[3][]
{
  \begingroup
    \def\childdocextract #2##1~~~{\def\childdoctmp{\childdocforward[#1]{#3##1}}}
    \expandafter\childdocextract\childdocname~~~
    \expandafter
  \endgroup
  \childdoctmp
}
%    \end{macrocode}

% \macro{\childdoc}
% The deprecated macro |\childdoc| is a legacy version of |\childdocmain|:
%    \begin{macrocode}
\newcommand{\childdoc}{\childdocmain}
%    \end{macrocode}

% \macro{\childdocredirect}
% The deprecated macro |\childdocredirect| is a legacy version
% of |\childdocforward| and |\childdocforwardprefix|:
%    \begin{macrocode}
\newcommand{\childdocredirect}[2][]
{
  \begingroup
    \if?#1?
      \def\childdoctmp{\childdocforward{#2}}
    \else
      \def\childdoctmp{\childdocforwardprefix{#1}{#2}}
    \fi
    \expandafter
  \endgroup
  \childdoctmp
}
%    \end{macrocode}

%\iffalse
%</package>
%\fi
%
\endinput

\childdocof{cdocsamp}
%    \end{macrocode}

%\iffalse
%</samplechap1|samplechap2>
%\fi
%
%\iffalse
%<*samplechap1>
%\fi
% Some text for chapter 1:
%    \begin{macrocode}
\section{one}
some text in chapter one
%    \end{macrocode}

%\iffalse
%</samplechap1>
%\fi
% Some text for chapter 2:
%\iffalse
%<*samplechap2>
%\fi
%    \begin{macrocode}
\section{two}
more text in chapter two
%    \end{macrocode}

%\iffalse
%</samplechap2>
%\fi
%
% %%%%%%%%%%%%%%%%%%%%%%%%%%%%%%%%%%%%%%
% \paragraph{Part Include Files.}
%
% The include files are called |cdocspt3.tex| and |cdocspt4.tex|.
%
%\iffalse
%<*samplepart3|samplepart4>
%\fi

% Optional override for |\version| flag:
%    \begin{macrocode}
%%\providecommand{\version}{final}
%    \end{macrocode}

% Include the main document:
%    \begin{macrocode}
% \iffalse
%
% childdoc.dtx Copyright (C) 2017-2018 Niklas Beisert
%
% This work may be distributed and/or modified under the
% conditions of the LaTeX Project Public License, either version 1.3
% of this license or (at your option) any later version.
% The latest version of this license is in
%   http://www.latex-project.org/lppl.txt
% and version 1.3 or later is part of all distributions of LaTeX
% version 2005/12/01 or later.
%
% This work has the LPPL maintenance status `maintained'.
%
% The Current Maintainer of this work is Niklas Beisert.
%
% This work consists of the files childdoc.dtx and childdoc.ins
% and the derived files childdoc.def and cdocsamp.tex with
% cdocsch1.tex, cdocsch2.tex, cdocsdrf.tex, cdocsfn1.tex, cdocsfn2.tex.
%
%<package>\ifdefined\childdocmain\endinput\fi
%<package>\ProvidesFile{childdoc.def}[2018/12/30 v2.0 child document driver]
%<samplemain>\ProvidesFile{cdocsamp.tex}[2018/12/30 v2.0 sample for childdoc]
%<*driver>
%\ProvidesFile{childdoc.drv}[2018/12/30 v2.0 childdoc reference manual file]
\PassOptionsToClass{10pt,a4paper}{article}
\documentclass{ltxdoc}

\usepackage[margin=35mm]{geometry}
\usepackage{hyperref}
\usepackage{hyperxmp}
\usepackage[usenames]{color}

\hypersetup{colorlinks=true}
\hypersetup{pdfstartview=FitH}
\hypersetup{pdfpagemode=UseNone}
\hypersetup{pdfsource={}}
\hypersetup{pdflang={en-UK}}
\hypersetup{pdfcopyright={Copyright 2017-2018 Niklas Beisert.
  This work may be distributed and/or modified under the
  conditions of the LaTeX Project Public License, either version 1.3
  of this license or (at your option) any later version.}}
\hypersetup{pdflicenseurl={http://www.latex-project.org/lppl.txt}}
\hypersetup{pdfcontactaddress={ETH Zurich, ITP, HIT K,
  Wolfgang-Pauli-Strasse 27}}
\hypersetup{pdfcontactpostcode={8093}}
\hypersetup{pdfcontactcity={Zurich}}
\hypersetup{pdfcontactcountry={Switzerland}}
\hypersetup{pdfcontactemail={nbeisert@itp.phys.ethz.ch}}
\hypersetup{pdfcontacturl={http://people.phys.ethz.ch/\xmptilde nbeisert/}}

\newcommand{\secref}[1]{\hyperref[#1]{section \ref*{#1}}}

\parskip1ex
\parindent0pt
\let\olditemize\itemize
\def\itemize{\olditemize\parskip0pt}

\begin{document}

\title{The \textsf{childdoc} Package}
\hypersetup{pdftitle={The childdoc Package}}
\author{Niklas Beisert\\[2ex]
  Institut f\"ur Theoretische Physik\\
  Eidgen\"ossische Technische Hochschule Z\"urich\\
  Wolfgang-Pauli-Strasse 27, 8093 Z\"urich, Switzerland\\[1ex]
  \href{mailto:nbeisert@itp.phys.ethz.ch}
  {\texttt{nbeisert@itp.phys.ethz.ch}}}
\hypersetup{pdfauthor={Niklas Beisert}}
\hypersetup{pdfsubject={Manual for the LaTeX2e Package childdoc}}
\date{30 December 2018, \textsf{v2.0}}
\maketitle

\begin{abstract}\noindent
\textsf{childdoc} is a \LaTeXe{} package
that enables the direct compilation
of document sections included by |\include|
to individual files.
\end{abstract}

\begingroup
\parskip0ex
\tableofcontents
\endgroup

%%%%%%%%%%%%%%%%%%%%%%%%%%%%%%%%%%%%%%%%%%%%%%%%%%%%%%%%%%%%%%%%%%%%%%%%%%%%%%%%
%%%%%%%%%%%%%%%%%%%%%%%%%%%%%%%%%%%%%%%%%%%%%%%%%%%%%%%%%%%%%%%%%%%%%%%%%%%%%%%%
\section{Introduction}

\LaTeX{} provides a mechanism to structure a large document (such as a book)
into a main file and several child files (containing the chapters)
using the |\include| command.
This mechanism is beneficial for documents
which span hundreds of pages in order to
make the source file(s) more manageable.
Moreover, compilation can be restricted to
selected child files by means of the |\includeonly| command.
The latter feature can be used to reduce the compilation time while editing
(this was significantly more useful in the earlier days of \LaTeX{})
or to generate a smaller document which is easier to navigate.
Another application of |\includeonly| is to generate
documents consisting of selected parts of the complete document.

However, there are a few drawbacks of the plain |\include| mechanism:
\begin{itemize}
\item
The child files cannot be compiled on their own,
they can only be compiled via the main file.
A naive editing environment
(such as a text editor with an option
to have the current file processed by \LaTeX)
may require one to switch to the main file before compiling;
attempting to compile the child file produces errors.
\item
The main file must be modified (each time)
to adjust the |\includeonly| command
to the present needs. This easily leaves the main file in a messy state.
\item
The generated document will always carry the filename
of the main document. This is inconvenient if
several child files are to be compiled and
to be kept for distribution.
\end{itemize}

The present package provides a simple interface
to make child files individually compilable by \LaTeX{}.
Compiling a child file then has the same effect as compiling
the main file with an |\includeonly| command
to select the appropriate child.
Moreover the generated document will carry the name of the child
rather than the main file.
This resolves all three above issues.

This feature is meant to make the editing of books,
thesis documents and lecture notes somewhat more convenient.
However, the package can also be used efficiently for
composing a series of documents (such as exercise sheets)
which are typically distributed individually.
It then assists the author in generating the individual documents
(potentially in different versions)
as well as a document containing the collected series.
Another application is in developing style files
or other kinds of included material
where compilation of the style file could redirect
to a sample or test file.

%%%%%%%%%%%%%%%%%%%%%%%%%%%%%%%%%%%%%%%%%%%%%%%%%%%%%%%%%%%%%%%%%%%%%%%%%%%%%%%%
%%%%%%%%%%%%%%%%%%%%%%%%%%%%%%%%%%%%%%%%%%%%%%%%%%%%%%%%%%%%%%%%%%%%%%%%%%%%%%%%
\section{Usage}

First of all, the package \textsf{childdoc} is \emph{not} a standard
\LaTeXe{} |.sty| style file! Therefore it needs to be invoked in
a non-standard way.

%%%%%%%%%%%%%%%%%%%%%%%%%%%%%%%%%%%%%%%%%%%%%%%%%%%%%%%%%%%%%%%%%%%%%%%%%%%%%%%%
\subsection{Included Files}
\label{sec:include}

%%%%%%%%%%%%%%%%%%%%%%%%%%%%%%%%%%%%%%%%
\DescribeMacro{\childdocmain}
To use the package, add the commands
\begin{center}
\begin{tabular}{l}
|\input{childdoc.def}|\\
|\childdocmain{}|\\
\end{tabular}
\end{center}
at the very top of the main \LaTeX{} file,
in particular \emph{before} the |\documentclass| statement!
The argument of |\childdocmain| should be left empty
(but it must be present).

%%%%%%%%%%%%%%%%%%%%%%%%%%%%%%%%%%%%%%%%
\DescribeMacro{\childdocof}
Furthermore, add the commands
\begin{center}
\begin{tabular}{l}
|\input{childdoc.def}|\\
|\childdocof{|\textit{main}|}|\\
\end{tabular}
\end{center}
at the top of every child file \textit{child}
which is included by |\include{|\textit{child}|}|
from within the main file
(or at least for those files to be compiled individually).
The argument \textit{main} must be the filename of the main file.

There are a couple of
considerations in setting up the main and child documents:

%%%%%%%%%%%%%%%%%%%%%%%%%%%%%%%%%%%%%%%%
\paragraph{Restrictions.}

Please note the following restrictions:
\begin{itemize}
\item
|\childdocmain| must be called with one argument \textit{main}
to ensure compatibility with earlier version of the package.
It must either be empty (|\childdocmain{}|)
or precisely match the filename of the main file in which it is specified.
See \secref{sec:detection} for further information.
\item
The filename \textit{main} must be specified without the |.tex| extension.
\item
The filename \textit{main} is case sensitive
(even in case-insensitive file systems)
due to internal string comparison.
\item
The argument \textit{main} should be fully expanded, it cannot be a macro.
\item
Subdirectories and special characters should be avoided in filenames.
\item
The command |\childdocmain{|\textit{main}|}| must be followed by a whitespace.
It should not be followed immediately by another command
or by a comment mark `|%|'.
This is because the \TeX{} parser reads the token immediately following
the argument of |\childdocmain| and puts it
at the beginning of every child section;
however, a white\-space is ignored.
\end{itemize}

%%%%%%%%%%%%%%%%%%%%%%%%%%%%%%%%%%%%%%%%
\paragraph{Content of Main File.}

It is advisable to place all content in the child files included by |\include|.
Any output contained in the main file will appear in all child documents
unless suppressed manually;
it cannot be suppressed automatically by the |\includeonly| directive
and thus should normally be avoided.
A method to include some content in the main file
by means of conditional processing is described in \secref{sec:conditional}.

%%%%%%%%%%%%%%%%%%%%%%%%%%%%%%%%%%%%%%%%
\paragraph{Page Numbering.}

When only a part of the document is compiled,
the appropriate numbering of pages
(as well as other status parameters)
is determined from the |.aux| files.
The latter contain information from previous passes.
However this information needs to propagate through
all intermediate child documents.
Therefore the page numbering in child documents may well
be inconsistent until the complete document is compiled at least once.

A useful (if unconventional) way to always ensure a consistent
page numbering is to restart the numbering in each child document
and denote the pages by `\textit{child}|.|\textit{page}'
where \textit{child} represents the chapter/section number of the child file.
This can be achieved by the command
|\numberwithin{page}{|\textit{child}|}|
of the \textsf{amsmath} package
where \textit{child} can be |chapter| or |section|
depending on the chosen structuring.
Alternatively, one can modify the macro |\thepage| appropriately
and reset the counter |page| at the start of each child file.

%%%%%%%%%%%%%%%%%%%%%%%%%%%%%%%%%%%%%%%%%%%%%%%%%%%%%%%%%%%%%%%%%%%%%%%%%%%%%%%%
\subsection{Conditional Processing}
\label{sec:conditional}

The package provides a mechanism to compile different versions
of a document. To customise the versions further some conditional processing
can come in handy to distinguish which version is being compiled.
The package provides two macros to describe the compilation context:

%%%%%%%%%%%%%%%%%%%%%%%%%%%%%%%%%%%%%%%%
\DescribeMacro{\ifchilddoc}
The conditional |\ifchilddoc| distinguishes between the compilation of
child documents and the main document:
%
\begin{center}
|\ifchilddoc |\textit{child-code}| |[|\||else |\textit{main-code}]| \||fi|
\end{center}

%%%%%%%%%%%%%%%%%%%%%%%%%%%%%%%%%%%%%%%%
\DescribeMacro{\childdocname}
\DescribeMacro{\childdocjob}
The macro |\childdocname| contains the filename (without extension)
of the main or child file being processed.
Note that |\childdocjob| will always contain the name of the main file.

%%%%%%%%%%%%%%%%%%%%%%%%%%%%%%%%%%%%%%%%
\paragraph{Title Page.}

Conditional processing can be used to include a title or banner page
in the main document when proper precautions are taken.
Importantly, the code in the main file should ensure that the page counter
(as well as other status parameters which are stored in the |.aux| files)
takes the same value after the conditional processing.
Otherwise the page numbers may take divergent values
depending on which part is compiled.

For example, a title page could be declared by:
%
\begin{center}
\begin{tabular}{l}
|\ifchilddoc\||else|\\
|\addtocounter{page}{-1}|\\
\textit{code for title page}\\
|\newpage|\\
|\||fi|
\end{tabular}
\end{center}
%
A banner page for the child documents can be generated by:
%
\begin{center}
\begin{tabular}{l}
|\ifchilddoc|\\
|\addtocounter{page}{-1}|\\
\textit{code for banner page}\\
|\newpage|\\
|\||fi|
\end{tabular}
\end{center}
%
Here one could write a message such as:
\begin{center}
|This is the part \childdocname{} of \childdocjob{}.|
\end{center}

%%%%%%%%%%%%%%%%%%%%%%%%%%%%%%%%%%%%%%%%%%%%%%%%%%%%%%%%%%%%%%%%%%%%%%%%%%%%%%%%
\subsection{Flags}
\label{sec:flags}

The package makes it easy to generate different versions
of the main or child documents.
To this end compilation flags can be defined
and assigned different default values.
They will be particularly useful in conjunction
with the forwarding mechanism described in \secref{sec:forward}.

For example, it may be useful to have a flag |\version|
which can be set to |draft| or |final|.
The document source will contain some conditional code
depending on the value of |\version|.
Suppose further, the flag should default to |final| for the main file
and to |draft| for child files
which is a natural assignment for editing the document.
This is achieved by placing the following code
in the preamble of the main document
(below the |\childdocmain| directive):
%
\begin{center}
\begin{tabular}{l}
|\ifchilddoc|\\
|\providecommand{\version}{draft}|\\
|\||else|\\
|\providecommand{\version}{final}|\\
|\||fi|
\end{tabular}
\end{center}
%
The definition by |\providecommand| makes sure
that previous definitions are not overwritten.
Further statements |\providecommand{\version}{...}|
can thus be added before the above code to override it.

For the main file, one might add a line
(between |\childdocmain| and the above block)
%
\begin{center}
|%\ifchilddoc\||else\providecommand{\version}{draft}\||fi|
\end{center}
%
which can be uncommented to produce a draft version.
Likewise one can add a line to the very top of a child file
(above the |\childdocof{|\textit{main}|}| directive)
%
\begin{center}
|%\providecommand{\version}{final}|
\end{center}
%
which can be uncommented to produce the final version of this child document.

%%%%%%%%%%%%%%%%%%%%%%%%%%%%%%%%%%%%%%%%%%%%%%%%%%%%%%%%%%%%%%%%%%%%%%%%%%%%%%%%
\subsection{Forwarding}
\label{sec:forward}

Different versions of the main or child documents
using compilation flags as described in \secref{sec:flags}
can be (permanently) stored in different files
for convenient compilation, viewing and distribution.
To this end, the package defines a command
to pass on compilation to a different file:

%%%%%%%%%%%%%%%%%%%%%%%%%%%%%%%%%%%%%%%%
\DescribeMacro{\childdocforward}
The command |\childdocforward| redirects processing to
another source file:
%
\begin{center}
\begin{tabular}{l}
|\input{childdoc.def}|\\
|\childdocforward[|\textit{main}|]{|\textit{dest}|}|\\
\end{tabular}
\end{center}
%
The argument \textit{dest} is the destination file
(without extension).
It should be the main file or one of the child files.
Note that further \textsf{childdoc} directives
such as |\childdocof| and |\childdocforward|
in the indicated file will be processed in this form.
The optional argument \textit{main}
passes on directly to the main file \textit{main}
while pretending to compile the child \textit{dest}.
This form behaves as if \textit{dest}
issues |\childdocof{|\textit{main}|}| right away,
and no further \textsf{childdoc} directives will be processed.

%%%%%%%%%%%%%%%%%%%%%%%%%%%%%%%%%%%%%%%%
\DescribeMacro{\...prefix}
In the alternative form |\childdocforwardprefix|,
%
\begin{center}
\begin{tabular}{l}
|\input{childdoc.def}|\\
|\childdocforwardprefix[|\textit{main}|]{|\textit{prefix}|}{|\textit{dest}|}|
\end{tabular}
\end{center}
%
the destination file is determined by a pattern
depending on the current file:
To make this work, the current file must be called
`{\textit{prefix}\hspace{0.2em}\textit{suffix}}'
with \textit{prefix} matching precisely the argument.
Processing is then passed on to the file
`{\textit{dest}\hspace{0.2em}\textit{suffix}}'.
Surely, the same effect is achieved by
directly specifying the
argument `{\textit{dest}\hspace{0.2em}\textit{suffix}}'
in the first form.
However, that requires to set up a different file
for each child. With the alternative form of the command
all these files can have exactly the same content
which simplifies setting them up and maintaining them.

For example, the following file |draft.tex|
with a compilation flag |\version| as described in \secref{sec:flags}
compiles the main document as a draft:
%
\begin{center}
\begin{tabular}{l}
|\def\version{draft}|\\
|\input{childdoc.def}|\\
|\childdocforward{|\textit{main}|}|
\end{tabular}
\end{center}
%
Likewise, the following files |final|\textit{nn}|.tex|
compile the final version of the child document
|child|\textit{nn}|.tex|:
%
\begin{center}
\begin{tabular}{l}
|\def\version{final}|\\
|\input{childdoc.def}|\\
|\childdocforwardprefix{final}{child}|
\end{tabular}
\end{center}
%

Note that when several versions of a main file and/or of each child file
are to be generated, it may be convenient to set up a |Makefile| or
shell script to automatise the process.

%%%%%%%%%%%%%%%%%%%%%%%%%%%%%%%%%%%%%%%%%%%%%%%%%%%%%%%%%%%%%%%%%%%%%%%%%%%%%%%%
\subsection{Command Line Processing}
\label{sec:commandline}

The effect of redirection files can also be achieved by invoking
the \LaTeX{} compiler with a more elaborate command line.
Most conveniently this should be done as part
of a shell script or a |Makefile|.

When using \textsf{childdoc} in the main file, the following
command lines effectively perform a redirection
(note that depending on the shell being used,
backslashes may have to be doubled: `|\|' $\to$ `|\\|'):
%
\begin{center}
|... -jobname "|\textit{target}|" |\\|"|[\textit{flags}]%
|\input{childdoc.def}\childdocforward[|\textit{main}|]{|\textit{dest}|}"|
\end{center}
%
Here \textit{target} is the name of the output file,
\textit{main} is the name of the main file
and \textit{dest} is the name of the main or child file to be processed
(all filenames without extensions).
The optional argument \textit{main} can be omitted
if \textit{main} matches \textit{dest}.
Optionally, compilation \textit{flags} can be defined via |\def| commands.
This command line makes the \TeX{} engine believe
it is compiling the file \textit{target}
whose content is specified as the latter parameter.
The provided code then forwards the processing to
\textit{main} or \textit{dest} as described in \secref{sec:forward}.

%%%%%%%%%%%%%%%%%%%%%%%%%%%%%%%%%%%%%%%%%%%%%%%%%%%%%%%%%%%%%%%%%%%%%%%%%%%%%%%%
\subsection{Include by Input}
\label{sec:input}

Including child documents by |\include| has some restrictions by design.
Most notably, the content of a child document always occupies
its own set of pages; pages cannot be shared between child documents.
Usually, this behaviour makes perfect sense
because each child document contain an essential part of the document.
However, in some situations it may be desirable to compose
a document from a collection of parts
without having mandatory page breaks between then.
For this case, the package
provides a mechanism to include parts
by |\input| which can also be processed individually.
However, by construction this mechanism
requires manual handling of the content to be output.

%%%%%%%%%%%%%%%%%%%%%%%%%%%%%%%%%%%%%%%%
\DescribeMacro{\ifchilddocmanual}
The main file should be prepared as usual, see \secref{sec:include}.
However, the document body must make a distinction
between processing of an individual part and of the main document, e.g.:
%
\begin{center}
\begin{tabular}{l}
|\ifchilddocmanual|\\
|\input{\childdocname}|\\
|\||else|\\
\textit{document body with }|\input{|\textit{part}|}|\\
|\||fi|
\end{tabular}
\end{center}
%
The conditional |\ifchilddocmanual| is true whenever
a part to be included by |\input| is being compiled,
and the name of the part is stored in |\childdocname|.

%%%%%%%%%%%%%%%%%%%%%%%%%%%%%%%%%%%%%%%%
\DescribeMacro{\childdocby}
Each part to be included by |\input| should start with:
%
\begin{center}
\begin{tabular}{l}
|\input{childdoc.def}|\\
|\childdocby{|\textit{main}|}|\\
\end{tabular}
\end{center}
%
The directive |\childdocby| is similar to |\childdocof|
described in \secref{sec:include},
but the subsequent selection of content must be done manually.
To that end, both |\ifchilddoc| and |\ifchilddocmanual|
will be true upon processing of a part,
and the name of the part is stored in |\childdocname|.
Note that |\jobname| will be set to the filename of the current part
so that each part receives an individual |.aux| file
that does not interfere with the |.aux| file(s) of the main document.
This behaviour can be altered by the alternative form
|\childdocby[*]{|\textit{main}|}| (with a non-empty optional argument)
which uses the |.aux| file of the main document
by setting |\jobname| to \textit{main}.

%%%%%%%%%%%%%%%%%%%%%%%%%%%%%%%%%%%%%%%%%%%%%%%%%%%%%%%%%%%%%%%%%%%%%%%%%%%%%%%%
\subsection{Driver Development}
\label{sec:driver}

The \textsf{childdoc} mechanism can also be use for the development
of definition files such as \LaTeX{} styles or classes.
This case differs from the above setup with multiple parts
included by |\include| in that no |\includeonly| should be invoked.
This can be achieved by starting the include file
(before |\ProvidesPackage|) with:
%
\begin{center}
\begin{tabular}{l}
|\input{childdoc.def}|\\
|\childdocforward{|\textit{main}|}|\\
\end{tabular}
\end{center}
%
or alternatively with:
%
\begin{center}
\begin{tabular}{l}
|\input{childdoc.def}|\\
|\childdocby{|\textit{main}|}|\\
\end{tabular}
\end{center}
%
Both forms have slightly different effects as described above.
The main file is prepared as usual, see \secref{sec:include}.

%%%%%%%%%%%%%%%%%%%%%%%%%%%%%%%%%%%%%%%%%%%%%%%%%%%%%%%%%%%%%%%%%%%%%%%%%%%%%%%%
\subsection{Legacy Detection}
\label{sec:detection}

The directive |\childdocmain| in the main file can detect
whether the complete document or merely a child is to be compiled
even without using the directive |\childdocof|.
This method is deprecated because it is less robust
and there is no compelling reason to use it;
it is merely provided for backward compatibility
and it may be removed in future versions.

If the detection mechanism is to be used,
it is mandatory to correctly specify
the filename of the main file as the argument of |\childdocmain|:
%
\begin{center}
\begin{tabular}{l}
|\input{childdoc.def}|\\
|\childdocmain{|\textit{main}|}|\\
\end{tabular}
\end{center}
%
If |\jobname| does not match the argument \textit{main} of |\childdocmain|,
it is assumed that |\jobname| points to the child file to be compiled.
When using |\childdocmain| with the main file specified as argument,
it suffices to start a child file
with just |\input{|\textit{main}|}|
without loading of the package and using |\childdocof|.
If instead all processing is done
with the appropriate \textsf{childdoc} directives,
the argument of \textit{main} of |\childdocmain| can be empty.

An alternative version of the command line processing described
in \secref{sec:commandline} using the detection mechanism reads:
%
\begin{center}
|... -jobname "|\textit{target}|" "|[\textit{flags}]%
[|\def\jobname{|\textit{dest}|}|]|\input{|\textit{main}|}"|
\end{center}

%%%%%%%%%%%%%%%%%%%%%%%%%%%%%%%%%%%%%%%%%%%%%%%%%%%%%%%%%%%%%%%%%%%%%%%%%%%%%%%%
\subsection{Manual Code}
\label{sec:manual}

In case one cannot be certain whether the definitions file |childdoc.def|
is installed on the target \TeX{} distribution
and one prefers not to ship it,
it is conceivable to paste a few relevant commands into the sources.

To that end, drop all statements |\input{childdoc.def}|
and perform the replacements as outlined below.
Instead of |\childdocmain{|\textit{main}|}| add the following code
to the top of the main file:
%
\begin{center}
\begin{tabular}{l}
|\||ifdefined\childdocname\endinput\||fi\newif\ifchilddoc|\\
|\edef\childdocname{\scantokens\expandafter{\jobname\noexpand}}|\\
|\def\childdocmain{|\textit{main}|}\||ifx\childdocmain\childdocname\||else|\\
|\childdoctrue\includeonly{\childdocname}\let\jobname\childdocmain\||fi|\\
\end{tabular}
\end{center}
%
Instead of |\childdocof{|\textit{main}|}| just include the main file
at the top of each child file:
%
\begin{center}
|\input{|\textit{main}|}|
\end{center}
%
A simple redirection |\childdocforward{|\textit{dest}|}| is achieved by:
%
\begin{center}
|\def\jobname{|\textit{dest}|}\input{\jobname}|
\end{center}
%
The redirection with prefix
|\childdocforwardprefix[|\textit{prefix}|]{|\textit{dest}|}|
is accomplished by:
%
\begin{center}
\begin{tabular}{l}
|{\edef\jobname{\scantokens\expandafter{\jobname\noexpand}}|\\
|\def\redirectjob |\textit{prefix}|#1~~~{\gdef\jobname{|\textit{dest}|#1}}|\\
|\expandafter\redirectjob\jobname~~~}\input{\jobname}|
\end{tabular}
\end{center}

In an alternative approach,
child documents can be compiled by a specific command line
without additional code or specific definitions:
%
\begin{center}
|... -jobname "|\textit{target}|" "|[\textit{flags}]%
|\includeonly{|\textit{dest}|}\input{|\textit{main}|}"|
\end{center}
%

%%%%%%%%%%%%%%%%%%%%%%%%%%%%%%%%%%%%%%%%%%%%%%%%%%%%%%%%%%%%%%%%%%%%%%%%%%%%%%%%
%%%%%%%%%%%%%%%%%%%%%%%%%%%%%%%%%%%%%%%%%%%%%%%%%%%%%%%%%%%%%%%%%%%%%%%%%%%%%%%%
\section{Information}

%%%%%%%%%%%%%%%%%%%%%%%%%%%%%%%%%%%%%%%%%%%%%%%%%%%%%%%%%%%%%%%%%%%%%%%%%%%%%%%%
\subsection{Copyright}

Copyright \copyright{} 2017--2018 Niklas Beisert

This work may be distributed and/or modified under the
conditions of the \LaTeX{} Project Public License, either version 1.3
of this license or (at your option) any later version.
The latest version of this license is in
  \url{http://www.latex-project.org/lppl.txt}
and version 1.3 or later is part of all distributions of \LaTeX{}
version 2005/12/01 or later.

This work has the LPPL maintenance status `maintained'.

The Current Maintainer of this work is Niklas Beisert.

This work consists of the files |README.txt|, |childdoc.ins| and |childdoc.dtx|
as well as the derived files |childdoc.def|, |cdocsamp.tex|
with |cdocsch1.tex|, |cdocsch2.tex|, |cdocspt3.tex|, |cdocspt4.tex|,
|cdocsdrf.tex|, |cdocsfn1.tex|, |cdocsfn2.tex|
as well as |childdoc.pdf|.

%%%%%%%%%%%%%%%%%%%%%%%%%%%%%%%%%%%%%%%%%%%%%%%%%%%%%%%%%%%%%%%%%%%%%%%%%%%%%%%%
\subsection{Files and Installation}

The package consists of the files:
%
\begin{center}
\begin{tabular}{ll}
    |README.txt|   & readme file \\
    |childdoc.ins| & installation file \\
    |childdoc.dtx| & source file \\
    |childdoc.def| & definition file \\
    |cdocsamp.tex| & sample main file \\
    |cdocsch1.tex| & sample include file \\
    |cdocsch2.tex| & sample include file \\
    |cdocspt3.tex| & sample part file \\
    |cdocspt4.tex| & sample part file \\
    |cdocsdrf.tex| & sample redirection file \\
    |cdocsfn1.tex| & sample redirection file \\
    |cdocsfn2.tex| & sample redirection file \\
    |childdoc.pdf| & manual
\end{tabular}
\end{center}
%
The distribution consists of the files
|README.txt|, |childdoc.ins| and |childdoc.dtx|.
%
\begin{itemize}
\item
Run (pdf)\LaTeX{} on |childdoc.dtx|
to compile the manual |childdoc.pdf| (this file).
\item
Run \LaTeX{} on |childdoc.ins| to create the definitions file |childdoc.def|
and the sample |cdocsamp.tex| with include files
|cdocsch1.tex|, |cdocsch2.tex|, |cdocspt3.tex|, |cdocspt4.tex|,
|cdocsdrf.tex|, |cdocsfn1.tex|, |cdocsfn2.tex|.
Then copy the file |childdoc.def| to an appropriate directory of your \LaTeX{}
distribution, e.g.\ \textit{texmf-root}|/tex/latex/childdoc|.
\end{itemize}

%%%%%%%%%%%%%%%%%%%%%%%%%%%%%%%%%%%%%%%%%%%%%%%%%%%%%%%%%%%%%%%%%%%%%%%%%%%%%%%%
\subsection{Related CTAN Packages}

There are several other packages which offer a similar functionality:
%
\begin{itemize}
\item
The packages
\href{http://ctan.org/pkg/docmute}{\textsf{docmute}},
\href{http://ctan.org/pkg/includex}{\textsf{includex}} and
\href{http://ctan.org/pkg/standalone}{\textsf{standalone}}
provide commands to include only the document body of
a child file thus allowing both files to be compiled individually.
\item
The packages \href{http://ctan.org/pkg/subdocs}{\textsf{subdocs}}
and \href{http://ctan.org/pkg/subfiles}{\textsf{subfiles}}
provide structures in which the main and child documents can be
encapsulated and allowing them to be compiled individually.
The inclusion mechanism is different from the conventional |\include|.
\item
The package \href{http://ctan.org/pkg/combine}{\textsf{combine}}
is an elaborate solution to combine several documents into one.
\end{itemize}
%
See also the CTAN topic \href{http://ctan.org/topic/subdocs}{\textsf{subdocs}}
for further related packages.
The present package differs from the above solutions in that
a document structure constructed with the conventional |\include| mechanism
just needs two extra commands at the top of every file
such that all constituent files can be compiled individually.

%%%%%%%%%%%%%%%%%%%%%%%%%%%%%%%%%%%%%%%%%%%%%%%%%%%%%%%%%%%%%%%%%%%%%%%%%%%%%%%%
%\subsection{Feature Suggestions}
%
%The following is a list of features which may be useful for future
%versions of this package:
%%
%\begin{itemize}
%\item
%\ldots
%\end{itemize}

%%%%%%%%%%%%%%%%%%%%%%%%%%%%%%%%%%%%%%%%%%%%%%%%%%%%%%%%%%%%%%%%%%%%%%%%%%%%%%%%
\subsection{Revision History}

%%%%%%%%%%%%%%%%%%%%%%%%%%%%%%%%%%%%%%%%
\paragraph{v2.0:} 2018/12/30

\begin{itemize}
\item
immediate forward processing
\item
added |\childdocby| mechanism
\item
manual restructured
\end{itemize}

%%%%%%%%%%%%%%%%%%%%%%%%%%%%%%%%%%%%%%%%
\paragraph{v1.6:} 2018/01/17

\begin{itemize}
\item
application for development of include files
\item
corrections to manual
\end{itemize}

%%%%%%%%%%%%%%%%%%%%%%%%%%%%%%%%%%%%%%%%
\paragraph{v1.5:} 2017/05/21

\begin{itemize}
\item
more complete structuring introduced
\item
|\childdocof| introduced
\item
|\childdoc| renamed to |\childdocmain|
\item
|\childredirect| renamed to |\childdocforward| and |\childdocforwardprefix|
and functionality expanded
\end{itemize}

%%%%%%%%%%%%%%%%%%%%%%%%%%%%%%%%%%%%%%%%
\paragraph{v1.0:} 2017/04/27

\begin{itemize}
\item
manual and install package
\item
first version published on CTAN
\end{itemize}

%%%%%%%%%%%%%%%%%%%%%%%%%%%%%%%%%%%%%%%%
\paragraph{v0.6:} 2017/04/26

\begin{itemize}
\item
redirection mechanism added
\end{itemize}

%%%%%%%%%%%%%%%%%%%%%%%%%%%%%%%%%%%%%%%%
\paragraph{v0.5:} 2017/04/26

\begin{itemize}
\item
functionality in definition file
\end{itemize}


%%%%%%%%%%%%%%%%%%%%%%%%%%%%%%%%%%%%%%%%%%%%%%%%%%%%%%%%%%%%%%%%%%%%%%%%%%%%%%%%
%%%%%%%%%%%%%%%%%%%%%%%%%%%%%%%%%%%%%%%%%%%%%%%%%%%%%%%%%%%%%%%%%%%%%%%%%%%%%%%%
%%%%%%%%%%%%%%%%%%%%%%%%%%%%%%%%%%%%%%%%%%%%%%%%%%%%%%%%%%%%%%%%%%%%%%%%%%%%%%%%
\appendix

\settowidth\MacroIndent{\rmfamily\scriptsize 000\ }

 \DocInput{childdoc.dtx}

\end{document}
%</driver>
% \fi
%
% %%%%%%%%%%%%%%%%%%%%%%%%%%%%%%%%%%%%%%%%%%%%%%%%%%%%%%%%%%%%%%%%%%%%%%%%%%%%%%
% %%%%%%%%%%%%%%%%%%%%%%%%%%%%%%%%%%%%%%%%%%%%%%%%%%%%%%%%%%%%%%%%%%%%%%%%%%%%%%
% \section{Sample}
%\iffalse
%<*samplemain>
%\fi
%
% The following presents a sample document
% with two chapters, two parts, a title page,
% a compile flag as well as three forwarding files to set the flag.
% It consists of eight |.tex| files:
% \begin{center}
% \begin{tabular}{ll}
% |cdocsamp.tex|&main file\\
% |cdocsch1.tex|&include file for chapter 1\\
% |cdocsch2.tex|&include file for chapter 2\\
% |cdocspt3.tex|&include file for part 3\\
% |cdocspt4.tex|&include file for part 4\\
% |cdocsdrf.tex|&forwarding file for main file in draft mode\\
% |cdocsfi1.tex|&forwarding file for final version of chapter 1\\
% |cdocsfi2.tex|&forwarding file for final version of chapter 2\\
% \end{tabular}
% \end{center}
% Each of the eight files can be compiled directly by the \LaTeX{} compiler.
%
% %%%%%%%%%%%%%%%%%%%%%%%%%%%%%%%%%%%%%%
% \paragraph{Main File.}
%
% The main file is called |cdocsamp.tex|.
%
% Load the \textsf{childdoc} definitions and
% declare the filename for the main document:
%    \begin{macrocode}
\input{childdoc.def}
\childdocmain{}
%    \end{macrocode}

% Optional override for |\version| flag:
%    \begin{macrocode}
%%\ifchilddoc\else\providecommand{\version}{draft}\fi
%    \end{macrocode}

% Define the default values for the |\version| flag
% (|final| for the main file and |draft| for childs):
%    \begin{macrocode}
\ifchilddoc
\providecommand{\version}{draft}
\else
\providecommand{\version}{final}
\fi
%    \end{macrocode}

% Load the standard document class:
%    \begin{macrocode}
\documentclass[12pt]{article}
%    \end{macrocode}

% Start the document body:
%    \begin{macrocode}
\begin{document}
%    \end{macrocode}

% Declare a title page.
% Print title, part of document being processed and version flag:
%    \begin{macrocode}
\addtocounter{page}{-1}
\begin{center}
{\LARGE\bfseries{}childdoc example\par}
\vspace{1cm}
\ifchilddoc
\ifchilddocmanual part\else chapter\fi:
`\childdocname' of `\childdocjob'\par
\else
main document: `\childdocjob'\par
\fi
version: \version\par
\end{center}
\newpage
%    \end{macrocode}

% Manually include selected file,
% otherwise process as usual:
%    \begin{macrocode}
\ifchilddocmanual
\section*{part `\childdocname'}
\input{\childdocname}
\else
%    \end{macrocode}

% Include the two chapters:
%    \begin{macrocode}
\include{cdocsch1}
\include{cdocsch2}
%    \end{macrocode}

% Include the two parts unless only chapters should be displayed:
%    \begin{macrocode}
\ifchilddoc\else
\section{part three}
\input{cdocspt3}
\section{part four}
\input{cdocspt4}
\fi
%    \end{macrocode}

% Process as usual until here:
%    \begin{macrocode}
\fi
%    \end{macrocode}

% End of document body:
%    \begin{macrocode}
\end{document}
%    \end{macrocode}
%\iffalse
%</samplemain>
%\fi
%
% %%%%%%%%%%%%%%%%%%%%%%%%%%%%%%%%%%%%%%
% \paragraph{Chapter Include Files.}
%
% The include files are called |cdocsch1.tex| and |cdocsch2.tex|.
%
%\iffalse
%<*samplechap1|samplechap2>
%\fi

% Optional override for |\version| flag:
%    \begin{macrocode}
%%\providecommand{\version}{final}
%    \end{macrocode}

% Include the main document:
%    \begin{macrocode}
\input{childdoc.def}
\childdocof{cdocsamp}
%    \end{macrocode}

%\iffalse
%</samplechap1|samplechap2>
%\fi
%
%\iffalse
%<*samplechap1>
%\fi
% Some text for chapter 1:
%    \begin{macrocode}
\section{one}
some text in chapter one
%    \end{macrocode}

%\iffalse
%</samplechap1>
%\fi
% Some text for chapter 2:
%\iffalse
%<*samplechap2>
%\fi
%    \begin{macrocode}
\section{two}
more text in chapter two
%    \end{macrocode}

%\iffalse
%</samplechap2>
%\fi
%
% %%%%%%%%%%%%%%%%%%%%%%%%%%%%%%%%%%%%%%
% \paragraph{Part Include Files.}
%
% The include files are called |cdocspt3.tex| and |cdocspt4.tex|.
%
%\iffalse
%<*samplepart3|samplepart4>
%\fi

% Optional override for |\version| flag:
%    \begin{macrocode}
%%\providecommand{\version}{final}
%    \end{macrocode}

% Include the main document:
%    \begin{macrocode}
\input{childdoc.def}
\childdocby{cdocsamp}
%    \end{macrocode}

%\iffalse
%</samplepart3|samplepart4>
%\fi
%
%\iffalse
%<*samplepart3>
%\fi
% Some text for part 3:
%    \begin{macrocode}
some text in part three
%    \end{macrocode}

%\iffalse
%</samplepart3>
%\fi
% Some text for part 4:
%\iffalse
%<*samplepart4>
%\fi
%    \begin{macrocode}
more text in part four
%    \end{macrocode}

%\iffalse
%</samplepart4>
%\fi
%
% %%%%%%%%%%%%%%%%%%%%%%%%%%%%%%%%%%%%%%
% \paragraph{Forwarding for a Complete Draft.}
%
% The following forwarding file |cdocsdrf.tex|
% compiles the main document in draft mode:
%\iffalse
%<*sampledraft>
%\fi
%    \begin{macrocode}
\def\version{draft}
\input{childdoc.def}
\childdocforward{cdocsamp}
%    \end{macrocode}

%\iffalse
%</sampledraft>
%\fi
%
% %%%%%%%%%%%%%%%%%%%%%%%%%%%%%%%%%%%%%%
% \paragraph{Forwarding for Final Version of the Chapters.}
%
% The following forwarding files |cdocsfn1.tex| and |cdocsfn2.tex|
% (with identical content)
% compile the final versions of the child documents
% |cdocsch1.tex| and |cdocsch2.tex|, respectively:
%\iffalse
%<*samplefinal>
%\fi
%    \begin{macrocode}
\def\version{final}
\input{childdoc.def}
\childdocforwardprefix[cdocsamp]{cdocsfn}{cdocsch}
%    \end{macrocode}

%\iffalse
%</samplefinal>
%\fi
%
% %%%%%%%%%%%%%%%%%%%%%%%%%%%%%%%%%%%%%%
% \paragraph{Command Line Processing.}
%
% The following three command lines generate the output files
% |cdocscld|, |cdocscl1| and |cdocscl2|
% which should be identical to
% |cdocsdrf|, |cdocsch1| and |cdocsfn2|, respectively:
% \begin{center}
% \begin{tabular}{l}
% |latex -jobname cdocscld \|\\
% |  "\def\version{draft}\input{childdoc.def}\childdocforward{cdocsamp}"|\\
% |latex -jobname cdocscl1 \|\\
% |  "\input{childdoc.def}\childdocforward[cdocsamp]{cdocsch1}"|\\
% |latex -jobname cdocscl2 \|\\
% |  "\def\version{final}\input{childdoc.def}\childdocforward{cdocsch2}"|
% \end{tabular}
% \end{center}
% Note that the trailing backslash on each first line
% merely continues the input to the second line
% (for convenient cut ant paste).
% Furthermore, the command |latex| can be replaced by any
% of its alternative versions such as |pdflatex|.
%
% %%%%%%%%%%%%%%%%%%%%%%%%%%%%%%%%%%%%%%%%%%%%%%%%%%%%%%%%%%%%%%%%%%%%%%%%%%%%%%
% %%%%%%%%%%%%%%%%%%%%%%%%%%%%%%%%%%%%%%%%%%%%%%%%%%%%%%%%%%%%%%%%%%%%%%%%%%%%%%
% \section{Implementation}
%\iffalse
%<*package>
%\fi
%
% This section describes the definitions file |childdoc.def|.

% The definitions cannot be loaded using |\usepackage| or |\RequirePackage|
% which has a mechanism to prevent loading a style file more than once.
% When loading the definitions by means of |\input|
% multiple instances have to be prevented manually:
%\iffalse
%This code needs to be before the `\ProvidesFile' directive
%which is defined at the beginning of this file.
%Therefore it is also placed there and commented out here.
%</package>
%<*discard>
%\fi
%    \begin{macrocode}
\ifdefined\childdocmain\endinput\fi
%    \end{macrocode}
%\iffalse
%</discard>
%<*package>
%\fi
%
% \macro{\ifchilddoc}
% \macro{\ifchilddocmanual}
% The conditional |\ifchilddoc| tells whether a
% child (true) or main (false) document is being compiled.
% The conditional |\ifchilddocmanual| tells whether
% the |\includeonly| mechanism is used (false) or
% the selection of child files must be performed manually (true).
% The definitions initialise to false:
%    \begin{macrocode}
\newif\ifchilddoc
\newif\ifchilddocmanual
%    \end{macrocode}

% \macro{\childdocname}
% \macro{\childdocjob}
% The macro |\childdocname| stores the name of the main document
% to be compiled. The macro |\childdocjob| stores the name of
% the document on which the \LaTeX{} compiler was originally invoked.
% The content of |\jobname| cannot be compared
% to filenames specified in the source due to different catcodes.
% The following code rescans |\jobname|, stores the result
% in |\childdocname| and saves a copy in |\childdocjob|:
%    \begin{macrocode}
\edef\childdocname{\scantokens\expandafter{\jobname\noexpand}}
\let\childdocjob\childdocname
%    \end{macrocode}

% \macro{\childdocdisable}
% The macro |\childdocdisable| prevents the main file
% from being processed more than once.
% At this stage, the main document command |\childdocmain|
% is assumed to be called once again where it should do nothing.
% Any subsequent call to it should prevent
% a secondary processing of the main document
% It overwrites the forwarding commands
% |\childdocof| and |\childdocforward|
% with empty macros to prevent further inclusions of the main document:
%    \begin{macrocode}
\newcommand{\childdocdisable}
{
  \renewcommand{\childdocmain}[1]{\renewcommand{\childdocmain}[1]{\endinput}}
  \renewcommand{\childdocof}[1]{}
  \renewcommand{\childdocby}[2][]{}
  \renewcommand{\childdocforward}[2][]{}
  \renewcommand{\childdocdisable}{}
}
%    \end{macrocode}

% \macro{\childdocmain}
% The macro |\childdocmain| is to be called at the top of the main file
% with nothing or the main filename (without extension) as argument.
% First, it breaks loops.
% If the argument is not empty and does not match |\childdocname|
% (which is set by the first inclusion of |childdoc.def|),
% |\ifchilddoc| is set to true, |\includeonly| is applied to the child file
% and |\jobname| is set to the main file
% (for proper handling of |.aux| files):
%    \begin{macrocode}
\newcommand{\childdocmain}[1]
{
  \childdocdisable\childdocmain{}
  \if?#1?\else
    \begingroup
      \def\childdoctmp{#1}
      \ifx\childdoctmp\childdocname
        \def\childdoctmp{}
      \else
        \def\childdoctmp
        {
          \childdoctrue
          \includeonly{\childdocname}
          \def\childdocjob{#1}
          \def\jobname{#1}
        }
      \fi
      \expandafter
    \endgroup
    \childdoctmp
  \fi
}
%    \end{macrocode}

% \macro{\childdocof}
% The command |\childdocof| redirects
% compilation to the main file |#1|.
%    \begin{macrocode}
\newcommand{\childdocof}[1]
{
  \childdocdisable
  \childdoctrue
  \includeonly{\childdocname}
  \def\jobname{#1}
  \def\childdocjob{#1}
  \input{#1}
}
%    \end{macrocode}

% \macro{\childdocby}
% The command |\childdocby| ....
%    \begin{macrocode}
\newcommand{\childdocby}[2][]
{
  \childdocdisable
  \childdoctrue
  \childdocmanualtrue
  \if?#1?\else
    \def\jobname{#2}
  \fi
  \def\childdocjob{#2}
  \input{#2}
  \endinput
}
%    \end{macrocode}

% \macro{\childdocforward}
% The command |\childdocforward| redirects
% compilation to the main file or
% (if the optional argument is given) a child file.
% Parameters are set as if the main file
% or a child file starting with |\childdocof| was compiled.
% Then compilation is handed over to the main file:
%    \begin{macrocode}
\newcommand{\childdocforward}[2][]
{
  \begingroup
    \if?#1?
      \def\childdoctmp
      {
        \def\childdocname{#2}
        \def\childdocjob{#2}
        \def\jobname{#2}
        \input{#2}
        \endinput
      }
    \else
      \def\childdoctmp
      {
        \childdocdisable
        \def\childdocname{#2}
        \childdoctrue
        \includeonly{#2}
        \def\childdocjob{#1}
        \def\jobname{#1}
        \input{#1}
        \endinput
      }
    \fi
    \expandafter
  \endgroup
  \childdoctmp
}
%    \end{macrocode}

% \macro{\childdocforwardprefix}
% The command |\childdocforwardprefix| redirects
% compilation to the main or a child file by means of a pattern.
% The prefix |#1| in the current filename is replaced by |#2|
% and the suffix of the current filename is kept
% (it is assumed that the filename does not contain the substring `|~~~|'
% which is used as a delimiter).
% Compilation is handed over to the new file by |\childdocforward|:
%    \begin{macrocode}
\newcommand{\childdocforwardprefix}[3][]
{
  \begingroup
    \def\childdocextract #2##1~~~{\def\childdoctmp{\childdocforward[#1]{#3##1}}}
    \expandafter\childdocextract\childdocname~~~
    \expandafter
  \endgroup
  \childdoctmp
}
%    \end{macrocode}

% \macro{\childdoc}
% The deprecated macro |\childdoc| is a legacy version of |\childdocmain|:
%    \begin{macrocode}
\newcommand{\childdoc}{\childdocmain}
%    \end{macrocode}

% \macro{\childdocredirect}
% The deprecated macro |\childdocredirect| is a legacy version
% of |\childdocforward| and |\childdocforwardprefix|:
%    \begin{macrocode}
\newcommand{\childdocredirect}[2][]
{
  \begingroup
    \if?#1?
      \def\childdoctmp{\childdocforward{#2}}
    \else
      \def\childdoctmp{\childdocforwardprefix{#1}{#2}}
    \fi
    \expandafter
  \endgroup
  \childdoctmp
}
%    \end{macrocode}

%\iffalse
%</package>
%\fi
%
\endinput

\childdocby{cdocsamp}
%    \end{macrocode}

%\iffalse
%</samplepart3|samplepart4>
%\fi
%
%\iffalse
%<*samplepart3>
%\fi
% Some text for part 3:
%    \begin{macrocode}
some text in part three
%    \end{macrocode}

%\iffalse
%</samplepart3>
%\fi
% Some text for part 4:
%\iffalse
%<*samplepart4>
%\fi
%    \begin{macrocode}
more text in part four
%    \end{macrocode}

%\iffalse
%</samplepart4>
%\fi
%
% %%%%%%%%%%%%%%%%%%%%%%%%%%%%%%%%%%%%%%
% \paragraph{Forwarding for a Complete Draft.}
%
% The following forwarding file |cdocsdrf.tex|
% compiles the main document in draft mode:
%\iffalse
%<*sampledraft>
%\fi
%    \begin{macrocode}
\def\version{draft}
% \iffalse
%
% childdoc.dtx Copyright (C) 2017-2018 Niklas Beisert
%
% This work may be distributed and/or modified under the
% conditions of the LaTeX Project Public License, either version 1.3
% of this license or (at your option) any later version.
% The latest version of this license is in
%   http://www.latex-project.org/lppl.txt
% and version 1.3 or later is part of all distributions of LaTeX
% version 2005/12/01 or later.
%
% This work has the LPPL maintenance status `maintained'.
%
% The Current Maintainer of this work is Niklas Beisert.
%
% This work consists of the files childdoc.dtx and childdoc.ins
% and the derived files childdoc.def and cdocsamp.tex with
% cdocsch1.tex, cdocsch2.tex, cdocsdrf.tex, cdocsfn1.tex, cdocsfn2.tex.
%
%<package>\ifdefined\childdocmain\endinput\fi
%<package>\ProvidesFile{childdoc.def}[2018/12/30 v2.0 child document driver]
%<samplemain>\ProvidesFile{cdocsamp.tex}[2018/12/30 v2.0 sample for childdoc]
%<*driver>
%\ProvidesFile{childdoc.drv}[2018/12/30 v2.0 childdoc reference manual file]
\PassOptionsToClass{10pt,a4paper}{article}
\documentclass{ltxdoc}

\usepackage[margin=35mm]{geometry}
\usepackage{hyperref}
\usepackage{hyperxmp}
\usepackage[usenames]{color}

\hypersetup{colorlinks=true}
\hypersetup{pdfstartview=FitH}
\hypersetup{pdfpagemode=UseNone}
\hypersetup{pdfsource={}}
\hypersetup{pdflang={en-UK}}
\hypersetup{pdfcopyright={Copyright 2017-2018 Niklas Beisert.
  This work may be distributed and/or modified under the
  conditions of the LaTeX Project Public License, either version 1.3
  of this license or (at your option) any later version.}}
\hypersetup{pdflicenseurl={http://www.latex-project.org/lppl.txt}}
\hypersetup{pdfcontactaddress={ETH Zurich, ITP, HIT K,
  Wolfgang-Pauli-Strasse 27}}
\hypersetup{pdfcontactpostcode={8093}}
\hypersetup{pdfcontactcity={Zurich}}
\hypersetup{pdfcontactcountry={Switzerland}}
\hypersetup{pdfcontactemail={nbeisert@itp.phys.ethz.ch}}
\hypersetup{pdfcontacturl={http://people.phys.ethz.ch/\xmptilde nbeisert/}}

\newcommand{\secref}[1]{\hyperref[#1]{section \ref*{#1}}}

\parskip1ex
\parindent0pt
\let\olditemize\itemize
\def\itemize{\olditemize\parskip0pt}

\begin{document}

\title{The \textsf{childdoc} Package}
\hypersetup{pdftitle={The childdoc Package}}
\author{Niklas Beisert\\[2ex]
  Institut f\"ur Theoretische Physik\\
  Eidgen\"ossische Technische Hochschule Z\"urich\\
  Wolfgang-Pauli-Strasse 27, 8093 Z\"urich, Switzerland\\[1ex]
  \href{mailto:nbeisert@itp.phys.ethz.ch}
  {\texttt{nbeisert@itp.phys.ethz.ch}}}
\hypersetup{pdfauthor={Niklas Beisert}}
\hypersetup{pdfsubject={Manual for the LaTeX2e Package childdoc}}
\date{30 December 2018, \textsf{v2.0}}
\maketitle

\begin{abstract}\noindent
\textsf{childdoc} is a \LaTeXe{} package
that enables the direct compilation
of document sections included by |\include|
to individual files.
\end{abstract}

\begingroup
\parskip0ex
\tableofcontents
\endgroup

%%%%%%%%%%%%%%%%%%%%%%%%%%%%%%%%%%%%%%%%%%%%%%%%%%%%%%%%%%%%%%%%%%%%%%%%%%%%%%%%
%%%%%%%%%%%%%%%%%%%%%%%%%%%%%%%%%%%%%%%%%%%%%%%%%%%%%%%%%%%%%%%%%%%%%%%%%%%%%%%%
\section{Introduction}

\LaTeX{} provides a mechanism to structure a large document (such as a book)
into a main file and several child files (containing the chapters)
using the |\include| command.
This mechanism is beneficial for documents
which span hundreds of pages in order to
make the source file(s) more manageable.
Moreover, compilation can be restricted to
selected child files by means of the |\includeonly| command.
The latter feature can be used to reduce the compilation time while editing
(this was significantly more useful in the earlier days of \LaTeX{})
or to generate a smaller document which is easier to navigate.
Another application of |\includeonly| is to generate
documents consisting of selected parts of the complete document.

However, there are a few drawbacks of the plain |\include| mechanism:
\begin{itemize}
\item
The child files cannot be compiled on their own,
they can only be compiled via the main file.
A naive editing environment
(such as a text editor with an option
to have the current file processed by \LaTeX)
may require one to switch to the main file before compiling;
attempting to compile the child file produces errors.
\item
The main file must be modified (each time)
to adjust the |\includeonly| command
to the present needs. This easily leaves the main file in a messy state.
\item
The generated document will always carry the filename
of the main document. This is inconvenient if
several child files are to be compiled and
to be kept for distribution.
\end{itemize}

The present package provides a simple interface
to make child files individually compilable by \LaTeX{}.
Compiling a child file then has the same effect as compiling
the main file with an |\includeonly| command
to select the appropriate child.
Moreover the generated document will carry the name of the child
rather than the main file.
This resolves all three above issues.

This feature is meant to make the editing of books,
thesis documents and lecture notes somewhat more convenient.
However, the package can also be used efficiently for
composing a series of documents (such as exercise sheets)
which are typically distributed individually.
It then assists the author in generating the individual documents
(potentially in different versions)
as well as a document containing the collected series.
Another application is in developing style files
or other kinds of included material
where compilation of the style file could redirect
to a sample or test file.

%%%%%%%%%%%%%%%%%%%%%%%%%%%%%%%%%%%%%%%%%%%%%%%%%%%%%%%%%%%%%%%%%%%%%%%%%%%%%%%%
%%%%%%%%%%%%%%%%%%%%%%%%%%%%%%%%%%%%%%%%%%%%%%%%%%%%%%%%%%%%%%%%%%%%%%%%%%%%%%%%
\section{Usage}

First of all, the package \textsf{childdoc} is \emph{not} a standard
\LaTeXe{} |.sty| style file! Therefore it needs to be invoked in
a non-standard way.

%%%%%%%%%%%%%%%%%%%%%%%%%%%%%%%%%%%%%%%%%%%%%%%%%%%%%%%%%%%%%%%%%%%%%%%%%%%%%%%%
\subsection{Included Files}
\label{sec:include}

%%%%%%%%%%%%%%%%%%%%%%%%%%%%%%%%%%%%%%%%
\DescribeMacro{\childdocmain}
To use the package, add the commands
\begin{center}
\begin{tabular}{l}
|\input{childdoc.def}|\\
|\childdocmain{}|\\
\end{tabular}
\end{center}
at the very top of the main \LaTeX{} file,
in particular \emph{before} the |\documentclass| statement!
The argument of |\childdocmain| should be left empty
(but it must be present).

%%%%%%%%%%%%%%%%%%%%%%%%%%%%%%%%%%%%%%%%
\DescribeMacro{\childdocof}
Furthermore, add the commands
\begin{center}
\begin{tabular}{l}
|\input{childdoc.def}|\\
|\childdocof{|\textit{main}|}|\\
\end{tabular}
\end{center}
at the top of every child file \textit{child}
which is included by |\include{|\textit{child}|}|
from within the main file
(or at least for those files to be compiled individually).
The argument \textit{main} must be the filename of the main file.

There are a couple of
considerations in setting up the main and child documents:

%%%%%%%%%%%%%%%%%%%%%%%%%%%%%%%%%%%%%%%%
\paragraph{Restrictions.}

Please note the following restrictions:
\begin{itemize}
\item
|\childdocmain| must be called with one argument \textit{main}
to ensure compatibility with earlier version of the package.
It must either be empty (|\childdocmain{}|)
or precisely match the filename of the main file in which it is specified.
See \secref{sec:detection} for further information.
\item
The filename \textit{main} must be specified without the |.tex| extension.
\item
The filename \textit{main} is case sensitive
(even in case-insensitive file systems)
due to internal string comparison.
\item
The argument \textit{main} should be fully expanded, it cannot be a macro.
\item
Subdirectories and special characters should be avoided in filenames.
\item
The command |\childdocmain{|\textit{main}|}| must be followed by a whitespace.
It should not be followed immediately by another command
or by a comment mark `|%|'.
This is because the \TeX{} parser reads the token immediately following
the argument of |\childdocmain| and puts it
at the beginning of every child section;
however, a white\-space is ignored.
\end{itemize}

%%%%%%%%%%%%%%%%%%%%%%%%%%%%%%%%%%%%%%%%
\paragraph{Content of Main File.}

It is advisable to place all content in the child files included by |\include|.
Any output contained in the main file will appear in all child documents
unless suppressed manually;
it cannot be suppressed automatically by the |\includeonly| directive
and thus should normally be avoided.
A method to include some content in the main file
by means of conditional processing is described in \secref{sec:conditional}.

%%%%%%%%%%%%%%%%%%%%%%%%%%%%%%%%%%%%%%%%
\paragraph{Page Numbering.}

When only a part of the document is compiled,
the appropriate numbering of pages
(as well as other status parameters)
is determined from the |.aux| files.
The latter contain information from previous passes.
However this information needs to propagate through
all intermediate child documents.
Therefore the page numbering in child documents may well
be inconsistent until the complete document is compiled at least once.

A useful (if unconventional) way to always ensure a consistent
page numbering is to restart the numbering in each child document
and denote the pages by `\textit{child}|.|\textit{page}'
where \textit{child} represents the chapter/section number of the child file.
This can be achieved by the command
|\numberwithin{page}{|\textit{child}|}|
of the \textsf{amsmath} package
where \textit{child} can be |chapter| or |section|
depending on the chosen structuring.
Alternatively, one can modify the macro |\thepage| appropriately
and reset the counter |page| at the start of each child file.

%%%%%%%%%%%%%%%%%%%%%%%%%%%%%%%%%%%%%%%%%%%%%%%%%%%%%%%%%%%%%%%%%%%%%%%%%%%%%%%%
\subsection{Conditional Processing}
\label{sec:conditional}

The package provides a mechanism to compile different versions
of a document. To customise the versions further some conditional processing
can come in handy to distinguish which version is being compiled.
The package provides two macros to describe the compilation context:

%%%%%%%%%%%%%%%%%%%%%%%%%%%%%%%%%%%%%%%%
\DescribeMacro{\ifchilddoc}
The conditional |\ifchilddoc| distinguishes between the compilation of
child documents and the main document:
%
\begin{center}
|\ifchilddoc |\textit{child-code}| |[|\||else |\textit{main-code}]| \||fi|
\end{center}

%%%%%%%%%%%%%%%%%%%%%%%%%%%%%%%%%%%%%%%%
\DescribeMacro{\childdocname}
\DescribeMacro{\childdocjob}
The macro |\childdocname| contains the filename (without extension)
of the main or child file being processed.
Note that |\childdocjob| will always contain the name of the main file.

%%%%%%%%%%%%%%%%%%%%%%%%%%%%%%%%%%%%%%%%
\paragraph{Title Page.}

Conditional processing can be used to include a title or banner page
in the main document when proper precautions are taken.
Importantly, the code in the main file should ensure that the page counter
(as well as other status parameters which are stored in the |.aux| files)
takes the same value after the conditional processing.
Otherwise the page numbers may take divergent values
depending on which part is compiled.

For example, a title page could be declared by:
%
\begin{center}
\begin{tabular}{l}
|\ifchilddoc\||else|\\
|\addtocounter{page}{-1}|\\
\textit{code for title page}\\
|\newpage|\\
|\||fi|
\end{tabular}
\end{center}
%
A banner page for the child documents can be generated by:
%
\begin{center}
\begin{tabular}{l}
|\ifchilddoc|\\
|\addtocounter{page}{-1}|\\
\textit{code for banner page}\\
|\newpage|\\
|\||fi|
\end{tabular}
\end{center}
%
Here one could write a message such as:
\begin{center}
|This is the part \childdocname{} of \childdocjob{}.|
\end{center}

%%%%%%%%%%%%%%%%%%%%%%%%%%%%%%%%%%%%%%%%%%%%%%%%%%%%%%%%%%%%%%%%%%%%%%%%%%%%%%%%
\subsection{Flags}
\label{sec:flags}

The package makes it easy to generate different versions
of the main or child documents.
To this end compilation flags can be defined
and assigned different default values.
They will be particularly useful in conjunction
with the forwarding mechanism described in \secref{sec:forward}.

For example, it may be useful to have a flag |\version|
which can be set to |draft| or |final|.
The document source will contain some conditional code
depending on the value of |\version|.
Suppose further, the flag should default to |final| for the main file
and to |draft| for child files
which is a natural assignment for editing the document.
This is achieved by placing the following code
in the preamble of the main document
(below the |\childdocmain| directive):
%
\begin{center}
\begin{tabular}{l}
|\ifchilddoc|\\
|\providecommand{\version}{draft}|\\
|\||else|\\
|\providecommand{\version}{final}|\\
|\||fi|
\end{tabular}
\end{center}
%
The definition by |\providecommand| makes sure
that previous definitions are not overwritten.
Further statements |\providecommand{\version}{...}|
can thus be added before the above code to override it.

For the main file, one might add a line
(between |\childdocmain| and the above block)
%
\begin{center}
|%\ifchilddoc\||else\providecommand{\version}{draft}\||fi|
\end{center}
%
which can be uncommented to produce a draft version.
Likewise one can add a line to the very top of a child file
(above the |\childdocof{|\textit{main}|}| directive)
%
\begin{center}
|%\providecommand{\version}{final}|
\end{center}
%
which can be uncommented to produce the final version of this child document.

%%%%%%%%%%%%%%%%%%%%%%%%%%%%%%%%%%%%%%%%%%%%%%%%%%%%%%%%%%%%%%%%%%%%%%%%%%%%%%%%
\subsection{Forwarding}
\label{sec:forward}

Different versions of the main or child documents
using compilation flags as described in \secref{sec:flags}
can be (permanently) stored in different files
for convenient compilation, viewing and distribution.
To this end, the package defines a command
to pass on compilation to a different file:

%%%%%%%%%%%%%%%%%%%%%%%%%%%%%%%%%%%%%%%%
\DescribeMacro{\childdocforward}
The command |\childdocforward| redirects processing to
another source file:
%
\begin{center}
\begin{tabular}{l}
|\input{childdoc.def}|\\
|\childdocforward[|\textit{main}|]{|\textit{dest}|}|\\
\end{tabular}
\end{center}
%
The argument \textit{dest} is the destination file
(without extension).
It should be the main file or one of the child files.
Note that further \textsf{childdoc} directives
such as |\childdocof| and |\childdocforward|
in the indicated file will be processed in this form.
The optional argument \textit{main}
passes on directly to the main file \textit{main}
while pretending to compile the child \textit{dest}.
This form behaves as if \textit{dest}
issues |\childdocof{|\textit{main}|}| right away,
and no further \textsf{childdoc} directives will be processed.

%%%%%%%%%%%%%%%%%%%%%%%%%%%%%%%%%%%%%%%%
\DescribeMacro{\...prefix}
In the alternative form |\childdocforwardprefix|,
%
\begin{center}
\begin{tabular}{l}
|\input{childdoc.def}|\\
|\childdocforwardprefix[|\textit{main}|]{|\textit{prefix}|}{|\textit{dest}|}|
\end{tabular}
\end{center}
%
the destination file is determined by a pattern
depending on the current file:
To make this work, the current file must be called
`{\textit{prefix}\hspace{0.2em}\textit{suffix}}'
with \textit{prefix} matching precisely the argument.
Processing is then passed on to the file
`{\textit{dest}\hspace{0.2em}\textit{suffix}}'.
Surely, the same effect is achieved by
directly specifying the
argument `{\textit{dest}\hspace{0.2em}\textit{suffix}}'
in the first form.
However, that requires to set up a different file
for each child. With the alternative form of the command
all these files can have exactly the same content
which simplifies setting them up and maintaining them.

For example, the following file |draft.tex|
with a compilation flag |\version| as described in \secref{sec:flags}
compiles the main document as a draft:
%
\begin{center}
\begin{tabular}{l}
|\def\version{draft}|\\
|\input{childdoc.def}|\\
|\childdocforward{|\textit{main}|}|
\end{tabular}
\end{center}
%
Likewise, the following files |final|\textit{nn}|.tex|
compile the final version of the child document
|child|\textit{nn}|.tex|:
%
\begin{center}
\begin{tabular}{l}
|\def\version{final}|\\
|\input{childdoc.def}|\\
|\childdocforwardprefix{final}{child}|
\end{tabular}
\end{center}
%

Note that when several versions of a main file and/or of each child file
are to be generated, it may be convenient to set up a |Makefile| or
shell script to automatise the process.

%%%%%%%%%%%%%%%%%%%%%%%%%%%%%%%%%%%%%%%%%%%%%%%%%%%%%%%%%%%%%%%%%%%%%%%%%%%%%%%%
\subsection{Command Line Processing}
\label{sec:commandline}

The effect of redirection files can also be achieved by invoking
the \LaTeX{} compiler with a more elaborate command line.
Most conveniently this should be done as part
of a shell script or a |Makefile|.

When using \textsf{childdoc} in the main file, the following
command lines effectively perform a redirection
(note that depending on the shell being used,
backslashes may have to be doubled: `|\|' $\to$ `|\\|'):
%
\begin{center}
|... -jobname "|\textit{target}|" |\\|"|[\textit{flags}]%
|\input{childdoc.def}\childdocforward[|\textit{main}|]{|\textit{dest}|}"|
\end{center}
%
Here \textit{target} is the name of the output file,
\textit{main} is the name of the main file
and \textit{dest} is the name of the main or child file to be processed
(all filenames without extensions).
The optional argument \textit{main} can be omitted
if \textit{main} matches \textit{dest}.
Optionally, compilation \textit{flags} can be defined via |\def| commands.
This command line makes the \TeX{} engine believe
it is compiling the file \textit{target}
whose content is specified as the latter parameter.
The provided code then forwards the processing to
\textit{main} or \textit{dest} as described in \secref{sec:forward}.

%%%%%%%%%%%%%%%%%%%%%%%%%%%%%%%%%%%%%%%%%%%%%%%%%%%%%%%%%%%%%%%%%%%%%%%%%%%%%%%%
\subsection{Include by Input}
\label{sec:input}

Including child documents by |\include| has some restrictions by design.
Most notably, the content of a child document always occupies
its own set of pages; pages cannot be shared between child documents.
Usually, this behaviour makes perfect sense
because each child document contain an essential part of the document.
However, in some situations it may be desirable to compose
a document from a collection of parts
without having mandatory page breaks between then.
For this case, the package
provides a mechanism to include parts
by |\input| which can also be processed individually.
However, by construction this mechanism
requires manual handling of the content to be output.

%%%%%%%%%%%%%%%%%%%%%%%%%%%%%%%%%%%%%%%%
\DescribeMacro{\ifchilddocmanual}
The main file should be prepared as usual, see \secref{sec:include}.
However, the document body must make a distinction
between processing of an individual part and of the main document, e.g.:
%
\begin{center}
\begin{tabular}{l}
|\ifchilddocmanual|\\
|\input{\childdocname}|\\
|\||else|\\
\textit{document body with }|\input{|\textit{part}|}|\\
|\||fi|
\end{tabular}
\end{center}
%
The conditional |\ifchilddocmanual| is true whenever
a part to be included by |\input| is being compiled,
and the name of the part is stored in |\childdocname|.

%%%%%%%%%%%%%%%%%%%%%%%%%%%%%%%%%%%%%%%%
\DescribeMacro{\childdocby}
Each part to be included by |\input| should start with:
%
\begin{center}
\begin{tabular}{l}
|\input{childdoc.def}|\\
|\childdocby{|\textit{main}|}|\\
\end{tabular}
\end{center}
%
The directive |\childdocby| is similar to |\childdocof|
described in \secref{sec:include},
but the subsequent selection of content must be done manually.
To that end, both |\ifchilddoc| and |\ifchilddocmanual|
will be true upon processing of a part,
and the name of the part is stored in |\childdocname|.
Note that |\jobname| will be set to the filename of the current part
so that each part receives an individual |.aux| file
that does not interfere with the |.aux| file(s) of the main document.
This behaviour can be altered by the alternative form
|\childdocby[*]{|\textit{main}|}| (with a non-empty optional argument)
which uses the |.aux| file of the main document
by setting |\jobname| to \textit{main}.

%%%%%%%%%%%%%%%%%%%%%%%%%%%%%%%%%%%%%%%%%%%%%%%%%%%%%%%%%%%%%%%%%%%%%%%%%%%%%%%%
\subsection{Driver Development}
\label{sec:driver}

The \textsf{childdoc} mechanism can also be use for the development
of definition files such as \LaTeX{} styles or classes.
This case differs from the above setup with multiple parts
included by |\include| in that no |\includeonly| should be invoked.
This can be achieved by starting the include file
(before |\ProvidesPackage|) with:
%
\begin{center}
\begin{tabular}{l}
|\input{childdoc.def}|\\
|\childdocforward{|\textit{main}|}|\\
\end{tabular}
\end{center}
%
or alternatively with:
%
\begin{center}
\begin{tabular}{l}
|\input{childdoc.def}|\\
|\childdocby{|\textit{main}|}|\\
\end{tabular}
\end{center}
%
Both forms have slightly different effects as described above.
The main file is prepared as usual, see \secref{sec:include}.

%%%%%%%%%%%%%%%%%%%%%%%%%%%%%%%%%%%%%%%%%%%%%%%%%%%%%%%%%%%%%%%%%%%%%%%%%%%%%%%%
\subsection{Legacy Detection}
\label{sec:detection}

The directive |\childdocmain| in the main file can detect
whether the complete document or merely a child is to be compiled
even without using the directive |\childdocof|.
This method is deprecated because it is less robust
and there is no compelling reason to use it;
it is merely provided for backward compatibility
and it may be removed in future versions.

If the detection mechanism is to be used,
it is mandatory to correctly specify
the filename of the main file as the argument of |\childdocmain|:
%
\begin{center}
\begin{tabular}{l}
|\input{childdoc.def}|\\
|\childdocmain{|\textit{main}|}|\\
\end{tabular}
\end{center}
%
If |\jobname| does not match the argument \textit{main} of |\childdocmain|,
it is assumed that |\jobname| points to the child file to be compiled.
When using |\childdocmain| with the main file specified as argument,
it suffices to start a child file
with just |\input{|\textit{main}|}|
without loading of the package and using |\childdocof|.
If instead all processing is done
with the appropriate \textsf{childdoc} directives,
the argument of \textit{main} of |\childdocmain| can be empty.

An alternative version of the command line processing described
in \secref{sec:commandline} using the detection mechanism reads:
%
\begin{center}
|... -jobname "|\textit{target}|" "|[\textit{flags}]%
[|\def\jobname{|\textit{dest}|}|]|\input{|\textit{main}|}"|
\end{center}

%%%%%%%%%%%%%%%%%%%%%%%%%%%%%%%%%%%%%%%%%%%%%%%%%%%%%%%%%%%%%%%%%%%%%%%%%%%%%%%%
\subsection{Manual Code}
\label{sec:manual}

In case one cannot be certain whether the definitions file |childdoc.def|
is installed on the target \TeX{} distribution
and one prefers not to ship it,
it is conceivable to paste a few relevant commands into the sources.

To that end, drop all statements |\input{childdoc.def}|
and perform the replacements as outlined below.
Instead of |\childdocmain{|\textit{main}|}| add the following code
to the top of the main file:
%
\begin{center}
\begin{tabular}{l}
|\||ifdefined\childdocname\endinput\||fi\newif\ifchilddoc|\\
|\edef\childdocname{\scantokens\expandafter{\jobname\noexpand}}|\\
|\def\childdocmain{|\textit{main}|}\||ifx\childdocmain\childdocname\||else|\\
|\childdoctrue\includeonly{\childdocname}\let\jobname\childdocmain\||fi|\\
\end{tabular}
\end{center}
%
Instead of |\childdocof{|\textit{main}|}| just include the main file
at the top of each child file:
%
\begin{center}
|\input{|\textit{main}|}|
\end{center}
%
A simple redirection |\childdocforward{|\textit{dest}|}| is achieved by:
%
\begin{center}
|\def\jobname{|\textit{dest}|}\input{\jobname}|
\end{center}
%
The redirection with prefix
|\childdocforwardprefix[|\textit{prefix}|]{|\textit{dest}|}|
is accomplished by:
%
\begin{center}
\begin{tabular}{l}
|{\edef\jobname{\scantokens\expandafter{\jobname\noexpand}}|\\
|\def\redirectjob |\textit{prefix}|#1~~~{\gdef\jobname{|\textit{dest}|#1}}|\\
|\expandafter\redirectjob\jobname~~~}\input{\jobname}|
\end{tabular}
\end{center}

In an alternative approach,
child documents can be compiled by a specific command line
without additional code or specific definitions:
%
\begin{center}
|... -jobname "|\textit{target}|" "|[\textit{flags}]%
|\includeonly{|\textit{dest}|}\input{|\textit{main}|}"|
\end{center}
%

%%%%%%%%%%%%%%%%%%%%%%%%%%%%%%%%%%%%%%%%%%%%%%%%%%%%%%%%%%%%%%%%%%%%%%%%%%%%%%%%
%%%%%%%%%%%%%%%%%%%%%%%%%%%%%%%%%%%%%%%%%%%%%%%%%%%%%%%%%%%%%%%%%%%%%%%%%%%%%%%%
\section{Information}

%%%%%%%%%%%%%%%%%%%%%%%%%%%%%%%%%%%%%%%%%%%%%%%%%%%%%%%%%%%%%%%%%%%%%%%%%%%%%%%%
\subsection{Copyright}

Copyright \copyright{} 2017--2018 Niklas Beisert

This work may be distributed and/or modified under the
conditions of the \LaTeX{} Project Public License, either version 1.3
of this license or (at your option) any later version.
The latest version of this license is in
  \url{http://www.latex-project.org/lppl.txt}
and version 1.3 or later is part of all distributions of \LaTeX{}
version 2005/12/01 or later.

This work has the LPPL maintenance status `maintained'.

The Current Maintainer of this work is Niklas Beisert.

This work consists of the files |README.txt|, |childdoc.ins| and |childdoc.dtx|
as well as the derived files |childdoc.def|, |cdocsamp.tex|
with |cdocsch1.tex|, |cdocsch2.tex|, |cdocspt3.tex|, |cdocspt4.tex|,
|cdocsdrf.tex|, |cdocsfn1.tex|, |cdocsfn2.tex|
as well as |childdoc.pdf|.

%%%%%%%%%%%%%%%%%%%%%%%%%%%%%%%%%%%%%%%%%%%%%%%%%%%%%%%%%%%%%%%%%%%%%%%%%%%%%%%%
\subsection{Files and Installation}

The package consists of the files:
%
\begin{center}
\begin{tabular}{ll}
    |README.txt|   & readme file \\
    |childdoc.ins| & installation file \\
    |childdoc.dtx| & source file \\
    |childdoc.def| & definition file \\
    |cdocsamp.tex| & sample main file \\
    |cdocsch1.tex| & sample include file \\
    |cdocsch2.tex| & sample include file \\
    |cdocspt3.tex| & sample part file \\
    |cdocspt4.tex| & sample part file \\
    |cdocsdrf.tex| & sample redirection file \\
    |cdocsfn1.tex| & sample redirection file \\
    |cdocsfn2.tex| & sample redirection file \\
    |childdoc.pdf| & manual
\end{tabular}
\end{center}
%
The distribution consists of the files
|README.txt|, |childdoc.ins| and |childdoc.dtx|.
%
\begin{itemize}
\item
Run (pdf)\LaTeX{} on |childdoc.dtx|
to compile the manual |childdoc.pdf| (this file).
\item
Run \LaTeX{} on |childdoc.ins| to create the definitions file |childdoc.def|
and the sample |cdocsamp.tex| with include files
|cdocsch1.tex|, |cdocsch2.tex|, |cdocspt3.tex|, |cdocspt4.tex|,
|cdocsdrf.tex|, |cdocsfn1.tex|, |cdocsfn2.tex|.
Then copy the file |childdoc.def| to an appropriate directory of your \LaTeX{}
distribution, e.g.\ \textit{texmf-root}|/tex/latex/childdoc|.
\end{itemize}

%%%%%%%%%%%%%%%%%%%%%%%%%%%%%%%%%%%%%%%%%%%%%%%%%%%%%%%%%%%%%%%%%%%%%%%%%%%%%%%%
\subsection{Related CTAN Packages}

There are several other packages which offer a similar functionality:
%
\begin{itemize}
\item
The packages
\href{http://ctan.org/pkg/docmute}{\textsf{docmute}},
\href{http://ctan.org/pkg/includex}{\textsf{includex}} and
\href{http://ctan.org/pkg/standalone}{\textsf{standalone}}
provide commands to include only the document body of
a child file thus allowing both files to be compiled individually.
\item
The packages \href{http://ctan.org/pkg/subdocs}{\textsf{subdocs}}
and \href{http://ctan.org/pkg/subfiles}{\textsf{subfiles}}
provide structures in which the main and child documents can be
encapsulated and allowing them to be compiled individually.
The inclusion mechanism is different from the conventional |\include|.
\item
The package \href{http://ctan.org/pkg/combine}{\textsf{combine}}
is an elaborate solution to combine several documents into one.
\end{itemize}
%
See also the CTAN topic \href{http://ctan.org/topic/subdocs}{\textsf{subdocs}}
for further related packages.
The present package differs from the above solutions in that
a document structure constructed with the conventional |\include| mechanism
just needs two extra commands at the top of every file
such that all constituent files can be compiled individually.

%%%%%%%%%%%%%%%%%%%%%%%%%%%%%%%%%%%%%%%%%%%%%%%%%%%%%%%%%%%%%%%%%%%%%%%%%%%%%%%%
%\subsection{Feature Suggestions}
%
%The following is a list of features which may be useful for future
%versions of this package:
%%
%\begin{itemize}
%\item
%\ldots
%\end{itemize}

%%%%%%%%%%%%%%%%%%%%%%%%%%%%%%%%%%%%%%%%%%%%%%%%%%%%%%%%%%%%%%%%%%%%%%%%%%%%%%%%
\subsection{Revision History}

%%%%%%%%%%%%%%%%%%%%%%%%%%%%%%%%%%%%%%%%
\paragraph{v2.0:} 2018/12/30

\begin{itemize}
\item
immediate forward processing
\item
added |\childdocby| mechanism
\item
manual restructured
\end{itemize}

%%%%%%%%%%%%%%%%%%%%%%%%%%%%%%%%%%%%%%%%
\paragraph{v1.6:} 2018/01/17

\begin{itemize}
\item
application for development of include files
\item
corrections to manual
\end{itemize}

%%%%%%%%%%%%%%%%%%%%%%%%%%%%%%%%%%%%%%%%
\paragraph{v1.5:} 2017/05/21

\begin{itemize}
\item
more complete structuring introduced
\item
|\childdocof| introduced
\item
|\childdoc| renamed to |\childdocmain|
\item
|\childredirect| renamed to |\childdocforward| and |\childdocforwardprefix|
and functionality expanded
\end{itemize}

%%%%%%%%%%%%%%%%%%%%%%%%%%%%%%%%%%%%%%%%
\paragraph{v1.0:} 2017/04/27

\begin{itemize}
\item
manual and install package
\item
first version published on CTAN
\end{itemize}

%%%%%%%%%%%%%%%%%%%%%%%%%%%%%%%%%%%%%%%%
\paragraph{v0.6:} 2017/04/26

\begin{itemize}
\item
redirection mechanism added
\end{itemize}

%%%%%%%%%%%%%%%%%%%%%%%%%%%%%%%%%%%%%%%%
\paragraph{v0.5:} 2017/04/26

\begin{itemize}
\item
functionality in definition file
\end{itemize}


%%%%%%%%%%%%%%%%%%%%%%%%%%%%%%%%%%%%%%%%%%%%%%%%%%%%%%%%%%%%%%%%%%%%%%%%%%%%%%%%
%%%%%%%%%%%%%%%%%%%%%%%%%%%%%%%%%%%%%%%%%%%%%%%%%%%%%%%%%%%%%%%%%%%%%%%%%%%%%%%%
%%%%%%%%%%%%%%%%%%%%%%%%%%%%%%%%%%%%%%%%%%%%%%%%%%%%%%%%%%%%%%%%%%%%%%%%%%%%%%%%
\appendix

\settowidth\MacroIndent{\rmfamily\scriptsize 000\ }

 \DocInput{childdoc.dtx}

\end{document}
%</driver>
% \fi
%
% %%%%%%%%%%%%%%%%%%%%%%%%%%%%%%%%%%%%%%%%%%%%%%%%%%%%%%%%%%%%%%%%%%%%%%%%%%%%%%
% %%%%%%%%%%%%%%%%%%%%%%%%%%%%%%%%%%%%%%%%%%%%%%%%%%%%%%%%%%%%%%%%%%%%%%%%%%%%%%
% \section{Sample}
%\iffalse
%<*samplemain>
%\fi
%
% The following presents a sample document
% with two chapters, two parts, a title page,
% a compile flag as well as three forwarding files to set the flag.
% It consists of eight |.tex| files:
% \begin{center}
% \begin{tabular}{ll}
% |cdocsamp.tex|&main file\\
% |cdocsch1.tex|&include file for chapter 1\\
% |cdocsch2.tex|&include file for chapter 2\\
% |cdocspt3.tex|&include file for part 3\\
% |cdocspt4.tex|&include file for part 4\\
% |cdocsdrf.tex|&forwarding file for main file in draft mode\\
% |cdocsfi1.tex|&forwarding file for final version of chapter 1\\
% |cdocsfi2.tex|&forwarding file for final version of chapter 2\\
% \end{tabular}
% \end{center}
% Each of the eight files can be compiled directly by the \LaTeX{} compiler.
%
% %%%%%%%%%%%%%%%%%%%%%%%%%%%%%%%%%%%%%%
% \paragraph{Main File.}
%
% The main file is called |cdocsamp.tex|.
%
% Load the \textsf{childdoc} definitions and
% declare the filename for the main document:
%    \begin{macrocode}
\input{childdoc.def}
\childdocmain{}
%    \end{macrocode}

% Optional override for |\version| flag:
%    \begin{macrocode}
%%\ifchilddoc\else\providecommand{\version}{draft}\fi
%    \end{macrocode}

% Define the default values for the |\version| flag
% (|final| for the main file and |draft| for childs):
%    \begin{macrocode}
\ifchilddoc
\providecommand{\version}{draft}
\else
\providecommand{\version}{final}
\fi
%    \end{macrocode}

% Load the standard document class:
%    \begin{macrocode}
\documentclass[12pt]{article}
%    \end{macrocode}

% Start the document body:
%    \begin{macrocode}
\begin{document}
%    \end{macrocode}

% Declare a title page.
% Print title, part of document being processed and version flag:
%    \begin{macrocode}
\addtocounter{page}{-1}
\begin{center}
{\LARGE\bfseries{}childdoc example\par}
\vspace{1cm}
\ifchilddoc
\ifchilddocmanual part\else chapter\fi:
`\childdocname' of `\childdocjob'\par
\else
main document: `\childdocjob'\par
\fi
version: \version\par
\end{center}
\newpage
%    \end{macrocode}

% Manually include selected file,
% otherwise process as usual:
%    \begin{macrocode}
\ifchilddocmanual
\section*{part `\childdocname'}
\input{\childdocname}
\else
%    \end{macrocode}

% Include the two chapters:
%    \begin{macrocode}
\include{cdocsch1}
\include{cdocsch2}
%    \end{macrocode}

% Include the two parts unless only chapters should be displayed:
%    \begin{macrocode}
\ifchilddoc\else
\section{part three}
\input{cdocspt3}
\section{part four}
\input{cdocspt4}
\fi
%    \end{macrocode}

% Process as usual until here:
%    \begin{macrocode}
\fi
%    \end{macrocode}

% End of document body:
%    \begin{macrocode}
\end{document}
%    \end{macrocode}
%\iffalse
%</samplemain>
%\fi
%
% %%%%%%%%%%%%%%%%%%%%%%%%%%%%%%%%%%%%%%
% \paragraph{Chapter Include Files.}
%
% The include files are called |cdocsch1.tex| and |cdocsch2.tex|.
%
%\iffalse
%<*samplechap1|samplechap2>
%\fi

% Optional override for |\version| flag:
%    \begin{macrocode}
%%\providecommand{\version}{final}
%    \end{macrocode}

% Include the main document:
%    \begin{macrocode}
\input{childdoc.def}
\childdocof{cdocsamp}
%    \end{macrocode}

%\iffalse
%</samplechap1|samplechap2>
%\fi
%
%\iffalse
%<*samplechap1>
%\fi
% Some text for chapter 1:
%    \begin{macrocode}
\section{one}
some text in chapter one
%    \end{macrocode}

%\iffalse
%</samplechap1>
%\fi
% Some text for chapter 2:
%\iffalse
%<*samplechap2>
%\fi
%    \begin{macrocode}
\section{two}
more text in chapter two
%    \end{macrocode}

%\iffalse
%</samplechap2>
%\fi
%
% %%%%%%%%%%%%%%%%%%%%%%%%%%%%%%%%%%%%%%
% \paragraph{Part Include Files.}
%
% The include files are called |cdocspt3.tex| and |cdocspt4.tex|.
%
%\iffalse
%<*samplepart3|samplepart4>
%\fi

% Optional override for |\version| flag:
%    \begin{macrocode}
%%\providecommand{\version}{final}
%    \end{macrocode}

% Include the main document:
%    \begin{macrocode}
\input{childdoc.def}
\childdocby{cdocsamp}
%    \end{macrocode}

%\iffalse
%</samplepart3|samplepart4>
%\fi
%
%\iffalse
%<*samplepart3>
%\fi
% Some text for part 3:
%    \begin{macrocode}
some text in part three
%    \end{macrocode}

%\iffalse
%</samplepart3>
%\fi
% Some text for part 4:
%\iffalse
%<*samplepart4>
%\fi
%    \begin{macrocode}
more text in part four
%    \end{macrocode}

%\iffalse
%</samplepart4>
%\fi
%
% %%%%%%%%%%%%%%%%%%%%%%%%%%%%%%%%%%%%%%
% \paragraph{Forwarding for a Complete Draft.}
%
% The following forwarding file |cdocsdrf.tex|
% compiles the main document in draft mode:
%\iffalse
%<*sampledraft>
%\fi
%    \begin{macrocode}
\def\version{draft}
\input{childdoc.def}
\childdocforward{cdocsamp}
%    \end{macrocode}

%\iffalse
%</sampledraft>
%\fi
%
% %%%%%%%%%%%%%%%%%%%%%%%%%%%%%%%%%%%%%%
% \paragraph{Forwarding for Final Version of the Chapters.}
%
% The following forwarding files |cdocsfn1.tex| and |cdocsfn2.tex|
% (with identical content)
% compile the final versions of the child documents
% |cdocsch1.tex| and |cdocsch2.tex|, respectively:
%\iffalse
%<*samplefinal>
%\fi
%    \begin{macrocode}
\def\version{final}
\input{childdoc.def}
\childdocforwardprefix[cdocsamp]{cdocsfn}{cdocsch}
%    \end{macrocode}

%\iffalse
%</samplefinal>
%\fi
%
% %%%%%%%%%%%%%%%%%%%%%%%%%%%%%%%%%%%%%%
% \paragraph{Command Line Processing.}
%
% The following three command lines generate the output files
% |cdocscld|, |cdocscl1| and |cdocscl2|
% which should be identical to
% |cdocsdrf|, |cdocsch1| and |cdocsfn2|, respectively:
% \begin{center}
% \begin{tabular}{l}
% |latex -jobname cdocscld \|\\
% |  "\def\version{draft}\input{childdoc.def}\childdocforward{cdocsamp}"|\\
% |latex -jobname cdocscl1 \|\\
% |  "\input{childdoc.def}\childdocforward[cdocsamp]{cdocsch1}"|\\
% |latex -jobname cdocscl2 \|\\
% |  "\def\version{final}\input{childdoc.def}\childdocforward{cdocsch2}"|
% \end{tabular}
% \end{center}
% Note that the trailing backslash on each first line
% merely continues the input to the second line
% (for convenient cut ant paste).
% Furthermore, the command |latex| can be replaced by any
% of its alternative versions such as |pdflatex|.
%
% %%%%%%%%%%%%%%%%%%%%%%%%%%%%%%%%%%%%%%%%%%%%%%%%%%%%%%%%%%%%%%%%%%%%%%%%%%%%%%
% %%%%%%%%%%%%%%%%%%%%%%%%%%%%%%%%%%%%%%%%%%%%%%%%%%%%%%%%%%%%%%%%%%%%%%%%%%%%%%
% \section{Implementation}
%\iffalse
%<*package>
%\fi
%
% This section describes the definitions file |childdoc.def|.

% The definitions cannot be loaded using |\usepackage| or |\RequirePackage|
% which has a mechanism to prevent loading a style file more than once.
% When loading the definitions by means of |\input|
% multiple instances have to be prevented manually:
%\iffalse
%This code needs to be before the `\ProvidesFile' directive
%which is defined at the beginning of this file.
%Therefore it is also placed there and commented out here.
%</package>
%<*discard>
%\fi
%    \begin{macrocode}
\ifdefined\childdocmain\endinput\fi
%    \end{macrocode}
%\iffalse
%</discard>
%<*package>
%\fi
%
% \macro{\ifchilddoc}
% \macro{\ifchilddocmanual}
% The conditional |\ifchilddoc| tells whether a
% child (true) or main (false) document is being compiled.
% The conditional |\ifchilddocmanual| tells whether
% the |\includeonly| mechanism is used (false) or
% the selection of child files must be performed manually (true).
% The definitions initialise to false:
%    \begin{macrocode}
\newif\ifchilddoc
\newif\ifchilddocmanual
%    \end{macrocode}

% \macro{\childdocname}
% \macro{\childdocjob}
% The macro |\childdocname| stores the name of the main document
% to be compiled. The macro |\childdocjob| stores the name of
% the document on which the \LaTeX{} compiler was originally invoked.
% The content of |\jobname| cannot be compared
% to filenames specified in the source due to different catcodes.
% The following code rescans |\jobname|, stores the result
% in |\childdocname| and saves a copy in |\childdocjob|:
%    \begin{macrocode}
\edef\childdocname{\scantokens\expandafter{\jobname\noexpand}}
\let\childdocjob\childdocname
%    \end{macrocode}

% \macro{\childdocdisable}
% The macro |\childdocdisable| prevents the main file
% from being processed more than once.
% At this stage, the main document command |\childdocmain|
% is assumed to be called once again where it should do nothing.
% Any subsequent call to it should prevent
% a secondary processing of the main document
% It overwrites the forwarding commands
% |\childdocof| and |\childdocforward|
% with empty macros to prevent further inclusions of the main document:
%    \begin{macrocode}
\newcommand{\childdocdisable}
{
  \renewcommand{\childdocmain}[1]{\renewcommand{\childdocmain}[1]{\endinput}}
  \renewcommand{\childdocof}[1]{}
  \renewcommand{\childdocby}[2][]{}
  \renewcommand{\childdocforward}[2][]{}
  \renewcommand{\childdocdisable}{}
}
%    \end{macrocode}

% \macro{\childdocmain}
% The macro |\childdocmain| is to be called at the top of the main file
% with nothing or the main filename (without extension) as argument.
% First, it breaks loops.
% If the argument is not empty and does not match |\childdocname|
% (which is set by the first inclusion of |childdoc.def|),
% |\ifchilddoc| is set to true, |\includeonly| is applied to the child file
% and |\jobname| is set to the main file
% (for proper handling of |.aux| files):
%    \begin{macrocode}
\newcommand{\childdocmain}[1]
{
  \childdocdisable\childdocmain{}
  \if?#1?\else
    \begingroup
      \def\childdoctmp{#1}
      \ifx\childdoctmp\childdocname
        \def\childdoctmp{}
      \else
        \def\childdoctmp
        {
          \childdoctrue
          \includeonly{\childdocname}
          \def\childdocjob{#1}
          \def\jobname{#1}
        }
      \fi
      \expandafter
    \endgroup
    \childdoctmp
  \fi
}
%    \end{macrocode}

% \macro{\childdocof}
% The command |\childdocof| redirects
% compilation to the main file |#1|.
%    \begin{macrocode}
\newcommand{\childdocof}[1]
{
  \childdocdisable
  \childdoctrue
  \includeonly{\childdocname}
  \def\jobname{#1}
  \def\childdocjob{#1}
  \input{#1}
}
%    \end{macrocode}

% \macro{\childdocby}
% The command |\childdocby| ....
%    \begin{macrocode}
\newcommand{\childdocby}[2][]
{
  \childdocdisable
  \childdoctrue
  \childdocmanualtrue
  \if?#1?\else
    \def\jobname{#2}
  \fi
  \def\childdocjob{#2}
  \input{#2}
  \endinput
}
%    \end{macrocode}

% \macro{\childdocforward}
% The command |\childdocforward| redirects
% compilation to the main file or
% (if the optional argument is given) a child file.
% Parameters are set as if the main file
% or a child file starting with |\childdocof| was compiled.
% Then compilation is handed over to the main file:
%    \begin{macrocode}
\newcommand{\childdocforward}[2][]
{
  \begingroup
    \if?#1?
      \def\childdoctmp
      {
        \def\childdocname{#2}
        \def\childdocjob{#2}
        \def\jobname{#2}
        \input{#2}
        \endinput
      }
    \else
      \def\childdoctmp
      {
        \childdocdisable
        \def\childdocname{#2}
        \childdoctrue
        \includeonly{#2}
        \def\childdocjob{#1}
        \def\jobname{#1}
        \input{#1}
        \endinput
      }
    \fi
    \expandafter
  \endgroup
  \childdoctmp
}
%    \end{macrocode}

% \macro{\childdocforwardprefix}
% The command |\childdocforwardprefix| redirects
% compilation to the main or a child file by means of a pattern.
% The prefix |#1| in the current filename is replaced by |#2|
% and the suffix of the current filename is kept
% (it is assumed that the filename does not contain the substring `|~~~|'
% which is used as a delimiter).
% Compilation is handed over to the new file by |\childdocforward|:
%    \begin{macrocode}
\newcommand{\childdocforwardprefix}[3][]
{
  \begingroup
    \def\childdocextract #2##1~~~{\def\childdoctmp{\childdocforward[#1]{#3##1}}}
    \expandafter\childdocextract\childdocname~~~
    \expandafter
  \endgroup
  \childdoctmp
}
%    \end{macrocode}

% \macro{\childdoc}
% The deprecated macro |\childdoc| is a legacy version of |\childdocmain|:
%    \begin{macrocode}
\newcommand{\childdoc}{\childdocmain}
%    \end{macrocode}

% \macro{\childdocredirect}
% The deprecated macro |\childdocredirect| is a legacy version
% of |\childdocforward| and |\childdocforwardprefix|:
%    \begin{macrocode}
\newcommand{\childdocredirect}[2][]
{
  \begingroup
    \if?#1?
      \def\childdoctmp{\childdocforward{#2}}
    \else
      \def\childdoctmp{\childdocforwardprefix{#1}{#2}}
    \fi
    \expandafter
  \endgroup
  \childdoctmp
}
%    \end{macrocode}

%\iffalse
%</package>
%\fi
%
\endinput

\childdocforward{cdocsamp}
%    \end{macrocode}

%\iffalse
%</sampledraft>
%\fi
%
% %%%%%%%%%%%%%%%%%%%%%%%%%%%%%%%%%%%%%%
% \paragraph{Forwarding for Final Version of the Chapters.}
%
% The following forwarding files |cdocsfn1.tex| and |cdocsfn2.tex|
% (with identical content)
% compile the final versions of the child documents
% |cdocsch1.tex| and |cdocsch2.tex|, respectively:
%\iffalse
%<*samplefinal>
%\fi
%    \begin{macrocode}
\def\version{final}
% \iffalse
%
% childdoc.dtx Copyright (C) 2017-2018 Niklas Beisert
%
% This work may be distributed and/or modified under the
% conditions of the LaTeX Project Public License, either version 1.3
% of this license or (at your option) any later version.
% The latest version of this license is in
%   http://www.latex-project.org/lppl.txt
% and version 1.3 or later is part of all distributions of LaTeX
% version 2005/12/01 or later.
%
% This work has the LPPL maintenance status `maintained'.
%
% The Current Maintainer of this work is Niklas Beisert.
%
% This work consists of the files childdoc.dtx and childdoc.ins
% and the derived files childdoc.def and cdocsamp.tex with
% cdocsch1.tex, cdocsch2.tex, cdocsdrf.tex, cdocsfn1.tex, cdocsfn2.tex.
%
%<package>\ifdefined\childdocmain\endinput\fi
%<package>\ProvidesFile{childdoc.def}[2018/12/30 v2.0 child document driver]
%<samplemain>\ProvidesFile{cdocsamp.tex}[2018/12/30 v2.0 sample for childdoc]
%<*driver>
%\ProvidesFile{childdoc.drv}[2018/12/30 v2.0 childdoc reference manual file]
\PassOptionsToClass{10pt,a4paper}{article}
\documentclass{ltxdoc}

\usepackage[margin=35mm]{geometry}
\usepackage{hyperref}
\usepackage{hyperxmp}
\usepackage[usenames]{color}

\hypersetup{colorlinks=true}
\hypersetup{pdfstartview=FitH}
\hypersetup{pdfpagemode=UseNone}
\hypersetup{pdfsource={}}
\hypersetup{pdflang={en-UK}}
\hypersetup{pdfcopyright={Copyright 2017-2018 Niklas Beisert.
  This work may be distributed and/or modified under the
  conditions of the LaTeX Project Public License, either version 1.3
  of this license or (at your option) any later version.}}
\hypersetup{pdflicenseurl={http://www.latex-project.org/lppl.txt}}
\hypersetup{pdfcontactaddress={ETH Zurich, ITP, HIT K,
  Wolfgang-Pauli-Strasse 27}}
\hypersetup{pdfcontactpostcode={8093}}
\hypersetup{pdfcontactcity={Zurich}}
\hypersetup{pdfcontactcountry={Switzerland}}
\hypersetup{pdfcontactemail={nbeisert@itp.phys.ethz.ch}}
\hypersetup{pdfcontacturl={http://people.phys.ethz.ch/\xmptilde nbeisert/}}

\newcommand{\secref}[1]{\hyperref[#1]{section \ref*{#1}}}

\parskip1ex
\parindent0pt
\let\olditemize\itemize
\def\itemize{\olditemize\parskip0pt}

\begin{document}

\title{The \textsf{childdoc} Package}
\hypersetup{pdftitle={The childdoc Package}}
\author{Niklas Beisert\\[2ex]
  Institut f\"ur Theoretische Physik\\
  Eidgen\"ossische Technische Hochschule Z\"urich\\
  Wolfgang-Pauli-Strasse 27, 8093 Z\"urich, Switzerland\\[1ex]
  \href{mailto:nbeisert@itp.phys.ethz.ch}
  {\texttt{nbeisert@itp.phys.ethz.ch}}}
\hypersetup{pdfauthor={Niklas Beisert}}
\hypersetup{pdfsubject={Manual for the LaTeX2e Package childdoc}}
\date{30 December 2018, \textsf{v2.0}}
\maketitle

\begin{abstract}\noindent
\textsf{childdoc} is a \LaTeXe{} package
that enables the direct compilation
of document sections included by |\include|
to individual files.
\end{abstract}

\begingroup
\parskip0ex
\tableofcontents
\endgroup

%%%%%%%%%%%%%%%%%%%%%%%%%%%%%%%%%%%%%%%%%%%%%%%%%%%%%%%%%%%%%%%%%%%%%%%%%%%%%%%%
%%%%%%%%%%%%%%%%%%%%%%%%%%%%%%%%%%%%%%%%%%%%%%%%%%%%%%%%%%%%%%%%%%%%%%%%%%%%%%%%
\section{Introduction}

\LaTeX{} provides a mechanism to structure a large document (such as a book)
into a main file and several child files (containing the chapters)
using the |\include| command.
This mechanism is beneficial for documents
which span hundreds of pages in order to
make the source file(s) more manageable.
Moreover, compilation can be restricted to
selected child files by means of the |\includeonly| command.
The latter feature can be used to reduce the compilation time while editing
(this was significantly more useful in the earlier days of \LaTeX{})
or to generate a smaller document which is easier to navigate.
Another application of |\includeonly| is to generate
documents consisting of selected parts of the complete document.

However, there are a few drawbacks of the plain |\include| mechanism:
\begin{itemize}
\item
The child files cannot be compiled on their own,
they can only be compiled via the main file.
A naive editing environment
(such as a text editor with an option
to have the current file processed by \LaTeX)
may require one to switch to the main file before compiling;
attempting to compile the child file produces errors.
\item
The main file must be modified (each time)
to adjust the |\includeonly| command
to the present needs. This easily leaves the main file in a messy state.
\item
The generated document will always carry the filename
of the main document. This is inconvenient if
several child files are to be compiled and
to be kept for distribution.
\end{itemize}

The present package provides a simple interface
to make child files individually compilable by \LaTeX{}.
Compiling a child file then has the same effect as compiling
the main file with an |\includeonly| command
to select the appropriate child.
Moreover the generated document will carry the name of the child
rather than the main file.
This resolves all three above issues.

This feature is meant to make the editing of books,
thesis documents and lecture notes somewhat more convenient.
However, the package can also be used efficiently for
composing a series of documents (such as exercise sheets)
which are typically distributed individually.
It then assists the author in generating the individual documents
(potentially in different versions)
as well as a document containing the collected series.
Another application is in developing style files
or other kinds of included material
where compilation of the style file could redirect
to a sample or test file.

%%%%%%%%%%%%%%%%%%%%%%%%%%%%%%%%%%%%%%%%%%%%%%%%%%%%%%%%%%%%%%%%%%%%%%%%%%%%%%%%
%%%%%%%%%%%%%%%%%%%%%%%%%%%%%%%%%%%%%%%%%%%%%%%%%%%%%%%%%%%%%%%%%%%%%%%%%%%%%%%%
\section{Usage}

First of all, the package \textsf{childdoc} is \emph{not} a standard
\LaTeXe{} |.sty| style file! Therefore it needs to be invoked in
a non-standard way.

%%%%%%%%%%%%%%%%%%%%%%%%%%%%%%%%%%%%%%%%%%%%%%%%%%%%%%%%%%%%%%%%%%%%%%%%%%%%%%%%
\subsection{Included Files}
\label{sec:include}

%%%%%%%%%%%%%%%%%%%%%%%%%%%%%%%%%%%%%%%%
\DescribeMacro{\childdocmain}
To use the package, add the commands
\begin{center}
\begin{tabular}{l}
|\input{childdoc.def}|\\
|\childdocmain{}|\\
\end{tabular}
\end{center}
at the very top of the main \LaTeX{} file,
in particular \emph{before} the |\documentclass| statement!
The argument of |\childdocmain| should be left empty
(but it must be present).

%%%%%%%%%%%%%%%%%%%%%%%%%%%%%%%%%%%%%%%%
\DescribeMacro{\childdocof}
Furthermore, add the commands
\begin{center}
\begin{tabular}{l}
|\input{childdoc.def}|\\
|\childdocof{|\textit{main}|}|\\
\end{tabular}
\end{center}
at the top of every child file \textit{child}
which is included by |\include{|\textit{child}|}|
from within the main file
(or at least for those files to be compiled individually).
The argument \textit{main} must be the filename of the main file.

There are a couple of
considerations in setting up the main and child documents:

%%%%%%%%%%%%%%%%%%%%%%%%%%%%%%%%%%%%%%%%
\paragraph{Restrictions.}

Please note the following restrictions:
\begin{itemize}
\item
|\childdocmain| must be called with one argument \textit{main}
to ensure compatibility with earlier version of the package.
It must either be empty (|\childdocmain{}|)
or precisely match the filename of the main file in which it is specified.
See \secref{sec:detection} for further information.
\item
The filename \textit{main} must be specified without the |.tex| extension.
\item
The filename \textit{main} is case sensitive
(even in case-insensitive file systems)
due to internal string comparison.
\item
The argument \textit{main} should be fully expanded, it cannot be a macro.
\item
Subdirectories and special characters should be avoided in filenames.
\item
The command |\childdocmain{|\textit{main}|}| must be followed by a whitespace.
It should not be followed immediately by another command
or by a comment mark `|%|'.
This is because the \TeX{} parser reads the token immediately following
the argument of |\childdocmain| and puts it
at the beginning of every child section;
however, a white\-space is ignored.
\end{itemize}

%%%%%%%%%%%%%%%%%%%%%%%%%%%%%%%%%%%%%%%%
\paragraph{Content of Main File.}

It is advisable to place all content in the child files included by |\include|.
Any output contained in the main file will appear in all child documents
unless suppressed manually;
it cannot be suppressed automatically by the |\includeonly| directive
and thus should normally be avoided.
A method to include some content in the main file
by means of conditional processing is described in \secref{sec:conditional}.

%%%%%%%%%%%%%%%%%%%%%%%%%%%%%%%%%%%%%%%%
\paragraph{Page Numbering.}

When only a part of the document is compiled,
the appropriate numbering of pages
(as well as other status parameters)
is determined from the |.aux| files.
The latter contain information from previous passes.
However this information needs to propagate through
all intermediate child documents.
Therefore the page numbering in child documents may well
be inconsistent until the complete document is compiled at least once.

A useful (if unconventional) way to always ensure a consistent
page numbering is to restart the numbering in each child document
and denote the pages by `\textit{child}|.|\textit{page}'
where \textit{child} represents the chapter/section number of the child file.
This can be achieved by the command
|\numberwithin{page}{|\textit{child}|}|
of the \textsf{amsmath} package
where \textit{child} can be |chapter| or |section|
depending on the chosen structuring.
Alternatively, one can modify the macro |\thepage| appropriately
and reset the counter |page| at the start of each child file.

%%%%%%%%%%%%%%%%%%%%%%%%%%%%%%%%%%%%%%%%%%%%%%%%%%%%%%%%%%%%%%%%%%%%%%%%%%%%%%%%
\subsection{Conditional Processing}
\label{sec:conditional}

The package provides a mechanism to compile different versions
of a document. To customise the versions further some conditional processing
can come in handy to distinguish which version is being compiled.
The package provides two macros to describe the compilation context:

%%%%%%%%%%%%%%%%%%%%%%%%%%%%%%%%%%%%%%%%
\DescribeMacro{\ifchilddoc}
The conditional |\ifchilddoc| distinguishes between the compilation of
child documents and the main document:
%
\begin{center}
|\ifchilddoc |\textit{child-code}| |[|\||else |\textit{main-code}]| \||fi|
\end{center}

%%%%%%%%%%%%%%%%%%%%%%%%%%%%%%%%%%%%%%%%
\DescribeMacro{\childdocname}
\DescribeMacro{\childdocjob}
The macro |\childdocname| contains the filename (without extension)
of the main or child file being processed.
Note that |\childdocjob| will always contain the name of the main file.

%%%%%%%%%%%%%%%%%%%%%%%%%%%%%%%%%%%%%%%%
\paragraph{Title Page.}

Conditional processing can be used to include a title or banner page
in the main document when proper precautions are taken.
Importantly, the code in the main file should ensure that the page counter
(as well as other status parameters which are stored in the |.aux| files)
takes the same value after the conditional processing.
Otherwise the page numbers may take divergent values
depending on which part is compiled.

For example, a title page could be declared by:
%
\begin{center}
\begin{tabular}{l}
|\ifchilddoc\||else|\\
|\addtocounter{page}{-1}|\\
\textit{code for title page}\\
|\newpage|\\
|\||fi|
\end{tabular}
\end{center}
%
A banner page for the child documents can be generated by:
%
\begin{center}
\begin{tabular}{l}
|\ifchilddoc|\\
|\addtocounter{page}{-1}|\\
\textit{code for banner page}\\
|\newpage|\\
|\||fi|
\end{tabular}
\end{center}
%
Here one could write a message such as:
\begin{center}
|This is the part \childdocname{} of \childdocjob{}.|
\end{center}

%%%%%%%%%%%%%%%%%%%%%%%%%%%%%%%%%%%%%%%%%%%%%%%%%%%%%%%%%%%%%%%%%%%%%%%%%%%%%%%%
\subsection{Flags}
\label{sec:flags}

The package makes it easy to generate different versions
of the main or child documents.
To this end compilation flags can be defined
and assigned different default values.
They will be particularly useful in conjunction
with the forwarding mechanism described in \secref{sec:forward}.

For example, it may be useful to have a flag |\version|
which can be set to |draft| or |final|.
The document source will contain some conditional code
depending on the value of |\version|.
Suppose further, the flag should default to |final| for the main file
and to |draft| for child files
which is a natural assignment for editing the document.
This is achieved by placing the following code
in the preamble of the main document
(below the |\childdocmain| directive):
%
\begin{center}
\begin{tabular}{l}
|\ifchilddoc|\\
|\providecommand{\version}{draft}|\\
|\||else|\\
|\providecommand{\version}{final}|\\
|\||fi|
\end{tabular}
\end{center}
%
The definition by |\providecommand| makes sure
that previous definitions are not overwritten.
Further statements |\providecommand{\version}{...}|
can thus be added before the above code to override it.

For the main file, one might add a line
(between |\childdocmain| and the above block)
%
\begin{center}
|%\ifchilddoc\||else\providecommand{\version}{draft}\||fi|
\end{center}
%
which can be uncommented to produce a draft version.
Likewise one can add a line to the very top of a child file
(above the |\childdocof{|\textit{main}|}| directive)
%
\begin{center}
|%\providecommand{\version}{final}|
\end{center}
%
which can be uncommented to produce the final version of this child document.

%%%%%%%%%%%%%%%%%%%%%%%%%%%%%%%%%%%%%%%%%%%%%%%%%%%%%%%%%%%%%%%%%%%%%%%%%%%%%%%%
\subsection{Forwarding}
\label{sec:forward}

Different versions of the main or child documents
using compilation flags as described in \secref{sec:flags}
can be (permanently) stored in different files
for convenient compilation, viewing and distribution.
To this end, the package defines a command
to pass on compilation to a different file:

%%%%%%%%%%%%%%%%%%%%%%%%%%%%%%%%%%%%%%%%
\DescribeMacro{\childdocforward}
The command |\childdocforward| redirects processing to
another source file:
%
\begin{center}
\begin{tabular}{l}
|\input{childdoc.def}|\\
|\childdocforward[|\textit{main}|]{|\textit{dest}|}|\\
\end{tabular}
\end{center}
%
The argument \textit{dest} is the destination file
(without extension).
It should be the main file or one of the child files.
Note that further \textsf{childdoc} directives
such as |\childdocof| and |\childdocforward|
in the indicated file will be processed in this form.
The optional argument \textit{main}
passes on directly to the main file \textit{main}
while pretending to compile the child \textit{dest}.
This form behaves as if \textit{dest}
issues |\childdocof{|\textit{main}|}| right away,
and no further \textsf{childdoc} directives will be processed.

%%%%%%%%%%%%%%%%%%%%%%%%%%%%%%%%%%%%%%%%
\DescribeMacro{\...prefix}
In the alternative form |\childdocforwardprefix|,
%
\begin{center}
\begin{tabular}{l}
|\input{childdoc.def}|\\
|\childdocforwardprefix[|\textit{main}|]{|\textit{prefix}|}{|\textit{dest}|}|
\end{tabular}
\end{center}
%
the destination file is determined by a pattern
depending on the current file:
To make this work, the current file must be called
`{\textit{prefix}\hspace{0.2em}\textit{suffix}}'
with \textit{prefix} matching precisely the argument.
Processing is then passed on to the file
`{\textit{dest}\hspace{0.2em}\textit{suffix}}'.
Surely, the same effect is achieved by
directly specifying the
argument `{\textit{dest}\hspace{0.2em}\textit{suffix}}'
in the first form.
However, that requires to set up a different file
for each child. With the alternative form of the command
all these files can have exactly the same content
which simplifies setting them up and maintaining them.

For example, the following file |draft.tex|
with a compilation flag |\version| as described in \secref{sec:flags}
compiles the main document as a draft:
%
\begin{center}
\begin{tabular}{l}
|\def\version{draft}|\\
|\input{childdoc.def}|\\
|\childdocforward{|\textit{main}|}|
\end{tabular}
\end{center}
%
Likewise, the following files |final|\textit{nn}|.tex|
compile the final version of the child document
|child|\textit{nn}|.tex|:
%
\begin{center}
\begin{tabular}{l}
|\def\version{final}|\\
|\input{childdoc.def}|\\
|\childdocforwardprefix{final}{child}|
\end{tabular}
\end{center}
%

Note that when several versions of a main file and/or of each child file
are to be generated, it may be convenient to set up a |Makefile| or
shell script to automatise the process.

%%%%%%%%%%%%%%%%%%%%%%%%%%%%%%%%%%%%%%%%%%%%%%%%%%%%%%%%%%%%%%%%%%%%%%%%%%%%%%%%
\subsection{Command Line Processing}
\label{sec:commandline}

The effect of redirection files can also be achieved by invoking
the \LaTeX{} compiler with a more elaborate command line.
Most conveniently this should be done as part
of a shell script or a |Makefile|.

When using \textsf{childdoc} in the main file, the following
command lines effectively perform a redirection
(note that depending on the shell being used,
backslashes may have to be doubled: `|\|' $\to$ `|\\|'):
%
\begin{center}
|... -jobname "|\textit{target}|" |\\|"|[\textit{flags}]%
|\input{childdoc.def}\childdocforward[|\textit{main}|]{|\textit{dest}|}"|
\end{center}
%
Here \textit{target} is the name of the output file,
\textit{main} is the name of the main file
and \textit{dest} is the name of the main or child file to be processed
(all filenames without extensions).
The optional argument \textit{main} can be omitted
if \textit{main} matches \textit{dest}.
Optionally, compilation \textit{flags} can be defined via |\def| commands.
This command line makes the \TeX{} engine believe
it is compiling the file \textit{target}
whose content is specified as the latter parameter.
The provided code then forwards the processing to
\textit{main} or \textit{dest} as described in \secref{sec:forward}.

%%%%%%%%%%%%%%%%%%%%%%%%%%%%%%%%%%%%%%%%%%%%%%%%%%%%%%%%%%%%%%%%%%%%%%%%%%%%%%%%
\subsection{Include by Input}
\label{sec:input}

Including child documents by |\include| has some restrictions by design.
Most notably, the content of a child document always occupies
its own set of pages; pages cannot be shared between child documents.
Usually, this behaviour makes perfect sense
because each child document contain an essential part of the document.
However, in some situations it may be desirable to compose
a document from a collection of parts
without having mandatory page breaks between then.
For this case, the package
provides a mechanism to include parts
by |\input| which can also be processed individually.
However, by construction this mechanism
requires manual handling of the content to be output.

%%%%%%%%%%%%%%%%%%%%%%%%%%%%%%%%%%%%%%%%
\DescribeMacro{\ifchilddocmanual}
The main file should be prepared as usual, see \secref{sec:include}.
However, the document body must make a distinction
between processing of an individual part and of the main document, e.g.:
%
\begin{center}
\begin{tabular}{l}
|\ifchilddocmanual|\\
|\input{\childdocname}|\\
|\||else|\\
\textit{document body with }|\input{|\textit{part}|}|\\
|\||fi|
\end{tabular}
\end{center}
%
The conditional |\ifchilddocmanual| is true whenever
a part to be included by |\input| is being compiled,
and the name of the part is stored in |\childdocname|.

%%%%%%%%%%%%%%%%%%%%%%%%%%%%%%%%%%%%%%%%
\DescribeMacro{\childdocby}
Each part to be included by |\input| should start with:
%
\begin{center}
\begin{tabular}{l}
|\input{childdoc.def}|\\
|\childdocby{|\textit{main}|}|\\
\end{tabular}
\end{center}
%
The directive |\childdocby| is similar to |\childdocof|
described in \secref{sec:include},
but the subsequent selection of content must be done manually.
To that end, both |\ifchilddoc| and |\ifchilddocmanual|
will be true upon processing of a part,
and the name of the part is stored in |\childdocname|.
Note that |\jobname| will be set to the filename of the current part
so that each part receives an individual |.aux| file
that does not interfere with the |.aux| file(s) of the main document.
This behaviour can be altered by the alternative form
|\childdocby[*]{|\textit{main}|}| (with a non-empty optional argument)
which uses the |.aux| file of the main document
by setting |\jobname| to \textit{main}.

%%%%%%%%%%%%%%%%%%%%%%%%%%%%%%%%%%%%%%%%%%%%%%%%%%%%%%%%%%%%%%%%%%%%%%%%%%%%%%%%
\subsection{Driver Development}
\label{sec:driver}

The \textsf{childdoc} mechanism can also be use for the development
of definition files such as \LaTeX{} styles or classes.
This case differs from the above setup with multiple parts
included by |\include| in that no |\includeonly| should be invoked.
This can be achieved by starting the include file
(before |\ProvidesPackage|) with:
%
\begin{center}
\begin{tabular}{l}
|\input{childdoc.def}|\\
|\childdocforward{|\textit{main}|}|\\
\end{tabular}
\end{center}
%
or alternatively with:
%
\begin{center}
\begin{tabular}{l}
|\input{childdoc.def}|\\
|\childdocby{|\textit{main}|}|\\
\end{tabular}
\end{center}
%
Both forms have slightly different effects as described above.
The main file is prepared as usual, see \secref{sec:include}.

%%%%%%%%%%%%%%%%%%%%%%%%%%%%%%%%%%%%%%%%%%%%%%%%%%%%%%%%%%%%%%%%%%%%%%%%%%%%%%%%
\subsection{Legacy Detection}
\label{sec:detection}

The directive |\childdocmain| in the main file can detect
whether the complete document or merely a child is to be compiled
even without using the directive |\childdocof|.
This method is deprecated because it is less robust
and there is no compelling reason to use it;
it is merely provided for backward compatibility
and it may be removed in future versions.

If the detection mechanism is to be used,
it is mandatory to correctly specify
the filename of the main file as the argument of |\childdocmain|:
%
\begin{center}
\begin{tabular}{l}
|\input{childdoc.def}|\\
|\childdocmain{|\textit{main}|}|\\
\end{tabular}
\end{center}
%
If |\jobname| does not match the argument \textit{main} of |\childdocmain|,
it is assumed that |\jobname| points to the child file to be compiled.
When using |\childdocmain| with the main file specified as argument,
it suffices to start a child file
with just |\input{|\textit{main}|}|
without loading of the package and using |\childdocof|.
If instead all processing is done
with the appropriate \textsf{childdoc} directives,
the argument of \textit{main} of |\childdocmain| can be empty.

An alternative version of the command line processing described
in \secref{sec:commandline} using the detection mechanism reads:
%
\begin{center}
|... -jobname "|\textit{target}|" "|[\textit{flags}]%
[|\def\jobname{|\textit{dest}|}|]|\input{|\textit{main}|}"|
\end{center}

%%%%%%%%%%%%%%%%%%%%%%%%%%%%%%%%%%%%%%%%%%%%%%%%%%%%%%%%%%%%%%%%%%%%%%%%%%%%%%%%
\subsection{Manual Code}
\label{sec:manual}

In case one cannot be certain whether the definitions file |childdoc.def|
is installed on the target \TeX{} distribution
and one prefers not to ship it,
it is conceivable to paste a few relevant commands into the sources.

To that end, drop all statements |\input{childdoc.def}|
and perform the replacements as outlined below.
Instead of |\childdocmain{|\textit{main}|}| add the following code
to the top of the main file:
%
\begin{center}
\begin{tabular}{l}
|\||ifdefined\childdocname\endinput\||fi\newif\ifchilddoc|\\
|\edef\childdocname{\scantokens\expandafter{\jobname\noexpand}}|\\
|\def\childdocmain{|\textit{main}|}\||ifx\childdocmain\childdocname\||else|\\
|\childdoctrue\includeonly{\childdocname}\let\jobname\childdocmain\||fi|\\
\end{tabular}
\end{center}
%
Instead of |\childdocof{|\textit{main}|}| just include the main file
at the top of each child file:
%
\begin{center}
|\input{|\textit{main}|}|
\end{center}
%
A simple redirection |\childdocforward{|\textit{dest}|}| is achieved by:
%
\begin{center}
|\def\jobname{|\textit{dest}|}\input{\jobname}|
\end{center}
%
The redirection with prefix
|\childdocforwardprefix[|\textit{prefix}|]{|\textit{dest}|}|
is accomplished by:
%
\begin{center}
\begin{tabular}{l}
|{\edef\jobname{\scantokens\expandafter{\jobname\noexpand}}|\\
|\def\redirectjob |\textit{prefix}|#1~~~{\gdef\jobname{|\textit{dest}|#1}}|\\
|\expandafter\redirectjob\jobname~~~}\input{\jobname}|
\end{tabular}
\end{center}

In an alternative approach,
child documents can be compiled by a specific command line
without additional code or specific definitions:
%
\begin{center}
|... -jobname "|\textit{target}|" "|[\textit{flags}]%
|\includeonly{|\textit{dest}|}\input{|\textit{main}|}"|
\end{center}
%

%%%%%%%%%%%%%%%%%%%%%%%%%%%%%%%%%%%%%%%%%%%%%%%%%%%%%%%%%%%%%%%%%%%%%%%%%%%%%%%%
%%%%%%%%%%%%%%%%%%%%%%%%%%%%%%%%%%%%%%%%%%%%%%%%%%%%%%%%%%%%%%%%%%%%%%%%%%%%%%%%
\section{Information}

%%%%%%%%%%%%%%%%%%%%%%%%%%%%%%%%%%%%%%%%%%%%%%%%%%%%%%%%%%%%%%%%%%%%%%%%%%%%%%%%
\subsection{Copyright}

Copyright \copyright{} 2017--2018 Niklas Beisert

This work may be distributed and/or modified under the
conditions of the \LaTeX{} Project Public License, either version 1.3
of this license or (at your option) any later version.
The latest version of this license is in
  \url{http://www.latex-project.org/lppl.txt}
and version 1.3 or later is part of all distributions of \LaTeX{}
version 2005/12/01 or later.

This work has the LPPL maintenance status `maintained'.

The Current Maintainer of this work is Niklas Beisert.

This work consists of the files |README.txt|, |childdoc.ins| and |childdoc.dtx|
as well as the derived files |childdoc.def|, |cdocsamp.tex|
with |cdocsch1.tex|, |cdocsch2.tex|, |cdocspt3.tex|, |cdocspt4.tex|,
|cdocsdrf.tex|, |cdocsfn1.tex|, |cdocsfn2.tex|
as well as |childdoc.pdf|.

%%%%%%%%%%%%%%%%%%%%%%%%%%%%%%%%%%%%%%%%%%%%%%%%%%%%%%%%%%%%%%%%%%%%%%%%%%%%%%%%
\subsection{Files and Installation}

The package consists of the files:
%
\begin{center}
\begin{tabular}{ll}
    |README.txt|   & readme file \\
    |childdoc.ins| & installation file \\
    |childdoc.dtx| & source file \\
    |childdoc.def| & definition file \\
    |cdocsamp.tex| & sample main file \\
    |cdocsch1.tex| & sample include file \\
    |cdocsch2.tex| & sample include file \\
    |cdocspt3.tex| & sample part file \\
    |cdocspt4.tex| & sample part file \\
    |cdocsdrf.tex| & sample redirection file \\
    |cdocsfn1.tex| & sample redirection file \\
    |cdocsfn2.tex| & sample redirection file \\
    |childdoc.pdf| & manual
\end{tabular}
\end{center}
%
The distribution consists of the files
|README.txt|, |childdoc.ins| and |childdoc.dtx|.
%
\begin{itemize}
\item
Run (pdf)\LaTeX{} on |childdoc.dtx|
to compile the manual |childdoc.pdf| (this file).
\item
Run \LaTeX{} on |childdoc.ins| to create the definitions file |childdoc.def|
and the sample |cdocsamp.tex| with include files
|cdocsch1.tex|, |cdocsch2.tex|, |cdocspt3.tex|, |cdocspt4.tex|,
|cdocsdrf.tex|, |cdocsfn1.tex|, |cdocsfn2.tex|.
Then copy the file |childdoc.def| to an appropriate directory of your \LaTeX{}
distribution, e.g.\ \textit{texmf-root}|/tex/latex/childdoc|.
\end{itemize}

%%%%%%%%%%%%%%%%%%%%%%%%%%%%%%%%%%%%%%%%%%%%%%%%%%%%%%%%%%%%%%%%%%%%%%%%%%%%%%%%
\subsection{Related CTAN Packages}

There are several other packages which offer a similar functionality:
%
\begin{itemize}
\item
The packages
\href{http://ctan.org/pkg/docmute}{\textsf{docmute}},
\href{http://ctan.org/pkg/includex}{\textsf{includex}} and
\href{http://ctan.org/pkg/standalone}{\textsf{standalone}}
provide commands to include only the document body of
a child file thus allowing both files to be compiled individually.
\item
The packages \href{http://ctan.org/pkg/subdocs}{\textsf{subdocs}}
and \href{http://ctan.org/pkg/subfiles}{\textsf{subfiles}}
provide structures in which the main and child documents can be
encapsulated and allowing them to be compiled individually.
The inclusion mechanism is different from the conventional |\include|.
\item
The package \href{http://ctan.org/pkg/combine}{\textsf{combine}}
is an elaborate solution to combine several documents into one.
\end{itemize}
%
See also the CTAN topic \href{http://ctan.org/topic/subdocs}{\textsf{subdocs}}
for further related packages.
The present package differs from the above solutions in that
a document structure constructed with the conventional |\include| mechanism
just needs two extra commands at the top of every file
such that all constituent files can be compiled individually.

%%%%%%%%%%%%%%%%%%%%%%%%%%%%%%%%%%%%%%%%%%%%%%%%%%%%%%%%%%%%%%%%%%%%%%%%%%%%%%%%
%\subsection{Feature Suggestions}
%
%The following is a list of features which may be useful for future
%versions of this package:
%%
%\begin{itemize}
%\item
%\ldots
%\end{itemize}

%%%%%%%%%%%%%%%%%%%%%%%%%%%%%%%%%%%%%%%%%%%%%%%%%%%%%%%%%%%%%%%%%%%%%%%%%%%%%%%%
\subsection{Revision History}

%%%%%%%%%%%%%%%%%%%%%%%%%%%%%%%%%%%%%%%%
\paragraph{v2.0:} 2018/12/30

\begin{itemize}
\item
immediate forward processing
\item
added |\childdocby| mechanism
\item
manual restructured
\end{itemize}

%%%%%%%%%%%%%%%%%%%%%%%%%%%%%%%%%%%%%%%%
\paragraph{v1.6:} 2018/01/17

\begin{itemize}
\item
application for development of include files
\item
corrections to manual
\end{itemize}

%%%%%%%%%%%%%%%%%%%%%%%%%%%%%%%%%%%%%%%%
\paragraph{v1.5:} 2017/05/21

\begin{itemize}
\item
more complete structuring introduced
\item
|\childdocof| introduced
\item
|\childdoc| renamed to |\childdocmain|
\item
|\childredirect| renamed to |\childdocforward| and |\childdocforwardprefix|
and functionality expanded
\end{itemize}

%%%%%%%%%%%%%%%%%%%%%%%%%%%%%%%%%%%%%%%%
\paragraph{v1.0:} 2017/04/27

\begin{itemize}
\item
manual and install package
\item
first version published on CTAN
\end{itemize}

%%%%%%%%%%%%%%%%%%%%%%%%%%%%%%%%%%%%%%%%
\paragraph{v0.6:} 2017/04/26

\begin{itemize}
\item
redirection mechanism added
\end{itemize}

%%%%%%%%%%%%%%%%%%%%%%%%%%%%%%%%%%%%%%%%
\paragraph{v0.5:} 2017/04/26

\begin{itemize}
\item
functionality in definition file
\end{itemize}


%%%%%%%%%%%%%%%%%%%%%%%%%%%%%%%%%%%%%%%%%%%%%%%%%%%%%%%%%%%%%%%%%%%%%%%%%%%%%%%%
%%%%%%%%%%%%%%%%%%%%%%%%%%%%%%%%%%%%%%%%%%%%%%%%%%%%%%%%%%%%%%%%%%%%%%%%%%%%%%%%
%%%%%%%%%%%%%%%%%%%%%%%%%%%%%%%%%%%%%%%%%%%%%%%%%%%%%%%%%%%%%%%%%%%%%%%%%%%%%%%%
\appendix

\settowidth\MacroIndent{\rmfamily\scriptsize 000\ }

 \DocInput{childdoc.dtx}

\end{document}
%</driver>
% \fi
%
% %%%%%%%%%%%%%%%%%%%%%%%%%%%%%%%%%%%%%%%%%%%%%%%%%%%%%%%%%%%%%%%%%%%%%%%%%%%%%%
% %%%%%%%%%%%%%%%%%%%%%%%%%%%%%%%%%%%%%%%%%%%%%%%%%%%%%%%%%%%%%%%%%%%%%%%%%%%%%%
% \section{Sample}
%\iffalse
%<*samplemain>
%\fi
%
% The following presents a sample document
% with two chapters, two parts, a title page,
% a compile flag as well as three forwarding files to set the flag.
% It consists of eight |.tex| files:
% \begin{center}
% \begin{tabular}{ll}
% |cdocsamp.tex|&main file\\
% |cdocsch1.tex|&include file for chapter 1\\
% |cdocsch2.tex|&include file for chapter 2\\
% |cdocspt3.tex|&include file for part 3\\
% |cdocspt4.tex|&include file for part 4\\
% |cdocsdrf.tex|&forwarding file for main file in draft mode\\
% |cdocsfi1.tex|&forwarding file for final version of chapter 1\\
% |cdocsfi2.tex|&forwarding file for final version of chapter 2\\
% \end{tabular}
% \end{center}
% Each of the eight files can be compiled directly by the \LaTeX{} compiler.
%
% %%%%%%%%%%%%%%%%%%%%%%%%%%%%%%%%%%%%%%
% \paragraph{Main File.}
%
% The main file is called |cdocsamp.tex|.
%
% Load the \textsf{childdoc} definitions and
% declare the filename for the main document:
%    \begin{macrocode}
\input{childdoc.def}
\childdocmain{}
%    \end{macrocode}

% Optional override for |\version| flag:
%    \begin{macrocode}
%%\ifchilddoc\else\providecommand{\version}{draft}\fi
%    \end{macrocode}

% Define the default values for the |\version| flag
% (|final| for the main file and |draft| for childs):
%    \begin{macrocode}
\ifchilddoc
\providecommand{\version}{draft}
\else
\providecommand{\version}{final}
\fi
%    \end{macrocode}

% Load the standard document class:
%    \begin{macrocode}
\documentclass[12pt]{article}
%    \end{macrocode}

% Start the document body:
%    \begin{macrocode}
\begin{document}
%    \end{macrocode}

% Declare a title page.
% Print title, part of document being processed and version flag:
%    \begin{macrocode}
\addtocounter{page}{-1}
\begin{center}
{\LARGE\bfseries{}childdoc example\par}
\vspace{1cm}
\ifchilddoc
\ifchilddocmanual part\else chapter\fi:
`\childdocname' of `\childdocjob'\par
\else
main document: `\childdocjob'\par
\fi
version: \version\par
\end{center}
\newpage
%    \end{macrocode}

% Manually include selected file,
% otherwise process as usual:
%    \begin{macrocode}
\ifchilddocmanual
\section*{part `\childdocname'}
\input{\childdocname}
\else
%    \end{macrocode}

% Include the two chapters:
%    \begin{macrocode}
\include{cdocsch1}
\include{cdocsch2}
%    \end{macrocode}

% Include the two parts unless only chapters should be displayed:
%    \begin{macrocode}
\ifchilddoc\else
\section{part three}
\input{cdocspt3}
\section{part four}
\input{cdocspt4}
\fi
%    \end{macrocode}

% Process as usual until here:
%    \begin{macrocode}
\fi
%    \end{macrocode}

% End of document body:
%    \begin{macrocode}
\end{document}
%    \end{macrocode}
%\iffalse
%</samplemain>
%\fi
%
% %%%%%%%%%%%%%%%%%%%%%%%%%%%%%%%%%%%%%%
% \paragraph{Chapter Include Files.}
%
% The include files are called |cdocsch1.tex| and |cdocsch2.tex|.
%
%\iffalse
%<*samplechap1|samplechap2>
%\fi

% Optional override for |\version| flag:
%    \begin{macrocode}
%%\providecommand{\version}{final}
%    \end{macrocode}

% Include the main document:
%    \begin{macrocode}
\input{childdoc.def}
\childdocof{cdocsamp}
%    \end{macrocode}

%\iffalse
%</samplechap1|samplechap2>
%\fi
%
%\iffalse
%<*samplechap1>
%\fi
% Some text for chapter 1:
%    \begin{macrocode}
\section{one}
some text in chapter one
%    \end{macrocode}

%\iffalse
%</samplechap1>
%\fi
% Some text for chapter 2:
%\iffalse
%<*samplechap2>
%\fi
%    \begin{macrocode}
\section{two}
more text in chapter two
%    \end{macrocode}

%\iffalse
%</samplechap2>
%\fi
%
% %%%%%%%%%%%%%%%%%%%%%%%%%%%%%%%%%%%%%%
% \paragraph{Part Include Files.}
%
% The include files are called |cdocspt3.tex| and |cdocspt4.tex|.
%
%\iffalse
%<*samplepart3|samplepart4>
%\fi

% Optional override for |\version| flag:
%    \begin{macrocode}
%%\providecommand{\version}{final}
%    \end{macrocode}

% Include the main document:
%    \begin{macrocode}
\input{childdoc.def}
\childdocby{cdocsamp}
%    \end{macrocode}

%\iffalse
%</samplepart3|samplepart4>
%\fi
%
%\iffalse
%<*samplepart3>
%\fi
% Some text for part 3:
%    \begin{macrocode}
some text in part three
%    \end{macrocode}

%\iffalse
%</samplepart3>
%\fi
% Some text for part 4:
%\iffalse
%<*samplepart4>
%\fi
%    \begin{macrocode}
more text in part four
%    \end{macrocode}

%\iffalse
%</samplepart4>
%\fi
%
% %%%%%%%%%%%%%%%%%%%%%%%%%%%%%%%%%%%%%%
% \paragraph{Forwarding for a Complete Draft.}
%
% The following forwarding file |cdocsdrf.tex|
% compiles the main document in draft mode:
%\iffalse
%<*sampledraft>
%\fi
%    \begin{macrocode}
\def\version{draft}
\input{childdoc.def}
\childdocforward{cdocsamp}
%    \end{macrocode}

%\iffalse
%</sampledraft>
%\fi
%
% %%%%%%%%%%%%%%%%%%%%%%%%%%%%%%%%%%%%%%
% \paragraph{Forwarding for Final Version of the Chapters.}
%
% The following forwarding files |cdocsfn1.tex| and |cdocsfn2.tex|
% (with identical content)
% compile the final versions of the child documents
% |cdocsch1.tex| and |cdocsch2.tex|, respectively:
%\iffalse
%<*samplefinal>
%\fi
%    \begin{macrocode}
\def\version{final}
\input{childdoc.def}
\childdocforwardprefix[cdocsamp]{cdocsfn}{cdocsch}
%    \end{macrocode}

%\iffalse
%</samplefinal>
%\fi
%
% %%%%%%%%%%%%%%%%%%%%%%%%%%%%%%%%%%%%%%
% \paragraph{Command Line Processing.}
%
% The following three command lines generate the output files
% |cdocscld|, |cdocscl1| and |cdocscl2|
% which should be identical to
% |cdocsdrf|, |cdocsch1| and |cdocsfn2|, respectively:
% \begin{center}
% \begin{tabular}{l}
% |latex -jobname cdocscld \|\\
% |  "\def\version{draft}\input{childdoc.def}\childdocforward{cdocsamp}"|\\
% |latex -jobname cdocscl1 \|\\
% |  "\input{childdoc.def}\childdocforward[cdocsamp]{cdocsch1}"|\\
% |latex -jobname cdocscl2 \|\\
% |  "\def\version{final}\input{childdoc.def}\childdocforward{cdocsch2}"|
% \end{tabular}
% \end{center}
% Note that the trailing backslash on each first line
% merely continues the input to the second line
% (for convenient cut ant paste).
% Furthermore, the command |latex| can be replaced by any
% of its alternative versions such as |pdflatex|.
%
% %%%%%%%%%%%%%%%%%%%%%%%%%%%%%%%%%%%%%%%%%%%%%%%%%%%%%%%%%%%%%%%%%%%%%%%%%%%%%%
% %%%%%%%%%%%%%%%%%%%%%%%%%%%%%%%%%%%%%%%%%%%%%%%%%%%%%%%%%%%%%%%%%%%%%%%%%%%%%%
% \section{Implementation}
%\iffalse
%<*package>
%\fi
%
% This section describes the definitions file |childdoc.def|.

% The definitions cannot be loaded using |\usepackage| or |\RequirePackage|
% which has a mechanism to prevent loading a style file more than once.
% When loading the definitions by means of |\input|
% multiple instances have to be prevented manually:
%\iffalse
%This code needs to be before the `\ProvidesFile' directive
%which is defined at the beginning of this file.
%Therefore it is also placed there and commented out here.
%</package>
%<*discard>
%\fi
%    \begin{macrocode}
\ifdefined\childdocmain\endinput\fi
%    \end{macrocode}
%\iffalse
%</discard>
%<*package>
%\fi
%
% \macro{\ifchilddoc}
% \macro{\ifchilddocmanual}
% The conditional |\ifchilddoc| tells whether a
% child (true) or main (false) document is being compiled.
% The conditional |\ifchilddocmanual| tells whether
% the |\includeonly| mechanism is used (false) or
% the selection of child files must be performed manually (true).
% The definitions initialise to false:
%    \begin{macrocode}
\newif\ifchilddoc
\newif\ifchilddocmanual
%    \end{macrocode}

% \macro{\childdocname}
% \macro{\childdocjob}
% The macro |\childdocname| stores the name of the main document
% to be compiled. The macro |\childdocjob| stores the name of
% the document on which the \LaTeX{} compiler was originally invoked.
% The content of |\jobname| cannot be compared
% to filenames specified in the source due to different catcodes.
% The following code rescans |\jobname|, stores the result
% in |\childdocname| and saves a copy in |\childdocjob|:
%    \begin{macrocode}
\edef\childdocname{\scantokens\expandafter{\jobname\noexpand}}
\let\childdocjob\childdocname
%    \end{macrocode}

% \macro{\childdocdisable}
% The macro |\childdocdisable| prevents the main file
% from being processed more than once.
% At this stage, the main document command |\childdocmain|
% is assumed to be called once again where it should do nothing.
% Any subsequent call to it should prevent
% a secondary processing of the main document
% It overwrites the forwarding commands
% |\childdocof| and |\childdocforward|
% with empty macros to prevent further inclusions of the main document:
%    \begin{macrocode}
\newcommand{\childdocdisable}
{
  \renewcommand{\childdocmain}[1]{\renewcommand{\childdocmain}[1]{\endinput}}
  \renewcommand{\childdocof}[1]{}
  \renewcommand{\childdocby}[2][]{}
  \renewcommand{\childdocforward}[2][]{}
  \renewcommand{\childdocdisable}{}
}
%    \end{macrocode}

% \macro{\childdocmain}
% The macro |\childdocmain| is to be called at the top of the main file
% with nothing or the main filename (without extension) as argument.
% First, it breaks loops.
% If the argument is not empty and does not match |\childdocname|
% (which is set by the first inclusion of |childdoc.def|),
% |\ifchilddoc| is set to true, |\includeonly| is applied to the child file
% and |\jobname| is set to the main file
% (for proper handling of |.aux| files):
%    \begin{macrocode}
\newcommand{\childdocmain}[1]
{
  \childdocdisable\childdocmain{}
  \if?#1?\else
    \begingroup
      \def\childdoctmp{#1}
      \ifx\childdoctmp\childdocname
        \def\childdoctmp{}
      \else
        \def\childdoctmp
        {
          \childdoctrue
          \includeonly{\childdocname}
          \def\childdocjob{#1}
          \def\jobname{#1}
        }
      \fi
      \expandafter
    \endgroup
    \childdoctmp
  \fi
}
%    \end{macrocode}

% \macro{\childdocof}
% The command |\childdocof| redirects
% compilation to the main file |#1|.
%    \begin{macrocode}
\newcommand{\childdocof}[1]
{
  \childdocdisable
  \childdoctrue
  \includeonly{\childdocname}
  \def\jobname{#1}
  \def\childdocjob{#1}
  \input{#1}
}
%    \end{macrocode}

% \macro{\childdocby}
% The command |\childdocby| ....
%    \begin{macrocode}
\newcommand{\childdocby}[2][]
{
  \childdocdisable
  \childdoctrue
  \childdocmanualtrue
  \if?#1?\else
    \def\jobname{#2}
  \fi
  \def\childdocjob{#2}
  \input{#2}
  \endinput
}
%    \end{macrocode}

% \macro{\childdocforward}
% The command |\childdocforward| redirects
% compilation to the main file or
% (if the optional argument is given) a child file.
% Parameters are set as if the main file
% or a child file starting with |\childdocof| was compiled.
% Then compilation is handed over to the main file:
%    \begin{macrocode}
\newcommand{\childdocforward}[2][]
{
  \begingroup
    \if?#1?
      \def\childdoctmp
      {
        \def\childdocname{#2}
        \def\childdocjob{#2}
        \def\jobname{#2}
        \input{#2}
        \endinput
      }
    \else
      \def\childdoctmp
      {
        \childdocdisable
        \def\childdocname{#2}
        \childdoctrue
        \includeonly{#2}
        \def\childdocjob{#1}
        \def\jobname{#1}
        \input{#1}
        \endinput
      }
    \fi
    \expandafter
  \endgroup
  \childdoctmp
}
%    \end{macrocode}

% \macro{\childdocforwardprefix}
% The command |\childdocforwardprefix| redirects
% compilation to the main or a child file by means of a pattern.
% The prefix |#1| in the current filename is replaced by |#2|
% and the suffix of the current filename is kept
% (it is assumed that the filename does not contain the substring `|~~~|'
% which is used as a delimiter).
% Compilation is handed over to the new file by |\childdocforward|:
%    \begin{macrocode}
\newcommand{\childdocforwardprefix}[3][]
{
  \begingroup
    \def\childdocextract #2##1~~~{\def\childdoctmp{\childdocforward[#1]{#3##1}}}
    \expandafter\childdocextract\childdocname~~~
    \expandafter
  \endgroup
  \childdoctmp
}
%    \end{macrocode}

% \macro{\childdoc}
% The deprecated macro |\childdoc| is a legacy version of |\childdocmain|:
%    \begin{macrocode}
\newcommand{\childdoc}{\childdocmain}
%    \end{macrocode}

% \macro{\childdocredirect}
% The deprecated macro |\childdocredirect| is a legacy version
% of |\childdocforward| and |\childdocforwardprefix|:
%    \begin{macrocode}
\newcommand{\childdocredirect}[2][]
{
  \begingroup
    \if?#1?
      \def\childdoctmp{\childdocforward{#2}}
    \else
      \def\childdoctmp{\childdocforwardprefix{#1}{#2}}
    \fi
    \expandafter
  \endgroup
  \childdoctmp
}
%    \end{macrocode}

%\iffalse
%</package>
%\fi
%
\endinput

\childdocforwardprefix[cdocsamp]{cdocsfn}{cdocsch}
%    \end{macrocode}

%\iffalse
%</samplefinal>
%\fi
%
% %%%%%%%%%%%%%%%%%%%%%%%%%%%%%%%%%%%%%%
% \paragraph{Command Line Processing.}
%
% The following three command lines generate the output files
% |cdocscld|, |cdocscl1| and |cdocscl2|
% which should be identical to
% |cdocsdrf|, |cdocsch1| and |cdocsfn2|, respectively:
% \begin{center}
% \begin{tabular}{l}
% |latex -jobname cdocscld \|\\
% |  "\def\version{draft}% \iffalse
%
% childdoc.dtx Copyright (C) 2017-2018 Niklas Beisert
%
% This work may be distributed and/or modified under the
% conditions of the LaTeX Project Public License, either version 1.3
% of this license or (at your option) any later version.
% The latest version of this license is in
%   http://www.latex-project.org/lppl.txt
% and version 1.3 or later is part of all distributions of LaTeX
% version 2005/12/01 or later.
%
% This work has the LPPL maintenance status `maintained'.
%
% The Current Maintainer of this work is Niklas Beisert.
%
% This work consists of the files childdoc.dtx and childdoc.ins
% and the derived files childdoc.def and cdocsamp.tex with
% cdocsch1.tex, cdocsch2.tex, cdocsdrf.tex, cdocsfn1.tex, cdocsfn2.tex.
%
%<package>\ifdefined\childdocmain\endinput\fi
%<package>\ProvidesFile{childdoc.def}[2018/12/30 v2.0 child document driver]
%<samplemain>\ProvidesFile{cdocsamp.tex}[2018/12/30 v2.0 sample for childdoc]
%<*driver>
%\ProvidesFile{childdoc.drv}[2018/12/30 v2.0 childdoc reference manual file]
\PassOptionsToClass{10pt,a4paper}{article}
\documentclass{ltxdoc}

\usepackage[margin=35mm]{geometry}
\usepackage{hyperref}
\usepackage{hyperxmp}
\usepackage[usenames]{color}

\hypersetup{colorlinks=true}
\hypersetup{pdfstartview=FitH}
\hypersetup{pdfpagemode=UseNone}
\hypersetup{pdfsource={}}
\hypersetup{pdflang={en-UK}}
\hypersetup{pdfcopyright={Copyright 2017-2018 Niklas Beisert.
  This work may be distributed and/or modified under the
  conditions of the LaTeX Project Public License, either version 1.3
  of this license or (at your option) any later version.}}
\hypersetup{pdflicenseurl={http://www.latex-project.org/lppl.txt}}
\hypersetup{pdfcontactaddress={ETH Zurich, ITP, HIT K,
  Wolfgang-Pauli-Strasse 27}}
\hypersetup{pdfcontactpostcode={8093}}
\hypersetup{pdfcontactcity={Zurich}}
\hypersetup{pdfcontactcountry={Switzerland}}
\hypersetup{pdfcontactemail={nbeisert@itp.phys.ethz.ch}}
\hypersetup{pdfcontacturl={http://people.phys.ethz.ch/\xmptilde nbeisert/}}

\newcommand{\secref}[1]{\hyperref[#1]{section \ref*{#1}}}

\parskip1ex
\parindent0pt
\let\olditemize\itemize
\def\itemize{\olditemize\parskip0pt}

\begin{document}

\title{The \textsf{childdoc} Package}
\hypersetup{pdftitle={The childdoc Package}}
\author{Niklas Beisert\\[2ex]
  Institut f\"ur Theoretische Physik\\
  Eidgen\"ossische Technische Hochschule Z\"urich\\
  Wolfgang-Pauli-Strasse 27, 8093 Z\"urich, Switzerland\\[1ex]
  \href{mailto:nbeisert@itp.phys.ethz.ch}
  {\texttt{nbeisert@itp.phys.ethz.ch}}}
\hypersetup{pdfauthor={Niklas Beisert}}
\hypersetup{pdfsubject={Manual for the LaTeX2e Package childdoc}}
\date{30 December 2018, \textsf{v2.0}}
\maketitle

\begin{abstract}\noindent
\textsf{childdoc} is a \LaTeXe{} package
that enables the direct compilation
of document sections included by |\include|
to individual files.
\end{abstract}

\begingroup
\parskip0ex
\tableofcontents
\endgroup

%%%%%%%%%%%%%%%%%%%%%%%%%%%%%%%%%%%%%%%%%%%%%%%%%%%%%%%%%%%%%%%%%%%%%%%%%%%%%%%%
%%%%%%%%%%%%%%%%%%%%%%%%%%%%%%%%%%%%%%%%%%%%%%%%%%%%%%%%%%%%%%%%%%%%%%%%%%%%%%%%
\section{Introduction}

\LaTeX{} provides a mechanism to structure a large document (such as a book)
into a main file and several child files (containing the chapters)
using the |\include| command.
This mechanism is beneficial for documents
which span hundreds of pages in order to
make the source file(s) more manageable.
Moreover, compilation can be restricted to
selected child files by means of the |\includeonly| command.
The latter feature can be used to reduce the compilation time while editing
(this was significantly more useful in the earlier days of \LaTeX{})
or to generate a smaller document which is easier to navigate.
Another application of |\includeonly| is to generate
documents consisting of selected parts of the complete document.

However, there are a few drawbacks of the plain |\include| mechanism:
\begin{itemize}
\item
The child files cannot be compiled on their own,
they can only be compiled via the main file.
A naive editing environment
(such as a text editor with an option
to have the current file processed by \LaTeX)
may require one to switch to the main file before compiling;
attempting to compile the child file produces errors.
\item
The main file must be modified (each time)
to adjust the |\includeonly| command
to the present needs. This easily leaves the main file in a messy state.
\item
The generated document will always carry the filename
of the main document. This is inconvenient if
several child files are to be compiled and
to be kept for distribution.
\end{itemize}

The present package provides a simple interface
to make child files individually compilable by \LaTeX{}.
Compiling a child file then has the same effect as compiling
the main file with an |\includeonly| command
to select the appropriate child.
Moreover the generated document will carry the name of the child
rather than the main file.
This resolves all three above issues.

This feature is meant to make the editing of books,
thesis documents and lecture notes somewhat more convenient.
However, the package can also be used efficiently for
composing a series of documents (such as exercise sheets)
which are typically distributed individually.
It then assists the author in generating the individual documents
(potentially in different versions)
as well as a document containing the collected series.
Another application is in developing style files
or other kinds of included material
where compilation of the style file could redirect
to a sample or test file.

%%%%%%%%%%%%%%%%%%%%%%%%%%%%%%%%%%%%%%%%%%%%%%%%%%%%%%%%%%%%%%%%%%%%%%%%%%%%%%%%
%%%%%%%%%%%%%%%%%%%%%%%%%%%%%%%%%%%%%%%%%%%%%%%%%%%%%%%%%%%%%%%%%%%%%%%%%%%%%%%%
\section{Usage}

First of all, the package \textsf{childdoc} is \emph{not} a standard
\LaTeXe{} |.sty| style file! Therefore it needs to be invoked in
a non-standard way.

%%%%%%%%%%%%%%%%%%%%%%%%%%%%%%%%%%%%%%%%%%%%%%%%%%%%%%%%%%%%%%%%%%%%%%%%%%%%%%%%
\subsection{Included Files}
\label{sec:include}

%%%%%%%%%%%%%%%%%%%%%%%%%%%%%%%%%%%%%%%%
\DescribeMacro{\childdocmain}
To use the package, add the commands
\begin{center}
\begin{tabular}{l}
|\input{childdoc.def}|\\
|\childdocmain{}|\\
\end{tabular}
\end{center}
at the very top of the main \LaTeX{} file,
in particular \emph{before} the |\documentclass| statement!
The argument of |\childdocmain| should be left empty
(but it must be present).

%%%%%%%%%%%%%%%%%%%%%%%%%%%%%%%%%%%%%%%%
\DescribeMacro{\childdocof}
Furthermore, add the commands
\begin{center}
\begin{tabular}{l}
|\input{childdoc.def}|\\
|\childdocof{|\textit{main}|}|\\
\end{tabular}
\end{center}
at the top of every child file \textit{child}
which is included by |\include{|\textit{child}|}|
from within the main file
(or at least for those files to be compiled individually).
The argument \textit{main} must be the filename of the main file.

There are a couple of
considerations in setting up the main and child documents:

%%%%%%%%%%%%%%%%%%%%%%%%%%%%%%%%%%%%%%%%
\paragraph{Restrictions.}

Please note the following restrictions:
\begin{itemize}
\item
|\childdocmain| must be called with one argument \textit{main}
to ensure compatibility with earlier version of the package.
It must either be empty (|\childdocmain{}|)
or precisely match the filename of the main file in which it is specified.
See \secref{sec:detection} for further information.
\item
The filename \textit{main} must be specified without the |.tex| extension.
\item
The filename \textit{main} is case sensitive
(even in case-insensitive file systems)
due to internal string comparison.
\item
The argument \textit{main} should be fully expanded, it cannot be a macro.
\item
Subdirectories and special characters should be avoided in filenames.
\item
The command |\childdocmain{|\textit{main}|}| must be followed by a whitespace.
It should not be followed immediately by another command
or by a comment mark `|%|'.
This is because the \TeX{} parser reads the token immediately following
the argument of |\childdocmain| and puts it
at the beginning of every child section;
however, a white\-space is ignored.
\end{itemize}

%%%%%%%%%%%%%%%%%%%%%%%%%%%%%%%%%%%%%%%%
\paragraph{Content of Main File.}

It is advisable to place all content in the child files included by |\include|.
Any output contained in the main file will appear in all child documents
unless suppressed manually;
it cannot be suppressed automatically by the |\includeonly| directive
and thus should normally be avoided.
A method to include some content in the main file
by means of conditional processing is described in \secref{sec:conditional}.

%%%%%%%%%%%%%%%%%%%%%%%%%%%%%%%%%%%%%%%%
\paragraph{Page Numbering.}

When only a part of the document is compiled,
the appropriate numbering of pages
(as well as other status parameters)
is determined from the |.aux| files.
The latter contain information from previous passes.
However this information needs to propagate through
all intermediate child documents.
Therefore the page numbering in child documents may well
be inconsistent until the complete document is compiled at least once.

A useful (if unconventional) way to always ensure a consistent
page numbering is to restart the numbering in each child document
and denote the pages by `\textit{child}|.|\textit{page}'
where \textit{child} represents the chapter/section number of the child file.
This can be achieved by the command
|\numberwithin{page}{|\textit{child}|}|
of the \textsf{amsmath} package
where \textit{child} can be |chapter| or |section|
depending on the chosen structuring.
Alternatively, one can modify the macro |\thepage| appropriately
and reset the counter |page| at the start of each child file.

%%%%%%%%%%%%%%%%%%%%%%%%%%%%%%%%%%%%%%%%%%%%%%%%%%%%%%%%%%%%%%%%%%%%%%%%%%%%%%%%
\subsection{Conditional Processing}
\label{sec:conditional}

The package provides a mechanism to compile different versions
of a document. To customise the versions further some conditional processing
can come in handy to distinguish which version is being compiled.
The package provides two macros to describe the compilation context:

%%%%%%%%%%%%%%%%%%%%%%%%%%%%%%%%%%%%%%%%
\DescribeMacro{\ifchilddoc}
The conditional |\ifchilddoc| distinguishes between the compilation of
child documents and the main document:
%
\begin{center}
|\ifchilddoc |\textit{child-code}| |[|\||else |\textit{main-code}]| \||fi|
\end{center}

%%%%%%%%%%%%%%%%%%%%%%%%%%%%%%%%%%%%%%%%
\DescribeMacro{\childdocname}
\DescribeMacro{\childdocjob}
The macro |\childdocname| contains the filename (without extension)
of the main or child file being processed.
Note that |\childdocjob| will always contain the name of the main file.

%%%%%%%%%%%%%%%%%%%%%%%%%%%%%%%%%%%%%%%%
\paragraph{Title Page.}

Conditional processing can be used to include a title or banner page
in the main document when proper precautions are taken.
Importantly, the code in the main file should ensure that the page counter
(as well as other status parameters which are stored in the |.aux| files)
takes the same value after the conditional processing.
Otherwise the page numbers may take divergent values
depending on which part is compiled.

For example, a title page could be declared by:
%
\begin{center}
\begin{tabular}{l}
|\ifchilddoc\||else|\\
|\addtocounter{page}{-1}|\\
\textit{code for title page}\\
|\newpage|\\
|\||fi|
\end{tabular}
\end{center}
%
A banner page for the child documents can be generated by:
%
\begin{center}
\begin{tabular}{l}
|\ifchilddoc|\\
|\addtocounter{page}{-1}|\\
\textit{code for banner page}\\
|\newpage|\\
|\||fi|
\end{tabular}
\end{center}
%
Here one could write a message such as:
\begin{center}
|This is the part \childdocname{} of \childdocjob{}.|
\end{center}

%%%%%%%%%%%%%%%%%%%%%%%%%%%%%%%%%%%%%%%%%%%%%%%%%%%%%%%%%%%%%%%%%%%%%%%%%%%%%%%%
\subsection{Flags}
\label{sec:flags}

The package makes it easy to generate different versions
of the main or child documents.
To this end compilation flags can be defined
and assigned different default values.
They will be particularly useful in conjunction
with the forwarding mechanism described in \secref{sec:forward}.

For example, it may be useful to have a flag |\version|
which can be set to |draft| or |final|.
The document source will contain some conditional code
depending on the value of |\version|.
Suppose further, the flag should default to |final| for the main file
and to |draft| for child files
which is a natural assignment for editing the document.
This is achieved by placing the following code
in the preamble of the main document
(below the |\childdocmain| directive):
%
\begin{center}
\begin{tabular}{l}
|\ifchilddoc|\\
|\providecommand{\version}{draft}|\\
|\||else|\\
|\providecommand{\version}{final}|\\
|\||fi|
\end{tabular}
\end{center}
%
The definition by |\providecommand| makes sure
that previous definitions are not overwritten.
Further statements |\providecommand{\version}{...}|
can thus be added before the above code to override it.

For the main file, one might add a line
(between |\childdocmain| and the above block)
%
\begin{center}
|%\ifchilddoc\||else\providecommand{\version}{draft}\||fi|
\end{center}
%
which can be uncommented to produce a draft version.
Likewise one can add a line to the very top of a child file
(above the |\childdocof{|\textit{main}|}| directive)
%
\begin{center}
|%\providecommand{\version}{final}|
\end{center}
%
which can be uncommented to produce the final version of this child document.

%%%%%%%%%%%%%%%%%%%%%%%%%%%%%%%%%%%%%%%%%%%%%%%%%%%%%%%%%%%%%%%%%%%%%%%%%%%%%%%%
\subsection{Forwarding}
\label{sec:forward}

Different versions of the main or child documents
using compilation flags as described in \secref{sec:flags}
can be (permanently) stored in different files
for convenient compilation, viewing and distribution.
To this end, the package defines a command
to pass on compilation to a different file:

%%%%%%%%%%%%%%%%%%%%%%%%%%%%%%%%%%%%%%%%
\DescribeMacro{\childdocforward}
The command |\childdocforward| redirects processing to
another source file:
%
\begin{center}
\begin{tabular}{l}
|\input{childdoc.def}|\\
|\childdocforward[|\textit{main}|]{|\textit{dest}|}|\\
\end{tabular}
\end{center}
%
The argument \textit{dest} is the destination file
(without extension).
It should be the main file or one of the child files.
Note that further \textsf{childdoc} directives
such as |\childdocof| and |\childdocforward|
in the indicated file will be processed in this form.
The optional argument \textit{main}
passes on directly to the main file \textit{main}
while pretending to compile the child \textit{dest}.
This form behaves as if \textit{dest}
issues |\childdocof{|\textit{main}|}| right away,
and no further \textsf{childdoc} directives will be processed.

%%%%%%%%%%%%%%%%%%%%%%%%%%%%%%%%%%%%%%%%
\DescribeMacro{\...prefix}
In the alternative form |\childdocforwardprefix|,
%
\begin{center}
\begin{tabular}{l}
|\input{childdoc.def}|\\
|\childdocforwardprefix[|\textit{main}|]{|\textit{prefix}|}{|\textit{dest}|}|
\end{tabular}
\end{center}
%
the destination file is determined by a pattern
depending on the current file:
To make this work, the current file must be called
`{\textit{prefix}\hspace{0.2em}\textit{suffix}}'
with \textit{prefix} matching precisely the argument.
Processing is then passed on to the file
`{\textit{dest}\hspace{0.2em}\textit{suffix}}'.
Surely, the same effect is achieved by
directly specifying the
argument `{\textit{dest}\hspace{0.2em}\textit{suffix}}'
in the first form.
However, that requires to set up a different file
for each child. With the alternative form of the command
all these files can have exactly the same content
which simplifies setting them up and maintaining them.

For example, the following file |draft.tex|
with a compilation flag |\version| as described in \secref{sec:flags}
compiles the main document as a draft:
%
\begin{center}
\begin{tabular}{l}
|\def\version{draft}|\\
|\input{childdoc.def}|\\
|\childdocforward{|\textit{main}|}|
\end{tabular}
\end{center}
%
Likewise, the following files |final|\textit{nn}|.tex|
compile the final version of the child document
|child|\textit{nn}|.tex|:
%
\begin{center}
\begin{tabular}{l}
|\def\version{final}|\\
|\input{childdoc.def}|\\
|\childdocforwardprefix{final}{child}|
\end{tabular}
\end{center}
%

Note that when several versions of a main file and/or of each child file
are to be generated, it may be convenient to set up a |Makefile| or
shell script to automatise the process.

%%%%%%%%%%%%%%%%%%%%%%%%%%%%%%%%%%%%%%%%%%%%%%%%%%%%%%%%%%%%%%%%%%%%%%%%%%%%%%%%
\subsection{Command Line Processing}
\label{sec:commandline}

The effect of redirection files can also be achieved by invoking
the \LaTeX{} compiler with a more elaborate command line.
Most conveniently this should be done as part
of a shell script or a |Makefile|.

When using \textsf{childdoc} in the main file, the following
command lines effectively perform a redirection
(note that depending on the shell being used,
backslashes may have to be doubled: `|\|' $\to$ `|\\|'):
%
\begin{center}
|... -jobname "|\textit{target}|" |\\|"|[\textit{flags}]%
|\input{childdoc.def}\childdocforward[|\textit{main}|]{|\textit{dest}|}"|
\end{center}
%
Here \textit{target} is the name of the output file,
\textit{main} is the name of the main file
and \textit{dest} is the name of the main or child file to be processed
(all filenames without extensions).
The optional argument \textit{main} can be omitted
if \textit{main} matches \textit{dest}.
Optionally, compilation \textit{flags} can be defined via |\def| commands.
This command line makes the \TeX{} engine believe
it is compiling the file \textit{target}
whose content is specified as the latter parameter.
The provided code then forwards the processing to
\textit{main} or \textit{dest} as described in \secref{sec:forward}.

%%%%%%%%%%%%%%%%%%%%%%%%%%%%%%%%%%%%%%%%%%%%%%%%%%%%%%%%%%%%%%%%%%%%%%%%%%%%%%%%
\subsection{Include by Input}
\label{sec:input}

Including child documents by |\include| has some restrictions by design.
Most notably, the content of a child document always occupies
its own set of pages; pages cannot be shared between child documents.
Usually, this behaviour makes perfect sense
because each child document contain an essential part of the document.
However, in some situations it may be desirable to compose
a document from a collection of parts
without having mandatory page breaks between then.
For this case, the package
provides a mechanism to include parts
by |\input| which can also be processed individually.
However, by construction this mechanism
requires manual handling of the content to be output.

%%%%%%%%%%%%%%%%%%%%%%%%%%%%%%%%%%%%%%%%
\DescribeMacro{\ifchilddocmanual}
The main file should be prepared as usual, see \secref{sec:include}.
However, the document body must make a distinction
between processing of an individual part and of the main document, e.g.:
%
\begin{center}
\begin{tabular}{l}
|\ifchilddocmanual|\\
|\input{\childdocname}|\\
|\||else|\\
\textit{document body with }|\input{|\textit{part}|}|\\
|\||fi|
\end{tabular}
\end{center}
%
The conditional |\ifchilddocmanual| is true whenever
a part to be included by |\input| is being compiled,
and the name of the part is stored in |\childdocname|.

%%%%%%%%%%%%%%%%%%%%%%%%%%%%%%%%%%%%%%%%
\DescribeMacro{\childdocby}
Each part to be included by |\input| should start with:
%
\begin{center}
\begin{tabular}{l}
|\input{childdoc.def}|\\
|\childdocby{|\textit{main}|}|\\
\end{tabular}
\end{center}
%
The directive |\childdocby| is similar to |\childdocof|
described in \secref{sec:include},
but the subsequent selection of content must be done manually.
To that end, both |\ifchilddoc| and |\ifchilddocmanual|
will be true upon processing of a part,
and the name of the part is stored in |\childdocname|.
Note that |\jobname| will be set to the filename of the current part
so that each part receives an individual |.aux| file
that does not interfere with the |.aux| file(s) of the main document.
This behaviour can be altered by the alternative form
|\childdocby[*]{|\textit{main}|}| (with a non-empty optional argument)
which uses the |.aux| file of the main document
by setting |\jobname| to \textit{main}.

%%%%%%%%%%%%%%%%%%%%%%%%%%%%%%%%%%%%%%%%%%%%%%%%%%%%%%%%%%%%%%%%%%%%%%%%%%%%%%%%
\subsection{Driver Development}
\label{sec:driver}

The \textsf{childdoc} mechanism can also be use for the development
of definition files such as \LaTeX{} styles or classes.
This case differs from the above setup with multiple parts
included by |\include| in that no |\includeonly| should be invoked.
This can be achieved by starting the include file
(before |\ProvidesPackage|) with:
%
\begin{center}
\begin{tabular}{l}
|\input{childdoc.def}|\\
|\childdocforward{|\textit{main}|}|\\
\end{tabular}
\end{center}
%
or alternatively with:
%
\begin{center}
\begin{tabular}{l}
|\input{childdoc.def}|\\
|\childdocby{|\textit{main}|}|\\
\end{tabular}
\end{center}
%
Both forms have slightly different effects as described above.
The main file is prepared as usual, see \secref{sec:include}.

%%%%%%%%%%%%%%%%%%%%%%%%%%%%%%%%%%%%%%%%%%%%%%%%%%%%%%%%%%%%%%%%%%%%%%%%%%%%%%%%
\subsection{Legacy Detection}
\label{sec:detection}

The directive |\childdocmain| in the main file can detect
whether the complete document or merely a child is to be compiled
even without using the directive |\childdocof|.
This method is deprecated because it is less robust
and there is no compelling reason to use it;
it is merely provided for backward compatibility
and it may be removed in future versions.

If the detection mechanism is to be used,
it is mandatory to correctly specify
the filename of the main file as the argument of |\childdocmain|:
%
\begin{center}
\begin{tabular}{l}
|\input{childdoc.def}|\\
|\childdocmain{|\textit{main}|}|\\
\end{tabular}
\end{center}
%
If |\jobname| does not match the argument \textit{main} of |\childdocmain|,
it is assumed that |\jobname| points to the child file to be compiled.
When using |\childdocmain| with the main file specified as argument,
it suffices to start a child file
with just |\input{|\textit{main}|}|
without loading of the package and using |\childdocof|.
If instead all processing is done
with the appropriate \textsf{childdoc} directives,
the argument of \textit{main} of |\childdocmain| can be empty.

An alternative version of the command line processing described
in \secref{sec:commandline} using the detection mechanism reads:
%
\begin{center}
|... -jobname "|\textit{target}|" "|[\textit{flags}]%
[|\def\jobname{|\textit{dest}|}|]|\input{|\textit{main}|}"|
\end{center}

%%%%%%%%%%%%%%%%%%%%%%%%%%%%%%%%%%%%%%%%%%%%%%%%%%%%%%%%%%%%%%%%%%%%%%%%%%%%%%%%
\subsection{Manual Code}
\label{sec:manual}

In case one cannot be certain whether the definitions file |childdoc.def|
is installed on the target \TeX{} distribution
and one prefers not to ship it,
it is conceivable to paste a few relevant commands into the sources.

To that end, drop all statements |\input{childdoc.def}|
and perform the replacements as outlined below.
Instead of |\childdocmain{|\textit{main}|}| add the following code
to the top of the main file:
%
\begin{center}
\begin{tabular}{l}
|\||ifdefined\childdocname\endinput\||fi\newif\ifchilddoc|\\
|\edef\childdocname{\scantokens\expandafter{\jobname\noexpand}}|\\
|\def\childdocmain{|\textit{main}|}\||ifx\childdocmain\childdocname\||else|\\
|\childdoctrue\includeonly{\childdocname}\let\jobname\childdocmain\||fi|\\
\end{tabular}
\end{center}
%
Instead of |\childdocof{|\textit{main}|}| just include the main file
at the top of each child file:
%
\begin{center}
|\input{|\textit{main}|}|
\end{center}
%
A simple redirection |\childdocforward{|\textit{dest}|}| is achieved by:
%
\begin{center}
|\def\jobname{|\textit{dest}|}\input{\jobname}|
\end{center}
%
The redirection with prefix
|\childdocforwardprefix[|\textit{prefix}|]{|\textit{dest}|}|
is accomplished by:
%
\begin{center}
\begin{tabular}{l}
|{\edef\jobname{\scantokens\expandafter{\jobname\noexpand}}|\\
|\def\redirectjob |\textit{prefix}|#1~~~{\gdef\jobname{|\textit{dest}|#1}}|\\
|\expandafter\redirectjob\jobname~~~}\input{\jobname}|
\end{tabular}
\end{center}

In an alternative approach,
child documents can be compiled by a specific command line
without additional code or specific definitions:
%
\begin{center}
|... -jobname "|\textit{target}|" "|[\textit{flags}]%
|\includeonly{|\textit{dest}|}\input{|\textit{main}|}"|
\end{center}
%

%%%%%%%%%%%%%%%%%%%%%%%%%%%%%%%%%%%%%%%%%%%%%%%%%%%%%%%%%%%%%%%%%%%%%%%%%%%%%%%%
%%%%%%%%%%%%%%%%%%%%%%%%%%%%%%%%%%%%%%%%%%%%%%%%%%%%%%%%%%%%%%%%%%%%%%%%%%%%%%%%
\section{Information}

%%%%%%%%%%%%%%%%%%%%%%%%%%%%%%%%%%%%%%%%%%%%%%%%%%%%%%%%%%%%%%%%%%%%%%%%%%%%%%%%
\subsection{Copyright}

Copyright \copyright{} 2017--2018 Niklas Beisert

This work may be distributed and/or modified under the
conditions of the \LaTeX{} Project Public License, either version 1.3
of this license or (at your option) any later version.
The latest version of this license is in
  \url{http://www.latex-project.org/lppl.txt}
and version 1.3 or later is part of all distributions of \LaTeX{}
version 2005/12/01 or later.

This work has the LPPL maintenance status `maintained'.

The Current Maintainer of this work is Niklas Beisert.

This work consists of the files |README.txt|, |childdoc.ins| and |childdoc.dtx|
as well as the derived files |childdoc.def|, |cdocsamp.tex|
with |cdocsch1.tex|, |cdocsch2.tex|, |cdocspt3.tex|, |cdocspt4.tex|,
|cdocsdrf.tex|, |cdocsfn1.tex|, |cdocsfn2.tex|
as well as |childdoc.pdf|.

%%%%%%%%%%%%%%%%%%%%%%%%%%%%%%%%%%%%%%%%%%%%%%%%%%%%%%%%%%%%%%%%%%%%%%%%%%%%%%%%
\subsection{Files and Installation}

The package consists of the files:
%
\begin{center}
\begin{tabular}{ll}
    |README.txt|   & readme file \\
    |childdoc.ins| & installation file \\
    |childdoc.dtx| & source file \\
    |childdoc.def| & definition file \\
    |cdocsamp.tex| & sample main file \\
    |cdocsch1.tex| & sample include file \\
    |cdocsch2.tex| & sample include file \\
    |cdocspt3.tex| & sample part file \\
    |cdocspt4.tex| & sample part file \\
    |cdocsdrf.tex| & sample redirection file \\
    |cdocsfn1.tex| & sample redirection file \\
    |cdocsfn2.tex| & sample redirection file \\
    |childdoc.pdf| & manual
\end{tabular}
\end{center}
%
The distribution consists of the files
|README.txt|, |childdoc.ins| and |childdoc.dtx|.
%
\begin{itemize}
\item
Run (pdf)\LaTeX{} on |childdoc.dtx|
to compile the manual |childdoc.pdf| (this file).
\item
Run \LaTeX{} on |childdoc.ins| to create the definitions file |childdoc.def|
and the sample |cdocsamp.tex| with include files
|cdocsch1.tex|, |cdocsch2.tex|, |cdocspt3.tex|, |cdocspt4.tex|,
|cdocsdrf.tex|, |cdocsfn1.tex|, |cdocsfn2.tex|.
Then copy the file |childdoc.def| to an appropriate directory of your \LaTeX{}
distribution, e.g.\ \textit{texmf-root}|/tex/latex/childdoc|.
\end{itemize}

%%%%%%%%%%%%%%%%%%%%%%%%%%%%%%%%%%%%%%%%%%%%%%%%%%%%%%%%%%%%%%%%%%%%%%%%%%%%%%%%
\subsection{Related CTAN Packages}

There are several other packages which offer a similar functionality:
%
\begin{itemize}
\item
The packages
\href{http://ctan.org/pkg/docmute}{\textsf{docmute}},
\href{http://ctan.org/pkg/includex}{\textsf{includex}} and
\href{http://ctan.org/pkg/standalone}{\textsf{standalone}}
provide commands to include only the document body of
a child file thus allowing both files to be compiled individually.
\item
The packages \href{http://ctan.org/pkg/subdocs}{\textsf{subdocs}}
and \href{http://ctan.org/pkg/subfiles}{\textsf{subfiles}}
provide structures in which the main and child documents can be
encapsulated and allowing them to be compiled individually.
The inclusion mechanism is different from the conventional |\include|.
\item
The package \href{http://ctan.org/pkg/combine}{\textsf{combine}}
is an elaborate solution to combine several documents into one.
\end{itemize}
%
See also the CTAN topic \href{http://ctan.org/topic/subdocs}{\textsf{subdocs}}
for further related packages.
The present package differs from the above solutions in that
a document structure constructed with the conventional |\include| mechanism
just needs two extra commands at the top of every file
such that all constituent files can be compiled individually.

%%%%%%%%%%%%%%%%%%%%%%%%%%%%%%%%%%%%%%%%%%%%%%%%%%%%%%%%%%%%%%%%%%%%%%%%%%%%%%%%
%\subsection{Feature Suggestions}
%
%The following is a list of features which may be useful for future
%versions of this package:
%%
%\begin{itemize}
%\item
%\ldots
%\end{itemize}

%%%%%%%%%%%%%%%%%%%%%%%%%%%%%%%%%%%%%%%%%%%%%%%%%%%%%%%%%%%%%%%%%%%%%%%%%%%%%%%%
\subsection{Revision History}

%%%%%%%%%%%%%%%%%%%%%%%%%%%%%%%%%%%%%%%%
\paragraph{v2.0:} 2018/12/30

\begin{itemize}
\item
immediate forward processing
\item
added |\childdocby| mechanism
\item
manual restructured
\end{itemize}

%%%%%%%%%%%%%%%%%%%%%%%%%%%%%%%%%%%%%%%%
\paragraph{v1.6:} 2018/01/17

\begin{itemize}
\item
application for development of include files
\item
corrections to manual
\end{itemize}

%%%%%%%%%%%%%%%%%%%%%%%%%%%%%%%%%%%%%%%%
\paragraph{v1.5:} 2017/05/21

\begin{itemize}
\item
more complete structuring introduced
\item
|\childdocof| introduced
\item
|\childdoc| renamed to |\childdocmain|
\item
|\childredirect| renamed to |\childdocforward| and |\childdocforwardprefix|
and functionality expanded
\end{itemize}

%%%%%%%%%%%%%%%%%%%%%%%%%%%%%%%%%%%%%%%%
\paragraph{v1.0:} 2017/04/27

\begin{itemize}
\item
manual and install package
\item
first version published on CTAN
\end{itemize}

%%%%%%%%%%%%%%%%%%%%%%%%%%%%%%%%%%%%%%%%
\paragraph{v0.6:} 2017/04/26

\begin{itemize}
\item
redirection mechanism added
\end{itemize}

%%%%%%%%%%%%%%%%%%%%%%%%%%%%%%%%%%%%%%%%
\paragraph{v0.5:} 2017/04/26

\begin{itemize}
\item
functionality in definition file
\end{itemize}


%%%%%%%%%%%%%%%%%%%%%%%%%%%%%%%%%%%%%%%%%%%%%%%%%%%%%%%%%%%%%%%%%%%%%%%%%%%%%%%%
%%%%%%%%%%%%%%%%%%%%%%%%%%%%%%%%%%%%%%%%%%%%%%%%%%%%%%%%%%%%%%%%%%%%%%%%%%%%%%%%
%%%%%%%%%%%%%%%%%%%%%%%%%%%%%%%%%%%%%%%%%%%%%%%%%%%%%%%%%%%%%%%%%%%%%%%%%%%%%%%%
\appendix

\settowidth\MacroIndent{\rmfamily\scriptsize 000\ }

 \DocInput{childdoc.dtx}

\end{document}
%</driver>
% \fi
%
% %%%%%%%%%%%%%%%%%%%%%%%%%%%%%%%%%%%%%%%%%%%%%%%%%%%%%%%%%%%%%%%%%%%%%%%%%%%%%%
% %%%%%%%%%%%%%%%%%%%%%%%%%%%%%%%%%%%%%%%%%%%%%%%%%%%%%%%%%%%%%%%%%%%%%%%%%%%%%%
% \section{Sample}
%\iffalse
%<*samplemain>
%\fi
%
% The following presents a sample document
% with two chapters, two parts, a title page,
% a compile flag as well as three forwarding files to set the flag.
% It consists of eight |.tex| files:
% \begin{center}
% \begin{tabular}{ll}
% |cdocsamp.tex|&main file\\
% |cdocsch1.tex|&include file for chapter 1\\
% |cdocsch2.tex|&include file for chapter 2\\
% |cdocspt3.tex|&include file for part 3\\
% |cdocspt4.tex|&include file for part 4\\
% |cdocsdrf.tex|&forwarding file for main file in draft mode\\
% |cdocsfi1.tex|&forwarding file for final version of chapter 1\\
% |cdocsfi2.tex|&forwarding file for final version of chapter 2\\
% \end{tabular}
% \end{center}
% Each of the eight files can be compiled directly by the \LaTeX{} compiler.
%
% %%%%%%%%%%%%%%%%%%%%%%%%%%%%%%%%%%%%%%
% \paragraph{Main File.}
%
% The main file is called |cdocsamp.tex|.
%
% Load the \textsf{childdoc} definitions and
% declare the filename for the main document:
%    \begin{macrocode}
\input{childdoc.def}
\childdocmain{}
%    \end{macrocode}

% Optional override for |\version| flag:
%    \begin{macrocode}
%%\ifchilddoc\else\providecommand{\version}{draft}\fi
%    \end{macrocode}

% Define the default values for the |\version| flag
% (|final| for the main file and |draft| for childs):
%    \begin{macrocode}
\ifchilddoc
\providecommand{\version}{draft}
\else
\providecommand{\version}{final}
\fi
%    \end{macrocode}

% Load the standard document class:
%    \begin{macrocode}
\documentclass[12pt]{article}
%    \end{macrocode}

% Start the document body:
%    \begin{macrocode}
\begin{document}
%    \end{macrocode}

% Declare a title page.
% Print title, part of document being processed and version flag:
%    \begin{macrocode}
\addtocounter{page}{-1}
\begin{center}
{\LARGE\bfseries{}childdoc example\par}
\vspace{1cm}
\ifchilddoc
\ifchilddocmanual part\else chapter\fi:
`\childdocname' of `\childdocjob'\par
\else
main document: `\childdocjob'\par
\fi
version: \version\par
\end{center}
\newpage
%    \end{macrocode}

% Manually include selected file,
% otherwise process as usual:
%    \begin{macrocode}
\ifchilddocmanual
\section*{part `\childdocname'}
\input{\childdocname}
\else
%    \end{macrocode}

% Include the two chapters:
%    \begin{macrocode}
\include{cdocsch1}
\include{cdocsch2}
%    \end{macrocode}

% Include the two parts unless only chapters should be displayed:
%    \begin{macrocode}
\ifchilddoc\else
\section{part three}
\input{cdocspt3}
\section{part four}
\input{cdocspt4}
\fi
%    \end{macrocode}

% Process as usual until here:
%    \begin{macrocode}
\fi
%    \end{macrocode}

% End of document body:
%    \begin{macrocode}
\end{document}
%    \end{macrocode}
%\iffalse
%</samplemain>
%\fi
%
% %%%%%%%%%%%%%%%%%%%%%%%%%%%%%%%%%%%%%%
% \paragraph{Chapter Include Files.}
%
% The include files are called |cdocsch1.tex| and |cdocsch2.tex|.
%
%\iffalse
%<*samplechap1|samplechap2>
%\fi

% Optional override for |\version| flag:
%    \begin{macrocode}
%%\providecommand{\version}{final}
%    \end{macrocode}

% Include the main document:
%    \begin{macrocode}
\input{childdoc.def}
\childdocof{cdocsamp}
%    \end{macrocode}

%\iffalse
%</samplechap1|samplechap2>
%\fi
%
%\iffalse
%<*samplechap1>
%\fi
% Some text for chapter 1:
%    \begin{macrocode}
\section{one}
some text in chapter one
%    \end{macrocode}

%\iffalse
%</samplechap1>
%\fi
% Some text for chapter 2:
%\iffalse
%<*samplechap2>
%\fi
%    \begin{macrocode}
\section{two}
more text in chapter two
%    \end{macrocode}

%\iffalse
%</samplechap2>
%\fi
%
% %%%%%%%%%%%%%%%%%%%%%%%%%%%%%%%%%%%%%%
% \paragraph{Part Include Files.}
%
% The include files are called |cdocspt3.tex| and |cdocspt4.tex|.
%
%\iffalse
%<*samplepart3|samplepart4>
%\fi

% Optional override for |\version| flag:
%    \begin{macrocode}
%%\providecommand{\version}{final}
%    \end{macrocode}

% Include the main document:
%    \begin{macrocode}
\input{childdoc.def}
\childdocby{cdocsamp}
%    \end{macrocode}

%\iffalse
%</samplepart3|samplepart4>
%\fi
%
%\iffalse
%<*samplepart3>
%\fi
% Some text for part 3:
%    \begin{macrocode}
some text in part three
%    \end{macrocode}

%\iffalse
%</samplepart3>
%\fi
% Some text for part 4:
%\iffalse
%<*samplepart4>
%\fi
%    \begin{macrocode}
more text in part four
%    \end{macrocode}

%\iffalse
%</samplepart4>
%\fi
%
% %%%%%%%%%%%%%%%%%%%%%%%%%%%%%%%%%%%%%%
% \paragraph{Forwarding for a Complete Draft.}
%
% The following forwarding file |cdocsdrf.tex|
% compiles the main document in draft mode:
%\iffalse
%<*sampledraft>
%\fi
%    \begin{macrocode}
\def\version{draft}
\input{childdoc.def}
\childdocforward{cdocsamp}
%    \end{macrocode}

%\iffalse
%</sampledraft>
%\fi
%
% %%%%%%%%%%%%%%%%%%%%%%%%%%%%%%%%%%%%%%
% \paragraph{Forwarding for Final Version of the Chapters.}
%
% The following forwarding files |cdocsfn1.tex| and |cdocsfn2.tex|
% (with identical content)
% compile the final versions of the child documents
% |cdocsch1.tex| and |cdocsch2.tex|, respectively:
%\iffalse
%<*samplefinal>
%\fi
%    \begin{macrocode}
\def\version{final}
\input{childdoc.def}
\childdocforwardprefix[cdocsamp]{cdocsfn}{cdocsch}
%    \end{macrocode}

%\iffalse
%</samplefinal>
%\fi
%
% %%%%%%%%%%%%%%%%%%%%%%%%%%%%%%%%%%%%%%
% \paragraph{Command Line Processing.}
%
% The following three command lines generate the output files
% |cdocscld|, |cdocscl1| and |cdocscl2|
% which should be identical to
% |cdocsdrf|, |cdocsch1| and |cdocsfn2|, respectively:
% \begin{center}
% \begin{tabular}{l}
% |latex -jobname cdocscld \|\\
% |  "\def\version{draft}\input{childdoc.def}\childdocforward{cdocsamp}"|\\
% |latex -jobname cdocscl1 \|\\
% |  "\input{childdoc.def}\childdocforward[cdocsamp]{cdocsch1}"|\\
% |latex -jobname cdocscl2 \|\\
% |  "\def\version{final}\input{childdoc.def}\childdocforward{cdocsch2}"|
% \end{tabular}
% \end{center}
% Note that the trailing backslash on each first line
% merely continues the input to the second line
% (for convenient cut ant paste).
% Furthermore, the command |latex| can be replaced by any
% of its alternative versions such as |pdflatex|.
%
% %%%%%%%%%%%%%%%%%%%%%%%%%%%%%%%%%%%%%%%%%%%%%%%%%%%%%%%%%%%%%%%%%%%%%%%%%%%%%%
% %%%%%%%%%%%%%%%%%%%%%%%%%%%%%%%%%%%%%%%%%%%%%%%%%%%%%%%%%%%%%%%%%%%%%%%%%%%%%%
% \section{Implementation}
%\iffalse
%<*package>
%\fi
%
% This section describes the definitions file |childdoc.def|.

% The definitions cannot be loaded using |\usepackage| or |\RequirePackage|
% which has a mechanism to prevent loading a style file more than once.
% When loading the definitions by means of |\input|
% multiple instances have to be prevented manually:
%\iffalse
%This code needs to be before the `\ProvidesFile' directive
%which is defined at the beginning of this file.
%Therefore it is also placed there and commented out here.
%</package>
%<*discard>
%\fi
%    \begin{macrocode}
\ifdefined\childdocmain\endinput\fi
%    \end{macrocode}
%\iffalse
%</discard>
%<*package>
%\fi
%
% \macro{\ifchilddoc}
% \macro{\ifchilddocmanual}
% The conditional |\ifchilddoc| tells whether a
% child (true) or main (false) document is being compiled.
% The conditional |\ifchilddocmanual| tells whether
% the |\includeonly| mechanism is used (false) or
% the selection of child files must be performed manually (true).
% The definitions initialise to false:
%    \begin{macrocode}
\newif\ifchilddoc
\newif\ifchilddocmanual
%    \end{macrocode}

% \macro{\childdocname}
% \macro{\childdocjob}
% The macro |\childdocname| stores the name of the main document
% to be compiled. The macro |\childdocjob| stores the name of
% the document on which the \LaTeX{} compiler was originally invoked.
% The content of |\jobname| cannot be compared
% to filenames specified in the source due to different catcodes.
% The following code rescans |\jobname|, stores the result
% in |\childdocname| and saves a copy in |\childdocjob|:
%    \begin{macrocode}
\edef\childdocname{\scantokens\expandafter{\jobname\noexpand}}
\let\childdocjob\childdocname
%    \end{macrocode}

% \macro{\childdocdisable}
% The macro |\childdocdisable| prevents the main file
% from being processed more than once.
% At this stage, the main document command |\childdocmain|
% is assumed to be called once again where it should do nothing.
% Any subsequent call to it should prevent
% a secondary processing of the main document
% It overwrites the forwarding commands
% |\childdocof| and |\childdocforward|
% with empty macros to prevent further inclusions of the main document:
%    \begin{macrocode}
\newcommand{\childdocdisable}
{
  \renewcommand{\childdocmain}[1]{\renewcommand{\childdocmain}[1]{\endinput}}
  \renewcommand{\childdocof}[1]{}
  \renewcommand{\childdocby}[2][]{}
  \renewcommand{\childdocforward}[2][]{}
  \renewcommand{\childdocdisable}{}
}
%    \end{macrocode}

% \macro{\childdocmain}
% The macro |\childdocmain| is to be called at the top of the main file
% with nothing or the main filename (without extension) as argument.
% First, it breaks loops.
% If the argument is not empty and does not match |\childdocname|
% (which is set by the first inclusion of |childdoc.def|),
% |\ifchilddoc| is set to true, |\includeonly| is applied to the child file
% and |\jobname| is set to the main file
% (for proper handling of |.aux| files):
%    \begin{macrocode}
\newcommand{\childdocmain}[1]
{
  \childdocdisable\childdocmain{}
  \if?#1?\else
    \begingroup
      \def\childdoctmp{#1}
      \ifx\childdoctmp\childdocname
        \def\childdoctmp{}
      \else
        \def\childdoctmp
        {
          \childdoctrue
          \includeonly{\childdocname}
          \def\childdocjob{#1}
          \def\jobname{#1}
        }
      \fi
      \expandafter
    \endgroup
    \childdoctmp
  \fi
}
%    \end{macrocode}

% \macro{\childdocof}
% The command |\childdocof| redirects
% compilation to the main file |#1|.
%    \begin{macrocode}
\newcommand{\childdocof}[1]
{
  \childdocdisable
  \childdoctrue
  \includeonly{\childdocname}
  \def\jobname{#1}
  \def\childdocjob{#1}
  \input{#1}
}
%    \end{macrocode}

% \macro{\childdocby}
% The command |\childdocby| ....
%    \begin{macrocode}
\newcommand{\childdocby}[2][]
{
  \childdocdisable
  \childdoctrue
  \childdocmanualtrue
  \if?#1?\else
    \def\jobname{#2}
  \fi
  \def\childdocjob{#2}
  \input{#2}
  \endinput
}
%    \end{macrocode}

% \macro{\childdocforward}
% The command |\childdocforward| redirects
% compilation to the main file or
% (if the optional argument is given) a child file.
% Parameters are set as if the main file
% or a child file starting with |\childdocof| was compiled.
% Then compilation is handed over to the main file:
%    \begin{macrocode}
\newcommand{\childdocforward}[2][]
{
  \begingroup
    \if?#1?
      \def\childdoctmp
      {
        \def\childdocname{#2}
        \def\childdocjob{#2}
        \def\jobname{#2}
        \input{#2}
        \endinput
      }
    \else
      \def\childdoctmp
      {
        \childdocdisable
        \def\childdocname{#2}
        \childdoctrue
        \includeonly{#2}
        \def\childdocjob{#1}
        \def\jobname{#1}
        \input{#1}
        \endinput
      }
    \fi
    \expandafter
  \endgroup
  \childdoctmp
}
%    \end{macrocode}

% \macro{\childdocforwardprefix}
% The command |\childdocforwardprefix| redirects
% compilation to the main or a child file by means of a pattern.
% The prefix |#1| in the current filename is replaced by |#2|
% and the suffix of the current filename is kept
% (it is assumed that the filename does not contain the substring `|~~~|'
% which is used as a delimiter).
% Compilation is handed over to the new file by |\childdocforward|:
%    \begin{macrocode}
\newcommand{\childdocforwardprefix}[3][]
{
  \begingroup
    \def\childdocextract #2##1~~~{\def\childdoctmp{\childdocforward[#1]{#3##1}}}
    \expandafter\childdocextract\childdocname~~~
    \expandafter
  \endgroup
  \childdoctmp
}
%    \end{macrocode}

% \macro{\childdoc}
% The deprecated macro |\childdoc| is a legacy version of |\childdocmain|:
%    \begin{macrocode}
\newcommand{\childdoc}{\childdocmain}
%    \end{macrocode}

% \macro{\childdocredirect}
% The deprecated macro |\childdocredirect| is a legacy version
% of |\childdocforward| and |\childdocforwardprefix|:
%    \begin{macrocode}
\newcommand{\childdocredirect}[2][]
{
  \begingroup
    \if?#1?
      \def\childdoctmp{\childdocforward{#2}}
    \else
      \def\childdoctmp{\childdocforwardprefix{#1}{#2}}
    \fi
    \expandafter
  \endgroup
  \childdoctmp
}
%    \end{macrocode}

%\iffalse
%</package>
%\fi
%
\endinput
\childdocforward{cdocsamp}"|\\
% |latex -jobname cdocscl1 \|\\
% |  "% \iffalse
%
% childdoc.dtx Copyright (C) 2017-2018 Niklas Beisert
%
% This work may be distributed and/or modified under the
% conditions of the LaTeX Project Public License, either version 1.3
% of this license or (at your option) any later version.
% The latest version of this license is in
%   http://www.latex-project.org/lppl.txt
% and version 1.3 or later is part of all distributions of LaTeX
% version 2005/12/01 or later.
%
% This work has the LPPL maintenance status `maintained'.
%
% The Current Maintainer of this work is Niklas Beisert.
%
% This work consists of the files childdoc.dtx and childdoc.ins
% and the derived files childdoc.def and cdocsamp.tex with
% cdocsch1.tex, cdocsch2.tex, cdocsdrf.tex, cdocsfn1.tex, cdocsfn2.tex.
%
%<package>\ifdefined\childdocmain\endinput\fi
%<package>\ProvidesFile{childdoc.def}[2018/12/30 v2.0 child document driver]
%<samplemain>\ProvidesFile{cdocsamp.tex}[2018/12/30 v2.0 sample for childdoc]
%<*driver>
%\ProvidesFile{childdoc.drv}[2018/12/30 v2.0 childdoc reference manual file]
\PassOptionsToClass{10pt,a4paper}{article}
\documentclass{ltxdoc}

\usepackage[margin=35mm]{geometry}
\usepackage{hyperref}
\usepackage{hyperxmp}
\usepackage[usenames]{color}

\hypersetup{colorlinks=true}
\hypersetup{pdfstartview=FitH}
\hypersetup{pdfpagemode=UseNone}
\hypersetup{pdfsource={}}
\hypersetup{pdflang={en-UK}}
\hypersetup{pdfcopyright={Copyright 2017-2018 Niklas Beisert.
  This work may be distributed and/or modified under the
  conditions of the LaTeX Project Public License, either version 1.3
  of this license or (at your option) any later version.}}
\hypersetup{pdflicenseurl={http://www.latex-project.org/lppl.txt}}
\hypersetup{pdfcontactaddress={ETH Zurich, ITP, HIT K,
  Wolfgang-Pauli-Strasse 27}}
\hypersetup{pdfcontactpostcode={8093}}
\hypersetup{pdfcontactcity={Zurich}}
\hypersetup{pdfcontactcountry={Switzerland}}
\hypersetup{pdfcontactemail={nbeisert@itp.phys.ethz.ch}}
\hypersetup{pdfcontacturl={http://people.phys.ethz.ch/\xmptilde nbeisert/}}

\newcommand{\secref}[1]{\hyperref[#1]{section \ref*{#1}}}

\parskip1ex
\parindent0pt
\let\olditemize\itemize
\def\itemize{\olditemize\parskip0pt}

\begin{document}

\title{The \textsf{childdoc} Package}
\hypersetup{pdftitle={The childdoc Package}}
\author{Niklas Beisert\\[2ex]
  Institut f\"ur Theoretische Physik\\
  Eidgen\"ossische Technische Hochschule Z\"urich\\
  Wolfgang-Pauli-Strasse 27, 8093 Z\"urich, Switzerland\\[1ex]
  \href{mailto:nbeisert@itp.phys.ethz.ch}
  {\texttt{nbeisert@itp.phys.ethz.ch}}}
\hypersetup{pdfauthor={Niklas Beisert}}
\hypersetup{pdfsubject={Manual for the LaTeX2e Package childdoc}}
\date{30 December 2018, \textsf{v2.0}}
\maketitle

\begin{abstract}\noindent
\textsf{childdoc} is a \LaTeXe{} package
that enables the direct compilation
of document sections included by |\include|
to individual files.
\end{abstract}

\begingroup
\parskip0ex
\tableofcontents
\endgroup

%%%%%%%%%%%%%%%%%%%%%%%%%%%%%%%%%%%%%%%%%%%%%%%%%%%%%%%%%%%%%%%%%%%%%%%%%%%%%%%%
%%%%%%%%%%%%%%%%%%%%%%%%%%%%%%%%%%%%%%%%%%%%%%%%%%%%%%%%%%%%%%%%%%%%%%%%%%%%%%%%
\section{Introduction}

\LaTeX{} provides a mechanism to structure a large document (such as a book)
into a main file and several child files (containing the chapters)
using the |\include| command.
This mechanism is beneficial for documents
which span hundreds of pages in order to
make the source file(s) more manageable.
Moreover, compilation can be restricted to
selected child files by means of the |\includeonly| command.
The latter feature can be used to reduce the compilation time while editing
(this was significantly more useful in the earlier days of \LaTeX{})
or to generate a smaller document which is easier to navigate.
Another application of |\includeonly| is to generate
documents consisting of selected parts of the complete document.

However, there are a few drawbacks of the plain |\include| mechanism:
\begin{itemize}
\item
The child files cannot be compiled on their own,
they can only be compiled via the main file.
A naive editing environment
(such as a text editor with an option
to have the current file processed by \LaTeX)
may require one to switch to the main file before compiling;
attempting to compile the child file produces errors.
\item
The main file must be modified (each time)
to adjust the |\includeonly| command
to the present needs. This easily leaves the main file in a messy state.
\item
The generated document will always carry the filename
of the main document. This is inconvenient if
several child files are to be compiled and
to be kept for distribution.
\end{itemize}

The present package provides a simple interface
to make child files individually compilable by \LaTeX{}.
Compiling a child file then has the same effect as compiling
the main file with an |\includeonly| command
to select the appropriate child.
Moreover the generated document will carry the name of the child
rather than the main file.
This resolves all three above issues.

This feature is meant to make the editing of books,
thesis documents and lecture notes somewhat more convenient.
However, the package can also be used efficiently for
composing a series of documents (such as exercise sheets)
which are typically distributed individually.
It then assists the author in generating the individual documents
(potentially in different versions)
as well as a document containing the collected series.
Another application is in developing style files
or other kinds of included material
where compilation of the style file could redirect
to a sample or test file.

%%%%%%%%%%%%%%%%%%%%%%%%%%%%%%%%%%%%%%%%%%%%%%%%%%%%%%%%%%%%%%%%%%%%%%%%%%%%%%%%
%%%%%%%%%%%%%%%%%%%%%%%%%%%%%%%%%%%%%%%%%%%%%%%%%%%%%%%%%%%%%%%%%%%%%%%%%%%%%%%%
\section{Usage}

First of all, the package \textsf{childdoc} is \emph{not} a standard
\LaTeXe{} |.sty| style file! Therefore it needs to be invoked in
a non-standard way.

%%%%%%%%%%%%%%%%%%%%%%%%%%%%%%%%%%%%%%%%%%%%%%%%%%%%%%%%%%%%%%%%%%%%%%%%%%%%%%%%
\subsection{Included Files}
\label{sec:include}

%%%%%%%%%%%%%%%%%%%%%%%%%%%%%%%%%%%%%%%%
\DescribeMacro{\childdocmain}
To use the package, add the commands
\begin{center}
\begin{tabular}{l}
|\input{childdoc.def}|\\
|\childdocmain{}|\\
\end{tabular}
\end{center}
at the very top of the main \LaTeX{} file,
in particular \emph{before} the |\documentclass| statement!
The argument of |\childdocmain| should be left empty
(but it must be present).

%%%%%%%%%%%%%%%%%%%%%%%%%%%%%%%%%%%%%%%%
\DescribeMacro{\childdocof}
Furthermore, add the commands
\begin{center}
\begin{tabular}{l}
|\input{childdoc.def}|\\
|\childdocof{|\textit{main}|}|\\
\end{tabular}
\end{center}
at the top of every child file \textit{child}
which is included by |\include{|\textit{child}|}|
from within the main file
(or at least for those files to be compiled individually).
The argument \textit{main} must be the filename of the main file.

There are a couple of
considerations in setting up the main and child documents:

%%%%%%%%%%%%%%%%%%%%%%%%%%%%%%%%%%%%%%%%
\paragraph{Restrictions.}

Please note the following restrictions:
\begin{itemize}
\item
|\childdocmain| must be called with one argument \textit{main}
to ensure compatibility with earlier version of the package.
It must either be empty (|\childdocmain{}|)
or precisely match the filename of the main file in which it is specified.
See \secref{sec:detection} for further information.
\item
The filename \textit{main} must be specified without the |.tex| extension.
\item
The filename \textit{main} is case sensitive
(even in case-insensitive file systems)
due to internal string comparison.
\item
The argument \textit{main} should be fully expanded, it cannot be a macro.
\item
Subdirectories and special characters should be avoided in filenames.
\item
The command |\childdocmain{|\textit{main}|}| must be followed by a whitespace.
It should not be followed immediately by another command
or by a comment mark `|%|'.
This is because the \TeX{} parser reads the token immediately following
the argument of |\childdocmain| and puts it
at the beginning of every child section;
however, a white\-space is ignored.
\end{itemize}

%%%%%%%%%%%%%%%%%%%%%%%%%%%%%%%%%%%%%%%%
\paragraph{Content of Main File.}

It is advisable to place all content in the child files included by |\include|.
Any output contained in the main file will appear in all child documents
unless suppressed manually;
it cannot be suppressed automatically by the |\includeonly| directive
and thus should normally be avoided.
A method to include some content in the main file
by means of conditional processing is described in \secref{sec:conditional}.

%%%%%%%%%%%%%%%%%%%%%%%%%%%%%%%%%%%%%%%%
\paragraph{Page Numbering.}

When only a part of the document is compiled,
the appropriate numbering of pages
(as well as other status parameters)
is determined from the |.aux| files.
The latter contain information from previous passes.
However this information needs to propagate through
all intermediate child documents.
Therefore the page numbering in child documents may well
be inconsistent until the complete document is compiled at least once.

A useful (if unconventional) way to always ensure a consistent
page numbering is to restart the numbering in each child document
and denote the pages by `\textit{child}|.|\textit{page}'
where \textit{child} represents the chapter/section number of the child file.
This can be achieved by the command
|\numberwithin{page}{|\textit{child}|}|
of the \textsf{amsmath} package
where \textit{child} can be |chapter| or |section|
depending on the chosen structuring.
Alternatively, one can modify the macro |\thepage| appropriately
and reset the counter |page| at the start of each child file.

%%%%%%%%%%%%%%%%%%%%%%%%%%%%%%%%%%%%%%%%%%%%%%%%%%%%%%%%%%%%%%%%%%%%%%%%%%%%%%%%
\subsection{Conditional Processing}
\label{sec:conditional}

The package provides a mechanism to compile different versions
of a document. To customise the versions further some conditional processing
can come in handy to distinguish which version is being compiled.
The package provides two macros to describe the compilation context:

%%%%%%%%%%%%%%%%%%%%%%%%%%%%%%%%%%%%%%%%
\DescribeMacro{\ifchilddoc}
The conditional |\ifchilddoc| distinguishes between the compilation of
child documents and the main document:
%
\begin{center}
|\ifchilddoc |\textit{child-code}| |[|\||else |\textit{main-code}]| \||fi|
\end{center}

%%%%%%%%%%%%%%%%%%%%%%%%%%%%%%%%%%%%%%%%
\DescribeMacro{\childdocname}
\DescribeMacro{\childdocjob}
The macro |\childdocname| contains the filename (without extension)
of the main or child file being processed.
Note that |\childdocjob| will always contain the name of the main file.

%%%%%%%%%%%%%%%%%%%%%%%%%%%%%%%%%%%%%%%%
\paragraph{Title Page.}

Conditional processing can be used to include a title or banner page
in the main document when proper precautions are taken.
Importantly, the code in the main file should ensure that the page counter
(as well as other status parameters which are stored in the |.aux| files)
takes the same value after the conditional processing.
Otherwise the page numbers may take divergent values
depending on which part is compiled.

For example, a title page could be declared by:
%
\begin{center}
\begin{tabular}{l}
|\ifchilddoc\||else|\\
|\addtocounter{page}{-1}|\\
\textit{code for title page}\\
|\newpage|\\
|\||fi|
\end{tabular}
\end{center}
%
A banner page for the child documents can be generated by:
%
\begin{center}
\begin{tabular}{l}
|\ifchilddoc|\\
|\addtocounter{page}{-1}|\\
\textit{code for banner page}\\
|\newpage|\\
|\||fi|
\end{tabular}
\end{center}
%
Here one could write a message such as:
\begin{center}
|This is the part \childdocname{} of \childdocjob{}.|
\end{center}

%%%%%%%%%%%%%%%%%%%%%%%%%%%%%%%%%%%%%%%%%%%%%%%%%%%%%%%%%%%%%%%%%%%%%%%%%%%%%%%%
\subsection{Flags}
\label{sec:flags}

The package makes it easy to generate different versions
of the main or child documents.
To this end compilation flags can be defined
and assigned different default values.
They will be particularly useful in conjunction
with the forwarding mechanism described in \secref{sec:forward}.

For example, it may be useful to have a flag |\version|
which can be set to |draft| or |final|.
The document source will contain some conditional code
depending on the value of |\version|.
Suppose further, the flag should default to |final| for the main file
and to |draft| for child files
which is a natural assignment for editing the document.
This is achieved by placing the following code
in the preamble of the main document
(below the |\childdocmain| directive):
%
\begin{center}
\begin{tabular}{l}
|\ifchilddoc|\\
|\providecommand{\version}{draft}|\\
|\||else|\\
|\providecommand{\version}{final}|\\
|\||fi|
\end{tabular}
\end{center}
%
The definition by |\providecommand| makes sure
that previous definitions are not overwritten.
Further statements |\providecommand{\version}{...}|
can thus be added before the above code to override it.

For the main file, one might add a line
(between |\childdocmain| and the above block)
%
\begin{center}
|%\ifchilddoc\||else\providecommand{\version}{draft}\||fi|
\end{center}
%
which can be uncommented to produce a draft version.
Likewise one can add a line to the very top of a child file
(above the |\childdocof{|\textit{main}|}| directive)
%
\begin{center}
|%\providecommand{\version}{final}|
\end{center}
%
which can be uncommented to produce the final version of this child document.

%%%%%%%%%%%%%%%%%%%%%%%%%%%%%%%%%%%%%%%%%%%%%%%%%%%%%%%%%%%%%%%%%%%%%%%%%%%%%%%%
\subsection{Forwarding}
\label{sec:forward}

Different versions of the main or child documents
using compilation flags as described in \secref{sec:flags}
can be (permanently) stored in different files
for convenient compilation, viewing and distribution.
To this end, the package defines a command
to pass on compilation to a different file:

%%%%%%%%%%%%%%%%%%%%%%%%%%%%%%%%%%%%%%%%
\DescribeMacro{\childdocforward}
The command |\childdocforward| redirects processing to
another source file:
%
\begin{center}
\begin{tabular}{l}
|\input{childdoc.def}|\\
|\childdocforward[|\textit{main}|]{|\textit{dest}|}|\\
\end{tabular}
\end{center}
%
The argument \textit{dest} is the destination file
(without extension).
It should be the main file or one of the child files.
Note that further \textsf{childdoc} directives
such as |\childdocof| and |\childdocforward|
in the indicated file will be processed in this form.
The optional argument \textit{main}
passes on directly to the main file \textit{main}
while pretending to compile the child \textit{dest}.
This form behaves as if \textit{dest}
issues |\childdocof{|\textit{main}|}| right away,
and no further \textsf{childdoc} directives will be processed.

%%%%%%%%%%%%%%%%%%%%%%%%%%%%%%%%%%%%%%%%
\DescribeMacro{\...prefix}
In the alternative form |\childdocforwardprefix|,
%
\begin{center}
\begin{tabular}{l}
|\input{childdoc.def}|\\
|\childdocforwardprefix[|\textit{main}|]{|\textit{prefix}|}{|\textit{dest}|}|
\end{tabular}
\end{center}
%
the destination file is determined by a pattern
depending on the current file:
To make this work, the current file must be called
`{\textit{prefix}\hspace{0.2em}\textit{suffix}}'
with \textit{prefix} matching precisely the argument.
Processing is then passed on to the file
`{\textit{dest}\hspace{0.2em}\textit{suffix}}'.
Surely, the same effect is achieved by
directly specifying the
argument `{\textit{dest}\hspace{0.2em}\textit{suffix}}'
in the first form.
However, that requires to set up a different file
for each child. With the alternative form of the command
all these files can have exactly the same content
which simplifies setting them up and maintaining them.

For example, the following file |draft.tex|
with a compilation flag |\version| as described in \secref{sec:flags}
compiles the main document as a draft:
%
\begin{center}
\begin{tabular}{l}
|\def\version{draft}|\\
|\input{childdoc.def}|\\
|\childdocforward{|\textit{main}|}|
\end{tabular}
\end{center}
%
Likewise, the following files |final|\textit{nn}|.tex|
compile the final version of the child document
|child|\textit{nn}|.tex|:
%
\begin{center}
\begin{tabular}{l}
|\def\version{final}|\\
|\input{childdoc.def}|\\
|\childdocforwardprefix{final}{child}|
\end{tabular}
\end{center}
%

Note that when several versions of a main file and/or of each child file
are to be generated, it may be convenient to set up a |Makefile| or
shell script to automatise the process.

%%%%%%%%%%%%%%%%%%%%%%%%%%%%%%%%%%%%%%%%%%%%%%%%%%%%%%%%%%%%%%%%%%%%%%%%%%%%%%%%
\subsection{Command Line Processing}
\label{sec:commandline}

The effect of redirection files can also be achieved by invoking
the \LaTeX{} compiler with a more elaborate command line.
Most conveniently this should be done as part
of a shell script or a |Makefile|.

When using \textsf{childdoc} in the main file, the following
command lines effectively perform a redirection
(note that depending on the shell being used,
backslashes may have to be doubled: `|\|' $\to$ `|\\|'):
%
\begin{center}
|... -jobname "|\textit{target}|" |\\|"|[\textit{flags}]%
|\input{childdoc.def}\childdocforward[|\textit{main}|]{|\textit{dest}|}"|
\end{center}
%
Here \textit{target} is the name of the output file,
\textit{main} is the name of the main file
and \textit{dest} is the name of the main or child file to be processed
(all filenames without extensions).
The optional argument \textit{main} can be omitted
if \textit{main} matches \textit{dest}.
Optionally, compilation \textit{flags} can be defined via |\def| commands.
This command line makes the \TeX{} engine believe
it is compiling the file \textit{target}
whose content is specified as the latter parameter.
The provided code then forwards the processing to
\textit{main} or \textit{dest} as described in \secref{sec:forward}.

%%%%%%%%%%%%%%%%%%%%%%%%%%%%%%%%%%%%%%%%%%%%%%%%%%%%%%%%%%%%%%%%%%%%%%%%%%%%%%%%
\subsection{Include by Input}
\label{sec:input}

Including child documents by |\include| has some restrictions by design.
Most notably, the content of a child document always occupies
its own set of pages; pages cannot be shared between child documents.
Usually, this behaviour makes perfect sense
because each child document contain an essential part of the document.
However, in some situations it may be desirable to compose
a document from a collection of parts
without having mandatory page breaks between then.
For this case, the package
provides a mechanism to include parts
by |\input| which can also be processed individually.
However, by construction this mechanism
requires manual handling of the content to be output.

%%%%%%%%%%%%%%%%%%%%%%%%%%%%%%%%%%%%%%%%
\DescribeMacro{\ifchilddocmanual}
The main file should be prepared as usual, see \secref{sec:include}.
However, the document body must make a distinction
between processing of an individual part and of the main document, e.g.:
%
\begin{center}
\begin{tabular}{l}
|\ifchilddocmanual|\\
|\input{\childdocname}|\\
|\||else|\\
\textit{document body with }|\input{|\textit{part}|}|\\
|\||fi|
\end{tabular}
\end{center}
%
The conditional |\ifchilddocmanual| is true whenever
a part to be included by |\input| is being compiled,
and the name of the part is stored in |\childdocname|.

%%%%%%%%%%%%%%%%%%%%%%%%%%%%%%%%%%%%%%%%
\DescribeMacro{\childdocby}
Each part to be included by |\input| should start with:
%
\begin{center}
\begin{tabular}{l}
|\input{childdoc.def}|\\
|\childdocby{|\textit{main}|}|\\
\end{tabular}
\end{center}
%
The directive |\childdocby| is similar to |\childdocof|
described in \secref{sec:include},
but the subsequent selection of content must be done manually.
To that end, both |\ifchilddoc| and |\ifchilddocmanual|
will be true upon processing of a part,
and the name of the part is stored in |\childdocname|.
Note that |\jobname| will be set to the filename of the current part
so that each part receives an individual |.aux| file
that does not interfere with the |.aux| file(s) of the main document.
This behaviour can be altered by the alternative form
|\childdocby[*]{|\textit{main}|}| (with a non-empty optional argument)
which uses the |.aux| file of the main document
by setting |\jobname| to \textit{main}.

%%%%%%%%%%%%%%%%%%%%%%%%%%%%%%%%%%%%%%%%%%%%%%%%%%%%%%%%%%%%%%%%%%%%%%%%%%%%%%%%
\subsection{Driver Development}
\label{sec:driver}

The \textsf{childdoc} mechanism can also be use for the development
of definition files such as \LaTeX{} styles or classes.
This case differs from the above setup with multiple parts
included by |\include| in that no |\includeonly| should be invoked.
This can be achieved by starting the include file
(before |\ProvidesPackage|) with:
%
\begin{center}
\begin{tabular}{l}
|\input{childdoc.def}|\\
|\childdocforward{|\textit{main}|}|\\
\end{tabular}
\end{center}
%
or alternatively with:
%
\begin{center}
\begin{tabular}{l}
|\input{childdoc.def}|\\
|\childdocby{|\textit{main}|}|\\
\end{tabular}
\end{center}
%
Both forms have slightly different effects as described above.
The main file is prepared as usual, see \secref{sec:include}.

%%%%%%%%%%%%%%%%%%%%%%%%%%%%%%%%%%%%%%%%%%%%%%%%%%%%%%%%%%%%%%%%%%%%%%%%%%%%%%%%
\subsection{Legacy Detection}
\label{sec:detection}

The directive |\childdocmain| in the main file can detect
whether the complete document or merely a child is to be compiled
even without using the directive |\childdocof|.
This method is deprecated because it is less robust
and there is no compelling reason to use it;
it is merely provided for backward compatibility
and it may be removed in future versions.

If the detection mechanism is to be used,
it is mandatory to correctly specify
the filename of the main file as the argument of |\childdocmain|:
%
\begin{center}
\begin{tabular}{l}
|\input{childdoc.def}|\\
|\childdocmain{|\textit{main}|}|\\
\end{tabular}
\end{center}
%
If |\jobname| does not match the argument \textit{main} of |\childdocmain|,
it is assumed that |\jobname| points to the child file to be compiled.
When using |\childdocmain| with the main file specified as argument,
it suffices to start a child file
with just |\input{|\textit{main}|}|
without loading of the package and using |\childdocof|.
If instead all processing is done
with the appropriate \textsf{childdoc} directives,
the argument of \textit{main} of |\childdocmain| can be empty.

An alternative version of the command line processing described
in \secref{sec:commandline} using the detection mechanism reads:
%
\begin{center}
|... -jobname "|\textit{target}|" "|[\textit{flags}]%
[|\def\jobname{|\textit{dest}|}|]|\input{|\textit{main}|}"|
\end{center}

%%%%%%%%%%%%%%%%%%%%%%%%%%%%%%%%%%%%%%%%%%%%%%%%%%%%%%%%%%%%%%%%%%%%%%%%%%%%%%%%
\subsection{Manual Code}
\label{sec:manual}

In case one cannot be certain whether the definitions file |childdoc.def|
is installed on the target \TeX{} distribution
and one prefers not to ship it,
it is conceivable to paste a few relevant commands into the sources.

To that end, drop all statements |\input{childdoc.def}|
and perform the replacements as outlined below.
Instead of |\childdocmain{|\textit{main}|}| add the following code
to the top of the main file:
%
\begin{center}
\begin{tabular}{l}
|\||ifdefined\childdocname\endinput\||fi\newif\ifchilddoc|\\
|\edef\childdocname{\scantokens\expandafter{\jobname\noexpand}}|\\
|\def\childdocmain{|\textit{main}|}\||ifx\childdocmain\childdocname\||else|\\
|\childdoctrue\includeonly{\childdocname}\let\jobname\childdocmain\||fi|\\
\end{tabular}
\end{center}
%
Instead of |\childdocof{|\textit{main}|}| just include the main file
at the top of each child file:
%
\begin{center}
|\input{|\textit{main}|}|
\end{center}
%
A simple redirection |\childdocforward{|\textit{dest}|}| is achieved by:
%
\begin{center}
|\def\jobname{|\textit{dest}|}\input{\jobname}|
\end{center}
%
The redirection with prefix
|\childdocforwardprefix[|\textit{prefix}|]{|\textit{dest}|}|
is accomplished by:
%
\begin{center}
\begin{tabular}{l}
|{\edef\jobname{\scantokens\expandafter{\jobname\noexpand}}|\\
|\def\redirectjob |\textit{prefix}|#1~~~{\gdef\jobname{|\textit{dest}|#1}}|\\
|\expandafter\redirectjob\jobname~~~}\input{\jobname}|
\end{tabular}
\end{center}

In an alternative approach,
child documents can be compiled by a specific command line
without additional code or specific definitions:
%
\begin{center}
|... -jobname "|\textit{target}|" "|[\textit{flags}]%
|\includeonly{|\textit{dest}|}\input{|\textit{main}|}"|
\end{center}
%

%%%%%%%%%%%%%%%%%%%%%%%%%%%%%%%%%%%%%%%%%%%%%%%%%%%%%%%%%%%%%%%%%%%%%%%%%%%%%%%%
%%%%%%%%%%%%%%%%%%%%%%%%%%%%%%%%%%%%%%%%%%%%%%%%%%%%%%%%%%%%%%%%%%%%%%%%%%%%%%%%
\section{Information}

%%%%%%%%%%%%%%%%%%%%%%%%%%%%%%%%%%%%%%%%%%%%%%%%%%%%%%%%%%%%%%%%%%%%%%%%%%%%%%%%
\subsection{Copyright}

Copyright \copyright{} 2017--2018 Niklas Beisert

This work may be distributed and/or modified under the
conditions of the \LaTeX{} Project Public License, either version 1.3
of this license or (at your option) any later version.
The latest version of this license is in
  \url{http://www.latex-project.org/lppl.txt}
and version 1.3 or later is part of all distributions of \LaTeX{}
version 2005/12/01 or later.

This work has the LPPL maintenance status `maintained'.

The Current Maintainer of this work is Niklas Beisert.

This work consists of the files |README.txt|, |childdoc.ins| and |childdoc.dtx|
as well as the derived files |childdoc.def|, |cdocsamp.tex|
with |cdocsch1.tex|, |cdocsch2.tex|, |cdocspt3.tex|, |cdocspt4.tex|,
|cdocsdrf.tex|, |cdocsfn1.tex|, |cdocsfn2.tex|
as well as |childdoc.pdf|.

%%%%%%%%%%%%%%%%%%%%%%%%%%%%%%%%%%%%%%%%%%%%%%%%%%%%%%%%%%%%%%%%%%%%%%%%%%%%%%%%
\subsection{Files and Installation}

The package consists of the files:
%
\begin{center}
\begin{tabular}{ll}
    |README.txt|   & readme file \\
    |childdoc.ins| & installation file \\
    |childdoc.dtx| & source file \\
    |childdoc.def| & definition file \\
    |cdocsamp.tex| & sample main file \\
    |cdocsch1.tex| & sample include file \\
    |cdocsch2.tex| & sample include file \\
    |cdocspt3.tex| & sample part file \\
    |cdocspt4.tex| & sample part file \\
    |cdocsdrf.tex| & sample redirection file \\
    |cdocsfn1.tex| & sample redirection file \\
    |cdocsfn2.tex| & sample redirection file \\
    |childdoc.pdf| & manual
\end{tabular}
\end{center}
%
The distribution consists of the files
|README.txt|, |childdoc.ins| and |childdoc.dtx|.
%
\begin{itemize}
\item
Run (pdf)\LaTeX{} on |childdoc.dtx|
to compile the manual |childdoc.pdf| (this file).
\item
Run \LaTeX{} on |childdoc.ins| to create the definitions file |childdoc.def|
and the sample |cdocsamp.tex| with include files
|cdocsch1.tex|, |cdocsch2.tex|, |cdocspt3.tex|, |cdocspt4.tex|,
|cdocsdrf.tex|, |cdocsfn1.tex|, |cdocsfn2.tex|.
Then copy the file |childdoc.def| to an appropriate directory of your \LaTeX{}
distribution, e.g.\ \textit{texmf-root}|/tex/latex/childdoc|.
\end{itemize}

%%%%%%%%%%%%%%%%%%%%%%%%%%%%%%%%%%%%%%%%%%%%%%%%%%%%%%%%%%%%%%%%%%%%%%%%%%%%%%%%
\subsection{Related CTAN Packages}

There are several other packages which offer a similar functionality:
%
\begin{itemize}
\item
The packages
\href{http://ctan.org/pkg/docmute}{\textsf{docmute}},
\href{http://ctan.org/pkg/includex}{\textsf{includex}} and
\href{http://ctan.org/pkg/standalone}{\textsf{standalone}}
provide commands to include only the document body of
a child file thus allowing both files to be compiled individually.
\item
The packages \href{http://ctan.org/pkg/subdocs}{\textsf{subdocs}}
and \href{http://ctan.org/pkg/subfiles}{\textsf{subfiles}}
provide structures in which the main and child documents can be
encapsulated and allowing them to be compiled individually.
The inclusion mechanism is different from the conventional |\include|.
\item
The package \href{http://ctan.org/pkg/combine}{\textsf{combine}}
is an elaborate solution to combine several documents into one.
\end{itemize}
%
See also the CTAN topic \href{http://ctan.org/topic/subdocs}{\textsf{subdocs}}
for further related packages.
The present package differs from the above solutions in that
a document structure constructed with the conventional |\include| mechanism
just needs two extra commands at the top of every file
such that all constituent files can be compiled individually.

%%%%%%%%%%%%%%%%%%%%%%%%%%%%%%%%%%%%%%%%%%%%%%%%%%%%%%%%%%%%%%%%%%%%%%%%%%%%%%%%
%\subsection{Feature Suggestions}
%
%The following is a list of features which may be useful for future
%versions of this package:
%%
%\begin{itemize}
%\item
%\ldots
%\end{itemize}

%%%%%%%%%%%%%%%%%%%%%%%%%%%%%%%%%%%%%%%%%%%%%%%%%%%%%%%%%%%%%%%%%%%%%%%%%%%%%%%%
\subsection{Revision History}

%%%%%%%%%%%%%%%%%%%%%%%%%%%%%%%%%%%%%%%%
\paragraph{v2.0:} 2018/12/30

\begin{itemize}
\item
immediate forward processing
\item
added |\childdocby| mechanism
\item
manual restructured
\end{itemize}

%%%%%%%%%%%%%%%%%%%%%%%%%%%%%%%%%%%%%%%%
\paragraph{v1.6:} 2018/01/17

\begin{itemize}
\item
application for development of include files
\item
corrections to manual
\end{itemize}

%%%%%%%%%%%%%%%%%%%%%%%%%%%%%%%%%%%%%%%%
\paragraph{v1.5:} 2017/05/21

\begin{itemize}
\item
more complete structuring introduced
\item
|\childdocof| introduced
\item
|\childdoc| renamed to |\childdocmain|
\item
|\childredirect| renamed to |\childdocforward| and |\childdocforwardprefix|
and functionality expanded
\end{itemize}

%%%%%%%%%%%%%%%%%%%%%%%%%%%%%%%%%%%%%%%%
\paragraph{v1.0:} 2017/04/27

\begin{itemize}
\item
manual and install package
\item
first version published on CTAN
\end{itemize}

%%%%%%%%%%%%%%%%%%%%%%%%%%%%%%%%%%%%%%%%
\paragraph{v0.6:} 2017/04/26

\begin{itemize}
\item
redirection mechanism added
\end{itemize}

%%%%%%%%%%%%%%%%%%%%%%%%%%%%%%%%%%%%%%%%
\paragraph{v0.5:} 2017/04/26

\begin{itemize}
\item
functionality in definition file
\end{itemize}


%%%%%%%%%%%%%%%%%%%%%%%%%%%%%%%%%%%%%%%%%%%%%%%%%%%%%%%%%%%%%%%%%%%%%%%%%%%%%%%%
%%%%%%%%%%%%%%%%%%%%%%%%%%%%%%%%%%%%%%%%%%%%%%%%%%%%%%%%%%%%%%%%%%%%%%%%%%%%%%%%
%%%%%%%%%%%%%%%%%%%%%%%%%%%%%%%%%%%%%%%%%%%%%%%%%%%%%%%%%%%%%%%%%%%%%%%%%%%%%%%%
\appendix

\settowidth\MacroIndent{\rmfamily\scriptsize 000\ }

 \DocInput{childdoc.dtx}

\end{document}
%</driver>
% \fi
%
% %%%%%%%%%%%%%%%%%%%%%%%%%%%%%%%%%%%%%%%%%%%%%%%%%%%%%%%%%%%%%%%%%%%%%%%%%%%%%%
% %%%%%%%%%%%%%%%%%%%%%%%%%%%%%%%%%%%%%%%%%%%%%%%%%%%%%%%%%%%%%%%%%%%%%%%%%%%%%%
% \section{Sample}
%\iffalse
%<*samplemain>
%\fi
%
% The following presents a sample document
% with two chapters, two parts, a title page,
% a compile flag as well as three forwarding files to set the flag.
% It consists of eight |.tex| files:
% \begin{center}
% \begin{tabular}{ll}
% |cdocsamp.tex|&main file\\
% |cdocsch1.tex|&include file for chapter 1\\
% |cdocsch2.tex|&include file for chapter 2\\
% |cdocspt3.tex|&include file for part 3\\
% |cdocspt4.tex|&include file for part 4\\
% |cdocsdrf.tex|&forwarding file for main file in draft mode\\
% |cdocsfi1.tex|&forwarding file for final version of chapter 1\\
% |cdocsfi2.tex|&forwarding file for final version of chapter 2\\
% \end{tabular}
% \end{center}
% Each of the eight files can be compiled directly by the \LaTeX{} compiler.
%
% %%%%%%%%%%%%%%%%%%%%%%%%%%%%%%%%%%%%%%
% \paragraph{Main File.}
%
% The main file is called |cdocsamp.tex|.
%
% Load the \textsf{childdoc} definitions and
% declare the filename for the main document:
%    \begin{macrocode}
\input{childdoc.def}
\childdocmain{}
%    \end{macrocode}

% Optional override for |\version| flag:
%    \begin{macrocode}
%%\ifchilddoc\else\providecommand{\version}{draft}\fi
%    \end{macrocode}

% Define the default values for the |\version| flag
% (|final| for the main file and |draft| for childs):
%    \begin{macrocode}
\ifchilddoc
\providecommand{\version}{draft}
\else
\providecommand{\version}{final}
\fi
%    \end{macrocode}

% Load the standard document class:
%    \begin{macrocode}
\documentclass[12pt]{article}
%    \end{macrocode}

% Start the document body:
%    \begin{macrocode}
\begin{document}
%    \end{macrocode}

% Declare a title page.
% Print title, part of document being processed and version flag:
%    \begin{macrocode}
\addtocounter{page}{-1}
\begin{center}
{\LARGE\bfseries{}childdoc example\par}
\vspace{1cm}
\ifchilddoc
\ifchilddocmanual part\else chapter\fi:
`\childdocname' of `\childdocjob'\par
\else
main document: `\childdocjob'\par
\fi
version: \version\par
\end{center}
\newpage
%    \end{macrocode}

% Manually include selected file,
% otherwise process as usual:
%    \begin{macrocode}
\ifchilddocmanual
\section*{part `\childdocname'}
\input{\childdocname}
\else
%    \end{macrocode}

% Include the two chapters:
%    \begin{macrocode}
\include{cdocsch1}
\include{cdocsch2}
%    \end{macrocode}

% Include the two parts unless only chapters should be displayed:
%    \begin{macrocode}
\ifchilddoc\else
\section{part three}
\input{cdocspt3}
\section{part four}
\input{cdocspt4}
\fi
%    \end{macrocode}

% Process as usual until here:
%    \begin{macrocode}
\fi
%    \end{macrocode}

% End of document body:
%    \begin{macrocode}
\end{document}
%    \end{macrocode}
%\iffalse
%</samplemain>
%\fi
%
% %%%%%%%%%%%%%%%%%%%%%%%%%%%%%%%%%%%%%%
% \paragraph{Chapter Include Files.}
%
% The include files are called |cdocsch1.tex| and |cdocsch2.tex|.
%
%\iffalse
%<*samplechap1|samplechap2>
%\fi

% Optional override for |\version| flag:
%    \begin{macrocode}
%%\providecommand{\version}{final}
%    \end{macrocode}

% Include the main document:
%    \begin{macrocode}
\input{childdoc.def}
\childdocof{cdocsamp}
%    \end{macrocode}

%\iffalse
%</samplechap1|samplechap2>
%\fi
%
%\iffalse
%<*samplechap1>
%\fi
% Some text for chapter 1:
%    \begin{macrocode}
\section{one}
some text in chapter one
%    \end{macrocode}

%\iffalse
%</samplechap1>
%\fi
% Some text for chapter 2:
%\iffalse
%<*samplechap2>
%\fi
%    \begin{macrocode}
\section{two}
more text in chapter two
%    \end{macrocode}

%\iffalse
%</samplechap2>
%\fi
%
% %%%%%%%%%%%%%%%%%%%%%%%%%%%%%%%%%%%%%%
% \paragraph{Part Include Files.}
%
% The include files are called |cdocspt3.tex| and |cdocspt4.tex|.
%
%\iffalse
%<*samplepart3|samplepart4>
%\fi

% Optional override for |\version| flag:
%    \begin{macrocode}
%%\providecommand{\version}{final}
%    \end{macrocode}

% Include the main document:
%    \begin{macrocode}
\input{childdoc.def}
\childdocby{cdocsamp}
%    \end{macrocode}

%\iffalse
%</samplepart3|samplepart4>
%\fi
%
%\iffalse
%<*samplepart3>
%\fi
% Some text for part 3:
%    \begin{macrocode}
some text in part three
%    \end{macrocode}

%\iffalse
%</samplepart3>
%\fi
% Some text for part 4:
%\iffalse
%<*samplepart4>
%\fi
%    \begin{macrocode}
more text in part four
%    \end{macrocode}

%\iffalse
%</samplepart4>
%\fi
%
% %%%%%%%%%%%%%%%%%%%%%%%%%%%%%%%%%%%%%%
% \paragraph{Forwarding for a Complete Draft.}
%
% The following forwarding file |cdocsdrf.tex|
% compiles the main document in draft mode:
%\iffalse
%<*sampledraft>
%\fi
%    \begin{macrocode}
\def\version{draft}
\input{childdoc.def}
\childdocforward{cdocsamp}
%    \end{macrocode}

%\iffalse
%</sampledraft>
%\fi
%
% %%%%%%%%%%%%%%%%%%%%%%%%%%%%%%%%%%%%%%
% \paragraph{Forwarding for Final Version of the Chapters.}
%
% The following forwarding files |cdocsfn1.tex| and |cdocsfn2.tex|
% (with identical content)
% compile the final versions of the child documents
% |cdocsch1.tex| and |cdocsch2.tex|, respectively:
%\iffalse
%<*samplefinal>
%\fi
%    \begin{macrocode}
\def\version{final}
\input{childdoc.def}
\childdocforwardprefix[cdocsamp]{cdocsfn}{cdocsch}
%    \end{macrocode}

%\iffalse
%</samplefinal>
%\fi
%
% %%%%%%%%%%%%%%%%%%%%%%%%%%%%%%%%%%%%%%
% \paragraph{Command Line Processing.}
%
% The following three command lines generate the output files
% |cdocscld|, |cdocscl1| and |cdocscl2|
% which should be identical to
% |cdocsdrf|, |cdocsch1| and |cdocsfn2|, respectively:
% \begin{center}
% \begin{tabular}{l}
% |latex -jobname cdocscld \|\\
% |  "\def\version{draft}\input{childdoc.def}\childdocforward{cdocsamp}"|\\
% |latex -jobname cdocscl1 \|\\
% |  "\input{childdoc.def}\childdocforward[cdocsamp]{cdocsch1}"|\\
% |latex -jobname cdocscl2 \|\\
% |  "\def\version{final}\input{childdoc.def}\childdocforward{cdocsch2}"|
% \end{tabular}
% \end{center}
% Note that the trailing backslash on each first line
% merely continues the input to the second line
% (for convenient cut ant paste).
% Furthermore, the command |latex| can be replaced by any
% of its alternative versions such as |pdflatex|.
%
% %%%%%%%%%%%%%%%%%%%%%%%%%%%%%%%%%%%%%%%%%%%%%%%%%%%%%%%%%%%%%%%%%%%%%%%%%%%%%%
% %%%%%%%%%%%%%%%%%%%%%%%%%%%%%%%%%%%%%%%%%%%%%%%%%%%%%%%%%%%%%%%%%%%%%%%%%%%%%%
% \section{Implementation}
%\iffalse
%<*package>
%\fi
%
% This section describes the definitions file |childdoc.def|.

% The definitions cannot be loaded using |\usepackage| or |\RequirePackage|
% which has a mechanism to prevent loading a style file more than once.
% When loading the definitions by means of |\input|
% multiple instances have to be prevented manually:
%\iffalse
%This code needs to be before the `\ProvidesFile' directive
%which is defined at the beginning of this file.
%Therefore it is also placed there and commented out here.
%</package>
%<*discard>
%\fi
%    \begin{macrocode}
\ifdefined\childdocmain\endinput\fi
%    \end{macrocode}
%\iffalse
%</discard>
%<*package>
%\fi
%
% \macro{\ifchilddoc}
% \macro{\ifchilddocmanual}
% The conditional |\ifchilddoc| tells whether a
% child (true) or main (false) document is being compiled.
% The conditional |\ifchilddocmanual| tells whether
% the |\includeonly| mechanism is used (false) or
% the selection of child files must be performed manually (true).
% The definitions initialise to false:
%    \begin{macrocode}
\newif\ifchilddoc
\newif\ifchilddocmanual
%    \end{macrocode}

% \macro{\childdocname}
% \macro{\childdocjob}
% The macro |\childdocname| stores the name of the main document
% to be compiled. The macro |\childdocjob| stores the name of
% the document on which the \LaTeX{} compiler was originally invoked.
% The content of |\jobname| cannot be compared
% to filenames specified in the source due to different catcodes.
% The following code rescans |\jobname|, stores the result
% in |\childdocname| and saves a copy in |\childdocjob|:
%    \begin{macrocode}
\edef\childdocname{\scantokens\expandafter{\jobname\noexpand}}
\let\childdocjob\childdocname
%    \end{macrocode}

% \macro{\childdocdisable}
% The macro |\childdocdisable| prevents the main file
% from being processed more than once.
% At this stage, the main document command |\childdocmain|
% is assumed to be called once again where it should do nothing.
% Any subsequent call to it should prevent
% a secondary processing of the main document
% It overwrites the forwarding commands
% |\childdocof| and |\childdocforward|
% with empty macros to prevent further inclusions of the main document:
%    \begin{macrocode}
\newcommand{\childdocdisable}
{
  \renewcommand{\childdocmain}[1]{\renewcommand{\childdocmain}[1]{\endinput}}
  \renewcommand{\childdocof}[1]{}
  \renewcommand{\childdocby}[2][]{}
  \renewcommand{\childdocforward}[2][]{}
  \renewcommand{\childdocdisable}{}
}
%    \end{macrocode}

% \macro{\childdocmain}
% The macro |\childdocmain| is to be called at the top of the main file
% with nothing or the main filename (without extension) as argument.
% First, it breaks loops.
% If the argument is not empty and does not match |\childdocname|
% (which is set by the first inclusion of |childdoc.def|),
% |\ifchilddoc| is set to true, |\includeonly| is applied to the child file
% and |\jobname| is set to the main file
% (for proper handling of |.aux| files):
%    \begin{macrocode}
\newcommand{\childdocmain}[1]
{
  \childdocdisable\childdocmain{}
  \if?#1?\else
    \begingroup
      \def\childdoctmp{#1}
      \ifx\childdoctmp\childdocname
        \def\childdoctmp{}
      \else
        \def\childdoctmp
        {
          \childdoctrue
          \includeonly{\childdocname}
          \def\childdocjob{#1}
          \def\jobname{#1}
        }
      \fi
      \expandafter
    \endgroup
    \childdoctmp
  \fi
}
%    \end{macrocode}

% \macro{\childdocof}
% The command |\childdocof| redirects
% compilation to the main file |#1|.
%    \begin{macrocode}
\newcommand{\childdocof}[1]
{
  \childdocdisable
  \childdoctrue
  \includeonly{\childdocname}
  \def\jobname{#1}
  \def\childdocjob{#1}
  \input{#1}
}
%    \end{macrocode}

% \macro{\childdocby}
% The command |\childdocby| ....
%    \begin{macrocode}
\newcommand{\childdocby}[2][]
{
  \childdocdisable
  \childdoctrue
  \childdocmanualtrue
  \if?#1?\else
    \def\jobname{#2}
  \fi
  \def\childdocjob{#2}
  \input{#2}
  \endinput
}
%    \end{macrocode}

% \macro{\childdocforward}
% The command |\childdocforward| redirects
% compilation to the main file or
% (if the optional argument is given) a child file.
% Parameters are set as if the main file
% or a child file starting with |\childdocof| was compiled.
% Then compilation is handed over to the main file:
%    \begin{macrocode}
\newcommand{\childdocforward}[2][]
{
  \begingroup
    \if?#1?
      \def\childdoctmp
      {
        \def\childdocname{#2}
        \def\childdocjob{#2}
        \def\jobname{#2}
        \input{#2}
        \endinput
      }
    \else
      \def\childdoctmp
      {
        \childdocdisable
        \def\childdocname{#2}
        \childdoctrue
        \includeonly{#2}
        \def\childdocjob{#1}
        \def\jobname{#1}
        \input{#1}
        \endinput
      }
    \fi
    \expandafter
  \endgroup
  \childdoctmp
}
%    \end{macrocode}

% \macro{\childdocforwardprefix}
% The command |\childdocforwardprefix| redirects
% compilation to the main or a child file by means of a pattern.
% The prefix |#1| in the current filename is replaced by |#2|
% and the suffix of the current filename is kept
% (it is assumed that the filename does not contain the substring `|~~~|'
% which is used as a delimiter).
% Compilation is handed over to the new file by |\childdocforward|:
%    \begin{macrocode}
\newcommand{\childdocforwardprefix}[3][]
{
  \begingroup
    \def\childdocextract #2##1~~~{\def\childdoctmp{\childdocforward[#1]{#3##1}}}
    \expandafter\childdocextract\childdocname~~~
    \expandafter
  \endgroup
  \childdoctmp
}
%    \end{macrocode}

% \macro{\childdoc}
% The deprecated macro |\childdoc| is a legacy version of |\childdocmain|:
%    \begin{macrocode}
\newcommand{\childdoc}{\childdocmain}
%    \end{macrocode}

% \macro{\childdocredirect}
% The deprecated macro |\childdocredirect| is a legacy version
% of |\childdocforward| and |\childdocforwardprefix|:
%    \begin{macrocode}
\newcommand{\childdocredirect}[2][]
{
  \begingroup
    \if?#1?
      \def\childdoctmp{\childdocforward{#2}}
    \else
      \def\childdoctmp{\childdocforwardprefix{#1}{#2}}
    \fi
    \expandafter
  \endgroup
  \childdoctmp
}
%    \end{macrocode}

%\iffalse
%</package>
%\fi
%
\endinput
\childdocforward[cdocsamp]{cdocsch1}"|\\
% |latex -jobname cdocscl2 \|\\
% |  "\def\version{final}% \iffalse
%
% childdoc.dtx Copyright (C) 2017-2018 Niklas Beisert
%
% This work may be distributed and/or modified under the
% conditions of the LaTeX Project Public License, either version 1.3
% of this license or (at your option) any later version.
% The latest version of this license is in
%   http://www.latex-project.org/lppl.txt
% and version 1.3 or later is part of all distributions of LaTeX
% version 2005/12/01 or later.
%
% This work has the LPPL maintenance status `maintained'.
%
% The Current Maintainer of this work is Niklas Beisert.
%
% This work consists of the files childdoc.dtx and childdoc.ins
% and the derived files childdoc.def and cdocsamp.tex with
% cdocsch1.tex, cdocsch2.tex, cdocsdrf.tex, cdocsfn1.tex, cdocsfn2.tex.
%
%<package>\ifdefined\childdocmain\endinput\fi
%<package>\ProvidesFile{childdoc.def}[2018/12/30 v2.0 child document driver]
%<samplemain>\ProvidesFile{cdocsamp.tex}[2018/12/30 v2.0 sample for childdoc]
%<*driver>
%\ProvidesFile{childdoc.drv}[2018/12/30 v2.0 childdoc reference manual file]
\PassOptionsToClass{10pt,a4paper}{article}
\documentclass{ltxdoc}

\usepackage[margin=35mm]{geometry}
\usepackage{hyperref}
\usepackage{hyperxmp}
\usepackage[usenames]{color}

\hypersetup{colorlinks=true}
\hypersetup{pdfstartview=FitH}
\hypersetup{pdfpagemode=UseNone}
\hypersetup{pdfsource={}}
\hypersetup{pdflang={en-UK}}
\hypersetup{pdfcopyright={Copyright 2017-2018 Niklas Beisert.
  This work may be distributed and/or modified under the
  conditions of the LaTeX Project Public License, either version 1.3
  of this license or (at your option) any later version.}}
\hypersetup{pdflicenseurl={http://www.latex-project.org/lppl.txt}}
\hypersetup{pdfcontactaddress={ETH Zurich, ITP, HIT K,
  Wolfgang-Pauli-Strasse 27}}
\hypersetup{pdfcontactpostcode={8093}}
\hypersetup{pdfcontactcity={Zurich}}
\hypersetup{pdfcontactcountry={Switzerland}}
\hypersetup{pdfcontactemail={nbeisert@itp.phys.ethz.ch}}
\hypersetup{pdfcontacturl={http://people.phys.ethz.ch/\xmptilde nbeisert/}}

\newcommand{\secref}[1]{\hyperref[#1]{section \ref*{#1}}}

\parskip1ex
\parindent0pt
\let\olditemize\itemize
\def\itemize{\olditemize\parskip0pt}

\begin{document}

\title{The \textsf{childdoc} Package}
\hypersetup{pdftitle={The childdoc Package}}
\author{Niklas Beisert\\[2ex]
  Institut f\"ur Theoretische Physik\\
  Eidgen\"ossische Technische Hochschule Z\"urich\\
  Wolfgang-Pauli-Strasse 27, 8093 Z\"urich, Switzerland\\[1ex]
  \href{mailto:nbeisert@itp.phys.ethz.ch}
  {\texttt{nbeisert@itp.phys.ethz.ch}}}
\hypersetup{pdfauthor={Niklas Beisert}}
\hypersetup{pdfsubject={Manual for the LaTeX2e Package childdoc}}
\date{30 December 2018, \textsf{v2.0}}
\maketitle

\begin{abstract}\noindent
\textsf{childdoc} is a \LaTeXe{} package
that enables the direct compilation
of document sections included by |\include|
to individual files.
\end{abstract}

\begingroup
\parskip0ex
\tableofcontents
\endgroup

%%%%%%%%%%%%%%%%%%%%%%%%%%%%%%%%%%%%%%%%%%%%%%%%%%%%%%%%%%%%%%%%%%%%%%%%%%%%%%%%
%%%%%%%%%%%%%%%%%%%%%%%%%%%%%%%%%%%%%%%%%%%%%%%%%%%%%%%%%%%%%%%%%%%%%%%%%%%%%%%%
\section{Introduction}

\LaTeX{} provides a mechanism to structure a large document (such as a book)
into a main file and several child files (containing the chapters)
using the |\include| command.
This mechanism is beneficial for documents
which span hundreds of pages in order to
make the source file(s) more manageable.
Moreover, compilation can be restricted to
selected child files by means of the |\includeonly| command.
The latter feature can be used to reduce the compilation time while editing
(this was significantly more useful in the earlier days of \LaTeX{})
or to generate a smaller document which is easier to navigate.
Another application of |\includeonly| is to generate
documents consisting of selected parts of the complete document.

However, there are a few drawbacks of the plain |\include| mechanism:
\begin{itemize}
\item
The child files cannot be compiled on their own,
they can only be compiled via the main file.
A naive editing environment
(such as a text editor with an option
to have the current file processed by \LaTeX)
may require one to switch to the main file before compiling;
attempting to compile the child file produces errors.
\item
The main file must be modified (each time)
to adjust the |\includeonly| command
to the present needs. This easily leaves the main file in a messy state.
\item
The generated document will always carry the filename
of the main document. This is inconvenient if
several child files are to be compiled and
to be kept for distribution.
\end{itemize}

The present package provides a simple interface
to make child files individually compilable by \LaTeX{}.
Compiling a child file then has the same effect as compiling
the main file with an |\includeonly| command
to select the appropriate child.
Moreover the generated document will carry the name of the child
rather than the main file.
This resolves all three above issues.

This feature is meant to make the editing of books,
thesis documents and lecture notes somewhat more convenient.
However, the package can also be used efficiently for
composing a series of documents (such as exercise sheets)
which are typically distributed individually.
It then assists the author in generating the individual documents
(potentially in different versions)
as well as a document containing the collected series.
Another application is in developing style files
or other kinds of included material
where compilation of the style file could redirect
to a sample or test file.

%%%%%%%%%%%%%%%%%%%%%%%%%%%%%%%%%%%%%%%%%%%%%%%%%%%%%%%%%%%%%%%%%%%%%%%%%%%%%%%%
%%%%%%%%%%%%%%%%%%%%%%%%%%%%%%%%%%%%%%%%%%%%%%%%%%%%%%%%%%%%%%%%%%%%%%%%%%%%%%%%
\section{Usage}

First of all, the package \textsf{childdoc} is \emph{not} a standard
\LaTeXe{} |.sty| style file! Therefore it needs to be invoked in
a non-standard way.

%%%%%%%%%%%%%%%%%%%%%%%%%%%%%%%%%%%%%%%%%%%%%%%%%%%%%%%%%%%%%%%%%%%%%%%%%%%%%%%%
\subsection{Included Files}
\label{sec:include}

%%%%%%%%%%%%%%%%%%%%%%%%%%%%%%%%%%%%%%%%
\DescribeMacro{\childdocmain}
To use the package, add the commands
\begin{center}
\begin{tabular}{l}
|\input{childdoc.def}|\\
|\childdocmain{}|\\
\end{tabular}
\end{center}
at the very top of the main \LaTeX{} file,
in particular \emph{before} the |\documentclass| statement!
The argument of |\childdocmain| should be left empty
(but it must be present).

%%%%%%%%%%%%%%%%%%%%%%%%%%%%%%%%%%%%%%%%
\DescribeMacro{\childdocof}
Furthermore, add the commands
\begin{center}
\begin{tabular}{l}
|\input{childdoc.def}|\\
|\childdocof{|\textit{main}|}|\\
\end{tabular}
\end{center}
at the top of every child file \textit{child}
which is included by |\include{|\textit{child}|}|
from within the main file
(or at least for those files to be compiled individually).
The argument \textit{main} must be the filename of the main file.

There are a couple of
considerations in setting up the main and child documents:

%%%%%%%%%%%%%%%%%%%%%%%%%%%%%%%%%%%%%%%%
\paragraph{Restrictions.}

Please note the following restrictions:
\begin{itemize}
\item
|\childdocmain| must be called with one argument \textit{main}
to ensure compatibility with earlier version of the package.
It must either be empty (|\childdocmain{}|)
or precisely match the filename of the main file in which it is specified.
See \secref{sec:detection} for further information.
\item
The filename \textit{main} must be specified without the |.tex| extension.
\item
The filename \textit{main} is case sensitive
(even in case-insensitive file systems)
due to internal string comparison.
\item
The argument \textit{main} should be fully expanded, it cannot be a macro.
\item
Subdirectories and special characters should be avoided in filenames.
\item
The command |\childdocmain{|\textit{main}|}| must be followed by a whitespace.
It should not be followed immediately by another command
or by a comment mark `|%|'.
This is because the \TeX{} parser reads the token immediately following
the argument of |\childdocmain| and puts it
at the beginning of every child section;
however, a white\-space is ignored.
\end{itemize}

%%%%%%%%%%%%%%%%%%%%%%%%%%%%%%%%%%%%%%%%
\paragraph{Content of Main File.}

It is advisable to place all content in the child files included by |\include|.
Any output contained in the main file will appear in all child documents
unless suppressed manually;
it cannot be suppressed automatically by the |\includeonly| directive
and thus should normally be avoided.
A method to include some content in the main file
by means of conditional processing is described in \secref{sec:conditional}.

%%%%%%%%%%%%%%%%%%%%%%%%%%%%%%%%%%%%%%%%
\paragraph{Page Numbering.}

When only a part of the document is compiled,
the appropriate numbering of pages
(as well as other status parameters)
is determined from the |.aux| files.
The latter contain information from previous passes.
However this information needs to propagate through
all intermediate child documents.
Therefore the page numbering in child documents may well
be inconsistent until the complete document is compiled at least once.

A useful (if unconventional) way to always ensure a consistent
page numbering is to restart the numbering in each child document
and denote the pages by `\textit{child}|.|\textit{page}'
where \textit{child} represents the chapter/section number of the child file.
This can be achieved by the command
|\numberwithin{page}{|\textit{child}|}|
of the \textsf{amsmath} package
where \textit{child} can be |chapter| or |section|
depending on the chosen structuring.
Alternatively, one can modify the macro |\thepage| appropriately
and reset the counter |page| at the start of each child file.

%%%%%%%%%%%%%%%%%%%%%%%%%%%%%%%%%%%%%%%%%%%%%%%%%%%%%%%%%%%%%%%%%%%%%%%%%%%%%%%%
\subsection{Conditional Processing}
\label{sec:conditional}

The package provides a mechanism to compile different versions
of a document. To customise the versions further some conditional processing
can come in handy to distinguish which version is being compiled.
The package provides two macros to describe the compilation context:

%%%%%%%%%%%%%%%%%%%%%%%%%%%%%%%%%%%%%%%%
\DescribeMacro{\ifchilddoc}
The conditional |\ifchilddoc| distinguishes between the compilation of
child documents and the main document:
%
\begin{center}
|\ifchilddoc |\textit{child-code}| |[|\||else |\textit{main-code}]| \||fi|
\end{center}

%%%%%%%%%%%%%%%%%%%%%%%%%%%%%%%%%%%%%%%%
\DescribeMacro{\childdocname}
\DescribeMacro{\childdocjob}
The macro |\childdocname| contains the filename (without extension)
of the main or child file being processed.
Note that |\childdocjob| will always contain the name of the main file.

%%%%%%%%%%%%%%%%%%%%%%%%%%%%%%%%%%%%%%%%
\paragraph{Title Page.}

Conditional processing can be used to include a title or banner page
in the main document when proper precautions are taken.
Importantly, the code in the main file should ensure that the page counter
(as well as other status parameters which are stored in the |.aux| files)
takes the same value after the conditional processing.
Otherwise the page numbers may take divergent values
depending on which part is compiled.

For example, a title page could be declared by:
%
\begin{center}
\begin{tabular}{l}
|\ifchilddoc\||else|\\
|\addtocounter{page}{-1}|\\
\textit{code for title page}\\
|\newpage|\\
|\||fi|
\end{tabular}
\end{center}
%
A banner page for the child documents can be generated by:
%
\begin{center}
\begin{tabular}{l}
|\ifchilddoc|\\
|\addtocounter{page}{-1}|\\
\textit{code for banner page}\\
|\newpage|\\
|\||fi|
\end{tabular}
\end{center}
%
Here one could write a message such as:
\begin{center}
|This is the part \childdocname{} of \childdocjob{}.|
\end{center}

%%%%%%%%%%%%%%%%%%%%%%%%%%%%%%%%%%%%%%%%%%%%%%%%%%%%%%%%%%%%%%%%%%%%%%%%%%%%%%%%
\subsection{Flags}
\label{sec:flags}

The package makes it easy to generate different versions
of the main or child documents.
To this end compilation flags can be defined
and assigned different default values.
They will be particularly useful in conjunction
with the forwarding mechanism described in \secref{sec:forward}.

For example, it may be useful to have a flag |\version|
which can be set to |draft| or |final|.
The document source will contain some conditional code
depending on the value of |\version|.
Suppose further, the flag should default to |final| for the main file
and to |draft| for child files
which is a natural assignment for editing the document.
This is achieved by placing the following code
in the preamble of the main document
(below the |\childdocmain| directive):
%
\begin{center}
\begin{tabular}{l}
|\ifchilddoc|\\
|\providecommand{\version}{draft}|\\
|\||else|\\
|\providecommand{\version}{final}|\\
|\||fi|
\end{tabular}
\end{center}
%
The definition by |\providecommand| makes sure
that previous definitions are not overwritten.
Further statements |\providecommand{\version}{...}|
can thus be added before the above code to override it.

For the main file, one might add a line
(between |\childdocmain| and the above block)
%
\begin{center}
|%\ifchilddoc\||else\providecommand{\version}{draft}\||fi|
\end{center}
%
which can be uncommented to produce a draft version.
Likewise one can add a line to the very top of a child file
(above the |\childdocof{|\textit{main}|}| directive)
%
\begin{center}
|%\providecommand{\version}{final}|
\end{center}
%
which can be uncommented to produce the final version of this child document.

%%%%%%%%%%%%%%%%%%%%%%%%%%%%%%%%%%%%%%%%%%%%%%%%%%%%%%%%%%%%%%%%%%%%%%%%%%%%%%%%
\subsection{Forwarding}
\label{sec:forward}

Different versions of the main or child documents
using compilation flags as described in \secref{sec:flags}
can be (permanently) stored in different files
for convenient compilation, viewing and distribution.
To this end, the package defines a command
to pass on compilation to a different file:

%%%%%%%%%%%%%%%%%%%%%%%%%%%%%%%%%%%%%%%%
\DescribeMacro{\childdocforward}
The command |\childdocforward| redirects processing to
another source file:
%
\begin{center}
\begin{tabular}{l}
|\input{childdoc.def}|\\
|\childdocforward[|\textit{main}|]{|\textit{dest}|}|\\
\end{tabular}
\end{center}
%
The argument \textit{dest} is the destination file
(without extension).
It should be the main file or one of the child files.
Note that further \textsf{childdoc} directives
such as |\childdocof| and |\childdocforward|
in the indicated file will be processed in this form.
The optional argument \textit{main}
passes on directly to the main file \textit{main}
while pretending to compile the child \textit{dest}.
This form behaves as if \textit{dest}
issues |\childdocof{|\textit{main}|}| right away,
and no further \textsf{childdoc} directives will be processed.

%%%%%%%%%%%%%%%%%%%%%%%%%%%%%%%%%%%%%%%%
\DescribeMacro{\...prefix}
In the alternative form |\childdocforwardprefix|,
%
\begin{center}
\begin{tabular}{l}
|\input{childdoc.def}|\\
|\childdocforwardprefix[|\textit{main}|]{|\textit{prefix}|}{|\textit{dest}|}|
\end{tabular}
\end{center}
%
the destination file is determined by a pattern
depending on the current file:
To make this work, the current file must be called
`{\textit{prefix}\hspace{0.2em}\textit{suffix}}'
with \textit{prefix} matching precisely the argument.
Processing is then passed on to the file
`{\textit{dest}\hspace{0.2em}\textit{suffix}}'.
Surely, the same effect is achieved by
directly specifying the
argument `{\textit{dest}\hspace{0.2em}\textit{suffix}}'
in the first form.
However, that requires to set up a different file
for each child. With the alternative form of the command
all these files can have exactly the same content
which simplifies setting them up and maintaining them.

For example, the following file |draft.tex|
with a compilation flag |\version| as described in \secref{sec:flags}
compiles the main document as a draft:
%
\begin{center}
\begin{tabular}{l}
|\def\version{draft}|\\
|\input{childdoc.def}|\\
|\childdocforward{|\textit{main}|}|
\end{tabular}
\end{center}
%
Likewise, the following files |final|\textit{nn}|.tex|
compile the final version of the child document
|child|\textit{nn}|.tex|:
%
\begin{center}
\begin{tabular}{l}
|\def\version{final}|\\
|\input{childdoc.def}|\\
|\childdocforwardprefix{final}{child}|
\end{tabular}
\end{center}
%

Note that when several versions of a main file and/or of each child file
are to be generated, it may be convenient to set up a |Makefile| or
shell script to automatise the process.

%%%%%%%%%%%%%%%%%%%%%%%%%%%%%%%%%%%%%%%%%%%%%%%%%%%%%%%%%%%%%%%%%%%%%%%%%%%%%%%%
\subsection{Command Line Processing}
\label{sec:commandline}

The effect of redirection files can also be achieved by invoking
the \LaTeX{} compiler with a more elaborate command line.
Most conveniently this should be done as part
of a shell script or a |Makefile|.

When using \textsf{childdoc} in the main file, the following
command lines effectively perform a redirection
(note that depending on the shell being used,
backslashes may have to be doubled: `|\|' $\to$ `|\\|'):
%
\begin{center}
|... -jobname "|\textit{target}|" |\\|"|[\textit{flags}]%
|\input{childdoc.def}\childdocforward[|\textit{main}|]{|\textit{dest}|}"|
\end{center}
%
Here \textit{target} is the name of the output file,
\textit{main} is the name of the main file
and \textit{dest} is the name of the main or child file to be processed
(all filenames without extensions).
The optional argument \textit{main} can be omitted
if \textit{main} matches \textit{dest}.
Optionally, compilation \textit{flags} can be defined via |\def| commands.
This command line makes the \TeX{} engine believe
it is compiling the file \textit{target}
whose content is specified as the latter parameter.
The provided code then forwards the processing to
\textit{main} or \textit{dest} as described in \secref{sec:forward}.

%%%%%%%%%%%%%%%%%%%%%%%%%%%%%%%%%%%%%%%%%%%%%%%%%%%%%%%%%%%%%%%%%%%%%%%%%%%%%%%%
\subsection{Include by Input}
\label{sec:input}

Including child documents by |\include| has some restrictions by design.
Most notably, the content of a child document always occupies
its own set of pages; pages cannot be shared between child documents.
Usually, this behaviour makes perfect sense
because each child document contain an essential part of the document.
However, in some situations it may be desirable to compose
a document from a collection of parts
without having mandatory page breaks between then.
For this case, the package
provides a mechanism to include parts
by |\input| which can also be processed individually.
However, by construction this mechanism
requires manual handling of the content to be output.

%%%%%%%%%%%%%%%%%%%%%%%%%%%%%%%%%%%%%%%%
\DescribeMacro{\ifchilddocmanual}
The main file should be prepared as usual, see \secref{sec:include}.
However, the document body must make a distinction
between processing of an individual part and of the main document, e.g.:
%
\begin{center}
\begin{tabular}{l}
|\ifchilddocmanual|\\
|\input{\childdocname}|\\
|\||else|\\
\textit{document body with }|\input{|\textit{part}|}|\\
|\||fi|
\end{tabular}
\end{center}
%
The conditional |\ifchilddocmanual| is true whenever
a part to be included by |\input| is being compiled,
and the name of the part is stored in |\childdocname|.

%%%%%%%%%%%%%%%%%%%%%%%%%%%%%%%%%%%%%%%%
\DescribeMacro{\childdocby}
Each part to be included by |\input| should start with:
%
\begin{center}
\begin{tabular}{l}
|\input{childdoc.def}|\\
|\childdocby{|\textit{main}|}|\\
\end{tabular}
\end{center}
%
The directive |\childdocby| is similar to |\childdocof|
described in \secref{sec:include},
but the subsequent selection of content must be done manually.
To that end, both |\ifchilddoc| and |\ifchilddocmanual|
will be true upon processing of a part,
and the name of the part is stored in |\childdocname|.
Note that |\jobname| will be set to the filename of the current part
so that each part receives an individual |.aux| file
that does not interfere with the |.aux| file(s) of the main document.
This behaviour can be altered by the alternative form
|\childdocby[*]{|\textit{main}|}| (with a non-empty optional argument)
which uses the |.aux| file of the main document
by setting |\jobname| to \textit{main}.

%%%%%%%%%%%%%%%%%%%%%%%%%%%%%%%%%%%%%%%%%%%%%%%%%%%%%%%%%%%%%%%%%%%%%%%%%%%%%%%%
\subsection{Driver Development}
\label{sec:driver}

The \textsf{childdoc} mechanism can also be use for the development
of definition files such as \LaTeX{} styles or classes.
This case differs from the above setup with multiple parts
included by |\include| in that no |\includeonly| should be invoked.
This can be achieved by starting the include file
(before |\ProvidesPackage|) with:
%
\begin{center}
\begin{tabular}{l}
|\input{childdoc.def}|\\
|\childdocforward{|\textit{main}|}|\\
\end{tabular}
\end{center}
%
or alternatively with:
%
\begin{center}
\begin{tabular}{l}
|\input{childdoc.def}|\\
|\childdocby{|\textit{main}|}|\\
\end{tabular}
\end{center}
%
Both forms have slightly different effects as described above.
The main file is prepared as usual, see \secref{sec:include}.

%%%%%%%%%%%%%%%%%%%%%%%%%%%%%%%%%%%%%%%%%%%%%%%%%%%%%%%%%%%%%%%%%%%%%%%%%%%%%%%%
\subsection{Legacy Detection}
\label{sec:detection}

The directive |\childdocmain| in the main file can detect
whether the complete document or merely a child is to be compiled
even without using the directive |\childdocof|.
This method is deprecated because it is less robust
and there is no compelling reason to use it;
it is merely provided for backward compatibility
and it may be removed in future versions.

If the detection mechanism is to be used,
it is mandatory to correctly specify
the filename of the main file as the argument of |\childdocmain|:
%
\begin{center}
\begin{tabular}{l}
|\input{childdoc.def}|\\
|\childdocmain{|\textit{main}|}|\\
\end{tabular}
\end{center}
%
If |\jobname| does not match the argument \textit{main} of |\childdocmain|,
it is assumed that |\jobname| points to the child file to be compiled.
When using |\childdocmain| with the main file specified as argument,
it suffices to start a child file
with just |\input{|\textit{main}|}|
without loading of the package and using |\childdocof|.
If instead all processing is done
with the appropriate \textsf{childdoc} directives,
the argument of \textit{main} of |\childdocmain| can be empty.

An alternative version of the command line processing described
in \secref{sec:commandline} using the detection mechanism reads:
%
\begin{center}
|... -jobname "|\textit{target}|" "|[\textit{flags}]%
[|\def\jobname{|\textit{dest}|}|]|\input{|\textit{main}|}"|
\end{center}

%%%%%%%%%%%%%%%%%%%%%%%%%%%%%%%%%%%%%%%%%%%%%%%%%%%%%%%%%%%%%%%%%%%%%%%%%%%%%%%%
\subsection{Manual Code}
\label{sec:manual}

In case one cannot be certain whether the definitions file |childdoc.def|
is installed on the target \TeX{} distribution
and one prefers not to ship it,
it is conceivable to paste a few relevant commands into the sources.

To that end, drop all statements |\input{childdoc.def}|
and perform the replacements as outlined below.
Instead of |\childdocmain{|\textit{main}|}| add the following code
to the top of the main file:
%
\begin{center}
\begin{tabular}{l}
|\||ifdefined\childdocname\endinput\||fi\newif\ifchilddoc|\\
|\edef\childdocname{\scantokens\expandafter{\jobname\noexpand}}|\\
|\def\childdocmain{|\textit{main}|}\||ifx\childdocmain\childdocname\||else|\\
|\childdoctrue\includeonly{\childdocname}\let\jobname\childdocmain\||fi|\\
\end{tabular}
\end{center}
%
Instead of |\childdocof{|\textit{main}|}| just include the main file
at the top of each child file:
%
\begin{center}
|\input{|\textit{main}|}|
\end{center}
%
A simple redirection |\childdocforward{|\textit{dest}|}| is achieved by:
%
\begin{center}
|\def\jobname{|\textit{dest}|}\input{\jobname}|
\end{center}
%
The redirection with prefix
|\childdocforwardprefix[|\textit{prefix}|]{|\textit{dest}|}|
is accomplished by:
%
\begin{center}
\begin{tabular}{l}
|{\edef\jobname{\scantokens\expandafter{\jobname\noexpand}}|\\
|\def\redirectjob |\textit{prefix}|#1~~~{\gdef\jobname{|\textit{dest}|#1}}|\\
|\expandafter\redirectjob\jobname~~~}\input{\jobname}|
\end{tabular}
\end{center}

In an alternative approach,
child documents can be compiled by a specific command line
without additional code or specific definitions:
%
\begin{center}
|... -jobname "|\textit{target}|" "|[\textit{flags}]%
|\includeonly{|\textit{dest}|}\input{|\textit{main}|}"|
\end{center}
%

%%%%%%%%%%%%%%%%%%%%%%%%%%%%%%%%%%%%%%%%%%%%%%%%%%%%%%%%%%%%%%%%%%%%%%%%%%%%%%%%
%%%%%%%%%%%%%%%%%%%%%%%%%%%%%%%%%%%%%%%%%%%%%%%%%%%%%%%%%%%%%%%%%%%%%%%%%%%%%%%%
\section{Information}

%%%%%%%%%%%%%%%%%%%%%%%%%%%%%%%%%%%%%%%%%%%%%%%%%%%%%%%%%%%%%%%%%%%%%%%%%%%%%%%%
\subsection{Copyright}

Copyright \copyright{} 2017--2018 Niklas Beisert

This work may be distributed and/or modified under the
conditions of the \LaTeX{} Project Public License, either version 1.3
of this license or (at your option) any later version.
The latest version of this license is in
  \url{http://www.latex-project.org/lppl.txt}
and version 1.3 or later is part of all distributions of \LaTeX{}
version 2005/12/01 or later.

This work has the LPPL maintenance status `maintained'.

The Current Maintainer of this work is Niklas Beisert.

This work consists of the files |README.txt|, |childdoc.ins| and |childdoc.dtx|
as well as the derived files |childdoc.def|, |cdocsamp.tex|
with |cdocsch1.tex|, |cdocsch2.tex|, |cdocspt3.tex|, |cdocspt4.tex|,
|cdocsdrf.tex|, |cdocsfn1.tex|, |cdocsfn2.tex|
as well as |childdoc.pdf|.

%%%%%%%%%%%%%%%%%%%%%%%%%%%%%%%%%%%%%%%%%%%%%%%%%%%%%%%%%%%%%%%%%%%%%%%%%%%%%%%%
\subsection{Files and Installation}

The package consists of the files:
%
\begin{center}
\begin{tabular}{ll}
    |README.txt|   & readme file \\
    |childdoc.ins| & installation file \\
    |childdoc.dtx| & source file \\
    |childdoc.def| & definition file \\
    |cdocsamp.tex| & sample main file \\
    |cdocsch1.tex| & sample include file \\
    |cdocsch2.tex| & sample include file \\
    |cdocspt3.tex| & sample part file \\
    |cdocspt4.tex| & sample part file \\
    |cdocsdrf.tex| & sample redirection file \\
    |cdocsfn1.tex| & sample redirection file \\
    |cdocsfn2.tex| & sample redirection file \\
    |childdoc.pdf| & manual
\end{tabular}
\end{center}
%
The distribution consists of the files
|README.txt|, |childdoc.ins| and |childdoc.dtx|.
%
\begin{itemize}
\item
Run (pdf)\LaTeX{} on |childdoc.dtx|
to compile the manual |childdoc.pdf| (this file).
\item
Run \LaTeX{} on |childdoc.ins| to create the definitions file |childdoc.def|
and the sample |cdocsamp.tex| with include files
|cdocsch1.tex|, |cdocsch2.tex|, |cdocspt3.tex|, |cdocspt4.tex|,
|cdocsdrf.tex|, |cdocsfn1.tex|, |cdocsfn2.tex|.
Then copy the file |childdoc.def| to an appropriate directory of your \LaTeX{}
distribution, e.g.\ \textit{texmf-root}|/tex/latex/childdoc|.
\end{itemize}

%%%%%%%%%%%%%%%%%%%%%%%%%%%%%%%%%%%%%%%%%%%%%%%%%%%%%%%%%%%%%%%%%%%%%%%%%%%%%%%%
\subsection{Related CTAN Packages}

There are several other packages which offer a similar functionality:
%
\begin{itemize}
\item
The packages
\href{http://ctan.org/pkg/docmute}{\textsf{docmute}},
\href{http://ctan.org/pkg/includex}{\textsf{includex}} and
\href{http://ctan.org/pkg/standalone}{\textsf{standalone}}
provide commands to include only the document body of
a child file thus allowing both files to be compiled individually.
\item
The packages \href{http://ctan.org/pkg/subdocs}{\textsf{subdocs}}
and \href{http://ctan.org/pkg/subfiles}{\textsf{subfiles}}
provide structures in which the main and child documents can be
encapsulated and allowing them to be compiled individually.
The inclusion mechanism is different from the conventional |\include|.
\item
The package \href{http://ctan.org/pkg/combine}{\textsf{combine}}
is an elaborate solution to combine several documents into one.
\end{itemize}
%
See also the CTAN topic \href{http://ctan.org/topic/subdocs}{\textsf{subdocs}}
for further related packages.
The present package differs from the above solutions in that
a document structure constructed with the conventional |\include| mechanism
just needs two extra commands at the top of every file
such that all constituent files can be compiled individually.

%%%%%%%%%%%%%%%%%%%%%%%%%%%%%%%%%%%%%%%%%%%%%%%%%%%%%%%%%%%%%%%%%%%%%%%%%%%%%%%%
%\subsection{Feature Suggestions}
%
%The following is a list of features which may be useful for future
%versions of this package:
%%
%\begin{itemize}
%\item
%\ldots
%\end{itemize}

%%%%%%%%%%%%%%%%%%%%%%%%%%%%%%%%%%%%%%%%%%%%%%%%%%%%%%%%%%%%%%%%%%%%%%%%%%%%%%%%
\subsection{Revision History}

%%%%%%%%%%%%%%%%%%%%%%%%%%%%%%%%%%%%%%%%
\paragraph{v2.0:} 2018/12/30

\begin{itemize}
\item
immediate forward processing
\item
added |\childdocby| mechanism
\item
manual restructured
\end{itemize}

%%%%%%%%%%%%%%%%%%%%%%%%%%%%%%%%%%%%%%%%
\paragraph{v1.6:} 2018/01/17

\begin{itemize}
\item
application for development of include files
\item
corrections to manual
\end{itemize}

%%%%%%%%%%%%%%%%%%%%%%%%%%%%%%%%%%%%%%%%
\paragraph{v1.5:} 2017/05/21

\begin{itemize}
\item
more complete structuring introduced
\item
|\childdocof| introduced
\item
|\childdoc| renamed to |\childdocmain|
\item
|\childredirect| renamed to |\childdocforward| and |\childdocforwardprefix|
and functionality expanded
\end{itemize}

%%%%%%%%%%%%%%%%%%%%%%%%%%%%%%%%%%%%%%%%
\paragraph{v1.0:} 2017/04/27

\begin{itemize}
\item
manual and install package
\item
first version published on CTAN
\end{itemize}

%%%%%%%%%%%%%%%%%%%%%%%%%%%%%%%%%%%%%%%%
\paragraph{v0.6:} 2017/04/26

\begin{itemize}
\item
redirection mechanism added
\end{itemize}

%%%%%%%%%%%%%%%%%%%%%%%%%%%%%%%%%%%%%%%%
\paragraph{v0.5:} 2017/04/26

\begin{itemize}
\item
functionality in definition file
\end{itemize}


%%%%%%%%%%%%%%%%%%%%%%%%%%%%%%%%%%%%%%%%%%%%%%%%%%%%%%%%%%%%%%%%%%%%%%%%%%%%%%%%
%%%%%%%%%%%%%%%%%%%%%%%%%%%%%%%%%%%%%%%%%%%%%%%%%%%%%%%%%%%%%%%%%%%%%%%%%%%%%%%%
%%%%%%%%%%%%%%%%%%%%%%%%%%%%%%%%%%%%%%%%%%%%%%%%%%%%%%%%%%%%%%%%%%%%%%%%%%%%%%%%
\appendix

\settowidth\MacroIndent{\rmfamily\scriptsize 000\ }

 \DocInput{childdoc.dtx}

\end{document}
%</driver>
% \fi
%
% %%%%%%%%%%%%%%%%%%%%%%%%%%%%%%%%%%%%%%%%%%%%%%%%%%%%%%%%%%%%%%%%%%%%%%%%%%%%%%
% %%%%%%%%%%%%%%%%%%%%%%%%%%%%%%%%%%%%%%%%%%%%%%%%%%%%%%%%%%%%%%%%%%%%%%%%%%%%%%
% \section{Sample}
%\iffalse
%<*samplemain>
%\fi
%
% The following presents a sample document
% with two chapters, two parts, a title page,
% a compile flag as well as three forwarding files to set the flag.
% It consists of eight |.tex| files:
% \begin{center}
% \begin{tabular}{ll}
% |cdocsamp.tex|&main file\\
% |cdocsch1.tex|&include file for chapter 1\\
% |cdocsch2.tex|&include file for chapter 2\\
% |cdocspt3.tex|&include file for part 3\\
% |cdocspt4.tex|&include file for part 4\\
% |cdocsdrf.tex|&forwarding file for main file in draft mode\\
% |cdocsfi1.tex|&forwarding file for final version of chapter 1\\
% |cdocsfi2.tex|&forwarding file for final version of chapter 2\\
% \end{tabular}
% \end{center}
% Each of the eight files can be compiled directly by the \LaTeX{} compiler.
%
% %%%%%%%%%%%%%%%%%%%%%%%%%%%%%%%%%%%%%%
% \paragraph{Main File.}
%
% The main file is called |cdocsamp.tex|.
%
% Load the \textsf{childdoc} definitions and
% declare the filename for the main document:
%    \begin{macrocode}
\input{childdoc.def}
\childdocmain{}
%    \end{macrocode}

% Optional override for |\version| flag:
%    \begin{macrocode}
%%\ifchilddoc\else\providecommand{\version}{draft}\fi
%    \end{macrocode}

% Define the default values for the |\version| flag
% (|final| for the main file and |draft| for childs):
%    \begin{macrocode}
\ifchilddoc
\providecommand{\version}{draft}
\else
\providecommand{\version}{final}
\fi
%    \end{macrocode}

% Load the standard document class:
%    \begin{macrocode}
\documentclass[12pt]{article}
%    \end{macrocode}

% Start the document body:
%    \begin{macrocode}
\begin{document}
%    \end{macrocode}

% Declare a title page.
% Print title, part of document being processed and version flag:
%    \begin{macrocode}
\addtocounter{page}{-1}
\begin{center}
{\LARGE\bfseries{}childdoc example\par}
\vspace{1cm}
\ifchilddoc
\ifchilddocmanual part\else chapter\fi:
`\childdocname' of `\childdocjob'\par
\else
main document: `\childdocjob'\par
\fi
version: \version\par
\end{center}
\newpage
%    \end{macrocode}

% Manually include selected file,
% otherwise process as usual:
%    \begin{macrocode}
\ifchilddocmanual
\section*{part `\childdocname'}
\input{\childdocname}
\else
%    \end{macrocode}

% Include the two chapters:
%    \begin{macrocode}
\include{cdocsch1}
\include{cdocsch2}
%    \end{macrocode}

% Include the two parts unless only chapters should be displayed:
%    \begin{macrocode}
\ifchilddoc\else
\section{part three}
\input{cdocspt3}
\section{part four}
\input{cdocspt4}
\fi
%    \end{macrocode}

% Process as usual until here:
%    \begin{macrocode}
\fi
%    \end{macrocode}

% End of document body:
%    \begin{macrocode}
\end{document}
%    \end{macrocode}
%\iffalse
%</samplemain>
%\fi
%
% %%%%%%%%%%%%%%%%%%%%%%%%%%%%%%%%%%%%%%
% \paragraph{Chapter Include Files.}
%
% The include files are called |cdocsch1.tex| and |cdocsch2.tex|.
%
%\iffalse
%<*samplechap1|samplechap2>
%\fi

% Optional override for |\version| flag:
%    \begin{macrocode}
%%\providecommand{\version}{final}
%    \end{macrocode}

% Include the main document:
%    \begin{macrocode}
\input{childdoc.def}
\childdocof{cdocsamp}
%    \end{macrocode}

%\iffalse
%</samplechap1|samplechap2>
%\fi
%
%\iffalse
%<*samplechap1>
%\fi
% Some text for chapter 1:
%    \begin{macrocode}
\section{one}
some text in chapter one
%    \end{macrocode}

%\iffalse
%</samplechap1>
%\fi
% Some text for chapter 2:
%\iffalse
%<*samplechap2>
%\fi
%    \begin{macrocode}
\section{two}
more text in chapter two
%    \end{macrocode}

%\iffalse
%</samplechap2>
%\fi
%
% %%%%%%%%%%%%%%%%%%%%%%%%%%%%%%%%%%%%%%
% \paragraph{Part Include Files.}
%
% The include files are called |cdocspt3.tex| and |cdocspt4.tex|.
%
%\iffalse
%<*samplepart3|samplepart4>
%\fi

% Optional override for |\version| flag:
%    \begin{macrocode}
%%\providecommand{\version}{final}
%    \end{macrocode}

% Include the main document:
%    \begin{macrocode}
\input{childdoc.def}
\childdocby{cdocsamp}
%    \end{macrocode}

%\iffalse
%</samplepart3|samplepart4>
%\fi
%
%\iffalse
%<*samplepart3>
%\fi
% Some text for part 3:
%    \begin{macrocode}
some text in part three
%    \end{macrocode}

%\iffalse
%</samplepart3>
%\fi
% Some text for part 4:
%\iffalse
%<*samplepart4>
%\fi
%    \begin{macrocode}
more text in part four
%    \end{macrocode}

%\iffalse
%</samplepart4>
%\fi
%
% %%%%%%%%%%%%%%%%%%%%%%%%%%%%%%%%%%%%%%
% \paragraph{Forwarding for a Complete Draft.}
%
% The following forwarding file |cdocsdrf.tex|
% compiles the main document in draft mode:
%\iffalse
%<*sampledraft>
%\fi
%    \begin{macrocode}
\def\version{draft}
\input{childdoc.def}
\childdocforward{cdocsamp}
%    \end{macrocode}

%\iffalse
%</sampledraft>
%\fi
%
% %%%%%%%%%%%%%%%%%%%%%%%%%%%%%%%%%%%%%%
% \paragraph{Forwarding for Final Version of the Chapters.}
%
% The following forwarding files |cdocsfn1.tex| and |cdocsfn2.tex|
% (with identical content)
% compile the final versions of the child documents
% |cdocsch1.tex| and |cdocsch2.tex|, respectively:
%\iffalse
%<*samplefinal>
%\fi
%    \begin{macrocode}
\def\version{final}
\input{childdoc.def}
\childdocforwardprefix[cdocsamp]{cdocsfn}{cdocsch}
%    \end{macrocode}

%\iffalse
%</samplefinal>
%\fi
%
% %%%%%%%%%%%%%%%%%%%%%%%%%%%%%%%%%%%%%%
% \paragraph{Command Line Processing.}
%
% The following three command lines generate the output files
% |cdocscld|, |cdocscl1| and |cdocscl2|
% which should be identical to
% |cdocsdrf|, |cdocsch1| and |cdocsfn2|, respectively:
% \begin{center}
% \begin{tabular}{l}
% |latex -jobname cdocscld \|\\
% |  "\def\version{draft}\input{childdoc.def}\childdocforward{cdocsamp}"|\\
% |latex -jobname cdocscl1 \|\\
% |  "\input{childdoc.def}\childdocforward[cdocsamp]{cdocsch1}"|\\
% |latex -jobname cdocscl2 \|\\
% |  "\def\version{final}\input{childdoc.def}\childdocforward{cdocsch2}"|
% \end{tabular}
% \end{center}
% Note that the trailing backslash on each first line
% merely continues the input to the second line
% (for convenient cut ant paste).
% Furthermore, the command |latex| can be replaced by any
% of its alternative versions such as |pdflatex|.
%
% %%%%%%%%%%%%%%%%%%%%%%%%%%%%%%%%%%%%%%%%%%%%%%%%%%%%%%%%%%%%%%%%%%%%%%%%%%%%%%
% %%%%%%%%%%%%%%%%%%%%%%%%%%%%%%%%%%%%%%%%%%%%%%%%%%%%%%%%%%%%%%%%%%%%%%%%%%%%%%
% \section{Implementation}
%\iffalse
%<*package>
%\fi
%
% This section describes the definitions file |childdoc.def|.

% The definitions cannot be loaded using |\usepackage| or |\RequirePackage|
% which has a mechanism to prevent loading a style file more than once.
% When loading the definitions by means of |\input|
% multiple instances have to be prevented manually:
%\iffalse
%This code needs to be before the `\ProvidesFile' directive
%which is defined at the beginning of this file.
%Therefore it is also placed there and commented out here.
%</package>
%<*discard>
%\fi
%    \begin{macrocode}
\ifdefined\childdocmain\endinput\fi
%    \end{macrocode}
%\iffalse
%</discard>
%<*package>
%\fi
%
% \macro{\ifchilddoc}
% \macro{\ifchilddocmanual}
% The conditional |\ifchilddoc| tells whether a
% child (true) or main (false) document is being compiled.
% The conditional |\ifchilddocmanual| tells whether
% the |\includeonly| mechanism is used (false) or
% the selection of child files must be performed manually (true).
% The definitions initialise to false:
%    \begin{macrocode}
\newif\ifchilddoc
\newif\ifchilddocmanual
%    \end{macrocode}

% \macro{\childdocname}
% \macro{\childdocjob}
% The macro |\childdocname| stores the name of the main document
% to be compiled. The macro |\childdocjob| stores the name of
% the document on which the \LaTeX{} compiler was originally invoked.
% The content of |\jobname| cannot be compared
% to filenames specified in the source due to different catcodes.
% The following code rescans |\jobname|, stores the result
% in |\childdocname| and saves a copy in |\childdocjob|:
%    \begin{macrocode}
\edef\childdocname{\scantokens\expandafter{\jobname\noexpand}}
\let\childdocjob\childdocname
%    \end{macrocode}

% \macro{\childdocdisable}
% The macro |\childdocdisable| prevents the main file
% from being processed more than once.
% At this stage, the main document command |\childdocmain|
% is assumed to be called once again where it should do nothing.
% Any subsequent call to it should prevent
% a secondary processing of the main document
% It overwrites the forwarding commands
% |\childdocof| and |\childdocforward|
% with empty macros to prevent further inclusions of the main document:
%    \begin{macrocode}
\newcommand{\childdocdisable}
{
  \renewcommand{\childdocmain}[1]{\renewcommand{\childdocmain}[1]{\endinput}}
  \renewcommand{\childdocof}[1]{}
  \renewcommand{\childdocby}[2][]{}
  \renewcommand{\childdocforward}[2][]{}
  \renewcommand{\childdocdisable}{}
}
%    \end{macrocode}

% \macro{\childdocmain}
% The macro |\childdocmain| is to be called at the top of the main file
% with nothing or the main filename (without extension) as argument.
% First, it breaks loops.
% If the argument is not empty and does not match |\childdocname|
% (which is set by the first inclusion of |childdoc.def|),
% |\ifchilddoc| is set to true, |\includeonly| is applied to the child file
% and |\jobname| is set to the main file
% (for proper handling of |.aux| files):
%    \begin{macrocode}
\newcommand{\childdocmain}[1]
{
  \childdocdisable\childdocmain{}
  \if?#1?\else
    \begingroup
      \def\childdoctmp{#1}
      \ifx\childdoctmp\childdocname
        \def\childdoctmp{}
      \else
        \def\childdoctmp
        {
          \childdoctrue
          \includeonly{\childdocname}
          \def\childdocjob{#1}
          \def\jobname{#1}
        }
      \fi
      \expandafter
    \endgroup
    \childdoctmp
  \fi
}
%    \end{macrocode}

% \macro{\childdocof}
% The command |\childdocof| redirects
% compilation to the main file |#1|.
%    \begin{macrocode}
\newcommand{\childdocof}[1]
{
  \childdocdisable
  \childdoctrue
  \includeonly{\childdocname}
  \def\jobname{#1}
  \def\childdocjob{#1}
  \input{#1}
}
%    \end{macrocode}

% \macro{\childdocby}
% The command |\childdocby| ....
%    \begin{macrocode}
\newcommand{\childdocby}[2][]
{
  \childdocdisable
  \childdoctrue
  \childdocmanualtrue
  \if?#1?\else
    \def\jobname{#2}
  \fi
  \def\childdocjob{#2}
  \input{#2}
  \endinput
}
%    \end{macrocode}

% \macro{\childdocforward}
% The command |\childdocforward| redirects
% compilation to the main file or
% (if the optional argument is given) a child file.
% Parameters are set as if the main file
% or a child file starting with |\childdocof| was compiled.
% Then compilation is handed over to the main file:
%    \begin{macrocode}
\newcommand{\childdocforward}[2][]
{
  \begingroup
    \if?#1?
      \def\childdoctmp
      {
        \def\childdocname{#2}
        \def\childdocjob{#2}
        \def\jobname{#2}
        \input{#2}
        \endinput
      }
    \else
      \def\childdoctmp
      {
        \childdocdisable
        \def\childdocname{#2}
        \childdoctrue
        \includeonly{#2}
        \def\childdocjob{#1}
        \def\jobname{#1}
        \input{#1}
        \endinput
      }
    \fi
    \expandafter
  \endgroup
  \childdoctmp
}
%    \end{macrocode}

% \macro{\childdocforwardprefix}
% The command |\childdocforwardprefix| redirects
% compilation to the main or a child file by means of a pattern.
% The prefix |#1| in the current filename is replaced by |#2|
% and the suffix of the current filename is kept
% (it is assumed that the filename does not contain the substring `|~~~|'
% which is used as a delimiter).
% Compilation is handed over to the new file by |\childdocforward|:
%    \begin{macrocode}
\newcommand{\childdocforwardprefix}[3][]
{
  \begingroup
    \def\childdocextract #2##1~~~{\def\childdoctmp{\childdocforward[#1]{#3##1}}}
    \expandafter\childdocextract\childdocname~~~
    \expandafter
  \endgroup
  \childdoctmp
}
%    \end{macrocode}

% \macro{\childdoc}
% The deprecated macro |\childdoc| is a legacy version of |\childdocmain|:
%    \begin{macrocode}
\newcommand{\childdoc}{\childdocmain}
%    \end{macrocode}

% \macro{\childdocredirect}
% The deprecated macro |\childdocredirect| is a legacy version
% of |\childdocforward| and |\childdocforwardprefix|:
%    \begin{macrocode}
\newcommand{\childdocredirect}[2][]
{
  \begingroup
    \if?#1?
      \def\childdoctmp{\childdocforward{#2}}
    \else
      \def\childdoctmp{\childdocforwardprefix{#1}{#2}}
    \fi
    \expandafter
  \endgroup
  \childdoctmp
}
%    \end{macrocode}

%\iffalse
%</package>
%\fi
%
\endinput
\childdocforward{cdocsch2}"|
% \end{tabular}
% \end{center}
% Note that the trailing backslash on each first line
% merely continues the input to the second line
% (for convenient cut ant paste).
% Furthermore, the command |latex| can be replaced by any
% of its alternative versions such as |pdflatex|.
%
% %%%%%%%%%%%%%%%%%%%%%%%%%%%%%%%%%%%%%%%%%%%%%%%%%%%%%%%%%%%%%%%%%%%%%%%%%%%%%%
% %%%%%%%%%%%%%%%%%%%%%%%%%%%%%%%%%%%%%%%%%%%%%%%%%%%%%%%%%%%%%%%%%%%%%%%%%%%%%%
% \section{Implementation}
%\iffalse
%<*package>
%\fi
%
% This section describes the definitions file |childdoc.def|.

% The definitions cannot be loaded using |\usepackage| or |\RequirePackage|
% which has a mechanism to prevent loading a style file more than once.
% When loading the definitions by means of |\input|
% multiple instances have to be prevented manually:
%\iffalse
%This code needs to be before the `\ProvidesFile' directive
%which is defined at the beginning of this file.
%Therefore it is also placed there and commented out here.
%</package>
%<*discard>
%\fi
%    \begin{macrocode}
\ifdefined\childdocmain\endinput\fi
%    \end{macrocode}
%\iffalse
%</discard>
%<*package>
%\fi
%
% \macro{\ifchilddoc}
% \macro{\ifchilddocmanual}
% The conditional |\ifchilddoc| tells whether a
% child (true) or main (false) document is being compiled.
% The conditional |\ifchilddocmanual| tells whether
% the |\includeonly| mechanism is used (false) or
% the selection of child files must be performed manually (true).
% The definitions initialise to false:
%    \begin{macrocode}
\newif\ifchilddoc
\newif\ifchilddocmanual
%    \end{macrocode}

% \macro{\childdocname}
% \macro{\childdocjob}
% The macro |\childdocname| stores the name of the main document
% to be compiled. The macro |\childdocjob| stores the name of
% the document on which the \LaTeX{} compiler was originally invoked.
% The content of |\jobname| cannot be compared
% to filenames specified in the source due to different catcodes.
% The following code rescans |\jobname|, stores the result
% in |\childdocname| and saves a copy in |\childdocjob|:
%    \begin{macrocode}
\edef\childdocname{\scantokens\expandafter{\jobname\noexpand}}
\let\childdocjob\childdocname
%    \end{macrocode}

% \macro{\childdocdisable}
% The macro |\childdocdisable| prevents the main file
% from being processed more than once.
% At this stage, the main document command |\childdocmain|
% is assumed to be called once again where it should do nothing.
% Any subsequent call to it should prevent
% a secondary processing of the main document
% It overwrites the forwarding commands
% |\childdocof| and |\childdocforward|
% with empty macros to prevent further inclusions of the main document:
%    \begin{macrocode}
\newcommand{\childdocdisable}
{
  \renewcommand{\childdocmain}[1]{\renewcommand{\childdocmain}[1]{\endinput}}
  \renewcommand{\childdocof}[1]{}
  \renewcommand{\childdocby}[2][]{}
  \renewcommand{\childdocforward}[2][]{}
  \renewcommand{\childdocdisable}{}
}
%    \end{macrocode}

% \macro{\childdocmain}
% The macro |\childdocmain| is to be called at the top of the main file
% with nothing or the main filename (without extension) as argument.
% First, it breaks loops.
% If the argument is not empty and does not match |\childdocname|
% (which is set by the first inclusion of |childdoc.def|),
% |\ifchilddoc| is set to true, |\includeonly| is applied to the child file
% and |\jobname| is set to the main file
% (for proper handling of |.aux| files):
%    \begin{macrocode}
\newcommand{\childdocmain}[1]
{
  \childdocdisable\childdocmain{}
  \if?#1?\else
    \begingroup
      \def\childdoctmp{#1}
      \ifx\childdoctmp\childdocname
        \def\childdoctmp{}
      \else
        \def\childdoctmp
        {
          \childdoctrue
          \includeonly{\childdocname}
          \def\childdocjob{#1}
          \def\jobname{#1}
        }
      \fi
      \expandafter
    \endgroup
    \childdoctmp
  \fi
}
%    \end{macrocode}

% \macro{\childdocof}
% The command |\childdocof| redirects
% compilation to the main file |#1|.
%    \begin{macrocode}
\newcommand{\childdocof}[1]
{
  \childdocdisable
  \childdoctrue
  \includeonly{\childdocname}
  \def\jobname{#1}
  \def\childdocjob{#1}
  \input{#1}
}
%    \end{macrocode}

% \macro{\childdocby}
% The command |\childdocby| ....
%    \begin{macrocode}
\newcommand{\childdocby}[2][]
{
  \childdocdisable
  \childdoctrue
  \childdocmanualtrue
  \if?#1?\else
    \def\jobname{#2}
  \fi
  \def\childdocjob{#2}
  \input{#2}
  \endinput
}
%    \end{macrocode}

% \macro{\childdocforward}
% The command |\childdocforward| redirects
% compilation to the main file or
% (if the optional argument is given) a child file.
% Parameters are set as if the main file
% or a child file starting with |\childdocof| was compiled.
% Then compilation is handed over to the main file:
%    \begin{macrocode}
\newcommand{\childdocforward}[2][]
{
  \begingroup
    \if?#1?
      \def\childdoctmp
      {
        \def\childdocname{#2}
        \def\childdocjob{#2}
        \def\jobname{#2}
        \input{#2}
        \endinput
      }
    \else
      \def\childdoctmp
      {
        \childdocdisable
        \def\childdocname{#2}
        \childdoctrue
        \includeonly{#2}
        \def\childdocjob{#1}
        \def\jobname{#1}
        \input{#1}
        \endinput
      }
    \fi
    \expandafter
  \endgroup
  \childdoctmp
}
%    \end{macrocode}

% \macro{\childdocforwardprefix}
% The command |\childdocforwardprefix| redirects
% compilation to the main or a child file by means of a pattern.
% The prefix |#1| in the current filename is replaced by |#2|
% and the suffix of the current filename is kept
% (it is assumed that the filename does not contain the substring `|~~~|'
% which is used as a delimiter).
% Compilation is handed over to the new file by |\childdocforward|:
%    \begin{macrocode}
\newcommand{\childdocforwardprefix}[3][]
{
  \begingroup
    \def\childdocextract #2##1~~~{\def\childdoctmp{\childdocforward[#1]{#3##1}}}
    \expandafter\childdocextract\childdocname~~~
    \expandafter
  \endgroup
  \childdoctmp
}
%    \end{macrocode}

% \macro{\childdoc}
% The deprecated macro |\childdoc| is a legacy version of |\childdocmain|:
%    \begin{macrocode}
\newcommand{\childdoc}{\childdocmain}
%    \end{macrocode}

% \macro{\childdocredirect}
% The deprecated macro |\childdocredirect| is a legacy version
% of |\childdocforward| and |\childdocforwardprefix|:
%    \begin{macrocode}
\newcommand{\childdocredirect}[2][]
{
  \begingroup
    \if?#1?
      \def\childdoctmp{\childdocforward{#2}}
    \else
      \def\childdoctmp{\childdocforwardprefix{#1}{#2}}
    \fi
    \expandafter
  \endgroup
  \childdoctmp
}
%    \end{macrocode}

%\iffalse
%</package>
%\fi
%
\endinput
|\\
|\childdocmain{|\textit{main}|}|\\
\end{tabular}
\end{center}
%
If |\jobname| does not match the argument \textit{main} of |\childdocmain|,
it is assumed that |\jobname| points to the child file to be compiled.
When using |\childdocmain| with the main file specified as argument,
it suffices to start a child file
with just |\input{|\textit{main}|}|
without loading of the package and using |\childdocof|.
If instead all processing is done
with the appropriate \textsf{childdoc} directives,
the argument of \textit{main} of |\childdocmain| can be empty.

An alternative version of the command line processing described
in \secref{sec:commandline} using the detection mechanism reads:
%
\begin{center}
|... -jobname "|\textit{target}|" "|[\textit{flags}]%
[|\def\jobname{|\textit{dest}|}|]|\input{|\textit{main}|}"|
\end{center}

%%%%%%%%%%%%%%%%%%%%%%%%%%%%%%%%%%%%%%%%%%%%%%%%%%%%%%%%%%%%%%%%%%%%%%%%%%%%%%%%
\subsection{Manual Code}
\label{sec:manual}

In case one cannot be certain whether the definitions file |childdoc.def|
is installed on the target \TeX{} distribution
and one prefers not to ship it,
it is conceivable to paste a few relevant commands into the sources.

To that end, drop all statements |% \iffalse
%
% childdoc.dtx Copyright (C) 2017-2018 Niklas Beisert
%
% This work may be distributed and/or modified under the
% conditions of the LaTeX Project Public License, either version 1.3
% of this license or (at your option) any later version.
% The latest version of this license is in
%   http://www.latex-project.org/lppl.txt
% and version 1.3 or later is part of all distributions of LaTeX
% version 2005/12/01 or later.
%
% This work has the LPPL maintenance status `maintained'.
%
% The Current Maintainer of this work is Niklas Beisert.
%
% This work consists of the files childdoc.dtx and childdoc.ins
% and the derived files childdoc.def and cdocsamp.tex with
% cdocsch1.tex, cdocsch2.tex, cdocsdrf.tex, cdocsfn1.tex, cdocsfn2.tex.
%
%<package>\ifdefined\childdocmain\endinput\fi
%<package>\ProvidesFile{childdoc.def}[2018/12/30 v2.0 child document driver]
%<samplemain>\ProvidesFile{cdocsamp.tex}[2018/12/30 v2.0 sample for childdoc]
%<*driver>
%\ProvidesFile{childdoc.drv}[2018/12/30 v2.0 childdoc reference manual file]
\PassOptionsToClass{10pt,a4paper}{article}
\documentclass{ltxdoc}

\usepackage[margin=35mm]{geometry}
\usepackage{hyperref}
\usepackage{hyperxmp}
\usepackage[usenames]{color}

\hypersetup{colorlinks=true}
\hypersetup{pdfstartview=FitH}
\hypersetup{pdfpagemode=UseNone}
\hypersetup{pdfsource={}}
\hypersetup{pdflang={en-UK}}
\hypersetup{pdfcopyright={Copyright 2017-2018 Niklas Beisert.
  This work may be distributed and/or modified under the
  conditions of the LaTeX Project Public License, either version 1.3
  of this license or (at your option) any later version.}}
\hypersetup{pdflicenseurl={http://www.latex-project.org/lppl.txt}}
\hypersetup{pdfcontactaddress={ETH Zurich, ITP, HIT K,
  Wolfgang-Pauli-Strasse 27}}
\hypersetup{pdfcontactpostcode={8093}}
\hypersetup{pdfcontactcity={Zurich}}
\hypersetup{pdfcontactcountry={Switzerland}}
\hypersetup{pdfcontactemail={nbeisert@itp.phys.ethz.ch}}
\hypersetup{pdfcontacturl={http://people.phys.ethz.ch/\xmptilde nbeisert/}}

\newcommand{\secref}[1]{\hyperref[#1]{section \ref*{#1}}}

\parskip1ex
\parindent0pt
\let\olditemize\itemize
\def\itemize{\olditemize\parskip0pt}

\begin{document}

\title{The \textsf{childdoc} Package}
\hypersetup{pdftitle={The childdoc Package}}
\author{Niklas Beisert\\[2ex]
  Institut f\"ur Theoretische Physik\\
  Eidgen\"ossische Technische Hochschule Z\"urich\\
  Wolfgang-Pauli-Strasse 27, 8093 Z\"urich, Switzerland\\[1ex]
  \href{mailto:nbeisert@itp.phys.ethz.ch}
  {\texttt{nbeisert@itp.phys.ethz.ch}}}
\hypersetup{pdfauthor={Niklas Beisert}}
\hypersetup{pdfsubject={Manual for the LaTeX2e Package childdoc}}
\date{30 December 2018, \textsf{v2.0}}
\maketitle

\begin{abstract}\noindent
\textsf{childdoc} is a \LaTeXe{} package
that enables the direct compilation
of document sections included by |\include|
to individual files.
\end{abstract}

\begingroup
\parskip0ex
\tableofcontents
\endgroup

%%%%%%%%%%%%%%%%%%%%%%%%%%%%%%%%%%%%%%%%%%%%%%%%%%%%%%%%%%%%%%%%%%%%%%%%%%%%%%%%
%%%%%%%%%%%%%%%%%%%%%%%%%%%%%%%%%%%%%%%%%%%%%%%%%%%%%%%%%%%%%%%%%%%%%%%%%%%%%%%%
\section{Introduction}

\LaTeX{} provides a mechanism to structure a large document (such as a book)
into a main file and several child files (containing the chapters)
using the |\include| command.
This mechanism is beneficial for documents
which span hundreds of pages in order to
make the source file(s) more manageable.
Moreover, compilation can be restricted to
selected child files by means of the |\includeonly| command.
The latter feature can be used to reduce the compilation time while editing
(this was significantly more useful in the earlier days of \LaTeX{})
or to generate a smaller document which is easier to navigate.
Another application of |\includeonly| is to generate
documents consisting of selected parts of the complete document.

However, there are a few drawbacks of the plain |\include| mechanism:
\begin{itemize}
\item
The child files cannot be compiled on their own,
they can only be compiled via the main file.
A naive editing environment
(such as a text editor with an option
to have the current file processed by \LaTeX)
may require one to switch to the main file before compiling;
attempting to compile the child file produces errors.
\item
The main file must be modified (each time)
to adjust the |\includeonly| command
to the present needs. This easily leaves the main file in a messy state.
\item
The generated document will always carry the filename
of the main document. This is inconvenient if
several child files are to be compiled and
to be kept for distribution.
\end{itemize}

The present package provides a simple interface
to make child files individually compilable by \LaTeX{}.
Compiling a child file then has the same effect as compiling
the main file with an |\includeonly| command
to select the appropriate child.
Moreover the generated document will carry the name of the child
rather than the main file.
This resolves all three above issues.

This feature is meant to make the editing of books,
thesis documents and lecture notes somewhat more convenient.
However, the package can also be used efficiently for
composing a series of documents (such as exercise sheets)
which are typically distributed individually.
It then assists the author in generating the individual documents
(potentially in different versions)
as well as a document containing the collected series.
Another application is in developing style files
or other kinds of included material
where compilation of the style file could redirect
to a sample or test file.

%%%%%%%%%%%%%%%%%%%%%%%%%%%%%%%%%%%%%%%%%%%%%%%%%%%%%%%%%%%%%%%%%%%%%%%%%%%%%%%%
%%%%%%%%%%%%%%%%%%%%%%%%%%%%%%%%%%%%%%%%%%%%%%%%%%%%%%%%%%%%%%%%%%%%%%%%%%%%%%%%
\section{Usage}

First of all, the package \textsf{childdoc} is \emph{not} a standard
\LaTeXe{} |.sty| style file! Therefore it needs to be invoked in
a non-standard way.

%%%%%%%%%%%%%%%%%%%%%%%%%%%%%%%%%%%%%%%%%%%%%%%%%%%%%%%%%%%%%%%%%%%%%%%%%%%%%%%%
\subsection{Included Files}
\label{sec:include}

%%%%%%%%%%%%%%%%%%%%%%%%%%%%%%%%%%%%%%%%
\DescribeMacro{\childdocmain}
To use the package, add the commands
\begin{center}
\begin{tabular}{l}
|% \iffalse
%
% childdoc.dtx Copyright (C) 2017-2018 Niklas Beisert
%
% This work may be distributed and/or modified under the
% conditions of the LaTeX Project Public License, either version 1.3
% of this license or (at your option) any later version.
% The latest version of this license is in
%   http://www.latex-project.org/lppl.txt
% and version 1.3 or later is part of all distributions of LaTeX
% version 2005/12/01 or later.
%
% This work has the LPPL maintenance status `maintained'.
%
% The Current Maintainer of this work is Niklas Beisert.
%
% This work consists of the files childdoc.dtx and childdoc.ins
% and the derived files childdoc.def and cdocsamp.tex with
% cdocsch1.tex, cdocsch2.tex, cdocsdrf.tex, cdocsfn1.tex, cdocsfn2.tex.
%
%<package>\ifdefined\childdocmain\endinput\fi
%<package>\ProvidesFile{childdoc.def}[2018/12/30 v2.0 child document driver]
%<samplemain>\ProvidesFile{cdocsamp.tex}[2018/12/30 v2.0 sample for childdoc]
%<*driver>
%\ProvidesFile{childdoc.drv}[2018/12/30 v2.0 childdoc reference manual file]
\PassOptionsToClass{10pt,a4paper}{article}
\documentclass{ltxdoc}

\usepackage[margin=35mm]{geometry}
\usepackage{hyperref}
\usepackage{hyperxmp}
\usepackage[usenames]{color}

\hypersetup{colorlinks=true}
\hypersetup{pdfstartview=FitH}
\hypersetup{pdfpagemode=UseNone}
\hypersetup{pdfsource={}}
\hypersetup{pdflang={en-UK}}
\hypersetup{pdfcopyright={Copyright 2017-2018 Niklas Beisert.
  This work may be distributed and/or modified under the
  conditions of the LaTeX Project Public License, either version 1.3
  of this license or (at your option) any later version.}}
\hypersetup{pdflicenseurl={http://www.latex-project.org/lppl.txt}}
\hypersetup{pdfcontactaddress={ETH Zurich, ITP, HIT K,
  Wolfgang-Pauli-Strasse 27}}
\hypersetup{pdfcontactpostcode={8093}}
\hypersetup{pdfcontactcity={Zurich}}
\hypersetup{pdfcontactcountry={Switzerland}}
\hypersetup{pdfcontactemail={nbeisert@itp.phys.ethz.ch}}
\hypersetup{pdfcontacturl={http://people.phys.ethz.ch/\xmptilde nbeisert/}}

\newcommand{\secref}[1]{\hyperref[#1]{section \ref*{#1}}}

\parskip1ex
\parindent0pt
\let\olditemize\itemize
\def\itemize{\olditemize\parskip0pt}

\begin{document}

\title{The \textsf{childdoc} Package}
\hypersetup{pdftitle={The childdoc Package}}
\author{Niklas Beisert\\[2ex]
  Institut f\"ur Theoretische Physik\\
  Eidgen\"ossische Technische Hochschule Z\"urich\\
  Wolfgang-Pauli-Strasse 27, 8093 Z\"urich, Switzerland\\[1ex]
  \href{mailto:nbeisert@itp.phys.ethz.ch}
  {\texttt{nbeisert@itp.phys.ethz.ch}}}
\hypersetup{pdfauthor={Niklas Beisert}}
\hypersetup{pdfsubject={Manual for the LaTeX2e Package childdoc}}
\date{30 December 2018, \textsf{v2.0}}
\maketitle

\begin{abstract}\noindent
\textsf{childdoc} is a \LaTeXe{} package
that enables the direct compilation
of document sections included by |\include|
to individual files.
\end{abstract}

\begingroup
\parskip0ex
\tableofcontents
\endgroup

%%%%%%%%%%%%%%%%%%%%%%%%%%%%%%%%%%%%%%%%%%%%%%%%%%%%%%%%%%%%%%%%%%%%%%%%%%%%%%%%
%%%%%%%%%%%%%%%%%%%%%%%%%%%%%%%%%%%%%%%%%%%%%%%%%%%%%%%%%%%%%%%%%%%%%%%%%%%%%%%%
\section{Introduction}

\LaTeX{} provides a mechanism to structure a large document (such as a book)
into a main file and several child files (containing the chapters)
using the |\include| command.
This mechanism is beneficial for documents
which span hundreds of pages in order to
make the source file(s) more manageable.
Moreover, compilation can be restricted to
selected child files by means of the |\includeonly| command.
The latter feature can be used to reduce the compilation time while editing
(this was significantly more useful in the earlier days of \LaTeX{})
or to generate a smaller document which is easier to navigate.
Another application of |\includeonly| is to generate
documents consisting of selected parts of the complete document.

However, there are a few drawbacks of the plain |\include| mechanism:
\begin{itemize}
\item
The child files cannot be compiled on their own,
they can only be compiled via the main file.
A naive editing environment
(such as a text editor with an option
to have the current file processed by \LaTeX)
may require one to switch to the main file before compiling;
attempting to compile the child file produces errors.
\item
The main file must be modified (each time)
to adjust the |\includeonly| command
to the present needs. This easily leaves the main file in a messy state.
\item
The generated document will always carry the filename
of the main document. This is inconvenient if
several child files are to be compiled and
to be kept for distribution.
\end{itemize}

The present package provides a simple interface
to make child files individually compilable by \LaTeX{}.
Compiling a child file then has the same effect as compiling
the main file with an |\includeonly| command
to select the appropriate child.
Moreover the generated document will carry the name of the child
rather than the main file.
This resolves all three above issues.

This feature is meant to make the editing of books,
thesis documents and lecture notes somewhat more convenient.
However, the package can also be used efficiently for
composing a series of documents (such as exercise sheets)
which are typically distributed individually.
It then assists the author in generating the individual documents
(potentially in different versions)
as well as a document containing the collected series.
Another application is in developing style files
or other kinds of included material
where compilation of the style file could redirect
to a sample or test file.

%%%%%%%%%%%%%%%%%%%%%%%%%%%%%%%%%%%%%%%%%%%%%%%%%%%%%%%%%%%%%%%%%%%%%%%%%%%%%%%%
%%%%%%%%%%%%%%%%%%%%%%%%%%%%%%%%%%%%%%%%%%%%%%%%%%%%%%%%%%%%%%%%%%%%%%%%%%%%%%%%
\section{Usage}

First of all, the package \textsf{childdoc} is \emph{not} a standard
\LaTeXe{} |.sty| style file! Therefore it needs to be invoked in
a non-standard way.

%%%%%%%%%%%%%%%%%%%%%%%%%%%%%%%%%%%%%%%%%%%%%%%%%%%%%%%%%%%%%%%%%%%%%%%%%%%%%%%%
\subsection{Included Files}
\label{sec:include}

%%%%%%%%%%%%%%%%%%%%%%%%%%%%%%%%%%%%%%%%
\DescribeMacro{\childdocmain}
To use the package, add the commands
\begin{center}
\begin{tabular}{l}
|\input{childdoc.def}|\\
|\childdocmain{}|\\
\end{tabular}
\end{center}
at the very top of the main \LaTeX{} file,
in particular \emph{before} the |\documentclass| statement!
The argument of |\childdocmain| should be left empty
(but it must be present).

%%%%%%%%%%%%%%%%%%%%%%%%%%%%%%%%%%%%%%%%
\DescribeMacro{\childdocof}
Furthermore, add the commands
\begin{center}
\begin{tabular}{l}
|\input{childdoc.def}|\\
|\childdocof{|\textit{main}|}|\\
\end{tabular}
\end{center}
at the top of every child file \textit{child}
which is included by |\include{|\textit{child}|}|
from within the main file
(or at least for those files to be compiled individually).
The argument \textit{main} must be the filename of the main file.

There are a couple of
considerations in setting up the main and child documents:

%%%%%%%%%%%%%%%%%%%%%%%%%%%%%%%%%%%%%%%%
\paragraph{Restrictions.}

Please note the following restrictions:
\begin{itemize}
\item
|\childdocmain| must be called with one argument \textit{main}
to ensure compatibility with earlier version of the package.
It must either be empty (|\childdocmain{}|)
or precisely match the filename of the main file in which it is specified.
See \secref{sec:detection} for further information.
\item
The filename \textit{main} must be specified without the |.tex| extension.
\item
The filename \textit{main} is case sensitive
(even in case-insensitive file systems)
due to internal string comparison.
\item
The argument \textit{main} should be fully expanded, it cannot be a macro.
\item
Subdirectories and special characters should be avoided in filenames.
\item
The command |\childdocmain{|\textit{main}|}| must be followed by a whitespace.
It should not be followed immediately by another command
or by a comment mark `|%|'.
This is because the \TeX{} parser reads the token immediately following
the argument of |\childdocmain| and puts it
at the beginning of every child section;
however, a white\-space is ignored.
\end{itemize}

%%%%%%%%%%%%%%%%%%%%%%%%%%%%%%%%%%%%%%%%
\paragraph{Content of Main File.}

It is advisable to place all content in the child files included by |\include|.
Any output contained in the main file will appear in all child documents
unless suppressed manually;
it cannot be suppressed automatically by the |\includeonly| directive
and thus should normally be avoided.
A method to include some content in the main file
by means of conditional processing is described in \secref{sec:conditional}.

%%%%%%%%%%%%%%%%%%%%%%%%%%%%%%%%%%%%%%%%
\paragraph{Page Numbering.}

When only a part of the document is compiled,
the appropriate numbering of pages
(as well as other status parameters)
is determined from the |.aux| files.
The latter contain information from previous passes.
However this information needs to propagate through
all intermediate child documents.
Therefore the page numbering in child documents may well
be inconsistent until the complete document is compiled at least once.

A useful (if unconventional) way to always ensure a consistent
page numbering is to restart the numbering in each child document
and denote the pages by `\textit{child}|.|\textit{page}'
where \textit{child} represents the chapter/section number of the child file.
This can be achieved by the command
|\numberwithin{page}{|\textit{child}|}|
of the \textsf{amsmath} package
where \textit{child} can be |chapter| or |section|
depending on the chosen structuring.
Alternatively, one can modify the macro |\thepage| appropriately
and reset the counter |page| at the start of each child file.

%%%%%%%%%%%%%%%%%%%%%%%%%%%%%%%%%%%%%%%%%%%%%%%%%%%%%%%%%%%%%%%%%%%%%%%%%%%%%%%%
\subsection{Conditional Processing}
\label{sec:conditional}

The package provides a mechanism to compile different versions
of a document. To customise the versions further some conditional processing
can come in handy to distinguish which version is being compiled.
The package provides two macros to describe the compilation context:

%%%%%%%%%%%%%%%%%%%%%%%%%%%%%%%%%%%%%%%%
\DescribeMacro{\ifchilddoc}
The conditional |\ifchilddoc| distinguishes between the compilation of
child documents and the main document:
%
\begin{center}
|\ifchilddoc |\textit{child-code}| |[|\||else |\textit{main-code}]| \||fi|
\end{center}

%%%%%%%%%%%%%%%%%%%%%%%%%%%%%%%%%%%%%%%%
\DescribeMacro{\childdocname}
\DescribeMacro{\childdocjob}
The macro |\childdocname| contains the filename (without extension)
of the main or child file being processed.
Note that |\childdocjob| will always contain the name of the main file.

%%%%%%%%%%%%%%%%%%%%%%%%%%%%%%%%%%%%%%%%
\paragraph{Title Page.}

Conditional processing can be used to include a title or banner page
in the main document when proper precautions are taken.
Importantly, the code in the main file should ensure that the page counter
(as well as other status parameters which are stored in the |.aux| files)
takes the same value after the conditional processing.
Otherwise the page numbers may take divergent values
depending on which part is compiled.

For example, a title page could be declared by:
%
\begin{center}
\begin{tabular}{l}
|\ifchilddoc\||else|\\
|\addtocounter{page}{-1}|\\
\textit{code for title page}\\
|\newpage|\\
|\||fi|
\end{tabular}
\end{center}
%
A banner page for the child documents can be generated by:
%
\begin{center}
\begin{tabular}{l}
|\ifchilddoc|\\
|\addtocounter{page}{-1}|\\
\textit{code for banner page}\\
|\newpage|\\
|\||fi|
\end{tabular}
\end{center}
%
Here one could write a message such as:
\begin{center}
|This is the part \childdocname{} of \childdocjob{}.|
\end{center}

%%%%%%%%%%%%%%%%%%%%%%%%%%%%%%%%%%%%%%%%%%%%%%%%%%%%%%%%%%%%%%%%%%%%%%%%%%%%%%%%
\subsection{Flags}
\label{sec:flags}

The package makes it easy to generate different versions
of the main or child documents.
To this end compilation flags can be defined
and assigned different default values.
They will be particularly useful in conjunction
with the forwarding mechanism described in \secref{sec:forward}.

For example, it may be useful to have a flag |\version|
which can be set to |draft| or |final|.
The document source will contain some conditional code
depending on the value of |\version|.
Suppose further, the flag should default to |final| for the main file
and to |draft| for child files
which is a natural assignment for editing the document.
This is achieved by placing the following code
in the preamble of the main document
(below the |\childdocmain| directive):
%
\begin{center}
\begin{tabular}{l}
|\ifchilddoc|\\
|\providecommand{\version}{draft}|\\
|\||else|\\
|\providecommand{\version}{final}|\\
|\||fi|
\end{tabular}
\end{center}
%
The definition by |\providecommand| makes sure
that previous definitions are not overwritten.
Further statements |\providecommand{\version}{...}|
can thus be added before the above code to override it.

For the main file, one might add a line
(between |\childdocmain| and the above block)
%
\begin{center}
|%\ifchilddoc\||else\providecommand{\version}{draft}\||fi|
\end{center}
%
which can be uncommented to produce a draft version.
Likewise one can add a line to the very top of a child file
(above the |\childdocof{|\textit{main}|}| directive)
%
\begin{center}
|%\providecommand{\version}{final}|
\end{center}
%
which can be uncommented to produce the final version of this child document.

%%%%%%%%%%%%%%%%%%%%%%%%%%%%%%%%%%%%%%%%%%%%%%%%%%%%%%%%%%%%%%%%%%%%%%%%%%%%%%%%
\subsection{Forwarding}
\label{sec:forward}

Different versions of the main or child documents
using compilation flags as described in \secref{sec:flags}
can be (permanently) stored in different files
for convenient compilation, viewing and distribution.
To this end, the package defines a command
to pass on compilation to a different file:

%%%%%%%%%%%%%%%%%%%%%%%%%%%%%%%%%%%%%%%%
\DescribeMacro{\childdocforward}
The command |\childdocforward| redirects processing to
another source file:
%
\begin{center}
\begin{tabular}{l}
|\input{childdoc.def}|\\
|\childdocforward[|\textit{main}|]{|\textit{dest}|}|\\
\end{tabular}
\end{center}
%
The argument \textit{dest} is the destination file
(without extension).
It should be the main file or one of the child files.
Note that further \textsf{childdoc} directives
such as |\childdocof| and |\childdocforward|
in the indicated file will be processed in this form.
The optional argument \textit{main}
passes on directly to the main file \textit{main}
while pretending to compile the child \textit{dest}.
This form behaves as if \textit{dest}
issues |\childdocof{|\textit{main}|}| right away,
and no further \textsf{childdoc} directives will be processed.

%%%%%%%%%%%%%%%%%%%%%%%%%%%%%%%%%%%%%%%%
\DescribeMacro{\...prefix}
In the alternative form |\childdocforwardprefix|,
%
\begin{center}
\begin{tabular}{l}
|\input{childdoc.def}|\\
|\childdocforwardprefix[|\textit{main}|]{|\textit{prefix}|}{|\textit{dest}|}|
\end{tabular}
\end{center}
%
the destination file is determined by a pattern
depending on the current file:
To make this work, the current file must be called
`{\textit{prefix}\hspace{0.2em}\textit{suffix}}'
with \textit{prefix} matching precisely the argument.
Processing is then passed on to the file
`{\textit{dest}\hspace{0.2em}\textit{suffix}}'.
Surely, the same effect is achieved by
directly specifying the
argument `{\textit{dest}\hspace{0.2em}\textit{suffix}}'
in the first form.
However, that requires to set up a different file
for each child. With the alternative form of the command
all these files can have exactly the same content
which simplifies setting them up and maintaining them.

For example, the following file |draft.tex|
with a compilation flag |\version| as described in \secref{sec:flags}
compiles the main document as a draft:
%
\begin{center}
\begin{tabular}{l}
|\def\version{draft}|\\
|\input{childdoc.def}|\\
|\childdocforward{|\textit{main}|}|
\end{tabular}
\end{center}
%
Likewise, the following files |final|\textit{nn}|.tex|
compile the final version of the child document
|child|\textit{nn}|.tex|:
%
\begin{center}
\begin{tabular}{l}
|\def\version{final}|\\
|\input{childdoc.def}|\\
|\childdocforwardprefix{final}{child}|
\end{tabular}
\end{center}
%

Note that when several versions of a main file and/or of each child file
are to be generated, it may be convenient to set up a |Makefile| or
shell script to automatise the process.

%%%%%%%%%%%%%%%%%%%%%%%%%%%%%%%%%%%%%%%%%%%%%%%%%%%%%%%%%%%%%%%%%%%%%%%%%%%%%%%%
\subsection{Command Line Processing}
\label{sec:commandline}

The effect of redirection files can also be achieved by invoking
the \LaTeX{} compiler with a more elaborate command line.
Most conveniently this should be done as part
of a shell script or a |Makefile|.

When using \textsf{childdoc} in the main file, the following
command lines effectively perform a redirection
(note that depending on the shell being used,
backslashes may have to be doubled: `|\|' $\to$ `|\\|'):
%
\begin{center}
|... -jobname "|\textit{target}|" |\\|"|[\textit{flags}]%
|\input{childdoc.def}\childdocforward[|\textit{main}|]{|\textit{dest}|}"|
\end{center}
%
Here \textit{target} is the name of the output file,
\textit{main} is the name of the main file
and \textit{dest} is the name of the main or child file to be processed
(all filenames without extensions).
The optional argument \textit{main} can be omitted
if \textit{main} matches \textit{dest}.
Optionally, compilation \textit{flags} can be defined via |\def| commands.
This command line makes the \TeX{} engine believe
it is compiling the file \textit{target}
whose content is specified as the latter parameter.
The provided code then forwards the processing to
\textit{main} or \textit{dest} as described in \secref{sec:forward}.

%%%%%%%%%%%%%%%%%%%%%%%%%%%%%%%%%%%%%%%%%%%%%%%%%%%%%%%%%%%%%%%%%%%%%%%%%%%%%%%%
\subsection{Include by Input}
\label{sec:input}

Including child documents by |\include| has some restrictions by design.
Most notably, the content of a child document always occupies
its own set of pages; pages cannot be shared between child documents.
Usually, this behaviour makes perfect sense
because each child document contain an essential part of the document.
However, in some situations it may be desirable to compose
a document from a collection of parts
without having mandatory page breaks between then.
For this case, the package
provides a mechanism to include parts
by |\input| which can also be processed individually.
However, by construction this mechanism
requires manual handling of the content to be output.

%%%%%%%%%%%%%%%%%%%%%%%%%%%%%%%%%%%%%%%%
\DescribeMacro{\ifchilddocmanual}
The main file should be prepared as usual, see \secref{sec:include}.
However, the document body must make a distinction
between processing of an individual part and of the main document, e.g.:
%
\begin{center}
\begin{tabular}{l}
|\ifchilddocmanual|\\
|\input{\childdocname}|\\
|\||else|\\
\textit{document body with }|\input{|\textit{part}|}|\\
|\||fi|
\end{tabular}
\end{center}
%
The conditional |\ifchilddocmanual| is true whenever
a part to be included by |\input| is being compiled,
and the name of the part is stored in |\childdocname|.

%%%%%%%%%%%%%%%%%%%%%%%%%%%%%%%%%%%%%%%%
\DescribeMacro{\childdocby}
Each part to be included by |\input| should start with:
%
\begin{center}
\begin{tabular}{l}
|\input{childdoc.def}|\\
|\childdocby{|\textit{main}|}|\\
\end{tabular}
\end{center}
%
The directive |\childdocby| is similar to |\childdocof|
described in \secref{sec:include},
but the subsequent selection of content must be done manually.
To that end, both |\ifchilddoc| and |\ifchilddocmanual|
will be true upon processing of a part,
and the name of the part is stored in |\childdocname|.
Note that |\jobname| will be set to the filename of the current part
so that each part receives an individual |.aux| file
that does not interfere with the |.aux| file(s) of the main document.
This behaviour can be altered by the alternative form
|\childdocby[*]{|\textit{main}|}| (with a non-empty optional argument)
which uses the |.aux| file of the main document
by setting |\jobname| to \textit{main}.

%%%%%%%%%%%%%%%%%%%%%%%%%%%%%%%%%%%%%%%%%%%%%%%%%%%%%%%%%%%%%%%%%%%%%%%%%%%%%%%%
\subsection{Driver Development}
\label{sec:driver}

The \textsf{childdoc} mechanism can also be use for the development
of definition files such as \LaTeX{} styles or classes.
This case differs from the above setup with multiple parts
included by |\include| in that no |\includeonly| should be invoked.
This can be achieved by starting the include file
(before |\ProvidesPackage|) with:
%
\begin{center}
\begin{tabular}{l}
|\input{childdoc.def}|\\
|\childdocforward{|\textit{main}|}|\\
\end{tabular}
\end{center}
%
or alternatively with:
%
\begin{center}
\begin{tabular}{l}
|\input{childdoc.def}|\\
|\childdocby{|\textit{main}|}|\\
\end{tabular}
\end{center}
%
Both forms have slightly different effects as described above.
The main file is prepared as usual, see \secref{sec:include}.

%%%%%%%%%%%%%%%%%%%%%%%%%%%%%%%%%%%%%%%%%%%%%%%%%%%%%%%%%%%%%%%%%%%%%%%%%%%%%%%%
\subsection{Legacy Detection}
\label{sec:detection}

The directive |\childdocmain| in the main file can detect
whether the complete document or merely a child is to be compiled
even without using the directive |\childdocof|.
This method is deprecated because it is less robust
and there is no compelling reason to use it;
it is merely provided for backward compatibility
and it may be removed in future versions.

If the detection mechanism is to be used,
it is mandatory to correctly specify
the filename of the main file as the argument of |\childdocmain|:
%
\begin{center}
\begin{tabular}{l}
|\input{childdoc.def}|\\
|\childdocmain{|\textit{main}|}|\\
\end{tabular}
\end{center}
%
If |\jobname| does not match the argument \textit{main} of |\childdocmain|,
it is assumed that |\jobname| points to the child file to be compiled.
When using |\childdocmain| with the main file specified as argument,
it suffices to start a child file
with just |\input{|\textit{main}|}|
without loading of the package and using |\childdocof|.
If instead all processing is done
with the appropriate \textsf{childdoc} directives,
the argument of \textit{main} of |\childdocmain| can be empty.

An alternative version of the command line processing described
in \secref{sec:commandline} using the detection mechanism reads:
%
\begin{center}
|... -jobname "|\textit{target}|" "|[\textit{flags}]%
[|\def\jobname{|\textit{dest}|}|]|\input{|\textit{main}|}"|
\end{center}

%%%%%%%%%%%%%%%%%%%%%%%%%%%%%%%%%%%%%%%%%%%%%%%%%%%%%%%%%%%%%%%%%%%%%%%%%%%%%%%%
\subsection{Manual Code}
\label{sec:manual}

In case one cannot be certain whether the definitions file |childdoc.def|
is installed on the target \TeX{} distribution
and one prefers not to ship it,
it is conceivable to paste a few relevant commands into the sources.

To that end, drop all statements |\input{childdoc.def}|
and perform the replacements as outlined below.
Instead of |\childdocmain{|\textit{main}|}| add the following code
to the top of the main file:
%
\begin{center}
\begin{tabular}{l}
|\||ifdefined\childdocname\endinput\||fi\newif\ifchilddoc|\\
|\edef\childdocname{\scantokens\expandafter{\jobname\noexpand}}|\\
|\def\childdocmain{|\textit{main}|}\||ifx\childdocmain\childdocname\||else|\\
|\childdoctrue\includeonly{\childdocname}\let\jobname\childdocmain\||fi|\\
\end{tabular}
\end{center}
%
Instead of |\childdocof{|\textit{main}|}| just include the main file
at the top of each child file:
%
\begin{center}
|\input{|\textit{main}|}|
\end{center}
%
A simple redirection |\childdocforward{|\textit{dest}|}| is achieved by:
%
\begin{center}
|\def\jobname{|\textit{dest}|}\input{\jobname}|
\end{center}
%
The redirection with prefix
|\childdocforwardprefix[|\textit{prefix}|]{|\textit{dest}|}|
is accomplished by:
%
\begin{center}
\begin{tabular}{l}
|{\edef\jobname{\scantokens\expandafter{\jobname\noexpand}}|\\
|\def\redirectjob |\textit{prefix}|#1~~~{\gdef\jobname{|\textit{dest}|#1}}|\\
|\expandafter\redirectjob\jobname~~~}\input{\jobname}|
\end{tabular}
\end{center}

In an alternative approach,
child documents can be compiled by a specific command line
without additional code or specific definitions:
%
\begin{center}
|... -jobname "|\textit{target}|" "|[\textit{flags}]%
|\includeonly{|\textit{dest}|}\input{|\textit{main}|}"|
\end{center}
%

%%%%%%%%%%%%%%%%%%%%%%%%%%%%%%%%%%%%%%%%%%%%%%%%%%%%%%%%%%%%%%%%%%%%%%%%%%%%%%%%
%%%%%%%%%%%%%%%%%%%%%%%%%%%%%%%%%%%%%%%%%%%%%%%%%%%%%%%%%%%%%%%%%%%%%%%%%%%%%%%%
\section{Information}

%%%%%%%%%%%%%%%%%%%%%%%%%%%%%%%%%%%%%%%%%%%%%%%%%%%%%%%%%%%%%%%%%%%%%%%%%%%%%%%%
\subsection{Copyright}

Copyright \copyright{} 2017--2018 Niklas Beisert

This work may be distributed and/or modified under the
conditions of the \LaTeX{} Project Public License, either version 1.3
of this license or (at your option) any later version.
The latest version of this license is in
  \url{http://www.latex-project.org/lppl.txt}
and version 1.3 or later is part of all distributions of \LaTeX{}
version 2005/12/01 or later.

This work has the LPPL maintenance status `maintained'.

The Current Maintainer of this work is Niklas Beisert.

This work consists of the files |README.txt|, |childdoc.ins| and |childdoc.dtx|
as well as the derived files |childdoc.def|, |cdocsamp.tex|
with |cdocsch1.tex|, |cdocsch2.tex|, |cdocspt3.tex|, |cdocspt4.tex|,
|cdocsdrf.tex|, |cdocsfn1.tex|, |cdocsfn2.tex|
as well as |childdoc.pdf|.

%%%%%%%%%%%%%%%%%%%%%%%%%%%%%%%%%%%%%%%%%%%%%%%%%%%%%%%%%%%%%%%%%%%%%%%%%%%%%%%%
\subsection{Files and Installation}

The package consists of the files:
%
\begin{center}
\begin{tabular}{ll}
    |README.txt|   & readme file \\
    |childdoc.ins| & installation file \\
    |childdoc.dtx| & source file \\
    |childdoc.def| & definition file \\
    |cdocsamp.tex| & sample main file \\
    |cdocsch1.tex| & sample include file \\
    |cdocsch2.tex| & sample include file \\
    |cdocspt3.tex| & sample part file \\
    |cdocspt4.tex| & sample part file \\
    |cdocsdrf.tex| & sample redirection file \\
    |cdocsfn1.tex| & sample redirection file \\
    |cdocsfn2.tex| & sample redirection file \\
    |childdoc.pdf| & manual
\end{tabular}
\end{center}
%
The distribution consists of the files
|README.txt|, |childdoc.ins| and |childdoc.dtx|.
%
\begin{itemize}
\item
Run (pdf)\LaTeX{} on |childdoc.dtx|
to compile the manual |childdoc.pdf| (this file).
\item
Run \LaTeX{} on |childdoc.ins| to create the definitions file |childdoc.def|
and the sample |cdocsamp.tex| with include files
|cdocsch1.tex|, |cdocsch2.tex|, |cdocspt3.tex|, |cdocspt4.tex|,
|cdocsdrf.tex|, |cdocsfn1.tex|, |cdocsfn2.tex|.
Then copy the file |childdoc.def| to an appropriate directory of your \LaTeX{}
distribution, e.g.\ \textit{texmf-root}|/tex/latex/childdoc|.
\end{itemize}

%%%%%%%%%%%%%%%%%%%%%%%%%%%%%%%%%%%%%%%%%%%%%%%%%%%%%%%%%%%%%%%%%%%%%%%%%%%%%%%%
\subsection{Related CTAN Packages}

There are several other packages which offer a similar functionality:
%
\begin{itemize}
\item
The packages
\href{http://ctan.org/pkg/docmute}{\textsf{docmute}},
\href{http://ctan.org/pkg/includex}{\textsf{includex}} and
\href{http://ctan.org/pkg/standalone}{\textsf{standalone}}
provide commands to include only the document body of
a child file thus allowing both files to be compiled individually.
\item
The packages \href{http://ctan.org/pkg/subdocs}{\textsf{subdocs}}
and \href{http://ctan.org/pkg/subfiles}{\textsf{subfiles}}
provide structures in which the main and child documents can be
encapsulated and allowing them to be compiled individually.
The inclusion mechanism is different from the conventional |\include|.
\item
The package \href{http://ctan.org/pkg/combine}{\textsf{combine}}
is an elaborate solution to combine several documents into one.
\end{itemize}
%
See also the CTAN topic \href{http://ctan.org/topic/subdocs}{\textsf{subdocs}}
for further related packages.
The present package differs from the above solutions in that
a document structure constructed with the conventional |\include| mechanism
just needs two extra commands at the top of every file
such that all constituent files can be compiled individually.

%%%%%%%%%%%%%%%%%%%%%%%%%%%%%%%%%%%%%%%%%%%%%%%%%%%%%%%%%%%%%%%%%%%%%%%%%%%%%%%%
%\subsection{Feature Suggestions}
%
%The following is a list of features which may be useful for future
%versions of this package:
%%
%\begin{itemize}
%\item
%\ldots
%\end{itemize}

%%%%%%%%%%%%%%%%%%%%%%%%%%%%%%%%%%%%%%%%%%%%%%%%%%%%%%%%%%%%%%%%%%%%%%%%%%%%%%%%
\subsection{Revision History}

%%%%%%%%%%%%%%%%%%%%%%%%%%%%%%%%%%%%%%%%
\paragraph{v2.0:} 2018/12/30

\begin{itemize}
\item
immediate forward processing
\item
added |\childdocby| mechanism
\item
manual restructured
\end{itemize}

%%%%%%%%%%%%%%%%%%%%%%%%%%%%%%%%%%%%%%%%
\paragraph{v1.6:} 2018/01/17

\begin{itemize}
\item
application for development of include files
\item
corrections to manual
\end{itemize}

%%%%%%%%%%%%%%%%%%%%%%%%%%%%%%%%%%%%%%%%
\paragraph{v1.5:} 2017/05/21

\begin{itemize}
\item
more complete structuring introduced
\item
|\childdocof| introduced
\item
|\childdoc| renamed to |\childdocmain|
\item
|\childredirect| renamed to |\childdocforward| and |\childdocforwardprefix|
and functionality expanded
\end{itemize}

%%%%%%%%%%%%%%%%%%%%%%%%%%%%%%%%%%%%%%%%
\paragraph{v1.0:} 2017/04/27

\begin{itemize}
\item
manual and install package
\item
first version published on CTAN
\end{itemize}

%%%%%%%%%%%%%%%%%%%%%%%%%%%%%%%%%%%%%%%%
\paragraph{v0.6:} 2017/04/26

\begin{itemize}
\item
redirection mechanism added
\end{itemize}

%%%%%%%%%%%%%%%%%%%%%%%%%%%%%%%%%%%%%%%%
\paragraph{v0.5:} 2017/04/26

\begin{itemize}
\item
functionality in definition file
\end{itemize}


%%%%%%%%%%%%%%%%%%%%%%%%%%%%%%%%%%%%%%%%%%%%%%%%%%%%%%%%%%%%%%%%%%%%%%%%%%%%%%%%
%%%%%%%%%%%%%%%%%%%%%%%%%%%%%%%%%%%%%%%%%%%%%%%%%%%%%%%%%%%%%%%%%%%%%%%%%%%%%%%%
%%%%%%%%%%%%%%%%%%%%%%%%%%%%%%%%%%%%%%%%%%%%%%%%%%%%%%%%%%%%%%%%%%%%%%%%%%%%%%%%
\appendix

\settowidth\MacroIndent{\rmfamily\scriptsize 000\ }

 \DocInput{childdoc.dtx}

\end{document}
%</driver>
% \fi
%
% %%%%%%%%%%%%%%%%%%%%%%%%%%%%%%%%%%%%%%%%%%%%%%%%%%%%%%%%%%%%%%%%%%%%%%%%%%%%%%
% %%%%%%%%%%%%%%%%%%%%%%%%%%%%%%%%%%%%%%%%%%%%%%%%%%%%%%%%%%%%%%%%%%%%%%%%%%%%%%
% \section{Sample}
%\iffalse
%<*samplemain>
%\fi
%
% The following presents a sample document
% with two chapters, two parts, a title page,
% a compile flag as well as three forwarding files to set the flag.
% It consists of eight |.tex| files:
% \begin{center}
% \begin{tabular}{ll}
% |cdocsamp.tex|&main file\\
% |cdocsch1.tex|&include file for chapter 1\\
% |cdocsch2.tex|&include file for chapter 2\\
% |cdocspt3.tex|&include file for part 3\\
% |cdocspt4.tex|&include file for part 4\\
% |cdocsdrf.tex|&forwarding file for main file in draft mode\\
% |cdocsfi1.tex|&forwarding file for final version of chapter 1\\
% |cdocsfi2.tex|&forwarding file for final version of chapter 2\\
% \end{tabular}
% \end{center}
% Each of the eight files can be compiled directly by the \LaTeX{} compiler.
%
% %%%%%%%%%%%%%%%%%%%%%%%%%%%%%%%%%%%%%%
% \paragraph{Main File.}
%
% The main file is called |cdocsamp.tex|.
%
% Load the \textsf{childdoc} definitions and
% declare the filename for the main document:
%    \begin{macrocode}
\input{childdoc.def}
\childdocmain{}
%    \end{macrocode}

% Optional override for |\version| flag:
%    \begin{macrocode}
%%\ifchilddoc\else\providecommand{\version}{draft}\fi
%    \end{macrocode}

% Define the default values for the |\version| flag
% (|final| for the main file and |draft| for childs):
%    \begin{macrocode}
\ifchilddoc
\providecommand{\version}{draft}
\else
\providecommand{\version}{final}
\fi
%    \end{macrocode}

% Load the standard document class:
%    \begin{macrocode}
\documentclass[12pt]{article}
%    \end{macrocode}

% Start the document body:
%    \begin{macrocode}
\begin{document}
%    \end{macrocode}

% Declare a title page.
% Print title, part of document being processed and version flag:
%    \begin{macrocode}
\addtocounter{page}{-1}
\begin{center}
{\LARGE\bfseries{}childdoc example\par}
\vspace{1cm}
\ifchilddoc
\ifchilddocmanual part\else chapter\fi:
`\childdocname' of `\childdocjob'\par
\else
main document: `\childdocjob'\par
\fi
version: \version\par
\end{center}
\newpage
%    \end{macrocode}

% Manually include selected file,
% otherwise process as usual:
%    \begin{macrocode}
\ifchilddocmanual
\section*{part `\childdocname'}
\input{\childdocname}
\else
%    \end{macrocode}

% Include the two chapters:
%    \begin{macrocode}
\include{cdocsch1}
\include{cdocsch2}
%    \end{macrocode}

% Include the two parts unless only chapters should be displayed:
%    \begin{macrocode}
\ifchilddoc\else
\section{part three}
\input{cdocspt3}
\section{part four}
\input{cdocspt4}
\fi
%    \end{macrocode}

% Process as usual until here:
%    \begin{macrocode}
\fi
%    \end{macrocode}

% End of document body:
%    \begin{macrocode}
\end{document}
%    \end{macrocode}
%\iffalse
%</samplemain>
%\fi
%
% %%%%%%%%%%%%%%%%%%%%%%%%%%%%%%%%%%%%%%
% \paragraph{Chapter Include Files.}
%
% The include files are called |cdocsch1.tex| and |cdocsch2.tex|.
%
%\iffalse
%<*samplechap1|samplechap2>
%\fi

% Optional override for |\version| flag:
%    \begin{macrocode}
%%\providecommand{\version}{final}
%    \end{macrocode}

% Include the main document:
%    \begin{macrocode}
\input{childdoc.def}
\childdocof{cdocsamp}
%    \end{macrocode}

%\iffalse
%</samplechap1|samplechap2>
%\fi
%
%\iffalse
%<*samplechap1>
%\fi
% Some text for chapter 1:
%    \begin{macrocode}
\section{one}
some text in chapter one
%    \end{macrocode}

%\iffalse
%</samplechap1>
%\fi
% Some text for chapter 2:
%\iffalse
%<*samplechap2>
%\fi
%    \begin{macrocode}
\section{two}
more text in chapter two
%    \end{macrocode}

%\iffalse
%</samplechap2>
%\fi
%
% %%%%%%%%%%%%%%%%%%%%%%%%%%%%%%%%%%%%%%
% \paragraph{Part Include Files.}
%
% The include files are called |cdocspt3.tex| and |cdocspt4.tex|.
%
%\iffalse
%<*samplepart3|samplepart4>
%\fi

% Optional override for |\version| flag:
%    \begin{macrocode}
%%\providecommand{\version}{final}
%    \end{macrocode}

% Include the main document:
%    \begin{macrocode}
\input{childdoc.def}
\childdocby{cdocsamp}
%    \end{macrocode}

%\iffalse
%</samplepart3|samplepart4>
%\fi
%
%\iffalse
%<*samplepart3>
%\fi
% Some text for part 3:
%    \begin{macrocode}
some text in part three
%    \end{macrocode}

%\iffalse
%</samplepart3>
%\fi
% Some text for part 4:
%\iffalse
%<*samplepart4>
%\fi
%    \begin{macrocode}
more text in part four
%    \end{macrocode}

%\iffalse
%</samplepart4>
%\fi
%
% %%%%%%%%%%%%%%%%%%%%%%%%%%%%%%%%%%%%%%
% \paragraph{Forwarding for a Complete Draft.}
%
% The following forwarding file |cdocsdrf.tex|
% compiles the main document in draft mode:
%\iffalse
%<*sampledraft>
%\fi
%    \begin{macrocode}
\def\version{draft}
\input{childdoc.def}
\childdocforward{cdocsamp}
%    \end{macrocode}

%\iffalse
%</sampledraft>
%\fi
%
% %%%%%%%%%%%%%%%%%%%%%%%%%%%%%%%%%%%%%%
% \paragraph{Forwarding for Final Version of the Chapters.}
%
% The following forwarding files |cdocsfn1.tex| and |cdocsfn2.tex|
% (with identical content)
% compile the final versions of the child documents
% |cdocsch1.tex| and |cdocsch2.tex|, respectively:
%\iffalse
%<*samplefinal>
%\fi
%    \begin{macrocode}
\def\version{final}
\input{childdoc.def}
\childdocforwardprefix[cdocsamp]{cdocsfn}{cdocsch}
%    \end{macrocode}

%\iffalse
%</samplefinal>
%\fi
%
% %%%%%%%%%%%%%%%%%%%%%%%%%%%%%%%%%%%%%%
% \paragraph{Command Line Processing.}
%
% The following three command lines generate the output files
% |cdocscld|, |cdocscl1| and |cdocscl2|
% which should be identical to
% |cdocsdrf|, |cdocsch1| and |cdocsfn2|, respectively:
% \begin{center}
% \begin{tabular}{l}
% |latex -jobname cdocscld \|\\
% |  "\def\version{draft}\input{childdoc.def}\childdocforward{cdocsamp}"|\\
% |latex -jobname cdocscl1 \|\\
% |  "\input{childdoc.def}\childdocforward[cdocsamp]{cdocsch1}"|\\
% |latex -jobname cdocscl2 \|\\
% |  "\def\version{final}\input{childdoc.def}\childdocforward{cdocsch2}"|
% \end{tabular}
% \end{center}
% Note that the trailing backslash on each first line
% merely continues the input to the second line
% (for convenient cut ant paste).
% Furthermore, the command |latex| can be replaced by any
% of its alternative versions such as |pdflatex|.
%
% %%%%%%%%%%%%%%%%%%%%%%%%%%%%%%%%%%%%%%%%%%%%%%%%%%%%%%%%%%%%%%%%%%%%%%%%%%%%%%
% %%%%%%%%%%%%%%%%%%%%%%%%%%%%%%%%%%%%%%%%%%%%%%%%%%%%%%%%%%%%%%%%%%%%%%%%%%%%%%
% \section{Implementation}
%\iffalse
%<*package>
%\fi
%
% This section describes the definitions file |childdoc.def|.

% The definitions cannot be loaded using |\usepackage| or |\RequirePackage|
% which has a mechanism to prevent loading a style file more than once.
% When loading the definitions by means of |\input|
% multiple instances have to be prevented manually:
%\iffalse
%This code needs to be before the `\ProvidesFile' directive
%which is defined at the beginning of this file.
%Therefore it is also placed there and commented out here.
%</package>
%<*discard>
%\fi
%    \begin{macrocode}
\ifdefined\childdocmain\endinput\fi
%    \end{macrocode}
%\iffalse
%</discard>
%<*package>
%\fi
%
% \macro{\ifchilddoc}
% \macro{\ifchilddocmanual}
% The conditional |\ifchilddoc| tells whether a
% child (true) or main (false) document is being compiled.
% The conditional |\ifchilddocmanual| tells whether
% the |\includeonly| mechanism is used (false) or
% the selection of child files must be performed manually (true).
% The definitions initialise to false:
%    \begin{macrocode}
\newif\ifchilddoc
\newif\ifchilddocmanual
%    \end{macrocode}

% \macro{\childdocname}
% \macro{\childdocjob}
% The macro |\childdocname| stores the name of the main document
% to be compiled. The macro |\childdocjob| stores the name of
% the document on which the \LaTeX{} compiler was originally invoked.
% The content of |\jobname| cannot be compared
% to filenames specified in the source due to different catcodes.
% The following code rescans |\jobname|, stores the result
% in |\childdocname| and saves a copy in |\childdocjob|:
%    \begin{macrocode}
\edef\childdocname{\scantokens\expandafter{\jobname\noexpand}}
\let\childdocjob\childdocname
%    \end{macrocode}

% \macro{\childdocdisable}
% The macro |\childdocdisable| prevents the main file
% from being processed more than once.
% At this stage, the main document command |\childdocmain|
% is assumed to be called once again where it should do nothing.
% Any subsequent call to it should prevent
% a secondary processing of the main document
% It overwrites the forwarding commands
% |\childdocof| and |\childdocforward|
% with empty macros to prevent further inclusions of the main document:
%    \begin{macrocode}
\newcommand{\childdocdisable}
{
  \renewcommand{\childdocmain}[1]{\renewcommand{\childdocmain}[1]{\endinput}}
  \renewcommand{\childdocof}[1]{}
  \renewcommand{\childdocby}[2][]{}
  \renewcommand{\childdocforward}[2][]{}
  \renewcommand{\childdocdisable}{}
}
%    \end{macrocode}

% \macro{\childdocmain}
% The macro |\childdocmain| is to be called at the top of the main file
% with nothing or the main filename (without extension) as argument.
% First, it breaks loops.
% If the argument is not empty and does not match |\childdocname|
% (which is set by the first inclusion of |childdoc.def|),
% |\ifchilddoc| is set to true, |\includeonly| is applied to the child file
% and |\jobname| is set to the main file
% (for proper handling of |.aux| files):
%    \begin{macrocode}
\newcommand{\childdocmain}[1]
{
  \childdocdisable\childdocmain{}
  \if?#1?\else
    \begingroup
      \def\childdoctmp{#1}
      \ifx\childdoctmp\childdocname
        \def\childdoctmp{}
      \else
        \def\childdoctmp
        {
          \childdoctrue
          \includeonly{\childdocname}
          \def\childdocjob{#1}
          \def\jobname{#1}
        }
      \fi
      \expandafter
    \endgroup
    \childdoctmp
  \fi
}
%    \end{macrocode}

% \macro{\childdocof}
% The command |\childdocof| redirects
% compilation to the main file |#1|.
%    \begin{macrocode}
\newcommand{\childdocof}[1]
{
  \childdocdisable
  \childdoctrue
  \includeonly{\childdocname}
  \def\jobname{#1}
  \def\childdocjob{#1}
  \input{#1}
}
%    \end{macrocode}

% \macro{\childdocby}
% The command |\childdocby| ....
%    \begin{macrocode}
\newcommand{\childdocby}[2][]
{
  \childdocdisable
  \childdoctrue
  \childdocmanualtrue
  \if?#1?\else
    \def\jobname{#2}
  \fi
  \def\childdocjob{#2}
  \input{#2}
  \endinput
}
%    \end{macrocode}

% \macro{\childdocforward}
% The command |\childdocforward| redirects
% compilation to the main file or
% (if the optional argument is given) a child file.
% Parameters are set as if the main file
% or a child file starting with |\childdocof| was compiled.
% Then compilation is handed over to the main file:
%    \begin{macrocode}
\newcommand{\childdocforward}[2][]
{
  \begingroup
    \if?#1?
      \def\childdoctmp
      {
        \def\childdocname{#2}
        \def\childdocjob{#2}
        \def\jobname{#2}
        \input{#2}
        \endinput
      }
    \else
      \def\childdoctmp
      {
        \childdocdisable
        \def\childdocname{#2}
        \childdoctrue
        \includeonly{#2}
        \def\childdocjob{#1}
        \def\jobname{#1}
        \input{#1}
        \endinput
      }
    \fi
    \expandafter
  \endgroup
  \childdoctmp
}
%    \end{macrocode}

% \macro{\childdocforwardprefix}
% The command |\childdocforwardprefix| redirects
% compilation to the main or a child file by means of a pattern.
% The prefix |#1| in the current filename is replaced by |#2|
% and the suffix of the current filename is kept
% (it is assumed that the filename does not contain the substring `|~~~|'
% which is used as a delimiter).
% Compilation is handed over to the new file by |\childdocforward|:
%    \begin{macrocode}
\newcommand{\childdocforwardprefix}[3][]
{
  \begingroup
    \def\childdocextract #2##1~~~{\def\childdoctmp{\childdocforward[#1]{#3##1}}}
    \expandafter\childdocextract\childdocname~~~
    \expandafter
  \endgroup
  \childdoctmp
}
%    \end{macrocode}

% \macro{\childdoc}
% The deprecated macro |\childdoc| is a legacy version of |\childdocmain|:
%    \begin{macrocode}
\newcommand{\childdoc}{\childdocmain}
%    \end{macrocode}

% \macro{\childdocredirect}
% The deprecated macro |\childdocredirect| is a legacy version
% of |\childdocforward| and |\childdocforwardprefix|:
%    \begin{macrocode}
\newcommand{\childdocredirect}[2][]
{
  \begingroup
    \if?#1?
      \def\childdoctmp{\childdocforward{#2}}
    \else
      \def\childdoctmp{\childdocforwardprefix{#1}{#2}}
    \fi
    \expandafter
  \endgroup
  \childdoctmp
}
%    \end{macrocode}

%\iffalse
%</package>
%\fi
%
\endinput
|\\
|\childdocmain{}|\\
\end{tabular}
\end{center}
at the very top of the main \LaTeX{} file,
in particular \emph{before} the |\documentclass| statement!
The argument of |\childdocmain| should be left empty
(but it must be present).

%%%%%%%%%%%%%%%%%%%%%%%%%%%%%%%%%%%%%%%%
\DescribeMacro{\childdocof}
Furthermore, add the commands
\begin{center}
\begin{tabular}{l}
|% \iffalse
%
% childdoc.dtx Copyright (C) 2017-2018 Niklas Beisert
%
% This work may be distributed and/or modified under the
% conditions of the LaTeX Project Public License, either version 1.3
% of this license or (at your option) any later version.
% The latest version of this license is in
%   http://www.latex-project.org/lppl.txt
% and version 1.3 or later is part of all distributions of LaTeX
% version 2005/12/01 or later.
%
% This work has the LPPL maintenance status `maintained'.
%
% The Current Maintainer of this work is Niklas Beisert.
%
% This work consists of the files childdoc.dtx and childdoc.ins
% and the derived files childdoc.def and cdocsamp.tex with
% cdocsch1.tex, cdocsch2.tex, cdocsdrf.tex, cdocsfn1.tex, cdocsfn2.tex.
%
%<package>\ifdefined\childdocmain\endinput\fi
%<package>\ProvidesFile{childdoc.def}[2018/12/30 v2.0 child document driver]
%<samplemain>\ProvidesFile{cdocsamp.tex}[2018/12/30 v2.0 sample for childdoc]
%<*driver>
%\ProvidesFile{childdoc.drv}[2018/12/30 v2.0 childdoc reference manual file]
\PassOptionsToClass{10pt,a4paper}{article}
\documentclass{ltxdoc}

\usepackage[margin=35mm]{geometry}
\usepackage{hyperref}
\usepackage{hyperxmp}
\usepackage[usenames]{color}

\hypersetup{colorlinks=true}
\hypersetup{pdfstartview=FitH}
\hypersetup{pdfpagemode=UseNone}
\hypersetup{pdfsource={}}
\hypersetup{pdflang={en-UK}}
\hypersetup{pdfcopyright={Copyright 2017-2018 Niklas Beisert.
  This work may be distributed and/or modified under the
  conditions of the LaTeX Project Public License, either version 1.3
  of this license or (at your option) any later version.}}
\hypersetup{pdflicenseurl={http://www.latex-project.org/lppl.txt}}
\hypersetup{pdfcontactaddress={ETH Zurich, ITP, HIT K,
  Wolfgang-Pauli-Strasse 27}}
\hypersetup{pdfcontactpostcode={8093}}
\hypersetup{pdfcontactcity={Zurich}}
\hypersetup{pdfcontactcountry={Switzerland}}
\hypersetup{pdfcontactemail={nbeisert@itp.phys.ethz.ch}}
\hypersetup{pdfcontacturl={http://people.phys.ethz.ch/\xmptilde nbeisert/}}

\newcommand{\secref}[1]{\hyperref[#1]{section \ref*{#1}}}

\parskip1ex
\parindent0pt
\let\olditemize\itemize
\def\itemize{\olditemize\parskip0pt}

\begin{document}

\title{The \textsf{childdoc} Package}
\hypersetup{pdftitle={The childdoc Package}}
\author{Niklas Beisert\\[2ex]
  Institut f\"ur Theoretische Physik\\
  Eidgen\"ossische Technische Hochschule Z\"urich\\
  Wolfgang-Pauli-Strasse 27, 8093 Z\"urich, Switzerland\\[1ex]
  \href{mailto:nbeisert@itp.phys.ethz.ch}
  {\texttt{nbeisert@itp.phys.ethz.ch}}}
\hypersetup{pdfauthor={Niklas Beisert}}
\hypersetup{pdfsubject={Manual for the LaTeX2e Package childdoc}}
\date{30 December 2018, \textsf{v2.0}}
\maketitle

\begin{abstract}\noindent
\textsf{childdoc} is a \LaTeXe{} package
that enables the direct compilation
of document sections included by |\include|
to individual files.
\end{abstract}

\begingroup
\parskip0ex
\tableofcontents
\endgroup

%%%%%%%%%%%%%%%%%%%%%%%%%%%%%%%%%%%%%%%%%%%%%%%%%%%%%%%%%%%%%%%%%%%%%%%%%%%%%%%%
%%%%%%%%%%%%%%%%%%%%%%%%%%%%%%%%%%%%%%%%%%%%%%%%%%%%%%%%%%%%%%%%%%%%%%%%%%%%%%%%
\section{Introduction}

\LaTeX{} provides a mechanism to structure a large document (such as a book)
into a main file and several child files (containing the chapters)
using the |\include| command.
This mechanism is beneficial for documents
which span hundreds of pages in order to
make the source file(s) more manageable.
Moreover, compilation can be restricted to
selected child files by means of the |\includeonly| command.
The latter feature can be used to reduce the compilation time while editing
(this was significantly more useful in the earlier days of \LaTeX{})
or to generate a smaller document which is easier to navigate.
Another application of |\includeonly| is to generate
documents consisting of selected parts of the complete document.

However, there are a few drawbacks of the plain |\include| mechanism:
\begin{itemize}
\item
The child files cannot be compiled on their own,
they can only be compiled via the main file.
A naive editing environment
(such as a text editor with an option
to have the current file processed by \LaTeX)
may require one to switch to the main file before compiling;
attempting to compile the child file produces errors.
\item
The main file must be modified (each time)
to adjust the |\includeonly| command
to the present needs. This easily leaves the main file in a messy state.
\item
The generated document will always carry the filename
of the main document. This is inconvenient if
several child files are to be compiled and
to be kept for distribution.
\end{itemize}

The present package provides a simple interface
to make child files individually compilable by \LaTeX{}.
Compiling a child file then has the same effect as compiling
the main file with an |\includeonly| command
to select the appropriate child.
Moreover the generated document will carry the name of the child
rather than the main file.
This resolves all three above issues.

This feature is meant to make the editing of books,
thesis documents and lecture notes somewhat more convenient.
However, the package can also be used efficiently for
composing a series of documents (such as exercise sheets)
which are typically distributed individually.
It then assists the author in generating the individual documents
(potentially in different versions)
as well as a document containing the collected series.
Another application is in developing style files
or other kinds of included material
where compilation of the style file could redirect
to a sample or test file.

%%%%%%%%%%%%%%%%%%%%%%%%%%%%%%%%%%%%%%%%%%%%%%%%%%%%%%%%%%%%%%%%%%%%%%%%%%%%%%%%
%%%%%%%%%%%%%%%%%%%%%%%%%%%%%%%%%%%%%%%%%%%%%%%%%%%%%%%%%%%%%%%%%%%%%%%%%%%%%%%%
\section{Usage}

First of all, the package \textsf{childdoc} is \emph{not} a standard
\LaTeXe{} |.sty| style file! Therefore it needs to be invoked in
a non-standard way.

%%%%%%%%%%%%%%%%%%%%%%%%%%%%%%%%%%%%%%%%%%%%%%%%%%%%%%%%%%%%%%%%%%%%%%%%%%%%%%%%
\subsection{Included Files}
\label{sec:include}

%%%%%%%%%%%%%%%%%%%%%%%%%%%%%%%%%%%%%%%%
\DescribeMacro{\childdocmain}
To use the package, add the commands
\begin{center}
\begin{tabular}{l}
|\input{childdoc.def}|\\
|\childdocmain{}|\\
\end{tabular}
\end{center}
at the very top of the main \LaTeX{} file,
in particular \emph{before} the |\documentclass| statement!
The argument of |\childdocmain| should be left empty
(but it must be present).

%%%%%%%%%%%%%%%%%%%%%%%%%%%%%%%%%%%%%%%%
\DescribeMacro{\childdocof}
Furthermore, add the commands
\begin{center}
\begin{tabular}{l}
|\input{childdoc.def}|\\
|\childdocof{|\textit{main}|}|\\
\end{tabular}
\end{center}
at the top of every child file \textit{child}
which is included by |\include{|\textit{child}|}|
from within the main file
(or at least for those files to be compiled individually).
The argument \textit{main} must be the filename of the main file.

There are a couple of
considerations in setting up the main and child documents:

%%%%%%%%%%%%%%%%%%%%%%%%%%%%%%%%%%%%%%%%
\paragraph{Restrictions.}

Please note the following restrictions:
\begin{itemize}
\item
|\childdocmain| must be called with one argument \textit{main}
to ensure compatibility with earlier version of the package.
It must either be empty (|\childdocmain{}|)
or precisely match the filename of the main file in which it is specified.
See \secref{sec:detection} for further information.
\item
The filename \textit{main} must be specified without the |.tex| extension.
\item
The filename \textit{main} is case sensitive
(even in case-insensitive file systems)
due to internal string comparison.
\item
The argument \textit{main} should be fully expanded, it cannot be a macro.
\item
Subdirectories and special characters should be avoided in filenames.
\item
The command |\childdocmain{|\textit{main}|}| must be followed by a whitespace.
It should not be followed immediately by another command
or by a comment mark `|%|'.
This is because the \TeX{} parser reads the token immediately following
the argument of |\childdocmain| and puts it
at the beginning of every child section;
however, a white\-space is ignored.
\end{itemize}

%%%%%%%%%%%%%%%%%%%%%%%%%%%%%%%%%%%%%%%%
\paragraph{Content of Main File.}

It is advisable to place all content in the child files included by |\include|.
Any output contained in the main file will appear in all child documents
unless suppressed manually;
it cannot be suppressed automatically by the |\includeonly| directive
and thus should normally be avoided.
A method to include some content in the main file
by means of conditional processing is described in \secref{sec:conditional}.

%%%%%%%%%%%%%%%%%%%%%%%%%%%%%%%%%%%%%%%%
\paragraph{Page Numbering.}

When only a part of the document is compiled,
the appropriate numbering of pages
(as well as other status parameters)
is determined from the |.aux| files.
The latter contain information from previous passes.
However this information needs to propagate through
all intermediate child documents.
Therefore the page numbering in child documents may well
be inconsistent until the complete document is compiled at least once.

A useful (if unconventional) way to always ensure a consistent
page numbering is to restart the numbering in each child document
and denote the pages by `\textit{child}|.|\textit{page}'
where \textit{child} represents the chapter/section number of the child file.
This can be achieved by the command
|\numberwithin{page}{|\textit{child}|}|
of the \textsf{amsmath} package
where \textit{child} can be |chapter| or |section|
depending on the chosen structuring.
Alternatively, one can modify the macro |\thepage| appropriately
and reset the counter |page| at the start of each child file.

%%%%%%%%%%%%%%%%%%%%%%%%%%%%%%%%%%%%%%%%%%%%%%%%%%%%%%%%%%%%%%%%%%%%%%%%%%%%%%%%
\subsection{Conditional Processing}
\label{sec:conditional}

The package provides a mechanism to compile different versions
of a document. To customise the versions further some conditional processing
can come in handy to distinguish which version is being compiled.
The package provides two macros to describe the compilation context:

%%%%%%%%%%%%%%%%%%%%%%%%%%%%%%%%%%%%%%%%
\DescribeMacro{\ifchilddoc}
The conditional |\ifchilddoc| distinguishes between the compilation of
child documents and the main document:
%
\begin{center}
|\ifchilddoc |\textit{child-code}| |[|\||else |\textit{main-code}]| \||fi|
\end{center}

%%%%%%%%%%%%%%%%%%%%%%%%%%%%%%%%%%%%%%%%
\DescribeMacro{\childdocname}
\DescribeMacro{\childdocjob}
The macro |\childdocname| contains the filename (without extension)
of the main or child file being processed.
Note that |\childdocjob| will always contain the name of the main file.

%%%%%%%%%%%%%%%%%%%%%%%%%%%%%%%%%%%%%%%%
\paragraph{Title Page.}

Conditional processing can be used to include a title or banner page
in the main document when proper precautions are taken.
Importantly, the code in the main file should ensure that the page counter
(as well as other status parameters which are stored in the |.aux| files)
takes the same value after the conditional processing.
Otherwise the page numbers may take divergent values
depending on which part is compiled.

For example, a title page could be declared by:
%
\begin{center}
\begin{tabular}{l}
|\ifchilddoc\||else|\\
|\addtocounter{page}{-1}|\\
\textit{code for title page}\\
|\newpage|\\
|\||fi|
\end{tabular}
\end{center}
%
A banner page for the child documents can be generated by:
%
\begin{center}
\begin{tabular}{l}
|\ifchilddoc|\\
|\addtocounter{page}{-1}|\\
\textit{code for banner page}\\
|\newpage|\\
|\||fi|
\end{tabular}
\end{center}
%
Here one could write a message such as:
\begin{center}
|This is the part \childdocname{} of \childdocjob{}.|
\end{center}

%%%%%%%%%%%%%%%%%%%%%%%%%%%%%%%%%%%%%%%%%%%%%%%%%%%%%%%%%%%%%%%%%%%%%%%%%%%%%%%%
\subsection{Flags}
\label{sec:flags}

The package makes it easy to generate different versions
of the main or child documents.
To this end compilation flags can be defined
and assigned different default values.
They will be particularly useful in conjunction
with the forwarding mechanism described in \secref{sec:forward}.

For example, it may be useful to have a flag |\version|
which can be set to |draft| or |final|.
The document source will contain some conditional code
depending on the value of |\version|.
Suppose further, the flag should default to |final| for the main file
and to |draft| for child files
which is a natural assignment for editing the document.
This is achieved by placing the following code
in the preamble of the main document
(below the |\childdocmain| directive):
%
\begin{center}
\begin{tabular}{l}
|\ifchilddoc|\\
|\providecommand{\version}{draft}|\\
|\||else|\\
|\providecommand{\version}{final}|\\
|\||fi|
\end{tabular}
\end{center}
%
The definition by |\providecommand| makes sure
that previous definitions are not overwritten.
Further statements |\providecommand{\version}{...}|
can thus be added before the above code to override it.

For the main file, one might add a line
(between |\childdocmain| and the above block)
%
\begin{center}
|%\ifchilddoc\||else\providecommand{\version}{draft}\||fi|
\end{center}
%
which can be uncommented to produce a draft version.
Likewise one can add a line to the very top of a child file
(above the |\childdocof{|\textit{main}|}| directive)
%
\begin{center}
|%\providecommand{\version}{final}|
\end{center}
%
which can be uncommented to produce the final version of this child document.

%%%%%%%%%%%%%%%%%%%%%%%%%%%%%%%%%%%%%%%%%%%%%%%%%%%%%%%%%%%%%%%%%%%%%%%%%%%%%%%%
\subsection{Forwarding}
\label{sec:forward}

Different versions of the main or child documents
using compilation flags as described in \secref{sec:flags}
can be (permanently) stored in different files
for convenient compilation, viewing and distribution.
To this end, the package defines a command
to pass on compilation to a different file:

%%%%%%%%%%%%%%%%%%%%%%%%%%%%%%%%%%%%%%%%
\DescribeMacro{\childdocforward}
The command |\childdocforward| redirects processing to
another source file:
%
\begin{center}
\begin{tabular}{l}
|\input{childdoc.def}|\\
|\childdocforward[|\textit{main}|]{|\textit{dest}|}|\\
\end{tabular}
\end{center}
%
The argument \textit{dest} is the destination file
(without extension).
It should be the main file or one of the child files.
Note that further \textsf{childdoc} directives
such as |\childdocof| and |\childdocforward|
in the indicated file will be processed in this form.
The optional argument \textit{main}
passes on directly to the main file \textit{main}
while pretending to compile the child \textit{dest}.
This form behaves as if \textit{dest}
issues |\childdocof{|\textit{main}|}| right away,
and no further \textsf{childdoc} directives will be processed.

%%%%%%%%%%%%%%%%%%%%%%%%%%%%%%%%%%%%%%%%
\DescribeMacro{\...prefix}
In the alternative form |\childdocforwardprefix|,
%
\begin{center}
\begin{tabular}{l}
|\input{childdoc.def}|\\
|\childdocforwardprefix[|\textit{main}|]{|\textit{prefix}|}{|\textit{dest}|}|
\end{tabular}
\end{center}
%
the destination file is determined by a pattern
depending on the current file:
To make this work, the current file must be called
`{\textit{prefix}\hspace{0.2em}\textit{suffix}}'
with \textit{prefix} matching precisely the argument.
Processing is then passed on to the file
`{\textit{dest}\hspace{0.2em}\textit{suffix}}'.
Surely, the same effect is achieved by
directly specifying the
argument `{\textit{dest}\hspace{0.2em}\textit{suffix}}'
in the first form.
However, that requires to set up a different file
for each child. With the alternative form of the command
all these files can have exactly the same content
which simplifies setting them up and maintaining them.

For example, the following file |draft.tex|
with a compilation flag |\version| as described in \secref{sec:flags}
compiles the main document as a draft:
%
\begin{center}
\begin{tabular}{l}
|\def\version{draft}|\\
|\input{childdoc.def}|\\
|\childdocforward{|\textit{main}|}|
\end{tabular}
\end{center}
%
Likewise, the following files |final|\textit{nn}|.tex|
compile the final version of the child document
|child|\textit{nn}|.tex|:
%
\begin{center}
\begin{tabular}{l}
|\def\version{final}|\\
|\input{childdoc.def}|\\
|\childdocforwardprefix{final}{child}|
\end{tabular}
\end{center}
%

Note that when several versions of a main file and/or of each child file
are to be generated, it may be convenient to set up a |Makefile| or
shell script to automatise the process.

%%%%%%%%%%%%%%%%%%%%%%%%%%%%%%%%%%%%%%%%%%%%%%%%%%%%%%%%%%%%%%%%%%%%%%%%%%%%%%%%
\subsection{Command Line Processing}
\label{sec:commandline}

The effect of redirection files can also be achieved by invoking
the \LaTeX{} compiler with a more elaborate command line.
Most conveniently this should be done as part
of a shell script or a |Makefile|.

When using \textsf{childdoc} in the main file, the following
command lines effectively perform a redirection
(note that depending on the shell being used,
backslashes may have to be doubled: `|\|' $\to$ `|\\|'):
%
\begin{center}
|... -jobname "|\textit{target}|" |\\|"|[\textit{flags}]%
|\input{childdoc.def}\childdocforward[|\textit{main}|]{|\textit{dest}|}"|
\end{center}
%
Here \textit{target} is the name of the output file,
\textit{main} is the name of the main file
and \textit{dest} is the name of the main or child file to be processed
(all filenames without extensions).
The optional argument \textit{main} can be omitted
if \textit{main} matches \textit{dest}.
Optionally, compilation \textit{flags} can be defined via |\def| commands.
This command line makes the \TeX{} engine believe
it is compiling the file \textit{target}
whose content is specified as the latter parameter.
The provided code then forwards the processing to
\textit{main} or \textit{dest} as described in \secref{sec:forward}.

%%%%%%%%%%%%%%%%%%%%%%%%%%%%%%%%%%%%%%%%%%%%%%%%%%%%%%%%%%%%%%%%%%%%%%%%%%%%%%%%
\subsection{Include by Input}
\label{sec:input}

Including child documents by |\include| has some restrictions by design.
Most notably, the content of a child document always occupies
its own set of pages; pages cannot be shared between child documents.
Usually, this behaviour makes perfect sense
because each child document contain an essential part of the document.
However, in some situations it may be desirable to compose
a document from a collection of parts
without having mandatory page breaks between then.
For this case, the package
provides a mechanism to include parts
by |\input| which can also be processed individually.
However, by construction this mechanism
requires manual handling of the content to be output.

%%%%%%%%%%%%%%%%%%%%%%%%%%%%%%%%%%%%%%%%
\DescribeMacro{\ifchilddocmanual}
The main file should be prepared as usual, see \secref{sec:include}.
However, the document body must make a distinction
between processing of an individual part and of the main document, e.g.:
%
\begin{center}
\begin{tabular}{l}
|\ifchilddocmanual|\\
|\input{\childdocname}|\\
|\||else|\\
\textit{document body with }|\input{|\textit{part}|}|\\
|\||fi|
\end{tabular}
\end{center}
%
The conditional |\ifchilddocmanual| is true whenever
a part to be included by |\input| is being compiled,
and the name of the part is stored in |\childdocname|.

%%%%%%%%%%%%%%%%%%%%%%%%%%%%%%%%%%%%%%%%
\DescribeMacro{\childdocby}
Each part to be included by |\input| should start with:
%
\begin{center}
\begin{tabular}{l}
|\input{childdoc.def}|\\
|\childdocby{|\textit{main}|}|\\
\end{tabular}
\end{center}
%
The directive |\childdocby| is similar to |\childdocof|
described in \secref{sec:include},
but the subsequent selection of content must be done manually.
To that end, both |\ifchilddoc| and |\ifchilddocmanual|
will be true upon processing of a part,
and the name of the part is stored in |\childdocname|.
Note that |\jobname| will be set to the filename of the current part
so that each part receives an individual |.aux| file
that does not interfere with the |.aux| file(s) of the main document.
This behaviour can be altered by the alternative form
|\childdocby[*]{|\textit{main}|}| (with a non-empty optional argument)
which uses the |.aux| file of the main document
by setting |\jobname| to \textit{main}.

%%%%%%%%%%%%%%%%%%%%%%%%%%%%%%%%%%%%%%%%%%%%%%%%%%%%%%%%%%%%%%%%%%%%%%%%%%%%%%%%
\subsection{Driver Development}
\label{sec:driver}

The \textsf{childdoc} mechanism can also be use for the development
of definition files such as \LaTeX{} styles or classes.
This case differs from the above setup with multiple parts
included by |\include| in that no |\includeonly| should be invoked.
This can be achieved by starting the include file
(before |\ProvidesPackage|) with:
%
\begin{center}
\begin{tabular}{l}
|\input{childdoc.def}|\\
|\childdocforward{|\textit{main}|}|\\
\end{tabular}
\end{center}
%
or alternatively with:
%
\begin{center}
\begin{tabular}{l}
|\input{childdoc.def}|\\
|\childdocby{|\textit{main}|}|\\
\end{tabular}
\end{center}
%
Both forms have slightly different effects as described above.
The main file is prepared as usual, see \secref{sec:include}.

%%%%%%%%%%%%%%%%%%%%%%%%%%%%%%%%%%%%%%%%%%%%%%%%%%%%%%%%%%%%%%%%%%%%%%%%%%%%%%%%
\subsection{Legacy Detection}
\label{sec:detection}

The directive |\childdocmain| in the main file can detect
whether the complete document or merely a child is to be compiled
even without using the directive |\childdocof|.
This method is deprecated because it is less robust
and there is no compelling reason to use it;
it is merely provided for backward compatibility
and it may be removed in future versions.

If the detection mechanism is to be used,
it is mandatory to correctly specify
the filename of the main file as the argument of |\childdocmain|:
%
\begin{center}
\begin{tabular}{l}
|\input{childdoc.def}|\\
|\childdocmain{|\textit{main}|}|\\
\end{tabular}
\end{center}
%
If |\jobname| does not match the argument \textit{main} of |\childdocmain|,
it is assumed that |\jobname| points to the child file to be compiled.
When using |\childdocmain| with the main file specified as argument,
it suffices to start a child file
with just |\input{|\textit{main}|}|
without loading of the package and using |\childdocof|.
If instead all processing is done
with the appropriate \textsf{childdoc} directives,
the argument of \textit{main} of |\childdocmain| can be empty.

An alternative version of the command line processing described
in \secref{sec:commandline} using the detection mechanism reads:
%
\begin{center}
|... -jobname "|\textit{target}|" "|[\textit{flags}]%
[|\def\jobname{|\textit{dest}|}|]|\input{|\textit{main}|}"|
\end{center}

%%%%%%%%%%%%%%%%%%%%%%%%%%%%%%%%%%%%%%%%%%%%%%%%%%%%%%%%%%%%%%%%%%%%%%%%%%%%%%%%
\subsection{Manual Code}
\label{sec:manual}

In case one cannot be certain whether the definitions file |childdoc.def|
is installed on the target \TeX{} distribution
and one prefers not to ship it,
it is conceivable to paste a few relevant commands into the sources.

To that end, drop all statements |\input{childdoc.def}|
and perform the replacements as outlined below.
Instead of |\childdocmain{|\textit{main}|}| add the following code
to the top of the main file:
%
\begin{center}
\begin{tabular}{l}
|\||ifdefined\childdocname\endinput\||fi\newif\ifchilddoc|\\
|\edef\childdocname{\scantokens\expandafter{\jobname\noexpand}}|\\
|\def\childdocmain{|\textit{main}|}\||ifx\childdocmain\childdocname\||else|\\
|\childdoctrue\includeonly{\childdocname}\let\jobname\childdocmain\||fi|\\
\end{tabular}
\end{center}
%
Instead of |\childdocof{|\textit{main}|}| just include the main file
at the top of each child file:
%
\begin{center}
|\input{|\textit{main}|}|
\end{center}
%
A simple redirection |\childdocforward{|\textit{dest}|}| is achieved by:
%
\begin{center}
|\def\jobname{|\textit{dest}|}\input{\jobname}|
\end{center}
%
The redirection with prefix
|\childdocforwardprefix[|\textit{prefix}|]{|\textit{dest}|}|
is accomplished by:
%
\begin{center}
\begin{tabular}{l}
|{\edef\jobname{\scantokens\expandafter{\jobname\noexpand}}|\\
|\def\redirectjob |\textit{prefix}|#1~~~{\gdef\jobname{|\textit{dest}|#1}}|\\
|\expandafter\redirectjob\jobname~~~}\input{\jobname}|
\end{tabular}
\end{center}

In an alternative approach,
child documents can be compiled by a specific command line
without additional code or specific definitions:
%
\begin{center}
|... -jobname "|\textit{target}|" "|[\textit{flags}]%
|\includeonly{|\textit{dest}|}\input{|\textit{main}|}"|
\end{center}
%

%%%%%%%%%%%%%%%%%%%%%%%%%%%%%%%%%%%%%%%%%%%%%%%%%%%%%%%%%%%%%%%%%%%%%%%%%%%%%%%%
%%%%%%%%%%%%%%%%%%%%%%%%%%%%%%%%%%%%%%%%%%%%%%%%%%%%%%%%%%%%%%%%%%%%%%%%%%%%%%%%
\section{Information}

%%%%%%%%%%%%%%%%%%%%%%%%%%%%%%%%%%%%%%%%%%%%%%%%%%%%%%%%%%%%%%%%%%%%%%%%%%%%%%%%
\subsection{Copyright}

Copyright \copyright{} 2017--2018 Niklas Beisert

This work may be distributed and/or modified under the
conditions of the \LaTeX{} Project Public License, either version 1.3
of this license or (at your option) any later version.
The latest version of this license is in
  \url{http://www.latex-project.org/lppl.txt}
and version 1.3 or later is part of all distributions of \LaTeX{}
version 2005/12/01 or later.

This work has the LPPL maintenance status `maintained'.

The Current Maintainer of this work is Niklas Beisert.

This work consists of the files |README.txt|, |childdoc.ins| and |childdoc.dtx|
as well as the derived files |childdoc.def|, |cdocsamp.tex|
with |cdocsch1.tex|, |cdocsch2.tex|, |cdocspt3.tex|, |cdocspt4.tex|,
|cdocsdrf.tex|, |cdocsfn1.tex|, |cdocsfn2.tex|
as well as |childdoc.pdf|.

%%%%%%%%%%%%%%%%%%%%%%%%%%%%%%%%%%%%%%%%%%%%%%%%%%%%%%%%%%%%%%%%%%%%%%%%%%%%%%%%
\subsection{Files and Installation}

The package consists of the files:
%
\begin{center}
\begin{tabular}{ll}
    |README.txt|   & readme file \\
    |childdoc.ins| & installation file \\
    |childdoc.dtx| & source file \\
    |childdoc.def| & definition file \\
    |cdocsamp.tex| & sample main file \\
    |cdocsch1.tex| & sample include file \\
    |cdocsch2.tex| & sample include file \\
    |cdocspt3.tex| & sample part file \\
    |cdocspt4.tex| & sample part file \\
    |cdocsdrf.tex| & sample redirection file \\
    |cdocsfn1.tex| & sample redirection file \\
    |cdocsfn2.tex| & sample redirection file \\
    |childdoc.pdf| & manual
\end{tabular}
\end{center}
%
The distribution consists of the files
|README.txt|, |childdoc.ins| and |childdoc.dtx|.
%
\begin{itemize}
\item
Run (pdf)\LaTeX{} on |childdoc.dtx|
to compile the manual |childdoc.pdf| (this file).
\item
Run \LaTeX{} on |childdoc.ins| to create the definitions file |childdoc.def|
and the sample |cdocsamp.tex| with include files
|cdocsch1.tex|, |cdocsch2.tex|, |cdocspt3.tex|, |cdocspt4.tex|,
|cdocsdrf.tex|, |cdocsfn1.tex|, |cdocsfn2.tex|.
Then copy the file |childdoc.def| to an appropriate directory of your \LaTeX{}
distribution, e.g.\ \textit{texmf-root}|/tex/latex/childdoc|.
\end{itemize}

%%%%%%%%%%%%%%%%%%%%%%%%%%%%%%%%%%%%%%%%%%%%%%%%%%%%%%%%%%%%%%%%%%%%%%%%%%%%%%%%
\subsection{Related CTAN Packages}

There are several other packages which offer a similar functionality:
%
\begin{itemize}
\item
The packages
\href{http://ctan.org/pkg/docmute}{\textsf{docmute}},
\href{http://ctan.org/pkg/includex}{\textsf{includex}} and
\href{http://ctan.org/pkg/standalone}{\textsf{standalone}}
provide commands to include only the document body of
a child file thus allowing both files to be compiled individually.
\item
The packages \href{http://ctan.org/pkg/subdocs}{\textsf{subdocs}}
and \href{http://ctan.org/pkg/subfiles}{\textsf{subfiles}}
provide structures in which the main and child documents can be
encapsulated and allowing them to be compiled individually.
The inclusion mechanism is different from the conventional |\include|.
\item
The package \href{http://ctan.org/pkg/combine}{\textsf{combine}}
is an elaborate solution to combine several documents into one.
\end{itemize}
%
See also the CTAN topic \href{http://ctan.org/topic/subdocs}{\textsf{subdocs}}
for further related packages.
The present package differs from the above solutions in that
a document structure constructed with the conventional |\include| mechanism
just needs two extra commands at the top of every file
such that all constituent files can be compiled individually.

%%%%%%%%%%%%%%%%%%%%%%%%%%%%%%%%%%%%%%%%%%%%%%%%%%%%%%%%%%%%%%%%%%%%%%%%%%%%%%%%
%\subsection{Feature Suggestions}
%
%The following is a list of features which may be useful for future
%versions of this package:
%%
%\begin{itemize}
%\item
%\ldots
%\end{itemize}

%%%%%%%%%%%%%%%%%%%%%%%%%%%%%%%%%%%%%%%%%%%%%%%%%%%%%%%%%%%%%%%%%%%%%%%%%%%%%%%%
\subsection{Revision History}

%%%%%%%%%%%%%%%%%%%%%%%%%%%%%%%%%%%%%%%%
\paragraph{v2.0:} 2018/12/30

\begin{itemize}
\item
immediate forward processing
\item
added |\childdocby| mechanism
\item
manual restructured
\end{itemize}

%%%%%%%%%%%%%%%%%%%%%%%%%%%%%%%%%%%%%%%%
\paragraph{v1.6:} 2018/01/17

\begin{itemize}
\item
application for development of include files
\item
corrections to manual
\end{itemize}

%%%%%%%%%%%%%%%%%%%%%%%%%%%%%%%%%%%%%%%%
\paragraph{v1.5:} 2017/05/21

\begin{itemize}
\item
more complete structuring introduced
\item
|\childdocof| introduced
\item
|\childdoc| renamed to |\childdocmain|
\item
|\childredirect| renamed to |\childdocforward| and |\childdocforwardprefix|
and functionality expanded
\end{itemize}

%%%%%%%%%%%%%%%%%%%%%%%%%%%%%%%%%%%%%%%%
\paragraph{v1.0:} 2017/04/27

\begin{itemize}
\item
manual and install package
\item
first version published on CTAN
\end{itemize}

%%%%%%%%%%%%%%%%%%%%%%%%%%%%%%%%%%%%%%%%
\paragraph{v0.6:} 2017/04/26

\begin{itemize}
\item
redirection mechanism added
\end{itemize}

%%%%%%%%%%%%%%%%%%%%%%%%%%%%%%%%%%%%%%%%
\paragraph{v0.5:} 2017/04/26

\begin{itemize}
\item
functionality in definition file
\end{itemize}


%%%%%%%%%%%%%%%%%%%%%%%%%%%%%%%%%%%%%%%%%%%%%%%%%%%%%%%%%%%%%%%%%%%%%%%%%%%%%%%%
%%%%%%%%%%%%%%%%%%%%%%%%%%%%%%%%%%%%%%%%%%%%%%%%%%%%%%%%%%%%%%%%%%%%%%%%%%%%%%%%
%%%%%%%%%%%%%%%%%%%%%%%%%%%%%%%%%%%%%%%%%%%%%%%%%%%%%%%%%%%%%%%%%%%%%%%%%%%%%%%%
\appendix

\settowidth\MacroIndent{\rmfamily\scriptsize 000\ }

 \DocInput{childdoc.dtx}

\end{document}
%</driver>
% \fi
%
% %%%%%%%%%%%%%%%%%%%%%%%%%%%%%%%%%%%%%%%%%%%%%%%%%%%%%%%%%%%%%%%%%%%%%%%%%%%%%%
% %%%%%%%%%%%%%%%%%%%%%%%%%%%%%%%%%%%%%%%%%%%%%%%%%%%%%%%%%%%%%%%%%%%%%%%%%%%%%%
% \section{Sample}
%\iffalse
%<*samplemain>
%\fi
%
% The following presents a sample document
% with two chapters, two parts, a title page,
% a compile flag as well as three forwarding files to set the flag.
% It consists of eight |.tex| files:
% \begin{center}
% \begin{tabular}{ll}
% |cdocsamp.tex|&main file\\
% |cdocsch1.tex|&include file for chapter 1\\
% |cdocsch2.tex|&include file for chapter 2\\
% |cdocspt3.tex|&include file for part 3\\
% |cdocspt4.tex|&include file for part 4\\
% |cdocsdrf.tex|&forwarding file for main file in draft mode\\
% |cdocsfi1.tex|&forwarding file for final version of chapter 1\\
% |cdocsfi2.tex|&forwarding file for final version of chapter 2\\
% \end{tabular}
% \end{center}
% Each of the eight files can be compiled directly by the \LaTeX{} compiler.
%
% %%%%%%%%%%%%%%%%%%%%%%%%%%%%%%%%%%%%%%
% \paragraph{Main File.}
%
% The main file is called |cdocsamp.tex|.
%
% Load the \textsf{childdoc} definitions and
% declare the filename for the main document:
%    \begin{macrocode}
\input{childdoc.def}
\childdocmain{}
%    \end{macrocode}

% Optional override for |\version| flag:
%    \begin{macrocode}
%%\ifchilddoc\else\providecommand{\version}{draft}\fi
%    \end{macrocode}

% Define the default values for the |\version| flag
% (|final| for the main file and |draft| for childs):
%    \begin{macrocode}
\ifchilddoc
\providecommand{\version}{draft}
\else
\providecommand{\version}{final}
\fi
%    \end{macrocode}

% Load the standard document class:
%    \begin{macrocode}
\documentclass[12pt]{article}
%    \end{macrocode}

% Start the document body:
%    \begin{macrocode}
\begin{document}
%    \end{macrocode}

% Declare a title page.
% Print title, part of document being processed and version flag:
%    \begin{macrocode}
\addtocounter{page}{-1}
\begin{center}
{\LARGE\bfseries{}childdoc example\par}
\vspace{1cm}
\ifchilddoc
\ifchilddocmanual part\else chapter\fi:
`\childdocname' of `\childdocjob'\par
\else
main document: `\childdocjob'\par
\fi
version: \version\par
\end{center}
\newpage
%    \end{macrocode}

% Manually include selected file,
% otherwise process as usual:
%    \begin{macrocode}
\ifchilddocmanual
\section*{part `\childdocname'}
\input{\childdocname}
\else
%    \end{macrocode}

% Include the two chapters:
%    \begin{macrocode}
\include{cdocsch1}
\include{cdocsch2}
%    \end{macrocode}

% Include the two parts unless only chapters should be displayed:
%    \begin{macrocode}
\ifchilddoc\else
\section{part three}
\input{cdocspt3}
\section{part four}
\input{cdocspt4}
\fi
%    \end{macrocode}

% Process as usual until here:
%    \begin{macrocode}
\fi
%    \end{macrocode}

% End of document body:
%    \begin{macrocode}
\end{document}
%    \end{macrocode}
%\iffalse
%</samplemain>
%\fi
%
% %%%%%%%%%%%%%%%%%%%%%%%%%%%%%%%%%%%%%%
% \paragraph{Chapter Include Files.}
%
% The include files are called |cdocsch1.tex| and |cdocsch2.tex|.
%
%\iffalse
%<*samplechap1|samplechap2>
%\fi

% Optional override for |\version| flag:
%    \begin{macrocode}
%%\providecommand{\version}{final}
%    \end{macrocode}

% Include the main document:
%    \begin{macrocode}
\input{childdoc.def}
\childdocof{cdocsamp}
%    \end{macrocode}

%\iffalse
%</samplechap1|samplechap2>
%\fi
%
%\iffalse
%<*samplechap1>
%\fi
% Some text for chapter 1:
%    \begin{macrocode}
\section{one}
some text in chapter one
%    \end{macrocode}

%\iffalse
%</samplechap1>
%\fi
% Some text for chapter 2:
%\iffalse
%<*samplechap2>
%\fi
%    \begin{macrocode}
\section{two}
more text in chapter two
%    \end{macrocode}

%\iffalse
%</samplechap2>
%\fi
%
% %%%%%%%%%%%%%%%%%%%%%%%%%%%%%%%%%%%%%%
% \paragraph{Part Include Files.}
%
% The include files are called |cdocspt3.tex| and |cdocspt4.tex|.
%
%\iffalse
%<*samplepart3|samplepart4>
%\fi

% Optional override for |\version| flag:
%    \begin{macrocode}
%%\providecommand{\version}{final}
%    \end{macrocode}

% Include the main document:
%    \begin{macrocode}
\input{childdoc.def}
\childdocby{cdocsamp}
%    \end{macrocode}

%\iffalse
%</samplepart3|samplepart4>
%\fi
%
%\iffalse
%<*samplepart3>
%\fi
% Some text for part 3:
%    \begin{macrocode}
some text in part three
%    \end{macrocode}

%\iffalse
%</samplepart3>
%\fi
% Some text for part 4:
%\iffalse
%<*samplepart4>
%\fi
%    \begin{macrocode}
more text in part four
%    \end{macrocode}

%\iffalse
%</samplepart4>
%\fi
%
% %%%%%%%%%%%%%%%%%%%%%%%%%%%%%%%%%%%%%%
% \paragraph{Forwarding for a Complete Draft.}
%
% The following forwarding file |cdocsdrf.tex|
% compiles the main document in draft mode:
%\iffalse
%<*sampledraft>
%\fi
%    \begin{macrocode}
\def\version{draft}
\input{childdoc.def}
\childdocforward{cdocsamp}
%    \end{macrocode}

%\iffalse
%</sampledraft>
%\fi
%
% %%%%%%%%%%%%%%%%%%%%%%%%%%%%%%%%%%%%%%
% \paragraph{Forwarding for Final Version of the Chapters.}
%
% The following forwarding files |cdocsfn1.tex| and |cdocsfn2.tex|
% (with identical content)
% compile the final versions of the child documents
% |cdocsch1.tex| and |cdocsch2.tex|, respectively:
%\iffalse
%<*samplefinal>
%\fi
%    \begin{macrocode}
\def\version{final}
\input{childdoc.def}
\childdocforwardprefix[cdocsamp]{cdocsfn}{cdocsch}
%    \end{macrocode}

%\iffalse
%</samplefinal>
%\fi
%
% %%%%%%%%%%%%%%%%%%%%%%%%%%%%%%%%%%%%%%
% \paragraph{Command Line Processing.}
%
% The following three command lines generate the output files
% |cdocscld|, |cdocscl1| and |cdocscl2|
% which should be identical to
% |cdocsdrf|, |cdocsch1| and |cdocsfn2|, respectively:
% \begin{center}
% \begin{tabular}{l}
% |latex -jobname cdocscld \|\\
% |  "\def\version{draft}\input{childdoc.def}\childdocforward{cdocsamp}"|\\
% |latex -jobname cdocscl1 \|\\
% |  "\input{childdoc.def}\childdocforward[cdocsamp]{cdocsch1}"|\\
% |latex -jobname cdocscl2 \|\\
% |  "\def\version{final}\input{childdoc.def}\childdocforward{cdocsch2}"|
% \end{tabular}
% \end{center}
% Note that the trailing backslash on each first line
% merely continues the input to the second line
% (for convenient cut ant paste).
% Furthermore, the command |latex| can be replaced by any
% of its alternative versions such as |pdflatex|.
%
% %%%%%%%%%%%%%%%%%%%%%%%%%%%%%%%%%%%%%%%%%%%%%%%%%%%%%%%%%%%%%%%%%%%%%%%%%%%%%%
% %%%%%%%%%%%%%%%%%%%%%%%%%%%%%%%%%%%%%%%%%%%%%%%%%%%%%%%%%%%%%%%%%%%%%%%%%%%%%%
% \section{Implementation}
%\iffalse
%<*package>
%\fi
%
% This section describes the definitions file |childdoc.def|.

% The definitions cannot be loaded using |\usepackage| or |\RequirePackage|
% which has a mechanism to prevent loading a style file more than once.
% When loading the definitions by means of |\input|
% multiple instances have to be prevented manually:
%\iffalse
%This code needs to be before the `\ProvidesFile' directive
%which is defined at the beginning of this file.
%Therefore it is also placed there and commented out here.
%</package>
%<*discard>
%\fi
%    \begin{macrocode}
\ifdefined\childdocmain\endinput\fi
%    \end{macrocode}
%\iffalse
%</discard>
%<*package>
%\fi
%
% \macro{\ifchilddoc}
% \macro{\ifchilddocmanual}
% The conditional |\ifchilddoc| tells whether a
% child (true) or main (false) document is being compiled.
% The conditional |\ifchilddocmanual| tells whether
% the |\includeonly| mechanism is used (false) or
% the selection of child files must be performed manually (true).
% The definitions initialise to false:
%    \begin{macrocode}
\newif\ifchilddoc
\newif\ifchilddocmanual
%    \end{macrocode}

% \macro{\childdocname}
% \macro{\childdocjob}
% The macro |\childdocname| stores the name of the main document
% to be compiled. The macro |\childdocjob| stores the name of
% the document on which the \LaTeX{} compiler was originally invoked.
% The content of |\jobname| cannot be compared
% to filenames specified in the source due to different catcodes.
% The following code rescans |\jobname|, stores the result
% in |\childdocname| and saves a copy in |\childdocjob|:
%    \begin{macrocode}
\edef\childdocname{\scantokens\expandafter{\jobname\noexpand}}
\let\childdocjob\childdocname
%    \end{macrocode}

% \macro{\childdocdisable}
% The macro |\childdocdisable| prevents the main file
% from being processed more than once.
% At this stage, the main document command |\childdocmain|
% is assumed to be called once again where it should do nothing.
% Any subsequent call to it should prevent
% a secondary processing of the main document
% It overwrites the forwarding commands
% |\childdocof| and |\childdocforward|
% with empty macros to prevent further inclusions of the main document:
%    \begin{macrocode}
\newcommand{\childdocdisable}
{
  \renewcommand{\childdocmain}[1]{\renewcommand{\childdocmain}[1]{\endinput}}
  \renewcommand{\childdocof}[1]{}
  \renewcommand{\childdocby}[2][]{}
  \renewcommand{\childdocforward}[2][]{}
  \renewcommand{\childdocdisable}{}
}
%    \end{macrocode}

% \macro{\childdocmain}
% The macro |\childdocmain| is to be called at the top of the main file
% with nothing or the main filename (without extension) as argument.
% First, it breaks loops.
% If the argument is not empty and does not match |\childdocname|
% (which is set by the first inclusion of |childdoc.def|),
% |\ifchilddoc| is set to true, |\includeonly| is applied to the child file
% and |\jobname| is set to the main file
% (for proper handling of |.aux| files):
%    \begin{macrocode}
\newcommand{\childdocmain}[1]
{
  \childdocdisable\childdocmain{}
  \if?#1?\else
    \begingroup
      \def\childdoctmp{#1}
      \ifx\childdoctmp\childdocname
        \def\childdoctmp{}
      \else
        \def\childdoctmp
        {
          \childdoctrue
          \includeonly{\childdocname}
          \def\childdocjob{#1}
          \def\jobname{#1}
        }
      \fi
      \expandafter
    \endgroup
    \childdoctmp
  \fi
}
%    \end{macrocode}

% \macro{\childdocof}
% The command |\childdocof| redirects
% compilation to the main file |#1|.
%    \begin{macrocode}
\newcommand{\childdocof}[1]
{
  \childdocdisable
  \childdoctrue
  \includeonly{\childdocname}
  \def\jobname{#1}
  \def\childdocjob{#1}
  \input{#1}
}
%    \end{macrocode}

% \macro{\childdocby}
% The command |\childdocby| ....
%    \begin{macrocode}
\newcommand{\childdocby}[2][]
{
  \childdocdisable
  \childdoctrue
  \childdocmanualtrue
  \if?#1?\else
    \def\jobname{#2}
  \fi
  \def\childdocjob{#2}
  \input{#2}
  \endinput
}
%    \end{macrocode}

% \macro{\childdocforward}
% The command |\childdocforward| redirects
% compilation to the main file or
% (if the optional argument is given) a child file.
% Parameters are set as if the main file
% or a child file starting with |\childdocof| was compiled.
% Then compilation is handed over to the main file:
%    \begin{macrocode}
\newcommand{\childdocforward}[2][]
{
  \begingroup
    \if?#1?
      \def\childdoctmp
      {
        \def\childdocname{#2}
        \def\childdocjob{#2}
        \def\jobname{#2}
        \input{#2}
        \endinput
      }
    \else
      \def\childdoctmp
      {
        \childdocdisable
        \def\childdocname{#2}
        \childdoctrue
        \includeonly{#2}
        \def\childdocjob{#1}
        \def\jobname{#1}
        \input{#1}
        \endinput
      }
    \fi
    \expandafter
  \endgroup
  \childdoctmp
}
%    \end{macrocode}

% \macro{\childdocforwardprefix}
% The command |\childdocforwardprefix| redirects
% compilation to the main or a child file by means of a pattern.
% The prefix |#1| in the current filename is replaced by |#2|
% and the suffix of the current filename is kept
% (it is assumed that the filename does not contain the substring `|~~~|'
% which is used as a delimiter).
% Compilation is handed over to the new file by |\childdocforward|:
%    \begin{macrocode}
\newcommand{\childdocforwardprefix}[3][]
{
  \begingroup
    \def\childdocextract #2##1~~~{\def\childdoctmp{\childdocforward[#1]{#3##1}}}
    \expandafter\childdocextract\childdocname~~~
    \expandafter
  \endgroup
  \childdoctmp
}
%    \end{macrocode}

% \macro{\childdoc}
% The deprecated macro |\childdoc| is a legacy version of |\childdocmain|:
%    \begin{macrocode}
\newcommand{\childdoc}{\childdocmain}
%    \end{macrocode}

% \macro{\childdocredirect}
% The deprecated macro |\childdocredirect| is a legacy version
% of |\childdocforward| and |\childdocforwardprefix|:
%    \begin{macrocode}
\newcommand{\childdocredirect}[2][]
{
  \begingroup
    \if?#1?
      \def\childdoctmp{\childdocforward{#2}}
    \else
      \def\childdoctmp{\childdocforwardprefix{#1}{#2}}
    \fi
    \expandafter
  \endgroup
  \childdoctmp
}
%    \end{macrocode}

%\iffalse
%</package>
%\fi
%
\endinput
|\\
|\childdocof{|\textit{main}|}|\\
\end{tabular}
\end{center}
at the top of every child file \textit{child}
which is included by |\include{|\textit{child}|}|
from within the main file
(or at least for those files to be compiled individually).
The argument \textit{main} must be the filename of the main file.

There are a couple of
considerations in setting up the main and child documents:

%%%%%%%%%%%%%%%%%%%%%%%%%%%%%%%%%%%%%%%%
\paragraph{Restrictions.}

Please note the following restrictions:
\begin{itemize}
\item
|\childdocmain| must be called with one argument \textit{main}
to ensure compatibility with earlier version of the package.
It must either be empty (|\childdocmain{}|)
or precisely match the filename of the main file in which it is specified.
See \secref{sec:detection} for further information.
\item
The filename \textit{main} must be specified without the |.tex| extension.
\item
The filename \textit{main} is case sensitive
(even in case-insensitive file systems)
due to internal string comparison.
\item
The argument \textit{main} should be fully expanded, it cannot be a macro.
\item
Subdirectories and special characters should be avoided in filenames.
\item
The command |\childdocmain{|\textit{main}|}| must be followed by a whitespace.
It should not be followed immediately by another command
or by a comment mark `|%|'.
This is because the \TeX{} parser reads the token immediately following
the argument of |\childdocmain| and puts it
at the beginning of every child section;
however, a white\-space is ignored.
\end{itemize}

%%%%%%%%%%%%%%%%%%%%%%%%%%%%%%%%%%%%%%%%
\paragraph{Content of Main File.}

It is advisable to place all content in the child files included by |\include|.
Any output contained in the main file will appear in all child documents
unless suppressed manually;
it cannot be suppressed automatically by the |\includeonly| directive
and thus should normally be avoided.
A method to include some content in the main file
by means of conditional processing is described in \secref{sec:conditional}.

%%%%%%%%%%%%%%%%%%%%%%%%%%%%%%%%%%%%%%%%
\paragraph{Page Numbering.}

When only a part of the document is compiled,
the appropriate numbering of pages
(as well as other status parameters)
is determined from the |.aux| files.
The latter contain information from previous passes.
However this information needs to propagate through
all intermediate child documents.
Therefore the page numbering in child documents may well
be inconsistent until the complete document is compiled at least once.

A useful (if unconventional) way to always ensure a consistent
page numbering is to restart the numbering in each child document
and denote the pages by `\textit{child}|.|\textit{page}'
where \textit{child} represents the chapter/section number of the child file.
This can be achieved by the command
|\numberwithin{page}{|\textit{child}|}|
of the \textsf{amsmath} package
where \textit{child} can be |chapter| or |section|
depending on the chosen structuring.
Alternatively, one can modify the macro |\thepage| appropriately
and reset the counter |page| at the start of each child file.

%%%%%%%%%%%%%%%%%%%%%%%%%%%%%%%%%%%%%%%%%%%%%%%%%%%%%%%%%%%%%%%%%%%%%%%%%%%%%%%%
\subsection{Conditional Processing}
\label{sec:conditional}

The package provides a mechanism to compile different versions
of a document. To customise the versions further some conditional processing
can come in handy to distinguish which version is being compiled.
The package provides two macros to describe the compilation context:

%%%%%%%%%%%%%%%%%%%%%%%%%%%%%%%%%%%%%%%%
\DescribeMacro{\ifchilddoc}
The conditional |\ifchilddoc| distinguishes between the compilation of
child documents and the main document:
%
\begin{center}
|\ifchilddoc |\textit{child-code}| |[|\||else |\textit{main-code}]| \||fi|
\end{center}

%%%%%%%%%%%%%%%%%%%%%%%%%%%%%%%%%%%%%%%%
\DescribeMacro{\childdocname}
\DescribeMacro{\childdocjob}
The macro |\childdocname| contains the filename (without extension)
of the main or child file being processed.
Note that |\childdocjob| will always contain the name of the main file.

%%%%%%%%%%%%%%%%%%%%%%%%%%%%%%%%%%%%%%%%
\paragraph{Title Page.}

Conditional processing can be used to include a title or banner page
in the main document when proper precautions are taken.
Importantly, the code in the main file should ensure that the page counter
(as well as other status parameters which are stored in the |.aux| files)
takes the same value after the conditional processing.
Otherwise the page numbers may take divergent values
depending on which part is compiled.

For example, a title page could be declared by:
%
\begin{center}
\begin{tabular}{l}
|\ifchilddoc\||else|\\
|\addtocounter{page}{-1}|\\
\textit{code for title page}\\
|\newpage|\\
|\||fi|
\end{tabular}
\end{center}
%
A banner page for the child documents can be generated by:
%
\begin{center}
\begin{tabular}{l}
|\ifchilddoc|\\
|\addtocounter{page}{-1}|\\
\textit{code for banner page}\\
|\newpage|\\
|\||fi|
\end{tabular}
\end{center}
%
Here one could write a message such as:
\begin{center}
|This is the part \childdocname{} of \childdocjob{}.|
\end{center}

%%%%%%%%%%%%%%%%%%%%%%%%%%%%%%%%%%%%%%%%%%%%%%%%%%%%%%%%%%%%%%%%%%%%%%%%%%%%%%%%
\subsection{Flags}
\label{sec:flags}

The package makes it easy to generate different versions
of the main or child documents.
To this end compilation flags can be defined
and assigned different default values.
They will be particularly useful in conjunction
with the forwarding mechanism described in \secref{sec:forward}.

For example, it may be useful to have a flag |\version|
which can be set to |draft| or |final|.
The document source will contain some conditional code
depending on the value of |\version|.
Suppose further, the flag should default to |final| for the main file
and to |draft| for child files
which is a natural assignment for editing the document.
This is achieved by placing the following code
in the preamble of the main document
(below the |\childdocmain| directive):
%
\begin{center}
\begin{tabular}{l}
|\ifchilddoc|\\
|\providecommand{\version}{draft}|\\
|\||else|\\
|\providecommand{\version}{final}|\\
|\||fi|
\end{tabular}
\end{center}
%
The definition by |\providecommand| makes sure
that previous definitions are not overwritten.
Further statements |\providecommand{\version}{...}|
can thus be added before the above code to override it.

For the main file, one might add a line
(between |\childdocmain| and the above block)
%
\begin{center}
|%\ifchilddoc\||else\providecommand{\version}{draft}\||fi|
\end{center}
%
which can be uncommented to produce a draft version.
Likewise one can add a line to the very top of a child file
(above the |\childdocof{|\textit{main}|}| directive)
%
\begin{center}
|%\providecommand{\version}{final}|
\end{center}
%
which can be uncommented to produce the final version of this child document.

%%%%%%%%%%%%%%%%%%%%%%%%%%%%%%%%%%%%%%%%%%%%%%%%%%%%%%%%%%%%%%%%%%%%%%%%%%%%%%%%
\subsection{Forwarding}
\label{sec:forward}

Different versions of the main or child documents
using compilation flags as described in \secref{sec:flags}
can be (permanently) stored in different files
for convenient compilation, viewing and distribution.
To this end, the package defines a command
to pass on compilation to a different file:

%%%%%%%%%%%%%%%%%%%%%%%%%%%%%%%%%%%%%%%%
\DescribeMacro{\childdocforward}
The command |\childdocforward| redirects processing to
another source file:
%
\begin{center}
\begin{tabular}{l}
|% \iffalse
%
% childdoc.dtx Copyright (C) 2017-2018 Niklas Beisert
%
% This work may be distributed and/or modified under the
% conditions of the LaTeX Project Public License, either version 1.3
% of this license or (at your option) any later version.
% The latest version of this license is in
%   http://www.latex-project.org/lppl.txt
% and version 1.3 or later is part of all distributions of LaTeX
% version 2005/12/01 or later.
%
% This work has the LPPL maintenance status `maintained'.
%
% The Current Maintainer of this work is Niklas Beisert.
%
% This work consists of the files childdoc.dtx and childdoc.ins
% and the derived files childdoc.def and cdocsamp.tex with
% cdocsch1.tex, cdocsch2.tex, cdocsdrf.tex, cdocsfn1.tex, cdocsfn2.tex.
%
%<package>\ifdefined\childdocmain\endinput\fi
%<package>\ProvidesFile{childdoc.def}[2018/12/30 v2.0 child document driver]
%<samplemain>\ProvidesFile{cdocsamp.tex}[2018/12/30 v2.0 sample for childdoc]
%<*driver>
%\ProvidesFile{childdoc.drv}[2018/12/30 v2.0 childdoc reference manual file]
\PassOptionsToClass{10pt,a4paper}{article}
\documentclass{ltxdoc}

\usepackage[margin=35mm]{geometry}
\usepackage{hyperref}
\usepackage{hyperxmp}
\usepackage[usenames]{color}

\hypersetup{colorlinks=true}
\hypersetup{pdfstartview=FitH}
\hypersetup{pdfpagemode=UseNone}
\hypersetup{pdfsource={}}
\hypersetup{pdflang={en-UK}}
\hypersetup{pdfcopyright={Copyright 2017-2018 Niklas Beisert.
  This work may be distributed and/or modified under the
  conditions of the LaTeX Project Public License, either version 1.3
  of this license or (at your option) any later version.}}
\hypersetup{pdflicenseurl={http://www.latex-project.org/lppl.txt}}
\hypersetup{pdfcontactaddress={ETH Zurich, ITP, HIT K,
  Wolfgang-Pauli-Strasse 27}}
\hypersetup{pdfcontactpostcode={8093}}
\hypersetup{pdfcontactcity={Zurich}}
\hypersetup{pdfcontactcountry={Switzerland}}
\hypersetup{pdfcontactemail={nbeisert@itp.phys.ethz.ch}}
\hypersetup{pdfcontacturl={http://people.phys.ethz.ch/\xmptilde nbeisert/}}

\newcommand{\secref}[1]{\hyperref[#1]{section \ref*{#1}}}

\parskip1ex
\parindent0pt
\let\olditemize\itemize
\def\itemize{\olditemize\parskip0pt}

\begin{document}

\title{The \textsf{childdoc} Package}
\hypersetup{pdftitle={The childdoc Package}}
\author{Niklas Beisert\\[2ex]
  Institut f\"ur Theoretische Physik\\
  Eidgen\"ossische Technische Hochschule Z\"urich\\
  Wolfgang-Pauli-Strasse 27, 8093 Z\"urich, Switzerland\\[1ex]
  \href{mailto:nbeisert@itp.phys.ethz.ch}
  {\texttt{nbeisert@itp.phys.ethz.ch}}}
\hypersetup{pdfauthor={Niklas Beisert}}
\hypersetup{pdfsubject={Manual for the LaTeX2e Package childdoc}}
\date{30 December 2018, \textsf{v2.0}}
\maketitle

\begin{abstract}\noindent
\textsf{childdoc} is a \LaTeXe{} package
that enables the direct compilation
of document sections included by |\include|
to individual files.
\end{abstract}

\begingroup
\parskip0ex
\tableofcontents
\endgroup

%%%%%%%%%%%%%%%%%%%%%%%%%%%%%%%%%%%%%%%%%%%%%%%%%%%%%%%%%%%%%%%%%%%%%%%%%%%%%%%%
%%%%%%%%%%%%%%%%%%%%%%%%%%%%%%%%%%%%%%%%%%%%%%%%%%%%%%%%%%%%%%%%%%%%%%%%%%%%%%%%
\section{Introduction}

\LaTeX{} provides a mechanism to structure a large document (such as a book)
into a main file and several child files (containing the chapters)
using the |\include| command.
This mechanism is beneficial for documents
which span hundreds of pages in order to
make the source file(s) more manageable.
Moreover, compilation can be restricted to
selected child files by means of the |\includeonly| command.
The latter feature can be used to reduce the compilation time while editing
(this was significantly more useful in the earlier days of \LaTeX{})
or to generate a smaller document which is easier to navigate.
Another application of |\includeonly| is to generate
documents consisting of selected parts of the complete document.

However, there are a few drawbacks of the plain |\include| mechanism:
\begin{itemize}
\item
The child files cannot be compiled on their own,
they can only be compiled via the main file.
A naive editing environment
(such as a text editor with an option
to have the current file processed by \LaTeX)
may require one to switch to the main file before compiling;
attempting to compile the child file produces errors.
\item
The main file must be modified (each time)
to adjust the |\includeonly| command
to the present needs. This easily leaves the main file in a messy state.
\item
The generated document will always carry the filename
of the main document. This is inconvenient if
several child files are to be compiled and
to be kept for distribution.
\end{itemize}

The present package provides a simple interface
to make child files individually compilable by \LaTeX{}.
Compiling a child file then has the same effect as compiling
the main file with an |\includeonly| command
to select the appropriate child.
Moreover the generated document will carry the name of the child
rather than the main file.
This resolves all three above issues.

This feature is meant to make the editing of books,
thesis documents and lecture notes somewhat more convenient.
However, the package can also be used efficiently for
composing a series of documents (such as exercise sheets)
which are typically distributed individually.
It then assists the author in generating the individual documents
(potentially in different versions)
as well as a document containing the collected series.
Another application is in developing style files
or other kinds of included material
where compilation of the style file could redirect
to a sample or test file.

%%%%%%%%%%%%%%%%%%%%%%%%%%%%%%%%%%%%%%%%%%%%%%%%%%%%%%%%%%%%%%%%%%%%%%%%%%%%%%%%
%%%%%%%%%%%%%%%%%%%%%%%%%%%%%%%%%%%%%%%%%%%%%%%%%%%%%%%%%%%%%%%%%%%%%%%%%%%%%%%%
\section{Usage}

First of all, the package \textsf{childdoc} is \emph{not} a standard
\LaTeXe{} |.sty| style file! Therefore it needs to be invoked in
a non-standard way.

%%%%%%%%%%%%%%%%%%%%%%%%%%%%%%%%%%%%%%%%%%%%%%%%%%%%%%%%%%%%%%%%%%%%%%%%%%%%%%%%
\subsection{Included Files}
\label{sec:include}

%%%%%%%%%%%%%%%%%%%%%%%%%%%%%%%%%%%%%%%%
\DescribeMacro{\childdocmain}
To use the package, add the commands
\begin{center}
\begin{tabular}{l}
|\input{childdoc.def}|\\
|\childdocmain{}|\\
\end{tabular}
\end{center}
at the very top of the main \LaTeX{} file,
in particular \emph{before} the |\documentclass| statement!
The argument of |\childdocmain| should be left empty
(but it must be present).

%%%%%%%%%%%%%%%%%%%%%%%%%%%%%%%%%%%%%%%%
\DescribeMacro{\childdocof}
Furthermore, add the commands
\begin{center}
\begin{tabular}{l}
|\input{childdoc.def}|\\
|\childdocof{|\textit{main}|}|\\
\end{tabular}
\end{center}
at the top of every child file \textit{child}
which is included by |\include{|\textit{child}|}|
from within the main file
(or at least for those files to be compiled individually).
The argument \textit{main} must be the filename of the main file.

There are a couple of
considerations in setting up the main and child documents:

%%%%%%%%%%%%%%%%%%%%%%%%%%%%%%%%%%%%%%%%
\paragraph{Restrictions.}

Please note the following restrictions:
\begin{itemize}
\item
|\childdocmain| must be called with one argument \textit{main}
to ensure compatibility with earlier version of the package.
It must either be empty (|\childdocmain{}|)
or precisely match the filename of the main file in which it is specified.
See \secref{sec:detection} for further information.
\item
The filename \textit{main} must be specified without the |.tex| extension.
\item
The filename \textit{main} is case sensitive
(even in case-insensitive file systems)
due to internal string comparison.
\item
The argument \textit{main} should be fully expanded, it cannot be a macro.
\item
Subdirectories and special characters should be avoided in filenames.
\item
The command |\childdocmain{|\textit{main}|}| must be followed by a whitespace.
It should not be followed immediately by another command
or by a comment mark `|%|'.
This is because the \TeX{} parser reads the token immediately following
the argument of |\childdocmain| and puts it
at the beginning of every child section;
however, a white\-space is ignored.
\end{itemize}

%%%%%%%%%%%%%%%%%%%%%%%%%%%%%%%%%%%%%%%%
\paragraph{Content of Main File.}

It is advisable to place all content in the child files included by |\include|.
Any output contained in the main file will appear in all child documents
unless suppressed manually;
it cannot be suppressed automatically by the |\includeonly| directive
and thus should normally be avoided.
A method to include some content in the main file
by means of conditional processing is described in \secref{sec:conditional}.

%%%%%%%%%%%%%%%%%%%%%%%%%%%%%%%%%%%%%%%%
\paragraph{Page Numbering.}

When only a part of the document is compiled,
the appropriate numbering of pages
(as well as other status parameters)
is determined from the |.aux| files.
The latter contain information from previous passes.
However this information needs to propagate through
all intermediate child documents.
Therefore the page numbering in child documents may well
be inconsistent until the complete document is compiled at least once.

A useful (if unconventional) way to always ensure a consistent
page numbering is to restart the numbering in each child document
and denote the pages by `\textit{child}|.|\textit{page}'
where \textit{child} represents the chapter/section number of the child file.
This can be achieved by the command
|\numberwithin{page}{|\textit{child}|}|
of the \textsf{amsmath} package
where \textit{child} can be |chapter| or |section|
depending on the chosen structuring.
Alternatively, one can modify the macro |\thepage| appropriately
and reset the counter |page| at the start of each child file.

%%%%%%%%%%%%%%%%%%%%%%%%%%%%%%%%%%%%%%%%%%%%%%%%%%%%%%%%%%%%%%%%%%%%%%%%%%%%%%%%
\subsection{Conditional Processing}
\label{sec:conditional}

The package provides a mechanism to compile different versions
of a document. To customise the versions further some conditional processing
can come in handy to distinguish which version is being compiled.
The package provides two macros to describe the compilation context:

%%%%%%%%%%%%%%%%%%%%%%%%%%%%%%%%%%%%%%%%
\DescribeMacro{\ifchilddoc}
The conditional |\ifchilddoc| distinguishes between the compilation of
child documents and the main document:
%
\begin{center}
|\ifchilddoc |\textit{child-code}| |[|\||else |\textit{main-code}]| \||fi|
\end{center}

%%%%%%%%%%%%%%%%%%%%%%%%%%%%%%%%%%%%%%%%
\DescribeMacro{\childdocname}
\DescribeMacro{\childdocjob}
The macro |\childdocname| contains the filename (without extension)
of the main or child file being processed.
Note that |\childdocjob| will always contain the name of the main file.

%%%%%%%%%%%%%%%%%%%%%%%%%%%%%%%%%%%%%%%%
\paragraph{Title Page.}

Conditional processing can be used to include a title or banner page
in the main document when proper precautions are taken.
Importantly, the code in the main file should ensure that the page counter
(as well as other status parameters which are stored in the |.aux| files)
takes the same value after the conditional processing.
Otherwise the page numbers may take divergent values
depending on which part is compiled.

For example, a title page could be declared by:
%
\begin{center}
\begin{tabular}{l}
|\ifchilddoc\||else|\\
|\addtocounter{page}{-1}|\\
\textit{code for title page}\\
|\newpage|\\
|\||fi|
\end{tabular}
\end{center}
%
A banner page for the child documents can be generated by:
%
\begin{center}
\begin{tabular}{l}
|\ifchilddoc|\\
|\addtocounter{page}{-1}|\\
\textit{code for banner page}\\
|\newpage|\\
|\||fi|
\end{tabular}
\end{center}
%
Here one could write a message such as:
\begin{center}
|This is the part \childdocname{} of \childdocjob{}.|
\end{center}

%%%%%%%%%%%%%%%%%%%%%%%%%%%%%%%%%%%%%%%%%%%%%%%%%%%%%%%%%%%%%%%%%%%%%%%%%%%%%%%%
\subsection{Flags}
\label{sec:flags}

The package makes it easy to generate different versions
of the main or child documents.
To this end compilation flags can be defined
and assigned different default values.
They will be particularly useful in conjunction
with the forwarding mechanism described in \secref{sec:forward}.

For example, it may be useful to have a flag |\version|
which can be set to |draft| or |final|.
The document source will contain some conditional code
depending on the value of |\version|.
Suppose further, the flag should default to |final| for the main file
and to |draft| for child files
which is a natural assignment for editing the document.
This is achieved by placing the following code
in the preamble of the main document
(below the |\childdocmain| directive):
%
\begin{center}
\begin{tabular}{l}
|\ifchilddoc|\\
|\providecommand{\version}{draft}|\\
|\||else|\\
|\providecommand{\version}{final}|\\
|\||fi|
\end{tabular}
\end{center}
%
The definition by |\providecommand| makes sure
that previous definitions are not overwritten.
Further statements |\providecommand{\version}{...}|
can thus be added before the above code to override it.

For the main file, one might add a line
(between |\childdocmain| and the above block)
%
\begin{center}
|%\ifchilddoc\||else\providecommand{\version}{draft}\||fi|
\end{center}
%
which can be uncommented to produce a draft version.
Likewise one can add a line to the very top of a child file
(above the |\childdocof{|\textit{main}|}| directive)
%
\begin{center}
|%\providecommand{\version}{final}|
\end{center}
%
which can be uncommented to produce the final version of this child document.

%%%%%%%%%%%%%%%%%%%%%%%%%%%%%%%%%%%%%%%%%%%%%%%%%%%%%%%%%%%%%%%%%%%%%%%%%%%%%%%%
\subsection{Forwarding}
\label{sec:forward}

Different versions of the main or child documents
using compilation flags as described in \secref{sec:flags}
can be (permanently) stored in different files
for convenient compilation, viewing and distribution.
To this end, the package defines a command
to pass on compilation to a different file:

%%%%%%%%%%%%%%%%%%%%%%%%%%%%%%%%%%%%%%%%
\DescribeMacro{\childdocforward}
The command |\childdocforward| redirects processing to
another source file:
%
\begin{center}
\begin{tabular}{l}
|\input{childdoc.def}|\\
|\childdocforward[|\textit{main}|]{|\textit{dest}|}|\\
\end{tabular}
\end{center}
%
The argument \textit{dest} is the destination file
(without extension).
It should be the main file or one of the child files.
Note that further \textsf{childdoc} directives
such as |\childdocof| and |\childdocforward|
in the indicated file will be processed in this form.
The optional argument \textit{main}
passes on directly to the main file \textit{main}
while pretending to compile the child \textit{dest}.
This form behaves as if \textit{dest}
issues |\childdocof{|\textit{main}|}| right away,
and no further \textsf{childdoc} directives will be processed.

%%%%%%%%%%%%%%%%%%%%%%%%%%%%%%%%%%%%%%%%
\DescribeMacro{\...prefix}
In the alternative form |\childdocforwardprefix|,
%
\begin{center}
\begin{tabular}{l}
|\input{childdoc.def}|\\
|\childdocforwardprefix[|\textit{main}|]{|\textit{prefix}|}{|\textit{dest}|}|
\end{tabular}
\end{center}
%
the destination file is determined by a pattern
depending on the current file:
To make this work, the current file must be called
`{\textit{prefix}\hspace{0.2em}\textit{suffix}}'
with \textit{prefix} matching precisely the argument.
Processing is then passed on to the file
`{\textit{dest}\hspace{0.2em}\textit{suffix}}'.
Surely, the same effect is achieved by
directly specifying the
argument `{\textit{dest}\hspace{0.2em}\textit{suffix}}'
in the first form.
However, that requires to set up a different file
for each child. With the alternative form of the command
all these files can have exactly the same content
which simplifies setting them up and maintaining them.

For example, the following file |draft.tex|
with a compilation flag |\version| as described in \secref{sec:flags}
compiles the main document as a draft:
%
\begin{center}
\begin{tabular}{l}
|\def\version{draft}|\\
|\input{childdoc.def}|\\
|\childdocforward{|\textit{main}|}|
\end{tabular}
\end{center}
%
Likewise, the following files |final|\textit{nn}|.tex|
compile the final version of the child document
|child|\textit{nn}|.tex|:
%
\begin{center}
\begin{tabular}{l}
|\def\version{final}|\\
|\input{childdoc.def}|\\
|\childdocforwardprefix{final}{child}|
\end{tabular}
\end{center}
%

Note that when several versions of a main file and/or of each child file
are to be generated, it may be convenient to set up a |Makefile| or
shell script to automatise the process.

%%%%%%%%%%%%%%%%%%%%%%%%%%%%%%%%%%%%%%%%%%%%%%%%%%%%%%%%%%%%%%%%%%%%%%%%%%%%%%%%
\subsection{Command Line Processing}
\label{sec:commandline}

The effect of redirection files can also be achieved by invoking
the \LaTeX{} compiler with a more elaborate command line.
Most conveniently this should be done as part
of a shell script or a |Makefile|.

When using \textsf{childdoc} in the main file, the following
command lines effectively perform a redirection
(note that depending on the shell being used,
backslashes may have to be doubled: `|\|' $\to$ `|\\|'):
%
\begin{center}
|... -jobname "|\textit{target}|" |\\|"|[\textit{flags}]%
|\input{childdoc.def}\childdocforward[|\textit{main}|]{|\textit{dest}|}"|
\end{center}
%
Here \textit{target} is the name of the output file,
\textit{main} is the name of the main file
and \textit{dest} is the name of the main or child file to be processed
(all filenames without extensions).
The optional argument \textit{main} can be omitted
if \textit{main} matches \textit{dest}.
Optionally, compilation \textit{flags} can be defined via |\def| commands.
This command line makes the \TeX{} engine believe
it is compiling the file \textit{target}
whose content is specified as the latter parameter.
The provided code then forwards the processing to
\textit{main} or \textit{dest} as described in \secref{sec:forward}.

%%%%%%%%%%%%%%%%%%%%%%%%%%%%%%%%%%%%%%%%%%%%%%%%%%%%%%%%%%%%%%%%%%%%%%%%%%%%%%%%
\subsection{Include by Input}
\label{sec:input}

Including child documents by |\include| has some restrictions by design.
Most notably, the content of a child document always occupies
its own set of pages; pages cannot be shared between child documents.
Usually, this behaviour makes perfect sense
because each child document contain an essential part of the document.
However, in some situations it may be desirable to compose
a document from a collection of parts
without having mandatory page breaks between then.
For this case, the package
provides a mechanism to include parts
by |\input| which can also be processed individually.
However, by construction this mechanism
requires manual handling of the content to be output.

%%%%%%%%%%%%%%%%%%%%%%%%%%%%%%%%%%%%%%%%
\DescribeMacro{\ifchilddocmanual}
The main file should be prepared as usual, see \secref{sec:include}.
However, the document body must make a distinction
between processing of an individual part and of the main document, e.g.:
%
\begin{center}
\begin{tabular}{l}
|\ifchilddocmanual|\\
|\input{\childdocname}|\\
|\||else|\\
\textit{document body with }|\input{|\textit{part}|}|\\
|\||fi|
\end{tabular}
\end{center}
%
The conditional |\ifchilddocmanual| is true whenever
a part to be included by |\input| is being compiled,
and the name of the part is stored in |\childdocname|.

%%%%%%%%%%%%%%%%%%%%%%%%%%%%%%%%%%%%%%%%
\DescribeMacro{\childdocby}
Each part to be included by |\input| should start with:
%
\begin{center}
\begin{tabular}{l}
|\input{childdoc.def}|\\
|\childdocby{|\textit{main}|}|\\
\end{tabular}
\end{center}
%
The directive |\childdocby| is similar to |\childdocof|
described in \secref{sec:include},
but the subsequent selection of content must be done manually.
To that end, both |\ifchilddoc| and |\ifchilddocmanual|
will be true upon processing of a part,
and the name of the part is stored in |\childdocname|.
Note that |\jobname| will be set to the filename of the current part
so that each part receives an individual |.aux| file
that does not interfere with the |.aux| file(s) of the main document.
This behaviour can be altered by the alternative form
|\childdocby[*]{|\textit{main}|}| (with a non-empty optional argument)
which uses the |.aux| file of the main document
by setting |\jobname| to \textit{main}.

%%%%%%%%%%%%%%%%%%%%%%%%%%%%%%%%%%%%%%%%%%%%%%%%%%%%%%%%%%%%%%%%%%%%%%%%%%%%%%%%
\subsection{Driver Development}
\label{sec:driver}

The \textsf{childdoc} mechanism can also be use for the development
of definition files such as \LaTeX{} styles or classes.
This case differs from the above setup with multiple parts
included by |\include| in that no |\includeonly| should be invoked.
This can be achieved by starting the include file
(before |\ProvidesPackage|) with:
%
\begin{center}
\begin{tabular}{l}
|\input{childdoc.def}|\\
|\childdocforward{|\textit{main}|}|\\
\end{tabular}
\end{center}
%
or alternatively with:
%
\begin{center}
\begin{tabular}{l}
|\input{childdoc.def}|\\
|\childdocby{|\textit{main}|}|\\
\end{tabular}
\end{center}
%
Both forms have slightly different effects as described above.
The main file is prepared as usual, see \secref{sec:include}.

%%%%%%%%%%%%%%%%%%%%%%%%%%%%%%%%%%%%%%%%%%%%%%%%%%%%%%%%%%%%%%%%%%%%%%%%%%%%%%%%
\subsection{Legacy Detection}
\label{sec:detection}

The directive |\childdocmain| in the main file can detect
whether the complete document or merely a child is to be compiled
even without using the directive |\childdocof|.
This method is deprecated because it is less robust
and there is no compelling reason to use it;
it is merely provided for backward compatibility
and it may be removed in future versions.

If the detection mechanism is to be used,
it is mandatory to correctly specify
the filename of the main file as the argument of |\childdocmain|:
%
\begin{center}
\begin{tabular}{l}
|\input{childdoc.def}|\\
|\childdocmain{|\textit{main}|}|\\
\end{tabular}
\end{center}
%
If |\jobname| does not match the argument \textit{main} of |\childdocmain|,
it is assumed that |\jobname| points to the child file to be compiled.
When using |\childdocmain| with the main file specified as argument,
it suffices to start a child file
with just |\input{|\textit{main}|}|
without loading of the package and using |\childdocof|.
If instead all processing is done
with the appropriate \textsf{childdoc} directives,
the argument of \textit{main} of |\childdocmain| can be empty.

An alternative version of the command line processing described
in \secref{sec:commandline} using the detection mechanism reads:
%
\begin{center}
|... -jobname "|\textit{target}|" "|[\textit{flags}]%
[|\def\jobname{|\textit{dest}|}|]|\input{|\textit{main}|}"|
\end{center}

%%%%%%%%%%%%%%%%%%%%%%%%%%%%%%%%%%%%%%%%%%%%%%%%%%%%%%%%%%%%%%%%%%%%%%%%%%%%%%%%
\subsection{Manual Code}
\label{sec:manual}

In case one cannot be certain whether the definitions file |childdoc.def|
is installed on the target \TeX{} distribution
and one prefers not to ship it,
it is conceivable to paste a few relevant commands into the sources.

To that end, drop all statements |\input{childdoc.def}|
and perform the replacements as outlined below.
Instead of |\childdocmain{|\textit{main}|}| add the following code
to the top of the main file:
%
\begin{center}
\begin{tabular}{l}
|\||ifdefined\childdocname\endinput\||fi\newif\ifchilddoc|\\
|\edef\childdocname{\scantokens\expandafter{\jobname\noexpand}}|\\
|\def\childdocmain{|\textit{main}|}\||ifx\childdocmain\childdocname\||else|\\
|\childdoctrue\includeonly{\childdocname}\let\jobname\childdocmain\||fi|\\
\end{tabular}
\end{center}
%
Instead of |\childdocof{|\textit{main}|}| just include the main file
at the top of each child file:
%
\begin{center}
|\input{|\textit{main}|}|
\end{center}
%
A simple redirection |\childdocforward{|\textit{dest}|}| is achieved by:
%
\begin{center}
|\def\jobname{|\textit{dest}|}\input{\jobname}|
\end{center}
%
The redirection with prefix
|\childdocforwardprefix[|\textit{prefix}|]{|\textit{dest}|}|
is accomplished by:
%
\begin{center}
\begin{tabular}{l}
|{\edef\jobname{\scantokens\expandafter{\jobname\noexpand}}|\\
|\def\redirectjob |\textit{prefix}|#1~~~{\gdef\jobname{|\textit{dest}|#1}}|\\
|\expandafter\redirectjob\jobname~~~}\input{\jobname}|
\end{tabular}
\end{center}

In an alternative approach,
child documents can be compiled by a specific command line
without additional code or specific definitions:
%
\begin{center}
|... -jobname "|\textit{target}|" "|[\textit{flags}]%
|\includeonly{|\textit{dest}|}\input{|\textit{main}|}"|
\end{center}
%

%%%%%%%%%%%%%%%%%%%%%%%%%%%%%%%%%%%%%%%%%%%%%%%%%%%%%%%%%%%%%%%%%%%%%%%%%%%%%%%%
%%%%%%%%%%%%%%%%%%%%%%%%%%%%%%%%%%%%%%%%%%%%%%%%%%%%%%%%%%%%%%%%%%%%%%%%%%%%%%%%
\section{Information}

%%%%%%%%%%%%%%%%%%%%%%%%%%%%%%%%%%%%%%%%%%%%%%%%%%%%%%%%%%%%%%%%%%%%%%%%%%%%%%%%
\subsection{Copyright}

Copyright \copyright{} 2017--2018 Niklas Beisert

This work may be distributed and/or modified under the
conditions of the \LaTeX{} Project Public License, either version 1.3
of this license or (at your option) any later version.
The latest version of this license is in
  \url{http://www.latex-project.org/lppl.txt}
and version 1.3 or later is part of all distributions of \LaTeX{}
version 2005/12/01 or later.

This work has the LPPL maintenance status `maintained'.

The Current Maintainer of this work is Niklas Beisert.

This work consists of the files |README.txt|, |childdoc.ins| and |childdoc.dtx|
as well as the derived files |childdoc.def|, |cdocsamp.tex|
with |cdocsch1.tex|, |cdocsch2.tex|, |cdocspt3.tex|, |cdocspt4.tex|,
|cdocsdrf.tex|, |cdocsfn1.tex|, |cdocsfn2.tex|
as well as |childdoc.pdf|.

%%%%%%%%%%%%%%%%%%%%%%%%%%%%%%%%%%%%%%%%%%%%%%%%%%%%%%%%%%%%%%%%%%%%%%%%%%%%%%%%
\subsection{Files and Installation}

The package consists of the files:
%
\begin{center}
\begin{tabular}{ll}
    |README.txt|   & readme file \\
    |childdoc.ins| & installation file \\
    |childdoc.dtx| & source file \\
    |childdoc.def| & definition file \\
    |cdocsamp.tex| & sample main file \\
    |cdocsch1.tex| & sample include file \\
    |cdocsch2.tex| & sample include file \\
    |cdocspt3.tex| & sample part file \\
    |cdocspt4.tex| & sample part file \\
    |cdocsdrf.tex| & sample redirection file \\
    |cdocsfn1.tex| & sample redirection file \\
    |cdocsfn2.tex| & sample redirection file \\
    |childdoc.pdf| & manual
\end{tabular}
\end{center}
%
The distribution consists of the files
|README.txt|, |childdoc.ins| and |childdoc.dtx|.
%
\begin{itemize}
\item
Run (pdf)\LaTeX{} on |childdoc.dtx|
to compile the manual |childdoc.pdf| (this file).
\item
Run \LaTeX{} on |childdoc.ins| to create the definitions file |childdoc.def|
and the sample |cdocsamp.tex| with include files
|cdocsch1.tex|, |cdocsch2.tex|, |cdocspt3.tex|, |cdocspt4.tex|,
|cdocsdrf.tex|, |cdocsfn1.tex|, |cdocsfn2.tex|.
Then copy the file |childdoc.def| to an appropriate directory of your \LaTeX{}
distribution, e.g.\ \textit{texmf-root}|/tex/latex/childdoc|.
\end{itemize}

%%%%%%%%%%%%%%%%%%%%%%%%%%%%%%%%%%%%%%%%%%%%%%%%%%%%%%%%%%%%%%%%%%%%%%%%%%%%%%%%
\subsection{Related CTAN Packages}

There are several other packages which offer a similar functionality:
%
\begin{itemize}
\item
The packages
\href{http://ctan.org/pkg/docmute}{\textsf{docmute}},
\href{http://ctan.org/pkg/includex}{\textsf{includex}} and
\href{http://ctan.org/pkg/standalone}{\textsf{standalone}}
provide commands to include only the document body of
a child file thus allowing both files to be compiled individually.
\item
The packages \href{http://ctan.org/pkg/subdocs}{\textsf{subdocs}}
and \href{http://ctan.org/pkg/subfiles}{\textsf{subfiles}}
provide structures in which the main and child documents can be
encapsulated and allowing them to be compiled individually.
The inclusion mechanism is different from the conventional |\include|.
\item
The package \href{http://ctan.org/pkg/combine}{\textsf{combine}}
is an elaborate solution to combine several documents into one.
\end{itemize}
%
See also the CTAN topic \href{http://ctan.org/topic/subdocs}{\textsf{subdocs}}
for further related packages.
The present package differs from the above solutions in that
a document structure constructed with the conventional |\include| mechanism
just needs two extra commands at the top of every file
such that all constituent files can be compiled individually.

%%%%%%%%%%%%%%%%%%%%%%%%%%%%%%%%%%%%%%%%%%%%%%%%%%%%%%%%%%%%%%%%%%%%%%%%%%%%%%%%
%\subsection{Feature Suggestions}
%
%The following is a list of features which may be useful for future
%versions of this package:
%%
%\begin{itemize}
%\item
%\ldots
%\end{itemize}

%%%%%%%%%%%%%%%%%%%%%%%%%%%%%%%%%%%%%%%%%%%%%%%%%%%%%%%%%%%%%%%%%%%%%%%%%%%%%%%%
\subsection{Revision History}

%%%%%%%%%%%%%%%%%%%%%%%%%%%%%%%%%%%%%%%%
\paragraph{v2.0:} 2018/12/30

\begin{itemize}
\item
immediate forward processing
\item
added |\childdocby| mechanism
\item
manual restructured
\end{itemize}

%%%%%%%%%%%%%%%%%%%%%%%%%%%%%%%%%%%%%%%%
\paragraph{v1.6:} 2018/01/17

\begin{itemize}
\item
application for development of include files
\item
corrections to manual
\end{itemize}

%%%%%%%%%%%%%%%%%%%%%%%%%%%%%%%%%%%%%%%%
\paragraph{v1.5:} 2017/05/21

\begin{itemize}
\item
more complete structuring introduced
\item
|\childdocof| introduced
\item
|\childdoc| renamed to |\childdocmain|
\item
|\childredirect| renamed to |\childdocforward| and |\childdocforwardprefix|
and functionality expanded
\end{itemize}

%%%%%%%%%%%%%%%%%%%%%%%%%%%%%%%%%%%%%%%%
\paragraph{v1.0:} 2017/04/27

\begin{itemize}
\item
manual and install package
\item
first version published on CTAN
\end{itemize}

%%%%%%%%%%%%%%%%%%%%%%%%%%%%%%%%%%%%%%%%
\paragraph{v0.6:} 2017/04/26

\begin{itemize}
\item
redirection mechanism added
\end{itemize}

%%%%%%%%%%%%%%%%%%%%%%%%%%%%%%%%%%%%%%%%
\paragraph{v0.5:} 2017/04/26

\begin{itemize}
\item
functionality in definition file
\end{itemize}


%%%%%%%%%%%%%%%%%%%%%%%%%%%%%%%%%%%%%%%%%%%%%%%%%%%%%%%%%%%%%%%%%%%%%%%%%%%%%%%%
%%%%%%%%%%%%%%%%%%%%%%%%%%%%%%%%%%%%%%%%%%%%%%%%%%%%%%%%%%%%%%%%%%%%%%%%%%%%%%%%
%%%%%%%%%%%%%%%%%%%%%%%%%%%%%%%%%%%%%%%%%%%%%%%%%%%%%%%%%%%%%%%%%%%%%%%%%%%%%%%%
\appendix

\settowidth\MacroIndent{\rmfamily\scriptsize 000\ }

 \DocInput{childdoc.dtx}

\end{document}
%</driver>
% \fi
%
% %%%%%%%%%%%%%%%%%%%%%%%%%%%%%%%%%%%%%%%%%%%%%%%%%%%%%%%%%%%%%%%%%%%%%%%%%%%%%%
% %%%%%%%%%%%%%%%%%%%%%%%%%%%%%%%%%%%%%%%%%%%%%%%%%%%%%%%%%%%%%%%%%%%%%%%%%%%%%%
% \section{Sample}
%\iffalse
%<*samplemain>
%\fi
%
% The following presents a sample document
% with two chapters, two parts, a title page,
% a compile flag as well as three forwarding files to set the flag.
% It consists of eight |.tex| files:
% \begin{center}
% \begin{tabular}{ll}
% |cdocsamp.tex|&main file\\
% |cdocsch1.tex|&include file for chapter 1\\
% |cdocsch2.tex|&include file for chapter 2\\
% |cdocspt3.tex|&include file for part 3\\
% |cdocspt4.tex|&include file for part 4\\
% |cdocsdrf.tex|&forwarding file for main file in draft mode\\
% |cdocsfi1.tex|&forwarding file for final version of chapter 1\\
% |cdocsfi2.tex|&forwarding file for final version of chapter 2\\
% \end{tabular}
% \end{center}
% Each of the eight files can be compiled directly by the \LaTeX{} compiler.
%
% %%%%%%%%%%%%%%%%%%%%%%%%%%%%%%%%%%%%%%
% \paragraph{Main File.}
%
% The main file is called |cdocsamp.tex|.
%
% Load the \textsf{childdoc} definitions and
% declare the filename for the main document:
%    \begin{macrocode}
\input{childdoc.def}
\childdocmain{}
%    \end{macrocode}

% Optional override for |\version| flag:
%    \begin{macrocode}
%%\ifchilddoc\else\providecommand{\version}{draft}\fi
%    \end{macrocode}

% Define the default values for the |\version| flag
% (|final| for the main file and |draft| for childs):
%    \begin{macrocode}
\ifchilddoc
\providecommand{\version}{draft}
\else
\providecommand{\version}{final}
\fi
%    \end{macrocode}

% Load the standard document class:
%    \begin{macrocode}
\documentclass[12pt]{article}
%    \end{macrocode}

% Start the document body:
%    \begin{macrocode}
\begin{document}
%    \end{macrocode}

% Declare a title page.
% Print title, part of document being processed and version flag:
%    \begin{macrocode}
\addtocounter{page}{-1}
\begin{center}
{\LARGE\bfseries{}childdoc example\par}
\vspace{1cm}
\ifchilddoc
\ifchilddocmanual part\else chapter\fi:
`\childdocname' of `\childdocjob'\par
\else
main document: `\childdocjob'\par
\fi
version: \version\par
\end{center}
\newpage
%    \end{macrocode}

% Manually include selected file,
% otherwise process as usual:
%    \begin{macrocode}
\ifchilddocmanual
\section*{part `\childdocname'}
\input{\childdocname}
\else
%    \end{macrocode}

% Include the two chapters:
%    \begin{macrocode}
\include{cdocsch1}
\include{cdocsch2}
%    \end{macrocode}

% Include the two parts unless only chapters should be displayed:
%    \begin{macrocode}
\ifchilddoc\else
\section{part three}
\input{cdocspt3}
\section{part four}
\input{cdocspt4}
\fi
%    \end{macrocode}

% Process as usual until here:
%    \begin{macrocode}
\fi
%    \end{macrocode}

% End of document body:
%    \begin{macrocode}
\end{document}
%    \end{macrocode}
%\iffalse
%</samplemain>
%\fi
%
% %%%%%%%%%%%%%%%%%%%%%%%%%%%%%%%%%%%%%%
% \paragraph{Chapter Include Files.}
%
% The include files are called |cdocsch1.tex| and |cdocsch2.tex|.
%
%\iffalse
%<*samplechap1|samplechap2>
%\fi

% Optional override for |\version| flag:
%    \begin{macrocode}
%%\providecommand{\version}{final}
%    \end{macrocode}

% Include the main document:
%    \begin{macrocode}
\input{childdoc.def}
\childdocof{cdocsamp}
%    \end{macrocode}

%\iffalse
%</samplechap1|samplechap2>
%\fi
%
%\iffalse
%<*samplechap1>
%\fi
% Some text for chapter 1:
%    \begin{macrocode}
\section{one}
some text in chapter one
%    \end{macrocode}

%\iffalse
%</samplechap1>
%\fi
% Some text for chapter 2:
%\iffalse
%<*samplechap2>
%\fi
%    \begin{macrocode}
\section{two}
more text in chapter two
%    \end{macrocode}

%\iffalse
%</samplechap2>
%\fi
%
% %%%%%%%%%%%%%%%%%%%%%%%%%%%%%%%%%%%%%%
% \paragraph{Part Include Files.}
%
% The include files are called |cdocspt3.tex| and |cdocspt4.tex|.
%
%\iffalse
%<*samplepart3|samplepart4>
%\fi

% Optional override for |\version| flag:
%    \begin{macrocode}
%%\providecommand{\version}{final}
%    \end{macrocode}

% Include the main document:
%    \begin{macrocode}
\input{childdoc.def}
\childdocby{cdocsamp}
%    \end{macrocode}

%\iffalse
%</samplepart3|samplepart4>
%\fi
%
%\iffalse
%<*samplepart3>
%\fi
% Some text for part 3:
%    \begin{macrocode}
some text in part three
%    \end{macrocode}

%\iffalse
%</samplepart3>
%\fi
% Some text for part 4:
%\iffalse
%<*samplepart4>
%\fi
%    \begin{macrocode}
more text in part four
%    \end{macrocode}

%\iffalse
%</samplepart4>
%\fi
%
% %%%%%%%%%%%%%%%%%%%%%%%%%%%%%%%%%%%%%%
% \paragraph{Forwarding for a Complete Draft.}
%
% The following forwarding file |cdocsdrf.tex|
% compiles the main document in draft mode:
%\iffalse
%<*sampledraft>
%\fi
%    \begin{macrocode}
\def\version{draft}
\input{childdoc.def}
\childdocforward{cdocsamp}
%    \end{macrocode}

%\iffalse
%</sampledraft>
%\fi
%
% %%%%%%%%%%%%%%%%%%%%%%%%%%%%%%%%%%%%%%
% \paragraph{Forwarding for Final Version of the Chapters.}
%
% The following forwarding files |cdocsfn1.tex| and |cdocsfn2.tex|
% (with identical content)
% compile the final versions of the child documents
% |cdocsch1.tex| and |cdocsch2.tex|, respectively:
%\iffalse
%<*samplefinal>
%\fi
%    \begin{macrocode}
\def\version{final}
\input{childdoc.def}
\childdocforwardprefix[cdocsamp]{cdocsfn}{cdocsch}
%    \end{macrocode}

%\iffalse
%</samplefinal>
%\fi
%
% %%%%%%%%%%%%%%%%%%%%%%%%%%%%%%%%%%%%%%
% \paragraph{Command Line Processing.}
%
% The following three command lines generate the output files
% |cdocscld|, |cdocscl1| and |cdocscl2|
% which should be identical to
% |cdocsdrf|, |cdocsch1| and |cdocsfn2|, respectively:
% \begin{center}
% \begin{tabular}{l}
% |latex -jobname cdocscld \|\\
% |  "\def\version{draft}\input{childdoc.def}\childdocforward{cdocsamp}"|\\
% |latex -jobname cdocscl1 \|\\
% |  "\input{childdoc.def}\childdocforward[cdocsamp]{cdocsch1}"|\\
% |latex -jobname cdocscl2 \|\\
% |  "\def\version{final}\input{childdoc.def}\childdocforward{cdocsch2}"|
% \end{tabular}
% \end{center}
% Note that the trailing backslash on each first line
% merely continues the input to the second line
% (for convenient cut ant paste).
% Furthermore, the command |latex| can be replaced by any
% of its alternative versions such as |pdflatex|.
%
% %%%%%%%%%%%%%%%%%%%%%%%%%%%%%%%%%%%%%%%%%%%%%%%%%%%%%%%%%%%%%%%%%%%%%%%%%%%%%%
% %%%%%%%%%%%%%%%%%%%%%%%%%%%%%%%%%%%%%%%%%%%%%%%%%%%%%%%%%%%%%%%%%%%%%%%%%%%%%%
% \section{Implementation}
%\iffalse
%<*package>
%\fi
%
% This section describes the definitions file |childdoc.def|.

% The definitions cannot be loaded using |\usepackage| or |\RequirePackage|
% which has a mechanism to prevent loading a style file more than once.
% When loading the definitions by means of |\input|
% multiple instances have to be prevented manually:
%\iffalse
%This code needs to be before the `\ProvidesFile' directive
%which is defined at the beginning of this file.
%Therefore it is also placed there and commented out here.
%</package>
%<*discard>
%\fi
%    \begin{macrocode}
\ifdefined\childdocmain\endinput\fi
%    \end{macrocode}
%\iffalse
%</discard>
%<*package>
%\fi
%
% \macro{\ifchilddoc}
% \macro{\ifchilddocmanual}
% The conditional |\ifchilddoc| tells whether a
% child (true) or main (false) document is being compiled.
% The conditional |\ifchilddocmanual| tells whether
% the |\includeonly| mechanism is used (false) or
% the selection of child files must be performed manually (true).
% The definitions initialise to false:
%    \begin{macrocode}
\newif\ifchilddoc
\newif\ifchilddocmanual
%    \end{macrocode}

% \macro{\childdocname}
% \macro{\childdocjob}
% The macro |\childdocname| stores the name of the main document
% to be compiled. The macro |\childdocjob| stores the name of
% the document on which the \LaTeX{} compiler was originally invoked.
% The content of |\jobname| cannot be compared
% to filenames specified in the source due to different catcodes.
% The following code rescans |\jobname|, stores the result
% in |\childdocname| and saves a copy in |\childdocjob|:
%    \begin{macrocode}
\edef\childdocname{\scantokens\expandafter{\jobname\noexpand}}
\let\childdocjob\childdocname
%    \end{macrocode}

% \macro{\childdocdisable}
% The macro |\childdocdisable| prevents the main file
% from being processed more than once.
% At this stage, the main document command |\childdocmain|
% is assumed to be called once again where it should do nothing.
% Any subsequent call to it should prevent
% a secondary processing of the main document
% It overwrites the forwarding commands
% |\childdocof| and |\childdocforward|
% with empty macros to prevent further inclusions of the main document:
%    \begin{macrocode}
\newcommand{\childdocdisable}
{
  \renewcommand{\childdocmain}[1]{\renewcommand{\childdocmain}[1]{\endinput}}
  \renewcommand{\childdocof}[1]{}
  \renewcommand{\childdocby}[2][]{}
  \renewcommand{\childdocforward}[2][]{}
  \renewcommand{\childdocdisable}{}
}
%    \end{macrocode}

% \macro{\childdocmain}
% The macro |\childdocmain| is to be called at the top of the main file
% with nothing or the main filename (without extension) as argument.
% First, it breaks loops.
% If the argument is not empty and does not match |\childdocname|
% (which is set by the first inclusion of |childdoc.def|),
% |\ifchilddoc| is set to true, |\includeonly| is applied to the child file
% and |\jobname| is set to the main file
% (for proper handling of |.aux| files):
%    \begin{macrocode}
\newcommand{\childdocmain}[1]
{
  \childdocdisable\childdocmain{}
  \if?#1?\else
    \begingroup
      \def\childdoctmp{#1}
      \ifx\childdoctmp\childdocname
        \def\childdoctmp{}
      \else
        \def\childdoctmp
        {
          \childdoctrue
          \includeonly{\childdocname}
          \def\childdocjob{#1}
          \def\jobname{#1}
        }
      \fi
      \expandafter
    \endgroup
    \childdoctmp
  \fi
}
%    \end{macrocode}

% \macro{\childdocof}
% The command |\childdocof| redirects
% compilation to the main file |#1|.
%    \begin{macrocode}
\newcommand{\childdocof}[1]
{
  \childdocdisable
  \childdoctrue
  \includeonly{\childdocname}
  \def\jobname{#1}
  \def\childdocjob{#1}
  \input{#1}
}
%    \end{macrocode}

% \macro{\childdocby}
% The command |\childdocby| ....
%    \begin{macrocode}
\newcommand{\childdocby}[2][]
{
  \childdocdisable
  \childdoctrue
  \childdocmanualtrue
  \if?#1?\else
    \def\jobname{#2}
  \fi
  \def\childdocjob{#2}
  \input{#2}
  \endinput
}
%    \end{macrocode}

% \macro{\childdocforward}
% The command |\childdocforward| redirects
% compilation to the main file or
% (if the optional argument is given) a child file.
% Parameters are set as if the main file
% or a child file starting with |\childdocof| was compiled.
% Then compilation is handed over to the main file:
%    \begin{macrocode}
\newcommand{\childdocforward}[2][]
{
  \begingroup
    \if?#1?
      \def\childdoctmp
      {
        \def\childdocname{#2}
        \def\childdocjob{#2}
        \def\jobname{#2}
        \input{#2}
        \endinput
      }
    \else
      \def\childdoctmp
      {
        \childdocdisable
        \def\childdocname{#2}
        \childdoctrue
        \includeonly{#2}
        \def\childdocjob{#1}
        \def\jobname{#1}
        \input{#1}
        \endinput
      }
    \fi
    \expandafter
  \endgroup
  \childdoctmp
}
%    \end{macrocode}

% \macro{\childdocforwardprefix}
% The command |\childdocforwardprefix| redirects
% compilation to the main or a child file by means of a pattern.
% The prefix |#1| in the current filename is replaced by |#2|
% and the suffix of the current filename is kept
% (it is assumed that the filename does not contain the substring `|~~~|'
% which is used as a delimiter).
% Compilation is handed over to the new file by |\childdocforward|:
%    \begin{macrocode}
\newcommand{\childdocforwardprefix}[3][]
{
  \begingroup
    \def\childdocextract #2##1~~~{\def\childdoctmp{\childdocforward[#1]{#3##1}}}
    \expandafter\childdocextract\childdocname~~~
    \expandafter
  \endgroup
  \childdoctmp
}
%    \end{macrocode}

% \macro{\childdoc}
% The deprecated macro |\childdoc| is a legacy version of |\childdocmain|:
%    \begin{macrocode}
\newcommand{\childdoc}{\childdocmain}
%    \end{macrocode}

% \macro{\childdocredirect}
% The deprecated macro |\childdocredirect| is a legacy version
% of |\childdocforward| and |\childdocforwardprefix|:
%    \begin{macrocode}
\newcommand{\childdocredirect}[2][]
{
  \begingroup
    \if?#1?
      \def\childdoctmp{\childdocforward{#2}}
    \else
      \def\childdoctmp{\childdocforwardprefix{#1}{#2}}
    \fi
    \expandafter
  \endgroup
  \childdoctmp
}
%    \end{macrocode}

%\iffalse
%</package>
%\fi
%
\endinput
|\\
|\childdocforward[|\textit{main}|]{|\textit{dest}|}|\\
\end{tabular}
\end{center}
%
The argument \textit{dest} is the destination file
(without extension).
It should be the main file or one of the child files.
Note that further \textsf{childdoc} directives
such as |\childdocof| and |\childdocforward|
in the indicated file will be processed in this form.
The optional argument \textit{main}
passes on directly to the main file \textit{main}
while pretending to compile the child \textit{dest}.
This form behaves as if \textit{dest}
issues |\childdocof{|\textit{main}|}| right away,
and no further \textsf{childdoc} directives will be processed.

%%%%%%%%%%%%%%%%%%%%%%%%%%%%%%%%%%%%%%%%
\DescribeMacro{\...prefix}
In the alternative form |\childdocforwardprefix|,
%
\begin{center}
\begin{tabular}{l}
|% \iffalse
%
% childdoc.dtx Copyright (C) 2017-2018 Niklas Beisert
%
% This work may be distributed and/or modified under the
% conditions of the LaTeX Project Public License, either version 1.3
% of this license or (at your option) any later version.
% The latest version of this license is in
%   http://www.latex-project.org/lppl.txt
% and version 1.3 or later is part of all distributions of LaTeX
% version 2005/12/01 or later.
%
% This work has the LPPL maintenance status `maintained'.
%
% The Current Maintainer of this work is Niklas Beisert.
%
% This work consists of the files childdoc.dtx and childdoc.ins
% and the derived files childdoc.def and cdocsamp.tex with
% cdocsch1.tex, cdocsch2.tex, cdocsdrf.tex, cdocsfn1.tex, cdocsfn2.tex.
%
%<package>\ifdefined\childdocmain\endinput\fi
%<package>\ProvidesFile{childdoc.def}[2018/12/30 v2.0 child document driver]
%<samplemain>\ProvidesFile{cdocsamp.tex}[2018/12/30 v2.0 sample for childdoc]
%<*driver>
%\ProvidesFile{childdoc.drv}[2018/12/30 v2.0 childdoc reference manual file]
\PassOptionsToClass{10pt,a4paper}{article}
\documentclass{ltxdoc}

\usepackage[margin=35mm]{geometry}
\usepackage{hyperref}
\usepackage{hyperxmp}
\usepackage[usenames]{color}

\hypersetup{colorlinks=true}
\hypersetup{pdfstartview=FitH}
\hypersetup{pdfpagemode=UseNone}
\hypersetup{pdfsource={}}
\hypersetup{pdflang={en-UK}}
\hypersetup{pdfcopyright={Copyright 2017-2018 Niklas Beisert.
  This work may be distributed and/or modified under the
  conditions of the LaTeX Project Public License, either version 1.3
  of this license or (at your option) any later version.}}
\hypersetup{pdflicenseurl={http://www.latex-project.org/lppl.txt}}
\hypersetup{pdfcontactaddress={ETH Zurich, ITP, HIT K,
  Wolfgang-Pauli-Strasse 27}}
\hypersetup{pdfcontactpostcode={8093}}
\hypersetup{pdfcontactcity={Zurich}}
\hypersetup{pdfcontactcountry={Switzerland}}
\hypersetup{pdfcontactemail={nbeisert@itp.phys.ethz.ch}}
\hypersetup{pdfcontacturl={http://people.phys.ethz.ch/\xmptilde nbeisert/}}

\newcommand{\secref}[1]{\hyperref[#1]{section \ref*{#1}}}

\parskip1ex
\parindent0pt
\let\olditemize\itemize
\def\itemize{\olditemize\parskip0pt}

\begin{document}

\title{The \textsf{childdoc} Package}
\hypersetup{pdftitle={The childdoc Package}}
\author{Niklas Beisert\\[2ex]
  Institut f\"ur Theoretische Physik\\
  Eidgen\"ossische Technische Hochschule Z\"urich\\
  Wolfgang-Pauli-Strasse 27, 8093 Z\"urich, Switzerland\\[1ex]
  \href{mailto:nbeisert@itp.phys.ethz.ch}
  {\texttt{nbeisert@itp.phys.ethz.ch}}}
\hypersetup{pdfauthor={Niklas Beisert}}
\hypersetup{pdfsubject={Manual for the LaTeX2e Package childdoc}}
\date{30 December 2018, \textsf{v2.0}}
\maketitle

\begin{abstract}\noindent
\textsf{childdoc} is a \LaTeXe{} package
that enables the direct compilation
of document sections included by |\include|
to individual files.
\end{abstract}

\begingroup
\parskip0ex
\tableofcontents
\endgroup

%%%%%%%%%%%%%%%%%%%%%%%%%%%%%%%%%%%%%%%%%%%%%%%%%%%%%%%%%%%%%%%%%%%%%%%%%%%%%%%%
%%%%%%%%%%%%%%%%%%%%%%%%%%%%%%%%%%%%%%%%%%%%%%%%%%%%%%%%%%%%%%%%%%%%%%%%%%%%%%%%
\section{Introduction}

\LaTeX{} provides a mechanism to structure a large document (such as a book)
into a main file and several child files (containing the chapters)
using the |\include| command.
This mechanism is beneficial for documents
which span hundreds of pages in order to
make the source file(s) more manageable.
Moreover, compilation can be restricted to
selected child files by means of the |\includeonly| command.
The latter feature can be used to reduce the compilation time while editing
(this was significantly more useful in the earlier days of \LaTeX{})
or to generate a smaller document which is easier to navigate.
Another application of |\includeonly| is to generate
documents consisting of selected parts of the complete document.

However, there are a few drawbacks of the plain |\include| mechanism:
\begin{itemize}
\item
The child files cannot be compiled on their own,
they can only be compiled via the main file.
A naive editing environment
(such as a text editor with an option
to have the current file processed by \LaTeX)
may require one to switch to the main file before compiling;
attempting to compile the child file produces errors.
\item
The main file must be modified (each time)
to adjust the |\includeonly| command
to the present needs. This easily leaves the main file in a messy state.
\item
The generated document will always carry the filename
of the main document. This is inconvenient if
several child files are to be compiled and
to be kept for distribution.
\end{itemize}

The present package provides a simple interface
to make child files individually compilable by \LaTeX{}.
Compiling a child file then has the same effect as compiling
the main file with an |\includeonly| command
to select the appropriate child.
Moreover the generated document will carry the name of the child
rather than the main file.
This resolves all three above issues.

This feature is meant to make the editing of books,
thesis documents and lecture notes somewhat more convenient.
However, the package can also be used efficiently for
composing a series of documents (such as exercise sheets)
which are typically distributed individually.
It then assists the author in generating the individual documents
(potentially in different versions)
as well as a document containing the collected series.
Another application is in developing style files
or other kinds of included material
where compilation of the style file could redirect
to a sample or test file.

%%%%%%%%%%%%%%%%%%%%%%%%%%%%%%%%%%%%%%%%%%%%%%%%%%%%%%%%%%%%%%%%%%%%%%%%%%%%%%%%
%%%%%%%%%%%%%%%%%%%%%%%%%%%%%%%%%%%%%%%%%%%%%%%%%%%%%%%%%%%%%%%%%%%%%%%%%%%%%%%%
\section{Usage}

First of all, the package \textsf{childdoc} is \emph{not} a standard
\LaTeXe{} |.sty| style file! Therefore it needs to be invoked in
a non-standard way.

%%%%%%%%%%%%%%%%%%%%%%%%%%%%%%%%%%%%%%%%%%%%%%%%%%%%%%%%%%%%%%%%%%%%%%%%%%%%%%%%
\subsection{Included Files}
\label{sec:include}

%%%%%%%%%%%%%%%%%%%%%%%%%%%%%%%%%%%%%%%%
\DescribeMacro{\childdocmain}
To use the package, add the commands
\begin{center}
\begin{tabular}{l}
|\input{childdoc.def}|\\
|\childdocmain{}|\\
\end{tabular}
\end{center}
at the very top of the main \LaTeX{} file,
in particular \emph{before} the |\documentclass| statement!
The argument of |\childdocmain| should be left empty
(but it must be present).

%%%%%%%%%%%%%%%%%%%%%%%%%%%%%%%%%%%%%%%%
\DescribeMacro{\childdocof}
Furthermore, add the commands
\begin{center}
\begin{tabular}{l}
|\input{childdoc.def}|\\
|\childdocof{|\textit{main}|}|\\
\end{tabular}
\end{center}
at the top of every child file \textit{child}
which is included by |\include{|\textit{child}|}|
from within the main file
(or at least for those files to be compiled individually).
The argument \textit{main} must be the filename of the main file.

There are a couple of
considerations in setting up the main and child documents:

%%%%%%%%%%%%%%%%%%%%%%%%%%%%%%%%%%%%%%%%
\paragraph{Restrictions.}

Please note the following restrictions:
\begin{itemize}
\item
|\childdocmain| must be called with one argument \textit{main}
to ensure compatibility with earlier version of the package.
It must either be empty (|\childdocmain{}|)
or precisely match the filename of the main file in which it is specified.
See \secref{sec:detection} for further information.
\item
The filename \textit{main} must be specified without the |.tex| extension.
\item
The filename \textit{main} is case sensitive
(even in case-insensitive file systems)
due to internal string comparison.
\item
The argument \textit{main} should be fully expanded, it cannot be a macro.
\item
Subdirectories and special characters should be avoided in filenames.
\item
The command |\childdocmain{|\textit{main}|}| must be followed by a whitespace.
It should not be followed immediately by another command
or by a comment mark `|%|'.
This is because the \TeX{} parser reads the token immediately following
the argument of |\childdocmain| and puts it
at the beginning of every child section;
however, a white\-space is ignored.
\end{itemize}

%%%%%%%%%%%%%%%%%%%%%%%%%%%%%%%%%%%%%%%%
\paragraph{Content of Main File.}

It is advisable to place all content in the child files included by |\include|.
Any output contained in the main file will appear in all child documents
unless suppressed manually;
it cannot be suppressed automatically by the |\includeonly| directive
and thus should normally be avoided.
A method to include some content in the main file
by means of conditional processing is described in \secref{sec:conditional}.

%%%%%%%%%%%%%%%%%%%%%%%%%%%%%%%%%%%%%%%%
\paragraph{Page Numbering.}

When only a part of the document is compiled,
the appropriate numbering of pages
(as well as other status parameters)
is determined from the |.aux| files.
The latter contain information from previous passes.
However this information needs to propagate through
all intermediate child documents.
Therefore the page numbering in child documents may well
be inconsistent until the complete document is compiled at least once.

A useful (if unconventional) way to always ensure a consistent
page numbering is to restart the numbering in each child document
and denote the pages by `\textit{child}|.|\textit{page}'
where \textit{child} represents the chapter/section number of the child file.
This can be achieved by the command
|\numberwithin{page}{|\textit{child}|}|
of the \textsf{amsmath} package
where \textit{child} can be |chapter| or |section|
depending on the chosen structuring.
Alternatively, one can modify the macro |\thepage| appropriately
and reset the counter |page| at the start of each child file.

%%%%%%%%%%%%%%%%%%%%%%%%%%%%%%%%%%%%%%%%%%%%%%%%%%%%%%%%%%%%%%%%%%%%%%%%%%%%%%%%
\subsection{Conditional Processing}
\label{sec:conditional}

The package provides a mechanism to compile different versions
of a document. To customise the versions further some conditional processing
can come in handy to distinguish which version is being compiled.
The package provides two macros to describe the compilation context:

%%%%%%%%%%%%%%%%%%%%%%%%%%%%%%%%%%%%%%%%
\DescribeMacro{\ifchilddoc}
The conditional |\ifchilddoc| distinguishes between the compilation of
child documents and the main document:
%
\begin{center}
|\ifchilddoc |\textit{child-code}| |[|\||else |\textit{main-code}]| \||fi|
\end{center}

%%%%%%%%%%%%%%%%%%%%%%%%%%%%%%%%%%%%%%%%
\DescribeMacro{\childdocname}
\DescribeMacro{\childdocjob}
The macro |\childdocname| contains the filename (without extension)
of the main or child file being processed.
Note that |\childdocjob| will always contain the name of the main file.

%%%%%%%%%%%%%%%%%%%%%%%%%%%%%%%%%%%%%%%%
\paragraph{Title Page.}

Conditional processing can be used to include a title or banner page
in the main document when proper precautions are taken.
Importantly, the code in the main file should ensure that the page counter
(as well as other status parameters which are stored in the |.aux| files)
takes the same value after the conditional processing.
Otherwise the page numbers may take divergent values
depending on which part is compiled.

For example, a title page could be declared by:
%
\begin{center}
\begin{tabular}{l}
|\ifchilddoc\||else|\\
|\addtocounter{page}{-1}|\\
\textit{code for title page}\\
|\newpage|\\
|\||fi|
\end{tabular}
\end{center}
%
A banner page for the child documents can be generated by:
%
\begin{center}
\begin{tabular}{l}
|\ifchilddoc|\\
|\addtocounter{page}{-1}|\\
\textit{code for banner page}\\
|\newpage|\\
|\||fi|
\end{tabular}
\end{center}
%
Here one could write a message such as:
\begin{center}
|This is the part \childdocname{} of \childdocjob{}.|
\end{center}

%%%%%%%%%%%%%%%%%%%%%%%%%%%%%%%%%%%%%%%%%%%%%%%%%%%%%%%%%%%%%%%%%%%%%%%%%%%%%%%%
\subsection{Flags}
\label{sec:flags}

The package makes it easy to generate different versions
of the main or child documents.
To this end compilation flags can be defined
and assigned different default values.
They will be particularly useful in conjunction
with the forwarding mechanism described in \secref{sec:forward}.

For example, it may be useful to have a flag |\version|
which can be set to |draft| or |final|.
The document source will contain some conditional code
depending on the value of |\version|.
Suppose further, the flag should default to |final| for the main file
and to |draft| for child files
which is a natural assignment for editing the document.
This is achieved by placing the following code
in the preamble of the main document
(below the |\childdocmain| directive):
%
\begin{center}
\begin{tabular}{l}
|\ifchilddoc|\\
|\providecommand{\version}{draft}|\\
|\||else|\\
|\providecommand{\version}{final}|\\
|\||fi|
\end{tabular}
\end{center}
%
The definition by |\providecommand| makes sure
that previous definitions are not overwritten.
Further statements |\providecommand{\version}{...}|
can thus be added before the above code to override it.

For the main file, one might add a line
(between |\childdocmain| and the above block)
%
\begin{center}
|%\ifchilddoc\||else\providecommand{\version}{draft}\||fi|
\end{center}
%
which can be uncommented to produce a draft version.
Likewise one can add a line to the very top of a child file
(above the |\childdocof{|\textit{main}|}| directive)
%
\begin{center}
|%\providecommand{\version}{final}|
\end{center}
%
which can be uncommented to produce the final version of this child document.

%%%%%%%%%%%%%%%%%%%%%%%%%%%%%%%%%%%%%%%%%%%%%%%%%%%%%%%%%%%%%%%%%%%%%%%%%%%%%%%%
\subsection{Forwarding}
\label{sec:forward}

Different versions of the main or child documents
using compilation flags as described in \secref{sec:flags}
can be (permanently) stored in different files
for convenient compilation, viewing and distribution.
To this end, the package defines a command
to pass on compilation to a different file:

%%%%%%%%%%%%%%%%%%%%%%%%%%%%%%%%%%%%%%%%
\DescribeMacro{\childdocforward}
The command |\childdocforward| redirects processing to
another source file:
%
\begin{center}
\begin{tabular}{l}
|\input{childdoc.def}|\\
|\childdocforward[|\textit{main}|]{|\textit{dest}|}|\\
\end{tabular}
\end{center}
%
The argument \textit{dest} is the destination file
(without extension).
It should be the main file or one of the child files.
Note that further \textsf{childdoc} directives
such as |\childdocof| and |\childdocforward|
in the indicated file will be processed in this form.
The optional argument \textit{main}
passes on directly to the main file \textit{main}
while pretending to compile the child \textit{dest}.
This form behaves as if \textit{dest}
issues |\childdocof{|\textit{main}|}| right away,
and no further \textsf{childdoc} directives will be processed.

%%%%%%%%%%%%%%%%%%%%%%%%%%%%%%%%%%%%%%%%
\DescribeMacro{\...prefix}
In the alternative form |\childdocforwardprefix|,
%
\begin{center}
\begin{tabular}{l}
|\input{childdoc.def}|\\
|\childdocforwardprefix[|\textit{main}|]{|\textit{prefix}|}{|\textit{dest}|}|
\end{tabular}
\end{center}
%
the destination file is determined by a pattern
depending on the current file:
To make this work, the current file must be called
`{\textit{prefix}\hspace{0.2em}\textit{suffix}}'
with \textit{prefix} matching precisely the argument.
Processing is then passed on to the file
`{\textit{dest}\hspace{0.2em}\textit{suffix}}'.
Surely, the same effect is achieved by
directly specifying the
argument `{\textit{dest}\hspace{0.2em}\textit{suffix}}'
in the first form.
However, that requires to set up a different file
for each child. With the alternative form of the command
all these files can have exactly the same content
which simplifies setting them up and maintaining them.

For example, the following file |draft.tex|
with a compilation flag |\version| as described in \secref{sec:flags}
compiles the main document as a draft:
%
\begin{center}
\begin{tabular}{l}
|\def\version{draft}|\\
|\input{childdoc.def}|\\
|\childdocforward{|\textit{main}|}|
\end{tabular}
\end{center}
%
Likewise, the following files |final|\textit{nn}|.tex|
compile the final version of the child document
|child|\textit{nn}|.tex|:
%
\begin{center}
\begin{tabular}{l}
|\def\version{final}|\\
|\input{childdoc.def}|\\
|\childdocforwardprefix{final}{child}|
\end{tabular}
\end{center}
%

Note that when several versions of a main file and/or of each child file
are to be generated, it may be convenient to set up a |Makefile| or
shell script to automatise the process.

%%%%%%%%%%%%%%%%%%%%%%%%%%%%%%%%%%%%%%%%%%%%%%%%%%%%%%%%%%%%%%%%%%%%%%%%%%%%%%%%
\subsection{Command Line Processing}
\label{sec:commandline}

The effect of redirection files can also be achieved by invoking
the \LaTeX{} compiler with a more elaborate command line.
Most conveniently this should be done as part
of a shell script or a |Makefile|.

When using \textsf{childdoc} in the main file, the following
command lines effectively perform a redirection
(note that depending on the shell being used,
backslashes may have to be doubled: `|\|' $\to$ `|\\|'):
%
\begin{center}
|... -jobname "|\textit{target}|" |\\|"|[\textit{flags}]%
|\input{childdoc.def}\childdocforward[|\textit{main}|]{|\textit{dest}|}"|
\end{center}
%
Here \textit{target} is the name of the output file,
\textit{main} is the name of the main file
and \textit{dest} is the name of the main or child file to be processed
(all filenames without extensions).
The optional argument \textit{main} can be omitted
if \textit{main} matches \textit{dest}.
Optionally, compilation \textit{flags} can be defined via |\def| commands.
This command line makes the \TeX{} engine believe
it is compiling the file \textit{target}
whose content is specified as the latter parameter.
The provided code then forwards the processing to
\textit{main} or \textit{dest} as described in \secref{sec:forward}.

%%%%%%%%%%%%%%%%%%%%%%%%%%%%%%%%%%%%%%%%%%%%%%%%%%%%%%%%%%%%%%%%%%%%%%%%%%%%%%%%
\subsection{Include by Input}
\label{sec:input}

Including child documents by |\include| has some restrictions by design.
Most notably, the content of a child document always occupies
its own set of pages; pages cannot be shared between child documents.
Usually, this behaviour makes perfect sense
because each child document contain an essential part of the document.
However, in some situations it may be desirable to compose
a document from a collection of parts
without having mandatory page breaks between then.
For this case, the package
provides a mechanism to include parts
by |\input| which can also be processed individually.
However, by construction this mechanism
requires manual handling of the content to be output.

%%%%%%%%%%%%%%%%%%%%%%%%%%%%%%%%%%%%%%%%
\DescribeMacro{\ifchilddocmanual}
The main file should be prepared as usual, see \secref{sec:include}.
However, the document body must make a distinction
between processing of an individual part and of the main document, e.g.:
%
\begin{center}
\begin{tabular}{l}
|\ifchilddocmanual|\\
|\input{\childdocname}|\\
|\||else|\\
\textit{document body with }|\input{|\textit{part}|}|\\
|\||fi|
\end{tabular}
\end{center}
%
The conditional |\ifchilddocmanual| is true whenever
a part to be included by |\input| is being compiled,
and the name of the part is stored in |\childdocname|.

%%%%%%%%%%%%%%%%%%%%%%%%%%%%%%%%%%%%%%%%
\DescribeMacro{\childdocby}
Each part to be included by |\input| should start with:
%
\begin{center}
\begin{tabular}{l}
|\input{childdoc.def}|\\
|\childdocby{|\textit{main}|}|\\
\end{tabular}
\end{center}
%
The directive |\childdocby| is similar to |\childdocof|
described in \secref{sec:include},
but the subsequent selection of content must be done manually.
To that end, both |\ifchilddoc| and |\ifchilddocmanual|
will be true upon processing of a part,
and the name of the part is stored in |\childdocname|.
Note that |\jobname| will be set to the filename of the current part
so that each part receives an individual |.aux| file
that does not interfere with the |.aux| file(s) of the main document.
This behaviour can be altered by the alternative form
|\childdocby[*]{|\textit{main}|}| (with a non-empty optional argument)
which uses the |.aux| file of the main document
by setting |\jobname| to \textit{main}.

%%%%%%%%%%%%%%%%%%%%%%%%%%%%%%%%%%%%%%%%%%%%%%%%%%%%%%%%%%%%%%%%%%%%%%%%%%%%%%%%
\subsection{Driver Development}
\label{sec:driver}

The \textsf{childdoc} mechanism can also be use for the development
of definition files such as \LaTeX{} styles or classes.
This case differs from the above setup with multiple parts
included by |\include| in that no |\includeonly| should be invoked.
This can be achieved by starting the include file
(before |\ProvidesPackage|) with:
%
\begin{center}
\begin{tabular}{l}
|\input{childdoc.def}|\\
|\childdocforward{|\textit{main}|}|\\
\end{tabular}
\end{center}
%
or alternatively with:
%
\begin{center}
\begin{tabular}{l}
|\input{childdoc.def}|\\
|\childdocby{|\textit{main}|}|\\
\end{tabular}
\end{center}
%
Both forms have slightly different effects as described above.
The main file is prepared as usual, see \secref{sec:include}.

%%%%%%%%%%%%%%%%%%%%%%%%%%%%%%%%%%%%%%%%%%%%%%%%%%%%%%%%%%%%%%%%%%%%%%%%%%%%%%%%
\subsection{Legacy Detection}
\label{sec:detection}

The directive |\childdocmain| in the main file can detect
whether the complete document or merely a child is to be compiled
even without using the directive |\childdocof|.
This method is deprecated because it is less robust
and there is no compelling reason to use it;
it is merely provided for backward compatibility
and it may be removed in future versions.

If the detection mechanism is to be used,
it is mandatory to correctly specify
the filename of the main file as the argument of |\childdocmain|:
%
\begin{center}
\begin{tabular}{l}
|\input{childdoc.def}|\\
|\childdocmain{|\textit{main}|}|\\
\end{tabular}
\end{center}
%
If |\jobname| does not match the argument \textit{main} of |\childdocmain|,
it is assumed that |\jobname| points to the child file to be compiled.
When using |\childdocmain| with the main file specified as argument,
it suffices to start a child file
with just |\input{|\textit{main}|}|
without loading of the package and using |\childdocof|.
If instead all processing is done
with the appropriate \textsf{childdoc} directives,
the argument of \textit{main} of |\childdocmain| can be empty.

An alternative version of the command line processing described
in \secref{sec:commandline} using the detection mechanism reads:
%
\begin{center}
|... -jobname "|\textit{target}|" "|[\textit{flags}]%
[|\def\jobname{|\textit{dest}|}|]|\input{|\textit{main}|}"|
\end{center}

%%%%%%%%%%%%%%%%%%%%%%%%%%%%%%%%%%%%%%%%%%%%%%%%%%%%%%%%%%%%%%%%%%%%%%%%%%%%%%%%
\subsection{Manual Code}
\label{sec:manual}

In case one cannot be certain whether the definitions file |childdoc.def|
is installed on the target \TeX{} distribution
and one prefers not to ship it,
it is conceivable to paste a few relevant commands into the sources.

To that end, drop all statements |\input{childdoc.def}|
and perform the replacements as outlined below.
Instead of |\childdocmain{|\textit{main}|}| add the following code
to the top of the main file:
%
\begin{center}
\begin{tabular}{l}
|\||ifdefined\childdocname\endinput\||fi\newif\ifchilddoc|\\
|\edef\childdocname{\scantokens\expandafter{\jobname\noexpand}}|\\
|\def\childdocmain{|\textit{main}|}\||ifx\childdocmain\childdocname\||else|\\
|\childdoctrue\includeonly{\childdocname}\let\jobname\childdocmain\||fi|\\
\end{tabular}
\end{center}
%
Instead of |\childdocof{|\textit{main}|}| just include the main file
at the top of each child file:
%
\begin{center}
|\input{|\textit{main}|}|
\end{center}
%
A simple redirection |\childdocforward{|\textit{dest}|}| is achieved by:
%
\begin{center}
|\def\jobname{|\textit{dest}|}\input{\jobname}|
\end{center}
%
The redirection with prefix
|\childdocforwardprefix[|\textit{prefix}|]{|\textit{dest}|}|
is accomplished by:
%
\begin{center}
\begin{tabular}{l}
|{\edef\jobname{\scantokens\expandafter{\jobname\noexpand}}|\\
|\def\redirectjob |\textit{prefix}|#1~~~{\gdef\jobname{|\textit{dest}|#1}}|\\
|\expandafter\redirectjob\jobname~~~}\input{\jobname}|
\end{tabular}
\end{center}

In an alternative approach,
child documents can be compiled by a specific command line
without additional code or specific definitions:
%
\begin{center}
|... -jobname "|\textit{target}|" "|[\textit{flags}]%
|\includeonly{|\textit{dest}|}\input{|\textit{main}|}"|
\end{center}
%

%%%%%%%%%%%%%%%%%%%%%%%%%%%%%%%%%%%%%%%%%%%%%%%%%%%%%%%%%%%%%%%%%%%%%%%%%%%%%%%%
%%%%%%%%%%%%%%%%%%%%%%%%%%%%%%%%%%%%%%%%%%%%%%%%%%%%%%%%%%%%%%%%%%%%%%%%%%%%%%%%
\section{Information}

%%%%%%%%%%%%%%%%%%%%%%%%%%%%%%%%%%%%%%%%%%%%%%%%%%%%%%%%%%%%%%%%%%%%%%%%%%%%%%%%
\subsection{Copyright}

Copyright \copyright{} 2017--2018 Niklas Beisert

This work may be distributed and/or modified under the
conditions of the \LaTeX{} Project Public License, either version 1.3
of this license or (at your option) any later version.
The latest version of this license is in
  \url{http://www.latex-project.org/lppl.txt}
and version 1.3 or later is part of all distributions of \LaTeX{}
version 2005/12/01 or later.

This work has the LPPL maintenance status `maintained'.

The Current Maintainer of this work is Niklas Beisert.

This work consists of the files |README.txt|, |childdoc.ins| and |childdoc.dtx|
as well as the derived files |childdoc.def|, |cdocsamp.tex|
with |cdocsch1.tex|, |cdocsch2.tex|, |cdocspt3.tex|, |cdocspt4.tex|,
|cdocsdrf.tex|, |cdocsfn1.tex|, |cdocsfn2.tex|
as well as |childdoc.pdf|.

%%%%%%%%%%%%%%%%%%%%%%%%%%%%%%%%%%%%%%%%%%%%%%%%%%%%%%%%%%%%%%%%%%%%%%%%%%%%%%%%
\subsection{Files and Installation}

The package consists of the files:
%
\begin{center}
\begin{tabular}{ll}
    |README.txt|   & readme file \\
    |childdoc.ins| & installation file \\
    |childdoc.dtx| & source file \\
    |childdoc.def| & definition file \\
    |cdocsamp.tex| & sample main file \\
    |cdocsch1.tex| & sample include file \\
    |cdocsch2.tex| & sample include file \\
    |cdocspt3.tex| & sample part file \\
    |cdocspt4.tex| & sample part file \\
    |cdocsdrf.tex| & sample redirection file \\
    |cdocsfn1.tex| & sample redirection file \\
    |cdocsfn2.tex| & sample redirection file \\
    |childdoc.pdf| & manual
\end{tabular}
\end{center}
%
The distribution consists of the files
|README.txt|, |childdoc.ins| and |childdoc.dtx|.
%
\begin{itemize}
\item
Run (pdf)\LaTeX{} on |childdoc.dtx|
to compile the manual |childdoc.pdf| (this file).
\item
Run \LaTeX{} on |childdoc.ins| to create the definitions file |childdoc.def|
and the sample |cdocsamp.tex| with include files
|cdocsch1.tex|, |cdocsch2.tex|, |cdocspt3.tex|, |cdocspt4.tex|,
|cdocsdrf.tex|, |cdocsfn1.tex|, |cdocsfn2.tex|.
Then copy the file |childdoc.def| to an appropriate directory of your \LaTeX{}
distribution, e.g.\ \textit{texmf-root}|/tex/latex/childdoc|.
\end{itemize}

%%%%%%%%%%%%%%%%%%%%%%%%%%%%%%%%%%%%%%%%%%%%%%%%%%%%%%%%%%%%%%%%%%%%%%%%%%%%%%%%
\subsection{Related CTAN Packages}

There are several other packages which offer a similar functionality:
%
\begin{itemize}
\item
The packages
\href{http://ctan.org/pkg/docmute}{\textsf{docmute}},
\href{http://ctan.org/pkg/includex}{\textsf{includex}} and
\href{http://ctan.org/pkg/standalone}{\textsf{standalone}}
provide commands to include only the document body of
a child file thus allowing both files to be compiled individually.
\item
The packages \href{http://ctan.org/pkg/subdocs}{\textsf{subdocs}}
and \href{http://ctan.org/pkg/subfiles}{\textsf{subfiles}}
provide structures in which the main and child documents can be
encapsulated and allowing them to be compiled individually.
The inclusion mechanism is different from the conventional |\include|.
\item
The package \href{http://ctan.org/pkg/combine}{\textsf{combine}}
is an elaborate solution to combine several documents into one.
\end{itemize}
%
See also the CTAN topic \href{http://ctan.org/topic/subdocs}{\textsf{subdocs}}
for further related packages.
The present package differs from the above solutions in that
a document structure constructed with the conventional |\include| mechanism
just needs two extra commands at the top of every file
such that all constituent files can be compiled individually.

%%%%%%%%%%%%%%%%%%%%%%%%%%%%%%%%%%%%%%%%%%%%%%%%%%%%%%%%%%%%%%%%%%%%%%%%%%%%%%%%
%\subsection{Feature Suggestions}
%
%The following is a list of features which may be useful for future
%versions of this package:
%%
%\begin{itemize}
%\item
%\ldots
%\end{itemize}

%%%%%%%%%%%%%%%%%%%%%%%%%%%%%%%%%%%%%%%%%%%%%%%%%%%%%%%%%%%%%%%%%%%%%%%%%%%%%%%%
\subsection{Revision History}

%%%%%%%%%%%%%%%%%%%%%%%%%%%%%%%%%%%%%%%%
\paragraph{v2.0:} 2018/12/30

\begin{itemize}
\item
immediate forward processing
\item
added |\childdocby| mechanism
\item
manual restructured
\end{itemize}

%%%%%%%%%%%%%%%%%%%%%%%%%%%%%%%%%%%%%%%%
\paragraph{v1.6:} 2018/01/17

\begin{itemize}
\item
application for development of include files
\item
corrections to manual
\end{itemize}

%%%%%%%%%%%%%%%%%%%%%%%%%%%%%%%%%%%%%%%%
\paragraph{v1.5:} 2017/05/21

\begin{itemize}
\item
more complete structuring introduced
\item
|\childdocof| introduced
\item
|\childdoc| renamed to |\childdocmain|
\item
|\childredirect| renamed to |\childdocforward| and |\childdocforwardprefix|
and functionality expanded
\end{itemize}

%%%%%%%%%%%%%%%%%%%%%%%%%%%%%%%%%%%%%%%%
\paragraph{v1.0:} 2017/04/27

\begin{itemize}
\item
manual and install package
\item
first version published on CTAN
\end{itemize}

%%%%%%%%%%%%%%%%%%%%%%%%%%%%%%%%%%%%%%%%
\paragraph{v0.6:} 2017/04/26

\begin{itemize}
\item
redirection mechanism added
\end{itemize}

%%%%%%%%%%%%%%%%%%%%%%%%%%%%%%%%%%%%%%%%
\paragraph{v0.5:} 2017/04/26

\begin{itemize}
\item
functionality in definition file
\end{itemize}


%%%%%%%%%%%%%%%%%%%%%%%%%%%%%%%%%%%%%%%%%%%%%%%%%%%%%%%%%%%%%%%%%%%%%%%%%%%%%%%%
%%%%%%%%%%%%%%%%%%%%%%%%%%%%%%%%%%%%%%%%%%%%%%%%%%%%%%%%%%%%%%%%%%%%%%%%%%%%%%%%
%%%%%%%%%%%%%%%%%%%%%%%%%%%%%%%%%%%%%%%%%%%%%%%%%%%%%%%%%%%%%%%%%%%%%%%%%%%%%%%%
\appendix

\settowidth\MacroIndent{\rmfamily\scriptsize 000\ }

 \DocInput{childdoc.dtx}

\end{document}
%</driver>
% \fi
%
% %%%%%%%%%%%%%%%%%%%%%%%%%%%%%%%%%%%%%%%%%%%%%%%%%%%%%%%%%%%%%%%%%%%%%%%%%%%%%%
% %%%%%%%%%%%%%%%%%%%%%%%%%%%%%%%%%%%%%%%%%%%%%%%%%%%%%%%%%%%%%%%%%%%%%%%%%%%%%%
% \section{Sample}
%\iffalse
%<*samplemain>
%\fi
%
% The following presents a sample document
% with two chapters, two parts, a title page,
% a compile flag as well as three forwarding files to set the flag.
% It consists of eight |.tex| files:
% \begin{center}
% \begin{tabular}{ll}
% |cdocsamp.tex|&main file\\
% |cdocsch1.tex|&include file for chapter 1\\
% |cdocsch2.tex|&include file for chapter 2\\
% |cdocspt3.tex|&include file for part 3\\
% |cdocspt4.tex|&include file for part 4\\
% |cdocsdrf.tex|&forwarding file for main file in draft mode\\
% |cdocsfi1.tex|&forwarding file for final version of chapter 1\\
% |cdocsfi2.tex|&forwarding file for final version of chapter 2\\
% \end{tabular}
% \end{center}
% Each of the eight files can be compiled directly by the \LaTeX{} compiler.
%
% %%%%%%%%%%%%%%%%%%%%%%%%%%%%%%%%%%%%%%
% \paragraph{Main File.}
%
% The main file is called |cdocsamp.tex|.
%
% Load the \textsf{childdoc} definitions and
% declare the filename for the main document:
%    \begin{macrocode}
\input{childdoc.def}
\childdocmain{}
%    \end{macrocode}

% Optional override for |\version| flag:
%    \begin{macrocode}
%%\ifchilddoc\else\providecommand{\version}{draft}\fi
%    \end{macrocode}

% Define the default values for the |\version| flag
% (|final| for the main file and |draft| for childs):
%    \begin{macrocode}
\ifchilddoc
\providecommand{\version}{draft}
\else
\providecommand{\version}{final}
\fi
%    \end{macrocode}

% Load the standard document class:
%    \begin{macrocode}
\documentclass[12pt]{article}
%    \end{macrocode}

% Start the document body:
%    \begin{macrocode}
\begin{document}
%    \end{macrocode}

% Declare a title page.
% Print title, part of document being processed and version flag:
%    \begin{macrocode}
\addtocounter{page}{-1}
\begin{center}
{\LARGE\bfseries{}childdoc example\par}
\vspace{1cm}
\ifchilddoc
\ifchilddocmanual part\else chapter\fi:
`\childdocname' of `\childdocjob'\par
\else
main document: `\childdocjob'\par
\fi
version: \version\par
\end{center}
\newpage
%    \end{macrocode}

% Manually include selected file,
% otherwise process as usual:
%    \begin{macrocode}
\ifchilddocmanual
\section*{part `\childdocname'}
\input{\childdocname}
\else
%    \end{macrocode}

% Include the two chapters:
%    \begin{macrocode}
\include{cdocsch1}
\include{cdocsch2}
%    \end{macrocode}

% Include the two parts unless only chapters should be displayed:
%    \begin{macrocode}
\ifchilddoc\else
\section{part three}
\input{cdocspt3}
\section{part four}
\input{cdocspt4}
\fi
%    \end{macrocode}

% Process as usual until here:
%    \begin{macrocode}
\fi
%    \end{macrocode}

% End of document body:
%    \begin{macrocode}
\end{document}
%    \end{macrocode}
%\iffalse
%</samplemain>
%\fi
%
% %%%%%%%%%%%%%%%%%%%%%%%%%%%%%%%%%%%%%%
% \paragraph{Chapter Include Files.}
%
% The include files are called |cdocsch1.tex| and |cdocsch2.tex|.
%
%\iffalse
%<*samplechap1|samplechap2>
%\fi

% Optional override for |\version| flag:
%    \begin{macrocode}
%%\providecommand{\version}{final}
%    \end{macrocode}

% Include the main document:
%    \begin{macrocode}
\input{childdoc.def}
\childdocof{cdocsamp}
%    \end{macrocode}

%\iffalse
%</samplechap1|samplechap2>
%\fi
%
%\iffalse
%<*samplechap1>
%\fi
% Some text for chapter 1:
%    \begin{macrocode}
\section{one}
some text in chapter one
%    \end{macrocode}

%\iffalse
%</samplechap1>
%\fi
% Some text for chapter 2:
%\iffalse
%<*samplechap2>
%\fi
%    \begin{macrocode}
\section{two}
more text in chapter two
%    \end{macrocode}

%\iffalse
%</samplechap2>
%\fi
%
% %%%%%%%%%%%%%%%%%%%%%%%%%%%%%%%%%%%%%%
% \paragraph{Part Include Files.}
%
% The include files are called |cdocspt3.tex| and |cdocspt4.tex|.
%
%\iffalse
%<*samplepart3|samplepart4>
%\fi

% Optional override for |\version| flag:
%    \begin{macrocode}
%%\providecommand{\version}{final}
%    \end{macrocode}

% Include the main document:
%    \begin{macrocode}
\input{childdoc.def}
\childdocby{cdocsamp}
%    \end{macrocode}

%\iffalse
%</samplepart3|samplepart4>
%\fi
%
%\iffalse
%<*samplepart3>
%\fi
% Some text for part 3:
%    \begin{macrocode}
some text in part three
%    \end{macrocode}

%\iffalse
%</samplepart3>
%\fi
% Some text for part 4:
%\iffalse
%<*samplepart4>
%\fi
%    \begin{macrocode}
more text in part four
%    \end{macrocode}

%\iffalse
%</samplepart4>
%\fi
%
% %%%%%%%%%%%%%%%%%%%%%%%%%%%%%%%%%%%%%%
% \paragraph{Forwarding for a Complete Draft.}
%
% The following forwarding file |cdocsdrf.tex|
% compiles the main document in draft mode:
%\iffalse
%<*sampledraft>
%\fi
%    \begin{macrocode}
\def\version{draft}
\input{childdoc.def}
\childdocforward{cdocsamp}
%    \end{macrocode}

%\iffalse
%</sampledraft>
%\fi
%
% %%%%%%%%%%%%%%%%%%%%%%%%%%%%%%%%%%%%%%
% \paragraph{Forwarding for Final Version of the Chapters.}
%
% The following forwarding files |cdocsfn1.tex| and |cdocsfn2.tex|
% (with identical content)
% compile the final versions of the child documents
% |cdocsch1.tex| and |cdocsch2.tex|, respectively:
%\iffalse
%<*samplefinal>
%\fi
%    \begin{macrocode}
\def\version{final}
\input{childdoc.def}
\childdocforwardprefix[cdocsamp]{cdocsfn}{cdocsch}
%    \end{macrocode}

%\iffalse
%</samplefinal>
%\fi
%
% %%%%%%%%%%%%%%%%%%%%%%%%%%%%%%%%%%%%%%
% \paragraph{Command Line Processing.}
%
% The following three command lines generate the output files
% |cdocscld|, |cdocscl1| and |cdocscl2|
% which should be identical to
% |cdocsdrf|, |cdocsch1| and |cdocsfn2|, respectively:
% \begin{center}
% \begin{tabular}{l}
% |latex -jobname cdocscld \|\\
% |  "\def\version{draft}\input{childdoc.def}\childdocforward{cdocsamp}"|\\
% |latex -jobname cdocscl1 \|\\
% |  "\input{childdoc.def}\childdocforward[cdocsamp]{cdocsch1}"|\\
% |latex -jobname cdocscl2 \|\\
% |  "\def\version{final}\input{childdoc.def}\childdocforward{cdocsch2}"|
% \end{tabular}
% \end{center}
% Note that the trailing backslash on each first line
% merely continues the input to the second line
% (for convenient cut ant paste).
% Furthermore, the command |latex| can be replaced by any
% of its alternative versions such as |pdflatex|.
%
% %%%%%%%%%%%%%%%%%%%%%%%%%%%%%%%%%%%%%%%%%%%%%%%%%%%%%%%%%%%%%%%%%%%%%%%%%%%%%%
% %%%%%%%%%%%%%%%%%%%%%%%%%%%%%%%%%%%%%%%%%%%%%%%%%%%%%%%%%%%%%%%%%%%%%%%%%%%%%%
% \section{Implementation}
%\iffalse
%<*package>
%\fi
%
% This section describes the definitions file |childdoc.def|.

% The definitions cannot be loaded using |\usepackage| or |\RequirePackage|
% which has a mechanism to prevent loading a style file more than once.
% When loading the definitions by means of |\input|
% multiple instances have to be prevented manually:
%\iffalse
%This code needs to be before the `\ProvidesFile' directive
%which is defined at the beginning of this file.
%Therefore it is also placed there and commented out here.
%</package>
%<*discard>
%\fi
%    \begin{macrocode}
\ifdefined\childdocmain\endinput\fi
%    \end{macrocode}
%\iffalse
%</discard>
%<*package>
%\fi
%
% \macro{\ifchilddoc}
% \macro{\ifchilddocmanual}
% The conditional |\ifchilddoc| tells whether a
% child (true) or main (false) document is being compiled.
% The conditional |\ifchilddocmanual| tells whether
% the |\includeonly| mechanism is used (false) or
% the selection of child files must be performed manually (true).
% The definitions initialise to false:
%    \begin{macrocode}
\newif\ifchilddoc
\newif\ifchilddocmanual
%    \end{macrocode}

% \macro{\childdocname}
% \macro{\childdocjob}
% The macro |\childdocname| stores the name of the main document
% to be compiled. The macro |\childdocjob| stores the name of
% the document on which the \LaTeX{} compiler was originally invoked.
% The content of |\jobname| cannot be compared
% to filenames specified in the source due to different catcodes.
% The following code rescans |\jobname|, stores the result
% in |\childdocname| and saves a copy in |\childdocjob|:
%    \begin{macrocode}
\edef\childdocname{\scantokens\expandafter{\jobname\noexpand}}
\let\childdocjob\childdocname
%    \end{macrocode}

% \macro{\childdocdisable}
% The macro |\childdocdisable| prevents the main file
% from being processed more than once.
% At this stage, the main document command |\childdocmain|
% is assumed to be called once again where it should do nothing.
% Any subsequent call to it should prevent
% a secondary processing of the main document
% It overwrites the forwarding commands
% |\childdocof| and |\childdocforward|
% with empty macros to prevent further inclusions of the main document:
%    \begin{macrocode}
\newcommand{\childdocdisable}
{
  \renewcommand{\childdocmain}[1]{\renewcommand{\childdocmain}[1]{\endinput}}
  \renewcommand{\childdocof}[1]{}
  \renewcommand{\childdocby}[2][]{}
  \renewcommand{\childdocforward}[2][]{}
  \renewcommand{\childdocdisable}{}
}
%    \end{macrocode}

% \macro{\childdocmain}
% The macro |\childdocmain| is to be called at the top of the main file
% with nothing or the main filename (without extension) as argument.
% First, it breaks loops.
% If the argument is not empty and does not match |\childdocname|
% (which is set by the first inclusion of |childdoc.def|),
% |\ifchilddoc| is set to true, |\includeonly| is applied to the child file
% and |\jobname| is set to the main file
% (for proper handling of |.aux| files):
%    \begin{macrocode}
\newcommand{\childdocmain}[1]
{
  \childdocdisable\childdocmain{}
  \if?#1?\else
    \begingroup
      \def\childdoctmp{#1}
      \ifx\childdoctmp\childdocname
        \def\childdoctmp{}
      \else
        \def\childdoctmp
        {
          \childdoctrue
          \includeonly{\childdocname}
          \def\childdocjob{#1}
          \def\jobname{#1}
        }
      \fi
      \expandafter
    \endgroup
    \childdoctmp
  \fi
}
%    \end{macrocode}

% \macro{\childdocof}
% The command |\childdocof| redirects
% compilation to the main file |#1|.
%    \begin{macrocode}
\newcommand{\childdocof}[1]
{
  \childdocdisable
  \childdoctrue
  \includeonly{\childdocname}
  \def\jobname{#1}
  \def\childdocjob{#1}
  \input{#1}
}
%    \end{macrocode}

% \macro{\childdocby}
% The command |\childdocby| ....
%    \begin{macrocode}
\newcommand{\childdocby}[2][]
{
  \childdocdisable
  \childdoctrue
  \childdocmanualtrue
  \if?#1?\else
    \def\jobname{#2}
  \fi
  \def\childdocjob{#2}
  \input{#2}
  \endinput
}
%    \end{macrocode}

% \macro{\childdocforward}
% The command |\childdocforward| redirects
% compilation to the main file or
% (if the optional argument is given) a child file.
% Parameters are set as if the main file
% or a child file starting with |\childdocof| was compiled.
% Then compilation is handed over to the main file:
%    \begin{macrocode}
\newcommand{\childdocforward}[2][]
{
  \begingroup
    \if?#1?
      \def\childdoctmp
      {
        \def\childdocname{#2}
        \def\childdocjob{#2}
        \def\jobname{#2}
        \input{#2}
        \endinput
      }
    \else
      \def\childdoctmp
      {
        \childdocdisable
        \def\childdocname{#2}
        \childdoctrue
        \includeonly{#2}
        \def\childdocjob{#1}
        \def\jobname{#1}
        \input{#1}
        \endinput
      }
    \fi
    \expandafter
  \endgroup
  \childdoctmp
}
%    \end{macrocode}

% \macro{\childdocforwardprefix}
% The command |\childdocforwardprefix| redirects
% compilation to the main or a child file by means of a pattern.
% The prefix |#1| in the current filename is replaced by |#2|
% and the suffix of the current filename is kept
% (it is assumed that the filename does not contain the substring `|~~~|'
% which is used as a delimiter).
% Compilation is handed over to the new file by |\childdocforward|:
%    \begin{macrocode}
\newcommand{\childdocforwardprefix}[3][]
{
  \begingroup
    \def\childdocextract #2##1~~~{\def\childdoctmp{\childdocforward[#1]{#3##1}}}
    \expandafter\childdocextract\childdocname~~~
    \expandafter
  \endgroup
  \childdoctmp
}
%    \end{macrocode}

% \macro{\childdoc}
% The deprecated macro |\childdoc| is a legacy version of |\childdocmain|:
%    \begin{macrocode}
\newcommand{\childdoc}{\childdocmain}
%    \end{macrocode}

% \macro{\childdocredirect}
% The deprecated macro |\childdocredirect| is a legacy version
% of |\childdocforward| and |\childdocforwardprefix|:
%    \begin{macrocode}
\newcommand{\childdocredirect}[2][]
{
  \begingroup
    \if?#1?
      \def\childdoctmp{\childdocforward{#2}}
    \else
      \def\childdoctmp{\childdocforwardprefix{#1}{#2}}
    \fi
    \expandafter
  \endgroup
  \childdoctmp
}
%    \end{macrocode}

%\iffalse
%</package>
%\fi
%
\endinput
|\\
|\childdocforwardprefix[|\textit{main}|]{|\textit{prefix}|}{|\textit{dest}|}|
\end{tabular}
\end{center}
%
the destination file is determined by a pattern
depending on the current file:
To make this work, the current file must be called
`{\textit{prefix}\hspace{0.2em}\textit{suffix}}'
with \textit{prefix} matching precisely the argument.
Processing is then passed on to the file
`{\textit{dest}\hspace{0.2em}\textit{suffix}}'.
Surely, the same effect is achieved by
directly specifying the
argument `{\textit{dest}\hspace{0.2em}\textit{suffix}}'
in the first form.
However, that requires to set up a different file
for each child. With the alternative form of the command
all these files can have exactly the same content
which simplifies setting them up and maintaining them.

For example, the following file |draft.tex|
with a compilation flag |\version| as described in \secref{sec:flags}
compiles the main document as a draft:
%
\begin{center}
\begin{tabular}{l}
|\def\version{draft}|\\
|% \iffalse
%
% childdoc.dtx Copyright (C) 2017-2018 Niklas Beisert
%
% This work may be distributed and/or modified under the
% conditions of the LaTeX Project Public License, either version 1.3
% of this license or (at your option) any later version.
% The latest version of this license is in
%   http://www.latex-project.org/lppl.txt
% and version 1.3 or later is part of all distributions of LaTeX
% version 2005/12/01 or later.
%
% This work has the LPPL maintenance status `maintained'.
%
% The Current Maintainer of this work is Niklas Beisert.
%
% This work consists of the files childdoc.dtx and childdoc.ins
% and the derived files childdoc.def and cdocsamp.tex with
% cdocsch1.tex, cdocsch2.tex, cdocsdrf.tex, cdocsfn1.tex, cdocsfn2.tex.
%
%<package>\ifdefined\childdocmain\endinput\fi
%<package>\ProvidesFile{childdoc.def}[2018/12/30 v2.0 child document driver]
%<samplemain>\ProvidesFile{cdocsamp.tex}[2018/12/30 v2.0 sample for childdoc]
%<*driver>
%\ProvidesFile{childdoc.drv}[2018/12/30 v2.0 childdoc reference manual file]
\PassOptionsToClass{10pt,a4paper}{article}
\documentclass{ltxdoc}

\usepackage[margin=35mm]{geometry}
\usepackage{hyperref}
\usepackage{hyperxmp}
\usepackage[usenames]{color}

\hypersetup{colorlinks=true}
\hypersetup{pdfstartview=FitH}
\hypersetup{pdfpagemode=UseNone}
\hypersetup{pdfsource={}}
\hypersetup{pdflang={en-UK}}
\hypersetup{pdfcopyright={Copyright 2017-2018 Niklas Beisert.
  This work may be distributed and/or modified under the
  conditions of the LaTeX Project Public License, either version 1.3
  of this license or (at your option) any later version.}}
\hypersetup{pdflicenseurl={http://www.latex-project.org/lppl.txt}}
\hypersetup{pdfcontactaddress={ETH Zurich, ITP, HIT K,
  Wolfgang-Pauli-Strasse 27}}
\hypersetup{pdfcontactpostcode={8093}}
\hypersetup{pdfcontactcity={Zurich}}
\hypersetup{pdfcontactcountry={Switzerland}}
\hypersetup{pdfcontactemail={nbeisert@itp.phys.ethz.ch}}
\hypersetup{pdfcontacturl={http://people.phys.ethz.ch/\xmptilde nbeisert/}}

\newcommand{\secref}[1]{\hyperref[#1]{section \ref*{#1}}}

\parskip1ex
\parindent0pt
\let\olditemize\itemize
\def\itemize{\olditemize\parskip0pt}

\begin{document}

\title{The \textsf{childdoc} Package}
\hypersetup{pdftitle={The childdoc Package}}
\author{Niklas Beisert\\[2ex]
  Institut f\"ur Theoretische Physik\\
  Eidgen\"ossische Technische Hochschule Z\"urich\\
  Wolfgang-Pauli-Strasse 27, 8093 Z\"urich, Switzerland\\[1ex]
  \href{mailto:nbeisert@itp.phys.ethz.ch}
  {\texttt{nbeisert@itp.phys.ethz.ch}}}
\hypersetup{pdfauthor={Niklas Beisert}}
\hypersetup{pdfsubject={Manual for the LaTeX2e Package childdoc}}
\date{30 December 2018, \textsf{v2.0}}
\maketitle

\begin{abstract}\noindent
\textsf{childdoc} is a \LaTeXe{} package
that enables the direct compilation
of document sections included by |\include|
to individual files.
\end{abstract}

\begingroup
\parskip0ex
\tableofcontents
\endgroup

%%%%%%%%%%%%%%%%%%%%%%%%%%%%%%%%%%%%%%%%%%%%%%%%%%%%%%%%%%%%%%%%%%%%%%%%%%%%%%%%
%%%%%%%%%%%%%%%%%%%%%%%%%%%%%%%%%%%%%%%%%%%%%%%%%%%%%%%%%%%%%%%%%%%%%%%%%%%%%%%%
\section{Introduction}

\LaTeX{} provides a mechanism to structure a large document (such as a book)
into a main file and several child files (containing the chapters)
using the |\include| command.
This mechanism is beneficial for documents
which span hundreds of pages in order to
make the source file(s) more manageable.
Moreover, compilation can be restricted to
selected child files by means of the |\includeonly| command.
The latter feature can be used to reduce the compilation time while editing
(this was significantly more useful in the earlier days of \LaTeX{})
or to generate a smaller document which is easier to navigate.
Another application of |\includeonly| is to generate
documents consisting of selected parts of the complete document.

However, there are a few drawbacks of the plain |\include| mechanism:
\begin{itemize}
\item
The child files cannot be compiled on their own,
they can only be compiled via the main file.
A naive editing environment
(such as a text editor with an option
to have the current file processed by \LaTeX)
may require one to switch to the main file before compiling;
attempting to compile the child file produces errors.
\item
The main file must be modified (each time)
to adjust the |\includeonly| command
to the present needs. This easily leaves the main file in a messy state.
\item
The generated document will always carry the filename
of the main document. This is inconvenient if
several child files are to be compiled and
to be kept for distribution.
\end{itemize}

The present package provides a simple interface
to make child files individually compilable by \LaTeX{}.
Compiling a child file then has the same effect as compiling
the main file with an |\includeonly| command
to select the appropriate child.
Moreover the generated document will carry the name of the child
rather than the main file.
This resolves all three above issues.

This feature is meant to make the editing of books,
thesis documents and lecture notes somewhat more convenient.
However, the package can also be used efficiently for
composing a series of documents (such as exercise sheets)
which are typically distributed individually.
It then assists the author in generating the individual documents
(potentially in different versions)
as well as a document containing the collected series.
Another application is in developing style files
or other kinds of included material
where compilation of the style file could redirect
to a sample or test file.

%%%%%%%%%%%%%%%%%%%%%%%%%%%%%%%%%%%%%%%%%%%%%%%%%%%%%%%%%%%%%%%%%%%%%%%%%%%%%%%%
%%%%%%%%%%%%%%%%%%%%%%%%%%%%%%%%%%%%%%%%%%%%%%%%%%%%%%%%%%%%%%%%%%%%%%%%%%%%%%%%
\section{Usage}

First of all, the package \textsf{childdoc} is \emph{not} a standard
\LaTeXe{} |.sty| style file! Therefore it needs to be invoked in
a non-standard way.

%%%%%%%%%%%%%%%%%%%%%%%%%%%%%%%%%%%%%%%%%%%%%%%%%%%%%%%%%%%%%%%%%%%%%%%%%%%%%%%%
\subsection{Included Files}
\label{sec:include}

%%%%%%%%%%%%%%%%%%%%%%%%%%%%%%%%%%%%%%%%
\DescribeMacro{\childdocmain}
To use the package, add the commands
\begin{center}
\begin{tabular}{l}
|\input{childdoc.def}|\\
|\childdocmain{}|\\
\end{tabular}
\end{center}
at the very top of the main \LaTeX{} file,
in particular \emph{before} the |\documentclass| statement!
The argument of |\childdocmain| should be left empty
(but it must be present).

%%%%%%%%%%%%%%%%%%%%%%%%%%%%%%%%%%%%%%%%
\DescribeMacro{\childdocof}
Furthermore, add the commands
\begin{center}
\begin{tabular}{l}
|\input{childdoc.def}|\\
|\childdocof{|\textit{main}|}|\\
\end{tabular}
\end{center}
at the top of every child file \textit{child}
which is included by |\include{|\textit{child}|}|
from within the main file
(or at least for those files to be compiled individually).
The argument \textit{main} must be the filename of the main file.

There are a couple of
considerations in setting up the main and child documents:

%%%%%%%%%%%%%%%%%%%%%%%%%%%%%%%%%%%%%%%%
\paragraph{Restrictions.}

Please note the following restrictions:
\begin{itemize}
\item
|\childdocmain| must be called with one argument \textit{main}
to ensure compatibility with earlier version of the package.
It must either be empty (|\childdocmain{}|)
or precisely match the filename of the main file in which it is specified.
See \secref{sec:detection} for further information.
\item
The filename \textit{main} must be specified without the |.tex| extension.
\item
The filename \textit{main} is case sensitive
(even in case-insensitive file systems)
due to internal string comparison.
\item
The argument \textit{main} should be fully expanded, it cannot be a macro.
\item
Subdirectories and special characters should be avoided in filenames.
\item
The command |\childdocmain{|\textit{main}|}| must be followed by a whitespace.
It should not be followed immediately by another command
or by a comment mark `|%|'.
This is because the \TeX{} parser reads the token immediately following
the argument of |\childdocmain| and puts it
at the beginning of every child section;
however, a white\-space is ignored.
\end{itemize}

%%%%%%%%%%%%%%%%%%%%%%%%%%%%%%%%%%%%%%%%
\paragraph{Content of Main File.}

It is advisable to place all content in the child files included by |\include|.
Any output contained in the main file will appear in all child documents
unless suppressed manually;
it cannot be suppressed automatically by the |\includeonly| directive
and thus should normally be avoided.
A method to include some content in the main file
by means of conditional processing is described in \secref{sec:conditional}.

%%%%%%%%%%%%%%%%%%%%%%%%%%%%%%%%%%%%%%%%
\paragraph{Page Numbering.}

When only a part of the document is compiled,
the appropriate numbering of pages
(as well as other status parameters)
is determined from the |.aux| files.
The latter contain information from previous passes.
However this information needs to propagate through
all intermediate child documents.
Therefore the page numbering in child documents may well
be inconsistent until the complete document is compiled at least once.

A useful (if unconventional) way to always ensure a consistent
page numbering is to restart the numbering in each child document
and denote the pages by `\textit{child}|.|\textit{page}'
where \textit{child} represents the chapter/section number of the child file.
This can be achieved by the command
|\numberwithin{page}{|\textit{child}|}|
of the \textsf{amsmath} package
where \textit{child} can be |chapter| or |section|
depending on the chosen structuring.
Alternatively, one can modify the macro |\thepage| appropriately
and reset the counter |page| at the start of each child file.

%%%%%%%%%%%%%%%%%%%%%%%%%%%%%%%%%%%%%%%%%%%%%%%%%%%%%%%%%%%%%%%%%%%%%%%%%%%%%%%%
\subsection{Conditional Processing}
\label{sec:conditional}

The package provides a mechanism to compile different versions
of a document. To customise the versions further some conditional processing
can come in handy to distinguish which version is being compiled.
The package provides two macros to describe the compilation context:

%%%%%%%%%%%%%%%%%%%%%%%%%%%%%%%%%%%%%%%%
\DescribeMacro{\ifchilddoc}
The conditional |\ifchilddoc| distinguishes between the compilation of
child documents and the main document:
%
\begin{center}
|\ifchilddoc |\textit{child-code}| |[|\||else |\textit{main-code}]| \||fi|
\end{center}

%%%%%%%%%%%%%%%%%%%%%%%%%%%%%%%%%%%%%%%%
\DescribeMacro{\childdocname}
\DescribeMacro{\childdocjob}
The macro |\childdocname| contains the filename (without extension)
of the main or child file being processed.
Note that |\childdocjob| will always contain the name of the main file.

%%%%%%%%%%%%%%%%%%%%%%%%%%%%%%%%%%%%%%%%
\paragraph{Title Page.}

Conditional processing can be used to include a title or banner page
in the main document when proper precautions are taken.
Importantly, the code in the main file should ensure that the page counter
(as well as other status parameters which are stored in the |.aux| files)
takes the same value after the conditional processing.
Otherwise the page numbers may take divergent values
depending on which part is compiled.

For example, a title page could be declared by:
%
\begin{center}
\begin{tabular}{l}
|\ifchilddoc\||else|\\
|\addtocounter{page}{-1}|\\
\textit{code for title page}\\
|\newpage|\\
|\||fi|
\end{tabular}
\end{center}
%
A banner page for the child documents can be generated by:
%
\begin{center}
\begin{tabular}{l}
|\ifchilddoc|\\
|\addtocounter{page}{-1}|\\
\textit{code for banner page}\\
|\newpage|\\
|\||fi|
\end{tabular}
\end{center}
%
Here one could write a message such as:
\begin{center}
|This is the part \childdocname{} of \childdocjob{}.|
\end{center}

%%%%%%%%%%%%%%%%%%%%%%%%%%%%%%%%%%%%%%%%%%%%%%%%%%%%%%%%%%%%%%%%%%%%%%%%%%%%%%%%
\subsection{Flags}
\label{sec:flags}

The package makes it easy to generate different versions
of the main or child documents.
To this end compilation flags can be defined
and assigned different default values.
They will be particularly useful in conjunction
with the forwarding mechanism described in \secref{sec:forward}.

For example, it may be useful to have a flag |\version|
which can be set to |draft| or |final|.
The document source will contain some conditional code
depending on the value of |\version|.
Suppose further, the flag should default to |final| for the main file
and to |draft| for child files
which is a natural assignment for editing the document.
This is achieved by placing the following code
in the preamble of the main document
(below the |\childdocmain| directive):
%
\begin{center}
\begin{tabular}{l}
|\ifchilddoc|\\
|\providecommand{\version}{draft}|\\
|\||else|\\
|\providecommand{\version}{final}|\\
|\||fi|
\end{tabular}
\end{center}
%
The definition by |\providecommand| makes sure
that previous definitions are not overwritten.
Further statements |\providecommand{\version}{...}|
can thus be added before the above code to override it.

For the main file, one might add a line
(between |\childdocmain| and the above block)
%
\begin{center}
|%\ifchilddoc\||else\providecommand{\version}{draft}\||fi|
\end{center}
%
which can be uncommented to produce a draft version.
Likewise one can add a line to the very top of a child file
(above the |\childdocof{|\textit{main}|}| directive)
%
\begin{center}
|%\providecommand{\version}{final}|
\end{center}
%
which can be uncommented to produce the final version of this child document.

%%%%%%%%%%%%%%%%%%%%%%%%%%%%%%%%%%%%%%%%%%%%%%%%%%%%%%%%%%%%%%%%%%%%%%%%%%%%%%%%
\subsection{Forwarding}
\label{sec:forward}

Different versions of the main or child documents
using compilation flags as described in \secref{sec:flags}
can be (permanently) stored in different files
for convenient compilation, viewing and distribution.
To this end, the package defines a command
to pass on compilation to a different file:

%%%%%%%%%%%%%%%%%%%%%%%%%%%%%%%%%%%%%%%%
\DescribeMacro{\childdocforward}
The command |\childdocforward| redirects processing to
another source file:
%
\begin{center}
\begin{tabular}{l}
|\input{childdoc.def}|\\
|\childdocforward[|\textit{main}|]{|\textit{dest}|}|\\
\end{tabular}
\end{center}
%
The argument \textit{dest} is the destination file
(without extension).
It should be the main file or one of the child files.
Note that further \textsf{childdoc} directives
such as |\childdocof| and |\childdocforward|
in the indicated file will be processed in this form.
The optional argument \textit{main}
passes on directly to the main file \textit{main}
while pretending to compile the child \textit{dest}.
This form behaves as if \textit{dest}
issues |\childdocof{|\textit{main}|}| right away,
and no further \textsf{childdoc} directives will be processed.

%%%%%%%%%%%%%%%%%%%%%%%%%%%%%%%%%%%%%%%%
\DescribeMacro{\...prefix}
In the alternative form |\childdocforwardprefix|,
%
\begin{center}
\begin{tabular}{l}
|\input{childdoc.def}|\\
|\childdocforwardprefix[|\textit{main}|]{|\textit{prefix}|}{|\textit{dest}|}|
\end{tabular}
\end{center}
%
the destination file is determined by a pattern
depending on the current file:
To make this work, the current file must be called
`{\textit{prefix}\hspace{0.2em}\textit{suffix}}'
with \textit{prefix} matching precisely the argument.
Processing is then passed on to the file
`{\textit{dest}\hspace{0.2em}\textit{suffix}}'.
Surely, the same effect is achieved by
directly specifying the
argument `{\textit{dest}\hspace{0.2em}\textit{suffix}}'
in the first form.
However, that requires to set up a different file
for each child. With the alternative form of the command
all these files can have exactly the same content
which simplifies setting them up and maintaining them.

For example, the following file |draft.tex|
with a compilation flag |\version| as described in \secref{sec:flags}
compiles the main document as a draft:
%
\begin{center}
\begin{tabular}{l}
|\def\version{draft}|\\
|\input{childdoc.def}|\\
|\childdocforward{|\textit{main}|}|
\end{tabular}
\end{center}
%
Likewise, the following files |final|\textit{nn}|.tex|
compile the final version of the child document
|child|\textit{nn}|.tex|:
%
\begin{center}
\begin{tabular}{l}
|\def\version{final}|\\
|\input{childdoc.def}|\\
|\childdocforwardprefix{final}{child}|
\end{tabular}
\end{center}
%

Note that when several versions of a main file and/or of each child file
are to be generated, it may be convenient to set up a |Makefile| or
shell script to automatise the process.

%%%%%%%%%%%%%%%%%%%%%%%%%%%%%%%%%%%%%%%%%%%%%%%%%%%%%%%%%%%%%%%%%%%%%%%%%%%%%%%%
\subsection{Command Line Processing}
\label{sec:commandline}

The effect of redirection files can also be achieved by invoking
the \LaTeX{} compiler with a more elaborate command line.
Most conveniently this should be done as part
of a shell script or a |Makefile|.

When using \textsf{childdoc} in the main file, the following
command lines effectively perform a redirection
(note that depending on the shell being used,
backslashes may have to be doubled: `|\|' $\to$ `|\\|'):
%
\begin{center}
|... -jobname "|\textit{target}|" |\\|"|[\textit{flags}]%
|\input{childdoc.def}\childdocforward[|\textit{main}|]{|\textit{dest}|}"|
\end{center}
%
Here \textit{target} is the name of the output file,
\textit{main} is the name of the main file
and \textit{dest} is the name of the main or child file to be processed
(all filenames without extensions).
The optional argument \textit{main} can be omitted
if \textit{main} matches \textit{dest}.
Optionally, compilation \textit{flags} can be defined via |\def| commands.
This command line makes the \TeX{} engine believe
it is compiling the file \textit{target}
whose content is specified as the latter parameter.
The provided code then forwards the processing to
\textit{main} or \textit{dest} as described in \secref{sec:forward}.

%%%%%%%%%%%%%%%%%%%%%%%%%%%%%%%%%%%%%%%%%%%%%%%%%%%%%%%%%%%%%%%%%%%%%%%%%%%%%%%%
\subsection{Include by Input}
\label{sec:input}

Including child documents by |\include| has some restrictions by design.
Most notably, the content of a child document always occupies
its own set of pages; pages cannot be shared between child documents.
Usually, this behaviour makes perfect sense
because each child document contain an essential part of the document.
However, in some situations it may be desirable to compose
a document from a collection of parts
without having mandatory page breaks between then.
For this case, the package
provides a mechanism to include parts
by |\input| which can also be processed individually.
However, by construction this mechanism
requires manual handling of the content to be output.

%%%%%%%%%%%%%%%%%%%%%%%%%%%%%%%%%%%%%%%%
\DescribeMacro{\ifchilddocmanual}
The main file should be prepared as usual, see \secref{sec:include}.
However, the document body must make a distinction
between processing of an individual part and of the main document, e.g.:
%
\begin{center}
\begin{tabular}{l}
|\ifchilddocmanual|\\
|\input{\childdocname}|\\
|\||else|\\
\textit{document body with }|\input{|\textit{part}|}|\\
|\||fi|
\end{tabular}
\end{center}
%
The conditional |\ifchilddocmanual| is true whenever
a part to be included by |\input| is being compiled,
and the name of the part is stored in |\childdocname|.

%%%%%%%%%%%%%%%%%%%%%%%%%%%%%%%%%%%%%%%%
\DescribeMacro{\childdocby}
Each part to be included by |\input| should start with:
%
\begin{center}
\begin{tabular}{l}
|\input{childdoc.def}|\\
|\childdocby{|\textit{main}|}|\\
\end{tabular}
\end{center}
%
The directive |\childdocby| is similar to |\childdocof|
described in \secref{sec:include},
but the subsequent selection of content must be done manually.
To that end, both |\ifchilddoc| and |\ifchilddocmanual|
will be true upon processing of a part,
and the name of the part is stored in |\childdocname|.
Note that |\jobname| will be set to the filename of the current part
so that each part receives an individual |.aux| file
that does not interfere with the |.aux| file(s) of the main document.
This behaviour can be altered by the alternative form
|\childdocby[*]{|\textit{main}|}| (with a non-empty optional argument)
which uses the |.aux| file of the main document
by setting |\jobname| to \textit{main}.

%%%%%%%%%%%%%%%%%%%%%%%%%%%%%%%%%%%%%%%%%%%%%%%%%%%%%%%%%%%%%%%%%%%%%%%%%%%%%%%%
\subsection{Driver Development}
\label{sec:driver}

The \textsf{childdoc} mechanism can also be use for the development
of definition files such as \LaTeX{} styles or classes.
This case differs from the above setup with multiple parts
included by |\include| in that no |\includeonly| should be invoked.
This can be achieved by starting the include file
(before |\ProvidesPackage|) with:
%
\begin{center}
\begin{tabular}{l}
|\input{childdoc.def}|\\
|\childdocforward{|\textit{main}|}|\\
\end{tabular}
\end{center}
%
or alternatively with:
%
\begin{center}
\begin{tabular}{l}
|\input{childdoc.def}|\\
|\childdocby{|\textit{main}|}|\\
\end{tabular}
\end{center}
%
Both forms have slightly different effects as described above.
The main file is prepared as usual, see \secref{sec:include}.

%%%%%%%%%%%%%%%%%%%%%%%%%%%%%%%%%%%%%%%%%%%%%%%%%%%%%%%%%%%%%%%%%%%%%%%%%%%%%%%%
\subsection{Legacy Detection}
\label{sec:detection}

The directive |\childdocmain| in the main file can detect
whether the complete document or merely a child is to be compiled
even without using the directive |\childdocof|.
This method is deprecated because it is less robust
and there is no compelling reason to use it;
it is merely provided for backward compatibility
and it may be removed in future versions.

If the detection mechanism is to be used,
it is mandatory to correctly specify
the filename of the main file as the argument of |\childdocmain|:
%
\begin{center}
\begin{tabular}{l}
|\input{childdoc.def}|\\
|\childdocmain{|\textit{main}|}|\\
\end{tabular}
\end{center}
%
If |\jobname| does not match the argument \textit{main} of |\childdocmain|,
it is assumed that |\jobname| points to the child file to be compiled.
When using |\childdocmain| with the main file specified as argument,
it suffices to start a child file
with just |\input{|\textit{main}|}|
without loading of the package and using |\childdocof|.
If instead all processing is done
with the appropriate \textsf{childdoc} directives,
the argument of \textit{main} of |\childdocmain| can be empty.

An alternative version of the command line processing described
in \secref{sec:commandline} using the detection mechanism reads:
%
\begin{center}
|... -jobname "|\textit{target}|" "|[\textit{flags}]%
[|\def\jobname{|\textit{dest}|}|]|\input{|\textit{main}|}"|
\end{center}

%%%%%%%%%%%%%%%%%%%%%%%%%%%%%%%%%%%%%%%%%%%%%%%%%%%%%%%%%%%%%%%%%%%%%%%%%%%%%%%%
\subsection{Manual Code}
\label{sec:manual}

In case one cannot be certain whether the definitions file |childdoc.def|
is installed on the target \TeX{} distribution
and one prefers not to ship it,
it is conceivable to paste a few relevant commands into the sources.

To that end, drop all statements |\input{childdoc.def}|
and perform the replacements as outlined below.
Instead of |\childdocmain{|\textit{main}|}| add the following code
to the top of the main file:
%
\begin{center}
\begin{tabular}{l}
|\||ifdefined\childdocname\endinput\||fi\newif\ifchilddoc|\\
|\edef\childdocname{\scantokens\expandafter{\jobname\noexpand}}|\\
|\def\childdocmain{|\textit{main}|}\||ifx\childdocmain\childdocname\||else|\\
|\childdoctrue\includeonly{\childdocname}\let\jobname\childdocmain\||fi|\\
\end{tabular}
\end{center}
%
Instead of |\childdocof{|\textit{main}|}| just include the main file
at the top of each child file:
%
\begin{center}
|\input{|\textit{main}|}|
\end{center}
%
A simple redirection |\childdocforward{|\textit{dest}|}| is achieved by:
%
\begin{center}
|\def\jobname{|\textit{dest}|}\input{\jobname}|
\end{center}
%
The redirection with prefix
|\childdocforwardprefix[|\textit{prefix}|]{|\textit{dest}|}|
is accomplished by:
%
\begin{center}
\begin{tabular}{l}
|{\edef\jobname{\scantokens\expandafter{\jobname\noexpand}}|\\
|\def\redirectjob |\textit{prefix}|#1~~~{\gdef\jobname{|\textit{dest}|#1}}|\\
|\expandafter\redirectjob\jobname~~~}\input{\jobname}|
\end{tabular}
\end{center}

In an alternative approach,
child documents can be compiled by a specific command line
without additional code or specific definitions:
%
\begin{center}
|... -jobname "|\textit{target}|" "|[\textit{flags}]%
|\includeonly{|\textit{dest}|}\input{|\textit{main}|}"|
\end{center}
%

%%%%%%%%%%%%%%%%%%%%%%%%%%%%%%%%%%%%%%%%%%%%%%%%%%%%%%%%%%%%%%%%%%%%%%%%%%%%%%%%
%%%%%%%%%%%%%%%%%%%%%%%%%%%%%%%%%%%%%%%%%%%%%%%%%%%%%%%%%%%%%%%%%%%%%%%%%%%%%%%%
\section{Information}

%%%%%%%%%%%%%%%%%%%%%%%%%%%%%%%%%%%%%%%%%%%%%%%%%%%%%%%%%%%%%%%%%%%%%%%%%%%%%%%%
\subsection{Copyright}

Copyright \copyright{} 2017--2018 Niklas Beisert

This work may be distributed and/or modified under the
conditions of the \LaTeX{} Project Public License, either version 1.3
of this license or (at your option) any later version.
The latest version of this license is in
  \url{http://www.latex-project.org/lppl.txt}
and version 1.3 or later is part of all distributions of \LaTeX{}
version 2005/12/01 or later.

This work has the LPPL maintenance status `maintained'.

The Current Maintainer of this work is Niklas Beisert.

This work consists of the files |README.txt|, |childdoc.ins| and |childdoc.dtx|
as well as the derived files |childdoc.def|, |cdocsamp.tex|
with |cdocsch1.tex|, |cdocsch2.tex|, |cdocspt3.tex|, |cdocspt4.tex|,
|cdocsdrf.tex|, |cdocsfn1.tex|, |cdocsfn2.tex|
as well as |childdoc.pdf|.

%%%%%%%%%%%%%%%%%%%%%%%%%%%%%%%%%%%%%%%%%%%%%%%%%%%%%%%%%%%%%%%%%%%%%%%%%%%%%%%%
\subsection{Files and Installation}

The package consists of the files:
%
\begin{center}
\begin{tabular}{ll}
    |README.txt|   & readme file \\
    |childdoc.ins| & installation file \\
    |childdoc.dtx| & source file \\
    |childdoc.def| & definition file \\
    |cdocsamp.tex| & sample main file \\
    |cdocsch1.tex| & sample include file \\
    |cdocsch2.tex| & sample include file \\
    |cdocspt3.tex| & sample part file \\
    |cdocspt4.tex| & sample part file \\
    |cdocsdrf.tex| & sample redirection file \\
    |cdocsfn1.tex| & sample redirection file \\
    |cdocsfn2.tex| & sample redirection file \\
    |childdoc.pdf| & manual
\end{tabular}
\end{center}
%
The distribution consists of the files
|README.txt|, |childdoc.ins| and |childdoc.dtx|.
%
\begin{itemize}
\item
Run (pdf)\LaTeX{} on |childdoc.dtx|
to compile the manual |childdoc.pdf| (this file).
\item
Run \LaTeX{} on |childdoc.ins| to create the definitions file |childdoc.def|
and the sample |cdocsamp.tex| with include files
|cdocsch1.tex|, |cdocsch2.tex|, |cdocspt3.tex|, |cdocspt4.tex|,
|cdocsdrf.tex|, |cdocsfn1.tex|, |cdocsfn2.tex|.
Then copy the file |childdoc.def| to an appropriate directory of your \LaTeX{}
distribution, e.g.\ \textit{texmf-root}|/tex/latex/childdoc|.
\end{itemize}

%%%%%%%%%%%%%%%%%%%%%%%%%%%%%%%%%%%%%%%%%%%%%%%%%%%%%%%%%%%%%%%%%%%%%%%%%%%%%%%%
\subsection{Related CTAN Packages}

There are several other packages which offer a similar functionality:
%
\begin{itemize}
\item
The packages
\href{http://ctan.org/pkg/docmute}{\textsf{docmute}},
\href{http://ctan.org/pkg/includex}{\textsf{includex}} and
\href{http://ctan.org/pkg/standalone}{\textsf{standalone}}
provide commands to include only the document body of
a child file thus allowing both files to be compiled individually.
\item
The packages \href{http://ctan.org/pkg/subdocs}{\textsf{subdocs}}
and \href{http://ctan.org/pkg/subfiles}{\textsf{subfiles}}
provide structures in which the main and child documents can be
encapsulated and allowing them to be compiled individually.
The inclusion mechanism is different from the conventional |\include|.
\item
The package \href{http://ctan.org/pkg/combine}{\textsf{combine}}
is an elaborate solution to combine several documents into one.
\end{itemize}
%
See also the CTAN topic \href{http://ctan.org/topic/subdocs}{\textsf{subdocs}}
for further related packages.
The present package differs from the above solutions in that
a document structure constructed with the conventional |\include| mechanism
just needs two extra commands at the top of every file
such that all constituent files can be compiled individually.

%%%%%%%%%%%%%%%%%%%%%%%%%%%%%%%%%%%%%%%%%%%%%%%%%%%%%%%%%%%%%%%%%%%%%%%%%%%%%%%%
%\subsection{Feature Suggestions}
%
%The following is a list of features which may be useful for future
%versions of this package:
%%
%\begin{itemize}
%\item
%\ldots
%\end{itemize}

%%%%%%%%%%%%%%%%%%%%%%%%%%%%%%%%%%%%%%%%%%%%%%%%%%%%%%%%%%%%%%%%%%%%%%%%%%%%%%%%
\subsection{Revision History}

%%%%%%%%%%%%%%%%%%%%%%%%%%%%%%%%%%%%%%%%
\paragraph{v2.0:} 2018/12/30

\begin{itemize}
\item
immediate forward processing
\item
added |\childdocby| mechanism
\item
manual restructured
\end{itemize}

%%%%%%%%%%%%%%%%%%%%%%%%%%%%%%%%%%%%%%%%
\paragraph{v1.6:} 2018/01/17

\begin{itemize}
\item
application for development of include files
\item
corrections to manual
\end{itemize}

%%%%%%%%%%%%%%%%%%%%%%%%%%%%%%%%%%%%%%%%
\paragraph{v1.5:} 2017/05/21

\begin{itemize}
\item
more complete structuring introduced
\item
|\childdocof| introduced
\item
|\childdoc| renamed to |\childdocmain|
\item
|\childredirect| renamed to |\childdocforward| and |\childdocforwardprefix|
and functionality expanded
\end{itemize}

%%%%%%%%%%%%%%%%%%%%%%%%%%%%%%%%%%%%%%%%
\paragraph{v1.0:} 2017/04/27

\begin{itemize}
\item
manual and install package
\item
first version published on CTAN
\end{itemize}

%%%%%%%%%%%%%%%%%%%%%%%%%%%%%%%%%%%%%%%%
\paragraph{v0.6:} 2017/04/26

\begin{itemize}
\item
redirection mechanism added
\end{itemize}

%%%%%%%%%%%%%%%%%%%%%%%%%%%%%%%%%%%%%%%%
\paragraph{v0.5:} 2017/04/26

\begin{itemize}
\item
functionality in definition file
\end{itemize}


%%%%%%%%%%%%%%%%%%%%%%%%%%%%%%%%%%%%%%%%%%%%%%%%%%%%%%%%%%%%%%%%%%%%%%%%%%%%%%%%
%%%%%%%%%%%%%%%%%%%%%%%%%%%%%%%%%%%%%%%%%%%%%%%%%%%%%%%%%%%%%%%%%%%%%%%%%%%%%%%%
%%%%%%%%%%%%%%%%%%%%%%%%%%%%%%%%%%%%%%%%%%%%%%%%%%%%%%%%%%%%%%%%%%%%%%%%%%%%%%%%
\appendix

\settowidth\MacroIndent{\rmfamily\scriptsize 000\ }

 \DocInput{childdoc.dtx}

\end{document}
%</driver>
% \fi
%
% %%%%%%%%%%%%%%%%%%%%%%%%%%%%%%%%%%%%%%%%%%%%%%%%%%%%%%%%%%%%%%%%%%%%%%%%%%%%%%
% %%%%%%%%%%%%%%%%%%%%%%%%%%%%%%%%%%%%%%%%%%%%%%%%%%%%%%%%%%%%%%%%%%%%%%%%%%%%%%
% \section{Sample}
%\iffalse
%<*samplemain>
%\fi
%
% The following presents a sample document
% with two chapters, two parts, a title page,
% a compile flag as well as three forwarding files to set the flag.
% It consists of eight |.tex| files:
% \begin{center}
% \begin{tabular}{ll}
% |cdocsamp.tex|&main file\\
% |cdocsch1.tex|&include file for chapter 1\\
% |cdocsch2.tex|&include file for chapter 2\\
% |cdocspt3.tex|&include file for part 3\\
% |cdocspt4.tex|&include file for part 4\\
% |cdocsdrf.tex|&forwarding file for main file in draft mode\\
% |cdocsfi1.tex|&forwarding file for final version of chapter 1\\
% |cdocsfi2.tex|&forwarding file for final version of chapter 2\\
% \end{tabular}
% \end{center}
% Each of the eight files can be compiled directly by the \LaTeX{} compiler.
%
% %%%%%%%%%%%%%%%%%%%%%%%%%%%%%%%%%%%%%%
% \paragraph{Main File.}
%
% The main file is called |cdocsamp.tex|.
%
% Load the \textsf{childdoc} definitions and
% declare the filename for the main document:
%    \begin{macrocode}
\input{childdoc.def}
\childdocmain{}
%    \end{macrocode}

% Optional override for |\version| flag:
%    \begin{macrocode}
%%\ifchilddoc\else\providecommand{\version}{draft}\fi
%    \end{macrocode}

% Define the default values for the |\version| flag
% (|final| for the main file and |draft| for childs):
%    \begin{macrocode}
\ifchilddoc
\providecommand{\version}{draft}
\else
\providecommand{\version}{final}
\fi
%    \end{macrocode}

% Load the standard document class:
%    \begin{macrocode}
\documentclass[12pt]{article}
%    \end{macrocode}

% Start the document body:
%    \begin{macrocode}
\begin{document}
%    \end{macrocode}

% Declare a title page.
% Print title, part of document being processed and version flag:
%    \begin{macrocode}
\addtocounter{page}{-1}
\begin{center}
{\LARGE\bfseries{}childdoc example\par}
\vspace{1cm}
\ifchilddoc
\ifchilddocmanual part\else chapter\fi:
`\childdocname' of `\childdocjob'\par
\else
main document: `\childdocjob'\par
\fi
version: \version\par
\end{center}
\newpage
%    \end{macrocode}

% Manually include selected file,
% otherwise process as usual:
%    \begin{macrocode}
\ifchilddocmanual
\section*{part `\childdocname'}
\input{\childdocname}
\else
%    \end{macrocode}

% Include the two chapters:
%    \begin{macrocode}
\include{cdocsch1}
\include{cdocsch2}
%    \end{macrocode}

% Include the two parts unless only chapters should be displayed:
%    \begin{macrocode}
\ifchilddoc\else
\section{part three}
\input{cdocspt3}
\section{part four}
\input{cdocspt4}
\fi
%    \end{macrocode}

% Process as usual until here:
%    \begin{macrocode}
\fi
%    \end{macrocode}

% End of document body:
%    \begin{macrocode}
\end{document}
%    \end{macrocode}
%\iffalse
%</samplemain>
%\fi
%
% %%%%%%%%%%%%%%%%%%%%%%%%%%%%%%%%%%%%%%
% \paragraph{Chapter Include Files.}
%
% The include files are called |cdocsch1.tex| and |cdocsch2.tex|.
%
%\iffalse
%<*samplechap1|samplechap2>
%\fi

% Optional override for |\version| flag:
%    \begin{macrocode}
%%\providecommand{\version}{final}
%    \end{macrocode}

% Include the main document:
%    \begin{macrocode}
\input{childdoc.def}
\childdocof{cdocsamp}
%    \end{macrocode}

%\iffalse
%</samplechap1|samplechap2>
%\fi
%
%\iffalse
%<*samplechap1>
%\fi
% Some text for chapter 1:
%    \begin{macrocode}
\section{one}
some text in chapter one
%    \end{macrocode}

%\iffalse
%</samplechap1>
%\fi
% Some text for chapter 2:
%\iffalse
%<*samplechap2>
%\fi
%    \begin{macrocode}
\section{two}
more text in chapter two
%    \end{macrocode}

%\iffalse
%</samplechap2>
%\fi
%
% %%%%%%%%%%%%%%%%%%%%%%%%%%%%%%%%%%%%%%
% \paragraph{Part Include Files.}
%
% The include files are called |cdocspt3.tex| and |cdocspt4.tex|.
%
%\iffalse
%<*samplepart3|samplepart4>
%\fi

% Optional override for |\version| flag:
%    \begin{macrocode}
%%\providecommand{\version}{final}
%    \end{macrocode}

% Include the main document:
%    \begin{macrocode}
\input{childdoc.def}
\childdocby{cdocsamp}
%    \end{macrocode}

%\iffalse
%</samplepart3|samplepart4>
%\fi
%
%\iffalse
%<*samplepart3>
%\fi
% Some text for part 3:
%    \begin{macrocode}
some text in part three
%    \end{macrocode}

%\iffalse
%</samplepart3>
%\fi
% Some text for part 4:
%\iffalse
%<*samplepart4>
%\fi
%    \begin{macrocode}
more text in part four
%    \end{macrocode}

%\iffalse
%</samplepart4>
%\fi
%
% %%%%%%%%%%%%%%%%%%%%%%%%%%%%%%%%%%%%%%
% \paragraph{Forwarding for a Complete Draft.}
%
% The following forwarding file |cdocsdrf.tex|
% compiles the main document in draft mode:
%\iffalse
%<*sampledraft>
%\fi
%    \begin{macrocode}
\def\version{draft}
\input{childdoc.def}
\childdocforward{cdocsamp}
%    \end{macrocode}

%\iffalse
%</sampledraft>
%\fi
%
% %%%%%%%%%%%%%%%%%%%%%%%%%%%%%%%%%%%%%%
% \paragraph{Forwarding for Final Version of the Chapters.}
%
% The following forwarding files |cdocsfn1.tex| and |cdocsfn2.tex|
% (with identical content)
% compile the final versions of the child documents
% |cdocsch1.tex| and |cdocsch2.tex|, respectively:
%\iffalse
%<*samplefinal>
%\fi
%    \begin{macrocode}
\def\version{final}
\input{childdoc.def}
\childdocforwardprefix[cdocsamp]{cdocsfn}{cdocsch}
%    \end{macrocode}

%\iffalse
%</samplefinal>
%\fi
%
% %%%%%%%%%%%%%%%%%%%%%%%%%%%%%%%%%%%%%%
% \paragraph{Command Line Processing.}
%
% The following three command lines generate the output files
% |cdocscld|, |cdocscl1| and |cdocscl2|
% which should be identical to
% |cdocsdrf|, |cdocsch1| and |cdocsfn2|, respectively:
% \begin{center}
% \begin{tabular}{l}
% |latex -jobname cdocscld \|\\
% |  "\def\version{draft}\input{childdoc.def}\childdocforward{cdocsamp}"|\\
% |latex -jobname cdocscl1 \|\\
% |  "\input{childdoc.def}\childdocforward[cdocsamp]{cdocsch1}"|\\
% |latex -jobname cdocscl2 \|\\
% |  "\def\version{final}\input{childdoc.def}\childdocforward{cdocsch2}"|
% \end{tabular}
% \end{center}
% Note that the trailing backslash on each first line
% merely continues the input to the second line
% (for convenient cut ant paste).
% Furthermore, the command |latex| can be replaced by any
% of its alternative versions such as |pdflatex|.
%
% %%%%%%%%%%%%%%%%%%%%%%%%%%%%%%%%%%%%%%%%%%%%%%%%%%%%%%%%%%%%%%%%%%%%%%%%%%%%%%
% %%%%%%%%%%%%%%%%%%%%%%%%%%%%%%%%%%%%%%%%%%%%%%%%%%%%%%%%%%%%%%%%%%%%%%%%%%%%%%
% \section{Implementation}
%\iffalse
%<*package>
%\fi
%
% This section describes the definitions file |childdoc.def|.

% The definitions cannot be loaded using |\usepackage| or |\RequirePackage|
% which has a mechanism to prevent loading a style file more than once.
% When loading the definitions by means of |\input|
% multiple instances have to be prevented manually:
%\iffalse
%This code needs to be before the `\ProvidesFile' directive
%which is defined at the beginning of this file.
%Therefore it is also placed there and commented out here.
%</package>
%<*discard>
%\fi
%    \begin{macrocode}
\ifdefined\childdocmain\endinput\fi
%    \end{macrocode}
%\iffalse
%</discard>
%<*package>
%\fi
%
% \macro{\ifchilddoc}
% \macro{\ifchilddocmanual}
% The conditional |\ifchilddoc| tells whether a
% child (true) or main (false) document is being compiled.
% The conditional |\ifchilddocmanual| tells whether
% the |\includeonly| mechanism is used (false) or
% the selection of child files must be performed manually (true).
% The definitions initialise to false:
%    \begin{macrocode}
\newif\ifchilddoc
\newif\ifchilddocmanual
%    \end{macrocode}

% \macro{\childdocname}
% \macro{\childdocjob}
% The macro |\childdocname| stores the name of the main document
% to be compiled. The macro |\childdocjob| stores the name of
% the document on which the \LaTeX{} compiler was originally invoked.
% The content of |\jobname| cannot be compared
% to filenames specified in the source due to different catcodes.
% The following code rescans |\jobname|, stores the result
% in |\childdocname| and saves a copy in |\childdocjob|:
%    \begin{macrocode}
\edef\childdocname{\scantokens\expandafter{\jobname\noexpand}}
\let\childdocjob\childdocname
%    \end{macrocode}

% \macro{\childdocdisable}
% The macro |\childdocdisable| prevents the main file
% from being processed more than once.
% At this stage, the main document command |\childdocmain|
% is assumed to be called once again where it should do nothing.
% Any subsequent call to it should prevent
% a secondary processing of the main document
% It overwrites the forwarding commands
% |\childdocof| and |\childdocforward|
% with empty macros to prevent further inclusions of the main document:
%    \begin{macrocode}
\newcommand{\childdocdisable}
{
  \renewcommand{\childdocmain}[1]{\renewcommand{\childdocmain}[1]{\endinput}}
  \renewcommand{\childdocof}[1]{}
  \renewcommand{\childdocby}[2][]{}
  \renewcommand{\childdocforward}[2][]{}
  \renewcommand{\childdocdisable}{}
}
%    \end{macrocode}

% \macro{\childdocmain}
% The macro |\childdocmain| is to be called at the top of the main file
% with nothing or the main filename (without extension) as argument.
% First, it breaks loops.
% If the argument is not empty and does not match |\childdocname|
% (which is set by the first inclusion of |childdoc.def|),
% |\ifchilddoc| is set to true, |\includeonly| is applied to the child file
% and |\jobname| is set to the main file
% (for proper handling of |.aux| files):
%    \begin{macrocode}
\newcommand{\childdocmain}[1]
{
  \childdocdisable\childdocmain{}
  \if?#1?\else
    \begingroup
      \def\childdoctmp{#1}
      \ifx\childdoctmp\childdocname
        \def\childdoctmp{}
      \else
        \def\childdoctmp
        {
          \childdoctrue
          \includeonly{\childdocname}
          \def\childdocjob{#1}
          \def\jobname{#1}
        }
      \fi
      \expandafter
    \endgroup
    \childdoctmp
  \fi
}
%    \end{macrocode}

% \macro{\childdocof}
% The command |\childdocof| redirects
% compilation to the main file |#1|.
%    \begin{macrocode}
\newcommand{\childdocof}[1]
{
  \childdocdisable
  \childdoctrue
  \includeonly{\childdocname}
  \def\jobname{#1}
  \def\childdocjob{#1}
  \input{#1}
}
%    \end{macrocode}

% \macro{\childdocby}
% The command |\childdocby| ....
%    \begin{macrocode}
\newcommand{\childdocby}[2][]
{
  \childdocdisable
  \childdoctrue
  \childdocmanualtrue
  \if?#1?\else
    \def\jobname{#2}
  \fi
  \def\childdocjob{#2}
  \input{#2}
  \endinput
}
%    \end{macrocode}

% \macro{\childdocforward}
% The command |\childdocforward| redirects
% compilation to the main file or
% (if the optional argument is given) a child file.
% Parameters are set as if the main file
% or a child file starting with |\childdocof| was compiled.
% Then compilation is handed over to the main file:
%    \begin{macrocode}
\newcommand{\childdocforward}[2][]
{
  \begingroup
    \if?#1?
      \def\childdoctmp
      {
        \def\childdocname{#2}
        \def\childdocjob{#2}
        \def\jobname{#2}
        \input{#2}
        \endinput
      }
    \else
      \def\childdoctmp
      {
        \childdocdisable
        \def\childdocname{#2}
        \childdoctrue
        \includeonly{#2}
        \def\childdocjob{#1}
        \def\jobname{#1}
        \input{#1}
        \endinput
      }
    \fi
    \expandafter
  \endgroup
  \childdoctmp
}
%    \end{macrocode}

% \macro{\childdocforwardprefix}
% The command |\childdocforwardprefix| redirects
% compilation to the main or a child file by means of a pattern.
% The prefix |#1| in the current filename is replaced by |#2|
% and the suffix of the current filename is kept
% (it is assumed that the filename does not contain the substring `|~~~|'
% which is used as a delimiter).
% Compilation is handed over to the new file by |\childdocforward|:
%    \begin{macrocode}
\newcommand{\childdocforwardprefix}[3][]
{
  \begingroup
    \def\childdocextract #2##1~~~{\def\childdoctmp{\childdocforward[#1]{#3##1}}}
    \expandafter\childdocextract\childdocname~~~
    \expandafter
  \endgroup
  \childdoctmp
}
%    \end{macrocode}

% \macro{\childdoc}
% The deprecated macro |\childdoc| is a legacy version of |\childdocmain|:
%    \begin{macrocode}
\newcommand{\childdoc}{\childdocmain}
%    \end{macrocode}

% \macro{\childdocredirect}
% The deprecated macro |\childdocredirect| is a legacy version
% of |\childdocforward| and |\childdocforwardprefix|:
%    \begin{macrocode}
\newcommand{\childdocredirect}[2][]
{
  \begingroup
    \if?#1?
      \def\childdoctmp{\childdocforward{#2}}
    \else
      \def\childdoctmp{\childdocforwardprefix{#1}{#2}}
    \fi
    \expandafter
  \endgroup
  \childdoctmp
}
%    \end{macrocode}

%\iffalse
%</package>
%\fi
%
\endinput
|\\
|\childdocforward{|\textit{main}|}|
\end{tabular}
\end{center}
%
Likewise, the following files |final|\textit{nn}|.tex|
compile the final version of the child document
|child|\textit{nn}|.tex|:
%
\begin{center}
\begin{tabular}{l}
|\def\version{final}|\\
|% \iffalse
%
% childdoc.dtx Copyright (C) 2017-2018 Niklas Beisert
%
% This work may be distributed and/or modified under the
% conditions of the LaTeX Project Public License, either version 1.3
% of this license or (at your option) any later version.
% The latest version of this license is in
%   http://www.latex-project.org/lppl.txt
% and version 1.3 or later is part of all distributions of LaTeX
% version 2005/12/01 or later.
%
% This work has the LPPL maintenance status `maintained'.
%
% The Current Maintainer of this work is Niklas Beisert.
%
% This work consists of the files childdoc.dtx and childdoc.ins
% and the derived files childdoc.def and cdocsamp.tex with
% cdocsch1.tex, cdocsch2.tex, cdocsdrf.tex, cdocsfn1.tex, cdocsfn2.tex.
%
%<package>\ifdefined\childdocmain\endinput\fi
%<package>\ProvidesFile{childdoc.def}[2018/12/30 v2.0 child document driver]
%<samplemain>\ProvidesFile{cdocsamp.tex}[2018/12/30 v2.0 sample for childdoc]
%<*driver>
%\ProvidesFile{childdoc.drv}[2018/12/30 v2.0 childdoc reference manual file]
\PassOptionsToClass{10pt,a4paper}{article}
\documentclass{ltxdoc}

\usepackage[margin=35mm]{geometry}
\usepackage{hyperref}
\usepackage{hyperxmp}
\usepackage[usenames]{color}

\hypersetup{colorlinks=true}
\hypersetup{pdfstartview=FitH}
\hypersetup{pdfpagemode=UseNone}
\hypersetup{pdfsource={}}
\hypersetup{pdflang={en-UK}}
\hypersetup{pdfcopyright={Copyright 2017-2018 Niklas Beisert.
  This work may be distributed and/or modified under the
  conditions of the LaTeX Project Public License, either version 1.3
  of this license or (at your option) any later version.}}
\hypersetup{pdflicenseurl={http://www.latex-project.org/lppl.txt}}
\hypersetup{pdfcontactaddress={ETH Zurich, ITP, HIT K,
  Wolfgang-Pauli-Strasse 27}}
\hypersetup{pdfcontactpostcode={8093}}
\hypersetup{pdfcontactcity={Zurich}}
\hypersetup{pdfcontactcountry={Switzerland}}
\hypersetup{pdfcontactemail={nbeisert@itp.phys.ethz.ch}}
\hypersetup{pdfcontacturl={http://people.phys.ethz.ch/\xmptilde nbeisert/}}

\newcommand{\secref}[1]{\hyperref[#1]{section \ref*{#1}}}

\parskip1ex
\parindent0pt
\let\olditemize\itemize
\def\itemize{\olditemize\parskip0pt}

\begin{document}

\title{The \textsf{childdoc} Package}
\hypersetup{pdftitle={The childdoc Package}}
\author{Niklas Beisert\\[2ex]
  Institut f\"ur Theoretische Physik\\
  Eidgen\"ossische Technische Hochschule Z\"urich\\
  Wolfgang-Pauli-Strasse 27, 8093 Z\"urich, Switzerland\\[1ex]
  \href{mailto:nbeisert@itp.phys.ethz.ch}
  {\texttt{nbeisert@itp.phys.ethz.ch}}}
\hypersetup{pdfauthor={Niklas Beisert}}
\hypersetup{pdfsubject={Manual for the LaTeX2e Package childdoc}}
\date{30 December 2018, \textsf{v2.0}}
\maketitle

\begin{abstract}\noindent
\textsf{childdoc} is a \LaTeXe{} package
that enables the direct compilation
of document sections included by |\include|
to individual files.
\end{abstract}

\begingroup
\parskip0ex
\tableofcontents
\endgroup

%%%%%%%%%%%%%%%%%%%%%%%%%%%%%%%%%%%%%%%%%%%%%%%%%%%%%%%%%%%%%%%%%%%%%%%%%%%%%%%%
%%%%%%%%%%%%%%%%%%%%%%%%%%%%%%%%%%%%%%%%%%%%%%%%%%%%%%%%%%%%%%%%%%%%%%%%%%%%%%%%
\section{Introduction}

\LaTeX{} provides a mechanism to structure a large document (such as a book)
into a main file and several child files (containing the chapters)
using the |\include| command.
This mechanism is beneficial for documents
which span hundreds of pages in order to
make the source file(s) more manageable.
Moreover, compilation can be restricted to
selected child files by means of the |\includeonly| command.
The latter feature can be used to reduce the compilation time while editing
(this was significantly more useful in the earlier days of \LaTeX{})
or to generate a smaller document which is easier to navigate.
Another application of |\includeonly| is to generate
documents consisting of selected parts of the complete document.

However, there are a few drawbacks of the plain |\include| mechanism:
\begin{itemize}
\item
The child files cannot be compiled on their own,
they can only be compiled via the main file.
A naive editing environment
(such as a text editor with an option
to have the current file processed by \LaTeX)
may require one to switch to the main file before compiling;
attempting to compile the child file produces errors.
\item
The main file must be modified (each time)
to adjust the |\includeonly| command
to the present needs. This easily leaves the main file in a messy state.
\item
The generated document will always carry the filename
of the main document. This is inconvenient if
several child files are to be compiled and
to be kept for distribution.
\end{itemize}

The present package provides a simple interface
to make child files individually compilable by \LaTeX{}.
Compiling a child file then has the same effect as compiling
the main file with an |\includeonly| command
to select the appropriate child.
Moreover the generated document will carry the name of the child
rather than the main file.
This resolves all three above issues.

This feature is meant to make the editing of books,
thesis documents and lecture notes somewhat more convenient.
However, the package can also be used efficiently for
composing a series of documents (such as exercise sheets)
which are typically distributed individually.
It then assists the author in generating the individual documents
(potentially in different versions)
as well as a document containing the collected series.
Another application is in developing style files
or other kinds of included material
where compilation of the style file could redirect
to a sample or test file.

%%%%%%%%%%%%%%%%%%%%%%%%%%%%%%%%%%%%%%%%%%%%%%%%%%%%%%%%%%%%%%%%%%%%%%%%%%%%%%%%
%%%%%%%%%%%%%%%%%%%%%%%%%%%%%%%%%%%%%%%%%%%%%%%%%%%%%%%%%%%%%%%%%%%%%%%%%%%%%%%%
\section{Usage}

First of all, the package \textsf{childdoc} is \emph{not} a standard
\LaTeXe{} |.sty| style file! Therefore it needs to be invoked in
a non-standard way.

%%%%%%%%%%%%%%%%%%%%%%%%%%%%%%%%%%%%%%%%%%%%%%%%%%%%%%%%%%%%%%%%%%%%%%%%%%%%%%%%
\subsection{Included Files}
\label{sec:include}

%%%%%%%%%%%%%%%%%%%%%%%%%%%%%%%%%%%%%%%%
\DescribeMacro{\childdocmain}
To use the package, add the commands
\begin{center}
\begin{tabular}{l}
|\input{childdoc.def}|\\
|\childdocmain{}|\\
\end{tabular}
\end{center}
at the very top of the main \LaTeX{} file,
in particular \emph{before} the |\documentclass| statement!
The argument of |\childdocmain| should be left empty
(but it must be present).

%%%%%%%%%%%%%%%%%%%%%%%%%%%%%%%%%%%%%%%%
\DescribeMacro{\childdocof}
Furthermore, add the commands
\begin{center}
\begin{tabular}{l}
|\input{childdoc.def}|\\
|\childdocof{|\textit{main}|}|\\
\end{tabular}
\end{center}
at the top of every child file \textit{child}
which is included by |\include{|\textit{child}|}|
from within the main file
(or at least for those files to be compiled individually).
The argument \textit{main} must be the filename of the main file.

There are a couple of
considerations in setting up the main and child documents:

%%%%%%%%%%%%%%%%%%%%%%%%%%%%%%%%%%%%%%%%
\paragraph{Restrictions.}

Please note the following restrictions:
\begin{itemize}
\item
|\childdocmain| must be called with one argument \textit{main}
to ensure compatibility with earlier version of the package.
It must either be empty (|\childdocmain{}|)
or precisely match the filename of the main file in which it is specified.
See \secref{sec:detection} for further information.
\item
The filename \textit{main} must be specified without the |.tex| extension.
\item
The filename \textit{main} is case sensitive
(even in case-insensitive file systems)
due to internal string comparison.
\item
The argument \textit{main} should be fully expanded, it cannot be a macro.
\item
Subdirectories and special characters should be avoided in filenames.
\item
The command |\childdocmain{|\textit{main}|}| must be followed by a whitespace.
It should not be followed immediately by another command
or by a comment mark `|%|'.
This is because the \TeX{} parser reads the token immediately following
the argument of |\childdocmain| and puts it
at the beginning of every child section;
however, a white\-space is ignored.
\end{itemize}

%%%%%%%%%%%%%%%%%%%%%%%%%%%%%%%%%%%%%%%%
\paragraph{Content of Main File.}

It is advisable to place all content in the child files included by |\include|.
Any output contained in the main file will appear in all child documents
unless suppressed manually;
it cannot be suppressed automatically by the |\includeonly| directive
and thus should normally be avoided.
A method to include some content in the main file
by means of conditional processing is described in \secref{sec:conditional}.

%%%%%%%%%%%%%%%%%%%%%%%%%%%%%%%%%%%%%%%%
\paragraph{Page Numbering.}

When only a part of the document is compiled,
the appropriate numbering of pages
(as well as other status parameters)
is determined from the |.aux| files.
The latter contain information from previous passes.
However this information needs to propagate through
all intermediate child documents.
Therefore the page numbering in child documents may well
be inconsistent until the complete document is compiled at least once.

A useful (if unconventional) way to always ensure a consistent
page numbering is to restart the numbering in each child document
and denote the pages by `\textit{child}|.|\textit{page}'
where \textit{child} represents the chapter/section number of the child file.
This can be achieved by the command
|\numberwithin{page}{|\textit{child}|}|
of the \textsf{amsmath} package
where \textit{child} can be |chapter| or |section|
depending on the chosen structuring.
Alternatively, one can modify the macro |\thepage| appropriately
and reset the counter |page| at the start of each child file.

%%%%%%%%%%%%%%%%%%%%%%%%%%%%%%%%%%%%%%%%%%%%%%%%%%%%%%%%%%%%%%%%%%%%%%%%%%%%%%%%
\subsection{Conditional Processing}
\label{sec:conditional}

The package provides a mechanism to compile different versions
of a document. To customise the versions further some conditional processing
can come in handy to distinguish which version is being compiled.
The package provides two macros to describe the compilation context:

%%%%%%%%%%%%%%%%%%%%%%%%%%%%%%%%%%%%%%%%
\DescribeMacro{\ifchilddoc}
The conditional |\ifchilddoc| distinguishes between the compilation of
child documents and the main document:
%
\begin{center}
|\ifchilddoc |\textit{child-code}| |[|\||else |\textit{main-code}]| \||fi|
\end{center}

%%%%%%%%%%%%%%%%%%%%%%%%%%%%%%%%%%%%%%%%
\DescribeMacro{\childdocname}
\DescribeMacro{\childdocjob}
The macro |\childdocname| contains the filename (without extension)
of the main or child file being processed.
Note that |\childdocjob| will always contain the name of the main file.

%%%%%%%%%%%%%%%%%%%%%%%%%%%%%%%%%%%%%%%%
\paragraph{Title Page.}

Conditional processing can be used to include a title or banner page
in the main document when proper precautions are taken.
Importantly, the code in the main file should ensure that the page counter
(as well as other status parameters which are stored in the |.aux| files)
takes the same value after the conditional processing.
Otherwise the page numbers may take divergent values
depending on which part is compiled.

For example, a title page could be declared by:
%
\begin{center}
\begin{tabular}{l}
|\ifchilddoc\||else|\\
|\addtocounter{page}{-1}|\\
\textit{code for title page}\\
|\newpage|\\
|\||fi|
\end{tabular}
\end{center}
%
A banner page for the child documents can be generated by:
%
\begin{center}
\begin{tabular}{l}
|\ifchilddoc|\\
|\addtocounter{page}{-1}|\\
\textit{code for banner page}\\
|\newpage|\\
|\||fi|
\end{tabular}
\end{center}
%
Here one could write a message such as:
\begin{center}
|This is the part \childdocname{} of \childdocjob{}.|
\end{center}

%%%%%%%%%%%%%%%%%%%%%%%%%%%%%%%%%%%%%%%%%%%%%%%%%%%%%%%%%%%%%%%%%%%%%%%%%%%%%%%%
\subsection{Flags}
\label{sec:flags}

The package makes it easy to generate different versions
of the main or child documents.
To this end compilation flags can be defined
and assigned different default values.
They will be particularly useful in conjunction
with the forwarding mechanism described in \secref{sec:forward}.

For example, it may be useful to have a flag |\version|
which can be set to |draft| or |final|.
The document source will contain some conditional code
depending on the value of |\version|.
Suppose further, the flag should default to |final| for the main file
and to |draft| for child files
which is a natural assignment for editing the document.
This is achieved by placing the following code
in the preamble of the main document
(below the |\childdocmain| directive):
%
\begin{center}
\begin{tabular}{l}
|\ifchilddoc|\\
|\providecommand{\version}{draft}|\\
|\||else|\\
|\providecommand{\version}{final}|\\
|\||fi|
\end{tabular}
\end{center}
%
The definition by |\providecommand| makes sure
that previous definitions are not overwritten.
Further statements |\providecommand{\version}{...}|
can thus be added before the above code to override it.

For the main file, one might add a line
(between |\childdocmain| and the above block)
%
\begin{center}
|%\ifchilddoc\||else\providecommand{\version}{draft}\||fi|
\end{center}
%
which can be uncommented to produce a draft version.
Likewise one can add a line to the very top of a child file
(above the |\childdocof{|\textit{main}|}| directive)
%
\begin{center}
|%\providecommand{\version}{final}|
\end{center}
%
which can be uncommented to produce the final version of this child document.

%%%%%%%%%%%%%%%%%%%%%%%%%%%%%%%%%%%%%%%%%%%%%%%%%%%%%%%%%%%%%%%%%%%%%%%%%%%%%%%%
\subsection{Forwarding}
\label{sec:forward}

Different versions of the main or child documents
using compilation flags as described in \secref{sec:flags}
can be (permanently) stored in different files
for convenient compilation, viewing and distribution.
To this end, the package defines a command
to pass on compilation to a different file:

%%%%%%%%%%%%%%%%%%%%%%%%%%%%%%%%%%%%%%%%
\DescribeMacro{\childdocforward}
The command |\childdocforward| redirects processing to
another source file:
%
\begin{center}
\begin{tabular}{l}
|\input{childdoc.def}|\\
|\childdocforward[|\textit{main}|]{|\textit{dest}|}|\\
\end{tabular}
\end{center}
%
The argument \textit{dest} is the destination file
(without extension).
It should be the main file or one of the child files.
Note that further \textsf{childdoc} directives
such as |\childdocof| and |\childdocforward|
in the indicated file will be processed in this form.
The optional argument \textit{main}
passes on directly to the main file \textit{main}
while pretending to compile the child \textit{dest}.
This form behaves as if \textit{dest}
issues |\childdocof{|\textit{main}|}| right away,
and no further \textsf{childdoc} directives will be processed.

%%%%%%%%%%%%%%%%%%%%%%%%%%%%%%%%%%%%%%%%
\DescribeMacro{\...prefix}
In the alternative form |\childdocforwardprefix|,
%
\begin{center}
\begin{tabular}{l}
|\input{childdoc.def}|\\
|\childdocforwardprefix[|\textit{main}|]{|\textit{prefix}|}{|\textit{dest}|}|
\end{tabular}
\end{center}
%
the destination file is determined by a pattern
depending on the current file:
To make this work, the current file must be called
`{\textit{prefix}\hspace{0.2em}\textit{suffix}}'
with \textit{prefix} matching precisely the argument.
Processing is then passed on to the file
`{\textit{dest}\hspace{0.2em}\textit{suffix}}'.
Surely, the same effect is achieved by
directly specifying the
argument `{\textit{dest}\hspace{0.2em}\textit{suffix}}'
in the first form.
However, that requires to set up a different file
for each child. With the alternative form of the command
all these files can have exactly the same content
which simplifies setting them up and maintaining them.

For example, the following file |draft.tex|
with a compilation flag |\version| as described in \secref{sec:flags}
compiles the main document as a draft:
%
\begin{center}
\begin{tabular}{l}
|\def\version{draft}|\\
|\input{childdoc.def}|\\
|\childdocforward{|\textit{main}|}|
\end{tabular}
\end{center}
%
Likewise, the following files |final|\textit{nn}|.tex|
compile the final version of the child document
|child|\textit{nn}|.tex|:
%
\begin{center}
\begin{tabular}{l}
|\def\version{final}|\\
|\input{childdoc.def}|\\
|\childdocforwardprefix{final}{child}|
\end{tabular}
\end{center}
%

Note that when several versions of a main file and/or of each child file
are to be generated, it may be convenient to set up a |Makefile| or
shell script to automatise the process.

%%%%%%%%%%%%%%%%%%%%%%%%%%%%%%%%%%%%%%%%%%%%%%%%%%%%%%%%%%%%%%%%%%%%%%%%%%%%%%%%
\subsection{Command Line Processing}
\label{sec:commandline}

The effect of redirection files can also be achieved by invoking
the \LaTeX{} compiler with a more elaborate command line.
Most conveniently this should be done as part
of a shell script or a |Makefile|.

When using \textsf{childdoc} in the main file, the following
command lines effectively perform a redirection
(note that depending on the shell being used,
backslashes may have to be doubled: `|\|' $\to$ `|\\|'):
%
\begin{center}
|... -jobname "|\textit{target}|" |\\|"|[\textit{flags}]%
|\input{childdoc.def}\childdocforward[|\textit{main}|]{|\textit{dest}|}"|
\end{center}
%
Here \textit{target} is the name of the output file,
\textit{main} is the name of the main file
and \textit{dest} is the name of the main or child file to be processed
(all filenames without extensions).
The optional argument \textit{main} can be omitted
if \textit{main} matches \textit{dest}.
Optionally, compilation \textit{flags} can be defined via |\def| commands.
This command line makes the \TeX{} engine believe
it is compiling the file \textit{target}
whose content is specified as the latter parameter.
The provided code then forwards the processing to
\textit{main} or \textit{dest} as described in \secref{sec:forward}.

%%%%%%%%%%%%%%%%%%%%%%%%%%%%%%%%%%%%%%%%%%%%%%%%%%%%%%%%%%%%%%%%%%%%%%%%%%%%%%%%
\subsection{Include by Input}
\label{sec:input}

Including child documents by |\include| has some restrictions by design.
Most notably, the content of a child document always occupies
its own set of pages; pages cannot be shared between child documents.
Usually, this behaviour makes perfect sense
because each child document contain an essential part of the document.
However, in some situations it may be desirable to compose
a document from a collection of parts
without having mandatory page breaks between then.
For this case, the package
provides a mechanism to include parts
by |\input| which can also be processed individually.
However, by construction this mechanism
requires manual handling of the content to be output.

%%%%%%%%%%%%%%%%%%%%%%%%%%%%%%%%%%%%%%%%
\DescribeMacro{\ifchilddocmanual}
The main file should be prepared as usual, see \secref{sec:include}.
However, the document body must make a distinction
between processing of an individual part and of the main document, e.g.:
%
\begin{center}
\begin{tabular}{l}
|\ifchilddocmanual|\\
|\input{\childdocname}|\\
|\||else|\\
\textit{document body with }|\input{|\textit{part}|}|\\
|\||fi|
\end{tabular}
\end{center}
%
The conditional |\ifchilddocmanual| is true whenever
a part to be included by |\input| is being compiled,
and the name of the part is stored in |\childdocname|.

%%%%%%%%%%%%%%%%%%%%%%%%%%%%%%%%%%%%%%%%
\DescribeMacro{\childdocby}
Each part to be included by |\input| should start with:
%
\begin{center}
\begin{tabular}{l}
|\input{childdoc.def}|\\
|\childdocby{|\textit{main}|}|\\
\end{tabular}
\end{center}
%
The directive |\childdocby| is similar to |\childdocof|
described in \secref{sec:include},
but the subsequent selection of content must be done manually.
To that end, both |\ifchilddoc| and |\ifchilddocmanual|
will be true upon processing of a part,
and the name of the part is stored in |\childdocname|.
Note that |\jobname| will be set to the filename of the current part
so that each part receives an individual |.aux| file
that does not interfere with the |.aux| file(s) of the main document.
This behaviour can be altered by the alternative form
|\childdocby[*]{|\textit{main}|}| (with a non-empty optional argument)
which uses the |.aux| file of the main document
by setting |\jobname| to \textit{main}.

%%%%%%%%%%%%%%%%%%%%%%%%%%%%%%%%%%%%%%%%%%%%%%%%%%%%%%%%%%%%%%%%%%%%%%%%%%%%%%%%
\subsection{Driver Development}
\label{sec:driver}

The \textsf{childdoc} mechanism can also be use for the development
of definition files such as \LaTeX{} styles or classes.
This case differs from the above setup with multiple parts
included by |\include| in that no |\includeonly| should be invoked.
This can be achieved by starting the include file
(before |\ProvidesPackage|) with:
%
\begin{center}
\begin{tabular}{l}
|\input{childdoc.def}|\\
|\childdocforward{|\textit{main}|}|\\
\end{tabular}
\end{center}
%
or alternatively with:
%
\begin{center}
\begin{tabular}{l}
|\input{childdoc.def}|\\
|\childdocby{|\textit{main}|}|\\
\end{tabular}
\end{center}
%
Both forms have slightly different effects as described above.
The main file is prepared as usual, see \secref{sec:include}.

%%%%%%%%%%%%%%%%%%%%%%%%%%%%%%%%%%%%%%%%%%%%%%%%%%%%%%%%%%%%%%%%%%%%%%%%%%%%%%%%
\subsection{Legacy Detection}
\label{sec:detection}

The directive |\childdocmain| in the main file can detect
whether the complete document or merely a child is to be compiled
even without using the directive |\childdocof|.
This method is deprecated because it is less robust
and there is no compelling reason to use it;
it is merely provided for backward compatibility
and it may be removed in future versions.

If the detection mechanism is to be used,
it is mandatory to correctly specify
the filename of the main file as the argument of |\childdocmain|:
%
\begin{center}
\begin{tabular}{l}
|\input{childdoc.def}|\\
|\childdocmain{|\textit{main}|}|\\
\end{tabular}
\end{center}
%
If |\jobname| does not match the argument \textit{main} of |\childdocmain|,
it is assumed that |\jobname| points to the child file to be compiled.
When using |\childdocmain| with the main file specified as argument,
it suffices to start a child file
with just |\input{|\textit{main}|}|
without loading of the package and using |\childdocof|.
If instead all processing is done
with the appropriate \textsf{childdoc} directives,
the argument of \textit{main} of |\childdocmain| can be empty.

An alternative version of the command line processing described
in \secref{sec:commandline} using the detection mechanism reads:
%
\begin{center}
|... -jobname "|\textit{target}|" "|[\textit{flags}]%
[|\def\jobname{|\textit{dest}|}|]|\input{|\textit{main}|}"|
\end{center}

%%%%%%%%%%%%%%%%%%%%%%%%%%%%%%%%%%%%%%%%%%%%%%%%%%%%%%%%%%%%%%%%%%%%%%%%%%%%%%%%
\subsection{Manual Code}
\label{sec:manual}

In case one cannot be certain whether the definitions file |childdoc.def|
is installed on the target \TeX{} distribution
and one prefers not to ship it,
it is conceivable to paste a few relevant commands into the sources.

To that end, drop all statements |\input{childdoc.def}|
and perform the replacements as outlined below.
Instead of |\childdocmain{|\textit{main}|}| add the following code
to the top of the main file:
%
\begin{center}
\begin{tabular}{l}
|\||ifdefined\childdocname\endinput\||fi\newif\ifchilddoc|\\
|\edef\childdocname{\scantokens\expandafter{\jobname\noexpand}}|\\
|\def\childdocmain{|\textit{main}|}\||ifx\childdocmain\childdocname\||else|\\
|\childdoctrue\includeonly{\childdocname}\let\jobname\childdocmain\||fi|\\
\end{tabular}
\end{center}
%
Instead of |\childdocof{|\textit{main}|}| just include the main file
at the top of each child file:
%
\begin{center}
|\input{|\textit{main}|}|
\end{center}
%
A simple redirection |\childdocforward{|\textit{dest}|}| is achieved by:
%
\begin{center}
|\def\jobname{|\textit{dest}|}\input{\jobname}|
\end{center}
%
The redirection with prefix
|\childdocforwardprefix[|\textit{prefix}|]{|\textit{dest}|}|
is accomplished by:
%
\begin{center}
\begin{tabular}{l}
|{\edef\jobname{\scantokens\expandafter{\jobname\noexpand}}|\\
|\def\redirectjob |\textit{prefix}|#1~~~{\gdef\jobname{|\textit{dest}|#1}}|\\
|\expandafter\redirectjob\jobname~~~}\input{\jobname}|
\end{tabular}
\end{center}

In an alternative approach,
child documents can be compiled by a specific command line
without additional code or specific definitions:
%
\begin{center}
|... -jobname "|\textit{target}|" "|[\textit{flags}]%
|\includeonly{|\textit{dest}|}\input{|\textit{main}|}"|
\end{center}
%

%%%%%%%%%%%%%%%%%%%%%%%%%%%%%%%%%%%%%%%%%%%%%%%%%%%%%%%%%%%%%%%%%%%%%%%%%%%%%%%%
%%%%%%%%%%%%%%%%%%%%%%%%%%%%%%%%%%%%%%%%%%%%%%%%%%%%%%%%%%%%%%%%%%%%%%%%%%%%%%%%
\section{Information}

%%%%%%%%%%%%%%%%%%%%%%%%%%%%%%%%%%%%%%%%%%%%%%%%%%%%%%%%%%%%%%%%%%%%%%%%%%%%%%%%
\subsection{Copyright}

Copyright \copyright{} 2017--2018 Niklas Beisert

This work may be distributed and/or modified under the
conditions of the \LaTeX{} Project Public License, either version 1.3
of this license or (at your option) any later version.
The latest version of this license is in
  \url{http://www.latex-project.org/lppl.txt}
and version 1.3 or later is part of all distributions of \LaTeX{}
version 2005/12/01 or later.

This work has the LPPL maintenance status `maintained'.

The Current Maintainer of this work is Niklas Beisert.

This work consists of the files |README.txt|, |childdoc.ins| and |childdoc.dtx|
as well as the derived files |childdoc.def|, |cdocsamp.tex|
with |cdocsch1.tex|, |cdocsch2.tex|, |cdocspt3.tex|, |cdocspt4.tex|,
|cdocsdrf.tex|, |cdocsfn1.tex|, |cdocsfn2.tex|
as well as |childdoc.pdf|.

%%%%%%%%%%%%%%%%%%%%%%%%%%%%%%%%%%%%%%%%%%%%%%%%%%%%%%%%%%%%%%%%%%%%%%%%%%%%%%%%
\subsection{Files and Installation}

The package consists of the files:
%
\begin{center}
\begin{tabular}{ll}
    |README.txt|   & readme file \\
    |childdoc.ins| & installation file \\
    |childdoc.dtx| & source file \\
    |childdoc.def| & definition file \\
    |cdocsamp.tex| & sample main file \\
    |cdocsch1.tex| & sample include file \\
    |cdocsch2.tex| & sample include file \\
    |cdocspt3.tex| & sample part file \\
    |cdocspt4.tex| & sample part file \\
    |cdocsdrf.tex| & sample redirection file \\
    |cdocsfn1.tex| & sample redirection file \\
    |cdocsfn2.tex| & sample redirection file \\
    |childdoc.pdf| & manual
\end{tabular}
\end{center}
%
The distribution consists of the files
|README.txt|, |childdoc.ins| and |childdoc.dtx|.
%
\begin{itemize}
\item
Run (pdf)\LaTeX{} on |childdoc.dtx|
to compile the manual |childdoc.pdf| (this file).
\item
Run \LaTeX{} on |childdoc.ins| to create the definitions file |childdoc.def|
and the sample |cdocsamp.tex| with include files
|cdocsch1.tex|, |cdocsch2.tex|, |cdocspt3.tex|, |cdocspt4.tex|,
|cdocsdrf.tex|, |cdocsfn1.tex|, |cdocsfn2.tex|.
Then copy the file |childdoc.def| to an appropriate directory of your \LaTeX{}
distribution, e.g.\ \textit{texmf-root}|/tex/latex/childdoc|.
\end{itemize}

%%%%%%%%%%%%%%%%%%%%%%%%%%%%%%%%%%%%%%%%%%%%%%%%%%%%%%%%%%%%%%%%%%%%%%%%%%%%%%%%
\subsection{Related CTAN Packages}

There are several other packages which offer a similar functionality:
%
\begin{itemize}
\item
The packages
\href{http://ctan.org/pkg/docmute}{\textsf{docmute}},
\href{http://ctan.org/pkg/includex}{\textsf{includex}} and
\href{http://ctan.org/pkg/standalone}{\textsf{standalone}}
provide commands to include only the document body of
a child file thus allowing both files to be compiled individually.
\item
The packages \href{http://ctan.org/pkg/subdocs}{\textsf{subdocs}}
and \href{http://ctan.org/pkg/subfiles}{\textsf{subfiles}}
provide structures in which the main and child documents can be
encapsulated and allowing them to be compiled individually.
The inclusion mechanism is different from the conventional |\include|.
\item
The package \href{http://ctan.org/pkg/combine}{\textsf{combine}}
is an elaborate solution to combine several documents into one.
\end{itemize}
%
See also the CTAN topic \href{http://ctan.org/topic/subdocs}{\textsf{subdocs}}
for further related packages.
The present package differs from the above solutions in that
a document structure constructed with the conventional |\include| mechanism
just needs two extra commands at the top of every file
such that all constituent files can be compiled individually.

%%%%%%%%%%%%%%%%%%%%%%%%%%%%%%%%%%%%%%%%%%%%%%%%%%%%%%%%%%%%%%%%%%%%%%%%%%%%%%%%
%\subsection{Feature Suggestions}
%
%The following is a list of features which may be useful for future
%versions of this package:
%%
%\begin{itemize}
%\item
%\ldots
%\end{itemize}

%%%%%%%%%%%%%%%%%%%%%%%%%%%%%%%%%%%%%%%%%%%%%%%%%%%%%%%%%%%%%%%%%%%%%%%%%%%%%%%%
\subsection{Revision History}

%%%%%%%%%%%%%%%%%%%%%%%%%%%%%%%%%%%%%%%%
\paragraph{v2.0:} 2018/12/30

\begin{itemize}
\item
immediate forward processing
\item
added |\childdocby| mechanism
\item
manual restructured
\end{itemize}

%%%%%%%%%%%%%%%%%%%%%%%%%%%%%%%%%%%%%%%%
\paragraph{v1.6:} 2018/01/17

\begin{itemize}
\item
application for development of include files
\item
corrections to manual
\end{itemize}

%%%%%%%%%%%%%%%%%%%%%%%%%%%%%%%%%%%%%%%%
\paragraph{v1.5:} 2017/05/21

\begin{itemize}
\item
more complete structuring introduced
\item
|\childdocof| introduced
\item
|\childdoc| renamed to |\childdocmain|
\item
|\childredirect| renamed to |\childdocforward| and |\childdocforwardprefix|
and functionality expanded
\end{itemize}

%%%%%%%%%%%%%%%%%%%%%%%%%%%%%%%%%%%%%%%%
\paragraph{v1.0:} 2017/04/27

\begin{itemize}
\item
manual and install package
\item
first version published on CTAN
\end{itemize}

%%%%%%%%%%%%%%%%%%%%%%%%%%%%%%%%%%%%%%%%
\paragraph{v0.6:} 2017/04/26

\begin{itemize}
\item
redirection mechanism added
\end{itemize}

%%%%%%%%%%%%%%%%%%%%%%%%%%%%%%%%%%%%%%%%
\paragraph{v0.5:} 2017/04/26

\begin{itemize}
\item
functionality in definition file
\end{itemize}


%%%%%%%%%%%%%%%%%%%%%%%%%%%%%%%%%%%%%%%%%%%%%%%%%%%%%%%%%%%%%%%%%%%%%%%%%%%%%%%%
%%%%%%%%%%%%%%%%%%%%%%%%%%%%%%%%%%%%%%%%%%%%%%%%%%%%%%%%%%%%%%%%%%%%%%%%%%%%%%%%
%%%%%%%%%%%%%%%%%%%%%%%%%%%%%%%%%%%%%%%%%%%%%%%%%%%%%%%%%%%%%%%%%%%%%%%%%%%%%%%%
\appendix

\settowidth\MacroIndent{\rmfamily\scriptsize 000\ }

 \DocInput{childdoc.dtx}

\end{document}
%</driver>
% \fi
%
% %%%%%%%%%%%%%%%%%%%%%%%%%%%%%%%%%%%%%%%%%%%%%%%%%%%%%%%%%%%%%%%%%%%%%%%%%%%%%%
% %%%%%%%%%%%%%%%%%%%%%%%%%%%%%%%%%%%%%%%%%%%%%%%%%%%%%%%%%%%%%%%%%%%%%%%%%%%%%%
% \section{Sample}
%\iffalse
%<*samplemain>
%\fi
%
% The following presents a sample document
% with two chapters, two parts, a title page,
% a compile flag as well as three forwarding files to set the flag.
% It consists of eight |.tex| files:
% \begin{center}
% \begin{tabular}{ll}
% |cdocsamp.tex|&main file\\
% |cdocsch1.tex|&include file for chapter 1\\
% |cdocsch2.tex|&include file for chapter 2\\
% |cdocspt3.tex|&include file for part 3\\
% |cdocspt4.tex|&include file for part 4\\
% |cdocsdrf.tex|&forwarding file for main file in draft mode\\
% |cdocsfi1.tex|&forwarding file for final version of chapter 1\\
% |cdocsfi2.tex|&forwarding file for final version of chapter 2\\
% \end{tabular}
% \end{center}
% Each of the eight files can be compiled directly by the \LaTeX{} compiler.
%
% %%%%%%%%%%%%%%%%%%%%%%%%%%%%%%%%%%%%%%
% \paragraph{Main File.}
%
% The main file is called |cdocsamp.tex|.
%
% Load the \textsf{childdoc} definitions and
% declare the filename for the main document:
%    \begin{macrocode}
\input{childdoc.def}
\childdocmain{}
%    \end{macrocode}

% Optional override for |\version| flag:
%    \begin{macrocode}
%%\ifchilddoc\else\providecommand{\version}{draft}\fi
%    \end{macrocode}

% Define the default values for the |\version| flag
% (|final| for the main file and |draft| for childs):
%    \begin{macrocode}
\ifchilddoc
\providecommand{\version}{draft}
\else
\providecommand{\version}{final}
\fi
%    \end{macrocode}

% Load the standard document class:
%    \begin{macrocode}
\documentclass[12pt]{article}
%    \end{macrocode}

% Start the document body:
%    \begin{macrocode}
\begin{document}
%    \end{macrocode}

% Declare a title page.
% Print title, part of document being processed and version flag:
%    \begin{macrocode}
\addtocounter{page}{-1}
\begin{center}
{\LARGE\bfseries{}childdoc example\par}
\vspace{1cm}
\ifchilddoc
\ifchilddocmanual part\else chapter\fi:
`\childdocname' of `\childdocjob'\par
\else
main document: `\childdocjob'\par
\fi
version: \version\par
\end{center}
\newpage
%    \end{macrocode}

% Manually include selected file,
% otherwise process as usual:
%    \begin{macrocode}
\ifchilddocmanual
\section*{part `\childdocname'}
\input{\childdocname}
\else
%    \end{macrocode}

% Include the two chapters:
%    \begin{macrocode}
\include{cdocsch1}
\include{cdocsch2}
%    \end{macrocode}

% Include the two parts unless only chapters should be displayed:
%    \begin{macrocode}
\ifchilddoc\else
\section{part three}
\input{cdocspt3}
\section{part four}
\input{cdocspt4}
\fi
%    \end{macrocode}

% Process as usual until here:
%    \begin{macrocode}
\fi
%    \end{macrocode}

% End of document body:
%    \begin{macrocode}
\end{document}
%    \end{macrocode}
%\iffalse
%</samplemain>
%\fi
%
% %%%%%%%%%%%%%%%%%%%%%%%%%%%%%%%%%%%%%%
% \paragraph{Chapter Include Files.}
%
% The include files are called |cdocsch1.tex| and |cdocsch2.tex|.
%
%\iffalse
%<*samplechap1|samplechap2>
%\fi

% Optional override for |\version| flag:
%    \begin{macrocode}
%%\providecommand{\version}{final}
%    \end{macrocode}

% Include the main document:
%    \begin{macrocode}
\input{childdoc.def}
\childdocof{cdocsamp}
%    \end{macrocode}

%\iffalse
%</samplechap1|samplechap2>
%\fi
%
%\iffalse
%<*samplechap1>
%\fi
% Some text for chapter 1:
%    \begin{macrocode}
\section{one}
some text in chapter one
%    \end{macrocode}

%\iffalse
%</samplechap1>
%\fi
% Some text for chapter 2:
%\iffalse
%<*samplechap2>
%\fi
%    \begin{macrocode}
\section{two}
more text in chapter two
%    \end{macrocode}

%\iffalse
%</samplechap2>
%\fi
%
% %%%%%%%%%%%%%%%%%%%%%%%%%%%%%%%%%%%%%%
% \paragraph{Part Include Files.}
%
% The include files are called |cdocspt3.tex| and |cdocspt4.tex|.
%
%\iffalse
%<*samplepart3|samplepart4>
%\fi

% Optional override for |\version| flag:
%    \begin{macrocode}
%%\providecommand{\version}{final}
%    \end{macrocode}

% Include the main document:
%    \begin{macrocode}
\input{childdoc.def}
\childdocby{cdocsamp}
%    \end{macrocode}

%\iffalse
%</samplepart3|samplepart4>
%\fi
%
%\iffalse
%<*samplepart3>
%\fi
% Some text for part 3:
%    \begin{macrocode}
some text in part three
%    \end{macrocode}

%\iffalse
%</samplepart3>
%\fi
% Some text for part 4:
%\iffalse
%<*samplepart4>
%\fi
%    \begin{macrocode}
more text in part four
%    \end{macrocode}

%\iffalse
%</samplepart4>
%\fi
%
% %%%%%%%%%%%%%%%%%%%%%%%%%%%%%%%%%%%%%%
% \paragraph{Forwarding for a Complete Draft.}
%
% The following forwarding file |cdocsdrf.tex|
% compiles the main document in draft mode:
%\iffalse
%<*sampledraft>
%\fi
%    \begin{macrocode}
\def\version{draft}
\input{childdoc.def}
\childdocforward{cdocsamp}
%    \end{macrocode}

%\iffalse
%</sampledraft>
%\fi
%
% %%%%%%%%%%%%%%%%%%%%%%%%%%%%%%%%%%%%%%
% \paragraph{Forwarding for Final Version of the Chapters.}
%
% The following forwarding files |cdocsfn1.tex| and |cdocsfn2.tex|
% (with identical content)
% compile the final versions of the child documents
% |cdocsch1.tex| and |cdocsch2.tex|, respectively:
%\iffalse
%<*samplefinal>
%\fi
%    \begin{macrocode}
\def\version{final}
\input{childdoc.def}
\childdocforwardprefix[cdocsamp]{cdocsfn}{cdocsch}
%    \end{macrocode}

%\iffalse
%</samplefinal>
%\fi
%
% %%%%%%%%%%%%%%%%%%%%%%%%%%%%%%%%%%%%%%
% \paragraph{Command Line Processing.}
%
% The following three command lines generate the output files
% |cdocscld|, |cdocscl1| and |cdocscl2|
% which should be identical to
% |cdocsdrf|, |cdocsch1| and |cdocsfn2|, respectively:
% \begin{center}
% \begin{tabular}{l}
% |latex -jobname cdocscld \|\\
% |  "\def\version{draft}\input{childdoc.def}\childdocforward{cdocsamp}"|\\
% |latex -jobname cdocscl1 \|\\
% |  "\input{childdoc.def}\childdocforward[cdocsamp]{cdocsch1}"|\\
% |latex -jobname cdocscl2 \|\\
% |  "\def\version{final}\input{childdoc.def}\childdocforward{cdocsch2}"|
% \end{tabular}
% \end{center}
% Note that the trailing backslash on each first line
% merely continues the input to the second line
% (for convenient cut ant paste).
% Furthermore, the command |latex| can be replaced by any
% of its alternative versions such as |pdflatex|.
%
% %%%%%%%%%%%%%%%%%%%%%%%%%%%%%%%%%%%%%%%%%%%%%%%%%%%%%%%%%%%%%%%%%%%%%%%%%%%%%%
% %%%%%%%%%%%%%%%%%%%%%%%%%%%%%%%%%%%%%%%%%%%%%%%%%%%%%%%%%%%%%%%%%%%%%%%%%%%%%%
% \section{Implementation}
%\iffalse
%<*package>
%\fi
%
% This section describes the definitions file |childdoc.def|.

% The definitions cannot be loaded using |\usepackage| or |\RequirePackage|
% which has a mechanism to prevent loading a style file more than once.
% When loading the definitions by means of |\input|
% multiple instances have to be prevented manually:
%\iffalse
%This code needs to be before the `\ProvidesFile' directive
%which is defined at the beginning of this file.
%Therefore it is also placed there and commented out here.
%</package>
%<*discard>
%\fi
%    \begin{macrocode}
\ifdefined\childdocmain\endinput\fi
%    \end{macrocode}
%\iffalse
%</discard>
%<*package>
%\fi
%
% \macro{\ifchilddoc}
% \macro{\ifchilddocmanual}
% The conditional |\ifchilddoc| tells whether a
% child (true) or main (false) document is being compiled.
% The conditional |\ifchilddocmanual| tells whether
% the |\includeonly| mechanism is used (false) or
% the selection of child files must be performed manually (true).
% The definitions initialise to false:
%    \begin{macrocode}
\newif\ifchilddoc
\newif\ifchilddocmanual
%    \end{macrocode}

% \macro{\childdocname}
% \macro{\childdocjob}
% The macro |\childdocname| stores the name of the main document
% to be compiled. The macro |\childdocjob| stores the name of
% the document on which the \LaTeX{} compiler was originally invoked.
% The content of |\jobname| cannot be compared
% to filenames specified in the source due to different catcodes.
% The following code rescans |\jobname|, stores the result
% in |\childdocname| and saves a copy in |\childdocjob|:
%    \begin{macrocode}
\edef\childdocname{\scantokens\expandafter{\jobname\noexpand}}
\let\childdocjob\childdocname
%    \end{macrocode}

% \macro{\childdocdisable}
% The macro |\childdocdisable| prevents the main file
% from being processed more than once.
% At this stage, the main document command |\childdocmain|
% is assumed to be called once again where it should do nothing.
% Any subsequent call to it should prevent
% a secondary processing of the main document
% It overwrites the forwarding commands
% |\childdocof| and |\childdocforward|
% with empty macros to prevent further inclusions of the main document:
%    \begin{macrocode}
\newcommand{\childdocdisable}
{
  \renewcommand{\childdocmain}[1]{\renewcommand{\childdocmain}[1]{\endinput}}
  \renewcommand{\childdocof}[1]{}
  \renewcommand{\childdocby}[2][]{}
  \renewcommand{\childdocforward}[2][]{}
  \renewcommand{\childdocdisable}{}
}
%    \end{macrocode}

% \macro{\childdocmain}
% The macro |\childdocmain| is to be called at the top of the main file
% with nothing or the main filename (without extension) as argument.
% First, it breaks loops.
% If the argument is not empty and does not match |\childdocname|
% (which is set by the first inclusion of |childdoc.def|),
% |\ifchilddoc| is set to true, |\includeonly| is applied to the child file
% and |\jobname| is set to the main file
% (for proper handling of |.aux| files):
%    \begin{macrocode}
\newcommand{\childdocmain}[1]
{
  \childdocdisable\childdocmain{}
  \if?#1?\else
    \begingroup
      \def\childdoctmp{#1}
      \ifx\childdoctmp\childdocname
        \def\childdoctmp{}
      \else
        \def\childdoctmp
        {
          \childdoctrue
          \includeonly{\childdocname}
          \def\childdocjob{#1}
          \def\jobname{#1}
        }
      \fi
      \expandafter
    \endgroup
    \childdoctmp
  \fi
}
%    \end{macrocode}

% \macro{\childdocof}
% The command |\childdocof| redirects
% compilation to the main file |#1|.
%    \begin{macrocode}
\newcommand{\childdocof}[1]
{
  \childdocdisable
  \childdoctrue
  \includeonly{\childdocname}
  \def\jobname{#1}
  \def\childdocjob{#1}
  \input{#1}
}
%    \end{macrocode}

% \macro{\childdocby}
% The command |\childdocby| ....
%    \begin{macrocode}
\newcommand{\childdocby}[2][]
{
  \childdocdisable
  \childdoctrue
  \childdocmanualtrue
  \if?#1?\else
    \def\jobname{#2}
  \fi
  \def\childdocjob{#2}
  \input{#2}
  \endinput
}
%    \end{macrocode}

% \macro{\childdocforward}
% The command |\childdocforward| redirects
% compilation to the main file or
% (if the optional argument is given) a child file.
% Parameters are set as if the main file
% or a child file starting with |\childdocof| was compiled.
% Then compilation is handed over to the main file:
%    \begin{macrocode}
\newcommand{\childdocforward}[2][]
{
  \begingroup
    \if?#1?
      \def\childdoctmp
      {
        \def\childdocname{#2}
        \def\childdocjob{#2}
        \def\jobname{#2}
        \input{#2}
        \endinput
      }
    \else
      \def\childdoctmp
      {
        \childdocdisable
        \def\childdocname{#2}
        \childdoctrue
        \includeonly{#2}
        \def\childdocjob{#1}
        \def\jobname{#1}
        \input{#1}
        \endinput
      }
    \fi
    \expandafter
  \endgroup
  \childdoctmp
}
%    \end{macrocode}

% \macro{\childdocforwardprefix}
% The command |\childdocforwardprefix| redirects
% compilation to the main or a child file by means of a pattern.
% The prefix |#1| in the current filename is replaced by |#2|
% and the suffix of the current filename is kept
% (it is assumed that the filename does not contain the substring `|~~~|'
% which is used as a delimiter).
% Compilation is handed over to the new file by |\childdocforward|:
%    \begin{macrocode}
\newcommand{\childdocforwardprefix}[3][]
{
  \begingroup
    \def\childdocextract #2##1~~~{\def\childdoctmp{\childdocforward[#1]{#3##1}}}
    \expandafter\childdocextract\childdocname~~~
    \expandafter
  \endgroup
  \childdoctmp
}
%    \end{macrocode}

% \macro{\childdoc}
% The deprecated macro |\childdoc| is a legacy version of |\childdocmain|:
%    \begin{macrocode}
\newcommand{\childdoc}{\childdocmain}
%    \end{macrocode}

% \macro{\childdocredirect}
% The deprecated macro |\childdocredirect| is a legacy version
% of |\childdocforward| and |\childdocforwardprefix|:
%    \begin{macrocode}
\newcommand{\childdocredirect}[2][]
{
  \begingroup
    \if?#1?
      \def\childdoctmp{\childdocforward{#2}}
    \else
      \def\childdoctmp{\childdocforwardprefix{#1}{#2}}
    \fi
    \expandafter
  \endgroup
  \childdoctmp
}
%    \end{macrocode}

%\iffalse
%</package>
%\fi
%
\endinput
|\\
|\childdocforwardprefix{final}{child}|
\end{tabular}
\end{center}
%

Note that when several versions of a main file and/or of each child file
are to be generated, it may be convenient to set up a |Makefile| or
shell script to automatise the process.

%%%%%%%%%%%%%%%%%%%%%%%%%%%%%%%%%%%%%%%%%%%%%%%%%%%%%%%%%%%%%%%%%%%%%%%%%%%%%%%%
\subsection{Command Line Processing}
\label{sec:commandline}

The effect of redirection files can also be achieved by invoking
the \LaTeX{} compiler with a more elaborate command line.
Most conveniently this should be done as part
of a shell script or a |Makefile|.

When using \textsf{childdoc} in the main file, the following
command lines effectively perform a redirection
(note that depending on the shell being used,
backslashes may have to be doubled: `|\|' $\to$ `|\\|'):
%
\begin{center}
|... -jobname "|\textit{target}|" |\\|"|[\textit{flags}]%
|% \iffalse
%
% childdoc.dtx Copyright (C) 2017-2018 Niklas Beisert
%
% This work may be distributed and/or modified under the
% conditions of the LaTeX Project Public License, either version 1.3
% of this license or (at your option) any later version.
% The latest version of this license is in
%   http://www.latex-project.org/lppl.txt
% and version 1.3 or later is part of all distributions of LaTeX
% version 2005/12/01 or later.
%
% This work has the LPPL maintenance status `maintained'.
%
% The Current Maintainer of this work is Niklas Beisert.
%
% This work consists of the files childdoc.dtx and childdoc.ins
% and the derived files childdoc.def and cdocsamp.tex with
% cdocsch1.tex, cdocsch2.tex, cdocsdrf.tex, cdocsfn1.tex, cdocsfn2.tex.
%
%<package>\ifdefined\childdocmain\endinput\fi
%<package>\ProvidesFile{childdoc.def}[2018/12/30 v2.0 child document driver]
%<samplemain>\ProvidesFile{cdocsamp.tex}[2018/12/30 v2.0 sample for childdoc]
%<*driver>
%\ProvidesFile{childdoc.drv}[2018/12/30 v2.0 childdoc reference manual file]
\PassOptionsToClass{10pt,a4paper}{article}
\documentclass{ltxdoc}

\usepackage[margin=35mm]{geometry}
\usepackage{hyperref}
\usepackage{hyperxmp}
\usepackage[usenames]{color}

\hypersetup{colorlinks=true}
\hypersetup{pdfstartview=FitH}
\hypersetup{pdfpagemode=UseNone}
\hypersetup{pdfsource={}}
\hypersetup{pdflang={en-UK}}
\hypersetup{pdfcopyright={Copyright 2017-2018 Niklas Beisert.
  This work may be distributed and/or modified under the
  conditions of the LaTeX Project Public License, either version 1.3
  of this license or (at your option) any later version.}}
\hypersetup{pdflicenseurl={http://www.latex-project.org/lppl.txt}}
\hypersetup{pdfcontactaddress={ETH Zurich, ITP, HIT K,
  Wolfgang-Pauli-Strasse 27}}
\hypersetup{pdfcontactpostcode={8093}}
\hypersetup{pdfcontactcity={Zurich}}
\hypersetup{pdfcontactcountry={Switzerland}}
\hypersetup{pdfcontactemail={nbeisert@itp.phys.ethz.ch}}
\hypersetup{pdfcontacturl={http://people.phys.ethz.ch/\xmptilde nbeisert/}}

\newcommand{\secref}[1]{\hyperref[#1]{section \ref*{#1}}}

\parskip1ex
\parindent0pt
\let\olditemize\itemize
\def\itemize{\olditemize\parskip0pt}

\begin{document}

\title{The \textsf{childdoc} Package}
\hypersetup{pdftitle={The childdoc Package}}
\author{Niklas Beisert\\[2ex]
  Institut f\"ur Theoretische Physik\\
  Eidgen\"ossische Technische Hochschule Z\"urich\\
  Wolfgang-Pauli-Strasse 27, 8093 Z\"urich, Switzerland\\[1ex]
  \href{mailto:nbeisert@itp.phys.ethz.ch}
  {\texttt{nbeisert@itp.phys.ethz.ch}}}
\hypersetup{pdfauthor={Niklas Beisert}}
\hypersetup{pdfsubject={Manual for the LaTeX2e Package childdoc}}
\date{30 December 2018, \textsf{v2.0}}
\maketitle

\begin{abstract}\noindent
\textsf{childdoc} is a \LaTeXe{} package
that enables the direct compilation
of document sections included by |\include|
to individual files.
\end{abstract}

\begingroup
\parskip0ex
\tableofcontents
\endgroup

%%%%%%%%%%%%%%%%%%%%%%%%%%%%%%%%%%%%%%%%%%%%%%%%%%%%%%%%%%%%%%%%%%%%%%%%%%%%%%%%
%%%%%%%%%%%%%%%%%%%%%%%%%%%%%%%%%%%%%%%%%%%%%%%%%%%%%%%%%%%%%%%%%%%%%%%%%%%%%%%%
\section{Introduction}

\LaTeX{} provides a mechanism to structure a large document (such as a book)
into a main file and several child files (containing the chapters)
using the |\include| command.
This mechanism is beneficial for documents
which span hundreds of pages in order to
make the source file(s) more manageable.
Moreover, compilation can be restricted to
selected child files by means of the |\includeonly| command.
The latter feature can be used to reduce the compilation time while editing
(this was significantly more useful in the earlier days of \LaTeX{})
or to generate a smaller document which is easier to navigate.
Another application of |\includeonly| is to generate
documents consisting of selected parts of the complete document.

However, there are a few drawbacks of the plain |\include| mechanism:
\begin{itemize}
\item
The child files cannot be compiled on their own,
they can only be compiled via the main file.
A naive editing environment
(such as a text editor with an option
to have the current file processed by \LaTeX)
may require one to switch to the main file before compiling;
attempting to compile the child file produces errors.
\item
The main file must be modified (each time)
to adjust the |\includeonly| command
to the present needs. This easily leaves the main file in a messy state.
\item
The generated document will always carry the filename
of the main document. This is inconvenient if
several child files are to be compiled and
to be kept for distribution.
\end{itemize}

The present package provides a simple interface
to make child files individually compilable by \LaTeX{}.
Compiling a child file then has the same effect as compiling
the main file with an |\includeonly| command
to select the appropriate child.
Moreover the generated document will carry the name of the child
rather than the main file.
This resolves all three above issues.

This feature is meant to make the editing of books,
thesis documents and lecture notes somewhat more convenient.
However, the package can also be used efficiently for
composing a series of documents (such as exercise sheets)
which are typically distributed individually.
It then assists the author in generating the individual documents
(potentially in different versions)
as well as a document containing the collected series.
Another application is in developing style files
or other kinds of included material
where compilation of the style file could redirect
to a sample or test file.

%%%%%%%%%%%%%%%%%%%%%%%%%%%%%%%%%%%%%%%%%%%%%%%%%%%%%%%%%%%%%%%%%%%%%%%%%%%%%%%%
%%%%%%%%%%%%%%%%%%%%%%%%%%%%%%%%%%%%%%%%%%%%%%%%%%%%%%%%%%%%%%%%%%%%%%%%%%%%%%%%
\section{Usage}

First of all, the package \textsf{childdoc} is \emph{not} a standard
\LaTeXe{} |.sty| style file! Therefore it needs to be invoked in
a non-standard way.

%%%%%%%%%%%%%%%%%%%%%%%%%%%%%%%%%%%%%%%%%%%%%%%%%%%%%%%%%%%%%%%%%%%%%%%%%%%%%%%%
\subsection{Included Files}
\label{sec:include}

%%%%%%%%%%%%%%%%%%%%%%%%%%%%%%%%%%%%%%%%
\DescribeMacro{\childdocmain}
To use the package, add the commands
\begin{center}
\begin{tabular}{l}
|\input{childdoc.def}|\\
|\childdocmain{}|\\
\end{tabular}
\end{center}
at the very top of the main \LaTeX{} file,
in particular \emph{before} the |\documentclass| statement!
The argument of |\childdocmain| should be left empty
(but it must be present).

%%%%%%%%%%%%%%%%%%%%%%%%%%%%%%%%%%%%%%%%
\DescribeMacro{\childdocof}
Furthermore, add the commands
\begin{center}
\begin{tabular}{l}
|\input{childdoc.def}|\\
|\childdocof{|\textit{main}|}|\\
\end{tabular}
\end{center}
at the top of every child file \textit{child}
which is included by |\include{|\textit{child}|}|
from within the main file
(or at least for those files to be compiled individually).
The argument \textit{main} must be the filename of the main file.

There are a couple of
considerations in setting up the main and child documents:

%%%%%%%%%%%%%%%%%%%%%%%%%%%%%%%%%%%%%%%%
\paragraph{Restrictions.}

Please note the following restrictions:
\begin{itemize}
\item
|\childdocmain| must be called with one argument \textit{main}
to ensure compatibility with earlier version of the package.
It must either be empty (|\childdocmain{}|)
or precisely match the filename of the main file in which it is specified.
See \secref{sec:detection} for further information.
\item
The filename \textit{main} must be specified without the |.tex| extension.
\item
The filename \textit{main} is case sensitive
(even in case-insensitive file systems)
due to internal string comparison.
\item
The argument \textit{main} should be fully expanded, it cannot be a macro.
\item
Subdirectories and special characters should be avoided in filenames.
\item
The command |\childdocmain{|\textit{main}|}| must be followed by a whitespace.
It should not be followed immediately by another command
or by a comment mark `|%|'.
This is because the \TeX{} parser reads the token immediately following
the argument of |\childdocmain| and puts it
at the beginning of every child section;
however, a white\-space is ignored.
\end{itemize}

%%%%%%%%%%%%%%%%%%%%%%%%%%%%%%%%%%%%%%%%
\paragraph{Content of Main File.}

It is advisable to place all content in the child files included by |\include|.
Any output contained in the main file will appear in all child documents
unless suppressed manually;
it cannot be suppressed automatically by the |\includeonly| directive
and thus should normally be avoided.
A method to include some content in the main file
by means of conditional processing is described in \secref{sec:conditional}.

%%%%%%%%%%%%%%%%%%%%%%%%%%%%%%%%%%%%%%%%
\paragraph{Page Numbering.}

When only a part of the document is compiled,
the appropriate numbering of pages
(as well as other status parameters)
is determined from the |.aux| files.
The latter contain information from previous passes.
However this information needs to propagate through
all intermediate child documents.
Therefore the page numbering in child documents may well
be inconsistent until the complete document is compiled at least once.

A useful (if unconventional) way to always ensure a consistent
page numbering is to restart the numbering in each child document
and denote the pages by `\textit{child}|.|\textit{page}'
where \textit{child} represents the chapter/section number of the child file.
This can be achieved by the command
|\numberwithin{page}{|\textit{child}|}|
of the \textsf{amsmath} package
where \textit{child} can be |chapter| or |section|
depending on the chosen structuring.
Alternatively, one can modify the macro |\thepage| appropriately
and reset the counter |page| at the start of each child file.

%%%%%%%%%%%%%%%%%%%%%%%%%%%%%%%%%%%%%%%%%%%%%%%%%%%%%%%%%%%%%%%%%%%%%%%%%%%%%%%%
\subsection{Conditional Processing}
\label{sec:conditional}

The package provides a mechanism to compile different versions
of a document. To customise the versions further some conditional processing
can come in handy to distinguish which version is being compiled.
The package provides two macros to describe the compilation context:

%%%%%%%%%%%%%%%%%%%%%%%%%%%%%%%%%%%%%%%%
\DescribeMacro{\ifchilddoc}
The conditional |\ifchilddoc| distinguishes between the compilation of
child documents and the main document:
%
\begin{center}
|\ifchilddoc |\textit{child-code}| |[|\||else |\textit{main-code}]| \||fi|
\end{center}

%%%%%%%%%%%%%%%%%%%%%%%%%%%%%%%%%%%%%%%%
\DescribeMacro{\childdocname}
\DescribeMacro{\childdocjob}
The macro |\childdocname| contains the filename (without extension)
of the main or child file being processed.
Note that |\childdocjob| will always contain the name of the main file.

%%%%%%%%%%%%%%%%%%%%%%%%%%%%%%%%%%%%%%%%
\paragraph{Title Page.}

Conditional processing can be used to include a title or banner page
in the main document when proper precautions are taken.
Importantly, the code in the main file should ensure that the page counter
(as well as other status parameters which are stored in the |.aux| files)
takes the same value after the conditional processing.
Otherwise the page numbers may take divergent values
depending on which part is compiled.

For example, a title page could be declared by:
%
\begin{center}
\begin{tabular}{l}
|\ifchilddoc\||else|\\
|\addtocounter{page}{-1}|\\
\textit{code for title page}\\
|\newpage|\\
|\||fi|
\end{tabular}
\end{center}
%
A banner page for the child documents can be generated by:
%
\begin{center}
\begin{tabular}{l}
|\ifchilddoc|\\
|\addtocounter{page}{-1}|\\
\textit{code for banner page}\\
|\newpage|\\
|\||fi|
\end{tabular}
\end{center}
%
Here one could write a message such as:
\begin{center}
|This is the part \childdocname{} of \childdocjob{}.|
\end{center}

%%%%%%%%%%%%%%%%%%%%%%%%%%%%%%%%%%%%%%%%%%%%%%%%%%%%%%%%%%%%%%%%%%%%%%%%%%%%%%%%
\subsection{Flags}
\label{sec:flags}

The package makes it easy to generate different versions
of the main or child documents.
To this end compilation flags can be defined
and assigned different default values.
They will be particularly useful in conjunction
with the forwarding mechanism described in \secref{sec:forward}.

For example, it may be useful to have a flag |\version|
which can be set to |draft| or |final|.
The document source will contain some conditional code
depending on the value of |\version|.
Suppose further, the flag should default to |final| for the main file
and to |draft| for child files
which is a natural assignment for editing the document.
This is achieved by placing the following code
in the preamble of the main document
(below the |\childdocmain| directive):
%
\begin{center}
\begin{tabular}{l}
|\ifchilddoc|\\
|\providecommand{\version}{draft}|\\
|\||else|\\
|\providecommand{\version}{final}|\\
|\||fi|
\end{tabular}
\end{center}
%
The definition by |\providecommand| makes sure
that previous definitions are not overwritten.
Further statements |\providecommand{\version}{...}|
can thus be added before the above code to override it.

For the main file, one might add a line
(between |\childdocmain| and the above block)
%
\begin{center}
|%\ifchilddoc\||else\providecommand{\version}{draft}\||fi|
\end{center}
%
which can be uncommented to produce a draft version.
Likewise one can add a line to the very top of a child file
(above the |\childdocof{|\textit{main}|}| directive)
%
\begin{center}
|%\providecommand{\version}{final}|
\end{center}
%
which can be uncommented to produce the final version of this child document.

%%%%%%%%%%%%%%%%%%%%%%%%%%%%%%%%%%%%%%%%%%%%%%%%%%%%%%%%%%%%%%%%%%%%%%%%%%%%%%%%
\subsection{Forwarding}
\label{sec:forward}

Different versions of the main or child documents
using compilation flags as described in \secref{sec:flags}
can be (permanently) stored in different files
for convenient compilation, viewing and distribution.
To this end, the package defines a command
to pass on compilation to a different file:

%%%%%%%%%%%%%%%%%%%%%%%%%%%%%%%%%%%%%%%%
\DescribeMacro{\childdocforward}
The command |\childdocforward| redirects processing to
another source file:
%
\begin{center}
\begin{tabular}{l}
|\input{childdoc.def}|\\
|\childdocforward[|\textit{main}|]{|\textit{dest}|}|\\
\end{tabular}
\end{center}
%
The argument \textit{dest} is the destination file
(without extension).
It should be the main file or one of the child files.
Note that further \textsf{childdoc} directives
such as |\childdocof| and |\childdocforward|
in the indicated file will be processed in this form.
The optional argument \textit{main}
passes on directly to the main file \textit{main}
while pretending to compile the child \textit{dest}.
This form behaves as if \textit{dest}
issues |\childdocof{|\textit{main}|}| right away,
and no further \textsf{childdoc} directives will be processed.

%%%%%%%%%%%%%%%%%%%%%%%%%%%%%%%%%%%%%%%%
\DescribeMacro{\...prefix}
In the alternative form |\childdocforwardprefix|,
%
\begin{center}
\begin{tabular}{l}
|\input{childdoc.def}|\\
|\childdocforwardprefix[|\textit{main}|]{|\textit{prefix}|}{|\textit{dest}|}|
\end{tabular}
\end{center}
%
the destination file is determined by a pattern
depending on the current file:
To make this work, the current file must be called
`{\textit{prefix}\hspace{0.2em}\textit{suffix}}'
with \textit{prefix} matching precisely the argument.
Processing is then passed on to the file
`{\textit{dest}\hspace{0.2em}\textit{suffix}}'.
Surely, the same effect is achieved by
directly specifying the
argument `{\textit{dest}\hspace{0.2em}\textit{suffix}}'
in the first form.
However, that requires to set up a different file
for each child. With the alternative form of the command
all these files can have exactly the same content
which simplifies setting them up and maintaining them.

For example, the following file |draft.tex|
with a compilation flag |\version| as described in \secref{sec:flags}
compiles the main document as a draft:
%
\begin{center}
\begin{tabular}{l}
|\def\version{draft}|\\
|\input{childdoc.def}|\\
|\childdocforward{|\textit{main}|}|
\end{tabular}
\end{center}
%
Likewise, the following files |final|\textit{nn}|.tex|
compile the final version of the child document
|child|\textit{nn}|.tex|:
%
\begin{center}
\begin{tabular}{l}
|\def\version{final}|\\
|\input{childdoc.def}|\\
|\childdocforwardprefix{final}{child}|
\end{tabular}
\end{center}
%

Note that when several versions of a main file and/or of each child file
are to be generated, it may be convenient to set up a |Makefile| or
shell script to automatise the process.

%%%%%%%%%%%%%%%%%%%%%%%%%%%%%%%%%%%%%%%%%%%%%%%%%%%%%%%%%%%%%%%%%%%%%%%%%%%%%%%%
\subsection{Command Line Processing}
\label{sec:commandline}

The effect of redirection files can also be achieved by invoking
the \LaTeX{} compiler with a more elaborate command line.
Most conveniently this should be done as part
of a shell script or a |Makefile|.

When using \textsf{childdoc} in the main file, the following
command lines effectively perform a redirection
(note that depending on the shell being used,
backslashes may have to be doubled: `|\|' $\to$ `|\\|'):
%
\begin{center}
|... -jobname "|\textit{target}|" |\\|"|[\textit{flags}]%
|\input{childdoc.def}\childdocforward[|\textit{main}|]{|\textit{dest}|}"|
\end{center}
%
Here \textit{target} is the name of the output file,
\textit{main} is the name of the main file
and \textit{dest} is the name of the main or child file to be processed
(all filenames without extensions).
The optional argument \textit{main} can be omitted
if \textit{main} matches \textit{dest}.
Optionally, compilation \textit{flags} can be defined via |\def| commands.
This command line makes the \TeX{} engine believe
it is compiling the file \textit{target}
whose content is specified as the latter parameter.
The provided code then forwards the processing to
\textit{main} or \textit{dest} as described in \secref{sec:forward}.

%%%%%%%%%%%%%%%%%%%%%%%%%%%%%%%%%%%%%%%%%%%%%%%%%%%%%%%%%%%%%%%%%%%%%%%%%%%%%%%%
\subsection{Include by Input}
\label{sec:input}

Including child documents by |\include| has some restrictions by design.
Most notably, the content of a child document always occupies
its own set of pages; pages cannot be shared between child documents.
Usually, this behaviour makes perfect sense
because each child document contain an essential part of the document.
However, in some situations it may be desirable to compose
a document from a collection of parts
without having mandatory page breaks between then.
For this case, the package
provides a mechanism to include parts
by |\input| which can also be processed individually.
However, by construction this mechanism
requires manual handling of the content to be output.

%%%%%%%%%%%%%%%%%%%%%%%%%%%%%%%%%%%%%%%%
\DescribeMacro{\ifchilddocmanual}
The main file should be prepared as usual, see \secref{sec:include}.
However, the document body must make a distinction
between processing of an individual part and of the main document, e.g.:
%
\begin{center}
\begin{tabular}{l}
|\ifchilddocmanual|\\
|\input{\childdocname}|\\
|\||else|\\
\textit{document body with }|\input{|\textit{part}|}|\\
|\||fi|
\end{tabular}
\end{center}
%
The conditional |\ifchilddocmanual| is true whenever
a part to be included by |\input| is being compiled,
and the name of the part is stored in |\childdocname|.

%%%%%%%%%%%%%%%%%%%%%%%%%%%%%%%%%%%%%%%%
\DescribeMacro{\childdocby}
Each part to be included by |\input| should start with:
%
\begin{center}
\begin{tabular}{l}
|\input{childdoc.def}|\\
|\childdocby{|\textit{main}|}|\\
\end{tabular}
\end{center}
%
The directive |\childdocby| is similar to |\childdocof|
described in \secref{sec:include},
but the subsequent selection of content must be done manually.
To that end, both |\ifchilddoc| and |\ifchilddocmanual|
will be true upon processing of a part,
and the name of the part is stored in |\childdocname|.
Note that |\jobname| will be set to the filename of the current part
so that each part receives an individual |.aux| file
that does not interfere with the |.aux| file(s) of the main document.
This behaviour can be altered by the alternative form
|\childdocby[*]{|\textit{main}|}| (with a non-empty optional argument)
which uses the |.aux| file of the main document
by setting |\jobname| to \textit{main}.

%%%%%%%%%%%%%%%%%%%%%%%%%%%%%%%%%%%%%%%%%%%%%%%%%%%%%%%%%%%%%%%%%%%%%%%%%%%%%%%%
\subsection{Driver Development}
\label{sec:driver}

The \textsf{childdoc} mechanism can also be use for the development
of definition files such as \LaTeX{} styles or classes.
This case differs from the above setup with multiple parts
included by |\include| in that no |\includeonly| should be invoked.
This can be achieved by starting the include file
(before |\ProvidesPackage|) with:
%
\begin{center}
\begin{tabular}{l}
|\input{childdoc.def}|\\
|\childdocforward{|\textit{main}|}|\\
\end{tabular}
\end{center}
%
or alternatively with:
%
\begin{center}
\begin{tabular}{l}
|\input{childdoc.def}|\\
|\childdocby{|\textit{main}|}|\\
\end{tabular}
\end{center}
%
Both forms have slightly different effects as described above.
The main file is prepared as usual, see \secref{sec:include}.

%%%%%%%%%%%%%%%%%%%%%%%%%%%%%%%%%%%%%%%%%%%%%%%%%%%%%%%%%%%%%%%%%%%%%%%%%%%%%%%%
\subsection{Legacy Detection}
\label{sec:detection}

The directive |\childdocmain| in the main file can detect
whether the complete document or merely a child is to be compiled
even without using the directive |\childdocof|.
This method is deprecated because it is less robust
and there is no compelling reason to use it;
it is merely provided for backward compatibility
and it may be removed in future versions.

If the detection mechanism is to be used,
it is mandatory to correctly specify
the filename of the main file as the argument of |\childdocmain|:
%
\begin{center}
\begin{tabular}{l}
|\input{childdoc.def}|\\
|\childdocmain{|\textit{main}|}|\\
\end{tabular}
\end{center}
%
If |\jobname| does not match the argument \textit{main} of |\childdocmain|,
it is assumed that |\jobname| points to the child file to be compiled.
When using |\childdocmain| with the main file specified as argument,
it suffices to start a child file
with just |\input{|\textit{main}|}|
without loading of the package and using |\childdocof|.
If instead all processing is done
with the appropriate \textsf{childdoc} directives,
the argument of \textit{main} of |\childdocmain| can be empty.

An alternative version of the command line processing described
in \secref{sec:commandline} using the detection mechanism reads:
%
\begin{center}
|... -jobname "|\textit{target}|" "|[\textit{flags}]%
[|\def\jobname{|\textit{dest}|}|]|\input{|\textit{main}|}"|
\end{center}

%%%%%%%%%%%%%%%%%%%%%%%%%%%%%%%%%%%%%%%%%%%%%%%%%%%%%%%%%%%%%%%%%%%%%%%%%%%%%%%%
\subsection{Manual Code}
\label{sec:manual}

In case one cannot be certain whether the definitions file |childdoc.def|
is installed on the target \TeX{} distribution
and one prefers not to ship it,
it is conceivable to paste a few relevant commands into the sources.

To that end, drop all statements |\input{childdoc.def}|
and perform the replacements as outlined below.
Instead of |\childdocmain{|\textit{main}|}| add the following code
to the top of the main file:
%
\begin{center}
\begin{tabular}{l}
|\||ifdefined\childdocname\endinput\||fi\newif\ifchilddoc|\\
|\edef\childdocname{\scantokens\expandafter{\jobname\noexpand}}|\\
|\def\childdocmain{|\textit{main}|}\||ifx\childdocmain\childdocname\||else|\\
|\childdoctrue\includeonly{\childdocname}\let\jobname\childdocmain\||fi|\\
\end{tabular}
\end{center}
%
Instead of |\childdocof{|\textit{main}|}| just include the main file
at the top of each child file:
%
\begin{center}
|\input{|\textit{main}|}|
\end{center}
%
A simple redirection |\childdocforward{|\textit{dest}|}| is achieved by:
%
\begin{center}
|\def\jobname{|\textit{dest}|}\input{\jobname}|
\end{center}
%
The redirection with prefix
|\childdocforwardprefix[|\textit{prefix}|]{|\textit{dest}|}|
is accomplished by:
%
\begin{center}
\begin{tabular}{l}
|{\edef\jobname{\scantokens\expandafter{\jobname\noexpand}}|\\
|\def\redirectjob |\textit{prefix}|#1~~~{\gdef\jobname{|\textit{dest}|#1}}|\\
|\expandafter\redirectjob\jobname~~~}\input{\jobname}|
\end{tabular}
\end{center}

In an alternative approach,
child documents can be compiled by a specific command line
without additional code or specific definitions:
%
\begin{center}
|... -jobname "|\textit{target}|" "|[\textit{flags}]%
|\includeonly{|\textit{dest}|}\input{|\textit{main}|}"|
\end{center}
%

%%%%%%%%%%%%%%%%%%%%%%%%%%%%%%%%%%%%%%%%%%%%%%%%%%%%%%%%%%%%%%%%%%%%%%%%%%%%%%%%
%%%%%%%%%%%%%%%%%%%%%%%%%%%%%%%%%%%%%%%%%%%%%%%%%%%%%%%%%%%%%%%%%%%%%%%%%%%%%%%%
\section{Information}

%%%%%%%%%%%%%%%%%%%%%%%%%%%%%%%%%%%%%%%%%%%%%%%%%%%%%%%%%%%%%%%%%%%%%%%%%%%%%%%%
\subsection{Copyright}

Copyright \copyright{} 2017--2018 Niklas Beisert

This work may be distributed and/or modified under the
conditions of the \LaTeX{} Project Public License, either version 1.3
of this license or (at your option) any later version.
The latest version of this license is in
  \url{http://www.latex-project.org/lppl.txt}
and version 1.3 or later is part of all distributions of \LaTeX{}
version 2005/12/01 or later.

This work has the LPPL maintenance status `maintained'.

The Current Maintainer of this work is Niklas Beisert.

This work consists of the files |README.txt|, |childdoc.ins| and |childdoc.dtx|
as well as the derived files |childdoc.def|, |cdocsamp.tex|
with |cdocsch1.tex|, |cdocsch2.tex|, |cdocspt3.tex|, |cdocspt4.tex|,
|cdocsdrf.tex|, |cdocsfn1.tex|, |cdocsfn2.tex|
as well as |childdoc.pdf|.

%%%%%%%%%%%%%%%%%%%%%%%%%%%%%%%%%%%%%%%%%%%%%%%%%%%%%%%%%%%%%%%%%%%%%%%%%%%%%%%%
\subsection{Files and Installation}

The package consists of the files:
%
\begin{center}
\begin{tabular}{ll}
    |README.txt|   & readme file \\
    |childdoc.ins| & installation file \\
    |childdoc.dtx| & source file \\
    |childdoc.def| & definition file \\
    |cdocsamp.tex| & sample main file \\
    |cdocsch1.tex| & sample include file \\
    |cdocsch2.tex| & sample include file \\
    |cdocspt3.tex| & sample part file \\
    |cdocspt4.tex| & sample part file \\
    |cdocsdrf.tex| & sample redirection file \\
    |cdocsfn1.tex| & sample redirection file \\
    |cdocsfn2.tex| & sample redirection file \\
    |childdoc.pdf| & manual
\end{tabular}
\end{center}
%
The distribution consists of the files
|README.txt|, |childdoc.ins| and |childdoc.dtx|.
%
\begin{itemize}
\item
Run (pdf)\LaTeX{} on |childdoc.dtx|
to compile the manual |childdoc.pdf| (this file).
\item
Run \LaTeX{} on |childdoc.ins| to create the definitions file |childdoc.def|
and the sample |cdocsamp.tex| with include files
|cdocsch1.tex|, |cdocsch2.tex|, |cdocspt3.tex|, |cdocspt4.tex|,
|cdocsdrf.tex|, |cdocsfn1.tex|, |cdocsfn2.tex|.
Then copy the file |childdoc.def| to an appropriate directory of your \LaTeX{}
distribution, e.g.\ \textit{texmf-root}|/tex/latex/childdoc|.
\end{itemize}

%%%%%%%%%%%%%%%%%%%%%%%%%%%%%%%%%%%%%%%%%%%%%%%%%%%%%%%%%%%%%%%%%%%%%%%%%%%%%%%%
\subsection{Related CTAN Packages}

There are several other packages which offer a similar functionality:
%
\begin{itemize}
\item
The packages
\href{http://ctan.org/pkg/docmute}{\textsf{docmute}},
\href{http://ctan.org/pkg/includex}{\textsf{includex}} and
\href{http://ctan.org/pkg/standalone}{\textsf{standalone}}
provide commands to include only the document body of
a child file thus allowing both files to be compiled individually.
\item
The packages \href{http://ctan.org/pkg/subdocs}{\textsf{subdocs}}
and \href{http://ctan.org/pkg/subfiles}{\textsf{subfiles}}
provide structures in which the main and child documents can be
encapsulated and allowing them to be compiled individually.
The inclusion mechanism is different from the conventional |\include|.
\item
The package \href{http://ctan.org/pkg/combine}{\textsf{combine}}
is an elaborate solution to combine several documents into one.
\end{itemize}
%
See also the CTAN topic \href{http://ctan.org/topic/subdocs}{\textsf{subdocs}}
for further related packages.
The present package differs from the above solutions in that
a document structure constructed with the conventional |\include| mechanism
just needs two extra commands at the top of every file
such that all constituent files can be compiled individually.

%%%%%%%%%%%%%%%%%%%%%%%%%%%%%%%%%%%%%%%%%%%%%%%%%%%%%%%%%%%%%%%%%%%%%%%%%%%%%%%%
%\subsection{Feature Suggestions}
%
%The following is a list of features which may be useful for future
%versions of this package:
%%
%\begin{itemize}
%\item
%\ldots
%\end{itemize}

%%%%%%%%%%%%%%%%%%%%%%%%%%%%%%%%%%%%%%%%%%%%%%%%%%%%%%%%%%%%%%%%%%%%%%%%%%%%%%%%
\subsection{Revision History}

%%%%%%%%%%%%%%%%%%%%%%%%%%%%%%%%%%%%%%%%
\paragraph{v2.0:} 2018/12/30

\begin{itemize}
\item
immediate forward processing
\item
added |\childdocby| mechanism
\item
manual restructured
\end{itemize}

%%%%%%%%%%%%%%%%%%%%%%%%%%%%%%%%%%%%%%%%
\paragraph{v1.6:} 2018/01/17

\begin{itemize}
\item
application for development of include files
\item
corrections to manual
\end{itemize}

%%%%%%%%%%%%%%%%%%%%%%%%%%%%%%%%%%%%%%%%
\paragraph{v1.5:} 2017/05/21

\begin{itemize}
\item
more complete structuring introduced
\item
|\childdocof| introduced
\item
|\childdoc| renamed to |\childdocmain|
\item
|\childredirect| renamed to |\childdocforward| and |\childdocforwardprefix|
and functionality expanded
\end{itemize}

%%%%%%%%%%%%%%%%%%%%%%%%%%%%%%%%%%%%%%%%
\paragraph{v1.0:} 2017/04/27

\begin{itemize}
\item
manual and install package
\item
first version published on CTAN
\end{itemize}

%%%%%%%%%%%%%%%%%%%%%%%%%%%%%%%%%%%%%%%%
\paragraph{v0.6:} 2017/04/26

\begin{itemize}
\item
redirection mechanism added
\end{itemize}

%%%%%%%%%%%%%%%%%%%%%%%%%%%%%%%%%%%%%%%%
\paragraph{v0.5:} 2017/04/26

\begin{itemize}
\item
functionality in definition file
\end{itemize}


%%%%%%%%%%%%%%%%%%%%%%%%%%%%%%%%%%%%%%%%%%%%%%%%%%%%%%%%%%%%%%%%%%%%%%%%%%%%%%%%
%%%%%%%%%%%%%%%%%%%%%%%%%%%%%%%%%%%%%%%%%%%%%%%%%%%%%%%%%%%%%%%%%%%%%%%%%%%%%%%%
%%%%%%%%%%%%%%%%%%%%%%%%%%%%%%%%%%%%%%%%%%%%%%%%%%%%%%%%%%%%%%%%%%%%%%%%%%%%%%%%
\appendix

\settowidth\MacroIndent{\rmfamily\scriptsize 000\ }

 \DocInput{childdoc.dtx}

\end{document}
%</driver>
% \fi
%
% %%%%%%%%%%%%%%%%%%%%%%%%%%%%%%%%%%%%%%%%%%%%%%%%%%%%%%%%%%%%%%%%%%%%%%%%%%%%%%
% %%%%%%%%%%%%%%%%%%%%%%%%%%%%%%%%%%%%%%%%%%%%%%%%%%%%%%%%%%%%%%%%%%%%%%%%%%%%%%
% \section{Sample}
%\iffalse
%<*samplemain>
%\fi
%
% The following presents a sample document
% with two chapters, two parts, a title page,
% a compile flag as well as three forwarding files to set the flag.
% It consists of eight |.tex| files:
% \begin{center}
% \begin{tabular}{ll}
% |cdocsamp.tex|&main file\\
% |cdocsch1.tex|&include file for chapter 1\\
% |cdocsch2.tex|&include file for chapter 2\\
% |cdocspt3.tex|&include file for part 3\\
% |cdocspt4.tex|&include file for part 4\\
% |cdocsdrf.tex|&forwarding file for main file in draft mode\\
% |cdocsfi1.tex|&forwarding file for final version of chapter 1\\
% |cdocsfi2.tex|&forwarding file for final version of chapter 2\\
% \end{tabular}
% \end{center}
% Each of the eight files can be compiled directly by the \LaTeX{} compiler.
%
% %%%%%%%%%%%%%%%%%%%%%%%%%%%%%%%%%%%%%%
% \paragraph{Main File.}
%
% The main file is called |cdocsamp.tex|.
%
% Load the \textsf{childdoc} definitions and
% declare the filename for the main document:
%    \begin{macrocode}
\input{childdoc.def}
\childdocmain{}
%    \end{macrocode}

% Optional override for |\version| flag:
%    \begin{macrocode}
%%\ifchilddoc\else\providecommand{\version}{draft}\fi
%    \end{macrocode}

% Define the default values for the |\version| flag
% (|final| for the main file and |draft| for childs):
%    \begin{macrocode}
\ifchilddoc
\providecommand{\version}{draft}
\else
\providecommand{\version}{final}
\fi
%    \end{macrocode}

% Load the standard document class:
%    \begin{macrocode}
\documentclass[12pt]{article}
%    \end{macrocode}

% Start the document body:
%    \begin{macrocode}
\begin{document}
%    \end{macrocode}

% Declare a title page.
% Print title, part of document being processed and version flag:
%    \begin{macrocode}
\addtocounter{page}{-1}
\begin{center}
{\LARGE\bfseries{}childdoc example\par}
\vspace{1cm}
\ifchilddoc
\ifchilddocmanual part\else chapter\fi:
`\childdocname' of `\childdocjob'\par
\else
main document: `\childdocjob'\par
\fi
version: \version\par
\end{center}
\newpage
%    \end{macrocode}

% Manually include selected file,
% otherwise process as usual:
%    \begin{macrocode}
\ifchilddocmanual
\section*{part `\childdocname'}
\input{\childdocname}
\else
%    \end{macrocode}

% Include the two chapters:
%    \begin{macrocode}
\include{cdocsch1}
\include{cdocsch2}
%    \end{macrocode}

% Include the two parts unless only chapters should be displayed:
%    \begin{macrocode}
\ifchilddoc\else
\section{part three}
\input{cdocspt3}
\section{part four}
\input{cdocspt4}
\fi
%    \end{macrocode}

% Process as usual until here:
%    \begin{macrocode}
\fi
%    \end{macrocode}

% End of document body:
%    \begin{macrocode}
\end{document}
%    \end{macrocode}
%\iffalse
%</samplemain>
%\fi
%
% %%%%%%%%%%%%%%%%%%%%%%%%%%%%%%%%%%%%%%
% \paragraph{Chapter Include Files.}
%
% The include files are called |cdocsch1.tex| and |cdocsch2.tex|.
%
%\iffalse
%<*samplechap1|samplechap2>
%\fi

% Optional override for |\version| flag:
%    \begin{macrocode}
%%\providecommand{\version}{final}
%    \end{macrocode}

% Include the main document:
%    \begin{macrocode}
\input{childdoc.def}
\childdocof{cdocsamp}
%    \end{macrocode}

%\iffalse
%</samplechap1|samplechap2>
%\fi
%
%\iffalse
%<*samplechap1>
%\fi
% Some text for chapter 1:
%    \begin{macrocode}
\section{one}
some text in chapter one
%    \end{macrocode}

%\iffalse
%</samplechap1>
%\fi
% Some text for chapter 2:
%\iffalse
%<*samplechap2>
%\fi
%    \begin{macrocode}
\section{two}
more text in chapter two
%    \end{macrocode}

%\iffalse
%</samplechap2>
%\fi
%
% %%%%%%%%%%%%%%%%%%%%%%%%%%%%%%%%%%%%%%
% \paragraph{Part Include Files.}
%
% The include files are called |cdocspt3.tex| and |cdocspt4.tex|.
%
%\iffalse
%<*samplepart3|samplepart4>
%\fi

% Optional override for |\version| flag:
%    \begin{macrocode}
%%\providecommand{\version}{final}
%    \end{macrocode}

% Include the main document:
%    \begin{macrocode}
\input{childdoc.def}
\childdocby{cdocsamp}
%    \end{macrocode}

%\iffalse
%</samplepart3|samplepart4>
%\fi
%
%\iffalse
%<*samplepart3>
%\fi
% Some text for part 3:
%    \begin{macrocode}
some text in part three
%    \end{macrocode}

%\iffalse
%</samplepart3>
%\fi
% Some text for part 4:
%\iffalse
%<*samplepart4>
%\fi
%    \begin{macrocode}
more text in part four
%    \end{macrocode}

%\iffalse
%</samplepart4>
%\fi
%
% %%%%%%%%%%%%%%%%%%%%%%%%%%%%%%%%%%%%%%
% \paragraph{Forwarding for a Complete Draft.}
%
% The following forwarding file |cdocsdrf.tex|
% compiles the main document in draft mode:
%\iffalse
%<*sampledraft>
%\fi
%    \begin{macrocode}
\def\version{draft}
\input{childdoc.def}
\childdocforward{cdocsamp}
%    \end{macrocode}

%\iffalse
%</sampledraft>
%\fi
%
% %%%%%%%%%%%%%%%%%%%%%%%%%%%%%%%%%%%%%%
% \paragraph{Forwarding for Final Version of the Chapters.}
%
% The following forwarding files |cdocsfn1.tex| and |cdocsfn2.tex|
% (with identical content)
% compile the final versions of the child documents
% |cdocsch1.tex| and |cdocsch2.tex|, respectively:
%\iffalse
%<*samplefinal>
%\fi
%    \begin{macrocode}
\def\version{final}
\input{childdoc.def}
\childdocforwardprefix[cdocsamp]{cdocsfn}{cdocsch}
%    \end{macrocode}

%\iffalse
%</samplefinal>
%\fi
%
% %%%%%%%%%%%%%%%%%%%%%%%%%%%%%%%%%%%%%%
% \paragraph{Command Line Processing.}
%
% The following three command lines generate the output files
% |cdocscld|, |cdocscl1| and |cdocscl2|
% which should be identical to
% |cdocsdrf|, |cdocsch1| and |cdocsfn2|, respectively:
% \begin{center}
% \begin{tabular}{l}
% |latex -jobname cdocscld \|\\
% |  "\def\version{draft}\input{childdoc.def}\childdocforward{cdocsamp}"|\\
% |latex -jobname cdocscl1 \|\\
% |  "\input{childdoc.def}\childdocforward[cdocsamp]{cdocsch1}"|\\
% |latex -jobname cdocscl2 \|\\
% |  "\def\version{final}\input{childdoc.def}\childdocforward{cdocsch2}"|
% \end{tabular}
% \end{center}
% Note that the trailing backslash on each first line
% merely continues the input to the second line
% (for convenient cut ant paste).
% Furthermore, the command |latex| can be replaced by any
% of its alternative versions such as |pdflatex|.
%
% %%%%%%%%%%%%%%%%%%%%%%%%%%%%%%%%%%%%%%%%%%%%%%%%%%%%%%%%%%%%%%%%%%%%%%%%%%%%%%
% %%%%%%%%%%%%%%%%%%%%%%%%%%%%%%%%%%%%%%%%%%%%%%%%%%%%%%%%%%%%%%%%%%%%%%%%%%%%%%
% \section{Implementation}
%\iffalse
%<*package>
%\fi
%
% This section describes the definitions file |childdoc.def|.

% The definitions cannot be loaded using |\usepackage| or |\RequirePackage|
% which has a mechanism to prevent loading a style file more than once.
% When loading the definitions by means of |\input|
% multiple instances have to be prevented manually:
%\iffalse
%This code needs to be before the `\ProvidesFile' directive
%which is defined at the beginning of this file.
%Therefore it is also placed there and commented out here.
%</package>
%<*discard>
%\fi
%    \begin{macrocode}
\ifdefined\childdocmain\endinput\fi
%    \end{macrocode}
%\iffalse
%</discard>
%<*package>
%\fi
%
% \macro{\ifchilddoc}
% \macro{\ifchilddocmanual}
% The conditional |\ifchilddoc| tells whether a
% child (true) or main (false) document is being compiled.
% The conditional |\ifchilddocmanual| tells whether
% the |\includeonly| mechanism is used (false) or
% the selection of child files must be performed manually (true).
% The definitions initialise to false:
%    \begin{macrocode}
\newif\ifchilddoc
\newif\ifchilddocmanual
%    \end{macrocode}

% \macro{\childdocname}
% \macro{\childdocjob}
% The macro |\childdocname| stores the name of the main document
% to be compiled. The macro |\childdocjob| stores the name of
% the document on which the \LaTeX{} compiler was originally invoked.
% The content of |\jobname| cannot be compared
% to filenames specified in the source due to different catcodes.
% The following code rescans |\jobname|, stores the result
% in |\childdocname| and saves a copy in |\childdocjob|:
%    \begin{macrocode}
\edef\childdocname{\scantokens\expandafter{\jobname\noexpand}}
\let\childdocjob\childdocname
%    \end{macrocode}

% \macro{\childdocdisable}
% The macro |\childdocdisable| prevents the main file
% from being processed more than once.
% At this stage, the main document command |\childdocmain|
% is assumed to be called once again where it should do nothing.
% Any subsequent call to it should prevent
% a secondary processing of the main document
% It overwrites the forwarding commands
% |\childdocof| and |\childdocforward|
% with empty macros to prevent further inclusions of the main document:
%    \begin{macrocode}
\newcommand{\childdocdisable}
{
  \renewcommand{\childdocmain}[1]{\renewcommand{\childdocmain}[1]{\endinput}}
  \renewcommand{\childdocof}[1]{}
  \renewcommand{\childdocby}[2][]{}
  \renewcommand{\childdocforward}[2][]{}
  \renewcommand{\childdocdisable}{}
}
%    \end{macrocode}

% \macro{\childdocmain}
% The macro |\childdocmain| is to be called at the top of the main file
% with nothing or the main filename (without extension) as argument.
% First, it breaks loops.
% If the argument is not empty and does not match |\childdocname|
% (which is set by the first inclusion of |childdoc.def|),
% |\ifchilddoc| is set to true, |\includeonly| is applied to the child file
% and |\jobname| is set to the main file
% (for proper handling of |.aux| files):
%    \begin{macrocode}
\newcommand{\childdocmain}[1]
{
  \childdocdisable\childdocmain{}
  \if?#1?\else
    \begingroup
      \def\childdoctmp{#1}
      \ifx\childdoctmp\childdocname
        \def\childdoctmp{}
      \else
        \def\childdoctmp
        {
          \childdoctrue
          \includeonly{\childdocname}
          \def\childdocjob{#1}
          \def\jobname{#1}
        }
      \fi
      \expandafter
    \endgroup
    \childdoctmp
  \fi
}
%    \end{macrocode}

% \macro{\childdocof}
% The command |\childdocof| redirects
% compilation to the main file |#1|.
%    \begin{macrocode}
\newcommand{\childdocof}[1]
{
  \childdocdisable
  \childdoctrue
  \includeonly{\childdocname}
  \def\jobname{#1}
  \def\childdocjob{#1}
  \input{#1}
}
%    \end{macrocode}

% \macro{\childdocby}
% The command |\childdocby| ....
%    \begin{macrocode}
\newcommand{\childdocby}[2][]
{
  \childdocdisable
  \childdoctrue
  \childdocmanualtrue
  \if?#1?\else
    \def\jobname{#2}
  \fi
  \def\childdocjob{#2}
  \input{#2}
  \endinput
}
%    \end{macrocode}

% \macro{\childdocforward}
% The command |\childdocforward| redirects
% compilation to the main file or
% (if the optional argument is given) a child file.
% Parameters are set as if the main file
% or a child file starting with |\childdocof| was compiled.
% Then compilation is handed over to the main file:
%    \begin{macrocode}
\newcommand{\childdocforward}[2][]
{
  \begingroup
    \if?#1?
      \def\childdoctmp
      {
        \def\childdocname{#2}
        \def\childdocjob{#2}
        \def\jobname{#2}
        \input{#2}
        \endinput
      }
    \else
      \def\childdoctmp
      {
        \childdocdisable
        \def\childdocname{#2}
        \childdoctrue
        \includeonly{#2}
        \def\childdocjob{#1}
        \def\jobname{#1}
        \input{#1}
        \endinput
      }
    \fi
    \expandafter
  \endgroup
  \childdoctmp
}
%    \end{macrocode}

% \macro{\childdocforwardprefix}
% The command |\childdocforwardprefix| redirects
% compilation to the main or a child file by means of a pattern.
% The prefix |#1| in the current filename is replaced by |#2|
% and the suffix of the current filename is kept
% (it is assumed that the filename does not contain the substring `|~~~|'
% which is used as a delimiter).
% Compilation is handed over to the new file by |\childdocforward|:
%    \begin{macrocode}
\newcommand{\childdocforwardprefix}[3][]
{
  \begingroup
    \def\childdocextract #2##1~~~{\def\childdoctmp{\childdocforward[#1]{#3##1}}}
    \expandafter\childdocextract\childdocname~~~
    \expandafter
  \endgroup
  \childdoctmp
}
%    \end{macrocode}

% \macro{\childdoc}
% The deprecated macro |\childdoc| is a legacy version of |\childdocmain|:
%    \begin{macrocode}
\newcommand{\childdoc}{\childdocmain}
%    \end{macrocode}

% \macro{\childdocredirect}
% The deprecated macro |\childdocredirect| is a legacy version
% of |\childdocforward| and |\childdocforwardprefix|:
%    \begin{macrocode}
\newcommand{\childdocredirect}[2][]
{
  \begingroup
    \if?#1?
      \def\childdoctmp{\childdocforward{#2}}
    \else
      \def\childdoctmp{\childdocforwardprefix{#1}{#2}}
    \fi
    \expandafter
  \endgroup
  \childdoctmp
}
%    \end{macrocode}

%\iffalse
%</package>
%\fi
%
\endinput
\childdocforward[|\textit{main}|]{|\textit{dest}|}"|
\end{center}
%
Here \textit{target} is the name of the output file,
\textit{main} is the name of the main file
and \textit{dest} is the name of the main or child file to be processed
(all filenames without extensions).
The optional argument \textit{main} can be omitted
if \textit{main} matches \textit{dest}.
Optionally, compilation \textit{flags} can be defined via |\def| commands.
This command line makes the \TeX{} engine believe
it is compiling the file \textit{target}
whose content is specified as the latter parameter.
The provided code then forwards the processing to
\textit{main} or \textit{dest} as described in \secref{sec:forward}.

%%%%%%%%%%%%%%%%%%%%%%%%%%%%%%%%%%%%%%%%%%%%%%%%%%%%%%%%%%%%%%%%%%%%%%%%%%%%%%%%
\subsection{Include by Input}
\label{sec:input}

Including child documents by |\include| has some restrictions by design.
Most notably, the content of a child document always occupies
its own set of pages; pages cannot be shared between child documents.
Usually, this behaviour makes perfect sense
because each child document contain an essential part of the document.
However, in some situations it may be desirable to compose
a document from a collection of parts
without having mandatory page breaks between then.
For this case, the package
provides a mechanism to include parts
by |\input| which can also be processed individually.
However, by construction this mechanism
requires manual handling of the content to be output.

%%%%%%%%%%%%%%%%%%%%%%%%%%%%%%%%%%%%%%%%
\DescribeMacro{\ifchilddocmanual}
The main file should be prepared as usual, see \secref{sec:include}.
However, the document body must make a distinction
between processing of an individual part and of the main document, e.g.:
%
\begin{center}
\begin{tabular}{l}
|\ifchilddocmanual|\\
|\input{\childdocname}|\\
|\||else|\\
\textit{document body with }|\input{|\textit{part}|}|\\
|\||fi|
\end{tabular}
\end{center}
%
The conditional |\ifchilddocmanual| is true whenever
a part to be included by |\input| is being compiled,
and the name of the part is stored in |\childdocname|.

%%%%%%%%%%%%%%%%%%%%%%%%%%%%%%%%%%%%%%%%
\DescribeMacro{\childdocby}
Each part to be included by |\input| should start with:
%
\begin{center}
\begin{tabular}{l}
|% \iffalse
%
% childdoc.dtx Copyright (C) 2017-2018 Niklas Beisert
%
% This work may be distributed and/or modified under the
% conditions of the LaTeX Project Public License, either version 1.3
% of this license or (at your option) any later version.
% The latest version of this license is in
%   http://www.latex-project.org/lppl.txt
% and version 1.3 or later is part of all distributions of LaTeX
% version 2005/12/01 or later.
%
% This work has the LPPL maintenance status `maintained'.
%
% The Current Maintainer of this work is Niklas Beisert.
%
% This work consists of the files childdoc.dtx and childdoc.ins
% and the derived files childdoc.def and cdocsamp.tex with
% cdocsch1.tex, cdocsch2.tex, cdocsdrf.tex, cdocsfn1.tex, cdocsfn2.tex.
%
%<package>\ifdefined\childdocmain\endinput\fi
%<package>\ProvidesFile{childdoc.def}[2018/12/30 v2.0 child document driver]
%<samplemain>\ProvidesFile{cdocsamp.tex}[2018/12/30 v2.0 sample for childdoc]
%<*driver>
%\ProvidesFile{childdoc.drv}[2018/12/30 v2.0 childdoc reference manual file]
\PassOptionsToClass{10pt,a4paper}{article}
\documentclass{ltxdoc}

\usepackage[margin=35mm]{geometry}
\usepackage{hyperref}
\usepackage{hyperxmp}
\usepackage[usenames]{color}

\hypersetup{colorlinks=true}
\hypersetup{pdfstartview=FitH}
\hypersetup{pdfpagemode=UseNone}
\hypersetup{pdfsource={}}
\hypersetup{pdflang={en-UK}}
\hypersetup{pdfcopyright={Copyright 2017-2018 Niklas Beisert.
  This work may be distributed and/or modified under the
  conditions of the LaTeX Project Public License, either version 1.3
  of this license or (at your option) any later version.}}
\hypersetup{pdflicenseurl={http://www.latex-project.org/lppl.txt}}
\hypersetup{pdfcontactaddress={ETH Zurich, ITP, HIT K,
  Wolfgang-Pauli-Strasse 27}}
\hypersetup{pdfcontactpostcode={8093}}
\hypersetup{pdfcontactcity={Zurich}}
\hypersetup{pdfcontactcountry={Switzerland}}
\hypersetup{pdfcontactemail={nbeisert@itp.phys.ethz.ch}}
\hypersetup{pdfcontacturl={http://people.phys.ethz.ch/\xmptilde nbeisert/}}

\newcommand{\secref}[1]{\hyperref[#1]{section \ref*{#1}}}

\parskip1ex
\parindent0pt
\let\olditemize\itemize
\def\itemize{\olditemize\parskip0pt}

\begin{document}

\title{The \textsf{childdoc} Package}
\hypersetup{pdftitle={The childdoc Package}}
\author{Niklas Beisert\\[2ex]
  Institut f\"ur Theoretische Physik\\
  Eidgen\"ossische Technische Hochschule Z\"urich\\
  Wolfgang-Pauli-Strasse 27, 8093 Z\"urich, Switzerland\\[1ex]
  \href{mailto:nbeisert@itp.phys.ethz.ch}
  {\texttt{nbeisert@itp.phys.ethz.ch}}}
\hypersetup{pdfauthor={Niklas Beisert}}
\hypersetup{pdfsubject={Manual for the LaTeX2e Package childdoc}}
\date{30 December 2018, \textsf{v2.0}}
\maketitle

\begin{abstract}\noindent
\textsf{childdoc} is a \LaTeXe{} package
that enables the direct compilation
of document sections included by |\include|
to individual files.
\end{abstract}

\begingroup
\parskip0ex
\tableofcontents
\endgroup

%%%%%%%%%%%%%%%%%%%%%%%%%%%%%%%%%%%%%%%%%%%%%%%%%%%%%%%%%%%%%%%%%%%%%%%%%%%%%%%%
%%%%%%%%%%%%%%%%%%%%%%%%%%%%%%%%%%%%%%%%%%%%%%%%%%%%%%%%%%%%%%%%%%%%%%%%%%%%%%%%
\section{Introduction}

\LaTeX{} provides a mechanism to structure a large document (such as a book)
into a main file and several child files (containing the chapters)
using the |\include| command.
This mechanism is beneficial for documents
which span hundreds of pages in order to
make the source file(s) more manageable.
Moreover, compilation can be restricted to
selected child files by means of the |\includeonly| command.
The latter feature can be used to reduce the compilation time while editing
(this was significantly more useful in the earlier days of \LaTeX{})
or to generate a smaller document which is easier to navigate.
Another application of |\includeonly| is to generate
documents consisting of selected parts of the complete document.

However, there are a few drawbacks of the plain |\include| mechanism:
\begin{itemize}
\item
The child files cannot be compiled on their own,
they can only be compiled via the main file.
A naive editing environment
(such as a text editor with an option
to have the current file processed by \LaTeX)
may require one to switch to the main file before compiling;
attempting to compile the child file produces errors.
\item
The main file must be modified (each time)
to adjust the |\includeonly| command
to the present needs. This easily leaves the main file in a messy state.
\item
The generated document will always carry the filename
of the main document. This is inconvenient if
several child files are to be compiled and
to be kept for distribution.
\end{itemize}

The present package provides a simple interface
to make child files individually compilable by \LaTeX{}.
Compiling a child file then has the same effect as compiling
the main file with an |\includeonly| command
to select the appropriate child.
Moreover the generated document will carry the name of the child
rather than the main file.
This resolves all three above issues.

This feature is meant to make the editing of books,
thesis documents and lecture notes somewhat more convenient.
However, the package can also be used efficiently for
composing a series of documents (such as exercise sheets)
which are typically distributed individually.
It then assists the author in generating the individual documents
(potentially in different versions)
as well as a document containing the collected series.
Another application is in developing style files
or other kinds of included material
where compilation of the style file could redirect
to a sample or test file.

%%%%%%%%%%%%%%%%%%%%%%%%%%%%%%%%%%%%%%%%%%%%%%%%%%%%%%%%%%%%%%%%%%%%%%%%%%%%%%%%
%%%%%%%%%%%%%%%%%%%%%%%%%%%%%%%%%%%%%%%%%%%%%%%%%%%%%%%%%%%%%%%%%%%%%%%%%%%%%%%%
\section{Usage}

First of all, the package \textsf{childdoc} is \emph{not} a standard
\LaTeXe{} |.sty| style file! Therefore it needs to be invoked in
a non-standard way.

%%%%%%%%%%%%%%%%%%%%%%%%%%%%%%%%%%%%%%%%%%%%%%%%%%%%%%%%%%%%%%%%%%%%%%%%%%%%%%%%
\subsection{Included Files}
\label{sec:include}

%%%%%%%%%%%%%%%%%%%%%%%%%%%%%%%%%%%%%%%%
\DescribeMacro{\childdocmain}
To use the package, add the commands
\begin{center}
\begin{tabular}{l}
|\input{childdoc.def}|\\
|\childdocmain{}|\\
\end{tabular}
\end{center}
at the very top of the main \LaTeX{} file,
in particular \emph{before} the |\documentclass| statement!
The argument of |\childdocmain| should be left empty
(but it must be present).

%%%%%%%%%%%%%%%%%%%%%%%%%%%%%%%%%%%%%%%%
\DescribeMacro{\childdocof}
Furthermore, add the commands
\begin{center}
\begin{tabular}{l}
|\input{childdoc.def}|\\
|\childdocof{|\textit{main}|}|\\
\end{tabular}
\end{center}
at the top of every child file \textit{child}
which is included by |\include{|\textit{child}|}|
from within the main file
(or at least for those files to be compiled individually).
The argument \textit{main} must be the filename of the main file.

There are a couple of
considerations in setting up the main and child documents:

%%%%%%%%%%%%%%%%%%%%%%%%%%%%%%%%%%%%%%%%
\paragraph{Restrictions.}

Please note the following restrictions:
\begin{itemize}
\item
|\childdocmain| must be called with one argument \textit{main}
to ensure compatibility with earlier version of the package.
It must either be empty (|\childdocmain{}|)
or precisely match the filename of the main file in which it is specified.
See \secref{sec:detection} for further information.
\item
The filename \textit{main} must be specified without the |.tex| extension.
\item
The filename \textit{main} is case sensitive
(even in case-insensitive file systems)
due to internal string comparison.
\item
The argument \textit{main} should be fully expanded, it cannot be a macro.
\item
Subdirectories and special characters should be avoided in filenames.
\item
The command |\childdocmain{|\textit{main}|}| must be followed by a whitespace.
It should not be followed immediately by another command
or by a comment mark `|%|'.
This is because the \TeX{} parser reads the token immediately following
the argument of |\childdocmain| and puts it
at the beginning of every child section;
however, a white\-space is ignored.
\end{itemize}

%%%%%%%%%%%%%%%%%%%%%%%%%%%%%%%%%%%%%%%%
\paragraph{Content of Main File.}

It is advisable to place all content in the child files included by |\include|.
Any output contained in the main file will appear in all child documents
unless suppressed manually;
it cannot be suppressed automatically by the |\includeonly| directive
and thus should normally be avoided.
A method to include some content in the main file
by means of conditional processing is described in \secref{sec:conditional}.

%%%%%%%%%%%%%%%%%%%%%%%%%%%%%%%%%%%%%%%%
\paragraph{Page Numbering.}

When only a part of the document is compiled,
the appropriate numbering of pages
(as well as other status parameters)
is determined from the |.aux| files.
The latter contain information from previous passes.
However this information needs to propagate through
all intermediate child documents.
Therefore the page numbering in child documents may well
be inconsistent until the complete document is compiled at least once.

A useful (if unconventional) way to always ensure a consistent
page numbering is to restart the numbering in each child document
and denote the pages by `\textit{child}|.|\textit{page}'
where \textit{child} represents the chapter/section number of the child file.
This can be achieved by the command
|\numberwithin{page}{|\textit{child}|}|
of the \textsf{amsmath} package
where \textit{child} can be |chapter| or |section|
depending on the chosen structuring.
Alternatively, one can modify the macro |\thepage| appropriately
and reset the counter |page| at the start of each child file.

%%%%%%%%%%%%%%%%%%%%%%%%%%%%%%%%%%%%%%%%%%%%%%%%%%%%%%%%%%%%%%%%%%%%%%%%%%%%%%%%
\subsection{Conditional Processing}
\label{sec:conditional}

The package provides a mechanism to compile different versions
of a document. To customise the versions further some conditional processing
can come in handy to distinguish which version is being compiled.
The package provides two macros to describe the compilation context:

%%%%%%%%%%%%%%%%%%%%%%%%%%%%%%%%%%%%%%%%
\DescribeMacro{\ifchilddoc}
The conditional |\ifchilddoc| distinguishes between the compilation of
child documents and the main document:
%
\begin{center}
|\ifchilddoc |\textit{child-code}| |[|\||else |\textit{main-code}]| \||fi|
\end{center}

%%%%%%%%%%%%%%%%%%%%%%%%%%%%%%%%%%%%%%%%
\DescribeMacro{\childdocname}
\DescribeMacro{\childdocjob}
The macro |\childdocname| contains the filename (without extension)
of the main or child file being processed.
Note that |\childdocjob| will always contain the name of the main file.

%%%%%%%%%%%%%%%%%%%%%%%%%%%%%%%%%%%%%%%%
\paragraph{Title Page.}

Conditional processing can be used to include a title or banner page
in the main document when proper precautions are taken.
Importantly, the code in the main file should ensure that the page counter
(as well as other status parameters which are stored in the |.aux| files)
takes the same value after the conditional processing.
Otherwise the page numbers may take divergent values
depending on which part is compiled.

For example, a title page could be declared by:
%
\begin{center}
\begin{tabular}{l}
|\ifchilddoc\||else|\\
|\addtocounter{page}{-1}|\\
\textit{code for title page}\\
|\newpage|\\
|\||fi|
\end{tabular}
\end{center}
%
A banner page for the child documents can be generated by:
%
\begin{center}
\begin{tabular}{l}
|\ifchilddoc|\\
|\addtocounter{page}{-1}|\\
\textit{code for banner page}\\
|\newpage|\\
|\||fi|
\end{tabular}
\end{center}
%
Here one could write a message such as:
\begin{center}
|This is the part \childdocname{} of \childdocjob{}.|
\end{center}

%%%%%%%%%%%%%%%%%%%%%%%%%%%%%%%%%%%%%%%%%%%%%%%%%%%%%%%%%%%%%%%%%%%%%%%%%%%%%%%%
\subsection{Flags}
\label{sec:flags}

The package makes it easy to generate different versions
of the main or child documents.
To this end compilation flags can be defined
and assigned different default values.
They will be particularly useful in conjunction
with the forwarding mechanism described in \secref{sec:forward}.

For example, it may be useful to have a flag |\version|
which can be set to |draft| or |final|.
The document source will contain some conditional code
depending on the value of |\version|.
Suppose further, the flag should default to |final| for the main file
and to |draft| for child files
which is a natural assignment for editing the document.
This is achieved by placing the following code
in the preamble of the main document
(below the |\childdocmain| directive):
%
\begin{center}
\begin{tabular}{l}
|\ifchilddoc|\\
|\providecommand{\version}{draft}|\\
|\||else|\\
|\providecommand{\version}{final}|\\
|\||fi|
\end{tabular}
\end{center}
%
The definition by |\providecommand| makes sure
that previous definitions are not overwritten.
Further statements |\providecommand{\version}{...}|
can thus be added before the above code to override it.

For the main file, one might add a line
(between |\childdocmain| and the above block)
%
\begin{center}
|%\ifchilddoc\||else\providecommand{\version}{draft}\||fi|
\end{center}
%
which can be uncommented to produce a draft version.
Likewise one can add a line to the very top of a child file
(above the |\childdocof{|\textit{main}|}| directive)
%
\begin{center}
|%\providecommand{\version}{final}|
\end{center}
%
which can be uncommented to produce the final version of this child document.

%%%%%%%%%%%%%%%%%%%%%%%%%%%%%%%%%%%%%%%%%%%%%%%%%%%%%%%%%%%%%%%%%%%%%%%%%%%%%%%%
\subsection{Forwarding}
\label{sec:forward}

Different versions of the main or child documents
using compilation flags as described in \secref{sec:flags}
can be (permanently) stored in different files
for convenient compilation, viewing and distribution.
To this end, the package defines a command
to pass on compilation to a different file:

%%%%%%%%%%%%%%%%%%%%%%%%%%%%%%%%%%%%%%%%
\DescribeMacro{\childdocforward}
The command |\childdocforward| redirects processing to
another source file:
%
\begin{center}
\begin{tabular}{l}
|\input{childdoc.def}|\\
|\childdocforward[|\textit{main}|]{|\textit{dest}|}|\\
\end{tabular}
\end{center}
%
The argument \textit{dest} is the destination file
(without extension).
It should be the main file or one of the child files.
Note that further \textsf{childdoc} directives
such as |\childdocof| and |\childdocforward|
in the indicated file will be processed in this form.
The optional argument \textit{main}
passes on directly to the main file \textit{main}
while pretending to compile the child \textit{dest}.
This form behaves as if \textit{dest}
issues |\childdocof{|\textit{main}|}| right away,
and no further \textsf{childdoc} directives will be processed.

%%%%%%%%%%%%%%%%%%%%%%%%%%%%%%%%%%%%%%%%
\DescribeMacro{\...prefix}
In the alternative form |\childdocforwardprefix|,
%
\begin{center}
\begin{tabular}{l}
|\input{childdoc.def}|\\
|\childdocforwardprefix[|\textit{main}|]{|\textit{prefix}|}{|\textit{dest}|}|
\end{tabular}
\end{center}
%
the destination file is determined by a pattern
depending on the current file:
To make this work, the current file must be called
`{\textit{prefix}\hspace{0.2em}\textit{suffix}}'
with \textit{prefix} matching precisely the argument.
Processing is then passed on to the file
`{\textit{dest}\hspace{0.2em}\textit{suffix}}'.
Surely, the same effect is achieved by
directly specifying the
argument `{\textit{dest}\hspace{0.2em}\textit{suffix}}'
in the first form.
However, that requires to set up a different file
for each child. With the alternative form of the command
all these files can have exactly the same content
which simplifies setting them up and maintaining them.

For example, the following file |draft.tex|
with a compilation flag |\version| as described in \secref{sec:flags}
compiles the main document as a draft:
%
\begin{center}
\begin{tabular}{l}
|\def\version{draft}|\\
|\input{childdoc.def}|\\
|\childdocforward{|\textit{main}|}|
\end{tabular}
\end{center}
%
Likewise, the following files |final|\textit{nn}|.tex|
compile the final version of the child document
|child|\textit{nn}|.tex|:
%
\begin{center}
\begin{tabular}{l}
|\def\version{final}|\\
|\input{childdoc.def}|\\
|\childdocforwardprefix{final}{child}|
\end{tabular}
\end{center}
%

Note that when several versions of a main file and/or of each child file
are to be generated, it may be convenient to set up a |Makefile| or
shell script to automatise the process.

%%%%%%%%%%%%%%%%%%%%%%%%%%%%%%%%%%%%%%%%%%%%%%%%%%%%%%%%%%%%%%%%%%%%%%%%%%%%%%%%
\subsection{Command Line Processing}
\label{sec:commandline}

The effect of redirection files can also be achieved by invoking
the \LaTeX{} compiler with a more elaborate command line.
Most conveniently this should be done as part
of a shell script or a |Makefile|.

When using \textsf{childdoc} in the main file, the following
command lines effectively perform a redirection
(note that depending on the shell being used,
backslashes may have to be doubled: `|\|' $\to$ `|\\|'):
%
\begin{center}
|... -jobname "|\textit{target}|" |\\|"|[\textit{flags}]%
|\input{childdoc.def}\childdocforward[|\textit{main}|]{|\textit{dest}|}"|
\end{center}
%
Here \textit{target} is the name of the output file,
\textit{main} is the name of the main file
and \textit{dest} is the name of the main or child file to be processed
(all filenames without extensions).
The optional argument \textit{main} can be omitted
if \textit{main} matches \textit{dest}.
Optionally, compilation \textit{flags} can be defined via |\def| commands.
This command line makes the \TeX{} engine believe
it is compiling the file \textit{target}
whose content is specified as the latter parameter.
The provided code then forwards the processing to
\textit{main} or \textit{dest} as described in \secref{sec:forward}.

%%%%%%%%%%%%%%%%%%%%%%%%%%%%%%%%%%%%%%%%%%%%%%%%%%%%%%%%%%%%%%%%%%%%%%%%%%%%%%%%
\subsection{Include by Input}
\label{sec:input}

Including child documents by |\include| has some restrictions by design.
Most notably, the content of a child document always occupies
its own set of pages; pages cannot be shared between child documents.
Usually, this behaviour makes perfect sense
because each child document contain an essential part of the document.
However, in some situations it may be desirable to compose
a document from a collection of parts
without having mandatory page breaks between then.
For this case, the package
provides a mechanism to include parts
by |\input| which can also be processed individually.
However, by construction this mechanism
requires manual handling of the content to be output.

%%%%%%%%%%%%%%%%%%%%%%%%%%%%%%%%%%%%%%%%
\DescribeMacro{\ifchilddocmanual}
The main file should be prepared as usual, see \secref{sec:include}.
However, the document body must make a distinction
between processing of an individual part and of the main document, e.g.:
%
\begin{center}
\begin{tabular}{l}
|\ifchilddocmanual|\\
|\input{\childdocname}|\\
|\||else|\\
\textit{document body with }|\input{|\textit{part}|}|\\
|\||fi|
\end{tabular}
\end{center}
%
The conditional |\ifchilddocmanual| is true whenever
a part to be included by |\input| is being compiled,
and the name of the part is stored in |\childdocname|.

%%%%%%%%%%%%%%%%%%%%%%%%%%%%%%%%%%%%%%%%
\DescribeMacro{\childdocby}
Each part to be included by |\input| should start with:
%
\begin{center}
\begin{tabular}{l}
|\input{childdoc.def}|\\
|\childdocby{|\textit{main}|}|\\
\end{tabular}
\end{center}
%
The directive |\childdocby| is similar to |\childdocof|
described in \secref{sec:include},
but the subsequent selection of content must be done manually.
To that end, both |\ifchilddoc| and |\ifchilddocmanual|
will be true upon processing of a part,
and the name of the part is stored in |\childdocname|.
Note that |\jobname| will be set to the filename of the current part
so that each part receives an individual |.aux| file
that does not interfere with the |.aux| file(s) of the main document.
This behaviour can be altered by the alternative form
|\childdocby[*]{|\textit{main}|}| (with a non-empty optional argument)
which uses the |.aux| file of the main document
by setting |\jobname| to \textit{main}.

%%%%%%%%%%%%%%%%%%%%%%%%%%%%%%%%%%%%%%%%%%%%%%%%%%%%%%%%%%%%%%%%%%%%%%%%%%%%%%%%
\subsection{Driver Development}
\label{sec:driver}

The \textsf{childdoc} mechanism can also be use for the development
of definition files such as \LaTeX{} styles or classes.
This case differs from the above setup with multiple parts
included by |\include| in that no |\includeonly| should be invoked.
This can be achieved by starting the include file
(before |\ProvidesPackage|) with:
%
\begin{center}
\begin{tabular}{l}
|\input{childdoc.def}|\\
|\childdocforward{|\textit{main}|}|\\
\end{tabular}
\end{center}
%
or alternatively with:
%
\begin{center}
\begin{tabular}{l}
|\input{childdoc.def}|\\
|\childdocby{|\textit{main}|}|\\
\end{tabular}
\end{center}
%
Both forms have slightly different effects as described above.
The main file is prepared as usual, see \secref{sec:include}.

%%%%%%%%%%%%%%%%%%%%%%%%%%%%%%%%%%%%%%%%%%%%%%%%%%%%%%%%%%%%%%%%%%%%%%%%%%%%%%%%
\subsection{Legacy Detection}
\label{sec:detection}

The directive |\childdocmain| in the main file can detect
whether the complete document or merely a child is to be compiled
even without using the directive |\childdocof|.
This method is deprecated because it is less robust
and there is no compelling reason to use it;
it is merely provided for backward compatibility
and it may be removed in future versions.

If the detection mechanism is to be used,
it is mandatory to correctly specify
the filename of the main file as the argument of |\childdocmain|:
%
\begin{center}
\begin{tabular}{l}
|\input{childdoc.def}|\\
|\childdocmain{|\textit{main}|}|\\
\end{tabular}
\end{center}
%
If |\jobname| does not match the argument \textit{main} of |\childdocmain|,
it is assumed that |\jobname| points to the child file to be compiled.
When using |\childdocmain| with the main file specified as argument,
it suffices to start a child file
with just |\input{|\textit{main}|}|
without loading of the package and using |\childdocof|.
If instead all processing is done
with the appropriate \textsf{childdoc} directives,
the argument of \textit{main} of |\childdocmain| can be empty.

An alternative version of the command line processing described
in \secref{sec:commandline} using the detection mechanism reads:
%
\begin{center}
|... -jobname "|\textit{target}|" "|[\textit{flags}]%
[|\def\jobname{|\textit{dest}|}|]|\input{|\textit{main}|}"|
\end{center}

%%%%%%%%%%%%%%%%%%%%%%%%%%%%%%%%%%%%%%%%%%%%%%%%%%%%%%%%%%%%%%%%%%%%%%%%%%%%%%%%
\subsection{Manual Code}
\label{sec:manual}

In case one cannot be certain whether the definitions file |childdoc.def|
is installed on the target \TeX{} distribution
and one prefers not to ship it,
it is conceivable to paste a few relevant commands into the sources.

To that end, drop all statements |\input{childdoc.def}|
and perform the replacements as outlined below.
Instead of |\childdocmain{|\textit{main}|}| add the following code
to the top of the main file:
%
\begin{center}
\begin{tabular}{l}
|\||ifdefined\childdocname\endinput\||fi\newif\ifchilddoc|\\
|\edef\childdocname{\scantokens\expandafter{\jobname\noexpand}}|\\
|\def\childdocmain{|\textit{main}|}\||ifx\childdocmain\childdocname\||else|\\
|\childdoctrue\includeonly{\childdocname}\let\jobname\childdocmain\||fi|\\
\end{tabular}
\end{center}
%
Instead of |\childdocof{|\textit{main}|}| just include the main file
at the top of each child file:
%
\begin{center}
|\input{|\textit{main}|}|
\end{center}
%
A simple redirection |\childdocforward{|\textit{dest}|}| is achieved by:
%
\begin{center}
|\def\jobname{|\textit{dest}|}\input{\jobname}|
\end{center}
%
The redirection with prefix
|\childdocforwardprefix[|\textit{prefix}|]{|\textit{dest}|}|
is accomplished by:
%
\begin{center}
\begin{tabular}{l}
|{\edef\jobname{\scantokens\expandafter{\jobname\noexpand}}|\\
|\def\redirectjob |\textit{prefix}|#1~~~{\gdef\jobname{|\textit{dest}|#1}}|\\
|\expandafter\redirectjob\jobname~~~}\input{\jobname}|
\end{tabular}
\end{center}

In an alternative approach,
child documents can be compiled by a specific command line
without additional code or specific definitions:
%
\begin{center}
|... -jobname "|\textit{target}|" "|[\textit{flags}]%
|\includeonly{|\textit{dest}|}\input{|\textit{main}|}"|
\end{center}
%

%%%%%%%%%%%%%%%%%%%%%%%%%%%%%%%%%%%%%%%%%%%%%%%%%%%%%%%%%%%%%%%%%%%%%%%%%%%%%%%%
%%%%%%%%%%%%%%%%%%%%%%%%%%%%%%%%%%%%%%%%%%%%%%%%%%%%%%%%%%%%%%%%%%%%%%%%%%%%%%%%
\section{Information}

%%%%%%%%%%%%%%%%%%%%%%%%%%%%%%%%%%%%%%%%%%%%%%%%%%%%%%%%%%%%%%%%%%%%%%%%%%%%%%%%
\subsection{Copyright}

Copyright \copyright{} 2017--2018 Niklas Beisert

This work may be distributed and/or modified under the
conditions of the \LaTeX{} Project Public License, either version 1.3
of this license or (at your option) any later version.
The latest version of this license is in
  \url{http://www.latex-project.org/lppl.txt}
and version 1.3 or later is part of all distributions of \LaTeX{}
version 2005/12/01 or later.

This work has the LPPL maintenance status `maintained'.

The Current Maintainer of this work is Niklas Beisert.

This work consists of the files |README.txt|, |childdoc.ins| and |childdoc.dtx|
as well as the derived files |childdoc.def|, |cdocsamp.tex|
with |cdocsch1.tex|, |cdocsch2.tex|, |cdocspt3.tex|, |cdocspt4.tex|,
|cdocsdrf.tex|, |cdocsfn1.tex|, |cdocsfn2.tex|
as well as |childdoc.pdf|.

%%%%%%%%%%%%%%%%%%%%%%%%%%%%%%%%%%%%%%%%%%%%%%%%%%%%%%%%%%%%%%%%%%%%%%%%%%%%%%%%
\subsection{Files and Installation}

The package consists of the files:
%
\begin{center}
\begin{tabular}{ll}
    |README.txt|   & readme file \\
    |childdoc.ins| & installation file \\
    |childdoc.dtx| & source file \\
    |childdoc.def| & definition file \\
    |cdocsamp.tex| & sample main file \\
    |cdocsch1.tex| & sample include file \\
    |cdocsch2.tex| & sample include file \\
    |cdocspt3.tex| & sample part file \\
    |cdocspt4.tex| & sample part file \\
    |cdocsdrf.tex| & sample redirection file \\
    |cdocsfn1.tex| & sample redirection file \\
    |cdocsfn2.tex| & sample redirection file \\
    |childdoc.pdf| & manual
\end{tabular}
\end{center}
%
The distribution consists of the files
|README.txt|, |childdoc.ins| and |childdoc.dtx|.
%
\begin{itemize}
\item
Run (pdf)\LaTeX{} on |childdoc.dtx|
to compile the manual |childdoc.pdf| (this file).
\item
Run \LaTeX{} on |childdoc.ins| to create the definitions file |childdoc.def|
and the sample |cdocsamp.tex| with include files
|cdocsch1.tex|, |cdocsch2.tex|, |cdocspt3.tex|, |cdocspt4.tex|,
|cdocsdrf.tex|, |cdocsfn1.tex|, |cdocsfn2.tex|.
Then copy the file |childdoc.def| to an appropriate directory of your \LaTeX{}
distribution, e.g.\ \textit{texmf-root}|/tex/latex/childdoc|.
\end{itemize}

%%%%%%%%%%%%%%%%%%%%%%%%%%%%%%%%%%%%%%%%%%%%%%%%%%%%%%%%%%%%%%%%%%%%%%%%%%%%%%%%
\subsection{Related CTAN Packages}

There are several other packages which offer a similar functionality:
%
\begin{itemize}
\item
The packages
\href{http://ctan.org/pkg/docmute}{\textsf{docmute}},
\href{http://ctan.org/pkg/includex}{\textsf{includex}} and
\href{http://ctan.org/pkg/standalone}{\textsf{standalone}}
provide commands to include only the document body of
a child file thus allowing both files to be compiled individually.
\item
The packages \href{http://ctan.org/pkg/subdocs}{\textsf{subdocs}}
and \href{http://ctan.org/pkg/subfiles}{\textsf{subfiles}}
provide structures in which the main and child documents can be
encapsulated and allowing them to be compiled individually.
The inclusion mechanism is different from the conventional |\include|.
\item
The package \href{http://ctan.org/pkg/combine}{\textsf{combine}}
is an elaborate solution to combine several documents into one.
\end{itemize}
%
See also the CTAN topic \href{http://ctan.org/topic/subdocs}{\textsf{subdocs}}
for further related packages.
The present package differs from the above solutions in that
a document structure constructed with the conventional |\include| mechanism
just needs two extra commands at the top of every file
such that all constituent files can be compiled individually.

%%%%%%%%%%%%%%%%%%%%%%%%%%%%%%%%%%%%%%%%%%%%%%%%%%%%%%%%%%%%%%%%%%%%%%%%%%%%%%%%
%\subsection{Feature Suggestions}
%
%The following is a list of features which may be useful for future
%versions of this package:
%%
%\begin{itemize}
%\item
%\ldots
%\end{itemize}

%%%%%%%%%%%%%%%%%%%%%%%%%%%%%%%%%%%%%%%%%%%%%%%%%%%%%%%%%%%%%%%%%%%%%%%%%%%%%%%%
\subsection{Revision History}

%%%%%%%%%%%%%%%%%%%%%%%%%%%%%%%%%%%%%%%%
\paragraph{v2.0:} 2018/12/30

\begin{itemize}
\item
immediate forward processing
\item
added |\childdocby| mechanism
\item
manual restructured
\end{itemize}

%%%%%%%%%%%%%%%%%%%%%%%%%%%%%%%%%%%%%%%%
\paragraph{v1.6:} 2018/01/17

\begin{itemize}
\item
application for development of include files
\item
corrections to manual
\end{itemize}

%%%%%%%%%%%%%%%%%%%%%%%%%%%%%%%%%%%%%%%%
\paragraph{v1.5:} 2017/05/21

\begin{itemize}
\item
more complete structuring introduced
\item
|\childdocof| introduced
\item
|\childdoc| renamed to |\childdocmain|
\item
|\childredirect| renamed to |\childdocforward| and |\childdocforwardprefix|
and functionality expanded
\end{itemize}

%%%%%%%%%%%%%%%%%%%%%%%%%%%%%%%%%%%%%%%%
\paragraph{v1.0:} 2017/04/27

\begin{itemize}
\item
manual and install package
\item
first version published on CTAN
\end{itemize}

%%%%%%%%%%%%%%%%%%%%%%%%%%%%%%%%%%%%%%%%
\paragraph{v0.6:} 2017/04/26

\begin{itemize}
\item
redirection mechanism added
\end{itemize}

%%%%%%%%%%%%%%%%%%%%%%%%%%%%%%%%%%%%%%%%
\paragraph{v0.5:} 2017/04/26

\begin{itemize}
\item
functionality in definition file
\end{itemize}


%%%%%%%%%%%%%%%%%%%%%%%%%%%%%%%%%%%%%%%%%%%%%%%%%%%%%%%%%%%%%%%%%%%%%%%%%%%%%%%%
%%%%%%%%%%%%%%%%%%%%%%%%%%%%%%%%%%%%%%%%%%%%%%%%%%%%%%%%%%%%%%%%%%%%%%%%%%%%%%%%
%%%%%%%%%%%%%%%%%%%%%%%%%%%%%%%%%%%%%%%%%%%%%%%%%%%%%%%%%%%%%%%%%%%%%%%%%%%%%%%%
\appendix

\settowidth\MacroIndent{\rmfamily\scriptsize 000\ }

 \DocInput{childdoc.dtx}

\end{document}
%</driver>
% \fi
%
% %%%%%%%%%%%%%%%%%%%%%%%%%%%%%%%%%%%%%%%%%%%%%%%%%%%%%%%%%%%%%%%%%%%%%%%%%%%%%%
% %%%%%%%%%%%%%%%%%%%%%%%%%%%%%%%%%%%%%%%%%%%%%%%%%%%%%%%%%%%%%%%%%%%%%%%%%%%%%%
% \section{Sample}
%\iffalse
%<*samplemain>
%\fi
%
% The following presents a sample document
% with two chapters, two parts, a title page,
% a compile flag as well as three forwarding files to set the flag.
% It consists of eight |.tex| files:
% \begin{center}
% \begin{tabular}{ll}
% |cdocsamp.tex|&main file\\
% |cdocsch1.tex|&include file for chapter 1\\
% |cdocsch2.tex|&include file for chapter 2\\
% |cdocspt3.tex|&include file for part 3\\
% |cdocspt4.tex|&include file for part 4\\
% |cdocsdrf.tex|&forwarding file for main file in draft mode\\
% |cdocsfi1.tex|&forwarding file for final version of chapter 1\\
% |cdocsfi2.tex|&forwarding file for final version of chapter 2\\
% \end{tabular}
% \end{center}
% Each of the eight files can be compiled directly by the \LaTeX{} compiler.
%
% %%%%%%%%%%%%%%%%%%%%%%%%%%%%%%%%%%%%%%
% \paragraph{Main File.}
%
% The main file is called |cdocsamp.tex|.
%
% Load the \textsf{childdoc} definitions and
% declare the filename for the main document:
%    \begin{macrocode}
\input{childdoc.def}
\childdocmain{}
%    \end{macrocode}

% Optional override for |\version| flag:
%    \begin{macrocode}
%%\ifchilddoc\else\providecommand{\version}{draft}\fi
%    \end{macrocode}

% Define the default values for the |\version| flag
% (|final| for the main file and |draft| for childs):
%    \begin{macrocode}
\ifchilddoc
\providecommand{\version}{draft}
\else
\providecommand{\version}{final}
\fi
%    \end{macrocode}

% Load the standard document class:
%    \begin{macrocode}
\documentclass[12pt]{article}
%    \end{macrocode}

% Start the document body:
%    \begin{macrocode}
\begin{document}
%    \end{macrocode}

% Declare a title page.
% Print title, part of document being processed and version flag:
%    \begin{macrocode}
\addtocounter{page}{-1}
\begin{center}
{\LARGE\bfseries{}childdoc example\par}
\vspace{1cm}
\ifchilddoc
\ifchilddocmanual part\else chapter\fi:
`\childdocname' of `\childdocjob'\par
\else
main document: `\childdocjob'\par
\fi
version: \version\par
\end{center}
\newpage
%    \end{macrocode}

% Manually include selected file,
% otherwise process as usual:
%    \begin{macrocode}
\ifchilddocmanual
\section*{part `\childdocname'}
\input{\childdocname}
\else
%    \end{macrocode}

% Include the two chapters:
%    \begin{macrocode}
\include{cdocsch1}
\include{cdocsch2}
%    \end{macrocode}

% Include the two parts unless only chapters should be displayed:
%    \begin{macrocode}
\ifchilddoc\else
\section{part three}
\input{cdocspt3}
\section{part four}
\input{cdocspt4}
\fi
%    \end{macrocode}

% Process as usual until here:
%    \begin{macrocode}
\fi
%    \end{macrocode}

% End of document body:
%    \begin{macrocode}
\end{document}
%    \end{macrocode}
%\iffalse
%</samplemain>
%\fi
%
% %%%%%%%%%%%%%%%%%%%%%%%%%%%%%%%%%%%%%%
% \paragraph{Chapter Include Files.}
%
% The include files are called |cdocsch1.tex| and |cdocsch2.tex|.
%
%\iffalse
%<*samplechap1|samplechap2>
%\fi

% Optional override for |\version| flag:
%    \begin{macrocode}
%%\providecommand{\version}{final}
%    \end{macrocode}

% Include the main document:
%    \begin{macrocode}
\input{childdoc.def}
\childdocof{cdocsamp}
%    \end{macrocode}

%\iffalse
%</samplechap1|samplechap2>
%\fi
%
%\iffalse
%<*samplechap1>
%\fi
% Some text for chapter 1:
%    \begin{macrocode}
\section{one}
some text in chapter one
%    \end{macrocode}

%\iffalse
%</samplechap1>
%\fi
% Some text for chapter 2:
%\iffalse
%<*samplechap2>
%\fi
%    \begin{macrocode}
\section{two}
more text in chapter two
%    \end{macrocode}

%\iffalse
%</samplechap2>
%\fi
%
% %%%%%%%%%%%%%%%%%%%%%%%%%%%%%%%%%%%%%%
% \paragraph{Part Include Files.}
%
% The include files are called |cdocspt3.tex| and |cdocspt4.tex|.
%
%\iffalse
%<*samplepart3|samplepart4>
%\fi

% Optional override for |\version| flag:
%    \begin{macrocode}
%%\providecommand{\version}{final}
%    \end{macrocode}

% Include the main document:
%    \begin{macrocode}
\input{childdoc.def}
\childdocby{cdocsamp}
%    \end{macrocode}

%\iffalse
%</samplepart3|samplepart4>
%\fi
%
%\iffalse
%<*samplepart3>
%\fi
% Some text for part 3:
%    \begin{macrocode}
some text in part three
%    \end{macrocode}

%\iffalse
%</samplepart3>
%\fi
% Some text for part 4:
%\iffalse
%<*samplepart4>
%\fi
%    \begin{macrocode}
more text in part four
%    \end{macrocode}

%\iffalse
%</samplepart4>
%\fi
%
% %%%%%%%%%%%%%%%%%%%%%%%%%%%%%%%%%%%%%%
% \paragraph{Forwarding for a Complete Draft.}
%
% The following forwarding file |cdocsdrf.tex|
% compiles the main document in draft mode:
%\iffalse
%<*sampledraft>
%\fi
%    \begin{macrocode}
\def\version{draft}
\input{childdoc.def}
\childdocforward{cdocsamp}
%    \end{macrocode}

%\iffalse
%</sampledraft>
%\fi
%
% %%%%%%%%%%%%%%%%%%%%%%%%%%%%%%%%%%%%%%
% \paragraph{Forwarding for Final Version of the Chapters.}
%
% The following forwarding files |cdocsfn1.tex| and |cdocsfn2.tex|
% (with identical content)
% compile the final versions of the child documents
% |cdocsch1.tex| and |cdocsch2.tex|, respectively:
%\iffalse
%<*samplefinal>
%\fi
%    \begin{macrocode}
\def\version{final}
\input{childdoc.def}
\childdocforwardprefix[cdocsamp]{cdocsfn}{cdocsch}
%    \end{macrocode}

%\iffalse
%</samplefinal>
%\fi
%
% %%%%%%%%%%%%%%%%%%%%%%%%%%%%%%%%%%%%%%
% \paragraph{Command Line Processing.}
%
% The following three command lines generate the output files
% |cdocscld|, |cdocscl1| and |cdocscl2|
% which should be identical to
% |cdocsdrf|, |cdocsch1| and |cdocsfn2|, respectively:
% \begin{center}
% \begin{tabular}{l}
% |latex -jobname cdocscld \|\\
% |  "\def\version{draft}\input{childdoc.def}\childdocforward{cdocsamp}"|\\
% |latex -jobname cdocscl1 \|\\
% |  "\input{childdoc.def}\childdocforward[cdocsamp]{cdocsch1}"|\\
% |latex -jobname cdocscl2 \|\\
% |  "\def\version{final}\input{childdoc.def}\childdocforward{cdocsch2}"|
% \end{tabular}
% \end{center}
% Note that the trailing backslash on each first line
% merely continues the input to the second line
% (for convenient cut ant paste).
% Furthermore, the command |latex| can be replaced by any
% of its alternative versions such as |pdflatex|.
%
% %%%%%%%%%%%%%%%%%%%%%%%%%%%%%%%%%%%%%%%%%%%%%%%%%%%%%%%%%%%%%%%%%%%%%%%%%%%%%%
% %%%%%%%%%%%%%%%%%%%%%%%%%%%%%%%%%%%%%%%%%%%%%%%%%%%%%%%%%%%%%%%%%%%%%%%%%%%%%%
% \section{Implementation}
%\iffalse
%<*package>
%\fi
%
% This section describes the definitions file |childdoc.def|.

% The definitions cannot be loaded using |\usepackage| or |\RequirePackage|
% which has a mechanism to prevent loading a style file more than once.
% When loading the definitions by means of |\input|
% multiple instances have to be prevented manually:
%\iffalse
%This code needs to be before the `\ProvidesFile' directive
%which is defined at the beginning of this file.
%Therefore it is also placed there and commented out here.
%</package>
%<*discard>
%\fi
%    \begin{macrocode}
\ifdefined\childdocmain\endinput\fi
%    \end{macrocode}
%\iffalse
%</discard>
%<*package>
%\fi
%
% \macro{\ifchilddoc}
% \macro{\ifchilddocmanual}
% The conditional |\ifchilddoc| tells whether a
% child (true) or main (false) document is being compiled.
% The conditional |\ifchilddocmanual| tells whether
% the |\includeonly| mechanism is used (false) or
% the selection of child files must be performed manually (true).
% The definitions initialise to false:
%    \begin{macrocode}
\newif\ifchilddoc
\newif\ifchilddocmanual
%    \end{macrocode}

% \macro{\childdocname}
% \macro{\childdocjob}
% The macro |\childdocname| stores the name of the main document
% to be compiled. The macro |\childdocjob| stores the name of
% the document on which the \LaTeX{} compiler was originally invoked.
% The content of |\jobname| cannot be compared
% to filenames specified in the source due to different catcodes.
% The following code rescans |\jobname|, stores the result
% in |\childdocname| and saves a copy in |\childdocjob|:
%    \begin{macrocode}
\edef\childdocname{\scantokens\expandafter{\jobname\noexpand}}
\let\childdocjob\childdocname
%    \end{macrocode}

% \macro{\childdocdisable}
% The macro |\childdocdisable| prevents the main file
% from being processed more than once.
% At this stage, the main document command |\childdocmain|
% is assumed to be called once again where it should do nothing.
% Any subsequent call to it should prevent
% a secondary processing of the main document
% It overwrites the forwarding commands
% |\childdocof| and |\childdocforward|
% with empty macros to prevent further inclusions of the main document:
%    \begin{macrocode}
\newcommand{\childdocdisable}
{
  \renewcommand{\childdocmain}[1]{\renewcommand{\childdocmain}[1]{\endinput}}
  \renewcommand{\childdocof}[1]{}
  \renewcommand{\childdocby}[2][]{}
  \renewcommand{\childdocforward}[2][]{}
  \renewcommand{\childdocdisable}{}
}
%    \end{macrocode}

% \macro{\childdocmain}
% The macro |\childdocmain| is to be called at the top of the main file
% with nothing or the main filename (without extension) as argument.
% First, it breaks loops.
% If the argument is not empty and does not match |\childdocname|
% (which is set by the first inclusion of |childdoc.def|),
% |\ifchilddoc| is set to true, |\includeonly| is applied to the child file
% and |\jobname| is set to the main file
% (for proper handling of |.aux| files):
%    \begin{macrocode}
\newcommand{\childdocmain}[1]
{
  \childdocdisable\childdocmain{}
  \if?#1?\else
    \begingroup
      \def\childdoctmp{#1}
      \ifx\childdoctmp\childdocname
        \def\childdoctmp{}
      \else
        \def\childdoctmp
        {
          \childdoctrue
          \includeonly{\childdocname}
          \def\childdocjob{#1}
          \def\jobname{#1}
        }
      \fi
      \expandafter
    \endgroup
    \childdoctmp
  \fi
}
%    \end{macrocode}

% \macro{\childdocof}
% The command |\childdocof| redirects
% compilation to the main file |#1|.
%    \begin{macrocode}
\newcommand{\childdocof}[1]
{
  \childdocdisable
  \childdoctrue
  \includeonly{\childdocname}
  \def\jobname{#1}
  \def\childdocjob{#1}
  \input{#1}
}
%    \end{macrocode}

% \macro{\childdocby}
% The command |\childdocby| ....
%    \begin{macrocode}
\newcommand{\childdocby}[2][]
{
  \childdocdisable
  \childdoctrue
  \childdocmanualtrue
  \if?#1?\else
    \def\jobname{#2}
  \fi
  \def\childdocjob{#2}
  \input{#2}
  \endinput
}
%    \end{macrocode}

% \macro{\childdocforward}
% The command |\childdocforward| redirects
% compilation to the main file or
% (if the optional argument is given) a child file.
% Parameters are set as if the main file
% or a child file starting with |\childdocof| was compiled.
% Then compilation is handed over to the main file:
%    \begin{macrocode}
\newcommand{\childdocforward}[2][]
{
  \begingroup
    \if?#1?
      \def\childdoctmp
      {
        \def\childdocname{#2}
        \def\childdocjob{#2}
        \def\jobname{#2}
        \input{#2}
        \endinput
      }
    \else
      \def\childdoctmp
      {
        \childdocdisable
        \def\childdocname{#2}
        \childdoctrue
        \includeonly{#2}
        \def\childdocjob{#1}
        \def\jobname{#1}
        \input{#1}
        \endinput
      }
    \fi
    \expandafter
  \endgroup
  \childdoctmp
}
%    \end{macrocode}

% \macro{\childdocforwardprefix}
% The command |\childdocforwardprefix| redirects
% compilation to the main or a child file by means of a pattern.
% The prefix |#1| in the current filename is replaced by |#2|
% and the suffix of the current filename is kept
% (it is assumed that the filename does not contain the substring `|~~~|'
% which is used as a delimiter).
% Compilation is handed over to the new file by |\childdocforward|:
%    \begin{macrocode}
\newcommand{\childdocforwardprefix}[3][]
{
  \begingroup
    \def\childdocextract #2##1~~~{\def\childdoctmp{\childdocforward[#1]{#3##1}}}
    \expandafter\childdocextract\childdocname~~~
    \expandafter
  \endgroup
  \childdoctmp
}
%    \end{macrocode}

% \macro{\childdoc}
% The deprecated macro |\childdoc| is a legacy version of |\childdocmain|:
%    \begin{macrocode}
\newcommand{\childdoc}{\childdocmain}
%    \end{macrocode}

% \macro{\childdocredirect}
% The deprecated macro |\childdocredirect| is a legacy version
% of |\childdocforward| and |\childdocforwardprefix|:
%    \begin{macrocode}
\newcommand{\childdocredirect}[2][]
{
  \begingroup
    \if?#1?
      \def\childdoctmp{\childdocforward{#2}}
    \else
      \def\childdoctmp{\childdocforwardprefix{#1}{#2}}
    \fi
    \expandafter
  \endgroup
  \childdoctmp
}
%    \end{macrocode}

%\iffalse
%</package>
%\fi
%
\endinput
|\\
|\childdocby{|\textit{main}|}|\\
\end{tabular}
\end{center}
%
The directive |\childdocby| is similar to |\childdocof|
described in \secref{sec:include},
but the subsequent selection of content must be done manually.
To that end, both |\ifchilddoc| and |\ifchilddocmanual|
will be true upon processing of a part,
and the name of the part is stored in |\childdocname|.
Note that |\jobname| will be set to the filename of the current part
so that each part receives an individual |.aux| file
that does not interfere with the |.aux| file(s) of the main document.
This behaviour can be altered by the alternative form
|\childdocby[*]{|\textit{main}|}| (with a non-empty optional argument)
which uses the |.aux| file of the main document
by setting |\jobname| to \textit{main}.

%%%%%%%%%%%%%%%%%%%%%%%%%%%%%%%%%%%%%%%%%%%%%%%%%%%%%%%%%%%%%%%%%%%%%%%%%%%%%%%%
\subsection{Driver Development}
\label{sec:driver}

The \textsf{childdoc} mechanism can also be use for the development
of definition files such as \LaTeX{} styles or classes.
This case differs from the above setup with multiple parts
included by |\include| in that no |\includeonly| should be invoked.
This can be achieved by starting the include file
(before |\ProvidesPackage|) with:
%
\begin{center}
\begin{tabular}{l}
|% \iffalse
%
% childdoc.dtx Copyright (C) 2017-2018 Niklas Beisert
%
% This work may be distributed and/or modified under the
% conditions of the LaTeX Project Public License, either version 1.3
% of this license or (at your option) any later version.
% The latest version of this license is in
%   http://www.latex-project.org/lppl.txt
% and version 1.3 or later is part of all distributions of LaTeX
% version 2005/12/01 or later.
%
% This work has the LPPL maintenance status `maintained'.
%
% The Current Maintainer of this work is Niklas Beisert.
%
% This work consists of the files childdoc.dtx and childdoc.ins
% and the derived files childdoc.def and cdocsamp.tex with
% cdocsch1.tex, cdocsch2.tex, cdocsdrf.tex, cdocsfn1.tex, cdocsfn2.tex.
%
%<package>\ifdefined\childdocmain\endinput\fi
%<package>\ProvidesFile{childdoc.def}[2018/12/30 v2.0 child document driver]
%<samplemain>\ProvidesFile{cdocsamp.tex}[2018/12/30 v2.0 sample for childdoc]
%<*driver>
%\ProvidesFile{childdoc.drv}[2018/12/30 v2.0 childdoc reference manual file]
\PassOptionsToClass{10pt,a4paper}{article}
\documentclass{ltxdoc}

\usepackage[margin=35mm]{geometry}
\usepackage{hyperref}
\usepackage{hyperxmp}
\usepackage[usenames]{color}

\hypersetup{colorlinks=true}
\hypersetup{pdfstartview=FitH}
\hypersetup{pdfpagemode=UseNone}
\hypersetup{pdfsource={}}
\hypersetup{pdflang={en-UK}}
\hypersetup{pdfcopyright={Copyright 2017-2018 Niklas Beisert.
  This work may be distributed and/or modified under the
  conditions of the LaTeX Project Public License, either version 1.3
  of this license or (at your option) any later version.}}
\hypersetup{pdflicenseurl={http://www.latex-project.org/lppl.txt}}
\hypersetup{pdfcontactaddress={ETH Zurich, ITP, HIT K,
  Wolfgang-Pauli-Strasse 27}}
\hypersetup{pdfcontactpostcode={8093}}
\hypersetup{pdfcontactcity={Zurich}}
\hypersetup{pdfcontactcountry={Switzerland}}
\hypersetup{pdfcontactemail={nbeisert@itp.phys.ethz.ch}}
\hypersetup{pdfcontacturl={http://people.phys.ethz.ch/\xmptilde nbeisert/}}

\newcommand{\secref}[1]{\hyperref[#1]{section \ref*{#1}}}

\parskip1ex
\parindent0pt
\let\olditemize\itemize
\def\itemize{\olditemize\parskip0pt}

\begin{document}

\title{The \textsf{childdoc} Package}
\hypersetup{pdftitle={The childdoc Package}}
\author{Niklas Beisert\\[2ex]
  Institut f\"ur Theoretische Physik\\
  Eidgen\"ossische Technische Hochschule Z\"urich\\
  Wolfgang-Pauli-Strasse 27, 8093 Z\"urich, Switzerland\\[1ex]
  \href{mailto:nbeisert@itp.phys.ethz.ch}
  {\texttt{nbeisert@itp.phys.ethz.ch}}}
\hypersetup{pdfauthor={Niklas Beisert}}
\hypersetup{pdfsubject={Manual for the LaTeX2e Package childdoc}}
\date{30 December 2018, \textsf{v2.0}}
\maketitle

\begin{abstract}\noindent
\textsf{childdoc} is a \LaTeXe{} package
that enables the direct compilation
of document sections included by |\include|
to individual files.
\end{abstract}

\begingroup
\parskip0ex
\tableofcontents
\endgroup

%%%%%%%%%%%%%%%%%%%%%%%%%%%%%%%%%%%%%%%%%%%%%%%%%%%%%%%%%%%%%%%%%%%%%%%%%%%%%%%%
%%%%%%%%%%%%%%%%%%%%%%%%%%%%%%%%%%%%%%%%%%%%%%%%%%%%%%%%%%%%%%%%%%%%%%%%%%%%%%%%
\section{Introduction}

\LaTeX{} provides a mechanism to structure a large document (such as a book)
into a main file and several child files (containing the chapters)
using the |\include| command.
This mechanism is beneficial for documents
which span hundreds of pages in order to
make the source file(s) more manageable.
Moreover, compilation can be restricted to
selected child files by means of the |\includeonly| command.
The latter feature can be used to reduce the compilation time while editing
(this was significantly more useful in the earlier days of \LaTeX{})
or to generate a smaller document which is easier to navigate.
Another application of |\includeonly| is to generate
documents consisting of selected parts of the complete document.

However, there are a few drawbacks of the plain |\include| mechanism:
\begin{itemize}
\item
The child files cannot be compiled on their own,
they can only be compiled via the main file.
A naive editing environment
(such as a text editor with an option
to have the current file processed by \LaTeX)
may require one to switch to the main file before compiling;
attempting to compile the child file produces errors.
\item
The main file must be modified (each time)
to adjust the |\includeonly| command
to the present needs. This easily leaves the main file in a messy state.
\item
The generated document will always carry the filename
of the main document. This is inconvenient if
several child files are to be compiled and
to be kept for distribution.
\end{itemize}

The present package provides a simple interface
to make child files individually compilable by \LaTeX{}.
Compiling a child file then has the same effect as compiling
the main file with an |\includeonly| command
to select the appropriate child.
Moreover the generated document will carry the name of the child
rather than the main file.
This resolves all three above issues.

This feature is meant to make the editing of books,
thesis documents and lecture notes somewhat more convenient.
However, the package can also be used efficiently for
composing a series of documents (such as exercise sheets)
which are typically distributed individually.
It then assists the author in generating the individual documents
(potentially in different versions)
as well as a document containing the collected series.
Another application is in developing style files
or other kinds of included material
where compilation of the style file could redirect
to a sample or test file.

%%%%%%%%%%%%%%%%%%%%%%%%%%%%%%%%%%%%%%%%%%%%%%%%%%%%%%%%%%%%%%%%%%%%%%%%%%%%%%%%
%%%%%%%%%%%%%%%%%%%%%%%%%%%%%%%%%%%%%%%%%%%%%%%%%%%%%%%%%%%%%%%%%%%%%%%%%%%%%%%%
\section{Usage}

First of all, the package \textsf{childdoc} is \emph{not} a standard
\LaTeXe{} |.sty| style file! Therefore it needs to be invoked in
a non-standard way.

%%%%%%%%%%%%%%%%%%%%%%%%%%%%%%%%%%%%%%%%%%%%%%%%%%%%%%%%%%%%%%%%%%%%%%%%%%%%%%%%
\subsection{Included Files}
\label{sec:include}

%%%%%%%%%%%%%%%%%%%%%%%%%%%%%%%%%%%%%%%%
\DescribeMacro{\childdocmain}
To use the package, add the commands
\begin{center}
\begin{tabular}{l}
|\input{childdoc.def}|\\
|\childdocmain{}|\\
\end{tabular}
\end{center}
at the very top of the main \LaTeX{} file,
in particular \emph{before} the |\documentclass| statement!
The argument of |\childdocmain| should be left empty
(but it must be present).

%%%%%%%%%%%%%%%%%%%%%%%%%%%%%%%%%%%%%%%%
\DescribeMacro{\childdocof}
Furthermore, add the commands
\begin{center}
\begin{tabular}{l}
|\input{childdoc.def}|\\
|\childdocof{|\textit{main}|}|\\
\end{tabular}
\end{center}
at the top of every child file \textit{child}
which is included by |\include{|\textit{child}|}|
from within the main file
(or at least for those files to be compiled individually).
The argument \textit{main} must be the filename of the main file.

There are a couple of
considerations in setting up the main and child documents:

%%%%%%%%%%%%%%%%%%%%%%%%%%%%%%%%%%%%%%%%
\paragraph{Restrictions.}

Please note the following restrictions:
\begin{itemize}
\item
|\childdocmain| must be called with one argument \textit{main}
to ensure compatibility with earlier version of the package.
It must either be empty (|\childdocmain{}|)
or precisely match the filename of the main file in which it is specified.
See \secref{sec:detection} for further information.
\item
The filename \textit{main} must be specified without the |.tex| extension.
\item
The filename \textit{main} is case sensitive
(even in case-insensitive file systems)
due to internal string comparison.
\item
The argument \textit{main} should be fully expanded, it cannot be a macro.
\item
Subdirectories and special characters should be avoided in filenames.
\item
The command |\childdocmain{|\textit{main}|}| must be followed by a whitespace.
It should not be followed immediately by another command
or by a comment mark `|%|'.
This is because the \TeX{} parser reads the token immediately following
the argument of |\childdocmain| and puts it
at the beginning of every child section;
however, a white\-space is ignored.
\end{itemize}

%%%%%%%%%%%%%%%%%%%%%%%%%%%%%%%%%%%%%%%%
\paragraph{Content of Main File.}

It is advisable to place all content in the child files included by |\include|.
Any output contained in the main file will appear in all child documents
unless suppressed manually;
it cannot be suppressed automatically by the |\includeonly| directive
and thus should normally be avoided.
A method to include some content in the main file
by means of conditional processing is described in \secref{sec:conditional}.

%%%%%%%%%%%%%%%%%%%%%%%%%%%%%%%%%%%%%%%%
\paragraph{Page Numbering.}

When only a part of the document is compiled,
the appropriate numbering of pages
(as well as other status parameters)
is determined from the |.aux| files.
The latter contain information from previous passes.
However this information needs to propagate through
all intermediate child documents.
Therefore the page numbering in child documents may well
be inconsistent until the complete document is compiled at least once.

A useful (if unconventional) way to always ensure a consistent
page numbering is to restart the numbering in each child document
and denote the pages by `\textit{child}|.|\textit{page}'
where \textit{child} represents the chapter/section number of the child file.
This can be achieved by the command
|\numberwithin{page}{|\textit{child}|}|
of the \textsf{amsmath} package
where \textit{child} can be |chapter| or |section|
depending on the chosen structuring.
Alternatively, one can modify the macro |\thepage| appropriately
and reset the counter |page| at the start of each child file.

%%%%%%%%%%%%%%%%%%%%%%%%%%%%%%%%%%%%%%%%%%%%%%%%%%%%%%%%%%%%%%%%%%%%%%%%%%%%%%%%
\subsection{Conditional Processing}
\label{sec:conditional}

The package provides a mechanism to compile different versions
of a document. To customise the versions further some conditional processing
can come in handy to distinguish which version is being compiled.
The package provides two macros to describe the compilation context:

%%%%%%%%%%%%%%%%%%%%%%%%%%%%%%%%%%%%%%%%
\DescribeMacro{\ifchilddoc}
The conditional |\ifchilddoc| distinguishes between the compilation of
child documents and the main document:
%
\begin{center}
|\ifchilddoc |\textit{child-code}| |[|\||else |\textit{main-code}]| \||fi|
\end{center}

%%%%%%%%%%%%%%%%%%%%%%%%%%%%%%%%%%%%%%%%
\DescribeMacro{\childdocname}
\DescribeMacro{\childdocjob}
The macro |\childdocname| contains the filename (without extension)
of the main or child file being processed.
Note that |\childdocjob| will always contain the name of the main file.

%%%%%%%%%%%%%%%%%%%%%%%%%%%%%%%%%%%%%%%%
\paragraph{Title Page.}

Conditional processing can be used to include a title or banner page
in the main document when proper precautions are taken.
Importantly, the code in the main file should ensure that the page counter
(as well as other status parameters which are stored in the |.aux| files)
takes the same value after the conditional processing.
Otherwise the page numbers may take divergent values
depending on which part is compiled.

For example, a title page could be declared by:
%
\begin{center}
\begin{tabular}{l}
|\ifchilddoc\||else|\\
|\addtocounter{page}{-1}|\\
\textit{code for title page}\\
|\newpage|\\
|\||fi|
\end{tabular}
\end{center}
%
A banner page for the child documents can be generated by:
%
\begin{center}
\begin{tabular}{l}
|\ifchilddoc|\\
|\addtocounter{page}{-1}|\\
\textit{code for banner page}\\
|\newpage|\\
|\||fi|
\end{tabular}
\end{center}
%
Here one could write a message such as:
\begin{center}
|This is the part \childdocname{} of \childdocjob{}.|
\end{center}

%%%%%%%%%%%%%%%%%%%%%%%%%%%%%%%%%%%%%%%%%%%%%%%%%%%%%%%%%%%%%%%%%%%%%%%%%%%%%%%%
\subsection{Flags}
\label{sec:flags}

The package makes it easy to generate different versions
of the main or child documents.
To this end compilation flags can be defined
and assigned different default values.
They will be particularly useful in conjunction
with the forwarding mechanism described in \secref{sec:forward}.

For example, it may be useful to have a flag |\version|
which can be set to |draft| or |final|.
The document source will contain some conditional code
depending on the value of |\version|.
Suppose further, the flag should default to |final| for the main file
and to |draft| for child files
which is a natural assignment for editing the document.
This is achieved by placing the following code
in the preamble of the main document
(below the |\childdocmain| directive):
%
\begin{center}
\begin{tabular}{l}
|\ifchilddoc|\\
|\providecommand{\version}{draft}|\\
|\||else|\\
|\providecommand{\version}{final}|\\
|\||fi|
\end{tabular}
\end{center}
%
The definition by |\providecommand| makes sure
that previous definitions are not overwritten.
Further statements |\providecommand{\version}{...}|
can thus be added before the above code to override it.

For the main file, one might add a line
(between |\childdocmain| and the above block)
%
\begin{center}
|%\ifchilddoc\||else\providecommand{\version}{draft}\||fi|
\end{center}
%
which can be uncommented to produce a draft version.
Likewise one can add a line to the very top of a child file
(above the |\childdocof{|\textit{main}|}| directive)
%
\begin{center}
|%\providecommand{\version}{final}|
\end{center}
%
which can be uncommented to produce the final version of this child document.

%%%%%%%%%%%%%%%%%%%%%%%%%%%%%%%%%%%%%%%%%%%%%%%%%%%%%%%%%%%%%%%%%%%%%%%%%%%%%%%%
\subsection{Forwarding}
\label{sec:forward}

Different versions of the main or child documents
using compilation flags as described in \secref{sec:flags}
can be (permanently) stored in different files
for convenient compilation, viewing and distribution.
To this end, the package defines a command
to pass on compilation to a different file:

%%%%%%%%%%%%%%%%%%%%%%%%%%%%%%%%%%%%%%%%
\DescribeMacro{\childdocforward}
The command |\childdocforward| redirects processing to
another source file:
%
\begin{center}
\begin{tabular}{l}
|\input{childdoc.def}|\\
|\childdocforward[|\textit{main}|]{|\textit{dest}|}|\\
\end{tabular}
\end{center}
%
The argument \textit{dest} is the destination file
(without extension).
It should be the main file or one of the child files.
Note that further \textsf{childdoc} directives
such as |\childdocof| and |\childdocforward|
in the indicated file will be processed in this form.
The optional argument \textit{main}
passes on directly to the main file \textit{main}
while pretending to compile the child \textit{dest}.
This form behaves as if \textit{dest}
issues |\childdocof{|\textit{main}|}| right away,
and no further \textsf{childdoc} directives will be processed.

%%%%%%%%%%%%%%%%%%%%%%%%%%%%%%%%%%%%%%%%
\DescribeMacro{\...prefix}
In the alternative form |\childdocforwardprefix|,
%
\begin{center}
\begin{tabular}{l}
|\input{childdoc.def}|\\
|\childdocforwardprefix[|\textit{main}|]{|\textit{prefix}|}{|\textit{dest}|}|
\end{tabular}
\end{center}
%
the destination file is determined by a pattern
depending on the current file:
To make this work, the current file must be called
`{\textit{prefix}\hspace{0.2em}\textit{suffix}}'
with \textit{prefix} matching precisely the argument.
Processing is then passed on to the file
`{\textit{dest}\hspace{0.2em}\textit{suffix}}'.
Surely, the same effect is achieved by
directly specifying the
argument `{\textit{dest}\hspace{0.2em}\textit{suffix}}'
in the first form.
However, that requires to set up a different file
for each child. With the alternative form of the command
all these files can have exactly the same content
which simplifies setting them up and maintaining them.

For example, the following file |draft.tex|
with a compilation flag |\version| as described in \secref{sec:flags}
compiles the main document as a draft:
%
\begin{center}
\begin{tabular}{l}
|\def\version{draft}|\\
|\input{childdoc.def}|\\
|\childdocforward{|\textit{main}|}|
\end{tabular}
\end{center}
%
Likewise, the following files |final|\textit{nn}|.tex|
compile the final version of the child document
|child|\textit{nn}|.tex|:
%
\begin{center}
\begin{tabular}{l}
|\def\version{final}|\\
|\input{childdoc.def}|\\
|\childdocforwardprefix{final}{child}|
\end{tabular}
\end{center}
%

Note that when several versions of a main file and/or of each child file
are to be generated, it may be convenient to set up a |Makefile| or
shell script to automatise the process.

%%%%%%%%%%%%%%%%%%%%%%%%%%%%%%%%%%%%%%%%%%%%%%%%%%%%%%%%%%%%%%%%%%%%%%%%%%%%%%%%
\subsection{Command Line Processing}
\label{sec:commandline}

The effect of redirection files can also be achieved by invoking
the \LaTeX{} compiler with a more elaborate command line.
Most conveniently this should be done as part
of a shell script or a |Makefile|.

When using \textsf{childdoc} in the main file, the following
command lines effectively perform a redirection
(note that depending on the shell being used,
backslashes may have to be doubled: `|\|' $\to$ `|\\|'):
%
\begin{center}
|... -jobname "|\textit{target}|" |\\|"|[\textit{flags}]%
|\input{childdoc.def}\childdocforward[|\textit{main}|]{|\textit{dest}|}"|
\end{center}
%
Here \textit{target} is the name of the output file,
\textit{main} is the name of the main file
and \textit{dest} is the name of the main or child file to be processed
(all filenames without extensions).
The optional argument \textit{main} can be omitted
if \textit{main} matches \textit{dest}.
Optionally, compilation \textit{flags} can be defined via |\def| commands.
This command line makes the \TeX{} engine believe
it is compiling the file \textit{target}
whose content is specified as the latter parameter.
The provided code then forwards the processing to
\textit{main} or \textit{dest} as described in \secref{sec:forward}.

%%%%%%%%%%%%%%%%%%%%%%%%%%%%%%%%%%%%%%%%%%%%%%%%%%%%%%%%%%%%%%%%%%%%%%%%%%%%%%%%
\subsection{Include by Input}
\label{sec:input}

Including child documents by |\include| has some restrictions by design.
Most notably, the content of a child document always occupies
its own set of pages; pages cannot be shared between child documents.
Usually, this behaviour makes perfect sense
because each child document contain an essential part of the document.
However, in some situations it may be desirable to compose
a document from a collection of parts
without having mandatory page breaks between then.
For this case, the package
provides a mechanism to include parts
by |\input| which can also be processed individually.
However, by construction this mechanism
requires manual handling of the content to be output.

%%%%%%%%%%%%%%%%%%%%%%%%%%%%%%%%%%%%%%%%
\DescribeMacro{\ifchilddocmanual}
The main file should be prepared as usual, see \secref{sec:include}.
However, the document body must make a distinction
between processing of an individual part and of the main document, e.g.:
%
\begin{center}
\begin{tabular}{l}
|\ifchilddocmanual|\\
|\input{\childdocname}|\\
|\||else|\\
\textit{document body with }|\input{|\textit{part}|}|\\
|\||fi|
\end{tabular}
\end{center}
%
The conditional |\ifchilddocmanual| is true whenever
a part to be included by |\input| is being compiled,
and the name of the part is stored in |\childdocname|.

%%%%%%%%%%%%%%%%%%%%%%%%%%%%%%%%%%%%%%%%
\DescribeMacro{\childdocby}
Each part to be included by |\input| should start with:
%
\begin{center}
\begin{tabular}{l}
|\input{childdoc.def}|\\
|\childdocby{|\textit{main}|}|\\
\end{tabular}
\end{center}
%
The directive |\childdocby| is similar to |\childdocof|
described in \secref{sec:include},
but the subsequent selection of content must be done manually.
To that end, both |\ifchilddoc| and |\ifchilddocmanual|
will be true upon processing of a part,
and the name of the part is stored in |\childdocname|.
Note that |\jobname| will be set to the filename of the current part
so that each part receives an individual |.aux| file
that does not interfere with the |.aux| file(s) of the main document.
This behaviour can be altered by the alternative form
|\childdocby[*]{|\textit{main}|}| (with a non-empty optional argument)
which uses the |.aux| file of the main document
by setting |\jobname| to \textit{main}.

%%%%%%%%%%%%%%%%%%%%%%%%%%%%%%%%%%%%%%%%%%%%%%%%%%%%%%%%%%%%%%%%%%%%%%%%%%%%%%%%
\subsection{Driver Development}
\label{sec:driver}

The \textsf{childdoc} mechanism can also be use for the development
of definition files such as \LaTeX{} styles or classes.
This case differs from the above setup with multiple parts
included by |\include| in that no |\includeonly| should be invoked.
This can be achieved by starting the include file
(before |\ProvidesPackage|) with:
%
\begin{center}
\begin{tabular}{l}
|\input{childdoc.def}|\\
|\childdocforward{|\textit{main}|}|\\
\end{tabular}
\end{center}
%
or alternatively with:
%
\begin{center}
\begin{tabular}{l}
|\input{childdoc.def}|\\
|\childdocby{|\textit{main}|}|\\
\end{tabular}
\end{center}
%
Both forms have slightly different effects as described above.
The main file is prepared as usual, see \secref{sec:include}.

%%%%%%%%%%%%%%%%%%%%%%%%%%%%%%%%%%%%%%%%%%%%%%%%%%%%%%%%%%%%%%%%%%%%%%%%%%%%%%%%
\subsection{Legacy Detection}
\label{sec:detection}

The directive |\childdocmain| in the main file can detect
whether the complete document or merely a child is to be compiled
even without using the directive |\childdocof|.
This method is deprecated because it is less robust
and there is no compelling reason to use it;
it is merely provided for backward compatibility
and it may be removed in future versions.

If the detection mechanism is to be used,
it is mandatory to correctly specify
the filename of the main file as the argument of |\childdocmain|:
%
\begin{center}
\begin{tabular}{l}
|\input{childdoc.def}|\\
|\childdocmain{|\textit{main}|}|\\
\end{tabular}
\end{center}
%
If |\jobname| does not match the argument \textit{main} of |\childdocmain|,
it is assumed that |\jobname| points to the child file to be compiled.
When using |\childdocmain| with the main file specified as argument,
it suffices to start a child file
with just |\input{|\textit{main}|}|
without loading of the package and using |\childdocof|.
If instead all processing is done
with the appropriate \textsf{childdoc} directives,
the argument of \textit{main} of |\childdocmain| can be empty.

An alternative version of the command line processing described
in \secref{sec:commandline} using the detection mechanism reads:
%
\begin{center}
|... -jobname "|\textit{target}|" "|[\textit{flags}]%
[|\def\jobname{|\textit{dest}|}|]|\input{|\textit{main}|}"|
\end{center}

%%%%%%%%%%%%%%%%%%%%%%%%%%%%%%%%%%%%%%%%%%%%%%%%%%%%%%%%%%%%%%%%%%%%%%%%%%%%%%%%
\subsection{Manual Code}
\label{sec:manual}

In case one cannot be certain whether the definitions file |childdoc.def|
is installed on the target \TeX{} distribution
and one prefers not to ship it,
it is conceivable to paste a few relevant commands into the sources.

To that end, drop all statements |\input{childdoc.def}|
and perform the replacements as outlined below.
Instead of |\childdocmain{|\textit{main}|}| add the following code
to the top of the main file:
%
\begin{center}
\begin{tabular}{l}
|\||ifdefined\childdocname\endinput\||fi\newif\ifchilddoc|\\
|\edef\childdocname{\scantokens\expandafter{\jobname\noexpand}}|\\
|\def\childdocmain{|\textit{main}|}\||ifx\childdocmain\childdocname\||else|\\
|\childdoctrue\includeonly{\childdocname}\let\jobname\childdocmain\||fi|\\
\end{tabular}
\end{center}
%
Instead of |\childdocof{|\textit{main}|}| just include the main file
at the top of each child file:
%
\begin{center}
|\input{|\textit{main}|}|
\end{center}
%
A simple redirection |\childdocforward{|\textit{dest}|}| is achieved by:
%
\begin{center}
|\def\jobname{|\textit{dest}|}\input{\jobname}|
\end{center}
%
The redirection with prefix
|\childdocforwardprefix[|\textit{prefix}|]{|\textit{dest}|}|
is accomplished by:
%
\begin{center}
\begin{tabular}{l}
|{\edef\jobname{\scantokens\expandafter{\jobname\noexpand}}|\\
|\def\redirectjob |\textit{prefix}|#1~~~{\gdef\jobname{|\textit{dest}|#1}}|\\
|\expandafter\redirectjob\jobname~~~}\input{\jobname}|
\end{tabular}
\end{center}

In an alternative approach,
child documents can be compiled by a specific command line
without additional code or specific definitions:
%
\begin{center}
|... -jobname "|\textit{target}|" "|[\textit{flags}]%
|\includeonly{|\textit{dest}|}\input{|\textit{main}|}"|
\end{center}
%

%%%%%%%%%%%%%%%%%%%%%%%%%%%%%%%%%%%%%%%%%%%%%%%%%%%%%%%%%%%%%%%%%%%%%%%%%%%%%%%%
%%%%%%%%%%%%%%%%%%%%%%%%%%%%%%%%%%%%%%%%%%%%%%%%%%%%%%%%%%%%%%%%%%%%%%%%%%%%%%%%
\section{Information}

%%%%%%%%%%%%%%%%%%%%%%%%%%%%%%%%%%%%%%%%%%%%%%%%%%%%%%%%%%%%%%%%%%%%%%%%%%%%%%%%
\subsection{Copyright}

Copyright \copyright{} 2017--2018 Niklas Beisert

This work may be distributed and/or modified under the
conditions of the \LaTeX{} Project Public License, either version 1.3
of this license or (at your option) any later version.
The latest version of this license is in
  \url{http://www.latex-project.org/lppl.txt}
and version 1.3 or later is part of all distributions of \LaTeX{}
version 2005/12/01 or later.

This work has the LPPL maintenance status `maintained'.

The Current Maintainer of this work is Niklas Beisert.

This work consists of the files |README.txt|, |childdoc.ins| and |childdoc.dtx|
as well as the derived files |childdoc.def|, |cdocsamp.tex|
with |cdocsch1.tex|, |cdocsch2.tex|, |cdocspt3.tex|, |cdocspt4.tex|,
|cdocsdrf.tex|, |cdocsfn1.tex|, |cdocsfn2.tex|
as well as |childdoc.pdf|.

%%%%%%%%%%%%%%%%%%%%%%%%%%%%%%%%%%%%%%%%%%%%%%%%%%%%%%%%%%%%%%%%%%%%%%%%%%%%%%%%
\subsection{Files and Installation}

The package consists of the files:
%
\begin{center}
\begin{tabular}{ll}
    |README.txt|   & readme file \\
    |childdoc.ins| & installation file \\
    |childdoc.dtx| & source file \\
    |childdoc.def| & definition file \\
    |cdocsamp.tex| & sample main file \\
    |cdocsch1.tex| & sample include file \\
    |cdocsch2.tex| & sample include file \\
    |cdocspt3.tex| & sample part file \\
    |cdocspt4.tex| & sample part file \\
    |cdocsdrf.tex| & sample redirection file \\
    |cdocsfn1.tex| & sample redirection file \\
    |cdocsfn2.tex| & sample redirection file \\
    |childdoc.pdf| & manual
\end{tabular}
\end{center}
%
The distribution consists of the files
|README.txt|, |childdoc.ins| and |childdoc.dtx|.
%
\begin{itemize}
\item
Run (pdf)\LaTeX{} on |childdoc.dtx|
to compile the manual |childdoc.pdf| (this file).
\item
Run \LaTeX{} on |childdoc.ins| to create the definitions file |childdoc.def|
and the sample |cdocsamp.tex| with include files
|cdocsch1.tex|, |cdocsch2.tex|, |cdocspt3.tex|, |cdocspt4.tex|,
|cdocsdrf.tex|, |cdocsfn1.tex|, |cdocsfn2.tex|.
Then copy the file |childdoc.def| to an appropriate directory of your \LaTeX{}
distribution, e.g.\ \textit{texmf-root}|/tex/latex/childdoc|.
\end{itemize}

%%%%%%%%%%%%%%%%%%%%%%%%%%%%%%%%%%%%%%%%%%%%%%%%%%%%%%%%%%%%%%%%%%%%%%%%%%%%%%%%
\subsection{Related CTAN Packages}

There are several other packages which offer a similar functionality:
%
\begin{itemize}
\item
The packages
\href{http://ctan.org/pkg/docmute}{\textsf{docmute}},
\href{http://ctan.org/pkg/includex}{\textsf{includex}} and
\href{http://ctan.org/pkg/standalone}{\textsf{standalone}}
provide commands to include only the document body of
a child file thus allowing both files to be compiled individually.
\item
The packages \href{http://ctan.org/pkg/subdocs}{\textsf{subdocs}}
and \href{http://ctan.org/pkg/subfiles}{\textsf{subfiles}}
provide structures in which the main and child documents can be
encapsulated and allowing them to be compiled individually.
The inclusion mechanism is different from the conventional |\include|.
\item
The package \href{http://ctan.org/pkg/combine}{\textsf{combine}}
is an elaborate solution to combine several documents into one.
\end{itemize}
%
See also the CTAN topic \href{http://ctan.org/topic/subdocs}{\textsf{subdocs}}
for further related packages.
The present package differs from the above solutions in that
a document structure constructed with the conventional |\include| mechanism
just needs two extra commands at the top of every file
such that all constituent files can be compiled individually.

%%%%%%%%%%%%%%%%%%%%%%%%%%%%%%%%%%%%%%%%%%%%%%%%%%%%%%%%%%%%%%%%%%%%%%%%%%%%%%%%
%\subsection{Feature Suggestions}
%
%The following is a list of features which may be useful for future
%versions of this package:
%%
%\begin{itemize}
%\item
%\ldots
%\end{itemize}

%%%%%%%%%%%%%%%%%%%%%%%%%%%%%%%%%%%%%%%%%%%%%%%%%%%%%%%%%%%%%%%%%%%%%%%%%%%%%%%%
\subsection{Revision History}

%%%%%%%%%%%%%%%%%%%%%%%%%%%%%%%%%%%%%%%%
\paragraph{v2.0:} 2018/12/30

\begin{itemize}
\item
immediate forward processing
\item
added |\childdocby| mechanism
\item
manual restructured
\end{itemize}

%%%%%%%%%%%%%%%%%%%%%%%%%%%%%%%%%%%%%%%%
\paragraph{v1.6:} 2018/01/17

\begin{itemize}
\item
application for development of include files
\item
corrections to manual
\end{itemize}

%%%%%%%%%%%%%%%%%%%%%%%%%%%%%%%%%%%%%%%%
\paragraph{v1.5:} 2017/05/21

\begin{itemize}
\item
more complete structuring introduced
\item
|\childdocof| introduced
\item
|\childdoc| renamed to |\childdocmain|
\item
|\childredirect| renamed to |\childdocforward| and |\childdocforwardprefix|
and functionality expanded
\end{itemize}

%%%%%%%%%%%%%%%%%%%%%%%%%%%%%%%%%%%%%%%%
\paragraph{v1.0:} 2017/04/27

\begin{itemize}
\item
manual and install package
\item
first version published on CTAN
\end{itemize}

%%%%%%%%%%%%%%%%%%%%%%%%%%%%%%%%%%%%%%%%
\paragraph{v0.6:} 2017/04/26

\begin{itemize}
\item
redirection mechanism added
\end{itemize}

%%%%%%%%%%%%%%%%%%%%%%%%%%%%%%%%%%%%%%%%
\paragraph{v0.5:} 2017/04/26

\begin{itemize}
\item
functionality in definition file
\end{itemize}


%%%%%%%%%%%%%%%%%%%%%%%%%%%%%%%%%%%%%%%%%%%%%%%%%%%%%%%%%%%%%%%%%%%%%%%%%%%%%%%%
%%%%%%%%%%%%%%%%%%%%%%%%%%%%%%%%%%%%%%%%%%%%%%%%%%%%%%%%%%%%%%%%%%%%%%%%%%%%%%%%
%%%%%%%%%%%%%%%%%%%%%%%%%%%%%%%%%%%%%%%%%%%%%%%%%%%%%%%%%%%%%%%%%%%%%%%%%%%%%%%%
\appendix

\settowidth\MacroIndent{\rmfamily\scriptsize 000\ }

 \DocInput{childdoc.dtx}

\end{document}
%</driver>
% \fi
%
% %%%%%%%%%%%%%%%%%%%%%%%%%%%%%%%%%%%%%%%%%%%%%%%%%%%%%%%%%%%%%%%%%%%%%%%%%%%%%%
% %%%%%%%%%%%%%%%%%%%%%%%%%%%%%%%%%%%%%%%%%%%%%%%%%%%%%%%%%%%%%%%%%%%%%%%%%%%%%%
% \section{Sample}
%\iffalse
%<*samplemain>
%\fi
%
% The following presents a sample document
% with two chapters, two parts, a title page,
% a compile flag as well as three forwarding files to set the flag.
% It consists of eight |.tex| files:
% \begin{center}
% \begin{tabular}{ll}
% |cdocsamp.tex|&main file\\
% |cdocsch1.tex|&include file for chapter 1\\
% |cdocsch2.tex|&include file for chapter 2\\
% |cdocspt3.tex|&include file for part 3\\
% |cdocspt4.tex|&include file for part 4\\
% |cdocsdrf.tex|&forwarding file for main file in draft mode\\
% |cdocsfi1.tex|&forwarding file for final version of chapter 1\\
% |cdocsfi2.tex|&forwarding file for final version of chapter 2\\
% \end{tabular}
% \end{center}
% Each of the eight files can be compiled directly by the \LaTeX{} compiler.
%
% %%%%%%%%%%%%%%%%%%%%%%%%%%%%%%%%%%%%%%
% \paragraph{Main File.}
%
% The main file is called |cdocsamp.tex|.
%
% Load the \textsf{childdoc} definitions and
% declare the filename for the main document:
%    \begin{macrocode}
\input{childdoc.def}
\childdocmain{}
%    \end{macrocode}

% Optional override for |\version| flag:
%    \begin{macrocode}
%%\ifchilddoc\else\providecommand{\version}{draft}\fi
%    \end{macrocode}

% Define the default values for the |\version| flag
% (|final| for the main file and |draft| for childs):
%    \begin{macrocode}
\ifchilddoc
\providecommand{\version}{draft}
\else
\providecommand{\version}{final}
\fi
%    \end{macrocode}

% Load the standard document class:
%    \begin{macrocode}
\documentclass[12pt]{article}
%    \end{macrocode}

% Start the document body:
%    \begin{macrocode}
\begin{document}
%    \end{macrocode}

% Declare a title page.
% Print title, part of document being processed and version flag:
%    \begin{macrocode}
\addtocounter{page}{-1}
\begin{center}
{\LARGE\bfseries{}childdoc example\par}
\vspace{1cm}
\ifchilddoc
\ifchilddocmanual part\else chapter\fi:
`\childdocname' of `\childdocjob'\par
\else
main document: `\childdocjob'\par
\fi
version: \version\par
\end{center}
\newpage
%    \end{macrocode}

% Manually include selected file,
% otherwise process as usual:
%    \begin{macrocode}
\ifchilddocmanual
\section*{part `\childdocname'}
\input{\childdocname}
\else
%    \end{macrocode}

% Include the two chapters:
%    \begin{macrocode}
\include{cdocsch1}
\include{cdocsch2}
%    \end{macrocode}

% Include the two parts unless only chapters should be displayed:
%    \begin{macrocode}
\ifchilddoc\else
\section{part three}
\input{cdocspt3}
\section{part four}
\input{cdocspt4}
\fi
%    \end{macrocode}

% Process as usual until here:
%    \begin{macrocode}
\fi
%    \end{macrocode}

% End of document body:
%    \begin{macrocode}
\end{document}
%    \end{macrocode}
%\iffalse
%</samplemain>
%\fi
%
% %%%%%%%%%%%%%%%%%%%%%%%%%%%%%%%%%%%%%%
% \paragraph{Chapter Include Files.}
%
% The include files are called |cdocsch1.tex| and |cdocsch2.tex|.
%
%\iffalse
%<*samplechap1|samplechap2>
%\fi

% Optional override for |\version| flag:
%    \begin{macrocode}
%%\providecommand{\version}{final}
%    \end{macrocode}

% Include the main document:
%    \begin{macrocode}
\input{childdoc.def}
\childdocof{cdocsamp}
%    \end{macrocode}

%\iffalse
%</samplechap1|samplechap2>
%\fi
%
%\iffalse
%<*samplechap1>
%\fi
% Some text for chapter 1:
%    \begin{macrocode}
\section{one}
some text in chapter one
%    \end{macrocode}

%\iffalse
%</samplechap1>
%\fi
% Some text for chapter 2:
%\iffalse
%<*samplechap2>
%\fi
%    \begin{macrocode}
\section{two}
more text in chapter two
%    \end{macrocode}

%\iffalse
%</samplechap2>
%\fi
%
% %%%%%%%%%%%%%%%%%%%%%%%%%%%%%%%%%%%%%%
% \paragraph{Part Include Files.}
%
% The include files are called |cdocspt3.tex| and |cdocspt4.tex|.
%
%\iffalse
%<*samplepart3|samplepart4>
%\fi

% Optional override for |\version| flag:
%    \begin{macrocode}
%%\providecommand{\version}{final}
%    \end{macrocode}

% Include the main document:
%    \begin{macrocode}
\input{childdoc.def}
\childdocby{cdocsamp}
%    \end{macrocode}

%\iffalse
%</samplepart3|samplepart4>
%\fi
%
%\iffalse
%<*samplepart3>
%\fi
% Some text for part 3:
%    \begin{macrocode}
some text in part three
%    \end{macrocode}

%\iffalse
%</samplepart3>
%\fi
% Some text for part 4:
%\iffalse
%<*samplepart4>
%\fi
%    \begin{macrocode}
more text in part four
%    \end{macrocode}

%\iffalse
%</samplepart4>
%\fi
%
% %%%%%%%%%%%%%%%%%%%%%%%%%%%%%%%%%%%%%%
% \paragraph{Forwarding for a Complete Draft.}
%
% The following forwarding file |cdocsdrf.tex|
% compiles the main document in draft mode:
%\iffalse
%<*sampledraft>
%\fi
%    \begin{macrocode}
\def\version{draft}
\input{childdoc.def}
\childdocforward{cdocsamp}
%    \end{macrocode}

%\iffalse
%</sampledraft>
%\fi
%
% %%%%%%%%%%%%%%%%%%%%%%%%%%%%%%%%%%%%%%
% \paragraph{Forwarding for Final Version of the Chapters.}
%
% The following forwarding files |cdocsfn1.tex| and |cdocsfn2.tex|
% (with identical content)
% compile the final versions of the child documents
% |cdocsch1.tex| and |cdocsch2.tex|, respectively:
%\iffalse
%<*samplefinal>
%\fi
%    \begin{macrocode}
\def\version{final}
\input{childdoc.def}
\childdocforwardprefix[cdocsamp]{cdocsfn}{cdocsch}
%    \end{macrocode}

%\iffalse
%</samplefinal>
%\fi
%
% %%%%%%%%%%%%%%%%%%%%%%%%%%%%%%%%%%%%%%
% \paragraph{Command Line Processing.}
%
% The following three command lines generate the output files
% |cdocscld|, |cdocscl1| and |cdocscl2|
% which should be identical to
% |cdocsdrf|, |cdocsch1| and |cdocsfn2|, respectively:
% \begin{center}
% \begin{tabular}{l}
% |latex -jobname cdocscld \|\\
% |  "\def\version{draft}\input{childdoc.def}\childdocforward{cdocsamp}"|\\
% |latex -jobname cdocscl1 \|\\
% |  "\input{childdoc.def}\childdocforward[cdocsamp]{cdocsch1}"|\\
% |latex -jobname cdocscl2 \|\\
% |  "\def\version{final}\input{childdoc.def}\childdocforward{cdocsch2}"|
% \end{tabular}
% \end{center}
% Note that the trailing backslash on each first line
% merely continues the input to the second line
% (for convenient cut ant paste).
% Furthermore, the command |latex| can be replaced by any
% of its alternative versions such as |pdflatex|.
%
% %%%%%%%%%%%%%%%%%%%%%%%%%%%%%%%%%%%%%%%%%%%%%%%%%%%%%%%%%%%%%%%%%%%%%%%%%%%%%%
% %%%%%%%%%%%%%%%%%%%%%%%%%%%%%%%%%%%%%%%%%%%%%%%%%%%%%%%%%%%%%%%%%%%%%%%%%%%%%%
% \section{Implementation}
%\iffalse
%<*package>
%\fi
%
% This section describes the definitions file |childdoc.def|.

% The definitions cannot be loaded using |\usepackage| or |\RequirePackage|
% which has a mechanism to prevent loading a style file more than once.
% When loading the definitions by means of |\input|
% multiple instances have to be prevented manually:
%\iffalse
%This code needs to be before the `\ProvidesFile' directive
%which is defined at the beginning of this file.
%Therefore it is also placed there and commented out here.
%</package>
%<*discard>
%\fi
%    \begin{macrocode}
\ifdefined\childdocmain\endinput\fi
%    \end{macrocode}
%\iffalse
%</discard>
%<*package>
%\fi
%
% \macro{\ifchilddoc}
% \macro{\ifchilddocmanual}
% The conditional |\ifchilddoc| tells whether a
% child (true) or main (false) document is being compiled.
% The conditional |\ifchilddocmanual| tells whether
% the |\includeonly| mechanism is used (false) or
% the selection of child files must be performed manually (true).
% The definitions initialise to false:
%    \begin{macrocode}
\newif\ifchilddoc
\newif\ifchilddocmanual
%    \end{macrocode}

% \macro{\childdocname}
% \macro{\childdocjob}
% The macro |\childdocname| stores the name of the main document
% to be compiled. The macro |\childdocjob| stores the name of
% the document on which the \LaTeX{} compiler was originally invoked.
% The content of |\jobname| cannot be compared
% to filenames specified in the source due to different catcodes.
% The following code rescans |\jobname|, stores the result
% in |\childdocname| and saves a copy in |\childdocjob|:
%    \begin{macrocode}
\edef\childdocname{\scantokens\expandafter{\jobname\noexpand}}
\let\childdocjob\childdocname
%    \end{macrocode}

% \macro{\childdocdisable}
% The macro |\childdocdisable| prevents the main file
% from being processed more than once.
% At this stage, the main document command |\childdocmain|
% is assumed to be called once again where it should do nothing.
% Any subsequent call to it should prevent
% a secondary processing of the main document
% It overwrites the forwarding commands
% |\childdocof| and |\childdocforward|
% with empty macros to prevent further inclusions of the main document:
%    \begin{macrocode}
\newcommand{\childdocdisable}
{
  \renewcommand{\childdocmain}[1]{\renewcommand{\childdocmain}[1]{\endinput}}
  \renewcommand{\childdocof}[1]{}
  \renewcommand{\childdocby}[2][]{}
  \renewcommand{\childdocforward}[2][]{}
  \renewcommand{\childdocdisable}{}
}
%    \end{macrocode}

% \macro{\childdocmain}
% The macro |\childdocmain| is to be called at the top of the main file
% with nothing or the main filename (without extension) as argument.
% First, it breaks loops.
% If the argument is not empty and does not match |\childdocname|
% (which is set by the first inclusion of |childdoc.def|),
% |\ifchilddoc| is set to true, |\includeonly| is applied to the child file
% and |\jobname| is set to the main file
% (for proper handling of |.aux| files):
%    \begin{macrocode}
\newcommand{\childdocmain}[1]
{
  \childdocdisable\childdocmain{}
  \if?#1?\else
    \begingroup
      \def\childdoctmp{#1}
      \ifx\childdoctmp\childdocname
        \def\childdoctmp{}
      \else
        \def\childdoctmp
        {
          \childdoctrue
          \includeonly{\childdocname}
          \def\childdocjob{#1}
          \def\jobname{#1}
        }
      \fi
      \expandafter
    \endgroup
    \childdoctmp
  \fi
}
%    \end{macrocode}

% \macro{\childdocof}
% The command |\childdocof| redirects
% compilation to the main file |#1|.
%    \begin{macrocode}
\newcommand{\childdocof}[1]
{
  \childdocdisable
  \childdoctrue
  \includeonly{\childdocname}
  \def\jobname{#1}
  \def\childdocjob{#1}
  \input{#1}
}
%    \end{macrocode}

% \macro{\childdocby}
% The command |\childdocby| ....
%    \begin{macrocode}
\newcommand{\childdocby}[2][]
{
  \childdocdisable
  \childdoctrue
  \childdocmanualtrue
  \if?#1?\else
    \def\jobname{#2}
  \fi
  \def\childdocjob{#2}
  \input{#2}
  \endinput
}
%    \end{macrocode}

% \macro{\childdocforward}
% The command |\childdocforward| redirects
% compilation to the main file or
% (if the optional argument is given) a child file.
% Parameters are set as if the main file
% or a child file starting with |\childdocof| was compiled.
% Then compilation is handed over to the main file:
%    \begin{macrocode}
\newcommand{\childdocforward}[2][]
{
  \begingroup
    \if?#1?
      \def\childdoctmp
      {
        \def\childdocname{#2}
        \def\childdocjob{#2}
        \def\jobname{#2}
        \input{#2}
        \endinput
      }
    \else
      \def\childdoctmp
      {
        \childdocdisable
        \def\childdocname{#2}
        \childdoctrue
        \includeonly{#2}
        \def\childdocjob{#1}
        \def\jobname{#1}
        \input{#1}
        \endinput
      }
    \fi
    \expandafter
  \endgroup
  \childdoctmp
}
%    \end{macrocode}

% \macro{\childdocforwardprefix}
% The command |\childdocforwardprefix| redirects
% compilation to the main or a child file by means of a pattern.
% The prefix |#1| in the current filename is replaced by |#2|
% and the suffix of the current filename is kept
% (it is assumed that the filename does not contain the substring `|~~~|'
% which is used as a delimiter).
% Compilation is handed over to the new file by |\childdocforward|:
%    \begin{macrocode}
\newcommand{\childdocforwardprefix}[3][]
{
  \begingroup
    \def\childdocextract #2##1~~~{\def\childdoctmp{\childdocforward[#1]{#3##1}}}
    \expandafter\childdocextract\childdocname~~~
    \expandafter
  \endgroup
  \childdoctmp
}
%    \end{macrocode}

% \macro{\childdoc}
% The deprecated macro |\childdoc| is a legacy version of |\childdocmain|:
%    \begin{macrocode}
\newcommand{\childdoc}{\childdocmain}
%    \end{macrocode}

% \macro{\childdocredirect}
% The deprecated macro |\childdocredirect| is a legacy version
% of |\childdocforward| and |\childdocforwardprefix|:
%    \begin{macrocode}
\newcommand{\childdocredirect}[2][]
{
  \begingroup
    \if?#1?
      \def\childdoctmp{\childdocforward{#2}}
    \else
      \def\childdoctmp{\childdocforwardprefix{#1}{#2}}
    \fi
    \expandafter
  \endgroup
  \childdoctmp
}
%    \end{macrocode}

%\iffalse
%</package>
%\fi
%
\endinput
|\\
|\childdocforward{|\textit{main}|}|\\
\end{tabular}
\end{center}
%
or alternatively with:
%
\begin{center}
\begin{tabular}{l}
|% \iffalse
%
% childdoc.dtx Copyright (C) 2017-2018 Niklas Beisert
%
% This work may be distributed and/or modified under the
% conditions of the LaTeX Project Public License, either version 1.3
% of this license or (at your option) any later version.
% The latest version of this license is in
%   http://www.latex-project.org/lppl.txt
% and version 1.3 or later is part of all distributions of LaTeX
% version 2005/12/01 or later.
%
% This work has the LPPL maintenance status `maintained'.
%
% The Current Maintainer of this work is Niklas Beisert.
%
% This work consists of the files childdoc.dtx and childdoc.ins
% and the derived files childdoc.def and cdocsamp.tex with
% cdocsch1.tex, cdocsch2.tex, cdocsdrf.tex, cdocsfn1.tex, cdocsfn2.tex.
%
%<package>\ifdefined\childdocmain\endinput\fi
%<package>\ProvidesFile{childdoc.def}[2018/12/30 v2.0 child document driver]
%<samplemain>\ProvidesFile{cdocsamp.tex}[2018/12/30 v2.0 sample for childdoc]
%<*driver>
%\ProvidesFile{childdoc.drv}[2018/12/30 v2.0 childdoc reference manual file]
\PassOptionsToClass{10pt,a4paper}{article}
\documentclass{ltxdoc}

\usepackage[margin=35mm]{geometry}
\usepackage{hyperref}
\usepackage{hyperxmp}
\usepackage[usenames]{color}

\hypersetup{colorlinks=true}
\hypersetup{pdfstartview=FitH}
\hypersetup{pdfpagemode=UseNone}
\hypersetup{pdfsource={}}
\hypersetup{pdflang={en-UK}}
\hypersetup{pdfcopyright={Copyright 2017-2018 Niklas Beisert.
  This work may be distributed and/or modified under the
  conditions of the LaTeX Project Public License, either version 1.3
  of this license or (at your option) any later version.}}
\hypersetup{pdflicenseurl={http://www.latex-project.org/lppl.txt}}
\hypersetup{pdfcontactaddress={ETH Zurich, ITP, HIT K,
  Wolfgang-Pauli-Strasse 27}}
\hypersetup{pdfcontactpostcode={8093}}
\hypersetup{pdfcontactcity={Zurich}}
\hypersetup{pdfcontactcountry={Switzerland}}
\hypersetup{pdfcontactemail={nbeisert@itp.phys.ethz.ch}}
\hypersetup{pdfcontacturl={http://people.phys.ethz.ch/\xmptilde nbeisert/}}

\newcommand{\secref}[1]{\hyperref[#1]{section \ref*{#1}}}

\parskip1ex
\parindent0pt
\let\olditemize\itemize
\def\itemize{\olditemize\parskip0pt}

\begin{document}

\title{The \textsf{childdoc} Package}
\hypersetup{pdftitle={The childdoc Package}}
\author{Niklas Beisert\\[2ex]
  Institut f\"ur Theoretische Physik\\
  Eidgen\"ossische Technische Hochschule Z\"urich\\
  Wolfgang-Pauli-Strasse 27, 8093 Z\"urich, Switzerland\\[1ex]
  \href{mailto:nbeisert@itp.phys.ethz.ch}
  {\texttt{nbeisert@itp.phys.ethz.ch}}}
\hypersetup{pdfauthor={Niklas Beisert}}
\hypersetup{pdfsubject={Manual for the LaTeX2e Package childdoc}}
\date{30 December 2018, \textsf{v2.0}}
\maketitle

\begin{abstract}\noindent
\textsf{childdoc} is a \LaTeXe{} package
that enables the direct compilation
of document sections included by |\include|
to individual files.
\end{abstract}

\begingroup
\parskip0ex
\tableofcontents
\endgroup

%%%%%%%%%%%%%%%%%%%%%%%%%%%%%%%%%%%%%%%%%%%%%%%%%%%%%%%%%%%%%%%%%%%%%%%%%%%%%%%%
%%%%%%%%%%%%%%%%%%%%%%%%%%%%%%%%%%%%%%%%%%%%%%%%%%%%%%%%%%%%%%%%%%%%%%%%%%%%%%%%
\section{Introduction}

\LaTeX{} provides a mechanism to structure a large document (such as a book)
into a main file and several child files (containing the chapters)
using the |\include| command.
This mechanism is beneficial for documents
which span hundreds of pages in order to
make the source file(s) more manageable.
Moreover, compilation can be restricted to
selected child files by means of the |\includeonly| command.
The latter feature can be used to reduce the compilation time while editing
(this was significantly more useful in the earlier days of \LaTeX{})
or to generate a smaller document which is easier to navigate.
Another application of |\includeonly| is to generate
documents consisting of selected parts of the complete document.

However, there are a few drawbacks of the plain |\include| mechanism:
\begin{itemize}
\item
The child files cannot be compiled on their own,
they can only be compiled via the main file.
A naive editing environment
(such as a text editor with an option
to have the current file processed by \LaTeX)
may require one to switch to the main file before compiling;
attempting to compile the child file produces errors.
\item
The main file must be modified (each time)
to adjust the |\includeonly| command
to the present needs. This easily leaves the main file in a messy state.
\item
The generated document will always carry the filename
of the main document. This is inconvenient if
several child files are to be compiled and
to be kept for distribution.
\end{itemize}

The present package provides a simple interface
to make child files individually compilable by \LaTeX{}.
Compiling a child file then has the same effect as compiling
the main file with an |\includeonly| command
to select the appropriate child.
Moreover the generated document will carry the name of the child
rather than the main file.
This resolves all three above issues.

This feature is meant to make the editing of books,
thesis documents and lecture notes somewhat more convenient.
However, the package can also be used efficiently for
composing a series of documents (such as exercise sheets)
which are typically distributed individually.
It then assists the author in generating the individual documents
(potentially in different versions)
as well as a document containing the collected series.
Another application is in developing style files
or other kinds of included material
where compilation of the style file could redirect
to a sample or test file.

%%%%%%%%%%%%%%%%%%%%%%%%%%%%%%%%%%%%%%%%%%%%%%%%%%%%%%%%%%%%%%%%%%%%%%%%%%%%%%%%
%%%%%%%%%%%%%%%%%%%%%%%%%%%%%%%%%%%%%%%%%%%%%%%%%%%%%%%%%%%%%%%%%%%%%%%%%%%%%%%%
\section{Usage}

First of all, the package \textsf{childdoc} is \emph{not} a standard
\LaTeXe{} |.sty| style file! Therefore it needs to be invoked in
a non-standard way.

%%%%%%%%%%%%%%%%%%%%%%%%%%%%%%%%%%%%%%%%%%%%%%%%%%%%%%%%%%%%%%%%%%%%%%%%%%%%%%%%
\subsection{Included Files}
\label{sec:include}

%%%%%%%%%%%%%%%%%%%%%%%%%%%%%%%%%%%%%%%%
\DescribeMacro{\childdocmain}
To use the package, add the commands
\begin{center}
\begin{tabular}{l}
|\input{childdoc.def}|\\
|\childdocmain{}|\\
\end{tabular}
\end{center}
at the very top of the main \LaTeX{} file,
in particular \emph{before} the |\documentclass| statement!
The argument of |\childdocmain| should be left empty
(but it must be present).

%%%%%%%%%%%%%%%%%%%%%%%%%%%%%%%%%%%%%%%%
\DescribeMacro{\childdocof}
Furthermore, add the commands
\begin{center}
\begin{tabular}{l}
|\input{childdoc.def}|\\
|\childdocof{|\textit{main}|}|\\
\end{tabular}
\end{center}
at the top of every child file \textit{child}
which is included by |\include{|\textit{child}|}|
from within the main file
(or at least for those files to be compiled individually).
The argument \textit{main} must be the filename of the main file.

There are a couple of
considerations in setting up the main and child documents:

%%%%%%%%%%%%%%%%%%%%%%%%%%%%%%%%%%%%%%%%
\paragraph{Restrictions.}

Please note the following restrictions:
\begin{itemize}
\item
|\childdocmain| must be called with one argument \textit{main}
to ensure compatibility with earlier version of the package.
It must either be empty (|\childdocmain{}|)
or precisely match the filename of the main file in which it is specified.
See \secref{sec:detection} for further information.
\item
The filename \textit{main} must be specified without the |.tex| extension.
\item
The filename \textit{main} is case sensitive
(even in case-insensitive file systems)
due to internal string comparison.
\item
The argument \textit{main} should be fully expanded, it cannot be a macro.
\item
Subdirectories and special characters should be avoided in filenames.
\item
The command |\childdocmain{|\textit{main}|}| must be followed by a whitespace.
It should not be followed immediately by another command
or by a comment mark `|%|'.
This is because the \TeX{} parser reads the token immediately following
the argument of |\childdocmain| and puts it
at the beginning of every child section;
however, a white\-space is ignored.
\end{itemize}

%%%%%%%%%%%%%%%%%%%%%%%%%%%%%%%%%%%%%%%%
\paragraph{Content of Main File.}

It is advisable to place all content in the child files included by |\include|.
Any output contained in the main file will appear in all child documents
unless suppressed manually;
it cannot be suppressed automatically by the |\includeonly| directive
and thus should normally be avoided.
A method to include some content in the main file
by means of conditional processing is described in \secref{sec:conditional}.

%%%%%%%%%%%%%%%%%%%%%%%%%%%%%%%%%%%%%%%%
\paragraph{Page Numbering.}

When only a part of the document is compiled,
the appropriate numbering of pages
(as well as other status parameters)
is determined from the |.aux| files.
The latter contain information from previous passes.
However this information needs to propagate through
all intermediate child documents.
Therefore the page numbering in child documents may well
be inconsistent until the complete document is compiled at least once.

A useful (if unconventional) way to always ensure a consistent
page numbering is to restart the numbering in each child document
and denote the pages by `\textit{child}|.|\textit{page}'
where \textit{child} represents the chapter/section number of the child file.
This can be achieved by the command
|\numberwithin{page}{|\textit{child}|}|
of the \textsf{amsmath} package
where \textit{child} can be |chapter| or |section|
depending on the chosen structuring.
Alternatively, one can modify the macro |\thepage| appropriately
and reset the counter |page| at the start of each child file.

%%%%%%%%%%%%%%%%%%%%%%%%%%%%%%%%%%%%%%%%%%%%%%%%%%%%%%%%%%%%%%%%%%%%%%%%%%%%%%%%
\subsection{Conditional Processing}
\label{sec:conditional}

The package provides a mechanism to compile different versions
of a document. To customise the versions further some conditional processing
can come in handy to distinguish which version is being compiled.
The package provides two macros to describe the compilation context:

%%%%%%%%%%%%%%%%%%%%%%%%%%%%%%%%%%%%%%%%
\DescribeMacro{\ifchilddoc}
The conditional |\ifchilddoc| distinguishes between the compilation of
child documents and the main document:
%
\begin{center}
|\ifchilddoc |\textit{child-code}| |[|\||else |\textit{main-code}]| \||fi|
\end{center}

%%%%%%%%%%%%%%%%%%%%%%%%%%%%%%%%%%%%%%%%
\DescribeMacro{\childdocname}
\DescribeMacro{\childdocjob}
The macro |\childdocname| contains the filename (without extension)
of the main or child file being processed.
Note that |\childdocjob| will always contain the name of the main file.

%%%%%%%%%%%%%%%%%%%%%%%%%%%%%%%%%%%%%%%%
\paragraph{Title Page.}

Conditional processing can be used to include a title or banner page
in the main document when proper precautions are taken.
Importantly, the code in the main file should ensure that the page counter
(as well as other status parameters which are stored in the |.aux| files)
takes the same value after the conditional processing.
Otherwise the page numbers may take divergent values
depending on which part is compiled.

For example, a title page could be declared by:
%
\begin{center}
\begin{tabular}{l}
|\ifchilddoc\||else|\\
|\addtocounter{page}{-1}|\\
\textit{code for title page}\\
|\newpage|\\
|\||fi|
\end{tabular}
\end{center}
%
A banner page for the child documents can be generated by:
%
\begin{center}
\begin{tabular}{l}
|\ifchilddoc|\\
|\addtocounter{page}{-1}|\\
\textit{code for banner page}\\
|\newpage|\\
|\||fi|
\end{tabular}
\end{center}
%
Here one could write a message such as:
\begin{center}
|This is the part \childdocname{} of \childdocjob{}.|
\end{center}

%%%%%%%%%%%%%%%%%%%%%%%%%%%%%%%%%%%%%%%%%%%%%%%%%%%%%%%%%%%%%%%%%%%%%%%%%%%%%%%%
\subsection{Flags}
\label{sec:flags}

The package makes it easy to generate different versions
of the main or child documents.
To this end compilation flags can be defined
and assigned different default values.
They will be particularly useful in conjunction
with the forwarding mechanism described in \secref{sec:forward}.

For example, it may be useful to have a flag |\version|
which can be set to |draft| or |final|.
The document source will contain some conditional code
depending on the value of |\version|.
Suppose further, the flag should default to |final| for the main file
and to |draft| for child files
which is a natural assignment for editing the document.
This is achieved by placing the following code
in the preamble of the main document
(below the |\childdocmain| directive):
%
\begin{center}
\begin{tabular}{l}
|\ifchilddoc|\\
|\providecommand{\version}{draft}|\\
|\||else|\\
|\providecommand{\version}{final}|\\
|\||fi|
\end{tabular}
\end{center}
%
The definition by |\providecommand| makes sure
that previous definitions are not overwritten.
Further statements |\providecommand{\version}{...}|
can thus be added before the above code to override it.

For the main file, one might add a line
(between |\childdocmain| and the above block)
%
\begin{center}
|%\ifchilddoc\||else\providecommand{\version}{draft}\||fi|
\end{center}
%
which can be uncommented to produce a draft version.
Likewise one can add a line to the very top of a child file
(above the |\childdocof{|\textit{main}|}| directive)
%
\begin{center}
|%\providecommand{\version}{final}|
\end{center}
%
which can be uncommented to produce the final version of this child document.

%%%%%%%%%%%%%%%%%%%%%%%%%%%%%%%%%%%%%%%%%%%%%%%%%%%%%%%%%%%%%%%%%%%%%%%%%%%%%%%%
\subsection{Forwarding}
\label{sec:forward}

Different versions of the main or child documents
using compilation flags as described in \secref{sec:flags}
can be (permanently) stored in different files
for convenient compilation, viewing and distribution.
To this end, the package defines a command
to pass on compilation to a different file:

%%%%%%%%%%%%%%%%%%%%%%%%%%%%%%%%%%%%%%%%
\DescribeMacro{\childdocforward}
The command |\childdocforward| redirects processing to
another source file:
%
\begin{center}
\begin{tabular}{l}
|\input{childdoc.def}|\\
|\childdocforward[|\textit{main}|]{|\textit{dest}|}|\\
\end{tabular}
\end{center}
%
The argument \textit{dest} is the destination file
(without extension).
It should be the main file or one of the child files.
Note that further \textsf{childdoc} directives
such as |\childdocof| and |\childdocforward|
in the indicated file will be processed in this form.
The optional argument \textit{main}
passes on directly to the main file \textit{main}
while pretending to compile the child \textit{dest}.
This form behaves as if \textit{dest}
issues |\childdocof{|\textit{main}|}| right away,
and no further \textsf{childdoc} directives will be processed.

%%%%%%%%%%%%%%%%%%%%%%%%%%%%%%%%%%%%%%%%
\DescribeMacro{\...prefix}
In the alternative form |\childdocforwardprefix|,
%
\begin{center}
\begin{tabular}{l}
|\input{childdoc.def}|\\
|\childdocforwardprefix[|\textit{main}|]{|\textit{prefix}|}{|\textit{dest}|}|
\end{tabular}
\end{center}
%
the destination file is determined by a pattern
depending on the current file:
To make this work, the current file must be called
`{\textit{prefix}\hspace{0.2em}\textit{suffix}}'
with \textit{prefix} matching precisely the argument.
Processing is then passed on to the file
`{\textit{dest}\hspace{0.2em}\textit{suffix}}'.
Surely, the same effect is achieved by
directly specifying the
argument `{\textit{dest}\hspace{0.2em}\textit{suffix}}'
in the first form.
However, that requires to set up a different file
for each child. With the alternative form of the command
all these files can have exactly the same content
which simplifies setting them up and maintaining them.

For example, the following file |draft.tex|
with a compilation flag |\version| as described in \secref{sec:flags}
compiles the main document as a draft:
%
\begin{center}
\begin{tabular}{l}
|\def\version{draft}|\\
|\input{childdoc.def}|\\
|\childdocforward{|\textit{main}|}|
\end{tabular}
\end{center}
%
Likewise, the following files |final|\textit{nn}|.tex|
compile the final version of the child document
|child|\textit{nn}|.tex|:
%
\begin{center}
\begin{tabular}{l}
|\def\version{final}|\\
|\input{childdoc.def}|\\
|\childdocforwardprefix{final}{child}|
\end{tabular}
\end{center}
%

Note that when several versions of a main file and/or of each child file
are to be generated, it may be convenient to set up a |Makefile| or
shell script to automatise the process.

%%%%%%%%%%%%%%%%%%%%%%%%%%%%%%%%%%%%%%%%%%%%%%%%%%%%%%%%%%%%%%%%%%%%%%%%%%%%%%%%
\subsection{Command Line Processing}
\label{sec:commandline}

The effect of redirection files can also be achieved by invoking
the \LaTeX{} compiler with a more elaborate command line.
Most conveniently this should be done as part
of a shell script or a |Makefile|.

When using \textsf{childdoc} in the main file, the following
command lines effectively perform a redirection
(note that depending on the shell being used,
backslashes may have to be doubled: `|\|' $\to$ `|\\|'):
%
\begin{center}
|... -jobname "|\textit{target}|" |\\|"|[\textit{flags}]%
|\input{childdoc.def}\childdocforward[|\textit{main}|]{|\textit{dest}|}"|
\end{center}
%
Here \textit{target} is the name of the output file,
\textit{main} is the name of the main file
and \textit{dest} is the name of the main or child file to be processed
(all filenames without extensions).
The optional argument \textit{main} can be omitted
if \textit{main} matches \textit{dest}.
Optionally, compilation \textit{flags} can be defined via |\def| commands.
This command line makes the \TeX{} engine believe
it is compiling the file \textit{target}
whose content is specified as the latter parameter.
The provided code then forwards the processing to
\textit{main} or \textit{dest} as described in \secref{sec:forward}.

%%%%%%%%%%%%%%%%%%%%%%%%%%%%%%%%%%%%%%%%%%%%%%%%%%%%%%%%%%%%%%%%%%%%%%%%%%%%%%%%
\subsection{Include by Input}
\label{sec:input}

Including child documents by |\include| has some restrictions by design.
Most notably, the content of a child document always occupies
its own set of pages; pages cannot be shared between child documents.
Usually, this behaviour makes perfect sense
because each child document contain an essential part of the document.
However, in some situations it may be desirable to compose
a document from a collection of parts
without having mandatory page breaks between then.
For this case, the package
provides a mechanism to include parts
by |\input| which can also be processed individually.
However, by construction this mechanism
requires manual handling of the content to be output.

%%%%%%%%%%%%%%%%%%%%%%%%%%%%%%%%%%%%%%%%
\DescribeMacro{\ifchilddocmanual}
The main file should be prepared as usual, see \secref{sec:include}.
However, the document body must make a distinction
between processing of an individual part and of the main document, e.g.:
%
\begin{center}
\begin{tabular}{l}
|\ifchilddocmanual|\\
|\input{\childdocname}|\\
|\||else|\\
\textit{document body with }|\input{|\textit{part}|}|\\
|\||fi|
\end{tabular}
\end{center}
%
The conditional |\ifchilddocmanual| is true whenever
a part to be included by |\input| is being compiled,
and the name of the part is stored in |\childdocname|.

%%%%%%%%%%%%%%%%%%%%%%%%%%%%%%%%%%%%%%%%
\DescribeMacro{\childdocby}
Each part to be included by |\input| should start with:
%
\begin{center}
\begin{tabular}{l}
|\input{childdoc.def}|\\
|\childdocby{|\textit{main}|}|\\
\end{tabular}
\end{center}
%
The directive |\childdocby| is similar to |\childdocof|
described in \secref{sec:include},
but the subsequent selection of content must be done manually.
To that end, both |\ifchilddoc| and |\ifchilddocmanual|
will be true upon processing of a part,
and the name of the part is stored in |\childdocname|.
Note that |\jobname| will be set to the filename of the current part
so that each part receives an individual |.aux| file
that does not interfere with the |.aux| file(s) of the main document.
This behaviour can be altered by the alternative form
|\childdocby[*]{|\textit{main}|}| (with a non-empty optional argument)
which uses the |.aux| file of the main document
by setting |\jobname| to \textit{main}.

%%%%%%%%%%%%%%%%%%%%%%%%%%%%%%%%%%%%%%%%%%%%%%%%%%%%%%%%%%%%%%%%%%%%%%%%%%%%%%%%
\subsection{Driver Development}
\label{sec:driver}

The \textsf{childdoc} mechanism can also be use for the development
of definition files such as \LaTeX{} styles or classes.
This case differs from the above setup with multiple parts
included by |\include| in that no |\includeonly| should be invoked.
This can be achieved by starting the include file
(before |\ProvidesPackage|) with:
%
\begin{center}
\begin{tabular}{l}
|\input{childdoc.def}|\\
|\childdocforward{|\textit{main}|}|\\
\end{tabular}
\end{center}
%
or alternatively with:
%
\begin{center}
\begin{tabular}{l}
|\input{childdoc.def}|\\
|\childdocby{|\textit{main}|}|\\
\end{tabular}
\end{center}
%
Both forms have slightly different effects as described above.
The main file is prepared as usual, see \secref{sec:include}.

%%%%%%%%%%%%%%%%%%%%%%%%%%%%%%%%%%%%%%%%%%%%%%%%%%%%%%%%%%%%%%%%%%%%%%%%%%%%%%%%
\subsection{Legacy Detection}
\label{sec:detection}

The directive |\childdocmain| in the main file can detect
whether the complete document or merely a child is to be compiled
even without using the directive |\childdocof|.
This method is deprecated because it is less robust
and there is no compelling reason to use it;
it is merely provided for backward compatibility
and it may be removed in future versions.

If the detection mechanism is to be used,
it is mandatory to correctly specify
the filename of the main file as the argument of |\childdocmain|:
%
\begin{center}
\begin{tabular}{l}
|\input{childdoc.def}|\\
|\childdocmain{|\textit{main}|}|\\
\end{tabular}
\end{center}
%
If |\jobname| does not match the argument \textit{main} of |\childdocmain|,
it is assumed that |\jobname| points to the child file to be compiled.
When using |\childdocmain| with the main file specified as argument,
it suffices to start a child file
with just |\input{|\textit{main}|}|
without loading of the package and using |\childdocof|.
If instead all processing is done
with the appropriate \textsf{childdoc} directives,
the argument of \textit{main} of |\childdocmain| can be empty.

An alternative version of the command line processing described
in \secref{sec:commandline} using the detection mechanism reads:
%
\begin{center}
|... -jobname "|\textit{target}|" "|[\textit{flags}]%
[|\def\jobname{|\textit{dest}|}|]|\input{|\textit{main}|}"|
\end{center}

%%%%%%%%%%%%%%%%%%%%%%%%%%%%%%%%%%%%%%%%%%%%%%%%%%%%%%%%%%%%%%%%%%%%%%%%%%%%%%%%
\subsection{Manual Code}
\label{sec:manual}

In case one cannot be certain whether the definitions file |childdoc.def|
is installed on the target \TeX{} distribution
and one prefers not to ship it,
it is conceivable to paste a few relevant commands into the sources.

To that end, drop all statements |\input{childdoc.def}|
and perform the replacements as outlined below.
Instead of |\childdocmain{|\textit{main}|}| add the following code
to the top of the main file:
%
\begin{center}
\begin{tabular}{l}
|\||ifdefined\childdocname\endinput\||fi\newif\ifchilddoc|\\
|\edef\childdocname{\scantokens\expandafter{\jobname\noexpand}}|\\
|\def\childdocmain{|\textit{main}|}\||ifx\childdocmain\childdocname\||else|\\
|\childdoctrue\includeonly{\childdocname}\let\jobname\childdocmain\||fi|\\
\end{tabular}
\end{center}
%
Instead of |\childdocof{|\textit{main}|}| just include the main file
at the top of each child file:
%
\begin{center}
|\input{|\textit{main}|}|
\end{center}
%
A simple redirection |\childdocforward{|\textit{dest}|}| is achieved by:
%
\begin{center}
|\def\jobname{|\textit{dest}|}\input{\jobname}|
\end{center}
%
The redirection with prefix
|\childdocforwardprefix[|\textit{prefix}|]{|\textit{dest}|}|
is accomplished by:
%
\begin{center}
\begin{tabular}{l}
|{\edef\jobname{\scantokens\expandafter{\jobname\noexpand}}|\\
|\def\redirectjob |\textit{prefix}|#1~~~{\gdef\jobname{|\textit{dest}|#1}}|\\
|\expandafter\redirectjob\jobname~~~}\input{\jobname}|
\end{tabular}
\end{center}

In an alternative approach,
child documents can be compiled by a specific command line
without additional code or specific definitions:
%
\begin{center}
|... -jobname "|\textit{target}|" "|[\textit{flags}]%
|\includeonly{|\textit{dest}|}\input{|\textit{main}|}"|
\end{center}
%

%%%%%%%%%%%%%%%%%%%%%%%%%%%%%%%%%%%%%%%%%%%%%%%%%%%%%%%%%%%%%%%%%%%%%%%%%%%%%%%%
%%%%%%%%%%%%%%%%%%%%%%%%%%%%%%%%%%%%%%%%%%%%%%%%%%%%%%%%%%%%%%%%%%%%%%%%%%%%%%%%
\section{Information}

%%%%%%%%%%%%%%%%%%%%%%%%%%%%%%%%%%%%%%%%%%%%%%%%%%%%%%%%%%%%%%%%%%%%%%%%%%%%%%%%
\subsection{Copyright}

Copyright \copyright{} 2017--2018 Niklas Beisert

This work may be distributed and/or modified under the
conditions of the \LaTeX{} Project Public License, either version 1.3
of this license or (at your option) any later version.
The latest version of this license is in
  \url{http://www.latex-project.org/lppl.txt}
and version 1.3 or later is part of all distributions of \LaTeX{}
version 2005/12/01 or later.

This work has the LPPL maintenance status `maintained'.

The Current Maintainer of this work is Niklas Beisert.

This work consists of the files |README.txt|, |childdoc.ins| and |childdoc.dtx|
as well as the derived files |childdoc.def|, |cdocsamp.tex|
with |cdocsch1.tex|, |cdocsch2.tex|, |cdocspt3.tex|, |cdocspt4.tex|,
|cdocsdrf.tex|, |cdocsfn1.tex|, |cdocsfn2.tex|
as well as |childdoc.pdf|.

%%%%%%%%%%%%%%%%%%%%%%%%%%%%%%%%%%%%%%%%%%%%%%%%%%%%%%%%%%%%%%%%%%%%%%%%%%%%%%%%
\subsection{Files and Installation}

The package consists of the files:
%
\begin{center}
\begin{tabular}{ll}
    |README.txt|   & readme file \\
    |childdoc.ins| & installation file \\
    |childdoc.dtx| & source file \\
    |childdoc.def| & definition file \\
    |cdocsamp.tex| & sample main file \\
    |cdocsch1.tex| & sample include file \\
    |cdocsch2.tex| & sample include file \\
    |cdocspt3.tex| & sample part file \\
    |cdocspt4.tex| & sample part file \\
    |cdocsdrf.tex| & sample redirection file \\
    |cdocsfn1.tex| & sample redirection file \\
    |cdocsfn2.tex| & sample redirection file \\
    |childdoc.pdf| & manual
\end{tabular}
\end{center}
%
The distribution consists of the files
|README.txt|, |childdoc.ins| and |childdoc.dtx|.
%
\begin{itemize}
\item
Run (pdf)\LaTeX{} on |childdoc.dtx|
to compile the manual |childdoc.pdf| (this file).
\item
Run \LaTeX{} on |childdoc.ins| to create the definitions file |childdoc.def|
and the sample |cdocsamp.tex| with include files
|cdocsch1.tex|, |cdocsch2.tex|, |cdocspt3.tex|, |cdocspt4.tex|,
|cdocsdrf.tex|, |cdocsfn1.tex|, |cdocsfn2.tex|.
Then copy the file |childdoc.def| to an appropriate directory of your \LaTeX{}
distribution, e.g.\ \textit{texmf-root}|/tex/latex/childdoc|.
\end{itemize}

%%%%%%%%%%%%%%%%%%%%%%%%%%%%%%%%%%%%%%%%%%%%%%%%%%%%%%%%%%%%%%%%%%%%%%%%%%%%%%%%
\subsection{Related CTAN Packages}

There are several other packages which offer a similar functionality:
%
\begin{itemize}
\item
The packages
\href{http://ctan.org/pkg/docmute}{\textsf{docmute}},
\href{http://ctan.org/pkg/includex}{\textsf{includex}} and
\href{http://ctan.org/pkg/standalone}{\textsf{standalone}}
provide commands to include only the document body of
a child file thus allowing both files to be compiled individually.
\item
The packages \href{http://ctan.org/pkg/subdocs}{\textsf{subdocs}}
and \href{http://ctan.org/pkg/subfiles}{\textsf{subfiles}}
provide structures in which the main and child documents can be
encapsulated and allowing them to be compiled individually.
The inclusion mechanism is different from the conventional |\include|.
\item
The package \href{http://ctan.org/pkg/combine}{\textsf{combine}}
is an elaborate solution to combine several documents into one.
\end{itemize}
%
See also the CTAN topic \href{http://ctan.org/topic/subdocs}{\textsf{subdocs}}
for further related packages.
The present package differs from the above solutions in that
a document structure constructed with the conventional |\include| mechanism
just needs two extra commands at the top of every file
such that all constituent files can be compiled individually.

%%%%%%%%%%%%%%%%%%%%%%%%%%%%%%%%%%%%%%%%%%%%%%%%%%%%%%%%%%%%%%%%%%%%%%%%%%%%%%%%
%\subsection{Feature Suggestions}
%
%The following is a list of features which may be useful for future
%versions of this package:
%%
%\begin{itemize}
%\item
%\ldots
%\end{itemize}

%%%%%%%%%%%%%%%%%%%%%%%%%%%%%%%%%%%%%%%%%%%%%%%%%%%%%%%%%%%%%%%%%%%%%%%%%%%%%%%%
\subsection{Revision History}

%%%%%%%%%%%%%%%%%%%%%%%%%%%%%%%%%%%%%%%%
\paragraph{v2.0:} 2018/12/30

\begin{itemize}
\item
immediate forward processing
\item
added |\childdocby| mechanism
\item
manual restructured
\end{itemize}

%%%%%%%%%%%%%%%%%%%%%%%%%%%%%%%%%%%%%%%%
\paragraph{v1.6:} 2018/01/17

\begin{itemize}
\item
application for development of include files
\item
corrections to manual
\end{itemize}

%%%%%%%%%%%%%%%%%%%%%%%%%%%%%%%%%%%%%%%%
\paragraph{v1.5:} 2017/05/21

\begin{itemize}
\item
more complete structuring introduced
\item
|\childdocof| introduced
\item
|\childdoc| renamed to |\childdocmain|
\item
|\childredirect| renamed to |\childdocforward| and |\childdocforwardprefix|
and functionality expanded
\end{itemize}

%%%%%%%%%%%%%%%%%%%%%%%%%%%%%%%%%%%%%%%%
\paragraph{v1.0:} 2017/04/27

\begin{itemize}
\item
manual and install package
\item
first version published on CTAN
\end{itemize}

%%%%%%%%%%%%%%%%%%%%%%%%%%%%%%%%%%%%%%%%
\paragraph{v0.6:} 2017/04/26

\begin{itemize}
\item
redirection mechanism added
\end{itemize}

%%%%%%%%%%%%%%%%%%%%%%%%%%%%%%%%%%%%%%%%
\paragraph{v0.5:} 2017/04/26

\begin{itemize}
\item
functionality in definition file
\end{itemize}


%%%%%%%%%%%%%%%%%%%%%%%%%%%%%%%%%%%%%%%%%%%%%%%%%%%%%%%%%%%%%%%%%%%%%%%%%%%%%%%%
%%%%%%%%%%%%%%%%%%%%%%%%%%%%%%%%%%%%%%%%%%%%%%%%%%%%%%%%%%%%%%%%%%%%%%%%%%%%%%%%
%%%%%%%%%%%%%%%%%%%%%%%%%%%%%%%%%%%%%%%%%%%%%%%%%%%%%%%%%%%%%%%%%%%%%%%%%%%%%%%%
\appendix

\settowidth\MacroIndent{\rmfamily\scriptsize 000\ }

 \DocInput{childdoc.dtx}

\end{document}
%</driver>
% \fi
%
% %%%%%%%%%%%%%%%%%%%%%%%%%%%%%%%%%%%%%%%%%%%%%%%%%%%%%%%%%%%%%%%%%%%%%%%%%%%%%%
% %%%%%%%%%%%%%%%%%%%%%%%%%%%%%%%%%%%%%%%%%%%%%%%%%%%%%%%%%%%%%%%%%%%%%%%%%%%%%%
% \section{Sample}
%\iffalse
%<*samplemain>
%\fi
%
% The following presents a sample document
% with two chapters, two parts, a title page,
% a compile flag as well as three forwarding files to set the flag.
% It consists of eight |.tex| files:
% \begin{center}
% \begin{tabular}{ll}
% |cdocsamp.tex|&main file\\
% |cdocsch1.tex|&include file for chapter 1\\
% |cdocsch2.tex|&include file for chapter 2\\
% |cdocspt3.tex|&include file for part 3\\
% |cdocspt4.tex|&include file for part 4\\
% |cdocsdrf.tex|&forwarding file for main file in draft mode\\
% |cdocsfi1.tex|&forwarding file for final version of chapter 1\\
% |cdocsfi2.tex|&forwarding file for final version of chapter 2\\
% \end{tabular}
% \end{center}
% Each of the eight files can be compiled directly by the \LaTeX{} compiler.
%
% %%%%%%%%%%%%%%%%%%%%%%%%%%%%%%%%%%%%%%
% \paragraph{Main File.}
%
% The main file is called |cdocsamp.tex|.
%
% Load the \textsf{childdoc} definitions and
% declare the filename for the main document:
%    \begin{macrocode}
\input{childdoc.def}
\childdocmain{}
%    \end{macrocode}

% Optional override for |\version| flag:
%    \begin{macrocode}
%%\ifchilddoc\else\providecommand{\version}{draft}\fi
%    \end{macrocode}

% Define the default values for the |\version| flag
% (|final| for the main file and |draft| for childs):
%    \begin{macrocode}
\ifchilddoc
\providecommand{\version}{draft}
\else
\providecommand{\version}{final}
\fi
%    \end{macrocode}

% Load the standard document class:
%    \begin{macrocode}
\documentclass[12pt]{article}
%    \end{macrocode}

% Start the document body:
%    \begin{macrocode}
\begin{document}
%    \end{macrocode}

% Declare a title page.
% Print title, part of document being processed and version flag:
%    \begin{macrocode}
\addtocounter{page}{-1}
\begin{center}
{\LARGE\bfseries{}childdoc example\par}
\vspace{1cm}
\ifchilddoc
\ifchilddocmanual part\else chapter\fi:
`\childdocname' of `\childdocjob'\par
\else
main document: `\childdocjob'\par
\fi
version: \version\par
\end{center}
\newpage
%    \end{macrocode}

% Manually include selected file,
% otherwise process as usual:
%    \begin{macrocode}
\ifchilddocmanual
\section*{part `\childdocname'}
\input{\childdocname}
\else
%    \end{macrocode}

% Include the two chapters:
%    \begin{macrocode}
\include{cdocsch1}
\include{cdocsch2}
%    \end{macrocode}

% Include the two parts unless only chapters should be displayed:
%    \begin{macrocode}
\ifchilddoc\else
\section{part three}
\input{cdocspt3}
\section{part four}
\input{cdocspt4}
\fi
%    \end{macrocode}

% Process as usual until here:
%    \begin{macrocode}
\fi
%    \end{macrocode}

% End of document body:
%    \begin{macrocode}
\end{document}
%    \end{macrocode}
%\iffalse
%</samplemain>
%\fi
%
% %%%%%%%%%%%%%%%%%%%%%%%%%%%%%%%%%%%%%%
% \paragraph{Chapter Include Files.}
%
% The include files are called |cdocsch1.tex| and |cdocsch2.tex|.
%
%\iffalse
%<*samplechap1|samplechap2>
%\fi

% Optional override for |\version| flag:
%    \begin{macrocode}
%%\providecommand{\version}{final}
%    \end{macrocode}

% Include the main document:
%    \begin{macrocode}
\input{childdoc.def}
\childdocof{cdocsamp}
%    \end{macrocode}

%\iffalse
%</samplechap1|samplechap2>
%\fi
%
%\iffalse
%<*samplechap1>
%\fi
% Some text for chapter 1:
%    \begin{macrocode}
\section{one}
some text in chapter one
%    \end{macrocode}

%\iffalse
%</samplechap1>
%\fi
% Some text for chapter 2:
%\iffalse
%<*samplechap2>
%\fi
%    \begin{macrocode}
\section{two}
more text in chapter two
%    \end{macrocode}

%\iffalse
%</samplechap2>
%\fi
%
% %%%%%%%%%%%%%%%%%%%%%%%%%%%%%%%%%%%%%%
% \paragraph{Part Include Files.}
%
% The include files are called |cdocspt3.tex| and |cdocspt4.tex|.
%
%\iffalse
%<*samplepart3|samplepart4>
%\fi

% Optional override for |\version| flag:
%    \begin{macrocode}
%%\providecommand{\version}{final}
%    \end{macrocode}

% Include the main document:
%    \begin{macrocode}
\input{childdoc.def}
\childdocby{cdocsamp}
%    \end{macrocode}

%\iffalse
%</samplepart3|samplepart4>
%\fi
%
%\iffalse
%<*samplepart3>
%\fi
% Some text for part 3:
%    \begin{macrocode}
some text in part three
%    \end{macrocode}

%\iffalse
%</samplepart3>
%\fi
% Some text for part 4:
%\iffalse
%<*samplepart4>
%\fi
%    \begin{macrocode}
more text in part four
%    \end{macrocode}

%\iffalse
%</samplepart4>
%\fi
%
% %%%%%%%%%%%%%%%%%%%%%%%%%%%%%%%%%%%%%%
% \paragraph{Forwarding for a Complete Draft.}
%
% The following forwarding file |cdocsdrf.tex|
% compiles the main document in draft mode:
%\iffalse
%<*sampledraft>
%\fi
%    \begin{macrocode}
\def\version{draft}
\input{childdoc.def}
\childdocforward{cdocsamp}
%    \end{macrocode}

%\iffalse
%</sampledraft>
%\fi
%
% %%%%%%%%%%%%%%%%%%%%%%%%%%%%%%%%%%%%%%
% \paragraph{Forwarding for Final Version of the Chapters.}
%
% The following forwarding files |cdocsfn1.tex| and |cdocsfn2.tex|
% (with identical content)
% compile the final versions of the child documents
% |cdocsch1.tex| and |cdocsch2.tex|, respectively:
%\iffalse
%<*samplefinal>
%\fi
%    \begin{macrocode}
\def\version{final}
\input{childdoc.def}
\childdocforwardprefix[cdocsamp]{cdocsfn}{cdocsch}
%    \end{macrocode}

%\iffalse
%</samplefinal>
%\fi
%
% %%%%%%%%%%%%%%%%%%%%%%%%%%%%%%%%%%%%%%
% \paragraph{Command Line Processing.}
%
% The following three command lines generate the output files
% |cdocscld|, |cdocscl1| and |cdocscl2|
% which should be identical to
% |cdocsdrf|, |cdocsch1| and |cdocsfn2|, respectively:
% \begin{center}
% \begin{tabular}{l}
% |latex -jobname cdocscld \|\\
% |  "\def\version{draft}\input{childdoc.def}\childdocforward{cdocsamp}"|\\
% |latex -jobname cdocscl1 \|\\
% |  "\input{childdoc.def}\childdocforward[cdocsamp]{cdocsch1}"|\\
% |latex -jobname cdocscl2 \|\\
% |  "\def\version{final}\input{childdoc.def}\childdocforward{cdocsch2}"|
% \end{tabular}
% \end{center}
% Note that the trailing backslash on each first line
% merely continues the input to the second line
% (for convenient cut ant paste).
% Furthermore, the command |latex| can be replaced by any
% of its alternative versions such as |pdflatex|.
%
% %%%%%%%%%%%%%%%%%%%%%%%%%%%%%%%%%%%%%%%%%%%%%%%%%%%%%%%%%%%%%%%%%%%%%%%%%%%%%%
% %%%%%%%%%%%%%%%%%%%%%%%%%%%%%%%%%%%%%%%%%%%%%%%%%%%%%%%%%%%%%%%%%%%%%%%%%%%%%%
% \section{Implementation}
%\iffalse
%<*package>
%\fi
%
% This section describes the definitions file |childdoc.def|.

% The definitions cannot be loaded using |\usepackage| or |\RequirePackage|
% which has a mechanism to prevent loading a style file more than once.
% When loading the definitions by means of |\input|
% multiple instances have to be prevented manually:
%\iffalse
%This code needs to be before the `\ProvidesFile' directive
%which is defined at the beginning of this file.
%Therefore it is also placed there and commented out here.
%</package>
%<*discard>
%\fi
%    \begin{macrocode}
\ifdefined\childdocmain\endinput\fi
%    \end{macrocode}
%\iffalse
%</discard>
%<*package>
%\fi
%
% \macro{\ifchilddoc}
% \macro{\ifchilddocmanual}
% The conditional |\ifchilddoc| tells whether a
% child (true) or main (false) document is being compiled.
% The conditional |\ifchilddocmanual| tells whether
% the |\includeonly| mechanism is used (false) or
% the selection of child files must be performed manually (true).
% The definitions initialise to false:
%    \begin{macrocode}
\newif\ifchilddoc
\newif\ifchilddocmanual
%    \end{macrocode}

% \macro{\childdocname}
% \macro{\childdocjob}
% The macro |\childdocname| stores the name of the main document
% to be compiled. The macro |\childdocjob| stores the name of
% the document on which the \LaTeX{} compiler was originally invoked.
% The content of |\jobname| cannot be compared
% to filenames specified in the source due to different catcodes.
% The following code rescans |\jobname|, stores the result
% in |\childdocname| and saves a copy in |\childdocjob|:
%    \begin{macrocode}
\edef\childdocname{\scantokens\expandafter{\jobname\noexpand}}
\let\childdocjob\childdocname
%    \end{macrocode}

% \macro{\childdocdisable}
% The macro |\childdocdisable| prevents the main file
% from being processed more than once.
% At this stage, the main document command |\childdocmain|
% is assumed to be called once again where it should do nothing.
% Any subsequent call to it should prevent
% a secondary processing of the main document
% It overwrites the forwarding commands
% |\childdocof| and |\childdocforward|
% with empty macros to prevent further inclusions of the main document:
%    \begin{macrocode}
\newcommand{\childdocdisable}
{
  \renewcommand{\childdocmain}[1]{\renewcommand{\childdocmain}[1]{\endinput}}
  \renewcommand{\childdocof}[1]{}
  \renewcommand{\childdocby}[2][]{}
  \renewcommand{\childdocforward}[2][]{}
  \renewcommand{\childdocdisable}{}
}
%    \end{macrocode}

% \macro{\childdocmain}
% The macro |\childdocmain| is to be called at the top of the main file
% with nothing or the main filename (without extension) as argument.
% First, it breaks loops.
% If the argument is not empty and does not match |\childdocname|
% (which is set by the first inclusion of |childdoc.def|),
% |\ifchilddoc| is set to true, |\includeonly| is applied to the child file
% and |\jobname| is set to the main file
% (for proper handling of |.aux| files):
%    \begin{macrocode}
\newcommand{\childdocmain}[1]
{
  \childdocdisable\childdocmain{}
  \if?#1?\else
    \begingroup
      \def\childdoctmp{#1}
      \ifx\childdoctmp\childdocname
        \def\childdoctmp{}
      \else
        \def\childdoctmp
        {
          \childdoctrue
          \includeonly{\childdocname}
          \def\childdocjob{#1}
          \def\jobname{#1}
        }
      \fi
      \expandafter
    \endgroup
    \childdoctmp
  \fi
}
%    \end{macrocode}

% \macro{\childdocof}
% The command |\childdocof| redirects
% compilation to the main file |#1|.
%    \begin{macrocode}
\newcommand{\childdocof}[1]
{
  \childdocdisable
  \childdoctrue
  \includeonly{\childdocname}
  \def\jobname{#1}
  \def\childdocjob{#1}
  \input{#1}
}
%    \end{macrocode}

% \macro{\childdocby}
% The command |\childdocby| ....
%    \begin{macrocode}
\newcommand{\childdocby}[2][]
{
  \childdocdisable
  \childdoctrue
  \childdocmanualtrue
  \if?#1?\else
    \def\jobname{#2}
  \fi
  \def\childdocjob{#2}
  \input{#2}
  \endinput
}
%    \end{macrocode}

% \macro{\childdocforward}
% The command |\childdocforward| redirects
% compilation to the main file or
% (if the optional argument is given) a child file.
% Parameters are set as if the main file
% or a child file starting with |\childdocof| was compiled.
% Then compilation is handed over to the main file:
%    \begin{macrocode}
\newcommand{\childdocforward}[2][]
{
  \begingroup
    \if?#1?
      \def\childdoctmp
      {
        \def\childdocname{#2}
        \def\childdocjob{#2}
        \def\jobname{#2}
        \input{#2}
        \endinput
      }
    \else
      \def\childdoctmp
      {
        \childdocdisable
        \def\childdocname{#2}
        \childdoctrue
        \includeonly{#2}
        \def\childdocjob{#1}
        \def\jobname{#1}
        \input{#1}
        \endinput
      }
    \fi
    \expandafter
  \endgroup
  \childdoctmp
}
%    \end{macrocode}

% \macro{\childdocforwardprefix}
% The command |\childdocforwardprefix| redirects
% compilation to the main or a child file by means of a pattern.
% The prefix |#1| in the current filename is replaced by |#2|
% and the suffix of the current filename is kept
% (it is assumed that the filename does not contain the substring `|~~~|'
% which is used as a delimiter).
% Compilation is handed over to the new file by |\childdocforward|:
%    \begin{macrocode}
\newcommand{\childdocforwardprefix}[3][]
{
  \begingroup
    \def\childdocextract #2##1~~~{\def\childdoctmp{\childdocforward[#1]{#3##1}}}
    \expandafter\childdocextract\childdocname~~~
    \expandafter
  \endgroup
  \childdoctmp
}
%    \end{macrocode}

% \macro{\childdoc}
% The deprecated macro |\childdoc| is a legacy version of |\childdocmain|:
%    \begin{macrocode}
\newcommand{\childdoc}{\childdocmain}
%    \end{macrocode}

% \macro{\childdocredirect}
% The deprecated macro |\childdocredirect| is a legacy version
% of |\childdocforward| and |\childdocforwardprefix|:
%    \begin{macrocode}
\newcommand{\childdocredirect}[2][]
{
  \begingroup
    \if?#1?
      \def\childdoctmp{\childdocforward{#2}}
    \else
      \def\childdoctmp{\childdocforwardprefix{#1}{#2}}
    \fi
    \expandafter
  \endgroup
  \childdoctmp
}
%    \end{macrocode}

%\iffalse
%</package>
%\fi
%
\endinput
|\\
|\childdocby{|\textit{main}|}|\\
\end{tabular}
\end{center}
%
Both forms have slightly different effects as described above.
The main file is prepared as usual, see \secref{sec:include}.

%%%%%%%%%%%%%%%%%%%%%%%%%%%%%%%%%%%%%%%%%%%%%%%%%%%%%%%%%%%%%%%%%%%%%%%%%%%%%%%%
\subsection{Legacy Detection}
\label{sec:detection}

The directive |\childdocmain| in the main file can detect
whether the complete document or merely a child is to be compiled
even without using the directive |\childdocof|.
This method is deprecated because it is less robust
and there is no compelling reason to use it;
it is merely provided for backward compatibility
and it may be removed in future versions.

If the detection mechanism is to be used,
it is mandatory to correctly specify
the filename of the main file as the argument of |\childdocmain|:
%
\begin{center}
\begin{tabular}{l}
|% \iffalse
%
% childdoc.dtx Copyright (C) 2017-2018 Niklas Beisert
%
% This work may be distributed and/or modified under the
% conditions of the LaTeX Project Public License, either version 1.3
% of this license or (at your option) any later version.
% The latest version of this license is in
%   http://www.latex-project.org/lppl.txt
% and version 1.3 or later is part of all distributions of LaTeX
% version 2005/12/01 or later.
%
% This work has the LPPL maintenance status `maintained'.
%
% The Current Maintainer of this work is Niklas Beisert.
%
% This work consists of the files childdoc.dtx and childdoc.ins
% and the derived files childdoc.def and cdocsamp.tex with
% cdocsch1.tex, cdocsch2.tex, cdocsdrf.tex, cdocsfn1.tex, cdocsfn2.tex.
%
%<package>\ifdefined\childdocmain\endinput\fi
%<package>\ProvidesFile{childdoc.def}[2018/12/30 v2.0 child document driver]
%<samplemain>\ProvidesFile{cdocsamp.tex}[2018/12/30 v2.0 sample for childdoc]
%<*driver>
%\ProvidesFile{childdoc.drv}[2018/12/30 v2.0 childdoc reference manual file]
\PassOptionsToClass{10pt,a4paper}{article}
\documentclass{ltxdoc}

\usepackage[margin=35mm]{geometry}
\usepackage{hyperref}
\usepackage{hyperxmp}
\usepackage[usenames]{color}

\hypersetup{colorlinks=true}
\hypersetup{pdfstartview=FitH}
\hypersetup{pdfpagemode=UseNone}
\hypersetup{pdfsource={}}
\hypersetup{pdflang={en-UK}}
\hypersetup{pdfcopyright={Copyright 2017-2018 Niklas Beisert.
  This work may be distributed and/or modified under the
  conditions of the LaTeX Project Public License, either version 1.3
  of this license or (at your option) any later version.}}
\hypersetup{pdflicenseurl={http://www.latex-project.org/lppl.txt}}
\hypersetup{pdfcontactaddress={ETH Zurich, ITP, HIT K,
  Wolfgang-Pauli-Strasse 27}}
\hypersetup{pdfcontactpostcode={8093}}
\hypersetup{pdfcontactcity={Zurich}}
\hypersetup{pdfcontactcountry={Switzerland}}
\hypersetup{pdfcontactemail={nbeisert@itp.phys.ethz.ch}}
\hypersetup{pdfcontacturl={http://people.phys.ethz.ch/\xmptilde nbeisert/}}

\newcommand{\secref}[1]{\hyperref[#1]{section \ref*{#1}}}

\parskip1ex
\parindent0pt
\let\olditemize\itemize
\def\itemize{\olditemize\parskip0pt}

\begin{document}

\title{The \textsf{childdoc} Package}
\hypersetup{pdftitle={The childdoc Package}}
\author{Niklas Beisert\\[2ex]
  Institut f\"ur Theoretische Physik\\
  Eidgen\"ossische Technische Hochschule Z\"urich\\
  Wolfgang-Pauli-Strasse 27, 8093 Z\"urich, Switzerland\\[1ex]
  \href{mailto:nbeisert@itp.phys.ethz.ch}
  {\texttt{nbeisert@itp.phys.ethz.ch}}}
\hypersetup{pdfauthor={Niklas Beisert}}
\hypersetup{pdfsubject={Manual for the LaTeX2e Package childdoc}}
\date{30 December 2018, \textsf{v2.0}}
\maketitle

\begin{abstract}\noindent
\textsf{childdoc} is a \LaTeXe{} package
that enables the direct compilation
of document sections included by |\include|
to individual files.
\end{abstract}

\begingroup
\parskip0ex
\tableofcontents
\endgroup

%%%%%%%%%%%%%%%%%%%%%%%%%%%%%%%%%%%%%%%%%%%%%%%%%%%%%%%%%%%%%%%%%%%%%%%%%%%%%%%%
%%%%%%%%%%%%%%%%%%%%%%%%%%%%%%%%%%%%%%%%%%%%%%%%%%%%%%%%%%%%%%%%%%%%%%%%%%%%%%%%
\section{Introduction}

\LaTeX{} provides a mechanism to structure a large document (such as a book)
into a main file and several child files (containing the chapters)
using the |\include| command.
This mechanism is beneficial for documents
which span hundreds of pages in order to
make the source file(s) more manageable.
Moreover, compilation can be restricted to
selected child files by means of the |\includeonly| command.
The latter feature can be used to reduce the compilation time while editing
(this was significantly more useful in the earlier days of \LaTeX{})
or to generate a smaller document which is easier to navigate.
Another application of |\includeonly| is to generate
documents consisting of selected parts of the complete document.

However, there are a few drawbacks of the plain |\include| mechanism:
\begin{itemize}
\item
The child files cannot be compiled on their own,
they can only be compiled via the main file.
A naive editing environment
(such as a text editor with an option
to have the current file processed by \LaTeX)
may require one to switch to the main file before compiling;
attempting to compile the child file produces errors.
\item
The main file must be modified (each time)
to adjust the |\includeonly| command
to the present needs. This easily leaves the main file in a messy state.
\item
The generated document will always carry the filename
of the main document. This is inconvenient if
several child files are to be compiled and
to be kept for distribution.
\end{itemize}

The present package provides a simple interface
to make child files individually compilable by \LaTeX{}.
Compiling a child file then has the same effect as compiling
the main file with an |\includeonly| command
to select the appropriate child.
Moreover the generated document will carry the name of the child
rather than the main file.
This resolves all three above issues.

This feature is meant to make the editing of books,
thesis documents and lecture notes somewhat more convenient.
However, the package can also be used efficiently for
composing a series of documents (such as exercise sheets)
which are typically distributed individually.
It then assists the author in generating the individual documents
(potentially in different versions)
as well as a document containing the collected series.
Another application is in developing style files
or other kinds of included material
where compilation of the style file could redirect
to a sample or test file.

%%%%%%%%%%%%%%%%%%%%%%%%%%%%%%%%%%%%%%%%%%%%%%%%%%%%%%%%%%%%%%%%%%%%%%%%%%%%%%%%
%%%%%%%%%%%%%%%%%%%%%%%%%%%%%%%%%%%%%%%%%%%%%%%%%%%%%%%%%%%%%%%%%%%%%%%%%%%%%%%%
\section{Usage}

First of all, the package \textsf{childdoc} is \emph{not} a standard
\LaTeXe{} |.sty| style file! Therefore it needs to be invoked in
a non-standard way.

%%%%%%%%%%%%%%%%%%%%%%%%%%%%%%%%%%%%%%%%%%%%%%%%%%%%%%%%%%%%%%%%%%%%%%%%%%%%%%%%
\subsection{Included Files}
\label{sec:include}

%%%%%%%%%%%%%%%%%%%%%%%%%%%%%%%%%%%%%%%%
\DescribeMacro{\childdocmain}
To use the package, add the commands
\begin{center}
\begin{tabular}{l}
|\input{childdoc.def}|\\
|\childdocmain{}|\\
\end{tabular}
\end{center}
at the very top of the main \LaTeX{} file,
in particular \emph{before} the |\documentclass| statement!
The argument of |\childdocmain| should be left empty
(but it must be present).

%%%%%%%%%%%%%%%%%%%%%%%%%%%%%%%%%%%%%%%%
\DescribeMacro{\childdocof}
Furthermore, add the commands
\begin{center}
\begin{tabular}{l}
|\input{childdoc.def}|\\
|\childdocof{|\textit{main}|}|\\
\end{tabular}
\end{center}
at the top of every child file \textit{child}
which is included by |\include{|\textit{child}|}|
from within the main file
(or at least for those files to be compiled individually).
The argument \textit{main} must be the filename of the main file.

There are a couple of
considerations in setting up the main and child documents:

%%%%%%%%%%%%%%%%%%%%%%%%%%%%%%%%%%%%%%%%
\paragraph{Restrictions.}

Please note the following restrictions:
\begin{itemize}
\item
|\childdocmain| must be called with one argument \textit{main}
to ensure compatibility with earlier version of the package.
It must either be empty (|\childdocmain{}|)
or precisely match the filename of the main file in which it is specified.
See \secref{sec:detection} for further information.
\item
The filename \textit{main} must be specified without the |.tex| extension.
\item
The filename \textit{main} is case sensitive
(even in case-insensitive file systems)
due to internal string comparison.
\item
The argument \textit{main} should be fully expanded, it cannot be a macro.
\item
Subdirectories and special characters should be avoided in filenames.
\item
The command |\childdocmain{|\textit{main}|}| must be followed by a whitespace.
It should not be followed immediately by another command
or by a comment mark `|%|'.
This is because the \TeX{} parser reads the token immediately following
the argument of |\childdocmain| and puts it
at the beginning of every child section;
however, a white\-space is ignored.
\end{itemize}

%%%%%%%%%%%%%%%%%%%%%%%%%%%%%%%%%%%%%%%%
\paragraph{Content of Main File.}

It is advisable to place all content in the child files included by |\include|.
Any output contained in the main file will appear in all child documents
unless suppressed manually;
it cannot be suppressed automatically by the |\includeonly| directive
and thus should normally be avoided.
A method to include some content in the main file
by means of conditional processing is described in \secref{sec:conditional}.

%%%%%%%%%%%%%%%%%%%%%%%%%%%%%%%%%%%%%%%%
\paragraph{Page Numbering.}

When only a part of the document is compiled,
the appropriate numbering of pages
(as well as other status parameters)
is determined from the |.aux| files.
The latter contain information from previous passes.
However this information needs to propagate through
all intermediate child documents.
Therefore the page numbering in child documents may well
be inconsistent until the complete document is compiled at least once.

A useful (if unconventional) way to always ensure a consistent
page numbering is to restart the numbering in each child document
and denote the pages by `\textit{child}|.|\textit{page}'
where \textit{child} represents the chapter/section number of the child file.
This can be achieved by the command
|\numberwithin{page}{|\textit{child}|}|
of the \textsf{amsmath} package
where \textit{child} can be |chapter| or |section|
depending on the chosen structuring.
Alternatively, one can modify the macro |\thepage| appropriately
and reset the counter |page| at the start of each child file.

%%%%%%%%%%%%%%%%%%%%%%%%%%%%%%%%%%%%%%%%%%%%%%%%%%%%%%%%%%%%%%%%%%%%%%%%%%%%%%%%
\subsection{Conditional Processing}
\label{sec:conditional}

The package provides a mechanism to compile different versions
of a document. To customise the versions further some conditional processing
can come in handy to distinguish which version is being compiled.
The package provides two macros to describe the compilation context:

%%%%%%%%%%%%%%%%%%%%%%%%%%%%%%%%%%%%%%%%
\DescribeMacro{\ifchilddoc}
The conditional |\ifchilddoc| distinguishes between the compilation of
child documents and the main document:
%
\begin{center}
|\ifchilddoc |\textit{child-code}| |[|\||else |\textit{main-code}]| \||fi|
\end{center}

%%%%%%%%%%%%%%%%%%%%%%%%%%%%%%%%%%%%%%%%
\DescribeMacro{\childdocname}
\DescribeMacro{\childdocjob}
The macro |\childdocname| contains the filename (without extension)
of the main or child file being processed.
Note that |\childdocjob| will always contain the name of the main file.

%%%%%%%%%%%%%%%%%%%%%%%%%%%%%%%%%%%%%%%%
\paragraph{Title Page.}

Conditional processing can be used to include a title or banner page
in the main document when proper precautions are taken.
Importantly, the code in the main file should ensure that the page counter
(as well as other status parameters which are stored in the |.aux| files)
takes the same value after the conditional processing.
Otherwise the page numbers may take divergent values
depending on which part is compiled.

For example, a title page could be declared by:
%
\begin{center}
\begin{tabular}{l}
|\ifchilddoc\||else|\\
|\addtocounter{page}{-1}|\\
\textit{code for title page}\\
|\newpage|\\
|\||fi|
\end{tabular}
\end{center}
%
A banner page for the child documents can be generated by:
%
\begin{center}
\begin{tabular}{l}
|\ifchilddoc|\\
|\addtocounter{page}{-1}|\\
\textit{code for banner page}\\
|\newpage|\\
|\||fi|
\end{tabular}
\end{center}
%
Here one could write a message such as:
\begin{center}
|This is the part \childdocname{} of \childdocjob{}.|
\end{center}

%%%%%%%%%%%%%%%%%%%%%%%%%%%%%%%%%%%%%%%%%%%%%%%%%%%%%%%%%%%%%%%%%%%%%%%%%%%%%%%%
\subsection{Flags}
\label{sec:flags}

The package makes it easy to generate different versions
of the main or child documents.
To this end compilation flags can be defined
and assigned different default values.
They will be particularly useful in conjunction
with the forwarding mechanism described in \secref{sec:forward}.

For example, it may be useful to have a flag |\version|
which can be set to |draft| or |final|.
The document source will contain some conditional code
depending on the value of |\version|.
Suppose further, the flag should default to |final| for the main file
and to |draft| for child files
which is a natural assignment for editing the document.
This is achieved by placing the following code
in the preamble of the main document
(below the |\childdocmain| directive):
%
\begin{center}
\begin{tabular}{l}
|\ifchilddoc|\\
|\providecommand{\version}{draft}|\\
|\||else|\\
|\providecommand{\version}{final}|\\
|\||fi|
\end{tabular}
\end{center}
%
The definition by |\providecommand| makes sure
that previous definitions are not overwritten.
Further statements |\providecommand{\version}{...}|
can thus be added before the above code to override it.

For the main file, one might add a line
(between |\childdocmain| and the above block)
%
\begin{center}
|%\ifchilddoc\||else\providecommand{\version}{draft}\||fi|
\end{center}
%
which can be uncommented to produce a draft version.
Likewise one can add a line to the very top of a child file
(above the |\childdocof{|\textit{main}|}| directive)
%
\begin{center}
|%\providecommand{\version}{final}|
\end{center}
%
which can be uncommented to produce the final version of this child document.

%%%%%%%%%%%%%%%%%%%%%%%%%%%%%%%%%%%%%%%%%%%%%%%%%%%%%%%%%%%%%%%%%%%%%%%%%%%%%%%%
\subsection{Forwarding}
\label{sec:forward}

Different versions of the main or child documents
using compilation flags as described in \secref{sec:flags}
can be (permanently) stored in different files
for convenient compilation, viewing and distribution.
To this end, the package defines a command
to pass on compilation to a different file:

%%%%%%%%%%%%%%%%%%%%%%%%%%%%%%%%%%%%%%%%
\DescribeMacro{\childdocforward}
The command |\childdocforward| redirects processing to
another source file:
%
\begin{center}
\begin{tabular}{l}
|\input{childdoc.def}|\\
|\childdocforward[|\textit{main}|]{|\textit{dest}|}|\\
\end{tabular}
\end{center}
%
The argument \textit{dest} is the destination file
(without extension).
It should be the main file or one of the child files.
Note that further \textsf{childdoc} directives
such as |\childdocof| and |\childdocforward|
in the indicated file will be processed in this form.
The optional argument \textit{main}
passes on directly to the main file \textit{main}
while pretending to compile the child \textit{dest}.
This form behaves as if \textit{dest}
issues |\childdocof{|\textit{main}|}| right away,
and no further \textsf{childdoc} directives will be processed.

%%%%%%%%%%%%%%%%%%%%%%%%%%%%%%%%%%%%%%%%
\DescribeMacro{\...prefix}
In the alternative form |\childdocforwardprefix|,
%
\begin{center}
\begin{tabular}{l}
|\input{childdoc.def}|\\
|\childdocforwardprefix[|\textit{main}|]{|\textit{prefix}|}{|\textit{dest}|}|
\end{tabular}
\end{center}
%
the destination file is determined by a pattern
depending on the current file:
To make this work, the current file must be called
`{\textit{prefix}\hspace{0.2em}\textit{suffix}}'
with \textit{prefix} matching precisely the argument.
Processing is then passed on to the file
`{\textit{dest}\hspace{0.2em}\textit{suffix}}'.
Surely, the same effect is achieved by
directly specifying the
argument `{\textit{dest}\hspace{0.2em}\textit{suffix}}'
in the first form.
However, that requires to set up a different file
for each child. With the alternative form of the command
all these files can have exactly the same content
which simplifies setting them up and maintaining them.

For example, the following file |draft.tex|
with a compilation flag |\version| as described in \secref{sec:flags}
compiles the main document as a draft:
%
\begin{center}
\begin{tabular}{l}
|\def\version{draft}|\\
|\input{childdoc.def}|\\
|\childdocforward{|\textit{main}|}|
\end{tabular}
\end{center}
%
Likewise, the following files |final|\textit{nn}|.tex|
compile the final version of the child document
|child|\textit{nn}|.tex|:
%
\begin{center}
\begin{tabular}{l}
|\def\version{final}|\\
|\input{childdoc.def}|\\
|\childdocforwardprefix{final}{child}|
\end{tabular}
\end{center}
%

Note that when several versions of a main file and/or of each child file
are to be generated, it may be convenient to set up a |Makefile| or
shell script to automatise the process.

%%%%%%%%%%%%%%%%%%%%%%%%%%%%%%%%%%%%%%%%%%%%%%%%%%%%%%%%%%%%%%%%%%%%%%%%%%%%%%%%
\subsection{Command Line Processing}
\label{sec:commandline}

The effect of redirection files can also be achieved by invoking
the \LaTeX{} compiler with a more elaborate command line.
Most conveniently this should be done as part
of a shell script or a |Makefile|.

When using \textsf{childdoc} in the main file, the following
command lines effectively perform a redirection
(note that depending on the shell being used,
backslashes may have to be doubled: `|\|' $\to$ `|\\|'):
%
\begin{center}
|... -jobname "|\textit{target}|" |\\|"|[\textit{flags}]%
|\input{childdoc.def}\childdocforward[|\textit{main}|]{|\textit{dest}|}"|
\end{center}
%
Here \textit{target} is the name of the output file,
\textit{main} is the name of the main file
and \textit{dest} is the name of the main or child file to be processed
(all filenames without extensions).
The optional argument \textit{main} can be omitted
if \textit{main} matches \textit{dest}.
Optionally, compilation \textit{flags} can be defined via |\def| commands.
This command line makes the \TeX{} engine believe
it is compiling the file \textit{target}
whose content is specified as the latter parameter.
The provided code then forwards the processing to
\textit{main} or \textit{dest} as described in \secref{sec:forward}.

%%%%%%%%%%%%%%%%%%%%%%%%%%%%%%%%%%%%%%%%%%%%%%%%%%%%%%%%%%%%%%%%%%%%%%%%%%%%%%%%
\subsection{Include by Input}
\label{sec:input}

Including child documents by |\include| has some restrictions by design.
Most notably, the content of a child document always occupies
its own set of pages; pages cannot be shared between child documents.
Usually, this behaviour makes perfect sense
because each child document contain an essential part of the document.
However, in some situations it may be desirable to compose
a document from a collection of parts
without having mandatory page breaks between then.
For this case, the package
provides a mechanism to include parts
by |\input| which can also be processed individually.
However, by construction this mechanism
requires manual handling of the content to be output.

%%%%%%%%%%%%%%%%%%%%%%%%%%%%%%%%%%%%%%%%
\DescribeMacro{\ifchilddocmanual}
The main file should be prepared as usual, see \secref{sec:include}.
However, the document body must make a distinction
between processing of an individual part and of the main document, e.g.:
%
\begin{center}
\begin{tabular}{l}
|\ifchilddocmanual|\\
|\input{\childdocname}|\\
|\||else|\\
\textit{document body with }|\input{|\textit{part}|}|\\
|\||fi|
\end{tabular}
\end{center}
%
The conditional |\ifchilddocmanual| is true whenever
a part to be included by |\input| is being compiled,
and the name of the part is stored in |\childdocname|.

%%%%%%%%%%%%%%%%%%%%%%%%%%%%%%%%%%%%%%%%
\DescribeMacro{\childdocby}
Each part to be included by |\input| should start with:
%
\begin{center}
\begin{tabular}{l}
|\input{childdoc.def}|\\
|\childdocby{|\textit{main}|}|\\
\end{tabular}
\end{center}
%
The directive |\childdocby| is similar to |\childdocof|
described in \secref{sec:include},
but the subsequent selection of content must be done manually.
To that end, both |\ifchilddoc| and |\ifchilddocmanual|
will be true upon processing of a part,
and the name of the part is stored in |\childdocname|.
Note that |\jobname| will be set to the filename of the current part
so that each part receives an individual |.aux| file
that does not interfere with the |.aux| file(s) of the main document.
This behaviour can be altered by the alternative form
|\childdocby[*]{|\textit{main}|}| (with a non-empty optional argument)
which uses the |.aux| file of the main document
by setting |\jobname| to \textit{main}.

%%%%%%%%%%%%%%%%%%%%%%%%%%%%%%%%%%%%%%%%%%%%%%%%%%%%%%%%%%%%%%%%%%%%%%%%%%%%%%%%
\subsection{Driver Development}
\label{sec:driver}

The \textsf{childdoc} mechanism can also be use for the development
of definition files such as \LaTeX{} styles or classes.
This case differs from the above setup with multiple parts
included by |\include| in that no |\includeonly| should be invoked.
This can be achieved by starting the include file
(before |\ProvidesPackage|) with:
%
\begin{center}
\begin{tabular}{l}
|\input{childdoc.def}|\\
|\childdocforward{|\textit{main}|}|\\
\end{tabular}
\end{center}
%
or alternatively with:
%
\begin{center}
\begin{tabular}{l}
|\input{childdoc.def}|\\
|\childdocby{|\textit{main}|}|\\
\end{tabular}
\end{center}
%
Both forms have slightly different effects as described above.
The main file is prepared as usual, see \secref{sec:include}.

%%%%%%%%%%%%%%%%%%%%%%%%%%%%%%%%%%%%%%%%%%%%%%%%%%%%%%%%%%%%%%%%%%%%%%%%%%%%%%%%
\subsection{Legacy Detection}
\label{sec:detection}

The directive |\childdocmain| in the main file can detect
whether the complete document or merely a child is to be compiled
even without using the directive |\childdocof|.
This method is deprecated because it is less robust
and there is no compelling reason to use it;
it is merely provided for backward compatibility
and it may be removed in future versions.

If the detection mechanism is to be used,
it is mandatory to correctly specify
the filename of the main file as the argument of |\childdocmain|:
%
\begin{center}
\begin{tabular}{l}
|\input{childdoc.def}|\\
|\childdocmain{|\textit{main}|}|\\
\end{tabular}
\end{center}
%
If |\jobname| does not match the argument \textit{main} of |\childdocmain|,
it is assumed that |\jobname| points to the child file to be compiled.
When using |\childdocmain| with the main file specified as argument,
it suffices to start a child file
with just |\input{|\textit{main}|}|
without loading of the package and using |\childdocof|.
If instead all processing is done
with the appropriate \textsf{childdoc} directives,
the argument of \textit{main} of |\childdocmain| can be empty.

An alternative version of the command line processing described
in \secref{sec:commandline} using the detection mechanism reads:
%
\begin{center}
|... -jobname "|\textit{target}|" "|[\textit{flags}]%
[|\def\jobname{|\textit{dest}|}|]|\input{|\textit{main}|}"|
\end{center}

%%%%%%%%%%%%%%%%%%%%%%%%%%%%%%%%%%%%%%%%%%%%%%%%%%%%%%%%%%%%%%%%%%%%%%%%%%%%%%%%
\subsection{Manual Code}
\label{sec:manual}

In case one cannot be certain whether the definitions file |childdoc.def|
is installed on the target \TeX{} distribution
and one prefers not to ship it,
it is conceivable to paste a few relevant commands into the sources.

To that end, drop all statements |\input{childdoc.def}|
and perform the replacements as outlined below.
Instead of |\childdocmain{|\textit{main}|}| add the following code
to the top of the main file:
%
\begin{center}
\begin{tabular}{l}
|\||ifdefined\childdocname\endinput\||fi\newif\ifchilddoc|\\
|\edef\childdocname{\scantokens\expandafter{\jobname\noexpand}}|\\
|\def\childdocmain{|\textit{main}|}\||ifx\childdocmain\childdocname\||else|\\
|\childdoctrue\includeonly{\childdocname}\let\jobname\childdocmain\||fi|\\
\end{tabular}
\end{center}
%
Instead of |\childdocof{|\textit{main}|}| just include the main file
at the top of each child file:
%
\begin{center}
|\input{|\textit{main}|}|
\end{center}
%
A simple redirection |\childdocforward{|\textit{dest}|}| is achieved by:
%
\begin{center}
|\def\jobname{|\textit{dest}|}\input{\jobname}|
\end{center}
%
The redirection with prefix
|\childdocforwardprefix[|\textit{prefix}|]{|\textit{dest}|}|
is accomplished by:
%
\begin{center}
\begin{tabular}{l}
|{\edef\jobname{\scantokens\expandafter{\jobname\noexpand}}|\\
|\def\redirectjob |\textit{prefix}|#1~~~{\gdef\jobname{|\textit{dest}|#1}}|\\
|\expandafter\redirectjob\jobname~~~}\input{\jobname}|
\end{tabular}
\end{center}

In an alternative approach,
child documents can be compiled by a specific command line
without additional code or specific definitions:
%
\begin{center}
|... -jobname "|\textit{target}|" "|[\textit{flags}]%
|\includeonly{|\textit{dest}|}\input{|\textit{main}|}"|
\end{center}
%

%%%%%%%%%%%%%%%%%%%%%%%%%%%%%%%%%%%%%%%%%%%%%%%%%%%%%%%%%%%%%%%%%%%%%%%%%%%%%%%%
%%%%%%%%%%%%%%%%%%%%%%%%%%%%%%%%%%%%%%%%%%%%%%%%%%%%%%%%%%%%%%%%%%%%%%%%%%%%%%%%
\section{Information}

%%%%%%%%%%%%%%%%%%%%%%%%%%%%%%%%%%%%%%%%%%%%%%%%%%%%%%%%%%%%%%%%%%%%%%%%%%%%%%%%
\subsection{Copyright}

Copyright \copyright{} 2017--2018 Niklas Beisert

This work may be distributed and/or modified under the
conditions of the \LaTeX{} Project Public License, either version 1.3
of this license or (at your option) any later version.
The latest version of this license is in
  \url{http://www.latex-project.org/lppl.txt}
and version 1.3 or later is part of all distributions of \LaTeX{}
version 2005/12/01 or later.

This work has the LPPL maintenance status `maintained'.

The Current Maintainer of this work is Niklas Beisert.

This work consists of the files |README.txt|, |childdoc.ins| and |childdoc.dtx|
as well as the derived files |childdoc.def|, |cdocsamp.tex|
with |cdocsch1.tex|, |cdocsch2.tex|, |cdocspt3.tex|, |cdocspt4.tex|,
|cdocsdrf.tex|, |cdocsfn1.tex|, |cdocsfn2.tex|
as well as |childdoc.pdf|.

%%%%%%%%%%%%%%%%%%%%%%%%%%%%%%%%%%%%%%%%%%%%%%%%%%%%%%%%%%%%%%%%%%%%%%%%%%%%%%%%
\subsection{Files and Installation}

The package consists of the files:
%
\begin{center}
\begin{tabular}{ll}
    |README.txt|   & readme file \\
    |childdoc.ins| & installation file \\
    |childdoc.dtx| & source file \\
    |childdoc.def| & definition file \\
    |cdocsamp.tex| & sample main file \\
    |cdocsch1.tex| & sample include file \\
    |cdocsch2.tex| & sample include file \\
    |cdocspt3.tex| & sample part file \\
    |cdocspt4.tex| & sample part file \\
    |cdocsdrf.tex| & sample redirection file \\
    |cdocsfn1.tex| & sample redirection file \\
    |cdocsfn2.tex| & sample redirection file \\
    |childdoc.pdf| & manual
\end{tabular}
\end{center}
%
The distribution consists of the files
|README.txt|, |childdoc.ins| and |childdoc.dtx|.
%
\begin{itemize}
\item
Run (pdf)\LaTeX{} on |childdoc.dtx|
to compile the manual |childdoc.pdf| (this file).
\item
Run \LaTeX{} on |childdoc.ins| to create the definitions file |childdoc.def|
and the sample |cdocsamp.tex| with include files
|cdocsch1.tex|, |cdocsch2.tex|, |cdocspt3.tex|, |cdocspt4.tex|,
|cdocsdrf.tex|, |cdocsfn1.tex|, |cdocsfn2.tex|.
Then copy the file |childdoc.def| to an appropriate directory of your \LaTeX{}
distribution, e.g.\ \textit{texmf-root}|/tex/latex/childdoc|.
\end{itemize}

%%%%%%%%%%%%%%%%%%%%%%%%%%%%%%%%%%%%%%%%%%%%%%%%%%%%%%%%%%%%%%%%%%%%%%%%%%%%%%%%
\subsection{Related CTAN Packages}

There are several other packages which offer a similar functionality:
%
\begin{itemize}
\item
The packages
\href{http://ctan.org/pkg/docmute}{\textsf{docmute}},
\href{http://ctan.org/pkg/includex}{\textsf{includex}} and
\href{http://ctan.org/pkg/standalone}{\textsf{standalone}}
provide commands to include only the document body of
a child file thus allowing both files to be compiled individually.
\item
The packages \href{http://ctan.org/pkg/subdocs}{\textsf{subdocs}}
and \href{http://ctan.org/pkg/subfiles}{\textsf{subfiles}}
provide structures in which the main and child documents can be
encapsulated and allowing them to be compiled individually.
The inclusion mechanism is different from the conventional |\include|.
\item
The package \href{http://ctan.org/pkg/combine}{\textsf{combine}}
is an elaborate solution to combine several documents into one.
\end{itemize}
%
See also the CTAN topic \href{http://ctan.org/topic/subdocs}{\textsf{subdocs}}
for further related packages.
The present package differs from the above solutions in that
a document structure constructed with the conventional |\include| mechanism
just needs two extra commands at the top of every file
such that all constituent files can be compiled individually.

%%%%%%%%%%%%%%%%%%%%%%%%%%%%%%%%%%%%%%%%%%%%%%%%%%%%%%%%%%%%%%%%%%%%%%%%%%%%%%%%
%\subsection{Feature Suggestions}
%
%The following is a list of features which may be useful for future
%versions of this package:
%%
%\begin{itemize}
%\item
%\ldots
%\end{itemize}

%%%%%%%%%%%%%%%%%%%%%%%%%%%%%%%%%%%%%%%%%%%%%%%%%%%%%%%%%%%%%%%%%%%%%%%%%%%%%%%%
\subsection{Revision History}

%%%%%%%%%%%%%%%%%%%%%%%%%%%%%%%%%%%%%%%%
\paragraph{v2.0:} 2018/12/30

\begin{itemize}
\item
immediate forward processing
\item
added |\childdocby| mechanism
\item
manual restructured
\end{itemize}

%%%%%%%%%%%%%%%%%%%%%%%%%%%%%%%%%%%%%%%%
\paragraph{v1.6:} 2018/01/17

\begin{itemize}
\item
application for development of include files
\item
corrections to manual
\end{itemize}

%%%%%%%%%%%%%%%%%%%%%%%%%%%%%%%%%%%%%%%%
\paragraph{v1.5:} 2017/05/21

\begin{itemize}
\item
more complete structuring introduced
\item
|\childdocof| introduced
\item
|\childdoc| renamed to |\childdocmain|
\item
|\childredirect| renamed to |\childdocforward| and |\childdocforwardprefix|
and functionality expanded
\end{itemize}

%%%%%%%%%%%%%%%%%%%%%%%%%%%%%%%%%%%%%%%%
\paragraph{v1.0:} 2017/04/27

\begin{itemize}
\item
manual and install package
\item
first version published on CTAN
\end{itemize}

%%%%%%%%%%%%%%%%%%%%%%%%%%%%%%%%%%%%%%%%
\paragraph{v0.6:} 2017/04/26

\begin{itemize}
\item
redirection mechanism added
\end{itemize}

%%%%%%%%%%%%%%%%%%%%%%%%%%%%%%%%%%%%%%%%
\paragraph{v0.5:} 2017/04/26

\begin{itemize}
\item
functionality in definition file
\end{itemize}


%%%%%%%%%%%%%%%%%%%%%%%%%%%%%%%%%%%%%%%%%%%%%%%%%%%%%%%%%%%%%%%%%%%%%%%%%%%%%%%%
%%%%%%%%%%%%%%%%%%%%%%%%%%%%%%%%%%%%%%%%%%%%%%%%%%%%%%%%%%%%%%%%%%%%%%%%%%%%%%%%
%%%%%%%%%%%%%%%%%%%%%%%%%%%%%%%%%%%%%%%%%%%%%%%%%%%%%%%%%%%%%%%%%%%%%%%%%%%%%%%%
\appendix

\settowidth\MacroIndent{\rmfamily\scriptsize 000\ }

 \DocInput{childdoc.dtx}

\end{document}
%</driver>
% \fi
%
% %%%%%%%%%%%%%%%%%%%%%%%%%%%%%%%%%%%%%%%%%%%%%%%%%%%%%%%%%%%%%%%%%%%%%%%%%%%%%%
% %%%%%%%%%%%%%%%%%%%%%%%%%%%%%%%%%%%%%%%%%%%%%%%%%%%%%%%%%%%%%%%%%%%%%%%%%%%%%%
% \section{Sample}
%\iffalse
%<*samplemain>
%\fi
%
% The following presents a sample document
% with two chapters, two parts, a title page,
% a compile flag as well as three forwarding files to set the flag.
% It consists of eight |.tex| files:
% \begin{center}
% \begin{tabular}{ll}
% |cdocsamp.tex|&main file\\
% |cdocsch1.tex|&include file for chapter 1\\
% |cdocsch2.tex|&include file for chapter 2\\
% |cdocspt3.tex|&include file for part 3\\
% |cdocspt4.tex|&include file for part 4\\
% |cdocsdrf.tex|&forwarding file for main file in draft mode\\
% |cdocsfi1.tex|&forwarding file for final version of chapter 1\\
% |cdocsfi2.tex|&forwarding file for final version of chapter 2\\
% \end{tabular}
% \end{center}
% Each of the eight files can be compiled directly by the \LaTeX{} compiler.
%
% %%%%%%%%%%%%%%%%%%%%%%%%%%%%%%%%%%%%%%
% \paragraph{Main File.}
%
% The main file is called |cdocsamp.tex|.
%
% Load the \textsf{childdoc} definitions and
% declare the filename for the main document:
%    \begin{macrocode}
\input{childdoc.def}
\childdocmain{}
%    \end{macrocode}

% Optional override for |\version| flag:
%    \begin{macrocode}
%%\ifchilddoc\else\providecommand{\version}{draft}\fi
%    \end{macrocode}

% Define the default values for the |\version| flag
% (|final| for the main file and |draft| for childs):
%    \begin{macrocode}
\ifchilddoc
\providecommand{\version}{draft}
\else
\providecommand{\version}{final}
\fi
%    \end{macrocode}

% Load the standard document class:
%    \begin{macrocode}
\documentclass[12pt]{article}
%    \end{macrocode}

% Start the document body:
%    \begin{macrocode}
\begin{document}
%    \end{macrocode}

% Declare a title page.
% Print title, part of document being processed and version flag:
%    \begin{macrocode}
\addtocounter{page}{-1}
\begin{center}
{\LARGE\bfseries{}childdoc example\par}
\vspace{1cm}
\ifchilddoc
\ifchilddocmanual part\else chapter\fi:
`\childdocname' of `\childdocjob'\par
\else
main document: `\childdocjob'\par
\fi
version: \version\par
\end{center}
\newpage
%    \end{macrocode}

% Manually include selected file,
% otherwise process as usual:
%    \begin{macrocode}
\ifchilddocmanual
\section*{part `\childdocname'}
\input{\childdocname}
\else
%    \end{macrocode}

% Include the two chapters:
%    \begin{macrocode}
\include{cdocsch1}
\include{cdocsch2}
%    \end{macrocode}

% Include the two parts unless only chapters should be displayed:
%    \begin{macrocode}
\ifchilddoc\else
\section{part three}
\input{cdocspt3}
\section{part four}
\input{cdocspt4}
\fi
%    \end{macrocode}

% Process as usual until here:
%    \begin{macrocode}
\fi
%    \end{macrocode}

% End of document body:
%    \begin{macrocode}
\end{document}
%    \end{macrocode}
%\iffalse
%</samplemain>
%\fi
%
% %%%%%%%%%%%%%%%%%%%%%%%%%%%%%%%%%%%%%%
% \paragraph{Chapter Include Files.}
%
% The include files are called |cdocsch1.tex| and |cdocsch2.tex|.
%
%\iffalse
%<*samplechap1|samplechap2>
%\fi

% Optional override for |\version| flag:
%    \begin{macrocode}
%%\providecommand{\version}{final}
%    \end{macrocode}

% Include the main document:
%    \begin{macrocode}
\input{childdoc.def}
\childdocof{cdocsamp}
%    \end{macrocode}

%\iffalse
%</samplechap1|samplechap2>
%\fi
%
%\iffalse
%<*samplechap1>
%\fi
% Some text for chapter 1:
%    \begin{macrocode}
\section{one}
some text in chapter one
%    \end{macrocode}

%\iffalse
%</samplechap1>
%\fi
% Some text for chapter 2:
%\iffalse
%<*samplechap2>
%\fi
%    \begin{macrocode}
\section{two}
more text in chapter two
%    \end{macrocode}

%\iffalse
%</samplechap2>
%\fi
%
% %%%%%%%%%%%%%%%%%%%%%%%%%%%%%%%%%%%%%%
% \paragraph{Part Include Files.}
%
% The include files are called |cdocspt3.tex| and |cdocspt4.tex|.
%
%\iffalse
%<*samplepart3|samplepart4>
%\fi

% Optional override for |\version| flag:
%    \begin{macrocode}
%%\providecommand{\version}{final}
%    \end{macrocode}

% Include the main document:
%    \begin{macrocode}
\input{childdoc.def}
\childdocby{cdocsamp}
%    \end{macrocode}

%\iffalse
%</samplepart3|samplepart4>
%\fi
%
%\iffalse
%<*samplepart3>
%\fi
% Some text for part 3:
%    \begin{macrocode}
some text in part three
%    \end{macrocode}

%\iffalse
%</samplepart3>
%\fi
% Some text for part 4:
%\iffalse
%<*samplepart4>
%\fi
%    \begin{macrocode}
more text in part four
%    \end{macrocode}

%\iffalse
%</samplepart4>
%\fi
%
% %%%%%%%%%%%%%%%%%%%%%%%%%%%%%%%%%%%%%%
% \paragraph{Forwarding for a Complete Draft.}
%
% The following forwarding file |cdocsdrf.tex|
% compiles the main document in draft mode:
%\iffalse
%<*sampledraft>
%\fi
%    \begin{macrocode}
\def\version{draft}
\input{childdoc.def}
\childdocforward{cdocsamp}
%    \end{macrocode}

%\iffalse
%</sampledraft>
%\fi
%
% %%%%%%%%%%%%%%%%%%%%%%%%%%%%%%%%%%%%%%
% \paragraph{Forwarding for Final Version of the Chapters.}
%
% The following forwarding files |cdocsfn1.tex| and |cdocsfn2.tex|
% (with identical content)
% compile the final versions of the child documents
% |cdocsch1.tex| and |cdocsch2.tex|, respectively:
%\iffalse
%<*samplefinal>
%\fi
%    \begin{macrocode}
\def\version{final}
\input{childdoc.def}
\childdocforwardprefix[cdocsamp]{cdocsfn}{cdocsch}
%    \end{macrocode}

%\iffalse
%</samplefinal>
%\fi
%
% %%%%%%%%%%%%%%%%%%%%%%%%%%%%%%%%%%%%%%
% \paragraph{Command Line Processing.}
%
% The following three command lines generate the output files
% |cdocscld|, |cdocscl1| and |cdocscl2|
% which should be identical to
% |cdocsdrf|, |cdocsch1| and |cdocsfn2|, respectively:
% \begin{center}
% \begin{tabular}{l}
% |latex -jobname cdocscld \|\\
% |  "\def\version{draft}\input{childdoc.def}\childdocforward{cdocsamp}"|\\
% |latex -jobname cdocscl1 \|\\
% |  "\input{childdoc.def}\childdocforward[cdocsamp]{cdocsch1}"|\\
% |latex -jobname cdocscl2 \|\\
% |  "\def\version{final}\input{childdoc.def}\childdocforward{cdocsch2}"|
% \end{tabular}
% \end{center}
% Note that the trailing backslash on each first line
% merely continues the input to the second line
% (for convenient cut ant paste).
% Furthermore, the command |latex| can be replaced by any
% of its alternative versions such as |pdflatex|.
%
% %%%%%%%%%%%%%%%%%%%%%%%%%%%%%%%%%%%%%%%%%%%%%%%%%%%%%%%%%%%%%%%%%%%%%%%%%%%%%%
% %%%%%%%%%%%%%%%%%%%%%%%%%%%%%%%%%%%%%%%%%%%%%%%%%%%%%%%%%%%%%%%%%%%%%%%%%%%%%%
% \section{Implementation}
%\iffalse
%<*package>
%\fi
%
% This section describes the definitions file |childdoc.def|.

% The definitions cannot be loaded using |\usepackage| or |\RequirePackage|
% which has a mechanism to prevent loading a style file more than once.
% When loading the definitions by means of |\input|
% multiple instances have to be prevented manually:
%\iffalse
%This code needs to be before the `\ProvidesFile' directive
%which is defined at the beginning of this file.
%Therefore it is also placed there and commented out here.
%</package>
%<*discard>
%\fi
%    \begin{macrocode}
\ifdefined\childdocmain\endinput\fi
%    \end{macrocode}
%\iffalse
%</discard>
%<*package>
%\fi
%
% \macro{\ifchilddoc}
% \macro{\ifchilddocmanual}
% The conditional |\ifchilddoc| tells whether a
% child (true) or main (false) document is being compiled.
% The conditional |\ifchilddocmanual| tells whether
% the |\includeonly| mechanism is used (false) or
% the selection of child files must be performed manually (true).
% The definitions initialise to false:
%    \begin{macrocode}
\newif\ifchilddoc
\newif\ifchilddocmanual
%    \end{macrocode}

% \macro{\childdocname}
% \macro{\childdocjob}
% The macro |\childdocname| stores the name of the main document
% to be compiled. The macro |\childdocjob| stores the name of
% the document on which the \LaTeX{} compiler was originally invoked.
% The content of |\jobname| cannot be compared
% to filenames specified in the source due to different catcodes.
% The following code rescans |\jobname|, stores the result
% in |\childdocname| and saves a copy in |\childdocjob|:
%    \begin{macrocode}
\edef\childdocname{\scantokens\expandafter{\jobname\noexpand}}
\let\childdocjob\childdocname
%    \end{macrocode}

% \macro{\childdocdisable}
% The macro |\childdocdisable| prevents the main file
% from being processed more than once.
% At this stage, the main document command |\childdocmain|
% is assumed to be called once again where it should do nothing.
% Any subsequent call to it should prevent
% a secondary processing of the main document
% It overwrites the forwarding commands
% |\childdocof| and |\childdocforward|
% with empty macros to prevent further inclusions of the main document:
%    \begin{macrocode}
\newcommand{\childdocdisable}
{
  \renewcommand{\childdocmain}[1]{\renewcommand{\childdocmain}[1]{\endinput}}
  \renewcommand{\childdocof}[1]{}
  \renewcommand{\childdocby}[2][]{}
  \renewcommand{\childdocforward}[2][]{}
  \renewcommand{\childdocdisable}{}
}
%    \end{macrocode}

% \macro{\childdocmain}
% The macro |\childdocmain| is to be called at the top of the main file
% with nothing or the main filename (without extension) as argument.
% First, it breaks loops.
% If the argument is not empty and does not match |\childdocname|
% (which is set by the first inclusion of |childdoc.def|),
% |\ifchilddoc| is set to true, |\includeonly| is applied to the child file
% and |\jobname| is set to the main file
% (for proper handling of |.aux| files):
%    \begin{macrocode}
\newcommand{\childdocmain}[1]
{
  \childdocdisable\childdocmain{}
  \if?#1?\else
    \begingroup
      \def\childdoctmp{#1}
      \ifx\childdoctmp\childdocname
        \def\childdoctmp{}
      \else
        \def\childdoctmp
        {
          \childdoctrue
          \includeonly{\childdocname}
          \def\childdocjob{#1}
          \def\jobname{#1}
        }
      \fi
      \expandafter
    \endgroup
    \childdoctmp
  \fi
}
%    \end{macrocode}

% \macro{\childdocof}
% The command |\childdocof| redirects
% compilation to the main file |#1|.
%    \begin{macrocode}
\newcommand{\childdocof}[1]
{
  \childdocdisable
  \childdoctrue
  \includeonly{\childdocname}
  \def\jobname{#1}
  \def\childdocjob{#1}
  \input{#1}
}
%    \end{macrocode}

% \macro{\childdocby}
% The command |\childdocby| ....
%    \begin{macrocode}
\newcommand{\childdocby}[2][]
{
  \childdocdisable
  \childdoctrue
  \childdocmanualtrue
  \if?#1?\else
    \def\jobname{#2}
  \fi
  \def\childdocjob{#2}
  \input{#2}
  \endinput
}
%    \end{macrocode}

% \macro{\childdocforward}
% The command |\childdocforward| redirects
% compilation to the main file or
% (if the optional argument is given) a child file.
% Parameters are set as if the main file
% or a child file starting with |\childdocof| was compiled.
% Then compilation is handed over to the main file:
%    \begin{macrocode}
\newcommand{\childdocforward}[2][]
{
  \begingroup
    \if?#1?
      \def\childdoctmp
      {
        \def\childdocname{#2}
        \def\childdocjob{#2}
        \def\jobname{#2}
        \input{#2}
        \endinput
      }
    \else
      \def\childdoctmp
      {
        \childdocdisable
        \def\childdocname{#2}
        \childdoctrue
        \includeonly{#2}
        \def\childdocjob{#1}
        \def\jobname{#1}
        \input{#1}
        \endinput
      }
    \fi
    \expandafter
  \endgroup
  \childdoctmp
}
%    \end{macrocode}

% \macro{\childdocforwardprefix}
% The command |\childdocforwardprefix| redirects
% compilation to the main or a child file by means of a pattern.
% The prefix |#1| in the current filename is replaced by |#2|
% and the suffix of the current filename is kept
% (it is assumed that the filename does not contain the substring `|~~~|'
% which is used as a delimiter).
% Compilation is handed over to the new file by |\childdocforward|:
%    \begin{macrocode}
\newcommand{\childdocforwardprefix}[3][]
{
  \begingroup
    \def\childdocextract #2##1~~~{\def\childdoctmp{\childdocforward[#1]{#3##1}}}
    \expandafter\childdocextract\childdocname~~~
    \expandafter
  \endgroup
  \childdoctmp
}
%    \end{macrocode}

% \macro{\childdoc}
% The deprecated macro |\childdoc| is a legacy version of |\childdocmain|:
%    \begin{macrocode}
\newcommand{\childdoc}{\childdocmain}
%    \end{macrocode}

% \macro{\childdocredirect}
% The deprecated macro |\childdocredirect| is a legacy version
% of |\childdocforward| and |\childdocforwardprefix|:
%    \begin{macrocode}
\newcommand{\childdocredirect}[2][]
{
  \begingroup
    \if?#1?
      \def\childdoctmp{\childdocforward{#2}}
    \else
      \def\childdoctmp{\childdocforwardprefix{#1}{#2}}
    \fi
    \expandafter
  \endgroup
  \childdoctmp
}
%    \end{macrocode}

%\iffalse
%</package>
%\fi
%
\endinput
|\\
|\childdocmain{|\textit{main}|}|\\
\end{tabular}
\end{center}
%
If |\jobname| does not match the argument \textit{main} of |\childdocmain|,
it is assumed that |\jobname| points to the child file to be compiled.
When using |\childdocmain| with the main file specified as argument,
it suffices to start a child file
with just |\input{|\textit{main}|}|
without loading of the package and using |\childdocof|.
If instead all processing is done
with the appropriate \textsf{childdoc} directives,
the argument of \textit{main} of |\childdocmain| can be empty.

An alternative version of the command line processing described
in \secref{sec:commandline} using the detection mechanism reads:
%
\begin{center}
|... -jobname "|\textit{target}|" "|[\textit{flags}]%
[|\def\jobname{|\textit{dest}|}|]|\input{|\textit{main}|}"|
\end{center}

%%%%%%%%%%%%%%%%%%%%%%%%%%%%%%%%%%%%%%%%%%%%%%%%%%%%%%%%%%%%%%%%%%%%%%%%%%%%%%%%
\subsection{Manual Code}
\label{sec:manual}

In case one cannot be certain whether the definitions file |childdoc.def|
is installed on the target \TeX{} distribution
and one prefers not to ship it,
it is conceivable to paste a few relevant commands into the sources.

To that end, drop all statements |% \iffalse
%
% childdoc.dtx Copyright (C) 2017-2018 Niklas Beisert
%
% This work may be distributed and/or modified under the
% conditions of the LaTeX Project Public License, either version 1.3
% of this license or (at your option) any later version.
% The latest version of this license is in
%   http://www.latex-project.org/lppl.txt
% and version 1.3 or later is part of all distributions of LaTeX
% version 2005/12/01 or later.
%
% This work has the LPPL maintenance status `maintained'.
%
% The Current Maintainer of this work is Niklas Beisert.
%
% This work consists of the files childdoc.dtx and childdoc.ins
% and the derived files childdoc.def and cdocsamp.tex with
% cdocsch1.tex, cdocsch2.tex, cdocsdrf.tex, cdocsfn1.tex, cdocsfn2.tex.
%
%<package>\ifdefined\childdocmain\endinput\fi
%<package>\ProvidesFile{childdoc.def}[2018/12/30 v2.0 child document driver]
%<samplemain>\ProvidesFile{cdocsamp.tex}[2018/12/30 v2.0 sample for childdoc]
%<*driver>
%\ProvidesFile{childdoc.drv}[2018/12/30 v2.0 childdoc reference manual file]
\PassOptionsToClass{10pt,a4paper}{article}
\documentclass{ltxdoc}

\usepackage[margin=35mm]{geometry}
\usepackage{hyperref}
\usepackage{hyperxmp}
\usepackage[usenames]{color}

\hypersetup{colorlinks=true}
\hypersetup{pdfstartview=FitH}
\hypersetup{pdfpagemode=UseNone}
\hypersetup{pdfsource={}}
\hypersetup{pdflang={en-UK}}
\hypersetup{pdfcopyright={Copyright 2017-2018 Niklas Beisert.
  This work may be distributed and/or modified under the
  conditions of the LaTeX Project Public License, either version 1.3
  of this license or (at your option) any later version.}}
\hypersetup{pdflicenseurl={http://www.latex-project.org/lppl.txt}}
\hypersetup{pdfcontactaddress={ETH Zurich, ITP, HIT K,
  Wolfgang-Pauli-Strasse 27}}
\hypersetup{pdfcontactpostcode={8093}}
\hypersetup{pdfcontactcity={Zurich}}
\hypersetup{pdfcontactcountry={Switzerland}}
\hypersetup{pdfcontactemail={nbeisert@itp.phys.ethz.ch}}
\hypersetup{pdfcontacturl={http://people.phys.ethz.ch/\xmptilde nbeisert/}}

\newcommand{\secref}[1]{\hyperref[#1]{section \ref*{#1}}}

\parskip1ex
\parindent0pt
\let\olditemize\itemize
\def\itemize{\olditemize\parskip0pt}

\begin{document}

\title{The \textsf{childdoc} Package}
\hypersetup{pdftitle={The childdoc Package}}
\author{Niklas Beisert\\[2ex]
  Institut f\"ur Theoretische Physik\\
  Eidgen\"ossische Technische Hochschule Z\"urich\\
  Wolfgang-Pauli-Strasse 27, 8093 Z\"urich, Switzerland\\[1ex]
  \href{mailto:nbeisert@itp.phys.ethz.ch}
  {\texttt{nbeisert@itp.phys.ethz.ch}}}
\hypersetup{pdfauthor={Niklas Beisert}}
\hypersetup{pdfsubject={Manual for the LaTeX2e Package childdoc}}
\date{30 December 2018, \textsf{v2.0}}
\maketitle

\begin{abstract}\noindent
\textsf{childdoc} is a \LaTeXe{} package
that enables the direct compilation
of document sections included by |\include|
to individual files.
\end{abstract}

\begingroup
\parskip0ex
\tableofcontents
\endgroup

%%%%%%%%%%%%%%%%%%%%%%%%%%%%%%%%%%%%%%%%%%%%%%%%%%%%%%%%%%%%%%%%%%%%%%%%%%%%%%%%
%%%%%%%%%%%%%%%%%%%%%%%%%%%%%%%%%%%%%%%%%%%%%%%%%%%%%%%%%%%%%%%%%%%%%%%%%%%%%%%%
\section{Introduction}

\LaTeX{} provides a mechanism to structure a large document (such as a book)
into a main file and several child files (containing the chapters)
using the |\include| command.
This mechanism is beneficial for documents
which span hundreds of pages in order to
make the source file(s) more manageable.
Moreover, compilation can be restricted to
selected child files by means of the |\includeonly| command.
The latter feature can be used to reduce the compilation time while editing
(this was significantly more useful in the earlier days of \LaTeX{})
or to generate a smaller document which is easier to navigate.
Another application of |\includeonly| is to generate
documents consisting of selected parts of the complete document.

However, there are a few drawbacks of the plain |\include| mechanism:
\begin{itemize}
\item
The child files cannot be compiled on their own,
they can only be compiled via the main file.
A naive editing environment
(such as a text editor with an option
to have the current file processed by \LaTeX)
may require one to switch to the main file before compiling;
attempting to compile the child file produces errors.
\item
The main file must be modified (each time)
to adjust the |\includeonly| command
to the present needs. This easily leaves the main file in a messy state.
\item
The generated document will always carry the filename
of the main document. This is inconvenient if
several child files are to be compiled and
to be kept for distribution.
\end{itemize}

The present package provides a simple interface
to make child files individually compilable by \LaTeX{}.
Compiling a child file then has the same effect as compiling
the main file with an |\includeonly| command
to select the appropriate child.
Moreover the generated document will carry the name of the child
rather than the main file.
This resolves all three above issues.

This feature is meant to make the editing of books,
thesis documents and lecture notes somewhat more convenient.
However, the package can also be used efficiently for
composing a series of documents (such as exercise sheets)
which are typically distributed individually.
It then assists the author in generating the individual documents
(potentially in different versions)
as well as a document containing the collected series.
Another application is in developing style files
or other kinds of included material
where compilation of the style file could redirect
to a sample or test file.

%%%%%%%%%%%%%%%%%%%%%%%%%%%%%%%%%%%%%%%%%%%%%%%%%%%%%%%%%%%%%%%%%%%%%%%%%%%%%%%%
%%%%%%%%%%%%%%%%%%%%%%%%%%%%%%%%%%%%%%%%%%%%%%%%%%%%%%%%%%%%%%%%%%%%%%%%%%%%%%%%
\section{Usage}

First of all, the package \textsf{childdoc} is \emph{not} a standard
\LaTeXe{} |.sty| style file! Therefore it needs to be invoked in
a non-standard way.

%%%%%%%%%%%%%%%%%%%%%%%%%%%%%%%%%%%%%%%%%%%%%%%%%%%%%%%%%%%%%%%%%%%%%%%%%%%%%%%%
\subsection{Included Files}
\label{sec:include}

%%%%%%%%%%%%%%%%%%%%%%%%%%%%%%%%%%%%%%%%
\DescribeMacro{\childdocmain}
To use the package, add the commands
\begin{center}
\begin{tabular}{l}
|\input{childdoc.def}|\\
|\childdocmain{}|\\
\end{tabular}
\end{center}
at the very top of the main \LaTeX{} file,
in particular \emph{before} the |\documentclass| statement!
The argument of |\childdocmain| should be left empty
(but it must be present).

%%%%%%%%%%%%%%%%%%%%%%%%%%%%%%%%%%%%%%%%
\DescribeMacro{\childdocof}
Furthermore, add the commands
\begin{center}
\begin{tabular}{l}
|\input{childdoc.def}|\\
|\childdocof{|\textit{main}|}|\\
\end{tabular}
\end{center}
at the top of every child file \textit{child}
which is included by |\include{|\textit{child}|}|
from within the main file
(or at least for those files to be compiled individually).
The argument \textit{main} must be the filename of the main file.

There are a couple of
considerations in setting up the main and child documents:

%%%%%%%%%%%%%%%%%%%%%%%%%%%%%%%%%%%%%%%%
\paragraph{Restrictions.}

Please note the following restrictions:
\begin{itemize}
\item
|\childdocmain| must be called with one argument \textit{main}
to ensure compatibility with earlier version of the package.
It must either be empty (|\childdocmain{}|)
or precisely match the filename of the main file in which it is specified.
See \secref{sec:detection} for further information.
\item
The filename \textit{main} must be specified without the |.tex| extension.
\item
The filename \textit{main} is case sensitive
(even in case-insensitive file systems)
due to internal string comparison.
\item
The argument \textit{main} should be fully expanded, it cannot be a macro.
\item
Subdirectories and special characters should be avoided in filenames.
\item
The command |\childdocmain{|\textit{main}|}| must be followed by a whitespace.
It should not be followed immediately by another command
or by a comment mark `|%|'.
This is because the \TeX{} parser reads the token immediately following
the argument of |\childdocmain| and puts it
at the beginning of every child section;
however, a white\-space is ignored.
\end{itemize}

%%%%%%%%%%%%%%%%%%%%%%%%%%%%%%%%%%%%%%%%
\paragraph{Content of Main File.}

It is advisable to place all content in the child files included by |\include|.
Any output contained in the main file will appear in all child documents
unless suppressed manually;
it cannot be suppressed automatically by the |\includeonly| directive
and thus should normally be avoided.
A method to include some content in the main file
by means of conditional processing is described in \secref{sec:conditional}.

%%%%%%%%%%%%%%%%%%%%%%%%%%%%%%%%%%%%%%%%
\paragraph{Page Numbering.}

When only a part of the document is compiled,
the appropriate numbering of pages
(as well as other status parameters)
is determined from the |.aux| files.
The latter contain information from previous passes.
However this information needs to propagate through
all intermediate child documents.
Therefore the page numbering in child documents may well
be inconsistent until the complete document is compiled at least once.

A useful (if unconventional) way to always ensure a consistent
page numbering is to restart the numbering in each child document
and denote the pages by `\textit{child}|.|\textit{page}'
where \textit{child} represents the chapter/section number of the child file.
This can be achieved by the command
|\numberwithin{page}{|\textit{child}|}|
of the \textsf{amsmath} package
where \textit{child} can be |chapter| or |section|
depending on the chosen structuring.
Alternatively, one can modify the macro |\thepage| appropriately
and reset the counter |page| at the start of each child file.

%%%%%%%%%%%%%%%%%%%%%%%%%%%%%%%%%%%%%%%%%%%%%%%%%%%%%%%%%%%%%%%%%%%%%%%%%%%%%%%%
\subsection{Conditional Processing}
\label{sec:conditional}

The package provides a mechanism to compile different versions
of a document. To customise the versions further some conditional processing
can come in handy to distinguish which version is being compiled.
The package provides two macros to describe the compilation context:

%%%%%%%%%%%%%%%%%%%%%%%%%%%%%%%%%%%%%%%%
\DescribeMacro{\ifchilddoc}
The conditional |\ifchilddoc| distinguishes between the compilation of
child documents and the main document:
%
\begin{center}
|\ifchilddoc |\textit{child-code}| |[|\||else |\textit{main-code}]| \||fi|
\end{center}

%%%%%%%%%%%%%%%%%%%%%%%%%%%%%%%%%%%%%%%%
\DescribeMacro{\childdocname}
\DescribeMacro{\childdocjob}
The macro |\childdocname| contains the filename (without extension)
of the main or child file being processed.
Note that |\childdocjob| will always contain the name of the main file.

%%%%%%%%%%%%%%%%%%%%%%%%%%%%%%%%%%%%%%%%
\paragraph{Title Page.}

Conditional processing can be used to include a title or banner page
in the main document when proper precautions are taken.
Importantly, the code in the main file should ensure that the page counter
(as well as other status parameters which are stored in the |.aux| files)
takes the same value after the conditional processing.
Otherwise the page numbers may take divergent values
depending on which part is compiled.

For example, a title page could be declared by:
%
\begin{center}
\begin{tabular}{l}
|\ifchilddoc\||else|\\
|\addtocounter{page}{-1}|\\
\textit{code for title page}\\
|\newpage|\\
|\||fi|
\end{tabular}
\end{center}
%
A banner page for the child documents can be generated by:
%
\begin{center}
\begin{tabular}{l}
|\ifchilddoc|\\
|\addtocounter{page}{-1}|\\
\textit{code for banner page}\\
|\newpage|\\
|\||fi|
\end{tabular}
\end{center}
%
Here one could write a message such as:
\begin{center}
|This is the part \childdocname{} of \childdocjob{}.|
\end{center}

%%%%%%%%%%%%%%%%%%%%%%%%%%%%%%%%%%%%%%%%%%%%%%%%%%%%%%%%%%%%%%%%%%%%%%%%%%%%%%%%
\subsection{Flags}
\label{sec:flags}

The package makes it easy to generate different versions
of the main or child documents.
To this end compilation flags can be defined
and assigned different default values.
They will be particularly useful in conjunction
with the forwarding mechanism described in \secref{sec:forward}.

For example, it may be useful to have a flag |\version|
which can be set to |draft| or |final|.
The document source will contain some conditional code
depending on the value of |\version|.
Suppose further, the flag should default to |final| for the main file
and to |draft| for child files
which is a natural assignment for editing the document.
This is achieved by placing the following code
in the preamble of the main document
(below the |\childdocmain| directive):
%
\begin{center}
\begin{tabular}{l}
|\ifchilddoc|\\
|\providecommand{\version}{draft}|\\
|\||else|\\
|\providecommand{\version}{final}|\\
|\||fi|
\end{tabular}
\end{center}
%
The definition by |\providecommand| makes sure
that previous definitions are not overwritten.
Further statements |\providecommand{\version}{...}|
can thus be added before the above code to override it.

For the main file, one might add a line
(between |\childdocmain| and the above block)
%
\begin{center}
|%\ifchilddoc\||else\providecommand{\version}{draft}\||fi|
\end{center}
%
which can be uncommented to produce a draft version.
Likewise one can add a line to the very top of a child file
(above the |\childdocof{|\textit{main}|}| directive)
%
\begin{center}
|%\providecommand{\version}{final}|
\end{center}
%
which can be uncommented to produce the final version of this child document.

%%%%%%%%%%%%%%%%%%%%%%%%%%%%%%%%%%%%%%%%%%%%%%%%%%%%%%%%%%%%%%%%%%%%%%%%%%%%%%%%
\subsection{Forwarding}
\label{sec:forward}

Different versions of the main or child documents
using compilation flags as described in \secref{sec:flags}
can be (permanently) stored in different files
for convenient compilation, viewing and distribution.
To this end, the package defines a command
to pass on compilation to a different file:

%%%%%%%%%%%%%%%%%%%%%%%%%%%%%%%%%%%%%%%%
\DescribeMacro{\childdocforward}
The command |\childdocforward| redirects processing to
another source file:
%
\begin{center}
\begin{tabular}{l}
|\input{childdoc.def}|\\
|\childdocforward[|\textit{main}|]{|\textit{dest}|}|\\
\end{tabular}
\end{center}
%
The argument \textit{dest} is the destination file
(without extension).
It should be the main file or one of the child files.
Note that further \textsf{childdoc} directives
such as |\childdocof| and |\childdocforward|
in the indicated file will be processed in this form.
The optional argument \textit{main}
passes on directly to the main file \textit{main}
while pretending to compile the child \textit{dest}.
This form behaves as if \textit{dest}
issues |\childdocof{|\textit{main}|}| right away,
and no further \textsf{childdoc} directives will be processed.

%%%%%%%%%%%%%%%%%%%%%%%%%%%%%%%%%%%%%%%%
\DescribeMacro{\...prefix}
In the alternative form |\childdocforwardprefix|,
%
\begin{center}
\begin{tabular}{l}
|\input{childdoc.def}|\\
|\childdocforwardprefix[|\textit{main}|]{|\textit{prefix}|}{|\textit{dest}|}|
\end{tabular}
\end{center}
%
the destination file is determined by a pattern
depending on the current file:
To make this work, the current file must be called
`{\textit{prefix}\hspace{0.2em}\textit{suffix}}'
with \textit{prefix} matching precisely the argument.
Processing is then passed on to the file
`{\textit{dest}\hspace{0.2em}\textit{suffix}}'.
Surely, the same effect is achieved by
directly specifying the
argument `{\textit{dest}\hspace{0.2em}\textit{suffix}}'
in the first form.
However, that requires to set up a different file
for each child. With the alternative form of the command
all these files can have exactly the same content
which simplifies setting them up and maintaining them.

For example, the following file |draft.tex|
with a compilation flag |\version| as described in \secref{sec:flags}
compiles the main document as a draft:
%
\begin{center}
\begin{tabular}{l}
|\def\version{draft}|\\
|\input{childdoc.def}|\\
|\childdocforward{|\textit{main}|}|
\end{tabular}
\end{center}
%
Likewise, the following files |final|\textit{nn}|.tex|
compile the final version of the child document
|child|\textit{nn}|.tex|:
%
\begin{center}
\begin{tabular}{l}
|\def\version{final}|\\
|\input{childdoc.def}|\\
|\childdocforwardprefix{final}{child}|
\end{tabular}
\end{center}
%

Note that when several versions of a main file and/or of each child file
are to be generated, it may be convenient to set up a |Makefile| or
shell script to automatise the process.

%%%%%%%%%%%%%%%%%%%%%%%%%%%%%%%%%%%%%%%%%%%%%%%%%%%%%%%%%%%%%%%%%%%%%%%%%%%%%%%%
\subsection{Command Line Processing}
\label{sec:commandline}

The effect of redirection files can also be achieved by invoking
the \LaTeX{} compiler with a more elaborate command line.
Most conveniently this should be done as part
of a shell script or a |Makefile|.

When using \textsf{childdoc} in the main file, the following
command lines effectively perform a redirection
(note that depending on the shell being used,
backslashes may have to be doubled: `|\|' $\to$ `|\\|'):
%
\begin{center}
|... -jobname "|\textit{target}|" |\\|"|[\textit{flags}]%
|\input{childdoc.def}\childdocforward[|\textit{main}|]{|\textit{dest}|}"|
\end{center}
%
Here \textit{target} is the name of the output file,
\textit{main} is the name of the main file
and \textit{dest} is the name of the main or child file to be processed
(all filenames without extensions).
The optional argument \textit{main} can be omitted
if \textit{main} matches \textit{dest}.
Optionally, compilation \textit{flags} can be defined via |\def| commands.
This command line makes the \TeX{} engine believe
it is compiling the file \textit{target}
whose content is specified as the latter parameter.
The provided code then forwards the processing to
\textit{main} or \textit{dest} as described in \secref{sec:forward}.

%%%%%%%%%%%%%%%%%%%%%%%%%%%%%%%%%%%%%%%%%%%%%%%%%%%%%%%%%%%%%%%%%%%%%%%%%%%%%%%%
\subsection{Include by Input}
\label{sec:input}

Including child documents by |\include| has some restrictions by design.
Most notably, the content of a child document always occupies
its own set of pages; pages cannot be shared between child documents.
Usually, this behaviour makes perfect sense
because each child document contain an essential part of the document.
However, in some situations it may be desirable to compose
a document from a collection of parts
without having mandatory page breaks between then.
For this case, the package
provides a mechanism to include parts
by |\input| which can also be processed individually.
However, by construction this mechanism
requires manual handling of the content to be output.

%%%%%%%%%%%%%%%%%%%%%%%%%%%%%%%%%%%%%%%%
\DescribeMacro{\ifchilddocmanual}
The main file should be prepared as usual, see \secref{sec:include}.
However, the document body must make a distinction
between processing of an individual part and of the main document, e.g.:
%
\begin{center}
\begin{tabular}{l}
|\ifchilddocmanual|\\
|\input{\childdocname}|\\
|\||else|\\
\textit{document body with }|\input{|\textit{part}|}|\\
|\||fi|
\end{tabular}
\end{center}
%
The conditional |\ifchilddocmanual| is true whenever
a part to be included by |\input| is being compiled,
and the name of the part is stored in |\childdocname|.

%%%%%%%%%%%%%%%%%%%%%%%%%%%%%%%%%%%%%%%%
\DescribeMacro{\childdocby}
Each part to be included by |\input| should start with:
%
\begin{center}
\begin{tabular}{l}
|\input{childdoc.def}|\\
|\childdocby{|\textit{main}|}|\\
\end{tabular}
\end{center}
%
The directive |\childdocby| is similar to |\childdocof|
described in \secref{sec:include},
but the subsequent selection of content must be done manually.
To that end, both |\ifchilddoc| and |\ifchilddocmanual|
will be true upon processing of a part,
and the name of the part is stored in |\childdocname|.
Note that |\jobname| will be set to the filename of the current part
so that each part receives an individual |.aux| file
that does not interfere with the |.aux| file(s) of the main document.
This behaviour can be altered by the alternative form
|\childdocby[*]{|\textit{main}|}| (with a non-empty optional argument)
which uses the |.aux| file of the main document
by setting |\jobname| to \textit{main}.

%%%%%%%%%%%%%%%%%%%%%%%%%%%%%%%%%%%%%%%%%%%%%%%%%%%%%%%%%%%%%%%%%%%%%%%%%%%%%%%%
\subsection{Driver Development}
\label{sec:driver}

The \textsf{childdoc} mechanism can also be use for the development
of definition files such as \LaTeX{} styles or classes.
This case differs from the above setup with multiple parts
included by |\include| in that no |\includeonly| should be invoked.
This can be achieved by starting the include file
(before |\ProvidesPackage|) with:
%
\begin{center}
\begin{tabular}{l}
|\input{childdoc.def}|\\
|\childdocforward{|\textit{main}|}|\\
\end{tabular}
\end{center}
%
or alternatively with:
%
\begin{center}
\begin{tabular}{l}
|\input{childdoc.def}|\\
|\childdocby{|\textit{main}|}|\\
\end{tabular}
\end{center}
%
Both forms have slightly different effects as described above.
The main file is prepared as usual, see \secref{sec:include}.

%%%%%%%%%%%%%%%%%%%%%%%%%%%%%%%%%%%%%%%%%%%%%%%%%%%%%%%%%%%%%%%%%%%%%%%%%%%%%%%%
\subsection{Legacy Detection}
\label{sec:detection}

The directive |\childdocmain| in the main file can detect
whether the complete document or merely a child is to be compiled
even without using the directive |\childdocof|.
This method is deprecated because it is less robust
and there is no compelling reason to use it;
it is merely provided for backward compatibility
and it may be removed in future versions.

If the detection mechanism is to be used,
it is mandatory to correctly specify
the filename of the main file as the argument of |\childdocmain|:
%
\begin{center}
\begin{tabular}{l}
|\input{childdoc.def}|\\
|\childdocmain{|\textit{main}|}|\\
\end{tabular}
\end{center}
%
If |\jobname| does not match the argument \textit{main} of |\childdocmain|,
it is assumed that |\jobname| points to the child file to be compiled.
When using |\childdocmain| with the main file specified as argument,
it suffices to start a child file
with just |\input{|\textit{main}|}|
without loading of the package and using |\childdocof|.
If instead all processing is done
with the appropriate \textsf{childdoc} directives,
the argument of \textit{main} of |\childdocmain| can be empty.

An alternative version of the command line processing described
in \secref{sec:commandline} using the detection mechanism reads:
%
\begin{center}
|... -jobname "|\textit{target}|" "|[\textit{flags}]%
[|\def\jobname{|\textit{dest}|}|]|\input{|\textit{main}|}"|
\end{center}

%%%%%%%%%%%%%%%%%%%%%%%%%%%%%%%%%%%%%%%%%%%%%%%%%%%%%%%%%%%%%%%%%%%%%%%%%%%%%%%%
\subsection{Manual Code}
\label{sec:manual}

In case one cannot be certain whether the definitions file |childdoc.def|
is installed on the target \TeX{} distribution
and one prefers not to ship it,
it is conceivable to paste a few relevant commands into the sources.

To that end, drop all statements |\input{childdoc.def}|
and perform the replacements as outlined below.
Instead of |\childdocmain{|\textit{main}|}| add the following code
to the top of the main file:
%
\begin{center}
\begin{tabular}{l}
|\||ifdefined\childdocname\endinput\||fi\newif\ifchilddoc|\\
|\edef\childdocname{\scantokens\expandafter{\jobname\noexpand}}|\\
|\def\childdocmain{|\textit{main}|}\||ifx\childdocmain\childdocname\||else|\\
|\childdoctrue\includeonly{\childdocname}\let\jobname\childdocmain\||fi|\\
\end{tabular}
\end{center}
%
Instead of |\childdocof{|\textit{main}|}| just include the main file
at the top of each child file:
%
\begin{center}
|\input{|\textit{main}|}|
\end{center}
%
A simple redirection |\childdocforward{|\textit{dest}|}| is achieved by:
%
\begin{center}
|\def\jobname{|\textit{dest}|}\input{\jobname}|
\end{center}
%
The redirection with prefix
|\childdocforwardprefix[|\textit{prefix}|]{|\textit{dest}|}|
is accomplished by:
%
\begin{center}
\begin{tabular}{l}
|{\edef\jobname{\scantokens\expandafter{\jobname\noexpand}}|\\
|\def\redirectjob |\textit{prefix}|#1~~~{\gdef\jobname{|\textit{dest}|#1}}|\\
|\expandafter\redirectjob\jobname~~~}\input{\jobname}|
\end{tabular}
\end{center}

In an alternative approach,
child documents can be compiled by a specific command line
without additional code or specific definitions:
%
\begin{center}
|... -jobname "|\textit{target}|" "|[\textit{flags}]%
|\includeonly{|\textit{dest}|}\input{|\textit{main}|}"|
\end{center}
%

%%%%%%%%%%%%%%%%%%%%%%%%%%%%%%%%%%%%%%%%%%%%%%%%%%%%%%%%%%%%%%%%%%%%%%%%%%%%%%%%
%%%%%%%%%%%%%%%%%%%%%%%%%%%%%%%%%%%%%%%%%%%%%%%%%%%%%%%%%%%%%%%%%%%%%%%%%%%%%%%%
\section{Information}

%%%%%%%%%%%%%%%%%%%%%%%%%%%%%%%%%%%%%%%%%%%%%%%%%%%%%%%%%%%%%%%%%%%%%%%%%%%%%%%%
\subsection{Copyright}

Copyright \copyright{} 2017--2018 Niklas Beisert

This work may be distributed and/or modified under the
conditions of the \LaTeX{} Project Public License, either version 1.3
of this license or (at your option) any later version.
The latest version of this license is in
  \url{http://www.latex-project.org/lppl.txt}
and version 1.3 or later is part of all distributions of \LaTeX{}
version 2005/12/01 or later.

This work has the LPPL maintenance status `maintained'.

The Current Maintainer of this work is Niklas Beisert.

This work consists of the files |README.txt|, |childdoc.ins| and |childdoc.dtx|
as well as the derived files |childdoc.def|, |cdocsamp.tex|
with |cdocsch1.tex|, |cdocsch2.tex|, |cdocspt3.tex|, |cdocspt4.tex|,
|cdocsdrf.tex|, |cdocsfn1.tex|, |cdocsfn2.tex|
as well as |childdoc.pdf|.

%%%%%%%%%%%%%%%%%%%%%%%%%%%%%%%%%%%%%%%%%%%%%%%%%%%%%%%%%%%%%%%%%%%%%%%%%%%%%%%%
\subsection{Files and Installation}

The package consists of the files:
%
\begin{center}
\begin{tabular}{ll}
    |README.txt|   & readme file \\
    |childdoc.ins| & installation file \\
    |childdoc.dtx| & source file \\
    |childdoc.def| & definition file \\
    |cdocsamp.tex| & sample main file \\
    |cdocsch1.tex| & sample include file \\
    |cdocsch2.tex| & sample include file \\
    |cdocspt3.tex| & sample part file \\
    |cdocspt4.tex| & sample part file \\
    |cdocsdrf.tex| & sample redirection file \\
    |cdocsfn1.tex| & sample redirection file \\
    |cdocsfn2.tex| & sample redirection file \\
    |childdoc.pdf| & manual
\end{tabular}
\end{center}
%
The distribution consists of the files
|README.txt|, |childdoc.ins| and |childdoc.dtx|.
%
\begin{itemize}
\item
Run (pdf)\LaTeX{} on |childdoc.dtx|
to compile the manual |childdoc.pdf| (this file).
\item
Run \LaTeX{} on |childdoc.ins| to create the definitions file |childdoc.def|
and the sample |cdocsamp.tex| with include files
|cdocsch1.tex|, |cdocsch2.tex|, |cdocspt3.tex|, |cdocspt4.tex|,
|cdocsdrf.tex|, |cdocsfn1.tex|, |cdocsfn2.tex|.
Then copy the file |childdoc.def| to an appropriate directory of your \LaTeX{}
distribution, e.g.\ \textit{texmf-root}|/tex/latex/childdoc|.
\end{itemize}

%%%%%%%%%%%%%%%%%%%%%%%%%%%%%%%%%%%%%%%%%%%%%%%%%%%%%%%%%%%%%%%%%%%%%%%%%%%%%%%%
\subsection{Related CTAN Packages}

There are several other packages which offer a similar functionality:
%
\begin{itemize}
\item
The packages
\href{http://ctan.org/pkg/docmute}{\textsf{docmute}},
\href{http://ctan.org/pkg/includex}{\textsf{includex}} and
\href{http://ctan.org/pkg/standalone}{\textsf{standalone}}
provide commands to include only the document body of
a child file thus allowing both files to be compiled individually.
\item
The packages \href{http://ctan.org/pkg/subdocs}{\textsf{subdocs}}
and \href{http://ctan.org/pkg/subfiles}{\textsf{subfiles}}
provide structures in which the main and child documents can be
encapsulated and allowing them to be compiled individually.
The inclusion mechanism is different from the conventional |\include|.
\item
The package \href{http://ctan.org/pkg/combine}{\textsf{combine}}
is an elaborate solution to combine several documents into one.
\end{itemize}
%
See also the CTAN topic \href{http://ctan.org/topic/subdocs}{\textsf{subdocs}}
for further related packages.
The present package differs from the above solutions in that
a document structure constructed with the conventional |\include| mechanism
just needs two extra commands at the top of every file
such that all constituent files can be compiled individually.

%%%%%%%%%%%%%%%%%%%%%%%%%%%%%%%%%%%%%%%%%%%%%%%%%%%%%%%%%%%%%%%%%%%%%%%%%%%%%%%%
%\subsection{Feature Suggestions}
%
%The following is a list of features which may be useful for future
%versions of this package:
%%
%\begin{itemize}
%\item
%\ldots
%\end{itemize}

%%%%%%%%%%%%%%%%%%%%%%%%%%%%%%%%%%%%%%%%%%%%%%%%%%%%%%%%%%%%%%%%%%%%%%%%%%%%%%%%
\subsection{Revision History}

%%%%%%%%%%%%%%%%%%%%%%%%%%%%%%%%%%%%%%%%
\paragraph{v2.0:} 2018/12/30

\begin{itemize}
\item
immediate forward processing
\item
added |\childdocby| mechanism
\item
manual restructured
\end{itemize}

%%%%%%%%%%%%%%%%%%%%%%%%%%%%%%%%%%%%%%%%
\paragraph{v1.6:} 2018/01/17

\begin{itemize}
\item
application for development of include files
\item
corrections to manual
\end{itemize}

%%%%%%%%%%%%%%%%%%%%%%%%%%%%%%%%%%%%%%%%
\paragraph{v1.5:} 2017/05/21

\begin{itemize}
\item
more complete structuring introduced
\item
|\childdocof| introduced
\item
|\childdoc| renamed to |\childdocmain|
\item
|\childredirect| renamed to |\childdocforward| and |\childdocforwardprefix|
and functionality expanded
\end{itemize}

%%%%%%%%%%%%%%%%%%%%%%%%%%%%%%%%%%%%%%%%
\paragraph{v1.0:} 2017/04/27

\begin{itemize}
\item
manual and install package
\item
first version published on CTAN
\end{itemize}

%%%%%%%%%%%%%%%%%%%%%%%%%%%%%%%%%%%%%%%%
\paragraph{v0.6:} 2017/04/26

\begin{itemize}
\item
redirection mechanism added
\end{itemize}

%%%%%%%%%%%%%%%%%%%%%%%%%%%%%%%%%%%%%%%%
\paragraph{v0.5:} 2017/04/26

\begin{itemize}
\item
functionality in definition file
\end{itemize}


%%%%%%%%%%%%%%%%%%%%%%%%%%%%%%%%%%%%%%%%%%%%%%%%%%%%%%%%%%%%%%%%%%%%%%%%%%%%%%%%
%%%%%%%%%%%%%%%%%%%%%%%%%%%%%%%%%%%%%%%%%%%%%%%%%%%%%%%%%%%%%%%%%%%%%%%%%%%%%%%%
%%%%%%%%%%%%%%%%%%%%%%%%%%%%%%%%%%%%%%%%%%%%%%%%%%%%%%%%%%%%%%%%%%%%%%%%%%%%%%%%
\appendix

\settowidth\MacroIndent{\rmfamily\scriptsize 000\ }

 \DocInput{childdoc.dtx}

\end{document}
%</driver>
% \fi
%
% %%%%%%%%%%%%%%%%%%%%%%%%%%%%%%%%%%%%%%%%%%%%%%%%%%%%%%%%%%%%%%%%%%%%%%%%%%%%%%
% %%%%%%%%%%%%%%%%%%%%%%%%%%%%%%%%%%%%%%%%%%%%%%%%%%%%%%%%%%%%%%%%%%%%%%%%%%%%%%
% \section{Sample}
%\iffalse
%<*samplemain>
%\fi
%
% The following presents a sample document
% with two chapters, two parts, a title page,
% a compile flag as well as three forwarding files to set the flag.
% It consists of eight |.tex| files:
% \begin{center}
% \begin{tabular}{ll}
% |cdocsamp.tex|&main file\\
% |cdocsch1.tex|&include file for chapter 1\\
% |cdocsch2.tex|&include file for chapter 2\\
% |cdocspt3.tex|&include file for part 3\\
% |cdocspt4.tex|&include file for part 4\\
% |cdocsdrf.tex|&forwarding file for main file in draft mode\\
% |cdocsfi1.tex|&forwarding file for final version of chapter 1\\
% |cdocsfi2.tex|&forwarding file for final version of chapter 2\\
% \end{tabular}
% \end{center}
% Each of the eight files can be compiled directly by the \LaTeX{} compiler.
%
% %%%%%%%%%%%%%%%%%%%%%%%%%%%%%%%%%%%%%%
% \paragraph{Main File.}
%
% The main file is called |cdocsamp.tex|.
%
% Load the \textsf{childdoc} definitions and
% declare the filename for the main document:
%    \begin{macrocode}
\input{childdoc.def}
\childdocmain{}
%    \end{macrocode}

% Optional override for |\version| flag:
%    \begin{macrocode}
%%\ifchilddoc\else\providecommand{\version}{draft}\fi
%    \end{macrocode}

% Define the default values for the |\version| flag
% (|final| for the main file and |draft| for childs):
%    \begin{macrocode}
\ifchilddoc
\providecommand{\version}{draft}
\else
\providecommand{\version}{final}
\fi
%    \end{macrocode}

% Load the standard document class:
%    \begin{macrocode}
\documentclass[12pt]{article}
%    \end{macrocode}

% Start the document body:
%    \begin{macrocode}
\begin{document}
%    \end{macrocode}

% Declare a title page.
% Print title, part of document being processed and version flag:
%    \begin{macrocode}
\addtocounter{page}{-1}
\begin{center}
{\LARGE\bfseries{}childdoc example\par}
\vspace{1cm}
\ifchilddoc
\ifchilddocmanual part\else chapter\fi:
`\childdocname' of `\childdocjob'\par
\else
main document: `\childdocjob'\par
\fi
version: \version\par
\end{center}
\newpage
%    \end{macrocode}

% Manually include selected file,
% otherwise process as usual:
%    \begin{macrocode}
\ifchilddocmanual
\section*{part `\childdocname'}
\input{\childdocname}
\else
%    \end{macrocode}

% Include the two chapters:
%    \begin{macrocode}
\include{cdocsch1}
\include{cdocsch2}
%    \end{macrocode}

% Include the two parts unless only chapters should be displayed:
%    \begin{macrocode}
\ifchilddoc\else
\section{part three}
\input{cdocspt3}
\section{part four}
\input{cdocspt4}
\fi
%    \end{macrocode}

% Process as usual until here:
%    \begin{macrocode}
\fi
%    \end{macrocode}

% End of document body:
%    \begin{macrocode}
\end{document}
%    \end{macrocode}
%\iffalse
%</samplemain>
%\fi
%
% %%%%%%%%%%%%%%%%%%%%%%%%%%%%%%%%%%%%%%
% \paragraph{Chapter Include Files.}
%
% The include files are called |cdocsch1.tex| and |cdocsch2.tex|.
%
%\iffalse
%<*samplechap1|samplechap2>
%\fi

% Optional override for |\version| flag:
%    \begin{macrocode}
%%\providecommand{\version}{final}
%    \end{macrocode}

% Include the main document:
%    \begin{macrocode}
\input{childdoc.def}
\childdocof{cdocsamp}
%    \end{macrocode}

%\iffalse
%</samplechap1|samplechap2>
%\fi
%
%\iffalse
%<*samplechap1>
%\fi
% Some text for chapter 1:
%    \begin{macrocode}
\section{one}
some text in chapter one
%    \end{macrocode}

%\iffalse
%</samplechap1>
%\fi
% Some text for chapter 2:
%\iffalse
%<*samplechap2>
%\fi
%    \begin{macrocode}
\section{two}
more text in chapter two
%    \end{macrocode}

%\iffalse
%</samplechap2>
%\fi
%
% %%%%%%%%%%%%%%%%%%%%%%%%%%%%%%%%%%%%%%
% \paragraph{Part Include Files.}
%
% The include files are called |cdocspt3.tex| and |cdocspt4.tex|.
%
%\iffalse
%<*samplepart3|samplepart4>
%\fi

% Optional override for |\version| flag:
%    \begin{macrocode}
%%\providecommand{\version}{final}
%    \end{macrocode}

% Include the main document:
%    \begin{macrocode}
\input{childdoc.def}
\childdocby{cdocsamp}
%    \end{macrocode}

%\iffalse
%</samplepart3|samplepart4>
%\fi
%
%\iffalse
%<*samplepart3>
%\fi
% Some text for part 3:
%    \begin{macrocode}
some text in part three
%    \end{macrocode}

%\iffalse
%</samplepart3>
%\fi
% Some text for part 4:
%\iffalse
%<*samplepart4>
%\fi
%    \begin{macrocode}
more text in part four
%    \end{macrocode}

%\iffalse
%</samplepart4>
%\fi
%
% %%%%%%%%%%%%%%%%%%%%%%%%%%%%%%%%%%%%%%
% \paragraph{Forwarding for a Complete Draft.}
%
% The following forwarding file |cdocsdrf.tex|
% compiles the main document in draft mode:
%\iffalse
%<*sampledraft>
%\fi
%    \begin{macrocode}
\def\version{draft}
\input{childdoc.def}
\childdocforward{cdocsamp}
%    \end{macrocode}

%\iffalse
%</sampledraft>
%\fi
%
% %%%%%%%%%%%%%%%%%%%%%%%%%%%%%%%%%%%%%%
% \paragraph{Forwarding for Final Version of the Chapters.}
%
% The following forwarding files |cdocsfn1.tex| and |cdocsfn2.tex|
% (with identical content)
% compile the final versions of the child documents
% |cdocsch1.tex| and |cdocsch2.tex|, respectively:
%\iffalse
%<*samplefinal>
%\fi
%    \begin{macrocode}
\def\version{final}
\input{childdoc.def}
\childdocforwardprefix[cdocsamp]{cdocsfn}{cdocsch}
%    \end{macrocode}

%\iffalse
%</samplefinal>
%\fi
%
% %%%%%%%%%%%%%%%%%%%%%%%%%%%%%%%%%%%%%%
% \paragraph{Command Line Processing.}
%
% The following three command lines generate the output files
% |cdocscld|, |cdocscl1| and |cdocscl2|
% which should be identical to
% |cdocsdrf|, |cdocsch1| and |cdocsfn2|, respectively:
% \begin{center}
% \begin{tabular}{l}
% |latex -jobname cdocscld \|\\
% |  "\def\version{draft}\input{childdoc.def}\childdocforward{cdocsamp}"|\\
% |latex -jobname cdocscl1 \|\\
% |  "\input{childdoc.def}\childdocforward[cdocsamp]{cdocsch1}"|\\
% |latex -jobname cdocscl2 \|\\
% |  "\def\version{final}\input{childdoc.def}\childdocforward{cdocsch2}"|
% \end{tabular}
% \end{center}
% Note that the trailing backslash on each first line
% merely continues the input to the second line
% (for convenient cut ant paste).
% Furthermore, the command |latex| can be replaced by any
% of its alternative versions such as |pdflatex|.
%
% %%%%%%%%%%%%%%%%%%%%%%%%%%%%%%%%%%%%%%%%%%%%%%%%%%%%%%%%%%%%%%%%%%%%%%%%%%%%%%
% %%%%%%%%%%%%%%%%%%%%%%%%%%%%%%%%%%%%%%%%%%%%%%%%%%%%%%%%%%%%%%%%%%%%%%%%%%%%%%
% \section{Implementation}
%\iffalse
%<*package>
%\fi
%
% This section describes the definitions file |childdoc.def|.

% The definitions cannot be loaded using |\usepackage| or |\RequirePackage|
% which has a mechanism to prevent loading a style file more than once.
% When loading the definitions by means of |\input|
% multiple instances have to be prevented manually:
%\iffalse
%This code needs to be before the `\ProvidesFile' directive
%which is defined at the beginning of this file.
%Therefore it is also placed there and commented out here.
%</package>
%<*discard>
%\fi
%    \begin{macrocode}
\ifdefined\childdocmain\endinput\fi
%    \end{macrocode}
%\iffalse
%</discard>
%<*package>
%\fi
%
% \macro{\ifchilddoc}
% \macro{\ifchilddocmanual}
% The conditional |\ifchilddoc| tells whether a
% child (true) or main (false) document is being compiled.
% The conditional |\ifchilddocmanual| tells whether
% the |\includeonly| mechanism is used (false) or
% the selection of child files must be performed manually (true).
% The definitions initialise to false:
%    \begin{macrocode}
\newif\ifchilddoc
\newif\ifchilddocmanual
%    \end{macrocode}

% \macro{\childdocname}
% \macro{\childdocjob}
% The macro |\childdocname| stores the name of the main document
% to be compiled. The macro |\childdocjob| stores the name of
% the document on which the \LaTeX{} compiler was originally invoked.
% The content of |\jobname| cannot be compared
% to filenames specified in the source due to different catcodes.
% The following code rescans |\jobname|, stores the result
% in |\childdocname| and saves a copy in |\childdocjob|:
%    \begin{macrocode}
\edef\childdocname{\scantokens\expandafter{\jobname\noexpand}}
\let\childdocjob\childdocname
%    \end{macrocode}

% \macro{\childdocdisable}
% The macro |\childdocdisable| prevents the main file
% from being processed more than once.
% At this stage, the main document command |\childdocmain|
% is assumed to be called once again where it should do nothing.
% Any subsequent call to it should prevent
% a secondary processing of the main document
% It overwrites the forwarding commands
% |\childdocof| and |\childdocforward|
% with empty macros to prevent further inclusions of the main document:
%    \begin{macrocode}
\newcommand{\childdocdisable}
{
  \renewcommand{\childdocmain}[1]{\renewcommand{\childdocmain}[1]{\endinput}}
  \renewcommand{\childdocof}[1]{}
  \renewcommand{\childdocby}[2][]{}
  \renewcommand{\childdocforward}[2][]{}
  \renewcommand{\childdocdisable}{}
}
%    \end{macrocode}

% \macro{\childdocmain}
% The macro |\childdocmain| is to be called at the top of the main file
% with nothing or the main filename (without extension) as argument.
% First, it breaks loops.
% If the argument is not empty and does not match |\childdocname|
% (which is set by the first inclusion of |childdoc.def|),
% |\ifchilddoc| is set to true, |\includeonly| is applied to the child file
% and |\jobname| is set to the main file
% (for proper handling of |.aux| files):
%    \begin{macrocode}
\newcommand{\childdocmain}[1]
{
  \childdocdisable\childdocmain{}
  \if?#1?\else
    \begingroup
      \def\childdoctmp{#1}
      \ifx\childdoctmp\childdocname
        \def\childdoctmp{}
      \else
        \def\childdoctmp
        {
          \childdoctrue
          \includeonly{\childdocname}
          \def\childdocjob{#1}
          \def\jobname{#1}
        }
      \fi
      \expandafter
    \endgroup
    \childdoctmp
  \fi
}
%    \end{macrocode}

% \macro{\childdocof}
% The command |\childdocof| redirects
% compilation to the main file |#1|.
%    \begin{macrocode}
\newcommand{\childdocof}[1]
{
  \childdocdisable
  \childdoctrue
  \includeonly{\childdocname}
  \def\jobname{#1}
  \def\childdocjob{#1}
  \input{#1}
}
%    \end{macrocode}

% \macro{\childdocby}
% The command |\childdocby| ....
%    \begin{macrocode}
\newcommand{\childdocby}[2][]
{
  \childdocdisable
  \childdoctrue
  \childdocmanualtrue
  \if?#1?\else
    \def\jobname{#2}
  \fi
  \def\childdocjob{#2}
  \input{#2}
  \endinput
}
%    \end{macrocode}

% \macro{\childdocforward}
% The command |\childdocforward| redirects
% compilation to the main file or
% (if the optional argument is given) a child file.
% Parameters are set as if the main file
% or a child file starting with |\childdocof| was compiled.
% Then compilation is handed over to the main file:
%    \begin{macrocode}
\newcommand{\childdocforward}[2][]
{
  \begingroup
    \if?#1?
      \def\childdoctmp
      {
        \def\childdocname{#2}
        \def\childdocjob{#2}
        \def\jobname{#2}
        \input{#2}
        \endinput
      }
    \else
      \def\childdoctmp
      {
        \childdocdisable
        \def\childdocname{#2}
        \childdoctrue
        \includeonly{#2}
        \def\childdocjob{#1}
        \def\jobname{#1}
        \input{#1}
        \endinput
      }
    \fi
    \expandafter
  \endgroup
  \childdoctmp
}
%    \end{macrocode}

% \macro{\childdocforwardprefix}
% The command |\childdocforwardprefix| redirects
% compilation to the main or a child file by means of a pattern.
% The prefix |#1| in the current filename is replaced by |#2|
% and the suffix of the current filename is kept
% (it is assumed that the filename does not contain the substring `|~~~|'
% which is used as a delimiter).
% Compilation is handed over to the new file by |\childdocforward|:
%    \begin{macrocode}
\newcommand{\childdocforwardprefix}[3][]
{
  \begingroup
    \def\childdocextract #2##1~~~{\def\childdoctmp{\childdocforward[#1]{#3##1}}}
    \expandafter\childdocextract\childdocname~~~
    \expandafter
  \endgroup
  \childdoctmp
}
%    \end{macrocode}

% \macro{\childdoc}
% The deprecated macro |\childdoc| is a legacy version of |\childdocmain|:
%    \begin{macrocode}
\newcommand{\childdoc}{\childdocmain}
%    \end{macrocode}

% \macro{\childdocredirect}
% The deprecated macro |\childdocredirect| is a legacy version
% of |\childdocforward| and |\childdocforwardprefix|:
%    \begin{macrocode}
\newcommand{\childdocredirect}[2][]
{
  \begingroup
    \if?#1?
      \def\childdoctmp{\childdocforward{#2}}
    \else
      \def\childdoctmp{\childdocforwardprefix{#1}{#2}}
    \fi
    \expandafter
  \endgroup
  \childdoctmp
}
%    \end{macrocode}

%\iffalse
%</package>
%\fi
%
\endinput
|
and perform the replacements as outlined below.
Instead of |\childdocmain{|\textit{main}|}| add the following code
to the top of the main file:
%
\begin{center}
\begin{tabular}{l}
|\||ifdefined\childdocname\endinput\||fi\newif\ifchilddoc|\\
|\edef\childdocname{\scantokens\expandafter{\jobname\noexpand}}|\\
|\def\childdocmain{|\textit{main}|}\||ifx\childdocmain\childdocname\||else|\\
|\childdoctrue\includeonly{\childdocname}\let\jobname\childdocmain\||fi|\\
\end{tabular}
\end{center}
%
Instead of |\childdocof{|\textit{main}|}| just include the main file
at the top of each child file:
%
\begin{center}
|\input{|\textit{main}|}|
\end{center}
%
A simple redirection |\childdocforward{|\textit{dest}|}| is achieved by:
%
\begin{center}
|\def\jobname{|\textit{dest}|}\input{\jobname}|
\end{center}
%
The redirection with prefix
|\childdocforwardprefix[|\textit{prefix}|]{|\textit{dest}|}|
is accomplished by:
%
\begin{center}
\begin{tabular}{l}
|{\edef\jobname{\scantokens\expandafter{\jobname\noexpand}}|\\
|\def\redirectjob |\textit{prefix}|#1~~~{\gdef\jobname{|\textit{dest}|#1}}|\\
|\expandafter\redirectjob\jobname~~~}\input{\jobname}|
\end{tabular}
\end{center}

In an alternative approach,
child documents can be compiled by a specific command line
without additional code or specific definitions:
%
\begin{center}
|... -jobname "|\textit{target}|" "|[\textit{flags}]%
|\includeonly{|\textit{dest}|}\input{|\textit{main}|}"|
\end{center}
%

%%%%%%%%%%%%%%%%%%%%%%%%%%%%%%%%%%%%%%%%%%%%%%%%%%%%%%%%%%%%%%%%%%%%%%%%%%%%%%%%
%%%%%%%%%%%%%%%%%%%%%%%%%%%%%%%%%%%%%%%%%%%%%%%%%%%%%%%%%%%%%%%%%%%%%%%%%%%%%%%%
\section{Information}

%%%%%%%%%%%%%%%%%%%%%%%%%%%%%%%%%%%%%%%%%%%%%%%%%%%%%%%%%%%%%%%%%%%%%%%%%%%%%%%%
\subsection{Copyright}

Copyright \copyright{} 2017--2018 Niklas Beisert

This work may be distributed and/or modified under the
conditions of the \LaTeX{} Project Public License, either version 1.3
of this license or (at your option) any later version.
The latest version of this license is in
  \url{http://www.latex-project.org/lppl.txt}
and version 1.3 or later is part of all distributions of \LaTeX{}
version 2005/12/01 or later.

This work has the LPPL maintenance status `maintained'.

The Current Maintainer of this work is Niklas Beisert.

This work consists of the files |README.txt|, |childdoc.ins| and |childdoc.dtx|
as well as the derived files |childdoc.def|, |cdocsamp.tex|
with |cdocsch1.tex|, |cdocsch2.tex|, |cdocspt3.tex|, |cdocspt4.tex|,
|cdocsdrf.tex|, |cdocsfn1.tex|, |cdocsfn2.tex|
as well as |childdoc.pdf|.

%%%%%%%%%%%%%%%%%%%%%%%%%%%%%%%%%%%%%%%%%%%%%%%%%%%%%%%%%%%%%%%%%%%%%%%%%%%%%%%%
\subsection{Files and Installation}

The package consists of the files:
%
\begin{center}
\begin{tabular}{ll}
    |README.txt|   & readme file \\
    |childdoc.ins| & installation file \\
    |childdoc.dtx| & source file \\
    |childdoc.def| & definition file \\
    |cdocsamp.tex| & sample main file \\
    |cdocsch1.tex| & sample include file \\
    |cdocsch2.tex| & sample include file \\
    |cdocspt3.tex| & sample part file \\
    |cdocspt4.tex| & sample part file \\
    |cdocsdrf.tex| & sample redirection file \\
    |cdocsfn1.tex| & sample redirection file \\
    |cdocsfn2.tex| & sample redirection file \\
    |childdoc.pdf| & manual
\end{tabular}
\end{center}
%
The distribution consists of the files
|README.txt|, |childdoc.ins| and |childdoc.dtx|.
%
\begin{itemize}
\item
Run (pdf)\LaTeX{} on |childdoc.dtx|
to compile the manual |childdoc.pdf| (this file).
\item
Run \LaTeX{} on |childdoc.ins| to create the definitions file |childdoc.def|
and the sample |cdocsamp.tex| with include files
|cdocsch1.tex|, |cdocsch2.tex|, |cdocspt3.tex|, |cdocspt4.tex|,
|cdocsdrf.tex|, |cdocsfn1.tex|, |cdocsfn2.tex|.
Then copy the file |childdoc.def| to an appropriate directory of your \LaTeX{}
distribution, e.g.\ \textit{texmf-root}|/tex/latex/childdoc|.
\end{itemize}

%%%%%%%%%%%%%%%%%%%%%%%%%%%%%%%%%%%%%%%%%%%%%%%%%%%%%%%%%%%%%%%%%%%%%%%%%%%%%%%%
\subsection{Related CTAN Packages}

There are several other packages which offer a similar functionality:
%
\begin{itemize}
\item
The packages
\href{http://ctan.org/pkg/docmute}{\textsf{docmute}},
\href{http://ctan.org/pkg/includex}{\textsf{includex}} and
\href{http://ctan.org/pkg/standalone}{\textsf{standalone}}
provide commands to include only the document body of
a child file thus allowing both files to be compiled individually.
\item
The packages \href{http://ctan.org/pkg/subdocs}{\textsf{subdocs}}
and \href{http://ctan.org/pkg/subfiles}{\textsf{subfiles}}
provide structures in which the main and child documents can be
encapsulated and allowing them to be compiled individually.
The inclusion mechanism is different from the conventional |\include|.
\item
The package \href{http://ctan.org/pkg/combine}{\textsf{combine}}
is an elaborate solution to combine several documents into one.
\end{itemize}
%
See also the CTAN topic \href{http://ctan.org/topic/subdocs}{\textsf{subdocs}}
for further related packages.
The present package differs from the above solutions in that
a document structure constructed with the conventional |\include| mechanism
just needs two extra commands at the top of every file
such that all constituent files can be compiled individually.

%%%%%%%%%%%%%%%%%%%%%%%%%%%%%%%%%%%%%%%%%%%%%%%%%%%%%%%%%%%%%%%%%%%%%%%%%%%%%%%%
%\subsection{Feature Suggestions}
%
%The following is a list of features which may be useful for future
%versions of this package:
%%
%\begin{itemize}
%\item
%\ldots
%\end{itemize}

%%%%%%%%%%%%%%%%%%%%%%%%%%%%%%%%%%%%%%%%%%%%%%%%%%%%%%%%%%%%%%%%%%%%%%%%%%%%%%%%
\subsection{Revision History}

%%%%%%%%%%%%%%%%%%%%%%%%%%%%%%%%%%%%%%%%
\paragraph{v2.0:} 2018/12/30

\begin{itemize}
\item
immediate forward processing
\item
added |\childdocby| mechanism
\item
manual restructured
\end{itemize}

%%%%%%%%%%%%%%%%%%%%%%%%%%%%%%%%%%%%%%%%
\paragraph{v1.6:} 2018/01/17

\begin{itemize}
\item
application for development of include files
\item
corrections to manual
\end{itemize}

%%%%%%%%%%%%%%%%%%%%%%%%%%%%%%%%%%%%%%%%
\paragraph{v1.5:} 2017/05/21

\begin{itemize}
\item
more complete structuring introduced
\item
|\childdocof| introduced
\item
|\childdoc| renamed to |\childdocmain|
\item
|\childredirect| renamed to |\childdocforward| and |\childdocforwardprefix|
and functionality expanded
\end{itemize}

%%%%%%%%%%%%%%%%%%%%%%%%%%%%%%%%%%%%%%%%
\paragraph{v1.0:} 2017/04/27

\begin{itemize}
\item
manual and install package
\item
first version published on CTAN
\end{itemize}

%%%%%%%%%%%%%%%%%%%%%%%%%%%%%%%%%%%%%%%%
\paragraph{v0.6:} 2017/04/26

\begin{itemize}
\item
redirection mechanism added
\end{itemize}

%%%%%%%%%%%%%%%%%%%%%%%%%%%%%%%%%%%%%%%%
\paragraph{v0.5:} 2017/04/26

\begin{itemize}
\item
functionality in definition file
\end{itemize}


%%%%%%%%%%%%%%%%%%%%%%%%%%%%%%%%%%%%%%%%%%%%%%%%%%%%%%%%%%%%%%%%%%%%%%%%%%%%%%%%
%%%%%%%%%%%%%%%%%%%%%%%%%%%%%%%%%%%%%%%%%%%%%%%%%%%%%%%%%%%%%%%%%%%%%%%%%%%%%%%%
%%%%%%%%%%%%%%%%%%%%%%%%%%%%%%%%%%%%%%%%%%%%%%%%%%%%%%%%%%%%%%%%%%%%%%%%%%%%%%%%
\appendix

\settowidth\MacroIndent{\rmfamily\scriptsize 000\ }

 \DocInput{childdoc.dtx}

\end{document}
%</driver>
% \fi
%
% %%%%%%%%%%%%%%%%%%%%%%%%%%%%%%%%%%%%%%%%%%%%%%%%%%%%%%%%%%%%%%%%%%%%%%%%%%%%%%
% %%%%%%%%%%%%%%%%%%%%%%%%%%%%%%%%%%%%%%%%%%%%%%%%%%%%%%%%%%%%%%%%%%%%%%%%%%%%%%
% \section{Sample}
%\iffalse
%<*samplemain>
%\fi
%
% The following presents a sample document
% with two chapters, two parts, a title page,
% a compile flag as well as three forwarding files to set the flag.
% It consists of eight |.tex| files:
% \begin{center}
% \begin{tabular}{ll}
% |cdocsamp.tex|&main file\\
% |cdocsch1.tex|&include file for chapter 1\\
% |cdocsch2.tex|&include file for chapter 2\\
% |cdocspt3.tex|&include file for part 3\\
% |cdocspt4.tex|&include file for part 4\\
% |cdocsdrf.tex|&forwarding file for main file in draft mode\\
% |cdocsfi1.tex|&forwarding file for final version of chapter 1\\
% |cdocsfi2.tex|&forwarding file for final version of chapter 2\\
% \end{tabular}
% \end{center}
% Each of the eight files can be compiled directly by the \LaTeX{} compiler.
%
% %%%%%%%%%%%%%%%%%%%%%%%%%%%%%%%%%%%%%%
% \paragraph{Main File.}
%
% The main file is called |cdocsamp.tex|.
%
% Load the \textsf{childdoc} definitions and
% declare the filename for the main document:
%    \begin{macrocode}
% \iffalse
%
% childdoc.dtx Copyright (C) 2017-2018 Niklas Beisert
%
% This work may be distributed and/or modified under the
% conditions of the LaTeX Project Public License, either version 1.3
% of this license or (at your option) any later version.
% The latest version of this license is in
%   http://www.latex-project.org/lppl.txt
% and version 1.3 or later is part of all distributions of LaTeX
% version 2005/12/01 or later.
%
% This work has the LPPL maintenance status `maintained'.
%
% The Current Maintainer of this work is Niklas Beisert.
%
% This work consists of the files childdoc.dtx and childdoc.ins
% and the derived files childdoc.def and cdocsamp.tex with
% cdocsch1.tex, cdocsch2.tex, cdocsdrf.tex, cdocsfn1.tex, cdocsfn2.tex.
%
%<package>\ifdefined\childdocmain\endinput\fi
%<package>\ProvidesFile{childdoc.def}[2018/12/30 v2.0 child document driver]
%<samplemain>\ProvidesFile{cdocsamp.tex}[2018/12/30 v2.0 sample for childdoc]
%<*driver>
%\ProvidesFile{childdoc.drv}[2018/12/30 v2.0 childdoc reference manual file]
\PassOptionsToClass{10pt,a4paper}{article}
\documentclass{ltxdoc}

\usepackage[margin=35mm]{geometry}
\usepackage{hyperref}
\usepackage{hyperxmp}
\usepackage[usenames]{color}

\hypersetup{colorlinks=true}
\hypersetup{pdfstartview=FitH}
\hypersetup{pdfpagemode=UseNone}
\hypersetup{pdfsource={}}
\hypersetup{pdflang={en-UK}}
\hypersetup{pdfcopyright={Copyright 2017-2018 Niklas Beisert.
  This work may be distributed and/or modified under the
  conditions of the LaTeX Project Public License, either version 1.3
  of this license or (at your option) any later version.}}
\hypersetup{pdflicenseurl={http://www.latex-project.org/lppl.txt}}
\hypersetup{pdfcontactaddress={ETH Zurich, ITP, HIT K,
  Wolfgang-Pauli-Strasse 27}}
\hypersetup{pdfcontactpostcode={8093}}
\hypersetup{pdfcontactcity={Zurich}}
\hypersetup{pdfcontactcountry={Switzerland}}
\hypersetup{pdfcontactemail={nbeisert@itp.phys.ethz.ch}}
\hypersetup{pdfcontacturl={http://people.phys.ethz.ch/\xmptilde nbeisert/}}

\newcommand{\secref}[1]{\hyperref[#1]{section \ref*{#1}}}

\parskip1ex
\parindent0pt
\let\olditemize\itemize
\def\itemize{\olditemize\parskip0pt}

\begin{document}

\title{The \textsf{childdoc} Package}
\hypersetup{pdftitle={The childdoc Package}}
\author{Niklas Beisert\\[2ex]
  Institut f\"ur Theoretische Physik\\
  Eidgen\"ossische Technische Hochschule Z\"urich\\
  Wolfgang-Pauli-Strasse 27, 8093 Z\"urich, Switzerland\\[1ex]
  \href{mailto:nbeisert@itp.phys.ethz.ch}
  {\texttt{nbeisert@itp.phys.ethz.ch}}}
\hypersetup{pdfauthor={Niklas Beisert}}
\hypersetup{pdfsubject={Manual for the LaTeX2e Package childdoc}}
\date{30 December 2018, \textsf{v2.0}}
\maketitle

\begin{abstract}\noindent
\textsf{childdoc} is a \LaTeXe{} package
that enables the direct compilation
of document sections included by |\include|
to individual files.
\end{abstract}

\begingroup
\parskip0ex
\tableofcontents
\endgroup

%%%%%%%%%%%%%%%%%%%%%%%%%%%%%%%%%%%%%%%%%%%%%%%%%%%%%%%%%%%%%%%%%%%%%%%%%%%%%%%%
%%%%%%%%%%%%%%%%%%%%%%%%%%%%%%%%%%%%%%%%%%%%%%%%%%%%%%%%%%%%%%%%%%%%%%%%%%%%%%%%
\section{Introduction}

\LaTeX{} provides a mechanism to structure a large document (such as a book)
into a main file and several child files (containing the chapters)
using the |\include| command.
This mechanism is beneficial for documents
which span hundreds of pages in order to
make the source file(s) more manageable.
Moreover, compilation can be restricted to
selected child files by means of the |\includeonly| command.
The latter feature can be used to reduce the compilation time while editing
(this was significantly more useful in the earlier days of \LaTeX{})
or to generate a smaller document which is easier to navigate.
Another application of |\includeonly| is to generate
documents consisting of selected parts of the complete document.

However, there are a few drawbacks of the plain |\include| mechanism:
\begin{itemize}
\item
The child files cannot be compiled on their own,
they can only be compiled via the main file.
A naive editing environment
(such as a text editor with an option
to have the current file processed by \LaTeX)
may require one to switch to the main file before compiling;
attempting to compile the child file produces errors.
\item
The main file must be modified (each time)
to adjust the |\includeonly| command
to the present needs. This easily leaves the main file in a messy state.
\item
The generated document will always carry the filename
of the main document. This is inconvenient if
several child files are to be compiled and
to be kept for distribution.
\end{itemize}

The present package provides a simple interface
to make child files individually compilable by \LaTeX{}.
Compiling a child file then has the same effect as compiling
the main file with an |\includeonly| command
to select the appropriate child.
Moreover the generated document will carry the name of the child
rather than the main file.
This resolves all three above issues.

This feature is meant to make the editing of books,
thesis documents and lecture notes somewhat more convenient.
However, the package can also be used efficiently for
composing a series of documents (such as exercise sheets)
which are typically distributed individually.
It then assists the author in generating the individual documents
(potentially in different versions)
as well as a document containing the collected series.
Another application is in developing style files
or other kinds of included material
where compilation of the style file could redirect
to a sample or test file.

%%%%%%%%%%%%%%%%%%%%%%%%%%%%%%%%%%%%%%%%%%%%%%%%%%%%%%%%%%%%%%%%%%%%%%%%%%%%%%%%
%%%%%%%%%%%%%%%%%%%%%%%%%%%%%%%%%%%%%%%%%%%%%%%%%%%%%%%%%%%%%%%%%%%%%%%%%%%%%%%%
\section{Usage}

First of all, the package \textsf{childdoc} is \emph{not} a standard
\LaTeXe{} |.sty| style file! Therefore it needs to be invoked in
a non-standard way.

%%%%%%%%%%%%%%%%%%%%%%%%%%%%%%%%%%%%%%%%%%%%%%%%%%%%%%%%%%%%%%%%%%%%%%%%%%%%%%%%
\subsection{Included Files}
\label{sec:include}

%%%%%%%%%%%%%%%%%%%%%%%%%%%%%%%%%%%%%%%%
\DescribeMacro{\childdocmain}
To use the package, add the commands
\begin{center}
\begin{tabular}{l}
|\input{childdoc.def}|\\
|\childdocmain{}|\\
\end{tabular}
\end{center}
at the very top of the main \LaTeX{} file,
in particular \emph{before} the |\documentclass| statement!
The argument of |\childdocmain| should be left empty
(but it must be present).

%%%%%%%%%%%%%%%%%%%%%%%%%%%%%%%%%%%%%%%%
\DescribeMacro{\childdocof}
Furthermore, add the commands
\begin{center}
\begin{tabular}{l}
|\input{childdoc.def}|\\
|\childdocof{|\textit{main}|}|\\
\end{tabular}
\end{center}
at the top of every child file \textit{child}
which is included by |\include{|\textit{child}|}|
from within the main file
(or at least for those files to be compiled individually).
The argument \textit{main} must be the filename of the main file.

There are a couple of
considerations in setting up the main and child documents:

%%%%%%%%%%%%%%%%%%%%%%%%%%%%%%%%%%%%%%%%
\paragraph{Restrictions.}

Please note the following restrictions:
\begin{itemize}
\item
|\childdocmain| must be called with one argument \textit{main}
to ensure compatibility with earlier version of the package.
It must either be empty (|\childdocmain{}|)
or precisely match the filename of the main file in which it is specified.
See \secref{sec:detection} for further information.
\item
The filename \textit{main} must be specified without the |.tex| extension.
\item
The filename \textit{main} is case sensitive
(even in case-insensitive file systems)
due to internal string comparison.
\item
The argument \textit{main} should be fully expanded, it cannot be a macro.
\item
Subdirectories and special characters should be avoided in filenames.
\item
The command |\childdocmain{|\textit{main}|}| must be followed by a whitespace.
It should not be followed immediately by another command
or by a comment mark `|%|'.
This is because the \TeX{} parser reads the token immediately following
the argument of |\childdocmain| and puts it
at the beginning of every child section;
however, a white\-space is ignored.
\end{itemize}

%%%%%%%%%%%%%%%%%%%%%%%%%%%%%%%%%%%%%%%%
\paragraph{Content of Main File.}

It is advisable to place all content in the child files included by |\include|.
Any output contained in the main file will appear in all child documents
unless suppressed manually;
it cannot be suppressed automatically by the |\includeonly| directive
and thus should normally be avoided.
A method to include some content in the main file
by means of conditional processing is described in \secref{sec:conditional}.

%%%%%%%%%%%%%%%%%%%%%%%%%%%%%%%%%%%%%%%%
\paragraph{Page Numbering.}

When only a part of the document is compiled,
the appropriate numbering of pages
(as well as other status parameters)
is determined from the |.aux| files.
The latter contain information from previous passes.
However this information needs to propagate through
all intermediate child documents.
Therefore the page numbering in child documents may well
be inconsistent until the complete document is compiled at least once.

A useful (if unconventional) way to always ensure a consistent
page numbering is to restart the numbering in each child document
and denote the pages by `\textit{child}|.|\textit{page}'
where \textit{child} represents the chapter/section number of the child file.
This can be achieved by the command
|\numberwithin{page}{|\textit{child}|}|
of the \textsf{amsmath} package
where \textit{child} can be |chapter| or |section|
depending on the chosen structuring.
Alternatively, one can modify the macro |\thepage| appropriately
and reset the counter |page| at the start of each child file.

%%%%%%%%%%%%%%%%%%%%%%%%%%%%%%%%%%%%%%%%%%%%%%%%%%%%%%%%%%%%%%%%%%%%%%%%%%%%%%%%
\subsection{Conditional Processing}
\label{sec:conditional}

The package provides a mechanism to compile different versions
of a document. To customise the versions further some conditional processing
can come in handy to distinguish which version is being compiled.
The package provides two macros to describe the compilation context:

%%%%%%%%%%%%%%%%%%%%%%%%%%%%%%%%%%%%%%%%
\DescribeMacro{\ifchilddoc}
The conditional |\ifchilddoc| distinguishes between the compilation of
child documents and the main document:
%
\begin{center}
|\ifchilddoc |\textit{child-code}| |[|\||else |\textit{main-code}]| \||fi|
\end{center}

%%%%%%%%%%%%%%%%%%%%%%%%%%%%%%%%%%%%%%%%
\DescribeMacro{\childdocname}
\DescribeMacro{\childdocjob}
The macro |\childdocname| contains the filename (without extension)
of the main or child file being processed.
Note that |\childdocjob| will always contain the name of the main file.

%%%%%%%%%%%%%%%%%%%%%%%%%%%%%%%%%%%%%%%%
\paragraph{Title Page.}

Conditional processing can be used to include a title or banner page
in the main document when proper precautions are taken.
Importantly, the code in the main file should ensure that the page counter
(as well as other status parameters which are stored in the |.aux| files)
takes the same value after the conditional processing.
Otherwise the page numbers may take divergent values
depending on which part is compiled.

For example, a title page could be declared by:
%
\begin{center}
\begin{tabular}{l}
|\ifchilddoc\||else|\\
|\addtocounter{page}{-1}|\\
\textit{code for title page}\\
|\newpage|\\
|\||fi|
\end{tabular}
\end{center}
%
A banner page for the child documents can be generated by:
%
\begin{center}
\begin{tabular}{l}
|\ifchilddoc|\\
|\addtocounter{page}{-1}|\\
\textit{code for banner page}\\
|\newpage|\\
|\||fi|
\end{tabular}
\end{center}
%
Here one could write a message such as:
\begin{center}
|This is the part \childdocname{} of \childdocjob{}.|
\end{center}

%%%%%%%%%%%%%%%%%%%%%%%%%%%%%%%%%%%%%%%%%%%%%%%%%%%%%%%%%%%%%%%%%%%%%%%%%%%%%%%%
\subsection{Flags}
\label{sec:flags}

The package makes it easy to generate different versions
of the main or child documents.
To this end compilation flags can be defined
and assigned different default values.
They will be particularly useful in conjunction
with the forwarding mechanism described in \secref{sec:forward}.

For example, it may be useful to have a flag |\version|
which can be set to |draft| or |final|.
The document source will contain some conditional code
depending on the value of |\version|.
Suppose further, the flag should default to |final| for the main file
and to |draft| for child files
which is a natural assignment for editing the document.
This is achieved by placing the following code
in the preamble of the main document
(below the |\childdocmain| directive):
%
\begin{center}
\begin{tabular}{l}
|\ifchilddoc|\\
|\providecommand{\version}{draft}|\\
|\||else|\\
|\providecommand{\version}{final}|\\
|\||fi|
\end{tabular}
\end{center}
%
The definition by |\providecommand| makes sure
that previous definitions are not overwritten.
Further statements |\providecommand{\version}{...}|
can thus be added before the above code to override it.

For the main file, one might add a line
(between |\childdocmain| and the above block)
%
\begin{center}
|%\ifchilddoc\||else\providecommand{\version}{draft}\||fi|
\end{center}
%
which can be uncommented to produce a draft version.
Likewise one can add a line to the very top of a child file
(above the |\childdocof{|\textit{main}|}| directive)
%
\begin{center}
|%\providecommand{\version}{final}|
\end{center}
%
which can be uncommented to produce the final version of this child document.

%%%%%%%%%%%%%%%%%%%%%%%%%%%%%%%%%%%%%%%%%%%%%%%%%%%%%%%%%%%%%%%%%%%%%%%%%%%%%%%%
\subsection{Forwarding}
\label{sec:forward}

Different versions of the main or child documents
using compilation flags as described in \secref{sec:flags}
can be (permanently) stored in different files
for convenient compilation, viewing and distribution.
To this end, the package defines a command
to pass on compilation to a different file:

%%%%%%%%%%%%%%%%%%%%%%%%%%%%%%%%%%%%%%%%
\DescribeMacro{\childdocforward}
The command |\childdocforward| redirects processing to
another source file:
%
\begin{center}
\begin{tabular}{l}
|\input{childdoc.def}|\\
|\childdocforward[|\textit{main}|]{|\textit{dest}|}|\\
\end{tabular}
\end{center}
%
The argument \textit{dest} is the destination file
(without extension).
It should be the main file or one of the child files.
Note that further \textsf{childdoc} directives
such as |\childdocof| and |\childdocforward|
in the indicated file will be processed in this form.
The optional argument \textit{main}
passes on directly to the main file \textit{main}
while pretending to compile the child \textit{dest}.
This form behaves as if \textit{dest}
issues |\childdocof{|\textit{main}|}| right away,
and no further \textsf{childdoc} directives will be processed.

%%%%%%%%%%%%%%%%%%%%%%%%%%%%%%%%%%%%%%%%
\DescribeMacro{\...prefix}
In the alternative form |\childdocforwardprefix|,
%
\begin{center}
\begin{tabular}{l}
|\input{childdoc.def}|\\
|\childdocforwardprefix[|\textit{main}|]{|\textit{prefix}|}{|\textit{dest}|}|
\end{tabular}
\end{center}
%
the destination file is determined by a pattern
depending on the current file:
To make this work, the current file must be called
`{\textit{prefix}\hspace{0.2em}\textit{suffix}}'
with \textit{prefix} matching precisely the argument.
Processing is then passed on to the file
`{\textit{dest}\hspace{0.2em}\textit{suffix}}'.
Surely, the same effect is achieved by
directly specifying the
argument `{\textit{dest}\hspace{0.2em}\textit{suffix}}'
in the first form.
However, that requires to set up a different file
for each child. With the alternative form of the command
all these files can have exactly the same content
which simplifies setting them up and maintaining them.

For example, the following file |draft.tex|
with a compilation flag |\version| as described in \secref{sec:flags}
compiles the main document as a draft:
%
\begin{center}
\begin{tabular}{l}
|\def\version{draft}|\\
|\input{childdoc.def}|\\
|\childdocforward{|\textit{main}|}|
\end{tabular}
\end{center}
%
Likewise, the following files |final|\textit{nn}|.tex|
compile the final version of the child document
|child|\textit{nn}|.tex|:
%
\begin{center}
\begin{tabular}{l}
|\def\version{final}|\\
|\input{childdoc.def}|\\
|\childdocforwardprefix{final}{child}|
\end{tabular}
\end{center}
%

Note that when several versions of a main file and/or of each child file
are to be generated, it may be convenient to set up a |Makefile| or
shell script to automatise the process.

%%%%%%%%%%%%%%%%%%%%%%%%%%%%%%%%%%%%%%%%%%%%%%%%%%%%%%%%%%%%%%%%%%%%%%%%%%%%%%%%
\subsection{Command Line Processing}
\label{sec:commandline}

The effect of redirection files can also be achieved by invoking
the \LaTeX{} compiler with a more elaborate command line.
Most conveniently this should be done as part
of a shell script or a |Makefile|.

When using \textsf{childdoc} in the main file, the following
command lines effectively perform a redirection
(note that depending on the shell being used,
backslashes may have to be doubled: `|\|' $\to$ `|\\|'):
%
\begin{center}
|... -jobname "|\textit{target}|" |\\|"|[\textit{flags}]%
|\input{childdoc.def}\childdocforward[|\textit{main}|]{|\textit{dest}|}"|
\end{center}
%
Here \textit{target} is the name of the output file,
\textit{main} is the name of the main file
and \textit{dest} is the name of the main or child file to be processed
(all filenames without extensions).
The optional argument \textit{main} can be omitted
if \textit{main} matches \textit{dest}.
Optionally, compilation \textit{flags} can be defined via |\def| commands.
This command line makes the \TeX{} engine believe
it is compiling the file \textit{target}
whose content is specified as the latter parameter.
The provided code then forwards the processing to
\textit{main} or \textit{dest} as described in \secref{sec:forward}.

%%%%%%%%%%%%%%%%%%%%%%%%%%%%%%%%%%%%%%%%%%%%%%%%%%%%%%%%%%%%%%%%%%%%%%%%%%%%%%%%
\subsection{Include by Input}
\label{sec:input}

Including child documents by |\include| has some restrictions by design.
Most notably, the content of a child document always occupies
its own set of pages; pages cannot be shared between child documents.
Usually, this behaviour makes perfect sense
because each child document contain an essential part of the document.
However, in some situations it may be desirable to compose
a document from a collection of parts
without having mandatory page breaks between then.
For this case, the package
provides a mechanism to include parts
by |\input| which can also be processed individually.
However, by construction this mechanism
requires manual handling of the content to be output.

%%%%%%%%%%%%%%%%%%%%%%%%%%%%%%%%%%%%%%%%
\DescribeMacro{\ifchilddocmanual}
The main file should be prepared as usual, see \secref{sec:include}.
However, the document body must make a distinction
between processing of an individual part and of the main document, e.g.:
%
\begin{center}
\begin{tabular}{l}
|\ifchilddocmanual|\\
|\input{\childdocname}|\\
|\||else|\\
\textit{document body with }|\input{|\textit{part}|}|\\
|\||fi|
\end{tabular}
\end{center}
%
The conditional |\ifchilddocmanual| is true whenever
a part to be included by |\input| is being compiled,
and the name of the part is stored in |\childdocname|.

%%%%%%%%%%%%%%%%%%%%%%%%%%%%%%%%%%%%%%%%
\DescribeMacro{\childdocby}
Each part to be included by |\input| should start with:
%
\begin{center}
\begin{tabular}{l}
|\input{childdoc.def}|\\
|\childdocby{|\textit{main}|}|\\
\end{tabular}
\end{center}
%
The directive |\childdocby| is similar to |\childdocof|
described in \secref{sec:include},
but the subsequent selection of content must be done manually.
To that end, both |\ifchilddoc| and |\ifchilddocmanual|
will be true upon processing of a part,
and the name of the part is stored in |\childdocname|.
Note that |\jobname| will be set to the filename of the current part
so that each part receives an individual |.aux| file
that does not interfere with the |.aux| file(s) of the main document.
This behaviour can be altered by the alternative form
|\childdocby[*]{|\textit{main}|}| (with a non-empty optional argument)
which uses the |.aux| file of the main document
by setting |\jobname| to \textit{main}.

%%%%%%%%%%%%%%%%%%%%%%%%%%%%%%%%%%%%%%%%%%%%%%%%%%%%%%%%%%%%%%%%%%%%%%%%%%%%%%%%
\subsection{Driver Development}
\label{sec:driver}

The \textsf{childdoc} mechanism can also be use for the development
of definition files such as \LaTeX{} styles or classes.
This case differs from the above setup with multiple parts
included by |\include| in that no |\includeonly| should be invoked.
This can be achieved by starting the include file
(before |\ProvidesPackage|) with:
%
\begin{center}
\begin{tabular}{l}
|\input{childdoc.def}|\\
|\childdocforward{|\textit{main}|}|\\
\end{tabular}
\end{center}
%
or alternatively with:
%
\begin{center}
\begin{tabular}{l}
|\input{childdoc.def}|\\
|\childdocby{|\textit{main}|}|\\
\end{tabular}
\end{center}
%
Both forms have slightly different effects as described above.
The main file is prepared as usual, see \secref{sec:include}.

%%%%%%%%%%%%%%%%%%%%%%%%%%%%%%%%%%%%%%%%%%%%%%%%%%%%%%%%%%%%%%%%%%%%%%%%%%%%%%%%
\subsection{Legacy Detection}
\label{sec:detection}

The directive |\childdocmain| in the main file can detect
whether the complete document or merely a child is to be compiled
even without using the directive |\childdocof|.
This method is deprecated because it is less robust
and there is no compelling reason to use it;
it is merely provided for backward compatibility
and it may be removed in future versions.

If the detection mechanism is to be used,
it is mandatory to correctly specify
the filename of the main file as the argument of |\childdocmain|:
%
\begin{center}
\begin{tabular}{l}
|\input{childdoc.def}|\\
|\childdocmain{|\textit{main}|}|\\
\end{tabular}
\end{center}
%
If |\jobname| does not match the argument \textit{main} of |\childdocmain|,
it is assumed that |\jobname| points to the child file to be compiled.
When using |\childdocmain| with the main file specified as argument,
it suffices to start a child file
with just |\input{|\textit{main}|}|
without loading of the package and using |\childdocof|.
If instead all processing is done
with the appropriate \textsf{childdoc} directives,
the argument of \textit{main} of |\childdocmain| can be empty.

An alternative version of the command line processing described
in \secref{sec:commandline} using the detection mechanism reads:
%
\begin{center}
|... -jobname "|\textit{target}|" "|[\textit{flags}]%
[|\def\jobname{|\textit{dest}|}|]|\input{|\textit{main}|}"|
\end{center}

%%%%%%%%%%%%%%%%%%%%%%%%%%%%%%%%%%%%%%%%%%%%%%%%%%%%%%%%%%%%%%%%%%%%%%%%%%%%%%%%
\subsection{Manual Code}
\label{sec:manual}

In case one cannot be certain whether the definitions file |childdoc.def|
is installed on the target \TeX{} distribution
and one prefers not to ship it,
it is conceivable to paste a few relevant commands into the sources.

To that end, drop all statements |\input{childdoc.def}|
and perform the replacements as outlined below.
Instead of |\childdocmain{|\textit{main}|}| add the following code
to the top of the main file:
%
\begin{center}
\begin{tabular}{l}
|\||ifdefined\childdocname\endinput\||fi\newif\ifchilddoc|\\
|\edef\childdocname{\scantokens\expandafter{\jobname\noexpand}}|\\
|\def\childdocmain{|\textit{main}|}\||ifx\childdocmain\childdocname\||else|\\
|\childdoctrue\includeonly{\childdocname}\let\jobname\childdocmain\||fi|\\
\end{tabular}
\end{center}
%
Instead of |\childdocof{|\textit{main}|}| just include the main file
at the top of each child file:
%
\begin{center}
|\input{|\textit{main}|}|
\end{center}
%
A simple redirection |\childdocforward{|\textit{dest}|}| is achieved by:
%
\begin{center}
|\def\jobname{|\textit{dest}|}\input{\jobname}|
\end{center}
%
The redirection with prefix
|\childdocforwardprefix[|\textit{prefix}|]{|\textit{dest}|}|
is accomplished by:
%
\begin{center}
\begin{tabular}{l}
|{\edef\jobname{\scantokens\expandafter{\jobname\noexpand}}|\\
|\def\redirectjob |\textit{prefix}|#1~~~{\gdef\jobname{|\textit{dest}|#1}}|\\
|\expandafter\redirectjob\jobname~~~}\input{\jobname}|
\end{tabular}
\end{center}

In an alternative approach,
child documents can be compiled by a specific command line
without additional code or specific definitions:
%
\begin{center}
|... -jobname "|\textit{target}|" "|[\textit{flags}]%
|\includeonly{|\textit{dest}|}\input{|\textit{main}|}"|
\end{center}
%

%%%%%%%%%%%%%%%%%%%%%%%%%%%%%%%%%%%%%%%%%%%%%%%%%%%%%%%%%%%%%%%%%%%%%%%%%%%%%%%%
%%%%%%%%%%%%%%%%%%%%%%%%%%%%%%%%%%%%%%%%%%%%%%%%%%%%%%%%%%%%%%%%%%%%%%%%%%%%%%%%
\section{Information}

%%%%%%%%%%%%%%%%%%%%%%%%%%%%%%%%%%%%%%%%%%%%%%%%%%%%%%%%%%%%%%%%%%%%%%%%%%%%%%%%
\subsection{Copyright}

Copyright \copyright{} 2017--2018 Niklas Beisert

This work may be distributed and/or modified under the
conditions of the \LaTeX{} Project Public License, either version 1.3
of this license or (at your option) any later version.
The latest version of this license is in
  \url{http://www.latex-project.org/lppl.txt}
and version 1.3 or later is part of all distributions of \LaTeX{}
version 2005/12/01 or later.

This work has the LPPL maintenance status `maintained'.

The Current Maintainer of this work is Niklas Beisert.

This work consists of the files |README.txt|, |childdoc.ins| and |childdoc.dtx|
as well as the derived files |childdoc.def|, |cdocsamp.tex|
with |cdocsch1.tex|, |cdocsch2.tex|, |cdocspt3.tex|, |cdocspt4.tex|,
|cdocsdrf.tex|, |cdocsfn1.tex|, |cdocsfn2.tex|
as well as |childdoc.pdf|.

%%%%%%%%%%%%%%%%%%%%%%%%%%%%%%%%%%%%%%%%%%%%%%%%%%%%%%%%%%%%%%%%%%%%%%%%%%%%%%%%
\subsection{Files and Installation}

The package consists of the files:
%
\begin{center}
\begin{tabular}{ll}
    |README.txt|   & readme file \\
    |childdoc.ins| & installation file \\
    |childdoc.dtx| & source file \\
    |childdoc.def| & definition file \\
    |cdocsamp.tex| & sample main file \\
    |cdocsch1.tex| & sample include file \\
    |cdocsch2.tex| & sample include file \\
    |cdocspt3.tex| & sample part file \\
    |cdocspt4.tex| & sample part file \\
    |cdocsdrf.tex| & sample redirection file \\
    |cdocsfn1.tex| & sample redirection file \\
    |cdocsfn2.tex| & sample redirection file \\
    |childdoc.pdf| & manual
\end{tabular}
\end{center}
%
The distribution consists of the files
|README.txt|, |childdoc.ins| and |childdoc.dtx|.
%
\begin{itemize}
\item
Run (pdf)\LaTeX{} on |childdoc.dtx|
to compile the manual |childdoc.pdf| (this file).
\item
Run \LaTeX{} on |childdoc.ins| to create the definitions file |childdoc.def|
and the sample |cdocsamp.tex| with include files
|cdocsch1.tex|, |cdocsch2.tex|, |cdocspt3.tex|, |cdocspt4.tex|,
|cdocsdrf.tex|, |cdocsfn1.tex|, |cdocsfn2.tex|.
Then copy the file |childdoc.def| to an appropriate directory of your \LaTeX{}
distribution, e.g.\ \textit{texmf-root}|/tex/latex/childdoc|.
\end{itemize}

%%%%%%%%%%%%%%%%%%%%%%%%%%%%%%%%%%%%%%%%%%%%%%%%%%%%%%%%%%%%%%%%%%%%%%%%%%%%%%%%
\subsection{Related CTAN Packages}

There are several other packages which offer a similar functionality:
%
\begin{itemize}
\item
The packages
\href{http://ctan.org/pkg/docmute}{\textsf{docmute}},
\href{http://ctan.org/pkg/includex}{\textsf{includex}} and
\href{http://ctan.org/pkg/standalone}{\textsf{standalone}}
provide commands to include only the document body of
a child file thus allowing both files to be compiled individually.
\item
The packages \href{http://ctan.org/pkg/subdocs}{\textsf{subdocs}}
and \href{http://ctan.org/pkg/subfiles}{\textsf{subfiles}}
provide structures in which the main and child documents can be
encapsulated and allowing them to be compiled individually.
The inclusion mechanism is different from the conventional |\include|.
\item
The package \href{http://ctan.org/pkg/combine}{\textsf{combine}}
is an elaborate solution to combine several documents into one.
\end{itemize}
%
See also the CTAN topic \href{http://ctan.org/topic/subdocs}{\textsf{subdocs}}
for further related packages.
The present package differs from the above solutions in that
a document structure constructed with the conventional |\include| mechanism
just needs two extra commands at the top of every file
such that all constituent files can be compiled individually.

%%%%%%%%%%%%%%%%%%%%%%%%%%%%%%%%%%%%%%%%%%%%%%%%%%%%%%%%%%%%%%%%%%%%%%%%%%%%%%%%
%\subsection{Feature Suggestions}
%
%The following is a list of features which may be useful for future
%versions of this package:
%%
%\begin{itemize}
%\item
%\ldots
%\end{itemize}

%%%%%%%%%%%%%%%%%%%%%%%%%%%%%%%%%%%%%%%%%%%%%%%%%%%%%%%%%%%%%%%%%%%%%%%%%%%%%%%%
\subsection{Revision History}

%%%%%%%%%%%%%%%%%%%%%%%%%%%%%%%%%%%%%%%%
\paragraph{v2.0:} 2018/12/30

\begin{itemize}
\item
immediate forward processing
\item
added |\childdocby| mechanism
\item
manual restructured
\end{itemize}

%%%%%%%%%%%%%%%%%%%%%%%%%%%%%%%%%%%%%%%%
\paragraph{v1.6:} 2018/01/17

\begin{itemize}
\item
application for development of include files
\item
corrections to manual
\end{itemize}

%%%%%%%%%%%%%%%%%%%%%%%%%%%%%%%%%%%%%%%%
\paragraph{v1.5:} 2017/05/21

\begin{itemize}
\item
more complete structuring introduced
\item
|\childdocof| introduced
\item
|\childdoc| renamed to |\childdocmain|
\item
|\childredirect| renamed to |\childdocforward| and |\childdocforwardprefix|
and functionality expanded
\end{itemize}

%%%%%%%%%%%%%%%%%%%%%%%%%%%%%%%%%%%%%%%%
\paragraph{v1.0:} 2017/04/27

\begin{itemize}
\item
manual and install package
\item
first version published on CTAN
\end{itemize}

%%%%%%%%%%%%%%%%%%%%%%%%%%%%%%%%%%%%%%%%
\paragraph{v0.6:} 2017/04/26

\begin{itemize}
\item
redirection mechanism added
\end{itemize}

%%%%%%%%%%%%%%%%%%%%%%%%%%%%%%%%%%%%%%%%
\paragraph{v0.5:} 2017/04/26

\begin{itemize}
\item
functionality in definition file
\end{itemize}


%%%%%%%%%%%%%%%%%%%%%%%%%%%%%%%%%%%%%%%%%%%%%%%%%%%%%%%%%%%%%%%%%%%%%%%%%%%%%%%%
%%%%%%%%%%%%%%%%%%%%%%%%%%%%%%%%%%%%%%%%%%%%%%%%%%%%%%%%%%%%%%%%%%%%%%%%%%%%%%%%
%%%%%%%%%%%%%%%%%%%%%%%%%%%%%%%%%%%%%%%%%%%%%%%%%%%%%%%%%%%%%%%%%%%%%%%%%%%%%%%%
\appendix

\settowidth\MacroIndent{\rmfamily\scriptsize 000\ }

 \DocInput{childdoc.dtx}

\end{document}
%</driver>
% \fi
%
% %%%%%%%%%%%%%%%%%%%%%%%%%%%%%%%%%%%%%%%%%%%%%%%%%%%%%%%%%%%%%%%%%%%%%%%%%%%%%%
% %%%%%%%%%%%%%%%%%%%%%%%%%%%%%%%%%%%%%%%%%%%%%%%%%%%%%%%%%%%%%%%%%%%%%%%%%%%%%%
% \section{Sample}
%\iffalse
%<*samplemain>
%\fi
%
% The following presents a sample document
% with two chapters, two parts, a title page,
% a compile flag as well as three forwarding files to set the flag.
% It consists of eight |.tex| files:
% \begin{center}
% \begin{tabular}{ll}
% |cdocsamp.tex|&main file\\
% |cdocsch1.tex|&include file for chapter 1\\
% |cdocsch2.tex|&include file for chapter 2\\
% |cdocspt3.tex|&include file for part 3\\
% |cdocspt4.tex|&include file for part 4\\
% |cdocsdrf.tex|&forwarding file for main file in draft mode\\
% |cdocsfi1.tex|&forwarding file for final version of chapter 1\\
% |cdocsfi2.tex|&forwarding file for final version of chapter 2\\
% \end{tabular}
% \end{center}
% Each of the eight files can be compiled directly by the \LaTeX{} compiler.
%
% %%%%%%%%%%%%%%%%%%%%%%%%%%%%%%%%%%%%%%
% \paragraph{Main File.}
%
% The main file is called |cdocsamp.tex|.
%
% Load the \textsf{childdoc} definitions and
% declare the filename for the main document:
%    \begin{macrocode}
\input{childdoc.def}
\childdocmain{}
%    \end{macrocode}

% Optional override for |\version| flag:
%    \begin{macrocode}
%%\ifchilddoc\else\providecommand{\version}{draft}\fi
%    \end{macrocode}

% Define the default values for the |\version| flag
% (|final| for the main file and |draft| for childs):
%    \begin{macrocode}
\ifchilddoc
\providecommand{\version}{draft}
\else
\providecommand{\version}{final}
\fi
%    \end{macrocode}

% Load the standard document class:
%    \begin{macrocode}
\documentclass[12pt]{article}
%    \end{macrocode}

% Start the document body:
%    \begin{macrocode}
\begin{document}
%    \end{macrocode}

% Declare a title page.
% Print title, part of document being processed and version flag:
%    \begin{macrocode}
\addtocounter{page}{-1}
\begin{center}
{\LARGE\bfseries{}childdoc example\par}
\vspace{1cm}
\ifchilddoc
\ifchilddocmanual part\else chapter\fi:
`\childdocname' of `\childdocjob'\par
\else
main document: `\childdocjob'\par
\fi
version: \version\par
\end{center}
\newpage
%    \end{macrocode}

% Manually include selected file,
% otherwise process as usual:
%    \begin{macrocode}
\ifchilddocmanual
\section*{part `\childdocname'}
\input{\childdocname}
\else
%    \end{macrocode}

% Include the two chapters:
%    \begin{macrocode}
\include{cdocsch1}
\include{cdocsch2}
%    \end{macrocode}

% Include the two parts unless only chapters should be displayed:
%    \begin{macrocode}
\ifchilddoc\else
\section{part three}
\input{cdocspt3}
\section{part four}
\input{cdocspt4}
\fi
%    \end{macrocode}

% Process as usual until here:
%    \begin{macrocode}
\fi
%    \end{macrocode}

% End of document body:
%    \begin{macrocode}
\end{document}
%    \end{macrocode}
%\iffalse
%</samplemain>
%\fi
%
% %%%%%%%%%%%%%%%%%%%%%%%%%%%%%%%%%%%%%%
% \paragraph{Chapter Include Files.}
%
% The include files are called |cdocsch1.tex| and |cdocsch2.tex|.
%
%\iffalse
%<*samplechap1|samplechap2>
%\fi

% Optional override for |\version| flag:
%    \begin{macrocode}
%%\providecommand{\version}{final}
%    \end{macrocode}

% Include the main document:
%    \begin{macrocode}
\input{childdoc.def}
\childdocof{cdocsamp}
%    \end{macrocode}

%\iffalse
%</samplechap1|samplechap2>
%\fi
%
%\iffalse
%<*samplechap1>
%\fi
% Some text for chapter 1:
%    \begin{macrocode}
\section{one}
some text in chapter one
%    \end{macrocode}

%\iffalse
%</samplechap1>
%\fi
% Some text for chapter 2:
%\iffalse
%<*samplechap2>
%\fi
%    \begin{macrocode}
\section{two}
more text in chapter two
%    \end{macrocode}

%\iffalse
%</samplechap2>
%\fi
%
% %%%%%%%%%%%%%%%%%%%%%%%%%%%%%%%%%%%%%%
% \paragraph{Part Include Files.}
%
% The include files are called |cdocspt3.tex| and |cdocspt4.tex|.
%
%\iffalse
%<*samplepart3|samplepart4>
%\fi

% Optional override for |\version| flag:
%    \begin{macrocode}
%%\providecommand{\version}{final}
%    \end{macrocode}

% Include the main document:
%    \begin{macrocode}
\input{childdoc.def}
\childdocby{cdocsamp}
%    \end{macrocode}

%\iffalse
%</samplepart3|samplepart4>
%\fi
%
%\iffalse
%<*samplepart3>
%\fi
% Some text for part 3:
%    \begin{macrocode}
some text in part three
%    \end{macrocode}

%\iffalse
%</samplepart3>
%\fi
% Some text for part 4:
%\iffalse
%<*samplepart4>
%\fi
%    \begin{macrocode}
more text in part four
%    \end{macrocode}

%\iffalse
%</samplepart4>
%\fi
%
% %%%%%%%%%%%%%%%%%%%%%%%%%%%%%%%%%%%%%%
% \paragraph{Forwarding for a Complete Draft.}
%
% The following forwarding file |cdocsdrf.tex|
% compiles the main document in draft mode:
%\iffalse
%<*sampledraft>
%\fi
%    \begin{macrocode}
\def\version{draft}
\input{childdoc.def}
\childdocforward{cdocsamp}
%    \end{macrocode}

%\iffalse
%</sampledraft>
%\fi
%
% %%%%%%%%%%%%%%%%%%%%%%%%%%%%%%%%%%%%%%
% \paragraph{Forwarding for Final Version of the Chapters.}
%
% The following forwarding files |cdocsfn1.tex| and |cdocsfn2.tex|
% (with identical content)
% compile the final versions of the child documents
% |cdocsch1.tex| and |cdocsch2.tex|, respectively:
%\iffalse
%<*samplefinal>
%\fi
%    \begin{macrocode}
\def\version{final}
\input{childdoc.def}
\childdocforwardprefix[cdocsamp]{cdocsfn}{cdocsch}
%    \end{macrocode}

%\iffalse
%</samplefinal>
%\fi
%
% %%%%%%%%%%%%%%%%%%%%%%%%%%%%%%%%%%%%%%
% \paragraph{Command Line Processing.}
%
% The following three command lines generate the output files
% |cdocscld|, |cdocscl1| and |cdocscl2|
% which should be identical to
% |cdocsdrf|, |cdocsch1| and |cdocsfn2|, respectively:
% \begin{center}
% \begin{tabular}{l}
% |latex -jobname cdocscld \|\\
% |  "\def\version{draft}\input{childdoc.def}\childdocforward{cdocsamp}"|\\
% |latex -jobname cdocscl1 \|\\
% |  "\input{childdoc.def}\childdocforward[cdocsamp]{cdocsch1}"|\\
% |latex -jobname cdocscl2 \|\\
% |  "\def\version{final}\input{childdoc.def}\childdocforward{cdocsch2}"|
% \end{tabular}
% \end{center}
% Note that the trailing backslash on each first line
% merely continues the input to the second line
% (for convenient cut ant paste).
% Furthermore, the command |latex| can be replaced by any
% of its alternative versions such as |pdflatex|.
%
% %%%%%%%%%%%%%%%%%%%%%%%%%%%%%%%%%%%%%%%%%%%%%%%%%%%%%%%%%%%%%%%%%%%%%%%%%%%%%%
% %%%%%%%%%%%%%%%%%%%%%%%%%%%%%%%%%%%%%%%%%%%%%%%%%%%%%%%%%%%%%%%%%%%%%%%%%%%%%%
% \section{Implementation}
%\iffalse
%<*package>
%\fi
%
% This section describes the definitions file |childdoc.def|.

% The definitions cannot be loaded using |\usepackage| or |\RequirePackage|
% which has a mechanism to prevent loading a style file more than once.
% When loading the definitions by means of |\input|
% multiple instances have to be prevented manually:
%\iffalse
%This code needs to be before the `\ProvidesFile' directive
%which is defined at the beginning of this file.
%Therefore it is also placed there and commented out here.
%</package>
%<*discard>
%\fi
%    \begin{macrocode}
\ifdefined\childdocmain\endinput\fi
%    \end{macrocode}
%\iffalse
%</discard>
%<*package>
%\fi
%
% \macro{\ifchilddoc}
% \macro{\ifchilddocmanual}
% The conditional |\ifchilddoc| tells whether a
% child (true) or main (false) document is being compiled.
% The conditional |\ifchilddocmanual| tells whether
% the |\includeonly| mechanism is used (false) or
% the selection of child files must be performed manually (true).
% The definitions initialise to false:
%    \begin{macrocode}
\newif\ifchilddoc
\newif\ifchilddocmanual
%    \end{macrocode}

% \macro{\childdocname}
% \macro{\childdocjob}
% The macro |\childdocname| stores the name of the main document
% to be compiled. The macro |\childdocjob| stores the name of
% the document on which the \LaTeX{} compiler was originally invoked.
% The content of |\jobname| cannot be compared
% to filenames specified in the source due to different catcodes.
% The following code rescans |\jobname|, stores the result
% in |\childdocname| and saves a copy in |\childdocjob|:
%    \begin{macrocode}
\edef\childdocname{\scantokens\expandafter{\jobname\noexpand}}
\let\childdocjob\childdocname
%    \end{macrocode}

% \macro{\childdocdisable}
% The macro |\childdocdisable| prevents the main file
% from being processed more than once.
% At this stage, the main document command |\childdocmain|
% is assumed to be called once again where it should do nothing.
% Any subsequent call to it should prevent
% a secondary processing of the main document
% It overwrites the forwarding commands
% |\childdocof| and |\childdocforward|
% with empty macros to prevent further inclusions of the main document:
%    \begin{macrocode}
\newcommand{\childdocdisable}
{
  \renewcommand{\childdocmain}[1]{\renewcommand{\childdocmain}[1]{\endinput}}
  \renewcommand{\childdocof}[1]{}
  \renewcommand{\childdocby}[2][]{}
  \renewcommand{\childdocforward}[2][]{}
  \renewcommand{\childdocdisable}{}
}
%    \end{macrocode}

% \macro{\childdocmain}
% The macro |\childdocmain| is to be called at the top of the main file
% with nothing or the main filename (without extension) as argument.
% First, it breaks loops.
% If the argument is not empty and does not match |\childdocname|
% (which is set by the first inclusion of |childdoc.def|),
% |\ifchilddoc| is set to true, |\includeonly| is applied to the child file
% and |\jobname| is set to the main file
% (for proper handling of |.aux| files):
%    \begin{macrocode}
\newcommand{\childdocmain}[1]
{
  \childdocdisable\childdocmain{}
  \if?#1?\else
    \begingroup
      \def\childdoctmp{#1}
      \ifx\childdoctmp\childdocname
        \def\childdoctmp{}
      \else
        \def\childdoctmp
        {
          \childdoctrue
          \includeonly{\childdocname}
          \def\childdocjob{#1}
          \def\jobname{#1}
        }
      \fi
      \expandafter
    \endgroup
    \childdoctmp
  \fi
}
%    \end{macrocode}

% \macro{\childdocof}
% The command |\childdocof| redirects
% compilation to the main file |#1|.
%    \begin{macrocode}
\newcommand{\childdocof}[1]
{
  \childdocdisable
  \childdoctrue
  \includeonly{\childdocname}
  \def\jobname{#1}
  \def\childdocjob{#1}
  \input{#1}
}
%    \end{macrocode}

% \macro{\childdocby}
% The command |\childdocby| ....
%    \begin{macrocode}
\newcommand{\childdocby}[2][]
{
  \childdocdisable
  \childdoctrue
  \childdocmanualtrue
  \if?#1?\else
    \def\jobname{#2}
  \fi
  \def\childdocjob{#2}
  \input{#2}
  \endinput
}
%    \end{macrocode}

% \macro{\childdocforward}
% The command |\childdocforward| redirects
% compilation to the main file or
% (if the optional argument is given) a child file.
% Parameters are set as if the main file
% or a child file starting with |\childdocof| was compiled.
% Then compilation is handed over to the main file:
%    \begin{macrocode}
\newcommand{\childdocforward}[2][]
{
  \begingroup
    \if?#1?
      \def\childdoctmp
      {
        \def\childdocname{#2}
        \def\childdocjob{#2}
        \def\jobname{#2}
        \input{#2}
        \endinput
      }
    \else
      \def\childdoctmp
      {
        \childdocdisable
        \def\childdocname{#2}
        \childdoctrue
        \includeonly{#2}
        \def\childdocjob{#1}
        \def\jobname{#1}
        \input{#1}
        \endinput
      }
    \fi
    \expandafter
  \endgroup
  \childdoctmp
}
%    \end{macrocode}

% \macro{\childdocforwardprefix}
% The command |\childdocforwardprefix| redirects
% compilation to the main or a child file by means of a pattern.
% The prefix |#1| in the current filename is replaced by |#2|
% and the suffix of the current filename is kept
% (it is assumed that the filename does not contain the substring `|~~~|'
% which is used as a delimiter).
% Compilation is handed over to the new file by |\childdocforward|:
%    \begin{macrocode}
\newcommand{\childdocforwardprefix}[3][]
{
  \begingroup
    \def\childdocextract #2##1~~~{\def\childdoctmp{\childdocforward[#1]{#3##1}}}
    \expandafter\childdocextract\childdocname~~~
    \expandafter
  \endgroup
  \childdoctmp
}
%    \end{macrocode}

% \macro{\childdoc}
% The deprecated macro |\childdoc| is a legacy version of |\childdocmain|:
%    \begin{macrocode}
\newcommand{\childdoc}{\childdocmain}
%    \end{macrocode}

% \macro{\childdocredirect}
% The deprecated macro |\childdocredirect| is a legacy version
% of |\childdocforward| and |\childdocforwardprefix|:
%    \begin{macrocode}
\newcommand{\childdocredirect}[2][]
{
  \begingroup
    \if?#1?
      \def\childdoctmp{\childdocforward{#2}}
    \else
      \def\childdoctmp{\childdocforwardprefix{#1}{#2}}
    \fi
    \expandafter
  \endgroup
  \childdoctmp
}
%    \end{macrocode}

%\iffalse
%</package>
%\fi
%
\endinput

\childdocmain{}
%    \end{macrocode}

% Optional override for |\version| flag:
%    \begin{macrocode}
%%\ifchilddoc\else\providecommand{\version}{draft}\fi
%    \end{macrocode}

% Define the default values for the |\version| flag
% (|final| for the main file and |draft| for childs):
%    \begin{macrocode}
\ifchilddoc
\providecommand{\version}{draft}
\else
\providecommand{\version}{final}
\fi
%    \end{macrocode}

% Load the standard document class:
%    \begin{macrocode}
\documentclass[12pt]{article}
%    \end{macrocode}

% Start the document body:
%    \begin{macrocode}
\begin{document}
%    \end{macrocode}

% Declare a title page.
% Print title, part of document being processed and version flag:
%    \begin{macrocode}
\addtocounter{page}{-1}
\begin{center}
{\LARGE\bfseries{}childdoc example\par}
\vspace{1cm}
\ifchilddoc
\ifchilddocmanual part\else chapter\fi:
`\childdocname' of `\childdocjob'\par
\else
main document: `\childdocjob'\par
\fi
version: \version\par
\end{center}
\newpage
%    \end{macrocode}

% Manually include selected file,
% otherwise process as usual:
%    \begin{macrocode}
\ifchilddocmanual
\section*{part `\childdocname'}
\input{\childdocname}
\else
%    \end{macrocode}

% Include the two chapters:
%    \begin{macrocode}
\include{cdocsch1}
\include{cdocsch2}
%    \end{macrocode}

% Include the two parts unless only chapters should be displayed:
%    \begin{macrocode}
\ifchilddoc\else
\section{part three}
\input{cdocspt3}
\section{part four}
\input{cdocspt4}
\fi
%    \end{macrocode}

% Process as usual until here:
%    \begin{macrocode}
\fi
%    \end{macrocode}

% End of document body:
%    \begin{macrocode}
\end{document}
%    \end{macrocode}
%\iffalse
%</samplemain>
%\fi
%
% %%%%%%%%%%%%%%%%%%%%%%%%%%%%%%%%%%%%%%
% \paragraph{Chapter Include Files.}
%
% The include files are called |cdocsch1.tex| and |cdocsch2.tex|.
%
%\iffalse
%<*samplechap1|samplechap2>
%\fi

% Optional override for |\version| flag:
%    \begin{macrocode}
%%\providecommand{\version}{final}
%    \end{macrocode}

% Include the main document:
%    \begin{macrocode}
% \iffalse
%
% childdoc.dtx Copyright (C) 2017-2018 Niklas Beisert
%
% This work may be distributed and/or modified under the
% conditions of the LaTeX Project Public License, either version 1.3
% of this license or (at your option) any later version.
% The latest version of this license is in
%   http://www.latex-project.org/lppl.txt
% and version 1.3 or later is part of all distributions of LaTeX
% version 2005/12/01 or later.
%
% This work has the LPPL maintenance status `maintained'.
%
% The Current Maintainer of this work is Niklas Beisert.
%
% This work consists of the files childdoc.dtx and childdoc.ins
% and the derived files childdoc.def and cdocsamp.tex with
% cdocsch1.tex, cdocsch2.tex, cdocsdrf.tex, cdocsfn1.tex, cdocsfn2.tex.
%
%<package>\ifdefined\childdocmain\endinput\fi
%<package>\ProvidesFile{childdoc.def}[2018/12/30 v2.0 child document driver]
%<samplemain>\ProvidesFile{cdocsamp.tex}[2018/12/30 v2.0 sample for childdoc]
%<*driver>
%\ProvidesFile{childdoc.drv}[2018/12/30 v2.0 childdoc reference manual file]
\PassOptionsToClass{10pt,a4paper}{article}
\documentclass{ltxdoc}

\usepackage[margin=35mm]{geometry}
\usepackage{hyperref}
\usepackage{hyperxmp}
\usepackage[usenames]{color}

\hypersetup{colorlinks=true}
\hypersetup{pdfstartview=FitH}
\hypersetup{pdfpagemode=UseNone}
\hypersetup{pdfsource={}}
\hypersetup{pdflang={en-UK}}
\hypersetup{pdfcopyright={Copyright 2017-2018 Niklas Beisert.
  This work may be distributed and/or modified under the
  conditions of the LaTeX Project Public License, either version 1.3
  of this license or (at your option) any later version.}}
\hypersetup{pdflicenseurl={http://www.latex-project.org/lppl.txt}}
\hypersetup{pdfcontactaddress={ETH Zurich, ITP, HIT K,
  Wolfgang-Pauli-Strasse 27}}
\hypersetup{pdfcontactpostcode={8093}}
\hypersetup{pdfcontactcity={Zurich}}
\hypersetup{pdfcontactcountry={Switzerland}}
\hypersetup{pdfcontactemail={nbeisert@itp.phys.ethz.ch}}
\hypersetup{pdfcontacturl={http://people.phys.ethz.ch/\xmptilde nbeisert/}}

\newcommand{\secref}[1]{\hyperref[#1]{section \ref*{#1}}}

\parskip1ex
\parindent0pt
\let\olditemize\itemize
\def\itemize{\olditemize\parskip0pt}

\begin{document}

\title{The \textsf{childdoc} Package}
\hypersetup{pdftitle={The childdoc Package}}
\author{Niklas Beisert\\[2ex]
  Institut f\"ur Theoretische Physik\\
  Eidgen\"ossische Technische Hochschule Z\"urich\\
  Wolfgang-Pauli-Strasse 27, 8093 Z\"urich, Switzerland\\[1ex]
  \href{mailto:nbeisert@itp.phys.ethz.ch}
  {\texttt{nbeisert@itp.phys.ethz.ch}}}
\hypersetup{pdfauthor={Niklas Beisert}}
\hypersetup{pdfsubject={Manual for the LaTeX2e Package childdoc}}
\date{30 December 2018, \textsf{v2.0}}
\maketitle

\begin{abstract}\noindent
\textsf{childdoc} is a \LaTeXe{} package
that enables the direct compilation
of document sections included by |\include|
to individual files.
\end{abstract}

\begingroup
\parskip0ex
\tableofcontents
\endgroup

%%%%%%%%%%%%%%%%%%%%%%%%%%%%%%%%%%%%%%%%%%%%%%%%%%%%%%%%%%%%%%%%%%%%%%%%%%%%%%%%
%%%%%%%%%%%%%%%%%%%%%%%%%%%%%%%%%%%%%%%%%%%%%%%%%%%%%%%%%%%%%%%%%%%%%%%%%%%%%%%%
\section{Introduction}

\LaTeX{} provides a mechanism to structure a large document (such as a book)
into a main file and several child files (containing the chapters)
using the |\include| command.
This mechanism is beneficial for documents
which span hundreds of pages in order to
make the source file(s) more manageable.
Moreover, compilation can be restricted to
selected child files by means of the |\includeonly| command.
The latter feature can be used to reduce the compilation time while editing
(this was significantly more useful in the earlier days of \LaTeX{})
or to generate a smaller document which is easier to navigate.
Another application of |\includeonly| is to generate
documents consisting of selected parts of the complete document.

However, there are a few drawbacks of the plain |\include| mechanism:
\begin{itemize}
\item
The child files cannot be compiled on their own,
they can only be compiled via the main file.
A naive editing environment
(such as a text editor with an option
to have the current file processed by \LaTeX)
may require one to switch to the main file before compiling;
attempting to compile the child file produces errors.
\item
The main file must be modified (each time)
to adjust the |\includeonly| command
to the present needs. This easily leaves the main file in a messy state.
\item
The generated document will always carry the filename
of the main document. This is inconvenient if
several child files are to be compiled and
to be kept for distribution.
\end{itemize}

The present package provides a simple interface
to make child files individually compilable by \LaTeX{}.
Compiling a child file then has the same effect as compiling
the main file with an |\includeonly| command
to select the appropriate child.
Moreover the generated document will carry the name of the child
rather than the main file.
This resolves all three above issues.

This feature is meant to make the editing of books,
thesis documents and lecture notes somewhat more convenient.
However, the package can also be used efficiently for
composing a series of documents (such as exercise sheets)
which are typically distributed individually.
It then assists the author in generating the individual documents
(potentially in different versions)
as well as a document containing the collected series.
Another application is in developing style files
or other kinds of included material
where compilation of the style file could redirect
to a sample or test file.

%%%%%%%%%%%%%%%%%%%%%%%%%%%%%%%%%%%%%%%%%%%%%%%%%%%%%%%%%%%%%%%%%%%%%%%%%%%%%%%%
%%%%%%%%%%%%%%%%%%%%%%%%%%%%%%%%%%%%%%%%%%%%%%%%%%%%%%%%%%%%%%%%%%%%%%%%%%%%%%%%
\section{Usage}

First of all, the package \textsf{childdoc} is \emph{not} a standard
\LaTeXe{} |.sty| style file! Therefore it needs to be invoked in
a non-standard way.

%%%%%%%%%%%%%%%%%%%%%%%%%%%%%%%%%%%%%%%%%%%%%%%%%%%%%%%%%%%%%%%%%%%%%%%%%%%%%%%%
\subsection{Included Files}
\label{sec:include}

%%%%%%%%%%%%%%%%%%%%%%%%%%%%%%%%%%%%%%%%
\DescribeMacro{\childdocmain}
To use the package, add the commands
\begin{center}
\begin{tabular}{l}
|\input{childdoc.def}|\\
|\childdocmain{}|\\
\end{tabular}
\end{center}
at the very top of the main \LaTeX{} file,
in particular \emph{before} the |\documentclass| statement!
The argument of |\childdocmain| should be left empty
(but it must be present).

%%%%%%%%%%%%%%%%%%%%%%%%%%%%%%%%%%%%%%%%
\DescribeMacro{\childdocof}
Furthermore, add the commands
\begin{center}
\begin{tabular}{l}
|\input{childdoc.def}|\\
|\childdocof{|\textit{main}|}|\\
\end{tabular}
\end{center}
at the top of every child file \textit{child}
which is included by |\include{|\textit{child}|}|
from within the main file
(or at least for those files to be compiled individually).
The argument \textit{main} must be the filename of the main file.

There are a couple of
considerations in setting up the main and child documents:

%%%%%%%%%%%%%%%%%%%%%%%%%%%%%%%%%%%%%%%%
\paragraph{Restrictions.}

Please note the following restrictions:
\begin{itemize}
\item
|\childdocmain| must be called with one argument \textit{main}
to ensure compatibility with earlier version of the package.
It must either be empty (|\childdocmain{}|)
or precisely match the filename of the main file in which it is specified.
See \secref{sec:detection} for further information.
\item
The filename \textit{main} must be specified without the |.tex| extension.
\item
The filename \textit{main} is case sensitive
(even in case-insensitive file systems)
due to internal string comparison.
\item
The argument \textit{main} should be fully expanded, it cannot be a macro.
\item
Subdirectories and special characters should be avoided in filenames.
\item
The command |\childdocmain{|\textit{main}|}| must be followed by a whitespace.
It should not be followed immediately by another command
or by a comment mark `|%|'.
This is because the \TeX{} parser reads the token immediately following
the argument of |\childdocmain| and puts it
at the beginning of every child section;
however, a white\-space is ignored.
\end{itemize}

%%%%%%%%%%%%%%%%%%%%%%%%%%%%%%%%%%%%%%%%
\paragraph{Content of Main File.}

It is advisable to place all content in the child files included by |\include|.
Any output contained in the main file will appear in all child documents
unless suppressed manually;
it cannot be suppressed automatically by the |\includeonly| directive
and thus should normally be avoided.
A method to include some content in the main file
by means of conditional processing is described in \secref{sec:conditional}.

%%%%%%%%%%%%%%%%%%%%%%%%%%%%%%%%%%%%%%%%
\paragraph{Page Numbering.}

When only a part of the document is compiled,
the appropriate numbering of pages
(as well as other status parameters)
is determined from the |.aux| files.
The latter contain information from previous passes.
However this information needs to propagate through
all intermediate child documents.
Therefore the page numbering in child documents may well
be inconsistent until the complete document is compiled at least once.

A useful (if unconventional) way to always ensure a consistent
page numbering is to restart the numbering in each child document
and denote the pages by `\textit{child}|.|\textit{page}'
where \textit{child} represents the chapter/section number of the child file.
This can be achieved by the command
|\numberwithin{page}{|\textit{child}|}|
of the \textsf{amsmath} package
where \textit{child} can be |chapter| or |section|
depending on the chosen structuring.
Alternatively, one can modify the macro |\thepage| appropriately
and reset the counter |page| at the start of each child file.

%%%%%%%%%%%%%%%%%%%%%%%%%%%%%%%%%%%%%%%%%%%%%%%%%%%%%%%%%%%%%%%%%%%%%%%%%%%%%%%%
\subsection{Conditional Processing}
\label{sec:conditional}

The package provides a mechanism to compile different versions
of a document. To customise the versions further some conditional processing
can come in handy to distinguish which version is being compiled.
The package provides two macros to describe the compilation context:

%%%%%%%%%%%%%%%%%%%%%%%%%%%%%%%%%%%%%%%%
\DescribeMacro{\ifchilddoc}
The conditional |\ifchilddoc| distinguishes between the compilation of
child documents and the main document:
%
\begin{center}
|\ifchilddoc |\textit{child-code}| |[|\||else |\textit{main-code}]| \||fi|
\end{center}

%%%%%%%%%%%%%%%%%%%%%%%%%%%%%%%%%%%%%%%%
\DescribeMacro{\childdocname}
\DescribeMacro{\childdocjob}
The macro |\childdocname| contains the filename (without extension)
of the main or child file being processed.
Note that |\childdocjob| will always contain the name of the main file.

%%%%%%%%%%%%%%%%%%%%%%%%%%%%%%%%%%%%%%%%
\paragraph{Title Page.}

Conditional processing can be used to include a title or banner page
in the main document when proper precautions are taken.
Importantly, the code in the main file should ensure that the page counter
(as well as other status parameters which are stored in the |.aux| files)
takes the same value after the conditional processing.
Otherwise the page numbers may take divergent values
depending on which part is compiled.

For example, a title page could be declared by:
%
\begin{center}
\begin{tabular}{l}
|\ifchilddoc\||else|\\
|\addtocounter{page}{-1}|\\
\textit{code for title page}\\
|\newpage|\\
|\||fi|
\end{tabular}
\end{center}
%
A banner page for the child documents can be generated by:
%
\begin{center}
\begin{tabular}{l}
|\ifchilddoc|\\
|\addtocounter{page}{-1}|\\
\textit{code for banner page}\\
|\newpage|\\
|\||fi|
\end{tabular}
\end{center}
%
Here one could write a message such as:
\begin{center}
|This is the part \childdocname{} of \childdocjob{}.|
\end{center}

%%%%%%%%%%%%%%%%%%%%%%%%%%%%%%%%%%%%%%%%%%%%%%%%%%%%%%%%%%%%%%%%%%%%%%%%%%%%%%%%
\subsection{Flags}
\label{sec:flags}

The package makes it easy to generate different versions
of the main or child documents.
To this end compilation flags can be defined
and assigned different default values.
They will be particularly useful in conjunction
with the forwarding mechanism described in \secref{sec:forward}.

For example, it may be useful to have a flag |\version|
which can be set to |draft| or |final|.
The document source will contain some conditional code
depending on the value of |\version|.
Suppose further, the flag should default to |final| for the main file
and to |draft| for child files
which is a natural assignment for editing the document.
This is achieved by placing the following code
in the preamble of the main document
(below the |\childdocmain| directive):
%
\begin{center}
\begin{tabular}{l}
|\ifchilddoc|\\
|\providecommand{\version}{draft}|\\
|\||else|\\
|\providecommand{\version}{final}|\\
|\||fi|
\end{tabular}
\end{center}
%
The definition by |\providecommand| makes sure
that previous definitions are not overwritten.
Further statements |\providecommand{\version}{...}|
can thus be added before the above code to override it.

For the main file, one might add a line
(between |\childdocmain| and the above block)
%
\begin{center}
|%\ifchilddoc\||else\providecommand{\version}{draft}\||fi|
\end{center}
%
which can be uncommented to produce a draft version.
Likewise one can add a line to the very top of a child file
(above the |\childdocof{|\textit{main}|}| directive)
%
\begin{center}
|%\providecommand{\version}{final}|
\end{center}
%
which can be uncommented to produce the final version of this child document.

%%%%%%%%%%%%%%%%%%%%%%%%%%%%%%%%%%%%%%%%%%%%%%%%%%%%%%%%%%%%%%%%%%%%%%%%%%%%%%%%
\subsection{Forwarding}
\label{sec:forward}

Different versions of the main or child documents
using compilation flags as described in \secref{sec:flags}
can be (permanently) stored in different files
for convenient compilation, viewing and distribution.
To this end, the package defines a command
to pass on compilation to a different file:

%%%%%%%%%%%%%%%%%%%%%%%%%%%%%%%%%%%%%%%%
\DescribeMacro{\childdocforward}
The command |\childdocforward| redirects processing to
another source file:
%
\begin{center}
\begin{tabular}{l}
|\input{childdoc.def}|\\
|\childdocforward[|\textit{main}|]{|\textit{dest}|}|\\
\end{tabular}
\end{center}
%
The argument \textit{dest} is the destination file
(without extension).
It should be the main file or one of the child files.
Note that further \textsf{childdoc} directives
such as |\childdocof| and |\childdocforward|
in the indicated file will be processed in this form.
The optional argument \textit{main}
passes on directly to the main file \textit{main}
while pretending to compile the child \textit{dest}.
This form behaves as if \textit{dest}
issues |\childdocof{|\textit{main}|}| right away,
and no further \textsf{childdoc} directives will be processed.

%%%%%%%%%%%%%%%%%%%%%%%%%%%%%%%%%%%%%%%%
\DescribeMacro{\...prefix}
In the alternative form |\childdocforwardprefix|,
%
\begin{center}
\begin{tabular}{l}
|\input{childdoc.def}|\\
|\childdocforwardprefix[|\textit{main}|]{|\textit{prefix}|}{|\textit{dest}|}|
\end{tabular}
\end{center}
%
the destination file is determined by a pattern
depending on the current file:
To make this work, the current file must be called
`{\textit{prefix}\hspace{0.2em}\textit{suffix}}'
with \textit{prefix} matching precisely the argument.
Processing is then passed on to the file
`{\textit{dest}\hspace{0.2em}\textit{suffix}}'.
Surely, the same effect is achieved by
directly specifying the
argument `{\textit{dest}\hspace{0.2em}\textit{suffix}}'
in the first form.
However, that requires to set up a different file
for each child. With the alternative form of the command
all these files can have exactly the same content
which simplifies setting them up and maintaining them.

For example, the following file |draft.tex|
with a compilation flag |\version| as described in \secref{sec:flags}
compiles the main document as a draft:
%
\begin{center}
\begin{tabular}{l}
|\def\version{draft}|\\
|\input{childdoc.def}|\\
|\childdocforward{|\textit{main}|}|
\end{tabular}
\end{center}
%
Likewise, the following files |final|\textit{nn}|.tex|
compile the final version of the child document
|child|\textit{nn}|.tex|:
%
\begin{center}
\begin{tabular}{l}
|\def\version{final}|\\
|\input{childdoc.def}|\\
|\childdocforwardprefix{final}{child}|
\end{tabular}
\end{center}
%

Note that when several versions of a main file and/or of each child file
are to be generated, it may be convenient to set up a |Makefile| or
shell script to automatise the process.

%%%%%%%%%%%%%%%%%%%%%%%%%%%%%%%%%%%%%%%%%%%%%%%%%%%%%%%%%%%%%%%%%%%%%%%%%%%%%%%%
\subsection{Command Line Processing}
\label{sec:commandline}

The effect of redirection files can also be achieved by invoking
the \LaTeX{} compiler with a more elaborate command line.
Most conveniently this should be done as part
of a shell script or a |Makefile|.

When using \textsf{childdoc} in the main file, the following
command lines effectively perform a redirection
(note that depending on the shell being used,
backslashes may have to be doubled: `|\|' $\to$ `|\\|'):
%
\begin{center}
|... -jobname "|\textit{target}|" |\\|"|[\textit{flags}]%
|\input{childdoc.def}\childdocforward[|\textit{main}|]{|\textit{dest}|}"|
\end{center}
%
Here \textit{target} is the name of the output file,
\textit{main} is the name of the main file
and \textit{dest} is the name of the main or child file to be processed
(all filenames without extensions).
The optional argument \textit{main} can be omitted
if \textit{main} matches \textit{dest}.
Optionally, compilation \textit{flags} can be defined via |\def| commands.
This command line makes the \TeX{} engine believe
it is compiling the file \textit{target}
whose content is specified as the latter parameter.
The provided code then forwards the processing to
\textit{main} or \textit{dest} as described in \secref{sec:forward}.

%%%%%%%%%%%%%%%%%%%%%%%%%%%%%%%%%%%%%%%%%%%%%%%%%%%%%%%%%%%%%%%%%%%%%%%%%%%%%%%%
\subsection{Include by Input}
\label{sec:input}

Including child documents by |\include| has some restrictions by design.
Most notably, the content of a child document always occupies
its own set of pages; pages cannot be shared between child documents.
Usually, this behaviour makes perfect sense
because each child document contain an essential part of the document.
However, in some situations it may be desirable to compose
a document from a collection of parts
without having mandatory page breaks between then.
For this case, the package
provides a mechanism to include parts
by |\input| which can also be processed individually.
However, by construction this mechanism
requires manual handling of the content to be output.

%%%%%%%%%%%%%%%%%%%%%%%%%%%%%%%%%%%%%%%%
\DescribeMacro{\ifchilddocmanual}
The main file should be prepared as usual, see \secref{sec:include}.
However, the document body must make a distinction
between processing of an individual part and of the main document, e.g.:
%
\begin{center}
\begin{tabular}{l}
|\ifchilddocmanual|\\
|\input{\childdocname}|\\
|\||else|\\
\textit{document body with }|\input{|\textit{part}|}|\\
|\||fi|
\end{tabular}
\end{center}
%
The conditional |\ifchilddocmanual| is true whenever
a part to be included by |\input| is being compiled,
and the name of the part is stored in |\childdocname|.

%%%%%%%%%%%%%%%%%%%%%%%%%%%%%%%%%%%%%%%%
\DescribeMacro{\childdocby}
Each part to be included by |\input| should start with:
%
\begin{center}
\begin{tabular}{l}
|\input{childdoc.def}|\\
|\childdocby{|\textit{main}|}|\\
\end{tabular}
\end{center}
%
The directive |\childdocby| is similar to |\childdocof|
described in \secref{sec:include},
but the subsequent selection of content must be done manually.
To that end, both |\ifchilddoc| and |\ifchilddocmanual|
will be true upon processing of a part,
and the name of the part is stored in |\childdocname|.
Note that |\jobname| will be set to the filename of the current part
so that each part receives an individual |.aux| file
that does not interfere with the |.aux| file(s) of the main document.
This behaviour can be altered by the alternative form
|\childdocby[*]{|\textit{main}|}| (with a non-empty optional argument)
which uses the |.aux| file of the main document
by setting |\jobname| to \textit{main}.

%%%%%%%%%%%%%%%%%%%%%%%%%%%%%%%%%%%%%%%%%%%%%%%%%%%%%%%%%%%%%%%%%%%%%%%%%%%%%%%%
\subsection{Driver Development}
\label{sec:driver}

The \textsf{childdoc} mechanism can also be use for the development
of definition files such as \LaTeX{} styles or classes.
This case differs from the above setup with multiple parts
included by |\include| in that no |\includeonly| should be invoked.
This can be achieved by starting the include file
(before |\ProvidesPackage|) with:
%
\begin{center}
\begin{tabular}{l}
|\input{childdoc.def}|\\
|\childdocforward{|\textit{main}|}|\\
\end{tabular}
\end{center}
%
or alternatively with:
%
\begin{center}
\begin{tabular}{l}
|\input{childdoc.def}|\\
|\childdocby{|\textit{main}|}|\\
\end{tabular}
\end{center}
%
Both forms have slightly different effects as described above.
The main file is prepared as usual, see \secref{sec:include}.

%%%%%%%%%%%%%%%%%%%%%%%%%%%%%%%%%%%%%%%%%%%%%%%%%%%%%%%%%%%%%%%%%%%%%%%%%%%%%%%%
\subsection{Legacy Detection}
\label{sec:detection}

The directive |\childdocmain| in the main file can detect
whether the complete document or merely a child is to be compiled
even without using the directive |\childdocof|.
This method is deprecated because it is less robust
and there is no compelling reason to use it;
it is merely provided for backward compatibility
and it may be removed in future versions.

If the detection mechanism is to be used,
it is mandatory to correctly specify
the filename of the main file as the argument of |\childdocmain|:
%
\begin{center}
\begin{tabular}{l}
|\input{childdoc.def}|\\
|\childdocmain{|\textit{main}|}|\\
\end{tabular}
\end{center}
%
If |\jobname| does not match the argument \textit{main} of |\childdocmain|,
it is assumed that |\jobname| points to the child file to be compiled.
When using |\childdocmain| with the main file specified as argument,
it suffices to start a child file
with just |\input{|\textit{main}|}|
without loading of the package and using |\childdocof|.
If instead all processing is done
with the appropriate \textsf{childdoc} directives,
the argument of \textit{main} of |\childdocmain| can be empty.

An alternative version of the command line processing described
in \secref{sec:commandline} using the detection mechanism reads:
%
\begin{center}
|... -jobname "|\textit{target}|" "|[\textit{flags}]%
[|\def\jobname{|\textit{dest}|}|]|\input{|\textit{main}|}"|
\end{center}

%%%%%%%%%%%%%%%%%%%%%%%%%%%%%%%%%%%%%%%%%%%%%%%%%%%%%%%%%%%%%%%%%%%%%%%%%%%%%%%%
\subsection{Manual Code}
\label{sec:manual}

In case one cannot be certain whether the definitions file |childdoc.def|
is installed on the target \TeX{} distribution
and one prefers not to ship it,
it is conceivable to paste a few relevant commands into the sources.

To that end, drop all statements |\input{childdoc.def}|
and perform the replacements as outlined below.
Instead of |\childdocmain{|\textit{main}|}| add the following code
to the top of the main file:
%
\begin{center}
\begin{tabular}{l}
|\||ifdefined\childdocname\endinput\||fi\newif\ifchilddoc|\\
|\edef\childdocname{\scantokens\expandafter{\jobname\noexpand}}|\\
|\def\childdocmain{|\textit{main}|}\||ifx\childdocmain\childdocname\||else|\\
|\childdoctrue\includeonly{\childdocname}\let\jobname\childdocmain\||fi|\\
\end{tabular}
\end{center}
%
Instead of |\childdocof{|\textit{main}|}| just include the main file
at the top of each child file:
%
\begin{center}
|\input{|\textit{main}|}|
\end{center}
%
A simple redirection |\childdocforward{|\textit{dest}|}| is achieved by:
%
\begin{center}
|\def\jobname{|\textit{dest}|}\input{\jobname}|
\end{center}
%
The redirection with prefix
|\childdocforwardprefix[|\textit{prefix}|]{|\textit{dest}|}|
is accomplished by:
%
\begin{center}
\begin{tabular}{l}
|{\edef\jobname{\scantokens\expandafter{\jobname\noexpand}}|\\
|\def\redirectjob |\textit{prefix}|#1~~~{\gdef\jobname{|\textit{dest}|#1}}|\\
|\expandafter\redirectjob\jobname~~~}\input{\jobname}|
\end{tabular}
\end{center}

In an alternative approach,
child documents can be compiled by a specific command line
without additional code or specific definitions:
%
\begin{center}
|... -jobname "|\textit{target}|" "|[\textit{flags}]%
|\includeonly{|\textit{dest}|}\input{|\textit{main}|}"|
\end{center}
%

%%%%%%%%%%%%%%%%%%%%%%%%%%%%%%%%%%%%%%%%%%%%%%%%%%%%%%%%%%%%%%%%%%%%%%%%%%%%%%%%
%%%%%%%%%%%%%%%%%%%%%%%%%%%%%%%%%%%%%%%%%%%%%%%%%%%%%%%%%%%%%%%%%%%%%%%%%%%%%%%%
\section{Information}

%%%%%%%%%%%%%%%%%%%%%%%%%%%%%%%%%%%%%%%%%%%%%%%%%%%%%%%%%%%%%%%%%%%%%%%%%%%%%%%%
\subsection{Copyright}

Copyright \copyright{} 2017--2018 Niklas Beisert

This work may be distributed and/or modified under the
conditions of the \LaTeX{} Project Public License, either version 1.3
of this license or (at your option) any later version.
The latest version of this license is in
  \url{http://www.latex-project.org/lppl.txt}
and version 1.3 or later is part of all distributions of \LaTeX{}
version 2005/12/01 or later.

This work has the LPPL maintenance status `maintained'.

The Current Maintainer of this work is Niklas Beisert.

This work consists of the files |README.txt|, |childdoc.ins| and |childdoc.dtx|
as well as the derived files |childdoc.def|, |cdocsamp.tex|
with |cdocsch1.tex|, |cdocsch2.tex|, |cdocspt3.tex|, |cdocspt4.tex|,
|cdocsdrf.tex|, |cdocsfn1.tex|, |cdocsfn2.tex|
as well as |childdoc.pdf|.

%%%%%%%%%%%%%%%%%%%%%%%%%%%%%%%%%%%%%%%%%%%%%%%%%%%%%%%%%%%%%%%%%%%%%%%%%%%%%%%%
\subsection{Files and Installation}

The package consists of the files:
%
\begin{center}
\begin{tabular}{ll}
    |README.txt|   & readme file \\
    |childdoc.ins| & installation file \\
    |childdoc.dtx| & source file \\
    |childdoc.def| & definition file \\
    |cdocsamp.tex| & sample main file \\
    |cdocsch1.tex| & sample include file \\
    |cdocsch2.tex| & sample include file \\
    |cdocspt3.tex| & sample part file \\
    |cdocspt4.tex| & sample part file \\
    |cdocsdrf.tex| & sample redirection file \\
    |cdocsfn1.tex| & sample redirection file \\
    |cdocsfn2.tex| & sample redirection file \\
    |childdoc.pdf| & manual
\end{tabular}
\end{center}
%
The distribution consists of the files
|README.txt|, |childdoc.ins| and |childdoc.dtx|.
%
\begin{itemize}
\item
Run (pdf)\LaTeX{} on |childdoc.dtx|
to compile the manual |childdoc.pdf| (this file).
\item
Run \LaTeX{} on |childdoc.ins| to create the definitions file |childdoc.def|
and the sample |cdocsamp.tex| with include files
|cdocsch1.tex|, |cdocsch2.tex|, |cdocspt3.tex|, |cdocspt4.tex|,
|cdocsdrf.tex|, |cdocsfn1.tex|, |cdocsfn2.tex|.
Then copy the file |childdoc.def| to an appropriate directory of your \LaTeX{}
distribution, e.g.\ \textit{texmf-root}|/tex/latex/childdoc|.
\end{itemize}

%%%%%%%%%%%%%%%%%%%%%%%%%%%%%%%%%%%%%%%%%%%%%%%%%%%%%%%%%%%%%%%%%%%%%%%%%%%%%%%%
\subsection{Related CTAN Packages}

There are several other packages which offer a similar functionality:
%
\begin{itemize}
\item
The packages
\href{http://ctan.org/pkg/docmute}{\textsf{docmute}},
\href{http://ctan.org/pkg/includex}{\textsf{includex}} and
\href{http://ctan.org/pkg/standalone}{\textsf{standalone}}
provide commands to include only the document body of
a child file thus allowing both files to be compiled individually.
\item
The packages \href{http://ctan.org/pkg/subdocs}{\textsf{subdocs}}
and \href{http://ctan.org/pkg/subfiles}{\textsf{subfiles}}
provide structures in which the main and child documents can be
encapsulated and allowing them to be compiled individually.
The inclusion mechanism is different from the conventional |\include|.
\item
The package \href{http://ctan.org/pkg/combine}{\textsf{combine}}
is an elaborate solution to combine several documents into one.
\end{itemize}
%
See also the CTAN topic \href{http://ctan.org/topic/subdocs}{\textsf{subdocs}}
for further related packages.
The present package differs from the above solutions in that
a document structure constructed with the conventional |\include| mechanism
just needs two extra commands at the top of every file
such that all constituent files can be compiled individually.

%%%%%%%%%%%%%%%%%%%%%%%%%%%%%%%%%%%%%%%%%%%%%%%%%%%%%%%%%%%%%%%%%%%%%%%%%%%%%%%%
%\subsection{Feature Suggestions}
%
%The following is a list of features which may be useful for future
%versions of this package:
%%
%\begin{itemize}
%\item
%\ldots
%\end{itemize}

%%%%%%%%%%%%%%%%%%%%%%%%%%%%%%%%%%%%%%%%%%%%%%%%%%%%%%%%%%%%%%%%%%%%%%%%%%%%%%%%
\subsection{Revision History}

%%%%%%%%%%%%%%%%%%%%%%%%%%%%%%%%%%%%%%%%
\paragraph{v2.0:} 2018/12/30

\begin{itemize}
\item
immediate forward processing
\item
added |\childdocby| mechanism
\item
manual restructured
\end{itemize}

%%%%%%%%%%%%%%%%%%%%%%%%%%%%%%%%%%%%%%%%
\paragraph{v1.6:} 2018/01/17

\begin{itemize}
\item
application for development of include files
\item
corrections to manual
\end{itemize}

%%%%%%%%%%%%%%%%%%%%%%%%%%%%%%%%%%%%%%%%
\paragraph{v1.5:} 2017/05/21

\begin{itemize}
\item
more complete structuring introduced
\item
|\childdocof| introduced
\item
|\childdoc| renamed to |\childdocmain|
\item
|\childredirect| renamed to |\childdocforward| and |\childdocforwardprefix|
and functionality expanded
\end{itemize}

%%%%%%%%%%%%%%%%%%%%%%%%%%%%%%%%%%%%%%%%
\paragraph{v1.0:} 2017/04/27

\begin{itemize}
\item
manual and install package
\item
first version published on CTAN
\end{itemize}

%%%%%%%%%%%%%%%%%%%%%%%%%%%%%%%%%%%%%%%%
\paragraph{v0.6:} 2017/04/26

\begin{itemize}
\item
redirection mechanism added
\end{itemize}

%%%%%%%%%%%%%%%%%%%%%%%%%%%%%%%%%%%%%%%%
\paragraph{v0.5:} 2017/04/26

\begin{itemize}
\item
functionality in definition file
\end{itemize}


%%%%%%%%%%%%%%%%%%%%%%%%%%%%%%%%%%%%%%%%%%%%%%%%%%%%%%%%%%%%%%%%%%%%%%%%%%%%%%%%
%%%%%%%%%%%%%%%%%%%%%%%%%%%%%%%%%%%%%%%%%%%%%%%%%%%%%%%%%%%%%%%%%%%%%%%%%%%%%%%%
%%%%%%%%%%%%%%%%%%%%%%%%%%%%%%%%%%%%%%%%%%%%%%%%%%%%%%%%%%%%%%%%%%%%%%%%%%%%%%%%
\appendix

\settowidth\MacroIndent{\rmfamily\scriptsize 000\ }

 \DocInput{childdoc.dtx}

\end{document}
%</driver>
% \fi
%
% %%%%%%%%%%%%%%%%%%%%%%%%%%%%%%%%%%%%%%%%%%%%%%%%%%%%%%%%%%%%%%%%%%%%%%%%%%%%%%
% %%%%%%%%%%%%%%%%%%%%%%%%%%%%%%%%%%%%%%%%%%%%%%%%%%%%%%%%%%%%%%%%%%%%%%%%%%%%%%
% \section{Sample}
%\iffalse
%<*samplemain>
%\fi
%
% The following presents a sample document
% with two chapters, two parts, a title page,
% a compile flag as well as three forwarding files to set the flag.
% It consists of eight |.tex| files:
% \begin{center}
% \begin{tabular}{ll}
% |cdocsamp.tex|&main file\\
% |cdocsch1.tex|&include file for chapter 1\\
% |cdocsch2.tex|&include file for chapter 2\\
% |cdocspt3.tex|&include file for part 3\\
% |cdocspt4.tex|&include file for part 4\\
% |cdocsdrf.tex|&forwarding file for main file in draft mode\\
% |cdocsfi1.tex|&forwarding file for final version of chapter 1\\
% |cdocsfi2.tex|&forwarding file for final version of chapter 2\\
% \end{tabular}
% \end{center}
% Each of the eight files can be compiled directly by the \LaTeX{} compiler.
%
% %%%%%%%%%%%%%%%%%%%%%%%%%%%%%%%%%%%%%%
% \paragraph{Main File.}
%
% The main file is called |cdocsamp.tex|.
%
% Load the \textsf{childdoc} definitions and
% declare the filename for the main document:
%    \begin{macrocode}
\input{childdoc.def}
\childdocmain{}
%    \end{macrocode}

% Optional override for |\version| flag:
%    \begin{macrocode}
%%\ifchilddoc\else\providecommand{\version}{draft}\fi
%    \end{macrocode}

% Define the default values for the |\version| flag
% (|final| for the main file and |draft| for childs):
%    \begin{macrocode}
\ifchilddoc
\providecommand{\version}{draft}
\else
\providecommand{\version}{final}
\fi
%    \end{macrocode}

% Load the standard document class:
%    \begin{macrocode}
\documentclass[12pt]{article}
%    \end{macrocode}

% Start the document body:
%    \begin{macrocode}
\begin{document}
%    \end{macrocode}

% Declare a title page.
% Print title, part of document being processed and version flag:
%    \begin{macrocode}
\addtocounter{page}{-1}
\begin{center}
{\LARGE\bfseries{}childdoc example\par}
\vspace{1cm}
\ifchilddoc
\ifchilddocmanual part\else chapter\fi:
`\childdocname' of `\childdocjob'\par
\else
main document: `\childdocjob'\par
\fi
version: \version\par
\end{center}
\newpage
%    \end{macrocode}

% Manually include selected file,
% otherwise process as usual:
%    \begin{macrocode}
\ifchilddocmanual
\section*{part `\childdocname'}
\input{\childdocname}
\else
%    \end{macrocode}

% Include the two chapters:
%    \begin{macrocode}
\include{cdocsch1}
\include{cdocsch2}
%    \end{macrocode}

% Include the two parts unless only chapters should be displayed:
%    \begin{macrocode}
\ifchilddoc\else
\section{part three}
\input{cdocspt3}
\section{part four}
\input{cdocspt4}
\fi
%    \end{macrocode}

% Process as usual until here:
%    \begin{macrocode}
\fi
%    \end{macrocode}

% End of document body:
%    \begin{macrocode}
\end{document}
%    \end{macrocode}
%\iffalse
%</samplemain>
%\fi
%
% %%%%%%%%%%%%%%%%%%%%%%%%%%%%%%%%%%%%%%
% \paragraph{Chapter Include Files.}
%
% The include files are called |cdocsch1.tex| and |cdocsch2.tex|.
%
%\iffalse
%<*samplechap1|samplechap2>
%\fi

% Optional override for |\version| flag:
%    \begin{macrocode}
%%\providecommand{\version}{final}
%    \end{macrocode}

% Include the main document:
%    \begin{macrocode}
\input{childdoc.def}
\childdocof{cdocsamp}
%    \end{macrocode}

%\iffalse
%</samplechap1|samplechap2>
%\fi
%
%\iffalse
%<*samplechap1>
%\fi
% Some text for chapter 1:
%    \begin{macrocode}
\section{one}
some text in chapter one
%    \end{macrocode}

%\iffalse
%</samplechap1>
%\fi
% Some text for chapter 2:
%\iffalse
%<*samplechap2>
%\fi
%    \begin{macrocode}
\section{two}
more text in chapter two
%    \end{macrocode}

%\iffalse
%</samplechap2>
%\fi
%
% %%%%%%%%%%%%%%%%%%%%%%%%%%%%%%%%%%%%%%
% \paragraph{Part Include Files.}
%
% The include files are called |cdocspt3.tex| and |cdocspt4.tex|.
%
%\iffalse
%<*samplepart3|samplepart4>
%\fi

% Optional override for |\version| flag:
%    \begin{macrocode}
%%\providecommand{\version}{final}
%    \end{macrocode}

% Include the main document:
%    \begin{macrocode}
\input{childdoc.def}
\childdocby{cdocsamp}
%    \end{macrocode}

%\iffalse
%</samplepart3|samplepart4>
%\fi
%
%\iffalse
%<*samplepart3>
%\fi
% Some text for part 3:
%    \begin{macrocode}
some text in part three
%    \end{macrocode}

%\iffalse
%</samplepart3>
%\fi
% Some text for part 4:
%\iffalse
%<*samplepart4>
%\fi
%    \begin{macrocode}
more text in part four
%    \end{macrocode}

%\iffalse
%</samplepart4>
%\fi
%
% %%%%%%%%%%%%%%%%%%%%%%%%%%%%%%%%%%%%%%
% \paragraph{Forwarding for a Complete Draft.}
%
% The following forwarding file |cdocsdrf.tex|
% compiles the main document in draft mode:
%\iffalse
%<*sampledraft>
%\fi
%    \begin{macrocode}
\def\version{draft}
\input{childdoc.def}
\childdocforward{cdocsamp}
%    \end{macrocode}

%\iffalse
%</sampledraft>
%\fi
%
% %%%%%%%%%%%%%%%%%%%%%%%%%%%%%%%%%%%%%%
% \paragraph{Forwarding for Final Version of the Chapters.}
%
% The following forwarding files |cdocsfn1.tex| and |cdocsfn2.tex|
% (with identical content)
% compile the final versions of the child documents
% |cdocsch1.tex| and |cdocsch2.tex|, respectively:
%\iffalse
%<*samplefinal>
%\fi
%    \begin{macrocode}
\def\version{final}
\input{childdoc.def}
\childdocforwardprefix[cdocsamp]{cdocsfn}{cdocsch}
%    \end{macrocode}

%\iffalse
%</samplefinal>
%\fi
%
% %%%%%%%%%%%%%%%%%%%%%%%%%%%%%%%%%%%%%%
% \paragraph{Command Line Processing.}
%
% The following three command lines generate the output files
% |cdocscld|, |cdocscl1| and |cdocscl2|
% which should be identical to
% |cdocsdrf|, |cdocsch1| and |cdocsfn2|, respectively:
% \begin{center}
% \begin{tabular}{l}
% |latex -jobname cdocscld \|\\
% |  "\def\version{draft}\input{childdoc.def}\childdocforward{cdocsamp}"|\\
% |latex -jobname cdocscl1 \|\\
% |  "\input{childdoc.def}\childdocforward[cdocsamp]{cdocsch1}"|\\
% |latex -jobname cdocscl2 \|\\
% |  "\def\version{final}\input{childdoc.def}\childdocforward{cdocsch2}"|
% \end{tabular}
% \end{center}
% Note that the trailing backslash on each first line
% merely continues the input to the second line
% (for convenient cut ant paste).
% Furthermore, the command |latex| can be replaced by any
% of its alternative versions such as |pdflatex|.
%
% %%%%%%%%%%%%%%%%%%%%%%%%%%%%%%%%%%%%%%%%%%%%%%%%%%%%%%%%%%%%%%%%%%%%%%%%%%%%%%
% %%%%%%%%%%%%%%%%%%%%%%%%%%%%%%%%%%%%%%%%%%%%%%%%%%%%%%%%%%%%%%%%%%%%%%%%%%%%%%
% \section{Implementation}
%\iffalse
%<*package>
%\fi
%
% This section describes the definitions file |childdoc.def|.

% The definitions cannot be loaded using |\usepackage| or |\RequirePackage|
% which has a mechanism to prevent loading a style file more than once.
% When loading the definitions by means of |\input|
% multiple instances have to be prevented manually:
%\iffalse
%This code needs to be before the `\ProvidesFile' directive
%which is defined at the beginning of this file.
%Therefore it is also placed there and commented out here.
%</package>
%<*discard>
%\fi
%    \begin{macrocode}
\ifdefined\childdocmain\endinput\fi
%    \end{macrocode}
%\iffalse
%</discard>
%<*package>
%\fi
%
% \macro{\ifchilddoc}
% \macro{\ifchilddocmanual}
% The conditional |\ifchilddoc| tells whether a
% child (true) or main (false) document is being compiled.
% The conditional |\ifchilddocmanual| tells whether
% the |\includeonly| mechanism is used (false) or
% the selection of child files must be performed manually (true).
% The definitions initialise to false:
%    \begin{macrocode}
\newif\ifchilddoc
\newif\ifchilddocmanual
%    \end{macrocode}

% \macro{\childdocname}
% \macro{\childdocjob}
% The macro |\childdocname| stores the name of the main document
% to be compiled. The macro |\childdocjob| stores the name of
% the document on which the \LaTeX{} compiler was originally invoked.
% The content of |\jobname| cannot be compared
% to filenames specified in the source due to different catcodes.
% The following code rescans |\jobname|, stores the result
% in |\childdocname| and saves a copy in |\childdocjob|:
%    \begin{macrocode}
\edef\childdocname{\scantokens\expandafter{\jobname\noexpand}}
\let\childdocjob\childdocname
%    \end{macrocode}

% \macro{\childdocdisable}
% The macro |\childdocdisable| prevents the main file
% from being processed more than once.
% At this stage, the main document command |\childdocmain|
% is assumed to be called once again where it should do nothing.
% Any subsequent call to it should prevent
% a secondary processing of the main document
% It overwrites the forwarding commands
% |\childdocof| and |\childdocforward|
% with empty macros to prevent further inclusions of the main document:
%    \begin{macrocode}
\newcommand{\childdocdisable}
{
  \renewcommand{\childdocmain}[1]{\renewcommand{\childdocmain}[1]{\endinput}}
  \renewcommand{\childdocof}[1]{}
  \renewcommand{\childdocby}[2][]{}
  \renewcommand{\childdocforward}[2][]{}
  \renewcommand{\childdocdisable}{}
}
%    \end{macrocode}

% \macro{\childdocmain}
% The macro |\childdocmain| is to be called at the top of the main file
% with nothing or the main filename (without extension) as argument.
% First, it breaks loops.
% If the argument is not empty and does not match |\childdocname|
% (which is set by the first inclusion of |childdoc.def|),
% |\ifchilddoc| is set to true, |\includeonly| is applied to the child file
% and |\jobname| is set to the main file
% (for proper handling of |.aux| files):
%    \begin{macrocode}
\newcommand{\childdocmain}[1]
{
  \childdocdisable\childdocmain{}
  \if?#1?\else
    \begingroup
      \def\childdoctmp{#1}
      \ifx\childdoctmp\childdocname
        \def\childdoctmp{}
      \else
        \def\childdoctmp
        {
          \childdoctrue
          \includeonly{\childdocname}
          \def\childdocjob{#1}
          \def\jobname{#1}
        }
      \fi
      \expandafter
    \endgroup
    \childdoctmp
  \fi
}
%    \end{macrocode}

% \macro{\childdocof}
% The command |\childdocof| redirects
% compilation to the main file |#1|.
%    \begin{macrocode}
\newcommand{\childdocof}[1]
{
  \childdocdisable
  \childdoctrue
  \includeonly{\childdocname}
  \def\jobname{#1}
  \def\childdocjob{#1}
  \input{#1}
}
%    \end{macrocode}

% \macro{\childdocby}
% The command |\childdocby| ....
%    \begin{macrocode}
\newcommand{\childdocby}[2][]
{
  \childdocdisable
  \childdoctrue
  \childdocmanualtrue
  \if?#1?\else
    \def\jobname{#2}
  \fi
  \def\childdocjob{#2}
  \input{#2}
  \endinput
}
%    \end{macrocode}

% \macro{\childdocforward}
% The command |\childdocforward| redirects
% compilation to the main file or
% (if the optional argument is given) a child file.
% Parameters are set as if the main file
% or a child file starting with |\childdocof| was compiled.
% Then compilation is handed over to the main file:
%    \begin{macrocode}
\newcommand{\childdocforward}[2][]
{
  \begingroup
    \if?#1?
      \def\childdoctmp
      {
        \def\childdocname{#2}
        \def\childdocjob{#2}
        \def\jobname{#2}
        \input{#2}
        \endinput
      }
    \else
      \def\childdoctmp
      {
        \childdocdisable
        \def\childdocname{#2}
        \childdoctrue
        \includeonly{#2}
        \def\childdocjob{#1}
        \def\jobname{#1}
        \input{#1}
        \endinput
      }
    \fi
    \expandafter
  \endgroup
  \childdoctmp
}
%    \end{macrocode}

% \macro{\childdocforwardprefix}
% The command |\childdocforwardprefix| redirects
% compilation to the main or a child file by means of a pattern.
% The prefix |#1| in the current filename is replaced by |#2|
% and the suffix of the current filename is kept
% (it is assumed that the filename does not contain the substring `|~~~|'
% which is used as a delimiter).
% Compilation is handed over to the new file by |\childdocforward|:
%    \begin{macrocode}
\newcommand{\childdocforwardprefix}[3][]
{
  \begingroup
    \def\childdocextract #2##1~~~{\def\childdoctmp{\childdocforward[#1]{#3##1}}}
    \expandafter\childdocextract\childdocname~~~
    \expandafter
  \endgroup
  \childdoctmp
}
%    \end{macrocode}

% \macro{\childdoc}
% The deprecated macro |\childdoc| is a legacy version of |\childdocmain|:
%    \begin{macrocode}
\newcommand{\childdoc}{\childdocmain}
%    \end{macrocode}

% \macro{\childdocredirect}
% The deprecated macro |\childdocredirect| is a legacy version
% of |\childdocforward| and |\childdocforwardprefix|:
%    \begin{macrocode}
\newcommand{\childdocredirect}[2][]
{
  \begingroup
    \if?#1?
      \def\childdoctmp{\childdocforward{#2}}
    \else
      \def\childdoctmp{\childdocforwardprefix{#1}{#2}}
    \fi
    \expandafter
  \endgroup
  \childdoctmp
}
%    \end{macrocode}

%\iffalse
%</package>
%\fi
%
\endinput

\childdocof{cdocsamp}
%    \end{macrocode}

%\iffalse
%</samplechap1|samplechap2>
%\fi
%
%\iffalse
%<*samplechap1>
%\fi
% Some text for chapter 1:
%    \begin{macrocode}
\section{one}
some text in chapter one
%    \end{macrocode}

%\iffalse
%</samplechap1>
%\fi
% Some text for chapter 2:
%\iffalse
%<*samplechap2>
%\fi
%    \begin{macrocode}
\section{two}
more text in chapter two
%    \end{macrocode}

%\iffalse
%</samplechap2>
%\fi
%
% %%%%%%%%%%%%%%%%%%%%%%%%%%%%%%%%%%%%%%
% \paragraph{Part Include Files.}
%
% The include files are called |cdocspt3.tex| and |cdocspt4.tex|.
%
%\iffalse
%<*samplepart3|samplepart4>
%\fi

% Optional override for |\version| flag:
%    \begin{macrocode}
%%\providecommand{\version}{final}
%    \end{macrocode}

% Include the main document:
%    \begin{macrocode}
% \iffalse
%
% childdoc.dtx Copyright (C) 2017-2018 Niklas Beisert
%
% This work may be distributed and/or modified under the
% conditions of the LaTeX Project Public License, either version 1.3
% of this license or (at your option) any later version.
% The latest version of this license is in
%   http://www.latex-project.org/lppl.txt
% and version 1.3 or later is part of all distributions of LaTeX
% version 2005/12/01 or later.
%
% This work has the LPPL maintenance status `maintained'.
%
% The Current Maintainer of this work is Niklas Beisert.
%
% This work consists of the files childdoc.dtx and childdoc.ins
% and the derived files childdoc.def and cdocsamp.tex with
% cdocsch1.tex, cdocsch2.tex, cdocsdrf.tex, cdocsfn1.tex, cdocsfn2.tex.
%
%<package>\ifdefined\childdocmain\endinput\fi
%<package>\ProvidesFile{childdoc.def}[2018/12/30 v2.0 child document driver]
%<samplemain>\ProvidesFile{cdocsamp.tex}[2018/12/30 v2.0 sample for childdoc]
%<*driver>
%\ProvidesFile{childdoc.drv}[2018/12/30 v2.0 childdoc reference manual file]
\PassOptionsToClass{10pt,a4paper}{article}
\documentclass{ltxdoc}

\usepackage[margin=35mm]{geometry}
\usepackage{hyperref}
\usepackage{hyperxmp}
\usepackage[usenames]{color}

\hypersetup{colorlinks=true}
\hypersetup{pdfstartview=FitH}
\hypersetup{pdfpagemode=UseNone}
\hypersetup{pdfsource={}}
\hypersetup{pdflang={en-UK}}
\hypersetup{pdfcopyright={Copyright 2017-2018 Niklas Beisert.
  This work may be distributed and/or modified under the
  conditions of the LaTeX Project Public License, either version 1.3
  of this license or (at your option) any later version.}}
\hypersetup{pdflicenseurl={http://www.latex-project.org/lppl.txt}}
\hypersetup{pdfcontactaddress={ETH Zurich, ITP, HIT K,
  Wolfgang-Pauli-Strasse 27}}
\hypersetup{pdfcontactpostcode={8093}}
\hypersetup{pdfcontactcity={Zurich}}
\hypersetup{pdfcontactcountry={Switzerland}}
\hypersetup{pdfcontactemail={nbeisert@itp.phys.ethz.ch}}
\hypersetup{pdfcontacturl={http://people.phys.ethz.ch/\xmptilde nbeisert/}}

\newcommand{\secref}[1]{\hyperref[#1]{section \ref*{#1}}}

\parskip1ex
\parindent0pt
\let\olditemize\itemize
\def\itemize{\olditemize\parskip0pt}

\begin{document}

\title{The \textsf{childdoc} Package}
\hypersetup{pdftitle={The childdoc Package}}
\author{Niklas Beisert\\[2ex]
  Institut f\"ur Theoretische Physik\\
  Eidgen\"ossische Technische Hochschule Z\"urich\\
  Wolfgang-Pauli-Strasse 27, 8093 Z\"urich, Switzerland\\[1ex]
  \href{mailto:nbeisert@itp.phys.ethz.ch}
  {\texttt{nbeisert@itp.phys.ethz.ch}}}
\hypersetup{pdfauthor={Niklas Beisert}}
\hypersetup{pdfsubject={Manual for the LaTeX2e Package childdoc}}
\date{30 December 2018, \textsf{v2.0}}
\maketitle

\begin{abstract}\noindent
\textsf{childdoc} is a \LaTeXe{} package
that enables the direct compilation
of document sections included by |\include|
to individual files.
\end{abstract}

\begingroup
\parskip0ex
\tableofcontents
\endgroup

%%%%%%%%%%%%%%%%%%%%%%%%%%%%%%%%%%%%%%%%%%%%%%%%%%%%%%%%%%%%%%%%%%%%%%%%%%%%%%%%
%%%%%%%%%%%%%%%%%%%%%%%%%%%%%%%%%%%%%%%%%%%%%%%%%%%%%%%%%%%%%%%%%%%%%%%%%%%%%%%%
\section{Introduction}

\LaTeX{} provides a mechanism to structure a large document (such as a book)
into a main file and several child files (containing the chapters)
using the |\include| command.
This mechanism is beneficial for documents
which span hundreds of pages in order to
make the source file(s) more manageable.
Moreover, compilation can be restricted to
selected child files by means of the |\includeonly| command.
The latter feature can be used to reduce the compilation time while editing
(this was significantly more useful in the earlier days of \LaTeX{})
or to generate a smaller document which is easier to navigate.
Another application of |\includeonly| is to generate
documents consisting of selected parts of the complete document.

However, there are a few drawbacks of the plain |\include| mechanism:
\begin{itemize}
\item
The child files cannot be compiled on their own,
they can only be compiled via the main file.
A naive editing environment
(such as a text editor with an option
to have the current file processed by \LaTeX)
may require one to switch to the main file before compiling;
attempting to compile the child file produces errors.
\item
The main file must be modified (each time)
to adjust the |\includeonly| command
to the present needs. This easily leaves the main file in a messy state.
\item
The generated document will always carry the filename
of the main document. This is inconvenient if
several child files are to be compiled and
to be kept for distribution.
\end{itemize}

The present package provides a simple interface
to make child files individually compilable by \LaTeX{}.
Compiling a child file then has the same effect as compiling
the main file with an |\includeonly| command
to select the appropriate child.
Moreover the generated document will carry the name of the child
rather than the main file.
This resolves all three above issues.

This feature is meant to make the editing of books,
thesis documents and lecture notes somewhat more convenient.
However, the package can also be used efficiently for
composing a series of documents (such as exercise sheets)
which are typically distributed individually.
It then assists the author in generating the individual documents
(potentially in different versions)
as well as a document containing the collected series.
Another application is in developing style files
or other kinds of included material
where compilation of the style file could redirect
to a sample or test file.

%%%%%%%%%%%%%%%%%%%%%%%%%%%%%%%%%%%%%%%%%%%%%%%%%%%%%%%%%%%%%%%%%%%%%%%%%%%%%%%%
%%%%%%%%%%%%%%%%%%%%%%%%%%%%%%%%%%%%%%%%%%%%%%%%%%%%%%%%%%%%%%%%%%%%%%%%%%%%%%%%
\section{Usage}

First of all, the package \textsf{childdoc} is \emph{not} a standard
\LaTeXe{} |.sty| style file! Therefore it needs to be invoked in
a non-standard way.

%%%%%%%%%%%%%%%%%%%%%%%%%%%%%%%%%%%%%%%%%%%%%%%%%%%%%%%%%%%%%%%%%%%%%%%%%%%%%%%%
\subsection{Included Files}
\label{sec:include}

%%%%%%%%%%%%%%%%%%%%%%%%%%%%%%%%%%%%%%%%
\DescribeMacro{\childdocmain}
To use the package, add the commands
\begin{center}
\begin{tabular}{l}
|\input{childdoc.def}|\\
|\childdocmain{}|\\
\end{tabular}
\end{center}
at the very top of the main \LaTeX{} file,
in particular \emph{before} the |\documentclass| statement!
The argument of |\childdocmain| should be left empty
(but it must be present).

%%%%%%%%%%%%%%%%%%%%%%%%%%%%%%%%%%%%%%%%
\DescribeMacro{\childdocof}
Furthermore, add the commands
\begin{center}
\begin{tabular}{l}
|\input{childdoc.def}|\\
|\childdocof{|\textit{main}|}|\\
\end{tabular}
\end{center}
at the top of every child file \textit{child}
which is included by |\include{|\textit{child}|}|
from within the main file
(or at least for those files to be compiled individually).
The argument \textit{main} must be the filename of the main file.

There are a couple of
considerations in setting up the main and child documents:

%%%%%%%%%%%%%%%%%%%%%%%%%%%%%%%%%%%%%%%%
\paragraph{Restrictions.}

Please note the following restrictions:
\begin{itemize}
\item
|\childdocmain| must be called with one argument \textit{main}
to ensure compatibility with earlier version of the package.
It must either be empty (|\childdocmain{}|)
or precisely match the filename of the main file in which it is specified.
See \secref{sec:detection} for further information.
\item
The filename \textit{main} must be specified without the |.tex| extension.
\item
The filename \textit{main} is case sensitive
(even in case-insensitive file systems)
due to internal string comparison.
\item
The argument \textit{main} should be fully expanded, it cannot be a macro.
\item
Subdirectories and special characters should be avoided in filenames.
\item
The command |\childdocmain{|\textit{main}|}| must be followed by a whitespace.
It should not be followed immediately by another command
or by a comment mark `|%|'.
This is because the \TeX{} parser reads the token immediately following
the argument of |\childdocmain| and puts it
at the beginning of every child section;
however, a white\-space is ignored.
\end{itemize}

%%%%%%%%%%%%%%%%%%%%%%%%%%%%%%%%%%%%%%%%
\paragraph{Content of Main File.}

It is advisable to place all content in the child files included by |\include|.
Any output contained in the main file will appear in all child documents
unless suppressed manually;
it cannot be suppressed automatically by the |\includeonly| directive
and thus should normally be avoided.
A method to include some content in the main file
by means of conditional processing is described in \secref{sec:conditional}.

%%%%%%%%%%%%%%%%%%%%%%%%%%%%%%%%%%%%%%%%
\paragraph{Page Numbering.}

When only a part of the document is compiled,
the appropriate numbering of pages
(as well as other status parameters)
is determined from the |.aux| files.
The latter contain information from previous passes.
However this information needs to propagate through
all intermediate child documents.
Therefore the page numbering in child documents may well
be inconsistent until the complete document is compiled at least once.

A useful (if unconventional) way to always ensure a consistent
page numbering is to restart the numbering in each child document
and denote the pages by `\textit{child}|.|\textit{page}'
where \textit{child} represents the chapter/section number of the child file.
This can be achieved by the command
|\numberwithin{page}{|\textit{child}|}|
of the \textsf{amsmath} package
where \textit{child} can be |chapter| or |section|
depending on the chosen structuring.
Alternatively, one can modify the macro |\thepage| appropriately
and reset the counter |page| at the start of each child file.

%%%%%%%%%%%%%%%%%%%%%%%%%%%%%%%%%%%%%%%%%%%%%%%%%%%%%%%%%%%%%%%%%%%%%%%%%%%%%%%%
\subsection{Conditional Processing}
\label{sec:conditional}

The package provides a mechanism to compile different versions
of a document. To customise the versions further some conditional processing
can come in handy to distinguish which version is being compiled.
The package provides two macros to describe the compilation context:

%%%%%%%%%%%%%%%%%%%%%%%%%%%%%%%%%%%%%%%%
\DescribeMacro{\ifchilddoc}
The conditional |\ifchilddoc| distinguishes between the compilation of
child documents and the main document:
%
\begin{center}
|\ifchilddoc |\textit{child-code}| |[|\||else |\textit{main-code}]| \||fi|
\end{center}

%%%%%%%%%%%%%%%%%%%%%%%%%%%%%%%%%%%%%%%%
\DescribeMacro{\childdocname}
\DescribeMacro{\childdocjob}
The macro |\childdocname| contains the filename (without extension)
of the main or child file being processed.
Note that |\childdocjob| will always contain the name of the main file.

%%%%%%%%%%%%%%%%%%%%%%%%%%%%%%%%%%%%%%%%
\paragraph{Title Page.}

Conditional processing can be used to include a title or banner page
in the main document when proper precautions are taken.
Importantly, the code in the main file should ensure that the page counter
(as well as other status parameters which are stored in the |.aux| files)
takes the same value after the conditional processing.
Otherwise the page numbers may take divergent values
depending on which part is compiled.

For example, a title page could be declared by:
%
\begin{center}
\begin{tabular}{l}
|\ifchilddoc\||else|\\
|\addtocounter{page}{-1}|\\
\textit{code for title page}\\
|\newpage|\\
|\||fi|
\end{tabular}
\end{center}
%
A banner page for the child documents can be generated by:
%
\begin{center}
\begin{tabular}{l}
|\ifchilddoc|\\
|\addtocounter{page}{-1}|\\
\textit{code for banner page}\\
|\newpage|\\
|\||fi|
\end{tabular}
\end{center}
%
Here one could write a message such as:
\begin{center}
|This is the part \childdocname{} of \childdocjob{}.|
\end{center}

%%%%%%%%%%%%%%%%%%%%%%%%%%%%%%%%%%%%%%%%%%%%%%%%%%%%%%%%%%%%%%%%%%%%%%%%%%%%%%%%
\subsection{Flags}
\label{sec:flags}

The package makes it easy to generate different versions
of the main or child documents.
To this end compilation flags can be defined
and assigned different default values.
They will be particularly useful in conjunction
with the forwarding mechanism described in \secref{sec:forward}.

For example, it may be useful to have a flag |\version|
which can be set to |draft| or |final|.
The document source will contain some conditional code
depending on the value of |\version|.
Suppose further, the flag should default to |final| for the main file
and to |draft| for child files
which is a natural assignment for editing the document.
This is achieved by placing the following code
in the preamble of the main document
(below the |\childdocmain| directive):
%
\begin{center}
\begin{tabular}{l}
|\ifchilddoc|\\
|\providecommand{\version}{draft}|\\
|\||else|\\
|\providecommand{\version}{final}|\\
|\||fi|
\end{tabular}
\end{center}
%
The definition by |\providecommand| makes sure
that previous definitions are not overwritten.
Further statements |\providecommand{\version}{...}|
can thus be added before the above code to override it.

For the main file, one might add a line
(between |\childdocmain| and the above block)
%
\begin{center}
|%\ifchilddoc\||else\providecommand{\version}{draft}\||fi|
\end{center}
%
which can be uncommented to produce a draft version.
Likewise one can add a line to the very top of a child file
(above the |\childdocof{|\textit{main}|}| directive)
%
\begin{center}
|%\providecommand{\version}{final}|
\end{center}
%
which can be uncommented to produce the final version of this child document.

%%%%%%%%%%%%%%%%%%%%%%%%%%%%%%%%%%%%%%%%%%%%%%%%%%%%%%%%%%%%%%%%%%%%%%%%%%%%%%%%
\subsection{Forwarding}
\label{sec:forward}

Different versions of the main or child documents
using compilation flags as described in \secref{sec:flags}
can be (permanently) stored in different files
for convenient compilation, viewing and distribution.
To this end, the package defines a command
to pass on compilation to a different file:

%%%%%%%%%%%%%%%%%%%%%%%%%%%%%%%%%%%%%%%%
\DescribeMacro{\childdocforward}
The command |\childdocforward| redirects processing to
another source file:
%
\begin{center}
\begin{tabular}{l}
|\input{childdoc.def}|\\
|\childdocforward[|\textit{main}|]{|\textit{dest}|}|\\
\end{tabular}
\end{center}
%
The argument \textit{dest} is the destination file
(without extension).
It should be the main file or one of the child files.
Note that further \textsf{childdoc} directives
such as |\childdocof| and |\childdocforward|
in the indicated file will be processed in this form.
The optional argument \textit{main}
passes on directly to the main file \textit{main}
while pretending to compile the child \textit{dest}.
This form behaves as if \textit{dest}
issues |\childdocof{|\textit{main}|}| right away,
and no further \textsf{childdoc} directives will be processed.

%%%%%%%%%%%%%%%%%%%%%%%%%%%%%%%%%%%%%%%%
\DescribeMacro{\...prefix}
In the alternative form |\childdocforwardprefix|,
%
\begin{center}
\begin{tabular}{l}
|\input{childdoc.def}|\\
|\childdocforwardprefix[|\textit{main}|]{|\textit{prefix}|}{|\textit{dest}|}|
\end{tabular}
\end{center}
%
the destination file is determined by a pattern
depending on the current file:
To make this work, the current file must be called
`{\textit{prefix}\hspace{0.2em}\textit{suffix}}'
with \textit{prefix} matching precisely the argument.
Processing is then passed on to the file
`{\textit{dest}\hspace{0.2em}\textit{suffix}}'.
Surely, the same effect is achieved by
directly specifying the
argument `{\textit{dest}\hspace{0.2em}\textit{suffix}}'
in the first form.
However, that requires to set up a different file
for each child. With the alternative form of the command
all these files can have exactly the same content
which simplifies setting them up and maintaining them.

For example, the following file |draft.tex|
with a compilation flag |\version| as described in \secref{sec:flags}
compiles the main document as a draft:
%
\begin{center}
\begin{tabular}{l}
|\def\version{draft}|\\
|\input{childdoc.def}|\\
|\childdocforward{|\textit{main}|}|
\end{tabular}
\end{center}
%
Likewise, the following files |final|\textit{nn}|.tex|
compile the final version of the child document
|child|\textit{nn}|.tex|:
%
\begin{center}
\begin{tabular}{l}
|\def\version{final}|\\
|\input{childdoc.def}|\\
|\childdocforwardprefix{final}{child}|
\end{tabular}
\end{center}
%

Note that when several versions of a main file and/or of each child file
are to be generated, it may be convenient to set up a |Makefile| or
shell script to automatise the process.

%%%%%%%%%%%%%%%%%%%%%%%%%%%%%%%%%%%%%%%%%%%%%%%%%%%%%%%%%%%%%%%%%%%%%%%%%%%%%%%%
\subsection{Command Line Processing}
\label{sec:commandline}

The effect of redirection files can also be achieved by invoking
the \LaTeX{} compiler with a more elaborate command line.
Most conveniently this should be done as part
of a shell script or a |Makefile|.

When using \textsf{childdoc} in the main file, the following
command lines effectively perform a redirection
(note that depending on the shell being used,
backslashes may have to be doubled: `|\|' $\to$ `|\\|'):
%
\begin{center}
|... -jobname "|\textit{target}|" |\\|"|[\textit{flags}]%
|\input{childdoc.def}\childdocforward[|\textit{main}|]{|\textit{dest}|}"|
\end{center}
%
Here \textit{target} is the name of the output file,
\textit{main} is the name of the main file
and \textit{dest} is the name of the main or child file to be processed
(all filenames without extensions).
The optional argument \textit{main} can be omitted
if \textit{main} matches \textit{dest}.
Optionally, compilation \textit{flags} can be defined via |\def| commands.
This command line makes the \TeX{} engine believe
it is compiling the file \textit{target}
whose content is specified as the latter parameter.
The provided code then forwards the processing to
\textit{main} or \textit{dest} as described in \secref{sec:forward}.

%%%%%%%%%%%%%%%%%%%%%%%%%%%%%%%%%%%%%%%%%%%%%%%%%%%%%%%%%%%%%%%%%%%%%%%%%%%%%%%%
\subsection{Include by Input}
\label{sec:input}

Including child documents by |\include| has some restrictions by design.
Most notably, the content of a child document always occupies
its own set of pages; pages cannot be shared between child documents.
Usually, this behaviour makes perfect sense
because each child document contain an essential part of the document.
However, in some situations it may be desirable to compose
a document from a collection of parts
without having mandatory page breaks between then.
For this case, the package
provides a mechanism to include parts
by |\input| which can also be processed individually.
However, by construction this mechanism
requires manual handling of the content to be output.

%%%%%%%%%%%%%%%%%%%%%%%%%%%%%%%%%%%%%%%%
\DescribeMacro{\ifchilddocmanual}
The main file should be prepared as usual, see \secref{sec:include}.
However, the document body must make a distinction
between processing of an individual part and of the main document, e.g.:
%
\begin{center}
\begin{tabular}{l}
|\ifchilddocmanual|\\
|\input{\childdocname}|\\
|\||else|\\
\textit{document body with }|\input{|\textit{part}|}|\\
|\||fi|
\end{tabular}
\end{center}
%
The conditional |\ifchilddocmanual| is true whenever
a part to be included by |\input| is being compiled,
and the name of the part is stored in |\childdocname|.

%%%%%%%%%%%%%%%%%%%%%%%%%%%%%%%%%%%%%%%%
\DescribeMacro{\childdocby}
Each part to be included by |\input| should start with:
%
\begin{center}
\begin{tabular}{l}
|\input{childdoc.def}|\\
|\childdocby{|\textit{main}|}|\\
\end{tabular}
\end{center}
%
The directive |\childdocby| is similar to |\childdocof|
described in \secref{sec:include},
but the subsequent selection of content must be done manually.
To that end, both |\ifchilddoc| and |\ifchilddocmanual|
will be true upon processing of a part,
and the name of the part is stored in |\childdocname|.
Note that |\jobname| will be set to the filename of the current part
so that each part receives an individual |.aux| file
that does not interfere with the |.aux| file(s) of the main document.
This behaviour can be altered by the alternative form
|\childdocby[*]{|\textit{main}|}| (with a non-empty optional argument)
which uses the |.aux| file of the main document
by setting |\jobname| to \textit{main}.

%%%%%%%%%%%%%%%%%%%%%%%%%%%%%%%%%%%%%%%%%%%%%%%%%%%%%%%%%%%%%%%%%%%%%%%%%%%%%%%%
\subsection{Driver Development}
\label{sec:driver}

The \textsf{childdoc} mechanism can also be use for the development
of definition files such as \LaTeX{} styles or classes.
This case differs from the above setup with multiple parts
included by |\include| in that no |\includeonly| should be invoked.
This can be achieved by starting the include file
(before |\ProvidesPackage|) with:
%
\begin{center}
\begin{tabular}{l}
|\input{childdoc.def}|\\
|\childdocforward{|\textit{main}|}|\\
\end{tabular}
\end{center}
%
or alternatively with:
%
\begin{center}
\begin{tabular}{l}
|\input{childdoc.def}|\\
|\childdocby{|\textit{main}|}|\\
\end{tabular}
\end{center}
%
Both forms have slightly different effects as described above.
The main file is prepared as usual, see \secref{sec:include}.

%%%%%%%%%%%%%%%%%%%%%%%%%%%%%%%%%%%%%%%%%%%%%%%%%%%%%%%%%%%%%%%%%%%%%%%%%%%%%%%%
\subsection{Legacy Detection}
\label{sec:detection}

The directive |\childdocmain| in the main file can detect
whether the complete document or merely a child is to be compiled
even without using the directive |\childdocof|.
This method is deprecated because it is less robust
and there is no compelling reason to use it;
it is merely provided for backward compatibility
and it may be removed in future versions.

If the detection mechanism is to be used,
it is mandatory to correctly specify
the filename of the main file as the argument of |\childdocmain|:
%
\begin{center}
\begin{tabular}{l}
|\input{childdoc.def}|\\
|\childdocmain{|\textit{main}|}|\\
\end{tabular}
\end{center}
%
If |\jobname| does not match the argument \textit{main} of |\childdocmain|,
it is assumed that |\jobname| points to the child file to be compiled.
When using |\childdocmain| with the main file specified as argument,
it suffices to start a child file
with just |\input{|\textit{main}|}|
without loading of the package and using |\childdocof|.
If instead all processing is done
with the appropriate \textsf{childdoc} directives,
the argument of \textit{main} of |\childdocmain| can be empty.

An alternative version of the command line processing described
in \secref{sec:commandline} using the detection mechanism reads:
%
\begin{center}
|... -jobname "|\textit{target}|" "|[\textit{flags}]%
[|\def\jobname{|\textit{dest}|}|]|\input{|\textit{main}|}"|
\end{center}

%%%%%%%%%%%%%%%%%%%%%%%%%%%%%%%%%%%%%%%%%%%%%%%%%%%%%%%%%%%%%%%%%%%%%%%%%%%%%%%%
\subsection{Manual Code}
\label{sec:manual}

In case one cannot be certain whether the definitions file |childdoc.def|
is installed on the target \TeX{} distribution
and one prefers not to ship it,
it is conceivable to paste a few relevant commands into the sources.

To that end, drop all statements |\input{childdoc.def}|
and perform the replacements as outlined below.
Instead of |\childdocmain{|\textit{main}|}| add the following code
to the top of the main file:
%
\begin{center}
\begin{tabular}{l}
|\||ifdefined\childdocname\endinput\||fi\newif\ifchilddoc|\\
|\edef\childdocname{\scantokens\expandafter{\jobname\noexpand}}|\\
|\def\childdocmain{|\textit{main}|}\||ifx\childdocmain\childdocname\||else|\\
|\childdoctrue\includeonly{\childdocname}\let\jobname\childdocmain\||fi|\\
\end{tabular}
\end{center}
%
Instead of |\childdocof{|\textit{main}|}| just include the main file
at the top of each child file:
%
\begin{center}
|\input{|\textit{main}|}|
\end{center}
%
A simple redirection |\childdocforward{|\textit{dest}|}| is achieved by:
%
\begin{center}
|\def\jobname{|\textit{dest}|}\input{\jobname}|
\end{center}
%
The redirection with prefix
|\childdocforwardprefix[|\textit{prefix}|]{|\textit{dest}|}|
is accomplished by:
%
\begin{center}
\begin{tabular}{l}
|{\edef\jobname{\scantokens\expandafter{\jobname\noexpand}}|\\
|\def\redirectjob |\textit{prefix}|#1~~~{\gdef\jobname{|\textit{dest}|#1}}|\\
|\expandafter\redirectjob\jobname~~~}\input{\jobname}|
\end{tabular}
\end{center}

In an alternative approach,
child documents can be compiled by a specific command line
without additional code or specific definitions:
%
\begin{center}
|... -jobname "|\textit{target}|" "|[\textit{flags}]%
|\includeonly{|\textit{dest}|}\input{|\textit{main}|}"|
\end{center}
%

%%%%%%%%%%%%%%%%%%%%%%%%%%%%%%%%%%%%%%%%%%%%%%%%%%%%%%%%%%%%%%%%%%%%%%%%%%%%%%%%
%%%%%%%%%%%%%%%%%%%%%%%%%%%%%%%%%%%%%%%%%%%%%%%%%%%%%%%%%%%%%%%%%%%%%%%%%%%%%%%%
\section{Information}

%%%%%%%%%%%%%%%%%%%%%%%%%%%%%%%%%%%%%%%%%%%%%%%%%%%%%%%%%%%%%%%%%%%%%%%%%%%%%%%%
\subsection{Copyright}

Copyright \copyright{} 2017--2018 Niklas Beisert

This work may be distributed and/or modified under the
conditions of the \LaTeX{} Project Public License, either version 1.3
of this license or (at your option) any later version.
The latest version of this license is in
  \url{http://www.latex-project.org/lppl.txt}
and version 1.3 or later is part of all distributions of \LaTeX{}
version 2005/12/01 or later.

This work has the LPPL maintenance status `maintained'.

The Current Maintainer of this work is Niklas Beisert.

This work consists of the files |README.txt|, |childdoc.ins| and |childdoc.dtx|
as well as the derived files |childdoc.def|, |cdocsamp.tex|
with |cdocsch1.tex|, |cdocsch2.tex|, |cdocspt3.tex|, |cdocspt4.tex|,
|cdocsdrf.tex|, |cdocsfn1.tex|, |cdocsfn2.tex|
as well as |childdoc.pdf|.

%%%%%%%%%%%%%%%%%%%%%%%%%%%%%%%%%%%%%%%%%%%%%%%%%%%%%%%%%%%%%%%%%%%%%%%%%%%%%%%%
\subsection{Files and Installation}

The package consists of the files:
%
\begin{center}
\begin{tabular}{ll}
    |README.txt|   & readme file \\
    |childdoc.ins| & installation file \\
    |childdoc.dtx| & source file \\
    |childdoc.def| & definition file \\
    |cdocsamp.tex| & sample main file \\
    |cdocsch1.tex| & sample include file \\
    |cdocsch2.tex| & sample include file \\
    |cdocspt3.tex| & sample part file \\
    |cdocspt4.tex| & sample part file \\
    |cdocsdrf.tex| & sample redirection file \\
    |cdocsfn1.tex| & sample redirection file \\
    |cdocsfn2.tex| & sample redirection file \\
    |childdoc.pdf| & manual
\end{tabular}
\end{center}
%
The distribution consists of the files
|README.txt|, |childdoc.ins| and |childdoc.dtx|.
%
\begin{itemize}
\item
Run (pdf)\LaTeX{} on |childdoc.dtx|
to compile the manual |childdoc.pdf| (this file).
\item
Run \LaTeX{} on |childdoc.ins| to create the definitions file |childdoc.def|
and the sample |cdocsamp.tex| with include files
|cdocsch1.tex|, |cdocsch2.tex|, |cdocspt3.tex|, |cdocspt4.tex|,
|cdocsdrf.tex|, |cdocsfn1.tex|, |cdocsfn2.tex|.
Then copy the file |childdoc.def| to an appropriate directory of your \LaTeX{}
distribution, e.g.\ \textit{texmf-root}|/tex/latex/childdoc|.
\end{itemize}

%%%%%%%%%%%%%%%%%%%%%%%%%%%%%%%%%%%%%%%%%%%%%%%%%%%%%%%%%%%%%%%%%%%%%%%%%%%%%%%%
\subsection{Related CTAN Packages}

There are several other packages which offer a similar functionality:
%
\begin{itemize}
\item
The packages
\href{http://ctan.org/pkg/docmute}{\textsf{docmute}},
\href{http://ctan.org/pkg/includex}{\textsf{includex}} and
\href{http://ctan.org/pkg/standalone}{\textsf{standalone}}
provide commands to include only the document body of
a child file thus allowing both files to be compiled individually.
\item
The packages \href{http://ctan.org/pkg/subdocs}{\textsf{subdocs}}
and \href{http://ctan.org/pkg/subfiles}{\textsf{subfiles}}
provide structures in which the main and child documents can be
encapsulated and allowing them to be compiled individually.
The inclusion mechanism is different from the conventional |\include|.
\item
The package \href{http://ctan.org/pkg/combine}{\textsf{combine}}
is an elaborate solution to combine several documents into one.
\end{itemize}
%
See also the CTAN topic \href{http://ctan.org/topic/subdocs}{\textsf{subdocs}}
for further related packages.
The present package differs from the above solutions in that
a document structure constructed with the conventional |\include| mechanism
just needs two extra commands at the top of every file
such that all constituent files can be compiled individually.

%%%%%%%%%%%%%%%%%%%%%%%%%%%%%%%%%%%%%%%%%%%%%%%%%%%%%%%%%%%%%%%%%%%%%%%%%%%%%%%%
%\subsection{Feature Suggestions}
%
%The following is a list of features which may be useful for future
%versions of this package:
%%
%\begin{itemize}
%\item
%\ldots
%\end{itemize}

%%%%%%%%%%%%%%%%%%%%%%%%%%%%%%%%%%%%%%%%%%%%%%%%%%%%%%%%%%%%%%%%%%%%%%%%%%%%%%%%
\subsection{Revision History}

%%%%%%%%%%%%%%%%%%%%%%%%%%%%%%%%%%%%%%%%
\paragraph{v2.0:} 2018/12/30

\begin{itemize}
\item
immediate forward processing
\item
added |\childdocby| mechanism
\item
manual restructured
\end{itemize}

%%%%%%%%%%%%%%%%%%%%%%%%%%%%%%%%%%%%%%%%
\paragraph{v1.6:} 2018/01/17

\begin{itemize}
\item
application for development of include files
\item
corrections to manual
\end{itemize}

%%%%%%%%%%%%%%%%%%%%%%%%%%%%%%%%%%%%%%%%
\paragraph{v1.5:} 2017/05/21

\begin{itemize}
\item
more complete structuring introduced
\item
|\childdocof| introduced
\item
|\childdoc| renamed to |\childdocmain|
\item
|\childredirect| renamed to |\childdocforward| and |\childdocforwardprefix|
and functionality expanded
\end{itemize}

%%%%%%%%%%%%%%%%%%%%%%%%%%%%%%%%%%%%%%%%
\paragraph{v1.0:} 2017/04/27

\begin{itemize}
\item
manual and install package
\item
first version published on CTAN
\end{itemize}

%%%%%%%%%%%%%%%%%%%%%%%%%%%%%%%%%%%%%%%%
\paragraph{v0.6:} 2017/04/26

\begin{itemize}
\item
redirection mechanism added
\end{itemize}

%%%%%%%%%%%%%%%%%%%%%%%%%%%%%%%%%%%%%%%%
\paragraph{v0.5:} 2017/04/26

\begin{itemize}
\item
functionality in definition file
\end{itemize}


%%%%%%%%%%%%%%%%%%%%%%%%%%%%%%%%%%%%%%%%%%%%%%%%%%%%%%%%%%%%%%%%%%%%%%%%%%%%%%%%
%%%%%%%%%%%%%%%%%%%%%%%%%%%%%%%%%%%%%%%%%%%%%%%%%%%%%%%%%%%%%%%%%%%%%%%%%%%%%%%%
%%%%%%%%%%%%%%%%%%%%%%%%%%%%%%%%%%%%%%%%%%%%%%%%%%%%%%%%%%%%%%%%%%%%%%%%%%%%%%%%
\appendix

\settowidth\MacroIndent{\rmfamily\scriptsize 000\ }

 \DocInput{childdoc.dtx}

\end{document}
%</driver>
% \fi
%
% %%%%%%%%%%%%%%%%%%%%%%%%%%%%%%%%%%%%%%%%%%%%%%%%%%%%%%%%%%%%%%%%%%%%%%%%%%%%%%
% %%%%%%%%%%%%%%%%%%%%%%%%%%%%%%%%%%%%%%%%%%%%%%%%%%%%%%%%%%%%%%%%%%%%%%%%%%%%%%
% \section{Sample}
%\iffalse
%<*samplemain>
%\fi
%
% The following presents a sample document
% with two chapters, two parts, a title page,
% a compile flag as well as three forwarding files to set the flag.
% It consists of eight |.tex| files:
% \begin{center}
% \begin{tabular}{ll}
% |cdocsamp.tex|&main file\\
% |cdocsch1.tex|&include file for chapter 1\\
% |cdocsch2.tex|&include file for chapter 2\\
% |cdocspt3.tex|&include file for part 3\\
% |cdocspt4.tex|&include file for part 4\\
% |cdocsdrf.tex|&forwarding file for main file in draft mode\\
% |cdocsfi1.tex|&forwarding file for final version of chapter 1\\
% |cdocsfi2.tex|&forwarding file for final version of chapter 2\\
% \end{tabular}
% \end{center}
% Each of the eight files can be compiled directly by the \LaTeX{} compiler.
%
% %%%%%%%%%%%%%%%%%%%%%%%%%%%%%%%%%%%%%%
% \paragraph{Main File.}
%
% The main file is called |cdocsamp.tex|.
%
% Load the \textsf{childdoc} definitions and
% declare the filename for the main document:
%    \begin{macrocode}
\input{childdoc.def}
\childdocmain{}
%    \end{macrocode}

% Optional override for |\version| flag:
%    \begin{macrocode}
%%\ifchilddoc\else\providecommand{\version}{draft}\fi
%    \end{macrocode}

% Define the default values for the |\version| flag
% (|final| for the main file and |draft| for childs):
%    \begin{macrocode}
\ifchilddoc
\providecommand{\version}{draft}
\else
\providecommand{\version}{final}
\fi
%    \end{macrocode}

% Load the standard document class:
%    \begin{macrocode}
\documentclass[12pt]{article}
%    \end{macrocode}

% Start the document body:
%    \begin{macrocode}
\begin{document}
%    \end{macrocode}

% Declare a title page.
% Print title, part of document being processed and version flag:
%    \begin{macrocode}
\addtocounter{page}{-1}
\begin{center}
{\LARGE\bfseries{}childdoc example\par}
\vspace{1cm}
\ifchilddoc
\ifchilddocmanual part\else chapter\fi:
`\childdocname' of `\childdocjob'\par
\else
main document: `\childdocjob'\par
\fi
version: \version\par
\end{center}
\newpage
%    \end{macrocode}

% Manually include selected file,
% otherwise process as usual:
%    \begin{macrocode}
\ifchilddocmanual
\section*{part `\childdocname'}
\input{\childdocname}
\else
%    \end{macrocode}

% Include the two chapters:
%    \begin{macrocode}
\include{cdocsch1}
\include{cdocsch2}
%    \end{macrocode}

% Include the two parts unless only chapters should be displayed:
%    \begin{macrocode}
\ifchilddoc\else
\section{part three}
\input{cdocspt3}
\section{part four}
\input{cdocspt4}
\fi
%    \end{macrocode}

% Process as usual until here:
%    \begin{macrocode}
\fi
%    \end{macrocode}

% End of document body:
%    \begin{macrocode}
\end{document}
%    \end{macrocode}
%\iffalse
%</samplemain>
%\fi
%
% %%%%%%%%%%%%%%%%%%%%%%%%%%%%%%%%%%%%%%
% \paragraph{Chapter Include Files.}
%
% The include files are called |cdocsch1.tex| and |cdocsch2.tex|.
%
%\iffalse
%<*samplechap1|samplechap2>
%\fi

% Optional override for |\version| flag:
%    \begin{macrocode}
%%\providecommand{\version}{final}
%    \end{macrocode}

% Include the main document:
%    \begin{macrocode}
\input{childdoc.def}
\childdocof{cdocsamp}
%    \end{macrocode}

%\iffalse
%</samplechap1|samplechap2>
%\fi
%
%\iffalse
%<*samplechap1>
%\fi
% Some text for chapter 1:
%    \begin{macrocode}
\section{one}
some text in chapter one
%    \end{macrocode}

%\iffalse
%</samplechap1>
%\fi
% Some text for chapter 2:
%\iffalse
%<*samplechap2>
%\fi
%    \begin{macrocode}
\section{two}
more text in chapter two
%    \end{macrocode}

%\iffalse
%</samplechap2>
%\fi
%
% %%%%%%%%%%%%%%%%%%%%%%%%%%%%%%%%%%%%%%
% \paragraph{Part Include Files.}
%
% The include files are called |cdocspt3.tex| and |cdocspt4.tex|.
%
%\iffalse
%<*samplepart3|samplepart4>
%\fi

% Optional override for |\version| flag:
%    \begin{macrocode}
%%\providecommand{\version}{final}
%    \end{macrocode}

% Include the main document:
%    \begin{macrocode}
\input{childdoc.def}
\childdocby{cdocsamp}
%    \end{macrocode}

%\iffalse
%</samplepart3|samplepart4>
%\fi
%
%\iffalse
%<*samplepart3>
%\fi
% Some text for part 3:
%    \begin{macrocode}
some text in part three
%    \end{macrocode}

%\iffalse
%</samplepart3>
%\fi
% Some text for part 4:
%\iffalse
%<*samplepart4>
%\fi
%    \begin{macrocode}
more text in part four
%    \end{macrocode}

%\iffalse
%</samplepart4>
%\fi
%
% %%%%%%%%%%%%%%%%%%%%%%%%%%%%%%%%%%%%%%
% \paragraph{Forwarding for a Complete Draft.}
%
% The following forwarding file |cdocsdrf.tex|
% compiles the main document in draft mode:
%\iffalse
%<*sampledraft>
%\fi
%    \begin{macrocode}
\def\version{draft}
\input{childdoc.def}
\childdocforward{cdocsamp}
%    \end{macrocode}

%\iffalse
%</sampledraft>
%\fi
%
% %%%%%%%%%%%%%%%%%%%%%%%%%%%%%%%%%%%%%%
% \paragraph{Forwarding for Final Version of the Chapters.}
%
% The following forwarding files |cdocsfn1.tex| and |cdocsfn2.tex|
% (with identical content)
% compile the final versions of the child documents
% |cdocsch1.tex| and |cdocsch2.tex|, respectively:
%\iffalse
%<*samplefinal>
%\fi
%    \begin{macrocode}
\def\version{final}
\input{childdoc.def}
\childdocforwardprefix[cdocsamp]{cdocsfn}{cdocsch}
%    \end{macrocode}

%\iffalse
%</samplefinal>
%\fi
%
% %%%%%%%%%%%%%%%%%%%%%%%%%%%%%%%%%%%%%%
% \paragraph{Command Line Processing.}
%
% The following three command lines generate the output files
% |cdocscld|, |cdocscl1| and |cdocscl2|
% which should be identical to
% |cdocsdrf|, |cdocsch1| and |cdocsfn2|, respectively:
% \begin{center}
% \begin{tabular}{l}
% |latex -jobname cdocscld \|\\
% |  "\def\version{draft}\input{childdoc.def}\childdocforward{cdocsamp}"|\\
% |latex -jobname cdocscl1 \|\\
% |  "\input{childdoc.def}\childdocforward[cdocsamp]{cdocsch1}"|\\
% |latex -jobname cdocscl2 \|\\
% |  "\def\version{final}\input{childdoc.def}\childdocforward{cdocsch2}"|
% \end{tabular}
% \end{center}
% Note that the trailing backslash on each first line
% merely continues the input to the second line
% (for convenient cut ant paste).
% Furthermore, the command |latex| can be replaced by any
% of its alternative versions such as |pdflatex|.
%
% %%%%%%%%%%%%%%%%%%%%%%%%%%%%%%%%%%%%%%%%%%%%%%%%%%%%%%%%%%%%%%%%%%%%%%%%%%%%%%
% %%%%%%%%%%%%%%%%%%%%%%%%%%%%%%%%%%%%%%%%%%%%%%%%%%%%%%%%%%%%%%%%%%%%%%%%%%%%%%
% \section{Implementation}
%\iffalse
%<*package>
%\fi
%
% This section describes the definitions file |childdoc.def|.

% The definitions cannot be loaded using |\usepackage| or |\RequirePackage|
% which has a mechanism to prevent loading a style file more than once.
% When loading the definitions by means of |\input|
% multiple instances have to be prevented manually:
%\iffalse
%This code needs to be before the `\ProvidesFile' directive
%which is defined at the beginning of this file.
%Therefore it is also placed there and commented out here.
%</package>
%<*discard>
%\fi
%    \begin{macrocode}
\ifdefined\childdocmain\endinput\fi
%    \end{macrocode}
%\iffalse
%</discard>
%<*package>
%\fi
%
% \macro{\ifchilddoc}
% \macro{\ifchilddocmanual}
% The conditional |\ifchilddoc| tells whether a
% child (true) or main (false) document is being compiled.
% The conditional |\ifchilddocmanual| tells whether
% the |\includeonly| mechanism is used (false) or
% the selection of child files must be performed manually (true).
% The definitions initialise to false:
%    \begin{macrocode}
\newif\ifchilddoc
\newif\ifchilddocmanual
%    \end{macrocode}

% \macro{\childdocname}
% \macro{\childdocjob}
% The macro |\childdocname| stores the name of the main document
% to be compiled. The macro |\childdocjob| stores the name of
% the document on which the \LaTeX{} compiler was originally invoked.
% The content of |\jobname| cannot be compared
% to filenames specified in the source due to different catcodes.
% The following code rescans |\jobname|, stores the result
% in |\childdocname| and saves a copy in |\childdocjob|:
%    \begin{macrocode}
\edef\childdocname{\scantokens\expandafter{\jobname\noexpand}}
\let\childdocjob\childdocname
%    \end{macrocode}

% \macro{\childdocdisable}
% The macro |\childdocdisable| prevents the main file
% from being processed more than once.
% At this stage, the main document command |\childdocmain|
% is assumed to be called once again where it should do nothing.
% Any subsequent call to it should prevent
% a secondary processing of the main document
% It overwrites the forwarding commands
% |\childdocof| and |\childdocforward|
% with empty macros to prevent further inclusions of the main document:
%    \begin{macrocode}
\newcommand{\childdocdisable}
{
  \renewcommand{\childdocmain}[1]{\renewcommand{\childdocmain}[1]{\endinput}}
  \renewcommand{\childdocof}[1]{}
  \renewcommand{\childdocby}[2][]{}
  \renewcommand{\childdocforward}[2][]{}
  \renewcommand{\childdocdisable}{}
}
%    \end{macrocode}

% \macro{\childdocmain}
% The macro |\childdocmain| is to be called at the top of the main file
% with nothing or the main filename (without extension) as argument.
% First, it breaks loops.
% If the argument is not empty and does not match |\childdocname|
% (which is set by the first inclusion of |childdoc.def|),
% |\ifchilddoc| is set to true, |\includeonly| is applied to the child file
% and |\jobname| is set to the main file
% (for proper handling of |.aux| files):
%    \begin{macrocode}
\newcommand{\childdocmain}[1]
{
  \childdocdisable\childdocmain{}
  \if?#1?\else
    \begingroup
      \def\childdoctmp{#1}
      \ifx\childdoctmp\childdocname
        \def\childdoctmp{}
      \else
        \def\childdoctmp
        {
          \childdoctrue
          \includeonly{\childdocname}
          \def\childdocjob{#1}
          \def\jobname{#1}
        }
      \fi
      \expandafter
    \endgroup
    \childdoctmp
  \fi
}
%    \end{macrocode}

% \macro{\childdocof}
% The command |\childdocof| redirects
% compilation to the main file |#1|.
%    \begin{macrocode}
\newcommand{\childdocof}[1]
{
  \childdocdisable
  \childdoctrue
  \includeonly{\childdocname}
  \def\jobname{#1}
  \def\childdocjob{#1}
  \input{#1}
}
%    \end{macrocode}

% \macro{\childdocby}
% The command |\childdocby| ....
%    \begin{macrocode}
\newcommand{\childdocby}[2][]
{
  \childdocdisable
  \childdoctrue
  \childdocmanualtrue
  \if?#1?\else
    \def\jobname{#2}
  \fi
  \def\childdocjob{#2}
  \input{#2}
  \endinput
}
%    \end{macrocode}

% \macro{\childdocforward}
% The command |\childdocforward| redirects
% compilation to the main file or
% (if the optional argument is given) a child file.
% Parameters are set as if the main file
% or a child file starting with |\childdocof| was compiled.
% Then compilation is handed over to the main file:
%    \begin{macrocode}
\newcommand{\childdocforward}[2][]
{
  \begingroup
    \if?#1?
      \def\childdoctmp
      {
        \def\childdocname{#2}
        \def\childdocjob{#2}
        \def\jobname{#2}
        \input{#2}
        \endinput
      }
    \else
      \def\childdoctmp
      {
        \childdocdisable
        \def\childdocname{#2}
        \childdoctrue
        \includeonly{#2}
        \def\childdocjob{#1}
        \def\jobname{#1}
        \input{#1}
        \endinput
      }
    \fi
    \expandafter
  \endgroup
  \childdoctmp
}
%    \end{macrocode}

% \macro{\childdocforwardprefix}
% The command |\childdocforwardprefix| redirects
% compilation to the main or a child file by means of a pattern.
% The prefix |#1| in the current filename is replaced by |#2|
% and the suffix of the current filename is kept
% (it is assumed that the filename does not contain the substring `|~~~|'
% which is used as a delimiter).
% Compilation is handed over to the new file by |\childdocforward|:
%    \begin{macrocode}
\newcommand{\childdocforwardprefix}[3][]
{
  \begingroup
    \def\childdocextract #2##1~~~{\def\childdoctmp{\childdocforward[#1]{#3##1}}}
    \expandafter\childdocextract\childdocname~~~
    \expandafter
  \endgroup
  \childdoctmp
}
%    \end{macrocode}

% \macro{\childdoc}
% The deprecated macro |\childdoc| is a legacy version of |\childdocmain|:
%    \begin{macrocode}
\newcommand{\childdoc}{\childdocmain}
%    \end{macrocode}

% \macro{\childdocredirect}
% The deprecated macro |\childdocredirect| is a legacy version
% of |\childdocforward| and |\childdocforwardprefix|:
%    \begin{macrocode}
\newcommand{\childdocredirect}[2][]
{
  \begingroup
    \if?#1?
      \def\childdoctmp{\childdocforward{#2}}
    \else
      \def\childdoctmp{\childdocforwardprefix{#1}{#2}}
    \fi
    \expandafter
  \endgroup
  \childdoctmp
}
%    \end{macrocode}

%\iffalse
%</package>
%\fi
%
\endinput

\childdocby{cdocsamp}
%    \end{macrocode}

%\iffalse
%</samplepart3|samplepart4>
%\fi
%
%\iffalse
%<*samplepart3>
%\fi
% Some text for part 3:
%    \begin{macrocode}
some text in part three
%    \end{macrocode}

%\iffalse
%</samplepart3>
%\fi
% Some text for part 4:
%\iffalse
%<*samplepart4>
%\fi
%    \begin{macrocode}
more text in part four
%    \end{macrocode}

%\iffalse
%</samplepart4>
%\fi
%
% %%%%%%%%%%%%%%%%%%%%%%%%%%%%%%%%%%%%%%
% \paragraph{Forwarding for a Complete Draft.}
%
% The following forwarding file |cdocsdrf.tex|
% compiles the main document in draft mode:
%\iffalse
%<*sampledraft>
%\fi
%    \begin{macrocode}
\def\version{draft}
% \iffalse
%
% childdoc.dtx Copyright (C) 2017-2018 Niklas Beisert
%
% This work may be distributed and/or modified under the
% conditions of the LaTeX Project Public License, either version 1.3
% of this license or (at your option) any later version.
% The latest version of this license is in
%   http://www.latex-project.org/lppl.txt
% and version 1.3 or later is part of all distributions of LaTeX
% version 2005/12/01 or later.
%
% This work has the LPPL maintenance status `maintained'.
%
% The Current Maintainer of this work is Niklas Beisert.
%
% This work consists of the files childdoc.dtx and childdoc.ins
% and the derived files childdoc.def and cdocsamp.tex with
% cdocsch1.tex, cdocsch2.tex, cdocsdrf.tex, cdocsfn1.tex, cdocsfn2.tex.
%
%<package>\ifdefined\childdocmain\endinput\fi
%<package>\ProvidesFile{childdoc.def}[2018/12/30 v2.0 child document driver]
%<samplemain>\ProvidesFile{cdocsamp.tex}[2018/12/30 v2.0 sample for childdoc]
%<*driver>
%\ProvidesFile{childdoc.drv}[2018/12/30 v2.0 childdoc reference manual file]
\PassOptionsToClass{10pt,a4paper}{article}
\documentclass{ltxdoc}

\usepackage[margin=35mm]{geometry}
\usepackage{hyperref}
\usepackage{hyperxmp}
\usepackage[usenames]{color}

\hypersetup{colorlinks=true}
\hypersetup{pdfstartview=FitH}
\hypersetup{pdfpagemode=UseNone}
\hypersetup{pdfsource={}}
\hypersetup{pdflang={en-UK}}
\hypersetup{pdfcopyright={Copyright 2017-2018 Niklas Beisert.
  This work may be distributed and/or modified under the
  conditions of the LaTeX Project Public License, either version 1.3
  of this license or (at your option) any later version.}}
\hypersetup{pdflicenseurl={http://www.latex-project.org/lppl.txt}}
\hypersetup{pdfcontactaddress={ETH Zurich, ITP, HIT K,
  Wolfgang-Pauli-Strasse 27}}
\hypersetup{pdfcontactpostcode={8093}}
\hypersetup{pdfcontactcity={Zurich}}
\hypersetup{pdfcontactcountry={Switzerland}}
\hypersetup{pdfcontactemail={nbeisert@itp.phys.ethz.ch}}
\hypersetup{pdfcontacturl={http://people.phys.ethz.ch/\xmptilde nbeisert/}}

\newcommand{\secref}[1]{\hyperref[#1]{section \ref*{#1}}}

\parskip1ex
\parindent0pt
\let\olditemize\itemize
\def\itemize{\olditemize\parskip0pt}

\begin{document}

\title{The \textsf{childdoc} Package}
\hypersetup{pdftitle={The childdoc Package}}
\author{Niklas Beisert\\[2ex]
  Institut f\"ur Theoretische Physik\\
  Eidgen\"ossische Technische Hochschule Z\"urich\\
  Wolfgang-Pauli-Strasse 27, 8093 Z\"urich, Switzerland\\[1ex]
  \href{mailto:nbeisert@itp.phys.ethz.ch}
  {\texttt{nbeisert@itp.phys.ethz.ch}}}
\hypersetup{pdfauthor={Niklas Beisert}}
\hypersetup{pdfsubject={Manual for the LaTeX2e Package childdoc}}
\date{30 December 2018, \textsf{v2.0}}
\maketitle

\begin{abstract}\noindent
\textsf{childdoc} is a \LaTeXe{} package
that enables the direct compilation
of document sections included by |\include|
to individual files.
\end{abstract}

\begingroup
\parskip0ex
\tableofcontents
\endgroup

%%%%%%%%%%%%%%%%%%%%%%%%%%%%%%%%%%%%%%%%%%%%%%%%%%%%%%%%%%%%%%%%%%%%%%%%%%%%%%%%
%%%%%%%%%%%%%%%%%%%%%%%%%%%%%%%%%%%%%%%%%%%%%%%%%%%%%%%%%%%%%%%%%%%%%%%%%%%%%%%%
\section{Introduction}

\LaTeX{} provides a mechanism to structure a large document (such as a book)
into a main file and several child files (containing the chapters)
using the |\include| command.
This mechanism is beneficial for documents
which span hundreds of pages in order to
make the source file(s) more manageable.
Moreover, compilation can be restricted to
selected child files by means of the |\includeonly| command.
The latter feature can be used to reduce the compilation time while editing
(this was significantly more useful in the earlier days of \LaTeX{})
or to generate a smaller document which is easier to navigate.
Another application of |\includeonly| is to generate
documents consisting of selected parts of the complete document.

However, there are a few drawbacks of the plain |\include| mechanism:
\begin{itemize}
\item
The child files cannot be compiled on their own,
they can only be compiled via the main file.
A naive editing environment
(such as a text editor with an option
to have the current file processed by \LaTeX)
may require one to switch to the main file before compiling;
attempting to compile the child file produces errors.
\item
The main file must be modified (each time)
to adjust the |\includeonly| command
to the present needs. This easily leaves the main file in a messy state.
\item
The generated document will always carry the filename
of the main document. This is inconvenient if
several child files are to be compiled and
to be kept for distribution.
\end{itemize}

The present package provides a simple interface
to make child files individually compilable by \LaTeX{}.
Compiling a child file then has the same effect as compiling
the main file with an |\includeonly| command
to select the appropriate child.
Moreover the generated document will carry the name of the child
rather than the main file.
This resolves all three above issues.

This feature is meant to make the editing of books,
thesis documents and lecture notes somewhat more convenient.
However, the package can also be used efficiently for
composing a series of documents (such as exercise sheets)
which are typically distributed individually.
It then assists the author in generating the individual documents
(potentially in different versions)
as well as a document containing the collected series.
Another application is in developing style files
or other kinds of included material
where compilation of the style file could redirect
to a sample or test file.

%%%%%%%%%%%%%%%%%%%%%%%%%%%%%%%%%%%%%%%%%%%%%%%%%%%%%%%%%%%%%%%%%%%%%%%%%%%%%%%%
%%%%%%%%%%%%%%%%%%%%%%%%%%%%%%%%%%%%%%%%%%%%%%%%%%%%%%%%%%%%%%%%%%%%%%%%%%%%%%%%
\section{Usage}

First of all, the package \textsf{childdoc} is \emph{not} a standard
\LaTeXe{} |.sty| style file! Therefore it needs to be invoked in
a non-standard way.

%%%%%%%%%%%%%%%%%%%%%%%%%%%%%%%%%%%%%%%%%%%%%%%%%%%%%%%%%%%%%%%%%%%%%%%%%%%%%%%%
\subsection{Included Files}
\label{sec:include}

%%%%%%%%%%%%%%%%%%%%%%%%%%%%%%%%%%%%%%%%
\DescribeMacro{\childdocmain}
To use the package, add the commands
\begin{center}
\begin{tabular}{l}
|\input{childdoc.def}|\\
|\childdocmain{}|\\
\end{tabular}
\end{center}
at the very top of the main \LaTeX{} file,
in particular \emph{before} the |\documentclass| statement!
The argument of |\childdocmain| should be left empty
(but it must be present).

%%%%%%%%%%%%%%%%%%%%%%%%%%%%%%%%%%%%%%%%
\DescribeMacro{\childdocof}
Furthermore, add the commands
\begin{center}
\begin{tabular}{l}
|\input{childdoc.def}|\\
|\childdocof{|\textit{main}|}|\\
\end{tabular}
\end{center}
at the top of every child file \textit{child}
which is included by |\include{|\textit{child}|}|
from within the main file
(or at least for those files to be compiled individually).
The argument \textit{main} must be the filename of the main file.

There are a couple of
considerations in setting up the main and child documents:

%%%%%%%%%%%%%%%%%%%%%%%%%%%%%%%%%%%%%%%%
\paragraph{Restrictions.}

Please note the following restrictions:
\begin{itemize}
\item
|\childdocmain| must be called with one argument \textit{main}
to ensure compatibility with earlier version of the package.
It must either be empty (|\childdocmain{}|)
or precisely match the filename of the main file in which it is specified.
See \secref{sec:detection} for further information.
\item
The filename \textit{main} must be specified without the |.tex| extension.
\item
The filename \textit{main} is case sensitive
(even in case-insensitive file systems)
due to internal string comparison.
\item
The argument \textit{main} should be fully expanded, it cannot be a macro.
\item
Subdirectories and special characters should be avoided in filenames.
\item
The command |\childdocmain{|\textit{main}|}| must be followed by a whitespace.
It should not be followed immediately by another command
or by a comment mark `|%|'.
This is because the \TeX{} parser reads the token immediately following
the argument of |\childdocmain| and puts it
at the beginning of every child section;
however, a white\-space is ignored.
\end{itemize}

%%%%%%%%%%%%%%%%%%%%%%%%%%%%%%%%%%%%%%%%
\paragraph{Content of Main File.}

It is advisable to place all content in the child files included by |\include|.
Any output contained in the main file will appear in all child documents
unless suppressed manually;
it cannot be suppressed automatically by the |\includeonly| directive
and thus should normally be avoided.
A method to include some content in the main file
by means of conditional processing is described in \secref{sec:conditional}.

%%%%%%%%%%%%%%%%%%%%%%%%%%%%%%%%%%%%%%%%
\paragraph{Page Numbering.}

When only a part of the document is compiled,
the appropriate numbering of pages
(as well as other status parameters)
is determined from the |.aux| files.
The latter contain information from previous passes.
However this information needs to propagate through
all intermediate child documents.
Therefore the page numbering in child documents may well
be inconsistent until the complete document is compiled at least once.

A useful (if unconventional) way to always ensure a consistent
page numbering is to restart the numbering in each child document
and denote the pages by `\textit{child}|.|\textit{page}'
where \textit{child} represents the chapter/section number of the child file.
This can be achieved by the command
|\numberwithin{page}{|\textit{child}|}|
of the \textsf{amsmath} package
where \textit{child} can be |chapter| or |section|
depending on the chosen structuring.
Alternatively, one can modify the macro |\thepage| appropriately
and reset the counter |page| at the start of each child file.

%%%%%%%%%%%%%%%%%%%%%%%%%%%%%%%%%%%%%%%%%%%%%%%%%%%%%%%%%%%%%%%%%%%%%%%%%%%%%%%%
\subsection{Conditional Processing}
\label{sec:conditional}

The package provides a mechanism to compile different versions
of a document. To customise the versions further some conditional processing
can come in handy to distinguish which version is being compiled.
The package provides two macros to describe the compilation context:

%%%%%%%%%%%%%%%%%%%%%%%%%%%%%%%%%%%%%%%%
\DescribeMacro{\ifchilddoc}
The conditional |\ifchilddoc| distinguishes between the compilation of
child documents and the main document:
%
\begin{center}
|\ifchilddoc |\textit{child-code}| |[|\||else |\textit{main-code}]| \||fi|
\end{center}

%%%%%%%%%%%%%%%%%%%%%%%%%%%%%%%%%%%%%%%%
\DescribeMacro{\childdocname}
\DescribeMacro{\childdocjob}
The macro |\childdocname| contains the filename (without extension)
of the main or child file being processed.
Note that |\childdocjob| will always contain the name of the main file.

%%%%%%%%%%%%%%%%%%%%%%%%%%%%%%%%%%%%%%%%
\paragraph{Title Page.}

Conditional processing can be used to include a title or banner page
in the main document when proper precautions are taken.
Importantly, the code in the main file should ensure that the page counter
(as well as other status parameters which are stored in the |.aux| files)
takes the same value after the conditional processing.
Otherwise the page numbers may take divergent values
depending on which part is compiled.

For example, a title page could be declared by:
%
\begin{center}
\begin{tabular}{l}
|\ifchilddoc\||else|\\
|\addtocounter{page}{-1}|\\
\textit{code for title page}\\
|\newpage|\\
|\||fi|
\end{tabular}
\end{center}
%
A banner page for the child documents can be generated by:
%
\begin{center}
\begin{tabular}{l}
|\ifchilddoc|\\
|\addtocounter{page}{-1}|\\
\textit{code for banner page}\\
|\newpage|\\
|\||fi|
\end{tabular}
\end{center}
%
Here one could write a message such as:
\begin{center}
|This is the part \childdocname{} of \childdocjob{}.|
\end{center}

%%%%%%%%%%%%%%%%%%%%%%%%%%%%%%%%%%%%%%%%%%%%%%%%%%%%%%%%%%%%%%%%%%%%%%%%%%%%%%%%
\subsection{Flags}
\label{sec:flags}

The package makes it easy to generate different versions
of the main or child documents.
To this end compilation flags can be defined
and assigned different default values.
They will be particularly useful in conjunction
with the forwarding mechanism described in \secref{sec:forward}.

For example, it may be useful to have a flag |\version|
which can be set to |draft| or |final|.
The document source will contain some conditional code
depending on the value of |\version|.
Suppose further, the flag should default to |final| for the main file
and to |draft| for child files
which is a natural assignment for editing the document.
This is achieved by placing the following code
in the preamble of the main document
(below the |\childdocmain| directive):
%
\begin{center}
\begin{tabular}{l}
|\ifchilddoc|\\
|\providecommand{\version}{draft}|\\
|\||else|\\
|\providecommand{\version}{final}|\\
|\||fi|
\end{tabular}
\end{center}
%
The definition by |\providecommand| makes sure
that previous definitions are not overwritten.
Further statements |\providecommand{\version}{...}|
can thus be added before the above code to override it.

For the main file, one might add a line
(between |\childdocmain| and the above block)
%
\begin{center}
|%\ifchilddoc\||else\providecommand{\version}{draft}\||fi|
\end{center}
%
which can be uncommented to produce a draft version.
Likewise one can add a line to the very top of a child file
(above the |\childdocof{|\textit{main}|}| directive)
%
\begin{center}
|%\providecommand{\version}{final}|
\end{center}
%
which can be uncommented to produce the final version of this child document.

%%%%%%%%%%%%%%%%%%%%%%%%%%%%%%%%%%%%%%%%%%%%%%%%%%%%%%%%%%%%%%%%%%%%%%%%%%%%%%%%
\subsection{Forwarding}
\label{sec:forward}

Different versions of the main or child documents
using compilation flags as described in \secref{sec:flags}
can be (permanently) stored in different files
for convenient compilation, viewing and distribution.
To this end, the package defines a command
to pass on compilation to a different file:

%%%%%%%%%%%%%%%%%%%%%%%%%%%%%%%%%%%%%%%%
\DescribeMacro{\childdocforward}
The command |\childdocforward| redirects processing to
another source file:
%
\begin{center}
\begin{tabular}{l}
|\input{childdoc.def}|\\
|\childdocforward[|\textit{main}|]{|\textit{dest}|}|\\
\end{tabular}
\end{center}
%
The argument \textit{dest} is the destination file
(without extension).
It should be the main file or one of the child files.
Note that further \textsf{childdoc} directives
such as |\childdocof| and |\childdocforward|
in the indicated file will be processed in this form.
The optional argument \textit{main}
passes on directly to the main file \textit{main}
while pretending to compile the child \textit{dest}.
This form behaves as if \textit{dest}
issues |\childdocof{|\textit{main}|}| right away,
and no further \textsf{childdoc} directives will be processed.

%%%%%%%%%%%%%%%%%%%%%%%%%%%%%%%%%%%%%%%%
\DescribeMacro{\...prefix}
In the alternative form |\childdocforwardprefix|,
%
\begin{center}
\begin{tabular}{l}
|\input{childdoc.def}|\\
|\childdocforwardprefix[|\textit{main}|]{|\textit{prefix}|}{|\textit{dest}|}|
\end{tabular}
\end{center}
%
the destination file is determined by a pattern
depending on the current file:
To make this work, the current file must be called
`{\textit{prefix}\hspace{0.2em}\textit{suffix}}'
with \textit{prefix} matching precisely the argument.
Processing is then passed on to the file
`{\textit{dest}\hspace{0.2em}\textit{suffix}}'.
Surely, the same effect is achieved by
directly specifying the
argument `{\textit{dest}\hspace{0.2em}\textit{suffix}}'
in the first form.
However, that requires to set up a different file
for each child. With the alternative form of the command
all these files can have exactly the same content
which simplifies setting them up and maintaining them.

For example, the following file |draft.tex|
with a compilation flag |\version| as described in \secref{sec:flags}
compiles the main document as a draft:
%
\begin{center}
\begin{tabular}{l}
|\def\version{draft}|\\
|\input{childdoc.def}|\\
|\childdocforward{|\textit{main}|}|
\end{tabular}
\end{center}
%
Likewise, the following files |final|\textit{nn}|.tex|
compile the final version of the child document
|child|\textit{nn}|.tex|:
%
\begin{center}
\begin{tabular}{l}
|\def\version{final}|\\
|\input{childdoc.def}|\\
|\childdocforwardprefix{final}{child}|
\end{tabular}
\end{center}
%

Note that when several versions of a main file and/or of each child file
are to be generated, it may be convenient to set up a |Makefile| or
shell script to automatise the process.

%%%%%%%%%%%%%%%%%%%%%%%%%%%%%%%%%%%%%%%%%%%%%%%%%%%%%%%%%%%%%%%%%%%%%%%%%%%%%%%%
\subsection{Command Line Processing}
\label{sec:commandline}

The effect of redirection files can also be achieved by invoking
the \LaTeX{} compiler with a more elaborate command line.
Most conveniently this should be done as part
of a shell script or a |Makefile|.

When using \textsf{childdoc} in the main file, the following
command lines effectively perform a redirection
(note that depending on the shell being used,
backslashes may have to be doubled: `|\|' $\to$ `|\\|'):
%
\begin{center}
|... -jobname "|\textit{target}|" |\\|"|[\textit{flags}]%
|\input{childdoc.def}\childdocforward[|\textit{main}|]{|\textit{dest}|}"|
\end{center}
%
Here \textit{target} is the name of the output file,
\textit{main} is the name of the main file
and \textit{dest} is the name of the main or child file to be processed
(all filenames without extensions).
The optional argument \textit{main} can be omitted
if \textit{main} matches \textit{dest}.
Optionally, compilation \textit{flags} can be defined via |\def| commands.
This command line makes the \TeX{} engine believe
it is compiling the file \textit{target}
whose content is specified as the latter parameter.
The provided code then forwards the processing to
\textit{main} or \textit{dest} as described in \secref{sec:forward}.

%%%%%%%%%%%%%%%%%%%%%%%%%%%%%%%%%%%%%%%%%%%%%%%%%%%%%%%%%%%%%%%%%%%%%%%%%%%%%%%%
\subsection{Include by Input}
\label{sec:input}

Including child documents by |\include| has some restrictions by design.
Most notably, the content of a child document always occupies
its own set of pages; pages cannot be shared between child documents.
Usually, this behaviour makes perfect sense
because each child document contain an essential part of the document.
However, in some situations it may be desirable to compose
a document from a collection of parts
without having mandatory page breaks between then.
For this case, the package
provides a mechanism to include parts
by |\input| which can also be processed individually.
However, by construction this mechanism
requires manual handling of the content to be output.

%%%%%%%%%%%%%%%%%%%%%%%%%%%%%%%%%%%%%%%%
\DescribeMacro{\ifchilddocmanual}
The main file should be prepared as usual, see \secref{sec:include}.
However, the document body must make a distinction
between processing of an individual part and of the main document, e.g.:
%
\begin{center}
\begin{tabular}{l}
|\ifchilddocmanual|\\
|\input{\childdocname}|\\
|\||else|\\
\textit{document body with }|\input{|\textit{part}|}|\\
|\||fi|
\end{tabular}
\end{center}
%
The conditional |\ifchilddocmanual| is true whenever
a part to be included by |\input| is being compiled,
and the name of the part is stored in |\childdocname|.

%%%%%%%%%%%%%%%%%%%%%%%%%%%%%%%%%%%%%%%%
\DescribeMacro{\childdocby}
Each part to be included by |\input| should start with:
%
\begin{center}
\begin{tabular}{l}
|\input{childdoc.def}|\\
|\childdocby{|\textit{main}|}|\\
\end{tabular}
\end{center}
%
The directive |\childdocby| is similar to |\childdocof|
described in \secref{sec:include},
but the subsequent selection of content must be done manually.
To that end, both |\ifchilddoc| and |\ifchilddocmanual|
will be true upon processing of a part,
and the name of the part is stored in |\childdocname|.
Note that |\jobname| will be set to the filename of the current part
so that each part receives an individual |.aux| file
that does not interfere with the |.aux| file(s) of the main document.
This behaviour can be altered by the alternative form
|\childdocby[*]{|\textit{main}|}| (with a non-empty optional argument)
which uses the |.aux| file of the main document
by setting |\jobname| to \textit{main}.

%%%%%%%%%%%%%%%%%%%%%%%%%%%%%%%%%%%%%%%%%%%%%%%%%%%%%%%%%%%%%%%%%%%%%%%%%%%%%%%%
\subsection{Driver Development}
\label{sec:driver}

The \textsf{childdoc} mechanism can also be use for the development
of definition files such as \LaTeX{} styles or classes.
This case differs from the above setup with multiple parts
included by |\include| in that no |\includeonly| should be invoked.
This can be achieved by starting the include file
(before |\ProvidesPackage|) with:
%
\begin{center}
\begin{tabular}{l}
|\input{childdoc.def}|\\
|\childdocforward{|\textit{main}|}|\\
\end{tabular}
\end{center}
%
or alternatively with:
%
\begin{center}
\begin{tabular}{l}
|\input{childdoc.def}|\\
|\childdocby{|\textit{main}|}|\\
\end{tabular}
\end{center}
%
Both forms have slightly different effects as described above.
The main file is prepared as usual, see \secref{sec:include}.

%%%%%%%%%%%%%%%%%%%%%%%%%%%%%%%%%%%%%%%%%%%%%%%%%%%%%%%%%%%%%%%%%%%%%%%%%%%%%%%%
\subsection{Legacy Detection}
\label{sec:detection}

The directive |\childdocmain| in the main file can detect
whether the complete document or merely a child is to be compiled
even without using the directive |\childdocof|.
This method is deprecated because it is less robust
and there is no compelling reason to use it;
it is merely provided for backward compatibility
and it may be removed in future versions.

If the detection mechanism is to be used,
it is mandatory to correctly specify
the filename of the main file as the argument of |\childdocmain|:
%
\begin{center}
\begin{tabular}{l}
|\input{childdoc.def}|\\
|\childdocmain{|\textit{main}|}|\\
\end{tabular}
\end{center}
%
If |\jobname| does not match the argument \textit{main} of |\childdocmain|,
it is assumed that |\jobname| points to the child file to be compiled.
When using |\childdocmain| with the main file specified as argument,
it suffices to start a child file
with just |\input{|\textit{main}|}|
without loading of the package and using |\childdocof|.
If instead all processing is done
with the appropriate \textsf{childdoc} directives,
the argument of \textit{main} of |\childdocmain| can be empty.

An alternative version of the command line processing described
in \secref{sec:commandline} using the detection mechanism reads:
%
\begin{center}
|... -jobname "|\textit{target}|" "|[\textit{flags}]%
[|\def\jobname{|\textit{dest}|}|]|\input{|\textit{main}|}"|
\end{center}

%%%%%%%%%%%%%%%%%%%%%%%%%%%%%%%%%%%%%%%%%%%%%%%%%%%%%%%%%%%%%%%%%%%%%%%%%%%%%%%%
\subsection{Manual Code}
\label{sec:manual}

In case one cannot be certain whether the definitions file |childdoc.def|
is installed on the target \TeX{} distribution
and one prefers not to ship it,
it is conceivable to paste a few relevant commands into the sources.

To that end, drop all statements |\input{childdoc.def}|
and perform the replacements as outlined below.
Instead of |\childdocmain{|\textit{main}|}| add the following code
to the top of the main file:
%
\begin{center}
\begin{tabular}{l}
|\||ifdefined\childdocname\endinput\||fi\newif\ifchilddoc|\\
|\edef\childdocname{\scantokens\expandafter{\jobname\noexpand}}|\\
|\def\childdocmain{|\textit{main}|}\||ifx\childdocmain\childdocname\||else|\\
|\childdoctrue\includeonly{\childdocname}\let\jobname\childdocmain\||fi|\\
\end{tabular}
\end{center}
%
Instead of |\childdocof{|\textit{main}|}| just include the main file
at the top of each child file:
%
\begin{center}
|\input{|\textit{main}|}|
\end{center}
%
A simple redirection |\childdocforward{|\textit{dest}|}| is achieved by:
%
\begin{center}
|\def\jobname{|\textit{dest}|}\input{\jobname}|
\end{center}
%
The redirection with prefix
|\childdocforwardprefix[|\textit{prefix}|]{|\textit{dest}|}|
is accomplished by:
%
\begin{center}
\begin{tabular}{l}
|{\edef\jobname{\scantokens\expandafter{\jobname\noexpand}}|\\
|\def\redirectjob |\textit{prefix}|#1~~~{\gdef\jobname{|\textit{dest}|#1}}|\\
|\expandafter\redirectjob\jobname~~~}\input{\jobname}|
\end{tabular}
\end{center}

In an alternative approach,
child documents can be compiled by a specific command line
without additional code or specific definitions:
%
\begin{center}
|... -jobname "|\textit{target}|" "|[\textit{flags}]%
|\includeonly{|\textit{dest}|}\input{|\textit{main}|}"|
\end{center}
%

%%%%%%%%%%%%%%%%%%%%%%%%%%%%%%%%%%%%%%%%%%%%%%%%%%%%%%%%%%%%%%%%%%%%%%%%%%%%%%%%
%%%%%%%%%%%%%%%%%%%%%%%%%%%%%%%%%%%%%%%%%%%%%%%%%%%%%%%%%%%%%%%%%%%%%%%%%%%%%%%%
\section{Information}

%%%%%%%%%%%%%%%%%%%%%%%%%%%%%%%%%%%%%%%%%%%%%%%%%%%%%%%%%%%%%%%%%%%%%%%%%%%%%%%%
\subsection{Copyright}

Copyright \copyright{} 2017--2018 Niklas Beisert

This work may be distributed and/or modified under the
conditions of the \LaTeX{} Project Public License, either version 1.3
of this license or (at your option) any later version.
The latest version of this license is in
  \url{http://www.latex-project.org/lppl.txt}
and version 1.3 or later is part of all distributions of \LaTeX{}
version 2005/12/01 or later.

This work has the LPPL maintenance status `maintained'.

The Current Maintainer of this work is Niklas Beisert.

This work consists of the files |README.txt|, |childdoc.ins| and |childdoc.dtx|
as well as the derived files |childdoc.def|, |cdocsamp.tex|
with |cdocsch1.tex|, |cdocsch2.tex|, |cdocspt3.tex|, |cdocspt4.tex|,
|cdocsdrf.tex|, |cdocsfn1.tex|, |cdocsfn2.tex|
as well as |childdoc.pdf|.

%%%%%%%%%%%%%%%%%%%%%%%%%%%%%%%%%%%%%%%%%%%%%%%%%%%%%%%%%%%%%%%%%%%%%%%%%%%%%%%%
\subsection{Files and Installation}

The package consists of the files:
%
\begin{center}
\begin{tabular}{ll}
    |README.txt|   & readme file \\
    |childdoc.ins| & installation file \\
    |childdoc.dtx| & source file \\
    |childdoc.def| & definition file \\
    |cdocsamp.tex| & sample main file \\
    |cdocsch1.tex| & sample include file \\
    |cdocsch2.tex| & sample include file \\
    |cdocspt3.tex| & sample part file \\
    |cdocspt4.tex| & sample part file \\
    |cdocsdrf.tex| & sample redirection file \\
    |cdocsfn1.tex| & sample redirection file \\
    |cdocsfn2.tex| & sample redirection file \\
    |childdoc.pdf| & manual
\end{tabular}
\end{center}
%
The distribution consists of the files
|README.txt|, |childdoc.ins| and |childdoc.dtx|.
%
\begin{itemize}
\item
Run (pdf)\LaTeX{} on |childdoc.dtx|
to compile the manual |childdoc.pdf| (this file).
\item
Run \LaTeX{} on |childdoc.ins| to create the definitions file |childdoc.def|
and the sample |cdocsamp.tex| with include files
|cdocsch1.tex|, |cdocsch2.tex|, |cdocspt3.tex|, |cdocspt4.tex|,
|cdocsdrf.tex|, |cdocsfn1.tex|, |cdocsfn2.tex|.
Then copy the file |childdoc.def| to an appropriate directory of your \LaTeX{}
distribution, e.g.\ \textit{texmf-root}|/tex/latex/childdoc|.
\end{itemize}

%%%%%%%%%%%%%%%%%%%%%%%%%%%%%%%%%%%%%%%%%%%%%%%%%%%%%%%%%%%%%%%%%%%%%%%%%%%%%%%%
\subsection{Related CTAN Packages}

There are several other packages which offer a similar functionality:
%
\begin{itemize}
\item
The packages
\href{http://ctan.org/pkg/docmute}{\textsf{docmute}},
\href{http://ctan.org/pkg/includex}{\textsf{includex}} and
\href{http://ctan.org/pkg/standalone}{\textsf{standalone}}
provide commands to include only the document body of
a child file thus allowing both files to be compiled individually.
\item
The packages \href{http://ctan.org/pkg/subdocs}{\textsf{subdocs}}
and \href{http://ctan.org/pkg/subfiles}{\textsf{subfiles}}
provide structures in which the main and child documents can be
encapsulated and allowing them to be compiled individually.
The inclusion mechanism is different from the conventional |\include|.
\item
The package \href{http://ctan.org/pkg/combine}{\textsf{combine}}
is an elaborate solution to combine several documents into one.
\end{itemize}
%
See also the CTAN topic \href{http://ctan.org/topic/subdocs}{\textsf{subdocs}}
for further related packages.
The present package differs from the above solutions in that
a document structure constructed with the conventional |\include| mechanism
just needs two extra commands at the top of every file
such that all constituent files can be compiled individually.

%%%%%%%%%%%%%%%%%%%%%%%%%%%%%%%%%%%%%%%%%%%%%%%%%%%%%%%%%%%%%%%%%%%%%%%%%%%%%%%%
%\subsection{Feature Suggestions}
%
%The following is a list of features which may be useful for future
%versions of this package:
%%
%\begin{itemize}
%\item
%\ldots
%\end{itemize}

%%%%%%%%%%%%%%%%%%%%%%%%%%%%%%%%%%%%%%%%%%%%%%%%%%%%%%%%%%%%%%%%%%%%%%%%%%%%%%%%
\subsection{Revision History}

%%%%%%%%%%%%%%%%%%%%%%%%%%%%%%%%%%%%%%%%
\paragraph{v2.0:} 2018/12/30

\begin{itemize}
\item
immediate forward processing
\item
added |\childdocby| mechanism
\item
manual restructured
\end{itemize}

%%%%%%%%%%%%%%%%%%%%%%%%%%%%%%%%%%%%%%%%
\paragraph{v1.6:} 2018/01/17

\begin{itemize}
\item
application for development of include files
\item
corrections to manual
\end{itemize}

%%%%%%%%%%%%%%%%%%%%%%%%%%%%%%%%%%%%%%%%
\paragraph{v1.5:} 2017/05/21

\begin{itemize}
\item
more complete structuring introduced
\item
|\childdocof| introduced
\item
|\childdoc| renamed to |\childdocmain|
\item
|\childredirect| renamed to |\childdocforward| and |\childdocforwardprefix|
and functionality expanded
\end{itemize}

%%%%%%%%%%%%%%%%%%%%%%%%%%%%%%%%%%%%%%%%
\paragraph{v1.0:} 2017/04/27

\begin{itemize}
\item
manual and install package
\item
first version published on CTAN
\end{itemize}

%%%%%%%%%%%%%%%%%%%%%%%%%%%%%%%%%%%%%%%%
\paragraph{v0.6:} 2017/04/26

\begin{itemize}
\item
redirection mechanism added
\end{itemize}

%%%%%%%%%%%%%%%%%%%%%%%%%%%%%%%%%%%%%%%%
\paragraph{v0.5:} 2017/04/26

\begin{itemize}
\item
functionality in definition file
\end{itemize}


%%%%%%%%%%%%%%%%%%%%%%%%%%%%%%%%%%%%%%%%%%%%%%%%%%%%%%%%%%%%%%%%%%%%%%%%%%%%%%%%
%%%%%%%%%%%%%%%%%%%%%%%%%%%%%%%%%%%%%%%%%%%%%%%%%%%%%%%%%%%%%%%%%%%%%%%%%%%%%%%%
%%%%%%%%%%%%%%%%%%%%%%%%%%%%%%%%%%%%%%%%%%%%%%%%%%%%%%%%%%%%%%%%%%%%%%%%%%%%%%%%
\appendix

\settowidth\MacroIndent{\rmfamily\scriptsize 000\ }

 \DocInput{childdoc.dtx}

\end{document}
%</driver>
% \fi
%
% %%%%%%%%%%%%%%%%%%%%%%%%%%%%%%%%%%%%%%%%%%%%%%%%%%%%%%%%%%%%%%%%%%%%%%%%%%%%%%
% %%%%%%%%%%%%%%%%%%%%%%%%%%%%%%%%%%%%%%%%%%%%%%%%%%%%%%%%%%%%%%%%%%%%%%%%%%%%%%
% \section{Sample}
%\iffalse
%<*samplemain>
%\fi
%
% The following presents a sample document
% with two chapters, two parts, a title page,
% a compile flag as well as three forwarding files to set the flag.
% It consists of eight |.tex| files:
% \begin{center}
% \begin{tabular}{ll}
% |cdocsamp.tex|&main file\\
% |cdocsch1.tex|&include file for chapter 1\\
% |cdocsch2.tex|&include file for chapter 2\\
% |cdocspt3.tex|&include file for part 3\\
% |cdocspt4.tex|&include file for part 4\\
% |cdocsdrf.tex|&forwarding file for main file in draft mode\\
% |cdocsfi1.tex|&forwarding file for final version of chapter 1\\
% |cdocsfi2.tex|&forwarding file for final version of chapter 2\\
% \end{tabular}
% \end{center}
% Each of the eight files can be compiled directly by the \LaTeX{} compiler.
%
% %%%%%%%%%%%%%%%%%%%%%%%%%%%%%%%%%%%%%%
% \paragraph{Main File.}
%
% The main file is called |cdocsamp.tex|.
%
% Load the \textsf{childdoc} definitions and
% declare the filename for the main document:
%    \begin{macrocode}
\input{childdoc.def}
\childdocmain{}
%    \end{macrocode}

% Optional override for |\version| flag:
%    \begin{macrocode}
%%\ifchilddoc\else\providecommand{\version}{draft}\fi
%    \end{macrocode}

% Define the default values for the |\version| flag
% (|final| for the main file and |draft| for childs):
%    \begin{macrocode}
\ifchilddoc
\providecommand{\version}{draft}
\else
\providecommand{\version}{final}
\fi
%    \end{macrocode}

% Load the standard document class:
%    \begin{macrocode}
\documentclass[12pt]{article}
%    \end{macrocode}

% Start the document body:
%    \begin{macrocode}
\begin{document}
%    \end{macrocode}

% Declare a title page.
% Print title, part of document being processed and version flag:
%    \begin{macrocode}
\addtocounter{page}{-1}
\begin{center}
{\LARGE\bfseries{}childdoc example\par}
\vspace{1cm}
\ifchilddoc
\ifchilddocmanual part\else chapter\fi:
`\childdocname' of `\childdocjob'\par
\else
main document: `\childdocjob'\par
\fi
version: \version\par
\end{center}
\newpage
%    \end{macrocode}

% Manually include selected file,
% otherwise process as usual:
%    \begin{macrocode}
\ifchilddocmanual
\section*{part `\childdocname'}
\input{\childdocname}
\else
%    \end{macrocode}

% Include the two chapters:
%    \begin{macrocode}
\include{cdocsch1}
\include{cdocsch2}
%    \end{macrocode}

% Include the two parts unless only chapters should be displayed:
%    \begin{macrocode}
\ifchilddoc\else
\section{part three}
\input{cdocspt3}
\section{part four}
\input{cdocspt4}
\fi
%    \end{macrocode}

% Process as usual until here:
%    \begin{macrocode}
\fi
%    \end{macrocode}

% End of document body:
%    \begin{macrocode}
\end{document}
%    \end{macrocode}
%\iffalse
%</samplemain>
%\fi
%
% %%%%%%%%%%%%%%%%%%%%%%%%%%%%%%%%%%%%%%
% \paragraph{Chapter Include Files.}
%
% The include files are called |cdocsch1.tex| and |cdocsch2.tex|.
%
%\iffalse
%<*samplechap1|samplechap2>
%\fi

% Optional override for |\version| flag:
%    \begin{macrocode}
%%\providecommand{\version}{final}
%    \end{macrocode}

% Include the main document:
%    \begin{macrocode}
\input{childdoc.def}
\childdocof{cdocsamp}
%    \end{macrocode}

%\iffalse
%</samplechap1|samplechap2>
%\fi
%
%\iffalse
%<*samplechap1>
%\fi
% Some text for chapter 1:
%    \begin{macrocode}
\section{one}
some text in chapter one
%    \end{macrocode}

%\iffalse
%</samplechap1>
%\fi
% Some text for chapter 2:
%\iffalse
%<*samplechap2>
%\fi
%    \begin{macrocode}
\section{two}
more text in chapter two
%    \end{macrocode}

%\iffalse
%</samplechap2>
%\fi
%
% %%%%%%%%%%%%%%%%%%%%%%%%%%%%%%%%%%%%%%
% \paragraph{Part Include Files.}
%
% The include files are called |cdocspt3.tex| and |cdocspt4.tex|.
%
%\iffalse
%<*samplepart3|samplepart4>
%\fi

% Optional override for |\version| flag:
%    \begin{macrocode}
%%\providecommand{\version}{final}
%    \end{macrocode}

% Include the main document:
%    \begin{macrocode}
\input{childdoc.def}
\childdocby{cdocsamp}
%    \end{macrocode}

%\iffalse
%</samplepart3|samplepart4>
%\fi
%
%\iffalse
%<*samplepart3>
%\fi
% Some text for part 3:
%    \begin{macrocode}
some text in part three
%    \end{macrocode}

%\iffalse
%</samplepart3>
%\fi
% Some text for part 4:
%\iffalse
%<*samplepart4>
%\fi
%    \begin{macrocode}
more text in part four
%    \end{macrocode}

%\iffalse
%</samplepart4>
%\fi
%
% %%%%%%%%%%%%%%%%%%%%%%%%%%%%%%%%%%%%%%
% \paragraph{Forwarding for a Complete Draft.}
%
% The following forwarding file |cdocsdrf.tex|
% compiles the main document in draft mode:
%\iffalse
%<*sampledraft>
%\fi
%    \begin{macrocode}
\def\version{draft}
\input{childdoc.def}
\childdocforward{cdocsamp}
%    \end{macrocode}

%\iffalse
%</sampledraft>
%\fi
%
% %%%%%%%%%%%%%%%%%%%%%%%%%%%%%%%%%%%%%%
% \paragraph{Forwarding for Final Version of the Chapters.}
%
% The following forwarding files |cdocsfn1.tex| and |cdocsfn2.tex|
% (with identical content)
% compile the final versions of the child documents
% |cdocsch1.tex| and |cdocsch2.tex|, respectively:
%\iffalse
%<*samplefinal>
%\fi
%    \begin{macrocode}
\def\version{final}
\input{childdoc.def}
\childdocforwardprefix[cdocsamp]{cdocsfn}{cdocsch}
%    \end{macrocode}

%\iffalse
%</samplefinal>
%\fi
%
% %%%%%%%%%%%%%%%%%%%%%%%%%%%%%%%%%%%%%%
% \paragraph{Command Line Processing.}
%
% The following three command lines generate the output files
% |cdocscld|, |cdocscl1| and |cdocscl2|
% which should be identical to
% |cdocsdrf|, |cdocsch1| and |cdocsfn2|, respectively:
% \begin{center}
% \begin{tabular}{l}
% |latex -jobname cdocscld \|\\
% |  "\def\version{draft}\input{childdoc.def}\childdocforward{cdocsamp}"|\\
% |latex -jobname cdocscl1 \|\\
% |  "\input{childdoc.def}\childdocforward[cdocsamp]{cdocsch1}"|\\
% |latex -jobname cdocscl2 \|\\
% |  "\def\version{final}\input{childdoc.def}\childdocforward{cdocsch2}"|
% \end{tabular}
% \end{center}
% Note that the trailing backslash on each first line
% merely continues the input to the second line
% (for convenient cut ant paste).
% Furthermore, the command |latex| can be replaced by any
% of its alternative versions such as |pdflatex|.
%
% %%%%%%%%%%%%%%%%%%%%%%%%%%%%%%%%%%%%%%%%%%%%%%%%%%%%%%%%%%%%%%%%%%%%%%%%%%%%%%
% %%%%%%%%%%%%%%%%%%%%%%%%%%%%%%%%%%%%%%%%%%%%%%%%%%%%%%%%%%%%%%%%%%%%%%%%%%%%%%
% \section{Implementation}
%\iffalse
%<*package>
%\fi
%
% This section describes the definitions file |childdoc.def|.

% The definitions cannot be loaded using |\usepackage| or |\RequirePackage|
% which has a mechanism to prevent loading a style file more than once.
% When loading the definitions by means of |\input|
% multiple instances have to be prevented manually:
%\iffalse
%This code needs to be before the `\ProvidesFile' directive
%which is defined at the beginning of this file.
%Therefore it is also placed there and commented out here.
%</package>
%<*discard>
%\fi
%    \begin{macrocode}
\ifdefined\childdocmain\endinput\fi
%    \end{macrocode}
%\iffalse
%</discard>
%<*package>
%\fi
%
% \macro{\ifchilddoc}
% \macro{\ifchilddocmanual}
% The conditional |\ifchilddoc| tells whether a
% child (true) or main (false) document is being compiled.
% The conditional |\ifchilddocmanual| tells whether
% the |\includeonly| mechanism is used (false) or
% the selection of child files must be performed manually (true).
% The definitions initialise to false:
%    \begin{macrocode}
\newif\ifchilddoc
\newif\ifchilddocmanual
%    \end{macrocode}

% \macro{\childdocname}
% \macro{\childdocjob}
% The macro |\childdocname| stores the name of the main document
% to be compiled. The macro |\childdocjob| stores the name of
% the document on which the \LaTeX{} compiler was originally invoked.
% The content of |\jobname| cannot be compared
% to filenames specified in the source due to different catcodes.
% The following code rescans |\jobname|, stores the result
% in |\childdocname| and saves a copy in |\childdocjob|:
%    \begin{macrocode}
\edef\childdocname{\scantokens\expandafter{\jobname\noexpand}}
\let\childdocjob\childdocname
%    \end{macrocode}

% \macro{\childdocdisable}
% The macro |\childdocdisable| prevents the main file
% from being processed more than once.
% At this stage, the main document command |\childdocmain|
% is assumed to be called once again where it should do nothing.
% Any subsequent call to it should prevent
% a secondary processing of the main document
% It overwrites the forwarding commands
% |\childdocof| and |\childdocforward|
% with empty macros to prevent further inclusions of the main document:
%    \begin{macrocode}
\newcommand{\childdocdisable}
{
  \renewcommand{\childdocmain}[1]{\renewcommand{\childdocmain}[1]{\endinput}}
  \renewcommand{\childdocof}[1]{}
  \renewcommand{\childdocby}[2][]{}
  \renewcommand{\childdocforward}[2][]{}
  \renewcommand{\childdocdisable}{}
}
%    \end{macrocode}

% \macro{\childdocmain}
% The macro |\childdocmain| is to be called at the top of the main file
% with nothing or the main filename (without extension) as argument.
% First, it breaks loops.
% If the argument is not empty and does not match |\childdocname|
% (which is set by the first inclusion of |childdoc.def|),
% |\ifchilddoc| is set to true, |\includeonly| is applied to the child file
% and |\jobname| is set to the main file
% (for proper handling of |.aux| files):
%    \begin{macrocode}
\newcommand{\childdocmain}[1]
{
  \childdocdisable\childdocmain{}
  \if?#1?\else
    \begingroup
      \def\childdoctmp{#1}
      \ifx\childdoctmp\childdocname
        \def\childdoctmp{}
      \else
        \def\childdoctmp
        {
          \childdoctrue
          \includeonly{\childdocname}
          \def\childdocjob{#1}
          \def\jobname{#1}
        }
      \fi
      \expandafter
    \endgroup
    \childdoctmp
  \fi
}
%    \end{macrocode}

% \macro{\childdocof}
% The command |\childdocof| redirects
% compilation to the main file |#1|.
%    \begin{macrocode}
\newcommand{\childdocof}[1]
{
  \childdocdisable
  \childdoctrue
  \includeonly{\childdocname}
  \def\jobname{#1}
  \def\childdocjob{#1}
  \input{#1}
}
%    \end{macrocode}

% \macro{\childdocby}
% The command |\childdocby| ....
%    \begin{macrocode}
\newcommand{\childdocby}[2][]
{
  \childdocdisable
  \childdoctrue
  \childdocmanualtrue
  \if?#1?\else
    \def\jobname{#2}
  \fi
  \def\childdocjob{#2}
  \input{#2}
  \endinput
}
%    \end{macrocode}

% \macro{\childdocforward}
% The command |\childdocforward| redirects
% compilation to the main file or
% (if the optional argument is given) a child file.
% Parameters are set as if the main file
% or a child file starting with |\childdocof| was compiled.
% Then compilation is handed over to the main file:
%    \begin{macrocode}
\newcommand{\childdocforward}[2][]
{
  \begingroup
    \if?#1?
      \def\childdoctmp
      {
        \def\childdocname{#2}
        \def\childdocjob{#2}
        \def\jobname{#2}
        \input{#2}
        \endinput
      }
    \else
      \def\childdoctmp
      {
        \childdocdisable
        \def\childdocname{#2}
        \childdoctrue
        \includeonly{#2}
        \def\childdocjob{#1}
        \def\jobname{#1}
        \input{#1}
        \endinput
      }
    \fi
    \expandafter
  \endgroup
  \childdoctmp
}
%    \end{macrocode}

% \macro{\childdocforwardprefix}
% The command |\childdocforwardprefix| redirects
% compilation to the main or a child file by means of a pattern.
% The prefix |#1| in the current filename is replaced by |#2|
% and the suffix of the current filename is kept
% (it is assumed that the filename does not contain the substring `|~~~|'
% which is used as a delimiter).
% Compilation is handed over to the new file by |\childdocforward|:
%    \begin{macrocode}
\newcommand{\childdocforwardprefix}[3][]
{
  \begingroup
    \def\childdocextract #2##1~~~{\def\childdoctmp{\childdocforward[#1]{#3##1}}}
    \expandafter\childdocextract\childdocname~~~
    \expandafter
  \endgroup
  \childdoctmp
}
%    \end{macrocode}

% \macro{\childdoc}
% The deprecated macro |\childdoc| is a legacy version of |\childdocmain|:
%    \begin{macrocode}
\newcommand{\childdoc}{\childdocmain}
%    \end{macrocode}

% \macro{\childdocredirect}
% The deprecated macro |\childdocredirect| is a legacy version
% of |\childdocforward| and |\childdocforwardprefix|:
%    \begin{macrocode}
\newcommand{\childdocredirect}[2][]
{
  \begingroup
    \if?#1?
      \def\childdoctmp{\childdocforward{#2}}
    \else
      \def\childdoctmp{\childdocforwardprefix{#1}{#2}}
    \fi
    \expandafter
  \endgroup
  \childdoctmp
}
%    \end{macrocode}

%\iffalse
%</package>
%\fi
%
\endinput

\childdocforward{cdocsamp}
%    \end{macrocode}

%\iffalse
%</sampledraft>
%\fi
%
% %%%%%%%%%%%%%%%%%%%%%%%%%%%%%%%%%%%%%%
% \paragraph{Forwarding for Final Version of the Chapters.}
%
% The following forwarding files |cdocsfn1.tex| and |cdocsfn2.tex|
% (with identical content)
% compile the final versions of the child documents
% |cdocsch1.tex| and |cdocsch2.tex|, respectively:
%\iffalse
%<*samplefinal>
%\fi
%    \begin{macrocode}
\def\version{final}
% \iffalse
%
% childdoc.dtx Copyright (C) 2017-2018 Niklas Beisert
%
% This work may be distributed and/or modified under the
% conditions of the LaTeX Project Public License, either version 1.3
% of this license or (at your option) any later version.
% The latest version of this license is in
%   http://www.latex-project.org/lppl.txt
% and version 1.3 or later is part of all distributions of LaTeX
% version 2005/12/01 or later.
%
% This work has the LPPL maintenance status `maintained'.
%
% The Current Maintainer of this work is Niklas Beisert.
%
% This work consists of the files childdoc.dtx and childdoc.ins
% and the derived files childdoc.def and cdocsamp.tex with
% cdocsch1.tex, cdocsch2.tex, cdocsdrf.tex, cdocsfn1.tex, cdocsfn2.tex.
%
%<package>\ifdefined\childdocmain\endinput\fi
%<package>\ProvidesFile{childdoc.def}[2018/12/30 v2.0 child document driver]
%<samplemain>\ProvidesFile{cdocsamp.tex}[2018/12/30 v2.0 sample for childdoc]
%<*driver>
%\ProvidesFile{childdoc.drv}[2018/12/30 v2.0 childdoc reference manual file]
\PassOptionsToClass{10pt,a4paper}{article}
\documentclass{ltxdoc}

\usepackage[margin=35mm]{geometry}
\usepackage{hyperref}
\usepackage{hyperxmp}
\usepackage[usenames]{color}

\hypersetup{colorlinks=true}
\hypersetup{pdfstartview=FitH}
\hypersetup{pdfpagemode=UseNone}
\hypersetup{pdfsource={}}
\hypersetup{pdflang={en-UK}}
\hypersetup{pdfcopyright={Copyright 2017-2018 Niklas Beisert.
  This work may be distributed and/or modified under the
  conditions of the LaTeX Project Public License, either version 1.3
  of this license or (at your option) any later version.}}
\hypersetup{pdflicenseurl={http://www.latex-project.org/lppl.txt}}
\hypersetup{pdfcontactaddress={ETH Zurich, ITP, HIT K,
  Wolfgang-Pauli-Strasse 27}}
\hypersetup{pdfcontactpostcode={8093}}
\hypersetup{pdfcontactcity={Zurich}}
\hypersetup{pdfcontactcountry={Switzerland}}
\hypersetup{pdfcontactemail={nbeisert@itp.phys.ethz.ch}}
\hypersetup{pdfcontacturl={http://people.phys.ethz.ch/\xmptilde nbeisert/}}

\newcommand{\secref}[1]{\hyperref[#1]{section \ref*{#1}}}

\parskip1ex
\parindent0pt
\let\olditemize\itemize
\def\itemize{\olditemize\parskip0pt}

\begin{document}

\title{The \textsf{childdoc} Package}
\hypersetup{pdftitle={The childdoc Package}}
\author{Niklas Beisert\\[2ex]
  Institut f\"ur Theoretische Physik\\
  Eidgen\"ossische Technische Hochschule Z\"urich\\
  Wolfgang-Pauli-Strasse 27, 8093 Z\"urich, Switzerland\\[1ex]
  \href{mailto:nbeisert@itp.phys.ethz.ch}
  {\texttt{nbeisert@itp.phys.ethz.ch}}}
\hypersetup{pdfauthor={Niklas Beisert}}
\hypersetup{pdfsubject={Manual for the LaTeX2e Package childdoc}}
\date{30 December 2018, \textsf{v2.0}}
\maketitle

\begin{abstract}\noindent
\textsf{childdoc} is a \LaTeXe{} package
that enables the direct compilation
of document sections included by |\include|
to individual files.
\end{abstract}

\begingroup
\parskip0ex
\tableofcontents
\endgroup

%%%%%%%%%%%%%%%%%%%%%%%%%%%%%%%%%%%%%%%%%%%%%%%%%%%%%%%%%%%%%%%%%%%%%%%%%%%%%%%%
%%%%%%%%%%%%%%%%%%%%%%%%%%%%%%%%%%%%%%%%%%%%%%%%%%%%%%%%%%%%%%%%%%%%%%%%%%%%%%%%
\section{Introduction}

\LaTeX{} provides a mechanism to structure a large document (such as a book)
into a main file and several child files (containing the chapters)
using the |\include| command.
This mechanism is beneficial for documents
which span hundreds of pages in order to
make the source file(s) more manageable.
Moreover, compilation can be restricted to
selected child files by means of the |\includeonly| command.
The latter feature can be used to reduce the compilation time while editing
(this was significantly more useful in the earlier days of \LaTeX{})
or to generate a smaller document which is easier to navigate.
Another application of |\includeonly| is to generate
documents consisting of selected parts of the complete document.

However, there are a few drawbacks of the plain |\include| mechanism:
\begin{itemize}
\item
The child files cannot be compiled on their own,
they can only be compiled via the main file.
A naive editing environment
(such as a text editor with an option
to have the current file processed by \LaTeX)
may require one to switch to the main file before compiling;
attempting to compile the child file produces errors.
\item
The main file must be modified (each time)
to adjust the |\includeonly| command
to the present needs. This easily leaves the main file in a messy state.
\item
The generated document will always carry the filename
of the main document. This is inconvenient if
several child files are to be compiled and
to be kept for distribution.
\end{itemize}

The present package provides a simple interface
to make child files individually compilable by \LaTeX{}.
Compiling a child file then has the same effect as compiling
the main file with an |\includeonly| command
to select the appropriate child.
Moreover the generated document will carry the name of the child
rather than the main file.
This resolves all three above issues.

This feature is meant to make the editing of books,
thesis documents and lecture notes somewhat more convenient.
However, the package can also be used efficiently for
composing a series of documents (such as exercise sheets)
which are typically distributed individually.
It then assists the author in generating the individual documents
(potentially in different versions)
as well as a document containing the collected series.
Another application is in developing style files
or other kinds of included material
where compilation of the style file could redirect
to a sample or test file.

%%%%%%%%%%%%%%%%%%%%%%%%%%%%%%%%%%%%%%%%%%%%%%%%%%%%%%%%%%%%%%%%%%%%%%%%%%%%%%%%
%%%%%%%%%%%%%%%%%%%%%%%%%%%%%%%%%%%%%%%%%%%%%%%%%%%%%%%%%%%%%%%%%%%%%%%%%%%%%%%%
\section{Usage}

First of all, the package \textsf{childdoc} is \emph{not} a standard
\LaTeXe{} |.sty| style file! Therefore it needs to be invoked in
a non-standard way.

%%%%%%%%%%%%%%%%%%%%%%%%%%%%%%%%%%%%%%%%%%%%%%%%%%%%%%%%%%%%%%%%%%%%%%%%%%%%%%%%
\subsection{Included Files}
\label{sec:include}

%%%%%%%%%%%%%%%%%%%%%%%%%%%%%%%%%%%%%%%%
\DescribeMacro{\childdocmain}
To use the package, add the commands
\begin{center}
\begin{tabular}{l}
|\input{childdoc.def}|\\
|\childdocmain{}|\\
\end{tabular}
\end{center}
at the very top of the main \LaTeX{} file,
in particular \emph{before} the |\documentclass| statement!
The argument of |\childdocmain| should be left empty
(but it must be present).

%%%%%%%%%%%%%%%%%%%%%%%%%%%%%%%%%%%%%%%%
\DescribeMacro{\childdocof}
Furthermore, add the commands
\begin{center}
\begin{tabular}{l}
|\input{childdoc.def}|\\
|\childdocof{|\textit{main}|}|\\
\end{tabular}
\end{center}
at the top of every child file \textit{child}
which is included by |\include{|\textit{child}|}|
from within the main file
(or at least for those files to be compiled individually).
The argument \textit{main} must be the filename of the main file.

There are a couple of
considerations in setting up the main and child documents:

%%%%%%%%%%%%%%%%%%%%%%%%%%%%%%%%%%%%%%%%
\paragraph{Restrictions.}

Please note the following restrictions:
\begin{itemize}
\item
|\childdocmain| must be called with one argument \textit{main}
to ensure compatibility with earlier version of the package.
It must either be empty (|\childdocmain{}|)
or precisely match the filename of the main file in which it is specified.
See \secref{sec:detection} for further information.
\item
The filename \textit{main} must be specified without the |.tex| extension.
\item
The filename \textit{main} is case sensitive
(even in case-insensitive file systems)
due to internal string comparison.
\item
The argument \textit{main} should be fully expanded, it cannot be a macro.
\item
Subdirectories and special characters should be avoided in filenames.
\item
The command |\childdocmain{|\textit{main}|}| must be followed by a whitespace.
It should not be followed immediately by another command
or by a comment mark `|%|'.
This is because the \TeX{} parser reads the token immediately following
the argument of |\childdocmain| and puts it
at the beginning of every child section;
however, a white\-space is ignored.
\end{itemize}

%%%%%%%%%%%%%%%%%%%%%%%%%%%%%%%%%%%%%%%%
\paragraph{Content of Main File.}

It is advisable to place all content in the child files included by |\include|.
Any output contained in the main file will appear in all child documents
unless suppressed manually;
it cannot be suppressed automatically by the |\includeonly| directive
and thus should normally be avoided.
A method to include some content in the main file
by means of conditional processing is described in \secref{sec:conditional}.

%%%%%%%%%%%%%%%%%%%%%%%%%%%%%%%%%%%%%%%%
\paragraph{Page Numbering.}

When only a part of the document is compiled,
the appropriate numbering of pages
(as well as other status parameters)
is determined from the |.aux| files.
The latter contain information from previous passes.
However this information needs to propagate through
all intermediate child documents.
Therefore the page numbering in child documents may well
be inconsistent until the complete document is compiled at least once.

A useful (if unconventional) way to always ensure a consistent
page numbering is to restart the numbering in each child document
and denote the pages by `\textit{child}|.|\textit{page}'
where \textit{child} represents the chapter/section number of the child file.
This can be achieved by the command
|\numberwithin{page}{|\textit{child}|}|
of the \textsf{amsmath} package
where \textit{child} can be |chapter| or |section|
depending on the chosen structuring.
Alternatively, one can modify the macro |\thepage| appropriately
and reset the counter |page| at the start of each child file.

%%%%%%%%%%%%%%%%%%%%%%%%%%%%%%%%%%%%%%%%%%%%%%%%%%%%%%%%%%%%%%%%%%%%%%%%%%%%%%%%
\subsection{Conditional Processing}
\label{sec:conditional}

The package provides a mechanism to compile different versions
of a document. To customise the versions further some conditional processing
can come in handy to distinguish which version is being compiled.
The package provides two macros to describe the compilation context:

%%%%%%%%%%%%%%%%%%%%%%%%%%%%%%%%%%%%%%%%
\DescribeMacro{\ifchilddoc}
The conditional |\ifchilddoc| distinguishes between the compilation of
child documents and the main document:
%
\begin{center}
|\ifchilddoc |\textit{child-code}| |[|\||else |\textit{main-code}]| \||fi|
\end{center}

%%%%%%%%%%%%%%%%%%%%%%%%%%%%%%%%%%%%%%%%
\DescribeMacro{\childdocname}
\DescribeMacro{\childdocjob}
The macro |\childdocname| contains the filename (without extension)
of the main or child file being processed.
Note that |\childdocjob| will always contain the name of the main file.

%%%%%%%%%%%%%%%%%%%%%%%%%%%%%%%%%%%%%%%%
\paragraph{Title Page.}

Conditional processing can be used to include a title or banner page
in the main document when proper precautions are taken.
Importantly, the code in the main file should ensure that the page counter
(as well as other status parameters which are stored in the |.aux| files)
takes the same value after the conditional processing.
Otherwise the page numbers may take divergent values
depending on which part is compiled.

For example, a title page could be declared by:
%
\begin{center}
\begin{tabular}{l}
|\ifchilddoc\||else|\\
|\addtocounter{page}{-1}|\\
\textit{code for title page}\\
|\newpage|\\
|\||fi|
\end{tabular}
\end{center}
%
A banner page for the child documents can be generated by:
%
\begin{center}
\begin{tabular}{l}
|\ifchilddoc|\\
|\addtocounter{page}{-1}|\\
\textit{code for banner page}\\
|\newpage|\\
|\||fi|
\end{tabular}
\end{center}
%
Here one could write a message such as:
\begin{center}
|This is the part \childdocname{} of \childdocjob{}.|
\end{center}

%%%%%%%%%%%%%%%%%%%%%%%%%%%%%%%%%%%%%%%%%%%%%%%%%%%%%%%%%%%%%%%%%%%%%%%%%%%%%%%%
\subsection{Flags}
\label{sec:flags}

The package makes it easy to generate different versions
of the main or child documents.
To this end compilation flags can be defined
and assigned different default values.
They will be particularly useful in conjunction
with the forwarding mechanism described in \secref{sec:forward}.

For example, it may be useful to have a flag |\version|
which can be set to |draft| or |final|.
The document source will contain some conditional code
depending on the value of |\version|.
Suppose further, the flag should default to |final| for the main file
and to |draft| for child files
which is a natural assignment for editing the document.
This is achieved by placing the following code
in the preamble of the main document
(below the |\childdocmain| directive):
%
\begin{center}
\begin{tabular}{l}
|\ifchilddoc|\\
|\providecommand{\version}{draft}|\\
|\||else|\\
|\providecommand{\version}{final}|\\
|\||fi|
\end{tabular}
\end{center}
%
The definition by |\providecommand| makes sure
that previous definitions are not overwritten.
Further statements |\providecommand{\version}{...}|
can thus be added before the above code to override it.

For the main file, one might add a line
(between |\childdocmain| and the above block)
%
\begin{center}
|%\ifchilddoc\||else\providecommand{\version}{draft}\||fi|
\end{center}
%
which can be uncommented to produce a draft version.
Likewise one can add a line to the very top of a child file
(above the |\childdocof{|\textit{main}|}| directive)
%
\begin{center}
|%\providecommand{\version}{final}|
\end{center}
%
which can be uncommented to produce the final version of this child document.

%%%%%%%%%%%%%%%%%%%%%%%%%%%%%%%%%%%%%%%%%%%%%%%%%%%%%%%%%%%%%%%%%%%%%%%%%%%%%%%%
\subsection{Forwarding}
\label{sec:forward}

Different versions of the main or child documents
using compilation flags as described in \secref{sec:flags}
can be (permanently) stored in different files
for convenient compilation, viewing and distribution.
To this end, the package defines a command
to pass on compilation to a different file:

%%%%%%%%%%%%%%%%%%%%%%%%%%%%%%%%%%%%%%%%
\DescribeMacro{\childdocforward}
The command |\childdocforward| redirects processing to
another source file:
%
\begin{center}
\begin{tabular}{l}
|\input{childdoc.def}|\\
|\childdocforward[|\textit{main}|]{|\textit{dest}|}|\\
\end{tabular}
\end{center}
%
The argument \textit{dest} is the destination file
(without extension).
It should be the main file or one of the child files.
Note that further \textsf{childdoc} directives
such as |\childdocof| and |\childdocforward|
in the indicated file will be processed in this form.
The optional argument \textit{main}
passes on directly to the main file \textit{main}
while pretending to compile the child \textit{dest}.
This form behaves as if \textit{dest}
issues |\childdocof{|\textit{main}|}| right away,
and no further \textsf{childdoc} directives will be processed.

%%%%%%%%%%%%%%%%%%%%%%%%%%%%%%%%%%%%%%%%
\DescribeMacro{\...prefix}
In the alternative form |\childdocforwardprefix|,
%
\begin{center}
\begin{tabular}{l}
|\input{childdoc.def}|\\
|\childdocforwardprefix[|\textit{main}|]{|\textit{prefix}|}{|\textit{dest}|}|
\end{tabular}
\end{center}
%
the destination file is determined by a pattern
depending on the current file:
To make this work, the current file must be called
`{\textit{prefix}\hspace{0.2em}\textit{suffix}}'
with \textit{prefix} matching precisely the argument.
Processing is then passed on to the file
`{\textit{dest}\hspace{0.2em}\textit{suffix}}'.
Surely, the same effect is achieved by
directly specifying the
argument `{\textit{dest}\hspace{0.2em}\textit{suffix}}'
in the first form.
However, that requires to set up a different file
for each child. With the alternative form of the command
all these files can have exactly the same content
which simplifies setting them up and maintaining them.

For example, the following file |draft.tex|
with a compilation flag |\version| as described in \secref{sec:flags}
compiles the main document as a draft:
%
\begin{center}
\begin{tabular}{l}
|\def\version{draft}|\\
|\input{childdoc.def}|\\
|\childdocforward{|\textit{main}|}|
\end{tabular}
\end{center}
%
Likewise, the following files |final|\textit{nn}|.tex|
compile the final version of the child document
|child|\textit{nn}|.tex|:
%
\begin{center}
\begin{tabular}{l}
|\def\version{final}|\\
|\input{childdoc.def}|\\
|\childdocforwardprefix{final}{child}|
\end{tabular}
\end{center}
%

Note that when several versions of a main file and/or of each child file
are to be generated, it may be convenient to set up a |Makefile| or
shell script to automatise the process.

%%%%%%%%%%%%%%%%%%%%%%%%%%%%%%%%%%%%%%%%%%%%%%%%%%%%%%%%%%%%%%%%%%%%%%%%%%%%%%%%
\subsection{Command Line Processing}
\label{sec:commandline}

The effect of redirection files can also be achieved by invoking
the \LaTeX{} compiler with a more elaborate command line.
Most conveniently this should be done as part
of a shell script or a |Makefile|.

When using \textsf{childdoc} in the main file, the following
command lines effectively perform a redirection
(note that depending on the shell being used,
backslashes may have to be doubled: `|\|' $\to$ `|\\|'):
%
\begin{center}
|... -jobname "|\textit{target}|" |\\|"|[\textit{flags}]%
|\input{childdoc.def}\childdocforward[|\textit{main}|]{|\textit{dest}|}"|
\end{center}
%
Here \textit{target} is the name of the output file,
\textit{main} is the name of the main file
and \textit{dest} is the name of the main or child file to be processed
(all filenames without extensions).
The optional argument \textit{main} can be omitted
if \textit{main} matches \textit{dest}.
Optionally, compilation \textit{flags} can be defined via |\def| commands.
This command line makes the \TeX{} engine believe
it is compiling the file \textit{target}
whose content is specified as the latter parameter.
The provided code then forwards the processing to
\textit{main} or \textit{dest} as described in \secref{sec:forward}.

%%%%%%%%%%%%%%%%%%%%%%%%%%%%%%%%%%%%%%%%%%%%%%%%%%%%%%%%%%%%%%%%%%%%%%%%%%%%%%%%
\subsection{Include by Input}
\label{sec:input}

Including child documents by |\include| has some restrictions by design.
Most notably, the content of a child document always occupies
its own set of pages; pages cannot be shared between child documents.
Usually, this behaviour makes perfect sense
because each child document contain an essential part of the document.
However, in some situations it may be desirable to compose
a document from a collection of parts
without having mandatory page breaks between then.
For this case, the package
provides a mechanism to include parts
by |\input| which can also be processed individually.
However, by construction this mechanism
requires manual handling of the content to be output.

%%%%%%%%%%%%%%%%%%%%%%%%%%%%%%%%%%%%%%%%
\DescribeMacro{\ifchilddocmanual}
The main file should be prepared as usual, see \secref{sec:include}.
However, the document body must make a distinction
between processing of an individual part and of the main document, e.g.:
%
\begin{center}
\begin{tabular}{l}
|\ifchilddocmanual|\\
|\input{\childdocname}|\\
|\||else|\\
\textit{document body with }|\input{|\textit{part}|}|\\
|\||fi|
\end{tabular}
\end{center}
%
The conditional |\ifchilddocmanual| is true whenever
a part to be included by |\input| is being compiled,
and the name of the part is stored in |\childdocname|.

%%%%%%%%%%%%%%%%%%%%%%%%%%%%%%%%%%%%%%%%
\DescribeMacro{\childdocby}
Each part to be included by |\input| should start with:
%
\begin{center}
\begin{tabular}{l}
|\input{childdoc.def}|\\
|\childdocby{|\textit{main}|}|\\
\end{tabular}
\end{center}
%
The directive |\childdocby| is similar to |\childdocof|
described in \secref{sec:include},
but the subsequent selection of content must be done manually.
To that end, both |\ifchilddoc| and |\ifchilddocmanual|
will be true upon processing of a part,
and the name of the part is stored in |\childdocname|.
Note that |\jobname| will be set to the filename of the current part
so that each part receives an individual |.aux| file
that does not interfere with the |.aux| file(s) of the main document.
This behaviour can be altered by the alternative form
|\childdocby[*]{|\textit{main}|}| (with a non-empty optional argument)
which uses the |.aux| file of the main document
by setting |\jobname| to \textit{main}.

%%%%%%%%%%%%%%%%%%%%%%%%%%%%%%%%%%%%%%%%%%%%%%%%%%%%%%%%%%%%%%%%%%%%%%%%%%%%%%%%
\subsection{Driver Development}
\label{sec:driver}

The \textsf{childdoc} mechanism can also be use for the development
of definition files such as \LaTeX{} styles or classes.
This case differs from the above setup with multiple parts
included by |\include| in that no |\includeonly| should be invoked.
This can be achieved by starting the include file
(before |\ProvidesPackage|) with:
%
\begin{center}
\begin{tabular}{l}
|\input{childdoc.def}|\\
|\childdocforward{|\textit{main}|}|\\
\end{tabular}
\end{center}
%
or alternatively with:
%
\begin{center}
\begin{tabular}{l}
|\input{childdoc.def}|\\
|\childdocby{|\textit{main}|}|\\
\end{tabular}
\end{center}
%
Both forms have slightly different effects as described above.
The main file is prepared as usual, see \secref{sec:include}.

%%%%%%%%%%%%%%%%%%%%%%%%%%%%%%%%%%%%%%%%%%%%%%%%%%%%%%%%%%%%%%%%%%%%%%%%%%%%%%%%
\subsection{Legacy Detection}
\label{sec:detection}

The directive |\childdocmain| in the main file can detect
whether the complete document or merely a child is to be compiled
even without using the directive |\childdocof|.
This method is deprecated because it is less robust
and there is no compelling reason to use it;
it is merely provided for backward compatibility
and it may be removed in future versions.

If the detection mechanism is to be used,
it is mandatory to correctly specify
the filename of the main file as the argument of |\childdocmain|:
%
\begin{center}
\begin{tabular}{l}
|\input{childdoc.def}|\\
|\childdocmain{|\textit{main}|}|\\
\end{tabular}
\end{center}
%
If |\jobname| does not match the argument \textit{main} of |\childdocmain|,
it is assumed that |\jobname| points to the child file to be compiled.
When using |\childdocmain| with the main file specified as argument,
it suffices to start a child file
with just |\input{|\textit{main}|}|
without loading of the package and using |\childdocof|.
If instead all processing is done
with the appropriate \textsf{childdoc} directives,
the argument of \textit{main} of |\childdocmain| can be empty.

An alternative version of the command line processing described
in \secref{sec:commandline} using the detection mechanism reads:
%
\begin{center}
|... -jobname "|\textit{target}|" "|[\textit{flags}]%
[|\def\jobname{|\textit{dest}|}|]|\input{|\textit{main}|}"|
\end{center}

%%%%%%%%%%%%%%%%%%%%%%%%%%%%%%%%%%%%%%%%%%%%%%%%%%%%%%%%%%%%%%%%%%%%%%%%%%%%%%%%
\subsection{Manual Code}
\label{sec:manual}

In case one cannot be certain whether the definitions file |childdoc.def|
is installed on the target \TeX{} distribution
and one prefers not to ship it,
it is conceivable to paste a few relevant commands into the sources.

To that end, drop all statements |\input{childdoc.def}|
and perform the replacements as outlined below.
Instead of |\childdocmain{|\textit{main}|}| add the following code
to the top of the main file:
%
\begin{center}
\begin{tabular}{l}
|\||ifdefined\childdocname\endinput\||fi\newif\ifchilddoc|\\
|\edef\childdocname{\scantokens\expandafter{\jobname\noexpand}}|\\
|\def\childdocmain{|\textit{main}|}\||ifx\childdocmain\childdocname\||else|\\
|\childdoctrue\includeonly{\childdocname}\let\jobname\childdocmain\||fi|\\
\end{tabular}
\end{center}
%
Instead of |\childdocof{|\textit{main}|}| just include the main file
at the top of each child file:
%
\begin{center}
|\input{|\textit{main}|}|
\end{center}
%
A simple redirection |\childdocforward{|\textit{dest}|}| is achieved by:
%
\begin{center}
|\def\jobname{|\textit{dest}|}\input{\jobname}|
\end{center}
%
The redirection with prefix
|\childdocforwardprefix[|\textit{prefix}|]{|\textit{dest}|}|
is accomplished by:
%
\begin{center}
\begin{tabular}{l}
|{\edef\jobname{\scantokens\expandafter{\jobname\noexpand}}|\\
|\def\redirectjob |\textit{prefix}|#1~~~{\gdef\jobname{|\textit{dest}|#1}}|\\
|\expandafter\redirectjob\jobname~~~}\input{\jobname}|
\end{tabular}
\end{center}

In an alternative approach,
child documents can be compiled by a specific command line
without additional code or specific definitions:
%
\begin{center}
|... -jobname "|\textit{target}|" "|[\textit{flags}]%
|\includeonly{|\textit{dest}|}\input{|\textit{main}|}"|
\end{center}
%

%%%%%%%%%%%%%%%%%%%%%%%%%%%%%%%%%%%%%%%%%%%%%%%%%%%%%%%%%%%%%%%%%%%%%%%%%%%%%%%%
%%%%%%%%%%%%%%%%%%%%%%%%%%%%%%%%%%%%%%%%%%%%%%%%%%%%%%%%%%%%%%%%%%%%%%%%%%%%%%%%
\section{Information}

%%%%%%%%%%%%%%%%%%%%%%%%%%%%%%%%%%%%%%%%%%%%%%%%%%%%%%%%%%%%%%%%%%%%%%%%%%%%%%%%
\subsection{Copyright}

Copyright \copyright{} 2017--2018 Niklas Beisert

This work may be distributed and/or modified under the
conditions of the \LaTeX{} Project Public License, either version 1.3
of this license or (at your option) any later version.
The latest version of this license is in
  \url{http://www.latex-project.org/lppl.txt}
and version 1.3 or later is part of all distributions of \LaTeX{}
version 2005/12/01 or later.

This work has the LPPL maintenance status `maintained'.

The Current Maintainer of this work is Niklas Beisert.

This work consists of the files |README.txt|, |childdoc.ins| and |childdoc.dtx|
as well as the derived files |childdoc.def|, |cdocsamp.tex|
with |cdocsch1.tex|, |cdocsch2.tex|, |cdocspt3.tex|, |cdocspt4.tex|,
|cdocsdrf.tex|, |cdocsfn1.tex|, |cdocsfn2.tex|
as well as |childdoc.pdf|.

%%%%%%%%%%%%%%%%%%%%%%%%%%%%%%%%%%%%%%%%%%%%%%%%%%%%%%%%%%%%%%%%%%%%%%%%%%%%%%%%
\subsection{Files and Installation}

The package consists of the files:
%
\begin{center}
\begin{tabular}{ll}
    |README.txt|   & readme file \\
    |childdoc.ins| & installation file \\
    |childdoc.dtx| & source file \\
    |childdoc.def| & definition file \\
    |cdocsamp.tex| & sample main file \\
    |cdocsch1.tex| & sample include file \\
    |cdocsch2.tex| & sample include file \\
    |cdocspt3.tex| & sample part file \\
    |cdocspt4.tex| & sample part file \\
    |cdocsdrf.tex| & sample redirection file \\
    |cdocsfn1.tex| & sample redirection file \\
    |cdocsfn2.tex| & sample redirection file \\
    |childdoc.pdf| & manual
\end{tabular}
\end{center}
%
The distribution consists of the files
|README.txt|, |childdoc.ins| and |childdoc.dtx|.
%
\begin{itemize}
\item
Run (pdf)\LaTeX{} on |childdoc.dtx|
to compile the manual |childdoc.pdf| (this file).
\item
Run \LaTeX{} on |childdoc.ins| to create the definitions file |childdoc.def|
and the sample |cdocsamp.tex| with include files
|cdocsch1.tex|, |cdocsch2.tex|, |cdocspt3.tex|, |cdocspt4.tex|,
|cdocsdrf.tex|, |cdocsfn1.tex|, |cdocsfn2.tex|.
Then copy the file |childdoc.def| to an appropriate directory of your \LaTeX{}
distribution, e.g.\ \textit{texmf-root}|/tex/latex/childdoc|.
\end{itemize}

%%%%%%%%%%%%%%%%%%%%%%%%%%%%%%%%%%%%%%%%%%%%%%%%%%%%%%%%%%%%%%%%%%%%%%%%%%%%%%%%
\subsection{Related CTAN Packages}

There are several other packages which offer a similar functionality:
%
\begin{itemize}
\item
The packages
\href{http://ctan.org/pkg/docmute}{\textsf{docmute}},
\href{http://ctan.org/pkg/includex}{\textsf{includex}} and
\href{http://ctan.org/pkg/standalone}{\textsf{standalone}}
provide commands to include only the document body of
a child file thus allowing both files to be compiled individually.
\item
The packages \href{http://ctan.org/pkg/subdocs}{\textsf{subdocs}}
and \href{http://ctan.org/pkg/subfiles}{\textsf{subfiles}}
provide structures in which the main and child documents can be
encapsulated and allowing them to be compiled individually.
The inclusion mechanism is different from the conventional |\include|.
\item
The package \href{http://ctan.org/pkg/combine}{\textsf{combine}}
is an elaborate solution to combine several documents into one.
\end{itemize}
%
See also the CTAN topic \href{http://ctan.org/topic/subdocs}{\textsf{subdocs}}
for further related packages.
The present package differs from the above solutions in that
a document structure constructed with the conventional |\include| mechanism
just needs two extra commands at the top of every file
such that all constituent files can be compiled individually.

%%%%%%%%%%%%%%%%%%%%%%%%%%%%%%%%%%%%%%%%%%%%%%%%%%%%%%%%%%%%%%%%%%%%%%%%%%%%%%%%
%\subsection{Feature Suggestions}
%
%The following is a list of features which may be useful for future
%versions of this package:
%%
%\begin{itemize}
%\item
%\ldots
%\end{itemize}

%%%%%%%%%%%%%%%%%%%%%%%%%%%%%%%%%%%%%%%%%%%%%%%%%%%%%%%%%%%%%%%%%%%%%%%%%%%%%%%%
\subsection{Revision History}

%%%%%%%%%%%%%%%%%%%%%%%%%%%%%%%%%%%%%%%%
\paragraph{v2.0:} 2018/12/30

\begin{itemize}
\item
immediate forward processing
\item
added |\childdocby| mechanism
\item
manual restructured
\end{itemize}

%%%%%%%%%%%%%%%%%%%%%%%%%%%%%%%%%%%%%%%%
\paragraph{v1.6:} 2018/01/17

\begin{itemize}
\item
application for development of include files
\item
corrections to manual
\end{itemize}

%%%%%%%%%%%%%%%%%%%%%%%%%%%%%%%%%%%%%%%%
\paragraph{v1.5:} 2017/05/21

\begin{itemize}
\item
more complete structuring introduced
\item
|\childdocof| introduced
\item
|\childdoc| renamed to |\childdocmain|
\item
|\childredirect| renamed to |\childdocforward| and |\childdocforwardprefix|
and functionality expanded
\end{itemize}

%%%%%%%%%%%%%%%%%%%%%%%%%%%%%%%%%%%%%%%%
\paragraph{v1.0:} 2017/04/27

\begin{itemize}
\item
manual and install package
\item
first version published on CTAN
\end{itemize}

%%%%%%%%%%%%%%%%%%%%%%%%%%%%%%%%%%%%%%%%
\paragraph{v0.6:} 2017/04/26

\begin{itemize}
\item
redirection mechanism added
\end{itemize}

%%%%%%%%%%%%%%%%%%%%%%%%%%%%%%%%%%%%%%%%
\paragraph{v0.5:} 2017/04/26

\begin{itemize}
\item
functionality in definition file
\end{itemize}


%%%%%%%%%%%%%%%%%%%%%%%%%%%%%%%%%%%%%%%%%%%%%%%%%%%%%%%%%%%%%%%%%%%%%%%%%%%%%%%%
%%%%%%%%%%%%%%%%%%%%%%%%%%%%%%%%%%%%%%%%%%%%%%%%%%%%%%%%%%%%%%%%%%%%%%%%%%%%%%%%
%%%%%%%%%%%%%%%%%%%%%%%%%%%%%%%%%%%%%%%%%%%%%%%%%%%%%%%%%%%%%%%%%%%%%%%%%%%%%%%%
\appendix

\settowidth\MacroIndent{\rmfamily\scriptsize 000\ }

 \DocInput{childdoc.dtx}

\end{document}
%</driver>
% \fi
%
% %%%%%%%%%%%%%%%%%%%%%%%%%%%%%%%%%%%%%%%%%%%%%%%%%%%%%%%%%%%%%%%%%%%%%%%%%%%%%%
% %%%%%%%%%%%%%%%%%%%%%%%%%%%%%%%%%%%%%%%%%%%%%%%%%%%%%%%%%%%%%%%%%%%%%%%%%%%%%%
% \section{Sample}
%\iffalse
%<*samplemain>
%\fi
%
% The following presents a sample document
% with two chapters, two parts, a title page,
% a compile flag as well as three forwarding files to set the flag.
% It consists of eight |.tex| files:
% \begin{center}
% \begin{tabular}{ll}
% |cdocsamp.tex|&main file\\
% |cdocsch1.tex|&include file for chapter 1\\
% |cdocsch2.tex|&include file for chapter 2\\
% |cdocspt3.tex|&include file for part 3\\
% |cdocspt4.tex|&include file for part 4\\
% |cdocsdrf.tex|&forwarding file for main file in draft mode\\
% |cdocsfi1.tex|&forwarding file for final version of chapter 1\\
% |cdocsfi2.tex|&forwarding file for final version of chapter 2\\
% \end{tabular}
% \end{center}
% Each of the eight files can be compiled directly by the \LaTeX{} compiler.
%
% %%%%%%%%%%%%%%%%%%%%%%%%%%%%%%%%%%%%%%
% \paragraph{Main File.}
%
% The main file is called |cdocsamp.tex|.
%
% Load the \textsf{childdoc} definitions and
% declare the filename for the main document:
%    \begin{macrocode}
\input{childdoc.def}
\childdocmain{}
%    \end{macrocode}

% Optional override for |\version| flag:
%    \begin{macrocode}
%%\ifchilddoc\else\providecommand{\version}{draft}\fi
%    \end{macrocode}

% Define the default values for the |\version| flag
% (|final| for the main file and |draft| for childs):
%    \begin{macrocode}
\ifchilddoc
\providecommand{\version}{draft}
\else
\providecommand{\version}{final}
\fi
%    \end{macrocode}

% Load the standard document class:
%    \begin{macrocode}
\documentclass[12pt]{article}
%    \end{macrocode}

% Start the document body:
%    \begin{macrocode}
\begin{document}
%    \end{macrocode}

% Declare a title page.
% Print title, part of document being processed and version flag:
%    \begin{macrocode}
\addtocounter{page}{-1}
\begin{center}
{\LARGE\bfseries{}childdoc example\par}
\vspace{1cm}
\ifchilddoc
\ifchilddocmanual part\else chapter\fi:
`\childdocname' of `\childdocjob'\par
\else
main document: `\childdocjob'\par
\fi
version: \version\par
\end{center}
\newpage
%    \end{macrocode}

% Manually include selected file,
% otherwise process as usual:
%    \begin{macrocode}
\ifchilddocmanual
\section*{part `\childdocname'}
\input{\childdocname}
\else
%    \end{macrocode}

% Include the two chapters:
%    \begin{macrocode}
\include{cdocsch1}
\include{cdocsch2}
%    \end{macrocode}

% Include the two parts unless only chapters should be displayed:
%    \begin{macrocode}
\ifchilddoc\else
\section{part three}
\input{cdocspt3}
\section{part four}
\input{cdocspt4}
\fi
%    \end{macrocode}

% Process as usual until here:
%    \begin{macrocode}
\fi
%    \end{macrocode}

% End of document body:
%    \begin{macrocode}
\end{document}
%    \end{macrocode}
%\iffalse
%</samplemain>
%\fi
%
% %%%%%%%%%%%%%%%%%%%%%%%%%%%%%%%%%%%%%%
% \paragraph{Chapter Include Files.}
%
% The include files are called |cdocsch1.tex| and |cdocsch2.tex|.
%
%\iffalse
%<*samplechap1|samplechap2>
%\fi

% Optional override for |\version| flag:
%    \begin{macrocode}
%%\providecommand{\version}{final}
%    \end{macrocode}

% Include the main document:
%    \begin{macrocode}
\input{childdoc.def}
\childdocof{cdocsamp}
%    \end{macrocode}

%\iffalse
%</samplechap1|samplechap2>
%\fi
%
%\iffalse
%<*samplechap1>
%\fi
% Some text for chapter 1:
%    \begin{macrocode}
\section{one}
some text in chapter one
%    \end{macrocode}

%\iffalse
%</samplechap1>
%\fi
% Some text for chapter 2:
%\iffalse
%<*samplechap2>
%\fi
%    \begin{macrocode}
\section{two}
more text in chapter two
%    \end{macrocode}

%\iffalse
%</samplechap2>
%\fi
%
% %%%%%%%%%%%%%%%%%%%%%%%%%%%%%%%%%%%%%%
% \paragraph{Part Include Files.}
%
% The include files are called |cdocspt3.tex| and |cdocspt4.tex|.
%
%\iffalse
%<*samplepart3|samplepart4>
%\fi

% Optional override for |\version| flag:
%    \begin{macrocode}
%%\providecommand{\version}{final}
%    \end{macrocode}

% Include the main document:
%    \begin{macrocode}
\input{childdoc.def}
\childdocby{cdocsamp}
%    \end{macrocode}

%\iffalse
%</samplepart3|samplepart4>
%\fi
%
%\iffalse
%<*samplepart3>
%\fi
% Some text for part 3:
%    \begin{macrocode}
some text in part three
%    \end{macrocode}

%\iffalse
%</samplepart3>
%\fi
% Some text for part 4:
%\iffalse
%<*samplepart4>
%\fi
%    \begin{macrocode}
more text in part four
%    \end{macrocode}

%\iffalse
%</samplepart4>
%\fi
%
% %%%%%%%%%%%%%%%%%%%%%%%%%%%%%%%%%%%%%%
% \paragraph{Forwarding for a Complete Draft.}
%
% The following forwarding file |cdocsdrf.tex|
% compiles the main document in draft mode:
%\iffalse
%<*sampledraft>
%\fi
%    \begin{macrocode}
\def\version{draft}
\input{childdoc.def}
\childdocforward{cdocsamp}
%    \end{macrocode}

%\iffalse
%</sampledraft>
%\fi
%
% %%%%%%%%%%%%%%%%%%%%%%%%%%%%%%%%%%%%%%
% \paragraph{Forwarding for Final Version of the Chapters.}
%
% The following forwarding files |cdocsfn1.tex| and |cdocsfn2.tex|
% (with identical content)
% compile the final versions of the child documents
% |cdocsch1.tex| and |cdocsch2.tex|, respectively:
%\iffalse
%<*samplefinal>
%\fi
%    \begin{macrocode}
\def\version{final}
\input{childdoc.def}
\childdocforwardprefix[cdocsamp]{cdocsfn}{cdocsch}
%    \end{macrocode}

%\iffalse
%</samplefinal>
%\fi
%
% %%%%%%%%%%%%%%%%%%%%%%%%%%%%%%%%%%%%%%
% \paragraph{Command Line Processing.}
%
% The following three command lines generate the output files
% |cdocscld|, |cdocscl1| and |cdocscl2|
% which should be identical to
% |cdocsdrf|, |cdocsch1| and |cdocsfn2|, respectively:
% \begin{center}
% \begin{tabular}{l}
% |latex -jobname cdocscld \|\\
% |  "\def\version{draft}\input{childdoc.def}\childdocforward{cdocsamp}"|\\
% |latex -jobname cdocscl1 \|\\
% |  "\input{childdoc.def}\childdocforward[cdocsamp]{cdocsch1}"|\\
% |latex -jobname cdocscl2 \|\\
% |  "\def\version{final}\input{childdoc.def}\childdocforward{cdocsch2}"|
% \end{tabular}
% \end{center}
% Note that the trailing backslash on each first line
% merely continues the input to the second line
% (for convenient cut ant paste).
% Furthermore, the command |latex| can be replaced by any
% of its alternative versions such as |pdflatex|.
%
% %%%%%%%%%%%%%%%%%%%%%%%%%%%%%%%%%%%%%%%%%%%%%%%%%%%%%%%%%%%%%%%%%%%%%%%%%%%%%%
% %%%%%%%%%%%%%%%%%%%%%%%%%%%%%%%%%%%%%%%%%%%%%%%%%%%%%%%%%%%%%%%%%%%%%%%%%%%%%%
% \section{Implementation}
%\iffalse
%<*package>
%\fi
%
% This section describes the definitions file |childdoc.def|.

% The definitions cannot be loaded using |\usepackage| or |\RequirePackage|
% which has a mechanism to prevent loading a style file more than once.
% When loading the definitions by means of |\input|
% multiple instances have to be prevented manually:
%\iffalse
%This code needs to be before the `\ProvidesFile' directive
%which is defined at the beginning of this file.
%Therefore it is also placed there and commented out here.
%</package>
%<*discard>
%\fi
%    \begin{macrocode}
\ifdefined\childdocmain\endinput\fi
%    \end{macrocode}
%\iffalse
%</discard>
%<*package>
%\fi
%
% \macro{\ifchilddoc}
% \macro{\ifchilddocmanual}
% The conditional |\ifchilddoc| tells whether a
% child (true) or main (false) document is being compiled.
% The conditional |\ifchilddocmanual| tells whether
% the |\includeonly| mechanism is used (false) or
% the selection of child files must be performed manually (true).
% The definitions initialise to false:
%    \begin{macrocode}
\newif\ifchilddoc
\newif\ifchilddocmanual
%    \end{macrocode}

% \macro{\childdocname}
% \macro{\childdocjob}
% The macro |\childdocname| stores the name of the main document
% to be compiled. The macro |\childdocjob| stores the name of
% the document on which the \LaTeX{} compiler was originally invoked.
% The content of |\jobname| cannot be compared
% to filenames specified in the source due to different catcodes.
% The following code rescans |\jobname|, stores the result
% in |\childdocname| and saves a copy in |\childdocjob|:
%    \begin{macrocode}
\edef\childdocname{\scantokens\expandafter{\jobname\noexpand}}
\let\childdocjob\childdocname
%    \end{macrocode}

% \macro{\childdocdisable}
% The macro |\childdocdisable| prevents the main file
% from being processed more than once.
% At this stage, the main document command |\childdocmain|
% is assumed to be called once again where it should do nothing.
% Any subsequent call to it should prevent
% a secondary processing of the main document
% It overwrites the forwarding commands
% |\childdocof| and |\childdocforward|
% with empty macros to prevent further inclusions of the main document:
%    \begin{macrocode}
\newcommand{\childdocdisable}
{
  \renewcommand{\childdocmain}[1]{\renewcommand{\childdocmain}[1]{\endinput}}
  \renewcommand{\childdocof}[1]{}
  \renewcommand{\childdocby}[2][]{}
  \renewcommand{\childdocforward}[2][]{}
  \renewcommand{\childdocdisable}{}
}
%    \end{macrocode}

% \macro{\childdocmain}
% The macro |\childdocmain| is to be called at the top of the main file
% with nothing or the main filename (without extension) as argument.
% First, it breaks loops.
% If the argument is not empty and does not match |\childdocname|
% (which is set by the first inclusion of |childdoc.def|),
% |\ifchilddoc| is set to true, |\includeonly| is applied to the child file
% and |\jobname| is set to the main file
% (for proper handling of |.aux| files):
%    \begin{macrocode}
\newcommand{\childdocmain}[1]
{
  \childdocdisable\childdocmain{}
  \if?#1?\else
    \begingroup
      \def\childdoctmp{#1}
      \ifx\childdoctmp\childdocname
        \def\childdoctmp{}
      \else
        \def\childdoctmp
        {
          \childdoctrue
          \includeonly{\childdocname}
          \def\childdocjob{#1}
          \def\jobname{#1}
        }
      \fi
      \expandafter
    \endgroup
    \childdoctmp
  \fi
}
%    \end{macrocode}

% \macro{\childdocof}
% The command |\childdocof| redirects
% compilation to the main file |#1|.
%    \begin{macrocode}
\newcommand{\childdocof}[1]
{
  \childdocdisable
  \childdoctrue
  \includeonly{\childdocname}
  \def\jobname{#1}
  \def\childdocjob{#1}
  \input{#1}
}
%    \end{macrocode}

% \macro{\childdocby}
% The command |\childdocby| ....
%    \begin{macrocode}
\newcommand{\childdocby}[2][]
{
  \childdocdisable
  \childdoctrue
  \childdocmanualtrue
  \if?#1?\else
    \def\jobname{#2}
  \fi
  \def\childdocjob{#2}
  \input{#2}
  \endinput
}
%    \end{macrocode}

% \macro{\childdocforward}
% The command |\childdocforward| redirects
% compilation to the main file or
% (if the optional argument is given) a child file.
% Parameters are set as if the main file
% or a child file starting with |\childdocof| was compiled.
% Then compilation is handed over to the main file:
%    \begin{macrocode}
\newcommand{\childdocforward}[2][]
{
  \begingroup
    \if?#1?
      \def\childdoctmp
      {
        \def\childdocname{#2}
        \def\childdocjob{#2}
        \def\jobname{#2}
        \input{#2}
        \endinput
      }
    \else
      \def\childdoctmp
      {
        \childdocdisable
        \def\childdocname{#2}
        \childdoctrue
        \includeonly{#2}
        \def\childdocjob{#1}
        \def\jobname{#1}
        \input{#1}
        \endinput
      }
    \fi
    \expandafter
  \endgroup
  \childdoctmp
}
%    \end{macrocode}

% \macro{\childdocforwardprefix}
% The command |\childdocforwardprefix| redirects
% compilation to the main or a child file by means of a pattern.
% The prefix |#1| in the current filename is replaced by |#2|
% and the suffix of the current filename is kept
% (it is assumed that the filename does not contain the substring `|~~~|'
% which is used as a delimiter).
% Compilation is handed over to the new file by |\childdocforward|:
%    \begin{macrocode}
\newcommand{\childdocforwardprefix}[3][]
{
  \begingroup
    \def\childdocextract #2##1~~~{\def\childdoctmp{\childdocforward[#1]{#3##1}}}
    \expandafter\childdocextract\childdocname~~~
    \expandafter
  \endgroup
  \childdoctmp
}
%    \end{macrocode}

% \macro{\childdoc}
% The deprecated macro |\childdoc| is a legacy version of |\childdocmain|:
%    \begin{macrocode}
\newcommand{\childdoc}{\childdocmain}
%    \end{macrocode}

% \macro{\childdocredirect}
% The deprecated macro |\childdocredirect| is a legacy version
% of |\childdocforward| and |\childdocforwardprefix|:
%    \begin{macrocode}
\newcommand{\childdocredirect}[2][]
{
  \begingroup
    \if?#1?
      \def\childdoctmp{\childdocforward{#2}}
    \else
      \def\childdoctmp{\childdocforwardprefix{#1}{#2}}
    \fi
    \expandafter
  \endgroup
  \childdoctmp
}
%    \end{macrocode}

%\iffalse
%</package>
%\fi
%
\endinput

\childdocforwardprefix[cdocsamp]{cdocsfn}{cdocsch}
%    \end{macrocode}

%\iffalse
%</samplefinal>
%\fi
%
% %%%%%%%%%%%%%%%%%%%%%%%%%%%%%%%%%%%%%%
% \paragraph{Command Line Processing.}
%
% The following three command lines generate the output files
% |cdocscld|, |cdocscl1| and |cdocscl2|
% which should be identical to
% |cdocsdrf|, |cdocsch1| and |cdocsfn2|, respectively:
% \begin{center}
% \begin{tabular}{l}
% |latex -jobname cdocscld \|\\
% |  "\def\version{draft}% \iffalse
%
% childdoc.dtx Copyright (C) 2017-2018 Niklas Beisert
%
% This work may be distributed and/or modified under the
% conditions of the LaTeX Project Public License, either version 1.3
% of this license or (at your option) any later version.
% The latest version of this license is in
%   http://www.latex-project.org/lppl.txt
% and version 1.3 or later is part of all distributions of LaTeX
% version 2005/12/01 or later.
%
% This work has the LPPL maintenance status `maintained'.
%
% The Current Maintainer of this work is Niklas Beisert.
%
% This work consists of the files childdoc.dtx and childdoc.ins
% and the derived files childdoc.def and cdocsamp.tex with
% cdocsch1.tex, cdocsch2.tex, cdocsdrf.tex, cdocsfn1.tex, cdocsfn2.tex.
%
%<package>\ifdefined\childdocmain\endinput\fi
%<package>\ProvidesFile{childdoc.def}[2018/12/30 v2.0 child document driver]
%<samplemain>\ProvidesFile{cdocsamp.tex}[2018/12/30 v2.0 sample for childdoc]
%<*driver>
%\ProvidesFile{childdoc.drv}[2018/12/30 v2.0 childdoc reference manual file]
\PassOptionsToClass{10pt,a4paper}{article}
\documentclass{ltxdoc}

\usepackage[margin=35mm]{geometry}
\usepackage{hyperref}
\usepackage{hyperxmp}
\usepackage[usenames]{color}

\hypersetup{colorlinks=true}
\hypersetup{pdfstartview=FitH}
\hypersetup{pdfpagemode=UseNone}
\hypersetup{pdfsource={}}
\hypersetup{pdflang={en-UK}}
\hypersetup{pdfcopyright={Copyright 2017-2018 Niklas Beisert.
  This work may be distributed and/or modified under the
  conditions of the LaTeX Project Public License, either version 1.3
  of this license or (at your option) any later version.}}
\hypersetup{pdflicenseurl={http://www.latex-project.org/lppl.txt}}
\hypersetup{pdfcontactaddress={ETH Zurich, ITP, HIT K,
  Wolfgang-Pauli-Strasse 27}}
\hypersetup{pdfcontactpostcode={8093}}
\hypersetup{pdfcontactcity={Zurich}}
\hypersetup{pdfcontactcountry={Switzerland}}
\hypersetup{pdfcontactemail={nbeisert@itp.phys.ethz.ch}}
\hypersetup{pdfcontacturl={http://people.phys.ethz.ch/\xmptilde nbeisert/}}

\newcommand{\secref}[1]{\hyperref[#1]{section \ref*{#1}}}

\parskip1ex
\parindent0pt
\let\olditemize\itemize
\def\itemize{\olditemize\parskip0pt}

\begin{document}

\title{The \textsf{childdoc} Package}
\hypersetup{pdftitle={The childdoc Package}}
\author{Niklas Beisert\\[2ex]
  Institut f\"ur Theoretische Physik\\
  Eidgen\"ossische Technische Hochschule Z\"urich\\
  Wolfgang-Pauli-Strasse 27, 8093 Z\"urich, Switzerland\\[1ex]
  \href{mailto:nbeisert@itp.phys.ethz.ch}
  {\texttt{nbeisert@itp.phys.ethz.ch}}}
\hypersetup{pdfauthor={Niklas Beisert}}
\hypersetup{pdfsubject={Manual for the LaTeX2e Package childdoc}}
\date{30 December 2018, \textsf{v2.0}}
\maketitle

\begin{abstract}\noindent
\textsf{childdoc} is a \LaTeXe{} package
that enables the direct compilation
of document sections included by |\include|
to individual files.
\end{abstract}

\begingroup
\parskip0ex
\tableofcontents
\endgroup

%%%%%%%%%%%%%%%%%%%%%%%%%%%%%%%%%%%%%%%%%%%%%%%%%%%%%%%%%%%%%%%%%%%%%%%%%%%%%%%%
%%%%%%%%%%%%%%%%%%%%%%%%%%%%%%%%%%%%%%%%%%%%%%%%%%%%%%%%%%%%%%%%%%%%%%%%%%%%%%%%
\section{Introduction}

\LaTeX{} provides a mechanism to structure a large document (such as a book)
into a main file and several child files (containing the chapters)
using the |\include| command.
This mechanism is beneficial for documents
which span hundreds of pages in order to
make the source file(s) more manageable.
Moreover, compilation can be restricted to
selected child files by means of the |\includeonly| command.
The latter feature can be used to reduce the compilation time while editing
(this was significantly more useful in the earlier days of \LaTeX{})
or to generate a smaller document which is easier to navigate.
Another application of |\includeonly| is to generate
documents consisting of selected parts of the complete document.

However, there are a few drawbacks of the plain |\include| mechanism:
\begin{itemize}
\item
The child files cannot be compiled on their own,
they can only be compiled via the main file.
A naive editing environment
(such as a text editor with an option
to have the current file processed by \LaTeX)
may require one to switch to the main file before compiling;
attempting to compile the child file produces errors.
\item
The main file must be modified (each time)
to adjust the |\includeonly| command
to the present needs. This easily leaves the main file in a messy state.
\item
The generated document will always carry the filename
of the main document. This is inconvenient if
several child files are to be compiled and
to be kept for distribution.
\end{itemize}

The present package provides a simple interface
to make child files individually compilable by \LaTeX{}.
Compiling a child file then has the same effect as compiling
the main file with an |\includeonly| command
to select the appropriate child.
Moreover the generated document will carry the name of the child
rather than the main file.
This resolves all three above issues.

This feature is meant to make the editing of books,
thesis documents and lecture notes somewhat more convenient.
However, the package can also be used efficiently for
composing a series of documents (such as exercise sheets)
which are typically distributed individually.
It then assists the author in generating the individual documents
(potentially in different versions)
as well as a document containing the collected series.
Another application is in developing style files
or other kinds of included material
where compilation of the style file could redirect
to a sample or test file.

%%%%%%%%%%%%%%%%%%%%%%%%%%%%%%%%%%%%%%%%%%%%%%%%%%%%%%%%%%%%%%%%%%%%%%%%%%%%%%%%
%%%%%%%%%%%%%%%%%%%%%%%%%%%%%%%%%%%%%%%%%%%%%%%%%%%%%%%%%%%%%%%%%%%%%%%%%%%%%%%%
\section{Usage}

First of all, the package \textsf{childdoc} is \emph{not} a standard
\LaTeXe{} |.sty| style file! Therefore it needs to be invoked in
a non-standard way.

%%%%%%%%%%%%%%%%%%%%%%%%%%%%%%%%%%%%%%%%%%%%%%%%%%%%%%%%%%%%%%%%%%%%%%%%%%%%%%%%
\subsection{Included Files}
\label{sec:include}

%%%%%%%%%%%%%%%%%%%%%%%%%%%%%%%%%%%%%%%%
\DescribeMacro{\childdocmain}
To use the package, add the commands
\begin{center}
\begin{tabular}{l}
|\input{childdoc.def}|\\
|\childdocmain{}|\\
\end{tabular}
\end{center}
at the very top of the main \LaTeX{} file,
in particular \emph{before} the |\documentclass| statement!
The argument of |\childdocmain| should be left empty
(but it must be present).

%%%%%%%%%%%%%%%%%%%%%%%%%%%%%%%%%%%%%%%%
\DescribeMacro{\childdocof}
Furthermore, add the commands
\begin{center}
\begin{tabular}{l}
|\input{childdoc.def}|\\
|\childdocof{|\textit{main}|}|\\
\end{tabular}
\end{center}
at the top of every child file \textit{child}
which is included by |\include{|\textit{child}|}|
from within the main file
(or at least for those files to be compiled individually).
The argument \textit{main} must be the filename of the main file.

There are a couple of
considerations in setting up the main and child documents:

%%%%%%%%%%%%%%%%%%%%%%%%%%%%%%%%%%%%%%%%
\paragraph{Restrictions.}

Please note the following restrictions:
\begin{itemize}
\item
|\childdocmain| must be called with one argument \textit{main}
to ensure compatibility with earlier version of the package.
It must either be empty (|\childdocmain{}|)
or precisely match the filename of the main file in which it is specified.
See \secref{sec:detection} for further information.
\item
The filename \textit{main} must be specified without the |.tex| extension.
\item
The filename \textit{main} is case sensitive
(even in case-insensitive file systems)
due to internal string comparison.
\item
The argument \textit{main} should be fully expanded, it cannot be a macro.
\item
Subdirectories and special characters should be avoided in filenames.
\item
The command |\childdocmain{|\textit{main}|}| must be followed by a whitespace.
It should not be followed immediately by another command
or by a comment mark `|%|'.
This is because the \TeX{} parser reads the token immediately following
the argument of |\childdocmain| and puts it
at the beginning of every child section;
however, a white\-space is ignored.
\end{itemize}

%%%%%%%%%%%%%%%%%%%%%%%%%%%%%%%%%%%%%%%%
\paragraph{Content of Main File.}

It is advisable to place all content in the child files included by |\include|.
Any output contained in the main file will appear in all child documents
unless suppressed manually;
it cannot be suppressed automatically by the |\includeonly| directive
and thus should normally be avoided.
A method to include some content in the main file
by means of conditional processing is described in \secref{sec:conditional}.

%%%%%%%%%%%%%%%%%%%%%%%%%%%%%%%%%%%%%%%%
\paragraph{Page Numbering.}

When only a part of the document is compiled,
the appropriate numbering of pages
(as well as other status parameters)
is determined from the |.aux| files.
The latter contain information from previous passes.
However this information needs to propagate through
all intermediate child documents.
Therefore the page numbering in child documents may well
be inconsistent until the complete document is compiled at least once.

A useful (if unconventional) way to always ensure a consistent
page numbering is to restart the numbering in each child document
and denote the pages by `\textit{child}|.|\textit{page}'
where \textit{child} represents the chapter/section number of the child file.
This can be achieved by the command
|\numberwithin{page}{|\textit{child}|}|
of the \textsf{amsmath} package
where \textit{child} can be |chapter| or |section|
depending on the chosen structuring.
Alternatively, one can modify the macro |\thepage| appropriately
and reset the counter |page| at the start of each child file.

%%%%%%%%%%%%%%%%%%%%%%%%%%%%%%%%%%%%%%%%%%%%%%%%%%%%%%%%%%%%%%%%%%%%%%%%%%%%%%%%
\subsection{Conditional Processing}
\label{sec:conditional}

The package provides a mechanism to compile different versions
of a document. To customise the versions further some conditional processing
can come in handy to distinguish which version is being compiled.
The package provides two macros to describe the compilation context:

%%%%%%%%%%%%%%%%%%%%%%%%%%%%%%%%%%%%%%%%
\DescribeMacro{\ifchilddoc}
The conditional |\ifchilddoc| distinguishes between the compilation of
child documents and the main document:
%
\begin{center}
|\ifchilddoc |\textit{child-code}| |[|\||else |\textit{main-code}]| \||fi|
\end{center}

%%%%%%%%%%%%%%%%%%%%%%%%%%%%%%%%%%%%%%%%
\DescribeMacro{\childdocname}
\DescribeMacro{\childdocjob}
The macro |\childdocname| contains the filename (without extension)
of the main or child file being processed.
Note that |\childdocjob| will always contain the name of the main file.

%%%%%%%%%%%%%%%%%%%%%%%%%%%%%%%%%%%%%%%%
\paragraph{Title Page.}

Conditional processing can be used to include a title or banner page
in the main document when proper precautions are taken.
Importantly, the code in the main file should ensure that the page counter
(as well as other status parameters which are stored in the |.aux| files)
takes the same value after the conditional processing.
Otherwise the page numbers may take divergent values
depending on which part is compiled.

For example, a title page could be declared by:
%
\begin{center}
\begin{tabular}{l}
|\ifchilddoc\||else|\\
|\addtocounter{page}{-1}|\\
\textit{code for title page}\\
|\newpage|\\
|\||fi|
\end{tabular}
\end{center}
%
A banner page for the child documents can be generated by:
%
\begin{center}
\begin{tabular}{l}
|\ifchilddoc|\\
|\addtocounter{page}{-1}|\\
\textit{code for banner page}\\
|\newpage|\\
|\||fi|
\end{tabular}
\end{center}
%
Here one could write a message such as:
\begin{center}
|This is the part \childdocname{} of \childdocjob{}.|
\end{center}

%%%%%%%%%%%%%%%%%%%%%%%%%%%%%%%%%%%%%%%%%%%%%%%%%%%%%%%%%%%%%%%%%%%%%%%%%%%%%%%%
\subsection{Flags}
\label{sec:flags}

The package makes it easy to generate different versions
of the main or child documents.
To this end compilation flags can be defined
and assigned different default values.
They will be particularly useful in conjunction
with the forwarding mechanism described in \secref{sec:forward}.

For example, it may be useful to have a flag |\version|
which can be set to |draft| or |final|.
The document source will contain some conditional code
depending on the value of |\version|.
Suppose further, the flag should default to |final| for the main file
and to |draft| for child files
which is a natural assignment for editing the document.
This is achieved by placing the following code
in the preamble of the main document
(below the |\childdocmain| directive):
%
\begin{center}
\begin{tabular}{l}
|\ifchilddoc|\\
|\providecommand{\version}{draft}|\\
|\||else|\\
|\providecommand{\version}{final}|\\
|\||fi|
\end{tabular}
\end{center}
%
The definition by |\providecommand| makes sure
that previous definitions are not overwritten.
Further statements |\providecommand{\version}{...}|
can thus be added before the above code to override it.

For the main file, one might add a line
(between |\childdocmain| and the above block)
%
\begin{center}
|%\ifchilddoc\||else\providecommand{\version}{draft}\||fi|
\end{center}
%
which can be uncommented to produce a draft version.
Likewise one can add a line to the very top of a child file
(above the |\childdocof{|\textit{main}|}| directive)
%
\begin{center}
|%\providecommand{\version}{final}|
\end{center}
%
which can be uncommented to produce the final version of this child document.

%%%%%%%%%%%%%%%%%%%%%%%%%%%%%%%%%%%%%%%%%%%%%%%%%%%%%%%%%%%%%%%%%%%%%%%%%%%%%%%%
\subsection{Forwarding}
\label{sec:forward}

Different versions of the main or child documents
using compilation flags as described in \secref{sec:flags}
can be (permanently) stored in different files
for convenient compilation, viewing and distribution.
To this end, the package defines a command
to pass on compilation to a different file:

%%%%%%%%%%%%%%%%%%%%%%%%%%%%%%%%%%%%%%%%
\DescribeMacro{\childdocforward}
The command |\childdocforward| redirects processing to
another source file:
%
\begin{center}
\begin{tabular}{l}
|\input{childdoc.def}|\\
|\childdocforward[|\textit{main}|]{|\textit{dest}|}|\\
\end{tabular}
\end{center}
%
The argument \textit{dest} is the destination file
(without extension).
It should be the main file or one of the child files.
Note that further \textsf{childdoc} directives
such as |\childdocof| and |\childdocforward|
in the indicated file will be processed in this form.
The optional argument \textit{main}
passes on directly to the main file \textit{main}
while pretending to compile the child \textit{dest}.
This form behaves as if \textit{dest}
issues |\childdocof{|\textit{main}|}| right away,
and no further \textsf{childdoc} directives will be processed.

%%%%%%%%%%%%%%%%%%%%%%%%%%%%%%%%%%%%%%%%
\DescribeMacro{\...prefix}
In the alternative form |\childdocforwardprefix|,
%
\begin{center}
\begin{tabular}{l}
|\input{childdoc.def}|\\
|\childdocforwardprefix[|\textit{main}|]{|\textit{prefix}|}{|\textit{dest}|}|
\end{tabular}
\end{center}
%
the destination file is determined by a pattern
depending on the current file:
To make this work, the current file must be called
`{\textit{prefix}\hspace{0.2em}\textit{suffix}}'
with \textit{prefix} matching precisely the argument.
Processing is then passed on to the file
`{\textit{dest}\hspace{0.2em}\textit{suffix}}'.
Surely, the same effect is achieved by
directly specifying the
argument `{\textit{dest}\hspace{0.2em}\textit{suffix}}'
in the first form.
However, that requires to set up a different file
for each child. With the alternative form of the command
all these files can have exactly the same content
which simplifies setting them up and maintaining them.

For example, the following file |draft.tex|
with a compilation flag |\version| as described in \secref{sec:flags}
compiles the main document as a draft:
%
\begin{center}
\begin{tabular}{l}
|\def\version{draft}|\\
|\input{childdoc.def}|\\
|\childdocforward{|\textit{main}|}|
\end{tabular}
\end{center}
%
Likewise, the following files |final|\textit{nn}|.tex|
compile the final version of the child document
|child|\textit{nn}|.tex|:
%
\begin{center}
\begin{tabular}{l}
|\def\version{final}|\\
|\input{childdoc.def}|\\
|\childdocforwardprefix{final}{child}|
\end{tabular}
\end{center}
%

Note that when several versions of a main file and/or of each child file
are to be generated, it may be convenient to set up a |Makefile| or
shell script to automatise the process.

%%%%%%%%%%%%%%%%%%%%%%%%%%%%%%%%%%%%%%%%%%%%%%%%%%%%%%%%%%%%%%%%%%%%%%%%%%%%%%%%
\subsection{Command Line Processing}
\label{sec:commandline}

The effect of redirection files can also be achieved by invoking
the \LaTeX{} compiler with a more elaborate command line.
Most conveniently this should be done as part
of a shell script or a |Makefile|.

When using \textsf{childdoc} in the main file, the following
command lines effectively perform a redirection
(note that depending on the shell being used,
backslashes may have to be doubled: `|\|' $\to$ `|\\|'):
%
\begin{center}
|... -jobname "|\textit{target}|" |\\|"|[\textit{flags}]%
|\input{childdoc.def}\childdocforward[|\textit{main}|]{|\textit{dest}|}"|
\end{center}
%
Here \textit{target} is the name of the output file,
\textit{main} is the name of the main file
and \textit{dest} is the name of the main or child file to be processed
(all filenames without extensions).
The optional argument \textit{main} can be omitted
if \textit{main} matches \textit{dest}.
Optionally, compilation \textit{flags} can be defined via |\def| commands.
This command line makes the \TeX{} engine believe
it is compiling the file \textit{target}
whose content is specified as the latter parameter.
The provided code then forwards the processing to
\textit{main} or \textit{dest} as described in \secref{sec:forward}.

%%%%%%%%%%%%%%%%%%%%%%%%%%%%%%%%%%%%%%%%%%%%%%%%%%%%%%%%%%%%%%%%%%%%%%%%%%%%%%%%
\subsection{Include by Input}
\label{sec:input}

Including child documents by |\include| has some restrictions by design.
Most notably, the content of a child document always occupies
its own set of pages; pages cannot be shared between child documents.
Usually, this behaviour makes perfect sense
because each child document contain an essential part of the document.
However, in some situations it may be desirable to compose
a document from a collection of parts
without having mandatory page breaks between then.
For this case, the package
provides a mechanism to include parts
by |\input| which can also be processed individually.
However, by construction this mechanism
requires manual handling of the content to be output.

%%%%%%%%%%%%%%%%%%%%%%%%%%%%%%%%%%%%%%%%
\DescribeMacro{\ifchilddocmanual}
The main file should be prepared as usual, see \secref{sec:include}.
However, the document body must make a distinction
between processing of an individual part and of the main document, e.g.:
%
\begin{center}
\begin{tabular}{l}
|\ifchilddocmanual|\\
|\input{\childdocname}|\\
|\||else|\\
\textit{document body with }|\input{|\textit{part}|}|\\
|\||fi|
\end{tabular}
\end{center}
%
The conditional |\ifchilddocmanual| is true whenever
a part to be included by |\input| is being compiled,
and the name of the part is stored in |\childdocname|.

%%%%%%%%%%%%%%%%%%%%%%%%%%%%%%%%%%%%%%%%
\DescribeMacro{\childdocby}
Each part to be included by |\input| should start with:
%
\begin{center}
\begin{tabular}{l}
|\input{childdoc.def}|\\
|\childdocby{|\textit{main}|}|\\
\end{tabular}
\end{center}
%
The directive |\childdocby| is similar to |\childdocof|
described in \secref{sec:include},
but the subsequent selection of content must be done manually.
To that end, both |\ifchilddoc| and |\ifchilddocmanual|
will be true upon processing of a part,
and the name of the part is stored in |\childdocname|.
Note that |\jobname| will be set to the filename of the current part
so that each part receives an individual |.aux| file
that does not interfere with the |.aux| file(s) of the main document.
This behaviour can be altered by the alternative form
|\childdocby[*]{|\textit{main}|}| (with a non-empty optional argument)
which uses the |.aux| file of the main document
by setting |\jobname| to \textit{main}.

%%%%%%%%%%%%%%%%%%%%%%%%%%%%%%%%%%%%%%%%%%%%%%%%%%%%%%%%%%%%%%%%%%%%%%%%%%%%%%%%
\subsection{Driver Development}
\label{sec:driver}

The \textsf{childdoc} mechanism can also be use for the development
of definition files such as \LaTeX{} styles or classes.
This case differs from the above setup with multiple parts
included by |\include| in that no |\includeonly| should be invoked.
This can be achieved by starting the include file
(before |\ProvidesPackage|) with:
%
\begin{center}
\begin{tabular}{l}
|\input{childdoc.def}|\\
|\childdocforward{|\textit{main}|}|\\
\end{tabular}
\end{center}
%
or alternatively with:
%
\begin{center}
\begin{tabular}{l}
|\input{childdoc.def}|\\
|\childdocby{|\textit{main}|}|\\
\end{tabular}
\end{center}
%
Both forms have slightly different effects as described above.
The main file is prepared as usual, see \secref{sec:include}.

%%%%%%%%%%%%%%%%%%%%%%%%%%%%%%%%%%%%%%%%%%%%%%%%%%%%%%%%%%%%%%%%%%%%%%%%%%%%%%%%
\subsection{Legacy Detection}
\label{sec:detection}

The directive |\childdocmain| in the main file can detect
whether the complete document or merely a child is to be compiled
even without using the directive |\childdocof|.
This method is deprecated because it is less robust
and there is no compelling reason to use it;
it is merely provided for backward compatibility
and it may be removed in future versions.

If the detection mechanism is to be used,
it is mandatory to correctly specify
the filename of the main file as the argument of |\childdocmain|:
%
\begin{center}
\begin{tabular}{l}
|\input{childdoc.def}|\\
|\childdocmain{|\textit{main}|}|\\
\end{tabular}
\end{center}
%
If |\jobname| does not match the argument \textit{main} of |\childdocmain|,
it is assumed that |\jobname| points to the child file to be compiled.
When using |\childdocmain| with the main file specified as argument,
it suffices to start a child file
with just |\input{|\textit{main}|}|
without loading of the package and using |\childdocof|.
If instead all processing is done
with the appropriate \textsf{childdoc} directives,
the argument of \textit{main} of |\childdocmain| can be empty.

An alternative version of the command line processing described
in \secref{sec:commandline} using the detection mechanism reads:
%
\begin{center}
|... -jobname "|\textit{target}|" "|[\textit{flags}]%
[|\def\jobname{|\textit{dest}|}|]|\input{|\textit{main}|}"|
\end{center}

%%%%%%%%%%%%%%%%%%%%%%%%%%%%%%%%%%%%%%%%%%%%%%%%%%%%%%%%%%%%%%%%%%%%%%%%%%%%%%%%
\subsection{Manual Code}
\label{sec:manual}

In case one cannot be certain whether the definitions file |childdoc.def|
is installed on the target \TeX{} distribution
and one prefers not to ship it,
it is conceivable to paste a few relevant commands into the sources.

To that end, drop all statements |\input{childdoc.def}|
and perform the replacements as outlined below.
Instead of |\childdocmain{|\textit{main}|}| add the following code
to the top of the main file:
%
\begin{center}
\begin{tabular}{l}
|\||ifdefined\childdocname\endinput\||fi\newif\ifchilddoc|\\
|\edef\childdocname{\scantokens\expandafter{\jobname\noexpand}}|\\
|\def\childdocmain{|\textit{main}|}\||ifx\childdocmain\childdocname\||else|\\
|\childdoctrue\includeonly{\childdocname}\let\jobname\childdocmain\||fi|\\
\end{tabular}
\end{center}
%
Instead of |\childdocof{|\textit{main}|}| just include the main file
at the top of each child file:
%
\begin{center}
|\input{|\textit{main}|}|
\end{center}
%
A simple redirection |\childdocforward{|\textit{dest}|}| is achieved by:
%
\begin{center}
|\def\jobname{|\textit{dest}|}\input{\jobname}|
\end{center}
%
The redirection with prefix
|\childdocforwardprefix[|\textit{prefix}|]{|\textit{dest}|}|
is accomplished by:
%
\begin{center}
\begin{tabular}{l}
|{\edef\jobname{\scantokens\expandafter{\jobname\noexpand}}|\\
|\def\redirectjob |\textit{prefix}|#1~~~{\gdef\jobname{|\textit{dest}|#1}}|\\
|\expandafter\redirectjob\jobname~~~}\input{\jobname}|
\end{tabular}
\end{center}

In an alternative approach,
child documents can be compiled by a specific command line
without additional code or specific definitions:
%
\begin{center}
|... -jobname "|\textit{target}|" "|[\textit{flags}]%
|\includeonly{|\textit{dest}|}\input{|\textit{main}|}"|
\end{center}
%

%%%%%%%%%%%%%%%%%%%%%%%%%%%%%%%%%%%%%%%%%%%%%%%%%%%%%%%%%%%%%%%%%%%%%%%%%%%%%%%%
%%%%%%%%%%%%%%%%%%%%%%%%%%%%%%%%%%%%%%%%%%%%%%%%%%%%%%%%%%%%%%%%%%%%%%%%%%%%%%%%
\section{Information}

%%%%%%%%%%%%%%%%%%%%%%%%%%%%%%%%%%%%%%%%%%%%%%%%%%%%%%%%%%%%%%%%%%%%%%%%%%%%%%%%
\subsection{Copyright}

Copyright \copyright{} 2017--2018 Niklas Beisert

This work may be distributed and/or modified under the
conditions of the \LaTeX{} Project Public License, either version 1.3
of this license or (at your option) any later version.
The latest version of this license is in
  \url{http://www.latex-project.org/lppl.txt}
and version 1.3 or later is part of all distributions of \LaTeX{}
version 2005/12/01 or later.

This work has the LPPL maintenance status `maintained'.

The Current Maintainer of this work is Niklas Beisert.

This work consists of the files |README.txt|, |childdoc.ins| and |childdoc.dtx|
as well as the derived files |childdoc.def|, |cdocsamp.tex|
with |cdocsch1.tex|, |cdocsch2.tex|, |cdocspt3.tex|, |cdocspt4.tex|,
|cdocsdrf.tex|, |cdocsfn1.tex|, |cdocsfn2.tex|
as well as |childdoc.pdf|.

%%%%%%%%%%%%%%%%%%%%%%%%%%%%%%%%%%%%%%%%%%%%%%%%%%%%%%%%%%%%%%%%%%%%%%%%%%%%%%%%
\subsection{Files and Installation}

The package consists of the files:
%
\begin{center}
\begin{tabular}{ll}
    |README.txt|   & readme file \\
    |childdoc.ins| & installation file \\
    |childdoc.dtx| & source file \\
    |childdoc.def| & definition file \\
    |cdocsamp.tex| & sample main file \\
    |cdocsch1.tex| & sample include file \\
    |cdocsch2.tex| & sample include file \\
    |cdocspt3.tex| & sample part file \\
    |cdocspt4.tex| & sample part file \\
    |cdocsdrf.tex| & sample redirection file \\
    |cdocsfn1.tex| & sample redirection file \\
    |cdocsfn2.tex| & sample redirection file \\
    |childdoc.pdf| & manual
\end{tabular}
\end{center}
%
The distribution consists of the files
|README.txt|, |childdoc.ins| and |childdoc.dtx|.
%
\begin{itemize}
\item
Run (pdf)\LaTeX{} on |childdoc.dtx|
to compile the manual |childdoc.pdf| (this file).
\item
Run \LaTeX{} on |childdoc.ins| to create the definitions file |childdoc.def|
and the sample |cdocsamp.tex| with include files
|cdocsch1.tex|, |cdocsch2.tex|, |cdocspt3.tex|, |cdocspt4.tex|,
|cdocsdrf.tex|, |cdocsfn1.tex|, |cdocsfn2.tex|.
Then copy the file |childdoc.def| to an appropriate directory of your \LaTeX{}
distribution, e.g.\ \textit{texmf-root}|/tex/latex/childdoc|.
\end{itemize}

%%%%%%%%%%%%%%%%%%%%%%%%%%%%%%%%%%%%%%%%%%%%%%%%%%%%%%%%%%%%%%%%%%%%%%%%%%%%%%%%
\subsection{Related CTAN Packages}

There are several other packages which offer a similar functionality:
%
\begin{itemize}
\item
The packages
\href{http://ctan.org/pkg/docmute}{\textsf{docmute}},
\href{http://ctan.org/pkg/includex}{\textsf{includex}} and
\href{http://ctan.org/pkg/standalone}{\textsf{standalone}}
provide commands to include only the document body of
a child file thus allowing both files to be compiled individually.
\item
The packages \href{http://ctan.org/pkg/subdocs}{\textsf{subdocs}}
and \href{http://ctan.org/pkg/subfiles}{\textsf{subfiles}}
provide structures in which the main and child documents can be
encapsulated and allowing them to be compiled individually.
The inclusion mechanism is different from the conventional |\include|.
\item
The package \href{http://ctan.org/pkg/combine}{\textsf{combine}}
is an elaborate solution to combine several documents into one.
\end{itemize}
%
See also the CTAN topic \href{http://ctan.org/topic/subdocs}{\textsf{subdocs}}
for further related packages.
The present package differs from the above solutions in that
a document structure constructed with the conventional |\include| mechanism
just needs two extra commands at the top of every file
such that all constituent files can be compiled individually.

%%%%%%%%%%%%%%%%%%%%%%%%%%%%%%%%%%%%%%%%%%%%%%%%%%%%%%%%%%%%%%%%%%%%%%%%%%%%%%%%
%\subsection{Feature Suggestions}
%
%The following is a list of features which may be useful for future
%versions of this package:
%%
%\begin{itemize}
%\item
%\ldots
%\end{itemize}

%%%%%%%%%%%%%%%%%%%%%%%%%%%%%%%%%%%%%%%%%%%%%%%%%%%%%%%%%%%%%%%%%%%%%%%%%%%%%%%%
\subsection{Revision History}

%%%%%%%%%%%%%%%%%%%%%%%%%%%%%%%%%%%%%%%%
\paragraph{v2.0:} 2018/12/30

\begin{itemize}
\item
immediate forward processing
\item
added |\childdocby| mechanism
\item
manual restructured
\end{itemize}

%%%%%%%%%%%%%%%%%%%%%%%%%%%%%%%%%%%%%%%%
\paragraph{v1.6:} 2018/01/17

\begin{itemize}
\item
application for development of include files
\item
corrections to manual
\end{itemize}

%%%%%%%%%%%%%%%%%%%%%%%%%%%%%%%%%%%%%%%%
\paragraph{v1.5:} 2017/05/21

\begin{itemize}
\item
more complete structuring introduced
\item
|\childdocof| introduced
\item
|\childdoc| renamed to |\childdocmain|
\item
|\childredirect| renamed to |\childdocforward| and |\childdocforwardprefix|
and functionality expanded
\end{itemize}

%%%%%%%%%%%%%%%%%%%%%%%%%%%%%%%%%%%%%%%%
\paragraph{v1.0:} 2017/04/27

\begin{itemize}
\item
manual and install package
\item
first version published on CTAN
\end{itemize}

%%%%%%%%%%%%%%%%%%%%%%%%%%%%%%%%%%%%%%%%
\paragraph{v0.6:} 2017/04/26

\begin{itemize}
\item
redirection mechanism added
\end{itemize}

%%%%%%%%%%%%%%%%%%%%%%%%%%%%%%%%%%%%%%%%
\paragraph{v0.5:} 2017/04/26

\begin{itemize}
\item
functionality in definition file
\end{itemize}


%%%%%%%%%%%%%%%%%%%%%%%%%%%%%%%%%%%%%%%%%%%%%%%%%%%%%%%%%%%%%%%%%%%%%%%%%%%%%%%%
%%%%%%%%%%%%%%%%%%%%%%%%%%%%%%%%%%%%%%%%%%%%%%%%%%%%%%%%%%%%%%%%%%%%%%%%%%%%%%%%
%%%%%%%%%%%%%%%%%%%%%%%%%%%%%%%%%%%%%%%%%%%%%%%%%%%%%%%%%%%%%%%%%%%%%%%%%%%%%%%%
\appendix

\settowidth\MacroIndent{\rmfamily\scriptsize 000\ }

 \DocInput{childdoc.dtx}

\end{document}
%</driver>
% \fi
%
% %%%%%%%%%%%%%%%%%%%%%%%%%%%%%%%%%%%%%%%%%%%%%%%%%%%%%%%%%%%%%%%%%%%%%%%%%%%%%%
% %%%%%%%%%%%%%%%%%%%%%%%%%%%%%%%%%%%%%%%%%%%%%%%%%%%%%%%%%%%%%%%%%%%%%%%%%%%%%%
% \section{Sample}
%\iffalse
%<*samplemain>
%\fi
%
% The following presents a sample document
% with two chapters, two parts, a title page,
% a compile flag as well as three forwarding files to set the flag.
% It consists of eight |.tex| files:
% \begin{center}
% \begin{tabular}{ll}
% |cdocsamp.tex|&main file\\
% |cdocsch1.tex|&include file for chapter 1\\
% |cdocsch2.tex|&include file for chapter 2\\
% |cdocspt3.tex|&include file for part 3\\
% |cdocspt4.tex|&include file for part 4\\
% |cdocsdrf.tex|&forwarding file for main file in draft mode\\
% |cdocsfi1.tex|&forwarding file for final version of chapter 1\\
% |cdocsfi2.tex|&forwarding file for final version of chapter 2\\
% \end{tabular}
% \end{center}
% Each of the eight files can be compiled directly by the \LaTeX{} compiler.
%
% %%%%%%%%%%%%%%%%%%%%%%%%%%%%%%%%%%%%%%
% \paragraph{Main File.}
%
% The main file is called |cdocsamp.tex|.
%
% Load the \textsf{childdoc} definitions and
% declare the filename for the main document:
%    \begin{macrocode}
\input{childdoc.def}
\childdocmain{}
%    \end{macrocode}

% Optional override for |\version| flag:
%    \begin{macrocode}
%%\ifchilddoc\else\providecommand{\version}{draft}\fi
%    \end{macrocode}

% Define the default values for the |\version| flag
% (|final| for the main file and |draft| for childs):
%    \begin{macrocode}
\ifchilddoc
\providecommand{\version}{draft}
\else
\providecommand{\version}{final}
\fi
%    \end{macrocode}

% Load the standard document class:
%    \begin{macrocode}
\documentclass[12pt]{article}
%    \end{macrocode}

% Start the document body:
%    \begin{macrocode}
\begin{document}
%    \end{macrocode}

% Declare a title page.
% Print title, part of document being processed and version flag:
%    \begin{macrocode}
\addtocounter{page}{-1}
\begin{center}
{\LARGE\bfseries{}childdoc example\par}
\vspace{1cm}
\ifchilddoc
\ifchilddocmanual part\else chapter\fi:
`\childdocname' of `\childdocjob'\par
\else
main document: `\childdocjob'\par
\fi
version: \version\par
\end{center}
\newpage
%    \end{macrocode}

% Manually include selected file,
% otherwise process as usual:
%    \begin{macrocode}
\ifchilddocmanual
\section*{part `\childdocname'}
\input{\childdocname}
\else
%    \end{macrocode}

% Include the two chapters:
%    \begin{macrocode}
\include{cdocsch1}
\include{cdocsch2}
%    \end{macrocode}

% Include the two parts unless only chapters should be displayed:
%    \begin{macrocode}
\ifchilddoc\else
\section{part three}
\input{cdocspt3}
\section{part four}
\input{cdocspt4}
\fi
%    \end{macrocode}

% Process as usual until here:
%    \begin{macrocode}
\fi
%    \end{macrocode}

% End of document body:
%    \begin{macrocode}
\end{document}
%    \end{macrocode}
%\iffalse
%</samplemain>
%\fi
%
% %%%%%%%%%%%%%%%%%%%%%%%%%%%%%%%%%%%%%%
% \paragraph{Chapter Include Files.}
%
% The include files are called |cdocsch1.tex| and |cdocsch2.tex|.
%
%\iffalse
%<*samplechap1|samplechap2>
%\fi

% Optional override for |\version| flag:
%    \begin{macrocode}
%%\providecommand{\version}{final}
%    \end{macrocode}

% Include the main document:
%    \begin{macrocode}
\input{childdoc.def}
\childdocof{cdocsamp}
%    \end{macrocode}

%\iffalse
%</samplechap1|samplechap2>
%\fi
%
%\iffalse
%<*samplechap1>
%\fi
% Some text for chapter 1:
%    \begin{macrocode}
\section{one}
some text in chapter one
%    \end{macrocode}

%\iffalse
%</samplechap1>
%\fi
% Some text for chapter 2:
%\iffalse
%<*samplechap2>
%\fi
%    \begin{macrocode}
\section{two}
more text in chapter two
%    \end{macrocode}

%\iffalse
%</samplechap2>
%\fi
%
% %%%%%%%%%%%%%%%%%%%%%%%%%%%%%%%%%%%%%%
% \paragraph{Part Include Files.}
%
% The include files are called |cdocspt3.tex| and |cdocspt4.tex|.
%
%\iffalse
%<*samplepart3|samplepart4>
%\fi

% Optional override for |\version| flag:
%    \begin{macrocode}
%%\providecommand{\version}{final}
%    \end{macrocode}

% Include the main document:
%    \begin{macrocode}
\input{childdoc.def}
\childdocby{cdocsamp}
%    \end{macrocode}

%\iffalse
%</samplepart3|samplepart4>
%\fi
%
%\iffalse
%<*samplepart3>
%\fi
% Some text for part 3:
%    \begin{macrocode}
some text in part three
%    \end{macrocode}

%\iffalse
%</samplepart3>
%\fi
% Some text for part 4:
%\iffalse
%<*samplepart4>
%\fi
%    \begin{macrocode}
more text in part four
%    \end{macrocode}

%\iffalse
%</samplepart4>
%\fi
%
% %%%%%%%%%%%%%%%%%%%%%%%%%%%%%%%%%%%%%%
% \paragraph{Forwarding for a Complete Draft.}
%
% The following forwarding file |cdocsdrf.tex|
% compiles the main document in draft mode:
%\iffalse
%<*sampledraft>
%\fi
%    \begin{macrocode}
\def\version{draft}
\input{childdoc.def}
\childdocforward{cdocsamp}
%    \end{macrocode}

%\iffalse
%</sampledraft>
%\fi
%
% %%%%%%%%%%%%%%%%%%%%%%%%%%%%%%%%%%%%%%
% \paragraph{Forwarding for Final Version of the Chapters.}
%
% The following forwarding files |cdocsfn1.tex| and |cdocsfn2.tex|
% (with identical content)
% compile the final versions of the child documents
% |cdocsch1.tex| and |cdocsch2.tex|, respectively:
%\iffalse
%<*samplefinal>
%\fi
%    \begin{macrocode}
\def\version{final}
\input{childdoc.def}
\childdocforwardprefix[cdocsamp]{cdocsfn}{cdocsch}
%    \end{macrocode}

%\iffalse
%</samplefinal>
%\fi
%
% %%%%%%%%%%%%%%%%%%%%%%%%%%%%%%%%%%%%%%
% \paragraph{Command Line Processing.}
%
% The following three command lines generate the output files
% |cdocscld|, |cdocscl1| and |cdocscl2|
% which should be identical to
% |cdocsdrf|, |cdocsch1| and |cdocsfn2|, respectively:
% \begin{center}
% \begin{tabular}{l}
% |latex -jobname cdocscld \|\\
% |  "\def\version{draft}\input{childdoc.def}\childdocforward{cdocsamp}"|\\
% |latex -jobname cdocscl1 \|\\
% |  "\input{childdoc.def}\childdocforward[cdocsamp]{cdocsch1}"|\\
% |latex -jobname cdocscl2 \|\\
% |  "\def\version{final}\input{childdoc.def}\childdocforward{cdocsch2}"|
% \end{tabular}
% \end{center}
% Note that the trailing backslash on each first line
% merely continues the input to the second line
% (for convenient cut ant paste).
% Furthermore, the command |latex| can be replaced by any
% of its alternative versions such as |pdflatex|.
%
% %%%%%%%%%%%%%%%%%%%%%%%%%%%%%%%%%%%%%%%%%%%%%%%%%%%%%%%%%%%%%%%%%%%%%%%%%%%%%%
% %%%%%%%%%%%%%%%%%%%%%%%%%%%%%%%%%%%%%%%%%%%%%%%%%%%%%%%%%%%%%%%%%%%%%%%%%%%%%%
% \section{Implementation}
%\iffalse
%<*package>
%\fi
%
% This section describes the definitions file |childdoc.def|.

% The definitions cannot be loaded using |\usepackage| or |\RequirePackage|
% which has a mechanism to prevent loading a style file more than once.
% When loading the definitions by means of |\input|
% multiple instances have to be prevented manually:
%\iffalse
%This code needs to be before the `\ProvidesFile' directive
%which is defined at the beginning of this file.
%Therefore it is also placed there and commented out here.
%</package>
%<*discard>
%\fi
%    \begin{macrocode}
\ifdefined\childdocmain\endinput\fi
%    \end{macrocode}
%\iffalse
%</discard>
%<*package>
%\fi
%
% \macro{\ifchilddoc}
% \macro{\ifchilddocmanual}
% The conditional |\ifchilddoc| tells whether a
% child (true) or main (false) document is being compiled.
% The conditional |\ifchilddocmanual| tells whether
% the |\includeonly| mechanism is used (false) or
% the selection of child files must be performed manually (true).
% The definitions initialise to false:
%    \begin{macrocode}
\newif\ifchilddoc
\newif\ifchilddocmanual
%    \end{macrocode}

% \macro{\childdocname}
% \macro{\childdocjob}
% The macro |\childdocname| stores the name of the main document
% to be compiled. The macro |\childdocjob| stores the name of
% the document on which the \LaTeX{} compiler was originally invoked.
% The content of |\jobname| cannot be compared
% to filenames specified in the source due to different catcodes.
% The following code rescans |\jobname|, stores the result
% in |\childdocname| and saves a copy in |\childdocjob|:
%    \begin{macrocode}
\edef\childdocname{\scantokens\expandafter{\jobname\noexpand}}
\let\childdocjob\childdocname
%    \end{macrocode}

% \macro{\childdocdisable}
% The macro |\childdocdisable| prevents the main file
% from being processed more than once.
% At this stage, the main document command |\childdocmain|
% is assumed to be called once again where it should do nothing.
% Any subsequent call to it should prevent
% a secondary processing of the main document
% It overwrites the forwarding commands
% |\childdocof| and |\childdocforward|
% with empty macros to prevent further inclusions of the main document:
%    \begin{macrocode}
\newcommand{\childdocdisable}
{
  \renewcommand{\childdocmain}[1]{\renewcommand{\childdocmain}[1]{\endinput}}
  \renewcommand{\childdocof}[1]{}
  \renewcommand{\childdocby}[2][]{}
  \renewcommand{\childdocforward}[2][]{}
  \renewcommand{\childdocdisable}{}
}
%    \end{macrocode}

% \macro{\childdocmain}
% The macro |\childdocmain| is to be called at the top of the main file
% with nothing or the main filename (without extension) as argument.
% First, it breaks loops.
% If the argument is not empty and does not match |\childdocname|
% (which is set by the first inclusion of |childdoc.def|),
% |\ifchilddoc| is set to true, |\includeonly| is applied to the child file
% and |\jobname| is set to the main file
% (for proper handling of |.aux| files):
%    \begin{macrocode}
\newcommand{\childdocmain}[1]
{
  \childdocdisable\childdocmain{}
  \if?#1?\else
    \begingroup
      \def\childdoctmp{#1}
      \ifx\childdoctmp\childdocname
        \def\childdoctmp{}
      \else
        \def\childdoctmp
        {
          \childdoctrue
          \includeonly{\childdocname}
          \def\childdocjob{#1}
          \def\jobname{#1}
        }
      \fi
      \expandafter
    \endgroup
    \childdoctmp
  \fi
}
%    \end{macrocode}

% \macro{\childdocof}
% The command |\childdocof| redirects
% compilation to the main file |#1|.
%    \begin{macrocode}
\newcommand{\childdocof}[1]
{
  \childdocdisable
  \childdoctrue
  \includeonly{\childdocname}
  \def\jobname{#1}
  \def\childdocjob{#1}
  \input{#1}
}
%    \end{macrocode}

% \macro{\childdocby}
% The command |\childdocby| ....
%    \begin{macrocode}
\newcommand{\childdocby}[2][]
{
  \childdocdisable
  \childdoctrue
  \childdocmanualtrue
  \if?#1?\else
    \def\jobname{#2}
  \fi
  \def\childdocjob{#2}
  \input{#2}
  \endinput
}
%    \end{macrocode}

% \macro{\childdocforward}
% The command |\childdocforward| redirects
% compilation to the main file or
% (if the optional argument is given) a child file.
% Parameters are set as if the main file
% or a child file starting with |\childdocof| was compiled.
% Then compilation is handed over to the main file:
%    \begin{macrocode}
\newcommand{\childdocforward}[2][]
{
  \begingroup
    \if?#1?
      \def\childdoctmp
      {
        \def\childdocname{#2}
        \def\childdocjob{#2}
        \def\jobname{#2}
        \input{#2}
        \endinput
      }
    \else
      \def\childdoctmp
      {
        \childdocdisable
        \def\childdocname{#2}
        \childdoctrue
        \includeonly{#2}
        \def\childdocjob{#1}
        \def\jobname{#1}
        \input{#1}
        \endinput
      }
    \fi
    \expandafter
  \endgroup
  \childdoctmp
}
%    \end{macrocode}

% \macro{\childdocforwardprefix}
% The command |\childdocforwardprefix| redirects
% compilation to the main or a child file by means of a pattern.
% The prefix |#1| in the current filename is replaced by |#2|
% and the suffix of the current filename is kept
% (it is assumed that the filename does not contain the substring `|~~~|'
% which is used as a delimiter).
% Compilation is handed over to the new file by |\childdocforward|:
%    \begin{macrocode}
\newcommand{\childdocforwardprefix}[3][]
{
  \begingroup
    \def\childdocextract #2##1~~~{\def\childdoctmp{\childdocforward[#1]{#3##1}}}
    \expandafter\childdocextract\childdocname~~~
    \expandafter
  \endgroup
  \childdoctmp
}
%    \end{macrocode}

% \macro{\childdoc}
% The deprecated macro |\childdoc| is a legacy version of |\childdocmain|:
%    \begin{macrocode}
\newcommand{\childdoc}{\childdocmain}
%    \end{macrocode}

% \macro{\childdocredirect}
% The deprecated macro |\childdocredirect| is a legacy version
% of |\childdocforward| and |\childdocforwardprefix|:
%    \begin{macrocode}
\newcommand{\childdocredirect}[2][]
{
  \begingroup
    \if?#1?
      \def\childdoctmp{\childdocforward{#2}}
    \else
      \def\childdoctmp{\childdocforwardprefix{#1}{#2}}
    \fi
    \expandafter
  \endgroup
  \childdoctmp
}
%    \end{macrocode}

%\iffalse
%</package>
%\fi
%
\endinput
\childdocforward{cdocsamp}"|\\
% |latex -jobname cdocscl1 \|\\
% |  "% \iffalse
%
% childdoc.dtx Copyright (C) 2017-2018 Niklas Beisert
%
% This work may be distributed and/or modified under the
% conditions of the LaTeX Project Public License, either version 1.3
% of this license or (at your option) any later version.
% The latest version of this license is in
%   http://www.latex-project.org/lppl.txt
% and version 1.3 or later is part of all distributions of LaTeX
% version 2005/12/01 or later.
%
% This work has the LPPL maintenance status `maintained'.
%
% The Current Maintainer of this work is Niklas Beisert.
%
% This work consists of the files childdoc.dtx and childdoc.ins
% and the derived files childdoc.def and cdocsamp.tex with
% cdocsch1.tex, cdocsch2.tex, cdocsdrf.tex, cdocsfn1.tex, cdocsfn2.tex.
%
%<package>\ifdefined\childdocmain\endinput\fi
%<package>\ProvidesFile{childdoc.def}[2018/12/30 v2.0 child document driver]
%<samplemain>\ProvidesFile{cdocsamp.tex}[2018/12/30 v2.0 sample for childdoc]
%<*driver>
%\ProvidesFile{childdoc.drv}[2018/12/30 v2.0 childdoc reference manual file]
\PassOptionsToClass{10pt,a4paper}{article}
\documentclass{ltxdoc}

\usepackage[margin=35mm]{geometry}
\usepackage{hyperref}
\usepackage{hyperxmp}
\usepackage[usenames]{color}

\hypersetup{colorlinks=true}
\hypersetup{pdfstartview=FitH}
\hypersetup{pdfpagemode=UseNone}
\hypersetup{pdfsource={}}
\hypersetup{pdflang={en-UK}}
\hypersetup{pdfcopyright={Copyright 2017-2018 Niklas Beisert.
  This work may be distributed and/or modified under the
  conditions of the LaTeX Project Public License, either version 1.3
  of this license or (at your option) any later version.}}
\hypersetup{pdflicenseurl={http://www.latex-project.org/lppl.txt}}
\hypersetup{pdfcontactaddress={ETH Zurich, ITP, HIT K,
  Wolfgang-Pauli-Strasse 27}}
\hypersetup{pdfcontactpostcode={8093}}
\hypersetup{pdfcontactcity={Zurich}}
\hypersetup{pdfcontactcountry={Switzerland}}
\hypersetup{pdfcontactemail={nbeisert@itp.phys.ethz.ch}}
\hypersetup{pdfcontacturl={http://people.phys.ethz.ch/\xmptilde nbeisert/}}

\newcommand{\secref}[1]{\hyperref[#1]{section \ref*{#1}}}

\parskip1ex
\parindent0pt
\let\olditemize\itemize
\def\itemize{\olditemize\parskip0pt}

\begin{document}

\title{The \textsf{childdoc} Package}
\hypersetup{pdftitle={The childdoc Package}}
\author{Niklas Beisert\\[2ex]
  Institut f\"ur Theoretische Physik\\
  Eidgen\"ossische Technische Hochschule Z\"urich\\
  Wolfgang-Pauli-Strasse 27, 8093 Z\"urich, Switzerland\\[1ex]
  \href{mailto:nbeisert@itp.phys.ethz.ch}
  {\texttt{nbeisert@itp.phys.ethz.ch}}}
\hypersetup{pdfauthor={Niklas Beisert}}
\hypersetup{pdfsubject={Manual for the LaTeX2e Package childdoc}}
\date{30 December 2018, \textsf{v2.0}}
\maketitle

\begin{abstract}\noindent
\textsf{childdoc} is a \LaTeXe{} package
that enables the direct compilation
of document sections included by |\include|
to individual files.
\end{abstract}

\begingroup
\parskip0ex
\tableofcontents
\endgroup

%%%%%%%%%%%%%%%%%%%%%%%%%%%%%%%%%%%%%%%%%%%%%%%%%%%%%%%%%%%%%%%%%%%%%%%%%%%%%%%%
%%%%%%%%%%%%%%%%%%%%%%%%%%%%%%%%%%%%%%%%%%%%%%%%%%%%%%%%%%%%%%%%%%%%%%%%%%%%%%%%
\section{Introduction}

\LaTeX{} provides a mechanism to structure a large document (such as a book)
into a main file and several child files (containing the chapters)
using the |\include| command.
This mechanism is beneficial for documents
which span hundreds of pages in order to
make the source file(s) more manageable.
Moreover, compilation can be restricted to
selected child files by means of the |\includeonly| command.
The latter feature can be used to reduce the compilation time while editing
(this was significantly more useful in the earlier days of \LaTeX{})
or to generate a smaller document which is easier to navigate.
Another application of |\includeonly| is to generate
documents consisting of selected parts of the complete document.

However, there are a few drawbacks of the plain |\include| mechanism:
\begin{itemize}
\item
The child files cannot be compiled on their own,
they can only be compiled via the main file.
A naive editing environment
(such as a text editor with an option
to have the current file processed by \LaTeX)
may require one to switch to the main file before compiling;
attempting to compile the child file produces errors.
\item
The main file must be modified (each time)
to adjust the |\includeonly| command
to the present needs. This easily leaves the main file in a messy state.
\item
The generated document will always carry the filename
of the main document. This is inconvenient if
several child files are to be compiled and
to be kept for distribution.
\end{itemize}

The present package provides a simple interface
to make child files individually compilable by \LaTeX{}.
Compiling a child file then has the same effect as compiling
the main file with an |\includeonly| command
to select the appropriate child.
Moreover the generated document will carry the name of the child
rather than the main file.
This resolves all three above issues.

This feature is meant to make the editing of books,
thesis documents and lecture notes somewhat more convenient.
However, the package can also be used efficiently for
composing a series of documents (such as exercise sheets)
which are typically distributed individually.
It then assists the author in generating the individual documents
(potentially in different versions)
as well as a document containing the collected series.
Another application is in developing style files
or other kinds of included material
where compilation of the style file could redirect
to a sample or test file.

%%%%%%%%%%%%%%%%%%%%%%%%%%%%%%%%%%%%%%%%%%%%%%%%%%%%%%%%%%%%%%%%%%%%%%%%%%%%%%%%
%%%%%%%%%%%%%%%%%%%%%%%%%%%%%%%%%%%%%%%%%%%%%%%%%%%%%%%%%%%%%%%%%%%%%%%%%%%%%%%%
\section{Usage}

First of all, the package \textsf{childdoc} is \emph{not} a standard
\LaTeXe{} |.sty| style file! Therefore it needs to be invoked in
a non-standard way.

%%%%%%%%%%%%%%%%%%%%%%%%%%%%%%%%%%%%%%%%%%%%%%%%%%%%%%%%%%%%%%%%%%%%%%%%%%%%%%%%
\subsection{Included Files}
\label{sec:include}

%%%%%%%%%%%%%%%%%%%%%%%%%%%%%%%%%%%%%%%%
\DescribeMacro{\childdocmain}
To use the package, add the commands
\begin{center}
\begin{tabular}{l}
|\input{childdoc.def}|\\
|\childdocmain{}|\\
\end{tabular}
\end{center}
at the very top of the main \LaTeX{} file,
in particular \emph{before} the |\documentclass| statement!
The argument of |\childdocmain| should be left empty
(but it must be present).

%%%%%%%%%%%%%%%%%%%%%%%%%%%%%%%%%%%%%%%%
\DescribeMacro{\childdocof}
Furthermore, add the commands
\begin{center}
\begin{tabular}{l}
|\input{childdoc.def}|\\
|\childdocof{|\textit{main}|}|\\
\end{tabular}
\end{center}
at the top of every child file \textit{child}
which is included by |\include{|\textit{child}|}|
from within the main file
(or at least for those files to be compiled individually).
The argument \textit{main} must be the filename of the main file.

There are a couple of
considerations in setting up the main and child documents:

%%%%%%%%%%%%%%%%%%%%%%%%%%%%%%%%%%%%%%%%
\paragraph{Restrictions.}

Please note the following restrictions:
\begin{itemize}
\item
|\childdocmain| must be called with one argument \textit{main}
to ensure compatibility with earlier version of the package.
It must either be empty (|\childdocmain{}|)
or precisely match the filename of the main file in which it is specified.
See \secref{sec:detection} for further information.
\item
The filename \textit{main} must be specified without the |.tex| extension.
\item
The filename \textit{main} is case sensitive
(even in case-insensitive file systems)
due to internal string comparison.
\item
The argument \textit{main} should be fully expanded, it cannot be a macro.
\item
Subdirectories and special characters should be avoided in filenames.
\item
The command |\childdocmain{|\textit{main}|}| must be followed by a whitespace.
It should not be followed immediately by another command
or by a comment mark `|%|'.
This is because the \TeX{} parser reads the token immediately following
the argument of |\childdocmain| and puts it
at the beginning of every child section;
however, a white\-space is ignored.
\end{itemize}

%%%%%%%%%%%%%%%%%%%%%%%%%%%%%%%%%%%%%%%%
\paragraph{Content of Main File.}

It is advisable to place all content in the child files included by |\include|.
Any output contained in the main file will appear in all child documents
unless suppressed manually;
it cannot be suppressed automatically by the |\includeonly| directive
and thus should normally be avoided.
A method to include some content in the main file
by means of conditional processing is described in \secref{sec:conditional}.

%%%%%%%%%%%%%%%%%%%%%%%%%%%%%%%%%%%%%%%%
\paragraph{Page Numbering.}

When only a part of the document is compiled,
the appropriate numbering of pages
(as well as other status parameters)
is determined from the |.aux| files.
The latter contain information from previous passes.
However this information needs to propagate through
all intermediate child documents.
Therefore the page numbering in child documents may well
be inconsistent until the complete document is compiled at least once.

A useful (if unconventional) way to always ensure a consistent
page numbering is to restart the numbering in each child document
and denote the pages by `\textit{child}|.|\textit{page}'
where \textit{child} represents the chapter/section number of the child file.
This can be achieved by the command
|\numberwithin{page}{|\textit{child}|}|
of the \textsf{amsmath} package
where \textit{child} can be |chapter| or |section|
depending on the chosen structuring.
Alternatively, one can modify the macro |\thepage| appropriately
and reset the counter |page| at the start of each child file.

%%%%%%%%%%%%%%%%%%%%%%%%%%%%%%%%%%%%%%%%%%%%%%%%%%%%%%%%%%%%%%%%%%%%%%%%%%%%%%%%
\subsection{Conditional Processing}
\label{sec:conditional}

The package provides a mechanism to compile different versions
of a document. To customise the versions further some conditional processing
can come in handy to distinguish which version is being compiled.
The package provides two macros to describe the compilation context:

%%%%%%%%%%%%%%%%%%%%%%%%%%%%%%%%%%%%%%%%
\DescribeMacro{\ifchilddoc}
The conditional |\ifchilddoc| distinguishes between the compilation of
child documents and the main document:
%
\begin{center}
|\ifchilddoc |\textit{child-code}| |[|\||else |\textit{main-code}]| \||fi|
\end{center}

%%%%%%%%%%%%%%%%%%%%%%%%%%%%%%%%%%%%%%%%
\DescribeMacro{\childdocname}
\DescribeMacro{\childdocjob}
The macro |\childdocname| contains the filename (without extension)
of the main or child file being processed.
Note that |\childdocjob| will always contain the name of the main file.

%%%%%%%%%%%%%%%%%%%%%%%%%%%%%%%%%%%%%%%%
\paragraph{Title Page.}

Conditional processing can be used to include a title or banner page
in the main document when proper precautions are taken.
Importantly, the code in the main file should ensure that the page counter
(as well as other status parameters which are stored in the |.aux| files)
takes the same value after the conditional processing.
Otherwise the page numbers may take divergent values
depending on which part is compiled.

For example, a title page could be declared by:
%
\begin{center}
\begin{tabular}{l}
|\ifchilddoc\||else|\\
|\addtocounter{page}{-1}|\\
\textit{code for title page}\\
|\newpage|\\
|\||fi|
\end{tabular}
\end{center}
%
A banner page for the child documents can be generated by:
%
\begin{center}
\begin{tabular}{l}
|\ifchilddoc|\\
|\addtocounter{page}{-1}|\\
\textit{code for banner page}\\
|\newpage|\\
|\||fi|
\end{tabular}
\end{center}
%
Here one could write a message such as:
\begin{center}
|This is the part \childdocname{} of \childdocjob{}.|
\end{center}

%%%%%%%%%%%%%%%%%%%%%%%%%%%%%%%%%%%%%%%%%%%%%%%%%%%%%%%%%%%%%%%%%%%%%%%%%%%%%%%%
\subsection{Flags}
\label{sec:flags}

The package makes it easy to generate different versions
of the main or child documents.
To this end compilation flags can be defined
and assigned different default values.
They will be particularly useful in conjunction
with the forwarding mechanism described in \secref{sec:forward}.

For example, it may be useful to have a flag |\version|
which can be set to |draft| or |final|.
The document source will contain some conditional code
depending on the value of |\version|.
Suppose further, the flag should default to |final| for the main file
and to |draft| for child files
which is a natural assignment for editing the document.
This is achieved by placing the following code
in the preamble of the main document
(below the |\childdocmain| directive):
%
\begin{center}
\begin{tabular}{l}
|\ifchilddoc|\\
|\providecommand{\version}{draft}|\\
|\||else|\\
|\providecommand{\version}{final}|\\
|\||fi|
\end{tabular}
\end{center}
%
The definition by |\providecommand| makes sure
that previous definitions are not overwritten.
Further statements |\providecommand{\version}{...}|
can thus be added before the above code to override it.

For the main file, one might add a line
(between |\childdocmain| and the above block)
%
\begin{center}
|%\ifchilddoc\||else\providecommand{\version}{draft}\||fi|
\end{center}
%
which can be uncommented to produce a draft version.
Likewise one can add a line to the very top of a child file
(above the |\childdocof{|\textit{main}|}| directive)
%
\begin{center}
|%\providecommand{\version}{final}|
\end{center}
%
which can be uncommented to produce the final version of this child document.

%%%%%%%%%%%%%%%%%%%%%%%%%%%%%%%%%%%%%%%%%%%%%%%%%%%%%%%%%%%%%%%%%%%%%%%%%%%%%%%%
\subsection{Forwarding}
\label{sec:forward}

Different versions of the main or child documents
using compilation flags as described in \secref{sec:flags}
can be (permanently) stored in different files
for convenient compilation, viewing and distribution.
To this end, the package defines a command
to pass on compilation to a different file:

%%%%%%%%%%%%%%%%%%%%%%%%%%%%%%%%%%%%%%%%
\DescribeMacro{\childdocforward}
The command |\childdocforward| redirects processing to
another source file:
%
\begin{center}
\begin{tabular}{l}
|\input{childdoc.def}|\\
|\childdocforward[|\textit{main}|]{|\textit{dest}|}|\\
\end{tabular}
\end{center}
%
The argument \textit{dest} is the destination file
(without extension).
It should be the main file or one of the child files.
Note that further \textsf{childdoc} directives
such as |\childdocof| and |\childdocforward|
in the indicated file will be processed in this form.
The optional argument \textit{main}
passes on directly to the main file \textit{main}
while pretending to compile the child \textit{dest}.
This form behaves as if \textit{dest}
issues |\childdocof{|\textit{main}|}| right away,
and no further \textsf{childdoc} directives will be processed.

%%%%%%%%%%%%%%%%%%%%%%%%%%%%%%%%%%%%%%%%
\DescribeMacro{\...prefix}
In the alternative form |\childdocforwardprefix|,
%
\begin{center}
\begin{tabular}{l}
|\input{childdoc.def}|\\
|\childdocforwardprefix[|\textit{main}|]{|\textit{prefix}|}{|\textit{dest}|}|
\end{tabular}
\end{center}
%
the destination file is determined by a pattern
depending on the current file:
To make this work, the current file must be called
`{\textit{prefix}\hspace{0.2em}\textit{suffix}}'
with \textit{prefix} matching precisely the argument.
Processing is then passed on to the file
`{\textit{dest}\hspace{0.2em}\textit{suffix}}'.
Surely, the same effect is achieved by
directly specifying the
argument `{\textit{dest}\hspace{0.2em}\textit{suffix}}'
in the first form.
However, that requires to set up a different file
for each child. With the alternative form of the command
all these files can have exactly the same content
which simplifies setting them up and maintaining them.

For example, the following file |draft.tex|
with a compilation flag |\version| as described in \secref{sec:flags}
compiles the main document as a draft:
%
\begin{center}
\begin{tabular}{l}
|\def\version{draft}|\\
|\input{childdoc.def}|\\
|\childdocforward{|\textit{main}|}|
\end{tabular}
\end{center}
%
Likewise, the following files |final|\textit{nn}|.tex|
compile the final version of the child document
|child|\textit{nn}|.tex|:
%
\begin{center}
\begin{tabular}{l}
|\def\version{final}|\\
|\input{childdoc.def}|\\
|\childdocforwardprefix{final}{child}|
\end{tabular}
\end{center}
%

Note that when several versions of a main file and/or of each child file
are to be generated, it may be convenient to set up a |Makefile| or
shell script to automatise the process.

%%%%%%%%%%%%%%%%%%%%%%%%%%%%%%%%%%%%%%%%%%%%%%%%%%%%%%%%%%%%%%%%%%%%%%%%%%%%%%%%
\subsection{Command Line Processing}
\label{sec:commandline}

The effect of redirection files can also be achieved by invoking
the \LaTeX{} compiler with a more elaborate command line.
Most conveniently this should be done as part
of a shell script or a |Makefile|.

When using \textsf{childdoc} in the main file, the following
command lines effectively perform a redirection
(note that depending on the shell being used,
backslashes may have to be doubled: `|\|' $\to$ `|\\|'):
%
\begin{center}
|... -jobname "|\textit{target}|" |\\|"|[\textit{flags}]%
|\input{childdoc.def}\childdocforward[|\textit{main}|]{|\textit{dest}|}"|
\end{center}
%
Here \textit{target} is the name of the output file,
\textit{main} is the name of the main file
and \textit{dest} is the name of the main or child file to be processed
(all filenames without extensions).
The optional argument \textit{main} can be omitted
if \textit{main} matches \textit{dest}.
Optionally, compilation \textit{flags} can be defined via |\def| commands.
This command line makes the \TeX{} engine believe
it is compiling the file \textit{target}
whose content is specified as the latter parameter.
The provided code then forwards the processing to
\textit{main} or \textit{dest} as described in \secref{sec:forward}.

%%%%%%%%%%%%%%%%%%%%%%%%%%%%%%%%%%%%%%%%%%%%%%%%%%%%%%%%%%%%%%%%%%%%%%%%%%%%%%%%
\subsection{Include by Input}
\label{sec:input}

Including child documents by |\include| has some restrictions by design.
Most notably, the content of a child document always occupies
its own set of pages; pages cannot be shared between child documents.
Usually, this behaviour makes perfect sense
because each child document contain an essential part of the document.
However, in some situations it may be desirable to compose
a document from a collection of parts
without having mandatory page breaks between then.
For this case, the package
provides a mechanism to include parts
by |\input| which can also be processed individually.
However, by construction this mechanism
requires manual handling of the content to be output.

%%%%%%%%%%%%%%%%%%%%%%%%%%%%%%%%%%%%%%%%
\DescribeMacro{\ifchilddocmanual}
The main file should be prepared as usual, see \secref{sec:include}.
However, the document body must make a distinction
between processing of an individual part and of the main document, e.g.:
%
\begin{center}
\begin{tabular}{l}
|\ifchilddocmanual|\\
|\input{\childdocname}|\\
|\||else|\\
\textit{document body with }|\input{|\textit{part}|}|\\
|\||fi|
\end{tabular}
\end{center}
%
The conditional |\ifchilddocmanual| is true whenever
a part to be included by |\input| is being compiled,
and the name of the part is stored in |\childdocname|.

%%%%%%%%%%%%%%%%%%%%%%%%%%%%%%%%%%%%%%%%
\DescribeMacro{\childdocby}
Each part to be included by |\input| should start with:
%
\begin{center}
\begin{tabular}{l}
|\input{childdoc.def}|\\
|\childdocby{|\textit{main}|}|\\
\end{tabular}
\end{center}
%
The directive |\childdocby| is similar to |\childdocof|
described in \secref{sec:include},
but the subsequent selection of content must be done manually.
To that end, both |\ifchilddoc| and |\ifchilddocmanual|
will be true upon processing of a part,
and the name of the part is stored in |\childdocname|.
Note that |\jobname| will be set to the filename of the current part
so that each part receives an individual |.aux| file
that does not interfere with the |.aux| file(s) of the main document.
This behaviour can be altered by the alternative form
|\childdocby[*]{|\textit{main}|}| (with a non-empty optional argument)
which uses the |.aux| file of the main document
by setting |\jobname| to \textit{main}.

%%%%%%%%%%%%%%%%%%%%%%%%%%%%%%%%%%%%%%%%%%%%%%%%%%%%%%%%%%%%%%%%%%%%%%%%%%%%%%%%
\subsection{Driver Development}
\label{sec:driver}

The \textsf{childdoc} mechanism can also be use for the development
of definition files such as \LaTeX{} styles or classes.
This case differs from the above setup with multiple parts
included by |\include| in that no |\includeonly| should be invoked.
This can be achieved by starting the include file
(before |\ProvidesPackage|) with:
%
\begin{center}
\begin{tabular}{l}
|\input{childdoc.def}|\\
|\childdocforward{|\textit{main}|}|\\
\end{tabular}
\end{center}
%
or alternatively with:
%
\begin{center}
\begin{tabular}{l}
|\input{childdoc.def}|\\
|\childdocby{|\textit{main}|}|\\
\end{tabular}
\end{center}
%
Both forms have slightly different effects as described above.
The main file is prepared as usual, see \secref{sec:include}.

%%%%%%%%%%%%%%%%%%%%%%%%%%%%%%%%%%%%%%%%%%%%%%%%%%%%%%%%%%%%%%%%%%%%%%%%%%%%%%%%
\subsection{Legacy Detection}
\label{sec:detection}

The directive |\childdocmain| in the main file can detect
whether the complete document or merely a child is to be compiled
even without using the directive |\childdocof|.
This method is deprecated because it is less robust
and there is no compelling reason to use it;
it is merely provided for backward compatibility
and it may be removed in future versions.

If the detection mechanism is to be used,
it is mandatory to correctly specify
the filename of the main file as the argument of |\childdocmain|:
%
\begin{center}
\begin{tabular}{l}
|\input{childdoc.def}|\\
|\childdocmain{|\textit{main}|}|\\
\end{tabular}
\end{center}
%
If |\jobname| does not match the argument \textit{main} of |\childdocmain|,
it is assumed that |\jobname| points to the child file to be compiled.
When using |\childdocmain| with the main file specified as argument,
it suffices to start a child file
with just |\input{|\textit{main}|}|
without loading of the package and using |\childdocof|.
If instead all processing is done
with the appropriate \textsf{childdoc} directives,
the argument of \textit{main} of |\childdocmain| can be empty.

An alternative version of the command line processing described
in \secref{sec:commandline} using the detection mechanism reads:
%
\begin{center}
|... -jobname "|\textit{target}|" "|[\textit{flags}]%
[|\def\jobname{|\textit{dest}|}|]|\input{|\textit{main}|}"|
\end{center}

%%%%%%%%%%%%%%%%%%%%%%%%%%%%%%%%%%%%%%%%%%%%%%%%%%%%%%%%%%%%%%%%%%%%%%%%%%%%%%%%
\subsection{Manual Code}
\label{sec:manual}

In case one cannot be certain whether the definitions file |childdoc.def|
is installed on the target \TeX{} distribution
and one prefers not to ship it,
it is conceivable to paste a few relevant commands into the sources.

To that end, drop all statements |\input{childdoc.def}|
and perform the replacements as outlined below.
Instead of |\childdocmain{|\textit{main}|}| add the following code
to the top of the main file:
%
\begin{center}
\begin{tabular}{l}
|\||ifdefined\childdocname\endinput\||fi\newif\ifchilddoc|\\
|\edef\childdocname{\scantokens\expandafter{\jobname\noexpand}}|\\
|\def\childdocmain{|\textit{main}|}\||ifx\childdocmain\childdocname\||else|\\
|\childdoctrue\includeonly{\childdocname}\let\jobname\childdocmain\||fi|\\
\end{tabular}
\end{center}
%
Instead of |\childdocof{|\textit{main}|}| just include the main file
at the top of each child file:
%
\begin{center}
|\input{|\textit{main}|}|
\end{center}
%
A simple redirection |\childdocforward{|\textit{dest}|}| is achieved by:
%
\begin{center}
|\def\jobname{|\textit{dest}|}\input{\jobname}|
\end{center}
%
The redirection with prefix
|\childdocforwardprefix[|\textit{prefix}|]{|\textit{dest}|}|
is accomplished by:
%
\begin{center}
\begin{tabular}{l}
|{\edef\jobname{\scantokens\expandafter{\jobname\noexpand}}|\\
|\def\redirectjob |\textit{prefix}|#1~~~{\gdef\jobname{|\textit{dest}|#1}}|\\
|\expandafter\redirectjob\jobname~~~}\input{\jobname}|
\end{tabular}
\end{center}

In an alternative approach,
child documents can be compiled by a specific command line
without additional code or specific definitions:
%
\begin{center}
|... -jobname "|\textit{target}|" "|[\textit{flags}]%
|\includeonly{|\textit{dest}|}\input{|\textit{main}|}"|
\end{center}
%

%%%%%%%%%%%%%%%%%%%%%%%%%%%%%%%%%%%%%%%%%%%%%%%%%%%%%%%%%%%%%%%%%%%%%%%%%%%%%%%%
%%%%%%%%%%%%%%%%%%%%%%%%%%%%%%%%%%%%%%%%%%%%%%%%%%%%%%%%%%%%%%%%%%%%%%%%%%%%%%%%
\section{Information}

%%%%%%%%%%%%%%%%%%%%%%%%%%%%%%%%%%%%%%%%%%%%%%%%%%%%%%%%%%%%%%%%%%%%%%%%%%%%%%%%
\subsection{Copyright}

Copyright \copyright{} 2017--2018 Niklas Beisert

This work may be distributed and/or modified under the
conditions of the \LaTeX{} Project Public License, either version 1.3
of this license or (at your option) any later version.
The latest version of this license is in
  \url{http://www.latex-project.org/lppl.txt}
and version 1.3 or later is part of all distributions of \LaTeX{}
version 2005/12/01 or later.

This work has the LPPL maintenance status `maintained'.

The Current Maintainer of this work is Niklas Beisert.

This work consists of the files |README.txt|, |childdoc.ins| and |childdoc.dtx|
as well as the derived files |childdoc.def|, |cdocsamp.tex|
with |cdocsch1.tex|, |cdocsch2.tex|, |cdocspt3.tex|, |cdocspt4.tex|,
|cdocsdrf.tex|, |cdocsfn1.tex|, |cdocsfn2.tex|
as well as |childdoc.pdf|.

%%%%%%%%%%%%%%%%%%%%%%%%%%%%%%%%%%%%%%%%%%%%%%%%%%%%%%%%%%%%%%%%%%%%%%%%%%%%%%%%
\subsection{Files and Installation}

The package consists of the files:
%
\begin{center}
\begin{tabular}{ll}
    |README.txt|   & readme file \\
    |childdoc.ins| & installation file \\
    |childdoc.dtx| & source file \\
    |childdoc.def| & definition file \\
    |cdocsamp.tex| & sample main file \\
    |cdocsch1.tex| & sample include file \\
    |cdocsch2.tex| & sample include file \\
    |cdocspt3.tex| & sample part file \\
    |cdocspt4.tex| & sample part file \\
    |cdocsdrf.tex| & sample redirection file \\
    |cdocsfn1.tex| & sample redirection file \\
    |cdocsfn2.tex| & sample redirection file \\
    |childdoc.pdf| & manual
\end{tabular}
\end{center}
%
The distribution consists of the files
|README.txt|, |childdoc.ins| and |childdoc.dtx|.
%
\begin{itemize}
\item
Run (pdf)\LaTeX{} on |childdoc.dtx|
to compile the manual |childdoc.pdf| (this file).
\item
Run \LaTeX{} on |childdoc.ins| to create the definitions file |childdoc.def|
and the sample |cdocsamp.tex| with include files
|cdocsch1.tex|, |cdocsch2.tex|, |cdocspt3.tex|, |cdocspt4.tex|,
|cdocsdrf.tex|, |cdocsfn1.tex|, |cdocsfn2.tex|.
Then copy the file |childdoc.def| to an appropriate directory of your \LaTeX{}
distribution, e.g.\ \textit{texmf-root}|/tex/latex/childdoc|.
\end{itemize}

%%%%%%%%%%%%%%%%%%%%%%%%%%%%%%%%%%%%%%%%%%%%%%%%%%%%%%%%%%%%%%%%%%%%%%%%%%%%%%%%
\subsection{Related CTAN Packages}

There are several other packages which offer a similar functionality:
%
\begin{itemize}
\item
The packages
\href{http://ctan.org/pkg/docmute}{\textsf{docmute}},
\href{http://ctan.org/pkg/includex}{\textsf{includex}} and
\href{http://ctan.org/pkg/standalone}{\textsf{standalone}}
provide commands to include only the document body of
a child file thus allowing both files to be compiled individually.
\item
The packages \href{http://ctan.org/pkg/subdocs}{\textsf{subdocs}}
and \href{http://ctan.org/pkg/subfiles}{\textsf{subfiles}}
provide structures in which the main and child documents can be
encapsulated and allowing them to be compiled individually.
The inclusion mechanism is different from the conventional |\include|.
\item
The package \href{http://ctan.org/pkg/combine}{\textsf{combine}}
is an elaborate solution to combine several documents into one.
\end{itemize}
%
See also the CTAN topic \href{http://ctan.org/topic/subdocs}{\textsf{subdocs}}
for further related packages.
The present package differs from the above solutions in that
a document structure constructed with the conventional |\include| mechanism
just needs two extra commands at the top of every file
such that all constituent files can be compiled individually.

%%%%%%%%%%%%%%%%%%%%%%%%%%%%%%%%%%%%%%%%%%%%%%%%%%%%%%%%%%%%%%%%%%%%%%%%%%%%%%%%
%\subsection{Feature Suggestions}
%
%The following is a list of features which may be useful for future
%versions of this package:
%%
%\begin{itemize}
%\item
%\ldots
%\end{itemize}

%%%%%%%%%%%%%%%%%%%%%%%%%%%%%%%%%%%%%%%%%%%%%%%%%%%%%%%%%%%%%%%%%%%%%%%%%%%%%%%%
\subsection{Revision History}

%%%%%%%%%%%%%%%%%%%%%%%%%%%%%%%%%%%%%%%%
\paragraph{v2.0:} 2018/12/30

\begin{itemize}
\item
immediate forward processing
\item
added |\childdocby| mechanism
\item
manual restructured
\end{itemize}

%%%%%%%%%%%%%%%%%%%%%%%%%%%%%%%%%%%%%%%%
\paragraph{v1.6:} 2018/01/17

\begin{itemize}
\item
application for development of include files
\item
corrections to manual
\end{itemize}

%%%%%%%%%%%%%%%%%%%%%%%%%%%%%%%%%%%%%%%%
\paragraph{v1.5:} 2017/05/21

\begin{itemize}
\item
more complete structuring introduced
\item
|\childdocof| introduced
\item
|\childdoc| renamed to |\childdocmain|
\item
|\childredirect| renamed to |\childdocforward| and |\childdocforwardprefix|
and functionality expanded
\end{itemize}

%%%%%%%%%%%%%%%%%%%%%%%%%%%%%%%%%%%%%%%%
\paragraph{v1.0:} 2017/04/27

\begin{itemize}
\item
manual and install package
\item
first version published on CTAN
\end{itemize}

%%%%%%%%%%%%%%%%%%%%%%%%%%%%%%%%%%%%%%%%
\paragraph{v0.6:} 2017/04/26

\begin{itemize}
\item
redirection mechanism added
\end{itemize}

%%%%%%%%%%%%%%%%%%%%%%%%%%%%%%%%%%%%%%%%
\paragraph{v0.5:} 2017/04/26

\begin{itemize}
\item
functionality in definition file
\end{itemize}


%%%%%%%%%%%%%%%%%%%%%%%%%%%%%%%%%%%%%%%%%%%%%%%%%%%%%%%%%%%%%%%%%%%%%%%%%%%%%%%%
%%%%%%%%%%%%%%%%%%%%%%%%%%%%%%%%%%%%%%%%%%%%%%%%%%%%%%%%%%%%%%%%%%%%%%%%%%%%%%%%
%%%%%%%%%%%%%%%%%%%%%%%%%%%%%%%%%%%%%%%%%%%%%%%%%%%%%%%%%%%%%%%%%%%%%%%%%%%%%%%%
\appendix

\settowidth\MacroIndent{\rmfamily\scriptsize 000\ }

 \DocInput{childdoc.dtx}

\end{document}
%</driver>
% \fi
%
% %%%%%%%%%%%%%%%%%%%%%%%%%%%%%%%%%%%%%%%%%%%%%%%%%%%%%%%%%%%%%%%%%%%%%%%%%%%%%%
% %%%%%%%%%%%%%%%%%%%%%%%%%%%%%%%%%%%%%%%%%%%%%%%%%%%%%%%%%%%%%%%%%%%%%%%%%%%%%%
% \section{Sample}
%\iffalse
%<*samplemain>
%\fi
%
% The following presents a sample document
% with two chapters, two parts, a title page,
% a compile flag as well as three forwarding files to set the flag.
% It consists of eight |.tex| files:
% \begin{center}
% \begin{tabular}{ll}
% |cdocsamp.tex|&main file\\
% |cdocsch1.tex|&include file for chapter 1\\
% |cdocsch2.tex|&include file for chapter 2\\
% |cdocspt3.tex|&include file for part 3\\
% |cdocspt4.tex|&include file for part 4\\
% |cdocsdrf.tex|&forwarding file for main file in draft mode\\
% |cdocsfi1.tex|&forwarding file for final version of chapter 1\\
% |cdocsfi2.tex|&forwarding file for final version of chapter 2\\
% \end{tabular}
% \end{center}
% Each of the eight files can be compiled directly by the \LaTeX{} compiler.
%
% %%%%%%%%%%%%%%%%%%%%%%%%%%%%%%%%%%%%%%
% \paragraph{Main File.}
%
% The main file is called |cdocsamp.tex|.
%
% Load the \textsf{childdoc} definitions and
% declare the filename for the main document:
%    \begin{macrocode}
\input{childdoc.def}
\childdocmain{}
%    \end{macrocode}

% Optional override for |\version| flag:
%    \begin{macrocode}
%%\ifchilddoc\else\providecommand{\version}{draft}\fi
%    \end{macrocode}

% Define the default values for the |\version| flag
% (|final| for the main file and |draft| for childs):
%    \begin{macrocode}
\ifchilddoc
\providecommand{\version}{draft}
\else
\providecommand{\version}{final}
\fi
%    \end{macrocode}

% Load the standard document class:
%    \begin{macrocode}
\documentclass[12pt]{article}
%    \end{macrocode}

% Start the document body:
%    \begin{macrocode}
\begin{document}
%    \end{macrocode}

% Declare a title page.
% Print title, part of document being processed and version flag:
%    \begin{macrocode}
\addtocounter{page}{-1}
\begin{center}
{\LARGE\bfseries{}childdoc example\par}
\vspace{1cm}
\ifchilddoc
\ifchilddocmanual part\else chapter\fi:
`\childdocname' of `\childdocjob'\par
\else
main document: `\childdocjob'\par
\fi
version: \version\par
\end{center}
\newpage
%    \end{macrocode}

% Manually include selected file,
% otherwise process as usual:
%    \begin{macrocode}
\ifchilddocmanual
\section*{part `\childdocname'}
\input{\childdocname}
\else
%    \end{macrocode}

% Include the two chapters:
%    \begin{macrocode}
\include{cdocsch1}
\include{cdocsch2}
%    \end{macrocode}

% Include the two parts unless only chapters should be displayed:
%    \begin{macrocode}
\ifchilddoc\else
\section{part three}
\input{cdocspt3}
\section{part four}
\input{cdocspt4}
\fi
%    \end{macrocode}

% Process as usual until here:
%    \begin{macrocode}
\fi
%    \end{macrocode}

% End of document body:
%    \begin{macrocode}
\end{document}
%    \end{macrocode}
%\iffalse
%</samplemain>
%\fi
%
% %%%%%%%%%%%%%%%%%%%%%%%%%%%%%%%%%%%%%%
% \paragraph{Chapter Include Files.}
%
% The include files are called |cdocsch1.tex| and |cdocsch2.tex|.
%
%\iffalse
%<*samplechap1|samplechap2>
%\fi

% Optional override for |\version| flag:
%    \begin{macrocode}
%%\providecommand{\version}{final}
%    \end{macrocode}

% Include the main document:
%    \begin{macrocode}
\input{childdoc.def}
\childdocof{cdocsamp}
%    \end{macrocode}

%\iffalse
%</samplechap1|samplechap2>
%\fi
%
%\iffalse
%<*samplechap1>
%\fi
% Some text for chapter 1:
%    \begin{macrocode}
\section{one}
some text in chapter one
%    \end{macrocode}

%\iffalse
%</samplechap1>
%\fi
% Some text for chapter 2:
%\iffalse
%<*samplechap2>
%\fi
%    \begin{macrocode}
\section{two}
more text in chapter two
%    \end{macrocode}

%\iffalse
%</samplechap2>
%\fi
%
% %%%%%%%%%%%%%%%%%%%%%%%%%%%%%%%%%%%%%%
% \paragraph{Part Include Files.}
%
% The include files are called |cdocspt3.tex| and |cdocspt4.tex|.
%
%\iffalse
%<*samplepart3|samplepart4>
%\fi

% Optional override for |\version| flag:
%    \begin{macrocode}
%%\providecommand{\version}{final}
%    \end{macrocode}

% Include the main document:
%    \begin{macrocode}
\input{childdoc.def}
\childdocby{cdocsamp}
%    \end{macrocode}

%\iffalse
%</samplepart3|samplepart4>
%\fi
%
%\iffalse
%<*samplepart3>
%\fi
% Some text for part 3:
%    \begin{macrocode}
some text in part three
%    \end{macrocode}

%\iffalse
%</samplepart3>
%\fi
% Some text for part 4:
%\iffalse
%<*samplepart4>
%\fi
%    \begin{macrocode}
more text in part four
%    \end{macrocode}

%\iffalse
%</samplepart4>
%\fi
%
% %%%%%%%%%%%%%%%%%%%%%%%%%%%%%%%%%%%%%%
% \paragraph{Forwarding for a Complete Draft.}
%
% The following forwarding file |cdocsdrf.tex|
% compiles the main document in draft mode:
%\iffalse
%<*sampledraft>
%\fi
%    \begin{macrocode}
\def\version{draft}
\input{childdoc.def}
\childdocforward{cdocsamp}
%    \end{macrocode}

%\iffalse
%</sampledraft>
%\fi
%
% %%%%%%%%%%%%%%%%%%%%%%%%%%%%%%%%%%%%%%
% \paragraph{Forwarding for Final Version of the Chapters.}
%
% The following forwarding files |cdocsfn1.tex| and |cdocsfn2.tex|
% (with identical content)
% compile the final versions of the child documents
% |cdocsch1.tex| and |cdocsch2.tex|, respectively:
%\iffalse
%<*samplefinal>
%\fi
%    \begin{macrocode}
\def\version{final}
\input{childdoc.def}
\childdocforwardprefix[cdocsamp]{cdocsfn}{cdocsch}
%    \end{macrocode}

%\iffalse
%</samplefinal>
%\fi
%
% %%%%%%%%%%%%%%%%%%%%%%%%%%%%%%%%%%%%%%
% \paragraph{Command Line Processing.}
%
% The following three command lines generate the output files
% |cdocscld|, |cdocscl1| and |cdocscl2|
% which should be identical to
% |cdocsdrf|, |cdocsch1| and |cdocsfn2|, respectively:
% \begin{center}
% \begin{tabular}{l}
% |latex -jobname cdocscld \|\\
% |  "\def\version{draft}\input{childdoc.def}\childdocforward{cdocsamp}"|\\
% |latex -jobname cdocscl1 \|\\
% |  "\input{childdoc.def}\childdocforward[cdocsamp]{cdocsch1}"|\\
% |latex -jobname cdocscl2 \|\\
% |  "\def\version{final}\input{childdoc.def}\childdocforward{cdocsch2}"|
% \end{tabular}
% \end{center}
% Note that the trailing backslash on each first line
% merely continues the input to the second line
% (for convenient cut ant paste).
% Furthermore, the command |latex| can be replaced by any
% of its alternative versions such as |pdflatex|.
%
% %%%%%%%%%%%%%%%%%%%%%%%%%%%%%%%%%%%%%%%%%%%%%%%%%%%%%%%%%%%%%%%%%%%%%%%%%%%%%%
% %%%%%%%%%%%%%%%%%%%%%%%%%%%%%%%%%%%%%%%%%%%%%%%%%%%%%%%%%%%%%%%%%%%%%%%%%%%%%%
% \section{Implementation}
%\iffalse
%<*package>
%\fi
%
% This section describes the definitions file |childdoc.def|.

% The definitions cannot be loaded using |\usepackage| or |\RequirePackage|
% which has a mechanism to prevent loading a style file more than once.
% When loading the definitions by means of |\input|
% multiple instances have to be prevented manually:
%\iffalse
%This code needs to be before the `\ProvidesFile' directive
%which is defined at the beginning of this file.
%Therefore it is also placed there and commented out here.
%</package>
%<*discard>
%\fi
%    \begin{macrocode}
\ifdefined\childdocmain\endinput\fi
%    \end{macrocode}
%\iffalse
%</discard>
%<*package>
%\fi
%
% \macro{\ifchilddoc}
% \macro{\ifchilddocmanual}
% The conditional |\ifchilddoc| tells whether a
% child (true) or main (false) document is being compiled.
% The conditional |\ifchilddocmanual| tells whether
% the |\includeonly| mechanism is used (false) or
% the selection of child files must be performed manually (true).
% The definitions initialise to false:
%    \begin{macrocode}
\newif\ifchilddoc
\newif\ifchilddocmanual
%    \end{macrocode}

% \macro{\childdocname}
% \macro{\childdocjob}
% The macro |\childdocname| stores the name of the main document
% to be compiled. The macro |\childdocjob| stores the name of
% the document on which the \LaTeX{} compiler was originally invoked.
% The content of |\jobname| cannot be compared
% to filenames specified in the source due to different catcodes.
% The following code rescans |\jobname|, stores the result
% in |\childdocname| and saves a copy in |\childdocjob|:
%    \begin{macrocode}
\edef\childdocname{\scantokens\expandafter{\jobname\noexpand}}
\let\childdocjob\childdocname
%    \end{macrocode}

% \macro{\childdocdisable}
% The macro |\childdocdisable| prevents the main file
% from being processed more than once.
% At this stage, the main document command |\childdocmain|
% is assumed to be called once again where it should do nothing.
% Any subsequent call to it should prevent
% a secondary processing of the main document
% It overwrites the forwarding commands
% |\childdocof| and |\childdocforward|
% with empty macros to prevent further inclusions of the main document:
%    \begin{macrocode}
\newcommand{\childdocdisable}
{
  \renewcommand{\childdocmain}[1]{\renewcommand{\childdocmain}[1]{\endinput}}
  \renewcommand{\childdocof}[1]{}
  \renewcommand{\childdocby}[2][]{}
  \renewcommand{\childdocforward}[2][]{}
  \renewcommand{\childdocdisable}{}
}
%    \end{macrocode}

% \macro{\childdocmain}
% The macro |\childdocmain| is to be called at the top of the main file
% with nothing or the main filename (without extension) as argument.
% First, it breaks loops.
% If the argument is not empty and does not match |\childdocname|
% (which is set by the first inclusion of |childdoc.def|),
% |\ifchilddoc| is set to true, |\includeonly| is applied to the child file
% and |\jobname| is set to the main file
% (for proper handling of |.aux| files):
%    \begin{macrocode}
\newcommand{\childdocmain}[1]
{
  \childdocdisable\childdocmain{}
  \if?#1?\else
    \begingroup
      \def\childdoctmp{#1}
      \ifx\childdoctmp\childdocname
        \def\childdoctmp{}
      \else
        \def\childdoctmp
        {
          \childdoctrue
          \includeonly{\childdocname}
          \def\childdocjob{#1}
          \def\jobname{#1}
        }
      \fi
      \expandafter
    \endgroup
    \childdoctmp
  \fi
}
%    \end{macrocode}

% \macro{\childdocof}
% The command |\childdocof| redirects
% compilation to the main file |#1|.
%    \begin{macrocode}
\newcommand{\childdocof}[1]
{
  \childdocdisable
  \childdoctrue
  \includeonly{\childdocname}
  \def\jobname{#1}
  \def\childdocjob{#1}
  \input{#1}
}
%    \end{macrocode}

% \macro{\childdocby}
% The command |\childdocby| ....
%    \begin{macrocode}
\newcommand{\childdocby}[2][]
{
  \childdocdisable
  \childdoctrue
  \childdocmanualtrue
  \if?#1?\else
    \def\jobname{#2}
  \fi
  \def\childdocjob{#2}
  \input{#2}
  \endinput
}
%    \end{macrocode}

% \macro{\childdocforward}
% The command |\childdocforward| redirects
% compilation to the main file or
% (if the optional argument is given) a child file.
% Parameters are set as if the main file
% or a child file starting with |\childdocof| was compiled.
% Then compilation is handed over to the main file:
%    \begin{macrocode}
\newcommand{\childdocforward}[2][]
{
  \begingroup
    \if?#1?
      \def\childdoctmp
      {
        \def\childdocname{#2}
        \def\childdocjob{#2}
        \def\jobname{#2}
        \input{#2}
        \endinput
      }
    \else
      \def\childdoctmp
      {
        \childdocdisable
        \def\childdocname{#2}
        \childdoctrue
        \includeonly{#2}
        \def\childdocjob{#1}
        \def\jobname{#1}
        \input{#1}
        \endinput
      }
    \fi
    \expandafter
  \endgroup
  \childdoctmp
}
%    \end{macrocode}

% \macro{\childdocforwardprefix}
% The command |\childdocforwardprefix| redirects
% compilation to the main or a child file by means of a pattern.
% The prefix |#1| in the current filename is replaced by |#2|
% and the suffix of the current filename is kept
% (it is assumed that the filename does not contain the substring `|~~~|'
% which is used as a delimiter).
% Compilation is handed over to the new file by |\childdocforward|:
%    \begin{macrocode}
\newcommand{\childdocforwardprefix}[3][]
{
  \begingroup
    \def\childdocextract #2##1~~~{\def\childdoctmp{\childdocforward[#1]{#3##1}}}
    \expandafter\childdocextract\childdocname~~~
    \expandafter
  \endgroup
  \childdoctmp
}
%    \end{macrocode}

% \macro{\childdoc}
% The deprecated macro |\childdoc| is a legacy version of |\childdocmain|:
%    \begin{macrocode}
\newcommand{\childdoc}{\childdocmain}
%    \end{macrocode}

% \macro{\childdocredirect}
% The deprecated macro |\childdocredirect| is a legacy version
% of |\childdocforward| and |\childdocforwardprefix|:
%    \begin{macrocode}
\newcommand{\childdocredirect}[2][]
{
  \begingroup
    \if?#1?
      \def\childdoctmp{\childdocforward{#2}}
    \else
      \def\childdoctmp{\childdocforwardprefix{#1}{#2}}
    \fi
    \expandafter
  \endgroup
  \childdoctmp
}
%    \end{macrocode}

%\iffalse
%</package>
%\fi
%
\endinput
\childdocforward[cdocsamp]{cdocsch1}"|\\
% |latex -jobname cdocscl2 \|\\
% |  "\def\version{final}% \iffalse
%
% childdoc.dtx Copyright (C) 2017-2018 Niklas Beisert
%
% This work may be distributed and/or modified under the
% conditions of the LaTeX Project Public License, either version 1.3
% of this license or (at your option) any later version.
% The latest version of this license is in
%   http://www.latex-project.org/lppl.txt
% and version 1.3 or later is part of all distributions of LaTeX
% version 2005/12/01 or later.
%
% This work has the LPPL maintenance status `maintained'.
%
% The Current Maintainer of this work is Niklas Beisert.
%
% This work consists of the files childdoc.dtx and childdoc.ins
% and the derived files childdoc.def and cdocsamp.tex with
% cdocsch1.tex, cdocsch2.tex, cdocsdrf.tex, cdocsfn1.tex, cdocsfn2.tex.
%
%<package>\ifdefined\childdocmain\endinput\fi
%<package>\ProvidesFile{childdoc.def}[2018/12/30 v2.0 child document driver]
%<samplemain>\ProvidesFile{cdocsamp.tex}[2018/12/30 v2.0 sample for childdoc]
%<*driver>
%\ProvidesFile{childdoc.drv}[2018/12/30 v2.0 childdoc reference manual file]
\PassOptionsToClass{10pt,a4paper}{article}
\documentclass{ltxdoc}

\usepackage[margin=35mm]{geometry}
\usepackage{hyperref}
\usepackage{hyperxmp}
\usepackage[usenames]{color}

\hypersetup{colorlinks=true}
\hypersetup{pdfstartview=FitH}
\hypersetup{pdfpagemode=UseNone}
\hypersetup{pdfsource={}}
\hypersetup{pdflang={en-UK}}
\hypersetup{pdfcopyright={Copyright 2017-2018 Niklas Beisert.
  This work may be distributed and/or modified under the
  conditions of the LaTeX Project Public License, either version 1.3
  of this license or (at your option) any later version.}}
\hypersetup{pdflicenseurl={http://www.latex-project.org/lppl.txt}}
\hypersetup{pdfcontactaddress={ETH Zurich, ITP, HIT K,
  Wolfgang-Pauli-Strasse 27}}
\hypersetup{pdfcontactpostcode={8093}}
\hypersetup{pdfcontactcity={Zurich}}
\hypersetup{pdfcontactcountry={Switzerland}}
\hypersetup{pdfcontactemail={nbeisert@itp.phys.ethz.ch}}
\hypersetup{pdfcontacturl={http://people.phys.ethz.ch/\xmptilde nbeisert/}}

\newcommand{\secref}[1]{\hyperref[#1]{section \ref*{#1}}}

\parskip1ex
\parindent0pt
\let\olditemize\itemize
\def\itemize{\olditemize\parskip0pt}

\begin{document}

\title{The \textsf{childdoc} Package}
\hypersetup{pdftitle={The childdoc Package}}
\author{Niklas Beisert\\[2ex]
  Institut f\"ur Theoretische Physik\\
  Eidgen\"ossische Technische Hochschule Z\"urich\\
  Wolfgang-Pauli-Strasse 27, 8093 Z\"urich, Switzerland\\[1ex]
  \href{mailto:nbeisert@itp.phys.ethz.ch}
  {\texttt{nbeisert@itp.phys.ethz.ch}}}
\hypersetup{pdfauthor={Niklas Beisert}}
\hypersetup{pdfsubject={Manual for the LaTeX2e Package childdoc}}
\date{30 December 2018, \textsf{v2.0}}
\maketitle

\begin{abstract}\noindent
\textsf{childdoc} is a \LaTeXe{} package
that enables the direct compilation
of document sections included by |\include|
to individual files.
\end{abstract}

\begingroup
\parskip0ex
\tableofcontents
\endgroup

%%%%%%%%%%%%%%%%%%%%%%%%%%%%%%%%%%%%%%%%%%%%%%%%%%%%%%%%%%%%%%%%%%%%%%%%%%%%%%%%
%%%%%%%%%%%%%%%%%%%%%%%%%%%%%%%%%%%%%%%%%%%%%%%%%%%%%%%%%%%%%%%%%%%%%%%%%%%%%%%%
\section{Introduction}

\LaTeX{} provides a mechanism to structure a large document (such as a book)
into a main file and several child files (containing the chapters)
using the |\include| command.
This mechanism is beneficial for documents
which span hundreds of pages in order to
make the source file(s) more manageable.
Moreover, compilation can be restricted to
selected child files by means of the |\includeonly| command.
The latter feature can be used to reduce the compilation time while editing
(this was significantly more useful in the earlier days of \LaTeX{})
or to generate a smaller document which is easier to navigate.
Another application of |\includeonly| is to generate
documents consisting of selected parts of the complete document.

However, there are a few drawbacks of the plain |\include| mechanism:
\begin{itemize}
\item
The child files cannot be compiled on their own,
they can only be compiled via the main file.
A naive editing environment
(such as a text editor with an option
to have the current file processed by \LaTeX)
may require one to switch to the main file before compiling;
attempting to compile the child file produces errors.
\item
The main file must be modified (each time)
to adjust the |\includeonly| command
to the present needs. This easily leaves the main file in a messy state.
\item
The generated document will always carry the filename
of the main document. This is inconvenient if
several child files are to be compiled and
to be kept for distribution.
\end{itemize}

The present package provides a simple interface
to make child files individually compilable by \LaTeX{}.
Compiling a child file then has the same effect as compiling
the main file with an |\includeonly| command
to select the appropriate child.
Moreover the generated document will carry the name of the child
rather than the main file.
This resolves all three above issues.

This feature is meant to make the editing of books,
thesis documents and lecture notes somewhat more convenient.
However, the package can also be used efficiently for
composing a series of documents (such as exercise sheets)
which are typically distributed individually.
It then assists the author in generating the individual documents
(potentially in different versions)
as well as a document containing the collected series.
Another application is in developing style files
or other kinds of included material
where compilation of the style file could redirect
to a sample or test file.

%%%%%%%%%%%%%%%%%%%%%%%%%%%%%%%%%%%%%%%%%%%%%%%%%%%%%%%%%%%%%%%%%%%%%%%%%%%%%%%%
%%%%%%%%%%%%%%%%%%%%%%%%%%%%%%%%%%%%%%%%%%%%%%%%%%%%%%%%%%%%%%%%%%%%%%%%%%%%%%%%
\section{Usage}

First of all, the package \textsf{childdoc} is \emph{not} a standard
\LaTeXe{} |.sty| style file! Therefore it needs to be invoked in
a non-standard way.

%%%%%%%%%%%%%%%%%%%%%%%%%%%%%%%%%%%%%%%%%%%%%%%%%%%%%%%%%%%%%%%%%%%%%%%%%%%%%%%%
\subsection{Included Files}
\label{sec:include}

%%%%%%%%%%%%%%%%%%%%%%%%%%%%%%%%%%%%%%%%
\DescribeMacro{\childdocmain}
To use the package, add the commands
\begin{center}
\begin{tabular}{l}
|\input{childdoc.def}|\\
|\childdocmain{}|\\
\end{tabular}
\end{center}
at the very top of the main \LaTeX{} file,
in particular \emph{before} the |\documentclass| statement!
The argument of |\childdocmain| should be left empty
(but it must be present).

%%%%%%%%%%%%%%%%%%%%%%%%%%%%%%%%%%%%%%%%
\DescribeMacro{\childdocof}
Furthermore, add the commands
\begin{center}
\begin{tabular}{l}
|\input{childdoc.def}|\\
|\childdocof{|\textit{main}|}|\\
\end{tabular}
\end{center}
at the top of every child file \textit{child}
which is included by |\include{|\textit{child}|}|
from within the main file
(or at least for those files to be compiled individually).
The argument \textit{main} must be the filename of the main file.

There are a couple of
considerations in setting up the main and child documents:

%%%%%%%%%%%%%%%%%%%%%%%%%%%%%%%%%%%%%%%%
\paragraph{Restrictions.}

Please note the following restrictions:
\begin{itemize}
\item
|\childdocmain| must be called with one argument \textit{main}
to ensure compatibility with earlier version of the package.
It must either be empty (|\childdocmain{}|)
or precisely match the filename of the main file in which it is specified.
See \secref{sec:detection} for further information.
\item
The filename \textit{main} must be specified without the |.tex| extension.
\item
The filename \textit{main} is case sensitive
(even in case-insensitive file systems)
due to internal string comparison.
\item
The argument \textit{main} should be fully expanded, it cannot be a macro.
\item
Subdirectories and special characters should be avoided in filenames.
\item
The command |\childdocmain{|\textit{main}|}| must be followed by a whitespace.
It should not be followed immediately by another command
or by a comment mark `|%|'.
This is because the \TeX{} parser reads the token immediately following
the argument of |\childdocmain| and puts it
at the beginning of every child section;
however, a white\-space is ignored.
\end{itemize}

%%%%%%%%%%%%%%%%%%%%%%%%%%%%%%%%%%%%%%%%
\paragraph{Content of Main File.}

It is advisable to place all content in the child files included by |\include|.
Any output contained in the main file will appear in all child documents
unless suppressed manually;
it cannot be suppressed automatically by the |\includeonly| directive
and thus should normally be avoided.
A method to include some content in the main file
by means of conditional processing is described in \secref{sec:conditional}.

%%%%%%%%%%%%%%%%%%%%%%%%%%%%%%%%%%%%%%%%
\paragraph{Page Numbering.}

When only a part of the document is compiled,
the appropriate numbering of pages
(as well as other status parameters)
is determined from the |.aux| files.
The latter contain information from previous passes.
However this information needs to propagate through
all intermediate child documents.
Therefore the page numbering in child documents may well
be inconsistent until the complete document is compiled at least once.

A useful (if unconventional) way to always ensure a consistent
page numbering is to restart the numbering in each child document
and denote the pages by `\textit{child}|.|\textit{page}'
where \textit{child} represents the chapter/section number of the child file.
This can be achieved by the command
|\numberwithin{page}{|\textit{child}|}|
of the \textsf{amsmath} package
where \textit{child} can be |chapter| or |section|
depending on the chosen structuring.
Alternatively, one can modify the macro |\thepage| appropriately
and reset the counter |page| at the start of each child file.

%%%%%%%%%%%%%%%%%%%%%%%%%%%%%%%%%%%%%%%%%%%%%%%%%%%%%%%%%%%%%%%%%%%%%%%%%%%%%%%%
\subsection{Conditional Processing}
\label{sec:conditional}

The package provides a mechanism to compile different versions
of a document. To customise the versions further some conditional processing
can come in handy to distinguish which version is being compiled.
The package provides two macros to describe the compilation context:

%%%%%%%%%%%%%%%%%%%%%%%%%%%%%%%%%%%%%%%%
\DescribeMacro{\ifchilddoc}
The conditional |\ifchilddoc| distinguishes between the compilation of
child documents and the main document:
%
\begin{center}
|\ifchilddoc |\textit{child-code}| |[|\||else |\textit{main-code}]| \||fi|
\end{center}

%%%%%%%%%%%%%%%%%%%%%%%%%%%%%%%%%%%%%%%%
\DescribeMacro{\childdocname}
\DescribeMacro{\childdocjob}
The macro |\childdocname| contains the filename (without extension)
of the main or child file being processed.
Note that |\childdocjob| will always contain the name of the main file.

%%%%%%%%%%%%%%%%%%%%%%%%%%%%%%%%%%%%%%%%
\paragraph{Title Page.}

Conditional processing can be used to include a title or banner page
in the main document when proper precautions are taken.
Importantly, the code in the main file should ensure that the page counter
(as well as other status parameters which are stored in the |.aux| files)
takes the same value after the conditional processing.
Otherwise the page numbers may take divergent values
depending on which part is compiled.

For example, a title page could be declared by:
%
\begin{center}
\begin{tabular}{l}
|\ifchilddoc\||else|\\
|\addtocounter{page}{-1}|\\
\textit{code for title page}\\
|\newpage|\\
|\||fi|
\end{tabular}
\end{center}
%
A banner page for the child documents can be generated by:
%
\begin{center}
\begin{tabular}{l}
|\ifchilddoc|\\
|\addtocounter{page}{-1}|\\
\textit{code for banner page}\\
|\newpage|\\
|\||fi|
\end{tabular}
\end{center}
%
Here one could write a message such as:
\begin{center}
|This is the part \childdocname{} of \childdocjob{}.|
\end{center}

%%%%%%%%%%%%%%%%%%%%%%%%%%%%%%%%%%%%%%%%%%%%%%%%%%%%%%%%%%%%%%%%%%%%%%%%%%%%%%%%
\subsection{Flags}
\label{sec:flags}

The package makes it easy to generate different versions
of the main or child documents.
To this end compilation flags can be defined
and assigned different default values.
They will be particularly useful in conjunction
with the forwarding mechanism described in \secref{sec:forward}.

For example, it may be useful to have a flag |\version|
which can be set to |draft| or |final|.
The document source will contain some conditional code
depending on the value of |\version|.
Suppose further, the flag should default to |final| for the main file
and to |draft| for child files
which is a natural assignment for editing the document.
This is achieved by placing the following code
in the preamble of the main document
(below the |\childdocmain| directive):
%
\begin{center}
\begin{tabular}{l}
|\ifchilddoc|\\
|\providecommand{\version}{draft}|\\
|\||else|\\
|\providecommand{\version}{final}|\\
|\||fi|
\end{tabular}
\end{center}
%
The definition by |\providecommand| makes sure
that previous definitions are not overwritten.
Further statements |\providecommand{\version}{...}|
can thus be added before the above code to override it.

For the main file, one might add a line
(between |\childdocmain| and the above block)
%
\begin{center}
|%\ifchilddoc\||else\providecommand{\version}{draft}\||fi|
\end{center}
%
which can be uncommented to produce a draft version.
Likewise one can add a line to the very top of a child file
(above the |\childdocof{|\textit{main}|}| directive)
%
\begin{center}
|%\providecommand{\version}{final}|
\end{center}
%
which can be uncommented to produce the final version of this child document.

%%%%%%%%%%%%%%%%%%%%%%%%%%%%%%%%%%%%%%%%%%%%%%%%%%%%%%%%%%%%%%%%%%%%%%%%%%%%%%%%
\subsection{Forwarding}
\label{sec:forward}

Different versions of the main or child documents
using compilation flags as described in \secref{sec:flags}
can be (permanently) stored in different files
for convenient compilation, viewing and distribution.
To this end, the package defines a command
to pass on compilation to a different file:

%%%%%%%%%%%%%%%%%%%%%%%%%%%%%%%%%%%%%%%%
\DescribeMacro{\childdocforward}
The command |\childdocforward| redirects processing to
another source file:
%
\begin{center}
\begin{tabular}{l}
|\input{childdoc.def}|\\
|\childdocforward[|\textit{main}|]{|\textit{dest}|}|\\
\end{tabular}
\end{center}
%
The argument \textit{dest} is the destination file
(without extension).
It should be the main file or one of the child files.
Note that further \textsf{childdoc} directives
such as |\childdocof| and |\childdocforward|
in the indicated file will be processed in this form.
The optional argument \textit{main}
passes on directly to the main file \textit{main}
while pretending to compile the child \textit{dest}.
This form behaves as if \textit{dest}
issues |\childdocof{|\textit{main}|}| right away,
and no further \textsf{childdoc} directives will be processed.

%%%%%%%%%%%%%%%%%%%%%%%%%%%%%%%%%%%%%%%%
\DescribeMacro{\...prefix}
In the alternative form |\childdocforwardprefix|,
%
\begin{center}
\begin{tabular}{l}
|\input{childdoc.def}|\\
|\childdocforwardprefix[|\textit{main}|]{|\textit{prefix}|}{|\textit{dest}|}|
\end{tabular}
\end{center}
%
the destination file is determined by a pattern
depending on the current file:
To make this work, the current file must be called
`{\textit{prefix}\hspace{0.2em}\textit{suffix}}'
with \textit{prefix} matching precisely the argument.
Processing is then passed on to the file
`{\textit{dest}\hspace{0.2em}\textit{suffix}}'.
Surely, the same effect is achieved by
directly specifying the
argument `{\textit{dest}\hspace{0.2em}\textit{suffix}}'
in the first form.
However, that requires to set up a different file
for each child. With the alternative form of the command
all these files can have exactly the same content
which simplifies setting them up and maintaining them.

For example, the following file |draft.tex|
with a compilation flag |\version| as described in \secref{sec:flags}
compiles the main document as a draft:
%
\begin{center}
\begin{tabular}{l}
|\def\version{draft}|\\
|\input{childdoc.def}|\\
|\childdocforward{|\textit{main}|}|
\end{tabular}
\end{center}
%
Likewise, the following files |final|\textit{nn}|.tex|
compile the final version of the child document
|child|\textit{nn}|.tex|:
%
\begin{center}
\begin{tabular}{l}
|\def\version{final}|\\
|\input{childdoc.def}|\\
|\childdocforwardprefix{final}{child}|
\end{tabular}
\end{center}
%

Note that when several versions of a main file and/or of each child file
are to be generated, it may be convenient to set up a |Makefile| or
shell script to automatise the process.

%%%%%%%%%%%%%%%%%%%%%%%%%%%%%%%%%%%%%%%%%%%%%%%%%%%%%%%%%%%%%%%%%%%%%%%%%%%%%%%%
\subsection{Command Line Processing}
\label{sec:commandline}

The effect of redirection files can also be achieved by invoking
the \LaTeX{} compiler with a more elaborate command line.
Most conveniently this should be done as part
of a shell script or a |Makefile|.

When using \textsf{childdoc} in the main file, the following
command lines effectively perform a redirection
(note that depending on the shell being used,
backslashes may have to be doubled: `|\|' $\to$ `|\\|'):
%
\begin{center}
|... -jobname "|\textit{target}|" |\\|"|[\textit{flags}]%
|\input{childdoc.def}\childdocforward[|\textit{main}|]{|\textit{dest}|}"|
\end{center}
%
Here \textit{target} is the name of the output file,
\textit{main} is the name of the main file
and \textit{dest} is the name of the main or child file to be processed
(all filenames without extensions).
The optional argument \textit{main} can be omitted
if \textit{main} matches \textit{dest}.
Optionally, compilation \textit{flags} can be defined via |\def| commands.
This command line makes the \TeX{} engine believe
it is compiling the file \textit{target}
whose content is specified as the latter parameter.
The provided code then forwards the processing to
\textit{main} or \textit{dest} as described in \secref{sec:forward}.

%%%%%%%%%%%%%%%%%%%%%%%%%%%%%%%%%%%%%%%%%%%%%%%%%%%%%%%%%%%%%%%%%%%%%%%%%%%%%%%%
\subsection{Include by Input}
\label{sec:input}

Including child documents by |\include| has some restrictions by design.
Most notably, the content of a child document always occupies
its own set of pages; pages cannot be shared between child documents.
Usually, this behaviour makes perfect sense
because each child document contain an essential part of the document.
However, in some situations it may be desirable to compose
a document from a collection of parts
without having mandatory page breaks between then.
For this case, the package
provides a mechanism to include parts
by |\input| which can also be processed individually.
However, by construction this mechanism
requires manual handling of the content to be output.

%%%%%%%%%%%%%%%%%%%%%%%%%%%%%%%%%%%%%%%%
\DescribeMacro{\ifchilddocmanual}
The main file should be prepared as usual, see \secref{sec:include}.
However, the document body must make a distinction
between processing of an individual part and of the main document, e.g.:
%
\begin{center}
\begin{tabular}{l}
|\ifchilddocmanual|\\
|\input{\childdocname}|\\
|\||else|\\
\textit{document body with }|\input{|\textit{part}|}|\\
|\||fi|
\end{tabular}
\end{center}
%
The conditional |\ifchilddocmanual| is true whenever
a part to be included by |\input| is being compiled,
and the name of the part is stored in |\childdocname|.

%%%%%%%%%%%%%%%%%%%%%%%%%%%%%%%%%%%%%%%%
\DescribeMacro{\childdocby}
Each part to be included by |\input| should start with:
%
\begin{center}
\begin{tabular}{l}
|\input{childdoc.def}|\\
|\childdocby{|\textit{main}|}|\\
\end{tabular}
\end{center}
%
The directive |\childdocby| is similar to |\childdocof|
described in \secref{sec:include},
but the subsequent selection of content must be done manually.
To that end, both |\ifchilddoc| and |\ifchilddocmanual|
will be true upon processing of a part,
and the name of the part is stored in |\childdocname|.
Note that |\jobname| will be set to the filename of the current part
so that each part receives an individual |.aux| file
that does not interfere with the |.aux| file(s) of the main document.
This behaviour can be altered by the alternative form
|\childdocby[*]{|\textit{main}|}| (with a non-empty optional argument)
which uses the |.aux| file of the main document
by setting |\jobname| to \textit{main}.

%%%%%%%%%%%%%%%%%%%%%%%%%%%%%%%%%%%%%%%%%%%%%%%%%%%%%%%%%%%%%%%%%%%%%%%%%%%%%%%%
\subsection{Driver Development}
\label{sec:driver}

The \textsf{childdoc} mechanism can also be use for the development
of definition files such as \LaTeX{} styles or classes.
This case differs from the above setup with multiple parts
included by |\include| in that no |\includeonly| should be invoked.
This can be achieved by starting the include file
(before |\ProvidesPackage|) with:
%
\begin{center}
\begin{tabular}{l}
|\input{childdoc.def}|\\
|\childdocforward{|\textit{main}|}|\\
\end{tabular}
\end{center}
%
or alternatively with:
%
\begin{center}
\begin{tabular}{l}
|\input{childdoc.def}|\\
|\childdocby{|\textit{main}|}|\\
\end{tabular}
\end{center}
%
Both forms have slightly different effects as described above.
The main file is prepared as usual, see \secref{sec:include}.

%%%%%%%%%%%%%%%%%%%%%%%%%%%%%%%%%%%%%%%%%%%%%%%%%%%%%%%%%%%%%%%%%%%%%%%%%%%%%%%%
\subsection{Legacy Detection}
\label{sec:detection}

The directive |\childdocmain| in the main file can detect
whether the complete document or merely a child is to be compiled
even without using the directive |\childdocof|.
This method is deprecated because it is less robust
and there is no compelling reason to use it;
it is merely provided for backward compatibility
and it may be removed in future versions.

If the detection mechanism is to be used,
it is mandatory to correctly specify
the filename of the main file as the argument of |\childdocmain|:
%
\begin{center}
\begin{tabular}{l}
|\input{childdoc.def}|\\
|\childdocmain{|\textit{main}|}|\\
\end{tabular}
\end{center}
%
If |\jobname| does not match the argument \textit{main} of |\childdocmain|,
it is assumed that |\jobname| points to the child file to be compiled.
When using |\childdocmain| with the main file specified as argument,
it suffices to start a child file
with just |\input{|\textit{main}|}|
without loading of the package and using |\childdocof|.
If instead all processing is done
with the appropriate \textsf{childdoc} directives,
the argument of \textit{main} of |\childdocmain| can be empty.

An alternative version of the command line processing described
in \secref{sec:commandline} using the detection mechanism reads:
%
\begin{center}
|... -jobname "|\textit{target}|" "|[\textit{flags}]%
[|\def\jobname{|\textit{dest}|}|]|\input{|\textit{main}|}"|
\end{center}

%%%%%%%%%%%%%%%%%%%%%%%%%%%%%%%%%%%%%%%%%%%%%%%%%%%%%%%%%%%%%%%%%%%%%%%%%%%%%%%%
\subsection{Manual Code}
\label{sec:manual}

In case one cannot be certain whether the definitions file |childdoc.def|
is installed on the target \TeX{} distribution
and one prefers not to ship it,
it is conceivable to paste a few relevant commands into the sources.

To that end, drop all statements |\input{childdoc.def}|
and perform the replacements as outlined below.
Instead of |\childdocmain{|\textit{main}|}| add the following code
to the top of the main file:
%
\begin{center}
\begin{tabular}{l}
|\||ifdefined\childdocname\endinput\||fi\newif\ifchilddoc|\\
|\edef\childdocname{\scantokens\expandafter{\jobname\noexpand}}|\\
|\def\childdocmain{|\textit{main}|}\||ifx\childdocmain\childdocname\||else|\\
|\childdoctrue\includeonly{\childdocname}\let\jobname\childdocmain\||fi|\\
\end{tabular}
\end{center}
%
Instead of |\childdocof{|\textit{main}|}| just include the main file
at the top of each child file:
%
\begin{center}
|\input{|\textit{main}|}|
\end{center}
%
A simple redirection |\childdocforward{|\textit{dest}|}| is achieved by:
%
\begin{center}
|\def\jobname{|\textit{dest}|}\input{\jobname}|
\end{center}
%
The redirection with prefix
|\childdocforwardprefix[|\textit{prefix}|]{|\textit{dest}|}|
is accomplished by:
%
\begin{center}
\begin{tabular}{l}
|{\edef\jobname{\scantokens\expandafter{\jobname\noexpand}}|\\
|\def\redirectjob |\textit{prefix}|#1~~~{\gdef\jobname{|\textit{dest}|#1}}|\\
|\expandafter\redirectjob\jobname~~~}\input{\jobname}|
\end{tabular}
\end{center}

In an alternative approach,
child documents can be compiled by a specific command line
without additional code or specific definitions:
%
\begin{center}
|... -jobname "|\textit{target}|" "|[\textit{flags}]%
|\includeonly{|\textit{dest}|}\input{|\textit{main}|}"|
\end{center}
%

%%%%%%%%%%%%%%%%%%%%%%%%%%%%%%%%%%%%%%%%%%%%%%%%%%%%%%%%%%%%%%%%%%%%%%%%%%%%%%%%
%%%%%%%%%%%%%%%%%%%%%%%%%%%%%%%%%%%%%%%%%%%%%%%%%%%%%%%%%%%%%%%%%%%%%%%%%%%%%%%%
\section{Information}

%%%%%%%%%%%%%%%%%%%%%%%%%%%%%%%%%%%%%%%%%%%%%%%%%%%%%%%%%%%%%%%%%%%%%%%%%%%%%%%%
\subsection{Copyright}

Copyright \copyright{} 2017--2018 Niklas Beisert

This work may be distributed and/or modified under the
conditions of the \LaTeX{} Project Public License, either version 1.3
of this license or (at your option) any later version.
The latest version of this license is in
  \url{http://www.latex-project.org/lppl.txt}
and version 1.3 or later is part of all distributions of \LaTeX{}
version 2005/12/01 or later.

This work has the LPPL maintenance status `maintained'.

The Current Maintainer of this work is Niklas Beisert.

This work consists of the files |README.txt|, |childdoc.ins| and |childdoc.dtx|
as well as the derived files |childdoc.def|, |cdocsamp.tex|
with |cdocsch1.tex|, |cdocsch2.tex|, |cdocspt3.tex|, |cdocspt4.tex|,
|cdocsdrf.tex|, |cdocsfn1.tex|, |cdocsfn2.tex|
as well as |childdoc.pdf|.

%%%%%%%%%%%%%%%%%%%%%%%%%%%%%%%%%%%%%%%%%%%%%%%%%%%%%%%%%%%%%%%%%%%%%%%%%%%%%%%%
\subsection{Files and Installation}

The package consists of the files:
%
\begin{center}
\begin{tabular}{ll}
    |README.txt|   & readme file \\
    |childdoc.ins| & installation file \\
    |childdoc.dtx| & source file \\
    |childdoc.def| & definition file \\
    |cdocsamp.tex| & sample main file \\
    |cdocsch1.tex| & sample include file \\
    |cdocsch2.tex| & sample include file \\
    |cdocspt3.tex| & sample part file \\
    |cdocspt4.tex| & sample part file \\
    |cdocsdrf.tex| & sample redirection file \\
    |cdocsfn1.tex| & sample redirection file \\
    |cdocsfn2.tex| & sample redirection file \\
    |childdoc.pdf| & manual
\end{tabular}
\end{center}
%
The distribution consists of the files
|README.txt|, |childdoc.ins| and |childdoc.dtx|.
%
\begin{itemize}
\item
Run (pdf)\LaTeX{} on |childdoc.dtx|
to compile the manual |childdoc.pdf| (this file).
\item
Run \LaTeX{} on |childdoc.ins| to create the definitions file |childdoc.def|
and the sample |cdocsamp.tex| with include files
|cdocsch1.tex|, |cdocsch2.tex|, |cdocspt3.tex|, |cdocspt4.tex|,
|cdocsdrf.tex|, |cdocsfn1.tex|, |cdocsfn2.tex|.
Then copy the file |childdoc.def| to an appropriate directory of your \LaTeX{}
distribution, e.g.\ \textit{texmf-root}|/tex/latex/childdoc|.
\end{itemize}

%%%%%%%%%%%%%%%%%%%%%%%%%%%%%%%%%%%%%%%%%%%%%%%%%%%%%%%%%%%%%%%%%%%%%%%%%%%%%%%%
\subsection{Related CTAN Packages}

There are several other packages which offer a similar functionality:
%
\begin{itemize}
\item
The packages
\href{http://ctan.org/pkg/docmute}{\textsf{docmute}},
\href{http://ctan.org/pkg/includex}{\textsf{includex}} and
\href{http://ctan.org/pkg/standalone}{\textsf{standalone}}
provide commands to include only the document body of
a child file thus allowing both files to be compiled individually.
\item
The packages \href{http://ctan.org/pkg/subdocs}{\textsf{subdocs}}
and \href{http://ctan.org/pkg/subfiles}{\textsf{subfiles}}
provide structures in which the main and child documents can be
encapsulated and allowing them to be compiled individually.
The inclusion mechanism is different from the conventional |\include|.
\item
The package \href{http://ctan.org/pkg/combine}{\textsf{combine}}
is an elaborate solution to combine several documents into one.
\end{itemize}
%
See also the CTAN topic \href{http://ctan.org/topic/subdocs}{\textsf{subdocs}}
for further related packages.
The present package differs from the above solutions in that
a document structure constructed with the conventional |\include| mechanism
just needs two extra commands at the top of every file
such that all constituent files can be compiled individually.

%%%%%%%%%%%%%%%%%%%%%%%%%%%%%%%%%%%%%%%%%%%%%%%%%%%%%%%%%%%%%%%%%%%%%%%%%%%%%%%%
%\subsection{Feature Suggestions}
%
%The following is a list of features which may be useful for future
%versions of this package:
%%
%\begin{itemize}
%\item
%\ldots
%\end{itemize}

%%%%%%%%%%%%%%%%%%%%%%%%%%%%%%%%%%%%%%%%%%%%%%%%%%%%%%%%%%%%%%%%%%%%%%%%%%%%%%%%
\subsection{Revision History}

%%%%%%%%%%%%%%%%%%%%%%%%%%%%%%%%%%%%%%%%
\paragraph{v2.0:} 2018/12/30

\begin{itemize}
\item
immediate forward processing
\item
added |\childdocby| mechanism
\item
manual restructured
\end{itemize}

%%%%%%%%%%%%%%%%%%%%%%%%%%%%%%%%%%%%%%%%
\paragraph{v1.6:} 2018/01/17

\begin{itemize}
\item
application for development of include files
\item
corrections to manual
\end{itemize}

%%%%%%%%%%%%%%%%%%%%%%%%%%%%%%%%%%%%%%%%
\paragraph{v1.5:} 2017/05/21

\begin{itemize}
\item
more complete structuring introduced
\item
|\childdocof| introduced
\item
|\childdoc| renamed to |\childdocmain|
\item
|\childredirect| renamed to |\childdocforward| and |\childdocforwardprefix|
and functionality expanded
\end{itemize}

%%%%%%%%%%%%%%%%%%%%%%%%%%%%%%%%%%%%%%%%
\paragraph{v1.0:} 2017/04/27

\begin{itemize}
\item
manual and install package
\item
first version published on CTAN
\end{itemize}

%%%%%%%%%%%%%%%%%%%%%%%%%%%%%%%%%%%%%%%%
\paragraph{v0.6:} 2017/04/26

\begin{itemize}
\item
redirection mechanism added
\end{itemize}

%%%%%%%%%%%%%%%%%%%%%%%%%%%%%%%%%%%%%%%%
\paragraph{v0.5:} 2017/04/26

\begin{itemize}
\item
functionality in definition file
\end{itemize}


%%%%%%%%%%%%%%%%%%%%%%%%%%%%%%%%%%%%%%%%%%%%%%%%%%%%%%%%%%%%%%%%%%%%%%%%%%%%%%%%
%%%%%%%%%%%%%%%%%%%%%%%%%%%%%%%%%%%%%%%%%%%%%%%%%%%%%%%%%%%%%%%%%%%%%%%%%%%%%%%%
%%%%%%%%%%%%%%%%%%%%%%%%%%%%%%%%%%%%%%%%%%%%%%%%%%%%%%%%%%%%%%%%%%%%%%%%%%%%%%%%
\appendix

\settowidth\MacroIndent{\rmfamily\scriptsize 000\ }

 \DocInput{childdoc.dtx}

\end{document}
%</driver>
% \fi
%
% %%%%%%%%%%%%%%%%%%%%%%%%%%%%%%%%%%%%%%%%%%%%%%%%%%%%%%%%%%%%%%%%%%%%%%%%%%%%%%
% %%%%%%%%%%%%%%%%%%%%%%%%%%%%%%%%%%%%%%%%%%%%%%%%%%%%%%%%%%%%%%%%%%%%%%%%%%%%%%
% \section{Sample}
%\iffalse
%<*samplemain>
%\fi
%
% The following presents a sample document
% with two chapters, two parts, a title page,
% a compile flag as well as three forwarding files to set the flag.
% It consists of eight |.tex| files:
% \begin{center}
% \begin{tabular}{ll}
% |cdocsamp.tex|&main file\\
% |cdocsch1.tex|&include file for chapter 1\\
% |cdocsch2.tex|&include file for chapter 2\\
% |cdocspt3.tex|&include file for part 3\\
% |cdocspt4.tex|&include file for part 4\\
% |cdocsdrf.tex|&forwarding file for main file in draft mode\\
% |cdocsfi1.tex|&forwarding file for final version of chapter 1\\
% |cdocsfi2.tex|&forwarding file for final version of chapter 2\\
% \end{tabular}
% \end{center}
% Each of the eight files can be compiled directly by the \LaTeX{} compiler.
%
% %%%%%%%%%%%%%%%%%%%%%%%%%%%%%%%%%%%%%%
% \paragraph{Main File.}
%
% The main file is called |cdocsamp.tex|.
%
% Load the \textsf{childdoc} definitions and
% declare the filename for the main document:
%    \begin{macrocode}
\input{childdoc.def}
\childdocmain{}
%    \end{macrocode}

% Optional override for |\version| flag:
%    \begin{macrocode}
%%\ifchilddoc\else\providecommand{\version}{draft}\fi
%    \end{macrocode}

% Define the default values for the |\version| flag
% (|final| for the main file and |draft| for childs):
%    \begin{macrocode}
\ifchilddoc
\providecommand{\version}{draft}
\else
\providecommand{\version}{final}
\fi
%    \end{macrocode}

% Load the standard document class:
%    \begin{macrocode}
\documentclass[12pt]{article}
%    \end{macrocode}

% Start the document body:
%    \begin{macrocode}
\begin{document}
%    \end{macrocode}

% Declare a title page.
% Print title, part of document being processed and version flag:
%    \begin{macrocode}
\addtocounter{page}{-1}
\begin{center}
{\LARGE\bfseries{}childdoc example\par}
\vspace{1cm}
\ifchilddoc
\ifchilddocmanual part\else chapter\fi:
`\childdocname' of `\childdocjob'\par
\else
main document: `\childdocjob'\par
\fi
version: \version\par
\end{center}
\newpage
%    \end{macrocode}

% Manually include selected file,
% otherwise process as usual:
%    \begin{macrocode}
\ifchilddocmanual
\section*{part `\childdocname'}
\input{\childdocname}
\else
%    \end{macrocode}

% Include the two chapters:
%    \begin{macrocode}
\include{cdocsch1}
\include{cdocsch2}
%    \end{macrocode}

% Include the two parts unless only chapters should be displayed:
%    \begin{macrocode}
\ifchilddoc\else
\section{part three}
\input{cdocspt3}
\section{part four}
\input{cdocspt4}
\fi
%    \end{macrocode}

% Process as usual until here:
%    \begin{macrocode}
\fi
%    \end{macrocode}

% End of document body:
%    \begin{macrocode}
\end{document}
%    \end{macrocode}
%\iffalse
%</samplemain>
%\fi
%
% %%%%%%%%%%%%%%%%%%%%%%%%%%%%%%%%%%%%%%
% \paragraph{Chapter Include Files.}
%
% The include files are called |cdocsch1.tex| and |cdocsch2.tex|.
%
%\iffalse
%<*samplechap1|samplechap2>
%\fi

% Optional override for |\version| flag:
%    \begin{macrocode}
%%\providecommand{\version}{final}
%    \end{macrocode}

% Include the main document:
%    \begin{macrocode}
\input{childdoc.def}
\childdocof{cdocsamp}
%    \end{macrocode}

%\iffalse
%</samplechap1|samplechap2>
%\fi
%
%\iffalse
%<*samplechap1>
%\fi
% Some text for chapter 1:
%    \begin{macrocode}
\section{one}
some text in chapter one
%    \end{macrocode}

%\iffalse
%</samplechap1>
%\fi
% Some text for chapter 2:
%\iffalse
%<*samplechap2>
%\fi
%    \begin{macrocode}
\section{two}
more text in chapter two
%    \end{macrocode}

%\iffalse
%</samplechap2>
%\fi
%
% %%%%%%%%%%%%%%%%%%%%%%%%%%%%%%%%%%%%%%
% \paragraph{Part Include Files.}
%
% The include files are called |cdocspt3.tex| and |cdocspt4.tex|.
%
%\iffalse
%<*samplepart3|samplepart4>
%\fi

% Optional override for |\version| flag:
%    \begin{macrocode}
%%\providecommand{\version}{final}
%    \end{macrocode}

% Include the main document:
%    \begin{macrocode}
\input{childdoc.def}
\childdocby{cdocsamp}
%    \end{macrocode}

%\iffalse
%</samplepart3|samplepart4>
%\fi
%
%\iffalse
%<*samplepart3>
%\fi
% Some text for part 3:
%    \begin{macrocode}
some text in part three
%    \end{macrocode}

%\iffalse
%</samplepart3>
%\fi
% Some text for part 4:
%\iffalse
%<*samplepart4>
%\fi
%    \begin{macrocode}
more text in part four
%    \end{macrocode}

%\iffalse
%</samplepart4>
%\fi
%
% %%%%%%%%%%%%%%%%%%%%%%%%%%%%%%%%%%%%%%
% \paragraph{Forwarding for a Complete Draft.}
%
% The following forwarding file |cdocsdrf.tex|
% compiles the main document in draft mode:
%\iffalse
%<*sampledraft>
%\fi
%    \begin{macrocode}
\def\version{draft}
\input{childdoc.def}
\childdocforward{cdocsamp}
%    \end{macrocode}

%\iffalse
%</sampledraft>
%\fi
%
% %%%%%%%%%%%%%%%%%%%%%%%%%%%%%%%%%%%%%%
% \paragraph{Forwarding for Final Version of the Chapters.}
%
% The following forwarding files |cdocsfn1.tex| and |cdocsfn2.tex|
% (with identical content)
% compile the final versions of the child documents
% |cdocsch1.tex| and |cdocsch2.tex|, respectively:
%\iffalse
%<*samplefinal>
%\fi
%    \begin{macrocode}
\def\version{final}
\input{childdoc.def}
\childdocforwardprefix[cdocsamp]{cdocsfn}{cdocsch}
%    \end{macrocode}

%\iffalse
%</samplefinal>
%\fi
%
% %%%%%%%%%%%%%%%%%%%%%%%%%%%%%%%%%%%%%%
% \paragraph{Command Line Processing.}
%
% The following three command lines generate the output files
% |cdocscld|, |cdocscl1| and |cdocscl2|
% which should be identical to
% |cdocsdrf|, |cdocsch1| and |cdocsfn2|, respectively:
% \begin{center}
% \begin{tabular}{l}
% |latex -jobname cdocscld \|\\
% |  "\def\version{draft}\input{childdoc.def}\childdocforward{cdocsamp}"|\\
% |latex -jobname cdocscl1 \|\\
% |  "\input{childdoc.def}\childdocforward[cdocsamp]{cdocsch1}"|\\
% |latex -jobname cdocscl2 \|\\
% |  "\def\version{final}\input{childdoc.def}\childdocforward{cdocsch2}"|
% \end{tabular}
% \end{center}
% Note that the trailing backslash on each first line
% merely continues the input to the second line
% (for convenient cut ant paste).
% Furthermore, the command |latex| can be replaced by any
% of its alternative versions such as |pdflatex|.
%
% %%%%%%%%%%%%%%%%%%%%%%%%%%%%%%%%%%%%%%%%%%%%%%%%%%%%%%%%%%%%%%%%%%%%%%%%%%%%%%
% %%%%%%%%%%%%%%%%%%%%%%%%%%%%%%%%%%%%%%%%%%%%%%%%%%%%%%%%%%%%%%%%%%%%%%%%%%%%%%
% \section{Implementation}
%\iffalse
%<*package>
%\fi
%
% This section describes the definitions file |childdoc.def|.

% The definitions cannot be loaded using |\usepackage| or |\RequirePackage|
% which has a mechanism to prevent loading a style file more than once.
% When loading the definitions by means of |\input|
% multiple instances have to be prevented manually:
%\iffalse
%This code needs to be before the `\ProvidesFile' directive
%which is defined at the beginning of this file.
%Therefore it is also placed there and commented out here.
%</package>
%<*discard>
%\fi
%    \begin{macrocode}
\ifdefined\childdocmain\endinput\fi
%    \end{macrocode}
%\iffalse
%</discard>
%<*package>
%\fi
%
% \macro{\ifchilddoc}
% \macro{\ifchilddocmanual}
% The conditional |\ifchilddoc| tells whether a
% child (true) or main (false) document is being compiled.
% The conditional |\ifchilddocmanual| tells whether
% the |\includeonly| mechanism is used (false) or
% the selection of child files must be performed manually (true).
% The definitions initialise to false:
%    \begin{macrocode}
\newif\ifchilddoc
\newif\ifchilddocmanual
%    \end{macrocode}

% \macro{\childdocname}
% \macro{\childdocjob}
% The macro |\childdocname| stores the name of the main document
% to be compiled. The macro |\childdocjob| stores the name of
% the document on which the \LaTeX{} compiler was originally invoked.
% The content of |\jobname| cannot be compared
% to filenames specified in the source due to different catcodes.
% The following code rescans |\jobname|, stores the result
% in |\childdocname| and saves a copy in |\childdocjob|:
%    \begin{macrocode}
\edef\childdocname{\scantokens\expandafter{\jobname\noexpand}}
\let\childdocjob\childdocname
%    \end{macrocode}

% \macro{\childdocdisable}
% The macro |\childdocdisable| prevents the main file
% from being processed more than once.
% At this stage, the main document command |\childdocmain|
% is assumed to be called once again where it should do nothing.
% Any subsequent call to it should prevent
% a secondary processing of the main document
% It overwrites the forwarding commands
% |\childdocof| and |\childdocforward|
% with empty macros to prevent further inclusions of the main document:
%    \begin{macrocode}
\newcommand{\childdocdisable}
{
  \renewcommand{\childdocmain}[1]{\renewcommand{\childdocmain}[1]{\endinput}}
  \renewcommand{\childdocof}[1]{}
  \renewcommand{\childdocby}[2][]{}
  \renewcommand{\childdocforward}[2][]{}
  \renewcommand{\childdocdisable}{}
}
%    \end{macrocode}

% \macro{\childdocmain}
% The macro |\childdocmain| is to be called at the top of the main file
% with nothing or the main filename (without extension) as argument.
% First, it breaks loops.
% If the argument is not empty and does not match |\childdocname|
% (which is set by the first inclusion of |childdoc.def|),
% |\ifchilddoc| is set to true, |\includeonly| is applied to the child file
% and |\jobname| is set to the main file
% (for proper handling of |.aux| files):
%    \begin{macrocode}
\newcommand{\childdocmain}[1]
{
  \childdocdisable\childdocmain{}
  \if?#1?\else
    \begingroup
      \def\childdoctmp{#1}
      \ifx\childdoctmp\childdocname
        \def\childdoctmp{}
      \else
        \def\childdoctmp
        {
          \childdoctrue
          \includeonly{\childdocname}
          \def\childdocjob{#1}
          \def\jobname{#1}
        }
      \fi
      \expandafter
    \endgroup
    \childdoctmp
  \fi
}
%    \end{macrocode}

% \macro{\childdocof}
% The command |\childdocof| redirects
% compilation to the main file |#1|.
%    \begin{macrocode}
\newcommand{\childdocof}[1]
{
  \childdocdisable
  \childdoctrue
  \includeonly{\childdocname}
  \def\jobname{#1}
  \def\childdocjob{#1}
  \input{#1}
}
%    \end{macrocode}

% \macro{\childdocby}
% The command |\childdocby| ....
%    \begin{macrocode}
\newcommand{\childdocby}[2][]
{
  \childdocdisable
  \childdoctrue
  \childdocmanualtrue
  \if?#1?\else
    \def\jobname{#2}
  \fi
  \def\childdocjob{#2}
  \input{#2}
  \endinput
}
%    \end{macrocode}

% \macro{\childdocforward}
% The command |\childdocforward| redirects
% compilation to the main file or
% (if the optional argument is given) a child file.
% Parameters are set as if the main file
% or a child file starting with |\childdocof| was compiled.
% Then compilation is handed over to the main file:
%    \begin{macrocode}
\newcommand{\childdocforward}[2][]
{
  \begingroup
    \if?#1?
      \def\childdoctmp
      {
        \def\childdocname{#2}
        \def\childdocjob{#2}
        \def\jobname{#2}
        \input{#2}
        \endinput
      }
    \else
      \def\childdoctmp
      {
        \childdocdisable
        \def\childdocname{#2}
        \childdoctrue
        \includeonly{#2}
        \def\childdocjob{#1}
        \def\jobname{#1}
        \input{#1}
        \endinput
      }
    \fi
    \expandafter
  \endgroup
  \childdoctmp
}
%    \end{macrocode}

% \macro{\childdocforwardprefix}
% The command |\childdocforwardprefix| redirects
% compilation to the main or a child file by means of a pattern.
% The prefix |#1| in the current filename is replaced by |#2|
% and the suffix of the current filename is kept
% (it is assumed that the filename does not contain the substring `|~~~|'
% which is used as a delimiter).
% Compilation is handed over to the new file by |\childdocforward|:
%    \begin{macrocode}
\newcommand{\childdocforwardprefix}[3][]
{
  \begingroup
    \def\childdocextract #2##1~~~{\def\childdoctmp{\childdocforward[#1]{#3##1}}}
    \expandafter\childdocextract\childdocname~~~
    \expandafter
  \endgroup
  \childdoctmp
}
%    \end{macrocode}

% \macro{\childdoc}
% The deprecated macro |\childdoc| is a legacy version of |\childdocmain|:
%    \begin{macrocode}
\newcommand{\childdoc}{\childdocmain}
%    \end{macrocode}

% \macro{\childdocredirect}
% The deprecated macro |\childdocredirect| is a legacy version
% of |\childdocforward| and |\childdocforwardprefix|:
%    \begin{macrocode}
\newcommand{\childdocredirect}[2][]
{
  \begingroup
    \if?#1?
      \def\childdoctmp{\childdocforward{#2}}
    \else
      \def\childdoctmp{\childdocforwardprefix{#1}{#2}}
    \fi
    \expandafter
  \endgroup
  \childdoctmp
}
%    \end{macrocode}

%\iffalse
%</package>
%\fi
%
\endinput
\childdocforward{cdocsch2}"|
% \end{tabular}
% \end{center}
% Note that the trailing backslash on each first line
% merely continues the input to the second line
% (for convenient cut ant paste).
% Furthermore, the command |latex| can be replaced by any
% of its alternative versions such as |pdflatex|.
%
% %%%%%%%%%%%%%%%%%%%%%%%%%%%%%%%%%%%%%%%%%%%%%%%%%%%%%%%%%%%%%%%%%%%%%%%%%%%%%%
% %%%%%%%%%%%%%%%%%%%%%%%%%%%%%%%%%%%%%%%%%%%%%%%%%%%%%%%%%%%%%%%%%%%%%%%%%%%%%%
% \section{Implementation}
%\iffalse
%<*package>
%\fi
%
% This section describes the definitions file |childdoc.def|.

% The definitions cannot be loaded using |\usepackage| or |\RequirePackage|
% which has a mechanism to prevent loading a style file more than once.
% When loading the definitions by means of |\input|
% multiple instances have to be prevented manually:
%\iffalse
%This code needs to be before the `\ProvidesFile' directive
%which is defined at the beginning of this file.
%Therefore it is also placed there and commented out here.
%</package>
%<*discard>
%\fi
%    \begin{macrocode}
\ifdefined\childdocmain\endinput\fi
%    \end{macrocode}
%\iffalse
%</discard>
%<*package>
%\fi
%
% \macro{\ifchilddoc}
% \macro{\ifchilddocmanual}
% The conditional |\ifchilddoc| tells whether a
% child (true) or main (false) document is being compiled.
% The conditional |\ifchilddocmanual| tells whether
% the |\includeonly| mechanism is used (false) or
% the selection of child files must be performed manually (true).
% The definitions initialise to false:
%    \begin{macrocode}
\newif\ifchilddoc
\newif\ifchilddocmanual
%    \end{macrocode}

% \macro{\childdocname}
% \macro{\childdocjob}
% The macro |\childdocname| stores the name of the main document
% to be compiled. The macro |\childdocjob| stores the name of
% the document on which the \LaTeX{} compiler was originally invoked.
% The content of |\jobname| cannot be compared
% to filenames specified in the source due to different catcodes.
% The following code rescans |\jobname|, stores the result
% in |\childdocname| and saves a copy in |\childdocjob|:
%    \begin{macrocode}
\edef\childdocname{\scantokens\expandafter{\jobname\noexpand}}
\let\childdocjob\childdocname
%    \end{macrocode}

% \macro{\childdocdisable}
% The macro |\childdocdisable| prevents the main file
% from being processed more than once.
% At this stage, the main document command |\childdocmain|
% is assumed to be called once again where it should do nothing.
% Any subsequent call to it should prevent
% a secondary processing of the main document
% It overwrites the forwarding commands
% |\childdocof| and |\childdocforward|
% with empty macros to prevent further inclusions of the main document:
%    \begin{macrocode}
\newcommand{\childdocdisable}
{
  \renewcommand{\childdocmain}[1]{\renewcommand{\childdocmain}[1]{\endinput}}
  \renewcommand{\childdocof}[1]{}
  \renewcommand{\childdocby}[2][]{}
  \renewcommand{\childdocforward}[2][]{}
  \renewcommand{\childdocdisable}{}
}
%    \end{macrocode}

% \macro{\childdocmain}
% The macro |\childdocmain| is to be called at the top of the main file
% with nothing or the main filename (without extension) as argument.
% First, it breaks loops.
% If the argument is not empty and does not match |\childdocname|
% (which is set by the first inclusion of |childdoc.def|),
% |\ifchilddoc| is set to true, |\includeonly| is applied to the child file
% and |\jobname| is set to the main file
% (for proper handling of |.aux| files):
%    \begin{macrocode}
\newcommand{\childdocmain}[1]
{
  \childdocdisable\childdocmain{}
  \if?#1?\else
    \begingroup
      \def\childdoctmp{#1}
      \ifx\childdoctmp\childdocname
        \def\childdoctmp{}
      \else
        \def\childdoctmp
        {
          \childdoctrue
          \includeonly{\childdocname}
          \def\childdocjob{#1}
          \def\jobname{#1}
        }
      \fi
      \expandafter
    \endgroup
    \childdoctmp
  \fi
}
%    \end{macrocode}

% \macro{\childdocof}
% The command |\childdocof| redirects
% compilation to the main file |#1|.
%    \begin{macrocode}
\newcommand{\childdocof}[1]
{
  \childdocdisable
  \childdoctrue
  \includeonly{\childdocname}
  \def\jobname{#1}
  \def\childdocjob{#1}
  \input{#1}
}
%    \end{macrocode}

% \macro{\childdocby}
% The command |\childdocby| ....
%    \begin{macrocode}
\newcommand{\childdocby}[2][]
{
  \childdocdisable
  \childdoctrue
  \childdocmanualtrue
  \if?#1?\else
    \def\jobname{#2}
  \fi
  \def\childdocjob{#2}
  \input{#2}
  \endinput
}
%    \end{macrocode}

% \macro{\childdocforward}
% The command |\childdocforward| redirects
% compilation to the main file or
% (if the optional argument is given) a child file.
% Parameters are set as if the main file
% or a child file starting with |\childdocof| was compiled.
% Then compilation is handed over to the main file:
%    \begin{macrocode}
\newcommand{\childdocforward}[2][]
{
  \begingroup
    \if?#1?
      \def\childdoctmp
      {
        \def\childdocname{#2}
        \def\childdocjob{#2}
        \def\jobname{#2}
        \input{#2}
        \endinput
      }
    \else
      \def\childdoctmp
      {
        \childdocdisable
        \def\childdocname{#2}
        \childdoctrue
        \includeonly{#2}
        \def\childdocjob{#1}
        \def\jobname{#1}
        \input{#1}
        \endinput
      }
    \fi
    \expandafter
  \endgroup
  \childdoctmp
}
%    \end{macrocode}

% \macro{\childdocforwardprefix}
% The command |\childdocforwardprefix| redirects
% compilation to the main or a child file by means of a pattern.
% The prefix |#1| in the current filename is replaced by |#2|
% and the suffix of the current filename is kept
% (it is assumed that the filename does not contain the substring `|~~~|'
% which is used as a delimiter).
% Compilation is handed over to the new file by |\childdocforward|:
%    \begin{macrocode}
\newcommand{\childdocforwardprefix}[3][]
{
  \begingroup
    \def\childdocextract #2##1~~~{\def\childdoctmp{\childdocforward[#1]{#3##1}}}
    \expandafter\childdocextract\childdocname~~~
    \expandafter
  \endgroup
  \childdoctmp
}
%    \end{macrocode}

% \macro{\childdoc}
% The deprecated macro |\childdoc| is a legacy version of |\childdocmain|:
%    \begin{macrocode}
\newcommand{\childdoc}{\childdocmain}
%    \end{macrocode}

% \macro{\childdocredirect}
% The deprecated macro |\childdocredirect| is a legacy version
% of |\childdocforward| and |\childdocforwardprefix|:
%    \begin{macrocode}
\newcommand{\childdocredirect}[2][]
{
  \begingroup
    \if?#1?
      \def\childdoctmp{\childdocforward{#2}}
    \else
      \def\childdoctmp{\childdocforwardprefix{#1}{#2}}
    \fi
    \expandafter
  \endgroup
  \childdoctmp
}
%    \end{macrocode}

%\iffalse
%</package>
%\fi
%
\endinput
|
and perform the replacements as outlined below.
Instead of |\childdocmain{|\textit{main}|}| add the following code
to the top of the main file:
%
\begin{center}
\begin{tabular}{l}
|\||ifdefined\childdocname\endinput\||fi\newif\ifchilddoc|\\
|\edef\childdocname{\scantokens\expandafter{\jobname\noexpand}}|\\
|\def\childdocmain{|\textit{main}|}\||ifx\childdocmain\childdocname\||else|\\
|\childdoctrue\includeonly{\childdocname}\let\jobname\childdocmain\||fi|\\
\end{tabular}
\end{center}
%
Instead of |\childdocof{|\textit{main}|}| just include the main file
at the top of each child file:
%
\begin{center}
|\input{|\textit{main}|}|
\end{center}
%
A simple redirection |\childdocforward{|\textit{dest}|}| is achieved by:
%
\begin{center}
|\def\jobname{|\textit{dest}|}\input{\jobname}|
\end{center}
%
The redirection with prefix
|\childdocforwardprefix[|\textit{prefix}|]{|\textit{dest}|}|
is accomplished by:
%
\begin{center}
\begin{tabular}{l}
|{\edef\jobname{\scantokens\expandafter{\jobname\noexpand}}|\\
|\def\redirectjob |\textit{prefix}|#1~~~{\gdef\jobname{|\textit{dest}|#1}}|\\
|\expandafter\redirectjob\jobname~~~}\input{\jobname}|
\end{tabular}
\end{center}

In an alternative approach,
child documents can be compiled by a specific command line
without additional code or specific definitions:
%
\begin{center}
|... -jobname "|\textit{target}|" "|[\textit{flags}]%
|\includeonly{|\textit{dest}|}\input{|\textit{main}|}"|
\end{center}
%

%%%%%%%%%%%%%%%%%%%%%%%%%%%%%%%%%%%%%%%%%%%%%%%%%%%%%%%%%%%%%%%%%%%%%%%%%%%%%%%%
%%%%%%%%%%%%%%%%%%%%%%%%%%%%%%%%%%%%%%%%%%%%%%%%%%%%%%%%%%%%%%%%%%%%%%%%%%%%%%%%
\section{Information}

%%%%%%%%%%%%%%%%%%%%%%%%%%%%%%%%%%%%%%%%%%%%%%%%%%%%%%%%%%%%%%%%%%%%%%%%%%%%%%%%
\subsection{Copyright}

Copyright \copyright{} 2017--2018 Niklas Beisert

This work may be distributed and/or modified under the
conditions of the \LaTeX{} Project Public License, either version 1.3
of this license or (at your option) any later version.
The latest version of this license is in
  \url{http://www.latex-project.org/lppl.txt}
and version 1.3 or later is part of all distributions of \LaTeX{}
version 2005/12/01 or later.

This work has the LPPL maintenance status `maintained'.

The Current Maintainer of this work is Niklas Beisert.

This work consists of the files |README.txt|, |childdoc.ins| and |childdoc.dtx|
as well as the derived files |childdoc.def|, |cdocsamp.tex|
with |cdocsch1.tex|, |cdocsch2.tex|, |cdocspt3.tex|, |cdocspt4.tex|,
|cdocsdrf.tex|, |cdocsfn1.tex|, |cdocsfn2.tex|
as well as |childdoc.pdf|.

%%%%%%%%%%%%%%%%%%%%%%%%%%%%%%%%%%%%%%%%%%%%%%%%%%%%%%%%%%%%%%%%%%%%%%%%%%%%%%%%
\subsection{Files and Installation}

The package consists of the files:
%
\begin{center}
\begin{tabular}{ll}
    |README.txt|   & readme file \\
    |childdoc.ins| & installation file \\
    |childdoc.dtx| & source file \\
    |childdoc.def| & definition file \\
    |cdocsamp.tex| & sample main file \\
    |cdocsch1.tex| & sample include file \\
    |cdocsch2.tex| & sample include file \\
    |cdocspt3.tex| & sample part file \\
    |cdocspt4.tex| & sample part file \\
    |cdocsdrf.tex| & sample redirection file \\
    |cdocsfn1.tex| & sample redirection file \\
    |cdocsfn2.tex| & sample redirection file \\
    |childdoc.pdf| & manual
\end{tabular}
\end{center}
%
The distribution consists of the files
|README.txt|, |childdoc.ins| and |childdoc.dtx|.
%
\begin{itemize}
\item
Run (pdf)\LaTeX{} on |childdoc.dtx|
to compile the manual |childdoc.pdf| (this file).
\item
Run \LaTeX{} on |childdoc.ins| to create the definitions file |childdoc.def|
and the sample |cdocsamp.tex| with include files
|cdocsch1.tex|, |cdocsch2.tex|, |cdocspt3.tex|, |cdocspt4.tex|,
|cdocsdrf.tex|, |cdocsfn1.tex|, |cdocsfn2.tex|.
Then copy the file |childdoc.def| to an appropriate directory of your \LaTeX{}
distribution, e.g.\ \textit{texmf-root}|/tex/latex/childdoc|.
\end{itemize}

%%%%%%%%%%%%%%%%%%%%%%%%%%%%%%%%%%%%%%%%%%%%%%%%%%%%%%%%%%%%%%%%%%%%%%%%%%%%%%%%
\subsection{Related CTAN Packages}

There are several other packages which offer a similar functionality:
%
\begin{itemize}
\item
The packages
\href{http://ctan.org/pkg/docmute}{\textsf{docmute}},
\href{http://ctan.org/pkg/includex}{\textsf{includex}} and
\href{http://ctan.org/pkg/standalone}{\textsf{standalone}}
provide commands to include only the document body of
a child file thus allowing both files to be compiled individually.
\item
The packages \href{http://ctan.org/pkg/subdocs}{\textsf{subdocs}}
and \href{http://ctan.org/pkg/subfiles}{\textsf{subfiles}}
provide structures in which the main and child documents can be
encapsulated and allowing them to be compiled individually.
The inclusion mechanism is different from the conventional |\include|.
\item
The package \href{http://ctan.org/pkg/combine}{\textsf{combine}}
is an elaborate solution to combine several documents into one.
\end{itemize}
%
See also the CTAN topic \href{http://ctan.org/topic/subdocs}{\textsf{subdocs}}
for further related packages.
The present package differs from the above solutions in that
a document structure constructed with the conventional |\include| mechanism
just needs two extra commands at the top of every file
such that all constituent files can be compiled individually.

%%%%%%%%%%%%%%%%%%%%%%%%%%%%%%%%%%%%%%%%%%%%%%%%%%%%%%%%%%%%%%%%%%%%%%%%%%%%%%%%
%\subsection{Feature Suggestions}
%
%The following is a list of features which may be useful for future
%versions of this package:
%%
%\begin{itemize}
%\item
%\ldots
%\end{itemize}

%%%%%%%%%%%%%%%%%%%%%%%%%%%%%%%%%%%%%%%%%%%%%%%%%%%%%%%%%%%%%%%%%%%%%%%%%%%%%%%%
\subsection{Revision History}

%%%%%%%%%%%%%%%%%%%%%%%%%%%%%%%%%%%%%%%%
\paragraph{v2.0:} 2018/12/30

\begin{itemize}
\item
immediate forward processing
\item
added |\childdocby| mechanism
\item
manual restructured
\end{itemize}

%%%%%%%%%%%%%%%%%%%%%%%%%%%%%%%%%%%%%%%%
\paragraph{v1.6:} 2018/01/17

\begin{itemize}
\item
application for development of include files
\item
corrections to manual
\end{itemize}

%%%%%%%%%%%%%%%%%%%%%%%%%%%%%%%%%%%%%%%%
\paragraph{v1.5:} 2017/05/21

\begin{itemize}
\item
more complete structuring introduced
\item
|\childdocof| introduced
\item
|\childdoc| renamed to |\childdocmain|
\item
|\childredirect| renamed to |\childdocforward| and |\childdocforwardprefix|
and functionality expanded
\end{itemize}

%%%%%%%%%%%%%%%%%%%%%%%%%%%%%%%%%%%%%%%%
\paragraph{v1.0:} 2017/04/27

\begin{itemize}
\item
manual and install package
\item
first version published on CTAN
\end{itemize}

%%%%%%%%%%%%%%%%%%%%%%%%%%%%%%%%%%%%%%%%
\paragraph{v0.6:} 2017/04/26

\begin{itemize}
\item
redirection mechanism added
\end{itemize}

%%%%%%%%%%%%%%%%%%%%%%%%%%%%%%%%%%%%%%%%
\paragraph{v0.5:} 2017/04/26

\begin{itemize}
\item
functionality in definition file
\end{itemize}


%%%%%%%%%%%%%%%%%%%%%%%%%%%%%%%%%%%%%%%%%%%%%%%%%%%%%%%%%%%%%%%%%%%%%%%%%%%%%%%%
%%%%%%%%%%%%%%%%%%%%%%%%%%%%%%%%%%%%%%%%%%%%%%%%%%%%%%%%%%%%%%%%%%%%%%%%%%%%%%%%
%%%%%%%%%%%%%%%%%%%%%%%%%%%%%%%%%%%%%%%%%%%%%%%%%%%%%%%%%%%%%%%%%%%%%%%%%%%%%%%%
\appendix

\settowidth\MacroIndent{\rmfamily\scriptsize 000\ }

 \DocInput{childdoc.dtx}

\end{document}
%</driver>
% \fi
%
% %%%%%%%%%%%%%%%%%%%%%%%%%%%%%%%%%%%%%%%%%%%%%%%%%%%%%%%%%%%%%%%%%%%%%%%%%%%%%%
% %%%%%%%%%%%%%%%%%%%%%%%%%%%%%%%%%%%%%%%%%%%%%%%%%%%%%%%%%%%%%%%%%%%%%%%%%%%%%%
% \section{Sample}
%\iffalse
%<*samplemain>
%\fi
%
% The following presents a sample document
% with two chapters, two parts, a title page,
% a compile flag as well as three forwarding files to set the flag.
% It consists of eight |.tex| files:
% \begin{center}
% \begin{tabular}{ll}
% |cdocsamp.tex|&main file\\
% |cdocsch1.tex|&include file for chapter 1\\
% |cdocsch2.tex|&include file for chapter 2\\
% |cdocspt3.tex|&include file for part 3\\
% |cdocspt4.tex|&include file for part 4\\
% |cdocsdrf.tex|&forwarding file for main file in draft mode\\
% |cdocsfi1.tex|&forwarding file for final version of chapter 1\\
% |cdocsfi2.tex|&forwarding file for final version of chapter 2\\
% \end{tabular}
% \end{center}
% Each of the eight files can be compiled directly by the \LaTeX{} compiler.
%
% %%%%%%%%%%%%%%%%%%%%%%%%%%%%%%%%%%%%%%
% \paragraph{Main File.}
%
% The main file is called |cdocsamp.tex|.
%
% Load the \textsf{childdoc} definitions and
% declare the filename for the main document:
%    \begin{macrocode}
% \iffalse
%
% childdoc.dtx Copyright (C) 2017-2018 Niklas Beisert
%
% This work may be distributed and/or modified under the
% conditions of the LaTeX Project Public License, either version 1.3
% of this license or (at your option) any later version.
% The latest version of this license is in
%   http://www.latex-project.org/lppl.txt
% and version 1.3 or later is part of all distributions of LaTeX
% version 2005/12/01 or later.
%
% This work has the LPPL maintenance status `maintained'.
%
% The Current Maintainer of this work is Niklas Beisert.
%
% This work consists of the files childdoc.dtx and childdoc.ins
% and the derived files childdoc.def and cdocsamp.tex with
% cdocsch1.tex, cdocsch2.tex, cdocsdrf.tex, cdocsfn1.tex, cdocsfn2.tex.
%
%<package>\ifdefined\childdocmain\endinput\fi
%<package>\ProvidesFile{childdoc.def}[2018/12/30 v2.0 child document driver]
%<samplemain>\ProvidesFile{cdocsamp.tex}[2018/12/30 v2.0 sample for childdoc]
%<*driver>
%\ProvidesFile{childdoc.drv}[2018/12/30 v2.0 childdoc reference manual file]
\PassOptionsToClass{10pt,a4paper}{article}
\documentclass{ltxdoc}

\usepackage[margin=35mm]{geometry}
\usepackage{hyperref}
\usepackage{hyperxmp}
\usepackage[usenames]{color}

\hypersetup{colorlinks=true}
\hypersetup{pdfstartview=FitH}
\hypersetup{pdfpagemode=UseNone}
\hypersetup{pdfsource={}}
\hypersetup{pdflang={en-UK}}
\hypersetup{pdfcopyright={Copyright 2017-2018 Niklas Beisert.
  This work may be distributed and/or modified under the
  conditions of the LaTeX Project Public License, either version 1.3
  of this license or (at your option) any later version.}}
\hypersetup{pdflicenseurl={http://www.latex-project.org/lppl.txt}}
\hypersetup{pdfcontactaddress={ETH Zurich, ITP, HIT K,
  Wolfgang-Pauli-Strasse 27}}
\hypersetup{pdfcontactpostcode={8093}}
\hypersetup{pdfcontactcity={Zurich}}
\hypersetup{pdfcontactcountry={Switzerland}}
\hypersetup{pdfcontactemail={nbeisert@itp.phys.ethz.ch}}
\hypersetup{pdfcontacturl={http://people.phys.ethz.ch/\xmptilde nbeisert/}}

\newcommand{\secref}[1]{\hyperref[#1]{section \ref*{#1}}}

\parskip1ex
\parindent0pt
\let\olditemize\itemize
\def\itemize{\olditemize\parskip0pt}

\begin{document}

\title{The \textsf{childdoc} Package}
\hypersetup{pdftitle={The childdoc Package}}
\author{Niklas Beisert\\[2ex]
  Institut f\"ur Theoretische Physik\\
  Eidgen\"ossische Technische Hochschule Z\"urich\\
  Wolfgang-Pauli-Strasse 27, 8093 Z\"urich, Switzerland\\[1ex]
  \href{mailto:nbeisert@itp.phys.ethz.ch}
  {\texttt{nbeisert@itp.phys.ethz.ch}}}
\hypersetup{pdfauthor={Niklas Beisert}}
\hypersetup{pdfsubject={Manual for the LaTeX2e Package childdoc}}
\date{30 December 2018, \textsf{v2.0}}
\maketitle

\begin{abstract}\noindent
\textsf{childdoc} is a \LaTeXe{} package
that enables the direct compilation
of document sections included by |\include|
to individual files.
\end{abstract}

\begingroup
\parskip0ex
\tableofcontents
\endgroup

%%%%%%%%%%%%%%%%%%%%%%%%%%%%%%%%%%%%%%%%%%%%%%%%%%%%%%%%%%%%%%%%%%%%%%%%%%%%%%%%
%%%%%%%%%%%%%%%%%%%%%%%%%%%%%%%%%%%%%%%%%%%%%%%%%%%%%%%%%%%%%%%%%%%%%%%%%%%%%%%%
\section{Introduction}

\LaTeX{} provides a mechanism to structure a large document (such as a book)
into a main file and several child files (containing the chapters)
using the |\include| command.
This mechanism is beneficial for documents
which span hundreds of pages in order to
make the source file(s) more manageable.
Moreover, compilation can be restricted to
selected child files by means of the |\includeonly| command.
The latter feature can be used to reduce the compilation time while editing
(this was significantly more useful in the earlier days of \LaTeX{})
or to generate a smaller document which is easier to navigate.
Another application of |\includeonly| is to generate
documents consisting of selected parts of the complete document.

However, there are a few drawbacks of the plain |\include| mechanism:
\begin{itemize}
\item
The child files cannot be compiled on their own,
they can only be compiled via the main file.
A naive editing environment
(such as a text editor with an option
to have the current file processed by \LaTeX)
may require one to switch to the main file before compiling;
attempting to compile the child file produces errors.
\item
The main file must be modified (each time)
to adjust the |\includeonly| command
to the present needs. This easily leaves the main file in a messy state.
\item
The generated document will always carry the filename
of the main document. This is inconvenient if
several child files are to be compiled and
to be kept for distribution.
\end{itemize}

The present package provides a simple interface
to make child files individually compilable by \LaTeX{}.
Compiling a child file then has the same effect as compiling
the main file with an |\includeonly| command
to select the appropriate child.
Moreover the generated document will carry the name of the child
rather than the main file.
This resolves all three above issues.

This feature is meant to make the editing of books,
thesis documents and lecture notes somewhat more convenient.
However, the package can also be used efficiently for
composing a series of documents (such as exercise sheets)
which are typically distributed individually.
It then assists the author in generating the individual documents
(potentially in different versions)
as well as a document containing the collected series.
Another application is in developing style files
or other kinds of included material
where compilation of the style file could redirect
to a sample or test file.

%%%%%%%%%%%%%%%%%%%%%%%%%%%%%%%%%%%%%%%%%%%%%%%%%%%%%%%%%%%%%%%%%%%%%%%%%%%%%%%%
%%%%%%%%%%%%%%%%%%%%%%%%%%%%%%%%%%%%%%%%%%%%%%%%%%%%%%%%%%%%%%%%%%%%%%%%%%%%%%%%
\section{Usage}

First of all, the package \textsf{childdoc} is \emph{not} a standard
\LaTeXe{} |.sty| style file! Therefore it needs to be invoked in
a non-standard way.

%%%%%%%%%%%%%%%%%%%%%%%%%%%%%%%%%%%%%%%%%%%%%%%%%%%%%%%%%%%%%%%%%%%%%%%%%%%%%%%%
\subsection{Included Files}
\label{sec:include}

%%%%%%%%%%%%%%%%%%%%%%%%%%%%%%%%%%%%%%%%
\DescribeMacro{\childdocmain}
To use the package, add the commands
\begin{center}
\begin{tabular}{l}
|% \iffalse
%
% childdoc.dtx Copyright (C) 2017-2018 Niklas Beisert
%
% This work may be distributed and/or modified under the
% conditions of the LaTeX Project Public License, either version 1.3
% of this license or (at your option) any later version.
% The latest version of this license is in
%   http://www.latex-project.org/lppl.txt
% and version 1.3 or later is part of all distributions of LaTeX
% version 2005/12/01 or later.
%
% This work has the LPPL maintenance status `maintained'.
%
% The Current Maintainer of this work is Niklas Beisert.
%
% This work consists of the files childdoc.dtx and childdoc.ins
% and the derived files childdoc.def and cdocsamp.tex with
% cdocsch1.tex, cdocsch2.tex, cdocsdrf.tex, cdocsfn1.tex, cdocsfn2.tex.
%
%<package>\ifdefined\childdocmain\endinput\fi
%<package>\ProvidesFile{childdoc.def}[2018/12/30 v2.0 child document driver]
%<samplemain>\ProvidesFile{cdocsamp.tex}[2018/12/30 v2.0 sample for childdoc]
%<*driver>
%\ProvidesFile{childdoc.drv}[2018/12/30 v2.0 childdoc reference manual file]
\PassOptionsToClass{10pt,a4paper}{article}
\documentclass{ltxdoc}

\usepackage[margin=35mm]{geometry}
\usepackage{hyperref}
\usepackage{hyperxmp}
\usepackage[usenames]{color}

\hypersetup{colorlinks=true}
\hypersetup{pdfstartview=FitH}
\hypersetup{pdfpagemode=UseNone}
\hypersetup{pdfsource={}}
\hypersetup{pdflang={en-UK}}
\hypersetup{pdfcopyright={Copyright 2017-2018 Niklas Beisert.
  This work may be distributed and/or modified under the
  conditions of the LaTeX Project Public License, either version 1.3
  of this license or (at your option) any later version.}}
\hypersetup{pdflicenseurl={http://www.latex-project.org/lppl.txt}}
\hypersetup{pdfcontactaddress={ETH Zurich, ITP, HIT K,
  Wolfgang-Pauli-Strasse 27}}
\hypersetup{pdfcontactpostcode={8093}}
\hypersetup{pdfcontactcity={Zurich}}
\hypersetup{pdfcontactcountry={Switzerland}}
\hypersetup{pdfcontactemail={nbeisert@itp.phys.ethz.ch}}
\hypersetup{pdfcontacturl={http://people.phys.ethz.ch/\xmptilde nbeisert/}}

\newcommand{\secref}[1]{\hyperref[#1]{section \ref*{#1}}}

\parskip1ex
\parindent0pt
\let\olditemize\itemize
\def\itemize{\olditemize\parskip0pt}

\begin{document}

\title{The \textsf{childdoc} Package}
\hypersetup{pdftitle={The childdoc Package}}
\author{Niklas Beisert\\[2ex]
  Institut f\"ur Theoretische Physik\\
  Eidgen\"ossische Technische Hochschule Z\"urich\\
  Wolfgang-Pauli-Strasse 27, 8093 Z\"urich, Switzerland\\[1ex]
  \href{mailto:nbeisert@itp.phys.ethz.ch}
  {\texttt{nbeisert@itp.phys.ethz.ch}}}
\hypersetup{pdfauthor={Niklas Beisert}}
\hypersetup{pdfsubject={Manual for the LaTeX2e Package childdoc}}
\date{30 December 2018, \textsf{v2.0}}
\maketitle

\begin{abstract}\noindent
\textsf{childdoc} is a \LaTeXe{} package
that enables the direct compilation
of document sections included by |\include|
to individual files.
\end{abstract}

\begingroup
\parskip0ex
\tableofcontents
\endgroup

%%%%%%%%%%%%%%%%%%%%%%%%%%%%%%%%%%%%%%%%%%%%%%%%%%%%%%%%%%%%%%%%%%%%%%%%%%%%%%%%
%%%%%%%%%%%%%%%%%%%%%%%%%%%%%%%%%%%%%%%%%%%%%%%%%%%%%%%%%%%%%%%%%%%%%%%%%%%%%%%%
\section{Introduction}

\LaTeX{} provides a mechanism to structure a large document (such as a book)
into a main file and several child files (containing the chapters)
using the |\include| command.
This mechanism is beneficial for documents
which span hundreds of pages in order to
make the source file(s) more manageable.
Moreover, compilation can be restricted to
selected child files by means of the |\includeonly| command.
The latter feature can be used to reduce the compilation time while editing
(this was significantly more useful in the earlier days of \LaTeX{})
or to generate a smaller document which is easier to navigate.
Another application of |\includeonly| is to generate
documents consisting of selected parts of the complete document.

However, there are a few drawbacks of the plain |\include| mechanism:
\begin{itemize}
\item
The child files cannot be compiled on their own,
they can only be compiled via the main file.
A naive editing environment
(such as a text editor with an option
to have the current file processed by \LaTeX)
may require one to switch to the main file before compiling;
attempting to compile the child file produces errors.
\item
The main file must be modified (each time)
to adjust the |\includeonly| command
to the present needs. This easily leaves the main file in a messy state.
\item
The generated document will always carry the filename
of the main document. This is inconvenient if
several child files are to be compiled and
to be kept for distribution.
\end{itemize}

The present package provides a simple interface
to make child files individually compilable by \LaTeX{}.
Compiling a child file then has the same effect as compiling
the main file with an |\includeonly| command
to select the appropriate child.
Moreover the generated document will carry the name of the child
rather than the main file.
This resolves all three above issues.

This feature is meant to make the editing of books,
thesis documents and lecture notes somewhat more convenient.
However, the package can also be used efficiently for
composing a series of documents (such as exercise sheets)
which are typically distributed individually.
It then assists the author in generating the individual documents
(potentially in different versions)
as well as a document containing the collected series.
Another application is in developing style files
or other kinds of included material
where compilation of the style file could redirect
to a sample or test file.

%%%%%%%%%%%%%%%%%%%%%%%%%%%%%%%%%%%%%%%%%%%%%%%%%%%%%%%%%%%%%%%%%%%%%%%%%%%%%%%%
%%%%%%%%%%%%%%%%%%%%%%%%%%%%%%%%%%%%%%%%%%%%%%%%%%%%%%%%%%%%%%%%%%%%%%%%%%%%%%%%
\section{Usage}

First of all, the package \textsf{childdoc} is \emph{not} a standard
\LaTeXe{} |.sty| style file! Therefore it needs to be invoked in
a non-standard way.

%%%%%%%%%%%%%%%%%%%%%%%%%%%%%%%%%%%%%%%%%%%%%%%%%%%%%%%%%%%%%%%%%%%%%%%%%%%%%%%%
\subsection{Included Files}
\label{sec:include}

%%%%%%%%%%%%%%%%%%%%%%%%%%%%%%%%%%%%%%%%
\DescribeMacro{\childdocmain}
To use the package, add the commands
\begin{center}
\begin{tabular}{l}
|\input{childdoc.def}|\\
|\childdocmain{}|\\
\end{tabular}
\end{center}
at the very top of the main \LaTeX{} file,
in particular \emph{before} the |\documentclass| statement!
The argument of |\childdocmain| should be left empty
(but it must be present).

%%%%%%%%%%%%%%%%%%%%%%%%%%%%%%%%%%%%%%%%
\DescribeMacro{\childdocof}
Furthermore, add the commands
\begin{center}
\begin{tabular}{l}
|\input{childdoc.def}|\\
|\childdocof{|\textit{main}|}|\\
\end{tabular}
\end{center}
at the top of every child file \textit{child}
which is included by |\include{|\textit{child}|}|
from within the main file
(or at least for those files to be compiled individually).
The argument \textit{main} must be the filename of the main file.

There are a couple of
considerations in setting up the main and child documents:

%%%%%%%%%%%%%%%%%%%%%%%%%%%%%%%%%%%%%%%%
\paragraph{Restrictions.}

Please note the following restrictions:
\begin{itemize}
\item
|\childdocmain| must be called with one argument \textit{main}
to ensure compatibility with earlier version of the package.
It must either be empty (|\childdocmain{}|)
or precisely match the filename of the main file in which it is specified.
See \secref{sec:detection} for further information.
\item
The filename \textit{main} must be specified without the |.tex| extension.
\item
The filename \textit{main} is case sensitive
(even in case-insensitive file systems)
due to internal string comparison.
\item
The argument \textit{main} should be fully expanded, it cannot be a macro.
\item
Subdirectories and special characters should be avoided in filenames.
\item
The command |\childdocmain{|\textit{main}|}| must be followed by a whitespace.
It should not be followed immediately by another command
or by a comment mark `|%|'.
This is because the \TeX{} parser reads the token immediately following
the argument of |\childdocmain| and puts it
at the beginning of every child section;
however, a white\-space is ignored.
\end{itemize}

%%%%%%%%%%%%%%%%%%%%%%%%%%%%%%%%%%%%%%%%
\paragraph{Content of Main File.}

It is advisable to place all content in the child files included by |\include|.
Any output contained in the main file will appear in all child documents
unless suppressed manually;
it cannot be suppressed automatically by the |\includeonly| directive
and thus should normally be avoided.
A method to include some content in the main file
by means of conditional processing is described in \secref{sec:conditional}.

%%%%%%%%%%%%%%%%%%%%%%%%%%%%%%%%%%%%%%%%
\paragraph{Page Numbering.}

When only a part of the document is compiled,
the appropriate numbering of pages
(as well as other status parameters)
is determined from the |.aux| files.
The latter contain information from previous passes.
However this information needs to propagate through
all intermediate child documents.
Therefore the page numbering in child documents may well
be inconsistent until the complete document is compiled at least once.

A useful (if unconventional) way to always ensure a consistent
page numbering is to restart the numbering in each child document
and denote the pages by `\textit{child}|.|\textit{page}'
where \textit{child} represents the chapter/section number of the child file.
This can be achieved by the command
|\numberwithin{page}{|\textit{child}|}|
of the \textsf{amsmath} package
where \textit{child} can be |chapter| or |section|
depending on the chosen structuring.
Alternatively, one can modify the macro |\thepage| appropriately
and reset the counter |page| at the start of each child file.

%%%%%%%%%%%%%%%%%%%%%%%%%%%%%%%%%%%%%%%%%%%%%%%%%%%%%%%%%%%%%%%%%%%%%%%%%%%%%%%%
\subsection{Conditional Processing}
\label{sec:conditional}

The package provides a mechanism to compile different versions
of a document. To customise the versions further some conditional processing
can come in handy to distinguish which version is being compiled.
The package provides two macros to describe the compilation context:

%%%%%%%%%%%%%%%%%%%%%%%%%%%%%%%%%%%%%%%%
\DescribeMacro{\ifchilddoc}
The conditional |\ifchilddoc| distinguishes between the compilation of
child documents and the main document:
%
\begin{center}
|\ifchilddoc |\textit{child-code}| |[|\||else |\textit{main-code}]| \||fi|
\end{center}

%%%%%%%%%%%%%%%%%%%%%%%%%%%%%%%%%%%%%%%%
\DescribeMacro{\childdocname}
\DescribeMacro{\childdocjob}
The macro |\childdocname| contains the filename (without extension)
of the main or child file being processed.
Note that |\childdocjob| will always contain the name of the main file.

%%%%%%%%%%%%%%%%%%%%%%%%%%%%%%%%%%%%%%%%
\paragraph{Title Page.}

Conditional processing can be used to include a title or banner page
in the main document when proper precautions are taken.
Importantly, the code in the main file should ensure that the page counter
(as well as other status parameters which are stored in the |.aux| files)
takes the same value after the conditional processing.
Otherwise the page numbers may take divergent values
depending on which part is compiled.

For example, a title page could be declared by:
%
\begin{center}
\begin{tabular}{l}
|\ifchilddoc\||else|\\
|\addtocounter{page}{-1}|\\
\textit{code for title page}\\
|\newpage|\\
|\||fi|
\end{tabular}
\end{center}
%
A banner page for the child documents can be generated by:
%
\begin{center}
\begin{tabular}{l}
|\ifchilddoc|\\
|\addtocounter{page}{-1}|\\
\textit{code for banner page}\\
|\newpage|\\
|\||fi|
\end{tabular}
\end{center}
%
Here one could write a message such as:
\begin{center}
|This is the part \childdocname{} of \childdocjob{}.|
\end{center}

%%%%%%%%%%%%%%%%%%%%%%%%%%%%%%%%%%%%%%%%%%%%%%%%%%%%%%%%%%%%%%%%%%%%%%%%%%%%%%%%
\subsection{Flags}
\label{sec:flags}

The package makes it easy to generate different versions
of the main or child documents.
To this end compilation flags can be defined
and assigned different default values.
They will be particularly useful in conjunction
with the forwarding mechanism described in \secref{sec:forward}.

For example, it may be useful to have a flag |\version|
which can be set to |draft| or |final|.
The document source will contain some conditional code
depending on the value of |\version|.
Suppose further, the flag should default to |final| for the main file
and to |draft| for child files
which is a natural assignment for editing the document.
This is achieved by placing the following code
in the preamble of the main document
(below the |\childdocmain| directive):
%
\begin{center}
\begin{tabular}{l}
|\ifchilddoc|\\
|\providecommand{\version}{draft}|\\
|\||else|\\
|\providecommand{\version}{final}|\\
|\||fi|
\end{tabular}
\end{center}
%
The definition by |\providecommand| makes sure
that previous definitions are not overwritten.
Further statements |\providecommand{\version}{...}|
can thus be added before the above code to override it.

For the main file, one might add a line
(between |\childdocmain| and the above block)
%
\begin{center}
|%\ifchilddoc\||else\providecommand{\version}{draft}\||fi|
\end{center}
%
which can be uncommented to produce a draft version.
Likewise one can add a line to the very top of a child file
(above the |\childdocof{|\textit{main}|}| directive)
%
\begin{center}
|%\providecommand{\version}{final}|
\end{center}
%
which can be uncommented to produce the final version of this child document.

%%%%%%%%%%%%%%%%%%%%%%%%%%%%%%%%%%%%%%%%%%%%%%%%%%%%%%%%%%%%%%%%%%%%%%%%%%%%%%%%
\subsection{Forwarding}
\label{sec:forward}

Different versions of the main or child documents
using compilation flags as described in \secref{sec:flags}
can be (permanently) stored in different files
for convenient compilation, viewing and distribution.
To this end, the package defines a command
to pass on compilation to a different file:

%%%%%%%%%%%%%%%%%%%%%%%%%%%%%%%%%%%%%%%%
\DescribeMacro{\childdocforward}
The command |\childdocforward| redirects processing to
another source file:
%
\begin{center}
\begin{tabular}{l}
|\input{childdoc.def}|\\
|\childdocforward[|\textit{main}|]{|\textit{dest}|}|\\
\end{tabular}
\end{center}
%
The argument \textit{dest} is the destination file
(without extension).
It should be the main file or one of the child files.
Note that further \textsf{childdoc} directives
such as |\childdocof| and |\childdocforward|
in the indicated file will be processed in this form.
The optional argument \textit{main}
passes on directly to the main file \textit{main}
while pretending to compile the child \textit{dest}.
This form behaves as if \textit{dest}
issues |\childdocof{|\textit{main}|}| right away,
and no further \textsf{childdoc} directives will be processed.

%%%%%%%%%%%%%%%%%%%%%%%%%%%%%%%%%%%%%%%%
\DescribeMacro{\...prefix}
In the alternative form |\childdocforwardprefix|,
%
\begin{center}
\begin{tabular}{l}
|\input{childdoc.def}|\\
|\childdocforwardprefix[|\textit{main}|]{|\textit{prefix}|}{|\textit{dest}|}|
\end{tabular}
\end{center}
%
the destination file is determined by a pattern
depending on the current file:
To make this work, the current file must be called
`{\textit{prefix}\hspace{0.2em}\textit{suffix}}'
with \textit{prefix} matching precisely the argument.
Processing is then passed on to the file
`{\textit{dest}\hspace{0.2em}\textit{suffix}}'.
Surely, the same effect is achieved by
directly specifying the
argument `{\textit{dest}\hspace{0.2em}\textit{suffix}}'
in the first form.
However, that requires to set up a different file
for each child. With the alternative form of the command
all these files can have exactly the same content
which simplifies setting them up and maintaining them.

For example, the following file |draft.tex|
with a compilation flag |\version| as described in \secref{sec:flags}
compiles the main document as a draft:
%
\begin{center}
\begin{tabular}{l}
|\def\version{draft}|\\
|\input{childdoc.def}|\\
|\childdocforward{|\textit{main}|}|
\end{tabular}
\end{center}
%
Likewise, the following files |final|\textit{nn}|.tex|
compile the final version of the child document
|child|\textit{nn}|.tex|:
%
\begin{center}
\begin{tabular}{l}
|\def\version{final}|\\
|\input{childdoc.def}|\\
|\childdocforwardprefix{final}{child}|
\end{tabular}
\end{center}
%

Note that when several versions of a main file and/or of each child file
are to be generated, it may be convenient to set up a |Makefile| or
shell script to automatise the process.

%%%%%%%%%%%%%%%%%%%%%%%%%%%%%%%%%%%%%%%%%%%%%%%%%%%%%%%%%%%%%%%%%%%%%%%%%%%%%%%%
\subsection{Command Line Processing}
\label{sec:commandline}

The effect of redirection files can also be achieved by invoking
the \LaTeX{} compiler with a more elaborate command line.
Most conveniently this should be done as part
of a shell script or a |Makefile|.

When using \textsf{childdoc} in the main file, the following
command lines effectively perform a redirection
(note that depending on the shell being used,
backslashes may have to be doubled: `|\|' $\to$ `|\\|'):
%
\begin{center}
|... -jobname "|\textit{target}|" |\\|"|[\textit{flags}]%
|\input{childdoc.def}\childdocforward[|\textit{main}|]{|\textit{dest}|}"|
\end{center}
%
Here \textit{target} is the name of the output file,
\textit{main} is the name of the main file
and \textit{dest} is the name of the main or child file to be processed
(all filenames without extensions).
The optional argument \textit{main} can be omitted
if \textit{main} matches \textit{dest}.
Optionally, compilation \textit{flags} can be defined via |\def| commands.
This command line makes the \TeX{} engine believe
it is compiling the file \textit{target}
whose content is specified as the latter parameter.
The provided code then forwards the processing to
\textit{main} or \textit{dest} as described in \secref{sec:forward}.

%%%%%%%%%%%%%%%%%%%%%%%%%%%%%%%%%%%%%%%%%%%%%%%%%%%%%%%%%%%%%%%%%%%%%%%%%%%%%%%%
\subsection{Include by Input}
\label{sec:input}

Including child documents by |\include| has some restrictions by design.
Most notably, the content of a child document always occupies
its own set of pages; pages cannot be shared between child documents.
Usually, this behaviour makes perfect sense
because each child document contain an essential part of the document.
However, in some situations it may be desirable to compose
a document from a collection of parts
without having mandatory page breaks between then.
For this case, the package
provides a mechanism to include parts
by |\input| which can also be processed individually.
However, by construction this mechanism
requires manual handling of the content to be output.

%%%%%%%%%%%%%%%%%%%%%%%%%%%%%%%%%%%%%%%%
\DescribeMacro{\ifchilddocmanual}
The main file should be prepared as usual, see \secref{sec:include}.
However, the document body must make a distinction
between processing of an individual part and of the main document, e.g.:
%
\begin{center}
\begin{tabular}{l}
|\ifchilddocmanual|\\
|\input{\childdocname}|\\
|\||else|\\
\textit{document body with }|\input{|\textit{part}|}|\\
|\||fi|
\end{tabular}
\end{center}
%
The conditional |\ifchilddocmanual| is true whenever
a part to be included by |\input| is being compiled,
and the name of the part is stored in |\childdocname|.

%%%%%%%%%%%%%%%%%%%%%%%%%%%%%%%%%%%%%%%%
\DescribeMacro{\childdocby}
Each part to be included by |\input| should start with:
%
\begin{center}
\begin{tabular}{l}
|\input{childdoc.def}|\\
|\childdocby{|\textit{main}|}|\\
\end{tabular}
\end{center}
%
The directive |\childdocby| is similar to |\childdocof|
described in \secref{sec:include},
but the subsequent selection of content must be done manually.
To that end, both |\ifchilddoc| and |\ifchilddocmanual|
will be true upon processing of a part,
and the name of the part is stored in |\childdocname|.
Note that |\jobname| will be set to the filename of the current part
so that each part receives an individual |.aux| file
that does not interfere with the |.aux| file(s) of the main document.
This behaviour can be altered by the alternative form
|\childdocby[*]{|\textit{main}|}| (with a non-empty optional argument)
which uses the |.aux| file of the main document
by setting |\jobname| to \textit{main}.

%%%%%%%%%%%%%%%%%%%%%%%%%%%%%%%%%%%%%%%%%%%%%%%%%%%%%%%%%%%%%%%%%%%%%%%%%%%%%%%%
\subsection{Driver Development}
\label{sec:driver}

The \textsf{childdoc} mechanism can also be use for the development
of definition files such as \LaTeX{} styles or classes.
This case differs from the above setup with multiple parts
included by |\include| in that no |\includeonly| should be invoked.
This can be achieved by starting the include file
(before |\ProvidesPackage|) with:
%
\begin{center}
\begin{tabular}{l}
|\input{childdoc.def}|\\
|\childdocforward{|\textit{main}|}|\\
\end{tabular}
\end{center}
%
or alternatively with:
%
\begin{center}
\begin{tabular}{l}
|\input{childdoc.def}|\\
|\childdocby{|\textit{main}|}|\\
\end{tabular}
\end{center}
%
Both forms have slightly different effects as described above.
The main file is prepared as usual, see \secref{sec:include}.

%%%%%%%%%%%%%%%%%%%%%%%%%%%%%%%%%%%%%%%%%%%%%%%%%%%%%%%%%%%%%%%%%%%%%%%%%%%%%%%%
\subsection{Legacy Detection}
\label{sec:detection}

The directive |\childdocmain| in the main file can detect
whether the complete document or merely a child is to be compiled
even without using the directive |\childdocof|.
This method is deprecated because it is less robust
and there is no compelling reason to use it;
it is merely provided for backward compatibility
and it may be removed in future versions.

If the detection mechanism is to be used,
it is mandatory to correctly specify
the filename of the main file as the argument of |\childdocmain|:
%
\begin{center}
\begin{tabular}{l}
|\input{childdoc.def}|\\
|\childdocmain{|\textit{main}|}|\\
\end{tabular}
\end{center}
%
If |\jobname| does not match the argument \textit{main} of |\childdocmain|,
it is assumed that |\jobname| points to the child file to be compiled.
When using |\childdocmain| with the main file specified as argument,
it suffices to start a child file
with just |\input{|\textit{main}|}|
without loading of the package and using |\childdocof|.
If instead all processing is done
with the appropriate \textsf{childdoc} directives,
the argument of \textit{main} of |\childdocmain| can be empty.

An alternative version of the command line processing described
in \secref{sec:commandline} using the detection mechanism reads:
%
\begin{center}
|... -jobname "|\textit{target}|" "|[\textit{flags}]%
[|\def\jobname{|\textit{dest}|}|]|\input{|\textit{main}|}"|
\end{center}

%%%%%%%%%%%%%%%%%%%%%%%%%%%%%%%%%%%%%%%%%%%%%%%%%%%%%%%%%%%%%%%%%%%%%%%%%%%%%%%%
\subsection{Manual Code}
\label{sec:manual}

In case one cannot be certain whether the definitions file |childdoc.def|
is installed on the target \TeX{} distribution
and one prefers not to ship it,
it is conceivable to paste a few relevant commands into the sources.

To that end, drop all statements |\input{childdoc.def}|
and perform the replacements as outlined below.
Instead of |\childdocmain{|\textit{main}|}| add the following code
to the top of the main file:
%
\begin{center}
\begin{tabular}{l}
|\||ifdefined\childdocname\endinput\||fi\newif\ifchilddoc|\\
|\edef\childdocname{\scantokens\expandafter{\jobname\noexpand}}|\\
|\def\childdocmain{|\textit{main}|}\||ifx\childdocmain\childdocname\||else|\\
|\childdoctrue\includeonly{\childdocname}\let\jobname\childdocmain\||fi|\\
\end{tabular}
\end{center}
%
Instead of |\childdocof{|\textit{main}|}| just include the main file
at the top of each child file:
%
\begin{center}
|\input{|\textit{main}|}|
\end{center}
%
A simple redirection |\childdocforward{|\textit{dest}|}| is achieved by:
%
\begin{center}
|\def\jobname{|\textit{dest}|}\input{\jobname}|
\end{center}
%
The redirection with prefix
|\childdocforwardprefix[|\textit{prefix}|]{|\textit{dest}|}|
is accomplished by:
%
\begin{center}
\begin{tabular}{l}
|{\edef\jobname{\scantokens\expandafter{\jobname\noexpand}}|\\
|\def\redirectjob |\textit{prefix}|#1~~~{\gdef\jobname{|\textit{dest}|#1}}|\\
|\expandafter\redirectjob\jobname~~~}\input{\jobname}|
\end{tabular}
\end{center}

In an alternative approach,
child documents can be compiled by a specific command line
without additional code or specific definitions:
%
\begin{center}
|... -jobname "|\textit{target}|" "|[\textit{flags}]%
|\includeonly{|\textit{dest}|}\input{|\textit{main}|}"|
\end{center}
%

%%%%%%%%%%%%%%%%%%%%%%%%%%%%%%%%%%%%%%%%%%%%%%%%%%%%%%%%%%%%%%%%%%%%%%%%%%%%%%%%
%%%%%%%%%%%%%%%%%%%%%%%%%%%%%%%%%%%%%%%%%%%%%%%%%%%%%%%%%%%%%%%%%%%%%%%%%%%%%%%%
\section{Information}

%%%%%%%%%%%%%%%%%%%%%%%%%%%%%%%%%%%%%%%%%%%%%%%%%%%%%%%%%%%%%%%%%%%%%%%%%%%%%%%%
\subsection{Copyright}

Copyright \copyright{} 2017--2018 Niklas Beisert

This work may be distributed and/or modified under the
conditions of the \LaTeX{} Project Public License, either version 1.3
of this license or (at your option) any later version.
The latest version of this license is in
  \url{http://www.latex-project.org/lppl.txt}
and version 1.3 or later is part of all distributions of \LaTeX{}
version 2005/12/01 or later.

This work has the LPPL maintenance status `maintained'.

The Current Maintainer of this work is Niklas Beisert.

This work consists of the files |README.txt|, |childdoc.ins| and |childdoc.dtx|
as well as the derived files |childdoc.def|, |cdocsamp.tex|
with |cdocsch1.tex|, |cdocsch2.tex|, |cdocspt3.tex|, |cdocspt4.tex|,
|cdocsdrf.tex|, |cdocsfn1.tex|, |cdocsfn2.tex|
as well as |childdoc.pdf|.

%%%%%%%%%%%%%%%%%%%%%%%%%%%%%%%%%%%%%%%%%%%%%%%%%%%%%%%%%%%%%%%%%%%%%%%%%%%%%%%%
\subsection{Files and Installation}

The package consists of the files:
%
\begin{center}
\begin{tabular}{ll}
    |README.txt|   & readme file \\
    |childdoc.ins| & installation file \\
    |childdoc.dtx| & source file \\
    |childdoc.def| & definition file \\
    |cdocsamp.tex| & sample main file \\
    |cdocsch1.tex| & sample include file \\
    |cdocsch2.tex| & sample include file \\
    |cdocspt3.tex| & sample part file \\
    |cdocspt4.tex| & sample part file \\
    |cdocsdrf.tex| & sample redirection file \\
    |cdocsfn1.tex| & sample redirection file \\
    |cdocsfn2.tex| & sample redirection file \\
    |childdoc.pdf| & manual
\end{tabular}
\end{center}
%
The distribution consists of the files
|README.txt|, |childdoc.ins| and |childdoc.dtx|.
%
\begin{itemize}
\item
Run (pdf)\LaTeX{} on |childdoc.dtx|
to compile the manual |childdoc.pdf| (this file).
\item
Run \LaTeX{} on |childdoc.ins| to create the definitions file |childdoc.def|
and the sample |cdocsamp.tex| with include files
|cdocsch1.tex|, |cdocsch2.tex|, |cdocspt3.tex|, |cdocspt4.tex|,
|cdocsdrf.tex|, |cdocsfn1.tex|, |cdocsfn2.tex|.
Then copy the file |childdoc.def| to an appropriate directory of your \LaTeX{}
distribution, e.g.\ \textit{texmf-root}|/tex/latex/childdoc|.
\end{itemize}

%%%%%%%%%%%%%%%%%%%%%%%%%%%%%%%%%%%%%%%%%%%%%%%%%%%%%%%%%%%%%%%%%%%%%%%%%%%%%%%%
\subsection{Related CTAN Packages}

There are several other packages which offer a similar functionality:
%
\begin{itemize}
\item
The packages
\href{http://ctan.org/pkg/docmute}{\textsf{docmute}},
\href{http://ctan.org/pkg/includex}{\textsf{includex}} and
\href{http://ctan.org/pkg/standalone}{\textsf{standalone}}
provide commands to include only the document body of
a child file thus allowing both files to be compiled individually.
\item
The packages \href{http://ctan.org/pkg/subdocs}{\textsf{subdocs}}
and \href{http://ctan.org/pkg/subfiles}{\textsf{subfiles}}
provide structures in which the main and child documents can be
encapsulated and allowing them to be compiled individually.
The inclusion mechanism is different from the conventional |\include|.
\item
The package \href{http://ctan.org/pkg/combine}{\textsf{combine}}
is an elaborate solution to combine several documents into one.
\end{itemize}
%
See also the CTAN topic \href{http://ctan.org/topic/subdocs}{\textsf{subdocs}}
for further related packages.
The present package differs from the above solutions in that
a document structure constructed with the conventional |\include| mechanism
just needs two extra commands at the top of every file
such that all constituent files can be compiled individually.

%%%%%%%%%%%%%%%%%%%%%%%%%%%%%%%%%%%%%%%%%%%%%%%%%%%%%%%%%%%%%%%%%%%%%%%%%%%%%%%%
%\subsection{Feature Suggestions}
%
%The following is a list of features which may be useful for future
%versions of this package:
%%
%\begin{itemize}
%\item
%\ldots
%\end{itemize}

%%%%%%%%%%%%%%%%%%%%%%%%%%%%%%%%%%%%%%%%%%%%%%%%%%%%%%%%%%%%%%%%%%%%%%%%%%%%%%%%
\subsection{Revision History}

%%%%%%%%%%%%%%%%%%%%%%%%%%%%%%%%%%%%%%%%
\paragraph{v2.0:} 2018/12/30

\begin{itemize}
\item
immediate forward processing
\item
added |\childdocby| mechanism
\item
manual restructured
\end{itemize}

%%%%%%%%%%%%%%%%%%%%%%%%%%%%%%%%%%%%%%%%
\paragraph{v1.6:} 2018/01/17

\begin{itemize}
\item
application for development of include files
\item
corrections to manual
\end{itemize}

%%%%%%%%%%%%%%%%%%%%%%%%%%%%%%%%%%%%%%%%
\paragraph{v1.5:} 2017/05/21

\begin{itemize}
\item
more complete structuring introduced
\item
|\childdocof| introduced
\item
|\childdoc| renamed to |\childdocmain|
\item
|\childredirect| renamed to |\childdocforward| and |\childdocforwardprefix|
and functionality expanded
\end{itemize}

%%%%%%%%%%%%%%%%%%%%%%%%%%%%%%%%%%%%%%%%
\paragraph{v1.0:} 2017/04/27

\begin{itemize}
\item
manual and install package
\item
first version published on CTAN
\end{itemize}

%%%%%%%%%%%%%%%%%%%%%%%%%%%%%%%%%%%%%%%%
\paragraph{v0.6:} 2017/04/26

\begin{itemize}
\item
redirection mechanism added
\end{itemize}

%%%%%%%%%%%%%%%%%%%%%%%%%%%%%%%%%%%%%%%%
\paragraph{v0.5:} 2017/04/26

\begin{itemize}
\item
functionality in definition file
\end{itemize}


%%%%%%%%%%%%%%%%%%%%%%%%%%%%%%%%%%%%%%%%%%%%%%%%%%%%%%%%%%%%%%%%%%%%%%%%%%%%%%%%
%%%%%%%%%%%%%%%%%%%%%%%%%%%%%%%%%%%%%%%%%%%%%%%%%%%%%%%%%%%%%%%%%%%%%%%%%%%%%%%%
%%%%%%%%%%%%%%%%%%%%%%%%%%%%%%%%%%%%%%%%%%%%%%%%%%%%%%%%%%%%%%%%%%%%%%%%%%%%%%%%
\appendix

\settowidth\MacroIndent{\rmfamily\scriptsize 000\ }

 \DocInput{childdoc.dtx}

\end{document}
%</driver>
% \fi
%
% %%%%%%%%%%%%%%%%%%%%%%%%%%%%%%%%%%%%%%%%%%%%%%%%%%%%%%%%%%%%%%%%%%%%%%%%%%%%%%
% %%%%%%%%%%%%%%%%%%%%%%%%%%%%%%%%%%%%%%%%%%%%%%%%%%%%%%%%%%%%%%%%%%%%%%%%%%%%%%
% \section{Sample}
%\iffalse
%<*samplemain>
%\fi
%
% The following presents a sample document
% with two chapters, two parts, a title page,
% a compile flag as well as three forwarding files to set the flag.
% It consists of eight |.tex| files:
% \begin{center}
% \begin{tabular}{ll}
% |cdocsamp.tex|&main file\\
% |cdocsch1.tex|&include file for chapter 1\\
% |cdocsch2.tex|&include file for chapter 2\\
% |cdocspt3.tex|&include file for part 3\\
% |cdocspt4.tex|&include file for part 4\\
% |cdocsdrf.tex|&forwarding file for main file in draft mode\\
% |cdocsfi1.tex|&forwarding file for final version of chapter 1\\
% |cdocsfi2.tex|&forwarding file for final version of chapter 2\\
% \end{tabular}
% \end{center}
% Each of the eight files can be compiled directly by the \LaTeX{} compiler.
%
% %%%%%%%%%%%%%%%%%%%%%%%%%%%%%%%%%%%%%%
% \paragraph{Main File.}
%
% The main file is called |cdocsamp.tex|.
%
% Load the \textsf{childdoc} definitions and
% declare the filename for the main document:
%    \begin{macrocode}
\input{childdoc.def}
\childdocmain{}
%    \end{macrocode}

% Optional override for |\version| flag:
%    \begin{macrocode}
%%\ifchilddoc\else\providecommand{\version}{draft}\fi
%    \end{macrocode}

% Define the default values for the |\version| flag
% (|final| for the main file and |draft| for childs):
%    \begin{macrocode}
\ifchilddoc
\providecommand{\version}{draft}
\else
\providecommand{\version}{final}
\fi
%    \end{macrocode}

% Load the standard document class:
%    \begin{macrocode}
\documentclass[12pt]{article}
%    \end{macrocode}

% Start the document body:
%    \begin{macrocode}
\begin{document}
%    \end{macrocode}

% Declare a title page.
% Print title, part of document being processed and version flag:
%    \begin{macrocode}
\addtocounter{page}{-1}
\begin{center}
{\LARGE\bfseries{}childdoc example\par}
\vspace{1cm}
\ifchilddoc
\ifchilddocmanual part\else chapter\fi:
`\childdocname' of `\childdocjob'\par
\else
main document: `\childdocjob'\par
\fi
version: \version\par
\end{center}
\newpage
%    \end{macrocode}

% Manually include selected file,
% otherwise process as usual:
%    \begin{macrocode}
\ifchilddocmanual
\section*{part `\childdocname'}
\input{\childdocname}
\else
%    \end{macrocode}

% Include the two chapters:
%    \begin{macrocode}
\include{cdocsch1}
\include{cdocsch2}
%    \end{macrocode}

% Include the two parts unless only chapters should be displayed:
%    \begin{macrocode}
\ifchilddoc\else
\section{part three}
\input{cdocspt3}
\section{part four}
\input{cdocspt4}
\fi
%    \end{macrocode}

% Process as usual until here:
%    \begin{macrocode}
\fi
%    \end{macrocode}

% End of document body:
%    \begin{macrocode}
\end{document}
%    \end{macrocode}
%\iffalse
%</samplemain>
%\fi
%
% %%%%%%%%%%%%%%%%%%%%%%%%%%%%%%%%%%%%%%
% \paragraph{Chapter Include Files.}
%
% The include files are called |cdocsch1.tex| and |cdocsch2.tex|.
%
%\iffalse
%<*samplechap1|samplechap2>
%\fi

% Optional override for |\version| flag:
%    \begin{macrocode}
%%\providecommand{\version}{final}
%    \end{macrocode}

% Include the main document:
%    \begin{macrocode}
\input{childdoc.def}
\childdocof{cdocsamp}
%    \end{macrocode}

%\iffalse
%</samplechap1|samplechap2>
%\fi
%
%\iffalse
%<*samplechap1>
%\fi
% Some text for chapter 1:
%    \begin{macrocode}
\section{one}
some text in chapter one
%    \end{macrocode}

%\iffalse
%</samplechap1>
%\fi
% Some text for chapter 2:
%\iffalse
%<*samplechap2>
%\fi
%    \begin{macrocode}
\section{two}
more text in chapter two
%    \end{macrocode}

%\iffalse
%</samplechap2>
%\fi
%
% %%%%%%%%%%%%%%%%%%%%%%%%%%%%%%%%%%%%%%
% \paragraph{Part Include Files.}
%
% The include files are called |cdocspt3.tex| and |cdocspt4.tex|.
%
%\iffalse
%<*samplepart3|samplepart4>
%\fi

% Optional override for |\version| flag:
%    \begin{macrocode}
%%\providecommand{\version}{final}
%    \end{macrocode}

% Include the main document:
%    \begin{macrocode}
\input{childdoc.def}
\childdocby{cdocsamp}
%    \end{macrocode}

%\iffalse
%</samplepart3|samplepart4>
%\fi
%
%\iffalse
%<*samplepart3>
%\fi
% Some text for part 3:
%    \begin{macrocode}
some text in part three
%    \end{macrocode}

%\iffalse
%</samplepart3>
%\fi
% Some text for part 4:
%\iffalse
%<*samplepart4>
%\fi
%    \begin{macrocode}
more text in part four
%    \end{macrocode}

%\iffalse
%</samplepart4>
%\fi
%
% %%%%%%%%%%%%%%%%%%%%%%%%%%%%%%%%%%%%%%
% \paragraph{Forwarding for a Complete Draft.}
%
% The following forwarding file |cdocsdrf.tex|
% compiles the main document in draft mode:
%\iffalse
%<*sampledraft>
%\fi
%    \begin{macrocode}
\def\version{draft}
\input{childdoc.def}
\childdocforward{cdocsamp}
%    \end{macrocode}

%\iffalse
%</sampledraft>
%\fi
%
% %%%%%%%%%%%%%%%%%%%%%%%%%%%%%%%%%%%%%%
% \paragraph{Forwarding for Final Version of the Chapters.}
%
% The following forwarding files |cdocsfn1.tex| and |cdocsfn2.tex|
% (with identical content)
% compile the final versions of the child documents
% |cdocsch1.tex| and |cdocsch2.tex|, respectively:
%\iffalse
%<*samplefinal>
%\fi
%    \begin{macrocode}
\def\version{final}
\input{childdoc.def}
\childdocforwardprefix[cdocsamp]{cdocsfn}{cdocsch}
%    \end{macrocode}

%\iffalse
%</samplefinal>
%\fi
%
% %%%%%%%%%%%%%%%%%%%%%%%%%%%%%%%%%%%%%%
% \paragraph{Command Line Processing.}
%
% The following three command lines generate the output files
% |cdocscld|, |cdocscl1| and |cdocscl2|
% which should be identical to
% |cdocsdrf|, |cdocsch1| and |cdocsfn2|, respectively:
% \begin{center}
% \begin{tabular}{l}
% |latex -jobname cdocscld \|\\
% |  "\def\version{draft}\input{childdoc.def}\childdocforward{cdocsamp}"|\\
% |latex -jobname cdocscl1 \|\\
% |  "\input{childdoc.def}\childdocforward[cdocsamp]{cdocsch1}"|\\
% |latex -jobname cdocscl2 \|\\
% |  "\def\version{final}\input{childdoc.def}\childdocforward{cdocsch2}"|
% \end{tabular}
% \end{center}
% Note that the trailing backslash on each first line
% merely continues the input to the second line
% (for convenient cut ant paste).
% Furthermore, the command |latex| can be replaced by any
% of its alternative versions such as |pdflatex|.
%
% %%%%%%%%%%%%%%%%%%%%%%%%%%%%%%%%%%%%%%%%%%%%%%%%%%%%%%%%%%%%%%%%%%%%%%%%%%%%%%
% %%%%%%%%%%%%%%%%%%%%%%%%%%%%%%%%%%%%%%%%%%%%%%%%%%%%%%%%%%%%%%%%%%%%%%%%%%%%%%
% \section{Implementation}
%\iffalse
%<*package>
%\fi
%
% This section describes the definitions file |childdoc.def|.

% The definitions cannot be loaded using |\usepackage| or |\RequirePackage|
% which has a mechanism to prevent loading a style file more than once.
% When loading the definitions by means of |\input|
% multiple instances have to be prevented manually:
%\iffalse
%This code needs to be before the `\ProvidesFile' directive
%which is defined at the beginning of this file.
%Therefore it is also placed there and commented out here.
%</package>
%<*discard>
%\fi
%    \begin{macrocode}
\ifdefined\childdocmain\endinput\fi
%    \end{macrocode}
%\iffalse
%</discard>
%<*package>
%\fi
%
% \macro{\ifchilddoc}
% \macro{\ifchilddocmanual}
% The conditional |\ifchilddoc| tells whether a
% child (true) or main (false) document is being compiled.
% The conditional |\ifchilddocmanual| tells whether
% the |\includeonly| mechanism is used (false) or
% the selection of child files must be performed manually (true).
% The definitions initialise to false:
%    \begin{macrocode}
\newif\ifchilddoc
\newif\ifchilddocmanual
%    \end{macrocode}

% \macro{\childdocname}
% \macro{\childdocjob}
% The macro |\childdocname| stores the name of the main document
% to be compiled. The macro |\childdocjob| stores the name of
% the document on which the \LaTeX{} compiler was originally invoked.
% The content of |\jobname| cannot be compared
% to filenames specified in the source due to different catcodes.
% The following code rescans |\jobname|, stores the result
% in |\childdocname| and saves a copy in |\childdocjob|:
%    \begin{macrocode}
\edef\childdocname{\scantokens\expandafter{\jobname\noexpand}}
\let\childdocjob\childdocname
%    \end{macrocode}

% \macro{\childdocdisable}
% The macro |\childdocdisable| prevents the main file
% from being processed more than once.
% At this stage, the main document command |\childdocmain|
% is assumed to be called once again where it should do nothing.
% Any subsequent call to it should prevent
% a secondary processing of the main document
% It overwrites the forwarding commands
% |\childdocof| and |\childdocforward|
% with empty macros to prevent further inclusions of the main document:
%    \begin{macrocode}
\newcommand{\childdocdisable}
{
  \renewcommand{\childdocmain}[1]{\renewcommand{\childdocmain}[1]{\endinput}}
  \renewcommand{\childdocof}[1]{}
  \renewcommand{\childdocby}[2][]{}
  \renewcommand{\childdocforward}[2][]{}
  \renewcommand{\childdocdisable}{}
}
%    \end{macrocode}

% \macro{\childdocmain}
% The macro |\childdocmain| is to be called at the top of the main file
% with nothing or the main filename (without extension) as argument.
% First, it breaks loops.
% If the argument is not empty and does not match |\childdocname|
% (which is set by the first inclusion of |childdoc.def|),
% |\ifchilddoc| is set to true, |\includeonly| is applied to the child file
% and |\jobname| is set to the main file
% (for proper handling of |.aux| files):
%    \begin{macrocode}
\newcommand{\childdocmain}[1]
{
  \childdocdisable\childdocmain{}
  \if?#1?\else
    \begingroup
      \def\childdoctmp{#1}
      \ifx\childdoctmp\childdocname
        \def\childdoctmp{}
      \else
        \def\childdoctmp
        {
          \childdoctrue
          \includeonly{\childdocname}
          \def\childdocjob{#1}
          \def\jobname{#1}
        }
      \fi
      \expandafter
    \endgroup
    \childdoctmp
  \fi
}
%    \end{macrocode}

% \macro{\childdocof}
% The command |\childdocof| redirects
% compilation to the main file |#1|.
%    \begin{macrocode}
\newcommand{\childdocof}[1]
{
  \childdocdisable
  \childdoctrue
  \includeonly{\childdocname}
  \def\jobname{#1}
  \def\childdocjob{#1}
  \input{#1}
}
%    \end{macrocode}

% \macro{\childdocby}
% The command |\childdocby| ....
%    \begin{macrocode}
\newcommand{\childdocby}[2][]
{
  \childdocdisable
  \childdoctrue
  \childdocmanualtrue
  \if?#1?\else
    \def\jobname{#2}
  \fi
  \def\childdocjob{#2}
  \input{#2}
  \endinput
}
%    \end{macrocode}

% \macro{\childdocforward}
% The command |\childdocforward| redirects
% compilation to the main file or
% (if the optional argument is given) a child file.
% Parameters are set as if the main file
% or a child file starting with |\childdocof| was compiled.
% Then compilation is handed over to the main file:
%    \begin{macrocode}
\newcommand{\childdocforward}[2][]
{
  \begingroup
    \if?#1?
      \def\childdoctmp
      {
        \def\childdocname{#2}
        \def\childdocjob{#2}
        \def\jobname{#2}
        \input{#2}
        \endinput
      }
    \else
      \def\childdoctmp
      {
        \childdocdisable
        \def\childdocname{#2}
        \childdoctrue
        \includeonly{#2}
        \def\childdocjob{#1}
        \def\jobname{#1}
        \input{#1}
        \endinput
      }
    \fi
    \expandafter
  \endgroup
  \childdoctmp
}
%    \end{macrocode}

% \macro{\childdocforwardprefix}
% The command |\childdocforwardprefix| redirects
% compilation to the main or a child file by means of a pattern.
% The prefix |#1| in the current filename is replaced by |#2|
% and the suffix of the current filename is kept
% (it is assumed that the filename does not contain the substring `|~~~|'
% which is used as a delimiter).
% Compilation is handed over to the new file by |\childdocforward|:
%    \begin{macrocode}
\newcommand{\childdocforwardprefix}[3][]
{
  \begingroup
    \def\childdocextract #2##1~~~{\def\childdoctmp{\childdocforward[#1]{#3##1}}}
    \expandafter\childdocextract\childdocname~~~
    \expandafter
  \endgroup
  \childdoctmp
}
%    \end{macrocode}

% \macro{\childdoc}
% The deprecated macro |\childdoc| is a legacy version of |\childdocmain|:
%    \begin{macrocode}
\newcommand{\childdoc}{\childdocmain}
%    \end{macrocode}

% \macro{\childdocredirect}
% The deprecated macro |\childdocredirect| is a legacy version
% of |\childdocforward| and |\childdocforwardprefix|:
%    \begin{macrocode}
\newcommand{\childdocredirect}[2][]
{
  \begingroup
    \if?#1?
      \def\childdoctmp{\childdocforward{#2}}
    \else
      \def\childdoctmp{\childdocforwardprefix{#1}{#2}}
    \fi
    \expandafter
  \endgroup
  \childdoctmp
}
%    \end{macrocode}

%\iffalse
%</package>
%\fi
%
\endinput
|\\
|\childdocmain{}|\\
\end{tabular}
\end{center}
at the very top of the main \LaTeX{} file,
in particular \emph{before} the |\documentclass| statement!
The argument of |\childdocmain| should be left empty
(but it must be present).

%%%%%%%%%%%%%%%%%%%%%%%%%%%%%%%%%%%%%%%%
\DescribeMacro{\childdocof}
Furthermore, add the commands
\begin{center}
\begin{tabular}{l}
|% \iffalse
%
% childdoc.dtx Copyright (C) 2017-2018 Niklas Beisert
%
% This work may be distributed and/or modified under the
% conditions of the LaTeX Project Public License, either version 1.3
% of this license or (at your option) any later version.
% The latest version of this license is in
%   http://www.latex-project.org/lppl.txt
% and version 1.3 or later is part of all distributions of LaTeX
% version 2005/12/01 or later.
%
% This work has the LPPL maintenance status `maintained'.
%
% The Current Maintainer of this work is Niklas Beisert.
%
% This work consists of the files childdoc.dtx and childdoc.ins
% and the derived files childdoc.def and cdocsamp.tex with
% cdocsch1.tex, cdocsch2.tex, cdocsdrf.tex, cdocsfn1.tex, cdocsfn2.tex.
%
%<package>\ifdefined\childdocmain\endinput\fi
%<package>\ProvidesFile{childdoc.def}[2018/12/30 v2.0 child document driver]
%<samplemain>\ProvidesFile{cdocsamp.tex}[2018/12/30 v2.0 sample for childdoc]
%<*driver>
%\ProvidesFile{childdoc.drv}[2018/12/30 v2.0 childdoc reference manual file]
\PassOptionsToClass{10pt,a4paper}{article}
\documentclass{ltxdoc}

\usepackage[margin=35mm]{geometry}
\usepackage{hyperref}
\usepackage{hyperxmp}
\usepackage[usenames]{color}

\hypersetup{colorlinks=true}
\hypersetup{pdfstartview=FitH}
\hypersetup{pdfpagemode=UseNone}
\hypersetup{pdfsource={}}
\hypersetup{pdflang={en-UK}}
\hypersetup{pdfcopyright={Copyright 2017-2018 Niklas Beisert.
  This work may be distributed and/or modified under the
  conditions of the LaTeX Project Public License, either version 1.3
  of this license or (at your option) any later version.}}
\hypersetup{pdflicenseurl={http://www.latex-project.org/lppl.txt}}
\hypersetup{pdfcontactaddress={ETH Zurich, ITP, HIT K,
  Wolfgang-Pauli-Strasse 27}}
\hypersetup{pdfcontactpostcode={8093}}
\hypersetup{pdfcontactcity={Zurich}}
\hypersetup{pdfcontactcountry={Switzerland}}
\hypersetup{pdfcontactemail={nbeisert@itp.phys.ethz.ch}}
\hypersetup{pdfcontacturl={http://people.phys.ethz.ch/\xmptilde nbeisert/}}

\newcommand{\secref}[1]{\hyperref[#1]{section \ref*{#1}}}

\parskip1ex
\parindent0pt
\let\olditemize\itemize
\def\itemize{\olditemize\parskip0pt}

\begin{document}

\title{The \textsf{childdoc} Package}
\hypersetup{pdftitle={The childdoc Package}}
\author{Niklas Beisert\\[2ex]
  Institut f\"ur Theoretische Physik\\
  Eidgen\"ossische Technische Hochschule Z\"urich\\
  Wolfgang-Pauli-Strasse 27, 8093 Z\"urich, Switzerland\\[1ex]
  \href{mailto:nbeisert@itp.phys.ethz.ch}
  {\texttt{nbeisert@itp.phys.ethz.ch}}}
\hypersetup{pdfauthor={Niklas Beisert}}
\hypersetup{pdfsubject={Manual for the LaTeX2e Package childdoc}}
\date{30 December 2018, \textsf{v2.0}}
\maketitle

\begin{abstract}\noindent
\textsf{childdoc} is a \LaTeXe{} package
that enables the direct compilation
of document sections included by |\include|
to individual files.
\end{abstract}

\begingroup
\parskip0ex
\tableofcontents
\endgroup

%%%%%%%%%%%%%%%%%%%%%%%%%%%%%%%%%%%%%%%%%%%%%%%%%%%%%%%%%%%%%%%%%%%%%%%%%%%%%%%%
%%%%%%%%%%%%%%%%%%%%%%%%%%%%%%%%%%%%%%%%%%%%%%%%%%%%%%%%%%%%%%%%%%%%%%%%%%%%%%%%
\section{Introduction}

\LaTeX{} provides a mechanism to structure a large document (such as a book)
into a main file and several child files (containing the chapters)
using the |\include| command.
This mechanism is beneficial for documents
which span hundreds of pages in order to
make the source file(s) more manageable.
Moreover, compilation can be restricted to
selected child files by means of the |\includeonly| command.
The latter feature can be used to reduce the compilation time while editing
(this was significantly more useful in the earlier days of \LaTeX{})
or to generate a smaller document which is easier to navigate.
Another application of |\includeonly| is to generate
documents consisting of selected parts of the complete document.

However, there are a few drawbacks of the plain |\include| mechanism:
\begin{itemize}
\item
The child files cannot be compiled on their own,
they can only be compiled via the main file.
A naive editing environment
(such as a text editor with an option
to have the current file processed by \LaTeX)
may require one to switch to the main file before compiling;
attempting to compile the child file produces errors.
\item
The main file must be modified (each time)
to adjust the |\includeonly| command
to the present needs. This easily leaves the main file in a messy state.
\item
The generated document will always carry the filename
of the main document. This is inconvenient if
several child files are to be compiled and
to be kept for distribution.
\end{itemize}

The present package provides a simple interface
to make child files individually compilable by \LaTeX{}.
Compiling a child file then has the same effect as compiling
the main file with an |\includeonly| command
to select the appropriate child.
Moreover the generated document will carry the name of the child
rather than the main file.
This resolves all three above issues.

This feature is meant to make the editing of books,
thesis documents and lecture notes somewhat more convenient.
However, the package can also be used efficiently for
composing a series of documents (such as exercise sheets)
which are typically distributed individually.
It then assists the author in generating the individual documents
(potentially in different versions)
as well as a document containing the collected series.
Another application is in developing style files
or other kinds of included material
where compilation of the style file could redirect
to a sample or test file.

%%%%%%%%%%%%%%%%%%%%%%%%%%%%%%%%%%%%%%%%%%%%%%%%%%%%%%%%%%%%%%%%%%%%%%%%%%%%%%%%
%%%%%%%%%%%%%%%%%%%%%%%%%%%%%%%%%%%%%%%%%%%%%%%%%%%%%%%%%%%%%%%%%%%%%%%%%%%%%%%%
\section{Usage}

First of all, the package \textsf{childdoc} is \emph{not} a standard
\LaTeXe{} |.sty| style file! Therefore it needs to be invoked in
a non-standard way.

%%%%%%%%%%%%%%%%%%%%%%%%%%%%%%%%%%%%%%%%%%%%%%%%%%%%%%%%%%%%%%%%%%%%%%%%%%%%%%%%
\subsection{Included Files}
\label{sec:include}

%%%%%%%%%%%%%%%%%%%%%%%%%%%%%%%%%%%%%%%%
\DescribeMacro{\childdocmain}
To use the package, add the commands
\begin{center}
\begin{tabular}{l}
|\input{childdoc.def}|\\
|\childdocmain{}|\\
\end{tabular}
\end{center}
at the very top of the main \LaTeX{} file,
in particular \emph{before} the |\documentclass| statement!
The argument of |\childdocmain| should be left empty
(but it must be present).

%%%%%%%%%%%%%%%%%%%%%%%%%%%%%%%%%%%%%%%%
\DescribeMacro{\childdocof}
Furthermore, add the commands
\begin{center}
\begin{tabular}{l}
|\input{childdoc.def}|\\
|\childdocof{|\textit{main}|}|\\
\end{tabular}
\end{center}
at the top of every child file \textit{child}
which is included by |\include{|\textit{child}|}|
from within the main file
(or at least for those files to be compiled individually).
The argument \textit{main} must be the filename of the main file.

There are a couple of
considerations in setting up the main and child documents:

%%%%%%%%%%%%%%%%%%%%%%%%%%%%%%%%%%%%%%%%
\paragraph{Restrictions.}

Please note the following restrictions:
\begin{itemize}
\item
|\childdocmain| must be called with one argument \textit{main}
to ensure compatibility with earlier version of the package.
It must either be empty (|\childdocmain{}|)
or precisely match the filename of the main file in which it is specified.
See \secref{sec:detection} for further information.
\item
The filename \textit{main} must be specified without the |.tex| extension.
\item
The filename \textit{main} is case sensitive
(even in case-insensitive file systems)
due to internal string comparison.
\item
The argument \textit{main} should be fully expanded, it cannot be a macro.
\item
Subdirectories and special characters should be avoided in filenames.
\item
The command |\childdocmain{|\textit{main}|}| must be followed by a whitespace.
It should not be followed immediately by another command
or by a comment mark `|%|'.
This is because the \TeX{} parser reads the token immediately following
the argument of |\childdocmain| and puts it
at the beginning of every child section;
however, a white\-space is ignored.
\end{itemize}

%%%%%%%%%%%%%%%%%%%%%%%%%%%%%%%%%%%%%%%%
\paragraph{Content of Main File.}

It is advisable to place all content in the child files included by |\include|.
Any output contained in the main file will appear in all child documents
unless suppressed manually;
it cannot be suppressed automatically by the |\includeonly| directive
and thus should normally be avoided.
A method to include some content in the main file
by means of conditional processing is described in \secref{sec:conditional}.

%%%%%%%%%%%%%%%%%%%%%%%%%%%%%%%%%%%%%%%%
\paragraph{Page Numbering.}

When only a part of the document is compiled,
the appropriate numbering of pages
(as well as other status parameters)
is determined from the |.aux| files.
The latter contain information from previous passes.
However this information needs to propagate through
all intermediate child documents.
Therefore the page numbering in child documents may well
be inconsistent until the complete document is compiled at least once.

A useful (if unconventional) way to always ensure a consistent
page numbering is to restart the numbering in each child document
and denote the pages by `\textit{child}|.|\textit{page}'
where \textit{child} represents the chapter/section number of the child file.
This can be achieved by the command
|\numberwithin{page}{|\textit{child}|}|
of the \textsf{amsmath} package
where \textit{child} can be |chapter| or |section|
depending on the chosen structuring.
Alternatively, one can modify the macro |\thepage| appropriately
and reset the counter |page| at the start of each child file.

%%%%%%%%%%%%%%%%%%%%%%%%%%%%%%%%%%%%%%%%%%%%%%%%%%%%%%%%%%%%%%%%%%%%%%%%%%%%%%%%
\subsection{Conditional Processing}
\label{sec:conditional}

The package provides a mechanism to compile different versions
of a document. To customise the versions further some conditional processing
can come in handy to distinguish which version is being compiled.
The package provides two macros to describe the compilation context:

%%%%%%%%%%%%%%%%%%%%%%%%%%%%%%%%%%%%%%%%
\DescribeMacro{\ifchilddoc}
The conditional |\ifchilddoc| distinguishes between the compilation of
child documents and the main document:
%
\begin{center}
|\ifchilddoc |\textit{child-code}| |[|\||else |\textit{main-code}]| \||fi|
\end{center}

%%%%%%%%%%%%%%%%%%%%%%%%%%%%%%%%%%%%%%%%
\DescribeMacro{\childdocname}
\DescribeMacro{\childdocjob}
The macro |\childdocname| contains the filename (without extension)
of the main or child file being processed.
Note that |\childdocjob| will always contain the name of the main file.

%%%%%%%%%%%%%%%%%%%%%%%%%%%%%%%%%%%%%%%%
\paragraph{Title Page.}

Conditional processing can be used to include a title or banner page
in the main document when proper precautions are taken.
Importantly, the code in the main file should ensure that the page counter
(as well as other status parameters which are stored in the |.aux| files)
takes the same value after the conditional processing.
Otherwise the page numbers may take divergent values
depending on which part is compiled.

For example, a title page could be declared by:
%
\begin{center}
\begin{tabular}{l}
|\ifchilddoc\||else|\\
|\addtocounter{page}{-1}|\\
\textit{code for title page}\\
|\newpage|\\
|\||fi|
\end{tabular}
\end{center}
%
A banner page for the child documents can be generated by:
%
\begin{center}
\begin{tabular}{l}
|\ifchilddoc|\\
|\addtocounter{page}{-1}|\\
\textit{code for banner page}\\
|\newpage|\\
|\||fi|
\end{tabular}
\end{center}
%
Here one could write a message such as:
\begin{center}
|This is the part \childdocname{} of \childdocjob{}.|
\end{center}

%%%%%%%%%%%%%%%%%%%%%%%%%%%%%%%%%%%%%%%%%%%%%%%%%%%%%%%%%%%%%%%%%%%%%%%%%%%%%%%%
\subsection{Flags}
\label{sec:flags}

The package makes it easy to generate different versions
of the main or child documents.
To this end compilation flags can be defined
and assigned different default values.
They will be particularly useful in conjunction
with the forwarding mechanism described in \secref{sec:forward}.

For example, it may be useful to have a flag |\version|
which can be set to |draft| or |final|.
The document source will contain some conditional code
depending on the value of |\version|.
Suppose further, the flag should default to |final| for the main file
and to |draft| for child files
which is a natural assignment for editing the document.
This is achieved by placing the following code
in the preamble of the main document
(below the |\childdocmain| directive):
%
\begin{center}
\begin{tabular}{l}
|\ifchilddoc|\\
|\providecommand{\version}{draft}|\\
|\||else|\\
|\providecommand{\version}{final}|\\
|\||fi|
\end{tabular}
\end{center}
%
The definition by |\providecommand| makes sure
that previous definitions are not overwritten.
Further statements |\providecommand{\version}{...}|
can thus be added before the above code to override it.

For the main file, one might add a line
(between |\childdocmain| and the above block)
%
\begin{center}
|%\ifchilddoc\||else\providecommand{\version}{draft}\||fi|
\end{center}
%
which can be uncommented to produce a draft version.
Likewise one can add a line to the very top of a child file
(above the |\childdocof{|\textit{main}|}| directive)
%
\begin{center}
|%\providecommand{\version}{final}|
\end{center}
%
which can be uncommented to produce the final version of this child document.

%%%%%%%%%%%%%%%%%%%%%%%%%%%%%%%%%%%%%%%%%%%%%%%%%%%%%%%%%%%%%%%%%%%%%%%%%%%%%%%%
\subsection{Forwarding}
\label{sec:forward}

Different versions of the main or child documents
using compilation flags as described in \secref{sec:flags}
can be (permanently) stored in different files
for convenient compilation, viewing and distribution.
To this end, the package defines a command
to pass on compilation to a different file:

%%%%%%%%%%%%%%%%%%%%%%%%%%%%%%%%%%%%%%%%
\DescribeMacro{\childdocforward}
The command |\childdocforward| redirects processing to
another source file:
%
\begin{center}
\begin{tabular}{l}
|\input{childdoc.def}|\\
|\childdocforward[|\textit{main}|]{|\textit{dest}|}|\\
\end{tabular}
\end{center}
%
The argument \textit{dest} is the destination file
(without extension).
It should be the main file or one of the child files.
Note that further \textsf{childdoc} directives
such as |\childdocof| and |\childdocforward|
in the indicated file will be processed in this form.
The optional argument \textit{main}
passes on directly to the main file \textit{main}
while pretending to compile the child \textit{dest}.
This form behaves as if \textit{dest}
issues |\childdocof{|\textit{main}|}| right away,
and no further \textsf{childdoc} directives will be processed.

%%%%%%%%%%%%%%%%%%%%%%%%%%%%%%%%%%%%%%%%
\DescribeMacro{\...prefix}
In the alternative form |\childdocforwardprefix|,
%
\begin{center}
\begin{tabular}{l}
|\input{childdoc.def}|\\
|\childdocforwardprefix[|\textit{main}|]{|\textit{prefix}|}{|\textit{dest}|}|
\end{tabular}
\end{center}
%
the destination file is determined by a pattern
depending on the current file:
To make this work, the current file must be called
`{\textit{prefix}\hspace{0.2em}\textit{suffix}}'
with \textit{prefix} matching precisely the argument.
Processing is then passed on to the file
`{\textit{dest}\hspace{0.2em}\textit{suffix}}'.
Surely, the same effect is achieved by
directly specifying the
argument `{\textit{dest}\hspace{0.2em}\textit{suffix}}'
in the first form.
However, that requires to set up a different file
for each child. With the alternative form of the command
all these files can have exactly the same content
which simplifies setting them up and maintaining them.

For example, the following file |draft.tex|
with a compilation flag |\version| as described in \secref{sec:flags}
compiles the main document as a draft:
%
\begin{center}
\begin{tabular}{l}
|\def\version{draft}|\\
|\input{childdoc.def}|\\
|\childdocforward{|\textit{main}|}|
\end{tabular}
\end{center}
%
Likewise, the following files |final|\textit{nn}|.tex|
compile the final version of the child document
|child|\textit{nn}|.tex|:
%
\begin{center}
\begin{tabular}{l}
|\def\version{final}|\\
|\input{childdoc.def}|\\
|\childdocforwardprefix{final}{child}|
\end{tabular}
\end{center}
%

Note that when several versions of a main file and/or of each child file
are to be generated, it may be convenient to set up a |Makefile| or
shell script to automatise the process.

%%%%%%%%%%%%%%%%%%%%%%%%%%%%%%%%%%%%%%%%%%%%%%%%%%%%%%%%%%%%%%%%%%%%%%%%%%%%%%%%
\subsection{Command Line Processing}
\label{sec:commandline}

The effect of redirection files can also be achieved by invoking
the \LaTeX{} compiler with a more elaborate command line.
Most conveniently this should be done as part
of a shell script or a |Makefile|.

When using \textsf{childdoc} in the main file, the following
command lines effectively perform a redirection
(note that depending on the shell being used,
backslashes may have to be doubled: `|\|' $\to$ `|\\|'):
%
\begin{center}
|... -jobname "|\textit{target}|" |\\|"|[\textit{flags}]%
|\input{childdoc.def}\childdocforward[|\textit{main}|]{|\textit{dest}|}"|
\end{center}
%
Here \textit{target} is the name of the output file,
\textit{main} is the name of the main file
and \textit{dest} is the name of the main or child file to be processed
(all filenames without extensions).
The optional argument \textit{main} can be omitted
if \textit{main} matches \textit{dest}.
Optionally, compilation \textit{flags} can be defined via |\def| commands.
This command line makes the \TeX{} engine believe
it is compiling the file \textit{target}
whose content is specified as the latter parameter.
The provided code then forwards the processing to
\textit{main} or \textit{dest} as described in \secref{sec:forward}.

%%%%%%%%%%%%%%%%%%%%%%%%%%%%%%%%%%%%%%%%%%%%%%%%%%%%%%%%%%%%%%%%%%%%%%%%%%%%%%%%
\subsection{Include by Input}
\label{sec:input}

Including child documents by |\include| has some restrictions by design.
Most notably, the content of a child document always occupies
its own set of pages; pages cannot be shared between child documents.
Usually, this behaviour makes perfect sense
because each child document contain an essential part of the document.
However, in some situations it may be desirable to compose
a document from a collection of parts
without having mandatory page breaks between then.
For this case, the package
provides a mechanism to include parts
by |\input| which can also be processed individually.
However, by construction this mechanism
requires manual handling of the content to be output.

%%%%%%%%%%%%%%%%%%%%%%%%%%%%%%%%%%%%%%%%
\DescribeMacro{\ifchilddocmanual}
The main file should be prepared as usual, see \secref{sec:include}.
However, the document body must make a distinction
between processing of an individual part and of the main document, e.g.:
%
\begin{center}
\begin{tabular}{l}
|\ifchilddocmanual|\\
|\input{\childdocname}|\\
|\||else|\\
\textit{document body with }|\input{|\textit{part}|}|\\
|\||fi|
\end{tabular}
\end{center}
%
The conditional |\ifchilddocmanual| is true whenever
a part to be included by |\input| is being compiled,
and the name of the part is stored in |\childdocname|.

%%%%%%%%%%%%%%%%%%%%%%%%%%%%%%%%%%%%%%%%
\DescribeMacro{\childdocby}
Each part to be included by |\input| should start with:
%
\begin{center}
\begin{tabular}{l}
|\input{childdoc.def}|\\
|\childdocby{|\textit{main}|}|\\
\end{tabular}
\end{center}
%
The directive |\childdocby| is similar to |\childdocof|
described in \secref{sec:include},
but the subsequent selection of content must be done manually.
To that end, both |\ifchilddoc| and |\ifchilddocmanual|
will be true upon processing of a part,
and the name of the part is stored in |\childdocname|.
Note that |\jobname| will be set to the filename of the current part
so that each part receives an individual |.aux| file
that does not interfere with the |.aux| file(s) of the main document.
This behaviour can be altered by the alternative form
|\childdocby[*]{|\textit{main}|}| (with a non-empty optional argument)
which uses the |.aux| file of the main document
by setting |\jobname| to \textit{main}.

%%%%%%%%%%%%%%%%%%%%%%%%%%%%%%%%%%%%%%%%%%%%%%%%%%%%%%%%%%%%%%%%%%%%%%%%%%%%%%%%
\subsection{Driver Development}
\label{sec:driver}

The \textsf{childdoc} mechanism can also be use for the development
of definition files such as \LaTeX{} styles or classes.
This case differs from the above setup with multiple parts
included by |\include| in that no |\includeonly| should be invoked.
This can be achieved by starting the include file
(before |\ProvidesPackage|) with:
%
\begin{center}
\begin{tabular}{l}
|\input{childdoc.def}|\\
|\childdocforward{|\textit{main}|}|\\
\end{tabular}
\end{center}
%
or alternatively with:
%
\begin{center}
\begin{tabular}{l}
|\input{childdoc.def}|\\
|\childdocby{|\textit{main}|}|\\
\end{tabular}
\end{center}
%
Both forms have slightly different effects as described above.
The main file is prepared as usual, see \secref{sec:include}.

%%%%%%%%%%%%%%%%%%%%%%%%%%%%%%%%%%%%%%%%%%%%%%%%%%%%%%%%%%%%%%%%%%%%%%%%%%%%%%%%
\subsection{Legacy Detection}
\label{sec:detection}

The directive |\childdocmain| in the main file can detect
whether the complete document or merely a child is to be compiled
even without using the directive |\childdocof|.
This method is deprecated because it is less robust
and there is no compelling reason to use it;
it is merely provided for backward compatibility
and it may be removed in future versions.

If the detection mechanism is to be used,
it is mandatory to correctly specify
the filename of the main file as the argument of |\childdocmain|:
%
\begin{center}
\begin{tabular}{l}
|\input{childdoc.def}|\\
|\childdocmain{|\textit{main}|}|\\
\end{tabular}
\end{center}
%
If |\jobname| does not match the argument \textit{main} of |\childdocmain|,
it is assumed that |\jobname| points to the child file to be compiled.
When using |\childdocmain| with the main file specified as argument,
it suffices to start a child file
with just |\input{|\textit{main}|}|
without loading of the package and using |\childdocof|.
If instead all processing is done
with the appropriate \textsf{childdoc} directives,
the argument of \textit{main} of |\childdocmain| can be empty.

An alternative version of the command line processing described
in \secref{sec:commandline} using the detection mechanism reads:
%
\begin{center}
|... -jobname "|\textit{target}|" "|[\textit{flags}]%
[|\def\jobname{|\textit{dest}|}|]|\input{|\textit{main}|}"|
\end{center}

%%%%%%%%%%%%%%%%%%%%%%%%%%%%%%%%%%%%%%%%%%%%%%%%%%%%%%%%%%%%%%%%%%%%%%%%%%%%%%%%
\subsection{Manual Code}
\label{sec:manual}

In case one cannot be certain whether the definitions file |childdoc.def|
is installed on the target \TeX{} distribution
and one prefers not to ship it,
it is conceivable to paste a few relevant commands into the sources.

To that end, drop all statements |\input{childdoc.def}|
and perform the replacements as outlined below.
Instead of |\childdocmain{|\textit{main}|}| add the following code
to the top of the main file:
%
\begin{center}
\begin{tabular}{l}
|\||ifdefined\childdocname\endinput\||fi\newif\ifchilddoc|\\
|\edef\childdocname{\scantokens\expandafter{\jobname\noexpand}}|\\
|\def\childdocmain{|\textit{main}|}\||ifx\childdocmain\childdocname\||else|\\
|\childdoctrue\includeonly{\childdocname}\let\jobname\childdocmain\||fi|\\
\end{tabular}
\end{center}
%
Instead of |\childdocof{|\textit{main}|}| just include the main file
at the top of each child file:
%
\begin{center}
|\input{|\textit{main}|}|
\end{center}
%
A simple redirection |\childdocforward{|\textit{dest}|}| is achieved by:
%
\begin{center}
|\def\jobname{|\textit{dest}|}\input{\jobname}|
\end{center}
%
The redirection with prefix
|\childdocforwardprefix[|\textit{prefix}|]{|\textit{dest}|}|
is accomplished by:
%
\begin{center}
\begin{tabular}{l}
|{\edef\jobname{\scantokens\expandafter{\jobname\noexpand}}|\\
|\def\redirectjob |\textit{prefix}|#1~~~{\gdef\jobname{|\textit{dest}|#1}}|\\
|\expandafter\redirectjob\jobname~~~}\input{\jobname}|
\end{tabular}
\end{center}

In an alternative approach,
child documents can be compiled by a specific command line
without additional code or specific definitions:
%
\begin{center}
|... -jobname "|\textit{target}|" "|[\textit{flags}]%
|\includeonly{|\textit{dest}|}\input{|\textit{main}|}"|
\end{center}
%

%%%%%%%%%%%%%%%%%%%%%%%%%%%%%%%%%%%%%%%%%%%%%%%%%%%%%%%%%%%%%%%%%%%%%%%%%%%%%%%%
%%%%%%%%%%%%%%%%%%%%%%%%%%%%%%%%%%%%%%%%%%%%%%%%%%%%%%%%%%%%%%%%%%%%%%%%%%%%%%%%
\section{Information}

%%%%%%%%%%%%%%%%%%%%%%%%%%%%%%%%%%%%%%%%%%%%%%%%%%%%%%%%%%%%%%%%%%%%%%%%%%%%%%%%
\subsection{Copyright}

Copyright \copyright{} 2017--2018 Niklas Beisert

This work may be distributed and/or modified under the
conditions of the \LaTeX{} Project Public License, either version 1.3
of this license or (at your option) any later version.
The latest version of this license is in
  \url{http://www.latex-project.org/lppl.txt}
and version 1.3 or later is part of all distributions of \LaTeX{}
version 2005/12/01 or later.

This work has the LPPL maintenance status `maintained'.

The Current Maintainer of this work is Niklas Beisert.

This work consists of the files |README.txt|, |childdoc.ins| and |childdoc.dtx|
as well as the derived files |childdoc.def|, |cdocsamp.tex|
with |cdocsch1.tex|, |cdocsch2.tex|, |cdocspt3.tex|, |cdocspt4.tex|,
|cdocsdrf.tex|, |cdocsfn1.tex|, |cdocsfn2.tex|
as well as |childdoc.pdf|.

%%%%%%%%%%%%%%%%%%%%%%%%%%%%%%%%%%%%%%%%%%%%%%%%%%%%%%%%%%%%%%%%%%%%%%%%%%%%%%%%
\subsection{Files and Installation}

The package consists of the files:
%
\begin{center}
\begin{tabular}{ll}
    |README.txt|   & readme file \\
    |childdoc.ins| & installation file \\
    |childdoc.dtx| & source file \\
    |childdoc.def| & definition file \\
    |cdocsamp.tex| & sample main file \\
    |cdocsch1.tex| & sample include file \\
    |cdocsch2.tex| & sample include file \\
    |cdocspt3.tex| & sample part file \\
    |cdocspt4.tex| & sample part file \\
    |cdocsdrf.tex| & sample redirection file \\
    |cdocsfn1.tex| & sample redirection file \\
    |cdocsfn2.tex| & sample redirection file \\
    |childdoc.pdf| & manual
\end{tabular}
\end{center}
%
The distribution consists of the files
|README.txt|, |childdoc.ins| and |childdoc.dtx|.
%
\begin{itemize}
\item
Run (pdf)\LaTeX{} on |childdoc.dtx|
to compile the manual |childdoc.pdf| (this file).
\item
Run \LaTeX{} on |childdoc.ins| to create the definitions file |childdoc.def|
and the sample |cdocsamp.tex| with include files
|cdocsch1.tex|, |cdocsch2.tex|, |cdocspt3.tex|, |cdocspt4.tex|,
|cdocsdrf.tex|, |cdocsfn1.tex|, |cdocsfn2.tex|.
Then copy the file |childdoc.def| to an appropriate directory of your \LaTeX{}
distribution, e.g.\ \textit{texmf-root}|/tex/latex/childdoc|.
\end{itemize}

%%%%%%%%%%%%%%%%%%%%%%%%%%%%%%%%%%%%%%%%%%%%%%%%%%%%%%%%%%%%%%%%%%%%%%%%%%%%%%%%
\subsection{Related CTAN Packages}

There are several other packages which offer a similar functionality:
%
\begin{itemize}
\item
The packages
\href{http://ctan.org/pkg/docmute}{\textsf{docmute}},
\href{http://ctan.org/pkg/includex}{\textsf{includex}} and
\href{http://ctan.org/pkg/standalone}{\textsf{standalone}}
provide commands to include only the document body of
a child file thus allowing both files to be compiled individually.
\item
The packages \href{http://ctan.org/pkg/subdocs}{\textsf{subdocs}}
and \href{http://ctan.org/pkg/subfiles}{\textsf{subfiles}}
provide structures in which the main and child documents can be
encapsulated and allowing them to be compiled individually.
The inclusion mechanism is different from the conventional |\include|.
\item
The package \href{http://ctan.org/pkg/combine}{\textsf{combine}}
is an elaborate solution to combine several documents into one.
\end{itemize}
%
See also the CTAN topic \href{http://ctan.org/topic/subdocs}{\textsf{subdocs}}
for further related packages.
The present package differs from the above solutions in that
a document structure constructed with the conventional |\include| mechanism
just needs two extra commands at the top of every file
such that all constituent files can be compiled individually.

%%%%%%%%%%%%%%%%%%%%%%%%%%%%%%%%%%%%%%%%%%%%%%%%%%%%%%%%%%%%%%%%%%%%%%%%%%%%%%%%
%\subsection{Feature Suggestions}
%
%The following is a list of features which may be useful for future
%versions of this package:
%%
%\begin{itemize}
%\item
%\ldots
%\end{itemize}

%%%%%%%%%%%%%%%%%%%%%%%%%%%%%%%%%%%%%%%%%%%%%%%%%%%%%%%%%%%%%%%%%%%%%%%%%%%%%%%%
\subsection{Revision History}

%%%%%%%%%%%%%%%%%%%%%%%%%%%%%%%%%%%%%%%%
\paragraph{v2.0:} 2018/12/30

\begin{itemize}
\item
immediate forward processing
\item
added |\childdocby| mechanism
\item
manual restructured
\end{itemize}

%%%%%%%%%%%%%%%%%%%%%%%%%%%%%%%%%%%%%%%%
\paragraph{v1.6:} 2018/01/17

\begin{itemize}
\item
application for development of include files
\item
corrections to manual
\end{itemize}

%%%%%%%%%%%%%%%%%%%%%%%%%%%%%%%%%%%%%%%%
\paragraph{v1.5:} 2017/05/21

\begin{itemize}
\item
more complete structuring introduced
\item
|\childdocof| introduced
\item
|\childdoc| renamed to |\childdocmain|
\item
|\childredirect| renamed to |\childdocforward| and |\childdocforwardprefix|
and functionality expanded
\end{itemize}

%%%%%%%%%%%%%%%%%%%%%%%%%%%%%%%%%%%%%%%%
\paragraph{v1.0:} 2017/04/27

\begin{itemize}
\item
manual and install package
\item
first version published on CTAN
\end{itemize}

%%%%%%%%%%%%%%%%%%%%%%%%%%%%%%%%%%%%%%%%
\paragraph{v0.6:} 2017/04/26

\begin{itemize}
\item
redirection mechanism added
\end{itemize}

%%%%%%%%%%%%%%%%%%%%%%%%%%%%%%%%%%%%%%%%
\paragraph{v0.5:} 2017/04/26

\begin{itemize}
\item
functionality in definition file
\end{itemize}


%%%%%%%%%%%%%%%%%%%%%%%%%%%%%%%%%%%%%%%%%%%%%%%%%%%%%%%%%%%%%%%%%%%%%%%%%%%%%%%%
%%%%%%%%%%%%%%%%%%%%%%%%%%%%%%%%%%%%%%%%%%%%%%%%%%%%%%%%%%%%%%%%%%%%%%%%%%%%%%%%
%%%%%%%%%%%%%%%%%%%%%%%%%%%%%%%%%%%%%%%%%%%%%%%%%%%%%%%%%%%%%%%%%%%%%%%%%%%%%%%%
\appendix

\settowidth\MacroIndent{\rmfamily\scriptsize 000\ }

 \DocInput{childdoc.dtx}

\end{document}
%</driver>
% \fi
%
% %%%%%%%%%%%%%%%%%%%%%%%%%%%%%%%%%%%%%%%%%%%%%%%%%%%%%%%%%%%%%%%%%%%%%%%%%%%%%%
% %%%%%%%%%%%%%%%%%%%%%%%%%%%%%%%%%%%%%%%%%%%%%%%%%%%%%%%%%%%%%%%%%%%%%%%%%%%%%%
% \section{Sample}
%\iffalse
%<*samplemain>
%\fi
%
% The following presents a sample document
% with two chapters, two parts, a title page,
% a compile flag as well as three forwarding files to set the flag.
% It consists of eight |.tex| files:
% \begin{center}
% \begin{tabular}{ll}
% |cdocsamp.tex|&main file\\
% |cdocsch1.tex|&include file for chapter 1\\
% |cdocsch2.tex|&include file for chapter 2\\
% |cdocspt3.tex|&include file for part 3\\
% |cdocspt4.tex|&include file for part 4\\
% |cdocsdrf.tex|&forwarding file for main file in draft mode\\
% |cdocsfi1.tex|&forwarding file for final version of chapter 1\\
% |cdocsfi2.tex|&forwarding file for final version of chapter 2\\
% \end{tabular}
% \end{center}
% Each of the eight files can be compiled directly by the \LaTeX{} compiler.
%
% %%%%%%%%%%%%%%%%%%%%%%%%%%%%%%%%%%%%%%
% \paragraph{Main File.}
%
% The main file is called |cdocsamp.tex|.
%
% Load the \textsf{childdoc} definitions and
% declare the filename for the main document:
%    \begin{macrocode}
\input{childdoc.def}
\childdocmain{}
%    \end{macrocode}

% Optional override for |\version| flag:
%    \begin{macrocode}
%%\ifchilddoc\else\providecommand{\version}{draft}\fi
%    \end{macrocode}

% Define the default values for the |\version| flag
% (|final| for the main file and |draft| for childs):
%    \begin{macrocode}
\ifchilddoc
\providecommand{\version}{draft}
\else
\providecommand{\version}{final}
\fi
%    \end{macrocode}

% Load the standard document class:
%    \begin{macrocode}
\documentclass[12pt]{article}
%    \end{macrocode}

% Start the document body:
%    \begin{macrocode}
\begin{document}
%    \end{macrocode}

% Declare a title page.
% Print title, part of document being processed and version flag:
%    \begin{macrocode}
\addtocounter{page}{-1}
\begin{center}
{\LARGE\bfseries{}childdoc example\par}
\vspace{1cm}
\ifchilddoc
\ifchilddocmanual part\else chapter\fi:
`\childdocname' of `\childdocjob'\par
\else
main document: `\childdocjob'\par
\fi
version: \version\par
\end{center}
\newpage
%    \end{macrocode}

% Manually include selected file,
% otherwise process as usual:
%    \begin{macrocode}
\ifchilddocmanual
\section*{part `\childdocname'}
\input{\childdocname}
\else
%    \end{macrocode}

% Include the two chapters:
%    \begin{macrocode}
\include{cdocsch1}
\include{cdocsch2}
%    \end{macrocode}

% Include the two parts unless only chapters should be displayed:
%    \begin{macrocode}
\ifchilddoc\else
\section{part three}
\input{cdocspt3}
\section{part four}
\input{cdocspt4}
\fi
%    \end{macrocode}

% Process as usual until here:
%    \begin{macrocode}
\fi
%    \end{macrocode}

% End of document body:
%    \begin{macrocode}
\end{document}
%    \end{macrocode}
%\iffalse
%</samplemain>
%\fi
%
% %%%%%%%%%%%%%%%%%%%%%%%%%%%%%%%%%%%%%%
% \paragraph{Chapter Include Files.}
%
% The include files are called |cdocsch1.tex| and |cdocsch2.tex|.
%
%\iffalse
%<*samplechap1|samplechap2>
%\fi

% Optional override for |\version| flag:
%    \begin{macrocode}
%%\providecommand{\version}{final}
%    \end{macrocode}

% Include the main document:
%    \begin{macrocode}
\input{childdoc.def}
\childdocof{cdocsamp}
%    \end{macrocode}

%\iffalse
%</samplechap1|samplechap2>
%\fi
%
%\iffalse
%<*samplechap1>
%\fi
% Some text for chapter 1:
%    \begin{macrocode}
\section{one}
some text in chapter one
%    \end{macrocode}

%\iffalse
%</samplechap1>
%\fi
% Some text for chapter 2:
%\iffalse
%<*samplechap2>
%\fi
%    \begin{macrocode}
\section{two}
more text in chapter two
%    \end{macrocode}

%\iffalse
%</samplechap2>
%\fi
%
% %%%%%%%%%%%%%%%%%%%%%%%%%%%%%%%%%%%%%%
% \paragraph{Part Include Files.}
%
% The include files are called |cdocspt3.tex| and |cdocspt4.tex|.
%
%\iffalse
%<*samplepart3|samplepart4>
%\fi

% Optional override for |\version| flag:
%    \begin{macrocode}
%%\providecommand{\version}{final}
%    \end{macrocode}

% Include the main document:
%    \begin{macrocode}
\input{childdoc.def}
\childdocby{cdocsamp}
%    \end{macrocode}

%\iffalse
%</samplepart3|samplepart4>
%\fi
%
%\iffalse
%<*samplepart3>
%\fi
% Some text for part 3:
%    \begin{macrocode}
some text in part three
%    \end{macrocode}

%\iffalse
%</samplepart3>
%\fi
% Some text for part 4:
%\iffalse
%<*samplepart4>
%\fi
%    \begin{macrocode}
more text in part four
%    \end{macrocode}

%\iffalse
%</samplepart4>
%\fi
%
% %%%%%%%%%%%%%%%%%%%%%%%%%%%%%%%%%%%%%%
% \paragraph{Forwarding for a Complete Draft.}
%
% The following forwarding file |cdocsdrf.tex|
% compiles the main document in draft mode:
%\iffalse
%<*sampledraft>
%\fi
%    \begin{macrocode}
\def\version{draft}
\input{childdoc.def}
\childdocforward{cdocsamp}
%    \end{macrocode}

%\iffalse
%</sampledraft>
%\fi
%
% %%%%%%%%%%%%%%%%%%%%%%%%%%%%%%%%%%%%%%
% \paragraph{Forwarding for Final Version of the Chapters.}
%
% The following forwarding files |cdocsfn1.tex| and |cdocsfn2.tex|
% (with identical content)
% compile the final versions of the child documents
% |cdocsch1.tex| and |cdocsch2.tex|, respectively:
%\iffalse
%<*samplefinal>
%\fi
%    \begin{macrocode}
\def\version{final}
\input{childdoc.def}
\childdocforwardprefix[cdocsamp]{cdocsfn}{cdocsch}
%    \end{macrocode}

%\iffalse
%</samplefinal>
%\fi
%
% %%%%%%%%%%%%%%%%%%%%%%%%%%%%%%%%%%%%%%
% \paragraph{Command Line Processing.}
%
% The following three command lines generate the output files
% |cdocscld|, |cdocscl1| and |cdocscl2|
% which should be identical to
% |cdocsdrf|, |cdocsch1| and |cdocsfn2|, respectively:
% \begin{center}
% \begin{tabular}{l}
% |latex -jobname cdocscld \|\\
% |  "\def\version{draft}\input{childdoc.def}\childdocforward{cdocsamp}"|\\
% |latex -jobname cdocscl1 \|\\
% |  "\input{childdoc.def}\childdocforward[cdocsamp]{cdocsch1}"|\\
% |latex -jobname cdocscl2 \|\\
% |  "\def\version{final}\input{childdoc.def}\childdocforward{cdocsch2}"|
% \end{tabular}
% \end{center}
% Note that the trailing backslash on each first line
% merely continues the input to the second line
% (for convenient cut ant paste).
% Furthermore, the command |latex| can be replaced by any
% of its alternative versions such as |pdflatex|.
%
% %%%%%%%%%%%%%%%%%%%%%%%%%%%%%%%%%%%%%%%%%%%%%%%%%%%%%%%%%%%%%%%%%%%%%%%%%%%%%%
% %%%%%%%%%%%%%%%%%%%%%%%%%%%%%%%%%%%%%%%%%%%%%%%%%%%%%%%%%%%%%%%%%%%%%%%%%%%%%%
% \section{Implementation}
%\iffalse
%<*package>
%\fi
%
% This section describes the definitions file |childdoc.def|.

% The definitions cannot be loaded using |\usepackage| or |\RequirePackage|
% which has a mechanism to prevent loading a style file more than once.
% When loading the definitions by means of |\input|
% multiple instances have to be prevented manually:
%\iffalse
%This code needs to be before the `\ProvidesFile' directive
%which is defined at the beginning of this file.
%Therefore it is also placed there and commented out here.
%</package>
%<*discard>
%\fi
%    \begin{macrocode}
\ifdefined\childdocmain\endinput\fi
%    \end{macrocode}
%\iffalse
%</discard>
%<*package>
%\fi
%
% \macro{\ifchilddoc}
% \macro{\ifchilddocmanual}
% The conditional |\ifchilddoc| tells whether a
% child (true) or main (false) document is being compiled.
% The conditional |\ifchilddocmanual| tells whether
% the |\includeonly| mechanism is used (false) or
% the selection of child files must be performed manually (true).
% The definitions initialise to false:
%    \begin{macrocode}
\newif\ifchilddoc
\newif\ifchilddocmanual
%    \end{macrocode}

% \macro{\childdocname}
% \macro{\childdocjob}
% The macro |\childdocname| stores the name of the main document
% to be compiled. The macro |\childdocjob| stores the name of
% the document on which the \LaTeX{} compiler was originally invoked.
% The content of |\jobname| cannot be compared
% to filenames specified in the source due to different catcodes.
% The following code rescans |\jobname|, stores the result
% in |\childdocname| and saves a copy in |\childdocjob|:
%    \begin{macrocode}
\edef\childdocname{\scantokens\expandafter{\jobname\noexpand}}
\let\childdocjob\childdocname
%    \end{macrocode}

% \macro{\childdocdisable}
% The macro |\childdocdisable| prevents the main file
% from being processed more than once.
% At this stage, the main document command |\childdocmain|
% is assumed to be called once again where it should do nothing.
% Any subsequent call to it should prevent
% a secondary processing of the main document
% It overwrites the forwarding commands
% |\childdocof| and |\childdocforward|
% with empty macros to prevent further inclusions of the main document:
%    \begin{macrocode}
\newcommand{\childdocdisable}
{
  \renewcommand{\childdocmain}[1]{\renewcommand{\childdocmain}[1]{\endinput}}
  \renewcommand{\childdocof}[1]{}
  \renewcommand{\childdocby}[2][]{}
  \renewcommand{\childdocforward}[2][]{}
  \renewcommand{\childdocdisable}{}
}
%    \end{macrocode}

% \macro{\childdocmain}
% The macro |\childdocmain| is to be called at the top of the main file
% with nothing or the main filename (without extension) as argument.
% First, it breaks loops.
% If the argument is not empty and does not match |\childdocname|
% (which is set by the first inclusion of |childdoc.def|),
% |\ifchilddoc| is set to true, |\includeonly| is applied to the child file
% and |\jobname| is set to the main file
% (for proper handling of |.aux| files):
%    \begin{macrocode}
\newcommand{\childdocmain}[1]
{
  \childdocdisable\childdocmain{}
  \if?#1?\else
    \begingroup
      \def\childdoctmp{#1}
      \ifx\childdoctmp\childdocname
        \def\childdoctmp{}
      \else
        \def\childdoctmp
        {
          \childdoctrue
          \includeonly{\childdocname}
          \def\childdocjob{#1}
          \def\jobname{#1}
        }
      \fi
      \expandafter
    \endgroup
    \childdoctmp
  \fi
}
%    \end{macrocode}

% \macro{\childdocof}
% The command |\childdocof| redirects
% compilation to the main file |#1|.
%    \begin{macrocode}
\newcommand{\childdocof}[1]
{
  \childdocdisable
  \childdoctrue
  \includeonly{\childdocname}
  \def\jobname{#1}
  \def\childdocjob{#1}
  \input{#1}
}
%    \end{macrocode}

% \macro{\childdocby}
% The command |\childdocby| ....
%    \begin{macrocode}
\newcommand{\childdocby}[2][]
{
  \childdocdisable
  \childdoctrue
  \childdocmanualtrue
  \if?#1?\else
    \def\jobname{#2}
  \fi
  \def\childdocjob{#2}
  \input{#2}
  \endinput
}
%    \end{macrocode}

% \macro{\childdocforward}
% The command |\childdocforward| redirects
% compilation to the main file or
% (if the optional argument is given) a child file.
% Parameters are set as if the main file
% or a child file starting with |\childdocof| was compiled.
% Then compilation is handed over to the main file:
%    \begin{macrocode}
\newcommand{\childdocforward}[2][]
{
  \begingroup
    \if?#1?
      \def\childdoctmp
      {
        \def\childdocname{#2}
        \def\childdocjob{#2}
        \def\jobname{#2}
        \input{#2}
        \endinput
      }
    \else
      \def\childdoctmp
      {
        \childdocdisable
        \def\childdocname{#2}
        \childdoctrue
        \includeonly{#2}
        \def\childdocjob{#1}
        \def\jobname{#1}
        \input{#1}
        \endinput
      }
    \fi
    \expandafter
  \endgroup
  \childdoctmp
}
%    \end{macrocode}

% \macro{\childdocforwardprefix}
% The command |\childdocforwardprefix| redirects
% compilation to the main or a child file by means of a pattern.
% The prefix |#1| in the current filename is replaced by |#2|
% and the suffix of the current filename is kept
% (it is assumed that the filename does not contain the substring `|~~~|'
% which is used as a delimiter).
% Compilation is handed over to the new file by |\childdocforward|:
%    \begin{macrocode}
\newcommand{\childdocforwardprefix}[3][]
{
  \begingroup
    \def\childdocextract #2##1~~~{\def\childdoctmp{\childdocforward[#1]{#3##1}}}
    \expandafter\childdocextract\childdocname~~~
    \expandafter
  \endgroup
  \childdoctmp
}
%    \end{macrocode}

% \macro{\childdoc}
% The deprecated macro |\childdoc| is a legacy version of |\childdocmain|:
%    \begin{macrocode}
\newcommand{\childdoc}{\childdocmain}
%    \end{macrocode}

% \macro{\childdocredirect}
% The deprecated macro |\childdocredirect| is a legacy version
% of |\childdocforward| and |\childdocforwardprefix|:
%    \begin{macrocode}
\newcommand{\childdocredirect}[2][]
{
  \begingroup
    \if?#1?
      \def\childdoctmp{\childdocforward{#2}}
    \else
      \def\childdoctmp{\childdocforwardprefix{#1}{#2}}
    \fi
    \expandafter
  \endgroup
  \childdoctmp
}
%    \end{macrocode}

%\iffalse
%</package>
%\fi
%
\endinput
|\\
|\childdocof{|\textit{main}|}|\\
\end{tabular}
\end{center}
at the top of every child file \textit{child}
which is included by |\include{|\textit{child}|}|
from within the main file
(or at least for those files to be compiled individually).
The argument \textit{main} must be the filename of the main file.

There are a couple of
considerations in setting up the main and child documents:

%%%%%%%%%%%%%%%%%%%%%%%%%%%%%%%%%%%%%%%%
\paragraph{Restrictions.}

Please note the following restrictions:
\begin{itemize}
\item
|\childdocmain| must be called with one argument \textit{main}
to ensure compatibility with earlier version of the package.
It must either be empty (|\childdocmain{}|)
or precisely match the filename of the main file in which it is specified.
See \secref{sec:detection} for further information.
\item
The filename \textit{main} must be specified without the |.tex| extension.
\item
The filename \textit{main} is case sensitive
(even in case-insensitive file systems)
due to internal string comparison.
\item
The argument \textit{main} should be fully expanded, it cannot be a macro.
\item
Subdirectories and special characters should be avoided in filenames.
\item
The command |\childdocmain{|\textit{main}|}| must be followed by a whitespace.
It should not be followed immediately by another command
or by a comment mark `|%|'.
This is because the \TeX{} parser reads the token immediately following
the argument of |\childdocmain| and puts it
at the beginning of every child section;
however, a white\-space is ignored.
\end{itemize}

%%%%%%%%%%%%%%%%%%%%%%%%%%%%%%%%%%%%%%%%
\paragraph{Content of Main File.}

It is advisable to place all content in the child files included by |\include|.
Any output contained in the main file will appear in all child documents
unless suppressed manually;
it cannot be suppressed automatically by the |\includeonly| directive
and thus should normally be avoided.
A method to include some content in the main file
by means of conditional processing is described in \secref{sec:conditional}.

%%%%%%%%%%%%%%%%%%%%%%%%%%%%%%%%%%%%%%%%
\paragraph{Page Numbering.}

When only a part of the document is compiled,
the appropriate numbering of pages
(as well as other status parameters)
is determined from the |.aux| files.
The latter contain information from previous passes.
However this information needs to propagate through
all intermediate child documents.
Therefore the page numbering in child documents may well
be inconsistent until the complete document is compiled at least once.

A useful (if unconventional) way to always ensure a consistent
page numbering is to restart the numbering in each child document
and denote the pages by `\textit{child}|.|\textit{page}'
where \textit{child} represents the chapter/section number of the child file.
This can be achieved by the command
|\numberwithin{page}{|\textit{child}|}|
of the \textsf{amsmath} package
where \textit{child} can be |chapter| or |section|
depending on the chosen structuring.
Alternatively, one can modify the macro |\thepage| appropriately
and reset the counter |page| at the start of each child file.

%%%%%%%%%%%%%%%%%%%%%%%%%%%%%%%%%%%%%%%%%%%%%%%%%%%%%%%%%%%%%%%%%%%%%%%%%%%%%%%%
\subsection{Conditional Processing}
\label{sec:conditional}

The package provides a mechanism to compile different versions
of a document. To customise the versions further some conditional processing
can come in handy to distinguish which version is being compiled.
The package provides two macros to describe the compilation context:

%%%%%%%%%%%%%%%%%%%%%%%%%%%%%%%%%%%%%%%%
\DescribeMacro{\ifchilddoc}
The conditional |\ifchilddoc| distinguishes between the compilation of
child documents and the main document:
%
\begin{center}
|\ifchilddoc |\textit{child-code}| |[|\||else |\textit{main-code}]| \||fi|
\end{center}

%%%%%%%%%%%%%%%%%%%%%%%%%%%%%%%%%%%%%%%%
\DescribeMacro{\childdocname}
\DescribeMacro{\childdocjob}
The macro |\childdocname| contains the filename (without extension)
of the main or child file being processed.
Note that |\childdocjob| will always contain the name of the main file.

%%%%%%%%%%%%%%%%%%%%%%%%%%%%%%%%%%%%%%%%
\paragraph{Title Page.}

Conditional processing can be used to include a title or banner page
in the main document when proper precautions are taken.
Importantly, the code in the main file should ensure that the page counter
(as well as other status parameters which are stored in the |.aux| files)
takes the same value after the conditional processing.
Otherwise the page numbers may take divergent values
depending on which part is compiled.

For example, a title page could be declared by:
%
\begin{center}
\begin{tabular}{l}
|\ifchilddoc\||else|\\
|\addtocounter{page}{-1}|\\
\textit{code for title page}\\
|\newpage|\\
|\||fi|
\end{tabular}
\end{center}
%
A banner page for the child documents can be generated by:
%
\begin{center}
\begin{tabular}{l}
|\ifchilddoc|\\
|\addtocounter{page}{-1}|\\
\textit{code for banner page}\\
|\newpage|\\
|\||fi|
\end{tabular}
\end{center}
%
Here one could write a message such as:
\begin{center}
|This is the part \childdocname{} of \childdocjob{}.|
\end{center}

%%%%%%%%%%%%%%%%%%%%%%%%%%%%%%%%%%%%%%%%%%%%%%%%%%%%%%%%%%%%%%%%%%%%%%%%%%%%%%%%
\subsection{Flags}
\label{sec:flags}

The package makes it easy to generate different versions
of the main or child documents.
To this end compilation flags can be defined
and assigned different default values.
They will be particularly useful in conjunction
with the forwarding mechanism described in \secref{sec:forward}.

For example, it may be useful to have a flag |\version|
which can be set to |draft| or |final|.
The document source will contain some conditional code
depending on the value of |\version|.
Suppose further, the flag should default to |final| for the main file
and to |draft| for child files
which is a natural assignment for editing the document.
This is achieved by placing the following code
in the preamble of the main document
(below the |\childdocmain| directive):
%
\begin{center}
\begin{tabular}{l}
|\ifchilddoc|\\
|\providecommand{\version}{draft}|\\
|\||else|\\
|\providecommand{\version}{final}|\\
|\||fi|
\end{tabular}
\end{center}
%
The definition by |\providecommand| makes sure
that previous definitions are not overwritten.
Further statements |\providecommand{\version}{...}|
can thus be added before the above code to override it.

For the main file, one might add a line
(between |\childdocmain| and the above block)
%
\begin{center}
|%\ifchilddoc\||else\providecommand{\version}{draft}\||fi|
\end{center}
%
which can be uncommented to produce a draft version.
Likewise one can add a line to the very top of a child file
(above the |\childdocof{|\textit{main}|}| directive)
%
\begin{center}
|%\providecommand{\version}{final}|
\end{center}
%
which can be uncommented to produce the final version of this child document.

%%%%%%%%%%%%%%%%%%%%%%%%%%%%%%%%%%%%%%%%%%%%%%%%%%%%%%%%%%%%%%%%%%%%%%%%%%%%%%%%
\subsection{Forwarding}
\label{sec:forward}

Different versions of the main or child documents
using compilation flags as described in \secref{sec:flags}
can be (permanently) stored in different files
for convenient compilation, viewing and distribution.
To this end, the package defines a command
to pass on compilation to a different file:

%%%%%%%%%%%%%%%%%%%%%%%%%%%%%%%%%%%%%%%%
\DescribeMacro{\childdocforward}
The command |\childdocforward| redirects processing to
another source file:
%
\begin{center}
\begin{tabular}{l}
|% \iffalse
%
% childdoc.dtx Copyright (C) 2017-2018 Niklas Beisert
%
% This work may be distributed and/or modified under the
% conditions of the LaTeX Project Public License, either version 1.3
% of this license or (at your option) any later version.
% The latest version of this license is in
%   http://www.latex-project.org/lppl.txt
% and version 1.3 or later is part of all distributions of LaTeX
% version 2005/12/01 or later.
%
% This work has the LPPL maintenance status `maintained'.
%
% The Current Maintainer of this work is Niklas Beisert.
%
% This work consists of the files childdoc.dtx and childdoc.ins
% and the derived files childdoc.def and cdocsamp.tex with
% cdocsch1.tex, cdocsch2.tex, cdocsdrf.tex, cdocsfn1.tex, cdocsfn2.tex.
%
%<package>\ifdefined\childdocmain\endinput\fi
%<package>\ProvidesFile{childdoc.def}[2018/12/30 v2.0 child document driver]
%<samplemain>\ProvidesFile{cdocsamp.tex}[2018/12/30 v2.0 sample for childdoc]
%<*driver>
%\ProvidesFile{childdoc.drv}[2018/12/30 v2.0 childdoc reference manual file]
\PassOptionsToClass{10pt,a4paper}{article}
\documentclass{ltxdoc}

\usepackage[margin=35mm]{geometry}
\usepackage{hyperref}
\usepackage{hyperxmp}
\usepackage[usenames]{color}

\hypersetup{colorlinks=true}
\hypersetup{pdfstartview=FitH}
\hypersetup{pdfpagemode=UseNone}
\hypersetup{pdfsource={}}
\hypersetup{pdflang={en-UK}}
\hypersetup{pdfcopyright={Copyright 2017-2018 Niklas Beisert.
  This work may be distributed and/or modified under the
  conditions of the LaTeX Project Public License, either version 1.3
  of this license or (at your option) any later version.}}
\hypersetup{pdflicenseurl={http://www.latex-project.org/lppl.txt}}
\hypersetup{pdfcontactaddress={ETH Zurich, ITP, HIT K,
  Wolfgang-Pauli-Strasse 27}}
\hypersetup{pdfcontactpostcode={8093}}
\hypersetup{pdfcontactcity={Zurich}}
\hypersetup{pdfcontactcountry={Switzerland}}
\hypersetup{pdfcontactemail={nbeisert@itp.phys.ethz.ch}}
\hypersetup{pdfcontacturl={http://people.phys.ethz.ch/\xmptilde nbeisert/}}

\newcommand{\secref}[1]{\hyperref[#1]{section \ref*{#1}}}

\parskip1ex
\parindent0pt
\let\olditemize\itemize
\def\itemize{\olditemize\parskip0pt}

\begin{document}

\title{The \textsf{childdoc} Package}
\hypersetup{pdftitle={The childdoc Package}}
\author{Niklas Beisert\\[2ex]
  Institut f\"ur Theoretische Physik\\
  Eidgen\"ossische Technische Hochschule Z\"urich\\
  Wolfgang-Pauli-Strasse 27, 8093 Z\"urich, Switzerland\\[1ex]
  \href{mailto:nbeisert@itp.phys.ethz.ch}
  {\texttt{nbeisert@itp.phys.ethz.ch}}}
\hypersetup{pdfauthor={Niklas Beisert}}
\hypersetup{pdfsubject={Manual for the LaTeX2e Package childdoc}}
\date{30 December 2018, \textsf{v2.0}}
\maketitle

\begin{abstract}\noindent
\textsf{childdoc} is a \LaTeXe{} package
that enables the direct compilation
of document sections included by |\include|
to individual files.
\end{abstract}

\begingroup
\parskip0ex
\tableofcontents
\endgroup

%%%%%%%%%%%%%%%%%%%%%%%%%%%%%%%%%%%%%%%%%%%%%%%%%%%%%%%%%%%%%%%%%%%%%%%%%%%%%%%%
%%%%%%%%%%%%%%%%%%%%%%%%%%%%%%%%%%%%%%%%%%%%%%%%%%%%%%%%%%%%%%%%%%%%%%%%%%%%%%%%
\section{Introduction}

\LaTeX{} provides a mechanism to structure a large document (such as a book)
into a main file and several child files (containing the chapters)
using the |\include| command.
This mechanism is beneficial for documents
which span hundreds of pages in order to
make the source file(s) more manageable.
Moreover, compilation can be restricted to
selected child files by means of the |\includeonly| command.
The latter feature can be used to reduce the compilation time while editing
(this was significantly more useful in the earlier days of \LaTeX{})
or to generate a smaller document which is easier to navigate.
Another application of |\includeonly| is to generate
documents consisting of selected parts of the complete document.

However, there are a few drawbacks of the plain |\include| mechanism:
\begin{itemize}
\item
The child files cannot be compiled on their own,
they can only be compiled via the main file.
A naive editing environment
(such as a text editor with an option
to have the current file processed by \LaTeX)
may require one to switch to the main file before compiling;
attempting to compile the child file produces errors.
\item
The main file must be modified (each time)
to adjust the |\includeonly| command
to the present needs. This easily leaves the main file in a messy state.
\item
The generated document will always carry the filename
of the main document. This is inconvenient if
several child files are to be compiled and
to be kept for distribution.
\end{itemize}

The present package provides a simple interface
to make child files individually compilable by \LaTeX{}.
Compiling a child file then has the same effect as compiling
the main file with an |\includeonly| command
to select the appropriate child.
Moreover the generated document will carry the name of the child
rather than the main file.
This resolves all three above issues.

This feature is meant to make the editing of books,
thesis documents and lecture notes somewhat more convenient.
However, the package can also be used efficiently for
composing a series of documents (such as exercise sheets)
which are typically distributed individually.
It then assists the author in generating the individual documents
(potentially in different versions)
as well as a document containing the collected series.
Another application is in developing style files
or other kinds of included material
where compilation of the style file could redirect
to a sample or test file.

%%%%%%%%%%%%%%%%%%%%%%%%%%%%%%%%%%%%%%%%%%%%%%%%%%%%%%%%%%%%%%%%%%%%%%%%%%%%%%%%
%%%%%%%%%%%%%%%%%%%%%%%%%%%%%%%%%%%%%%%%%%%%%%%%%%%%%%%%%%%%%%%%%%%%%%%%%%%%%%%%
\section{Usage}

First of all, the package \textsf{childdoc} is \emph{not} a standard
\LaTeXe{} |.sty| style file! Therefore it needs to be invoked in
a non-standard way.

%%%%%%%%%%%%%%%%%%%%%%%%%%%%%%%%%%%%%%%%%%%%%%%%%%%%%%%%%%%%%%%%%%%%%%%%%%%%%%%%
\subsection{Included Files}
\label{sec:include}

%%%%%%%%%%%%%%%%%%%%%%%%%%%%%%%%%%%%%%%%
\DescribeMacro{\childdocmain}
To use the package, add the commands
\begin{center}
\begin{tabular}{l}
|\input{childdoc.def}|\\
|\childdocmain{}|\\
\end{tabular}
\end{center}
at the very top of the main \LaTeX{} file,
in particular \emph{before} the |\documentclass| statement!
The argument of |\childdocmain| should be left empty
(but it must be present).

%%%%%%%%%%%%%%%%%%%%%%%%%%%%%%%%%%%%%%%%
\DescribeMacro{\childdocof}
Furthermore, add the commands
\begin{center}
\begin{tabular}{l}
|\input{childdoc.def}|\\
|\childdocof{|\textit{main}|}|\\
\end{tabular}
\end{center}
at the top of every child file \textit{child}
which is included by |\include{|\textit{child}|}|
from within the main file
(or at least for those files to be compiled individually).
The argument \textit{main} must be the filename of the main file.

There are a couple of
considerations in setting up the main and child documents:

%%%%%%%%%%%%%%%%%%%%%%%%%%%%%%%%%%%%%%%%
\paragraph{Restrictions.}

Please note the following restrictions:
\begin{itemize}
\item
|\childdocmain| must be called with one argument \textit{main}
to ensure compatibility with earlier version of the package.
It must either be empty (|\childdocmain{}|)
or precisely match the filename of the main file in which it is specified.
See \secref{sec:detection} for further information.
\item
The filename \textit{main} must be specified without the |.tex| extension.
\item
The filename \textit{main} is case sensitive
(even in case-insensitive file systems)
due to internal string comparison.
\item
The argument \textit{main} should be fully expanded, it cannot be a macro.
\item
Subdirectories and special characters should be avoided in filenames.
\item
The command |\childdocmain{|\textit{main}|}| must be followed by a whitespace.
It should not be followed immediately by another command
or by a comment mark `|%|'.
This is because the \TeX{} parser reads the token immediately following
the argument of |\childdocmain| and puts it
at the beginning of every child section;
however, a white\-space is ignored.
\end{itemize}

%%%%%%%%%%%%%%%%%%%%%%%%%%%%%%%%%%%%%%%%
\paragraph{Content of Main File.}

It is advisable to place all content in the child files included by |\include|.
Any output contained in the main file will appear in all child documents
unless suppressed manually;
it cannot be suppressed automatically by the |\includeonly| directive
and thus should normally be avoided.
A method to include some content in the main file
by means of conditional processing is described in \secref{sec:conditional}.

%%%%%%%%%%%%%%%%%%%%%%%%%%%%%%%%%%%%%%%%
\paragraph{Page Numbering.}

When only a part of the document is compiled,
the appropriate numbering of pages
(as well as other status parameters)
is determined from the |.aux| files.
The latter contain information from previous passes.
However this information needs to propagate through
all intermediate child documents.
Therefore the page numbering in child documents may well
be inconsistent until the complete document is compiled at least once.

A useful (if unconventional) way to always ensure a consistent
page numbering is to restart the numbering in each child document
and denote the pages by `\textit{child}|.|\textit{page}'
where \textit{child} represents the chapter/section number of the child file.
This can be achieved by the command
|\numberwithin{page}{|\textit{child}|}|
of the \textsf{amsmath} package
where \textit{child} can be |chapter| or |section|
depending on the chosen structuring.
Alternatively, one can modify the macro |\thepage| appropriately
and reset the counter |page| at the start of each child file.

%%%%%%%%%%%%%%%%%%%%%%%%%%%%%%%%%%%%%%%%%%%%%%%%%%%%%%%%%%%%%%%%%%%%%%%%%%%%%%%%
\subsection{Conditional Processing}
\label{sec:conditional}

The package provides a mechanism to compile different versions
of a document. To customise the versions further some conditional processing
can come in handy to distinguish which version is being compiled.
The package provides two macros to describe the compilation context:

%%%%%%%%%%%%%%%%%%%%%%%%%%%%%%%%%%%%%%%%
\DescribeMacro{\ifchilddoc}
The conditional |\ifchilddoc| distinguishes between the compilation of
child documents and the main document:
%
\begin{center}
|\ifchilddoc |\textit{child-code}| |[|\||else |\textit{main-code}]| \||fi|
\end{center}

%%%%%%%%%%%%%%%%%%%%%%%%%%%%%%%%%%%%%%%%
\DescribeMacro{\childdocname}
\DescribeMacro{\childdocjob}
The macro |\childdocname| contains the filename (without extension)
of the main or child file being processed.
Note that |\childdocjob| will always contain the name of the main file.

%%%%%%%%%%%%%%%%%%%%%%%%%%%%%%%%%%%%%%%%
\paragraph{Title Page.}

Conditional processing can be used to include a title or banner page
in the main document when proper precautions are taken.
Importantly, the code in the main file should ensure that the page counter
(as well as other status parameters which are stored in the |.aux| files)
takes the same value after the conditional processing.
Otherwise the page numbers may take divergent values
depending on which part is compiled.

For example, a title page could be declared by:
%
\begin{center}
\begin{tabular}{l}
|\ifchilddoc\||else|\\
|\addtocounter{page}{-1}|\\
\textit{code for title page}\\
|\newpage|\\
|\||fi|
\end{tabular}
\end{center}
%
A banner page for the child documents can be generated by:
%
\begin{center}
\begin{tabular}{l}
|\ifchilddoc|\\
|\addtocounter{page}{-1}|\\
\textit{code for banner page}\\
|\newpage|\\
|\||fi|
\end{tabular}
\end{center}
%
Here one could write a message such as:
\begin{center}
|This is the part \childdocname{} of \childdocjob{}.|
\end{center}

%%%%%%%%%%%%%%%%%%%%%%%%%%%%%%%%%%%%%%%%%%%%%%%%%%%%%%%%%%%%%%%%%%%%%%%%%%%%%%%%
\subsection{Flags}
\label{sec:flags}

The package makes it easy to generate different versions
of the main or child documents.
To this end compilation flags can be defined
and assigned different default values.
They will be particularly useful in conjunction
with the forwarding mechanism described in \secref{sec:forward}.

For example, it may be useful to have a flag |\version|
which can be set to |draft| or |final|.
The document source will contain some conditional code
depending on the value of |\version|.
Suppose further, the flag should default to |final| for the main file
and to |draft| for child files
which is a natural assignment for editing the document.
This is achieved by placing the following code
in the preamble of the main document
(below the |\childdocmain| directive):
%
\begin{center}
\begin{tabular}{l}
|\ifchilddoc|\\
|\providecommand{\version}{draft}|\\
|\||else|\\
|\providecommand{\version}{final}|\\
|\||fi|
\end{tabular}
\end{center}
%
The definition by |\providecommand| makes sure
that previous definitions are not overwritten.
Further statements |\providecommand{\version}{...}|
can thus be added before the above code to override it.

For the main file, one might add a line
(between |\childdocmain| and the above block)
%
\begin{center}
|%\ifchilddoc\||else\providecommand{\version}{draft}\||fi|
\end{center}
%
which can be uncommented to produce a draft version.
Likewise one can add a line to the very top of a child file
(above the |\childdocof{|\textit{main}|}| directive)
%
\begin{center}
|%\providecommand{\version}{final}|
\end{center}
%
which can be uncommented to produce the final version of this child document.

%%%%%%%%%%%%%%%%%%%%%%%%%%%%%%%%%%%%%%%%%%%%%%%%%%%%%%%%%%%%%%%%%%%%%%%%%%%%%%%%
\subsection{Forwarding}
\label{sec:forward}

Different versions of the main or child documents
using compilation flags as described in \secref{sec:flags}
can be (permanently) stored in different files
for convenient compilation, viewing and distribution.
To this end, the package defines a command
to pass on compilation to a different file:

%%%%%%%%%%%%%%%%%%%%%%%%%%%%%%%%%%%%%%%%
\DescribeMacro{\childdocforward}
The command |\childdocforward| redirects processing to
another source file:
%
\begin{center}
\begin{tabular}{l}
|\input{childdoc.def}|\\
|\childdocforward[|\textit{main}|]{|\textit{dest}|}|\\
\end{tabular}
\end{center}
%
The argument \textit{dest} is the destination file
(without extension).
It should be the main file or one of the child files.
Note that further \textsf{childdoc} directives
such as |\childdocof| and |\childdocforward|
in the indicated file will be processed in this form.
The optional argument \textit{main}
passes on directly to the main file \textit{main}
while pretending to compile the child \textit{dest}.
This form behaves as if \textit{dest}
issues |\childdocof{|\textit{main}|}| right away,
and no further \textsf{childdoc} directives will be processed.

%%%%%%%%%%%%%%%%%%%%%%%%%%%%%%%%%%%%%%%%
\DescribeMacro{\...prefix}
In the alternative form |\childdocforwardprefix|,
%
\begin{center}
\begin{tabular}{l}
|\input{childdoc.def}|\\
|\childdocforwardprefix[|\textit{main}|]{|\textit{prefix}|}{|\textit{dest}|}|
\end{tabular}
\end{center}
%
the destination file is determined by a pattern
depending on the current file:
To make this work, the current file must be called
`{\textit{prefix}\hspace{0.2em}\textit{suffix}}'
with \textit{prefix} matching precisely the argument.
Processing is then passed on to the file
`{\textit{dest}\hspace{0.2em}\textit{suffix}}'.
Surely, the same effect is achieved by
directly specifying the
argument `{\textit{dest}\hspace{0.2em}\textit{suffix}}'
in the first form.
However, that requires to set up a different file
for each child. With the alternative form of the command
all these files can have exactly the same content
which simplifies setting them up and maintaining them.

For example, the following file |draft.tex|
with a compilation flag |\version| as described in \secref{sec:flags}
compiles the main document as a draft:
%
\begin{center}
\begin{tabular}{l}
|\def\version{draft}|\\
|\input{childdoc.def}|\\
|\childdocforward{|\textit{main}|}|
\end{tabular}
\end{center}
%
Likewise, the following files |final|\textit{nn}|.tex|
compile the final version of the child document
|child|\textit{nn}|.tex|:
%
\begin{center}
\begin{tabular}{l}
|\def\version{final}|\\
|\input{childdoc.def}|\\
|\childdocforwardprefix{final}{child}|
\end{tabular}
\end{center}
%

Note that when several versions of a main file and/or of each child file
are to be generated, it may be convenient to set up a |Makefile| or
shell script to automatise the process.

%%%%%%%%%%%%%%%%%%%%%%%%%%%%%%%%%%%%%%%%%%%%%%%%%%%%%%%%%%%%%%%%%%%%%%%%%%%%%%%%
\subsection{Command Line Processing}
\label{sec:commandline}

The effect of redirection files can also be achieved by invoking
the \LaTeX{} compiler with a more elaborate command line.
Most conveniently this should be done as part
of a shell script or a |Makefile|.

When using \textsf{childdoc} in the main file, the following
command lines effectively perform a redirection
(note that depending on the shell being used,
backslashes may have to be doubled: `|\|' $\to$ `|\\|'):
%
\begin{center}
|... -jobname "|\textit{target}|" |\\|"|[\textit{flags}]%
|\input{childdoc.def}\childdocforward[|\textit{main}|]{|\textit{dest}|}"|
\end{center}
%
Here \textit{target} is the name of the output file,
\textit{main} is the name of the main file
and \textit{dest} is the name of the main or child file to be processed
(all filenames without extensions).
The optional argument \textit{main} can be omitted
if \textit{main} matches \textit{dest}.
Optionally, compilation \textit{flags} can be defined via |\def| commands.
This command line makes the \TeX{} engine believe
it is compiling the file \textit{target}
whose content is specified as the latter parameter.
The provided code then forwards the processing to
\textit{main} or \textit{dest} as described in \secref{sec:forward}.

%%%%%%%%%%%%%%%%%%%%%%%%%%%%%%%%%%%%%%%%%%%%%%%%%%%%%%%%%%%%%%%%%%%%%%%%%%%%%%%%
\subsection{Include by Input}
\label{sec:input}

Including child documents by |\include| has some restrictions by design.
Most notably, the content of a child document always occupies
its own set of pages; pages cannot be shared between child documents.
Usually, this behaviour makes perfect sense
because each child document contain an essential part of the document.
However, in some situations it may be desirable to compose
a document from a collection of parts
without having mandatory page breaks between then.
For this case, the package
provides a mechanism to include parts
by |\input| which can also be processed individually.
However, by construction this mechanism
requires manual handling of the content to be output.

%%%%%%%%%%%%%%%%%%%%%%%%%%%%%%%%%%%%%%%%
\DescribeMacro{\ifchilddocmanual}
The main file should be prepared as usual, see \secref{sec:include}.
However, the document body must make a distinction
between processing of an individual part and of the main document, e.g.:
%
\begin{center}
\begin{tabular}{l}
|\ifchilddocmanual|\\
|\input{\childdocname}|\\
|\||else|\\
\textit{document body with }|\input{|\textit{part}|}|\\
|\||fi|
\end{tabular}
\end{center}
%
The conditional |\ifchilddocmanual| is true whenever
a part to be included by |\input| is being compiled,
and the name of the part is stored in |\childdocname|.

%%%%%%%%%%%%%%%%%%%%%%%%%%%%%%%%%%%%%%%%
\DescribeMacro{\childdocby}
Each part to be included by |\input| should start with:
%
\begin{center}
\begin{tabular}{l}
|\input{childdoc.def}|\\
|\childdocby{|\textit{main}|}|\\
\end{tabular}
\end{center}
%
The directive |\childdocby| is similar to |\childdocof|
described in \secref{sec:include},
but the subsequent selection of content must be done manually.
To that end, both |\ifchilddoc| and |\ifchilddocmanual|
will be true upon processing of a part,
and the name of the part is stored in |\childdocname|.
Note that |\jobname| will be set to the filename of the current part
so that each part receives an individual |.aux| file
that does not interfere with the |.aux| file(s) of the main document.
This behaviour can be altered by the alternative form
|\childdocby[*]{|\textit{main}|}| (with a non-empty optional argument)
which uses the |.aux| file of the main document
by setting |\jobname| to \textit{main}.

%%%%%%%%%%%%%%%%%%%%%%%%%%%%%%%%%%%%%%%%%%%%%%%%%%%%%%%%%%%%%%%%%%%%%%%%%%%%%%%%
\subsection{Driver Development}
\label{sec:driver}

The \textsf{childdoc} mechanism can also be use for the development
of definition files such as \LaTeX{} styles or classes.
This case differs from the above setup with multiple parts
included by |\include| in that no |\includeonly| should be invoked.
This can be achieved by starting the include file
(before |\ProvidesPackage|) with:
%
\begin{center}
\begin{tabular}{l}
|\input{childdoc.def}|\\
|\childdocforward{|\textit{main}|}|\\
\end{tabular}
\end{center}
%
or alternatively with:
%
\begin{center}
\begin{tabular}{l}
|\input{childdoc.def}|\\
|\childdocby{|\textit{main}|}|\\
\end{tabular}
\end{center}
%
Both forms have slightly different effects as described above.
The main file is prepared as usual, see \secref{sec:include}.

%%%%%%%%%%%%%%%%%%%%%%%%%%%%%%%%%%%%%%%%%%%%%%%%%%%%%%%%%%%%%%%%%%%%%%%%%%%%%%%%
\subsection{Legacy Detection}
\label{sec:detection}

The directive |\childdocmain| in the main file can detect
whether the complete document or merely a child is to be compiled
even without using the directive |\childdocof|.
This method is deprecated because it is less robust
and there is no compelling reason to use it;
it is merely provided for backward compatibility
and it may be removed in future versions.

If the detection mechanism is to be used,
it is mandatory to correctly specify
the filename of the main file as the argument of |\childdocmain|:
%
\begin{center}
\begin{tabular}{l}
|\input{childdoc.def}|\\
|\childdocmain{|\textit{main}|}|\\
\end{tabular}
\end{center}
%
If |\jobname| does not match the argument \textit{main} of |\childdocmain|,
it is assumed that |\jobname| points to the child file to be compiled.
When using |\childdocmain| with the main file specified as argument,
it suffices to start a child file
with just |\input{|\textit{main}|}|
without loading of the package and using |\childdocof|.
If instead all processing is done
with the appropriate \textsf{childdoc} directives,
the argument of \textit{main} of |\childdocmain| can be empty.

An alternative version of the command line processing described
in \secref{sec:commandline} using the detection mechanism reads:
%
\begin{center}
|... -jobname "|\textit{target}|" "|[\textit{flags}]%
[|\def\jobname{|\textit{dest}|}|]|\input{|\textit{main}|}"|
\end{center}

%%%%%%%%%%%%%%%%%%%%%%%%%%%%%%%%%%%%%%%%%%%%%%%%%%%%%%%%%%%%%%%%%%%%%%%%%%%%%%%%
\subsection{Manual Code}
\label{sec:manual}

In case one cannot be certain whether the definitions file |childdoc.def|
is installed on the target \TeX{} distribution
and one prefers not to ship it,
it is conceivable to paste a few relevant commands into the sources.

To that end, drop all statements |\input{childdoc.def}|
and perform the replacements as outlined below.
Instead of |\childdocmain{|\textit{main}|}| add the following code
to the top of the main file:
%
\begin{center}
\begin{tabular}{l}
|\||ifdefined\childdocname\endinput\||fi\newif\ifchilddoc|\\
|\edef\childdocname{\scantokens\expandafter{\jobname\noexpand}}|\\
|\def\childdocmain{|\textit{main}|}\||ifx\childdocmain\childdocname\||else|\\
|\childdoctrue\includeonly{\childdocname}\let\jobname\childdocmain\||fi|\\
\end{tabular}
\end{center}
%
Instead of |\childdocof{|\textit{main}|}| just include the main file
at the top of each child file:
%
\begin{center}
|\input{|\textit{main}|}|
\end{center}
%
A simple redirection |\childdocforward{|\textit{dest}|}| is achieved by:
%
\begin{center}
|\def\jobname{|\textit{dest}|}\input{\jobname}|
\end{center}
%
The redirection with prefix
|\childdocforwardprefix[|\textit{prefix}|]{|\textit{dest}|}|
is accomplished by:
%
\begin{center}
\begin{tabular}{l}
|{\edef\jobname{\scantokens\expandafter{\jobname\noexpand}}|\\
|\def\redirectjob |\textit{prefix}|#1~~~{\gdef\jobname{|\textit{dest}|#1}}|\\
|\expandafter\redirectjob\jobname~~~}\input{\jobname}|
\end{tabular}
\end{center}

In an alternative approach,
child documents can be compiled by a specific command line
without additional code or specific definitions:
%
\begin{center}
|... -jobname "|\textit{target}|" "|[\textit{flags}]%
|\includeonly{|\textit{dest}|}\input{|\textit{main}|}"|
\end{center}
%

%%%%%%%%%%%%%%%%%%%%%%%%%%%%%%%%%%%%%%%%%%%%%%%%%%%%%%%%%%%%%%%%%%%%%%%%%%%%%%%%
%%%%%%%%%%%%%%%%%%%%%%%%%%%%%%%%%%%%%%%%%%%%%%%%%%%%%%%%%%%%%%%%%%%%%%%%%%%%%%%%
\section{Information}

%%%%%%%%%%%%%%%%%%%%%%%%%%%%%%%%%%%%%%%%%%%%%%%%%%%%%%%%%%%%%%%%%%%%%%%%%%%%%%%%
\subsection{Copyright}

Copyright \copyright{} 2017--2018 Niklas Beisert

This work may be distributed and/or modified under the
conditions of the \LaTeX{} Project Public License, either version 1.3
of this license or (at your option) any later version.
The latest version of this license is in
  \url{http://www.latex-project.org/lppl.txt}
and version 1.3 or later is part of all distributions of \LaTeX{}
version 2005/12/01 or later.

This work has the LPPL maintenance status `maintained'.

The Current Maintainer of this work is Niklas Beisert.

This work consists of the files |README.txt|, |childdoc.ins| and |childdoc.dtx|
as well as the derived files |childdoc.def|, |cdocsamp.tex|
with |cdocsch1.tex|, |cdocsch2.tex|, |cdocspt3.tex|, |cdocspt4.tex|,
|cdocsdrf.tex|, |cdocsfn1.tex|, |cdocsfn2.tex|
as well as |childdoc.pdf|.

%%%%%%%%%%%%%%%%%%%%%%%%%%%%%%%%%%%%%%%%%%%%%%%%%%%%%%%%%%%%%%%%%%%%%%%%%%%%%%%%
\subsection{Files and Installation}

The package consists of the files:
%
\begin{center}
\begin{tabular}{ll}
    |README.txt|   & readme file \\
    |childdoc.ins| & installation file \\
    |childdoc.dtx| & source file \\
    |childdoc.def| & definition file \\
    |cdocsamp.tex| & sample main file \\
    |cdocsch1.tex| & sample include file \\
    |cdocsch2.tex| & sample include file \\
    |cdocspt3.tex| & sample part file \\
    |cdocspt4.tex| & sample part file \\
    |cdocsdrf.tex| & sample redirection file \\
    |cdocsfn1.tex| & sample redirection file \\
    |cdocsfn2.tex| & sample redirection file \\
    |childdoc.pdf| & manual
\end{tabular}
\end{center}
%
The distribution consists of the files
|README.txt|, |childdoc.ins| and |childdoc.dtx|.
%
\begin{itemize}
\item
Run (pdf)\LaTeX{} on |childdoc.dtx|
to compile the manual |childdoc.pdf| (this file).
\item
Run \LaTeX{} on |childdoc.ins| to create the definitions file |childdoc.def|
and the sample |cdocsamp.tex| with include files
|cdocsch1.tex|, |cdocsch2.tex|, |cdocspt3.tex|, |cdocspt4.tex|,
|cdocsdrf.tex|, |cdocsfn1.tex|, |cdocsfn2.tex|.
Then copy the file |childdoc.def| to an appropriate directory of your \LaTeX{}
distribution, e.g.\ \textit{texmf-root}|/tex/latex/childdoc|.
\end{itemize}

%%%%%%%%%%%%%%%%%%%%%%%%%%%%%%%%%%%%%%%%%%%%%%%%%%%%%%%%%%%%%%%%%%%%%%%%%%%%%%%%
\subsection{Related CTAN Packages}

There are several other packages which offer a similar functionality:
%
\begin{itemize}
\item
The packages
\href{http://ctan.org/pkg/docmute}{\textsf{docmute}},
\href{http://ctan.org/pkg/includex}{\textsf{includex}} and
\href{http://ctan.org/pkg/standalone}{\textsf{standalone}}
provide commands to include only the document body of
a child file thus allowing both files to be compiled individually.
\item
The packages \href{http://ctan.org/pkg/subdocs}{\textsf{subdocs}}
and \href{http://ctan.org/pkg/subfiles}{\textsf{subfiles}}
provide structures in which the main and child documents can be
encapsulated and allowing them to be compiled individually.
The inclusion mechanism is different from the conventional |\include|.
\item
The package \href{http://ctan.org/pkg/combine}{\textsf{combine}}
is an elaborate solution to combine several documents into one.
\end{itemize}
%
See also the CTAN topic \href{http://ctan.org/topic/subdocs}{\textsf{subdocs}}
for further related packages.
The present package differs from the above solutions in that
a document structure constructed with the conventional |\include| mechanism
just needs two extra commands at the top of every file
such that all constituent files can be compiled individually.

%%%%%%%%%%%%%%%%%%%%%%%%%%%%%%%%%%%%%%%%%%%%%%%%%%%%%%%%%%%%%%%%%%%%%%%%%%%%%%%%
%\subsection{Feature Suggestions}
%
%The following is a list of features which may be useful for future
%versions of this package:
%%
%\begin{itemize}
%\item
%\ldots
%\end{itemize}

%%%%%%%%%%%%%%%%%%%%%%%%%%%%%%%%%%%%%%%%%%%%%%%%%%%%%%%%%%%%%%%%%%%%%%%%%%%%%%%%
\subsection{Revision History}

%%%%%%%%%%%%%%%%%%%%%%%%%%%%%%%%%%%%%%%%
\paragraph{v2.0:} 2018/12/30

\begin{itemize}
\item
immediate forward processing
\item
added |\childdocby| mechanism
\item
manual restructured
\end{itemize}

%%%%%%%%%%%%%%%%%%%%%%%%%%%%%%%%%%%%%%%%
\paragraph{v1.6:} 2018/01/17

\begin{itemize}
\item
application for development of include files
\item
corrections to manual
\end{itemize}

%%%%%%%%%%%%%%%%%%%%%%%%%%%%%%%%%%%%%%%%
\paragraph{v1.5:} 2017/05/21

\begin{itemize}
\item
more complete structuring introduced
\item
|\childdocof| introduced
\item
|\childdoc| renamed to |\childdocmain|
\item
|\childredirect| renamed to |\childdocforward| and |\childdocforwardprefix|
and functionality expanded
\end{itemize}

%%%%%%%%%%%%%%%%%%%%%%%%%%%%%%%%%%%%%%%%
\paragraph{v1.0:} 2017/04/27

\begin{itemize}
\item
manual and install package
\item
first version published on CTAN
\end{itemize}

%%%%%%%%%%%%%%%%%%%%%%%%%%%%%%%%%%%%%%%%
\paragraph{v0.6:} 2017/04/26

\begin{itemize}
\item
redirection mechanism added
\end{itemize}

%%%%%%%%%%%%%%%%%%%%%%%%%%%%%%%%%%%%%%%%
\paragraph{v0.5:} 2017/04/26

\begin{itemize}
\item
functionality in definition file
\end{itemize}


%%%%%%%%%%%%%%%%%%%%%%%%%%%%%%%%%%%%%%%%%%%%%%%%%%%%%%%%%%%%%%%%%%%%%%%%%%%%%%%%
%%%%%%%%%%%%%%%%%%%%%%%%%%%%%%%%%%%%%%%%%%%%%%%%%%%%%%%%%%%%%%%%%%%%%%%%%%%%%%%%
%%%%%%%%%%%%%%%%%%%%%%%%%%%%%%%%%%%%%%%%%%%%%%%%%%%%%%%%%%%%%%%%%%%%%%%%%%%%%%%%
\appendix

\settowidth\MacroIndent{\rmfamily\scriptsize 000\ }

 \DocInput{childdoc.dtx}

\end{document}
%</driver>
% \fi
%
% %%%%%%%%%%%%%%%%%%%%%%%%%%%%%%%%%%%%%%%%%%%%%%%%%%%%%%%%%%%%%%%%%%%%%%%%%%%%%%
% %%%%%%%%%%%%%%%%%%%%%%%%%%%%%%%%%%%%%%%%%%%%%%%%%%%%%%%%%%%%%%%%%%%%%%%%%%%%%%
% \section{Sample}
%\iffalse
%<*samplemain>
%\fi
%
% The following presents a sample document
% with two chapters, two parts, a title page,
% a compile flag as well as three forwarding files to set the flag.
% It consists of eight |.tex| files:
% \begin{center}
% \begin{tabular}{ll}
% |cdocsamp.tex|&main file\\
% |cdocsch1.tex|&include file for chapter 1\\
% |cdocsch2.tex|&include file for chapter 2\\
% |cdocspt3.tex|&include file for part 3\\
% |cdocspt4.tex|&include file for part 4\\
% |cdocsdrf.tex|&forwarding file for main file in draft mode\\
% |cdocsfi1.tex|&forwarding file for final version of chapter 1\\
% |cdocsfi2.tex|&forwarding file for final version of chapter 2\\
% \end{tabular}
% \end{center}
% Each of the eight files can be compiled directly by the \LaTeX{} compiler.
%
% %%%%%%%%%%%%%%%%%%%%%%%%%%%%%%%%%%%%%%
% \paragraph{Main File.}
%
% The main file is called |cdocsamp.tex|.
%
% Load the \textsf{childdoc} definitions and
% declare the filename for the main document:
%    \begin{macrocode}
\input{childdoc.def}
\childdocmain{}
%    \end{macrocode}

% Optional override for |\version| flag:
%    \begin{macrocode}
%%\ifchilddoc\else\providecommand{\version}{draft}\fi
%    \end{macrocode}

% Define the default values for the |\version| flag
% (|final| for the main file and |draft| for childs):
%    \begin{macrocode}
\ifchilddoc
\providecommand{\version}{draft}
\else
\providecommand{\version}{final}
\fi
%    \end{macrocode}

% Load the standard document class:
%    \begin{macrocode}
\documentclass[12pt]{article}
%    \end{macrocode}

% Start the document body:
%    \begin{macrocode}
\begin{document}
%    \end{macrocode}

% Declare a title page.
% Print title, part of document being processed and version flag:
%    \begin{macrocode}
\addtocounter{page}{-1}
\begin{center}
{\LARGE\bfseries{}childdoc example\par}
\vspace{1cm}
\ifchilddoc
\ifchilddocmanual part\else chapter\fi:
`\childdocname' of `\childdocjob'\par
\else
main document: `\childdocjob'\par
\fi
version: \version\par
\end{center}
\newpage
%    \end{macrocode}

% Manually include selected file,
% otherwise process as usual:
%    \begin{macrocode}
\ifchilddocmanual
\section*{part `\childdocname'}
\input{\childdocname}
\else
%    \end{macrocode}

% Include the two chapters:
%    \begin{macrocode}
\include{cdocsch1}
\include{cdocsch2}
%    \end{macrocode}

% Include the two parts unless only chapters should be displayed:
%    \begin{macrocode}
\ifchilddoc\else
\section{part three}
\input{cdocspt3}
\section{part four}
\input{cdocspt4}
\fi
%    \end{macrocode}

% Process as usual until here:
%    \begin{macrocode}
\fi
%    \end{macrocode}

% End of document body:
%    \begin{macrocode}
\end{document}
%    \end{macrocode}
%\iffalse
%</samplemain>
%\fi
%
% %%%%%%%%%%%%%%%%%%%%%%%%%%%%%%%%%%%%%%
% \paragraph{Chapter Include Files.}
%
% The include files are called |cdocsch1.tex| and |cdocsch2.tex|.
%
%\iffalse
%<*samplechap1|samplechap2>
%\fi

% Optional override for |\version| flag:
%    \begin{macrocode}
%%\providecommand{\version}{final}
%    \end{macrocode}

% Include the main document:
%    \begin{macrocode}
\input{childdoc.def}
\childdocof{cdocsamp}
%    \end{macrocode}

%\iffalse
%</samplechap1|samplechap2>
%\fi
%
%\iffalse
%<*samplechap1>
%\fi
% Some text for chapter 1:
%    \begin{macrocode}
\section{one}
some text in chapter one
%    \end{macrocode}

%\iffalse
%</samplechap1>
%\fi
% Some text for chapter 2:
%\iffalse
%<*samplechap2>
%\fi
%    \begin{macrocode}
\section{two}
more text in chapter two
%    \end{macrocode}

%\iffalse
%</samplechap2>
%\fi
%
% %%%%%%%%%%%%%%%%%%%%%%%%%%%%%%%%%%%%%%
% \paragraph{Part Include Files.}
%
% The include files are called |cdocspt3.tex| and |cdocspt4.tex|.
%
%\iffalse
%<*samplepart3|samplepart4>
%\fi

% Optional override for |\version| flag:
%    \begin{macrocode}
%%\providecommand{\version}{final}
%    \end{macrocode}

% Include the main document:
%    \begin{macrocode}
\input{childdoc.def}
\childdocby{cdocsamp}
%    \end{macrocode}

%\iffalse
%</samplepart3|samplepart4>
%\fi
%
%\iffalse
%<*samplepart3>
%\fi
% Some text for part 3:
%    \begin{macrocode}
some text in part three
%    \end{macrocode}

%\iffalse
%</samplepart3>
%\fi
% Some text for part 4:
%\iffalse
%<*samplepart4>
%\fi
%    \begin{macrocode}
more text in part four
%    \end{macrocode}

%\iffalse
%</samplepart4>
%\fi
%
% %%%%%%%%%%%%%%%%%%%%%%%%%%%%%%%%%%%%%%
% \paragraph{Forwarding for a Complete Draft.}
%
% The following forwarding file |cdocsdrf.tex|
% compiles the main document in draft mode:
%\iffalse
%<*sampledraft>
%\fi
%    \begin{macrocode}
\def\version{draft}
\input{childdoc.def}
\childdocforward{cdocsamp}
%    \end{macrocode}

%\iffalse
%</sampledraft>
%\fi
%
% %%%%%%%%%%%%%%%%%%%%%%%%%%%%%%%%%%%%%%
% \paragraph{Forwarding for Final Version of the Chapters.}
%
% The following forwarding files |cdocsfn1.tex| and |cdocsfn2.tex|
% (with identical content)
% compile the final versions of the child documents
% |cdocsch1.tex| and |cdocsch2.tex|, respectively:
%\iffalse
%<*samplefinal>
%\fi
%    \begin{macrocode}
\def\version{final}
\input{childdoc.def}
\childdocforwardprefix[cdocsamp]{cdocsfn}{cdocsch}
%    \end{macrocode}

%\iffalse
%</samplefinal>
%\fi
%
% %%%%%%%%%%%%%%%%%%%%%%%%%%%%%%%%%%%%%%
% \paragraph{Command Line Processing.}
%
% The following three command lines generate the output files
% |cdocscld|, |cdocscl1| and |cdocscl2|
% which should be identical to
% |cdocsdrf|, |cdocsch1| and |cdocsfn2|, respectively:
% \begin{center}
% \begin{tabular}{l}
% |latex -jobname cdocscld \|\\
% |  "\def\version{draft}\input{childdoc.def}\childdocforward{cdocsamp}"|\\
% |latex -jobname cdocscl1 \|\\
% |  "\input{childdoc.def}\childdocforward[cdocsamp]{cdocsch1}"|\\
% |latex -jobname cdocscl2 \|\\
% |  "\def\version{final}\input{childdoc.def}\childdocforward{cdocsch2}"|
% \end{tabular}
% \end{center}
% Note that the trailing backslash on each first line
% merely continues the input to the second line
% (for convenient cut ant paste).
% Furthermore, the command |latex| can be replaced by any
% of its alternative versions such as |pdflatex|.
%
% %%%%%%%%%%%%%%%%%%%%%%%%%%%%%%%%%%%%%%%%%%%%%%%%%%%%%%%%%%%%%%%%%%%%%%%%%%%%%%
% %%%%%%%%%%%%%%%%%%%%%%%%%%%%%%%%%%%%%%%%%%%%%%%%%%%%%%%%%%%%%%%%%%%%%%%%%%%%%%
% \section{Implementation}
%\iffalse
%<*package>
%\fi
%
% This section describes the definitions file |childdoc.def|.

% The definitions cannot be loaded using |\usepackage| or |\RequirePackage|
% which has a mechanism to prevent loading a style file more than once.
% When loading the definitions by means of |\input|
% multiple instances have to be prevented manually:
%\iffalse
%This code needs to be before the `\ProvidesFile' directive
%which is defined at the beginning of this file.
%Therefore it is also placed there and commented out here.
%</package>
%<*discard>
%\fi
%    \begin{macrocode}
\ifdefined\childdocmain\endinput\fi
%    \end{macrocode}
%\iffalse
%</discard>
%<*package>
%\fi
%
% \macro{\ifchilddoc}
% \macro{\ifchilddocmanual}
% The conditional |\ifchilddoc| tells whether a
% child (true) or main (false) document is being compiled.
% The conditional |\ifchilddocmanual| tells whether
% the |\includeonly| mechanism is used (false) or
% the selection of child files must be performed manually (true).
% The definitions initialise to false:
%    \begin{macrocode}
\newif\ifchilddoc
\newif\ifchilddocmanual
%    \end{macrocode}

% \macro{\childdocname}
% \macro{\childdocjob}
% The macro |\childdocname| stores the name of the main document
% to be compiled. The macro |\childdocjob| stores the name of
% the document on which the \LaTeX{} compiler was originally invoked.
% The content of |\jobname| cannot be compared
% to filenames specified in the source due to different catcodes.
% The following code rescans |\jobname|, stores the result
% in |\childdocname| and saves a copy in |\childdocjob|:
%    \begin{macrocode}
\edef\childdocname{\scantokens\expandafter{\jobname\noexpand}}
\let\childdocjob\childdocname
%    \end{macrocode}

% \macro{\childdocdisable}
% The macro |\childdocdisable| prevents the main file
% from being processed more than once.
% At this stage, the main document command |\childdocmain|
% is assumed to be called once again where it should do nothing.
% Any subsequent call to it should prevent
% a secondary processing of the main document
% It overwrites the forwarding commands
% |\childdocof| and |\childdocforward|
% with empty macros to prevent further inclusions of the main document:
%    \begin{macrocode}
\newcommand{\childdocdisable}
{
  \renewcommand{\childdocmain}[1]{\renewcommand{\childdocmain}[1]{\endinput}}
  \renewcommand{\childdocof}[1]{}
  \renewcommand{\childdocby}[2][]{}
  \renewcommand{\childdocforward}[2][]{}
  \renewcommand{\childdocdisable}{}
}
%    \end{macrocode}

% \macro{\childdocmain}
% The macro |\childdocmain| is to be called at the top of the main file
% with nothing or the main filename (without extension) as argument.
% First, it breaks loops.
% If the argument is not empty and does not match |\childdocname|
% (which is set by the first inclusion of |childdoc.def|),
% |\ifchilddoc| is set to true, |\includeonly| is applied to the child file
% and |\jobname| is set to the main file
% (for proper handling of |.aux| files):
%    \begin{macrocode}
\newcommand{\childdocmain}[1]
{
  \childdocdisable\childdocmain{}
  \if?#1?\else
    \begingroup
      \def\childdoctmp{#1}
      \ifx\childdoctmp\childdocname
        \def\childdoctmp{}
      \else
        \def\childdoctmp
        {
          \childdoctrue
          \includeonly{\childdocname}
          \def\childdocjob{#1}
          \def\jobname{#1}
        }
      \fi
      \expandafter
    \endgroup
    \childdoctmp
  \fi
}
%    \end{macrocode}

% \macro{\childdocof}
% The command |\childdocof| redirects
% compilation to the main file |#1|.
%    \begin{macrocode}
\newcommand{\childdocof}[1]
{
  \childdocdisable
  \childdoctrue
  \includeonly{\childdocname}
  \def\jobname{#1}
  \def\childdocjob{#1}
  \input{#1}
}
%    \end{macrocode}

% \macro{\childdocby}
% The command |\childdocby| ....
%    \begin{macrocode}
\newcommand{\childdocby}[2][]
{
  \childdocdisable
  \childdoctrue
  \childdocmanualtrue
  \if?#1?\else
    \def\jobname{#2}
  \fi
  \def\childdocjob{#2}
  \input{#2}
  \endinput
}
%    \end{macrocode}

% \macro{\childdocforward}
% The command |\childdocforward| redirects
% compilation to the main file or
% (if the optional argument is given) a child file.
% Parameters are set as if the main file
% or a child file starting with |\childdocof| was compiled.
% Then compilation is handed over to the main file:
%    \begin{macrocode}
\newcommand{\childdocforward}[2][]
{
  \begingroup
    \if?#1?
      \def\childdoctmp
      {
        \def\childdocname{#2}
        \def\childdocjob{#2}
        \def\jobname{#2}
        \input{#2}
        \endinput
      }
    \else
      \def\childdoctmp
      {
        \childdocdisable
        \def\childdocname{#2}
        \childdoctrue
        \includeonly{#2}
        \def\childdocjob{#1}
        \def\jobname{#1}
        \input{#1}
        \endinput
      }
    \fi
    \expandafter
  \endgroup
  \childdoctmp
}
%    \end{macrocode}

% \macro{\childdocforwardprefix}
% The command |\childdocforwardprefix| redirects
% compilation to the main or a child file by means of a pattern.
% The prefix |#1| in the current filename is replaced by |#2|
% and the suffix of the current filename is kept
% (it is assumed that the filename does not contain the substring `|~~~|'
% which is used as a delimiter).
% Compilation is handed over to the new file by |\childdocforward|:
%    \begin{macrocode}
\newcommand{\childdocforwardprefix}[3][]
{
  \begingroup
    \def\childdocextract #2##1~~~{\def\childdoctmp{\childdocforward[#1]{#3##1}}}
    \expandafter\childdocextract\childdocname~~~
    \expandafter
  \endgroup
  \childdoctmp
}
%    \end{macrocode}

% \macro{\childdoc}
% The deprecated macro |\childdoc| is a legacy version of |\childdocmain|:
%    \begin{macrocode}
\newcommand{\childdoc}{\childdocmain}
%    \end{macrocode}

% \macro{\childdocredirect}
% The deprecated macro |\childdocredirect| is a legacy version
% of |\childdocforward| and |\childdocforwardprefix|:
%    \begin{macrocode}
\newcommand{\childdocredirect}[2][]
{
  \begingroup
    \if?#1?
      \def\childdoctmp{\childdocforward{#2}}
    \else
      \def\childdoctmp{\childdocforwardprefix{#1}{#2}}
    \fi
    \expandafter
  \endgroup
  \childdoctmp
}
%    \end{macrocode}

%\iffalse
%</package>
%\fi
%
\endinput
|\\
|\childdocforward[|\textit{main}|]{|\textit{dest}|}|\\
\end{tabular}
\end{center}
%
The argument \textit{dest} is the destination file
(without extension).
It should be the main file or one of the child files.
Note that further \textsf{childdoc} directives
such as |\childdocof| and |\childdocforward|
in the indicated file will be processed in this form.
The optional argument \textit{main}
passes on directly to the main file \textit{main}
while pretending to compile the child \textit{dest}.
This form behaves as if \textit{dest}
issues |\childdocof{|\textit{main}|}| right away,
and no further \textsf{childdoc} directives will be processed.

%%%%%%%%%%%%%%%%%%%%%%%%%%%%%%%%%%%%%%%%
\DescribeMacro{\...prefix}
In the alternative form |\childdocforwardprefix|,
%
\begin{center}
\begin{tabular}{l}
|% \iffalse
%
% childdoc.dtx Copyright (C) 2017-2018 Niklas Beisert
%
% This work may be distributed and/or modified under the
% conditions of the LaTeX Project Public License, either version 1.3
% of this license or (at your option) any later version.
% The latest version of this license is in
%   http://www.latex-project.org/lppl.txt
% and version 1.3 or later is part of all distributions of LaTeX
% version 2005/12/01 or later.
%
% This work has the LPPL maintenance status `maintained'.
%
% The Current Maintainer of this work is Niklas Beisert.
%
% This work consists of the files childdoc.dtx and childdoc.ins
% and the derived files childdoc.def and cdocsamp.tex with
% cdocsch1.tex, cdocsch2.tex, cdocsdrf.tex, cdocsfn1.tex, cdocsfn2.tex.
%
%<package>\ifdefined\childdocmain\endinput\fi
%<package>\ProvidesFile{childdoc.def}[2018/12/30 v2.0 child document driver]
%<samplemain>\ProvidesFile{cdocsamp.tex}[2018/12/30 v2.0 sample for childdoc]
%<*driver>
%\ProvidesFile{childdoc.drv}[2018/12/30 v2.0 childdoc reference manual file]
\PassOptionsToClass{10pt,a4paper}{article}
\documentclass{ltxdoc}

\usepackage[margin=35mm]{geometry}
\usepackage{hyperref}
\usepackage{hyperxmp}
\usepackage[usenames]{color}

\hypersetup{colorlinks=true}
\hypersetup{pdfstartview=FitH}
\hypersetup{pdfpagemode=UseNone}
\hypersetup{pdfsource={}}
\hypersetup{pdflang={en-UK}}
\hypersetup{pdfcopyright={Copyright 2017-2018 Niklas Beisert.
  This work may be distributed and/or modified under the
  conditions of the LaTeX Project Public License, either version 1.3
  of this license or (at your option) any later version.}}
\hypersetup{pdflicenseurl={http://www.latex-project.org/lppl.txt}}
\hypersetup{pdfcontactaddress={ETH Zurich, ITP, HIT K,
  Wolfgang-Pauli-Strasse 27}}
\hypersetup{pdfcontactpostcode={8093}}
\hypersetup{pdfcontactcity={Zurich}}
\hypersetup{pdfcontactcountry={Switzerland}}
\hypersetup{pdfcontactemail={nbeisert@itp.phys.ethz.ch}}
\hypersetup{pdfcontacturl={http://people.phys.ethz.ch/\xmptilde nbeisert/}}

\newcommand{\secref}[1]{\hyperref[#1]{section \ref*{#1}}}

\parskip1ex
\parindent0pt
\let\olditemize\itemize
\def\itemize{\olditemize\parskip0pt}

\begin{document}

\title{The \textsf{childdoc} Package}
\hypersetup{pdftitle={The childdoc Package}}
\author{Niklas Beisert\\[2ex]
  Institut f\"ur Theoretische Physik\\
  Eidgen\"ossische Technische Hochschule Z\"urich\\
  Wolfgang-Pauli-Strasse 27, 8093 Z\"urich, Switzerland\\[1ex]
  \href{mailto:nbeisert@itp.phys.ethz.ch}
  {\texttt{nbeisert@itp.phys.ethz.ch}}}
\hypersetup{pdfauthor={Niklas Beisert}}
\hypersetup{pdfsubject={Manual for the LaTeX2e Package childdoc}}
\date{30 December 2018, \textsf{v2.0}}
\maketitle

\begin{abstract}\noindent
\textsf{childdoc} is a \LaTeXe{} package
that enables the direct compilation
of document sections included by |\include|
to individual files.
\end{abstract}

\begingroup
\parskip0ex
\tableofcontents
\endgroup

%%%%%%%%%%%%%%%%%%%%%%%%%%%%%%%%%%%%%%%%%%%%%%%%%%%%%%%%%%%%%%%%%%%%%%%%%%%%%%%%
%%%%%%%%%%%%%%%%%%%%%%%%%%%%%%%%%%%%%%%%%%%%%%%%%%%%%%%%%%%%%%%%%%%%%%%%%%%%%%%%
\section{Introduction}

\LaTeX{} provides a mechanism to structure a large document (such as a book)
into a main file and several child files (containing the chapters)
using the |\include| command.
This mechanism is beneficial for documents
which span hundreds of pages in order to
make the source file(s) more manageable.
Moreover, compilation can be restricted to
selected child files by means of the |\includeonly| command.
The latter feature can be used to reduce the compilation time while editing
(this was significantly more useful in the earlier days of \LaTeX{})
or to generate a smaller document which is easier to navigate.
Another application of |\includeonly| is to generate
documents consisting of selected parts of the complete document.

However, there are a few drawbacks of the plain |\include| mechanism:
\begin{itemize}
\item
The child files cannot be compiled on their own,
they can only be compiled via the main file.
A naive editing environment
(such as a text editor with an option
to have the current file processed by \LaTeX)
may require one to switch to the main file before compiling;
attempting to compile the child file produces errors.
\item
The main file must be modified (each time)
to adjust the |\includeonly| command
to the present needs. This easily leaves the main file in a messy state.
\item
The generated document will always carry the filename
of the main document. This is inconvenient if
several child files are to be compiled and
to be kept for distribution.
\end{itemize}

The present package provides a simple interface
to make child files individually compilable by \LaTeX{}.
Compiling a child file then has the same effect as compiling
the main file with an |\includeonly| command
to select the appropriate child.
Moreover the generated document will carry the name of the child
rather than the main file.
This resolves all three above issues.

This feature is meant to make the editing of books,
thesis documents and lecture notes somewhat more convenient.
However, the package can also be used efficiently for
composing a series of documents (such as exercise sheets)
which are typically distributed individually.
It then assists the author in generating the individual documents
(potentially in different versions)
as well as a document containing the collected series.
Another application is in developing style files
or other kinds of included material
where compilation of the style file could redirect
to a sample or test file.

%%%%%%%%%%%%%%%%%%%%%%%%%%%%%%%%%%%%%%%%%%%%%%%%%%%%%%%%%%%%%%%%%%%%%%%%%%%%%%%%
%%%%%%%%%%%%%%%%%%%%%%%%%%%%%%%%%%%%%%%%%%%%%%%%%%%%%%%%%%%%%%%%%%%%%%%%%%%%%%%%
\section{Usage}

First of all, the package \textsf{childdoc} is \emph{not} a standard
\LaTeXe{} |.sty| style file! Therefore it needs to be invoked in
a non-standard way.

%%%%%%%%%%%%%%%%%%%%%%%%%%%%%%%%%%%%%%%%%%%%%%%%%%%%%%%%%%%%%%%%%%%%%%%%%%%%%%%%
\subsection{Included Files}
\label{sec:include}

%%%%%%%%%%%%%%%%%%%%%%%%%%%%%%%%%%%%%%%%
\DescribeMacro{\childdocmain}
To use the package, add the commands
\begin{center}
\begin{tabular}{l}
|\input{childdoc.def}|\\
|\childdocmain{}|\\
\end{tabular}
\end{center}
at the very top of the main \LaTeX{} file,
in particular \emph{before} the |\documentclass| statement!
The argument of |\childdocmain| should be left empty
(but it must be present).

%%%%%%%%%%%%%%%%%%%%%%%%%%%%%%%%%%%%%%%%
\DescribeMacro{\childdocof}
Furthermore, add the commands
\begin{center}
\begin{tabular}{l}
|\input{childdoc.def}|\\
|\childdocof{|\textit{main}|}|\\
\end{tabular}
\end{center}
at the top of every child file \textit{child}
which is included by |\include{|\textit{child}|}|
from within the main file
(or at least for those files to be compiled individually).
The argument \textit{main} must be the filename of the main file.

There are a couple of
considerations in setting up the main and child documents:

%%%%%%%%%%%%%%%%%%%%%%%%%%%%%%%%%%%%%%%%
\paragraph{Restrictions.}

Please note the following restrictions:
\begin{itemize}
\item
|\childdocmain| must be called with one argument \textit{main}
to ensure compatibility with earlier version of the package.
It must either be empty (|\childdocmain{}|)
or precisely match the filename of the main file in which it is specified.
See \secref{sec:detection} for further information.
\item
The filename \textit{main} must be specified without the |.tex| extension.
\item
The filename \textit{main} is case sensitive
(even in case-insensitive file systems)
due to internal string comparison.
\item
The argument \textit{main} should be fully expanded, it cannot be a macro.
\item
Subdirectories and special characters should be avoided in filenames.
\item
The command |\childdocmain{|\textit{main}|}| must be followed by a whitespace.
It should not be followed immediately by another command
or by a comment mark `|%|'.
This is because the \TeX{} parser reads the token immediately following
the argument of |\childdocmain| and puts it
at the beginning of every child section;
however, a white\-space is ignored.
\end{itemize}

%%%%%%%%%%%%%%%%%%%%%%%%%%%%%%%%%%%%%%%%
\paragraph{Content of Main File.}

It is advisable to place all content in the child files included by |\include|.
Any output contained in the main file will appear in all child documents
unless suppressed manually;
it cannot be suppressed automatically by the |\includeonly| directive
and thus should normally be avoided.
A method to include some content in the main file
by means of conditional processing is described in \secref{sec:conditional}.

%%%%%%%%%%%%%%%%%%%%%%%%%%%%%%%%%%%%%%%%
\paragraph{Page Numbering.}

When only a part of the document is compiled,
the appropriate numbering of pages
(as well as other status parameters)
is determined from the |.aux| files.
The latter contain information from previous passes.
However this information needs to propagate through
all intermediate child documents.
Therefore the page numbering in child documents may well
be inconsistent until the complete document is compiled at least once.

A useful (if unconventional) way to always ensure a consistent
page numbering is to restart the numbering in each child document
and denote the pages by `\textit{child}|.|\textit{page}'
where \textit{child} represents the chapter/section number of the child file.
This can be achieved by the command
|\numberwithin{page}{|\textit{child}|}|
of the \textsf{amsmath} package
where \textit{child} can be |chapter| or |section|
depending on the chosen structuring.
Alternatively, one can modify the macro |\thepage| appropriately
and reset the counter |page| at the start of each child file.

%%%%%%%%%%%%%%%%%%%%%%%%%%%%%%%%%%%%%%%%%%%%%%%%%%%%%%%%%%%%%%%%%%%%%%%%%%%%%%%%
\subsection{Conditional Processing}
\label{sec:conditional}

The package provides a mechanism to compile different versions
of a document. To customise the versions further some conditional processing
can come in handy to distinguish which version is being compiled.
The package provides two macros to describe the compilation context:

%%%%%%%%%%%%%%%%%%%%%%%%%%%%%%%%%%%%%%%%
\DescribeMacro{\ifchilddoc}
The conditional |\ifchilddoc| distinguishes between the compilation of
child documents and the main document:
%
\begin{center}
|\ifchilddoc |\textit{child-code}| |[|\||else |\textit{main-code}]| \||fi|
\end{center}

%%%%%%%%%%%%%%%%%%%%%%%%%%%%%%%%%%%%%%%%
\DescribeMacro{\childdocname}
\DescribeMacro{\childdocjob}
The macro |\childdocname| contains the filename (without extension)
of the main or child file being processed.
Note that |\childdocjob| will always contain the name of the main file.

%%%%%%%%%%%%%%%%%%%%%%%%%%%%%%%%%%%%%%%%
\paragraph{Title Page.}

Conditional processing can be used to include a title or banner page
in the main document when proper precautions are taken.
Importantly, the code in the main file should ensure that the page counter
(as well as other status parameters which are stored in the |.aux| files)
takes the same value after the conditional processing.
Otherwise the page numbers may take divergent values
depending on which part is compiled.

For example, a title page could be declared by:
%
\begin{center}
\begin{tabular}{l}
|\ifchilddoc\||else|\\
|\addtocounter{page}{-1}|\\
\textit{code for title page}\\
|\newpage|\\
|\||fi|
\end{tabular}
\end{center}
%
A banner page for the child documents can be generated by:
%
\begin{center}
\begin{tabular}{l}
|\ifchilddoc|\\
|\addtocounter{page}{-1}|\\
\textit{code for banner page}\\
|\newpage|\\
|\||fi|
\end{tabular}
\end{center}
%
Here one could write a message such as:
\begin{center}
|This is the part \childdocname{} of \childdocjob{}.|
\end{center}

%%%%%%%%%%%%%%%%%%%%%%%%%%%%%%%%%%%%%%%%%%%%%%%%%%%%%%%%%%%%%%%%%%%%%%%%%%%%%%%%
\subsection{Flags}
\label{sec:flags}

The package makes it easy to generate different versions
of the main or child documents.
To this end compilation flags can be defined
and assigned different default values.
They will be particularly useful in conjunction
with the forwarding mechanism described in \secref{sec:forward}.

For example, it may be useful to have a flag |\version|
which can be set to |draft| or |final|.
The document source will contain some conditional code
depending on the value of |\version|.
Suppose further, the flag should default to |final| for the main file
and to |draft| for child files
which is a natural assignment for editing the document.
This is achieved by placing the following code
in the preamble of the main document
(below the |\childdocmain| directive):
%
\begin{center}
\begin{tabular}{l}
|\ifchilddoc|\\
|\providecommand{\version}{draft}|\\
|\||else|\\
|\providecommand{\version}{final}|\\
|\||fi|
\end{tabular}
\end{center}
%
The definition by |\providecommand| makes sure
that previous definitions are not overwritten.
Further statements |\providecommand{\version}{...}|
can thus be added before the above code to override it.

For the main file, one might add a line
(between |\childdocmain| and the above block)
%
\begin{center}
|%\ifchilddoc\||else\providecommand{\version}{draft}\||fi|
\end{center}
%
which can be uncommented to produce a draft version.
Likewise one can add a line to the very top of a child file
(above the |\childdocof{|\textit{main}|}| directive)
%
\begin{center}
|%\providecommand{\version}{final}|
\end{center}
%
which can be uncommented to produce the final version of this child document.

%%%%%%%%%%%%%%%%%%%%%%%%%%%%%%%%%%%%%%%%%%%%%%%%%%%%%%%%%%%%%%%%%%%%%%%%%%%%%%%%
\subsection{Forwarding}
\label{sec:forward}

Different versions of the main or child documents
using compilation flags as described in \secref{sec:flags}
can be (permanently) stored in different files
for convenient compilation, viewing and distribution.
To this end, the package defines a command
to pass on compilation to a different file:

%%%%%%%%%%%%%%%%%%%%%%%%%%%%%%%%%%%%%%%%
\DescribeMacro{\childdocforward}
The command |\childdocforward| redirects processing to
another source file:
%
\begin{center}
\begin{tabular}{l}
|\input{childdoc.def}|\\
|\childdocforward[|\textit{main}|]{|\textit{dest}|}|\\
\end{tabular}
\end{center}
%
The argument \textit{dest} is the destination file
(without extension).
It should be the main file or one of the child files.
Note that further \textsf{childdoc} directives
such as |\childdocof| and |\childdocforward|
in the indicated file will be processed in this form.
The optional argument \textit{main}
passes on directly to the main file \textit{main}
while pretending to compile the child \textit{dest}.
This form behaves as if \textit{dest}
issues |\childdocof{|\textit{main}|}| right away,
and no further \textsf{childdoc} directives will be processed.

%%%%%%%%%%%%%%%%%%%%%%%%%%%%%%%%%%%%%%%%
\DescribeMacro{\...prefix}
In the alternative form |\childdocforwardprefix|,
%
\begin{center}
\begin{tabular}{l}
|\input{childdoc.def}|\\
|\childdocforwardprefix[|\textit{main}|]{|\textit{prefix}|}{|\textit{dest}|}|
\end{tabular}
\end{center}
%
the destination file is determined by a pattern
depending on the current file:
To make this work, the current file must be called
`{\textit{prefix}\hspace{0.2em}\textit{suffix}}'
with \textit{prefix} matching precisely the argument.
Processing is then passed on to the file
`{\textit{dest}\hspace{0.2em}\textit{suffix}}'.
Surely, the same effect is achieved by
directly specifying the
argument `{\textit{dest}\hspace{0.2em}\textit{suffix}}'
in the first form.
However, that requires to set up a different file
for each child. With the alternative form of the command
all these files can have exactly the same content
which simplifies setting them up and maintaining them.

For example, the following file |draft.tex|
with a compilation flag |\version| as described in \secref{sec:flags}
compiles the main document as a draft:
%
\begin{center}
\begin{tabular}{l}
|\def\version{draft}|\\
|\input{childdoc.def}|\\
|\childdocforward{|\textit{main}|}|
\end{tabular}
\end{center}
%
Likewise, the following files |final|\textit{nn}|.tex|
compile the final version of the child document
|child|\textit{nn}|.tex|:
%
\begin{center}
\begin{tabular}{l}
|\def\version{final}|\\
|\input{childdoc.def}|\\
|\childdocforwardprefix{final}{child}|
\end{tabular}
\end{center}
%

Note that when several versions of a main file and/or of each child file
are to be generated, it may be convenient to set up a |Makefile| or
shell script to automatise the process.

%%%%%%%%%%%%%%%%%%%%%%%%%%%%%%%%%%%%%%%%%%%%%%%%%%%%%%%%%%%%%%%%%%%%%%%%%%%%%%%%
\subsection{Command Line Processing}
\label{sec:commandline}

The effect of redirection files can also be achieved by invoking
the \LaTeX{} compiler with a more elaborate command line.
Most conveniently this should be done as part
of a shell script or a |Makefile|.

When using \textsf{childdoc} in the main file, the following
command lines effectively perform a redirection
(note that depending on the shell being used,
backslashes may have to be doubled: `|\|' $\to$ `|\\|'):
%
\begin{center}
|... -jobname "|\textit{target}|" |\\|"|[\textit{flags}]%
|\input{childdoc.def}\childdocforward[|\textit{main}|]{|\textit{dest}|}"|
\end{center}
%
Here \textit{target} is the name of the output file,
\textit{main} is the name of the main file
and \textit{dest} is the name of the main or child file to be processed
(all filenames without extensions).
The optional argument \textit{main} can be omitted
if \textit{main} matches \textit{dest}.
Optionally, compilation \textit{flags} can be defined via |\def| commands.
This command line makes the \TeX{} engine believe
it is compiling the file \textit{target}
whose content is specified as the latter parameter.
The provided code then forwards the processing to
\textit{main} or \textit{dest} as described in \secref{sec:forward}.

%%%%%%%%%%%%%%%%%%%%%%%%%%%%%%%%%%%%%%%%%%%%%%%%%%%%%%%%%%%%%%%%%%%%%%%%%%%%%%%%
\subsection{Include by Input}
\label{sec:input}

Including child documents by |\include| has some restrictions by design.
Most notably, the content of a child document always occupies
its own set of pages; pages cannot be shared between child documents.
Usually, this behaviour makes perfect sense
because each child document contain an essential part of the document.
However, in some situations it may be desirable to compose
a document from a collection of parts
without having mandatory page breaks between then.
For this case, the package
provides a mechanism to include parts
by |\input| which can also be processed individually.
However, by construction this mechanism
requires manual handling of the content to be output.

%%%%%%%%%%%%%%%%%%%%%%%%%%%%%%%%%%%%%%%%
\DescribeMacro{\ifchilddocmanual}
The main file should be prepared as usual, see \secref{sec:include}.
However, the document body must make a distinction
between processing of an individual part and of the main document, e.g.:
%
\begin{center}
\begin{tabular}{l}
|\ifchilddocmanual|\\
|\input{\childdocname}|\\
|\||else|\\
\textit{document body with }|\input{|\textit{part}|}|\\
|\||fi|
\end{tabular}
\end{center}
%
The conditional |\ifchilddocmanual| is true whenever
a part to be included by |\input| is being compiled,
and the name of the part is stored in |\childdocname|.

%%%%%%%%%%%%%%%%%%%%%%%%%%%%%%%%%%%%%%%%
\DescribeMacro{\childdocby}
Each part to be included by |\input| should start with:
%
\begin{center}
\begin{tabular}{l}
|\input{childdoc.def}|\\
|\childdocby{|\textit{main}|}|\\
\end{tabular}
\end{center}
%
The directive |\childdocby| is similar to |\childdocof|
described in \secref{sec:include},
but the subsequent selection of content must be done manually.
To that end, both |\ifchilddoc| and |\ifchilddocmanual|
will be true upon processing of a part,
and the name of the part is stored in |\childdocname|.
Note that |\jobname| will be set to the filename of the current part
so that each part receives an individual |.aux| file
that does not interfere with the |.aux| file(s) of the main document.
This behaviour can be altered by the alternative form
|\childdocby[*]{|\textit{main}|}| (with a non-empty optional argument)
which uses the |.aux| file of the main document
by setting |\jobname| to \textit{main}.

%%%%%%%%%%%%%%%%%%%%%%%%%%%%%%%%%%%%%%%%%%%%%%%%%%%%%%%%%%%%%%%%%%%%%%%%%%%%%%%%
\subsection{Driver Development}
\label{sec:driver}

The \textsf{childdoc} mechanism can also be use for the development
of definition files such as \LaTeX{} styles or classes.
This case differs from the above setup with multiple parts
included by |\include| in that no |\includeonly| should be invoked.
This can be achieved by starting the include file
(before |\ProvidesPackage|) with:
%
\begin{center}
\begin{tabular}{l}
|\input{childdoc.def}|\\
|\childdocforward{|\textit{main}|}|\\
\end{tabular}
\end{center}
%
or alternatively with:
%
\begin{center}
\begin{tabular}{l}
|\input{childdoc.def}|\\
|\childdocby{|\textit{main}|}|\\
\end{tabular}
\end{center}
%
Both forms have slightly different effects as described above.
The main file is prepared as usual, see \secref{sec:include}.

%%%%%%%%%%%%%%%%%%%%%%%%%%%%%%%%%%%%%%%%%%%%%%%%%%%%%%%%%%%%%%%%%%%%%%%%%%%%%%%%
\subsection{Legacy Detection}
\label{sec:detection}

The directive |\childdocmain| in the main file can detect
whether the complete document or merely a child is to be compiled
even without using the directive |\childdocof|.
This method is deprecated because it is less robust
and there is no compelling reason to use it;
it is merely provided for backward compatibility
and it may be removed in future versions.

If the detection mechanism is to be used,
it is mandatory to correctly specify
the filename of the main file as the argument of |\childdocmain|:
%
\begin{center}
\begin{tabular}{l}
|\input{childdoc.def}|\\
|\childdocmain{|\textit{main}|}|\\
\end{tabular}
\end{center}
%
If |\jobname| does not match the argument \textit{main} of |\childdocmain|,
it is assumed that |\jobname| points to the child file to be compiled.
When using |\childdocmain| with the main file specified as argument,
it suffices to start a child file
with just |\input{|\textit{main}|}|
without loading of the package and using |\childdocof|.
If instead all processing is done
with the appropriate \textsf{childdoc} directives,
the argument of \textit{main} of |\childdocmain| can be empty.

An alternative version of the command line processing described
in \secref{sec:commandline} using the detection mechanism reads:
%
\begin{center}
|... -jobname "|\textit{target}|" "|[\textit{flags}]%
[|\def\jobname{|\textit{dest}|}|]|\input{|\textit{main}|}"|
\end{center}

%%%%%%%%%%%%%%%%%%%%%%%%%%%%%%%%%%%%%%%%%%%%%%%%%%%%%%%%%%%%%%%%%%%%%%%%%%%%%%%%
\subsection{Manual Code}
\label{sec:manual}

In case one cannot be certain whether the definitions file |childdoc.def|
is installed on the target \TeX{} distribution
and one prefers not to ship it,
it is conceivable to paste a few relevant commands into the sources.

To that end, drop all statements |\input{childdoc.def}|
and perform the replacements as outlined below.
Instead of |\childdocmain{|\textit{main}|}| add the following code
to the top of the main file:
%
\begin{center}
\begin{tabular}{l}
|\||ifdefined\childdocname\endinput\||fi\newif\ifchilddoc|\\
|\edef\childdocname{\scantokens\expandafter{\jobname\noexpand}}|\\
|\def\childdocmain{|\textit{main}|}\||ifx\childdocmain\childdocname\||else|\\
|\childdoctrue\includeonly{\childdocname}\let\jobname\childdocmain\||fi|\\
\end{tabular}
\end{center}
%
Instead of |\childdocof{|\textit{main}|}| just include the main file
at the top of each child file:
%
\begin{center}
|\input{|\textit{main}|}|
\end{center}
%
A simple redirection |\childdocforward{|\textit{dest}|}| is achieved by:
%
\begin{center}
|\def\jobname{|\textit{dest}|}\input{\jobname}|
\end{center}
%
The redirection with prefix
|\childdocforwardprefix[|\textit{prefix}|]{|\textit{dest}|}|
is accomplished by:
%
\begin{center}
\begin{tabular}{l}
|{\edef\jobname{\scantokens\expandafter{\jobname\noexpand}}|\\
|\def\redirectjob |\textit{prefix}|#1~~~{\gdef\jobname{|\textit{dest}|#1}}|\\
|\expandafter\redirectjob\jobname~~~}\input{\jobname}|
\end{tabular}
\end{center}

In an alternative approach,
child documents can be compiled by a specific command line
without additional code or specific definitions:
%
\begin{center}
|... -jobname "|\textit{target}|" "|[\textit{flags}]%
|\includeonly{|\textit{dest}|}\input{|\textit{main}|}"|
\end{center}
%

%%%%%%%%%%%%%%%%%%%%%%%%%%%%%%%%%%%%%%%%%%%%%%%%%%%%%%%%%%%%%%%%%%%%%%%%%%%%%%%%
%%%%%%%%%%%%%%%%%%%%%%%%%%%%%%%%%%%%%%%%%%%%%%%%%%%%%%%%%%%%%%%%%%%%%%%%%%%%%%%%
\section{Information}

%%%%%%%%%%%%%%%%%%%%%%%%%%%%%%%%%%%%%%%%%%%%%%%%%%%%%%%%%%%%%%%%%%%%%%%%%%%%%%%%
\subsection{Copyright}

Copyright \copyright{} 2017--2018 Niklas Beisert

This work may be distributed and/or modified under the
conditions of the \LaTeX{} Project Public License, either version 1.3
of this license or (at your option) any later version.
The latest version of this license is in
  \url{http://www.latex-project.org/lppl.txt}
and version 1.3 or later is part of all distributions of \LaTeX{}
version 2005/12/01 or later.

This work has the LPPL maintenance status `maintained'.

The Current Maintainer of this work is Niklas Beisert.

This work consists of the files |README.txt|, |childdoc.ins| and |childdoc.dtx|
as well as the derived files |childdoc.def|, |cdocsamp.tex|
with |cdocsch1.tex|, |cdocsch2.tex|, |cdocspt3.tex|, |cdocspt4.tex|,
|cdocsdrf.tex|, |cdocsfn1.tex|, |cdocsfn2.tex|
as well as |childdoc.pdf|.

%%%%%%%%%%%%%%%%%%%%%%%%%%%%%%%%%%%%%%%%%%%%%%%%%%%%%%%%%%%%%%%%%%%%%%%%%%%%%%%%
\subsection{Files and Installation}

The package consists of the files:
%
\begin{center}
\begin{tabular}{ll}
    |README.txt|   & readme file \\
    |childdoc.ins| & installation file \\
    |childdoc.dtx| & source file \\
    |childdoc.def| & definition file \\
    |cdocsamp.tex| & sample main file \\
    |cdocsch1.tex| & sample include file \\
    |cdocsch2.tex| & sample include file \\
    |cdocspt3.tex| & sample part file \\
    |cdocspt4.tex| & sample part file \\
    |cdocsdrf.tex| & sample redirection file \\
    |cdocsfn1.tex| & sample redirection file \\
    |cdocsfn2.tex| & sample redirection file \\
    |childdoc.pdf| & manual
\end{tabular}
\end{center}
%
The distribution consists of the files
|README.txt|, |childdoc.ins| and |childdoc.dtx|.
%
\begin{itemize}
\item
Run (pdf)\LaTeX{} on |childdoc.dtx|
to compile the manual |childdoc.pdf| (this file).
\item
Run \LaTeX{} on |childdoc.ins| to create the definitions file |childdoc.def|
and the sample |cdocsamp.tex| with include files
|cdocsch1.tex|, |cdocsch2.tex|, |cdocspt3.tex|, |cdocspt4.tex|,
|cdocsdrf.tex|, |cdocsfn1.tex|, |cdocsfn2.tex|.
Then copy the file |childdoc.def| to an appropriate directory of your \LaTeX{}
distribution, e.g.\ \textit{texmf-root}|/tex/latex/childdoc|.
\end{itemize}

%%%%%%%%%%%%%%%%%%%%%%%%%%%%%%%%%%%%%%%%%%%%%%%%%%%%%%%%%%%%%%%%%%%%%%%%%%%%%%%%
\subsection{Related CTAN Packages}

There are several other packages which offer a similar functionality:
%
\begin{itemize}
\item
The packages
\href{http://ctan.org/pkg/docmute}{\textsf{docmute}},
\href{http://ctan.org/pkg/includex}{\textsf{includex}} and
\href{http://ctan.org/pkg/standalone}{\textsf{standalone}}
provide commands to include only the document body of
a child file thus allowing both files to be compiled individually.
\item
The packages \href{http://ctan.org/pkg/subdocs}{\textsf{subdocs}}
and \href{http://ctan.org/pkg/subfiles}{\textsf{subfiles}}
provide structures in which the main and child documents can be
encapsulated and allowing them to be compiled individually.
The inclusion mechanism is different from the conventional |\include|.
\item
The package \href{http://ctan.org/pkg/combine}{\textsf{combine}}
is an elaborate solution to combine several documents into one.
\end{itemize}
%
See also the CTAN topic \href{http://ctan.org/topic/subdocs}{\textsf{subdocs}}
for further related packages.
The present package differs from the above solutions in that
a document structure constructed with the conventional |\include| mechanism
just needs two extra commands at the top of every file
such that all constituent files can be compiled individually.

%%%%%%%%%%%%%%%%%%%%%%%%%%%%%%%%%%%%%%%%%%%%%%%%%%%%%%%%%%%%%%%%%%%%%%%%%%%%%%%%
%\subsection{Feature Suggestions}
%
%The following is a list of features which may be useful for future
%versions of this package:
%%
%\begin{itemize}
%\item
%\ldots
%\end{itemize}

%%%%%%%%%%%%%%%%%%%%%%%%%%%%%%%%%%%%%%%%%%%%%%%%%%%%%%%%%%%%%%%%%%%%%%%%%%%%%%%%
\subsection{Revision History}

%%%%%%%%%%%%%%%%%%%%%%%%%%%%%%%%%%%%%%%%
\paragraph{v2.0:} 2018/12/30

\begin{itemize}
\item
immediate forward processing
\item
added |\childdocby| mechanism
\item
manual restructured
\end{itemize}

%%%%%%%%%%%%%%%%%%%%%%%%%%%%%%%%%%%%%%%%
\paragraph{v1.6:} 2018/01/17

\begin{itemize}
\item
application for development of include files
\item
corrections to manual
\end{itemize}

%%%%%%%%%%%%%%%%%%%%%%%%%%%%%%%%%%%%%%%%
\paragraph{v1.5:} 2017/05/21

\begin{itemize}
\item
more complete structuring introduced
\item
|\childdocof| introduced
\item
|\childdoc| renamed to |\childdocmain|
\item
|\childredirect| renamed to |\childdocforward| and |\childdocforwardprefix|
and functionality expanded
\end{itemize}

%%%%%%%%%%%%%%%%%%%%%%%%%%%%%%%%%%%%%%%%
\paragraph{v1.0:} 2017/04/27

\begin{itemize}
\item
manual and install package
\item
first version published on CTAN
\end{itemize}

%%%%%%%%%%%%%%%%%%%%%%%%%%%%%%%%%%%%%%%%
\paragraph{v0.6:} 2017/04/26

\begin{itemize}
\item
redirection mechanism added
\end{itemize}

%%%%%%%%%%%%%%%%%%%%%%%%%%%%%%%%%%%%%%%%
\paragraph{v0.5:} 2017/04/26

\begin{itemize}
\item
functionality in definition file
\end{itemize}


%%%%%%%%%%%%%%%%%%%%%%%%%%%%%%%%%%%%%%%%%%%%%%%%%%%%%%%%%%%%%%%%%%%%%%%%%%%%%%%%
%%%%%%%%%%%%%%%%%%%%%%%%%%%%%%%%%%%%%%%%%%%%%%%%%%%%%%%%%%%%%%%%%%%%%%%%%%%%%%%%
%%%%%%%%%%%%%%%%%%%%%%%%%%%%%%%%%%%%%%%%%%%%%%%%%%%%%%%%%%%%%%%%%%%%%%%%%%%%%%%%
\appendix

\settowidth\MacroIndent{\rmfamily\scriptsize 000\ }

 \DocInput{childdoc.dtx}

\end{document}
%</driver>
% \fi
%
% %%%%%%%%%%%%%%%%%%%%%%%%%%%%%%%%%%%%%%%%%%%%%%%%%%%%%%%%%%%%%%%%%%%%%%%%%%%%%%
% %%%%%%%%%%%%%%%%%%%%%%%%%%%%%%%%%%%%%%%%%%%%%%%%%%%%%%%%%%%%%%%%%%%%%%%%%%%%%%
% \section{Sample}
%\iffalse
%<*samplemain>
%\fi
%
% The following presents a sample document
% with two chapters, two parts, a title page,
% a compile flag as well as three forwarding files to set the flag.
% It consists of eight |.tex| files:
% \begin{center}
% \begin{tabular}{ll}
% |cdocsamp.tex|&main file\\
% |cdocsch1.tex|&include file for chapter 1\\
% |cdocsch2.tex|&include file for chapter 2\\
% |cdocspt3.tex|&include file for part 3\\
% |cdocspt4.tex|&include file for part 4\\
% |cdocsdrf.tex|&forwarding file for main file in draft mode\\
% |cdocsfi1.tex|&forwarding file for final version of chapter 1\\
% |cdocsfi2.tex|&forwarding file for final version of chapter 2\\
% \end{tabular}
% \end{center}
% Each of the eight files can be compiled directly by the \LaTeX{} compiler.
%
% %%%%%%%%%%%%%%%%%%%%%%%%%%%%%%%%%%%%%%
% \paragraph{Main File.}
%
% The main file is called |cdocsamp.tex|.
%
% Load the \textsf{childdoc} definitions and
% declare the filename for the main document:
%    \begin{macrocode}
\input{childdoc.def}
\childdocmain{}
%    \end{macrocode}

% Optional override for |\version| flag:
%    \begin{macrocode}
%%\ifchilddoc\else\providecommand{\version}{draft}\fi
%    \end{macrocode}

% Define the default values for the |\version| flag
% (|final| for the main file and |draft| for childs):
%    \begin{macrocode}
\ifchilddoc
\providecommand{\version}{draft}
\else
\providecommand{\version}{final}
\fi
%    \end{macrocode}

% Load the standard document class:
%    \begin{macrocode}
\documentclass[12pt]{article}
%    \end{macrocode}

% Start the document body:
%    \begin{macrocode}
\begin{document}
%    \end{macrocode}

% Declare a title page.
% Print title, part of document being processed and version flag:
%    \begin{macrocode}
\addtocounter{page}{-1}
\begin{center}
{\LARGE\bfseries{}childdoc example\par}
\vspace{1cm}
\ifchilddoc
\ifchilddocmanual part\else chapter\fi:
`\childdocname' of `\childdocjob'\par
\else
main document: `\childdocjob'\par
\fi
version: \version\par
\end{center}
\newpage
%    \end{macrocode}

% Manually include selected file,
% otherwise process as usual:
%    \begin{macrocode}
\ifchilddocmanual
\section*{part `\childdocname'}
\input{\childdocname}
\else
%    \end{macrocode}

% Include the two chapters:
%    \begin{macrocode}
\include{cdocsch1}
\include{cdocsch2}
%    \end{macrocode}

% Include the two parts unless only chapters should be displayed:
%    \begin{macrocode}
\ifchilddoc\else
\section{part three}
\input{cdocspt3}
\section{part four}
\input{cdocspt4}
\fi
%    \end{macrocode}

% Process as usual until here:
%    \begin{macrocode}
\fi
%    \end{macrocode}

% End of document body:
%    \begin{macrocode}
\end{document}
%    \end{macrocode}
%\iffalse
%</samplemain>
%\fi
%
% %%%%%%%%%%%%%%%%%%%%%%%%%%%%%%%%%%%%%%
% \paragraph{Chapter Include Files.}
%
% The include files are called |cdocsch1.tex| and |cdocsch2.tex|.
%
%\iffalse
%<*samplechap1|samplechap2>
%\fi

% Optional override for |\version| flag:
%    \begin{macrocode}
%%\providecommand{\version}{final}
%    \end{macrocode}

% Include the main document:
%    \begin{macrocode}
\input{childdoc.def}
\childdocof{cdocsamp}
%    \end{macrocode}

%\iffalse
%</samplechap1|samplechap2>
%\fi
%
%\iffalse
%<*samplechap1>
%\fi
% Some text for chapter 1:
%    \begin{macrocode}
\section{one}
some text in chapter one
%    \end{macrocode}

%\iffalse
%</samplechap1>
%\fi
% Some text for chapter 2:
%\iffalse
%<*samplechap2>
%\fi
%    \begin{macrocode}
\section{two}
more text in chapter two
%    \end{macrocode}

%\iffalse
%</samplechap2>
%\fi
%
% %%%%%%%%%%%%%%%%%%%%%%%%%%%%%%%%%%%%%%
% \paragraph{Part Include Files.}
%
% The include files are called |cdocspt3.tex| and |cdocspt4.tex|.
%
%\iffalse
%<*samplepart3|samplepart4>
%\fi

% Optional override for |\version| flag:
%    \begin{macrocode}
%%\providecommand{\version}{final}
%    \end{macrocode}

% Include the main document:
%    \begin{macrocode}
\input{childdoc.def}
\childdocby{cdocsamp}
%    \end{macrocode}

%\iffalse
%</samplepart3|samplepart4>
%\fi
%
%\iffalse
%<*samplepart3>
%\fi
% Some text for part 3:
%    \begin{macrocode}
some text in part three
%    \end{macrocode}

%\iffalse
%</samplepart3>
%\fi
% Some text for part 4:
%\iffalse
%<*samplepart4>
%\fi
%    \begin{macrocode}
more text in part four
%    \end{macrocode}

%\iffalse
%</samplepart4>
%\fi
%
% %%%%%%%%%%%%%%%%%%%%%%%%%%%%%%%%%%%%%%
% \paragraph{Forwarding for a Complete Draft.}
%
% The following forwarding file |cdocsdrf.tex|
% compiles the main document in draft mode:
%\iffalse
%<*sampledraft>
%\fi
%    \begin{macrocode}
\def\version{draft}
\input{childdoc.def}
\childdocforward{cdocsamp}
%    \end{macrocode}

%\iffalse
%</sampledraft>
%\fi
%
% %%%%%%%%%%%%%%%%%%%%%%%%%%%%%%%%%%%%%%
% \paragraph{Forwarding for Final Version of the Chapters.}
%
% The following forwarding files |cdocsfn1.tex| and |cdocsfn2.tex|
% (with identical content)
% compile the final versions of the child documents
% |cdocsch1.tex| and |cdocsch2.tex|, respectively:
%\iffalse
%<*samplefinal>
%\fi
%    \begin{macrocode}
\def\version{final}
\input{childdoc.def}
\childdocforwardprefix[cdocsamp]{cdocsfn}{cdocsch}
%    \end{macrocode}

%\iffalse
%</samplefinal>
%\fi
%
% %%%%%%%%%%%%%%%%%%%%%%%%%%%%%%%%%%%%%%
% \paragraph{Command Line Processing.}
%
% The following three command lines generate the output files
% |cdocscld|, |cdocscl1| and |cdocscl2|
% which should be identical to
% |cdocsdrf|, |cdocsch1| and |cdocsfn2|, respectively:
% \begin{center}
% \begin{tabular}{l}
% |latex -jobname cdocscld \|\\
% |  "\def\version{draft}\input{childdoc.def}\childdocforward{cdocsamp}"|\\
% |latex -jobname cdocscl1 \|\\
% |  "\input{childdoc.def}\childdocforward[cdocsamp]{cdocsch1}"|\\
% |latex -jobname cdocscl2 \|\\
% |  "\def\version{final}\input{childdoc.def}\childdocforward{cdocsch2}"|
% \end{tabular}
% \end{center}
% Note that the trailing backslash on each first line
% merely continues the input to the second line
% (for convenient cut ant paste).
% Furthermore, the command |latex| can be replaced by any
% of its alternative versions such as |pdflatex|.
%
% %%%%%%%%%%%%%%%%%%%%%%%%%%%%%%%%%%%%%%%%%%%%%%%%%%%%%%%%%%%%%%%%%%%%%%%%%%%%%%
% %%%%%%%%%%%%%%%%%%%%%%%%%%%%%%%%%%%%%%%%%%%%%%%%%%%%%%%%%%%%%%%%%%%%%%%%%%%%%%
% \section{Implementation}
%\iffalse
%<*package>
%\fi
%
% This section describes the definitions file |childdoc.def|.

% The definitions cannot be loaded using |\usepackage| or |\RequirePackage|
% which has a mechanism to prevent loading a style file more than once.
% When loading the definitions by means of |\input|
% multiple instances have to be prevented manually:
%\iffalse
%This code needs to be before the `\ProvidesFile' directive
%which is defined at the beginning of this file.
%Therefore it is also placed there and commented out here.
%</package>
%<*discard>
%\fi
%    \begin{macrocode}
\ifdefined\childdocmain\endinput\fi
%    \end{macrocode}
%\iffalse
%</discard>
%<*package>
%\fi
%
% \macro{\ifchilddoc}
% \macro{\ifchilddocmanual}
% The conditional |\ifchilddoc| tells whether a
% child (true) or main (false) document is being compiled.
% The conditional |\ifchilddocmanual| tells whether
% the |\includeonly| mechanism is used (false) or
% the selection of child files must be performed manually (true).
% The definitions initialise to false:
%    \begin{macrocode}
\newif\ifchilddoc
\newif\ifchilddocmanual
%    \end{macrocode}

% \macro{\childdocname}
% \macro{\childdocjob}
% The macro |\childdocname| stores the name of the main document
% to be compiled. The macro |\childdocjob| stores the name of
% the document on which the \LaTeX{} compiler was originally invoked.
% The content of |\jobname| cannot be compared
% to filenames specified in the source due to different catcodes.
% The following code rescans |\jobname|, stores the result
% in |\childdocname| and saves a copy in |\childdocjob|:
%    \begin{macrocode}
\edef\childdocname{\scantokens\expandafter{\jobname\noexpand}}
\let\childdocjob\childdocname
%    \end{macrocode}

% \macro{\childdocdisable}
% The macro |\childdocdisable| prevents the main file
% from being processed more than once.
% At this stage, the main document command |\childdocmain|
% is assumed to be called once again where it should do nothing.
% Any subsequent call to it should prevent
% a secondary processing of the main document
% It overwrites the forwarding commands
% |\childdocof| and |\childdocforward|
% with empty macros to prevent further inclusions of the main document:
%    \begin{macrocode}
\newcommand{\childdocdisable}
{
  \renewcommand{\childdocmain}[1]{\renewcommand{\childdocmain}[1]{\endinput}}
  \renewcommand{\childdocof}[1]{}
  \renewcommand{\childdocby}[2][]{}
  \renewcommand{\childdocforward}[2][]{}
  \renewcommand{\childdocdisable}{}
}
%    \end{macrocode}

% \macro{\childdocmain}
% The macro |\childdocmain| is to be called at the top of the main file
% with nothing or the main filename (without extension) as argument.
% First, it breaks loops.
% If the argument is not empty and does not match |\childdocname|
% (which is set by the first inclusion of |childdoc.def|),
% |\ifchilddoc| is set to true, |\includeonly| is applied to the child file
% and |\jobname| is set to the main file
% (for proper handling of |.aux| files):
%    \begin{macrocode}
\newcommand{\childdocmain}[1]
{
  \childdocdisable\childdocmain{}
  \if?#1?\else
    \begingroup
      \def\childdoctmp{#1}
      \ifx\childdoctmp\childdocname
        \def\childdoctmp{}
      \else
        \def\childdoctmp
        {
          \childdoctrue
          \includeonly{\childdocname}
          \def\childdocjob{#1}
          \def\jobname{#1}
        }
      \fi
      \expandafter
    \endgroup
    \childdoctmp
  \fi
}
%    \end{macrocode}

% \macro{\childdocof}
% The command |\childdocof| redirects
% compilation to the main file |#1|.
%    \begin{macrocode}
\newcommand{\childdocof}[1]
{
  \childdocdisable
  \childdoctrue
  \includeonly{\childdocname}
  \def\jobname{#1}
  \def\childdocjob{#1}
  \input{#1}
}
%    \end{macrocode}

% \macro{\childdocby}
% The command |\childdocby| ....
%    \begin{macrocode}
\newcommand{\childdocby}[2][]
{
  \childdocdisable
  \childdoctrue
  \childdocmanualtrue
  \if?#1?\else
    \def\jobname{#2}
  \fi
  \def\childdocjob{#2}
  \input{#2}
  \endinput
}
%    \end{macrocode}

% \macro{\childdocforward}
% The command |\childdocforward| redirects
% compilation to the main file or
% (if the optional argument is given) a child file.
% Parameters are set as if the main file
% or a child file starting with |\childdocof| was compiled.
% Then compilation is handed over to the main file:
%    \begin{macrocode}
\newcommand{\childdocforward}[2][]
{
  \begingroup
    \if?#1?
      \def\childdoctmp
      {
        \def\childdocname{#2}
        \def\childdocjob{#2}
        \def\jobname{#2}
        \input{#2}
        \endinput
      }
    \else
      \def\childdoctmp
      {
        \childdocdisable
        \def\childdocname{#2}
        \childdoctrue
        \includeonly{#2}
        \def\childdocjob{#1}
        \def\jobname{#1}
        \input{#1}
        \endinput
      }
    \fi
    \expandafter
  \endgroup
  \childdoctmp
}
%    \end{macrocode}

% \macro{\childdocforwardprefix}
% The command |\childdocforwardprefix| redirects
% compilation to the main or a child file by means of a pattern.
% The prefix |#1| in the current filename is replaced by |#2|
% and the suffix of the current filename is kept
% (it is assumed that the filename does not contain the substring `|~~~|'
% which is used as a delimiter).
% Compilation is handed over to the new file by |\childdocforward|:
%    \begin{macrocode}
\newcommand{\childdocforwardprefix}[3][]
{
  \begingroup
    \def\childdocextract #2##1~~~{\def\childdoctmp{\childdocforward[#1]{#3##1}}}
    \expandafter\childdocextract\childdocname~~~
    \expandafter
  \endgroup
  \childdoctmp
}
%    \end{macrocode}

% \macro{\childdoc}
% The deprecated macro |\childdoc| is a legacy version of |\childdocmain|:
%    \begin{macrocode}
\newcommand{\childdoc}{\childdocmain}
%    \end{macrocode}

% \macro{\childdocredirect}
% The deprecated macro |\childdocredirect| is a legacy version
% of |\childdocforward| and |\childdocforwardprefix|:
%    \begin{macrocode}
\newcommand{\childdocredirect}[2][]
{
  \begingroup
    \if?#1?
      \def\childdoctmp{\childdocforward{#2}}
    \else
      \def\childdoctmp{\childdocforwardprefix{#1}{#2}}
    \fi
    \expandafter
  \endgroup
  \childdoctmp
}
%    \end{macrocode}

%\iffalse
%</package>
%\fi
%
\endinput
|\\
|\childdocforwardprefix[|\textit{main}|]{|\textit{prefix}|}{|\textit{dest}|}|
\end{tabular}
\end{center}
%
the destination file is determined by a pattern
depending on the current file:
To make this work, the current file must be called
`{\textit{prefix}\hspace{0.2em}\textit{suffix}}'
with \textit{prefix} matching precisely the argument.
Processing is then passed on to the file
`{\textit{dest}\hspace{0.2em}\textit{suffix}}'.
Surely, the same effect is achieved by
directly specifying the
argument `{\textit{dest}\hspace{0.2em}\textit{suffix}}'
in the first form.
However, that requires to set up a different file
for each child. With the alternative form of the command
all these files can have exactly the same content
which simplifies setting them up and maintaining them.

For example, the following file |draft.tex|
with a compilation flag |\version| as described in \secref{sec:flags}
compiles the main document as a draft:
%
\begin{center}
\begin{tabular}{l}
|\def\version{draft}|\\
|% \iffalse
%
% childdoc.dtx Copyright (C) 2017-2018 Niklas Beisert
%
% This work may be distributed and/or modified under the
% conditions of the LaTeX Project Public License, either version 1.3
% of this license or (at your option) any later version.
% The latest version of this license is in
%   http://www.latex-project.org/lppl.txt
% and version 1.3 or later is part of all distributions of LaTeX
% version 2005/12/01 or later.
%
% This work has the LPPL maintenance status `maintained'.
%
% The Current Maintainer of this work is Niklas Beisert.
%
% This work consists of the files childdoc.dtx and childdoc.ins
% and the derived files childdoc.def and cdocsamp.tex with
% cdocsch1.tex, cdocsch2.tex, cdocsdrf.tex, cdocsfn1.tex, cdocsfn2.tex.
%
%<package>\ifdefined\childdocmain\endinput\fi
%<package>\ProvidesFile{childdoc.def}[2018/12/30 v2.0 child document driver]
%<samplemain>\ProvidesFile{cdocsamp.tex}[2018/12/30 v2.0 sample for childdoc]
%<*driver>
%\ProvidesFile{childdoc.drv}[2018/12/30 v2.0 childdoc reference manual file]
\PassOptionsToClass{10pt,a4paper}{article}
\documentclass{ltxdoc}

\usepackage[margin=35mm]{geometry}
\usepackage{hyperref}
\usepackage{hyperxmp}
\usepackage[usenames]{color}

\hypersetup{colorlinks=true}
\hypersetup{pdfstartview=FitH}
\hypersetup{pdfpagemode=UseNone}
\hypersetup{pdfsource={}}
\hypersetup{pdflang={en-UK}}
\hypersetup{pdfcopyright={Copyright 2017-2018 Niklas Beisert.
  This work may be distributed and/or modified under the
  conditions of the LaTeX Project Public License, either version 1.3
  of this license or (at your option) any later version.}}
\hypersetup{pdflicenseurl={http://www.latex-project.org/lppl.txt}}
\hypersetup{pdfcontactaddress={ETH Zurich, ITP, HIT K,
  Wolfgang-Pauli-Strasse 27}}
\hypersetup{pdfcontactpostcode={8093}}
\hypersetup{pdfcontactcity={Zurich}}
\hypersetup{pdfcontactcountry={Switzerland}}
\hypersetup{pdfcontactemail={nbeisert@itp.phys.ethz.ch}}
\hypersetup{pdfcontacturl={http://people.phys.ethz.ch/\xmptilde nbeisert/}}

\newcommand{\secref}[1]{\hyperref[#1]{section \ref*{#1}}}

\parskip1ex
\parindent0pt
\let\olditemize\itemize
\def\itemize{\olditemize\parskip0pt}

\begin{document}

\title{The \textsf{childdoc} Package}
\hypersetup{pdftitle={The childdoc Package}}
\author{Niklas Beisert\\[2ex]
  Institut f\"ur Theoretische Physik\\
  Eidgen\"ossische Technische Hochschule Z\"urich\\
  Wolfgang-Pauli-Strasse 27, 8093 Z\"urich, Switzerland\\[1ex]
  \href{mailto:nbeisert@itp.phys.ethz.ch}
  {\texttt{nbeisert@itp.phys.ethz.ch}}}
\hypersetup{pdfauthor={Niklas Beisert}}
\hypersetup{pdfsubject={Manual for the LaTeX2e Package childdoc}}
\date{30 December 2018, \textsf{v2.0}}
\maketitle

\begin{abstract}\noindent
\textsf{childdoc} is a \LaTeXe{} package
that enables the direct compilation
of document sections included by |\include|
to individual files.
\end{abstract}

\begingroup
\parskip0ex
\tableofcontents
\endgroup

%%%%%%%%%%%%%%%%%%%%%%%%%%%%%%%%%%%%%%%%%%%%%%%%%%%%%%%%%%%%%%%%%%%%%%%%%%%%%%%%
%%%%%%%%%%%%%%%%%%%%%%%%%%%%%%%%%%%%%%%%%%%%%%%%%%%%%%%%%%%%%%%%%%%%%%%%%%%%%%%%
\section{Introduction}

\LaTeX{} provides a mechanism to structure a large document (such as a book)
into a main file and several child files (containing the chapters)
using the |\include| command.
This mechanism is beneficial for documents
which span hundreds of pages in order to
make the source file(s) more manageable.
Moreover, compilation can be restricted to
selected child files by means of the |\includeonly| command.
The latter feature can be used to reduce the compilation time while editing
(this was significantly more useful in the earlier days of \LaTeX{})
or to generate a smaller document which is easier to navigate.
Another application of |\includeonly| is to generate
documents consisting of selected parts of the complete document.

However, there are a few drawbacks of the plain |\include| mechanism:
\begin{itemize}
\item
The child files cannot be compiled on their own,
they can only be compiled via the main file.
A naive editing environment
(such as a text editor with an option
to have the current file processed by \LaTeX)
may require one to switch to the main file before compiling;
attempting to compile the child file produces errors.
\item
The main file must be modified (each time)
to adjust the |\includeonly| command
to the present needs. This easily leaves the main file in a messy state.
\item
The generated document will always carry the filename
of the main document. This is inconvenient if
several child files are to be compiled and
to be kept for distribution.
\end{itemize}

The present package provides a simple interface
to make child files individually compilable by \LaTeX{}.
Compiling a child file then has the same effect as compiling
the main file with an |\includeonly| command
to select the appropriate child.
Moreover the generated document will carry the name of the child
rather than the main file.
This resolves all three above issues.

This feature is meant to make the editing of books,
thesis documents and lecture notes somewhat more convenient.
However, the package can also be used efficiently for
composing a series of documents (such as exercise sheets)
which are typically distributed individually.
It then assists the author in generating the individual documents
(potentially in different versions)
as well as a document containing the collected series.
Another application is in developing style files
or other kinds of included material
where compilation of the style file could redirect
to a sample or test file.

%%%%%%%%%%%%%%%%%%%%%%%%%%%%%%%%%%%%%%%%%%%%%%%%%%%%%%%%%%%%%%%%%%%%%%%%%%%%%%%%
%%%%%%%%%%%%%%%%%%%%%%%%%%%%%%%%%%%%%%%%%%%%%%%%%%%%%%%%%%%%%%%%%%%%%%%%%%%%%%%%
\section{Usage}

First of all, the package \textsf{childdoc} is \emph{not} a standard
\LaTeXe{} |.sty| style file! Therefore it needs to be invoked in
a non-standard way.

%%%%%%%%%%%%%%%%%%%%%%%%%%%%%%%%%%%%%%%%%%%%%%%%%%%%%%%%%%%%%%%%%%%%%%%%%%%%%%%%
\subsection{Included Files}
\label{sec:include}

%%%%%%%%%%%%%%%%%%%%%%%%%%%%%%%%%%%%%%%%
\DescribeMacro{\childdocmain}
To use the package, add the commands
\begin{center}
\begin{tabular}{l}
|\input{childdoc.def}|\\
|\childdocmain{}|\\
\end{tabular}
\end{center}
at the very top of the main \LaTeX{} file,
in particular \emph{before} the |\documentclass| statement!
The argument of |\childdocmain| should be left empty
(but it must be present).

%%%%%%%%%%%%%%%%%%%%%%%%%%%%%%%%%%%%%%%%
\DescribeMacro{\childdocof}
Furthermore, add the commands
\begin{center}
\begin{tabular}{l}
|\input{childdoc.def}|\\
|\childdocof{|\textit{main}|}|\\
\end{tabular}
\end{center}
at the top of every child file \textit{child}
which is included by |\include{|\textit{child}|}|
from within the main file
(or at least for those files to be compiled individually).
The argument \textit{main} must be the filename of the main file.

There are a couple of
considerations in setting up the main and child documents:

%%%%%%%%%%%%%%%%%%%%%%%%%%%%%%%%%%%%%%%%
\paragraph{Restrictions.}

Please note the following restrictions:
\begin{itemize}
\item
|\childdocmain| must be called with one argument \textit{main}
to ensure compatibility with earlier version of the package.
It must either be empty (|\childdocmain{}|)
or precisely match the filename of the main file in which it is specified.
See \secref{sec:detection} for further information.
\item
The filename \textit{main} must be specified without the |.tex| extension.
\item
The filename \textit{main} is case sensitive
(even in case-insensitive file systems)
due to internal string comparison.
\item
The argument \textit{main} should be fully expanded, it cannot be a macro.
\item
Subdirectories and special characters should be avoided in filenames.
\item
The command |\childdocmain{|\textit{main}|}| must be followed by a whitespace.
It should not be followed immediately by another command
or by a comment mark `|%|'.
This is because the \TeX{} parser reads the token immediately following
the argument of |\childdocmain| and puts it
at the beginning of every child section;
however, a white\-space is ignored.
\end{itemize}

%%%%%%%%%%%%%%%%%%%%%%%%%%%%%%%%%%%%%%%%
\paragraph{Content of Main File.}

It is advisable to place all content in the child files included by |\include|.
Any output contained in the main file will appear in all child documents
unless suppressed manually;
it cannot be suppressed automatically by the |\includeonly| directive
and thus should normally be avoided.
A method to include some content in the main file
by means of conditional processing is described in \secref{sec:conditional}.

%%%%%%%%%%%%%%%%%%%%%%%%%%%%%%%%%%%%%%%%
\paragraph{Page Numbering.}

When only a part of the document is compiled,
the appropriate numbering of pages
(as well as other status parameters)
is determined from the |.aux| files.
The latter contain information from previous passes.
However this information needs to propagate through
all intermediate child documents.
Therefore the page numbering in child documents may well
be inconsistent until the complete document is compiled at least once.

A useful (if unconventional) way to always ensure a consistent
page numbering is to restart the numbering in each child document
and denote the pages by `\textit{child}|.|\textit{page}'
where \textit{child} represents the chapter/section number of the child file.
This can be achieved by the command
|\numberwithin{page}{|\textit{child}|}|
of the \textsf{amsmath} package
where \textit{child} can be |chapter| or |section|
depending on the chosen structuring.
Alternatively, one can modify the macro |\thepage| appropriately
and reset the counter |page| at the start of each child file.

%%%%%%%%%%%%%%%%%%%%%%%%%%%%%%%%%%%%%%%%%%%%%%%%%%%%%%%%%%%%%%%%%%%%%%%%%%%%%%%%
\subsection{Conditional Processing}
\label{sec:conditional}

The package provides a mechanism to compile different versions
of a document. To customise the versions further some conditional processing
can come in handy to distinguish which version is being compiled.
The package provides two macros to describe the compilation context:

%%%%%%%%%%%%%%%%%%%%%%%%%%%%%%%%%%%%%%%%
\DescribeMacro{\ifchilddoc}
The conditional |\ifchilddoc| distinguishes between the compilation of
child documents and the main document:
%
\begin{center}
|\ifchilddoc |\textit{child-code}| |[|\||else |\textit{main-code}]| \||fi|
\end{center}

%%%%%%%%%%%%%%%%%%%%%%%%%%%%%%%%%%%%%%%%
\DescribeMacro{\childdocname}
\DescribeMacro{\childdocjob}
The macro |\childdocname| contains the filename (without extension)
of the main or child file being processed.
Note that |\childdocjob| will always contain the name of the main file.

%%%%%%%%%%%%%%%%%%%%%%%%%%%%%%%%%%%%%%%%
\paragraph{Title Page.}

Conditional processing can be used to include a title or banner page
in the main document when proper precautions are taken.
Importantly, the code in the main file should ensure that the page counter
(as well as other status parameters which are stored in the |.aux| files)
takes the same value after the conditional processing.
Otherwise the page numbers may take divergent values
depending on which part is compiled.

For example, a title page could be declared by:
%
\begin{center}
\begin{tabular}{l}
|\ifchilddoc\||else|\\
|\addtocounter{page}{-1}|\\
\textit{code for title page}\\
|\newpage|\\
|\||fi|
\end{tabular}
\end{center}
%
A banner page for the child documents can be generated by:
%
\begin{center}
\begin{tabular}{l}
|\ifchilddoc|\\
|\addtocounter{page}{-1}|\\
\textit{code for banner page}\\
|\newpage|\\
|\||fi|
\end{tabular}
\end{center}
%
Here one could write a message such as:
\begin{center}
|This is the part \childdocname{} of \childdocjob{}.|
\end{center}

%%%%%%%%%%%%%%%%%%%%%%%%%%%%%%%%%%%%%%%%%%%%%%%%%%%%%%%%%%%%%%%%%%%%%%%%%%%%%%%%
\subsection{Flags}
\label{sec:flags}

The package makes it easy to generate different versions
of the main or child documents.
To this end compilation flags can be defined
and assigned different default values.
They will be particularly useful in conjunction
with the forwarding mechanism described in \secref{sec:forward}.

For example, it may be useful to have a flag |\version|
which can be set to |draft| or |final|.
The document source will contain some conditional code
depending on the value of |\version|.
Suppose further, the flag should default to |final| for the main file
and to |draft| for child files
which is a natural assignment for editing the document.
This is achieved by placing the following code
in the preamble of the main document
(below the |\childdocmain| directive):
%
\begin{center}
\begin{tabular}{l}
|\ifchilddoc|\\
|\providecommand{\version}{draft}|\\
|\||else|\\
|\providecommand{\version}{final}|\\
|\||fi|
\end{tabular}
\end{center}
%
The definition by |\providecommand| makes sure
that previous definitions are not overwritten.
Further statements |\providecommand{\version}{...}|
can thus be added before the above code to override it.

For the main file, one might add a line
(between |\childdocmain| and the above block)
%
\begin{center}
|%\ifchilddoc\||else\providecommand{\version}{draft}\||fi|
\end{center}
%
which can be uncommented to produce a draft version.
Likewise one can add a line to the very top of a child file
(above the |\childdocof{|\textit{main}|}| directive)
%
\begin{center}
|%\providecommand{\version}{final}|
\end{center}
%
which can be uncommented to produce the final version of this child document.

%%%%%%%%%%%%%%%%%%%%%%%%%%%%%%%%%%%%%%%%%%%%%%%%%%%%%%%%%%%%%%%%%%%%%%%%%%%%%%%%
\subsection{Forwarding}
\label{sec:forward}

Different versions of the main or child documents
using compilation flags as described in \secref{sec:flags}
can be (permanently) stored in different files
for convenient compilation, viewing and distribution.
To this end, the package defines a command
to pass on compilation to a different file:

%%%%%%%%%%%%%%%%%%%%%%%%%%%%%%%%%%%%%%%%
\DescribeMacro{\childdocforward}
The command |\childdocforward| redirects processing to
another source file:
%
\begin{center}
\begin{tabular}{l}
|\input{childdoc.def}|\\
|\childdocforward[|\textit{main}|]{|\textit{dest}|}|\\
\end{tabular}
\end{center}
%
The argument \textit{dest} is the destination file
(without extension).
It should be the main file or one of the child files.
Note that further \textsf{childdoc} directives
such as |\childdocof| and |\childdocforward|
in the indicated file will be processed in this form.
The optional argument \textit{main}
passes on directly to the main file \textit{main}
while pretending to compile the child \textit{dest}.
This form behaves as if \textit{dest}
issues |\childdocof{|\textit{main}|}| right away,
and no further \textsf{childdoc} directives will be processed.

%%%%%%%%%%%%%%%%%%%%%%%%%%%%%%%%%%%%%%%%
\DescribeMacro{\...prefix}
In the alternative form |\childdocforwardprefix|,
%
\begin{center}
\begin{tabular}{l}
|\input{childdoc.def}|\\
|\childdocforwardprefix[|\textit{main}|]{|\textit{prefix}|}{|\textit{dest}|}|
\end{tabular}
\end{center}
%
the destination file is determined by a pattern
depending on the current file:
To make this work, the current file must be called
`{\textit{prefix}\hspace{0.2em}\textit{suffix}}'
with \textit{prefix} matching precisely the argument.
Processing is then passed on to the file
`{\textit{dest}\hspace{0.2em}\textit{suffix}}'.
Surely, the same effect is achieved by
directly specifying the
argument `{\textit{dest}\hspace{0.2em}\textit{suffix}}'
in the first form.
However, that requires to set up a different file
for each child. With the alternative form of the command
all these files can have exactly the same content
which simplifies setting them up and maintaining them.

For example, the following file |draft.tex|
with a compilation flag |\version| as described in \secref{sec:flags}
compiles the main document as a draft:
%
\begin{center}
\begin{tabular}{l}
|\def\version{draft}|\\
|\input{childdoc.def}|\\
|\childdocforward{|\textit{main}|}|
\end{tabular}
\end{center}
%
Likewise, the following files |final|\textit{nn}|.tex|
compile the final version of the child document
|child|\textit{nn}|.tex|:
%
\begin{center}
\begin{tabular}{l}
|\def\version{final}|\\
|\input{childdoc.def}|\\
|\childdocforwardprefix{final}{child}|
\end{tabular}
\end{center}
%

Note that when several versions of a main file and/or of each child file
are to be generated, it may be convenient to set up a |Makefile| or
shell script to automatise the process.

%%%%%%%%%%%%%%%%%%%%%%%%%%%%%%%%%%%%%%%%%%%%%%%%%%%%%%%%%%%%%%%%%%%%%%%%%%%%%%%%
\subsection{Command Line Processing}
\label{sec:commandline}

The effect of redirection files can also be achieved by invoking
the \LaTeX{} compiler with a more elaborate command line.
Most conveniently this should be done as part
of a shell script or a |Makefile|.

When using \textsf{childdoc} in the main file, the following
command lines effectively perform a redirection
(note that depending on the shell being used,
backslashes may have to be doubled: `|\|' $\to$ `|\\|'):
%
\begin{center}
|... -jobname "|\textit{target}|" |\\|"|[\textit{flags}]%
|\input{childdoc.def}\childdocforward[|\textit{main}|]{|\textit{dest}|}"|
\end{center}
%
Here \textit{target} is the name of the output file,
\textit{main} is the name of the main file
and \textit{dest} is the name of the main or child file to be processed
(all filenames without extensions).
The optional argument \textit{main} can be omitted
if \textit{main} matches \textit{dest}.
Optionally, compilation \textit{flags} can be defined via |\def| commands.
This command line makes the \TeX{} engine believe
it is compiling the file \textit{target}
whose content is specified as the latter parameter.
The provided code then forwards the processing to
\textit{main} or \textit{dest} as described in \secref{sec:forward}.

%%%%%%%%%%%%%%%%%%%%%%%%%%%%%%%%%%%%%%%%%%%%%%%%%%%%%%%%%%%%%%%%%%%%%%%%%%%%%%%%
\subsection{Include by Input}
\label{sec:input}

Including child documents by |\include| has some restrictions by design.
Most notably, the content of a child document always occupies
its own set of pages; pages cannot be shared between child documents.
Usually, this behaviour makes perfect sense
because each child document contain an essential part of the document.
However, in some situations it may be desirable to compose
a document from a collection of parts
without having mandatory page breaks between then.
For this case, the package
provides a mechanism to include parts
by |\input| which can also be processed individually.
However, by construction this mechanism
requires manual handling of the content to be output.

%%%%%%%%%%%%%%%%%%%%%%%%%%%%%%%%%%%%%%%%
\DescribeMacro{\ifchilddocmanual}
The main file should be prepared as usual, see \secref{sec:include}.
However, the document body must make a distinction
between processing of an individual part and of the main document, e.g.:
%
\begin{center}
\begin{tabular}{l}
|\ifchilddocmanual|\\
|\input{\childdocname}|\\
|\||else|\\
\textit{document body with }|\input{|\textit{part}|}|\\
|\||fi|
\end{tabular}
\end{center}
%
The conditional |\ifchilddocmanual| is true whenever
a part to be included by |\input| is being compiled,
and the name of the part is stored in |\childdocname|.

%%%%%%%%%%%%%%%%%%%%%%%%%%%%%%%%%%%%%%%%
\DescribeMacro{\childdocby}
Each part to be included by |\input| should start with:
%
\begin{center}
\begin{tabular}{l}
|\input{childdoc.def}|\\
|\childdocby{|\textit{main}|}|\\
\end{tabular}
\end{center}
%
The directive |\childdocby| is similar to |\childdocof|
described in \secref{sec:include},
but the subsequent selection of content must be done manually.
To that end, both |\ifchilddoc| and |\ifchilddocmanual|
will be true upon processing of a part,
and the name of the part is stored in |\childdocname|.
Note that |\jobname| will be set to the filename of the current part
so that each part receives an individual |.aux| file
that does not interfere with the |.aux| file(s) of the main document.
This behaviour can be altered by the alternative form
|\childdocby[*]{|\textit{main}|}| (with a non-empty optional argument)
which uses the |.aux| file of the main document
by setting |\jobname| to \textit{main}.

%%%%%%%%%%%%%%%%%%%%%%%%%%%%%%%%%%%%%%%%%%%%%%%%%%%%%%%%%%%%%%%%%%%%%%%%%%%%%%%%
\subsection{Driver Development}
\label{sec:driver}

The \textsf{childdoc} mechanism can also be use for the development
of definition files such as \LaTeX{} styles or classes.
This case differs from the above setup with multiple parts
included by |\include| in that no |\includeonly| should be invoked.
This can be achieved by starting the include file
(before |\ProvidesPackage|) with:
%
\begin{center}
\begin{tabular}{l}
|\input{childdoc.def}|\\
|\childdocforward{|\textit{main}|}|\\
\end{tabular}
\end{center}
%
or alternatively with:
%
\begin{center}
\begin{tabular}{l}
|\input{childdoc.def}|\\
|\childdocby{|\textit{main}|}|\\
\end{tabular}
\end{center}
%
Both forms have slightly different effects as described above.
The main file is prepared as usual, see \secref{sec:include}.

%%%%%%%%%%%%%%%%%%%%%%%%%%%%%%%%%%%%%%%%%%%%%%%%%%%%%%%%%%%%%%%%%%%%%%%%%%%%%%%%
\subsection{Legacy Detection}
\label{sec:detection}

The directive |\childdocmain| in the main file can detect
whether the complete document or merely a child is to be compiled
even without using the directive |\childdocof|.
This method is deprecated because it is less robust
and there is no compelling reason to use it;
it is merely provided for backward compatibility
and it may be removed in future versions.

If the detection mechanism is to be used,
it is mandatory to correctly specify
the filename of the main file as the argument of |\childdocmain|:
%
\begin{center}
\begin{tabular}{l}
|\input{childdoc.def}|\\
|\childdocmain{|\textit{main}|}|\\
\end{tabular}
\end{center}
%
If |\jobname| does not match the argument \textit{main} of |\childdocmain|,
it is assumed that |\jobname| points to the child file to be compiled.
When using |\childdocmain| with the main file specified as argument,
it suffices to start a child file
with just |\input{|\textit{main}|}|
without loading of the package and using |\childdocof|.
If instead all processing is done
with the appropriate \textsf{childdoc} directives,
the argument of \textit{main} of |\childdocmain| can be empty.

An alternative version of the command line processing described
in \secref{sec:commandline} using the detection mechanism reads:
%
\begin{center}
|... -jobname "|\textit{target}|" "|[\textit{flags}]%
[|\def\jobname{|\textit{dest}|}|]|\input{|\textit{main}|}"|
\end{center}

%%%%%%%%%%%%%%%%%%%%%%%%%%%%%%%%%%%%%%%%%%%%%%%%%%%%%%%%%%%%%%%%%%%%%%%%%%%%%%%%
\subsection{Manual Code}
\label{sec:manual}

In case one cannot be certain whether the definitions file |childdoc.def|
is installed on the target \TeX{} distribution
and one prefers not to ship it,
it is conceivable to paste a few relevant commands into the sources.

To that end, drop all statements |\input{childdoc.def}|
and perform the replacements as outlined below.
Instead of |\childdocmain{|\textit{main}|}| add the following code
to the top of the main file:
%
\begin{center}
\begin{tabular}{l}
|\||ifdefined\childdocname\endinput\||fi\newif\ifchilddoc|\\
|\edef\childdocname{\scantokens\expandafter{\jobname\noexpand}}|\\
|\def\childdocmain{|\textit{main}|}\||ifx\childdocmain\childdocname\||else|\\
|\childdoctrue\includeonly{\childdocname}\let\jobname\childdocmain\||fi|\\
\end{tabular}
\end{center}
%
Instead of |\childdocof{|\textit{main}|}| just include the main file
at the top of each child file:
%
\begin{center}
|\input{|\textit{main}|}|
\end{center}
%
A simple redirection |\childdocforward{|\textit{dest}|}| is achieved by:
%
\begin{center}
|\def\jobname{|\textit{dest}|}\input{\jobname}|
\end{center}
%
The redirection with prefix
|\childdocforwardprefix[|\textit{prefix}|]{|\textit{dest}|}|
is accomplished by:
%
\begin{center}
\begin{tabular}{l}
|{\edef\jobname{\scantokens\expandafter{\jobname\noexpand}}|\\
|\def\redirectjob |\textit{prefix}|#1~~~{\gdef\jobname{|\textit{dest}|#1}}|\\
|\expandafter\redirectjob\jobname~~~}\input{\jobname}|
\end{tabular}
\end{center}

In an alternative approach,
child documents can be compiled by a specific command line
without additional code or specific definitions:
%
\begin{center}
|... -jobname "|\textit{target}|" "|[\textit{flags}]%
|\includeonly{|\textit{dest}|}\input{|\textit{main}|}"|
\end{center}
%

%%%%%%%%%%%%%%%%%%%%%%%%%%%%%%%%%%%%%%%%%%%%%%%%%%%%%%%%%%%%%%%%%%%%%%%%%%%%%%%%
%%%%%%%%%%%%%%%%%%%%%%%%%%%%%%%%%%%%%%%%%%%%%%%%%%%%%%%%%%%%%%%%%%%%%%%%%%%%%%%%
\section{Information}

%%%%%%%%%%%%%%%%%%%%%%%%%%%%%%%%%%%%%%%%%%%%%%%%%%%%%%%%%%%%%%%%%%%%%%%%%%%%%%%%
\subsection{Copyright}

Copyright \copyright{} 2017--2018 Niklas Beisert

This work may be distributed and/or modified under the
conditions of the \LaTeX{} Project Public License, either version 1.3
of this license or (at your option) any later version.
The latest version of this license is in
  \url{http://www.latex-project.org/lppl.txt}
and version 1.3 or later is part of all distributions of \LaTeX{}
version 2005/12/01 or later.

This work has the LPPL maintenance status `maintained'.

The Current Maintainer of this work is Niklas Beisert.

This work consists of the files |README.txt|, |childdoc.ins| and |childdoc.dtx|
as well as the derived files |childdoc.def|, |cdocsamp.tex|
with |cdocsch1.tex|, |cdocsch2.tex|, |cdocspt3.tex|, |cdocspt4.tex|,
|cdocsdrf.tex|, |cdocsfn1.tex|, |cdocsfn2.tex|
as well as |childdoc.pdf|.

%%%%%%%%%%%%%%%%%%%%%%%%%%%%%%%%%%%%%%%%%%%%%%%%%%%%%%%%%%%%%%%%%%%%%%%%%%%%%%%%
\subsection{Files and Installation}

The package consists of the files:
%
\begin{center}
\begin{tabular}{ll}
    |README.txt|   & readme file \\
    |childdoc.ins| & installation file \\
    |childdoc.dtx| & source file \\
    |childdoc.def| & definition file \\
    |cdocsamp.tex| & sample main file \\
    |cdocsch1.tex| & sample include file \\
    |cdocsch2.tex| & sample include file \\
    |cdocspt3.tex| & sample part file \\
    |cdocspt4.tex| & sample part file \\
    |cdocsdrf.tex| & sample redirection file \\
    |cdocsfn1.tex| & sample redirection file \\
    |cdocsfn2.tex| & sample redirection file \\
    |childdoc.pdf| & manual
\end{tabular}
\end{center}
%
The distribution consists of the files
|README.txt|, |childdoc.ins| and |childdoc.dtx|.
%
\begin{itemize}
\item
Run (pdf)\LaTeX{} on |childdoc.dtx|
to compile the manual |childdoc.pdf| (this file).
\item
Run \LaTeX{} on |childdoc.ins| to create the definitions file |childdoc.def|
and the sample |cdocsamp.tex| with include files
|cdocsch1.tex|, |cdocsch2.tex|, |cdocspt3.tex|, |cdocspt4.tex|,
|cdocsdrf.tex|, |cdocsfn1.tex|, |cdocsfn2.tex|.
Then copy the file |childdoc.def| to an appropriate directory of your \LaTeX{}
distribution, e.g.\ \textit{texmf-root}|/tex/latex/childdoc|.
\end{itemize}

%%%%%%%%%%%%%%%%%%%%%%%%%%%%%%%%%%%%%%%%%%%%%%%%%%%%%%%%%%%%%%%%%%%%%%%%%%%%%%%%
\subsection{Related CTAN Packages}

There are several other packages which offer a similar functionality:
%
\begin{itemize}
\item
The packages
\href{http://ctan.org/pkg/docmute}{\textsf{docmute}},
\href{http://ctan.org/pkg/includex}{\textsf{includex}} and
\href{http://ctan.org/pkg/standalone}{\textsf{standalone}}
provide commands to include only the document body of
a child file thus allowing both files to be compiled individually.
\item
The packages \href{http://ctan.org/pkg/subdocs}{\textsf{subdocs}}
and \href{http://ctan.org/pkg/subfiles}{\textsf{subfiles}}
provide structures in which the main and child documents can be
encapsulated and allowing them to be compiled individually.
The inclusion mechanism is different from the conventional |\include|.
\item
The package \href{http://ctan.org/pkg/combine}{\textsf{combine}}
is an elaborate solution to combine several documents into one.
\end{itemize}
%
See also the CTAN topic \href{http://ctan.org/topic/subdocs}{\textsf{subdocs}}
for further related packages.
The present package differs from the above solutions in that
a document structure constructed with the conventional |\include| mechanism
just needs two extra commands at the top of every file
such that all constituent files can be compiled individually.

%%%%%%%%%%%%%%%%%%%%%%%%%%%%%%%%%%%%%%%%%%%%%%%%%%%%%%%%%%%%%%%%%%%%%%%%%%%%%%%%
%\subsection{Feature Suggestions}
%
%The following is a list of features which may be useful for future
%versions of this package:
%%
%\begin{itemize}
%\item
%\ldots
%\end{itemize}

%%%%%%%%%%%%%%%%%%%%%%%%%%%%%%%%%%%%%%%%%%%%%%%%%%%%%%%%%%%%%%%%%%%%%%%%%%%%%%%%
\subsection{Revision History}

%%%%%%%%%%%%%%%%%%%%%%%%%%%%%%%%%%%%%%%%
\paragraph{v2.0:} 2018/12/30

\begin{itemize}
\item
immediate forward processing
\item
added |\childdocby| mechanism
\item
manual restructured
\end{itemize}

%%%%%%%%%%%%%%%%%%%%%%%%%%%%%%%%%%%%%%%%
\paragraph{v1.6:} 2018/01/17

\begin{itemize}
\item
application for development of include files
\item
corrections to manual
\end{itemize}

%%%%%%%%%%%%%%%%%%%%%%%%%%%%%%%%%%%%%%%%
\paragraph{v1.5:} 2017/05/21

\begin{itemize}
\item
more complete structuring introduced
\item
|\childdocof| introduced
\item
|\childdoc| renamed to |\childdocmain|
\item
|\childredirect| renamed to |\childdocforward| and |\childdocforwardprefix|
and functionality expanded
\end{itemize}

%%%%%%%%%%%%%%%%%%%%%%%%%%%%%%%%%%%%%%%%
\paragraph{v1.0:} 2017/04/27

\begin{itemize}
\item
manual and install package
\item
first version published on CTAN
\end{itemize}

%%%%%%%%%%%%%%%%%%%%%%%%%%%%%%%%%%%%%%%%
\paragraph{v0.6:} 2017/04/26

\begin{itemize}
\item
redirection mechanism added
\end{itemize}

%%%%%%%%%%%%%%%%%%%%%%%%%%%%%%%%%%%%%%%%
\paragraph{v0.5:} 2017/04/26

\begin{itemize}
\item
functionality in definition file
\end{itemize}


%%%%%%%%%%%%%%%%%%%%%%%%%%%%%%%%%%%%%%%%%%%%%%%%%%%%%%%%%%%%%%%%%%%%%%%%%%%%%%%%
%%%%%%%%%%%%%%%%%%%%%%%%%%%%%%%%%%%%%%%%%%%%%%%%%%%%%%%%%%%%%%%%%%%%%%%%%%%%%%%%
%%%%%%%%%%%%%%%%%%%%%%%%%%%%%%%%%%%%%%%%%%%%%%%%%%%%%%%%%%%%%%%%%%%%%%%%%%%%%%%%
\appendix

\settowidth\MacroIndent{\rmfamily\scriptsize 000\ }

 \DocInput{childdoc.dtx}

\end{document}
%</driver>
% \fi
%
% %%%%%%%%%%%%%%%%%%%%%%%%%%%%%%%%%%%%%%%%%%%%%%%%%%%%%%%%%%%%%%%%%%%%%%%%%%%%%%
% %%%%%%%%%%%%%%%%%%%%%%%%%%%%%%%%%%%%%%%%%%%%%%%%%%%%%%%%%%%%%%%%%%%%%%%%%%%%%%
% \section{Sample}
%\iffalse
%<*samplemain>
%\fi
%
% The following presents a sample document
% with two chapters, two parts, a title page,
% a compile flag as well as three forwarding files to set the flag.
% It consists of eight |.tex| files:
% \begin{center}
% \begin{tabular}{ll}
% |cdocsamp.tex|&main file\\
% |cdocsch1.tex|&include file for chapter 1\\
% |cdocsch2.tex|&include file for chapter 2\\
% |cdocspt3.tex|&include file for part 3\\
% |cdocspt4.tex|&include file for part 4\\
% |cdocsdrf.tex|&forwarding file for main file in draft mode\\
% |cdocsfi1.tex|&forwarding file for final version of chapter 1\\
% |cdocsfi2.tex|&forwarding file for final version of chapter 2\\
% \end{tabular}
% \end{center}
% Each of the eight files can be compiled directly by the \LaTeX{} compiler.
%
% %%%%%%%%%%%%%%%%%%%%%%%%%%%%%%%%%%%%%%
% \paragraph{Main File.}
%
% The main file is called |cdocsamp.tex|.
%
% Load the \textsf{childdoc} definitions and
% declare the filename for the main document:
%    \begin{macrocode}
\input{childdoc.def}
\childdocmain{}
%    \end{macrocode}

% Optional override for |\version| flag:
%    \begin{macrocode}
%%\ifchilddoc\else\providecommand{\version}{draft}\fi
%    \end{macrocode}

% Define the default values for the |\version| flag
% (|final| for the main file and |draft| for childs):
%    \begin{macrocode}
\ifchilddoc
\providecommand{\version}{draft}
\else
\providecommand{\version}{final}
\fi
%    \end{macrocode}

% Load the standard document class:
%    \begin{macrocode}
\documentclass[12pt]{article}
%    \end{macrocode}

% Start the document body:
%    \begin{macrocode}
\begin{document}
%    \end{macrocode}

% Declare a title page.
% Print title, part of document being processed and version flag:
%    \begin{macrocode}
\addtocounter{page}{-1}
\begin{center}
{\LARGE\bfseries{}childdoc example\par}
\vspace{1cm}
\ifchilddoc
\ifchilddocmanual part\else chapter\fi:
`\childdocname' of `\childdocjob'\par
\else
main document: `\childdocjob'\par
\fi
version: \version\par
\end{center}
\newpage
%    \end{macrocode}

% Manually include selected file,
% otherwise process as usual:
%    \begin{macrocode}
\ifchilddocmanual
\section*{part `\childdocname'}
\input{\childdocname}
\else
%    \end{macrocode}

% Include the two chapters:
%    \begin{macrocode}
\include{cdocsch1}
\include{cdocsch2}
%    \end{macrocode}

% Include the two parts unless only chapters should be displayed:
%    \begin{macrocode}
\ifchilddoc\else
\section{part three}
\input{cdocspt3}
\section{part four}
\input{cdocspt4}
\fi
%    \end{macrocode}

% Process as usual until here:
%    \begin{macrocode}
\fi
%    \end{macrocode}

% End of document body:
%    \begin{macrocode}
\end{document}
%    \end{macrocode}
%\iffalse
%</samplemain>
%\fi
%
% %%%%%%%%%%%%%%%%%%%%%%%%%%%%%%%%%%%%%%
% \paragraph{Chapter Include Files.}
%
% The include files are called |cdocsch1.tex| and |cdocsch2.tex|.
%
%\iffalse
%<*samplechap1|samplechap2>
%\fi

% Optional override for |\version| flag:
%    \begin{macrocode}
%%\providecommand{\version}{final}
%    \end{macrocode}

% Include the main document:
%    \begin{macrocode}
\input{childdoc.def}
\childdocof{cdocsamp}
%    \end{macrocode}

%\iffalse
%</samplechap1|samplechap2>
%\fi
%
%\iffalse
%<*samplechap1>
%\fi
% Some text for chapter 1:
%    \begin{macrocode}
\section{one}
some text in chapter one
%    \end{macrocode}

%\iffalse
%</samplechap1>
%\fi
% Some text for chapter 2:
%\iffalse
%<*samplechap2>
%\fi
%    \begin{macrocode}
\section{two}
more text in chapter two
%    \end{macrocode}

%\iffalse
%</samplechap2>
%\fi
%
% %%%%%%%%%%%%%%%%%%%%%%%%%%%%%%%%%%%%%%
% \paragraph{Part Include Files.}
%
% The include files are called |cdocspt3.tex| and |cdocspt4.tex|.
%
%\iffalse
%<*samplepart3|samplepart4>
%\fi

% Optional override for |\version| flag:
%    \begin{macrocode}
%%\providecommand{\version}{final}
%    \end{macrocode}

% Include the main document:
%    \begin{macrocode}
\input{childdoc.def}
\childdocby{cdocsamp}
%    \end{macrocode}

%\iffalse
%</samplepart3|samplepart4>
%\fi
%
%\iffalse
%<*samplepart3>
%\fi
% Some text for part 3:
%    \begin{macrocode}
some text in part three
%    \end{macrocode}

%\iffalse
%</samplepart3>
%\fi
% Some text for part 4:
%\iffalse
%<*samplepart4>
%\fi
%    \begin{macrocode}
more text in part four
%    \end{macrocode}

%\iffalse
%</samplepart4>
%\fi
%
% %%%%%%%%%%%%%%%%%%%%%%%%%%%%%%%%%%%%%%
% \paragraph{Forwarding for a Complete Draft.}
%
% The following forwarding file |cdocsdrf.tex|
% compiles the main document in draft mode:
%\iffalse
%<*sampledraft>
%\fi
%    \begin{macrocode}
\def\version{draft}
\input{childdoc.def}
\childdocforward{cdocsamp}
%    \end{macrocode}

%\iffalse
%</sampledraft>
%\fi
%
% %%%%%%%%%%%%%%%%%%%%%%%%%%%%%%%%%%%%%%
% \paragraph{Forwarding for Final Version of the Chapters.}
%
% The following forwarding files |cdocsfn1.tex| and |cdocsfn2.tex|
% (with identical content)
% compile the final versions of the child documents
% |cdocsch1.tex| and |cdocsch2.tex|, respectively:
%\iffalse
%<*samplefinal>
%\fi
%    \begin{macrocode}
\def\version{final}
\input{childdoc.def}
\childdocforwardprefix[cdocsamp]{cdocsfn}{cdocsch}
%    \end{macrocode}

%\iffalse
%</samplefinal>
%\fi
%
% %%%%%%%%%%%%%%%%%%%%%%%%%%%%%%%%%%%%%%
% \paragraph{Command Line Processing.}
%
% The following three command lines generate the output files
% |cdocscld|, |cdocscl1| and |cdocscl2|
% which should be identical to
% |cdocsdrf|, |cdocsch1| and |cdocsfn2|, respectively:
% \begin{center}
% \begin{tabular}{l}
% |latex -jobname cdocscld \|\\
% |  "\def\version{draft}\input{childdoc.def}\childdocforward{cdocsamp}"|\\
% |latex -jobname cdocscl1 \|\\
% |  "\input{childdoc.def}\childdocforward[cdocsamp]{cdocsch1}"|\\
% |latex -jobname cdocscl2 \|\\
% |  "\def\version{final}\input{childdoc.def}\childdocforward{cdocsch2}"|
% \end{tabular}
% \end{center}
% Note that the trailing backslash on each first line
% merely continues the input to the second line
% (for convenient cut ant paste).
% Furthermore, the command |latex| can be replaced by any
% of its alternative versions such as |pdflatex|.
%
% %%%%%%%%%%%%%%%%%%%%%%%%%%%%%%%%%%%%%%%%%%%%%%%%%%%%%%%%%%%%%%%%%%%%%%%%%%%%%%
% %%%%%%%%%%%%%%%%%%%%%%%%%%%%%%%%%%%%%%%%%%%%%%%%%%%%%%%%%%%%%%%%%%%%%%%%%%%%%%
% \section{Implementation}
%\iffalse
%<*package>
%\fi
%
% This section describes the definitions file |childdoc.def|.

% The definitions cannot be loaded using |\usepackage| or |\RequirePackage|
% which has a mechanism to prevent loading a style file more than once.
% When loading the definitions by means of |\input|
% multiple instances have to be prevented manually:
%\iffalse
%This code needs to be before the `\ProvidesFile' directive
%which is defined at the beginning of this file.
%Therefore it is also placed there and commented out here.
%</package>
%<*discard>
%\fi
%    \begin{macrocode}
\ifdefined\childdocmain\endinput\fi
%    \end{macrocode}
%\iffalse
%</discard>
%<*package>
%\fi
%
% \macro{\ifchilddoc}
% \macro{\ifchilddocmanual}
% The conditional |\ifchilddoc| tells whether a
% child (true) or main (false) document is being compiled.
% The conditional |\ifchilddocmanual| tells whether
% the |\includeonly| mechanism is used (false) or
% the selection of child files must be performed manually (true).
% The definitions initialise to false:
%    \begin{macrocode}
\newif\ifchilddoc
\newif\ifchilddocmanual
%    \end{macrocode}

% \macro{\childdocname}
% \macro{\childdocjob}
% The macro |\childdocname| stores the name of the main document
% to be compiled. The macro |\childdocjob| stores the name of
% the document on which the \LaTeX{} compiler was originally invoked.
% The content of |\jobname| cannot be compared
% to filenames specified in the source due to different catcodes.
% The following code rescans |\jobname|, stores the result
% in |\childdocname| and saves a copy in |\childdocjob|:
%    \begin{macrocode}
\edef\childdocname{\scantokens\expandafter{\jobname\noexpand}}
\let\childdocjob\childdocname
%    \end{macrocode}

% \macro{\childdocdisable}
% The macro |\childdocdisable| prevents the main file
% from being processed more than once.
% At this stage, the main document command |\childdocmain|
% is assumed to be called once again where it should do nothing.
% Any subsequent call to it should prevent
% a secondary processing of the main document
% It overwrites the forwarding commands
% |\childdocof| and |\childdocforward|
% with empty macros to prevent further inclusions of the main document:
%    \begin{macrocode}
\newcommand{\childdocdisable}
{
  \renewcommand{\childdocmain}[1]{\renewcommand{\childdocmain}[1]{\endinput}}
  \renewcommand{\childdocof}[1]{}
  \renewcommand{\childdocby}[2][]{}
  \renewcommand{\childdocforward}[2][]{}
  \renewcommand{\childdocdisable}{}
}
%    \end{macrocode}

% \macro{\childdocmain}
% The macro |\childdocmain| is to be called at the top of the main file
% with nothing or the main filename (without extension) as argument.
% First, it breaks loops.
% If the argument is not empty and does not match |\childdocname|
% (which is set by the first inclusion of |childdoc.def|),
% |\ifchilddoc| is set to true, |\includeonly| is applied to the child file
% and |\jobname| is set to the main file
% (for proper handling of |.aux| files):
%    \begin{macrocode}
\newcommand{\childdocmain}[1]
{
  \childdocdisable\childdocmain{}
  \if?#1?\else
    \begingroup
      \def\childdoctmp{#1}
      \ifx\childdoctmp\childdocname
        \def\childdoctmp{}
      \else
        \def\childdoctmp
        {
          \childdoctrue
          \includeonly{\childdocname}
          \def\childdocjob{#1}
          \def\jobname{#1}
        }
      \fi
      \expandafter
    \endgroup
    \childdoctmp
  \fi
}
%    \end{macrocode}

% \macro{\childdocof}
% The command |\childdocof| redirects
% compilation to the main file |#1|.
%    \begin{macrocode}
\newcommand{\childdocof}[1]
{
  \childdocdisable
  \childdoctrue
  \includeonly{\childdocname}
  \def\jobname{#1}
  \def\childdocjob{#1}
  \input{#1}
}
%    \end{macrocode}

% \macro{\childdocby}
% The command |\childdocby| ....
%    \begin{macrocode}
\newcommand{\childdocby}[2][]
{
  \childdocdisable
  \childdoctrue
  \childdocmanualtrue
  \if?#1?\else
    \def\jobname{#2}
  \fi
  \def\childdocjob{#2}
  \input{#2}
  \endinput
}
%    \end{macrocode}

% \macro{\childdocforward}
% The command |\childdocforward| redirects
% compilation to the main file or
% (if the optional argument is given) a child file.
% Parameters are set as if the main file
% or a child file starting with |\childdocof| was compiled.
% Then compilation is handed over to the main file:
%    \begin{macrocode}
\newcommand{\childdocforward}[2][]
{
  \begingroup
    \if?#1?
      \def\childdoctmp
      {
        \def\childdocname{#2}
        \def\childdocjob{#2}
        \def\jobname{#2}
        \input{#2}
        \endinput
      }
    \else
      \def\childdoctmp
      {
        \childdocdisable
        \def\childdocname{#2}
        \childdoctrue
        \includeonly{#2}
        \def\childdocjob{#1}
        \def\jobname{#1}
        \input{#1}
        \endinput
      }
    \fi
    \expandafter
  \endgroup
  \childdoctmp
}
%    \end{macrocode}

% \macro{\childdocforwardprefix}
% The command |\childdocforwardprefix| redirects
% compilation to the main or a child file by means of a pattern.
% The prefix |#1| in the current filename is replaced by |#2|
% and the suffix of the current filename is kept
% (it is assumed that the filename does not contain the substring `|~~~|'
% which is used as a delimiter).
% Compilation is handed over to the new file by |\childdocforward|:
%    \begin{macrocode}
\newcommand{\childdocforwardprefix}[3][]
{
  \begingroup
    \def\childdocextract #2##1~~~{\def\childdoctmp{\childdocforward[#1]{#3##1}}}
    \expandafter\childdocextract\childdocname~~~
    \expandafter
  \endgroup
  \childdoctmp
}
%    \end{macrocode}

% \macro{\childdoc}
% The deprecated macro |\childdoc| is a legacy version of |\childdocmain|:
%    \begin{macrocode}
\newcommand{\childdoc}{\childdocmain}
%    \end{macrocode}

% \macro{\childdocredirect}
% The deprecated macro |\childdocredirect| is a legacy version
% of |\childdocforward| and |\childdocforwardprefix|:
%    \begin{macrocode}
\newcommand{\childdocredirect}[2][]
{
  \begingroup
    \if?#1?
      \def\childdoctmp{\childdocforward{#2}}
    \else
      \def\childdoctmp{\childdocforwardprefix{#1}{#2}}
    \fi
    \expandafter
  \endgroup
  \childdoctmp
}
%    \end{macrocode}

%\iffalse
%</package>
%\fi
%
\endinput
|\\
|\childdocforward{|\textit{main}|}|
\end{tabular}
\end{center}
%
Likewise, the following files |final|\textit{nn}|.tex|
compile the final version of the child document
|child|\textit{nn}|.tex|:
%
\begin{center}
\begin{tabular}{l}
|\def\version{final}|\\
|% \iffalse
%
% childdoc.dtx Copyright (C) 2017-2018 Niklas Beisert
%
% This work may be distributed and/or modified under the
% conditions of the LaTeX Project Public License, either version 1.3
% of this license or (at your option) any later version.
% The latest version of this license is in
%   http://www.latex-project.org/lppl.txt
% and version 1.3 or later is part of all distributions of LaTeX
% version 2005/12/01 or later.
%
% This work has the LPPL maintenance status `maintained'.
%
% The Current Maintainer of this work is Niklas Beisert.
%
% This work consists of the files childdoc.dtx and childdoc.ins
% and the derived files childdoc.def and cdocsamp.tex with
% cdocsch1.tex, cdocsch2.tex, cdocsdrf.tex, cdocsfn1.tex, cdocsfn2.tex.
%
%<package>\ifdefined\childdocmain\endinput\fi
%<package>\ProvidesFile{childdoc.def}[2018/12/30 v2.0 child document driver]
%<samplemain>\ProvidesFile{cdocsamp.tex}[2018/12/30 v2.0 sample for childdoc]
%<*driver>
%\ProvidesFile{childdoc.drv}[2018/12/30 v2.0 childdoc reference manual file]
\PassOptionsToClass{10pt,a4paper}{article}
\documentclass{ltxdoc}

\usepackage[margin=35mm]{geometry}
\usepackage{hyperref}
\usepackage{hyperxmp}
\usepackage[usenames]{color}

\hypersetup{colorlinks=true}
\hypersetup{pdfstartview=FitH}
\hypersetup{pdfpagemode=UseNone}
\hypersetup{pdfsource={}}
\hypersetup{pdflang={en-UK}}
\hypersetup{pdfcopyright={Copyright 2017-2018 Niklas Beisert.
  This work may be distributed and/or modified under the
  conditions of the LaTeX Project Public License, either version 1.3
  of this license or (at your option) any later version.}}
\hypersetup{pdflicenseurl={http://www.latex-project.org/lppl.txt}}
\hypersetup{pdfcontactaddress={ETH Zurich, ITP, HIT K,
  Wolfgang-Pauli-Strasse 27}}
\hypersetup{pdfcontactpostcode={8093}}
\hypersetup{pdfcontactcity={Zurich}}
\hypersetup{pdfcontactcountry={Switzerland}}
\hypersetup{pdfcontactemail={nbeisert@itp.phys.ethz.ch}}
\hypersetup{pdfcontacturl={http://people.phys.ethz.ch/\xmptilde nbeisert/}}

\newcommand{\secref}[1]{\hyperref[#1]{section \ref*{#1}}}

\parskip1ex
\parindent0pt
\let\olditemize\itemize
\def\itemize{\olditemize\parskip0pt}

\begin{document}

\title{The \textsf{childdoc} Package}
\hypersetup{pdftitle={The childdoc Package}}
\author{Niklas Beisert\\[2ex]
  Institut f\"ur Theoretische Physik\\
  Eidgen\"ossische Technische Hochschule Z\"urich\\
  Wolfgang-Pauli-Strasse 27, 8093 Z\"urich, Switzerland\\[1ex]
  \href{mailto:nbeisert@itp.phys.ethz.ch}
  {\texttt{nbeisert@itp.phys.ethz.ch}}}
\hypersetup{pdfauthor={Niklas Beisert}}
\hypersetup{pdfsubject={Manual for the LaTeX2e Package childdoc}}
\date{30 December 2018, \textsf{v2.0}}
\maketitle

\begin{abstract}\noindent
\textsf{childdoc} is a \LaTeXe{} package
that enables the direct compilation
of document sections included by |\include|
to individual files.
\end{abstract}

\begingroup
\parskip0ex
\tableofcontents
\endgroup

%%%%%%%%%%%%%%%%%%%%%%%%%%%%%%%%%%%%%%%%%%%%%%%%%%%%%%%%%%%%%%%%%%%%%%%%%%%%%%%%
%%%%%%%%%%%%%%%%%%%%%%%%%%%%%%%%%%%%%%%%%%%%%%%%%%%%%%%%%%%%%%%%%%%%%%%%%%%%%%%%
\section{Introduction}

\LaTeX{} provides a mechanism to structure a large document (such as a book)
into a main file and several child files (containing the chapters)
using the |\include| command.
This mechanism is beneficial for documents
which span hundreds of pages in order to
make the source file(s) more manageable.
Moreover, compilation can be restricted to
selected child files by means of the |\includeonly| command.
The latter feature can be used to reduce the compilation time while editing
(this was significantly more useful in the earlier days of \LaTeX{})
or to generate a smaller document which is easier to navigate.
Another application of |\includeonly| is to generate
documents consisting of selected parts of the complete document.

However, there are a few drawbacks of the plain |\include| mechanism:
\begin{itemize}
\item
The child files cannot be compiled on their own,
they can only be compiled via the main file.
A naive editing environment
(such as a text editor with an option
to have the current file processed by \LaTeX)
may require one to switch to the main file before compiling;
attempting to compile the child file produces errors.
\item
The main file must be modified (each time)
to adjust the |\includeonly| command
to the present needs. This easily leaves the main file in a messy state.
\item
The generated document will always carry the filename
of the main document. This is inconvenient if
several child files are to be compiled and
to be kept for distribution.
\end{itemize}

The present package provides a simple interface
to make child files individually compilable by \LaTeX{}.
Compiling a child file then has the same effect as compiling
the main file with an |\includeonly| command
to select the appropriate child.
Moreover the generated document will carry the name of the child
rather than the main file.
This resolves all three above issues.

This feature is meant to make the editing of books,
thesis documents and lecture notes somewhat more convenient.
However, the package can also be used efficiently for
composing a series of documents (such as exercise sheets)
which are typically distributed individually.
It then assists the author in generating the individual documents
(potentially in different versions)
as well as a document containing the collected series.
Another application is in developing style files
or other kinds of included material
where compilation of the style file could redirect
to a sample or test file.

%%%%%%%%%%%%%%%%%%%%%%%%%%%%%%%%%%%%%%%%%%%%%%%%%%%%%%%%%%%%%%%%%%%%%%%%%%%%%%%%
%%%%%%%%%%%%%%%%%%%%%%%%%%%%%%%%%%%%%%%%%%%%%%%%%%%%%%%%%%%%%%%%%%%%%%%%%%%%%%%%
\section{Usage}

First of all, the package \textsf{childdoc} is \emph{not} a standard
\LaTeXe{} |.sty| style file! Therefore it needs to be invoked in
a non-standard way.

%%%%%%%%%%%%%%%%%%%%%%%%%%%%%%%%%%%%%%%%%%%%%%%%%%%%%%%%%%%%%%%%%%%%%%%%%%%%%%%%
\subsection{Included Files}
\label{sec:include}

%%%%%%%%%%%%%%%%%%%%%%%%%%%%%%%%%%%%%%%%
\DescribeMacro{\childdocmain}
To use the package, add the commands
\begin{center}
\begin{tabular}{l}
|\input{childdoc.def}|\\
|\childdocmain{}|\\
\end{tabular}
\end{center}
at the very top of the main \LaTeX{} file,
in particular \emph{before} the |\documentclass| statement!
The argument of |\childdocmain| should be left empty
(but it must be present).

%%%%%%%%%%%%%%%%%%%%%%%%%%%%%%%%%%%%%%%%
\DescribeMacro{\childdocof}
Furthermore, add the commands
\begin{center}
\begin{tabular}{l}
|\input{childdoc.def}|\\
|\childdocof{|\textit{main}|}|\\
\end{tabular}
\end{center}
at the top of every child file \textit{child}
which is included by |\include{|\textit{child}|}|
from within the main file
(or at least for those files to be compiled individually).
The argument \textit{main} must be the filename of the main file.

There are a couple of
considerations in setting up the main and child documents:

%%%%%%%%%%%%%%%%%%%%%%%%%%%%%%%%%%%%%%%%
\paragraph{Restrictions.}

Please note the following restrictions:
\begin{itemize}
\item
|\childdocmain| must be called with one argument \textit{main}
to ensure compatibility with earlier version of the package.
It must either be empty (|\childdocmain{}|)
or precisely match the filename of the main file in which it is specified.
See \secref{sec:detection} for further information.
\item
The filename \textit{main} must be specified without the |.tex| extension.
\item
The filename \textit{main} is case sensitive
(even in case-insensitive file systems)
due to internal string comparison.
\item
The argument \textit{main} should be fully expanded, it cannot be a macro.
\item
Subdirectories and special characters should be avoided in filenames.
\item
The command |\childdocmain{|\textit{main}|}| must be followed by a whitespace.
It should not be followed immediately by another command
or by a comment mark `|%|'.
This is because the \TeX{} parser reads the token immediately following
the argument of |\childdocmain| and puts it
at the beginning of every child section;
however, a white\-space is ignored.
\end{itemize}

%%%%%%%%%%%%%%%%%%%%%%%%%%%%%%%%%%%%%%%%
\paragraph{Content of Main File.}

It is advisable to place all content in the child files included by |\include|.
Any output contained in the main file will appear in all child documents
unless suppressed manually;
it cannot be suppressed automatically by the |\includeonly| directive
and thus should normally be avoided.
A method to include some content in the main file
by means of conditional processing is described in \secref{sec:conditional}.

%%%%%%%%%%%%%%%%%%%%%%%%%%%%%%%%%%%%%%%%
\paragraph{Page Numbering.}

When only a part of the document is compiled,
the appropriate numbering of pages
(as well as other status parameters)
is determined from the |.aux| files.
The latter contain information from previous passes.
However this information needs to propagate through
all intermediate child documents.
Therefore the page numbering in child documents may well
be inconsistent until the complete document is compiled at least once.

A useful (if unconventional) way to always ensure a consistent
page numbering is to restart the numbering in each child document
and denote the pages by `\textit{child}|.|\textit{page}'
where \textit{child} represents the chapter/section number of the child file.
This can be achieved by the command
|\numberwithin{page}{|\textit{child}|}|
of the \textsf{amsmath} package
where \textit{child} can be |chapter| or |section|
depending on the chosen structuring.
Alternatively, one can modify the macro |\thepage| appropriately
and reset the counter |page| at the start of each child file.

%%%%%%%%%%%%%%%%%%%%%%%%%%%%%%%%%%%%%%%%%%%%%%%%%%%%%%%%%%%%%%%%%%%%%%%%%%%%%%%%
\subsection{Conditional Processing}
\label{sec:conditional}

The package provides a mechanism to compile different versions
of a document. To customise the versions further some conditional processing
can come in handy to distinguish which version is being compiled.
The package provides two macros to describe the compilation context:

%%%%%%%%%%%%%%%%%%%%%%%%%%%%%%%%%%%%%%%%
\DescribeMacro{\ifchilddoc}
The conditional |\ifchilddoc| distinguishes between the compilation of
child documents and the main document:
%
\begin{center}
|\ifchilddoc |\textit{child-code}| |[|\||else |\textit{main-code}]| \||fi|
\end{center}

%%%%%%%%%%%%%%%%%%%%%%%%%%%%%%%%%%%%%%%%
\DescribeMacro{\childdocname}
\DescribeMacro{\childdocjob}
The macro |\childdocname| contains the filename (without extension)
of the main or child file being processed.
Note that |\childdocjob| will always contain the name of the main file.

%%%%%%%%%%%%%%%%%%%%%%%%%%%%%%%%%%%%%%%%
\paragraph{Title Page.}

Conditional processing can be used to include a title or banner page
in the main document when proper precautions are taken.
Importantly, the code in the main file should ensure that the page counter
(as well as other status parameters which are stored in the |.aux| files)
takes the same value after the conditional processing.
Otherwise the page numbers may take divergent values
depending on which part is compiled.

For example, a title page could be declared by:
%
\begin{center}
\begin{tabular}{l}
|\ifchilddoc\||else|\\
|\addtocounter{page}{-1}|\\
\textit{code for title page}\\
|\newpage|\\
|\||fi|
\end{tabular}
\end{center}
%
A banner page for the child documents can be generated by:
%
\begin{center}
\begin{tabular}{l}
|\ifchilddoc|\\
|\addtocounter{page}{-1}|\\
\textit{code for banner page}\\
|\newpage|\\
|\||fi|
\end{tabular}
\end{center}
%
Here one could write a message such as:
\begin{center}
|This is the part \childdocname{} of \childdocjob{}.|
\end{center}

%%%%%%%%%%%%%%%%%%%%%%%%%%%%%%%%%%%%%%%%%%%%%%%%%%%%%%%%%%%%%%%%%%%%%%%%%%%%%%%%
\subsection{Flags}
\label{sec:flags}

The package makes it easy to generate different versions
of the main or child documents.
To this end compilation flags can be defined
and assigned different default values.
They will be particularly useful in conjunction
with the forwarding mechanism described in \secref{sec:forward}.

For example, it may be useful to have a flag |\version|
which can be set to |draft| or |final|.
The document source will contain some conditional code
depending on the value of |\version|.
Suppose further, the flag should default to |final| for the main file
and to |draft| for child files
which is a natural assignment for editing the document.
This is achieved by placing the following code
in the preamble of the main document
(below the |\childdocmain| directive):
%
\begin{center}
\begin{tabular}{l}
|\ifchilddoc|\\
|\providecommand{\version}{draft}|\\
|\||else|\\
|\providecommand{\version}{final}|\\
|\||fi|
\end{tabular}
\end{center}
%
The definition by |\providecommand| makes sure
that previous definitions are not overwritten.
Further statements |\providecommand{\version}{...}|
can thus be added before the above code to override it.

For the main file, one might add a line
(between |\childdocmain| and the above block)
%
\begin{center}
|%\ifchilddoc\||else\providecommand{\version}{draft}\||fi|
\end{center}
%
which can be uncommented to produce a draft version.
Likewise one can add a line to the very top of a child file
(above the |\childdocof{|\textit{main}|}| directive)
%
\begin{center}
|%\providecommand{\version}{final}|
\end{center}
%
which can be uncommented to produce the final version of this child document.

%%%%%%%%%%%%%%%%%%%%%%%%%%%%%%%%%%%%%%%%%%%%%%%%%%%%%%%%%%%%%%%%%%%%%%%%%%%%%%%%
\subsection{Forwarding}
\label{sec:forward}

Different versions of the main or child documents
using compilation flags as described in \secref{sec:flags}
can be (permanently) stored in different files
for convenient compilation, viewing and distribution.
To this end, the package defines a command
to pass on compilation to a different file:

%%%%%%%%%%%%%%%%%%%%%%%%%%%%%%%%%%%%%%%%
\DescribeMacro{\childdocforward}
The command |\childdocforward| redirects processing to
another source file:
%
\begin{center}
\begin{tabular}{l}
|\input{childdoc.def}|\\
|\childdocforward[|\textit{main}|]{|\textit{dest}|}|\\
\end{tabular}
\end{center}
%
The argument \textit{dest} is the destination file
(without extension).
It should be the main file or one of the child files.
Note that further \textsf{childdoc} directives
such as |\childdocof| and |\childdocforward|
in the indicated file will be processed in this form.
The optional argument \textit{main}
passes on directly to the main file \textit{main}
while pretending to compile the child \textit{dest}.
This form behaves as if \textit{dest}
issues |\childdocof{|\textit{main}|}| right away,
and no further \textsf{childdoc} directives will be processed.

%%%%%%%%%%%%%%%%%%%%%%%%%%%%%%%%%%%%%%%%
\DescribeMacro{\...prefix}
In the alternative form |\childdocforwardprefix|,
%
\begin{center}
\begin{tabular}{l}
|\input{childdoc.def}|\\
|\childdocforwardprefix[|\textit{main}|]{|\textit{prefix}|}{|\textit{dest}|}|
\end{tabular}
\end{center}
%
the destination file is determined by a pattern
depending on the current file:
To make this work, the current file must be called
`{\textit{prefix}\hspace{0.2em}\textit{suffix}}'
with \textit{prefix} matching precisely the argument.
Processing is then passed on to the file
`{\textit{dest}\hspace{0.2em}\textit{suffix}}'.
Surely, the same effect is achieved by
directly specifying the
argument `{\textit{dest}\hspace{0.2em}\textit{suffix}}'
in the first form.
However, that requires to set up a different file
for each child. With the alternative form of the command
all these files can have exactly the same content
which simplifies setting them up and maintaining them.

For example, the following file |draft.tex|
with a compilation flag |\version| as described in \secref{sec:flags}
compiles the main document as a draft:
%
\begin{center}
\begin{tabular}{l}
|\def\version{draft}|\\
|\input{childdoc.def}|\\
|\childdocforward{|\textit{main}|}|
\end{tabular}
\end{center}
%
Likewise, the following files |final|\textit{nn}|.tex|
compile the final version of the child document
|child|\textit{nn}|.tex|:
%
\begin{center}
\begin{tabular}{l}
|\def\version{final}|\\
|\input{childdoc.def}|\\
|\childdocforwardprefix{final}{child}|
\end{tabular}
\end{center}
%

Note that when several versions of a main file and/or of each child file
are to be generated, it may be convenient to set up a |Makefile| or
shell script to automatise the process.

%%%%%%%%%%%%%%%%%%%%%%%%%%%%%%%%%%%%%%%%%%%%%%%%%%%%%%%%%%%%%%%%%%%%%%%%%%%%%%%%
\subsection{Command Line Processing}
\label{sec:commandline}

The effect of redirection files can also be achieved by invoking
the \LaTeX{} compiler with a more elaborate command line.
Most conveniently this should be done as part
of a shell script or a |Makefile|.

When using \textsf{childdoc} in the main file, the following
command lines effectively perform a redirection
(note that depending on the shell being used,
backslashes may have to be doubled: `|\|' $\to$ `|\\|'):
%
\begin{center}
|... -jobname "|\textit{target}|" |\\|"|[\textit{flags}]%
|\input{childdoc.def}\childdocforward[|\textit{main}|]{|\textit{dest}|}"|
\end{center}
%
Here \textit{target} is the name of the output file,
\textit{main} is the name of the main file
and \textit{dest} is the name of the main or child file to be processed
(all filenames without extensions).
The optional argument \textit{main} can be omitted
if \textit{main} matches \textit{dest}.
Optionally, compilation \textit{flags} can be defined via |\def| commands.
This command line makes the \TeX{} engine believe
it is compiling the file \textit{target}
whose content is specified as the latter parameter.
The provided code then forwards the processing to
\textit{main} or \textit{dest} as described in \secref{sec:forward}.

%%%%%%%%%%%%%%%%%%%%%%%%%%%%%%%%%%%%%%%%%%%%%%%%%%%%%%%%%%%%%%%%%%%%%%%%%%%%%%%%
\subsection{Include by Input}
\label{sec:input}

Including child documents by |\include| has some restrictions by design.
Most notably, the content of a child document always occupies
its own set of pages; pages cannot be shared between child documents.
Usually, this behaviour makes perfect sense
because each child document contain an essential part of the document.
However, in some situations it may be desirable to compose
a document from a collection of parts
without having mandatory page breaks between then.
For this case, the package
provides a mechanism to include parts
by |\input| which can also be processed individually.
However, by construction this mechanism
requires manual handling of the content to be output.

%%%%%%%%%%%%%%%%%%%%%%%%%%%%%%%%%%%%%%%%
\DescribeMacro{\ifchilddocmanual}
The main file should be prepared as usual, see \secref{sec:include}.
However, the document body must make a distinction
between processing of an individual part and of the main document, e.g.:
%
\begin{center}
\begin{tabular}{l}
|\ifchilddocmanual|\\
|\input{\childdocname}|\\
|\||else|\\
\textit{document body with }|\input{|\textit{part}|}|\\
|\||fi|
\end{tabular}
\end{center}
%
The conditional |\ifchilddocmanual| is true whenever
a part to be included by |\input| is being compiled,
and the name of the part is stored in |\childdocname|.

%%%%%%%%%%%%%%%%%%%%%%%%%%%%%%%%%%%%%%%%
\DescribeMacro{\childdocby}
Each part to be included by |\input| should start with:
%
\begin{center}
\begin{tabular}{l}
|\input{childdoc.def}|\\
|\childdocby{|\textit{main}|}|\\
\end{tabular}
\end{center}
%
The directive |\childdocby| is similar to |\childdocof|
described in \secref{sec:include},
but the subsequent selection of content must be done manually.
To that end, both |\ifchilddoc| and |\ifchilddocmanual|
will be true upon processing of a part,
and the name of the part is stored in |\childdocname|.
Note that |\jobname| will be set to the filename of the current part
so that each part receives an individual |.aux| file
that does not interfere with the |.aux| file(s) of the main document.
This behaviour can be altered by the alternative form
|\childdocby[*]{|\textit{main}|}| (with a non-empty optional argument)
which uses the |.aux| file of the main document
by setting |\jobname| to \textit{main}.

%%%%%%%%%%%%%%%%%%%%%%%%%%%%%%%%%%%%%%%%%%%%%%%%%%%%%%%%%%%%%%%%%%%%%%%%%%%%%%%%
\subsection{Driver Development}
\label{sec:driver}

The \textsf{childdoc} mechanism can also be use for the development
of definition files such as \LaTeX{} styles or classes.
This case differs from the above setup with multiple parts
included by |\include| in that no |\includeonly| should be invoked.
This can be achieved by starting the include file
(before |\ProvidesPackage|) with:
%
\begin{center}
\begin{tabular}{l}
|\input{childdoc.def}|\\
|\childdocforward{|\textit{main}|}|\\
\end{tabular}
\end{center}
%
or alternatively with:
%
\begin{center}
\begin{tabular}{l}
|\input{childdoc.def}|\\
|\childdocby{|\textit{main}|}|\\
\end{tabular}
\end{center}
%
Both forms have slightly different effects as described above.
The main file is prepared as usual, see \secref{sec:include}.

%%%%%%%%%%%%%%%%%%%%%%%%%%%%%%%%%%%%%%%%%%%%%%%%%%%%%%%%%%%%%%%%%%%%%%%%%%%%%%%%
\subsection{Legacy Detection}
\label{sec:detection}

The directive |\childdocmain| in the main file can detect
whether the complete document or merely a child is to be compiled
even without using the directive |\childdocof|.
This method is deprecated because it is less robust
and there is no compelling reason to use it;
it is merely provided for backward compatibility
and it may be removed in future versions.

If the detection mechanism is to be used,
it is mandatory to correctly specify
the filename of the main file as the argument of |\childdocmain|:
%
\begin{center}
\begin{tabular}{l}
|\input{childdoc.def}|\\
|\childdocmain{|\textit{main}|}|\\
\end{tabular}
\end{center}
%
If |\jobname| does not match the argument \textit{main} of |\childdocmain|,
it is assumed that |\jobname| points to the child file to be compiled.
When using |\childdocmain| with the main file specified as argument,
it suffices to start a child file
with just |\input{|\textit{main}|}|
without loading of the package and using |\childdocof|.
If instead all processing is done
with the appropriate \textsf{childdoc} directives,
the argument of \textit{main} of |\childdocmain| can be empty.

An alternative version of the command line processing described
in \secref{sec:commandline} using the detection mechanism reads:
%
\begin{center}
|... -jobname "|\textit{target}|" "|[\textit{flags}]%
[|\def\jobname{|\textit{dest}|}|]|\input{|\textit{main}|}"|
\end{center}

%%%%%%%%%%%%%%%%%%%%%%%%%%%%%%%%%%%%%%%%%%%%%%%%%%%%%%%%%%%%%%%%%%%%%%%%%%%%%%%%
\subsection{Manual Code}
\label{sec:manual}

In case one cannot be certain whether the definitions file |childdoc.def|
is installed on the target \TeX{} distribution
and one prefers not to ship it,
it is conceivable to paste a few relevant commands into the sources.

To that end, drop all statements |\input{childdoc.def}|
and perform the replacements as outlined below.
Instead of |\childdocmain{|\textit{main}|}| add the following code
to the top of the main file:
%
\begin{center}
\begin{tabular}{l}
|\||ifdefined\childdocname\endinput\||fi\newif\ifchilddoc|\\
|\edef\childdocname{\scantokens\expandafter{\jobname\noexpand}}|\\
|\def\childdocmain{|\textit{main}|}\||ifx\childdocmain\childdocname\||else|\\
|\childdoctrue\includeonly{\childdocname}\let\jobname\childdocmain\||fi|\\
\end{tabular}
\end{center}
%
Instead of |\childdocof{|\textit{main}|}| just include the main file
at the top of each child file:
%
\begin{center}
|\input{|\textit{main}|}|
\end{center}
%
A simple redirection |\childdocforward{|\textit{dest}|}| is achieved by:
%
\begin{center}
|\def\jobname{|\textit{dest}|}\input{\jobname}|
\end{center}
%
The redirection with prefix
|\childdocforwardprefix[|\textit{prefix}|]{|\textit{dest}|}|
is accomplished by:
%
\begin{center}
\begin{tabular}{l}
|{\edef\jobname{\scantokens\expandafter{\jobname\noexpand}}|\\
|\def\redirectjob |\textit{prefix}|#1~~~{\gdef\jobname{|\textit{dest}|#1}}|\\
|\expandafter\redirectjob\jobname~~~}\input{\jobname}|
\end{tabular}
\end{center}

In an alternative approach,
child documents can be compiled by a specific command line
without additional code or specific definitions:
%
\begin{center}
|... -jobname "|\textit{target}|" "|[\textit{flags}]%
|\includeonly{|\textit{dest}|}\input{|\textit{main}|}"|
\end{center}
%

%%%%%%%%%%%%%%%%%%%%%%%%%%%%%%%%%%%%%%%%%%%%%%%%%%%%%%%%%%%%%%%%%%%%%%%%%%%%%%%%
%%%%%%%%%%%%%%%%%%%%%%%%%%%%%%%%%%%%%%%%%%%%%%%%%%%%%%%%%%%%%%%%%%%%%%%%%%%%%%%%
\section{Information}

%%%%%%%%%%%%%%%%%%%%%%%%%%%%%%%%%%%%%%%%%%%%%%%%%%%%%%%%%%%%%%%%%%%%%%%%%%%%%%%%
\subsection{Copyright}

Copyright \copyright{} 2017--2018 Niklas Beisert

This work may be distributed and/or modified under the
conditions of the \LaTeX{} Project Public License, either version 1.3
of this license or (at your option) any later version.
The latest version of this license is in
  \url{http://www.latex-project.org/lppl.txt}
and version 1.3 or later is part of all distributions of \LaTeX{}
version 2005/12/01 or later.

This work has the LPPL maintenance status `maintained'.

The Current Maintainer of this work is Niklas Beisert.

This work consists of the files |README.txt|, |childdoc.ins| and |childdoc.dtx|
as well as the derived files |childdoc.def|, |cdocsamp.tex|
with |cdocsch1.tex|, |cdocsch2.tex|, |cdocspt3.tex|, |cdocspt4.tex|,
|cdocsdrf.tex|, |cdocsfn1.tex|, |cdocsfn2.tex|
as well as |childdoc.pdf|.

%%%%%%%%%%%%%%%%%%%%%%%%%%%%%%%%%%%%%%%%%%%%%%%%%%%%%%%%%%%%%%%%%%%%%%%%%%%%%%%%
\subsection{Files and Installation}

The package consists of the files:
%
\begin{center}
\begin{tabular}{ll}
    |README.txt|   & readme file \\
    |childdoc.ins| & installation file \\
    |childdoc.dtx| & source file \\
    |childdoc.def| & definition file \\
    |cdocsamp.tex| & sample main file \\
    |cdocsch1.tex| & sample include file \\
    |cdocsch2.tex| & sample include file \\
    |cdocspt3.tex| & sample part file \\
    |cdocspt4.tex| & sample part file \\
    |cdocsdrf.tex| & sample redirection file \\
    |cdocsfn1.tex| & sample redirection file \\
    |cdocsfn2.tex| & sample redirection file \\
    |childdoc.pdf| & manual
\end{tabular}
\end{center}
%
The distribution consists of the files
|README.txt|, |childdoc.ins| and |childdoc.dtx|.
%
\begin{itemize}
\item
Run (pdf)\LaTeX{} on |childdoc.dtx|
to compile the manual |childdoc.pdf| (this file).
\item
Run \LaTeX{} on |childdoc.ins| to create the definitions file |childdoc.def|
and the sample |cdocsamp.tex| with include files
|cdocsch1.tex|, |cdocsch2.tex|, |cdocspt3.tex|, |cdocspt4.tex|,
|cdocsdrf.tex|, |cdocsfn1.tex|, |cdocsfn2.tex|.
Then copy the file |childdoc.def| to an appropriate directory of your \LaTeX{}
distribution, e.g.\ \textit{texmf-root}|/tex/latex/childdoc|.
\end{itemize}

%%%%%%%%%%%%%%%%%%%%%%%%%%%%%%%%%%%%%%%%%%%%%%%%%%%%%%%%%%%%%%%%%%%%%%%%%%%%%%%%
\subsection{Related CTAN Packages}

There are several other packages which offer a similar functionality:
%
\begin{itemize}
\item
The packages
\href{http://ctan.org/pkg/docmute}{\textsf{docmute}},
\href{http://ctan.org/pkg/includex}{\textsf{includex}} and
\href{http://ctan.org/pkg/standalone}{\textsf{standalone}}
provide commands to include only the document body of
a child file thus allowing both files to be compiled individually.
\item
The packages \href{http://ctan.org/pkg/subdocs}{\textsf{subdocs}}
and \href{http://ctan.org/pkg/subfiles}{\textsf{subfiles}}
provide structures in which the main and child documents can be
encapsulated and allowing them to be compiled individually.
The inclusion mechanism is different from the conventional |\include|.
\item
The package \href{http://ctan.org/pkg/combine}{\textsf{combine}}
is an elaborate solution to combine several documents into one.
\end{itemize}
%
See also the CTAN topic \href{http://ctan.org/topic/subdocs}{\textsf{subdocs}}
for further related packages.
The present package differs from the above solutions in that
a document structure constructed with the conventional |\include| mechanism
just needs two extra commands at the top of every file
such that all constituent files can be compiled individually.

%%%%%%%%%%%%%%%%%%%%%%%%%%%%%%%%%%%%%%%%%%%%%%%%%%%%%%%%%%%%%%%%%%%%%%%%%%%%%%%%
%\subsection{Feature Suggestions}
%
%The following is a list of features which may be useful for future
%versions of this package:
%%
%\begin{itemize}
%\item
%\ldots
%\end{itemize}

%%%%%%%%%%%%%%%%%%%%%%%%%%%%%%%%%%%%%%%%%%%%%%%%%%%%%%%%%%%%%%%%%%%%%%%%%%%%%%%%
\subsection{Revision History}

%%%%%%%%%%%%%%%%%%%%%%%%%%%%%%%%%%%%%%%%
\paragraph{v2.0:} 2018/12/30

\begin{itemize}
\item
immediate forward processing
\item
added |\childdocby| mechanism
\item
manual restructured
\end{itemize}

%%%%%%%%%%%%%%%%%%%%%%%%%%%%%%%%%%%%%%%%
\paragraph{v1.6:} 2018/01/17

\begin{itemize}
\item
application for development of include files
\item
corrections to manual
\end{itemize}

%%%%%%%%%%%%%%%%%%%%%%%%%%%%%%%%%%%%%%%%
\paragraph{v1.5:} 2017/05/21

\begin{itemize}
\item
more complete structuring introduced
\item
|\childdocof| introduced
\item
|\childdoc| renamed to |\childdocmain|
\item
|\childredirect| renamed to |\childdocforward| and |\childdocforwardprefix|
and functionality expanded
\end{itemize}

%%%%%%%%%%%%%%%%%%%%%%%%%%%%%%%%%%%%%%%%
\paragraph{v1.0:} 2017/04/27

\begin{itemize}
\item
manual and install package
\item
first version published on CTAN
\end{itemize}

%%%%%%%%%%%%%%%%%%%%%%%%%%%%%%%%%%%%%%%%
\paragraph{v0.6:} 2017/04/26

\begin{itemize}
\item
redirection mechanism added
\end{itemize}

%%%%%%%%%%%%%%%%%%%%%%%%%%%%%%%%%%%%%%%%
\paragraph{v0.5:} 2017/04/26

\begin{itemize}
\item
functionality in definition file
\end{itemize}


%%%%%%%%%%%%%%%%%%%%%%%%%%%%%%%%%%%%%%%%%%%%%%%%%%%%%%%%%%%%%%%%%%%%%%%%%%%%%%%%
%%%%%%%%%%%%%%%%%%%%%%%%%%%%%%%%%%%%%%%%%%%%%%%%%%%%%%%%%%%%%%%%%%%%%%%%%%%%%%%%
%%%%%%%%%%%%%%%%%%%%%%%%%%%%%%%%%%%%%%%%%%%%%%%%%%%%%%%%%%%%%%%%%%%%%%%%%%%%%%%%
\appendix

\settowidth\MacroIndent{\rmfamily\scriptsize 000\ }

 \DocInput{childdoc.dtx}

\end{document}
%</driver>
% \fi
%
% %%%%%%%%%%%%%%%%%%%%%%%%%%%%%%%%%%%%%%%%%%%%%%%%%%%%%%%%%%%%%%%%%%%%%%%%%%%%%%
% %%%%%%%%%%%%%%%%%%%%%%%%%%%%%%%%%%%%%%%%%%%%%%%%%%%%%%%%%%%%%%%%%%%%%%%%%%%%%%
% \section{Sample}
%\iffalse
%<*samplemain>
%\fi
%
% The following presents a sample document
% with two chapters, two parts, a title page,
% a compile flag as well as three forwarding files to set the flag.
% It consists of eight |.tex| files:
% \begin{center}
% \begin{tabular}{ll}
% |cdocsamp.tex|&main file\\
% |cdocsch1.tex|&include file for chapter 1\\
% |cdocsch2.tex|&include file for chapter 2\\
% |cdocspt3.tex|&include file for part 3\\
% |cdocspt4.tex|&include file for part 4\\
% |cdocsdrf.tex|&forwarding file for main file in draft mode\\
% |cdocsfi1.tex|&forwarding file for final version of chapter 1\\
% |cdocsfi2.tex|&forwarding file for final version of chapter 2\\
% \end{tabular}
% \end{center}
% Each of the eight files can be compiled directly by the \LaTeX{} compiler.
%
% %%%%%%%%%%%%%%%%%%%%%%%%%%%%%%%%%%%%%%
% \paragraph{Main File.}
%
% The main file is called |cdocsamp.tex|.
%
% Load the \textsf{childdoc} definitions and
% declare the filename for the main document:
%    \begin{macrocode}
\input{childdoc.def}
\childdocmain{}
%    \end{macrocode}

% Optional override for |\version| flag:
%    \begin{macrocode}
%%\ifchilddoc\else\providecommand{\version}{draft}\fi
%    \end{macrocode}

% Define the default values for the |\version| flag
% (|final| for the main file and |draft| for childs):
%    \begin{macrocode}
\ifchilddoc
\providecommand{\version}{draft}
\else
\providecommand{\version}{final}
\fi
%    \end{macrocode}

% Load the standard document class:
%    \begin{macrocode}
\documentclass[12pt]{article}
%    \end{macrocode}

% Start the document body:
%    \begin{macrocode}
\begin{document}
%    \end{macrocode}

% Declare a title page.
% Print title, part of document being processed and version flag:
%    \begin{macrocode}
\addtocounter{page}{-1}
\begin{center}
{\LARGE\bfseries{}childdoc example\par}
\vspace{1cm}
\ifchilddoc
\ifchilddocmanual part\else chapter\fi:
`\childdocname' of `\childdocjob'\par
\else
main document: `\childdocjob'\par
\fi
version: \version\par
\end{center}
\newpage
%    \end{macrocode}

% Manually include selected file,
% otherwise process as usual:
%    \begin{macrocode}
\ifchilddocmanual
\section*{part `\childdocname'}
\input{\childdocname}
\else
%    \end{macrocode}

% Include the two chapters:
%    \begin{macrocode}
\include{cdocsch1}
\include{cdocsch2}
%    \end{macrocode}

% Include the two parts unless only chapters should be displayed:
%    \begin{macrocode}
\ifchilddoc\else
\section{part three}
\input{cdocspt3}
\section{part four}
\input{cdocspt4}
\fi
%    \end{macrocode}

% Process as usual until here:
%    \begin{macrocode}
\fi
%    \end{macrocode}

% End of document body:
%    \begin{macrocode}
\end{document}
%    \end{macrocode}
%\iffalse
%</samplemain>
%\fi
%
% %%%%%%%%%%%%%%%%%%%%%%%%%%%%%%%%%%%%%%
% \paragraph{Chapter Include Files.}
%
% The include files are called |cdocsch1.tex| and |cdocsch2.tex|.
%
%\iffalse
%<*samplechap1|samplechap2>
%\fi

% Optional override for |\version| flag:
%    \begin{macrocode}
%%\providecommand{\version}{final}
%    \end{macrocode}

% Include the main document:
%    \begin{macrocode}
\input{childdoc.def}
\childdocof{cdocsamp}
%    \end{macrocode}

%\iffalse
%</samplechap1|samplechap2>
%\fi
%
%\iffalse
%<*samplechap1>
%\fi
% Some text for chapter 1:
%    \begin{macrocode}
\section{one}
some text in chapter one
%    \end{macrocode}

%\iffalse
%</samplechap1>
%\fi
% Some text for chapter 2:
%\iffalse
%<*samplechap2>
%\fi
%    \begin{macrocode}
\section{two}
more text in chapter two
%    \end{macrocode}

%\iffalse
%</samplechap2>
%\fi
%
% %%%%%%%%%%%%%%%%%%%%%%%%%%%%%%%%%%%%%%
% \paragraph{Part Include Files.}
%
% The include files are called |cdocspt3.tex| and |cdocspt4.tex|.
%
%\iffalse
%<*samplepart3|samplepart4>
%\fi

% Optional override for |\version| flag:
%    \begin{macrocode}
%%\providecommand{\version}{final}
%    \end{macrocode}

% Include the main document:
%    \begin{macrocode}
\input{childdoc.def}
\childdocby{cdocsamp}
%    \end{macrocode}

%\iffalse
%</samplepart3|samplepart4>
%\fi
%
%\iffalse
%<*samplepart3>
%\fi
% Some text for part 3:
%    \begin{macrocode}
some text in part three
%    \end{macrocode}

%\iffalse
%</samplepart3>
%\fi
% Some text for part 4:
%\iffalse
%<*samplepart4>
%\fi
%    \begin{macrocode}
more text in part four
%    \end{macrocode}

%\iffalse
%</samplepart4>
%\fi
%
% %%%%%%%%%%%%%%%%%%%%%%%%%%%%%%%%%%%%%%
% \paragraph{Forwarding for a Complete Draft.}
%
% The following forwarding file |cdocsdrf.tex|
% compiles the main document in draft mode:
%\iffalse
%<*sampledraft>
%\fi
%    \begin{macrocode}
\def\version{draft}
\input{childdoc.def}
\childdocforward{cdocsamp}
%    \end{macrocode}

%\iffalse
%</sampledraft>
%\fi
%
% %%%%%%%%%%%%%%%%%%%%%%%%%%%%%%%%%%%%%%
% \paragraph{Forwarding for Final Version of the Chapters.}
%
% The following forwarding files |cdocsfn1.tex| and |cdocsfn2.tex|
% (with identical content)
% compile the final versions of the child documents
% |cdocsch1.tex| and |cdocsch2.tex|, respectively:
%\iffalse
%<*samplefinal>
%\fi
%    \begin{macrocode}
\def\version{final}
\input{childdoc.def}
\childdocforwardprefix[cdocsamp]{cdocsfn}{cdocsch}
%    \end{macrocode}

%\iffalse
%</samplefinal>
%\fi
%
% %%%%%%%%%%%%%%%%%%%%%%%%%%%%%%%%%%%%%%
% \paragraph{Command Line Processing.}
%
% The following three command lines generate the output files
% |cdocscld|, |cdocscl1| and |cdocscl2|
% which should be identical to
% |cdocsdrf|, |cdocsch1| and |cdocsfn2|, respectively:
% \begin{center}
% \begin{tabular}{l}
% |latex -jobname cdocscld \|\\
% |  "\def\version{draft}\input{childdoc.def}\childdocforward{cdocsamp}"|\\
% |latex -jobname cdocscl1 \|\\
% |  "\input{childdoc.def}\childdocforward[cdocsamp]{cdocsch1}"|\\
% |latex -jobname cdocscl2 \|\\
% |  "\def\version{final}\input{childdoc.def}\childdocforward{cdocsch2}"|
% \end{tabular}
% \end{center}
% Note that the trailing backslash on each first line
% merely continues the input to the second line
% (for convenient cut ant paste).
% Furthermore, the command |latex| can be replaced by any
% of its alternative versions such as |pdflatex|.
%
% %%%%%%%%%%%%%%%%%%%%%%%%%%%%%%%%%%%%%%%%%%%%%%%%%%%%%%%%%%%%%%%%%%%%%%%%%%%%%%
% %%%%%%%%%%%%%%%%%%%%%%%%%%%%%%%%%%%%%%%%%%%%%%%%%%%%%%%%%%%%%%%%%%%%%%%%%%%%%%
% \section{Implementation}
%\iffalse
%<*package>
%\fi
%
% This section describes the definitions file |childdoc.def|.

% The definitions cannot be loaded using |\usepackage| or |\RequirePackage|
% which has a mechanism to prevent loading a style file more than once.
% When loading the definitions by means of |\input|
% multiple instances have to be prevented manually:
%\iffalse
%This code needs to be before the `\ProvidesFile' directive
%which is defined at the beginning of this file.
%Therefore it is also placed there and commented out here.
%</package>
%<*discard>
%\fi
%    \begin{macrocode}
\ifdefined\childdocmain\endinput\fi
%    \end{macrocode}
%\iffalse
%</discard>
%<*package>
%\fi
%
% \macro{\ifchilddoc}
% \macro{\ifchilddocmanual}
% The conditional |\ifchilddoc| tells whether a
% child (true) or main (false) document is being compiled.
% The conditional |\ifchilddocmanual| tells whether
% the |\includeonly| mechanism is used (false) or
% the selection of child files must be performed manually (true).
% The definitions initialise to false:
%    \begin{macrocode}
\newif\ifchilddoc
\newif\ifchilddocmanual
%    \end{macrocode}

% \macro{\childdocname}
% \macro{\childdocjob}
% The macro |\childdocname| stores the name of the main document
% to be compiled. The macro |\childdocjob| stores the name of
% the document on which the \LaTeX{} compiler was originally invoked.
% The content of |\jobname| cannot be compared
% to filenames specified in the source due to different catcodes.
% The following code rescans |\jobname|, stores the result
% in |\childdocname| and saves a copy in |\childdocjob|:
%    \begin{macrocode}
\edef\childdocname{\scantokens\expandafter{\jobname\noexpand}}
\let\childdocjob\childdocname
%    \end{macrocode}

% \macro{\childdocdisable}
% The macro |\childdocdisable| prevents the main file
% from being processed more than once.
% At this stage, the main document command |\childdocmain|
% is assumed to be called once again where it should do nothing.
% Any subsequent call to it should prevent
% a secondary processing of the main document
% It overwrites the forwarding commands
% |\childdocof| and |\childdocforward|
% with empty macros to prevent further inclusions of the main document:
%    \begin{macrocode}
\newcommand{\childdocdisable}
{
  \renewcommand{\childdocmain}[1]{\renewcommand{\childdocmain}[1]{\endinput}}
  \renewcommand{\childdocof}[1]{}
  \renewcommand{\childdocby}[2][]{}
  \renewcommand{\childdocforward}[2][]{}
  \renewcommand{\childdocdisable}{}
}
%    \end{macrocode}

% \macro{\childdocmain}
% The macro |\childdocmain| is to be called at the top of the main file
% with nothing or the main filename (without extension) as argument.
% First, it breaks loops.
% If the argument is not empty and does not match |\childdocname|
% (which is set by the first inclusion of |childdoc.def|),
% |\ifchilddoc| is set to true, |\includeonly| is applied to the child file
% and |\jobname| is set to the main file
% (for proper handling of |.aux| files):
%    \begin{macrocode}
\newcommand{\childdocmain}[1]
{
  \childdocdisable\childdocmain{}
  \if?#1?\else
    \begingroup
      \def\childdoctmp{#1}
      \ifx\childdoctmp\childdocname
        \def\childdoctmp{}
      \else
        \def\childdoctmp
        {
          \childdoctrue
          \includeonly{\childdocname}
          \def\childdocjob{#1}
          \def\jobname{#1}
        }
      \fi
      \expandafter
    \endgroup
    \childdoctmp
  \fi
}
%    \end{macrocode}

% \macro{\childdocof}
% The command |\childdocof| redirects
% compilation to the main file |#1|.
%    \begin{macrocode}
\newcommand{\childdocof}[1]
{
  \childdocdisable
  \childdoctrue
  \includeonly{\childdocname}
  \def\jobname{#1}
  \def\childdocjob{#1}
  \input{#1}
}
%    \end{macrocode}

% \macro{\childdocby}
% The command |\childdocby| ....
%    \begin{macrocode}
\newcommand{\childdocby}[2][]
{
  \childdocdisable
  \childdoctrue
  \childdocmanualtrue
  \if?#1?\else
    \def\jobname{#2}
  \fi
  \def\childdocjob{#2}
  \input{#2}
  \endinput
}
%    \end{macrocode}

% \macro{\childdocforward}
% The command |\childdocforward| redirects
% compilation to the main file or
% (if the optional argument is given) a child file.
% Parameters are set as if the main file
% or a child file starting with |\childdocof| was compiled.
% Then compilation is handed over to the main file:
%    \begin{macrocode}
\newcommand{\childdocforward}[2][]
{
  \begingroup
    \if?#1?
      \def\childdoctmp
      {
        \def\childdocname{#2}
        \def\childdocjob{#2}
        \def\jobname{#2}
        \input{#2}
        \endinput
      }
    \else
      \def\childdoctmp
      {
        \childdocdisable
        \def\childdocname{#2}
        \childdoctrue
        \includeonly{#2}
        \def\childdocjob{#1}
        \def\jobname{#1}
        \input{#1}
        \endinput
      }
    \fi
    \expandafter
  \endgroup
  \childdoctmp
}
%    \end{macrocode}

% \macro{\childdocforwardprefix}
% The command |\childdocforwardprefix| redirects
% compilation to the main or a child file by means of a pattern.
% The prefix |#1| in the current filename is replaced by |#2|
% and the suffix of the current filename is kept
% (it is assumed that the filename does not contain the substring `|~~~|'
% which is used as a delimiter).
% Compilation is handed over to the new file by |\childdocforward|:
%    \begin{macrocode}
\newcommand{\childdocforwardprefix}[3][]
{
  \begingroup
    \def\childdocextract #2##1~~~{\def\childdoctmp{\childdocforward[#1]{#3##1}}}
    \expandafter\childdocextract\childdocname~~~
    \expandafter
  \endgroup
  \childdoctmp
}
%    \end{macrocode}

% \macro{\childdoc}
% The deprecated macro |\childdoc| is a legacy version of |\childdocmain|:
%    \begin{macrocode}
\newcommand{\childdoc}{\childdocmain}
%    \end{macrocode}

% \macro{\childdocredirect}
% The deprecated macro |\childdocredirect| is a legacy version
% of |\childdocforward| and |\childdocforwardprefix|:
%    \begin{macrocode}
\newcommand{\childdocredirect}[2][]
{
  \begingroup
    \if?#1?
      \def\childdoctmp{\childdocforward{#2}}
    \else
      \def\childdoctmp{\childdocforwardprefix{#1}{#2}}
    \fi
    \expandafter
  \endgroup
  \childdoctmp
}
%    \end{macrocode}

%\iffalse
%</package>
%\fi
%
\endinput
|\\
|\childdocforwardprefix{final}{child}|
\end{tabular}
\end{center}
%

Note that when several versions of a main file and/or of each child file
are to be generated, it may be convenient to set up a |Makefile| or
shell script to automatise the process.

%%%%%%%%%%%%%%%%%%%%%%%%%%%%%%%%%%%%%%%%%%%%%%%%%%%%%%%%%%%%%%%%%%%%%%%%%%%%%%%%
\subsection{Command Line Processing}
\label{sec:commandline}

The effect of redirection files can also be achieved by invoking
the \LaTeX{} compiler with a more elaborate command line.
Most conveniently this should be done as part
of a shell script or a |Makefile|.

When using \textsf{childdoc} in the main file, the following
command lines effectively perform a redirection
(note that depending on the shell being used,
backslashes may have to be doubled: `|\|' $\to$ `|\\|'):
%
\begin{center}
|... -jobname "|\textit{target}|" |\\|"|[\textit{flags}]%
|% \iffalse
%
% childdoc.dtx Copyright (C) 2017-2018 Niklas Beisert
%
% This work may be distributed and/or modified under the
% conditions of the LaTeX Project Public License, either version 1.3
% of this license or (at your option) any later version.
% The latest version of this license is in
%   http://www.latex-project.org/lppl.txt
% and version 1.3 or later is part of all distributions of LaTeX
% version 2005/12/01 or later.
%
% This work has the LPPL maintenance status `maintained'.
%
% The Current Maintainer of this work is Niklas Beisert.
%
% This work consists of the files childdoc.dtx and childdoc.ins
% and the derived files childdoc.def and cdocsamp.tex with
% cdocsch1.tex, cdocsch2.tex, cdocsdrf.tex, cdocsfn1.tex, cdocsfn2.tex.
%
%<package>\ifdefined\childdocmain\endinput\fi
%<package>\ProvidesFile{childdoc.def}[2018/12/30 v2.0 child document driver]
%<samplemain>\ProvidesFile{cdocsamp.tex}[2018/12/30 v2.0 sample for childdoc]
%<*driver>
%\ProvidesFile{childdoc.drv}[2018/12/30 v2.0 childdoc reference manual file]
\PassOptionsToClass{10pt,a4paper}{article}
\documentclass{ltxdoc}

\usepackage[margin=35mm]{geometry}
\usepackage{hyperref}
\usepackage{hyperxmp}
\usepackage[usenames]{color}

\hypersetup{colorlinks=true}
\hypersetup{pdfstartview=FitH}
\hypersetup{pdfpagemode=UseNone}
\hypersetup{pdfsource={}}
\hypersetup{pdflang={en-UK}}
\hypersetup{pdfcopyright={Copyright 2017-2018 Niklas Beisert.
  This work may be distributed and/or modified under the
  conditions of the LaTeX Project Public License, either version 1.3
  of this license or (at your option) any later version.}}
\hypersetup{pdflicenseurl={http://www.latex-project.org/lppl.txt}}
\hypersetup{pdfcontactaddress={ETH Zurich, ITP, HIT K,
  Wolfgang-Pauli-Strasse 27}}
\hypersetup{pdfcontactpostcode={8093}}
\hypersetup{pdfcontactcity={Zurich}}
\hypersetup{pdfcontactcountry={Switzerland}}
\hypersetup{pdfcontactemail={nbeisert@itp.phys.ethz.ch}}
\hypersetup{pdfcontacturl={http://people.phys.ethz.ch/\xmptilde nbeisert/}}

\newcommand{\secref}[1]{\hyperref[#1]{section \ref*{#1}}}

\parskip1ex
\parindent0pt
\let\olditemize\itemize
\def\itemize{\olditemize\parskip0pt}

\begin{document}

\title{The \textsf{childdoc} Package}
\hypersetup{pdftitle={The childdoc Package}}
\author{Niklas Beisert\\[2ex]
  Institut f\"ur Theoretische Physik\\
  Eidgen\"ossische Technische Hochschule Z\"urich\\
  Wolfgang-Pauli-Strasse 27, 8093 Z\"urich, Switzerland\\[1ex]
  \href{mailto:nbeisert@itp.phys.ethz.ch}
  {\texttt{nbeisert@itp.phys.ethz.ch}}}
\hypersetup{pdfauthor={Niklas Beisert}}
\hypersetup{pdfsubject={Manual for the LaTeX2e Package childdoc}}
\date{30 December 2018, \textsf{v2.0}}
\maketitle

\begin{abstract}\noindent
\textsf{childdoc} is a \LaTeXe{} package
that enables the direct compilation
of document sections included by |\include|
to individual files.
\end{abstract}

\begingroup
\parskip0ex
\tableofcontents
\endgroup

%%%%%%%%%%%%%%%%%%%%%%%%%%%%%%%%%%%%%%%%%%%%%%%%%%%%%%%%%%%%%%%%%%%%%%%%%%%%%%%%
%%%%%%%%%%%%%%%%%%%%%%%%%%%%%%%%%%%%%%%%%%%%%%%%%%%%%%%%%%%%%%%%%%%%%%%%%%%%%%%%
\section{Introduction}

\LaTeX{} provides a mechanism to structure a large document (such as a book)
into a main file and several child files (containing the chapters)
using the |\include| command.
This mechanism is beneficial for documents
which span hundreds of pages in order to
make the source file(s) more manageable.
Moreover, compilation can be restricted to
selected child files by means of the |\includeonly| command.
The latter feature can be used to reduce the compilation time while editing
(this was significantly more useful in the earlier days of \LaTeX{})
or to generate a smaller document which is easier to navigate.
Another application of |\includeonly| is to generate
documents consisting of selected parts of the complete document.

However, there are a few drawbacks of the plain |\include| mechanism:
\begin{itemize}
\item
The child files cannot be compiled on their own,
they can only be compiled via the main file.
A naive editing environment
(such as a text editor with an option
to have the current file processed by \LaTeX)
may require one to switch to the main file before compiling;
attempting to compile the child file produces errors.
\item
The main file must be modified (each time)
to adjust the |\includeonly| command
to the present needs. This easily leaves the main file in a messy state.
\item
The generated document will always carry the filename
of the main document. This is inconvenient if
several child files are to be compiled and
to be kept for distribution.
\end{itemize}

The present package provides a simple interface
to make child files individually compilable by \LaTeX{}.
Compiling a child file then has the same effect as compiling
the main file with an |\includeonly| command
to select the appropriate child.
Moreover the generated document will carry the name of the child
rather than the main file.
This resolves all three above issues.

This feature is meant to make the editing of books,
thesis documents and lecture notes somewhat more convenient.
However, the package can also be used efficiently for
composing a series of documents (such as exercise sheets)
which are typically distributed individually.
It then assists the author in generating the individual documents
(potentially in different versions)
as well as a document containing the collected series.
Another application is in developing style files
or other kinds of included material
where compilation of the style file could redirect
to a sample or test file.

%%%%%%%%%%%%%%%%%%%%%%%%%%%%%%%%%%%%%%%%%%%%%%%%%%%%%%%%%%%%%%%%%%%%%%%%%%%%%%%%
%%%%%%%%%%%%%%%%%%%%%%%%%%%%%%%%%%%%%%%%%%%%%%%%%%%%%%%%%%%%%%%%%%%%%%%%%%%%%%%%
\section{Usage}

First of all, the package \textsf{childdoc} is \emph{not} a standard
\LaTeXe{} |.sty| style file! Therefore it needs to be invoked in
a non-standard way.

%%%%%%%%%%%%%%%%%%%%%%%%%%%%%%%%%%%%%%%%%%%%%%%%%%%%%%%%%%%%%%%%%%%%%%%%%%%%%%%%
\subsection{Included Files}
\label{sec:include}

%%%%%%%%%%%%%%%%%%%%%%%%%%%%%%%%%%%%%%%%
\DescribeMacro{\childdocmain}
To use the package, add the commands
\begin{center}
\begin{tabular}{l}
|\input{childdoc.def}|\\
|\childdocmain{}|\\
\end{tabular}
\end{center}
at the very top of the main \LaTeX{} file,
in particular \emph{before} the |\documentclass| statement!
The argument of |\childdocmain| should be left empty
(but it must be present).

%%%%%%%%%%%%%%%%%%%%%%%%%%%%%%%%%%%%%%%%
\DescribeMacro{\childdocof}
Furthermore, add the commands
\begin{center}
\begin{tabular}{l}
|\input{childdoc.def}|\\
|\childdocof{|\textit{main}|}|\\
\end{tabular}
\end{center}
at the top of every child file \textit{child}
which is included by |\include{|\textit{child}|}|
from within the main file
(or at least for those files to be compiled individually).
The argument \textit{main} must be the filename of the main file.

There are a couple of
considerations in setting up the main and child documents:

%%%%%%%%%%%%%%%%%%%%%%%%%%%%%%%%%%%%%%%%
\paragraph{Restrictions.}

Please note the following restrictions:
\begin{itemize}
\item
|\childdocmain| must be called with one argument \textit{main}
to ensure compatibility with earlier version of the package.
It must either be empty (|\childdocmain{}|)
or precisely match the filename of the main file in which it is specified.
See \secref{sec:detection} for further information.
\item
The filename \textit{main} must be specified without the |.tex| extension.
\item
The filename \textit{main} is case sensitive
(even in case-insensitive file systems)
due to internal string comparison.
\item
The argument \textit{main} should be fully expanded, it cannot be a macro.
\item
Subdirectories and special characters should be avoided in filenames.
\item
The command |\childdocmain{|\textit{main}|}| must be followed by a whitespace.
It should not be followed immediately by another command
or by a comment mark `|%|'.
This is because the \TeX{} parser reads the token immediately following
the argument of |\childdocmain| and puts it
at the beginning of every child section;
however, a white\-space is ignored.
\end{itemize}

%%%%%%%%%%%%%%%%%%%%%%%%%%%%%%%%%%%%%%%%
\paragraph{Content of Main File.}

It is advisable to place all content in the child files included by |\include|.
Any output contained in the main file will appear in all child documents
unless suppressed manually;
it cannot be suppressed automatically by the |\includeonly| directive
and thus should normally be avoided.
A method to include some content in the main file
by means of conditional processing is described in \secref{sec:conditional}.

%%%%%%%%%%%%%%%%%%%%%%%%%%%%%%%%%%%%%%%%
\paragraph{Page Numbering.}

When only a part of the document is compiled,
the appropriate numbering of pages
(as well as other status parameters)
is determined from the |.aux| files.
The latter contain information from previous passes.
However this information needs to propagate through
all intermediate child documents.
Therefore the page numbering in child documents may well
be inconsistent until the complete document is compiled at least once.

A useful (if unconventional) way to always ensure a consistent
page numbering is to restart the numbering in each child document
and denote the pages by `\textit{child}|.|\textit{page}'
where \textit{child} represents the chapter/section number of the child file.
This can be achieved by the command
|\numberwithin{page}{|\textit{child}|}|
of the \textsf{amsmath} package
where \textit{child} can be |chapter| or |section|
depending on the chosen structuring.
Alternatively, one can modify the macro |\thepage| appropriately
and reset the counter |page| at the start of each child file.

%%%%%%%%%%%%%%%%%%%%%%%%%%%%%%%%%%%%%%%%%%%%%%%%%%%%%%%%%%%%%%%%%%%%%%%%%%%%%%%%
\subsection{Conditional Processing}
\label{sec:conditional}

The package provides a mechanism to compile different versions
of a document. To customise the versions further some conditional processing
can come in handy to distinguish which version is being compiled.
The package provides two macros to describe the compilation context:

%%%%%%%%%%%%%%%%%%%%%%%%%%%%%%%%%%%%%%%%
\DescribeMacro{\ifchilddoc}
The conditional |\ifchilddoc| distinguishes between the compilation of
child documents and the main document:
%
\begin{center}
|\ifchilddoc |\textit{child-code}| |[|\||else |\textit{main-code}]| \||fi|
\end{center}

%%%%%%%%%%%%%%%%%%%%%%%%%%%%%%%%%%%%%%%%
\DescribeMacro{\childdocname}
\DescribeMacro{\childdocjob}
The macro |\childdocname| contains the filename (without extension)
of the main or child file being processed.
Note that |\childdocjob| will always contain the name of the main file.

%%%%%%%%%%%%%%%%%%%%%%%%%%%%%%%%%%%%%%%%
\paragraph{Title Page.}

Conditional processing can be used to include a title or banner page
in the main document when proper precautions are taken.
Importantly, the code in the main file should ensure that the page counter
(as well as other status parameters which are stored in the |.aux| files)
takes the same value after the conditional processing.
Otherwise the page numbers may take divergent values
depending on which part is compiled.

For example, a title page could be declared by:
%
\begin{center}
\begin{tabular}{l}
|\ifchilddoc\||else|\\
|\addtocounter{page}{-1}|\\
\textit{code for title page}\\
|\newpage|\\
|\||fi|
\end{tabular}
\end{center}
%
A banner page for the child documents can be generated by:
%
\begin{center}
\begin{tabular}{l}
|\ifchilddoc|\\
|\addtocounter{page}{-1}|\\
\textit{code for banner page}\\
|\newpage|\\
|\||fi|
\end{tabular}
\end{center}
%
Here one could write a message such as:
\begin{center}
|This is the part \childdocname{} of \childdocjob{}.|
\end{center}

%%%%%%%%%%%%%%%%%%%%%%%%%%%%%%%%%%%%%%%%%%%%%%%%%%%%%%%%%%%%%%%%%%%%%%%%%%%%%%%%
\subsection{Flags}
\label{sec:flags}

The package makes it easy to generate different versions
of the main or child documents.
To this end compilation flags can be defined
and assigned different default values.
They will be particularly useful in conjunction
with the forwarding mechanism described in \secref{sec:forward}.

For example, it may be useful to have a flag |\version|
which can be set to |draft| or |final|.
The document source will contain some conditional code
depending on the value of |\version|.
Suppose further, the flag should default to |final| for the main file
and to |draft| for child files
which is a natural assignment for editing the document.
This is achieved by placing the following code
in the preamble of the main document
(below the |\childdocmain| directive):
%
\begin{center}
\begin{tabular}{l}
|\ifchilddoc|\\
|\providecommand{\version}{draft}|\\
|\||else|\\
|\providecommand{\version}{final}|\\
|\||fi|
\end{tabular}
\end{center}
%
The definition by |\providecommand| makes sure
that previous definitions are not overwritten.
Further statements |\providecommand{\version}{...}|
can thus be added before the above code to override it.

For the main file, one might add a line
(between |\childdocmain| and the above block)
%
\begin{center}
|%\ifchilddoc\||else\providecommand{\version}{draft}\||fi|
\end{center}
%
which can be uncommented to produce a draft version.
Likewise one can add a line to the very top of a child file
(above the |\childdocof{|\textit{main}|}| directive)
%
\begin{center}
|%\providecommand{\version}{final}|
\end{center}
%
which can be uncommented to produce the final version of this child document.

%%%%%%%%%%%%%%%%%%%%%%%%%%%%%%%%%%%%%%%%%%%%%%%%%%%%%%%%%%%%%%%%%%%%%%%%%%%%%%%%
\subsection{Forwarding}
\label{sec:forward}

Different versions of the main or child documents
using compilation flags as described in \secref{sec:flags}
can be (permanently) stored in different files
for convenient compilation, viewing and distribution.
To this end, the package defines a command
to pass on compilation to a different file:

%%%%%%%%%%%%%%%%%%%%%%%%%%%%%%%%%%%%%%%%
\DescribeMacro{\childdocforward}
The command |\childdocforward| redirects processing to
another source file:
%
\begin{center}
\begin{tabular}{l}
|\input{childdoc.def}|\\
|\childdocforward[|\textit{main}|]{|\textit{dest}|}|\\
\end{tabular}
\end{center}
%
The argument \textit{dest} is the destination file
(without extension).
It should be the main file or one of the child files.
Note that further \textsf{childdoc} directives
such as |\childdocof| and |\childdocforward|
in the indicated file will be processed in this form.
The optional argument \textit{main}
passes on directly to the main file \textit{main}
while pretending to compile the child \textit{dest}.
This form behaves as if \textit{dest}
issues |\childdocof{|\textit{main}|}| right away,
and no further \textsf{childdoc} directives will be processed.

%%%%%%%%%%%%%%%%%%%%%%%%%%%%%%%%%%%%%%%%
\DescribeMacro{\...prefix}
In the alternative form |\childdocforwardprefix|,
%
\begin{center}
\begin{tabular}{l}
|\input{childdoc.def}|\\
|\childdocforwardprefix[|\textit{main}|]{|\textit{prefix}|}{|\textit{dest}|}|
\end{tabular}
\end{center}
%
the destination file is determined by a pattern
depending on the current file:
To make this work, the current file must be called
`{\textit{prefix}\hspace{0.2em}\textit{suffix}}'
with \textit{prefix} matching precisely the argument.
Processing is then passed on to the file
`{\textit{dest}\hspace{0.2em}\textit{suffix}}'.
Surely, the same effect is achieved by
directly specifying the
argument `{\textit{dest}\hspace{0.2em}\textit{suffix}}'
in the first form.
However, that requires to set up a different file
for each child. With the alternative form of the command
all these files can have exactly the same content
which simplifies setting them up and maintaining them.

For example, the following file |draft.tex|
with a compilation flag |\version| as described in \secref{sec:flags}
compiles the main document as a draft:
%
\begin{center}
\begin{tabular}{l}
|\def\version{draft}|\\
|\input{childdoc.def}|\\
|\childdocforward{|\textit{main}|}|
\end{tabular}
\end{center}
%
Likewise, the following files |final|\textit{nn}|.tex|
compile the final version of the child document
|child|\textit{nn}|.tex|:
%
\begin{center}
\begin{tabular}{l}
|\def\version{final}|\\
|\input{childdoc.def}|\\
|\childdocforwardprefix{final}{child}|
\end{tabular}
\end{center}
%

Note that when several versions of a main file and/or of each child file
are to be generated, it may be convenient to set up a |Makefile| or
shell script to automatise the process.

%%%%%%%%%%%%%%%%%%%%%%%%%%%%%%%%%%%%%%%%%%%%%%%%%%%%%%%%%%%%%%%%%%%%%%%%%%%%%%%%
\subsection{Command Line Processing}
\label{sec:commandline}

The effect of redirection files can also be achieved by invoking
the \LaTeX{} compiler with a more elaborate command line.
Most conveniently this should be done as part
of a shell script or a |Makefile|.

When using \textsf{childdoc} in the main file, the following
command lines effectively perform a redirection
(note that depending on the shell being used,
backslashes may have to be doubled: `|\|' $\to$ `|\\|'):
%
\begin{center}
|... -jobname "|\textit{target}|" |\\|"|[\textit{flags}]%
|\input{childdoc.def}\childdocforward[|\textit{main}|]{|\textit{dest}|}"|
\end{center}
%
Here \textit{target} is the name of the output file,
\textit{main} is the name of the main file
and \textit{dest} is the name of the main or child file to be processed
(all filenames without extensions).
The optional argument \textit{main} can be omitted
if \textit{main} matches \textit{dest}.
Optionally, compilation \textit{flags} can be defined via |\def| commands.
This command line makes the \TeX{} engine believe
it is compiling the file \textit{target}
whose content is specified as the latter parameter.
The provided code then forwards the processing to
\textit{main} or \textit{dest} as described in \secref{sec:forward}.

%%%%%%%%%%%%%%%%%%%%%%%%%%%%%%%%%%%%%%%%%%%%%%%%%%%%%%%%%%%%%%%%%%%%%%%%%%%%%%%%
\subsection{Include by Input}
\label{sec:input}

Including child documents by |\include| has some restrictions by design.
Most notably, the content of a child document always occupies
its own set of pages; pages cannot be shared between child documents.
Usually, this behaviour makes perfect sense
because each child document contain an essential part of the document.
However, in some situations it may be desirable to compose
a document from a collection of parts
without having mandatory page breaks between then.
For this case, the package
provides a mechanism to include parts
by |\input| which can also be processed individually.
However, by construction this mechanism
requires manual handling of the content to be output.

%%%%%%%%%%%%%%%%%%%%%%%%%%%%%%%%%%%%%%%%
\DescribeMacro{\ifchilddocmanual}
The main file should be prepared as usual, see \secref{sec:include}.
However, the document body must make a distinction
between processing of an individual part and of the main document, e.g.:
%
\begin{center}
\begin{tabular}{l}
|\ifchilddocmanual|\\
|\input{\childdocname}|\\
|\||else|\\
\textit{document body with }|\input{|\textit{part}|}|\\
|\||fi|
\end{tabular}
\end{center}
%
The conditional |\ifchilddocmanual| is true whenever
a part to be included by |\input| is being compiled,
and the name of the part is stored in |\childdocname|.

%%%%%%%%%%%%%%%%%%%%%%%%%%%%%%%%%%%%%%%%
\DescribeMacro{\childdocby}
Each part to be included by |\input| should start with:
%
\begin{center}
\begin{tabular}{l}
|\input{childdoc.def}|\\
|\childdocby{|\textit{main}|}|\\
\end{tabular}
\end{center}
%
The directive |\childdocby| is similar to |\childdocof|
described in \secref{sec:include},
but the subsequent selection of content must be done manually.
To that end, both |\ifchilddoc| and |\ifchilddocmanual|
will be true upon processing of a part,
and the name of the part is stored in |\childdocname|.
Note that |\jobname| will be set to the filename of the current part
so that each part receives an individual |.aux| file
that does not interfere with the |.aux| file(s) of the main document.
This behaviour can be altered by the alternative form
|\childdocby[*]{|\textit{main}|}| (with a non-empty optional argument)
which uses the |.aux| file of the main document
by setting |\jobname| to \textit{main}.

%%%%%%%%%%%%%%%%%%%%%%%%%%%%%%%%%%%%%%%%%%%%%%%%%%%%%%%%%%%%%%%%%%%%%%%%%%%%%%%%
\subsection{Driver Development}
\label{sec:driver}

The \textsf{childdoc} mechanism can also be use for the development
of definition files such as \LaTeX{} styles or classes.
This case differs from the above setup with multiple parts
included by |\include| in that no |\includeonly| should be invoked.
This can be achieved by starting the include file
(before |\ProvidesPackage|) with:
%
\begin{center}
\begin{tabular}{l}
|\input{childdoc.def}|\\
|\childdocforward{|\textit{main}|}|\\
\end{tabular}
\end{center}
%
or alternatively with:
%
\begin{center}
\begin{tabular}{l}
|\input{childdoc.def}|\\
|\childdocby{|\textit{main}|}|\\
\end{tabular}
\end{center}
%
Both forms have slightly different effects as described above.
The main file is prepared as usual, see \secref{sec:include}.

%%%%%%%%%%%%%%%%%%%%%%%%%%%%%%%%%%%%%%%%%%%%%%%%%%%%%%%%%%%%%%%%%%%%%%%%%%%%%%%%
\subsection{Legacy Detection}
\label{sec:detection}

The directive |\childdocmain| in the main file can detect
whether the complete document or merely a child is to be compiled
even without using the directive |\childdocof|.
This method is deprecated because it is less robust
and there is no compelling reason to use it;
it is merely provided for backward compatibility
and it may be removed in future versions.

If the detection mechanism is to be used,
it is mandatory to correctly specify
the filename of the main file as the argument of |\childdocmain|:
%
\begin{center}
\begin{tabular}{l}
|\input{childdoc.def}|\\
|\childdocmain{|\textit{main}|}|\\
\end{tabular}
\end{center}
%
If |\jobname| does not match the argument \textit{main} of |\childdocmain|,
it is assumed that |\jobname| points to the child file to be compiled.
When using |\childdocmain| with the main file specified as argument,
it suffices to start a child file
with just |\input{|\textit{main}|}|
without loading of the package and using |\childdocof|.
If instead all processing is done
with the appropriate \textsf{childdoc} directives,
the argument of \textit{main} of |\childdocmain| can be empty.

An alternative version of the command line processing described
in \secref{sec:commandline} using the detection mechanism reads:
%
\begin{center}
|... -jobname "|\textit{target}|" "|[\textit{flags}]%
[|\def\jobname{|\textit{dest}|}|]|\input{|\textit{main}|}"|
\end{center}

%%%%%%%%%%%%%%%%%%%%%%%%%%%%%%%%%%%%%%%%%%%%%%%%%%%%%%%%%%%%%%%%%%%%%%%%%%%%%%%%
\subsection{Manual Code}
\label{sec:manual}

In case one cannot be certain whether the definitions file |childdoc.def|
is installed on the target \TeX{} distribution
and one prefers not to ship it,
it is conceivable to paste a few relevant commands into the sources.

To that end, drop all statements |\input{childdoc.def}|
and perform the replacements as outlined below.
Instead of |\childdocmain{|\textit{main}|}| add the following code
to the top of the main file:
%
\begin{center}
\begin{tabular}{l}
|\||ifdefined\childdocname\endinput\||fi\newif\ifchilddoc|\\
|\edef\childdocname{\scantokens\expandafter{\jobname\noexpand}}|\\
|\def\childdocmain{|\textit{main}|}\||ifx\childdocmain\childdocname\||else|\\
|\childdoctrue\includeonly{\childdocname}\let\jobname\childdocmain\||fi|\\
\end{tabular}
\end{center}
%
Instead of |\childdocof{|\textit{main}|}| just include the main file
at the top of each child file:
%
\begin{center}
|\input{|\textit{main}|}|
\end{center}
%
A simple redirection |\childdocforward{|\textit{dest}|}| is achieved by:
%
\begin{center}
|\def\jobname{|\textit{dest}|}\input{\jobname}|
\end{center}
%
The redirection with prefix
|\childdocforwardprefix[|\textit{prefix}|]{|\textit{dest}|}|
is accomplished by:
%
\begin{center}
\begin{tabular}{l}
|{\edef\jobname{\scantokens\expandafter{\jobname\noexpand}}|\\
|\def\redirectjob |\textit{prefix}|#1~~~{\gdef\jobname{|\textit{dest}|#1}}|\\
|\expandafter\redirectjob\jobname~~~}\input{\jobname}|
\end{tabular}
\end{center}

In an alternative approach,
child documents can be compiled by a specific command line
without additional code or specific definitions:
%
\begin{center}
|... -jobname "|\textit{target}|" "|[\textit{flags}]%
|\includeonly{|\textit{dest}|}\input{|\textit{main}|}"|
\end{center}
%

%%%%%%%%%%%%%%%%%%%%%%%%%%%%%%%%%%%%%%%%%%%%%%%%%%%%%%%%%%%%%%%%%%%%%%%%%%%%%%%%
%%%%%%%%%%%%%%%%%%%%%%%%%%%%%%%%%%%%%%%%%%%%%%%%%%%%%%%%%%%%%%%%%%%%%%%%%%%%%%%%
\section{Information}

%%%%%%%%%%%%%%%%%%%%%%%%%%%%%%%%%%%%%%%%%%%%%%%%%%%%%%%%%%%%%%%%%%%%%%%%%%%%%%%%
\subsection{Copyright}

Copyright \copyright{} 2017--2018 Niklas Beisert

This work may be distributed and/or modified under the
conditions of the \LaTeX{} Project Public License, either version 1.3
of this license or (at your option) any later version.
The latest version of this license is in
  \url{http://www.latex-project.org/lppl.txt}
and version 1.3 or later is part of all distributions of \LaTeX{}
version 2005/12/01 or later.

This work has the LPPL maintenance status `maintained'.

The Current Maintainer of this work is Niklas Beisert.

This work consists of the files |README.txt|, |childdoc.ins| and |childdoc.dtx|
as well as the derived files |childdoc.def|, |cdocsamp.tex|
with |cdocsch1.tex|, |cdocsch2.tex|, |cdocspt3.tex|, |cdocspt4.tex|,
|cdocsdrf.tex|, |cdocsfn1.tex|, |cdocsfn2.tex|
as well as |childdoc.pdf|.

%%%%%%%%%%%%%%%%%%%%%%%%%%%%%%%%%%%%%%%%%%%%%%%%%%%%%%%%%%%%%%%%%%%%%%%%%%%%%%%%
\subsection{Files and Installation}

The package consists of the files:
%
\begin{center}
\begin{tabular}{ll}
    |README.txt|   & readme file \\
    |childdoc.ins| & installation file \\
    |childdoc.dtx| & source file \\
    |childdoc.def| & definition file \\
    |cdocsamp.tex| & sample main file \\
    |cdocsch1.tex| & sample include file \\
    |cdocsch2.tex| & sample include file \\
    |cdocspt3.tex| & sample part file \\
    |cdocspt4.tex| & sample part file \\
    |cdocsdrf.tex| & sample redirection file \\
    |cdocsfn1.tex| & sample redirection file \\
    |cdocsfn2.tex| & sample redirection file \\
    |childdoc.pdf| & manual
\end{tabular}
\end{center}
%
The distribution consists of the files
|README.txt|, |childdoc.ins| and |childdoc.dtx|.
%
\begin{itemize}
\item
Run (pdf)\LaTeX{} on |childdoc.dtx|
to compile the manual |childdoc.pdf| (this file).
\item
Run \LaTeX{} on |childdoc.ins| to create the definitions file |childdoc.def|
and the sample |cdocsamp.tex| with include files
|cdocsch1.tex|, |cdocsch2.tex|, |cdocspt3.tex|, |cdocspt4.tex|,
|cdocsdrf.tex|, |cdocsfn1.tex|, |cdocsfn2.tex|.
Then copy the file |childdoc.def| to an appropriate directory of your \LaTeX{}
distribution, e.g.\ \textit{texmf-root}|/tex/latex/childdoc|.
\end{itemize}

%%%%%%%%%%%%%%%%%%%%%%%%%%%%%%%%%%%%%%%%%%%%%%%%%%%%%%%%%%%%%%%%%%%%%%%%%%%%%%%%
\subsection{Related CTAN Packages}

There are several other packages which offer a similar functionality:
%
\begin{itemize}
\item
The packages
\href{http://ctan.org/pkg/docmute}{\textsf{docmute}},
\href{http://ctan.org/pkg/includex}{\textsf{includex}} and
\href{http://ctan.org/pkg/standalone}{\textsf{standalone}}
provide commands to include only the document body of
a child file thus allowing both files to be compiled individually.
\item
The packages \href{http://ctan.org/pkg/subdocs}{\textsf{subdocs}}
and \href{http://ctan.org/pkg/subfiles}{\textsf{subfiles}}
provide structures in which the main and child documents can be
encapsulated and allowing them to be compiled individually.
The inclusion mechanism is different from the conventional |\include|.
\item
The package \href{http://ctan.org/pkg/combine}{\textsf{combine}}
is an elaborate solution to combine several documents into one.
\end{itemize}
%
See also the CTAN topic \href{http://ctan.org/topic/subdocs}{\textsf{subdocs}}
for further related packages.
The present package differs from the above solutions in that
a document structure constructed with the conventional |\include| mechanism
just needs two extra commands at the top of every file
such that all constituent files can be compiled individually.

%%%%%%%%%%%%%%%%%%%%%%%%%%%%%%%%%%%%%%%%%%%%%%%%%%%%%%%%%%%%%%%%%%%%%%%%%%%%%%%%
%\subsection{Feature Suggestions}
%
%The following is a list of features which may be useful for future
%versions of this package:
%%
%\begin{itemize}
%\item
%\ldots
%\end{itemize}

%%%%%%%%%%%%%%%%%%%%%%%%%%%%%%%%%%%%%%%%%%%%%%%%%%%%%%%%%%%%%%%%%%%%%%%%%%%%%%%%
\subsection{Revision History}

%%%%%%%%%%%%%%%%%%%%%%%%%%%%%%%%%%%%%%%%
\paragraph{v2.0:} 2018/12/30

\begin{itemize}
\item
immediate forward processing
\item
added |\childdocby| mechanism
\item
manual restructured
\end{itemize}

%%%%%%%%%%%%%%%%%%%%%%%%%%%%%%%%%%%%%%%%
\paragraph{v1.6:} 2018/01/17

\begin{itemize}
\item
application for development of include files
\item
corrections to manual
\end{itemize}

%%%%%%%%%%%%%%%%%%%%%%%%%%%%%%%%%%%%%%%%
\paragraph{v1.5:} 2017/05/21

\begin{itemize}
\item
more complete structuring introduced
\item
|\childdocof| introduced
\item
|\childdoc| renamed to |\childdocmain|
\item
|\childredirect| renamed to |\childdocforward| and |\childdocforwardprefix|
and functionality expanded
\end{itemize}

%%%%%%%%%%%%%%%%%%%%%%%%%%%%%%%%%%%%%%%%
\paragraph{v1.0:} 2017/04/27

\begin{itemize}
\item
manual and install package
\item
first version published on CTAN
\end{itemize}

%%%%%%%%%%%%%%%%%%%%%%%%%%%%%%%%%%%%%%%%
\paragraph{v0.6:} 2017/04/26

\begin{itemize}
\item
redirection mechanism added
\end{itemize}

%%%%%%%%%%%%%%%%%%%%%%%%%%%%%%%%%%%%%%%%
\paragraph{v0.5:} 2017/04/26

\begin{itemize}
\item
functionality in definition file
\end{itemize}


%%%%%%%%%%%%%%%%%%%%%%%%%%%%%%%%%%%%%%%%%%%%%%%%%%%%%%%%%%%%%%%%%%%%%%%%%%%%%%%%
%%%%%%%%%%%%%%%%%%%%%%%%%%%%%%%%%%%%%%%%%%%%%%%%%%%%%%%%%%%%%%%%%%%%%%%%%%%%%%%%
%%%%%%%%%%%%%%%%%%%%%%%%%%%%%%%%%%%%%%%%%%%%%%%%%%%%%%%%%%%%%%%%%%%%%%%%%%%%%%%%
\appendix

\settowidth\MacroIndent{\rmfamily\scriptsize 000\ }

 \DocInput{childdoc.dtx}

\end{document}
%</driver>
% \fi
%
% %%%%%%%%%%%%%%%%%%%%%%%%%%%%%%%%%%%%%%%%%%%%%%%%%%%%%%%%%%%%%%%%%%%%%%%%%%%%%%
% %%%%%%%%%%%%%%%%%%%%%%%%%%%%%%%%%%%%%%%%%%%%%%%%%%%%%%%%%%%%%%%%%%%%%%%%%%%%%%
% \section{Sample}
%\iffalse
%<*samplemain>
%\fi
%
% The following presents a sample document
% with two chapters, two parts, a title page,
% a compile flag as well as three forwarding files to set the flag.
% It consists of eight |.tex| files:
% \begin{center}
% \begin{tabular}{ll}
% |cdocsamp.tex|&main file\\
% |cdocsch1.tex|&include file for chapter 1\\
% |cdocsch2.tex|&include file for chapter 2\\
% |cdocspt3.tex|&include file for part 3\\
% |cdocspt4.tex|&include file for part 4\\
% |cdocsdrf.tex|&forwarding file for main file in draft mode\\
% |cdocsfi1.tex|&forwarding file for final version of chapter 1\\
% |cdocsfi2.tex|&forwarding file for final version of chapter 2\\
% \end{tabular}
% \end{center}
% Each of the eight files can be compiled directly by the \LaTeX{} compiler.
%
% %%%%%%%%%%%%%%%%%%%%%%%%%%%%%%%%%%%%%%
% \paragraph{Main File.}
%
% The main file is called |cdocsamp.tex|.
%
% Load the \textsf{childdoc} definitions and
% declare the filename for the main document:
%    \begin{macrocode}
\input{childdoc.def}
\childdocmain{}
%    \end{macrocode}

% Optional override for |\version| flag:
%    \begin{macrocode}
%%\ifchilddoc\else\providecommand{\version}{draft}\fi
%    \end{macrocode}

% Define the default values for the |\version| flag
% (|final| for the main file and |draft| for childs):
%    \begin{macrocode}
\ifchilddoc
\providecommand{\version}{draft}
\else
\providecommand{\version}{final}
\fi
%    \end{macrocode}

% Load the standard document class:
%    \begin{macrocode}
\documentclass[12pt]{article}
%    \end{macrocode}

% Start the document body:
%    \begin{macrocode}
\begin{document}
%    \end{macrocode}

% Declare a title page.
% Print title, part of document being processed and version flag:
%    \begin{macrocode}
\addtocounter{page}{-1}
\begin{center}
{\LARGE\bfseries{}childdoc example\par}
\vspace{1cm}
\ifchilddoc
\ifchilddocmanual part\else chapter\fi:
`\childdocname' of `\childdocjob'\par
\else
main document: `\childdocjob'\par
\fi
version: \version\par
\end{center}
\newpage
%    \end{macrocode}

% Manually include selected file,
% otherwise process as usual:
%    \begin{macrocode}
\ifchilddocmanual
\section*{part `\childdocname'}
\input{\childdocname}
\else
%    \end{macrocode}

% Include the two chapters:
%    \begin{macrocode}
\include{cdocsch1}
\include{cdocsch2}
%    \end{macrocode}

% Include the two parts unless only chapters should be displayed:
%    \begin{macrocode}
\ifchilddoc\else
\section{part three}
\input{cdocspt3}
\section{part four}
\input{cdocspt4}
\fi
%    \end{macrocode}

% Process as usual until here:
%    \begin{macrocode}
\fi
%    \end{macrocode}

% End of document body:
%    \begin{macrocode}
\end{document}
%    \end{macrocode}
%\iffalse
%</samplemain>
%\fi
%
% %%%%%%%%%%%%%%%%%%%%%%%%%%%%%%%%%%%%%%
% \paragraph{Chapter Include Files.}
%
% The include files are called |cdocsch1.tex| and |cdocsch2.tex|.
%
%\iffalse
%<*samplechap1|samplechap2>
%\fi

% Optional override for |\version| flag:
%    \begin{macrocode}
%%\providecommand{\version}{final}
%    \end{macrocode}

% Include the main document:
%    \begin{macrocode}
\input{childdoc.def}
\childdocof{cdocsamp}
%    \end{macrocode}

%\iffalse
%</samplechap1|samplechap2>
%\fi
%
%\iffalse
%<*samplechap1>
%\fi
% Some text for chapter 1:
%    \begin{macrocode}
\section{one}
some text in chapter one
%    \end{macrocode}

%\iffalse
%</samplechap1>
%\fi
% Some text for chapter 2:
%\iffalse
%<*samplechap2>
%\fi
%    \begin{macrocode}
\section{two}
more text in chapter two
%    \end{macrocode}

%\iffalse
%</samplechap2>
%\fi
%
% %%%%%%%%%%%%%%%%%%%%%%%%%%%%%%%%%%%%%%
% \paragraph{Part Include Files.}
%
% The include files are called |cdocspt3.tex| and |cdocspt4.tex|.
%
%\iffalse
%<*samplepart3|samplepart4>
%\fi

% Optional override for |\version| flag:
%    \begin{macrocode}
%%\providecommand{\version}{final}
%    \end{macrocode}

% Include the main document:
%    \begin{macrocode}
\input{childdoc.def}
\childdocby{cdocsamp}
%    \end{macrocode}

%\iffalse
%</samplepart3|samplepart4>
%\fi
%
%\iffalse
%<*samplepart3>
%\fi
% Some text for part 3:
%    \begin{macrocode}
some text in part three
%    \end{macrocode}

%\iffalse
%</samplepart3>
%\fi
% Some text for part 4:
%\iffalse
%<*samplepart4>
%\fi
%    \begin{macrocode}
more text in part four
%    \end{macrocode}

%\iffalse
%</samplepart4>
%\fi
%
% %%%%%%%%%%%%%%%%%%%%%%%%%%%%%%%%%%%%%%
% \paragraph{Forwarding for a Complete Draft.}
%
% The following forwarding file |cdocsdrf.tex|
% compiles the main document in draft mode:
%\iffalse
%<*sampledraft>
%\fi
%    \begin{macrocode}
\def\version{draft}
\input{childdoc.def}
\childdocforward{cdocsamp}
%    \end{macrocode}

%\iffalse
%</sampledraft>
%\fi
%
% %%%%%%%%%%%%%%%%%%%%%%%%%%%%%%%%%%%%%%
% \paragraph{Forwarding for Final Version of the Chapters.}
%
% The following forwarding files |cdocsfn1.tex| and |cdocsfn2.tex|
% (with identical content)
% compile the final versions of the child documents
% |cdocsch1.tex| and |cdocsch2.tex|, respectively:
%\iffalse
%<*samplefinal>
%\fi
%    \begin{macrocode}
\def\version{final}
\input{childdoc.def}
\childdocforwardprefix[cdocsamp]{cdocsfn}{cdocsch}
%    \end{macrocode}

%\iffalse
%</samplefinal>
%\fi
%
% %%%%%%%%%%%%%%%%%%%%%%%%%%%%%%%%%%%%%%
% \paragraph{Command Line Processing.}
%
% The following three command lines generate the output files
% |cdocscld|, |cdocscl1| and |cdocscl2|
% which should be identical to
% |cdocsdrf|, |cdocsch1| and |cdocsfn2|, respectively:
% \begin{center}
% \begin{tabular}{l}
% |latex -jobname cdocscld \|\\
% |  "\def\version{draft}\input{childdoc.def}\childdocforward{cdocsamp}"|\\
% |latex -jobname cdocscl1 \|\\
% |  "\input{childdoc.def}\childdocforward[cdocsamp]{cdocsch1}"|\\
% |latex -jobname cdocscl2 \|\\
% |  "\def\version{final}\input{childdoc.def}\childdocforward{cdocsch2}"|
% \end{tabular}
% \end{center}
% Note that the trailing backslash on each first line
% merely continues the input to the second line
% (for convenient cut ant paste).
% Furthermore, the command |latex| can be replaced by any
% of its alternative versions such as |pdflatex|.
%
% %%%%%%%%%%%%%%%%%%%%%%%%%%%%%%%%%%%%%%%%%%%%%%%%%%%%%%%%%%%%%%%%%%%%%%%%%%%%%%
% %%%%%%%%%%%%%%%%%%%%%%%%%%%%%%%%%%%%%%%%%%%%%%%%%%%%%%%%%%%%%%%%%%%%%%%%%%%%%%
% \section{Implementation}
%\iffalse
%<*package>
%\fi
%
% This section describes the definitions file |childdoc.def|.

% The definitions cannot be loaded using |\usepackage| or |\RequirePackage|
% which has a mechanism to prevent loading a style file more than once.
% When loading the definitions by means of |\input|
% multiple instances have to be prevented manually:
%\iffalse
%This code needs to be before the `\ProvidesFile' directive
%which is defined at the beginning of this file.
%Therefore it is also placed there and commented out here.
%</package>
%<*discard>
%\fi
%    \begin{macrocode}
\ifdefined\childdocmain\endinput\fi
%    \end{macrocode}
%\iffalse
%</discard>
%<*package>
%\fi
%
% \macro{\ifchilddoc}
% \macro{\ifchilddocmanual}
% The conditional |\ifchilddoc| tells whether a
% child (true) or main (false) document is being compiled.
% The conditional |\ifchilddocmanual| tells whether
% the |\includeonly| mechanism is used (false) or
% the selection of child files must be performed manually (true).
% The definitions initialise to false:
%    \begin{macrocode}
\newif\ifchilddoc
\newif\ifchilddocmanual
%    \end{macrocode}

% \macro{\childdocname}
% \macro{\childdocjob}
% The macro |\childdocname| stores the name of the main document
% to be compiled. The macro |\childdocjob| stores the name of
% the document on which the \LaTeX{} compiler was originally invoked.
% The content of |\jobname| cannot be compared
% to filenames specified in the source due to different catcodes.
% The following code rescans |\jobname|, stores the result
% in |\childdocname| and saves a copy in |\childdocjob|:
%    \begin{macrocode}
\edef\childdocname{\scantokens\expandafter{\jobname\noexpand}}
\let\childdocjob\childdocname
%    \end{macrocode}

% \macro{\childdocdisable}
% The macro |\childdocdisable| prevents the main file
% from being processed more than once.
% At this stage, the main document command |\childdocmain|
% is assumed to be called once again where it should do nothing.
% Any subsequent call to it should prevent
% a secondary processing of the main document
% It overwrites the forwarding commands
% |\childdocof| and |\childdocforward|
% with empty macros to prevent further inclusions of the main document:
%    \begin{macrocode}
\newcommand{\childdocdisable}
{
  \renewcommand{\childdocmain}[1]{\renewcommand{\childdocmain}[1]{\endinput}}
  \renewcommand{\childdocof}[1]{}
  \renewcommand{\childdocby}[2][]{}
  \renewcommand{\childdocforward}[2][]{}
  \renewcommand{\childdocdisable}{}
}
%    \end{macrocode}

% \macro{\childdocmain}
% The macro |\childdocmain| is to be called at the top of the main file
% with nothing or the main filename (without extension) as argument.
% First, it breaks loops.
% If the argument is not empty and does not match |\childdocname|
% (which is set by the first inclusion of |childdoc.def|),
% |\ifchilddoc| is set to true, |\includeonly| is applied to the child file
% and |\jobname| is set to the main file
% (for proper handling of |.aux| files):
%    \begin{macrocode}
\newcommand{\childdocmain}[1]
{
  \childdocdisable\childdocmain{}
  \if?#1?\else
    \begingroup
      \def\childdoctmp{#1}
      \ifx\childdoctmp\childdocname
        \def\childdoctmp{}
      \else
        \def\childdoctmp
        {
          \childdoctrue
          \includeonly{\childdocname}
          \def\childdocjob{#1}
          \def\jobname{#1}
        }
      \fi
      \expandafter
    \endgroup
    \childdoctmp
  \fi
}
%    \end{macrocode}

% \macro{\childdocof}
% The command |\childdocof| redirects
% compilation to the main file |#1|.
%    \begin{macrocode}
\newcommand{\childdocof}[1]
{
  \childdocdisable
  \childdoctrue
  \includeonly{\childdocname}
  \def\jobname{#1}
  \def\childdocjob{#1}
  \input{#1}
}
%    \end{macrocode}

% \macro{\childdocby}
% The command |\childdocby| ....
%    \begin{macrocode}
\newcommand{\childdocby}[2][]
{
  \childdocdisable
  \childdoctrue
  \childdocmanualtrue
  \if?#1?\else
    \def\jobname{#2}
  \fi
  \def\childdocjob{#2}
  \input{#2}
  \endinput
}
%    \end{macrocode}

% \macro{\childdocforward}
% The command |\childdocforward| redirects
% compilation to the main file or
% (if the optional argument is given) a child file.
% Parameters are set as if the main file
% or a child file starting with |\childdocof| was compiled.
% Then compilation is handed over to the main file:
%    \begin{macrocode}
\newcommand{\childdocforward}[2][]
{
  \begingroup
    \if?#1?
      \def\childdoctmp
      {
        \def\childdocname{#2}
        \def\childdocjob{#2}
        \def\jobname{#2}
        \input{#2}
        \endinput
      }
    \else
      \def\childdoctmp
      {
        \childdocdisable
        \def\childdocname{#2}
        \childdoctrue
        \includeonly{#2}
        \def\childdocjob{#1}
        \def\jobname{#1}
        \input{#1}
        \endinput
      }
    \fi
    \expandafter
  \endgroup
  \childdoctmp
}
%    \end{macrocode}

% \macro{\childdocforwardprefix}
% The command |\childdocforwardprefix| redirects
% compilation to the main or a child file by means of a pattern.
% The prefix |#1| in the current filename is replaced by |#2|
% and the suffix of the current filename is kept
% (it is assumed that the filename does not contain the substring `|~~~|'
% which is used as a delimiter).
% Compilation is handed over to the new file by |\childdocforward|:
%    \begin{macrocode}
\newcommand{\childdocforwardprefix}[3][]
{
  \begingroup
    \def\childdocextract #2##1~~~{\def\childdoctmp{\childdocforward[#1]{#3##1}}}
    \expandafter\childdocextract\childdocname~~~
    \expandafter
  \endgroup
  \childdoctmp
}
%    \end{macrocode}

% \macro{\childdoc}
% The deprecated macro |\childdoc| is a legacy version of |\childdocmain|:
%    \begin{macrocode}
\newcommand{\childdoc}{\childdocmain}
%    \end{macrocode}

% \macro{\childdocredirect}
% The deprecated macro |\childdocredirect| is a legacy version
% of |\childdocforward| and |\childdocforwardprefix|:
%    \begin{macrocode}
\newcommand{\childdocredirect}[2][]
{
  \begingroup
    \if?#1?
      \def\childdoctmp{\childdocforward{#2}}
    \else
      \def\childdoctmp{\childdocforwardprefix{#1}{#2}}
    \fi
    \expandafter
  \endgroup
  \childdoctmp
}
%    \end{macrocode}

%\iffalse
%</package>
%\fi
%
\endinput
\childdocforward[|\textit{main}|]{|\textit{dest}|}"|
\end{center}
%
Here \textit{target} is the name of the output file,
\textit{main} is the name of the main file
and \textit{dest} is the name of the main or child file to be processed
(all filenames without extensions).
The optional argument \textit{main} can be omitted
if \textit{main} matches \textit{dest}.
Optionally, compilation \textit{flags} can be defined via |\def| commands.
This command line makes the \TeX{} engine believe
it is compiling the file \textit{target}
whose content is specified as the latter parameter.
The provided code then forwards the processing to
\textit{main} or \textit{dest} as described in \secref{sec:forward}.

%%%%%%%%%%%%%%%%%%%%%%%%%%%%%%%%%%%%%%%%%%%%%%%%%%%%%%%%%%%%%%%%%%%%%%%%%%%%%%%%
\subsection{Include by Input}
\label{sec:input}

Including child documents by |\include| has some restrictions by design.
Most notably, the content of a child document always occupies
its own set of pages; pages cannot be shared between child documents.
Usually, this behaviour makes perfect sense
because each child document contain an essential part of the document.
However, in some situations it may be desirable to compose
a document from a collection of parts
without having mandatory page breaks between then.
For this case, the package
provides a mechanism to include parts
by |\input| which can also be processed individually.
However, by construction this mechanism
requires manual handling of the content to be output.

%%%%%%%%%%%%%%%%%%%%%%%%%%%%%%%%%%%%%%%%
\DescribeMacro{\ifchilddocmanual}
The main file should be prepared as usual, see \secref{sec:include}.
However, the document body must make a distinction
between processing of an individual part and of the main document, e.g.:
%
\begin{center}
\begin{tabular}{l}
|\ifchilddocmanual|\\
|\input{\childdocname}|\\
|\||else|\\
\textit{document body with }|\input{|\textit{part}|}|\\
|\||fi|
\end{tabular}
\end{center}
%
The conditional |\ifchilddocmanual| is true whenever
a part to be included by |\input| is being compiled,
and the name of the part is stored in |\childdocname|.

%%%%%%%%%%%%%%%%%%%%%%%%%%%%%%%%%%%%%%%%
\DescribeMacro{\childdocby}
Each part to be included by |\input| should start with:
%
\begin{center}
\begin{tabular}{l}
|% \iffalse
%
% childdoc.dtx Copyright (C) 2017-2018 Niklas Beisert
%
% This work may be distributed and/or modified under the
% conditions of the LaTeX Project Public License, either version 1.3
% of this license or (at your option) any later version.
% The latest version of this license is in
%   http://www.latex-project.org/lppl.txt
% and version 1.3 or later is part of all distributions of LaTeX
% version 2005/12/01 or later.
%
% This work has the LPPL maintenance status `maintained'.
%
% The Current Maintainer of this work is Niklas Beisert.
%
% This work consists of the files childdoc.dtx and childdoc.ins
% and the derived files childdoc.def and cdocsamp.tex with
% cdocsch1.tex, cdocsch2.tex, cdocsdrf.tex, cdocsfn1.tex, cdocsfn2.tex.
%
%<package>\ifdefined\childdocmain\endinput\fi
%<package>\ProvidesFile{childdoc.def}[2018/12/30 v2.0 child document driver]
%<samplemain>\ProvidesFile{cdocsamp.tex}[2018/12/30 v2.0 sample for childdoc]
%<*driver>
%\ProvidesFile{childdoc.drv}[2018/12/30 v2.0 childdoc reference manual file]
\PassOptionsToClass{10pt,a4paper}{article}
\documentclass{ltxdoc}

\usepackage[margin=35mm]{geometry}
\usepackage{hyperref}
\usepackage{hyperxmp}
\usepackage[usenames]{color}

\hypersetup{colorlinks=true}
\hypersetup{pdfstartview=FitH}
\hypersetup{pdfpagemode=UseNone}
\hypersetup{pdfsource={}}
\hypersetup{pdflang={en-UK}}
\hypersetup{pdfcopyright={Copyright 2017-2018 Niklas Beisert.
  This work may be distributed and/or modified under the
  conditions of the LaTeX Project Public License, either version 1.3
  of this license or (at your option) any later version.}}
\hypersetup{pdflicenseurl={http://www.latex-project.org/lppl.txt}}
\hypersetup{pdfcontactaddress={ETH Zurich, ITP, HIT K,
  Wolfgang-Pauli-Strasse 27}}
\hypersetup{pdfcontactpostcode={8093}}
\hypersetup{pdfcontactcity={Zurich}}
\hypersetup{pdfcontactcountry={Switzerland}}
\hypersetup{pdfcontactemail={nbeisert@itp.phys.ethz.ch}}
\hypersetup{pdfcontacturl={http://people.phys.ethz.ch/\xmptilde nbeisert/}}

\newcommand{\secref}[1]{\hyperref[#1]{section \ref*{#1}}}

\parskip1ex
\parindent0pt
\let\olditemize\itemize
\def\itemize{\olditemize\parskip0pt}

\begin{document}

\title{The \textsf{childdoc} Package}
\hypersetup{pdftitle={The childdoc Package}}
\author{Niklas Beisert\\[2ex]
  Institut f\"ur Theoretische Physik\\
  Eidgen\"ossische Technische Hochschule Z\"urich\\
  Wolfgang-Pauli-Strasse 27, 8093 Z\"urich, Switzerland\\[1ex]
  \href{mailto:nbeisert@itp.phys.ethz.ch}
  {\texttt{nbeisert@itp.phys.ethz.ch}}}
\hypersetup{pdfauthor={Niklas Beisert}}
\hypersetup{pdfsubject={Manual for the LaTeX2e Package childdoc}}
\date{30 December 2018, \textsf{v2.0}}
\maketitle

\begin{abstract}\noindent
\textsf{childdoc} is a \LaTeXe{} package
that enables the direct compilation
of document sections included by |\include|
to individual files.
\end{abstract}

\begingroup
\parskip0ex
\tableofcontents
\endgroup

%%%%%%%%%%%%%%%%%%%%%%%%%%%%%%%%%%%%%%%%%%%%%%%%%%%%%%%%%%%%%%%%%%%%%%%%%%%%%%%%
%%%%%%%%%%%%%%%%%%%%%%%%%%%%%%%%%%%%%%%%%%%%%%%%%%%%%%%%%%%%%%%%%%%%%%%%%%%%%%%%
\section{Introduction}

\LaTeX{} provides a mechanism to structure a large document (such as a book)
into a main file and several child files (containing the chapters)
using the |\include| command.
This mechanism is beneficial for documents
which span hundreds of pages in order to
make the source file(s) more manageable.
Moreover, compilation can be restricted to
selected child files by means of the |\includeonly| command.
The latter feature can be used to reduce the compilation time while editing
(this was significantly more useful in the earlier days of \LaTeX{})
or to generate a smaller document which is easier to navigate.
Another application of |\includeonly| is to generate
documents consisting of selected parts of the complete document.

However, there are a few drawbacks of the plain |\include| mechanism:
\begin{itemize}
\item
The child files cannot be compiled on their own,
they can only be compiled via the main file.
A naive editing environment
(such as a text editor with an option
to have the current file processed by \LaTeX)
may require one to switch to the main file before compiling;
attempting to compile the child file produces errors.
\item
The main file must be modified (each time)
to adjust the |\includeonly| command
to the present needs. This easily leaves the main file in a messy state.
\item
The generated document will always carry the filename
of the main document. This is inconvenient if
several child files are to be compiled and
to be kept for distribution.
\end{itemize}

The present package provides a simple interface
to make child files individually compilable by \LaTeX{}.
Compiling a child file then has the same effect as compiling
the main file with an |\includeonly| command
to select the appropriate child.
Moreover the generated document will carry the name of the child
rather than the main file.
This resolves all three above issues.

This feature is meant to make the editing of books,
thesis documents and lecture notes somewhat more convenient.
However, the package can also be used efficiently for
composing a series of documents (such as exercise sheets)
which are typically distributed individually.
It then assists the author in generating the individual documents
(potentially in different versions)
as well as a document containing the collected series.
Another application is in developing style files
or other kinds of included material
where compilation of the style file could redirect
to a sample or test file.

%%%%%%%%%%%%%%%%%%%%%%%%%%%%%%%%%%%%%%%%%%%%%%%%%%%%%%%%%%%%%%%%%%%%%%%%%%%%%%%%
%%%%%%%%%%%%%%%%%%%%%%%%%%%%%%%%%%%%%%%%%%%%%%%%%%%%%%%%%%%%%%%%%%%%%%%%%%%%%%%%
\section{Usage}

First of all, the package \textsf{childdoc} is \emph{not} a standard
\LaTeXe{} |.sty| style file! Therefore it needs to be invoked in
a non-standard way.

%%%%%%%%%%%%%%%%%%%%%%%%%%%%%%%%%%%%%%%%%%%%%%%%%%%%%%%%%%%%%%%%%%%%%%%%%%%%%%%%
\subsection{Included Files}
\label{sec:include}

%%%%%%%%%%%%%%%%%%%%%%%%%%%%%%%%%%%%%%%%
\DescribeMacro{\childdocmain}
To use the package, add the commands
\begin{center}
\begin{tabular}{l}
|\input{childdoc.def}|\\
|\childdocmain{}|\\
\end{tabular}
\end{center}
at the very top of the main \LaTeX{} file,
in particular \emph{before} the |\documentclass| statement!
The argument of |\childdocmain| should be left empty
(but it must be present).

%%%%%%%%%%%%%%%%%%%%%%%%%%%%%%%%%%%%%%%%
\DescribeMacro{\childdocof}
Furthermore, add the commands
\begin{center}
\begin{tabular}{l}
|\input{childdoc.def}|\\
|\childdocof{|\textit{main}|}|\\
\end{tabular}
\end{center}
at the top of every child file \textit{child}
which is included by |\include{|\textit{child}|}|
from within the main file
(or at least for those files to be compiled individually).
The argument \textit{main} must be the filename of the main file.

There are a couple of
considerations in setting up the main and child documents:

%%%%%%%%%%%%%%%%%%%%%%%%%%%%%%%%%%%%%%%%
\paragraph{Restrictions.}

Please note the following restrictions:
\begin{itemize}
\item
|\childdocmain| must be called with one argument \textit{main}
to ensure compatibility with earlier version of the package.
It must either be empty (|\childdocmain{}|)
or precisely match the filename of the main file in which it is specified.
See \secref{sec:detection} for further information.
\item
The filename \textit{main} must be specified without the |.tex| extension.
\item
The filename \textit{main} is case sensitive
(even in case-insensitive file systems)
due to internal string comparison.
\item
The argument \textit{main} should be fully expanded, it cannot be a macro.
\item
Subdirectories and special characters should be avoided in filenames.
\item
The command |\childdocmain{|\textit{main}|}| must be followed by a whitespace.
It should not be followed immediately by another command
or by a comment mark `|%|'.
This is because the \TeX{} parser reads the token immediately following
the argument of |\childdocmain| and puts it
at the beginning of every child section;
however, a white\-space is ignored.
\end{itemize}

%%%%%%%%%%%%%%%%%%%%%%%%%%%%%%%%%%%%%%%%
\paragraph{Content of Main File.}

It is advisable to place all content in the child files included by |\include|.
Any output contained in the main file will appear in all child documents
unless suppressed manually;
it cannot be suppressed automatically by the |\includeonly| directive
and thus should normally be avoided.
A method to include some content in the main file
by means of conditional processing is described in \secref{sec:conditional}.

%%%%%%%%%%%%%%%%%%%%%%%%%%%%%%%%%%%%%%%%
\paragraph{Page Numbering.}

When only a part of the document is compiled,
the appropriate numbering of pages
(as well as other status parameters)
is determined from the |.aux| files.
The latter contain information from previous passes.
However this information needs to propagate through
all intermediate child documents.
Therefore the page numbering in child documents may well
be inconsistent until the complete document is compiled at least once.

A useful (if unconventional) way to always ensure a consistent
page numbering is to restart the numbering in each child document
and denote the pages by `\textit{child}|.|\textit{page}'
where \textit{child} represents the chapter/section number of the child file.
This can be achieved by the command
|\numberwithin{page}{|\textit{child}|}|
of the \textsf{amsmath} package
where \textit{child} can be |chapter| or |section|
depending on the chosen structuring.
Alternatively, one can modify the macro |\thepage| appropriately
and reset the counter |page| at the start of each child file.

%%%%%%%%%%%%%%%%%%%%%%%%%%%%%%%%%%%%%%%%%%%%%%%%%%%%%%%%%%%%%%%%%%%%%%%%%%%%%%%%
\subsection{Conditional Processing}
\label{sec:conditional}

The package provides a mechanism to compile different versions
of a document. To customise the versions further some conditional processing
can come in handy to distinguish which version is being compiled.
The package provides two macros to describe the compilation context:

%%%%%%%%%%%%%%%%%%%%%%%%%%%%%%%%%%%%%%%%
\DescribeMacro{\ifchilddoc}
The conditional |\ifchilddoc| distinguishes between the compilation of
child documents and the main document:
%
\begin{center}
|\ifchilddoc |\textit{child-code}| |[|\||else |\textit{main-code}]| \||fi|
\end{center}

%%%%%%%%%%%%%%%%%%%%%%%%%%%%%%%%%%%%%%%%
\DescribeMacro{\childdocname}
\DescribeMacro{\childdocjob}
The macro |\childdocname| contains the filename (without extension)
of the main or child file being processed.
Note that |\childdocjob| will always contain the name of the main file.

%%%%%%%%%%%%%%%%%%%%%%%%%%%%%%%%%%%%%%%%
\paragraph{Title Page.}

Conditional processing can be used to include a title or banner page
in the main document when proper precautions are taken.
Importantly, the code in the main file should ensure that the page counter
(as well as other status parameters which are stored in the |.aux| files)
takes the same value after the conditional processing.
Otherwise the page numbers may take divergent values
depending on which part is compiled.

For example, a title page could be declared by:
%
\begin{center}
\begin{tabular}{l}
|\ifchilddoc\||else|\\
|\addtocounter{page}{-1}|\\
\textit{code for title page}\\
|\newpage|\\
|\||fi|
\end{tabular}
\end{center}
%
A banner page for the child documents can be generated by:
%
\begin{center}
\begin{tabular}{l}
|\ifchilddoc|\\
|\addtocounter{page}{-1}|\\
\textit{code for banner page}\\
|\newpage|\\
|\||fi|
\end{tabular}
\end{center}
%
Here one could write a message such as:
\begin{center}
|This is the part \childdocname{} of \childdocjob{}.|
\end{center}

%%%%%%%%%%%%%%%%%%%%%%%%%%%%%%%%%%%%%%%%%%%%%%%%%%%%%%%%%%%%%%%%%%%%%%%%%%%%%%%%
\subsection{Flags}
\label{sec:flags}

The package makes it easy to generate different versions
of the main or child documents.
To this end compilation flags can be defined
and assigned different default values.
They will be particularly useful in conjunction
with the forwarding mechanism described in \secref{sec:forward}.

For example, it may be useful to have a flag |\version|
which can be set to |draft| or |final|.
The document source will contain some conditional code
depending on the value of |\version|.
Suppose further, the flag should default to |final| for the main file
and to |draft| for child files
which is a natural assignment for editing the document.
This is achieved by placing the following code
in the preamble of the main document
(below the |\childdocmain| directive):
%
\begin{center}
\begin{tabular}{l}
|\ifchilddoc|\\
|\providecommand{\version}{draft}|\\
|\||else|\\
|\providecommand{\version}{final}|\\
|\||fi|
\end{tabular}
\end{center}
%
The definition by |\providecommand| makes sure
that previous definitions are not overwritten.
Further statements |\providecommand{\version}{...}|
can thus be added before the above code to override it.

For the main file, one might add a line
(between |\childdocmain| and the above block)
%
\begin{center}
|%\ifchilddoc\||else\providecommand{\version}{draft}\||fi|
\end{center}
%
which can be uncommented to produce a draft version.
Likewise one can add a line to the very top of a child file
(above the |\childdocof{|\textit{main}|}| directive)
%
\begin{center}
|%\providecommand{\version}{final}|
\end{center}
%
which can be uncommented to produce the final version of this child document.

%%%%%%%%%%%%%%%%%%%%%%%%%%%%%%%%%%%%%%%%%%%%%%%%%%%%%%%%%%%%%%%%%%%%%%%%%%%%%%%%
\subsection{Forwarding}
\label{sec:forward}

Different versions of the main or child documents
using compilation flags as described in \secref{sec:flags}
can be (permanently) stored in different files
for convenient compilation, viewing and distribution.
To this end, the package defines a command
to pass on compilation to a different file:

%%%%%%%%%%%%%%%%%%%%%%%%%%%%%%%%%%%%%%%%
\DescribeMacro{\childdocforward}
The command |\childdocforward| redirects processing to
another source file:
%
\begin{center}
\begin{tabular}{l}
|\input{childdoc.def}|\\
|\childdocforward[|\textit{main}|]{|\textit{dest}|}|\\
\end{tabular}
\end{center}
%
The argument \textit{dest} is the destination file
(without extension).
It should be the main file or one of the child files.
Note that further \textsf{childdoc} directives
such as |\childdocof| and |\childdocforward|
in the indicated file will be processed in this form.
The optional argument \textit{main}
passes on directly to the main file \textit{main}
while pretending to compile the child \textit{dest}.
This form behaves as if \textit{dest}
issues |\childdocof{|\textit{main}|}| right away,
and no further \textsf{childdoc} directives will be processed.

%%%%%%%%%%%%%%%%%%%%%%%%%%%%%%%%%%%%%%%%
\DescribeMacro{\...prefix}
In the alternative form |\childdocforwardprefix|,
%
\begin{center}
\begin{tabular}{l}
|\input{childdoc.def}|\\
|\childdocforwardprefix[|\textit{main}|]{|\textit{prefix}|}{|\textit{dest}|}|
\end{tabular}
\end{center}
%
the destination file is determined by a pattern
depending on the current file:
To make this work, the current file must be called
`{\textit{prefix}\hspace{0.2em}\textit{suffix}}'
with \textit{prefix} matching precisely the argument.
Processing is then passed on to the file
`{\textit{dest}\hspace{0.2em}\textit{suffix}}'.
Surely, the same effect is achieved by
directly specifying the
argument `{\textit{dest}\hspace{0.2em}\textit{suffix}}'
in the first form.
However, that requires to set up a different file
for each child. With the alternative form of the command
all these files can have exactly the same content
which simplifies setting them up and maintaining them.

For example, the following file |draft.tex|
with a compilation flag |\version| as described in \secref{sec:flags}
compiles the main document as a draft:
%
\begin{center}
\begin{tabular}{l}
|\def\version{draft}|\\
|\input{childdoc.def}|\\
|\childdocforward{|\textit{main}|}|
\end{tabular}
\end{center}
%
Likewise, the following files |final|\textit{nn}|.tex|
compile the final version of the child document
|child|\textit{nn}|.tex|:
%
\begin{center}
\begin{tabular}{l}
|\def\version{final}|\\
|\input{childdoc.def}|\\
|\childdocforwardprefix{final}{child}|
\end{tabular}
\end{center}
%

Note that when several versions of a main file and/or of each child file
are to be generated, it may be convenient to set up a |Makefile| or
shell script to automatise the process.

%%%%%%%%%%%%%%%%%%%%%%%%%%%%%%%%%%%%%%%%%%%%%%%%%%%%%%%%%%%%%%%%%%%%%%%%%%%%%%%%
\subsection{Command Line Processing}
\label{sec:commandline}

The effect of redirection files can also be achieved by invoking
the \LaTeX{} compiler with a more elaborate command line.
Most conveniently this should be done as part
of a shell script or a |Makefile|.

When using \textsf{childdoc} in the main file, the following
command lines effectively perform a redirection
(note that depending on the shell being used,
backslashes may have to be doubled: `|\|' $\to$ `|\\|'):
%
\begin{center}
|... -jobname "|\textit{target}|" |\\|"|[\textit{flags}]%
|\input{childdoc.def}\childdocforward[|\textit{main}|]{|\textit{dest}|}"|
\end{center}
%
Here \textit{target} is the name of the output file,
\textit{main} is the name of the main file
and \textit{dest} is the name of the main or child file to be processed
(all filenames without extensions).
The optional argument \textit{main} can be omitted
if \textit{main} matches \textit{dest}.
Optionally, compilation \textit{flags} can be defined via |\def| commands.
This command line makes the \TeX{} engine believe
it is compiling the file \textit{target}
whose content is specified as the latter parameter.
The provided code then forwards the processing to
\textit{main} or \textit{dest} as described in \secref{sec:forward}.

%%%%%%%%%%%%%%%%%%%%%%%%%%%%%%%%%%%%%%%%%%%%%%%%%%%%%%%%%%%%%%%%%%%%%%%%%%%%%%%%
\subsection{Include by Input}
\label{sec:input}

Including child documents by |\include| has some restrictions by design.
Most notably, the content of a child document always occupies
its own set of pages; pages cannot be shared between child documents.
Usually, this behaviour makes perfect sense
because each child document contain an essential part of the document.
However, in some situations it may be desirable to compose
a document from a collection of parts
without having mandatory page breaks between then.
For this case, the package
provides a mechanism to include parts
by |\input| which can also be processed individually.
However, by construction this mechanism
requires manual handling of the content to be output.

%%%%%%%%%%%%%%%%%%%%%%%%%%%%%%%%%%%%%%%%
\DescribeMacro{\ifchilddocmanual}
The main file should be prepared as usual, see \secref{sec:include}.
However, the document body must make a distinction
between processing of an individual part and of the main document, e.g.:
%
\begin{center}
\begin{tabular}{l}
|\ifchilddocmanual|\\
|\input{\childdocname}|\\
|\||else|\\
\textit{document body with }|\input{|\textit{part}|}|\\
|\||fi|
\end{tabular}
\end{center}
%
The conditional |\ifchilddocmanual| is true whenever
a part to be included by |\input| is being compiled,
and the name of the part is stored in |\childdocname|.

%%%%%%%%%%%%%%%%%%%%%%%%%%%%%%%%%%%%%%%%
\DescribeMacro{\childdocby}
Each part to be included by |\input| should start with:
%
\begin{center}
\begin{tabular}{l}
|\input{childdoc.def}|\\
|\childdocby{|\textit{main}|}|\\
\end{tabular}
\end{center}
%
The directive |\childdocby| is similar to |\childdocof|
described in \secref{sec:include},
but the subsequent selection of content must be done manually.
To that end, both |\ifchilddoc| and |\ifchilddocmanual|
will be true upon processing of a part,
and the name of the part is stored in |\childdocname|.
Note that |\jobname| will be set to the filename of the current part
so that each part receives an individual |.aux| file
that does not interfere with the |.aux| file(s) of the main document.
This behaviour can be altered by the alternative form
|\childdocby[*]{|\textit{main}|}| (with a non-empty optional argument)
which uses the |.aux| file of the main document
by setting |\jobname| to \textit{main}.

%%%%%%%%%%%%%%%%%%%%%%%%%%%%%%%%%%%%%%%%%%%%%%%%%%%%%%%%%%%%%%%%%%%%%%%%%%%%%%%%
\subsection{Driver Development}
\label{sec:driver}

The \textsf{childdoc} mechanism can also be use for the development
of definition files such as \LaTeX{} styles or classes.
This case differs from the above setup with multiple parts
included by |\include| in that no |\includeonly| should be invoked.
This can be achieved by starting the include file
(before |\ProvidesPackage|) with:
%
\begin{center}
\begin{tabular}{l}
|\input{childdoc.def}|\\
|\childdocforward{|\textit{main}|}|\\
\end{tabular}
\end{center}
%
or alternatively with:
%
\begin{center}
\begin{tabular}{l}
|\input{childdoc.def}|\\
|\childdocby{|\textit{main}|}|\\
\end{tabular}
\end{center}
%
Both forms have slightly different effects as described above.
The main file is prepared as usual, see \secref{sec:include}.

%%%%%%%%%%%%%%%%%%%%%%%%%%%%%%%%%%%%%%%%%%%%%%%%%%%%%%%%%%%%%%%%%%%%%%%%%%%%%%%%
\subsection{Legacy Detection}
\label{sec:detection}

The directive |\childdocmain| in the main file can detect
whether the complete document or merely a child is to be compiled
even without using the directive |\childdocof|.
This method is deprecated because it is less robust
and there is no compelling reason to use it;
it is merely provided for backward compatibility
and it may be removed in future versions.

If the detection mechanism is to be used,
it is mandatory to correctly specify
the filename of the main file as the argument of |\childdocmain|:
%
\begin{center}
\begin{tabular}{l}
|\input{childdoc.def}|\\
|\childdocmain{|\textit{main}|}|\\
\end{tabular}
\end{center}
%
If |\jobname| does not match the argument \textit{main} of |\childdocmain|,
it is assumed that |\jobname| points to the child file to be compiled.
When using |\childdocmain| with the main file specified as argument,
it suffices to start a child file
with just |\input{|\textit{main}|}|
without loading of the package and using |\childdocof|.
If instead all processing is done
with the appropriate \textsf{childdoc} directives,
the argument of \textit{main} of |\childdocmain| can be empty.

An alternative version of the command line processing described
in \secref{sec:commandline} using the detection mechanism reads:
%
\begin{center}
|... -jobname "|\textit{target}|" "|[\textit{flags}]%
[|\def\jobname{|\textit{dest}|}|]|\input{|\textit{main}|}"|
\end{center}

%%%%%%%%%%%%%%%%%%%%%%%%%%%%%%%%%%%%%%%%%%%%%%%%%%%%%%%%%%%%%%%%%%%%%%%%%%%%%%%%
\subsection{Manual Code}
\label{sec:manual}

In case one cannot be certain whether the definitions file |childdoc.def|
is installed on the target \TeX{} distribution
and one prefers not to ship it,
it is conceivable to paste a few relevant commands into the sources.

To that end, drop all statements |\input{childdoc.def}|
and perform the replacements as outlined below.
Instead of |\childdocmain{|\textit{main}|}| add the following code
to the top of the main file:
%
\begin{center}
\begin{tabular}{l}
|\||ifdefined\childdocname\endinput\||fi\newif\ifchilddoc|\\
|\edef\childdocname{\scantokens\expandafter{\jobname\noexpand}}|\\
|\def\childdocmain{|\textit{main}|}\||ifx\childdocmain\childdocname\||else|\\
|\childdoctrue\includeonly{\childdocname}\let\jobname\childdocmain\||fi|\\
\end{tabular}
\end{center}
%
Instead of |\childdocof{|\textit{main}|}| just include the main file
at the top of each child file:
%
\begin{center}
|\input{|\textit{main}|}|
\end{center}
%
A simple redirection |\childdocforward{|\textit{dest}|}| is achieved by:
%
\begin{center}
|\def\jobname{|\textit{dest}|}\input{\jobname}|
\end{center}
%
The redirection with prefix
|\childdocforwardprefix[|\textit{prefix}|]{|\textit{dest}|}|
is accomplished by:
%
\begin{center}
\begin{tabular}{l}
|{\edef\jobname{\scantokens\expandafter{\jobname\noexpand}}|\\
|\def\redirectjob |\textit{prefix}|#1~~~{\gdef\jobname{|\textit{dest}|#1}}|\\
|\expandafter\redirectjob\jobname~~~}\input{\jobname}|
\end{tabular}
\end{center}

In an alternative approach,
child documents can be compiled by a specific command line
without additional code or specific definitions:
%
\begin{center}
|... -jobname "|\textit{target}|" "|[\textit{flags}]%
|\includeonly{|\textit{dest}|}\input{|\textit{main}|}"|
\end{center}
%

%%%%%%%%%%%%%%%%%%%%%%%%%%%%%%%%%%%%%%%%%%%%%%%%%%%%%%%%%%%%%%%%%%%%%%%%%%%%%%%%
%%%%%%%%%%%%%%%%%%%%%%%%%%%%%%%%%%%%%%%%%%%%%%%%%%%%%%%%%%%%%%%%%%%%%%%%%%%%%%%%
\section{Information}

%%%%%%%%%%%%%%%%%%%%%%%%%%%%%%%%%%%%%%%%%%%%%%%%%%%%%%%%%%%%%%%%%%%%%%%%%%%%%%%%
\subsection{Copyright}

Copyright \copyright{} 2017--2018 Niklas Beisert

This work may be distributed and/or modified under the
conditions of the \LaTeX{} Project Public License, either version 1.3
of this license or (at your option) any later version.
The latest version of this license is in
  \url{http://www.latex-project.org/lppl.txt}
and version 1.3 or later is part of all distributions of \LaTeX{}
version 2005/12/01 or later.

This work has the LPPL maintenance status `maintained'.

The Current Maintainer of this work is Niklas Beisert.

This work consists of the files |README.txt|, |childdoc.ins| and |childdoc.dtx|
as well as the derived files |childdoc.def|, |cdocsamp.tex|
with |cdocsch1.tex|, |cdocsch2.tex|, |cdocspt3.tex|, |cdocspt4.tex|,
|cdocsdrf.tex|, |cdocsfn1.tex|, |cdocsfn2.tex|
as well as |childdoc.pdf|.

%%%%%%%%%%%%%%%%%%%%%%%%%%%%%%%%%%%%%%%%%%%%%%%%%%%%%%%%%%%%%%%%%%%%%%%%%%%%%%%%
\subsection{Files and Installation}

The package consists of the files:
%
\begin{center}
\begin{tabular}{ll}
    |README.txt|   & readme file \\
    |childdoc.ins| & installation file \\
    |childdoc.dtx| & source file \\
    |childdoc.def| & definition file \\
    |cdocsamp.tex| & sample main file \\
    |cdocsch1.tex| & sample include file \\
    |cdocsch2.tex| & sample include file \\
    |cdocspt3.tex| & sample part file \\
    |cdocspt4.tex| & sample part file \\
    |cdocsdrf.tex| & sample redirection file \\
    |cdocsfn1.tex| & sample redirection file \\
    |cdocsfn2.tex| & sample redirection file \\
    |childdoc.pdf| & manual
\end{tabular}
\end{center}
%
The distribution consists of the files
|README.txt|, |childdoc.ins| and |childdoc.dtx|.
%
\begin{itemize}
\item
Run (pdf)\LaTeX{} on |childdoc.dtx|
to compile the manual |childdoc.pdf| (this file).
\item
Run \LaTeX{} on |childdoc.ins| to create the definitions file |childdoc.def|
and the sample |cdocsamp.tex| with include files
|cdocsch1.tex|, |cdocsch2.tex|, |cdocspt3.tex|, |cdocspt4.tex|,
|cdocsdrf.tex|, |cdocsfn1.tex|, |cdocsfn2.tex|.
Then copy the file |childdoc.def| to an appropriate directory of your \LaTeX{}
distribution, e.g.\ \textit{texmf-root}|/tex/latex/childdoc|.
\end{itemize}

%%%%%%%%%%%%%%%%%%%%%%%%%%%%%%%%%%%%%%%%%%%%%%%%%%%%%%%%%%%%%%%%%%%%%%%%%%%%%%%%
\subsection{Related CTAN Packages}

There are several other packages which offer a similar functionality:
%
\begin{itemize}
\item
The packages
\href{http://ctan.org/pkg/docmute}{\textsf{docmute}},
\href{http://ctan.org/pkg/includex}{\textsf{includex}} and
\href{http://ctan.org/pkg/standalone}{\textsf{standalone}}
provide commands to include only the document body of
a child file thus allowing both files to be compiled individually.
\item
The packages \href{http://ctan.org/pkg/subdocs}{\textsf{subdocs}}
and \href{http://ctan.org/pkg/subfiles}{\textsf{subfiles}}
provide structures in which the main and child documents can be
encapsulated and allowing them to be compiled individually.
The inclusion mechanism is different from the conventional |\include|.
\item
The package \href{http://ctan.org/pkg/combine}{\textsf{combine}}
is an elaborate solution to combine several documents into one.
\end{itemize}
%
See also the CTAN topic \href{http://ctan.org/topic/subdocs}{\textsf{subdocs}}
for further related packages.
The present package differs from the above solutions in that
a document structure constructed with the conventional |\include| mechanism
just needs two extra commands at the top of every file
such that all constituent files can be compiled individually.

%%%%%%%%%%%%%%%%%%%%%%%%%%%%%%%%%%%%%%%%%%%%%%%%%%%%%%%%%%%%%%%%%%%%%%%%%%%%%%%%
%\subsection{Feature Suggestions}
%
%The following is a list of features which may be useful for future
%versions of this package:
%%
%\begin{itemize}
%\item
%\ldots
%\end{itemize}

%%%%%%%%%%%%%%%%%%%%%%%%%%%%%%%%%%%%%%%%%%%%%%%%%%%%%%%%%%%%%%%%%%%%%%%%%%%%%%%%
\subsection{Revision History}

%%%%%%%%%%%%%%%%%%%%%%%%%%%%%%%%%%%%%%%%
\paragraph{v2.0:} 2018/12/30

\begin{itemize}
\item
immediate forward processing
\item
added |\childdocby| mechanism
\item
manual restructured
\end{itemize}

%%%%%%%%%%%%%%%%%%%%%%%%%%%%%%%%%%%%%%%%
\paragraph{v1.6:} 2018/01/17

\begin{itemize}
\item
application for development of include files
\item
corrections to manual
\end{itemize}

%%%%%%%%%%%%%%%%%%%%%%%%%%%%%%%%%%%%%%%%
\paragraph{v1.5:} 2017/05/21

\begin{itemize}
\item
more complete structuring introduced
\item
|\childdocof| introduced
\item
|\childdoc| renamed to |\childdocmain|
\item
|\childredirect| renamed to |\childdocforward| and |\childdocforwardprefix|
and functionality expanded
\end{itemize}

%%%%%%%%%%%%%%%%%%%%%%%%%%%%%%%%%%%%%%%%
\paragraph{v1.0:} 2017/04/27

\begin{itemize}
\item
manual and install package
\item
first version published on CTAN
\end{itemize}

%%%%%%%%%%%%%%%%%%%%%%%%%%%%%%%%%%%%%%%%
\paragraph{v0.6:} 2017/04/26

\begin{itemize}
\item
redirection mechanism added
\end{itemize}

%%%%%%%%%%%%%%%%%%%%%%%%%%%%%%%%%%%%%%%%
\paragraph{v0.5:} 2017/04/26

\begin{itemize}
\item
functionality in definition file
\end{itemize}


%%%%%%%%%%%%%%%%%%%%%%%%%%%%%%%%%%%%%%%%%%%%%%%%%%%%%%%%%%%%%%%%%%%%%%%%%%%%%%%%
%%%%%%%%%%%%%%%%%%%%%%%%%%%%%%%%%%%%%%%%%%%%%%%%%%%%%%%%%%%%%%%%%%%%%%%%%%%%%%%%
%%%%%%%%%%%%%%%%%%%%%%%%%%%%%%%%%%%%%%%%%%%%%%%%%%%%%%%%%%%%%%%%%%%%%%%%%%%%%%%%
\appendix

\settowidth\MacroIndent{\rmfamily\scriptsize 000\ }

 \DocInput{childdoc.dtx}

\end{document}
%</driver>
% \fi
%
% %%%%%%%%%%%%%%%%%%%%%%%%%%%%%%%%%%%%%%%%%%%%%%%%%%%%%%%%%%%%%%%%%%%%%%%%%%%%%%
% %%%%%%%%%%%%%%%%%%%%%%%%%%%%%%%%%%%%%%%%%%%%%%%%%%%%%%%%%%%%%%%%%%%%%%%%%%%%%%
% \section{Sample}
%\iffalse
%<*samplemain>
%\fi
%
% The following presents a sample document
% with two chapters, two parts, a title page,
% a compile flag as well as three forwarding files to set the flag.
% It consists of eight |.tex| files:
% \begin{center}
% \begin{tabular}{ll}
% |cdocsamp.tex|&main file\\
% |cdocsch1.tex|&include file for chapter 1\\
% |cdocsch2.tex|&include file for chapter 2\\
% |cdocspt3.tex|&include file for part 3\\
% |cdocspt4.tex|&include file for part 4\\
% |cdocsdrf.tex|&forwarding file for main file in draft mode\\
% |cdocsfi1.tex|&forwarding file for final version of chapter 1\\
% |cdocsfi2.tex|&forwarding file for final version of chapter 2\\
% \end{tabular}
% \end{center}
% Each of the eight files can be compiled directly by the \LaTeX{} compiler.
%
% %%%%%%%%%%%%%%%%%%%%%%%%%%%%%%%%%%%%%%
% \paragraph{Main File.}
%
% The main file is called |cdocsamp.tex|.
%
% Load the \textsf{childdoc} definitions and
% declare the filename for the main document:
%    \begin{macrocode}
\input{childdoc.def}
\childdocmain{}
%    \end{macrocode}

% Optional override for |\version| flag:
%    \begin{macrocode}
%%\ifchilddoc\else\providecommand{\version}{draft}\fi
%    \end{macrocode}

% Define the default values for the |\version| flag
% (|final| for the main file and |draft| for childs):
%    \begin{macrocode}
\ifchilddoc
\providecommand{\version}{draft}
\else
\providecommand{\version}{final}
\fi
%    \end{macrocode}

% Load the standard document class:
%    \begin{macrocode}
\documentclass[12pt]{article}
%    \end{macrocode}

% Start the document body:
%    \begin{macrocode}
\begin{document}
%    \end{macrocode}

% Declare a title page.
% Print title, part of document being processed and version flag:
%    \begin{macrocode}
\addtocounter{page}{-1}
\begin{center}
{\LARGE\bfseries{}childdoc example\par}
\vspace{1cm}
\ifchilddoc
\ifchilddocmanual part\else chapter\fi:
`\childdocname' of `\childdocjob'\par
\else
main document: `\childdocjob'\par
\fi
version: \version\par
\end{center}
\newpage
%    \end{macrocode}

% Manually include selected file,
% otherwise process as usual:
%    \begin{macrocode}
\ifchilddocmanual
\section*{part `\childdocname'}
\input{\childdocname}
\else
%    \end{macrocode}

% Include the two chapters:
%    \begin{macrocode}
\include{cdocsch1}
\include{cdocsch2}
%    \end{macrocode}

% Include the two parts unless only chapters should be displayed:
%    \begin{macrocode}
\ifchilddoc\else
\section{part three}
\input{cdocspt3}
\section{part four}
\input{cdocspt4}
\fi
%    \end{macrocode}

% Process as usual until here:
%    \begin{macrocode}
\fi
%    \end{macrocode}

% End of document body:
%    \begin{macrocode}
\end{document}
%    \end{macrocode}
%\iffalse
%</samplemain>
%\fi
%
% %%%%%%%%%%%%%%%%%%%%%%%%%%%%%%%%%%%%%%
% \paragraph{Chapter Include Files.}
%
% The include files are called |cdocsch1.tex| and |cdocsch2.tex|.
%
%\iffalse
%<*samplechap1|samplechap2>
%\fi

% Optional override for |\version| flag:
%    \begin{macrocode}
%%\providecommand{\version}{final}
%    \end{macrocode}

% Include the main document:
%    \begin{macrocode}
\input{childdoc.def}
\childdocof{cdocsamp}
%    \end{macrocode}

%\iffalse
%</samplechap1|samplechap2>
%\fi
%
%\iffalse
%<*samplechap1>
%\fi
% Some text for chapter 1:
%    \begin{macrocode}
\section{one}
some text in chapter one
%    \end{macrocode}

%\iffalse
%</samplechap1>
%\fi
% Some text for chapter 2:
%\iffalse
%<*samplechap2>
%\fi
%    \begin{macrocode}
\section{two}
more text in chapter two
%    \end{macrocode}

%\iffalse
%</samplechap2>
%\fi
%
% %%%%%%%%%%%%%%%%%%%%%%%%%%%%%%%%%%%%%%
% \paragraph{Part Include Files.}
%
% The include files are called |cdocspt3.tex| and |cdocspt4.tex|.
%
%\iffalse
%<*samplepart3|samplepart4>
%\fi

% Optional override for |\version| flag:
%    \begin{macrocode}
%%\providecommand{\version}{final}
%    \end{macrocode}

% Include the main document:
%    \begin{macrocode}
\input{childdoc.def}
\childdocby{cdocsamp}
%    \end{macrocode}

%\iffalse
%</samplepart3|samplepart4>
%\fi
%
%\iffalse
%<*samplepart3>
%\fi
% Some text for part 3:
%    \begin{macrocode}
some text in part three
%    \end{macrocode}

%\iffalse
%</samplepart3>
%\fi
% Some text for part 4:
%\iffalse
%<*samplepart4>
%\fi
%    \begin{macrocode}
more text in part four
%    \end{macrocode}

%\iffalse
%</samplepart4>
%\fi
%
% %%%%%%%%%%%%%%%%%%%%%%%%%%%%%%%%%%%%%%
% \paragraph{Forwarding for a Complete Draft.}
%
% The following forwarding file |cdocsdrf.tex|
% compiles the main document in draft mode:
%\iffalse
%<*sampledraft>
%\fi
%    \begin{macrocode}
\def\version{draft}
\input{childdoc.def}
\childdocforward{cdocsamp}
%    \end{macrocode}

%\iffalse
%</sampledraft>
%\fi
%
% %%%%%%%%%%%%%%%%%%%%%%%%%%%%%%%%%%%%%%
% \paragraph{Forwarding for Final Version of the Chapters.}
%
% The following forwarding files |cdocsfn1.tex| and |cdocsfn2.tex|
% (with identical content)
% compile the final versions of the child documents
% |cdocsch1.tex| and |cdocsch2.tex|, respectively:
%\iffalse
%<*samplefinal>
%\fi
%    \begin{macrocode}
\def\version{final}
\input{childdoc.def}
\childdocforwardprefix[cdocsamp]{cdocsfn}{cdocsch}
%    \end{macrocode}

%\iffalse
%</samplefinal>
%\fi
%
% %%%%%%%%%%%%%%%%%%%%%%%%%%%%%%%%%%%%%%
% \paragraph{Command Line Processing.}
%
% The following three command lines generate the output files
% |cdocscld|, |cdocscl1| and |cdocscl2|
% which should be identical to
% |cdocsdrf|, |cdocsch1| and |cdocsfn2|, respectively:
% \begin{center}
% \begin{tabular}{l}
% |latex -jobname cdocscld \|\\
% |  "\def\version{draft}\input{childdoc.def}\childdocforward{cdocsamp}"|\\
% |latex -jobname cdocscl1 \|\\
% |  "\input{childdoc.def}\childdocforward[cdocsamp]{cdocsch1}"|\\
% |latex -jobname cdocscl2 \|\\
% |  "\def\version{final}\input{childdoc.def}\childdocforward{cdocsch2}"|
% \end{tabular}
% \end{center}
% Note that the trailing backslash on each first line
% merely continues the input to the second line
% (for convenient cut ant paste).
% Furthermore, the command |latex| can be replaced by any
% of its alternative versions such as |pdflatex|.
%
% %%%%%%%%%%%%%%%%%%%%%%%%%%%%%%%%%%%%%%%%%%%%%%%%%%%%%%%%%%%%%%%%%%%%%%%%%%%%%%
% %%%%%%%%%%%%%%%%%%%%%%%%%%%%%%%%%%%%%%%%%%%%%%%%%%%%%%%%%%%%%%%%%%%%%%%%%%%%%%
% \section{Implementation}
%\iffalse
%<*package>
%\fi
%
% This section describes the definitions file |childdoc.def|.

% The definitions cannot be loaded using |\usepackage| or |\RequirePackage|
% which has a mechanism to prevent loading a style file more than once.
% When loading the definitions by means of |\input|
% multiple instances have to be prevented manually:
%\iffalse
%This code needs to be before the `\ProvidesFile' directive
%which is defined at the beginning of this file.
%Therefore it is also placed there and commented out here.
%</package>
%<*discard>
%\fi
%    \begin{macrocode}
\ifdefined\childdocmain\endinput\fi
%    \end{macrocode}
%\iffalse
%</discard>
%<*package>
%\fi
%
% \macro{\ifchilddoc}
% \macro{\ifchilddocmanual}
% The conditional |\ifchilddoc| tells whether a
% child (true) or main (false) document is being compiled.
% The conditional |\ifchilddocmanual| tells whether
% the |\includeonly| mechanism is used (false) or
% the selection of child files must be performed manually (true).
% The definitions initialise to false:
%    \begin{macrocode}
\newif\ifchilddoc
\newif\ifchilddocmanual
%    \end{macrocode}

% \macro{\childdocname}
% \macro{\childdocjob}
% The macro |\childdocname| stores the name of the main document
% to be compiled. The macro |\childdocjob| stores the name of
% the document on which the \LaTeX{} compiler was originally invoked.
% The content of |\jobname| cannot be compared
% to filenames specified in the source due to different catcodes.
% The following code rescans |\jobname|, stores the result
% in |\childdocname| and saves a copy in |\childdocjob|:
%    \begin{macrocode}
\edef\childdocname{\scantokens\expandafter{\jobname\noexpand}}
\let\childdocjob\childdocname
%    \end{macrocode}

% \macro{\childdocdisable}
% The macro |\childdocdisable| prevents the main file
% from being processed more than once.
% At this stage, the main document command |\childdocmain|
% is assumed to be called once again where it should do nothing.
% Any subsequent call to it should prevent
% a secondary processing of the main document
% It overwrites the forwarding commands
% |\childdocof| and |\childdocforward|
% with empty macros to prevent further inclusions of the main document:
%    \begin{macrocode}
\newcommand{\childdocdisable}
{
  \renewcommand{\childdocmain}[1]{\renewcommand{\childdocmain}[1]{\endinput}}
  \renewcommand{\childdocof}[1]{}
  \renewcommand{\childdocby}[2][]{}
  \renewcommand{\childdocforward}[2][]{}
  \renewcommand{\childdocdisable}{}
}
%    \end{macrocode}

% \macro{\childdocmain}
% The macro |\childdocmain| is to be called at the top of the main file
% with nothing or the main filename (without extension) as argument.
% First, it breaks loops.
% If the argument is not empty and does not match |\childdocname|
% (which is set by the first inclusion of |childdoc.def|),
% |\ifchilddoc| is set to true, |\includeonly| is applied to the child file
% and |\jobname| is set to the main file
% (for proper handling of |.aux| files):
%    \begin{macrocode}
\newcommand{\childdocmain}[1]
{
  \childdocdisable\childdocmain{}
  \if?#1?\else
    \begingroup
      \def\childdoctmp{#1}
      \ifx\childdoctmp\childdocname
        \def\childdoctmp{}
      \else
        \def\childdoctmp
        {
          \childdoctrue
          \includeonly{\childdocname}
          \def\childdocjob{#1}
          \def\jobname{#1}
        }
      \fi
      \expandafter
    \endgroup
    \childdoctmp
  \fi
}
%    \end{macrocode}

% \macro{\childdocof}
% The command |\childdocof| redirects
% compilation to the main file |#1|.
%    \begin{macrocode}
\newcommand{\childdocof}[1]
{
  \childdocdisable
  \childdoctrue
  \includeonly{\childdocname}
  \def\jobname{#1}
  \def\childdocjob{#1}
  \input{#1}
}
%    \end{macrocode}

% \macro{\childdocby}
% The command |\childdocby| ....
%    \begin{macrocode}
\newcommand{\childdocby}[2][]
{
  \childdocdisable
  \childdoctrue
  \childdocmanualtrue
  \if?#1?\else
    \def\jobname{#2}
  \fi
  \def\childdocjob{#2}
  \input{#2}
  \endinput
}
%    \end{macrocode}

% \macro{\childdocforward}
% The command |\childdocforward| redirects
% compilation to the main file or
% (if the optional argument is given) a child file.
% Parameters are set as if the main file
% or a child file starting with |\childdocof| was compiled.
% Then compilation is handed over to the main file:
%    \begin{macrocode}
\newcommand{\childdocforward}[2][]
{
  \begingroup
    \if?#1?
      \def\childdoctmp
      {
        \def\childdocname{#2}
        \def\childdocjob{#2}
        \def\jobname{#2}
        \input{#2}
        \endinput
      }
    \else
      \def\childdoctmp
      {
        \childdocdisable
        \def\childdocname{#2}
        \childdoctrue
        \includeonly{#2}
        \def\childdocjob{#1}
        \def\jobname{#1}
        \input{#1}
        \endinput
      }
    \fi
    \expandafter
  \endgroup
  \childdoctmp
}
%    \end{macrocode}

% \macro{\childdocforwardprefix}
% The command |\childdocforwardprefix| redirects
% compilation to the main or a child file by means of a pattern.
% The prefix |#1| in the current filename is replaced by |#2|
% and the suffix of the current filename is kept
% (it is assumed that the filename does not contain the substring `|~~~|'
% which is used as a delimiter).
% Compilation is handed over to the new file by |\childdocforward|:
%    \begin{macrocode}
\newcommand{\childdocforwardprefix}[3][]
{
  \begingroup
    \def\childdocextract #2##1~~~{\def\childdoctmp{\childdocforward[#1]{#3##1}}}
    \expandafter\childdocextract\childdocname~~~
    \expandafter
  \endgroup
  \childdoctmp
}
%    \end{macrocode}

% \macro{\childdoc}
% The deprecated macro |\childdoc| is a legacy version of |\childdocmain|:
%    \begin{macrocode}
\newcommand{\childdoc}{\childdocmain}
%    \end{macrocode}

% \macro{\childdocredirect}
% The deprecated macro |\childdocredirect| is a legacy version
% of |\childdocforward| and |\childdocforwardprefix|:
%    \begin{macrocode}
\newcommand{\childdocredirect}[2][]
{
  \begingroup
    \if?#1?
      \def\childdoctmp{\childdocforward{#2}}
    \else
      \def\childdoctmp{\childdocforwardprefix{#1}{#2}}
    \fi
    \expandafter
  \endgroup
  \childdoctmp
}
%    \end{macrocode}

%\iffalse
%</package>
%\fi
%
\endinput
|\\
|\childdocby{|\textit{main}|}|\\
\end{tabular}
\end{center}
%
The directive |\childdocby| is similar to |\childdocof|
described in \secref{sec:include},
but the subsequent selection of content must be done manually.
To that end, both |\ifchilddoc| and |\ifchilddocmanual|
will be true upon processing of a part,
and the name of the part is stored in |\childdocname|.
Note that |\jobname| will be set to the filename of the current part
so that each part receives an individual |.aux| file
that does not interfere with the |.aux| file(s) of the main document.
This behaviour can be altered by the alternative form
|\childdocby[*]{|\textit{main}|}| (with a non-empty optional argument)
which uses the |.aux| file of the main document
by setting |\jobname| to \textit{main}.

%%%%%%%%%%%%%%%%%%%%%%%%%%%%%%%%%%%%%%%%%%%%%%%%%%%%%%%%%%%%%%%%%%%%%%%%%%%%%%%%
\subsection{Driver Development}
\label{sec:driver}

The \textsf{childdoc} mechanism can also be use for the development
of definition files such as \LaTeX{} styles or classes.
This case differs from the above setup with multiple parts
included by |\include| in that no |\includeonly| should be invoked.
This can be achieved by starting the include file
(before |\ProvidesPackage|) with:
%
\begin{center}
\begin{tabular}{l}
|% \iffalse
%
% childdoc.dtx Copyright (C) 2017-2018 Niklas Beisert
%
% This work may be distributed and/or modified under the
% conditions of the LaTeX Project Public License, either version 1.3
% of this license or (at your option) any later version.
% The latest version of this license is in
%   http://www.latex-project.org/lppl.txt
% and version 1.3 or later is part of all distributions of LaTeX
% version 2005/12/01 or later.
%
% This work has the LPPL maintenance status `maintained'.
%
% The Current Maintainer of this work is Niklas Beisert.
%
% This work consists of the files childdoc.dtx and childdoc.ins
% and the derived files childdoc.def and cdocsamp.tex with
% cdocsch1.tex, cdocsch2.tex, cdocsdrf.tex, cdocsfn1.tex, cdocsfn2.tex.
%
%<package>\ifdefined\childdocmain\endinput\fi
%<package>\ProvidesFile{childdoc.def}[2018/12/30 v2.0 child document driver]
%<samplemain>\ProvidesFile{cdocsamp.tex}[2018/12/30 v2.0 sample for childdoc]
%<*driver>
%\ProvidesFile{childdoc.drv}[2018/12/30 v2.0 childdoc reference manual file]
\PassOptionsToClass{10pt,a4paper}{article}
\documentclass{ltxdoc}

\usepackage[margin=35mm]{geometry}
\usepackage{hyperref}
\usepackage{hyperxmp}
\usepackage[usenames]{color}

\hypersetup{colorlinks=true}
\hypersetup{pdfstartview=FitH}
\hypersetup{pdfpagemode=UseNone}
\hypersetup{pdfsource={}}
\hypersetup{pdflang={en-UK}}
\hypersetup{pdfcopyright={Copyright 2017-2018 Niklas Beisert.
  This work may be distributed and/or modified under the
  conditions of the LaTeX Project Public License, either version 1.3
  of this license or (at your option) any later version.}}
\hypersetup{pdflicenseurl={http://www.latex-project.org/lppl.txt}}
\hypersetup{pdfcontactaddress={ETH Zurich, ITP, HIT K,
  Wolfgang-Pauli-Strasse 27}}
\hypersetup{pdfcontactpostcode={8093}}
\hypersetup{pdfcontactcity={Zurich}}
\hypersetup{pdfcontactcountry={Switzerland}}
\hypersetup{pdfcontactemail={nbeisert@itp.phys.ethz.ch}}
\hypersetup{pdfcontacturl={http://people.phys.ethz.ch/\xmptilde nbeisert/}}

\newcommand{\secref}[1]{\hyperref[#1]{section \ref*{#1}}}

\parskip1ex
\parindent0pt
\let\olditemize\itemize
\def\itemize{\olditemize\parskip0pt}

\begin{document}

\title{The \textsf{childdoc} Package}
\hypersetup{pdftitle={The childdoc Package}}
\author{Niklas Beisert\\[2ex]
  Institut f\"ur Theoretische Physik\\
  Eidgen\"ossische Technische Hochschule Z\"urich\\
  Wolfgang-Pauli-Strasse 27, 8093 Z\"urich, Switzerland\\[1ex]
  \href{mailto:nbeisert@itp.phys.ethz.ch}
  {\texttt{nbeisert@itp.phys.ethz.ch}}}
\hypersetup{pdfauthor={Niklas Beisert}}
\hypersetup{pdfsubject={Manual for the LaTeX2e Package childdoc}}
\date{30 December 2018, \textsf{v2.0}}
\maketitle

\begin{abstract}\noindent
\textsf{childdoc} is a \LaTeXe{} package
that enables the direct compilation
of document sections included by |\include|
to individual files.
\end{abstract}

\begingroup
\parskip0ex
\tableofcontents
\endgroup

%%%%%%%%%%%%%%%%%%%%%%%%%%%%%%%%%%%%%%%%%%%%%%%%%%%%%%%%%%%%%%%%%%%%%%%%%%%%%%%%
%%%%%%%%%%%%%%%%%%%%%%%%%%%%%%%%%%%%%%%%%%%%%%%%%%%%%%%%%%%%%%%%%%%%%%%%%%%%%%%%
\section{Introduction}

\LaTeX{} provides a mechanism to structure a large document (such as a book)
into a main file and several child files (containing the chapters)
using the |\include| command.
This mechanism is beneficial for documents
which span hundreds of pages in order to
make the source file(s) more manageable.
Moreover, compilation can be restricted to
selected child files by means of the |\includeonly| command.
The latter feature can be used to reduce the compilation time while editing
(this was significantly more useful in the earlier days of \LaTeX{})
or to generate a smaller document which is easier to navigate.
Another application of |\includeonly| is to generate
documents consisting of selected parts of the complete document.

However, there are a few drawbacks of the plain |\include| mechanism:
\begin{itemize}
\item
The child files cannot be compiled on their own,
they can only be compiled via the main file.
A naive editing environment
(such as a text editor with an option
to have the current file processed by \LaTeX)
may require one to switch to the main file before compiling;
attempting to compile the child file produces errors.
\item
The main file must be modified (each time)
to adjust the |\includeonly| command
to the present needs. This easily leaves the main file in a messy state.
\item
The generated document will always carry the filename
of the main document. This is inconvenient if
several child files are to be compiled and
to be kept for distribution.
\end{itemize}

The present package provides a simple interface
to make child files individually compilable by \LaTeX{}.
Compiling a child file then has the same effect as compiling
the main file with an |\includeonly| command
to select the appropriate child.
Moreover the generated document will carry the name of the child
rather than the main file.
This resolves all three above issues.

This feature is meant to make the editing of books,
thesis documents and lecture notes somewhat more convenient.
However, the package can also be used efficiently for
composing a series of documents (such as exercise sheets)
which are typically distributed individually.
It then assists the author in generating the individual documents
(potentially in different versions)
as well as a document containing the collected series.
Another application is in developing style files
or other kinds of included material
where compilation of the style file could redirect
to a sample or test file.

%%%%%%%%%%%%%%%%%%%%%%%%%%%%%%%%%%%%%%%%%%%%%%%%%%%%%%%%%%%%%%%%%%%%%%%%%%%%%%%%
%%%%%%%%%%%%%%%%%%%%%%%%%%%%%%%%%%%%%%%%%%%%%%%%%%%%%%%%%%%%%%%%%%%%%%%%%%%%%%%%
\section{Usage}

First of all, the package \textsf{childdoc} is \emph{not} a standard
\LaTeXe{} |.sty| style file! Therefore it needs to be invoked in
a non-standard way.

%%%%%%%%%%%%%%%%%%%%%%%%%%%%%%%%%%%%%%%%%%%%%%%%%%%%%%%%%%%%%%%%%%%%%%%%%%%%%%%%
\subsection{Included Files}
\label{sec:include}

%%%%%%%%%%%%%%%%%%%%%%%%%%%%%%%%%%%%%%%%
\DescribeMacro{\childdocmain}
To use the package, add the commands
\begin{center}
\begin{tabular}{l}
|\input{childdoc.def}|\\
|\childdocmain{}|\\
\end{tabular}
\end{center}
at the very top of the main \LaTeX{} file,
in particular \emph{before} the |\documentclass| statement!
The argument of |\childdocmain| should be left empty
(but it must be present).

%%%%%%%%%%%%%%%%%%%%%%%%%%%%%%%%%%%%%%%%
\DescribeMacro{\childdocof}
Furthermore, add the commands
\begin{center}
\begin{tabular}{l}
|\input{childdoc.def}|\\
|\childdocof{|\textit{main}|}|\\
\end{tabular}
\end{center}
at the top of every child file \textit{child}
which is included by |\include{|\textit{child}|}|
from within the main file
(or at least for those files to be compiled individually).
The argument \textit{main} must be the filename of the main file.

There are a couple of
considerations in setting up the main and child documents:

%%%%%%%%%%%%%%%%%%%%%%%%%%%%%%%%%%%%%%%%
\paragraph{Restrictions.}

Please note the following restrictions:
\begin{itemize}
\item
|\childdocmain| must be called with one argument \textit{main}
to ensure compatibility with earlier version of the package.
It must either be empty (|\childdocmain{}|)
or precisely match the filename of the main file in which it is specified.
See \secref{sec:detection} for further information.
\item
The filename \textit{main} must be specified without the |.tex| extension.
\item
The filename \textit{main} is case sensitive
(even in case-insensitive file systems)
due to internal string comparison.
\item
The argument \textit{main} should be fully expanded, it cannot be a macro.
\item
Subdirectories and special characters should be avoided in filenames.
\item
The command |\childdocmain{|\textit{main}|}| must be followed by a whitespace.
It should not be followed immediately by another command
or by a comment mark `|%|'.
This is because the \TeX{} parser reads the token immediately following
the argument of |\childdocmain| and puts it
at the beginning of every child section;
however, a white\-space is ignored.
\end{itemize}

%%%%%%%%%%%%%%%%%%%%%%%%%%%%%%%%%%%%%%%%
\paragraph{Content of Main File.}

It is advisable to place all content in the child files included by |\include|.
Any output contained in the main file will appear in all child documents
unless suppressed manually;
it cannot be suppressed automatically by the |\includeonly| directive
and thus should normally be avoided.
A method to include some content in the main file
by means of conditional processing is described in \secref{sec:conditional}.

%%%%%%%%%%%%%%%%%%%%%%%%%%%%%%%%%%%%%%%%
\paragraph{Page Numbering.}

When only a part of the document is compiled,
the appropriate numbering of pages
(as well as other status parameters)
is determined from the |.aux| files.
The latter contain information from previous passes.
However this information needs to propagate through
all intermediate child documents.
Therefore the page numbering in child documents may well
be inconsistent until the complete document is compiled at least once.

A useful (if unconventional) way to always ensure a consistent
page numbering is to restart the numbering in each child document
and denote the pages by `\textit{child}|.|\textit{page}'
where \textit{child} represents the chapter/section number of the child file.
This can be achieved by the command
|\numberwithin{page}{|\textit{child}|}|
of the \textsf{amsmath} package
where \textit{child} can be |chapter| or |section|
depending on the chosen structuring.
Alternatively, one can modify the macro |\thepage| appropriately
and reset the counter |page| at the start of each child file.

%%%%%%%%%%%%%%%%%%%%%%%%%%%%%%%%%%%%%%%%%%%%%%%%%%%%%%%%%%%%%%%%%%%%%%%%%%%%%%%%
\subsection{Conditional Processing}
\label{sec:conditional}

The package provides a mechanism to compile different versions
of a document. To customise the versions further some conditional processing
can come in handy to distinguish which version is being compiled.
The package provides two macros to describe the compilation context:

%%%%%%%%%%%%%%%%%%%%%%%%%%%%%%%%%%%%%%%%
\DescribeMacro{\ifchilddoc}
The conditional |\ifchilddoc| distinguishes between the compilation of
child documents and the main document:
%
\begin{center}
|\ifchilddoc |\textit{child-code}| |[|\||else |\textit{main-code}]| \||fi|
\end{center}

%%%%%%%%%%%%%%%%%%%%%%%%%%%%%%%%%%%%%%%%
\DescribeMacro{\childdocname}
\DescribeMacro{\childdocjob}
The macro |\childdocname| contains the filename (without extension)
of the main or child file being processed.
Note that |\childdocjob| will always contain the name of the main file.

%%%%%%%%%%%%%%%%%%%%%%%%%%%%%%%%%%%%%%%%
\paragraph{Title Page.}

Conditional processing can be used to include a title or banner page
in the main document when proper precautions are taken.
Importantly, the code in the main file should ensure that the page counter
(as well as other status parameters which are stored in the |.aux| files)
takes the same value after the conditional processing.
Otherwise the page numbers may take divergent values
depending on which part is compiled.

For example, a title page could be declared by:
%
\begin{center}
\begin{tabular}{l}
|\ifchilddoc\||else|\\
|\addtocounter{page}{-1}|\\
\textit{code for title page}\\
|\newpage|\\
|\||fi|
\end{tabular}
\end{center}
%
A banner page for the child documents can be generated by:
%
\begin{center}
\begin{tabular}{l}
|\ifchilddoc|\\
|\addtocounter{page}{-1}|\\
\textit{code for banner page}\\
|\newpage|\\
|\||fi|
\end{tabular}
\end{center}
%
Here one could write a message such as:
\begin{center}
|This is the part \childdocname{} of \childdocjob{}.|
\end{center}

%%%%%%%%%%%%%%%%%%%%%%%%%%%%%%%%%%%%%%%%%%%%%%%%%%%%%%%%%%%%%%%%%%%%%%%%%%%%%%%%
\subsection{Flags}
\label{sec:flags}

The package makes it easy to generate different versions
of the main or child documents.
To this end compilation flags can be defined
and assigned different default values.
They will be particularly useful in conjunction
with the forwarding mechanism described in \secref{sec:forward}.

For example, it may be useful to have a flag |\version|
which can be set to |draft| or |final|.
The document source will contain some conditional code
depending on the value of |\version|.
Suppose further, the flag should default to |final| for the main file
and to |draft| for child files
which is a natural assignment for editing the document.
This is achieved by placing the following code
in the preamble of the main document
(below the |\childdocmain| directive):
%
\begin{center}
\begin{tabular}{l}
|\ifchilddoc|\\
|\providecommand{\version}{draft}|\\
|\||else|\\
|\providecommand{\version}{final}|\\
|\||fi|
\end{tabular}
\end{center}
%
The definition by |\providecommand| makes sure
that previous definitions are not overwritten.
Further statements |\providecommand{\version}{...}|
can thus be added before the above code to override it.

For the main file, one might add a line
(between |\childdocmain| and the above block)
%
\begin{center}
|%\ifchilddoc\||else\providecommand{\version}{draft}\||fi|
\end{center}
%
which can be uncommented to produce a draft version.
Likewise one can add a line to the very top of a child file
(above the |\childdocof{|\textit{main}|}| directive)
%
\begin{center}
|%\providecommand{\version}{final}|
\end{center}
%
which can be uncommented to produce the final version of this child document.

%%%%%%%%%%%%%%%%%%%%%%%%%%%%%%%%%%%%%%%%%%%%%%%%%%%%%%%%%%%%%%%%%%%%%%%%%%%%%%%%
\subsection{Forwarding}
\label{sec:forward}

Different versions of the main or child documents
using compilation flags as described in \secref{sec:flags}
can be (permanently) stored in different files
for convenient compilation, viewing and distribution.
To this end, the package defines a command
to pass on compilation to a different file:

%%%%%%%%%%%%%%%%%%%%%%%%%%%%%%%%%%%%%%%%
\DescribeMacro{\childdocforward}
The command |\childdocforward| redirects processing to
another source file:
%
\begin{center}
\begin{tabular}{l}
|\input{childdoc.def}|\\
|\childdocforward[|\textit{main}|]{|\textit{dest}|}|\\
\end{tabular}
\end{center}
%
The argument \textit{dest} is the destination file
(without extension).
It should be the main file or one of the child files.
Note that further \textsf{childdoc} directives
such as |\childdocof| and |\childdocforward|
in the indicated file will be processed in this form.
The optional argument \textit{main}
passes on directly to the main file \textit{main}
while pretending to compile the child \textit{dest}.
This form behaves as if \textit{dest}
issues |\childdocof{|\textit{main}|}| right away,
and no further \textsf{childdoc} directives will be processed.

%%%%%%%%%%%%%%%%%%%%%%%%%%%%%%%%%%%%%%%%
\DescribeMacro{\...prefix}
In the alternative form |\childdocforwardprefix|,
%
\begin{center}
\begin{tabular}{l}
|\input{childdoc.def}|\\
|\childdocforwardprefix[|\textit{main}|]{|\textit{prefix}|}{|\textit{dest}|}|
\end{tabular}
\end{center}
%
the destination file is determined by a pattern
depending on the current file:
To make this work, the current file must be called
`{\textit{prefix}\hspace{0.2em}\textit{suffix}}'
with \textit{prefix} matching precisely the argument.
Processing is then passed on to the file
`{\textit{dest}\hspace{0.2em}\textit{suffix}}'.
Surely, the same effect is achieved by
directly specifying the
argument `{\textit{dest}\hspace{0.2em}\textit{suffix}}'
in the first form.
However, that requires to set up a different file
for each child. With the alternative form of the command
all these files can have exactly the same content
which simplifies setting them up and maintaining them.

For example, the following file |draft.tex|
with a compilation flag |\version| as described in \secref{sec:flags}
compiles the main document as a draft:
%
\begin{center}
\begin{tabular}{l}
|\def\version{draft}|\\
|\input{childdoc.def}|\\
|\childdocforward{|\textit{main}|}|
\end{tabular}
\end{center}
%
Likewise, the following files |final|\textit{nn}|.tex|
compile the final version of the child document
|child|\textit{nn}|.tex|:
%
\begin{center}
\begin{tabular}{l}
|\def\version{final}|\\
|\input{childdoc.def}|\\
|\childdocforwardprefix{final}{child}|
\end{tabular}
\end{center}
%

Note that when several versions of a main file and/or of each child file
are to be generated, it may be convenient to set up a |Makefile| or
shell script to automatise the process.

%%%%%%%%%%%%%%%%%%%%%%%%%%%%%%%%%%%%%%%%%%%%%%%%%%%%%%%%%%%%%%%%%%%%%%%%%%%%%%%%
\subsection{Command Line Processing}
\label{sec:commandline}

The effect of redirection files can also be achieved by invoking
the \LaTeX{} compiler with a more elaborate command line.
Most conveniently this should be done as part
of a shell script or a |Makefile|.

When using \textsf{childdoc} in the main file, the following
command lines effectively perform a redirection
(note that depending on the shell being used,
backslashes may have to be doubled: `|\|' $\to$ `|\\|'):
%
\begin{center}
|... -jobname "|\textit{target}|" |\\|"|[\textit{flags}]%
|\input{childdoc.def}\childdocforward[|\textit{main}|]{|\textit{dest}|}"|
\end{center}
%
Here \textit{target} is the name of the output file,
\textit{main} is the name of the main file
and \textit{dest} is the name of the main or child file to be processed
(all filenames without extensions).
The optional argument \textit{main} can be omitted
if \textit{main} matches \textit{dest}.
Optionally, compilation \textit{flags} can be defined via |\def| commands.
This command line makes the \TeX{} engine believe
it is compiling the file \textit{target}
whose content is specified as the latter parameter.
The provided code then forwards the processing to
\textit{main} or \textit{dest} as described in \secref{sec:forward}.

%%%%%%%%%%%%%%%%%%%%%%%%%%%%%%%%%%%%%%%%%%%%%%%%%%%%%%%%%%%%%%%%%%%%%%%%%%%%%%%%
\subsection{Include by Input}
\label{sec:input}

Including child documents by |\include| has some restrictions by design.
Most notably, the content of a child document always occupies
its own set of pages; pages cannot be shared between child documents.
Usually, this behaviour makes perfect sense
because each child document contain an essential part of the document.
However, in some situations it may be desirable to compose
a document from a collection of parts
without having mandatory page breaks between then.
For this case, the package
provides a mechanism to include parts
by |\input| which can also be processed individually.
However, by construction this mechanism
requires manual handling of the content to be output.

%%%%%%%%%%%%%%%%%%%%%%%%%%%%%%%%%%%%%%%%
\DescribeMacro{\ifchilddocmanual}
The main file should be prepared as usual, see \secref{sec:include}.
However, the document body must make a distinction
between processing of an individual part and of the main document, e.g.:
%
\begin{center}
\begin{tabular}{l}
|\ifchilddocmanual|\\
|\input{\childdocname}|\\
|\||else|\\
\textit{document body with }|\input{|\textit{part}|}|\\
|\||fi|
\end{tabular}
\end{center}
%
The conditional |\ifchilddocmanual| is true whenever
a part to be included by |\input| is being compiled,
and the name of the part is stored in |\childdocname|.

%%%%%%%%%%%%%%%%%%%%%%%%%%%%%%%%%%%%%%%%
\DescribeMacro{\childdocby}
Each part to be included by |\input| should start with:
%
\begin{center}
\begin{tabular}{l}
|\input{childdoc.def}|\\
|\childdocby{|\textit{main}|}|\\
\end{tabular}
\end{center}
%
The directive |\childdocby| is similar to |\childdocof|
described in \secref{sec:include},
but the subsequent selection of content must be done manually.
To that end, both |\ifchilddoc| and |\ifchilddocmanual|
will be true upon processing of a part,
and the name of the part is stored in |\childdocname|.
Note that |\jobname| will be set to the filename of the current part
so that each part receives an individual |.aux| file
that does not interfere with the |.aux| file(s) of the main document.
This behaviour can be altered by the alternative form
|\childdocby[*]{|\textit{main}|}| (with a non-empty optional argument)
which uses the |.aux| file of the main document
by setting |\jobname| to \textit{main}.

%%%%%%%%%%%%%%%%%%%%%%%%%%%%%%%%%%%%%%%%%%%%%%%%%%%%%%%%%%%%%%%%%%%%%%%%%%%%%%%%
\subsection{Driver Development}
\label{sec:driver}

The \textsf{childdoc} mechanism can also be use for the development
of definition files such as \LaTeX{} styles or classes.
This case differs from the above setup with multiple parts
included by |\include| in that no |\includeonly| should be invoked.
This can be achieved by starting the include file
(before |\ProvidesPackage|) with:
%
\begin{center}
\begin{tabular}{l}
|\input{childdoc.def}|\\
|\childdocforward{|\textit{main}|}|\\
\end{tabular}
\end{center}
%
or alternatively with:
%
\begin{center}
\begin{tabular}{l}
|\input{childdoc.def}|\\
|\childdocby{|\textit{main}|}|\\
\end{tabular}
\end{center}
%
Both forms have slightly different effects as described above.
The main file is prepared as usual, see \secref{sec:include}.

%%%%%%%%%%%%%%%%%%%%%%%%%%%%%%%%%%%%%%%%%%%%%%%%%%%%%%%%%%%%%%%%%%%%%%%%%%%%%%%%
\subsection{Legacy Detection}
\label{sec:detection}

The directive |\childdocmain| in the main file can detect
whether the complete document or merely a child is to be compiled
even without using the directive |\childdocof|.
This method is deprecated because it is less robust
and there is no compelling reason to use it;
it is merely provided for backward compatibility
and it may be removed in future versions.

If the detection mechanism is to be used,
it is mandatory to correctly specify
the filename of the main file as the argument of |\childdocmain|:
%
\begin{center}
\begin{tabular}{l}
|\input{childdoc.def}|\\
|\childdocmain{|\textit{main}|}|\\
\end{tabular}
\end{center}
%
If |\jobname| does not match the argument \textit{main} of |\childdocmain|,
it is assumed that |\jobname| points to the child file to be compiled.
When using |\childdocmain| with the main file specified as argument,
it suffices to start a child file
with just |\input{|\textit{main}|}|
without loading of the package and using |\childdocof|.
If instead all processing is done
with the appropriate \textsf{childdoc} directives,
the argument of \textit{main} of |\childdocmain| can be empty.

An alternative version of the command line processing described
in \secref{sec:commandline} using the detection mechanism reads:
%
\begin{center}
|... -jobname "|\textit{target}|" "|[\textit{flags}]%
[|\def\jobname{|\textit{dest}|}|]|\input{|\textit{main}|}"|
\end{center}

%%%%%%%%%%%%%%%%%%%%%%%%%%%%%%%%%%%%%%%%%%%%%%%%%%%%%%%%%%%%%%%%%%%%%%%%%%%%%%%%
\subsection{Manual Code}
\label{sec:manual}

In case one cannot be certain whether the definitions file |childdoc.def|
is installed on the target \TeX{} distribution
and one prefers not to ship it,
it is conceivable to paste a few relevant commands into the sources.

To that end, drop all statements |\input{childdoc.def}|
and perform the replacements as outlined below.
Instead of |\childdocmain{|\textit{main}|}| add the following code
to the top of the main file:
%
\begin{center}
\begin{tabular}{l}
|\||ifdefined\childdocname\endinput\||fi\newif\ifchilddoc|\\
|\edef\childdocname{\scantokens\expandafter{\jobname\noexpand}}|\\
|\def\childdocmain{|\textit{main}|}\||ifx\childdocmain\childdocname\||else|\\
|\childdoctrue\includeonly{\childdocname}\let\jobname\childdocmain\||fi|\\
\end{tabular}
\end{center}
%
Instead of |\childdocof{|\textit{main}|}| just include the main file
at the top of each child file:
%
\begin{center}
|\input{|\textit{main}|}|
\end{center}
%
A simple redirection |\childdocforward{|\textit{dest}|}| is achieved by:
%
\begin{center}
|\def\jobname{|\textit{dest}|}\input{\jobname}|
\end{center}
%
The redirection with prefix
|\childdocforwardprefix[|\textit{prefix}|]{|\textit{dest}|}|
is accomplished by:
%
\begin{center}
\begin{tabular}{l}
|{\edef\jobname{\scantokens\expandafter{\jobname\noexpand}}|\\
|\def\redirectjob |\textit{prefix}|#1~~~{\gdef\jobname{|\textit{dest}|#1}}|\\
|\expandafter\redirectjob\jobname~~~}\input{\jobname}|
\end{tabular}
\end{center}

In an alternative approach,
child documents can be compiled by a specific command line
without additional code or specific definitions:
%
\begin{center}
|... -jobname "|\textit{target}|" "|[\textit{flags}]%
|\includeonly{|\textit{dest}|}\input{|\textit{main}|}"|
\end{center}
%

%%%%%%%%%%%%%%%%%%%%%%%%%%%%%%%%%%%%%%%%%%%%%%%%%%%%%%%%%%%%%%%%%%%%%%%%%%%%%%%%
%%%%%%%%%%%%%%%%%%%%%%%%%%%%%%%%%%%%%%%%%%%%%%%%%%%%%%%%%%%%%%%%%%%%%%%%%%%%%%%%
\section{Information}

%%%%%%%%%%%%%%%%%%%%%%%%%%%%%%%%%%%%%%%%%%%%%%%%%%%%%%%%%%%%%%%%%%%%%%%%%%%%%%%%
\subsection{Copyright}

Copyright \copyright{} 2017--2018 Niklas Beisert

This work may be distributed and/or modified under the
conditions of the \LaTeX{} Project Public License, either version 1.3
of this license or (at your option) any later version.
The latest version of this license is in
  \url{http://www.latex-project.org/lppl.txt}
and version 1.3 or later is part of all distributions of \LaTeX{}
version 2005/12/01 or later.

This work has the LPPL maintenance status `maintained'.

The Current Maintainer of this work is Niklas Beisert.

This work consists of the files |README.txt|, |childdoc.ins| and |childdoc.dtx|
as well as the derived files |childdoc.def|, |cdocsamp.tex|
with |cdocsch1.tex|, |cdocsch2.tex|, |cdocspt3.tex|, |cdocspt4.tex|,
|cdocsdrf.tex|, |cdocsfn1.tex|, |cdocsfn2.tex|
as well as |childdoc.pdf|.

%%%%%%%%%%%%%%%%%%%%%%%%%%%%%%%%%%%%%%%%%%%%%%%%%%%%%%%%%%%%%%%%%%%%%%%%%%%%%%%%
\subsection{Files and Installation}

The package consists of the files:
%
\begin{center}
\begin{tabular}{ll}
    |README.txt|   & readme file \\
    |childdoc.ins| & installation file \\
    |childdoc.dtx| & source file \\
    |childdoc.def| & definition file \\
    |cdocsamp.tex| & sample main file \\
    |cdocsch1.tex| & sample include file \\
    |cdocsch2.tex| & sample include file \\
    |cdocspt3.tex| & sample part file \\
    |cdocspt4.tex| & sample part file \\
    |cdocsdrf.tex| & sample redirection file \\
    |cdocsfn1.tex| & sample redirection file \\
    |cdocsfn2.tex| & sample redirection file \\
    |childdoc.pdf| & manual
\end{tabular}
\end{center}
%
The distribution consists of the files
|README.txt|, |childdoc.ins| and |childdoc.dtx|.
%
\begin{itemize}
\item
Run (pdf)\LaTeX{} on |childdoc.dtx|
to compile the manual |childdoc.pdf| (this file).
\item
Run \LaTeX{} on |childdoc.ins| to create the definitions file |childdoc.def|
and the sample |cdocsamp.tex| with include files
|cdocsch1.tex|, |cdocsch2.tex|, |cdocspt3.tex|, |cdocspt4.tex|,
|cdocsdrf.tex|, |cdocsfn1.tex|, |cdocsfn2.tex|.
Then copy the file |childdoc.def| to an appropriate directory of your \LaTeX{}
distribution, e.g.\ \textit{texmf-root}|/tex/latex/childdoc|.
\end{itemize}

%%%%%%%%%%%%%%%%%%%%%%%%%%%%%%%%%%%%%%%%%%%%%%%%%%%%%%%%%%%%%%%%%%%%%%%%%%%%%%%%
\subsection{Related CTAN Packages}

There are several other packages which offer a similar functionality:
%
\begin{itemize}
\item
The packages
\href{http://ctan.org/pkg/docmute}{\textsf{docmute}},
\href{http://ctan.org/pkg/includex}{\textsf{includex}} and
\href{http://ctan.org/pkg/standalone}{\textsf{standalone}}
provide commands to include only the document body of
a child file thus allowing both files to be compiled individually.
\item
The packages \href{http://ctan.org/pkg/subdocs}{\textsf{subdocs}}
and \href{http://ctan.org/pkg/subfiles}{\textsf{subfiles}}
provide structures in which the main and child documents can be
encapsulated and allowing them to be compiled individually.
The inclusion mechanism is different from the conventional |\include|.
\item
The package \href{http://ctan.org/pkg/combine}{\textsf{combine}}
is an elaborate solution to combine several documents into one.
\end{itemize}
%
See also the CTAN topic \href{http://ctan.org/topic/subdocs}{\textsf{subdocs}}
for further related packages.
The present package differs from the above solutions in that
a document structure constructed with the conventional |\include| mechanism
just needs two extra commands at the top of every file
such that all constituent files can be compiled individually.

%%%%%%%%%%%%%%%%%%%%%%%%%%%%%%%%%%%%%%%%%%%%%%%%%%%%%%%%%%%%%%%%%%%%%%%%%%%%%%%%
%\subsection{Feature Suggestions}
%
%The following is a list of features which may be useful for future
%versions of this package:
%%
%\begin{itemize}
%\item
%\ldots
%\end{itemize}

%%%%%%%%%%%%%%%%%%%%%%%%%%%%%%%%%%%%%%%%%%%%%%%%%%%%%%%%%%%%%%%%%%%%%%%%%%%%%%%%
\subsection{Revision History}

%%%%%%%%%%%%%%%%%%%%%%%%%%%%%%%%%%%%%%%%
\paragraph{v2.0:} 2018/12/30

\begin{itemize}
\item
immediate forward processing
\item
added |\childdocby| mechanism
\item
manual restructured
\end{itemize}

%%%%%%%%%%%%%%%%%%%%%%%%%%%%%%%%%%%%%%%%
\paragraph{v1.6:} 2018/01/17

\begin{itemize}
\item
application for development of include files
\item
corrections to manual
\end{itemize}

%%%%%%%%%%%%%%%%%%%%%%%%%%%%%%%%%%%%%%%%
\paragraph{v1.5:} 2017/05/21

\begin{itemize}
\item
more complete structuring introduced
\item
|\childdocof| introduced
\item
|\childdoc| renamed to |\childdocmain|
\item
|\childredirect| renamed to |\childdocforward| and |\childdocforwardprefix|
and functionality expanded
\end{itemize}

%%%%%%%%%%%%%%%%%%%%%%%%%%%%%%%%%%%%%%%%
\paragraph{v1.0:} 2017/04/27

\begin{itemize}
\item
manual and install package
\item
first version published on CTAN
\end{itemize}

%%%%%%%%%%%%%%%%%%%%%%%%%%%%%%%%%%%%%%%%
\paragraph{v0.6:} 2017/04/26

\begin{itemize}
\item
redirection mechanism added
\end{itemize}

%%%%%%%%%%%%%%%%%%%%%%%%%%%%%%%%%%%%%%%%
\paragraph{v0.5:} 2017/04/26

\begin{itemize}
\item
functionality in definition file
\end{itemize}


%%%%%%%%%%%%%%%%%%%%%%%%%%%%%%%%%%%%%%%%%%%%%%%%%%%%%%%%%%%%%%%%%%%%%%%%%%%%%%%%
%%%%%%%%%%%%%%%%%%%%%%%%%%%%%%%%%%%%%%%%%%%%%%%%%%%%%%%%%%%%%%%%%%%%%%%%%%%%%%%%
%%%%%%%%%%%%%%%%%%%%%%%%%%%%%%%%%%%%%%%%%%%%%%%%%%%%%%%%%%%%%%%%%%%%%%%%%%%%%%%%
\appendix

\settowidth\MacroIndent{\rmfamily\scriptsize 000\ }

 \DocInput{childdoc.dtx}

\end{document}
%</driver>
% \fi
%
% %%%%%%%%%%%%%%%%%%%%%%%%%%%%%%%%%%%%%%%%%%%%%%%%%%%%%%%%%%%%%%%%%%%%%%%%%%%%%%
% %%%%%%%%%%%%%%%%%%%%%%%%%%%%%%%%%%%%%%%%%%%%%%%%%%%%%%%%%%%%%%%%%%%%%%%%%%%%%%
% \section{Sample}
%\iffalse
%<*samplemain>
%\fi
%
% The following presents a sample document
% with two chapters, two parts, a title page,
% a compile flag as well as three forwarding files to set the flag.
% It consists of eight |.tex| files:
% \begin{center}
% \begin{tabular}{ll}
% |cdocsamp.tex|&main file\\
% |cdocsch1.tex|&include file for chapter 1\\
% |cdocsch2.tex|&include file for chapter 2\\
% |cdocspt3.tex|&include file for part 3\\
% |cdocspt4.tex|&include file for part 4\\
% |cdocsdrf.tex|&forwarding file for main file in draft mode\\
% |cdocsfi1.tex|&forwarding file for final version of chapter 1\\
% |cdocsfi2.tex|&forwarding file for final version of chapter 2\\
% \end{tabular}
% \end{center}
% Each of the eight files can be compiled directly by the \LaTeX{} compiler.
%
% %%%%%%%%%%%%%%%%%%%%%%%%%%%%%%%%%%%%%%
% \paragraph{Main File.}
%
% The main file is called |cdocsamp.tex|.
%
% Load the \textsf{childdoc} definitions and
% declare the filename for the main document:
%    \begin{macrocode}
\input{childdoc.def}
\childdocmain{}
%    \end{macrocode}

% Optional override for |\version| flag:
%    \begin{macrocode}
%%\ifchilddoc\else\providecommand{\version}{draft}\fi
%    \end{macrocode}

% Define the default values for the |\version| flag
% (|final| for the main file and |draft| for childs):
%    \begin{macrocode}
\ifchilddoc
\providecommand{\version}{draft}
\else
\providecommand{\version}{final}
\fi
%    \end{macrocode}

% Load the standard document class:
%    \begin{macrocode}
\documentclass[12pt]{article}
%    \end{macrocode}

% Start the document body:
%    \begin{macrocode}
\begin{document}
%    \end{macrocode}

% Declare a title page.
% Print title, part of document being processed and version flag:
%    \begin{macrocode}
\addtocounter{page}{-1}
\begin{center}
{\LARGE\bfseries{}childdoc example\par}
\vspace{1cm}
\ifchilddoc
\ifchilddocmanual part\else chapter\fi:
`\childdocname' of `\childdocjob'\par
\else
main document: `\childdocjob'\par
\fi
version: \version\par
\end{center}
\newpage
%    \end{macrocode}

% Manually include selected file,
% otherwise process as usual:
%    \begin{macrocode}
\ifchilddocmanual
\section*{part `\childdocname'}
\input{\childdocname}
\else
%    \end{macrocode}

% Include the two chapters:
%    \begin{macrocode}
\include{cdocsch1}
\include{cdocsch2}
%    \end{macrocode}

% Include the two parts unless only chapters should be displayed:
%    \begin{macrocode}
\ifchilddoc\else
\section{part three}
\input{cdocspt3}
\section{part four}
\input{cdocspt4}
\fi
%    \end{macrocode}

% Process as usual until here:
%    \begin{macrocode}
\fi
%    \end{macrocode}

% End of document body:
%    \begin{macrocode}
\end{document}
%    \end{macrocode}
%\iffalse
%</samplemain>
%\fi
%
% %%%%%%%%%%%%%%%%%%%%%%%%%%%%%%%%%%%%%%
% \paragraph{Chapter Include Files.}
%
% The include files are called |cdocsch1.tex| and |cdocsch2.tex|.
%
%\iffalse
%<*samplechap1|samplechap2>
%\fi

% Optional override for |\version| flag:
%    \begin{macrocode}
%%\providecommand{\version}{final}
%    \end{macrocode}

% Include the main document:
%    \begin{macrocode}
\input{childdoc.def}
\childdocof{cdocsamp}
%    \end{macrocode}

%\iffalse
%</samplechap1|samplechap2>
%\fi
%
%\iffalse
%<*samplechap1>
%\fi
% Some text for chapter 1:
%    \begin{macrocode}
\section{one}
some text in chapter one
%    \end{macrocode}

%\iffalse
%</samplechap1>
%\fi
% Some text for chapter 2:
%\iffalse
%<*samplechap2>
%\fi
%    \begin{macrocode}
\section{two}
more text in chapter two
%    \end{macrocode}

%\iffalse
%</samplechap2>
%\fi
%
% %%%%%%%%%%%%%%%%%%%%%%%%%%%%%%%%%%%%%%
% \paragraph{Part Include Files.}
%
% The include files are called |cdocspt3.tex| and |cdocspt4.tex|.
%
%\iffalse
%<*samplepart3|samplepart4>
%\fi

% Optional override for |\version| flag:
%    \begin{macrocode}
%%\providecommand{\version}{final}
%    \end{macrocode}

% Include the main document:
%    \begin{macrocode}
\input{childdoc.def}
\childdocby{cdocsamp}
%    \end{macrocode}

%\iffalse
%</samplepart3|samplepart4>
%\fi
%
%\iffalse
%<*samplepart3>
%\fi
% Some text for part 3:
%    \begin{macrocode}
some text in part three
%    \end{macrocode}

%\iffalse
%</samplepart3>
%\fi
% Some text for part 4:
%\iffalse
%<*samplepart4>
%\fi
%    \begin{macrocode}
more text in part four
%    \end{macrocode}

%\iffalse
%</samplepart4>
%\fi
%
% %%%%%%%%%%%%%%%%%%%%%%%%%%%%%%%%%%%%%%
% \paragraph{Forwarding for a Complete Draft.}
%
% The following forwarding file |cdocsdrf.tex|
% compiles the main document in draft mode:
%\iffalse
%<*sampledraft>
%\fi
%    \begin{macrocode}
\def\version{draft}
\input{childdoc.def}
\childdocforward{cdocsamp}
%    \end{macrocode}

%\iffalse
%</sampledraft>
%\fi
%
% %%%%%%%%%%%%%%%%%%%%%%%%%%%%%%%%%%%%%%
% \paragraph{Forwarding for Final Version of the Chapters.}
%
% The following forwarding files |cdocsfn1.tex| and |cdocsfn2.tex|
% (with identical content)
% compile the final versions of the child documents
% |cdocsch1.tex| and |cdocsch2.tex|, respectively:
%\iffalse
%<*samplefinal>
%\fi
%    \begin{macrocode}
\def\version{final}
\input{childdoc.def}
\childdocforwardprefix[cdocsamp]{cdocsfn}{cdocsch}
%    \end{macrocode}

%\iffalse
%</samplefinal>
%\fi
%
% %%%%%%%%%%%%%%%%%%%%%%%%%%%%%%%%%%%%%%
% \paragraph{Command Line Processing.}
%
% The following three command lines generate the output files
% |cdocscld|, |cdocscl1| and |cdocscl2|
% which should be identical to
% |cdocsdrf|, |cdocsch1| and |cdocsfn2|, respectively:
% \begin{center}
% \begin{tabular}{l}
% |latex -jobname cdocscld \|\\
% |  "\def\version{draft}\input{childdoc.def}\childdocforward{cdocsamp}"|\\
% |latex -jobname cdocscl1 \|\\
% |  "\input{childdoc.def}\childdocforward[cdocsamp]{cdocsch1}"|\\
% |latex -jobname cdocscl2 \|\\
% |  "\def\version{final}\input{childdoc.def}\childdocforward{cdocsch2}"|
% \end{tabular}
% \end{center}
% Note that the trailing backslash on each first line
% merely continues the input to the second line
% (for convenient cut ant paste).
% Furthermore, the command |latex| can be replaced by any
% of its alternative versions such as |pdflatex|.
%
% %%%%%%%%%%%%%%%%%%%%%%%%%%%%%%%%%%%%%%%%%%%%%%%%%%%%%%%%%%%%%%%%%%%%%%%%%%%%%%
% %%%%%%%%%%%%%%%%%%%%%%%%%%%%%%%%%%%%%%%%%%%%%%%%%%%%%%%%%%%%%%%%%%%%%%%%%%%%%%
% \section{Implementation}
%\iffalse
%<*package>
%\fi
%
% This section describes the definitions file |childdoc.def|.

% The definitions cannot be loaded using |\usepackage| or |\RequirePackage|
% which has a mechanism to prevent loading a style file more than once.
% When loading the definitions by means of |\input|
% multiple instances have to be prevented manually:
%\iffalse
%This code needs to be before the `\ProvidesFile' directive
%which is defined at the beginning of this file.
%Therefore it is also placed there and commented out here.
%</package>
%<*discard>
%\fi
%    \begin{macrocode}
\ifdefined\childdocmain\endinput\fi
%    \end{macrocode}
%\iffalse
%</discard>
%<*package>
%\fi
%
% \macro{\ifchilddoc}
% \macro{\ifchilddocmanual}
% The conditional |\ifchilddoc| tells whether a
% child (true) or main (false) document is being compiled.
% The conditional |\ifchilddocmanual| tells whether
% the |\includeonly| mechanism is used (false) or
% the selection of child files must be performed manually (true).
% The definitions initialise to false:
%    \begin{macrocode}
\newif\ifchilddoc
\newif\ifchilddocmanual
%    \end{macrocode}

% \macro{\childdocname}
% \macro{\childdocjob}
% The macro |\childdocname| stores the name of the main document
% to be compiled. The macro |\childdocjob| stores the name of
% the document on which the \LaTeX{} compiler was originally invoked.
% The content of |\jobname| cannot be compared
% to filenames specified in the source due to different catcodes.
% The following code rescans |\jobname|, stores the result
% in |\childdocname| and saves a copy in |\childdocjob|:
%    \begin{macrocode}
\edef\childdocname{\scantokens\expandafter{\jobname\noexpand}}
\let\childdocjob\childdocname
%    \end{macrocode}

% \macro{\childdocdisable}
% The macro |\childdocdisable| prevents the main file
% from being processed more than once.
% At this stage, the main document command |\childdocmain|
% is assumed to be called once again where it should do nothing.
% Any subsequent call to it should prevent
% a secondary processing of the main document
% It overwrites the forwarding commands
% |\childdocof| and |\childdocforward|
% with empty macros to prevent further inclusions of the main document:
%    \begin{macrocode}
\newcommand{\childdocdisable}
{
  \renewcommand{\childdocmain}[1]{\renewcommand{\childdocmain}[1]{\endinput}}
  \renewcommand{\childdocof}[1]{}
  \renewcommand{\childdocby}[2][]{}
  \renewcommand{\childdocforward}[2][]{}
  \renewcommand{\childdocdisable}{}
}
%    \end{macrocode}

% \macro{\childdocmain}
% The macro |\childdocmain| is to be called at the top of the main file
% with nothing or the main filename (without extension) as argument.
% First, it breaks loops.
% If the argument is not empty and does not match |\childdocname|
% (which is set by the first inclusion of |childdoc.def|),
% |\ifchilddoc| is set to true, |\includeonly| is applied to the child file
% and |\jobname| is set to the main file
% (for proper handling of |.aux| files):
%    \begin{macrocode}
\newcommand{\childdocmain}[1]
{
  \childdocdisable\childdocmain{}
  \if?#1?\else
    \begingroup
      \def\childdoctmp{#1}
      \ifx\childdoctmp\childdocname
        \def\childdoctmp{}
      \else
        \def\childdoctmp
        {
          \childdoctrue
          \includeonly{\childdocname}
          \def\childdocjob{#1}
          \def\jobname{#1}
        }
      \fi
      \expandafter
    \endgroup
    \childdoctmp
  \fi
}
%    \end{macrocode}

% \macro{\childdocof}
% The command |\childdocof| redirects
% compilation to the main file |#1|.
%    \begin{macrocode}
\newcommand{\childdocof}[1]
{
  \childdocdisable
  \childdoctrue
  \includeonly{\childdocname}
  \def\jobname{#1}
  \def\childdocjob{#1}
  \input{#1}
}
%    \end{macrocode}

% \macro{\childdocby}
% The command |\childdocby| ....
%    \begin{macrocode}
\newcommand{\childdocby}[2][]
{
  \childdocdisable
  \childdoctrue
  \childdocmanualtrue
  \if?#1?\else
    \def\jobname{#2}
  \fi
  \def\childdocjob{#2}
  \input{#2}
  \endinput
}
%    \end{macrocode}

% \macro{\childdocforward}
% The command |\childdocforward| redirects
% compilation to the main file or
% (if the optional argument is given) a child file.
% Parameters are set as if the main file
% or a child file starting with |\childdocof| was compiled.
% Then compilation is handed over to the main file:
%    \begin{macrocode}
\newcommand{\childdocforward}[2][]
{
  \begingroup
    \if?#1?
      \def\childdoctmp
      {
        \def\childdocname{#2}
        \def\childdocjob{#2}
        \def\jobname{#2}
        \input{#2}
        \endinput
      }
    \else
      \def\childdoctmp
      {
        \childdocdisable
        \def\childdocname{#2}
        \childdoctrue
        \includeonly{#2}
        \def\childdocjob{#1}
        \def\jobname{#1}
        \input{#1}
        \endinput
      }
    \fi
    \expandafter
  \endgroup
  \childdoctmp
}
%    \end{macrocode}

% \macro{\childdocforwardprefix}
% The command |\childdocforwardprefix| redirects
% compilation to the main or a child file by means of a pattern.
% The prefix |#1| in the current filename is replaced by |#2|
% and the suffix of the current filename is kept
% (it is assumed that the filename does not contain the substring `|~~~|'
% which is used as a delimiter).
% Compilation is handed over to the new file by |\childdocforward|:
%    \begin{macrocode}
\newcommand{\childdocforwardprefix}[3][]
{
  \begingroup
    \def\childdocextract #2##1~~~{\def\childdoctmp{\childdocforward[#1]{#3##1}}}
    \expandafter\childdocextract\childdocname~~~
    \expandafter
  \endgroup
  \childdoctmp
}
%    \end{macrocode}

% \macro{\childdoc}
% The deprecated macro |\childdoc| is a legacy version of |\childdocmain|:
%    \begin{macrocode}
\newcommand{\childdoc}{\childdocmain}
%    \end{macrocode}

% \macro{\childdocredirect}
% The deprecated macro |\childdocredirect| is a legacy version
% of |\childdocforward| and |\childdocforwardprefix|:
%    \begin{macrocode}
\newcommand{\childdocredirect}[2][]
{
  \begingroup
    \if?#1?
      \def\childdoctmp{\childdocforward{#2}}
    \else
      \def\childdoctmp{\childdocforwardprefix{#1}{#2}}
    \fi
    \expandafter
  \endgroup
  \childdoctmp
}
%    \end{macrocode}

%\iffalse
%</package>
%\fi
%
\endinput
|\\
|\childdocforward{|\textit{main}|}|\\
\end{tabular}
\end{center}
%
or alternatively with:
%
\begin{center}
\begin{tabular}{l}
|% \iffalse
%
% childdoc.dtx Copyright (C) 2017-2018 Niklas Beisert
%
% This work may be distributed and/or modified under the
% conditions of the LaTeX Project Public License, either version 1.3
% of this license or (at your option) any later version.
% The latest version of this license is in
%   http://www.latex-project.org/lppl.txt
% and version 1.3 or later is part of all distributions of LaTeX
% version 2005/12/01 or later.
%
% This work has the LPPL maintenance status `maintained'.
%
% The Current Maintainer of this work is Niklas Beisert.
%
% This work consists of the files childdoc.dtx and childdoc.ins
% and the derived files childdoc.def and cdocsamp.tex with
% cdocsch1.tex, cdocsch2.tex, cdocsdrf.tex, cdocsfn1.tex, cdocsfn2.tex.
%
%<package>\ifdefined\childdocmain\endinput\fi
%<package>\ProvidesFile{childdoc.def}[2018/12/30 v2.0 child document driver]
%<samplemain>\ProvidesFile{cdocsamp.tex}[2018/12/30 v2.0 sample for childdoc]
%<*driver>
%\ProvidesFile{childdoc.drv}[2018/12/30 v2.0 childdoc reference manual file]
\PassOptionsToClass{10pt,a4paper}{article}
\documentclass{ltxdoc}

\usepackage[margin=35mm]{geometry}
\usepackage{hyperref}
\usepackage{hyperxmp}
\usepackage[usenames]{color}

\hypersetup{colorlinks=true}
\hypersetup{pdfstartview=FitH}
\hypersetup{pdfpagemode=UseNone}
\hypersetup{pdfsource={}}
\hypersetup{pdflang={en-UK}}
\hypersetup{pdfcopyright={Copyright 2017-2018 Niklas Beisert.
  This work may be distributed and/or modified under the
  conditions of the LaTeX Project Public License, either version 1.3
  of this license or (at your option) any later version.}}
\hypersetup{pdflicenseurl={http://www.latex-project.org/lppl.txt}}
\hypersetup{pdfcontactaddress={ETH Zurich, ITP, HIT K,
  Wolfgang-Pauli-Strasse 27}}
\hypersetup{pdfcontactpostcode={8093}}
\hypersetup{pdfcontactcity={Zurich}}
\hypersetup{pdfcontactcountry={Switzerland}}
\hypersetup{pdfcontactemail={nbeisert@itp.phys.ethz.ch}}
\hypersetup{pdfcontacturl={http://people.phys.ethz.ch/\xmptilde nbeisert/}}

\newcommand{\secref}[1]{\hyperref[#1]{section \ref*{#1}}}

\parskip1ex
\parindent0pt
\let\olditemize\itemize
\def\itemize{\olditemize\parskip0pt}

\begin{document}

\title{The \textsf{childdoc} Package}
\hypersetup{pdftitle={The childdoc Package}}
\author{Niklas Beisert\\[2ex]
  Institut f\"ur Theoretische Physik\\
  Eidgen\"ossische Technische Hochschule Z\"urich\\
  Wolfgang-Pauli-Strasse 27, 8093 Z\"urich, Switzerland\\[1ex]
  \href{mailto:nbeisert@itp.phys.ethz.ch}
  {\texttt{nbeisert@itp.phys.ethz.ch}}}
\hypersetup{pdfauthor={Niklas Beisert}}
\hypersetup{pdfsubject={Manual for the LaTeX2e Package childdoc}}
\date{30 December 2018, \textsf{v2.0}}
\maketitle

\begin{abstract}\noindent
\textsf{childdoc} is a \LaTeXe{} package
that enables the direct compilation
of document sections included by |\include|
to individual files.
\end{abstract}

\begingroup
\parskip0ex
\tableofcontents
\endgroup

%%%%%%%%%%%%%%%%%%%%%%%%%%%%%%%%%%%%%%%%%%%%%%%%%%%%%%%%%%%%%%%%%%%%%%%%%%%%%%%%
%%%%%%%%%%%%%%%%%%%%%%%%%%%%%%%%%%%%%%%%%%%%%%%%%%%%%%%%%%%%%%%%%%%%%%%%%%%%%%%%
\section{Introduction}

\LaTeX{} provides a mechanism to structure a large document (such as a book)
into a main file and several child files (containing the chapters)
using the |\include| command.
This mechanism is beneficial for documents
which span hundreds of pages in order to
make the source file(s) more manageable.
Moreover, compilation can be restricted to
selected child files by means of the |\includeonly| command.
The latter feature can be used to reduce the compilation time while editing
(this was significantly more useful in the earlier days of \LaTeX{})
or to generate a smaller document which is easier to navigate.
Another application of |\includeonly| is to generate
documents consisting of selected parts of the complete document.

However, there are a few drawbacks of the plain |\include| mechanism:
\begin{itemize}
\item
The child files cannot be compiled on their own,
they can only be compiled via the main file.
A naive editing environment
(such as a text editor with an option
to have the current file processed by \LaTeX)
may require one to switch to the main file before compiling;
attempting to compile the child file produces errors.
\item
The main file must be modified (each time)
to adjust the |\includeonly| command
to the present needs. This easily leaves the main file in a messy state.
\item
The generated document will always carry the filename
of the main document. This is inconvenient if
several child files are to be compiled and
to be kept for distribution.
\end{itemize}

The present package provides a simple interface
to make child files individually compilable by \LaTeX{}.
Compiling a child file then has the same effect as compiling
the main file with an |\includeonly| command
to select the appropriate child.
Moreover the generated document will carry the name of the child
rather than the main file.
This resolves all three above issues.

This feature is meant to make the editing of books,
thesis documents and lecture notes somewhat more convenient.
However, the package can also be used efficiently for
composing a series of documents (such as exercise sheets)
which are typically distributed individually.
It then assists the author in generating the individual documents
(potentially in different versions)
as well as a document containing the collected series.
Another application is in developing style files
or other kinds of included material
where compilation of the style file could redirect
to a sample or test file.

%%%%%%%%%%%%%%%%%%%%%%%%%%%%%%%%%%%%%%%%%%%%%%%%%%%%%%%%%%%%%%%%%%%%%%%%%%%%%%%%
%%%%%%%%%%%%%%%%%%%%%%%%%%%%%%%%%%%%%%%%%%%%%%%%%%%%%%%%%%%%%%%%%%%%%%%%%%%%%%%%
\section{Usage}

First of all, the package \textsf{childdoc} is \emph{not} a standard
\LaTeXe{} |.sty| style file! Therefore it needs to be invoked in
a non-standard way.

%%%%%%%%%%%%%%%%%%%%%%%%%%%%%%%%%%%%%%%%%%%%%%%%%%%%%%%%%%%%%%%%%%%%%%%%%%%%%%%%
\subsection{Included Files}
\label{sec:include}

%%%%%%%%%%%%%%%%%%%%%%%%%%%%%%%%%%%%%%%%
\DescribeMacro{\childdocmain}
To use the package, add the commands
\begin{center}
\begin{tabular}{l}
|\input{childdoc.def}|\\
|\childdocmain{}|\\
\end{tabular}
\end{center}
at the very top of the main \LaTeX{} file,
in particular \emph{before} the |\documentclass| statement!
The argument of |\childdocmain| should be left empty
(but it must be present).

%%%%%%%%%%%%%%%%%%%%%%%%%%%%%%%%%%%%%%%%
\DescribeMacro{\childdocof}
Furthermore, add the commands
\begin{center}
\begin{tabular}{l}
|\input{childdoc.def}|\\
|\childdocof{|\textit{main}|}|\\
\end{tabular}
\end{center}
at the top of every child file \textit{child}
which is included by |\include{|\textit{child}|}|
from within the main file
(or at least for those files to be compiled individually).
The argument \textit{main} must be the filename of the main file.

There are a couple of
considerations in setting up the main and child documents:

%%%%%%%%%%%%%%%%%%%%%%%%%%%%%%%%%%%%%%%%
\paragraph{Restrictions.}

Please note the following restrictions:
\begin{itemize}
\item
|\childdocmain| must be called with one argument \textit{main}
to ensure compatibility with earlier version of the package.
It must either be empty (|\childdocmain{}|)
or precisely match the filename of the main file in which it is specified.
See \secref{sec:detection} for further information.
\item
The filename \textit{main} must be specified without the |.tex| extension.
\item
The filename \textit{main} is case sensitive
(even in case-insensitive file systems)
due to internal string comparison.
\item
The argument \textit{main} should be fully expanded, it cannot be a macro.
\item
Subdirectories and special characters should be avoided in filenames.
\item
The command |\childdocmain{|\textit{main}|}| must be followed by a whitespace.
It should not be followed immediately by another command
or by a comment mark `|%|'.
This is because the \TeX{} parser reads the token immediately following
the argument of |\childdocmain| and puts it
at the beginning of every child section;
however, a white\-space is ignored.
\end{itemize}

%%%%%%%%%%%%%%%%%%%%%%%%%%%%%%%%%%%%%%%%
\paragraph{Content of Main File.}

It is advisable to place all content in the child files included by |\include|.
Any output contained in the main file will appear in all child documents
unless suppressed manually;
it cannot be suppressed automatically by the |\includeonly| directive
and thus should normally be avoided.
A method to include some content in the main file
by means of conditional processing is described in \secref{sec:conditional}.

%%%%%%%%%%%%%%%%%%%%%%%%%%%%%%%%%%%%%%%%
\paragraph{Page Numbering.}

When only a part of the document is compiled,
the appropriate numbering of pages
(as well as other status parameters)
is determined from the |.aux| files.
The latter contain information from previous passes.
However this information needs to propagate through
all intermediate child documents.
Therefore the page numbering in child documents may well
be inconsistent until the complete document is compiled at least once.

A useful (if unconventional) way to always ensure a consistent
page numbering is to restart the numbering in each child document
and denote the pages by `\textit{child}|.|\textit{page}'
where \textit{child} represents the chapter/section number of the child file.
This can be achieved by the command
|\numberwithin{page}{|\textit{child}|}|
of the \textsf{amsmath} package
where \textit{child} can be |chapter| or |section|
depending on the chosen structuring.
Alternatively, one can modify the macro |\thepage| appropriately
and reset the counter |page| at the start of each child file.

%%%%%%%%%%%%%%%%%%%%%%%%%%%%%%%%%%%%%%%%%%%%%%%%%%%%%%%%%%%%%%%%%%%%%%%%%%%%%%%%
\subsection{Conditional Processing}
\label{sec:conditional}

The package provides a mechanism to compile different versions
of a document. To customise the versions further some conditional processing
can come in handy to distinguish which version is being compiled.
The package provides two macros to describe the compilation context:

%%%%%%%%%%%%%%%%%%%%%%%%%%%%%%%%%%%%%%%%
\DescribeMacro{\ifchilddoc}
The conditional |\ifchilddoc| distinguishes between the compilation of
child documents and the main document:
%
\begin{center}
|\ifchilddoc |\textit{child-code}| |[|\||else |\textit{main-code}]| \||fi|
\end{center}

%%%%%%%%%%%%%%%%%%%%%%%%%%%%%%%%%%%%%%%%
\DescribeMacro{\childdocname}
\DescribeMacro{\childdocjob}
The macro |\childdocname| contains the filename (without extension)
of the main or child file being processed.
Note that |\childdocjob| will always contain the name of the main file.

%%%%%%%%%%%%%%%%%%%%%%%%%%%%%%%%%%%%%%%%
\paragraph{Title Page.}

Conditional processing can be used to include a title or banner page
in the main document when proper precautions are taken.
Importantly, the code in the main file should ensure that the page counter
(as well as other status parameters which are stored in the |.aux| files)
takes the same value after the conditional processing.
Otherwise the page numbers may take divergent values
depending on which part is compiled.

For example, a title page could be declared by:
%
\begin{center}
\begin{tabular}{l}
|\ifchilddoc\||else|\\
|\addtocounter{page}{-1}|\\
\textit{code for title page}\\
|\newpage|\\
|\||fi|
\end{tabular}
\end{center}
%
A banner page for the child documents can be generated by:
%
\begin{center}
\begin{tabular}{l}
|\ifchilddoc|\\
|\addtocounter{page}{-1}|\\
\textit{code for banner page}\\
|\newpage|\\
|\||fi|
\end{tabular}
\end{center}
%
Here one could write a message such as:
\begin{center}
|This is the part \childdocname{} of \childdocjob{}.|
\end{center}

%%%%%%%%%%%%%%%%%%%%%%%%%%%%%%%%%%%%%%%%%%%%%%%%%%%%%%%%%%%%%%%%%%%%%%%%%%%%%%%%
\subsection{Flags}
\label{sec:flags}

The package makes it easy to generate different versions
of the main or child documents.
To this end compilation flags can be defined
and assigned different default values.
They will be particularly useful in conjunction
with the forwarding mechanism described in \secref{sec:forward}.

For example, it may be useful to have a flag |\version|
which can be set to |draft| or |final|.
The document source will contain some conditional code
depending on the value of |\version|.
Suppose further, the flag should default to |final| for the main file
and to |draft| for child files
which is a natural assignment for editing the document.
This is achieved by placing the following code
in the preamble of the main document
(below the |\childdocmain| directive):
%
\begin{center}
\begin{tabular}{l}
|\ifchilddoc|\\
|\providecommand{\version}{draft}|\\
|\||else|\\
|\providecommand{\version}{final}|\\
|\||fi|
\end{tabular}
\end{center}
%
The definition by |\providecommand| makes sure
that previous definitions are not overwritten.
Further statements |\providecommand{\version}{...}|
can thus be added before the above code to override it.

For the main file, one might add a line
(between |\childdocmain| and the above block)
%
\begin{center}
|%\ifchilddoc\||else\providecommand{\version}{draft}\||fi|
\end{center}
%
which can be uncommented to produce a draft version.
Likewise one can add a line to the very top of a child file
(above the |\childdocof{|\textit{main}|}| directive)
%
\begin{center}
|%\providecommand{\version}{final}|
\end{center}
%
which can be uncommented to produce the final version of this child document.

%%%%%%%%%%%%%%%%%%%%%%%%%%%%%%%%%%%%%%%%%%%%%%%%%%%%%%%%%%%%%%%%%%%%%%%%%%%%%%%%
\subsection{Forwarding}
\label{sec:forward}

Different versions of the main or child documents
using compilation flags as described in \secref{sec:flags}
can be (permanently) stored in different files
for convenient compilation, viewing and distribution.
To this end, the package defines a command
to pass on compilation to a different file:

%%%%%%%%%%%%%%%%%%%%%%%%%%%%%%%%%%%%%%%%
\DescribeMacro{\childdocforward}
The command |\childdocforward| redirects processing to
another source file:
%
\begin{center}
\begin{tabular}{l}
|\input{childdoc.def}|\\
|\childdocforward[|\textit{main}|]{|\textit{dest}|}|\\
\end{tabular}
\end{center}
%
The argument \textit{dest} is the destination file
(without extension).
It should be the main file or one of the child files.
Note that further \textsf{childdoc} directives
such as |\childdocof| and |\childdocforward|
in the indicated file will be processed in this form.
The optional argument \textit{main}
passes on directly to the main file \textit{main}
while pretending to compile the child \textit{dest}.
This form behaves as if \textit{dest}
issues |\childdocof{|\textit{main}|}| right away,
and no further \textsf{childdoc} directives will be processed.

%%%%%%%%%%%%%%%%%%%%%%%%%%%%%%%%%%%%%%%%
\DescribeMacro{\...prefix}
In the alternative form |\childdocforwardprefix|,
%
\begin{center}
\begin{tabular}{l}
|\input{childdoc.def}|\\
|\childdocforwardprefix[|\textit{main}|]{|\textit{prefix}|}{|\textit{dest}|}|
\end{tabular}
\end{center}
%
the destination file is determined by a pattern
depending on the current file:
To make this work, the current file must be called
`{\textit{prefix}\hspace{0.2em}\textit{suffix}}'
with \textit{prefix} matching precisely the argument.
Processing is then passed on to the file
`{\textit{dest}\hspace{0.2em}\textit{suffix}}'.
Surely, the same effect is achieved by
directly specifying the
argument `{\textit{dest}\hspace{0.2em}\textit{suffix}}'
in the first form.
However, that requires to set up a different file
for each child. With the alternative form of the command
all these files can have exactly the same content
which simplifies setting them up and maintaining them.

For example, the following file |draft.tex|
with a compilation flag |\version| as described in \secref{sec:flags}
compiles the main document as a draft:
%
\begin{center}
\begin{tabular}{l}
|\def\version{draft}|\\
|\input{childdoc.def}|\\
|\childdocforward{|\textit{main}|}|
\end{tabular}
\end{center}
%
Likewise, the following files |final|\textit{nn}|.tex|
compile the final version of the child document
|child|\textit{nn}|.tex|:
%
\begin{center}
\begin{tabular}{l}
|\def\version{final}|\\
|\input{childdoc.def}|\\
|\childdocforwardprefix{final}{child}|
\end{tabular}
\end{center}
%

Note that when several versions of a main file and/or of each child file
are to be generated, it may be convenient to set up a |Makefile| or
shell script to automatise the process.

%%%%%%%%%%%%%%%%%%%%%%%%%%%%%%%%%%%%%%%%%%%%%%%%%%%%%%%%%%%%%%%%%%%%%%%%%%%%%%%%
\subsection{Command Line Processing}
\label{sec:commandline}

The effect of redirection files can also be achieved by invoking
the \LaTeX{} compiler with a more elaborate command line.
Most conveniently this should be done as part
of a shell script or a |Makefile|.

When using \textsf{childdoc} in the main file, the following
command lines effectively perform a redirection
(note that depending on the shell being used,
backslashes may have to be doubled: `|\|' $\to$ `|\\|'):
%
\begin{center}
|... -jobname "|\textit{target}|" |\\|"|[\textit{flags}]%
|\input{childdoc.def}\childdocforward[|\textit{main}|]{|\textit{dest}|}"|
\end{center}
%
Here \textit{target} is the name of the output file,
\textit{main} is the name of the main file
and \textit{dest} is the name of the main or child file to be processed
(all filenames without extensions).
The optional argument \textit{main} can be omitted
if \textit{main} matches \textit{dest}.
Optionally, compilation \textit{flags} can be defined via |\def| commands.
This command line makes the \TeX{} engine believe
it is compiling the file \textit{target}
whose content is specified as the latter parameter.
The provided code then forwards the processing to
\textit{main} or \textit{dest} as described in \secref{sec:forward}.

%%%%%%%%%%%%%%%%%%%%%%%%%%%%%%%%%%%%%%%%%%%%%%%%%%%%%%%%%%%%%%%%%%%%%%%%%%%%%%%%
\subsection{Include by Input}
\label{sec:input}

Including child documents by |\include| has some restrictions by design.
Most notably, the content of a child document always occupies
its own set of pages; pages cannot be shared between child documents.
Usually, this behaviour makes perfect sense
because each child document contain an essential part of the document.
However, in some situations it may be desirable to compose
a document from a collection of parts
without having mandatory page breaks between then.
For this case, the package
provides a mechanism to include parts
by |\input| which can also be processed individually.
However, by construction this mechanism
requires manual handling of the content to be output.

%%%%%%%%%%%%%%%%%%%%%%%%%%%%%%%%%%%%%%%%
\DescribeMacro{\ifchilddocmanual}
The main file should be prepared as usual, see \secref{sec:include}.
However, the document body must make a distinction
between processing of an individual part and of the main document, e.g.:
%
\begin{center}
\begin{tabular}{l}
|\ifchilddocmanual|\\
|\input{\childdocname}|\\
|\||else|\\
\textit{document body with }|\input{|\textit{part}|}|\\
|\||fi|
\end{tabular}
\end{center}
%
The conditional |\ifchilddocmanual| is true whenever
a part to be included by |\input| is being compiled,
and the name of the part is stored in |\childdocname|.

%%%%%%%%%%%%%%%%%%%%%%%%%%%%%%%%%%%%%%%%
\DescribeMacro{\childdocby}
Each part to be included by |\input| should start with:
%
\begin{center}
\begin{tabular}{l}
|\input{childdoc.def}|\\
|\childdocby{|\textit{main}|}|\\
\end{tabular}
\end{center}
%
The directive |\childdocby| is similar to |\childdocof|
described in \secref{sec:include},
but the subsequent selection of content must be done manually.
To that end, both |\ifchilddoc| and |\ifchilddocmanual|
will be true upon processing of a part,
and the name of the part is stored in |\childdocname|.
Note that |\jobname| will be set to the filename of the current part
so that each part receives an individual |.aux| file
that does not interfere with the |.aux| file(s) of the main document.
This behaviour can be altered by the alternative form
|\childdocby[*]{|\textit{main}|}| (with a non-empty optional argument)
which uses the |.aux| file of the main document
by setting |\jobname| to \textit{main}.

%%%%%%%%%%%%%%%%%%%%%%%%%%%%%%%%%%%%%%%%%%%%%%%%%%%%%%%%%%%%%%%%%%%%%%%%%%%%%%%%
\subsection{Driver Development}
\label{sec:driver}

The \textsf{childdoc} mechanism can also be use for the development
of definition files such as \LaTeX{} styles or classes.
This case differs from the above setup with multiple parts
included by |\include| in that no |\includeonly| should be invoked.
This can be achieved by starting the include file
(before |\ProvidesPackage|) with:
%
\begin{center}
\begin{tabular}{l}
|\input{childdoc.def}|\\
|\childdocforward{|\textit{main}|}|\\
\end{tabular}
\end{center}
%
or alternatively with:
%
\begin{center}
\begin{tabular}{l}
|\input{childdoc.def}|\\
|\childdocby{|\textit{main}|}|\\
\end{tabular}
\end{center}
%
Both forms have slightly different effects as described above.
The main file is prepared as usual, see \secref{sec:include}.

%%%%%%%%%%%%%%%%%%%%%%%%%%%%%%%%%%%%%%%%%%%%%%%%%%%%%%%%%%%%%%%%%%%%%%%%%%%%%%%%
\subsection{Legacy Detection}
\label{sec:detection}

The directive |\childdocmain| in the main file can detect
whether the complete document or merely a child is to be compiled
even without using the directive |\childdocof|.
This method is deprecated because it is less robust
and there is no compelling reason to use it;
it is merely provided for backward compatibility
and it may be removed in future versions.

If the detection mechanism is to be used,
it is mandatory to correctly specify
the filename of the main file as the argument of |\childdocmain|:
%
\begin{center}
\begin{tabular}{l}
|\input{childdoc.def}|\\
|\childdocmain{|\textit{main}|}|\\
\end{tabular}
\end{center}
%
If |\jobname| does not match the argument \textit{main} of |\childdocmain|,
it is assumed that |\jobname| points to the child file to be compiled.
When using |\childdocmain| with the main file specified as argument,
it suffices to start a child file
with just |\input{|\textit{main}|}|
without loading of the package and using |\childdocof|.
If instead all processing is done
with the appropriate \textsf{childdoc} directives,
the argument of \textit{main} of |\childdocmain| can be empty.

An alternative version of the command line processing described
in \secref{sec:commandline} using the detection mechanism reads:
%
\begin{center}
|... -jobname "|\textit{target}|" "|[\textit{flags}]%
[|\def\jobname{|\textit{dest}|}|]|\input{|\textit{main}|}"|
\end{center}

%%%%%%%%%%%%%%%%%%%%%%%%%%%%%%%%%%%%%%%%%%%%%%%%%%%%%%%%%%%%%%%%%%%%%%%%%%%%%%%%
\subsection{Manual Code}
\label{sec:manual}

In case one cannot be certain whether the definitions file |childdoc.def|
is installed on the target \TeX{} distribution
and one prefers not to ship it,
it is conceivable to paste a few relevant commands into the sources.

To that end, drop all statements |\input{childdoc.def}|
and perform the replacements as outlined below.
Instead of |\childdocmain{|\textit{main}|}| add the following code
to the top of the main file:
%
\begin{center}
\begin{tabular}{l}
|\||ifdefined\childdocname\endinput\||fi\newif\ifchilddoc|\\
|\edef\childdocname{\scantokens\expandafter{\jobname\noexpand}}|\\
|\def\childdocmain{|\textit{main}|}\||ifx\childdocmain\childdocname\||else|\\
|\childdoctrue\includeonly{\childdocname}\let\jobname\childdocmain\||fi|\\
\end{tabular}
\end{center}
%
Instead of |\childdocof{|\textit{main}|}| just include the main file
at the top of each child file:
%
\begin{center}
|\input{|\textit{main}|}|
\end{center}
%
A simple redirection |\childdocforward{|\textit{dest}|}| is achieved by:
%
\begin{center}
|\def\jobname{|\textit{dest}|}\input{\jobname}|
\end{center}
%
The redirection with prefix
|\childdocforwardprefix[|\textit{prefix}|]{|\textit{dest}|}|
is accomplished by:
%
\begin{center}
\begin{tabular}{l}
|{\edef\jobname{\scantokens\expandafter{\jobname\noexpand}}|\\
|\def\redirectjob |\textit{prefix}|#1~~~{\gdef\jobname{|\textit{dest}|#1}}|\\
|\expandafter\redirectjob\jobname~~~}\input{\jobname}|
\end{tabular}
\end{center}

In an alternative approach,
child documents can be compiled by a specific command line
without additional code or specific definitions:
%
\begin{center}
|... -jobname "|\textit{target}|" "|[\textit{flags}]%
|\includeonly{|\textit{dest}|}\input{|\textit{main}|}"|
\end{center}
%

%%%%%%%%%%%%%%%%%%%%%%%%%%%%%%%%%%%%%%%%%%%%%%%%%%%%%%%%%%%%%%%%%%%%%%%%%%%%%%%%
%%%%%%%%%%%%%%%%%%%%%%%%%%%%%%%%%%%%%%%%%%%%%%%%%%%%%%%%%%%%%%%%%%%%%%%%%%%%%%%%
\section{Information}

%%%%%%%%%%%%%%%%%%%%%%%%%%%%%%%%%%%%%%%%%%%%%%%%%%%%%%%%%%%%%%%%%%%%%%%%%%%%%%%%
\subsection{Copyright}

Copyright \copyright{} 2017--2018 Niklas Beisert

This work may be distributed and/or modified under the
conditions of the \LaTeX{} Project Public License, either version 1.3
of this license or (at your option) any later version.
The latest version of this license is in
  \url{http://www.latex-project.org/lppl.txt}
and version 1.3 or later is part of all distributions of \LaTeX{}
version 2005/12/01 or later.

This work has the LPPL maintenance status `maintained'.

The Current Maintainer of this work is Niklas Beisert.

This work consists of the files |README.txt|, |childdoc.ins| and |childdoc.dtx|
as well as the derived files |childdoc.def|, |cdocsamp.tex|
with |cdocsch1.tex|, |cdocsch2.tex|, |cdocspt3.tex|, |cdocspt4.tex|,
|cdocsdrf.tex|, |cdocsfn1.tex|, |cdocsfn2.tex|
as well as |childdoc.pdf|.

%%%%%%%%%%%%%%%%%%%%%%%%%%%%%%%%%%%%%%%%%%%%%%%%%%%%%%%%%%%%%%%%%%%%%%%%%%%%%%%%
\subsection{Files and Installation}

The package consists of the files:
%
\begin{center}
\begin{tabular}{ll}
    |README.txt|   & readme file \\
    |childdoc.ins| & installation file \\
    |childdoc.dtx| & source file \\
    |childdoc.def| & definition file \\
    |cdocsamp.tex| & sample main file \\
    |cdocsch1.tex| & sample include file \\
    |cdocsch2.tex| & sample include file \\
    |cdocspt3.tex| & sample part file \\
    |cdocspt4.tex| & sample part file \\
    |cdocsdrf.tex| & sample redirection file \\
    |cdocsfn1.tex| & sample redirection file \\
    |cdocsfn2.tex| & sample redirection file \\
    |childdoc.pdf| & manual
\end{tabular}
\end{center}
%
The distribution consists of the files
|README.txt|, |childdoc.ins| and |childdoc.dtx|.
%
\begin{itemize}
\item
Run (pdf)\LaTeX{} on |childdoc.dtx|
to compile the manual |childdoc.pdf| (this file).
\item
Run \LaTeX{} on |childdoc.ins| to create the definitions file |childdoc.def|
and the sample |cdocsamp.tex| with include files
|cdocsch1.tex|, |cdocsch2.tex|, |cdocspt3.tex|, |cdocspt4.tex|,
|cdocsdrf.tex|, |cdocsfn1.tex|, |cdocsfn2.tex|.
Then copy the file |childdoc.def| to an appropriate directory of your \LaTeX{}
distribution, e.g.\ \textit{texmf-root}|/tex/latex/childdoc|.
\end{itemize}

%%%%%%%%%%%%%%%%%%%%%%%%%%%%%%%%%%%%%%%%%%%%%%%%%%%%%%%%%%%%%%%%%%%%%%%%%%%%%%%%
\subsection{Related CTAN Packages}

There are several other packages which offer a similar functionality:
%
\begin{itemize}
\item
The packages
\href{http://ctan.org/pkg/docmute}{\textsf{docmute}},
\href{http://ctan.org/pkg/includex}{\textsf{includex}} and
\href{http://ctan.org/pkg/standalone}{\textsf{standalone}}
provide commands to include only the document body of
a child file thus allowing both files to be compiled individually.
\item
The packages \href{http://ctan.org/pkg/subdocs}{\textsf{subdocs}}
and \href{http://ctan.org/pkg/subfiles}{\textsf{subfiles}}
provide structures in which the main and child documents can be
encapsulated and allowing them to be compiled individually.
The inclusion mechanism is different from the conventional |\include|.
\item
The package \href{http://ctan.org/pkg/combine}{\textsf{combine}}
is an elaborate solution to combine several documents into one.
\end{itemize}
%
See also the CTAN topic \href{http://ctan.org/topic/subdocs}{\textsf{subdocs}}
for further related packages.
The present package differs from the above solutions in that
a document structure constructed with the conventional |\include| mechanism
just needs two extra commands at the top of every file
such that all constituent files can be compiled individually.

%%%%%%%%%%%%%%%%%%%%%%%%%%%%%%%%%%%%%%%%%%%%%%%%%%%%%%%%%%%%%%%%%%%%%%%%%%%%%%%%
%\subsection{Feature Suggestions}
%
%The following is a list of features which may be useful for future
%versions of this package:
%%
%\begin{itemize}
%\item
%\ldots
%\end{itemize}

%%%%%%%%%%%%%%%%%%%%%%%%%%%%%%%%%%%%%%%%%%%%%%%%%%%%%%%%%%%%%%%%%%%%%%%%%%%%%%%%
\subsection{Revision History}

%%%%%%%%%%%%%%%%%%%%%%%%%%%%%%%%%%%%%%%%
\paragraph{v2.0:} 2018/12/30

\begin{itemize}
\item
immediate forward processing
\item
added |\childdocby| mechanism
\item
manual restructured
\end{itemize}

%%%%%%%%%%%%%%%%%%%%%%%%%%%%%%%%%%%%%%%%
\paragraph{v1.6:} 2018/01/17

\begin{itemize}
\item
application for development of include files
\item
corrections to manual
\end{itemize}

%%%%%%%%%%%%%%%%%%%%%%%%%%%%%%%%%%%%%%%%
\paragraph{v1.5:} 2017/05/21

\begin{itemize}
\item
more complete structuring introduced
\item
|\childdocof| introduced
\item
|\childdoc| renamed to |\childdocmain|
\item
|\childredirect| renamed to |\childdocforward| and |\childdocforwardprefix|
and functionality expanded
\end{itemize}

%%%%%%%%%%%%%%%%%%%%%%%%%%%%%%%%%%%%%%%%
\paragraph{v1.0:} 2017/04/27

\begin{itemize}
\item
manual and install package
\item
first version published on CTAN
\end{itemize}

%%%%%%%%%%%%%%%%%%%%%%%%%%%%%%%%%%%%%%%%
\paragraph{v0.6:} 2017/04/26

\begin{itemize}
\item
redirection mechanism added
\end{itemize}

%%%%%%%%%%%%%%%%%%%%%%%%%%%%%%%%%%%%%%%%
\paragraph{v0.5:} 2017/04/26

\begin{itemize}
\item
functionality in definition file
\end{itemize}


%%%%%%%%%%%%%%%%%%%%%%%%%%%%%%%%%%%%%%%%%%%%%%%%%%%%%%%%%%%%%%%%%%%%%%%%%%%%%%%%
%%%%%%%%%%%%%%%%%%%%%%%%%%%%%%%%%%%%%%%%%%%%%%%%%%%%%%%%%%%%%%%%%%%%%%%%%%%%%%%%
%%%%%%%%%%%%%%%%%%%%%%%%%%%%%%%%%%%%%%%%%%%%%%%%%%%%%%%%%%%%%%%%%%%%%%%%%%%%%%%%
\appendix

\settowidth\MacroIndent{\rmfamily\scriptsize 000\ }

 \DocInput{childdoc.dtx}

\end{document}
%</driver>
% \fi
%
% %%%%%%%%%%%%%%%%%%%%%%%%%%%%%%%%%%%%%%%%%%%%%%%%%%%%%%%%%%%%%%%%%%%%%%%%%%%%%%
% %%%%%%%%%%%%%%%%%%%%%%%%%%%%%%%%%%%%%%%%%%%%%%%%%%%%%%%%%%%%%%%%%%%%%%%%%%%%%%
% \section{Sample}
%\iffalse
%<*samplemain>
%\fi
%
% The following presents a sample document
% with two chapters, two parts, a title page,
% a compile flag as well as three forwarding files to set the flag.
% It consists of eight |.tex| files:
% \begin{center}
% \begin{tabular}{ll}
% |cdocsamp.tex|&main file\\
% |cdocsch1.tex|&include file for chapter 1\\
% |cdocsch2.tex|&include file for chapter 2\\
% |cdocspt3.tex|&include file for part 3\\
% |cdocspt4.tex|&include file for part 4\\
% |cdocsdrf.tex|&forwarding file for main file in draft mode\\
% |cdocsfi1.tex|&forwarding file for final version of chapter 1\\
% |cdocsfi2.tex|&forwarding file for final version of chapter 2\\
% \end{tabular}
% \end{center}
% Each of the eight files can be compiled directly by the \LaTeX{} compiler.
%
% %%%%%%%%%%%%%%%%%%%%%%%%%%%%%%%%%%%%%%
% \paragraph{Main File.}
%
% The main file is called |cdocsamp.tex|.
%
% Load the \textsf{childdoc} definitions and
% declare the filename for the main document:
%    \begin{macrocode}
\input{childdoc.def}
\childdocmain{}
%    \end{macrocode}

% Optional override for |\version| flag:
%    \begin{macrocode}
%%\ifchilddoc\else\providecommand{\version}{draft}\fi
%    \end{macrocode}

% Define the default values for the |\version| flag
% (|final| for the main file and |draft| for childs):
%    \begin{macrocode}
\ifchilddoc
\providecommand{\version}{draft}
\else
\providecommand{\version}{final}
\fi
%    \end{macrocode}

% Load the standard document class:
%    \begin{macrocode}
\documentclass[12pt]{article}
%    \end{macrocode}

% Start the document body:
%    \begin{macrocode}
\begin{document}
%    \end{macrocode}

% Declare a title page.
% Print title, part of document being processed and version flag:
%    \begin{macrocode}
\addtocounter{page}{-1}
\begin{center}
{\LARGE\bfseries{}childdoc example\par}
\vspace{1cm}
\ifchilddoc
\ifchilddocmanual part\else chapter\fi:
`\childdocname' of `\childdocjob'\par
\else
main document: `\childdocjob'\par
\fi
version: \version\par
\end{center}
\newpage
%    \end{macrocode}

% Manually include selected file,
% otherwise process as usual:
%    \begin{macrocode}
\ifchilddocmanual
\section*{part `\childdocname'}
\input{\childdocname}
\else
%    \end{macrocode}

% Include the two chapters:
%    \begin{macrocode}
\include{cdocsch1}
\include{cdocsch2}
%    \end{macrocode}

% Include the two parts unless only chapters should be displayed:
%    \begin{macrocode}
\ifchilddoc\else
\section{part three}
\input{cdocspt3}
\section{part four}
\input{cdocspt4}
\fi
%    \end{macrocode}

% Process as usual until here:
%    \begin{macrocode}
\fi
%    \end{macrocode}

% End of document body:
%    \begin{macrocode}
\end{document}
%    \end{macrocode}
%\iffalse
%</samplemain>
%\fi
%
% %%%%%%%%%%%%%%%%%%%%%%%%%%%%%%%%%%%%%%
% \paragraph{Chapter Include Files.}
%
% The include files are called |cdocsch1.tex| and |cdocsch2.tex|.
%
%\iffalse
%<*samplechap1|samplechap2>
%\fi

% Optional override for |\version| flag:
%    \begin{macrocode}
%%\providecommand{\version}{final}
%    \end{macrocode}

% Include the main document:
%    \begin{macrocode}
\input{childdoc.def}
\childdocof{cdocsamp}
%    \end{macrocode}

%\iffalse
%</samplechap1|samplechap2>
%\fi
%
%\iffalse
%<*samplechap1>
%\fi
% Some text for chapter 1:
%    \begin{macrocode}
\section{one}
some text in chapter one
%    \end{macrocode}

%\iffalse
%</samplechap1>
%\fi
% Some text for chapter 2:
%\iffalse
%<*samplechap2>
%\fi
%    \begin{macrocode}
\section{two}
more text in chapter two
%    \end{macrocode}

%\iffalse
%</samplechap2>
%\fi
%
% %%%%%%%%%%%%%%%%%%%%%%%%%%%%%%%%%%%%%%
% \paragraph{Part Include Files.}
%
% The include files are called |cdocspt3.tex| and |cdocspt4.tex|.
%
%\iffalse
%<*samplepart3|samplepart4>
%\fi

% Optional override for |\version| flag:
%    \begin{macrocode}
%%\providecommand{\version}{final}
%    \end{macrocode}

% Include the main document:
%    \begin{macrocode}
\input{childdoc.def}
\childdocby{cdocsamp}
%    \end{macrocode}

%\iffalse
%</samplepart3|samplepart4>
%\fi
%
%\iffalse
%<*samplepart3>
%\fi
% Some text for part 3:
%    \begin{macrocode}
some text in part three
%    \end{macrocode}

%\iffalse
%</samplepart3>
%\fi
% Some text for part 4:
%\iffalse
%<*samplepart4>
%\fi
%    \begin{macrocode}
more text in part four
%    \end{macrocode}

%\iffalse
%</samplepart4>
%\fi
%
% %%%%%%%%%%%%%%%%%%%%%%%%%%%%%%%%%%%%%%
% \paragraph{Forwarding for a Complete Draft.}
%
% The following forwarding file |cdocsdrf.tex|
% compiles the main document in draft mode:
%\iffalse
%<*sampledraft>
%\fi
%    \begin{macrocode}
\def\version{draft}
\input{childdoc.def}
\childdocforward{cdocsamp}
%    \end{macrocode}

%\iffalse
%</sampledraft>
%\fi
%
% %%%%%%%%%%%%%%%%%%%%%%%%%%%%%%%%%%%%%%
% \paragraph{Forwarding for Final Version of the Chapters.}
%
% The following forwarding files |cdocsfn1.tex| and |cdocsfn2.tex|
% (with identical content)
% compile the final versions of the child documents
% |cdocsch1.tex| and |cdocsch2.tex|, respectively:
%\iffalse
%<*samplefinal>
%\fi
%    \begin{macrocode}
\def\version{final}
\input{childdoc.def}
\childdocforwardprefix[cdocsamp]{cdocsfn}{cdocsch}
%    \end{macrocode}

%\iffalse
%</samplefinal>
%\fi
%
% %%%%%%%%%%%%%%%%%%%%%%%%%%%%%%%%%%%%%%
% \paragraph{Command Line Processing.}
%
% The following three command lines generate the output files
% |cdocscld|, |cdocscl1| and |cdocscl2|
% which should be identical to
% |cdocsdrf|, |cdocsch1| and |cdocsfn2|, respectively:
% \begin{center}
% \begin{tabular}{l}
% |latex -jobname cdocscld \|\\
% |  "\def\version{draft}\input{childdoc.def}\childdocforward{cdocsamp}"|\\
% |latex -jobname cdocscl1 \|\\
% |  "\input{childdoc.def}\childdocforward[cdocsamp]{cdocsch1}"|\\
% |latex -jobname cdocscl2 \|\\
% |  "\def\version{final}\input{childdoc.def}\childdocforward{cdocsch2}"|
% \end{tabular}
% \end{center}
% Note that the trailing backslash on each first line
% merely continues the input to the second line
% (for convenient cut ant paste).
% Furthermore, the command |latex| can be replaced by any
% of its alternative versions such as |pdflatex|.
%
% %%%%%%%%%%%%%%%%%%%%%%%%%%%%%%%%%%%%%%%%%%%%%%%%%%%%%%%%%%%%%%%%%%%%%%%%%%%%%%
% %%%%%%%%%%%%%%%%%%%%%%%%%%%%%%%%%%%%%%%%%%%%%%%%%%%%%%%%%%%%%%%%%%%%%%%%%%%%%%
% \section{Implementation}
%\iffalse
%<*package>
%\fi
%
% This section describes the definitions file |childdoc.def|.

% The definitions cannot be loaded using |\usepackage| or |\RequirePackage|
% which has a mechanism to prevent loading a style file more than once.
% When loading the definitions by means of |\input|
% multiple instances have to be prevented manually:
%\iffalse
%This code needs to be before the `\ProvidesFile' directive
%which is defined at the beginning of this file.
%Therefore it is also placed there and commented out here.
%</package>
%<*discard>
%\fi
%    \begin{macrocode}
\ifdefined\childdocmain\endinput\fi
%    \end{macrocode}
%\iffalse
%</discard>
%<*package>
%\fi
%
% \macro{\ifchilddoc}
% \macro{\ifchilddocmanual}
% The conditional |\ifchilddoc| tells whether a
% child (true) or main (false) document is being compiled.
% The conditional |\ifchilddocmanual| tells whether
% the |\includeonly| mechanism is used (false) or
% the selection of child files must be performed manually (true).
% The definitions initialise to false:
%    \begin{macrocode}
\newif\ifchilddoc
\newif\ifchilddocmanual
%    \end{macrocode}

% \macro{\childdocname}
% \macro{\childdocjob}
% The macro |\childdocname| stores the name of the main document
% to be compiled. The macro |\childdocjob| stores the name of
% the document on which the \LaTeX{} compiler was originally invoked.
% The content of |\jobname| cannot be compared
% to filenames specified in the source due to different catcodes.
% The following code rescans |\jobname|, stores the result
% in |\childdocname| and saves a copy in |\childdocjob|:
%    \begin{macrocode}
\edef\childdocname{\scantokens\expandafter{\jobname\noexpand}}
\let\childdocjob\childdocname
%    \end{macrocode}

% \macro{\childdocdisable}
% The macro |\childdocdisable| prevents the main file
% from being processed more than once.
% At this stage, the main document command |\childdocmain|
% is assumed to be called once again where it should do nothing.
% Any subsequent call to it should prevent
% a secondary processing of the main document
% It overwrites the forwarding commands
% |\childdocof| and |\childdocforward|
% with empty macros to prevent further inclusions of the main document:
%    \begin{macrocode}
\newcommand{\childdocdisable}
{
  \renewcommand{\childdocmain}[1]{\renewcommand{\childdocmain}[1]{\endinput}}
  \renewcommand{\childdocof}[1]{}
  \renewcommand{\childdocby}[2][]{}
  \renewcommand{\childdocforward}[2][]{}
  \renewcommand{\childdocdisable}{}
}
%    \end{macrocode}

% \macro{\childdocmain}
% The macro |\childdocmain| is to be called at the top of the main file
% with nothing or the main filename (without extension) as argument.
% First, it breaks loops.
% If the argument is not empty and does not match |\childdocname|
% (which is set by the first inclusion of |childdoc.def|),
% |\ifchilddoc| is set to true, |\includeonly| is applied to the child file
% and |\jobname| is set to the main file
% (for proper handling of |.aux| files):
%    \begin{macrocode}
\newcommand{\childdocmain}[1]
{
  \childdocdisable\childdocmain{}
  \if?#1?\else
    \begingroup
      \def\childdoctmp{#1}
      \ifx\childdoctmp\childdocname
        \def\childdoctmp{}
      \else
        \def\childdoctmp
        {
          \childdoctrue
          \includeonly{\childdocname}
          \def\childdocjob{#1}
          \def\jobname{#1}
        }
      \fi
      \expandafter
    \endgroup
    \childdoctmp
  \fi
}
%    \end{macrocode}

% \macro{\childdocof}
% The command |\childdocof| redirects
% compilation to the main file |#1|.
%    \begin{macrocode}
\newcommand{\childdocof}[1]
{
  \childdocdisable
  \childdoctrue
  \includeonly{\childdocname}
  \def\jobname{#1}
  \def\childdocjob{#1}
  \input{#1}
}
%    \end{macrocode}

% \macro{\childdocby}
% The command |\childdocby| ....
%    \begin{macrocode}
\newcommand{\childdocby}[2][]
{
  \childdocdisable
  \childdoctrue
  \childdocmanualtrue
  \if?#1?\else
    \def\jobname{#2}
  \fi
  \def\childdocjob{#2}
  \input{#2}
  \endinput
}
%    \end{macrocode}

% \macro{\childdocforward}
% The command |\childdocforward| redirects
% compilation to the main file or
% (if the optional argument is given) a child file.
% Parameters are set as if the main file
% or a child file starting with |\childdocof| was compiled.
% Then compilation is handed over to the main file:
%    \begin{macrocode}
\newcommand{\childdocforward}[2][]
{
  \begingroup
    \if?#1?
      \def\childdoctmp
      {
        \def\childdocname{#2}
        \def\childdocjob{#2}
        \def\jobname{#2}
        \input{#2}
        \endinput
      }
    \else
      \def\childdoctmp
      {
        \childdocdisable
        \def\childdocname{#2}
        \childdoctrue
        \includeonly{#2}
        \def\childdocjob{#1}
        \def\jobname{#1}
        \input{#1}
        \endinput
      }
    \fi
    \expandafter
  \endgroup
  \childdoctmp
}
%    \end{macrocode}

% \macro{\childdocforwardprefix}
% The command |\childdocforwardprefix| redirects
% compilation to the main or a child file by means of a pattern.
% The prefix |#1| in the current filename is replaced by |#2|
% and the suffix of the current filename is kept
% (it is assumed that the filename does not contain the substring `|~~~|'
% which is used as a delimiter).
% Compilation is handed over to the new file by |\childdocforward|:
%    \begin{macrocode}
\newcommand{\childdocforwardprefix}[3][]
{
  \begingroup
    \def\childdocextract #2##1~~~{\def\childdoctmp{\childdocforward[#1]{#3##1}}}
    \expandafter\childdocextract\childdocname~~~
    \expandafter
  \endgroup
  \childdoctmp
}
%    \end{macrocode}

% \macro{\childdoc}
% The deprecated macro |\childdoc| is a legacy version of |\childdocmain|:
%    \begin{macrocode}
\newcommand{\childdoc}{\childdocmain}
%    \end{macrocode}

% \macro{\childdocredirect}
% The deprecated macro |\childdocredirect| is a legacy version
% of |\childdocforward| and |\childdocforwardprefix|:
%    \begin{macrocode}
\newcommand{\childdocredirect}[2][]
{
  \begingroup
    \if?#1?
      \def\childdoctmp{\childdocforward{#2}}
    \else
      \def\childdoctmp{\childdocforwardprefix{#1}{#2}}
    \fi
    \expandafter
  \endgroup
  \childdoctmp
}
%    \end{macrocode}

%\iffalse
%</package>
%\fi
%
\endinput
|\\
|\childdocby{|\textit{main}|}|\\
\end{tabular}
\end{center}
%
Both forms have slightly different effects as described above.
The main file is prepared as usual, see \secref{sec:include}.

%%%%%%%%%%%%%%%%%%%%%%%%%%%%%%%%%%%%%%%%%%%%%%%%%%%%%%%%%%%%%%%%%%%%%%%%%%%%%%%%
\subsection{Legacy Detection}
\label{sec:detection}

The directive |\childdocmain| in the main file can detect
whether the complete document or merely a child is to be compiled
even without using the directive |\childdocof|.
This method is deprecated because it is less robust
and there is no compelling reason to use it;
it is merely provided for backward compatibility
and it may be removed in future versions.

If the detection mechanism is to be used,
it is mandatory to correctly specify
the filename of the main file as the argument of |\childdocmain|:
%
\begin{center}
\begin{tabular}{l}
|% \iffalse
%
% childdoc.dtx Copyright (C) 2017-2018 Niklas Beisert
%
% This work may be distributed and/or modified under the
% conditions of the LaTeX Project Public License, either version 1.3
% of this license or (at your option) any later version.
% The latest version of this license is in
%   http://www.latex-project.org/lppl.txt
% and version 1.3 or later is part of all distributions of LaTeX
% version 2005/12/01 or later.
%
% This work has the LPPL maintenance status `maintained'.
%
% The Current Maintainer of this work is Niklas Beisert.
%
% This work consists of the files childdoc.dtx and childdoc.ins
% and the derived files childdoc.def and cdocsamp.tex with
% cdocsch1.tex, cdocsch2.tex, cdocsdrf.tex, cdocsfn1.tex, cdocsfn2.tex.
%
%<package>\ifdefined\childdocmain\endinput\fi
%<package>\ProvidesFile{childdoc.def}[2018/12/30 v2.0 child document driver]
%<samplemain>\ProvidesFile{cdocsamp.tex}[2018/12/30 v2.0 sample for childdoc]
%<*driver>
%\ProvidesFile{childdoc.drv}[2018/12/30 v2.0 childdoc reference manual file]
\PassOptionsToClass{10pt,a4paper}{article}
\documentclass{ltxdoc}

\usepackage[margin=35mm]{geometry}
\usepackage{hyperref}
\usepackage{hyperxmp}
\usepackage[usenames]{color}

\hypersetup{colorlinks=true}
\hypersetup{pdfstartview=FitH}
\hypersetup{pdfpagemode=UseNone}
\hypersetup{pdfsource={}}
\hypersetup{pdflang={en-UK}}
\hypersetup{pdfcopyright={Copyright 2017-2018 Niklas Beisert.
  This work may be distributed and/or modified under the
  conditions of the LaTeX Project Public License, either version 1.3
  of this license or (at your option) any later version.}}
\hypersetup{pdflicenseurl={http://www.latex-project.org/lppl.txt}}
\hypersetup{pdfcontactaddress={ETH Zurich, ITP, HIT K,
  Wolfgang-Pauli-Strasse 27}}
\hypersetup{pdfcontactpostcode={8093}}
\hypersetup{pdfcontactcity={Zurich}}
\hypersetup{pdfcontactcountry={Switzerland}}
\hypersetup{pdfcontactemail={nbeisert@itp.phys.ethz.ch}}
\hypersetup{pdfcontacturl={http://people.phys.ethz.ch/\xmptilde nbeisert/}}

\newcommand{\secref}[1]{\hyperref[#1]{section \ref*{#1}}}

\parskip1ex
\parindent0pt
\let\olditemize\itemize
\def\itemize{\olditemize\parskip0pt}

\begin{document}

\title{The \textsf{childdoc} Package}
\hypersetup{pdftitle={The childdoc Package}}
\author{Niklas Beisert\\[2ex]
  Institut f\"ur Theoretische Physik\\
  Eidgen\"ossische Technische Hochschule Z\"urich\\
  Wolfgang-Pauli-Strasse 27, 8093 Z\"urich, Switzerland\\[1ex]
  \href{mailto:nbeisert@itp.phys.ethz.ch}
  {\texttt{nbeisert@itp.phys.ethz.ch}}}
\hypersetup{pdfauthor={Niklas Beisert}}
\hypersetup{pdfsubject={Manual for the LaTeX2e Package childdoc}}
\date{30 December 2018, \textsf{v2.0}}
\maketitle

\begin{abstract}\noindent
\textsf{childdoc} is a \LaTeXe{} package
that enables the direct compilation
of document sections included by |\include|
to individual files.
\end{abstract}

\begingroup
\parskip0ex
\tableofcontents
\endgroup

%%%%%%%%%%%%%%%%%%%%%%%%%%%%%%%%%%%%%%%%%%%%%%%%%%%%%%%%%%%%%%%%%%%%%%%%%%%%%%%%
%%%%%%%%%%%%%%%%%%%%%%%%%%%%%%%%%%%%%%%%%%%%%%%%%%%%%%%%%%%%%%%%%%%%%%%%%%%%%%%%
\section{Introduction}

\LaTeX{} provides a mechanism to structure a large document (such as a book)
into a main file and several child files (containing the chapters)
using the |\include| command.
This mechanism is beneficial for documents
which span hundreds of pages in order to
make the source file(s) more manageable.
Moreover, compilation can be restricted to
selected child files by means of the |\includeonly| command.
The latter feature can be used to reduce the compilation time while editing
(this was significantly more useful in the earlier days of \LaTeX{})
or to generate a smaller document which is easier to navigate.
Another application of |\includeonly| is to generate
documents consisting of selected parts of the complete document.

However, there are a few drawbacks of the plain |\include| mechanism:
\begin{itemize}
\item
The child files cannot be compiled on their own,
they can only be compiled via the main file.
A naive editing environment
(such as a text editor with an option
to have the current file processed by \LaTeX)
may require one to switch to the main file before compiling;
attempting to compile the child file produces errors.
\item
The main file must be modified (each time)
to adjust the |\includeonly| command
to the present needs. This easily leaves the main file in a messy state.
\item
The generated document will always carry the filename
of the main document. This is inconvenient if
several child files are to be compiled and
to be kept for distribution.
\end{itemize}

The present package provides a simple interface
to make child files individually compilable by \LaTeX{}.
Compiling a child file then has the same effect as compiling
the main file with an |\includeonly| command
to select the appropriate child.
Moreover the generated document will carry the name of the child
rather than the main file.
This resolves all three above issues.

This feature is meant to make the editing of books,
thesis documents and lecture notes somewhat more convenient.
However, the package can also be used efficiently for
composing a series of documents (such as exercise sheets)
which are typically distributed individually.
It then assists the author in generating the individual documents
(potentially in different versions)
as well as a document containing the collected series.
Another application is in developing style files
or other kinds of included material
where compilation of the style file could redirect
to a sample or test file.

%%%%%%%%%%%%%%%%%%%%%%%%%%%%%%%%%%%%%%%%%%%%%%%%%%%%%%%%%%%%%%%%%%%%%%%%%%%%%%%%
%%%%%%%%%%%%%%%%%%%%%%%%%%%%%%%%%%%%%%%%%%%%%%%%%%%%%%%%%%%%%%%%%%%%%%%%%%%%%%%%
\section{Usage}

First of all, the package \textsf{childdoc} is \emph{not} a standard
\LaTeXe{} |.sty| style file! Therefore it needs to be invoked in
a non-standard way.

%%%%%%%%%%%%%%%%%%%%%%%%%%%%%%%%%%%%%%%%%%%%%%%%%%%%%%%%%%%%%%%%%%%%%%%%%%%%%%%%
\subsection{Included Files}
\label{sec:include}

%%%%%%%%%%%%%%%%%%%%%%%%%%%%%%%%%%%%%%%%
\DescribeMacro{\childdocmain}
To use the package, add the commands
\begin{center}
\begin{tabular}{l}
|\input{childdoc.def}|\\
|\childdocmain{}|\\
\end{tabular}
\end{center}
at the very top of the main \LaTeX{} file,
in particular \emph{before} the |\documentclass| statement!
The argument of |\childdocmain| should be left empty
(but it must be present).

%%%%%%%%%%%%%%%%%%%%%%%%%%%%%%%%%%%%%%%%
\DescribeMacro{\childdocof}
Furthermore, add the commands
\begin{center}
\begin{tabular}{l}
|\input{childdoc.def}|\\
|\childdocof{|\textit{main}|}|\\
\end{tabular}
\end{center}
at the top of every child file \textit{child}
which is included by |\include{|\textit{child}|}|
from within the main file
(or at least for those files to be compiled individually).
The argument \textit{main} must be the filename of the main file.

There are a couple of
considerations in setting up the main and child documents:

%%%%%%%%%%%%%%%%%%%%%%%%%%%%%%%%%%%%%%%%
\paragraph{Restrictions.}

Please note the following restrictions:
\begin{itemize}
\item
|\childdocmain| must be called with one argument \textit{main}
to ensure compatibility with earlier version of the package.
It must either be empty (|\childdocmain{}|)
or precisely match the filename of the main file in which it is specified.
See \secref{sec:detection} for further information.
\item
The filename \textit{main} must be specified without the |.tex| extension.
\item
The filename \textit{main} is case sensitive
(even in case-insensitive file systems)
due to internal string comparison.
\item
The argument \textit{main} should be fully expanded, it cannot be a macro.
\item
Subdirectories and special characters should be avoided in filenames.
\item
The command |\childdocmain{|\textit{main}|}| must be followed by a whitespace.
It should not be followed immediately by another command
or by a comment mark `|%|'.
This is because the \TeX{} parser reads the token immediately following
the argument of |\childdocmain| and puts it
at the beginning of every child section;
however, a white\-space is ignored.
\end{itemize}

%%%%%%%%%%%%%%%%%%%%%%%%%%%%%%%%%%%%%%%%
\paragraph{Content of Main File.}

It is advisable to place all content in the child files included by |\include|.
Any output contained in the main file will appear in all child documents
unless suppressed manually;
it cannot be suppressed automatically by the |\includeonly| directive
and thus should normally be avoided.
A method to include some content in the main file
by means of conditional processing is described in \secref{sec:conditional}.

%%%%%%%%%%%%%%%%%%%%%%%%%%%%%%%%%%%%%%%%
\paragraph{Page Numbering.}

When only a part of the document is compiled,
the appropriate numbering of pages
(as well as other status parameters)
is determined from the |.aux| files.
The latter contain information from previous passes.
However this information needs to propagate through
all intermediate child documents.
Therefore the page numbering in child documents may well
be inconsistent until the complete document is compiled at least once.

A useful (if unconventional) way to always ensure a consistent
page numbering is to restart the numbering in each child document
and denote the pages by `\textit{child}|.|\textit{page}'
where \textit{child} represents the chapter/section number of the child file.
This can be achieved by the command
|\numberwithin{page}{|\textit{child}|}|
of the \textsf{amsmath} package
where \textit{child} can be |chapter| or |section|
depending on the chosen structuring.
Alternatively, one can modify the macro |\thepage| appropriately
and reset the counter |page| at the start of each child file.

%%%%%%%%%%%%%%%%%%%%%%%%%%%%%%%%%%%%%%%%%%%%%%%%%%%%%%%%%%%%%%%%%%%%%%%%%%%%%%%%
\subsection{Conditional Processing}
\label{sec:conditional}

The package provides a mechanism to compile different versions
of a document. To customise the versions further some conditional processing
can come in handy to distinguish which version is being compiled.
The package provides two macros to describe the compilation context:

%%%%%%%%%%%%%%%%%%%%%%%%%%%%%%%%%%%%%%%%
\DescribeMacro{\ifchilddoc}
The conditional |\ifchilddoc| distinguishes between the compilation of
child documents and the main document:
%
\begin{center}
|\ifchilddoc |\textit{child-code}| |[|\||else |\textit{main-code}]| \||fi|
\end{center}

%%%%%%%%%%%%%%%%%%%%%%%%%%%%%%%%%%%%%%%%
\DescribeMacro{\childdocname}
\DescribeMacro{\childdocjob}
The macro |\childdocname| contains the filename (without extension)
of the main or child file being processed.
Note that |\childdocjob| will always contain the name of the main file.

%%%%%%%%%%%%%%%%%%%%%%%%%%%%%%%%%%%%%%%%
\paragraph{Title Page.}

Conditional processing can be used to include a title or banner page
in the main document when proper precautions are taken.
Importantly, the code in the main file should ensure that the page counter
(as well as other status parameters which are stored in the |.aux| files)
takes the same value after the conditional processing.
Otherwise the page numbers may take divergent values
depending on which part is compiled.

For example, a title page could be declared by:
%
\begin{center}
\begin{tabular}{l}
|\ifchilddoc\||else|\\
|\addtocounter{page}{-1}|\\
\textit{code for title page}\\
|\newpage|\\
|\||fi|
\end{tabular}
\end{center}
%
A banner page for the child documents can be generated by:
%
\begin{center}
\begin{tabular}{l}
|\ifchilddoc|\\
|\addtocounter{page}{-1}|\\
\textit{code for banner page}\\
|\newpage|\\
|\||fi|
\end{tabular}
\end{center}
%
Here one could write a message such as:
\begin{center}
|This is the part \childdocname{} of \childdocjob{}.|
\end{center}

%%%%%%%%%%%%%%%%%%%%%%%%%%%%%%%%%%%%%%%%%%%%%%%%%%%%%%%%%%%%%%%%%%%%%%%%%%%%%%%%
\subsection{Flags}
\label{sec:flags}

The package makes it easy to generate different versions
of the main or child documents.
To this end compilation flags can be defined
and assigned different default values.
They will be particularly useful in conjunction
with the forwarding mechanism described in \secref{sec:forward}.

For example, it may be useful to have a flag |\version|
which can be set to |draft| or |final|.
The document source will contain some conditional code
depending on the value of |\version|.
Suppose further, the flag should default to |final| for the main file
and to |draft| for child files
which is a natural assignment for editing the document.
This is achieved by placing the following code
in the preamble of the main document
(below the |\childdocmain| directive):
%
\begin{center}
\begin{tabular}{l}
|\ifchilddoc|\\
|\providecommand{\version}{draft}|\\
|\||else|\\
|\providecommand{\version}{final}|\\
|\||fi|
\end{tabular}
\end{center}
%
The definition by |\providecommand| makes sure
that previous definitions are not overwritten.
Further statements |\providecommand{\version}{...}|
can thus be added before the above code to override it.

For the main file, one might add a line
(between |\childdocmain| and the above block)
%
\begin{center}
|%\ifchilddoc\||else\providecommand{\version}{draft}\||fi|
\end{center}
%
which can be uncommented to produce a draft version.
Likewise one can add a line to the very top of a child file
(above the |\childdocof{|\textit{main}|}| directive)
%
\begin{center}
|%\providecommand{\version}{final}|
\end{center}
%
which can be uncommented to produce the final version of this child document.

%%%%%%%%%%%%%%%%%%%%%%%%%%%%%%%%%%%%%%%%%%%%%%%%%%%%%%%%%%%%%%%%%%%%%%%%%%%%%%%%
\subsection{Forwarding}
\label{sec:forward}

Different versions of the main or child documents
using compilation flags as described in \secref{sec:flags}
can be (permanently) stored in different files
for convenient compilation, viewing and distribution.
To this end, the package defines a command
to pass on compilation to a different file:

%%%%%%%%%%%%%%%%%%%%%%%%%%%%%%%%%%%%%%%%
\DescribeMacro{\childdocforward}
The command |\childdocforward| redirects processing to
another source file:
%
\begin{center}
\begin{tabular}{l}
|\input{childdoc.def}|\\
|\childdocforward[|\textit{main}|]{|\textit{dest}|}|\\
\end{tabular}
\end{center}
%
The argument \textit{dest} is the destination file
(without extension).
It should be the main file or one of the child files.
Note that further \textsf{childdoc} directives
such as |\childdocof| and |\childdocforward|
in the indicated file will be processed in this form.
The optional argument \textit{main}
passes on directly to the main file \textit{main}
while pretending to compile the child \textit{dest}.
This form behaves as if \textit{dest}
issues |\childdocof{|\textit{main}|}| right away,
and no further \textsf{childdoc} directives will be processed.

%%%%%%%%%%%%%%%%%%%%%%%%%%%%%%%%%%%%%%%%
\DescribeMacro{\...prefix}
In the alternative form |\childdocforwardprefix|,
%
\begin{center}
\begin{tabular}{l}
|\input{childdoc.def}|\\
|\childdocforwardprefix[|\textit{main}|]{|\textit{prefix}|}{|\textit{dest}|}|
\end{tabular}
\end{center}
%
the destination file is determined by a pattern
depending on the current file:
To make this work, the current file must be called
`{\textit{prefix}\hspace{0.2em}\textit{suffix}}'
with \textit{prefix} matching precisely the argument.
Processing is then passed on to the file
`{\textit{dest}\hspace{0.2em}\textit{suffix}}'.
Surely, the same effect is achieved by
directly specifying the
argument `{\textit{dest}\hspace{0.2em}\textit{suffix}}'
in the first form.
However, that requires to set up a different file
for each child. With the alternative form of the command
all these files can have exactly the same content
which simplifies setting them up and maintaining them.

For example, the following file |draft.tex|
with a compilation flag |\version| as described in \secref{sec:flags}
compiles the main document as a draft:
%
\begin{center}
\begin{tabular}{l}
|\def\version{draft}|\\
|\input{childdoc.def}|\\
|\childdocforward{|\textit{main}|}|
\end{tabular}
\end{center}
%
Likewise, the following files |final|\textit{nn}|.tex|
compile the final version of the child document
|child|\textit{nn}|.tex|:
%
\begin{center}
\begin{tabular}{l}
|\def\version{final}|\\
|\input{childdoc.def}|\\
|\childdocforwardprefix{final}{child}|
\end{tabular}
\end{center}
%

Note that when several versions of a main file and/or of each child file
are to be generated, it may be convenient to set up a |Makefile| or
shell script to automatise the process.

%%%%%%%%%%%%%%%%%%%%%%%%%%%%%%%%%%%%%%%%%%%%%%%%%%%%%%%%%%%%%%%%%%%%%%%%%%%%%%%%
\subsection{Command Line Processing}
\label{sec:commandline}

The effect of redirection files can also be achieved by invoking
the \LaTeX{} compiler with a more elaborate command line.
Most conveniently this should be done as part
of a shell script or a |Makefile|.

When using \textsf{childdoc} in the main file, the following
command lines effectively perform a redirection
(note that depending on the shell being used,
backslashes may have to be doubled: `|\|' $\to$ `|\\|'):
%
\begin{center}
|... -jobname "|\textit{target}|" |\\|"|[\textit{flags}]%
|\input{childdoc.def}\childdocforward[|\textit{main}|]{|\textit{dest}|}"|
\end{center}
%
Here \textit{target} is the name of the output file,
\textit{main} is the name of the main file
and \textit{dest} is the name of the main or child file to be processed
(all filenames without extensions).
The optional argument \textit{main} can be omitted
if \textit{main} matches \textit{dest}.
Optionally, compilation \textit{flags} can be defined via |\def| commands.
This command line makes the \TeX{} engine believe
it is compiling the file \textit{target}
whose content is specified as the latter parameter.
The provided code then forwards the processing to
\textit{main} or \textit{dest} as described in \secref{sec:forward}.

%%%%%%%%%%%%%%%%%%%%%%%%%%%%%%%%%%%%%%%%%%%%%%%%%%%%%%%%%%%%%%%%%%%%%%%%%%%%%%%%
\subsection{Include by Input}
\label{sec:input}

Including child documents by |\include| has some restrictions by design.
Most notably, the content of a child document always occupies
its own set of pages; pages cannot be shared between child documents.
Usually, this behaviour makes perfect sense
because each child document contain an essential part of the document.
However, in some situations it may be desirable to compose
a document from a collection of parts
without having mandatory page breaks between then.
For this case, the package
provides a mechanism to include parts
by |\input| which can also be processed individually.
However, by construction this mechanism
requires manual handling of the content to be output.

%%%%%%%%%%%%%%%%%%%%%%%%%%%%%%%%%%%%%%%%
\DescribeMacro{\ifchilddocmanual}
The main file should be prepared as usual, see \secref{sec:include}.
However, the document body must make a distinction
between processing of an individual part and of the main document, e.g.:
%
\begin{center}
\begin{tabular}{l}
|\ifchilddocmanual|\\
|\input{\childdocname}|\\
|\||else|\\
\textit{document body with }|\input{|\textit{part}|}|\\
|\||fi|
\end{tabular}
\end{center}
%
The conditional |\ifchilddocmanual| is true whenever
a part to be included by |\input| is being compiled,
and the name of the part is stored in |\childdocname|.

%%%%%%%%%%%%%%%%%%%%%%%%%%%%%%%%%%%%%%%%
\DescribeMacro{\childdocby}
Each part to be included by |\input| should start with:
%
\begin{center}
\begin{tabular}{l}
|\input{childdoc.def}|\\
|\childdocby{|\textit{main}|}|\\
\end{tabular}
\end{center}
%
The directive |\childdocby| is similar to |\childdocof|
described in \secref{sec:include},
but the subsequent selection of content must be done manually.
To that end, both |\ifchilddoc| and |\ifchilddocmanual|
will be true upon processing of a part,
and the name of the part is stored in |\childdocname|.
Note that |\jobname| will be set to the filename of the current part
so that each part receives an individual |.aux| file
that does not interfere with the |.aux| file(s) of the main document.
This behaviour can be altered by the alternative form
|\childdocby[*]{|\textit{main}|}| (with a non-empty optional argument)
which uses the |.aux| file of the main document
by setting |\jobname| to \textit{main}.

%%%%%%%%%%%%%%%%%%%%%%%%%%%%%%%%%%%%%%%%%%%%%%%%%%%%%%%%%%%%%%%%%%%%%%%%%%%%%%%%
\subsection{Driver Development}
\label{sec:driver}

The \textsf{childdoc} mechanism can also be use for the development
of definition files such as \LaTeX{} styles or classes.
This case differs from the above setup with multiple parts
included by |\include| in that no |\includeonly| should be invoked.
This can be achieved by starting the include file
(before |\ProvidesPackage|) with:
%
\begin{center}
\begin{tabular}{l}
|\input{childdoc.def}|\\
|\childdocforward{|\textit{main}|}|\\
\end{tabular}
\end{center}
%
or alternatively with:
%
\begin{center}
\begin{tabular}{l}
|\input{childdoc.def}|\\
|\childdocby{|\textit{main}|}|\\
\end{tabular}
\end{center}
%
Both forms have slightly different effects as described above.
The main file is prepared as usual, see \secref{sec:include}.

%%%%%%%%%%%%%%%%%%%%%%%%%%%%%%%%%%%%%%%%%%%%%%%%%%%%%%%%%%%%%%%%%%%%%%%%%%%%%%%%
\subsection{Legacy Detection}
\label{sec:detection}

The directive |\childdocmain| in the main file can detect
whether the complete document or merely a child is to be compiled
even without using the directive |\childdocof|.
This method is deprecated because it is less robust
and there is no compelling reason to use it;
it is merely provided for backward compatibility
and it may be removed in future versions.

If the detection mechanism is to be used,
it is mandatory to correctly specify
the filename of the main file as the argument of |\childdocmain|:
%
\begin{center}
\begin{tabular}{l}
|\input{childdoc.def}|\\
|\childdocmain{|\textit{main}|}|\\
\end{tabular}
\end{center}
%
If |\jobname| does not match the argument \textit{main} of |\childdocmain|,
it is assumed that |\jobname| points to the child file to be compiled.
When using |\childdocmain| with the main file specified as argument,
it suffices to start a child file
with just |\input{|\textit{main}|}|
without loading of the package and using |\childdocof|.
If instead all processing is done
with the appropriate \textsf{childdoc} directives,
the argument of \textit{main} of |\childdocmain| can be empty.

An alternative version of the command line processing described
in \secref{sec:commandline} using the detection mechanism reads:
%
\begin{center}
|... -jobname "|\textit{target}|" "|[\textit{flags}]%
[|\def\jobname{|\textit{dest}|}|]|\input{|\textit{main}|}"|
\end{center}

%%%%%%%%%%%%%%%%%%%%%%%%%%%%%%%%%%%%%%%%%%%%%%%%%%%%%%%%%%%%%%%%%%%%%%%%%%%%%%%%
\subsection{Manual Code}
\label{sec:manual}

In case one cannot be certain whether the definitions file |childdoc.def|
is installed on the target \TeX{} distribution
and one prefers not to ship it,
it is conceivable to paste a few relevant commands into the sources.

To that end, drop all statements |\input{childdoc.def}|
and perform the replacements as outlined below.
Instead of |\childdocmain{|\textit{main}|}| add the following code
to the top of the main file:
%
\begin{center}
\begin{tabular}{l}
|\||ifdefined\childdocname\endinput\||fi\newif\ifchilddoc|\\
|\edef\childdocname{\scantokens\expandafter{\jobname\noexpand}}|\\
|\def\childdocmain{|\textit{main}|}\||ifx\childdocmain\childdocname\||else|\\
|\childdoctrue\includeonly{\childdocname}\let\jobname\childdocmain\||fi|\\
\end{tabular}
\end{center}
%
Instead of |\childdocof{|\textit{main}|}| just include the main file
at the top of each child file:
%
\begin{center}
|\input{|\textit{main}|}|
\end{center}
%
A simple redirection |\childdocforward{|\textit{dest}|}| is achieved by:
%
\begin{center}
|\def\jobname{|\textit{dest}|}\input{\jobname}|
\end{center}
%
The redirection with prefix
|\childdocforwardprefix[|\textit{prefix}|]{|\textit{dest}|}|
is accomplished by:
%
\begin{center}
\begin{tabular}{l}
|{\edef\jobname{\scantokens\expandafter{\jobname\noexpand}}|\\
|\def\redirectjob |\textit{prefix}|#1~~~{\gdef\jobname{|\textit{dest}|#1}}|\\
|\expandafter\redirectjob\jobname~~~}\input{\jobname}|
\end{tabular}
\end{center}

In an alternative approach,
child documents can be compiled by a specific command line
without additional code or specific definitions:
%
\begin{center}
|... -jobname "|\textit{target}|" "|[\textit{flags}]%
|\includeonly{|\textit{dest}|}\input{|\textit{main}|}"|
\end{center}
%

%%%%%%%%%%%%%%%%%%%%%%%%%%%%%%%%%%%%%%%%%%%%%%%%%%%%%%%%%%%%%%%%%%%%%%%%%%%%%%%%
%%%%%%%%%%%%%%%%%%%%%%%%%%%%%%%%%%%%%%%%%%%%%%%%%%%%%%%%%%%%%%%%%%%%%%%%%%%%%%%%
\section{Information}

%%%%%%%%%%%%%%%%%%%%%%%%%%%%%%%%%%%%%%%%%%%%%%%%%%%%%%%%%%%%%%%%%%%%%%%%%%%%%%%%
\subsection{Copyright}

Copyright \copyright{} 2017--2018 Niklas Beisert

This work may be distributed and/or modified under the
conditions of the \LaTeX{} Project Public License, either version 1.3
of this license or (at your option) any later version.
The latest version of this license is in
  \url{http://www.latex-project.org/lppl.txt}
and version 1.3 or later is part of all distributions of \LaTeX{}
version 2005/12/01 or later.

This work has the LPPL maintenance status `maintained'.

The Current Maintainer of this work is Niklas Beisert.

This work consists of the files |README.txt|, |childdoc.ins| and |childdoc.dtx|
as well as the derived files |childdoc.def|, |cdocsamp.tex|
with |cdocsch1.tex|, |cdocsch2.tex|, |cdocspt3.tex|, |cdocspt4.tex|,
|cdocsdrf.tex|, |cdocsfn1.tex|, |cdocsfn2.tex|
as well as |childdoc.pdf|.

%%%%%%%%%%%%%%%%%%%%%%%%%%%%%%%%%%%%%%%%%%%%%%%%%%%%%%%%%%%%%%%%%%%%%%%%%%%%%%%%
\subsection{Files and Installation}

The package consists of the files:
%
\begin{center}
\begin{tabular}{ll}
    |README.txt|   & readme file \\
    |childdoc.ins| & installation file \\
    |childdoc.dtx| & source file \\
    |childdoc.def| & definition file \\
    |cdocsamp.tex| & sample main file \\
    |cdocsch1.tex| & sample include file \\
    |cdocsch2.tex| & sample include file \\
    |cdocspt3.tex| & sample part file \\
    |cdocspt4.tex| & sample part file \\
    |cdocsdrf.tex| & sample redirection file \\
    |cdocsfn1.tex| & sample redirection file \\
    |cdocsfn2.tex| & sample redirection file \\
    |childdoc.pdf| & manual
\end{tabular}
\end{center}
%
The distribution consists of the files
|README.txt|, |childdoc.ins| and |childdoc.dtx|.
%
\begin{itemize}
\item
Run (pdf)\LaTeX{} on |childdoc.dtx|
to compile the manual |childdoc.pdf| (this file).
\item
Run \LaTeX{} on |childdoc.ins| to create the definitions file |childdoc.def|
and the sample |cdocsamp.tex| with include files
|cdocsch1.tex|, |cdocsch2.tex|, |cdocspt3.tex|, |cdocspt4.tex|,
|cdocsdrf.tex|, |cdocsfn1.tex|, |cdocsfn2.tex|.
Then copy the file |childdoc.def| to an appropriate directory of your \LaTeX{}
distribution, e.g.\ \textit{texmf-root}|/tex/latex/childdoc|.
\end{itemize}

%%%%%%%%%%%%%%%%%%%%%%%%%%%%%%%%%%%%%%%%%%%%%%%%%%%%%%%%%%%%%%%%%%%%%%%%%%%%%%%%
\subsection{Related CTAN Packages}

There are several other packages which offer a similar functionality:
%
\begin{itemize}
\item
The packages
\href{http://ctan.org/pkg/docmute}{\textsf{docmute}},
\href{http://ctan.org/pkg/includex}{\textsf{includex}} and
\href{http://ctan.org/pkg/standalone}{\textsf{standalone}}
provide commands to include only the document body of
a child file thus allowing both files to be compiled individually.
\item
The packages \href{http://ctan.org/pkg/subdocs}{\textsf{subdocs}}
and \href{http://ctan.org/pkg/subfiles}{\textsf{subfiles}}
provide structures in which the main and child documents can be
encapsulated and allowing them to be compiled individually.
The inclusion mechanism is different from the conventional |\include|.
\item
The package \href{http://ctan.org/pkg/combine}{\textsf{combine}}
is an elaborate solution to combine several documents into one.
\end{itemize}
%
See also the CTAN topic \href{http://ctan.org/topic/subdocs}{\textsf{subdocs}}
for further related packages.
The present package differs from the above solutions in that
a document structure constructed with the conventional |\include| mechanism
just needs two extra commands at the top of every file
such that all constituent files can be compiled individually.

%%%%%%%%%%%%%%%%%%%%%%%%%%%%%%%%%%%%%%%%%%%%%%%%%%%%%%%%%%%%%%%%%%%%%%%%%%%%%%%%
%\subsection{Feature Suggestions}
%
%The following is a list of features which may be useful for future
%versions of this package:
%%
%\begin{itemize}
%\item
%\ldots
%\end{itemize}

%%%%%%%%%%%%%%%%%%%%%%%%%%%%%%%%%%%%%%%%%%%%%%%%%%%%%%%%%%%%%%%%%%%%%%%%%%%%%%%%
\subsection{Revision History}

%%%%%%%%%%%%%%%%%%%%%%%%%%%%%%%%%%%%%%%%
\paragraph{v2.0:} 2018/12/30

\begin{itemize}
\item
immediate forward processing
\item
added |\childdocby| mechanism
\item
manual restructured
\end{itemize}

%%%%%%%%%%%%%%%%%%%%%%%%%%%%%%%%%%%%%%%%
\paragraph{v1.6:} 2018/01/17

\begin{itemize}
\item
application for development of include files
\item
corrections to manual
\end{itemize}

%%%%%%%%%%%%%%%%%%%%%%%%%%%%%%%%%%%%%%%%
\paragraph{v1.5:} 2017/05/21

\begin{itemize}
\item
more complete structuring introduced
\item
|\childdocof| introduced
\item
|\childdoc| renamed to |\childdocmain|
\item
|\childredirect| renamed to |\childdocforward| and |\childdocforwardprefix|
and functionality expanded
\end{itemize}

%%%%%%%%%%%%%%%%%%%%%%%%%%%%%%%%%%%%%%%%
\paragraph{v1.0:} 2017/04/27

\begin{itemize}
\item
manual and install package
\item
first version published on CTAN
\end{itemize}

%%%%%%%%%%%%%%%%%%%%%%%%%%%%%%%%%%%%%%%%
\paragraph{v0.6:} 2017/04/26

\begin{itemize}
\item
redirection mechanism added
\end{itemize}

%%%%%%%%%%%%%%%%%%%%%%%%%%%%%%%%%%%%%%%%
\paragraph{v0.5:} 2017/04/26

\begin{itemize}
\item
functionality in definition file
\end{itemize}


%%%%%%%%%%%%%%%%%%%%%%%%%%%%%%%%%%%%%%%%%%%%%%%%%%%%%%%%%%%%%%%%%%%%%%%%%%%%%%%%
%%%%%%%%%%%%%%%%%%%%%%%%%%%%%%%%%%%%%%%%%%%%%%%%%%%%%%%%%%%%%%%%%%%%%%%%%%%%%%%%
%%%%%%%%%%%%%%%%%%%%%%%%%%%%%%%%%%%%%%%%%%%%%%%%%%%%%%%%%%%%%%%%%%%%%%%%%%%%%%%%
\appendix

\settowidth\MacroIndent{\rmfamily\scriptsize 000\ }

 \DocInput{childdoc.dtx}

\end{document}
%</driver>
% \fi
%
% %%%%%%%%%%%%%%%%%%%%%%%%%%%%%%%%%%%%%%%%%%%%%%%%%%%%%%%%%%%%%%%%%%%%%%%%%%%%%%
% %%%%%%%%%%%%%%%%%%%%%%%%%%%%%%%%%%%%%%%%%%%%%%%%%%%%%%%%%%%%%%%%%%%%%%%%%%%%%%
% \section{Sample}
%\iffalse
%<*samplemain>
%\fi
%
% The following presents a sample document
% with two chapters, two parts, a title page,
% a compile flag as well as three forwarding files to set the flag.
% It consists of eight |.tex| files:
% \begin{center}
% \begin{tabular}{ll}
% |cdocsamp.tex|&main file\\
% |cdocsch1.tex|&include file for chapter 1\\
% |cdocsch2.tex|&include file for chapter 2\\
% |cdocspt3.tex|&include file for part 3\\
% |cdocspt4.tex|&include file for part 4\\
% |cdocsdrf.tex|&forwarding file for main file in draft mode\\
% |cdocsfi1.tex|&forwarding file for final version of chapter 1\\
% |cdocsfi2.tex|&forwarding file for final version of chapter 2\\
% \end{tabular}
% \end{center}
% Each of the eight files can be compiled directly by the \LaTeX{} compiler.
%
% %%%%%%%%%%%%%%%%%%%%%%%%%%%%%%%%%%%%%%
% \paragraph{Main File.}
%
% The main file is called |cdocsamp.tex|.
%
% Load the \textsf{childdoc} definitions and
% declare the filename for the main document:
%    \begin{macrocode}
\input{childdoc.def}
\childdocmain{}
%    \end{macrocode}

% Optional override for |\version| flag:
%    \begin{macrocode}
%%\ifchilddoc\else\providecommand{\version}{draft}\fi
%    \end{macrocode}

% Define the default values for the |\version| flag
% (|final| for the main file and |draft| for childs):
%    \begin{macrocode}
\ifchilddoc
\providecommand{\version}{draft}
\else
\providecommand{\version}{final}
\fi
%    \end{macrocode}

% Load the standard document class:
%    \begin{macrocode}
\documentclass[12pt]{article}
%    \end{macrocode}

% Start the document body:
%    \begin{macrocode}
\begin{document}
%    \end{macrocode}

% Declare a title page.
% Print title, part of document being processed and version flag:
%    \begin{macrocode}
\addtocounter{page}{-1}
\begin{center}
{\LARGE\bfseries{}childdoc example\par}
\vspace{1cm}
\ifchilddoc
\ifchilddocmanual part\else chapter\fi:
`\childdocname' of `\childdocjob'\par
\else
main document: `\childdocjob'\par
\fi
version: \version\par
\end{center}
\newpage
%    \end{macrocode}

% Manually include selected file,
% otherwise process as usual:
%    \begin{macrocode}
\ifchilddocmanual
\section*{part `\childdocname'}
\input{\childdocname}
\else
%    \end{macrocode}

% Include the two chapters:
%    \begin{macrocode}
\include{cdocsch1}
\include{cdocsch2}
%    \end{macrocode}

% Include the two parts unless only chapters should be displayed:
%    \begin{macrocode}
\ifchilddoc\else
\section{part three}
\input{cdocspt3}
\section{part four}
\input{cdocspt4}
\fi
%    \end{macrocode}

% Process as usual until here:
%    \begin{macrocode}
\fi
%    \end{macrocode}

% End of document body:
%    \begin{macrocode}
\end{document}
%    \end{macrocode}
%\iffalse
%</samplemain>
%\fi
%
% %%%%%%%%%%%%%%%%%%%%%%%%%%%%%%%%%%%%%%
% \paragraph{Chapter Include Files.}
%
% The include files are called |cdocsch1.tex| and |cdocsch2.tex|.
%
%\iffalse
%<*samplechap1|samplechap2>
%\fi

% Optional override for |\version| flag:
%    \begin{macrocode}
%%\providecommand{\version}{final}
%    \end{macrocode}

% Include the main document:
%    \begin{macrocode}
\input{childdoc.def}
\childdocof{cdocsamp}
%    \end{macrocode}

%\iffalse
%</samplechap1|samplechap2>
%\fi
%
%\iffalse
%<*samplechap1>
%\fi
% Some text for chapter 1:
%    \begin{macrocode}
\section{one}
some text in chapter one
%    \end{macrocode}

%\iffalse
%</samplechap1>
%\fi
% Some text for chapter 2:
%\iffalse
%<*samplechap2>
%\fi
%    \begin{macrocode}
\section{two}
more text in chapter two
%    \end{macrocode}

%\iffalse
%</samplechap2>
%\fi
%
% %%%%%%%%%%%%%%%%%%%%%%%%%%%%%%%%%%%%%%
% \paragraph{Part Include Files.}
%
% The include files are called |cdocspt3.tex| and |cdocspt4.tex|.
%
%\iffalse
%<*samplepart3|samplepart4>
%\fi

% Optional override for |\version| flag:
%    \begin{macrocode}
%%\providecommand{\version}{final}
%    \end{macrocode}

% Include the main document:
%    \begin{macrocode}
\input{childdoc.def}
\childdocby{cdocsamp}
%    \end{macrocode}

%\iffalse
%</samplepart3|samplepart4>
%\fi
%
%\iffalse
%<*samplepart3>
%\fi
% Some text for part 3:
%    \begin{macrocode}
some text in part three
%    \end{macrocode}

%\iffalse
%</samplepart3>
%\fi
% Some text for part 4:
%\iffalse
%<*samplepart4>
%\fi
%    \begin{macrocode}
more text in part four
%    \end{macrocode}

%\iffalse
%</samplepart4>
%\fi
%
% %%%%%%%%%%%%%%%%%%%%%%%%%%%%%%%%%%%%%%
% \paragraph{Forwarding for a Complete Draft.}
%
% The following forwarding file |cdocsdrf.tex|
% compiles the main document in draft mode:
%\iffalse
%<*sampledraft>
%\fi
%    \begin{macrocode}
\def\version{draft}
\input{childdoc.def}
\childdocforward{cdocsamp}
%    \end{macrocode}

%\iffalse
%</sampledraft>
%\fi
%
% %%%%%%%%%%%%%%%%%%%%%%%%%%%%%%%%%%%%%%
% \paragraph{Forwarding for Final Version of the Chapters.}
%
% The following forwarding files |cdocsfn1.tex| and |cdocsfn2.tex|
% (with identical content)
% compile the final versions of the child documents
% |cdocsch1.tex| and |cdocsch2.tex|, respectively:
%\iffalse
%<*samplefinal>
%\fi
%    \begin{macrocode}
\def\version{final}
\input{childdoc.def}
\childdocforwardprefix[cdocsamp]{cdocsfn}{cdocsch}
%    \end{macrocode}

%\iffalse
%</samplefinal>
%\fi
%
% %%%%%%%%%%%%%%%%%%%%%%%%%%%%%%%%%%%%%%
% \paragraph{Command Line Processing.}
%
% The following three command lines generate the output files
% |cdocscld|, |cdocscl1| and |cdocscl2|
% which should be identical to
% |cdocsdrf|, |cdocsch1| and |cdocsfn2|, respectively:
% \begin{center}
% \begin{tabular}{l}
% |latex -jobname cdocscld \|\\
% |  "\def\version{draft}\input{childdoc.def}\childdocforward{cdocsamp}"|\\
% |latex -jobname cdocscl1 \|\\
% |  "\input{childdoc.def}\childdocforward[cdocsamp]{cdocsch1}"|\\
% |latex -jobname cdocscl2 \|\\
% |  "\def\version{final}\input{childdoc.def}\childdocforward{cdocsch2}"|
% \end{tabular}
% \end{center}
% Note that the trailing backslash on each first line
% merely continues the input to the second line
% (for convenient cut ant paste).
% Furthermore, the command |latex| can be replaced by any
% of its alternative versions such as |pdflatex|.
%
% %%%%%%%%%%%%%%%%%%%%%%%%%%%%%%%%%%%%%%%%%%%%%%%%%%%%%%%%%%%%%%%%%%%%%%%%%%%%%%
% %%%%%%%%%%%%%%%%%%%%%%%%%%%%%%%%%%%%%%%%%%%%%%%%%%%%%%%%%%%%%%%%%%%%%%%%%%%%%%
% \section{Implementation}
%\iffalse
%<*package>
%\fi
%
% This section describes the definitions file |childdoc.def|.

% The definitions cannot be loaded using |\usepackage| or |\RequirePackage|
% which has a mechanism to prevent loading a style file more than once.
% When loading the definitions by means of |\input|
% multiple instances have to be prevented manually:
%\iffalse
%This code needs to be before the `\ProvidesFile' directive
%which is defined at the beginning of this file.
%Therefore it is also placed there and commented out here.
%</package>
%<*discard>
%\fi
%    \begin{macrocode}
\ifdefined\childdocmain\endinput\fi
%    \end{macrocode}
%\iffalse
%</discard>
%<*package>
%\fi
%
% \macro{\ifchilddoc}
% \macro{\ifchilddocmanual}
% The conditional |\ifchilddoc| tells whether a
% child (true) or main (false) document is being compiled.
% The conditional |\ifchilddocmanual| tells whether
% the |\includeonly| mechanism is used (false) or
% the selection of child files must be performed manually (true).
% The definitions initialise to false:
%    \begin{macrocode}
\newif\ifchilddoc
\newif\ifchilddocmanual
%    \end{macrocode}

% \macro{\childdocname}
% \macro{\childdocjob}
% The macro |\childdocname| stores the name of the main document
% to be compiled. The macro |\childdocjob| stores the name of
% the document on which the \LaTeX{} compiler was originally invoked.
% The content of |\jobname| cannot be compared
% to filenames specified in the source due to different catcodes.
% The following code rescans |\jobname|, stores the result
% in |\childdocname| and saves a copy in |\childdocjob|:
%    \begin{macrocode}
\edef\childdocname{\scantokens\expandafter{\jobname\noexpand}}
\let\childdocjob\childdocname
%    \end{macrocode}

% \macro{\childdocdisable}
% The macro |\childdocdisable| prevents the main file
% from being processed more than once.
% At this stage, the main document command |\childdocmain|
% is assumed to be called once again where it should do nothing.
% Any subsequent call to it should prevent
% a secondary processing of the main document
% It overwrites the forwarding commands
% |\childdocof| and |\childdocforward|
% with empty macros to prevent further inclusions of the main document:
%    \begin{macrocode}
\newcommand{\childdocdisable}
{
  \renewcommand{\childdocmain}[1]{\renewcommand{\childdocmain}[1]{\endinput}}
  \renewcommand{\childdocof}[1]{}
  \renewcommand{\childdocby}[2][]{}
  \renewcommand{\childdocforward}[2][]{}
  \renewcommand{\childdocdisable}{}
}
%    \end{macrocode}

% \macro{\childdocmain}
% The macro |\childdocmain| is to be called at the top of the main file
% with nothing or the main filename (without extension) as argument.
% First, it breaks loops.
% If the argument is not empty and does not match |\childdocname|
% (which is set by the first inclusion of |childdoc.def|),
% |\ifchilddoc| is set to true, |\includeonly| is applied to the child file
% and |\jobname| is set to the main file
% (for proper handling of |.aux| files):
%    \begin{macrocode}
\newcommand{\childdocmain}[1]
{
  \childdocdisable\childdocmain{}
  \if?#1?\else
    \begingroup
      \def\childdoctmp{#1}
      \ifx\childdoctmp\childdocname
        \def\childdoctmp{}
      \else
        \def\childdoctmp
        {
          \childdoctrue
          \includeonly{\childdocname}
          \def\childdocjob{#1}
          \def\jobname{#1}
        }
      \fi
      \expandafter
    \endgroup
    \childdoctmp
  \fi
}
%    \end{macrocode}

% \macro{\childdocof}
% The command |\childdocof| redirects
% compilation to the main file |#1|.
%    \begin{macrocode}
\newcommand{\childdocof}[1]
{
  \childdocdisable
  \childdoctrue
  \includeonly{\childdocname}
  \def\jobname{#1}
  \def\childdocjob{#1}
  \input{#1}
}
%    \end{macrocode}

% \macro{\childdocby}
% The command |\childdocby| ....
%    \begin{macrocode}
\newcommand{\childdocby}[2][]
{
  \childdocdisable
  \childdoctrue
  \childdocmanualtrue
  \if?#1?\else
    \def\jobname{#2}
  \fi
  \def\childdocjob{#2}
  \input{#2}
  \endinput
}
%    \end{macrocode}

% \macro{\childdocforward}
% The command |\childdocforward| redirects
% compilation to the main file or
% (if the optional argument is given) a child file.
% Parameters are set as if the main file
% or a child file starting with |\childdocof| was compiled.
% Then compilation is handed over to the main file:
%    \begin{macrocode}
\newcommand{\childdocforward}[2][]
{
  \begingroup
    \if?#1?
      \def\childdoctmp
      {
        \def\childdocname{#2}
        \def\childdocjob{#2}
        \def\jobname{#2}
        \input{#2}
        \endinput
      }
    \else
      \def\childdoctmp
      {
        \childdocdisable
        \def\childdocname{#2}
        \childdoctrue
        \includeonly{#2}
        \def\childdocjob{#1}
        \def\jobname{#1}
        \input{#1}
        \endinput
      }
    \fi
    \expandafter
  \endgroup
  \childdoctmp
}
%    \end{macrocode}

% \macro{\childdocforwardprefix}
% The command |\childdocforwardprefix| redirects
% compilation to the main or a child file by means of a pattern.
% The prefix |#1| in the current filename is replaced by |#2|
% and the suffix of the current filename is kept
% (it is assumed that the filename does not contain the substring `|~~~|'
% which is used as a delimiter).
% Compilation is handed over to the new file by |\childdocforward|:
%    \begin{macrocode}
\newcommand{\childdocforwardprefix}[3][]
{
  \begingroup
    \def\childdocextract #2##1~~~{\def\childdoctmp{\childdocforward[#1]{#3##1}}}
    \expandafter\childdocextract\childdocname~~~
    \expandafter
  \endgroup
  \childdoctmp
}
%    \end{macrocode}

% \macro{\childdoc}
% The deprecated macro |\childdoc| is a legacy version of |\childdocmain|:
%    \begin{macrocode}
\newcommand{\childdoc}{\childdocmain}
%    \end{macrocode}

% \macro{\childdocredirect}
% The deprecated macro |\childdocredirect| is a legacy version
% of |\childdocforward| and |\childdocforwardprefix|:
%    \begin{macrocode}
\newcommand{\childdocredirect}[2][]
{
  \begingroup
    \if?#1?
      \def\childdoctmp{\childdocforward{#2}}
    \else
      \def\childdoctmp{\childdocforwardprefix{#1}{#2}}
    \fi
    \expandafter
  \endgroup
  \childdoctmp
}
%    \end{macrocode}

%\iffalse
%</package>
%\fi
%
\endinput
|\\
|\childdocmain{|\textit{main}|}|\\
\end{tabular}
\end{center}
%
If |\jobname| does not match the argument \textit{main} of |\childdocmain|,
it is assumed that |\jobname| points to the child file to be compiled.
When using |\childdocmain| with the main file specified as argument,
it suffices to start a child file
with just |\input{|\textit{main}|}|
without loading of the package and using |\childdocof|.
If instead all processing is done
with the appropriate \textsf{childdoc} directives,
the argument of \textit{main} of |\childdocmain| can be empty.

An alternative version of the command line processing described
in \secref{sec:commandline} using the detection mechanism reads:
%
\begin{center}
|... -jobname "|\textit{target}|" "|[\textit{flags}]%
[|\def\jobname{|\textit{dest}|}|]|\input{|\textit{main}|}"|
\end{center}

%%%%%%%%%%%%%%%%%%%%%%%%%%%%%%%%%%%%%%%%%%%%%%%%%%%%%%%%%%%%%%%%%%%%%%%%%%%%%%%%
\subsection{Manual Code}
\label{sec:manual}

In case one cannot be certain whether the definitions file |childdoc.def|
is installed on the target \TeX{} distribution
and one prefers not to ship it,
it is conceivable to paste a few relevant commands into the sources.

To that end, drop all statements |% \iffalse
%
% childdoc.dtx Copyright (C) 2017-2018 Niklas Beisert
%
% This work may be distributed and/or modified under the
% conditions of the LaTeX Project Public License, either version 1.3
% of this license or (at your option) any later version.
% The latest version of this license is in
%   http://www.latex-project.org/lppl.txt
% and version 1.3 or later is part of all distributions of LaTeX
% version 2005/12/01 or later.
%
% This work has the LPPL maintenance status `maintained'.
%
% The Current Maintainer of this work is Niklas Beisert.
%
% This work consists of the files childdoc.dtx and childdoc.ins
% and the derived files childdoc.def and cdocsamp.tex with
% cdocsch1.tex, cdocsch2.tex, cdocsdrf.tex, cdocsfn1.tex, cdocsfn2.tex.
%
%<package>\ifdefined\childdocmain\endinput\fi
%<package>\ProvidesFile{childdoc.def}[2018/12/30 v2.0 child document driver]
%<samplemain>\ProvidesFile{cdocsamp.tex}[2018/12/30 v2.0 sample for childdoc]
%<*driver>
%\ProvidesFile{childdoc.drv}[2018/12/30 v2.0 childdoc reference manual file]
\PassOptionsToClass{10pt,a4paper}{article}
\documentclass{ltxdoc}

\usepackage[margin=35mm]{geometry}
\usepackage{hyperref}
\usepackage{hyperxmp}
\usepackage[usenames]{color}

\hypersetup{colorlinks=true}
\hypersetup{pdfstartview=FitH}
\hypersetup{pdfpagemode=UseNone}
\hypersetup{pdfsource={}}
\hypersetup{pdflang={en-UK}}
\hypersetup{pdfcopyright={Copyright 2017-2018 Niklas Beisert.
  This work may be distributed and/or modified under the
  conditions of the LaTeX Project Public License, either version 1.3
  of this license or (at your option) any later version.}}
\hypersetup{pdflicenseurl={http://www.latex-project.org/lppl.txt}}
\hypersetup{pdfcontactaddress={ETH Zurich, ITP, HIT K,
  Wolfgang-Pauli-Strasse 27}}
\hypersetup{pdfcontactpostcode={8093}}
\hypersetup{pdfcontactcity={Zurich}}
\hypersetup{pdfcontactcountry={Switzerland}}
\hypersetup{pdfcontactemail={nbeisert@itp.phys.ethz.ch}}
\hypersetup{pdfcontacturl={http://people.phys.ethz.ch/\xmptilde nbeisert/}}

\newcommand{\secref}[1]{\hyperref[#1]{section \ref*{#1}}}

\parskip1ex
\parindent0pt
\let\olditemize\itemize
\def\itemize{\olditemize\parskip0pt}

\begin{document}

\title{The \textsf{childdoc} Package}
\hypersetup{pdftitle={The childdoc Package}}
\author{Niklas Beisert\\[2ex]
  Institut f\"ur Theoretische Physik\\
  Eidgen\"ossische Technische Hochschule Z\"urich\\
  Wolfgang-Pauli-Strasse 27, 8093 Z\"urich, Switzerland\\[1ex]
  \href{mailto:nbeisert@itp.phys.ethz.ch}
  {\texttt{nbeisert@itp.phys.ethz.ch}}}
\hypersetup{pdfauthor={Niklas Beisert}}
\hypersetup{pdfsubject={Manual for the LaTeX2e Package childdoc}}
\date{30 December 2018, \textsf{v2.0}}
\maketitle

\begin{abstract}\noindent
\textsf{childdoc} is a \LaTeXe{} package
that enables the direct compilation
of document sections included by |\include|
to individual files.
\end{abstract}

\begingroup
\parskip0ex
\tableofcontents
\endgroup

%%%%%%%%%%%%%%%%%%%%%%%%%%%%%%%%%%%%%%%%%%%%%%%%%%%%%%%%%%%%%%%%%%%%%%%%%%%%%%%%
%%%%%%%%%%%%%%%%%%%%%%%%%%%%%%%%%%%%%%%%%%%%%%%%%%%%%%%%%%%%%%%%%%%%%%%%%%%%%%%%
\section{Introduction}

\LaTeX{} provides a mechanism to structure a large document (such as a book)
into a main file and several child files (containing the chapters)
using the |\include| command.
This mechanism is beneficial for documents
which span hundreds of pages in order to
make the source file(s) more manageable.
Moreover, compilation can be restricted to
selected child files by means of the |\includeonly| command.
The latter feature can be used to reduce the compilation time while editing
(this was significantly more useful in the earlier days of \LaTeX{})
or to generate a smaller document which is easier to navigate.
Another application of |\includeonly| is to generate
documents consisting of selected parts of the complete document.

However, there are a few drawbacks of the plain |\include| mechanism:
\begin{itemize}
\item
The child files cannot be compiled on their own,
they can only be compiled via the main file.
A naive editing environment
(such as a text editor with an option
to have the current file processed by \LaTeX)
may require one to switch to the main file before compiling;
attempting to compile the child file produces errors.
\item
The main file must be modified (each time)
to adjust the |\includeonly| command
to the present needs. This easily leaves the main file in a messy state.
\item
The generated document will always carry the filename
of the main document. This is inconvenient if
several child files are to be compiled and
to be kept for distribution.
\end{itemize}

The present package provides a simple interface
to make child files individually compilable by \LaTeX{}.
Compiling a child file then has the same effect as compiling
the main file with an |\includeonly| command
to select the appropriate child.
Moreover the generated document will carry the name of the child
rather than the main file.
This resolves all three above issues.

This feature is meant to make the editing of books,
thesis documents and lecture notes somewhat more convenient.
However, the package can also be used efficiently for
composing a series of documents (such as exercise sheets)
which are typically distributed individually.
It then assists the author in generating the individual documents
(potentially in different versions)
as well as a document containing the collected series.
Another application is in developing style files
or other kinds of included material
where compilation of the style file could redirect
to a sample or test file.

%%%%%%%%%%%%%%%%%%%%%%%%%%%%%%%%%%%%%%%%%%%%%%%%%%%%%%%%%%%%%%%%%%%%%%%%%%%%%%%%
%%%%%%%%%%%%%%%%%%%%%%%%%%%%%%%%%%%%%%%%%%%%%%%%%%%%%%%%%%%%%%%%%%%%%%%%%%%%%%%%
\section{Usage}

First of all, the package \textsf{childdoc} is \emph{not} a standard
\LaTeXe{} |.sty| style file! Therefore it needs to be invoked in
a non-standard way.

%%%%%%%%%%%%%%%%%%%%%%%%%%%%%%%%%%%%%%%%%%%%%%%%%%%%%%%%%%%%%%%%%%%%%%%%%%%%%%%%
\subsection{Included Files}
\label{sec:include}

%%%%%%%%%%%%%%%%%%%%%%%%%%%%%%%%%%%%%%%%
\DescribeMacro{\childdocmain}
To use the package, add the commands
\begin{center}
\begin{tabular}{l}
|\input{childdoc.def}|\\
|\childdocmain{}|\\
\end{tabular}
\end{center}
at the very top of the main \LaTeX{} file,
in particular \emph{before} the |\documentclass| statement!
The argument of |\childdocmain| should be left empty
(but it must be present).

%%%%%%%%%%%%%%%%%%%%%%%%%%%%%%%%%%%%%%%%
\DescribeMacro{\childdocof}
Furthermore, add the commands
\begin{center}
\begin{tabular}{l}
|\input{childdoc.def}|\\
|\childdocof{|\textit{main}|}|\\
\end{tabular}
\end{center}
at the top of every child file \textit{child}
which is included by |\include{|\textit{child}|}|
from within the main file
(or at least for those files to be compiled individually).
The argument \textit{main} must be the filename of the main file.

There are a couple of
considerations in setting up the main and child documents:

%%%%%%%%%%%%%%%%%%%%%%%%%%%%%%%%%%%%%%%%
\paragraph{Restrictions.}

Please note the following restrictions:
\begin{itemize}
\item
|\childdocmain| must be called with one argument \textit{main}
to ensure compatibility with earlier version of the package.
It must either be empty (|\childdocmain{}|)
or precisely match the filename of the main file in which it is specified.
See \secref{sec:detection} for further information.
\item
The filename \textit{main} must be specified without the |.tex| extension.
\item
The filename \textit{main} is case sensitive
(even in case-insensitive file systems)
due to internal string comparison.
\item
The argument \textit{main} should be fully expanded, it cannot be a macro.
\item
Subdirectories and special characters should be avoided in filenames.
\item
The command |\childdocmain{|\textit{main}|}| must be followed by a whitespace.
It should not be followed immediately by another command
or by a comment mark `|%|'.
This is because the \TeX{} parser reads the token immediately following
the argument of |\childdocmain| and puts it
at the beginning of every child section;
however, a white\-space is ignored.
\end{itemize}

%%%%%%%%%%%%%%%%%%%%%%%%%%%%%%%%%%%%%%%%
\paragraph{Content of Main File.}

It is advisable to place all content in the child files included by |\include|.
Any output contained in the main file will appear in all child documents
unless suppressed manually;
it cannot be suppressed automatically by the |\includeonly| directive
and thus should normally be avoided.
A method to include some content in the main file
by means of conditional processing is described in \secref{sec:conditional}.

%%%%%%%%%%%%%%%%%%%%%%%%%%%%%%%%%%%%%%%%
\paragraph{Page Numbering.}

When only a part of the document is compiled,
the appropriate numbering of pages
(as well as other status parameters)
is determined from the |.aux| files.
The latter contain information from previous passes.
However this information needs to propagate through
all intermediate child documents.
Therefore the page numbering in child documents may well
be inconsistent until the complete document is compiled at least once.

A useful (if unconventional) way to always ensure a consistent
page numbering is to restart the numbering in each child document
and denote the pages by `\textit{child}|.|\textit{page}'
where \textit{child} represents the chapter/section number of the child file.
This can be achieved by the command
|\numberwithin{page}{|\textit{child}|}|
of the \textsf{amsmath} package
where \textit{child} can be |chapter| or |section|
depending on the chosen structuring.
Alternatively, one can modify the macro |\thepage| appropriately
and reset the counter |page| at the start of each child file.

%%%%%%%%%%%%%%%%%%%%%%%%%%%%%%%%%%%%%%%%%%%%%%%%%%%%%%%%%%%%%%%%%%%%%%%%%%%%%%%%
\subsection{Conditional Processing}
\label{sec:conditional}

The package provides a mechanism to compile different versions
of a document. To customise the versions further some conditional processing
can come in handy to distinguish which version is being compiled.
The package provides two macros to describe the compilation context:

%%%%%%%%%%%%%%%%%%%%%%%%%%%%%%%%%%%%%%%%
\DescribeMacro{\ifchilddoc}
The conditional |\ifchilddoc| distinguishes between the compilation of
child documents and the main document:
%
\begin{center}
|\ifchilddoc |\textit{child-code}| |[|\||else |\textit{main-code}]| \||fi|
\end{center}

%%%%%%%%%%%%%%%%%%%%%%%%%%%%%%%%%%%%%%%%
\DescribeMacro{\childdocname}
\DescribeMacro{\childdocjob}
The macro |\childdocname| contains the filename (without extension)
of the main or child file being processed.
Note that |\childdocjob| will always contain the name of the main file.

%%%%%%%%%%%%%%%%%%%%%%%%%%%%%%%%%%%%%%%%
\paragraph{Title Page.}

Conditional processing can be used to include a title or banner page
in the main document when proper precautions are taken.
Importantly, the code in the main file should ensure that the page counter
(as well as other status parameters which are stored in the |.aux| files)
takes the same value after the conditional processing.
Otherwise the page numbers may take divergent values
depending on which part is compiled.

For example, a title page could be declared by:
%
\begin{center}
\begin{tabular}{l}
|\ifchilddoc\||else|\\
|\addtocounter{page}{-1}|\\
\textit{code for title page}\\
|\newpage|\\
|\||fi|
\end{tabular}
\end{center}
%
A banner page for the child documents can be generated by:
%
\begin{center}
\begin{tabular}{l}
|\ifchilddoc|\\
|\addtocounter{page}{-1}|\\
\textit{code for banner page}\\
|\newpage|\\
|\||fi|
\end{tabular}
\end{center}
%
Here one could write a message such as:
\begin{center}
|This is the part \childdocname{} of \childdocjob{}.|
\end{center}

%%%%%%%%%%%%%%%%%%%%%%%%%%%%%%%%%%%%%%%%%%%%%%%%%%%%%%%%%%%%%%%%%%%%%%%%%%%%%%%%
\subsection{Flags}
\label{sec:flags}

The package makes it easy to generate different versions
of the main or child documents.
To this end compilation flags can be defined
and assigned different default values.
They will be particularly useful in conjunction
with the forwarding mechanism described in \secref{sec:forward}.

For example, it may be useful to have a flag |\version|
which can be set to |draft| or |final|.
The document source will contain some conditional code
depending on the value of |\version|.
Suppose further, the flag should default to |final| for the main file
and to |draft| for child files
which is a natural assignment for editing the document.
This is achieved by placing the following code
in the preamble of the main document
(below the |\childdocmain| directive):
%
\begin{center}
\begin{tabular}{l}
|\ifchilddoc|\\
|\providecommand{\version}{draft}|\\
|\||else|\\
|\providecommand{\version}{final}|\\
|\||fi|
\end{tabular}
\end{center}
%
The definition by |\providecommand| makes sure
that previous definitions are not overwritten.
Further statements |\providecommand{\version}{...}|
can thus be added before the above code to override it.

For the main file, one might add a line
(between |\childdocmain| and the above block)
%
\begin{center}
|%\ifchilddoc\||else\providecommand{\version}{draft}\||fi|
\end{center}
%
which can be uncommented to produce a draft version.
Likewise one can add a line to the very top of a child file
(above the |\childdocof{|\textit{main}|}| directive)
%
\begin{center}
|%\providecommand{\version}{final}|
\end{center}
%
which can be uncommented to produce the final version of this child document.

%%%%%%%%%%%%%%%%%%%%%%%%%%%%%%%%%%%%%%%%%%%%%%%%%%%%%%%%%%%%%%%%%%%%%%%%%%%%%%%%
\subsection{Forwarding}
\label{sec:forward}

Different versions of the main or child documents
using compilation flags as described in \secref{sec:flags}
can be (permanently) stored in different files
for convenient compilation, viewing and distribution.
To this end, the package defines a command
to pass on compilation to a different file:

%%%%%%%%%%%%%%%%%%%%%%%%%%%%%%%%%%%%%%%%
\DescribeMacro{\childdocforward}
The command |\childdocforward| redirects processing to
another source file:
%
\begin{center}
\begin{tabular}{l}
|\input{childdoc.def}|\\
|\childdocforward[|\textit{main}|]{|\textit{dest}|}|\\
\end{tabular}
\end{center}
%
The argument \textit{dest} is the destination file
(without extension).
It should be the main file or one of the child files.
Note that further \textsf{childdoc} directives
such as |\childdocof| and |\childdocforward|
in the indicated file will be processed in this form.
The optional argument \textit{main}
passes on directly to the main file \textit{main}
while pretending to compile the child \textit{dest}.
This form behaves as if \textit{dest}
issues |\childdocof{|\textit{main}|}| right away,
and no further \textsf{childdoc} directives will be processed.

%%%%%%%%%%%%%%%%%%%%%%%%%%%%%%%%%%%%%%%%
\DescribeMacro{\...prefix}
In the alternative form |\childdocforwardprefix|,
%
\begin{center}
\begin{tabular}{l}
|\input{childdoc.def}|\\
|\childdocforwardprefix[|\textit{main}|]{|\textit{prefix}|}{|\textit{dest}|}|
\end{tabular}
\end{center}
%
the destination file is determined by a pattern
depending on the current file:
To make this work, the current file must be called
`{\textit{prefix}\hspace{0.2em}\textit{suffix}}'
with \textit{prefix} matching precisely the argument.
Processing is then passed on to the file
`{\textit{dest}\hspace{0.2em}\textit{suffix}}'.
Surely, the same effect is achieved by
directly specifying the
argument `{\textit{dest}\hspace{0.2em}\textit{suffix}}'
in the first form.
However, that requires to set up a different file
for each child. With the alternative form of the command
all these files can have exactly the same content
which simplifies setting them up and maintaining them.

For example, the following file |draft.tex|
with a compilation flag |\version| as described in \secref{sec:flags}
compiles the main document as a draft:
%
\begin{center}
\begin{tabular}{l}
|\def\version{draft}|\\
|\input{childdoc.def}|\\
|\childdocforward{|\textit{main}|}|
\end{tabular}
\end{center}
%
Likewise, the following files |final|\textit{nn}|.tex|
compile the final version of the child document
|child|\textit{nn}|.tex|:
%
\begin{center}
\begin{tabular}{l}
|\def\version{final}|\\
|\input{childdoc.def}|\\
|\childdocforwardprefix{final}{child}|
\end{tabular}
\end{center}
%

Note that when several versions of a main file and/or of each child file
are to be generated, it may be convenient to set up a |Makefile| or
shell script to automatise the process.

%%%%%%%%%%%%%%%%%%%%%%%%%%%%%%%%%%%%%%%%%%%%%%%%%%%%%%%%%%%%%%%%%%%%%%%%%%%%%%%%
\subsection{Command Line Processing}
\label{sec:commandline}

The effect of redirection files can also be achieved by invoking
the \LaTeX{} compiler with a more elaborate command line.
Most conveniently this should be done as part
of a shell script or a |Makefile|.

When using \textsf{childdoc} in the main file, the following
command lines effectively perform a redirection
(note that depending on the shell being used,
backslashes may have to be doubled: `|\|' $\to$ `|\\|'):
%
\begin{center}
|... -jobname "|\textit{target}|" |\\|"|[\textit{flags}]%
|\input{childdoc.def}\childdocforward[|\textit{main}|]{|\textit{dest}|}"|
\end{center}
%
Here \textit{target} is the name of the output file,
\textit{main} is the name of the main file
and \textit{dest} is the name of the main or child file to be processed
(all filenames without extensions).
The optional argument \textit{main} can be omitted
if \textit{main} matches \textit{dest}.
Optionally, compilation \textit{flags} can be defined via |\def| commands.
This command line makes the \TeX{} engine believe
it is compiling the file \textit{target}
whose content is specified as the latter parameter.
The provided code then forwards the processing to
\textit{main} or \textit{dest} as described in \secref{sec:forward}.

%%%%%%%%%%%%%%%%%%%%%%%%%%%%%%%%%%%%%%%%%%%%%%%%%%%%%%%%%%%%%%%%%%%%%%%%%%%%%%%%
\subsection{Include by Input}
\label{sec:input}

Including child documents by |\include| has some restrictions by design.
Most notably, the content of a child document always occupies
its own set of pages; pages cannot be shared between child documents.
Usually, this behaviour makes perfect sense
because each child document contain an essential part of the document.
However, in some situations it may be desirable to compose
a document from a collection of parts
without having mandatory page breaks between then.
For this case, the package
provides a mechanism to include parts
by |\input| which can also be processed individually.
However, by construction this mechanism
requires manual handling of the content to be output.

%%%%%%%%%%%%%%%%%%%%%%%%%%%%%%%%%%%%%%%%
\DescribeMacro{\ifchilddocmanual}
The main file should be prepared as usual, see \secref{sec:include}.
However, the document body must make a distinction
between processing of an individual part and of the main document, e.g.:
%
\begin{center}
\begin{tabular}{l}
|\ifchilddocmanual|\\
|\input{\childdocname}|\\
|\||else|\\
\textit{document body with }|\input{|\textit{part}|}|\\
|\||fi|
\end{tabular}
\end{center}
%
The conditional |\ifchilddocmanual| is true whenever
a part to be included by |\input| is being compiled,
and the name of the part is stored in |\childdocname|.

%%%%%%%%%%%%%%%%%%%%%%%%%%%%%%%%%%%%%%%%
\DescribeMacro{\childdocby}
Each part to be included by |\input| should start with:
%
\begin{center}
\begin{tabular}{l}
|\input{childdoc.def}|\\
|\childdocby{|\textit{main}|}|\\
\end{tabular}
\end{center}
%
The directive |\childdocby| is similar to |\childdocof|
described in \secref{sec:include},
but the subsequent selection of content must be done manually.
To that end, both |\ifchilddoc| and |\ifchilddocmanual|
will be true upon processing of a part,
and the name of the part is stored in |\childdocname|.
Note that |\jobname| will be set to the filename of the current part
so that each part receives an individual |.aux| file
that does not interfere with the |.aux| file(s) of the main document.
This behaviour can be altered by the alternative form
|\childdocby[*]{|\textit{main}|}| (with a non-empty optional argument)
which uses the |.aux| file of the main document
by setting |\jobname| to \textit{main}.

%%%%%%%%%%%%%%%%%%%%%%%%%%%%%%%%%%%%%%%%%%%%%%%%%%%%%%%%%%%%%%%%%%%%%%%%%%%%%%%%
\subsection{Driver Development}
\label{sec:driver}

The \textsf{childdoc} mechanism can also be use for the development
of definition files such as \LaTeX{} styles or classes.
This case differs from the above setup with multiple parts
included by |\include| in that no |\includeonly| should be invoked.
This can be achieved by starting the include file
(before |\ProvidesPackage|) with:
%
\begin{center}
\begin{tabular}{l}
|\input{childdoc.def}|\\
|\childdocforward{|\textit{main}|}|\\
\end{tabular}
\end{center}
%
or alternatively with:
%
\begin{center}
\begin{tabular}{l}
|\input{childdoc.def}|\\
|\childdocby{|\textit{main}|}|\\
\end{tabular}
\end{center}
%
Both forms have slightly different effects as described above.
The main file is prepared as usual, see \secref{sec:include}.

%%%%%%%%%%%%%%%%%%%%%%%%%%%%%%%%%%%%%%%%%%%%%%%%%%%%%%%%%%%%%%%%%%%%%%%%%%%%%%%%
\subsection{Legacy Detection}
\label{sec:detection}

The directive |\childdocmain| in the main file can detect
whether the complete document or merely a child is to be compiled
even without using the directive |\childdocof|.
This method is deprecated because it is less robust
and there is no compelling reason to use it;
it is merely provided for backward compatibility
and it may be removed in future versions.

If the detection mechanism is to be used,
it is mandatory to correctly specify
the filename of the main file as the argument of |\childdocmain|:
%
\begin{center}
\begin{tabular}{l}
|\input{childdoc.def}|\\
|\childdocmain{|\textit{main}|}|\\
\end{tabular}
\end{center}
%
If |\jobname| does not match the argument \textit{main} of |\childdocmain|,
it is assumed that |\jobname| points to the child file to be compiled.
When using |\childdocmain| with the main file specified as argument,
it suffices to start a child file
with just |\input{|\textit{main}|}|
without loading of the package and using |\childdocof|.
If instead all processing is done
with the appropriate \textsf{childdoc} directives,
the argument of \textit{main} of |\childdocmain| can be empty.

An alternative version of the command line processing described
in \secref{sec:commandline} using the detection mechanism reads:
%
\begin{center}
|... -jobname "|\textit{target}|" "|[\textit{flags}]%
[|\def\jobname{|\textit{dest}|}|]|\input{|\textit{main}|}"|
\end{center}

%%%%%%%%%%%%%%%%%%%%%%%%%%%%%%%%%%%%%%%%%%%%%%%%%%%%%%%%%%%%%%%%%%%%%%%%%%%%%%%%
\subsection{Manual Code}
\label{sec:manual}

In case one cannot be certain whether the definitions file |childdoc.def|
is installed on the target \TeX{} distribution
and one prefers not to ship it,
it is conceivable to paste a few relevant commands into the sources.

To that end, drop all statements |\input{childdoc.def}|
and perform the replacements as outlined below.
Instead of |\childdocmain{|\textit{main}|}| add the following code
to the top of the main file:
%
\begin{center}
\begin{tabular}{l}
|\||ifdefined\childdocname\endinput\||fi\newif\ifchilddoc|\\
|\edef\childdocname{\scantokens\expandafter{\jobname\noexpand}}|\\
|\def\childdocmain{|\textit{main}|}\||ifx\childdocmain\childdocname\||else|\\
|\childdoctrue\includeonly{\childdocname}\let\jobname\childdocmain\||fi|\\
\end{tabular}
\end{center}
%
Instead of |\childdocof{|\textit{main}|}| just include the main file
at the top of each child file:
%
\begin{center}
|\input{|\textit{main}|}|
\end{center}
%
A simple redirection |\childdocforward{|\textit{dest}|}| is achieved by:
%
\begin{center}
|\def\jobname{|\textit{dest}|}\input{\jobname}|
\end{center}
%
The redirection with prefix
|\childdocforwardprefix[|\textit{prefix}|]{|\textit{dest}|}|
is accomplished by:
%
\begin{center}
\begin{tabular}{l}
|{\edef\jobname{\scantokens\expandafter{\jobname\noexpand}}|\\
|\def\redirectjob |\textit{prefix}|#1~~~{\gdef\jobname{|\textit{dest}|#1}}|\\
|\expandafter\redirectjob\jobname~~~}\input{\jobname}|
\end{tabular}
\end{center}

In an alternative approach,
child documents can be compiled by a specific command line
without additional code or specific definitions:
%
\begin{center}
|... -jobname "|\textit{target}|" "|[\textit{flags}]%
|\includeonly{|\textit{dest}|}\input{|\textit{main}|}"|
\end{center}
%

%%%%%%%%%%%%%%%%%%%%%%%%%%%%%%%%%%%%%%%%%%%%%%%%%%%%%%%%%%%%%%%%%%%%%%%%%%%%%%%%
%%%%%%%%%%%%%%%%%%%%%%%%%%%%%%%%%%%%%%%%%%%%%%%%%%%%%%%%%%%%%%%%%%%%%%%%%%%%%%%%
\section{Information}

%%%%%%%%%%%%%%%%%%%%%%%%%%%%%%%%%%%%%%%%%%%%%%%%%%%%%%%%%%%%%%%%%%%%%%%%%%%%%%%%
\subsection{Copyright}

Copyright \copyright{} 2017--2018 Niklas Beisert

This work may be distributed and/or modified under the
conditions of the \LaTeX{} Project Public License, either version 1.3
of this license or (at your option) any later version.
The latest version of this license is in
  \url{http://www.latex-project.org/lppl.txt}
and version 1.3 or later is part of all distributions of \LaTeX{}
version 2005/12/01 or later.

This work has the LPPL maintenance status `maintained'.

The Current Maintainer of this work is Niklas Beisert.

This work consists of the files |README.txt|, |childdoc.ins| and |childdoc.dtx|
as well as the derived files |childdoc.def|, |cdocsamp.tex|
with |cdocsch1.tex|, |cdocsch2.tex|, |cdocspt3.tex|, |cdocspt4.tex|,
|cdocsdrf.tex|, |cdocsfn1.tex|, |cdocsfn2.tex|
as well as |childdoc.pdf|.

%%%%%%%%%%%%%%%%%%%%%%%%%%%%%%%%%%%%%%%%%%%%%%%%%%%%%%%%%%%%%%%%%%%%%%%%%%%%%%%%
\subsection{Files and Installation}

The package consists of the files:
%
\begin{center}
\begin{tabular}{ll}
    |README.txt|   & readme file \\
    |childdoc.ins| & installation file \\
    |childdoc.dtx| & source file \\
    |childdoc.def| & definition file \\
    |cdocsamp.tex| & sample main file \\
    |cdocsch1.tex| & sample include file \\
    |cdocsch2.tex| & sample include file \\
    |cdocspt3.tex| & sample part file \\
    |cdocspt4.tex| & sample part file \\
    |cdocsdrf.tex| & sample redirection file \\
    |cdocsfn1.tex| & sample redirection file \\
    |cdocsfn2.tex| & sample redirection file \\
    |childdoc.pdf| & manual
\end{tabular}
\end{center}
%
The distribution consists of the files
|README.txt|, |childdoc.ins| and |childdoc.dtx|.
%
\begin{itemize}
\item
Run (pdf)\LaTeX{} on |childdoc.dtx|
to compile the manual |childdoc.pdf| (this file).
\item
Run \LaTeX{} on |childdoc.ins| to create the definitions file |childdoc.def|
and the sample |cdocsamp.tex| with include files
|cdocsch1.tex|, |cdocsch2.tex|, |cdocspt3.tex|, |cdocspt4.tex|,
|cdocsdrf.tex|, |cdocsfn1.tex|, |cdocsfn2.tex|.
Then copy the file |childdoc.def| to an appropriate directory of your \LaTeX{}
distribution, e.g.\ \textit{texmf-root}|/tex/latex/childdoc|.
\end{itemize}

%%%%%%%%%%%%%%%%%%%%%%%%%%%%%%%%%%%%%%%%%%%%%%%%%%%%%%%%%%%%%%%%%%%%%%%%%%%%%%%%
\subsection{Related CTAN Packages}

There are several other packages which offer a similar functionality:
%
\begin{itemize}
\item
The packages
\href{http://ctan.org/pkg/docmute}{\textsf{docmute}},
\href{http://ctan.org/pkg/includex}{\textsf{includex}} and
\href{http://ctan.org/pkg/standalone}{\textsf{standalone}}
provide commands to include only the document body of
a child file thus allowing both files to be compiled individually.
\item
The packages \href{http://ctan.org/pkg/subdocs}{\textsf{subdocs}}
and \href{http://ctan.org/pkg/subfiles}{\textsf{subfiles}}
provide structures in which the main and child documents can be
encapsulated and allowing them to be compiled individually.
The inclusion mechanism is different from the conventional |\include|.
\item
The package \href{http://ctan.org/pkg/combine}{\textsf{combine}}
is an elaborate solution to combine several documents into one.
\end{itemize}
%
See also the CTAN topic \href{http://ctan.org/topic/subdocs}{\textsf{subdocs}}
for further related packages.
The present package differs from the above solutions in that
a document structure constructed with the conventional |\include| mechanism
just needs two extra commands at the top of every file
such that all constituent files can be compiled individually.

%%%%%%%%%%%%%%%%%%%%%%%%%%%%%%%%%%%%%%%%%%%%%%%%%%%%%%%%%%%%%%%%%%%%%%%%%%%%%%%%
%\subsection{Feature Suggestions}
%
%The following is a list of features which may be useful for future
%versions of this package:
%%
%\begin{itemize}
%\item
%\ldots
%\end{itemize}

%%%%%%%%%%%%%%%%%%%%%%%%%%%%%%%%%%%%%%%%%%%%%%%%%%%%%%%%%%%%%%%%%%%%%%%%%%%%%%%%
\subsection{Revision History}

%%%%%%%%%%%%%%%%%%%%%%%%%%%%%%%%%%%%%%%%
\paragraph{v2.0:} 2018/12/30

\begin{itemize}
\item
immediate forward processing
\item
added |\childdocby| mechanism
\item
manual restructured
\end{itemize}

%%%%%%%%%%%%%%%%%%%%%%%%%%%%%%%%%%%%%%%%
\paragraph{v1.6:} 2018/01/17

\begin{itemize}
\item
application for development of include files
\item
corrections to manual
\end{itemize}

%%%%%%%%%%%%%%%%%%%%%%%%%%%%%%%%%%%%%%%%
\paragraph{v1.5:} 2017/05/21

\begin{itemize}
\item
more complete structuring introduced
\item
|\childdocof| introduced
\item
|\childdoc| renamed to |\childdocmain|
\item
|\childredirect| renamed to |\childdocforward| and |\childdocforwardprefix|
and functionality expanded
\end{itemize}

%%%%%%%%%%%%%%%%%%%%%%%%%%%%%%%%%%%%%%%%
\paragraph{v1.0:} 2017/04/27

\begin{itemize}
\item
manual and install package
\item
first version published on CTAN
\end{itemize}

%%%%%%%%%%%%%%%%%%%%%%%%%%%%%%%%%%%%%%%%
\paragraph{v0.6:} 2017/04/26

\begin{itemize}
\item
redirection mechanism added
\end{itemize}

%%%%%%%%%%%%%%%%%%%%%%%%%%%%%%%%%%%%%%%%
\paragraph{v0.5:} 2017/04/26

\begin{itemize}
\item
functionality in definition file
\end{itemize}


%%%%%%%%%%%%%%%%%%%%%%%%%%%%%%%%%%%%%%%%%%%%%%%%%%%%%%%%%%%%%%%%%%%%%%%%%%%%%%%%
%%%%%%%%%%%%%%%%%%%%%%%%%%%%%%%%%%%%%%%%%%%%%%%%%%%%%%%%%%%%%%%%%%%%%%%%%%%%%%%%
%%%%%%%%%%%%%%%%%%%%%%%%%%%%%%%%%%%%%%%%%%%%%%%%%%%%%%%%%%%%%%%%%%%%%%%%%%%%%%%%
\appendix

\settowidth\MacroIndent{\rmfamily\scriptsize 000\ }

 \DocInput{childdoc.dtx}

\end{document}
%</driver>
% \fi
%
% %%%%%%%%%%%%%%%%%%%%%%%%%%%%%%%%%%%%%%%%%%%%%%%%%%%%%%%%%%%%%%%%%%%%%%%%%%%%%%
% %%%%%%%%%%%%%%%%%%%%%%%%%%%%%%%%%%%%%%%%%%%%%%%%%%%%%%%%%%%%%%%%%%%%%%%%%%%%%%
% \section{Sample}
%\iffalse
%<*samplemain>
%\fi
%
% The following presents a sample document
% with two chapters, two parts, a title page,
% a compile flag as well as three forwarding files to set the flag.
% It consists of eight |.tex| files:
% \begin{center}
% \begin{tabular}{ll}
% |cdocsamp.tex|&main file\\
% |cdocsch1.tex|&include file for chapter 1\\
% |cdocsch2.tex|&include file for chapter 2\\
% |cdocspt3.tex|&include file for part 3\\
% |cdocspt4.tex|&include file for part 4\\
% |cdocsdrf.tex|&forwarding file for main file in draft mode\\
% |cdocsfi1.tex|&forwarding file for final version of chapter 1\\
% |cdocsfi2.tex|&forwarding file for final version of chapter 2\\
% \end{tabular}
% \end{center}
% Each of the eight files can be compiled directly by the \LaTeX{} compiler.
%
% %%%%%%%%%%%%%%%%%%%%%%%%%%%%%%%%%%%%%%
% \paragraph{Main File.}
%
% The main file is called |cdocsamp.tex|.
%
% Load the \textsf{childdoc} definitions and
% declare the filename for the main document:
%    \begin{macrocode}
\input{childdoc.def}
\childdocmain{}
%    \end{macrocode}

% Optional override for |\version| flag:
%    \begin{macrocode}
%%\ifchilddoc\else\providecommand{\version}{draft}\fi
%    \end{macrocode}

% Define the default values for the |\version| flag
% (|final| for the main file and |draft| for childs):
%    \begin{macrocode}
\ifchilddoc
\providecommand{\version}{draft}
\else
\providecommand{\version}{final}
\fi
%    \end{macrocode}

% Load the standard document class:
%    \begin{macrocode}
\documentclass[12pt]{article}
%    \end{macrocode}

% Start the document body:
%    \begin{macrocode}
\begin{document}
%    \end{macrocode}

% Declare a title page.
% Print title, part of document being processed and version flag:
%    \begin{macrocode}
\addtocounter{page}{-1}
\begin{center}
{\LARGE\bfseries{}childdoc example\par}
\vspace{1cm}
\ifchilddoc
\ifchilddocmanual part\else chapter\fi:
`\childdocname' of `\childdocjob'\par
\else
main document: `\childdocjob'\par
\fi
version: \version\par
\end{center}
\newpage
%    \end{macrocode}

% Manually include selected file,
% otherwise process as usual:
%    \begin{macrocode}
\ifchilddocmanual
\section*{part `\childdocname'}
\input{\childdocname}
\else
%    \end{macrocode}

% Include the two chapters:
%    \begin{macrocode}
\include{cdocsch1}
\include{cdocsch2}
%    \end{macrocode}

% Include the two parts unless only chapters should be displayed:
%    \begin{macrocode}
\ifchilddoc\else
\section{part three}
\input{cdocspt3}
\section{part four}
\input{cdocspt4}
\fi
%    \end{macrocode}

% Process as usual until here:
%    \begin{macrocode}
\fi
%    \end{macrocode}

% End of document body:
%    \begin{macrocode}
\end{document}
%    \end{macrocode}
%\iffalse
%</samplemain>
%\fi
%
% %%%%%%%%%%%%%%%%%%%%%%%%%%%%%%%%%%%%%%
% \paragraph{Chapter Include Files.}
%
% The include files are called |cdocsch1.tex| and |cdocsch2.tex|.
%
%\iffalse
%<*samplechap1|samplechap2>
%\fi

% Optional override for |\version| flag:
%    \begin{macrocode}
%%\providecommand{\version}{final}
%    \end{macrocode}

% Include the main document:
%    \begin{macrocode}
\input{childdoc.def}
\childdocof{cdocsamp}
%    \end{macrocode}

%\iffalse
%</samplechap1|samplechap2>
%\fi
%
%\iffalse
%<*samplechap1>
%\fi
% Some text for chapter 1:
%    \begin{macrocode}
\section{one}
some text in chapter one
%    \end{macrocode}

%\iffalse
%</samplechap1>
%\fi
% Some text for chapter 2:
%\iffalse
%<*samplechap2>
%\fi
%    \begin{macrocode}
\section{two}
more text in chapter two
%    \end{macrocode}

%\iffalse
%</samplechap2>
%\fi
%
% %%%%%%%%%%%%%%%%%%%%%%%%%%%%%%%%%%%%%%
% \paragraph{Part Include Files.}
%
% The include files are called |cdocspt3.tex| and |cdocspt4.tex|.
%
%\iffalse
%<*samplepart3|samplepart4>
%\fi

% Optional override for |\version| flag:
%    \begin{macrocode}
%%\providecommand{\version}{final}
%    \end{macrocode}

% Include the main document:
%    \begin{macrocode}
\input{childdoc.def}
\childdocby{cdocsamp}
%    \end{macrocode}

%\iffalse
%</samplepart3|samplepart4>
%\fi
%
%\iffalse
%<*samplepart3>
%\fi
% Some text for part 3:
%    \begin{macrocode}
some text in part three
%    \end{macrocode}

%\iffalse
%</samplepart3>
%\fi
% Some text for part 4:
%\iffalse
%<*samplepart4>
%\fi
%    \begin{macrocode}
more text in part four
%    \end{macrocode}

%\iffalse
%</samplepart4>
%\fi
%
% %%%%%%%%%%%%%%%%%%%%%%%%%%%%%%%%%%%%%%
% \paragraph{Forwarding for a Complete Draft.}
%
% The following forwarding file |cdocsdrf.tex|
% compiles the main document in draft mode:
%\iffalse
%<*sampledraft>
%\fi
%    \begin{macrocode}
\def\version{draft}
\input{childdoc.def}
\childdocforward{cdocsamp}
%    \end{macrocode}

%\iffalse
%</sampledraft>
%\fi
%
% %%%%%%%%%%%%%%%%%%%%%%%%%%%%%%%%%%%%%%
% \paragraph{Forwarding for Final Version of the Chapters.}
%
% The following forwarding files |cdocsfn1.tex| and |cdocsfn2.tex|
% (with identical content)
% compile the final versions of the child documents
% |cdocsch1.tex| and |cdocsch2.tex|, respectively:
%\iffalse
%<*samplefinal>
%\fi
%    \begin{macrocode}
\def\version{final}
\input{childdoc.def}
\childdocforwardprefix[cdocsamp]{cdocsfn}{cdocsch}
%    \end{macrocode}

%\iffalse
%</samplefinal>
%\fi
%
% %%%%%%%%%%%%%%%%%%%%%%%%%%%%%%%%%%%%%%
% \paragraph{Command Line Processing.}
%
% The following three command lines generate the output files
% |cdocscld|, |cdocscl1| and |cdocscl2|
% which should be identical to
% |cdocsdrf|, |cdocsch1| and |cdocsfn2|, respectively:
% \begin{center}
% \begin{tabular}{l}
% |latex -jobname cdocscld \|\\
% |  "\def\version{draft}\input{childdoc.def}\childdocforward{cdocsamp}"|\\
% |latex -jobname cdocscl1 \|\\
% |  "\input{childdoc.def}\childdocforward[cdocsamp]{cdocsch1}"|\\
% |latex -jobname cdocscl2 \|\\
% |  "\def\version{final}\input{childdoc.def}\childdocforward{cdocsch2}"|
% \end{tabular}
% \end{center}
% Note that the trailing backslash on each first line
% merely continues the input to the second line
% (for convenient cut ant paste).
% Furthermore, the command |latex| can be replaced by any
% of its alternative versions such as |pdflatex|.
%
% %%%%%%%%%%%%%%%%%%%%%%%%%%%%%%%%%%%%%%%%%%%%%%%%%%%%%%%%%%%%%%%%%%%%%%%%%%%%%%
% %%%%%%%%%%%%%%%%%%%%%%%%%%%%%%%%%%%%%%%%%%%%%%%%%%%%%%%%%%%%%%%%%%%%%%%%%%%%%%
% \section{Implementation}
%\iffalse
%<*package>
%\fi
%
% This section describes the definitions file |childdoc.def|.

% The definitions cannot be loaded using |\usepackage| or |\RequirePackage|
% which has a mechanism to prevent loading a style file more than once.
% When loading the definitions by means of |\input|
% multiple instances have to be prevented manually:
%\iffalse
%This code needs to be before the `\ProvidesFile' directive
%which is defined at the beginning of this file.
%Therefore it is also placed there and commented out here.
%</package>
%<*discard>
%\fi
%    \begin{macrocode}
\ifdefined\childdocmain\endinput\fi
%    \end{macrocode}
%\iffalse
%</discard>
%<*package>
%\fi
%
% \macro{\ifchilddoc}
% \macro{\ifchilddocmanual}
% The conditional |\ifchilddoc| tells whether a
% child (true) or main (false) document is being compiled.
% The conditional |\ifchilddocmanual| tells whether
% the |\includeonly| mechanism is used (false) or
% the selection of child files must be performed manually (true).
% The definitions initialise to false:
%    \begin{macrocode}
\newif\ifchilddoc
\newif\ifchilddocmanual
%    \end{macrocode}

% \macro{\childdocname}
% \macro{\childdocjob}
% The macro |\childdocname| stores the name of the main document
% to be compiled. The macro |\childdocjob| stores the name of
% the document on which the \LaTeX{} compiler was originally invoked.
% The content of |\jobname| cannot be compared
% to filenames specified in the source due to different catcodes.
% The following code rescans |\jobname|, stores the result
% in |\childdocname| and saves a copy in |\childdocjob|:
%    \begin{macrocode}
\edef\childdocname{\scantokens\expandafter{\jobname\noexpand}}
\let\childdocjob\childdocname
%    \end{macrocode}

% \macro{\childdocdisable}
% The macro |\childdocdisable| prevents the main file
% from being processed more than once.
% At this stage, the main document command |\childdocmain|
% is assumed to be called once again where it should do nothing.
% Any subsequent call to it should prevent
% a secondary processing of the main document
% It overwrites the forwarding commands
% |\childdocof| and |\childdocforward|
% with empty macros to prevent further inclusions of the main document:
%    \begin{macrocode}
\newcommand{\childdocdisable}
{
  \renewcommand{\childdocmain}[1]{\renewcommand{\childdocmain}[1]{\endinput}}
  \renewcommand{\childdocof}[1]{}
  \renewcommand{\childdocby}[2][]{}
  \renewcommand{\childdocforward}[2][]{}
  \renewcommand{\childdocdisable}{}
}
%    \end{macrocode}

% \macro{\childdocmain}
% The macro |\childdocmain| is to be called at the top of the main file
% with nothing or the main filename (without extension) as argument.
% First, it breaks loops.
% If the argument is not empty and does not match |\childdocname|
% (which is set by the first inclusion of |childdoc.def|),
% |\ifchilddoc| is set to true, |\includeonly| is applied to the child file
% and |\jobname| is set to the main file
% (for proper handling of |.aux| files):
%    \begin{macrocode}
\newcommand{\childdocmain}[1]
{
  \childdocdisable\childdocmain{}
  \if?#1?\else
    \begingroup
      \def\childdoctmp{#1}
      \ifx\childdoctmp\childdocname
        \def\childdoctmp{}
      \else
        \def\childdoctmp
        {
          \childdoctrue
          \includeonly{\childdocname}
          \def\childdocjob{#1}
          \def\jobname{#1}
        }
      \fi
      \expandafter
    \endgroup
    \childdoctmp
  \fi
}
%    \end{macrocode}

% \macro{\childdocof}
% The command |\childdocof| redirects
% compilation to the main file |#1|.
%    \begin{macrocode}
\newcommand{\childdocof}[1]
{
  \childdocdisable
  \childdoctrue
  \includeonly{\childdocname}
  \def\jobname{#1}
  \def\childdocjob{#1}
  \input{#1}
}
%    \end{macrocode}

% \macro{\childdocby}
% The command |\childdocby| ....
%    \begin{macrocode}
\newcommand{\childdocby}[2][]
{
  \childdocdisable
  \childdoctrue
  \childdocmanualtrue
  \if?#1?\else
    \def\jobname{#2}
  \fi
  \def\childdocjob{#2}
  \input{#2}
  \endinput
}
%    \end{macrocode}

% \macro{\childdocforward}
% The command |\childdocforward| redirects
% compilation to the main file or
% (if the optional argument is given) a child file.
% Parameters are set as if the main file
% or a child file starting with |\childdocof| was compiled.
% Then compilation is handed over to the main file:
%    \begin{macrocode}
\newcommand{\childdocforward}[2][]
{
  \begingroup
    \if?#1?
      \def\childdoctmp
      {
        \def\childdocname{#2}
        \def\childdocjob{#2}
        \def\jobname{#2}
        \input{#2}
        \endinput
      }
    \else
      \def\childdoctmp
      {
        \childdocdisable
        \def\childdocname{#2}
        \childdoctrue
        \includeonly{#2}
        \def\childdocjob{#1}
        \def\jobname{#1}
        \input{#1}
        \endinput
      }
    \fi
    \expandafter
  \endgroup
  \childdoctmp
}
%    \end{macrocode}

% \macro{\childdocforwardprefix}
% The command |\childdocforwardprefix| redirects
% compilation to the main or a child file by means of a pattern.
% The prefix |#1| in the current filename is replaced by |#2|
% and the suffix of the current filename is kept
% (it is assumed that the filename does not contain the substring `|~~~|'
% which is used as a delimiter).
% Compilation is handed over to the new file by |\childdocforward|:
%    \begin{macrocode}
\newcommand{\childdocforwardprefix}[3][]
{
  \begingroup
    \def\childdocextract #2##1~~~{\def\childdoctmp{\childdocforward[#1]{#3##1}}}
    \expandafter\childdocextract\childdocname~~~
    \expandafter
  \endgroup
  \childdoctmp
}
%    \end{macrocode}

% \macro{\childdoc}
% The deprecated macro |\childdoc| is a legacy version of |\childdocmain|:
%    \begin{macrocode}
\newcommand{\childdoc}{\childdocmain}
%    \end{macrocode}

% \macro{\childdocredirect}
% The deprecated macro |\childdocredirect| is a legacy version
% of |\childdocforward| and |\childdocforwardprefix|:
%    \begin{macrocode}
\newcommand{\childdocredirect}[2][]
{
  \begingroup
    \if?#1?
      \def\childdoctmp{\childdocforward{#2}}
    \else
      \def\childdoctmp{\childdocforwardprefix{#1}{#2}}
    \fi
    \expandafter
  \endgroup
  \childdoctmp
}
%    \end{macrocode}

%\iffalse
%</package>
%\fi
%
\endinput
|
and perform the replacements as outlined below.
Instead of |\childdocmain{|\textit{main}|}| add the following code
to the top of the main file:
%
\begin{center}
\begin{tabular}{l}
|\||ifdefined\childdocname\endinput\||fi\newif\ifchilddoc|\\
|\edef\childdocname{\scantokens\expandafter{\jobname\noexpand}}|\\
|\def\childdocmain{|\textit{main}|}\||ifx\childdocmain\childdocname\||else|\\
|\childdoctrue\includeonly{\childdocname}\let\jobname\childdocmain\||fi|\\
\end{tabular}
\end{center}
%
Instead of |\childdocof{|\textit{main}|}| just include the main file
at the top of each child file:
%
\begin{center}
|\input{|\textit{main}|}|
\end{center}
%
A simple redirection |\childdocforward{|\textit{dest}|}| is achieved by:
%
\begin{center}
|\def\jobname{|\textit{dest}|}\input{\jobname}|
\end{center}
%
The redirection with prefix
|\childdocforwardprefix[|\textit{prefix}|]{|\textit{dest}|}|
is accomplished by:
%
\begin{center}
\begin{tabular}{l}
|{\edef\jobname{\scantokens\expandafter{\jobname\noexpand}}|\\
|\def\redirectjob |\textit{prefix}|#1~~~{\gdef\jobname{|\textit{dest}|#1}}|\\
|\expandafter\redirectjob\jobname~~~}\input{\jobname}|
\end{tabular}
\end{center}

In an alternative approach,
child documents can be compiled by a specific command line
without additional code or specific definitions:
%
\begin{center}
|... -jobname "|\textit{target}|" "|[\textit{flags}]%
|\includeonly{|\textit{dest}|}\input{|\textit{main}|}"|
\end{center}
%

%%%%%%%%%%%%%%%%%%%%%%%%%%%%%%%%%%%%%%%%%%%%%%%%%%%%%%%%%%%%%%%%%%%%%%%%%%%%%%%%
%%%%%%%%%%%%%%%%%%%%%%%%%%%%%%%%%%%%%%%%%%%%%%%%%%%%%%%%%%%%%%%%%%%%%%%%%%%%%%%%
\section{Information}

%%%%%%%%%%%%%%%%%%%%%%%%%%%%%%%%%%%%%%%%%%%%%%%%%%%%%%%%%%%%%%%%%%%%%%%%%%%%%%%%
\subsection{Copyright}

Copyright \copyright{} 2017--2018 Niklas Beisert

This work may be distributed and/or modified under the
conditions of the \LaTeX{} Project Public License, either version 1.3
of this license or (at your option) any later version.
The latest version of this license is in
  \url{http://www.latex-project.org/lppl.txt}
and version 1.3 or later is part of all distributions of \LaTeX{}
version 2005/12/01 or later.

This work has the LPPL maintenance status `maintained'.

The Current Maintainer of this work is Niklas Beisert.

This work consists of the files |README.txt|, |childdoc.ins| and |childdoc.dtx|
as well as the derived files |childdoc.def|, |cdocsamp.tex|
with |cdocsch1.tex|, |cdocsch2.tex|, |cdocspt3.tex|, |cdocspt4.tex|,
|cdocsdrf.tex|, |cdocsfn1.tex|, |cdocsfn2.tex|
as well as |childdoc.pdf|.

%%%%%%%%%%%%%%%%%%%%%%%%%%%%%%%%%%%%%%%%%%%%%%%%%%%%%%%%%%%%%%%%%%%%%%%%%%%%%%%%
\subsection{Files and Installation}

The package consists of the files:
%
\begin{center}
\begin{tabular}{ll}
    |README.txt|   & readme file \\
    |childdoc.ins| & installation file \\
    |childdoc.dtx| & source file \\
    |childdoc.def| & definition file \\
    |cdocsamp.tex| & sample main file \\
    |cdocsch1.tex| & sample include file \\
    |cdocsch2.tex| & sample include file \\
    |cdocspt3.tex| & sample part file \\
    |cdocspt4.tex| & sample part file \\
    |cdocsdrf.tex| & sample redirection file \\
    |cdocsfn1.tex| & sample redirection file \\
    |cdocsfn2.tex| & sample redirection file \\
    |childdoc.pdf| & manual
\end{tabular}
\end{center}
%
The distribution consists of the files
|README.txt|, |childdoc.ins| and |childdoc.dtx|.
%
\begin{itemize}
\item
Run (pdf)\LaTeX{} on |childdoc.dtx|
to compile the manual |childdoc.pdf| (this file).
\item
Run \LaTeX{} on |childdoc.ins| to create the definitions file |childdoc.def|
and the sample |cdocsamp.tex| with include files
|cdocsch1.tex|, |cdocsch2.tex|, |cdocspt3.tex|, |cdocspt4.tex|,
|cdocsdrf.tex|, |cdocsfn1.tex|, |cdocsfn2.tex|.
Then copy the file |childdoc.def| to an appropriate directory of your \LaTeX{}
distribution, e.g.\ \textit{texmf-root}|/tex/latex/childdoc|.
\end{itemize}

%%%%%%%%%%%%%%%%%%%%%%%%%%%%%%%%%%%%%%%%%%%%%%%%%%%%%%%%%%%%%%%%%%%%%%%%%%%%%%%%
\subsection{Related CTAN Packages}

There are several other packages which offer a similar functionality:
%
\begin{itemize}
\item
The packages
\href{http://ctan.org/pkg/docmute}{\textsf{docmute}},
\href{http://ctan.org/pkg/includex}{\textsf{includex}} and
\href{http://ctan.org/pkg/standalone}{\textsf{standalone}}
provide commands to include only the document body of
a child file thus allowing both files to be compiled individually.
\item
The packages \href{http://ctan.org/pkg/subdocs}{\textsf{subdocs}}
and \href{http://ctan.org/pkg/subfiles}{\textsf{subfiles}}
provide structures in which the main and child documents can be
encapsulated and allowing them to be compiled individually.
The inclusion mechanism is different from the conventional |\include|.
\item
The package \href{http://ctan.org/pkg/combine}{\textsf{combine}}
is an elaborate solution to combine several documents into one.
\end{itemize}
%
See also the CTAN topic \href{http://ctan.org/topic/subdocs}{\textsf{subdocs}}
for further related packages.
The present package differs from the above solutions in that
a document structure constructed with the conventional |\include| mechanism
just needs two extra commands at the top of every file
such that all constituent files can be compiled individually.

%%%%%%%%%%%%%%%%%%%%%%%%%%%%%%%%%%%%%%%%%%%%%%%%%%%%%%%%%%%%%%%%%%%%%%%%%%%%%%%%
%\subsection{Feature Suggestions}
%
%The following is a list of features which may be useful for future
%versions of this package:
%%
%\begin{itemize}
%\item
%\ldots
%\end{itemize}

%%%%%%%%%%%%%%%%%%%%%%%%%%%%%%%%%%%%%%%%%%%%%%%%%%%%%%%%%%%%%%%%%%%%%%%%%%%%%%%%
\subsection{Revision History}

%%%%%%%%%%%%%%%%%%%%%%%%%%%%%%%%%%%%%%%%
\paragraph{v2.0:} 2018/12/30

\begin{itemize}
\item
immediate forward processing
\item
added |\childdocby| mechanism
\item
manual restructured
\end{itemize}

%%%%%%%%%%%%%%%%%%%%%%%%%%%%%%%%%%%%%%%%
\paragraph{v1.6:} 2018/01/17

\begin{itemize}
\item
application for development of include files
\item
corrections to manual
\end{itemize}

%%%%%%%%%%%%%%%%%%%%%%%%%%%%%%%%%%%%%%%%
\paragraph{v1.5:} 2017/05/21

\begin{itemize}
\item
more complete structuring introduced
\item
|\childdocof| introduced
\item
|\childdoc| renamed to |\childdocmain|
\item
|\childredirect| renamed to |\childdocforward| and |\childdocforwardprefix|
and functionality expanded
\end{itemize}

%%%%%%%%%%%%%%%%%%%%%%%%%%%%%%%%%%%%%%%%
\paragraph{v1.0:} 2017/04/27

\begin{itemize}
\item
manual and install package
\item
first version published on CTAN
\end{itemize}

%%%%%%%%%%%%%%%%%%%%%%%%%%%%%%%%%%%%%%%%
\paragraph{v0.6:} 2017/04/26

\begin{itemize}
\item
redirection mechanism added
\end{itemize}

%%%%%%%%%%%%%%%%%%%%%%%%%%%%%%%%%%%%%%%%
\paragraph{v0.5:} 2017/04/26

\begin{itemize}
\item
functionality in definition file
\end{itemize}


%%%%%%%%%%%%%%%%%%%%%%%%%%%%%%%%%%%%%%%%%%%%%%%%%%%%%%%%%%%%%%%%%%%%%%%%%%%%%%%%
%%%%%%%%%%%%%%%%%%%%%%%%%%%%%%%%%%%%%%%%%%%%%%%%%%%%%%%%%%%%%%%%%%%%%%%%%%%%%%%%
%%%%%%%%%%%%%%%%%%%%%%%%%%%%%%%%%%%%%%%%%%%%%%%%%%%%%%%%%%%%%%%%%%%%%%%%%%%%%%%%
\appendix

\settowidth\MacroIndent{\rmfamily\scriptsize 000\ }

 \DocInput{childdoc.dtx}

\end{document}
%</driver>
% \fi
%
% %%%%%%%%%%%%%%%%%%%%%%%%%%%%%%%%%%%%%%%%%%%%%%%%%%%%%%%%%%%%%%%%%%%%%%%%%%%%%%
% %%%%%%%%%%%%%%%%%%%%%%%%%%%%%%%%%%%%%%%%%%%%%%%%%%%%%%%%%%%%%%%%%%%%%%%%%%%%%%
% \section{Sample}
%\iffalse
%<*samplemain>
%\fi
%
% The following presents a sample document
% with two chapters, two parts, a title page,
% a compile flag as well as three forwarding files to set the flag.
% It consists of eight |.tex| files:
% \begin{center}
% \begin{tabular}{ll}
% |cdocsamp.tex|&main file\\
% |cdocsch1.tex|&include file for chapter 1\\
% |cdocsch2.tex|&include file for chapter 2\\
% |cdocspt3.tex|&include file for part 3\\
% |cdocspt4.tex|&include file for part 4\\
% |cdocsdrf.tex|&forwarding file for main file in draft mode\\
% |cdocsfi1.tex|&forwarding file for final version of chapter 1\\
% |cdocsfi2.tex|&forwarding file for final version of chapter 2\\
% \end{tabular}
% \end{center}
% Each of the eight files can be compiled directly by the \LaTeX{} compiler.
%
% %%%%%%%%%%%%%%%%%%%%%%%%%%%%%%%%%%%%%%
% \paragraph{Main File.}
%
% The main file is called |cdocsamp.tex|.
%
% Load the \textsf{childdoc} definitions and
% declare the filename for the main document:
%    \begin{macrocode}
% \iffalse
%
% childdoc.dtx Copyright (C) 2017-2018 Niklas Beisert
%
% This work may be distributed and/or modified under the
% conditions of the LaTeX Project Public License, either version 1.3
% of this license or (at your option) any later version.
% The latest version of this license is in
%   http://www.latex-project.org/lppl.txt
% and version 1.3 or later is part of all distributions of LaTeX
% version 2005/12/01 or later.
%
% This work has the LPPL maintenance status `maintained'.
%
% The Current Maintainer of this work is Niklas Beisert.
%
% This work consists of the files childdoc.dtx and childdoc.ins
% and the derived files childdoc.def and cdocsamp.tex with
% cdocsch1.tex, cdocsch2.tex, cdocsdrf.tex, cdocsfn1.tex, cdocsfn2.tex.
%
%<package>\ifdefined\childdocmain\endinput\fi
%<package>\ProvidesFile{childdoc.def}[2018/12/30 v2.0 child document driver]
%<samplemain>\ProvidesFile{cdocsamp.tex}[2018/12/30 v2.0 sample for childdoc]
%<*driver>
%\ProvidesFile{childdoc.drv}[2018/12/30 v2.0 childdoc reference manual file]
\PassOptionsToClass{10pt,a4paper}{article}
\documentclass{ltxdoc}

\usepackage[margin=35mm]{geometry}
\usepackage{hyperref}
\usepackage{hyperxmp}
\usepackage[usenames]{color}

\hypersetup{colorlinks=true}
\hypersetup{pdfstartview=FitH}
\hypersetup{pdfpagemode=UseNone}
\hypersetup{pdfsource={}}
\hypersetup{pdflang={en-UK}}
\hypersetup{pdfcopyright={Copyright 2017-2018 Niklas Beisert.
  This work may be distributed and/or modified under the
  conditions of the LaTeX Project Public License, either version 1.3
  of this license or (at your option) any later version.}}
\hypersetup{pdflicenseurl={http://www.latex-project.org/lppl.txt}}
\hypersetup{pdfcontactaddress={ETH Zurich, ITP, HIT K,
  Wolfgang-Pauli-Strasse 27}}
\hypersetup{pdfcontactpostcode={8093}}
\hypersetup{pdfcontactcity={Zurich}}
\hypersetup{pdfcontactcountry={Switzerland}}
\hypersetup{pdfcontactemail={nbeisert@itp.phys.ethz.ch}}
\hypersetup{pdfcontacturl={http://people.phys.ethz.ch/\xmptilde nbeisert/}}

\newcommand{\secref}[1]{\hyperref[#1]{section \ref*{#1}}}

\parskip1ex
\parindent0pt
\let\olditemize\itemize
\def\itemize{\olditemize\parskip0pt}

\begin{document}

\title{The \textsf{childdoc} Package}
\hypersetup{pdftitle={The childdoc Package}}
\author{Niklas Beisert\\[2ex]
  Institut f\"ur Theoretische Physik\\
  Eidgen\"ossische Technische Hochschule Z\"urich\\
  Wolfgang-Pauli-Strasse 27, 8093 Z\"urich, Switzerland\\[1ex]
  \href{mailto:nbeisert@itp.phys.ethz.ch}
  {\texttt{nbeisert@itp.phys.ethz.ch}}}
\hypersetup{pdfauthor={Niklas Beisert}}
\hypersetup{pdfsubject={Manual for the LaTeX2e Package childdoc}}
\date{30 December 2018, \textsf{v2.0}}
\maketitle

\begin{abstract}\noindent
\textsf{childdoc} is a \LaTeXe{} package
that enables the direct compilation
of document sections included by |\include|
to individual files.
\end{abstract}

\begingroup
\parskip0ex
\tableofcontents
\endgroup

%%%%%%%%%%%%%%%%%%%%%%%%%%%%%%%%%%%%%%%%%%%%%%%%%%%%%%%%%%%%%%%%%%%%%%%%%%%%%%%%
%%%%%%%%%%%%%%%%%%%%%%%%%%%%%%%%%%%%%%%%%%%%%%%%%%%%%%%%%%%%%%%%%%%%%%%%%%%%%%%%
\section{Introduction}

\LaTeX{} provides a mechanism to structure a large document (such as a book)
into a main file and several child files (containing the chapters)
using the |\include| command.
This mechanism is beneficial for documents
which span hundreds of pages in order to
make the source file(s) more manageable.
Moreover, compilation can be restricted to
selected child files by means of the |\includeonly| command.
The latter feature can be used to reduce the compilation time while editing
(this was significantly more useful in the earlier days of \LaTeX{})
or to generate a smaller document which is easier to navigate.
Another application of |\includeonly| is to generate
documents consisting of selected parts of the complete document.

However, there are a few drawbacks of the plain |\include| mechanism:
\begin{itemize}
\item
The child files cannot be compiled on their own,
they can only be compiled via the main file.
A naive editing environment
(such as a text editor with an option
to have the current file processed by \LaTeX)
may require one to switch to the main file before compiling;
attempting to compile the child file produces errors.
\item
The main file must be modified (each time)
to adjust the |\includeonly| command
to the present needs. This easily leaves the main file in a messy state.
\item
The generated document will always carry the filename
of the main document. This is inconvenient if
several child files are to be compiled and
to be kept for distribution.
\end{itemize}

The present package provides a simple interface
to make child files individually compilable by \LaTeX{}.
Compiling a child file then has the same effect as compiling
the main file with an |\includeonly| command
to select the appropriate child.
Moreover the generated document will carry the name of the child
rather than the main file.
This resolves all three above issues.

This feature is meant to make the editing of books,
thesis documents and lecture notes somewhat more convenient.
However, the package can also be used efficiently for
composing a series of documents (such as exercise sheets)
which are typically distributed individually.
It then assists the author in generating the individual documents
(potentially in different versions)
as well as a document containing the collected series.
Another application is in developing style files
or other kinds of included material
where compilation of the style file could redirect
to a sample or test file.

%%%%%%%%%%%%%%%%%%%%%%%%%%%%%%%%%%%%%%%%%%%%%%%%%%%%%%%%%%%%%%%%%%%%%%%%%%%%%%%%
%%%%%%%%%%%%%%%%%%%%%%%%%%%%%%%%%%%%%%%%%%%%%%%%%%%%%%%%%%%%%%%%%%%%%%%%%%%%%%%%
\section{Usage}

First of all, the package \textsf{childdoc} is \emph{not} a standard
\LaTeXe{} |.sty| style file! Therefore it needs to be invoked in
a non-standard way.

%%%%%%%%%%%%%%%%%%%%%%%%%%%%%%%%%%%%%%%%%%%%%%%%%%%%%%%%%%%%%%%%%%%%%%%%%%%%%%%%
\subsection{Included Files}
\label{sec:include}

%%%%%%%%%%%%%%%%%%%%%%%%%%%%%%%%%%%%%%%%
\DescribeMacro{\childdocmain}
To use the package, add the commands
\begin{center}
\begin{tabular}{l}
|\input{childdoc.def}|\\
|\childdocmain{}|\\
\end{tabular}
\end{center}
at the very top of the main \LaTeX{} file,
in particular \emph{before} the |\documentclass| statement!
The argument of |\childdocmain| should be left empty
(but it must be present).

%%%%%%%%%%%%%%%%%%%%%%%%%%%%%%%%%%%%%%%%
\DescribeMacro{\childdocof}
Furthermore, add the commands
\begin{center}
\begin{tabular}{l}
|\input{childdoc.def}|\\
|\childdocof{|\textit{main}|}|\\
\end{tabular}
\end{center}
at the top of every child file \textit{child}
which is included by |\include{|\textit{child}|}|
from within the main file
(or at least for those files to be compiled individually).
The argument \textit{main} must be the filename of the main file.

There are a couple of
considerations in setting up the main and child documents:

%%%%%%%%%%%%%%%%%%%%%%%%%%%%%%%%%%%%%%%%
\paragraph{Restrictions.}

Please note the following restrictions:
\begin{itemize}
\item
|\childdocmain| must be called with one argument \textit{main}
to ensure compatibility with earlier version of the package.
It must either be empty (|\childdocmain{}|)
or precisely match the filename of the main file in which it is specified.
See \secref{sec:detection} for further information.
\item
The filename \textit{main} must be specified without the |.tex| extension.
\item
The filename \textit{main} is case sensitive
(even in case-insensitive file systems)
due to internal string comparison.
\item
The argument \textit{main} should be fully expanded, it cannot be a macro.
\item
Subdirectories and special characters should be avoided in filenames.
\item
The command |\childdocmain{|\textit{main}|}| must be followed by a whitespace.
It should not be followed immediately by another command
or by a comment mark `|%|'.
This is because the \TeX{} parser reads the token immediately following
the argument of |\childdocmain| and puts it
at the beginning of every child section;
however, a white\-space is ignored.
\end{itemize}

%%%%%%%%%%%%%%%%%%%%%%%%%%%%%%%%%%%%%%%%
\paragraph{Content of Main File.}

It is advisable to place all content in the child files included by |\include|.
Any output contained in the main file will appear in all child documents
unless suppressed manually;
it cannot be suppressed automatically by the |\includeonly| directive
and thus should normally be avoided.
A method to include some content in the main file
by means of conditional processing is described in \secref{sec:conditional}.

%%%%%%%%%%%%%%%%%%%%%%%%%%%%%%%%%%%%%%%%
\paragraph{Page Numbering.}

When only a part of the document is compiled,
the appropriate numbering of pages
(as well as other status parameters)
is determined from the |.aux| files.
The latter contain information from previous passes.
However this information needs to propagate through
all intermediate child documents.
Therefore the page numbering in child documents may well
be inconsistent until the complete document is compiled at least once.

A useful (if unconventional) way to always ensure a consistent
page numbering is to restart the numbering in each child document
and denote the pages by `\textit{child}|.|\textit{page}'
where \textit{child} represents the chapter/section number of the child file.
This can be achieved by the command
|\numberwithin{page}{|\textit{child}|}|
of the \textsf{amsmath} package
where \textit{child} can be |chapter| or |section|
depending on the chosen structuring.
Alternatively, one can modify the macro |\thepage| appropriately
and reset the counter |page| at the start of each child file.

%%%%%%%%%%%%%%%%%%%%%%%%%%%%%%%%%%%%%%%%%%%%%%%%%%%%%%%%%%%%%%%%%%%%%%%%%%%%%%%%
\subsection{Conditional Processing}
\label{sec:conditional}

The package provides a mechanism to compile different versions
of a document. To customise the versions further some conditional processing
can come in handy to distinguish which version is being compiled.
The package provides two macros to describe the compilation context:

%%%%%%%%%%%%%%%%%%%%%%%%%%%%%%%%%%%%%%%%
\DescribeMacro{\ifchilddoc}
The conditional |\ifchilddoc| distinguishes between the compilation of
child documents and the main document:
%
\begin{center}
|\ifchilddoc |\textit{child-code}| |[|\||else |\textit{main-code}]| \||fi|
\end{center}

%%%%%%%%%%%%%%%%%%%%%%%%%%%%%%%%%%%%%%%%
\DescribeMacro{\childdocname}
\DescribeMacro{\childdocjob}
The macro |\childdocname| contains the filename (without extension)
of the main or child file being processed.
Note that |\childdocjob| will always contain the name of the main file.

%%%%%%%%%%%%%%%%%%%%%%%%%%%%%%%%%%%%%%%%
\paragraph{Title Page.}

Conditional processing can be used to include a title or banner page
in the main document when proper precautions are taken.
Importantly, the code in the main file should ensure that the page counter
(as well as other status parameters which are stored in the |.aux| files)
takes the same value after the conditional processing.
Otherwise the page numbers may take divergent values
depending on which part is compiled.

For example, a title page could be declared by:
%
\begin{center}
\begin{tabular}{l}
|\ifchilddoc\||else|\\
|\addtocounter{page}{-1}|\\
\textit{code for title page}\\
|\newpage|\\
|\||fi|
\end{tabular}
\end{center}
%
A banner page for the child documents can be generated by:
%
\begin{center}
\begin{tabular}{l}
|\ifchilddoc|\\
|\addtocounter{page}{-1}|\\
\textit{code for banner page}\\
|\newpage|\\
|\||fi|
\end{tabular}
\end{center}
%
Here one could write a message such as:
\begin{center}
|This is the part \childdocname{} of \childdocjob{}.|
\end{center}

%%%%%%%%%%%%%%%%%%%%%%%%%%%%%%%%%%%%%%%%%%%%%%%%%%%%%%%%%%%%%%%%%%%%%%%%%%%%%%%%
\subsection{Flags}
\label{sec:flags}

The package makes it easy to generate different versions
of the main or child documents.
To this end compilation flags can be defined
and assigned different default values.
They will be particularly useful in conjunction
with the forwarding mechanism described in \secref{sec:forward}.

For example, it may be useful to have a flag |\version|
which can be set to |draft| or |final|.
The document source will contain some conditional code
depending on the value of |\version|.
Suppose further, the flag should default to |final| for the main file
and to |draft| for child files
which is a natural assignment for editing the document.
This is achieved by placing the following code
in the preamble of the main document
(below the |\childdocmain| directive):
%
\begin{center}
\begin{tabular}{l}
|\ifchilddoc|\\
|\providecommand{\version}{draft}|\\
|\||else|\\
|\providecommand{\version}{final}|\\
|\||fi|
\end{tabular}
\end{center}
%
The definition by |\providecommand| makes sure
that previous definitions are not overwritten.
Further statements |\providecommand{\version}{...}|
can thus be added before the above code to override it.

For the main file, one might add a line
(between |\childdocmain| and the above block)
%
\begin{center}
|%\ifchilddoc\||else\providecommand{\version}{draft}\||fi|
\end{center}
%
which can be uncommented to produce a draft version.
Likewise one can add a line to the very top of a child file
(above the |\childdocof{|\textit{main}|}| directive)
%
\begin{center}
|%\providecommand{\version}{final}|
\end{center}
%
which can be uncommented to produce the final version of this child document.

%%%%%%%%%%%%%%%%%%%%%%%%%%%%%%%%%%%%%%%%%%%%%%%%%%%%%%%%%%%%%%%%%%%%%%%%%%%%%%%%
\subsection{Forwarding}
\label{sec:forward}

Different versions of the main or child documents
using compilation flags as described in \secref{sec:flags}
can be (permanently) stored in different files
for convenient compilation, viewing and distribution.
To this end, the package defines a command
to pass on compilation to a different file:

%%%%%%%%%%%%%%%%%%%%%%%%%%%%%%%%%%%%%%%%
\DescribeMacro{\childdocforward}
The command |\childdocforward| redirects processing to
another source file:
%
\begin{center}
\begin{tabular}{l}
|\input{childdoc.def}|\\
|\childdocforward[|\textit{main}|]{|\textit{dest}|}|\\
\end{tabular}
\end{center}
%
The argument \textit{dest} is the destination file
(without extension).
It should be the main file or one of the child files.
Note that further \textsf{childdoc} directives
such as |\childdocof| and |\childdocforward|
in the indicated file will be processed in this form.
The optional argument \textit{main}
passes on directly to the main file \textit{main}
while pretending to compile the child \textit{dest}.
This form behaves as if \textit{dest}
issues |\childdocof{|\textit{main}|}| right away,
and no further \textsf{childdoc} directives will be processed.

%%%%%%%%%%%%%%%%%%%%%%%%%%%%%%%%%%%%%%%%
\DescribeMacro{\...prefix}
In the alternative form |\childdocforwardprefix|,
%
\begin{center}
\begin{tabular}{l}
|\input{childdoc.def}|\\
|\childdocforwardprefix[|\textit{main}|]{|\textit{prefix}|}{|\textit{dest}|}|
\end{tabular}
\end{center}
%
the destination file is determined by a pattern
depending on the current file:
To make this work, the current file must be called
`{\textit{prefix}\hspace{0.2em}\textit{suffix}}'
with \textit{prefix} matching precisely the argument.
Processing is then passed on to the file
`{\textit{dest}\hspace{0.2em}\textit{suffix}}'.
Surely, the same effect is achieved by
directly specifying the
argument `{\textit{dest}\hspace{0.2em}\textit{suffix}}'
in the first form.
However, that requires to set up a different file
for each child. With the alternative form of the command
all these files can have exactly the same content
which simplifies setting them up and maintaining them.

For example, the following file |draft.tex|
with a compilation flag |\version| as described in \secref{sec:flags}
compiles the main document as a draft:
%
\begin{center}
\begin{tabular}{l}
|\def\version{draft}|\\
|\input{childdoc.def}|\\
|\childdocforward{|\textit{main}|}|
\end{tabular}
\end{center}
%
Likewise, the following files |final|\textit{nn}|.tex|
compile the final version of the child document
|child|\textit{nn}|.tex|:
%
\begin{center}
\begin{tabular}{l}
|\def\version{final}|\\
|\input{childdoc.def}|\\
|\childdocforwardprefix{final}{child}|
\end{tabular}
\end{center}
%

Note that when several versions of a main file and/or of each child file
are to be generated, it may be convenient to set up a |Makefile| or
shell script to automatise the process.

%%%%%%%%%%%%%%%%%%%%%%%%%%%%%%%%%%%%%%%%%%%%%%%%%%%%%%%%%%%%%%%%%%%%%%%%%%%%%%%%
\subsection{Command Line Processing}
\label{sec:commandline}

The effect of redirection files can also be achieved by invoking
the \LaTeX{} compiler with a more elaborate command line.
Most conveniently this should be done as part
of a shell script or a |Makefile|.

When using \textsf{childdoc} in the main file, the following
command lines effectively perform a redirection
(note that depending on the shell being used,
backslashes may have to be doubled: `|\|' $\to$ `|\\|'):
%
\begin{center}
|... -jobname "|\textit{target}|" |\\|"|[\textit{flags}]%
|\input{childdoc.def}\childdocforward[|\textit{main}|]{|\textit{dest}|}"|
\end{center}
%
Here \textit{target} is the name of the output file,
\textit{main} is the name of the main file
and \textit{dest} is the name of the main or child file to be processed
(all filenames without extensions).
The optional argument \textit{main} can be omitted
if \textit{main} matches \textit{dest}.
Optionally, compilation \textit{flags} can be defined via |\def| commands.
This command line makes the \TeX{} engine believe
it is compiling the file \textit{target}
whose content is specified as the latter parameter.
The provided code then forwards the processing to
\textit{main} or \textit{dest} as described in \secref{sec:forward}.

%%%%%%%%%%%%%%%%%%%%%%%%%%%%%%%%%%%%%%%%%%%%%%%%%%%%%%%%%%%%%%%%%%%%%%%%%%%%%%%%
\subsection{Include by Input}
\label{sec:input}

Including child documents by |\include| has some restrictions by design.
Most notably, the content of a child document always occupies
its own set of pages; pages cannot be shared between child documents.
Usually, this behaviour makes perfect sense
because each child document contain an essential part of the document.
However, in some situations it may be desirable to compose
a document from a collection of parts
without having mandatory page breaks between then.
For this case, the package
provides a mechanism to include parts
by |\input| which can also be processed individually.
However, by construction this mechanism
requires manual handling of the content to be output.

%%%%%%%%%%%%%%%%%%%%%%%%%%%%%%%%%%%%%%%%
\DescribeMacro{\ifchilddocmanual}
The main file should be prepared as usual, see \secref{sec:include}.
However, the document body must make a distinction
between processing of an individual part and of the main document, e.g.:
%
\begin{center}
\begin{tabular}{l}
|\ifchilddocmanual|\\
|\input{\childdocname}|\\
|\||else|\\
\textit{document body with }|\input{|\textit{part}|}|\\
|\||fi|
\end{tabular}
\end{center}
%
The conditional |\ifchilddocmanual| is true whenever
a part to be included by |\input| is being compiled,
and the name of the part is stored in |\childdocname|.

%%%%%%%%%%%%%%%%%%%%%%%%%%%%%%%%%%%%%%%%
\DescribeMacro{\childdocby}
Each part to be included by |\input| should start with:
%
\begin{center}
\begin{tabular}{l}
|\input{childdoc.def}|\\
|\childdocby{|\textit{main}|}|\\
\end{tabular}
\end{center}
%
The directive |\childdocby| is similar to |\childdocof|
described in \secref{sec:include},
but the subsequent selection of content must be done manually.
To that end, both |\ifchilddoc| and |\ifchilddocmanual|
will be true upon processing of a part,
and the name of the part is stored in |\childdocname|.
Note that |\jobname| will be set to the filename of the current part
so that each part receives an individual |.aux| file
that does not interfere with the |.aux| file(s) of the main document.
This behaviour can be altered by the alternative form
|\childdocby[*]{|\textit{main}|}| (with a non-empty optional argument)
which uses the |.aux| file of the main document
by setting |\jobname| to \textit{main}.

%%%%%%%%%%%%%%%%%%%%%%%%%%%%%%%%%%%%%%%%%%%%%%%%%%%%%%%%%%%%%%%%%%%%%%%%%%%%%%%%
\subsection{Driver Development}
\label{sec:driver}

The \textsf{childdoc} mechanism can also be use for the development
of definition files such as \LaTeX{} styles or classes.
This case differs from the above setup with multiple parts
included by |\include| in that no |\includeonly| should be invoked.
This can be achieved by starting the include file
(before |\ProvidesPackage|) with:
%
\begin{center}
\begin{tabular}{l}
|\input{childdoc.def}|\\
|\childdocforward{|\textit{main}|}|\\
\end{tabular}
\end{center}
%
or alternatively with:
%
\begin{center}
\begin{tabular}{l}
|\input{childdoc.def}|\\
|\childdocby{|\textit{main}|}|\\
\end{tabular}
\end{center}
%
Both forms have slightly different effects as described above.
The main file is prepared as usual, see \secref{sec:include}.

%%%%%%%%%%%%%%%%%%%%%%%%%%%%%%%%%%%%%%%%%%%%%%%%%%%%%%%%%%%%%%%%%%%%%%%%%%%%%%%%
\subsection{Legacy Detection}
\label{sec:detection}

The directive |\childdocmain| in the main file can detect
whether the complete document or merely a child is to be compiled
even without using the directive |\childdocof|.
This method is deprecated because it is less robust
and there is no compelling reason to use it;
it is merely provided for backward compatibility
and it may be removed in future versions.

If the detection mechanism is to be used,
it is mandatory to correctly specify
the filename of the main file as the argument of |\childdocmain|:
%
\begin{center}
\begin{tabular}{l}
|\input{childdoc.def}|\\
|\childdocmain{|\textit{main}|}|\\
\end{tabular}
\end{center}
%
If |\jobname| does not match the argument \textit{main} of |\childdocmain|,
it is assumed that |\jobname| points to the child file to be compiled.
When using |\childdocmain| with the main file specified as argument,
it suffices to start a child file
with just |\input{|\textit{main}|}|
without loading of the package and using |\childdocof|.
If instead all processing is done
with the appropriate \textsf{childdoc} directives,
the argument of \textit{main} of |\childdocmain| can be empty.

An alternative version of the command line processing described
in \secref{sec:commandline} using the detection mechanism reads:
%
\begin{center}
|... -jobname "|\textit{target}|" "|[\textit{flags}]%
[|\def\jobname{|\textit{dest}|}|]|\input{|\textit{main}|}"|
\end{center}

%%%%%%%%%%%%%%%%%%%%%%%%%%%%%%%%%%%%%%%%%%%%%%%%%%%%%%%%%%%%%%%%%%%%%%%%%%%%%%%%
\subsection{Manual Code}
\label{sec:manual}

In case one cannot be certain whether the definitions file |childdoc.def|
is installed on the target \TeX{} distribution
and one prefers not to ship it,
it is conceivable to paste a few relevant commands into the sources.

To that end, drop all statements |\input{childdoc.def}|
and perform the replacements as outlined below.
Instead of |\childdocmain{|\textit{main}|}| add the following code
to the top of the main file:
%
\begin{center}
\begin{tabular}{l}
|\||ifdefined\childdocname\endinput\||fi\newif\ifchilddoc|\\
|\edef\childdocname{\scantokens\expandafter{\jobname\noexpand}}|\\
|\def\childdocmain{|\textit{main}|}\||ifx\childdocmain\childdocname\||else|\\
|\childdoctrue\includeonly{\childdocname}\let\jobname\childdocmain\||fi|\\
\end{tabular}
\end{center}
%
Instead of |\childdocof{|\textit{main}|}| just include the main file
at the top of each child file:
%
\begin{center}
|\input{|\textit{main}|}|
\end{center}
%
A simple redirection |\childdocforward{|\textit{dest}|}| is achieved by:
%
\begin{center}
|\def\jobname{|\textit{dest}|}\input{\jobname}|
\end{center}
%
The redirection with prefix
|\childdocforwardprefix[|\textit{prefix}|]{|\textit{dest}|}|
is accomplished by:
%
\begin{center}
\begin{tabular}{l}
|{\edef\jobname{\scantokens\expandafter{\jobname\noexpand}}|\\
|\def\redirectjob |\textit{prefix}|#1~~~{\gdef\jobname{|\textit{dest}|#1}}|\\
|\expandafter\redirectjob\jobname~~~}\input{\jobname}|
\end{tabular}
\end{center}

In an alternative approach,
child documents can be compiled by a specific command line
without additional code or specific definitions:
%
\begin{center}
|... -jobname "|\textit{target}|" "|[\textit{flags}]%
|\includeonly{|\textit{dest}|}\input{|\textit{main}|}"|
\end{center}
%

%%%%%%%%%%%%%%%%%%%%%%%%%%%%%%%%%%%%%%%%%%%%%%%%%%%%%%%%%%%%%%%%%%%%%%%%%%%%%%%%
%%%%%%%%%%%%%%%%%%%%%%%%%%%%%%%%%%%%%%%%%%%%%%%%%%%%%%%%%%%%%%%%%%%%%%%%%%%%%%%%
\section{Information}

%%%%%%%%%%%%%%%%%%%%%%%%%%%%%%%%%%%%%%%%%%%%%%%%%%%%%%%%%%%%%%%%%%%%%%%%%%%%%%%%
\subsection{Copyright}

Copyright \copyright{} 2017--2018 Niklas Beisert

This work may be distributed and/or modified under the
conditions of the \LaTeX{} Project Public License, either version 1.3
of this license or (at your option) any later version.
The latest version of this license is in
  \url{http://www.latex-project.org/lppl.txt}
and version 1.3 or later is part of all distributions of \LaTeX{}
version 2005/12/01 or later.

This work has the LPPL maintenance status `maintained'.

The Current Maintainer of this work is Niklas Beisert.

This work consists of the files |README.txt|, |childdoc.ins| and |childdoc.dtx|
as well as the derived files |childdoc.def|, |cdocsamp.tex|
with |cdocsch1.tex|, |cdocsch2.tex|, |cdocspt3.tex|, |cdocspt4.tex|,
|cdocsdrf.tex|, |cdocsfn1.tex|, |cdocsfn2.tex|
as well as |childdoc.pdf|.

%%%%%%%%%%%%%%%%%%%%%%%%%%%%%%%%%%%%%%%%%%%%%%%%%%%%%%%%%%%%%%%%%%%%%%%%%%%%%%%%
\subsection{Files and Installation}

The package consists of the files:
%
\begin{center}
\begin{tabular}{ll}
    |README.txt|   & readme file \\
    |childdoc.ins| & installation file \\
    |childdoc.dtx| & source file \\
    |childdoc.def| & definition file \\
    |cdocsamp.tex| & sample main file \\
    |cdocsch1.tex| & sample include file \\
    |cdocsch2.tex| & sample include file \\
    |cdocspt3.tex| & sample part file \\
    |cdocspt4.tex| & sample part file \\
    |cdocsdrf.tex| & sample redirection file \\
    |cdocsfn1.tex| & sample redirection file \\
    |cdocsfn2.tex| & sample redirection file \\
    |childdoc.pdf| & manual
\end{tabular}
\end{center}
%
The distribution consists of the files
|README.txt|, |childdoc.ins| and |childdoc.dtx|.
%
\begin{itemize}
\item
Run (pdf)\LaTeX{} on |childdoc.dtx|
to compile the manual |childdoc.pdf| (this file).
\item
Run \LaTeX{} on |childdoc.ins| to create the definitions file |childdoc.def|
and the sample |cdocsamp.tex| with include files
|cdocsch1.tex|, |cdocsch2.tex|, |cdocspt3.tex|, |cdocspt4.tex|,
|cdocsdrf.tex|, |cdocsfn1.tex|, |cdocsfn2.tex|.
Then copy the file |childdoc.def| to an appropriate directory of your \LaTeX{}
distribution, e.g.\ \textit{texmf-root}|/tex/latex/childdoc|.
\end{itemize}

%%%%%%%%%%%%%%%%%%%%%%%%%%%%%%%%%%%%%%%%%%%%%%%%%%%%%%%%%%%%%%%%%%%%%%%%%%%%%%%%
\subsection{Related CTAN Packages}

There are several other packages which offer a similar functionality:
%
\begin{itemize}
\item
The packages
\href{http://ctan.org/pkg/docmute}{\textsf{docmute}},
\href{http://ctan.org/pkg/includex}{\textsf{includex}} and
\href{http://ctan.org/pkg/standalone}{\textsf{standalone}}
provide commands to include only the document body of
a child file thus allowing both files to be compiled individually.
\item
The packages \href{http://ctan.org/pkg/subdocs}{\textsf{subdocs}}
and \href{http://ctan.org/pkg/subfiles}{\textsf{subfiles}}
provide structures in which the main and child documents can be
encapsulated and allowing them to be compiled individually.
The inclusion mechanism is different from the conventional |\include|.
\item
The package \href{http://ctan.org/pkg/combine}{\textsf{combine}}
is an elaborate solution to combine several documents into one.
\end{itemize}
%
See also the CTAN topic \href{http://ctan.org/topic/subdocs}{\textsf{subdocs}}
for further related packages.
The present package differs from the above solutions in that
a document structure constructed with the conventional |\include| mechanism
just needs two extra commands at the top of every file
such that all constituent files can be compiled individually.

%%%%%%%%%%%%%%%%%%%%%%%%%%%%%%%%%%%%%%%%%%%%%%%%%%%%%%%%%%%%%%%%%%%%%%%%%%%%%%%%
%\subsection{Feature Suggestions}
%
%The following is a list of features which may be useful for future
%versions of this package:
%%
%\begin{itemize}
%\item
%\ldots
%\end{itemize}

%%%%%%%%%%%%%%%%%%%%%%%%%%%%%%%%%%%%%%%%%%%%%%%%%%%%%%%%%%%%%%%%%%%%%%%%%%%%%%%%
\subsection{Revision History}

%%%%%%%%%%%%%%%%%%%%%%%%%%%%%%%%%%%%%%%%
\paragraph{v2.0:} 2018/12/30

\begin{itemize}
\item
immediate forward processing
\item
added |\childdocby| mechanism
\item
manual restructured
\end{itemize}

%%%%%%%%%%%%%%%%%%%%%%%%%%%%%%%%%%%%%%%%
\paragraph{v1.6:} 2018/01/17

\begin{itemize}
\item
application for development of include files
\item
corrections to manual
\end{itemize}

%%%%%%%%%%%%%%%%%%%%%%%%%%%%%%%%%%%%%%%%
\paragraph{v1.5:} 2017/05/21

\begin{itemize}
\item
more complete structuring introduced
\item
|\childdocof| introduced
\item
|\childdoc| renamed to |\childdocmain|
\item
|\childredirect| renamed to |\childdocforward| and |\childdocforwardprefix|
and functionality expanded
\end{itemize}

%%%%%%%%%%%%%%%%%%%%%%%%%%%%%%%%%%%%%%%%
\paragraph{v1.0:} 2017/04/27

\begin{itemize}
\item
manual and install package
\item
first version published on CTAN
\end{itemize}

%%%%%%%%%%%%%%%%%%%%%%%%%%%%%%%%%%%%%%%%
\paragraph{v0.6:} 2017/04/26

\begin{itemize}
\item
redirection mechanism added
\end{itemize}

%%%%%%%%%%%%%%%%%%%%%%%%%%%%%%%%%%%%%%%%
\paragraph{v0.5:} 2017/04/26

\begin{itemize}
\item
functionality in definition file
\end{itemize}


%%%%%%%%%%%%%%%%%%%%%%%%%%%%%%%%%%%%%%%%%%%%%%%%%%%%%%%%%%%%%%%%%%%%%%%%%%%%%%%%
%%%%%%%%%%%%%%%%%%%%%%%%%%%%%%%%%%%%%%%%%%%%%%%%%%%%%%%%%%%%%%%%%%%%%%%%%%%%%%%%
%%%%%%%%%%%%%%%%%%%%%%%%%%%%%%%%%%%%%%%%%%%%%%%%%%%%%%%%%%%%%%%%%%%%%%%%%%%%%%%%
\appendix

\settowidth\MacroIndent{\rmfamily\scriptsize 000\ }

 \DocInput{childdoc.dtx}

\end{document}
%</driver>
% \fi
%
% %%%%%%%%%%%%%%%%%%%%%%%%%%%%%%%%%%%%%%%%%%%%%%%%%%%%%%%%%%%%%%%%%%%%%%%%%%%%%%
% %%%%%%%%%%%%%%%%%%%%%%%%%%%%%%%%%%%%%%%%%%%%%%%%%%%%%%%%%%%%%%%%%%%%%%%%%%%%%%
% \section{Sample}
%\iffalse
%<*samplemain>
%\fi
%
% The following presents a sample document
% with two chapters, two parts, a title page,
% a compile flag as well as three forwarding files to set the flag.
% It consists of eight |.tex| files:
% \begin{center}
% \begin{tabular}{ll}
% |cdocsamp.tex|&main file\\
% |cdocsch1.tex|&include file for chapter 1\\
% |cdocsch2.tex|&include file for chapter 2\\
% |cdocspt3.tex|&include file for part 3\\
% |cdocspt4.tex|&include file for part 4\\
% |cdocsdrf.tex|&forwarding file for main file in draft mode\\
% |cdocsfi1.tex|&forwarding file for final version of chapter 1\\
% |cdocsfi2.tex|&forwarding file for final version of chapter 2\\
% \end{tabular}
% \end{center}
% Each of the eight files can be compiled directly by the \LaTeX{} compiler.
%
% %%%%%%%%%%%%%%%%%%%%%%%%%%%%%%%%%%%%%%
% \paragraph{Main File.}
%
% The main file is called |cdocsamp.tex|.
%
% Load the \textsf{childdoc} definitions and
% declare the filename for the main document:
%    \begin{macrocode}
\input{childdoc.def}
\childdocmain{}
%    \end{macrocode}

% Optional override for |\version| flag:
%    \begin{macrocode}
%%\ifchilddoc\else\providecommand{\version}{draft}\fi
%    \end{macrocode}

% Define the default values for the |\version| flag
% (|final| for the main file and |draft| for childs):
%    \begin{macrocode}
\ifchilddoc
\providecommand{\version}{draft}
\else
\providecommand{\version}{final}
\fi
%    \end{macrocode}

% Load the standard document class:
%    \begin{macrocode}
\documentclass[12pt]{article}
%    \end{macrocode}

% Start the document body:
%    \begin{macrocode}
\begin{document}
%    \end{macrocode}

% Declare a title page.
% Print title, part of document being processed and version flag:
%    \begin{macrocode}
\addtocounter{page}{-1}
\begin{center}
{\LARGE\bfseries{}childdoc example\par}
\vspace{1cm}
\ifchilddoc
\ifchilddocmanual part\else chapter\fi:
`\childdocname' of `\childdocjob'\par
\else
main document: `\childdocjob'\par
\fi
version: \version\par
\end{center}
\newpage
%    \end{macrocode}

% Manually include selected file,
% otherwise process as usual:
%    \begin{macrocode}
\ifchilddocmanual
\section*{part `\childdocname'}
\input{\childdocname}
\else
%    \end{macrocode}

% Include the two chapters:
%    \begin{macrocode}
\include{cdocsch1}
\include{cdocsch2}
%    \end{macrocode}

% Include the two parts unless only chapters should be displayed:
%    \begin{macrocode}
\ifchilddoc\else
\section{part three}
\input{cdocspt3}
\section{part four}
\input{cdocspt4}
\fi
%    \end{macrocode}

% Process as usual until here:
%    \begin{macrocode}
\fi
%    \end{macrocode}

% End of document body:
%    \begin{macrocode}
\end{document}
%    \end{macrocode}
%\iffalse
%</samplemain>
%\fi
%
% %%%%%%%%%%%%%%%%%%%%%%%%%%%%%%%%%%%%%%
% \paragraph{Chapter Include Files.}
%
% The include files are called |cdocsch1.tex| and |cdocsch2.tex|.
%
%\iffalse
%<*samplechap1|samplechap2>
%\fi

% Optional override for |\version| flag:
%    \begin{macrocode}
%%\providecommand{\version}{final}
%    \end{macrocode}

% Include the main document:
%    \begin{macrocode}
\input{childdoc.def}
\childdocof{cdocsamp}
%    \end{macrocode}

%\iffalse
%</samplechap1|samplechap2>
%\fi
%
%\iffalse
%<*samplechap1>
%\fi
% Some text for chapter 1:
%    \begin{macrocode}
\section{one}
some text in chapter one
%    \end{macrocode}

%\iffalse
%</samplechap1>
%\fi
% Some text for chapter 2:
%\iffalse
%<*samplechap2>
%\fi
%    \begin{macrocode}
\section{two}
more text in chapter two
%    \end{macrocode}

%\iffalse
%</samplechap2>
%\fi
%
% %%%%%%%%%%%%%%%%%%%%%%%%%%%%%%%%%%%%%%
% \paragraph{Part Include Files.}
%
% The include files are called |cdocspt3.tex| and |cdocspt4.tex|.
%
%\iffalse
%<*samplepart3|samplepart4>
%\fi

% Optional override for |\version| flag:
%    \begin{macrocode}
%%\providecommand{\version}{final}
%    \end{macrocode}

% Include the main document:
%    \begin{macrocode}
\input{childdoc.def}
\childdocby{cdocsamp}
%    \end{macrocode}

%\iffalse
%</samplepart3|samplepart4>
%\fi
%
%\iffalse
%<*samplepart3>
%\fi
% Some text for part 3:
%    \begin{macrocode}
some text in part three
%    \end{macrocode}

%\iffalse
%</samplepart3>
%\fi
% Some text for part 4:
%\iffalse
%<*samplepart4>
%\fi
%    \begin{macrocode}
more text in part four
%    \end{macrocode}

%\iffalse
%</samplepart4>
%\fi
%
% %%%%%%%%%%%%%%%%%%%%%%%%%%%%%%%%%%%%%%
% \paragraph{Forwarding for a Complete Draft.}
%
% The following forwarding file |cdocsdrf.tex|
% compiles the main document in draft mode:
%\iffalse
%<*sampledraft>
%\fi
%    \begin{macrocode}
\def\version{draft}
\input{childdoc.def}
\childdocforward{cdocsamp}
%    \end{macrocode}

%\iffalse
%</sampledraft>
%\fi
%
% %%%%%%%%%%%%%%%%%%%%%%%%%%%%%%%%%%%%%%
% \paragraph{Forwarding for Final Version of the Chapters.}
%
% The following forwarding files |cdocsfn1.tex| and |cdocsfn2.tex|
% (with identical content)
% compile the final versions of the child documents
% |cdocsch1.tex| and |cdocsch2.tex|, respectively:
%\iffalse
%<*samplefinal>
%\fi
%    \begin{macrocode}
\def\version{final}
\input{childdoc.def}
\childdocforwardprefix[cdocsamp]{cdocsfn}{cdocsch}
%    \end{macrocode}

%\iffalse
%</samplefinal>
%\fi
%
% %%%%%%%%%%%%%%%%%%%%%%%%%%%%%%%%%%%%%%
% \paragraph{Command Line Processing.}
%
% The following three command lines generate the output files
% |cdocscld|, |cdocscl1| and |cdocscl2|
% which should be identical to
% |cdocsdrf|, |cdocsch1| and |cdocsfn2|, respectively:
% \begin{center}
% \begin{tabular}{l}
% |latex -jobname cdocscld \|\\
% |  "\def\version{draft}\input{childdoc.def}\childdocforward{cdocsamp}"|\\
% |latex -jobname cdocscl1 \|\\
% |  "\input{childdoc.def}\childdocforward[cdocsamp]{cdocsch1}"|\\
% |latex -jobname cdocscl2 \|\\
% |  "\def\version{final}\input{childdoc.def}\childdocforward{cdocsch2}"|
% \end{tabular}
% \end{center}
% Note that the trailing backslash on each first line
% merely continues the input to the second line
% (for convenient cut ant paste).
% Furthermore, the command |latex| can be replaced by any
% of its alternative versions such as |pdflatex|.
%
% %%%%%%%%%%%%%%%%%%%%%%%%%%%%%%%%%%%%%%%%%%%%%%%%%%%%%%%%%%%%%%%%%%%%%%%%%%%%%%
% %%%%%%%%%%%%%%%%%%%%%%%%%%%%%%%%%%%%%%%%%%%%%%%%%%%%%%%%%%%%%%%%%%%%%%%%%%%%%%
% \section{Implementation}
%\iffalse
%<*package>
%\fi
%
% This section describes the definitions file |childdoc.def|.

% The definitions cannot be loaded using |\usepackage| or |\RequirePackage|
% which has a mechanism to prevent loading a style file more than once.
% When loading the definitions by means of |\input|
% multiple instances have to be prevented manually:
%\iffalse
%This code needs to be before the `\ProvidesFile' directive
%which is defined at the beginning of this file.
%Therefore it is also placed there and commented out here.
%</package>
%<*discard>
%\fi
%    \begin{macrocode}
\ifdefined\childdocmain\endinput\fi
%    \end{macrocode}
%\iffalse
%</discard>
%<*package>
%\fi
%
% \macro{\ifchilddoc}
% \macro{\ifchilddocmanual}
% The conditional |\ifchilddoc| tells whether a
% child (true) or main (false) document is being compiled.
% The conditional |\ifchilddocmanual| tells whether
% the |\includeonly| mechanism is used (false) or
% the selection of child files must be performed manually (true).
% The definitions initialise to false:
%    \begin{macrocode}
\newif\ifchilddoc
\newif\ifchilddocmanual
%    \end{macrocode}

% \macro{\childdocname}
% \macro{\childdocjob}
% The macro |\childdocname| stores the name of the main document
% to be compiled. The macro |\childdocjob| stores the name of
% the document on which the \LaTeX{} compiler was originally invoked.
% The content of |\jobname| cannot be compared
% to filenames specified in the source due to different catcodes.
% The following code rescans |\jobname|, stores the result
% in |\childdocname| and saves a copy in |\childdocjob|:
%    \begin{macrocode}
\edef\childdocname{\scantokens\expandafter{\jobname\noexpand}}
\let\childdocjob\childdocname
%    \end{macrocode}

% \macro{\childdocdisable}
% The macro |\childdocdisable| prevents the main file
% from being processed more than once.
% At this stage, the main document command |\childdocmain|
% is assumed to be called once again where it should do nothing.
% Any subsequent call to it should prevent
% a secondary processing of the main document
% It overwrites the forwarding commands
% |\childdocof| and |\childdocforward|
% with empty macros to prevent further inclusions of the main document:
%    \begin{macrocode}
\newcommand{\childdocdisable}
{
  \renewcommand{\childdocmain}[1]{\renewcommand{\childdocmain}[1]{\endinput}}
  \renewcommand{\childdocof}[1]{}
  \renewcommand{\childdocby}[2][]{}
  \renewcommand{\childdocforward}[2][]{}
  \renewcommand{\childdocdisable}{}
}
%    \end{macrocode}

% \macro{\childdocmain}
% The macro |\childdocmain| is to be called at the top of the main file
% with nothing or the main filename (without extension) as argument.
% First, it breaks loops.
% If the argument is not empty and does not match |\childdocname|
% (which is set by the first inclusion of |childdoc.def|),
% |\ifchilddoc| is set to true, |\includeonly| is applied to the child file
% and |\jobname| is set to the main file
% (for proper handling of |.aux| files):
%    \begin{macrocode}
\newcommand{\childdocmain}[1]
{
  \childdocdisable\childdocmain{}
  \if?#1?\else
    \begingroup
      \def\childdoctmp{#1}
      \ifx\childdoctmp\childdocname
        \def\childdoctmp{}
      \else
        \def\childdoctmp
        {
          \childdoctrue
          \includeonly{\childdocname}
          \def\childdocjob{#1}
          \def\jobname{#1}
        }
      \fi
      \expandafter
    \endgroup
    \childdoctmp
  \fi
}
%    \end{macrocode}

% \macro{\childdocof}
% The command |\childdocof| redirects
% compilation to the main file |#1|.
%    \begin{macrocode}
\newcommand{\childdocof}[1]
{
  \childdocdisable
  \childdoctrue
  \includeonly{\childdocname}
  \def\jobname{#1}
  \def\childdocjob{#1}
  \input{#1}
}
%    \end{macrocode}

% \macro{\childdocby}
% The command |\childdocby| ....
%    \begin{macrocode}
\newcommand{\childdocby}[2][]
{
  \childdocdisable
  \childdoctrue
  \childdocmanualtrue
  \if?#1?\else
    \def\jobname{#2}
  \fi
  \def\childdocjob{#2}
  \input{#2}
  \endinput
}
%    \end{macrocode}

% \macro{\childdocforward}
% The command |\childdocforward| redirects
% compilation to the main file or
% (if the optional argument is given) a child file.
% Parameters are set as if the main file
% or a child file starting with |\childdocof| was compiled.
% Then compilation is handed over to the main file:
%    \begin{macrocode}
\newcommand{\childdocforward}[2][]
{
  \begingroup
    \if?#1?
      \def\childdoctmp
      {
        \def\childdocname{#2}
        \def\childdocjob{#2}
        \def\jobname{#2}
        \input{#2}
        \endinput
      }
    \else
      \def\childdoctmp
      {
        \childdocdisable
        \def\childdocname{#2}
        \childdoctrue
        \includeonly{#2}
        \def\childdocjob{#1}
        \def\jobname{#1}
        \input{#1}
        \endinput
      }
    \fi
    \expandafter
  \endgroup
  \childdoctmp
}
%    \end{macrocode}

% \macro{\childdocforwardprefix}
% The command |\childdocforwardprefix| redirects
% compilation to the main or a child file by means of a pattern.
% The prefix |#1| in the current filename is replaced by |#2|
% and the suffix of the current filename is kept
% (it is assumed that the filename does not contain the substring `|~~~|'
% which is used as a delimiter).
% Compilation is handed over to the new file by |\childdocforward|:
%    \begin{macrocode}
\newcommand{\childdocforwardprefix}[3][]
{
  \begingroup
    \def\childdocextract #2##1~~~{\def\childdoctmp{\childdocforward[#1]{#3##1}}}
    \expandafter\childdocextract\childdocname~~~
    \expandafter
  \endgroup
  \childdoctmp
}
%    \end{macrocode}

% \macro{\childdoc}
% The deprecated macro |\childdoc| is a legacy version of |\childdocmain|:
%    \begin{macrocode}
\newcommand{\childdoc}{\childdocmain}
%    \end{macrocode}

% \macro{\childdocredirect}
% The deprecated macro |\childdocredirect| is a legacy version
% of |\childdocforward| and |\childdocforwardprefix|:
%    \begin{macrocode}
\newcommand{\childdocredirect}[2][]
{
  \begingroup
    \if?#1?
      \def\childdoctmp{\childdocforward{#2}}
    \else
      \def\childdoctmp{\childdocforwardprefix{#1}{#2}}
    \fi
    \expandafter
  \endgroup
  \childdoctmp
}
%    \end{macrocode}

%\iffalse
%</package>
%\fi
%
\endinput

\childdocmain{}
%    \end{macrocode}

% Optional override for |\version| flag:
%    \begin{macrocode}
%%\ifchilddoc\else\providecommand{\version}{draft}\fi
%    \end{macrocode}

% Define the default values for the |\version| flag
% (|final| for the main file and |draft| for childs):
%    \begin{macrocode}
\ifchilddoc
\providecommand{\version}{draft}
\else
\providecommand{\version}{final}
\fi
%    \end{macrocode}

% Load the standard document class:
%    \begin{macrocode}
\documentclass[12pt]{article}
%    \end{macrocode}

% Start the document body:
%    \begin{macrocode}
\begin{document}
%    \end{macrocode}

% Declare a title page.
% Print title, part of document being processed and version flag:
%    \begin{macrocode}
\addtocounter{page}{-1}
\begin{center}
{\LARGE\bfseries{}childdoc example\par}
\vspace{1cm}
\ifchilddoc
\ifchilddocmanual part\else chapter\fi:
`\childdocname' of `\childdocjob'\par
\else
main document: `\childdocjob'\par
\fi
version: \version\par
\end{center}
\newpage
%    \end{macrocode}

% Manually include selected file,
% otherwise process as usual:
%    \begin{macrocode}
\ifchilddocmanual
\section*{part `\childdocname'}
\input{\childdocname}
\else
%    \end{macrocode}

% Include the two chapters:
%    \begin{macrocode}
\include{cdocsch1}
\include{cdocsch2}
%    \end{macrocode}

% Include the two parts unless only chapters should be displayed:
%    \begin{macrocode}
\ifchilddoc\else
\section{part three}
\input{cdocspt3}
\section{part four}
\input{cdocspt4}
\fi
%    \end{macrocode}

% Process as usual until here:
%    \begin{macrocode}
\fi
%    \end{macrocode}

% End of document body:
%    \begin{macrocode}
\end{document}
%    \end{macrocode}
%\iffalse
%</samplemain>
%\fi
%
% %%%%%%%%%%%%%%%%%%%%%%%%%%%%%%%%%%%%%%
% \paragraph{Chapter Include Files.}
%
% The include files are called |cdocsch1.tex| and |cdocsch2.tex|.
%
%\iffalse
%<*samplechap1|samplechap2>
%\fi

% Optional override for |\version| flag:
%    \begin{macrocode}
%%\providecommand{\version}{final}
%    \end{macrocode}

% Include the main document:
%    \begin{macrocode}
% \iffalse
%
% childdoc.dtx Copyright (C) 2017-2018 Niklas Beisert
%
% This work may be distributed and/or modified under the
% conditions of the LaTeX Project Public License, either version 1.3
% of this license or (at your option) any later version.
% The latest version of this license is in
%   http://www.latex-project.org/lppl.txt
% and version 1.3 or later is part of all distributions of LaTeX
% version 2005/12/01 or later.
%
% This work has the LPPL maintenance status `maintained'.
%
% The Current Maintainer of this work is Niklas Beisert.
%
% This work consists of the files childdoc.dtx and childdoc.ins
% and the derived files childdoc.def and cdocsamp.tex with
% cdocsch1.tex, cdocsch2.tex, cdocsdrf.tex, cdocsfn1.tex, cdocsfn2.tex.
%
%<package>\ifdefined\childdocmain\endinput\fi
%<package>\ProvidesFile{childdoc.def}[2018/12/30 v2.0 child document driver]
%<samplemain>\ProvidesFile{cdocsamp.tex}[2018/12/30 v2.0 sample for childdoc]
%<*driver>
%\ProvidesFile{childdoc.drv}[2018/12/30 v2.0 childdoc reference manual file]
\PassOptionsToClass{10pt,a4paper}{article}
\documentclass{ltxdoc}

\usepackage[margin=35mm]{geometry}
\usepackage{hyperref}
\usepackage{hyperxmp}
\usepackage[usenames]{color}

\hypersetup{colorlinks=true}
\hypersetup{pdfstartview=FitH}
\hypersetup{pdfpagemode=UseNone}
\hypersetup{pdfsource={}}
\hypersetup{pdflang={en-UK}}
\hypersetup{pdfcopyright={Copyright 2017-2018 Niklas Beisert.
  This work may be distributed and/or modified under the
  conditions of the LaTeX Project Public License, either version 1.3
  of this license or (at your option) any later version.}}
\hypersetup{pdflicenseurl={http://www.latex-project.org/lppl.txt}}
\hypersetup{pdfcontactaddress={ETH Zurich, ITP, HIT K,
  Wolfgang-Pauli-Strasse 27}}
\hypersetup{pdfcontactpostcode={8093}}
\hypersetup{pdfcontactcity={Zurich}}
\hypersetup{pdfcontactcountry={Switzerland}}
\hypersetup{pdfcontactemail={nbeisert@itp.phys.ethz.ch}}
\hypersetup{pdfcontacturl={http://people.phys.ethz.ch/\xmptilde nbeisert/}}

\newcommand{\secref}[1]{\hyperref[#1]{section \ref*{#1}}}

\parskip1ex
\parindent0pt
\let\olditemize\itemize
\def\itemize{\olditemize\parskip0pt}

\begin{document}

\title{The \textsf{childdoc} Package}
\hypersetup{pdftitle={The childdoc Package}}
\author{Niklas Beisert\\[2ex]
  Institut f\"ur Theoretische Physik\\
  Eidgen\"ossische Technische Hochschule Z\"urich\\
  Wolfgang-Pauli-Strasse 27, 8093 Z\"urich, Switzerland\\[1ex]
  \href{mailto:nbeisert@itp.phys.ethz.ch}
  {\texttt{nbeisert@itp.phys.ethz.ch}}}
\hypersetup{pdfauthor={Niklas Beisert}}
\hypersetup{pdfsubject={Manual for the LaTeX2e Package childdoc}}
\date{30 December 2018, \textsf{v2.0}}
\maketitle

\begin{abstract}\noindent
\textsf{childdoc} is a \LaTeXe{} package
that enables the direct compilation
of document sections included by |\include|
to individual files.
\end{abstract}

\begingroup
\parskip0ex
\tableofcontents
\endgroup

%%%%%%%%%%%%%%%%%%%%%%%%%%%%%%%%%%%%%%%%%%%%%%%%%%%%%%%%%%%%%%%%%%%%%%%%%%%%%%%%
%%%%%%%%%%%%%%%%%%%%%%%%%%%%%%%%%%%%%%%%%%%%%%%%%%%%%%%%%%%%%%%%%%%%%%%%%%%%%%%%
\section{Introduction}

\LaTeX{} provides a mechanism to structure a large document (such as a book)
into a main file and several child files (containing the chapters)
using the |\include| command.
This mechanism is beneficial for documents
which span hundreds of pages in order to
make the source file(s) more manageable.
Moreover, compilation can be restricted to
selected child files by means of the |\includeonly| command.
The latter feature can be used to reduce the compilation time while editing
(this was significantly more useful in the earlier days of \LaTeX{})
or to generate a smaller document which is easier to navigate.
Another application of |\includeonly| is to generate
documents consisting of selected parts of the complete document.

However, there are a few drawbacks of the plain |\include| mechanism:
\begin{itemize}
\item
The child files cannot be compiled on their own,
they can only be compiled via the main file.
A naive editing environment
(such as a text editor with an option
to have the current file processed by \LaTeX)
may require one to switch to the main file before compiling;
attempting to compile the child file produces errors.
\item
The main file must be modified (each time)
to adjust the |\includeonly| command
to the present needs. This easily leaves the main file in a messy state.
\item
The generated document will always carry the filename
of the main document. This is inconvenient if
several child files are to be compiled and
to be kept for distribution.
\end{itemize}

The present package provides a simple interface
to make child files individually compilable by \LaTeX{}.
Compiling a child file then has the same effect as compiling
the main file with an |\includeonly| command
to select the appropriate child.
Moreover the generated document will carry the name of the child
rather than the main file.
This resolves all three above issues.

This feature is meant to make the editing of books,
thesis documents and lecture notes somewhat more convenient.
However, the package can also be used efficiently for
composing a series of documents (such as exercise sheets)
which are typically distributed individually.
It then assists the author in generating the individual documents
(potentially in different versions)
as well as a document containing the collected series.
Another application is in developing style files
or other kinds of included material
where compilation of the style file could redirect
to a sample or test file.

%%%%%%%%%%%%%%%%%%%%%%%%%%%%%%%%%%%%%%%%%%%%%%%%%%%%%%%%%%%%%%%%%%%%%%%%%%%%%%%%
%%%%%%%%%%%%%%%%%%%%%%%%%%%%%%%%%%%%%%%%%%%%%%%%%%%%%%%%%%%%%%%%%%%%%%%%%%%%%%%%
\section{Usage}

First of all, the package \textsf{childdoc} is \emph{not} a standard
\LaTeXe{} |.sty| style file! Therefore it needs to be invoked in
a non-standard way.

%%%%%%%%%%%%%%%%%%%%%%%%%%%%%%%%%%%%%%%%%%%%%%%%%%%%%%%%%%%%%%%%%%%%%%%%%%%%%%%%
\subsection{Included Files}
\label{sec:include}

%%%%%%%%%%%%%%%%%%%%%%%%%%%%%%%%%%%%%%%%
\DescribeMacro{\childdocmain}
To use the package, add the commands
\begin{center}
\begin{tabular}{l}
|\input{childdoc.def}|\\
|\childdocmain{}|\\
\end{tabular}
\end{center}
at the very top of the main \LaTeX{} file,
in particular \emph{before} the |\documentclass| statement!
The argument of |\childdocmain| should be left empty
(but it must be present).

%%%%%%%%%%%%%%%%%%%%%%%%%%%%%%%%%%%%%%%%
\DescribeMacro{\childdocof}
Furthermore, add the commands
\begin{center}
\begin{tabular}{l}
|\input{childdoc.def}|\\
|\childdocof{|\textit{main}|}|\\
\end{tabular}
\end{center}
at the top of every child file \textit{child}
which is included by |\include{|\textit{child}|}|
from within the main file
(or at least for those files to be compiled individually).
The argument \textit{main} must be the filename of the main file.

There are a couple of
considerations in setting up the main and child documents:

%%%%%%%%%%%%%%%%%%%%%%%%%%%%%%%%%%%%%%%%
\paragraph{Restrictions.}

Please note the following restrictions:
\begin{itemize}
\item
|\childdocmain| must be called with one argument \textit{main}
to ensure compatibility with earlier version of the package.
It must either be empty (|\childdocmain{}|)
or precisely match the filename of the main file in which it is specified.
See \secref{sec:detection} for further information.
\item
The filename \textit{main} must be specified without the |.tex| extension.
\item
The filename \textit{main} is case sensitive
(even in case-insensitive file systems)
due to internal string comparison.
\item
The argument \textit{main} should be fully expanded, it cannot be a macro.
\item
Subdirectories and special characters should be avoided in filenames.
\item
The command |\childdocmain{|\textit{main}|}| must be followed by a whitespace.
It should not be followed immediately by another command
or by a comment mark `|%|'.
This is because the \TeX{} parser reads the token immediately following
the argument of |\childdocmain| and puts it
at the beginning of every child section;
however, a white\-space is ignored.
\end{itemize}

%%%%%%%%%%%%%%%%%%%%%%%%%%%%%%%%%%%%%%%%
\paragraph{Content of Main File.}

It is advisable to place all content in the child files included by |\include|.
Any output contained in the main file will appear in all child documents
unless suppressed manually;
it cannot be suppressed automatically by the |\includeonly| directive
and thus should normally be avoided.
A method to include some content in the main file
by means of conditional processing is described in \secref{sec:conditional}.

%%%%%%%%%%%%%%%%%%%%%%%%%%%%%%%%%%%%%%%%
\paragraph{Page Numbering.}

When only a part of the document is compiled,
the appropriate numbering of pages
(as well as other status parameters)
is determined from the |.aux| files.
The latter contain information from previous passes.
However this information needs to propagate through
all intermediate child documents.
Therefore the page numbering in child documents may well
be inconsistent until the complete document is compiled at least once.

A useful (if unconventional) way to always ensure a consistent
page numbering is to restart the numbering in each child document
and denote the pages by `\textit{child}|.|\textit{page}'
where \textit{child} represents the chapter/section number of the child file.
This can be achieved by the command
|\numberwithin{page}{|\textit{child}|}|
of the \textsf{amsmath} package
where \textit{child} can be |chapter| or |section|
depending on the chosen structuring.
Alternatively, one can modify the macro |\thepage| appropriately
and reset the counter |page| at the start of each child file.

%%%%%%%%%%%%%%%%%%%%%%%%%%%%%%%%%%%%%%%%%%%%%%%%%%%%%%%%%%%%%%%%%%%%%%%%%%%%%%%%
\subsection{Conditional Processing}
\label{sec:conditional}

The package provides a mechanism to compile different versions
of a document. To customise the versions further some conditional processing
can come in handy to distinguish which version is being compiled.
The package provides two macros to describe the compilation context:

%%%%%%%%%%%%%%%%%%%%%%%%%%%%%%%%%%%%%%%%
\DescribeMacro{\ifchilddoc}
The conditional |\ifchilddoc| distinguishes between the compilation of
child documents and the main document:
%
\begin{center}
|\ifchilddoc |\textit{child-code}| |[|\||else |\textit{main-code}]| \||fi|
\end{center}

%%%%%%%%%%%%%%%%%%%%%%%%%%%%%%%%%%%%%%%%
\DescribeMacro{\childdocname}
\DescribeMacro{\childdocjob}
The macro |\childdocname| contains the filename (without extension)
of the main or child file being processed.
Note that |\childdocjob| will always contain the name of the main file.

%%%%%%%%%%%%%%%%%%%%%%%%%%%%%%%%%%%%%%%%
\paragraph{Title Page.}

Conditional processing can be used to include a title or banner page
in the main document when proper precautions are taken.
Importantly, the code in the main file should ensure that the page counter
(as well as other status parameters which are stored in the |.aux| files)
takes the same value after the conditional processing.
Otherwise the page numbers may take divergent values
depending on which part is compiled.

For example, a title page could be declared by:
%
\begin{center}
\begin{tabular}{l}
|\ifchilddoc\||else|\\
|\addtocounter{page}{-1}|\\
\textit{code for title page}\\
|\newpage|\\
|\||fi|
\end{tabular}
\end{center}
%
A banner page for the child documents can be generated by:
%
\begin{center}
\begin{tabular}{l}
|\ifchilddoc|\\
|\addtocounter{page}{-1}|\\
\textit{code for banner page}\\
|\newpage|\\
|\||fi|
\end{tabular}
\end{center}
%
Here one could write a message such as:
\begin{center}
|This is the part \childdocname{} of \childdocjob{}.|
\end{center}

%%%%%%%%%%%%%%%%%%%%%%%%%%%%%%%%%%%%%%%%%%%%%%%%%%%%%%%%%%%%%%%%%%%%%%%%%%%%%%%%
\subsection{Flags}
\label{sec:flags}

The package makes it easy to generate different versions
of the main or child documents.
To this end compilation flags can be defined
and assigned different default values.
They will be particularly useful in conjunction
with the forwarding mechanism described in \secref{sec:forward}.

For example, it may be useful to have a flag |\version|
which can be set to |draft| or |final|.
The document source will contain some conditional code
depending on the value of |\version|.
Suppose further, the flag should default to |final| for the main file
and to |draft| for child files
which is a natural assignment for editing the document.
This is achieved by placing the following code
in the preamble of the main document
(below the |\childdocmain| directive):
%
\begin{center}
\begin{tabular}{l}
|\ifchilddoc|\\
|\providecommand{\version}{draft}|\\
|\||else|\\
|\providecommand{\version}{final}|\\
|\||fi|
\end{tabular}
\end{center}
%
The definition by |\providecommand| makes sure
that previous definitions are not overwritten.
Further statements |\providecommand{\version}{...}|
can thus be added before the above code to override it.

For the main file, one might add a line
(between |\childdocmain| and the above block)
%
\begin{center}
|%\ifchilddoc\||else\providecommand{\version}{draft}\||fi|
\end{center}
%
which can be uncommented to produce a draft version.
Likewise one can add a line to the very top of a child file
(above the |\childdocof{|\textit{main}|}| directive)
%
\begin{center}
|%\providecommand{\version}{final}|
\end{center}
%
which can be uncommented to produce the final version of this child document.

%%%%%%%%%%%%%%%%%%%%%%%%%%%%%%%%%%%%%%%%%%%%%%%%%%%%%%%%%%%%%%%%%%%%%%%%%%%%%%%%
\subsection{Forwarding}
\label{sec:forward}

Different versions of the main or child documents
using compilation flags as described in \secref{sec:flags}
can be (permanently) stored in different files
for convenient compilation, viewing and distribution.
To this end, the package defines a command
to pass on compilation to a different file:

%%%%%%%%%%%%%%%%%%%%%%%%%%%%%%%%%%%%%%%%
\DescribeMacro{\childdocforward}
The command |\childdocforward| redirects processing to
another source file:
%
\begin{center}
\begin{tabular}{l}
|\input{childdoc.def}|\\
|\childdocforward[|\textit{main}|]{|\textit{dest}|}|\\
\end{tabular}
\end{center}
%
The argument \textit{dest} is the destination file
(without extension).
It should be the main file or one of the child files.
Note that further \textsf{childdoc} directives
such as |\childdocof| and |\childdocforward|
in the indicated file will be processed in this form.
The optional argument \textit{main}
passes on directly to the main file \textit{main}
while pretending to compile the child \textit{dest}.
This form behaves as if \textit{dest}
issues |\childdocof{|\textit{main}|}| right away,
and no further \textsf{childdoc} directives will be processed.

%%%%%%%%%%%%%%%%%%%%%%%%%%%%%%%%%%%%%%%%
\DescribeMacro{\...prefix}
In the alternative form |\childdocforwardprefix|,
%
\begin{center}
\begin{tabular}{l}
|\input{childdoc.def}|\\
|\childdocforwardprefix[|\textit{main}|]{|\textit{prefix}|}{|\textit{dest}|}|
\end{tabular}
\end{center}
%
the destination file is determined by a pattern
depending on the current file:
To make this work, the current file must be called
`{\textit{prefix}\hspace{0.2em}\textit{suffix}}'
with \textit{prefix} matching precisely the argument.
Processing is then passed on to the file
`{\textit{dest}\hspace{0.2em}\textit{suffix}}'.
Surely, the same effect is achieved by
directly specifying the
argument `{\textit{dest}\hspace{0.2em}\textit{suffix}}'
in the first form.
However, that requires to set up a different file
for each child. With the alternative form of the command
all these files can have exactly the same content
which simplifies setting them up and maintaining them.

For example, the following file |draft.tex|
with a compilation flag |\version| as described in \secref{sec:flags}
compiles the main document as a draft:
%
\begin{center}
\begin{tabular}{l}
|\def\version{draft}|\\
|\input{childdoc.def}|\\
|\childdocforward{|\textit{main}|}|
\end{tabular}
\end{center}
%
Likewise, the following files |final|\textit{nn}|.tex|
compile the final version of the child document
|child|\textit{nn}|.tex|:
%
\begin{center}
\begin{tabular}{l}
|\def\version{final}|\\
|\input{childdoc.def}|\\
|\childdocforwardprefix{final}{child}|
\end{tabular}
\end{center}
%

Note that when several versions of a main file and/or of each child file
are to be generated, it may be convenient to set up a |Makefile| or
shell script to automatise the process.

%%%%%%%%%%%%%%%%%%%%%%%%%%%%%%%%%%%%%%%%%%%%%%%%%%%%%%%%%%%%%%%%%%%%%%%%%%%%%%%%
\subsection{Command Line Processing}
\label{sec:commandline}

The effect of redirection files can also be achieved by invoking
the \LaTeX{} compiler with a more elaborate command line.
Most conveniently this should be done as part
of a shell script or a |Makefile|.

When using \textsf{childdoc} in the main file, the following
command lines effectively perform a redirection
(note that depending on the shell being used,
backslashes may have to be doubled: `|\|' $\to$ `|\\|'):
%
\begin{center}
|... -jobname "|\textit{target}|" |\\|"|[\textit{flags}]%
|\input{childdoc.def}\childdocforward[|\textit{main}|]{|\textit{dest}|}"|
\end{center}
%
Here \textit{target} is the name of the output file,
\textit{main} is the name of the main file
and \textit{dest} is the name of the main or child file to be processed
(all filenames without extensions).
The optional argument \textit{main} can be omitted
if \textit{main} matches \textit{dest}.
Optionally, compilation \textit{flags} can be defined via |\def| commands.
This command line makes the \TeX{} engine believe
it is compiling the file \textit{target}
whose content is specified as the latter parameter.
The provided code then forwards the processing to
\textit{main} or \textit{dest} as described in \secref{sec:forward}.

%%%%%%%%%%%%%%%%%%%%%%%%%%%%%%%%%%%%%%%%%%%%%%%%%%%%%%%%%%%%%%%%%%%%%%%%%%%%%%%%
\subsection{Include by Input}
\label{sec:input}

Including child documents by |\include| has some restrictions by design.
Most notably, the content of a child document always occupies
its own set of pages; pages cannot be shared between child documents.
Usually, this behaviour makes perfect sense
because each child document contain an essential part of the document.
However, in some situations it may be desirable to compose
a document from a collection of parts
without having mandatory page breaks between then.
For this case, the package
provides a mechanism to include parts
by |\input| which can also be processed individually.
However, by construction this mechanism
requires manual handling of the content to be output.

%%%%%%%%%%%%%%%%%%%%%%%%%%%%%%%%%%%%%%%%
\DescribeMacro{\ifchilddocmanual}
The main file should be prepared as usual, see \secref{sec:include}.
However, the document body must make a distinction
between processing of an individual part and of the main document, e.g.:
%
\begin{center}
\begin{tabular}{l}
|\ifchilddocmanual|\\
|\input{\childdocname}|\\
|\||else|\\
\textit{document body with }|\input{|\textit{part}|}|\\
|\||fi|
\end{tabular}
\end{center}
%
The conditional |\ifchilddocmanual| is true whenever
a part to be included by |\input| is being compiled,
and the name of the part is stored in |\childdocname|.

%%%%%%%%%%%%%%%%%%%%%%%%%%%%%%%%%%%%%%%%
\DescribeMacro{\childdocby}
Each part to be included by |\input| should start with:
%
\begin{center}
\begin{tabular}{l}
|\input{childdoc.def}|\\
|\childdocby{|\textit{main}|}|\\
\end{tabular}
\end{center}
%
The directive |\childdocby| is similar to |\childdocof|
described in \secref{sec:include},
but the subsequent selection of content must be done manually.
To that end, both |\ifchilddoc| and |\ifchilddocmanual|
will be true upon processing of a part,
and the name of the part is stored in |\childdocname|.
Note that |\jobname| will be set to the filename of the current part
so that each part receives an individual |.aux| file
that does not interfere with the |.aux| file(s) of the main document.
This behaviour can be altered by the alternative form
|\childdocby[*]{|\textit{main}|}| (with a non-empty optional argument)
which uses the |.aux| file of the main document
by setting |\jobname| to \textit{main}.

%%%%%%%%%%%%%%%%%%%%%%%%%%%%%%%%%%%%%%%%%%%%%%%%%%%%%%%%%%%%%%%%%%%%%%%%%%%%%%%%
\subsection{Driver Development}
\label{sec:driver}

The \textsf{childdoc} mechanism can also be use for the development
of definition files such as \LaTeX{} styles or classes.
This case differs from the above setup with multiple parts
included by |\include| in that no |\includeonly| should be invoked.
This can be achieved by starting the include file
(before |\ProvidesPackage|) with:
%
\begin{center}
\begin{tabular}{l}
|\input{childdoc.def}|\\
|\childdocforward{|\textit{main}|}|\\
\end{tabular}
\end{center}
%
or alternatively with:
%
\begin{center}
\begin{tabular}{l}
|\input{childdoc.def}|\\
|\childdocby{|\textit{main}|}|\\
\end{tabular}
\end{center}
%
Both forms have slightly different effects as described above.
The main file is prepared as usual, see \secref{sec:include}.

%%%%%%%%%%%%%%%%%%%%%%%%%%%%%%%%%%%%%%%%%%%%%%%%%%%%%%%%%%%%%%%%%%%%%%%%%%%%%%%%
\subsection{Legacy Detection}
\label{sec:detection}

The directive |\childdocmain| in the main file can detect
whether the complete document or merely a child is to be compiled
even without using the directive |\childdocof|.
This method is deprecated because it is less robust
and there is no compelling reason to use it;
it is merely provided for backward compatibility
and it may be removed in future versions.

If the detection mechanism is to be used,
it is mandatory to correctly specify
the filename of the main file as the argument of |\childdocmain|:
%
\begin{center}
\begin{tabular}{l}
|\input{childdoc.def}|\\
|\childdocmain{|\textit{main}|}|\\
\end{tabular}
\end{center}
%
If |\jobname| does not match the argument \textit{main} of |\childdocmain|,
it is assumed that |\jobname| points to the child file to be compiled.
When using |\childdocmain| with the main file specified as argument,
it suffices to start a child file
with just |\input{|\textit{main}|}|
without loading of the package and using |\childdocof|.
If instead all processing is done
with the appropriate \textsf{childdoc} directives,
the argument of \textit{main} of |\childdocmain| can be empty.

An alternative version of the command line processing described
in \secref{sec:commandline} using the detection mechanism reads:
%
\begin{center}
|... -jobname "|\textit{target}|" "|[\textit{flags}]%
[|\def\jobname{|\textit{dest}|}|]|\input{|\textit{main}|}"|
\end{center}

%%%%%%%%%%%%%%%%%%%%%%%%%%%%%%%%%%%%%%%%%%%%%%%%%%%%%%%%%%%%%%%%%%%%%%%%%%%%%%%%
\subsection{Manual Code}
\label{sec:manual}

In case one cannot be certain whether the definitions file |childdoc.def|
is installed on the target \TeX{} distribution
and one prefers not to ship it,
it is conceivable to paste a few relevant commands into the sources.

To that end, drop all statements |\input{childdoc.def}|
and perform the replacements as outlined below.
Instead of |\childdocmain{|\textit{main}|}| add the following code
to the top of the main file:
%
\begin{center}
\begin{tabular}{l}
|\||ifdefined\childdocname\endinput\||fi\newif\ifchilddoc|\\
|\edef\childdocname{\scantokens\expandafter{\jobname\noexpand}}|\\
|\def\childdocmain{|\textit{main}|}\||ifx\childdocmain\childdocname\||else|\\
|\childdoctrue\includeonly{\childdocname}\let\jobname\childdocmain\||fi|\\
\end{tabular}
\end{center}
%
Instead of |\childdocof{|\textit{main}|}| just include the main file
at the top of each child file:
%
\begin{center}
|\input{|\textit{main}|}|
\end{center}
%
A simple redirection |\childdocforward{|\textit{dest}|}| is achieved by:
%
\begin{center}
|\def\jobname{|\textit{dest}|}\input{\jobname}|
\end{center}
%
The redirection with prefix
|\childdocforwardprefix[|\textit{prefix}|]{|\textit{dest}|}|
is accomplished by:
%
\begin{center}
\begin{tabular}{l}
|{\edef\jobname{\scantokens\expandafter{\jobname\noexpand}}|\\
|\def\redirectjob |\textit{prefix}|#1~~~{\gdef\jobname{|\textit{dest}|#1}}|\\
|\expandafter\redirectjob\jobname~~~}\input{\jobname}|
\end{tabular}
\end{center}

In an alternative approach,
child documents can be compiled by a specific command line
without additional code or specific definitions:
%
\begin{center}
|... -jobname "|\textit{target}|" "|[\textit{flags}]%
|\includeonly{|\textit{dest}|}\input{|\textit{main}|}"|
\end{center}
%

%%%%%%%%%%%%%%%%%%%%%%%%%%%%%%%%%%%%%%%%%%%%%%%%%%%%%%%%%%%%%%%%%%%%%%%%%%%%%%%%
%%%%%%%%%%%%%%%%%%%%%%%%%%%%%%%%%%%%%%%%%%%%%%%%%%%%%%%%%%%%%%%%%%%%%%%%%%%%%%%%
\section{Information}

%%%%%%%%%%%%%%%%%%%%%%%%%%%%%%%%%%%%%%%%%%%%%%%%%%%%%%%%%%%%%%%%%%%%%%%%%%%%%%%%
\subsection{Copyright}

Copyright \copyright{} 2017--2018 Niklas Beisert

This work may be distributed and/or modified under the
conditions of the \LaTeX{} Project Public License, either version 1.3
of this license or (at your option) any later version.
The latest version of this license is in
  \url{http://www.latex-project.org/lppl.txt}
and version 1.3 or later is part of all distributions of \LaTeX{}
version 2005/12/01 or later.

This work has the LPPL maintenance status `maintained'.

The Current Maintainer of this work is Niklas Beisert.

This work consists of the files |README.txt|, |childdoc.ins| and |childdoc.dtx|
as well as the derived files |childdoc.def|, |cdocsamp.tex|
with |cdocsch1.tex|, |cdocsch2.tex|, |cdocspt3.tex|, |cdocspt4.tex|,
|cdocsdrf.tex|, |cdocsfn1.tex|, |cdocsfn2.tex|
as well as |childdoc.pdf|.

%%%%%%%%%%%%%%%%%%%%%%%%%%%%%%%%%%%%%%%%%%%%%%%%%%%%%%%%%%%%%%%%%%%%%%%%%%%%%%%%
\subsection{Files and Installation}

The package consists of the files:
%
\begin{center}
\begin{tabular}{ll}
    |README.txt|   & readme file \\
    |childdoc.ins| & installation file \\
    |childdoc.dtx| & source file \\
    |childdoc.def| & definition file \\
    |cdocsamp.tex| & sample main file \\
    |cdocsch1.tex| & sample include file \\
    |cdocsch2.tex| & sample include file \\
    |cdocspt3.tex| & sample part file \\
    |cdocspt4.tex| & sample part file \\
    |cdocsdrf.tex| & sample redirection file \\
    |cdocsfn1.tex| & sample redirection file \\
    |cdocsfn2.tex| & sample redirection file \\
    |childdoc.pdf| & manual
\end{tabular}
\end{center}
%
The distribution consists of the files
|README.txt|, |childdoc.ins| and |childdoc.dtx|.
%
\begin{itemize}
\item
Run (pdf)\LaTeX{} on |childdoc.dtx|
to compile the manual |childdoc.pdf| (this file).
\item
Run \LaTeX{} on |childdoc.ins| to create the definitions file |childdoc.def|
and the sample |cdocsamp.tex| with include files
|cdocsch1.tex|, |cdocsch2.tex|, |cdocspt3.tex|, |cdocspt4.tex|,
|cdocsdrf.tex|, |cdocsfn1.tex|, |cdocsfn2.tex|.
Then copy the file |childdoc.def| to an appropriate directory of your \LaTeX{}
distribution, e.g.\ \textit{texmf-root}|/tex/latex/childdoc|.
\end{itemize}

%%%%%%%%%%%%%%%%%%%%%%%%%%%%%%%%%%%%%%%%%%%%%%%%%%%%%%%%%%%%%%%%%%%%%%%%%%%%%%%%
\subsection{Related CTAN Packages}

There are several other packages which offer a similar functionality:
%
\begin{itemize}
\item
The packages
\href{http://ctan.org/pkg/docmute}{\textsf{docmute}},
\href{http://ctan.org/pkg/includex}{\textsf{includex}} and
\href{http://ctan.org/pkg/standalone}{\textsf{standalone}}
provide commands to include only the document body of
a child file thus allowing both files to be compiled individually.
\item
The packages \href{http://ctan.org/pkg/subdocs}{\textsf{subdocs}}
and \href{http://ctan.org/pkg/subfiles}{\textsf{subfiles}}
provide structures in which the main and child documents can be
encapsulated and allowing them to be compiled individually.
The inclusion mechanism is different from the conventional |\include|.
\item
The package \href{http://ctan.org/pkg/combine}{\textsf{combine}}
is an elaborate solution to combine several documents into one.
\end{itemize}
%
See also the CTAN topic \href{http://ctan.org/topic/subdocs}{\textsf{subdocs}}
for further related packages.
The present package differs from the above solutions in that
a document structure constructed with the conventional |\include| mechanism
just needs two extra commands at the top of every file
such that all constituent files can be compiled individually.

%%%%%%%%%%%%%%%%%%%%%%%%%%%%%%%%%%%%%%%%%%%%%%%%%%%%%%%%%%%%%%%%%%%%%%%%%%%%%%%%
%\subsection{Feature Suggestions}
%
%The following is a list of features which may be useful for future
%versions of this package:
%%
%\begin{itemize}
%\item
%\ldots
%\end{itemize}

%%%%%%%%%%%%%%%%%%%%%%%%%%%%%%%%%%%%%%%%%%%%%%%%%%%%%%%%%%%%%%%%%%%%%%%%%%%%%%%%
\subsection{Revision History}

%%%%%%%%%%%%%%%%%%%%%%%%%%%%%%%%%%%%%%%%
\paragraph{v2.0:} 2018/12/30

\begin{itemize}
\item
immediate forward processing
\item
added |\childdocby| mechanism
\item
manual restructured
\end{itemize}

%%%%%%%%%%%%%%%%%%%%%%%%%%%%%%%%%%%%%%%%
\paragraph{v1.6:} 2018/01/17

\begin{itemize}
\item
application for development of include files
\item
corrections to manual
\end{itemize}

%%%%%%%%%%%%%%%%%%%%%%%%%%%%%%%%%%%%%%%%
\paragraph{v1.5:} 2017/05/21

\begin{itemize}
\item
more complete structuring introduced
\item
|\childdocof| introduced
\item
|\childdoc| renamed to |\childdocmain|
\item
|\childredirect| renamed to |\childdocforward| and |\childdocforwardprefix|
and functionality expanded
\end{itemize}

%%%%%%%%%%%%%%%%%%%%%%%%%%%%%%%%%%%%%%%%
\paragraph{v1.0:} 2017/04/27

\begin{itemize}
\item
manual and install package
\item
first version published on CTAN
\end{itemize}

%%%%%%%%%%%%%%%%%%%%%%%%%%%%%%%%%%%%%%%%
\paragraph{v0.6:} 2017/04/26

\begin{itemize}
\item
redirection mechanism added
\end{itemize}

%%%%%%%%%%%%%%%%%%%%%%%%%%%%%%%%%%%%%%%%
\paragraph{v0.5:} 2017/04/26

\begin{itemize}
\item
functionality in definition file
\end{itemize}


%%%%%%%%%%%%%%%%%%%%%%%%%%%%%%%%%%%%%%%%%%%%%%%%%%%%%%%%%%%%%%%%%%%%%%%%%%%%%%%%
%%%%%%%%%%%%%%%%%%%%%%%%%%%%%%%%%%%%%%%%%%%%%%%%%%%%%%%%%%%%%%%%%%%%%%%%%%%%%%%%
%%%%%%%%%%%%%%%%%%%%%%%%%%%%%%%%%%%%%%%%%%%%%%%%%%%%%%%%%%%%%%%%%%%%%%%%%%%%%%%%
\appendix

\settowidth\MacroIndent{\rmfamily\scriptsize 000\ }

 \DocInput{childdoc.dtx}

\end{document}
%</driver>
% \fi
%
% %%%%%%%%%%%%%%%%%%%%%%%%%%%%%%%%%%%%%%%%%%%%%%%%%%%%%%%%%%%%%%%%%%%%%%%%%%%%%%
% %%%%%%%%%%%%%%%%%%%%%%%%%%%%%%%%%%%%%%%%%%%%%%%%%%%%%%%%%%%%%%%%%%%%%%%%%%%%%%
% \section{Sample}
%\iffalse
%<*samplemain>
%\fi
%
% The following presents a sample document
% with two chapters, two parts, a title page,
% a compile flag as well as three forwarding files to set the flag.
% It consists of eight |.tex| files:
% \begin{center}
% \begin{tabular}{ll}
% |cdocsamp.tex|&main file\\
% |cdocsch1.tex|&include file for chapter 1\\
% |cdocsch2.tex|&include file for chapter 2\\
% |cdocspt3.tex|&include file for part 3\\
% |cdocspt4.tex|&include file for part 4\\
% |cdocsdrf.tex|&forwarding file for main file in draft mode\\
% |cdocsfi1.tex|&forwarding file for final version of chapter 1\\
% |cdocsfi2.tex|&forwarding file for final version of chapter 2\\
% \end{tabular}
% \end{center}
% Each of the eight files can be compiled directly by the \LaTeX{} compiler.
%
% %%%%%%%%%%%%%%%%%%%%%%%%%%%%%%%%%%%%%%
% \paragraph{Main File.}
%
% The main file is called |cdocsamp.tex|.
%
% Load the \textsf{childdoc} definitions and
% declare the filename for the main document:
%    \begin{macrocode}
\input{childdoc.def}
\childdocmain{}
%    \end{macrocode}

% Optional override for |\version| flag:
%    \begin{macrocode}
%%\ifchilddoc\else\providecommand{\version}{draft}\fi
%    \end{macrocode}

% Define the default values for the |\version| flag
% (|final| for the main file and |draft| for childs):
%    \begin{macrocode}
\ifchilddoc
\providecommand{\version}{draft}
\else
\providecommand{\version}{final}
\fi
%    \end{macrocode}

% Load the standard document class:
%    \begin{macrocode}
\documentclass[12pt]{article}
%    \end{macrocode}

% Start the document body:
%    \begin{macrocode}
\begin{document}
%    \end{macrocode}

% Declare a title page.
% Print title, part of document being processed and version flag:
%    \begin{macrocode}
\addtocounter{page}{-1}
\begin{center}
{\LARGE\bfseries{}childdoc example\par}
\vspace{1cm}
\ifchilddoc
\ifchilddocmanual part\else chapter\fi:
`\childdocname' of `\childdocjob'\par
\else
main document: `\childdocjob'\par
\fi
version: \version\par
\end{center}
\newpage
%    \end{macrocode}

% Manually include selected file,
% otherwise process as usual:
%    \begin{macrocode}
\ifchilddocmanual
\section*{part `\childdocname'}
\input{\childdocname}
\else
%    \end{macrocode}

% Include the two chapters:
%    \begin{macrocode}
\include{cdocsch1}
\include{cdocsch2}
%    \end{macrocode}

% Include the two parts unless only chapters should be displayed:
%    \begin{macrocode}
\ifchilddoc\else
\section{part three}
\input{cdocspt3}
\section{part four}
\input{cdocspt4}
\fi
%    \end{macrocode}

% Process as usual until here:
%    \begin{macrocode}
\fi
%    \end{macrocode}

% End of document body:
%    \begin{macrocode}
\end{document}
%    \end{macrocode}
%\iffalse
%</samplemain>
%\fi
%
% %%%%%%%%%%%%%%%%%%%%%%%%%%%%%%%%%%%%%%
% \paragraph{Chapter Include Files.}
%
% The include files are called |cdocsch1.tex| and |cdocsch2.tex|.
%
%\iffalse
%<*samplechap1|samplechap2>
%\fi

% Optional override for |\version| flag:
%    \begin{macrocode}
%%\providecommand{\version}{final}
%    \end{macrocode}

% Include the main document:
%    \begin{macrocode}
\input{childdoc.def}
\childdocof{cdocsamp}
%    \end{macrocode}

%\iffalse
%</samplechap1|samplechap2>
%\fi
%
%\iffalse
%<*samplechap1>
%\fi
% Some text for chapter 1:
%    \begin{macrocode}
\section{one}
some text in chapter one
%    \end{macrocode}

%\iffalse
%</samplechap1>
%\fi
% Some text for chapter 2:
%\iffalse
%<*samplechap2>
%\fi
%    \begin{macrocode}
\section{two}
more text in chapter two
%    \end{macrocode}

%\iffalse
%</samplechap2>
%\fi
%
% %%%%%%%%%%%%%%%%%%%%%%%%%%%%%%%%%%%%%%
% \paragraph{Part Include Files.}
%
% The include files are called |cdocspt3.tex| and |cdocspt4.tex|.
%
%\iffalse
%<*samplepart3|samplepart4>
%\fi

% Optional override for |\version| flag:
%    \begin{macrocode}
%%\providecommand{\version}{final}
%    \end{macrocode}

% Include the main document:
%    \begin{macrocode}
\input{childdoc.def}
\childdocby{cdocsamp}
%    \end{macrocode}

%\iffalse
%</samplepart3|samplepart4>
%\fi
%
%\iffalse
%<*samplepart3>
%\fi
% Some text for part 3:
%    \begin{macrocode}
some text in part three
%    \end{macrocode}

%\iffalse
%</samplepart3>
%\fi
% Some text for part 4:
%\iffalse
%<*samplepart4>
%\fi
%    \begin{macrocode}
more text in part four
%    \end{macrocode}

%\iffalse
%</samplepart4>
%\fi
%
% %%%%%%%%%%%%%%%%%%%%%%%%%%%%%%%%%%%%%%
% \paragraph{Forwarding for a Complete Draft.}
%
% The following forwarding file |cdocsdrf.tex|
% compiles the main document in draft mode:
%\iffalse
%<*sampledraft>
%\fi
%    \begin{macrocode}
\def\version{draft}
\input{childdoc.def}
\childdocforward{cdocsamp}
%    \end{macrocode}

%\iffalse
%</sampledraft>
%\fi
%
% %%%%%%%%%%%%%%%%%%%%%%%%%%%%%%%%%%%%%%
% \paragraph{Forwarding for Final Version of the Chapters.}
%
% The following forwarding files |cdocsfn1.tex| and |cdocsfn2.tex|
% (with identical content)
% compile the final versions of the child documents
% |cdocsch1.tex| and |cdocsch2.tex|, respectively:
%\iffalse
%<*samplefinal>
%\fi
%    \begin{macrocode}
\def\version{final}
\input{childdoc.def}
\childdocforwardprefix[cdocsamp]{cdocsfn}{cdocsch}
%    \end{macrocode}

%\iffalse
%</samplefinal>
%\fi
%
% %%%%%%%%%%%%%%%%%%%%%%%%%%%%%%%%%%%%%%
% \paragraph{Command Line Processing.}
%
% The following three command lines generate the output files
% |cdocscld|, |cdocscl1| and |cdocscl2|
% which should be identical to
% |cdocsdrf|, |cdocsch1| and |cdocsfn2|, respectively:
% \begin{center}
% \begin{tabular}{l}
% |latex -jobname cdocscld \|\\
% |  "\def\version{draft}\input{childdoc.def}\childdocforward{cdocsamp}"|\\
% |latex -jobname cdocscl1 \|\\
% |  "\input{childdoc.def}\childdocforward[cdocsamp]{cdocsch1}"|\\
% |latex -jobname cdocscl2 \|\\
% |  "\def\version{final}\input{childdoc.def}\childdocforward{cdocsch2}"|
% \end{tabular}
% \end{center}
% Note that the trailing backslash on each first line
% merely continues the input to the second line
% (for convenient cut ant paste).
% Furthermore, the command |latex| can be replaced by any
% of its alternative versions such as |pdflatex|.
%
% %%%%%%%%%%%%%%%%%%%%%%%%%%%%%%%%%%%%%%%%%%%%%%%%%%%%%%%%%%%%%%%%%%%%%%%%%%%%%%
% %%%%%%%%%%%%%%%%%%%%%%%%%%%%%%%%%%%%%%%%%%%%%%%%%%%%%%%%%%%%%%%%%%%%%%%%%%%%%%
% \section{Implementation}
%\iffalse
%<*package>
%\fi
%
% This section describes the definitions file |childdoc.def|.

% The definitions cannot be loaded using |\usepackage| or |\RequirePackage|
% which has a mechanism to prevent loading a style file more than once.
% When loading the definitions by means of |\input|
% multiple instances have to be prevented manually:
%\iffalse
%This code needs to be before the `\ProvidesFile' directive
%which is defined at the beginning of this file.
%Therefore it is also placed there and commented out here.
%</package>
%<*discard>
%\fi
%    \begin{macrocode}
\ifdefined\childdocmain\endinput\fi
%    \end{macrocode}
%\iffalse
%</discard>
%<*package>
%\fi
%
% \macro{\ifchilddoc}
% \macro{\ifchilddocmanual}
% The conditional |\ifchilddoc| tells whether a
% child (true) or main (false) document is being compiled.
% The conditional |\ifchilddocmanual| tells whether
% the |\includeonly| mechanism is used (false) or
% the selection of child files must be performed manually (true).
% The definitions initialise to false:
%    \begin{macrocode}
\newif\ifchilddoc
\newif\ifchilddocmanual
%    \end{macrocode}

% \macro{\childdocname}
% \macro{\childdocjob}
% The macro |\childdocname| stores the name of the main document
% to be compiled. The macro |\childdocjob| stores the name of
% the document on which the \LaTeX{} compiler was originally invoked.
% The content of |\jobname| cannot be compared
% to filenames specified in the source due to different catcodes.
% The following code rescans |\jobname|, stores the result
% in |\childdocname| and saves a copy in |\childdocjob|:
%    \begin{macrocode}
\edef\childdocname{\scantokens\expandafter{\jobname\noexpand}}
\let\childdocjob\childdocname
%    \end{macrocode}

% \macro{\childdocdisable}
% The macro |\childdocdisable| prevents the main file
% from being processed more than once.
% At this stage, the main document command |\childdocmain|
% is assumed to be called once again where it should do nothing.
% Any subsequent call to it should prevent
% a secondary processing of the main document
% It overwrites the forwarding commands
% |\childdocof| and |\childdocforward|
% with empty macros to prevent further inclusions of the main document:
%    \begin{macrocode}
\newcommand{\childdocdisable}
{
  \renewcommand{\childdocmain}[1]{\renewcommand{\childdocmain}[1]{\endinput}}
  \renewcommand{\childdocof}[1]{}
  \renewcommand{\childdocby}[2][]{}
  \renewcommand{\childdocforward}[2][]{}
  \renewcommand{\childdocdisable}{}
}
%    \end{macrocode}

% \macro{\childdocmain}
% The macro |\childdocmain| is to be called at the top of the main file
% with nothing or the main filename (without extension) as argument.
% First, it breaks loops.
% If the argument is not empty and does not match |\childdocname|
% (which is set by the first inclusion of |childdoc.def|),
% |\ifchilddoc| is set to true, |\includeonly| is applied to the child file
% and |\jobname| is set to the main file
% (for proper handling of |.aux| files):
%    \begin{macrocode}
\newcommand{\childdocmain}[1]
{
  \childdocdisable\childdocmain{}
  \if?#1?\else
    \begingroup
      \def\childdoctmp{#1}
      \ifx\childdoctmp\childdocname
        \def\childdoctmp{}
      \else
        \def\childdoctmp
        {
          \childdoctrue
          \includeonly{\childdocname}
          \def\childdocjob{#1}
          \def\jobname{#1}
        }
      \fi
      \expandafter
    \endgroup
    \childdoctmp
  \fi
}
%    \end{macrocode}

% \macro{\childdocof}
% The command |\childdocof| redirects
% compilation to the main file |#1|.
%    \begin{macrocode}
\newcommand{\childdocof}[1]
{
  \childdocdisable
  \childdoctrue
  \includeonly{\childdocname}
  \def\jobname{#1}
  \def\childdocjob{#1}
  \input{#1}
}
%    \end{macrocode}

% \macro{\childdocby}
% The command |\childdocby| ....
%    \begin{macrocode}
\newcommand{\childdocby}[2][]
{
  \childdocdisable
  \childdoctrue
  \childdocmanualtrue
  \if?#1?\else
    \def\jobname{#2}
  \fi
  \def\childdocjob{#2}
  \input{#2}
  \endinput
}
%    \end{macrocode}

% \macro{\childdocforward}
% The command |\childdocforward| redirects
% compilation to the main file or
% (if the optional argument is given) a child file.
% Parameters are set as if the main file
% or a child file starting with |\childdocof| was compiled.
% Then compilation is handed over to the main file:
%    \begin{macrocode}
\newcommand{\childdocforward}[2][]
{
  \begingroup
    \if?#1?
      \def\childdoctmp
      {
        \def\childdocname{#2}
        \def\childdocjob{#2}
        \def\jobname{#2}
        \input{#2}
        \endinput
      }
    \else
      \def\childdoctmp
      {
        \childdocdisable
        \def\childdocname{#2}
        \childdoctrue
        \includeonly{#2}
        \def\childdocjob{#1}
        \def\jobname{#1}
        \input{#1}
        \endinput
      }
    \fi
    \expandafter
  \endgroup
  \childdoctmp
}
%    \end{macrocode}

% \macro{\childdocforwardprefix}
% The command |\childdocforwardprefix| redirects
% compilation to the main or a child file by means of a pattern.
% The prefix |#1| in the current filename is replaced by |#2|
% and the suffix of the current filename is kept
% (it is assumed that the filename does not contain the substring `|~~~|'
% which is used as a delimiter).
% Compilation is handed over to the new file by |\childdocforward|:
%    \begin{macrocode}
\newcommand{\childdocforwardprefix}[3][]
{
  \begingroup
    \def\childdocextract #2##1~~~{\def\childdoctmp{\childdocforward[#1]{#3##1}}}
    \expandafter\childdocextract\childdocname~~~
    \expandafter
  \endgroup
  \childdoctmp
}
%    \end{macrocode}

% \macro{\childdoc}
% The deprecated macro |\childdoc| is a legacy version of |\childdocmain|:
%    \begin{macrocode}
\newcommand{\childdoc}{\childdocmain}
%    \end{macrocode}

% \macro{\childdocredirect}
% The deprecated macro |\childdocredirect| is a legacy version
% of |\childdocforward| and |\childdocforwardprefix|:
%    \begin{macrocode}
\newcommand{\childdocredirect}[2][]
{
  \begingroup
    \if?#1?
      \def\childdoctmp{\childdocforward{#2}}
    \else
      \def\childdoctmp{\childdocforwardprefix{#1}{#2}}
    \fi
    \expandafter
  \endgroup
  \childdoctmp
}
%    \end{macrocode}

%\iffalse
%</package>
%\fi
%
\endinput

\childdocof{cdocsamp}
%    \end{macrocode}

%\iffalse
%</samplechap1|samplechap2>
%\fi
%
%\iffalse
%<*samplechap1>
%\fi
% Some text for chapter 1:
%    \begin{macrocode}
\section{one}
some text in chapter one
%    \end{macrocode}

%\iffalse
%</samplechap1>
%\fi
% Some text for chapter 2:
%\iffalse
%<*samplechap2>
%\fi
%    \begin{macrocode}
\section{two}
more text in chapter two
%    \end{macrocode}

%\iffalse
%</samplechap2>
%\fi
%
% %%%%%%%%%%%%%%%%%%%%%%%%%%%%%%%%%%%%%%
% \paragraph{Part Include Files.}
%
% The include files are called |cdocspt3.tex| and |cdocspt4.tex|.
%
%\iffalse
%<*samplepart3|samplepart4>
%\fi

% Optional override for |\version| flag:
%    \begin{macrocode}
%%\providecommand{\version}{final}
%    \end{macrocode}

% Include the main document:
%    \begin{macrocode}
% \iffalse
%
% childdoc.dtx Copyright (C) 2017-2018 Niklas Beisert
%
% This work may be distributed and/or modified under the
% conditions of the LaTeX Project Public License, either version 1.3
% of this license or (at your option) any later version.
% The latest version of this license is in
%   http://www.latex-project.org/lppl.txt
% and version 1.3 or later is part of all distributions of LaTeX
% version 2005/12/01 or later.
%
% This work has the LPPL maintenance status `maintained'.
%
% The Current Maintainer of this work is Niklas Beisert.
%
% This work consists of the files childdoc.dtx and childdoc.ins
% and the derived files childdoc.def and cdocsamp.tex with
% cdocsch1.tex, cdocsch2.tex, cdocsdrf.tex, cdocsfn1.tex, cdocsfn2.tex.
%
%<package>\ifdefined\childdocmain\endinput\fi
%<package>\ProvidesFile{childdoc.def}[2018/12/30 v2.0 child document driver]
%<samplemain>\ProvidesFile{cdocsamp.tex}[2018/12/30 v2.0 sample for childdoc]
%<*driver>
%\ProvidesFile{childdoc.drv}[2018/12/30 v2.0 childdoc reference manual file]
\PassOptionsToClass{10pt,a4paper}{article}
\documentclass{ltxdoc}

\usepackage[margin=35mm]{geometry}
\usepackage{hyperref}
\usepackage{hyperxmp}
\usepackage[usenames]{color}

\hypersetup{colorlinks=true}
\hypersetup{pdfstartview=FitH}
\hypersetup{pdfpagemode=UseNone}
\hypersetup{pdfsource={}}
\hypersetup{pdflang={en-UK}}
\hypersetup{pdfcopyright={Copyright 2017-2018 Niklas Beisert.
  This work may be distributed and/or modified under the
  conditions of the LaTeX Project Public License, either version 1.3
  of this license or (at your option) any later version.}}
\hypersetup{pdflicenseurl={http://www.latex-project.org/lppl.txt}}
\hypersetup{pdfcontactaddress={ETH Zurich, ITP, HIT K,
  Wolfgang-Pauli-Strasse 27}}
\hypersetup{pdfcontactpostcode={8093}}
\hypersetup{pdfcontactcity={Zurich}}
\hypersetup{pdfcontactcountry={Switzerland}}
\hypersetup{pdfcontactemail={nbeisert@itp.phys.ethz.ch}}
\hypersetup{pdfcontacturl={http://people.phys.ethz.ch/\xmptilde nbeisert/}}

\newcommand{\secref}[1]{\hyperref[#1]{section \ref*{#1}}}

\parskip1ex
\parindent0pt
\let\olditemize\itemize
\def\itemize{\olditemize\parskip0pt}

\begin{document}

\title{The \textsf{childdoc} Package}
\hypersetup{pdftitle={The childdoc Package}}
\author{Niklas Beisert\\[2ex]
  Institut f\"ur Theoretische Physik\\
  Eidgen\"ossische Technische Hochschule Z\"urich\\
  Wolfgang-Pauli-Strasse 27, 8093 Z\"urich, Switzerland\\[1ex]
  \href{mailto:nbeisert@itp.phys.ethz.ch}
  {\texttt{nbeisert@itp.phys.ethz.ch}}}
\hypersetup{pdfauthor={Niklas Beisert}}
\hypersetup{pdfsubject={Manual for the LaTeX2e Package childdoc}}
\date{30 December 2018, \textsf{v2.0}}
\maketitle

\begin{abstract}\noindent
\textsf{childdoc} is a \LaTeXe{} package
that enables the direct compilation
of document sections included by |\include|
to individual files.
\end{abstract}

\begingroup
\parskip0ex
\tableofcontents
\endgroup

%%%%%%%%%%%%%%%%%%%%%%%%%%%%%%%%%%%%%%%%%%%%%%%%%%%%%%%%%%%%%%%%%%%%%%%%%%%%%%%%
%%%%%%%%%%%%%%%%%%%%%%%%%%%%%%%%%%%%%%%%%%%%%%%%%%%%%%%%%%%%%%%%%%%%%%%%%%%%%%%%
\section{Introduction}

\LaTeX{} provides a mechanism to structure a large document (such as a book)
into a main file and several child files (containing the chapters)
using the |\include| command.
This mechanism is beneficial for documents
which span hundreds of pages in order to
make the source file(s) more manageable.
Moreover, compilation can be restricted to
selected child files by means of the |\includeonly| command.
The latter feature can be used to reduce the compilation time while editing
(this was significantly more useful in the earlier days of \LaTeX{})
or to generate a smaller document which is easier to navigate.
Another application of |\includeonly| is to generate
documents consisting of selected parts of the complete document.

However, there are a few drawbacks of the plain |\include| mechanism:
\begin{itemize}
\item
The child files cannot be compiled on their own,
they can only be compiled via the main file.
A naive editing environment
(such as a text editor with an option
to have the current file processed by \LaTeX)
may require one to switch to the main file before compiling;
attempting to compile the child file produces errors.
\item
The main file must be modified (each time)
to adjust the |\includeonly| command
to the present needs. This easily leaves the main file in a messy state.
\item
The generated document will always carry the filename
of the main document. This is inconvenient if
several child files are to be compiled and
to be kept for distribution.
\end{itemize}

The present package provides a simple interface
to make child files individually compilable by \LaTeX{}.
Compiling a child file then has the same effect as compiling
the main file with an |\includeonly| command
to select the appropriate child.
Moreover the generated document will carry the name of the child
rather than the main file.
This resolves all three above issues.

This feature is meant to make the editing of books,
thesis documents and lecture notes somewhat more convenient.
However, the package can also be used efficiently for
composing a series of documents (such as exercise sheets)
which are typically distributed individually.
It then assists the author in generating the individual documents
(potentially in different versions)
as well as a document containing the collected series.
Another application is in developing style files
or other kinds of included material
where compilation of the style file could redirect
to a sample or test file.

%%%%%%%%%%%%%%%%%%%%%%%%%%%%%%%%%%%%%%%%%%%%%%%%%%%%%%%%%%%%%%%%%%%%%%%%%%%%%%%%
%%%%%%%%%%%%%%%%%%%%%%%%%%%%%%%%%%%%%%%%%%%%%%%%%%%%%%%%%%%%%%%%%%%%%%%%%%%%%%%%
\section{Usage}

First of all, the package \textsf{childdoc} is \emph{not} a standard
\LaTeXe{} |.sty| style file! Therefore it needs to be invoked in
a non-standard way.

%%%%%%%%%%%%%%%%%%%%%%%%%%%%%%%%%%%%%%%%%%%%%%%%%%%%%%%%%%%%%%%%%%%%%%%%%%%%%%%%
\subsection{Included Files}
\label{sec:include}

%%%%%%%%%%%%%%%%%%%%%%%%%%%%%%%%%%%%%%%%
\DescribeMacro{\childdocmain}
To use the package, add the commands
\begin{center}
\begin{tabular}{l}
|\input{childdoc.def}|\\
|\childdocmain{}|\\
\end{tabular}
\end{center}
at the very top of the main \LaTeX{} file,
in particular \emph{before} the |\documentclass| statement!
The argument of |\childdocmain| should be left empty
(but it must be present).

%%%%%%%%%%%%%%%%%%%%%%%%%%%%%%%%%%%%%%%%
\DescribeMacro{\childdocof}
Furthermore, add the commands
\begin{center}
\begin{tabular}{l}
|\input{childdoc.def}|\\
|\childdocof{|\textit{main}|}|\\
\end{tabular}
\end{center}
at the top of every child file \textit{child}
which is included by |\include{|\textit{child}|}|
from within the main file
(or at least for those files to be compiled individually).
The argument \textit{main} must be the filename of the main file.

There are a couple of
considerations in setting up the main and child documents:

%%%%%%%%%%%%%%%%%%%%%%%%%%%%%%%%%%%%%%%%
\paragraph{Restrictions.}

Please note the following restrictions:
\begin{itemize}
\item
|\childdocmain| must be called with one argument \textit{main}
to ensure compatibility with earlier version of the package.
It must either be empty (|\childdocmain{}|)
or precisely match the filename of the main file in which it is specified.
See \secref{sec:detection} for further information.
\item
The filename \textit{main} must be specified without the |.tex| extension.
\item
The filename \textit{main} is case sensitive
(even in case-insensitive file systems)
due to internal string comparison.
\item
The argument \textit{main} should be fully expanded, it cannot be a macro.
\item
Subdirectories and special characters should be avoided in filenames.
\item
The command |\childdocmain{|\textit{main}|}| must be followed by a whitespace.
It should not be followed immediately by another command
or by a comment mark `|%|'.
This is because the \TeX{} parser reads the token immediately following
the argument of |\childdocmain| and puts it
at the beginning of every child section;
however, a white\-space is ignored.
\end{itemize}

%%%%%%%%%%%%%%%%%%%%%%%%%%%%%%%%%%%%%%%%
\paragraph{Content of Main File.}

It is advisable to place all content in the child files included by |\include|.
Any output contained in the main file will appear in all child documents
unless suppressed manually;
it cannot be suppressed automatically by the |\includeonly| directive
and thus should normally be avoided.
A method to include some content in the main file
by means of conditional processing is described in \secref{sec:conditional}.

%%%%%%%%%%%%%%%%%%%%%%%%%%%%%%%%%%%%%%%%
\paragraph{Page Numbering.}

When only a part of the document is compiled,
the appropriate numbering of pages
(as well as other status parameters)
is determined from the |.aux| files.
The latter contain information from previous passes.
However this information needs to propagate through
all intermediate child documents.
Therefore the page numbering in child documents may well
be inconsistent until the complete document is compiled at least once.

A useful (if unconventional) way to always ensure a consistent
page numbering is to restart the numbering in each child document
and denote the pages by `\textit{child}|.|\textit{page}'
where \textit{child} represents the chapter/section number of the child file.
This can be achieved by the command
|\numberwithin{page}{|\textit{child}|}|
of the \textsf{amsmath} package
where \textit{child} can be |chapter| or |section|
depending on the chosen structuring.
Alternatively, one can modify the macro |\thepage| appropriately
and reset the counter |page| at the start of each child file.

%%%%%%%%%%%%%%%%%%%%%%%%%%%%%%%%%%%%%%%%%%%%%%%%%%%%%%%%%%%%%%%%%%%%%%%%%%%%%%%%
\subsection{Conditional Processing}
\label{sec:conditional}

The package provides a mechanism to compile different versions
of a document. To customise the versions further some conditional processing
can come in handy to distinguish which version is being compiled.
The package provides two macros to describe the compilation context:

%%%%%%%%%%%%%%%%%%%%%%%%%%%%%%%%%%%%%%%%
\DescribeMacro{\ifchilddoc}
The conditional |\ifchilddoc| distinguishes between the compilation of
child documents and the main document:
%
\begin{center}
|\ifchilddoc |\textit{child-code}| |[|\||else |\textit{main-code}]| \||fi|
\end{center}

%%%%%%%%%%%%%%%%%%%%%%%%%%%%%%%%%%%%%%%%
\DescribeMacro{\childdocname}
\DescribeMacro{\childdocjob}
The macro |\childdocname| contains the filename (without extension)
of the main or child file being processed.
Note that |\childdocjob| will always contain the name of the main file.

%%%%%%%%%%%%%%%%%%%%%%%%%%%%%%%%%%%%%%%%
\paragraph{Title Page.}

Conditional processing can be used to include a title or banner page
in the main document when proper precautions are taken.
Importantly, the code in the main file should ensure that the page counter
(as well as other status parameters which are stored in the |.aux| files)
takes the same value after the conditional processing.
Otherwise the page numbers may take divergent values
depending on which part is compiled.

For example, a title page could be declared by:
%
\begin{center}
\begin{tabular}{l}
|\ifchilddoc\||else|\\
|\addtocounter{page}{-1}|\\
\textit{code for title page}\\
|\newpage|\\
|\||fi|
\end{tabular}
\end{center}
%
A banner page for the child documents can be generated by:
%
\begin{center}
\begin{tabular}{l}
|\ifchilddoc|\\
|\addtocounter{page}{-1}|\\
\textit{code for banner page}\\
|\newpage|\\
|\||fi|
\end{tabular}
\end{center}
%
Here one could write a message such as:
\begin{center}
|This is the part \childdocname{} of \childdocjob{}.|
\end{center}

%%%%%%%%%%%%%%%%%%%%%%%%%%%%%%%%%%%%%%%%%%%%%%%%%%%%%%%%%%%%%%%%%%%%%%%%%%%%%%%%
\subsection{Flags}
\label{sec:flags}

The package makes it easy to generate different versions
of the main or child documents.
To this end compilation flags can be defined
and assigned different default values.
They will be particularly useful in conjunction
with the forwarding mechanism described in \secref{sec:forward}.

For example, it may be useful to have a flag |\version|
which can be set to |draft| or |final|.
The document source will contain some conditional code
depending on the value of |\version|.
Suppose further, the flag should default to |final| for the main file
and to |draft| for child files
which is a natural assignment for editing the document.
This is achieved by placing the following code
in the preamble of the main document
(below the |\childdocmain| directive):
%
\begin{center}
\begin{tabular}{l}
|\ifchilddoc|\\
|\providecommand{\version}{draft}|\\
|\||else|\\
|\providecommand{\version}{final}|\\
|\||fi|
\end{tabular}
\end{center}
%
The definition by |\providecommand| makes sure
that previous definitions are not overwritten.
Further statements |\providecommand{\version}{...}|
can thus be added before the above code to override it.

For the main file, one might add a line
(between |\childdocmain| and the above block)
%
\begin{center}
|%\ifchilddoc\||else\providecommand{\version}{draft}\||fi|
\end{center}
%
which can be uncommented to produce a draft version.
Likewise one can add a line to the very top of a child file
(above the |\childdocof{|\textit{main}|}| directive)
%
\begin{center}
|%\providecommand{\version}{final}|
\end{center}
%
which can be uncommented to produce the final version of this child document.

%%%%%%%%%%%%%%%%%%%%%%%%%%%%%%%%%%%%%%%%%%%%%%%%%%%%%%%%%%%%%%%%%%%%%%%%%%%%%%%%
\subsection{Forwarding}
\label{sec:forward}

Different versions of the main or child documents
using compilation flags as described in \secref{sec:flags}
can be (permanently) stored in different files
for convenient compilation, viewing and distribution.
To this end, the package defines a command
to pass on compilation to a different file:

%%%%%%%%%%%%%%%%%%%%%%%%%%%%%%%%%%%%%%%%
\DescribeMacro{\childdocforward}
The command |\childdocforward| redirects processing to
another source file:
%
\begin{center}
\begin{tabular}{l}
|\input{childdoc.def}|\\
|\childdocforward[|\textit{main}|]{|\textit{dest}|}|\\
\end{tabular}
\end{center}
%
The argument \textit{dest} is the destination file
(without extension).
It should be the main file or one of the child files.
Note that further \textsf{childdoc} directives
such as |\childdocof| and |\childdocforward|
in the indicated file will be processed in this form.
The optional argument \textit{main}
passes on directly to the main file \textit{main}
while pretending to compile the child \textit{dest}.
This form behaves as if \textit{dest}
issues |\childdocof{|\textit{main}|}| right away,
and no further \textsf{childdoc} directives will be processed.

%%%%%%%%%%%%%%%%%%%%%%%%%%%%%%%%%%%%%%%%
\DescribeMacro{\...prefix}
In the alternative form |\childdocforwardprefix|,
%
\begin{center}
\begin{tabular}{l}
|\input{childdoc.def}|\\
|\childdocforwardprefix[|\textit{main}|]{|\textit{prefix}|}{|\textit{dest}|}|
\end{tabular}
\end{center}
%
the destination file is determined by a pattern
depending on the current file:
To make this work, the current file must be called
`{\textit{prefix}\hspace{0.2em}\textit{suffix}}'
with \textit{prefix} matching precisely the argument.
Processing is then passed on to the file
`{\textit{dest}\hspace{0.2em}\textit{suffix}}'.
Surely, the same effect is achieved by
directly specifying the
argument `{\textit{dest}\hspace{0.2em}\textit{suffix}}'
in the first form.
However, that requires to set up a different file
for each child. With the alternative form of the command
all these files can have exactly the same content
which simplifies setting them up and maintaining them.

For example, the following file |draft.tex|
with a compilation flag |\version| as described in \secref{sec:flags}
compiles the main document as a draft:
%
\begin{center}
\begin{tabular}{l}
|\def\version{draft}|\\
|\input{childdoc.def}|\\
|\childdocforward{|\textit{main}|}|
\end{tabular}
\end{center}
%
Likewise, the following files |final|\textit{nn}|.tex|
compile the final version of the child document
|child|\textit{nn}|.tex|:
%
\begin{center}
\begin{tabular}{l}
|\def\version{final}|\\
|\input{childdoc.def}|\\
|\childdocforwardprefix{final}{child}|
\end{tabular}
\end{center}
%

Note that when several versions of a main file and/or of each child file
are to be generated, it may be convenient to set up a |Makefile| or
shell script to automatise the process.

%%%%%%%%%%%%%%%%%%%%%%%%%%%%%%%%%%%%%%%%%%%%%%%%%%%%%%%%%%%%%%%%%%%%%%%%%%%%%%%%
\subsection{Command Line Processing}
\label{sec:commandline}

The effect of redirection files can also be achieved by invoking
the \LaTeX{} compiler with a more elaborate command line.
Most conveniently this should be done as part
of a shell script or a |Makefile|.

When using \textsf{childdoc} in the main file, the following
command lines effectively perform a redirection
(note that depending on the shell being used,
backslashes may have to be doubled: `|\|' $\to$ `|\\|'):
%
\begin{center}
|... -jobname "|\textit{target}|" |\\|"|[\textit{flags}]%
|\input{childdoc.def}\childdocforward[|\textit{main}|]{|\textit{dest}|}"|
\end{center}
%
Here \textit{target} is the name of the output file,
\textit{main} is the name of the main file
and \textit{dest} is the name of the main or child file to be processed
(all filenames without extensions).
The optional argument \textit{main} can be omitted
if \textit{main} matches \textit{dest}.
Optionally, compilation \textit{flags} can be defined via |\def| commands.
This command line makes the \TeX{} engine believe
it is compiling the file \textit{target}
whose content is specified as the latter parameter.
The provided code then forwards the processing to
\textit{main} or \textit{dest} as described in \secref{sec:forward}.

%%%%%%%%%%%%%%%%%%%%%%%%%%%%%%%%%%%%%%%%%%%%%%%%%%%%%%%%%%%%%%%%%%%%%%%%%%%%%%%%
\subsection{Include by Input}
\label{sec:input}

Including child documents by |\include| has some restrictions by design.
Most notably, the content of a child document always occupies
its own set of pages; pages cannot be shared between child documents.
Usually, this behaviour makes perfect sense
because each child document contain an essential part of the document.
However, in some situations it may be desirable to compose
a document from a collection of parts
without having mandatory page breaks between then.
For this case, the package
provides a mechanism to include parts
by |\input| which can also be processed individually.
However, by construction this mechanism
requires manual handling of the content to be output.

%%%%%%%%%%%%%%%%%%%%%%%%%%%%%%%%%%%%%%%%
\DescribeMacro{\ifchilddocmanual}
The main file should be prepared as usual, see \secref{sec:include}.
However, the document body must make a distinction
between processing of an individual part and of the main document, e.g.:
%
\begin{center}
\begin{tabular}{l}
|\ifchilddocmanual|\\
|\input{\childdocname}|\\
|\||else|\\
\textit{document body with }|\input{|\textit{part}|}|\\
|\||fi|
\end{tabular}
\end{center}
%
The conditional |\ifchilddocmanual| is true whenever
a part to be included by |\input| is being compiled,
and the name of the part is stored in |\childdocname|.

%%%%%%%%%%%%%%%%%%%%%%%%%%%%%%%%%%%%%%%%
\DescribeMacro{\childdocby}
Each part to be included by |\input| should start with:
%
\begin{center}
\begin{tabular}{l}
|\input{childdoc.def}|\\
|\childdocby{|\textit{main}|}|\\
\end{tabular}
\end{center}
%
The directive |\childdocby| is similar to |\childdocof|
described in \secref{sec:include},
but the subsequent selection of content must be done manually.
To that end, both |\ifchilddoc| and |\ifchilddocmanual|
will be true upon processing of a part,
and the name of the part is stored in |\childdocname|.
Note that |\jobname| will be set to the filename of the current part
so that each part receives an individual |.aux| file
that does not interfere with the |.aux| file(s) of the main document.
This behaviour can be altered by the alternative form
|\childdocby[*]{|\textit{main}|}| (with a non-empty optional argument)
which uses the |.aux| file of the main document
by setting |\jobname| to \textit{main}.

%%%%%%%%%%%%%%%%%%%%%%%%%%%%%%%%%%%%%%%%%%%%%%%%%%%%%%%%%%%%%%%%%%%%%%%%%%%%%%%%
\subsection{Driver Development}
\label{sec:driver}

The \textsf{childdoc} mechanism can also be use for the development
of definition files such as \LaTeX{} styles or classes.
This case differs from the above setup with multiple parts
included by |\include| in that no |\includeonly| should be invoked.
This can be achieved by starting the include file
(before |\ProvidesPackage|) with:
%
\begin{center}
\begin{tabular}{l}
|\input{childdoc.def}|\\
|\childdocforward{|\textit{main}|}|\\
\end{tabular}
\end{center}
%
or alternatively with:
%
\begin{center}
\begin{tabular}{l}
|\input{childdoc.def}|\\
|\childdocby{|\textit{main}|}|\\
\end{tabular}
\end{center}
%
Both forms have slightly different effects as described above.
The main file is prepared as usual, see \secref{sec:include}.

%%%%%%%%%%%%%%%%%%%%%%%%%%%%%%%%%%%%%%%%%%%%%%%%%%%%%%%%%%%%%%%%%%%%%%%%%%%%%%%%
\subsection{Legacy Detection}
\label{sec:detection}

The directive |\childdocmain| in the main file can detect
whether the complete document or merely a child is to be compiled
even without using the directive |\childdocof|.
This method is deprecated because it is less robust
and there is no compelling reason to use it;
it is merely provided for backward compatibility
and it may be removed in future versions.

If the detection mechanism is to be used,
it is mandatory to correctly specify
the filename of the main file as the argument of |\childdocmain|:
%
\begin{center}
\begin{tabular}{l}
|\input{childdoc.def}|\\
|\childdocmain{|\textit{main}|}|\\
\end{tabular}
\end{center}
%
If |\jobname| does not match the argument \textit{main} of |\childdocmain|,
it is assumed that |\jobname| points to the child file to be compiled.
When using |\childdocmain| with the main file specified as argument,
it suffices to start a child file
with just |\input{|\textit{main}|}|
without loading of the package and using |\childdocof|.
If instead all processing is done
with the appropriate \textsf{childdoc} directives,
the argument of \textit{main} of |\childdocmain| can be empty.

An alternative version of the command line processing described
in \secref{sec:commandline} using the detection mechanism reads:
%
\begin{center}
|... -jobname "|\textit{target}|" "|[\textit{flags}]%
[|\def\jobname{|\textit{dest}|}|]|\input{|\textit{main}|}"|
\end{center}

%%%%%%%%%%%%%%%%%%%%%%%%%%%%%%%%%%%%%%%%%%%%%%%%%%%%%%%%%%%%%%%%%%%%%%%%%%%%%%%%
\subsection{Manual Code}
\label{sec:manual}

In case one cannot be certain whether the definitions file |childdoc.def|
is installed on the target \TeX{} distribution
and one prefers not to ship it,
it is conceivable to paste a few relevant commands into the sources.

To that end, drop all statements |\input{childdoc.def}|
and perform the replacements as outlined below.
Instead of |\childdocmain{|\textit{main}|}| add the following code
to the top of the main file:
%
\begin{center}
\begin{tabular}{l}
|\||ifdefined\childdocname\endinput\||fi\newif\ifchilddoc|\\
|\edef\childdocname{\scantokens\expandafter{\jobname\noexpand}}|\\
|\def\childdocmain{|\textit{main}|}\||ifx\childdocmain\childdocname\||else|\\
|\childdoctrue\includeonly{\childdocname}\let\jobname\childdocmain\||fi|\\
\end{tabular}
\end{center}
%
Instead of |\childdocof{|\textit{main}|}| just include the main file
at the top of each child file:
%
\begin{center}
|\input{|\textit{main}|}|
\end{center}
%
A simple redirection |\childdocforward{|\textit{dest}|}| is achieved by:
%
\begin{center}
|\def\jobname{|\textit{dest}|}\input{\jobname}|
\end{center}
%
The redirection with prefix
|\childdocforwardprefix[|\textit{prefix}|]{|\textit{dest}|}|
is accomplished by:
%
\begin{center}
\begin{tabular}{l}
|{\edef\jobname{\scantokens\expandafter{\jobname\noexpand}}|\\
|\def\redirectjob |\textit{prefix}|#1~~~{\gdef\jobname{|\textit{dest}|#1}}|\\
|\expandafter\redirectjob\jobname~~~}\input{\jobname}|
\end{tabular}
\end{center}

In an alternative approach,
child documents can be compiled by a specific command line
without additional code or specific definitions:
%
\begin{center}
|... -jobname "|\textit{target}|" "|[\textit{flags}]%
|\includeonly{|\textit{dest}|}\input{|\textit{main}|}"|
\end{center}
%

%%%%%%%%%%%%%%%%%%%%%%%%%%%%%%%%%%%%%%%%%%%%%%%%%%%%%%%%%%%%%%%%%%%%%%%%%%%%%%%%
%%%%%%%%%%%%%%%%%%%%%%%%%%%%%%%%%%%%%%%%%%%%%%%%%%%%%%%%%%%%%%%%%%%%%%%%%%%%%%%%
\section{Information}

%%%%%%%%%%%%%%%%%%%%%%%%%%%%%%%%%%%%%%%%%%%%%%%%%%%%%%%%%%%%%%%%%%%%%%%%%%%%%%%%
\subsection{Copyright}

Copyright \copyright{} 2017--2018 Niklas Beisert

This work may be distributed and/or modified under the
conditions of the \LaTeX{} Project Public License, either version 1.3
of this license or (at your option) any later version.
The latest version of this license is in
  \url{http://www.latex-project.org/lppl.txt}
and version 1.3 or later is part of all distributions of \LaTeX{}
version 2005/12/01 or later.

This work has the LPPL maintenance status `maintained'.

The Current Maintainer of this work is Niklas Beisert.

This work consists of the files |README.txt|, |childdoc.ins| and |childdoc.dtx|
as well as the derived files |childdoc.def|, |cdocsamp.tex|
with |cdocsch1.tex|, |cdocsch2.tex|, |cdocspt3.tex|, |cdocspt4.tex|,
|cdocsdrf.tex|, |cdocsfn1.tex|, |cdocsfn2.tex|
as well as |childdoc.pdf|.

%%%%%%%%%%%%%%%%%%%%%%%%%%%%%%%%%%%%%%%%%%%%%%%%%%%%%%%%%%%%%%%%%%%%%%%%%%%%%%%%
\subsection{Files and Installation}

The package consists of the files:
%
\begin{center}
\begin{tabular}{ll}
    |README.txt|   & readme file \\
    |childdoc.ins| & installation file \\
    |childdoc.dtx| & source file \\
    |childdoc.def| & definition file \\
    |cdocsamp.tex| & sample main file \\
    |cdocsch1.tex| & sample include file \\
    |cdocsch2.tex| & sample include file \\
    |cdocspt3.tex| & sample part file \\
    |cdocspt4.tex| & sample part file \\
    |cdocsdrf.tex| & sample redirection file \\
    |cdocsfn1.tex| & sample redirection file \\
    |cdocsfn2.tex| & sample redirection file \\
    |childdoc.pdf| & manual
\end{tabular}
\end{center}
%
The distribution consists of the files
|README.txt|, |childdoc.ins| and |childdoc.dtx|.
%
\begin{itemize}
\item
Run (pdf)\LaTeX{} on |childdoc.dtx|
to compile the manual |childdoc.pdf| (this file).
\item
Run \LaTeX{} on |childdoc.ins| to create the definitions file |childdoc.def|
and the sample |cdocsamp.tex| with include files
|cdocsch1.tex|, |cdocsch2.tex|, |cdocspt3.tex|, |cdocspt4.tex|,
|cdocsdrf.tex|, |cdocsfn1.tex|, |cdocsfn2.tex|.
Then copy the file |childdoc.def| to an appropriate directory of your \LaTeX{}
distribution, e.g.\ \textit{texmf-root}|/tex/latex/childdoc|.
\end{itemize}

%%%%%%%%%%%%%%%%%%%%%%%%%%%%%%%%%%%%%%%%%%%%%%%%%%%%%%%%%%%%%%%%%%%%%%%%%%%%%%%%
\subsection{Related CTAN Packages}

There are several other packages which offer a similar functionality:
%
\begin{itemize}
\item
The packages
\href{http://ctan.org/pkg/docmute}{\textsf{docmute}},
\href{http://ctan.org/pkg/includex}{\textsf{includex}} and
\href{http://ctan.org/pkg/standalone}{\textsf{standalone}}
provide commands to include only the document body of
a child file thus allowing both files to be compiled individually.
\item
The packages \href{http://ctan.org/pkg/subdocs}{\textsf{subdocs}}
and \href{http://ctan.org/pkg/subfiles}{\textsf{subfiles}}
provide structures in which the main and child documents can be
encapsulated and allowing them to be compiled individually.
The inclusion mechanism is different from the conventional |\include|.
\item
The package \href{http://ctan.org/pkg/combine}{\textsf{combine}}
is an elaborate solution to combine several documents into one.
\end{itemize}
%
See also the CTAN topic \href{http://ctan.org/topic/subdocs}{\textsf{subdocs}}
for further related packages.
The present package differs from the above solutions in that
a document structure constructed with the conventional |\include| mechanism
just needs two extra commands at the top of every file
such that all constituent files can be compiled individually.

%%%%%%%%%%%%%%%%%%%%%%%%%%%%%%%%%%%%%%%%%%%%%%%%%%%%%%%%%%%%%%%%%%%%%%%%%%%%%%%%
%\subsection{Feature Suggestions}
%
%The following is a list of features which may be useful for future
%versions of this package:
%%
%\begin{itemize}
%\item
%\ldots
%\end{itemize}

%%%%%%%%%%%%%%%%%%%%%%%%%%%%%%%%%%%%%%%%%%%%%%%%%%%%%%%%%%%%%%%%%%%%%%%%%%%%%%%%
\subsection{Revision History}

%%%%%%%%%%%%%%%%%%%%%%%%%%%%%%%%%%%%%%%%
\paragraph{v2.0:} 2018/12/30

\begin{itemize}
\item
immediate forward processing
\item
added |\childdocby| mechanism
\item
manual restructured
\end{itemize}

%%%%%%%%%%%%%%%%%%%%%%%%%%%%%%%%%%%%%%%%
\paragraph{v1.6:} 2018/01/17

\begin{itemize}
\item
application for development of include files
\item
corrections to manual
\end{itemize}

%%%%%%%%%%%%%%%%%%%%%%%%%%%%%%%%%%%%%%%%
\paragraph{v1.5:} 2017/05/21

\begin{itemize}
\item
more complete structuring introduced
\item
|\childdocof| introduced
\item
|\childdoc| renamed to |\childdocmain|
\item
|\childredirect| renamed to |\childdocforward| and |\childdocforwardprefix|
and functionality expanded
\end{itemize}

%%%%%%%%%%%%%%%%%%%%%%%%%%%%%%%%%%%%%%%%
\paragraph{v1.0:} 2017/04/27

\begin{itemize}
\item
manual and install package
\item
first version published on CTAN
\end{itemize}

%%%%%%%%%%%%%%%%%%%%%%%%%%%%%%%%%%%%%%%%
\paragraph{v0.6:} 2017/04/26

\begin{itemize}
\item
redirection mechanism added
\end{itemize}

%%%%%%%%%%%%%%%%%%%%%%%%%%%%%%%%%%%%%%%%
\paragraph{v0.5:} 2017/04/26

\begin{itemize}
\item
functionality in definition file
\end{itemize}


%%%%%%%%%%%%%%%%%%%%%%%%%%%%%%%%%%%%%%%%%%%%%%%%%%%%%%%%%%%%%%%%%%%%%%%%%%%%%%%%
%%%%%%%%%%%%%%%%%%%%%%%%%%%%%%%%%%%%%%%%%%%%%%%%%%%%%%%%%%%%%%%%%%%%%%%%%%%%%%%%
%%%%%%%%%%%%%%%%%%%%%%%%%%%%%%%%%%%%%%%%%%%%%%%%%%%%%%%%%%%%%%%%%%%%%%%%%%%%%%%%
\appendix

\settowidth\MacroIndent{\rmfamily\scriptsize 000\ }

 \DocInput{childdoc.dtx}

\end{document}
%</driver>
% \fi
%
% %%%%%%%%%%%%%%%%%%%%%%%%%%%%%%%%%%%%%%%%%%%%%%%%%%%%%%%%%%%%%%%%%%%%%%%%%%%%%%
% %%%%%%%%%%%%%%%%%%%%%%%%%%%%%%%%%%%%%%%%%%%%%%%%%%%%%%%%%%%%%%%%%%%%%%%%%%%%%%
% \section{Sample}
%\iffalse
%<*samplemain>
%\fi
%
% The following presents a sample document
% with two chapters, two parts, a title page,
% a compile flag as well as three forwarding files to set the flag.
% It consists of eight |.tex| files:
% \begin{center}
% \begin{tabular}{ll}
% |cdocsamp.tex|&main file\\
% |cdocsch1.tex|&include file for chapter 1\\
% |cdocsch2.tex|&include file for chapter 2\\
% |cdocspt3.tex|&include file for part 3\\
% |cdocspt4.tex|&include file for part 4\\
% |cdocsdrf.tex|&forwarding file for main file in draft mode\\
% |cdocsfi1.tex|&forwarding file for final version of chapter 1\\
% |cdocsfi2.tex|&forwarding file for final version of chapter 2\\
% \end{tabular}
% \end{center}
% Each of the eight files can be compiled directly by the \LaTeX{} compiler.
%
% %%%%%%%%%%%%%%%%%%%%%%%%%%%%%%%%%%%%%%
% \paragraph{Main File.}
%
% The main file is called |cdocsamp.tex|.
%
% Load the \textsf{childdoc} definitions and
% declare the filename for the main document:
%    \begin{macrocode}
\input{childdoc.def}
\childdocmain{}
%    \end{macrocode}

% Optional override for |\version| flag:
%    \begin{macrocode}
%%\ifchilddoc\else\providecommand{\version}{draft}\fi
%    \end{macrocode}

% Define the default values for the |\version| flag
% (|final| for the main file and |draft| for childs):
%    \begin{macrocode}
\ifchilddoc
\providecommand{\version}{draft}
\else
\providecommand{\version}{final}
\fi
%    \end{macrocode}

% Load the standard document class:
%    \begin{macrocode}
\documentclass[12pt]{article}
%    \end{macrocode}

% Start the document body:
%    \begin{macrocode}
\begin{document}
%    \end{macrocode}

% Declare a title page.
% Print title, part of document being processed and version flag:
%    \begin{macrocode}
\addtocounter{page}{-1}
\begin{center}
{\LARGE\bfseries{}childdoc example\par}
\vspace{1cm}
\ifchilddoc
\ifchilddocmanual part\else chapter\fi:
`\childdocname' of `\childdocjob'\par
\else
main document: `\childdocjob'\par
\fi
version: \version\par
\end{center}
\newpage
%    \end{macrocode}

% Manually include selected file,
% otherwise process as usual:
%    \begin{macrocode}
\ifchilddocmanual
\section*{part `\childdocname'}
\input{\childdocname}
\else
%    \end{macrocode}

% Include the two chapters:
%    \begin{macrocode}
\include{cdocsch1}
\include{cdocsch2}
%    \end{macrocode}

% Include the two parts unless only chapters should be displayed:
%    \begin{macrocode}
\ifchilddoc\else
\section{part three}
\input{cdocspt3}
\section{part four}
\input{cdocspt4}
\fi
%    \end{macrocode}

% Process as usual until here:
%    \begin{macrocode}
\fi
%    \end{macrocode}

% End of document body:
%    \begin{macrocode}
\end{document}
%    \end{macrocode}
%\iffalse
%</samplemain>
%\fi
%
% %%%%%%%%%%%%%%%%%%%%%%%%%%%%%%%%%%%%%%
% \paragraph{Chapter Include Files.}
%
% The include files are called |cdocsch1.tex| and |cdocsch2.tex|.
%
%\iffalse
%<*samplechap1|samplechap2>
%\fi

% Optional override for |\version| flag:
%    \begin{macrocode}
%%\providecommand{\version}{final}
%    \end{macrocode}

% Include the main document:
%    \begin{macrocode}
\input{childdoc.def}
\childdocof{cdocsamp}
%    \end{macrocode}

%\iffalse
%</samplechap1|samplechap2>
%\fi
%
%\iffalse
%<*samplechap1>
%\fi
% Some text for chapter 1:
%    \begin{macrocode}
\section{one}
some text in chapter one
%    \end{macrocode}

%\iffalse
%</samplechap1>
%\fi
% Some text for chapter 2:
%\iffalse
%<*samplechap2>
%\fi
%    \begin{macrocode}
\section{two}
more text in chapter two
%    \end{macrocode}

%\iffalse
%</samplechap2>
%\fi
%
% %%%%%%%%%%%%%%%%%%%%%%%%%%%%%%%%%%%%%%
% \paragraph{Part Include Files.}
%
% The include files are called |cdocspt3.tex| and |cdocspt4.tex|.
%
%\iffalse
%<*samplepart3|samplepart4>
%\fi

% Optional override for |\version| flag:
%    \begin{macrocode}
%%\providecommand{\version}{final}
%    \end{macrocode}

% Include the main document:
%    \begin{macrocode}
\input{childdoc.def}
\childdocby{cdocsamp}
%    \end{macrocode}

%\iffalse
%</samplepart3|samplepart4>
%\fi
%
%\iffalse
%<*samplepart3>
%\fi
% Some text for part 3:
%    \begin{macrocode}
some text in part three
%    \end{macrocode}

%\iffalse
%</samplepart3>
%\fi
% Some text for part 4:
%\iffalse
%<*samplepart4>
%\fi
%    \begin{macrocode}
more text in part four
%    \end{macrocode}

%\iffalse
%</samplepart4>
%\fi
%
% %%%%%%%%%%%%%%%%%%%%%%%%%%%%%%%%%%%%%%
% \paragraph{Forwarding for a Complete Draft.}
%
% The following forwarding file |cdocsdrf.tex|
% compiles the main document in draft mode:
%\iffalse
%<*sampledraft>
%\fi
%    \begin{macrocode}
\def\version{draft}
\input{childdoc.def}
\childdocforward{cdocsamp}
%    \end{macrocode}

%\iffalse
%</sampledraft>
%\fi
%
% %%%%%%%%%%%%%%%%%%%%%%%%%%%%%%%%%%%%%%
% \paragraph{Forwarding for Final Version of the Chapters.}
%
% The following forwarding files |cdocsfn1.tex| and |cdocsfn2.tex|
% (with identical content)
% compile the final versions of the child documents
% |cdocsch1.tex| and |cdocsch2.tex|, respectively:
%\iffalse
%<*samplefinal>
%\fi
%    \begin{macrocode}
\def\version{final}
\input{childdoc.def}
\childdocforwardprefix[cdocsamp]{cdocsfn}{cdocsch}
%    \end{macrocode}

%\iffalse
%</samplefinal>
%\fi
%
% %%%%%%%%%%%%%%%%%%%%%%%%%%%%%%%%%%%%%%
% \paragraph{Command Line Processing.}
%
% The following three command lines generate the output files
% |cdocscld|, |cdocscl1| and |cdocscl2|
% which should be identical to
% |cdocsdrf|, |cdocsch1| and |cdocsfn2|, respectively:
% \begin{center}
% \begin{tabular}{l}
% |latex -jobname cdocscld \|\\
% |  "\def\version{draft}\input{childdoc.def}\childdocforward{cdocsamp}"|\\
% |latex -jobname cdocscl1 \|\\
% |  "\input{childdoc.def}\childdocforward[cdocsamp]{cdocsch1}"|\\
% |latex -jobname cdocscl2 \|\\
% |  "\def\version{final}\input{childdoc.def}\childdocforward{cdocsch2}"|
% \end{tabular}
% \end{center}
% Note that the trailing backslash on each first line
% merely continues the input to the second line
% (for convenient cut ant paste).
% Furthermore, the command |latex| can be replaced by any
% of its alternative versions such as |pdflatex|.
%
% %%%%%%%%%%%%%%%%%%%%%%%%%%%%%%%%%%%%%%%%%%%%%%%%%%%%%%%%%%%%%%%%%%%%%%%%%%%%%%
% %%%%%%%%%%%%%%%%%%%%%%%%%%%%%%%%%%%%%%%%%%%%%%%%%%%%%%%%%%%%%%%%%%%%%%%%%%%%%%
% \section{Implementation}
%\iffalse
%<*package>
%\fi
%
% This section describes the definitions file |childdoc.def|.

% The definitions cannot be loaded using |\usepackage| or |\RequirePackage|
% which has a mechanism to prevent loading a style file more than once.
% When loading the definitions by means of |\input|
% multiple instances have to be prevented manually:
%\iffalse
%This code needs to be before the `\ProvidesFile' directive
%which is defined at the beginning of this file.
%Therefore it is also placed there and commented out here.
%</package>
%<*discard>
%\fi
%    \begin{macrocode}
\ifdefined\childdocmain\endinput\fi
%    \end{macrocode}
%\iffalse
%</discard>
%<*package>
%\fi
%
% \macro{\ifchilddoc}
% \macro{\ifchilddocmanual}
% The conditional |\ifchilddoc| tells whether a
% child (true) or main (false) document is being compiled.
% The conditional |\ifchilddocmanual| tells whether
% the |\includeonly| mechanism is used (false) or
% the selection of child files must be performed manually (true).
% The definitions initialise to false:
%    \begin{macrocode}
\newif\ifchilddoc
\newif\ifchilddocmanual
%    \end{macrocode}

% \macro{\childdocname}
% \macro{\childdocjob}
% The macro |\childdocname| stores the name of the main document
% to be compiled. The macro |\childdocjob| stores the name of
% the document on which the \LaTeX{} compiler was originally invoked.
% The content of |\jobname| cannot be compared
% to filenames specified in the source due to different catcodes.
% The following code rescans |\jobname|, stores the result
% in |\childdocname| and saves a copy in |\childdocjob|:
%    \begin{macrocode}
\edef\childdocname{\scantokens\expandafter{\jobname\noexpand}}
\let\childdocjob\childdocname
%    \end{macrocode}

% \macro{\childdocdisable}
% The macro |\childdocdisable| prevents the main file
% from being processed more than once.
% At this stage, the main document command |\childdocmain|
% is assumed to be called once again where it should do nothing.
% Any subsequent call to it should prevent
% a secondary processing of the main document
% It overwrites the forwarding commands
% |\childdocof| and |\childdocforward|
% with empty macros to prevent further inclusions of the main document:
%    \begin{macrocode}
\newcommand{\childdocdisable}
{
  \renewcommand{\childdocmain}[1]{\renewcommand{\childdocmain}[1]{\endinput}}
  \renewcommand{\childdocof}[1]{}
  \renewcommand{\childdocby}[2][]{}
  \renewcommand{\childdocforward}[2][]{}
  \renewcommand{\childdocdisable}{}
}
%    \end{macrocode}

% \macro{\childdocmain}
% The macro |\childdocmain| is to be called at the top of the main file
% with nothing or the main filename (without extension) as argument.
% First, it breaks loops.
% If the argument is not empty and does not match |\childdocname|
% (which is set by the first inclusion of |childdoc.def|),
% |\ifchilddoc| is set to true, |\includeonly| is applied to the child file
% and |\jobname| is set to the main file
% (for proper handling of |.aux| files):
%    \begin{macrocode}
\newcommand{\childdocmain}[1]
{
  \childdocdisable\childdocmain{}
  \if?#1?\else
    \begingroup
      \def\childdoctmp{#1}
      \ifx\childdoctmp\childdocname
        \def\childdoctmp{}
      \else
        \def\childdoctmp
        {
          \childdoctrue
          \includeonly{\childdocname}
          \def\childdocjob{#1}
          \def\jobname{#1}
        }
      \fi
      \expandafter
    \endgroup
    \childdoctmp
  \fi
}
%    \end{macrocode}

% \macro{\childdocof}
% The command |\childdocof| redirects
% compilation to the main file |#1|.
%    \begin{macrocode}
\newcommand{\childdocof}[1]
{
  \childdocdisable
  \childdoctrue
  \includeonly{\childdocname}
  \def\jobname{#1}
  \def\childdocjob{#1}
  \input{#1}
}
%    \end{macrocode}

% \macro{\childdocby}
% The command |\childdocby| ....
%    \begin{macrocode}
\newcommand{\childdocby}[2][]
{
  \childdocdisable
  \childdoctrue
  \childdocmanualtrue
  \if?#1?\else
    \def\jobname{#2}
  \fi
  \def\childdocjob{#2}
  \input{#2}
  \endinput
}
%    \end{macrocode}

% \macro{\childdocforward}
% The command |\childdocforward| redirects
% compilation to the main file or
% (if the optional argument is given) a child file.
% Parameters are set as if the main file
% or a child file starting with |\childdocof| was compiled.
% Then compilation is handed over to the main file:
%    \begin{macrocode}
\newcommand{\childdocforward}[2][]
{
  \begingroup
    \if?#1?
      \def\childdoctmp
      {
        \def\childdocname{#2}
        \def\childdocjob{#2}
        \def\jobname{#2}
        \input{#2}
        \endinput
      }
    \else
      \def\childdoctmp
      {
        \childdocdisable
        \def\childdocname{#2}
        \childdoctrue
        \includeonly{#2}
        \def\childdocjob{#1}
        \def\jobname{#1}
        \input{#1}
        \endinput
      }
    \fi
    \expandafter
  \endgroup
  \childdoctmp
}
%    \end{macrocode}

% \macro{\childdocforwardprefix}
% The command |\childdocforwardprefix| redirects
% compilation to the main or a child file by means of a pattern.
% The prefix |#1| in the current filename is replaced by |#2|
% and the suffix of the current filename is kept
% (it is assumed that the filename does not contain the substring `|~~~|'
% which is used as a delimiter).
% Compilation is handed over to the new file by |\childdocforward|:
%    \begin{macrocode}
\newcommand{\childdocforwardprefix}[3][]
{
  \begingroup
    \def\childdocextract #2##1~~~{\def\childdoctmp{\childdocforward[#1]{#3##1}}}
    \expandafter\childdocextract\childdocname~~~
    \expandafter
  \endgroup
  \childdoctmp
}
%    \end{macrocode}

% \macro{\childdoc}
% The deprecated macro |\childdoc| is a legacy version of |\childdocmain|:
%    \begin{macrocode}
\newcommand{\childdoc}{\childdocmain}
%    \end{macrocode}

% \macro{\childdocredirect}
% The deprecated macro |\childdocredirect| is a legacy version
% of |\childdocforward| and |\childdocforwardprefix|:
%    \begin{macrocode}
\newcommand{\childdocredirect}[2][]
{
  \begingroup
    \if?#1?
      \def\childdoctmp{\childdocforward{#2}}
    \else
      \def\childdoctmp{\childdocforwardprefix{#1}{#2}}
    \fi
    \expandafter
  \endgroup
  \childdoctmp
}
%    \end{macrocode}

%\iffalse
%</package>
%\fi
%
\endinput

\childdocby{cdocsamp}
%    \end{macrocode}

%\iffalse
%</samplepart3|samplepart4>
%\fi
%
%\iffalse
%<*samplepart3>
%\fi
% Some text for part 3:
%    \begin{macrocode}
some text in part three
%    \end{macrocode}

%\iffalse
%</samplepart3>
%\fi
% Some text for part 4:
%\iffalse
%<*samplepart4>
%\fi
%    \begin{macrocode}
more text in part four
%    \end{macrocode}

%\iffalse
%</samplepart4>
%\fi
%
% %%%%%%%%%%%%%%%%%%%%%%%%%%%%%%%%%%%%%%
% \paragraph{Forwarding for a Complete Draft.}
%
% The following forwarding file |cdocsdrf.tex|
% compiles the main document in draft mode:
%\iffalse
%<*sampledraft>
%\fi
%    \begin{macrocode}
\def\version{draft}
% \iffalse
%
% childdoc.dtx Copyright (C) 2017-2018 Niklas Beisert
%
% This work may be distributed and/or modified under the
% conditions of the LaTeX Project Public License, either version 1.3
% of this license or (at your option) any later version.
% The latest version of this license is in
%   http://www.latex-project.org/lppl.txt
% and version 1.3 or later is part of all distributions of LaTeX
% version 2005/12/01 or later.
%
% This work has the LPPL maintenance status `maintained'.
%
% The Current Maintainer of this work is Niklas Beisert.
%
% This work consists of the files childdoc.dtx and childdoc.ins
% and the derived files childdoc.def and cdocsamp.tex with
% cdocsch1.tex, cdocsch2.tex, cdocsdrf.tex, cdocsfn1.tex, cdocsfn2.tex.
%
%<package>\ifdefined\childdocmain\endinput\fi
%<package>\ProvidesFile{childdoc.def}[2018/12/30 v2.0 child document driver]
%<samplemain>\ProvidesFile{cdocsamp.tex}[2018/12/30 v2.0 sample for childdoc]
%<*driver>
%\ProvidesFile{childdoc.drv}[2018/12/30 v2.0 childdoc reference manual file]
\PassOptionsToClass{10pt,a4paper}{article}
\documentclass{ltxdoc}

\usepackage[margin=35mm]{geometry}
\usepackage{hyperref}
\usepackage{hyperxmp}
\usepackage[usenames]{color}

\hypersetup{colorlinks=true}
\hypersetup{pdfstartview=FitH}
\hypersetup{pdfpagemode=UseNone}
\hypersetup{pdfsource={}}
\hypersetup{pdflang={en-UK}}
\hypersetup{pdfcopyright={Copyright 2017-2018 Niklas Beisert.
  This work may be distributed and/or modified under the
  conditions of the LaTeX Project Public License, either version 1.3
  of this license or (at your option) any later version.}}
\hypersetup{pdflicenseurl={http://www.latex-project.org/lppl.txt}}
\hypersetup{pdfcontactaddress={ETH Zurich, ITP, HIT K,
  Wolfgang-Pauli-Strasse 27}}
\hypersetup{pdfcontactpostcode={8093}}
\hypersetup{pdfcontactcity={Zurich}}
\hypersetup{pdfcontactcountry={Switzerland}}
\hypersetup{pdfcontactemail={nbeisert@itp.phys.ethz.ch}}
\hypersetup{pdfcontacturl={http://people.phys.ethz.ch/\xmptilde nbeisert/}}

\newcommand{\secref}[1]{\hyperref[#1]{section \ref*{#1}}}

\parskip1ex
\parindent0pt
\let\olditemize\itemize
\def\itemize{\olditemize\parskip0pt}

\begin{document}

\title{The \textsf{childdoc} Package}
\hypersetup{pdftitle={The childdoc Package}}
\author{Niklas Beisert\\[2ex]
  Institut f\"ur Theoretische Physik\\
  Eidgen\"ossische Technische Hochschule Z\"urich\\
  Wolfgang-Pauli-Strasse 27, 8093 Z\"urich, Switzerland\\[1ex]
  \href{mailto:nbeisert@itp.phys.ethz.ch}
  {\texttt{nbeisert@itp.phys.ethz.ch}}}
\hypersetup{pdfauthor={Niklas Beisert}}
\hypersetup{pdfsubject={Manual for the LaTeX2e Package childdoc}}
\date{30 December 2018, \textsf{v2.0}}
\maketitle

\begin{abstract}\noindent
\textsf{childdoc} is a \LaTeXe{} package
that enables the direct compilation
of document sections included by |\include|
to individual files.
\end{abstract}

\begingroup
\parskip0ex
\tableofcontents
\endgroup

%%%%%%%%%%%%%%%%%%%%%%%%%%%%%%%%%%%%%%%%%%%%%%%%%%%%%%%%%%%%%%%%%%%%%%%%%%%%%%%%
%%%%%%%%%%%%%%%%%%%%%%%%%%%%%%%%%%%%%%%%%%%%%%%%%%%%%%%%%%%%%%%%%%%%%%%%%%%%%%%%
\section{Introduction}

\LaTeX{} provides a mechanism to structure a large document (such as a book)
into a main file and several child files (containing the chapters)
using the |\include| command.
This mechanism is beneficial for documents
which span hundreds of pages in order to
make the source file(s) more manageable.
Moreover, compilation can be restricted to
selected child files by means of the |\includeonly| command.
The latter feature can be used to reduce the compilation time while editing
(this was significantly more useful in the earlier days of \LaTeX{})
or to generate a smaller document which is easier to navigate.
Another application of |\includeonly| is to generate
documents consisting of selected parts of the complete document.

However, there are a few drawbacks of the plain |\include| mechanism:
\begin{itemize}
\item
The child files cannot be compiled on their own,
they can only be compiled via the main file.
A naive editing environment
(such as a text editor with an option
to have the current file processed by \LaTeX)
may require one to switch to the main file before compiling;
attempting to compile the child file produces errors.
\item
The main file must be modified (each time)
to adjust the |\includeonly| command
to the present needs. This easily leaves the main file in a messy state.
\item
The generated document will always carry the filename
of the main document. This is inconvenient if
several child files are to be compiled and
to be kept for distribution.
\end{itemize}

The present package provides a simple interface
to make child files individually compilable by \LaTeX{}.
Compiling a child file then has the same effect as compiling
the main file with an |\includeonly| command
to select the appropriate child.
Moreover the generated document will carry the name of the child
rather than the main file.
This resolves all three above issues.

This feature is meant to make the editing of books,
thesis documents and lecture notes somewhat more convenient.
However, the package can also be used efficiently for
composing a series of documents (such as exercise sheets)
which are typically distributed individually.
It then assists the author in generating the individual documents
(potentially in different versions)
as well as a document containing the collected series.
Another application is in developing style files
or other kinds of included material
where compilation of the style file could redirect
to a sample or test file.

%%%%%%%%%%%%%%%%%%%%%%%%%%%%%%%%%%%%%%%%%%%%%%%%%%%%%%%%%%%%%%%%%%%%%%%%%%%%%%%%
%%%%%%%%%%%%%%%%%%%%%%%%%%%%%%%%%%%%%%%%%%%%%%%%%%%%%%%%%%%%%%%%%%%%%%%%%%%%%%%%
\section{Usage}

First of all, the package \textsf{childdoc} is \emph{not} a standard
\LaTeXe{} |.sty| style file! Therefore it needs to be invoked in
a non-standard way.

%%%%%%%%%%%%%%%%%%%%%%%%%%%%%%%%%%%%%%%%%%%%%%%%%%%%%%%%%%%%%%%%%%%%%%%%%%%%%%%%
\subsection{Included Files}
\label{sec:include}

%%%%%%%%%%%%%%%%%%%%%%%%%%%%%%%%%%%%%%%%
\DescribeMacro{\childdocmain}
To use the package, add the commands
\begin{center}
\begin{tabular}{l}
|\input{childdoc.def}|\\
|\childdocmain{}|\\
\end{tabular}
\end{center}
at the very top of the main \LaTeX{} file,
in particular \emph{before} the |\documentclass| statement!
The argument of |\childdocmain| should be left empty
(but it must be present).

%%%%%%%%%%%%%%%%%%%%%%%%%%%%%%%%%%%%%%%%
\DescribeMacro{\childdocof}
Furthermore, add the commands
\begin{center}
\begin{tabular}{l}
|\input{childdoc.def}|\\
|\childdocof{|\textit{main}|}|\\
\end{tabular}
\end{center}
at the top of every child file \textit{child}
which is included by |\include{|\textit{child}|}|
from within the main file
(or at least for those files to be compiled individually).
The argument \textit{main} must be the filename of the main file.

There are a couple of
considerations in setting up the main and child documents:

%%%%%%%%%%%%%%%%%%%%%%%%%%%%%%%%%%%%%%%%
\paragraph{Restrictions.}

Please note the following restrictions:
\begin{itemize}
\item
|\childdocmain| must be called with one argument \textit{main}
to ensure compatibility with earlier version of the package.
It must either be empty (|\childdocmain{}|)
or precisely match the filename of the main file in which it is specified.
See \secref{sec:detection} for further information.
\item
The filename \textit{main} must be specified without the |.tex| extension.
\item
The filename \textit{main} is case sensitive
(even in case-insensitive file systems)
due to internal string comparison.
\item
The argument \textit{main} should be fully expanded, it cannot be a macro.
\item
Subdirectories and special characters should be avoided in filenames.
\item
The command |\childdocmain{|\textit{main}|}| must be followed by a whitespace.
It should not be followed immediately by another command
or by a comment mark `|%|'.
This is because the \TeX{} parser reads the token immediately following
the argument of |\childdocmain| and puts it
at the beginning of every child section;
however, a white\-space is ignored.
\end{itemize}

%%%%%%%%%%%%%%%%%%%%%%%%%%%%%%%%%%%%%%%%
\paragraph{Content of Main File.}

It is advisable to place all content in the child files included by |\include|.
Any output contained in the main file will appear in all child documents
unless suppressed manually;
it cannot be suppressed automatically by the |\includeonly| directive
and thus should normally be avoided.
A method to include some content in the main file
by means of conditional processing is described in \secref{sec:conditional}.

%%%%%%%%%%%%%%%%%%%%%%%%%%%%%%%%%%%%%%%%
\paragraph{Page Numbering.}

When only a part of the document is compiled,
the appropriate numbering of pages
(as well as other status parameters)
is determined from the |.aux| files.
The latter contain information from previous passes.
However this information needs to propagate through
all intermediate child documents.
Therefore the page numbering in child documents may well
be inconsistent until the complete document is compiled at least once.

A useful (if unconventional) way to always ensure a consistent
page numbering is to restart the numbering in each child document
and denote the pages by `\textit{child}|.|\textit{page}'
where \textit{child} represents the chapter/section number of the child file.
This can be achieved by the command
|\numberwithin{page}{|\textit{child}|}|
of the \textsf{amsmath} package
where \textit{child} can be |chapter| or |section|
depending on the chosen structuring.
Alternatively, one can modify the macro |\thepage| appropriately
and reset the counter |page| at the start of each child file.

%%%%%%%%%%%%%%%%%%%%%%%%%%%%%%%%%%%%%%%%%%%%%%%%%%%%%%%%%%%%%%%%%%%%%%%%%%%%%%%%
\subsection{Conditional Processing}
\label{sec:conditional}

The package provides a mechanism to compile different versions
of a document. To customise the versions further some conditional processing
can come in handy to distinguish which version is being compiled.
The package provides two macros to describe the compilation context:

%%%%%%%%%%%%%%%%%%%%%%%%%%%%%%%%%%%%%%%%
\DescribeMacro{\ifchilddoc}
The conditional |\ifchilddoc| distinguishes between the compilation of
child documents and the main document:
%
\begin{center}
|\ifchilddoc |\textit{child-code}| |[|\||else |\textit{main-code}]| \||fi|
\end{center}

%%%%%%%%%%%%%%%%%%%%%%%%%%%%%%%%%%%%%%%%
\DescribeMacro{\childdocname}
\DescribeMacro{\childdocjob}
The macro |\childdocname| contains the filename (without extension)
of the main or child file being processed.
Note that |\childdocjob| will always contain the name of the main file.

%%%%%%%%%%%%%%%%%%%%%%%%%%%%%%%%%%%%%%%%
\paragraph{Title Page.}

Conditional processing can be used to include a title or banner page
in the main document when proper precautions are taken.
Importantly, the code in the main file should ensure that the page counter
(as well as other status parameters which are stored in the |.aux| files)
takes the same value after the conditional processing.
Otherwise the page numbers may take divergent values
depending on which part is compiled.

For example, a title page could be declared by:
%
\begin{center}
\begin{tabular}{l}
|\ifchilddoc\||else|\\
|\addtocounter{page}{-1}|\\
\textit{code for title page}\\
|\newpage|\\
|\||fi|
\end{tabular}
\end{center}
%
A banner page for the child documents can be generated by:
%
\begin{center}
\begin{tabular}{l}
|\ifchilddoc|\\
|\addtocounter{page}{-1}|\\
\textit{code for banner page}\\
|\newpage|\\
|\||fi|
\end{tabular}
\end{center}
%
Here one could write a message such as:
\begin{center}
|This is the part \childdocname{} of \childdocjob{}.|
\end{center}

%%%%%%%%%%%%%%%%%%%%%%%%%%%%%%%%%%%%%%%%%%%%%%%%%%%%%%%%%%%%%%%%%%%%%%%%%%%%%%%%
\subsection{Flags}
\label{sec:flags}

The package makes it easy to generate different versions
of the main or child documents.
To this end compilation flags can be defined
and assigned different default values.
They will be particularly useful in conjunction
with the forwarding mechanism described in \secref{sec:forward}.

For example, it may be useful to have a flag |\version|
which can be set to |draft| or |final|.
The document source will contain some conditional code
depending on the value of |\version|.
Suppose further, the flag should default to |final| for the main file
and to |draft| for child files
which is a natural assignment for editing the document.
This is achieved by placing the following code
in the preamble of the main document
(below the |\childdocmain| directive):
%
\begin{center}
\begin{tabular}{l}
|\ifchilddoc|\\
|\providecommand{\version}{draft}|\\
|\||else|\\
|\providecommand{\version}{final}|\\
|\||fi|
\end{tabular}
\end{center}
%
The definition by |\providecommand| makes sure
that previous definitions are not overwritten.
Further statements |\providecommand{\version}{...}|
can thus be added before the above code to override it.

For the main file, one might add a line
(between |\childdocmain| and the above block)
%
\begin{center}
|%\ifchilddoc\||else\providecommand{\version}{draft}\||fi|
\end{center}
%
which can be uncommented to produce a draft version.
Likewise one can add a line to the very top of a child file
(above the |\childdocof{|\textit{main}|}| directive)
%
\begin{center}
|%\providecommand{\version}{final}|
\end{center}
%
which can be uncommented to produce the final version of this child document.

%%%%%%%%%%%%%%%%%%%%%%%%%%%%%%%%%%%%%%%%%%%%%%%%%%%%%%%%%%%%%%%%%%%%%%%%%%%%%%%%
\subsection{Forwarding}
\label{sec:forward}

Different versions of the main or child documents
using compilation flags as described in \secref{sec:flags}
can be (permanently) stored in different files
for convenient compilation, viewing and distribution.
To this end, the package defines a command
to pass on compilation to a different file:

%%%%%%%%%%%%%%%%%%%%%%%%%%%%%%%%%%%%%%%%
\DescribeMacro{\childdocforward}
The command |\childdocforward| redirects processing to
another source file:
%
\begin{center}
\begin{tabular}{l}
|\input{childdoc.def}|\\
|\childdocforward[|\textit{main}|]{|\textit{dest}|}|\\
\end{tabular}
\end{center}
%
The argument \textit{dest} is the destination file
(without extension).
It should be the main file or one of the child files.
Note that further \textsf{childdoc} directives
such as |\childdocof| and |\childdocforward|
in the indicated file will be processed in this form.
The optional argument \textit{main}
passes on directly to the main file \textit{main}
while pretending to compile the child \textit{dest}.
This form behaves as if \textit{dest}
issues |\childdocof{|\textit{main}|}| right away,
and no further \textsf{childdoc} directives will be processed.

%%%%%%%%%%%%%%%%%%%%%%%%%%%%%%%%%%%%%%%%
\DescribeMacro{\...prefix}
In the alternative form |\childdocforwardprefix|,
%
\begin{center}
\begin{tabular}{l}
|\input{childdoc.def}|\\
|\childdocforwardprefix[|\textit{main}|]{|\textit{prefix}|}{|\textit{dest}|}|
\end{tabular}
\end{center}
%
the destination file is determined by a pattern
depending on the current file:
To make this work, the current file must be called
`{\textit{prefix}\hspace{0.2em}\textit{suffix}}'
with \textit{prefix} matching precisely the argument.
Processing is then passed on to the file
`{\textit{dest}\hspace{0.2em}\textit{suffix}}'.
Surely, the same effect is achieved by
directly specifying the
argument `{\textit{dest}\hspace{0.2em}\textit{suffix}}'
in the first form.
However, that requires to set up a different file
for each child. With the alternative form of the command
all these files can have exactly the same content
which simplifies setting them up and maintaining them.

For example, the following file |draft.tex|
with a compilation flag |\version| as described in \secref{sec:flags}
compiles the main document as a draft:
%
\begin{center}
\begin{tabular}{l}
|\def\version{draft}|\\
|\input{childdoc.def}|\\
|\childdocforward{|\textit{main}|}|
\end{tabular}
\end{center}
%
Likewise, the following files |final|\textit{nn}|.tex|
compile the final version of the child document
|child|\textit{nn}|.tex|:
%
\begin{center}
\begin{tabular}{l}
|\def\version{final}|\\
|\input{childdoc.def}|\\
|\childdocforwardprefix{final}{child}|
\end{tabular}
\end{center}
%

Note that when several versions of a main file and/or of each child file
are to be generated, it may be convenient to set up a |Makefile| or
shell script to automatise the process.

%%%%%%%%%%%%%%%%%%%%%%%%%%%%%%%%%%%%%%%%%%%%%%%%%%%%%%%%%%%%%%%%%%%%%%%%%%%%%%%%
\subsection{Command Line Processing}
\label{sec:commandline}

The effect of redirection files can also be achieved by invoking
the \LaTeX{} compiler with a more elaborate command line.
Most conveniently this should be done as part
of a shell script or a |Makefile|.

When using \textsf{childdoc} in the main file, the following
command lines effectively perform a redirection
(note that depending on the shell being used,
backslashes may have to be doubled: `|\|' $\to$ `|\\|'):
%
\begin{center}
|... -jobname "|\textit{target}|" |\\|"|[\textit{flags}]%
|\input{childdoc.def}\childdocforward[|\textit{main}|]{|\textit{dest}|}"|
\end{center}
%
Here \textit{target} is the name of the output file,
\textit{main} is the name of the main file
and \textit{dest} is the name of the main or child file to be processed
(all filenames without extensions).
The optional argument \textit{main} can be omitted
if \textit{main} matches \textit{dest}.
Optionally, compilation \textit{flags} can be defined via |\def| commands.
This command line makes the \TeX{} engine believe
it is compiling the file \textit{target}
whose content is specified as the latter parameter.
The provided code then forwards the processing to
\textit{main} or \textit{dest} as described in \secref{sec:forward}.

%%%%%%%%%%%%%%%%%%%%%%%%%%%%%%%%%%%%%%%%%%%%%%%%%%%%%%%%%%%%%%%%%%%%%%%%%%%%%%%%
\subsection{Include by Input}
\label{sec:input}

Including child documents by |\include| has some restrictions by design.
Most notably, the content of a child document always occupies
its own set of pages; pages cannot be shared between child documents.
Usually, this behaviour makes perfect sense
because each child document contain an essential part of the document.
However, in some situations it may be desirable to compose
a document from a collection of parts
without having mandatory page breaks between then.
For this case, the package
provides a mechanism to include parts
by |\input| which can also be processed individually.
However, by construction this mechanism
requires manual handling of the content to be output.

%%%%%%%%%%%%%%%%%%%%%%%%%%%%%%%%%%%%%%%%
\DescribeMacro{\ifchilddocmanual}
The main file should be prepared as usual, see \secref{sec:include}.
However, the document body must make a distinction
between processing of an individual part and of the main document, e.g.:
%
\begin{center}
\begin{tabular}{l}
|\ifchilddocmanual|\\
|\input{\childdocname}|\\
|\||else|\\
\textit{document body with }|\input{|\textit{part}|}|\\
|\||fi|
\end{tabular}
\end{center}
%
The conditional |\ifchilddocmanual| is true whenever
a part to be included by |\input| is being compiled,
and the name of the part is stored in |\childdocname|.

%%%%%%%%%%%%%%%%%%%%%%%%%%%%%%%%%%%%%%%%
\DescribeMacro{\childdocby}
Each part to be included by |\input| should start with:
%
\begin{center}
\begin{tabular}{l}
|\input{childdoc.def}|\\
|\childdocby{|\textit{main}|}|\\
\end{tabular}
\end{center}
%
The directive |\childdocby| is similar to |\childdocof|
described in \secref{sec:include},
but the subsequent selection of content must be done manually.
To that end, both |\ifchilddoc| and |\ifchilddocmanual|
will be true upon processing of a part,
and the name of the part is stored in |\childdocname|.
Note that |\jobname| will be set to the filename of the current part
so that each part receives an individual |.aux| file
that does not interfere with the |.aux| file(s) of the main document.
This behaviour can be altered by the alternative form
|\childdocby[*]{|\textit{main}|}| (with a non-empty optional argument)
which uses the |.aux| file of the main document
by setting |\jobname| to \textit{main}.

%%%%%%%%%%%%%%%%%%%%%%%%%%%%%%%%%%%%%%%%%%%%%%%%%%%%%%%%%%%%%%%%%%%%%%%%%%%%%%%%
\subsection{Driver Development}
\label{sec:driver}

The \textsf{childdoc} mechanism can also be use for the development
of definition files such as \LaTeX{} styles or classes.
This case differs from the above setup with multiple parts
included by |\include| in that no |\includeonly| should be invoked.
This can be achieved by starting the include file
(before |\ProvidesPackage|) with:
%
\begin{center}
\begin{tabular}{l}
|\input{childdoc.def}|\\
|\childdocforward{|\textit{main}|}|\\
\end{tabular}
\end{center}
%
or alternatively with:
%
\begin{center}
\begin{tabular}{l}
|\input{childdoc.def}|\\
|\childdocby{|\textit{main}|}|\\
\end{tabular}
\end{center}
%
Both forms have slightly different effects as described above.
The main file is prepared as usual, see \secref{sec:include}.

%%%%%%%%%%%%%%%%%%%%%%%%%%%%%%%%%%%%%%%%%%%%%%%%%%%%%%%%%%%%%%%%%%%%%%%%%%%%%%%%
\subsection{Legacy Detection}
\label{sec:detection}

The directive |\childdocmain| in the main file can detect
whether the complete document or merely a child is to be compiled
even without using the directive |\childdocof|.
This method is deprecated because it is less robust
and there is no compelling reason to use it;
it is merely provided for backward compatibility
and it may be removed in future versions.

If the detection mechanism is to be used,
it is mandatory to correctly specify
the filename of the main file as the argument of |\childdocmain|:
%
\begin{center}
\begin{tabular}{l}
|\input{childdoc.def}|\\
|\childdocmain{|\textit{main}|}|\\
\end{tabular}
\end{center}
%
If |\jobname| does not match the argument \textit{main} of |\childdocmain|,
it is assumed that |\jobname| points to the child file to be compiled.
When using |\childdocmain| with the main file specified as argument,
it suffices to start a child file
with just |\input{|\textit{main}|}|
without loading of the package and using |\childdocof|.
If instead all processing is done
with the appropriate \textsf{childdoc} directives,
the argument of \textit{main} of |\childdocmain| can be empty.

An alternative version of the command line processing described
in \secref{sec:commandline} using the detection mechanism reads:
%
\begin{center}
|... -jobname "|\textit{target}|" "|[\textit{flags}]%
[|\def\jobname{|\textit{dest}|}|]|\input{|\textit{main}|}"|
\end{center}

%%%%%%%%%%%%%%%%%%%%%%%%%%%%%%%%%%%%%%%%%%%%%%%%%%%%%%%%%%%%%%%%%%%%%%%%%%%%%%%%
\subsection{Manual Code}
\label{sec:manual}

In case one cannot be certain whether the definitions file |childdoc.def|
is installed on the target \TeX{} distribution
and one prefers not to ship it,
it is conceivable to paste a few relevant commands into the sources.

To that end, drop all statements |\input{childdoc.def}|
and perform the replacements as outlined below.
Instead of |\childdocmain{|\textit{main}|}| add the following code
to the top of the main file:
%
\begin{center}
\begin{tabular}{l}
|\||ifdefined\childdocname\endinput\||fi\newif\ifchilddoc|\\
|\edef\childdocname{\scantokens\expandafter{\jobname\noexpand}}|\\
|\def\childdocmain{|\textit{main}|}\||ifx\childdocmain\childdocname\||else|\\
|\childdoctrue\includeonly{\childdocname}\let\jobname\childdocmain\||fi|\\
\end{tabular}
\end{center}
%
Instead of |\childdocof{|\textit{main}|}| just include the main file
at the top of each child file:
%
\begin{center}
|\input{|\textit{main}|}|
\end{center}
%
A simple redirection |\childdocforward{|\textit{dest}|}| is achieved by:
%
\begin{center}
|\def\jobname{|\textit{dest}|}\input{\jobname}|
\end{center}
%
The redirection with prefix
|\childdocforwardprefix[|\textit{prefix}|]{|\textit{dest}|}|
is accomplished by:
%
\begin{center}
\begin{tabular}{l}
|{\edef\jobname{\scantokens\expandafter{\jobname\noexpand}}|\\
|\def\redirectjob |\textit{prefix}|#1~~~{\gdef\jobname{|\textit{dest}|#1}}|\\
|\expandafter\redirectjob\jobname~~~}\input{\jobname}|
\end{tabular}
\end{center}

In an alternative approach,
child documents can be compiled by a specific command line
without additional code or specific definitions:
%
\begin{center}
|... -jobname "|\textit{target}|" "|[\textit{flags}]%
|\includeonly{|\textit{dest}|}\input{|\textit{main}|}"|
\end{center}
%

%%%%%%%%%%%%%%%%%%%%%%%%%%%%%%%%%%%%%%%%%%%%%%%%%%%%%%%%%%%%%%%%%%%%%%%%%%%%%%%%
%%%%%%%%%%%%%%%%%%%%%%%%%%%%%%%%%%%%%%%%%%%%%%%%%%%%%%%%%%%%%%%%%%%%%%%%%%%%%%%%
\section{Information}

%%%%%%%%%%%%%%%%%%%%%%%%%%%%%%%%%%%%%%%%%%%%%%%%%%%%%%%%%%%%%%%%%%%%%%%%%%%%%%%%
\subsection{Copyright}

Copyright \copyright{} 2017--2018 Niklas Beisert

This work may be distributed and/or modified under the
conditions of the \LaTeX{} Project Public License, either version 1.3
of this license or (at your option) any later version.
The latest version of this license is in
  \url{http://www.latex-project.org/lppl.txt}
and version 1.3 or later is part of all distributions of \LaTeX{}
version 2005/12/01 or later.

This work has the LPPL maintenance status `maintained'.

The Current Maintainer of this work is Niklas Beisert.

This work consists of the files |README.txt|, |childdoc.ins| and |childdoc.dtx|
as well as the derived files |childdoc.def|, |cdocsamp.tex|
with |cdocsch1.tex|, |cdocsch2.tex|, |cdocspt3.tex|, |cdocspt4.tex|,
|cdocsdrf.tex|, |cdocsfn1.tex|, |cdocsfn2.tex|
as well as |childdoc.pdf|.

%%%%%%%%%%%%%%%%%%%%%%%%%%%%%%%%%%%%%%%%%%%%%%%%%%%%%%%%%%%%%%%%%%%%%%%%%%%%%%%%
\subsection{Files and Installation}

The package consists of the files:
%
\begin{center}
\begin{tabular}{ll}
    |README.txt|   & readme file \\
    |childdoc.ins| & installation file \\
    |childdoc.dtx| & source file \\
    |childdoc.def| & definition file \\
    |cdocsamp.tex| & sample main file \\
    |cdocsch1.tex| & sample include file \\
    |cdocsch2.tex| & sample include file \\
    |cdocspt3.tex| & sample part file \\
    |cdocspt4.tex| & sample part file \\
    |cdocsdrf.tex| & sample redirection file \\
    |cdocsfn1.tex| & sample redirection file \\
    |cdocsfn2.tex| & sample redirection file \\
    |childdoc.pdf| & manual
\end{tabular}
\end{center}
%
The distribution consists of the files
|README.txt|, |childdoc.ins| and |childdoc.dtx|.
%
\begin{itemize}
\item
Run (pdf)\LaTeX{} on |childdoc.dtx|
to compile the manual |childdoc.pdf| (this file).
\item
Run \LaTeX{} on |childdoc.ins| to create the definitions file |childdoc.def|
and the sample |cdocsamp.tex| with include files
|cdocsch1.tex|, |cdocsch2.tex|, |cdocspt3.tex|, |cdocspt4.tex|,
|cdocsdrf.tex|, |cdocsfn1.tex|, |cdocsfn2.tex|.
Then copy the file |childdoc.def| to an appropriate directory of your \LaTeX{}
distribution, e.g.\ \textit{texmf-root}|/tex/latex/childdoc|.
\end{itemize}

%%%%%%%%%%%%%%%%%%%%%%%%%%%%%%%%%%%%%%%%%%%%%%%%%%%%%%%%%%%%%%%%%%%%%%%%%%%%%%%%
\subsection{Related CTAN Packages}

There are several other packages which offer a similar functionality:
%
\begin{itemize}
\item
The packages
\href{http://ctan.org/pkg/docmute}{\textsf{docmute}},
\href{http://ctan.org/pkg/includex}{\textsf{includex}} and
\href{http://ctan.org/pkg/standalone}{\textsf{standalone}}
provide commands to include only the document body of
a child file thus allowing both files to be compiled individually.
\item
The packages \href{http://ctan.org/pkg/subdocs}{\textsf{subdocs}}
and \href{http://ctan.org/pkg/subfiles}{\textsf{subfiles}}
provide structures in which the main and child documents can be
encapsulated and allowing them to be compiled individually.
The inclusion mechanism is different from the conventional |\include|.
\item
The package \href{http://ctan.org/pkg/combine}{\textsf{combine}}
is an elaborate solution to combine several documents into one.
\end{itemize}
%
See also the CTAN topic \href{http://ctan.org/topic/subdocs}{\textsf{subdocs}}
for further related packages.
The present package differs from the above solutions in that
a document structure constructed with the conventional |\include| mechanism
just needs two extra commands at the top of every file
such that all constituent files can be compiled individually.

%%%%%%%%%%%%%%%%%%%%%%%%%%%%%%%%%%%%%%%%%%%%%%%%%%%%%%%%%%%%%%%%%%%%%%%%%%%%%%%%
%\subsection{Feature Suggestions}
%
%The following is a list of features which may be useful for future
%versions of this package:
%%
%\begin{itemize}
%\item
%\ldots
%\end{itemize}

%%%%%%%%%%%%%%%%%%%%%%%%%%%%%%%%%%%%%%%%%%%%%%%%%%%%%%%%%%%%%%%%%%%%%%%%%%%%%%%%
\subsection{Revision History}

%%%%%%%%%%%%%%%%%%%%%%%%%%%%%%%%%%%%%%%%
\paragraph{v2.0:} 2018/12/30

\begin{itemize}
\item
immediate forward processing
\item
added |\childdocby| mechanism
\item
manual restructured
\end{itemize}

%%%%%%%%%%%%%%%%%%%%%%%%%%%%%%%%%%%%%%%%
\paragraph{v1.6:} 2018/01/17

\begin{itemize}
\item
application for development of include files
\item
corrections to manual
\end{itemize}

%%%%%%%%%%%%%%%%%%%%%%%%%%%%%%%%%%%%%%%%
\paragraph{v1.5:} 2017/05/21

\begin{itemize}
\item
more complete structuring introduced
\item
|\childdocof| introduced
\item
|\childdoc| renamed to |\childdocmain|
\item
|\childredirect| renamed to |\childdocforward| and |\childdocforwardprefix|
and functionality expanded
\end{itemize}

%%%%%%%%%%%%%%%%%%%%%%%%%%%%%%%%%%%%%%%%
\paragraph{v1.0:} 2017/04/27

\begin{itemize}
\item
manual and install package
\item
first version published on CTAN
\end{itemize}

%%%%%%%%%%%%%%%%%%%%%%%%%%%%%%%%%%%%%%%%
\paragraph{v0.6:} 2017/04/26

\begin{itemize}
\item
redirection mechanism added
\end{itemize}

%%%%%%%%%%%%%%%%%%%%%%%%%%%%%%%%%%%%%%%%
\paragraph{v0.5:} 2017/04/26

\begin{itemize}
\item
functionality in definition file
\end{itemize}


%%%%%%%%%%%%%%%%%%%%%%%%%%%%%%%%%%%%%%%%%%%%%%%%%%%%%%%%%%%%%%%%%%%%%%%%%%%%%%%%
%%%%%%%%%%%%%%%%%%%%%%%%%%%%%%%%%%%%%%%%%%%%%%%%%%%%%%%%%%%%%%%%%%%%%%%%%%%%%%%%
%%%%%%%%%%%%%%%%%%%%%%%%%%%%%%%%%%%%%%%%%%%%%%%%%%%%%%%%%%%%%%%%%%%%%%%%%%%%%%%%
\appendix

\settowidth\MacroIndent{\rmfamily\scriptsize 000\ }

 \DocInput{childdoc.dtx}

\end{document}
%</driver>
% \fi
%
% %%%%%%%%%%%%%%%%%%%%%%%%%%%%%%%%%%%%%%%%%%%%%%%%%%%%%%%%%%%%%%%%%%%%%%%%%%%%%%
% %%%%%%%%%%%%%%%%%%%%%%%%%%%%%%%%%%%%%%%%%%%%%%%%%%%%%%%%%%%%%%%%%%%%%%%%%%%%%%
% \section{Sample}
%\iffalse
%<*samplemain>
%\fi
%
% The following presents a sample document
% with two chapters, two parts, a title page,
% a compile flag as well as three forwarding files to set the flag.
% It consists of eight |.tex| files:
% \begin{center}
% \begin{tabular}{ll}
% |cdocsamp.tex|&main file\\
% |cdocsch1.tex|&include file for chapter 1\\
% |cdocsch2.tex|&include file for chapter 2\\
% |cdocspt3.tex|&include file for part 3\\
% |cdocspt4.tex|&include file for part 4\\
% |cdocsdrf.tex|&forwarding file for main file in draft mode\\
% |cdocsfi1.tex|&forwarding file for final version of chapter 1\\
% |cdocsfi2.tex|&forwarding file for final version of chapter 2\\
% \end{tabular}
% \end{center}
% Each of the eight files can be compiled directly by the \LaTeX{} compiler.
%
% %%%%%%%%%%%%%%%%%%%%%%%%%%%%%%%%%%%%%%
% \paragraph{Main File.}
%
% The main file is called |cdocsamp.tex|.
%
% Load the \textsf{childdoc} definitions and
% declare the filename for the main document:
%    \begin{macrocode}
\input{childdoc.def}
\childdocmain{}
%    \end{macrocode}

% Optional override for |\version| flag:
%    \begin{macrocode}
%%\ifchilddoc\else\providecommand{\version}{draft}\fi
%    \end{macrocode}

% Define the default values for the |\version| flag
% (|final| for the main file and |draft| for childs):
%    \begin{macrocode}
\ifchilddoc
\providecommand{\version}{draft}
\else
\providecommand{\version}{final}
\fi
%    \end{macrocode}

% Load the standard document class:
%    \begin{macrocode}
\documentclass[12pt]{article}
%    \end{macrocode}

% Start the document body:
%    \begin{macrocode}
\begin{document}
%    \end{macrocode}

% Declare a title page.
% Print title, part of document being processed and version flag:
%    \begin{macrocode}
\addtocounter{page}{-1}
\begin{center}
{\LARGE\bfseries{}childdoc example\par}
\vspace{1cm}
\ifchilddoc
\ifchilddocmanual part\else chapter\fi:
`\childdocname' of `\childdocjob'\par
\else
main document: `\childdocjob'\par
\fi
version: \version\par
\end{center}
\newpage
%    \end{macrocode}

% Manually include selected file,
% otherwise process as usual:
%    \begin{macrocode}
\ifchilddocmanual
\section*{part `\childdocname'}
\input{\childdocname}
\else
%    \end{macrocode}

% Include the two chapters:
%    \begin{macrocode}
\include{cdocsch1}
\include{cdocsch2}
%    \end{macrocode}

% Include the two parts unless only chapters should be displayed:
%    \begin{macrocode}
\ifchilddoc\else
\section{part three}
\input{cdocspt3}
\section{part four}
\input{cdocspt4}
\fi
%    \end{macrocode}

% Process as usual until here:
%    \begin{macrocode}
\fi
%    \end{macrocode}

% End of document body:
%    \begin{macrocode}
\end{document}
%    \end{macrocode}
%\iffalse
%</samplemain>
%\fi
%
% %%%%%%%%%%%%%%%%%%%%%%%%%%%%%%%%%%%%%%
% \paragraph{Chapter Include Files.}
%
% The include files are called |cdocsch1.tex| and |cdocsch2.tex|.
%
%\iffalse
%<*samplechap1|samplechap2>
%\fi

% Optional override for |\version| flag:
%    \begin{macrocode}
%%\providecommand{\version}{final}
%    \end{macrocode}

% Include the main document:
%    \begin{macrocode}
\input{childdoc.def}
\childdocof{cdocsamp}
%    \end{macrocode}

%\iffalse
%</samplechap1|samplechap2>
%\fi
%
%\iffalse
%<*samplechap1>
%\fi
% Some text for chapter 1:
%    \begin{macrocode}
\section{one}
some text in chapter one
%    \end{macrocode}

%\iffalse
%</samplechap1>
%\fi
% Some text for chapter 2:
%\iffalse
%<*samplechap2>
%\fi
%    \begin{macrocode}
\section{two}
more text in chapter two
%    \end{macrocode}

%\iffalse
%</samplechap2>
%\fi
%
% %%%%%%%%%%%%%%%%%%%%%%%%%%%%%%%%%%%%%%
% \paragraph{Part Include Files.}
%
% The include files are called |cdocspt3.tex| and |cdocspt4.tex|.
%
%\iffalse
%<*samplepart3|samplepart4>
%\fi

% Optional override for |\version| flag:
%    \begin{macrocode}
%%\providecommand{\version}{final}
%    \end{macrocode}

% Include the main document:
%    \begin{macrocode}
\input{childdoc.def}
\childdocby{cdocsamp}
%    \end{macrocode}

%\iffalse
%</samplepart3|samplepart4>
%\fi
%
%\iffalse
%<*samplepart3>
%\fi
% Some text for part 3:
%    \begin{macrocode}
some text in part three
%    \end{macrocode}

%\iffalse
%</samplepart3>
%\fi
% Some text for part 4:
%\iffalse
%<*samplepart4>
%\fi
%    \begin{macrocode}
more text in part four
%    \end{macrocode}

%\iffalse
%</samplepart4>
%\fi
%
% %%%%%%%%%%%%%%%%%%%%%%%%%%%%%%%%%%%%%%
% \paragraph{Forwarding for a Complete Draft.}
%
% The following forwarding file |cdocsdrf.tex|
% compiles the main document in draft mode:
%\iffalse
%<*sampledraft>
%\fi
%    \begin{macrocode}
\def\version{draft}
\input{childdoc.def}
\childdocforward{cdocsamp}
%    \end{macrocode}

%\iffalse
%</sampledraft>
%\fi
%
% %%%%%%%%%%%%%%%%%%%%%%%%%%%%%%%%%%%%%%
% \paragraph{Forwarding for Final Version of the Chapters.}
%
% The following forwarding files |cdocsfn1.tex| and |cdocsfn2.tex|
% (with identical content)
% compile the final versions of the child documents
% |cdocsch1.tex| and |cdocsch2.tex|, respectively:
%\iffalse
%<*samplefinal>
%\fi
%    \begin{macrocode}
\def\version{final}
\input{childdoc.def}
\childdocforwardprefix[cdocsamp]{cdocsfn}{cdocsch}
%    \end{macrocode}

%\iffalse
%</samplefinal>
%\fi
%
% %%%%%%%%%%%%%%%%%%%%%%%%%%%%%%%%%%%%%%
% \paragraph{Command Line Processing.}
%
% The following three command lines generate the output files
% |cdocscld|, |cdocscl1| and |cdocscl2|
% which should be identical to
% |cdocsdrf|, |cdocsch1| and |cdocsfn2|, respectively:
% \begin{center}
% \begin{tabular}{l}
% |latex -jobname cdocscld \|\\
% |  "\def\version{draft}\input{childdoc.def}\childdocforward{cdocsamp}"|\\
% |latex -jobname cdocscl1 \|\\
% |  "\input{childdoc.def}\childdocforward[cdocsamp]{cdocsch1}"|\\
% |latex -jobname cdocscl2 \|\\
% |  "\def\version{final}\input{childdoc.def}\childdocforward{cdocsch2}"|
% \end{tabular}
% \end{center}
% Note that the trailing backslash on each first line
% merely continues the input to the second line
% (for convenient cut ant paste).
% Furthermore, the command |latex| can be replaced by any
% of its alternative versions such as |pdflatex|.
%
% %%%%%%%%%%%%%%%%%%%%%%%%%%%%%%%%%%%%%%%%%%%%%%%%%%%%%%%%%%%%%%%%%%%%%%%%%%%%%%
% %%%%%%%%%%%%%%%%%%%%%%%%%%%%%%%%%%%%%%%%%%%%%%%%%%%%%%%%%%%%%%%%%%%%%%%%%%%%%%
% \section{Implementation}
%\iffalse
%<*package>
%\fi
%
% This section describes the definitions file |childdoc.def|.

% The definitions cannot be loaded using |\usepackage| or |\RequirePackage|
% which has a mechanism to prevent loading a style file more than once.
% When loading the definitions by means of |\input|
% multiple instances have to be prevented manually:
%\iffalse
%This code needs to be before the `\ProvidesFile' directive
%which is defined at the beginning of this file.
%Therefore it is also placed there and commented out here.
%</package>
%<*discard>
%\fi
%    \begin{macrocode}
\ifdefined\childdocmain\endinput\fi
%    \end{macrocode}
%\iffalse
%</discard>
%<*package>
%\fi
%
% \macro{\ifchilddoc}
% \macro{\ifchilddocmanual}
% The conditional |\ifchilddoc| tells whether a
% child (true) or main (false) document is being compiled.
% The conditional |\ifchilddocmanual| tells whether
% the |\includeonly| mechanism is used (false) or
% the selection of child files must be performed manually (true).
% The definitions initialise to false:
%    \begin{macrocode}
\newif\ifchilddoc
\newif\ifchilddocmanual
%    \end{macrocode}

% \macro{\childdocname}
% \macro{\childdocjob}
% The macro |\childdocname| stores the name of the main document
% to be compiled. The macro |\childdocjob| stores the name of
% the document on which the \LaTeX{} compiler was originally invoked.
% The content of |\jobname| cannot be compared
% to filenames specified in the source due to different catcodes.
% The following code rescans |\jobname|, stores the result
% in |\childdocname| and saves a copy in |\childdocjob|:
%    \begin{macrocode}
\edef\childdocname{\scantokens\expandafter{\jobname\noexpand}}
\let\childdocjob\childdocname
%    \end{macrocode}

% \macro{\childdocdisable}
% The macro |\childdocdisable| prevents the main file
% from being processed more than once.
% At this stage, the main document command |\childdocmain|
% is assumed to be called once again where it should do nothing.
% Any subsequent call to it should prevent
% a secondary processing of the main document
% It overwrites the forwarding commands
% |\childdocof| and |\childdocforward|
% with empty macros to prevent further inclusions of the main document:
%    \begin{macrocode}
\newcommand{\childdocdisable}
{
  \renewcommand{\childdocmain}[1]{\renewcommand{\childdocmain}[1]{\endinput}}
  \renewcommand{\childdocof}[1]{}
  \renewcommand{\childdocby}[2][]{}
  \renewcommand{\childdocforward}[2][]{}
  \renewcommand{\childdocdisable}{}
}
%    \end{macrocode}

% \macro{\childdocmain}
% The macro |\childdocmain| is to be called at the top of the main file
% with nothing or the main filename (without extension) as argument.
% First, it breaks loops.
% If the argument is not empty and does not match |\childdocname|
% (which is set by the first inclusion of |childdoc.def|),
% |\ifchilddoc| is set to true, |\includeonly| is applied to the child file
% and |\jobname| is set to the main file
% (for proper handling of |.aux| files):
%    \begin{macrocode}
\newcommand{\childdocmain}[1]
{
  \childdocdisable\childdocmain{}
  \if?#1?\else
    \begingroup
      \def\childdoctmp{#1}
      \ifx\childdoctmp\childdocname
        \def\childdoctmp{}
      \else
        \def\childdoctmp
        {
          \childdoctrue
          \includeonly{\childdocname}
          \def\childdocjob{#1}
          \def\jobname{#1}
        }
      \fi
      \expandafter
    \endgroup
    \childdoctmp
  \fi
}
%    \end{macrocode}

% \macro{\childdocof}
% The command |\childdocof| redirects
% compilation to the main file |#1|.
%    \begin{macrocode}
\newcommand{\childdocof}[1]
{
  \childdocdisable
  \childdoctrue
  \includeonly{\childdocname}
  \def\jobname{#1}
  \def\childdocjob{#1}
  \input{#1}
}
%    \end{macrocode}

% \macro{\childdocby}
% The command |\childdocby| ....
%    \begin{macrocode}
\newcommand{\childdocby}[2][]
{
  \childdocdisable
  \childdoctrue
  \childdocmanualtrue
  \if?#1?\else
    \def\jobname{#2}
  \fi
  \def\childdocjob{#2}
  \input{#2}
  \endinput
}
%    \end{macrocode}

% \macro{\childdocforward}
% The command |\childdocforward| redirects
% compilation to the main file or
% (if the optional argument is given) a child file.
% Parameters are set as if the main file
% or a child file starting with |\childdocof| was compiled.
% Then compilation is handed over to the main file:
%    \begin{macrocode}
\newcommand{\childdocforward}[2][]
{
  \begingroup
    \if?#1?
      \def\childdoctmp
      {
        \def\childdocname{#2}
        \def\childdocjob{#2}
        \def\jobname{#2}
        \input{#2}
        \endinput
      }
    \else
      \def\childdoctmp
      {
        \childdocdisable
        \def\childdocname{#2}
        \childdoctrue
        \includeonly{#2}
        \def\childdocjob{#1}
        \def\jobname{#1}
        \input{#1}
        \endinput
      }
    \fi
    \expandafter
  \endgroup
  \childdoctmp
}
%    \end{macrocode}

% \macro{\childdocforwardprefix}
% The command |\childdocforwardprefix| redirects
% compilation to the main or a child file by means of a pattern.
% The prefix |#1| in the current filename is replaced by |#2|
% and the suffix of the current filename is kept
% (it is assumed that the filename does not contain the substring `|~~~|'
% which is used as a delimiter).
% Compilation is handed over to the new file by |\childdocforward|:
%    \begin{macrocode}
\newcommand{\childdocforwardprefix}[3][]
{
  \begingroup
    \def\childdocextract #2##1~~~{\def\childdoctmp{\childdocforward[#1]{#3##1}}}
    \expandafter\childdocextract\childdocname~~~
    \expandafter
  \endgroup
  \childdoctmp
}
%    \end{macrocode}

% \macro{\childdoc}
% The deprecated macro |\childdoc| is a legacy version of |\childdocmain|:
%    \begin{macrocode}
\newcommand{\childdoc}{\childdocmain}
%    \end{macrocode}

% \macro{\childdocredirect}
% The deprecated macro |\childdocredirect| is a legacy version
% of |\childdocforward| and |\childdocforwardprefix|:
%    \begin{macrocode}
\newcommand{\childdocredirect}[2][]
{
  \begingroup
    \if?#1?
      \def\childdoctmp{\childdocforward{#2}}
    \else
      \def\childdoctmp{\childdocforwardprefix{#1}{#2}}
    \fi
    \expandafter
  \endgroup
  \childdoctmp
}
%    \end{macrocode}

%\iffalse
%</package>
%\fi
%
\endinput

\childdocforward{cdocsamp}
%    \end{macrocode}

%\iffalse
%</sampledraft>
%\fi
%
% %%%%%%%%%%%%%%%%%%%%%%%%%%%%%%%%%%%%%%
% \paragraph{Forwarding for Final Version of the Chapters.}
%
% The following forwarding files |cdocsfn1.tex| and |cdocsfn2.tex|
% (with identical content)
% compile the final versions of the child documents
% |cdocsch1.tex| and |cdocsch2.tex|, respectively:
%\iffalse
%<*samplefinal>
%\fi
%    \begin{macrocode}
\def\version{final}
% \iffalse
%
% childdoc.dtx Copyright (C) 2017-2018 Niklas Beisert
%
% This work may be distributed and/or modified under the
% conditions of the LaTeX Project Public License, either version 1.3
% of this license or (at your option) any later version.
% The latest version of this license is in
%   http://www.latex-project.org/lppl.txt
% and version 1.3 or later is part of all distributions of LaTeX
% version 2005/12/01 or later.
%
% This work has the LPPL maintenance status `maintained'.
%
% The Current Maintainer of this work is Niklas Beisert.
%
% This work consists of the files childdoc.dtx and childdoc.ins
% and the derived files childdoc.def and cdocsamp.tex with
% cdocsch1.tex, cdocsch2.tex, cdocsdrf.tex, cdocsfn1.tex, cdocsfn2.tex.
%
%<package>\ifdefined\childdocmain\endinput\fi
%<package>\ProvidesFile{childdoc.def}[2018/12/30 v2.0 child document driver]
%<samplemain>\ProvidesFile{cdocsamp.tex}[2018/12/30 v2.0 sample for childdoc]
%<*driver>
%\ProvidesFile{childdoc.drv}[2018/12/30 v2.0 childdoc reference manual file]
\PassOptionsToClass{10pt,a4paper}{article}
\documentclass{ltxdoc}

\usepackage[margin=35mm]{geometry}
\usepackage{hyperref}
\usepackage{hyperxmp}
\usepackage[usenames]{color}

\hypersetup{colorlinks=true}
\hypersetup{pdfstartview=FitH}
\hypersetup{pdfpagemode=UseNone}
\hypersetup{pdfsource={}}
\hypersetup{pdflang={en-UK}}
\hypersetup{pdfcopyright={Copyright 2017-2018 Niklas Beisert.
  This work may be distributed and/or modified under the
  conditions of the LaTeX Project Public License, either version 1.3
  of this license or (at your option) any later version.}}
\hypersetup{pdflicenseurl={http://www.latex-project.org/lppl.txt}}
\hypersetup{pdfcontactaddress={ETH Zurich, ITP, HIT K,
  Wolfgang-Pauli-Strasse 27}}
\hypersetup{pdfcontactpostcode={8093}}
\hypersetup{pdfcontactcity={Zurich}}
\hypersetup{pdfcontactcountry={Switzerland}}
\hypersetup{pdfcontactemail={nbeisert@itp.phys.ethz.ch}}
\hypersetup{pdfcontacturl={http://people.phys.ethz.ch/\xmptilde nbeisert/}}

\newcommand{\secref}[1]{\hyperref[#1]{section \ref*{#1}}}

\parskip1ex
\parindent0pt
\let\olditemize\itemize
\def\itemize{\olditemize\parskip0pt}

\begin{document}

\title{The \textsf{childdoc} Package}
\hypersetup{pdftitle={The childdoc Package}}
\author{Niklas Beisert\\[2ex]
  Institut f\"ur Theoretische Physik\\
  Eidgen\"ossische Technische Hochschule Z\"urich\\
  Wolfgang-Pauli-Strasse 27, 8093 Z\"urich, Switzerland\\[1ex]
  \href{mailto:nbeisert@itp.phys.ethz.ch}
  {\texttt{nbeisert@itp.phys.ethz.ch}}}
\hypersetup{pdfauthor={Niklas Beisert}}
\hypersetup{pdfsubject={Manual for the LaTeX2e Package childdoc}}
\date{30 December 2018, \textsf{v2.0}}
\maketitle

\begin{abstract}\noindent
\textsf{childdoc} is a \LaTeXe{} package
that enables the direct compilation
of document sections included by |\include|
to individual files.
\end{abstract}

\begingroup
\parskip0ex
\tableofcontents
\endgroup

%%%%%%%%%%%%%%%%%%%%%%%%%%%%%%%%%%%%%%%%%%%%%%%%%%%%%%%%%%%%%%%%%%%%%%%%%%%%%%%%
%%%%%%%%%%%%%%%%%%%%%%%%%%%%%%%%%%%%%%%%%%%%%%%%%%%%%%%%%%%%%%%%%%%%%%%%%%%%%%%%
\section{Introduction}

\LaTeX{} provides a mechanism to structure a large document (such as a book)
into a main file and several child files (containing the chapters)
using the |\include| command.
This mechanism is beneficial for documents
which span hundreds of pages in order to
make the source file(s) more manageable.
Moreover, compilation can be restricted to
selected child files by means of the |\includeonly| command.
The latter feature can be used to reduce the compilation time while editing
(this was significantly more useful in the earlier days of \LaTeX{})
or to generate a smaller document which is easier to navigate.
Another application of |\includeonly| is to generate
documents consisting of selected parts of the complete document.

However, there are a few drawbacks of the plain |\include| mechanism:
\begin{itemize}
\item
The child files cannot be compiled on their own,
they can only be compiled via the main file.
A naive editing environment
(such as a text editor with an option
to have the current file processed by \LaTeX)
may require one to switch to the main file before compiling;
attempting to compile the child file produces errors.
\item
The main file must be modified (each time)
to adjust the |\includeonly| command
to the present needs. This easily leaves the main file in a messy state.
\item
The generated document will always carry the filename
of the main document. This is inconvenient if
several child files are to be compiled and
to be kept for distribution.
\end{itemize}

The present package provides a simple interface
to make child files individually compilable by \LaTeX{}.
Compiling a child file then has the same effect as compiling
the main file with an |\includeonly| command
to select the appropriate child.
Moreover the generated document will carry the name of the child
rather than the main file.
This resolves all three above issues.

This feature is meant to make the editing of books,
thesis documents and lecture notes somewhat more convenient.
However, the package can also be used efficiently for
composing a series of documents (such as exercise sheets)
which are typically distributed individually.
It then assists the author in generating the individual documents
(potentially in different versions)
as well as a document containing the collected series.
Another application is in developing style files
or other kinds of included material
where compilation of the style file could redirect
to a sample or test file.

%%%%%%%%%%%%%%%%%%%%%%%%%%%%%%%%%%%%%%%%%%%%%%%%%%%%%%%%%%%%%%%%%%%%%%%%%%%%%%%%
%%%%%%%%%%%%%%%%%%%%%%%%%%%%%%%%%%%%%%%%%%%%%%%%%%%%%%%%%%%%%%%%%%%%%%%%%%%%%%%%
\section{Usage}

First of all, the package \textsf{childdoc} is \emph{not} a standard
\LaTeXe{} |.sty| style file! Therefore it needs to be invoked in
a non-standard way.

%%%%%%%%%%%%%%%%%%%%%%%%%%%%%%%%%%%%%%%%%%%%%%%%%%%%%%%%%%%%%%%%%%%%%%%%%%%%%%%%
\subsection{Included Files}
\label{sec:include}

%%%%%%%%%%%%%%%%%%%%%%%%%%%%%%%%%%%%%%%%
\DescribeMacro{\childdocmain}
To use the package, add the commands
\begin{center}
\begin{tabular}{l}
|\input{childdoc.def}|\\
|\childdocmain{}|\\
\end{tabular}
\end{center}
at the very top of the main \LaTeX{} file,
in particular \emph{before} the |\documentclass| statement!
The argument of |\childdocmain| should be left empty
(but it must be present).

%%%%%%%%%%%%%%%%%%%%%%%%%%%%%%%%%%%%%%%%
\DescribeMacro{\childdocof}
Furthermore, add the commands
\begin{center}
\begin{tabular}{l}
|\input{childdoc.def}|\\
|\childdocof{|\textit{main}|}|\\
\end{tabular}
\end{center}
at the top of every child file \textit{child}
which is included by |\include{|\textit{child}|}|
from within the main file
(or at least for those files to be compiled individually).
The argument \textit{main} must be the filename of the main file.

There are a couple of
considerations in setting up the main and child documents:

%%%%%%%%%%%%%%%%%%%%%%%%%%%%%%%%%%%%%%%%
\paragraph{Restrictions.}

Please note the following restrictions:
\begin{itemize}
\item
|\childdocmain| must be called with one argument \textit{main}
to ensure compatibility with earlier version of the package.
It must either be empty (|\childdocmain{}|)
or precisely match the filename of the main file in which it is specified.
See \secref{sec:detection} for further information.
\item
The filename \textit{main} must be specified without the |.tex| extension.
\item
The filename \textit{main} is case sensitive
(even in case-insensitive file systems)
due to internal string comparison.
\item
The argument \textit{main} should be fully expanded, it cannot be a macro.
\item
Subdirectories and special characters should be avoided in filenames.
\item
The command |\childdocmain{|\textit{main}|}| must be followed by a whitespace.
It should not be followed immediately by another command
or by a comment mark `|%|'.
This is because the \TeX{} parser reads the token immediately following
the argument of |\childdocmain| and puts it
at the beginning of every child section;
however, a white\-space is ignored.
\end{itemize}

%%%%%%%%%%%%%%%%%%%%%%%%%%%%%%%%%%%%%%%%
\paragraph{Content of Main File.}

It is advisable to place all content in the child files included by |\include|.
Any output contained in the main file will appear in all child documents
unless suppressed manually;
it cannot be suppressed automatically by the |\includeonly| directive
and thus should normally be avoided.
A method to include some content in the main file
by means of conditional processing is described in \secref{sec:conditional}.

%%%%%%%%%%%%%%%%%%%%%%%%%%%%%%%%%%%%%%%%
\paragraph{Page Numbering.}

When only a part of the document is compiled,
the appropriate numbering of pages
(as well as other status parameters)
is determined from the |.aux| files.
The latter contain information from previous passes.
However this information needs to propagate through
all intermediate child documents.
Therefore the page numbering in child documents may well
be inconsistent until the complete document is compiled at least once.

A useful (if unconventional) way to always ensure a consistent
page numbering is to restart the numbering in each child document
and denote the pages by `\textit{child}|.|\textit{page}'
where \textit{child} represents the chapter/section number of the child file.
This can be achieved by the command
|\numberwithin{page}{|\textit{child}|}|
of the \textsf{amsmath} package
where \textit{child} can be |chapter| or |section|
depending on the chosen structuring.
Alternatively, one can modify the macro |\thepage| appropriately
and reset the counter |page| at the start of each child file.

%%%%%%%%%%%%%%%%%%%%%%%%%%%%%%%%%%%%%%%%%%%%%%%%%%%%%%%%%%%%%%%%%%%%%%%%%%%%%%%%
\subsection{Conditional Processing}
\label{sec:conditional}

The package provides a mechanism to compile different versions
of a document. To customise the versions further some conditional processing
can come in handy to distinguish which version is being compiled.
The package provides two macros to describe the compilation context:

%%%%%%%%%%%%%%%%%%%%%%%%%%%%%%%%%%%%%%%%
\DescribeMacro{\ifchilddoc}
The conditional |\ifchilddoc| distinguishes between the compilation of
child documents and the main document:
%
\begin{center}
|\ifchilddoc |\textit{child-code}| |[|\||else |\textit{main-code}]| \||fi|
\end{center}

%%%%%%%%%%%%%%%%%%%%%%%%%%%%%%%%%%%%%%%%
\DescribeMacro{\childdocname}
\DescribeMacro{\childdocjob}
The macro |\childdocname| contains the filename (without extension)
of the main or child file being processed.
Note that |\childdocjob| will always contain the name of the main file.

%%%%%%%%%%%%%%%%%%%%%%%%%%%%%%%%%%%%%%%%
\paragraph{Title Page.}

Conditional processing can be used to include a title or banner page
in the main document when proper precautions are taken.
Importantly, the code in the main file should ensure that the page counter
(as well as other status parameters which are stored in the |.aux| files)
takes the same value after the conditional processing.
Otherwise the page numbers may take divergent values
depending on which part is compiled.

For example, a title page could be declared by:
%
\begin{center}
\begin{tabular}{l}
|\ifchilddoc\||else|\\
|\addtocounter{page}{-1}|\\
\textit{code for title page}\\
|\newpage|\\
|\||fi|
\end{tabular}
\end{center}
%
A banner page for the child documents can be generated by:
%
\begin{center}
\begin{tabular}{l}
|\ifchilddoc|\\
|\addtocounter{page}{-1}|\\
\textit{code for banner page}\\
|\newpage|\\
|\||fi|
\end{tabular}
\end{center}
%
Here one could write a message such as:
\begin{center}
|This is the part \childdocname{} of \childdocjob{}.|
\end{center}

%%%%%%%%%%%%%%%%%%%%%%%%%%%%%%%%%%%%%%%%%%%%%%%%%%%%%%%%%%%%%%%%%%%%%%%%%%%%%%%%
\subsection{Flags}
\label{sec:flags}

The package makes it easy to generate different versions
of the main or child documents.
To this end compilation flags can be defined
and assigned different default values.
They will be particularly useful in conjunction
with the forwarding mechanism described in \secref{sec:forward}.

For example, it may be useful to have a flag |\version|
which can be set to |draft| or |final|.
The document source will contain some conditional code
depending on the value of |\version|.
Suppose further, the flag should default to |final| for the main file
and to |draft| for child files
which is a natural assignment for editing the document.
This is achieved by placing the following code
in the preamble of the main document
(below the |\childdocmain| directive):
%
\begin{center}
\begin{tabular}{l}
|\ifchilddoc|\\
|\providecommand{\version}{draft}|\\
|\||else|\\
|\providecommand{\version}{final}|\\
|\||fi|
\end{tabular}
\end{center}
%
The definition by |\providecommand| makes sure
that previous definitions are not overwritten.
Further statements |\providecommand{\version}{...}|
can thus be added before the above code to override it.

For the main file, one might add a line
(between |\childdocmain| and the above block)
%
\begin{center}
|%\ifchilddoc\||else\providecommand{\version}{draft}\||fi|
\end{center}
%
which can be uncommented to produce a draft version.
Likewise one can add a line to the very top of a child file
(above the |\childdocof{|\textit{main}|}| directive)
%
\begin{center}
|%\providecommand{\version}{final}|
\end{center}
%
which can be uncommented to produce the final version of this child document.

%%%%%%%%%%%%%%%%%%%%%%%%%%%%%%%%%%%%%%%%%%%%%%%%%%%%%%%%%%%%%%%%%%%%%%%%%%%%%%%%
\subsection{Forwarding}
\label{sec:forward}

Different versions of the main or child documents
using compilation flags as described in \secref{sec:flags}
can be (permanently) stored in different files
for convenient compilation, viewing and distribution.
To this end, the package defines a command
to pass on compilation to a different file:

%%%%%%%%%%%%%%%%%%%%%%%%%%%%%%%%%%%%%%%%
\DescribeMacro{\childdocforward}
The command |\childdocforward| redirects processing to
another source file:
%
\begin{center}
\begin{tabular}{l}
|\input{childdoc.def}|\\
|\childdocforward[|\textit{main}|]{|\textit{dest}|}|\\
\end{tabular}
\end{center}
%
The argument \textit{dest} is the destination file
(without extension).
It should be the main file or one of the child files.
Note that further \textsf{childdoc} directives
such as |\childdocof| and |\childdocforward|
in the indicated file will be processed in this form.
The optional argument \textit{main}
passes on directly to the main file \textit{main}
while pretending to compile the child \textit{dest}.
This form behaves as if \textit{dest}
issues |\childdocof{|\textit{main}|}| right away,
and no further \textsf{childdoc} directives will be processed.

%%%%%%%%%%%%%%%%%%%%%%%%%%%%%%%%%%%%%%%%
\DescribeMacro{\...prefix}
In the alternative form |\childdocforwardprefix|,
%
\begin{center}
\begin{tabular}{l}
|\input{childdoc.def}|\\
|\childdocforwardprefix[|\textit{main}|]{|\textit{prefix}|}{|\textit{dest}|}|
\end{tabular}
\end{center}
%
the destination file is determined by a pattern
depending on the current file:
To make this work, the current file must be called
`{\textit{prefix}\hspace{0.2em}\textit{suffix}}'
with \textit{prefix} matching precisely the argument.
Processing is then passed on to the file
`{\textit{dest}\hspace{0.2em}\textit{suffix}}'.
Surely, the same effect is achieved by
directly specifying the
argument `{\textit{dest}\hspace{0.2em}\textit{suffix}}'
in the first form.
However, that requires to set up a different file
for each child. With the alternative form of the command
all these files can have exactly the same content
which simplifies setting them up and maintaining them.

For example, the following file |draft.tex|
with a compilation flag |\version| as described in \secref{sec:flags}
compiles the main document as a draft:
%
\begin{center}
\begin{tabular}{l}
|\def\version{draft}|\\
|\input{childdoc.def}|\\
|\childdocforward{|\textit{main}|}|
\end{tabular}
\end{center}
%
Likewise, the following files |final|\textit{nn}|.tex|
compile the final version of the child document
|child|\textit{nn}|.tex|:
%
\begin{center}
\begin{tabular}{l}
|\def\version{final}|\\
|\input{childdoc.def}|\\
|\childdocforwardprefix{final}{child}|
\end{tabular}
\end{center}
%

Note that when several versions of a main file and/or of each child file
are to be generated, it may be convenient to set up a |Makefile| or
shell script to automatise the process.

%%%%%%%%%%%%%%%%%%%%%%%%%%%%%%%%%%%%%%%%%%%%%%%%%%%%%%%%%%%%%%%%%%%%%%%%%%%%%%%%
\subsection{Command Line Processing}
\label{sec:commandline}

The effect of redirection files can also be achieved by invoking
the \LaTeX{} compiler with a more elaborate command line.
Most conveniently this should be done as part
of a shell script or a |Makefile|.

When using \textsf{childdoc} in the main file, the following
command lines effectively perform a redirection
(note that depending on the shell being used,
backslashes may have to be doubled: `|\|' $\to$ `|\\|'):
%
\begin{center}
|... -jobname "|\textit{target}|" |\\|"|[\textit{flags}]%
|\input{childdoc.def}\childdocforward[|\textit{main}|]{|\textit{dest}|}"|
\end{center}
%
Here \textit{target} is the name of the output file,
\textit{main} is the name of the main file
and \textit{dest} is the name of the main or child file to be processed
(all filenames without extensions).
The optional argument \textit{main} can be omitted
if \textit{main} matches \textit{dest}.
Optionally, compilation \textit{flags} can be defined via |\def| commands.
This command line makes the \TeX{} engine believe
it is compiling the file \textit{target}
whose content is specified as the latter parameter.
The provided code then forwards the processing to
\textit{main} or \textit{dest} as described in \secref{sec:forward}.

%%%%%%%%%%%%%%%%%%%%%%%%%%%%%%%%%%%%%%%%%%%%%%%%%%%%%%%%%%%%%%%%%%%%%%%%%%%%%%%%
\subsection{Include by Input}
\label{sec:input}

Including child documents by |\include| has some restrictions by design.
Most notably, the content of a child document always occupies
its own set of pages; pages cannot be shared between child documents.
Usually, this behaviour makes perfect sense
because each child document contain an essential part of the document.
However, in some situations it may be desirable to compose
a document from a collection of parts
without having mandatory page breaks between then.
For this case, the package
provides a mechanism to include parts
by |\input| which can also be processed individually.
However, by construction this mechanism
requires manual handling of the content to be output.

%%%%%%%%%%%%%%%%%%%%%%%%%%%%%%%%%%%%%%%%
\DescribeMacro{\ifchilddocmanual}
The main file should be prepared as usual, see \secref{sec:include}.
However, the document body must make a distinction
between processing of an individual part and of the main document, e.g.:
%
\begin{center}
\begin{tabular}{l}
|\ifchilddocmanual|\\
|\input{\childdocname}|\\
|\||else|\\
\textit{document body with }|\input{|\textit{part}|}|\\
|\||fi|
\end{tabular}
\end{center}
%
The conditional |\ifchilddocmanual| is true whenever
a part to be included by |\input| is being compiled,
and the name of the part is stored in |\childdocname|.

%%%%%%%%%%%%%%%%%%%%%%%%%%%%%%%%%%%%%%%%
\DescribeMacro{\childdocby}
Each part to be included by |\input| should start with:
%
\begin{center}
\begin{tabular}{l}
|\input{childdoc.def}|\\
|\childdocby{|\textit{main}|}|\\
\end{tabular}
\end{center}
%
The directive |\childdocby| is similar to |\childdocof|
described in \secref{sec:include},
but the subsequent selection of content must be done manually.
To that end, both |\ifchilddoc| and |\ifchilddocmanual|
will be true upon processing of a part,
and the name of the part is stored in |\childdocname|.
Note that |\jobname| will be set to the filename of the current part
so that each part receives an individual |.aux| file
that does not interfere with the |.aux| file(s) of the main document.
This behaviour can be altered by the alternative form
|\childdocby[*]{|\textit{main}|}| (with a non-empty optional argument)
which uses the |.aux| file of the main document
by setting |\jobname| to \textit{main}.

%%%%%%%%%%%%%%%%%%%%%%%%%%%%%%%%%%%%%%%%%%%%%%%%%%%%%%%%%%%%%%%%%%%%%%%%%%%%%%%%
\subsection{Driver Development}
\label{sec:driver}

The \textsf{childdoc} mechanism can also be use for the development
of definition files such as \LaTeX{} styles or classes.
This case differs from the above setup with multiple parts
included by |\include| in that no |\includeonly| should be invoked.
This can be achieved by starting the include file
(before |\ProvidesPackage|) with:
%
\begin{center}
\begin{tabular}{l}
|\input{childdoc.def}|\\
|\childdocforward{|\textit{main}|}|\\
\end{tabular}
\end{center}
%
or alternatively with:
%
\begin{center}
\begin{tabular}{l}
|\input{childdoc.def}|\\
|\childdocby{|\textit{main}|}|\\
\end{tabular}
\end{center}
%
Both forms have slightly different effects as described above.
The main file is prepared as usual, see \secref{sec:include}.

%%%%%%%%%%%%%%%%%%%%%%%%%%%%%%%%%%%%%%%%%%%%%%%%%%%%%%%%%%%%%%%%%%%%%%%%%%%%%%%%
\subsection{Legacy Detection}
\label{sec:detection}

The directive |\childdocmain| in the main file can detect
whether the complete document or merely a child is to be compiled
even without using the directive |\childdocof|.
This method is deprecated because it is less robust
and there is no compelling reason to use it;
it is merely provided for backward compatibility
and it may be removed in future versions.

If the detection mechanism is to be used,
it is mandatory to correctly specify
the filename of the main file as the argument of |\childdocmain|:
%
\begin{center}
\begin{tabular}{l}
|\input{childdoc.def}|\\
|\childdocmain{|\textit{main}|}|\\
\end{tabular}
\end{center}
%
If |\jobname| does not match the argument \textit{main} of |\childdocmain|,
it is assumed that |\jobname| points to the child file to be compiled.
When using |\childdocmain| with the main file specified as argument,
it suffices to start a child file
with just |\input{|\textit{main}|}|
without loading of the package and using |\childdocof|.
If instead all processing is done
with the appropriate \textsf{childdoc} directives,
the argument of \textit{main} of |\childdocmain| can be empty.

An alternative version of the command line processing described
in \secref{sec:commandline} using the detection mechanism reads:
%
\begin{center}
|... -jobname "|\textit{target}|" "|[\textit{flags}]%
[|\def\jobname{|\textit{dest}|}|]|\input{|\textit{main}|}"|
\end{center}

%%%%%%%%%%%%%%%%%%%%%%%%%%%%%%%%%%%%%%%%%%%%%%%%%%%%%%%%%%%%%%%%%%%%%%%%%%%%%%%%
\subsection{Manual Code}
\label{sec:manual}

In case one cannot be certain whether the definitions file |childdoc.def|
is installed on the target \TeX{} distribution
and one prefers not to ship it,
it is conceivable to paste a few relevant commands into the sources.

To that end, drop all statements |\input{childdoc.def}|
and perform the replacements as outlined below.
Instead of |\childdocmain{|\textit{main}|}| add the following code
to the top of the main file:
%
\begin{center}
\begin{tabular}{l}
|\||ifdefined\childdocname\endinput\||fi\newif\ifchilddoc|\\
|\edef\childdocname{\scantokens\expandafter{\jobname\noexpand}}|\\
|\def\childdocmain{|\textit{main}|}\||ifx\childdocmain\childdocname\||else|\\
|\childdoctrue\includeonly{\childdocname}\let\jobname\childdocmain\||fi|\\
\end{tabular}
\end{center}
%
Instead of |\childdocof{|\textit{main}|}| just include the main file
at the top of each child file:
%
\begin{center}
|\input{|\textit{main}|}|
\end{center}
%
A simple redirection |\childdocforward{|\textit{dest}|}| is achieved by:
%
\begin{center}
|\def\jobname{|\textit{dest}|}\input{\jobname}|
\end{center}
%
The redirection with prefix
|\childdocforwardprefix[|\textit{prefix}|]{|\textit{dest}|}|
is accomplished by:
%
\begin{center}
\begin{tabular}{l}
|{\edef\jobname{\scantokens\expandafter{\jobname\noexpand}}|\\
|\def\redirectjob |\textit{prefix}|#1~~~{\gdef\jobname{|\textit{dest}|#1}}|\\
|\expandafter\redirectjob\jobname~~~}\input{\jobname}|
\end{tabular}
\end{center}

In an alternative approach,
child documents can be compiled by a specific command line
without additional code or specific definitions:
%
\begin{center}
|... -jobname "|\textit{target}|" "|[\textit{flags}]%
|\includeonly{|\textit{dest}|}\input{|\textit{main}|}"|
\end{center}
%

%%%%%%%%%%%%%%%%%%%%%%%%%%%%%%%%%%%%%%%%%%%%%%%%%%%%%%%%%%%%%%%%%%%%%%%%%%%%%%%%
%%%%%%%%%%%%%%%%%%%%%%%%%%%%%%%%%%%%%%%%%%%%%%%%%%%%%%%%%%%%%%%%%%%%%%%%%%%%%%%%
\section{Information}

%%%%%%%%%%%%%%%%%%%%%%%%%%%%%%%%%%%%%%%%%%%%%%%%%%%%%%%%%%%%%%%%%%%%%%%%%%%%%%%%
\subsection{Copyright}

Copyright \copyright{} 2017--2018 Niklas Beisert

This work may be distributed and/or modified under the
conditions of the \LaTeX{} Project Public License, either version 1.3
of this license or (at your option) any later version.
The latest version of this license is in
  \url{http://www.latex-project.org/lppl.txt}
and version 1.3 or later is part of all distributions of \LaTeX{}
version 2005/12/01 or later.

This work has the LPPL maintenance status `maintained'.

The Current Maintainer of this work is Niklas Beisert.

This work consists of the files |README.txt|, |childdoc.ins| and |childdoc.dtx|
as well as the derived files |childdoc.def|, |cdocsamp.tex|
with |cdocsch1.tex|, |cdocsch2.tex|, |cdocspt3.tex|, |cdocspt4.tex|,
|cdocsdrf.tex|, |cdocsfn1.tex|, |cdocsfn2.tex|
as well as |childdoc.pdf|.

%%%%%%%%%%%%%%%%%%%%%%%%%%%%%%%%%%%%%%%%%%%%%%%%%%%%%%%%%%%%%%%%%%%%%%%%%%%%%%%%
\subsection{Files and Installation}

The package consists of the files:
%
\begin{center}
\begin{tabular}{ll}
    |README.txt|   & readme file \\
    |childdoc.ins| & installation file \\
    |childdoc.dtx| & source file \\
    |childdoc.def| & definition file \\
    |cdocsamp.tex| & sample main file \\
    |cdocsch1.tex| & sample include file \\
    |cdocsch2.tex| & sample include file \\
    |cdocspt3.tex| & sample part file \\
    |cdocspt4.tex| & sample part file \\
    |cdocsdrf.tex| & sample redirection file \\
    |cdocsfn1.tex| & sample redirection file \\
    |cdocsfn2.tex| & sample redirection file \\
    |childdoc.pdf| & manual
\end{tabular}
\end{center}
%
The distribution consists of the files
|README.txt|, |childdoc.ins| and |childdoc.dtx|.
%
\begin{itemize}
\item
Run (pdf)\LaTeX{} on |childdoc.dtx|
to compile the manual |childdoc.pdf| (this file).
\item
Run \LaTeX{} on |childdoc.ins| to create the definitions file |childdoc.def|
and the sample |cdocsamp.tex| with include files
|cdocsch1.tex|, |cdocsch2.tex|, |cdocspt3.tex|, |cdocspt4.tex|,
|cdocsdrf.tex|, |cdocsfn1.tex|, |cdocsfn2.tex|.
Then copy the file |childdoc.def| to an appropriate directory of your \LaTeX{}
distribution, e.g.\ \textit{texmf-root}|/tex/latex/childdoc|.
\end{itemize}

%%%%%%%%%%%%%%%%%%%%%%%%%%%%%%%%%%%%%%%%%%%%%%%%%%%%%%%%%%%%%%%%%%%%%%%%%%%%%%%%
\subsection{Related CTAN Packages}

There are several other packages which offer a similar functionality:
%
\begin{itemize}
\item
The packages
\href{http://ctan.org/pkg/docmute}{\textsf{docmute}},
\href{http://ctan.org/pkg/includex}{\textsf{includex}} and
\href{http://ctan.org/pkg/standalone}{\textsf{standalone}}
provide commands to include only the document body of
a child file thus allowing both files to be compiled individually.
\item
The packages \href{http://ctan.org/pkg/subdocs}{\textsf{subdocs}}
and \href{http://ctan.org/pkg/subfiles}{\textsf{subfiles}}
provide structures in which the main and child documents can be
encapsulated and allowing them to be compiled individually.
The inclusion mechanism is different from the conventional |\include|.
\item
The package \href{http://ctan.org/pkg/combine}{\textsf{combine}}
is an elaborate solution to combine several documents into one.
\end{itemize}
%
See also the CTAN topic \href{http://ctan.org/topic/subdocs}{\textsf{subdocs}}
for further related packages.
The present package differs from the above solutions in that
a document structure constructed with the conventional |\include| mechanism
just needs two extra commands at the top of every file
such that all constituent files can be compiled individually.

%%%%%%%%%%%%%%%%%%%%%%%%%%%%%%%%%%%%%%%%%%%%%%%%%%%%%%%%%%%%%%%%%%%%%%%%%%%%%%%%
%\subsection{Feature Suggestions}
%
%The following is a list of features which may be useful for future
%versions of this package:
%%
%\begin{itemize}
%\item
%\ldots
%\end{itemize}

%%%%%%%%%%%%%%%%%%%%%%%%%%%%%%%%%%%%%%%%%%%%%%%%%%%%%%%%%%%%%%%%%%%%%%%%%%%%%%%%
\subsection{Revision History}

%%%%%%%%%%%%%%%%%%%%%%%%%%%%%%%%%%%%%%%%
\paragraph{v2.0:} 2018/12/30

\begin{itemize}
\item
immediate forward processing
\item
added |\childdocby| mechanism
\item
manual restructured
\end{itemize}

%%%%%%%%%%%%%%%%%%%%%%%%%%%%%%%%%%%%%%%%
\paragraph{v1.6:} 2018/01/17

\begin{itemize}
\item
application for development of include files
\item
corrections to manual
\end{itemize}

%%%%%%%%%%%%%%%%%%%%%%%%%%%%%%%%%%%%%%%%
\paragraph{v1.5:} 2017/05/21

\begin{itemize}
\item
more complete structuring introduced
\item
|\childdocof| introduced
\item
|\childdoc| renamed to |\childdocmain|
\item
|\childredirect| renamed to |\childdocforward| and |\childdocforwardprefix|
and functionality expanded
\end{itemize}

%%%%%%%%%%%%%%%%%%%%%%%%%%%%%%%%%%%%%%%%
\paragraph{v1.0:} 2017/04/27

\begin{itemize}
\item
manual and install package
\item
first version published on CTAN
\end{itemize}

%%%%%%%%%%%%%%%%%%%%%%%%%%%%%%%%%%%%%%%%
\paragraph{v0.6:} 2017/04/26

\begin{itemize}
\item
redirection mechanism added
\end{itemize}

%%%%%%%%%%%%%%%%%%%%%%%%%%%%%%%%%%%%%%%%
\paragraph{v0.5:} 2017/04/26

\begin{itemize}
\item
functionality in definition file
\end{itemize}


%%%%%%%%%%%%%%%%%%%%%%%%%%%%%%%%%%%%%%%%%%%%%%%%%%%%%%%%%%%%%%%%%%%%%%%%%%%%%%%%
%%%%%%%%%%%%%%%%%%%%%%%%%%%%%%%%%%%%%%%%%%%%%%%%%%%%%%%%%%%%%%%%%%%%%%%%%%%%%%%%
%%%%%%%%%%%%%%%%%%%%%%%%%%%%%%%%%%%%%%%%%%%%%%%%%%%%%%%%%%%%%%%%%%%%%%%%%%%%%%%%
\appendix

\settowidth\MacroIndent{\rmfamily\scriptsize 000\ }

 \DocInput{childdoc.dtx}

\end{document}
%</driver>
% \fi
%
% %%%%%%%%%%%%%%%%%%%%%%%%%%%%%%%%%%%%%%%%%%%%%%%%%%%%%%%%%%%%%%%%%%%%%%%%%%%%%%
% %%%%%%%%%%%%%%%%%%%%%%%%%%%%%%%%%%%%%%%%%%%%%%%%%%%%%%%%%%%%%%%%%%%%%%%%%%%%%%
% \section{Sample}
%\iffalse
%<*samplemain>
%\fi
%
% The following presents a sample document
% with two chapters, two parts, a title page,
% a compile flag as well as three forwarding files to set the flag.
% It consists of eight |.tex| files:
% \begin{center}
% \begin{tabular}{ll}
% |cdocsamp.tex|&main file\\
% |cdocsch1.tex|&include file for chapter 1\\
% |cdocsch2.tex|&include file for chapter 2\\
% |cdocspt3.tex|&include file for part 3\\
% |cdocspt4.tex|&include file for part 4\\
% |cdocsdrf.tex|&forwarding file for main file in draft mode\\
% |cdocsfi1.tex|&forwarding file for final version of chapter 1\\
% |cdocsfi2.tex|&forwarding file for final version of chapter 2\\
% \end{tabular}
% \end{center}
% Each of the eight files can be compiled directly by the \LaTeX{} compiler.
%
% %%%%%%%%%%%%%%%%%%%%%%%%%%%%%%%%%%%%%%
% \paragraph{Main File.}
%
% The main file is called |cdocsamp.tex|.
%
% Load the \textsf{childdoc} definitions and
% declare the filename for the main document:
%    \begin{macrocode}
\input{childdoc.def}
\childdocmain{}
%    \end{macrocode}

% Optional override for |\version| flag:
%    \begin{macrocode}
%%\ifchilddoc\else\providecommand{\version}{draft}\fi
%    \end{macrocode}

% Define the default values for the |\version| flag
% (|final| for the main file and |draft| for childs):
%    \begin{macrocode}
\ifchilddoc
\providecommand{\version}{draft}
\else
\providecommand{\version}{final}
\fi
%    \end{macrocode}

% Load the standard document class:
%    \begin{macrocode}
\documentclass[12pt]{article}
%    \end{macrocode}

% Start the document body:
%    \begin{macrocode}
\begin{document}
%    \end{macrocode}

% Declare a title page.
% Print title, part of document being processed and version flag:
%    \begin{macrocode}
\addtocounter{page}{-1}
\begin{center}
{\LARGE\bfseries{}childdoc example\par}
\vspace{1cm}
\ifchilddoc
\ifchilddocmanual part\else chapter\fi:
`\childdocname' of `\childdocjob'\par
\else
main document: `\childdocjob'\par
\fi
version: \version\par
\end{center}
\newpage
%    \end{macrocode}

% Manually include selected file,
% otherwise process as usual:
%    \begin{macrocode}
\ifchilddocmanual
\section*{part `\childdocname'}
\input{\childdocname}
\else
%    \end{macrocode}

% Include the two chapters:
%    \begin{macrocode}
\include{cdocsch1}
\include{cdocsch2}
%    \end{macrocode}

% Include the two parts unless only chapters should be displayed:
%    \begin{macrocode}
\ifchilddoc\else
\section{part three}
\input{cdocspt3}
\section{part four}
\input{cdocspt4}
\fi
%    \end{macrocode}

% Process as usual until here:
%    \begin{macrocode}
\fi
%    \end{macrocode}

% End of document body:
%    \begin{macrocode}
\end{document}
%    \end{macrocode}
%\iffalse
%</samplemain>
%\fi
%
% %%%%%%%%%%%%%%%%%%%%%%%%%%%%%%%%%%%%%%
% \paragraph{Chapter Include Files.}
%
% The include files are called |cdocsch1.tex| and |cdocsch2.tex|.
%
%\iffalse
%<*samplechap1|samplechap2>
%\fi

% Optional override for |\version| flag:
%    \begin{macrocode}
%%\providecommand{\version}{final}
%    \end{macrocode}

% Include the main document:
%    \begin{macrocode}
\input{childdoc.def}
\childdocof{cdocsamp}
%    \end{macrocode}

%\iffalse
%</samplechap1|samplechap2>
%\fi
%
%\iffalse
%<*samplechap1>
%\fi
% Some text for chapter 1:
%    \begin{macrocode}
\section{one}
some text in chapter one
%    \end{macrocode}

%\iffalse
%</samplechap1>
%\fi
% Some text for chapter 2:
%\iffalse
%<*samplechap2>
%\fi
%    \begin{macrocode}
\section{two}
more text in chapter two
%    \end{macrocode}

%\iffalse
%</samplechap2>
%\fi
%
% %%%%%%%%%%%%%%%%%%%%%%%%%%%%%%%%%%%%%%
% \paragraph{Part Include Files.}
%
% The include files are called |cdocspt3.tex| and |cdocspt4.tex|.
%
%\iffalse
%<*samplepart3|samplepart4>
%\fi

% Optional override for |\version| flag:
%    \begin{macrocode}
%%\providecommand{\version}{final}
%    \end{macrocode}

% Include the main document:
%    \begin{macrocode}
\input{childdoc.def}
\childdocby{cdocsamp}
%    \end{macrocode}

%\iffalse
%</samplepart3|samplepart4>
%\fi
%
%\iffalse
%<*samplepart3>
%\fi
% Some text for part 3:
%    \begin{macrocode}
some text in part three
%    \end{macrocode}

%\iffalse
%</samplepart3>
%\fi
% Some text for part 4:
%\iffalse
%<*samplepart4>
%\fi
%    \begin{macrocode}
more text in part four
%    \end{macrocode}

%\iffalse
%</samplepart4>
%\fi
%
% %%%%%%%%%%%%%%%%%%%%%%%%%%%%%%%%%%%%%%
% \paragraph{Forwarding for a Complete Draft.}
%
% The following forwarding file |cdocsdrf.tex|
% compiles the main document in draft mode:
%\iffalse
%<*sampledraft>
%\fi
%    \begin{macrocode}
\def\version{draft}
\input{childdoc.def}
\childdocforward{cdocsamp}
%    \end{macrocode}

%\iffalse
%</sampledraft>
%\fi
%
% %%%%%%%%%%%%%%%%%%%%%%%%%%%%%%%%%%%%%%
% \paragraph{Forwarding for Final Version of the Chapters.}
%
% The following forwarding files |cdocsfn1.tex| and |cdocsfn2.tex|
% (with identical content)
% compile the final versions of the child documents
% |cdocsch1.tex| and |cdocsch2.tex|, respectively:
%\iffalse
%<*samplefinal>
%\fi
%    \begin{macrocode}
\def\version{final}
\input{childdoc.def}
\childdocforwardprefix[cdocsamp]{cdocsfn}{cdocsch}
%    \end{macrocode}

%\iffalse
%</samplefinal>
%\fi
%
% %%%%%%%%%%%%%%%%%%%%%%%%%%%%%%%%%%%%%%
% \paragraph{Command Line Processing.}
%
% The following three command lines generate the output files
% |cdocscld|, |cdocscl1| and |cdocscl2|
% which should be identical to
% |cdocsdrf|, |cdocsch1| and |cdocsfn2|, respectively:
% \begin{center}
% \begin{tabular}{l}
% |latex -jobname cdocscld \|\\
% |  "\def\version{draft}\input{childdoc.def}\childdocforward{cdocsamp}"|\\
% |latex -jobname cdocscl1 \|\\
% |  "\input{childdoc.def}\childdocforward[cdocsamp]{cdocsch1}"|\\
% |latex -jobname cdocscl2 \|\\
% |  "\def\version{final}\input{childdoc.def}\childdocforward{cdocsch2}"|
% \end{tabular}
% \end{center}
% Note that the trailing backslash on each first line
% merely continues the input to the second line
% (for convenient cut ant paste).
% Furthermore, the command |latex| can be replaced by any
% of its alternative versions such as |pdflatex|.
%
% %%%%%%%%%%%%%%%%%%%%%%%%%%%%%%%%%%%%%%%%%%%%%%%%%%%%%%%%%%%%%%%%%%%%%%%%%%%%%%
% %%%%%%%%%%%%%%%%%%%%%%%%%%%%%%%%%%%%%%%%%%%%%%%%%%%%%%%%%%%%%%%%%%%%%%%%%%%%%%
% \section{Implementation}
%\iffalse
%<*package>
%\fi
%
% This section describes the definitions file |childdoc.def|.

% The definitions cannot be loaded using |\usepackage| or |\RequirePackage|
% which has a mechanism to prevent loading a style file more than once.
% When loading the definitions by means of |\input|
% multiple instances have to be prevented manually:
%\iffalse
%This code needs to be before the `\ProvidesFile' directive
%which is defined at the beginning of this file.
%Therefore it is also placed there and commented out here.
%</package>
%<*discard>
%\fi
%    \begin{macrocode}
\ifdefined\childdocmain\endinput\fi
%    \end{macrocode}
%\iffalse
%</discard>
%<*package>
%\fi
%
% \macro{\ifchilddoc}
% \macro{\ifchilddocmanual}
% The conditional |\ifchilddoc| tells whether a
% child (true) or main (false) document is being compiled.
% The conditional |\ifchilddocmanual| tells whether
% the |\includeonly| mechanism is used (false) or
% the selection of child files must be performed manually (true).
% The definitions initialise to false:
%    \begin{macrocode}
\newif\ifchilddoc
\newif\ifchilddocmanual
%    \end{macrocode}

% \macro{\childdocname}
% \macro{\childdocjob}
% The macro |\childdocname| stores the name of the main document
% to be compiled. The macro |\childdocjob| stores the name of
% the document on which the \LaTeX{} compiler was originally invoked.
% The content of |\jobname| cannot be compared
% to filenames specified in the source due to different catcodes.
% The following code rescans |\jobname|, stores the result
% in |\childdocname| and saves a copy in |\childdocjob|:
%    \begin{macrocode}
\edef\childdocname{\scantokens\expandafter{\jobname\noexpand}}
\let\childdocjob\childdocname
%    \end{macrocode}

% \macro{\childdocdisable}
% The macro |\childdocdisable| prevents the main file
% from being processed more than once.
% At this stage, the main document command |\childdocmain|
% is assumed to be called once again where it should do nothing.
% Any subsequent call to it should prevent
% a secondary processing of the main document
% It overwrites the forwarding commands
% |\childdocof| and |\childdocforward|
% with empty macros to prevent further inclusions of the main document:
%    \begin{macrocode}
\newcommand{\childdocdisable}
{
  \renewcommand{\childdocmain}[1]{\renewcommand{\childdocmain}[1]{\endinput}}
  \renewcommand{\childdocof}[1]{}
  \renewcommand{\childdocby}[2][]{}
  \renewcommand{\childdocforward}[2][]{}
  \renewcommand{\childdocdisable}{}
}
%    \end{macrocode}

% \macro{\childdocmain}
% The macro |\childdocmain| is to be called at the top of the main file
% with nothing or the main filename (without extension) as argument.
% First, it breaks loops.
% If the argument is not empty and does not match |\childdocname|
% (which is set by the first inclusion of |childdoc.def|),
% |\ifchilddoc| is set to true, |\includeonly| is applied to the child file
% and |\jobname| is set to the main file
% (for proper handling of |.aux| files):
%    \begin{macrocode}
\newcommand{\childdocmain}[1]
{
  \childdocdisable\childdocmain{}
  \if?#1?\else
    \begingroup
      \def\childdoctmp{#1}
      \ifx\childdoctmp\childdocname
        \def\childdoctmp{}
      \else
        \def\childdoctmp
        {
          \childdoctrue
          \includeonly{\childdocname}
          \def\childdocjob{#1}
          \def\jobname{#1}
        }
      \fi
      \expandafter
    \endgroup
    \childdoctmp
  \fi
}
%    \end{macrocode}

% \macro{\childdocof}
% The command |\childdocof| redirects
% compilation to the main file |#1|.
%    \begin{macrocode}
\newcommand{\childdocof}[1]
{
  \childdocdisable
  \childdoctrue
  \includeonly{\childdocname}
  \def\jobname{#1}
  \def\childdocjob{#1}
  \input{#1}
}
%    \end{macrocode}

% \macro{\childdocby}
% The command |\childdocby| ....
%    \begin{macrocode}
\newcommand{\childdocby}[2][]
{
  \childdocdisable
  \childdoctrue
  \childdocmanualtrue
  \if?#1?\else
    \def\jobname{#2}
  \fi
  \def\childdocjob{#2}
  \input{#2}
  \endinput
}
%    \end{macrocode}

% \macro{\childdocforward}
% The command |\childdocforward| redirects
% compilation to the main file or
% (if the optional argument is given) a child file.
% Parameters are set as if the main file
% or a child file starting with |\childdocof| was compiled.
% Then compilation is handed over to the main file:
%    \begin{macrocode}
\newcommand{\childdocforward}[2][]
{
  \begingroup
    \if?#1?
      \def\childdoctmp
      {
        \def\childdocname{#2}
        \def\childdocjob{#2}
        \def\jobname{#2}
        \input{#2}
        \endinput
      }
    \else
      \def\childdoctmp
      {
        \childdocdisable
        \def\childdocname{#2}
        \childdoctrue
        \includeonly{#2}
        \def\childdocjob{#1}
        \def\jobname{#1}
        \input{#1}
        \endinput
      }
    \fi
    \expandafter
  \endgroup
  \childdoctmp
}
%    \end{macrocode}

% \macro{\childdocforwardprefix}
% The command |\childdocforwardprefix| redirects
% compilation to the main or a child file by means of a pattern.
% The prefix |#1| in the current filename is replaced by |#2|
% and the suffix of the current filename is kept
% (it is assumed that the filename does not contain the substring `|~~~|'
% which is used as a delimiter).
% Compilation is handed over to the new file by |\childdocforward|:
%    \begin{macrocode}
\newcommand{\childdocforwardprefix}[3][]
{
  \begingroup
    \def\childdocextract #2##1~~~{\def\childdoctmp{\childdocforward[#1]{#3##1}}}
    \expandafter\childdocextract\childdocname~~~
    \expandafter
  \endgroup
  \childdoctmp
}
%    \end{macrocode}

% \macro{\childdoc}
% The deprecated macro |\childdoc| is a legacy version of |\childdocmain|:
%    \begin{macrocode}
\newcommand{\childdoc}{\childdocmain}
%    \end{macrocode}

% \macro{\childdocredirect}
% The deprecated macro |\childdocredirect| is a legacy version
% of |\childdocforward| and |\childdocforwardprefix|:
%    \begin{macrocode}
\newcommand{\childdocredirect}[2][]
{
  \begingroup
    \if?#1?
      \def\childdoctmp{\childdocforward{#2}}
    \else
      \def\childdoctmp{\childdocforwardprefix{#1}{#2}}
    \fi
    \expandafter
  \endgroup
  \childdoctmp
}
%    \end{macrocode}

%\iffalse
%</package>
%\fi
%
\endinput

\childdocforwardprefix[cdocsamp]{cdocsfn}{cdocsch}
%    \end{macrocode}

%\iffalse
%</samplefinal>
%\fi
%
% %%%%%%%%%%%%%%%%%%%%%%%%%%%%%%%%%%%%%%
% \paragraph{Command Line Processing.}
%
% The following three command lines generate the output files
% |cdocscld|, |cdocscl1| and |cdocscl2|
% which should be identical to
% |cdocsdrf|, |cdocsch1| and |cdocsfn2|, respectively:
% \begin{center}
% \begin{tabular}{l}
% |latex -jobname cdocscld \|\\
% |  "\def\version{draft}% \iffalse
%
% childdoc.dtx Copyright (C) 2017-2018 Niklas Beisert
%
% This work may be distributed and/or modified under the
% conditions of the LaTeX Project Public License, either version 1.3
% of this license or (at your option) any later version.
% The latest version of this license is in
%   http://www.latex-project.org/lppl.txt
% and version 1.3 or later is part of all distributions of LaTeX
% version 2005/12/01 or later.
%
% This work has the LPPL maintenance status `maintained'.
%
% The Current Maintainer of this work is Niklas Beisert.
%
% This work consists of the files childdoc.dtx and childdoc.ins
% and the derived files childdoc.def and cdocsamp.tex with
% cdocsch1.tex, cdocsch2.tex, cdocsdrf.tex, cdocsfn1.tex, cdocsfn2.tex.
%
%<package>\ifdefined\childdocmain\endinput\fi
%<package>\ProvidesFile{childdoc.def}[2018/12/30 v2.0 child document driver]
%<samplemain>\ProvidesFile{cdocsamp.tex}[2018/12/30 v2.0 sample for childdoc]
%<*driver>
%\ProvidesFile{childdoc.drv}[2018/12/30 v2.0 childdoc reference manual file]
\PassOptionsToClass{10pt,a4paper}{article}
\documentclass{ltxdoc}

\usepackage[margin=35mm]{geometry}
\usepackage{hyperref}
\usepackage{hyperxmp}
\usepackage[usenames]{color}

\hypersetup{colorlinks=true}
\hypersetup{pdfstartview=FitH}
\hypersetup{pdfpagemode=UseNone}
\hypersetup{pdfsource={}}
\hypersetup{pdflang={en-UK}}
\hypersetup{pdfcopyright={Copyright 2017-2018 Niklas Beisert.
  This work may be distributed and/or modified under the
  conditions of the LaTeX Project Public License, either version 1.3
  of this license or (at your option) any later version.}}
\hypersetup{pdflicenseurl={http://www.latex-project.org/lppl.txt}}
\hypersetup{pdfcontactaddress={ETH Zurich, ITP, HIT K,
  Wolfgang-Pauli-Strasse 27}}
\hypersetup{pdfcontactpostcode={8093}}
\hypersetup{pdfcontactcity={Zurich}}
\hypersetup{pdfcontactcountry={Switzerland}}
\hypersetup{pdfcontactemail={nbeisert@itp.phys.ethz.ch}}
\hypersetup{pdfcontacturl={http://people.phys.ethz.ch/\xmptilde nbeisert/}}

\newcommand{\secref}[1]{\hyperref[#1]{section \ref*{#1}}}

\parskip1ex
\parindent0pt
\let\olditemize\itemize
\def\itemize{\olditemize\parskip0pt}

\begin{document}

\title{The \textsf{childdoc} Package}
\hypersetup{pdftitle={The childdoc Package}}
\author{Niklas Beisert\\[2ex]
  Institut f\"ur Theoretische Physik\\
  Eidgen\"ossische Technische Hochschule Z\"urich\\
  Wolfgang-Pauli-Strasse 27, 8093 Z\"urich, Switzerland\\[1ex]
  \href{mailto:nbeisert@itp.phys.ethz.ch}
  {\texttt{nbeisert@itp.phys.ethz.ch}}}
\hypersetup{pdfauthor={Niklas Beisert}}
\hypersetup{pdfsubject={Manual for the LaTeX2e Package childdoc}}
\date{30 December 2018, \textsf{v2.0}}
\maketitle

\begin{abstract}\noindent
\textsf{childdoc} is a \LaTeXe{} package
that enables the direct compilation
of document sections included by |\include|
to individual files.
\end{abstract}

\begingroup
\parskip0ex
\tableofcontents
\endgroup

%%%%%%%%%%%%%%%%%%%%%%%%%%%%%%%%%%%%%%%%%%%%%%%%%%%%%%%%%%%%%%%%%%%%%%%%%%%%%%%%
%%%%%%%%%%%%%%%%%%%%%%%%%%%%%%%%%%%%%%%%%%%%%%%%%%%%%%%%%%%%%%%%%%%%%%%%%%%%%%%%
\section{Introduction}

\LaTeX{} provides a mechanism to structure a large document (such as a book)
into a main file and several child files (containing the chapters)
using the |\include| command.
This mechanism is beneficial for documents
which span hundreds of pages in order to
make the source file(s) more manageable.
Moreover, compilation can be restricted to
selected child files by means of the |\includeonly| command.
The latter feature can be used to reduce the compilation time while editing
(this was significantly more useful in the earlier days of \LaTeX{})
or to generate a smaller document which is easier to navigate.
Another application of |\includeonly| is to generate
documents consisting of selected parts of the complete document.

However, there are a few drawbacks of the plain |\include| mechanism:
\begin{itemize}
\item
The child files cannot be compiled on their own,
they can only be compiled via the main file.
A naive editing environment
(such as a text editor with an option
to have the current file processed by \LaTeX)
may require one to switch to the main file before compiling;
attempting to compile the child file produces errors.
\item
The main file must be modified (each time)
to adjust the |\includeonly| command
to the present needs. This easily leaves the main file in a messy state.
\item
The generated document will always carry the filename
of the main document. This is inconvenient if
several child files are to be compiled and
to be kept for distribution.
\end{itemize}

The present package provides a simple interface
to make child files individually compilable by \LaTeX{}.
Compiling a child file then has the same effect as compiling
the main file with an |\includeonly| command
to select the appropriate child.
Moreover the generated document will carry the name of the child
rather than the main file.
This resolves all three above issues.

This feature is meant to make the editing of books,
thesis documents and lecture notes somewhat more convenient.
However, the package can also be used efficiently for
composing a series of documents (such as exercise sheets)
which are typically distributed individually.
It then assists the author in generating the individual documents
(potentially in different versions)
as well as a document containing the collected series.
Another application is in developing style files
or other kinds of included material
where compilation of the style file could redirect
to a sample or test file.

%%%%%%%%%%%%%%%%%%%%%%%%%%%%%%%%%%%%%%%%%%%%%%%%%%%%%%%%%%%%%%%%%%%%%%%%%%%%%%%%
%%%%%%%%%%%%%%%%%%%%%%%%%%%%%%%%%%%%%%%%%%%%%%%%%%%%%%%%%%%%%%%%%%%%%%%%%%%%%%%%
\section{Usage}

First of all, the package \textsf{childdoc} is \emph{not} a standard
\LaTeXe{} |.sty| style file! Therefore it needs to be invoked in
a non-standard way.

%%%%%%%%%%%%%%%%%%%%%%%%%%%%%%%%%%%%%%%%%%%%%%%%%%%%%%%%%%%%%%%%%%%%%%%%%%%%%%%%
\subsection{Included Files}
\label{sec:include}

%%%%%%%%%%%%%%%%%%%%%%%%%%%%%%%%%%%%%%%%
\DescribeMacro{\childdocmain}
To use the package, add the commands
\begin{center}
\begin{tabular}{l}
|\input{childdoc.def}|\\
|\childdocmain{}|\\
\end{tabular}
\end{center}
at the very top of the main \LaTeX{} file,
in particular \emph{before} the |\documentclass| statement!
The argument of |\childdocmain| should be left empty
(but it must be present).

%%%%%%%%%%%%%%%%%%%%%%%%%%%%%%%%%%%%%%%%
\DescribeMacro{\childdocof}
Furthermore, add the commands
\begin{center}
\begin{tabular}{l}
|\input{childdoc.def}|\\
|\childdocof{|\textit{main}|}|\\
\end{tabular}
\end{center}
at the top of every child file \textit{child}
which is included by |\include{|\textit{child}|}|
from within the main file
(or at least for those files to be compiled individually).
The argument \textit{main} must be the filename of the main file.

There are a couple of
considerations in setting up the main and child documents:

%%%%%%%%%%%%%%%%%%%%%%%%%%%%%%%%%%%%%%%%
\paragraph{Restrictions.}

Please note the following restrictions:
\begin{itemize}
\item
|\childdocmain| must be called with one argument \textit{main}
to ensure compatibility with earlier version of the package.
It must either be empty (|\childdocmain{}|)
or precisely match the filename of the main file in which it is specified.
See \secref{sec:detection} for further information.
\item
The filename \textit{main} must be specified without the |.tex| extension.
\item
The filename \textit{main} is case sensitive
(even in case-insensitive file systems)
due to internal string comparison.
\item
The argument \textit{main} should be fully expanded, it cannot be a macro.
\item
Subdirectories and special characters should be avoided in filenames.
\item
The command |\childdocmain{|\textit{main}|}| must be followed by a whitespace.
It should not be followed immediately by another command
or by a comment mark `|%|'.
This is because the \TeX{} parser reads the token immediately following
the argument of |\childdocmain| and puts it
at the beginning of every child section;
however, a white\-space is ignored.
\end{itemize}

%%%%%%%%%%%%%%%%%%%%%%%%%%%%%%%%%%%%%%%%
\paragraph{Content of Main File.}

It is advisable to place all content in the child files included by |\include|.
Any output contained in the main file will appear in all child documents
unless suppressed manually;
it cannot be suppressed automatically by the |\includeonly| directive
and thus should normally be avoided.
A method to include some content in the main file
by means of conditional processing is described in \secref{sec:conditional}.

%%%%%%%%%%%%%%%%%%%%%%%%%%%%%%%%%%%%%%%%
\paragraph{Page Numbering.}

When only a part of the document is compiled,
the appropriate numbering of pages
(as well as other status parameters)
is determined from the |.aux| files.
The latter contain information from previous passes.
However this information needs to propagate through
all intermediate child documents.
Therefore the page numbering in child documents may well
be inconsistent until the complete document is compiled at least once.

A useful (if unconventional) way to always ensure a consistent
page numbering is to restart the numbering in each child document
and denote the pages by `\textit{child}|.|\textit{page}'
where \textit{child} represents the chapter/section number of the child file.
This can be achieved by the command
|\numberwithin{page}{|\textit{child}|}|
of the \textsf{amsmath} package
where \textit{child} can be |chapter| or |section|
depending on the chosen structuring.
Alternatively, one can modify the macro |\thepage| appropriately
and reset the counter |page| at the start of each child file.

%%%%%%%%%%%%%%%%%%%%%%%%%%%%%%%%%%%%%%%%%%%%%%%%%%%%%%%%%%%%%%%%%%%%%%%%%%%%%%%%
\subsection{Conditional Processing}
\label{sec:conditional}

The package provides a mechanism to compile different versions
of a document. To customise the versions further some conditional processing
can come in handy to distinguish which version is being compiled.
The package provides two macros to describe the compilation context:

%%%%%%%%%%%%%%%%%%%%%%%%%%%%%%%%%%%%%%%%
\DescribeMacro{\ifchilddoc}
The conditional |\ifchilddoc| distinguishes between the compilation of
child documents and the main document:
%
\begin{center}
|\ifchilddoc |\textit{child-code}| |[|\||else |\textit{main-code}]| \||fi|
\end{center}

%%%%%%%%%%%%%%%%%%%%%%%%%%%%%%%%%%%%%%%%
\DescribeMacro{\childdocname}
\DescribeMacro{\childdocjob}
The macro |\childdocname| contains the filename (without extension)
of the main or child file being processed.
Note that |\childdocjob| will always contain the name of the main file.

%%%%%%%%%%%%%%%%%%%%%%%%%%%%%%%%%%%%%%%%
\paragraph{Title Page.}

Conditional processing can be used to include a title or banner page
in the main document when proper precautions are taken.
Importantly, the code in the main file should ensure that the page counter
(as well as other status parameters which are stored in the |.aux| files)
takes the same value after the conditional processing.
Otherwise the page numbers may take divergent values
depending on which part is compiled.

For example, a title page could be declared by:
%
\begin{center}
\begin{tabular}{l}
|\ifchilddoc\||else|\\
|\addtocounter{page}{-1}|\\
\textit{code for title page}\\
|\newpage|\\
|\||fi|
\end{tabular}
\end{center}
%
A banner page for the child documents can be generated by:
%
\begin{center}
\begin{tabular}{l}
|\ifchilddoc|\\
|\addtocounter{page}{-1}|\\
\textit{code for banner page}\\
|\newpage|\\
|\||fi|
\end{tabular}
\end{center}
%
Here one could write a message such as:
\begin{center}
|This is the part \childdocname{} of \childdocjob{}.|
\end{center}

%%%%%%%%%%%%%%%%%%%%%%%%%%%%%%%%%%%%%%%%%%%%%%%%%%%%%%%%%%%%%%%%%%%%%%%%%%%%%%%%
\subsection{Flags}
\label{sec:flags}

The package makes it easy to generate different versions
of the main or child documents.
To this end compilation flags can be defined
and assigned different default values.
They will be particularly useful in conjunction
with the forwarding mechanism described in \secref{sec:forward}.

For example, it may be useful to have a flag |\version|
which can be set to |draft| or |final|.
The document source will contain some conditional code
depending on the value of |\version|.
Suppose further, the flag should default to |final| for the main file
and to |draft| for child files
which is a natural assignment for editing the document.
This is achieved by placing the following code
in the preamble of the main document
(below the |\childdocmain| directive):
%
\begin{center}
\begin{tabular}{l}
|\ifchilddoc|\\
|\providecommand{\version}{draft}|\\
|\||else|\\
|\providecommand{\version}{final}|\\
|\||fi|
\end{tabular}
\end{center}
%
The definition by |\providecommand| makes sure
that previous definitions are not overwritten.
Further statements |\providecommand{\version}{...}|
can thus be added before the above code to override it.

For the main file, one might add a line
(between |\childdocmain| and the above block)
%
\begin{center}
|%\ifchilddoc\||else\providecommand{\version}{draft}\||fi|
\end{center}
%
which can be uncommented to produce a draft version.
Likewise one can add a line to the very top of a child file
(above the |\childdocof{|\textit{main}|}| directive)
%
\begin{center}
|%\providecommand{\version}{final}|
\end{center}
%
which can be uncommented to produce the final version of this child document.

%%%%%%%%%%%%%%%%%%%%%%%%%%%%%%%%%%%%%%%%%%%%%%%%%%%%%%%%%%%%%%%%%%%%%%%%%%%%%%%%
\subsection{Forwarding}
\label{sec:forward}

Different versions of the main or child documents
using compilation flags as described in \secref{sec:flags}
can be (permanently) stored in different files
for convenient compilation, viewing and distribution.
To this end, the package defines a command
to pass on compilation to a different file:

%%%%%%%%%%%%%%%%%%%%%%%%%%%%%%%%%%%%%%%%
\DescribeMacro{\childdocforward}
The command |\childdocforward| redirects processing to
another source file:
%
\begin{center}
\begin{tabular}{l}
|\input{childdoc.def}|\\
|\childdocforward[|\textit{main}|]{|\textit{dest}|}|\\
\end{tabular}
\end{center}
%
The argument \textit{dest} is the destination file
(without extension).
It should be the main file or one of the child files.
Note that further \textsf{childdoc} directives
such as |\childdocof| and |\childdocforward|
in the indicated file will be processed in this form.
The optional argument \textit{main}
passes on directly to the main file \textit{main}
while pretending to compile the child \textit{dest}.
This form behaves as if \textit{dest}
issues |\childdocof{|\textit{main}|}| right away,
and no further \textsf{childdoc} directives will be processed.

%%%%%%%%%%%%%%%%%%%%%%%%%%%%%%%%%%%%%%%%
\DescribeMacro{\...prefix}
In the alternative form |\childdocforwardprefix|,
%
\begin{center}
\begin{tabular}{l}
|\input{childdoc.def}|\\
|\childdocforwardprefix[|\textit{main}|]{|\textit{prefix}|}{|\textit{dest}|}|
\end{tabular}
\end{center}
%
the destination file is determined by a pattern
depending on the current file:
To make this work, the current file must be called
`{\textit{prefix}\hspace{0.2em}\textit{suffix}}'
with \textit{prefix} matching precisely the argument.
Processing is then passed on to the file
`{\textit{dest}\hspace{0.2em}\textit{suffix}}'.
Surely, the same effect is achieved by
directly specifying the
argument `{\textit{dest}\hspace{0.2em}\textit{suffix}}'
in the first form.
However, that requires to set up a different file
for each child. With the alternative form of the command
all these files can have exactly the same content
which simplifies setting them up and maintaining them.

For example, the following file |draft.tex|
with a compilation flag |\version| as described in \secref{sec:flags}
compiles the main document as a draft:
%
\begin{center}
\begin{tabular}{l}
|\def\version{draft}|\\
|\input{childdoc.def}|\\
|\childdocforward{|\textit{main}|}|
\end{tabular}
\end{center}
%
Likewise, the following files |final|\textit{nn}|.tex|
compile the final version of the child document
|child|\textit{nn}|.tex|:
%
\begin{center}
\begin{tabular}{l}
|\def\version{final}|\\
|\input{childdoc.def}|\\
|\childdocforwardprefix{final}{child}|
\end{tabular}
\end{center}
%

Note that when several versions of a main file and/or of each child file
are to be generated, it may be convenient to set up a |Makefile| or
shell script to automatise the process.

%%%%%%%%%%%%%%%%%%%%%%%%%%%%%%%%%%%%%%%%%%%%%%%%%%%%%%%%%%%%%%%%%%%%%%%%%%%%%%%%
\subsection{Command Line Processing}
\label{sec:commandline}

The effect of redirection files can also be achieved by invoking
the \LaTeX{} compiler with a more elaborate command line.
Most conveniently this should be done as part
of a shell script or a |Makefile|.

When using \textsf{childdoc} in the main file, the following
command lines effectively perform a redirection
(note that depending on the shell being used,
backslashes may have to be doubled: `|\|' $\to$ `|\\|'):
%
\begin{center}
|... -jobname "|\textit{target}|" |\\|"|[\textit{flags}]%
|\input{childdoc.def}\childdocforward[|\textit{main}|]{|\textit{dest}|}"|
\end{center}
%
Here \textit{target} is the name of the output file,
\textit{main} is the name of the main file
and \textit{dest} is the name of the main or child file to be processed
(all filenames without extensions).
The optional argument \textit{main} can be omitted
if \textit{main} matches \textit{dest}.
Optionally, compilation \textit{flags} can be defined via |\def| commands.
This command line makes the \TeX{} engine believe
it is compiling the file \textit{target}
whose content is specified as the latter parameter.
The provided code then forwards the processing to
\textit{main} or \textit{dest} as described in \secref{sec:forward}.

%%%%%%%%%%%%%%%%%%%%%%%%%%%%%%%%%%%%%%%%%%%%%%%%%%%%%%%%%%%%%%%%%%%%%%%%%%%%%%%%
\subsection{Include by Input}
\label{sec:input}

Including child documents by |\include| has some restrictions by design.
Most notably, the content of a child document always occupies
its own set of pages; pages cannot be shared between child documents.
Usually, this behaviour makes perfect sense
because each child document contain an essential part of the document.
However, in some situations it may be desirable to compose
a document from a collection of parts
without having mandatory page breaks between then.
For this case, the package
provides a mechanism to include parts
by |\input| which can also be processed individually.
However, by construction this mechanism
requires manual handling of the content to be output.

%%%%%%%%%%%%%%%%%%%%%%%%%%%%%%%%%%%%%%%%
\DescribeMacro{\ifchilddocmanual}
The main file should be prepared as usual, see \secref{sec:include}.
However, the document body must make a distinction
between processing of an individual part and of the main document, e.g.:
%
\begin{center}
\begin{tabular}{l}
|\ifchilddocmanual|\\
|\input{\childdocname}|\\
|\||else|\\
\textit{document body with }|\input{|\textit{part}|}|\\
|\||fi|
\end{tabular}
\end{center}
%
The conditional |\ifchilddocmanual| is true whenever
a part to be included by |\input| is being compiled,
and the name of the part is stored in |\childdocname|.

%%%%%%%%%%%%%%%%%%%%%%%%%%%%%%%%%%%%%%%%
\DescribeMacro{\childdocby}
Each part to be included by |\input| should start with:
%
\begin{center}
\begin{tabular}{l}
|\input{childdoc.def}|\\
|\childdocby{|\textit{main}|}|\\
\end{tabular}
\end{center}
%
The directive |\childdocby| is similar to |\childdocof|
described in \secref{sec:include},
but the subsequent selection of content must be done manually.
To that end, both |\ifchilddoc| and |\ifchilddocmanual|
will be true upon processing of a part,
and the name of the part is stored in |\childdocname|.
Note that |\jobname| will be set to the filename of the current part
so that each part receives an individual |.aux| file
that does not interfere with the |.aux| file(s) of the main document.
This behaviour can be altered by the alternative form
|\childdocby[*]{|\textit{main}|}| (with a non-empty optional argument)
which uses the |.aux| file of the main document
by setting |\jobname| to \textit{main}.

%%%%%%%%%%%%%%%%%%%%%%%%%%%%%%%%%%%%%%%%%%%%%%%%%%%%%%%%%%%%%%%%%%%%%%%%%%%%%%%%
\subsection{Driver Development}
\label{sec:driver}

The \textsf{childdoc} mechanism can also be use for the development
of definition files such as \LaTeX{} styles or classes.
This case differs from the above setup with multiple parts
included by |\include| in that no |\includeonly| should be invoked.
This can be achieved by starting the include file
(before |\ProvidesPackage|) with:
%
\begin{center}
\begin{tabular}{l}
|\input{childdoc.def}|\\
|\childdocforward{|\textit{main}|}|\\
\end{tabular}
\end{center}
%
or alternatively with:
%
\begin{center}
\begin{tabular}{l}
|\input{childdoc.def}|\\
|\childdocby{|\textit{main}|}|\\
\end{tabular}
\end{center}
%
Both forms have slightly different effects as described above.
The main file is prepared as usual, see \secref{sec:include}.

%%%%%%%%%%%%%%%%%%%%%%%%%%%%%%%%%%%%%%%%%%%%%%%%%%%%%%%%%%%%%%%%%%%%%%%%%%%%%%%%
\subsection{Legacy Detection}
\label{sec:detection}

The directive |\childdocmain| in the main file can detect
whether the complete document or merely a child is to be compiled
even without using the directive |\childdocof|.
This method is deprecated because it is less robust
and there is no compelling reason to use it;
it is merely provided for backward compatibility
and it may be removed in future versions.

If the detection mechanism is to be used,
it is mandatory to correctly specify
the filename of the main file as the argument of |\childdocmain|:
%
\begin{center}
\begin{tabular}{l}
|\input{childdoc.def}|\\
|\childdocmain{|\textit{main}|}|\\
\end{tabular}
\end{center}
%
If |\jobname| does not match the argument \textit{main} of |\childdocmain|,
it is assumed that |\jobname| points to the child file to be compiled.
When using |\childdocmain| with the main file specified as argument,
it suffices to start a child file
with just |\input{|\textit{main}|}|
without loading of the package and using |\childdocof|.
If instead all processing is done
with the appropriate \textsf{childdoc} directives,
the argument of \textit{main} of |\childdocmain| can be empty.

An alternative version of the command line processing described
in \secref{sec:commandline} using the detection mechanism reads:
%
\begin{center}
|... -jobname "|\textit{target}|" "|[\textit{flags}]%
[|\def\jobname{|\textit{dest}|}|]|\input{|\textit{main}|}"|
\end{center}

%%%%%%%%%%%%%%%%%%%%%%%%%%%%%%%%%%%%%%%%%%%%%%%%%%%%%%%%%%%%%%%%%%%%%%%%%%%%%%%%
\subsection{Manual Code}
\label{sec:manual}

In case one cannot be certain whether the definitions file |childdoc.def|
is installed on the target \TeX{} distribution
and one prefers not to ship it,
it is conceivable to paste a few relevant commands into the sources.

To that end, drop all statements |\input{childdoc.def}|
and perform the replacements as outlined below.
Instead of |\childdocmain{|\textit{main}|}| add the following code
to the top of the main file:
%
\begin{center}
\begin{tabular}{l}
|\||ifdefined\childdocname\endinput\||fi\newif\ifchilddoc|\\
|\edef\childdocname{\scantokens\expandafter{\jobname\noexpand}}|\\
|\def\childdocmain{|\textit{main}|}\||ifx\childdocmain\childdocname\||else|\\
|\childdoctrue\includeonly{\childdocname}\let\jobname\childdocmain\||fi|\\
\end{tabular}
\end{center}
%
Instead of |\childdocof{|\textit{main}|}| just include the main file
at the top of each child file:
%
\begin{center}
|\input{|\textit{main}|}|
\end{center}
%
A simple redirection |\childdocforward{|\textit{dest}|}| is achieved by:
%
\begin{center}
|\def\jobname{|\textit{dest}|}\input{\jobname}|
\end{center}
%
The redirection with prefix
|\childdocforwardprefix[|\textit{prefix}|]{|\textit{dest}|}|
is accomplished by:
%
\begin{center}
\begin{tabular}{l}
|{\edef\jobname{\scantokens\expandafter{\jobname\noexpand}}|\\
|\def\redirectjob |\textit{prefix}|#1~~~{\gdef\jobname{|\textit{dest}|#1}}|\\
|\expandafter\redirectjob\jobname~~~}\input{\jobname}|
\end{tabular}
\end{center}

In an alternative approach,
child documents can be compiled by a specific command line
without additional code or specific definitions:
%
\begin{center}
|... -jobname "|\textit{target}|" "|[\textit{flags}]%
|\includeonly{|\textit{dest}|}\input{|\textit{main}|}"|
\end{center}
%

%%%%%%%%%%%%%%%%%%%%%%%%%%%%%%%%%%%%%%%%%%%%%%%%%%%%%%%%%%%%%%%%%%%%%%%%%%%%%%%%
%%%%%%%%%%%%%%%%%%%%%%%%%%%%%%%%%%%%%%%%%%%%%%%%%%%%%%%%%%%%%%%%%%%%%%%%%%%%%%%%
\section{Information}

%%%%%%%%%%%%%%%%%%%%%%%%%%%%%%%%%%%%%%%%%%%%%%%%%%%%%%%%%%%%%%%%%%%%%%%%%%%%%%%%
\subsection{Copyright}

Copyright \copyright{} 2017--2018 Niklas Beisert

This work may be distributed and/or modified under the
conditions of the \LaTeX{} Project Public License, either version 1.3
of this license or (at your option) any later version.
The latest version of this license is in
  \url{http://www.latex-project.org/lppl.txt}
and version 1.3 or later is part of all distributions of \LaTeX{}
version 2005/12/01 or later.

This work has the LPPL maintenance status `maintained'.

The Current Maintainer of this work is Niklas Beisert.

This work consists of the files |README.txt|, |childdoc.ins| and |childdoc.dtx|
as well as the derived files |childdoc.def|, |cdocsamp.tex|
with |cdocsch1.tex|, |cdocsch2.tex|, |cdocspt3.tex|, |cdocspt4.tex|,
|cdocsdrf.tex|, |cdocsfn1.tex|, |cdocsfn2.tex|
as well as |childdoc.pdf|.

%%%%%%%%%%%%%%%%%%%%%%%%%%%%%%%%%%%%%%%%%%%%%%%%%%%%%%%%%%%%%%%%%%%%%%%%%%%%%%%%
\subsection{Files and Installation}

The package consists of the files:
%
\begin{center}
\begin{tabular}{ll}
    |README.txt|   & readme file \\
    |childdoc.ins| & installation file \\
    |childdoc.dtx| & source file \\
    |childdoc.def| & definition file \\
    |cdocsamp.tex| & sample main file \\
    |cdocsch1.tex| & sample include file \\
    |cdocsch2.tex| & sample include file \\
    |cdocspt3.tex| & sample part file \\
    |cdocspt4.tex| & sample part file \\
    |cdocsdrf.tex| & sample redirection file \\
    |cdocsfn1.tex| & sample redirection file \\
    |cdocsfn2.tex| & sample redirection file \\
    |childdoc.pdf| & manual
\end{tabular}
\end{center}
%
The distribution consists of the files
|README.txt|, |childdoc.ins| and |childdoc.dtx|.
%
\begin{itemize}
\item
Run (pdf)\LaTeX{} on |childdoc.dtx|
to compile the manual |childdoc.pdf| (this file).
\item
Run \LaTeX{} on |childdoc.ins| to create the definitions file |childdoc.def|
and the sample |cdocsamp.tex| with include files
|cdocsch1.tex|, |cdocsch2.tex|, |cdocspt3.tex|, |cdocspt4.tex|,
|cdocsdrf.tex|, |cdocsfn1.tex|, |cdocsfn2.tex|.
Then copy the file |childdoc.def| to an appropriate directory of your \LaTeX{}
distribution, e.g.\ \textit{texmf-root}|/tex/latex/childdoc|.
\end{itemize}

%%%%%%%%%%%%%%%%%%%%%%%%%%%%%%%%%%%%%%%%%%%%%%%%%%%%%%%%%%%%%%%%%%%%%%%%%%%%%%%%
\subsection{Related CTAN Packages}

There are several other packages which offer a similar functionality:
%
\begin{itemize}
\item
The packages
\href{http://ctan.org/pkg/docmute}{\textsf{docmute}},
\href{http://ctan.org/pkg/includex}{\textsf{includex}} and
\href{http://ctan.org/pkg/standalone}{\textsf{standalone}}
provide commands to include only the document body of
a child file thus allowing both files to be compiled individually.
\item
The packages \href{http://ctan.org/pkg/subdocs}{\textsf{subdocs}}
and \href{http://ctan.org/pkg/subfiles}{\textsf{subfiles}}
provide structures in which the main and child documents can be
encapsulated and allowing them to be compiled individually.
The inclusion mechanism is different from the conventional |\include|.
\item
The package \href{http://ctan.org/pkg/combine}{\textsf{combine}}
is an elaborate solution to combine several documents into one.
\end{itemize}
%
See also the CTAN topic \href{http://ctan.org/topic/subdocs}{\textsf{subdocs}}
for further related packages.
The present package differs from the above solutions in that
a document structure constructed with the conventional |\include| mechanism
just needs two extra commands at the top of every file
such that all constituent files can be compiled individually.

%%%%%%%%%%%%%%%%%%%%%%%%%%%%%%%%%%%%%%%%%%%%%%%%%%%%%%%%%%%%%%%%%%%%%%%%%%%%%%%%
%\subsection{Feature Suggestions}
%
%The following is a list of features which may be useful for future
%versions of this package:
%%
%\begin{itemize}
%\item
%\ldots
%\end{itemize}

%%%%%%%%%%%%%%%%%%%%%%%%%%%%%%%%%%%%%%%%%%%%%%%%%%%%%%%%%%%%%%%%%%%%%%%%%%%%%%%%
\subsection{Revision History}

%%%%%%%%%%%%%%%%%%%%%%%%%%%%%%%%%%%%%%%%
\paragraph{v2.0:} 2018/12/30

\begin{itemize}
\item
immediate forward processing
\item
added |\childdocby| mechanism
\item
manual restructured
\end{itemize}

%%%%%%%%%%%%%%%%%%%%%%%%%%%%%%%%%%%%%%%%
\paragraph{v1.6:} 2018/01/17

\begin{itemize}
\item
application for development of include files
\item
corrections to manual
\end{itemize}

%%%%%%%%%%%%%%%%%%%%%%%%%%%%%%%%%%%%%%%%
\paragraph{v1.5:} 2017/05/21

\begin{itemize}
\item
more complete structuring introduced
\item
|\childdocof| introduced
\item
|\childdoc| renamed to |\childdocmain|
\item
|\childredirect| renamed to |\childdocforward| and |\childdocforwardprefix|
and functionality expanded
\end{itemize}

%%%%%%%%%%%%%%%%%%%%%%%%%%%%%%%%%%%%%%%%
\paragraph{v1.0:} 2017/04/27

\begin{itemize}
\item
manual and install package
\item
first version published on CTAN
\end{itemize}

%%%%%%%%%%%%%%%%%%%%%%%%%%%%%%%%%%%%%%%%
\paragraph{v0.6:} 2017/04/26

\begin{itemize}
\item
redirection mechanism added
\end{itemize}

%%%%%%%%%%%%%%%%%%%%%%%%%%%%%%%%%%%%%%%%
\paragraph{v0.5:} 2017/04/26

\begin{itemize}
\item
functionality in definition file
\end{itemize}


%%%%%%%%%%%%%%%%%%%%%%%%%%%%%%%%%%%%%%%%%%%%%%%%%%%%%%%%%%%%%%%%%%%%%%%%%%%%%%%%
%%%%%%%%%%%%%%%%%%%%%%%%%%%%%%%%%%%%%%%%%%%%%%%%%%%%%%%%%%%%%%%%%%%%%%%%%%%%%%%%
%%%%%%%%%%%%%%%%%%%%%%%%%%%%%%%%%%%%%%%%%%%%%%%%%%%%%%%%%%%%%%%%%%%%%%%%%%%%%%%%
\appendix

\settowidth\MacroIndent{\rmfamily\scriptsize 000\ }

 \DocInput{childdoc.dtx}

\end{document}
%</driver>
% \fi
%
% %%%%%%%%%%%%%%%%%%%%%%%%%%%%%%%%%%%%%%%%%%%%%%%%%%%%%%%%%%%%%%%%%%%%%%%%%%%%%%
% %%%%%%%%%%%%%%%%%%%%%%%%%%%%%%%%%%%%%%%%%%%%%%%%%%%%%%%%%%%%%%%%%%%%%%%%%%%%%%
% \section{Sample}
%\iffalse
%<*samplemain>
%\fi
%
% The following presents a sample document
% with two chapters, two parts, a title page,
% a compile flag as well as three forwarding files to set the flag.
% It consists of eight |.tex| files:
% \begin{center}
% \begin{tabular}{ll}
% |cdocsamp.tex|&main file\\
% |cdocsch1.tex|&include file for chapter 1\\
% |cdocsch2.tex|&include file for chapter 2\\
% |cdocspt3.tex|&include file for part 3\\
% |cdocspt4.tex|&include file for part 4\\
% |cdocsdrf.tex|&forwarding file for main file in draft mode\\
% |cdocsfi1.tex|&forwarding file for final version of chapter 1\\
% |cdocsfi2.tex|&forwarding file for final version of chapter 2\\
% \end{tabular}
% \end{center}
% Each of the eight files can be compiled directly by the \LaTeX{} compiler.
%
% %%%%%%%%%%%%%%%%%%%%%%%%%%%%%%%%%%%%%%
% \paragraph{Main File.}
%
% The main file is called |cdocsamp.tex|.
%
% Load the \textsf{childdoc} definitions and
% declare the filename for the main document:
%    \begin{macrocode}
\input{childdoc.def}
\childdocmain{}
%    \end{macrocode}

% Optional override for |\version| flag:
%    \begin{macrocode}
%%\ifchilddoc\else\providecommand{\version}{draft}\fi
%    \end{macrocode}

% Define the default values for the |\version| flag
% (|final| for the main file and |draft| for childs):
%    \begin{macrocode}
\ifchilddoc
\providecommand{\version}{draft}
\else
\providecommand{\version}{final}
\fi
%    \end{macrocode}

% Load the standard document class:
%    \begin{macrocode}
\documentclass[12pt]{article}
%    \end{macrocode}

% Start the document body:
%    \begin{macrocode}
\begin{document}
%    \end{macrocode}

% Declare a title page.
% Print title, part of document being processed and version flag:
%    \begin{macrocode}
\addtocounter{page}{-1}
\begin{center}
{\LARGE\bfseries{}childdoc example\par}
\vspace{1cm}
\ifchilddoc
\ifchilddocmanual part\else chapter\fi:
`\childdocname' of `\childdocjob'\par
\else
main document: `\childdocjob'\par
\fi
version: \version\par
\end{center}
\newpage
%    \end{macrocode}

% Manually include selected file,
% otherwise process as usual:
%    \begin{macrocode}
\ifchilddocmanual
\section*{part `\childdocname'}
\input{\childdocname}
\else
%    \end{macrocode}

% Include the two chapters:
%    \begin{macrocode}
\include{cdocsch1}
\include{cdocsch2}
%    \end{macrocode}

% Include the two parts unless only chapters should be displayed:
%    \begin{macrocode}
\ifchilddoc\else
\section{part three}
\input{cdocspt3}
\section{part four}
\input{cdocspt4}
\fi
%    \end{macrocode}

% Process as usual until here:
%    \begin{macrocode}
\fi
%    \end{macrocode}

% End of document body:
%    \begin{macrocode}
\end{document}
%    \end{macrocode}
%\iffalse
%</samplemain>
%\fi
%
% %%%%%%%%%%%%%%%%%%%%%%%%%%%%%%%%%%%%%%
% \paragraph{Chapter Include Files.}
%
% The include files are called |cdocsch1.tex| and |cdocsch2.tex|.
%
%\iffalse
%<*samplechap1|samplechap2>
%\fi

% Optional override for |\version| flag:
%    \begin{macrocode}
%%\providecommand{\version}{final}
%    \end{macrocode}

% Include the main document:
%    \begin{macrocode}
\input{childdoc.def}
\childdocof{cdocsamp}
%    \end{macrocode}

%\iffalse
%</samplechap1|samplechap2>
%\fi
%
%\iffalse
%<*samplechap1>
%\fi
% Some text for chapter 1:
%    \begin{macrocode}
\section{one}
some text in chapter one
%    \end{macrocode}

%\iffalse
%</samplechap1>
%\fi
% Some text for chapter 2:
%\iffalse
%<*samplechap2>
%\fi
%    \begin{macrocode}
\section{two}
more text in chapter two
%    \end{macrocode}

%\iffalse
%</samplechap2>
%\fi
%
% %%%%%%%%%%%%%%%%%%%%%%%%%%%%%%%%%%%%%%
% \paragraph{Part Include Files.}
%
% The include files are called |cdocspt3.tex| and |cdocspt4.tex|.
%
%\iffalse
%<*samplepart3|samplepart4>
%\fi

% Optional override for |\version| flag:
%    \begin{macrocode}
%%\providecommand{\version}{final}
%    \end{macrocode}

% Include the main document:
%    \begin{macrocode}
\input{childdoc.def}
\childdocby{cdocsamp}
%    \end{macrocode}

%\iffalse
%</samplepart3|samplepart4>
%\fi
%
%\iffalse
%<*samplepart3>
%\fi
% Some text for part 3:
%    \begin{macrocode}
some text in part three
%    \end{macrocode}

%\iffalse
%</samplepart3>
%\fi
% Some text for part 4:
%\iffalse
%<*samplepart4>
%\fi
%    \begin{macrocode}
more text in part four
%    \end{macrocode}

%\iffalse
%</samplepart4>
%\fi
%
% %%%%%%%%%%%%%%%%%%%%%%%%%%%%%%%%%%%%%%
% \paragraph{Forwarding for a Complete Draft.}
%
% The following forwarding file |cdocsdrf.tex|
% compiles the main document in draft mode:
%\iffalse
%<*sampledraft>
%\fi
%    \begin{macrocode}
\def\version{draft}
\input{childdoc.def}
\childdocforward{cdocsamp}
%    \end{macrocode}

%\iffalse
%</sampledraft>
%\fi
%
% %%%%%%%%%%%%%%%%%%%%%%%%%%%%%%%%%%%%%%
% \paragraph{Forwarding for Final Version of the Chapters.}
%
% The following forwarding files |cdocsfn1.tex| and |cdocsfn2.tex|
% (with identical content)
% compile the final versions of the child documents
% |cdocsch1.tex| and |cdocsch2.tex|, respectively:
%\iffalse
%<*samplefinal>
%\fi
%    \begin{macrocode}
\def\version{final}
\input{childdoc.def}
\childdocforwardprefix[cdocsamp]{cdocsfn}{cdocsch}
%    \end{macrocode}

%\iffalse
%</samplefinal>
%\fi
%
% %%%%%%%%%%%%%%%%%%%%%%%%%%%%%%%%%%%%%%
% \paragraph{Command Line Processing.}
%
% The following three command lines generate the output files
% |cdocscld|, |cdocscl1| and |cdocscl2|
% which should be identical to
% |cdocsdrf|, |cdocsch1| and |cdocsfn2|, respectively:
% \begin{center}
% \begin{tabular}{l}
% |latex -jobname cdocscld \|\\
% |  "\def\version{draft}\input{childdoc.def}\childdocforward{cdocsamp}"|\\
% |latex -jobname cdocscl1 \|\\
% |  "\input{childdoc.def}\childdocforward[cdocsamp]{cdocsch1}"|\\
% |latex -jobname cdocscl2 \|\\
% |  "\def\version{final}\input{childdoc.def}\childdocforward{cdocsch2}"|
% \end{tabular}
% \end{center}
% Note that the trailing backslash on each first line
% merely continues the input to the second line
% (for convenient cut ant paste).
% Furthermore, the command |latex| can be replaced by any
% of its alternative versions such as |pdflatex|.
%
% %%%%%%%%%%%%%%%%%%%%%%%%%%%%%%%%%%%%%%%%%%%%%%%%%%%%%%%%%%%%%%%%%%%%%%%%%%%%%%
% %%%%%%%%%%%%%%%%%%%%%%%%%%%%%%%%%%%%%%%%%%%%%%%%%%%%%%%%%%%%%%%%%%%%%%%%%%%%%%
% \section{Implementation}
%\iffalse
%<*package>
%\fi
%
% This section describes the definitions file |childdoc.def|.

% The definitions cannot be loaded using |\usepackage| or |\RequirePackage|
% which has a mechanism to prevent loading a style file more than once.
% When loading the definitions by means of |\input|
% multiple instances have to be prevented manually:
%\iffalse
%This code needs to be before the `\ProvidesFile' directive
%which is defined at the beginning of this file.
%Therefore it is also placed there and commented out here.
%</package>
%<*discard>
%\fi
%    \begin{macrocode}
\ifdefined\childdocmain\endinput\fi
%    \end{macrocode}
%\iffalse
%</discard>
%<*package>
%\fi
%
% \macro{\ifchilddoc}
% \macro{\ifchilddocmanual}
% The conditional |\ifchilddoc| tells whether a
% child (true) or main (false) document is being compiled.
% The conditional |\ifchilddocmanual| tells whether
% the |\includeonly| mechanism is used (false) or
% the selection of child files must be performed manually (true).
% The definitions initialise to false:
%    \begin{macrocode}
\newif\ifchilddoc
\newif\ifchilddocmanual
%    \end{macrocode}

% \macro{\childdocname}
% \macro{\childdocjob}
% The macro |\childdocname| stores the name of the main document
% to be compiled. The macro |\childdocjob| stores the name of
% the document on which the \LaTeX{} compiler was originally invoked.
% The content of |\jobname| cannot be compared
% to filenames specified in the source due to different catcodes.
% The following code rescans |\jobname|, stores the result
% in |\childdocname| and saves a copy in |\childdocjob|:
%    \begin{macrocode}
\edef\childdocname{\scantokens\expandafter{\jobname\noexpand}}
\let\childdocjob\childdocname
%    \end{macrocode}

% \macro{\childdocdisable}
% The macro |\childdocdisable| prevents the main file
% from being processed more than once.
% At this stage, the main document command |\childdocmain|
% is assumed to be called once again where it should do nothing.
% Any subsequent call to it should prevent
% a secondary processing of the main document
% It overwrites the forwarding commands
% |\childdocof| and |\childdocforward|
% with empty macros to prevent further inclusions of the main document:
%    \begin{macrocode}
\newcommand{\childdocdisable}
{
  \renewcommand{\childdocmain}[1]{\renewcommand{\childdocmain}[1]{\endinput}}
  \renewcommand{\childdocof}[1]{}
  \renewcommand{\childdocby}[2][]{}
  \renewcommand{\childdocforward}[2][]{}
  \renewcommand{\childdocdisable}{}
}
%    \end{macrocode}

% \macro{\childdocmain}
% The macro |\childdocmain| is to be called at the top of the main file
% with nothing or the main filename (without extension) as argument.
% First, it breaks loops.
% If the argument is not empty and does not match |\childdocname|
% (which is set by the first inclusion of |childdoc.def|),
% |\ifchilddoc| is set to true, |\includeonly| is applied to the child file
% and |\jobname| is set to the main file
% (for proper handling of |.aux| files):
%    \begin{macrocode}
\newcommand{\childdocmain}[1]
{
  \childdocdisable\childdocmain{}
  \if?#1?\else
    \begingroup
      \def\childdoctmp{#1}
      \ifx\childdoctmp\childdocname
        \def\childdoctmp{}
      \else
        \def\childdoctmp
        {
          \childdoctrue
          \includeonly{\childdocname}
          \def\childdocjob{#1}
          \def\jobname{#1}
        }
      \fi
      \expandafter
    \endgroup
    \childdoctmp
  \fi
}
%    \end{macrocode}

% \macro{\childdocof}
% The command |\childdocof| redirects
% compilation to the main file |#1|.
%    \begin{macrocode}
\newcommand{\childdocof}[1]
{
  \childdocdisable
  \childdoctrue
  \includeonly{\childdocname}
  \def\jobname{#1}
  \def\childdocjob{#1}
  \input{#1}
}
%    \end{macrocode}

% \macro{\childdocby}
% The command |\childdocby| ....
%    \begin{macrocode}
\newcommand{\childdocby}[2][]
{
  \childdocdisable
  \childdoctrue
  \childdocmanualtrue
  \if?#1?\else
    \def\jobname{#2}
  \fi
  \def\childdocjob{#2}
  \input{#2}
  \endinput
}
%    \end{macrocode}

% \macro{\childdocforward}
% The command |\childdocforward| redirects
% compilation to the main file or
% (if the optional argument is given) a child file.
% Parameters are set as if the main file
% or a child file starting with |\childdocof| was compiled.
% Then compilation is handed over to the main file:
%    \begin{macrocode}
\newcommand{\childdocforward}[2][]
{
  \begingroup
    \if?#1?
      \def\childdoctmp
      {
        \def\childdocname{#2}
        \def\childdocjob{#2}
        \def\jobname{#2}
        \input{#2}
        \endinput
      }
    \else
      \def\childdoctmp
      {
        \childdocdisable
        \def\childdocname{#2}
        \childdoctrue
        \includeonly{#2}
        \def\childdocjob{#1}
        \def\jobname{#1}
        \input{#1}
        \endinput
      }
    \fi
    \expandafter
  \endgroup
  \childdoctmp
}
%    \end{macrocode}

% \macro{\childdocforwardprefix}
% The command |\childdocforwardprefix| redirects
% compilation to the main or a child file by means of a pattern.
% The prefix |#1| in the current filename is replaced by |#2|
% and the suffix of the current filename is kept
% (it is assumed that the filename does not contain the substring `|~~~|'
% which is used as a delimiter).
% Compilation is handed over to the new file by |\childdocforward|:
%    \begin{macrocode}
\newcommand{\childdocforwardprefix}[3][]
{
  \begingroup
    \def\childdocextract #2##1~~~{\def\childdoctmp{\childdocforward[#1]{#3##1}}}
    \expandafter\childdocextract\childdocname~~~
    \expandafter
  \endgroup
  \childdoctmp
}
%    \end{macrocode}

% \macro{\childdoc}
% The deprecated macro |\childdoc| is a legacy version of |\childdocmain|:
%    \begin{macrocode}
\newcommand{\childdoc}{\childdocmain}
%    \end{macrocode}

% \macro{\childdocredirect}
% The deprecated macro |\childdocredirect| is a legacy version
% of |\childdocforward| and |\childdocforwardprefix|:
%    \begin{macrocode}
\newcommand{\childdocredirect}[2][]
{
  \begingroup
    \if?#1?
      \def\childdoctmp{\childdocforward{#2}}
    \else
      \def\childdoctmp{\childdocforwardprefix{#1}{#2}}
    \fi
    \expandafter
  \endgroup
  \childdoctmp
}
%    \end{macrocode}

%\iffalse
%</package>
%\fi
%
\endinput
\childdocforward{cdocsamp}"|\\
% |latex -jobname cdocscl1 \|\\
% |  "% \iffalse
%
% childdoc.dtx Copyright (C) 2017-2018 Niklas Beisert
%
% This work may be distributed and/or modified under the
% conditions of the LaTeX Project Public License, either version 1.3
% of this license or (at your option) any later version.
% The latest version of this license is in
%   http://www.latex-project.org/lppl.txt
% and version 1.3 or later is part of all distributions of LaTeX
% version 2005/12/01 or later.
%
% This work has the LPPL maintenance status `maintained'.
%
% The Current Maintainer of this work is Niklas Beisert.
%
% This work consists of the files childdoc.dtx and childdoc.ins
% and the derived files childdoc.def and cdocsamp.tex with
% cdocsch1.tex, cdocsch2.tex, cdocsdrf.tex, cdocsfn1.tex, cdocsfn2.tex.
%
%<package>\ifdefined\childdocmain\endinput\fi
%<package>\ProvidesFile{childdoc.def}[2018/12/30 v2.0 child document driver]
%<samplemain>\ProvidesFile{cdocsamp.tex}[2018/12/30 v2.0 sample for childdoc]
%<*driver>
%\ProvidesFile{childdoc.drv}[2018/12/30 v2.0 childdoc reference manual file]
\PassOptionsToClass{10pt,a4paper}{article}
\documentclass{ltxdoc}

\usepackage[margin=35mm]{geometry}
\usepackage{hyperref}
\usepackage{hyperxmp}
\usepackage[usenames]{color}

\hypersetup{colorlinks=true}
\hypersetup{pdfstartview=FitH}
\hypersetup{pdfpagemode=UseNone}
\hypersetup{pdfsource={}}
\hypersetup{pdflang={en-UK}}
\hypersetup{pdfcopyright={Copyright 2017-2018 Niklas Beisert.
  This work may be distributed and/or modified under the
  conditions of the LaTeX Project Public License, either version 1.3
  of this license or (at your option) any later version.}}
\hypersetup{pdflicenseurl={http://www.latex-project.org/lppl.txt}}
\hypersetup{pdfcontactaddress={ETH Zurich, ITP, HIT K,
  Wolfgang-Pauli-Strasse 27}}
\hypersetup{pdfcontactpostcode={8093}}
\hypersetup{pdfcontactcity={Zurich}}
\hypersetup{pdfcontactcountry={Switzerland}}
\hypersetup{pdfcontactemail={nbeisert@itp.phys.ethz.ch}}
\hypersetup{pdfcontacturl={http://people.phys.ethz.ch/\xmptilde nbeisert/}}

\newcommand{\secref}[1]{\hyperref[#1]{section \ref*{#1}}}

\parskip1ex
\parindent0pt
\let\olditemize\itemize
\def\itemize{\olditemize\parskip0pt}

\begin{document}

\title{The \textsf{childdoc} Package}
\hypersetup{pdftitle={The childdoc Package}}
\author{Niklas Beisert\\[2ex]
  Institut f\"ur Theoretische Physik\\
  Eidgen\"ossische Technische Hochschule Z\"urich\\
  Wolfgang-Pauli-Strasse 27, 8093 Z\"urich, Switzerland\\[1ex]
  \href{mailto:nbeisert@itp.phys.ethz.ch}
  {\texttt{nbeisert@itp.phys.ethz.ch}}}
\hypersetup{pdfauthor={Niklas Beisert}}
\hypersetup{pdfsubject={Manual for the LaTeX2e Package childdoc}}
\date{30 December 2018, \textsf{v2.0}}
\maketitle

\begin{abstract}\noindent
\textsf{childdoc} is a \LaTeXe{} package
that enables the direct compilation
of document sections included by |\include|
to individual files.
\end{abstract}

\begingroup
\parskip0ex
\tableofcontents
\endgroup

%%%%%%%%%%%%%%%%%%%%%%%%%%%%%%%%%%%%%%%%%%%%%%%%%%%%%%%%%%%%%%%%%%%%%%%%%%%%%%%%
%%%%%%%%%%%%%%%%%%%%%%%%%%%%%%%%%%%%%%%%%%%%%%%%%%%%%%%%%%%%%%%%%%%%%%%%%%%%%%%%
\section{Introduction}

\LaTeX{} provides a mechanism to structure a large document (such as a book)
into a main file and several child files (containing the chapters)
using the |\include| command.
This mechanism is beneficial for documents
which span hundreds of pages in order to
make the source file(s) more manageable.
Moreover, compilation can be restricted to
selected child files by means of the |\includeonly| command.
The latter feature can be used to reduce the compilation time while editing
(this was significantly more useful in the earlier days of \LaTeX{})
or to generate a smaller document which is easier to navigate.
Another application of |\includeonly| is to generate
documents consisting of selected parts of the complete document.

However, there are a few drawbacks of the plain |\include| mechanism:
\begin{itemize}
\item
The child files cannot be compiled on their own,
they can only be compiled via the main file.
A naive editing environment
(such as a text editor with an option
to have the current file processed by \LaTeX)
may require one to switch to the main file before compiling;
attempting to compile the child file produces errors.
\item
The main file must be modified (each time)
to adjust the |\includeonly| command
to the present needs. This easily leaves the main file in a messy state.
\item
The generated document will always carry the filename
of the main document. This is inconvenient if
several child files are to be compiled and
to be kept for distribution.
\end{itemize}

The present package provides a simple interface
to make child files individually compilable by \LaTeX{}.
Compiling a child file then has the same effect as compiling
the main file with an |\includeonly| command
to select the appropriate child.
Moreover the generated document will carry the name of the child
rather than the main file.
This resolves all three above issues.

This feature is meant to make the editing of books,
thesis documents and lecture notes somewhat more convenient.
However, the package can also be used efficiently for
composing a series of documents (such as exercise sheets)
which are typically distributed individually.
It then assists the author in generating the individual documents
(potentially in different versions)
as well as a document containing the collected series.
Another application is in developing style files
or other kinds of included material
where compilation of the style file could redirect
to a sample or test file.

%%%%%%%%%%%%%%%%%%%%%%%%%%%%%%%%%%%%%%%%%%%%%%%%%%%%%%%%%%%%%%%%%%%%%%%%%%%%%%%%
%%%%%%%%%%%%%%%%%%%%%%%%%%%%%%%%%%%%%%%%%%%%%%%%%%%%%%%%%%%%%%%%%%%%%%%%%%%%%%%%
\section{Usage}

First of all, the package \textsf{childdoc} is \emph{not} a standard
\LaTeXe{} |.sty| style file! Therefore it needs to be invoked in
a non-standard way.

%%%%%%%%%%%%%%%%%%%%%%%%%%%%%%%%%%%%%%%%%%%%%%%%%%%%%%%%%%%%%%%%%%%%%%%%%%%%%%%%
\subsection{Included Files}
\label{sec:include}

%%%%%%%%%%%%%%%%%%%%%%%%%%%%%%%%%%%%%%%%
\DescribeMacro{\childdocmain}
To use the package, add the commands
\begin{center}
\begin{tabular}{l}
|\input{childdoc.def}|\\
|\childdocmain{}|\\
\end{tabular}
\end{center}
at the very top of the main \LaTeX{} file,
in particular \emph{before} the |\documentclass| statement!
The argument of |\childdocmain| should be left empty
(but it must be present).

%%%%%%%%%%%%%%%%%%%%%%%%%%%%%%%%%%%%%%%%
\DescribeMacro{\childdocof}
Furthermore, add the commands
\begin{center}
\begin{tabular}{l}
|\input{childdoc.def}|\\
|\childdocof{|\textit{main}|}|\\
\end{tabular}
\end{center}
at the top of every child file \textit{child}
which is included by |\include{|\textit{child}|}|
from within the main file
(or at least for those files to be compiled individually).
The argument \textit{main} must be the filename of the main file.

There are a couple of
considerations in setting up the main and child documents:

%%%%%%%%%%%%%%%%%%%%%%%%%%%%%%%%%%%%%%%%
\paragraph{Restrictions.}

Please note the following restrictions:
\begin{itemize}
\item
|\childdocmain| must be called with one argument \textit{main}
to ensure compatibility with earlier version of the package.
It must either be empty (|\childdocmain{}|)
or precisely match the filename of the main file in which it is specified.
See \secref{sec:detection} for further information.
\item
The filename \textit{main} must be specified without the |.tex| extension.
\item
The filename \textit{main} is case sensitive
(even in case-insensitive file systems)
due to internal string comparison.
\item
The argument \textit{main} should be fully expanded, it cannot be a macro.
\item
Subdirectories and special characters should be avoided in filenames.
\item
The command |\childdocmain{|\textit{main}|}| must be followed by a whitespace.
It should not be followed immediately by another command
or by a comment mark `|%|'.
This is because the \TeX{} parser reads the token immediately following
the argument of |\childdocmain| and puts it
at the beginning of every child section;
however, a white\-space is ignored.
\end{itemize}

%%%%%%%%%%%%%%%%%%%%%%%%%%%%%%%%%%%%%%%%
\paragraph{Content of Main File.}

It is advisable to place all content in the child files included by |\include|.
Any output contained in the main file will appear in all child documents
unless suppressed manually;
it cannot be suppressed automatically by the |\includeonly| directive
and thus should normally be avoided.
A method to include some content in the main file
by means of conditional processing is described in \secref{sec:conditional}.

%%%%%%%%%%%%%%%%%%%%%%%%%%%%%%%%%%%%%%%%
\paragraph{Page Numbering.}

When only a part of the document is compiled,
the appropriate numbering of pages
(as well as other status parameters)
is determined from the |.aux| files.
The latter contain information from previous passes.
However this information needs to propagate through
all intermediate child documents.
Therefore the page numbering in child documents may well
be inconsistent until the complete document is compiled at least once.

A useful (if unconventional) way to always ensure a consistent
page numbering is to restart the numbering in each child document
and denote the pages by `\textit{child}|.|\textit{page}'
where \textit{child} represents the chapter/section number of the child file.
This can be achieved by the command
|\numberwithin{page}{|\textit{child}|}|
of the \textsf{amsmath} package
where \textit{child} can be |chapter| or |section|
depending on the chosen structuring.
Alternatively, one can modify the macro |\thepage| appropriately
and reset the counter |page| at the start of each child file.

%%%%%%%%%%%%%%%%%%%%%%%%%%%%%%%%%%%%%%%%%%%%%%%%%%%%%%%%%%%%%%%%%%%%%%%%%%%%%%%%
\subsection{Conditional Processing}
\label{sec:conditional}

The package provides a mechanism to compile different versions
of a document. To customise the versions further some conditional processing
can come in handy to distinguish which version is being compiled.
The package provides two macros to describe the compilation context:

%%%%%%%%%%%%%%%%%%%%%%%%%%%%%%%%%%%%%%%%
\DescribeMacro{\ifchilddoc}
The conditional |\ifchilddoc| distinguishes between the compilation of
child documents and the main document:
%
\begin{center}
|\ifchilddoc |\textit{child-code}| |[|\||else |\textit{main-code}]| \||fi|
\end{center}

%%%%%%%%%%%%%%%%%%%%%%%%%%%%%%%%%%%%%%%%
\DescribeMacro{\childdocname}
\DescribeMacro{\childdocjob}
The macro |\childdocname| contains the filename (without extension)
of the main or child file being processed.
Note that |\childdocjob| will always contain the name of the main file.

%%%%%%%%%%%%%%%%%%%%%%%%%%%%%%%%%%%%%%%%
\paragraph{Title Page.}

Conditional processing can be used to include a title or banner page
in the main document when proper precautions are taken.
Importantly, the code in the main file should ensure that the page counter
(as well as other status parameters which are stored in the |.aux| files)
takes the same value after the conditional processing.
Otherwise the page numbers may take divergent values
depending on which part is compiled.

For example, a title page could be declared by:
%
\begin{center}
\begin{tabular}{l}
|\ifchilddoc\||else|\\
|\addtocounter{page}{-1}|\\
\textit{code for title page}\\
|\newpage|\\
|\||fi|
\end{tabular}
\end{center}
%
A banner page for the child documents can be generated by:
%
\begin{center}
\begin{tabular}{l}
|\ifchilddoc|\\
|\addtocounter{page}{-1}|\\
\textit{code for banner page}\\
|\newpage|\\
|\||fi|
\end{tabular}
\end{center}
%
Here one could write a message such as:
\begin{center}
|This is the part \childdocname{} of \childdocjob{}.|
\end{center}

%%%%%%%%%%%%%%%%%%%%%%%%%%%%%%%%%%%%%%%%%%%%%%%%%%%%%%%%%%%%%%%%%%%%%%%%%%%%%%%%
\subsection{Flags}
\label{sec:flags}

The package makes it easy to generate different versions
of the main or child documents.
To this end compilation flags can be defined
and assigned different default values.
They will be particularly useful in conjunction
with the forwarding mechanism described in \secref{sec:forward}.

For example, it may be useful to have a flag |\version|
which can be set to |draft| or |final|.
The document source will contain some conditional code
depending on the value of |\version|.
Suppose further, the flag should default to |final| for the main file
and to |draft| for child files
which is a natural assignment for editing the document.
This is achieved by placing the following code
in the preamble of the main document
(below the |\childdocmain| directive):
%
\begin{center}
\begin{tabular}{l}
|\ifchilddoc|\\
|\providecommand{\version}{draft}|\\
|\||else|\\
|\providecommand{\version}{final}|\\
|\||fi|
\end{tabular}
\end{center}
%
The definition by |\providecommand| makes sure
that previous definitions are not overwritten.
Further statements |\providecommand{\version}{...}|
can thus be added before the above code to override it.

For the main file, one might add a line
(between |\childdocmain| and the above block)
%
\begin{center}
|%\ifchilddoc\||else\providecommand{\version}{draft}\||fi|
\end{center}
%
which can be uncommented to produce a draft version.
Likewise one can add a line to the very top of a child file
(above the |\childdocof{|\textit{main}|}| directive)
%
\begin{center}
|%\providecommand{\version}{final}|
\end{center}
%
which can be uncommented to produce the final version of this child document.

%%%%%%%%%%%%%%%%%%%%%%%%%%%%%%%%%%%%%%%%%%%%%%%%%%%%%%%%%%%%%%%%%%%%%%%%%%%%%%%%
\subsection{Forwarding}
\label{sec:forward}

Different versions of the main or child documents
using compilation flags as described in \secref{sec:flags}
can be (permanently) stored in different files
for convenient compilation, viewing and distribution.
To this end, the package defines a command
to pass on compilation to a different file:

%%%%%%%%%%%%%%%%%%%%%%%%%%%%%%%%%%%%%%%%
\DescribeMacro{\childdocforward}
The command |\childdocforward| redirects processing to
another source file:
%
\begin{center}
\begin{tabular}{l}
|\input{childdoc.def}|\\
|\childdocforward[|\textit{main}|]{|\textit{dest}|}|\\
\end{tabular}
\end{center}
%
The argument \textit{dest} is the destination file
(without extension).
It should be the main file or one of the child files.
Note that further \textsf{childdoc} directives
such as |\childdocof| and |\childdocforward|
in the indicated file will be processed in this form.
The optional argument \textit{main}
passes on directly to the main file \textit{main}
while pretending to compile the child \textit{dest}.
This form behaves as if \textit{dest}
issues |\childdocof{|\textit{main}|}| right away,
and no further \textsf{childdoc} directives will be processed.

%%%%%%%%%%%%%%%%%%%%%%%%%%%%%%%%%%%%%%%%
\DescribeMacro{\...prefix}
In the alternative form |\childdocforwardprefix|,
%
\begin{center}
\begin{tabular}{l}
|\input{childdoc.def}|\\
|\childdocforwardprefix[|\textit{main}|]{|\textit{prefix}|}{|\textit{dest}|}|
\end{tabular}
\end{center}
%
the destination file is determined by a pattern
depending on the current file:
To make this work, the current file must be called
`{\textit{prefix}\hspace{0.2em}\textit{suffix}}'
with \textit{prefix} matching precisely the argument.
Processing is then passed on to the file
`{\textit{dest}\hspace{0.2em}\textit{suffix}}'.
Surely, the same effect is achieved by
directly specifying the
argument `{\textit{dest}\hspace{0.2em}\textit{suffix}}'
in the first form.
However, that requires to set up a different file
for each child. With the alternative form of the command
all these files can have exactly the same content
which simplifies setting them up and maintaining them.

For example, the following file |draft.tex|
with a compilation flag |\version| as described in \secref{sec:flags}
compiles the main document as a draft:
%
\begin{center}
\begin{tabular}{l}
|\def\version{draft}|\\
|\input{childdoc.def}|\\
|\childdocforward{|\textit{main}|}|
\end{tabular}
\end{center}
%
Likewise, the following files |final|\textit{nn}|.tex|
compile the final version of the child document
|child|\textit{nn}|.tex|:
%
\begin{center}
\begin{tabular}{l}
|\def\version{final}|\\
|\input{childdoc.def}|\\
|\childdocforwardprefix{final}{child}|
\end{tabular}
\end{center}
%

Note that when several versions of a main file and/or of each child file
are to be generated, it may be convenient to set up a |Makefile| or
shell script to automatise the process.

%%%%%%%%%%%%%%%%%%%%%%%%%%%%%%%%%%%%%%%%%%%%%%%%%%%%%%%%%%%%%%%%%%%%%%%%%%%%%%%%
\subsection{Command Line Processing}
\label{sec:commandline}

The effect of redirection files can also be achieved by invoking
the \LaTeX{} compiler with a more elaborate command line.
Most conveniently this should be done as part
of a shell script or a |Makefile|.

When using \textsf{childdoc} in the main file, the following
command lines effectively perform a redirection
(note that depending on the shell being used,
backslashes may have to be doubled: `|\|' $\to$ `|\\|'):
%
\begin{center}
|... -jobname "|\textit{target}|" |\\|"|[\textit{flags}]%
|\input{childdoc.def}\childdocforward[|\textit{main}|]{|\textit{dest}|}"|
\end{center}
%
Here \textit{target} is the name of the output file,
\textit{main} is the name of the main file
and \textit{dest} is the name of the main or child file to be processed
(all filenames without extensions).
The optional argument \textit{main} can be omitted
if \textit{main} matches \textit{dest}.
Optionally, compilation \textit{flags} can be defined via |\def| commands.
This command line makes the \TeX{} engine believe
it is compiling the file \textit{target}
whose content is specified as the latter parameter.
The provided code then forwards the processing to
\textit{main} or \textit{dest} as described in \secref{sec:forward}.

%%%%%%%%%%%%%%%%%%%%%%%%%%%%%%%%%%%%%%%%%%%%%%%%%%%%%%%%%%%%%%%%%%%%%%%%%%%%%%%%
\subsection{Include by Input}
\label{sec:input}

Including child documents by |\include| has some restrictions by design.
Most notably, the content of a child document always occupies
its own set of pages; pages cannot be shared between child documents.
Usually, this behaviour makes perfect sense
because each child document contain an essential part of the document.
However, in some situations it may be desirable to compose
a document from a collection of parts
without having mandatory page breaks between then.
For this case, the package
provides a mechanism to include parts
by |\input| which can also be processed individually.
However, by construction this mechanism
requires manual handling of the content to be output.

%%%%%%%%%%%%%%%%%%%%%%%%%%%%%%%%%%%%%%%%
\DescribeMacro{\ifchilddocmanual}
The main file should be prepared as usual, see \secref{sec:include}.
However, the document body must make a distinction
between processing of an individual part and of the main document, e.g.:
%
\begin{center}
\begin{tabular}{l}
|\ifchilddocmanual|\\
|\input{\childdocname}|\\
|\||else|\\
\textit{document body with }|\input{|\textit{part}|}|\\
|\||fi|
\end{tabular}
\end{center}
%
The conditional |\ifchilddocmanual| is true whenever
a part to be included by |\input| is being compiled,
and the name of the part is stored in |\childdocname|.

%%%%%%%%%%%%%%%%%%%%%%%%%%%%%%%%%%%%%%%%
\DescribeMacro{\childdocby}
Each part to be included by |\input| should start with:
%
\begin{center}
\begin{tabular}{l}
|\input{childdoc.def}|\\
|\childdocby{|\textit{main}|}|\\
\end{tabular}
\end{center}
%
The directive |\childdocby| is similar to |\childdocof|
described in \secref{sec:include},
but the subsequent selection of content must be done manually.
To that end, both |\ifchilddoc| and |\ifchilddocmanual|
will be true upon processing of a part,
and the name of the part is stored in |\childdocname|.
Note that |\jobname| will be set to the filename of the current part
so that each part receives an individual |.aux| file
that does not interfere with the |.aux| file(s) of the main document.
This behaviour can be altered by the alternative form
|\childdocby[*]{|\textit{main}|}| (with a non-empty optional argument)
which uses the |.aux| file of the main document
by setting |\jobname| to \textit{main}.

%%%%%%%%%%%%%%%%%%%%%%%%%%%%%%%%%%%%%%%%%%%%%%%%%%%%%%%%%%%%%%%%%%%%%%%%%%%%%%%%
\subsection{Driver Development}
\label{sec:driver}

The \textsf{childdoc} mechanism can also be use for the development
of definition files such as \LaTeX{} styles or classes.
This case differs from the above setup with multiple parts
included by |\include| in that no |\includeonly| should be invoked.
This can be achieved by starting the include file
(before |\ProvidesPackage|) with:
%
\begin{center}
\begin{tabular}{l}
|\input{childdoc.def}|\\
|\childdocforward{|\textit{main}|}|\\
\end{tabular}
\end{center}
%
or alternatively with:
%
\begin{center}
\begin{tabular}{l}
|\input{childdoc.def}|\\
|\childdocby{|\textit{main}|}|\\
\end{tabular}
\end{center}
%
Both forms have slightly different effects as described above.
The main file is prepared as usual, see \secref{sec:include}.

%%%%%%%%%%%%%%%%%%%%%%%%%%%%%%%%%%%%%%%%%%%%%%%%%%%%%%%%%%%%%%%%%%%%%%%%%%%%%%%%
\subsection{Legacy Detection}
\label{sec:detection}

The directive |\childdocmain| in the main file can detect
whether the complete document or merely a child is to be compiled
even without using the directive |\childdocof|.
This method is deprecated because it is less robust
and there is no compelling reason to use it;
it is merely provided for backward compatibility
and it may be removed in future versions.

If the detection mechanism is to be used,
it is mandatory to correctly specify
the filename of the main file as the argument of |\childdocmain|:
%
\begin{center}
\begin{tabular}{l}
|\input{childdoc.def}|\\
|\childdocmain{|\textit{main}|}|\\
\end{tabular}
\end{center}
%
If |\jobname| does not match the argument \textit{main} of |\childdocmain|,
it is assumed that |\jobname| points to the child file to be compiled.
When using |\childdocmain| with the main file specified as argument,
it suffices to start a child file
with just |\input{|\textit{main}|}|
without loading of the package and using |\childdocof|.
If instead all processing is done
with the appropriate \textsf{childdoc} directives,
the argument of \textit{main} of |\childdocmain| can be empty.

An alternative version of the command line processing described
in \secref{sec:commandline} using the detection mechanism reads:
%
\begin{center}
|... -jobname "|\textit{target}|" "|[\textit{flags}]%
[|\def\jobname{|\textit{dest}|}|]|\input{|\textit{main}|}"|
\end{center}

%%%%%%%%%%%%%%%%%%%%%%%%%%%%%%%%%%%%%%%%%%%%%%%%%%%%%%%%%%%%%%%%%%%%%%%%%%%%%%%%
\subsection{Manual Code}
\label{sec:manual}

In case one cannot be certain whether the definitions file |childdoc.def|
is installed on the target \TeX{} distribution
and one prefers not to ship it,
it is conceivable to paste a few relevant commands into the sources.

To that end, drop all statements |\input{childdoc.def}|
and perform the replacements as outlined below.
Instead of |\childdocmain{|\textit{main}|}| add the following code
to the top of the main file:
%
\begin{center}
\begin{tabular}{l}
|\||ifdefined\childdocname\endinput\||fi\newif\ifchilddoc|\\
|\edef\childdocname{\scantokens\expandafter{\jobname\noexpand}}|\\
|\def\childdocmain{|\textit{main}|}\||ifx\childdocmain\childdocname\||else|\\
|\childdoctrue\includeonly{\childdocname}\let\jobname\childdocmain\||fi|\\
\end{tabular}
\end{center}
%
Instead of |\childdocof{|\textit{main}|}| just include the main file
at the top of each child file:
%
\begin{center}
|\input{|\textit{main}|}|
\end{center}
%
A simple redirection |\childdocforward{|\textit{dest}|}| is achieved by:
%
\begin{center}
|\def\jobname{|\textit{dest}|}\input{\jobname}|
\end{center}
%
The redirection with prefix
|\childdocforwardprefix[|\textit{prefix}|]{|\textit{dest}|}|
is accomplished by:
%
\begin{center}
\begin{tabular}{l}
|{\edef\jobname{\scantokens\expandafter{\jobname\noexpand}}|\\
|\def\redirectjob |\textit{prefix}|#1~~~{\gdef\jobname{|\textit{dest}|#1}}|\\
|\expandafter\redirectjob\jobname~~~}\input{\jobname}|
\end{tabular}
\end{center}

In an alternative approach,
child documents can be compiled by a specific command line
without additional code or specific definitions:
%
\begin{center}
|... -jobname "|\textit{target}|" "|[\textit{flags}]%
|\includeonly{|\textit{dest}|}\input{|\textit{main}|}"|
\end{center}
%

%%%%%%%%%%%%%%%%%%%%%%%%%%%%%%%%%%%%%%%%%%%%%%%%%%%%%%%%%%%%%%%%%%%%%%%%%%%%%%%%
%%%%%%%%%%%%%%%%%%%%%%%%%%%%%%%%%%%%%%%%%%%%%%%%%%%%%%%%%%%%%%%%%%%%%%%%%%%%%%%%
\section{Information}

%%%%%%%%%%%%%%%%%%%%%%%%%%%%%%%%%%%%%%%%%%%%%%%%%%%%%%%%%%%%%%%%%%%%%%%%%%%%%%%%
\subsection{Copyright}

Copyright \copyright{} 2017--2018 Niklas Beisert

This work may be distributed and/or modified under the
conditions of the \LaTeX{} Project Public License, either version 1.3
of this license or (at your option) any later version.
The latest version of this license is in
  \url{http://www.latex-project.org/lppl.txt}
and version 1.3 or later is part of all distributions of \LaTeX{}
version 2005/12/01 or later.

This work has the LPPL maintenance status `maintained'.

The Current Maintainer of this work is Niklas Beisert.

This work consists of the files |README.txt|, |childdoc.ins| and |childdoc.dtx|
as well as the derived files |childdoc.def|, |cdocsamp.tex|
with |cdocsch1.tex|, |cdocsch2.tex|, |cdocspt3.tex|, |cdocspt4.tex|,
|cdocsdrf.tex|, |cdocsfn1.tex|, |cdocsfn2.tex|
as well as |childdoc.pdf|.

%%%%%%%%%%%%%%%%%%%%%%%%%%%%%%%%%%%%%%%%%%%%%%%%%%%%%%%%%%%%%%%%%%%%%%%%%%%%%%%%
\subsection{Files and Installation}

The package consists of the files:
%
\begin{center}
\begin{tabular}{ll}
    |README.txt|   & readme file \\
    |childdoc.ins| & installation file \\
    |childdoc.dtx| & source file \\
    |childdoc.def| & definition file \\
    |cdocsamp.tex| & sample main file \\
    |cdocsch1.tex| & sample include file \\
    |cdocsch2.tex| & sample include file \\
    |cdocspt3.tex| & sample part file \\
    |cdocspt4.tex| & sample part file \\
    |cdocsdrf.tex| & sample redirection file \\
    |cdocsfn1.tex| & sample redirection file \\
    |cdocsfn2.tex| & sample redirection file \\
    |childdoc.pdf| & manual
\end{tabular}
\end{center}
%
The distribution consists of the files
|README.txt|, |childdoc.ins| and |childdoc.dtx|.
%
\begin{itemize}
\item
Run (pdf)\LaTeX{} on |childdoc.dtx|
to compile the manual |childdoc.pdf| (this file).
\item
Run \LaTeX{} on |childdoc.ins| to create the definitions file |childdoc.def|
and the sample |cdocsamp.tex| with include files
|cdocsch1.tex|, |cdocsch2.tex|, |cdocspt3.tex|, |cdocspt4.tex|,
|cdocsdrf.tex|, |cdocsfn1.tex|, |cdocsfn2.tex|.
Then copy the file |childdoc.def| to an appropriate directory of your \LaTeX{}
distribution, e.g.\ \textit{texmf-root}|/tex/latex/childdoc|.
\end{itemize}

%%%%%%%%%%%%%%%%%%%%%%%%%%%%%%%%%%%%%%%%%%%%%%%%%%%%%%%%%%%%%%%%%%%%%%%%%%%%%%%%
\subsection{Related CTAN Packages}

There are several other packages which offer a similar functionality:
%
\begin{itemize}
\item
The packages
\href{http://ctan.org/pkg/docmute}{\textsf{docmute}},
\href{http://ctan.org/pkg/includex}{\textsf{includex}} and
\href{http://ctan.org/pkg/standalone}{\textsf{standalone}}
provide commands to include only the document body of
a child file thus allowing both files to be compiled individually.
\item
The packages \href{http://ctan.org/pkg/subdocs}{\textsf{subdocs}}
and \href{http://ctan.org/pkg/subfiles}{\textsf{subfiles}}
provide structures in which the main and child documents can be
encapsulated and allowing them to be compiled individually.
The inclusion mechanism is different from the conventional |\include|.
\item
The package \href{http://ctan.org/pkg/combine}{\textsf{combine}}
is an elaborate solution to combine several documents into one.
\end{itemize}
%
See also the CTAN topic \href{http://ctan.org/topic/subdocs}{\textsf{subdocs}}
for further related packages.
The present package differs from the above solutions in that
a document structure constructed with the conventional |\include| mechanism
just needs two extra commands at the top of every file
such that all constituent files can be compiled individually.

%%%%%%%%%%%%%%%%%%%%%%%%%%%%%%%%%%%%%%%%%%%%%%%%%%%%%%%%%%%%%%%%%%%%%%%%%%%%%%%%
%\subsection{Feature Suggestions}
%
%The following is a list of features which may be useful for future
%versions of this package:
%%
%\begin{itemize}
%\item
%\ldots
%\end{itemize}

%%%%%%%%%%%%%%%%%%%%%%%%%%%%%%%%%%%%%%%%%%%%%%%%%%%%%%%%%%%%%%%%%%%%%%%%%%%%%%%%
\subsection{Revision History}

%%%%%%%%%%%%%%%%%%%%%%%%%%%%%%%%%%%%%%%%
\paragraph{v2.0:} 2018/12/30

\begin{itemize}
\item
immediate forward processing
\item
added |\childdocby| mechanism
\item
manual restructured
\end{itemize}

%%%%%%%%%%%%%%%%%%%%%%%%%%%%%%%%%%%%%%%%
\paragraph{v1.6:} 2018/01/17

\begin{itemize}
\item
application for development of include files
\item
corrections to manual
\end{itemize}

%%%%%%%%%%%%%%%%%%%%%%%%%%%%%%%%%%%%%%%%
\paragraph{v1.5:} 2017/05/21

\begin{itemize}
\item
more complete structuring introduced
\item
|\childdocof| introduced
\item
|\childdoc| renamed to |\childdocmain|
\item
|\childredirect| renamed to |\childdocforward| and |\childdocforwardprefix|
and functionality expanded
\end{itemize}

%%%%%%%%%%%%%%%%%%%%%%%%%%%%%%%%%%%%%%%%
\paragraph{v1.0:} 2017/04/27

\begin{itemize}
\item
manual and install package
\item
first version published on CTAN
\end{itemize}

%%%%%%%%%%%%%%%%%%%%%%%%%%%%%%%%%%%%%%%%
\paragraph{v0.6:} 2017/04/26

\begin{itemize}
\item
redirection mechanism added
\end{itemize}

%%%%%%%%%%%%%%%%%%%%%%%%%%%%%%%%%%%%%%%%
\paragraph{v0.5:} 2017/04/26

\begin{itemize}
\item
functionality in definition file
\end{itemize}


%%%%%%%%%%%%%%%%%%%%%%%%%%%%%%%%%%%%%%%%%%%%%%%%%%%%%%%%%%%%%%%%%%%%%%%%%%%%%%%%
%%%%%%%%%%%%%%%%%%%%%%%%%%%%%%%%%%%%%%%%%%%%%%%%%%%%%%%%%%%%%%%%%%%%%%%%%%%%%%%%
%%%%%%%%%%%%%%%%%%%%%%%%%%%%%%%%%%%%%%%%%%%%%%%%%%%%%%%%%%%%%%%%%%%%%%%%%%%%%%%%
\appendix

\settowidth\MacroIndent{\rmfamily\scriptsize 000\ }

 \DocInput{childdoc.dtx}

\end{document}
%</driver>
% \fi
%
% %%%%%%%%%%%%%%%%%%%%%%%%%%%%%%%%%%%%%%%%%%%%%%%%%%%%%%%%%%%%%%%%%%%%%%%%%%%%%%
% %%%%%%%%%%%%%%%%%%%%%%%%%%%%%%%%%%%%%%%%%%%%%%%%%%%%%%%%%%%%%%%%%%%%%%%%%%%%%%
% \section{Sample}
%\iffalse
%<*samplemain>
%\fi
%
% The following presents a sample document
% with two chapters, two parts, a title page,
% a compile flag as well as three forwarding files to set the flag.
% It consists of eight |.tex| files:
% \begin{center}
% \begin{tabular}{ll}
% |cdocsamp.tex|&main file\\
% |cdocsch1.tex|&include file for chapter 1\\
% |cdocsch2.tex|&include file for chapter 2\\
% |cdocspt3.tex|&include file for part 3\\
% |cdocspt4.tex|&include file for part 4\\
% |cdocsdrf.tex|&forwarding file for main file in draft mode\\
% |cdocsfi1.tex|&forwarding file for final version of chapter 1\\
% |cdocsfi2.tex|&forwarding file for final version of chapter 2\\
% \end{tabular}
% \end{center}
% Each of the eight files can be compiled directly by the \LaTeX{} compiler.
%
% %%%%%%%%%%%%%%%%%%%%%%%%%%%%%%%%%%%%%%
% \paragraph{Main File.}
%
% The main file is called |cdocsamp.tex|.
%
% Load the \textsf{childdoc} definitions and
% declare the filename for the main document:
%    \begin{macrocode}
\input{childdoc.def}
\childdocmain{}
%    \end{macrocode}

% Optional override for |\version| flag:
%    \begin{macrocode}
%%\ifchilddoc\else\providecommand{\version}{draft}\fi
%    \end{macrocode}

% Define the default values for the |\version| flag
% (|final| for the main file and |draft| for childs):
%    \begin{macrocode}
\ifchilddoc
\providecommand{\version}{draft}
\else
\providecommand{\version}{final}
\fi
%    \end{macrocode}

% Load the standard document class:
%    \begin{macrocode}
\documentclass[12pt]{article}
%    \end{macrocode}

% Start the document body:
%    \begin{macrocode}
\begin{document}
%    \end{macrocode}

% Declare a title page.
% Print title, part of document being processed and version flag:
%    \begin{macrocode}
\addtocounter{page}{-1}
\begin{center}
{\LARGE\bfseries{}childdoc example\par}
\vspace{1cm}
\ifchilddoc
\ifchilddocmanual part\else chapter\fi:
`\childdocname' of `\childdocjob'\par
\else
main document: `\childdocjob'\par
\fi
version: \version\par
\end{center}
\newpage
%    \end{macrocode}

% Manually include selected file,
% otherwise process as usual:
%    \begin{macrocode}
\ifchilddocmanual
\section*{part `\childdocname'}
\input{\childdocname}
\else
%    \end{macrocode}

% Include the two chapters:
%    \begin{macrocode}
\include{cdocsch1}
\include{cdocsch2}
%    \end{macrocode}

% Include the two parts unless only chapters should be displayed:
%    \begin{macrocode}
\ifchilddoc\else
\section{part three}
\input{cdocspt3}
\section{part four}
\input{cdocspt4}
\fi
%    \end{macrocode}

% Process as usual until here:
%    \begin{macrocode}
\fi
%    \end{macrocode}

% End of document body:
%    \begin{macrocode}
\end{document}
%    \end{macrocode}
%\iffalse
%</samplemain>
%\fi
%
% %%%%%%%%%%%%%%%%%%%%%%%%%%%%%%%%%%%%%%
% \paragraph{Chapter Include Files.}
%
% The include files are called |cdocsch1.tex| and |cdocsch2.tex|.
%
%\iffalse
%<*samplechap1|samplechap2>
%\fi

% Optional override for |\version| flag:
%    \begin{macrocode}
%%\providecommand{\version}{final}
%    \end{macrocode}

% Include the main document:
%    \begin{macrocode}
\input{childdoc.def}
\childdocof{cdocsamp}
%    \end{macrocode}

%\iffalse
%</samplechap1|samplechap2>
%\fi
%
%\iffalse
%<*samplechap1>
%\fi
% Some text for chapter 1:
%    \begin{macrocode}
\section{one}
some text in chapter one
%    \end{macrocode}

%\iffalse
%</samplechap1>
%\fi
% Some text for chapter 2:
%\iffalse
%<*samplechap2>
%\fi
%    \begin{macrocode}
\section{two}
more text in chapter two
%    \end{macrocode}

%\iffalse
%</samplechap2>
%\fi
%
% %%%%%%%%%%%%%%%%%%%%%%%%%%%%%%%%%%%%%%
% \paragraph{Part Include Files.}
%
% The include files are called |cdocspt3.tex| and |cdocspt4.tex|.
%
%\iffalse
%<*samplepart3|samplepart4>
%\fi

% Optional override for |\version| flag:
%    \begin{macrocode}
%%\providecommand{\version}{final}
%    \end{macrocode}

% Include the main document:
%    \begin{macrocode}
\input{childdoc.def}
\childdocby{cdocsamp}
%    \end{macrocode}

%\iffalse
%</samplepart3|samplepart4>
%\fi
%
%\iffalse
%<*samplepart3>
%\fi
% Some text for part 3:
%    \begin{macrocode}
some text in part three
%    \end{macrocode}

%\iffalse
%</samplepart3>
%\fi
% Some text for part 4:
%\iffalse
%<*samplepart4>
%\fi
%    \begin{macrocode}
more text in part four
%    \end{macrocode}

%\iffalse
%</samplepart4>
%\fi
%
% %%%%%%%%%%%%%%%%%%%%%%%%%%%%%%%%%%%%%%
% \paragraph{Forwarding for a Complete Draft.}
%
% The following forwarding file |cdocsdrf.tex|
% compiles the main document in draft mode:
%\iffalse
%<*sampledraft>
%\fi
%    \begin{macrocode}
\def\version{draft}
\input{childdoc.def}
\childdocforward{cdocsamp}
%    \end{macrocode}

%\iffalse
%</sampledraft>
%\fi
%
% %%%%%%%%%%%%%%%%%%%%%%%%%%%%%%%%%%%%%%
% \paragraph{Forwarding for Final Version of the Chapters.}
%
% The following forwarding files |cdocsfn1.tex| and |cdocsfn2.tex|
% (with identical content)
% compile the final versions of the child documents
% |cdocsch1.tex| and |cdocsch2.tex|, respectively:
%\iffalse
%<*samplefinal>
%\fi
%    \begin{macrocode}
\def\version{final}
\input{childdoc.def}
\childdocforwardprefix[cdocsamp]{cdocsfn}{cdocsch}
%    \end{macrocode}

%\iffalse
%</samplefinal>
%\fi
%
% %%%%%%%%%%%%%%%%%%%%%%%%%%%%%%%%%%%%%%
% \paragraph{Command Line Processing.}
%
% The following three command lines generate the output files
% |cdocscld|, |cdocscl1| and |cdocscl2|
% which should be identical to
% |cdocsdrf|, |cdocsch1| and |cdocsfn2|, respectively:
% \begin{center}
% \begin{tabular}{l}
% |latex -jobname cdocscld \|\\
% |  "\def\version{draft}\input{childdoc.def}\childdocforward{cdocsamp}"|\\
% |latex -jobname cdocscl1 \|\\
% |  "\input{childdoc.def}\childdocforward[cdocsamp]{cdocsch1}"|\\
% |latex -jobname cdocscl2 \|\\
% |  "\def\version{final}\input{childdoc.def}\childdocforward{cdocsch2}"|
% \end{tabular}
% \end{center}
% Note that the trailing backslash on each first line
% merely continues the input to the second line
% (for convenient cut ant paste).
% Furthermore, the command |latex| can be replaced by any
% of its alternative versions such as |pdflatex|.
%
% %%%%%%%%%%%%%%%%%%%%%%%%%%%%%%%%%%%%%%%%%%%%%%%%%%%%%%%%%%%%%%%%%%%%%%%%%%%%%%
% %%%%%%%%%%%%%%%%%%%%%%%%%%%%%%%%%%%%%%%%%%%%%%%%%%%%%%%%%%%%%%%%%%%%%%%%%%%%%%
% \section{Implementation}
%\iffalse
%<*package>
%\fi
%
% This section describes the definitions file |childdoc.def|.

% The definitions cannot be loaded using |\usepackage| or |\RequirePackage|
% which has a mechanism to prevent loading a style file more than once.
% When loading the definitions by means of |\input|
% multiple instances have to be prevented manually:
%\iffalse
%This code needs to be before the `\ProvidesFile' directive
%which is defined at the beginning of this file.
%Therefore it is also placed there and commented out here.
%</package>
%<*discard>
%\fi
%    \begin{macrocode}
\ifdefined\childdocmain\endinput\fi
%    \end{macrocode}
%\iffalse
%</discard>
%<*package>
%\fi
%
% \macro{\ifchilddoc}
% \macro{\ifchilddocmanual}
% The conditional |\ifchilddoc| tells whether a
% child (true) or main (false) document is being compiled.
% The conditional |\ifchilddocmanual| tells whether
% the |\includeonly| mechanism is used (false) or
% the selection of child files must be performed manually (true).
% The definitions initialise to false:
%    \begin{macrocode}
\newif\ifchilddoc
\newif\ifchilddocmanual
%    \end{macrocode}

% \macro{\childdocname}
% \macro{\childdocjob}
% The macro |\childdocname| stores the name of the main document
% to be compiled. The macro |\childdocjob| stores the name of
% the document on which the \LaTeX{} compiler was originally invoked.
% The content of |\jobname| cannot be compared
% to filenames specified in the source due to different catcodes.
% The following code rescans |\jobname|, stores the result
% in |\childdocname| and saves a copy in |\childdocjob|:
%    \begin{macrocode}
\edef\childdocname{\scantokens\expandafter{\jobname\noexpand}}
\let\childdocjob\childdocname
%    \end{macrocode}

% \macro{\childdocdisable}
% The macro |\childdocdisable| prevents the main file
% from being processed more than once.
% At this stage, the main document command |\childdocmain|
% is assumed to be called once again where it should do nothing.
% Any subsequent call to it should prevent
% a secondary processing of the main document
% It overwrites the forwarding commands
% |\childdocof| and |\childdocforward|
% with empty macros to prevent further inclusions of the main document:
%    \begin{macrocode}
\newcommand{\childdocdisable}
{
  \renewcommand{\childdocmain}[1]{\renewcommand{\childdocmain}[1]{\endinput}}
  \renewcommand{\childdocof}[1]{}
  \renewcommand{\childdocby}[2][]{}
  \renewcommand{\childdocforward}[2][]{}
  \renewcommand{\childdocdisable}{}
}
%    \end{macrocode}

% \macro{\childdocmain}
% The macro |\childdocmain| is to be called at the top of the main file
% with nothing or the main filename (without extension) as argument.
% First, it breaks loops.
% If the argument is not empty and does not match |\childdocname|
% (which is set by the first inclusion of |childdoc.def|),
% |\ifchilddoc| is set to true, |\includeonly| is applied to the child file
% and |\jobname| is set to the main file
% (for proper handling of |.aux| files):
%    \begin{macrocode}
\newcommand{\childdocmain}[1]
{
  \childdocdisable\childdocmain{}
  \if?#1?\else
    \begingroup
      \def\childdoctmp{#1}
      \ifx\childdoctmp\childdocname
        \def\childdoctmp{}
      \else
        \def\childdoctmp
        {
          \childdoctrue
          \includeonly{\childdocname}
          \def\childdocjob{#1}
          \def\jobname{#1}
        }
      \fi
      \expandafter
    \endgroup
    \childdoctmp
  \fi
}
%    \end{macrocode}

% \macro{\childdocof}
% The command |\childdocof| redirects
% compilation to the main file |#1|.
%    \begin{macrocode}
\newcommand{\childdocof}[1]
{
  \childdocdisable
  \childdoctrue
  \includeonly{\childdocname}
  \def\jobname{#1}
  \def\childdocjob{#1}
  \input{#1}
}
%    \end{macrocode}

% \macro{\childdocby}
% The command |\childdocby| ....
%    \begin{macrocode}
\newcommand{\childdocby}[2][]
{
  \childdocdisable
  \childdoctrue
  \childdocmanualtrue
  \if?#1?\else
    \def\jobname{#2}
  \fi
  \def\childdocjob{#2}
  \input{#2}
  \endinput
}
%    \end{macrocode}

% \macro{\childdocforward}
% The command |\childdocforward| redirects
% compilation to the main file or
% (if the optional argument is given) a child file.
% Parameters are set as if the main file
% or a child file starting with |\childdocof| was compiled.
% Then compilation is handed over to the main file:
%    \begin{macrocode}
\newcommand{\childdocforward}[2][]
{
  \begingroup
    \if?#1?
      \def\childdoctmp
      {
        \def\childdocname{#2}
        \def\childdocjob{#2}
        \def\jobname{#2}
        \input{#2}
        \endinput
      }
    \else
      \def\childdoctmp
      {
        \childdocdisable
        \def\childdocname{#2}
        \childdoctrue
        \includeonly{#2}
        \def\childdocjob{#1}
        \def\jobname{#1}
        \input{#1}
        \endinput
      }
    \fi
    \expandafter
  \endgroup
  \childdoctmp
}
%    \end{macrocode}

% \macro{\childdocforwardprefix}
% The command |\childdocforwardprefix| redirects
% compilation to the main or a child file by means of a pattern.
% The prefix |#1| in the current filename is replaced by |#2|
% and the suffix of the current filename is kept
% (it is assumed that the filename does not contain the substring `|~~~|'
% which is used as a delimiter).
% Compilation is handed over to the new file by |\childdocforward|:
%    \begin{macrocode}
\newcommand{\childdocforwardprefix}[3][]
{
  \begingroup
    \def\childdocextract #2##1~~~{\def\childdoctmp{\childdocforward[#1]{#3##1}}}
    \expandafter\childdocextract\childdocname~~~
    \expandafter
  \endgroup
  \childdoctmp
}
%    \end{macrocode}

% \macro{\childdoc}
% The deprecated macro |\childdoc| is a legacy version of |\childdocmain|:
%    \begin{macrocode}
\newcommand{\childdoc}{\childdocmain}
%    \end{macrocode}

% \macro{\childdocredirect}
% The deprecated macro |\childdocredirect| is a legacy version
% of |\childdocforward| and |\childdocforwardprefix|:
%    \begin{macrocode}
\newcommand{\childdocredirect}[2][]
{
  \begingroup
    \if?#1?
      \def\childdoctmp{\childdocforward{#2}}
    \else
      \def\childdoctmp{\childdocforwardprefix{#1}{#2}}
    \fi
    \expandafter
  \endgroup
  \childdoctmp
}
%    \end{macrocode}

%\iffalse
%</package>
%\fi
%
\endinput
\childdocforward[cdocsamp]{cdocsch1}"|\\
% |latex -jobname cdocscl2 \|\\
% |  "\def\version{final}% \iffalse
%
% childdoc.dtx Copyright (C) 2017-2018 Niklas Beisert
%
% This work may be distributed and/or modified under the
% conditions of the LaTeX Project Public License, either version 1.3
% of this license or (at your option) any later version.
% The latest version of this license is in
%   http://www.latex-project.org/lppl.txt
% and version 1.3 or later is part of all distributions of LaTeX
% version 2005/12/01 or later.
%
% This work has the LPPL maintenance status `maintained'.
%
% The Current Maintainer of this work is Niklas Beisert.
%
% This work consists of the files childdoc.dtx and childdoc.ins
% and the derived files childdoc.def and cdocsamp.tex with
% cdocsch1.tex, cdocsch2.tex, cdocsdrf.tex, cdocsfn1.tex, cdocsfn2.tex.
%
%<package>\ifdefined\childdocmain\endinput\fi
%<package>\ProvidesFile{childdoc.def}[2018/12/30 v2.0 child document driver]
%<samplemain>\ProvidesFile{cdocsamp.tex}[2018/12/30 v2.0 sample for childdoc]
%<*driver>
%\ProvidesFile{childdoc.drv}[2018/12/30 v2.0 childdoc reference manual file]
\PassOptionsToClass{10pt,a4paper}{article}
\documentclass{ltxdoc}

\usepackage[margin=35mm]{geometry}
\usepackage{hyperref}
\usepackage{hyperxmp}
\usepackage[usenames]{color}

\hypersetup{colorlinks=true}
\hypersetup{pdfstartview=FitH}
\hypersetup{pdfpagemode=UseNone}
\hypersetup{pdfsource={}}
\hypersetup{pdflang={en-UK}}
\hypersetup{pdfcopyright={Copyright 2017-2018 Niklas Beisert.
  This work may be distributed and/or modified under the
  conditions of the LaTeX Project Public License, either version 1.3
  of this license or (at your option) any later version.}}
\hypersetup{pdflicenseurl={http://www.latex-project.org/lppl.txt}}
\hypersetup{pdfcontactaddress={ETH Zurich, ITP, HIT K,
  Wolfgang-Pauli-Strasse 27}}
\hypersetup{pdfcontactpostcode={8093}}
\hypersetup{pdfcontactcity={Zurich}}
\hypersetup{pdfcontactcountry={Switzerland}}
\hypersetup{pdfcontactemail={nbeisert@itp.phys.ethz.ch}}
\hypersetup{pdfcontacturl={http://people.phys.ethz.ch/\xmptilde nbeisert/}}

\newcommand{\secref}[1]{\hyperref[#1]{section \ref*{#1}}}

\parskip1ex
\parindent0pt
\let\olditemize\itemize
\def\itemize{\olditemize\parskip0pt}

\begin{document}

\title{The \textsf{childdoc} Package}
\hypersetup{pdftitle={The childdoc Package}}
\author{Niklas Beisert\\[2ex]
  Institut f\"ur Theoretische Physik\\
  Eidgen\"ossische Technische Hochschule Z\"urich\\
  Wolfgang-Pauli-Strasse 27, 8093 Z\"urich, Switzerland\\[1ex]
  \href{mailto:nbeisert@itp.phys.ethz.ch}
  {\texttt{nbeisert@itp.phys.ethz.ch}}}
\hypersetup{pdfauthor={Niklas Beisert}}
\hypersetup{pdfsubject={Manual for the LaTeX2e Package childdoc}}
\date{30 December 2018, \textsf{v2.0}}
\maketitle

\begin{abstract}\noindent
\textsf{childdoc} is a \LaTeXe{} package
that enables the direct compilation
of document sections included by |\include|
to individual files.
\end{abstract}

\begingroup
\parskip0ex
\tableofcontents
\endgroup

%%%%%%%%%%%%%%%%%%%%%%%%%%%%%%%%%%%%%%%%%%%%%%%%%%%%%%%%%%%%%%%%%%%%%%%%%%%%%%%%
%%%%%%%%%%%%%%%%%%%%%%%%%%%%%%%%%%%%%%%%%%%%%%%%%%%%%%%%%%%%%%%%%%%%%%%%%%%%%%%%
\section{Introduction}

\LaTeX{} provides a mechanism to structure a large document (such as a book)
into a main file and several child files (containing the chapters)
using the |\include| command.
This mechanism is beneficial for documents
which span hundreds of pages in order to
make the source file(s) more manageable.
Moreover, compilation can be restricted to
selected child files by means of the |\includeonly| command.
The latter feature can be used to reduce the compilation time while editing
(this was significantly more useful in the earlier days of \LaTeX{})
or to generate a smaller document which is easier to navigate.
Another application of |\includeonly| is to generate
documents consisting of selected parts of the complete document.

However, there are a few drawbacks of the plain |\include| mechanism:
\begin{itemize}
\item
The child files cannot be compiled on their own,
they can only be compiled via the main file.
A naive editing environment
(such as a text editor with an option
to have the current file processed by \LaTeX)
may require one to switch to the main file before compiling;
attempting to compile the child file produces errors.
\item
The main file must be modified (each time)
to adjust the |\includeonly| command
to the present needs. This easily leaves the main file in a messy state.
\item
The generated document will always carry the filename
of the main document. This is inconvenient if
several child files are to be compiled and
to be kept for distribution.
\end{itemize}

The present package provides a simple interface
to make child files individually compilable by \LaTeX{}.
Compiling a child file then has the same effect as compiling
the main file with an |\includeonly| command
to select the appropriate child.
Moreover the generated document will carry the name of the child
rather than the main file.
This resolves all three above issues.

This feature is meant to make the editing of books,
thesis documents and lecture notes somewhat more convenient.
However, the package can also be used efficiently for
composing a series of documents (such as exercise sheets)
which are typically distributed individually.
It then assists the author in generating the individual documents
(potentially in different versions)
as well as a document containing the collected series.
Another application is in developing style files
or other kinds of included material
where compilation of the style file could redirect
to a sample or test file.

%%%%%%%%%%%%%%%%%%%%%%%%%%%%%%%%%%%%%%%%%%%%%%%%%%%%%%%%%%%%%%%%%%%%%%%%%%%%%%%%
%%%%%%%%%%%%%%%%%%%%%%%%%%%%%%%%%%%%%%%%%%%%%%%%%%%%%%%%%%%%%%%%%%%%%%%%%%%%%%%%
\section{Usage}

First of all, the package \textsf{childdoc} is \emph{not} a standard
\LaTeXe{} |.sty| style file! Therefore it needs to be invoked in
a non-standard way.

%%%%%%%%%%%%%%%%%%%%%%%%%%%%%%%%%%%%%%%%%%%%%%%%%%%%%%%%%%%%%%%%%%%%%%%%%%%%%%%%
\subsection{Included Files}
\label{sec:include}

%%%%%%%%%%%%%%%%%%%%%%%%%%%%%%%%%%%%%%%%
\DescribeMacro{\childdocmain}
To use the package, add the commands
\begin{center}
\begin{tabular}{l}
|\input{childdoc.def}|\\
|\childdocmain{}|\\
\end{tabular}
\end{center}
at the very top of the main \LaTeX{} file,
in particular \emph{before} the |\documentclass| statement!
The argument of |\childdocmain| should be left empty
(but it must be present).

%%%%%%%%%%%%%%%%%%%%%%%%%%%%%%%%%%%%%%%%
\DescribeMacro{\childdocof}
Furthermore, add the commands
\begin{center}
\begin{tabular}{l}
|\input{childdoc.def}|\\
|\childdocof{|\textit{main}|}|\\
\end{tabular}
\end{center}
at the top of every child file \textit{child}
which is included by |\include{|\textit{child}|}|
from within the main file
(or at least for those files to be compiled individually).
The argument \textit{main} must be the filename of the main file.

There are a couple of
considerations in setting up the main and child documents:

%%%%%%%%%%%%%%%%%%%%%%%%%%%%%%%%%%%%%%%%
\paragraph{Restrictions.}

Please note the following restrictions:
\begin{itemize}
\item
|\childdocmain| must be called with one argument \textit{main}
to ensure compatibility with earlier version of the package.
It must either be empty (|\childdocmain{}|)
or precisely match the filename of the main file in which it is specified.
See \secref{sec:detection} for further information.
\item
The filename \textit{main} must be specified without the |.tex| extension.
\item
The filename \textit{main} is case sensitive
(even in case-insensitive file systems)
due to internal string comparison.
\item
The argument \textit{main} should be fully expanded, it cannot be a macro.
\item
Subdirectories and special characters should be avoided in filenames.
\item
The command |\childdocmain{|\textit{main}|}| must be followed by a whitespace.
It should not be followed immediately by another command
or by a comment mark `|%|'.
This is because the \TeX{} parser reads the token immediately following
the argument of |\childdocmain| and puts it
at the beginning of every child section;
however, a white\-space is ignored.
\end{itemize}

%%%%%%%%%%%%%%%%%%%%%%%%%%%%%%%%%%%%%%%%
\paragraph{Content of Main File.}

It is advisable to place all content in the child files included by |\include|.
Any output contained in the main file will appear in all child documents
unless suppressed manually;
it cannot be suppressed automatically by the |\includeonly| directive
and thus should normally be avoided.
A method to include some content in the main file
by means of conditional processing is described in \secref{sec:conditional}.

%%%%%%%%%%%%%%%%%%%%%%%%%%%%%%%%%%%%%%%%
\paragraph{Page Numbering.}

When only a part of the document is compiled,
the appropriate numbering of pages
(as well as other status parameters)
is determined from the |.aux| files.
The latter contain information from previous passes.
However this information needs to propagate through
all intermediate child documents.
Therefore the page numbering in child documents may well
be inconsistent until the complete document is compiled at least once.

A useful (if unconventional) way to always ensure a consistent
page numbering is to restart the numbering in each child document
and denote the pages by `\textit{child}|.|\textit{page}'
where \textit{child} represents the chapter/section number of the child file.
This can be achieved by the command
|\numberwithin{page}{|\textit{child}|}|
of the \textsf{amsmath} package
where \textit{child} can be |chapter| or |section|
depending on the chosen structuring.
Alternatively, one can modify the macro |\thepage| appropriately
and reset the counter |page| at the start of each child file.

%%%%%%%%%%%%%%%%%%%%%%%%%%%%%%%%%%%%%%%%%%%%%%%%%%%%%%%%%%%%%%%%%%%%%%%%%%%%%%%%
\subsection{Conditional Processing}
\label{sec:conditional}

The package provides a mechanism to compile different versions
of a document. To customise the versions further some conditional processing
can come in handy to distinguish which version is being compiled.
The package provides two macros to describe the compilation context:

%%%%%%%%%%%%%%%%%%%%%%%%%%%%%%%%%%%%%%%%
\DescribeMacro{\ifchilddoc}
The conditional |\ifchilddoc| distinguishes between the compilation of
child documents and the main document:
%
\begin{center}
|\ifchilddoc |\textit{child-code}| |[|\||else |\textit{main-code}]| \||fi|
\end{center}

%%%%%%%%%%%%%%%%%%%%%%%%%%%%%%%%%%%%%%%%
\DescribeMacro{\childdocname}
\DescribeMacro{\childdocjob}
The macro |\childdocname| contains the filename (without extension)
of the main or child file being processed.
Note that |\childdocjob| will always contain the name of the main file.

%%%%%%%%%%%%%%%%%%%%%%%%%%%%%%%%%%%%%%%%
\paragraph{Title Page.}

Conditional processing can be used to include a title or banner page
in the main document when proper precautions are taken.
Importantly, the code in the main file should ensure that the page counter
(as well as other status parameters which are stored in the |.aux| files)
takes the same value after the conditional processing.
Otherwise the page numbers may take divergent values
depending on which part is compiled.

For example, a title page could be declared by:
%
\begin{center}
\begin{tabular}{l}
|\ifchilddoc\||else|\\
|\addtocounter{page}{-1}|\\
\textit{code for title page}\\
|\newpage|\\
|\||fi|
\end{tabular}
\end{center}
%
A banner page for the child documents can be generated by:
%
\begin{center}
\begin{tabular}{l}
|\ifchilddoc|\\
|\addtocounter{page}{-1}|\\
\textit{code for banner page}\\
|\newpage|\\
|\||fi|
\end{tabular}
\end{center}
%
Here one could write a message such as:
\begin{center}
|This is the part \childdocname{} of \childdocjob{}.|
\end{center}

%%%%%%%%%%%%%%%%%%%%%%%%%%%%%%%%%%%%%%%%%%%%%%%%%%%%%%%%%%%%%%%%%%%%%%%%%%%%%%%%
\subsection{Flags}
\label{sec:flags}

The package makes it easy to generate different versions
of the main or child documents.
To this end compilation flags can be defined
and assigned different default values.
They will be particularly useful in conjunction
with the forwarding mechanism described in \secref{sec:forward}.

For example, it may be useful to have a flag |\version|
which can be set to |draft| or |final|.
The document source will contain some conditional code
depending on the value of |\version|.
Suppose further, the flag should default to |final| for the main file
and to |draft| for child files
which is a natural assignment for editing the document.
This is achieved by placing the following code
in the preamble of the main document
(below the |\childdocmain| directive):
%
\begin{center}
\begin{tabular}{l}
|\ifchilddoc|\\
|\providecommand{\version}{draft}|\\
|\||else|\\
|\providecommand{\version}{final}|\\
|\||fi|
\end{tabular}
\end{center}
%
The definition by |\providecommand| makes sure
that previous definitions are not overwritten.
Further statements |\providecommand{\version}{...}|
can thus be added before the above code to override it.

For the main file, one might add a line
(between |\childdocmain| and the above block)
%
\begin{center}
|%\ifchilddoc\||else\providecommand{\version}{draft}\||fi|
\end{center}
%
which can be uncommented to produce a draft version.
Likewise one can add a line to the very top of a child file
(above the |\childdocof{|\textit{main}|}| directive)
%
\begin{center}
|%\providecommand{\version}{final}|
\end{center}
%
which can be uncommented to produce the final version of this child document.

%%%%%%%%%%%%%%%%%%%%%%%%%%%%%%%%%%%%%%%%%%%%%%%%%%%%%%%%%%%%%%%%%%%%%%%%%%%%%%%%
\subsection{Forwarding}
\label{sec:forward}

Different versions of the main or child documents
using compilation flags as described in \secref{sec:flags}
can be (permanently) stored in different files
for convenient compilation, viewing and distribution.
To this end, the package defines a command
to pass on compilation to a different file:

%%%%%%%%%%%%%%%%%%%%%%%%%%%%%%%%%%%%%%%%
\DescribeMacro{\childdocforward}
The command |\childdocforward| redirects processing to
another source file:
%
\begin{center}
\begin{tabular}{l}
|\input{childdoc.def}|\\
|\childdocforward[|\textit{main}|]{|\textit{dest}|}|\\
\end{tabular}
\end{center}
%
The argument \textit{dest} is the destination file
(without extension).
It should be the main file or one of the child files.
Note that further \textsf{childdoc} directives
such as |\childdocof| and |\childdocforward|
in the indicated file will be processed in this form.
The optional argument \textit{main}
passes on directly to the main file \textit{main}
while pretending to compile the child \textit{dest}.
This form behaves as if \textit{dest}
issues |\childdocof{|\textit{main}|}| right away,
and no further \textsf{childdoc} directives will be processed.

%%%%%%%%%%%%%%%%%%%%%%%%%%%%%%%%%%%%%%%%
\DescribeMacro{\...prefix}
In the alternative form |\childdocforwardprefix|,
%
\begin{center}
\begin{tabular}{l}
|\input{childdoc.def}|\\
|\childdocforwardprefix[|\textit{main}|]{|\textit{prefix}|}{|\textit{dest}|}|
\end{tabular}
\end{center}
%
the destination file is determined by a pattern
depending on the current file:
To make this work, the current file must be called
`{\textit{prefix}\hspace{0.2em}\textit{suffix}}'
with \textit{prefix} matching precisely the argument.
Processing is then passed on to the file
`{\textit{dest}\hspace{0.2em}\textit{suffix}}'.
Surely, the same effect is achieved by
directly specifying the
argument `{\textit{dest}\hspace{0.2em}\textit{suffix}}'
in the first form.
However, that requires to set up a different file
for each child. With the alternative form of the command
all these files can have exactly the same content
which simplifies setting them up and maintaining them.

For example, the following file |draft.tex|
with a compilation flag |\version| as described in \secref{sec:flags}
compiles the main document as a draft:
%
\begin{center}
\begin{tabular}{l}
|\def\version{draft}|\\
|\input{childdoc.def}|\\
|\childdocforward{|\textit{main}|}|
\end{tabular}
\end{center}
%
Likewise, the following files |final|\textit{nn}|.tex|
compile the final version of the child document
|child|\textit{nn}|.tex|:
%
\begin{center}
\begin{tabular}{l}
|\def\version{final}|\\
|\input{childdoc.def}|\\
|\childdocforwardprefix{final}{child}|
\end{tabular}
\end{center}
%

Note that when several versions of a main file and/or of each child file
are to be generated, it may be convenient to set up a |Makefile| or
shell script to automatise the process.

%%%%%%%%%%%%%%%%%%%%%%%%%%%%%%%%%%%%%%%%%%%%%%%%%%%%%%%%%%%%%%%%%%%%%%%%%%%%%%%%
\subsection{Command Line Processing}
\label{sec:commandline}

The effect of redirection files can also be achieved by invoking
the \LaTeX{} compiler with a more elaborate command line.
Most conveniently this should be done as part
of a shell script or a |Makefile|.

When using \textsf{childdoc} in the main file, the following
command lines effectively perform a redirection
(note that depending on the shell being used,
backslashes may have to be doubled: `|\|' $\to$ `|\\|'):
%
\begin{center}
|... -jobname "|\textit{target}|" |\\|"|[\textit{flags}]%
|\input{childdoc.def}\childdocforward[|\textit{main}|]{|\textit{dest}|}"|
\end{center}
%
Here \textit{target} is the name of the output file,
\textit{main} is the name of the main file
and \textit{dest} is the name of the main or child file to be processed
(all filenames without extensions).
The optional argument \textit{main} can be omitted
if \textit{main} matches \textit{dest}.
Optionally, compilation \textit{flags} can be defined via |\def| commands.
This command line makes the \TeX{} engine believe
it is compiling the file \textit{target}
whose content is specified as the latter parameter.
The provided code then forwards the processing to
\textit{main} or \textit{dest} as described in \secref{sec:forward}.

%%%%%%%%%%%%%%%%%%%%%%%%%%%%%%%%%%%%%%%%%%%%%%%%%%%%%%%%%%%%%%%%%%%%%%%%%%%%%%%%
\subsection{Include by Input}
\label{sec:input}

Including child documents by |\include| has some restrictions by design.
Most notably, the content of a child document always occupies
its own set of pages; pages cannot be shared between child documents.
Usually, this behaviour makes perfect sense
because each child document contain an essential part of the document.
However, in some situations it may be desirable to compose
a document from a collection of parts
without having mandatory page breaks between then.
For this case, the package
provides a mechanism to include parts
by |\input| which can also be processed individually.
However, by construction this mechanism
requires manual handling of the content to be output.

%%%%%%%%%%%%%%%%%%%%%%%%%%%%%%%%%%%%%%%%
\DescribeMacro{\ifchilddocmanual}
The main file should be prepared as usual, see \secref{sec:include}.
However, the document body must make a distinction
between processing of an individual part and of the main document, e.g.:
%
\begin{center}
\begin{tabular}{l}
|\ifchilddocmanual|\\
|\input{\childdocname}|\\
|\||else|\\
\textit{document body with }|\input{|\textit{part}|}|\\
|\||fi|
\end{tabular}
\end{center}
%
The conditional |\ifchilddocmanual| is true whenever
a part to be included by |\input| is being compiled,
and the name of the part is stored in |\childdocname|.

%%%%%%%%%%%%%%%%%%%%%%%%%%%%%%%%%%%%%%%%
\DescribeMacro{\childdocby}
Each part to be included by |\input| should start with:
%
\begin{center}
\begin{tabular}{l}
|\input{childdoc.def}|\\
|\childdocby{|\textit{main}|}|\\
\end{tabular}
\end{center}
%
The directive |\childdocby| is similar to |\childdocof|
described in \secref{sec:include},
but the subsequent selection of content must be done manually.
To that end, both |\ifchilddoc| and |\ifchilddocmanual|
will be true upon processing of a part,
and the name of the part is stored in |\childdocname|.
Note that |\jobname| will be set to the filename of the current part
so that each part receives an individual |.aux| file
that does not interfere with the |.aux| file(s) of the main document.
This behaviour can be altered by the alternative form
|\childdocby[*]{|\textit{main}|}| (with a non-empty optional argument)
which uses the |.aux| file of the main document
by setting |\jobname| to \textit{main}.

%%%%%%%%%%%%%%%%%%%%%%%%%%%%%%%%%%%%%%%%%%%%%%%%%%%%%%%%%%%%%%%%%%%%%%%%%%%%%%%%
\subsection{Driver Development}
\label{sec:driver}

The \textsf{childdoc} mechanism can also be use for the development
of definition files such as \LaTeX{} styles or classes.
This case differs from the above setup with multiple parts
included by |\include| in that no |\includeonly| should be invoked.
This can be achieved by starting the include file
(before |\ProvidesPackage|) with:
%
\begin{center}
\begin{tabular}{l}
|\input{childdoc.def}|\\
|\childdocforward{|\textit{main}|}|\\
\end{tabular}
\end{center}
%
or alternatively with:
%
\begin{center}
\begin{tabular}{l}
|\input{childdoc.def}|\\
|\childdocby{|\textit{main}|}|\\
\end{tabular}
\end{center}
%
Both forms have slightly different effects as described above.
The main file is prepared as usual, see \secref{sec:include}.

%%%%%%%%%%%%%%%%%%%%%%%%%%%%%%%%%%%%%%%%%%%%%%%%%%%%%%%%%%%%%%%%%%%%%%%%%%%%%%%%
\subsection{Legacy Detection}
\label{sec:detection}

The directive |\childdocmain| in the main file can detect
whether the complete document or merely a child is to be compiled
even without using the directive |\childdocof|.
This method is deprecated because it is less robust
and there is no compelling reason to use it;
it is merely provided for backward compatibility
and it may be removed in future versions.

If the detection mechanism is to be used,
it is mandatory to correctly specify
the filename of the main file as the argument of |\childdocmain|:
%
\begin{center}
\begin{tabular}{l}
|\input{childdoc.def}|\\
|\childdocmain{|\textit{main}|}|\\
\end{tabular}
\end{center}
%
If |\jobname| does not match the argument \textit{main} of |\childdocmain|,
it is assumed that |\jobname| points to the child file to be compiled.
When using |\childdocmain| with the main file specified as argument,
it suffices to start a child file
with just |\input{|\textit{main}|}|
without loading of the package and using |\childdocof|.
If instead all processing is done
with the appropriate \textsf{childdoc} directives,
the argument of \textit{main} of |\childdocmain| can be empty.

An alternative version of the command line processing described
in \secref{sec:commandline} using the detection mechanism reads:
%
\begin{center}
|... -jobname "|\textit{target}|" "|[\textit{flags}]%
[|\def\jobname{|\textit{dest}|}|]|\input{|\textit{main}|}"|
\end{center}

%%%%%%%%%%%%%%%%%%%%%%%%%%%%%%%%%%%%%%%%%%%%%%%%%%%%%%%%%%%%%%%%%%%%%%%%%%%%%%%%
\subsection{Manual Code}
\label{sec:manual}

In case one cannot be certain whether the definitions file |childdoc.def|
is installed on the target \TeX{} distribution
and one prefers not to ship it,
it is conceivable to paste a few relevant commands into the sources.

To that end, drop all statements |\input{childdoc.def}|
and perform the replacements as outlined below.
Instead of |\childdocmain{|\textit{main}|}| add the following code
to the top of the main file:
%
\begin{center}
\begin{tabular}{l}
|\||ifdefined\childdocname\endinput\||fi\newif\ifchilddoc|\\
|\edef\childdocname{\scantokens\expandafter{\jobname\noexpand}}|\\
|\def\childdocmain{|\textit{main}|}\||ifx\childdocmain\childdocname\||else|\\
|\childdoctrue\includeonly{\childdocname}\let\jobname\childdocmain\||fi|\\
\end{tabular}
\end{center}
%
Instead of |\childdocof{|\textit{main}|}| just include the main file
at the top of each child file:
%
\begin{center}
|\input{|\textit{main}|}|
\end{center}
%
A simple redirection |\childdocforward{|\textit{dest}|}| is achieved by:
%
\begin{center}
|\def\jobname{|\textit{dest}|}\input{\jobname}|
\end{center}
%
The redirection with prefix
|\childdocforwardprefix[|\textit{prefix}|]{|\textit{dest}|}|
is accomplished by:
%
\begin{center}
\begin{tabular}{l}
|{\edef\jobname{\scantokens\expandafter{\jobname\noexpand}}|\\
|\def\redirectjob |\textit{prefix}|#1~~~{\gdef\jobname{|\textit{dest}|#1}}|\\
|\expandafter\redirectjob\jobname~~~}\input{\jobname}|
\end{tabular}
\end{center}

In an alternative approach,
child documents can be compiled by a specific command line
without additional code or specific definitions:
%
\begin{center}
|... -jobname "|\textit{target}|" "|[\textit{flags}]%
|\includeonly{|\textit{dest}|}\input{|\textit{main}|}"|
\end{center}
%

%%%%%%%%%%%%%%%%%%%%%%%%%%%%%%%%%%%%%%%%%%%%%%%%%%%%%%%%%%%%%%%%%%%%%%%%%%%%%%%%
%%%%%%%%%%%%%%%%%%%%%%%%%%%%%%%%%%%%%%%%%%%%%%%%%%%%%%%%%%%%%%%%%%%%%%%%%%%%%%%%
\section{Information}

%%%%%%%%%%%%%%%%%%%%%%%%%%%%%%%%%%%%%%%%%%%%%%%%%%%%%%%%%%%%%%%%%%%%%%%%%%%%%%%%
\subsection{Copyright}

Copyright \copyright{} 2017--2018 Niklas Beisert

This work may be distributed and/or modified under the
conditions of the \LaTeX{} Project Public License, either version 1.3
of this license or (at your option) any later version.
The latest version of this license is in
  \url{http://www.latex-project.org/lppl.txt}
and version 1.3 or later is part of all distributions of \LaTeX{}
version 2005/12/01 or later.

This work has the LPPL maintenance status `maintained'.

The Current Maintainer of this work is Niklas Beisert.

This work consists of the files |README.txt|, |childdoc.ins| and |childdoc.dtx|
as well as the derived files |childdoc.def|, |cdocsamp.tex|
with |cdocsch1.tex|, |cdocsch2.tex|, |cdocspt3.tex|, |cdocspt4.tex|,
|cdocsdrf.tex|, |cdocsfn1.tex|, |cdocsfn2.tex|
as well as |childdoc.pdf|.

%%%%%%%%%%%%%%%%%%%%%%%%%%%%%%%%%%%%%%%%%%%%%%%%%%%%%%%%%%%%%%%%%%%%%%%%%%%%%%%%
\subsection{Files and Installation}

The package consists of the files:
%
\begin{center}
\begin{tabular}{ll}
    |README.txt|   & readme file \\
    |childdoc.ins| & installation file \\
    |childdoc.dtx| & source file \\
    |childdoc.def| & definition file \\
    |cdocsamp.tex| & sample main file \\
    |cdocsch1.tex| & sample include file \\
    |cdocsch2.tex| & sample include file \\
    |cdocspt3.tex| & sample part file \\
    |cdocspt4.tex| & sample part file \\
    |cdocsdrf.tex| & sample redirection file \\
    |cdocsfn1.tex| & sample redirection file \\
    |cdocsfn2.tex| & sample redirection file \\
    |childdoc.pdf| & manual
\end{tabular}
\end{center}
%
The distribution consists of the files
|README.txt|, |childdoc.ins| and |childdoc.dtx|.
%
\begin{itemize}
\item
Run (pdf)\LaTeX{} on |childdoc.dtx|
to compile the manual |childdoc.pdf| (this file).
\item
Run \LaTeX{} on |childdoc.ins| to create the definitions file |childdoc.def|
and the sample |cdocsamp.tex| with include files
|cdocsch1.tex|, |cdocsch2.tex|, |cdocspt3.tex|, |cdocspt4.tex|,
|cdocsdrf.tex|, |cdocsfn1.tex|, |cdocsfn2.tex|.
Then copy the file |childdoc.def| to an appropriate directory of your \LaTeX{}
distribution, e.g.\ \textit{texmf-root}|/tex/latex/childdoc|.
\end{itemize}

%%%%%%%%%%%%%%%%%%%%%%%%%%%%%%%%%%%%%%%%%%%%%%%%%%%%%%%%%%%%%%%%%%%%%%%%%%%%%%%%
\subsection{Related CTAN Packages}

There are several other packages which offer a similar functionality:
%
\begin{itemize}
\item
The packages
\href{http://ctan.org/pkg/docmute}{\textsf{docmute}},
\href{http://ctan.org/pkg/includex}{\textsf{includex}} and
\href{http://ctan.org/pkg/standalone}{\textsf{standalone}}
provide commands to include only the document body of
a child file thus allowing both files to be compiled individually.
\item
The packages \href{http://ctan.org/pkg/subdocs}{\textsf{subdocs}}
and \href{http://ctan.org/pkg/subfiles}{\textsf{subfiles}}
provide structures in which the main and child documents can be
encapsulated and allowing them to be compiled individually.
The inclusion mechanism is different from the conventional |\include|.
\item
The package \href{http://ctan.org/pkg/combine}{\textsf{combine}}
is an elaborate solution to combine several documents into one.
\end{itemize}
%
See also the CTAN topic \href{http://ctan.org/topic/subdocs}{\textsf{subdocs}}
for further related packages.
The present package differs from the above solutions in that
a document structure constructed with the conventional |\include| mechanism
just needs two extra commands at the top of every file
such that all constituent files can be compiled individually.

%%%%%%%%%%%%%%%%%%%%%%%%%%%%%%%%%%%%%%%%%%%%%%%%%%%%%%%%%%%%%%%%%%%%%%%%%%%%%%%%
%\subsection{Feature Suggestions}
%
%The following is a list of features which may be useful for future
%versions of this package:
%%
%\begin{itemize}
%\item
%\ldots
%\end{itemize}

%%%%%%%%%%%%%%%%%%%%%%%%%%%%%%%%%%%%%%%%%%%%%%%%%%%%%%%%%%%%%%%%%%%%%%%%%%%%%%%%
\subsection{Revision History}

%%%%%%%%%%%%%%%%%%%%%%%%%%%%%%%%%%%%%%%%
\paragraph{v2.0:} 2018/12/30

\begin{itemize}
\item
immediate forward processing
\item
added |\childdocby| mechanism
\item
manual restructured
\end{itemize}

%%%%%%%%%%%%%%%%%%%%%%%%%%%%%%%%%%%%%%%%
\paragraph{v1.6:} 2018/01/17

\begin{itemize}
\item
application for development of include files
\item
corrections to manual
\end{itemize}

%%%%%%%%%%%%%%%%%%%%%%%%%%%%%%%%%%%%%%%%
\paragraph{v1.5:} 2017/05/21

\begin{itemize}
\item
more complete structuring introduced
\item
|\childdocof| introduced
\item
|\childdoc| renamed to |\childdocmain|
\item
|\childredirect| renamed to |\childdocforward| and |\childdocforwardprefix|
and functionality expanded
\end{itemize}

%%%%%%%%%%%%%%%%%%%%%%%%%%%%%%%%%%%%%%%%
\paragraph{v1.0:} 2017/04/27

\begin{itemize}
\item
manual and install package
\item
first version published on CTAN
\end{itemize}

%%%%%%%%%%%%%%%%%%%%%%%%%%%%%%%%%%%%%%%%
\paragraph{v0.6:} 2017/04/26

\begin{itemize}
\item
redirection mechanism added
\end{itemize}

%%%%%%%%%%%%%%%%%%%%%%%%%%%%%%%%%%%%%%%%
\paragraph{v0.5:} 2017/04/26

\begin{itemize}
\item
functionality in definition file
\end{itemize}


%%%%%%%%%%%%%%%%%%%%%%%%%%%%%%%%%%%%%%%%%%%%%%%%%%%%%%%%%%%%%%%%%%%%%%%%%%%%%%%%
%%%%%%%%%%%%%%%%%%%%%%%%%%%%%%%%%%%%%%%%%%%%%%%%%%%%%%%%%%%%%%%%%%%%%%%%%%%%%%%%
%%%%%%%%%%%%%%%%%%%%%%%%%%%%%%%%%%%%%%%%%%%%%%%%%%%%%%%%%%%%%%%%%%%%%%%%%%%%%%%%
\appendix

\settowidth\MacroIndent{\rmfamily\scriptsize 000\ }

 \DocInput{childdoc.dtx}

\end{document}
%</driver>
% \fi
%
% %%%%%%%%%%%%%%%%%%%%%%%%%%%%%%%%%%%%%%%%%%%%%%%%%%%%%%%%%%%%%%%%%%%%%%%%%%%%%%
% %%%%%%%%%%%%%%%%%%%%%%%%%%%%%%%%%%%%%%%%%%%%%%%%%%%%%%%%%%%%%%%%%%%%%%%%%%%%%%
% \section{Sample}
%\iffalse
%<*samplemain>
%\fi
%
% The following presents a sample document
% with two chapters, two parts, a title page,
% a compile flag as well as three forwarding files to set the flag.
% It consists of eight |.tex| files:
% \begin{center}
% \begin{tabular}{ll}
% |cdocsamp.tex|&main file\\
% |cdocsch1.tex|&include file for chapter 1\\
% |cdocsch2.tex|&include file for chapter 2\\
% |cdocspt3.tex|&include file for part 3\\
% |cdocspt4.tex|&include file for part 4\\
% |cdocsdrf.tex|&forwarding file for main file in draft mode\\
% |cdocsfi1.tex|&forwarding file for final version of chapter 1\\
% |cdocsfi2.tex|&forwarding file for final version of chapter 2\\
% \end{tabular}
% \end{center}
% Each of the eight files can be compiled directly by the \LaTeX{} compiler.
%
% %%%%%%%%%%%%%%%%%%%%%%%%%%%%%%%%%%%%%%
% \paragraph{Main File.}
%
% The main file is called |cdocsamp.tex|.
%
% Load the \textsf{childdoc} definitions and
% declare the filename for the main document:
%    \begin{macrocode}
\input{childdoc.def}
\childdocmain{}
%    \end{macrocode}

% Optional override for |\version| flag:
%    \begin{macrocode}
%%\ifchilddoc\else\providecommand{\version}{draft}\fi
%    \end{macrocode}

% Define the default values for the |\version| flag
% (|final| for the main file and |draft| for childs):
%    \begin{macrocode}
\ifchilddoc
\providecommand{\version}{draft}
\else
\providecommand{\version}{final}
\fi
%    \end{macrocode}

% Load the standard document class:
%    \begin{macrocode}
\documentclass[12pt]{article}
%    \end{macrocode}

% Start the document body:
%    \begin{macrocode}
\begin{document}
%    \end{macrocode}

% Declare a title page.
% Print title, part of document being processed and version flag:
%    \begin{macrocode}
\addtocounter{page}{-1}
\begin{center}
{\LARGE\bfseries{}childdoc example\par}
\vspace{1cm}
\ifchilddoc
\ifchilddocmanual part\else chapter\fi:
`\childdocname' of `\childdocjob'\par
\else
main document: `\childdocjob'\par
\fi
version: \version\par
\end{center}
\newpage
%    \end{macrocode}

% Manually include selected file,
% otherwise process as usual:
%    \begin{macrocode}
\ifchilddocmanual
\section*{part `\childdocname'}
\input{\childdocname}
\else
%    \end{macrocode}

% Include the two chapters:
%    \begin{macrocode}
\include{cdocsch1}
\include{cdocsch2}
%    \end{macrocode}

% Include the two parts unless only chapters should be displayed:
%    \begin{macrocode}
\ifchilddoc\else
\section{part three}
\input{cdocspt3}
\section{part four}
\input{cdocspt4}
\fi
%    \end{macrocode}

% Process as usual until here:
%    \begin{macrocode}
\fi
%    \end{macrocode}

% End of document body:
%    \begin{macrocode}
\end{document}
%    \end{macrocode}
%\iffalse
%</samplemain>
%\fi
%
% %%%%%%%%%%%%%%%%%%%%%%%%%%%%%%%%%%%%%%
% \paragraph{Chapter Include Files.}
%
% The include files are called |cdocsch1.tex| and |cdocsch2.tex|.
%
%\iffalse
%<*samplechap1|samplechap2>
%\fi

% Optional override for |\version| flag:
%    \begin{macrocode}
%%\providecommand{\version}{final}
%    \end{macrocode}

% Include the main document:
%    \begin{macrocode}
\input{childdoc.def}
\childdocof{cdocsamp}
%    \end{macrocode}

%\iffalse
%</samplechap1|samplechap2>
%\fi
%
%\iffalse
%<*samplechap1>
%\fi
% Some text for chapter 1:
%    \begin{macrocode}
\section{one}
some text in chapter one
%    \end{macrocode}

%\iffalse
%</samplechap1>
%\fi
% Some text for chapter 2:
%\iffalse
%<*samplechap2>
%\fi
%    \begin{macrocode}
\section{two}
more text in chapter two
%    \end{macrocode}

%\iffalse
%</samplechap2>
%\fi
%
% %%%%%%%%%%%%%%%%%%%%%%%%%%%%%%%%%%%%%%
% \paragraph{Part Include Files.}
%
% The include files are called |cdocspt3.tex| and |cdocspt4.tex|.
%
%\iffalse
%<*samplepart3|samplepart4>
%\fi

% Optional override for |\version| flag:
%    \begin{macrocode}
%%\providecommand{\version}{final}
%    \end{macrocode}

% Include the main document:
%    \begin{macrocode}
\input{childdoc.def}
\childdocby{cdocsamp}
%    \end{macrocode}

%\iffalse
%</samplepart3|samplepart4>
%\fi
%
%\iffalse
%<*samplepart3>
%\fi
% Some text for part 3:
%    \begin{macrocode}
some text in part three
%    \end{macrocode}

%\iffalse
%</samplepart3>
%\fi
% Some text for part 4:
%\iffalse
%<*samplepart4>
%\fi
%    \begin{macrocode}
more text in part four
%    \end{macrocode}

%\iffalse
%</samplepart4>
%\fi
%
% %%%%%%%%%%%%%%%%%%%%%%%%%%%%%%%%%%%%%%
% \paragraph{Forwarding for a Complete Draft.}
%
% The following forwarding file |cdocsdrf.tex|
% compiles the main document in draft mode:
%\iffalse
%<*sampledraft>
%\fi
%    \begin{macrocode}
\def\version{draft}
\input{childdoc.def}
\childdocforward{cdocsamp}
%    \end{macrocode}

%\iffalse
%</sampledraft>
%\fi
%
% %%%%%%%%%%%%%%%%%%%%%%%%%%%%%%%%%%%%%%
% \paragraph{Forwarding for Final Version of the Chapters.}
%
% The following forwarding files |cdocsfn1.tex| and |cdocsfn2.tex|
% (with identical content)
% compile the final versions of the child documents
% |cdocsch1.tex| and |cdocsch2.tex|, respectively:
%\iffalse
%<*samplefinal>
%\fi
%    \begin{macrocode}
\def\version{final}
\input{childdoc.def}
\childdocforwardprefix[cdocsamp]{cdocsfn}{cdocsch}
%    \end{macrocode}

%\iffalse
%</samplefinal>
%\fi
%
% %%%%%%%%%%%%%%%%%%%%%%%%%%%%%%%%%%%%%%
% \paragraph{Command Line Processing.}
%
% The following three command lines generate the output files
% |cdocscld|, |cdocscl1| and |cdocscl2|
% which should be identical to
% |cdocsdrf|, |cdocsch1| and |cdocsfn2|, respectively:
% \begin{center}
% \begin{tabular}{l}
% |latex -jobname cdocscld \|\\
% |  "\def\version{draft}\input{childdoc.def}\childdocforward{cdocsamp}"|\\
% |latex -jobname cdocscl1 \|\\
% |  "\input{childdoc.def}\childdocforward[cdocsamp]{cdocsch1}"|\\
% |latex -jobname cdocscl2 \|\\
% |  "\def\version{final}\input{childdoc.def}\childdocforward{cdocsch2}"|
% \end{tabular}
% \end{center}
% Note that the trailing backslash on each first line
% merely continues the input to the second line
% (for convenient cut ant paste).
% Furthermore, the command |latex| can be replaced by any
% of its alternative versions such as |pdflatex|.
%
% %%%%%%%%%%%%%%%%%%%%%%%%%%%%%%%%%%%%%%%%%%%%%%%%%%%%%%%%%%%%%%%%%%%%%%%%%%%%%%
% %%%%%%%%%%%%%%%%%%%%%%%%%%%%%%%%%%%%%%%%%%%%%%%%%%%%%%%%%%%%%%%%%%%%%%%%%%%%%%
% \section{Implementation}
%\iffalse
%<*package>
%\fi
%
% This section describes the definitions file |childdoc.def|.

% The definitions cannot be loaded using |\usepackage| or |\RequirePackage|
% which has a mechanism to prevent loading a style file more than once.
% When loading the definitions by means of |\input|
% multiple instances have to be prevented manually:
%\iffalse
%This code needs to be before the `\ProvidesFile' directive
%which is defined at the beginning of this file.
%Therefore it is also placed there and commented out here.
%</package>
%<*discard>
%\fi
%    \begin{macrocode}
\ifdefined\childdocmain\endinput\fi
%    \end{macrocode}
%\iffalse
%</discard>
%<*package>
%\fi
%
% \macro{\ifchilddoc}
% \macro{\ifchilddocmanual}
% The conditional |\ifchilddoc| tells whether a
% child (true) or main (false) document is being compiled.
% The conditional |\ifchilddocmanual| tells whether
% the |\includeonly| mechanism is used (false) or
% the selection of child files must be performed manually (true).
% The definitions initialise to false:
%    \begin{macrocode}
\newif\ifchilddoc
\newif\ifchilddocmanual
%    \end{macrocode}

% \macro{\childdocname}
% \macro{\childdocjob}
% The macro |\childdocname| stores the name of the main document
% to be compiled. The macro |\childdocjob| stores the name of
% the document on which the \LaTeX{} compiler was originally invoked.
% The content of |\jobname| cannot be compared
% to filenames specified in the source due to different catcodes.
% The following code rescans |\jobname|, stores the result
% in |\childdocname| and saves a copy in |\childdocjob|:
%    \begin{macrocode}
\edef\childdocname{\scantokens\expandafter{\jobname\noexpand}}
\let\childdocjob\childdocname
%    \end{macrocode}

% \macro{\childdocdisable}
% The macro |\childdocdisable| prevents the main file
% from being processed more than once.
% At this stage, the main document command |\childdocmain|
% is assumed to be called once again where it should do nothing.
% Any subsequent call to it should prevent
% a secondary processing of the main document
% It overwrites the forwarding commands
% |\childdocof| and |\childdocforward|
% with empty macros to prevent further inclusions of the main document:
%    \begin{macrocode}
\newcommand{\childdocdisable}
{
  \renewcommand{\childdocmain}[1]{\renewcommand{\childdocmain}[1]{\endinput}}
  \renewcommand{\childdocof}[1]{}
  \renewcommand{\childdocby}[2][]{}
  \renewcommand{\childdocforward}[2][]{}
  \renewcommand{\childdocdisable}{}
}
%    \end{macrocode}

% \macro{\childdocmain}
% The macro |\childdocmain| is to be called at the top of the main file
% with nothing or the main filename (without extension) as argument.
% First, it breaks loops.
% If the argument is not empty and does not match |\childdocname|
% (which is set by the first inclusion of |childdoc.def|),
% |\ifchilddoc| is set to true, |\includeonly| is applied to the child file
% and |\jobname| is set to the main file
% (for proper handling of |.aux| files):
%    \begin{macrocode}
\newcommand{\childdocmain}[1]
{
  \childdocdisable\childdocmain{}
  \if?#1?\else
    \begingroup
      \def\childdoctmp{#1}
      \ifx\childdoctmp\childdocname
        \def\childdoctmp{}
      \else
        \def\childdoctmp
        {
          \childdoctrue
          \includeonly{\childdocname}
          \def\childdocjob{#1}
          \def\jobname{#1}
        }
      \fi
      \expandafter
    \endgroup
    \childdoctmp
  \fi
}
%    \end{macrocode}

% \macro{\childdocof}
% The command |\childdocof| redirects
% compilation to the main file |#1|.
%    \begin{macrocode}
\newcommand{\childdocof}[1]
{
  \childdocdisable
  \childdoctrue
  \includeonly{\childdocname}
  \def\jobname{#1}
  \def\childdocjob{#1}
  \input{#1}
}
%    \end{macrocode}

% \macro{\childdocby}
% The command |\childdocby| ....
%    \begin{macrocode}
\newcommand{\childdocby}[2][]
{
  \childdocdisable
  \childdoctrue
  \childdocmanualtrue
  \if?#1?\else
    \def\jobname{#2}
  \fi
  \def\childdocjob{#2}
  \input{#2}
  \endinput
}
%    \end{macrocode}

% \macro{\childdocforward}
% The command |\childdocforward| redirects
% compilation to the main file or
% (if the optional argument is given) a child file.
% Parameters are set as if the main file
% or a child file starting with |\childdocof| was compiled.
% Then compilation is handed over to the main file:
%    \begin{macrocode}
\newcommand{\childdocforward}[2][]
{
  \begingroup
    \if?#1?
      \def\childdoctmp
      {
        \def\childdocname{#2}
        \def\childdocjob{#2}
        \def\jobname{#2}
        \input{#2}
        \endinput
      }
    \else
      \def\childdoctmp
      {
        \childdocdisable
        \def\childdocname{#2}
        \childdoctrue
        \includeonly{#2}
        \def\childdocjob{#1}
        \def\jobname{#1}
        \input{#1}
        \endinput
      }
    \fi
    \expandafter
  \endgroup
  \childdoctmp
}
%    \end{macrocode}

% \macro{\childdocforwardprefix}
% The command |\childdocforwardprefix| redirects
% compilation to the main or a child file by means of a pattern.
% The prefix |#1| in the current filename is replaced by |#2|
% and the suffix of the current filename is kept
% (it is assumed that the filename does not contain the substring `|~~~|'
% which is used as a delimiter).
% Compilation is handed over to the new file by |\childdocforward|:
%    \begin{macrocode}
\newcommand{\childdocforwardprefix}[3][]
{
  \begingroup
    \def\childdocextract #2##1~~~{\def\childdoctmp{\childdocforward[#1]{#3##1}}}
    \expandafter\childdocextract\childdocname~~~
    \expandafter
  \endgroup
  \childdoctmp
}
%    \end{macrocode}

% \macro{\childdoc}
% The deprecated macro |\childdoc| is a legacy version of |\childdocmain|:
%    \begin{macrocode}
\newcommand{\childdoc}{\childdocmain}
%    \end{macrocode}

% \macro{\childdocredirect}
% The deprecated macro |\childdocredirect| is a legacy version
% of |\childdocforward| and |\childdocforwardprefix|:
%    \begin{macrocode}
\newcommand{\childdocredirect}[2][]
{
  \begingroup
    \if?#1?
      \def\childdoctmp{\childdocforward{#2}}
    \else
      \def\childdoctmp{\childdocforwardprefix{#1}{#2}}
    \fi
    \expandafter
  \endgroup
  \childdoctmp
}
%    \end{macrocode}

%\iffalse
%</package>
%\fi
%
\endinput
\childdocforward{cdocsch2}"|
% \end{tabular}
% \end{center}
% Note that the trailing backslash on each first line
% merely continues the input to the second line
% (for convenient cut ant paste).
% Furthermore, the command |latex| can be replaced by any
% of its alternative versions such as |pdflatex|.
%
% %%%%%%%%%%%%%%%%%%%%%%%%%%%%%%%%%%%%%%%%%%%%%%%%%%%%%%%%%%%%%%%%%%%%%%%%%%%%%%
% %%%%%%%%%%%%%%%%%%%%%%%%%%%%%%%%%%%%%%%%%%%%%%%%%%%%%%%%%%%%%%%%%%%%%%%%%%%%%%
% \section{Implementation}
%\iffalse
%<*package>
%\fi
%
% This section describes the definitions file |childdoc.def|.

% The definitions cannot be loaded using |\usepackage| or |\RequirePackage|
% which has a mechanism to prevent loading a style file more than once.
% When loading the definitions by means of |\input|
% multiple instances have to be prevented manually:
%\iffalse
%This code needs to be before the `\ProvidesFile' directive
%which is defined at the beginning of this file.
%Therefore it is also placed there and commented out here.
%</package>
%<*discard>
%\fi
%    \begin{macrocode}
\ifdefined\childdocmain\endinput\fi
%    \end{macrocode}
%\iffalse
%</discard>
%<*package>
%\fi
%
% \macro{\ifchilddoc}
% \macro{\ifchilddocmanual}
% The conditional |\ifchilddoc| tells whether a
% child (true) or main (false) document is being compiled.
% The conditional |\ifchilddocmanual| tells whether
% the |\includeonly| mechanism is used (false) or
% the selection of child files must be performed manually (true).
% The definitions initialise to false:
%    \begin{macrocode}
\newif\ifchilddoc
\newif\ifchilddocmanual
%    \end{macrocode}

% \macro{\childdocname}
% \macro{\childdocjob}
% The macro |\childdocname| stores the name of the main document
% to be compiled. The macro |\childdocjob| stores the name of
% the document on which the \LaTeX{} compiler was originally invoked.
% The content of |\jobname| cannot be compared
% to filenames specified in the source due to different catcodes.
% The following code rescans |\jobname|, stores the result
% in |\childdocname| and saves a copy in |\childdocjob|:
%    \begin{macrocode}
\edef\childdocname{\scantokens\expandafter{\jobname\noexpand}}
\let\childdocjob\childdocname
%    \end{macrocode}

% \macro{\childdocdisable}
% The macro |\childdocdisable| prevents the main file
% from being processed more than once.
% At this stage, the main document command |\childdocmain|
% is assumed to be called once again where it should do nothing.
% Any subsequent call to it should prevent
% a secondary processing of the main document
% It overwrites the forwarding commands
% |\childdocof| and |\childdocforward|
% with empty macros to prevent further inclusions of the main document:
%    \begin{macrocode}
\newcommand{\childdocdisable}
{
  \renewcommand{\childdocmain}[1]{\renewcommand{\childdocmain}[1]{\endinput}}
  \renewcommand{\childdocof}[1]{}
  \renewcommand{\childdocby}[2][]{}
  \renewcommand{\childdocforward}[2][]{}
  \renewcommand{\childdocdisable}{}
}
%    \end{macrocode}

% \macro{\childdocmain}
% The macro |\childdocmain| is to be called at the top of the main file
% with nothing or the main filename (without extension) as argument.
% First, it breaks loops.
% If the argument is not empty and does not match |\childdocname|
% (which is set by the first inclusion of |childdoc.def|),
% |\ifchilddoc| is set to true, |\includeonly| is applied to the child file
% and |\jobname| is set to the main file
% (for proper handling of |.aux| files):
%    \begin{macrocode}
\newcommand{\childdocmain}[1]
{
  \childdocdisable\childdocmain{}
  \if?#1?\else
    \begingroup
      \def\childdoctmp{#1}
      \ifx\childdoctmp\childdocname
        \def\childdoctmp{}
      \else
        \def\childdoctmp
        {
          \childdoctrue
          \includeonly{\childdocname}
          \def\childdocjob{#1}
          \def\jobname{#1}
        }
      \fi
      \expandafter
    \endgroup
    \childdoctmp
  \fi
}
%    \end{macrocode}

% \macro{\childdocof}
% The command |\childdocof| redirects
% compilation to the main file |#1|.
%    \begin{macrocode}
\newcommand{\childdocof}[1]
{
  \childdocdisable
  \childdoctrue
  \includeonly{\childdocname}
  \def\jobname{#1}
  \def\childdocjob{#1}
  \input{#1}
}
%    \end{macrocode}

% \macro{\childdocby}
% The command |\childdocby| ....
%    \begin{macrocode}
\newcommand{\childdocby}[2][]
{
  \childdocdisable
  \childdoctrue
  \childdocmanualtrue
  \if?#1?\else
    \def\jobname{#2}
  \fi
  \def\childdocjob{#2}
  \input{#2}
  \endinput
}
%    \end{macrocode}

% \macro{\childdocforward}
% The command |\childdocforward| redirects
% compilation to the main file or
% (if the optional argument is given) a child file.
% Parameters are set as if the main file
% or a child file starting with |\childdocof| was compiled.
% Then compilation is handed over to the main file:
%    \begin{macrocode}
\newcommand{\childdocforward}[2][]
{
  \begingroup
    \if?#1?
      \def\childdoctmp
      {
        \def\childdocname{#2}
        \def\childdocjob{#2}
        \def\jobname{#2}
        \input{#2}
        \endinput
      }
    \else
      \def\childdoctmp
      {
        \childdocdisable
        \def\childdocname{#2}
        \childdoctrue
        \includeonly{#2}
        \def\childdocjob{#1}
        \def\jobname{#1}
        \input{#1}
        \endinput
      }
    \fi
    \expandafter
  \endgroup
  \childdoctmp
}
%    \end{macrocode}

% \macro{\childdocforwardprefix}
% The command |\childdocforwardprefix| redirects
% compilation to the main or a child file by means of a pattern.
% The prefix |#1| in the current filename is replaced by |#2|
% and the suffix of the current filename is kept
% (it is assumed that the filename does not contain the substring `|~~~|'
% which is used as a delimiter).
% Compilation is handed over to the new file by |\childdocforward|:
%    \begin{macrocode}
\newcommand{\childdocforwardprefix}[3][]
{
  \begingroup
    \def\childdocextract #2##1~~~{\def\childdoctmp{\childdocforward[#1]{#3##1}}}
    \expandafter\childdocextract\childdocname~~~
    \expandafter
  \endgroup
  \childdoctmp
}
%    \end{macrocode}

% \macro{\childdoc}
% The deprecated macro |\childdoc| is a legacy version of |\childdocmain|:
%    \begin{macrocode}
\newcommand{\childdoc}{\childdocmain}
%    \end{macrocode}

% \macro{\childdocredirect}
% The deprecated macro |\childdocredirect| is a legacy version
% of |\childdocforward| and |\childdocforwardprefix|:
%    \begin{macrocode}
\newcommand{\childdocredirect}[2][]
{
  \begingroup
    \if?#1?
      \def\childdoctmp{\childdocforward{#2}}
    \else
      \def\childdoctmp{\childdocforwardprefix{#1}{#2}}
    \fi
    \expandafter
  \endgroup
  \childdoctmp
}
%    \end{macrocode}

%\iffalse
%</package>
%\fi
%
\endinput

\childdocmain{}
%    \end{macrocode}

% Optional override for |\version| flag:
%    \begin{macrocode}
%%\ifchilddoc\else\providecommand{\version}{draft}\fi
%    \end{macrocode}

% Define the default values for the |\version| flag
% (|final| for the main file and |draft| for childs):
%    \begin{macrocode}
\ifchilddoc
\providecommand{\version}{draft}
\else
\providecommand{\version}{final}
\fi
%    \end{macrocode}

% Load the standard document class:
%    \begin{macrocode}
\documentclass[12pt]{article}
%    \end{macrocode}

% Start the document body:
%    \begin{macrocode}
\begin{document}
%    \end{macrocode}

% Declare a title page.
% Print title, part of document being processed and version flag:
%    \begin{macrocode}
\addtocounter{page}{-1}
\begin{center}
{\LARGE\bfseries{}childdoc example\par}
\vspace{1cm}
\ifchilddoc
\ifchilddocmanual part\else chapter\fi:
`\childdocname' of `\childdocjob'\par
\else
main document: `\childdocjob'\par
\fi
version: \version\par
\end{center}
\newpage
%    \end{macrocode}

% Manually include selected file,
% otherwise process as usual:
%    \begin{macrocode}
\ifchilddocmanual
\section*{part `\childdocname'}
\input{\childdocname}
\else
%    \end{macrocode}

% Include the two chapters:
%    \begin{macrocode}
\include{cdocsch1}
\include{cdocsch2}
%    \end{macrocode}

% Include the two parts unless only chapters should be displayed:
%    \begin{macrocode}
\ifchilddoc\else
\section{part three}
\input{cdocspt3}
\section{part four}
\input{cdocspt4}
\fi
%    \end{macrocode}

% Process as usual until here:
%    \begin{macrocode}
\fi
%    \end{macrocode}

% End of document body:
%    \begin{macrocode}
\end{document}
%    \end{macrocode}
%\iffalse
%</samplemain>
%\fi
%
% %%%%%%%%%%%%%%%%%%%%%%%%%%%%%%%%%%%%%%
% \paragraph{Chapter Include Files.}
%
% The include files are called |cdocsch1.tex| and |cdocsch2.tex|.
%
%\iffalse
%<*samplechap1|samplechap2>
%\fi

% Optional override for |\version| flag:
%    \begin{macrocode}
%%\providecommand{\version}{final}
%    \end{macrocode}

% Include the main document:
%    \begin{macrocode}
% \iffalse
%
% childdoc.dtx Copyright (C) 2017-2018 Niklas Beisert
%
% This work may be distributed and/or modified under the
% conditions of the LaTeX Project Public License, either version 1.3
% of this license or (at your option) any later version.
% The latest version of this license is in
%   http://www.latex-project.org/lppl.txt
% and version 1.3 or later is part of all distributions of LaTeX
% version 2005/12/01 or later.
%
% This work has the LPPL maintenance status `maintained'.
%
% The Current Maintainer of this work is Niklas Beisert.
%
% This work consists of the files childdoc.dtx and childdoc.ins
% and the derived files childdoc.def and cdocsamp.tex with
% cdocsch1.tex, cdocsch2.tex, cdocsdrf.tex, cdocsfn1.tex, cdocsfn2.tex.
%
%<package>\ifdefined\childdocmain\endinput\fi
%<package>\ProvidesFile{childdoc.def}[2018/12/30 v2.0 child document driver]
%<samplemain>\ProvidesFile{cdocsamp.tex}[2018/12/30 v2.0 sample for childdoc]
%<*driver>
%\ProvidesFile{childdoc.drv}[2018/12/30 v2.0 childdoc reference manual file]
\PassOptionsToClass{10pt,a4paper}{article}
\documentclass{ltxdoc}

\usepackage[margin=35mm]{geometry}
\usepackage{hyperref}
\usepackage{hyperxmp}
\usepackage[usenames]{color}

\hypersetup{colorlinks=true}
\hypersetup{pdfstartview=FitH}
\hypersetup{pdfpagemode=UseNone}
\hypersetup{pdfsource={}}
\hypersetup{pdflang={en-UK}}
\hypersetup{pdfcopyright={Copyright 2017-2018 Niklas Beisert.
  This work may be distributed and/or modified under the
  conditions of the LaTeX Project Public License, either version 1.3
  of this license or (at your option) any later version.}}
\hypersetup{pdflicenseurl={http://www.latex-project.org/lppl.txt}}
\hypersetup{pdfcontactaddress={ETH Zurich, ITP, HIT K,
  Wolfgang-Pauli-Strasse 27}}
\hypersetup{pdfcontactpostcode={8093}}
\hypersetup{pdfcontactcity={Zurich}}
\hypersetup{pdfcontactcountry={Switzerland}}
\hypersetup{pdfcontactemail={nbeisert@itp.phys.ethz.ch}}
\hypersetup{pdfcontacturl={http://people.phys.ethz.ch/\xmptilde nbeisert/}}

\newcommand{\secref}[1]{\hyperref[#1]{section \ref*{#1}}}

\parskip1ex
\parindent0pt
\let\olditemize\itemize
\def\itemize{\olditemize\parskip0pt}

\begin{document}

\title{The \textsf{childdoc} Package}
\hypersetup{pdftitle={The childdoc Package}}
\author{Niklas Beisert\\[2ex]
  Institut f\"ur Theoretische Physik\\
  Eidgen\"ossische Technische Hochschule Z\"urich\\
  Wolfgang-Pauli-Strasse 27, 8093 Z\"urich, Switzerland\\[1ex]
  \href{mailto:nbeisert@itp.phys.ethz.ch}
  {\texttt{nbeisert@itp.phys.ethz.ch}}}
\hypersetup{pdfauthor={Niklas Beisert}}
\hypersetup{pdfsubject={Manual for the LaTeX2e Package childdoc}}
\date{30 December 2018, \textsf{v2.0}}
\maketitle

\begin{abstract}\noindent
\textsf{childdoc} is a \LaTeXe{} package
that enables the direct compilation
of document sections included by |\include|
to individual files.
\end{abstract}

\begingroup
\parskip0ex
\tableofcontents
\endgroup

%%%%%%%%%%%%%%%%%%%%%%%%%%%%%%%%%%%%%%%%%%%%%%%%%%%%%%%%%%%%%%%%%%%%%%%%%%%%%%%%
%%%%%%%%%%%%%%%%%%%%%%%%%%%%%%%%%%%%%%%%%%%%%%%%%%%%%%%%%%%%%%%%%%%%%%%%%%%%%%%%
\section{Introduction}

\LaTeX{} provides a mechanism to structure a large document (such as a book)
into a main file and several child files (containing the chapters)
using the |\include| command.
This mechanism is beneficial for documents
which span hundreds of pages in order to
make the source file(s) more manageable.
Moreover, compilation can be restricted to
selected child files by means of the |\includeonly| command.
The latter feature can be used to reduce the compilation time while editing
(this was significantly more useful in the earlier days of \LaTeX{})
or to generate a smaller document which is easier to navigate.
Another application of |\includeonly| is to generate
documents consisting of selected parts of the complete document.

However, there are a few drawbacks of the plain |\include| mechanism:
\begin{itemize}
\item
The child files cannot be compiled on their own,
they can only be compiled via the main file.
A naive editing environment
(such as a text editor with an option
to have the current file processed by \LaTeX)
may require one to switch to the main file before compiling;
attempting to compile the child file produces errors.
\item
The main file must be modified (each time)
to adjust the |\includeonly| command
to the present needs. This easily leaves the main file in a messy state.
\item
The generated document will always carry the filename
of the main document. This is inconvenient if
several child files are to be compiled and
to be kept for distribution.
\end{itemize}

The present package provides a simple interface
to make child files individually compilable by \LaTeX{}.
Compiling a child file then has the same effect as compiling
the main file with an |\includeonly| command
to select the appropriate child.
Moreover the generated document will carry the name of the child
rather than the main file.
This resolves all three above issues.

This feature is meant to make the editing of books,
thesis documents and lecture notes somewhat more convenient.
However, the package can also be used efficiently for
composing a series of documents (such as exercise sheets)
which are typically distributed individually.
It then assists the author in generating the individual documents
(potentially in different versions)
as well as a document containing the collected series.
Another application is in developing style files
or other kinds of included material
where compilation of the style file could redirect
to a sample or test file.

%%%%%%%%%%%%%%%%%%%%%%%%%%%%%%%%%%%%%%%%%%%%%%%%%%%%%%%%%%%%%%%%%%%%%%%%%%%%%%%%
%%%%%%%%%%%%%%%%%%%%%%%%%%%%%%%%%%%%%%%%%%%%%%%%%%%%%%%%%%%%%%%%%%%%%%%%%%%%%%%%
\section{Usage}

First of all, the package \textsf{childdoc} is \emph{not} a standard
\LaTeXe{} |.sty| style file! Therefore it needs to be invoked in
a non-standard way.

%%%%%%%%%%%%%%%%%%%%%%%%%%%%%%%%%%%%%%%%%%%%%%%%%%%%%%%%%%%%%%%%%%%%%%%%%%%%%%%%
\subsection{Included Files}
\label{sec:include}

%%%%%%%%%%%%%%%%%%%%%%%%%%%%%%%%%%%%%%%%
\DescribeMacro{\childdocmain}
To use the package, add the commands
\begin{center}
\begin{tabular}{l}
|% \iffalse
%
% childdoc.dtx Copyright (C) 2017-2018 Niklas Beisert
%
% This work may be distributed and/or modified under the
% conditions of the LaTeX Project Public License, either version 1.3
% of this license or (at your option) any later version.
% The latest version of this license is in
%   http://www.latex-project.org/lppl.txt
% and version 1.3 or later is part of all distributions of LaTeX
% version 2005/12/01 or later.
%
% This work has the LPPL maintenance status `maintained'.
%
% The Current Maintainer of this work is Niklas Beisert.
%
% This work consists of the files childdoc.dtx and childdoc.ins
% and the derived files childdoc.def and cdocsamp.tex with
% cdocsch1.tex, cdocsch2.tex, cdocsdrf.tex, cdocsfn1.tex, cdocsfn2.tex.
%
%<package>\ifdefined\childdocmain\endinput\fi
%<package>\ProvidesFile{childdoc.def}[2018/12/30 v2.0 child document driver]
%<samplemain>\ProvidesFile{cdocsamp.tex}[2018/12/30 v2.0 sample for childdoc]
%<*driver>
%\ProvidesFile{childdoc.drv}[2018/12/30 v2.0 childdoc reference manual file]
\PassOptionsToClass{10pt,a4paper}{article}
\documentclass{ltxdoc}

\usepackage[margin=35mm]{geometry}
\usepackage{hyperref}
\usepackage{hyperxmp}
\usepackage[usenames]{color}

\hypersetup{colorlinks=true}
\hypersetup{pdfstartview=FitH}
\hypersetup{pdfpagemode=UseNone}
\hypersetup{pdfsource={}}
\hypersetup{pdflang={en-UK}}
\hypersetup{pdfcopyright={Copyright 2017-2018 Niklas Beisert.
  This work may be distributed and/or modified under the
  conditions of the LaTeX Project Public License, either version 1.3
  of this license or (at your option) any later version.}}
\hypersetup{pdflicenseurl={http://www.latex-project.org/lppl.txt}}
\hypersetup{pdfcontactaddress={ETH Zurich, ITP, HIT K,
  Wolfgang-Pauli-Strasse 27}}
\hypersetup{pdfcontactpostcode={8093}}
\hypersetup{pdfcontactcity={Zurich}}
\hypersetup{pdfcontactcountry={Switzerland}}
\hypersetup{pdfcontactemail={nbeisert@itp.phys.ethz.ch}}
\hypersetup{pdfcontacturl={http://people.phys.ethz.ch/\xmptilde nbeisert/}}

\newcommand{\secref}[1]{\hyperref[#1]{section \ref*{#1}}}

\parskip1ex
\parindent0pt
\let\olditemize\itemize
\def\itemize{\olditemize\parskip0pt}

\begin{document}

\title{The \textsf{childdoc} Package}
\hypersetup{pdftitle={The childdoc Package}}
\author{Niklas Beisert\\[2ex]
  Institut f\"ur Theoretische Physik\\
  Eidgen\"ossische Technische Hochschule Z\"urich\\
  Wolfgang-Pauli-Strasse 27, 8093 Z\"urich, Switzerland\\[1ex]
  \href{mailto:nbeisert@itp.phys.ethz.ch}
  {\texttt{nbeisert@itp.phys.ethz.ch}}}
\hypersetup{pdfauthor={Niklas Beisert}}
\hypersetup{pdfsubject={Manual for the LaTeX2e Package childdoc}}
\date{30 December 2018, \textsf{v2.0}}
\maketitle

\begin{abstract}\noindent
\textsf{childdoc} is a \LaTeXe{} package
that enables the direct compilation
of document sections included by |\include|
to individual files.
\end{abstract}

\begingroup
\parskip0ex
\tableofcontents
\endgroup

%%%%%%%%%%%%%%%%%%%%%%%%%%%%%%%%%%%%%%%%%%%%%%%%%%%%%%%%%%%%%%%%%%%%%%%%%%%%%%%%
%%%%%%%%%%%%%%%%%%%%%%%%%%%%%%%%%%%%%%%%%%%%%%%%%%%%%%%%%%%%%%%%%%%%%%%%%%%%%%%%
\section{Introduction}

\LaTeX{} provides a mechanism to structure a large document (such as a book)
into a main file and several child files (containing the chapters)
using the |\include| command.
This mechanism is beneficial for documents
which span hundreds of pages in order to
make the source file(s) more manageable.
Moreover, compilation can be restricted to
selected child files by means of the |\includeonly| command.
The latter feature can be used to reduce the compilation time while editing
(this was significantly more useful in the earlier days of \LaTeX{})
or to generate a smaller document which is easier to navigate.
Another application of |\includeonly| is to generate
documents consisting of selected parts of the complete document.

However, there are a few drawbacks of the plain |\include| mechanism:
\begin{itemize}
\item
The child files cannot be compiled on their own,
they can only be compiled via the main file.
A naive editing environment
(such as a text editor with an option
to have the current file processed by \LaTeX)
may require one to switch to the main file before compiling;
attempting to compile the child file produces errors.
\item
The main file must be modified (each time)
to adjust the |\includeonly| command
to the present needs. This easily leaves the main file in a messy state.
\item
The generated document will always carry the filename
of the main document. This is inconvenient if
several child files are to be compiled and
to be kept for distribution.
\end{itemize}

The present package provides a simple interface
to make child files individually compilable by \LaTeX{}.
Compiling a child file then has the same effect as compiling
the main file with an |\includeonly| command
to select the appropriate child.
Moreover the generated document will carry the name of the child
rather than the main file.
This resolves all three above issues.

This feature is meant to make the editing of books,
thesis documents and lecture notes somewhat more convenient.
However, the package can also be used efficiently for
composing a series of documents (such as exercise sheets)
which are typically distributed individually.
It then assists the author in generating the individual documents
(potentially in different versions)
as well as a document containing the collected series.
Another application is in developing style files
or other kinds of included material
where compilation of the style file could redirect
to a sample or test file.

%%%%%%%%%%%%%%%%%%%%%%%%%%%%%%%%%%%%%%%%%%%%%%%%%%%%%%%%%%%%%%%%%%%%%%%%%%%%%%%%
%%%%%%%%%%%%%%%%%%%%%%%%%%%%%%%%%%%%%%%%%%%%%%%%%%%%%%%%%%%%%%%%%%%%%%%%%%%%%%%%
\section{Usage}

First of all, the package \textsf{childdoc} is \emph{not} a standard
\LaTeXe{} |.sty| style file! Therefore it needs to be invoked in
a non-standard way.

%%%%%%%%%%%%%%%%%%%%%%%%%%%%%%%%%%%%%%%%%%%%%%%%%%%%%%%%%%%%%%%%%%%%%%%%%%%%%%%%
\subsection{Included Files}
\label{sec:include}

%%%%%%%%%%%%%%%%%%%%%%%%%%%%%%%%%%%%%%%%
\DescribeMacro{\childdocmain}
To use the package, add the commands
\begin{center}
\begin{tabular}{l}
|\input{childdoc.def}|\\
|\childdocmain{}|\\
\end{tabular}
\end{center}
at the very top of the main \LaTeX{} file,
in particular \emph{before} the |\documentclass| statement!
The argument of |\childdocmain| should be left empty
(but it must be present).

%%%%%%%%%%%%%%%%%%%%%%%%%%%%%%%%%%%%%%%%
\DescribeMacro{\childdocof}
Furthermore, add the commands
\begin{center}
\begin{tabular}{l}
|\input{childdoc.def}|\\
|\childdocof{|\textit{main}|}|\\
\end{tabular}
\end{center}
at the top of every child file \textit{child}
which is included by |\include{|\textit{child}|}|
from within the main file
(or at least for those files to be compiled individually).
The argument \textit{main} must be the filename of the main file.

There are a couple of
considerations in setting up the main and child documents:

%%%%%%%%%%%%%%%%%%%%%%%%%%%%%%%%%%%%%%%%
\paragraph{Restrictions.}

Please note the following restrictions:
\begin{itemize}
\item
|\childdocmain| must be called with one argument \textit{main}
to ensure compatibility with earlier version of the package.
It must either be empty (|\childdocmain{}|)
or precisely match the filename of the main file in which it is specified.
See \secref{sec:detection} for further information.
\item
The filename \textit{main} must be specified without the |.tex| extension.
\item
The filename \textit{main} is case sensitive
(even in case-insensitive file systems)
due to internal string comparison.
\item
The argument \textit{main} should be fully expanded, it cannot be a macro.
\item
Subdirectories and special characters should be avoided in filenames.
\item
The command |\childdocmain{|\textit{main}|}| must be followed by a whitespace.
It should not be followed immediately by another command
or by a comment mark `|%|'.
This is because the \TeX{} parser reads the token immediately following
the argument of |\childdocmain| and puts it
at the beginning of every child section;
however, a white\-space is ignored.
\end{itemize}

%%%%%%%%%%%%%%%%%%%%%%%%%%%%%%%%%%%%%%%%
\paragraph{Content of Main File.}

It is advisable to place all content in the child files included by |\include|.
Any output contained in the main file will appear in all child documents
unless suppressed manually;
it cannot be suppressed automatically by the |\includeonly| directive
and thus should normally be avoided.
A method to include some content in the main file
by means of conditional processing is described in \secref{sec:conditional}.

%%%%%%%%%%%%%%%%%%%%%%%%%%%%%%%%%%%%%%%%
\paragraph{Page Numbering.}

When only a part of the document is compiled,
the appropriate numbering of pages
(as well as other status parameters)
is determined from the |.aux| files.
The latter contain information from previous passes.
However this information needs to propagate through
all intermediate child documents.
Therefore the page numbering in child documents may well
be inconsistent until the complete document is compiled at least once.

A useful (if unconventional) way to always ensure a consistent
page numbering is to restart the numbering in each child document
and denote the pages by `\textit{child}|.|\textit{page}'
where \textit{child} represents the chapter/section number of the child file.
This can be achieved by the command
|\numberwithin{page}{|\textit{child}|}|
of the \textsf{amsmath} package
where \textit{child} can be |chapter| or |section|
depending on the chosen structuring.
Alternatively, one can modify the macro |\thepage| appropriately
and reset the counter |page| at the start of each child file.

%%%%%%%%%%%%%%%%%%%%%%%%%%%%%%%%%%%%%%%%%%%%%%%%%%%%%%%%%%%%%%%%%%%%%%%%%%%%%%%%
\subsection{Conditional Processing}
\label{sec:conditional}

The package provides a mechanism to compile different versions
of a document. To customise the versions further some conditional processing
can come in handy to distinguish which version is being compiled.
The package provides two macros to describe the compilation context:

%%%%%%%%%%%%%%%%%%%%%%%%%%%%%%%%%%%%%%%%
\DescribeMacro{\ifchilddoc}
The conditional |\ifchilddoc| distinguishes between the compilation of
child documents and the main document:
%
\begin{center}
|\ifchilddoc |\textit{child-code}| |[|\||else |\textit{main-code}]| \||fi|
\end{center}

%%%%%%%%%%%%%%%%%%%%%%%%%%%%%%%%%%%%%%%%
\DescribeMacro{\childdocname}
\DescribeMacro{\childdocjob}
The macro |\childdocname| contains the filename (without extension)
of the main or child file being processed.
Note that |\childdocjob| will always contain the name of the main file.

%%%%%%%%%%%%%%%%%%%%%%%%%%%%%%%%%%%%%%%%
\paragraph{Title Page.}

Conditional processing can be used to include a title or banner page
in the main document when proper precautions are taken.
Importantly, the code in the main file should ensure that the page counter
(as well as other status parameters which are stored in the |.aux| files)
takes the same value after the conditional processing.
Otherwise the page numbers may take divergent values
depending on which part is compiled.

For example, a title page could be declared by:
%
\begin{center}
\begin{tabular}{l}
|\ifchilddoc\||else|\\
|\addtocounter{page}{-1}|\\
\textit{code for title page}\\
|\newpage|\\
|\||fi|
\end{tabular}
\end{center}
%
A banner page for the child documents can be generated by:
%
\begin{center}
\begin{tabular}{l}
|\ifchilddoc|\\
|\addtocounter{page}{-1}|\\
\textit{code for banner page}\\
|\newpage|\\
|\||fi|
\end{tabular}
\end{center}
%
Here one could write a message such as:
\begin{center}
|This is the part \childdocname{} of \childdocjob{}.|
\end{center}

%%%%%%%%%%%%%%%%%%%%%%%%%%%%%%%%%%%%%%%%%%%%%%%%%%%%%%%%%%%%%%%%%%%%%%%%%%%%%%%%
\subsection{Flags}
\label{sec:flags}

The package makes it easy to generate different versions
of the main or child documents.
To this end compilation flags can be defined
and assigned different default values.
They will be particularly useful in conjunction
with the forwarding mechanism described in \secref{sec:forward}.

For example, it may be useful to have a flag |\version|
which can be set to |draft| or |final|.
The document source will contain some conditional code
depending on the value of |\version|.
Suppose further, the flag should default to |final| for the main file
and to |draft| for child files
which is a natural assignment for editing the document.
This is achieved by placing the following code
in the preamble of the main document
(below the |\childdocmain| directive):
%
\begin{center}
\begin{tabular}{l}
|\ifchilddoc|\\
|\providecommand{\version}{draft}|\\
|\||else|\\
|\providecommand{\version}{final}|\\
|\||fi|
\end{tabular}
\end{center}
%
The definition by |\providecommand| makes sure
that previous definitions are not overwritten.
Further statements |\providecommand{\version}{...}|
can thus be added before the above code to override it.

For the main file, one might add a line
(between |\childdocmain| and the above block)
%
\begin{center}
|%\ifchilddoc\||else\providecommand{\version}{draft}\||fi|
\end{center}
%
which can be uncommented to produce a draft version.
Likewise one can add a line to the very top of a child file
(above the |\childdocof{|\textit{main}|}| directive)
%
\begin{center}
|%\providecommand{\version}{final}|
\end{center}
%
which can be uncommented to produce the final version of this child document.

%%%%%%%%%%%%%%%%%%%%%%%%%%%%%%%%%%%%%%%%%%%%%%%%%%%%%%%%%%%%%%%%%%%%%%%%%%%%%%%%
\subsection{Forwarding}
\label{sec:forward}

Different versions of the main or child documents
using compilation flags as described in \secref{sec:flags}
can be (permanently) stored in different files
for convenient compilation, viewing and distribution.
To this end, the package defines a command
to pass on compilation to a different file:

%%%%%%%%%%%%%%%%%%%%%%%%%%%%%%%%%%%%%%%%
\DescribeMacro{\childdocforward}
The command |\childdocforward| redirects processing to
another source file:
%
\begin{center}
\begin{tabular}{l}
|\input{childdoc.def}|\\
|\childdocforward[|\textit{main}|]{|\textit{dest}|}|\\
\end{tabular}
\end{center}
%
The argument \textit{dest} is the destination file
(without extension).
It should be the main file or one of the child files.
Note that further \textsf{childdoc} directives
such as |\childdocof| and |\childdocforward|
in the indicated file will be processed in this form.
The optional argument \textit{main}
passes on directly to the main file \textit{main}
while pretending to compile the child \textit{dest}.
This form behaves as if \textit{dest}
issues |\childdocof{|\textit{main}|}| right away,
and no further \textsf{childdoc} directives will be processed.

%%%%%%%%%%%%%%%%%%%%%%%%%%%%%%%%%%%%%%%%
\DescribeMacro{\...prefix}
In the alternative form |\childdocforwardprefix|,
%
\begin{center}
\begin{tabular}{l}
|\input{childdoc.def}|\\
|\childdocforwardprefix[|\textit{main}|]{|\textit{prefix}|}{|\textit{dest}|}|
\end{tabular}
\end{center}
%
the destination file is determined by a pattern
depending on the current file:
To make this work, the current file must be called
`{\textit{prefix}\hspace{0.2em}\textit{suffix}}'
with \textit{prefix} matching precisely the argument.
Processing is then passed on to the file
`{\textit{dest}\hspace{0.2em}\textit{suffix}}'.
Surely, the same effect is achieved by
directly specifying the
argument `{\textit{dest}\hspace{0.2em}\textit{suffix}}'
in the first form.
However, that requires to set up a different file
for each child. With the alternative form of the command
all these files can have exactly the same content
which simplifies setting them up and maintaining them.

For example, the following file |draft.tex|
with a compilation flag |\version| as described in \secref{sec:flags}
compiles the main document as a draft:
%
\begin{center}
\begin{tabular}{l}
|\def\version{draft}|\\
|\input{childdoc.def}|\\
|\childdocforward{|\textit{main}|}|
\end{tabular}
\end{center}
%
Likewise, the following files |final|\textit{nn}|.tex|
compile the final version of the child document
|child|\textit{nn}|.tex|:
%
\begin{center}
\begin{tabular}{l}
|\def\version{final}|\\
|\input{childdoc.def}|\\
|\childdocforwardprefix{final}{child}|
\end{tabular}
\end{center}
%

Note that when several versions of a main file and/or of each child file
are to be generated, it may be convenient to set up a |Makefile| or
shell script to automatise the process.

%%%%%%%%%%%%%%%%%%%%%%%%%%%%%%%%%%%%%%%%%%%%%%%%%%%%%%%%%%%%%%%%%%%%%%%%%%%%%%%%
\subsection{Command Line Processing}
\label{sec:commandline}

The effect of redirection files can also be achieved by invoking
the \LaTeX{} compiler with a more elaborate command line.
Most conveniently this should be done as part
of a shell script or a |Makefile|.

When using \textsf{childdoc} in the main file, the following
command lines effectively perform a redirection
(note that depending on the shell being used,
backslashes may have to be doubled: `|\|' $\to$ `|\\|'):
%
\begin{center}
|... -jobname "|\textit{target}|" |\\|"|[\textit{flags}]%
|\input{childdoc.def}\childdocforward[|\textit{main}|]{|\textit{dest}|}"|
\end{center}
%
Here \textit{target} is the name of the output file,
\textit{main} is the name of the main file
and \textit{dest} is the name of the main or child file to be processed
(all filenames without extensions).
The optional argument \textit{main} can be omitted
if \textit{main} matches \textit{dest}.
Optionally, compilation \textit{flags} can be defined via |\def| commands.
This command line makes the \TeX{} engine believe
it is compiling the file \textit{target}
whose content is specified as the latter parameter.
The provided code then forwards the processing to
\textit{main} or \textit{dest} as described in \secref{sec:forward}.

%%%%%%%%%%%%%%%%%%%%%%%%%%%%%%%%%%%%%%%%%%%%%%%%%%%%%%%%%%%%%%%%%%%%%%%%%%%%%%%%
\subsection{Include by Input}
\label{sec:input}

Including child documents by |\include| has some restrictions by design.
Most notably, the content of a child document always occupies
its own set of pages; pages cannot be shared between child documents.
Usually, this behaviour makes perfect sense
because each child document contain an essential part of the document.
However, in some situations it may be desirable to compose
a document from a collection of parts
without having mandatory page breaks between then.
For this case, the package
provides a mechanism to include parts
by |\input| which can also be processed individually.
However, by construction this mechanism
requires manual handling of the content to be output.

%%%%%%%%%%%%%%%%%%%%%%%%%%%%%%%%%%%%%%%%
\DescribeMacro{\ifchilddocmanual}
The main file should be prepared as usual, see \secref{sec:include}.
However, the document body must make a distinction
between processing of an individual part and of the main document, e.g.:
%
\begin{center}
\begin{tabular}{l}
|\ifchilddocmanual|\\
|\input{\childdocname}|\\
|\||else|\\
\textit{document body with }|\input{|\textit{part}|}|\\
|\||fi|
\end{tabular}
\end{center}
%
The conditional |\ifchilddocmanual| is true whenever
a part to be included by |\input| is being compiled,
and the name of the part is stored in |\childdocname|.

%%%%%%%%%%%%%%%%%%%%%%%%%%%%%%%%%%%%%%%%
\DescribeMacro{\childdocby}
Each part to be included by |\input| should start with:
%
\begin{center}
\begin{tabular}{l}
|\input{childdoc.def}|\\
|\childdocby{|\textit{main}|}|\\
\end{tabular}
\end{center}
%
The directive |\childdocby| is similar to |\childdocof|
described in \secref{sec:include},
but the subsequent selection of content must be done manually.
To that end, both |\ifchilddoc| and |\ifchilddocmanual|
will be true upon processing of a part,
and the name of the part is stored in |\childdocname|.
Note that |\jobname| will be set to the filename of the current part
so that each part receives an individual |.aux| file
that does not interfere with the |.aux| file(s) of the main document.
This behaviour can be altered by the alternative form
|\childdocby[*]{|\textit{main}|}| (with a non-empty optional argument)
which uses the |.aux| file of the main document
by setting |\jobname| to \textit{main}.

%%%%%%%%%%%%%%%%%%%%%%%%%%%%%%%%%%%%%%%%%%%%%%%%%%%%%%%%%%%%%%%%%%%%%%%%%%%%%%%%
\subsection{Driver Development}
\label{sec:driver}

The \textsf{childdoc} mechanism can also be use for the development
of definition files such as \LaTeX{} styles or classes.
This case differs from the above setup with multiple parts
included by |\include| in that no |\includeonly| should be invoked.
This can be achieved by starting the include file
(before |\ProvidesPackage|) with:
%
\begin{center}
\begin{tabular}{l}
|\input{childdoc.def}|\\
|\childdocforward{|\textit{main}|}|\\
\end{tabular}
\end{center}
%
or alternatively with:
%
\begin{center}
\begin{tabular}{l}
|\input{childdoc.def}|\\
|\childdocby{|\textit{main}|}|\\
\end{tabular}
\end{center}
%
Both forms have slightly different effects as described above.
The main file is prepared as usual, see \secref{sec:include}.

%%%%%%%%%%%%%%%%%%%%%%%%%%%%%%%%%%%%%%%%%%%%%%%%%%%%%%%%%%%%%%%%%%%%%%%%%%%%%%%%
\subsection{Legacy Detection}
\label{sec:detection}

The directive |\childdocmain| in the main file can detect
whether the complete document or merely a child is to be compiled
even without using the directive |\childdocof|.
This method is deprecated because it is less robust
and there is no compelling reason to use it;
it is merely provided for backward compatibility
and it may be removed in future versions.

If the detection mechanism is to be used,
it is mandatory to correctly specify
the filename of the main file as the argument of |\childdocmain|:
%
\begin{center}
\begin{tabular}{l}
|\input{childdoc.def}|\\
|\childdocmain{|\textit{main}|}|\\
\end{tabular}
\end{center}
%
If |\jobname| does not match the argument \textit{main} of |\childdocmain|,
it is assumed that |\jobname| points to the child file to be compiled.
When using |\childdocmain| with the main file specified as argument,
it suffices to start a child file
with just |\input{|\textit{main}|}|
without loading of the package and using |\childdocof|.
If instead all processing is done
with the appropriate \textsf{childdoc} directives,
the argument of \textit{main} of |\childdocmain| can be empty.

An alternative version of the command line processing described
in \secref{sec:commandline} using the detection mechanism reads:
%
\begin{center}
|... -jobname "|\textit{target}|" "|[\textit{flags}]%
[|\def\jobname{|\textit{dest}|}|]|\input{|\textit{main}|}"|
\end{center}

%%%%%%%%%%%%%%%%%%%%%%%%%%%%%%%%%%%%%%%%%%%%%%%%%%%%%%%%%%%%%%%%%%%%%%%%%%%%%%%%
\subsection{Manual Code}
\label{sec:manual}

In case one cannot be certain whether the definitions file |childdoc.def|
is installed on the target \TeX{} distribution
and one prefers not to ship it,
it is conceivable to paste a few relevant commands into the sources.

To that end, drop all statements |\input{childdoc.def}|
and perform the replacements as outlined below.
Instead of |\childdocmain{|\textit{main}|}| add the following code
to the top of the main file:
%
\begin{center}
\begin{tabular}{l}
|\||ifdefined\childdocname\endinput\||fi\newif\ifchilddoc|\\
|\edef\childdocname{\scantokens\expandafter{\jobname\noexpand}}|\\
|\def\childdocmain{|\textit{main}|}\||ifx\childdocmain\childdocname\||else|\\
|\childdoctrue\includeonly{\childdocname}\let\jobname\childdocmain\||fi|\\
\end{tabular}
\end{center}
%
Instead of |\childdocof{|\textit{main}|}| just include the main file
at the top of each child file:
%
\begin{center}
|\input{|\textit{main}|}|
\end{center}
%
A simple redirection |\childdocforward{|\textit{dest}|}| is achieved by:
%
\begin{center}
|\def\jobname{|\textit{dest}|}\input{\jobname}|
\end{center}
%
The redirection with prefix
|\childdocforwardprefix[|\textit{prefix}|]{|\textit{dest}|}|
is accomplished by:
%
\begin{center}
\begin{tabular}{l}
|{\edef\jobname{\scantokens\expandafter{\jobname\noexpand}}|\\
|\def\redirectjob |\textit{prefix}|#1~~~{\gdef\jobname{|\textit{dest}|#1}}|\\
|\expandafter\redirectjob\jobname~~~}\input{\jobname}|
\end{tabular}
\end{center}

In an alternative approach,
child documents can be compiled by a specific command line
without additional code or specific definitions:
%
\begin{center}
|... -jobname "|\textit{target}|" "|[\textit{flags}]%
|\includeonly{|\textit{dest}|}\input{|\textit{main}|}"|
\end{center}
%

%%%%%%%%%%%%%%%%%%%%%%%%%%%%%%%%%%%%%%%%%%%%%%%%%%%%%%%%%%%%%%%%%%%%%%%%%%%%%%%%
%%%%%%%%%%%%%%%%%%%%%%%%%%%%%%%%%%%%%%%%%%%%%%%%%%%%%%%%%%%%%%%%%%%%%%%%%%%%%%%%
\section{Information}

%%%%%%%%%%%%%%%%%%%%%%%%%%%%%%%%%%%%%%%%%%%%%%%%%%%%%%%%%%%%%%%%%%%%%%%%%%%%%%%%
\subsection{Copyright}

Copyright \copyright{} 2017--2018 Niklas Beisert

This work may be distributed and/or modified under the
conditions of the \LaTeX{} Project Public License, either version 1.3
of this license or (at your option) any later version.
The latest version of this license is in
  \url{http://www.latex-project.org/lppl.txt}
and version 1.3 or later is part of all distributions of \LaTeX{}
version 2005/12/01 or later.

This work has the LPPL maintenance status `maintained'.

The Current Maintainer of this work is Niklas Beisert.

This work consists of the files |README.txt|, |childdoc.ins| and |childdoc.dtx|
as well as the derived files |childdoc.def|, |cdocsamp.tex|
with |cdocsch1.tex|, |cdocsch2.tex|, |cdocspt3.tex|, |cdocspt4.tex|,
|cdocsdrf.tex|, |cdocsfn1.tex|, |cdocsfn2.tex|
as well as |childdoc.pdf|.

%%%%%%%%%%%%%%%%%%%%%%%%%%%%%%%%%%%%%%%%%%%%%%%%%%%%%%%%%%%%%%%%%%%%%%%%%%%%%%%%
\subsection{Files and Installation}

The package consists of the files:
%
\begin{center}
\begin{tabular}{ll}
    |README.txt|   & readme file \\
    |childdoc.ins| & installation file \\
    |childdoc.dtx| & source file \\
    |childdoc.def| & definition file \\
    |cdocsamp.tex| & sample main file \\
    |cdocsch1.tex| & sample include file \\
    |cdocsch2.tex| & sample include file \\
    |cdocspt3.tex| & sample part file \\
    |cdocspt4.tex| & sample part file \\
    |cdocsdrf.tex| & sample redirection file \\
    |cdocsfn1.tex| & sample redirection file \\
    |cdocsfn2.tex| & sample redirection file \\
    |childdoc.pdf| & manual
\end{tabular}
\end{center}
%
The distribution consists of the files
|README.txt|, |childdoc.ins| and |childdoc.dtx|.
%
\begin{itemize}
\item
Run (pdf)\LaTeX{} on |childdoc.dtx|
to compile the manual |childdoc.pdf| (this file).
\item
Run \LaTeX{} on |childdoc.ins| to create the definitions file |childdoc.def|
and the sample |cdocsamp.tex| with include files
|cdocsch1.tex|, |cdocsch2.tex|, |cdocspt3.tex|, |cdocspt4.tex|,
|cdocsdrf.tex|, |cdocsfn1.tex|, |cdocsfn2.tex|.
Then copy the file |childdoc.def| to an appropriate directory of your \LaTeX{}
distribution, e.g.\ \textit{texmf-root}|/tex/latex/childdoc|.
\end{itemize}

%%%%%%%%%%%%%%%%%%%%%%%%%%%%%%%%%%%%%%%%%%%%%%%%%%%%%%%%%%%%%%%%%%%%%%%%%%%%%%%%
\subsection{Related CTAN Packages}

There are several other packages which offer a similar functionality:
%
\begin{itemize}
\item
The packages
\href{http://ctan.org/pkg/docmute}{\textsf{docmute}},
\href{http://ctan.org/pkg/includex}{\textsf{includex}} and
\href{http://ctan.org/pkg/standalone}{\textsf{standalone}}
provide commands to include only the document body of
a child file thus allowing both files to be compiled individually.
\item
The packages \href{http://ctan.org/pkg/subdocs}{\textsf{subdocs}}
and \href{http://ctan.org/pkg/subfiles}{\textsf{subfiles}}
provide structures in which the main and child documents can be
encapsulated and allowing them to be compiled individually.
The inclusion mechanism is different from the conventional |\include|.
\item
The package \href{http://ctan.org/pkg/combine}{\textsf{combine}}
is an elaborate solution to combine several documents into one.
\end{itemize}
%
See also the CTAN topic \href{http://ctan.org/topic/subdocs}{\textsf{subdocs}}
for further related packages.
The present package differs from the above solutions in that
a document structure constructed with the conventional |\include| mechanism
just needs two extra commands at the top of every file
such that all constituent files can be compiled individually.

%%%%%%%%%%%%%%%%%%%%%%%%%%%%%%%%%%%%%%%%%%%%%%%%%%%%%%%%%%%%%%%%%%%%%%%%%%%%%%%%
%\subsection{Feature Suggestions}
%
%The following is a list of features which may be useful for future
%versions of this package:
%%
%\begin{itemize}
%\item
%\ldots
%\end{itemize}

%%%%%%%%%%%%%%%%%%%%%%%%%%%%%%%%%%%%%%%%%%%%%%%%%%%%%%%%%%%%%%%%%%%%%%%%%%%%%%%%
\subsection{Revision History}

%%%%%%%%%%%%%%%%%%%%%%%%%%%%%%%%%%%%%%%%
\paragraph{v2.0:} 2018/12/30

\begin{itemize}
\item
immediate forward processing
\item
added |\childdocby| mechanism
\item
manual restructured
\end{itemize}

%%%%%%%%%%%%%%%%%%%%%%%%%%%%%%%%%%%%%%%%
\paragraph{v1.6:} 2018/01/17

\begin{itemize}
\item
application for development of include files
\item
corrections to manual
\end{itemize}

%%%%%%%%%%%%%%%%%%%%%%%%%%%%%%%%%%%%%%%%
\paragraph{v1.5:} 2017/05/21

\begin{itemize}
\item
more complete structuring introduced
\item
|\childdocof| introduced
\item
|\childdoc| renamed to |\childdocmain|
\item
|\childredirect| renamed to |\childdocforward| and |\childdocforwardprefix|
and functionality expanded
\end{itemize}

%%%%%%%%%%%%%%%%%%%%%%%%%%%%%%%%%%%%%%%%
\paragraph{v1.0:} 2017/04/27

\begin{itemize}
\item
manual and install package
\item
first version published on CTAN
\end{itemize}

%%%%%%%%%%%%%%%%%%%%%%%%%%%%%%%%%%%%%%%%
\paragraph{v0.6:} 2017/04/26

\begin{itemize}
\item
redirection mechanism added
\end{itemize}

%%%%%%%%%%%%%%%%%%%%%%%%%%%%%%%%%%%%%%%%
\paragraph{v0.5:} 2017/04/26

\begin{itemize}
\item
functionality in definition file
\end{itemize}


%%%%%%%%%%%%%%%%%%%%%%%%%%%%%%%%%%%%%%%%%%%%%%%%%%%%%%%%%%%%%%%%%%%%%%%%%%%%%%%%
%%%%%%%%%%%%%%%%%%%%%%%%%%%%%%%%%%%%%%%%%%%%%%%%%%%%%%%%%%%%%%%%%%%%%%%%%%%%%%%%
%%%%%%%%%%%%%%%%%%%%%%%%%%%%%%%%%%%%%%%%%%%%%%%%%%%%%%%%%%%%%%%%%%%%%%%%%%%%%%%%
\appendix

\settowidth\MacroIndent{\rmfamily\scriptsize 000\ }

 \DocInput{childdoc.dtx}

\end{document}
%</driver>
% \fi
%
% %%%%%%%%%%%%%%%%%%%%%%%%%%%%%%%%%%%%%%%%%%%%%%%%%%%%%%%%%%%%%%%%%%%%%%%%%%%%%%
% %%%%%%%%%%%%%%%%%%%%%%%%%%%%%%%%%%%%%%%%%%%%%%%%%%%%%%%%%%%%%%%%%%%%%%%%%%%%%%
% \section{Sample}
%\iffalse
%<*samplemain>
%\fi
%
% The following presents a sample document
% with two chapters, two parts, a title page,
% a compile flag as well as three forwarding files to set the flag.
% It consists of eight |.tex| files:
% \begin{center}
% \begin{tabular}{ll}
% |cdocsamp.tex|&main file\\
% |cdocsch1.tex|&include file for chapter 1\\
% |cdocsch2.tex|&include file for chapter 2\\
% |cdocspt3.tex|&include file for part 3\\
% |cdocspt4.tex|&include file for part 4\\
% |cdocsdrf.tex|&forwarding file for main file in draft mode\\
% |cdocsfi1.tex|&forwarding file for final version of chapter 1\\
% |cdocsfi2.tex|&forwarding file for final version of chapter 2\\
% \end{tabular}
% \end{center}
% Each of the eight files can be compiled directly by the \LaTeX{} compiler.
%
% %%%%%%%%%%%%%%%%%%%%%%%%%%%%%%%%%%%%%%
% \paragraph{Main File.}
%
% The main file is called |cdocsamp.tex|.
%
% Load the \textsf{childdoc} definitions and
% declare the filename for the main document:
%    \begin{macrocode}
\input{childdoc.def}
\childdocmain{}
%    \end{macrocode}

% Optional override for |\version| flag:
%    \begin{macrocode}
%%\ifchilddoc\else\providecommand{\version}{draft}\fi
%    \end{macrocode}

% Define the default values for the |\version| flag
% (|final| for the main file and |draft| for childs):
%    \begin{macrocode}
\ifchilddoc
\providecommand{\version}{draft}
\else
\providecommand{\version}{final}
\fi
%    \end{macrocode}

% Load the standard document class:
%    \begin{macrocode}
\documentclass[12pt]{article}
%    \end{macrocode}

% Start the document body:
%    \begin{macrocode}
\begin{document}
%    \end{macrocode}

% Declare a title page.
% Print title, part of document being processed and version flag:
%    \begin{macrocode}
\addtocounter{page}{-1}
\begin{center}
{\LARGE\bfseries{}childdoc example\par}
\vspace{1cm}
\ifchilddoc
\ifchilddocmanual part\else chapter\fi:
`\childdocname' of `\childdocjob'\par
\else
main document: `\childdocjob'\par
\fi
version: \version\par
\end{center}
\newpage
%    \end{macrocode}

% Manually include selected file,
% otherwise process as usual:
%    \begin{macrocode}
\ifchilddocmanual
\section*{part `\childdocname'}
\input{\childdocname}
\else
%    \end{macrocode}

% Include the two chapters:
%    \begin{macrocode}
\include{cdocsch1}
\include{cdocsch2}
%    \end{macrocode}

% Include the two parts unless only chapters should be displayed:
%    \begin{macrocode}
\ifchilddoc\else
\section{part three}
\input{cdocspt3}
\section{part four}
\input{cdocspt4}
\fi
%    \end{macrocode}

% Process as usual until here:
%    \begin{macrocode}
\fi
%    \end{macrocode}

% End of document body:
%    \begin{macrocode}
\end{document}
%    \end{macrocode}
%\iffalse
%</samplemain>
%\fi
%
% %%%%%%%%%%%%%%%%%%%%%%%%%%%%%%%%%%%%%%
% \paragraph{Chapter Include Files.}
%
% The include files are called |cdocsch1.tex| and |cdocsch2.tex|.
%
%\iffalse
%<*samplechap1|samplechap2>
%\fi

% Optional override for |\version| flag:
%    \begin{macrocode}
%%\providecommand{\version}{final}
%    \end{macrocode}

% Include the main document:
%    \begin{macrocode}
\input{childdoc.def}
\childdocof{cdocsamp}
%    \end{macrocode}

%\iffalse
%</samplechap1|samplechap2>
%\fi
%
%\iffalse
%<*samplechap1>
%\fi
% Some text for chapter 1:
%    \begin{macrocode}
\section{one}
some text in chapter one
%    \end{macrocode}

%\iffalse
%</samplechap1>
%\fi
% Some text for chapter 2:
%\iffalse
%<*samplechap2>
%\fi
%    \begin{macrocode}
\section{two}
more text in chapter two
%    \end{macrocode}

%\iffalse
%</samplechap2>
%\fi
%
% %%%%%%%%%%%%%%%%%%%%%%%%%%%%%%%%%%%%%%
% \paragraph{Part Include Files.}
%
% The include files are called |cdocspt3.tex| and |cdocspt4.tex|.
%
%\iffalse
%<*samplepart3|samplepart4>
%\fi

% Optional override for |\version| flag:
%    \begin{macrocode}
%%\providecommand{\version}{final}
%    \end{macrocode}

% Include the main document:
%    \begin{macrocode}
\input{childdoc.def}
\childdocby{cdocsamp}
%    \end{macrocode}

%\iffalse
%</samplepart3|samplepart4>
%\fi
%
%\iffalse
%<*samplepart3>
%\fi
% Some text for part 3:
%    \begin{macrocode}
some text in part three
%    \end{macrocode}

%\iffalse
%</samplepart3>
%\fi
% Some text for part 4:
%\iffalse
%<*samplepart4>
%\fi
%    \begin{macrocode}
more text in part four
%    \end{macrocode}

%\iffalse
%</samplepart4>
%\fi
%
% %%%%%%%%%%%%%%%%%%%%%%%%%%%%%%%%%%%%%%
% \paragraph{Forwarding for a Complete Draft.}
%
% The following forwarding file |cdocsdrf.tex|
% compiles the main document in draft mode:
%\iffalse
%<*sampledraft>
%\fi
%    \begin{macrocode}
\def\version{draft}
\input{childdoc.def}
\childdocforward{cdocsamp}
%    \end{macrocode}

%\iffalse
%</sampledraft>
%\fi
%
% %%%%%%%%%%%%%%%%%%%%%%%%%%%%%%%%%%%%%%
% \paragraph{Forwarding for Final Version of the Chapters.}
%
% The following forwarding files |cdocsfn1.tex| and |cdocsfn2.tex|
% (with identical content)
% compile the final versions of the child documents
% |cdocsch1.tex| and |cdocsch2.tex|, respectively:
%\iffalse
%<*samplefinal>
%\fi
%    \begin{macrocode}
\def\version{final}
\input{childdoc.def}
\childdocforwardprefix[cdocsamp]{cdocsfn}{cdocsch}
%    \end{macrocode}

%\iffalse
%</samplefinal>
%\fi
%
% %%%%%%%%%%%%%%%%%%%%%%%%%%%%%%%%%%%%%%
% \paragraph{Command Line Processing.}
%
% The following three command lines generate the output files
% |cdocscld|, |cdocscl1| and |cdocscl2|
% which should be identical to
% |cdocsdrf|, |cdocsch1| and |cdocsfn2|, respectively:
% \begin{center}
% \begin{tabular}{l}
% |latex -jobname cdocscld \|\\
% |  "\def\version{draft}\input{childdoc.def}\childdocforward{cdocsamp}"|\\
% |latex -jobname cdocscl1 \|\\
% |  "\input{childdoc.def}\childdocforward[cdocsamp]{cdocsch1}"|\\
% |latex -jobname cdocscl2 \|\\
% |  "\def\version{final}\input{childdoc.def}\childdocforward{cdocsch2}"|
% \end{tabular}
% \end{center}
% Note that the trailing backslash on each first line
% merely continues the input to the second line
% (for convenient cut ant paste).
% Furthermore, the command |latex| can be replaced by any
% of its alternative versions such as |pdflatex|.
%
% %%%%%%%%%%%%%%%%%%%%%%%%%%%%%%%%%%%%%%%%%%%%%%%%%%%%%%%%%%%%%%%%%%%%%%%%%%%%%%
% %%%%%%%%%%%%%%%%%%%%%%%%%%%%%%%%%%%%%%%%%%%%%%%%%%%%%%%%%%%%%%%%%%%%%%%%%%%%%%
% \section{Implementation}
%\iffalse
%<*package>
%\fi
%
% This section describes the definitions file |childdoc.def|.

% The definitions cannot be loaded using |\usepackage| or |\RequirePackage|
% which has a mechanism to prevent loading a style file more than once.
% When loading the definitions by means of |\input|
% multiple instances have to be prevented manually:
%\iffalse
%This code needs to be before the `\ProvidesFile' directive
%which is defined at the beginning of this file.
%Therefore it is also placed there and commented out here.
%</package>
%<*discard>
%\fi
%    \begin{macrocode}
\ifdefined\childdocmain\endinput\fi
%    \end{macrocode}
%\iffalse
%</discard>
%<*package>
%\fi
%
% \macro{\ifchilddoc}
% \macro{\ifchilddocmanual}
% The conditional |\ifchilddoc| tells whether a
% child (true) or main (false) document is being compiled.
% The conditional |\ifchilddocmanual| tells whether
% the |\includeonly| mechanism is used (false) or
% the selection of child files must be performed manually (true).
% The definitions initialise to false:
%    \begin{macrocode}
\newif\ifchilddoc
\newif\ifchilddocmanual
%    \end{macrocode}

% \macro{\childdocname}
% \macro{\childdocjob}
% The macro |\childdocname| stores the name of the main document
% to be compiled. The macro |\childdocjob| stores the name of
% the document on which the \LaTeX{} compiler was originally invoked.
% The content of |\jobname| cannot be compared
% to filenames specified in the source due to different catcodes.
% The following code rescans |\jobname|, stores the result
% in |\childdocname| and saves a copy in |\childdocjob|:
%    \begin{macrocode}
\edef\childdocname{\scantokens\expandafter{\jobname\noexpand}}
\let\childdocjob\childdocname
%    \end{macrocode}

% \macro{\childdocdisable}
% The macro |\childdocdisable| prevents the main file
% from being processed more than once.
% At this stage, the main document command |\childdocmain|
% is assumed to be called once again where it should do nothing.
% Any subsequent call to it should prevent
% a secondary processing of the main document
% It overwrites the forwarding commands
% |\childdocof| and |\childdocforward|
% with empty macros to prevent further inclusions of the main document:
%    \begin{macrocode}
\newcommand{\childdocdisable}
{
  \renewcommand{\childdocmain}[1]{\renewcommand{\childdocmain}[1]{\endinput}}
  \renewcommand{\childdocof}[1]{}
  \renewcommand{\childdocby}[2][]{}
  \renewcommand{\childdocforward}[2][]{}
  \renewcommand{\childdocdisable}{}
}
%    \end{macrocode}

% \macro{\childdocmain}
% The macro |\childdocmain| is to be called at the top of the main file
% with nothing or the main filename (without extension) as argument.
% First, it breaks loops.
% If the argument is not empty and does not match |\childdocname|
% (which is set by the first inclusion of |childdoc.def|),
% |\ifchilddoc| is set to true, |\includeonly| is applied to the child file
% and |\jobname| is set to the main file
% (for proper handling of |.aux| files):
%    \begin{macrocode}
\newcommand{\childdocmain}[1]
{
  \childdocdisable\childdocmain{}
  \if?#1?\else
    \begingroup
      \def\childdoctmp{#1}
      \ifx\childdoctmp\childdocname
        \def\childdoctmp{}
      \else
        \def\childdoctmp
        {
          \childdoctrue
          \includeonly{\childdocname}
          \def\childdocjob{#1}
          \def\jobname{#1}
        }
      \fi
      \expandafter
    \endgroup
    \childdoctmp
  \fi
}
%    \end{macrocode}

% \macro{\childdocof}
% The command |\childdocof| redirects
% compilation to the main file |#1|.
%    \begin{macrocode}
\newcommand{\childdocof}[1]
{
  \childdocdisable
  \childdoctrue
  \includeonly{\childdocname}
  \def\jobname{#1}
  \def\childdocjob{#1}
  \input{#1}
}
%    \end{macrocode}

% \macro{\childdocby}
% The command |\childdocby| ....
%    \begin{macrocode}
\newcommand{\childdocby}[2][]
{
  \childdocdisable
  \childdoctrue
  \childdocmanualtrue
  \if?#1?\else
    \def\jobname{#2}
  \fi
  \def\childdocjob{#2}
  \input{#2}
  \endinput
}
%    \end{macrocode}

% \macro{\childdocforward}
% The command |\childdocforward| redirects
% compilation to the main file or
% (if the optional argument is given) a child file.
% Parameters are set as if the main file
% or a child file starting with |\childdocof| was compiled.
% Then compilation is handed over to the main file:
%    \begin{macrocode}
\newcommand{\childdocforward}[2][]
{
  \begingroup
    \if?#1?
      \def\childdoctmp
      {
        \def\childdocname{#2}
        \def\childdocjob{#2}
        \def\jobname{#2}
        \input{#2}
        \endinput
      }
    \else
      \def\childdoctmp
      {
        \childdocdisable
        \def\childdocname{#2}
        \childdoctrue
        \includeonly{#2}
        \def\childdocjob{#1}
        \def\jobname{#1}
        \input{#1}
        \endinput
      }
    \fi
    \expandafter
  \endgroup
  \childdoctmp
}
%    \end{macrocode}

% \macro{\childdocforwardprefix}
% The command |\childdocforwardprefix| redirects
% compilation to the main or a child file by means of a pattern.
% The prefix |#1| in the current filename is replaced by |#2|
% and the suffix of the current filename is kept
% (it is assumed that the filename does not contain the substring `|~~~|'
% which is used as a delimiter).
% Compilation is handed over to the new file by |\childdocforward|:
%    \begin{macrocode}
\newcommand{\childdocforwardprefix}[3][]
{
  \begingroup
    \def\childdocextract #2##1~~~{\def\childdoctmp{\childdocforward[#1]{#3##1}}}
    \expandafter\childdocextract\childdocname~~~
    \expandafter
  \endgroup
  \childdoctmp
}
%    \end{macrocode}

% \macro{\childdoc}
% The deprecated macro |\childdoc| is a legacy version of |\childdocmain|:
%    \begin{macrocode}
\newcommand{\childdoc}{\childdocmain}
%    \end{macrocode}

% \macro{\childdocredirect}
% The deprecated macro |\childdocredirect| is a legacy version
% of |\childdocforward| and |\childdocforwardprefix|:
%    \begin{macrocode}
\newcommand{\childdocredirect}[2][]
{
  \begingroup
    \if?#1?
      \def\childdoctmp{\childdocforward{#2}}
    \else
      \def\childdoctmp{\childdocforwardprefix{#1}{#2}}
    \fi
    \expandafter
  \endgroup
  \childdoctmp
}
%    \end{macrocode}

%\iffalse
%</package>
%\fi
%
\endinput
|\\
|\childdocmain{}|\\
\end{tabular}
\end{center}
at the very top of the main \LaTeX{} file,
in particular \emph{before} the |\documentclass| statement!
The argument of |\childdocmain| should be left empty
(but it must be present).

%%%%%%%%%%%%%%%%%%%%%%%%%%%%%%%%%%%%%%%%
\DescribeMacro{\childdocof}
Furthermore, add the commands
\begin{center}
\begin{tabular}{l}
|% \iffalse
%
% childdoc.dtx Copyright (C) 2017-2018 Niklas Beisert
%
% This work may be distributed and/or modified under the
% conditions of the LaTeX Project Public License, either version 1.3
% of this license or (at your option) any later version.
% The latest version of this license is in
%   http://www.latex-project.org/lppl.txt
% and version 1.3 or later is part of all distributions of LaTeX
% version 2005/12/01 or later.
%
% This work has the LPPL maintenance status `maintained'.
%
% The Current Maintainer of this work is Niklas Beisert.
%
% This work consists of the files childdoc.dtx and childdoc.ins
% and the derived files childdoc.def and cdocsamp.tex with
% cdocsch1.tex, cdocsch2.tex, cdocsdrf.tex, cdocsfn1.tex, cdocsfn2.tex.
%
%<package>\ifdefined\childdocmain\endinput\fi
%<package>\ProvidesFile{childdoc.def}[2018/12/30 v2.0 child document driver]
%<samplemain>\ProvidesFile{cdocsamp.tex}[2018/12/30 v2.0 sample for childdoc]
%<*driver>
%\ProvidesFile{childdoc.drv}[2018/12/30 v2.0 childdoc reference manual file]
\PassOptionsToClass{10pt,a4paper}{article}
\documentclass{ltxdoc}

\usepackage[margin=35mm]{geometry}
\usepackage{hyperref}
\usepackage{hyperxmp}
\usepackage[usenames]{color}

\hypersetup{colorlinks=true}
\hypersetup{pdfstartview=FitH}
\hypersetup{pdfpagemode=UseNone}
\hypersetup{pdfsource={}}
\hypersetup{pdflang={en-UK}}
\hypersetup{pdfcopyright={Copyright 2017-2018 Niklas Beisert.
  This work may be distributed and/or modified under the
  conditions of the LaTeX Project Public License, either version 1.3
  of this license or (at your option) any later version.}}
\hypersetup{pdflicenseurl={http://www.latex-project.org/lppl.txt}}
\hypersetup{pdfcontactaddress={ETH Zurich, ITP, HIT K,
  Wolfgang-Pauli-Strasse 27}}
\hypersetup{pdfcontactpostcode={8093}}
\hypersetup{pdfcontactcity={Zurich}}
\hypersetup{pdfcontactcountry={Switzerland}}
\hypersetup{pdfcontactemail={nbeisert@itp.phys.ethz.ch}}
\hypersetup{pdfcontacturl={http://people.phys.ethz.ch/\xmptilde nbeisert/}}

\newcommand{\secref}[1]{\hyperref[#1]{section \ref*{#1}}}

\parskip1ex
\parindent0pt
\let\olditemize\itemize
\def\itemize{\olditemize\parskip0pt}

\begin{document}

\title{The \textsf{childdoc} Package}
\hypersetup{pdftitle={The childdoc Package}}
\author{Niklas Beisert\\[2ex]
  Institut f\"ur Theoretische Physik\\
  Eidgen\"ossische Technische Hochschule Z\"urich\\
  Wolfgang-Pauli-Strasse 27, 8093 Z\"urich, Switzerland\\[1ex]
  \href{mailto:nbeisert@itp.phys.ethz.ch}
  {\texttt{nbeisert@itp.phys.ethz.ch}}}
\hypersetup{pdfauthor={Niklas Beisert}}
\hypersetup{pdfsubject={Manual for the LaTeX2e Package childdoc}}
\date{30 December 2018, \textsf{v2.0}}
\maketitle

\begin{abstract}\noindent
\textsf{childdoc} is a \LaTeXe{} package
that enables the direct compilation
of document sections included by |\include|
to individual files.
\end{abstract}

\begingroup
\parskip0ex
\tableofcontents
\endgroup

%%%%%%%%%%%%%%%%%%%%%%%%%%%%%%%%%%%%%%%%%%%%%%%%%%%%%%%%%%%%%%%%%%%%%%%%%%%%%%%%
%%%%%%%%%%%%%%%%%%%%%%%%%%%%%%%%%%%%%%%%%%%%%%%%%%%%%%%%%%%%%%%%%%%%%%%%%%%%%%%%
\section{Introduction}

\LaTeX{} provides a mechanism to structure a large document (such as a book)
into a main file and several child files (containing the chapters)
using the |\include| command.
This mechanism is beneficial for documents
which span hundreds of pages in order to
make the source file(s) more manageable.
Moreover, compilation can be restricted to
selected child files by means of the |\includeonly| command.
The latter feature can be used to reduce the compilation time while editing
(this was significantly more useful in the earlier days of \LaTeX{})
or to generate a smaller document which is easier to navigate.
Another application of |\includeonly| is to generate
documents consisting of selected parts of the complete document.

However, there are a few drawbacks of the plain |\include| mechanism:
\begin{itemize}
\item
The child files cannot be compiled on their own,
they can only be compiled via the main file.
A naive editing environment
(such as a text editor with an option
to have the current file processed by \LaTeX)
may require one to switch to the main file before compiling;
attempting to compile the child file produces errors.
\item
The main file must be modified (each time)
to adjust the |\includeonly| command
to the present needs. This easily leaves the main file in a messy state.
\item
The generated document will always carry the filename
of the main document. This is inconvenient if
several child files are to be compiled and
to be kept for distribution.
\end{itemize}

The present package provides a simple interface
to make child files individually compilable by \LaTeX{}.
Compiling a child file then has the same effect as compiling
the main file with an |\includeonly| command
to select the appropriate child.
Moreover the generated document will carry the name of the child
rather than the main file.
This resolves all three above issues.

This feature is meant to make the editing of books,
thesis documents and lecture notes somewhat more convenient.
However, the package can also be used efficiently for
composing a series of documents (such as exercise sheets)
which are typically distributed individually.
It then assists the author in generating the individual documents
(potentially in different versions)
as well as a document containing the collected series.
Another application is in developing style files
or other kinds of included material
where compilation of the style file could redirect
to a sample or test file.

%%%%%%%%%%%%%%%%%%%%%%%%%%%%%%%%%%%%%%%%%%%%%%%%%%%%%%%%%%%%%%%%%%%%%%%%%%%%%%%%
%%%%%%%%%%%%%%%%%%%%%%%%%%%%%%%%%%%%%%%%%%%%%%%%%%%%%%%%%%%%%%%%%%%%%%%%%%%%%%%%
\section{Usage}

First of all, the package \textsf{childdoc} is \emph{not} a standard
\LaTeXe{} |.sty| style file! Therefore it needs to be invoked in
a non-standard way.

%%%%%%%%%%%%%%%%%%%%%%%%%%%%%%%%%%%%%%%%%%%%%%%%%%%%%%%%%%%%%%%%%%%%%%%%%%%%%%%%
\subsection{Included Files}
\label{sec:include}

%%%%%%%%%%%%%%%%%%%%%%%%%%%%%%%%%%%%%%%%
\DescribeMacro{\childdocmain}
To use the package, add the commands
\begin{center}
\begin{tabular}{l}
|\input{childdoc.def}|\\
|\childdocmain{}|\\
\end{tabular}
\end{center}
at the very top of the main \LaTeX{} file,
in particular \emph{before} the |\documentclass| statement!
The argument of |\childdocmain| should be left empty
(but it must be present).

%%%%%%%%%%%%%%%%%%%%%%%%%%%%%%%%%%%%%%%%
\DescribeMacro{\childdocof}
Furthermore, add the commands
\begin{center}
\begin{tabular}{l}
|\input{childdoc.def}|\\
|\childdocof{|\textit{main}|}|\\
\end{tabular}
\end{center}
at the top of every child file \textit{child}
which is included by |\include{|\textit{child}|}|
from within the main file
(or at least for those files to be compiled individually).
The argument \textit{main} must be the filename of the main file.

There are a couple of
considerations in setting up the main and child documents:

%%%%%%%%%%%%%%%%%%%%%%%%%%%%%%%%%%%%%%%%
\paragraph{Restrictions.}

Please note the following restrictions:
\begin{itemize}
\item
|\childdocmain| must be called with one argument \textit{main}
to ensure compatibility with earlier version of the package.
It must either be empty (|\childdocmain{}|)
or precisely match the filename of the main file in which it is specified.
See \secref{sec:detection} for further information.
\item
The filename \textit{main} must be specified without the |.tex| extension.
\item
The filename \textit{main} is case sensitive
(even in case-insensitive file systems)
due to internal string comparison.
\item
The argument \textit{main} should be fully expanded, it cannot be a macro.
\item
Subdirectories and special characters should be avoided in filenames.
\item
The command |\childdocmain{|\textit{main}|}| must be followed by a whitespace.
It should not be followed immediately by another command
or by a comment mark `|%|'.
This is because the \TeX{} parser reads the token immediately following
the argument of |\childdocmain| and puts it
at the beginning of every child section;
however, a white\-space is ignored.
\end{itemize}

%%%%%%%%%%%%%%%%%%%%%%%%%%%%%%%%%%%%%%%%
\paragraph{Content of Main File.}

It is advisable to place all content in the child files included by |\include|.
Any output contained in the main file will appear in all child documents
unless suppressed manually;
it cannot be suppressed automatically by the |\includeonly| directive
and thus should normally be avoided.
A method to include some content in the main file
by means of conditional processing is described in \secref{sec:conditional}.

%%%%%%%%%%%%%%%%%%%%%%%%%%%%%%%%%%%%%%%%
\paragraph{Page Numbering.}

When only a part of the document is compiled,
the appropriate numbering of pages
(as well as other status parameters)
is determined from the |.aux| files.
The latter contain information from previous passes.
However this information needs to propagate through
all intermediate child documents.
Therefore the page numbering in child documents may well
be inconsistent until the complete document is compiled at least once.

A useful (if unconventional) way to always ensure a consistent
page numbering is to restart the numbering in each child document
and denote the pages by `\textit{child}|.|\textit{page}'
where \textit{child} represents the chapter/section number of the child file.
This can be achieved by the command
|\numberwithin{page}{|\textit{child}|}|
of the \textsf{amsmath} package
where \textit{child} can be |chapter| or |section|
depending on the chosen structuring.
Alternatively, one can modify the macro |\thepage| appropriately
and reset the counter |page| at the start of each child file.

%%%%%%%%%%%%%%%%%%%%%%%%%%%%%%%%%%%%%%%%%%%%%%%%%%%%%%%%%%%%%%%%%%%%%%%%%%%%%%%%
\subsection{Conditional Processing}
\label{sec:conditional}

The package provides a mechanism to compile different versions
of a document. To customise the versions further some conditional processing
can come in handy to distinguish which version is being compiled.
The package provides two macros to describe the compilation context:

%%%%%%%%%%%%%%%%%%%%%%%%%%%%%%%%%%%%%%%%
\DescribeMacro{\ifchilddoc}
The conditional |\ifchilddoc| distinguishes between the compilation of
child documents and the main document:
%
\begin{center}
|\ifchilddoc |\textit{child-code}| |[|\||else |\textit{main-code}]| \||fi|
\end{center}

%%%%%%%%%%%%%%%%%%%%%%%%%%%%%%%%%%%%%%%%
\DescribeMacro{\childdocname}
\DescribeMacro{\childdocjob}
The macro |\childdocname| contains the filename (without extension)
of the main or child file being processed.
Note that |\childdocjob| will always contain the name of the main file.

%%%%%%%%%%%%%%%%%%%%%%%%%%%%%%%%%%%%%%%%
\paragraph{Title Page.}

Conditional processing can be used to include a title or banner page
in the main document when proper precautions are taken.
Importantly, the code in the main file should ensure that the page counter
(as well as other status parameters which are stored in the |.aux| files)
takes the same value after the conditional processing.
Otherwise the page numbers may take divergent values
depending on which part is compiled.

For example, a title page could be declared by:
%
\begin{center}
\begin{tabular}{l}
|\ifchilddoc\||else|\\
|\addtocounter{page}{-1}|\\
\textit{code for title page}\\
|\newpage|\\
|\||fi|
\end{tabular}
\end{center}
%
A banner page for the child documents can be generated by:
%
\begin{center}
\begin{tabular}{l}
|\ifchilddoc|\\
|\addtocounter{page}{-1}|\\
\textit{code for banner page}\\
|\newpage|\\
|\||fi|
\end{tabular}
\end{center}
%
Here one could write a message such as:
\begin{center}
|This is the part \childdocname{} of \childdocjob{}.|
\end{center}

%%%%%%%%%%%%%%%%%%%%%%%%%%%%%%%%%%%%%%%%%%%%%%%%%%%%%%%%%%%%%%%%%%%%%%%%%%%%%%%%
\subsection{Flags}
\label{sec:flags}

The package makes it easy to generate different versions
of the main or child documents.
To this end compilation flags can be defined
and assigned different default values.
They will be particularly useful in conjunction
with the forwarding mechanism described in \secref{sec:forward}.

For example, it may be useful to have a flag |\version|
which can be set to |draft| or |final|.
The document source will contain some conditional code
depending on the value of |\version|.
Suppose further, the flag should default to |final| for the main file
and to |draft| for child files
which is a natural assignment for editing the document.
This is achieved by placing the following code
in the preamble of the main document
(below the |\childdocmain| directive):
%
\begin{center}
\begin{tabular}{l}
|\ifchilddoc|\\
|\providecommand{\version}{draft}|\\
|\||else|\\
|\providecommand{\version}{final}|\\
|\||fi|
\end{tabular}
\end{center}
%
The definition by |\providecommand| makes sure
that previous definitions are not overwritten.
Further statements |\providecommand{\version}{...}|
can thus be added before the above code to override it.

For the main file, one might add a line
(between |\childdocmain| and the above block)
%
\begin{center}
|%\ifchilddoc\||else\providecommand{\version}{draft}\||fi|
\end{center}
%
which can be uncommented to produce a draft version.
Likewise one can add a line to the very top of a child file
(above the |\childdocof{|\textit{main}|}| directive)
%
\begin{center}
|%\providecommand{\version}{final}|
\end{center}
%
which can be uncommented to produce the final version of this child document.

%%%%%%%%%%%%%%%%%%%%%%%%%%%%%%%%%%%%%%%%%%%%%%%%%%%%%%%%%%%%%%%%%%%%%%%%%%%%%%%%
\subsection{Forwarding}
\label{sec:forward}

Different versions of the main or child documents
using compilation flags as described in \secref{sec:flags}
can be (permanently) stored in different files
for convenient compilation, viewing and distribution.
To this end, the package defines a command
to pass on compilation to a different file:

%%%%%%%%%%%%%%%%%%%%%%%%%%%%%%%%%%%%%%%%
\DescribeMacro{\childdocforward}
The command |\childdocforward| redirects processing to
another source file:
%
\begin{center}
\begin{tabular}{l}
|\input{childdoc.def}|\\
|\childdocforward[|\textit{main}|]{|\textit{dest}|}|\\
\end{tabular}
\end{center}
%
The argument \textit{dest} is the destination file
(without extension).
It should be the main file or one of the child files.
Note that further \textsf{childdoc} directives
such as |\childdocof| and |\childdocforward|
in the indicated file will be processed in this form.
The optional argument \textit{main}
passes on directly to the main file \textit{main}
while pretending to compile the child \textit{dest}.
This form behaves as if \textit{dest}
issues |\childdocof{|\textit{main}|}| right away,
and no further \textsf{childdoc} directives will be processed.

%%%%%%%%%%%%%%%%%%%%%%%%%%%%%%%%%%%%%%%%
\DescribeMacro{\...prefix}
In the alternative form |\childdocforwardprefix|,
%
\begin{center}
\begin{tabular}{l}
|\input{childdoc.def}|\\
|\childdocforwardprefix[|\textit{main}|]{|\textit{prefix}|}{|\textit{dest}|}|
\end{tabular}
\end{center}
%
the destination file is determined by a pattern
depending on the current file:
To make this work, the current file must be called
`{\textit{prefix}\hspace{0.2em}\textit{suffix}}'
with \textit{prefix} matching precisely the argument.
Processing is then passed on to the file
`{\textit{dest}\hspace{0.2em}\textit{suffix}}'.
Surely, the same effect is achieved by
directly specifying the
argument `{\textit{dest}\hspace{0.2em}\textit{suffix}}'
in the first form.
However, that requires to set up a different file
for each child. With the alternative form of the command
all these files can have exactly the same content
which simplifies setting them up and maintaining them.

For example, the following file |draft.tex|
with a compilation flag |\version| as described in \secref{sec:flags}
compiles the main document as a draft:
%
\begin{center}
\begin{tabular}{l}
|\def\version{draft}|\\
|\input{childdoc.def}|\\
|\childdocforward{|\textit{main}|}|
\end{tabular}
\end{center}
%
Likewise, the following files |final|\textit{nn}|.tex|
compile the final version of the child document
|child|\textit{nn}|.tex|:
%
\begin{center}
\begin{tabular}{l}
|\def\version{final}|\\
|\input{childdoc.def}|\\
|\childdocforwardprefix{final}{child}|
\end{tabular}
\end{center}
%

Note that when several versions of a main file and/or of each child file
are to be generated, it may be convenient to set up a |Makefile| or
shell script to automatise the process.

%%%%%%%%%%%%%%%%%%%%%%%%%%%%%%%%%%%%%%%%%%%%%%%%%%%%%%%%%%%%%%%%%%%%%%%%%%%%%%%%
\subsection{Command Line Processing}
\label{sec:commandline}

The effect of redirection files can also be achieved by invoking
the \LaTeX{} compiler with a more elaborate command line.
Most conveniently this should be done as part
of a shell script or a |Makefile|.

When using \textsf{childdoc} in the main file, the following
command lines effectively perform a redirection
(note that depending on the shell being used,
backslashes may have to be doubled: `|\|' $\to$ `|\\|'):
%
\begin{center}
|... -jobname "|\textit{target}|" |\\|"|[\textit{flags}]%
|\input{childdoc.def}\childdocforward[|\textit{main}|]{|\textit{dest}|}"|
\end{center}
%
Here \textit{target} is the name of the output file,
\textit{main} is the name of the main file
and \textit{dest} is the name of the main or child file to be processed
(all filenames without extensions).
The optional argument \textit{main} can be omitted
if \textit{main} matches \textit{dest}.
Optionally, compilation \textit{flags} can be defined via |\def| commands.
This command line makes the \TeX{} engine believe
it is compiling the file \textit{target}
whose content is specified as the latter parameter.
The provided code then forwards the processing to
\textit{main} or \textit{dest} as described in \secref{sec:forward}.

%%%%%%%%%%%%%%%%%%%%%%%%%%%%%%%%%%%%%%%%%%%%%%%%%%%%%%%%%%%%%%%%%%%%%%%%%%%%%%%%
\subsection{Include by Input}
\label{sec:input}

Including child documents by |\include| has some restrictions by design.
Most notably, the content of a child document always occupies
its own set of pages; pages cannot be shared between child documents.
Usually, this behaviour makes perfect sense
because each child document contain an essential part of the document.
However, in some situations it may be desirable to compose
a document from a collection of parts
without having mandatory page breaks between then.
For this case, the package
provides a mechanism to include parts
by |\input| which can also be processed individually.
However, by construction this mechanism
requires manual handling of the content to be output.

%%%%%%%%%%%%%%%%%%%%%%%%%%%%%%%%%%%%%%%%
\DescribeMacro{\ifchilddocmanual}
The main file should be prepared as usual, see \secref{sec:include}.
However, the document body must make a distinction
between processing of an individual part and of the main document, e.g.:
%
\begin{center}
\begin{tabular}{l}
|\ifchilddocmanual|\\
|\input{\childdocname}|\\
|\||else|\\
\textit{document body with }|\input{|\textit{part}|}|\\
|\||fi|
\end{tabular}
\end{center}
%
The conditional |\ifchilddocmanual| is true whenever
a part to be included by |\input| is being compiled,
and the name of the part is stored in |\childdocname|.

%%%%%%%%%%%%%%%%%%%%%%%%%%%%%%%%%%%%%%%%
\DescribeMacro{\childdocby}
Each part to be included by |\input| should start with:
%
\begin{center}
\begin{tabular}{l}
|\input{childdoc.def}|\\
|\childdocby{|\textit{main}|}|\\
\end{tabular}
\end{center}
%
The directive |\childdocby| is similar to |\childdocof|
described in \secref{sec:include},
but the subsequent selection of content must be done manually.
To that end, both |\ifchilddoc| and |\ifchilddocmanual|
will be true upon processing of a part,
and the name of the part is stored in |\childdocname|.
Note that |\jobname| will be set to the filename of the current part
so that each part receives an individual |.aux| file
that does not interfere with the |.aux| file(s) of the main document.
This behaviour can be altered by the alternative form
|\childdocby[*]{|\textit{main}|}| (with a non-empty optional argument)
which uses the |.aux| file of the main document
by setting |\jobname| to \textit{main}.

%%%%%%%%%%%%%%%%%%%%%%%%%%%%%%%%%%%%%%%%%%%%%%%%%%%%%%%%%%%%%%%%%%%%%%%%%%%%%%%%
\subsection{Driver Development}
\label{sec:driver}

The \textsf{childdoc} mechanism can also be use for the development
of definition files such as \LaTeX{} styles or classes.
This case differs from the above setup with multiple parts
included by |\include| in that no |\includeonly| should be invoked.
This can be achieved by starting the include file
(before |\ProvidesPackage|) with:
%
\begin{center}
\begin{tabular}{l}
|\input{childdoc.def}|\\
|\childdocforward{|\textit{main}|}|\\
\end{tabular}
\end{center}
%
or alternatively with:
%
\begin{center}
\begin{tabular}{l}
|\input{childdoc.def}|\\
|\childdocby{|\textit{main}|}|\\
\end{tabular}
\end{center}
%
Both forms have slightly different effects as described above.
The main file is prepared as usual, see \secref{sec:include}.

%%%%%%%%%%%%%%%%%%%%%%%%%%%%%%%%%%%%%%%%%%%%%%%%%%%%%%%%%%%%%%%%%%%%%%%%%%%%%%%%
\subsection{Legacy Detection}
\label{sec:detection}

The directive |\childdocmain| in the main file can detect
whether the complete document or merely a child is to be compiled
even without using the directive |\childdocof|.
This method is deprecated because it is less robust
and there is no compelling reason to use it;
it is merely provided for backward compatibility
and it may be removed in future versions.

If the detection mechanism is to be used,
it is mandatory to correctly specify
the filename of the main file as the argument of |\childdocmain|:
%
\begin{center}
\begin{tabular}{l}
|\input{childdoc.def}|\\
|\childdocmain{|\textit{main}|}|\\
\end{tabular}
\end{center}
%
If |\jobname| does not match the argument \textit{main} of |\childdocmain|,
it is assumed that |\jobname| points to the child file to be compiled.
When using |\childdocmain| with the main file specified as argument,
it suffices to start a child file
with just |\input{|\textit{main}|}|
without loading of the package and using |\childdocof|.
If instead all processing is done
with the appropriate \textsf{childdoc} directives,
the argument of \textit{main} of |\childdocmain| can be empty.

An alternative version of the command line processing described
in \secref{sec:commandline} using the detection mechanism reads:
%
\begin{center}
|... -jobname "|\textit{target}|" "|[\textit{flags}]%
[|\def\jobname{|\textit{dest}|}|]|\input{|\textit{main}|}"|
\end{center}

%%%%%%%%%%%%%%%%%%%%%%%%%%%%%%%%%%%%%%%%%%%%%%%%%%%%%%%%%%%%%%%%%%%%%%%%%%%%%%%%
\subsection{Manual Code}
\label{sec:manual}

In case one cannot be certain whether the definitions file |childdoc.def|
is installed on the target \TeX{} distribution
and one prefers not to ship it,
it is conceivable to paste a few relevant commands into the sources.

To that end, drop all statements |\input{childdoc.def}|
and perform the replacements as outlined below.
Instead of |\childdocmain{|\textit{main}|}| add the following code
to the top of the main file:
%
\begin{center}
\begin{tabular}{l}
|\||ifdefined\childdocname\endinput\||fi\newif\ifchilddoc|\\
|\edef\childdocname{\scantokens\expandafter{\jobname\noexpand}}|\\
|\def\childdocmain{|\textit{main}|}\||ifx\childdocmain\childdocname\||else|\\
|\childdoctrue\includeonly{\childdocname}\let\jobname\childdocmain\||fi|\\
\end{tabular}
\end{center}
%
Instead of |\childdocof{|\textit{main}|}| just include the main file
at the top of each child file:
%
\begin{center}
|\input{|\textit{main}|}|
\end{center}
%
A simple redirection |\childdocforward{|\textit{dest}|}| is achieved by:
%
\begin{center}
|\def\jobname{|\textit{dest}|}\input{\jobname}|
\end{center}
%
The redirection with prefix
|\childdocforwardprefix[|\textit{prefix}|]{|\textit{dest}|}|
is accomplished by:
%
\begin{center}
\begin{tabular}{l}
|{\edef\jobname{\scantokens\expandafter{\jobname\noexpand}}|\\
|\def\redirectjob |\textit{prefix}|#1~~~{\gdef\jobname{|\textit{dest}|#1}}|\\
|\expandafter\redirectjob\jobname~~~}\input{\jobname}|
\end{tabular}
\end{center}

In an alternative approach,
child documents can be compiled by a specific command line
without additional code or specific definitions:
%
\begin{center}
|... -jobname "|\textit{target}|" "|[\textit{flags}]%
|\includeonly{|\textit{dest}|}\input{|\textit{main}|}"|
\end{center}
%

%%%%%%%%%%%%%%%%%%%%%%%%%%%%%%%%%%%%%%%%%%%%%%%%%%%%%%%%%%%%%%%%%%%%%%%%%%%%%%%%
%%%%%%%%%%%%%%%%%%%%%%%%%%%%%%%%%%%%%%%%%%%%%%%%%%%%%%%%%%%%%%%%%%%%%%%%%%%%%%%%
\section{Information}

%%%%%%%%%%%%%%%%%%%%%%%%%%%%%%%%%%%%%%%%%%%%%%%%%%%%%%%%%%%%%%%%%%%%%%%%%%%%%%%%
\subsection{Copyright}

Copyright \copyright{} 2017--2018 Niklas Beisert

This work may be distributed and/or modified under the
conditions of the \LaTeX{} Project Public License, either version 1.3
of this license or (at your option) any later version.
The latest version of this license is in
  \url{http://www.latex-project.org/lppl.txt}
and version 1.3 or later is part of all distributions of \LaTeX{}
version 2005/12/01 or later.

This work has the LPPL maintenance status `maintained'.

The Current Maintainer of this work is Niklas Beisert.

This work consists of the files |README.txt|, |childdoc.ins| and |childdoc.dtx|
as well as the derived files |childdoc.def|, |cdocsamp.tex|
with |cdocsch1.tex|, |cdocsch2.tex|, |cdocspt3.tex|, |cdocspt4.tex|,
|cdocsdrf.tex|, |cdocsfn1.tex|, |cdocsfn2.tex|
as well as |childdoc.pdf|.

%%%%%%%%%%%%%%%%%%%%%%%%%%%%%%%%%%%%%%%%%%%%%%%%%%%%%%%%%%%%%%%%%%%%%%%%%%%%%%%%
\subsection{Files and Installation}

The package consists of the files:
%
\begin{center}
\begin{tabular}{ll}
    |README.txt|   & readme file \\
    |childdoc.ins| & installation file \\
    |childdoc.dtx| & source file \\
    |childdoc.def| & definition file \\
    |cdocsamp.tex| & sample main file \\
    |cdocsch1.tex| & sample include file \\
    |cdocsch2.tex| & sample include file \\
    |cdocspt3.tex| & sample part file \\
    |cdocspt4.tex| & sample part file \\
    |cdocsdrf.tex| & sample redirection file \\
    |cdocsfn1.tex| & sample redirection file \\
    |cdocsfn2.tex| & sample redirection file \\
    |childdoc.pdf| & manual
\end{tabular}
\end{center}
%
The distribution consists of the files
|README.txt|, |childdoc.ins| and |childdoc.dtx|.
%
\begin{itemize}
\item
Run (pdf)\LaTeX{} on |childdoc.dtx|
to compile the manual |childdoc.pdf| (this file).
\item
Run \LaTeX{} on |childdoc.ins| to create the definitions file |childdoc.def|
and the sample |cdocsamp.tex| with include files
|cdocsch1.tex|, |cdocsch2.tex|, |cdocspt3.tex|, |cdocspt4.tex|,
|cdocsdrf.tex|, |cdocsfn1.tex|, |cdocsfn2.tex|.
Then copy the file |childdoc.def| to an appropriate directory of your \LaTeX{}
distribution, e.g.\ \textit{texmf-root}|/tex/latex/childdoc|.
\end{itemize}

%%%%%%%%%%%%%%%%%%%%%%%%%%%%%%%%%%%%%%%%%%%%%%%%%%%%%%%%%%%%%%%%%%%%%%%%%%%%%%%%
\subsection{Related CTAN Packages}

There are several other packages which offer a similar functionality:
%
\begin{itemize}
\item
The packages
\href{http://ctan.org/pkg/docmute}{\textsf{docmute}},
\href{http://ctan.org/pkg/includex}{\textsf{includex}} and
\href{http://ctan.org/pkg/standalone}{\textsf{standalone}}
provide commands to include only the document body of
a child file thus allowing both files to be compiled individually.
\item
The packages \href{http://ctan.org/pkg/subdocs}{\textsf{subdocs}}
and \href{http://ctan.org/pkg/subfiles}{\textsf{subfiles}}
provide structures in which the main and child documents can be
encapsulated and allowing them to be compiled individually.
The inclusion mechanism is different from the conventional |\include|.
\item
The package \href{http://ctan.org/pkg/combine}{\textsf{combine}}
is an elaborate solution to combine several documents into one.
\end{itemize}
%
See also the CTAN topic \href{http://ctan.org/topic/subdocs}{\textsf{subdocs}}
for further related packages.
The present package differs from the above solutions in that
a document structure constructed with the conventional |\include| mechanism
just needs two extra commands at the top of every file
such that all constituent files can be compiled individually.

%%%%%%%%%%%%%%%%%%%%%%%%%%%%%%%%%%%%%%%%%%%%%%%%%%%%%%%%%%%%%%%%%%%%%%%%%%%%%%%%
%\subsection{Feature Suggestions}
%
%The following is a list of features which may be useful for future
%versions of this package:
%%
%\begin{itemize}
%\item
%\ldots
%\end{itemize}

%%%%%%%%%%%%%%%%%%%%%%%%%%%%%%%%%%%%%%%%%%%%%%%%%%%%%%%%%%%%%%%%%%%%%%%%%%%%%%%%
\subsection{Revision History}

%%%%%%%%%%%%%%%%%%%%%%%%%%%%%%%%%%%%%%%%
\paragraph{v2.0:} 2018/12/30

\begin{itemize}
\item
immediate forward processing
\item
added |\childdocby| mechanism
\item
manual restructured
\end{itemize}

%%%%%%%%%%%%%%%%%%%%%%%%%%%%%%%%%%%%%%%%
\paragraph{v1.6:} 2018/01/17

\begin{itemize}
\item
application for development of include files
\item
corrections to manual
\end{itemize}

%%%%%%%%%%%%%%%%%%%%%%%%%%%%%%%%%%%%%%%%
\paragraph{v1.5:} 2017/05/21

\begin{itemize}
\item
more complete structuring introduced
\item
|\childdocof| introduced
\item
|\childdoc| renamed to |\childdocmain|
\item
|\childredirect| renamed to |\childdocforward| and |\childdocforwardprefix|
and functionality expanded
\end{itemize}

%%%%%%%%%%%%%%%%%%%%%%%%%%%%%%%%%%%%%%%%
\paragraph{v1.0:} 2017/04/27

\begin{itemize}
\item
manual and install package
\item
first version published on CTAN
\end{itemize}

%%%%%%%%%%%%%%%%%%%%%%%%%%%%%%%%%%%%%%%%
\paragraph{v0.6:} 2017/04/26

\begin{itemize}
\item
redirection mechanism added
\end{itemize}

%%%%%%%%%%%%%%%%%%%%%%%%%%%%%%%%%%%%%%%%
\paragraph{v0.5:} 2017/04/26

\begin{itemize}
\item
functionality in definition file
\end{itemize}


%%%%%%%%%%%%%%%%%%%%%%%%%%%%%%%%%%%%%%%%%%%%%%%%%%%%%%%%%%%%%%%%%%%%%%%%%%%%%%%%
%%%%%%%%%%%%%%%%%%%%%%%%%%%%%%%%%%%%%%%%%%%%%%%%%%%%%%%%%%%%%%%%%%%%%%%%%%%%%%%%
%%%%%%%%%%%%%%%%%%%%%%%%%%%%%%%%%%%%%%%%%%%%%%%%%%%%%%%%%%%%%%%%%%%%%%%%%%%%%%%%
\appendix

\settowidth\MacroIndent{\rmfamily\scriptsize 000\ }

 \DocInput{childdoc.dtx}

\end{document}
%</driver>
% \fi
%
% %%%%%%%%%%%%%%%%%%%%%%%%%%%%%%%%%%%%%%%%%%%%%%%%%%%%%%%%%%%%%%%%%%%%%%%%%%%%%%
% %%%%%%%%%%%%%%%%%%%%%%%%%%%%%%%%%%%%%%%%%%%%%%%%%%%%%%%%%%%%%%%%%%%%%%%%%%%%%%
% \section{Sample}
%\iffalse
%<*samplemain>
%\fi
%
% The following presents a sample document
% with two chapters, two parts, a title page,
% a compile flag as well as three forwarding files to set the flag.
% It consists of eight |.tex| files:
% \begin{center}
% \begin{tabular}{ll}
% |cdocsamp.tex|&main file\\
% |cdocsch1.tex|&include file for chapter 1\\
% |cdocsch2.tex|&include file for chapter 2\\
% |cdocspt3.tex|&include file for part 3\\
% |cdocspt4.tex|&include file for part 4\\
% |cdocsdrf.tex|&forwarding file for main file in draft mode\\
% |cdocsfi1.tex|&forwarding file for final version of chapter 1\\
% |cdocsfi2.tex|&forwarding file for final version of chapter 2\\
% \end{tabular}
% \end{center}
% Each of the eight files can be compiled directly by the \LaTeX{} compiler.
%
% %%%%%%%%%%%%%%%%%%%%%%%%%%%%%%%%%%%%%%
% \paragraph{Main File.}
%
% The main file is called |cdocsamp.tex|.
%
% Load the \textsf{childdoc} definitions and
% declare the filename for the main document:
%    \begin{macrocode}
\input{childdoc.def}
\childdocmain{}
%    \end{macrocode}

% Optional override for |\version| flag:
%    \begin{macrocode}
%%\ifchilddoc\else\providecommand{\version}{draft}\fi
%    \end{macrocode}

% Define the default values for the |\version| flag
% (|final| for the main file and |draft| for childs):
%    \begin{macrocode}
\ifchilddoc
\providecommand{\version}{draft}
\else
\providecommand{\version}{final}
\fi
%    \end{macrocode}

% Load the standard document class:
%    \begin{macrocode}
\documentclass[12pt]{article}
%    \end{macrocode}

% Start the document body:
%    \begin{macrocode}
\begin{document}
%    \end{macrocode}

% Declare a title page.
% Print title, part of document being processed and version flag:
%    \begin{macrocode}
\addtocounter{page}{-1}
\begin{center}
{\LARGE\bfseries{}childdoc example\par}
\vspace{1cm}
\ifchilddoc
\ifchilddocmanual part\else chapter\fi:
`\childdocname' of `\childdocjob'\par
\else
main document: `\childdocjob'\par
\fi
version: \version\par
\end{center}
\newpage
%    \end{macrocode}

% Manually include selected file,
% otherwise process as usual:
%    \begin{macrocode}
\ifchilddocmanual
\section*{part `\childdocname'}
\input{\childdocname}
\else
%    \end{macrocode}

% Include the two chapters:
%    \begin{macrocode}
\include{cdocsch1}
\include{cdocsch2}
%    \end{macrocode}

% Include the two parts unless only chapters should be displayed:
%    \begin{macrocode}
\ifchilddoc\else
\section{part three}
\input{cdocspt3}
\section{part four}
\input{cdocspt4}
\fi
%    \end{macrocode}

% Process as usual until here:
%    \begin{macrocode}
\fi
%    \end{macrocode}

% End of document body:
%    \begin{macrocode}
\end{document}
%    \end{macrocode}
%\iffalse
%</samplemain>
%\fi
%
% %%%%%%%%%%%%%%%%%%%%%%%%%%%%%%%%%%%%%%
% \paragraph{Chapter Include Files.}
%
% The include files are called |cdocsch1.tex| and |cdocsch2.tex|.
%
%\iffalse
%<*samplechap1|samplechap2>
%\fi

% Optional override for |\version| flag:
%    \begin{macrocode}
%%\providecommand{\version}{final}
%    \end{macrocode}

% Include the main document:
%    \begin{macrocode}
\input{childdoc.def}
\childdocof{cdocsamp}
%    \end{macrocode}

%\iffalse
%</samplechap1|samplechap2>
%\fi
%
%\iffalse
%<*samplechap1>
%\fi
% Some text for chapter 1:
%    \begin{macrocode}
\section{one}
some text in chapter one
%    \end{macrocode}

%\iffalse
%</samplechap1>
%\fi
% Some text for chapter 2:
%\iffalse
%<*samplechap2>
%\fi
%    \begin{macrocode}
\section{two}
more text in chapter two
%    \end{macrocode}

%\iffalse
%</samplechap2>
%\fi
%
% %%%%%%%%%%%%%%%%%%%%%%%%%%%%%%%%%%%%%%
% \paragraph{Part Include Files.}
%
% The include files are called |cdocspt3.tex| and |cdocspt4.tex|.
%
%\iffalse
%<*samplepart3|samplepart4>
%\fi

% Optional override for |\version| flag:
%    \begin{macrocode}
%%\providecommand{\version}{final}
%    \end{macrocode}

% Include the main document:
%    \begin{macrocode}
\input{childdoc.def}
\childdocby{cdocsamp}
%    \end{macrocode}

%\iffalse
%</samplepart3|samplepart4>
%\fi
%
%\iffalse
%<*samplepart3>
%\fi
% Some text for part 3:
%    \begin{macrocode}
some text in part three
%    \end{macrocode}

%\iffalse
%</samplepart3>
%\fi
% Some text for part 4:
%\iffalse
%<*samplepart4>
%\fi
%    \begin{macrocode}
more text in part four
%    \end{macrocode}

%\iffalse
%</samplepart4>
%\fi
%
% %%%%%%%%%%%%%%%%%%%%%%%%%%%%%%%%%%%%%%
% \paragraph{Forwarding for a Complete Draft.}
%
% The following forwarding file |cdocsdrf.tex|
% compiles the main document in draft mode:
%\iffalse
%<*sampledraft>
%\fi
%    \begin{macrocode}
\def\version{draft}
\input{childdoc.def}
\childdocforward{cdocsamp}
%    \end{macrocode}

%\iffalse
%</sampledraft>
%\fi
%
% %%%%%%%%%%%%%%%%%%%%%%%%%%%%%%%%%%%%%%
% \paragraph{Forwarding for Final Version of the Chapters.}
%
% The following forwarding files |cdocsfn1.tex| and |cdocsfn2.tex|
% (with identical content)
% compile the final versions of the child documents
% |cdocsch1.tex| and |cdocsch2.tex|, respectively:
%\iffalse
%<*samplefinal>
%\fi
%    \begin{macrocode}
\def\version{final}
\input{childdoc.def}
\childdocforwardprefix[cdocsamp]{cdocsfn}{cdocsch}
%    \end{macrocode}

%\iffalse
%</samplefinal>
%\fi
%
% %%%%%%%%%%%%%%%%%%%%%%%%%%%%%%%%%%%%%%
% \paragraph{Command Line Processing.}
%
% The following three command lines generate the output files
% |cdocscld|, |cdocscl1| and |cdocscl2|
% which should be identical to
% |cdocsdrf|, |cdocsch1| and |cdocsfn2|, respectively:
% \begin{center}
% \begin{tabular}{l}
% |latex -jobname cdocscld \|\\
% |  "\def\version{draft}\input{childdoc.def}\childdocforward{cdocsamp}"|\\
% |latex -jobname cdocscl1 \|\\
% |  "\input{childdoc.def}\childdocforward[cdocsamp]{cdocsch1}"|\\
% |latex -jobname cdocscl2 \|\\
% |  "\def\version{final}\input{childdoc.def}\childdocforward{cdocsch2}"|
% \end{tabular}
% \end{center}
% Note that the trailing backslash on each first line
% merely continues the input to the second line
% (for convenient cut ant paste).
% Furthermore, the command |latex| can be replaced by any
% of its alternative versions such as |pdflatex|.
%
% %%%%%%%%%%%%%%%%%%%%%%%%%%%%%%%%%%%%%%%%%%%%%%%%%%%%%%%%%%%%%%%%%%%%%%%%%%%%%%
% %%%%%%%%%%%%%%%%%%%%%%%%%%%%%%%%%%%%%%%%%%%%%%%%%%%%%%%%%%%%%%%%%%%%%%%%%%%%%%
% \section{Implementation}
%\iffalse
%<*package>
%\fi
%
% This section describes the definitions file |childdoc.def|.

% The definitions cannot be loaded using |\usepackage| or |\RequirePackage|
% which has a mechanism to prevent loading a style file more than once.
% When loading the definitions by means of |\input|
% multiple instances have to be prevented manually:
%\iffalse
%This code needs to be before the `\ProvidesFile' directive
%which is defined at the beginning of this file.
%Therefore it is also placed there and commented out here.
%</package>
%<*discard>
%\fi
%    \begin{macrocode}
\ifdefined\childdocmain\endinput\fi
%    \end{macrocode}
%\iffalse
%</discard>
%<*package>
%\fi
%
% \macro{\ifchilddoc}
% \macro{\ifchilddocmanual}
% The conditional |\ifchilddoc| tells whether a
% child (true) or main (false) document is being compiled.
% The conditional |\ifchilddocmanual| tells whether
% the |\includeonly| mechanism is used (false) or
% the selection of child files must be performed manually (true).
% The definitions initialise to false:
%    \begin{macrocode}
\newif\ifchilddoc
\newif\ifchilddocmanual
%    \end{macrocode}

% \macro{\childdocname}
% \macro{\childdocjob}
% The macro |\childdocname| stores the name of the main document
% to be compiled. The macro |\childdocjob| stores the name of
% the document on which the \LaTeX{} compiler was originally invoked.
% The content of |\jobname| cannot be compared
% to filenames specified in the source due to different catcodes.
% The following code rescans |\jobname|, stores the result
% in |\childdocname| and saves a copy in |\childdocjob|:
%    \begin{macrocode}
\edef\childdocname{\scantokens\expandafter{\jobname\noexpand}}
\let\childdocjob\childdocname
%    \end{macrocode}

% \macro{\childdocdisable}
% The macro |\childdocdisable| prevents the main file
% from being processed more than once.
% At this stage, the main document command |\childdocmain|
% is assumed to be called once again where it should do nothing.
% Any subsequent call to it should prevent
% a secondary processing of the main document
% It overwrites the forwarding commands
% |\childdocof| and |\childdocforward|
% with empty macros to prevent further inclusions of the main document:
%    \begin{macrocode}
\newcommand{\childdocdisable}
{
  \renewcommand{\childdocmain}[1]{\renewcommand{\childdocmain}[1]{\endinput}}
  \renewcommand{\childdocof}[1]{}
  \renewcommand{\childdocby}[2][]{}
  \renewcommand{\childdocforward}[2][]{}
  \renewcommand{\childdocdisable}{}
}
%    \end{macrocode}

% \macro{\childdocmain}
% The macro |\childdocmain| is to be called at the top of the main file
% with nothing or the main filename (without extension) as argument.
% First, it breaks loops.
% If the argument is not empty and does not match |\childdocname|
% (which is set by the first inclusion of |childdoc.def|),
% |\ifchilddoc| is set to true, |\includeonly| is applied to the child file
% and |\jobname| is set to the main file
% (for proper handling of |.aux| files):
%    \begin{macrocode}
\newcommand{\childdocmain}[1]
{
  \childdocdisable\childdocmain{}
  \if?#1?\else
    \begingroup
      \def\childdoctmp{#1}
      \ifx\childdoctmp\childdocname
        \def\childdoctmp{}
      \else
        \def\childdoctmp
        {
          \childdoctrue
          \includeonly{\childdocname}
          \def\childdocjob{#1}
          \def\jobname{#1}
        }
      \fi
      \expandafter
    \endgroup
    \childdoctmp
  \fi
}
%    \end{macrocode}

% \macro{\childdocof}
% The command |\childdocof| redirects
% compilation to the main file |#1|.
%    \begin{macrocode}
\newcommand{\childdocof}[1]
{
  \childdocdisable
  \childdoctrue
  \includeonly{\childdocname}
  \def\jobname{#1}
  \def\childdocjob{#1}
  \input{#1}
}
%    \end{macrocode}

% \macro{\childdocby}
% The command |\childdocby| ....
%    \begin{macrocode}
\newcommand{\childdocby}[2][]
{
  \childdocdisable
  \childdoctrue
  \childdocmanualtrue
  \if?#1?\else
    \def\jobname{#2}
  \fi
  \def\childdocjob{#2}
  \input{#2}
  \endinput
}
%    \end{macrocode}

% \macro{\childdocforward}
% The command |\childdocforward| redirects
% compilation to the main file or
% (if the optional argument is given) a child file.
% Parameters are set as if the main file
% or a child file starting with |\childdocof| was compiled.
% Then compilation is handed over to the main file:
%    \begin{macrocode}
\newcommand{\childdocforward}[2][]
{
  \begingroup
    \if?#1?
      \def\childdoctmp
      {
        \def\childdocname{#2}
        \def\childdocjob{#2}
        \def\jobname{#2}
        \input{#2}
        \endinput
      }
    \else
      \def\childdoctmp
      {
        \childdocdisable
        \def\childdocname{#2}
        \childdoctrue
        \includeonly{#2}
        \def\childdocjob{#1}
        \def\jobname{#1}
        \input{#1}
        \endinput
      }
    \fi
    \expandafter
  \endgroup
  \childdoctmp
}
%    \end{macrocode}

% \macro{\childdocforwardprefix}
% The command |\childdocforwardprefix| redirects
% compilation to the main or a child file by means of a pattern.
% The prefix |#1| in the current filename is replaced by |#2|
% and the suffix of the current filename is kept
% (it is assumed that the filename does not contain the substring `|~~~|'
% which is used as a delimiter).
% Compilation is handed over to the new file by |\childdocforward|:
%    \begin{macrocode}
\newcommand{\childdocforwardprefix}[3][]
{
  \begingroup
    \def\childdocextract #2##1~~~{\def\childdoctmp{\childdocforward[#1]{#3##1}}}
    \expandafter\childdocextract\childdocname~~~
    \expandafter
  \endgroup
  \childdoctmp
}
%    \end{macrocode}

% \macro{\childdoc}
% The deprecated macro |\childdoc| is a legacy version of |\childdocmain|:
%    \begin{macrocode}
\newcommand{\childdoc}{\childdocmain}
%    \end{macrocode}

% \macro{\childdocredirect}
% The deprecated macro |\childdocredirect| is a legacy version
% of |\childdocforward| and |\childdocforwardprefix|:
%    \begin{macrocode}
\newcommand{\childdocredirect}[2][]
{
  \begingroup
    \if?#1?
      \def\childdoctmp{\childdocforward{#2}}
    \else
      \def\childdoctmp{\childdocforwardprefix{#1}{#2}}
    \fi
    \expandafter
  \endgroup
  \childdoctmp
}
%    \end{macrocode}

%\iffalse
%</package>
%\fi
%
\endinput
|\\
|\childdocof{|\textit{main}|}|\\
\end{tabular}
\end{center}
at the top of every child file \textit{child}
which is included by |\include{|\textit{child}|}|
from within the main file
(or at least for those files to be compiled individually).
The argument \textit{main} must be the filename of the main file.

There are a couple of
considerations in setting up the main and child documents:

%%%%%%%%%%%%%%%%%%%%%%%%%%%%%%%%%%%%%%%%
\paragraph{Restrictions.}

Please note the following restrictions:
\begin{itemize}
\item
|\childdocmain| must be called with one argument \textit{main}
to ensure compatibility with earlier version of the package.
It must either be empty (|\childdocmain{}|)
or precisely match the filename of the main file in which it is specified.
See \secref{sec:detection} for further information.
\item
The filename \textit{main} must be specified without the |.tex| extension.
\item
The filename \textit{main} is case sensitive
(even in case-insensitive file systems)
due to internal string comparison.
\item
The argument \textit{main} should be fully expanded, it cannot be a macro.
\item
Subdirectories and special characters should be avoided in filenames.
\item
The command |\childdocmain{|\textit{main}|}| must be followed by a whitespace.
It should not be followed immediately by another command
or by a comment mark `|%|'.
This is because the \TeX{} parser reads the token immediately following
the argument of |\childdocmain| and puts it
at the beginning of every child section;
however, a white\-space is ignored.
\end{itemize}

%%%%%%%%%%%%%%%%%%%%%%%%%%%%%%%%%%%%%%%%
\paragraph{Content of Main File.}

It is advisable to place all content in the child files included by |\include|.
Any output contained in the main file will appear in all child documents
unless suppressed manually;
it cannot be suppressed automatically by the |\includeonly| directive
and thus should normally be avoided.
A method to include some content in the main file
by means of conditional processing is described in \secref{sec:conditional}.

%%%%%%%%%%%%%%%%%%%%%%%%%%%%%%%%%%%%%%%%
\paragraph{Page Numbering.}

When only a part of the document is compiled,
the appropriate numbering of pages
(as well as other status parameters)
is determined from the |.aux| files.
The latter contain information from previous passes.
However this information needs to propagate through
all intermediate child documents.
Therefore the page numbering in child documents may well
be inconsistent until the complete document is compiled at least once.

A useful (if unconventional) way to always ensure a consistent
page numbering is to restart the numbering in each child document
and denote the pages by `\textit{child}|.|\textit{page}'
where \textit{child} represents the chapter/section number of the child file.
This can be achieved by the command
|\numberwithin{page}{|\textit{child}|}|
of the \textsf{amsmath} package
where \textit{child} can be |chapter| or |section|
depending on the chosen structuring.
Alternatively, one can modify the macro |\thepage| appropriately
and reset the counter |page| at the start of each child file.

%%%%%%%%%%%%%%%%%%%%%%%%%%%%%%%%%%%%%%%%%%%%%%%%%%%%%%%%%%%%%%%%%%%%%%%%%%%%%%%%
\subsection{Conditional Processing}
\label{sec:conditional}

The package provides a mechanism to compile different versions
of a document. To customise the versions further some conditional processing
can come in handy to distinguish which version is being compiled.
The package provides two macros to describe the compilation context:

%%%%%%%%%%%%%%%%%%%%%%%%%%%%%%%%%%%%%%%%
\DescribeMacro{\ifchilddoc}
The conditional |\ifchilddoc| distinguishes between the compilation of
child documents and the main document:
%
\begin{center}
|\ifchilddoc |\textit{child-code}| |[|\||else |\textit{main-code}]| \||fi|
\end{center}

%%%%%%%%%%%%%%%%%%%%%%%%%%%%%%%%%%%%%%%%
\DescribeMacro{\childdocname}
\DescribeMacro{\childdocjob}
The macro |\childdocname| contains the filename (without extension)
of the main or child file being processed.
Note that |\childdocjob| will always contain the name of the main file.

%%%%%%%%%%%%%%%%%%%%%%%%%%%%%%%%%%%%%%%%
\paragraph{Title Page.}

Conditional processing can be used to include a title or banner page
in the main document when proper precautions are taken.
Importantly, the code in the main file should ensure that the page counter
(as well as other status parameters which are stored in the |.aux| files)
takes the same value after the conditional processing.
Otherwise the page numbers may take divergent values
depending on which part is compiled.

For example, a title page could be declared by:
%
\begin{center}
\begin{tabular}{l}
|\ifchilddoc\||else|\\
|\addtocounter{page}{-1}|\\
\textit{code for title page}\\
|\newpage|\\
|\||fi|
\end{tabular}
\end{center}
%
A banner page for the child documents can be generated by:
%
\begin{center}
\begin{tabular}{l}
|\ifchilddoc|\\
|\addtocounter{page}{-1}|\\
\textit{code for banner page}\\
|\newpage|\\
|\||fi|
\end{tabular}
\end{center}
%
Here one could write a message such as:
\begin{center}
|This is the part \childdocname{} of \childdocjob{}.|
\end{center}

%%%%%%%%%%%%%%%%%%%%%%%%%%%%%%%%%%%%%%%%%%%%%%%%%%%%%%%%%%%%%%%%%%%%%%%%%%%%%%%%
\subsection{Flags}
\label{sec:flags}

The package makes it easy to generate different versions
of the main or child documents.
To this end compilation flags can be defined
and assigned different default values.
They will be particularly useful in conjunction
with the forwarding mechanism described in \secref{sec:forward}.

For example, it may be useful to have a flag |\version|
which can be set to |draft| or |final|.
The document source will contain some conditional code
depending on the value of |\version|.
Suppose further, the flag should default to |final| for the main file
and to |draft| for child files
which is a natural assignment for editing the document.
This is achieved by placing the following code
in the preamble of the main document
(below the |\childdocmain| directive):
%
\begin{center}
\begin{tabular}{l}
|\ifchilddoc|\\
|\providecommand{\version}{draft}|\\
|\||else|\\
|\providecommand{\version}{final}|\\
|\||fi|
\end{tabular}
\end{center}
%
The definition by |\providecommand| makes sure
that previous definitions are not overwritten.
Further statements |\providecommand{\version}{...}|
can thus be added before the above code to override it.

For the main file, one might add a line
(between |\childdocmain| and the above block)
%
\begin{center}
|%\ifchilddoc\||else\providecommand{\version}{draft}\||fi|
\end{center}
%
which can be uncommented to produce a draft version.
Likewise one can add a line to the very top of a child file
(above the |\childdocof{|\textit{main}|}| directive)
%
\begin{center}
|%\providecommand{\version}{final}|
\end{center}
%
which can be uncommented to produce the final version of this child document.

%%%%%%%%%%%%%%%%%%%%%%%%%%%%%%%%%%%%%%%%%%%%%%%%%%%%%%%%%%%%%%%%%%%%%%%%%%%%%%%%
\subsection{Forwarding}
\label{sec:forward}

Different versions of the main or child documents
using compilation flags as described in \secref{sec:flags}
can be (permanently) stored in different files
for convenient compilation, viewing and distribution.
To this end, the package defines a command
to pass on compilation to a different file:

%%%%%%%%%%%%%%%%%%%%%%%%%%%%%%%%%%%%%%%%
\DescribeMacro{\childdocforward}
The command |\childdocforward| redirects processing to
another source file:
%
\begin{center}
\begin{tabular}{l}
|% \iffalse
%
% childdoc.dtx Copyright (C) 2017-2018 Niklas Beisert
%
% This work may be distributed and/or modified under the
% conditions of the LaTeX Project Public License, either version 1.3
% of this license or (at your option) any later version.
% The latest version of this license is in
%   http://www.latex-project.org/lppl.txt
% and version 1.3 or later is part of all distributions of LaTeX
% version 2005/12/01 or later.
%
% This work has the LPPL maintenance status `maintained'.
%
% The Current Maintainer of this work is Niklas Beisert.
%
% This work consists of the files childdoc.dtx and childdoc.ins
% and the derived files childdoc.def and cdocsamp.tex with
% cdocsch1.tex, cdocsch2.tex, cdocsdrf.tex, cdocsfn1.tex, cdocsfn2.tex.
%
%<package>\ifdefined\childdocmain\endinput\fi
%<package>\ProvidesFile{childdoc.def}[2018/12/30 v2.0 child document driver]
%<samplemain>\ProvidesFile{cdocsamp.tex}[2018/12/30 v2.0 sample for childdoc]
%<*driver>
%\ProvidesFile{childdoc.drv}[2018/12/30 v2.0 childdoc reference manual file]
\PassOptionsToClass{10pt,a4paper}{article}
\documentclass{ltxdoc}

\usepackage[margin=35mm]{geometry}
\usepackage{hyperref}
\usepackage{hyperxmp}
\usepackage[usenames]{color}

\hypersetup{colorlinks=true}
\hypersetup{pdfstartview=FitH}
\hypersetup{pdfpagemode=UseNone}
\hypersetup{pdfsource={}}
\hypersetup{pdflang={en-UK}}
\hypersetup{pdfcopyright={Copyright 2017-2018 Niklas Beisert.
  This work may be distributed and/or modified under the
  conditions of the LaTeX Project Public License, either version 1.3
  of this license or (at your option) any later version.}}
\hypersetup{pdflicenseurl={http://www.latex-project.org/lppl.txt}}
\hypersetup{pdfcontactaddress={ETH Zurich, ITP, HIT K,
  Wolfgang-Pauli-Strasse 27}}
\hypersetup{pdfcontactpostcode={8093}}
\hypersetup{pdfcontactcity={Zurich}}
\hypersetup{pdfcontactcountry={Switzerland}}
\hypersetup{pdfcontactemail={nbeisert@itp.phys.ethz.ch}}
\hypersetup{pdfcontacturl={http://people.phys.ethz.ch/\xmptilde nbeisert/}}

\newcommand{\secref}[1]{\hyperref[#1]{section \ref*{#1}}}

\parskip1ex
\parindent0pt
\let\olditemize\itemize
\def\itemize{\olditemize\parskip0pt}

\begin{document}

\title{The \textsf{childdoc} Package}
\hypersetup{pdftitle={The childdoc Package}}
\author{Niklas Beisert\\[2ex]
  Institut f\"ur Theoretische Physik\\
  Eidgen\"ossische Technische Hochschule Z\"urich\\
  Wolfgang-Pauli-Strasse 27, 8093 Z\"urich, Switzerland\\[1ex]
  \href{mailto:nbeisert@itp.phys.ethz.ch}
  {\texttt{nbeisert@itp.phys.ethz.ch}}}
\hypersetup{pdfauthor={Niklas Beisert}}
\hypersetup{pdfsubject={Manual for the LaTeX2e Package childdoc}}
\date{30 December 2018, \textsf{v2.0}}
\maketitle

\begin{abstract}\noindent
\textsf{childdoc} is a \LaTeXe{} package
that enables the direct compilation
of document sections included by |\include|
to individual files.
\end{abstract}

\begingroup
\parskip0ex
\tableofcontents
\endgroup

%%%%%%%%%%%%%%%%%%%%%%%%%%%%%%%%%%%%%%%%%%%%%%%%%%%%%%%%%%%%%%%%%%%%%%%%%%%%%%%%
%%%%%%%%%%%%%%%%%%%%%%%%%%%%%%%%%%%%%%%%%%%%%%%%%%%%%%%%%%%%%%%%%%%%%%%%%%%%%%%%
\section{Introduction}

\LaTeX{} provides a mechanism to structure a large document (such as a book)
into a main file and several child files (containing the chapters)
using the |\include| command.
This mechanism is beneficial for documents
which span hundreds of pages in order to
make the source file(s) more manageable.
Moreover, compilation can be restricted to
selected child files by means of the |\includeonly| command.
The latter feature can be used to reduce the compilation time while editing
(this was significantly more useful in the earlier days of \LaTeX{})
or to generate a smaller document which is easier to navigate.
Another application of |\includeonly| is to generate
documents consisting of selected parts of the complete document.

However, there are a few drawbacks of the plain |\include| mechanism:
\begin{itemize}
\item
The child files cannot be compiled on their own,
they can only be compiled via the main file.
A naive editing environment
(such as a text editor with an option
to have the current file processed by \LaTeX)
may require one to switch to the main file before compiling;
attempting to compile the child file produces errors.
\item
The main file must be modified (each time)
to adjust the |\includeonly| command
to the present needs. This easily leaves the main file in a messy state.
\item
The generated document will always carry the filename
of the main document. This is inconvenient if
several child files are to be compiled and
to be kept for distribution.
\end{itemize}

The present package provides a simple interface
to make child files individually compilable by \LaTeX{}.
Compiling a child file then has the same effect as compiling
the main file with an |\includeonly| command
to select the appropriate child.
Moreover the generated document will carry the name of the child
rather than the main file.
This resolves all three above issues.

This feature is meant to make the editing of books,
thesis documents and lecture notes somewhat more convenient.
However, the package can also be used efficiently for
composing a series of documents (such as exercise sheets)
which are typically distributed individually.
It then assists the author in generating the individual documents
(potentially in different versions)
as well as a document containing the collected series.
Another application is in developing style files
or other kinds of included material
where compilation of the style file could redirect
to a sample or test file.

%%%%%%%%%%%%%%%%%%%%%%%%%%%%%%%%%%%%%%%%%%%%%%%%%%%%%%%%%%%%%%%%%%%%%%%%%%%%%%%%
%%%%%%%%%%%%%%%%%%%%%%%%%%%%%%%%%%%%%%%%%%%%%%%%%%%%%%%%%%%%%%%%%%%%%%%%%%%%%%%%
\section{Usage}

First of all, the package \textsf{childdoc} is \emph{not} a standard
\LaTeXe{} |.sty| style file! Therefore it needs to be invoked in
a non-standard way.

%%%%%%%%%%%%%%%%%%%%%%%%%%%%%%%%%%%%%%%%%%%%%%%%%%%%%%%%%%%%%%%%%%%%%%%%%%%%%%%%
\subsection{Included Files}
\label{sec:include}

%%%%%%%%%%%%%%%%%%%%%%%%%%%%%%%%%%%%%%%%
\DescribeMacro{\childdocmain}
To use the package, add the commands
\begin{center}
\begin{tabular}{l}
|\input{childdoc.def}|\\
|\childdocmain{}|\\
\end{tabular}
\end{center}
at the very top of the main \LaTeX{} file,
in particular \emph{before} the |\documentclass| statement!
The argument of |\childdocmain| should be left empty
(but it must be present).

%%%%%%%%%%%%%%%%%%%%%%%%%%%%%%%%%%%%%%%%
\DescribeMacro{\childdocof}
Furthermore, add the commands
\begin{center}
\begin{tabular}{l}
|\input{childdoc.def}|\\
|\childdocof{|\textit{main}|}|\\
\end{tabular}
\end{center}
at the top of every child file \textit{child}
which is included by |\include{|\textit{child}|}|
from within the main file
(or at least for those files to be compiled individually).
The argument \textit{main} must be the filename of the main file.

There are a couple of
considerations in setting up the main and child documents:

%%%%%%%%%%%%%%%%%%%%%%%%%%%%%%%%%%%%%%%%
\paragraph{Restrictions.}

Please note the following restrictions:
\begin{itemize}
\item
|\childdocmain| must be called with one argument \textit{main}
to ensure compatibility with earlier version of the package.
It must either be empty (|\childdocmain{}|)
or precisely match the filename of the main file in which it is specified.
See \secref{sec:detection} for further information.
\item
The filename \textit{main} must be specified without the |.tex| extension.
\item
The filename \textit{main} is case sensitive
(even in case-insensitive file systems)
due to internal string comparison.
\item
The argument \textit{main} should be fully expanded, it cannot be a macro.
\item
Subdirectories and special characters should be avoided in filenames.
\item
The command |\childdocmain{|\textit{main}|}| must be followed by a whitespace.
It should not be followed immediately by another command
or by a comment mark `|%|'.
This is because the \TeX{} parser reads the token immediately following
the argument of |\childdocmain| and puts it
at the beginning of every child section;
however, a white\-space is ignored.
\end{itemize}

%%%%%%%%%%%%%%%%%%%%%%%%%%%%%%%%%%%%%%%%
\paragraph{Content of Main File.}

It is advisable to place all content in the child files included by |\include|.
Any output contained in the main file will appear in all child documents
unless suppressed manually;
it cannot be suppressed automatically by the |\includeonly| directive
and thus should normally be avoided.
A method to include some content in the main file
by means of conditional processing is described in \secref{sec:conditional}.

%%%%%%%%%%%%%%%%%%%%%%%%%%%%%%%%%%%%%%%%
\paragraph{Page Numbering.}

When only a part of the document is compiled,
the appropriate numbering of pages
(as well as other status parameters)
is determined from the |.aux| files.
The latter contain information from previous passes.
However this information needs to propagate through
all intermediate child documents.
Therefore the page numbering in child documents may well
be inconsistent until the complete document is compiled at least once.

A useful (if unconventional) way to always ensure a consistent
page numbering is to restart the numbering in each child document
and denote the pages by `\textit{child}|.|\textit{page}'
where \textit{child} represents the chapter/section number of the child file.
This can be achieved by the command
|\numberwithin{page}{|\textit{child}|}|
of the \textsf{amsmath} package
where \textit{child} can be |chapter| or |section|
depending on the chosen structuring.
Alternatively, one can modify the macro |\thepage| appropriately
and reset the counter |page| at the start of each child file.

%%%%%%%%%%%%%%%%%%%%%%%%%%%%%%%%%%%%%%%%%%%%%%%%%%%%%%%%%%%%%%%%%%%%%%%%%%%%%%%%
\subsection{Conditional Processing}
\label{sec:conditional}

The package provides a mechanism to compile different versions
of a document. To customise the versions further some conditional processing
can come in handy to distinguish which version is being compiled.
The package provides two macros to describe the compilation context:

%%%%%%%%%%%%%%%%%%%%%%%%%%%%%%%%%%%%%%%%
\DescribeMacro{\ifchilddoc}
The conditional |\ifchilddoc| distinguishes between the compilation of
child documents and the main document:
%
\begin{center}
|\ifchilddoc |\textit{child-code}| |[|\||else |\textit{main-code}]| \||fi|
\end{center}

%%%%%%%%%%%%%%%%%%%%%%%%%%%%%%%%%%%%%%%%
\DescribeMacro{\childdocname}
\DescribeMacro{\childdocjob}
The macro |\childdocname| contains the filename (without extension)
of the main or child file being processed.
Note that |\childdocjob| will always contain the name of the main file.

%%%%%%%%%%%%%%%%%%%%%%%%%%%%%%%%%%%%%%%%
\paragraph{Title Page.}

Conditional processing can be used to include a title or banner page
in the main document when proper precautions are taken.
Importantly, the code in the main file should ensure that the page counter
(as well as other status parameters which are stored in the |.aux| files)
takes the same value after the conditional processing.
Otherwise the page numbers may take divergent values
depending on which part is compiled.

For example, a title page could be declared by:
%
\begin{center}
\begin{tabular}{l}
|\ifchilddoc\||else|\\
|\addtocounter{page}{-1}|\\
\textit{code for title page}\\
|\newpage|\\
|\||fi|
\end{tabular}
\end{center}
%
A banner page for the child documents can be generated by:
%
\begin{center}
\begin{tabular}{l}
|\ifchilddoc|\\
|\addtocounter{page}{-1}|\\
\textit{code for banner page}\\
|\newpage|\\
|\||fi|
\end{tabular}
\end{center}
%
Here one could write a message such as:
\begin{center}
|This is the part \childdocname{} of \childdocjob{}.|
\end{center}

%%%%%%%%%%%%%%%%%%%%%%%%%%%%%%%%%%%%%%%%%%%%%%%%%%%%%%%%%%%%%%%%%%%%%%%%%%%%%%%%
\subsection{Flags}
\label{sec:flags}

The package makes it easy to generate different versions
of the main or child documents.
To this end compilation flags can be defined
and assigned different default values.
They will be particularly useful in conjunction
with the forwarding mechanism described in \secref{sec:forward}.

For example, it may be useful to have a flag |\version|
which can be set to |draft| or |final|.
The document source will contain some conditional code
depending on the value of |\version|.
Suppose further, the flag should default to |final| for the main file
and to |draft| for child files
which is a natural assignment for editing the document.
This is achieved by placing the following code
in the preamble of the main document
(below the |\childdocmain| directive):
%
\begin{center}
\begin{tabular}{l}
|\ifchilddoc|\\
|\providecommand{\version}{draft}|\\
|\||else|\\
|\providecommand{\version}{final}|\\
|\||fi|
\end{tabular}
\end{center}
%
The definition by |\providecommand| makes sure
that previous definitions are not overwritten.
Further statements |\providecommand{\version}{...}|
can thus be added before the above code to override it.

For the main file, one might add a line
(between |\childdocmain| and the above block)
%
\begin{center}
|%\ifchilddoc\||else\providecommand{\version}{draft}\||fi|
\end{center}
%
which can be uncommented to produce a draft version.
Likewise one can add a line to the very top of a child file
(above the |\childdocof{|\textit{main}|}| directive)
%
\begin{center}
|%\providecommand{\version}{final}|
\end{center}
%
which can be uncommented to produce the final version of this child document.

%%%%%%%%%%%%%%%%%%%%%%%%%%%%%%%%%%%%%%%%%%%%%%%%%%%%%%%%%%%%%%%%%%%%%%%%%%%%%%%%
\subsection{Forwarding}
\label{sec:forward}

Different versions of the main or child documents
using compilation flags as described in \secref{sec:flags}
can be (permanently) stored in different files
for convenient compilation, viewing and distribution.
To this end, the package defines a command
to pass on compilation to a different file:

%%%%%%%%%%%%%%%%%%%%%%%%%%%%%%%%%%%%%%%%
\DescribeMacro{\childdocforward}
The command |\childdocforward| redirects processing to
another source file:
%
\begin{center}
\begin{tabular}{l}
|\input{childdoc.def}|\\
|\childdocforward[|\textit{main}|]{|\textit{dest}|}|\\
\end{tabular}
\end{center}
%
The argument \textit{dest} is the destination file
(without extension).
It should be the main file or one of the child files.
Note that further \textsf{childdoc} directives
such as |\childdocof| and |\childdocforward|
in the indicated file will be processed in this form.
The optional argument \textit{main}
passes on directly to the main file \textit{main}
while pretending to compile the child \textit{dest}.
This form behaves as if \textit{dest}
issues |\childdocof{|\textit{main}|}| right away,
and no further \textsf{childdoc} directives will be processed.

%%%%%%%%%%%%%%%%%%%%%%%%%%%%%%%%%%%%%%%%
\DescribeMacro{\...prefix}
In the alternative form |\childdocforwardprefix|,
%
\begin{center}
\begin{tabular}{l}
|\input{childdoc.def}|\\
|\childdocforwardprefix[|\textit{main}|]{|\textit{prefix}|}{|\textit{dest}|}|
\end{tabular}
\end{center}
%
the destination file is determined by a pattern
depending on the current file:
To make this work, the current file must be called
`{\textit{prefix}\hspace{0.2em}\textit{suffix}}'
with \textit{prefix} matching precisely the argument.
Processing is then passed on to the file
`{\textit{dest}\hspace{0.2em}\textit{suffix}}'.
Surely, the same effect is achieved by
directly specifying the
argument `{\textit{dest}\hspace{0.2em}\textit{suffix}}'
in the first form.
However, that requires to set up a different file
for each child. With the alternative form of the command
all these files can have exactly the same content
which simplifies setting them up and maintaining them.

For example, the following file |draft.tex|
with a compilation flag |\version| as described in \secref{sec:flags}
compiles the main document as a draft:
%
\begin{center}
\begin{tabular}{l}
|\def\version{draft}|\\
|\input{childdoc.def}|\\
|\childdocforward{|\textit{main}|}|
\end{tabular}
\end{center}
%
Likewise, the following files |final|\textit{nn}|.tex|
compile the final version of the child document
|child|\textit{nn}|.tex|:
%
\begin{center}
\begin{tabular}{l}
|\def\version{final}|\\
|\input{childdoc.def}|\\
|\childdocforwardprefix{final}{child}|
\end{tabular}
\end{center}
%

Note that when several versions of a main file and/or of each child file
are to be generated, it may be convenient to set up a |Makefile| or
shell script to automatise the process.

%%%%%%%%%%%%%%%%%%%%%%%%%%%%%%%%%%%%%%%%%%%%%%%%%%%%%%%%%%%%%%%%%%%%%%%%%%%%%%%%
\subsection{Command Line Processing}
\label{sec:commandline}

The effect of redirection files can also be achieved by invoking
the \LaTeX{} compiler with a more elaborate command line.
Most conveniently this should be done as part
of a shell script or a |Makefile|.

When using \textsf{childdoc} in the main file, the following
command lines effectively perform a redirection
(note that depending on the shell being used,
backslashes may have to be doubled: `|\|' $\to$ `|\\|'):
%
\begin{center}
|... -jobname "|\textit{target}|" |\\|"|[\textit{flags}]%
|\input{childdoc.def}\childdocforward[|\textit{main}|]{|\textit{dest}|}"|
\end{center}
%
Here \textit{target} is the name of the output file,
\textit{main} is the name of the main file
and \textit{dest} is the name of the main or child file to be processed
(all filenames without extensions).
The optional argument \textit{main} can be omitted
if \textit{main} matches \textit{dest}.
Optionally, compilation \textit{flags} can be defined via |\def| commands.
This command line makes the \TeX{} engine believe
it is compiling the file \textit{target}
whose content is specified as the latter parameter.
The provided code then forwards the processing to
\textit{main} or \textit{dest} as described in \secref{sec:forward}.

%%%%%%%%%%%%%%%%%%%%%%%%%%%%%%%%%%%%%%%%%%%%%%%%%%%%%%%%%%%%%%%%%%%%%%%%%%%%%%%%
\subsection{Include by Input}
\label{sec:input}

Including child documents by |\include| has some restrictions by design.
Most notably, the content of a child document always occupies
its own set of pages; pages cannot be shared between child documents.
Usually, this behaviour makes perfect sense
because each child document contain an essential part of the document.
However, in some situations it may be desirable to compose
a document from a collection of parts
without having mandatory page breaks between then.
For this case, the package
provides a mechanism to include parts
by |\input| which can also be processed individually.
However, by construction this mechanism
requires manual handling of the content to be output.

%%%%%%%%%%%%%%%%%%%%%%%%%%%%%%%%%%%%%%%%
\DescribeMacro{\ifchilddocmanual}
The main file should be prepared as usual, see \secref{sec:include}.
However, the document body must make a distinction
between processing of an individual part and of the main document, e.g.:
%
\begin{center}
\begin{tabular}{l}
|\ifchilddocmanual|\\
|\input{\childdocname}|\\
|\||else|\\
\textit{document body with }|\input{|\textit{part}|}|\\
|\||fi|
\end{tabular}
\end{center}
%
The conditional |\ifchilddocmanual| is true whenever
a part to be included by |\input| is being compiled,
and the name of the part is stored in |\childdocname|.

%%%%%%%%%%%%%%%%%%%%%%%%%%%%%%%%%%%%%%%%
\DescribeMacro{\childdocby}
Each part to be included by |\input| should start with:
%
\begin{center}
\begin{tabular}{l}
|\input{childdoc.def}|\\
|\childdocby{|\textit{main}|}|\\
\end{tabular}
\end{center}
%
The directive |\childdocby| is similar to |\childdocof|
described in \secref{sec:include},
but the subsequent selection of content must be done manually.
To that end, both |\ifchilddoc| and |\ifchilddocmanual|
will be true upon processing of a part,
and the name of the part is stored in |\childdocname|.
Note that |\jobname| will be set to the filename of the current part
so that each part receives an individual |.aux| file
that does not interfere with the |.aux| file(s) of the main document.
This behaviour can be altered by the alternative form
|\childdocby[*]{|\textit{main}|}| (with a non-empty optional argument)
which uses the |.aux| file of the main document
by setting |\jobname| to \textit{main}.

%%%%%%%%%%%%%%%%%%%%%%%%%%%%%%%%%%%%%%%%%%%%%%%%%%%%%%%%%%%%%%%%%%%%%%%%%%%%%%%%
\subsection{Driver Development}
\label{sec:driver}

The \textsf{childdoc} mechanism can also be use for the development
of definition files such as \LaTeX{} styles or classes.
This case differs from the above setup with multiple parts
included by |\include| in that no |\includeonly| should be invoked.
This can be achieved by starting the include file
(before |\ProvidesPackage|) with:
%
\begin{center}
\begin{tabular}{l}
|\input{childdoc.def}|\\
|\childdocforward{|\textit{main}|}|\\
\end{tabular}
\end{center}
%
or alternatively with:
%
\begin{center}
\begin{tabular}{l}
|\input{childdoc.def}|\\
|\childdocby{|\textit{main}|}|\\
\end{tabular}
\end{center}
%
Both forms have slightly different effects as described above.
The main file is prepared as usual, see \secref{sec:include}.

%%%%%%%%%%%%%%%%%%%%%%%%%%%%%%%%%%%%%%%%%%%%%%%%%%%%%%%%%%%%%%%%%%%%%%%%%%%%%%%%
\subsection{Legacy Detection}
\label{sec:detection}

The directive |\childdocmain| in the main file can detect
whether the complete document or merely a child is to be compiled
even without using the directive |\childdocof|.
This method is deprecated because it is less robust
and there is no compelling reason to use it;
it is merely provided for backward compatibility
and it may be removed in future versions.

If the detection mechanism is to be used,
it is mandatory to correctly specify
the filename of the main file as the argument of |\childdocmain|:
%
\begin{center}
\begin{tabular}{l}
|\input{childdoc.def}|\\
|\childdocmain{|\textit{main}|}|\\
\end{tabular}
\end{center}
%
If |\jobname| does not match the argument \textit{main} of |\childdocmain|,
it is assumed that |\jobname| points to the child file to be compiled.
When using |\childdocmain| with the main file specified as argument,
it suffices to start a child file
with just |\input{|\textit{main}|}|
without loading of the package and using |\childdocof|.
If instead all processing is done
with the appropriate \textsf{childdoc} directives,
the argument of \textit{main} of |\childdocmain| can be empty.

An alternative version of the command line processing described
in \secref{sec:commandline} using the detection mechanism reads:
%
\begin{center}
|... -jobname "|\textit{target}|" "|[\textit{flags}]%
[|\def\jobname{|\textit{dest}|}|]|\input{|\textit{main}|}"|
\end{center}

%%%%%%%%%%%%%%%%%%%%%%%%%%%%%%%%%%%%%%%%%%%%%%%%%%%%%%%%%%%%%%%%%%%%%%%%%%%%%%%%
\subsection{Manual Code}
\label{sec:manual}

In case one cannot be certain whether the definitions file |childdoc.def|
is installed on the target \TeX{} distribution
and one prefers not to ship it,
it is conceivable to paste a few relevant commands into the sources.

To that end, drop all statements |\input{childdoc.def}|
and perform the replacements as outlined below.
Instead of |\childdocmain{|\textit{main}|}| add the following code
to the top of the main file:
%
\begin{center}
\begin{tabular}{l}
|\||ifdefined\childdocname\endinput\||fi\newif\ifchilddoc|\\
|\edef\childdocname{\scantokens\expandafter{\jobname\noexpand}}|\\
|\def\childdocmain{|\textit{main}|}\||ifx\childdocmain\childdocname\||else|\\
|\childdoctrue\includeonly{\childdocname}\let\jobname\childdocmain\||fi|\\
\end{tabular}
\end{center}
%
Instead of |\childdocof{|\textit{main}|}| just include the main file
at the top of each child file:
%
\begin{center}
|\input{|\textit{main}|}|
\end{center}
%
A simple redirection |\childdocforward{|\textit{dest}|}| is achieved by:
%
\begin{center}
|\def\jobname{|\textit{dest}|}\input{\jobname}|
\end{center}
%
The redirection with prefix
|\childdocforwardprefix[|\textit{prefix}|]{|\textit{dest}|}|
is accomplished by:
%
\begin{center}
\begin{tabular}{l}
|{\edef\jobname{\scantokens\expandafter{\jobname\noexpand}}|\\
|\def\redirectjob |\textit{prefix}|#1~~~{\gdef\jobname{|\textit{dest}|#1}}|\\
|\expandafter\redirectjob\jobname~~~}\input{\jobname}|
\end{tabular}
\end{center}

In an alternative approach,
child documents can be compiled by a specific command line
without additional code or specific definitions:
%
\begin{center}
|... -jobname "|\textit{target}|" "|[\textit{flags}]%
|\includeonly{|\textit{dest}|}\input{|\textit{main}|}"|
\end{center}
%

%%%%%%%%%%%%%%%%%%%%%%%%%%%%%%%%%%%%%%%%%%%%%%%%%%%%%%%%%%%%%%%%%%%%%%%%%%%%%%%%
%%%%%%%%%%%%%%%%%%%%%%%%%%%%%%%%%%%%%%%%%%%%%%%%%%%%%%%%%%%%%%%%%%%%%%%%%%%%%%%%
\section{Information}

%%%%%%%%%%%%%%%%%%%%%%%%%%%%%%%%%%%%%%%%%%%%%%%%%%%%%%%%%%%%%%%%%%%%%%%%%%%%%%%%
\subsection{Copyright}

Copyright \copyright{} 2017--2018 Niklas Beisert

This work may be distributed and/or modified under the
conditions of the \LaTeX{} Project Public License, either version 1.3
of this license or (at your option) any later version.
The latest version of this license is in
  \url{http://www.latex-project.org/lppl.txt}
and version 1.3 or later is part of all distributions of \LaTeX{}
version 2005/12/01 or later.

This work has the LPPL maintenance status `maintained'.

The Current Maintainer of this work is Niklas Beisert.

This work consists of the files |README.txt|, |childdoc.ins| and |childdoc.dtx|
as well as the derived files |childdoc.def|, |cdocsamp.tex|
with |cdocsch1.tex|, |cdocsch2.tex|, |cdocspt3.tex|, |cdocspt4.tex|,
|cdocsdrf.tex|, |cdocsfn1.tex|, |cdocsfn2.tex|
as well as |childdoc.pdf|.

%%%%%%%%%%%%%%%%%%%%%%%%%%%%%%%%%%%%%%%%%%%%%%%%%%%%%%%%%%%%%%%%%%%%%%%%%%%%%%%%
\subsection{Files and Installation}

The package consists of the files:
%
\begin{center}
\begin{tabular}{ll}
    |README.txt|   & readme file \\
    |childdoc.ins| & installation file \\
    |childdoc.dtx| & source file \\
    |childdoc.def| & definition file \\
    |cdocsamp.tex| & sample main file \\
    |cdocsch1.tex| & sample include file \\
    |cdocsch2.tex| & sample include file \\
    |cdocspt3.tex| & sample part file \\
    |cdocspt4.tex| & sample part file \\
    |cdocsdrf.tex| & sample redirection file \\
    |cdocsfn1.tex| & sample redirection file \\
    |cdocsfn2.tex| & sample redirection file \\
    |childdoc.pdf| & manual
\end{tabular}
\end{center}
%
The distribution consists of the files
|README.txt|, |childdoc.ins| and |childdoc.dtx|.
%
\begin{itemize}
\item
Run (pdf)\LaTeX{} on |childdoc.dtx|
to compile the manual |childdoc.pdf| (this file).
\item
Run \LaTeX{} on |childdoc.ins| to create the definitions file |childdoc.def|
and the sample |cdocsamp.tex| with include files
|cdocsch1.tex|, |cdocsch2.tex|, |cdocspt3.tex|, |cdocspt4.tex|,
|cdocsdrf.tex|, |cdocsfn1.tex|, |cdocsfn2.tex|.
Then copy the file |childdoc.def| to an appropriate directory of your \LaTeX{}
distribution, e.g.\ \textit{texmf-root}|/tex/latex/childdoc|.
\end{itemize}

%%%%%%%%%%%%%%%%%%%%%%%%%%%%%%%%%%%%%%%%%%%%%%%%%%%%%%%%%%%%%%%%%%%%%%%%%%%%%%%%
\subsection{Related CTAN Packages}

There are several other packages which offer a similar functionality:
%
\begin{itemize}
\item
The packages
\href{http://ctan.org/pkg/docmute}{\textsf{docmute}},
\href{http://ctan.org/pkg/includex}{\textsf{includex}} and
\href{http://ctan.org/pkg/standalone}{\textsf{standalone}}
provide commands to include only the document body of
a child file thus allowing both files to be compiled individually.
\item
The packages \href{http://ctan.org/pkg/subdocs}{\textsf{subdocs}}
and \href{http://ctan.org/pkg/subfiles}{\textsf{subfiles}}
provide structures in which the main and child documents can be
encapsulated and allowing them to be compiled individually.
The inclusion mechanism is different from the conventional |\include|.
\item
The package \href{http://ctan.org/pkg/combine}{\textsf{combine}}
is an elaborate solution to combine several documents into one.
\end{itemize}
%
See also the CTAN topic \href{http://ctan.org/topic/subdocs}{\textsf{subdocs}}
for further related packages.
The present package differs from the above solutions in that
a document structure constructed with the conventional |\include| mechanism
just needs two extra commands at the top of every file
such that all constituent files can be compiled individually.

%%%%%%%%%%%%%%%%%%%%%%%%%%%%%%%%%%%%%%%%%%%%%%%%%%%%%%%%%%%%%%%%%%%%%%%%%%%%%%%%
%\subsection{Feature Suggestions}
%
%The following is a list of features which may be useful for future
%versions of this package:
%%
%\begin{itemize}
%\item
%\ldots
%\end{itemize}

%%%%%%%%%%%%%%%%%%%%%%%%%%%%%%%%%%%%%%%%%%%%%%%%%%%%%%%%%%%%%%%%%%%%%%%%%%%%%%%%
\subsection{Revision History}

%%%%%%%%%%%%%%%%%%%%%%%%%%%%%%%%%%%%%%%%
\paragraph{v2.0:} 2018/12/30

\begin{itemize}
\item
immediate forward processing
\item
added |\childdocby| mechanism
\item
manual restructured
\end{itemize}

%%%%%%%%%%%%%%%%%%%%%%%%%%%%%%%%%%%%%%%%
\paragraph{v1.6:} 2018/01/17

\begin{itemize}
\item
application for development of include files
\item
corrections to manual
\end{itemize}

%%%%%%%%%%%%%%%%%%%%%%%%%%%%%%%%%%%%%%%%
\paragraph{v1.5:} 2017/05/21

\begin{itemize}
\item
more complete structuring introduced
\item
|\childdocof| introduced
\item
|\childdoc| renamed to |\childdocmain|
\item
|\childredirect| renamed to |\childdocforward| and |\childdocforwardprefix|
and functionality expanded
\end{itemize}

%%%%%%%%%%%%%%%%%%%%%%%%%%%%%%%%%%%%%%%%
\paragraph{v1.0:} 2017/04/27

\begin{itemize}
\item
manual and install package
\item
first version published on CTAN
\end{itemize}

%%%%%%%%%%%%%%%%%%%%%%%%%%%%%%%%%%%%%%%%
\paragraph{v0.6:} 2017/04/26

\begin{itemize}
\item
redirection mechanism added
\end{itemize}

%%%%%%%%%%%%%%%%%%%%%%%%%%%%%%%%%%%%%%%%
\paragraph{v0.5:} 2017/04/26

\begin{itemize}
\item
functionality in definition file
\end{itemize}


%%%%%%%%%%%%%%%%%%%%%%%%%%%%%%%%%%%%%%%%%%%%%%%%%%%%%%%%%%%%%%%%%%%%%%%%%%%%%%%%
%%%%%%%%%%%%%%%%%%%%%%%%%%%%%%%%%%%%%%%%%%%%%%%%%%%%%%%%%%%%%%%%%%%%%%%%%%%%%%%%
%%%%%%%%%%%%%%%%%%%%%%%%%%%%%%%%%%%%%%%%%%%%%%%%%%%%%%%%%%%%%%%%%%%%%%%%%%%%%%%%
\appendix

\settowidth\MacroIndent{\rmfamily\scriptsize 000\ }

 \DocInput{childdoc.dtx}

\end{document}
%</driver>
% \fi
%
% %%%%%%%%%%%%%%%%%%%%%%%%%%%%%%%%%%%%%%%%%%%%%%%%%%%%%%%%%%%%%%%%%%%%%%%%%%%%%%
% %%%%%%%%%%%%%%%%%%%%%%%%%%%%%%%%%%%%%%%%%%%%%%%%%%%%%%%%%%%%%%%%%%%%%%%%%%%%%%
% \section{Sample}
%\iffalse
%<*samplemain>
%\fi
%
% The following presents a sample document
% with two chapters, two parts, a title page,
% a compile flag as well as three forwarding files to set the flag.
% It consists of eight |.tex| files:
% \begin{center}
% \begin{tabular}{ll}
% |cdocsamp.tex|&main file\\
% |cdocsch1.tex|&include file for chapter 1\\
% |cdocsch2.tex|&include file for chapter 2\\
% |cdocspt3.tex|&include file for part 3\\
% |cdocspt4.tex|&include file for part 4\\
% |cdocsdrf.tex|&forwarding file for main file in draft mode\\
% |cdocsfi1.tex|&forwarding file for final version of chapter 1\\
% |cdocsfi2.tex|&forwarding file for final version of chapter 2\\
% \end{tabular}
% \end{center}
% Each of the eight files can be compiled directly by the \LaTeX{} compiler.
%
% %%%%%%%%%%%%%%%%%%%%%%%%%%%%%%%%%%%%%%
% \paragraph{Main File.}
%
% The main file is called |cdocsamp.tex|.
%
% Load the \textsf{childdoc} definitions and
% declare the filename for the main document:
%    \begin{macrocode}
\input{childdoc.def}
\childdocmain{}
%    \end{macrocode}

% Optional override for |\version| flag:
%    \begin{macrocode}
%%\ifchilddoc\else\providecommand{\version}{draft}\fi
%    \end{macrocode}

% Define the default values for the |\version| flag
% (|final| for the main file and |draft| for childs):
%    \begin{macrocode}
\ifchilddoc
\providecommand{\version}{draft}
\else
\providecommand{\version}{final}
\fi
%    \end{macrocode}

% Load the standard document class:
%    \begin{macrocode}
\documentclass[12pt]{article}
%    \end{macrocode}

% Start the document body:
%    \begin{macrocode}
\begin{document}
%    \end{macrocode}

% Declare a title page.
% Print title, part of document being processed and version flag:
%    \begin{macrocode}
\addtocounter{page}{-1}
\begin{center}
{\LARGE\bfseries{}childdoc example\par}
\vspace{1cm}
\ifchilddoc
\ifchilddocmanual part\else chapter\fi:
`\childdocname' of `\childdocjob'\par
\else
main document: `\childdocjob'\par
\fi
version: \version\par
\end{center}
\newpage
%    \end{macrocode}

% Manually include selected file,
% otherwise process as usual:
%    \begin{macrocode}
\ifchilddocmanual
\section*{part `\childdocname'}
\input{\childdocname}
\else
%    \end{macrocode}

% Include the two chapters:
%    \begin{macrocode}
\include{cdocsch1}
\include{cdocsch2}
%    \end{macrocode}

% Include the two parts unless only chapters should be displayed:
%    \begin{macrocode}
\ifchilddoc\else
\section{part three}
\input{cdocspt3}
\section{part four}
\input{cdocspt4}
\fi
%    \end{macrocode}

% Process as usual until here:
%    \begin{macrocode}
\fi
%    \end{macrocode}

% End of document body:
%    \begin{macrocode}
\end{document}
%    \end{macrocode}
%\iffalse
%</samplemain>
%\fi
%
% %%%%%%%%%%%%%%%%%%%%%%%%%%%%%%%%%%%%%%
% \paragraph{Chapter Include Files.}
%
% The include files are called |cdocsch1.tex| and |cdocsch2.tex|.
%
%\iffalse
%<*samplechap1|samplechap2>
%\fi

% Optional override for |\version| flag:
%    \begin{macrocode}
%%\providecommand{\version}{final}
%    \end{macrocode}

% Include the main document:
%    \begin{macrocode}
\input{childdoc.def}
\childdocof{cdocsamp}
%    \end{macrocode}

%\iffalse
%</samplechap1|samplechap2>
%\fi
%
%\iffalse
%<*samplechap1>
%\fi
% Some text for chapter 1:
%    \begin{macrocode}
\section{one}
some text in chapter one
%    \end{macrocode}

%\iffalse
%</samplechap1>
%\fi
% Some text for chapter 2:
%\iffalse
%<*samplechap2>
%\fi
%    \begin{macrocode}
\section{two}
more text in chapter two
%    \end{macrocode}

%\iffalse
%</samplechap2>
%\fi
%
% %%%%%%%%%%%%%%%%%%%%%%%%%%%%%%%%%%%%%%
% \paragraph{Part Include Files.}
%
% The include files are called |cdocspt3.tex| and |cdocspt4.tex|.
%
%\iffalse
%<*samplepart3|samplepart4>
%\fi

% Optional override for |\version| flag:
%    \begin{macrocode}
%%\providecommand{\version}{final}
%    \end{macrocode}

% Include the main document:
%    \begin{macrocode}
\input{childdoc.def}
\childdocby{cdocsamp}
%    \end{macrocode}

%\iffalse
%</samplepart3|samplepart4>
%\fi
%
%\iffalse
%<*samplepart3>
%\fi
% Some text for part 3:
%    \begin{macrocode}
some text in part three
%    \end{macrocode}

%\iffalse
%</samplepart3>
%\fi
% Some text for part 4:
%\iffalse
%<*samplepart4>
%\fi
%    \begin{macrocode}
more text in part four
%    \end{macrocode}

%\iffalse
%</samplepart4>
%\fi
%
% %%%%%%%%%%%%%%%%%%%%%%%%%%%%%%%%%%%%%%
% \paragraph{Forwarding for a Complete Draft.}
%
% The following forwarding file |cdocsdrf.tex|
% compiles the main document in draft mode:
%\iffalse
%<*sampledraft>
%\fi
%    \begin{macrocode}
\def\version{draft}
\input{childdoc.def}
\childdocforward{cdocsamp}
%    \end{macrocode}

%\iffalse
%</sampledraft>
%\fi
%
% %%%%%%%%%%%%%%%%%%%%%%%%%%%%%%%%%%%%%%
% \paragraph{Forwarding for Final Version of the Chapters.}
%
% The following forwarding files |cdocsfn1.tex| and |cdocsfn2.tex|
% (with identical content)
% compile the final versions of the child documents
% |cdocsch1.tex| and |cdocsch2.tex|, respectively:
%\iffalse
%<*samplefinal>
%\fi
%    \begin{macrocode}
\def\version{final}
\input{childdoc.def}
\childdocforwardprefix[cdocsamp]{cdocsfn}{cdocsch}
%    \end{macrocode}

%\iffalse
%</samplefinal>
%\fi
%
% %%%%%%%%%%%%%%%%%%%%%%%%%%%%%%%%%%%%%%
% \paragraph{Command Line Processing.}
%
% The following three command lines generate the output files
% |cdocscld|, |cdocscl1| and |cdocscl2|
% which should be identical to
% |cdocsdrf|, |cdocsch1| and |cdocsfn2|, respectively:
% \begin{center}
% \begin{tabular}{l}
% |latex -jobname cdocscld \|\\
% |  "\def\version{draft}\input{childdoc.def}\childdocforward{cdocsamp}"|\\
% |latex -jobname cdocscl1 \|\\
% |  "\input{childdoc.def}\childdocforward[cdocsamp]{cdocsch1}"|\\
% |latex -jobname cdocscl2 \|\\
% |  "\def\version{final}\input{childdoc.def}\childdocforward{cdocsch2}"|
% \end{tabular}
% \end{center}
% Note that the trailing backslash on each first line
% merely continues the input to the second line
% (for convenient cut ant paste).
% Furthermore, the command |latex| can be replaced by any
% of its alternative versions such as |pdflatex|.
%
% %%%%%%%%%%%%%%%%%%%%%%%%%%%%%%%%%%%%%%%%%%%%%%%%%%%%%%%%%%%%%%%%%%%%%%%%%%%%%%
% %%%%%%%%%%%%%%%%%%%%%%%%%%%%%%%%%%%%%%%%%%%%%%%%%%%%%%%%%%%%%%%%%%%%%%%%%%%%%%
% \section{Implementation}
%\iffalse
%<*package>
%\fi
%
% This section describes the definitions file |childdoc.def|.

% The definitions cannot be loaded using |\usepackage| or |\RequirePackage|
% which has a mechanism to prevent loading a style file more than once.
% When loading the definitions by means of |\input|
% multiple instances have to be prevented manually:
%\iffalse
%This code needs to be before the `\ProvidesFile' directive
%which is defined at the beginning of this file.
%Therefore it is also placed there and commented out here.
%</package>
%<*discard>
%\fi
%    \begin{macrocode}
\ifdefined\childdocmain\endinput\fi
%    \end{macrocode}
%\iffalse
%</discard>
%<*package>
%\fi
%
% \macro{\ifchilddoc}
% \macro{\ifchilddocmanual}
% The conditional |\ifchilddoc| tells whether a
% child (true) or main (false) document is being compiled.
% The conditional |\ifchilddocmanual| tells whether
% the |\includeonly| mechanism is used (false) or
% the selection of child files must be performed manually (true).
% The definitions initialise to false:
%    \begin{macrocode}
\newif\ifchilddoc
\newif\ifchilddocmanual
%    \end{macrocode}

% \macro{\childdocname}
% \macro{\childdocjob}
% The macro |\childdocname| stores the name of the main document
% to be compiled. The macro |\childdocjob| stores the name of
% the document on which the \LaTeX{} compiler was originally invoked.
% The content of |\jobname| cannot be compared
% to filenames specified in the source due to different catcodes.
% The following code rescans |\jobname|, stores the result
% in |\childdocname| and saves a copy in |\childdocjob|:
%    \begin{macrocode}
\edef\childdocname{\scantokens\expandafter{\jobname\noexpand}}
\let\childdocjob\childdocname
%    \end{macrocode}

% \macro{\childdocdisable}
% The macro |\childdocdisable| prevents the main file
% from being processed more than once.
% At this stage, the main document command |\childdocmain|
% is assumed to be called once again where it should do nothing.
% Any subsequent call to it should prevent
% a secondary processing of the main document
% It overwrites the forwarding commands
% |\childdocof| and |\childdocforward|
% with empty macros to prevent further inclusions of the main document:
%    \begin{macrocode}
\newcommand{\childdocdisable}
{
  \renewcommand{\childdocmain}[1]{\renewcommand{\childdocmain}[1]{\endinput}}
  \renewcommand{\childdocof}[1]{}
  \renewcommand{\childdocby}[2][]{}
  \renewcommand{\childdocforward}[2][]{}
  \renewcommand{\childdocdisable}{}
}
%    \end{macrocode}

% \macro{\childdocmain}
% The macro |\childdocmain| is to be called at the top of the main file
% with nothing or the main filename (without extension) as argument.
% First, it breaks loops.
% If the argument is not empty and does not match |\childdocname|
% (which is set by the first inclusion of |childdoc.def|),
% |\ifchilddoc| is set to true, |\includeonly| is applied to the child file
% and |\jobname| is set to the main file
% (for proper handling of |.aux| files):
%    \begin{macrocode}
\newcommand{\childdocmain}[1]
{
  \childdocdisable\childdocmain{}
  \if?#1?\else
    \begingroup
      \def\childdoctmp{#1}
      \ifx\childdoctmp\childdocname
        \def\childdoctmp{}
      \else
        \def\childdoctmp
        {
          \childdoctrue
          \includeonly{\childdocname}
          \def\childdocjob{#1}
          \def\jobname{#1}
        }
      \fi
      \expandafter
    \endgroup
    \childdoctmp
  \fi
}
%    \end{macrocode}

% \macro{\childdocof}
% The command |\childdocof| redirects
% compilation to the main file |#1|.
%    \begin{macrocode}
\newcommand{\childdocof}[1]
{
  \childdocdisable
  \childdoctrue
  \includeonly{\childdocname}
  \def\jobname{#1}
  \def\childdocjob{#1}
  \input{#1}
}
%    \end{macrocode}

% \macro{\childdocby}
% The command |\childdocby| ....
%    \begin{macrocode}
\newcommand{\childdocby}[2][]
{
  \childdocdisable
  \childdoctrue
  \childdocmanualtrue
  \if?#1?\else
    \def\jobname{#2}
  \fi
  \def\childdocjob{#2}
  \input{#2}
  \endinput
}
%    \end{macrocode}

% \macro{\childdocforward}
% The command |\childdocforward| redirects
% compilation to the main file or
% (if the optional argument is given) a child file.
% Parameters are set as if the main file
% or a child file starting with |\childdocof| was compiled.
% Then compilation is handed over to the main file:
%    \begin{macrocode}
\newcommand{\childdocforward}[2][]
{
  \begingroup
    \if?#1?
      \def\childdoctmp
      {
        \def\childdocname{#2}
        \def\childdocjob{#2}
        \def\jobname{#2}
        \input{#2}
        \endinput
      }
    \else
      \def\childdoctmp
      {
        \childdocdisable
        \def\childdocname{#2}
        \childdoctrue
        \includeonly{#2}
        \def\childdocjob{#1}
        \def\jobname{#1}
        \input{#1}
        \endinput
      }
    \fi
    \expandafter
  \endgroup
  \childdoctmp
}
%    \end{macrocode}

% \macro{\childdocforwardprefix}
% The command |\childdocforwardprefix| redirects
% compilation to the main or a child file by means of a pattern.
% The prefix |#1| in the current filename is replaced by |#2|
% and the suffix of the current filename is kept
% (it is assumed that the filename does not contain the substring `|~~~|'
% which is used as a delimiter).
% Compilation is handed over to the new file by |\childdocforward|:
%    \begin{macrocode}
\newcommand{\childdocforwardprefix}[3][]
{
  \begingroup
    \def\childdocextract #2##1~~~{\def\childdoctmp{\childdocforward[#1]{#3##1}}}
    \expandafter\childdocextract\childdocname~~~
    \expandafter
  \endgroup
  \childdoctmp
}
%    \end{macrocode}

% \macro{\childdoc}
% The deprecated macro |\childdoc| is a legacy version of |\childdocmain|:
%    \begin{macrocode}
\newcommand{\childdoc}{\childdocmain}
%    \end{macrocode}

% \macro{\childdocredirect}
% The deprecated macro |\childdocredirect| is a legacy version
% of |\childdocforward| and |\childdocforwardprefix|:
%    \begin{macrocode}
\newcommand{\childdocredirect}[2][]
{
  \begingroup
    \if?#1?
      \def\childdoctmp{\childdocforward{#2}}
    \else
      \def\childdoctmp{\childdocforwardprefix{#1}{#2}}
    \fi
    \expandafter
  \endgroup
  \childdoctmp
}
%    \end{macrocode}

%\iffalse
%</package>
%\fi
%
\endinput
|\\
|\childdocforward[|\textit{main}|]{|\textit{dest}|}|\\
\end{tabular}
\end{center}
%
The argument \textit{dest} is the destination file
(without extension).
It should be the main file or one of the child files.
Note that further \textsf{childdoc} directives
such as |\childdocof| and |\childdocforward|
in the indicated file will be processed in this form.
The optional argument \textit{main}
passes on directly to the main file \textit{main}
while pretending to compile the child \textit{dest}.
This form behaves as if \textit{dest}
issues |\childdocof{|\textit{main}|}| right away,
and no further \textsf{childdoc} directives will be processed.

%%%%%%%%%%%%%%%%%%%%%%%%%%%%%%%%%%%%%%%%
\DescribeMacro{\...prefix}
In the alternative form |\childdocforwardprefix|,
%
\begin{center}
\begin{tabular}{l}
|% \iffalse
%
% childdoc.dtx Copyright (C) 2017-2018 Niklas Beisert
%
% This work may be distributed and/or modified under the
% conditions of the LaTeX Project Public License, either version 1.3
% of this license or (at your option) any later version.
% The latest version of this license is in
%   http://www.latex-project.org/lppl.txt
% and version 1.3 or later is part of all distributions of LaTeX
% version 2005/12/01 or later.
%
% This work has the LPPL maintenance status `maintained'.
%
% The Current Maintainer of this work is Niklas Beisert.
%
% This work consists of the files childdoc.dtx and childdoc.ins
% and the derived files childdoc.def and cdocsamp.tex with
% cdocsch1.tex, cdocsch2.tex, cdocsdrf.tex, cdocsfn1.tex, cdocsfn2.tex.
%
%<package>\ifdefined\childdocmain\endinput\fi
%<package>\ProvidesFile{childdoc.def}[2018/12/30 v2.0 child document driver]
%<samplemain>\ProvidesFile{cdocsamp.tex}[2018/12/30 v2.0 sample for childdoc]
%<*driver>
%\ProvidesFile{childdoc.drv}[2018/12/30 v2.0 childdoc reference manual file]
\PassOptionsToClass{10pt,a4paper}{article}
\documentclass{ltxdoc}

\usepackage[margin=35mm]{geometry}
\usepackage{hyperref}
\usepackage{hyperxmp}
\usepackage[usenames]{color}

\hypersetup{colorlinks=true}
\hypersetup{pdfstartview=FitH}
\hypersetup{pdfpagemode=UseNone}
\hypersetup{pdfsource={}}
\hypersetup{pdflang={en-UK}}
\hypersetup{pdfcopyright={Copyright 2017-2018 Niklas Beisert.
  This work may be distributed and/or modified under the
  conditions of the LaTeX Project Public License, either version 1.3
  of this license or (at your option) any later version.}}
\hypersetup{pdflicenseurl={http://www.latex-project.org/lppl.txt}}
\hypersetup{pdfcontactaddress={ETH Zurich, ITP, HIT K,
  Wolfgang-Pauli-Strasse 27}}
\hypersetup{pdfcontactpostcode={8093}}
\hypersetup{pdfcontactcity={Zurich}}
\hypersetup{pdfcontactcountry={Switzerland}}
\hypersetup{pdfcontactemail={nbeisert@itp.phys.ethz.ch}}
\hypersetup{pdfcontacturl={http://people.phys.ethz.ch/\xmptilde nbeisert/}}

\newcommand{\secref}[1]{\hyperref[#1]{section \ref*{#1}}}

\parskip1ex
\parindent0pt
\let\olditemize\itemize
\def\itemize{\olditemize\parskip0pt}

\begin{document}

\title{The \textsf{childdoc} Package}
\hypersetup{pdftitle={The childdoc Package}}
\author{Niklas Beisert\\[2ex]
  Institut f\"ur Theoretische Physik\\
  Eidgen\"ossische Technische Hochschule Z\"urich\\
  Wolfgang-Pauli-Strasse 27, 8093 Z\"urich, Switzerland\\[1ex]
  \href{mailto:nbeisert@itp.phys.ethz.ch}
  {\texttt{nbeisert@itp.phys.ethz.ch}}}
\hypersetup{pdfauthor={Niklas Beisert}}
\hypersetup{pdfsubject={Manual for the LaTeX2e Package childdoc}}
\date{30 December 2018, \textsf{v2.0}}
\maketitle

\begin{abstract}\noindent
\textsf{childdoc} is a \LaTeXe{} package
that enables the direct compilation
of document sections included by |\include|
to individual files.
\end{abstract}

\begingroup
\parskip0ex
\tableofcontents
\endgroup

%%%%%%%%%%%%%%%%%%%%%%%%%%%%%%%%%%%%%%%%%%%%%%%%%%%%%%%%%%%%%%%%%%%%%%%%%%%%%%%%
%%%%%%%%%%%%%%%%%%%%%%%%%%%%%%%%%%%%%%%%%%%%%%%%%%%%%%%%%%%%%%%%%%%%%%%%%%%%%%%%
\section{Introduction}

\LaTeX{} provides a mechanism to structure a large document (such as a book)
into a main file and several child files (containing the chapters)
using the |\include| command.
This mechanism is beneficial for documents
which span hundreds of pages in order to
make the source file(s) more manageable.
Moreover, compilation can be restricted to
selected child files by means of the |\includeonly| command.
The latter feature can be used to reduce the compilation time while editing
(this was significantly more useful in the earlier days of \LaTeX{})
or to generate a smaller document which is easier to navigate.
Another application of |\includeonly| is to generate
documents consisting of selected parts of the complete document.

However, there are a few drawbacks of the plain |\include| mechanism:
\begin{itemize}
\item
The child files cannot be compiled on their own,
they can only be compiled via the main file.
A naive editing environment
(such as a text editor with an option
to have the current file processed by \LaTeX)
may require one to switch to the main file before compiling;
attempting to compile the child file produces errors.
\item
The main file must be modified (each time)
to adjust the |\includeonly| command
to the present needs. This easily leaves the main file in a messy state.
\item
The generated document will always carry the filename
of the main document. This is inconvenient if
several child files are to be compiled and
to be kept for distribution.
\end{itemize}

The present package provides a simple interface
to make child files individually compilable by \LaTeX{}.
Compiling a child file then has the same effect as compiling
the main file with an |\includeonly| command
to select the appropriate child.
Moreover the generated document will carry the name of the child
rather than the main file.
This resolves all three above issues.

This feature is meant to make the editing of books,
thesis documents and lecture notes somewhat more convenient.
However, the package can also be used efficiently for
composing a series of documents (such as exercise sheets)
which are typically distributed individually.
It then assists the author in generating the individual documents
(potentially in different versions)
as well as a document containing the collected series.
Another application is in developing style files
or other kinds of included material
where compilation of the style file could redirect
to a sample or test file.

%%%%%%%%%%%%%%%%%%%%%%%%%%%%%%%%%%%%%%%%%%%%%%%%%%%%%%%%%%%%%%%%%%%%%%%%%%%%%%%%
%%%%%%%%%%%%%%%%%%%%%%%%%%%%%%%%%%%%%%%%%%%%%%%%%%%%%%%%%%%%%%%%%%%%%%%%%%%%%%%%
\section{Usage}

First of all, the package \textsf{childdoc} is \emph{not} a standard
\LaTeXe{} |.sty| style file! Therefore it needs to be invoked in
a non-standard way.

%%%%%%%%%%%%%%%%%%%%%%%%%%%%%%%%%%%%%%%%%%%%%%%%%%%%%%%%%%%%%%%%%%%%%%%%%%%%%%%%
\subsection{Included Files}
\label{sec:include}

%%%%%%%%%%%%%%%%%%%%%%%%%%%%%%%%%%%%%%%%
\DescribeMacro{\childdocmain}
To use the package, add the commands
\begin{center}
\begin{tabular}{l}
|\input{childdoc.def}|\\
|\childdocmain{}|\\
\end{tabular}
\end{center}
at the very top of the main \LaTeX{} file,
in particular \emph{before} the |\documentclass| statement!
The argument of |\childdocmain| should be left empty
(but it must be present).

%%%%%%%%%%%%%%%%%%%%%%%%%%%%%%%%%%%%%%%%
\DescribeMacro{\childdocof}
Furthermore, add the commands
\begin{center}
\begin{tabular}{l}
|\input{childdoc.def}|\\
|\childdocof{|\textit{main}|}|\\
\end{tabular}
\end{center}
at the top of every child file \textit{child}
which is included by |\include{|\textit{child}|}|
from within the main file
(or at least for those files to be compiled individually).
The argument \textit{main} must be the filename of the main file.

There are a couple of
considerations in setting up the main and child documents:

%%%%%%%%%%%%%%%%%%%%%%%%%%%%%%%%%%%%%%%%
\paragraph{Restrictions.}

Please note the following restrictions:
\begin{itemize}
\item
|\childdocmain| must be called with one argument \textit{main}
to ensure compatibility with earlier version of the package.
It must either be empty (|\childdocmain{}|)
or precisely match the filename of the main file in which it is specified.
See \secref{sec:detection} for further information.
\item
The filename \textit{main} must be specified without the |.tex| extension.
\item
The filename \textit{main} is case sensitive
(even in case-insensitive file systems)
due to internal string comparison.
\item
The argument \textit{main} should be fully expanded, it cannot be a macro.
\item
Subdirectories and special characters should be avoided in filenames.
\item
The command |\childdocmain{|\textit{main}|}| must be followed by a whitespace.
It should not be followed immediately by another command
or by a comment mark `|%|'.
This is because the \TeX{} parser reads the token immediately following
the argument of |\childdocmain| and puts it
at the beginning of every child section;
however, a white\-space is ignored.
\end{itemize}

%%%%%%%%%%%%%%%%%%%%%%%%%%%%%%%%%%%%%%%%
\paragraph{Content of Main File.}

It is advisable to place all content in the child files included by |\include|.
Any output contained in the main file will appear in all child documents
unless suppressed manually;
it cannot be suppressed automatically by the |\includeonly| directive
and thus should normally be avoided.
A method to include some content in the main file
by means of conditional processing is described in \secref{sec:conditional}.

%%%%%%%%%%%%%%%%%%%%%%%%%%%%%%%%%%%%%%%%
\paragraph{Page Numbering.}

When only a part of the document is compiled,
the appropriate numbering of pages
(as well as other status parameters)
is determined from the |.aux| files.
The latter contain information from previous passes.
However this information needs to propagate through
all intermediate child documents.
Therefore the page numbering in child documents may well
be inconsistent until the complete document is compiled at least once.

A useful (if unconventional) way to always ensure a consistent
page numbering is to restart the numbering in each child document
and denote the pages by `\textit{child}|.|\textit{page}'
where \textit{child} represents the chapter/section number of the child file.
This can be achieved by the command
|\numberwithin{page}{|\textit{child}|}|
of the \textsf{amsmath} package
where \textit{child} can be |chapter| or |section|
depending on the chosen structuring.
Alternatively, one can modify the macro |\thepage| appropriately
and reset the counter |page| at the start of each child file.

%%%%%%%%%%%%%%%%%%%%%%%%%%%%%%%%%%%%%%%%%%%%%%%%%%%%%%%%%%%%%%%%%%%%%%%%%%%%%%%%
\subsection{Conditional Processing}
\label{sec:conditional}

The package provides a mechanism to compile different versions
of a document. To customise the versions further some conditional processing
can come in handy to distinguish which version is being compiled.
The package provides two macros to describe the compilation context:

%%%%%%%%%%%%%%%%%%%%%%%%%%%%%%%%%%%%%%%%
\DescribeMacro{\ifchilddoc}
The conditional |\ifchilddoc| distinguishes between the compilation of
child documents and the main document:
%
\begin{center}
|\ifchilddoc |\textit{child-code}| |[|\||else |\textit{main-code}]| \||fi|
\end{center}

%%%%%%%%%%%%%%%%%%%%%%%%%%%%%%%%%%%%%%%%
\DescribeMacro{\childdocname}
\DescribeMacro{\childdocjob}
The macro |\childdocname| contains the filename (without extension)
of the main or child file being processed.
Note that |\childdocjob| will always contain the name of the main file.

%%%%%%%%%%%%%%%%%%%%%%%%%%%%%%%%%%%%%%%%
\paragraph{Title Page.}

Conditional processing can be used to include a title or banner page
in the main document when proper precautions are taken.
Importantly, the code in the main file should ensure that the page counter
(as well as other status parameters which are stored in the |.aux| files)
takes the same value after the conditional processing.
Otherwise the page numbers may take divergent values
depending on which part is compiled.

For example, a title page could be declared by:
%
\begin{center}
\begin{tabular}{l}
|\ifchilddoc\||else|\\
|\addtocounter{page}{-1}|\\
\textit{code for title page}\\
|\newpage|\\
|\||fi|
\end{tabular}
\end{center}
%
A banner page for the child documents can be generated by:
%
\begin{center}
\begin{tabular}{l}
|\ifchilddoc|\\
|\addtocounter{page}{-1}|\\
\textit{code for banner page}\\
|\newpage|\\
|\||fi|
\end{tabular}
\end{center}
%
Here one could write a message such as:
\begin{center}
|This is the part \childdocname{} of \childdocjob{}.|
\end{center}

%%%%%%%%%%%%%%%%%%%%%%%%%%%%%%%%%%%%%%%%%%%%%%%%%%%%%%%%%%%%%%%%%%%%%%%%%%%%%%%%
\subsection{Flags}
\label{sec:flags}

The package makes it easy to generate different versions
of the main or child documents.
To this end compilation flags can be defined
and assigned different default values.
They will be particularly useful in conjunction
with the forwarding mechanism described in \secref{sec:forward}.

For example, it may be useful to have a flag |\version|
which can be set to |draft| or |final|.
The document source will contain some conditional code
depending on the value of |\version|.
Suppose further, the flag should default to |final| for the main file
and to |draft| for child files
which is a natural assignment for editing the document.
This is achieved by placing the following code
in the preamble of the main document
(below the |\childdocmain| directive):
%
\begin{center}
\begin{tabular}{l}
|\ifchilddoc|\\
|\providecommand{\version}{draft}|\\
|\||else|\\
|\providecommand{\version}{final}|\\
|\||fi|
\end{tabular}
\end{center}
%
The definition by |\providecommand| makes sure
that previous definitions are not overwritten.
Further statements |\providecommand{\version}{...}|
can thus be added before the above code to override it.

For the main file, one might add a line
(between |\childdocmain| and the above block)
%
\begin{center}
|%\ifchilddoc\||else\providecommand{\version}{draft}\||fi|
\end{center}
%
which can be uncommented to produce a draft version.
Likewise one can add a line to the very top of a child file
(above the |\childdocof{|\textit{main}|}| directive)
%
\begin{center}
|%\providecommand{\version}{final}|
\end{center}
%
which can be uncommented to produce the final version of this child document.

%%%%%%%%%%%%%%%%%%%%%%%%%%%%%%%%%%%%%%%%%%%%%%%%%%%%%%%%%%%%%%%%%%%%%%%%%%%%%%%%
\subsection{Forwarding}
\label{sec:forward}

Different versions of the main or child documents
using compilation flags as described in \secref{sec:flags}
can be (permanently) stored in different files
for convenient compilation, viewing and distribution.
To this end, the package defines a command
to pass on compilation to a different file:

%%%%%%%%%%%%%%%%%%%%%%%%%%%%%%%%%%%%%%%%
\DescribeMacro{\childdocforward}
The command |\childdocforward| redirects processing to
another source file:
%
\begin{center}
\begin{tabular}{l}
|\input{childdoc.def}|\\
|\childdocforward[|\textit{main}|]{|\textit{dest}|}|\\
\end{tabular}
\end{center}
%
The argument \textit{dest} is the destination file
(without extension).
It should be the main file or one of the child files.
Note that further \textsf{childdoc} directives
such as |\childdocof| and |\childdocforward|
in the indicated file will be processed in this form.
The optional argument \textit{main}
passes on directly to the main file \textit{main}
while pretending to compile the child \textit{dest}.
This form behaves as if \textit{dest}
issues |\childdocof{|\textit{main}|}| right away,
and no further \textsf{childdoc} directives will be processed.

%%%%%%%%%%%%%%%%%%%%%%%%%%%%%%%%%%%%%%%%
\DescribeMacro{\...prefix}
In the alternative form |\childdocforwardprefix|,
%
\begin{center}
\begin{tabular}{l}
|\input{childdoc.def}|\\
|\childdocforwardprefix[|\textit{main}|]{|\textit{prefix}|}{|\textit{dest}|}|
\end{tabular}
\end{center}
%
the destination file is determined by a pattern
depending on the current file:
To make this work, the current file must be called
`{\textit{prefix}\hspace{0.2em}\textit{suffix}}'
with \textit{prefix} matching precisely the argument.
Processing is then passed on to the file
`{\textit{dest}\hspace{0.2em}\textit{suffix}}'.
Surely, the same effect is achieved by
directly specifying the
argument `{\textit{dest}\hspace{0.2em}\textit{suffix}}'
in the first form.
However, that requires to set up a different file
for each child. With the alternative form of the command
all these files can have exactly the same content
which simplifies setting them up and maintaining them.

For example, the following file |draft.tex|
with a compilation flag |\version| as described in \secref{sec:flags}
compiles the main document as a draft:
%
\begin{center}
\begin{tabular}{l}
|\def\version{draft}|\\
|\input{childdoc.def}|\\
|\childdocforward{|\textit{main}|}|
\end{tabular}
\end{center}
%
Likewise, the following files |final|\textit{nn}|.tex|
compile the final version of the child document
|child|\textit{nn}|.tex|:
%
\begin{center}
\begin{tabular}{l}
|\def\version{final}|\\
|\input{childdoc.def}|\\
|\childdocforwardprefix{final}{child}|
\end{tabular}
\end{center}
%

Note that when several versions of a main file and/or of each child file
are to be generated, it may be convenient to set up a |Makefile| or
shell script to automatise the process.

%%%%%%%%%%%%%%%%%%%%%%%%%%%%%%%%%%%%%%%%%%%%%%%%%%%%%%%%%%%%%%%%%%%%%%%%%%%%%%%%
\subsection{Command Line Processing}
\label{sec:commandline}

The effect of redirection files can also be achieved by invoking
the \LaTeX{} compiler with a more elaborate command line.
Most conveniently this should be done as part
of a shell script or a |Makefile|.

When using \textsf{childdoc} in the main file, the following
command lines effectively perform a redirection
(note that depending on the shell being used,
backslashes may have to be doubled: `|\|' $\to$ `|\\|'):
%
\begin{center}
|... -jobname "|\textit{target}|" |\\|"|[\textit{flags}]%
|\input{childdoc.def}\childdocforward[|\textit{main}|]{|\textit{dest}|}"|
\end{center}
%
Here \textit{target} is the name of the output file,
\textit{main} is the name of the main file
and \textit{dest} is the name of the main or child file to be processed
(all filenames without extensions).
The optional argument \textit{main} can be omitted
if \textit{main} matches \textit{dest}.
Optionally, compilation \textit{flags} can be defined via |\def| commands.
This command line makes the \TeX{} engine believe
it is compiling the file \textit{target}
whose content is specified as the latter parameter.
The provided code then forwards the processing to
\textit{main} or \textit{dest} as described in \secref{sec:forward}.

%%%%%%%%%%%%%%%%%%%%%%%%%%%%%%%%%%%%%%%%%%%%%%%%%%%%%%%%%%%%%%%%%%%%%%%%%%%%%%%%
\subsection{Include by Input}
\label{sec:input}

Including child documents by |\include| has some restrictions by design.
Most notably, the content of a child document always occupies
its own set of pages; pages cannot be shared between child documents.
Usually, this behaviour makes perfect sense
because each child document contain an essential part of the document.
However, in some situations it may be desirable to compose
a document from a collection of parts
without having mandatory page breaks between then.
For this case, the package
provides a mechanism to include parts
by |\input| which can also be processed individually.
However, by construction this mechanism
requires manual handling of the content to be output.

%%%%%%%%%%%%%%%%%%%%%%%%%%%%%%%%%%%%%%%%
\DescribeMacro{\ifchilddocmanual}
The main file should be prepared as usual, see \secref{sec:include}.
However, the document body must make a distinction
between processing of an individual part and of the main document, e.g.:
%
\begin{center}
\begin{tabular}{l}
|\ifchilddocmanual|\\
|\input{\childdocname}|\\
|\||else|\\
\textit{document body with }|\input{|\textit{part}|}|\\
|\||fi|
\end{tabular}
\end{center}
%
The conditional |\ifchilddocmanual| is true whenever
a part to be included by |\input| is being compiled,
and the name of the part is stored in |\childdocname|.

%%%%%%%%%%%%%%%%%%%%%%%%%%%%%%%%%%%%%%%%
\DescribeMacro{\childdocby}
Each part to be included by |\input| should start with:
%
\begin{center}
\begin{tabular}{l}
|\input{childdoc.def}|\\
|\childdocby{|\textit{main}|}|\\
\end{tabular}
\end{center}
%
The directive |\childdocby| is similar to |\childdocof|
described in \secref{sec:include},
but the subsequent selection of content must be done manually.
To that end, both |\ifchilddoc| and |\ifchilddocmanual|
will be true upon processing of a part,
and the name of the part is stored in |\childdocname|.
Note that |\jobname| will be set to the filename of the current part
so that each part receives an individual |.aux| file
that does not interfere with the |.aux| file(s) of the main document.
This behaviour can be altered by the alternative form
|\childdocby[*]{|\textit{main}|}| (with a non-empty optional argument)
which uses the |.aux| file of the main document
by setting |\jobname| to \textit{main}.

%%%%%%%%%%%%%%%%%%%%%%%%%%%%%%%%%%%%%%%%%%%%%%%%%%%%%%%%%%%%%%%%%%%%%%%%%%%%%%%%
\subsection{Driver Development}
\label{sec:driver}

The \textsf{childdoc} mechanism can also be use for the development
of definition files such as \LaTeX{} styles or classes.
This case differs from the above setup with multiple parts
included by |\include| in that no |\includeonly| should be invoked.
This can be achieved by starting the include file
(before |\ProvidesPackage|) with:
%
\begin{center}
\begin{tabular}{l}
|\input{childdoc.def}|\\
|\childdocforward{|\textit{main}|}|\\
\end{tabular}
\end{center}
%
or alternatively with:
%
\begin{center}
\begin{tabular}{l}
|\input{childdoc.def}|\\
|\childdocby{|\textit{main}|}|\\
\end{tabular}
\end{center}
%
Both forms have slightly different effects as described above.
The main file is prepared as usual, see \secref{sec:include}.

%%%%%%%%%%%%%%%%%%%%%%%%%%%%%%%%%%%%%%%%%%%%%%%%%%%%%%%%%%%%%%%%%%%%%%%%%%%%%%%%
\subsection{Legacy Detection}
\label{sec:detection}

The directive |\childdocmain| in the main file can detect
whether the complete document or merely a child is to be compiled
even without using the directive |\childdocof|.
This method is deprecated because it is less robust
and there is no compelling reason to use it;
it is merely provided for backward compatibility
and it may be removed in future versions.

If the detection mechanism is to be used,
it is mandatory to correctly specify
the filename of the main file as the argument of |\childdocmain|:
%
\begin{center}
\begin{tabular}{l}
|\input{childdoc.def}|\\
|\childdocmain{|\textit{main}|}|\\
\end{tabular}
\end{center}
%
If |\jobname| does not match the argument \textit{main} of |\childdocmain|,
it is assumed that |\jobname| points to the child file to be compiled.
When using |\childdocmain| with the main file specified as argument,
it suffices to start a child file
with just |\input{|\textit{main}|}|
without loading of the package and using |\childdocof|.
If instead all processing is done
with the appropriate \textsf{childdoc} directives,
the argument of \textit{main} of |\childdocmain| can be empty.

An alternative version of the command line processing described
in \secref{sec:commandline} using the detection mechanism reads:
%
\begin{center}
|... -jobname "|\textit{target}|" "|[\textit{flags}]%
[|\def\jobname{|\textit{dest}|}|]|\input{|\textit{main}|}"|
\end{center}

%%%%%%%%%%%%%%%%%%%%%%%%%%%%%%%%%%%%%%%%%%%%%%%%%%%%%%%%%%%%%%%%%%%%%%%%%%%%%%%%
\subsection{Manual Code}
\label{sec:manual}

In case one cannot be certain whether the definitions file |childdoc.def|
is installed on the target \TeX{} distribution
and one prefers not to ship it,
it is conceivable to paste a few relevant commands into the sources.

To that end, drop all statements |\input{childdoc.def}|
and perform the replacements as outlined below.
Instead of |\childdocmain{|\textit{main}|}| add the following code
to the top of the main file:
%
\begin{center}
\begin{tabular}{l}
|\||ifdefined\childdocname\endinput\||fi\newif\ifchilddoc|\\
|\edef\childdocname{\scantokens\expandafter{\jobname\noexpand}}|\\
|\def\childdocmain{|\textit{main}|}\||ifx\childdocmain\childdocname\||else|\\
|\childdoctrue\includeonly{\childdocname}\let\jobname\childdocmain\||fi|\\
\end{tabular}
\end{center}
%
Instead of |\childdocof{|\textit{main}|}| just include the main file
at the top of each child file:
%
\begin{center}
|\input{|\textit{main}|}|
\end{center}
%
A simple redirection |\childdocforward{|\textit{dest}|}| is achieved by:
%
\begin{center}
|\def\jobname{|\textit{dest}|}\input{\jobname}|
\end{center}
%
The redirection with prefix
|\childdocforwardprefix[|\textit{prefix}|]{|\textit{dest}|}|
is accomplished by:
%
\begin{center}
\begin{tabular}{l}
|{\edef\jobname{\scantokens\expandafter{\jobname\noexpand}}|\\
|\def\redirectjob |\textit{prefix}|#1~~~{\gdef\jobname{|\textit{dest}|#1}}|\\
|\expandafter\redirectjob\jobname~~~}\input{\jobname}|
\end{tabular}
\end{center}

In an alternative approach,
child documents can be compiled by a specific command line
without additional code or specific definitions:
%
\begin{center}
|... -jobname "|\textit{target}|" "|[\textit{flags}]%
|\includeonly{|\textit{dest}|}\input{|\textit{main}|}"|
\end{center}
%

%%%%%%%%%%%%%%%%%%%%%%%%%%%%%%%%%%%%%%%%%%%%%%%%%%%%%%%%%%%%%%%%%%%%%%%%%%%%%%%%
%%%%%%%%%%%%%%%%%%%%%%%%%%%%%%%%%%%%%%%%%%%%%%%%%%%%%%%%%%%%%%%%%%%%%%%%%%%%%%%%
\section{Information}

%%%%%%%%%%%%%%%%%%%%%%%%%%%%%%%%%%%%%%%%%%%%%%%%%%%%%%%%%%%%%%%%%%%%%%%%%%%%%%%%
\subsection{Copyright}

Copyright \copyright{} 2017--2018 Niklas Beisert

This work may be distributed and/or modified under the
conditions of the \LaTeX{} Project Public License, either version 1.3
of this license or (at your option) any later version.
The latest version of this license is in
  \url{http://www.latex-project.org/lppl.txt}
and version 1.3 or later is part of all distributions of \LaTeX{}
version 2005/12/01 or later.

This work has the LPPL maintenance status `maintained'.

The Current Maintainer of this work is Niklas Beisert.

This work consists of the files |README.txt|, |childdoc.ins| and |childdoc.dtx|
as well as the derived files |childdoc.def|, |cdocsamp.tex|
with |cdocsch1.tex|, |cdocsch2.tex|, |cdocspt3.tex|, |cdocspt4.tex|,
|cdocsdrf.tex|, |cdocsfn1.tex|, |cdocsfn2.tex|
as well as |childdoc.pdf|.

%%%%%%%%%%%%%%%%%%%%%%%%%%%%%%%%%%%%%%%%%%%%%%%%%%%%%%%%%%%%%%%%%%%%%%%%%%%%%%%%
\subsection{Files and Installation}

The package consists of the files:
%
\begin{center}
\begin{tabular}{ll}
    |README.txt|   & readme file \\
    |childdoc.ins| & installation file \\
    |childdoc.dtx| & source file \\
    |childdoc.def| & definition file \\
    |cdocsamp.tex| & sample main file \\
    |cdocsch1.tex| & sample include file \\
    |cdocsch2.tex| & sample include file \\
    |cdocspt3.tex| & sample part file \\
    |cdocspt4.tex| & sample part file \\
    |cdocsdrf.tex| & sample redirection file \\
    |cdocsfn1.tex| & sample redirection file \\
    |cdocsfn2.tex| & sample redirection file \\
    |childdoc.pdf| & manual
\end{tabular}
\end{center}
%
The distribution consists of the files
|README.txt|, |childdoc.ins| and |childdoc.dtx|.
%
\begin{itemize}
\item
Run (pdf)\LaTeX{} on |childdoc.dtx|
to compile the manual |childdoc.pdf| (this file).
\item
Run \LaTeX{} on |childdoc.ins| to create the definitions file |childdoc.def|
and the sample |cdocsamp.tex| with include files
|cdocsch1.tex|, |cdocsch2.tex|, |cdocspt3.tex|, |cdocspt4.tex|,
|cdocsdrf.tex|, |cdocsfn1.tex|, |cdocsfn2.tex|.
Then copy the file |childdoc.def| to an appropriate directory of your \LaTeX{}
distribution, e.g.\ \textit{texmf-root}|/tex/latex/childdoc|.
\end{itemize}

%%%%%%%%%%%%%%%%%%%%%%%%%%%%%%%%%%%%%%%%%%%%%%%%%%%%%%%%%%%%%%%%%%%%%%%%%%%%%%%%
\subsection{Related CTAN Packages}

There are several other packages which offer a similar functionality:
%
\begin{itemize}
\item
The packages
\href{http://ctan.org/pkg/docmute}{\textsf{docmute}},
\href{http://ctan.org/pkg/includex}{\textsf{includex}} and
\href{http://ctan.org/pkg/standalone}{\textsf{standalone}}
provide commands to include only the document body of
a child file thus allowing both files to be compiled individually.
\item
The packages \href{http://ctan.org/pkg/subdocs}{\textsf{subdocs}}
and \href{http://ctan.org/pkg/subfiles}{\textsf{subfiles}}
provide structures in which the main and child documents can be
encapsulated and allowing them to be compiled individually.
The inclusion mechanism is different from the conventional |\include|.
\item
The package \href{http://ctan.org/pkg/combine}{\textsf{combine}}
is an elaborate solution to combine several documents into one.
\end{itemize}
%
See also the CTAN topic \href{http://ctan.org/topic/subdocs}{\textsf{subdocs}}
for further related packages.
The present package differs from the above solutions in that
a document structure constructed with the conventional |\include| mechanism
just needs two extra commands at the top of every file
such that all constituent files can be compiled individually.

%%%%%%%%%%%%%%%%%%%%%%%%%%%%%%%%%%%%%%%%%%%%%%%%%%%%%%%%%%%%%%%%%%%%%%%%%%%%%%%%
%\subsection{Feature Suggestions}
%
%The following is a list of features which may be useful for future
%versions of this package:
%%
%\begin{itemize}
%\item
%\ldots
%\end{itemize}

%%%%%%%%%%%%%%%%%%%%%%%%%%%%%%%%%%%%%%%%%%%%%%%%%%%%%%%%%%%%%%%%%%%%%%%%%%%%%%%%
\subsection{Revision History}

%%%%%%%%%%%%%%%%%%%%%%%%%%%%%%%%%%%%%%%%
\paragraph{v2.0:} 2018/12/30

\begin{itemize}
\item
immediate forward processing
\item
added |\childdocby| mechanism
\item
manual restructured
\end{itemize}

%%%%%%%%%%%%%%%%%%%%%%%%%%%%%%%%%%%%%%%%
\paragraph{v1.6:} 2018/01/17

\begin{itemize}
\item
application for development of include files
\item
corrections to manual
\end{itemize}

%%%%%%%%%%%%%%%%%%%%%%%%%%%%%%%%%%%%%%%%
\paragraph{v1.5:} 2017/05/21

\begin{itemize}
\item
more complete structuring introduced
\item
|\childdocof| introduced
\item
|\childdoc| renamed to |\childdocmain|
\item
|\childredirect| renamed to |\childdocforward| and |\childdocforwardprefix|
and functionality expanded
\end{itemize}

%%%%%%%%%%%%%%%%%%%%%%%%%%%%%%%%%%%%%%%%
\paragraph{v1.0:} 2017/04/27

\begin{itemize}
\item
manual and install package
\item
first version published on CTAN
\end{itemize}

%%%%%%%%%%%%%%%%%%%%%%%%%%%%%%%%%%%%%%%%
\paragraph{v0.6:} 2017/04/26

\begin{itemize}
\item
redirection mechanism added
\end{itemize}

%%%%%%%%%%%%%%%%%%%%%%%%%%%%%%%%%%%%%%%%
\paragraph{v0.5:} 2017/04/26

\begin{itemize}
\item
functionality in definition file
\end{itemize}


%%%%%%%%%%%%%%%%%%%%%%%%%%%%%%%%%%%%%%%%%%%%%%%%%%%%%%%%%%%%%%%%%%%%%%%%%%%%%%%%
%%%%%%%%%%%%%%%%%%%%%%%%%%%%%%%%%%%%%%%%%%%%%%%%%%%%%%%%%%%%%%%%%%%%%%%%%%%%%%%%
%%%%%%%%%%%%%%%%%%%%%%%%%%%%%%%%%%%%%%%%%%%%%%%%%%%%%%%%%%%%%%%%%%%%%%%%%%%%%%%%
\appendix

\settowidth\MacroIndent{\rmfamily\scriptsize 000\ }

 \DocInput{childdoc.dtx}

\end{document}
%</driver>
% \fi
%
% %%%%%%%%%%%%%%%%%%%%%%%%%%%%%%%%%%%%%%%%%%%%%%%%%%%%%%%%%%%%%%%%%%%%%%%%%%%%%%
% %%%%%%%%%%%%%%%%%%%%%%%%%%%%%%%%%%%%%%%%%%%%%%%%%%%%%%%%%%%%%%%%%%%%%%%%%%%%%%
% \section{Sample}
%\iffalse
%<*samplemain>
%\fi
%
% The following presents a sample document
% with two chapters, two parts, a title page,
% a compile flag as well as three forwarding files to set the flag.
% It consists of eight |.tex| files:
% \begin{center}
% \begin{tabular}{ll}
% |cdocsamp.tex|&main file\\
% |cdocsch1.tex|&include file for chapter 1\\
% |cdocsch2.tex|&include file for chapter 2\\
% |cdocspt3.tex|&include file for part 3\\
% |cdocspt4.tex|&include file for part 4\\
% |cdocsdrf.tex|&forwarding file for main file in draft mode\\
% |cdocsfi1.tex|&forwarding file for final version of chapter 1\\
% |cdocsfi2.tex|&forwarding file for final version of chapter 2\\
% \end{tabular}
% \end{center}
% Each of the eight files can be compiled directly by the \LaTeX{} compiler.
%
% %%%%%%%%%%%%%%%%%%%%%%%%%%%%%%%%%%%%%%
% \paragraph{Main File.}
%
% The main file is called |cdocsamp.tex|.
%
% Load the \textsf{childdoc} definitions and
% declare the filename for the main document:
%    \begin{macrocode}
\input{childdoc.def}
\childdocmain{}
%    \end{macrocode}

% Optional override for |\version| flag:
%    \begin{macrocode}
%%\ifchilddoc\else\providecommand{\version}{draft}\fi
%    \end{macrocode}

% Define the default values for the |\version| flag
% (|final| for the main file and |draft| for childs):
%    \begin{macrocode}
\ifchilddoc
\providecommand{\version}{draft}
\else
\providecommand{\version}{final}
\fi
%    \end{macrocode}

% Load the standard document class:
%    \begin{macrocode}
\documentclass[12pt]{article}
%    \end{macrocode}

% Start the document body:
%    \begin{macrocode}
\begin{document}
%    \end{macrocode}

% Declare a title page.
% Print title, part of document being processed and version flag:
%    \begin{macrocode}
\addtocounter{page}{-1}
\begin{center}
{\LARGE\bfseries{}childdoc example\par}
\vspace{1cm}
\ifchilddoc
\ifchilddocmanual part\else chapter\fi:
`\childdocname' of `\childdocjob'\par
\else
main document: `\childdocjob'\par
\fi
version: \version\par
\end{center}
\newpage
%    \end{macrocode}

% Manually include selected file,
% otherwise process as usual:
%    \begin{macrocode}
\ifchilddocmanual
\section*{part `\childdocname'}
\input{\childdocname}
\else
%    \end{macrocode}

% Include the two chapters:
%    \begin{macrocode}
\include{cdocsch1}
\include{cdocsch2}
%    \end{macrocode}

% Include the two parts unless only chapters should be displayed:
%    \begin{macrocode}
\ifchilddoc\else
\section{part three}
\input{cdocspt3}
\section{part four}
\input{cdocspt4}
\fi
%    \end{macrocode}

% Process as usual until here:
%    \begin{macrocode}
\fi
%    \end{macrocode}

% End of document body:
%    \begin{macrocode}
\end{document}
%    \end{macrocode}
%\iffalse
%</samplemain>
%\fi
%
% %%%%%%%%%%%%%%%%%%%%%%%%%%%%%%%%%%%%%%
% \paragraph{Chapter Include Files.}
%
% The include files are called |cdocsch1.tex| and |cdocsch2.tex|.
%
%\iffalse
%<*samplechap1|samplechap2>
%\fi

% Optional override for |\version| flag:
%    \begin{macrocode}
%%\providecommand{\version}{final}
%    \end{macrocode}

% Include the main document:
%    \begin{macrocode}
\input{childdoc.def}
\childdocof{cdocsamp}
%    \end{macrocode}

%\iffalse
%</samplechap1|samplechap2>
%\fi
%
%\iffalse
%<*samplechap1>
%\fi
% Some text for chapter 1:
%    \begin{macrocode}
\section{one}
some text in chapter one
%    \end{macrocode}

%\iffalse
%</samplechap1>
%\fi
% Some text for chapter 2:
%\iffalse
%<*samplechap2>
%\fi
%    \begin{macrocode}
\section{two}
more text in chapter two
%    \end{macrocode}

%\iffalse
%</samplechap2>
%\fi
%
% %%%%%%%%%%%%%%%%%%%%%%%%%%%%%%%%%%%%%%
% \paragraph{Part Include Files.}
%
% The include files are called |cdocspt3.tex| and |cdocspt4.tex|.
%
%\iffalse
%<*samplepart3|samplepart4>
%\fi

% Optional override for |\version| flag:
%    \begin{macrocode}
%%\providecommand{\version}{final}
%    \end{macrocode}

% Include the main document:
%    \begin{macrocode}
\input{childdoc.def}
\childdocby{cdocsamp}
%    \end{macrocode}

%\iffalse
%</samplepart3|samplepart4>
%\fi
%
%\iffalse
%<*samplepart3>
%\fi
% Some text for part 3:
%    \begin{macrocode}
some text in part three
%    \end{macrocode}

%\iffalse
%</samplepart3>
%\fi
% Some text for part 4:
%\iffalse
%<*samplepart4>
%\fi
%    \begin{macrocode}
more text in part four
%    \end{macrocode}

%\iffalse
%</samplepart4>
%\fi
%
% %%%%%%%%%%%%%%%%%%%%%%%%%%%%%%%%%%%%%%
% \paragraph{Forwarding for a Complete Draft.}
%
% The following forwarding file |cdocsdrf.tex|
% compiles the main document in draft mode:
%\iffalse
%<*sampledraft>
%\fi
%    \begin{macrocode}
\def\version{draft}
\input{childdoc.def}
\childdocforward{cdocsamp}
%    \end{macrocode}

%\iffalse
%</sampledraft>
%\fi
%
% %%%%%%%%%%%%%%%%%%%%%%%%%%%%%%%%%%%%%%
% \paragraph{Forwarding for Final Version of the Chapters.}
%
% The following forwarding files |cdocsfn1.tex| and |cdocsfn2.tex|
% (with identical content)
% compile the final versions of the child documents
% |cdocsch1.tex| and |cdocsch2.tex|, respectively:
%\iffalse
%<*samplefinal>
%\fi
%    \begin{macrocode}
\def\version{final}
\input{childdoc.def}
\childdocforwardprefix[cdocsamp]{cdocsfn}{cdocsch}
%    \end{macrocode}

%\iffalse
%</samplefinal>
%\fi
%
% %%%%%%%%%%%%%%%%%%%%%%%%%%%%%%%%%%%%%%
% \paragraph{Command Line Processing.}
%
% The following three command lines generate the output files
% |cdocscld|, |cdocscl1| and |cdocscl2|
% which should be identical to
% |cdocsdrf|, |cdocsch1| and |cdocsfn2|, respectively:
% \begin{center}
% \begin{tabular}{l}
% |latex -jobname cdocscld \|\\
% |  "\def\version{draft}\input{childdoc.def}\childdocforward{cdocsamp}"|\\
% |latex -jobname cdocscl1 \|\\
% |  "\input{childdoc.def}\childdocforward[cdocsamp]{cdocsch1}"|\\
% |latex -jobname cdocscl2 \|\\
% |  "\def\version{final}\input{childdoc.def}\childdocforward{cdocsch2}"|
% \end{tabular}
% \end{center}
% Note that the trailing backslash on each first line
% merely continues the input to the second line
% (for convenient cut ant paste).
% Furthermore, the command |latex| can be replaced by any
% of its alternative versions such as |pdflatex|.
%
% %%%%%%%%%%%%%%%%%%%%%%%%%%%%%%%%%%%%%%%%%%%%%%%%%%%%%%%%%%%%%%%%%%%%%%%%%%%%%%
% %%%%%%%%%%%%%%%%%%%%%%%%%%%%%%%%%%%%%%%%%%%%%%%%%%%%%%%%%%%%%%%%%%%%%%%%%%%%%%
% \section{Implementation}
%\iffalse
%<*package>
%\fi
%
% This section describes the definitions file |childdoc.def|.

% The definitions cannot be loaded using |\usepackage| or |\RequirePackage|
% which has a mechanism to prevent loading a style file more than once.
% When loading the definitions by means of |\input|
% multiple instances have to be prevented manually:
%\iffalse
%This code needs to be before the `\ProvidesFile' directive
%which is defined at the beginning of this file.
%Therefore it is also placed there and commented out here.
%</package>
%<*discard>
%\fi
%    \begin{macrocode}
\ifdefined\childdocmain\endinput\fi
%    \end{macrocode}
%\iffalse
%</discard>
%<*package>
%\fi
%
% \macro{\ifchilddoc}
% \macro{\ifchilddocmanual}
% The conditional |\ifchilddoc| tells whether a
% child (true) or main (false) document is being compiled.
% The conditional |\ifchilddocmanual| tells whether
% the |\includeonly| mechanism is used (false) or
% the selection of child files must be performed manually (true).
% The definitions initialise to false:
%    \begin{macrocode}
\newif\ifchilddoc
\newif\ifchilddocmanual
%    \end{macrocode}

% \macro{\childdocname}
% \macro{\childdocjob}
% The macro |\childdocname| stores the name of the main document
% to be compiled. The macro |\childdocjob| stores the name of
% the document on which the \LaTeX{} compiler was originally invoked.
% The content of |\jobname| cannot be compared
% to filenames specified in the source due to different catcodes.
% The following code rescans |\jobname|, stores the result
% in |\childdocname| and saves a copy in |\childdocjob|:
%    \begin{macrocode}
\edef\childdocname{\scantokens\expandafter{\jobname\noexpand}}
\let\childdocjob\childdocname
%    \end{macrocode}

% \macro{\childdocdisable}
% The macro |\childdocdisable| prevents the main file
% from being processed more than once.
% At this stage, the main document command |\childdocmain|
% is assumed to be called once again where it should do nothing.
% Any subsequent call to it should prevent
% a secondary processing of the main document
% It overwrites the forwarding commands
% |\childdocof| and |\childdocforward|
% with empty macros to prevent further inclusions of the main document:
%    \begin{macrocode}
\newcommand{\childdocdisable}
{
  \renewcommand{\childdocmain}[1]{\renewcommand{\childdocmain}[1]{\endinput}}
  \renewcommand{\childdocof}[1]{}
  \renewcommand{\childdocby}[2][]{}
  \renewcommand{\childdocforward}[2][]{}
  \renewcommand{\childdocdisable}{}
}
%    \end{macrocode}

% \macro{\childdocmain}
% The macro |\childdocmain| is to be called at the top of the main file
% with nothing or the main filename (without extension) as argument.
% First, it breaks loops.
% If the argument is not empty and does not match |\childdocname|
% (which is set by the first inclusion of |childdoc.def|),
% |\ifchilddoc| is set to true, |\includeonly| is applied to the child file
% and |\jobname| is set to the main file
% (for proper handling of |.aux| files):
%    \begin{macrocode}
\newcommand{\childdocmain}[1]
{
  \childdocdisable\childdocmain{}
  \if?#1?\else
    \begingroup
      \def\childdoctmp{#1}
      \ifx\childdoctmp\childdocname
        \def\childdoctmp{}
      \else
        \def\childdoctmp
        {
          \childdoctrue
          \includeonly{\childdocname}
          \def\childdocjob{#1}
          \def\jobname{#1}
        }
      \fi
      \expandafter
    \endgroup
    \childdoctmp
  \fi
}
%    \end{macrocode}

% \macro{\childdocof}
% The command |\childdocof| redirects
% compilation to the main file |#1|.
%    \begin{macrocode}
\newcommand{\childdocof}[1]
{
  \childdocdisable
  \childdoctrue
  \includeonly{\childdocname}
  \def\jobname{#1}
  \def\childdocjob{#1}
  \input{#1}
}
%    \end{macrocode}

% \macro{\childdocby}
% The command |\childdocby| ....
%    \begin{macrocode}
\newcommand{\childdocby}[2][]
{
  \childdocdisable
  \childdoctrue
  \childdocmanualtrue
  \if?#1?\else
    \def\jobname{#2}
  \fi
  \def\childdocjob{#2}
  \input{#2}
  \endinput
}
%    \end{macrocode}

% \macro{\childdocforward}
% The command |\childdocforward| redirects
% compilation to the main file or
% (if the optional argument is given) a child file.
% Parameters are set as if the main file
% or a child file starting with |\childdocof| was compiled.
% Then compilation is handed over to the main file:
%    \begin{macrocode}
\newcommand{\childdocforward}[2][]
{
  \begingroup
    \if?#1?
      \def\childdoctmp
      {
        \def\childdocname{#2}
        \def\childdocjob{#2}
        \def\jobname{#2}
        \input{#2}
        \endinput
      }
    \else
      \def\childdoctmp
      {
        \childdocdisable
        \def\childdocname{#2}
        \childdoctrue
        \includeonly{#2}
        \def\childdocjob{#1}
        \def\jobname{#1}
        \input{#1}
        \endinput
      }
    \fi
    \expandafter
  \endgroup
  \childdoctmp
}
%    \end{macrocode}

% \macro{\childdocforwardprefix}
% The command |\childdocforwardprefix| redirects
% compilation to the main or a child file by means of a pattern.
% The prefix |#1| in the current filename is replaced by |#2|
% and the suffix of the current filename is kept
% (it is assumed that the filename does not contain the substring `|~~~|'
% which is used as a delimiter).
% Compilation is handed over to the new file by |\childdocforward|:
%    \begin{macrocode}
\newcommand{\childdocforwardprefix}[3][]
{
  \begingroup
    \def\childdocextract #2##1~~~{\def\childdoctmp{\childdocforward[#1]{#3##1}}}
    \expandafter\childdocextract\childdocname~~~
    \expandafter
  \endgroup
  \childdoctmp
}
%    \end{macrocode}

% \macro{\childdoc}
% The deprecated macro |\childdoc| is a legacy version of |\childdocmain|:
%    \begin{macrocode}
\newcommand{\childdoc}{\childdocmain}
%    \end{macrocode}

% \macro{\childdocredirect}
% The deprecated macro |\childdocredirect| is a legacy version
% of |\childdocforward| and |\childdocforwardprefix|:
%    \begin{macrocode}
\newcommand{\childdocredirect}[2][]
{
  \begingroup
    \if?#1?
      \def\childdoctmp{\childdocforward{#2}}
    \else
      \def\childdoctmp{\childdocforwardprefix{#1}{#2}}
    \fi
    \expandafter
  \endgroup
  \childdoctmp
}
%    \end{macrocode}

%\iffalse
%</package>
%\fi
%
\endinput
|\\
|\childdocforwardprefix[|\textit{main}|]{|\textit{prefix}|}{|\textit{dest}|}|
\end{tabular}
\end{center}
%
the destination file is determined by a pattern
depending on the current file:
To make this work, the current file must be called
`{\textit{prefix}\hspace{0.2em}\textit{suffix}}'
with \textit{prefix} matching precisely the argument.
Processing is then passed on to the file
`{\textit{dest}\hspace{0.2em}\textit{suffix}}'.
Surely, the same effect is achieved by
directly specifying the
argument `{\textit{dest}\hspace{0.2em}\textit{suffix}}'
in the first form.
However, that requires to set up a different file
for each child. With the alternative form of the command
all these files can have exactly the same content
which simplifies setting them up and maintaining them.

For example, the following file |draft.tex|
with a compilation flag |\version| as described in \secref{sec:flags}
compiles the main document as a draft:
%
\begin{center}
\begin{tabular}{l}
|\def\version{draft}|\\
|% \iffalse
%
% childdoc.dtx Copyright (C) 2017-2018 Niklas Beisert
%
% This work may be distributed and/or modified under the
% conditions of the LaTeX Project Public License, either version 1.3
% of this license or (at your option) any later version.
% The latest version of this license is in
%   http://www.latex-project.org/lppl.txt
% and version 1.3 or later is part of all distributions of LaTeX
% version 2005/12/01 or later.
%
% This work has the LPPL maintenance status `maintained'.
%
% The Current Maintainer of this work is Niklas Beisert.
%
% This work consists of the files childdoc.dtx and childdoc.ins
% and the derived files childdoc.def and cdocsamp.tex with
% cdocsch1.tex, cdocsch2.tex, cdocsdrf.tex, cdocsfn1.tex, cdocsfn2.tex.
%
%<package>\ifdefined\childdocmain\endinput\fi
%<package>\ProvidesFile{childdoc.def}[2018/12/30 v2.0 child document driver]
%<samplemain>\ProvidesFile{cdocsamp.tex}[2018/12/30 v2.0 sample for childdoc]
%<*driver>
%\ProvidesFile{childdoc.drv}[2018/12/30 v2.0 childdoc reference manual file]
\PassOptionsToClass{10pt,a4paper}{article}
\documentclass{ltxdoc}

\usepackage[margin=35mm]{geometry}
\usepackage{hyperref}
\usepackage{hyperxmp}
\usepackage[usenames]{color}

\hypersetup{colorlinks=true}
\hypersetup{pdfstartview=FitH}
\hypersetup{pdfpagemode=UseNone}
\hypersetup{pdfsource={}}
\hypersetup{pdflang={en-UK}}
\hypersetup{pdfcopyright={Copyright 2017-2018 Niklas Beisert.
  This work may be distributed and/or modified under the
  conditions of the LaTeX Project Public License, either version 1.3
  of this license or (at your option) any later version.}}
\hypersetup{pdflicenseurl={http://www.latex-project.org/lppl.txt}}
\hypersetup{pdfcontactaddress={ETH Zurich, ITP, HIT K,
  Wolfgang-Pauli-Strasse 27}}
\hypersetup{pdfcontactpostcode={8093}}
\hypersetup{pdfcontactcity={Zurich}}
\hypersetup{pdfcontactcountry={Switzerland}}
\hypersetup{pdfcontactemail={nbeisert@itp.phys.ethz.ch}}
\hypersetup{pdfcontacturl={http://people.phys.ethz.ch/\xmptilde nbeisert/}}

\newcommand{\secref}[1]{\hyperref[#1]{section \ref*{#1}}}

\parskip1ex
\parindent0pt
\let\olditemize\itemize
\def\itemize{\olditemize\parskip0pt}

\begin{document}

\title{The \textsf{childdoc} Package}
\hypersetup{pdftitle={The childdoc Package}}
\author{Niklas Beisert\\[2ex]
  Institut f\"ur Theoretische Physik\\
  Eidgen\"ossische Technische Hochschule Z\"urich\\
  Wolfgang-Pauli-Strasse 27, 8093 Z\"urich, Switzerland\\[1ex]
  \href{mailto:nbeisert@itp.phys.ethz.ch}
  {\texttt{nbeisert@itp.phys.ethz.ch}}}
\hypersetup{pdfauthor={Niklas Beisert}}
\hypersetup{pdfsubject={Manual for the LaTeX2e Package childdoc}}
\date{30 December 2018, \textsf{v2.0}}
\maketitle

\begin{abstract}\noindent
\textsf{childdoc} is a \LaTeXe{} package
that enables the direct compilation
of document sections included by |\include|
to individual files.
\end{abstract}

\begingroup
\parskip0ex
\tableofcontents
\endgroup

%%%%%%%%%%%%%%%%%%%%%%%%%%%%%%%%%%%%%%%%%%%%%%%%%%%%%%%%%%%%%%%%%%%%%%%%%%%%%%%%
%%%%%%%%%%%%%%%%%%%%%%%%%%%%%%%%%%%%%%%%%%%%%%%%%%%%%%%%%%%%%%%%%%%%%%%%%%%%%%%%
\section{Introduction}

\LaTeX{} provides a mechanism to structure a large document (such as a book)
into a main file and several child files (containing the chapters)
using the |\include| command.
This mechanism is beneficial for documents
which span hundreds of pages in order to
make the source file(s) more manageable.
Moreover, compilation can be restricted to
selected child files by means of the |\includeonly| command.
The latter feature can be used to reduce the compilation time while editing
(this was significantly more useful in the earlier days of \LaTeX{})
or to generate a smaller document which is easier to navigate.
Another application of |\includeonly| is to generate
documents consisting of selected parts of the complete document.

However, there are a few drawbacks of the plain |\include| mechanism:
\begin{itemize}
\item
The child files cannot be compiled on their own,
they can only be compiled via the main file.
A naive editing environment
(such as a text editor with an option
to have the current file processed by \LaTeX)
may require one to switch to the main file before compiling;
attempting to compile the child file produces errors.
\item
The main file must be modified (each time)
to adjust the |\includeonly| command
to the present needs. This easily leaves the main file in a messy state.
\item
The generated document will always carry the filename
of the main document. This is inconvenient if
several child files are to be compiled and
to be kept for distribution.
\end{itemize}

The present package provides a simple interface
to make child files individually compilable by \LaTeX{}.
Compiling a child file then has the same effect as compiling
the main file with an |\includeonly| command
to select the appropriate child.
Moreover the generated document will carry the name of the child
rather than the main file.
This resolves all three above issues.

This feature is meant to make the editing of books,
thesis documents and lecture notes somewhat more convenient.
However, the package can also be used efficiently for
composing a series of documents (such as exercise sheets)
which are typically distributed individually.
It then assists the author in generating the individual documents
(potentially in different versions)
as well as a document containing the collected series.
Another application is in developing style files
or other kinds of included material
where compilation of the style file could redirect
to a sample or test file.

%%%%%%%%%%%%%%%%%%%%%%%%%%%%%%%%%%%%%%%%%%%%%%%%%%%%%%%%%%%%%%%%%%%%%%%%%%%%%%%%
%%%%%%%%%%%%%%%%%%%%%%%%%%%%%%%%%%%%%%%%%%%%%%%%%%%%%%%%%%%%%%%%%%%%%%%%%%%%%%%%
\section{Usage}

First of all, the package \textsf{childdoc} is \emph{not} a standard
\LaTeXe{} |.sty| style file! Therefore it needs to be invoked in
a non-standard way.

%%%%%%%%%%%%%%%%%%%%%%%%%%%%%%%%%%%%%%%%%%%%%%%%%%%%%%%%%%%%%%%%%%%%%%%%%%%%%%%%
\subsection{Included Files}
\label{sec:include}

%%%%%%%%%%%%%%%%%%%%%%%%%%%%%%%%%%%%%%%%
\DescribeMacro{\childdocmain}
To use the package, add the commands
\begin{center}
\begin{tabular}{l}
|\input{childdoc.def}|\\
|\childdocmain{}|\\
\end{tabular}
\end{center}
at the very top of the main \LaTeX{} file,
in particular \emph{before} the |\documentclass| statement!
The argument of |\childdocmain| should be left empty
(but it must be present).

%%%%%%%%%%%%%%%%%%%%%%%%%%%%%%%%%%%%%%%%
\DescribeMacro{\childdocof}
Furthermore, add the commands
\begin{center}
\begin{tabular}{l}
|\input{childdoc.def}|\\
|\childdocof{|\textit{main}|}|\\
\end{tabular}
\end{center}
at the top of every child file \textit{child}
which is included by |\include{|\textit{child}|}|
from within the main file
(or at least for those files to be compiled individually).
The argument \textit{main} must be the filename of the main file.

There are a couple of
considerations in setting up the main and child documents:

%%%%%%%%%%%%%%%%%%%%%%%%%%%%%%%%%%%%%%%%
\paragraph{Restrictions.}

Please note the following restrictions:
\begin{itemize}
\item
|\childdocmain| must be called with one argument \textit{main}
to ensure compatibility with earlier version of the package.
It must either be empty (|\childdocmain{}|)
or precisely match the filename of the main file in which it is specified.
See \secref{sec:detection} for further information.
\item
The filename \textit{main} must be specified without the |.tex| extension.
\item
The filename \textit{main} is case sensitive
(even in case-insensitive file systems)
due to internal string comparison.
\item
The argument \textit{main} should be fully expanded, it cannot be a macro.
\item
Subdirectories and special characters should be avoided in filenames.
\item
The command |\childdocmain{|\textit{main}|}| must be followed by a whitespace.
It should not be followed immediately by another command
or by a comment mark `|%|'.
This is because the \TeX{} parser reads the token immediately following
the argument of |\childdocmain| and puts it
at the beginning of every child section;
however, a white\-space is ignored.
\end{itemize}

%%%%%%%%%%%%%%%%%%%%%%%%%%%%%%%%%%%%%%%%
\paragraph{Content of Main File.}

It is advisable to place all content in the child files included by |\include|.
Any output contained in the main file will appear in all child documents
unless suppressed manually;
it cannot be suppressed automatically by the |\includeonly| directive
and thus should normally be avoided.
A method to include some content in the main file
by means of conditional processing is described in \secref{sec:conditional}.

%%%%%%%%%%%%%%%%%%%%%%%%%%%%%%%%%%%%%%%%
\paragraph{Page Numbering.}

When only a part of the document is compiled,
the appropriate numbering of pages
(as well as other status parameters)
is determined from the |.aux| files.
The latter contain information from previous passes.
However this information needs to propagate through
all intermediate child documents.
Therefore the page numbering in child documents may well
be inconsistent until the complete document is compiled at least once.

A useful (if unconventional) way to always ensure a consistent
page numbering is to restart the numbering in each child document
and denote the pages by `\textit{child}|.|\textit{page}'
where \textit{child} represents the chapter/section number of the child file.
This can be achieved by the command
|\numberwithin{page}{|\textit{child}|}|
of the \textsf{amsmath} package
where \textit{child} can be |chapter| or |section|
depending on the chosen structuring.
Alternatively, one can modify the macro |\thepage| appropriately
and reset the counter |page| at the start of each child file.

%%%%%%%%%%%%%%%%%%%%%%%%%%%%%%%%%%%%%%%%%%%%%%%%%%%%%%%%%%%%%%%%%%%%%%%%%%%%%%%%
\subsection{Conditional Processing}
\label{sec:conditional}

The package provides a mechanism to compile different versions
of a document. To customise the versions further some conditional processing
can come in handy to distinguish which version is being compiled.
The package provides two macros to describe the compilation context:

%%%%%%%%%%%%%%%%%%%%%%%%%%%%%%%%%%%%%%%%
\DescribeMacro{\ifchilddoc}
The conditional |\ifchilddoc| distinguishes between the compilation of
child documents and the main document:
%
\begin{center}
|\ifchilddoc |\textit{child-code}| |[|\||else |\textit{main-code}]| \||fi|
\end{center}

%%%%%%%%%%%%%%%%%%%%%%%%%%%%%%%%%%%%%%%%
\DescribeMacro{\childdocname}
\DescribeMacro{\childdocjob}
The macro |\childdocname| contains the filename (without extension)
of the main or child file being processed.
Note that |\childdocjob| will always contain the name of the main file.

%%%%%%%%%%%%%%%%%%%%%%%%%%%%%%%%%%%%%%%%
\paragraph{Title Page.}

Conditional processing can be used to include a title or banner page
in the main document when proper precautions are taken.
Importantly, the code in the main file should ensure that the page counter
(as well as other status parameters which are stored in the |.aux| files)
takes the same value after the conditional processing.
Otherwise the page numbers may take divergent values
depending on which part is compiled.

For example, a title page could be declared by:
%
\begin{center}
\begin{tabular}{l}
|\ifchilddoc\||else|\\
|\addtocounter{page}{-1}|\\
\textit{code for title page}\\
|\newpage|\\
|\||fi|
\end{tabular}
\end{center}
%
A banner page for the child documents can be generated by:
%
\begin{center}
\begin{tabular}{l}
|\ifchilddoc|\\
|\addtocounter{page}{-1}|\\
\textit{code for banner page}\\
|\newpage|\\
|\||fi|
\end{tabular}
\end{center}
%
Here one could write a message such as:
\begin{center}
|This is the part \childdocname{} of \childdocjob{}.|
\end{center}

%%%%%%%%%%%%%%%%%%%%%%%%%%%%%%%%%%%%%%%%%%%%%%%%%%%%%%%%%%%%%%%%%%%%%%%%%%%%%%%%
\subsection{Flags}
\label{sec:flags}

The package makes it easy to generate different versions
of the main or child documents.
To this end compilation flags can be defined
and assigned different default values.
They will be particularly useful in conjunction
with the forwarding mechanism described in \secref{sec:forward}.

For example, it may be useful to have a flag |\version|
which can be set to |draft| or |final|.
The document source will contain some conditional code
depending on the value of |\version|.
Suppose further, the flag should default to |final| for the main file
and to |draft| for child files
which is a natural assignment for editing the document.
This is achieved by placing the following code
in the preamble of the main document
(below the |\childdocmain| directive):
%
\begin{center}
\begin{tabular}{l}
|\ifchilddoc|\\
|\providecommand{\version}{draft}|\\
|\||else|\\
|\providecommand{\version}{final}|\\
|\||fi|
\end{tabular}
\end{center}
%
The definition by |\providecommand| makes sure
that previous definitions are not overwritten.
Further statements |\providecommand{\version}{...}|
can thus be added before the above code to override it.

For the main file, one might add a line
(between |\childdocmain| and the above block)
%
\begin{center}
|%\ifchilddoc\||else\providecommand{\version}{draft}\||fi|
\end{center}
%
which can be uncommented to produce a draft version.
Likewise one can add a line to the very top of a child file
(above the |\childdocof{|\textit{main}|}| directive)
%
\begin{center}
|%\providecommand{\version}{final}|
\end{center}
%
which can be uncommented to produce the final version of this child document.

%%%%%%%%%%%%%%%%%%%%%%%%%%%%%%%%%%%%%%%%%%%%%%%%%%%%%%%%%%%%%%%%%%%%%%%%%%%%%%%%
\subsection{Forwarding}
\label{sec:forward}

Different versions of the main or child documents
using compilation flags as described in \secref{sec:flags}
can be (permanently) stored in different files
for convenient compilation, viewing and distribution.
To this end, the package defines a command
to pass on compilation to a different file:

%%%%%%%%%%%%%%%%%%%%%%%%%%%%%%%%%%%%%%%%
\DescribeMacro{\childdocforward}
The command |\childdocforward| redirects processing to
another source file:
%
\begin{center}
\begin{tabular}{l}
|\input{childdoc.def}|\\
|\childdocforward[|\textit{main}|]{|\textit{dest}|}|\\
\end{tabular}
\end{center}
%
The argument \textit{dest} is the destination file
(without extension).
It should be the main file or one of the child files.
Note that further \textsf{childdoc} directives
such as |\childdocof| and |\childdocforward|
in the indicated file will be processed in this form.
The optional argument \textit{main}
passes on directly to the main file \textit{main}
while pretending to compile the child \textit{dest}.
This form behaves as if \textit{dest}
issues |\childdocof{|\textit{main}|}| right away,
and no further \textsf{childdoc} directives will be processed.

%%%%%%%%%%%%%%%%%%%%%%%%%%%%%%%%%%%%%%%%
\DescribeMacro{\...prefix}
In the alternative form |\childdocforwardprefix|,
%
\begin{center}
\begin{tabular}{l}
|\input{childdoc.def}|\\
|\childdocforwardprefix[|\textit{main}|]{|\textit{prefix}|}{|\textit{dest}|}|
\end{tabular}
\end{center}
%
the destination file is determined by a pattern
depending on the current file:
To make this work, the current file must be called
`{\textit{prefix}\hspace{0.2em}\textit{suffix}}'
with \textit{prefix} matching precisely the argument.
Processing is then passed on to the file
`{\textit{dest}\hspace{0.2em}\textit{suffix}}'.
Surely, the same effect is achieved by
directly specifying the
argument `{\textit{dest}\hspace{0.2em}\textit{suffix}}'
in the first form.
However, that requires to set up a different file
for each child. With the alternative form of the command
all these files can have exactly the same content
which simplifies setting them up and maintaining them.

For example, the following file |draft.tex|
with a compilation flag |\version| as described in \secref{sec:flags}
compiles the main document as a draft:
%
\begin{center}
\begin{tabular}{l}
|\def\version{draft}|\\
|\input{childdoc.def}|\\
|\childdocforward{|\textit{main}|}|
\end{tabular}
\end{center}
%
Likewise, the following files |final|\textit{nn}|.tex|
compile the final version of the child document
|child|\textit{nn}|.tex|:
%
\begin{center}
\begin{tabular}{l}
|\def\version{final}|\\
|\input{childdoc.def}|\\
|\childdocforwardprefix{final}{child}|
\end{tabular}
\end{center}
%

Note that when several versions of a main file and/or of each child file
are to be generated, it may be convenient to set up a |Makefile| or
shell script to automatise the process.

%%%%%%%%%%%%%%%%%%%%%%%%%%%%%%%%%%%%%%%%%%%%%%%%%%%%%%%%%%%%%%%%%%%%%%%%%%%%%%%%
\subsection{Command Line Processing}
\label{sec:commandline}

The effect of redirection files can also be achieved by invoking
the \LaTeX{} compiler with a more elaborate command line.
Most conveniently this should be done as part
of a shell script or a |Makefile|.

When using \textsf{childdoc} in the main file, the following
command lines effectively perform a redirection
(note that depending on the shell being used,
backslashes may have to be doubled: `|\|' $\to$ `|\\|'):
%
\begin{center}
|... -jobname "|\textit{target}|" |\\|"|[\textit{flags}]%
|\input{childdoc.def}\childdocforward[|\textit{main}|]{|\textit{dest}|}"|
\end{center}
%
Here \textit{target} is the name of the output file,
\textit{main} is the name of the main file
and \textit{dest} is the name of the main or child file to be processed
(all filenames without extensions).
The optional argument \textit{main} can be omitted
if \textit{main} matches \textit{dest}.
Optionally, compilation \textit{flags} can be defined via |\def| commands.
This command line makes the \TeX{} engine believe
it is compiling the file \textit{target}
whose content is specified as the latter parameter.
The provided code then forwards the processing to
\textit{main} or \textit{dest} as described in \secref{sec:forward}.

%%%%%%%%%%%%%%%%%%%%%%%%%%%%%%%%%%%%%%%%%%%%%%%%%%%%%%%%%%%%%%%%%%%%%%%%%%%%%%%%
\subsection{Include by Input}
\label{sec:input}

Including child documents by |\include| has some restrictions by design.
Most notably, the content of a child document always occupies
its own set of pages; pages cannot be shared between child documents.
Usually, this behaviour makes perfect sense
because each child document contain an essential part of the document.
However, in some situations it may be desirable to compose
a document from a collection of parts
without having mandatory page breaks between then.
For this case, the package
provides a mechanism to include parts
by |\input| which can also be processed individually.
However, by construction this mechanism
requires manual handling of the content to be output.

%%%%%%%%%%%%%%%%%%%%%%%%%%%%%%%%%%%%%%%%
\DescribeMacro{\ifchilddocmanual}
The main file should be prepared as usual, see \secref{sec:include}.
However, the document body must make a distinction
between processing of an individual part and of the main document, e.g.:
%
\begin{center}
\begin{tabular}{l}
|\ifchilddocmanual|\\
|\input{\childdocname}|\\
|\||else|\\
\textit{document body with }|\input{|\textit{part}|}|\\
|\||fi|
\end{tabular}
\end{center}
%
The conditional |\ifchilddocmanual| is true whenever
a part to be included by |\input| is being compiled,
and the name of the part is stored in |\childdocname|.

%%%%%%%%%%%%%%%%%%%%%%%%%%%%%%%%%%%%%%%%
\DescribeMacro{\childdocby}
Each part to be included by |\input| should start with:
%
\begin{center}
\begin{tabular}{l}
|\input{childdoc.def}|\\
|\childdocby{|\textit{main}|}|\\
\end{tabular}
\end{center}
%
The directive |\childdocby| is similar to |\childdocof|
described in \secref{sec:include},
but the subsequent selection of content must be done manually.
To that end, both |\ifchilddoc| and |\ifchilddocmanual|
will be true upon processing of a part,
and the name of the part is stored in |\childdocname|.
Note that |\jobname| will be set to the filename of the current part
so that each part receives an individual |.aux| file
that does not interfere with the |.aux| file(s) of the main document.
This behaviour can be altered by the alternative form
|\childdocby[*]{|\textit{main}|}| (with a non-empty optional argument)
which uses the |.aux| file of the main document
by setting |\jobname| to \textit{main}.

%%%%%%%%%%%%%%%%%%%%%%%%%%%%%%%%%%%%%%%%%%%%%%%%%%%%%%%%%%%%%%%%%%%%%%%%%%%%%%%%
\subsection{Driver Development}
\label{sec:driver}

The \textsf{childdoc} mechanism can also be use for the development
of definition files such as \LaTeX{} styles or classes.
This case differs from the above setup with multiple parts
included by |\include| in that no |\includeonly| should be invoked.
This can be achieved by starting the include file
(before |\ProvidesPackage|) with:
%
\begin{center}
\begin{tabular}{l}
|\input{childdoc.def}|\\
|\childdocforward{|\textit{main}|}|\\
\end{tabular}
\end{center}
%
or alternatively with:
%
\begin{center}
\begin{tabular}{l}
|\input{childdoc.def}|\\
|\childdocby{|\textit{main}|}|\\
\end{tabular}
\end{center}
%
Both forms have slightly different effects as described above.
The main file is prepared as usual, see \secref{sec:include}.

%%%%%%%%%%%%%%%%%%%%%%%%%%%%%%%%%%%%%%%%%%%%%%%%%%%%%%%%%%%%%%%%%%%%%%%%%%%%%%%%
\subsection{Legacy Detection}
\label{sec:detection}

The directive |\childdocmain| in the main file can detect
whether the complete document or merely a child is to be compiled
even without using the directive |\childdocof|.
This method is deprecated because it is less robust
and there is no compelling reason to use it;
it is merely provided for backward compatibility
and it may be removed in future versions.

If the detection mechanism is to be used,
it is mandatory to correctly specify
the filename of the main file as the argument of |\childdocmain|:
%
\begin{center}
\begin{tabular}{l}
|\input{childdoc.def}|\\
|\childdocmain{|\textit{main}|}|\\
\end{tabular}
\end{center}
%
If |\jobname| does not match the argument \textit{main} of |\childdocmain|,
it is assumed that |\jobname| points to the child file to be compiled.
When using |\childdocmain| with the main file specified as argument,
it suffices to start a child file
with just |\input{|\textit{main}|}|
without loading of the package and using |\childdocof|.
If instead all processing is done
with the appropriate \textsf{childdoc} directives,
the argument of \textit{main} of |\childdocmain| can be empty.

An alternative version of the command line processing described
in \secref{sec:commandline} using the detection mechanism reads:
%
\begin{center}
|... -jobname "|\textit{target}|" "|[\textit{flags}]%
[|\def\jobname{|\textit{dest}|}|]|\input{|\textit{main}|}"|
\end{center}

%%%%%%%%%%%%%%%%%%%%%%%%%%%%%%%%%%%%%%%%%%%%%%%%%%%%%%%%%%%%%%%%%%%%%%%%%%%%%%%%
\subsection{Manual Code}
\label{sec:manual}

In case one cannot be certain whether the definitions file |childdoc.def|
is installed on the target \TeX{} distribution
and one prefers not to ship it,
it is conceivable to paste a few relevant commands into the sources.

To that end, drop all statements |\input{childdoc.def}|
and perform the replacements as outlined below.
Instead of |\childdocmain{|\textit{main}|}| add the following code
to the top of the main file:
%
\begin{center}
\begin{tabular}{l}
|\||ifdefined\childdocname\endinput\||fi\newif\ifchilddoc|\\
|\edef\childdocname{\scantokens\expandafter{\jobname\noexpand}}|\\
|\def\childdocmain{|\textit{main}|}\||ifx\childdocmain\childdocname\||else|\\
|\childdoctrue\includeonly{\childdocname}\let\jobname\childdocmain\||fi|\\
\end{tabular}
\end{center}
%
Instead of |\childdocof{|\textit{main}|}| just include the main file
at the top of each child file:
%
\begin{center}
|\input{|\textit{main}|}|
\end{center}
%
A simple redirection |\childdocforward{|\textit{dest}|}| is achieved by:
%
\begin{center}
|\def\jobname{|\textit{dest}|}\input{\jobname}|
\end{center}
%
The redirection with prefix
|\childdocforwardprefix[|\textit{prefix}|]{|\textit{dest}|}|
is accomplished by:
%
\begin{center}
\begin{tabular}{l}
|{\edef\jobname{\scantokens\expandafter{\jobname\noexpand}}|\\
|\def\redirectjob |\textit{prefix}|#1~~~{\gdef\jobname{|\textit{dest}|#1}}|\\
|\expandafter\redirectjob\jobname~~~}\input{\jobname}|
\end{tabular}
\end{center}

In an alternative approach,
child documents can be compiled by a specific command line
without additional code or specific definitions:
%
\begin{center}
|... -jobname "|\textit{target}|" "|[\textit{flags}]%
|\includeonly{|\textit{dest}|}\input{|\textit{main}|}"|
\end{center}
%

%%%%%%%%%%%%%%%%%%%%%%%%%%%%%%%%%%%%%%%%%%%%%%%%%%%%%%%%%%%%%%%%%%%%%%%%%%%%%%%%
%%%%%%%%%%%%%%%%%%%%%%%%%%%%%%%%%%%%%%%%%%%%%%%%%%%%%%%%%%%%%%%%%%%%%%%%%%%%%%%%
\section{Information}

%%%%%%%%%%%%%%%%%%%%%%%%%%%%%%%%%%%%%%%%%%%%%%%%%%%%%%%%%%%%%%%%%%%%%%%%%%%%%%%%
\subsection{Copyright}

Copyright \copyright{} 2017--2018 Niklas Beisert

This work may be distributed and/or modified under the
conditions of the \LaTeX{} Project Public License, either version 1.3
of this license or (at your option) any later version.
The latest version of this license is in
  \url{http://www.latex-project.org/lppl.txt}
and version 1.3 or later is part of all distributions of \LaTeX{}
version 2005/12/01 or later.

This work has the LPPL maintenance status `maintained'.

The Current Maintainer of this work is Niklas Beisert.

This work consists of the files |README.txt|, |childdoc.ins| and |childdoc.dtx|
as well as the derived files |childdoc.def|, |cdocsamp.tex|
with |cdocsch1.tex|, |cdocsch2.tex|, |cdocspt3.tex|, |cdocspt4.tex|,
|cdocsdrf.tex|, |cdocsfn1.tex|, |cdocsfn2.tex|
as well as |childdoc.pdf|.

%%%%%%%%%%%%%%%%%%%%%%%%%%%%%%%%%%%%%%%%%%%%%%%%%%%%%%%%%%%%%%%%%%%%%%%%%%%%%%%%
\subsection{Files and Installation}

The package consists of the files:
%
\begin{center}
\begin{tabular}{ll}
    |README.txt|   & readme file \\
    |childdoc.ins| & installation file \\
    |childdoc.dtx| & source file \\
    |childdoc.def| & definition file \\
    |cdocsamp.tex| & sample main file \\
    |cdocsch1.tex| & sample include file \\
    |cdocsch2.tex| & sample include file \\
    |cdocspt3.tex| & sample part file \\
    |cdocspt4.tex| & sample part file \\
    |cdocsdrf.tex| & sample redirection file \\
    |cdocsfn1.tex| & sample redirection file \\
    |cdocsfn2.tex| & sample redirection file \\
    |childdoc.pdf| & manual
\end{tabular}
\end{center}
%
The distribution consists of the files
|README.txt|, |childdoc.ins| and |childdoc.dtx|.
%
\begin{itemize}
\item
Run (pdf)\LaTeX{} on |childdoc.dtx|
to compile the manual |childdoc.pdf| (this file).
\item
Run \LaTeX{} on |childdoc.ins| to create the definitions file |childdoc.def|
and the sample |cdocsamp.tex| with include files
|cdocsch1.tex|, |cdocsch2.tex|, |cdocspt3.tex|, |cdocspt4.tex|,
|cdocsdrf.tex|, |cdocsfn1.tex|, |cdocsfn2.tex|.
Then copy the file |childdoc.def| to an appropriate directory of your \LaTeX{}
distribution, e.g.\ \textit{texmf-root}|/tex/latex/childdoc|.
\end{itemize}

%%%%%%%%%%%%%%%%%%%%%%%%%%%%%%%%%%%%%%%%%%%%%%%%%%%%%%%%%%%%%%%%%%%%%%%%%%%%%%%%
\subsection{Related CTAN Packages}

There are several other packages which offer a similar functionality:
%
\begin{itemize}
\item
The packages
\href{http://ctan.org/pkg/docmute}{\textsf{docmute}},
\href{http://ctan.org/pkg/includex}{\textsf{includex}} and
\href{http://ctan.org/pkg/standalone}{\textsf{standalone}}
provide commands to include only the document body of
a child file thus allowing both files to be compiled individually.
\item
The packages \href{http://ctan.org/pkg/subdocs}{\textsf{subdocs}}
and \href{http://ctan.org/pkg/subfiles}{\textsf{subfiles}}
provide structures in which the main and child documents can be
encapsulated and allowing them to be compiled individually.
The inclusion mechanism is different from the conventional |\include|.
\item
The package \href{http://ctan.org/pkg/combine}{\textsf{combine}}
is an elaborate solution to combine several documents into one.
\end{itemize}
%
See also the CTAN topic \href{http://ctan.org/topic/subdocs}{\textsf{subdocs}}
for further related packages.
The present package differs from the above solutions in that
a document structure constructed with the conventional |\include| mechanism
just needs two extra commands at the top of every file
such that all constituent files can be compiled individually.

%%%%%%%%%%%%%%%%%%%%%%%%%%%%%%%%%%%%%%%%%%%%%%%%%%%%%%%%%%%%%%%%%%%%%%%%%%%%%%%%
%\subsection{Feature Suggestions}
%
%The following is a list of features which may be useful for future
%versions of this package:
%%
%\begin{itemize}
%\item
%\ldots
%\end{itemize}

%%%%%%%%%%%%%%%%%%%%%%%%%%%%%%%%%%%%%%%%%%%%%%%%%%%%%%%%%%%%%%%%%%%%%%%%%%%%%%%%
\subsection{Revision History}

%%%%%%%%%%%%%%%%%%%%%%%%%%%%%%%%%%%%%%%%
\paragraph{v2.0:} 2018/12/30

\begin{itemize}
\item
immediate forward processing
\item
added |\childdocby| mechanism
\item
manual restructured
\end{itemize}

%%%%%%%%%%%%%%%%%%%%%%%%%%%%%%%%%%%%%%%%
\paragraph{v1.6:} 2018/01/17

\begin{itemize}
\item
application for development of include files
\item
corrections to manual
\end{itemize}

%%%%%%%%%%%%%%%%%%%%%%%%%%%%%%%%%%%%%%%%
\paragraph{v1.5:} 2017/05/21

\begin{itemize}
\item
more complete structuring introduced
\item
|\childdocof| introduced
\item
|\childdoc| renamed to |\childdocmain|
\item
|\childredirect| renamed to |\childdocforward| and |\childdocforwardprefix|
and functionality expanded
\end{itemize}

%%%%%%%%%%%%%%%%%%%%%%%%%%%%%%%%%%%%%%%%
\paragraph{v1.0:} 2017/04/27

\begin{itemize}
\item
manual and install package
\item
first version published on CTAN
\end{itemize}

%%%%%%%%%%%%%%%%%%%%%%%%%%%%%%%%%%%%%%%%
\paragraph{v0.6:} 2017/04/26

\begin{itemize}
\item
redirection mechanism added
\end{itemize}

%%%%%%%%%%%%%%%%%%%%%%%%%%%%%%%%%%%%%%%%
\paragraph{v0.5:} 2017/04/26

\begin{itemize}
\item
functionality in definition file
\end{itemize}


%%%%%%%%%%%%%%%%%%%%%%%%%%%%%%%%%%%%%%%%%%%%%%%%%%%%%%%%%%%%%%%%%%%%%%%%%%%%%%%%
%%%%%%%%%%%%%%%%%%%%%%%%%%%%%%%%%%%%%%%%%%%%%%%%%%%%%%%%%%%%%%%%%%%%%%%%%%%%%%%%
%%%%%%%%%%%%%%%%%%%%%%%%%%%%%%%%%%%%%%%%%%%%%%%%%%%%%%%%%%%%%%%%%%%%%%%%%%%%%%%%
\appendix

\settowidth\MacroIndent{\rmfamily\scriptsize 000\ }

 \DocInput{childdoc.dtx}

\end{document}
%</driver>
% \fi
%
% %%%%%%%%%%%%%%%%%%%%%%%%%%%%%%%%%%%%%%%%%%%%%%%%%%%%%%%%%%%%%%%%%%%%%%%%%%%%%%
% %%%%%%%%%%%%%%%%%%%%%%%%%%%%%%%%%%%%%%%%%%%%%%%%%%%%%%%%%%%%%%%%%%%%%%%%%%%%%%
% \section{Sample}
%\iffalse
%<*samplemain>
%\fi
%
% The following presents a sample document
% with two chapters, two parts, a title page,
% a compile flag as well as three forwarding files to set the flag.
% It consists of eight |.tex| files:
% \begin{center}
% \begin{tabular}{ll}
% |cdocsamp.tex|&main file\\
% |cdocsch1.tex|&include file for chapter 1\\
% |cdocsch2.tex|&include file for chapter 2\\
% |cdocspt3.tex|&include file for part 3\\
% |cdocspt4.tex|&include file for part 4\\
% |cdocsdrf.tex|&forwarding file for main file in draft mode\\
% |cdocsfi1.tex|&forwarding file for final version of chapter 1\\
% |cdocsfi2.tex|&forwarding file for final version of chapter 2\\
% \end{tabular}
% \end{center}
% Each of the eight files can be compiled directly by the \LaTeX{} compiler.
%
% %%%%%%%%%%%%%%%%%%%%%%%%%%%%%%%%%%%%%%
% \paragraph{Main File.}
%
% The main file is called |cdocsamp.tex|.
%
% Load the \textsf{childdoc} definitions and
% declare the filename for the main document:
%    \begin{macrocode}
\input{childdoc.def}
\childdocmain{}
%    \end{macrocode}

% Optional override for |\version| flag:
%    \begin{macrocode}
%%\ifchilddoc\else\providecommand{\version}{draft}\fi
%    \end{macrocode}

% Define the default values for the |\version| flag
% (|final| for the main file and |draft| for childs):
%    \begin{macrocode}
\ifchilddoc
\providecommand{\version}{draft}
\else
\providecommand{\version}{final}
\fi
%    \end{macrocode}

% Load the standard document class:
%    \begin{macrocode}
\documentclass[12pt]{article}
%    \end{macrocode}

% Start the document body:
%    \begin{macrocode}
\begin{document}
%    \end{macrocode}

% Declare a title page.
% Print title, part of document being processed and version flag:
%    \begin{macrocode}
\addtocounter{page}{-1}
\begin{center}
{\LARGE\bfseries{}childdoc example\par}
\vspace{1cm}
\ifchilddoc
\ifchilddocmanual part\else chapter\fi:
`\childdocname' of `\childdocjob'\par
\else
main document: `\childdocjob'\par
\fi
version: \version\par
\end{center}
\newpage
%    \end{macrocode}

% Manually include selected file,
% otherwise process as usual:
%    \begin{macrocode}
\ifchilddocmanual
\section*{part `\childdocname'}
\input{\childdocname}
\else
%    \end{macrocode}

% Include the two chapters:
%    \begin{macrocode}
\include{cdocsch1}
\include{cdocsch2}
%    \end{macrocode}

% Include the two parts unless only chapters should be displayed:
%    \begin{macrocode}
\ifchilddoc\else
\section{part three}
\input{cdocspt3}
\section{part four}
\input{cdocspt4}
\fi
%    \end{macrocode}

% Process as usual until here:
%    \begin{macrocode}
\fi
%    \end{macrocode}

% End of document body:
%    \begin{macrocode}
\end{document}
%    \end{macrocode}
%\iffalse
%</samplemain>
%\fi
%
% %%%%%%%%%%%%%%%%%%%%%%%%%%%%%%%%%%%%%%
% \paragraph{Chapter Include Files.}
%
% The include files are called |cdocsch1.tex| and |cdocsch2.tex|.
%
%\iffalse
%<*samplechap1|samplechap2>
%\fi

% Optional override for |\version| flag:
%    \begin{macrocode}
%%\providecommand{\version}{final}
%    \end{macrocode}

% Include the main document:
%    \begin{macrocode}
\input{childdoc.def}
\childdocof{cdocsamp}
%    \end{macrocode}

%\iffalse
%</samplechap1|samplechap2>
%\fi
%
%\iffalse
%<*samplechap1>
%\fi
% Some text for chapter 1:
%    \begin{macrocode}
\section{one}
some text in chapter one
%    \end{macrocode}

%\iffalse
%</samplechap1>
%\fi
% Some text for chapter 2:
%\iffalse
%<*samplechap2>
%\fi
%    \begin{macrocode}
\section{two}
more text in chapter two
%    \end{macrocode}

%\iffalse
%</samplechap2>
%\fi
%
% %%%%%%%%%%%%%%%%%%%%%%%%%%%%%%%%%%%%%%
% \paragraph{Part Include Files.}
%
% The include files are called |cdocspt3.tex| and |cdocspt4.tex|.
%
%\iffalse
%<*samplepart3|samplepart4>
%\fi

% Optional override for |\version| flag:
%    \begin{macrocode}
%%\providecommand{\version}{final}
%    \end{macrocode}

% Include the main document:
%    \begin{macrocode}
\input{childdoc.def}
\childdocby{cdocsamp}
%    \end{macrocode}

%\iffalse
%</samplepart3|samplepart4>
%\fi
%
%\iffalse
%<*samplepart3>
%\fi
% Some text for part 3:
%    \begin{macrocode}
some text in part three
%    \end{macrocode}

%\iffalse
%</samplepart3>
%\fi
% Some text for part 4:
%\iffalse
%<*samplepart4>
%\fi
%    \begin{macrocode}
more text in part four
%    \end{macrocode}

%\iffalse
%</samplepart4>
%\fi
%
% %%%%%%%%%%%%%%%%%%%%%%%%%%%%%%%%%%%%%%
% \paragraph{Forwarding for a Complete Draft.}
%
% The following forwarding file |cdocsdrf.tex|
% compiles the main document in draft mode:
%\iffalse
%<*sampledraft>
%\fi
%    \begin{macrocode}
\def\version{draft}
\input{childdoc.def}
\childdocforward{cdocsamp}
%    \end{macrocode}

%\iffalse
%</sampledraft>
%\fi
%
% %%%%%%%%%%%%%%%%%%%%%%%%%%%%%%%%%%%%%%
% \paragraph{Forwarding for Final Version of the Chapters.}
%
% The following forwarding files |cdocsfn1.tex| and |cdocsfn2.tex|
% (with identical content)
% compile the final versions of the child documents
% |cdocsch1.tex| and |cdocsch2.tex|, respectively:
%\iffalse
%<*samplefinal>
%\fi
%    \begin{macrocode}
\def\version{final}
\input{childdoc.def}
\childdocforwardprefix[cdocsamp]{cdocsfn}{cdocsch}
%    \end{macrocode}

%\iffalse
%</samplefinal>
%\fi
%
% %%%%%%%%%%%%%%%%%%%%%%%%%%%%%%%%%%%%%%
% \paragraph{Command Line Processing.}
%
% The following three command lines generate the output files
% |cdocscld|, |cdocscl1| and |cdocscl2|
% which should be identical to
% |cdocsdrf|, |cdocsch1| and |cdocsfn2|, respectively:
% \begin{center}
% \begin{tabular}{l}
% |latex -jobname cdocscld \|\\
% |  "\def\version{draft}\input{childdoc.def}\childdocforward{cdocsamp}"|\\
% |latex -jobname cdocscl1 \|\\
% |  "\input{childdoc.def}\childdocforward[cdocsamp]{cdocsch1}"|\\
% |latex -jobname cdocscl2 \|\\
% |  "\def\version{final}\input{childdoc.def}\childdocforward{cdocsch2}"|
% \end{tabular}
% \end{center}
% Note that the trailing backslash on each first line
% merely continues the input to the second line
% (for convenient cut ant paste).
% Furthermore, the command |latex| can be replaced by any
% of its alternative versions such as |pdflatex|.
%
% %%%%%%%%%%%%%%%%%%%%%%%%%%%%%%%%%%%%%%%%%%%%%%%%%%%%%%%%%%%%%%%%%%%%%%%%%%%%%%
% %%%%%%%%%%%%%%%%%%%%%%%%%%%%%%%%%%%%%%%%%%%%%%%%%%%%%%%%%%%%%%%%%%%%%%%%%%%%%%
% \section{Implementation}
%\iffalse
%<*package>
%\fi
%
% This section describes the definitions file |childdoc.def|.

% The definitions cannot be loaded using |\usepackage| or |\RequirePackage|
% which has a mechanism to prevent loading a style file more than once.
% When loading the definitions by means of |\input|
% multiple instances have to be prevented manually:
%\iffalse
%This code needs to be before the `\ProvidesFile' directive
%which is defined at the beginning of this file.
%Therefore it is also placed there and commented out here.
%</package>
%<*discard>
%\fi
%    \begin{macrocode}
\ifdefined\childdocmain\endinput\fi
%    \end{macrocode}
%\iffalse
%</discard>
%<*package>
%\fi
%
% \macro{\ifchilddoc}
% \macro{\ifchilddocmanual}
% The conditional |\ifchilddoc| tells whether a
% child (true) or main (false) document is being compiled.
% The conditional |\ifchilddocmanual| tells whether
% the |\includeonly| mechanism is used (false) or
% the selection of child files must be performed manually (true).
% The definitions initialise to false:
%    \begin{macrocode}
\newif\ifchilddoc
\newif\ifchilddocmanual
%    \end{macrocode}

% \macro{\childdocname}
% \macro{\childdocjob}
% The macro |\childdocname| stores the name of the main document
% to be compiled. The macro |\childdocjob| stores the name of
% the document on which the \LaTeX{} compiler was originally invoked.
% The content of |\jobname| cannot be compared
% to filenames specified in the source due to different catcodes.
% The following code rescans |\jobname|, stores the result
% in |\childdocname| and saves a copy in |\childdocjob|:
%    \begin{macrocode}
\edef\childdocname{\scantokens\expandafter{\jobname\noexpand}}
\let\childdocjob\childdocname
%    \end{macrocode}

% \macro{\childdocdisable}
% The macro |\childdocdisable| prevents the main file
% from being processed more than once.
% At this stage, the main document command |\childdocmain|
% is assumed to be called once again where it should do nothing.
% Any subsequent call to it should prevent
% a secondary processing of the main document
% It overwrites the forwarding commands
% |\childdocof| and |\childdocforward|
% with empty macros to prevent further inclusions of the main document:
%    \begin{macrocode}
\newcommand{\childdocdisable}
{
  \renewcommand{\childdocmain}[1]{\renewcommand{\childdocmain}[1]{\endinput}}
  \renewcommand{\childdocof}[1]{}
  \renewcommand{\childdocby}[2][]{}
  \renewcommand{\childdocforward}[2][]{}
  \renewcommand{\childdocdisable}{}
}
%    \end{macrocode}

% \macro{\childdocmain}
% The macro |\childdocmain| is to be called at the top of the main file
% with nothing or the main filename (without extension) as argument.
% First, it breaks loops.
% If the argument is not empty and does not match |\childdocname|
% (which is set by the first inclusion of |childdoc.def|),
% |\ifchilddoc| is set to true, |\includeonly| is applied to the child file
% and |\jobname| is set to the main file
% (for proper handling of |.aux| files):
%    \begin{macrocode}
\newcommand{\childdocmain}[1]
{
  \childdocdisable\childdocmain{}
  \if?#1?\else
    \begingroup
      \def\childdoctmp{#1}
      \ifx\childdoctmp\childdocname
        \def\childdoctmp{}
      \else
        \def\childdoctmp
        {
          \childdoctrue
          \includeonly{\childdocname}
          \def\childdocjob{#1}
          \def\jobname{#1}
        }
      \fi
      \expandafter
    \endgroup
    \childdoctmp
  \fi
}
%    \end{macrocode}

% \macro{\childdocof}
% The command |\childdocof| redirects
% compilation to the main file |#1|.
%    \begin{macrocode}
\newcommand{\childdocof}[1]
{
  \childdocdisable
  \childdoctrue
  \includeonly{\childdocname}
  \def\jobname{#1}
  \def\childdocjob{#1}
  \input{#1}
}
%    \end{macrocode}

% \macro{\childdocby}
% The command |\childdocby| ....
%    \begin{macrocode}
\newcommand{\childdocby}[2][]
{
  \childdocdisable
  \childdoctrue
  \childdocmanualtrue
  \if?#1?\else
    \def\jobname{#2}
  \fi
  \def\childdocjob{#2}
  \input{#2}
  \endinput
}
%    \end{macrocode}

% \macro{\childdocforward}
% The command |\childdocforward| redirects
% compilation to the main file or
% (if the optional argument is given) a child file.
% Parameters are set as if the main file
% or a child file starting with |\childdocof| was compiled.
% Then compilation is handed over to the main file:
%    \begin{macrocode}
\newcommand{\childdocforward}[2][]
{
  \begingroup
    \if?#1?
      \def\childdoctmp
      {
        \def\childdocname{#2}
        \def\childdocjob{#2}
        \def\jobname{#2}
        \input{#2}
        \endinput
      }
    \else
      \def\childdoctmp
      {
        \childdocdisable
        \def\childdocname{#2}
        \childdoctrue
        \includeonly{#2}
        \def\childdocjob{#1}
        \def\jobname{#1}
        \input{#1}
        \endinput
      }
    \fi
    \expandafter
  \endgroup
  \childdoctmp
}
%    \end{macrocode}

% \macro{\childdocforwardprefix}
% The command |\childdocforwardprefix| redirects
% compilation to the main or a child file by means of a pattern.
% The prefix |#1| in the current filename is replaced by |#2|
% and the suffix of the current filename is kept
% (it is assumed that the filename does not contain the substring `|~~~|'
% which is used as a delimiter).
% Compilation is handed over to the new file by |\childdocforward|:
%    \begin{macrocode}
\newcommand{\childdocforwardprefix}[3][]
{
  \begingroup
    \def\childdocextract #2##1~~~{\def\childdoctmp{\childdocforward[#1]{#3##1}}}
    \expandafter\childdocextract\childdocname~~~
    \expandafter
  \endgroup
  \childdoctmp
}
%    \end{macrocode}

% \macro{\childdoc}
% The deprecated macro |\childdoc| is a legacy version of |\childdocmain|:
%    \begin{macrocode}
\newcommand{\childdoc}{\childdocmain}
%    \end{macrocode}

% \macro{\childdocredirect}
% The deprecated macro |\childdocredirect| is a legacy version
% of |\childdocforward| and |\childdocforwardprefix|:
%    \begin{macrocode}
\newcommand{\childdocredirect}[2][]
{
  \begingroup
    \if?#1?
      \def\childdoctmp{\childdocforward{#2}}
    \else
      \def\childdoctmp{\childdocforwardprefix{#1}{#2}}
    \fi
    \expandafter
  \endgroup
  \childdoctmp
}
%    \end{macrocode}

%\iffalse
%</package>
%\fi
%
\endinput
|\\
|\childdocforward{|\textit{main}|}|
\end{tabular}
\end{center}
%
Likewise, the following files |final|\textit{nn}|.tex|
compile the final version of the child document
|child|\textit{nn}|.tex|:
%
\begin{center}
\begin{tabular}{l}
|\def\version{final}|\\
|% \iffalse
%
% childdoc.dtx Copyright (C) 2017-2018 Niklas Beisert
%
% This work may be distributed and/or modified under the
% conditions of the LaTeX Project Public License, either version 1.3
% of this license or (at your option) any later version.
% The latest version of this license is in
%   http://www.latex-project.org/lppl.txt
% and version 1.3 or later is part of all distributions of LaTeX
% version 2005/12/01 or later.
%
% This work has the LPPL maintenance status `maintained'.
%
% The Current Maintainer of this work is Niklas Beisert.
%
% This work consists of the files childdoc.dtx and childdoc.ins
% and the derived files childdoc.def and cdocsamp.tex with
% cdocsch1.tex, cdocsch2.tex, cdocsdrf.tex, cdocsfn1.tex, cdocsfn2.tex.
%
%<package>\ifdefined\childdocmain\endinput\fi
%<package>\ProvidesFile{childdoc.def}[2018/12/30 v2.0 child document driver]
%<samplemain>\ProvidesFile{cdocsamp.tex}[2018/12/30 v2.0 sample for childdoc]
%<*driver>
%\ProvidesFile{childdoc.drv}[2018/12/30 v2.0 childdoc reference manual file]
\PassOptionsToClass{10pt,a4paper}{article}
\documentclass{ltxdoc}

\usepackage[margin=35mm]{geometry}
\usepackage{hyperref}
\usepackage{hyperxmp}
\usepackage[usenames]{color}

\hypersetup{colorlinks=true}
\hypersetup{pdfstartview=FitH}
\hypersetup{pdfpagemode=UseNone}
\hypersetup{pdfsource={}}
\hypersetup{pdflang={en-UK}}
\hypersetup{pdfcopyright={Copyright 2017-2018 Niklas Beisert.
  This work may be distributed and/or modified under the
  conditions of the LaTeX Project Public License, either version 1.3
  of this license or (at your option) any later version.}}
\hypersetup{pdflicenseurl={http://www.latex-project.org/lppl.txt}}
\hypersetup{pdfcontactaddress={ETH Zurich, ITP, HIT K,
  Wolfgang-Pauli-Strasse 27}}
\hypersetup{pdfcontactpostcode={8093}}
\hypersetup{pdfcontactcity={Zurich}}
\hypersetup{pdfcontactcountry={Switzerland}}
\hypersetup{pdfcontactemail={nbeisert@itp.phys.ethz.ch}}
\hypersetup{pdfcontacturl={http://people.phys.ethz.ch/\xmptilde nbeisert/}}

\newcommand{\secref}[1]{\hyperref[#1]{section \ref*{#1}}}

\parskip1ex
\parindent0pt
\let\olditemize\itemize
\def\itemize{\olditemize\parskip0pt}

\begin{document}

\title{The \textsf{childdoc} Package}
\hypersetup{pdftitle={The childdoc Package}}
\author{Niklas Beisert\\[2ex]
  Institut f\"ur Theoretische Physik\\
  Eidgen\"ossische Technische Hochschule Z\"urich\\
  Wolfgang-Pauli-Strasse 27, 8093 Z\"urich, Switzerland\\[1ex]
  \href{mailto:nbeisert@itp.phys.ethz.ch}
  {\texttt{nbeisert@itp.phys.ethz.ch}}}
\hypersetup{pdfauthor={Niklas Beisert}}
\hypersetup{pdfsubject={Manual for the LaTeX2e Package childdoc}}
\date{30 December 2018, \textsf{v2.0}}
\maketitle

\begin{abstract}\noindent
\textsf{childdoc} is a \LaTeXe{} package
that enables the direct compilation
of document sections included by |\include|
to individual files.
\end{abstract}

\begingroup
\parskip0ex
\tableofcontents
\endgroup

%%%%%%%%%%%%%%%%%%%%%%%%%%%%%%%%%%%%%%%%%%%%%%%%%%%%%%%%%%%%%%%%%%%%%%%%%%%%%%%%
%%%%%%%%%%%%%%%%%%%%%%%%%%%%%%%%%%%%%%%%%%%%%%%%%%%%%%%%%%%%%%%%%%%%%%%%%%%%%%%%
\section{Introduction}

\LaTeX{} provides a mechanism to structure a large document (such as a book)
into a main file and several child files (containing the chapters)
using the |\include| command.
This mechanism is beneficial for documents
which span hundreds of pages in order to
make the source file(s) more manageable.
Moreover, compilation can be restricted to
selected child files by means of the |\includeonly| command.
The latter feature can be used to reduce the compilation time while editing
(this was significantly more useful in the earlier days of \LaTeX{})
or to generate a smaller document which is easier to navigate.
Another application of |\includeonly| is to generate
documents consisting of selected parts of the complete document.

However, there are a few drawbacks of the plain |\include| mechanism:
\begin{itemize}
\item
The child files cannot be compiled on their own,
they can only be compiled via the main file.
A naive editing environment
(such as a text editor with an option
to have the current file processed by \LaTeX)
may require one to switch to the main file before compiling;
attempting to compile the child file produces errors.
\item
The main file must be modified (each time)
to adjust the |\includeonly| command
to the present needs. This easily leaves the main file in a messy state.
\item
The generated document will always carry the filename
of the main document. This is inconvenient if
several child files are to be compiled and
to be kept for distribution.
\end{itemize}

The present package provides a simple interface
to make child files individually compilable by \LaTeX{}.
Compiling a child file then has the same effect as compiling
the main file with an |\includeonly| command
to select the appropriate child.
Moreover the generated document will carry the name of the child
rather than the main file.
This resolves all three above issues.

This feature is meant to make the editing of books,
thesis documents and lecture notes somewhat more convenient.
However, the package can also be used efficiently for
composing a series of documents (such as exercise sheets)
which are typically distributed individually.
It then assists the author in generating the individual documents
(potentially in different versions)
as well as a document containing the collected series.
Another application is in developing style files
or other kinds of included material
where compilation of the style file could redirect
to a sample or test file.

%%%%%%%%%%%%%%%%%%%%%%%%%%%%%%%%%%%%%%%%%%%%%%%%%%%%%%%%%%%%%%%%%%%%%%%%%%%%%%%%
%%%%%%%%%%%%%%%%%%%%%%%%%%%%%%%%%%%%%%%%%%%%%%%%%%%%%%%%%%%%%%%%%%%%%%%%%%%%%%%%
\section{Usage}

First of all, the package \textsf{childdoc} is \emph{not} a standard
\LaTeXe{} |.sty| style file! Therefore it needs to be invoked in
a non-standard way.

%%%%%%%%%%%%%%%%%%%%%%%%%%%%%%%%%%%%%%%%%%%%%%%%%%%%%%%%%%%%%%%%%%%%%%%%%%%%%%%%
\subsection{Included Files}
\label{sec:include}

%%%%%%%%%%%%%%%%%%%%%%%%%%%%%%%%%%%%%%%%
\DescribeMacro{\childdocmain}
To use the package, add the commands
\begin{center}
\begin{tabular}{l}
|\input{childdoc.def}|\\
|\childdocmain{}|\\
\end{tabular}
\end{center}
at the very top of the main \LaTeX{} file,
in particular \emph{before} the |\documentclass| statement!
The argument of |\childdocmain| should be left empty
(but it must be present).

%%%%%%%%%%%%%%%%%%%%%%%%%%%%%%%%%%%%%%%%
\DescribeMacro{\childdocof}
Furthermore, add the commands
\begin{center}
\begin{tabular}{l}
|\input{childdoc.def}|\\
|\childdocof{|\textit{main}|}|\\
\end{tabular}
\end{center}
at the top of every child file \textit{child}
which is included by |\include{|\textit{child}|}|
from within the main file
(or at least for those files to be compiled individually).
The argument \textit{main} must be the filename of the main file.

There are a couple of
considerations in setting up the main and child documents:

%%%%%%%%%%%%%%%%%%%%%%%%%%%%%%%%%%%%%%%%
\paragraph{Restrictions.}

Please note the following restrictions:
\begin{itemize}
\item
|\childdocmain| must be called with one argument \textit{main}
to ensure compatibility with earlier version of the package.
It must either be empty (|\childdocmain{}|)
or precisely match the filename of the main file in which it is specified.
See \secref{sec:detection} for further information.
\item
The filename \textit{main} must be specified without the |.tex| extension.
\item
The filename \textit{main} is case sensitive
(even in case-insensitive file systems)
due to internal string comparison.
\item
The argument \textit{main} should be fully expanded, it cannot be a macro.
\item
Subdirectories and special characters should be avoided in filenames.
\item
The command |\childdocmain{|\textit{main}|}| must be followed by a whitespace.
It should not be followed immediately by another command
or by a comment mark `|%|'.
This is because the \TeX{} parser reads the token immediately following
the argument of |\childdocmain| and puts it
at the beginning of every child section;
however, a white\-space is ignored.
\end{itemize}

%%%%%%%%%%%%%%%%%%%%%%%%%%%%%%%%%%%%%%%%
\paragraph{Content of Main File.}

It is advisable to place all content in the child files included by |\include|.
Any output contained in the main file will appear in all child documents
unless suppressed manually;
it cannot be suppressed automatically by the |\includeonly| directive
and thus should normally be avoided.
A method to include some content in the main file
by means of conditional processing is described in \secref{sec:conditional}.

%%%%%%%%%%%%%%%%%%%%%%%%%%%%%%%%%%%%%%%%
\paragraph{Page Numbering.}

When only a part of the document is compiled,
the appropriate numbering of pages
(as well as other status parameters)
is determined from the |.aux| files.
The latter contain information from previous passes.
However this information needs to propagate through
all intermediate child documents.
Therefore the page numbering in child documents may well
be inconsistent until the complete document is compiled at least once.

A useful (if unconventional) way to always ensure a consistent
page numbering is to restart the numbering in each child document
and denote the pages by `\textit{child}|.|\textit{page}'
where \textit{child} represents the chapter/section number of the child file.
This can be achieved by the command
|\numberwithin{page}{|\textit{child}|}|
of the \textsf{amsmath} package
where \textit{child} can be |chapter| or |section|
depending on the chosen structuring.
Alternatively, one can modify the macro |\thepage| appropriately
and reset the counter |page| at the start of each child file.

%%%%%%%%%%%%%%%%%%%%%%%%%%%%%%%%%%%%%%%%%%%%%%%%%%%%%%%%%%%%%%%%%%%%%%%%%%%%%%%%
\subsection{Conditional Processing}
\label{sec:conditional}

The package provides a mechanism to compile different versions
of a document. To customise the versions further some conditional processing
can come in handy to distinguish which version is being compiled.
The package provides two macros to describe the compilation context:

%%%%%%%%%%%%%%%%%%%%%%%%%%%%%%%%%%%%%%%%
\DescribeMacro{\ifchilddoc}
The conditional |\ifchilddoc| distinguishes between the compilation of
child documents and the main document:
%
\begin{center}
|\ifchilddoc |\textit{child-code}| |[|\||else |\textit{main-code}]| \||fi|
\end{center}

%%%%%%%%%%%%%%%%%%%%%%%%%%%%%%%%%%%%%%%%
\DescribeMacro{\childdocname}
\DescribeMacro{\childdocjob}
The macro |\childdocname| contains the filename (without extension)
of the main or child file being processed.
Note that |\childdocjob| will always contain the name of the main file.

%%%%%%%%%%%%%%%%%%%%%%%%%%%%%%%%%%%%%%%%
\paragraph{Title Page.}

Conditional processing can be used to include a title or banner page
in the main document when proper precautions are taken.
Importantly, the code in the main file should ensure that the page counter
(as well as other status parameters which are stored in the |.aux| files)
takes the same value after the conditional processing.
Otherwise the page numbers may take divergent values
depending on which part is compiled.

For example, a title page could be declared by:
%
\begin{center}
\begin{tabular}{l}
|\ifchilddoc\||else|\\
|\addtocounter{page}{-1}|\\
\textit{code for title page}\\
|\newpage|\\
|\||fi|
\end{tabular}
\end{center}
%
A banner page for the child documents can be generated by:
%
\begin{center}
\begin{tabular}{l}
|\ifchilddoc|\\
|\addtocounter{page}{-1}|\\
\textit{code for banner page}\\
|\newpage|\\
|\||fi|
\end{tabular}
\end{center}
%
Here one could write a message such as:
\begin{center}
|This is the part \childdocname{} of \childdocjob{}.|
\end{center}

%%%%%%%%%%%%%%%%%%%%%%%%%%%%%%%%%%%%%%%%%%%%%%%%%%%%%%%%%%%%%%%%%%%%%%%%%%%%%%%%
\subsection{Flags}
\label{sec:flags}

The package makes it easy to generate different versions
of the main or child documents.
To this end compilation flags can be defined
and assigned different default values.
They will be particularly useful in conjunction
with the forwarding mechanism described in \secref{sec:forward}.

For example, it may be useful to have a flag |\version|
which can be set to |draft| or |final|.
The document source will contain some conditional code
depending on the value of |\version|.
Suppose further, the flag should default to |final| for the main file
and to |draft| for child files
which is a natural assignment for editing the document.
This is achieved by placing the following code
in the preamble of the main document
(below the |\childdocmain| directive):
%
\begin{center}
\begin{tabular}{l}
|\ifchilddoc|\\
|\providecommand{\version}{draft}|\\
|\||else|\\
|\providecommand{\version}{final}|\\
|\||fi|
\end{tabular}
\end{center}
%
The definition by |\providecommand| makes sure
that previous definitions are not overwritten.
Further statements |\providecommand{\version}{...}|
can thus be added before the above code to override it.

For the main file, one might add a line
(between |\childdocmain| and the above block)
%
\begin{center}
|%\ifchilddoc\||else\providecommand{\version}{draft}\||fi|
\end{center}
%
which can be uncommented to produce a draft version.
Likewise one can add a line to the very top of a child file
(above the |\childdocof{|\textit{main}|}| directive)
%
\begin{center}
|%\providecommand{\version}{final}|
\end{center}
%
which can be uncommented to produce the final version of this child document.

%%%%%%%%%%%%%%%%%%%%%%%%%%%%%%%%%%%%%%%%%%%%%%%%%%%%%%%%%%%%%%%%%%%%%%%%%%%%%%%%
\subsection{Forwarding}
\label{sec:forward}

Different versions of the main or child documents
using compilation flags as described in \secref{sec:flags}
can be (permanently) stored in different files
for convenient compilation, viewing and distribution.
To this end, the package defines a command
to pass on compilation to a different file:

%%%%%%%%%%%%%%%%%%%%%%%%%%%%%%%%%%%%%%%%
\DescribeMacro{\childdocforward}
The command |\childdocforward| redirects processing to
another source file:
%
\begin{center}
\begin{tabular}{l}
|\input{childdoc.def}|\\
|\childdocforward[|\textit{main}|]{|\textit{dest}|}|\\
\end{tabular}
\end{center}
%
The argument \textit{dest} is the destination file
(without extension).
It should be the main file or one of the child files.
Note that further \textsf{childdoc} directives
such as |\childdocof| and |\childdocforward|
in the indicated file will be processed in this form.
The optional argument \textit{main}
passes on directly to the main file \textit{main}
while pretending to compile the child \textit{dest}.
This form behaves as if \textit{dest}
issues |\childdocof{|\textit{main}|}| right away,
and no further \textsf{childdoc} directives will be processed.

%%%%%%%%%%%%%%%%%%%%%%%%%%%%%%%%%%%%%%%%
\DescribeMacro{\...prefix}
In the alternative form |\childdocforwardprefix|,
%
\begin{center}
\begin{tabular}{l}
|\input{childdoc.def}|\\
|\childdocforwardprefix[|\textit{main}|]{|\textit{prefix}|}{|\textit{dest}|}|
\end{tabular}
\end{center}
%
the destination file is determined by a pattern
depending on the current file:
To make this work, the current file must be called
`{\textit{prefix}\hspace{0.2em}\textit{suffix}}'
with \textit{prefix} matching precisely the argument.
Processing is then passed on to the file
`{\textit{dest}\hspace{0.2em}\textit{suffix}}'.
Surely, the same effect is achieved by
directly specifying the
argument `{\textit{dest}\hspace{0.2em}\textit{suffix}}'
in the first form.
However, that requires to set up a different file
for each child. With the alternative form of the command
all these files can have exactly the same content
which simplifies setting them up and maintaining them.

For example, the following file |draft.tex|
with a compilation flag |\version| as described in \secref{sec:flags}
compiles the main document as a draft:
%
\begin{center}
\begin{tabular}{l}
|\def\version{draft}|\\
|\input{childdoc.def}|\\
|\childdocforward{|\textit{main}|}|
\end{tabular}
\end{center}
%
Likewise, the following files |final|\textit{nn}|.tex|
compile the final version of the child document
|child|\textit{nn}|.tex|:
%
\begin{center}
\begin{tabular}{l}
|\def\version{final}|\\
|\input{childdoc.def}|\\
|\childdocforwardprefix{final}{child}|
\end{tabular}
\end{center}
%

Note that when several versions of a main file and/or of each child file
are to be generated, it may be convenient to set up a |Makefile| or
shell script to automatise the process.

%%%%%%%%%%%%%%%%%%%%%%%%%%%%%%%%%%%%%%%%%%%%%%%%%%%%%%%%%%%%%%%%%%%%%%%%%%%%%%%%
\subsection{Command Line Processing}
\label{sec:commandline}

The effect of redirection files can also be achieved by invoking
the \LaTeX{} compiler with a more elaborate command line.
Most conveniently this should be done as part
of a shell script or a |Makefile|.

When using \textsf{childdoc} in the main file, the following
command lines effectively perform a redirection
(note that depending on the shell being used,
backslashes may have to be doubled: `|\|' $\to$ `|\\|'):
%
\begin{center}
|... -jobname "|\textit{target}|" |\\|"|[\textit{flags}]%
|\input{childdoc.def}\childdocforward[|\textit{main}|]{|\textit{dest}|}"|
\end{center}
%
Here \textit{target} is the name of the output file,
\textit{main} is the name of the main file
and \textit{dest} is the name of the main or child file to be processed
(all filenames without extensions).
The optional argument \textit{main} can be omitted
if \textit{main} matches \textit{dest}.
Optionally, compilation \textit{flags} can be defined via |\def| commands.
This command line makes the \TeX{} engine believe
it is compiling the file \textit{target}
whose content is specified as the latter parameter.
The provided code then forwards the processing to
\textit{main} or \textit{dest} as described in \secref{sec:forward}.

%%%%%%%%%%%%%%%%%%%%%%%%%%%%%%%%%%%%%%%%%%%%%%%%%%%%%%%%%%%%%%%%%%%%%%%%%%%%%%%%
\subsection{Include by Input}
\label{sec:input}

Including child documents by |\include| has some restrictions by design.
Most notably, the content of a child document always occupies
its own set of pages; pages cannot be shared between child documents.
Usually, this behaviour makes perfect sense
because each child document contain an essential part of the document.
However, in some situations it may be desirable to compose
a document from a collection of parts
without having mandatory page breaks between then.
For this case, the package
provides a mechanism to include parts
by |\input| which can also be processed individually.
However, by construction this mechanism
requires manual handling of the content to be output.

%%%%%%%%%%%%%%%%%%%%%%%%%%%%%%%%%%%%%%%%
\DescribeMacro{\ifchilddocmanual}
The main file should be prepared as usual, see \secref{sec:include}.
However, the document body must make a distinction
between processing of an individual part and of the main document, e.g.:
%
\begin{center}
\begin{tabular}{l}
|\ifchilddocmanual|\\
|\input{\childdocname}|\\
|\||else|\\
\textit{document body with }|\input{|\textit{part}|}|\\
|\||fi|
\end{tabular}
\end{center}
%
The conditional |\ifchilddocmanual| is true whenever
a part to be included by |\input| is being compiled,
and the name of the part is stored in |\childdocname|.

%%%%%%%%%%%%%%%%%%%%%%%%%%%%%%%%%%%%%%%%
\DescribeMacro{\childdocby}
Each part to be included by |\input| should start with:
%
\begin{center}
\begin{tabular}{l}
|\input{childdoc.def}|\\
|\childdocby{|\textit{main}|}|\\
\end{tabular}
\end{center}
%
The directive |\childdocby| is similar to |\childdocof|
described in \secref{sec:include},
but the subsequent selection of content must be done manually.
To that end, both |\ifchilddoc| and |\ifchilddocmanual|
will be true upon processing of a part,
and the name of the part is stored in |\childdocname|.
Note that |\jobname| will be set to the filename of the current part
so that each part receives an individual |.aux| file
that does not interfere with the |.aux| file(s) of the main document.
This behaviour can be altered by the alternative form
|\childdocby[*]{|\textit{main}|}| (with a non-empty optional argument)
which uses the |.aux| file of the main document
by setting |\jobname| to \textit{main}.

%%%%%%%%%%%%%%%%%%%%%%%%%%%%%%%%%%%%%%%%%%%%%%%%%%%%%%%%%%%%%%%%%%%%%%%%%%%%%%%%
\subsection{Driver Development}
\label{sec:driver}

The \textsf{childdoc} mechanism can also be use for the development
of definition files such as \LaTeX{} styles or classes.
This case differs from the above setup with multiple parts
included by |\include| in that no |\includeonly| should be invoked.
This can be achieved by starting the include file
(before |\ProvidesPackage|) with:
%
\begin{center}
\begin{tabular}{l}
|\input{childdoc.def}|\\
|\childdocforward{|\textit{main}|}|\\
\end{tabular}
\end{center}
%
or alternatively with:
%
\begin{center}
\begin{tabular}{l}
|\input{childdoc.def}|\\
|\childdocby{|\textit{main}|}|\\
\end{tabular}
\end{center}
%
Both forms have slightly different effects as described above.
The main file is prepared as usual, see \secref{sec:include}.

%%%%%%%%%%%%%%%%%%%%%%%%%%%%%%%%%%%%%%%%%%%%%%%%%%%%%%%%%%%%%%%%%%%%%%%%%%%%%%%%
\subsection{Legacy Detection}
\label{sec:detection}

The directive |\childdocmain| in the main file can detect
whether the complete document or merely a child is to be compiled
even without using the directive |\childdocof|.
This method is deprecated because it is less robust
and there is no compelling reason to use it;
it is merely provided for backward compatibility
and it may be removed in future versions.

If the detection mechanism is to be used,
it is mandatory to correctly specify
the filename of the main file as the argument of |\childdocmain|:
%
\begin{center}
\begin{tabular}{l}
|\input{childdoc.def}|\\
|\childdocmain{|\textit{main}|}|\\
\end{tabular}
\end{center}
%
If |\jobname| does not match the argument \textit{main} of |\childdocmain|,
it is assumed that |\jobname| points to the child file to be compiled.
When using |\childdocmain| with the main file specified as argument,
it suffices to start a child file
with just |\input{|\textit{main}|}|
without loading of the package and using |\childdocof|.
If instead all processing is done
with the appropriate \textsf{childdoc} directives,
the argument of \textit{main} of |\childdocmain| can be empty.

An alternative version of the command line processing described
in \secref{sec:commandline} using the detection mechanism reads:
%
\begin{center}
|... -jobname "|\textit{target}|" "|[\textit{flags}]%
[|\def\jobname{|\textit{dest}|}|]|\input{|\textit{main}|}"|
\end{center}

%%%%%%%%%%%%%%%%%%%%%%%%%%%%%%%%%%%%%%%%%%%%%%%%%%%%%%%%%%%%%%%%%%%%%%%%%%%%%%%%
\subsection{Manual Code}
\label{sec:manual}

In case one cannot be certain whether the definitions file |childdoc.def|
is installed on the target \TeX{} distribution
and one prefers not to ship it,
it is conceivable to paste a few relevant commands into the sources.

To that end, drop all statements |\input{childdoc.def}|
and perform the replacements as outlined below.
Instead of |\childdocmain{|\textit{main}|}| add the following code
to the top of the main file:
%
\begin{center}
\begin{tabular}{l}
|\||ifdefined\childdocname\endinput\||fi\newif\ifchilddoc|\\
|\edef\childdocname{\scantokens\expandafter{\jobname\noexpand}}|\\
|\def\childdocmain{|\textit{main}|}\||ifx\childdocmain\childdocname\||else|\\
|\childdoctrue\includeonly{\childdocname}\let\jobname\childdocmain\||fi|\\
\end{tabular}
\end{center}
%
Instead of |\childdocof{|\textit{main}|}| just include the main file
at the top of each child file:
%
\begin{center}
|\input{|\textit{main}|}|
\end{center}
%
A simple redirection |\childdocforward{|\textit{dest}|}| is achieved by:
%
\begin{center}
|\def\jobname{|\textit{dest}|}\input{\jobname}|
\end{center}
%
The redirection with prefix
|\childdocforwardprefix[|\textit{prefix}|]{|\textit{dest}|}|
is accomplished by:
%
\begin{center}
\begin{tabular}{l}
|{\edef\jobname{\scantokens\expandafter{\jobname\noexpand}}|\\
|\def\redirectjob |\textit{prefix}|#1~~~{\gdef\jobname{|\textit{dest}|#1}}|\\
|\expandafter\redirectjob\jobname~~~}\input{\jobname}|
\end{tabular}
\end{center}

In an alternative approach,
child documents can be compiled by a specific command line
without additional code or specific definitions:
%
\begin{center}
|... -jobname "|\textit{target}|" "|[\textit{flags}]%
|\includeonly{|\textit{dest}|}\input{|\textit{main}|}"|
\end{center}
%

%%%%%%%%%%%%%%%%%%%%%%%%%%%%%%%%%%%%%%%%%%%%%%%%%%%%%%%%%%%%%%%%%%%%%%%%%%%%%%%%
%%%%%%%%%%%%%%%%%%%%%%%%%%%%%%%%%%%%%%%%%%%%%%%%%%%%%%%%%%%%%%%%%%%%%%%%%%%%%%%%
\section{Information}

%%%%%%%%%%%%%%%%%%%%%%%%%%%%%%%%%%%%%%%%%%%%%%%%%%%%%%%%%%%%%%%%%%%%%%%%%%%%%%%%
\subsection{Copyright}

Copyright \copyright{} 2017--2018 Niklas Beisert

This work may be distributed and/or modified under the
conditions of the \LaTeX{} Project Public License, either version 1.3
of this license or (at your option) any later version.
The latest version of this license is in
  \url{http://www.latex-project.org/lppl.txt}
and version 1.3 or later is part of all distributions of \LaTeX{}
version 2005/12/01 or later.

This work has the LPPL maintenance status `maintained'.

The Current Maintainer of this work is Niklas Beisert.

This work consists of the files |README.txt|, |childdoc.ins| and |childdoc.dtx|
as well as the derived files |childdoc.def|, |cdocsamp.tex|
with |cdocsch1.tex|, |cdocsch2.tex|, |cdocspt3.tex|, |cdocspt4.tex|,
|cdocsdrf.tex|, |cdocsfn1.tex|, |cdocsfn2.tex|
as well as |childdoc.pdf|.

%%%%%%%%%%%%%%%%%%%%%%%%%%%%%%%%%%%%%%%%%%%%%%%%%%%%%%%%%%%%%%%%%%%%%%%%%%%%%%%%
\subsection{Files and Installation}

The package consists of the files:
%
\begin{center}
\begin{tabular}{ll}
    |README.txt|   & readme file \\
    |childdoc.ins| & installation file \\
    |childdoc.dtx| & source file \\
    |childdoc.def| & definition file \\
    |cdocsamp.tex| & sample main file \\
    |cdocsch1.tex| & sample include file \\
    |cdocsch2.tex| & sample include file \\
    |cdocspt3.tex| & sample part file \\
    |cdocspt4.tex| & sample part file \\
    |cdocsdrf.tex| & sample redirection file \\
    |cdocsfn1.tex| & sample redirection file \\
    |cdocsfn2.tex| & sample redirection file \\
    |childdoc.pdf| & manual
\end{tabular}
\end{center}
%
The distribution consists of the files
|README.txt|, |childdoc.ins| and |childdoc.dtx|.
%
\begin{itemize}
\item
Run (pdf)\LaTeX{} on |childdoc.dtx|
to compile the manual |childdoc.pdf| (this file).
\item
Run \LaTeX{} on |childdoc.ins| to create the definitions file |childdoc.def|
and the sample |cdocsamp.tex| with include files
|cdocsch1.tex|, |cdocsch2.tex|, |cdocspt3.tex|, |cdocspt4.tex|,
|cdocsdrf.tex|, |cdocsfn1.tex|, |cdocsfn2.tex|.
Then copy the file |childdoc.def| to an appropriate directory of your \LaTeX{}
distribution, e.g.\ \textit{texmf-root}|/tex/latex/childdoc|.
\end{itemize}

%%%%%%%%%%%%%%%%%%%%%%%%%%%%%%%%%%%%%%%%%%%%%%%%%%%%%%%%%%%%%%%%%%%%%%%%%%%%%%%%
\subsection{Related CTAN Packages}

There are several other packages which offer a similar functionality:
%
\begin{itemize}
\item
The packages
\href{http://ctan.org/pkg/docmute}{\textsf{docmute}},
\href{http://ctan.org/pkg/includex}{\textsf{includex}} and
\href{http://ctan.org/pkg/standalone}{\textsf{standalone}}
provide commands to include only the document body of
a child file thus allowing both files to be compiled individually.
\item
The packages \href{http://ctan.org/pkg/subdocs}{\textsf{subdocs}}
and \href{http://ctan.org/pkg/subfiles}{\textsf{subfiles}}
provide structures in which the main and child documents can be
encapsulated and allowing them to be compiled individually.
The inclusion mechanism is different from the conventional |\include|.
\item
The package \href{http://ctan.org/pkg/combine}{\textsf{combine}}
is an elaborate solution to combine several documents into one.
\end{itemize}
%
See also the CTAN topic \href{http://ctan.org/topic/subdocs}{\textsf{subdocs}}
for further related packages.
The present package differs from the above solutions in that
a document structure constructed with the conventional |\include| mechanism
just needs two extra commands at the top of every file
such that all constituent files can be compiled individually.

%%%%%%%%%%%%%%%%%%%%%%%%%%%%%%%%%%%%%%%%%%%%%%%%%%%%%%%%%%%%%%%%%%%%%%%%%%%%%%%%
%\subsection{Feature Suggestions}
%
%The following is a list of features which may be useful for future
%versions of this package:
%%
%\begin{itemize}
%\item
%\ldots
%\end{itemize}

%%%%%%%%%%%%%%%%%%%%%%%%%%%%%%%%%%%%%%%%%%%%%%%%%%%%%%%%%%%%%%%%%%%%%%%%%%%%%%%%
\subsection{Revision History}

%%%%%%%%%%%%%%%%%%%%%%%%%%%%%%%%%%%%%%%%
\paragraph{v2.0:} 2018/12/30

\begin{itemize}
\item
immediate forward processing
\item
added |\childdocby| mechanism
\item
manual restructured
\end{itemize}

%%%%%%%%%%%%%%%%%%%%%%%%%%%%%%%%%%%%%%%%
\paragraph{v1.6:} 2018/01/17

\begin{itemize}
\item
application for development of include files
\item
corrections to manual
\end{itemize}

%%%%%%%%%%%%%%%%%%%%%%%%%%%%%%%%%%%%%%%%
\paragraph{v1.5:} 2017/05/21

\begin{itemize}
\item
more complete structuring introduced
\item
|\childdocof| introduced
\item
|\childdoc| renamed to |\childdocmain|
\item
|\childredirect| renamed to |\childdocforward| and |\childdocforwardprefix|
and functionality expanded
\end{itemize}

%%%%%%%%%%%%%%%%%%%%%%%%%%%%%%%%%%%%%%%%
\paragraph{v1.0:} 2017/04/27

\begin{itemize}
\item
manual and install package
\item
first version published on CTAN
\end{itemize}

%%%%%%%%%%%%%%%%%%%%%%%%%%%%%%%%%%%%%%%%
\paragraph{v0.6:} 2017/04/26

\begin{itemize}
\item
redirection mechanism added
\end{itemize}

%%%%%%%%%%%%%%%%%%%%%%%%%%%%%%%%%%%%%%%%
\paragraph{v0.5:} 2017/04/26

\begin{itemize}
\item
functionality in definition file
\end{itemize}


%%%%%%%%%%%%%%%%%%%%%%%%%%%%%%%%%%%%%%%%%%%%%%%%%%%%%%%%%%%%%%%%%%%%%%%%%%%%%%%%
%%%%%%%%%%%%%%%%%%%%%%%%%%%%%%%%%%%%%%%%%%%%%%%%%%%%%%%%%%%%%%%%%%%%%%%%%%%%%%%%
%%%%%%%%%%%%%%%%%%%%%%%%%%%%%%%%%%%%%%%%%%%%%%%%%%%%%%%%%%%%%%%%%%%%%%%%%%%%%%%%
\appendix

\settowidth\MacroIndent{\rmfamily\scriptsize 000\ }

 \DocInput{childdoc.dtx}

\end{document}
%</driver>
% \fi
%
% %%%%%%%%%%%%%%%%%%%%%%%%%%%%%%%%%%%%%%%%%%%%%%%%%%%%%%%%%%%%%%%%%%%%%%%%%%%%%%
% %%%%%%%%%%%%%%%%%%%%%%%%%%%%%%%%%%%%%%%%%%%%%%%%%%%%%%%%%%%%%%%%%%%%%%%%%%%%%%
% \section{Sample}
%\iffalse
%<*samplemain>
%\fi
%
% The following presents a sample document
% with two chapters, two parts, a title page,
% a compile flag as well as three forwarding files to set the flag.
% It consists of eight |.tex| files:
% \begin{center}
% \begin{tabular}{ll}
% |cdocsamp.tex|&main file\\
% |cdocsch1.tex|&include file for chapter 1\\
% |cdocsch2.tex|&include file for chapter 2\\
% |cdocspt3.tex|&include file for part 3\\
% |cdocspt4.tex|&include file for part 4\\
% |cdocsdrf.tex|&forwarding file for main file in draft mode\\
% |cdocsfi1.tex|&forwarding file for final version of chapter 1\\
% |cdocsfi2.tex|&forwarding file for final version of chapter 2\\
% \end{tabular}
% \end{center}
% Each of the eight files can be compiled directly by the \LaTeX{} compiler.
%
% %%%%%%%%%%%%%%%%%%%%%%%%%%%%%%%%%%%%%%
% \paragraph{Main File.}
%
% The main file is called |cdocsamp.tex|.
%
% Load the \textsf{childdoc} definitions and
% declare the filename for the main document:
%    \begin{macrocode}
\input{childdoc.def}
\childdocmain{}
%    \end{macrocode}

% Optional override for |\version| flag:
%    \begin{macrocode}
%%\ifchilddoc\else\providecommand{\version}{draft}\fi
%    \end{macrocode}

% Define the default values for the |\version| flag
% (|final| for the main file and |draft| for childs):
%    \begin{macrocode}
\ifchilddoc
\providecommand{\version}{draft}
\else
\providecommand{\version}{final}
\fi
%    \end{macrocode}

% Load the standard document class:
%    \begin{macrocode}
\documentclass[12pt]{article}
%    \end{macrocode}

% Start the document body:
%    \begin{macrocode}
\begin{document}
%    \end{macrocode}

% Declare a title page.
% Print title, part of document being processed and version flag:
%    \begin{macrocode}
\addtocounter{page}{-1}
\begin{center}
{\LARGE\bfseries{}childdoc example\par}
\vspace{1cm}
\ifchilddoc
\ifchilddocmanual part\else chapter\fi:
`\childdocname' of `\childdocjob'\par
\else
main document: `\childdocjob'\par
\fi
version: \version\par
\end{center}
\newpage
%    \end{macrocode}

% Manually include selected file,
% otherwise process as usual:
%    \begin{macrocode}
\ifchilddocmanual
\section*{part `\childdocname'}
\input{\childdocname}
\else
%    \end{macrocode}

% Include the two chapters:
%    \begin{macrocode}
\include{cdocsch1}
\include{cdocsch2}
%    \end{macrocode}

% Include the two parts unless only chapters should be displayed:
%    \begin{macrocode}
\ifchilddoc\else
\section{part three}
\input{cdocspt3}
\section{part four}
\input{cdocspt4}
\fi
%    \end{macrocode}

% Process as usual until here:
%    \begin{macrocode}
\fi
%    \end{macrocode}

% End of document body:
%    \begin{macrocode}
\end{document}
%    \end{macrocode}
%\iffalse
%</samplemain>
%\fi
%
% %%%%%%%%%%%%%%%%%%%%%%%%%%%%%%%%%%%%%%
% \paragraph{Chapter Include Files.}
%
% The include files are called |cdocsch1.tex| and |cdocsch2.tex|.
%
%\iffalse
%<*samplechap1|samplechap2>
%\fi

% Optional override for |\version| flag:
%    \begin{macrocode}
%%\providecommand{\version}{final}
%    \end{macrocode}

% Include the main document:
%    \begin{macrocode}
\input{childdoc.def}
\childdocof{cdocsamp}
%    \end{macrocode}

%\iffalse
%</samplechap1|samplechap2>
%\fi
%
%\iffalse
%<*samplechap1>
%\fi
% Some text for chapter 1:
%    \begin{macrocode}
\section{one}
some text in chapter one
%    \end{macrocode}

%\iffalse
%</samplechap1>
%\fi
% Some text for chapter 2:
%\iffalse
%<*samplechap2>
%\fi
%    \begin{macrocode}
\section{two}
more text in chapter two
%    \end{macrocode}

%\iffalse
%</samplechap2>
%\fi
%
% %%%%%%%%%%%%%%%%%%%%%%%%%%%%%%%%%%%%%%
% \paragraph{Part Include Files.}
%
% The include files are called |cdocspt3.tex| and |cdocspt4.tex|.
%
%\iffalse
%<*samplepart3|samplepart4>
%\fi

% Optional override for |\version| flag:
%    \begin{macrocode}
%%\providecommand{\version}{final}
%    \end{macrocode}

% Include the main document:
%    \begin{macrocode}
\input{childdoc.def}
\childdocby{cdocsamp}
%    \end{macrocode}

%\iffalse
%</samplepart3|samplepart4>
%\fi
%
%\iffalse
%<*samplepart3>
%\fi
% Some text for part 3:
%    \begin{macrocode}
some text in part three
%    \end{macrocode}

%\iffalse
%</samplepart3>
%\fi
% Some text for part 4:
%\iffalse
%<*samplepart4>
%\fi
%    \begin{macrocode}
more text in part four
%    \end{macrocode}

%\iffalse
%</samplepart4>
%\fi
%
% %%%%%%%%%%%%%%%%%%%%%%%%%%%%%%%%%%%%%%
% \paragraph{Forwarding for a Complete Draft.}
%
% The following forwarding file |cdocsdrf.tex|
% compiles the main document in draft mode:
%\iffalse
%<*sampledraft>
%\fi
%    \begin{macrocode}
\def\version{draft}
\input{childdoc.def}
\childdocforward{cdocsamp}
%    \end{macrocode}

%\iffalse
%</sampledraft>
%\fi
%
% %%%%%%%%%%%%%%%%%%%%%%%%%%%%%%%%%%%%%%
% \paragraph{Forwarding for Final Version of the Chapters.}
%
% The following forwarding files |cdocsfn1.tex| and |cdocsfn2.tex|
% (with identical content)
% compile the final versions of the child documents
% |cdocsch1.tex| and |cdocsch2.tex|, respectively:
%\iffalse
%<*samplefinal>
%\fi
%    \begin{macrocode}
\def\version{final}
\input{childdoc.def}
\childdocforwardprefix[cdocsamp]{cdocsfn}{cdocsch}
%    \end{macrocode}

%\iffalse
%</samplefinal>
%\fi
%
% %%%%%%%%%%%%%%%%%%%%%%%%%%%%%%%%%%%%%%
% \paragraph{Command Line Processing.}
%
% The following three command lines generate the output files
% |cdocscld|, |cdocscl1| and |cdocscl2|
% which should be identical to
% |cdocsdrf|, |cdocsch1| and |cdocsfn2|, respectively:
% \begin{center}
% \begin{tabular}{l}
% |latex -jobname cdocscld \|\\
% |  "\def\version{draft}\input{childdoc.def}\childdocforward{cdocsamp}"|\\
% |latex -jobname cdocscl1 \|\\
% |  "\input{childdoc.def}\childdocforward[cdocsamp]{cdocsch1}"|\\
% |latex -jobname cdocscl2 \|\\
% |  "\def\version{final}\input{childdoc.def}\childdocforward{cdocsch2}"|
% \end{tabular}
% \end{center}
% Note that the trailing backslash on each first line
% merely continues the input to the second line
% (for convenient cut ant paste).
% Furthermore, the command |latex| can be replaced by any
% of its alternative versions such as |pdflatex|.
%
% %%%%%%%%%%%%%%%%%%%%%%%%%%%%%%%%%%%%%%%%%%%%%%%%%%%%%%%%%%%%%%%%%%%%%%%%%%%%%%
% %%%%%%%%%%%%%%%%%%%%%%%%%%%%%%%%%%%%%%%%%%%%%%%%%%%%%%%%%%%%%%%%%%%%%%%%%%%%%%
% \section{Implementation}
%\iffalse
%<*package>
%\fi
%
% This section describes the definitions file |childdoc.def|.

% The definitions cannot be loaded using |\usepackage| or |\RequirePackage|
% which has a mechanism to prevent loading a style file more than once.
% When loading the definitions by means of |\input|
% multiple instances have to be prevented manually:
%\iffalse
%This code needs to be before the `\ProvidesFile' directive
%which is defined at the beginning of this file.
%Therefore it is also placed there and commented out here.
%</package>
%<*discard>
%\fi
%    \begin{macrocode}
\ifdefined\childdocmain\endinput\fi
%    \end{macrocode}
%\iffalse
%</discard>
%<*package>
%\fi
%
% \macro{\ifchilddoc}
% \macro{\ifchilddocmanual}
% The conditional |\ifchilddoc| tells whether a
% child (true) or main (false) document is being compiled.
% The conditional |\ifchilddocmanual| tells whether
% the |\includeonly| mechanism is used (false) or
% the selection of child files must be performed manually (true).
% The definitions initialise to false:
%    \begin{macrocode}
\newif\ifchilddoc
\newif\ifchilddocmanual
%    \end{macrocode}

% \macro{\childdocname}
% \macro{\childdocjob}
% The macro |\childdocname| stores the name of the main document
% to be compiled. The macro |\childdocjob| stores the name of
% the document on which the \LaTeX{} compiler was originally invoked.
% The content of |\jobname| cannot be compared
% to filenames specified in the source due to different catcodes.
% The following code rescans |\jobname|, stores the result
% in |\childdocname| and saves a copy in |\childdocjob|:
%    \begin{macrocode}
\edef\childdocname{\scantokens\expandafter{\jobname\noexpand}}
\let\childdocjob\childdocname
%    \end{macrocode}

% \macro{\childdocdisable}
% The macro |\childdocdisable| prevents the main file
% from being processed more than once.
% At this stage, the main document command |\childdocmain|
% is assumed to be called once again where it should do nothing.
% Any subsequent call to it should prevent
% a secondary processing of the main document
% It overwrites the forwarding commands
% |\childdocof| and |\childdocforward|
% with empty macros to prevent further inclusions of the main document:
%    \begin{macrocode}
\newcommand{\childdocdisable}
{
  \renewcommand{\childdocmain}[1]{\renewcommand{\childdocmain}[1]{\endinput}}
  \renewcommand{\childdocof}[1]{}
  \renewcommand{\childdocby}[2][]{}
  \renewcommand{\childdocforward}[2][]{}
  \renewcommand{\childdocdisable}{}
}
%    \end{macrocode}

% \macro{\childdocmain}
% The macro |\childdocmain| is to be called at the top of the main file
% with nothing or the main filename (without extension) as argument.
% First, it breaks loops.
% If the argument is not empty and does not match |\childdocname|
% (which is set by the first inclusion of |childdoc.def|),
% |\ifchilddoc| is set to true, |\includeonly| is applied to the child file
% and |\jobname| is set to the main file
% (for proper handling of |.aux| files):
%    \begin{macrocode}
\newcommand{\childdocmain}[1]
{
  \childdocdisable\childdocmain{}
  \if?#1?\else
    \begingroup
      \def\childdoctmp{#1}
      \ifx\childdoctmp\childdocname
        \def\childdoctmp{}
      \else
        \def\childdoctmp
        {
          \childdoctrue
          \includeonly{\childdocname}
          \def\childdocjob{#1}
          \def\jobname{#1}
        }
      \fi
      \expandafter
    \endgroup
    \childdoctmp
  \fi
}
%    \end{macrocode}

% \macro{\childdocof}
% The command |\childdocof| redirects
% compilation to the main file |#1|.
%    \begin{macrocode}
\newcommand{\childdocof}[1]
{
  \childdocdisable
  \childdoctrue
  \includeonly{\childdocname}
  \def\jobname{#1}
  \def\childdocjob{#1}
  \input{#1}
}
%    \end{macrocode}

% \macro{\childdocby}
% The command |\childdocby| ....
%    \begin{macrocode}
\newcommand{\childdocby}[2][]
{
  \childdocdisable
  \childdoctrue
  \childdocmanualtrue
  \if?#1?\else
    \def\jobname{#2}
  \fi
  \def\childdocjob{#2}
  \input{#2}
  \endinput
}
%    \end{macrocode}

% \macro{\childdocforward}
% The command |\childdocforward| redirects
% compilation to the main file or
% (if the optional argument is given) a child file.
% Parameters are set as if the main file
% or a child file starting with |\childdocof| was compiled.
% Then compilation is handed over to the main file:
%    \begin{macrocode}
\newcommand{\childdocforward}[2][]
{
  \begingroup
    \if?#1?
      \def\childdoctmp
      {
        \def\childdocname{#2}
        \def\childdocjob{#2}
        \def\jobname{#2}
        \input{#2}
        \endinput
      }
    \else
      \def\childdoctmp
      {
        \childdocdisable
        \def\childdocname{#2}
        \childdoctrue
        \includeonly{#2}
        \def\childdocjob{#1}
        \def\jobname{#1}
        \input{#1}
        \endinput
      }
    \fi
    \expandafter
  \endgroup
  \childdoctmp
}
%    \end{macrocode}

% \macro{\childdocforwardprefix}
% The command |\childdocforwardprefix| redirects
% compilation to the main or a child file by means of a pattern.
% The prefix |#1| in the current filename is replaced by |#2|
% and the suffix of the current filename is kept
% (it is assumed that the filename does not contain the substring `|~~~|'
% which is used as a delimiter).
% Compilation is handed over to the new file by |\childdocforward|:
%    \begin{macrocode}
\newcommand{\childdocforwardprefix}[3][]
{
  \begingroup
    \def\childdocextract #2##1~~~{\def\childdoctmp{\childdocforward[#1]{#3##1}}}
    \expandafter\childdocextract\childdocname~~~
    \expandafter
  \endgroup
  \childdoctmp
}
%    \end{macrocode}

% \macro{\childdoc}
% The deprecated macro |\childdoc| is a legacy version of |\childdocmain|:
%    \begin{macrocode}
\newcommand{\childdoc}{\childdocmain}
%    \end{macrocode}

% \macro{\childdocredirect}
% The deprecated macro |\childdocredirect| is a legacy version
% of |\childdocforward| and |\childdocforwardprefix|:
%    \begin{macrocode}
\newcommand{\childdocredirect}[2][]
{
  \begingroup
    \if?#1?
      \def\childdoctmp{\childdocforward{#2}}
    \else
      \def\childdoctmp{\childdocforwardprefix{#1}{#2}}
    \fi
    \expandafter
  \endgroup
  \childdoctmp
}
%    \end{macrocode}

%\iffalse
%</package>
%\fi
%
\endinput
|\\
|\childdocforwardprefix{final}{child}|
\end{tabular}
\end{center}
%

Note that when several versions of a main file and/or of each child file
are to be generated, it may be convenient to set up a |Makefile| or
shell script to automatise the process.

%%%%%%%%%%%%%%%%%%%%%%%%%%%%%%%%%%%%%%%%%%%%%%%%%%%%%%%%%%%%%%%%%%%%%%%%%%%%%%%%
\subsection{Command Line Processing}
\label{sec:commandline}

The effect of redirection files can also be achieved by invoking
the \LaTeX{} compiler with a more elaborate command line.
Most conveniently this should be done as part
of a shell script or a |Makefile|.

When using \textsf{childdoc} in the main file, the following
command lines effectively perform a redirection
(note that depending on the shell being used,
backslashes may have to be doubled: `|\|' $\to$ `|\\|'):
%
\begin{center}
|... -jobname "|\textit{target}|" |\\|"|[\textit{flags}]%
|% \iffalse
%
% childdoc.dtx Copyright (C) 2017-2018 Niklas Beisert
%
% This work may be distributed and/or modified under the
% conditions of the LaTeX Project Public License, either version 1.3
% of this license or (at your option) any later version.
% The latest version of this license is in
%   http://www.latex-project.org/lppl.txt
% and version 1.3 or later is part of all distributions of LaTeX
% version 2005/12/01 or later.
%
% This work has the LPPL maintenance status `maintained'.
%
% The Current Maintainer of this work is Niklas Beisert.
%
% This work consists of the files childdoc.dtx and childdoc.ins
% and the derived files childdoc.def and cdocsamp.tex with
% cdocsch1.tex, cdocsch2.tex, cdocsdrf.tex, cdocsfn1.tex, cdocsfn2.tex.
%
%<package>\ifdefined\childdocmain\endinput\fi
%<package>\ProvidesFile{childdoc.def}[2018/12/30 v2.0 child document driver]
%<samplemain>\ProvidesFile{cdocsamp.tex}[2018/12/30 v2.0 sample for childdoc]
%<*driver>
%\ProvidesFile{childdoc.drv}[2018/12/30 v2.0 childdoc reference manual file]
\PassOptionsToClass{10pt,a4paper}{article}
\documentclass{ltxdoc}

\usepackage[margin=35mm]{geometry}
\usepackage{hyperref}
\usepackage{hyperxmp}
\usepackage[usenames]{color}

\hypersetup{colorlinks=true}
\hypersetup{pdfstartview=FitH}
\hypersetup{pdfpagemode=UseNone}
\hypersetup{pdfsource={}}
\hypersetup{pdflang={en-UK}}
\hypersetup{pdfcopyright={Copyright 2017-2018 Niklas Beisert.
  This work may be distributed and/or modified under the
  conditions of the LaTeX Project Public License, either version 1.3
  of this license or (at your option) any later version.}}
\hypersetup{pdflicenseurl={http://www.latex-project.org/lppl.txt}}
\hypersetup{pdfcontactaddress={ETH Zurich, ITP, HIT K,
  Wolfgang-Pauli-Strasse 27}}
\hypersetup{pdfcontactpostcode={8093}}
\hypersetup{pdfcontactcity={Zurich}}
\hypersetup{pdfcontactcountry={Switzerland}}
\hypersetup{pdfcontactemail={nbeisert@itp.phys.ethz.ch}}
\hypersetup{pdfcontacturl={http://people.phys.ethz.ch/\xmptilde nbeisert/}}

\newcommand{\secref}[1]{\hyperref[#1]{section \ref*{#1}}}

\parskip1ex
\parindent0pt
\let\olditemize\itemize
\def\itemize{\olditemize\parskip0pt}

\begin{document}

\title{The \textsf{childdoc} Package}
\hypersetup{pdftitle={The childdoc Package}}
\author{Niklas Beisert\\[2ex]
  Institut f\"ur Theoretische Physik\\
  Eidgen\"ossische Technische Hochschule Z\"urich\\
  Wolfgang-Pauli-Strasse 27, 8093 Z\"urich, Switzerland\\[1ex]
  \href{mailto:nbeisert@itp.phys.ethz.ch}
  {\texttt{nbeisert@itp.phys.ethz.ch}}}
\hypersetup{pdfauthor={Niklas Beisert}}
\hypersetup{pdfsubject={Manual for the LaTeX2e Package childdoc}}
\date{30 December 2018, \textsf{v2.0}}
\maketitle

\begin{abstract}\noindent
\textsf{childdoc} is a \LaTeXe{} package
that enables the direct compilation
of document sections included by |\include|
to individual files.
\end{abstract}

\begingroup
\parskip0ex
\tableofcontents
\endgroup

%%%%%%%%%%%%%%%%%%%%%%%%%%%%%%%%%%%%%%%%%%%%%%%%%%%%%%%%%%%%%%%%%%%%%%%%%%%%%%%%
%%%%%%%%%%%%%%%%%%%%%%%%%%%%%%%%%%%%%%%%%%%%%%%%%%%%%%%%%%%%%%%%%%%%%%%%%%%%%%%%
\section{Introduction}

\LaTeX{} provides a mechanism to structure a large document (such as a book)
into a main file and several child files (containing the chapters)
using the |\include| command.
This mechanism is beneficial for documents
which span hundreds of pages in order to
make the source file(s) more manageable.
Moreover, compilation can be restricted to
selected child files by means of the |\includeonly| command.
The latter feature can be used to reduce the compilation time while editing
(this was significantly more useful in the earlier days of \LaTeX{})
or to generate a smaller document which is easier to navigate.
Another application of |\includeonly| is to generate
documents consisting of selected parts of the complete document.

However, there are a few drawbacks of the plain |\include| mechanism:
\begin{itemize}
\item
The child files cannot be compiled on their own,
they can only be compiled via the main file.
A naive editing environment
(such as a text editor with an option
to have the current file processed by \LaTeX)
may require one to switch to the main file before compiling;
attempting to compile the child file produces errors.
\item
The main file must be modified (each time)
to adjust the |\includeonly| command
to the present needs. This easily leaves the main file in a messy state.
\item
The generated document will always carry the filename
of the main document. This is inconvenient if
several child files are to be compiled and
to be kept for distribution.
\end{itemize}

The present package provides a simple interface
to make child files individually compilable by \LaTeX{}.
Compiling a child file then has the same effect as compiling
the main file with an |\includeonly| command
to select the appropriate child.
Moreover the generated document will carry the name of the child
rather than the main file.
This resolves all three above issues.

This feature is meant to make the editing of books,
thesis documents and lecture notes somewhat more convenient.
However, the package can also be used efficiently for
composing a series of documents (such as exercise sheets)
which are typically distributed individually.
It then assists the author in generating the individual documents
(potentially in different versions)
as well as a document containing the collected series.
Another application is in developing style files
or other kinds of included material
where compilation of the style file could redirect
to a sample or test file.

%%%%%%%%%%%%%%%%%%%%%%%%%%%%%%%%%%%%%%%%%%%%%%%%%%%%%%%%%%%%%%%%%%%%%%%%%%%%%%%%
%%%%%%%%%%%%%%%%%%%%%%%%%%%%%%%%%%%%%%%%%%%%%%%%%%%%%%%%%%%%%%%%%%%%%%%%%%%%%%%%
\section{Usage}

First of all, the package \textsf{childdoc} is \emph{not} a standard
\LaTeXe{} |.sty| style file! Therefore it needs to be invoked in
a non-standard way.

%%%%%%%%%%%%%%%%%%%%%%%%%%%%%%%%%%%%%%%%%%%%%%%%%%%%%%%%%%%%%%%%%%%%%%%%%%%%%%%%
\subsection{Included Files}
\label{sec:include}

%%%%%%%%%%%%%%%%%%%%%%%%%%%%%%%%%%%%%%%%
\DescribeMacro{\childdocmain}
To use the package, add the commands
\begin{center}
\begin{tabular}{l}
|\input{childdoc.def}|\\
|\childdocmain{}|\\
\end{tabular}
\end{center}
at the very top of the main \LaTeX{} file,
in particular \emph{before} the |\documentclass| statement!
The argument of |\childdocmain| should be left empty
(but it must be present).

%%%%%%%%%%%%%%%%%%%%%%%%%%%%%%%%%%%%%%%%
\DescribeMacro{\childdocof}
Furthermore, add the commands
\begin{center}
\begin{tabular}{l}
|\input{childdoc.def}|\\
|\childdocof{|\textit{main}|}|\\
\end{tabular}
\end{center}
at the top of every child file \textit{child}
which is included by |\include{|\textit{child}|}|
from within the main file
(or at least for those files to be compiled individually).
The argument \textit{main} must be the filename of the main file.

There are a couple of
considerations in setting up the main and child documents:

%%%%%%%%%%%%%%%%%%%%%%%%%%%%%%%%%%%%%%%%
\paragraph{Restrictions.}

Please note the following restrictions:
\begin{itemize}
\item
|\childdocmain| must be called with one argument \textit{main}
to ensure compatibility with earlier version of the package.
It must either be empty (|\childdocmain{}|)
or precisely match the filename of the main file in which it is specified.
See \secref{sec:detection} for further information.
\item
The filename \textit{main} must be specified without the |.tex| extension.
\item
The filename \textit{main} is case sensitive
(even in case-insensitive file systems)
due to internal string comparison.
\item
The argument \textit{main} should be fully expanded, it cannot be a macro.
\item
Subdirectories and special characters should be avoided in filenames.
\item
The command |\childdocmain{|\textit{main}|}| must be followed by a whitespace.
It should not be followed immediately by another command
or by a comment mark `|%|'.
This is because the \TeX{} parser reads the token immediately following
the argument of |\childdocmain| and puts it
at the beginning of every child section;
however, a white\-space is ignored.
\end{itemize}

%%%%%%%%%%%%%%%%%%%%%%%%%%%%%%%%%%%%%%%%
\paragraph{Content of Main File.}

It is advisable to place all content in the child files included by |\include|.
Any output contained in the main file will appear in all child documents
unless suppressed manually;
it cannot be suppressed automatically by the |\includeonly| directive
and thus should normally be avoided.
A method to include some content in the main file
by means of conditional processing is described in \secref{sec:conditional}.

%%%%%%%%%%%%%%%%%%%%%%%%%%%%%%%%%%%%%%%%
\paragraph{Page Numbering.}

When only a part of the document is compiled,
the appropriate numbering of pages
(as well as other status parameters)
is determined from the |.aux| files.
The latter contain information from previous passes.
However this information needs to propagate through
all intermediate child documents.
Therefore the page numbering in child documents may well
be inconsistent until the complete document is compiled at least once.

A useful (if unconventional) way to always ensure a consistent
page numbering is to restart the numbering in each child document
and denote the pages by `\textit{child}|.|\textit{page}'
where \textit{child} represents the chapter/section number of the child file.
This can be achieved by the command
|\numberwithin{page}{|\textit{child}|}|
of the \textsf{amsmath} package
where \textit{child} can be |chapter| or |section|
depending on the chosen structuring.
Alternatively, one can modify the macro |\thepage| appropriately
and reset the counter |page| at the start of each child file.

%%%%%%%%%%%%%%%%%%%%%%%%%%%%%%%%%%%%%%%%%%%%%%%%%%%%%%%%%%%%%%%%%%%%%%%%%%%%%%%%
\subsection{Conditional Processing}
\label{sec:conditional}

The package provides a mechanism to compile different versions
of a document. To customise the versions further some conditional processing
can come in handy to distinguish which version is being compiled.
The package provides two macros to describe the compilation context:

%%%%%%%%%%%%%%%%%%%%%%%%%%%%%%%%%%%%%%%%
\DescribeMacro{\ifchilddoc}
The conditional |\ifchilddoc| distinguishes between the compilation of
child documents and the main document:
%
\begin{center}
|\ifchilddoc |\textit{child-code}| |[|\||else |\textit{main-code}]| \||fi|
\end{center}

%%%%%%%%%%%%%%%%%%%%%%%%%%%%%%%%%%%%%%%%
\DescribeMacro{\childdocname}
\DescribeMacro{\childdocjob}
The macro |\childdocname| contains the filename (without extension)
of the main or child file being processed.
Note that |\childdocjob| will always contain the name of the main file.

%%%%%%%%%%%%%%%%%%%%%%%%%%%%%%%%%%%%%%%%
\paragraph{Title Page.}

Conditional processing can be used to include a title or banner page
in the main document when proper precautions are taken.
Importantly, the code in the main file should ensure that the page counter
(as well as other status parameters which are stored in the |.aux| files)
takes the same value after the conditional processing.
Otherwise the page numbers may take divergent values
depending on which part is compiled.

For example, a title page could be declared by:
%
\begin{center}
\begin{tabular}{l}
|\ifchilddoc\||else|\\
|\addtocounter{page}{-1}|\\
\textit{code for title page}\\
|\newpage|\\
|\||fi|
\end{tabular}
\end{center}
%
A banner page for the child documents can be generated by:
%
\begin{center}
\begin{tabular}{l}
|\ifchilddoc|\\
|\addtocounter{page}{-1}|\\
\textit{code for banner page}\\
|\newpage|\\
|\||fi|
\end{tabular}
\end{center}
%
Here one could write a message such as:
\begin{center}
|This is the part \childdocname{} of \childdocjob{}.|
\end{center}

%%%%%%%%%%%%%%%%%%%%%%%%%%%%%%%%%%%%%%%%%%%%%%%%%%%%%%%%%%%%%%%%%%%%%%%%%%%%%%%%
\subsection{Flags}
\label{sec:flags}

The package makes it easy to generate different versions
of the main or child documents.
To this end compilation flags can be defined
and assigned different default values.
They will be particularly useful in conjunction
with the forwarding mechanism described in \secref{sec:forward}.

For example, it may be useful to have a flag |\version|
which can be set to |draft| or |final|.
The document source will contain some conditional code
depending on the value of |\version|.
Suppose further, the flag should default to |final| for the main file
and to |draft| for child files
which is a natural assignment for editing the document.
This is achieved by placing the following code
in the preamble of the main document
(below the |\childdocmain| directive):
%
\begin{center}
\begin{tabular}{l}
|\ifchilddoc|\\
|\providecommand{\version}{draft}|\\
|\||else|\\
|\providecommand{\version}{final}|\\
|\||fi|
\end{tabular}
\end{center}
%
The definition by |\providecommand| makes sure
that previous definitions are not overwritten.
Further statements |\providecommand{\version}{...}|
can thus be added before the above code to override it.

For the main file, one might add a line
(between |\childdocmain| and the above block)
%
\begin{center}
|%\ifchilddoc\||else\providecommand{\version}{draft}\||fi|
\end{center}
%
which can be uncommented to produce a draft version.
Likewise one can add a line to the very top of a child file
(above the |\childdocof{|\textit{main}|}| directive)
%
\begin{center}
|%\providecommand{\version}{final}|
\end{center}
%
which can be uncommented to produce the final version of this child document.

%%%%%%%%%%%%%%%%%%%%%%%%%%%%%%%%%%%%%%%%%%%%%%%%%%%%%%%%%%%%%%%%%%%%%%%%%%%%%%%%
\subsection{Forwarding}
\label{sec:forward}

Different versions of the main or child documents
using compilation flags as described in \secref{sec:flags}
can be (permanently) stored in different files
for convenient compilation, viewing and distribution.
To this end, the package defines a command
to pass on compilation to a different file:

%%%%%%%%%%%%%%%%%%%%%%%%%%%%%%%%%%%%%%%%
\DescribeMacro{\childdocforward}
The command |\childdocforward| redirects processing to
another source file:
%
\begin{center}
\begin{tabular}{l}
|\input{childdoc.def}|\\
|\childdocforward[|\textit{main}|]{|\textit{dest}|}|\\
\end{tabular}
\end{center}
%
The argument \textit{dest} is the destination file
(without extension).
It should be the main file or one of the child files.
Note that further \textsf{childdoc} directives
such as |\childdocof| and |\childdocforward|
in the indicated file will be processed in this form.
The optional argument \textit{main}
passes on directly to the main file \textit{main}
while pretending to compile the child \textit{dest}.
This form behaves as if \textit{dest}
issues |\childdocof{|\textit{main}|}| right away,
and no further \textsf{childdoc} directives will be processed.

%%%%%%%%%%%%%%%%%%%%%%%%%%%%%%%%%%%%%%%%
\DescribeMacro{\...prefix}
In the alternative form |\childdocforwardprefix|,
%
\begin{center}
\begin{tabular}{l}
|\input{childdoc.def}|\\
|\childdocforwardprefix[|\textit{main}|]{|\textit{prefix}|}{|\textit{dest}|}|
\end{tabular}
\end{center}
%
the destination file is determined by a pattern
depending on the current file:
To make this work, the current file must be called
`{\textit{prefix}\hspace{0.2em}\textit{suffix}}'
with \textit{prefix} matching precisely the argument.
Processing is then passed on to the file
`{\textit{dest}\hspace{0.2em}\textit{suffix}}'.
Surely, the same effect is achieved by
directly specifying the
argument `{\textit{dest}\hspace{0.2em}\textit{suffix}}'
in the first form.
However, that requires to set up a different file
for each child. With the alternative form of the command
all these files can have exactly the same content
which simplifies setting them up and maintaining them.

For example, the following file |draft.tex|
with a compilation flag |\version| as described in \secref{sec:flags}
compiles the main document as a draft:
%
\begin{center}
\begin{tabular}{l}
|\def\version{draft}|\\
|\input{childdoc.def}|\\
|\childdocforward{|\textit{main}|}|
\end{tabular}
\end{center}
%
Likewise, the following files |final|\textit{nn}|.tex|
compile the final version of the child document
|child|\textit{nn}|.tex|:
%
\begin{center}
\begin{tabular}{l}
|\def\version{final}|\\
|\input{childdoc.def}|\\
|\childdocforwardprefix{final}{child}|
\end{tabular}
\end{center}
%

Note that when several versions of a main file and/or of each child file
are to be generated, it may be convenient to set up a |Makefile| or
shell script to automatise the process.

%%%%%%%%%%%%%%%%%%%%%%%%%%%%%%%%%%%%%%%%%%%%%%%%%%%%%%%%%%%%%%%%%%%%%%%%%%%%%%%%
\subsection{Command Line Processing}
\label{sec:commandline}

The effect of redirection files can also be achieved by invoking
the \LaTeX{} compiler with a more elaborate command line.
Most conveniently this should be done as part
of a shell script or a |Makefile|.

When using \textsf{childdoc} in the main file, the following
command lines effectively perform a redirection
(note that depending on the shell being used,
backslashes may have to be doubled: `|\|' $\to$ `|\\|'):
%
\begin{center}
|... -jobname "|\textit{target}|" |\\|"|[\textit{flags}]%
|\input{childdoc.def}\childdocforward[|\textit{main}|]{|\textit{dest}|}"|
\end{center}
%
Here \textit{target} is the name of the output file,
\textit{main} is the name of the main file
and \textit{dest} is the name of the main or child file to be processed
(all filenames without extensions).
The optional argument \textit{main} can be omitted
if \textit{main} matches \textit{dest}.
Optionally, compilation \textit{flags} can be defined via |\def| commands.
This command line makes the \TeX{} engine believe
it is compiling the file \textit{target}
whose content is specified as the latter parameter.
The provided code then forwards the processing to
\textit{main} or \textit{dest} as described in \secref{sec:forward}.

%%%%%%%%%%%%%%%%%%%%%%%%%%%%%%%%%%%%%%%%%%%%%%%%%%%%%%%%%%%%%%%%%%%%%%%%%%%%%%%%
\subsection{Include by Input}
\label{sec:input}

Including child documents by |\include| has some restrictions by design.
Most notably, the content of a child document always occupies
its own set of pages; pages cannot be shared between child documents.
Usually, this behaviour makes perfect sense
because each child document contain an essential part of the document.
However, in some situations it may be desirable to compose
a document from a collection of parts
without having mandatory page breaks between then.
For this case, the package
provides a mechanism to include parts
by |\input| which can also be processed individually.
However, by construction this mechanism
requires manual handling of the content to be output.

%%%%%%%%%%%%%%%%%%%%%%%%%%%%%%%%%%%%%%%%
\DescribeMacro{\ifchilddocmanual}
The main file should be prepared as usual, see \secref{sec:include}.
However, the document body must make a distinction
between processing of an individual part and of the main document, e.g.:
%
\begin{center}
\begin{tabular}{l}
|\ifchilddocmanual|\\
|\input{\childdocname}|\\
|\||else|\\
\textit{document body with }|\input{|\textit{part}|}|\\
|\||fi|
\end{tabular}
\end{center}
%
The conditional |\ifchilddocmanual| is true whenever
a part to be included by |\input| is being compiled,
and the name of the part is stored in |\childdocname|.

%%%%%%%%%%%%%%%%%%%%%%%%%%%%%%%%%%%%%%%%
\DescribeMacro{\childdocby}
Each part to be included by |\input| should start with:
%
\begin{center}
\begin{tabular}{l}
|\input{childdoc.def}|\\
|\childdocby{|\textit{main}|}|\\
\end{tabular}
\end{center}
%
The directive |\childdocby| is similar to |\childdocof|
described in \secref{sec:include},
but the subsequent selection of content must be done manually.
To that end, both |\ifchilddoc| and |\ifchilddocmanual|
will be true upon processing of a part,
and the name of the part is stored in |\childdocname|.
Note that |\jobname| will be set to the filename of the current part
so that each part receives an individual |.aux| file
that does not interfere with the |.aux| file(s) of the main document.
This behaviour can be altered by the alternative form
|\childdocby[*]{|\textit{main}|}| (with a non-empty optional argument)
which uses the |.aux| file of the main document
by setting |\jobname| to \textit{main}.

%%%%%%%%%%%%%%%%%%%%%%%%%%%%%%%%%%%%%%%%%%%%%%%%%%%%%%%%%%%%%%%%%%%%%%%%%%%%%%%%
\subsection{Driver Development}
\label{sec:driver}

The \textsf{childdoc} mechanism can also be use for the development
of definition files such as \LaTeX{} styles or classes.
This case differs from the above setup with multiple parts
included by |\include| in that no |\includeonly| should be invoked.
This can be achieved by starting the include file
(before |\ProvidesPackage|) with:
%
\begin{center}
\begin{tabular}{l}
|\input{childdoc.def}|\\
|\childdocforward{|\textit{main}|}|\\
\end{tabular}
\end{center}
%
or alternatively with:
%
\begin{center}
\begin{tabular}{l}
|\input{childdoc.def}|\\
|\childdocby{|\textit{main}|}|\\
\end{tabular}
\end{center}
%
Both forms have slightly different effects as described above.
The main file is prepared as usual, see \secref{sec:include}.

%%%%%%%%%%%%%%%%%%%%%%%%%%%%%%%%%%%%%%%%%%%%%%%%%%%%%%%%%%%%%%%%%%%%%%%%%%%%%%%%
\subsection{Legacy Detection}
\label{sec:detection}

The directive |\childdocmain| in the main file can detect
whether the complete document or merely a child is to be compiled
even without using the directive |\childdocof|.
This method is deprecated because it is less robust
and there is no compelling reason to use it;
it is merely provided for backward compatibility
and it may be removed in future versions.

If the detection mechanism is to be used,
it is mandatory to correctly specify
the filename of the main file as the argument of |\childdocmain|:
%
\begin{center}
\begin{tabular}{l}
|\input{childdoc.def}|\\
|\childdocmain{|\textit{main}|}|\\
\end{tabular}
\end{center}
%
If |\jobname| does not match the argument \textit{main} of |\childdocmain|,
it is assumed that |\jobname| points to the child file to be compiled.
When using |\childdocmain| with the main file specified as argument,
it suffices to start a child file
with just |\input{|\textit{main}|}|
without loading of the package and using |\childdocof|.
If instead all processing is done
with the appropriate \textsf{childdoc} directives,
the argument of \textit{main} of |\childdocmain| can be empty.

An alternative version of the command line processing described
in \secref{sec:commandline} using the detection mechanism reads:
%
\begin{center}
|... -jobname "|\textit{target}|" "|[\textit{flags}]%
[|\def\jobname{|\textit{dest}|}|]|\input{|\textit{main}|}"|
\end{center}

%%%%%%%%%%%%%%%%%%%%%%%%%%%%%%%%%%%%%%%%%%%%%%%%%%%%%%%%%%%%%%%%%%%%%%%%%%%%%%%%
\subsection{Manual Code}
\label{sec:manual}

In case one cannot be certain whether the definitions file |childdoc.def|
is installed on the target \TeX{} distribution
and one prefers not to ship it,
it is conceivable to paste a few relevant commands into the sources.

To that end, drop all statements |\input{childdoc.def}|
and perform the replacements as outlined below.
Instead of |\childdocmain{|\textit{main}|}| add the following code
to the top of the main file:
%
\begin{center}
\begin{tabular}{l}
|\||ifdefined\childdocname\endinput\||fi\newif\ifchilddoc|\\
|\edef\childdocname{\scantokens\expandafter{\jobname\noexpand}}|\\
|\def\childdocmain{|\textit{main}|}\||ifx\childdocmain\childdocname\||else|\\
|\childdoctrue\includeonly{\childdocname}\let\jobname\childdocmain\||fi|\\
\end{tabular}
\end{center}
%
Instead of |\childdocof{|\textit{main}|}| just include the main file
at the top of each child file:
%
\begin{center}
|\input{|\textit{main}|}|
\end{center}
%
A simple redirection |\childdocforward{|\textit{dest}|}| is achieved by:
%
\begin{center}
|\def\jobname{|\textit{dest}|}\input{\jobname}|
\end{center}
%
The redirection with prefix
|\childdocforwardprefix[|\textit{prefix}|]{|\textit{dest}|}|
is accomplished by:
%
\begin{center}
\begin{tabular}{l}
|{\edef\jobname{\scantokens\expandafter{\jobname\noexpand}}|\\
|\def\redirectjob |\textit{prefix}|#1~~~{\gdef\jobname{|\textit{dest}|#1}}|\\
|\expandafter\redirectjob\jobname~~~}\input{\jobname}|
\end{tabular}
\end{center}

In an alternative approach,
child documents can be compiled by a specific command line
without additional code or specific definitions:
%
\begin{center}
|... -jobname "|\textit{target}|" "|[\textit{flags}]%
|\includeonly{|\textit{dest}|}\input{|\textit{main}|}"|
\end{center}
%

%%%%%%%%%%%%%%%%%%%%%%%%%%%%%%%%%%%%%%%%%%%%%%%%%%%%%%%%%%%%%%%%%%%%%%%%%%%%%%%%
%%%%%%%%%%%%%%%%%%%%%%%%%%%%%%%%%%%%%%%%%%%%%%%%%%%%%%%%%%%%%%%%%%%%%%%%%%%%%%%%
\section{Information}

%%%%%%%%%%%%%%%%%%%%%%%%%%%%%%%%%%%%%%%%%%%%%%%%%%%%%%%%%%%%%%%%%%%%%%%%%%%%%%%%
\subsection{Copyright}

Copyright \copyright{} 2017--2018 Niklas Beisert

This work may be distributed and/or modified under the
conditions of the \LaTeX{} Project Public License, either version 1.3
of this license or (at your option) any later version.
The latest version of this license is in
  \url{http://www.latex-project.org/lppl.txt}
and version 1.3 or later is part of all distributions of \LaTeX{}
version 2005/12/01 or later.

This work has the LPPL maintenance status `maintained'.

The Current Maintainer of this work is Niklas Beisert.

This work consists of the files |README.txt|, |childdoc.ins| and |childdoc.dtx|
as well as the derived files |childdoc.def|, |cdocsamp.tex|
with |cdocsch1.tex|, |cdocsch2.tex|, |cdocspt3.tex|, |cdocspt4.tex|,
|cdocsdrf.tex|, |cdocsfn1.tex|, |cdocsfn2.tex|
as well as |childdoc.pdf|.

%%%%%%%%%%%%%%%%%%%%%%%%%%%%%%%%%%%%%%%%%%%%%%%%%%%%%%%%%%%%%%%%%%%%%%%%%%%%%%%%
\subsection{Files and Installation}

The package consists of the files:
%
\begin{center}
\begin{tabular}{ll}
    |README.txt|   & readme file \\
    |childdoc.ins| & installation file \\
    |childdoc.dtx| & source file \\
    |childdoc.def| & definition file \\
    |cdocsamp.tex| & sample main file \\
    |cdocsch1.tex| & sample include file \\
    |cdocsch2.tex| & sample include file \\
    |cdocspt3.tex| & sample part file \\
    |cdocspt4.tex| & sample part file \\
    |cdocsdrf.tex| & sample redirection file \\
    |cdocsfn1.tex| & sample redirection file \\
    |cdocsfn2.tex| & sample redirection file \\
    |childdoc.pdf| & manual
\end{tabular}
\end{center}
%
The distribution consists of the files
|README.txt|, |childdoc.ins| and |childdoc.dtx|.
%
\begin{itemize}
\item
Run (pdf)\LaTeX{} on |childdoc.dtx|
to compile the manual |childdoc.pdf| (this file).
\item
Run \LaTeX{} on |childdoc.ins| to create the definitions file |childdoc.def|
and the sample |cdocsamp.tex| with include files
|cdocsch1.tex|, |cdocsch2.tex|, |cdocspt3.tex|, |cdocspt4.tex|,
|cdocsdrf.tex|, |cdocsfn1.tex|, |cdocsfn2.tex|.
Then copy the file |childdoc.def| to an appropriate directory of your \LaTeX{}
distribution, e.g.\ \textit{texmf-root}|/tex/latex/childdoc|.
\end{itemize}

%%%%%%%%%%%%%%%%%%%%%%%%%%%%%%%%%%%%%%%%%%%%%%%%%%%%%%%%%%%%%%%%%%%%%%%%%%%%%%%%
\subsection{Related CTAN Packages}

There are several other packages which offer a similar functionality:
%
\begin{itemize}
\item
The packages
\href{http://ctan.org/pkg/docmute}{\textsf{docmute}},
\href{http://ctan.org/pkg/includex}{\textsf{includex}} and
\href{http://ctan.org/pkg/standalone}{\textsf{standalone}}
provide commands to include only the document body of
a child file thus allowing both files to be compiled individually.
\item
The packages \href{http://ctan.org/pkg/subdocs}{\textsf{subdocs}}
and \href{http://ctan.org/pkg/subfiles}{\textsf{subfiles}}
provide structures in which the main and child documents can be
encapsulated and allowing them to be compiled individually.
The inclusion mechanism is different from the conventional |\include|.
\item
The package \href{http://ctan.org/pkg/combine}{\textsf{combine}}
is an elaborate solution to combine several documents into one.
\end{itemize}
%
See also the CTAN topic \href{http://ctan.org/topic/subdocs}{\textsf{subdocs}}
for further related packages.
The present package differs from the above solutions in that
a document structure constructed with the conventional |\include| mechanism
just needs two extra commands at the top of every file
such that all constituent files can be compiled individually.

%%%%%%%%%%%%%%%%%%%%%%%%%%%%%%%%%%%%%%%%%%%%%%%%%%%%%%%%%%%%%%%%%%%%%%%%%%%%%%%%
%\subsection{Feature Suggestions}
%
%The following is a list of features which may be useful for future
%versions of this package:
%%
%\begin{itemize}
%\item
%\ldots
%\end{itemize}

%%%%%%%%%%%%%%%%%%%%%%%%%%%%%%%%%%%%%%%%%%%%%%%%%%%%%%%%%%%%%%%%%%%%%%%%%%%%%%%%
\subsection{Revision History}

%%%%%%%%%%%%%%%%%%%%%%%%%%%%%%%%%%%%%%%%
\paragraph{v2.0:} 2018/12/30

\begin{itemize}
\item
immediate forward processing
\item
added |\childdocby| mechanism
\item
manual restructured
\end{itemize}

%%%%%%%%%%%%%%%%%%%%%%%%%%%%%%%%%%%%%%%%
\paragraph{v1.6:} 2018/01/17

\begin{itemize}
\item
application for development of include files
\item
corrections to manual
\end{itemize}

%%%%%%%%%%%%%%%%%%%%%%%%%%%%%%%%%%%%%%%%
\paragraph{v1.5:} 2017/05/21

\begin{itemize}
\item
more complete structuring introduced
\item
|\childdocof| introduced
\item
|\childdoc| renamed to |\childdocmain|
\item
|\childredirect| renamed to |\childdocforward| and |\childdocforwardprefix|
and functionality expanded
\end{itemize}

%%%%%%%%%%%%%%%%%%%%%%%%%%%%%%%%%%%%%%%%
\paragraph{v1.0:} 2017/04/27

\begin{itemize}
\item
manual and install package
\item
first version published on CTAN
\end{itemize}

%%%%%%%%%%%%%%%%%%%%%%%%%%%%%%%%%%%%%%%%
\paragraph{v0.6:} 2017/04/26

\begin{itemize}
\item
redirection mechanism added
\end{itemize}

%%%%%%%%%%%%%%%%%%%%%%%%%%%%%%%%%%%%%%%%
\paragraph{v0.5:} 2017/04/26

\begin{itemize}
\item
functionality in definition file
\end{itemize}


%%%%%%%%%%%%%%%%%%%%%%%%%%%%%%%%%%%%%%%%%%%%%%%%%%%%%%%%%%%%%%%%%%%%%%%%%%%%%%%%
%%%%%%%%%%%%%%%%%%%%%%%%%%%%%%%%%%%%%%%%%%%%%%%%%%%%%%%%%%%%%%%%%%%%%%%%%%%%%%%%
%%%%%%%%%%%%%%%%%%%%%%%%%%%%%%%%%%%%%%%%%%%%%%%%%%%%%%%%%%%%%%%%%%%%%%%%%%%%%%%%
\appendix

\settowidth\MacroIndent{\rmfamily\scriptsize 000\ }

 \DocInput{childdoc.dtx}

\end{document}
%</driver>
% \fi
%
% %%%%%%%%%%%%%%%%%%%%%%%%%%%%%%%%%%%%%%%%%%%%%%%%%%%%%%%%%%%%%%%%%%%%%%%%%%%%%%
% %%%%%%%%%%%%%%%%%%%%%%%%%%%%%%%%%%%%%%%%%%%%%%%%%%%%%%%%%%%%%%%%%%%%%%%%%%%%%%
% \section{Sample}
%\iffalse
%<*samplemain>
%\fi
%
% The following presents a sample document
% with two chapters, two parts, a title page,
% a compile flag as well as three forwarding files to set the flag.
% It consists of eight |.tex| files:
% \begin{center}
% \begin{tabular}{ll}
% |cdocsamp.tex|&main file\\
% |cdocsch1.tex|&include file for chapter 1\\
% |cdocsch2.tex|&include file for chapter 2\\
% |cdocspt3.tex|&include file for part 3\\
% |cdocspt4.tex|&include file for part 4\\
% |cdocsdrf.tex|&forwarding file for main file in draft mode\\
% |cdocsfi1.tex|&forwarding file for final version of chapter 1\\
% |cdocsfi2.tex|&forwarding file for final version of chapter 2\\
% \end{tabular}
% \end{center}
% Each of the eight files can be compiled directly by the \LaTeX{} compiler.
%
% %%%%%%%%%%%%%%%%%%%%%%%%%%%%%%%%%%%%%%
% \paragraph{Main File.}
%
% The main file is called |cdocsamp.tex|.
%
% Load the \textsf{childdoc} definitions and
% declare the filename for the main document:
%    \begin{macrocode}
\input{childdoc.def}
\childdocmain{}
%    \end{macrocode}

% Optional override for |\version| flag:
%    \begin{macrocode}
%%\ifchilddoc\else\providecommand{\version}{draft}\fi
%    \end{macrocode}

% Define the default values for the |\version| flag
% (|final| for the main file and |draft| for childs):
%    \begin{macrocode}
\ifchilddoc
\providecommand{\version}{draft}
\else
\providecommand{\version}{final}
\fi
%    \end{macrocode}

% Load the standard document class:
%    \begin{macrocode}
\documentclass[12pt]{article}
%    \end{macrocode}

% Start the document body:
%    \begin{macrocode}
\begin{document}
%    \end{macrocode}

% Declare a title page.
% Print title, part of document being processed and version flag:
%    \begin{macrocode}
\addtocounter{page}{-1}
\begin{center}
{\LARGE\bfseries{}childdoc example\par}
\vspace{1cm}
\ifchilddoc
\ifchilddocmanual part\else chapter\fi:
`\childdocname' of `\childdocjob'\par
\else
main document: `\childdocjob'\par
\fi
version: \version\par
\end{center}
\newpage
%    \end{macrocode}

% Manually include selected file,
% otherwise process as usual:
%    \begin{macrocode}
\ifchilddocmanual
\section*{part `\childdocname'}
\input{\childdocname}
\else
%    \end{macrocode}

% Include the two chapters:
%    \begin{macrocode}
\include{cdocsch1}
\include{cdocsch2}
%    \end{macrocode}

% Include the two parts unless only chapters should be displayed:
%    \begin{macrocode}
\ifchilddoc\else
\section{part three}
\input{cdocspt3}
\section{part four}
\input{cdocspt4}
\fi
%    \end{macrocode}

% Process as usual until here:
%    \begin{macrocode}
\fi
%    \end{macrocode}

% End of document body:
%    \begin{macrocode}
\end{document}
%    \end{macrocode}
%\iffalse
%</samplemain>
%\fi
%
% %%%%%%%%%%%%%%%%%%%%%%%%%%%%%%%%%%%%%%
% \paragraph{Chapter Include Files.}
%
% The include files are called |cdocsch1.tex| and |cdocsch2.tex|.
%
%\iffalse
%<*samplechap1|samplechap2>
%\fi

% Optional override for |\version| flag:
%    \begin{macrocode}
%%\providecommand{\version}{final}
%    \end{macrocode}

% Include the main document:
%    \begin{macrocode}
\input{childdoc.def}
\childdocof{cdocsamp}
%    \end{macrocode}

%\iffalse
%</samplechap1|samplechap2>
%\fi
%
%\iffalse
%<*samplechap1>
%\fi
% Some text for chapter 1:
%    \begin{macrocode}
\section{one}
some text in chapter one
%    \end{macrocode}

%\iffalse
%</samplechap1>
%\fi
% Some text for chapter 2:
%\iffalse
%<*samplechap2>
%\fi
%    \begin{macrocode}
\section{two}
more text in chapter two
%    \end{macrocode}

%\iffalse
%</samplechap2>
%\fi
%
% %%%%%%%%%%%%%%%%%%%%%%%%%%%%%%%%%%%%%%
% \paragraph{Part Include Files.}
%
% The include files are called |cdocspt3.tex| and |cdocspt4.tex|.
%
%\iffalse
%<*samplepart3|samplepart4>
%\fi

% Optional override for |\version| flag:
%    \begin{macrocode}
%%\providecommand{\version}{final}
%    \end{macrocode}

% Include the main document:
%    \begin{macrocode}
\input{childdoc.def}
\childdocby{cdocsamp}
%    \end{macrocode}

%\iffalse
%</samplepart3|samplepart4>
%\fi
%
%\iffalse
%<*samplepart3>
%\fi
% Some text for part 3:
%    \begin{macrocode}
some text in part three
%    \end{macrocode}

%\iffalse
%</samplepart3>
%\fi
% Some text for part 4:
%\iffalse
%<*samplepart4>
%\fi
%    \begin{macrocode}
more text in part four
%    \end{macrocode}

%\iffalse
%</samplepart4>
%\fi
%
% %%%%%%%%%%%%%%%%%%%%%%%%%%%%%%%%%%%%%%
% \paragraph{Forwarding for a Complete Draft.}
%
% The following forwarding file |cdocsdrf.tex|
% compiles the main document in draft mode:
%\iffalse
%<*sampledraft>
%\fi
%    \begin{macrocode}
\def\version{draft}
\input{childdoc.def}
\childdocforward{cdocsamp}
%    \end{macrocode}

%\iffalse
%</sampledraft>
%\fi
%
% %%%%%%%%%%%%%%%%%%%%%%%%%%%%%%%%%%%%%%
% \paragraph{Forwarding for Final Version of the Chapters.}
%
% The following forwarding files |cdocsfn1.tex| and |cdocsfn2.tex|
% (with identical content)
% compile the final versions of the child documents
% |cdocsch1.tex| and |cdocsch2.tex|, respectively:
%\iffalse
%<*samplefinal>
%\fi
%    \begin{macrocode}
\def\version{final}
\input{childdoc.def}
\childdocforwardprefix[cdocsamp]{cdocsfn}{cdocsch}
%    \end{macrocode}

%\iffalse
%</samplefinal>
%\fi
%
% %%%%%%%%%%%%%%%%%%%%%%%%%%%%%%%%%%%%%%
% \paragraph{Command Line Processing.}
%
% The following three command lines generate the output files
% |cdocscld|, |cdocscl1| and |cdocscl2|
% which should be identical to
% |cdocsdrf|, |cdocsch1| and |cdocsfn2|, respectively:
% \begin{center}
% \begin{tabular}{l}
% |latex -jobname cdocscld \|\\
% |  "\def\version{draft}\input{childdoc.def}\childdocforward{cdocsamp}"|\\
% |latex -jobname cdocscl1 \|\\
% |  "\input{childdoc.def}\childdocforward[cdocsamp]{cdocsch1}"|\\
% |latex -jobname cdocscl2 \|\\
% |  "\def\version{final}\input{childdoc.def}\childdocforward{cdocsch2}"|
% \end{tabular}
% \end{center}
% Note that the trailing backslash on each first line
% merely continues the input to the second line
% (for convenient cut ant paste).
% Furthermore, the command |latex| can be replaced by any
% of its alternative versions such as |pdflatex|.
%
% %%%%%%%%%%%%%%%%%%%%%%%%%%%%%%%%%%%%%%%%%%%%%%%%%%%%%%%%%%%%%%%%%%%%%%%%%%%%%%
% %%%%%%%%%%%%%%%%%%%%%%%%%%%%%%%%%%%%%%%%%%%%%%%%%%%%%%%%%%%%%%%%%%%%%%%%%%%%%%
% \section{Implementation}
%\iffalse
%<*package>
%\fi
%
% This section describes the definitions file |childdoc.def|.

% The definitions cannot be loaded using |\usepackage| or |\RequirePackage|
% which has a mechanism to prevent loading a style file more than once.
% When loading the definitions by means of |\input|
% multiple instances have to be prevented manually:
%\iffalse
%This code needs to be before the `\ProvidesFile' directive
%which is defined at the beginning of this file.
%Therefore it is also placed there and commented out here.
%</package>
%<*discard>
%\fi
%    \begin{macrocode}
\ifdefined\childdocmain\endinput\fi
%    \end{macrocode}
%\iffalse
%</discard>
%<*package>
%\fi
%
% \macro{\ifchilddoc}
% \macro{\ifchilddocmanual}
% The conditional |\ifchilddoc| tells whether a
% child (true) or main (false) document is being compiled.
% The conditional |\ifchilddocmanual| tells whether
% the |\includeonly| mechanism is used (false) or
% the selection of child files must be performed manually (true).
% The definitions initialise to false:
%    \begin{macrocode}
\newif\ifchilddoc
\newif\ifchilddocmanual
%    \end{macrocode}

% \macro{\childdocname}
% \macro{\childdocjob}
% The macro |\childdocname| stores the name of the main document
% to be compiled. The macro |\childdocjob| stores the name of
% the document on which the \LaTeX{} compiler was originally invoked.
% The content of |\jobname| cannot be compared
% to filenames specified in the source due to different catcodes.
% The following code rescans |\jobname|, stores the result
% in |\childdocname| and saves a copy in |\childdocjob|:
%    \begin{macrocode}
\edef\childdocname{\scantokens\expandafter{\jobname\noexpand}}
\let\childdocjob\childdocname
%    \end{macrocode}

% \macro{\childdocdisable}
% The macro |\childdocdisable| prevents the main file
% from being processed more than once.
% At this stage, the main document command |\childdocmain|
% is assumed to be called once again where it should do nothing.
% Any subsequent call to it should prevent
% a secondary processing of the main document
% It overwrites the forwarding commands
% |\childdocof| and |\childdocforward|
% with empty macros to prevent further inclusions of the main document:
%    \begin{macrocode}
\newcommand{\childdocdisable}
{
  \renewcommand{\childdocmain}[1]{\renewcommand{\childdocmain}[1]{\endinput}}
  \renewcommand{\childdocof}[1]{}
  \renewcommand{\childdocby}[2][]{}
  \renewcommand{\childdocforward}[2][]{}
  \renewcommand{\childdocdisable}{}
}
%    \end{macrocode}

% \macro{\childdocmain}
% The macro |\childdocmain| is to be called at the top of the main file
% with nothing or the main filename (without extension) as argument.
% First, it breaks loops.
% If the argument is not empty and does not match |\childdocname|
% (which is set by the first inclusion of |childdoc.def|),
% |\ifchilddoc| is set to true, |\includeonly| is applied to the child file
% and |\jobname| is set to the main file
% (for proper handling of |.aux| files):
%    \begin{macrocode}
\newcommand{\childdocmain}[1]
{
  \childdocdisable\childdocmain{}
  \if?#1?\else
    \begingroup
      \def\childdoctmp{#1}
      \ifx\childdoctmp\childdocname
        \def\childdoctmp{}
      \else
        \def\childdoctmp
        {
          \childdoctrue
          \includeonly{\childdocname}
          \def\childdocjob{#1}
          \def\jobname{#1}
        }
      \fi
      \expandafter
    \endgroup
    \childdoctmp
  \fi
}
%    \end{macrocode}

% \macro{\childdocof}
% The command |\childdocof| redirects
% compilation to the main file |#1|.
%    \begin{macrocode}
\newcommand{\childdocof}[1]
{
  \childdocdisable
  \childdoctrue
  \includeonly{\childdocname}
  \def\jobname{#1}
  \def\childdocjob{#1}
  \input{#1}
}
%    \end{macrocode}

% \macro{\childdocby}
% The command |\childdocby| ....
%    \begin{macrocode}
\newcommand{\childdocby}[2][]
{
  \childdocdisable
  \childdoctrue
  \childdocmanualtrue
  \if?#1?\else
    \def\jobname{#2}
  \fi
  \def\childdocjob{#2}
  \input{#2}
  \endinput
}
%    \end{macrocode}

% \macro{\childdocforward}
% The command |\childdocforward| redirects
% compilation to the main file or
% (if the optional argument is given) a child file.
% Parameters are set as if the main file
% or a child file starting with |\childdocof| was compiled.
% Then compilation is handed over to the main file:
%    \begin{macrocode}
\newcommand{\childdocforward}[2][]
{
  \begingroup
    \if?#1?
      \def\childdoctmp
      {
        \def\childdocname{#2}
        \def\childdocjob{#2}
        \def\jobname{#2}
        \input{#2}
        \endinput
      }
    \else
      \def\childdoctmp
      {
        \childdocdisable
        \def\childdocname{#2}
        \childdoctrue
        \includeonly{#2}
        \def\childdocjob{#1}
        \def\jobname{#1}
        \input{#1}
        \endinput
      }
    \fi
    \expandafter
  \endgroup
  \childdoctmp
}
%    \end{macrocode}

% \macro{\childdocforwardprefix}
% The command |\childdocforwardprefix| redirects
% compilation to the main or a child file by means of a pattern.
% The prefix |#1| in the current filename is replaced by |#2|
% and the suffix of the current filename is kept
% (it is assumed that the filename does not contain the substring `|~~~|'
% which is used as a delimiter).
% Compilation is handed over to the new file by |\childdocforward|:
%    \begin{macrocode}
\newcommand{\childdocforwardprefix}[3][]
{
  \begingroup
    \def\childdocextract #2##1~~~{\def\childdoctmp{\childdocforward[#1]{#3##1}}}
    \expandafter\childdocextract\childdocname~~~
    \expandafter
  \endgroup
  \childdoctmp
}
%    \end{macrocode}

% \macro{\childdoc}
% The deprecated macro |\childdoc| is a legacy version of |\childdocmain|:
%    \begin{macrocode}
\newcommand{\childdoc}{\childdocmain}
%    \end{macrocode}

% \macro{\childdocredirect}
% The deprecated macro |\childdocredirect| is a legacy version
% of |\childdocforward| and |\childdocforwardprefix|:
%    \begin{macrocode}
\newcommand{\childdocredirect}[2][]
{
  \begingroup
    \if?#1?
      \def\childdoctmp{\childdocforward{#2}}
    \else
      \def\childdoctmp{\childdocforwardprefix{#1}{#2}}
    \fi
    \expandafter
  \endgroup
  \childdoctmp
}
%    \end{macrocode}

%\iffalse
%</package>
%\fi
%
\endinput
\childdocforward[|\textit{main}|]{|\textit{dest}|}"|
\end{center}
%
Here \textit{target} is the name of the output file,
\textit{main} is the name of the main file
and \textit{dest} is the name of the main or child file to be processed
(all filenames without extensions).
The optional argument \textit{main} can be omitted
if \textit{main} matches \textit{dest}.
Optionally, compilation \textit{flags} can be defined via |\def| commands.
This command line makes the \TeX{} engine believe
it is compiling the file \textit{target}
whose content is specified as the latter parameter.
The provided code then forwards the processing to
\textit{main} or \textit{dest} as described in \secref{sec:forward}.

%%%%%%%%%%%%%%%%%%%%%%%%%%%%%%%%%%%%%%%%%%%%%%%%%%%%%%%%%%%%%%%%%%%%%%%%%%%%%%%%
\subsection{Include by Input}
\label{sec:input}

Including child documents by |\include| has some restrictions by design.
Most notably, the content of a child document always occupies
its own set of pages; pages cannot be shared between child documents.
Usually, this behaviour makes perfect sense
because each child document contain an essential part of the document.
However, in some situations it may be desirable to compose
a document from a collection of parts
without having mandatory page breaks between then.
For this case, the package
provides a mechanism to include parts
by |\input| which can also be processed individually.
However, by construction this mechanism
requires manual handling of the content to be output.

%%%%%%%%%%%%%%%%%%%%%%%%%%%%%%%%%%%%%%%%
\DescribeMacro{\ifchilddocmanual}
The main file should be prepared as usual, see \secref{sec:include}.
However, the document body must make a distinction
between processing of an individual part and of the main document, e.g.:
%
\begin{center}
\begin{tabular}{l}
|\ifchilddocmanual|\\
|\input{\childdocname}|\\
|\||else|\\
\textit{document body with }|\input{|\textit{part}|}|\\
|\||fi|
\end{tabular}
\end{center}
%
The conditional |\ifchilddocmanual| is true whenever
a part to be included by |\input| is being compiled,
and the name of the part is stored in |\childdocname|.

%%%%%%%%%%%%%%%%%%%%%%%%%%%%%%%%%%%%%%%%
\DescribeMacro{\childdocby}
Each part to be included by |\input| should start with:
%
\begin{center}
\begin{tabular}{l}
|% \iffalse
%
% childdoc.dtx Copyright (C) 2017-2018 Niklas Beisert
%
% This work may be distributed and/or modified under the
% conditions of the LaTeX Project Public License, either version 1.3
% of this license or (at your option) any later version.
% The latest version of this license is in
%   http://www.latex-project.org/lppl.txt
% and version 1.3 or later is part of all distributions of LaTeX
% version 2005/12/01 or later.
%
% This work has the LPPL maintenance status `maintained'.
%
% The Current Maintainer of this work is Niklas Beisert.
%
% This work consists of the files childdoc.dtx and childdoc.ins
% and the derived files childdoc.def and cdocsamp.tex with
% cdocsch1.tex, cdocsch2.tex, cdocsdrf.tex, cdocsfn1.tex, cdocsfn2.tex.
%
%<package>\ifdefined\childdocmain\endinput\fi
%<package>\ProvidesFile{childdoc.def}[2018/12/30 v2.0 child document driver]
%<samplemain>\ProvidesFile{cdocsamp.tex}[2018/12/30 v2.0 sample for childdoc]
%<*driver>
%\ProvidesFile{childdoc.drv}[2018/12/30 v2.0 childdoc reference manual file]
\PassOptionsToClass{10pt,a4paper}{article}
\documentclass{ltxdoc}

\usepackage[margin=35mm]{geometry}
\usepackage{hyperref}
\usepackage{hyperxmp}
\usepackage[usenames]{color}

\hypersetup{colorlinks=true}
\hypersetup{pdfstartview=FitH}
\hypersetup{pdfpagemode=UseNone}
\hypersetup{pdfsource={}}
\hypersetup{pdflang={en-UK}}
\hypersetup{pdfcopyright={Copyright 2017-2018 Niklas Beisert.
  This work may be distributed and/or modified under the
  conditions of the LaTeX Project Public License, either version 1.3
  of this license or (at your option) any later version.}}
\hypersetup{pdflicenseurl={http://www.latex-project.org/lppl.txt}}
\hypersetup{pdfcontactaddress={ETH Zurich, ITP, HIT K,
  Wolfgang-Pauli-Strasse 27}}
\hypersetup{pdfcontactpostcode={8093}}
\hypersetup{pdfcontactcity={Zurich}}
\hypersetup{pdfcontactcountry={Switzerland}}
\hypersetup{pdfcontactemail={nbeisert@itp.phys.ethz.ch}}
\hypersetup{pdfcontacturl={http://people.phys.ethz.ch/\xmptilde nbeisert/}}

\newcommand{\secref}[1]{\hyperref[#1]{section \ref*{#1}}}

\parskip1ex
\parindent0pt
\let\olditemize\itemize
\def\itemize{\olditemize\parskip0pt}

\begin{document}

\title{The \textsf{childdoc} Package}
\hypersetup{pdftitle={The childdoc Package}}
\author{Niklas Beisert\\[2ex]
  Institut f\"ur Theoretische Physik\\
  Eidgen\"ossische Technische Hochschule Z\"urich\\
  Wolfgang-Pauli-Strasse 27, 8093 Z\"urich, Switzerland\\[1ex]
  \href{mailto:nbeisert@itp.phys.ethz.ch}
  {\texttt{nbeisert@itp.phys.ethz.ch}}}
\hypersetup{pdfauthor={Niklas Beisert}}
\hypersetup{pdfsubject={Manual for the LaTeX2e Package childdoc}}
\date{30 December 2018, \textsf{v2.0}}
\maketitle

\begin{abstract}\noindent
\textsf{childdoc} is a \LaTeXe{} package
that enables the direct compilation
of document sections included by |\include|
to individual files.
\end{abstract}

\begingroup
\parskip0ex
\tableofcontents
\endgroup

%%%%%%%%%%%%%%%%%%%%%%%%%%%%%%%%%%%%%%%%%%%%%%%%%%%%%%%%%%%%%%%%%%%%%%%%%%%%%%%%
%%%%%%%%%%%%%%%%%%%%%%%%%%%%%%%%%%%%%%%%%%%%%%%%%%%%%%%%%%%%%%%%%%%%%%%%%%%%%%%%
\section{Introduction}

\LaTeX{} provides a mechanism to structure a large document (such as a book)
into a main file and several child files (containing the chapters)
using the |\include| command.
This mechanism is beneficial for documents
which span hundreds of pages in order to
make the source file(s) more manageable.
Moreover, compilation can be restricted to
selected child files by means of the |\includeonly| command.
The latter feature can be used to reduce the compilation time while editing
(this was significantly more useful in the earlier days of \LaTeX{})
or to generate a smaller document which is easier to navigate.
Another application of |\includeonly| is to generate
documents consisting of selected parts of the complete document.

However, there are a few drawbacks of the plain |\include| mechanism:
\begin{itemize}
\item
The child files cannot be compiled on their own,
they can only be compiled via the main file.
A naive editing environment
(such as a text editor with an option
to have the current file processed by \LaTeX)
may require one to switch to the main file before compiling;
attempting to compile the child file produces errors.
\item
The main file must be modified (each time)
to adjust the |\includeonly| command
to the present needs. This easily leaves the main file in a messy state.
\item
The generated document will always carry the filename
of the main document. This is inconvenient if
several child files are to be compiled and
to be kept for distribution.
\end{itemize}

The present package provides a simple interface
to make child files individually compilable by \LaTeX{}.
Compiling a child file then has the same effect as compiling
the main file with an |\includeonly| command
to select the appropriate child.
Moreover the generated document will carry the name of the child
rather than the main file.
This resolves all three above issues.

This feature is meant to make the editing of books,
thesis documents and lecture notes somewhat more convenient.
However, the package can also be used efficiently for
composing a series of documents (such as exercise sheets)
which are typically distributed individually.
It then assists the author in generating the individual documents
(potentially in different versions)
as well as a document containing the collected series.
Another application is in developing style files
or other kinds of included material
where compilation of the style file could redirect
to a sample or test file.

%%%%%%%%%%%%%%%%%%%%%%%%%%%%%%%%%%%%%%%%%%%%%%%%%%%%%%%%%%%%%%%%%%%%%%%%%%%%%%%%
%%%%%%%%%%%%%%%%%%%%%%%%%%%%%%%%%%%%%%%%%%%%%%%%%%%%%%%%%%%%%%%%%%%%%%%%%%%%%%%%
\section{Usage}

First of all, the package \textsf{childdoc} is \emph{not} a standard
\LaTeXe{} |.sty| style file! Therefore it needs to be invoked in
a non-standard way.

%%%%%%%%%%%%%%%%%%%%%%%%%%%%%%%%%%%%%%%%%%%%%%%%%%%%%%%%%%%%%%%%%%%%%%%%%%%%%%%%
\subsection{Included Files}
\label{sec:include}

%%%%%%%%%%%%%%%%%%%%%%%%%%%%%%%%%%%%%%%%
\DescribeMacro{\childdocmain}
To use the package, add the commands
\begin{center}
\begin{tabular}{l}
|\input{childdoc.def}|\\
|\childdocmain{}|\\
\end{tabular}
\end{center}
at the very top of the main \LaTeX{} file,
in particular \emph{before} the |\documentclass| statement!
The argument of |\childdocmain| should be left empty
(but it must be present).

%%%%%%%%%%%%%%%%%%%%%%%%%%%%%%%%%%%%%%%%
\DescribeMacro{\childdocof}
Furthermore, add the commands
\begin{center}
\begin{tabular}{l}
|\input{childdoc.def}|\\
|\childdocof{|\textit{main}|}|\\
\end{tabular}
\end{center}
at the top of every child file \textit{child}
which is included by |\include{|\textit{child}|}|
from within the main file
(or at least for those files to be compiled individually).
The argument \textit{main} must be the filename of the main file.

There are a couple of
considerations in setting up the main and child documents:

%%%%%%%%%%%%%%%%%%%%%%%%%%%%%%%%%%%%%%%%
\paragraph{Restrictions.}

Please note the following restrictions:
\begin{itemize}
\item
|\childdocmain| must be called with one argument \textit{main}
to ensure compatibility with earlier version of the package.
It must either be empty (|\childdocmain{}|)
or precisely match the filename of the main file in which it is specified.
See \secref{sec:detection} for further information.
\item
The filename \textit{main} must be specified without the |.tex| extension.
\item
The filename \textit{main} is case sensitive
(even in case-insensitive file systems)
due to internal string comparison.
\item
The argument \textit{main} should be fully expanded, it cannot be a macro.
\item
Subdirectories and special characters should be avoided in filenames.
\item
The command |\childdocmain{|\textit{main}|}| must be followed by a whitespace.
It should not be followed immediately by another command
or by a comment mark `|%|'.
This is because the \TeX{} parser reads the token immediately following
the argument of |\childdocmain| and puts it
at the beginning of every child section;
however, a white\-space is ignored.
\end{itemize}

%%%%%%%%%%%%%%%%%%%%%%%%%%%%%%%%%%%%%%%%
\paragraph{Content of Main File.}

It is advisable to place all content in the child files included by |\include|.
Any output contained in the main file will appear in all child documents
unless suppressed manually;
it cannot be suppressed automatically by the |\includeonly| directive
and thus should normally be avoided.
A method to include some content in the main file
by means of conditional processing is described in \secref{sec:conditional}.

%%%%%%%%%%%%%%%%%%%%%%%%%%%%%%%%%%%%%%%%
\paragraph{Page Numbering.}

When only a part of the document is compiled,
the appropriate numbering of pages
(as well as other status parameters)
is determined from the |.aux| files.
The latter contain information from previous passes.
However this information needs to propagate through
all intermediate child documents.
Therefore the page numbering in child documents may well
be inconsistent until the complete document is compiled at least once.

A useful (if unconventional) way to always ensure a consistent
page numbering is to restart the numbering in each child document
and denote the pages by `\textit{child}|.|\textit{page}'
where \textit{child} represents the chapter/section number of the child file.
This can be achieved by the command
|\numberwithin{page}{|\textit{child}|}|
of the \textsf{amsmath} package
where \textit{child} can be |chapter| or |section|
depending on the chosen structuring.
Alternatively, one can modify the macro |\thepage| appropriately
and reset the counter |page| at the start of each child file.

%%%%%%%%%%%%%%%%%%%%%%%%%%%%%%%%%%%%%%%%%%%%%%%%%%%%%%%%%%%%%%%%%%%%%%%%%%%%%%%%
\subsection{Conditional Processing}
\label{sec:conditional}

The package provides a mechanism to compile different versions
of a document. To customise the versions further some conditional processing
can come in handy to distinguish which version is being compiled.
The package provides two macros to describe the compilation context:

%%%%%%%%%%%%%%%%%%%%%%%%%%%%%%%%%%%%%%%%
\DescribeMacro{\ifchilddoc}
The conditional |\ifchilddoc| distinguishes between the compilation of
child documents and the main document:
%
\begin{center}
|\ifchilddoc |\textit{child-code}| |[|\||else |\textit{main-code}]| \||fi|
\end{center}

%%%%%%%%%%%%%%%%%%%%%%%%%%%%%%%%%%%%%%%%
\DescribeMacro{\childdocname}
\DescribeMacro{\childdocjob}
The macro |\childdocname| contains the filename (without extension)
of the main or child file being processed.
Note that |\childdocjob| will always contain the name of the main file.

%%%%%%%%%%%%%%%%%%%%%%%%%%%%%%%%%%%%%%%%
\paragraph{Title Page.}

Conditional processing can be used to include a title or banner page
in the main document when proper precautions are taken.
Importantly, the code in the main file should ensure that the page counter
(as well as other status parameters which are stored in the |.aux| files)
takes the same value after the conditional processing.
Otherwise the page numbers may take divergent values
depending on which part is compiled.

For example, a title page could be declared by:
%
\begin{center}
\begin{tabular}{l}
|\ifchilddoc\||else|\\
|\addtocounter{page}{-1}|\\
\textit{code for title page}\\
|\newpage|\\
|\||fi|
\end{tabular}
\end{center}
%
A banner page for the child documents can be generated by:
%
\begin{center}
\begin{tabular}{l}
|\ifchilddoc|\\
|\addtocounter{page}{-1}|\\
\textit{code for banner page}\\
|\newpage|\\
|\||fi|
\end{tabular}
\end{center}
%
Here one could write a message such as:
\begin{center}
|This is the part \childdocname{} of \childdocjob{}.|
\end{center}

%%%%%%%%%%%%%%%%%%%%%%%%%%%%%%%%%%%%%%%%%%%%%%%%%%%%%%%%%%%%%%%%%%%%%%%%%%%%%%%%
\subsection{Flags}
\label{sec:flags}

The package makes it easy to generate different versions
of the main or child documents.
To this end compilation flags can be defined
and assigned different default values.
They will be particularly useful in conjunction
with the forwarding mechanism described in \secref{sec:forward}.

For example, it may be useful to have a flag |\version|
which can be set to |draft| or |final|.
The document source will contain some conditional code
depending on the value of |\version|.
Suppose further, the flag should default to |final| for the main file
and to |draft| for child files
which is a natural assignment for editing the document.
This is achieved by placing the following code
in the preamble of the main document
(below the |\childdocmain| directive):
%
\begin{center}
\begin{tabular}{l}
|\ifchilddoc|\\
|\providecommand{\version}{draft}|\\
|\||else|\\
|\providecommand{\version}{final}|\\
|\||fi|
\end{tabular}
\end{center}
%
The definition by |\providecommand| makes sure
that previous definitions are not overwritten.
Further statements |\providecommand{\version}{...}|
can thus be added before the above code to override it.

For the main file, one might add a line
(between |\childdocmain| and the above block)
%
\begin{center}
|%\ifchilddoc\||else\providecommand{\version}{draft}\||fi|
\end{center}
%
which can be uncommented to produce a draft version.
Likewise one can add a line to the very top of a child file
(above the |\childdocof{|\textit{main}|}| directive)
%
\begin{center}
|%\providecommand{\version}{final}|
\end{center}
%
which can be uncommented to produce the final version of this child document.

%%%%%%%%%%%%%%%%%%%%%%%%%%%%%%%%%%%%%%%%%%%%%%%%%%%%%%%%%%%%%%%%%%%%%%%%%%%%%%%%
\subsection{Forwarding}
\label{sec:forward}

Different versions of the main or child documents
using compilation flags as described in \secref{sec:flags}
can be (permanently) stored in different files
for convenient compilation, viewing and distribution.
To this end, the package defines a command
to pass on compilation to a different file:

%%%%%%%%%%%%%%%%%%%%%%%%%%%%%%%%%%%%%%%%
\DescribeMacro{\childdocforward}
The command |\childdocforward| redirects processing to
another source file:
%
\begin{center}
\begin{tabular}{l}
|\input{childdoc.def}|\\
|\childdocforward[|\textit{main}|]{|\textit{dest}|}|\\
\end{tabular}
\end{center}
%
The argument \textit{dest} is the destination file
(without extension).
It should be the main file or one of the child files.
Note that further \textsf{childdoc} directives
such as |\childdocof| and |\childdocforward|
in the indicated file will be processed in this form.
The optional argument \textit{main}
passes on directly to the main file \textit{main}
while pretending to compile the child \textit{dest}.
This form behaves as if \textit{dest}
issues |\childdocof{|\textit{main}|}| right away,
and no further \textsf{childdoc} directives will be processed.

%%%%%%%%%%%%%%%%%%%%%%%%%%%%%%%%%%%%%%%%
\DescribeMacro{\...prefix}
In the alternative form |\childdocforwardprefix|,
%
\begin{center}
\begin{tabular}{l}
|\input{childdoc.def}|\\
|\childdocforwardprefix[|\textit{main}|]{|\textit{prefix}|}{|\textit{dest}|}|
\end{tabular}
\end{center}
%
the destination file is determined by a pattern
depending on the current file:
To make this work, the current file must be called
`{\textit{prefix}\hspace{0.2em}\textit{suffix}}'
with \textit{prefix} matching precisely the argument.
Processing is then passed on to the file
`{\textit{dest}\hspace{0.2em}\textit{suffix}}'.
Surely, the same effect is achieved by
directly specifying the
argument `{\textit{dest}\hspace{0.2em}\textit{suffix}}'
in the first form.
However, that requires to set up a different file
for each child. With the alternative form of the command
all these files can have exactly the same content
which simplifies setting them up and maintaining them.

For example, the following file |draft.tex|
with a compilation flag |\version| as described in \secref{sec:flags}
compiles the main document as a draft:
%
\begin{center}
\begin{tabular}{l}
|\def\version{draft}|\\
|\input{childdoc.def}|\\
|\childdocforward{|\textit{main}|}|
\end{tabular}
\end{center}
%
Likewise, the following files |final|\textit{nn}|.tex|
compile the final version of the child document
|child|\textit{nn}|.tex|:
%
\begin{center}
\begin{tabular}{l}
|\def\version{final}|\\
|\input{childdoc.def}|\\
|\childdocforwardprefix{final}{child}|
\end{tabular}
\end{center}
%

Note that when several versions of a main file and/or of each child file
are to be generated, it may be convenient to set up a |Makefile| or
shell script to automatise the process.

%%%%%%%%%%%%%%%%%%%%%%%%%%%%%%%%%%%%%%%%%%%%%%%%%%%%%%%%%%%%%%%%%%%%%%%%%%%%%%%%
\subsection{Command Line Processing}
\label{sec:commandline}

The effect of redirection files can also be achieved by invoking
the \LaTeX{} compiler with a more elaborate command line.
Most conveniently this should be done as part
of a shell script or a |Makefile|.

When using \textsf{childdoc} in the main file, the following
command lines effectively perform a redirection
(note that depending on the shell being used,
backslashes may have to be doubled: `|\|' $\to$ `|\\|'):
%
\begin{center}
|... -jobname "|\textit{target}|" |\\|"|[\textit{flags}]%
|\input{childdoc.def}\childdocforward[|\textit{main}|]{|\textit{dest}|}"|
\end{center}
%
Here \textit{target} is the name of the output file,
\textit{main} is the name of the main file
and \textit{dest} is the name of the main or child file to be processed
(all filenames without extensions).
The optional argument \textit{main} can be omitted
if \textit{main} matches \textit{dest}.
Optionally, compilation \textit{flags} can be defined via |\def| commands.
This command line makes the \TeX{} engine believe
it is compiling the file \textit{target}
whose content is specified as the latter parameter.
The provided code then forwards the processing to
\textit{main} or \textit{dest} as described in \secref{sec:forward}.

%%%%%%%%%%%%%%%%%%%%%%%%%%%%%%%%%%%%%%%%%%%%%%%%%%%%%%%%%%%%%%%%%%%%%%%%%%%%%%%%
\subsection{Include by Input}
\label{sec:input}

Including child documents by |\include| has some restrictions by design.
Most notably, the content of a child document always occupies
its own set of pages; pages cannot be shared between child documents.
Usually, this behaviour makes perfect sense
because each child document contain an essential part of the document.
However, in some situations it may be desirable to compose
a document from a collection of parts
without having mandatory page breaks between then.
For this case, the package
provides a mechanism to include parts
by |\input| which can also be processed individually.
However, by construction this mechanism
requires manual handling of the content to be output.

%%%%%%%%%%%%%%%%%%%%%%%%%%%%%%%%%%%%%%%%
\DescribeMacro{\ifchilddocmanual}
The main file should be prepared as usual, see \secref{sec:include}.
However, the document body must make a distinction
between processing of an individual part and of the main document, e.g.:
%
\begin{center}
\begin{tabular}{l}
|\ifchilddocmanual|\\
|\input{\childdocname}|\\
|\||else|\\
\textit{document body with }|\input{|\textit{part}|}|\\
|\||fi|
\end{tabular}
\end{center}
%
The conditional |\ifchilddocmanual| is true whenever
a part to be included by |\input| is being compiled,
and the name of the part is stored in |\childdocname|.

%%%%%%%%%%%%%%%%%%%%%%%%%%%%%%%%%%%%%%%%
\DescribeMacro{\childdocby}
Each part to be included by |\input| should start with:
%
\begin{center}
\begin{tabular}{l}
|\input{childdoc.def}|\\
|\childdocby{|\textit{main}|}|\\
\end{tabular}
\end{center}
%
The directive |\childdocby| is similar to |\childdocof|
described in \secref{sec:include},
but the subsequent selection of content must be done manually.
To that end, both |\ifchilddoc| and |\ifchilddocmanual|
will be true upon processing of a part,
and the name of the part is stored in |\childdocname|.
Note that |\jobname| will be set to the filename of the current part
so that each part receives an individual |.aux| file
that does not interfere with the |.aux| file(s) of the main document.
This behaviour can be altered by the alternative form
|\childdocby[*]{|\textit{main}|}| (with a non-empty optional argument)
which uses the |.aux| file of the main document
by setting |\jobname| to \textit{main}.

%%%%%%%%%%%%%%%%%%%%%%%%%%%%%%%%%%%%%%%%%%%%%%%%%%%%%%%%%%%%%%%%%%%%%%%%%%%%%%%%
\subsection{Driver Development}
\label{sec:driver}

The \textsf{childdoc} mechanism can also be use for the development
of definition files such as \LaTeX{} styles or classes.
This case differs from the above setup with multiple parts
included by |\include| in that no |\includeonly| should be invoked.
This can be achieved by starting the include file
(before |\ProvidesPackage|) with:
%
\begin{center}
\begin{tabular}{l}
|\input{childdoc.def}|\\
|\childdocforward{|\textit{main}|}|\\
\end{tabular}
\end{center}
%
or alternatively with:
%
\begin{center}
\begin{tabular}{l}
|\input{childdoc.def}|\\
|\childdocby{|\textit{main}|}|\\
\end{tabular}
\end{center}
%
Both forms have slightly different effects as described above.
The main file is prepared as usual, see \secref{sec:include}.

%%%%%%%%%%%%%%%%%%%%%%%%%%%%%%%%%%%%%%%%%%%%%%%%%%%%%%%%%%%%%%%%%%%%%%%%%%%%%%%%
\subsection{Legacy Detection}
\label{sec:detection}

The directive |\childdocmain| in the main file can detect
whether the complete document or merely a child is to be compiled
even without using the directive |\childdocof|.
This method is deprecated because it is less robust
and there is no compelling reason to use it;
it is merely provided for backward compatibility
and it may be removed in future versions.

If the detection mechanism is to be used,
it is mandatory to correctly specify
the filename of the main file as the argument of |\childdocmain|:
%
\begin{center}
\begin{tabular}{l}
|\input{childdoc.def}|\\
|\childdocmain{|\textit{main}|}|\\
\end{tabular}
\end{center}
%
If |\jobname| does not match the argument \textit{main} of |\childdocmain|,
it is assumed that |\jobname| points to the child file to be compiled.
When using |\childdocmain| with the main file specified as argument,
it suffices to start a child file
with just |\input{|\textit{main}|}|
without loading of the package and using |\childdocof|.
If instead all processing is done
with the appropriate \textsf{childdoc} directives,
the argument of \textit{main} of |\childdocmain| can be empty.

An alternative version of the command line processing described
in \secref{sec:commandline} using the detection mechanism reads:
%
\begin{center}
|... -jobname "|\textit{target}|" "|[\textit{flags}]%
[|\def\jobname{|\textit{dest}|}|]|\input{|\textit{main}|}"|
\end{center}

%%%%%%%%%%%%%%%%%%%%%%%%%%%%%%%%%%%%%%%%%%%%%%%%%%%%%%%%%%%%%%%%%%%%%%%%%%%%%%%%
\subsection{Manual Code}
\label{sec:manual}

In case one cannot be certain whether the definitions file |childdoc.def|
is installed on the target \TeX{} distribution
and one prefers not to ship it,
it is conceivable to paste a few relevant commands into the sources.

To that end, drop all statements |\input{childdoc.def}|
and perform the replacements as outlined below.
Instead of |\childdocmain{|\textit{main}|}| add the following code
to the top of the main file:
%
\begin{center}
\begin{tabular}{l}
|\||ifdefined\childdocname\endinput\||fi\newif\ifchilddoc|\\
|\edef\childdocname{\scantokens\expandafter{\jobname\noexpand}}|\\
|\def\childdocmain{|\textit{main}|}\||ifx\childdocmain\childdocname\||else|\\
|\childdoctrue\includeonly{\childdocname}\let\jobname\childdocmain\||fi|\\
\end{tabular}
\end{center}
%
Instead of |\childdocof{|\textit{main}|}| just include the main file
at the top of each child file:
%
\begin{center}
|\input{|\textit{main}|}|
\end{center}
%
A simple redirection |\childdocforward{|\textit{dest}|}| is achieved by:
%
\begin{center}
|\def\jobname{|\textit{dest}|}\input{\jobname}|
\end{center}
%
The redirection with prefix
|\childdocforwardprefix[|\textit{prefix}|]{|\textit{dest}|}|
is accomplished by:
%
\begin{center}
\begin{tabular}{l}
|{\edef\jobname{\scantokens\expandafter{\jobname\noexpand}}|\\
|\def\redirectjob |\textit{prefix}|#1~~~{\gdef\jobname{|\textit{dest}|#1}}|\\
|\expandafter\redirectjob\jobname~~~}\input{\jobname}|
\end{tabular}
\end{center}

In an alternative approach,
child documents can be compiled by a specific command line
without additional code or specific definitions:
%
\begin{center}
|... -jobname "|\textit{target}|" "|[\textit{flags}]%
|\includeonly{|\textit{dest}|}\input{|\textit{main}|}"|
\end{center}
%

%%%%%%%%%%%%%%%%%%%%%%%%%%%%%%%%%%%%%%%%%%%%%%%%%%%%%%%%%%%%%%%%%%%%%%%%%%%%%%%%
%%%%%%%%%%%%%%%%%%%%%%%%%%%%%%%%%%%%%%%%%%%%%%%%%%%%%%%%%%%%%%%%%%%%%%%%%%%%%%%%
\section{Information}

%%%%%%%%%%%%%%%%%%%%%%%%%%%%%%%%%%%%%%%%%%%%%%%%%%%%%%%%%%%%%%%%%%%%%%%%%%%%%%%%
\subsection{Copyright}

Copyright \copyright{} 2017--2018 Niklas Beisert

This work may be distributed and/or modified under the
conditions of the \LaTeX{} Project Public License, either version 1.3
of this license or (at your option) any later version.
The latest version of this license is in
  \url{http://www.latex-project.org/lppl.txt}
and version 1.3 or later is part of all distributions of \LaTeX{}
version 2005/12/01 or later.

This work has the LPPL maintenance status `maintained'.

The Current Maintainer of this work is Niklas Beisert.

This work consists of the files |README.txt|, |childdoc.ins| and |childdoc.dtx|
as well as the derived files |childdoc.def|, |cdocsamp.tex|
with |cdocsch1.tex|, |cdocsch2.tex|, |cdocspt3.tex|, |cdocspt4.tex|,
|cdocsdrf.tex|, |cdocsfn1.tex|, |cdocsfn2.tex|
as well as |childdoc.pdf|.

%%%%%%%%%%%%%%%%%%%%%%%%%%%%%%%%%%%%%%%%%%%%%%%%%%%%%%%%%%%%%%%%%%%%%%%%%%%%%%%%
\subsection{Files and Installation}

The package consists of the files:
%
\begin{center}
\begin{tabular}{ll}
    |README.txt|   & readme file \\
    |childdoc.ins| & installation file \\
    |childdoc.dtx| & source file \\
    |childdoc.def| & definition file \\
    |cdocsamp.tex| & sample main file \\
    |cdocsch1.tex| & sample include file \\
    |cdocsch2.tex| & sample include file \\
    |cdocspt3.tex| & sample part file \\
    |cdocspt4.tex| & sample part file \\
    |cdocsdrf.tex| & sample redirection file \\
    |cdocsfn1.tex| & sample redirection file \\
    |cdocsfn2.tex| & sample redirection file \\
    |childdoc.pdf| & manual
\end{tabular}
\end{center}
%
The distribution consists of the files
|README.txt|, |childdoc.ins| and |childdoc.dtx|.
%
\begin{itemize}
\item
Run (pdf)\LaTeX{} on |childdoc.dtx|
to compile the manual |childdoc.pdf| (this file).
\item
Run \LaTeX{} on |childdoc.ins| to create the definitions file |childdoc.def|
and the sample |cdocsamp.tex| with include files
|cdocsch1.tex|, |cdocsch2.tex|, |cdocspt3.tex|, |cdocspt4.tex|,
|cdocsdrf.tex|, |cdocsfn1.tex|, |cdocsfn2.tex|.
Then copy the file |childdoc.def| to an appropriate directory of your \LaTeX{}
distribution, e.g.\ \textit{texmf-root}|/tex/latex/childdoc|.
\end{itemize}

%%%%%%%%%%%%%%%%%%%%%%%%%%%%%%%%%%%%%%%%%%%%%%%%%%%%%%%%%%%%%%%%%%%%%%%%%%%%%%%%
\subsection{Related CTAN Packages}

There are several other packages which offer a similar functionality:
%
\begin{itemize}
\item
The packages
\href{http://ctan.org/pkg/docmute}{\textsf{docmute}},
\href{http://ctan.org/pkg/includex}{\textsf{includex}} and
\href{http://ctan.org/pkg/standalone}{\textsf{standalone}}
provide commands to include only the document body of
a child file thus allowing both files to be compiled individually.
\item
The packages \href{http://ctan.org/pkg/subdocs}{\textsf{subdocs}}
and \href{http://ctan.org/pkg/subfiles}{\textsf{subfiles}}
provide structures in which the main and child documents can be
encapsulated and allowing them to be compiled individually.
The inclusion mechanism is different from the conventional |\include|.
\item
The package \href{http://ctan.org/pkg/combine}{\textsf{combine}}
is an elaborate solution to combine several documents into one.
\end{itemize}
%
See also the CTAN topic \href{http://ctan.org/topic/subdocs}{\textsf{subdocs}}
for further related packages.
The present package differs from the above solutions in that
a document structure constructed with the conventional |\include| mechanism
just needs two extra commands at the top of every file
such that all constituent files can be compiled individually.

%%%%%%%%%%%%%%%%%%%%%%%%%%%%%%%%%%%%%%%%%%%%%%%%%%%%%%%%%%%%%%%%%%%%%%%%%%%%%%%%
%\subsection{Feature Suggestions}
%
%The following is a list of features which may be useful for future
%versions of this package:
%%
%\begin{itemize}
%\item
%\ldots
%\end{itemize}

%%%%%%%%%%%%%%%%%%%%%%%%%%%%%%%%%%%%%%%%%%%%%%%%%%%%%%%%%%%%%%%%%%%%%%%%%%%%%%%%
\subsection{Revision History}

%%%%%%%%%%%%%%%%%%%%%%%%%%%%%%%%%%%%%%%%
\paragraph{v2.0:} 2018/12/30

\begin{itemize}
\item
immediate forward processing
\item
added |\childdocby| mechanism
\item
manual restructured
\end{itemize}

%%%%%%%%%%%%%%%%%%%%%%%%%%%%%%%%%%%%%%%%
\paragraph{v1.6:} 2018/01/17

\begin{itemize}
\item
application for development of include files
\item
corrections to manual
\end{itemize}

%%%%%%%%%%%%%%%%%%%%%%%%%%%%%%%%%%%%%%%%
\paragraph{v1.5:} 2017/05/21

\begin{itemize}
\item
more complete structuring introduced
\item
|\childdocof| introduced
\item
|\childdoc| renamed to |\childdocmain|
\item
|\childredirect| renamed to |\childdocforward| and |\childdocforwardprefix|
and functionality expanded
\end{itemize}

%%%%%%%%%%%%%%%%%%%%%%%%%%%%%%%%%%%%%%%%
\paragraph{v1.0:} 2017/04/27

\begin{itemize}
\item
manual and install package
\item
first version published on CTAN
\end{itemize}

%%%%%%%%%%%%%%%%%%%%%%%%%%%%%%%%%%%%%%%%
\paragraph{v0.6:} 2017/04/26

\begin{itemize}
\item
redirection mechanism added
\end{itemize}

%%%%%%%%%%%%%%%%%%%%%%%%%%%%%%%%%%%%%%%%
\paragraph{v0.5:} 2017/04/26

\begin{itemize}
\item
functionality in definition file
\end{itemize}


%%%%%%%%%%%%%%%%%%%%%%%%%%%%%%%%%%%%%%%%%%%%%%%%%%%%%%%%%%%%%%%%%%%%%%%%%%%%%%%%
%%%%%%%%%%%%%%%%%%%%%%%%%%%%%%%%%%%%%%%%%%%%%%%%%%%%%%%%%%%%%%%%%%%%%%%%%%%%%%%%
%%%%%%%%%%%%%%%%%%%%%%%%%%%%%%%%%%%%%%%%%%%%%%%%%%%%%%%%%%%%%%%%%%%%%%%%%%%%%%%%
\appendix

\settowidth\MacroIndent{\rmfamily\scriptsize 000\ }

 \DocInput{childdoc.dtx}

\end{document}
%</driver>
% \fi
%
% %%%%%%%%%%%%%%%%%%%%%%%%%%%%%%%%%%%%%%%%%%%%%%%%%%%%%%%%%%%%%%%%%%%%%%%%%%%%%%
% %%%%%%%%%%%%%%%%%%%%%%%%%%%%%%%%%%%%%%%%%%%%%%%%%%%%%%%%%%%%%%%%%%%%%%%%%%%%%%
% \section{Sample}
%\iffalse
%<*samplemain>
%\fi
%
% The following presents a sample document
% with two chapters, two parts, a title page,
% a compile flag as well as three forwarding files to set the flag.
% It consists of eight |.tex| files:
% \begin{center}
% \begin{tabular}{ll}
% |cdocsamp.tex|&main file\\
% |cdocsch1.tex|&include file for chapter 1\\
% |cdocsch2.tex|&include file for chapter 2\\
% |cdocspt3.tex|&include file for part 3\\
% |cdocspt4.tex|&include file for part 4\\
% |cdocsdrf.tex|&forwarding file for main file in draft mode\\
% |cdocsfi1.tex|&forwarding file for final version of chapter 1\\
% |cdocsfi2.tex|&forwarding file for final version of chapter 2\\
% \end{tabular}
% \end{center}
% Each of the eight files can be compiled directly by the \LaTeX{} compiler.
%
% %%%%%%%%%%%%%%%%%%%%%%%%%%%%%%%%%%%%%%
% \paragraph{Main File.}
%
% The main file is called |cdocsamp.tex|.
%
% Load the \textsf{childdoc} definitions and
% declare the filename for the main document:
%    \begin{macrocode}
\input{childdoc.def}
\childdocmain{}
%    \end{macrocode}

% Optional override for |\version| flag:
%    \begin{macrocode}
%%\ifchilddoc\else\providecommand{\version}{draft}\fi
%    \end{macrocode}

% Define the default values for the |\version| flag
% (|final| for the main file and |draft| for childs):
%    \begin{macrocode}
\ifchilddoc
\providecommand{\version}{draft}
\else
\providecommand{\version}{final}
\fi
%    \end{macrocode}

% Load the standard document class:
%    \begin{macrocode}
\documentclass[12pt]{article}
%    \end{macrocode}

% Start the document body:
%    \begin{macrocode}
\begin{document}
%    \end{macrocode}

% Declare a title page.
% Print title, part of document being processed and version flag:
%    \begin{macrocode}
\addtocounter{page}{-1}
\begin{center}
{\LARGE\bfseries{}childdoc example\par}
\vspace{1cm}
\ifchilddoc
\ifchilddocmanual part\else chapter\fi:
`\childdocname' of `\childdocjob'\par
\else
main document: `\childdocjob'\par
\fi
version: \version\par
\end{center}
\newpage
%    \end{macrocode}

% Manually include selected file,
% otherwise process as usual:
%    \begin{macrocode}
\ifchilddocmanual
\section*{part `\childdocname'}
\input{\childdocname}
\else
%    \end{macrocode}

% Include the two chapters:
%    \begin{macrocode}
\include{cdocsch1}
\include{cdocsch2}
%    \end{macrocode}

% Include the two parts unless only chapters should be displayed:
%    \begin{macrocode}
\ifchilddoc\else
\section{part three}
\input{cdocspt3}
\section{part four}
\input{cdocspt4}
\fi
%    \end{macrocode}

% Process as usual until here:
%    \begin{macrocode}
\fi
%    \end{macrocode}

% End of document body:
%    \begin{macrocode}
\end{document}
%    \end{macrocode}
%\iffalse
%</samplemain>
%\fi
%
% %%%%%%%%%%%%%%%%%%%%%%%%%%%%%%%%%%%%%%
% \paragraph{Chapter Include Files.}
%
% The include files are called |cdocsch1.tex| and |cdocsch2.tex|.
%
%\iffalse
%<*samplechap1|samplechap2>
%\fi

% Optional override for |\version| flag:
%    \begin{macrocode}
%%\providecommand{\version}{final}
%    \end{macrocode}

% Include the main document:
%    \begin{macrocode}
\input{childdoc.def}
\childdocof{cdocsamp}
%    \end{macrocode}

%\iffalse
%</samplechap1|samplechap2>
%\fi
%
%\iffalse
%<*samplechap1>
%\fi
% Some text for chapter 1:
%    \begin{macrocode}
\section{one}
some text in chapter one
%    \end{macrocode}

%\iffalse
%</samplechap1>
%\fi
% Some text for chapter 2:
%\iffalse
%<*samplechap2>
%\fi
%    \begin{macrocode}
\section{two}
more text in chapter two
%    \end{macrocode}

%\iffalse
%</samplechap2>
%\fi
%
% %%%%%%%%%%%%%%%%%%%%%%%%%%%%%%%%%%%%%%
% \paragraph{Part Include Files.}
%
% The include files are called |cdocspt3.tex| and |cdocspt4.tex|.
%
%\iffalse
%<*samplepart3|samplepart4>
%\fi

% Optional override for |\version| flag:
%    \begin{macrocode}
%%\providecommand{\version}{final}
%    \end{macrocode}

% Include the main document:
%    \begin{macrocode}
\input{childdoc.def}
\childdocby{cdocsamp}
%    \end{macrocode}

%\iffalse
%</samplepart3|samplepart4>
%\fi
%
%\iffalse
%<*samplepart3>
%\fi
% Some text for part 3:
%    \begin{macrocode}
some text in part three
%    \end{macrocode}

%\iffalse
%</samplepart3>
%\fi
% Some text for part 4:
%\iffalse
%<*samplepart4>
%\fi
%    \begin{macrocode}
more text in part four
%    \end{macrocode}

%\iffalse
%</samplepart4>
%\fi
%
% %%%%%%%%%%%%%%%%%%%%%%%%%%%%%%%%%%%%%%
% \paragraph{Forwarding for a Complete Draft.}
%
% The following forwarding file |cdocsdrf.tex|
% compiles the main document in draft mode:
%\iffalse
%<*sampledraft>
%\fi
%    \begin{macrocode}
\def\version{draft}
\input{childdoc.def}
\childdocforward{cdocsamp}
%    \end{macrocode}

%\iffalse
%</sampledraft>
%\fi
%
% %%%%%%%%%%%%%%%%%%%%%%%%%%%%%%%%%%%%%%
% \paragraph{Forwarding for Final Version of the Chapters.}
%
% The following forwarding files |cdocsfn1.tex| and |cdocsfn2.tex|
% (with identical content)
% compile the final versions of the child documents
% |cdocsch1.tex| and |cdocsch2.tex|, respectively:
%\iffalse
%<*samplefinal>
%\fi
%    \begin{macrocode}
\def\version{final}
\input{childdoc.def}
\childdocforwardprefix[cdocsamp]{cdocsfn}{cdocsch}
%    \end{macrocode}

%\iffalse
%</samplefinal>
%\fi
%
% %%%%%%%%%%%%%%%%%%%%%%%%%%%%%%%%%%%%%%
% \paragraph{Command Line Processing.}
%
% The following three command lines generate the output files
% |cdocscld|, |cdocscl1| and |cdocscl2|
% which should be identical to
% |cdocsdrf|, |cdocsch1| and |cdocsfn2|, respectively:
% \begin{center}
% \begin{tabular}{l}
% |latex -jobname cdocscld \|\\
% |  "\def\version{draft}\input{childdoc.def}\childdocforward{cdocsamp}"|\\
% |latex -jobname cdocscl1 \|\\
% |  "\input{childdoc.def}\childdocforward[cdocsamp]{cdocsch1}"|\\
% |latex -jobname cdocscl2 \|\\
% |  "\def\version{final}\input{childdoc.def}\childdocforward{cdocsch2}"|
% \end{tabular}
% \end{center}
% Note that the trailing backslash on each first line
% merely continues the input to the second line
% (for convenient cut ant paste).
% Furthermore, the command |latex| can be replaced by any
% of its alternative versions such as |pdflatex|.
%
% %%%%%%%%%%%%%%%%%%%%%%%%%%%%%%%%%%%%%%%%%%%%%%%%%%%%%%%%%%%%%%%%%%%%%%%%%%%%%%
% %%%%%%%%%%%%%%%%%%%%%%%%%%%%%%%%%%%%%%%%%%%%%%%%%%%%%%%%%%%%%%%%%%%%%%%%%%%%%%
% \section{Implementation}
%\iffalse
%<*package>
%\fi
%
% This section describes the definitions file |childdoc.def|.

% The definitions cannot be loaded using |\usepackage| or |\RequirePackage|
% which has a mechanism to prevent loading a style file more than once.
% When loading the definitions by means of |\input|
% multiple instances have to be prevented manually:
%\iffalse
%This code needs to be before the `\ProvidesFile' directive
%which is defined at the beginning of this file.
%Therefore it is also placed there and commented out here.
%</package>
%<*discard>
%\fi
%    \begin{macrocode}
\ifdefined\childdocmain\endinput\fi
%    \end{macrocode}
%\iffalse
%</discard>
%<*package>
%\fi
%
% \macro{\ifchilddoc}
% \macro{\ifchilddocmanual}
% The conditional |\ifchilddoc| tells whether a
% child (true) or main (false) document is being compiled.
% The conditional |\ifchilddocmanual| tells whether
% the |\includeonly| mechanism is used (false) or
% the selection of child files must be performed manually (true).
% The definitions initialise to false:
%    \begin{macrocode}
\newif\ifchilddoc
\newif\ifchilddocmanual
%    \end{macrocode}

% \macro{\childdocname}
% \macro{\childdocjob}
% The macro |\childdocname| stores the name of the main document
% to be compiled. The macro |\childdocjob| stores the name of
% the document on which the \LaTeX{} compiler was originally invoked.
% The content of |\jobname| cannot be compared
% to filenames specified in the source due to different catcodes.
% The following code rescans |\jobname|, stores the result
% in |\childdocname| and saves a copy in |\childdocjob|:
%    \begin{macrocode}
\edef\childdocname{\scantokens\expandafter{\jobname\noexpand}}
\let\childdocjob\childdocname
%    \end{macrocode}

% \macro{\childdocdisable}
% The macro |\childdocdisable| prevents the main file
% from being processed more than once.
% At this stage, the main document command |\childdocmain|
% is assumed to be called once again where it should do nothing.
% Any subsequent call to it should prevent
% a secondary processing of the main document
% It overwrites the forwarding commands
% |\childdocof| and |\childdocforward|
% with empty macros to prevent further inclusions of the main document:
%    \begin{macrocode}
\newcommand{\childdocdisable}
{
  \renewcommand{\childdocmain}[1]{\renewcommand{\childdocmain}[1]{\endinput}}
  \renewcommand{\childdocof}[1]{}
  \renewcommand{\childdocby}[2][]{}
  \renewcommand{\childdocforward}[2][]{}
  \renewcommand{\childdocdisable}{}
}
%    \end{macrocode}

% \macro{\childdocmain}
% The macro |\childdocmain| is to be called at the top of the main file
% with nothing or the main filename (without extension) as argument.
% First, it breaks loops.
% If the argument is not empty and does not match |\childdocname|
% (which is set by the first inclusion of |childdoc.def|),
% |\ifchilddoc| is set to true, |\includeonly| is applied to the child file
% and |\jobname| is set to the main file
% (for proper handling of |.aux| files):
%    \begin{macrocode}
\newcommand{\childdocmain}[1]
{
  \childdocdisable\childdocmain{}
  \if?#1?\else
    \begingroup
      \def\childdoctmp{#1}
      \ifx\childdoctmp\childdocname
        \def\childdoctmp{}
      \else
        \def\childdoctmp
        {
          \childdoctrue
          \includeonly{\childdocname}
          \def\childdocjob{#1}
          \def\jobname{#1}
        }
      \fi
      \expandafter
    \endgroup
    \childdoctmp
  \fi
}
%    \end{macrocode}

% \macro{\childdocof}
% The command |\childdocof| redirects
% compilation to the main file |#1|.
%    \begin{macrocode}
\newcommand{\childdocof}[1]
{
  \childdocdisable
  \childdoctrue
  \includeonly{\childdocname}
  \def\jobname{#1}
  \def\childdocjob{#1}
  \input{#1}
}
%    \end{macrocode}

% \macro{\childdocby}
% The command |\childdocby| ....
%    \begin{macrocode}
\newcommand{\childdocby}[2][]
{
  \childdocdisable
  \childdoctrue
  \childdocmanualtrue
  \if?#1?\else
    \def\jobname{#2}
  \fi
  \def\childdocjob{#2}
  \input{#2}
  \endinput
}
%    \end{macrocode}

% \macro{\childdocforward}
% The command |\childdocforward| redirects
% compilation to the main file or
% (if the optional argument is given) a child file.
% Parameters are set as if the main file
% or a child file starting with |\childdocof| was compiled.
% Then compilation is handed over to the main file:
%    \begin{macrocode}
\newcommand{\childdocforward}[2][]
{
  \begingroup
    \if?#1?
      \def\childdoctmp
      {
        \def\childdocname{#2}
        \def\childdocjob{#2}
        \def\jobname{#2}
        \input{#2}
        \endinput
      }
    \else
      \def\childdoctmp
      {
        \childdocdisable
        \def\childdocname{#2}
        \childdoctrue
        \includeonly{#2}
        \def\childdocjob{#1}
        \def\jobname{#1}
        \input{#1}
        \endinput
      }
    \fi
    \expandafter
  \endgroup
  \childdoctmp
}
%    \end{macrocode}

% \macro{\childdocforwardprefix}
% The command |\childdocforwardprefix| redirects
% compilation to the main or a child file by means of a pattern.
% The prefix |#1| in the current filename is replaced by |#2|
% and the suffix of the current filename is kept
% (it is assumed that the filename does not contain the substring `|~~~|'
% which is used as a delimiter).
% Compilation is handed over to the new file by |\childdocforward|:
%    \begin{macrocode}
\newcommand{\childdocforwardprefix}[3][]
{
  \begingroup
    \def\childdocextract #2##1~~~{\def\childdoctmp{\childdocforward[#1]{#3##1}}}
    \expandafter\childdocextract\childdocname~~~
    \expandafter
  \endgroup
  \childdoctmp
}
%    \end{macrocode}

% \macro{\childdoc}
% The deprecated macro |\childdoc| is a legacy version of |\childdocmain|:
%    \begin{macrocode}
\newcommand{\childdoc}{\childdocmain}
%    \end{macrocode}

% \macro{\childdocredirect}
% The deprecated macro |\childdocredirect| is a legacy version
% of |\childdocforward| and |\childdocforwardprefix|:
%    \begin{macrocode}
\newcommand{\childdocredirect}[2][]
{
  \begingroup
    \if?#1?
      \def\childdoctmp{\childdocforward{#2}}
    \else
      \def\childdoctmp{\childdocforwardprefix{#1}{#2}}
    \fi
    \expandafter
  \endgroup
  \childdoctmp
}
%    \end{macrocode}

%\iffalse
%</package>
%\fi
%
\endinput
|\\
|\childdocby{|\textit{main}|}|\\
\end{tabular}
\end{center}
%
The directive |\childdocby| is similar to |\childdocof|
described in \secref{sec:include},
but the subsequent selection of content must be done manually.
To that end, both |\ifchilddoc| and |\ifchilddocmanual|
will be true upon processing of a part,
and the name of the part is stored in |\childdocname|.
Note that |\jobname| will be set to the filename of the current part
so that each part receives an individual |.aux| file
that does not interfere with the |.aux| file(s) of the main document.
This behaviour can be altered by the alternative form
|\childdocby[*]{|\textit{main}|}| (with a non-empty optional argument)
which uses the |.aux| file of the main document
by setting |\jobname| to \textit{main}.

%%%%%%%%%%%%%%%%%%%%%%%%%%%%%%%%%%%%%%%%%%%%%%%%%%%%%%%%%%%%%%%%%%%%%%%%%%%%%%%%
\subsection{Driver Development}
\label{sec:driver}

The \textsf{childdoc} mechanism can also be use for the development
of definition files such as \LaTeX{} styles or classes.
This case differs from the above setup with multiple parts
included by |\include| in that no |\includeonly| should be invoked.
This can be achieved by starting the include file
(before |\ProvidesPackage|) with:
%
\begin{center}
\begin{tabular}{l}
|% \iffalse
%
% childdoc.dtx Copyright (C) 2017-2018 Niklas Beisert
%
% This work may be distributed and/or modified under the
% conditions of the LaTeX Project Public License, either version 1.3
% of this license or (at your option) any later version.
% The latest version of this license is in
%   http://www.latex-project.org/lppl.txt
% and version 1.3 or later is part of all distributions of LaTeX
% version 2005/12/01 or later.
%
% This work has the LPPL maintenance status `maintained'.
%
% The Current Maintainer of this work is Niklas Beisert.
%
% This work consists of the files childdoc.dtx and childdoc.ins
% and the derived files childdoc.def and cdocsamp.tex with
% cdocsch1.tex, cdocsch2.tex, cdocsdrf.tex, cdocsfn1.tex, cdocsfn2.tex.
%
%<package>\ifdefined\childdocmain\endinput\fi
%<package>\ProvidesFile{childdoc.def}[2018/12/30 v2.0 child document driver]
%<samplemain>\ProvidesFile{cdocsamp.tex}[2018/12/30 v2.0 sample for childdoc]
%<*driver>
%\ProvidesFile{childdoc.drv}[2018/12/30 v2.0 childdoc reference manual file]
\PassOptionsToClass{10pt,a4paper}{article}
\documentclass{ltxdoc}

\usepackage[margin=35mm]{geometry}
\usepackage{hyperref}
\usepackage{hyperxmp}
\usepackage[usenames]{color}

\hypersetup{colorlinks=true}
\hypersetup{pdfstartview=FitH}
\hypersetup{pdfpagemode=UseNone}
\hypersetup{pdfsource={}}
\hypersetup{pdflang={en-UK}}
\hypersetup{pdfcopyright={Copyright 2017-2018 Niklas Beisert.
  This work may be distributed and/or modified under the
  conditions of the LaTeX Project Public License, either version 1.3
  of this license or (at your option) any later version.}}
\hypersetup{pdflicenseurl={http://www.latex-project.org/lppl.txt}}
\hypersetup{pdfcontactaddress={ETH Zurich, ITP, HIT K,
  Wolfgang-Pauli-Strasse 27}}
\hypersetup{pdfcontactpostcode={8093}}
\hypersetup{pdfcontactcity={Zurich}}
\hypersetup{pdfcontactcountry={Switzerland}}
\hypersetup{pdfcontactemail={nbeisert@itp.phys.ethz.ch}}
\hypersetup{pdfcontacturl={http://people.phys.ethz.ch/\xmptilde nbeisert/}}

\newcommand{\secref}[1]{\hyperref[#1]{section \ref*{#1}}}

\parskip1ex
\parindent0pt
\let\olditemize\itemize
\def\itemize{\olditemize\parskip0pt}

\begin{document}

\title{The \textsf{childdoc} Package}
\hypersetup{pdftitle={The childdoc Package}}
\author{Niklas Beisert\\[2ex]
  Institut f\"ur Theoretische Physik\\
  Eidgen\"ossische Technische Hochschule Z\"urich\\
  Wolfgang-Pauli-Strasse 27, 8093 Z\"urich, Switzerland\\[1ex]
  \href{mailto:nbeisert@itp.phys.ethz.ch}
  {\texttt{nbeisert@itp.phys.ethz.ch}}}
\hypersetup{pdfauthor={Niklas Beisert}}
\hypersetup{pdfsubject={Manual for the LaTeX2e Package childdoc}}
\date{30 December 2018, \textsf{v2.0}}
\maketitle

\begin{abstract}\noindent
\textsf{childdoc} is a \LaTeXe{} package
that enables the direct compilation
of document sections included by |\include|
to individual files.
\end{abstract}

\begingroup
\parskip0ex
\tableofcontents
\endgroup

%%%%%%%%%%%%%%%%%%%%%%%%%%%%%%%%%%%%%%%%%%%%%%%%%%%%%%%%%%%%%%%%%%%%%%%%%%%%%%%%
%%%%%%%%%%%%%%%%%%%%%%%%%%%%%%%%%%%%%%%%%%%%%%%%%%%%%%%%%%%%%%%%%%%%%%%%%%%%%%%%
\section{Introduction}

\LaTeX{} provides a mechanism to structure a large document (such as a book)
into a main file and several child files (containing the chapters)
using the |\include| command.
This mechanism is beneficial for documents
which span hundreds of pages in order to
make the source file(s) more manageable.
Moreover, compilation can be restricted to
selected child files by means of the |\includeonly| command.
The latter feature can be used to reduce the compilation time while editing
(this was significantly more useful in the earlier days of \LaTeX{})
or to generate a smaller document which is easier to navigate.
Another application of |\includeonly| is to generate
documents consisting of selected parts of the complete document.

However, there are a few drawbacks of the plain |\include| mechanism:
\begin{itemize}
\item
The child files cannot be compiled on their own,
they can only be compiled via the main file.
A naive editing environment
(such as a text editor with an option
to have the current file processed by \LaTeX)
may require one to switch to the main file before compiling;
attempting to compile the child file produces errors.
\item
The main file must be modified (each time)
to adjust the |\includeonly| command
to the present needs. This easily leaves the main file in a messy state.
\item
The generated document will always carry the filename
of the main document. This is inconvenient if
several child files are to be compiled and
to be kept for distribution.
\end{itemize}

The present package provides a simple interface
to make child files individually compilable by \LaTeX{}.
Compiling a child file then has the same effect as compiling
the main file with an |\includeonly| command
to select the appropriate child.
Moreover the generated document will carry the name of the child
rather than the main file.
This resolves all three above issues.

This feature is meant to make the editing of books,
thesis documents and lecture notes somewhat more convenient.
However, the package can also be used efficiently for
composing a series of documents (such as exercise sheets)
which are typically distributed individually.
It then assists the author in generating the individual documents
(potentially in different versions)
as well as a document containing the collected series.
Another application is in developing style files
or other kinds of included material
where compilation of the style file could redirect
to a sample or test file.

%%%%%%%%%%%%%%%%%%%%%%%%%%%%%%%%%%%%%%%%%%%%%%%%%%%%%%%%%%%%%%%%%%%%%%%%%%%%%%%%
%%%%%%%%%%%%%%%%%%%%%%%%%%%%%%%%%%%%%%%%%%%%%%%%%%%%%%%%%%%%%%%%%%%%%%%%%%%%%%%%
\section{Usage}

First of all, the package \textsf{childdoc} is \emph{not} a standard
\LaTeXe{} |.sty| style file! Therefore it needs to be invoked in
a non-standard way.

%%%%%%%%%%%%%%%%%%%%%%%%%%%%%%%%%%%%%%%%%%%%%%%%%%%%%%%%%%%%%%%%%%%%%%%%%%%%%%%%
\subsection{Included Files}
\label{sec:include}

%%%%%%%%%%%%%%%%%%%%%%%%%%%%%%%%%%%%%%%%
\DescribeMacro{\childdocmain}
To use the package, add the commands
\begin{center}
\begin{tabular}{l}
|\input{childdoc.def}|\\
|\childdocmain{}|\\
\end{tabular}
\end{center}
at the very top of the main \LaTeX{} file,
in particular \emph{before} the |\documentclass| statement!
The argument of |\childdocmain| should be left empty
(but it must be present).

%%%%%%%%%%%%%%%%%%%%%%%%%%%%%%%%%%%%%%%%
\DescribeMacro{\childdocof}
Furthermore, add the commands
\begin{center}
\begin{tabular}{l}
|\input{childdoc.def}|\\
|\childdocof{|\textit{main}|}|\\
\end{tabular}
\end{center}
at the top of every child file \textit{child}
which is included by |\include{|\textit{child}|}|
from within the main file
(or at least for those files to be compiled individually).
The argument \textit{main} must be the filename of the main file.

There are a couple of
considerations in setting up the main and child documents:

%%%%%%%%%%%%%%%%%%%%%%%%%%%%%%%%%%%%%%%%
\paragraph{Restrictions.}

Please note the following restrictions:
\begin{itemize}
\item
|\childdocmain| must be called with one argument \textit{main}
to ensure compatibility with earlier version of the package.
It must either be empty (|\childdocmain{}|)
or precisely match the filename of the main file in which it is specified.
See \secref{sec:detection} for further information.
\item
The filename \textit{main} must be specified without the |.tex| extension.
\item
The filename \textit{main} is case sensitive
(even in case-insensitive file systems)
due to internal string comparison.
\item
The argument \textit{main} should be fully expanded, it cannot be a macro.
\item
Subdirectories and special characters should be avoided in filenames.
\item
The command |\childdocmain{|\textit{main}|}| must be followed by a whitespace.
It should not be followed immediately by another command
or by a comment mark `|%|'.
This is because the \TeX{} parser reads the token immediately following
the argument of |\childdocmain| and puts it
at the beginning of every child section;
however, a white\-space is ignored.
\end{itemize}

%%%%%%%%%%%%%%%%%%%%%%%%%%%%%%%%%%%%%%%%
\paragraph{Content of Main File.}

It is advisable to place all content in the child files included by |\include|.
Any output contained in the main file will appear in all child documents
unless suppressed manually;
it cannot be suppressed automatically by the |\includeonly| directive
and thus should normally be avoided.
A method to include some content in the main file
by means of conditional processing is described in \secref{sec:conditional}.

%%%%%%%%%%%%%%%%%%%%%%%%%%%%%%%%%%%%%%%%
\paragraph{Page Numbering.}

When only a part of the document is compiled,
the appropriate numbering of pages
(as well as other status parameters)
is determined from the |.aux| files.
The latter contain information from previous passes.
However this information needs to propagate through
all intermediate child documents.
Therefore the page numbering in child documents may well
be inconsistent until the complete document is compiled at least once.

A useful (if unconventional) way to always ensure a consistent
page numbering is to restart the numbering in each child document
and denote the pages by `\textit{child}|.|\textit{page}'
where \textit{child} represents the chapter/section number of the child file.
This can be achieved by the command
|\numberwithin{page}{|\textit{child}|}|
of the \textsf{amsmath} package
where \textit{child} can be |chapter| or |section|
depending on the chosen structuring.
Alternatively, one can modify the macro |\thepage| appropriately
and reset the counter |page| at the start of each child file.

%%%%%%%%%%%%%%%%%%%%%%%%%%%%%%%%%%%%%%%%%%%%%%%%%%%%%%%%%%%%%%%%%%%%%%%%%%%%%%%%
\subsection{Conditional Processing}
\label{sec:conditional}

The package provides a mechanism to compile different versions
of a document. To customise the versions further some conditional processing
can come in handy to distinguish which version is being compiled.
The package provides two macros to describe the compilation context:

%%%%%%%%%%%%%%%%%%%%%%%%%%%%%%%%%%%%%%%%
\DescribeMacro{\ifchilddoc}
The conditional |\ifchilddoc| distinguishes between the compilation of
child documents and the main document:
%
\begin{center}
|\ifchilddoc |\textit{child-code}| |[|\||else |\textit{main-code}]| \||fi|
\end{center}

%%%%%%%%%%%%%%%%%%%%%%%%%%%%%%%%%%%%%%%%
\DescribeMacro{\childdocname}
\DescribeMacro{\childdocjob}
The macro |\childdocname| contains the filename (without extension)
of the main or child file being processed.
Note that |\childdocjob| will always contain the name of the main file.

%%%%%%%%%%%%%%%%%%%%%%%%%%%%%%%%%%%%%%%%
\paragraph{Title Page.}

Conditional processing can be used to include a title or banner page
in the main document when proper precautions are taken.
Importantly, the code in the main file should ensure that the page counter
(as well as other status parameters which are stored in the |.aux| files)
takes the same value after the conditional processing.
Otherwise the page numbers may take divergent values
depending on which part is compiled.

For example, a title page could be declared by:
%
\begin{center}
\begin{tabular}{l}
|\ifchilddoc\||else|\\
|\addtocounter{page}{-1}|\\
\textit{code for title page}\\
|\newpage|\\
|\||fi|
\end{tabular}
\end{center}
%
A banner page for the child documents can be generated by:
%
\begin{center}
\begin{tabular}{l}
|\ifchilddoc|\\
|\addtocounter{page}{-1}|\\
\textit{code for banner page}\\
|\newpage|\\
|\||fi|
\end{tabular}
\end{center}
%
Here one could write a message such as:
\begin{center}
|This is the part \childdocname{} of \childdocjob{}.|
\end{center}

%%%%%%%%%%%%%%%%%%%%%%%%%%%%%%%%%%%%%%%%%%%%%%%%%%%%%%%%%%%%%%%%%%%%%%%%%%%%%%%%
\subsection{Flags}
\label{sec:flags}

The package makes it easy to generate different versions
of the main or child documents.
To this end compilation flags can be defined
and assigned different default values.
They will be particularly useful in conjunction
with the forwarding mechanism described in \secref{sec:forward}.

For example, it may be useful to have a flag |\version|
which can be set to |draft| or |final|.
The document source will contain some conditional code
depending on the value of |\version|.
Suppose further, the flag should default to |final| for the main file
and to |draft| for child files
which is a natural assignment for editing the document.
This is achieved by placing the following code
in the preamble of the main document
(below the |\childdocmain| directive):
%
\begin{center}
\begin{tabular}{l}
|\ifchilddoc|\\
|\providecommand{\version}{draft}|\\
|\||else|\\
|\providecommand{\version}{final}|\\
|\||fi|
\end{tabular}
\end{center}
%
The definition by |\providecommand| makes sure
that previous definitions are not overwritten.
Further statements |\providecommand{\version}{...}|
can thus be added before the above code to override it.

For the main file, one might add a line
(between |\childdocmain| and the above block)
%
\begin{center}
|%\ifchilddoc\||else\providecommand{\version}{draft}\||fi|
\end{center}
%
which can be uncommented to produce a draft version.
Likewise one can add a line to the very top of a child file
(above the |\childdocof{|\textit{main}|}| directive)
%
\begin{center}
|%\providecommand{\version}{final}|
\end{center}
%
which can be uncommented to produce the final version of this child document.

%%%%%%%%%%%%%%%%%%%%%%%%%%%%%%%%%%%%%%%%%%%%%%%%%%%%%%%%%%%%%%%%%%%%%%%%%%%%%%%%
\subsection{Forwarding}
\label{sec:forward}

Different versions of the main or child documents
using compilation flags as described in \secref{sec:flags}
can be (permanently) stored in different files
for convenient compilation, viewing and distribution.
To this end, the package defines a command
to pass on compilation to a different file:

%%%%%%%%%%%%%%%%%%%%%%%%%%%%%%%%%%%%%%%%
\DescribeMacro{\childdocforward}
The command |\childdocforward| redirects processing to
another source file:
%
\begin{center}
\begin{tabular}{l}
|\input{childdoc.def}|\\
|\childdocforward[|\textit{main}|]{|\textit{dest}|}|\\
\end{tabular}
\end{center}
%
The argument \textit{dest} is the destination file
(without extension).
It should be the main file or one of the child files.
Note that further \textsf{childdoc} directives
such as |\childdocof| and |\childdocforward|
in the indicated file will be processed in this form.
The optional argument \textit{main}
passes on directly to the main file \textit{main}
while pretending to compile the child \textit{dest}.
This form behaves as if \textit{dest}
issues |\childdocof{|\textit{main}|}| right away,
and no further \textsf{childdoc} directives will be processed.

%%%%%%%%%%%%%%%%%%%%%%%%%%%%%%%%%%%%%%%%
\DescribeMacro{\...prefix}
In the alternative form |\childdocforwardprefix|,
%
\begin{center}
\begin{tabular}{l}
|\input{childdoc.def}|\\
|\childdocforwardprefix[|\textit{main}|]{|\textit{prefix}|}{|\textit{dest}|}|
\end{tabular}
\end{center}
%
the destination file is determined by a pattern
depending on the current file:
To make this work, the current file must be called
`{\textit{prefix}\hspace{0.2em}\textit{suffix}}'
with \textit{prefix} matching precisely the argument.
Processing is then passed on to the file
`{\textit{dest}\hspace{0.2em}\textit{suffix}}'.
Surely, the same effect is achieved by
directly specifying the
argument `{\textit{dest}\hspace{0.2em}\textit{suffix}}'
in the first form.
However, that requires to set up a different file
for each child. With the alternative form of the command
all these files can have exactly the same content
which simplifies setting them up and maintaining them.

For example, the following file |draft.tex|
with a compilation flag |\version| as described in \secref{sec:flags}
compiles the main document as a draft:
%
\begin{center}
\begin{tabular}{l}
|\def\version{draft}|\\
|\input{childdoc.def}|\\
|\childdocforward{|\textit{main}|}|
\end{tabular}
\end{center}
%
Likewise, the following files |final|\textit{nn}|.tex|
compile the final version of the child document
|child|\textit{nn}|.tex|:
%
\begin{center}
\begin{tabular}{l}
|\def\version{final}|\\
|\input{childdoc.def}|\\
|\childdocforwardprefix{final}{child}|
\end{tabular}
\end{center}
%

Note that when several versions of a main file and/or of each child file
are to be generated, it may be convenient to set up a |Makefile| or
shell script to automatise the process.

%%%%%%%%%%%%%%%%%%%%%%%%%%%%%%%%%%%%%%%%%%%%%%%%%%%%%%%%%%%%%%%%%%%%%%%%%%%%%%%%
\subsection{Command Line Processing}
\label{sec:commandline}

The effect of redirection files can also be achieved by invoking
the \LaTeX{} compiler with a more elaborate command line.
Most conveniently this should be done as part
of a shell script or a |Makefile|.

When using \textsf{childdoc} in the main file, the following
command lines effectively perform a redirection
(note that depending on the shell being used,
backslashes may have to be doubled: `|\|' $\to$ `|\\|'):
%
\begin{center}
|... -jobname "|\textit{target}|" |\\|"|[\textit{flags}]%
|\input{childdoc.def}\childdocforward[|\textit{main}|]{|\textit{dest}|}"|
\end{center}
%
Here \textit{target} is the name of the output file,
\textit{main} is the name of the main file
and \textit{dest} is the name of the main or child file to be processed
(all filenames without extensions).
The optional argument \textit{main} can be omitted
if \textit{main} matches \textit{dest}.
Optionally, compilation \textit{flags} can be defined via |\def| commands.
This command line makes the \TeX{} engine believe
it is compiling the file \textit{target}
whose content is specified as the latter parameter.
The provided code then forwards the processing to
\textit{main} or \textit{dest} as described in \secref{sec:forward}.

%%%%%%%%%%%%%%%%%%%%%%%%%%%%%%%%%%%%%%%%%%%%%%%%%%%%%%%%%%%%%%%%%%%%%%%%%%%%%%%%
\subsection{Include by Input}
\label{sec:input}

Including child documents by |\include| has some restrictions by design.
Most notably, the content of a child document always occupies
its own set of pages; pages cannot be shared between child documents.
Usually, this behaviour makes perfect sense
because each child document contain an essential part of the document.
However, in some situations it may be desirable to compose
a document from a collection of parts
without having mandatory page breaks between then.
For this case, the package
provides a mechanism to include parts
by |\input| which can also be processed individually.
However, by construction this mechanism
requires manual handling of the content to be output.

%%%%%%%%%%%%%%%%%%%%%%%%%%%%%%%%%%%%%%%%
\DescribeMacro{\ifchilddocmanual}
The main file should be prepared as usual, see \secref{sec:include}.
However, the document body must make a distinction
between processing of an individual part and of the main document, e.g.:
%
\begin{center}
\begin{tabular}{l}
|\ifchilddocmanual|\\
|\input{\childdocname}|\\
|\||else|\\
\textit{document body with }|\input{|\textit{part}|}|\\
|\||fi|
\end{tabular}
\end{center}
%
The conditional |\ifchilddocmanual| is true whenever
a part to be included by |\input| is being compiled,
and the name of the part is stored in |\childdocname|.

%%%%%%%%%%%%%%%%%%%%%%%%%%%%%%%%%%%%%%%%
\DescribeMacro{\childdocby}
Each part to be included by |\input| should start with:
%
\begin{center}
\begin{tabular}{l}
|\input{childdoc.def}|\\
|\childdocby{|\textit{main}|}|\\
\end{tabular}
\end{center}
%
The directive |\childdocby| is similar to |\childdocof|
described in \secref{sec:include},
but the subsequent selection of content must be done manually.
To that end, both |\ifchilddoc| and |\ifchilddocmanual|
will be true upon processing of a part,
and the name of the part is stored in |\childdocname|.
Note that |\jobname| will be set to the filename of the current part
so that each part receives an individual |.aux| file
that does not interfere with the |.aux| file(s) of the main document.
This behaviour can be altered by the alternative form
|\childdocby[*]{|\textit{main}|}| (with a non-empty optional argument)
which uses the |.aux| file of the main document
by setting |\jobname| to \textit{main}.

%%%%%%%%%%%%%%%%%%%%%%%%%%%%%%%%%%%%%%%%%%%%%%%%%%%%%%%%%%%%%%%%%%%%%%%%%%%%%%%%
\subsection{Driver Development}
\label{sec:driver}

The \textsf{childdoc} mechanism can also be use for the development
of definition files such as \LaTeX{} styles or classes.
This case differs from the above setup with multiple parts
included by |\include| in that no |\includeonly| should be invoked.
This can be achieved by starting the include file
(before |\ProvidesPackage|) with:
%
\begin{center}
\begin{tabular}{l}
|\input{childdoc.def}|\\
|\childdocforward{|\textit{main}|}|\\
\end{tabular}
\end{center}
%
or alternatively with:
%
\begin{center}
\begin{tabular}{l}
|\input{childdoc.def}|\\
|\childdocby{|\textit{main}|}|\\
\end{tabular}
\end{center}
%
Both forms have slightly different effects as described above.
The main file is prepared as usual, see \secref{sec:include}.

%%%%%%%%%%%%%%%%%%%%%%%%%%%%%%%%%%%%%%%%%%%%%%%%%%%%%%%%%%%%%%%%%%%%%%%%%%%%%%%%
\subsection{Legacy Detection}
\label{sec:detection}

The directive |\childdocmain| in the main file can detect
whether the complete document or merely a child is to be compiled
even without using the directive |\childdocof|.
This method is deprecated because it is less robust
and there is no compelling reason to use it;
it is merely provided for backward compatibility
and it may be removed in future versions.

If the detection mechanism is to be used,
it is mandatory to correctly specify
the filename of the main file as the argument of |\childdocmain|:
%
\begin{center}
\begin{tabular}{l}
|\input{childdoc.def}|\\
|\childdocmain{|\textit{main}|}|\\
\end{tabular}
\end{center}
%
If |\jobname| does not match the argument \textit{main} of |\childdocmain|,
it is assumed that |\jobname| points to the child file to be compiled.
When using |\childdocmain| with the main file specified as argument,
it suffices to start a child file
with just |\input{|\textit{main}|}|
without loading of the package and using |\childdocof|.
If instead all processing is done
with the appropriate \textsf{childdoc} directives,
the argument of \textit{main} of |\childdocmain| can be empty.

An alternative version of the command line processing described
in \secref{sec:commandline} using the detection mechanism reads:
%
\begin{center}
|... -jobname "|\textit{target}|" "|[\textit{flags}]%
[|\def\jobname{|\textit{dest}|}|]|\input{|\textit{main}|}"|
\end{center}

%%%%%%%%%%%%%%%%%%%%%%%%%%%%%%%%%%%%%%%%%%%%%%%%%%%%%%%%%%%%%%%%%%%%%%%%%%%%%%%%
\subsection{Manual Code}
\label{sec:manual}

In case one cannot be certain whether the definitions file |childdoc.def|
is installed on the target \TeX{} distribution
and one prefers not to ship it,
it is conceivable to paste a few relevant commands into the sources.

To that end, drop all statements |\input{childdoc.def}|
and perform the replacements as outlined below.
Instead of |\childdocmain{|\textit{main}|}| add the following code
to the top of the main file:
%
\begin{center}
\begin{tabular}{l}
|\||ifdefined\childdocname\endinput\||fi\newif\ifchilddoc|\\
|\edef\childdocname{\scantokens\expandafter{\jobname\noexpand}}|\\
|\def\childdocmain{|\textit{main}|}\||ifx\childdocmain\childdocname\||else|\\
|\childdoctrue\includeonly{\childdocname}\let\jobname\childdocmain\||fi|\\
\end{tabular}
\end{center}
%
Instead of |\childdocof{|\textit{main}|}| just include the main file
at the top of each child file:
%
\begin{center}
|\input{|\textit{main}|}|
\end{center}
%
A simple redirection |\childdocforward{|\textit{dest}|}| is achieved by:
%
\begin{center}
|\def\jobname{|\textit{dest}|}\input{\jobname}|
\end{center}
%
The redirection with prefix
|\childdocforwardprefix[|\textit{prefix}|]{|\textit{dest}|}|
is accomplished by:
%
\begin{center}
\begin{tabular}{l}
|{\edef\jobname{\scantokens\expandafter{\jobname\noexpand}}|\\
|\def\redirectjob |\textit{prefix}|#1~~~{\gdef\jobname{|\textit{dest}|#1}}|\\
|\expandafter\redirectjob\jobname~~~}\input{\jobname}|
\end{tabular}
\end{center}

In an alternative approach,
child documents can be compiled by a specific command line
without additional code or specific definitions:
%
\begin{center}
|... -jobname "|\textit{target}|" "|[\textit{flags}]%
|\includeonly{|\textit{dest}|}\input{|\textit{main}|}"|
\end{center}
%

%%%%%%%%%%%%%%%%%%%%%%%%%%%%%%%%%%%%%%%%%%%%%%%%%%%%%%%%%%%%%%%%%%%%%%%%%%%%%%%%
%%%%%%%%%%%%%%%%%%%%%%%%%%%%%%%%%%%%%%%%%%%%%%%%%%%%%%%%%%%%%%%%%%%%%%%%%%%%%%%%
\section{Information}

%%%%%%%%%%%%%%%%%%%%%%%%%%%%%%%%%%%%%%%%%%%%%%%%%%%%%%%%%%%%%%%%%%%%%%%%%%%%%%%%
\subsection{Copyright}

Copyright \copyright{} 2017--2018 Niklas Beisert

This work may be distributed and/or modified under the
conditions of the \LaTeX{} Project Public License, either version 1.3
of this license or (at your option) any later version.
The latest version of this license is in
  \url{http://www.latex-project.org/lppl.txt}
and version 1.3 or later is part of all distributions of \LaTeX{}
version 2005/12/01 or later.

This work has the LPPL maintenance status `maintained'.

The Current Maintainer of this work is Niklas Beisert.

This work consists of the files |README.txt|, |childdoc.ins| and |childdoc.dtx|
as well as the derived files |childdoc.def|, |cdocsamp.tex|
with |cdocsch1.tex|, |cdocsch2.tex|, |cdocspt3.tex|, |cdocspt4.tex|,
|cdocsdrf.tex|, |cdocsfn1.tex|, |cdocsfn2.tex|
as well as |childdoc.pdf|.

%%%%%%%%%%%%%%%%%%%%%%%%%%%%%%%%%%%%%%%%%%%%%%%%%%%%%%%%%%%%%%%%%%%%%%%%%%%%%%%%
\subsection{Files and Installation}

The package consists of the files:
%
\begin{center}
\begin{tabular}{ll}
    |README.txt|   & readme file \\
    |childdoc.ins| & installation file \\
    |childdoc.dtx| & source file \\
    |childdoc.def| & definition file \\
    |cdocsamp.tex| & sample main file \\
    |cdocsch1.tex| & sample include file \\
    |cdocsch2.tex| & sample include file \\
    |cdocspt3.tex| & sample part file \\
    |cdocspt4.tex| & sample part file \\
    |cdocsdrf.tex| & sample redirection file \\
    |cdocsfn1.tex| & sample redirection file \\
    |cdocsfn2.tex| & sample redirection file \\
    |childdoc.pdf| & manual
\end{tabular}
\end{center}
%
The distribution consists of the files
|README.txt|, |childdoc.ins| and |childdoc.dtx|.
%
\begin{itemize}
\item
Run (pdf)\LaTeX{} on |childdoc.dtx|
to compile the manual |childdoc.pdf| (this file).
\item
Run \LaTeX{} on |childdoc.ins| to create the definitions file |childdoc.def|
and the sample |cdocsamp.tex| with include files
|cdocsch1.tex|, |cdocsch2.tex|, |cdocspt3.tex|, |cdocspt4.tex|,
|cdocsdrf.tex|, |cdocsfn1.tex|, |cdocsfn2.tex|.
Then copy the file |childdoc.def| to an appropriate directory of your \LaTeX{}
distribution, e.g.\ \textit{texmf-root}|/tex/latex/childdoc|.
\end{itemize}

%%%%%%%%%%%%%%%%%%%%%%%%%%%%%%%%%%%%%%%%%%%%%%%%%%%%%%%%%%%%%%%%%%%%%%%%%%%%%%%%
\subsection{Related CTAN Packages}

There are several other packages which offer a similar functionality:
%
\begin{itemize}
\item
The packages
\href{http://ctan.org/pkg/docmute}{\textsf{docmute}},
\href{http://ctan.org/pkg/includex}{\textsf{includex}} and
\href{http://ctan.org/pkg/standalone}{\textsf{standalone}}
provide commands to include only the document body of
a child file thus allowing both files to be compiled individually.
\item
The packages \href{http://ctan.org/pkg/subdocs}{\textsf{subdocs}}
and \href{http://ctan.org/pkg/subfiles}{\textsf{subfiles}}
provide structures in which the main and child documents can be
encapsulated and allowing them to be compiled individually.
The inclusion mechanism is different from the conventional |\include|.
\item
The package \href{http://ctan.org/pkg/combine}{\textsf{combine}}
is an elaborate solution to combine several documents into one.
\end{itemize}
%
See also the CTAN topic \href{http://ctan.org/topic/subdocs}{\textsf{subdocs}}
for further related packages.
The present package differs from the above solutions in that
a document structure constructed with the conventional |\include| mechanism
just needs two extra commands at the top of every file
such that all constituent files can be compiled individually.

%%%%%%%%%%%%%%%%%%%%%%%%%%%%%%%%%%%%%%%%%%%%%%%%%%%%%%%%%%%%%%%%%%%%%%%%%%%%%%%%
%\subsection{Feature Suggestions}
%
%The following is a list of features which may be useful for future
%versions of this package:
%%
%\begin{itemize}
%\item
%\ldots
%\end{itemize}

%%%%%%%%%%%%%%%%%%%%%%%%%%%%%%%%%%%%%%%%%%%%%%%%%%%%%%%%%%%%%%%%%%%%%%%%%%%%%%%%
\subsection{Revision History}

%%%%%%%%%%%%%%%%%%%%%%%%%%%%%%%%%%%%%%%%
\paragraph{v2.0:} 2018/12/30

\begin{itemize}
\item
immediate forward processing
\item
added |\childdocby| mechanism
\item
manual restructured
\end{itemize}

%%%%%%%%%%%%%%%%%%%%%%%%%%%%%%%%%%%%%%%%
\paragraph{v1.6:} 2018/01/17

\begin{itemize}
\item
application for development of include files
\item
corrections to manual
\end{itemize}

%%%%%%%%%%%%%%%%%%%%%%%%%%%%%%%%%%%%%%%%
\paragraph{v1.5:} 2017/05/21

\begin{itemize}
\item
more complete structuring introduced
\item
|\childdocof| introduced
\item
|\childdoc| renamed to |\childdocmain|
\item
|\childredirect| renamed to |\childdocforward| and |\childdocforwardprefix|
and functionality expanded
\end{itemize}

%%%%%%%%%%%%%%%%%%%%%%%%%%%%%%%%%%%%%%%%
\paragraph{v1.0:} 2017/04/27

\begin{itemize}
\item
manual and install package
\item
first version published on CTAN
\end{itemize}

%%%%%%%%%%%%%%%%%%%%%%%%%%%%%%%%%%%%%%%%
\paragraph{v0.6:} 2017/04/26

\begin{itemize}
\item
redirection mechanism added
\end{itemize}

%%%%%%%%%%%%%%%%%%%%%%%%%%%%%%%%%%%%%%%%
\paragraph{v0.5:} 2017/04/26

\begin{itemize}
\item
functionality in definition file
\end{itemize}


%%%%%%%%%%%%%%%%%%%%%%%%%%%%%%%%%%%%%%%%%%%%%%%%%%%%%%%%%%%%%%%%%%%%%%%%%%%%%%%%
%%%%%%%%%%%%%%%%%%%%%%%%%%%%%%%%%%%%%%%%%%%%%%%%%%%%%%%%%%%%%%%%%%%%%%%%%%%%%%%%
%%%%%%%%%%%%%%%%%%%%%%%%%%%%%%%%%%%%%%%%%%%%%%%%%%%%%%%%%%%%%%%%%%%%%%%%%%%%%%%%
\appendix

\settowidth\MacroIndent{\rmfamily\scriptsize 000\ }

 \DocInput{childdoc.dtx}

\end{document}
%</driver>
% \fi
%
% %%%%%%%%%%%%%%%%%%%%%%%%%%%%%%%%%%%%%%%%%%%%%%%%%%%%%%%%%%%%%%%%%%%%%%%%%%%%%%
% %%%%%%%%%%%%%%%%%%%%%%%%%%%%%%%%%%%%%%%%%%%%%%%%%%%%%%%%%%%%%%%%%%%%%%%%%%%%%%
% \section{Sample}
%\iffalse
%<*samplemain>
%\fi
%
% The following presents a sample document
% with two chapters, two parts, a title page,
% a compile flag as well as three forwarding files to set the flag.
% It consists of eight |.tex| files:
% \begin{center}
% \begin{tabular}{ll}
% |cdocsamp.tex|&main file\\
% |cdocsch1.tex|&include file for chapter 1\\
% |cdocsch2.tex|&include file for chapter 2\\
% |cdocspt3.tex|&include file for part 3\\
% |cdocspt4.tex|&include file for part 4\\
% |cdocsdrf.tex|&forwarding file for main file in draft mode\\
% |cdocsfi1.tex|&forwarding file for final version of chapter 1\\
% |cdocsfi2.tex|&forwarding file for final version of chapter 2\\
% \end{tabular}
% \end{center}
% Each of the eight files can be compiled directly by the \LaTeX{} compiler.
%
% %%%%%%%%%%%%%%%%%%%%%%%%%%%%%%%%%%%%%%
% \paragraph{Main File.}
%
% The main file is called |cdocsamp.tex|.
%
% Load the \textsf{childdoc} definitions and
% declare the filename for the main document:
%    \begin{macrocode}
\input{childdoc.def}
\childdocmain{}
%    \end{macrocode}

% Optional override for |\version| flag:
%    \begin{macrocode}
%%\ifchilddoc\else\providecommand{\version}{draft}\fi
%    \end{macrocode}

% Define the default values for the |\version| flag
% (|final| for the main file and |draft| for childs):
%    \begin{macrocode}
\ifchilddoc
\providecommand{\version}{draft}
\else
\providecommand{\version}{final}
\fi
%    \end{macrocode}

% Load the standard document class:
%    \begin{macrocode}
\documentclass[12pt]{article}
%    \end{macrocode}

% Start the document body:
%    \begin{macrocode}
\begin{document}
%    \end{macrocode}

% Declare a title page.
% Print title, part of document being processed and version flag:
%    \begin{macrocode}
\addtocounter{page}{-1}
\begin{center}
{\LARGE\bfseries{}childdoc example\par}
\vspace{1cm}
\ifchilddoc
\ifchilddocmanual part\else chapter\fi:
`\childdocname' of `\childdocjob'\par
\else
main document: `\childdocjob'\par
\fi
version: \version\par
\end{center}
\newpage
%    \end{macrocode}

% Manually include selected file,
% otherwise process as usual:
%    \begin{macrocode}
\ifchilddocmanual
\section*{part `\childdocname'}
\input{\childdocname}
\else
%    \end{macrocode}

% Include the two chapters:
%    \begin{macrocode}
\include{cdocsch1}
\include{cdocsch2}
%    \end{macrocode}

% Include the two parts unless only chapters should be displayed:
%    \begin{macrocode}
\ifchilddoc\else
\section{part three}
\input{cdocspt3}
\section{part four}
\input{cdocspt4}
\fi
%    \end{macrocode}

% Process as usual until here:
%    \begin{macrocode}
\fi
%    \end{macrocode}

% End of document body:
%    \begin{macrocode}
\end{document}
%    \end{macrocode}
%\iffalse
%</samplemain>
%\fi
%
% %%%%%%%%%%%%%%%%%%%%%%%%%%%%%%%%%%%%%%
% \paragraph{Chapter Include Files.}
%
% The include files are called |cdocsch1.tex| and |cdocsch2.tex|.
%
%\iffalse
%<*samplechap1|samplechap2>
%\fi

% Optional override for |\version| flag:
%    \begin{macrocode}
%%\providecommand{\version}{final}
%    \end{macrocode}

% Include the main document:
%    \begin{macrocode}
\input{childdoc.def}
\childdocof{cdocsamp}
%    \end{macrocode}

%\iffalse
%</samplechap1|samplechap2>
%\fi
%
%\iffalse
%<*samplechap1>
%\fi
% Some text for chapter 1:
%    \begin{macrocode}
\section{one}
some text in chapter one
%    \end{macrocode}

%\iffalse
%</samplechap1>
%\fi
% Some text for chapter 2:
%\iffalse
%<*samplechap2>
%\fi
%    \begin{macrocode}
\section{two}
more text in chapter two
%    \end{macrocode}

%\iffalse
%</samplechap2>
%\fi
%
% %%%%%%%%%%%%%%%%%%%%%%%%%%%%%%%%%%%%%%
% \paragraph{Part Include Files.}
%
% The include files are called |cdocspt3.tex| and |cdocspt4.tex|.
%
%\iffalse
%<*samplepart3|samplepart4>
%\fi

% Optional override for |\version| flag:
%    \begin{macrocode}
%%\providecommand{\version}{final}
%    \end{macrocode}

% Include the main document:
%    \begin{macrocode}
\input{childdoc.def}
\childdocby{cdocsamp}
%    \end{macrocode}

%\iffalse
%</samplepart3|samplepart4>
%\fi
%
%\iffalse
%<*samplepart3>
%\fi
% Some text for part 3:
%    \begin{macrocode}
some text in part three
%    \end{macrocode}

%\iffalse
%</samplepart3>
%\fi
% Some text for part 4:
%\iffalse
%<*samplepart4>
%\fi
%    \begin{macrocode}
more text in part four
%    \end{macrocode}

%\iffalse
%</samplepart4>
%\fi
%
% %%%%%%%%%%%%%%%%%%%%%%%%%%%%%%%%%%%%%%
% \paragraph{Forwarding for a Complete Draft.}
%
% The following forwarding file |cdocsdrf.tex|
% compiles the main document in draft mode:
%\iffalse
%<*sampledraft>
%\fi
%    \begin{macrocode}
\def\version{draft}
\input{childdoc.def}
\childdocforward{cdocsamp}
%    \end{macrocode}

%\iffalse
%</sampledraft>
%\fi
%
% %%%%%%%%%%%%%%%%%%%%%%%%%%%%%%%%%%%%%%
% \paragraph{Forwarding for Final Version of the Chapters.}
%
% The following forwarding files |cdocsfn1.tex| and |cdocsfn2.tex|
% (with identical content)
% compile the final versions of the child documents
% |cdocsch1.tex| and |cdocsch2.tex|, respectively:
%\iffalse
%<*samplefinal>
%\fi
%    \begin{macrocode}
\def\version{final}
\input{childdoc.def}
\childdocforwardprefix[cdocsamp]{cdocsfn}{cdocsch}
%    \end{macrocode}

%\iffalse
%</samplefinal>
%\fi
%
% %%%%%%%%%%%%%%%%%%%%%%%%%%%%%%%%%%%%%%
% \paragraph{Command Line Processing.}
%
% The following three command lines generate the output files
% |cdocscld|, |cdocscl1| and |cdocscl2|
% which should be identical to
% |cdocsdrf|, |cdocsch1| and |cdocsfn2|, respectively:
% \begin{center}
% \begin{tabular}{l}
% |latex -jobname cdocscld \|\\
% |  "\def\version{draft}\input{childdoc.def}\childdocforward{cdocsamp}"|\\
% |latex -jobname cdocscl1 \|\\
% |  "\input{childdoc.def}\childdocforward[cdocsamp]{cdocsch1}"|\\
% |latex -jobname cdocscl2 \|\\
% |  "\def\version{final}\input{childdoc.def}\childdocforward{cdocsch2}"|
% \end{tabular}
% \end{center}
% Note that the trailing backslash on each first line
% merely continues the input to the second line
% (for convenient cut ant paste).
% Furthermore, the command |latex| can be replaced by any
% of its alternative versions such as |pdflatex|.
%
% %%%%%%%%%%%%%%%%%%%%%%%%%%%%%%%%%%%%%%%%%%%%%%%%%%%%%%%%%%%%%%%%%%%%%%%%%%%%%%
% %%%%%%%%%%%%%%%%%%%%%%%%%%%%%%%%%%%%%%%%%%%%%%%%%%%%%%%%%%%%%%%%%%%%%%%%%%%%%%
% \section{Implementation}
%\iffalse
%<*package>
%\fi
%
% This section describes the definitions file |childdoc.def|.

% The definitions cannot be loaded using |\usepackage| or |\RequirePackage|
% which has a mechanism to prevent loading a style file more than once.
% When loading the definitions by means of |\input|
% multiple instances have to be prevented manually:
%\iffalse
%This code needs to be before the `\ProvidesFile' directive
%which is defined at the beginning of this file.
%Therefore it is also placed there and commented out here.
%</package>
%<*discard>
%\fi
%    \begin{macrocode}
\ifdefined\childdocmain\endinput\fi
%    \end{macrocode}
%\iffalse
%</discard>
%<*package>
%\fi
%
% \macro{\ifchilddoc}
% \macro{\ifchilddocmanual}
% The conditional |\ifchilddoc| tells whether a
% child (true) or main (false) document is being compiled.
% The conditional |\ifchilddocmanual| tells whether
% the |\includeonly| mechanism is used (false) or
% the selection of child files must be performed manually (true).
% The definitions initialise to false:
%    \begin{macrocode}
\newif\ifchilddoc
\newif\ifchilddocmanual
%    \end{macrocode}

% \macro{\childdocname}
% \macro{\childdocjob}
% The macro |\childdocname| stores the name of the main document
% to be compiled. The macro |\childdocjob| stores the name of
% the document on which the \LaTeX{} compiler was originally invoked.
% The content of |\jobname| cannot be compared
% to filenames specified in the source due to different catcodes.
% The following code rescans |\jobname|, stores the result
% in |\childdocname| and saves a copy in |\childdocjob|:
%    \begin{macrocode}
\edef\childdocname{\scantokens\expandafter{\jobname\noexpand}}
\let\childdocjob\childdocname
%    \end{macrocode}

% \macro{\childdocdisable}
% The macro |\childdocdisable| prevents the main file
% from being processed more than once.
% At this stage, the main document command |\childdocmain|
% is assumed to be called once again where it should do nothing.
% Any subsequent call to it should prevent
% a secondary processing of the main document
% It overwrites the forwarding commands
% |\childdocof| and |\childdocforward|
% with empty macros to prevent further inclusions of the main document:
%    \begin{macrocode}
\newcommand{\childdocdisable}
{
  \renewcommand{\childdocmain}[1]{\renewcommand{\childdocmain}[1]{\endinput}}
  \renewcommand{\childdocof}[1]{}
  \renewcommand{\childdocby}[2][]{}
  \renewcommand{\childdocforward}[2][]{}
  \renewcommand{\childdocdisable}{}
}
%    \end{macrocode}

% \macro{\childdocmain}
% The macro |\childdocmain| is to be called at the top of the main file
% with nothing or the main filename (without extension) as argument.
% First, it breaks loops.
% If the argument is not empty and does not match |\childdocname|
% (which is set by the first inclusion of |childdoc.def|),
% |\ifchilddoc| is set to true, |\includeonly| is applied to the child file
% and |\jobname| is set to the main file
% (for proper handling of |.aux| files):
%    \begin{macrocode}
\newcommand{\childdocmain}[1]
{
  \childdocdisable\childdocmain{}
  \if?#1?\else
    \begingroup
      \def\childdoctmp{#1}
      \ifx\childdoctmp\childdocname
        \def\childdoctmp{}
      \else
        \def\childdoctmp
        {
          \childdoctrue
          \includeonly{\childdocname}
          \def\childdocjob{#1}
          \def\jobname{#1}
        }
      \fi
      \expandafter
    \endgroup
    \childdoctmp
  \fi
}
%    \end{macrocode}

% \macro{\childdocof}
% The command |\childdocof| redirects
% compilation to the main file |#1|.
%    \begin{macrocode}
\newcommand{\childdocof}[1]
{
  \childdocdisable
  \childdoctrue
  \includeonly{\childdocname}
  \def\jobname{#1}
  \def\childdocjob{#1}
  \input{#1}
}
%    \end{macrocode}

% \macro{\childdocby}
% The command |\childdocby| ....
%    \begin{macrocode}
\newcommand{\childdocby}[2][]
{
  \childdocdisable
  \childdoctrue
  \childdocmanualtrue
  \if?#1?\else
    \def\jobname{#2}
  \fi
  \def\childdocjob{#2}
  \input{#2}
  \endinput
}
%    \end{macrocode}

% \macro{\childdocforward}
% The command |\childdocforward| redirects
% compilation to the main file or
% (if the optional argument is given) a child file.
% Parameters are set as if the main file
% or a child file starting with |\childdocof| was compiled.
% Then compilation is handed over to the main file:
%    \begin{macrocode}
\newcommand{\childdocforward}[2][]
{
  \begingroup
    \if?#1?
      \def\childdoctmp
      {
        \def\childdocname{#2}
        \def\childdocjob{#2}
        \def\jobname{#2}
        \input{#2}
        \endinput
      }
    \else
      \def\childdoctmp
      {
        \childdocdisable
        \def\childdocname{#2}
        \childdoctrue
        \includeonly{#2}
        \def\childdocjob{#1}
        \def\jobname{#1}
        \input{#1}
        \endinput
      }
    \fi
    \expandafter
  \endgroup
  \childdoctmp
}
%    \end{macrocode}

% \macro{\childdocforwardprefix}
% The command |\childdocforwardprefix| redirects
% compilation to the main or a child file by means of a pattern.
% The prefix |#1| in the current filename is replaced by |#2|
% and the suffix of the current filename is kept
% (it is assumed that the filename does not contain the substring `|~~~|'
% which is used as a delimiter).
% Compilation is handed over to the new file by |\childdocforward|:
%    \begin{macrocode}
\newcommand{\childdocforwardprefix}[3][]
{
  \begingroup
    \def\childdocextract #2##1~~~{\def\childdoctmp{\childdocforward[#1]{#3##1}}}
    \expandafter\childdocextract\childdocname~~~
    \expandafter
  \endgroup
  \childdoctmp
}
%    \end{macrocode}

% \macro{\childdoc}
% The deprecated macro |\childdoc| is a legacy version of |\childdocmain|:
%    \begin{macrocode}
\newcommand{\childdoc}{\childdocmain}
%    \end{macrocode}

% \macro{\childdocredirect}
% The deprecated macro |\childdocredirect| is a legacy version
% of |\childdocforward| and |\childdocforwardprefix|:
%    \begin{macrocode}
\newcommand{\childdocredirect}[2][]
{
  \begingroup
    \if?#1?
      \def\childdoctmp{\childdocforward{#2}}
    \else
      \def\childdoctmp{\childdocforwardprefix{#1}{#2}}
    \fi
    \expandafter
  \endgroup
  \childdoctmp
}
%    \end{macrocode}

%\iffalse
%</package>
%\fi
%
\endinput
|\\
|\childdocforward{|\textit{main}|}|\\
\end{tabular}
\end{center}
%
or alternatively with:
%
\begin{center}
\begin{tabular}{l}
|% \iffalse
%
% childdoc.dtx Copyright (C) 2017-2018 Niklas Beisert
%
% This work may be distributed and/or modified under the
% conditions of the LaTeX Project Public License, either version 1.3
% of this license or (at your option) any later version.
% The latest version of this license is in
%   http://www.latex-project.org/lppl.txt
% and version 1.3 or later is part of all distributions of LaTeX
% version 2005/12/01 or later.
%
% This work has the LPPL maintenance status `maintained'.
%
% The Current Maintainer of this work is Niklas Beisert.
%
% This work consists of the files childdoc.dtx and childdoc.ins
% and the derived files childdoc.def and cdocsamp.tex with
% cdocsch1.tex, cdocsch2.tex, cdocsdrf.tex, cdocsfn1.tex, cdocsfn2.tex.
%
%<package>\ifdefined\childdocmain\endinput\fi
%<package>\ProvidesFile{childdoc.def}[2018/12/30 v2.0 child document driver]
%<samplemain>\ProvidesFile{cdocsamp.tex}[2018/12/30 v2.0 sample for childdoc]
%<*driver>
%\ProvidesFile{childdoc.drv}[2018/12/30 v2.0 childdoc reference manual file]
\PassOptionsToClass{10pt,a4paper}{article}
\documentclass{ltxdoc}

\usepackage[margin=35mm]{geometry}
\usepackage{hyperref}
\usepackage{hyperxmp}
\usepackage[usenames]{color}

\hypersetup{colorlinks=true}
\hypersetup{pdfstartview=FitH}
\hypersetup{pdfpagemode=UseNone}
\hypersetup{pdfsource={}}
\hypersetup{pdflang={en-UK}}
\hypersetup{pdfcopyright={Copyright 2017-2018 Niklas Beisert.
  This work may be distributed and/or modified under the
  conditions of the LaTeX Project Public License, either version 1.3
  of this license or (at your option) any later version.}}
\hypersetup{pdflicenseurl={http://www.latex-project.org/lppl.txt}}
\hypersetup{pdfcontactaddress={ETH Zurich, ITP, HIT K,
  Wolfgang-Pauli-Strasse 27}}
\hypersetup{pdfcontactpostcode={8093}}
\hypersetup{pdfcontactcity={Zurich}}
\hypersetup{pdfcontactcountry={Switzerland}}
\hypersetup{pdfcontactemail={nbeisert@itp.phys.ethz.ch}}
\hypersetup{pdfcontacturl={http://people.phys.ethz.ch/\xmptilde nbeisert/}}

\newcommand{\secref}[1]{\hyperref[#1]{section \ref*{#1}}}

\parskip1ex
\parindent0pt
\let\olditemize\itemize
\def\itemize{\olditemize\parskip0pt}

\begin{document}

\title{The \textsf{childdoc} Package}
\hypersetup{pdftitle={The childdoc Package}}
\author{Niklas Beisert\\[2ex]
  Institut f\"ur Theoretische Physik\\
  Eidgen\"ossische Technische Hochschule Z\"urich\\
  Wolfgang-Pauli-Strasse 27, 8093 Z\"urich, Switzerland\\[1ex]
  \href{mailto:nbeisert@itp.phys.ethz.ch}
  {\texttt{nbeisert@itp.phys.ethz.ch}}}
\hypersetup{pdfauthor={Niklas Beisert}}
\hypersetup{pdfsubject={Manual for the LaTeX2e Package childdoc}}
\date{30 December 2018, \textsf{v2.0}}
\maketitle

\begin{abstract}\noindent
\textsf{childdoc} is a \LaTeXe{} package
that enables the direct compilation
of document sections included by |\include|
to individual files.
\end{abstract}

\begingroup
\parskip0ex
\tableofcontents
\endgroup

%%%%%%%%%%%%%%%%%%%%%%%%%%%%%%%%%%%%%%%%%%%%%%%%%%%%%%%%%%%%%%%%%%%%%%%%%%%%%%%%
%%%%%%%%%%%%%%%%%%%%%%%%%%%%%%%%%%%%%%%%%%%%%%%%%%%%%%%%%%%%%%%%%%%%%%%%%%%%%%%%
\section{Introduction}

\LaTeX{} provides a mechanism to structure a large document (such as a book)
into a main file and several child files (containing the chapters)
using the |\include| command.
This mechanism is beneficial for documents
which span hundreds of pages in order to
make the source file(s) more manageable.
Moreover, compilation can be restricted to
selected child files by means of the |\includeonly| command.
The latter feature can be used to reduce the compilation time while editing
(this was significantly more useful in the earlier days of \LaTeX{})
or to generate a smaller document which is easier to navigate.
Another application of |\includeonly| is to generate
documents consisting of selected parts of the complete document.

However, there are a few drawbacks of the plain |\include| mechanism:
\begin{itemize}
\item
The child files cannot be compiled on their own,
they can only be compiled via the main file.
A naive editing environment
(such as a text editor with an option
to have the current file processed by \LaTeX)
may require one to switch to the main file before compiling;
attempting to compile the child file produces errors.
\item
The main file must be modified (each time)
to adjust the |\includeonly| command
to the present needs. This easily leaves the main file in a messy state.
\item
The generated document will always carry the filename
of the main document. This is inconvenient if
several child files are to be compiled and
to be kept for distribution.
\end{itemize}

The present package provides a simple interface
to make child files individually compilable by \LaTeX{}.
Compiling a child file then has the same effect as compiling
the main file with an |\includeonly| command
to select the appropriate child.
Moreover the generated document will carry the name of the child
rather than the main file.
This resolves all three above issues.

This feature is meant to make the editing of books,
thesis documents and lecture notes somewhat more convenient.
However, the package can also be used efficiently for
composing a series of documents (such as exercise sheets)
which are typically distributed individually.
It then assists the author in generating the individual documents
(potentially in different versions)
as well as a document containing the collected series.
Another application is in developing style files
or other kinds of included material
where compilation of the style file could redirect
to a sample or test file.

%%%%%%%%%%%%%%%%%%%%%%%%%%%%%%%%%%%%%%%%%%%%%%%%%%%%%%%%%%%%%%%%%%%%%%%%%%%%%%%%
%%%%%%%%%%%%%%%%%%%%%%%%%%%%%%%%%%%%%%%%%%%%%%%%%%%%%%%%%%%%%%%%%%%%%%%%%%%%%%%%
\section{Usage}

First of all, the package \textsf{childdoc} is \emph{not} a standard
\LaTeXe{} |.sty| style file! Therefore it needs to be invoked in
a non-standard way.

%%%%%%%%%%%%%%%%%%%%%%%%%%%%%%%%%%%%%%%%%%%%%%%%%%%%%%%%%%%%%%%%%%%%%%%%%%%%%%%%
\subsection{Included Files}
\label{sec:include}

%%%%%%%%%%%%%%%%%%%%%%%%%%%%%%%%%%%%%%%%
\DescribeMacro{\childdocmain}
To use the package, add the commands
\begin{center}
\begin{tabular}{l}
|\input{childdoc.def}|\\
|\childdocmain{}|\\
\end{tabular}
\end{center}
at the very top of the main \LaTeX{} file,
in particular \emph{before} the |\documentclass| statement!
The argument of |\childdocmain| should be left empty
(but it must be present).

%%%%%%%%%%%%%%%%%%%%%%%%%%%%%%%%%%%%%%%%
\DescribeMacro{\childdocof}
Furthermore, add the commands
\begin{center}
\begin{tabular}{l}
|\input{childdoc.def}|\\
|\childdocof{|\textit{main}|}|\\
\end{tabular}
\end{center}
at the top of every child file \textit{child}
which is included by |\include{|\textit{child}|}|
from within the main file
(or at least for those files to be compiled individually).
The argument \textit{main} must be the filename of the main file.

There are a couple of
considerations in setting up the main and child documents:

%%%%%%%%%%%%%%%%%%%%%%%%%%%%%%%%%%%%%%%%
\paragraph{Restrictions.}

Please note the following restrictions:
\begin{itemize}
\item
|\childdocmain| must be called with one argument \textit{main}
to ensure compatibility with earlier version of the package.
It must either be empty (|\childdocmain{}|)
or precisely match the filename of the main file in which it is specified.
See \secref{sec:detection} for further information.
\item
The filename \textit{main} must be specified without the |.tex| extension.
\item
The filename \textit{main} is case sensitive
(even in case-insensitive file systems)
due to internal string comparison.
\item
The argument \textit{main} should be fully expanded, it cannot be a macro.
\item
Subdirectories and special characters should be avoided in filenames.
\item
The command |\childdocmain{|\textit{main}|}| must be followed by a whitespace.
It should not be followed immediately by another command
or by a comment mark `|%|'.
This is because the \TeX{} parser reads the token immediately following
the argument of |\childdocmain| and puts it
at the beginning of every child section;
however, a white\-space is ignored.
\end{itemize}

%%%%%%%%%%%%%%%%%%%%%%%%%%%%%%%%%%%%%%%%
\paragraph{Content of Main File.}

It is advisable to place all content in the child files included by |\include|.
Any output contained in the main file will appear in all child documents
unless suppressed manually;
it cannot be suppressed automatically by the |\includeonly| directive
and thus should normally be avoided.
A method to include some content in the main file
by means of conditional processing is described in \secref{sec:conditional}.

%%%%%%%%%%%%%%%%%%%%%%%%%%%%%%%%%%%%%%%%
\paragraph{Page Numbering.}

When only a part of the document is compiled,
the appropriate numbering of pages
(as well as other status parameters)
is determined from the |.aux| files.
The latter contain information from previous passes.
However this information needs to propagate through
all intermediate child documents.
Therefore the page numbering in child documents may well
be inconsistent until the complete document is compiled at least once.

A useful (if unconventional) way to always ensure a consistent
page numbering is to restart the numbering in each child document
and denote the pages by `\textit{child}|.|\textit{page}'
where \textit{child} represents the chapter/section number of the child file.
This can be achieved by the command
|\numberwithin{page}{|\textit{child}|}|
of the \textsf{amsmath} package
where \textit{child} can be |chapter| or |section|
depending on the chosen structuring.
Alternatively, one can modify the macro |\thepage| appropriately
and reset the counter |page| at the start of each child file.

%%%%%%%%%%%%%%%%%%%%%%%%%%%%%%%%%%%%%%%%%%%%%%%%%%%%%%%%%%%%%%%%%%%%%%%%%%%%%%%%
\subsection{Conditional Processing}
\label{sec:conditional}

The package provides a mechanism to compile different versions
of a document. To customise the versions further some conditional processing
can come in handy to distinguish which version is being compiled.
The package provides two macros to describe the compilation context:

%%%%%%%%%%%%%%%%%%%%%%%%%%%%%%%%%%%%%%%%
\DescribeMacro{\ifchilddoc}
The conditional |\ifchilddoc| distinguishes between the compilation of
child documents and the main document:
%
\begin{center}
|\ifchilddoc |\textit{child-code}| |[|\||else |\textit{main-code}]| \||fi|
\end{center}

%%%%%%%%%%%%%%%%%%%%%%%%%%%%%%%%%%%%%%%%
\DescribeMacro{\childdocname}
\DescribeMacro{\childdocjob}
The macro |\childdocname| contains the filename (without extension)
of the main or child file being processed.
Note that |\childdocjob| will always contain the name of the main file.

%%%%%%%%%%%%%%%%%%%%%%%%%%%%%%%%%%%%%%%%
\paragraph{Title Page.}

Conditional processing can be used to include a title or banner page
in the main document when proper precautions are taken.
Importantly, the code in the main file should ensure that the page counter
(as well as other status parameters which are stored in the |.aux| files)
takes the same value after the conditional processing.
Otherwise the page numbers may take divergent values
depending on which part is compiled.

For example, a title page could be declared by:
%
\begin{center}
\begin{tabular}{l}
|\ifchilddoc\||else|\\
|\addtocounter{page}{-1}|\\
\textit{code for title page}\\
|\newpage|\\
|\||fi|
\end{tabular}
\end{center}
%
A banner page for the child documents can be generated by:
%
\begin{center}
\begin{tabular}{l}
|\ifchilddoc|\\
|\addtocounter{page}{-1}|\\
\textit{code for banner page}\\
|\newpage|\\
|\||fi|
\end{tabular}
\end{center}
%
Here one could write a message such as:
\begin{center}
|This is the part \childdocname{} of \childdocjob{}.|
\end{center}

%%%%%%%%%%%%%%%%%%%%%%%%%%%%%%%%%%%%%%%%%%%%%%%%%%%%%%%%%%%%%%%%%%%%%%%%%%%%%%%%
\subsection{Flags}
\label{sec:flags}

The package makes it easy to generate different versions
of the main or child documents.
To this end compilation flags can be defined
and assigned different default values.
They will be particularly useful in conjunction
with the forwarding mechanism described in \secref{sec:forward}.

For example, it may be useful to have a flag |\version|
which can be set to |draft| or |final|.
The document source will contain some conditional code
depending on the value of |\version|.
Suppose further, the flag should default to |final| for the main file
and to |draft| for child files
which is a natural assignment for editing the document.
This is achieved by placing the following code
in the preamble of the main document
(below the |\childdocmain| directive):
%
\begin{center}
\begin{tabular}{l}
|\ifchilddoc|\\
|\providecommand{\version}{draft}|\\
|\||else|\\
|\providecommand{\version}{final}|\\
|\||fi|
\end{tabular}
\end{center}
%
The definition by |\providecommand| makes sure
that previous definitions are not overwritten.
Further statements |\providecommand{\version}{...}|
can thus be added before the above code to override it.

For the main file, one might add a line
(between |\childdocmain| and the above block)
%
\begin{center}
|%\ifchilddoc\||else\providecommand{\version}{draft}\||fi|
\end{center}
%
which can be uncommented to produce a draft version.
Likewise one can add a line to the very top of a child file
(above the |\childdocof{|\textit{main}|}| directive)
%
\begin{center}
|%\providecommand{\version}{final}|
\end{center}
%
which can be uncommented to produce the final version of this child document.

%%%%%%%%%%%%%%%%%%%%%%%%%%%%%%%%%%%%%%%%%%%%%%%%%%%%%%%%%%%%%%%%%%%%%%%%%%%%%%%%
\subsection{Forwarding}
\label{sec:forward}

Different versions of the main or child documents
using compilation flags as described in \secref{sec:flags}
can be (permanently) stored in different files
for convenient compilation, viewing and distribution.
To this end, the package defines a command
to pass on compilation to a different file:

%%%%%%%%%%%%%%%%%%%%%%%%%%%%%%%%%%%%%%%%
\DescribeMacro{\childdocforward}
The command |\childdocforward| redirects processing to
another source file:
%
\begin{center}
\begin{tabular}{l}
|\input{childdoc.def}|\\
|\childdocforward[|\textit{main}|]{|\textit{dest}|}|\\
\end{tabular}
\end{center}
%
The argument \textit{dest} is the destination file
(without extension).
It should be the main file or one of the child files.
Note that further \textsf{childdoc} directives
such as |\childdocof| and |\childdocforward|
in the indicated file will be processed in this form.
The optional argument \textit{main}
passes on directly to the main file \textit{main}
while pretending to compile the child \textit{dest}.
This form behaves as if \textit{dest}
issues |\childdocof{|\textit{main}|}| right away,
and no further \textsf{childdoc} directives will be processed.

%%%%%%%%%%%%%%%%%%%%%%%%%%%%%%%%%%%%%%%%
\DescribeMacro{\...prefix}
In the alternative form |\childdocforwardprefix|,
%
\begin{center}
\begin{tabular}{l}
|\input{childdoc.def}|\\
|\childdocforwardprefix[|\textit{main}|]{|\textit{prefix}|}{|\textit{dest}|}|
\end{tabular}
\end{center}
%
the destination file is determined by a pattern
depending on the current file:
To make this work, the current file must be called
`{\textit{prefix}\hspace{0.2em}\textit{suffix}}'
with \textit{prefix} matching precisely the argument.
Processing is then passed on to the file
`{\textit{dest}\hspace{0.2em}\textit{suffix}}'.
Surely, the same effect is achieved by
directly specifying the
argument `{\textit{dest}\hspace{0.2em}\textit{suffix}}'
in the first form.
However, that requires to set up a different file
for each child. With the alternative form of the command
all these files can have exactly the same content
which simplifies setting them up and maintaining them.

For example, the following file |draft.tex|
with a compilation flag |\version| as described in \secref{sec:flags}
compiles the main document as a draft:
%
\begin{center}
\begin{tabular}{l}
|\def\version{draft}|\\
|\input{childdoc.def}|\\
|\childdocforward{|\textit{main}|}|
\end{tabular}
\end{center}
%
Likewise, the following files |final|\textit{nn}|.tex|
compile the final version of the child document
|child|\textit{nn}|.tex|:
%
\begin{center}
\begin{tabular}{l}
|\def\version{final}|\\
|\input{childdoc.def}|\\
|\childdocforwardprefix{final}{child}|
\end{tabular}
\end{center}
%

Note that when several versions of a main file and/or of each child file
are to be generated, it may be convenient to set up a |Makefile| or
shell script to automatise the process.

%%%%%%%%%%%%%%%%%%%%%%%%%%%%%%%%%%%%%%%%%%%%%%%%%%%%%%%%%%%%%%%%%%%%%%%%%%%%%%%%
\subsection{Command Line Processing}
\label{sec:commandline}

The effect of redirection files can also be achieved by invoking
the \LaTeX{} compiler with a more elaborate command line.
Most conveniently this should be done as part
of a shell script or a |Makefile|.

When using \textsf{childdoc} in the main file, the following
command lines effectively perform a redirection
(note that depending on the shell being used,
backslashes may have to be doubled: `|\|' $\to$ `|\\|'):
%
\begin{center}
|... -jobname "|\textit{target}|" |\\|"|[\textit{flags}]%
|\input{childdoc.def}\childdocforward[|\textit{main}|]{|\textit{dest}|}"|
\end{center}
%
Here \textit{target} is the name of the output file,
\textit{main} is the name of the main file
and \textit{dest} is the name of the main or child file to be processed
(all filenames without extensions).
The optional argument \textit{main} can be omitted
if \textit{main} matches \textit{dest}.
Optionally, compilation \textit{flags} can be defined via |\def| commands.
This command line makes the \TeX{} engine believe
it is compiling the file \textit{target}
whose content is specified as the latter parameter.
The provided code then forwards the processing to
\textit{main} or \textit{dest} as described in \secref{sec:forward}.

%%%%%%%%%%%%%%%%%%%%%%%%%%%%%%%%%%%%%%%%%%%%%%%%%%%%%%%%%%%%%%%%%%%%%%%%%%%%%%%%
\subsection{Include by Input}
\label{sec:input}

Including child documents by |\include| has some restrictions by design.
Most notably, the content of a child document always occupies
its own set of pages; pages cannot be shared between child documents.
Usually, this behaviour makes perfect sense
because each child document contain an essential part of the document.
However, in some situations it may be desirable to compose
a document from a collection of parts
without having mandatory page breaks between then.
For this case, the package
provides a mechanism to include parts
by |\input| which can also be processed individually.
However, by construction this mechanism
requires manual handling of the content to be output.

%%%%%%%%%%%%%%%%%%%%%%%%%%%%%%%%%%%%%%%%
\DescribeMacro{\ifchilddocmanual}
The main file should be prepared as usual, see \secref{sec:include}.
However, the document body must make a distinction
between processing of an individual part and of the main document, e.g.:
%
\begin{center}
\begin{tabular}{l}
|\ifchilddocmanual|\\
|\input{\childdocname}|\\
|\||else|\\
\textit{document body with }|\input{|\textit{part}|}|\\
|\||fi|
\end{tabular}
\end{center}
%
The conditional |\ifchilddocmanual| is true whenever
a part to be included by |\input| is being compiled,
and the name of the part is stored in |\childdocname|.

%%%%%%%%%%%%%%%%%%%%%%%%%%%%%%%%%%%%%%%%
\DescribeMacro{\childdocby}
Each part to be included by |\input| should start with:
%
\begin{center}
\begin{tabular}{l}
|\input{childdoc.def}|\\
|\childdocby{|\textit{main}|}|\\
\end{tabular}
\end{center}
%
The directive |\childdocby| is similar to |\childdocof|
described in \secref{sec:include},
but the subsequent selection of content must be done manually.
To that end, both |\ifchilddoc| and |\ifchilddocmanual|
will be true upon processing of a part,
and the name of the part is stored in |\childdocname|.
Note that |\jobname| will be set to the filename of the current part
so that each part receives an individual |.aux| file
that does not interfere with the |.aux| file(s) of the main document.
This behaviour can be altered by the alternative form
|\childdocby[*]{|\textit{main}|}| (with a non-empty optional argument)
which uses the |.aux| file of the main document
by setting |\jobname| to \textit{main}.

%%%%%%%%%%%%%%%%%%%%%%%%%%%%%%%%%%%%%%%%%%%%%%%%%%%%%%%%%%%%%%%%%%%%%%%%%%%%%%%%
\subsection{Driver Development}
\label{sec:driver}

The \textsf{childdoc} mechanism can also be use for the development
of definition files such as \LaTeX{} styles or classes.
This case differs from the above setup with multiple parts
included by |\include| in that no |\includeonly| should be invoked.
This can be achieved by starting the include file
(before |\ProvidesPackage|) with:
%
\begin{center}
\begin{tabular}{l}
|\input{childdoc.def}|\\
|\childdocforward{|\textit{main}|}|\\
\end{tabular}
\end{center}
%
or alternatively with:
%
\begin{center}
\begin{tabular}{l}
|\input{childdoc.def}|\\
|\childdocby{|\textit{main}|}|\\
\end{tabular}
\end{center}
%
Both forms have slightly different effects as described above.
The main file is prepared as usual, see \secref{sec:include}.

%%%%%%%%%%%%%%%%%%%%%%%%%%%%%%%%%%%%%%%%%%%%%%%%%%%%%%%%%%%%%%%%%%%%%%%%%%%%%%%%
\subsection{Legacy Detection}
\label{sec:detection}

The directive |\childdocmain| in the main file can detect
whether the complete document or merely a child is to be compiled
even without using the directive |\childdocof|.
This method is deprecated because it is less robust
and there is no compelling reason to use it;
it is merely provided for backward compatibility
and it may be removed in future versions.

If the detection mechanism is to be used,
it is mandatory to correctly specify
the filename of the main file as the argument of |\childdocmain|:
%
\begin{center}
\begin{tabular}{l}
|\input{childdoc.def}|\\
|\childdocmain{|\textit{main}|}|\\
\end{tabular}
\end{center}
%
If |\jobname| does not match the argument \textit{main} of |\childdocmain|,
it is assumed that |\jobname| points to the child file to be compiled.
When using |\childdocmain| with the main file specified as argument,
it suffices to start a child file
with just |\input{|\textit{main}|}|
without loading of the package and using |\childdocof|.
If instead all processing is done
with the appropriate \textsf{childdoc} directives,
the argument of \textit{main} of |\childdocmain| can be empty.

An alternative version of the command line processing described
in \secref{sec:commandline} using the detection mechanism reads:
%
\begin{center}
|... -jobname "|\textit{target}|" "|[\textit{flags}]%
[|\def\jobname{|\textit{dest}|}|]|\input{|\textit{main}|}"|
\end{center}

%%%%%%%%%%%%%%%%%%%%%%%%%%%%%%%%%%%%%%%%%%%%%%%%%%%%%%%%%%%%%%%%%%%%%%%%%%%%%%%%
\subsection{Manual Code}
\label{sec:manual}

In case one cannot be certain whether the definitions file |childdoc.def|
is installed on the target \TeX{} distribution
and one prefers not to ship it,
it is conceivable to paste a few relevant commands into the sources.

To that end, drop all statements |\input{childdoc.def}|
and perform the replacements as outlined below.
Instead of |\childdocmain{|\textit{main}|}| add the following code
to the top of the main file:
%
\begin{center}
\begin{tabular}{l}
|\||ifdefined\childdocname\endinput\||fi\newif\ifchilddoc|\\
|\edef\childdocname{\scantokens\expandafter{\jobname\noexpand}}|\\
|\def\childdocmain{|\textit{main}|}\||ifx\childdocmain\childdocname\||else|\\
|\childdoctrue\includeonly{\childdocname}\let\jobname\childdocmain\||fi|\\
\end{tabular}
\end{center}
%
Instead of |\childdocof{|\textit{main}|}| just include the main file
at the top of each child file:
%
\begin{center}
|\input{|\textit{main}|}|
\end{center}
%
A simple redirection |\childdocforward{|\textit{dest}|}| is achieved by:
%
\begin{center}
|\def\jobname{|\textit{dest}|}\input{\jobname}|
\end{center}
%
The redirection with prefix
|\childdocforwardprefix[|\textit{prefix}|]{|\textit{dest}|}|
is accomplished by:
%
\begin{center}
\begin{tabular}{l}
|{\edef\jobname{\scantokens\expandafter{\jobname\noexpand}}|\\
|\def\redirectjob |\textit{prefix}|#1~~~{\gdef\jobname{|\textit{dest}|#1}}|\\
|\expandafter\redirectjob\jobname~~~}\input{\jobname}|
\end{tabular}
\end{center}

In an alternative approach,
child documents can be compiled by a specific command line
without additional code or specific definitions:
%
\begin{center}
|... -jobname "|\textit{target}|" "|[\textit{flags}]%
|\includeonly{|\textit{dest}|}\input{|\textit{main}|}"|
\end{center}
%

%%%%%%%%%%%%%%%%%%%%%%%%%%%%%%%%%%%%%%%%%%%%%%%%%%%%%%%%%%%%%%%%%%%%%%%%%%%%%%%%
%%%%%%%%%%%%%%%%%%%%%%%%%%%%%%%%%%%%%%%%%%%%%%%%%%%%%%%%%%%%%%%%%%%%%%%%%%%%%%%%
\section{Information}

%%%%%%%%%%%%%%%%%%%%%%%%%%%%%%%%%%%%%%%%%%%%%%%%%%%%%%%%%%%%%%%%%%%%%%%%%%%%%%%%
\subsection{Copyright}

Copyright \copyright{} 2017--2018 Niklas Beisert

This work may be distributed and/or modified under the
conditions of the \LaTeX{} Project Public License, either version 1.3
of this license or (at your option) any later version.
The latest version of this license is in
  \url{http://www.latex-project.org/lppl.txt}
and version 1.3 or later is part of all distributions of \LaTeX{}
version 2005/12/01 or later.

This work has the LPPL maintenance status `maintained'.

The Current Maintainer of this work is Niklas Beisert.

This work consists of the files |README.txt|, |childdoc.ins| and |childdoc.dtx|
as well as the derived files |childdoc.def|, |cdocsamp.tex|
with |cdocsch1.tex|, |cdocsch2.tex|, |cdocspt3.tex|, |cdocspt4.tex|,
|cdocsdrf.tex|, |cdocsfn1.tex|, |cdocsfn2.tex|
as well as |childdoc.pdf|.

%%%%%%%%%%%%%%%%%%%%%%%%%%%%%%%%%%%%%%%%%%%%%%%%%%%%%%%%%%%%%%%%%%%%%%%%%%%%%%%%
\subsection{Files and Installation}

The package consists of the files:
%
\begin{center}
\begin{tabular}{ll}
    |README.txt|   & readme file \\
    |childdoc.ins| & installation file \\
    |childdoc.dtx| & source file \\
    |childdoc.def| & definition file \\
    |cdocsamp.tex| & sample main file \\
    |cdocsch1.tex| & sample include file \\
    |cdocsch2.tex| & sample include file \\
    |cdocspt3.tex| & sample part file \\
    |cdocspt4.tex| & sample part file \\
    |cdocsdrf.tex| & sample redirection file \\
    |cdocsfn1.tex| & sample redirection file \\
    |cdocsfn2.tex| & sample redirection file \\
    |childdoc.pdf| & manual
\end{tabular}
\end{center}
%
The distribution consists of the files
|README.txt|, |childdoc.ins| and |childdoc.dtx|.
%
\begin{itemize}
\item
Run (pdf)\LaTeX{} on |childdoc.dtx|
to compile the manual |childdoc.pdf| (this file).
\item
Run \LaTeX{} on |childdoc.ins| to create the definitions file |childdoc.def|
and the sample |cdocsamp.tex| with include files
|cdocsch1.tex|, |cdocsch2.tex|, |cdocspt3.tex|, |cdocspt4.tex|,
|cdocsdrf.tex|, |cdocsfn1.tex|, |cdocsfn2.tex|.
Then copy the file |childdoc.def| to an appropriate directory of your \LaTeX{}
distribution, e.g.\ \textit{texmf-root}|/tex/latex/childdoc|.
\end{itemize}

%%%%%%%%%%%%%%%%%%%%%%%%%%%%%%%%%%%%%%%%%%%%%%%%%%%%%%%%%%%%%%%%%%%%%%%%%%%%%%%%
\subsection{Related CTAN Packages}

There are several other packages which offer a similar functionality:
%
\begin{itemize}
\item
The packages
\href{http://ctan.org/pkg/docmute}{\textsf{docmute}},
\href{http://ctan.org/pkg/includex}{\textsf{includex}} and
\href{http://ctan.org/pkg/standalone}{\textsf{standalone}}
provide commands to include only the document body of
a child file thus allowing both files to be compiled individually.
\item
The packages \href{http://ctan.org/pkg/subdocs}{\textsf{subdocs}}
and \href{http://ctan.org/pkg/subfiles}{\textsf{subfiles}}
provide structures in which the main and child documents can be
encapsulated and allowing them to be compiled individually.
The inclusion mechanism is different from the conventional |\include|.
\item
The package \href{http://ctan.org/pkg/combine}{\textsf{combine}}
is an elaborate solution to combine several documents into one.
\end{itemize}
%
See also the CTAN topic \href{http://ctan.org/topic/subdocs}{\textsf{subdocs}}
for further related packages.
The present package differs from the above solutions in that
a document structure constructed with the conventional |\include| mechanism
just needs two extra commands at the top of every file
such that all constituent files can be compiled individually.

%%%%%%%%%%%%%%%%%%%%%%%%%%%%%%%%%%%%%%%%%%%%%%%%%%%%%%%%%%%%%%%%%%%%%%%%%%%%%%%%
%\subsection{Feature Suggestions}
%
%The following is a list of features which may be useful for future
%versions of this package:
%%
%\begin{itemize}
%\item
%\ldots
%\end{itemize}

%%%%%%%%%%%%%%%%%%%%%%%%%%%%%%%%%%%%%%%%%%%%%%%%%%%%%%%%%%%%%%%%%%%%%%%%%%%%%%%%
\subsection{Revision History}

%%%%%%%%%%%%%%%%%%%%%%%%%%%%%%%%%%%%%%%%
\paragraph{v2.0:} 2018/12/30

\begin{itemize}
\item
immediate forward processing
\item
added |\childdocby| mechanism
\item
manual restructured
\end{itemize}

%%%%%%%%%%%%%%%%%%%%%%%%%%%%%%%%%%%%%%%%
\paragraph{v1.6:} 2018/01/17

\begin{itemize}
\item
application for development of include files
\item
corrections to manual
\end{itemize}

%%%%%%%%%%%%%%%%%%%%%%%%%%%%%%%%%%%%%%%%
\paragraph{v1.5:} 2017/05/21

\begin{itemize}
\item
more complete structuring introduced
\item
|\childdocof| introduced
\item
|\childdoc| renamed to |\childdocmain|
\item
|\childredirect| renamed to |\childdocforward| and |\childdocforwardprefix|
and functionality expanded
\end{itemize}

%%%%%%%%%%%%%%%%%%%%%%%%%%%%%%%%%%%%%%%%
\paragraph{v1.0:} 2017/04/27

\begin{itemize}
\item
manual and install package
\item
first version published on CTAN
\end{itemize}

%%%%%%%%%%%%%%%%%%%%%%%%%%%%%%%%%%%%%%%%
\paragraph{v0.6:} 2017/04/26

\begin{itemize}
\item
redirection mechanism added
\end{itemize}

%%%%%%%%%%%%%%%%%%%%%%%%%%%%%%%%%%%%%%%%
\paragraph{v0.5:} 2017/04/26

\begin{itemize}
\item
functionality in definition file
\end{itemize}


%%%%%%%%%%%%%%%%%%%%%%%%%%%%%%%%%%%%%%%%%%%%%%%%%%%%%%%%%%%%%%%%%%%%%%%%%%%%%%%%
%%%%%%%%%%%%%%%%%%%%%%%%%%%%%%%%%%%%%%%%%%%%%%%%%%%%%%%%%%%%%%%%%%%%%%%%%%%%%%%%
%%%%%%%%%%%%%%%%%%%%%%%%%%%%%%%%%%%%%%%%%%%%%%%%%%%%%%%%%%%%%%%%%%%%%%%%%%%%%%%%
\appendix

\settowidth\MacroIndent{\rmfamily\scriptsize 000\ }

 \DocInput{childdoc.dtx}

\end{document}
%</driver>
% \fi
%
% %%%%%%%%%%%%%%%%%%%%%%%%%%%%%%%%%%%%%%%%%%%%%%%%%%%%%%%%%%%%%%%%%%%%%%%%%%%%%%
% %%%%%%%%%%%%%%%%%%%%%%%%%%%%%%%%%%%%%%%%%%%%%%%%%%%%%%%%%%%%%%%%%%%%%%%%%%%%%%
% \section{Sample}
%\iffalse
%<*samplemain>
%\fi
%
% The following presents a sample document
% with two chapters, two parts, a title page,
% a compile flag as well as three forwarding files to set the flag.
% It consists of eight |.tex| files:
% \begin{center}
% \begin{tabular}{ll}
% |cdocsamp.tex|&main file\\
% |cdocsch1.tex|&include file for chapter 1\\
% |cdocsch2.tex|&include file for chapter 2\\
% |cdocspt3.tex|&include file for part 3\\
% |cdocspt4.tex|&include file for part 4\\
% |cdocsdrf.tex|&forwarding file for main file in draft mode\\
% |cdocsfi1.tex|&forwarding file for final version of chapter 1\\
% |cdocsfi2.tex|&forwarding file for final version of chapter 2\\
% \end{tabular}
% \end{center}
% Each of the eight files can be compiled directly by the \LaTeX{} compiler.
%
% %%%%%%%%%%%%%%%%%%%%%%%%%%%%%%%%%%%%%%
% \paragraph{Main File.}
%
% The main file is called |cdocsamp.tex|.
%
% Load the \textsf{childdoc} definitions and
% declare the filename for the main document:
%    \begin{macrocode}
\input{childdoc.def}
\childdocmain{}
%    \end{macrocode}

% Optional override for |\version| flag:
%    \begin{macrocode}
%%\ifchilddoc\else\providecommand{\version}{draft}\fi
%    \end{macrocode}

% Define the default values for the |\version| flag
% (|final| for the main file and |draft| for childs):
%    \begin{macrocode}
\ifchilddoc
\providecommand{\version}{draft}
\else
\providecommand{\version}{final}
\fi
%    \end{macrocode}

% Load the standard document class:
%    \begin{macrocode}
\documentclass[12pt]{article}
%    \end{macrocode}

% Start the document body:
%    \begin{macrocode}
\begin{document}
%    \end{macrocode}

% Declare a title page.
% Print title, part of document being processed and version flag:
%    \begin{macrocode}
\addtocounter{page}{-1}
\begin{center}
{\LARGE\bfseries{}childdoc example\par}
\vspace{1cm}
\ifchilddoc
\ifchilddocmanual part\else chapter\fi:
`\childdocname' of `\childdocjob'\par
\else
main document: `\childdocjob'\par
\fi
version: \version\par
\end{center}
\newpage
%    \end{macrocode}

% Manually include selected file,
% otherwise process as usual:
%    \begin{macrocode}
\ifchilddocmanual
\section*{part `\childdocname'}
\input{\childdocname}
\else
%    \end{macrocode}

% Include the two chapters:
%    \begin{macrocode}
\include{cdocsch1}
\include{cdocsch2}
%    \end{macrocode}

% Include the two parts unless only chapters should be displayed:
%    \begin{macrocode}
\ifchilddoc\else
\section{part three}
\input{cdocspt3}
\section{part four}
\input{cdocspt4}
\fi
%    \end{macrocode}

% Process as usual until here:
%    \begin{macrocode}
\fi
%    \end{macrocode}

% End of document body:
%    \begin{macrocode}
\end{document}
%    \end{macrocode}
%\iffalse
%</samplemain>
%\fi
%
% %%%%%%%%%%%%%%%%%%%%%%%%%%%%%%%%%%%%%%
% \paragraph{Chapter Include Files.}
%
% The include files are called |cdocsch1.tex| and |cdocsch2.tex|.
%
%\iffalse
%<*samplechap1|samplechap2>
%\fi

% Optional override for |\version| flag:
%    \begin{macrocode}
%%\providecommand{\version}{final}
%    \end{macrocode}

% Include the main document:
%    \begin{macrocode}
\input{childdoc.def}
\childdocof{cdocsamp}
%    \end{macrocode}

%\iffalse
%</samplechap1|samplechap2>
%\fi
%
%\iffalse
%<*samplechap1>
%\fi
% Some text for chapter 1:
%    \begin{macrocode}
\section{one}
some text in chapter one
%    \end{macrocode}

%\iffalse
%</samplechap1>
%\fi
% Some text for chapter 2:
%\iffalse
%<*samplechap2>
%\fi
%    \begin{macrocode}
\section{two}
more text in chapter two
%    \end{macrocode}

%\iffalse
%</samplechap2>
%\fi
%
% %%%%%%%%%%%%%%%%%%%%%%%%%%%%%%%%%%%%%%
% \paragraph{Part Include Files.}
%
% The include files are called |cdocspt3.tex| and |cdocspt4.tex|.
%
%\iffalse
%<*samplepart3|samplepart4>
%\fi

% Optional override for |\version| flag:
%    \begin{macrocode}
%%\providecommand{\version}{final}
%    \end{macrocode}

% Include the main document:
%    \begin{macrocode}
\input{childdoc.def}
\childdocby{cdocsamp}
%    \end{macrocode}

%\iffalse
%</samplepart3|samplepart4>
%\fi
%
%\iffalse
%<*samplepart3>
%\fi
% Some text for part 3:
%    \begin{macrocode}
some text in part three
%    \end{macrocode}

%\iffalse
%</samplepart3>
%\fi
% Some text for part 4:
%\iffalse
%<*samplepart4>
%\fi
%    \begin{macrocode}
more text in part four
%    \end{macrocode}

%\iffalse
%</samplepart4>
%\fi
%
% %%%%%%%%%%%%%%%%%%%%%%%%%%%%%%%%%%%%%%
% \paragraph{Forwarding for a Complete Draft.}
%
% The following forwarding file |cdocsdrf.tex|
% compiles the main document in draft mode:
%\iffalse
%<*sampledraft>
%\fi
%    \begin{macrocode}
\def\version{draft}
\input{childdoc.def}
\childdocforward{cdocsamp}
%    \end{macrocode}

%\iffalse
%</sampledraft>
%\fi
%
% %%%%%%%%%%%%%%%%%%%%%%%%%%%%%%%%%%%%%%
% \paragraph{Forwarding for Final Version of the Chapters.}
%
% The following forwarding files |cdocsfn1.tex| and |cdocsfn2.tex|
% (with identical content)
% compile the final versions of the child documents
% |cdocsch1.tex| and |cdocsch2.tex|, respectively:
%\iffalse
%<*samplefinal>
%\fi
%    \begin{macrocode}
\def\version{final}
\input{childdoc.def}
\childdocforwardprefix[cdocsamp]{cdocsfn}{cdocsch}
%    \end{macrocode}

%\iffalse
%</samplefinal>
%\fi
%
% %%%%%%%%%%%%%%%%%%%%%%%%%%%%%%%%%%%%%%
% \paragraph{Command Line Processing.}
%
% The following three command lines generate the output files
% |cdocscld|, |cdocscl1| and |cdocscl2|
% which should be identical to
% |cdocsdrf|, |cdocsch1| and |cdocsfn2|, respectively:
% \begin{center}
% \begin{tabular}{l}
% |latex -jobname cdocscld \|\\
% |  "\def\version{draft}\input{childdoc.def}\childdocforward{cdocsamp}"|\\
% |latex -jobname cdocscl1 \|\\
% |  "\input{childdoc.def}\childdocforward[cdocsamp]{cdocsch1}"|\\
% |latex -jobname cdocscl2 \|\\
% |  "\def\version{final}\input{childdoc.def}\childdocforward{cdocsch2}"|
% \end{tabular}
% \end{center}
% Note that the trailing backslash on each first line
% merely continues the input to the second line
% (for convenient cut ant paste).
% Furthermore, the command |latex| can be replaced by any
% of its alternative versions such as |pdflatex|.
%
% %%%%%%%%%%%%%%%%%%%%%%%%%%%%%%%%%%%%%%%%%%%%%%%%%%%%%%%%%%%%%%%%%%%%%%%%%%%%%%
% %%%%%%%%%%%%%%%%%%%%%%%%%%%%%%%%%%%%%%%%%%%%%%%%%%%%%%%%%%%%%%%%%%%%%%%%%%%%%%
% \section{Implementation}
%\iffalse
%<*package>
%\fi
%
% This section describes the definitions file |childdoc.def|.

% The definitions cannot be loaded using |\usepackage| or |\RequirePackage|
% which has a mechanism to prevent loading a style file more than once.
% When loading the definitions by means of |\input|
% multiple instances have to be prevented manually:
%\iffalse
%This code needs to be before the `\ProvidesFile' directive
%which is defined at the beginning of this file.
%Therefore it is also placed there and commented out here.
%</package>
%<*discard>
%\fi
%    \begin{macrocode}
\ifdefined\childdocmain\endinput\fi
%    \end{macrocode}
%\iffalse
%</discard>
%<*package>
%\fi
%
% \macro{\ifchilddoc}
% \macro{\ifchilddocmanual}
% The conditional |\ifchilddoc| tells whether a
% child (true) or main (false) document is being compiled.
% The conditional |\ifchilddocmanual| tells whether
% the |\includeonly| mechanism is used (false) or
% the selection of child files must be performed manually (true).
% The definitions initialise to false:
%    \begin{macrocode}
\newif\ifchilddoc
\newif\ifchilddocmanual
%    \end{macrocode}

% \macro{\childdocname}
% \macro{\childdocjob}
% The macro |\childdocname| stores the name of the main document
% to be compiled. The macro |\childdocjob| stores the name of
% the document on which the \LaTeX{} compiler was originally invoked.
% The content of |\jobname| cannot be compared
% to filenames specified in the source due to different catcodes.
% The following code rescans |\jobname|, stores the result
% in |\childdocname| and saves a copy in |\childdocjob|:
%    \begin{macrocode}
\edef\childdocname{\scantokens\expandafter{\jobname\noexpand}}
\let\childdocjob\childdocname
%    \end{macrocode}

% \macro{\childdocdisable}
% The macro |\childdocdisable| prevents the main file
% from being processed more than once.
% At this stage, the main document command |\childdocmain|
% is assumed to be called once again where it should do nothing.
% Any subsequent call to it should prevent
% a secondary processing of the main document
% It overwrites the forwarding commands
% |\childdocof| and |\childdocforward|
% with empty macros to prevent further inclusions of the main document:
%    \begin{macrocode}
\newcommand{\childdocdisable}
{
  \renewcommand{\childdocmain}[1]{\renewcommand{\childdocmain}[1]{\endinput}}
  \renewcommand{\childdocof}[1]{}
  \renewcommand{\childdocby}[2][]{}
  \renewcommand{\childdocforward}[2][]{}
  \renewcommand{\childdocdisable}{}
}
%    \end{macrocode}

% \macro{\childdocmain}
% The macro |\childdocmain| is to be called at the top of the main file
% with nothing or the main filename (without extension) as argument.
% First, it breaks loops.
% If the argument is not empty and does not match |\childdocname|
% (which is set by the first inclusion of |childdoc.def|),
% |\ifchilddoc| is set to true, |\includeonly| is applied to the child file
% and |\jobname| is set to the main file
% (for proper handling of |.aux| files):
%    \begin{macrocode}
\newcommand{\childdocmain}[1]
{
  \childdocdisable\childdocmain{}
  \if?#1?\else
    \begingroup
      \def\childdoctmp{#1}
      \ifx\childdoctmp\childdocname
        \def\childdoctmp{}
      \else
        \def\childdoctmp
        {
          \childdoctrue
          \includeonly{\childdocname}
          \def\childdocjob{#1}
          \def\jobname{#1}
        }
      \fi
      \expandafter
    \endgroup
    \childdoctmp
  \fi
}
%    \end{macrocode}

% \macro{\childdocof}
% The command |\childdocof| redirects
% compilation to the main file |#1|.
%    \begin{macrocode}
\newcommand{\childdocof}[1]
{
  \childdocdisable
  \childdoctrue
  \includeonly{\childdocname}
  \def\jobname{#1}
  \def\childdocjob{#1}
  \input{#1}
}
%    \end{macrocode}

% \macro{\childdocby}
% The command |\childdocby| ....
%    \begin{macrocode}
\newcommand{\childdocby}[2][]
{
  \childdocdisable
  \childdoctrue
  \childdocmanualtrue
  \if?#1?\else
    \def\jobname{#2}
  \fi
  \def\childdocjob{#2}
  \input{#2}
  \endinput
}
%    \end{macrocode}

% \macro{\childdocforward}
% The command |\childdocforward| redirects
% compilation to the main file or
% (if the optional argument is given) a child file.
% Parameters are set as if the main file
% or a child file starting with |\childdocof| was compiled.
% Then compilation is handed over to the main file:
%    \begin{macrocode}
\newcommand{\childdocforward}[2][]
{
  \begingroup
    \if?#1?
      \def\childdoctmp
      {
        \def\childdocname{#2}
        \def\childdocjob{#2}
        \def\jobname{#2}
        \input{#2}
        \endinput
      }
    \else
      \def\childdoctmp
      {
        \childdocdisable
        \def\childdocname{#2}
        \childdoctrue
        \includeonly{#2}
        \def\childdocjob{#1}
        \def\jobname{#1}
        \input{#1}
        \endinput
      }
    \fi
    \expandafter
  \endgroup
  \childdoctmp
}
%    \end{macrocode}

% \macro{\childdocforwardprefix}
% The command |\childdocforwardprefix| redirects
% compilation to the main or a child file by means of a pattern.
% The prefix |#1| in the current filename is replaced by |#2|
% and the suffix of the current filename is kept
% (it is assumed that the filename does not contain the substring `|~~~|'
% which is used as a delimiter).
% Compilation is handed over to the new file by |\childdocforward|:
%    \begin{macrocode}
\newcommand{\childdocforwardprefix}[3][]
{
  \begingroup
    \def\childdocextract #2##1~~~{\def\childdoctmp{\childdocforward[#1]{#3##1}}}
    \expandafter\childdocextract\childdocname~~~
    \expandafter
  \endgroup
  \childdoctmp
}
%    \end{macrocode}

% \macro{\childdoc}
% The deprecated macro |\childdoc| is a legacy version of |\childdocmain|:
%    \begin{macrocode}
\newcommand{\childdoc}{\childdocmain}
%    \end{macrocode}

% \macro{\childdocredirect}
% The deprecated macro |\childdocredirect| is a legacy version
% of |\childdocforward| and |\childdocforwardprefix|:
%    \begin{macrocode}
\newcommand{\childdocredirect}[2][]
{
  \begingroup
    \if?#1?
      \def\childdoctmp{\childdocforward{#2}}
    \else
      \def\childdoctmp{\childdocforwardprefix{#1}{#2}}
    \fi
    \expandafter
  \endgroup
  \childdoctmp
}
%    \end{macrocode}

%\iffalse
%</package>
%\fi
%
\endinput
|\\
|\childdocby{|\textit{main}|}|\\
\end{tabular}
\end{center}
%
Both forms have slightly different effects as described above.
The main file is prepared as usual, see \secref{sec:include}.

%%%%%%%%%%%%%%%%%%%%%%%%%%%%%%%%%%%%%%%%%%%%%%%%%%%%%%%%%%%%%%%%%%%%%%%%%%%%%%%%
\subsection{Legacy Detection}
\label{sec:detection}

The directive |\childdocmain| in the main file can detect
whether the complete document or merely a child is to be compiled
even without using the directive |\childdocof|.
This method is deprecated because it is less robust
and there is no compelling reason to use it;
it is merely provided for backward compatibility
and it may be removed in future versions.

If the detection mechanism is to be used,
it is mandatory to correctly specify
the filename of the main file as the argument of |\childdocmain|:
%
\begin{center}
\begin{tabular}{l}
|% \iffalse
%
% childdoc.dtx Copyright (C) 2017-2018 Niklas Beisert
%
% This work may be distributed and/or modified under the
% conditions of the LaTeX Project Public License, either version 1.3
% of this license or (at your option) any later version.
% The latest version of this license is in
%   http://www.latex-project.org/lppl.txt
% and version 1.3 or later is part of all distributions of LaTeX
% version 2005/12/01 or later.
%
% This work has the LPPL maintenance status `maintained'.
%
% The Current Maintainer of this work is Niklas Beisert.
%
% This work consists of the files childdoc.dtx and childdoc.ins
% and the derived files childdoc.def and cdocsamp.tex with
% cdocsch1.tex, cdocsch2.tex, cdocsdrf.tex, cdocsfn1.tex, cdocsfn2.tex.
%
%<package>\ifdefined\childdocmain\endinput\fi
%<package>\ProvidesFile{childdoc.def}[2018/12/30 v2.0 child document driver]
%<samplemain>\ProvidesFile{cdocsamp.tex}[2018/12/30 v2.0 sample for childdoc]
%<*driver>
%\ProvidesFile{childdoc.drv}[2018/12/30 v2.0 childdoc reference manual file]
\PassOptionsToClass{10pt,a4paper}{article}
\documentclass{ltxdoc}

\usepackage[margin=35mm]{geometry}
\usepackage{hyperref}
\usepackage{hyperxmp}
\usepackage[usenames]{color}

\hypersetup{colorlinks=true}
\hypersetup{pdfstartview=FitH}
\hypersetup{pdfpagemode=UseNone}
\hypersetup{pdfsource={}}
\hypersetup{pdflang={en-UK}}
\hypersetup{pdfcopyright={Copyright 2017-2018 Niklas Beisert.
  This work may be distributed and/or modified under the
  conditions of the LaTeX Project Public License, either version 1.3
  of this license or (at your option) any later version.}}
\hypersetup{pdflicenseurl={http://www.latex-project.org/lppl.txt}}
\hypersetup{pdfcontactaddress={ETH Zurich, ITP, HIT K,
  Wolfgang-Pauli-Strasse 27}}
\hypersetup{pdfcontactpostcode={8093}}
\hypersetup{pdfcontactcity={Zurich}}
\hypersetup{pdfcontactcountry={Switzerland}}
\hypersetup{pdfcontactemail={nbeisert@itp.phys.ethz.ch}}
\hypersetup{pdfcontacturl={http://people.phys.ethz.ch/\xmptilde nbeisert/}}

\newcommand{\secref}[1]{\hyperref[#1]{section \ref*{#1}}}

\parskip1ex
\parindent0pt
\let\olditemize\itemize
\def\itemize{\olditemize\parskip0pt}

\begin{document}

\title{The \textsf{childdoc} Package}
\hypersetup{pdftitle={The childdoc Package}}
\author{Niklas Beisert\\[2ex]
  Institut f\"ur Theoretische Physik\\
  Eidgen\"ossische Technische Hochschule Z\"urich\\
  Wolfgang-Pauli-Strasse 27, 8093 Z\"urich, Switzerland\\[1ex]
  \href{mailto:nbeisert@itp.phys.ethz.ch}
  {\texttt{nbeisert@itp.phys.ethz.ch}}}
\hypersetup{pdfauthor={Niklas Beisert}}
\hypersetup{pdfsubject={Manual for the LaTeX2e Package childdoc}}
\date{30 December 2018, \textsf{v2.0}}
\maketitle

\begin{abstract}\noindent
\textsf{childdoc} is a \LaTeXe{} package
that enables the direct compilation
of document sections included by |\include|
to individual files.
\end{abstract}

\begingroup
\parskip0ex
\tableofcontents
\endgroup

%%%%%%%%%%%%%%%%%%%%%%%%%%%%%%%%%%%%%%%%%%%%%%%%%%%%%%%%%%%%%%%%%%%%%%%%%%%%%%%%
%%%%%%%%%%%%%%%%%%%%%%%%%%%%%%%%%%%%%%%%%%%%%%%%%%%%%%%%%%%%%%%%%%%%%%%%%%%%%%%%
\section{Introduction}

\LaTeX{} provides a mechanism to structure a large document (such as a book)
into a main file and several child files (containing the chapters)
using the |\include| command.
This mechanism is beneficial for documents
which span hundreds of pages in order to
make the source file(s) more manageable.
Moreover, compilation can be restricted to
selected child files by means of the |\includeonly| command.
The latter feature can be used to reduce the compilation time while editing
(this was significantly more useful in the earlier days of \LaTeX{})
or to generate a smaller document which is easier to navigate.
Another application of |\includeonly| is to generate
documents consisting of selected parts of the complete document.

However, there are a few drawbacks of the plain |\include| mechanism:
\begin{itemize}
\item
The child files cannot be compiled on their own,
they can only be compiled via the main file.
A naive editing environment
(such as a text editor with an option
to have the current file processed by \LaTeX)
may require one to switch to the main file before compiling;
attempting to compile the child file produces errors.
\item
The main file must be modified (each time)
to adjust the |\includeonly| command
to the present needs. This easily leaves the main file in a messy state.
\item
The generated document will always carry the filename
of the main document. This is inconvenient if
several child files are to be compiled and
to be kept for distribution.
\end{itemize}

The present package provides a simple interface
to make child files individually compilable by \LaTeX{}.
Compiling a child file then has the same effect as compiling
the main file with an |\includeonly| command
to select the appropriate child.
Moreover the generated document will carry the name of the child
rather than the main file.
This resolves all three above issues.

This feature is meant to make the editing of books,
thesis documents and lecture notes somewhat more convenient.
However, the package can also be used efficiently for
composing a series of documents (such as exercise sheets)
which are typically distributed individually.
It then assists the author in generating the individual documents
(potentially in different versions)
as well as a document containing the collected series.
Another application is in developing style files
or other kinds of included material
where compilation of the style file could redirect
to a sample or test file.

%%%%%%%%%%%%%%%%%%%%%%%%%%%%%%%%%%%%%%%%%%%%%%%%%%%%%%%%%%%%%%%%%%%%%%%%%%%%%%%%
%%%%%%%%%%%%%%%%%%%%%%%%%%%%%%%%%%%%%%%%%%%%%%%%%%%%%%%%%%%%%%%%%%%%%%%%%%%%%%%%
\section{Usage}

First of all, the package \textsf{childdoc} is \emph{not} a standard
\LaTeXe{} |.sty| style file! Therefore it needs to be invoked in
a non-standard way.

%%%%%%%%%%%%%%%%%%%%%%%%%%%%%%%%%%%%%%%%%%%%%%%%%%%%%%%%%%%%%%%%%%%%%%%%%%%%%%%%
\subsection{Included Files}
\label{sec:include}

%%%%%%%%%%%%%%%%%%%%%%%%%%%%%%%%%%%%%%%%
\DescribeMacro{\childdocmain}
To use the package, add the commands
\begin{center}
\begin{tabular}{l}
|\input{childdoc.def}|\\
|\childdocmain{}|\\
\end{tabular}
\end{center}
at the very top of the main \LaTeX{} file,
in particular \emph{before} the |\documentclass| statement!
The argument of |\childdocmain| should be left empty
(but it must be present).

%%%%%%%%%%%%%%%%%%%%%%%%%%%%%%%%%%%%%%%%
\DescribeMacro{\childdocof}
Furthermore, add the commands
\begin{center}
\begin{tabular}{l}
|\input{childdoc.def}|\\
|\childdocof{|\textit{main}|}|\\
\end{tabular}
\end{center}
at the top of every child file \textit{child}
which is included by |\include{|\textit{child}|}|
from within the main file
(or at least for those files to be compiled individually).
The argument \textit{main} must be the filename of the main file.

There are a couple of
considerations in setting up the main and child documents:

%%%%%%%%%%%%%%%%%%%%%%%%%%%%%%%%%%%%%%%%
\paragraph{Restrictions.}

Please note the following restrictions:
\begin{itemize}
\item
|\childdocmain| must be called with one argument \textit{main}
to ensure compatibility with earlier version of the package.
It must either be empty (|\childdocmain{}|)
or precisely match the filename of the main file in which it is specified.
See \secref{sec:detection} for further information.
\item
The filename \textit{main} must be specified without the |.tex| extension.
\item
The filename \textit{main} is case sensitive
(even in case-insensitive file systems)
due to internal string comparison.
\item
The argument \textit{main} should be fully expanded, it cannot be a macro.
\item
Subdirectories and special characters should be avoided in filenames.
\item
The command |\childdocmain{|\textit{main}|}| must be followed by a whitespace.
It should not be followed immediately by another command
or by a comment mark `|%|'.
This is because the \TeX{} parser reads the token immediately following
the argument of |\childdocmain| and puts it
at the beginning of every child section;
however, a white\-space is ignored.
\end{itemize}

%%%%%%%%%%%%%%%%%%%%%%%%%%%%%%%%%%%%%%%%
\paragraph{Content of Main File.}

It is advisable to place all content in the child files included by |\include|.
Any output contained in the main file will appear in all child documents
unless suppressed manually;
it cannot be suppressed automatically by the |\includeonly| directive
and thus should normally be avoided.
A method to include some content in the main file
by means of conditional processing is described in \secref{sec:conditional}.

%%%%%%%%%%%%%%%%%%%%%%%%%%%%%%%%%%%%%%%%
\paragraph{Page Numbering.}

When only a part of the document is compiled,
the appropriate numbering of pages
(as well as other status parameters)
is determined from the |.aux| files.
The latter contain information from previous passes.
However this information needs to propagate through
all intermediate child documents.
Therefore the page numbering in child documents may well
be inconsistent until the complete document is compiled at least once.

A useful (if unconventional) way to always ensure a consistent
page numbering is to restart the numbering in each child document
and denote the pages by `\textit{child}|.|\textit{page}'
where \textit{child} represents the chapter/section number of the child file.
This can be achieved by the command
|\numberwithin{page}{|\textit{child}|}|
of the \textsf{amsmath} package
where \textit{child} can be |chapter| or |section|
depending on the chosen structuring.
Alternatively, one can modify the macro |\thepage| appropriately
and reset the counter |page| at the start of each child file.

%%%%%%%%%%%%%%%%%%%%%%%%%%%%%%%%%%%%%%%%%%%%%%%%%%%%%%%%%%%%%%%%%%%%%%%%%%%%%%%%
\subsection{Conditional Processing}
\label{sec:conditional}

The package provides a mechanism to compile different versions
of a document. To customise the versions further some conditional processing
can come in handy to distinguish which version is being compiled.
The package provides two macros to describe the compilation context:

%%%%%%%%%%%%%%%%%%%%%%%%%%%%%%%%%%%%%%%%
\DescribeMacro{\ifchilddoc}
The conditional |\ifchilddoc| distinguishes between the compilation of
child documents and the main document:
%
\begin{center}
|\ifchilddoc |\textit{child-code}| |[|\||else |\textit{main-code}]| \||fi|
\end{center}

%%%%%%%%%%%%%%%%%%%%%%%%%%%%%%%%%%%%%%%%
\DescribeMacro{\childdocname}
\DescribeMacro{\childdocjob}
The macro |\childdocname| contains the filename (without extension)
of the main or child file being processed.
Note that |\childdocjob| will always contain the name of the main file.

%%%%%%%%%%%%%%%%%%%%%%%%%%%%%%%%%%%%%%%%
\paragraph{Title Page.}

Conditional processing can be used to include a title or banner page
in the main document when proper precautions are taken.
Importantly, the code in the main file should ensure that the page counter
(as well as other status parameters which are stored in the |.aux| files)
takes the same value after the conditional processing.
Otherwise the page numbers may take divergent values
depending on which part is compiled.

For example, a title page could be declared by:
%
\begin{center}
\begin{tabular}{l}
|\ifchilddoc\||else|\\
|\addtocounter{page}{-1}|\\
\textit{code for title page}\\
|\newpage|\\
|\||fi|
\end{tabular}
\end{center}
%
A banner page for the child documents can be generated by:
%
\begin{center}
\begin{tabular}{l}
|\ifchilddoc|\\
|\addtocounter{page}{-1}|\\
\textit{code for banner page}\\
|\newpage|\\
|\||fi|
\end{tabular}
\end{center}
%
Here one could write a message such as:
\begin{center}
|This is the part \childdocname{} of \childdocjob{}.|
\end{center}

%%%%%%%%%%%%%%%%%%%%%%%%%%%%%%%%%%%%%%%%%%%%%%%%%%%%%%%%%%%%%%%%%%%%%%%%%%%%%%%%
\subsection{Flags}
\label{sec:flags}

The package makes it easy to generate different versions
of the main or child documents.
To this end compilation flags can be defined
and assigned different default values.
They will be particularly useful in conjunction
with the forwarding mechanism described in \secref{sec:forward}.

For example, it may be useful to have a flag |\version|
which can be set to |draft| or |final|.
The document source will contain some conditional code
depending on the value of |\version|.
Suppose further, the flag should default to |final| for the main file
and to |draft| for child files
which is a natural assignment for editing the document.
This is achieved by placing the following code
in the preamble of the main document
(below the |\childdocmain| directive):
%
\begin{center}
\begin{tabular}{l}
|\ifchilddoc|\\
|\providecommand{\version}{draft}|\\
|\||else|\\
|\providecommand{\version}{final}|\\
|\||fi|
\end{tabular}
\end{center}
%
The definition by |\providecommand| makes sure
that previous definitions are not overwritten.
Further statements |\providecommand{\version}{...}|
can thus be added before the above code to override it.

For the main file, one might add a line
(between |\childdocmain| and the above block)
%
\begin{center}
|%\ifchilddoc\||else\providecommand{\version}{draft}\||fi|
\end{center}
%
which can be uncommented to produce a draft version.
Likewise one can add a line to the very top of a child file
(above the |\childdocof{|\textit{main}|}| directive)
%
\begin{center}
|%\providecommand{\version}{final}|
\end{center}
%
which can be uncommented to produce the final version of this child document.

%%%%%%%%%%%%%%%%%%%%%%%%%%%%%%%%%%%%%%%%%%%%%%%%%%%%%%%%%%%%%%%%%%%%%%%%%%%%%%%%
\subsection{Forwarding}
\label{sec:forward}

Different versions of the main or child documents
using compilation flags as described in \secref{sec:flags}
can be (permanently) stored in different files
for convenient compilation, viewing and distribution.
To this end, the package defines a command
to pass on compilation to a different file:

%%%%%%%%%%%%%%%%%%%%%%%%%%%%%%%%%%%%%%%%
\DescribeMacro{\childdocforward}
The command |\childdocforward| redirects processing to
another source file:
%
\begin{center}
\begin{tabular}{l}
|\input{childdoc.def}|\\
|\childdocforward[|\textit{main}|]{|\textit{dest}|}|\\
\end{tabular}
\end{center}
%
The argument \textit{dest} is the destination file
(without extension).
It should be the main file or one of the child files.
Note that further \textsf{childdoc} directives
such as |\childdocof| and |\childdocforward|
in the indicated file will be processed in this form.
The optional argument \textit{main}
passes on directly to the main file \textit{main}
while pretending to compile the child \textit{dest}.
This form behaves as if \textit{dest}
issues |\childdocof{|\textit{main}|}| right away,
and no further \textsf{childdoc} directives will be processed.

%%%%%%%%%%%%%%%%%%%%%%%%%%%%%%%%%%%%%%%%
\DescribeMacro{\...prefix}
In the alternative form |\childdocforwardprefix|,
%
\begin{center}
\begin{tabular}{l}
|\input{childdoc.def}|\\
|\childdocforwardprefix[|\textit{main}|]{|\textit{prefix}|}{|\textit{dest}|}|
\end{tabular}
\end{center}
%
the destination file is determined by a pattern
depending on the current file:
To make this work, the current file must be called
`{\textit{prefix}\hspace{0.2em}\textit{suffix}}'
with \textit{prefix} matching precisely the argument.
Processing is then passed on to the file
`{\textit{dest}\hspace{0.2em}\textit{suffix}}'.
Surely, the same effect is achieved by
directly specifying the
argument `{\textit{dest}\hspace{0.2em}\textit{suffix}}'
in the first form.
However, that requires to set up a different file
for each child. With the alternative form of the command
all these files can have exactly the same content
which simplifies setting them up and maintaining them.

For example, the following file |draft.tex|
with a compilation flag |\version| as described in \secref{sec:flags}
compiles the main document as a draft:
%
\begin{center}
\begin{tabular}{l}
|\def\version{draft}|\\
|\input{childdoc.def}|\\
|\childdocforward{|\textit{main}|}|
\end{tabular}
\end{center}
%
Likewise, the following files |final|\textit{nn}|.tex|
compile the final version of the child document
|child|\textit{nn}|.tex|:
%
\begin{center}
\begin{tabular}{l}
|\def\version{final}|\\
|\input{childdoc.def}|\\
|\childdocforwardprefix{final}{child}|
\end{tabular}
\end{center}
%

Note that when several versions of a main file and/or of each child file
are to be generated, it may be convenient to set up a |Makefile| or
shell script to automatise the process.

%%%%%%%%%%%%%%%%%%%%%%%%%%%%%%%%%%%%%%%%%%%%%%%%%%%%%%%%%%%%%%%%%%%%%%%%%%%%%%%%
\subsection{Command Line Processing}
\label{sec:commandline}

The effect of redirection files can also be achieved by invoking
the \LaTeX{} compiler with a more elaborate command line.
Most conveniently this should be done as part
of a shell script or a |Makefile|.

When using \textsf{childdoc} in the main file, the following
command lines effectively perform a redirection
(note that depending on the shell being used,
backslashes may have to be doubled: `|\|' $\to$ `|\\|'):
%
\begin{center}
|... -jobname "|\textit{target}|" |\\|"|[\textit{flags}]%
|\input{childdoc.def}\childdocforward[|\textit{main}|]{|\textit{dest}|}"|
\end{center}
%
Here \textit{target} is the name of the output file,
\textit{main} is the name of the main file
and \textit{dest} is the name of the main or child file to be processed
(all filenames without extensions).
The optional argument \textit{main} can be omitted
if \textit{main} matches \textit{dest}.
Optionally, compilation \textit{flags} can be defined via |\def| commands.
This command line makes the \TeX{} engine believe
it is compiling the file \textit{target}
whose content is specified as the latter parameter.
The provided code then forwards the processing to
\textit{main} or \textit{dest} as described in \secref{sec:forward}.

%%%%%%%%%%%%%%%%%%%%%%%%%%%%%%%%%%%%%%%%%%%%%%%%%%%%%%%%%%%%%%%%%%%%%%%%%%%%%%%%
\subsection{Include by Input}
\label{sec:input}

Including child documents by |\include| has some restrictions by design.
Most notably, the content of a child document always occupies
its own set of pages; pages cannot be shared between child documents.
Usually, this behaviour makes perfect sense
because each child document contain an essential part of the document.
However, in some situations it may be desirable to compose
a document from a collection of parts
without having mandatory page breaks between then.
For this case, the package
provides a mechanism to include parts
by |\input| which can also be processed individually.
However, by construction this mechanism
requires manual handling of the content to be output.

%%%%%%%%%%%%%%%%%%%%%%%%%%%%%%%%%%%%%%%%
\DescribeMacro{\ifchilddocmanual}
The main file should be prepared as usual, see \secref{sec:include}.
However, the document body must make a distinction
between processing of an individual part and of the main document, e.g.:
%
\begin{center}
\begin{tabular}{l}
|\ifchilddocmanual|\\
|\input{\childdocname}|\\
|\||else|\\
\textit{document body with }|\input{|\textit{part}|}|\\
|\||fi|
\end{tabular}
\end{center}
%
The conditional |\ifchilddocmanual| is true whenever
a part to be included by |\input| is being compiled,
and the name of the part is stored in |\childdocname|.

%%%%%%%%%%%%%%%%%%%%%%%%%%%%%%%%%%%%%%%%
\DescribeMacro{\childdocby}
Each part to be included by |\input| should start with:
%
\begin{center}
\begin{tabular}{l}
|\input{childdoc.def}|\\
|\childdocby{|\textit{main}|}|\\
\end{tabular}
\end{center}
%
The directive |\childdocby| is similar to |\childdocof|
described in \secref{sec:include},
but the subsequent selection of content must be done manually.
To that end, both |\ifchilddoc| and |\ifchilddocmanual|
will be true upon processing of a part,
and the name of the part is stored in |\childdocname|.
Note that |\jobname| will be set to the filename of the current part
so that each part receives an individual |.aux| file
that does not interfere with the |.aux| file(s) of the main document.
This behaviour can be altered by the alternative form
|\childdocby[*]{|\textit{main}|}| (with a non-empty optional argument)
which uses the |.aux| file of the main document
by setting |\jobname| to \textit{main}.

%%%%%%%%%%%%%%%%%%%%%%%%%%%%%%%%%%%%%%%%%%%%%%%%%%%%%%%%%%%%%%%%%%%%%%%%%%%%%%%%
\subsection{Driver Development}
\label{sec:driver}

The \textsf{childdoc} mechanism can also be use for the development
of definition files such as \LaTeX{} styles or classes.
This case differs from the above setup with multiple parts
included by |\include| in that no |\includeonly| should be invoked.
This can be achieved by starting the include file
(before |\ProvidesPackage|) with:
%
\begin{center}
\begin{tabular}{l}
|\input{childdoc.def}|\\
|\childdocforward{|\textit{main}|}|\\
\end{tabular}
\end{center}
%
or alternatively with:
%
\begin{center}
\begin{tabular}{l}
|\input{childdoc.def}|\\
|\childdocby{|\textit{main}|}|\\
\end{tabular}
\end{center}
%
Both forms have slightly different effects as described above.
The main file is prepared as usual, see \secref{sec:include}.

%%%%%%%%%%%%%%%%%%%%%%%%%%%%%%%%%%%%%%%%%%%%%%%%%%%%%%%%%%%%%%%%%%%%%%%%%%%%%%%%
\subsection{Legacy Detection}
\label{sec:detection}

The directive |\childdocmain| in the main file can detect
whether the complete document or merely a child is to be compiled
even without using the directive |\childdocof|.
This method is deprecated because it is less robust
and there is no compelling reason to use it;
it is merely provided for backward compatibility
and it may be removed in future versions.

If the detection mechanism is to be used,
it is mandatory to correctly specify
the filename of the main file as the argument of |\childdocmain|:
%
\begin{center}
\begin{tabular}{l}
|\input{childdoc.def}|\\
|\childdocmain{|\textit{main}|}|\\
\end{tabular}
\end{center}
%
If |\jobname| does not match the argument \textit{main} of |\childdocmain|,
it is assumed that |\jobname| points to the child file to be compiled.
When using |\childdocmain| with the main file specified as argument,
it suffices to start a child file
with just |\input{|\textit{main}|}|
without loading of the package and using |\childdocof|.
If instead all processing is done
with the appropriate \textsf{childdoc} directives,
the argument of \textit{main} of |\childdocmain| can be empty.

An alternative version of the command line processing described
in \secref{sec:commandline} using the detection mechanism reads:
%
\begin{center}
|... -jobname "|\textit{target}|" "|[\textit{flags}]%
[|\def\jobname{|\textit{dest}|}|]|\input{|\textit{main}|}"|
\end{center}

%%%%%%%%%%%%%%%%%%%%%%%%%%%%%%%%%%%%%%%%%%%%%%%%%%%%%%%%%%%%%%%%%%%%%%%%%%%%%%%%
\subsection{Manual Code}
\label{sec:manual}

In case one cannot be certain whether the definitions file |childdoc.def|
is installed on the target \TeX{} distribution
and one prefers not to ship it,
it is conceivable to paste a few relevant commands into the sources.

To that end, drop all statements |\input{childdoc.def}|
and perform the replacements as outlined below.
Instead of |\childdocmain{|\textit{main}|}| add the following code
to the top of the main file:
%
\begin{center}
\begin{tabular}{l}
|\||ifdefined\childdocname\endinput\||fi\newif\ifchilddoc|\\
|\edef\childdocname{\scantokens\expandafter{\jobname\noexpand}}|\\
|\def\childdocmain{|\textit{main}|}\||ifx\childdocmain\childdocname\||else|\\
|\childdoctrue\includeonly{\childdocname}\let\jobname\childdocmain\||fi|\\
\end{tabular}
\end{center}
%
Instead of |\childdocof{|\textit{main}|}| just include the main file
at the top of each child file:
%
\begin{center}
|\input{|\textit{main}|}|
\end{center}
%
A simple redirection |\childdocforward{|\textit{dest}|}| is achieved by:
%
\begin{center}
|\def\jobname{|\textit{dest}|}\input{\jobname}|
\end{center}
%
The redirection with prefix
|\childdocforwardprefix[|\textit{prefix}|]{|\textit{dest}|}|
is accomplished by:
%
\begin{center}
\begin{tabular}{l}
|{\edef\jobname{\scantokens\expandafter{\jobname\noexpand}}|\\
|\def\redirectjob |\textit{prefix}|#1~~~{\gdef\jobname{|\textit{dest}|#1}}|\\
|\expandafter\redirectjob\jobname~~~}\input{\jobname}|
\end{tabular}
\end{center}

In an alternative approach,
child documents can be compiled by a specific command line
without additional code or specific definitions:
%
\begin{center}
|... -jobname "|\textit{target}|" "|[\textit{flags}]%
|\includeonly{|\textit{dest}|}\input{|\textit{main}|}"|
\end{center}
%

%%%%%%%%%%%%%%%%%%%%%%%%%%%%%%%%%%%%%%%%%%%%%%%%%%%%%%%%%%%%%%%%%%%%%%%%%%%%%%%%
%%%%%%%%%%%%%%%%%%%%%%%%%%%%%%%%%%%%%%%%%%%%%%%%%%%%%%%%%%%%%%%%%%%%%%%%%%%%%%%%
\section{Information}

%%%%%%%%%%%%%%%%%%%%%%%%%%%%%%%%%%%%%%%%%%%%%%%%%%%%%%%%%%%%%%%%%%%%%%%%%%%%%%%%
\subsection{Copyright}

Copyright \copyright{} 2017--2018 Niklas Beisert

This work may be distributed and/or modified under the
conditions of the \LaTeX{} Project Public License, either version 1.3
of this license or (at your option) any later version.
The latest version of this license is in
  \url{http://www.latex-project.org/lppl.txt}
and version 1.3 or later is part of all distributions of \LaTeX{}
version 2005/12/01 or later.

This work has the LPPL maintenance status `maintained'.

The Current Maintainer of this work is Niklas Beisert.

This work consists of the files |README.txt|, |childdoc.ins| and |childdoc.dtx|
as well as the derived files |childdoc.def|, |cdocsamp.tex|
with |cdocsch1.tex|, |cdocsch2.tex|, |cdocspt3.tex|, |cdocspt4.tex|,
|cdocsdrf.tex|, |cdocsfn1.tex|, |cdocsfn2.tex|
as well as |childdoc.pdf|.

%%%%%%%%%%%%%%%%%%%%%%%%%%%%%%%%%%%%%%%%%%%%%%%%%%%%%%%%%%%%%%%%%%%%%%%%%%%%%%%%
\subsection{Files and Installation}

The package consists of the files:
%
\begin{center}
\begin{tabular}{ll}
    |README.txt|   & readme file \\
    |childdoc.ins| & installation file \\
    |childdoc.dtx| & source file \\
    |childdoc.def| & definition file \\
    |cdocsamp.tex| & sample main file \\
    |cdocsch1.tex| & sample include file \\
    |cdocsch2.tex| & sample include file \\
    |cdocspt3.tex| & sample part file \\
    |cdocspt4.tex| & sample part file \\
    |cdocsdrf.tex| & sample redirection file \\
    |cdocsfn1.tex| & sample redirection file \\
    |cdocsfn2.tex| & sample redirection file \\
    |childdoc.pdf| & manual
\end{tabular}
\end{center}
%
The distribution consists of the files
|README.txt|, |childdoc.ins| and |childdoc.dtx|.
%
\begin{itemize}
\item
Run (pdf)\LaTeX{} on |childdoc.dtx|
to compile the manual |childdoc.pdf| (this file).
\item
Run \LaTeX{} on |childdoc.ins| to create the definitions file |childdoc.def|
and the sample |cdocsamp.tex| with include files
|cdocsch1.tex|, |cdocsch2.tex|, |cdocspt3.tex|, |cdocspt4.tex|,
|cdocsdrf.tex|, |cdocsfn1.tex|, |cdocsfn2.tex|.
Then copy the file |childdoc.def| to an appropriate directory of your \LaTeX{}
distribution, e.g.\ \textit{texmf-root}|/tex/latex/childdoc|.
\end{itemize}

%%%%%%%%%%%%%%%%%%%%%%%%%%%%%%%%%%%%%%%%%%%%%%%%%%%%%%%%%%%%%%%%%%%%%%%%%%%%%%%%
\subsection{Related CTAN Packages}

There are several other packages which offer a similar functionality:
%
\begin{itemize}
\item
The packages
\href{http://ctan.org/pkg/docmute}{\textsf{docmute}},
\href{http://ctan.org/pkg/includex}{\textsf{includex}} and
\href{http://ctan.org/pkg/standalone}{\textsf{standalone}}
provide commands to include only the document body of
a child file thus allowing both files to be compiled individually.
\item
The packages \href{http://ctan.org/pkg/subdocs}{\textsf{subdocs}}
and \href{http://ctan.org/pkg/subfiles}{\textsf{subfiles}}
provide structures in which the main and child documents can be
encapsulated and allowing them to be compiled individually.
The inclusion mechanism is different from the conventional |\include|.
\item
The package \href{http://ctan.org/pkg/combine}{\textsf{combine}}
is an elaborate solution to combine several documents into one.
\end{itemize}
%
See also the CTAN topic \href{http://ctan.org/topic/subdocs}{\textsf{subdocs}}
for further related packages.
The present package differs from the above solutions in that
a document structure constructed with the conventional |\include| mechanism
just needs two extra commands at the top of every file
such that all constituent files can be compiled individually.

%%%%%%%%%%%%%%%%%%%%%%%%%%%%%%%%%%%%%%%%%%%%%%%%%%%%%%%%%%%%%%%%%%%%%%%%%%%%%%%%
%\subsection{Feature Suggestions}
%
%The following is a list of features which may be useful for future
%versions of this package:
%%
%\begin{itemize}
%\item
%\ldots
%\end{itemize}

%%%%%%%%%%%%%%%%%%%%%%%%%%%%%%%%%%%%%%%%%%%%%%%%%%%%%%%%%%%%%%%%%%%%%%%%%%%%%%%%
\subsection{Revision History}

%%%%%%%%%%%%%%%%%%%%%%%%%%%%%%%%%%%%%%%%
\paragraph{v2.0:} 2018/12/30

\begin{itemize}
\item
immediate forward processing
\item
added |\childdocby| mechanism
\item
manual restructured
\end{itemize}

%%%%%%%%%%%%%%%%%%%%%%%%%%%%%%%%%%%%%%%%
\paragraph{v1.6:} 2018/01/17

\begin{itemize}
\item
application for development of include files
\item
corrections to manual
\end{itemize}

%%%%%%%%%%%%%%%%%%%%%%%%%%%%%%%%%%%%%%%%
\paragraph{v1.5:} 2017/05/21

\begin{itemize}
\item
more complete structuring introduced
\item
|\childdocof| introduced
\item
|\childdoc| renamed to |\childdocmain|
\item
|\childredirect| renamed to |\childdocforward| and |\childdocforwardprefix|
and functionality expanded
\end{itemize}

%%%%%%%%%%%%%%%%%%%%%%%%%%%%%%%%%%%%%%%%
\paragraph{v1.0:} 2017/04/27

\begin{itemize}
\item
manual and install package
\item
first version published on CTAN
\end{itemize}

%%%%%%%%%%%%%%%%%%%%%%%%%%%%%%%%%%%%%%%%
\paragraph{v0.6:} 2017/04/26

\begin{itemize}
\item
redirection mechanism added
\end{itemize}

%%%%%%%%%%%%%%%%%%%%%%%%%%%%%%%%%%%%%%%%
\paragraph{v0.5:} 2017/04/26

\begin{itemize}
\item
functionality in definition file
\end{itemize}


%%%%%%%%%%%%%%%%%%%%%%%%%%%%%%%%%%%%%%%%%%%%%%%%%%%%%%%%%%%%%%%%%%%%%%%%%%%%%%%%
%%%%%%%%%%%%%%%%%%%%%%%%%%%%%%%%%%%%%%%%%%%%%%%%%%%%%%%%%%%%%%%%%%%%%%%%%%%%%%%%
%%%%%%%%%%%%%%%%%%%%%%%%%%%%%%%%%%%%%%%%%%%%%%%%%%%%%%%%%%%%%%%%%%%%%%%%%%%%%%%%
\appendix

\settowidth\MacroIndent{\rmfamily\scriptsize 000\ }

 \DocInput{childdoc.dtx}

\end{document}
%</driver>
% \fi
%
% %%%%%%%%%%%%%%%%%%%%%%%%%%%%%%%%%%%%%%%%%%%%%%%%%%%%%%%%%%%%%%%%%%%%%%%%%%%%%%
% %%%%%%%%%%%%%%%%%%%%%%%%%%%%%%%%%%%%%%%%%%%%%%%%%%%%%%%%%%%%%%%%%%%%%%%%%%%%%%
% \section{Sample}
%\iffalse
%<*samplemain>
%\fi
%
% The following presents a sample document
% with two chapters, two parts, a title page,
% a compile flag as well as three forwarding files to set the flag.
% It consists of eight |.tex| files:
% \begin{center}
% \begin{tabular}{ll}
% |cdocsamp.tex|&main file\\
% |cdocsch1.tex|&include file for chapter 1\\
% |cdocsch2.tex|&include file for chapter 2\\
% |cdocspt3.tex|&include file for part 3\\
% |cdocspt4.tex|&include file for part 4\\
% |cdocsdrf.tex|&forwarding file for main file in draft mode\\
% |cdocsfi1.tex|&forwarding file for final version of chapter 1\\
% |cdocsfi2.tex|&forwarding file for final version of chapter 2\\
% \end{tabular}
% \end{center}
% Each of the eight files can be compiled directly by the \LaTeX{} compiler.
%
% %%%%%%%%%%%%%%%%%%%%%%%%%%%%%%%%%%%%%%
% \paragraph{Main File.}
%
% The main file is called |cdocsamp.tex|.
%
% Load the \textsf{childdoc} definitions and
% declare the filename for the main document:
%    \begin{macrocode}
\input{childdoc.def}
\childdocmain{}
%    \end{macrocode}

% Optional override for |\version| flag:
%    \begin{macrocode}
%%\ifchilddoc\else\providecommand{\version}{draft}\fi
%    \end{macrocode}

% Define the default values for the |\version| flag
% (|final| for the main file and |draft| for childs):
%    \begin{macrocode}
\ifchilddoc
\providecommand{\version}{draft}
\else
\providecommand{\version}{final}
\fi
%    \end{macrocode}

% Load the standard document class:
%    \begin{macrocode}
\documentclass[12pt]{article}
%    \end{macrocode}

% Start the document body:
%    \begin{macrocode}
\begin{document}
%    \end{macrocode}

% Declare a title page.
% Print title, part of document being processed and version flag:
%    \begin{macrocode}
\addtocounter{page}{-1}
\begin{center}
{\LARGE\bfseries{}childdoc example\par}
\vspace{1cm}
\ifchilddoc
\ifchilddocmanual part\else chapter\fi:
`\childdocname' of `\childdocjob'\par
\else
main document: `\childdocjob'\par
\fi
version: \version\par
\end{center}
\newpage
%    \end{macrocode}

% Manually include selected file,
% otherwise process as usual:
%    \begin{macrocode}
\ifchilddocmanual
\section*{part `\childdocname'}
\input{\childdocname}
\else
%    \end{macrocode}

% Include the two chapters:
%    \begin{macrocode}
\include{cdocsch1}
\include{cdocsch2}
%    \end{macrocode}

% Include the two parts unless only chapters should be displayed:
%    \begin{macrocode}
\ifchilddoc\else
\section{part three}
\input{cdocspt3}
\section{part four}
\input{cdocspt4}
\fi
%    \end{macrocode}

% Process as usual until here:
%    \begin{macrocode}
\fi
%    \end{macrocode}

% End of document body:
%    \begin{macrocode}
\end{document}
%    \end{macrocode}
%\iffalse
%</samplemain>
%\fi
%
% %%%%%%%%%%%%%%%%%%%%%%%%%%%%%%%%%%%%%%
% \paragraph{Chapter Include Files.}
%
% The include files are called |cdocsch1.tex| and |cdocsch2.tex|.
%
%\iffalse
%<*samplechap1|samplechap2>
%\fi

% Optional override for |\version| flag:
%    \begin{macrocode}
%%\providecommand{\version}{final}
%    \end{macrocode}

% Include the main document:
%    \begin{macrocode}
\input{childdoc.def}
\childdocof{cdocsamp}
%    \end{macrocode}

%\iffalse
%</samplechap1|samplechap2>
%\fi
%
%\iffalse
%<*samplechap1>
%\fi
% Some text for chapter 1:
%    \begin{macrocode}
\section{one}
some text in chapter one
%    \end{macrocode}

%\iffalse
%</samplechap1>
%\fi
% Some text for chapter 2:
%\iffalse
%<*samplechap2>
%\fi
%    \begin{macrocode}
\section{two}
more text in chapter two
%    \end{macrocode}

%\iffalse
%</samplechap2>
%\fi
%
% %%%%%%%%%%%%%%%%%%%%%%%%%%%%%%%%%%%%%%
% \paragraph{Part Include Files.}
%
% The include files are called |cdocspt3.tex| and |cdocspt4.tex|.
%
%\iffalse
%<*samplepart3|samplepart4>
%\fi

% Optional override for |\version| flag:
%    \begin{macrocode}
%%\providecommand{\version}{final}
%    \end{macrocode}

% Include the main document:
%    \begin{macrocode}
\input{childdoc.def}
\childdocby{cdocsamp}
%    \end{macrocode}

%\iffalse
%</samplepart3|samplepart4>
%\fi
%
%\iffalse
%<*samplepart3>
%\fi
% Some text for part 3:
%    \begin{macrocode}
some text in part three
%    \end{macrocode}

%\iffalse
%</samplepart3>
%\fi
% Some text for part 4:
%\iffalse
%<*samplepart4>
%\fi
%    \begin{macrocode}
more text in part four
%    \end{macrocode}

%\iffalse
%</samplepart4>
%\fi
%
% %%%%%%%%%%%%%%%%%%%%%%%%%%%%%%%%%%%%%%
% \paragraph{Forwarding for a Complete Draft.}
%
% The following forwarding file |cdocsdrf.tex|
% compiles the main document in draft mode:
%\iffalse
%<*sampledraft>
%\fi
%    \begin{macrocode}
\def\version{draft}
\input{childdoc.def}
\childdocforward{cdocsamp}
%    \end{macrocode}

%\iffalse
%</sampledraft>
%\fi
%
% %%%%%%%%%%%%%%%%%%%%%%%%%%%%%%%%%%%%%%
% \paragraph{Forwarding for Final Version of the Chapters.}
%
% The following forwarding files |cdocsfn1.tex| and |cdocsfn2.tex|
% (with identical content)
% compile the final versions of the child documents
% |cdocsch1.tex| and |cdocsch2.tex|, respectively:
%\iffalse
%<*samplefinal>
%\fi
%    \begin{macrocode}
\def\version{final}
\input{childdoc.def}
\childdocforwardprefix[cdocsamp]{cdocsfn}{cdocsch}
%    \end{macrocode}

%\iffalse
%</samplefinal>
%\fi
%
% %%%%%%%%%%%%%%%%%%%%%%%%%%%%%%%%%%%%%%
% \paragraph{Command Line Processing.}
%
% The following three command lines generate the output files
% |cdocscld|, |cdocscl1| and |cdocscl2|
% which should be identical to
% |cdocsdrf|, |cdocsch1| and |cdocsfn2|, respectively:
% \begin{center}
% \begin{tabular}{l}
% |latex -jobname cdocscld \|\\
% |  "\def\version{draft}\input{childdoc.def}\childdocforward{cdocsamp}"|\\
% |latex -jobname cdocscl1 \|\\
% |  "\input{childdoc.def}\childdocforward[cdocsamp]{cdocsch1}"|\\
% |latex -jobname cdocscl2 \|\\
% |  "\def\version{final}\input{childdoc.def}\childdocforward{cdocsch2}"|
% \end{tabular}
% \end{center}
% Note that the trailing backslash on each first line
% merely continues the input to the second line
% (for convenient cut ant paste).
% Furthermore, the command |latex| can be replaced by any
% of its alternative versions such as |pdflatex|.
%
% %%%%%%%%%%%%%%%%%%%%%%%%%%%%%%%%%%%%%%%%%%%%%%%%%%%%%%%%%%%%%%%%%%%%%%%%%%%%%%
% %%%%%%%%%%%%%%%%%%%%%%%%%%%%%%%%%%%%%%%%%%%%%%%%%%%%%%%%%%%%%%%%%%%%%%%%%%%%%%
% \section{Implementation}
%\iffalse
%<*package>
%\fi
%
% This section describes the definitions file |childdoc.def|.

% The definitions cannot be loaded using |\usepackage| or |\RequirePackage|
% which has a mechanism to prevent loading a style file more than once.
% When loading the definitions by means of |\input|
% multiple instances have to be prevented manually:
%\iffalse
%This code needs to be before the `\ProvidesFile' directive
%which is defined at the beginning of this file.
%Therefore it is also placed there and commented out here.
%</package>
%<*discard>
%\fi
%    \begin{macrocode}
\ifdefined\childdocmain\endinput\fi
%    \end{macrocode}
%\iffalse
%</discard>
%<*package>
%\fi
%
% \macro{\ifchilddoc}
% \macro{\ifchilddocmanual}
% The conditional |\ifchilddoc| tells whether a
% child (true) or main (false) document is being compiled.
% The conditional |\ifchilddocmanual| tells whether
% the |\includeonly| mechanism is used (false) or
% the selection of child files must be performed manually (true).
% The definitions initialise to false:
%    \begin{macrocode}
\newif\ifchilddoc
\newif\ifchilddocmanual
%    \end{macrocode}

% \macro{\childdocname}
% \macro{\childdocjob}
% The macro |\childdocname| stores the name of the main document
% to be compiled. The macro |\childdocjob| stores the name of
% the document on which the \LaTeX{} compiler was originally invoked.
% The content of |\jobname| cannot be compared
% to filenames specified in the source due to different catcodes.
% The following code rescans |\jobname|, stores the result
% in |\childdocname| and saves a copy in |\childdocjob|:
%    \begin{macrocode}
\edef\childdocname{\scantokens\expandafter{\jobname\noexpand}}
\let\childdocjob\childdocname
%    \end{macrocode}

% \macro{\childdocdisable}
% The macro |\childdocdisable| prevents the main file
% from being processed more than once.
% At this stage, the main document command |\childdocmain|
% is assumed to be called once again where it should do nothing.
% Any subsequent call to it should prevent
% a secondary processing of the main document
% It overwrites the forwarding commands
% |\childdocof| and |\childdocforward|
% with empty macros to prevent further inclusions of the main document:
%    \begin{macrocode}
\newcommand{\childdocdisable}
{
  \renewcommand{\childdocmain}[1]{\renewcommand{\childdocmain}[1]{\endinput}}
  \renewcommand{\childdocof}[1]{}
  \renewcommand{\childdocby}[2][]{}
  \renewcommand{\childdocforward}[2][]{}
  \renewcommand{\childdocdisable}{}
}
%    \end{macrocode}

% \macro{\childdocmain}
% The macro |\childdocmain| is to be called at the top of the main file
% with nothing or the main filename (without extension) as argument.
% First, it breaks loops.
% If the argument is not empty and does not match |\childdocname|
% (which is set by the first inclusion of |childdoc.def|),
% |\ifchilddoc| is set to true, |\includeonly| is applied to the child file
% and |\jobname| is set to the main file
% (for proper handling of |.aux| files):
%    \begin{macrocode}
\newcommand{\childdocmain}[1]
{
  \childdocdisable\childdocmain{}
  \if?#1?\else
    \begingroup
      \def\childdoctmp{#1}
      \ifx\childdoctmp\childdocname
        \def\childdoctmp{}
      \else
        \def\childdoctmp
        {
          \childdoctrue
          \includeonly{\childdocname}
          \def\childdocjob{#1}
          \def\jobname{#1}
        }
      \fi
      \expandafter
    \endgroup
    \childdoctmp
  \fi
}
%    \end{macrocode}

% \macro{\childdocof}
% The command |\childdocof| redirects
% compilation to the main file |#1|.
%    \begin{macrocode}
\newcommand{\childdocof}[1]
{
  \childdocdisable
  \childdoctrue
  \includeonly{\childdocname}
  \def\jobname{#1}
  \def\childdocjob{#1}
  \input{#1}
}
%    \end{macrocode}

% \macro{\childdocby}
% The command |\childdocby| ....
%    \begin{macrocode}
\newcommand{\childdocby}[2][]
{
  \childdocdisable
  \childdoctrue
  \childdocmanualtrue
  \if?#1?\else
    \def\jobname{#2}
  \fi
  \def\childdocjob{#2}
  \input{#2}
  \endinput
}
%    \end{macrocode}

% \macro{\childdocforward}
% The command |\childdocforward| redirects
% compilation to the main file or
% (if the optional argument is given) a child file.
% Parameters are set as if the main file
% or a child file starting with |\childdocof| was compiled.
% Then compilation is handed over to the main file:
%    \begin{macrocode}
\newcommand{\childdocforward}[2][]
{
  \begingroup
    \if?#1?
      \def\childdoctmp
      {
        \def\childdocname{#2}
        \def\childdocjob{#2}
        \def\jobname{#2}
        \input{#2}
        \endinput
      }
    \else
      \def\childdoctmp
      {
        \childdocdisable
        \def\childdocname{#2}
        \childdoctrue
        \includeonly{#2}
        \def\childdocjob{#1}
        \def\jobname{#1}
        \input{#1}
        \endinput
      }
    \fi
    \expandafter
  \endgroup
  \childdoctmp
}
%    \end{macrocode}

% \macro{\childdocforwardprefix}
% The command |\childdocforwardprefix| redirects
% compilation to the main or a child file by means of a pattern.
% The prefix |#1| in the current filename is replaced by |#2|
% and the suffix of the current filename is kept
% (it is assumed that the filename does not contain the substring `|~~~|'
% which is used as a delimiter).
% Compilation is handed over to the new file by |\childdocforward|:
%    \begin{macrocode}
\newcommand{\childdocforwardprefix}[3][]
{
  \begingroup
    \def\childdocextract #2##1~~~{\def\childdoctmp{\childdocforward[#1]{#3##1}}}
    \expandafter\childdocextract\childdocname~~~
    \expandafter
  \endgroup
  \childdoctmp
}
%    \end{macrocode}

% \macro{\childdoc}
% The deprecated macro |\childdoc| is a legacy version of |\childdocmain|:
%    \begin{macrocode}
\newcommand{\childdoc}{\childdocmain}
%    \end{macrocode}

% \macro{\childdocredirect}
% The deprecated macro |\childdocredirect| is a legacy version
% of |\childdocforward| and |\childdocforwardprefix|:
%    \begin{macrocode}
\newcommand{\childdocredirect}[2][]
{
  \begingroup
    \if?#1?
      \def\childdoctmp{\childdocforward{#2}}
    \else
      \def\childdoctmp{\childdocforwardprefix{#1}{#2}}
    \fi
    \expandafter
  \endgroup
  \childdoctmp
}
%    \end{macrocode}

%\iffalse
%</package>
%\fi
%
\endinput
|\\
|\childdocmain{|\textit{main}|}|\\
\end{tabular}
\end{center}
%
If |\jobname| does not match the argument \textit{main} of |\childdocmain|,
it is assumed that |\jobname| points to the child file to be compiled.
When using |\childdocmain| with the main file specified as argument,
it suffices to start a child file
with just |\input{|\textit{main}|}|
without loading of the package and using |\childdocof|.
If instead all processing is done
with the appropriate \textsf{childdoc} directives,
the argument of \textit{main} of |\childdocmain| can be empty.

An alternative version of the command line processing described
in \secref{sec:commandline} using the detection mechanism reads:
%
\begin{center}
|... -jobname "|\textit{target}|" "|[\textit{flags}]%
[|\def\jobname{|\textit{dest}|}|]|\input{|\textit{main}|}"|
\end{center}

%%%%%%%%%%%%%%%%%%%%%%%%%%%%%%%%%%%%%%%%%%%%%%%%%%%%%%%%%%%%%%%%%%%%%%%%%%%%%%%%
\subsection{Manual Code}
\label{sec:manual}

In case one cannot be certain whether the definitions file |childdoc.def|
is installed on the target \TeX{} distribution
and one prefers not to ship it,
it is conceivable to paste a few relevant commands into the sources.

To that end, drop all statements |% \iffalse
%
% childdoc.dtx Copyright (C) 2017-2018 Niklas Beisert
%
% This work may be distributed and/or modified under the
% conditions of the LaTeX Project Public License, either version 1.3
% of this license or (at your option) any later version.
% The latest version of this license is in
%   http://www.latex-project.org/lppl.txt
% and version 1.3 or later is part of all distributions of LaTeX
% version 2005/12/01 or later.
%
% This work has the LPPL maintenance status `maintained'.
%
% The Current Maintainer of this work is Niklas Beisert.
%
% This work consists of the files childdoc.dtx and childdoc.ins
% and the derived files childdoc.def and cdocsamp.tex with
% cdocsch1.tex, cdocsch2.tex, cdocsdrf.tex, cdocsfn1.tex, cdocsfn2.tex.
%
%<package>\ifdefined\childdocmain\endinput\fi
%<package>\ProvidesFile{childdoc.def}[2018/12/30 v2.0 child document driver]
%<samplemain>\ProvidesFile{cdocsamp.tex}[2018/12/30 v2.0 sample for childdoc]
%<*driver>
%\ProvidesFile{childdoc.drv}[2018/12/30 v2.0 childdoc reference manual file]
\PassOptionsToClass{10pt,a4paper}{article}
\documentclass{ltxdoc}

\usepackage[margin=35mm]{geometry}
\usepackage{hyperref}
\usepackage{hyperxmp}
\usepackage[usenames]{color}

\hypersetup{colorlinks=true}
\hypersetup{pdfstartview=FitH}
\hypersetup{pdfpagemode=UseNone}
\hypersetup{pdfsource={}}
\hypersetup{pdflang={en-UK}}
\hypersetup{pdfcopyright={Copyright 2017-2018 Niklas Beisert.
  This work may be distributed and/or modified under the
  conditions of the LaTeX Project Public License, either version 1.3
  of this license or (at your option) any later version.}}
\hypersetup{pdflicenseurl={http://www.latex-project.org/lppl.txt}}
\hypersetup{pdfcontactaddress={ETH Zurich, ITP, HIT K,
  Wolfgang-Pauli-Strasse 27}}
\hypersetup{pdfcontactpostcode={8093}}
\hypersetup{pdfcontactcity={Zurich}}
\hypersetup{pdfcontactcountry={Switzerland}}
\hypersetup{pdfcontactemail={nbeisert@itp.phys.ethz.ch}}
\hypersetup{pdfcontacturl={http://people.phys.ethz.ch/\xmptilde nbeisert/}}

\newcommand{\secref}[1]{\hyperref[#1]{section \ref*{#1}}}

\parskip1ex
\parindent0pt
\let\olditemize\itemize
\def\itemize{\olditemize\parskip0pt}

\begin{document}

\title{The \textsf{childdoc} Package}
\hypersetup{pdftitle={The childdoc Package}}
\author{Niklas Beisert\\[2ex]
  Institut f\"ur Theoretische Physik\\
  Eidgen\"ossische Technische Hochschule Z\"urich\\
  Wolfgang-Pauli-Strasse 27, 8093 Z\"urich, Switzerland\\[1ex]
  \href{mailto:nbeisert@itp.phys.ethz.ch}
  {\texttt{nbeisert@itp.phys.ethz.ch}}}
\hypersetup{pdfauthor={Niklas Beisert}}
\hypersetup{pdfsubject={Manual for the LaTeX2e Package childdoc}}
\date{30 December 2018, \textsf{v2.0}}
\maketitle

\begin{abstract}\noindent
\textsf{childdoc} is a \LaTeXe{} package
that enables the direct compilation
of document sections included by |\include|
to individual files.
\end{abstract}

\begingroup
\parskip0ex
\tableofcontents
\endgroup

%%%%%%%%%%%%%%%%%%%%%%%%%%%%%%%%%%%%%%%%%%%%%%%%%%%%%%%%%%%%%%%%%%%%%%%%%%%%%%%%
%%%%%%%%%%%%%%%%%%%%%%%%%%%%%%%%%%%%%%%%%%%%%%%%%%%%%%%%%%%%%%%%%%%%%%%%%%%%%%%%
\section{Introduction}

\LaTeX{} provides a mechanism to structure a large document (such as a book)
into a main file and several child files (containing the chapters)
using the |\include| command.
This mechanism is beneficial for documents
which span hundreds of pages in order to
make the source file(s) more manageable.
Moreover, compilation can be restricted to
selected child files by means of the |\includeonly| command.
The latter feature can be used to reduce the compilation time while editing
(this was significantly more useful in the earlier days of \LaTeX{})
or to generate a smaller document which is easier to navigate.
Another application of |\includeonly| is to generate
documents consisting of selected parts of the complete document.

However, there are a few drawbacks of the plain |\include| mechanism:
\begin{itemize}
\item
The child files cannot be compiled on their own,
they can only be compiled via the main file.
A naive editing environment
(such as a text editor with an option
to have the current file processed by \LaTeX)
may require one to switch to the main file before compiling;
attempting to compile the child file produces errors.
\item
The main file must be modified (each time)
to adjust the |\includeonly| command
to the present needs. This easily leaves the main file in a messy state.
\item
The generated document will always carry the filename
of the main document. This is inconvenient if
several child files are to be compiled and
to be kept for distribution.
\end{itemize}

The present package provides a simple interface
to make child files individually compilable by \LaTeX{}.
Compiling a child file then has the same effect as compiling
the main file with an |\includeonly| command
to select the appropriate child.
Moreover the generated document will carry the name of the child
rather than the main file.
This resolves all three above issues.

This feature is meant to make the editing of books,
thesis documents and lecture notes somewhat more convenient.
However, the package can also be used efficiently for
composing a series of documents (such as exercise sheets)
which are typically distributed individually.
It then assists the author in generating the individual documents
(potentially in different versions)
as well as a document containing the collected series.
Another application is in developing style files
or other kinds of included material
where compilation of the style file could redirect
to a sample or test file.

%%%%%%%%%%%%%%%%%%%%%%%%%%%%%%%%%%%%%%%%%%%%%%%%%%%%%%%%%%%%%%%%%%%%%%%%%%%%%%%%
%%%%%%%%%%%%%%%%%%%%%%%%%%%%%%%%%%%%%%%%%%%%%%%%%%%%%%%%%%%%%%%%%%%%%%%%%%%%%%%%
\section{Usage}

First of all, the package \textsf{childdoc} is \emph{not} a standard
\LaTeXe{} |.sty| style file! Therefore it needs to be invoked in
a non-standard way.

%%%%%%%%%%%%%%%%%%%%%%%%%%%%%%%%%%%%%%%%%%%%%%%%%%%%%%%%%%%%%%%%%%%%%%%%%%%%%%%%
\subsection{Included Files}
\label{sec:include}

%%%%%%%%%%%%%%%%%%%%%%%%%%%%%%%%%%%%%%%%
\DescribeMacro{\childdocmain}
To use the package, add the commands
\begin{center}
\begin{tabular}{l}
|\input{childdoc.def}|\\
|\childdocmain{}|\\
\end{tabular}
\end{center}
at the very top of the main \LaTeX{} file,
in particular \emph{before} the |\documentclass| statement!
The argument of |\childdocmain| should be left empty
(but it must be present).

%%%%%%%%%%%%%%%%%%%%%%%%%%%%%%%%%%%%%%%%
\DescribeMacro{\childdocof}
Furthermore, add the commands
\begin{center}
\begin{tabular}{l}
|\input{childdoc.def}|\\
|\childdocof{|\textit{main}|}|\\
\end{tabular}
\end{center}
at the top of every child file \textit{child}
which is included by |\include{|\textit{child}|}|
from within the main file
(or at least for those files to be compiled individually).
The argument \textit{main} must be the filename of the main file.

There are a couple of
considerations in setting up the main and child documents:

%%%%%%%%%%%%%%%%%%%%%%%%%%%%%%%%%%%%%%%%
\paragraph{Restrictions.}

Please note the following restrictions:
\begin{itemize}
\item
|\childdocmain| must be called with one argument \textit{main}
to ensure compatibility with earlier version of the package.
It must either be empty (|\childdocmain{}|)
or precisely match the filename of the main file in which it is specified.
See \secref{sec:detection} for further information.
\item
The filename \textit{main} must be specified without the |.tex| extension.
\item
The filename \textit{main} is case sensitive
(even in case-insensitive file systems)
due to internal string comparison.
\item
The argument \textit{main} should be fully expanded, it cannot be a macro.
\item
Subdirectories and special characters should be avoided in filenames.
\item
The command |\childdocmain{|\textit{main}|}| must be followed by a whitespace.
It should not be followed immediately by another command
or by a comment mark `|%|'.
This is because the \TeX{} parser reads the token immediately following
the argument of |\childdocmain| and puts it
at the beginning of every child section;
however, a white\-space is ignored.
\end{itemize}

%%%%%%%%%%%%%%%%%%%%%%%%%%%%%%%%%%%%%%%%
\paragraph{Content of Main File.}

It is advisable to place all content in the child files included by |\include|.
Any output contained in the main file will appear in all child documents
unless suppressed manually;
it cannot be suppressed automatically by the |\includeonly| directive
and thus should normally be avoided.
A method to include some content in the main file
by means of conditional processing is described in \secref{sec:conditional}.

%%%%%%%%%%%%%%%%%%%%%%%%%%%%%%%%%%%%%%%%
\paragraph{Page Numbering.}

When only a part of the document is compiled,
the appropriate numbering of pages
(as well as other status parameters)
is determined from the |.aux| files.
The latter contain information from previous passes.
However this information needs to propagate through
all intermediate child documents.
Therefore the page numbering in child documents may well
be inconsistent until the complete document is compiled at least once.

A useful (if unconventional) way to always ensure a consistent
page numbering is to restart the numbering in each child document
and denote the pages by `\textit{child}|.|\textit{page}'
where \textit{child} represents the chapter/section number of the child file.
This can be achieved by the command
|\numberwithin{page}{|\textit{child}|}|
of the \textsf{amsmath} package
where \textit{child} can be |chapter| or |section|
depending on the chosen structuring.
Alternatively, one can modify the macro |\thepage| appropriately
and reset the counter |page| at the start of each child file.

%%%%%%%%%%%%%%%%%%%%%%%%%%%%%%%%%%%%%%%%%%%%%%%%%%%%%%%%%%%%%%%%%%%%%%%%%%%%%%%%
\subsection{Conditional Processing}
\label{sec:conditional}

The package provides a mechanism to compile different versions
of a document. To customise the versions further some conditional processing
can come in handy to distinguish which version is being compiled.
The package provides two macros to describe the compilation context:

%%%%%%%%%%%%%%%%%%%%%%%%%%%%%%%%%%%%%%%%
\DescribeMacro{\ifchilddoc}
The conditional |\ifchilddoc| distinguishes between the compilation of
child documents and the main document:
%
\begin{center}
|\ifchilddoc |\textit{child-code}| |[|\||else |\textit{main-code}]| \||fi|
\end{center}

%%%%%%%%%%%%%%%%%%%%%%%%%%%%%%%%%%%%%%%%
\DescribeMacro{\childdocname}
\DescribeMacro{\childdocjob}
The macro |\childdocname| contains the filename (without extension)
of the main or child file being processed.
Note that |\childdocjob| will always contain the name of the main file.

%%%%%%%%%%%%%%%%%%%%%%%%%%%%%%%%%%%%%%%%
\paragraph{Title Page.}

Conditional processing can be used to include a title or banner page
in the main document when proper precautions are taken.
Importantly, the code in the main file should ensure that the page counter
(as well as other status parameters which are stored in the |.aux| files)
takes the same value after the conditional processing.
Otherwise the page numbers may take divergent values
depending on which part is compiled.

For example, a title page could be declared by:
%
\begin{center}
\begin{tabular}{l}
|\ifchilddoc\||else|\\
|\addtocounter{page}{-1}|\\
\textit{code for title page}\\
|\newpage|\\
|\||fi|
\end{tabular}
\end{center}
%
A banner page for the child documents can be generated by:
%
\begin{center}
\begin{tabular}{l}
|\ifchilddoc|\\
|\addtocounter{page}{-1}|\\
\textit{code for banner page}\\
|\newpage|\\
|\||fi|
\end{tabular}
\end{center}
%
Here one could write a message such as:
\begin{center}
|This is the part \childdocname{} of \childdocjob{}.|
\end{center}

%%%%%%%%%%%%%%%%%%%%%%%%%%%%%%%%%%%%%%%%%%%%%%%%%%%%%%%%%%%%%%%%%%%%%%%%%%%%%%%%
\subsection{Flags}
\label{sec:flags}

The package makes it easy to generate different versions
of the main or child documents.
To this end compilation flags can be defined
and assigned different default values.
They will be particularly useful in conjunction
with the forwarding mechanism described in \secref{sec:forward}.

For example, it may be useful to have a flag |\version|
which can be set to |draft| or |final|.
The document source will contain some conditional code
depending on the value of |\version|.
Suppose further, the flag should default to |final| for the main file
and to |draft| for child files
which is a natural assignment for editing the document.
This is achieved by placing the following code
in the preamble of the main document
(below the |\childdocmain| directive):
%
\begin{center}
\begin{tabular}{l}
|\ifchilddoc|\\
|\providecommand{\version}{draft}|\\
|\||else|\\
|\providecommand{\version}{final}|\\
|\||fi|
\end{tabular}
\end{center}
%
The definition by |\providecommand| makes sure
that previous definitions are not overwritten.
Further statements |\providecommand{\version}{...}|
can thus be added before the above code to override it.

For the main file, one might add a line
(between |\childdocmain| and the above block)
%
\begin{center}
|%\ifchilddoc\||else\providecommand{\version}{draft}\||fi|
\end{center}
%
which can be uncommented to produce a draft version.
Likewise one can add a line to the very top of a child file
(above the |\childdocof{|\textit{main}|}| directive)
%
\begin{center}
|%\providecommand{\version}{final}|
\end{center}
%
which can be uncommented to produce the final version of this child document.

%%%%%%%%%%%%%%%%%%%%%%%%%%%%%%%%%%%%%%%%%%%%%%%%%%%%%%%%%%%%%%%%%%%%%%%%%%%%%%%%
\subsection{Forwarding}
\label{sec:forward}

Different versions of the main or child documents
using compilation flags as described in \secref{sec:flags}
can be (permanently) stored in different files
for convenient compilation, viewing and distribution.
To this end, the package defines a command
to pass on compilation to a different file:

%%%%%%%%%%%%%%%%%%%%%%%%%%%%%%%%%%%%%%%%
\DescribeMacro{\childdocforward}
The command |\childdocforward| redirects processing to
another source file:
%
\begin{center}
\begin{tabular}{l}
|\input{childdoc.def}|\\
|\childdocforward[|\textit{main}|]{|\textit{dest}|}|\\
\end{tabular}
\end{center}
%
The argument \textit{dest} is the destination file
(without extension).
It should be the main file or one of the child files.
Note that further \textsf{childdoc} directives
such as |\childdocof| and |\childdocforward|
in the indicated file will be processed in this form.
The optional argument \textit{main}
passes on directly to the main file \textit{main}
while pretending to compile the child \textit{dest}.
This form behaves as if \textit{dest}
issues |\childdocof{|\textit{main}|}| right away,
and no further \textsf{childdoc} directives will be processed.

%%%%%%%%%%%%%%%%%%%%%%%%%%%%%%%%%%%%%%%%
\DescribeMacro{\...prefix}
In the alternative form |\childdocforwardprefix|,
%
\begin{center}
\begin{tabular}{l}
|\input{childdoc.def}|\\
|\childdocforwardprefix[|\textit{main}|]{|\textit{prefix}|}{|\textit{dest}|}|
\end{tabular}
\end{center}
%
the destination file is determined by a pattern
depending on the current file:
To make this work, the current file must be called
`{\textit{prefix}\hspace{0.2em}\textit{suffix}}'
with \textit{prefix} matching precisely the argument.
Processing is then passed on to the file
`{\textit{dest}\hspace{0.2em}\textit{suffix}}'.
Surely, the same effect is achieved by
directly specifying the
argument `{\textit{dest}\hspace{0.2em}\textit{suffix}}'
in the first form.
However, that requires to set up a different file
for each child. With the alternative form of the command
all these files can have exactly the same content
which simplifies setting them up and maintaining them.

For example, the following file |draft.tex|
with a compilation flag |\version| as described in \secref{sec:flags}
compiles the main document as a draft:
%
\begin{center}
\begin{tabular}{l}
|\def\version{draft}|\\
|\input{childdoc.def}|\\
|\childdocforward{|\textit{main}|}|
\end{tabular}
\end{center}
%
Likewise, the following files |final|\textit{nn}|.tex|
compile the final version of the child document
|child|\textit{nn}|.tex|:
%
\begin{center}
\begin{tabular}{l}
|\def\version{final}|\\
|\input{childdoc.def}|\\
|\childdocforwardprefix{final}{child}|
\end{tabular}
\end{center}
%

Note that when several versions of a main file and/or of each child file
are to be generated, it may be convenient to set up a |Makefile| or
shell script to automatise the process.

%%%%%%%%%%%%%%%%%%%%%%%%%%%%%%%%%%%%%%%%%%%%%%%%%%%%%%%%%%%%%%%%%%%%%%%%%%%%%%%%
\subsection{Command Line Processing}
\label{sec:commandline}

The effect of redirection files can also be achieved by invoking
the \LaTeX{} compiler with a more elaborate command line.
Most conveniently this should be done as part
of a shell script or a |Makefile|.

When using \textsf{childdoc} in the main file, the following
command lines effectively perform a redirection
(note that depending on the shell being used,
backslashes may have to be doubled: `|\|' $\to$ `|\\|'):
%
\begin{center}
|... -jobname "|\textit{target}|" |\\|"|[\textit{flags}]%
|\input{childdoc.def}\childdocforward[|\textit{main}|]{|\textit{dest}|}"|
\end{center}
%
Here \textit{target} is the name of the output file,
\textit{main} is the name of the main file
and \textit{dest} is the name of the main or child file to be processed
(all filenames without extensions).
The optional argument \textit{main} can be omitted
if \textit{main} matches \textit{dest}.
Optionally, compilation \textit{flags} can be defined via |\def| commands.
This command line makes the \TeX{} engine believe
it is compiling the file \textit{target}
whose content is specified as the latter parameter.
The provided code then forwards the processing to
\textit{main} or \textit{dest} as described in \secref{sec:forward}.

%%%%%%%%%%%%%%%%%%%%%%%%%%%%%%%%%%%%%%%%%%%%%%%%%%%%%%%%%%%%%%%%%%%%%%%%%%%%%%%%
\subsection{Include by Input}
\label{sec:input}

Including child documents by |\include| has some restrictions by design.
Most notably, the content of a child document always occupies
its own set of pages; pages cannot be shared between child documents.
Usually, this behaviour makes perfect sense
because each child document contain an essential part of the document.
However, in some situations it may be desirable to compose
a document from a collection of parts
without having mandatory page breaks between then.
For this case, the package
provides a mechanism to include parts
by |\input| which can also be processed individually.
However, by construction this mechanism
requires manual handling of the content to be output.

%%%%%%%%%%%%%%%%%%%%%%%%%%%%%%%%%%%%%%%%
\DescribeMacro{\ifchilddocmanual}
The main file should be prepared as usual, see \secref{sec:include}.
However, the document body must make a distinction
between processing of an individual part and of the main document, e.g.:
%
\begin{center}
\begin{tabular}{l}
|\ifchilddocmanual|\\
|\input{\childdocname}|\\
|\||else|\\
\textit{document body with }|\input{|\textit{part}|}|\\
|\||fi|
\end{tabular}
\end{center}
%
The conditional |\ifchilddocmanual| is true whenever
a part to be included by |\input| is being compiled,
and the name of the part is stored in |\childdocname|.

%%%%%%%%%%%%%%%%%%%%%%%%%%%%%%%%%%%%%%%%
\DescribeMacro{\childdocby}
Each part to be included by |\input| should start with:
%
\begin{center}
\begin{tabular}{l}
|\input{childdoc.def}|\\
|\childdocby{|\textit{main}|}|\\
\end{tabular}
\end{center}
%
The directive |\childdocby| is similar to |\childdocof|
described in \secref{sec:include},
but the subsequent selection of content must be done manually.
To that end, both |\ifchilddoc| and |\ifchilddocmanual|
will be true upon processing of a part,
and the name of the part is stored in |\childdocname|.
Note that |\jobname| will be set to the filename of the current part
so that each part receives an individual |.aux| file
that does not interfere with the |.aux| file(s) of the main document.
This behaviour can be altered by the alternative form
|\childdocby[*]{|\textit{main}|}| (with a non-empty optional argument)
which uses the |.aux| file of the main document
by setting |\jobname| to \textit{main}.

%%%%%%%%%%%%%%%%%%%%%%%%%%%%%%%%%%%%%%%%%%%%%%%%%%%%%%%%%%%%%%%%%%%%%%%%%%%%%%%%
\subsection{Driver Development}
\label{sec:driver}

The \textsf{childdoc} mechanism can also be use for the development
of definition files such as \LaTeX{} styles or classes.
This case differs from the above setup with multiple parts
included by |\include| in that no |\includeonly| should be invoked.
This can be achieved by starting the include file
(before |\ProvidesPackage|) with:
%
\begin{center}
\begin{tabular}{l}
|\input{childdoc.def}|\\
|\childdocforward{|\textit{main}|}|\\
\end{tabular}
\end{center}
%
or alternatively with:
%
\begin{center}
\begin{tabular}{l}
|\input{childdoc.def}|\\
|\childdocby{|\textit{main}|}|\\
\end{tabular}
\end{center}
%
Both forms have slightly different effects as described above.
The main file is prepared as usual, see \secref{sec:include}.

%%%%%%%%%%%%%%%%%%%%%%%%%%%%%%%%%%%%%%%%%%%%%%%%%%%%%%%%%%%%%%%%%%%%%%%%%%%%%%%%
\subsection{Legacy Detection}
\label{sec:detection}

The directive |\childdocmain| in the main file can detect
whether the complete document or merely a child is to be compiled
even without using the directive |\childdocof|.
This method is deprecated because it is less robust
and there is no compelling reason to use it;
it is merely provided for backward compatibility
and it may be removed in future versions.

If the detection mechanism is to be used,
it is mandatory to correctly specify
the filename of the main file as the argument of |\childdocmain|:
%
\begin{center}
\begin{tabular}{l}
|\input{childdoc.def}|\\
|\childdocmain{|\textit{main}|}|\\
\end{tabular}
\end{center}
%
If |\jobname| does not match the argument \textit{main} of |\childdocmain|,
it is assumed that |\jobname| points to the child file to be compiled.
When using |\childdocmain| with the main file specified as argument,
it suffices to start a child file
with just |\input{|\textit{main}|}|
without loading of the package and using |\childdocof|.
If instead all processing is done
with the appropriate \textsf{childdoc} directives,
the argument of \textit{main} of |\childdocmain| can be empty.

An alternative version of the command line processing described
in \secref{sec:commandline} using the detection mechanism reads:
%
\begin{center}
|... -jobname "|\textit{target}|" "|[\textit{flags}]%
[|\def\jobname{|\textit{dest}|}|]|\input{|\textit{main}|}"|
\end{center}

%%%%%%%%%%%%%%%%%%%%%%%%%%%%%%%%%%%%%%%%%%%%%%%%%%%%%%%%%%%%%%%%%%%%%%%%%%%%%%%%
\subsection{Manual Code}
\label{sec:manual}

In case one cannot be certain whether the definitions file |childdoc.def|
is installed on the target \TeX{} distribution
and one prefers not to ship it,
it is conceivable to paste a few relevant commands into the sources.

To that end, drop all statements |\input{childdoc.def}|
and perform the replacements as outlined below.
Instead of |\childdocmain{|\textit{main}|}| add the following code
to the top of the main file:
%
\begin{center}
\begin{tabular}{l}
|\||ifdefined\childdocname\endinput\||fi\newif\ifchilddoc|\\
|\edef\childdocname{\scantokens\expandafter{\jobname\noexpand}}|\\
|\def\childdocmain{|\textit{main}|}\||ifx\childdocmain\childdocname\||else|\\
|\childdoctrue\includeonly{\childdocname}\let\jobname\childdocmain\||fi|\\
\end{tabular}
\end{center}
%
Instead of |\childdocof{|\textit{main}|}| just include the main file
at the top of each child file:
%
\begin{center}
|\input{|\textit{main}|}|
\end{center}
%
A simple redirection |\childdocforward{|\textit{dest}|}| is achieved by:
%
\begin{center}
|\def\jobname{|\textit{dest}|}\input{\jobname}|
\end{center}
%
The redirection with prefix
|\childdocforwardprefix[|\textit{prefix}|]{|\textit{dest}|}|
is accomplished by:
%
\begin{center}
\begin{tabular}{l}
|{\edef\jobname{\scantokens\expandafter{\jobname\noexpand}}|\\
|\def\redirectjob |\textit{prefix}|#1~~~{\gdef\jobname{|\textit{dest}|#1}}|\\
|\expandafter\redirectjob\jobname~~~}\input{\jobname}|
\end{tabular}
\end{center}

In an alternative approach,
child documents can be compiled by a specific command line
without additional code or specific definitions:
%
\begin{center}
|... -jobname "|\textit{target}|" "|[\textit{flags}]%
|\includeonly{|\textit{dest}|}\input{|\textit{main}|}"|
\end{center}
%

%%%%%%%%%%%%%%%%%%%%%%%%%%%%%%%%%%%%%%%%%%%%%%%%%%%%%%%%%%%%%%%%%%%%%%%%%%%%%%%%
%%%%%%%%%%%%%%%%%%%%%%%%%%%%%%%%%%%%%%%%%%%%%%%%%%%%%%%%%%%%%%%%%%%%%%%%%%%%%%%%
\section{Information}

%%%%%%%%%%%%%%%%%%%%%%%%%%%%%%%%%%%%%%%%%%%%%%%%%%%%%%%%%%%%%%%%%%%%%%%%%%%%%%%%
\subsection{Copyright}

Copyright \copyright{} 2017--2018 Niklas Beisert

This work may be distributed and/or modified under the
conditions of the \LaTeX{} Project Public License, either version 1.3
of this license or (at your option) any later version.
The latest version of this license is in
  \url{http://www.latex-project.org/lppl.txt}
and version 1.3 or later is part of all distributions of \LaTeX{}
version 2005/12/01 or later.

This work has the LPPL maintenance status `maintained'.

The Current Maintainer of this work is Niklas Beisert.

This work consists of the files |README.txt|, |childdoc.ins| and |childdoc.dtx|
as well as the derived files |childdoc.def|, |cdocsamp.tex|
with |cdocsch1.tex|, |cdocsch2.tex|, |cdocspt3.tex|, |cdocspt4.tex|,
|cdocsdrf.tex|, |cdocsfn1.tex|, |cdocsfn2.tex|
as well as |childdoc.pdf|.

%%%%%%%%%%%%%%%%%%%%%%%%%%%%%%%%%%%%%%%%%%%%%%%%%%%%%%%%%%%%%%%%%%%%%%%%%%%%%%%%
\subsection{Files and Installation}

The package consists of the files:
%
\begin{center}
\begin{tabular}{ll}
    |README.txt|   & readme file \\
    |childdoc.ins| & installation file \\
    |childdoc.dtx| & source file \\
    |childdoc.def| & definition file \\
    |cdocsamp.tex| & sample main file \\
    |cdocsch1.tex| & sample include file \\
    |cdocsch2.tex| & sample include file \\
    |cdocspt3.tex| & sample part file \\
    |cdocspt4.tex| & sample part file \\
    |cdocsdrf.tex| & sample redirection file \\
    |cdocsfn1.tex| & sample redirection file \\
    |cdocsfn2.tex| & sample redirection file \\
    |childdoc.pdf| & manual
\end{tabular}
\end{center}
%
The distribution consists of the files
|README.txt|, |childdoc.ins| and |childdoc.dtx|.
%
\begin{itemize}
\item
Run (pdf)\LaTeX{} on |childdoc.dtx|
to compile the manual |childdoc.pdf| (this file).
\item
Run \LaTeX{} on |childdoc.ins| to create the definitions file |childdoc.def|
and the sample |cdocsamp.tex| with include files
|cdocsch1.tex|, |cdocsch2.tex|, |cdocspt3.tex|, |cdocspt4.tex|,
|cdocsdrf.tex|, |cdocsfn1.tex|, |cdocsfn2.tex|.
Then copy the file |childdoc.def| to an appropriate directory of your \LaTeX{}
distribution, e.g.\ \textit{texmf-root}|/tex/latex/childdoc|.
\end{itemize}

%%%%%%%%%%%%%%%%%%%%%%%%%%%%%%%%%%%%%%%%%%%%%%%%%%%%%%%%%%%%%%%%%%%%%%%%%%%%%%%%
\subsection{Related CTAN Packages}

There are several other packages which offer a similar functionality:
%
\begin{itemize}
\item
The packages
\href{http://ctan.org/pkg/docmute}{\textsf{docmute}},
\href{http://ctan.org/pkg/includex}{\textsf{includex}} and
\href{http://ctan.org/pkg/standalone}{\textsf{standalone}}
provide commands to include only the document body of
a child file thus allowing both files to be compiled individually.
\item
The packages \href{http://ctan.org/pkg/subdocs}{\textsf{subdocs}}
and \href{http://ctan.org/pkg/subfiles}{\textsf{subfiles}}
provide structures in which the main and child documents can be
encapsulated and allowing them to be compiled individually.
The inclusion mechanism is different from the conventional |\include|.
\item
The package \href{http://ctan.org/pkg/combine}{\textsf{combine}}
is an elaborate solution to combine several documents into one.
\end{itemize}
%
See also the CTAN topic \href{http://ctan.org/topic/subdocs}{\textsf{subdocs}}
for further related packages.
The present package differs from the above solutions in that
a document structure constructed with the conventional |\include| mechanism
just needs two extra commands at the top of every file
such that all constituent files can be compiled individually.

%%%%%%%%%%%%%%%%%%%%%%%%%%%%%%%%%%%%%%%%%%%%%%%%%%%%%%%%%%%%%%%%%%%%%%%%%%%%%%%%
%\subsection{Feature Suggestions}
%
%The following is a list of features which may be useful for future
%versions of this package:
%%
%\begin{itemize}
%\item
%\ldots
%\end{itemize}

%%%%%%%%%%%%%%%%%%%%%%%%%%%%%%%%%%%%%%%%%%%%%%%%%%%%%%%%%%%%%%%%%%%%%%%%%%%%%%%%
\subsection{Revision History}

%%%%%%%%%%%%%%%%%%%%%%%%%%%%%%%%%%%%%%%%
\paragraph{v2.0:} 2018/12/30

\begin{itemize}
\item
immediate forward processing
\item
added |\childdocby| mechanism
\item
manual restructured
\end{itemize}

%%%%%%%%%%%%%%%%%%%%%%%%%%%%%%%%%%%%%%%%
\paragraph{v1.6:} 2018/01/17

\begin{itemize}
\item
application for development of include files
\item
corrections to manual
\end{itemize}

%%%%%%%%%%%%%%%%%%%%%%%%%%%%%%%%%%%%%%%%
\paragraph{v1.5:} 2017/05/21

\begin{itemize}
\item
more complete structuring introduced
\item
|\childdocof| introduced
\item
|\childdoc| renamed to |\childdocmain|
\item
|\childredirect| renamed to |\childdocforward| and |\childdocforwardprefix|
and functionality expanded
\end{itemize}

%%%%%%%%%%%%%%%%%%%%%%%%%%%%%%%%%%%%%%%%
\paragraph{v1.0:} 2017/04/27

\begin{itemize}
\item
manual and install package
\item
first version published on CTAN
\end{itemize}

%%%%%%%%%%%%%%%%%%%%%%%%%%%%%%%%%%%%%%%%
\paragraph{v0.6:} 2017/04/26

\begin{itemize}
\item
redirection mechanism added
\end{itemize}

%%%%%%%%%%%%%%%%%%%%%%%%%%%%%%%%%%%%%%%%
\paragraph{v0.5:} 2017/04/26

\begin{itemize}
\item
functionality in definition file
\end{itemize}


%%%%%%%%%%%%%%%%%%%%%%%%%%%%%%%%%%%%%%%%%%%%%%%%%%%%%%%%%%%%%%%%%%%%%%%%%%%%%%%%
%%%%%%%%%%%%%%%%%%%%%%%%%%%%%%%%%%%%%%%%%%%%%%%%%%%%%%%%%%%%%%%%%%%%%%%%%%%%%%%%
%%%%%%%%%%%%%%%%%%%%%%%%%%%%%%%%%%%%%%%%%%%%%%%%%%%%%%%%%%%%%%%%%%%%%%%%%%%%%%%%
\appendix

\settowidth\MacroIndent{\rmfamily\scriptsize 000\ }

 \DocInput{childdoc.dtx}

\end{document}
%</driver>
% \fi
%
% %%%%%%%%%%%%%%%%%%%%%%%%%%%%%%%%%%%%%%%%%%%%%%%%%%%%%%%%%%%%%%%%%%%%%%%%%%%%%%
% %%%%%%%%%%%%%%%%%%%%%%%%%%%%%%%%%%%%%%%%%%%%%%%%%%%%%%%%%%%%%%%%%%%%%%%%%%%%%%
% \section{Sample}
%\iffalse
%<*samplemain>
%\fi
%
% The following presents a sample document
% with two chapters, two parts, a title page,
% a compile flag as well as three forwarding files to set the flag.
% It consists of eight |.tex| files:
% \begin{center}
% \begin{tabular}{ll}
% |cdocsamp.tex|&main file\\
% |cdocsch1.tex|&include file for chapter 1\\
% |cdocsch2.tex|&include file for chapter 2\\
% |cdocspt3.tex|&include file for part 3\\
% |cdocspt4.tex|&include file for part 4\\
% |cdocsdrf.tex|&forwarding file for main file in draft mode\\
% |cdocsfi1.tex|&forwarding file for final version of chapter 1\\
% |cdocsfi2.tex|&forwarding file for final version of chapter 2\\
% \end{tabular}
% \end{center}
% Each of the eight files can be compiled directly by the \LaTeX{} compiler.
%
% %%%%%%%%%%%%%%%%%%%%%%%%%%%%%%%%%%%%%%
% \paragraph{Main File.}
%
% The main file is called |cdocsamp.tex|.
%
% Load the \textsf{childdoc} definitions and
% declare the filename for the main document:
%    \begin{macrocode}
\input{childdoc.def}
\childdocmain{}
%    \end{macrocode}

% Optional override for |\version| flag:
%    \begin{macrocode}
%%\ifchilddoc\else\providecommand{\version}{draft}\fi
%    \end{macrocode}

% Define the default values for the |\version| flag
% (|final| for the main file and |draft| for childs):
%    \begin{macrocode}
\ifchilddoc
\providecommand{\version}{draft}
\else
\providecommand{\version}{final}
\fi
%    \end{macrocode}

% Load the standard document class:
%    \begin{macrocode}
\documentclass[12pt]{article}
%    \end{macrocode}

% Start the document body:
%    \begin{macrocode}
\begin{document}
%    \end{macrocode}

% Declare a title page.
% Print title, part of document being processed and version flag:
%    \begin{macrocode}
\addtocounter{page}{-1}
\begin{center}
{\LARGE\bfseries{}childdoc example\par}
\vspace{1cm}
\ifchilddoc
\ifchilddocmanual part\else chapter\fi:
`\childdocname' of `\childdocjob'\par
\else
main document: `\childdocjob'\par
\fi
version: \version\par
\end{center}
\newpage
%    \end{macrocode}

% Manually include selected file,
% otherwise process as usual:
%    \begin{macrocode}
\ifchilddocmanual
\section*{part `\childdocname'}
\input{\childdocname}
\else
%    \end{macrocode}

% Include the two chapters:
%    \begin{macrocode}
\include{cdocsch1}
\include{cdocsch2}
%    \end{macrocode}

% Include the two parts unless only chapters should be displayed:
%    \begin{macrocode}
\ifchilddoc\else
\section{part three}
\input{cdocspt3}
\section{part four}
\input{cdocspt4}
\fi
%    \end{macrocode}

% Process as usual until here:
%    \begin{macrocode}
\fi
%    \end{macrocode}

% End of document body:
%    \begin{macrocode}
\end{document}
%    \end{macrocode}
%\iffalse
%</samplemain>
%\fi
%
% %%%%%%%%%%%%%%%%%%%%%%%%%%%%%%%%%%%%%%
% \paragraph{Chapter Include Files.}
%
% The include files are called |cdocsch1.tex| and |cdocsch2.tex|.
%
%\iffalse
%<*samplechap1|samplechap2>
%\fi

% Optional override for |\version| flag:
%    \begin{macrocode}
%%\providecommand{\version}{final}
%    \end{macrocode}

% Include the main document:
%    \begin{macrocode}
\input{childdoc.def}
\childdocof{cdocsamp}
%    \end{macrocode}

%\iffalse
%</samplechap1|samplechap2>
%\fi
%
%\iffalse
%<*samplechap1>
%\fi
% Some text for chapter 1:
%    \begin{macrocode}
\section{one}
some text in chapter one
%    \end{macrocode}

%\iffalse
%</samplechap1>
%\fi
% Some text for chapter 2:
%\iffalse
%<*samplechap2>
%\fi
%    \begin{macrocode}
\section{two}
more text in chapter two
%    \end{macrocode}

%\iffalse
%</samplechap2>
%\fi
%
% %%%%%%%%%%%%%%%%%%%%%%%%%%%%%%%%%%%%%%
% \paragraph{Part Include Files.}
%
% The include files are called |cdocspt3.tex| and |cdocspt4.tex|.
%
%\iffalse
%<*samplepart3|samplepart4>
%\fi

% Optional override for |\version| flag:
%    \begin{macrocode}
%%\providecommand{\version}{final}
%    \end{macrocode}

% Include the main document:
%    \begin{macrocode}
\input{childdoc.def}
\childdocby{cdocsamp}
%    \end{macrocode}

%\iffalse
%</samplepart3|samplepart4>
%\fi
%
%\iffalse
%<*samplepart3>
%\fi
% Some text for part 3:
%    \begin{macrocode}
some text in part three
%    \end{macrocode}

%\iffalse
%</samplepart3>
%\fi
% Some text for part 4:
%\iffalse
%<*samplepart4>
%\fi
%    \begin{macrocode}
more text in part four
%    \end{macrocode}

%\iffalse
%</samplepart4>
%\fi
%
% %%%%%%%%%%%%%%%%%%%%%%%%%%%%%%%%%%%%%%
% \paragraph{Forwarding for a Complete Draft.}
%
% The following forwarding file |cdocsdrf.tex|
% compiles the main document in draft mode:
%\iffalse
%<*sampledraft>
%\fi
%    \begin{macrocode}
\def\version{draft}
\input{childdoc.def}
\childdocforward{cdocsamp}
%    \end{macrocode}

%\iffalse
%</sampledraft>
%\fi
%
% %%%%%%%%%%%%%%%%%%%%%%%%%%%%%%%%%%%%%%
% \paragraph{Forwarding for Final Version of the Chapters.}
%
% The following forwarding files |cdocsfn1.tex| and |cdocsfn2.tex|
% (with identical content)
% compile the final versions of the child documents
% |cdocsch1.tex| and |cdocsch2.tex|, respectively:
%\iffalse
%<*samplefinal>
%\fi
%    \begin{macrocode}
\def\version{final}
\input{childdoc.def}
\childdocforwardprefix[cdocsamp]{cdocsfn}{cdocsch}
%    \end{macrocode}

%\iffalse
%</samplefinal>
%\fi
%
% %%%%%%%%%%%%%%%%%%%%%%%%%%%%%%%%%%%%%%
% \paragraph{Command Line Processing.}
%
% The following three command lines generate the output files
% |cdocscld|, |cdocscl1| and |cdocscl2|
% which should be identical to
% |cdocsdrf|, |cdocsch1| and |cdocsfn2|, respectively:
% \begin{center}
% \begin{tabular}{l}
% |latex -jobname cdocscld \|\\
% |  "\def\version{draft}\input{childdoc.def}\childdocforward{cdocsamp}"|\\
% |latex -jobname cdocscl1 \|\\
% |  "\input{childdoc.def}\childdocforward[cdocsamp]{cdocsch1}"|\\
% |latex -jobname cdocscl2 \|\\
% |  "\def\version{final}\input{childdoc.def}\childdocforward{cdocsch2}"|
% \end{tabular}
% \end{center}
% Note that the trailing backslash on each first line
% merely continues the input to the second line
% (for convenient cut ant paste).
% Furthermore, the command |latex| can be replaced by any
% of its alternative versions such as |pdflatex|.
%
% %%%%%%%%%%%%%%%%%%%%%%%%%%%%%%%%%%%%%%%%%%%%%%%%%%%%%%%%%%%%%%%%%%%%%%%%%%%%%%
% %%%%%%%%%%%%%%%%%%%%%%%%%%%%%%%%%%%%%%%%%%%%%%%%%%%%%%%%%%%%%%%%%%%%%%%%%%%%%%
% \section{Implementation}
%\iffalse
%<*package>
%\fi
%
% This section describes the definitions file |childdoc.def|.

% The definitions cannot be loaded using |\usepackage| or |\RequirePackage|
% which has a mechanism to prevent loading a style file more than once.
% When loading the definitions by means of |\input|
% multiple instances have to be prevented manually:
%\iffalse
%This code needs to be before the `\ProvidesFile' directive
%which is defined at the beginning of this file.
%Therefore it is also placed there and commented out here.
%</package>
%<*discard>
%\fi
%    \begin{macrocode}
\ifdefined\childdocmain\endinput\fi
%    \end{macrocode}
%\iffalse
%</discard>
%<*package>
%\fi
%
% \macro{\ifchilddoc}
% \macro{\ifchilddocmanual}
% The conditional |\ifchilddoc| tells whether a
% child (true) or main (false) document is being compiled.
% The conditional |\ifchilddocmanual| tells whether
% the |\includeonly| mechanism is used (false) or
% the selection of child files must be performed manually (true).
% The definitions initialise to false:
%    \begin{macrocode}
\newif\ifchilddoc
\newif\ifchilddocmanual
%    \end{macrocode}

% \macro{\childdocname}
% \macro{\childdocjob}
% The macro |\childdocname| stores the name of the main document
% to be compiled. The macro |\childdocjob| stores the name of
% the document on which the \LaTeX{} compiler was originally invoked.
% The content of |\jobname| cannot be compared
% to filenames specified in the source due to different catcodes.
% The following code rescans |\jobname|, stores the result
% in |\childdocname| and saves a copy in |\childdocjob|:
%    \begin{macrocode}
\edef\childdocname{\scantokens\expandafter{\jobname\noexpand}}
\let\childdocjob\childdocname
%    \end{macrocode}

% \macro{\childdocdisable}
% The macro |\childdocdisable| prevents the main file
% from being processed more than once.
% At this stage, the main document command |\childdocmain|
% is assumed to be called once again where it should do nothing.
% Any subsequent call to it should prevent
% a secondary processing of the main document
% It overwrites the forwarding commands
% |\childdocof| and |\childdocforward|
% with empty macros to prevent further inclusions of the main document:
%    \begin{macrocode}
\newcommand{\childdocdisable}
{
  \renewcommand{\childdocmain}[1]{\renewcommand{\childdocmain}[1]{\endinput}}
  \renewcommand{\childdocof}[1]{}
  \renewcommand{\childdocby}[2][]{}
  \renewcommand{\childdocforward}[2][]{}
  \renewcommand{\childdocdisable}{}
}
%    \end{macrocode}

% \macro{\childdocmain}
% The macro |\childdocmain| is to be called at the top of the main file
% with nothing or the main filename (without extension) as argument.
% First, it breaks loops.
% If the argument is not empty and does not match |\childdocname|
% (which is set by the first inclusion of |childdoc.def|),
% |\ifchilddoc| is set to true, |\includeonly| is applied to the child file
% and |\jobname| is set to the main file
% (for proper handling of |.aux| files):
%    \begin{macrocode}
\newcommand{\childdocmain}[1]
{
  \childdocdisable\childdocmain{}
  \if?#1?\else
    \begingroup
      \def\childdoctmp{#1}
      \ifx\childdoctmp\childdocname
        \def\childdoctmp{}
      \else
        \def\childdoctmp
        {
          \childdoctrue
          \includeonly{\childdocname}
          \def\childdocjob{#1}
          \def\jobname{#1}
        }
      \fi
      \expandafter
    \endgroup
    \childdoctmp
  \fi
}
%    \end{macrocode}

% \macro{\childdocof}
% The command |\childdocof| redirects
% compilation to the main file |#1|.
%    \begin{macrocode}
\newcommand{\childdocof}[1]
{
  \childdocdisable
  \childdoctrue
  \includeonly{\childdocname}
  \def\jobname{#1}
  \def\childdocjob{#1}
  \input{#1}
}
%    \end{macrocode}

% \macro{\childdocby}
% The command |\childdocby| ....
%    \begin{macrocode}
\newcommand{\childdocby}[2][]
{
  \childdocdisable
  \childdoctrue
  \childdocmanualtrue
  \if?#1?\else
    \def\jobname{#2}
  \fi
  \def\childdocjob{#2}
  \input{#2}
  \endinput
}
%    \end{macrocode}

% \macro{\childdocforward}
% The command |\childdocforward| redirects
% compilation to the main file or
% (if the optional argument is given) a child file.
% Parameters are set as if the main file
% or a child file starting with |\childdocof| was compiled.
% Then compilation is handed over to the main file:
%    \begin{macrocode}
\newcommand{\childdocforward}[2][]
{
  \begingroup
    \if?#1?
      \def\childdoctmp
      {
        \def\childdocname{#2}
        \def\childdocjob{#2}
        \def\jobname{#2}
        \input{#2}
        \endinput
      }
    \else
      \def\childdoctmp
      {
        \childdocdisable
        \def\childdocname{#2}
        \childdoctrue
        \includeonly{#2}
        \def\childdocjob{#1}
        \def\jobname{#1}
        \input{#1}
        \endinput
      }
    \fi
    \expandafter
  \endgroup
  \childdoctmp
}
%    \end{macrocode}

% \macro{\childdocforwardprefix}
% The command |\childdocforwardprefix| redirects
% compilation to the main or a child file by means of a pattern.
% The prefix |#1| in the current filename is replaced by |#2|
% and the suffix of the current filename is kept
% (it is assumed that the filename does not contain the substring `|~~~|'
% which is used as a delimiter).
% Compilation is handed over to the new file by |\childdocforward|:
%    \begin{macrocode}
\newcommand{\childdocforwardprefix}[3][]
{
  \begingroup
    \def\childdocextract #2##1~~~{\def\childdoctmp{\childdocforward[#1]{#3##1}}}
    \expandafter\childdocextract\childdocname~~~
    \expandafter
  \endgroup
  \childdoctmp
}
%    \end{macrocode}

% \macro{\childdoc}
% The deprecated macro |\childdoc| is a legacy version of |\childdocmain|:
%    \begin{macrocode}
\newcommand{\childdoc}{\childdocmain}
%    \end{macrocode}

% \macro{\childdocredirect}
% The deprecated macro |\childdocredirect| is a legacy version
% of |\childdocforward| and |\childdocforwardprefix|:
%    \begin{macrocode}
\newcommand{\childdocredirect}[2][]
{
  \begingroup
    \if?#1?
      \def\childdoctmp{\childdocforward{#2}}
    \else
      \def\childdoctmp{\childdocforwardprefix{#1}{#2}}
    \fi
    \expandafter
  \endgroup
  \childdoctmp
}
%    \end{macrocode}

%\iffalse
%</package>
%\fi
%
\endinput
|
and perform the replacements as outlined below.
Instead of |\childdocmain{|\textit{main}|}| add the following code
to the top of the main file:
%
\begin{center}
\begin{tabular}{l}
|\||ifdefined\childdocname\endinput\||fi\newif\ifchilddoc|\\
|\edef\childdocname{\scantokens\expandafter{\jobname\noexpand}}|\\
|\def\childdocmain{|\textit{main}|}\||ifx\childdocmain\childdocname\||else|\\
|\childdoctrue\includeonly{\childdocname}\let\jobname\childdocmain\||fi|\\
\end{tabular}
\end{center}
%
Instead of |\childdocof{|\textit{main}|}| just include the main file
at the top of each child file:
%
\begin{center}
|\input{|\textit{main}|}|
\end{center}
%
A simple redirection |\childdocforward{|\textit{dest}|}| is achieved by:
%
\begin{center}
|\def\jobname{|\textit{dest}|}\input{\jobname}|
\end{center}
%
The redirection with prefix
|\childdocforwardprefix[|\textit{prefix}|]{|\textit{dest}|}|
is accomplished by:
%
\begin{center}
\begin{tabular}{l}
|{\edef\jobname{\scantokens\expandafter{\jobname\noexpand}}|\\
|\def\redirectjob |\textit{prefix}|#1~~~{\gdef\jobname{|\textit{dest}|#1}}|\\
|\expandafter\redirectjob\jobname~~~}\input{\jobname}|
\end{tabular}
\end{center}

In an alternative approach,
child documents can be compiled by a specific command line
without additional code or specific definitions:
%
\begin{center}
|... -jobname "|\textit{target}|" "|[\textit{flags}]%
|\includeonly{|\textit{dest}|}\input{|\textit{main}|}"|
\end{center}
%

%%%%%%%%%%%%%%%%%%%%%%%%%%%%%%%%%%%%%%%%%%%%%%%%%%%%%%%%%%%%%%%%%%%%%%%%%%%%%%%%
%%%%%%%%%%%%%%%%%%%%%%%%%%%%%%%%%%%%%%%%%%%%%%%%%%%%%%%%%%%%%%%%%%%%%%%%%%%%%%%%
\section{Information}

%%%%%%%%%%%%%%%%%%%%%%%%%%%%%%%%%%%%%%%%%%%%%%%%%%%%%%%%%%%%%%%%%%%%%%%%%%%%%%%%
\subsection{Copyright}

Copyright \copyright{} 2017--2018 Niklas Beisert

This work may be distributed and/or modified under the
conditions of the \LaTeX{} Project Public License, either version 1.3
of this license or (at your option) any later version.
The latest version of this license is in
  \url{http://www.latex-project.org/lppl.txt}
and version 1.3 or later is part of all distributions of \LaTeX{}
version 2005/12/01 or later.

This work has the LPPL maintenance status `maintained'.

The Current Maintainer of this work is Niklas Beisert.

This work consists of the files |README.txt|, |childdoc.ins| and |childdoc.dtx|
as well as the derived files |childdoc.def|, |cdocsamp.tex|
with |cdocsch1.tex|, |cdocsch2.tex|, |cdocspt3.tex|, |cdocspt4.tex|,
|cdocsdrf.tex|, |cdocsfn1.tex|, |cdocsfn2.tex|
as well as |childdoc.pdf|.

%%%%%%%%%%%%%%%%%%%%%%%%%%%%%%%%%%%%%%%%%%%%%%%%%%%%%%%%%%%%%%%%%%%%%%%%%%%%%%%%
\subsection{Files and Installation}

The package consists of the files:
%
\begin{center}
\begin{tabular}{ll}
    |README.txt|   & readme file \\
    |childdoc.ins| & installation file \\
    |childdoc.dtx| & source file \\
    |childdoc.def| & definition file \\
    |cdocsamp.tex| & sample main file \\
    |cdocsch1.tex| & sample include file \\
    |cdocsch2.tex| & sample include file \\
    |cdocspt3.tex| & sample part file \\
    |cdocspt4.tex| & sample part file \\
    |cdocsdrf.tex| & sample redirection file \\
    |cdocsfn1.tex| & sample redirection file \\
    |cdocsfn2.tex| & sample redirection file \\
    |childdoc.pdf| & manual
\end{tabular}
\end{center}
%
The distribution consists of the files
|README.txt|, |childdoc.ins| and |childdoc.dtx|.
%
\begin{itemize}
\item
Run (pdf)\LaTeX{} on |childdoc.dtx|
to compile the manual |childdoc.pdf| (this file).
\item
Run \LaTeX{} on |childdoc.ins| to create the definitions file |childdoc.def|
and the sample |cdocsamp.tex| with include files
|cdocsch1.tex|, |cdocsch2.tex|, |cdocspt3.tex|, |cdocspt4.tex|,
|cdocsdrf.tex|, |cdocsfn1.tex|, |cdocsfn2.tex|.
Then copy the file |childdoc.def| to an appropriate directory of your \LaTeX{}
distribution, e.g.\ \textit{texmf-root}|/tex/latex/childdoc|.
\end{itemize}

%%%%%%%%%%%%%%%%%%%%%%%%%%%%%%%%%%%%%%%%%%%%%%%%%%%%%%%%%%%%%%%%%%%%%%%%%%%%%%%%
\subsection{Related CTAN Packages}

There are several other packages which offer a similar functionality:
%
\begin{itemize}
\item
The packages
\href{http://ctan.org/pkg/docmute}{\textsf{docmute}},
\href{http://ctan.org/pkg/includex}{\textsf{includex}} and
\href{http://ctan.org/pkg/standalone}{\textsf{standalone}}
provide commands to include only the document body of
a child file thus allowing both files to be compiled individually.
\item
The packages \href{http://ctan.org/pkg/subdocs}{\textsf{subdocs}}
and \href{http://ctan.org/pkg/subfiles}{\textsf{subfiles}}
provide structures in which the main and child documents can be
encapsulated and allowing them to be compiled individually.
The inclusion mechanism is different from the conventional |\include|.
\item
The package \href{http://ctan.org/pkg/combine}{\textsf{combine}}
is an elaborate solution to combine several documents into one.
\end{itemize}
%
See also the CTAN topic \href{http://ctan.org/topic/subdocs}{\textsf{subdocs}}
for further related packages.
The present package differs from the above solutions in that
a document structure constructed with the conventional |\include| mechanism
just needs two extra commands at the top of every file
such that all constituent files can be compiled individually.

%%%%%%%%%%%%%%%%%%%%%%%%%%%%%%%%%%%%%%%%%%%%%%%%%%%%%%%%%%%%%%%%%%%%%%%%%%%%%%%%
%\subsection{Feature Suggestions}
%
%The following is a list of features which may be useful for future
%versions of this package:
%%
%\begin{itemize}
%\item
%\ldots
%\end{itemize}

%%%%%%%%%%%%%%%%%%%%%%%%%%%%%%%%%%%%%%%%%%%%%%%%%%%%%%%%%%%%%%%%%%%%%%%%%%%%%%%%
\subsection{Revision History}

%%%%%%%%%%%%%%%%%%%%%%%%%%%%%%%%%%%%%%%%
\paragraph{v2.0:} 2018/12/30

\begin{itemize}
\item
immediate forward processing
\item
added |\childdocby| mechanism
\item
manual restructured
\end{itemize}

%%%%%%%%%%%%%%%%%%%%%%%%%%%%%%%%%%%%%%%%
\paragraph{v1.6:} 2018/01/17

\begin{itemize}
\item
application for development of include files
\item
corrections to manual
\end{itemize}

%%%%%%%%%%%%%%%%%%%%%%%%%%%%%%%%%%%%%%%%
\paragraph{v1.5:} 2017/05/21

\begin{itemize}
\item
more complete structuring introduced
\item
|\childdocof| introduced
\item
|\childdoc| renamed to |\childdocmain|
\item
|\childredirect| renamed to |\childdocforward| and |\childdocforwardprefix|
and functionality expanded
\end{itemize}

%%%%%%%%%%%%%%%%%%%%%%%%%%%%%%%%%%%%%%%%
\paragraph{v1.0:} 2017/04/27

\begin{itemize}
\item
manual and install package
\item
first version published on CTAN
\end{itemize}

%%%%%%%%%%%%%%%%%%%%%%%%%%%%%%%%%%%%%%%%
\paragraph{v0.6:} 2017/04/26

\begin{itemize}
\item
redirection mechanism added
\end{itemize}

%%%%%%%%%%%%%%%%%%%%%%%%%%%%%%%%%%%%%%%%
\paragraph{v0.5:} 2017/04/26

\begin{itemize}
\item
functionality in definition file
\end{itemize}


%%%%%%%%%%%%%%%%%%%%%%%%%%%%%%%%%%%%%%%%%%%%%%%%%%%%%%%%%%%%%%%%%%%%%%%%%%%%%%%%
%%%%%%%%%%%%%%%%%%%%%%%%%%%%%%%%%%%%%%%%%%%%%%%%%%%%%%%%%%%%%%%%%%%%%%%%%%%%%%%%
%%%%%%%%%%%%%%%%%%%%%%%%%%%%%%%%%%%%%%%%%%%%%%%%%%%%%%%%%%%%%%%%%%%%%%%%%%%%%%%%
\appendix

\settowidth\MacroIndent{\rmfamily\scriptsize 000\ }

 \DocInput{childdoc.dtx}

\end{document}
%</driver>
% \fi
%
% %%%%%%%%%%%%%%%%%%%%%%%%%%%%%%%%%%%%%%%%%%%%%%%%%%%%%%%%%%%%%%%%%%%%%%%%%%%%%%
% %%%%%%%%%%%%%%%%%%%%%%%%%%%%%%%%%%%%%%%%%%%%%%%%%%%%%%%%%%%%%%%%%%%%%%%%%%%%%%
% \section{Sample}
%\iffalse
%<*samplemain>
%\fi
%
% The following presents a sample document
% with two chapters, two parts, a title page,
% a compile flag as well as three forwarding files to set the flag.
% It consists of eight |.tex| files:
% \begin{center}
% \begin{tabular}{ll}
% |cdocsamp.tex|&main file\\
% |cdocsch1.tex|&include file for chapter 1\\
% |cdocsch2.tex|&include file for chapter 2\\
% |cdocspt3.tex|&include file for part 3\\
% |cdocspt4.tex|&include file for part 4\\
% |cdocsdrf.tex|&forwarding file for main file in draft mode\\
% |cdocsfi1.tex|&forwarding file for final version of chapter 1\\
% |cdocsfi2.tex|&forwarding file for final version of chapter 2\\
% \end{tabular}
% \end{center}
% Each of the eight files can be compiled directly by the \LaTeX{} compiler.
%
% %%%%%%%%%%%%%%%%%%%%%%%%%%%%%%%%%%%%%%
% \paragraph{Main File.}
%
% The main file is called |cdocsamp.tex|.
%
% Load the \textsf{childdoc} definitions and
% declare the filename for the main document:
%    \begin{macrocode}
% \iffalse
%
% childdoc.dtx Copyright (C) 2017-2018 Niklas Beisert
%
% This work may be distributed and/or modified under the
% conditions of the LaTeX Project Public License, either version 1.3
% of this license or (at your option) any later version.
% The latest version of this license is in
%   http://www.latex-project.org/lppl.txt
% and version 1.3 or later is part of all distributions of LaTeX
% version 2005/12/01 or later.
%
% This work has the LPPL maintenance status `maintained'.
%
% The Current Maintainer of this work is Niklas Beisert.
%
% This work consists of the files childdoc.dtx and childdoc.ins
% and the derived files childdoc.def and cdocsamp.tex with
% cdocsch1.tex, cdocsch2.tex, cdocsdrf.tex, cdocsfn1.tex, cdocsfn2.tex.
%
%<package>\ifdefined\childdocmain\endinput\fi
%<package>\ProvidesFile{childdoc.def}[2018/12/30 v2.0 child document driver]
%<samplemain>\ProvidesFile{cdocsamp.tex}[2018/12/30 v2.0 sample for childdoc]
%<*driver>
%\ProvidesFile{childdoc.drv}[2018/12/30 v2.0 childdoc reference manual file]
\PassOptionsToClass{10pt,a4paper}{article}
\documentclass{ltxdoc}

\usepackage[margin=35mm]{geometry}
\usepackage{hyperref}
\usepackage{hyperxmp}
\usepackage[usenames]{color}

\hypersetup{colorlinks=true}
\hypersetup{pdfstartview=FitH}
\hypersetup{pdfpagemode=UseNone}
\hypersetup{pdfsource={}}
\hypersetup{pdflang={en-UK}}
\hypersetup{pdfcopyright={Copyright 2017-2018 Niklas Beisert.
  This work may be distributed and/or modified under the
  conditions of the LaTeX Project Public License, either version 1.3
  of this license or (at your option) any later version.}}
\hypersetup{pdflicenseurl={http://www.latex-project.org/lppl.txt}}
\hypersetup{pdfcontactaddress={ETH Zurich, ITP, HIT K,
  Wolfgang-Pauli-Strasse 27}}
\hypersetup{pdfcontactpostcode={8093}}
\hypersetup{pdfcontactcity={Zurich}}
\hypersetup{pdfcontactcountry={Switzerland}}
\hypersetup{pdfcontactemail={nbeisert@itp.phys.ethz.ch}}
\hypersetup{pdfcontacturl={http://people.phys.ethz.ch/\xmptilde nbeisert/}}

\newcommand{\secref}[1]{\hyperref[#1]{section \ref*{#1}}}

\parskip1ex
\parindent0pt
\let\olditemize\itemize
\def\itemize{\olditemize\parskip0pt}

\begin{document}

\title{The \textsf{childdoc} Package}
\hypersetup{pdftitle={The childdoc Package}}
\author{Niklas Beisert\\[2ex]
  Institut f\"ur Theoretische Physik\\
  Eidgen\"ossische Technische Hochschule Z\"urich\\
  Wolfgang-Pauli-Strasse 27, 8093 Z\"urich, Switzerland\\[1ex]
  \href{mailto:nbeisert@itp.phys.ethz.ch}
  {\texttt{nbeisert@itp.phys.ethz.ch}}}
\hypersetup{pdfauthor={Niklas Beisert}}
\hypersetup{pdfsubject={Manual for the LaTeX2e Package childdoc}}
\date{30 December 2018, \textsf{v2.0}}
\maketitle

\begin{abstract}\noindent
\textsf{childdoc} is a \LaTeXe{} package
that enables the direct compilation
of document sections included by |\include|
to individual files.
\end{abstract}

\begingroup
\parskip0ex
\tableofcontents
\endgroup

%%%%%%%%%%%%%%%%%%%%%%%%%%%%%%%%%%%%%%%%%%%%%%%%%%%%%%%%%%%%%%%%%%%%%%%%%%%%%%%%
%%%%%%%%%%%%%%%%%%%%%%%%%%%%%%%%%%%%%%%%%%%%%%%%%%%%%%%%%%%%%%%%%%%%%%%%%%%%%%%%
\section{Introduction}

\LaTeX{} provides a mechanism to structure a large document (such as a book)
into a main file and several child files (containing the chapters)
using the |\include| command.
This mechanism is beneficial for documents
which span hundreds of pages in order to
make the source file(s) more manageable.
Moreover, compilation can be restricted to
selected child files by means of the |\includeonly| command.
The latter feature can be used to reduce the compilation time while editing
(this was significantly more useful in the earlier days of \LaTeX{})
or to generate a smaller document which is easier to navigate.
Another application of |\includeonly| is to generate
documents consisting of selected parts of the complete document.

However, there are a few drawbacks of the plain |\include| mechanism:
\begin{itemize}
\item
The child files cannot be compiled on their own,
they can only be compiled via the main file.
A naive editing environment
(such as a text editor with an option
to have the current file processed by \LaTeX)
may require one to switch to the main file before compiling;
attempting to compile the child file produces errors.
\item
The main file must be modified (each time)
to adjust the |\includeonly| command
to the present needs. This easily leaves the main file in a messy state.
\item
The generated document will always carry the filename
of the main document. This is inconvenient if
several child files are to be compiled and
to be kept for distribution.
\end{itemize}

The present package provides a simple interface
to make child files individually compilable by \LaTeX{}.
Compiling a child file then has the same effect as compiling
the main file with an |\includeonly| command
to select the appropriate child.
Moreover the generated document will carry the name of the child
rather than the main file.
This resolves all three above issues.

This feature is meant to make the editing of books,
thesis documents and lecture notes somewhat more convenient.
However, the package can also be used efficiently for
composing a series of documents (such as exercise sheets)
which are typically distributed individually.
It then assists the author in generating the individual documents
(potentially in different versions)
as well as a document containing the collected series.
Another application is in developing style files
or other kinds of included material
where compilation of the style file could redirect
to a sample or test file.

%%%%%%%%%%%%%%%%%%%%%%%%%%%%%%%%%%%%%%%%%%%%%%%%%%%%%%%%%%%%%%%%%%%%%%%%%%%%%%%%
%%%%%%%%%%%%%%%%%%%%%%%%%%%%%%%%%%%%%%%%%%%%%%%%%%%%%%%%%%%%%%%%%%%%%%%%%%%%%%%%
\section{Usage}

First of all, the package \textsf{childdoc} is \emph{not} a standard
\LaTeXe{} |.sty| style file! Therefore it needs to be invoked in
a non-standard way.

%%%%%%%%%%%%%%%%%%%%%%%%%%%%%%%%%%%%%%%%%%%%%%%%%%%%%%%%%%%%%%%%%%%%%%%%%%%%%%%%
\subsection{Included Files}
\label{sec:include}

%%%%%%%%%%%%%%%%%%%%%%%%%%%%%%%%%%%%%%%%
\DescribeMacro{\childdocmain}
To use the package, add the commands
\begin{center}
\begin{tabular}{l}
|\input{childdoc.def}|\\
|\childdocmain{}|\\
\end{tabular}
\end{center}
at the very top of the main \LaTeX{} file,
in particular \emph{before} the |\documentclass| statement!
The argument of |\childdocmain| should be left empty
(but it must be present).

%%%%%%%%%%%%%%%%%%%%%%%%%%%%%%%%%%%%%%%%
\DescribeMacro{\childdocof}
Furthermore, add the commands
\begin{center}
\begin{tabular}{l}
|\input{childdoc.def}|\\
|\childdocof{|\textit{main}|}|\\
\end{tabular}
\end{center}
at the top of every child file \textit{child}
which is included by |\include{|\textit{child}|}|
from within the main file
(or at least for those files to be compiled individually).
The argument \textit{main} must be the filename of the main file.

There are a couple of
considerations in setting up the main and child documents:

%%%%%%%%%%%%%%%%%%%%%%%%%%%%%%%%%%%%%%%%
\paragraph{Restrictions.}

Please note the following restrictions:
\begin{itemize}
\item
|\childdocmain| must be called with one argument \textit{main}
to ensure compatibility with earlier version of the package.
It must either be empty (|\childdocmain{}|)
or precisely match the filename of the main file in which it is specified.
See \secref{sec:detection} for further information.
\item
The filename \textit{main} must be specified without the |.tex| extension.
\item
The filename \textit{main} is case sensitive
(even in case-insensitive file systems)
due to internal string comparison.
\item
The argument \textit{main} should be fully expanded, it cannot be a macro.
\item
Subdirectories and special characters should be avoided in filenames.
\item
The command |\childdocmain{|\textit{main}|}| must be followed by a whitespace.
It should not be followed immediately by another command
or by a comment mark `|%|'.
This is because the \TeX{} parser reads the token immediately following
the argument of |\childdocmain| and puts it
at the beginning of every child section;
however, a white\-space is ignored.
\end{itemize}

%%%%%%%%%%%%%%%%%%%%%%%%%%%%%%%%%%%%%%%%
\paragraph{Content of Main File.}

It is advisable to place all content in the child files included by |\include|.
Any output contained in the main file will appear in all child documents
unless suppressed manually;
it cannot be suppressed automatically by the |\includeonly| directive
and thus should normally be avoided.
A method to include some content in the main file
by means of conditional processing is described in \secref{sec:conditional}.

%%%%%%%%%%%%%%%%%%%%%%%%%%%%%%%%%%%%%%%%
\paragraph{Page Numbering.}

When only a part of the document is compiled,
the appropriate numbering of pages
(as well as other status parameters)
is determined from the |.aux| files.
The latter contain information from previous passes.
However this information needs to propagate through
all intermediate child documents.
Therefore the page numbering in child documents may well
be inconsistent until the complete document is compiled at least once.

A useful (if unconventional) way to always ensure a consistent
page numbering is to restart the numbering in each child document
and denote the pages by `\textit{child}|.|\textit{page}'
where \textit{child} represents the chapter/section number of the child file.
This can be achieved by the command
|\numberwithin{page}{|\textit{child}|}|
of the \textsf{amsmath} package
where \textit{child} can be |chapter| or |section|
depending on the chosen structuring.
Alternatively, one can modify the macro |\thepage| appropriately
and reset the counter |page| at the start of each child file.

%%%%%%%%%%%%%%%%%%%%%%%%%%%%%%%%%%%%%%%%%%%%%%%%%%%%%%%%%%%%%%%%%%%%%%%%%%%%%%%%
\subsection{Conditional Processing}
\label{sec:conditional}

The package provides a mechanism to compile different versions
of a document. To customise the versions further some conditional processing
can come in handy to distinguish which version is being compiled.
The package provides two macros to describe the compilation context:

%%%%%%%%%%%%%%%%%%%%%%%%%%%%%%%%%%%%%%%%
\DescribeMacro{\ifchilddoc}
The conditional |\ifchilddoc| distinguishes between the compilation of
child documents and the main document:
%
\begin{center}
|\ifchilddoc |\textit{child-code}| |[|\||else |\textit{main-code}]| \||fi|
\end{center}

%%%%%%%%%%%%%%%%%%%%%%%%%%%%%%%%%%%%%%%%
\DescribeMacro{\childdocname}
\DescribeMacro{\childdocjob}
The macro |\childdocname| contains the filename (without extension)
of the main or child file being processed.
Note that |\childdocjob| will always contain the name of the main file.

%%%%%%%%%%%%%%%%%%%%%%%%%%%%%%%%%%%%%%%%
\paragraph{Title Page.}

Conditional processing can be used to include a title or banner page
in the main document when proper precautions are taken.
Importantly, the code in the main file should ensure that the page counter
(as well as other status parameters which are stored in the |.aux| files)
takes the same value after the conditional processing.
Otherwise the page numbers may take divergent values
depending on which part is compiled.

For example, a title page could be declared by:
%
\begin{center}
\begin{tabular}{l}
|\ifchilddoc\||else|\\
|\addtocounter{page}{-1}|\\
\textit{code for title page}\\
|\newpage|\\
|\||fi|
\end{tabular}
\end{center}
%
A banner page for the child documents can be generated by:
%
\begin{center}
\begin{tabular}{l}
|\ifchilddoc|\\
|\addtocounter{page}{-1}|\\
\textit{code for banner page}\\
|\newpage|\\
|\||fi|
\end{tabular}
\end{center}
%
Here one could write a message such as:
\begin{center}
|This is the part \childdocname{} of \childdocjob{}.|
\end{center}

%%%%%%%%%%%%%%%%%%%%%%%%%%%%%%%%%%%%%%%%%%%%%%%%%%%%%%%%%%%%%%%%%%%%%%%%%%%%%%%%
\subsection{Flags}
\label{sec:flags}

The package makes it easy to generate different versions
of the main or child documents.
To this end compilation flags can be defined
and assigned different default values.
They will be particularly useful in conjunction
with the forwarding mechanism described in \secref{sec:forward}.

For example, it may be useful to have a flag |\version|
which can be set to |draft| or |final|.
The document source will contain some conditional code
depending on the value of |\version|.
Suppose further, the flag should default to |final| for the main file
and to |draft| for child files
which is a natural assignment for editing the document.
This is achieved by placing the following code
in the preamble of the main document
(below the |\childdocmain| directive):
%
\begin{center}
\begin{tabular}{l}
|\ifchilddoc|\\
|\providecommand{\version}{draft}|\\
|\||else|\\
|\providecommand{\version}{final}|\\
|\||fi|
\end{tabular}
\end{center}
%
The definition by |\providecommand| makes sure
that previous definitions are not overwritten.
Further statements |\providecommand{\version}{...}|
can thus be added before the above code to override it.

For the main file, one might add a line
(between |\childdocmain| and the above block)
%
\begin{center}
|%\ifchilddoc\||else\providecommand{\version}{draft}\||fi|
\end{center}
%
which can be uncommented to produce a draft version.
Likewise one can add a line to the very top of a child file
(above the |\childdocof{|\textit{main}|}| directive)
%
\begin{center}
|%\providecommand{\version}{final}|
\end{center}
%
which can be uncommented to produce the final version of this child document.

%%%%%%%%%%%%%%%%%%%%%%%%%%%%%%%%%%%%%%%%%%%%%%%%%%%%%%%%%%%%%%%%%%%%%%%%%%%%%%%%
\subsection{Forwarding}
\label{sec:forward}

Different versions of the main or child documents
using compilation flags as described in \secref{sec:flags}
can be (permanently) stored in different files
for convenient compilation, viewing and distribution.
To this end, the package defines a command
to pass on compilation to a different file:

%%%%%%%%%%%%%%%%%%%%%%%%%%%%%%%%%%%%%%%%
\DescribeMacro{\childdocforward}
The command |\childdocforward| redirects processing to
another source file:
%
\begin{center}
\begin{tabular}{l}
|\input{childdoc.def}|\\
|\childdocforward[|\textit{main}|]{|\textit{dest}|}|\\
\end{tabular}
\end{center}
%
The argument \textit{dest} is the destination file
(without extension).
It should be the main file or one of the child files.
Note that further \textsf{childdoc} directives
such as |\childdocof| and |\childdocforward|
in the indicated file will be processed in this form.
The optional argument \textit{main}
passes on directly to the main file \textit{main}
while pretending to compile the child \textit{dest}.
This form behaves as if \textit{dest}
issues |\childdocof{|\textit{main}|}| right away,
and no further \textsf{childdoc} directives will be processed.

%%%%%%%%%%%%%%%%%%%%%%%%%%%%%%%%%%%%%%%%
\DescribeMacro{\...prefix}
In the alternative form |\childdocforwardprefix|,
%
\begin{center}
\begin{tabular}{l}
|\input{childdoc.def}|\\
|\childdocforwardprefix[|\textit{main}|]{|\textit{prefix}|}{|\textit{dest}|}|
\end{tabular}
\end{center}
%
the destination file is determined by a pattern
depending on the current file:
To make this work, the current file must be called
`{\textit{prefix}\hspace{0.2em}\textit{suffix}}'
with \textit{prefix} matching precisely the argument.
Processing is then passed on to the file
`{\textit{dest}\hspace{0.2em}\textit{suffix}}'.
Surely, the same effect is achieved by
directly specifying the
argument `{\textit{dest}\hspace{0.2em}\textit{suffix}}'
in the first form.
However, that requires to set up a different file
for each child. With the alternative form of the command
all these files can have exactly the same content
which simplifies setting them up and maintaining them.

For example, the following file |draft.tex|
with a compilation flag |\version| as described in \secref{sec:flags}
compiles the main document as a draft:
%
\begin{center}
\begin{tabular}{l}
|\def\version{draft}|\\
|\input{childdoc.def}|\\
|\childdocforward{|\textit{main}|}|
\end{tabular}
\end{center}
%
Likewise, the following files |final|\textit{nn}|.tex|
compile the final version of the child document
|child|\textit{nn}|.tex|:
%
\begin{center}
\begin{tabular}{l}
|\def\version{final}|\\
|\input{childdoc.def}|\\
|\childdocforwardprefix{final}{child}|
\end{tabular}
\end{center}
%

Note that when several versions of a main file and/or of each child file
are to be generated, it may be convenient to set up a |Makefile| or
shell script to automatise the process.

%%%%%%%%%%%%%%%%%%%%%%%%%%%%%%%%%%%%%%%%%%%%%%%%%%%%%%%%%%%%%%%%%%%%%%%%%%%%%%%%
\subsection{Command Line Processing}
\label{sec:commandline}

The effect of redirection files can also be achieved by invoking
the \LaTeX{} compiler with a more elaborate command line.
Most conveniently this should be done as part
of a shell script or a |Makefile|.

When using \textsf{childdoc} in the main file, the following
command lines effectively perform a redirection
(note that depending on the shell being used,
backslashes may have to be doubled: `|\|' $\to$ `|\\|'):
%
\begin{center}
|... -jobname "|\textit{target}|" |\\|"|[\textit{flags}]%
|\input{childdoc.def}\childdocforward[|\textit{main}|]{|\textit{dest}|}"|
\end{center}
%
Here \textit{target} is the name of the output file,
\textit{main} is the name of the main file
and \textit{dest} is the name of the main or child file to be processed
(all filenames without extensions).
The optional argument \textit{main} can be omitted
if \textit{main} matches \textit{dest}.
Optionally, compilation \textit{flags} can be defined via |\def| commands.
This command line makes the \TeX{} engine believe
it is compiling the file \textit{target}
whose content is specified as the latter parameter.
The provided code then forwards the processing to
\textit{main} or \textit{dest} as described in \secref{sec:forward}.

%%%%%%%%%%%%%%%%%%%%%%%%%%%%%%%%%%%%%%%%%%%%%%%%%%%%%%%%%%%%%%%%%%%%%%%%%%%%%%%%
\subsection{Include by Input}
\label{sec:input}

Including child documents by |\include| has some restrictions by design.
Most notably, the content of a child document always occupies
its own set of pages; pages cannot be shared between child documents.
Usually, this behaviour makes perfect sense
because each child document contain an essential part of the document.
However, in some situations it may be desirable to compose
a document from a collection of parts
without having mandatory page breaks between then.
For this case, the package
provides a mechanism to include parts
by |\input| which can also be processed individually.
However, by construction this mechanism
requires manual handling of the content to be output.

%%%%%%%%%%%%%%%%%%%%%%%%%%%%%%%%%%%%%%%%
\DescribeMacro{\ifchilddocmanual}
The main file should be prepared as usual, see \secref{sec:include}.
However, the document body must make a distinction
between processing of an individual part and of the main document, e.g.:
%
\begin{center}
\begin{tabular}{l}
|\ifchilddocmanual|\\
|\input{\childdocname}|\\
|\||else|\\
\textit{document body with }|\input{|\textit{part}|}|\\
|\||fi|
\end{tabular}
\end{center}
%
The conditional |\ifchilddocmanual| is true whenever
a part to be included by |\input| is being compiled,
and the name of the part is stored in |\childdocname|.

%%%%%%%%%%%%%%%%%%%%%%%%%%%%%%%%%%%%%%%%
\DescribeMacro{\childdocby}
Each part to be included by |\input| should start with:
%
\begin{center}
\begin{tabular}{l}
|\input{childdoc.def}|\\
|\childdocby{|\textit{main}|}|\\
\end{tabular}
\end{center}
%
The directive |\childdocby| is similar to |\childdocof|
described in \secref{sec:include},
but the subsequent selection of content must be done manually.
To that end, both |\ifchilddoc| and |\ifchilddocmanual|
will be true upon processing of a part,
and the name of the part is stored in |\childdocname|.
Note that |\jobname| will be set to the filename of the current part
so that each part receives an individual |.aux| file
that does not interfere with the |.aux| file(s) of the main document.
This behaviour can be altered by the alternative form
|\childdocby[*]{|\textit{main}|}| (with a non-empty optional argument)
which uses the |.aux| file of the main document
by setting |\jobname| to \textit{main}.

%%%%%%%%%%%%%%%%%%%%%%%%%%%%%%%%%%%%%%%%%%%%%%%%%%%%%%%%%%%%%%%%%%%%%%%%%%%%%%%%
\subsection{Driver Development}
\label{sec:driver}

The \textsf{childdoc} mechanism can also be use for the development
of definition files such as \LaTeX{} styles or classes.
This case differs from the above setup with multiple parts
included by |\include| in that no |\includeonly| should be invoked.
This can be achieved by starting the include file
(before |\ProvidesPackage|) with:
%
\begin{center}
\begin{tabular}{l}
|\input{childdoc.def}|\\
|\childdocforward{|\textit{main}|}|\\
\end{tabular}
\end{center}
%
or alternatively with:
%
\begin{center}
\begin{tabular}{l}
|\input{childdoc.def}|\\
|\childdocby{|\textit{main}|}|\\
\end{tabular}
\end{center}
%
Both forms have slightly different effects as described above.
The main file is prepared as usual, see \secref{sec:include}.

%%%%%%%%%%%%%%%%%%%%%%%%%%%%%%%%%%%%%%%%%%%%%%%%%%%%%%%%%%%%%%%%%%%%%%%%%%%%%%%%
\subsection{Legacy Detection}
\label{sec:detection}

The directive |\childdocmain| in the main file can detect
whether the complete document or merely a child is to be compiled
even without using the directive |\childdocof|.
This method is deprecated because it is less robust
and there is no compelling reason to use it;
it is merely provided for backward compatibility
and it may be removed in future versions.

If the detection mechanism is to be used,
it is mandatory to correctly specify
the filename of the main file as the argument of |\childdocmain|:
%
\begin{center}
\begin{tabular}{l}
|\input{childdoc.def}|\\
|\childdocmain{|\textit{main}|}|\\
\end{tabular}
\end{center}
%
If |\jobname| does not match the argument \textit{main} of |\childdocmain|,
it is assumed that |\jobname| points to the child file to be compiled.
When using |\childdocmain| with the main file specified as argument,
it suffices to start a child file
with just |\input{|\textit{main}|}|
without loading of the package and using |\childdocof|.
If instead all processing is done
with the appropriate \textsf{childdoc} directives,
the argument of \textit{main} of |\childdocmain| can be empty.

An alternative version of the command line processing described
in \secref{sec:commandline} using the detection mechanism reads:
%
\begin{center}
|... -jobname "|\textit{target}|" "|[\textit{flags}]%
[|\def\jobname{|\textit{dest}|}|]|\input{|\textit{main}|}"|
\end{center}

%%%%%%%%%%%%%%%%%%%%%%%%%%%%%%%%%%%%%%%%%%%%%%%%%%%%%%%%%%%%%%%%%%%%%%%%%%%%%%%%
\subsection{Manual Code}
\label{sec:manual}

In case one cannot be certain whether the definitions file |childdoc.def|
is installed on the target \TeX{} distribution
and one prefers not to ship it,
it is conceivable to paste a few relevant commands into the sources.

To that end, drop all statements |\input{childdoc.def}|
and perform the replacements as outlined below.
Instead of |\childdocmain{|\textit{main}|}| add the following code
to the top of the main file:
%
\begin{center}
\begin{tabular}{l}
|\||ifdefined\childdocname\endinput\||fi\newif\ifchilddoc|\\
|\edef\childdocname{\scantokens\expandafter{\jobname\noexpand}}|\\
|\def\childdocmain{|\textit{main}|}\||ifx\childdocmain\childdocname\||else|\\
|\childdoctrue\includeonly{\childdocname}\let\jobname\childdocmain\||fi|\\
\end{tabular}
\end{center}
%
Instead of |\childdocof{|\textit{main}|}| just include the main file
at the top of each child file:
%
\begin{center}
|\input{|\textit{main}|}|
\end{center}
%
A simple redirection |\childdocforward{|\textit{dest}|}| is achieved by:
%
\begin{center}
|\def\jobname{|\textit{dest}|}\input{\jobname}|
\end{center}
%
The redirection with prefix
|\childdocforwardprefix[|\textit{prefix}|]{|\textit{dest}|}|
is accomplished by:
%
\begin{center}
\begin{tabular}{l}
|{\edef\jobname{\scantokens\expandafter{\jobname\noexpand}}|\\
|\def\redirectjob |\textit{prefix}|#1~~~{\gdef\jobname{|\textit{dest}|#1}}|\\
|\expandafter\redirectjob\jobname~~~}\input{\jobname}|
\end{tabular}
\end{center}

In an alternative approach,
child documents can be compiled by a specific command line
without additional code or specific definitions:
%
\begin{center}
|... -jobname "|\textit{target}|" "|[\textit{flags}]%
|\includeonly{|\textit{dest}|}\input{|\textit{main}|}"|
\end{center}
%

%%%%%%%%%%%%%%%%%%%%%%%%%%%%%%%%%%%%%%%%%%%%%%%%%%%%%%%%%%%%%%%%%%%%%%%%%%%%%%%%
%%%%%%%%%%%%%%%%%%%%%%%%%%%%%%%%%%%%%%%%%%%%%%%%%%%%%%%%%%%%%%%%%%%%%%%%%%%%%%%%
\section{Information}

%%%%%%%%%%%%%%%%%%%%%%%%%%%%%%%%%%%%%%%%%%%%%%%%%%%%%%%%%%%%%%%%%%%%%%%%%%%%%%%%
\subsection{Copyright}

Copyright \copyright{} 2017--2018 Niklas Beisert

This work may be distributed and/or modified under the
conditions of the \LaTeX{} Project Public License, either version 1.3
of this license or (at your option) any later version.
The latest version of this license is in
  \url{http://www.latex-project.org/lppl.txt}
and version 1.3 or later is part of all distributions of \LaTeX{}
version 2005/12/01 or later.

This work has the LPPL maintenance status `maintained'.

The Current Maintainer of this work is Niklas Beisert.

This work consists of the files |README.txt|, |childdoc.ins| and |childdoc.dtx|
as well as the derived files |childdoc.def|, |cdocsamp.tex|
with |cdocsch1.tex|, |cdocsch2.tex|, |cdocspt3.tex|, |cdocspt4.tex|,
|cdocsdrf.tex|, |cdocsfn1.tex|, |cdocsfn2.tex|
as well as |childdoc.pdf|.

%%%%%%%%%%%%%%%%%%%%%%%%%%%%%%%%%%%%%%%%%%%%%%%%%%%%%%%%%%%%%%%%%%%%%%%%%%%%%%%%
\subsection{Files and Installation}

The package consists of the files:
%
\begin{center}
\begin{tabular}{ll}
    |README.txt|   & readme file \\
    |childdoc.ins| & installation file \\
    |childdoc.dtx| & source file \\
    |childdoc.def| & definition file \\
    |cdocsamp.tex| & sample main file \\
    |cdocsch1.tex| & sample include file \\
    |cdocsch2.tex| & sample include file \\
    |cdocspt3.tex| & sample part file \\
    |cdocspt4.tex| & sample part file \\
    |cdocsdrf.tex| & sample redirection file \\
    |cdocsfn1.tex| & sample redirection file \\
    |cdocsfn2.tex| & sample redirection file \\
    |childdoc.pdf| & manual
\end{tabular}
\end{center}
%
The distribution consists of the files
|README.txt|, |childdoc.ins| and |childdoc.dtx|.
%
\begin{itemize}
\item
Run (pdf)\LaTeX{} on |childdoc.dtx|
to compile the manual |childdoc.pdf| (this file).
\item
Run \LaTeX{} on |childdoc.ins| to create the definitions file |childdoc.def|
and the sample |cdocsamp.tex| with include files
|cdocsch1.tex|, |cdocsch2.tex|, |cdocspt3.tex|, |cdocspt4.tex|,
|cdocsdrf.tex|, |cdocsfn1.tex|, |cdocsfn2.tex|.
Then copy the file |childdoc.def| to an appropriate directory of your \LaTeX{}
distribution, e.g.\ \textit{texmf-root}|/tex/latex/childdoc|.
\end{itemize}

%%%%%%%%%%%%%%%%%%%%%%%%%%%%%%%%%%%%%%%%%%%%%%%%%%%%%%%%%%%%%%%%%%%%%%%%%%%%%%%%
\subsection{Related CTAN Packages}

There are several other packages which offer a similar functionality:
%
\begin{itemize}
\item
The packages
\href{http://ctan.org/pkg/docmute}{\textsf{docmute}},
\href{http://ctan.org/pkg/includex}{\textsf{includex}} and
\href{http://ctan.org/pkg/standalone}{\textsf{standalone}}
provide commands to include only the document body of
a child file thus allowing both files to be compiled individually.
\item
The packages \href{http://ctan.org/pkg/subdocs}{\textsf{subdocs}}
and \href{http://ctan.org/pkg/subfiles}{\textsf{subfiles}}
provide structures in which the main and child documents can be
encapsulated and allowing them to be compiled individually.
The inclusion mechanism is different from the conventional |\include|.
\item
The package \href{http://ctan.org/pkg/combine}{\textsf{combine}}
is an elaborate solution to combine several documents into one.
\end{itemize}
%
See also the CTAN topic \href{http://ctan.org/topic/subdocs}{\textsf{subdocs}}
for further related packages.
The present package differs from the above solutions in that
a document structure constructed with the conventional |\include| mechanism
just needs two extra commands at the top of every file
such that all constituent files can be compiled individually.

%%%%%%%%%%%%%%%%%%%%%%%%%%%%%%%%%%%%%%%%%%%%%%%%%%%%%%%%%%%%%%%%%%%%%%%%%%%%%%%%
%\subsection{Feature Suggestions}
%
%The following is a list of features which may be useful for future
%versions of this package:
%%
%\begin{itemize}
%\item
%\ldots
%\end{itemize}

%%%%%%%%%%%%%%%%%%%%%%%%%%%%%%%%%%%%%%%%%%%%%%%%%%%%%%%%%%%%%%%%%%%%%%%%%%%%%%%%
\subsection{Revision History}

%%%%%%%%%%%%%%%%%%%%%%%%%%%%%%%%%%%%%%%%
\paragraph{v2.0:} 2018/12/30

\begin{itemize}
\item
immediate forward processing
\item
added |\childdocby| mechanism
\item
manual restructured
\end{itemize}

%%%%%%%%%%%%%%%%%%%%%%%%%%%%%%%%%%%%%%%%
\paragraph{v1.6:} 2018/01/17

\begin{itemize}
\item
application for development of include files
\item
corrections to manual
\end{itemize}

%%%%%%%%%%%%%%%%%%%%%%%%%%%%%%%%%%%%%%%%
\paragraph{v1.5:} 2017/05/21

\begin{itemize}
\item
more complete structuring introduced
\item
|\childdocof| introduced
\item
|\childdoc| renamed to |\childdocmain|
\item
|\childredirect| renamed to |\childdocforward| and |\childdocforwardprefix|
and functionality expanded
\end{itemize}

%%%%%%%%%%%%%%%%%%%%%%%%%%%%%%%%%%%%%%%%
\paragraph{v1.0:} 2017/04/27

\begin{itemize}
\item
manual and install package
\item
first version published on CTAN
\end{itemize}

%%%%%%%%%%%%%%%%%%%%%%%%%%%%%%%%%%%%%%%%
\paragraph{v0.6:} 2017/04/26

\begin{itemize}
\item
redirection mechanism added
\end{itemize}

%%%%%%%%%%%%%%%%%%%%%%%%%%%%%%%%%%%%%%%%
\paragraph{v0.5:} 2017/04/26

\begin{itemize}
\item
functionality in definition file
\end{itemize}


%%%%%%%%%%%%%%%%%%%%%%%%%%%%%%%%%%%%%%%%%%%%%%%%%%%%%%%%%%%%%%%%%%%%%%%%%%%%%%%%
%%%%%%%%%%%%%%%%%%%%%%%%%%%%%%%%%%%%%%%%%%%%%%%%%%%%%%%%%%%%%%%%%%%%%%%%%%%%%%%%
%%%%%%%%%%%%%%%%%%%%%%%%%%%%%%%%%%%%%%%%%%%%%%%%%%%%%%%%%%%%%%%%%%%%%%%%%%%%%%%%
\appendix

\settowidth\MacroIndent{\rmfamily\scriptsize 000\ }

 \DocInput{childdoc.dtx}

\end{document}
%</driver>
% \fi
%
% %%%%%%%%%%%%%%%%%%%%%%%%%%%%%%%%%%%%%%%%%%%%%%%%%%%%%%%%%%%%%%%%%%%%%%%%%%%%%%
% %%%%%%%%%%%%%%%%%%%%%%%%%%%%%%%%%%%%%%%%%%%%%%%%%%%%%%%%%%%%%%%%%%%%%%%%%%%%%%
% \section{Sample}
%\iffalse
%<*samplemain>
%\fi
%
% The following presents a sample document
% with two chapters, two parts, a title page,
% a compile flag as well as three forwarding files to set the flag.
% It consists of eight |.tex| files:
% \begin{center}
% \begin{tabular}{ll}
% |cdocsamp.tex|&main file\\
% |cdocsch1.tex|&include file for chapter 1\\
% |cdocsch2.tex|&include file for chapter 2\\
% |cdocspt3.tex|&include file for part 3\\
% |cdocspt4.tex|&include file for part 4\\
% |cdocsdrf.tex|&forwarding file for main file in draft mode\\
% |cdocsfi1.tex|&forwarding file for final version of chapter 1\\
% |cdocsfi2.tex|&forwarding file for final version of chapter 2\\
% \end{tabular}
% \end{center}
% Each of the eight files can be compiled directly by the \LaTeX{} compiler.
%
% %%%%%%%%%%%%%%%%%%%%%%%%%%%%%%%%%%%%%%
% \paragraph{Main File.}
%
% The main file is called |cdocsamp.tex|.
%
% Load the \textsf{childdoc} definitions and
% declare the filename for the main document:
%    \begin{macrocode}
\input{childdoc.def}
\childdocmain{}
%    \end{macrocode}

% Optional override for |\version| flag:
%    \begin{macrocode}
%%\ifchilddoc\else\providecommand{\version}{draft}\fi
%    \end{macrocode}

% Define the default values for the |\version| flag
% (|final| for the main file and |draft| for childs):
%    \begin{macrocode}
\ifchilddoc
\providecommand{\version}{draft}
\else
\providecommand{\version}{final}
\fi
%    \end{macrocode}

% Load the standard document class:
%    \begin{macrocode}
\documentclass[12pt]{article}
%    \end{macrocode}

% Start the document body:
%    \begin{macrocode}
\begin{document}
%    \end{macrocode}

% Declare a title page.
% Print title, part of document being processed and version flag:
%    \begin{macrocode}
\addtocounter{page}{-1}
\begin{center}
{\LARGE\bfseries{}childdoc example\par}
\vspace{1cm}
\ifchilddoc
\ifchilddocmanual part\else chapter\fi:
`\childdocname' of `\childdocjob'\par
\else
main document: `\childdocjob'\par
\fi
version: \version\par
\end{center}
\newpage
%    \end{macrocode}

% Manually include selected file,
% otherwise process as usual:
%    \begin{macrocode}
\ifchilddocmanual
\section*{part `\childdocname'}
\input{\childdocname}
\else
%    \end{macrocode}

% Include the two chapters:
%    \begin{macrocode}
\include{cdocsch1}
\include{cdocsch2}
%    \end{macrocode}

% Include the two parts unless only chapters should be displayed:
%    \begin{macrocode}
\ifchilddoc\else
\section{part three}
\input{cdocspt3}
\section{part four}
\input{cdocspt4}
\fi
%    \end{macrocode}

% Process as usual until here:
%    \begin{macrocode}
\fi
%    \end{macrocode}

% End of document body:
%    \begin{macrocode}
\end{document}
%    \end{macrocode}
%\iffalse
%</samplemain>
%\fi
%
% %%%%%%%%%%%%%%%%%%%%%%%%%%%%%%%%%%%%%%
% \paragraph{Chapter Include Files.}
%
% The include files are called |cdocsch1.tex| and |cdocsch2.tex|.
%
%\iffalse
%<*samplechap1|samplechap2>
%\fi

% Optional override for |\version| flag:
%    \begin{macrocode}
%%\providecommand{\version}{final}
%    \end{macrocode}

% Include the main document:
%    \begin{macrocode}
\input{childdoc.def}
\childdocof{cdocsamp}
%    \end{macrocode}

%\iffalse
%</samplechap1|samplechap2>
%\fi
%
%\iffalse
%<*samplechap1>
%\fi
% Some text for chapter 1:
%    \begin{macrocode}
\section{one}
some text in chapter one
%    \end{macrocode}

%\iffalse
%</samplechap1>
%\fi
% Some text for chapter 2:
%\iffalse
%<*samplechap2>
%\fi
%    \begin{macrocode}
\section{two}
more text in chapter two
%    \end{macrocode}

%\iffalse
%</samplechap2>
%\fi
%
% %%%%%%%%%%%%%%%%%%%%%%%%%%%%%%%%%%%%%%
% \paragraph{Part Include Files.}
%
% The include files are called |cdocspt3.tex| and |cdocspt4.tex|.
%
%\iffalse
%<*samplepart3|samplepart4>
%\fi

% Optional override for |\version| flag:
%    \begin{macrocode}
%%\providecommand{\version}{final}
%    \end{macrocode}

% Include the main document:
%    \begin{macrocode}
\input{childdoc.def}
\childdocby{cdocsamp}
%    \end{macrocode}

%\iffalse
%</samplepart3|samplepart4>
%\fi
%
%\iffalse
%<*samplepart3>
%\fi
% Some text for part 3:
%    \begin{macrocode}
some text in part three
%    \end{macrocode}

%\iffalse
%</samplepart3>
%\fi
% Some text for part 4:
%\iffalse
%<*samplepart4>
%\fi
%    \begin{macrocode}
more text in part four
%    \end{macrocode}

%\iffalse
%</samplepart4>
%\fi
%
% %%%%%%%%%%%%%%%%%%%%%%%%%%%%%%%%%%%%%%
% \paragraph{Forwarding for a Complete Draft.}
%
% The following forwarding file |cdocsdrf.tex|
% compiles the main document in draft mode:
%\iffalse
%<*sampledraft>
%\fi
%    \begin{macrocode}
\def\version{draft}
\input{childdoc.def}
\childdocforward{cdocsamp}
%    \end{macrocode}

%\iffalse
%</sampledraft>
%\fi
%
% %%%%%%%%%%%%%%%%%%%%%%%%%%%%%%%%%%%%%%
% \paragraph{Forwarding for Final Version of the Chapters.}
%
% The following forwarding files |cdocsfn1.tex| and |cdocsfn2.tex|
% (with identical content)
% compile the final versions of the child documents
% |cdocsch1.tex| and |cdocsch2.tex|, respectively:
%\iffalse
%<*samplefinal>
%\fi
%    \begin{macrocode}
\def\version{final}
\input{childdoc.def}
\childdocforwardprefix[cdocsamp]{cdocsfn}{cdocsch}
%    \end{macrocode}

%\iffalse
%</samplefinal>
%\fi
%
% %%%%%%%%%%%%%%%%%%%%%%%%%%%%%%%%%%%%%%
% \paragraph{Command Line Processing.}
%
% The following three command lines generate the output files
% |cdocscld|, |cdocscl1| and |cdocscl2|
% which should be identical to
% |cdocsdrf|, |cdocsch1| and |cdocsfn2|, respectively:
% \begin{center}
% \begin{tabular}{l}
% |latex -jobname cdocscld \|\\
% |  "\def\version{draft}\input{childdoc.def}\childdocforward{cdocsamp}"|\\
% |latex -jobname cdocscl1 \|\\
% |  "\input{childdoc.def}\childdocforward[cdocsamp]{cdocsch1}"|\\
% |latex -jobname cdocscl2 \|\\
% |  "\def\version{final}\input{childdoc.def}\childdocforward{cdocsch2}"|
% \end{tabular}
% \end{center}
% Note that the trailing backslash on each first line
% merely continues the input to the second line
% (for convenient cut ant paste).
% Furthermore, the command |latex| can be replaced by any
% of its alternative versions such as |pdflatex|.
%
% %%%%%%%%%%%%%%%%%%%%%%%%%%%%%%%%%%%%%%%%%%%%%%%%%%%%%%%%%%%%%%%%%%%%%%%%%%%%%%
% %%%%%%%%%%%%%%%%%%%%%%%%%%%%%%%%%%%%%%%%%%%%%%%%%%%%%%%%%%%%%%%%%%%%%%%%%%%%%%
% \section{Implementation}
%\iffalse
%<*package>
%\fi
%
% This section describes the definitions file |childdoc.def|.

% The definitions cannot be loaded using |\usepackage| or |\RequirePackage|
% which has a mechanism to prevent loading a style file more than once.
% When loading the definitions by means of |\input|
% multiple instances have to be prevented manually:
%\iffalse
%This code needs to be before the `\ProvidesFile' directive
%which is defined at the beginning of this file.
%Therefore it is also placed there and commented out here.
%</package>
%<*discard>
%\fi
%    \begin{macrocode}
\ifdefined\childdocmain\endinput\fi
%    \end{macrocode}
%\iffalse
%</discard>
%<*package>
%\fi
%
% \macro{\ifchilddoc}
% \macro{\ifchilddocmanual}
% The conditional |\ifchilddoc| tells whether a
% child (true) or main (false) document is being compiled.
% The conditional |\ifchilddocmanual| tells whether
% the |\includeonly| mechanism is used (false) or
% the selection of child files must be performed manually (true).
% The definitions initialise to false:
%    \begin{macrocode}
\newif\ifchilddoc
\newif\ifchilddocmanual
%    \end{macrocode}

% \macro{\childdocname}
% \macro{\childdocjob}
% The macro |\childdocname| stores the name of the main document
% to be compiled. The macro |\childdocjob| stores the name of
% the document on which the \LaTeX{} compiler was originally invoked.
% The content of |\jobname| cannot be compared
% to filenames specified in the source due to different catcodes.
% The following code rescans |\jobname|, stores the result
% in |\childdocname| and saves a copy in |\childdocjob|:
%    \begin{macrocode}
\edef\childdocname{\scantokens\expandafter{\jobname\noexpand}}
\let\childdocjob\childdocname
%    \end{macrocode}

% \macro{\childdocdisable}
% The macro |\childdocdisable| prevents the main file
% from being processed more than once.
% At this stage, the main document command |\childdocmain|
% is assumed to be called once again where it should do nothing.
% Any subsequent call to it should prevent
% a secondary processing of the main document
% It overwrites the forwarding commands
% |\childdocof| and |\childdocforward|
% with empty macros to prevent further inclusions of the main document:
%    \begin{macrocode}
\newcommand{\childdocdisable}
{
  \renewcommand{\childdocmain}[1]{\renewcommand{\childdocmain}[1]{\endinput}}
  \renewcommand{\childdocof}[1]{}
  \renewcommand{\childdocby}[2][]{}
  \renewcommand{\childdocforward}[2][]{}
  \renewcommand{\childdocdisable}{}
}
%    \end{macrocode}

% \macro{\childdocmain}
% The macro |\childdocmain| is to be called at the top of the main file
% with nothing or the main filename (without extension) as argument.
% First, it breaks loops.
% If the argument is not empty and does not match |\childdocname|
% (which is set by the first inclusion of |childdoc.def|),
% |\ifchilddoc| is set to true, |\includeonly| is applied to the child file
% and |\jobname| is set to the main file
% (for proper handling of |.aux| files):
%    \begin{macrocode}
\newcommand{\childdocmain}[1]
{
  \childdocdisable\childdocmain{}
  \if?#1?\else
    \begingroup
      \def\childdoctmp{#1}
      \ifx\childdoctmp\childdocname
        \def\childdoctmp{}
      \else
        \def\childdoctmp
        {
          \childdoctrue
          \includeonly{\childdocname}
          \def\childdocjob{#1}
          \def\jobname{#1}
        }
      \fi
      \expandafter
    \endgroup
    \childdoctmp
  \fi
}
%    \end{macrocode}

% \macro{\childdocof}
% The command |\childdocof| redirects
% compilation to the main file |#1|.
%    \begin{macrocode}
\newcommand{\childdocof}[1]
{
  \childdocdisable
  \childdoctrue
  \includeonly{\childdocname}
  \def\jobname{#1}
  \def\childdocjob{#1}
  \input{#1}
}
%    \end{macrocode}

% \macro{\childdocby}
% The command |\childdocby| ....
%    \begin{macrocode}
\newcommand{\childdocby}[2][]
{
  \childdocdisable
  \childdoctrue
  \childdocmanualtrue
  \if?#1?\else
    \def\jobname{#2}
  \fi
  \def\childdocjob{#2}
  \input{#2}
  \endinput
}
%    \end{macrocode}

% \macro{\childdocforward}
% The command |\childdocforward| redirects
% compilation to the main file or
% (if the optional argument is given) a child file.
% Parameters are set as if the main file
% or a child file starting with |\childdocof| was compiled.
% Then compilation is handed over to the main file:
%    \begin{macrocode}
\newcommand{\childdocforward}[2][]
{
  \begingroup
    \if?#1?
      \def\childdoctmp
      {
        \def\childdocname{#2}
        \def\childdocjob{#2}
        \def\jobname{#2}
        \input{#2}
        \endinput
      }
    \else
      \def\childdoctmp
      {
        \childdocdisable
        \def\childdocname{#2}
        \childdoctrue
        \includeonly{#2}
        \def\childdocjob{#1}
        \def\jobname{#1}
        \input{#1}
        \endinput
      }
    \fi
    \expandafter
  \endgroup
  \childdoctmp
}
%    \end{macrocode}

% \macro{\childdocforwardprefix}
% The command |\childdocforwardprefix| redirects
% compilation to the main or a child file by means of a pattern.
% The prefix |#1| in the current filename is replaced by |#2|
% and the suffix of the current filename is kept
% (it is assumed that the filename does not contain the substring `|~~~|'
% which is used as a delimiter).
% Compilation is handed over to the new file by |\childdocforward|:
%    \begin{macrocode}
\newcommand{\childdocforwardprefix}[3][]
{
  \begingroup
    \def\childdocextract #2##1~~~{\def\childdoctmp{\childdocforward[#1]{#3##1}}}
    \expandafter\childdocextract\childdocname~~~
    \expandafter
  \endgroup
  \childdoctmp
}
%    \end{macrocode}

% \macro{\childdoc}
% The deprecated macro |\childdoc| is a legacy version of |\childdocmain|:
%    \begin{macrocode}
\newcommand{\childdoc}{\childdocmain}
%    \end{macrocode}

% \macro{\childdocredirect}
% The deprecated macro |\childdocredirect| is a legacy version
% of |\childdocforward| and |\childdocforwardprefix|:
%    \begin{macrocode}
\newcommand{\childdocredirect}[2][]
{
  \begingroup
    \if?#1?
      \def\childdoctmp{\childdocforward{#2}}
    \else
      \def\childdoctmp{\childdocforwardprefix{#1}{#2}}
    \fi
    \expandafter
  \endgroup
  \childdoctmp
}
%    \end{macrocode}

%\iffalse
%</package>
%\fi
%
\endinput

\childdocmain{}
%    \end{macrocode}

% Optional override for |\version| flag:
%    \begin{macrocode}
%%\ifchilddoc\else\providecommand{\version}{draft}\fi
%    \end{macrocode}

% Define the default values for the |\version| flag
% (|final| for the main file and |draft| for childs):
%    \begin{macrocode}
\ifchilddoc
\providecommand{\version}{draft}
\else
\providecommand{\version}{final}
\fi
%    \end{macrocode}

% Load the standard document class:
%    \begin{macrocode}
\documentclass[12pt]{article}
%    \end{macrocode}

% Start the document body:
%    \begin{macrocode}
\begin{document}
%    \end{macrocode}

% Declare a title page.
% Print title, part of document being processed and version flag:
%    \begin{macrocode}
\addtocounter{page}{-1}
\begin{center}
{\LARGE\bfseries{}childdoc example\par}
\vspace{1cm}
\ifchilddoc
\ifchilddocmanual part\else chapter\fi:
`\childdocname' of `\childdocjob'\par
\else
main document: `\childdocjob'\par
\fi
version: \version\par
\end{center}
\newpage
%    \end{macrocode}

% Manually include selected file,
% otherwise process as usual:
%    \begin{macrocode}
\ifchilddocmanual
\section*{part `\childdocname'}
\input{\childdocname}
\else
%    \end{macrocode}

% Include the two chapters:
%    \begin{macrocode}
\include{cdocsch1}
\include{cdocsch2}
%    \end{macrocode}

% Include the two parts unless only chapters should be displayed:
%    \begin{macrocode}
\ifchilddoc\else
\section{part three}
\input{cdocspt3}
\section{part four}
\input{cdocspt4}
\fi
%    \end{macrocode}

% Process as usual until here:
%    \begin{macrocode}
\fi
%    \end{macrocode}

% End of document body:
%    \begin{macrocode}
\end{document}
%    \end{macrocode}
%\iffalse
%</samplemain>
%\fi
%
% %%%%%%%%%%%%%%%%%%%%%%%%%%%%%%%%%%%%%%
% \paragraph{Chapter Include Files.}
%
% The include files are called |cdocsch1.tex| and |cdocsch2.tex|.
%
%\iffalse
%<*samplechap1|samplechap2>
%\fi

% Optional override for |\version| flag:
%    \begin{macrocode}
%%\providecommand{\version}{final}
%    \end{macrocode}

% Include the main document:
%    \begin{macrocode}
% \iffalse
%
% childdoc.dtx Copyright (C) 2017-2018 Niklas Beisert
%
% This work may be distributed and/or modified under the
% conditions of the LaTeX Project Public License, either version 1.3
% of this license or (at your option) any later version.
% The latest version of this license is in
%   http://www.latex-project.org/lppl.txt
% and version 1.3 or later is part of all distributions of LaTeX
% version 2005/12/01 or later.
%
% This work has the LPPL maintenance status `maintained'.
%
% The Current Maintainer of this work is Niklas Beisert.
%
% This work consists of the files childdoc.dtx and childdoc.ins
% and the derived files childdoc.def and cdocsamp.tex with
% cdocsch1.tex, cdocsch2.tex, cdocsdrf.tex, cdocsfn1.tex, cdocsfn2.tex.
%
%<package>\ifdefined\childdocmain\endinput\fi
%<package>\ProvidesFile{childdoc.def}[2018/12/30 v2.0 child document driver]
%<samplemain>\ProvidesFile{cdocsamp.tex}[2018/12/30 v2.0 sample for childdoc]
%<*driver>
%\ProvidesFile{childdoc.drv}[2018/12/30 v2.0 childdoc reference manual file]
\PassOptionsToClass{10pt,a4paper}{article}
\documentclass{ltxdoc}

\usepackage[margin=35mm]{geometry}
\usepackage{hyperref}
\usepackage{hyperxmp}
\usepackage[usenames]{color}

\hypersetup{colorlinks=true}
\hypersetup{pdfstartview=FitH}
\hypersetup{pdfpagemode=UseNone}
\hypersetup{pdfsource={}}
\hypersetup{pdflang={en-UK}}
\hypersetup{pdfcopyright={Copyright 2017-2018 Niklas Beisert.
  This work may be distributed and/or modified under the
  conditions of the LaTeX Project Public License, either version 1.3
  of this license or (at your option) any later version.}}
\hypersetup{pdflicenseurl={http://www.latex-project.org/lppl.txt}}
\hypersetup{pdfcontactaddress={ETH Zurich, ITP, HIT K,
  Wolfgang-Pauli-Strasse 27}}
\hypersetup{pdfcontactpostcode={8093}}
\hypersetup{pdfcontactcity={Zurich}}
\hypersetup{pdfcontactcountry={Switzerland}}
\hypersetup{pdfcontactemail={nbeisert@itp.phys.ethz.ch}}
\hypersetup{pdfcontacturl={http://people.phys.ethz.ch/\xmptilde nbeisert/}}

\newcommand{\secref}[1]{\hyperref[#1]{section \ref*{#1}}}

\parskip1ex
\parindent0pt
\let\olditemize\itemize
\def\itemize{\olditemize\parskip0pt}

\begin{document}

\title{The \textsf{childdoc} Package}
\hypersetup{pdftitle={The childdoc Package}}
\author{Niklas Beisert\\[2ex]
  Institut f\"ur Theoretische Physik\\
  Eidgen\"ossische Technische Hochschule Z\"urich\\
  Wolfgang-Pauli-Strasse 27, 8093 Z\"urich, Switzerland\\[1ex]
  \href{mailto:nbeisert@itp.phys.ethz.ch}
  {\texttt{nbeisert@itp.phys.ethz.ch}}}
\hypersetup{pdfauthor={Niklas Beisert}}
\hypersetup{pdfsubject={Manual for the LaTeX2e Package childdoc}}
\date{30 December 2018, \textsf{v2.0}}
\maketitle

\begin{abstract}\noindent
\textsf{childdoc} is a \LaTeXe{} package
that enables the direct compilation
of document sections included by |\include|
to individual files.
\end{abstract}

\begingroup
\parskip0ex
\tableofcontents
\endgroup

%%%%%%%%%%%%%%%%%%%%%%%%%%%%%%%%%%%%%%%%%%%%%%%%%%%%%%%%%%%%%%%%%%%%%%%%%%%%%%%%
%%%%%%%%%%%%%%%%%%%%%%%%%%%%%%%%%%%%%%%%%%%%%%%%%%%%%%%%%%%%%%%%%%%%%%%%%%%%%%%%
\section{Introduction}

\LaTeX{} provides a mechanism to structure a large document (such as a book)
into a main file and several child files (containing the chapters)
using the |\include| command.
This mechanism is beneficial for documents
which span hundreds of pages in order to
make the source file(s) more manageable.
Moreover, compilation can be restricted to
selected child files by means of the |\includeonly| command.
The latter feature can be used to reduce the compilation time while editing
(this was significantly more useful in the earlier days of \LaTeX{})
or to generate a smaller document which is easier to navigate.
Another application of |\includeonly| is to generate
documents consisting of selected parts of the complete document.

However, there are a few drawbacks of the plain |\include| mechanism:
\begin{itemize}
\item
The child files cannot be compiled on their own,
they can only be compiled via the main file.
A naive editing environment
(such as a text editor with an option
to have the current file processed by \LaTeX)
may require one to switch to the main file before compiling;
attempting to compile the child file produces errors.
\item
The main file must be modified (each time)
to adjust the |\includeonly| command
to the present needs. This easily leaves the main file in a messy state.
\item
The generated document will always carry the filename
of the main document. This is inconvenient if
several child files are to be compiled and
to be kept for distribution.
\end{itemize}

The present package provides a simple interface
to make child files individually compilable by \LaTeX{}.
Compiling a child file then has the same effect as compiling
the main file with an |\includeonly| command
to select the appropriate child.
Moreover the generated document will carry the name of the child
rather than the main file.
This resolves all three above issues.

This feature is meant to make the editing of books,
thesis documents and lecture notes somewhat more convenient.
However, the package can also be used efficiently for
composing a series of documents (such as exercise sheets)
which are typically distributed individually.
It then assists the author in generating the individual documents
(potentially in different versions)
as well as a document containing the collected series.
Another application is in developing style files
or other kinds of included material
where compilation of the style file could redirect
to a sample or test file.

%%%%%%%%%%%%%%%%%%%%%%%%%%%%%%%%%%%%%%%%%%%%%%%%%%%%%%%%%%%%%%%%%%%%%%%%%%%%%%%%
%%%%%%%%%%%%%%%%%%%%%%%%%%%%%%%%%%%%%%%%%%%%%%%%%%%%%%%%%%%%%%%%%%%%%%%%%%%%%%%%
\section{Usage}

First of all, the package \textsf{childdoc} is \emph{not} a standard
\LaTeXe{} |.sty| style file! Therefore it needs to be invoked in
a non-standard way.

%%%%%%%%%%%%%%%%%%%%%%%%%%%%%%%%%%%%%%%%%%%%%%%%%%%%%%%%%%%%%%%%%%%%%%%%%%%%%%%%
\subsection{Included Files}
\label{sec:include}

%%%%%%%%%%%%%%%%%%%%%%%%%%%%%%%%%%%%%%%%
\DescribeMacro{\childdocmain}
To use the package, add the commands
\begin{center}
\begin{tabular}{l}
|\input{childdoc.def}|\\
|\childdocmain{}|\\
\end{tabular}
\end{center}
at the very top of the main \LaTeX{} file,
in particular \emph{before} the |\documentclass| statement!
The argument of |\childdocmain| should be left empty
(but it must be present).

%%%%%%%%%%%%%%%%%%%%%%%%%%%%%%%%%%%%%%%%
\DescribeMacro{\childdocof}
Furthermore, add the commands
\begin{center}
\begin{tabular}{l}
|\input{childdoc.def}|\\
|\childdocof{|\textit{main}|}|\\
\end{tabular}
\end{center}
at the top of every child file \textit{child}
which is included by |\include{|\textit{child}|}|
from within the main file
(or at least for those files to be compiled individually).
The argument \textit{main} must be the filename of the main file.

There are a couple of
considerations in setting up the main and child documents:

%%%%%%%%%%%%%%%%%%%%%%%%%%%%%%%%%%%%%%%%
\paragraph{Restrictions.}

Please note the following restrictions:
\begin{itemize}
\item
|\childdocmain| must be called with one argument \textit{main}
to ensure compatibility with earlier version of the package.
It must either be empty (|\childdocmain{}|)
or precisely match the filename of the main file in which it is specified.
See \secref{sec:detection} for further information.
\item
The filename \textit{main} must be specified without the |.tex| extension.
\item
The filename \textit{main} is case sensitive
(even in case-insensitive file systems)
due to internal string comparison.
\item
The argument \textit{main} should be fully expanded, it cannot be a macro.
\item
Subdirectories and special characters should be avoided in filenames.
\item
The command |\childdocmain{|\textit{main}|}| must be followed by a whitespace.
It should not be followed immediately by another command
or by a comment mark `|%|'.
This is because the \TeX{} parser reads the token immediately following
the argument of |\childdocmain| and puts it
at the beginning of every child section;
however, a white\-space is ignored.
\end{itemize}

%%%%%%%%%%%%%%%%%%%%%%%%%%%%%%%%%%%%%%%%
\paragraph{Content of Main File.}

It is advisable to place all content in the child files included by |\include|.
Any output contained in the main file will appear in all child documents
unless suppressed manually;
it cannot be suppressed automatically by the |\includeonly| directive
and thus should normally be avoided.
A method to include some content in the main file
by means of conditional processing is described in \secref{sec:conditional}.

%%%%%%%%%%%%%%%%%%%%%%%%%%%%%%%%%%%%%%%%
\paragraph{Page Numbering.}

When only a part of the document is compiled,
the appropriate numbering of pages
(as well as other status parameters)
is determined from the |.aux| files.
The latter contain information from previous passes.
However this information needs to propagate through
all intermediate child documents.
Therefore the page numbering in child documents may well
be inconsistent until the complete document is compiled at least once.

A useful (if unconventional) way to always ensure a consistent
page numbering is to restart the numbering in each child document
and denote the pages by `\textit{child}|.|\textit{page}'
where \textit{child} represents the chapter/section number of the child file.
This can be achieved by the command
|\numberwithin{page}{|\textit{child}|}|
of the \textsf{amsmath} package
where \textit{child} can be |chapter| or |section|
depending on the chosen structuring.
Alternatively, one can modify the macro |\thepage| appropriately
and reset the counter |page| at the start of each child file.

%%%%%%%%%%%%%%%%%%%%%%%%%%%%%%%%%%%%%%%%%%%%%%%%%%%%%%%%%%%%%%%%%%%%%%%%%%%%%%%%
\subsection{Conditional Processing}
\label{sec:conditional}

The package provides a mechanism to compile different versions
of a document. To customise the versions further some conditional processing
can come in handy to distinguish which version is being compiled.
The package provides two macros to describe the compilation context:

%%%%%%%%%%%%%%%%%%%%%%%%%%%%%%%%%%%%%%%%
\DescribeMacro{\ifchilddoc}
The conditional |\ifchilddoc| distinguishes between the compilation of
child documents and the main document:
%
\begin{center}
|\ifchilddoc |\textit{child-code}| |[|\||else |\textit{main-code}]| \||fi|
\end{center}

%%%%%%%%%%%%%%%%%%%%%%%%%%%%%%%%%%%%%%%%
\DescribeMacro{\childdocname}
\DescribeMacro{\childdocjob}
The macro |\childdocname| contains the filename (without extension)
of the main or child file being processed.
Note that |\childdocjob| will always contain the name of the main file.

%%%%%%%%%%%%%%%%%%%%%%%%%%%%%%%%%%%%%%%%
\paragraph{Title Page.}

Conditional processing can be used to include a title or banner page
in the main document when proper precautions are taken.
Importantly, the code in the main file should ensure that the page counter
(as well as other status parameters which are stored in the |.aux| files)
takes the same value after the conditional processing.
Otherwise the page numbers may take divergent values
depending on which part is compiled.

For example, a title page could be declared by:
%
\begin{center}
\begin{tabular}{l}
|\ifchilddoc\||else|\\
|\addtocounter{page}{-1}|\\
\textit{code for title page}\\
|\newpage|\\
|\||fi|
\end{tabular}
\end{center}
%
A banner page for the child documents can be generated by:
%
\begin{center}
\begin{tabular}{l}
|\ifchilddoc|\\
|\addtocounter{page}{-1}|\\
\textit{code for banner page}\\
|\newpage|\\
|\||fi|
\end{tabular}
\end{center}
%
Here one could write a message such as:
\begin{center}
|This is the part \childdocname{} of \childdocjob{}.|
\end{center}

%%%%%%%%%%%%%%%%%%%%%%%%%%%%%%%%%%%%%%%%%%%%%%%%%%%%%%%%%%%%%%%%%%%%%%%%%%%%%%%%
\subsection{Flags}
\label{sec:flags}

The package makes it easy to generate different versions
of the main or child documents.
To this end compilation flags can be defined
and assigned different default values.
They will be particularly useful in conjunction
with the forwarding mechanism described in \secref{sec:forward}.

For example, it may be useful to have a flag |\version|
which can be set to |draft| or |final|.
The document source will contain some conditional code
depending on the value of |\version|.
Suppose further, the flag should default to |final| for the main file
and to |draft| for child files
which is a natural assignment for editing the document.
This is achieved by placing the following code
in the preamble of the main document
(below the |\childdocmain| directive):
%
\begin{center}
\begin{tabular}{l}
|\ifchilddoc|\\
|\providecommand{\version}{draft}|\\
|\||else|\\
|\providecommand{\version}{final}|\\
|\||fi|
\end{tabular}
\end{center}
%
The definition by |\providecommand| makes sure
that previous definitions are not overwritten.
Further statements |\providecommand{\version}{...}|
can thus be added before the above code to override it.

For the main file, one might add a line
(between |\childdocmain| and the above block)
%
\begin{center}
|%\ifchilddoc\||else\providecommand{\version}{draft}\||fi|
\end{center}
%
which can be uncommented to produce a draft version.
Likewise one can add a line to the very top of a child file
(above the |\childdocof{|\textit{main}|}| directive)
%
\begin{center}
|%\providecommand{\version}{final}|
\end{center}
%
which can be uncommented to produce the final version of this child document.

%%%%%%%%%%%%%%%%%%%%%%%%%%%%%%%%%%%%%%%%%%%%%%%%%%%%%%%%%%%%%%%%%%%%%%%%%%%%%%%%
\subsection{Forwarding}
\label{sec:forward}

Different versions of the main or child documents
using compilation flags as described in \secref{sec:flags}
can be (permanently) stored in different files
for convenient compilation, viewing and distribution.
To this end, the package defines a command
to pass on compilation to a different file:

%%%%%%%%%%%%%%%%%%%%%%%%%%%%%%%%%%%%%%%%
\DescribeMacro{\childdocforward}
The command |\childdocforward| redirects processing to
another source file:
%
\begin{center}
\begin{tabular}{l}
|\input{childdoc.def}|\\
|\childdocforward[|\textit{main}|]{|\textit{dest}|}|\\
\end{tabular}
\end{center}
%
The argument \textit{dest} is the destination file
(without extension).
It should be the main file or one of the child files.
Note that further \textsf{childdoc} directives
such as |\childdocof| and |\childdocforward|
in the indicated file will be processed in this form.
The optional argument \textit{main}
passes on directly to the main file \textit{main}
while pretending to compile the child \textit{dest}.
This form behaves as if \textit{dest}
issues |\childdocof{|\textit{main}|}| right away,
and no further \textsf{childdoc} directives will be processed.

%%%%%%%%%%%%%%%%%%%%%%%%%%%%%%%%%%%%%%%%
\DescribeMacro{\...prefix}
In the alternative form |\childdocforwardprefix|,
%
\begin{center}
\begin{tabular}{l}
|\input{childdoc.def}|\\
|\childdocforwardprefix[|\textit{main}|]{|\textit{prefix}|}{|\textit{dest}|}|
\end{tabular}
\end{center}
%
the destination file is determined by a pattern
depending on the current file:
To make this work, the current file must be called
`{\textit{prefix}\hspace{0.2em}\textit{suffix}}'
with \textit{prefix} matching precisely the argument.
Processing is then passed on to the file
`{\textit{dest}\hspace{0.2em}\textit{suffix}}'.
Surely, the same effect is achieved by
directly specifying the
argument `{\textit{dest}\hspace{0.2em}\textit{suffix}}'
in the first form.
However, that requires to set up a different file
for each child. With the alternative form of the command
all these files can have exactly the same content
which simplifies setting them up and maintaining them.

For example, the following file |draft.tex|
with a compilation flag |\version| as described in \secref{sec:flags}
compiles the main document as a draft:
%
\begin{center}
\begin{tabular}{l}
|\def\version{draft}|\\
|\input{childdoc.def}|\\
|\childdocforward{|\textit{main}|}|
\end{tabular}
\end{center}
%
Likewise, the following files |final|\textit{nn}|.tex|
compile the final version of the child document
|child|\textit{nn}|.tex|:
%
\begin{center}
\begin{tabular}{l}
|\def\version{final}|\\
|\input{childdoc.def}|\\
|\childdocforwardprefix{final}{child}|
\end{tabular}
\end{center}
%

Note that when several versions of a main file and/or of each child file
are to be generated, it may be convenient to set up a |Makefile| or
shell script to automatise the process.

%%%%%%%%%%%%%%%%%%%%%%%%%%%%%%%%%%%%%%%%%%%%%%%%%%%%%%%%%%%%%%%%%%%%%%%%%%%%%%%%
\subsection{Command Line Processing}
\label{sec:commandline}

The effect of redirection files can also be achieved by invoking
the \LaTeX{} compiler with a more elaborate command line.
Most conveniently this should be done as part
of a shell script or a |Makefile|.

When using \textsf{childdoc} in the main file, the following
command lines effectively perform a redirection
(note that depending on the shell being used,
backslashes may have to be doubled: `|\|' $\to$ `|\\|'):
%
\begin{center}
|... -jobname "|\textit{target}|" |\\|"|[\textit{flags}]%
|\input{childdoc.def}\childdocforward[|\textit{main}|]{|\textit{dest}|}"|
\end{center}
%
Here \textit{target} is the name of the output file,
\textit{main} is the name of the main file
and \textit{dest} is the name of the main or child file to be processed
(all filenames without extensions).
The optional argument \textit{main} can be omitted
if \textit{main} matches \textit{dest}.
Optionally, compilation \textit{flags} can be defined via |\def| commands.
This command line makes the \TeX{} engine believe
it is compiling the file \textit{target}
whose content is specified as the latter parameter.
The provided code then forwards the processing to
\textit{main} or \textit{dest} as described in \secref{sec:forward}.

%%%%%%%%%%%%%%%%%%%%%%%%%%%%%%%%%%%%%%%%%%%%%%%%%%%%%%%%%%%%%%%%%%%%%%%%%%%%%%%%
\subsection{Include by Input}
\label{sec:input}

Including child documents by |\include| has some restrictions by design.
Most notably, the content of a child document always occupies
its own set of pages; pages cannot be shared between child documents.
Usually, this behaviour makes perfect sense
because each child document contain an essential part of the document.
However, in some situations it may be desirable to compose
a document from a collection of parts
without having mandatory page breaks between then.
For this case, the package
provides a mechanism to include parts
by |\input| which can also be processed individually.
However, by construction this mechanism
requires manual handling of the content to be output.

%%%%%%%%%%%%%%%%%%%%%%%%%%%%%%%%%%%%%%%%
\DescribeMacro{\ifchilddocmanual}
The main file should be prepared as usual, see \secref{sec:include}.
However, the document body must make a distinction
between processing of an individual part and of the main document, e.g.:
%
\begin{center}
\begin{tabular}{l}
|\ifchilddocmanual|\\
|\input{\childdocname}|\\
|\||else|\\
\textit{document body with }|\input{|\textit{part}|}|\\
|\||fi|
\end{tabular}
\end{center}
%
The conditional |\ifchilddocmanual| is true whenever
a part to be included by |\input| is being compiled,
and the name of the part is stored in |\childdocname|.

%%%%%%%%%%%%%%%%%%%%%%%%%%%%%%%%%%%%%%%%
\DescribeMacro{\childdocby}
Each part to be included by |\input| should start with:
%
\begin{center}
\begin{tabular}{l}
|\input{childdoc.def}|\\
|\childdocby{|\textit{main}|}|\\
\end{tabular}
\end{center}
%
The directive |\childdocby| is similar to |\childdocof|
described in \secref{sec:include},
but the subsequent selection of content must be done manually.
To that end, both |\ifchilddoc| and |\ifchilddocmanual|
will be true upon processing of a part,
and the name of the part is stored in |\childdocname|.
Note that |\jobname| will be set to the filename of the current part
so that each part receives an individual |.aux| file
that does not interfere with the |.aux| file(s) of the main document.
This behaviour can be altered by the alternative form
|\childdocby[*]{|\textit{main}|}| (with a non-empty optional argument)
which uses the |.aux| file of the main document
by setting |\jobname| to \textit{main}.

%%%%%%%%%%%%%%%%%%%%%%%%%%%%%%%%%%%%%%%%%%%%%%%%%%%%%%%%%%%%%%%%%%%%%%%%%%%%%%%%
\subsection{Driver Development}
\label{sec:driver}

The \textsf{childdoc} mechanism can also be use for the development
of definition files such as \LaTeX{} styles or classes.
This case differs from the above setup with multiple parts
included by |\include| in that no |\includeonly| should be invoked.
This can be achieved by starting the include file
(before |\ProvidesPackage|) with:
%
\begin{center}
\begin{tabular}{l}
|\input{childdoc.def}|\\
|\childdocforward{|\textit{main}|}|\\
\end{tabular}
\end{center}
%
or alternatively with:
%
\begin{center}
\begin{tabular}{l}
|\input{childdoc.def}|\\
|\childdocby{|\textit{main}|}|\\
\end{tabular}
\end{center}
%
Both forms have slightly different effects as described above.
The main file is prepared as usual, see \secref{sec:include}.

%%%%%%%%%%%%%%%%%%%%%%%%%%%%%%%%%%%%%%%%%%%%%%%%%%%%%%%%%%%%%%%%%%%%%%%%%%%%%%%%
\subsection{Legacy Detection}
\label{sec:detection}

The directive |\childdocmain| in the main file can detect
whether the complete document or merely a child is to be compiled
even without using the directive |\childdocof|.
This method is deprecated because it is less robust
and there is no compelling reason to use it;
it is merely provided for backward compatibility
and it may be removed in future versions.

If the detection mechanism is to be used,
it is mandatory to correctly specify
the filename of the main file as the argument of |\childdocmain|:
%
\begin{center}
\begin{tabular}{l}
|\input{childdoc.def}|\\
|\childdocmain{|\textit{main}|}|\\
\end{tabular}
\end{center}
%
If |\jobname| does not match the argument \textit{main} of |\childdocmain|,
it is assumed that |\jobname| points to the child file to be compiled.
When using |\childdocmain| with the main file specified as argument,
it suffices to start a child file
with just |\input{|\textit{main}|}|
without loading of the package and using |\childdocof|.
If instead all processing is done
with the appropriate \textsf{childdoc} directives,
the argument of \textit{main} of |\childdocmain| can be empty.

An alternative version of the command line processing described
in \secref{sec:commandline} using the detection mechanism reads:
%
\begin{center}
|... -jobname "|\textit{target}|" "|[\textit{flags}]%
[|\def\jobname{|\textit{dest}|}|]|\input{|\textit{main}|}"|
\end{center}

%%%%%%%%%%%%%%%%%%%%%%%%%%%%%%%%%%%%%%%%%%%%%%%%%%%%%%%%%%%%%%%%%%%%%%%%%%%%%%%%
\subsection{Manual Code}
\label{sec:manual}

In case one cannot be certain whether the definitions file |childdoc.def|
is installed on the target \TeX{} distribution
and one prefers not to ship it,
it is conceivable to paste a few relevant commands into the sources.

To that end, drop all statements |\input{childdoc.def}|
and perform the replacements as outlined below.
Instead of |\childdocmain{|\textit{main}|}| add the following code
to the top of the main file:
%
\begin{center}
\begin{tabular}{l}
|\||ifdefined\childdocname\endinput\||fi\newif\ifchilddoc|\\
|\edef\childdocname{\scantokens\expandafter{\jobname\noexpand}}|\\
|\def\childdocmain{|\textit{main}|}\||ifx\childdocmain\childdocname\||else|\\
|\childdoctrue\includeonly{\childdocname}\let\jobname\childdocmain\||fi|\\
\end{tabular}
\end{center}
%
Instead of |\childdocof{|\textit{main}|}| just include the main file
at the top of each child file:
%
\begin{center}
|\input{|\textit{main}|}|
\end{center}
%
A simple redirection |\childdocforward{|\textit{dest}|}| is achieved by:
%
\begin{center}
|\def\jobname{|\textit{dest}|}\input{\jobname}|
\end{center}
%
The redirection with prefix
|\childdocforwardprefix[|\textit{prefix}|]{|\textit{dest}|}|
is accomplished by:
%
\begin{center}
\begin{tabular}{l}
|{\edef\jobname{\scantokens\expandafter{\jobname\noexpand}}|\\
|\def\redirectjob |\textit{prefix}|#1~~~{\gdef\jobname{|\textit{dest}|#1}}|\\
|\expandafter\redirectjob\jobname~~~}\input{\jobname}|
\end{tabular}
\end{center}

In an alternative approach,
child documents can be compiled by a specific command line
without additional code or specific definitions:
%
\begin{center}
|... -jobname "|\textit{target}|" "|[\textit{flags}]%
|\includeonly{|\textit{dest}|}\input{|\textit{main}|}"|
\end{center}
%

%%%%%%%%%%%%%%%%%%%%%%%%%%%%%%%%%%%%%%%%%%%%%%%%%%%%%%%%%%%%%%%%%%%%%%%%%%%%%%%%
%%%%%%%%%%%%%%%%%%%%%%%%%%%%%%%%%%%%%%%%%%%%%%%%%%%%%%%%%%%%%%%%%%%%%%%%%%%%%%%%
\section{Information}

%%%%%%%%%%%%%%%%%%%%%%%%%%%%%%%%%%%%%%%%%%%%%%%%%%%%%%%%%%%%%%%%%%%%%%%%%%%%%%%%
\subsection{Copyright}

Copyright \copyright{} 2017--2018 Niklas Beisert

This work may be distributed and/or modified under the
conditions of the \LaTeX{} Project Public License, either version 1.3
of this license or (at your option) any later version.
The latest version of this license is in
  \url{http://www.latex-project.org/lppl.txt}
and version 1.3 or later is part of all distributions of \LaTeX{}
version 2005/12/01 or later.

This work has the LPPL maintenance status `maintained'.

The Current Maintainer of this work is Niklas Beisert.

This work consists of the files |README.txt|, |childdoc.ins| and |childdoc.dtx|
as well as the derived files |childdoc.def|, |cdocsamp.tex|
with |cdocsch1.tex|, |cdocsch2.tex|, |cdocspt3.tex|, |cdocspt4.tex|,
|cdocsdrf.tex|, |cdocsfn1.tex|, |cdocsfn2.tex|
as well as |childdoc.pdf|.

%%%%%%%%%%%%%%%%%%%%%%%%%%%%%%%%%%%%%%%%%%%%%%%%%%%%%%%%%%%%%%%%%%%%%%%%%%%%%%%%
\subsection{Files and Installation}

The package consists of the files:
%
\begin{center}
\begin{tabular}{ll}
    |README.txt|   & readme file \\
    |childdoc.ins| & installation file \\
    |childdoc.dtx| & source file \\
    |childdoc.def| & definition file \\
    |cdocsamp.tex| & sample main file \\
    |cdocsch1.tex| & sample include file \\
    |cdocsch2.tex| & sample include file \\
    |cdocspt3.tex| & sample part file \\
    |cdocspt4.tex| & sample part file \\
    |cdocsdrf.tex| & sample redirection file \\
    |cdocsfn1.tex| & sample redirection file \\
    |cdocsfn2.tex| & sample redirection file \\
    |childdoc.pdf| & manual
\end{tabular}
\end{center}
%
The distribution consists of the files
|README.txt|, |childdoc.ins| and |childdoc.dtx|.
%
\begin{itemize}
\item
Run (pdf)\LaTeX{} on |childdoc.dtx|
to compile the manual |childdoc.pdf| (this file).
\item
Run \LaTeX{} on |childdoc.ins| to create the definitions file |childdoc.def|
and the sample |cdocsamp.tex| with include files
|cdocsch1.tex|, |cdocsch2.tex|, |cdocspt3.tex|, |cdocspt4.tex|,
|cdocsdrf.tex|, |cdocsfn1.tex|, |cdocsfn2.tex|.
Then copy the file |childdoc.def| to an appropriate directory of your \LaTeX{}
distribution, e.g.\ \textit{texmf-root}|/tex/latex/childdoc|.
\end{itemize}

%%%%%%%%%%%%%%%%%%%%%%%%%%%%%%%%%%%%%%%%%%%%%%%%%%%%%%%%%%%%%%%%%%%%%%%%%%%%%%%%
\subsection{Related CTAN Packages}

There are several other packages which offer a similar functionality:
%
\begin{itemize}
\item
The packages
\href{http://ctan.org/pkg/docmute}{\textsf{docmute}},
\href{http://ctan.org/pkg/includex}{\textsf{includex}} and
\href{http://ctan.org/pkg/standalone}{\textsf{standalone}}
provide commands to include only the document body of
a child file thus allowing both files to be compiled individually.
\item
The packages \href{http://ctan.org/pkg/subdocs}{\textsf{subdocs}}
and \href{http://ctan.org/pkg/subfiles}{\textsf{subfiles}}
provide structures in which the main and child documents can be
encapsulated and allowing them to be compiled individually.
The inclusion mechanism is different from the conventional |\include|.
\item
The package \href{http://ctan.org/pkg/combine}{\textsf{combine}}
is an elaborate solution to combine several documents into one.
\end{itemize}
%
See also the CTAN topic \href{http://ctan.org/topic/subdocs}{\textsf{subdocs}}
for further related packages.
The present package differs from the above solutions in that
a document structure constructed with the conventional |\include| mechanism
just needs two extra commands at the top of every file
such that all constituent files can be compiled individually.

%%%%%%%%%%%%%%%%%%%%%%%%%%%%%%%%%%%%%%%%%%%%%%%%%%%%%%%%%%%%%%%%%%%%%%%%%%%%%%%%
%\subsection{Feature Suggestions}
%
%The following is a list of features which may be useful for future
%versions of this package:
%%
%\begin{itemize}
%\item
%\ldots
%\end{itemize}

%%%%%%%%%%%%%%%%%%%%%%%%%%%%%%%%%%%%%%%%%%%%%%%%%%%%%%%%%%%%%%%%%%%%%%%%%%%%%%%%
\subsection{Revision History}

%%%%%%%%%%%%%%%%%%%%%%%%%%%%%%%%%%%%%%%%
\paragraph{v2.0:} 2018/12/30

\begin{itemize}
\item
immediate forward processing
\item
added |\childdocby| mechanism
\item
manual restructured
\end{itemize}

%%%%%%%%%%%%%%%%%%%%%%%%%%%%%%%%%%%%%%%%
\paragraph{v1.6:} 2018/01/17

\begin{itemize}
\item
application for development of include files
\item
corrections to manual
\end{itemize}

%%%%%%%%%%%%%%%%%%%%%%%%%%%%%%%%%%%%%%%%
\paragraph{v1.5:} 2017/05/21

\begin{itemize}
\item
more complete structuring introduced
\item
|\childdocof| introduced
\item
|\childdoc| renamed to |\childdocmain|
\item
|\childredirect| renamed to |\childdocforward| and |\childdocforwardprefix|
and functionality expanded
\end{itemize}

%%%%%%%%%%%%%%%%%%%%%%%%%%%%%%%%%%%%%%%%
\paragraph{v1.0:} 2017/04/27

\begin{itemize}
\item
manual and install package
\item
first version published on CTAN
\end{itemize}

%%%%%%%%%%%%%%%%%%%%%%%%%%%%%%%%%%%%%%%%
\paragraph{v0.6:} 2017/04/26

\begin{itemize}
\item
redirection mechanism added
\end{itemize}

%%%%%%%%%%%%%%%%%%%%%%%%%%%%%%%%%%%%%%%%
\paragraph{v0.5:} 2017/04/26

\begin{itemize}
\item
functionality in definition file
\end{itemize}


%%%%%%%%%%%%%%%%%%%%%%%%%%%%%%%%%%%%%%%%%%%%%%%%%%%%%%%%%%%%%%%%%%%%%%%%%%%%%%%%
%%%%%%%%%%%%%%%%%%%%%%%%%%%%%%%%%%%%%%%%%%%%%%%%%%%%%%%%%%%%%%%%%%%%%%%%%%%%%%%%
%%%%%%%%%%%%%%%%%%%%%%%%%%%%%%%%%%%%%%%%%%%%%%%%%%%%%%%%%%%%%%%%%%%%%%%%%%%%%%%%
\appendix

\settowidth\MacroIndent{\rmfamily\scriptsize 000\ }

 \DocInput{childdoc.dtx}

\end{document}
%</driver>
% \fi
%
% %%%%%%%%%%%%%%%%%%%%%%%%%%%%%%%%%%%%%%%%%%%%%%%%%%%%%%%%%%%%%%%%%%%%%%%%%%%%%%
% %%%%%%%%%%%%%%%%%%%%%%%%%%%%%%%%%%%%%%%%%%%%%%%%%%%%%%%%%%%%%%%%%%%%%%%%%%%%%%
% \section{Sample}
%\iffalse
%<*samplemain>
%\fi
%
% The following presents a sample document
% with two chapters, two parts, a title page,
% a compile flag as well as three forwarding files to set the flag.
% It consists of eight |.tex| files:
% \begin{center}
% \begin{tabular}{ll}
% |cdocsamp.tex|&main file\\
% |cdocsch1.tex|&include file for chapter 1\\
% |cdocsch2.tex|&include file for chapter 2\\
% |cdocspt3.tex|&include file for part 3\\
% |cdocspt4.tex|&include file for part 4\\
% |cdocsdrf.tex|&forwarding file for main file in draft mode\\
% |cdocsfi1.tex|&forwarding file for final version of chapter 1\\
% |cdocsfi2.tex|&forwarding file for final version of chapter 2\\
% \end{tabular}
% \end{center}
% Each of the eight files can be compiled directly by the \LaTeX{} compiler.
%
% %%%%%%%%%%%%%%%%%%%%%%%%%%%%%%%%%%%%%%
% \paragraph{Main File.}
%
% The main file is called |cdocsamp.tex|.
%
% Load the \textsf{childdoc} definitions and
% declare the filename for the main document:
%    \begin{macrocode}
\input{childdoc.def}
\childdocmain{}
%    \end{macrocode}

% Optional override for |\version| flag:
%    \begin{macrocode}
%%\ifchilddoc\else\providecommand{\version}{draft}\fi
%    \end{macrocode}

% Define the default values for the |\version| flag
% (|final| for the main file and |draft| for childs):
%    \begin{macrocode}
\ifchilddoc
\providecommand{\version}{draft}
\else
\providecommand{\version}{final}
\fi
%    \end{macrocode}

% Load the standard document class:
%    \begin{macrocode}
\documentclass[12pt]{article}
%    \end{macrocode}

% Start the document body:
%    \begin{macrocode}
\begin{document}
%    \end{macrocode}

% Declare a title page.
% Print title, part of document being processed and version flag:
%    \begin{macrocode}
\addtocounter{page}{-1}
\begin{center}
{\LARGE\bfseries{}childdoc example\par}
\vspace{1cm}
\ifchilddoc
\ifchilddocmanual part\else chapter\fi:
`\childdocname' of `\childdocjob'\par
\else
main document: `\childdocjob'\par
\fi
version: \version\par
\end{center}
\newpage
%    \end{macrocode}

% Manually include selected file,
% otherwise process as usual:
%    \begin{macrocode}
\ifchilddocmanual
\section*{part `\childdocname'}
\input{\childdocname}
\else
%    \end{macrocode}

% Include the two chapters:
%    \begin{macrocode}
\include{cdocsch1}
\include{cdocsch2}
%    \end{macrocode}

% Include the two parts unless only chapters should be displayed:
%    \begin{macrocode}
\ifchilddoc\else
\section{part three}
\input{cdocspt3}
\section{part four}
\input{cdocspt4}
\fi
%    \end{macrocode}

% Process as usual until here:
%    \begin{macrocode}
\fi
%    \end{macrocode}

% End of document body:
%    \begin{macrocode}
\end{document}
%    \end{macrocode}
%\iffalse
%</samplemain>
%\fi
%
% %%%%%%%%%%%%%%%%%%%%%%%%%%%%%%%%%%%%%%
% \paragraph{Chapter Include Files.}
%
% The include files are called |cdocsch1.tex| and |cdocsch2.tex|.
%
%\iffalse
%<*samplechap1|samplechap2>
%\fi

% Optional override for |\version| flag:
%    \begin{macrocode}
%%\providecommand{\version}{final}
%    \end{macrocode}

% Include the main document:
%    \begin{macrocode}
\input{childdoc.def}
\childdocof{cdocsamp}
%    \end{macrocode}

%\iffalse
%</samplechap1|samplechap2>
%\fi
%
%\iffalse
%<*samplechap1>
%\fi
% Some text for chapter 1:
%    \begin{macrocode}
\section{one}
some text in chapter one
%    \end{macrocode}

%\iffalse
%</samplechap1>
%\fi
% Some text for chapter 2:
%\iffalse
%<*samplechap2>
%\fi
%    \begin{macrocode}
\section{two}
more text in chapter two
%    \end{macrocode}

%\iffalse
%</samplechap2>
%\fi
%
% %%%%%%%%%%%%%%%%%%%%%%%%%%%%%%%%%%%%%%
% \paragraph{Part Include Files.}
%
% The include files are called |cdocspt3.tex| and |cdocspt4.tex|.
%
%\iffalse
%<*samplepart3|samplepart4>
%\fi

% Optional override for |\version| flag:
%    \begin{macrocode}
%%\providecommand{\version}{final}
%    \end{macrocode}

% Include the main document:
%    \begin{macrocode}
\input{childdoc.def}
\childdocby{cdocsamp}
%    \end{macrocode}

%\iffalse
%</samplepart3|samplepart4>
%\fi
%
%\iffalse
%<*samplepart3>
%\fi
% Some text for part 3:
%    \begin{macrocode}
some text in part three
%    \end{macrocode}

%\iffalse
%</samplepart3>
%\fi
% Some text for part 4:
%\iffalse
%<*samplepart4>
%\fi
%    \begin{macrocode}
more text in part four
%    \end{macrocode}

%\iffalse
%</samplepart4>
%\fi
%
% %%%%%%%%%%%%%%%%%%%%%%%%%%%%%%%%%%%%%%
% \paragraph{Forwarding for a Complete Draft.}
%
% The following forwarding file |cdocsdrf.tex|
% compiles the main document in draft mode:
%\iffalse
%<*sampledraft>
%\fi
%    \begin{macrocode}
\def\version{draft}
\input{childdoc.def}
\childdocforward{cdocsamp}
%    \end{macrocode}

%\iffalse
%</sampledraft>
%\fi
%
% %%%%%%%%%%%%%%%%%%%%%%%%%%%%%%%%%%%%%%
% \paragraph{Forwarding for Final Version of the Chapters.}
%
% The following forwarding files |cdocsfn1.tex| and |cdocsfn2.tex|
% (with identical content)
% compile the final versions of the child documents
% |cdocsch1.tex| and |cdocsch2.tex|, respectively:
%\iffalse
%<*samplefinal>
%\fi
%    \begin{macrocode}
\def\version{final}
\input{childdoc.def}
\childdocforwardprefix[cdocsamp]{cdocsfn}{cdocsch}
%    \end{macrocode}

%\iffalse
%</samplefinal>
%\fi
%
% %%%%%%%%%%%%%%%%%%%%%%%%%%%%%%%%%%%%%%
% \paragraph{Command Line Processing.}
%
% The following three command lines generate the output files
% |cdocscld|, |cdocscl1| and |cdocscl2|
% which should be identical to
% |cdocsdrf|, |cdocsch1| and |cdocsfn2|, respectively:
% \begin{center}
% \begin{tabular}{l}
% |latex -jobname cdocscld \|\\
% |  "\def\version{draft}\input{childdoc.def}\childdocforward{cdocsamp}"|\\
% |latex -jobname cdocscl1 \|\\
% |  "\input{childdoc.def}\childdocforward[cdocsamp]{cdocsch1}"|\\
% |latex -jobname cdocscl2 \|\\
% |  "\def\version{final}\input{childdoc.def}\childdocforward{cdocsch2}"|
% \end{tabular}
% \end{center}
% Note that the trailing backslash on each first line
% merely continues the input to the second line
% (for convenient cut ant paste).
% Furthermore, the command |latex| can be replaced by any
% of its alternative versions such as |pdflatex|.
%
% %%%%%%%%%%%%%%%%%%%%%%%%%%%%%%%%%%%%%%%%%%%%%%%%%%%%%%%%%%%%%%%%%%%%%%%%%%%%%%
% %%%%%%%%%%%%%%%%%%%%%%%%%%%%%%%%%%%%%%%%%%%%%%%%%%%%%%%%%%%%%%%%%%%%%%%%%%%%%%
% \section{Implementation}
%\iffalse
%<*package>
%\fi
%
% This section describes the definitions file |childdoc.def|.

% The definitions cannot be loaded using |\usepackage| or |\RequirePackage|
% which has a mechanism to prevent loading a style file more than once.
% When loading the definitions by means of |\input|
% multiple instances have to be prevented manually:
%\iffalse
%This code needs to be before the `\ProvidesFile' directive
%which is defined at the beginning of this file.
%Therefore it is also placed there and commented out here.
%</package>
%<*discard>
%\fi
%    \begin{macrocode}
\ifdefined\childdocmain\endinput\fi
%    \end{macrocode}
%\iffalse
%</discard>
%<*package>
%\fi
%
% \macro{\ifchilddoc}
% \macro{\ifchilddocmanual}
% The conditional |\ifchilddoc| tells whether a
% child (true) or main (false) document is being compiled.
% The conditional |\ifchilddocmanual| tells whether
% the |\includeonly| mechanism is used (false) or
% the selection of child files must be performed manually (true).
% The definitions initialise to false:
%    \begin{macrocode}
\newif\ifchilddoc
\newif\ifchilddocmanual
%    \end{macrocode}

% \macro{\childdocname}
% \macro{\childdocjob}
% The macro |\childdocname| stores the name of the main document
% to be compiled. The macro |\childdocjob| stores the name of
% the document on which the \LaTeX{} compiler was originally invoked.
% The content of |\jobname| cannot be compared
% to filenames specified in the source due to different catcodes.
% The following code rescans |\jobname|, stores the result
% in |\childdocname| and saves a copy in |\childdocjob|:
%    \begin{macrocode}
\edef\childdocname{\scantokens\expandafter{\jobname\noexpand}}
\let\childdocjob\childdocname
%    \end{macrocode}

% \macro{\childdocdisable}
% The macro |\childdocdisable| prevents the main file
% from being processed more than once.
% At this stage, the main document command |\childdocmain|
% is assumed to be called once again where it should do nothing.
% Any subsequent call to it should prevent
% a secondary processing of the main document
% It overwrites the forwarding commands
% |\childdocof| and |\childdocforward|
% with empty macros to prevent further inclusions of the main document:
%    \begin{macrocode}
\newcommand{\childdocdisable}
{
  \renewcommand{\childdocmain}[1]{\renewcommand{\childdocmain}[1]{\endinput}}
  \renewcommand{\childdocof}[1]{}
  \renewcommand{\childdocby}[2][]{}
  \renewcommand{\childdocforward}[2][]{}
  \renewcommand{\childdocdisable}{}
}
%    \end{macrocode}

% \macro{\childdocmain}
% The macro |\childdocmain| is to be called at the top of the main file
% with nothing or the main filename (without extension) as argument.
% First, it breaks loops.
% If the argument is not empty and does not match |\childdocname|
% (which is set by the first inclusion of |childdoc.def|),
% |\ifchilddoc| is set to true, |\includeonly| is applied to the child file
% and |\jobname| is set to the main file
% (for proper handling of |.aux| files):
%    \begin{macrocode}
\newcommand{\childdocmain}[1]
{
  \childdocdisable\childdocmain{}
  \if?#1?\else
    \begingroup
      \def\childdoctmp{#1}
      \ifx\childdoctmp\childdocname
        \def\childdoctmp{}
      \else
        \def\childdoctmp
        {
          \childdoctrue
          \includeonly{\childdocname}
          \def\childdocjob{#1}
          \def\jobname{#1}
        }
      \fi
      \expandafter
    \endgroup
    \childdoctmp
  \fi
}
%    \end{macrocode}

% \macro{\childdocof}
% The command |\childdocof| redirects
% compilation to the main file |#1|.
%    \begin{macrocode}
\newcommand{\childdocof}[1]
{
  \childdocdisable
  \childdoctrue
  \includeonly{\childdocname}
  \def\jobname{#1}
  \def\childdocjob{#1}
  \input{#1}
}
%    \end{macrocode}

% \macro{\childdocby}
% The command |\childdocby| ....
%    \begin{macrocode}
\newcommand{\childdocby}[2][]
{
  \childdocdisable
  \childdoctrue
  \childdocmanualtrue
  \if?#1?\else
    \def\jobname{#2}
  \fi
  \def\childdocjob{#2}
  \input{#2}
  \endinput
}
%    \end{macrocode}

% \macro{\childdocforward}
% The command |\childdocforward| redirects
% compilation to the main file or
% (if the optional argument is given) a child file.
% Parameters are set as if the main file
% or a child file starting with |\childdocof| was compiled.
% Then compilation is handed over to the main file:
%    \begin{macrocode}
\newcommand{\childdocforward}[2][]
{
  \begingroup
    \if?#1?
      \def\childdoctmp
      {
        \def\childdocname{#2}
        \def\childdocjob{#2}
        \def\jobname{#2}
        \input{#2}
        \endinput
      }
    \else
      \def\childdoctmp
      {
        \childdocdisable
        \def\childdocname{#2}
        \childdoctrue
        \includeonly{#2}
        \def\childdocjob{#1}
        \def\jobname{#1}
        \input{#1}
        \endinput
      }
    \fi
    \expandafter
  \endgroup
  \childdoctmp
}
%    \end{macrocode}

% \macro{\childdocforwardprefix}
% The command |\childdocforwardprefix| redirects
% compilation to the main or a child file by means of a pattern.
% The prefix |#1| in the current filename is replaced by |#2|
% and the suffix of the current filename is kept
% (it is assumed that the filename does not contain the substring `|~~~|'
% which is used as a delimiter).
% Compilation is handed over to the new file by |\childdocforward|:
%    \begin{macrocode}
\newcommand{\childdocforwardprefix}[3][]
{
  \begingroup
    \def\childdocextract #2##1~~~{\def\childdoctmp{\childdocforward[#1]{#3##1}}}
    \expandafter\childdocextract\childdocname~~~
    \expandafter
  \endgroup
  \childdoctmp
}
%    \end{macrocode}

% \macro{\childdoc}
% The deprecated macro |\childdoc| is a legacy version of |\childdocmain|:
%    \begin{macrocode}
\newcommand{\childdoc}{\childdocmain}
%    \end{macrocode}

% \macro{\childdocredirect}
% The deprecated macro |\childdocredirect| is a legacy version
% of |\childdocforward| and |\childdocforwardprefix|:
%    \begin{macrocode}
\newcommand{\childdocredirect}[2][]
{
  \begingroup
    \if?#1?
      \def\childdoctmp{\childdocforward{#2}}
    \else
      \def\childdoctmp{\childdocforwardprefix{#1}{#2}}
    \fi
    \expandafter
  \endgroup
  \childdoctmp
}
%    \end{macrocode}

%\iffalse
%</package>
%\fi
%
\endinput

\childdocof{cdocsamp}
%    \end{macrocode}

%\iffalse
%</samplechap1|samplechap2>
%\fi
%
%\iffalse
%<*samplechap1>
%\fi
% Some text for chapter 1:
%    \begin{macrocode}
\section{one}
some text in chapter one
%    \end{macrocode}

%\iffalse
%</samplechap1>
%\fi
% Some text for chapter 2:
%\iffalse
%<*samplechap2>
%\fi
%    \begin{macrocode}
\section{two}
more text in chapter two
%    \end{macrocode}

%\iffalse
%</samplechap2>
%\fi
%
% %%%%%%%%%%%%%%%%%%%%%%%%%%%%%%%%%%%%%%
% \paragraph{Part Include Files.}
%
% The include files are called |cdocspt3.tex| and |cdocspt4.tex|.
%
%\iffalse
%<*samplepart3|samplepart4>
%\fi

% Optional override for |\version| flag:
%    \begin{macrocode}
%%\providecommand{\version}{final}
%    \end{macrocode}

% Include the main document:
%    \begin{macrocode}
% \iffalse
%
% childdoc.dtx Copyright (C) 2017-2018 Niklas Beisert
%
% This work may be distributed and/or modified under the
% conditions of the LaTeX Project Public License, either version 1.3
% of this license or (at your option) any later version.
% The latest version of this license is in
%   http://www.latex-project.org/lppl.txt
% and version 1.3 or later is part of all distributions of LaTeX
% version 2005/12/01 or later.
%
% This work has the LPPL maintenance status `maintained'.
%
% The Current Maintainer of this work is Niklas Beisert.
%
% This work consists of the files childdoc.dtx and childdoc.ins
% and the derived files childdoc.def and cdocsamp.tex with
% cdocsch1.tex, cdocsch2.tex, cdocsdrf.tex, cdocsfn1.tex, cdocsfn2.tex.
%
%<package>\ifdefined\childdocmain\endinput\fi
%<package>\ProvidesFile{childdoc.def}[2018/12/30 v2.0 child document driver]
%<samplemain>\ProvidesFile{cdocsamp.tex}[2018/12/30 v2.0 sample for childdoc]
%<*driver>
%\ProvidesFile{childdoc.drv}[2018/12/30 v2.0 childdoc reference manual file]
\PassOptionsToClass{10pt,a4paper}{article}
\documentclass{ltxdoc}

\usepackage[margin=35mm]{geometry}
\usepackage{hyperref}
\usepackage{hyperxmp}
\usepackage[usenames]{color}

\hypersetup{colorlinks=true}
\hypersetup{pdfstartview=FitH}
\hypersetup{pdfpagemode=UseNone}
\hypersetup{pdfsource={}}
\hypersetup{pdflang={en-UK}}
\hypersetup{pdfcopyright={Copyright 2017-2018 Niklas Beisert.
  This work may be distributed and/or modified under the
  conditions of the LaTeX Project Public License, either version 1.3
  of this license or (at your option) any later version.}}
\hypersetup{pdflicenseurl={http://www.latex-project.org/lppl.txt}}
\hypersetup{pdfcontactaddress={ETH Zurich, ITP, HIT K,
  Wolfgang-Pauli-Strasse 27}}
\hypersetup{pdfcontactpostcode={8093}}
\hypersetup{pdfcontactcity={Zurich}}
\hypersetup{pdfcontactcountry={Switzerland}}
\hypersetup{pdfcontactemail={nbeisert@itp.phys.ethz.ch}}
\hypersetup{pdfcontacturl={http://people.phys.ethz.ch/\xmptilde nbeisert/}}

\newcommand{\secref}[1]{\hyperref[#1]{section \ref*{#1}}}

\parskip1ex
\parindent0pt
\let\olditemize\itemize
\def\itemize{\olditemize\parskip0pt}

\begin{document}

\title{The \textsf{childdoc} Package}
\hypersetup{pdftitle={The childdoc Package}}
\author{Niklas Beisert\\[2ex]
  Institut f\"ur Theoretische Physik\\
  Eidgen\"ossische Technische Hochschule Z\"urich\\
  Wolfgang-Pauli-Strasse 27, 8093 Z\"urich, Switzerland\\[1ex]
  \href{mailto:nbeisert@itp.phys.ethz.ch}
  {\texttt{nbeisert@itp.phys.ethz.ch}}}
\hypersetup{pdfauthor={Niklas Beisert}}
\hypersetup{pdfsubject={Manual for the LaTeX2e Package childdoc}}
\date{30 December 2018, \textsf{v2.0}}
\maketitle

\begin{abstract}\noindent
\textsf{childdoc} is a \LaTeXe{} package
that enables the direct compilation
of document sections included by |\include|
to individual files.
\end{abstract}

\begingroup
\parskip0ex
\tableofcontents
\endgroup

%%%%%%%%%%%%%%%%%%%%%%%%%%%%%%%%%%%%%%%%%%%%%%%%%%%%%%%%%%%%%%%%%%%%%%%%%%%%%%%%
%%%%%%%%%%%%%%%%%%%%%%%%%%%%%%%%%%%%%%%%%%%%%%%%%%%%%%%%%%%%%%%%%%%%%%%%%%%%%%%%
\section{Introduction}

\LaTeX{} provides a mechanism to structure a large document (such as a book)
into a main file and several child files (containing the chapters)
using the |\include| command.
This mechanism is beneficial for documents
which span hundreds of pages in order to
make the source file(s) more manageable.
Moreover, compilation can be restricted to
selected child files by means of the |\includeonly| command.
The latter feature can be used to reduce the compilation time while editing
(this was significantly more useful in the earlier days of \LaTeX{})
or to generate a smaller document which is easier to navigate.
Another application of |\includeonly| is to generate
documents consisting of selected parts of the complete document.

However, there are a few drawbacks of the plain |\include| mechanism:
\begin{itemize}
\item
The child files cannot be compiled on their own,
they can only be compiled via the main file.
A naive editing environment
(such as a text editor with an option
to have the current file processed by \LaTeX)
may require one to switch to the main file before compiling;
attempting to compile the child file produces errors.
\item
The main file must be modified (each time)
to adjust the |\includeonly| command
to the present needs. This easily leaves the main file in a messy state.
\item
The generated document will always carry the filename
of the main document. This is inconvenient if
several child files are to be compiled and
to be kept for distribution.
\end{itemize}

The present package provides a simple interface
to make child files individually compilable by \LaTeX{}.
Compiling a child file then has the same effect as compiling
the main file with an |\includeonly| command
to select the appropriate child.
Moreover the generated document will carry the name of the child
rather than the main file.
This resolves all three above issues.

This feature is meant to make the editing of books,
thesis documents and lecture notes somewhat more convenient.
However, the package can also be used efficiently for
composing a series of documents (such as exercise sheets)
which are typically distributed individually.
It then assists the author in generating the individual documents
(potentially in different versions)
as well as a document containing the collected series.
Another application is in developing style files
or other kinds of included material
where compilation of the style file could redirect
to a sample or test file.

%%%%%%%%%%%%%%%%%%%%%%%%%%%%%%%%%%%%%%%%%%%%%%%%%%%%%%%%%%%%%%%%%%%%%%%%%%%%%%%%
%%%%%%%%%%%%%%%%%%%%%%%%%%%%%%%%%%%%%%%%%%%%%%%%%%%%%%%%%%%%%%%%%%%%%%%%%%%%%%%%
\section{Usage}

First of all, the package \textsf{childdoc} is \emph{not} a standard
\LaTeXe{} |.sty| style file! Therefore it needs to be invoked in
a non-standard way.

%%%%%%%%%%%%%%%%%%%%%%%%%%%%%%%%%%%%%%%%%%%%%%%%%%%%%%%%%%%%%%%%%%%%%%%%%%%%%%%%
\subsection{Included Files}
\label{sec:include}

%%%%%%%%%%%%%%%%%%%%%%%%%%%%%%%%%%%%%%%%
\DescribeMacro{\childdocmain}
To use the package, add the commands
\begin{center}
\begin{tabular}{l}
|\input{childdoc.def}|\\
|\childdocmain{}|\\
\end{tabular}
\end{center}
at the very top of the main \LaTeX{} file,
in particular \emph{before} the |\documentclass| statement!
The argument of |\childdocmain| should be left empty
(but it must be present).

%%%%%%%%%%%%%%%%%%%%%%%%%%%%%%%%%%%%%%%%
\DescribeMacro{\childdocof}
Furthermore, add the commands
\begin{center}
\begin{tabular}{l}
|\input{childdoc.def}|\\
|\childdocof{|\textit{main}|}|\\
\end{tabular}
\end{center}
at the top of every child file \textit{child}
which is included by |\include{|\textit{child}|}|
from within the main file
(or at least for those files to be compiled individually).
The argument \textit{main} must be the filename of the main file.

There are a couple of
considerations in setting up the main and child documents:

%%%%%%%%%%%%%%%%%%%%%%%%%%%%%%%%%%%%%%%%
\paragraph{Restrictions.}

Please note the following restrictions:
\begin{itemize}
\item
|\childdocmain| must be called with one argument \textit{main}
to ensure compatibility with earlier version of the package.
It must either be empty (|\childdocmain{}|)
or precisely match the filename of the main file in which it is specified.
See \secref{sec:detection} for further information.
\item
The filename \textit{main} must be specified without the |.tex| extension.
\item
The filename \textit{main} is case sensitive
(even in case-insensitive file systems)
due to internal string comparison.
\item
The argument \textit{main} should be fully expanded, it cannot be a macro.
\item
Subdirectories and special characters should be avoided in filenames.
\item
The command |\childdocmain{|\textit{main}|}| must be followed by a whitespace.
It should not be followed immediately by another command
or by a comment mark `|%|'.
This is because the \TeX{} parser reads the token immediately following
the argument of |\childdocmain| and puts it
at the beginning of every child section;
however, a white\-space is ignored.
\end{itemize}

%%%%%%%%%%%%%%%%%%%%%%%%%%%%%%%%%%%%%%%%
\paragraph{Content of Main File.}

It is advisable to place all content in the child files included by |\include|.
Any output contained in the main file will appear in all child documents
unless suppressed manually;
it cannot be suppressed automatically by the |\includeonly| directive
and thus should normally be avoided.
A method to include some content in the main file
by means of conditional processing is described in \secref{sec:conditional}.

%%%%%%%%%%%%%%%%%%%%%%%%%%%%%%%%%%%%%%%%
\paragraph{Page Numbering.}

When only a part of the document is compiled,
the appropriate numbering of pages
(as well as other status parameters)
is determined from the |.aux| files.
The latter contain information from previous passes.
However this information needs to propagate through
all intermediate child documents.
Therefore the page numbering in child documents may well
be inconsistent until the complete document is compiled at least once.

A useful (if unconventional) way to always ensure a consistent
page numbering is to restart the numbering in each child document
and denote the pages by `\textit{child}|.|\textit{page}'
where \textit{child} represents the chapter/section number of the child file.
This can be achieved by the command
|\numberwithin{page}{|\textit{child}|}|
of the \textsf{amsmath} package
where \textit{child} can be |chapter| or |section|
depending on the chosen structuring.
Alternatively, one can modify the macro |\thepage| appropriately
and reset the counter |page| at the start of each child file.

%%%%%%%%%%%%%%%%%%%%%%%%%%%%%%%%%%%%%%%%%%%%%%%%%%%%%%%%%%%%%%%%%%%%%%%%%%%%%%%%
\subsection{Conditional Processing}
\label{sec:conditional}

The package provides a mechanism to compile different versions
of a document. To customise the versions further some conditional processing
can come in handy to distinguish which version is being compiled.
The package provides two macros to describe the compilation context:

%%%%%%%%%%%%%%%%%%%%%%%%%%%%%%%%%%%%%%%%
\DescribeMacro{\ifchilddoc}
The conditional |\ifchilddoc| distinguishes between the compilation of
child documents and the main document:
%
\begin{center}
|\ifchilddoc |\textit{child-code}| |[|\||else |\textit{main-code}]| \||fi|
\end{center}

%%%%%%%%%%%%%%%%%%%%%%%%%%%%%%%%%%%%%%%%
\DescribeMacro{\childdocname}
\DescribeMacro{\childdocjob}
The macro |\childdocname| contains the filename (without extension)
of the main or child file being processed.
Note that |\childdocjob| will always contain the name of the main file.

%%%%%%%%%%%%%%%%%%%%%%%%%%%%%%%%%%%%%%%%
\paragraph{Title Page.}

Conditional processing can be used to include a title or banner page
in the main document when proper precautions are taken.
Importantly, the code in the main file should ensure that the page counter
(as well as other status parameters which are stored in the |.aux| files)
takes the same value after the conditional processing.
Otherwise the page numbers may take divergent values
depending on which part is compiled.

For example, a title page could be declared by:
%
\begin{center}
\begin{tabular}{l}
|\ifchilddoc\||else|\\
|\addtocounter{page}{-1}|\\
\textit{code for title page}\\
|\newpage|\\
|\||fi|
\end{tabular}
\end{center}
%
A banner page for the child documents can be generated by:
%
\begin{center}
\begin{tabular}{l}
|\ifchilddoc|\\
|\addtocounter{page}{-1}|\\
\textit{code for banner page}\\
|\newpage|\\
|\||fi|
\end{tabular}
\end{center}
%
Here one could write a message such as:
\begin{center}
|This is the part \childdocname{} of \childdocjob{}.|
\end{center}

%%%%%%%%%%%%%%%%%%%%%%%%%%%%%%%%%%%%%%%%%%%%%%%%%%%%%%%%%%%%%%%%%%%%%%%%%%%%%%%%
\subsection{Flags}
\label{sec:flags}

The package makes it easy to generate different versions
of the main or child documents.
To this end compilation flags can be defined
and assigned different default values.
They will be particularly useful in conjunction
with the forwarding mechanism described in \secref{sec:forward}.

For example, it may be useful to have a flag |\version|
which can be set to |draft| or |final|.
The document source will contain some conditional code
depending on the value of |\version|.
Suppose further, the flag should default to |final| for the main file
and to |draft| for child files
which is a natural assignment for editing the document.
This is achieved by placing the following code
in the preamble of the main document
(below the |\childdocmain| directive):
%
\begin{center}
\begin{tabular}{l}
|\ifchilddoc|\\
|\providecommand{\version}{draft}|\\
|\||else|\\
|\providecommand{\version}{final}|\\
|\||fi|
\end{tabular}
\end{center}
%
The definition by |\providecommand| makes sure
that previous definitions are not overwritten.
Further statements |\providecommand{\version}{...}|
can thus be added before the above code to override it.

For the main file, one might add a line
(between |\childdocmain| and the above block)
%
\begin{center}
|%\ifchilddoc\||else\providecommand{\version}{draft}\||fi|
\end{center}
%
which can be uncommented to produce a draft version.
Likewise one can add a line to the very top of a child file
(above the |\childdocof{|\textit{main}|}| directive)
%
\begin{center}
|%\providecommand{\version}{final}|
\end{center}
%
which can be uncommented to produce the final version of this child document.

%%%%%%%%%%%%%%%%%%%%%%%%%%%%%%%%%%%%%%%%%%%%%%%%%%%%%%%%%%%%%%%%%%%%%%%%%%%%%%%%
\subsection{Forwarding}
\label{sec:forward}

Different versions of the main or child documents
using compilation flags as described in \secref{sec:flags}
can be (permanently) stored in different files
for convenient compilation, viewing and distribution.
To this end, the package defines a command
to pass on compilation to a different file:

%%%%%%%%%%%%%%%%%%%%%%%%%%%%%%%%%%%%%%%%
\DescribeMacro{\childdocforward}
The command |\childdocforward| redirects processing to
another source file:
%
\begin{center}
\begin{tabular}{l}
|\input{childdoc.def}|\\
|\childdocforward[|\textit{main}|]{|\textit{dest}|}|\\
\end{tabular}
\end{center}
%
The argument \textit{dest} is the destination file
(without extension).
It should be the main file or one of the child files.
Note that further \textsf{childdoc} directives
such as |\childdocof| and |\childdocforward|
in the indicated file will be processed in this form.
The optional argument \textit{main}
passes on directly to the main file \textit{main}
while pretending to compile the child \textit{dest}.
This form behaves as if \textit{dest}
issues |\childdocof{|\textit{main}|}| right away,
and no further \textsf{childdoc} directives will be processed.

%%%%%%%%%%%%%%%%%%%%%%%%%%%%%%%%%%%%%%%%
\DescribeMacro{\...prefix}
In the alternative form |\childdocforwardprefix|,
%
\begin{center}
\begin{tabular}{l}
|\input{childdoc.def}|\\
|\childdocforwardprefix[|\textit{main}|]{|\textit{prefix}|}{|\textit{dest}|}|
\end{tabular}
\end{center}
%
the destination file is determined by a pattern
depending on the current file:
To make this work, the current file must be called
`{\textit{prefix}\hspace{0.2em}\textit{suffix}}'
with \textit{prefix} matching precisely the argument.
Processing is then passed on to the file
`{\textit{dest}\hspace{0.2em}\textit{suffix}}'.
Surely, the same effect is achieved by
directly specifying the
argument `{\textit{dest}\hspace{0.2em}\textit{suffix}}'
in the first form.
However, that requires to set up a different file
for each child. With the alternative form of the command
all these files can have exactly the same content
which simplifies setting them up and maintaining them.

For example, the following file |draft.tex|
with a compilation flag |\version| as described in \secref{sec:flags}
compiles the main document as a draft:
%
\begin{center}
\begin{tabular}{l}
|\def\version{draft}|\\
|\input{childdoc.def}|\\
|\childdocforward{|\textit{main}|}|
\end{tabular}
\end{center}
%
Likewise, the following files |final|\textit{nn}|.tex|
compile the final version of the child document
|child|\textit{nn}|.tex|:
%
\begin{center}
\begin{tabular}{l}
|\def\version{final}|\\
|\input{childdoc.def}|\\
|\childdocforwardprefix{final}{child}|
\end{tabular}
\end{center}
%

Note that when several versions of a main file and/or of each child file
are to be generated, it may be convenient to set up a |Makefile| or
shell script to automatise the process.

%%%%%%%%%%%%%%%%%%%%%%%%%%%%%%%%%%%%%%%%%%%%%%%%%%%%%%%%%%%%%%%%%%%%%%%%%%%%%%%%
\subsection{Command Line Processing}
\label{sec:commandline}

The effect of redirection files can also be achieved by invoking
the \LaTeX{} compiler with a more elaborate command line.
Most conveniently this should be done as part
of a shell script or a |Makefile|.

When using \textsf{childdoc} in the main file, the following
command lines effectively perform a redirection
(note that depending on the shell being used,
backslashes may have to be doubled: `|\|' $\to$ `|\\|'):
%
\begin{center}
|... -jobname "|\textit{target}|" |\\|"|[\textit{flags}]%
|\input{childdoc.def}\childdocforward[|\textit{main}|]{|\textit{dest}|}"|
\end{center}
%
Here \textit{target} is the name of the output file,
\textit{main} is the name of the main file
and \textit{dest} is the name of the main or child file to be processed
(all filenames without extensions).
The optional argument \textit{main} can be omitted
if \textit{main} matches \textit{dest}.
Optionally, compilation \textit{flags} can be defined via |\def| commands.
This command line makes the \TeX{} engine believe
it is compiling the file \textit{target}
whose content is specified as the latter parameter.
The provided code then forwards the processing to
\textit{main} or \textit{dest} as described in \secref{sec:forward}.

%%%%%%%%%%%%%%%%%%%%%%%%%%%%%%%%%%%%%%%%%%%%%%%%%%%%%%%%%%%%%%%%%%%%%%%%%%%%%%%%
\subsection{Include by Input}
\label{sec:input}

Including child documents by |\include| has some restrictions by design.
Most notably, the content of a child document always occupies
its own set of pages; pages cannot be shared between child documents.
Usually, this behaviour makes perfect sense
because each child document contain an essential part of the document.
However, in some situations it may be desirable to compose
a document from a collection of parts
without having mandatory page breaks between then.
For this case, the package
provides a mechanism to include parts
by |\input| which can also be processed individually.
However, by construction this mechanism
requires manual handling of the content to be output.

%%%%%%%%%%%%%%%%%%%%%%%%%%%%%%%%%%%%%%%%
\DescribeMacro{\ifchilddocmanual}
The main file should be prepared as usual, see \secref{sec:include}.
However, the document body must make a distinction
between processing of an individual part and of the main document, e.g.:
%
\begin{center}
\begin{tabular}{l}
|\ifchilddocmanual|\\
|\input{\childdocname}|\\
|\||else|\\
\textit{document body with }|\input{|\textit{part}|}|\\
|\||fi|
\end{tabular}
\end{center}
%
The conditional |\ifchilddocmanual| is true whenever
a part to be included by |\input| is being compiled,
and the name of the part is stored in |\childdocname|.

%%%%%%%%%%%%%%%%%%%%%%%%%%%%%%%%%%%%%%%%
\DescribeMacro{\childdocby}
Each part to be included by |\input| should start with:
%
\begin{center}
\begin{tabular}{l}
|\input{childdoc.def}|\\
|\childdocby{|\textit{main}|}|\\
\end{tabular}
\end{center}
%
The directive |\childdocby| is similar to |\childdocof|
described in \secref{sec:include},
but the subsequent selection of content must be done manually.
To that end, both |\ifchilddoc| and |\ifchilddocmanual|
will be true upon processing of a part,
and the name of the part is stored in |\childdocname|.
Note that |\jobname| will be set to the filename of the current part
so that each part receives an individual |.aux| file
that does not interfere with the |.aux| file(s) of the main document.
This behaviour can be altered by the alternative form
|\childdocby[*]{|\textit{main}|}| (with a non-empty optional argument)
which uses the |.aux| file of the main document
by setting |\jobname| to \textit{main}.

%%%%%%%%%%%%%%%%%%%%%%%%%%%%%%%%%%%%%%%%%%%%%%%%%%%%%%%%%%%%%%%%%%%%%%%%%%%%%%%%
\subsection{Driver Development}
\label{sec:driver}

The \textsf{childdoc} mechanism can also be use for the development
of definition files such as \LaTeX{} styles or classes.
This case differs from the above setup with multiple parts
included by |\include| in that no |\includeonly| should be invoked.
This can be achieved by starting the include file
(before |\ProvidesPackage|) with:
%
\begin{center}
\begin{tabular}{l}
|\input{childdoc.def}|\\
|\childdocforward{|\textit{main}|}|\\
\end{tabular}
\end{center}
%
or alternatively with:
%
\begin{center}
\begin{tabular}{l}
|\input{childdoc.def}|\\
|\childdocby{|\textit{main}|}|\\
\end{tabular}
\end{center}
%
Both forms have slightly different effects as described above.
The main file is prepared as usual, see \secref{sec:include}.

%%%%%%%%%%%%%%%%%%%%%%%%%%%%%%%%%%%%%%%%%%%%%%%%%%%%%%%%%%%%%%%%%%%%%%%%%%%%%%%%
\subsection{Legacy Detection}
\label{sec:detection}

The directive |\childdocmain| in the main file can detect
whether the complete document or merely a child is to be compiled
even without using the directive |\childdocof|.
This method is deprecated because it is less robust
and there is no compelling reason to use it;
it is merely provided for backward compatibility
and it may be removed in future versions.

If the detection mechanism is to be used,
it is mandatory to correctly specify
the filename of the main file as the argument of |\childdocmain|:
%
\begin{center}
\begin{tabular}{l}
|\input{childdoc.def}|\\
|\childdocmain{|\textit{main}|}|\\
\end{tabular}
\end{center}
%
If |\jobname| does not match the argument \textit{main} of |\childdocmain|,
it is assumed that |\jobname| points to the child file to be compiled.
When using |\childdocmain| with the main file specified as argument,
it suffices to start a child file
with just |\input{|\textit{main}|}|
without loading of the package and using |\childdocof|.
If instead all processing is done
with the appropriate \textsf{childdoc} directives,
the argument of \textit{main} of |\childdocmain| can be empty.

An alternative version of the command line processing described
in \secref{sec:commandline} using the detection mechanism reads:
%
\begin{center}
|... -jobname "|\textit{target}|" "|[\textit{flags}]%
[|\def\jobname{|\textit{dest}|}|]|\input{|\textit{main}|}"|
\end{center}

%%%%%%%%%%%%%%%%%%%%%%%%%%%%%%%%%%%%%%%%%%%%%%%%%%%%%%%%%%%%%%%%%%%%%%%%%%%%%%%%
\subsection{Manual Code}
\label{sec:manual}

In case one cannot be certain whether the definitions file |childdoc.def|
is installed on the target \TeX{} distribution
and one prefers not to ship it,
it is conceivable to paste a few relevant commands into the sources.

To that end, drop all statements |\input{childdoc.def}|
and perform the replacements as outlined below.
Instead of |\childdocmain{|\textit{main}|}| add the following code
to the top of the main file:
%
\begin{center}
\begin{tabular}{l}
|\||ifdefined\childdocname\endinput\||fi\newif\ifchilddoc|\\
|\edef\childdocname{\scantokens\expandafter{\jobname\noexpand}}|\\
|\def\childdocmain{|\textit{main}|}\||ifx\childdocmain\childdocname\||else|\\
|\childdoctrue\includeonly{\childdocname}\let\jobname\childdocmain\||fi|\\
\end{tabular}
\end{center}
%
Instead of |\childdocof{|\textit{main}|}| just include the main file
at the top of each child file:
%
\begin{center}
|\input{|\textit{main}|}|
\end{center}
%
A simple redirection |\childdocforward{|\textit{dest}|}| is achieved by:
%
\begin{center}
|\def\jobname{|\textit{dest}|}\input{\jobname}|
\end{center}
%
The redirection with prefix
|\childdocforwardprefix[|\textit{prefix}|]{|\textit{dest}|}|
is accomplished by:
%
\begin{center}
\begin{tabular}{l}
|{\edef\jobname{\scantokens\expandafter{\jobname\noexpand}}|\\
|\def\redirectjob |\textit{prefix}|#1~~~{\gdef\jobname{|\textit{dest}|#1}}|\\
|\expandafter\redirectjob\jobname~~~}\input{\jobname}|
\end{tabular}
\end{center}

In an alternative approach,
child documents can be compiled by a specific command line
without additional code or specific definitions:
%
\begin{center}
|... -jobname "|\textit{target}|" "|[\textit{flags}]%
|\includeonly{|\textit{dest}|}\input{|\textit{main}|}"|
\end{center}
%

%%%%%%%%%%%%%%%%%%%%%%%%%%%%%%%%%%%%%%%%%%%%%%%%%%%%%%%%%%%%%%%%%%%%%%%%%%%%%%%%
%%%%%%%%%%%%%%%%%%%%%%%%%%%%%%%%%%%%%%%%%%%%%%%%%%%%%%%%%%%%%%%%%%%%%%%%%%%%%%%%
\section{Information}

%%%%%%%%%%%%%%%%%%%%%%%%%%%%%%%%%%%%%%%%%%%%%%%%%%%%%%%%%%%%%%%%%%%%%%%%%%%%%%%%
\subsection{Copyright}

Copyright \copyright{} 2017--2018 Niklas Beisert

This work may be distributed and/or modified under the
conditions of the \LaTeX{} Project Public License, either version 1.3
of this license or (at your option) any later version.
The latest version of this license is in
  \url{http://www.latex-project.org/lppl.txt}
and version 1.3 or later is part of all distributions of \LaTeX{}
version 2005/12/01 or later.

This work has the LPPL maintenance status `maintained'.

The Current Maintainer of this work is Niklas Beisert.

This work consists of the files |README.txt|, |childdoc.ins| and |childdoc.dtx|
as well as the derived files |childdoc.def|, |cdocsamp.tex|
with |cdocsch1.tex|, |cdocsch2.tex|, |cdocspt3.tex|, |cdocspt4.tex|,
|cdocsdrf.tex|, |cdocsfn1.tex|, |cdocsfn2.tex|
as well as |childdoc.pdf|.

%%%%%%%%%%%%%%%%%%%%%%%%%%%%%%%%%%%%%%%%%%%%%%%%%%%%%%%%%%%%%%%%%%%%%%%%%%%%%%%%
\subsection{Files and Installation}

The package consists of the files:
%
\begin{center}
\begin{tabular}{ll}
    |README.txt|   & readme file \\
    |childdoc.ins| & installation file \\
    |childdoc.dtx| & source file \\
    |childdoc.def| & definition file \\
    |cdocsamp.tex| & sample main file \\
    |cdocsch1.tex| & sample include file \\
    |cdocsch2.tex| & sample include file \\
    |cdocspt3.tex| & sample part file \\
    |cdocspt4.tex| & sample part file \\
    |cdocsdrf.tex| & sample redirection file \\
    |cdocsfn1.tex| & sample redirection file \\
    |cdocsfn2.tex| & sample redirection file \\
    |childdoc.pdf| & manual
\end{tabular}
\end{center}
%
The distribution consists of the files
|README.txt|, |childdoc.ins| and |childdoc.dtx|.
%
\begin{itemize}
\item
Run (pdf)\LaTeX{} on |childdoc.dtx|
to compile the manual |childdoc.pdf| (this file).
\item
Run \LaTeX{} on |childdoc.ins| to create the definitions file |childdoc.def|
and the sample |cdocsamp.tex| with include files
|cdocsch1.tex|, |cdocsch2.tex|, |cdocspt3.tex|, |cdocspt4.tex|,
|cdocsdrf.tex|, |cdocsfn1.tex|, |cdocsfn2.tex|.
Then copy the file |childdoc.def| to an appropriate directory of your \LaTeX{}
distribution, e.g.\ \textit{texmf-root}|/tex/latex/childdoc|.
\end{itemize}

%%%%%%%%%%%%%%%%%%%%%%%%%%%%%%%%%%%%%%%%%%%%%%%%%%%%%%%%%%%%%%%%%%%%%%%%%%%%%%%%
\subsection{Related CTAN Packages}

There are several other packages which offer a similar functionality:
%
\begin{itemize}
\item
The packages
\href{http://ctan.org/pkg/docmute}{\textsf{docmute}},
\href{http://ctan.org/pkg/includex}{\textsf{includex}} and
\href{http://ctan.org/pkg/standalone}{\textsf{standalone}}
provide commands to include only the document body of
a child file thus allowing both files to be compiled individually.
\item
The packages \href{http://ctan.org/pkg/subdocs}{\textsf{subdocs}}
and \href{http://ctan.org/pkg/subfiles}{\textsf{subfiles}}
provide structures in which the main and child documents can be
encapsulated and allowing them to be compiled individually.
The inclusion mechanism is different from the conventional |\include|.
\item
The package \href{http://ctan.org/pkg/combine}{\textsf{combine}}
is an elaborate solution to combine several documents into one.
\end{itemize}
%
See also the CTAN topic \href{http://ctan.org/topic/subdocs}{\textsf{subdocs}}
for further related packages.
The present package differs from the above solutions in that
a document structure constructed with the conventional |\include| mechanism
just needs two extra commands at the top of every file
such that all constituent files can be compiled individually.

%%%%%%%%%%%%%%%%%%%%%%%%%%%%%%%%%%%%%%%%%%%%%%%%%%%%%%%%%%%%%%%%%%%%%%%%%%%%%%%%
%\subsection{Feature Suggestions}
%
%The following is a list of features which may be useful for future
%versions of this package:
%%
%\begin{itemize}
%\item
%\ldots
%\end{itemize}

%%%%%%%%%%%%%%%%%%%%%%%%%%%%%%%%%%%%%%%%%%%%%%%%%%%%%%%%%%%%%%%%%%%%%%%%%%%%%%%%
\subsection{Revision History}

%%%%%%%%%%%%%%%%%%%%%%%%%%%%%%%%%%%%%%%%
\paragraph{v2.0:} 2018/12/30

\begin{itemize}
\item
immediate forward processing
\item
added |\childdocby| mechanism
\item
manual restructured
\end{itemize}

%%%%%%%%%%%%%%%%%%%%%%%%%%%%%%%%%%%%%%%%
\paragraph{v1.6:} 2018/01/17

\begin{itemize}
\item
application for development of include files
\item
corrections to manual
\end{itemize}

%%%%%%%%%%%%%%%%%%%%%%%%%%%%%%%%%%%%%%%%
\paragraph{v1.5:} 2017/05/21

\begin{itemize}
\item
more complete structuring introduced
\item
|\childdocof| introduced
\item
|\childdoc| renamed to |\childdocmain|
\item
|\childredirect| renamed to |\childdocforward| and |\childdocforwardprefix|
and functionality expanded
\end{itemize}

%%%%%%%%%%%%%%%%%%%%%%%%%%%%%%%%%%%%%%%%
\paragraph{v1.0:} 2017/04/27

\begin{itemize}
\item
manual and install package
\item
first version published on CTAN
\end{itemize}

%%%%%%%%%%%%%%%%%%%%%%%%%%%%%%%%%%%%%%%%
\paragraph{v0.6:} 2017/04/26

\begin{itemize}
\item
redirection mechanism added
\end{itemize}

%%%%%%%%%%%%%%%%%%%%%%%%%%%%%%%%%%%%%%%%
\paragraph{v0.5:} 2017/04/26

\begin{itemize}
\item
functionality in definition file
\end{itemize}


%%%%%%%%%%%%%%%%%%%%%%%%%%%%%%%%%%%%%%%%%%%%%%%%%%%%%%%%%%%%%%%%%%%%%%%%%%%%%%%%
%%%%%%%%%%%%%%%%%%%%%%%%%%%%%%%%%%%%%%%%%%%%%%%%%%%%%%%%%%%%%%%%%%%%%%%%%%%%%%%%
%%%%%%%%%%%%%%%%%%%%%%%%%%%%%%%%%%%%%%%%%%%%%%%%%%%%%%%%%%%%%%%%%%%%%%%%%%%%%%%%
\appendix

\settowidth\MacroIndent{\rmfamily\scriptsize 000\ }

 \DocInput{childdoc.dtx}

\end{document}
%</driver>
% \fi
%
% %%%%%%%%%%%%%%%%%%%%%%%%%%%%%%%%%%%%%%%%%%%%%%%%%%%%%%%%%%%%%%%%%%%%%%%%%%%%%%
% %%%%%%%%%%%%%%%%%%%%%%%%%%%%%%%%%%%%%%%%%%%%%%%%%%%%%%%%%%%%%%%%%%%%%%%%%%%%%%
% \section{Sample}
%\iffalse
%<*samplemain>
%\fi
%
% The following presents a sample document
% with two chapters, two parts, a title page,
% a compile flag as well as three forwarding files to set the flag.
% It consists of eight |.tex| files:
% \begin{center}
% \begin{tabular}{ll}
% |cdocsamp.tex|&main file\\
% |cdocsch1.tex|&include file for chapter 1\\
% |cdocsch2.tex|&include file for chapter 2\\
% |cdocspt3.tex|&include file for part 3\\
% |cdocspt4.tex|&include file for part 4\\
% |cdocsdrf.tex|&forwarding file for main file in draft mode\\
% |cdocsfi1.tex|&forwarding file for final version of chapter 1\\
% |cdocsfi2.tex|&forwarding file for final version of chapter 2\\
% \end{tabular}
% \end{center}
% Each of the eight files can be compiled directly by the \LaTeX{} compiler.
%
% %%%%%%%%%%%%%%%%%%%%%%%%%%%%%%%%%%%%%%
% \paragraph{Main File.}
%
% The main file is called |cdocsamp.tex|.
%
% Load the \textsf{childdoc} definitions and
% declare the filename for the main document:
%    \begin{macrocode}
\input{childdoc.def}
\childdocmain{}
%    \end{macrocode}

% Optional override for |\version| flag:
%    \begin{macrocode}
%%\ifchilddoc\else\providecommand{\version}{draft}\fi
%    \end{macrocode}

% Define the default values for the |\version| flag
% (|final| for the main file and |draft| for childs):
%    \begin{macrocode}
\ifchilddoc
\providecommand{\version}{draft}
\else
\providecommand{\version}{final}
\fi
%    \end{macrocode}

% Load the standard document class:
%    \begin{macrocode}
\documentclass[12pt]{article}
%    \end{macrocode}

% Start the document body:
%    \begin{macrocode}
\begin{document}
%    \end{macrocode}

% Declare a title page.
% Print title, part of document being processed and version flag:
%    \begin{macrocode}
\addtocounter{page}{-1}
\begin{center}
{\LARGE\bfseries{}childdoc example\par}
\vspace{1cm}
\ifchilddoc
\ifchilddocmanual part\else chapter\fi:
`\childdocname' of `\childdocjob'\par
\else
main document: `\childdocjob'\par
\fi
version: \version\par
\end{center}
\newpage
%    \end{macrocode}

% Manually include selected file,
% otherwise process as usual:
%    \begin{macrocode}
\ifchilddocmanual
\section*{part `\childdocname'}
\input{\childdocname}
\else
%    \end{macrocode}

% Include the two chapters:
%    \begin{macrocode}
\include{cdocsch1}
\include{cdocsch2}
%    \end{macrocode}

% Include the two parts unless only chapters should be displayed:
%    \begin{macrocode}
\ifchilddoc\else
\section{part three}
\input{cdocspt3}
\section{part four}
\input{cdocspt4}
\fi
%    \end{macrocode}

% Process as usual until here:
%    \begin{macrocode}
\fi
%    \end{macrocode}

% End of document body:
%    \begin{macrocode}
\end{document}
%    \end{macrocode}
%\iffalse
%</samplemain>
%\fi
%
% %%%%%%%%%%%%%%%%%%%%%%%%%%%%%%%%%%%%%%
% \paragraph{Chapter Include Files.}
%
% The include files are called |cdocsch1.tex| and |cdocsch2.tex|.
%
%\iffalse
%<*samplechap1|samplechap2>
%\fi

% Optional override for |\version| flag:
%    \begin{macrocode}
%%\providecommand{\version}{final}
%    \end{macrocode}

% Include the main document:
%    \begin{macrocode}
\input{childdoc.def}
\childdocof{cdocsamp}
%    \end{macrocode}

%\iffalse
%</samplechap1|samplechap2>
%\fi
%
%\iffalse
%<*samplechap1>
%\fi
% Some text for chapter 1:
%    \begin{macrocode}
\section{one}
some text in chapter one
%    \end{macrocode}

%\iffalse
%</samplechap1>
%\fi
% Some text for chapter 2:
%\iffalse
%<*samplechap2>
%\fi
%    \begin{macrocode}
\section{two}
more text in chapter two
%    \end{macrocode}

%\iffalse
%</samplechap2>
%\fi
%
% %%%%%%%%%%%%%%%%%%%%%%%%%%%%%%%%%%%%%%
% \paragraph{Part Include Files.}
%
% The include files are called |cdocspt3.tex| and |cdocspt4.tex|.
%
%\iffalse
%<*samplepart3|samplepart4>
%\fi

% Optional override for |\version| flag:
%    \begin{macrocode}
%%\providecommand{\version}{final}
%    \end{macrocode}

% Include the main document:
%    \begin{macrocode}
\input{childdoc.def}
\childdocby{cdocsamp}
%    \end{macrocode}

%\iffalse
%</samplepart3|samplepart4>
%\fi
%
%\iffalse
%<*samplepart3>
%\fi
% Some text for part 3:
%    \begin{macrocode}
some text in part three
%    \end{macrocode}

%\iffalse
%</samplepart3>
%\fi
% Some text for part 4:
%\iffalse
%<*samplepart4>
%\fi
%    \begin{macrocode}
more text in part four
%    \end{macrocode}

%\iffalse
%</samplepart4>
%\fi
%
% %%%%%%%%%%%%%%%%%%%%%%%%%%%%%%%%%%%%%%
% \paragraph{Forwarding for a Complete Draft.}
%
% The following forwarding file |cdocsdrf.tex|
% compiles the main document in draft mode:
%\iffalse
%<*sampledraft>
%\fi
%    \begin{macrocode}
\def\version{draft}
\input{childdoc.def}
\childdocforward{cdocsamp}
%    \end{macrocode}

%\iffalse
%</sampledraft>
%\fi
%
% %%%%%%%%%%%%%%%%%%%%%%%%%%%%%%%%%%%%%%
% \paragraph{Forwarding for Final Version of the Chapters.}
%
% The following forwarding files |cdocsfn1.tex| and |cdocsfn2.tex|
% (with identical content)
% compile the final versions of the child documents
% |cdocsch1.tex| and |cdocsch2.tex|, respectively:
%\iffalse
%<*samplefinal>
%\fi
%    \begin{macrocode}
\def\version{final}
\input{childdoc.def}
\childdocforwardprefix[cdocsamp]{cdocsfn}{cdocsch}
%    \end{macrocode}

%\iffalse
%</samplefinal>
%\fi
%
% %%%%%%%%%%%%%%%%%%%%%%%%%%%%%%%%%%%%%%
% \paragraph{Command Line Processing.}
%
% The following three command lines generate the output files
% |cdocscld|, |cdocscl1| and |cdocscl2|
% which should be identical to
% |cdocsdrf|, |cdocsch1| and |cdocsfn2|, respectively:
% \begin{center}
% \begin{tabular}{l}
% |latex -jobname cdocscld \|\\
% |  "\def\version{draft}\input{childdoc.def}\childdocforward{cdocsamp}"|\\
% |latex -jobname cdocscl1 \|\\
% |  "\input{childdoc.def}\childdocforward[cdocsamp]{cdocsch1}"|\\
% |latex -jobname cdocscl2 \|\\
% |  "\def\version{final}\input{childdoc.def}\childdocforward{cdocsch2}"|
% \end{tabular}
% \end{center}
% Note that the trailing backslash on each first line
% merely continues the input to the second line
% (for convenient cut ant paste).
% Furthermore, the command |latex| can be replaced by any
% of its alternative versions such as |pdflatex|.
%
% %%%%%%%%%%%%%%%%%%%%%%%%%%%%%%%%%%%%%%%%%%%%%%%%%%%%%%%%%%%%%%%%%%%%%%%%%%%%%%
% %%%%%%%%%%%%%%%%%%%%%%%%%%%%%%%%%%%%%%%%%%%%%%%%%%%%%%%%%%%%%%%%%%%%%%%%%%%%%%
% \section{Implementation}
%\iffalse
%<*package>
%\fi
%
% This section describes the definitions file |childdoc.def|.

% The definitions cannot be loaded using |\usepackage| or |\RequirePackage|
% which has a mechanism to prevent loading a style file more than once.
% When loading the definitions by means of |\input|
% multiple instances have to be prevented manually:
%\iffalse
%This code needs to be before the `\ProvidesFile' directive
%which is defined at the beginning of this file.
%Therefore it is also placed there and commented out here.
%</package>
%<*discard>
%\fi
%    \begin{macrocode}
\ifdefined\childdocmain\endinput\fi
%    \end{macrocode}
%\iffalse
%</discard>
%<*package>
%\fi
%
% \macro{\ifchilddoc}
% \macro{\ifchilddocmanual}
% The conditional |\ifchilddoc| tells whether a
% child (true) or main (false) document is being compiled.
% The conditional |\ifchilddocmanual| tells whether
% the |\includeonly| mechanism is used (false) or
% the selection of child files must be performed manually (true).
% The definitions initialise to false:
%    \begin{macrocode}
\newif\ifchilddoc
\newif\ifchilddocmanual
%    \end{macrocode}

% \macro{\childdocname}
% \macro{\childdocjob}
% The macro |\childdocname| stores the name of the main document
% to be compiled. The macro |\childdocjob| stores the name of
% the document on which the \LaTeX{} compiler was originally invoked.
% The content of |\jobname| cannot be compared
% to filenames specified in the source due to different catcodes.
% The following code rescans |\jobname|, stores the result
% in |\childdocname| and saves a copy in |\childdocjob|:
%    \begin{macrocode}
\edef\childdocname{\scantokens\expandafter{\jobname\noexpand}}
\let\childdocjob\childdocname
%    \end{macrocode}

% \macro{\childdocdisable}
% The macro |\childdocdisable| prevents the main file
% from being processed more than once.
% At this stage, the main document command |\childdocmain|
% is assumed to be called once again where it should do nothing.
% Any subsequent call to it should prevent
% a secondary processing of the main document
% It overwrites the forwarding commands
% |\childdocof| and |\childdocforward|
% with empty macros to prevent further inclusions of the main document:
%    \begin{macrocode}
\newcommand{\childdocdisable}
{
  \renewcommand{\childdocmain}[1]{\renewcommand{\childdocmain}[1]{\endinput}}
  \renewcommand{\childdocof}[1]{}
  \renewcommand{\childdocby}[2][]{}
  \renewcommand{\childdocforward}[2][]{}
  \renewcommand{\childdocdisable}{}
}
%    \end{macrocode}

% \macro{\childdocmain}
% The macro |\childdocmain| is to be called at the top of the main file
% with nothing or the main filename (without extension) as argument.
% First, it breaks loops.
% If the argument is not empty and does not match |\childdocname|
% (which is set by the first inclusion of |childdoc.def|),
% |\ifchilddoc| is set to true, |\includeonly| is applied to the child file
% and |\jobname| is set to the main file
% (for proper handling of |.aux| files):
%    \begin{macrocode}
\newcommand{\childdocmain}[1]
{
  \childdocdisable\childdocmain{}
  \if?#1?\else
    \begingroup
      \def\childdoctmp{#1}
      \ifx\childdoctmp\childdocname
        \def\childdoctmp{}
      \else
        \def\childdoctmp
        {
          \childdoctrue
          \includeonly{\childdocname}
          \def\childdocjob{#1}
          \def\jobname{#1}
        }
      \fi
      \expandafter
    \endgroup
    \childdoctmp
  \fi
}
%    \end{macrocode}

% \macro{\childdocof}
% The command |\childdocof| redirects
% compilation to the main file |#1|.
%    \begin{macrocode}
\newcommand{\childdocof}[1]
{
  \childdocdisable
  \childdoctrue
  \includeonly{\childdocname}
  \def\jobname{#1}
  \def\childdocjob{#1}
  \input{#1}
}
%    \end{macrocode}

% \macro{\childdocby}
% The command |\childdocby| ....
%    \begin{macrocode}
\newcommand{\childdocby}[2][]
{
  \childdocdisable
  \childdoctrue
  \childdocmanualtrue
  \if?#1?\else
    \def\jobname{#2}
  \fi
  \def\childdocjob{#2}
  \input{#2}
  \endinput
}
%    \end{macrocode}

% \macro{\childdocforward}
% The command |\childdocforward| redirects
% compilation to the main file or
% (if the optional argument is given) a child file.
% Parameters are set as if the main file
% or a child file starting with |\childdocof| was compiled.
% Then compilation is handed over to the main file:
%    \begin{macrocode}
\newcommand{\childdocforward}[2][]
{
  \begingroup
    \if?#1?
      \def\childdoctmp
      {
        \def\childdocname{#2}
        \def\childdocjob{#2}
        \def\jobname{#2}
        \input{#2}
        \endinput
      }
    \else
      \def\childdoctmp
      {
        \childdocdisable
        \def\childdocname{#2}
        \childdoctrue
        \includeonly{#2}
        \def\childdocjob{#1}
        \def\jobname{#1}
        \input{#1}
        \endinput
      }
    \fi
    \expandafter
  \endgroup
  \childdoctmp
}
%    \end{macrocode}

% \macro{\childdocforwardprefix}
% The command |\childdocforwardprefix| redirects
% compilation to the main or a child file by means of a pattern.
% The prefix |#1| in the current filename is replaced by |#2|
% and the suffix of the current filename is kept
% (it is assumed that the filename does not contain the substring `|~~~|'
% which is used as a delimiter).
% Compilation is handed over to the new file by |\childdocforward|:
%    \begin{macrocode}
\newcommand{\childdocforwardprefix}[3][]
{
  \begingroup
    \def\childdocextract #2##1~~~{\def\childdoctmp{\childdocforward[#1]{#3##1}}}
    \expandafter\childdocextract\childdocname~~~
    \expandafter
  \endgroup
  \childdoctmp
}
%    \end{macrocode}

% \macro{\childdoc}
% The deprecated macro |\childdoc| is a legacy version of |\childdocmain|:
%    \begin{macrocode}
\newcommand{\childdoc}{\childdocmain}
%    \end{macrocode}

% \macro{\childdocredirect}
% The deprecated macro |\childdocredirect| is a legacy version
% of |\childdocforward| and |\childdocforwardprefix|:
%    \begin{macrocode}
\newcommand{\childdocredirect}[2][]
{
  \begingroup
    \if?#1?
      \def\childdoctmp{\childdocforward{#2}}
    \else
      \def\childdoctmp{\childdocforwardprefix{#1}{#2}}
    \fi
    \expandafter
  \endgroup
  \childdoctmp
}
%    \end{macrocode}

%\iffalse
%</package>
%\fi
%
\endinput

\childdocby{cdocsamp}
%    \end{macrocode}

%\iffalse
%</samplepart3|samplepart4>
%\fi
%
%\iffalse
%<*samplepart3>
%\fi
% Some text for part 3:
%    \begin{macrocode}
some text in part three
%    \end{macrocode}

%\iffalse
%</samplepart3>
%\fi
% Some text for part 4:
%\iffalse
%<*samplepart4>
%\fi
%    \begin{macrocode}
more text in part four
%    \end{macrocode}

%\iffalse
%</samplepart4>
%\fi
%
% %%%%%%%%%%%%%%%%%%%%%%%%%%%%%%%%%%%%%%
% \paragraph{Forwarding for a Complete Draft.}
%
% The following forwarding file |cdocsdrf.tex|
% compiles the main document in draft mode:
%\iffalse
%<*sampledraft>
%\fi
%    \begin{macrocode}
\def\version{draft}
% \iffalse
%
% childdoc.dtx Copyright (C) 2017-2018 Niklas Beisert
%
% This work may be distributed and/or modified under the
% conditions of the LaTeX Project Public License, either version 1.3
% of this license or (at your option) any later version.
% The latest version of this license is in
%   http://www.latex-project.org/lppl.txt
% and version 1.3 or later is part of all distributions of LaTeX
% version 2005/12/01 or later.
%
% This work has the LPPL maintenance status `maintained'.
%
% The Current Maintainer of this work is Niklas Beisert.
%
% This work consists of the files childdoc.dtx and childdoc.ins
% and the derived files childdoc.def and cdocsamp.tex with
% cdocsch1.tex, cdocsch2.tex, cdocsdrf.tex, cdocsfn1.tex, cdocsfn2.tex.
%
%<package>\ifdefined\childdocmain\endinput\fi
%<package>\ProvidesFile{childdoc.def}[2018/12/30 v2.0 child document driver]
%<samplemain>\ProvidesFile{cdocsamp.tex}[2018/12/30 v2.0 sample for childdoc]
%<*driver>
%\ProvidesFile{childdoc.drv}[2018/12/30 v2.0 childdoc reference manual file]
\PassOptionsToClass{10pt,a4paper}{article}
\documentclass{ltxdoc}

\usepackage[margin=35mm]{geometry}
\usepackage{hyperref}
\usepackage{hyperxmp}
\usepackage[usenames]{color}

\hypersetup{colorlinks=true}
\hypersetup{pdfstartview=FitH}
\hypersetup{pdfpagemode=UseNone}
\hypersetup{pdfsource={}}
\hypersetup{pdflang={en-UK}}
\hypersetup{pdfcopyright={Copyright 2017-2018 Niklas Beisert.
  This work may be distributed and/or modified under the
  conditions of the LaTeX Project Public License, either version 1.3
  of this license or (at your option) any later version.}}
\hypersetup{pdflicenseurl={http://www.latex-project.org/lppl.txt}}
\hypersetup{pdfcontactaddress={ETH Zurich, ITP, HIT K,
  Wolfgang-Pauli-Strasse 27}}
\hypersetup{pdfcontactpostcode={8093}}
\hypersetup{pdfcontactcity={Zurich}}
\hypersetup{pdfcontactcountry={Switzerland}}
\hypersetup{pdfcontactemail={nbeisert@itp.phys.ethz.ch}}
\hypersetup{pdfcontacturl={http://people.phys.ethz.ch/\xmptilde nbeisert/}}

\newcommand{\secref}[1]{\hyperref[#1]{section \ref*{#1}}}

\parskip1ex
\parindent0pt
\let\olditemize\itemize
\def\itemize{\olditemize\parskip0pt}

\begin{document}

\title{The \textsf{childdoc} Package}
\hypersetup{pdftitle={The childdoc Package}}
\author{Niklas Beisert\\[2ex]
  Institut f\"ur Theoretische Physik\\
  Eidgen\"ossische Technische Hochschule Z\"urich\\
  Wolfgang-Pauli-Strasse 27, 8093 Z\"urich, Switzerland\\[1ex]
  \href{mailto:nbeisert@itp.phys.ethz.ch}
  {\texttt{nbeisert@itp.phys.ethz.ch}}}
\hypersetup{pdfauthor={Niklas Beisert}}
\hypersetup{pdfsubject={Manual for the LaTeX2e Package childdoc}}
\date{30 December 2018, \textsf{v2.0}}
\maketitle

\begin{abstract}\noindent
\textsf{childdoc} is a \LaTeXe{} package
that enables the direct compilation
of document sections included by |\include|
to individual files.
\end{abstract}

\begingroup
\parskip0ex
\tableofcontents
\endgroup

%%%%%%%%%%%%%%%%%%%%%%%%%%%%%%%%%%%%%%%%%%%%%%%%%%%%%%%%%%%%%%%%%%%%%%%%%%%%%%%%
%%%%%%%%%%%%%%%%%%%%%%%%%%%%%%%%%%%%%%%%%%%%%%%%%%%%%%%%%%%%%%%%%%%%%%%%%%%%%%%%
\section{Introduction}

\LaTeX{} provides a mechanism to structure a large document (such as a book)
into a main file and several child files (containing the chapters)
using the |\include| command.
This mechanism is beneficial for documents
which span hundreds of pages in order to
make the source file(s) more manageable.
Moreover, compilation can be restricted to
selected child files by means of the |\includeonly| command.
The latter feature can be used to reduce the compilation time while editing
(this was significantly more useful in the earlier days of \LaTeX{})
or to generate a smaller document which is easier to navigate.
Another application of |\includeonly| is to generate
documents consisting of selected parts of the complete document.

However, there are a few drawbacks of the plain |\include| mechanism:
\begin{itemize}
\item
The child files cannot be compiled on their own,
they can only be compiled via the main file.
A naive editing environment
(such as a text editor with an option
to have the current file processed by \LaTeX)
may require one to switch to the main file before compiling;
attempting to compile the child file produces errors.
\item
The main file must be modified (each time)
to adjust the |\includeonly| command
to the present needs. This easily leaves the main file in a messy state.
\item
The generated document will always carry the filename
of the main document. This is inconvenient if
several child files are to be compiled and
to be kept for distribution.
\end{itemize}

The present package provides a simple interface
to make child files individually compilable by \LaTeX{}.
Compiling a child file then has the same effect as compiling
the main file with an |\includeonly| command
to select the appropriate child.
Moreover the generated document will carry the name of the child
rather than the main file.
This resolves all three above issues.

This feature is meant to make the editing of books,
thesis documents and lecture notes somewhat more convenient.
However, the package can also be used efficiently for
composing a series of documents (such as exercise sheets)
which are typically distributed individually.
It then assists the author in generating the individual documents
(potentially in different versions)
as well as a document containing the collected series.
Another application is in developing style files
or other kinds of included material
where compilation of the style file could redirect
to a sample or test file.

%%%%%%%%%%%%%%%%%%%%%%%%%%%%%%%%%%%%%%%%%%%%%%%%%%%%%%%%%%%%%%%%%%%%%%%%%%%%%%%%
%%%%%%%%%%%%%%%%%%%%%%%%%%%%%%%%%%%%%%%%%%%%%%%%%%%%%%%%%%%%%%%%%%%%%%%%%%%%%%%%
\section{Usage}

First of all, the package \textsf{childdoc} is \emph{not} a standard
\LaTeXe{} |.sty| style file! Therefore it needs to be invoked in
a non-standard way.

%%%%%%%%%%%%%%%%%%%%%%%%%%%%%%%%%%%%%%%%%%%%%%%%%%%%%%%%%%%%%%%%%%%%%%%%%%%%%%%%
\subsection{Included Files}
\label{sec:include}

%%%%%%%%%%%%%%%%%%%%%%%%%%%%%%%%%%%%%%%%
\DescribeMacro{\childdocmain}
To use the package, add the commands
\begin{center}
\begin{tabular}{l}
|\input{childdoc.def}|\\
|\childdocmain{}|\\
\end{tabular}
\end{center}
at the very top of the main \LaTeX{} file,
in particular \emph{before} the |\documentclass| statement!
The argument of |\childdocmain| should be left empty
(but it must be present).

%%%%%%%%%%%%%%%%%%%%%%%%%%%%%%%%%%%%%%%%
\DescribeMacro{\childdocof}
Furthermore, add the commands
\begin{center}
\begin{tabular}{l}
|\input{childdoc.def}|\\
|\childdocof{|\textit{main}|}|\\
\end{tabular}
\end{center}
at the top of every child file \textit{child}
which is included by |\include{|\textit{child}|}|
from within the main file
(or at least for those files to be compiled individually).
The argument \textit{main} must be the filename of the main file.

There are a couple of
considerations in setting up the main and child documents:

%%%%%%%%%%%%%%%%%%%%%%%%%%%%%%%%%%%%%%%%
\paragraph{Restrictions.}

Please note the following restrictions:
\begin{itemize}
\item
|\childdocmain| must be called with one argument \textit{main}
to ensure compatibility with earlier version of the package.
It must either be empty (|\childdocmain{}|)
or precisely match the filename of the main file in which it is specified.
See \secref{sec:detection} for further information.
\item
The filename \textit{main} must be specified without the |.tex| extension.
\item
The filename \textit{main} is case sensitive
(even in case-insensitive file systems)
due to internal string comparison.
\item
The argument \textit{main} should be fully expanded, it cannot be a macro.
\item
Subdirectories and special characters should be avoided in filenames.
\item
The command |\childdocmain{|\textit{main}|}| must be followed by a whitespace.
It should not be followed immediately by another command
or by a comment mark `|%|'.
This is because the \TeX{} parser reads the token immediately following
the argument of |\childdocmain| and puts it
at the beginning of every child section;
however, a white\-space is ignored.
\end{itemize}

%%%%%%%%%%%%%%%%%%%%%%%%%%%%%%%%%%%%%%%%
\paragraph{Content of Main File.}

It is advisable to place all content in the child files included by |\include|.
Any output contained in the main file will appear in all child documents
unless suppressed manually;
it cannot be suppressed automatically by the |\includeonly| directive
and thus should normally be avoided.
A method to include some content in the main file
by means of conditional processing is described in \secref{sec:conditional}.

%%%%%%%%%%%%%%%%%%%%%%%%%%%%%%%%%%%%%%%%
\paragraph{Page Numbering.}

When only a part of the document is compiled,
the appropriate numbering of pages
(as well as other status parameters)
is determined from the |.aux| files.
The latter contain information from previous passes.
However this information needs to propagate through
all intermediate child documents.
Therefore the page numbering in child documents may well
be inconsistent until the complete document is compiled at least once.

A useful (if unconventional) way to always ensure a consistent
page numbering is to restart the numbering in each child document
and denote the pages by `\textit{child}|.|\textit{page}'
where \textit{child} represents the chapter/section number of the child file.
This can be achieved by the command
|\numberwithin{page}{|\textit{child}|}|
of the \textsf{amsmath} package
where \textit{child} can be |chapter| or |section|
depending on the chosen structuring.
Alternatively, one can modify the macro |\thepage| appropriately
and reset the counter |page| at the start of each child file.

%%%%%%%%%%%%%%%%%%%%%%%%%%%%%%%%%%%%%%%%%%%%%%%%%%%%%%%%%%%%%%%%%%%%%%%%%%%%%%%%
\subsection{Conditional Processing}
\label{sec:conditional}

The package provides a mechanism to compile different versions
of a document. To customise the versions further some conditional processing
can come in handy to distinguish which version is being compiled.
The package provides two macros to describe the compilation context:

%%%%%%%%%%%%%%%%%%%%%%%%%%%%%%%%%%%%%%%%
\DescribeMacro{\ifchilddoc}
The conditional |\ifchilddoc| distinguishes between the compilation of
child documents and the main document:
%
\begin{center}
|\ifchilddoc |\textit{child-code}| |[|\||else |\textit{main-code}]| \||fi|
\end{center}

%%%%%%%%%%%%%%%%%%%%%%%%%%%%%%%%%%%%%%%%
\DescribeMacro{\childdocname}
\DescribeMacro{\childdocjob}
The macro |\childdocname| contains the filename (without extension)
of the main or child file being processed.
Note that |\childdocjob| will always contain the name of the main file.

%%%%%%%%%%%%%%%%%%%%%%%%%%%%%%%%%%%%%%%%
\paragraph{Title Page.}

Conditional processing can be used to include a title or banner page
in the main document when proper precautions are taken.
Importantly, the code in the main file should ensure that the page counter
(as well as other status parameters which are stored in the |.aux| files)
takes the same value after the conditional processing.
Otherwise the page numbers may take divergent values
depending on which part is compiled.

For example, a title page could be declared by:
%
\begin{center}
\begin{tabular}{l}
|\ifchilddoc\||else|\\
|\addtocounter{page}{-1}|\\
\textit{code for title page}\\
|\newpage|\\
|\||fi|
\end{tabular}
\end{center}
%
A banner page for the child documents can be generated by:
%
\begin{center}
\begin{tabular}{l}
|\ifchilddoc|\\
|\addtocounter{page}{-1}|\\
\textit{code for banner page}\\
|\newpage|\\
|\||fi|
\end{tabular}
\end{center}
%
Here one could write a message such as:
\begin{center}
|This is the part \childdocname{} of \childdocjob{}.|
\end{center}

%%%%%%%%%%%%%%%%%%%%%%%%%%%%%%%%%%%%%%%%%%%%%%%%%%%%%%%%%%%%%%%%%%%%%%%%%%%%%%%%
\subsection{Flags}
\label{sec:flags}

The package makes it easy to generate different versions
of the main or child documents.
To this end compilation flags can be defined
and assigned different default values.
They will be particularly useful in conjunction
with the forwarding mechanism described in \secref{sec:forward}.

For example, it may be useful to have a flag |\version|
which can be set to |draft| or |final|.
The document source will contain some conditional code
depending on the value of |\version|.
Suppose further, the flag should default to |final| for the main file
and to |draft| for child files
which is a natural assignment for editing the document.
This is achieved by placing the following code
in the preamble of the main document
(below the |\childdocmain| directive):
%
\begin{center}
\begin{tabular}{l}
|\ifchilddoc|\\
|\providecommand{\version}{draft}|\\
|\||else|\\
|\providecommand{\version}{final}|\\
|\||fi|
\end{tabular}
\end{center}
%
The definition by |\providecommand| makes sure
that previous definitions are not overwritten.
Further statements |\providecommand{\version}{...}|
can thus be added before the above code to override it.

For the main file, one might add a line
(between |\childdocmain| and the above block)
%
\begin{center}
|%\ifchilddoc\||else\providecommand{\version}{draft}\||fi|
\end{center}
%
which can be uncommented to produce a draft version.
Likewise one can add a line to the very top of a child file
(above the |\childdocof{|\textit{main}|}| directive)
%
\begin{center}
|%\providecommand{\version}{final}|
\end{center}
%
which can be uncommented to produce the final version of this child document.

%%%%%%%%%%%%%%%%%%%%%%%%%%%%%%%%%%%%%%%%%%%%%%%%%%%%%%%%%%%%%%%%%%%%%%%%%%%%%%%%
\subsection{Forwarding}
\label{sec:forward}

Different versions of the main or child documents
using compilation flags as described in \secref{sec:flags}
can be (permanently) stored in different files
for convenient compilation, viewing and distribution.
To this end, the package defines a command
to pass on compilation to a different file:

%%%%%%%%%%%%%%%%%%%%%%%%%%%%%%%%%%%%%%%%
\DescribeMacro{\childdocforward}
The command |\childdocforward| redirects processing to
another source file:
%
\begin{center}
\begin{tabular}{l}
|\input{childdoc.def}|\\
|\childdocforward[|\textit{main}|]{|\textit{dest}|}|\\
\end{tabular}
\end{center}
%
The argument \textit{dest} is the destination file
(without extension).
It should be the main file or one of the child files.
Note that further \textsf{childdoc} directives
such as |\childdocof| and |\childdocforward|
in the indicated file will be processed in this form.
The optional argument \textit{main}
passes on directly to the main file \textit{main}
while pretending to compile the child \textit{dest}.
This form behaves as if \textit{dest}
issues |\childdocof{|\textit{main}|}| right away,
and no further \textsf{childdoc} directives will be processed.

%%%%%%%%%%%%%%%%%%%%%%%%%%%%%%%%%%%%%%%%
\DescribeMacro{\...prefix}
In the alternative form |\childdocforwardprefix|,
%
\begin{center}
\begin{tabular}{l}
|\input{childdoc.def}|\\
|\childdocforwardprefix[|\textit{main}|]{|\textit{prefix}|}{|\textit{dest}|}|
\end{tabular}
\end{center}
%
the destination file is determined by a pattern
depending on the current file:
To make this work, the current file must be called
`{\textit{prefix}\hspace{0.2em}\textit{suffix}}'
with \textit{prefix} matching precisely the argument.
Processing is then passed on to the file
`{\textit{dest}\hspace{0.2em}\textit{suffix}}'.
Surely, the same effect is achieved by
directly specifying the
argument `{\textit{dest}\hspace{0.2em}\textit{suffix}}'
in the first form.
However, that requires to set up a different file
for each child. With the alternative form of the command
all these files can have exactly the same content
which simplifies setting them up and maintaining them.

For example, the following file |draft.tex|
with a compilation flag |\version| as described in \secref{sec:flags}
compiles the main document as a draft:
%
\begin{center}
\begin{tabular}{l}
|\def\version{draft}|\\
|\input{childdoc.def}|\\
|\childdocforward{|\textit{main}|}|
\end{tabular}
\end{center}
%
Likewise, the following files |final|\textit{nn}|.tex|
compile the final version of the child document
|child|\textit{nn}|.tex|:
%
\begin{center}
\begin{tabular}{l}
|\def\version{final}|\\
|\input{childdoc.def}|\\
|\childdocforwardprefix{final}{child}|
\end{tabular}
\end{center}
%

Note that when several versions of a main file and/or of each child file
are to be generated, it may be convenient to set up a |Makefile| or
shell script to automatise the process.

%%%%%%%%%%%%%%%%%%%%%%%%%%%%%%%%%%%%%%%%%%%%%%%%%%%%%%%%%%%%%%%%%%%%%%%%%%%%%%%%
\subsection{Command Line Processing}
\label{sec:commandline}

The effect of redirection files can also be achieved by invoking
the \LaTeX{} compiler with a more elaborate command line.
Most conveniently this should be done as part
of a shell script or a |Makefile|.

When using \textsf{childdoc} in the main file, the following
command lines effectively perform a redirection
(note that depending on the shell being used,
backslashes may have to be doubled: `|\|' $\to$ `|\\|'):
%
\begin{center}
|... -jobname "|\textit{target}|" |\\|"|[\textit{flags}]%
|\input{childdoc.def}\childdocforward[|\textit{main}|]{|\textit{dest}|}"|
\end{center}
%
Here \textit{target} is the name of the output file,
\textit{main} is the name of the main file
and \textit{dest} is the name of the main or child file to be processed
(all filenames without extensions).
The optional argument \textit{main} can be omitted
if \textit{main} matches \textit{dest}.
Optionally, compilation \textit{flags} can be defined via |\def| commands.
This command line makes the \TeX{} engine believe
it is compiling the file \textit{target}
whose content is specified as the latter parameter.
The provided code then forwards the processing to
\textit{main} or \textit{dest} as described in \secref{sec:forward}.

%%%%%%%%%%%%%%%%%%%%%%%%%%%%%%%%%%%%%%%%%%%%%%%%%%%%%%%%%%%%%%%%%%%%%%%%%%%%%%%%
\subsection{Include by Input}
\label{sec:input}

Including child documents by |\include| has some restrictions by design.
Most notably, the content of a child document always occupies
its own set of pages; pages cannot be shared between child documents.
Usually, this behaviour makes perfect sense
because each child document contain an essential part of the document.
However, in some situations it may be desirable to compose
a document from a collection of parts
without having mandatory page breaks between then.
For this case, the package
provides a mechanism to include parts
by |\input| which can also be processed individually.
However, by construction this mechanism
requires manual handling of the content to be output.

%%%%%%%%%%%%%%%%%%%%%%%%%%%%%%%%%%%%%%%%
\DescribeMacro{\ifchilddocmanual}
The main file should be prepared as usual, see \secref{sec:include}.
However, the document body must make a distinction
between processing of an individual part and of the main document, e.g.:
%
\begin{center}
\begin{tabular}{l}
|\ifchilddocmanual|\\
|\input{\childdocname}|\\
|\||else|\\
\textit{document body with }|\input{|\textit{part}|}|\\
|\||fi|
\end{tabular}
\end{center}
%
The conditional |\ifchilddocmanual| is true whenever
a part to be included by |\input| is being compiled,
and the name of the part is stored in |\childdocname|.

%%%%%%%%%%%%%%%%%%%%%%%%%%%%%%%%%%%%%%%%
\DescribeMacro{\childdocby}
Each part to be included by |\input| should start with:
%
\begin{center}
\begin{tabular}{l}
|\input{childdoc.def}|\\
|\childdocby{|\textit{main}|}|\\
\end{tabular}
\end{center}
%
The directive |\childdocby| is similar to |\childdocof|
described in \secref{sec:include},
but the subsequent selection of content must be done manually.
To that end, both |\ifchilddoc| and |\ifchilddocmanual|
will be true upon processing of a part,
and the name of the part is stored in |\childdocname|.
Note that |\jobname| will be set to the filename of the current part
so that each part receives an individual |.aux| file
that does not interfere with the |.aux| file(s) of the main document.
This behaviour can be altered by the alternative form
|\childdocby[*]{|\textit{main}|}| (with a non-empty optional argument)
which uses the |.aux| file of the main document
by setting |\jobname| to \textit{main}.

%%%%%%%%%%%%%%%%%%%%%%%%%%%%%%%%%%%%%%%%%%%%%%%%%%%%%%%%%%%%%%%%%%%%%%%%%%%%%%%%
\subsection{Driver Development}
\label{sec:driver}

The \textsf{childdoc} mechanism can also be use for the development
of definition files such as \LaTeX{} styles or classes.
This case differs from the above setup with multiple parts
included by |\include| in that no |\includeonly| should be invoked.
This can be achieved by starting the include file
(before |\ProvidesPackage|) with:
%
\begin{center}
\begin{tabular}{l}
|\input{childdoc.def}|\\
|\childdocforward{|\textit{main}|}|\\
\end{tabular}
\end{center}
%
or alternatively with:
%
\begin{center}
\begin{tabular}{l}
|\input{childdoc.def}|\\
|\childdocby{|\textit{main}|}|\\
\end{tabular}
\end{center}
%
Both forms have slightly different effects as described above.
The main file is prepared as usual, see \secref{sec:include}.

%%%%%%%%%%%%%%%%%%%%%%%%%%%%%%%%%%%%%%%%%%%%%%%%%%%%%%%%%%%%%%%%%%%%%%%%%%%%%%%%
\subsection{Legacy Detection}
\label{sec:detection}

The directive |\childdocmain| in the main file can detect
whether the complete document or merely a child is to be compiled
even without using the directive |\childdocof|.
This method is deprecated because it is less robust
and there is no compelling reason to use it;
it is merely provided for backward compatibility
and it may be removed in future versions.

If the detection mechanism is to be used,
it is mandatory to correctly specify
the filename of the main file as the argument of |\childdocmain|:
%
\begin{center}
\begin{tabular}{l}
|\input{childdoc.def}|\\
|\childdocmain{|\textit{main}|}|\\
\end{tabular}
\end{center}
%
If |\jobname| does not match the argument \textit{main} of |\childdocmain|,
it is assumed that |\jobname| points to the child file to be compiled.
When using |\childdocmain| with the main file specified as argument,
it suffices to start a child file
with just |\input{|\textit{main}|}|
without loading of the package and using |\childdocof|.
If instead all processing is done
with the appropriate \textsf{childdoc} directives,
the argument of \textit{main} of |\childdocmain| can be empty.

An alternative version of the command line processing described
in \secref{sec:commandline} using the detection mechanism reads:
%
\begin{center}
|... -jobname "|\textit{target}|" "|[\textit{flags}]%
[|\def\jobname{|\textit{dest}|}|]|\input{|\textit{main}|}"|
\end{center}

%%%%%%%%%%%%%%%%%%%%%%%%%%%%%%%%%%%%%%%%%%%%%%%%%%%%%%%%%%%%%%%%%%%%%%%%%%%%%%%%
\subsection{Manual Code}
\label{sec:manual}

In case one cannot be certain whether the definitions file |childdoc.def|
is installed on the target \TeX{} distribution
and one prefers not to ship it,
it is conceivable to paste a few relevant commands into the sources.

To that end, drop all statements |\input{childdoc.def}|
and perform the replacements as outlined below.
Instead of |\childdocmain{|\textit{main}|}| add the following code
to the top of the main file:
%
\begin{center}
\begin{tabular}{l}
|\||ifdefined\childdocname\endinput\||fi\newif\ifchilddoc|\\
|\edef\childdocname{\scantokens\expandafter{\jobname\noexpand}}|\\
|\def\childdocmain{|\textit{main}|}\||ifx\childdocmain\childdocname\||else|\\
|\childdoctrue\includeonly{\childdocname}\let\jobname\childdocmain\||fi|\\
\end{tabular}
\end{center}
%
Instead of |\childdocof{|\textit{main}|}| just include the main file
at the top of each child file:
%
\begin{center}
|\input{|\textit{main}|}|
\end{center}
%
A simple redirection |\childdocforward{|\textit{dest}|}| is achieved by:
%
\begin{center}
|\def\jobname{|\textit{dest}|}\input{\jobname}|
\end{center}
%
The redirection with prefix
|\childdocforwardprefix[|\textit{prefix}|]{|\textit{dest}|}|
is accomplished by:
%
\begin{center}
\begin{tabular}{l}
|{\edef\jobname{\scantokens\expandafter{\jobname\noexpand}}|\\
|\def\redirectjob |\textit{prefix}|#1~~~{\gdef\jobname{|\textit{dest}|#1}}|\\
|\expandafter\redirectjob\jobname~~~}\input{\jobname}|
\end{tabular}
\end{center}

In an alternative approach,
child documents can be compiled by a specific command line
without additional code or specific definitions:
%
\begin{center}
|... -jobname "|\textit{target}|" "|[\textit{flags}]%
|\includeonly{|\textit{dest}|}\input{|\textit{main}|}"|
\end{center}
%

%%%%%%%%%%%%%%%%%%%%%%%%%%%%%%%%%%%%%%%%%%%%%%%%%%%%%%%%%%%%%%%%%%%%%%%%%%%%%%%%
%%%%%%%%%%%%%%%%%%%%%%%%%%%%%%%%%%%%%%%%%%%%%%%%%%%%%%%%%%%%%%%%%%%%%%%%%%%%%%%%
\section{Information}

%%%%%%%%%%%%%%%%%%%%%%%%%%%%%%%%%%%%%%%%%%%%%%%%%%%%%%%%%%%%%%%%%%%%%%%%%%%%%%%%
\subsection{Copyright}

Copyright \copyright{} 2017--2018 Niklas Beisert

This work may be distributed and/or modified under the
conditions of the \LaTeX{} Project Public License, either version 1.3
of this license or (at your option) any later version.
The latest version of this license is in
  \url{http://www.latex-project.org/lppl.txt}
and version 1.3 or later is part of all distributions of \LaTeX{}
version 2005/12/01 or later.

This work has the LPPL maintenance status `maintained'.

The Current Maintainer of this work is Niklas Beisert.

This work consists of the files |README.txt|, |childdoc.ins| and |childdoc.dtx|
as well as the derived files |childdoc.def|, |cdocsamp.tex|
with |cdocsch1.tex|, |cdocsch2.tex|, |cdocspt3.tex|, |cdocspt4.tex|,
|cdocsdrf.tex|, |cdocsfn1.tex|, |cdocsfn2.tex|
as well as |childdoc.pdf|.

%%%%%%%%%%%%%%%%%%%%%%%%%%%%%%%%%%%%%%%%%%%%%%%%%%%%%%%%%%%%%%%%%%%%%%%%%%%%%%%%
\subsection{Files and Installation}

The package consists of the files:
%
\begin{center}
\begin{tabular}{ll}
    |README.txt|   & readme file \\
    |childdoc.ins| & installation file \\
    |childdoc.dtx| & source file \\
    |childdoc.def| & definition file \\
    |cdocsamp.tex| & sample main file \\
    |cdocsch1.tex| & sample include file \\
    |cdocsch2.tex| & sample include file \\
    |cdocspt3.tex| & sample part file \\
    |cdocspt4.tex| & sample part file \\
    |cdocsdrf.tex| & sample redirection file \\
    |cdocsfn1.tex| & sample redirection file \\
    |cdocsfn2.tex| & sample redirection file \\
    |childdoc.pdf| & manual
\end{tabular}
\end{center}
%
The distribution consists of the files
|README.txt|, |childdoc.ins| and |childdoc.dtx|.
%
\begin{itemize}
\item
Run (pdf)\LaTeX{} on |childdoc.dtx|
to compile the manual |childdoc.pdf| (this file).
\item
Run \LaTeX{} on |childdoc.ins| to create the definitions file |childdoc.def|
and the sample |cdocsamp.tex| with include files
|cdocsch1.tex|, |cdocsch2.tex|, |cdocspt3.tex|, |cdocspt4.tex|,
|cdocsdrf.tex|, |cdocsfn1.tex|, |cdocsfn2.tex|.
Then copy the file |childdoc.def| to an appropriate directory of your \LaTeX{}
distribution, e.g.\ \textit{texmf-root}|/tex/latex/childdoc|.
\end{itemize}

%%%%%%%%%%%%%%%%%%%%%%%%%%%%%%%%%%%%%%%%%%%%%%%%%%%%%%%%%%%%%%%%%%%%%%%%%%%%%%%%
\subsection{Related CTAN Packages}

There are several other packages which offer a similar functionality:
%
\begin{itemize}
\item
The packages
\href{http://ctan.org/pkg/docmute}{\textsf{docmute}},
\href{http://ctan.org/pkg/includex}{\textsf{includex}} and
\href{http://ctan.org/pkg/standalone}{\textsf{standalone}}
provide commands to include only the document body of
a child file thus allowing both files to be compiled individually.
\item
The packages \href{http://ctan.org/pkg/subdocs}{\textsf{subdocs}}
and \href{http://ctan.org/pkg/subfiles}{\textsf{subfiles}}
provide structures in which the main and child documents can be
encapsulated and allowing them to be compiled individually.
The inclusion mechanism is different from the conventional |\include|.
\item
The package \href{http://ctan.org/pkg/combine}{\textsf{combine}}
is an elaborate solution to combine several documents into one.
\end{itemize}
%
See also the CTAN topic \href{http://ctan.org/topic/subdocs}{\textsf{subdocs}}
for further related packages.
The present package differs from the above solutions in that
a document structure constructed with the conventional |\include| mechanism
just needs two extra commands at the top of every file
such that all constituent files can be compiled individually.

%%%%%%%%%%%%%%%%%%%%%%%%%%%%%%%%%%%%%%%%%%%%%%%%%%%%%%%%%%%%%%%%%%%%%%%%%%%%%%%%
%\subsection{Feature Suggestions}
%
%The following is a list of features which may be useful for future
%versions of this package:
%%
%\begin{itemize}
%\item
%\ldots
%\end{itemize}

%%%%%%%%%%%%%%%%%%%%%%%%%%%%%%%%%%%%%%%%%%%%%%%%%%%%%%%%%%%%%%%%%%%%%%%%%%%%%%%%
\subsection{Revision History}

%%%%%%%%%%%%%%%%%%%%%%%%%%%%%%%%%%%%%%%%
\paragraph{v2.0:} 2018/12/30

\begin{itemize}
\item
immediate forward processing
\item
added |\childdocby| mechanism
\item
manual restructured
\end{itemize}

%%%%%%%%%%%%%%%%%%%%%%%%%%%%%%%%%%%%%%%%
\paragraph{v1.6:} 2018/01/17

\begin{itemize}
\item
application for development of include files
\item
corrections to manual
\end{itemize}

%%%%%%%%%%%%%%%%%%%%%%%%%%%%%%%%%%%%%%%%
\paragraph{v1.5:} 2017/05/21

\begin{itemize}
\item
more complete structuring introduced
\item
|\childdocof| introduced
\item
|\childdoc| renamed to |\childdocmain|
\item
|\childredirect| renamed to |\childdocforward| and |\childdocforwardprefix|
and functionality expanded
\end{itemize}

%%%%%%%%%%%%%%%%%%%%%%%%%%%%%%%%%%%%%%%%
\paragraph{v1.0:} 2017/04/27

\begin{itemize}
\item
manual and install package
\item
first version published on CTAN
\end{itemize}

%%%%%%%%%%%%%%%%%%%%%%%%%%%%%%%%%%%%%%%%
\paragraph{v0.6:} 2017/04/26

\begin{itemize}
\item
redirection mechanism added
\end{itemize}

%%%%%%%%%%%%%%%%%%%%%%%%%%%%%%%%%%%%%%%%
\paragraph{v0.5:} 2017/04/26

\begin{itemize}
\item
functionality in definition file
\end{itemize}


%%%%%%%%%%%%%%%%%%%%%%%%%%%%%%%%%%%%%%%%%%%%%%%%%%%%%%%%%%%%%%%%%%%%%%%%%%%%%%%%
%%%%%%%%%%%%%%%%%%%%%%%%%%%%%%%%%%%%%%%%%%%%%%%%%%%%%%%%%%%%%%%%%%%%%%%%%%%%%%%%
%%%%%%%%%%%%%%%%%%%%%%%%%%%%%%%%%%%%%%%%%%%%%%%%%%%%%%%%%%%%%%%%%%%%%%%%%%%%%%%%
\appendix

\settowidth\MacroIndent{\rmfamily\scriptsize 000\ }

 \DocInput{childdoc.dtx}

\end{document}
%</driver>
% \fi
%
% %%%%%%%%%%%%%%%%%%%%%%%%%%%%%%%%%%%%%%%%%%%%%%%%%%%%%%%%%%%%%%%%%%%%%%%%%%%%%%
% %%%%%%%%%%%%%%%%%%%%%%%%%%%%%%%%%%%%%%%%%%%%%%%%%%%%%%%%%%%%%%%%%%%%%%%%%%%%%%
% \section{Sample}
%\iffalse
%<*samplemain>
%\fi
%
% The following presents a sample document
% with two chapters, two parts, a title page,
% a compile flag as well as three forwarding files to set the flag.
% It consists of eight |.tex| files:
% \begin{center}
% \begin{tabular}{ll}
% |cdocsamp.tex|&main file\\
% |cdocsch1.tex|&include file for chapter 1\\
% |cdocsch2.tex|&include file for chapter 2\\
% |cdocspt3.tex|&include file for part 3\\
% |cdocspt4.tex|&include file for part 4\\
% |cdocsdrf.tex|&forwarding file for main file in draft mode\\
% |cdocsfi1.tex|&forwarding file for final version of chapter 1\\
% |cdocsfi2.tex|&forwarding file for final version of chapter 2\\
% \end{tabular}
% \end{center}
% Each of the eight files can be compiled directly by the \LaTeX{} compiler.
%
% %%%%%%%%%%%%%%%%%%%%%%%%%%%%%%%%%%%%%%
% \paragraph{Main File.}
%
% The main file is called |cdocsamp.tex|.
%
% Load the \textsf{childdoc} definitions and
% declare the filename for the main document:
%    \begin{macrocode}
\input{childdoc.def}
\childdocmain{}
%    \end{macrocode}

% Optional override for |\version| flag:
%    \begin{macrocode}
%%\ifchilddoc\else\providecommand{\version}{draft}\fi
%    \end{macrocode}

% Define the default values for the |\version| flag
% (|final| for the main file and |draft| for childs):
%    \begin{macrocode}
\ifchilddoc
\providecommand{\version}{draft}
\else
\providecommand{\version}{final}
\fi
%    \end{macrocode}

% Load the standard document class:
%    \begin{macrocode}
\documentclass[12pt]{article}
%    \end{macrocode}

% Start the document body:
%    \begin{macrocode}
\begin{document}
%    \end{macrocode}

% Declare a title page.
% Print title, part of document being processed and version flag:
%    \begin{macrocode}
\addtocounter{page}{-1}
\begin{center}
{\LARGE\bfseries{}childdoc example\par}
\vspace{1cm}
\ifchilddoc
\ifchilddocmanual part\else chapter\fi:
`\childdocname' of `\childdocjob'\par
\else
main document: `\childdocjob'\par
\fi
version: \version\par
\end{center}
\newpage
%    \end{macrocode}

% Manually include selected file,
% otherwise process as usual:
%    \begin{macrocode}
\ifchilddocmanual
\section*{part `\childdocname'}
\input{\childdocname}
\else
%    \end{macrocode}

% Include the two chapters:
%    \begin{macrocode}
\include{cdocsch1}
\include{cdocsch2}
%    \end{macrocode}

% Include the two parts unless only chapters should be displayed:
%    \begin{macrocode}
\ifchilddoc\else
\section{part three}
\input{cdocspt3}
\section{part four}
\input{cdocspt4}
\fi
%    \end{macrocode}

% Process as usual until here:
%    \begin{macrocode}
\fi
%    \end{macrocode}

% End of document body:
%    \begin{macrocode}
\end{document}
%    \end{macrocode}
%\iffalse
%</samplemain>
%\fi
%
% %%%%%%%%%%%%%%%%%%%%%%%%%%%%%%%%%%%%%%
% \paragraph{Chapter Include Files.}
%
% The include files are called |cdocsch1.tex| and |cdocsch2.tex|.
%
%\iffalse
%<*samplechap1|samplechap2>
%\fi

% Optional override for |\version| flag:
%    \begin{macrocode}
%%\providecommand{\version}{final}
%    \end{macrocode}

% Include the main document:
%    \begin{macrocode}
\input{childdoc.def}
\childdocof{cdocsamp}
%    \end{macrocode}

%\iffalse
%</samplechap1|samplechap2>
%\fi
%
%\iffalse
%<*samplechap1>
%\fi
% Some text for chapter 1:
%    \begin{macrocode}
\section{one}
some text in chapter one
%    \end{macrocode}

%\iffalse
%</samplechap1>
%\fi
% Some text for chapter 2:
%\iffalse
%<*samplechap2>
%\fi
%    \begin{macrocode}
\section{two}
more text in chapter two
%    \end{macrocode}

%\iffalse
%</samplechap2>
%\fi
%
% %%%%%%%%%%%%%%%%%%%%%%%%%%%%%%%%%%%%%%
% \paragraph{Part Include Files.}
%
% The include files are called |cdocspt3.tex| and |cdocspt4.tex|.
%
%\iffalse
%<*samplepart3|samplepart4>
%\fi

% Optional override for |\version| flag:
%    \begin{macrocode}
%%\providecommand{\version}{final}
%    \end{macrocode}

% Include the main document:
%    \begin{macrocode}
\input{childdoc.def}
\childdocby{cdocsamp}
%    \end{macrocode}

%\iffalse
%</samplepart3|samplepart4>
%\fi
%
%\iffalse
%<*samplepart3>
%\fi
% Some text for part 3:
%    \begin{macrocode}
some text in part three
%    \end{macrocode}

%\iffalse
%</samplepart3>
%\fi
% Some text for part 4:
%\iffalse
%<*samplepart4>
%\fi
%    \begin{macrocode}
more text in part four
%    \end{macrocode}

%\iffalse
%</samplepart4>
%\fi
%
% %%%%%%%%%%%%%%%%%%%%%%%%%%%%%%%%%%%%%%
% \paragraph{Forwarding for a Complete Draft.}
%
% The following forwarding file |cdocsdrf.tex|
% compiles the main document in draft mode:
%\iffalse
%<*sampledraft>
%\fi
%    \begin{macrocode}
\def\version{draft}
\input{childdoc.def}
\childdocforward{cdocsamp}
%    \end{macrocode}

%\iffalse
%</sampledraft>
%\fi
%
% %%%%%%%%%%%%%%%%%%%%%%%%%%%%%%%%%%%%%%
% \paragraph{Forwarding for Final Version of the Chapters.}
%
% The following forwarding files |cdocsfn1.tex| and |cdocsfn2.tex|
% (with identical content)
% compile the final versions of the child documents
% |cdocsch1.tex| and |cdocsch2.tex|, respectively:
%\iffalse
%<*samplefinal>
%\fi
%    \begin{macrocode}
\def\version{final}
\input{childdoc.def}
\childdocforwardprefix[cdocsamp]{cdocsfn}{cdocsch}
%    \end{macrocode}

%\iffalse
%</samplefinal>
%\fi
%
% %%%%%%%%%%%%%%%%%%%%%%%%%%%%%%%%%%%%%%
% \paragraph{Command Line Processing.}
%
% The following three command lines generate the output files
% |cdocscld|, |cdocscl1| and |cdocscl2|
% which should be identical to
% |cdocsdrf|, |cdocsch1| and |cdocsfn2|, respectively:
% \begin{center}
% \begin{tabular}{l}
% |latex -jobname cdocscld \|\\
% |  "\def\version{draft}\input{childdoc.def}\childdocforward{cdocsamp}"|\\
% |latex -jobname cdocscl1 \|\\
% |  "\input{childdoc.def}\childdocforward[cdocsamp]{cdocsch1}"|\\
% |latex -jobname cdocscl2 \|\\
% |  "\def\version{final}\input{childdoc.def}\childdocforward{cdocsch2}"|
% \end{tabular}
% \end{center}
% Note that the trailing backslash on each first line
% merely continues the input to the second line
% (for convenient cut ant paste).
% Furthermore, the command |latex| can be replaced by any
% of its alternative versions such as |pdflatex|.
%
% %%%%%%%%%%%%%%%%%%%%%%%%%%%%%%%%%%%%%%%%%%%%%%%%%%%%%%%%%%%%%%%%%%%%%%%%%%%%%%
% %%%%%%%%%%%%%%%%%%%%%%%%%%%%%%%%%%%%%%%%%%%%%%%%%%%%%%%%%%%%%%%%%%%%%%%%%%%%%%
% \section{Implementation}
%\iffalse
%<*package>
%\fi
%
% This section describes the definitions file |childdoc.def|.

% The definitions cannot be loaded using |\usepackage| or |\RequirePackage|
% which has a mechanism to prevent loading a style file more than once.
% When loading the definitions by means of |\input|
% multiple instances have to be prevented manually:
%\iffalse
%This code needs to be before the `\ProvidesFile' directive
%which is defined at the beginning of this file.
%Therefore it is also placed there and commented out here.
%</package>
%<*discard>
%\fi
%    \begin{macrocode}
\ifdefined\childdocmain\endinput\fi
%    \end{macrocode}
%\iffalse
%</discard>
%<*package>
%\fi
%
% \macro{\ifchilddoc}
% \macro{\ifchilddocmanual}
% The conditional |\ifchilddoc| tells whether a
% child (true) or main (false) document is being compiled.
% The conditional |\ifchilddocmanual| tells whether
% the |\includeonly| mechanism is used (false) or
% the selection of child files must be performed manually (true).
% The definitions initialise to false:
%    \begin{macrocode}
\newif\ifchilddoc
\newif\ifchilddocmanual
%    \end{macrocode}

% \macro{\childdocname}
% \macro{\childdocjob}
% The macro |\childdocname| stores the name of the main document
% to be compiled. The macro |\childdocjob| stores the name of
% the document on which the \LaTeX{} compiler was originally invoked.
% The content of |\jobname| cannot be compared
% to filenames specified in the source due to different catcodes.
% The following code rescans |\jobname|, stores the result
% in |\childdocname| and saves a copy in |\childdocjob|:
%    \begin{macrocode}
\edef\childdocname{\scantokens\expandafter{\jobname\noexpand}}
\let\childdocjob\childdocname
%    \end{macrocode}

% \macro{\childdocdisable}
% The macro |\childdocdisable| prevents the main file
% from being processed more than once.
% At this stage, the main document command |\childdocmain|
% is assumed to be called once again where it should do nothing.
% Any subsequent call to it should prevent
% a secondary processing of the main document
% It overwrites the forwarding commands
% |\childdocof| and |\childdocforward|
% with empty macros to prevent further inclusions of the main document:
%    \begin{macrocode}
\newcommand{\childdocdisable}
{
  \renewcommand{\childdocmain}[1]{\renewcommand{\childdocmain}[1]{\endinput}}
  \renewcommand{\childdocof}[1]{}
  \renewcommand{\childdocby}[2][]{}
  \renewcommand{\childdocforward}[2][]{}
  \renewcommand{\childdocdisable}{}
}
%    \end{macrocode}

% \macro{\childdocmain}
% The macro |\childdocmain| is to be called at the top of the main file
% with nothing or the main filename (without extension) as argument.
% First, it breaks loops.
% If the argument is not empty and does not match |\childdocname|
% (which is set by the first inclusion of |childdoc.def|),
% |\ifchilddoc| is set to true, |\includeonly| is applied to the child file
% and |\jobname| is set to the main file
% (for proper handling of |.aux| files):
%    \begin{macrocode}
\newcommand{\childdocmain}[1]
{
  \childdocdisable\childdocmain{}
  \if?#1?\else
    \begingroup
      \def\childdoctmp{#1}
      \ifx\childdoctmp\childdocname
        \def\childdoctmp{}
      \else
        \def\childdoctmp
        {
          \childdoctrue
          \includeonly{\childdocname}
          \def\childdocjob{#1}
          \def\jobname{#1}
        }
      \fi
      \expandafter
    \endgroup
    \childdoctmp
  \fi
}
%    \end{macrocode}

% \macro{\childdocof}
% The command |\childdocof| redirects
% compilation to the main file |#1|.
%    \begin{macrocode}
\newcommand{\childdocof}[1]
{
  \childdocdisable
  \childdoctrue
  \includeonly{\childdocname}
  \def\jobname{#1}
  \def\childdocjob{#1}
  \input{#1}
}
%    \end{macrocode}

% \macro{\childdocby}
% The command |\childdocby| ....
%    \begin{macrocode}
\newcommand{\childdocby}[2][]
{
  \childdocdisable
  \childdoctrue
  \childdocmanualtrue
  \if?#1?\else
    \def\jobname{#2}
  \fi
  \def\childdocjob{#2}
  \input{#2}
  \endinput
}
%    \end{macrocode}

% \macro{\childdocforward}
% The command |\childdocforward| redirects
% compilation to the main file or
% (if the optional argument is given) a child file.
% Parameters are set as if the main file
% or a child file starting with |\childdocof| was compiled.
% Then compilation is handed over to the main file:
%    \begin{macrocode}
\newcommand{\childdocforward}[2][]
{
  \begingroup
    \if?#1?
      \def\childdoctmp
      {
        \def\childdocname{#2}
        \def\childdocjob{#2}
        \def\jobname{#2}
        \input{#2}
        \endinput
      }
    \else
      \def\childdoctmp
      {
        \childdocdisable
        \def\childdocname{#2}
        \childdoctrue
        \includeonly{#2}
        \def\childdocjob{#1}
        \def\jobname{#1}
        \input{#1}
        \endinput
      }
    \fi
    \expandafter
  \endgroup
  \childdoctmp
}
%    \end{macrocode}

% \macro{\childdocforwardprefix}
% The command |\childdocforwardprefix| redirects
% compilation to the main or a child file by means of a pattern.
% The prefix |#1| in the current filename is replaced by |#2|
% and the suffix of the current filename is kept
% (it is assumed that the filename does not contain the substring `|~~~|'
% which is used as a delimiter).
% Compilation is handed over to the new file by |\childdocforward|:
%    \begin{macrocode}
\newcommand{\childdocforwardprefix}[3][]
{
  \begingroup
    \def\childdocextract #2##1~~~{\def\childdoctmp{\childdocforward[#1]{#3##1}}}
    \expandafter\childdocextract\childdocname~~~
    \expandafter
  \endgroup
  \childdoctmp
}
%    \end{macrocode}

% \macro{\childdoc}
% The deprecated macro |\childdoc| is a legacy version of |\childdocmain|:
%    \begin{macrocode}
\newcommand{\childdoc}{\childdocmain}
%    \end{macrocode}

% \macro{\childdocredirect}
% The deprecated macro |\childdocredirect| is a legacy version
% of |\childdocforward| and |\childdocforwardprefix|:
%    \begin{macrocode}
\newcommand{\childdocredirect}[2][]
{
  \begingroup
    \if?#1?
      \def\childdoctmp{\childdocforward{#2}}
    \else
      \def\childdoctmp{\childdocforwardprefix{#1}{#2}}
    \fi
    \expandafter
  \endgroup
  \childdoctmp
}
%    \end{macrocode}

%\iffalse
%</package>
%\fi
%
\endinput

\childdocforward{cdocsamp}
%    \end{macrocode}

%\iffalse
%</sampledraft>
%\fi
%
% %%%%%%%%%%%%%%%%%%%%%%%%%%%%%%%%%%%%%%
% \paragraph{Forwarding for Final Version of the Chapters.}
%
% The following forwarding files |cdocsfn1.tex| and |cdocsfn2.tex|
% (with identical content)
% compile the final versions of the child documents
% |cdocsch1.tex| and |cdocsch2.tex|, respectively:
%\iffalse
%<*samplefinal>
%\fi
%    \begin{macrocode}
\def\version{final}
% \iffalse
%
% childdoc.dtx Copyright (C) 2017-2018 Niklas Beisert
%
% This work may be distributed and/or modified under the
% conditions of the LaTeX Project Public License, either version 1.3
% of this license or (at your option) any later version.
% The latest version of this license is in
%   http://www.latex-project.org/lppl.txt
% and version 1.3 or later is part of all distributions of LaTeX
% version 2005/12/01 or later.
%
% This work has the LPPL maintenance status `maintained'.
%
% The Current Maintainer of this work is Niklas Beisert.
%
% This work consists of the files childdoc.dtx and childdoc.ins
% and the derived files childdoc.def and cdocsamp.tex with
% cdocsch1.tex, cdocsch2.tex, cdocsdrf.tex, cdocsfn1.tex, cdocsfn2.tex.
%
%<package>\ifdefined\childdocmain\endinput\fi
%<package>\ProvidesFile{childdoc.def}[2018/12/30 v2.0 child document driver]
%<samplemain>\ProvidesFile{cdocsamp.tex}[2018/12/30 v2.0 sample for childdoc]
%<*driver>
%\ProvidesFile{childdoc.drv}[2018/12/30 v2.0 childdoc reference manual file]
\PassOptionsToClass{10pt,a4paper}{article}
\documentclass{ltxdoc}

\usepackage[margin=35mm]{geometry}
\usepackage{hyperref}
\usepackage{hyperxmp}
\usepackage[usenames]{color}

\hypersetup{colorlinks=true}
\hypersetup{pdfstartview=FitH}
\hypersetup{pdfpagemode=UseNone}
\hypersetup{pdfsource={}}
\hypersetup{pdflang={en-UK}}
\hypersetup{pdfcopyright={Copyright 2017-2018 Niklas Beisert.
  This work may be distributed and/or modified under the
  conditions of the LaTeX Project Public License, either version 1.3
  of this license or (at your option) any later version.}}
\hypersetup{pdflicenseurl={http://www.latex-project.org/lppl.txt}}
\hypersetup{pdfcontactaddress={ETH Zurich, ITP, HIT K,
  Wolfgang-Pauli-Strasse 27}}
\hypersetup{pdfcontactpostcode={8093}}
\hypersetup{pdfcontactcity={Zurich}}
\hypersetup{pdfcontactcountry={Switzerland}}
\hypersetup{pdfcontactemail={nbeisert@itp.phys.ethz.ch}}
\hypersetup{pdfcontacturl={http://people.phys.ethz.ch/\xmptilde nbeisert/}}

\newcommand{\secref}[1]{\hyperref[#1]{section \ref*{#1}}}

\parskip1ex
\parindent0pt
\let\olditemize\itemize
\def\itemize{\olditemize\parskip0pt}

\begin{document}

\title{The \textsf{childdoc} Package}
\hypersetup{pdftitle={The childdoc Package}}
\author{Niklas Beisert\\[2ex]
  Institut f\"ur Theoretische Physik\\
  Eidgen\"ossische Technische Hochschule Z\"urich\\
  Wolfgang-Pauli-Strasse 27, 8093 Z\"urich, Switzerland\\[1ex]
  \href{mailto:nbeisert@itp.phys.ethz.ch}
  {\texttt{nbeisert@itp.phys.ethz.ch}}}
\hypersetup{pdfauthor={Niklas Beisert}}
\hypersetup{pdfsubject={Manual for the LaTeX2e Package childdoc}}
\date{30 December 2018, \textsf{v2.0}}
\maketitle

\begin{abstract}\noindent
\textsf{childdoc} is a \LaTeXe{} package
that enables the direct compilation
of document sections included by |\include|
to individual files.
\end{abstract}

\begingroup
\parskip0ex
\tableofcontents
\endgroup

%%%%%%%%%%%%%%%%%%%%%%%%%%%%%%%%%%%%%%%%%%%%%%%%%%%%%%%%%%%%%%%%%%%%%%%%%%%%%%%%
%%%%%%%%%%%%%%%%%%%%%%%%%%%%%%%%%%%%%%%%%%%%%%%%%%%%%%%%%%%%%%%%%%%%%%%%%%%%%%%%
\section{Introduction}

\LaTeX{} provides a mechanism to structure a large document (such as a book)
into a main file and several child files (containing the chapters)
using the |\include| command.
This mechanism is beneficial for documents
which span hundreds of pages in order to
make the source file(s) more manageable.
Moreover, compilation can be restricted to
selected child files by means of the |\includeonly| command.
The latter feature can be used to reduce the compilation time while editing
(this was significantly more useful in the earlier days of \LaTeX{})
or to generate a smaller document which is easier to navigate.
Another application of |\includeonly| is to generate
documents consisting of selected parts of the complete document.

However, there are a few drawbacks of the plain |\include| mechanism:
\begin{itemize}
\item
The child files cannot be compiled on their own,
they can only be compiled via the main file.
A naive editing environment
(such as a text editor with an option
to have the current file processed by \LaTeX)
may require one to switch to the main file before compiling;
attempting to compile the child file produces errors.
\item
The main file must be modified (each time)
to adjust the |\includeonly| command
to the present needs. This easily leaves the main file in a messy state.
\item
The generated document will always carry the filename
of the main document. This is inconvenient if
several child files are to be compiled and
to be kept for distribution.
\end{itemize}

The present package provides a simple interface
to make child files individually compilable by \LaTeX{}.
Compiling a child file then has the same effect as compiling
the main file with an |\includeonly| command
to select the appropriate child.
Moreover the generated document will carry the name of the child
rather than the main file.
This resolves all three above issues.

This feature is meant to make the editing of books,
thesis documents and lecture notes somewhat more convenient.
However, the package can also be used efficiently for
composing a series of documents (such as exercise sheets)
which are typically distributed individually.
It then assists the author in generating the individual documents
(potentially in different versions)
as well as a document containing the collected series.
Another application is in developing style files
or other kinds of included material
where compilation of the style file could redirect
to a sample or test file.

%%%%%%%%%%%%%%%%%%%%%%%%%%%%%%%%%%%%%%%%%%%%%%%%%%%%%%%%%%%%%%%%%%%%%%%%%%%%%%%%
%%%%%%%%%%%%%%%%%%%%%%%%%%%%%%%%%%%%%%%%%%%%%%%%%%%%%%%%%%%%%%%%%%%%%%%%%%%%%%%%
\section{Usage}

First of all, the package \textsf{childdoc} is \emph{not} a standard
\LaTeXe{} |.sty| style file! Therefore it needs to be invoked in
a non-standard way.

%%%%%%%%%%%%%%%%%%%%%%%%%%%%%%%%%%%%%%%%%%%%%%%%%%%%%%%%%%%%%%%%%%%%%%%%%%%%%%%%
\subsection{Included Files}
\label{sec:include}

%%%%%%%%%%%%%%%%%%%%%%%%%%%%%%%%%%%%%%%%
\DescribeMacro{\childdocmain}
To use the package, add the commands
\begin{center}
\begin{tabular}{l}
|\input{childdoc.def}|\\
|\childdocmain{}|\\
\end{tabular}
\end{center}
at the very top of the main \LaTeX{} file,
in particular \emph{before} the |\documentclass| statement!
The argument of |\childdocmain| should be left empty
(but it must be present).

%%%%%%%%%%%%%%%%%%%%%%%%%%%%%%%%%%%%%%%%
\DescribeMacro{\childdocof}
Furthermore, add the commands
\begin{center}
\begin{tabular}{l}
|\input{childdoc.def}|\\
|\childdocof{|\textit{main}|}|\\
\end{tabular}
\end{center}
at the top of every child file \textit{child}
which is included by |\include{|\textit{child}|}|
from within the main file
(or at least for those files to be compiled individually).
The argument \textit{main} must be the filename of the main file.

There are a couple of
considerations in setting up the main and child documents:

%%%%%%%%%%%%%%%%%%%%%%%%%%%%%%%%%%%%%%%%
\paragraph{Restrictions.}

Please note the following restrictions:
\begin{itemize}
\item
|\childdocmain| must be called with one argument \textit{main}
to ensure compatibility with earlier version of the package.
It must either be empty (|\childdocmain{}|)
or precisely match the filename of the main file in which it is specified.
See \secref{sec:detection} for further information.
\item
The filename \textit{main} must be specified without the |.tex| extension.
\item
The filename \textit{main} is case sensitive
(even in case-insensitive file systems)
due to internal string comparison.
\item
The argument \textit{main} should be fully expanded, it cannot be a macro.
\item
Subdirectories and special characters should be avoided in filenames.
\item
The command |\childdocmain{|\textit{main}|}| must be followed by a whitespace.
It should not be followed immediately by another command
or by a comment mark `|%|'.
This is because the \TeX{} parser reads the token immediately following
the argument of |\childdocmain| and puts it
at the beginning of every child section;
however, a white\-space is ignored.
\end{itemize}

%%%%%%%%%%%%%%%%%%%%%%%%%%%%%%%%%%%%%%%%
\paragraph{Content of Main File.}

It is advisable to place all content in the child files included by |\include|.
Any output contained in the main file will appear in all child documents
unless suppressed manually;
it cannot be suppressed automatically by the |\includeonly| directive
and thus should normally be avoided.
A method to include some content in the main file
by means of conditional processing is described in \secref{sec:conditional}.

%%%%%%%%%%%%%%%%%%%%%%%%%%%%%%%%%%%%%%%%
\paragraph{Page Numbering.}

When only a part of the document is compiled,
the appropriate numbering of pages
(as well as other status parameters)
is determined from the |.aux| files.
The latter contain information from previous passes.
However this information needs to propagate through
all intermediate child documents.
Therefore the page numbering in child documents may well
be inconsistent until the complete document is compiled at least once.

A useful (if unconventional) way to always ensure a consistent
page numbering is to restart the numbering in each child document
and denote the pages by `\textit{child}|.|\textit{page}'
where \textit{child} represents the chapter/section number of the child file.
This can be achieved by the command
|\numberwithin{page}{|\textit{child}|}|
of the \textsf{amsmath} package
where \textit{child} can be |chapter| or |section|
depending on the chosen structuring.
Alternatively, one can modify the macro |\thepage| appropriately
and reset the counter |page| at the start of each child file.

%%%%%%%%%%%%%%%%%%%%%%%%%%%%%%%%%%%%%%%%%%%%%%%%%%%%%%%%%%%%%%%%%%%%%%%%%%%%%%%%
\subsection{Conditional Processing}
\label{sec:conditional}

The package provides a mechanism to compile different versions
of a document. To customise the versions further some conditional processing
can come in handy to distinguish which version is being compiled.
The package provides two macros to describe the compilation context:

%%%%%%%%%%%%%%%%%%%%%%%%%%%%%%%%%%%%%%%%
\DescribeMacro{\ifchilddoc}
The conditional |\ifchilddoc| distinguishes between the compilation of
child documents and the main document:
%
\begin{center}
|\ifchilddoc |\textit{child-code}| |[|\||else |\textit{main-code}]| \||fi|
\end{center}

%%%%%%%%%%%%%%%%%%%%%%%%%%%%%%%%%%%%%%%%
\DescribeMacro{\childdocname}
\DescribeMacro{\childdocjob}
The macro |\childdocname| contains the filename (without extension)
of the main or child file being processed.
Note that |\childdocjob| will always contain the name of the main file.

%%%%%%%%%%%%%%%%%%%%%%%%%%%%%%%%%%%%%%%%
\paragraph{Title Page.}

Conditional processing can be used to include a title or banner page
in the main document when proper precautions are taken.
Importantly, the code in the main file should ensure that the page counter
(as well as other status parameters which are stored in the |.aux| files)
takes the same value after the conditional processing.
Otherwise the page numbers may take divergent values
depending on which part is compiled.

For example, a title page could be declared by:
%
\begin{center}
\begin{tabular}{l}
|\ifchilddoc\||else|\\
|\addtocounter{page}{-1}|\\
\textit{code for title page}\\
|\newpage|\\
|\||fi|
\end{tabular}
\end{center}
%
A banner page for the child documents can be generated by:
%
\begin{center}
\begin{tabular}{l}
|\ifchilddoc|\\
|\addtocounter{page}{-1}|\\
\textit{code for banner page}\\
|\newpage|\\
|\||fi|
\end{tabular}
\end{center}
%
Here one could write a message such as:
\begin{center}
|This is the part \childdocname{} of \childdocjob{}.|
\end{center}

%%%%%%%%%%%%%%%%%%%%%%%%%%%%%%%%%%%%%%%%%%%%%%%%%%%%%%%%%%%%%%%%%%%%%%%%%%%%%%%%
\subsection{Flags}
\label{sec:flags}

The package makes it easy to generate different versions
of the main or child documents.
To this end compilation flags can be defined
and assigned different default values.
They will be particularly useful in conjunction
with the forwarding mechanism described in \secref{sec:forward}.

For example, it may be useful to have a flag |\version|
which can be set to |draft| or |final|.
The document source will contain some conditional code
depending on the value of |\version|.
Suppose further, the flag should default to |final| for the main file
and to |draft| for child files
which is a natural assignment for editing the document.
This is achieved by placing the following code
in the preamble of the main document
(below the |\childdocmain| directive):
%
\begin{center}
\begin{tabular}{l}
|\ifchilddoc|\\
|\providecommand{\version}{draft}|\\
|\||else|\\
|\providecommand{\version}{final}|\\
|\||fi|
\end{tabular}
\end{center}
%
The definition by |\providecommand| makes sure
that previous definitions are not overwritten.
Further statements |\providecommand{\version}{...}|
can thus be added before the above code to override it.

For the main file, one might add a line
(between |\childdocmain| and the above block)
%
\begin{center}
|%\ifchilddoc\||else\providecommand{\version}{draft}\||fi|
\end{center}
%
which can be uncommented to produce a draft version.
Likewise one can add a line to the very top of a child file
(above the |\childdocof{|\textit{main}|}| directive)
%
\begin{center}
|%\providecommand{\version}{final}|
\end{center}
%
which can be uncommented to produce the final version of this child document.

%%%%%%%%%%%%%%%%%%%%%%%%%%%%%%%%%%%%%%%%%%%%%%%%%%%%%%%%%%%%%%%%%%%%%%%%%%%%%%%%
\subsection{Forwarding}
\label{sec:forward}

Different versions of the main or child documents
using compilation flags as described in \secref{sec:flags}
can be (permanently) stored in different files
for convenient compilation, viewing and distribution.
To this end, the package defines a command
to pass on compilation to a different file:

%%%%%%%%%%%%%%%%%%%%%%%%%%%%%%%%%%%%%%%%
\DescribeMacro{\childdocforward}
The command |\childdocforward| redirects processing to
another source file:
%
\begin{center}
\begin{tabular}{l}
|\input{childdoc.def}|\\
|\childdocforward[|\textit{main}|]{|\textit{dest}|}|\\
\end{tabular}
\end{center}
%
The argument \textit{dest} is the destination file
(without extension).
It should be the main file or one of the child files.
Note that further \textsf{childdoc} directives
such as |\childdocof| and |\childdocforward|
in the indicated file will be processed in this form.
The optional argument \textit{main}
passes on directly to the main file \textit{main}
while pretending to compile the child \textit{dest}.
This form behaves as if \textit{dest}
issues |\childdocof{|\textit{main}|}| right away,
and no further \textsf{childdoc} directives will be processed.

%%%%%%%%%%%%%%%%%%%%%%%%%%%%%%%%%%%%%%%%
\DescribeMacro{\...prefix}
In the alternative form |\childdocforwardprefix|,
%
\begin{center}
\begin{tabular}{l}
|\input{childdoc.def}|\\
|\childdocforwardprefix[|\textit{main}|]{|\textit{prefix}|}{|\textit{dest}|}|
\end{tabular}
\end{center}
%
the destination file is determined by a pattern
depending on the current file:
To make this work, the current file must be called
`{\textit{prefix}\hspace{0.2em}\textit{suffix}}'
with \textit{prefix} matching precisely the argument.
Processing is then passed on to the file
`{\textit{dest}\hspace{0.2em}\textit{suffix}}'.
Surely, the same effect is achieved by
directly specifying the
argument `{\textit{dest}\hspace{0.2em}\textit{suffix}}'
in the first form.
However, that requires to set up a different file
for each child. With the alternative form of the command
all these files can have exactly the same content
which simplifies setting them up and maintaining them.

For example, the following file |draft.tex|
with a compilation flag |\version| as described in \secref{sec:flags}
compiles the main document as a draft:
%
\begin{center}
\begin{tabular}{l}
|\def\version{draft}|\\
|\input{childdoc.def}|\\
|\childdocforward{|\textit{main}|}|
\end{tabular}
\end{center}
%
Likewise, the following files |final|\textit{nn}|.tex|
compile the final version of the child document
|child|\textit{nn}|.tex|:
%
\begin{center}
\begin{tabular}{l}
|\def\version{final}|\\
|\input{childdoc.def}|\\
|\childdocforwardprefix{final}{child}|
\end{tabular}
\end{center}
%

Note that when several versions of a main file and/or of each child file
are to be generated, it may be convenient to set up a |Makefile| or
shell script to automatise the process.

%%%%%%%%%%%%%%%%%%%%%%%%%%%%%%%%%%%%%%%%%%%%%%%%%%%%%%%%%%%%%%%%%%%%%%%%%%%%%%%%
\subsection{Command Line Processing}
\label{sec:commandline}

The effect of redirection files can also be achieved by invoking
the \LaTeX{} compiler with a more elaborate command line.
Most conveniently this should be done as part
of a shell script or a |Makefile|.

When using \textsf{childdoc} in the main file, the following
command lines effectively perform a redirection
(note that depending on the shell being used,
backslashes may have to be doubled: `|\|' $\to$ `|\\|'):
%
\begin{center}
|... -jobname "|\textit{target}|" |\\|"|[\textit{flags}]%
|\input{childdoc.def}\childdocforward[|\textit{main}|]{|\textit{dest}|}"|
\end{center}
%
Here \textit{target} is the name of the output file,
\textit{main} is the name of the main file
and \textit{dest} is the name of the main or child file to be processed
(all filenames without extensions).
The optional argument \textit{main} can be omitted
if \textit{main} matches \textit{dest}.
Optionally, compilation \textit{flags} can be defined via |\def| commands.
This command line makes the \TeX{} engine believe
it is compiling the file \textit{target}
whose content is specified as the latter parameter.
The provided code then forwards the processing to
\textit{main} or \textit{dest} as described in \secref{sec:forward}.

%%%%%%%%%%%%%%%%%%%%%%%%%%%%%%%%%%%%%%%%%%%%%%%%%%%%%%%%%%%%%%%%%%%%%%%%%%%%%%%%
\subsection{Include by Input}
\label{sec:input}

Including child documents by |\include| has some restrictions by design.
Most notably, the content of a child document always occupies
its own set of pages; pages cannot be shared between child documents.
Usually, this behaviour makes perfect sense
because each child document contain an essential part of the document.
However, in some situations it may be desirable to compose
a document from a collection of parts
without having mandatory page breaks between then.
For this case, the package
provides a mechanism to include parts
by |\input| which can also be processed individually.
However, by construction this mechanism
requires manual handling of the content to be output.

%%%%%%%%%%%%%%%%%%%%%%%%%%%%%%%%%%%%%%%%
\DescribeMacro{\ifchilddocmanual}
The main file should be prepared as usual, see \secref{sec:include}.
However, the document body must make a distinction
between processing of an individual part and of the main document, e.g.:
%
\begin{center}
\begin{tabular}{l}
|\ifchilddocmanual|\\
|\input{\childdocname}|\\
|\||else|\\
\textit{document body with }|\input{|\textit{part}|}|\\
|\||fi|
\end{tabular}
\end{center}
%
The conditional |\ifchilddocmanual| is true whenever
a part to be included by |\input| is being compiled,
and the name of the part is stored in |\childdocname|.

%%%%%%%%%%%%%%%%%%%%%%%%%%%%%%%%%%%%%%%%
\DescribeMacro{\childdocby}
Each part to be included by |\input| should start with:
%
\begin{center}
\begin{tabular}{l}
|\input{childdoc.def}|\\
|\childdocby{|\textit{main}|}|\\
\end{tabular}
\end{center}
%
The directive |\childdocby| is similar to |\childdocof|
described in \secref{sec:include},
but the subsequent selection of content must be done manually.
To that end, both |\ifchilddoc| and |\ifchilddocmanual|
will be true upon processing of a part,
and the name of the part is stored in |\childdocname|.
Note that |\jobname| will be set to the filename of the current part
so that each part receives an individual |.aux| file
that does not interfere with the |.aux| file(s) of the main document.
This behaviour can be altered by the alternative form
|\childdocby[*]{|\textit{main}|}| (with a non-empty optional argument)
which uses the |.aux| file of the main document
by setting |\jobname| to \textit{main}.

%%%%%%%%%%%%%%%%%%%%%%%%%%%%%%%%%%%%%%%%%%%%%%%%%%%%%%%%%%%%%%%%%%%%%%%%%%%%%%%%
\subsection{Driver Development}
\label{sec:driver}

The \textsf{childdoc} mechanism can also be use for the development
of definition files such as \LaTeX{} styles or classes.
This case differs from the above setup with multiple parts
included by |\include| in that no |\includeonly| should be invoked.
This can be achieved by starting the include file
(before |\ProvidesPackage|) with:
%
\begin{center}
\begin{tabular}{l}
|\input{childdoc.def}|\\
|\childdocforward{|\textit{main}|}|\\
\end{tabular}
\end{center}
%
or alternatively with:
%
\begin{center}
\begin{tabular}{l}
|\input{childdoc.def}|\\
|\childdocby{|\textit{main}|}|\\
\end{tabular}
\end{center}
%
Both forms have slightly different effects as described above.
The main file is prepared as usual, see \secref{sec:include}.

%%%%%%%%%%%%%%%%%%%%%%%%%%%%%%%%%%%%%%%%%%%%%%%%%%%%%%%%%%%%%%%%%%%%%%%%%%%%%%%%
\subsection{Legacy Detection}
\label{sec:detection}

The directive |\childdocmain| in the main file can detect
whether the complete document or merely a child is to be compiled
even without using the directive |\childdocof|.
This method is deprecated because it is less robust
and there is no compelling reason to use it;
it is merely provided for backward compatibility
and it may be removed in future versions.

If the detection mechanism is to be used,
it is mandatory to correctly specify
the filename of the main file as the argument of |\childdocmain|:
%
\begin{center}
\begin{tabular}{l}
|\input{childdoc.def}|\\
|\childdocmain{|\textit{main}|}|\\
\end{tabular}
\end{center}
%
If |\jobname| does not match the argument \textit{main} of |\childdocmain|,
it is assumed that |\jobname| points to the child file to be compiled.
When using |\childdocmain| with the main file specified as argument,
it suffices to start a child file
with just |\input{|\textit{main}|}|
without loading of the package and using |\childdocof|.
If instead all processing is done
with the appropriate \textsf{childdoc} directives,
the argument of \textit{main} of |\childdocmain| can be empty.

An alternative version of the command line processing described
in \secref{sec:commandline} using the detection mechanism reads:
%
\begin{center}
|... -jobname "|\textit{target}|" "|[\textit{flags}]%
[|\def\jobname{|\textit{dest}|}|]|\input{|\textit{main}|}"|
\end{center}

%%%%%%%%%%%%%%%%%%%%%%%%%%%%%%%%%%%%%%%%%%%%%%%%%%%%%%%%%%%%%%%%%%%%%%%%%%%%%%%%
\subsection{Manual Code}
\label{sec:manual}

In case one cannot be certain whether the definitions file |childdoc.def|
is installed on the target \TeX{} distribution
and one prefers not to ship it,
it is conceivable to paste a few relevant commands into the sources.

To that end, drop all statements |\input{childdoc.def}|
and perform the replacements as outlined below.
Instead of |\childdocmain{|\textit{main}|}| add the following code
to the top of the main file:
%
\begin{center}
\begin{tabular}{l}
|\||ifdefined\childdocname\endinput\||fi\newif\ifchilddoc|\\
|\edef\childdocname{\scantokens\expandafter{\jobname\noexpand}}|\\
|\def\childdocmain{|\textit{main}|}\||ifx\childdocmain\childdocname\||else|\\
|\childdoctrue\includeonly{\childdocname}\let\jobname\childdocmain\||fi|\\
\end{tabular}
\end{center}
%
Instead of |\childdocof{|\textit{main}|}| just include the main file
at the top of each child file:
%
\begin{center}
|\input{|\textit{main}|}|
\end{center}
%
A simple redirection |\childdocforward{|\textit{dest}|}| is achieved by:
%
\begin{center}
|\def\jobname{|\textit{dest}|}\input{\jobname}|
\end{center}
%
The redirection with prefix
|\childdocforwardprefix[|\textit{prefix}|]{|\textit{dest}|}|
is accomplished by:
%
\begin{center}
\begin{tabular}{l}
|{\edef\jobname{\scantokens\expandafter{\jobname\noexpand}}|\\
|\def\redirectjob |\textit{prefix}|#1~~~{\gdef\jobname{|\textit{dest}|#1}}|\\
|\expandafter\redirectjob\jobname~~~}\input{\jobname}|
\end{tabular}
\end{center}

In an alternative approach,
child documents can be compiled by a specific command line
without additional code or specific definitions:
%
\begin{center}
|... -jobname "|\textit{target}|" "|[\textit{flags}]%
|\includeonly{|\textit{dest}|}\input{|\textit{main}|}"|
\end{center}
%

%%%%%%%%%%%%%%%%%%%%%%%%%%%%%%%%%%%%%%%%%%%%%%%%%%%%%%%%%%%%%%%%%%%%%%%%%%%%%%%%
%%%%%%%%%%%%%%%%%%%%%%%%%%%%%%%%%%%%%%%%%%%%%%%%%%%%%%%%%%%%%%%%%%%%%%%%%%%%%%%%
\section{Information}

%%%%%%%%%%%%%%%%%%%%%%%%%%%%%%%%%%%%%%%%%%%%%%%%%%%%%%%%%%%%%%%%%%%%%%%%%%%%%%%%
\subsection{Copyright}

Copyright \copyright{} 2017--2018 Niklas Beisert

This work may be distributed and/or modified under the
conditions of the \LaTeX{} Project Public License, either version 1.3
of this license or (at your option) any later version.
The latest version of this license is in
  \url{http://www.latex-project.org/lppl.txt}
and version 1.3 or later is part of all distributions of \LaTeX{}
version 2005/12/01 or later.

This work has the LPPL maintenance status `maintained'.

The Current Maintainer of this work is Niklas Beisert.

This work consists of the files |README.txt|, |childdoc.ins| and |childdoc.dtx|
as well as the derived files |childdoc.def|, |cdocsamp.tex|
with |cdocsch1.tex|, |cdocsch2.tex|, |cdocspt3.tex|, |cdocspt4.tex|,
|cdocsdrf.tex|, |cdocsfn1.tex|, |cdocsfn2.tex|
as well as |childdoc.pdf|.

%%%%%%%%%%%%%%%%%%%%%%%%%%%%%%%%%%%%%%%%%%%%%%%%%%%%%%%%%%%%%%%%%%%%%%%%%%%%%%%%
\subsection{Files and Installation}

The package consists of the files:
%
\begin{center}
\begin{tabular}{ll}
    |README.txt|   & readme file \\
    |childdoc.ins| & installation file \\
    |childdoc.dtx| & source file \\
    |childdoc.def| & definition file \\
    |cdocsamp.tex| & sample main file \\
    |cdocsch1.tex| & sample include file \\
    |cdocsch2.tex| & sample include file \\
    |cdocspt3.tex| & sample part file \\
    |cdocspt4.tex| & sample part file \\
    |cdocsdrf.tex| & sample redirection file \\
    |cdocsfn1.tex| & sample redirection file \\
    |cdocsfn2.tex| & sample redirection file \\
    |childdoc.pdf| & manual
\end{tabular}
\end{center}
%
The distribution consists of the files
|README.txt|, |childdoc.ins| and |childdoc.dtx|.
%
\begin{itemize}
\item
Run (pdf)\LaTeX{} on |childdoc.dtx|
to compile the manual |childdoc.pdf| (this file).
\item
Run \LaTeX{} on |childdoc.ins| to create the definitions file |childdoc.def|
and the sample |cdocsamp.tex| with include files
|cdocsch1.tex|, |cdocsch2.tex|, |cdocspt3.tex|, |cdocspt4.tex|,
|cdocsdrf.tex|, |cdocsfn1.tex|, |cdocsfn2.tex|.
Then copy the file |childdoc.def| to an appropriate directory of your \LaTeX{}
distribution, e.g.\ \textit{texmf-root}|/tex/latex/childdoc|.
\end{itemize}

%%%%%%%%%%%%%%%%%%%%%%%%%%%%%%%%%%%%%%%%%%%%%%%%%%%%%%%%%%%%%%%%%%%%%%%%%%%%%%%%
\subsection{Related CTAN Packages}

There are several other packages which offer a similar functionality:
%
\begin{itemize}
\item
The packages
\href{http://ctan.org/pkg/docmute}{\textsf{docmute}},
\href{http://ctan.org/pkg/includex}{\textsf{includex}} and
\href{http://ctan.org/pkg/standalone}{\textsf{standalone}}
provide commands to include only the document body of
a child file thus allowing both files to be compiled individually.
\item
The packages \href{http://ctan.org/pkg/subdocs}{\textsf{subdocs}}
and \href{http://ctan.org/pkg/subfiles}{\textsf{subfiles}}
provide structures in which the main and child documents can be
encapsulated and allowing them to be compiled individually.
The inclusion mechanism is different from the conventional |\include|.
\item
The package \href{http://ctan.org/pkg/combine}{\textsf{combine}}
is an elaborate solution to combine several documents into one.
\end{itemize}
%
See also the CTAN topic \href{http://ctan.org/topic/subdocs}{\textsf{subdocs}}
for further related packages.
The present package differs from the above solutions in that
a document structure constructed with the conventional |\include| mechanism
just needs two extra commands at the top of every file
such that all constituent files can be compiled individually.

%%%%%%%%%%%%%%%%%%%%%%%%%%%%%%%%%%%%%%%%%%%%%%%%%%%%%%%%%%%%%%%%%%%%%%%%%%%%%%%%
%\subsection{Feature Suggestions}
%
%The following is a list of features which may be useful for future
%versions of this package:
%%
%\begin{itemize}
%\item
%\ldots
%\end{itemize}

%%%%%%%%%%%%%%%%%%%%%%%%%%%%%%%%%%%%%%%%%%%%%%%%%%%%%%%%%%%%%%%%%%%%%%%%%%%%%%%%
\subsection{Revision History}

%%%%%%%%%%%%%%%%%%%%%%%%%%%%%%%%%%%%%%%%
\paragraph{v2.0:} 2018/12/30

\begin{itemize}
\item
immediate forward processing
\item
added |\childdocby| mechanism
\item
manual restructured
\end{itemize}

%%%%%%%%%%%%%%%%%%%%%%%%%%%%%%%%%%%%%%%%
\paragraph{v1.6:} 2018/01/17

\begin{itemize}
\item
application for development of include files
\item
corrections to manual
\end{itemize}

%%%%%%%%%%%%%%%%%%%%%%%%%%%%%%%%%%%%%%%%
\paragraph{v1.5:} 2017/05/21

\begin{itemize}
\item
more complete structuring introduced
\item
|\childdocof| introduced
\item
|\childdoc| renamed to |\childdocmain|
\item
|\childredirect| renamed to |\childdocforward| and |\childdocforwardprefix|
and functionality expanded
\end{itemize}

%%%%%%%%%%%%%%%%%%%%%%%%%%%%%%%%%%%%%%%%
\paragraph{v1.0:} 2017/04/27

\begin{itemize}
\item
manual and install package
\item
first version published on CTAN
\end{itemize}

%%%%%%%%%%%%%%%%%%%%%%%%%%%%%%%%%%%%%%%%
\paragraph{v0.6:} 2017/04/26

\begin{itemize}
\item
redirection mechanism added
\end{itemize}

%%%%%%%%%%%%%%%%%%%%%%%%%%%%%%%%%%%%%%%%
\paragraph{v0.5:} 2017/04/26

\begin{itemize}
\item
functionality in definition file
\end{itemize}


%%%%%%%%%%%%%%%%%%%%%%%%%%%%%%%%%%%%%%%%%%%%%%%%%%%%%%%%%%%%%%%%%%%%%%%%%%%%%%%%
%%%%%%%%%%%%%%%%%%%%%%%%%%%%%%%%%%%%%%%%%%%%%%%%%%%%%%%%%%%%%%%%%%%%%%%%%%%%%%%%
%%%%%%%%%%%%%%%%%%%%%%%%%%%%%%%%%%%%%%%%%%%%%%%%%%%%%%%%%%%%%%%%%%%%%%%%%%%%%%%%
\appendix

\settowidth\MacroIndent{\rmfamily\scriptsize 000\ }

 \DocInput{childdoc.dtx}

\end{document}
%</driver>
% \fi
%
% %%%%%%%%%%%%%%%%%%%%%%%%%%%%%%%%%%%%%%%%%%%%%%%%%%%%%%%%%%%%%%%%%%%%%%%%%%%%%%
% %%%%%%%%%%%%%%%%%%%%%%%%%%%%%%%%%%%%%%%%%%%%%%%%%%%%%%%%%%%%%%%%%%%%%%%%%%%%%%
% \section{Sample}
%\iffalse
%<*samplemain>
%\fi
%
% The following presents a sample document
% with two chapters, two parts, a title page,
% a compile flag as well as three forwarding files to set the flag.
% It consists of eight |.tex| files:
% \begin{center}
% \begin{tabular}{ll}
% |cdocsamp.tex|&main file\\
% |cdocsch1.tex|&include file for chapter 1\\
% |cdocsch2.tex|&include file for chapter 2\\
% |cdocspt3.tex|&include file for part 3\\
% |cdocspt4.tex|&include file for part 4\\
% |cdocsdrf.tex|&forwarding file for main file in draft mode\\
% |cdocsfi1.tex|&forwarding file for final version of chapter 1\\
% |cdocsfi2.tex|&forwarding file for final version of chapter 2\\
% \end{tabular}
% \end{center}
% Each of the eight files can be compiled directly by the \LaTeX{} compiler.
%
% %%%%%%%%%%%%%%%%%%%%%%%%%%%%%%%%%%%%%%
% \paragraph{Main File.}
%
% The main file is called |cdocsamp.tex|.
%
% Load the \textsf{childdoc} definitions and
% declare the filename for the main document:
%    \begin{macrocode}
\input{childdoc.def}
\childdocmain{}
%    \end{macrocode}

% Optional override for |\version| flag:
%    \begin{macrocode}
%%\ifchilddoc\else\providecommand{\version}{draft}\fi
%    \end{macrocode}

% Define the default values for the |\version| flag
% (|final| for the main file and |draft| for childs):
%    \begin{macrocode}
\ifchilddoc
\providecommand{\version}{draft}
\else
\providecommand{\version}{final}
\fi
%    \end{macrocode}

% Load the standard document class:
%    \begin{macrocode}
\documentclass[12pt]{article}
%    \end{macrocode}

% Start the document body:
%    \begin{macrocode}
\begin{document}
%    \end{macrocode}

% Declare a title page.
% Print title, part of document being processed and version flag:
%    \begin{macrocode}
\addtocounter{page}{-1}
\begin{center}
{\LARGE\bfseries{}childdoc example\par}
\vspace{1cm}
\ifchilddoc
\ifchilddocmanual part\else chapter\fi:
`\childdocname' of `\childdocjob'\par
\else
main document: `\childdocjob'\par
\fi
version: \version\par
\end{center}
\newpage
%    \end{macrocode}

% Manually include selected file,
% otherwise process as usual:
%    \begin{macrocode}
\ifchilddocmanual
\section*{part `\childdocname'}
\input{\childdocname}
\else
%    \end{macrocode}

% Include the two chapters:
%    \begin{macrocode}
\include{cdocsch1}
\include{cdocsch2}
%    \end{macrocode}

% Include the two parts unless only chapters should be displayed:
%    \begin{macrocode}
\ifchilddoc\else
\section{part three}
\input{cdocspt3}
\section{part four}
\input{cdocspt4}
\fi
%    \end{macrocode}

% Process as usual until here:
%    \begin{macrocode}
\fi
%    \end{macrocode}

% End of document body:
%    \begin{macrocode}
\end{document}
%    \end{macrocode}
%\iffalse
%</samplemain>
%\fi
%
% %%%%%%%%%%%%%%%%%%%%%%%%%%%%%%%%%%%%%%
% \paragraph{Chapter Include Files.}
%
% The include files are called |cdocsch1.tex| and |cdocsch2.tex|.
%
%\iffalse
%<*samplechap1|samplechap2>
%\fi

% Optional override for |\version| flag:
%    \begin{macrocode}
%%\providecommand{\version}{final}
%    \end{macrocode}

% Include the main document:
%    \begin{macrocode}
\input{childdoc.def}
\childdocof{cdocsamp}
%    \end{macrocode}

%\iffalse
%</samplechap1|samplechap2>
%\fi
%
%\iffalse
%<*samplechap1>
%\fi
% Some text for chapter 1:
%    \begin{macrocode}
\section{one}
some text in chapter one
%    \end{macrocode}

%\iffalse
%</samplechap1>
%\fi
% Some text for chapter 2:
%\iffalse
%<*samplechap2>
%\fi
%    \begin{macrocode}
\section{two}
more text in chapter two
%    \end{macrocode}

%\iffalse
%</samplechap2>
%\fi
%
% %%%%%%%%%%%%%%%%%%%%%%%%%%%%%%%%%%%%%%
% \paragraph{Part Include Files.}
%
% The include files are called |cdocspt3.tex| and |cdocspt4.tex|.
%
%\iffalse
%<*samplepart3|samplepart4>
%\fi

% Optional override for |\version| flag:
%    \begin{macrocode}
%%\providecommand{\version}{final}
%    \end{macrocode}

% Include the main document:
%    \begin{macrocode}
\input{childdoc.def}
\childdocby{cdocsamp}
%    \end{macrocode}

%\iffalse
%</samplepart3|samplepart4>
%\fi
%
%\iffalse
%<*samplepart3>
%\fi
% Some text for part 3:
%    \begin{macrocode}
some text in part three
%    \end{macrocode}

%\iffalse
%</samplepart3>
%\fi
% Some text for part 4:
%\iffalse
%<*samplepart4>
%\fi
%    \begin{macrocode}
more text in part four
%    \end{macrocode}

%\iffalse
%</samplepart4>
%\fi
%
% %%%%%%%%%%%%%%%%%%%%%%%%%%%%%%%%%%%%%%
% \paragraph{Forwarding for a Complete Draft.}
%
% The following forwarding file |cdocsdrf.tex|
% compiles the main document in draft mode:
%\iffalse
%<*sampledraft>
%\fi
%    \begin{macrocode}
\def\version{draft}
\input{childdoc.def}
\childdocforward{cdocsamp}
%    \end{macrocode}

%\iffalse
%</sampledraft>
%\fi
%
% %%%%%%%%%%%%%%%%%%%%%%%%%%%%%%%%%%%%%%
% \paragraph{Forwarding for Final Version of the Chapters.}
%
% The following forwarding files |cdocsfn1.tex| and |cdocsfn2.tex|
% (with identical content)
% compile the final versions of the child documents
% |cdocsch1.tex| and |cdocsch2.tex|, respectively:
%\iffalse
%<*samplefinal>
%\fi
%    \begin{macrocode}
\def\version{final}
\input{childdoc.def}
\childdocforwardprefix[cdocsamp]{cdocsfn}{cdocsch}
%    \end{macrocode}

%\iffalse
%</samplefinal>
%\fi
%
% %%%%%%%%%%%%%%%%%%%%%%%%%%%%%%%%%%%%%%
% \paragraph{Command Line Processing.}
%
% The following three command lines generate the output files
% |cdocscld|, |cdocscl1| and |cdocscl2|
% which should be identical to
% |cdocsdrf|, |cdocsch1| and |cdocsfn2|, respectively:
% \begin{center}
% \begin{tabular}{l}
% |latex -jobname cdocscld \|\\
% |  "\def\version{draft}\input{childdoc.def}\childdocforward{cdocsamp}"|\\
% |latex -jobname cdocscl1 \|\\
% |  "\input{childdoc.def}\childdocforward[cdocsamp]{cdocsch1}"|\\
% |latex -jobname cdocscl2 \|\\
% |  "\def\version{final}\input{childdoc.def}\childdocforward{cdocsch2}"|
% \end{tabular}
% \end{center}
% Note that the trailing backslash on each first line
% merely continues the input to the second line
% (for convenient cut ant paste).
% Furthermore, the command |latex| can be replaced by any
% of its alternative versions such as |pdflatex|.
%
% %%%%%%%%%%%%%%%%%%%%%%%%%%%%%%%%%%%%%%%%%%%%%%%%%%%%%%%%%%%%%%%%%%%%%%%%%%%%%%
% %%%%%%%%%%%%%%%%%%%%%%%%%%%%%%%%%%%%%%%%%%%%%%%%%%%%%%%%%%%%%%%%%%%%%%%%%%%%%%
% \section{Implementation}
%\iffalse
%<*package>
%\fi
%
% This section describes the definitions file |childdoc.def|.

% The definitions cannot be loaded using |\usepackage| or |\RequirePackage|
% which has a mechanism to prevent loading a style file more than once.
% When loading the definitions by means of |\input|
% multiple instances have to be prevented manually:
%\iffalse
%This code needs to be before the `\ProvidesFile' directive
%which is defined at the beginning of this file.
%Therefore it is also placed there and commented out here.
%</package>
%<*discard>
%\fi
%    \begin{macrocode}
\ifdefined\childdocmain\endinput\fi
%    \end{macrocode}
%\iffalse
%</discard>
%<*package>
%\fi
%
% \macro{\ifchilddoc}
% \macro{\ifchilddocmanual}
% The conditional |\ifchilddoc| tells whether a
% child (true) or main (false) document is being compiled.
% The conditional |\ifchilddocmanual| tells whether
% the |\includeonly| mechanism is used (false) or
% the selection of child files must be performed manually (true).
% The definitions initialise to false:
%    \begin{macrocode}
\newif\ifchilddoc
\newif\ifchilddocmanual
%    \end{macrocode}

% \macro{\childdocname}
% \macro{\childdocjob}
% The macro |\childdocname| stores the name of the main document
% to be compiled. The macro |\childdocjob| stores the name of
% the document on which the \LaTeX{} compiler was originally invoked.
% The content of |\jobname| cannot be compared
% to filenames specified in the source due to different catcodes.
% The following code rescans |\jobname|, stores the result
% in |\childdocname| and saves a copy in |\childdocjob|:
%    \begin{macrocode}
\edef\childdocname{\scantokens\expandafter{\jobname\noexpand}}
\let\childdocjob\childdocname
%    \end{macrocode}

% \macro{\childdocdisable}
% The macro |\childdocdisable| prevents the main file
% from being processed more than once.
% At this stage, the main document command |\childdocmain|
% is assumed to be called once again where it should do nothing.
% Any subsequent call to it should prevent
% a secondary processing of the main document
% It overwrites the forwarding commands
% |\childdocof| and |\childdocforward|
% with empty macros to prevent further inclusions of the main document:
%    \begin{macrocode}
\newcommand{\childdocdisable}
{
  \renewcommand{\childdocmain}[1]{\renewcommand{\childdocmain}[1]{\endinput}}
  \renewcommand{\childdocof}[1]{}
  \renewcommand{\childdocby}[2][]{}
  \renewcommand{\childdocforward}[2][]{}
  \renewcommand{\childdocdisable}{}
}
%    \end{macrocode}

% \macro{\childdocmain}
% The macro |\childdocmain| is to be called at the top of the main file
% with nothing or the main filename (without extension) as argument.
% First, it breaks loops.
% If the argument is not empty and does not match |\childdocname|
% (which is set by the first inclusion of |childdoc.def|),
% |\ifchilddoc| is set to true, |\includeonly| is applied to the child file
% and |\jobname| is set to the main file
% (for proper handling of |.aux| files):
%    \begin{macrocode}
\newcommand{\childdocmain}[1]
{
  \childdocdisable\childdocmain{}
  \if?#1?\else
    \begingroup
      \def\childdoctmp{#1}
      \ifx\childdoctmp\childdocname
        \def\childdoctmp{}
      \else
        \def\childdoctmp
        {
          \childdoctrue
          \includeonly{\childdocname}
          \def\childdocjob{#1}
          \def\jobname{#1}
        }
      \fi
      \expandafter
    \endgroup
    \childdoctmp
  \fi
}
%    \end{macrocode}

% \macro{\childdocof}
% The command |\childdocof| redirects
% compilation to the main file |#1|.
%    \begin{macrocode}
\newcommand{\childdocof}[1]
{
  \childdocdisable
  \childdoctrue
  \includeonly{\childdocname}
  \def\jobname{#1}
  \def\childdocjob{#1}
  \input{#1}
}
%    \end{macrocode}

% \macro{\childdocby}
% The command |\childdocby| ....
%    \begin{macrocode}
\newcommand{\childdocby}[2][]
{
  \childdocdisable
  \childdoctrue
  \childdocmanualtrue
  \if?#1?\else
    \def\jobname{#2}
  \fi
  \def\childdocjob{#2}
  \input{#2}
  \endinput
}
%    \end{macrocode}

% \macro{\childdocforward}
% The command |\childdocforward| redirects
% compilation to the main file or
% (if the optional argument is given) a child file.
% Parameters are set as if the main file
% or a child file starting with |\childdocof| was compiled.
% Then compilation is handed over to the main file:
%    \begin{macrocode}
\newcommand{\childdocforward}[2][]
{
  \begingroup
    \if?#1?
      \def\childdoctmp
      {
        \def\childdocname{#2}
        \def\childdocjob{#2}
        \def\jobname{#2}
        \input{#2}
        \endinput
      }
    \else
      \def\childdoctmp
      {
        \childdocdisable
        \def\childdocname{#2}
        \childdoctrue
        \includeonly{#2}
        \def\childdocjob{#1}
        \def\jobname{#1}
        \input{#1}
        \endinput
      }
    \fi
    \expandafter
  \endgroup
  \childdoctmp
}
%    \end{macrocode}

% \macro{\childdocforwardprefix}
% The command |\childdocforwardprefix| redirects
% compilation to the main or a child file by means of a pattern.
% The prefix |#1| in the current filename is replaced by |#2|
% and the suffix of the current filename is kept
% (it is assumed that the filename does not contain the substring `|~~~|'
% which is used as a delimiter).
% Compilation is handed over to the new file by |\childdocforward|:
%    \begin{macrocode}
\newcommand{\childdocforwardprefix}[3][]
{
  \begingroup
    \def\childdocextract #2##1~~~{\def\childdoctmp{\childdocforward[#1]{#3##1}}}
    \expandafter\childdocextract\childdocname~~~
    \expandafter
  \endgroup
  \childdoctmp
}
%    \end{macrocode}

% \macro{\childdoc}
% The deprecated macro |\childdoc| is a legacy version of |\childdocmain|:
%    \begin{macrocode}
\newcommand{\childdoc}{\childdocmain}
%    \end{macrocode}

% \macro{\childdocredirect}
% The deprecated macro |\childdocredirect| is a legacy version
% of |\childdocforward| and |\childdocforwardprefix|:
%    \begin{macrocode}
\newcommand{\childdocredirect}[2][]
{
  \begingroup
    \if?#1?
      \def\childdoctmp{\childdocforward{#2}}
    \else
      \def\childdoctmp{\childdocforwardprefix{#1}{#2}}
    \fi
    \expandafter
  \endgroup
  \childdoctmp
}
%    \end{macrocode}

%\iffalse
%</package>
%\fi
%
\endinput

\childdocforwardprefix[cdocsamp]{cdocsfn}{cdocsch}
%    \end{macrocode}

%\iffalse
%</samplefinal>
%\fi
%
% %%%%%%%%%%%%%%%%%%%%%%%%%%%%%%%%%%%%%%
% \paragraph{Command Line Processing.}
%
% The following three command lines generate the output files
% |cdocscld|, |cdocscl1| and |cdocscl2|
% which should be identical to
% |cdocsdrf|, |cdocsch1| and |cdocsfn2|, respectively:
% \begin{center}
% \begin{tabular}{l}
% |latex -jobname cdocscld \|\\
% |  "\def\version{draft}% \iffalse
%
% childdoc.dtx Copyright (C) 2017-2018 Niklas Beisert
%
% This work may be distributed and/or modified under the
% conditions of the LaTeX Project Public License, either version 1.3
% of this license or (at your option) any later version.
% The latest version of this license is in
%   http://www.latex-project.org/lppl.txt
% and version 1.3 or later is part of all distributions of LaTeX
% version 2005/12/01 or later.
%
% This work has the LPPL maintenance status `maintained'.
%
% The Current Maintainer of this work is Niklas Beisert.
%
% This work consists of the files childdoc.dtx and childdoc.ins
% and the derived files childdoc.def and cdocsamp.tex with
% cdocsch1.tex, cdocsch2.tex, cdocsdrf.tex, cdocsfn1.tex, cdocsfn2.tex.
%
%<package>\ifdefined\childdocmain\endinput\fi
%<package>\ProvidesFile{childdoc.def}[2018/12/30 v2.0 child document driver]
%<samplemain>\ProvidesFile{cdocsamp.tex}[2018/12/30 v2.0 sample for childdoc]
%<*driver>
%\ProvidesFile{childdoc.drv}[2018/12/30 v2.0 childdoc reference manual file]
\PassOptionsToClass{10pt,a4paper}{article}
\documentclass{ltxdoc}

\usepackage[margin=35mm]{geometry}
\usepackage{hyperref}
\usepackage{hyperxmp}
\usepackage[usenames]{color}

\hypersetup{colorlinks=true}
\hypersetup{pdfstartview=FitH}
\hypersetup{pdfpagemode=UseNone}
\hypersetup{pdfsource={}}
\hypersetup{pdflang={en-UK}}
\hypersetup{pdfcopyright={Copyright 2017-2018 Niklas Beisert.
  This work may be distributed and/or modified under the
  conditions of the LaTeX Project Public License, either version 1.3
  of this license or (at your option) any later version.}}
\hypersetup{pdflicenseurl={http://www.latex-project.org/lppl.txt}}
\hypersetup{pdfcontactaddress={ETH Zurich, ITP, HIT K,
  Wolfgang-Pauli-Strasse 27}}
\hypersetup{pdfcontactpostcode={8093}}
\hypersetup{pdfcontactcity={Zurich}}
\hypersetup{pdfcontactcountry={Switzerland}}
\hypersetup{pdfcontactemail={nbeisert@itp.phys.ethz.ch}}
\hypersetup{pdfcontacturl={http://people.phys.ethz.ch/\xmptilde nbeisert/}}

\newcommand{\secref}[1]{\hyperref[#1]{section \ref*{#1}}}

\parskip1ex
\parindent0pt
\let\olditemize\itemize
\def\itemize{\olditemize\parskip0pt}

\begin{document}

\title{The \textsf{childdoc} Package}
\hypersetup{pdftitle={The childdoc Package}}
\author{Niklas Beisert\\[2ex]
  Institut f\"ur Theoretische Physik\\
  Eidgen\"ossische Technische Hochschule Z\"urich\\
  Wolfgang-Pauli-Strasse 27, 8093 Z\"urich, Switzerland\\[1ex]
  \href{mailto:nbeisert@itp.phys.ethz.ch}
  {\texttt{nbeisert@itp.phys.ethz.ch}}}
\hypersetup{pdfauthor={Niklas Beisert}}
\hypersetup{pdfsubject={Manual for the LaTeX2e Package childdoc}}
\date{30 December 2018, \textsf{v2.0}}
\maketitle

\begin{abstract}\noindent
\textsf{childdoc} is a \LaTeXe{} package
that enables the direct compilation
of document sections included by |\include|
to individual files.
\end{abstract}

\begingroup
\parskip0ex
\tableofcontents
\endgroup

%%%%%%%%%%%%%%%%%%%%%%%%%%%%%%%%%%%%%%%%%%%%%%%%%%%%%%%%%%%%%%%%%%%%%%%%%%%%%%%%
%%%%%%%%%%%%%%%%%%%%%%%%%%%%%%%%%%%%%%%%%%%%%%%%%%%%%%%%%%%%%%%%%%%%%%%%%%%%%%%%
\section{Introduction}

\LaTeX{} provides a mechanism to structure a large document (such as a book)
into a main file and several child files (containing the chapters)
using the |\include| command.
This mechanism is beneficial for documents
which span hundreds of pages in order to
make the source file(s) more manageable.
Moreover, compilation can be restricted to
selected child files by means of the |\includeonly| command.
The latter feature can be used to reduce the compilation time while editing
(this was significantly more useful in the earlier days of \LaTeX{})
or to generate a smaller document which is easier to navigate.
Another application of |\includeonly| is to generate
documents consisting of selected parts of the complete document.

However, there are a few drawbacks of the plain |\include| mechanism:
\begin{itemize}
\item
The child files cannot be compiled on their own,
they can only be compiled via the main file.
A naive editing environment
(such as a text editor with an option
to have the current file processed by \LaTeX)
may require one to switch to the main file before compiling;
attempting to compile the child file produces errors.
\item
The main file must be modified (each time)
to adjust the |\includeonly| command
to the present needs. This easily leaves the main file in a messy state.
\item
The generated document will always carry the filename
of the main document. This is inconvenient if
several child files are to be compiled and
to be kept for distribution.
\end{itemize}

The present package provides a simple interface
to make child files individually compilable by \LaTeX{}.
Compiling a child file then has the same effect as compiling
the main file with an |\includeonly| command
to select the appropriate child.
Moreover the generated document will carry the name of the child
rather than the main file.
This resolves all three above issues.

This feature is meant to make the editing of books,
thesis documents and lecture notes somewhat more convenient.
However, the package can also be used efficiently for
composing a series of documents (such as exercise sheets)
which are typically distributed individually.
It then assists the author in generating the individual documents
(potentially in different versions)
as well as a document containing the collected series.
Another application is in developing style files
or other kinds of included material
where compilation of the style file could redirect
to a sample or test file.

%%%%%%%%%%%%%%%%%%%%%%%%%%%%%%%%%%%%%%%%%%%%%%%%%%%%%%%%%%%%%%%%%%%%%%%%%%%%%%%%
%%%%%%%%%%%%%%%%%%%%%%%%%%%%%%%%%%%%%%%%%%%%%%%%%%%%%%%%%%%%%%%%%%%%%%%%%%%%%%%%
\section{Usage}

First of all, the package \textsf{childdoc} is \emph{not} a standard
\LaTeXe{} |.sty| style file! Therefore it needs to be invoked in
a non-standard way.

%%%%%%%%%%%%%%%%%%%%%%%%%%%%%%%%%%%%%%%%%%%%%%%%%%%%%%%%%%%%%%%%%%%%%%%%%%%%%%%%
\subsection{Included Files}
\label{sec:include}

%%%%%%%%%%%%%%%%%%%%%%%%%%%%%%%%%%%%%%%%
\DescribeMacro{\childdocmain}
To use the package, add the commands
\begin{center}
\begin{tabular}{l}
|\input{childdoc.def}|\\
|\childdocmain{}|\\
\end{tabular}
\end{center}
at the very top of the main \LaTeX{} file,
in particular \emph{before} the |\documentclass| statement!
The argument of |\childdocmain| should be left empty
(but it must be present).

%%%%%%%%%%%%%%%%%%%%%%%%%%%%%%%%%%%%%%%%
\DescribeMacro{\childdocof}
Furthermore, add the commands
\begin{center}
\begin{tabular}{l}
|\input{childdoc.def}|\\
|\childdocof{|\textit{main}|}|\\
\end{tabular}
\end{center}
at the top of every child file \textit{child}
which is included by |\include{|\textit{child}|}|
from within the main file
(or at least for those files to be compiled individually).
The argument \textit{main} must be the filename of the main file.

There are a couple of
considerations in setting up the main and child documents:

%%%%%%%%%%%%%%%%%%%%%%%%%%%%%%%%%%%%%%%%
\paragraph{Restrictions.}

Please note the following restrictions:
\begin{itemize}
\item
|\childdocmain| must be called with one argument \textit{main}
to ensure compatibility with earlier version of the package.
It must either be empty (|\childdocmain{}|)
or precisely match the filename of the main file in which it is specified.
See \secref{sec:detection} for further information.
\item
The filename \textit{main} must be specified without the |.tex| extension.
\item
The filename \textit{main} is case sensitive
(even in case-insensitive file systems)
due to internal string comparison.
\item
The argument \textit{main} should be fully expanded, it cannot be a macro.
\item
Subdirectories and special characters should be avoided in filenames.
\item
The command |\childdocmain{|\textit{main}|}| must be followed by a whitespace.
It should not be followed immediately by another command
or by a comment mark `|%|'.
This is because the \TeX{} parser reads the token immediately following
the argument of |\childdocmain| and puts it
at the beginning of every child section;
however, a white\-space is ignored.
\end{itemize}

%%%%%%%%%%%%%%%%%%%%%%%%%%%%%%%%%%%%%%%%
\paragraph{Content of Main File.}

It is advisable to place all content in the child files included by |\include|.
Any output contained in the main file will appear in all child documents
unless suppressed manually;
it cannot be suppressed automatically by the |\includeonly| directive
and thus should normally be avoided.
A method to include some content in the main file
by means of conditional processing is described in \secref{sec:conditional}.

%%%%%%%%%%%%%%%%%%%%%%%%%%%%%%%%%%%%%%%%
\paragraph{Page Numbering.}

When only a part of the document is compiled,
the appropriate numbering of pages
(as well as other status parameters)
is determined from the |.aux| files.
The latter contain information from previous passes.
However this information needs to propagate through
all intermediate child documents.
Therefore the page numbering in child documents may well
be inconsistent until the complete document is compiled at least once.

A useful (if unconventional) way to always ensure a consistent
page numbering is to restart the numbering in each child document
and denote the pages by `\textit{child}|.|\textit{page}'
where \textit{child} represents the chapter/section number of the child file.
This can be achieved by the command
|\numberwithin{page}{|\textit{child}|}|
of the \textsf{amsmath} package
where \textit{child} can be |chapter| or |section|
depending on the chosen structuring.
Alternatively, one can modify the macro |\thepage| appropriately
and reset the counter |page| at the start of each child file.

%%%%%%%%%%%%%%%%%%%%%%%%%%%%%%%%%%%%%%%%%%%%%%%%%%%%%%%%%%%%%%%%%%%%%%%%%%%%%%%%
\subsection{Conditional Processing}
\label{sec:conditional}

The package provides a mechanism to compile different versions
of a document. To customise the versions further some conditional processing
can come in handy to distinguish which version is being compiled.
The package provides two macros to describe the compilation context:

%%%%%%%%%%%%%%%%%%%%%%%%%%%%%%%%%%%%%%%%
\DescribeMacro{\ifchilddoc}
The conditional |\ifchilddoc| distinguishes between the compilation of
child documents and the main document:
%
\begin{center}
|\ifchilddoc |\textit{child-code}| |[|\||else |\textit{main-code}]| \||fi|
\end{center}

%%%%%%%%%%%%%%%%%%%%%%%%%%%%%%%%%%%%%%%%
\DescribeMacro{\childdocname}
\DescribeMacro{\childdocjob}
The macro |\childdocname| contains the filename (without extension)
of the main or child file being processed.
Note that |\childdocjob| will always contain the name of the main file.

%%%%%%%%%%%%%%%%%%%%%%%%%%%%%%%%%%%%%%%%
\paragraph{Title Page.}

Conditional processing can be used to include a title or banner page
in the main document when proper precautions are taken.
Importantly, the code in the main file should ensure that the page counter
(as well as other status parameters which are stored in the |.aux| files)
takes the same value after the conditional processing.
Otherwise the page numbers may take divergent values
depending on which part is compiled.

For example, a title page could be declared by:
%
\begin{center}
\begin{tabular}{l}
|\ifchilddoc\||else|\\
|\addtocounter{page}{-1}|\\
\textit{code for title page}\\
|\newpage|\\
|\||fi|
\end{tabular}
\end{center}
%
A banner page for the child documents can be generated by:
%
\begin{center}
\begin{tabular}{l}
|\ifchilddoc|\\
|\addtocounter{page}{-1}|\\
\textit{code for banner page}\\
|\newpage|\\
|\||fi|
\end{tabular}
\end{center}
%
Here one could write a message such as:
\begin{center}
|This is the part \childdocname{} of \childdocjob{}.|
\end{center}

%%%%%%%%%%%%%%%%%%%%%%%%%%%%%%%%%%%%%%%%%%%%%%%%%%%%%%%%%%%%%%%%%%%%%%%%%%%%%%%%
\subsection{Flags}
\label{sec:flags}

The package makes it easy to generate different versions
of the main or child documents.
To this end compilation flags can be defined
and assigned different default values.
They will be particularly useful in conjunction
with the forwarding mechanism described in \secref{sec:forward}.

For example, it may be useful to have a flag |\version|
which can be set to |draft| or |final|.
The document source will contain some conditional code
depending on the value of |\version|.
Suppose further, the flag should default to |final| for the main file
and to |draft| for child files
which is a natural assignment for editing the document.
This is achieved by placing the following code
in the preamble of the main document
(below the |\childdocmain| directive):
%
\begin{center}
\begin{tabular}{l}
|\ifchilddoc|\\
|\providecommand{\version}{draft}|\\
|\||else|\\
|\providecommand{\version}{final}|\\
|\||fi|
\end{tabular}
\end{center}
%
The definition by |\providecommand| makes sure
that previous definitions are not overwritten.
Further statements |\providecommand{\version}{...}|
can thus be added before the above code to override it.

For the main file, one might add a line
(between |\childdocmain| and the above block)
%
\begin{center}
|%\ifchilddoc\||else\providecommand{\version}{draft}\||fi|
\end{center}
%
which can be uncommented to produce a draft version.
Likewise one can add a line to the very top of a child file
(above the |\childdocof{|\textit{main}|}| directive)
%
\begin{center}
|%\providecommand{\version}{final}|
\end{center}
%
which can be uncommented to produce the final version of this child document.

%%%%%%%%%%%%%%%%%%%%%%%%%%%%%%%%%%%%%%%%%%%%%%%%%%%%%%%%%%%%%%%%%%%%%%%%%%%%%%%%
\subsection{Forwarding}
\label{sec:forward}

Different versions of the main or child documents
using compilation flags as described in \secref{sec:flags}
can be (permanently) stored in different files
for convenient compilation, viewing and distribution.
To this end, the package defines a command
to pass on compilation to a different file:

%%%%%%%%%%%%%%%%%%%%%%%%%%%%%%%%%%%%%%%%
\DescribeMacro{\childdocforward}
The command |\childdocforward| redirects processing to
another source file:
%
\begin{center}
\begin{tabular}{l}
|\input{childdoc.def}|\\
|\childdocforward[|\textit{main}|]{|\textit{dest}|}|\\
\end{tabular}
\end{center}
%
The argument \textit{dest} is the destination file
(without extension).
It should be the main file or one of the child files.
Note that further \textsf{childdoc} directives
such as |\childdocof| and |\childdocforward|
in the indicated file will be processed in this form.
The optional argument \textit{main}
passes on directly to the main file \textit{main}
while pretending to compile the child \textit{dest}.
This form behaves as if \textit{dest}
issues |\childdocof{|\textit{main}|}| right away,
and no further \textsf{childdoc} directives will be processed.

%%%%%%%%%%%%%%%%%%%%%%%%%%%%%%%%%%%%%%%%
\DescribeMacro{\...prefix}
In the alternative form |\childdocforwardprefix|,
%
\begin{center}
\begin{tabular}{l}
|\input{childdoc.def}|\\
|\childdocforwardprefix[|\textit{main}|]{|\textit{prefix}|}{|\textit{dest}|}|
\end{tabular}
\end{center}
%
the destination file is determined by a pattern
depending on the current file:
To make this work, the current file must be called
`{\textit{prefix}\hspace{0.2em}\textit{suffix}}'
with \textit{prefix} matching precisely the argument.
Processing is then passed on to the file
`{\textit{dest}\hspace{0.2em}\textit{suffix}}'.
Surely, the same effect is achieved by
directly specifying the
argument `{\textit{dest}\hspace{0.2em}\textit{suffix}}'
in the first form.
However, that requires to set up a different file
for each child. With the alternative form of the command
all these files can have exactly the same content
which simplifies setting them up and maintaining them.

For example, the following file |draft.tex|
with a compilation flag |\version| as described in \secref{sec:flags}
compiles the main document as a draft:
%
\begin{center}
\begin{tabular}{l}
|\def\version{draft}|\\
|\input{childdoc.def}|\\
|\childdocforward{|\textit{main}|}|
\end{tabular}
\end{center}
%
Likewise, the following files |final|\textit{nn}|.tex|
compile the final version of the child document
|child|\textit{nn}|.tex|:
%
\begin{center}
\begin{tabular}{l}
|\def\version{final}|\\
|\input{childdoc.def}|\\
|\childdocforwardprefix{final}{child}|
\end{tabular}
\end{center}
%

Note that when several versions of a main file and/or of each child file
are to be generated, it may be convenient to set up a |Makefile| or
shell script to automatise the process.

%%%%%%%%%%%%%%%%%%%%%%%%%%%%%%%%%%%%%%%%%%%%%%%%%%%%%%%%%%%%%%%%%%%%%%%%%%%%%%%%
\subsection{Command Line Processing}
\label{sec:commandline}

The effect of redirection files can also be achieved by invoking
the \LaTeX{} compiler with a more elaborate command line.
Most conveniently this should be done as part
of a shell script or a |Makefile|.

When using \textsf{childdoc} in the main file, the following
command lines effectively perform a redirection
(note that depending on the shell being used,
backslashes may have to be doubled: `|\|' $\to$ `|\\|'):
%
\begin{center}
|... -jobname "|\textit{target}|" |\\|"|[\textit{flags}]%
|\input{childdoc.def}\childdocforward[|\textit{main}|]{|\textit{dest}|}"|
\end{center}
%
Here \textit{target} is the name of the output file,
\textit{main} is the name of the main file
and \textit{dest} is the name of the main or child file to be processed
(all filenames without extensions).
The optional argument \textit{main} can be omitted
if \textit{main} matches \textit{dest}.
Optionally, compilation \textit{flags} can be defined via |\def| commands.
This command line makes the \TeX{} engine believe
it is compiling the file \textit{target}
whose content is specified as the latter parameter.
The provided code then forwards the processing to
\textit{main} or \textit{dest} as described in \secref{sec:forward}.

%%%%%%%%%%%%%%%%%%%%%%%%%%%%%%%%%%%%%%%%%%%%%%%%%%%%%%%%%%%%%%%%%%%%%%%%%%%%%%%%
\subsection{Include by Input}
\label{sec:input}

Including child documents by |\include| has some restrictions by design.
Most notably, the content of a child document always occupies
its own set of pages; pages cannot be shared between child documents.
Usually, this behaviour makes perfect sense
because each child document contain an essential part of the document.
However, in some situations it may be desirable to compose
a document from a collection of parts
without having mandatory page breaks between then.
For this case, the package
provides a mechanism to include parts
by |\input| which can also be processed individually.
However, by construction this mechanism
requires manual handling of the content to be output.

%%%%%%%%%%%%%%%%%%%%%%%%%%%%%%%%%%%%%%%%
\DescribeMacro{\ifchilddocmanual}
The main file should be prepared as usual, see \secref{sec:include}.
However, the document body must make a distinction
between processing of an individual part and of the main document, e.g.:
%
\begin{center}
\begin{tabular}{l}
|\ifchilddocmanual|\\
|\input{\childdocname}|\\
|\||else|\\
\textit{document body with }|\input{|\textit{part}|}|\\
|\||fi|
\end{tabular}
\end{center}
%
The conditional |\ifchilddocmanual| is true whenever
a part to be included by |\input| is being compiled,
and the name of the part is stored in |\childdocname|.

%%%%%%%%%%%%%%%%%%%%%%%%%%%%%%%%%%%%%%%%
\DescribeMacro{\childdocby}
Each part to be included by |\input| should start with:
%
\begin{center}
\begin{tabular}{l}
|\input{childdoc.def}|\\
|\childdocby{|\textit{main}|}|\\
\end{tabular}
\end{center}
%
The directive |\childdocby| is similar to |\childdocof|
described in \secref{sec:include},
but the subsequent selection of content must be done manually.
To that end, both |\ifchilddoc| and |\ifchilddocmanual|
will be true upon processing of a part,
and the name of the part is stored in |\childdocname|.
Note that |\jobname| will be set to the filename of the current part
so that each part receives an individual |.aux| file
that does not interfere with the |.aux| file(s) of the main document.
This behaviour can be altered by the alternative form
|\childdocby[*]{|\textit{main}|}| (with a non-empty optional argument)
which uses the |.aux| file of the main document
by setting |\jobname| to \textit{main}.

%%%%%%%%%%%%%%%%%%%%%%%%%%%%%%%%%%%%%%%%%%%%%%%%%%%%%%%%%%%%%%%%%%%%%%%%%%%%%%%%
\subsection{Driver Development}
\label{sec:driver}

The \textsf{childdoc} mechanism can also be use for the development
of definition files such as \LaTeX{} styles or classes.
This case differs from the above setup with multiple parts
included by |\include| in that no |\includeonly| should be invoked.
This can be achieved by starting the include file
(before |\ProvidesPackage|) with:
%
\begin{center}
\begin{tabular}{l}
|\input{childdoc.def}|\\
|\childdocforward{|\textit{main}|}|\\
\end{tabular}
\end{center}
%
or alternatively with:
%
\begin{center}
\begin{tabular}{l}
|\input{childdoc.def}|\\
|\childdocby{|\textit{main}|}|\\
\end{tabular}
\end{center}
%
Both forms have slightly different effects as described above.
The main file is prepared as usual, see \secref{sec:include}.

%%%%%%%%%%%%%%%%%%%%%%%%%%%%%%%%%%%%%%%%%%%%%%%%%%%%%%%%%%%%%%%%%%%%%%%%%%%%%%%%
\subsection{Legacy Detection}
\label{sec:detection}

The directive |\childdocmain| in the main file can detect
whether the complete document or merely a child is to be compiled
even without using the directive |\childdocof|.
This method is deprecated because it is less robust
and there is no compelling reason to use it;
it is merely provided for backward compatibility
and it may be removed in future versions.

If the detection mechanism is to be used,
it is mandatory to correctly specify
the filename of the main file as the argument of |\childdocmain|:
%
\begin{center}
\begin{tabular}{l}
|\input{childdoc.def}|\\
|\childdocmain{|\textit{main}|}|\\
\end{tabular}
\end{center}
%
If |\jobname| does not match the argument \textit{main} of |\childdocmain|,
it is assumed that |\jobname| points to the child file to be compiled.
When using |\childdocmain| with the main file specified as argument,
it suffices to start a child file
with just |\input{|\textit{main}|}|
without loading of the package and using |\childdocof|.
If instead all processing is done
with the appropriate \textsf{childdoc} directives,
the argument of \textit{main} of |\childdocmain| can be empty.

An alternative version of the command line processing described
in \secref{sec:commandline} using the detection mechanism reads:
%
\begin{center}
|... -jobname "|\textit{target}|" "|[\textit{flags}]%
[|\def\jobname{|\textit{dest}|}|]|\input{|\textit{main}|}"|
\end{center}

%%%%%%%%%%%%%%%%%%%%%%%%%%%%%%%%%%%%%%%%%%%%%%%%%%%%%%%%%%%%%%%%%%%%%%%%%%%%%%%%
\subsection{Manual Code}
\label{sec:manual}

In case one cannot be certain whether the definitions file |childdoc.def|
is installed on the target \TeX{} distribution
and one prefers not to ship it,
it is conceivable to paste a few relevant commands into the sources.

To that end, drop all statements |\input{childdoc.def}|
and perform the replacements as outlined below.
Instead of |\childdocmain{|\textit{main}|}| add the following code
to the top of the main file:
%
\begin{center}
\begin{tabular}{l}
|\||ifdefined\childdocname\endinput\||fi\newif\ifchilddoc|\\
|\edef\childdocname{\scantokens\expandafter{\jobname\noexpand}}|\\
|\def\childdocmain{|\textit{main}|}\||ifx\childdocmain\childdocname\||else|\\
|\childdoctrue\includeonly{\childdocname}\let\jobname\childdocmain\||fi|\\
\end{tabular}
\end{center}
%
Instead of |\childdocof{|\textit{main}|}| just include the main file
at the top of each child file:
%
\begin{center}
|\input{|\textit{main}|}|
\end{center}
%
A simple redirection |\childdocforward{|\textit{dest}|}| is achieved by:
%
\begin{center}
|\def\jobname{|\textit{dest}|}\input{\jobname}|
\end{center}
%
The redirection with prefix
|\childdocforwardprefix[|\textit{prefix}|]{|\textit{dest}|}|
is accomplished by:
%
\begin{center}
\begin{tabular}{l}
|{\edef\jobname{\scantokens\expandafter{\jobname\noexpand}}|\\
|\def\redirectjob |\textit{prefix}|#1~~~{\gdef\jobname{|\textit{dest}|#1}}|\\
|\expandafter\redirectjob\jobname~~~}\input{\jobname}|
\end{tabular}
\end{center}

In an alternative approach,
child documents can be compiled by a specific command line
without additional code or specific definitions:
%
\begin{center}
|... -jobname "|\textit{target}|" "|[\textit{flags}]%
|\includeonly{|\textit{dest}|}\input{|\textit{main}|}"|
\end{center}
%

%%%%%%%%%%%%%%%%%%%%%%%%%%%%%%%%%%%%%%%%%%%%%%%%%%%%%%%%%%%%%%%%%%%%%%%%%%%%%%%%
%%%%%%%%%%%%%%%%%%%%%%%%%%%%%%%%%%%%%%%%%%%%%%%%%%%%%%%%%%%%%%%%%%%%%%%%%%%%%%%%
\section{Information}

%%%%%%%%%%%%%%%%%%%%%%%%%%%%%%%%%%%%%%%%%%%%%%%%%%%%%%%%%%%%%%%%%%%%%%%%%%%%%%%%
\subsection{Copyright}

Copyright \copyright{} 2017--2018 Niklas Beisert

This work may be distributed and/or modified under the
conditions of the \LaTeX{} Project Public License, either version 1.3
of this license or (at your option) any later version.
The latest version of this license is in
  \url{http://www.latex-project.org/lppl.txt}
and version 1.3 or later is part of all distributions of \LaTeX{}
version 2005/12/01 or later.

This work has the LPPL maintenance status `maintained'.

The Current Maintainer of this work is Niklas Beisert.

This work consists of the files |README.txt|, |childdoc.ins| and |childdoc.dtx|
as well as the derived files |childdoc.def|, |cdocsamp.tex|
with |cdocsch1.tex|, |cdocsch2.tex|, |cdocspt3.tex|, |cdocspt4.tex|,
|cdocsdrf.tex|, |cdocsfn1.tex|, |cdocsfn2.tex|
as well as |childdoc.pdf|.

%%%%%%%%%%%%%%%%%%%%%%%%%%%%%%%%%%%%%%%%%%%%%%%%%%%%%%%%%%%%%%%%%%%%%%%%%%%%%%%%
\subsection{Files and Installation}

The package consists of the files:
%
\begin{center}
\begin{tabular}{ll}
    |README.txt|   & readme file \\
    |childdoc.ins| & installation file \\
    |childdoc.dtx| & source file \\
    |childdoc.def| & definition file \\
    |cdocsamp.tex| & sample main file \\
    |cdocsch1.tex| & sample include file \\
    |cdocsch2.tex| & sample include file \\
    |cdocspt3.tex| & sample part file \\
    |cdocspt4.tex| & sample part file \\
    |cdocsdrf.tex| & sample redirection file \\
    |cdocsfn1.tex| & sample redirection file \\
    |cdocsfn2.tex| & sample redirection file \\
    |childdoc.pdf| & manual
\end{tabular}
\end{center}
%
The distribution consists of the files
|README.txt|, |childdoc.ins| and |childdoc.dtx|.
%
\begin{itemize}
\item
Run (pdf)\LaTeX{} on |childdoc.dtx|
to compile the manual |childdoc.pdf| (this file).
\item
Run \LaTeX{} on |childdoc.ins| to create the definitions file |childdoc.def|
and the sample |cdocsamp.tex| with include files
|cdocsch1.tex|, |cdocsch2.tex|, |cdocspt3.tex|, |cdocspt4.tex|,
|cdocsdrf.tex|, |cdocsfn1.tex|, |cdocsfn2.tex|.
Then copy the file |childdoc.def| to an appropriate directory of your \LaTeX{}
distribution, e.g.\ \textit{texmf-root}|/tex/latex/childdoc|.
\end{itemize}

%%%%%%%%%%%%%%%%%%%%%%%%%%%%%%%%%%%%%%%%%%%%%%%%%%%%%%%%%%%%%%%%%%%%%%%%%%%%%%%%
\subsection{Related CTAN Packages}

There are several other packages which offer a similar functionality:
%
\begin{itemize}
\item
The packages
\href{http://ctan.org/pkg/docmute}{\textsf{docmute}},
\href{http://ctan.org/pkg/includex}{\textsf{includex}} and
\href{http://ctan.org/pkg/standalone}{\textsf{standalone}}
provide commands to include only the document body of
a child file thus allowing both files to be compiled individually.
\item
The packages \href{http://ctan.org/pkg/subdocs}{\textsf{subdocs}}
and \href{http://ctan.org/pkg/subfiles}{\textsf{subfiles}}
provide structures in which the main and child documents can be
encapsulated and allowing them to be compiled individually.
The inclusion mechanism is different from the conventional |\include|.
\item
The package \href{http://ctan.org/pkg/combine}{\textsf{combine}}
is an elaborate solution to combine several documents into one.
\end{itemize}
%
See also the CTAN topic \href{http://ctan.org/topic/subdocs}{\textsf{subdocs}}
for further related packages.
The present package differs from the above solutions in that
a document structure constructed with the conventional |\include| mechanism
just needs two extra commands at the top of every file
such that all constituent files can be compiled individually.

%%%%%%%%%%%%%%%%%%%%%%%%%%%%%%%%%%%%%%%%%%%%%%%%%%%%%%%%%%%%%%%%%%%%%%%%%%%%%%%%
%\subsection{Feature Suggestions}
%
%The following is a list of features which may be useful for future
%versions of this package:
%%
%\begin{itemize}
%\item
%\ldots
%\end{itemize}

%%%%%%%%%%%%%%%%%%%%%%%%%%%%%%%%%%%%%%%%%%%%%%%%%%%%%%%%%%%%%%%%%%%%%%%%%%%%%%%%
\subsection{Revision History}

%%%%%%%%%%%%%%%%%%%%%%%%%%%%%%%%%%%%%%%%
\paragraph{v2.0:} 2018/12/30

\begin{itemize}
\item
immediate forward processing
\item
added |\childdocby| mechanism
\item
manual restructured
\end{itemize}

%%%%%%%%%%%%%%%%%%%%%%%%%%%%%%%%%%%%%%%%
\paragraph{v1.6:} 2018/01/17

\begin{itemize}
\item
application for development of include files
\item
corrections to manual
\end{itemize}

%%%%%%%%%%%%%%%%%%%%%%%%%%%%%%%%%%%%%%%%
\paragraph{v1.5:} 2017/05/21

\begin{itemize}
\item
more complete structuring introduced
\item
|\childdocof| introduced
\item
|\childdoc| renamed to |\childdocmain|
\item
|\childredirect| renamed to |\childdocforward| and |\childdocforwardprefix|
and functionality expanded
\end{itemize}

%%%%%%%%%%%%%%%%%%%%%%%%%%%%%%%%%%%%%%%%
\paragraph{v1.0:} 2017/04/27

\begin{itemize}
\item
manual and install package
\item
first version published on CTAN
\end{itemize}

%%%%%%%%%%%%%%%%%%%%%%%%%%%%%%%%%%%%%%%%
\paragraph{v0.6:} 2017/04/26

\begin{itemize}
\item
redirection mechanism added
\end{itemize}

%%%%%%%%%%%%%%%%%%%%%%%%%%%%%%%%%%%%%%%%
\paragraph{v0.5:} 2017/04/26

\begin{itemize}
\item
functionality in definition file
\end{itemize}


%%%%%%%%%%%%%%%%%%%%%%%%%%%%%%%%%%%%%%%%%%%%%%%%%%%%%%%%%%%%%%%%%%%%%%%%%%%%%%%%
%%%%%%%%%%%%%%%%%%%%%%%%%%%%%%%%%%%%%%%%%%%%%%%%%%%%%%%%%%%%%%%%%%%%%%%%%%%%%%%%
%%%%%%%%%%%%%%%%%%%%%%%%%%%%%%%%%%%%%%%%%%%%%%%%%%%%%%%%%%%%%%%%%%%%%%%%%%%%%%%%
\appendix

\settowidth\MacroIndent{\rmfamily\scriptsize 000\ }

 \DocInput{childdoc.dtx}

\end{document}
%</driver>
% \fi
%
% %%%%%%%%%%%%%%%%%%%%%%%%%%%%%%%%%%%%%%%%%%%%%%%%%%%%%%%%%%%%%%%%%%%%%%%%%%%%%%
% %%%%%%%%%%%%%%%%%%%%%%%%%%%%%%%%%%%%%%%%%%%%%%%%%%%%%%%%%%%%%%%%%%%%%%%%%%%%%%
% \section{Sample}
%\iffalse
%<*samplemain>
%\fi
%
% The following presents a sample document
% with two chapters, two parts, a title page,
% a compile flag as well as three forwarding files to set the flag.
% It consists of eight |.tex| files:
% \begin{center}
% \begin{tabular}{ll}
% |cdocsamp.tex|&main file\\
% |cdocsch1.tex|&include file for chapter 1\\
% |cdocsch2.tex|&include file for chapter 2\\
% |cdocspt3.tex|&include file for part 3\\
% |cdocspt4.tex|&include file for part 4\\
% |cdocsdrf.tex|&forwarding file for main file in draft mode\\
% |cdocsfi1.tex|&forwarding file for final version of chapter 1\\
% |cdocsfi2.tex|&forwarding file for final version of chapter 2\\
% \end{tabular}
% \end{center}
% Each of the eight files can be compiled directly by the \LaTeX{} compiler.
%
% %%%%%%%%%%%%%%%%%%%%%%%%%%%%%%%%%%%%%%
% \paragraph{Main File.}
%
% The main file is called |cdocsamp.tex|.
%
% Load the \textsf{childdoc} definitions and
% declare the filename for the main document:
%    \begin{macrocode}
\input{childdoc.def}
\childdocmain{}
%    \end{macrocode}

% Optional override for |\version| flag:
%    \begin{macrocode}
%%\ifchilddoc\else\providecommand{\version}{draft}\fi
%    \end{macrocode}

% Define the default values for the |\version| flag
% (|final| for the main file and |draft| for childs):
%    \begin{macrocode}
\ifchilddoc
\providecommand{\version}{draft}
\else
\providecommand{\version}{final}
\fi
%    \end{macrocode}

% Load the standard document class:
%    \begin{macrocode}
\documentclass[12pt]{article}
%    \end{macrocode}

% Start the document body:
%    \begin{macrocode}
\begin{document}
%    \end{macrocode}

% Declare a title page.
% Print title, part of document being processed and version flag:
%    \begin{macrocode}
\addtocounter{page}{-1}
\begin{center}
{\LARGE\bfseries{}childdoc example\par}
\vspace{1cm}
\ifchilddoc
\ifchilddocmanual part\else chapter\fi:
`\childdocname' of `\childdocjob'\par
\else
main document: `\childdocjob'\par
\fi
version: \version\par
\end{center}
\newpage
%    \end{macrocode}

% Manually include selected file,
% otherwise process as usual:
%    \begin{macrocode}
\ifchilddocmanual
\section*{part `\childdocname'}
\input{\childdocname}
\else
%    \end{macrocode}

% Include the two chapters:
%    \begin{macrocode}
\include{cdocsch1}
\include{cdocsch2}
%    \end{macrocode}

% Include the two parts unless only chapters should be displayed:
%    \begin{macrocode}
\ifchilddoc\else
\section{part three}
\input{cdocspt3}
\section{part four}
\input{cdocspt4}
\fi
%    \end{macrocode}

% Process as usual until here:
%    \begin{macrocode}
\fi
%    \end{macrocode}

% End of document body:
%    \begin{macrocode}
\end{document}
%    \end{macrocode}
%\iffalse
%</samplemain>
%\fi
%
% %%%%%%%%%%%%%%%%%%%%%%%%%%%%%%%%%%%%%%
% \paragraph{Chapter Include Files.}
%
% The include files are called |cdocsch1.tex| and |cdocsch2.tex|.
%
%\iffalse
%<*samplechap1|samplechap2>
%\fi

% Optional override for |\version| flag:
%    \begin{macrocode}
%%\providecommand{\version}{final}
%    \end{macrocode}

% Include the main document:
%    \begin{macrocode}
\input{childdoc.def}
\childdocof{cdocsamp}
%    \end{macrocode}

%\iffalse
%</samplechap1|samplechap2>
%\fi
%
%\iffalse
%<*samplechap1>
%\fi
% Some text for chapter 1:
%    \begin{macrocode}
\section{one}
some text in chapter one
%    \end{macrocode}

%\iffalse
%</samplechap1>
%\fi
% Some text for chapter 2:
%\iffalse
%<*samplechap2>
%\fi
%    \begin{macrocode}
\section{two}
more text in chapter two
%    \end{macrocode}

%\iffalse
%</samplechap2>
%\fi
%
% %%%%%%%%%%%%%%%%%%%%%%%%%%%%%%%%%%%%%%
% \paragraph{Part Include Files.}
%
% The include files are called |cdocspt3.tex| and |cdocspt4.tex|.
%
%\iffalse
%<*samplepart3|samplepart4>
%\fi

% Optional override for |\version| flag:
%    \begin{macrocode}
%%\providecommand{\version}{final}
%    \end{macrocode}

% Include the main document:
%    \begin{macrocode}
\input{childdoc.def}
\childdocby{cdocsamp}
%    \end{macrocode}

%\iffalse
%</samplepart3|samplepart4>
%\fi
%
%\iffalse
%<*samplepart3>
%\fi
% Some text for part 3:
%    \begin{macrocode}
some text in part three
%    \end{macrocode}

%\iffalse
%</samplepart3>
%\fi
% Some text for part 4:
%\iffalse
%<*samplepart4>
%\fi
%    \begin{macrocode}
more text in part four
%    \end{macrocode}

%\iffalse
%</samplepart4>
%\fi
%
% %%%%%%%%%%%%%%%%%%%%%%%%%%%%%%%%%%%%%%
% \paragraph{Forwarding for a Complete Draft.}
%
% The following forwarding file |cdocsdrf.tex|
% compiles the main document in draft mode:
%\iffalse
%<*sampledraft>
%\fi
%    \begin{macrocode}
\def\version{draft}
\input{childdoc.def}
\childdocforward{cdocsamp}
%    \end{macrocode}

%\iffalse
%</sampledraft>
%\fi
%
% %%%%%%%%%%%%%%%%%%%%%%%%%%%%%%%%%%%%%%
% \paragraph{Forwarding for Final Version of the Chapters.}
%
% The following forwarding files |cdocsfn1.tex| and |cdocsfn2.tex|
% (with identical content)
% compile the final versions of the child documents
% |cdocsch1.tex| and |cdocsch2.tex|, respectively:
%\iffalse
%<*samplefinal>
%\fi
%    \begin{macrocode}
\def\version{final}
\input{childdoc.def}
\childdocforwardprefix[cdocsamp]{cdocsfn}{cdocsch}
%    \end{macrocode}

%\iffalse
%</samplefinal>
%\fi
%
% %%%%%%%%%%%%%%%%%%%%%%%%%%%%%%%%%%%%%%
% \paragraph{Command Line Processing.}
%
% The following three command lines generate the output files
% |cdocscld|, |cdocscl1| and |cdocscl2|
% which should be identical to
% |cdocsdrf|, |cdocsch1| and |cdocsfn2|, respectively:
% \begin{center}
% \begin{tabular}{l}
% |latex -jobname cdocscld \|\\
% |  "\def\version{draft}\input{childdoc.def}\childdocforward{cdocsamp}"|\\
% |latex -jobname cdocscl1 \|\\
% |  "\input{childdoc.def}\childdocforward[cdocsamp]{cdocsch1}"|\\
% |latex -jobname cdocscl2 \|\\
% |  "\def\version{final}\input{childdoc.def}\childdocforward{cdocsch2}"|
% \end{tabular}
% \end{center}
% Note that the trailing backslash on each first line
% merely continues the input to the second line
% (for convenient cut ant paste).
% Furthermore, the command |latex| can be replaced by any
% of its alternative versions such as |pdflatex|.
%
% %%%%%%%%%%%%%%%%%%%%%%%%%%%%%%%%%%%%%%%%%%%%%%%%%%%%%%%%%%%%%%%%%%%%%%%%%%%%%%
% %%%%%%%%%%%%%%%%%%%%%%%%%%%%%%%%%%%%%%%%%%%%%%%%%%%%%%%%%%%%%%%%%%%%%%%%%%%%%%
% \section{Implementation}
%\iffalse
%<*package>
%\fi
%
% This section describes the definitions file |childdoc.def|.

% The definitions cannot be loaded using |\usepackage| or |\RequirePackage|
% which has a mechanism to prevent loading a style file more than once.
% When loading the definitions by means of |\input|
% multiple instances have to be prevented manually:
%\iffalse
%This code needs to be before the `\ProvidesFile' directive
%which is defined at the beginning of this file.
%Therefore it is also placed there and commented out here.
%</package>
%<*discard>
%\fi
%    \begin{macrocode}
\ifdefined\childdocmain\endinput\fi
%    \end{macrocode}
%\iffalse
%</discard>
%<*package>
%\fi
%
% \macro{\ifchilddoc}
% \macro{\ifchilddocmanual}
% The conditional |\ifchilddoc| tells whether a
% child (true) or main (false) document is being compiled.
% The conditional |\ifchilddocmanual| tells whether
% the |\includeonly| mechanism is used (false) or
% the selection of child files must be performed manually (true).
% The definitions initialise to false:
%    \begin{macrocode}
\newif\ifchilddoc
\newif\ifchilddocmanual
%    \end{macrocode}

% \macro{\childdocname}
% \macro{\childdocjob}
% The macro |\childdocname| stores the name of the main document
% to be compiled. The macro |\childdocjob| stores the name of
% the document on which the \LaTeX{} compiler was originally invoked.
% The content of |\jobname| cannot be compared
% to filenames specified in the source due to different catcodes.
% The following code rescans |\jobname|, stores the result
% in |\childdocname| and saves a copy in |\childdocjob|:
%    \begin{macrocode}
\edef\childdocname{\scantokens\expandafter{\jobname\noexpand}}
\let\childdocjob\childdocname
%    \end{macrocode}

% \macro{\childdocdisable}
% The macro |\childdocdisable| prevents the main file
% from being processed more than once.
% At this stage, the main document command |\childdocmain|
% is assumed to be called once again where it should do nothing.
% Any subsequent call to it should prevent
% a secondary processing of the main document
% It overwrites the forwarding commands
% |\childdocof| and |\childdocforward|
% with empty macros to prevent further inclusions of the main document:
%    \begin{macrocode}
\newcommand{\childdocdisable}
{
  \renewcommand{\childdocmain}[1]{\renewcommand{\childdocmain}[1]{\endinput}}
  \renewcommand{\childdocof}[1]{}
  \renewcommand{\childdocby}[2][]{}
  \renewcommand{\childdocforward}[2][]{}
  \renewcommand{\childdocdisable}{}
}
%    \end{macrocode}

% \macro{\childdocmain}
% The macro |\childdocmain| is to be called at the top of the main file
% with nothing or the main filename (without extension) as argument.
% First, it breaks loops.
% If the argument is not empty and does not match |\childdocname|
% (which is set by the first inclusion of |childdoc.def|),
% |\ifchilddoc| is set to true, |\includeonly| is applied to the child file
% and |\jobname| is set to the main file
% (for proper handling of |.aux| files):
%    \begin{macrocode}
\newcommand{\childdocmain}[1]
{
  \childdocdisable\childdocmain{}
  \if?#1?\else
    \begingroup
      \def\childdoctmp{#1}
      \ifx\childdoctmp\childdocname
        \def\childdoctmp{}
      \else
        \def\childdoctmp
        {
          \childdoctrue
          \includeonly{\childdocname}
          \def\childdocjob{#1}
          \def\jobname{#1}
        }
      \fi
      \expandafter
    \endgroup
    \childdoctmp
  \fi
}
%    \end{macrocode}

% \macro{\childdocof}
% The command |\childdocof| redirects
% compilation to the main file |#1|.
%    \begin{macrocode}
\newcommand{\childdocof}[1]
{
  \childdocdisable
  \childdoctrue
  \includeonly{\childdocname}
  \def\jobname{#1}
  \def\childdocjob{#1}
  \input{#1}
}
%    \end{macrocode}

% \macro{\childdocby}
% The command |\childdocby| ....
%    \begin{macrocode}
\newcommand{\childdocby}[2][]
{
  \childdocdisable
  \childdoctrue
  \childdocmanualtrue
  \if?#1?\else
    \def\jobname{#2}
  \fi
  \def\childdocjob{#2}
  \input{#2}
  \endinput
}
%    \end{macrocode}

% \macro{\childdocforward}
% The command |\childdocforward| redirects
% compilation to the main file or
% (if the optional argument is given) a child file.
% Parameters are set as if the main file
% or a child file starting with |\childdocof| was compiled.
% Then compilation is handed over to the main file:
%    \begin{macrocode}
\newcommand{\childdocforward}[2][]
{
  \begingroup
    \if?#1?
      \def\childdoctmp
      {
        \def\childdocname{#2}
        \def\childdocjob{#2}
        \def\jobname{#2}
        \input{#2}
        \endinput
      }
    \else
      \def\childdoctmp
      {
        \childdocdisable
        \def\childdocname{#2}
        \childdoctrue
        \includeonly{#2}
        \def\childdocjob{#1}
        \def\jobname{#1}
        \input{#1}
        \endinput
      }
    \fi
    \expandafter
  \endgroup
  \childdoctmp
}
%    \end{macrocode}

% \macro{\childdocforwardprefix}
% The command |\childdocforwardprefix| redirects
% compilation to the main or a child file by means of a pattern.
% The prefix |#1| in the current filename is replaced by |#2|
% and the suffix of the current filename is kept
% (it is assumed that the filename does not contain the substring `|~~~|'
% which is used as a delimiter).
% Compilation is handed over to the new file by |\childdocforward|:
%    \begin{macrocode}
\newcommand{\childdocforwardprefix}[3][]
{
  \begingroup
    \def\childdocextract #2##1~~~{\def\childdoctmp{\childdocforward[#1]{#3##1}}}
    \expandafter\childdocextract\childdocname~~~
    \expandafter
  \endgroup
  \childdoctmp
}
%    \end{macrocode}

% \macro{\childdoc}
% The deprecated macro |\childdoc| is a legacy version of |\childdocmain|:
%    \begin{macrocode}
\newcommand{\childdoc}{\childdocmain}
%    \end{macrocode}

% \macro{\childdocredirect}
% The deprecated macro |\childdocredirect| is a legacy version
% of |\childdocforward| and |\childdocforwardprefix|:
%    \begin{macrocode}
\newcommand{\childdocredirect}[2][]
{
  \begingroup
    \if?#1?
      \def\childdoctmp{\childdocforward{#2}}
    \else
      \def\childdoctmp{\childdocforwardprefix{#1}{#2}}
    \fi
    \expandafter
  \endgroup
  \childdoctmp
}
%    \end{macrocode}

%\iffalse
%</package>
%\fi
%
\endinput
\childdocforward{cdocsamp}"|\\
% |latex -jobname cdocscl1 \|\\
% |  "% \iffalse
%
% childdoc.dtx Copyright (C) 2017-2018 Niklas Beisert
%
% This work may be distributed and/or modified under the
% conditions of the LaTeX Project Public License, either version 1.3
% of this license or (at your option) any later version.
% The latest version of this license is in
%   http://www.latex-project.org/lppl.txt
% and version 1.3 or later is part of all distributions of LaTeX
% version 2005/12/01 or later.
%
% This work has the LPPL maintenance status `maintained'.
%
% The Current Maintainer of this work is Niklas Beisert.
%
% This work consists of the files childdoc.dtx and childdoc.ins
% and the derived files childdoc.def and cdocsamp.tex with
% cdocsch1.tex, cdocsch2.tex, cdocsdrf.tex, cdocsfn1.tex, cdocsfn2.tex.
%
%<package>\ifdefined\childdocmain\endinput\fi
%<package>\ProvidesFile{childdoc.def}[2018/12/30 v2.0 child document driver]
%<samplemain>\ProvidesFile{cdocsamp.tex}[2018/12/30 v2.0 sample for childdoc]
%<*driver>
%\ProvidesFile{childdoc.drv}[2018/12/30 v2.0 childdoc reference manual file]
\PassOptionsToClass{10pt,a4paper}{article}
\documentclass{ltxdoc}

\usepackage[margin=35mm]{geometry}
\usepackage{hyperref}
\usepackage{hyperxmp}
\usepackage[usenames]{color}

\hypersetup{colorlinks=true}
\hypersetup{pdfstartview=FitH}
\hypersetup{pdfpagemode=UseNone}
\hypersetup{pdfsource={}}
\hypersetup{pdflang={en-UK}}
\hypersetup{pdfcopyright={Copyright 2017-2018 Niklas Beisert.
  This work may be distributed and/or modified under the
  conditions of the LaTeX Project Public License, either version 1.3
  of this license or (at your option) any later version.}}
\hypersetup{pdflicenseurl={http://www.latex-project.org/lppl.txt}}
\hypersetup{pdfcontactaddress={ETH Zurich, ITP, HIT K,
  Wolfgang-Pauli-Strasse 27}}
\hypersetup{pdfcontactpostcode={8093}}
\hypersetup{pdfcontactcity={Zurich}}
\hypersetup{pdfcontactcountry={Switzerland}}
\hypersetup{pdfcontactemail={nbeisert@itp.phys.ethz.ch}}
\hypersetup{pdfcontacturl={http://people.phys.ethz.ch/\xmptilde nbeisert/}}

\newcommand{\secref}[1]{\hyperref[#1]{section \ref*{#1}}}

\parskip1ex
\parindent0pt
\let\olditemize\itemize
\def\itemize{\olditemize\parskip0pt}

\begin{document}

\title{The \textsf{childdoc} Package}
\hypersetup{pdftitle={The childdoc Package}}
\author{Niklas Beisert\\[2ex]
  Institut f\"ur Theoretische Physik\\
  Eidgen\"ossische Technische Hochschule Z\"urich\\
  Wolfgang-Pauli-Strasse 27, 8093 Z\"urich, Switzerland\\[1ex]
  \href{mailto:nbeisert@itp.phys.ethz.ch}
  {\texttt{nbeisert@itp.phys.ethz.ch}}}
\hypersetup{pdfauthor={Niklas Beisert}}
\hypersetup{pdfsubject={Manual for the LaTeX2e Package childdoc}}
\date{30 December 2018, \textsf{v2.0}}
\maketitle

\begin{abstract}\noindent
\textsf{childdoc} is a \LaTeXe{} package
that enables the direct compilation
of document sections included by |\include|
to individual files.
\end{abstract}

\begingroup
\parskip0ex
\tableofcontents
\endgroup

%%%%%%%%%%%%%%%%%%%%%%%%%%%%%%%%%%%%%%%%%%%%%%%%%%%%%%%%%%%%%%%%%%%%%%%%%%%%%%%%
%%%%%%%%%%%%%%%%%%%%%%%%%%%%%%%%%%%%%%%%%%%%%%%%%%%%%%%%%%%%%%%%%%%%%%%%%%%%%%%%
\section{Introduction}

\LaTeX{} provides a mechanism to structure a large document (such as a book)
into a main file and several child files (containing the chapters)
using the |\include| command.
This mechanism is beneficial for documents
which span hundreds of pages in order to
make the source file(s) more manageable.
Moreover, compilation can be restricted to
selected child files by means of the |\includeonly| command.
The latter feature can be used to reduce the compilation time while editing
(this was significantly more useful in the earlier days of \LaTeX{})
or to generate a smaller document which is easier to navigate.
Another application of |\includeonly| is to generate
documents consisting of selected parts of the complete document.

However, there are a few drawbacks of the plain |\include| mechanism:
\begin{itemize}
\item
The child files cannot be compiled on their own,
they can only be compiled via the main file.
A naive editing environment
(such as a text editor with an option
to have the current file processed by \LaTeX)
may require one to switch to the main file before compiling;
attempting to compile the child file produces errors.
\item
The main file must be modified (each time)
to adjust the |\includeonly| command
to the present needs. This easily leaves the main file in a messy state.
\item
The generated document will always carry the filename
of the main document. This is inconvenient if
several child files are to be compiled and
to be kept for distribution.
\end{itemize}

The present package provides a simple interface
to make child files individually compilable by \LaTeX{}.
Compiling a child file then has the same effect as compiling
the main file with an |\includeonly| command
to select the appropriate child.
Moreover the generated document will carry the name of the child
rather than the main file.
This resolves all three above issues.

This feature is meant to make the editing of books,
thesis documents and lecture notes somewhat more convenient.
However, the package can also be used efficiently for
composing a series of documents (such as exercise sheets)
which are typically distributed individually.
It then assists the author in generating the individual documents
(potentially in different versions)
as well as a document containing the collected series.
Another application is in developing style files
or other kinds of included material
where compilation of the style file could redirect
to a sample or test file.

%%%%%%%%%%%%%%%%%%%%%%%%%%%%%%%%%%%%%%%%%%%%%%%%%%%%%%%%%%%%%%%%%%%%%%%%%%%%%%%%
%%%%%%%%%%%%%%%%%%%%%%%%%%%%%%%%%%%%%%%%%%%%%%%%%%%%%%%%%%%%%%%%%%%%%%%%%%%%%%%%
\section{Usage}

First of all, the package \textsf{childdoc} is \emph{not} a standard
\LaTeXe{} |.sty| style file! Therefore it needs to be invoked in
a non-standard way.

%%%%%%%%%%%%%%%%%%%%%%%%%%%%%%%%%%%%%%%%%%%%%%%%%%%%%%%%%%%%%%%%%%%%%%%%%%%%%%%%
\subsection{Included Files}
\label{sec:include}

%%%%%%%%%%%%%%%%%%%%%%%%%%%%%%%%%%%%%%%%
\DescribeMacro{\childdocmain}
To use the package, add the commands
\begin{center}
\begin{tabular}{l}
|\input{childdoc.def}|\\
|\childdocmain{}|\\
\end{tabular}
\end{center}
at the very top of the main \LaTeX{} file,
in particular \emph{before} the |\documentclass| statement!
The argument of |\childdocmain| should be left empty
(but it must be present).

%%%%%%%%%%%%%%%%%%%%%%%%%%%%%%%%%%%%%%%%
\DescribeMacro{\childdocof}
Furthermore, add the commands
\begin{center}
\begin{tabular}{l}
|\input{childdoc.def}|\\
|\childdocof{|\textit{main}|}|\\
\end{tabular}
\end{center}
at the top of every child file \textit{child}
which is included by |\include{|\textit{child}|}|
from within the main file
(or at least for those files to be compiled individually).
The argument \textit{main} must be the filename of the main file.

There are a couple of
considerations in setting up the main and child documents:

%%%%%%%%%%%%%%%%%%%%%%%%%%%%%%%%%%%%%%%%
\paragraph{Restrictions.}

Please note the following restrictions:
\begin{itemize}
\item
|\childdocmain| must be called with one argument \textit{main}
to ensure compatibility with earlier version of the package.
It must either be empty (|\childdocmain{}|)
or precisely match the filename of the main file in which it is specified.
See \secref{sec:detection} for further information.
\item
The filename \textit{main} must be specified without the |.tex| extension.
\item
The filename \textit{main} is case sensitive
(even in case-insensitive file systems)
due to internal string comparison.
\item
The argument \textit{main} should be fully expanded, it cannot be a macro.
\item
Subdirectories and special characters should be avoided in filenames.
\item
The command |\childdocmain{|\textit{main}|}| must be followed by a whitespace.
It should not be followed immediately by another command
or by a comment mark `|%|'.
This is because the \TeX{} parser reads the token immediately following
the argument of |\childdocmain| and puts it
at the beginning of every child section;
however, a white\-space is ignored.
\end{itemize}

%%%%%%%%%%%%%%%%%%%%%%%%%%%%%%%%%%%%%%%%
\paragraph{Content of Main File.}

It is advisable to place all content in the child files included by |\include|.
Any output contained in the main file will appear in all child documents
unless suppressed manually;
it cannot be suppressed automatically by the |\includeonly| directive
and thus should normally be avoided.
A method to include some content in the main file
by means of conditional processing is described in \secref{sec:conditional}.

%%%%%%%%%%%%%%%%%%%%%%%%%%%%%%%%%%%%%%%%
\paragraph{Page Numbering.}

When only a part of the document is compiled,
the appropriate numbering of pages
(as well as other status parameters)
is determined from the |.aux| files.
The latter contain information from previous passes.
However this information needs to propagate through
all intermediate child documents.
Therefore the page numbering in child documents may well
be inconsistent until the complete document is compiled at least once.

A useful (if unconventional) way to always ensure a consistent
page numbering is to restart the numbering in each child document
and denote the pages by `\textit{child}|.|\textit{page}'
where \textit{child} represents the chapter/section number of the child file.
This can be achieved by the command
|\numberwithin{page}{|\textit{child}|}|
of the \textsf{amsmath} package
where \textit{child} can be |chapter| or |section|
depending on the chosen structuring.
Alternatively, one can modify the macro |\thepage| appropriately
and reset the counter |page| at the start of each child file.

%%%%%%%%%%%%%%%%%%%%%%%%%%%%%%%%%%%%%%%%%%%%%%%%%%%%%%%%%%%%%%%%%%%%%%%%%%%%%%%%
\subsection{Conditional Processing}
\label{sec:conditional}

The package provides a mechanism to compile different versions
of a document. To customise the versions further some conditional processing
can come in handy to distinguish which version is being compiled.
The package provides two macros to describe the compilation context:

%%%%%%%%%%%%%%%%%%%%%%%%%%%%%%%%%%%%%%%%
\DescribeMacro{\ifchilddoc}
The conditional |\ifchilddoc| distinguishes between the compilation of
child documents and the main document:
%
\begin{center}
|\ifchilddoc |\textit{child-code}| |[|\||else |\textit{main-code}]| \||fi|
\end{center}

%%%%%%%%%%%%%%%%%%%%%%%%%%%%%%%%%%%%%%%%
\DescribeMacro{\childdocname}
\DescribeMacro{\childdocjob}
The macro |\childdocname| contains the filename (without extension)
of the main or child file being processed.
Note that |\childdocjob| will always contain the name of the main file.

%%%%%%%%%%%%%%%%%%%%%%%%%%%%%%%%%%%%%%%%
\paragraph{Title Page.}

Conditional processing can be used to include a title or banner page
in the main document when proper precautions are taken.
Importantly, the code in the main file should ensure that the page counter
(as well as other status parameters which are stored in the |.aux| files)
takes the same value after the conditional processing.
Otherwise the page numbers may take divergent values
depending on which part is compiled.

For example, a title page could be declared by:
%
\begin{center}
\begin{tabular}{l}
|\ifchilddoc\||else|\\
|\addtocounter{page}{-1}|\\
\textit{code for title page}\\
|\newpage|\\
|\||fi|
\end{tabular}
\end{center}
%
A banner page for the child documents can be generated by:
%
\begin{center}
\begin{tabular}{l}
|\ifchilddoc|\\
|\addtocounter{page}{-1}|\\
\textit{code for banner page}\\
|\newpage|\\
|\||fi|
\end{tabular}
\end{center}
%
Here one could write a message such as:
\begin{center}
|This is the part \childdocname{} of \childdocjob{}.|
\end{center}

%%%%%%%%%%%%%%%%%%%%%%%%%%%%%%%%%%%%%%%%%%%%%%%%%%%%%%%%%%%%%%%%%%%%%%%%%%%%%%%%
\subsection{Flags}
\label{sec:flags}

The package makes it easy to generate different versions
of the main or child documents.
To this end compilation flags can be defined
and assigned different default values.
They will be particularly useful in conjunction
with the forwarding mechanism described in \secref{sec:forward}.

For example, it may be useful to have a flag |\version|
which can be set to |draft| or |final|.
The document source will contain some conditional code
depending on the value of |\version|.
Suppose further, the flag should default to |final| for the main file
and to |draft| for child files
which is a natural assignment for editing the document.
This is achieved by placing the following code
in the preamble of the main document
(below the |\childdocmain| directive):
%
\begin{center}
\begin{tabular}{l}
|\ifchilddoc|\\
|\providecommand{\version}{draft}|\\
|\||else|\\
|\providecommand{\version}{final}|\\
|\||fi|
\end{tabular}
\end{center}
%
The definition by |\providecommand| makes sure
that previous definitions are not overwritten.
Further statements |\providecommand{\version}{...}|
can thus be added before the above code to override it.

For the main file, one might add a line
(between |\childdocmain| and the above block)
%
\begin{center}
|%\ifchilddoc\||else\providecommand{\version}{draft}\||fi|
\end{center}
%
which can be uncommented to produce a draft version.
Likewise one can add a line to the very top of a child file
(above the |\childdocof{|\textit{main}|}| directive)
%
\begin{center}
|%\providecommand{\version}{final}|
\end{center}
%
which can be uncommented to produce the final version of this child document.

%%%%%%%%%%%%%%%%%%%%%%%%%%%%%%%%%%%%%%%%%%%%%%%%%%%%%%%%%%%%%%%%%%%%%%%%%%%%%%%%
\subsection{Forwarding}
\label{sec:forward}

Different versions of the main or child documents
using compilation flags as described in \secref{sec:flags}
can be (permanently) stored in different files
for convenient compilation, viewing and distribution.
To this end, the package defines a command
to pass on compilation to a different file:

%%%%%%%%%%%%%%%%%%%%%%%%%%%%%%%%%%%%%%%%
\DescribeMacro{\childdocforward}
The command |\childdocforward| redirects processing to
another source file:
%
\begin{center}
\begin{tabular}{l}
|\input{childdoc.def}|\\
|\childdocforward[|\textit{main}|]{|\textit{dest}|}|\\
\end{tabular}
\end{center}
%
The argument \textit{dest} is the destination file
(without extension).
It should be the main file or one of the child files.
Note that further \textsf{childdoc} directives
such as |\childdocof| and |\childdocforward|
in the indicated file will be processed in this form.
The optional argument \textit{main}
passes on directly to the main file \textit{main}
while pretending to compile the child \textit{dest}.
This form behaves as if \textit{dest}
issues |\childdocof{|\textit{main}|}| right away,
and no further \textsf{childdoc} directives will be processed.

%%%%%%%%%%%%%%%%%%%%%%%%%%%%%%%%%%%%%%%%
\DescribeMacro{\...prefix}
In the alternative form |\childdocforwardprefix|,
%
\begin{center}
\begin{tabular}{l}
|\input{childdoc.def}|\\
|\childdocforwardprefix[|\textit{main}|]{|\textit{prefix}|}{|\textit{dest}|}|
\end{tabular}
\end{center}
%
the destination file is determined by a pattern
depending on the current file:
To make this work, the current file must be called
`{\textit{prefix}\hspace{0.2em}\textit{suffix}}'
with \textit{prefix} matching precisely the argument.
Processing is then passed on to the file
`{\textit{dest}\hspace{0.2em}\textit{suffix}}'.
Surely, the same effect is achieved by
directly specifying the
argument `{\textit{dest}\hspace{0.2em}\textit{suffix}}'
in the first form.
However, that requires to set up a different file
for each child. With the alternative form of the command
all these files can have exactly the same content
which simplifies setting them up and maintaining them.

For example, the following file |draft.tex|
with a compilation flag |\version| as described in \secref{sec:flags}
compiles the main document as a draft:
%
\begin{center}
\begin{tabular}{l}
|\def\version{draft}|\\
|\input{childdoc.def}|\\
|\childdocforward{|\textit{main}|}|
\end{tabular}
\end{center}
%
Likewise, the following files |final|\textit{nn}|.tex|
compile the final version of the child document
|child|\textit{nn}|.tex|:
%
\begin{center}
\begin{tabular}{l}
|\def\version{final}|\\
|\input{childdoc.def}|\\
|\childdocforwardprefix{final}{child}|
\end{tabular}
\end{center}
%

Note that when several versions of a main file and/or of each child file
are to be generated, it may be convenient to set up a |Makefile| or
shell script to automatise the process.

%%%%%%%%%%%%%%%%%%%%%%%%%%%%%%%%%%%%%%%%%%%%%%%%%%%%%%%%%%%%%%%%%%%%%%%%%%%%%%%%
\subsection{Command Line Processing}
\label{sec:commandline}

The effect of redirection files can also be achieved by invoking
the \LaTeX{} compiler with a more elaborate command line.
Most conveniently this should be done as part
of a shell script or a |Makefile|.

When using \textsf{childdoc} in the main file, the following
command lines effectively perform a redirection
(note that depending on the shell being used,
backslashes may have to be doubled: `|\|' $\to$ `|\\|'):
%
\begin{center}
|... -jobname "|\textit{target}|" |\\|"|[\textit{flags}]%
|\input{childdoc.def}\childdocforward[|\textit{main}|]{|\textit{dest}|}"|
\end{center}
%
Here \textit{target} is the name of the output file,
\textit{main} is the name of the main file
and \textit{dest} is the name of the main or child file to be processed
(all filenames without extensions).
The optional argument \textit{main} can be omitted
if \textit{main} matches \textit{dest}.
Optionally, compilation \textit{flags} can be defined via |\def| commands.
This command line makes the \TeX{} engine believe
it is compiling the file \textit{target}
whose content is specified as the latter parameter.
The provided code then forwards the processing to
\textit{main} or \textit{dest} as described in \secref{sec:forward}.

%%%%%%%%%%%%%%%%%%%%%%%%%%%%%%%%%%%%%%%%%%%%%%%%%%%%%%%%%%%%%%%%%%%%%%%%%%%%%%%%
\subsection{Include by Input}
\label{sec:input}

Including child documents by |\include| has some restrictions by design.
Most notably, the content of a child document always occupies
its own set of pages; pages cannot be shared between child documents.
Usually, this behaviour makes perfect sense
because each child document contain an essential part of the document.
However, in some situations it may be desirable to compose
a document from a collection of parts
without having mandatory page breaks between then.
For this case, the package
provides a mechanism to include parts
by |\input| which can also be processed individually.
However, by construction this mechanism
requires manual handling of the content to be output.

%%%%%%%%%%%%%%%%%%%%%%%%%%%%%%%%%%%%%%%%
\DescribeMacro{\ifchilddocmanual}
The main file should be prepared as usual, see \secref{sec:include}.
However, the document body must make a distinction
between processing of an individual part and of the main document, e.g.:
%
\begin{center}
\begin{tabular}{l}
|\ifchilddocmanual|\\
|\input{\childdocname}|\\
|\||else|\\
\textit{document body with }|\input{|\textit{part}|}|\\
|\||fi|
\end{tabular}
\end{center}
%
The conditional |\ifchilddocmanual| is true whenever
a part to be included by |\input| is being compiled,
and the name of the part is stored in |\childdocname|.

%%%%%%%%%%%%%%%%%%%%%%%%%%%%%%%%%%%%%%%%
\DescribeMacro{\childdocby}
Each part to be included by |\input| should start with:
%
\begin{center}
\begin{tabular}{l}
|\input{childdoc.def}|\\
|\childdocby{|\textit{main}|}|\\
\end{tabular}
\end{center}
%
The directive |\childdocby| is similar to |\childdocof|
described in \secref{sec:include},
but the subsequent selection of content must be done manually.
To that end, both |\ifchilddoc| and |\ifchilddocmanual|
will be true upon processing of a part,
and the name of the part is stored in |\childdocname|.
Note that |\jobname| will be set to the filename of the current part
so that each part receives an individual |.aux| file
that does not interfere with the |.aux| file(s) of the main document.
This behaviour can be altered by the alternative form
|\childdocby[*]{|\textit{main}|}| (with a non-empty optional argument)
which uses the |.aux| file of the main document
by setting |\jobname| to \textit{main}.

%%%%%%%%%%%%%%%%%%%%%%%%%%%%%%%%%%%%%%%%%%%%%%%%%%%%%%%%%%%%%%%%%%%%%%%%%%%%%%%%
\subsection{Driver Development}
\label{sec:driver}

The \textsf{childdoc} mechanism can also be use for the development
of definition files such as \LaTeX{} styles or classes.
This case differs from the above setup with multiple parts
included by |\include| in that no |\includeonly| should be invoked.
This can be achieved by starting the include file
(before |\ProvidesPackage|) with:
%
\begin{center}
\begin{tabular}{l}
|\input{childdoc.def}|\\
|\childdocforward{|\textit{main}|}|\\
\end{tabular}
\end{center}
%
or alternatively with:
%
\begin{center}
\begin{tabular}{l}
|\input{childdoc.def}|\\
|\childdocby{|\textit{main}|}|\\
\end{tabular}
\end{center}
%
Both forms have slightly different effects as described above.
The main file is prepared as usual, see \secref{sec:include}.

%%%%%%%%%%%%%%%%%%%%%%%%%%%%%%%%%%%%%%%%%%%%%%%%%%%%%%%%%%%%%%%%%%%%%%%%%%%%%%%%
\subsection{Legacy Detection}
\label{sec:detection}

The directive |\childdocmain| in the main file can detect
whether the complete document or merely a child is to be compiled
even without using the directive |\childdocof|.
This method is deprecated because it is less robust
and there is no compelling reason to use it;
it is merely provided for backward compatibility
and it may be removed in future versions.

If the detection mechanism is to be used,
it is mandatory to correctly specify
the filename of the main file as the argument of |\childdocmain|:
%
\begin{center}
\begin{tabular}{l}
|\input{childdoc.def}|\\
|\childdocmain{|\textit{main}|}|\\
\end{tabular}
\end{center}
%
If |\jobname| does not match the argument \textit{main} of |\childdocmain|,
it is assumed that |\jobname| points to the child file to be compiled.
When using |\childdocmain| with the main file specified as argument,
it suffices to start a child file
with just |\input{|\textit{main}|}|
without loading of the package and using |\childdocof|.
If instead all processing is done
with the appropriate \textsf{childdoc} directives,
the argument of \textit{main} of |\childdocmain| can be empty.

An alternative version of the command line processing described
in \secref{sec:commandline} using the detection mechanism reads:
%
\begin{center}
|... -jobname "|\textit{target}|" "|[\textit{flags}]%
[|\def\jobname{|\textit{dest}|}|]|\input{|\textit{main}|}"|
\end{center}

%%%%%%%%%%%%%%%%%%%%%%%%%%%%%%%%%%%%%%%%%%%%%%%%%%%%%%%%%%%%%%%%%%%%%%%%%%%%%%%%
\subsection{Manual Code}
\label{sec:manual}

In case one cannot be certain whether the definitions file |childdoc.def|
is installed on the target \TeX{} distribution
and one prefers not to ship it,
it is conceivable to paste a few relevant commands into the sources.

To that end, drop all statements |\input{childdoc.def}|
and perform the replacements as outlined below.
Instead of |\childdocmain{|\textit{main}|}| add the following code
to the top of the main file:
%
\begin{center}
\begin{tabular}{l}
|\||ifdefined\childdocname\endinput\||fi\newif\ifchilddoc|\\
|\edef\childdocname{\scantokens\expandafter{\jobname\noexpand}}|\\
|\def\childdocmain{|\textit{main}|}\||ifx\childdocmain\childdocname\||else|\\
|\childdoctrue\includeonly{\childdocname}\let\jobname\childdocmain\||fi|\\
\end{tabular}
\end{center}
%
Instead of |\childdocof{|\textit{main}|}| just include the main file
at the top of each child file:
%
\begin{center}
|\input{|\textit{main}|}|
\end{center}
%
A simple redirection |\childdocforward{|\textit{dest}|}| is achieved by:
%
\begin{center}
|\def\jobname{|\textit{dest}|}\input{\jobname}|
\end{center}
%
The redirection with prefix
|\childdocforwardprefix[|\textit{prefix}|]{|\textit{dest}|}|
is accomplished by:
%
\begin{center}
\begin{tabular}{l}
|{\edef\jobname{\scantokens\expandafter{\jobname\noexpand}}|\\
|\def\redirectjob |\textit{prefix}|#1~~~{\gdef\jobname{|\textit{dest}|#1}}|\\
|\expandafter\redirectjob\jobname~~~}\input{\jobname}|
\end{tabular}
\end{center}

In an alternative approach,
child documents can be compiled by a specific command line
without additional code or specific definitions:
%
\begin{center}
|... -jobname "|\textit{target}|" "|[\textit{flags}]%
|\includeonly{|\textit{dest}|}\input{|\textit{main}|}"|
\end{center}
%

%%%%%%%%%%%%%%%%%%%%%%%%%%%%%%%%%%%%%%%%%%%%%%%%%%%%%%%%%%%%%%%%%%%%%%%%%%%%%%%%
%%%%%%%%%%%%%%%%%%%%%%%%%%%%%%%%%%%%%%%%%%%%%%%%%%%%%%%%%%%%%%%%%%%%%%%%%%%%%%%%
\section{Information}

%%%%%%%%%%%%%%%%%%%%%%%%%%%%%%%%%%%%%%%%%%%%%%%%%%%%%%%%%%%%%%%%%%%%%%%%%%%%%%%%
\subsection{Copyright}

Copyright \copyright{} 2017--2018 Niklas Beisert

This work may be distributed and/or modified under the
conditions of the \LaTeX{} Project Public License, either version 1.3
of this license or (at your option) any later version.
The latest version of this license is in
  \url{http://www.latex-project.org/lppl.txt}
and version 1.3 or later is part of all distributions of \LaTeX{}
version 2005/12/01 or later.

This work has the LPPL maintenance status `maintained'.

The Current Maintainer of this work is Niklas Beisert.

This work consists of the files |README.txt|, |childdoc.ins| and |childdoc.dtx|
as well as the derived files |childdoc.def|, |cdocsamp.tex|
with |cdocsch1.tex|, |cdocsch2.tex|, |cdocspt3.tex|, |cdocspt4.tex|,
|cdocsdrf.tex|, |cdocsfn1.tex|, |cdocsfn2.tex|
as well as |childdoc.pdf|.

%%%%%%%%%%%%%%%%%%%%%%%%%%%%%%%%%%%%%%%%%%%%%%%%%%%%%%%%%%%%%%%%%%%%%%%%%%%%%%%%
\subsection{Files and Installation}

The package consists of the files:
%
\begin{center}
\begin{tabular}{ll}
    |README.txt|   & readme file \\
    |childdoc.ins| & installation file \\
    |childdoc.dtx| & source file \\
    |childdoc.def| & definition file \\
    |cdocsamp.tex| & sample main file \\
    |cdocsch1.tex| & sample include file \\
    |cdocsch2.tex| & sample include file \\
    |cdocspt3.tex| & sample part file \\
    |cdocspt4.tex| & sample part file \\
    |cdocsdrf.tex| & sample redirection file \\
    |cdocsfn1.tex| & sample redirection file \\
    |cdocsfn2.tex| & sample redirection file \\
    |childdoc.pdf| & manual
\end{tabular}
\end{center}
%
The distribution consists of the files
|README.txt|, |childdoc.ins| and |childdoc.dtx|.
%
\begin{itemize}
\item
Run (pdf)\LaTeX{} on |childdoc.dtx|
to compile the manual |childdoc.pdf| (this file).
\item
Run \LaTeX{} on |childdoc.ins| to create the definitions file |childdoc.def|
and the sample |cdocsamp.tex| with include files
|cdocsch1.tex|, |cdocsch2.tex|, |cdocspt3.tex|, |cdocspt4.tex|,
|cdocsdrf.tex|, |cdocsfn1.tex|, |cdocsfn2.tex|.
Then copy the file |childdoc.def| to an appropriate directory of your \LaTeX{}
distribution, e.g.\ \textit{texmf-root}|/tex/latex/childdoc|.
\end{itemize}

%%%%%%%%%%%%%%%%%%%%%%%%%%%%%%%%%%%%%%%%%%%%%%%%%%%%%%%%%%%%%%%%%%%%%%%%%%%%%%%%
\subsection{Related CTAN Packages}

There are several other packages which offer a similar functionality:
%
\begin{itemize}
\item
The packages
\href{http://ctan.org/pkg/docmute}{\textsf{docmute}},
\href{http://ctan.org/pkg/includex}{\textsf{includex}} and
\href{http://ctan.org/pkg/standalone}{\textsf{standalone}}
provide commands to include only the document body of
a child file thus allowing both files to be compiled individually.
\item
The packages \href{http://ctan.org/pkg/subdocs}{\textsf{subdocs}}
and \href{http://ctan.org/pkg/subfiles}{\textsf{subfiles}}
provide structures in which the main and child documents can be
encapsulated and allowing them to be compiled individually.
The inclusion mechanism is different from the conventional |\include|.
\item
The package \href{http://ctan.org/pkg/combine}{\textsf{combine}}
is an elaborate solution to combine several documents into one.
\end{itemize}
%
See also the CTAN topic \href{http://ctan.org/topic/subdocs}{\textsf{subdocs}}
for further related packages.
The present package differs from the above solutions in that
a document structure constructed with the conventional |\include| mechanism
just needs two extra commands at the top of every file
such that all constituent files can be compiled individually.

%%%%%%%%%%%%%%%%%%%%%%%%%%%%%%%%%%%%%%%%%%%%%%%%%%%%%%%%%%%%%%%%%%%%%%%%%%%%%%%%
%\subsection{Feature Suggestions}
%
%The following is a list of features which may be useful for future
%versions of this package:
%%
%\begin{itemize}
%\item
%\ldots
%\end{itemize}

%%%%%%%%%%%%%%%%%%%%%%%%%%%%%%%%%%%%%%%%%%%%%%%%%%%%%%%%%%%%%%%%%%%%%%%%%%%%%%%%
\subsection{Revision History}

%%%%%%%%%%%%%%%%%%%%%%%%%%%%%%%%%%%%%%%%
\paragraph{v2.0:} 2018/12/30

\begin{itemize}
\item
immediate forward processing
\item
added |\childdocby| mechanism
\item
manual restructured
\end{itemize}

%%%%%%%%%%%%%%%%%%%%%%%%%%%%%%%%%%%%%%%%
\paragraph{v1.6:} 2018/01/17

\begin{itemize}
\item
application for development of include files
\item
corrections to manual
\end{itemize}

%%%%%%%%%%%%%%%%%%%%%%%%%%%%%%%%%%%%%%%%
\paragraph{v1.5:} 2017/05/21

\begin{itemize}
\item
more complete structuring introduced
\item
|\childdocof| introduced
\item
|\childdoc| renamed to |\childdocmain|
\item
|\childredirect| renamed to |\childdocforward| and |\childdocforwardprefix|
and functionality expanded
\end{itemize}

%%%%%%%%%%%%%%%%%%%%%%%%%%%%%%%%%%%%%%%%
\paragraph{v1.0:} 2017/04/27

\begin{itemize}
\item
manual and install package
\item
first version published on CTAN
\end{itemize}

%%%%%%%%%%%%%%%%%%%%%%%%%%%%%%%%%%%%%%%%
\paragraph{v0.6:} 2017/04/26

\begin{itemize}
\item
redirection mechanism added
\end{itemize}

%%%%%%%%%%%%%%%%%%%%%%%%%%%%%%%%%%%%%%%%
\paragraph{v0.5:} 2017/04/26

\begin{itemize}
\item
functionality in definition file
\end{itemize}


%%%%%%%%%%%%%%%%%%%%%%%%%%%%%%%%%%%%%%%%%%%%%%%%%%%%%%%%%%%%%%%%%%%%%%%%%%%%%%%%
%%%%%%%%%%%%%%%%%%%%%%%%%%%%%%%%%%%%%%%%%%%%%%%%%%%%%%%%%%%%%%%%%%%%%%%%%%%%%%%%
%%%%%%%%%%%%%%%%%%%%%%%%%%%%%%%%%%%%%%%%%%%%%%%%%%%%%%%%%%%%%%%%%%%%%%%%%%%%%%%%
\appendix

\settowidth\MacroIndent{\rmfamily\scriptsize 000\ }

 \DocInput{childdoc.dtx}

\end{document}
%</driver>
% \fi
%
% %%%%%%%%%%%%%%%%%%%%%%%%%%%%%%%%%%%%%%%%%%%%%%%%%%%%%%%%%%%%%%%%%%%%%%%%%%%%%%
% %%%%%%%%%%%%%%%%%%%%%%%%%%%%%%%%%%%%%%%%%%%%%%%%%%%%%%%%%%%%%%%%%%%%%%%%%%%%%%
% \section{Sample}
%\iffalse
%<*samplemain>
%\fi
%
% The following presents a sample document
% with two chapters, two parts, a title page,
% a compile flag as well as three forwarding files to set the flag.
% It consists of eight |.tex| files:
% \begin{center}
% \begin{tabular}{ll}
% |cdocsamp.tex|&main file\\
% |cdocsch1.tex|&include file for chapter 1\\
% |cdocsch2.tex|&include file for chapter 2\\
% |cdocspt3.tex|&include file for part 3\\
% |cdocspt4.tex|&include file for part 4\\
% |cdocsdrf.tex|&forwarding file for main file in draft mode\\
% |cdocsfi1.tex|&forwarding file for final version of chapter 1\\
% |cdocsfi2.tex|&forwarding file for final version of chapter 2\\
% \end{tabular}
% \end{center}
% Each of the eight files can be compiled directly by the \LaTeX{} compiler.
%
% %%%%%%%%%%%%%%%%%%%%%%%%%%%%%%%%%%%%%%
% \paragraph{Main File.}
%
% The main file is called |cdocsamp.tex|.
%
% Load the \textsf{childdoc} definitions and
% declare the filename for the main document:
%    \begin{macrocode}
\input{childdoc.def}
\childdocmain{}
%    \end{macrocode}

% Optional override for |\version| flag:
%    \begin{macrocode}
%%\ifchilddoc\else\providecommand{\version}{draft}\fi
%    \end{macrocode}

% Define the default values for the |\version| flag
% (|final| for the main file and |draft| for childs):
%    \begin{macrocode}
\ifchilddoc
\providecommand{\version}{draft}
\else
\providecommand{\version}{final}
\fi
%    \end{macrocode}

% Load the standard document class:
%    \begin{macrocode}
\documentclass[12pt]{article}
%    \end{macrocode}

% Start the document body:
%    \begin{macrocode}
\begin{document}
%    \end{macrocode}

% Declare a title page.
% Print title, part of document being processed and version flag:
%    \begin{macrocode}
\addtocounter{page}{-1}
\begin{center}
{\LARGE\bfseries{}childdoc example\par}
\vspace{1cm}
\ifchilddoc
\ifchilddocmanual part\else chapter\fi:
`\childdocname' of `\childdocjob'\par
\else
main document: `\childdocjob'\par
\fi
version: \version\par
\end{center}
\newpage
%    \end{macrocode}

% Manually include selected file,
% otherwise process as usual:
%    \begin{macrocode}
\ifchilddocmanual
\section*{part `\childdocname'}
\input{\childdocname}
\else
%    \end{macrocode}

% Include the two chapters:
%    \begin{macrocode}
\include{cdocsch1}
\include{cdocsch2}
%    \end{macrocode}

% Include the two parts unless only chapters should be displayed:
%    \begin{macrocode}
\ifchilddoc\else
\section{part three}
\input{cdocspt3}
\section{part four}
\input{cdocspt4}
\fi
%    \end{macrocode}

% Process as usual until here:
%    \begin{macrocode}
\fi
%    \end{macrocode}

% End of document body:
%    \begin{macrocode}
\end{document}
%    \end{macrocode}
%\iffalse
%</samplemain>
%\fi
%
% %%%%%%%%%%%%%%%%%%%%%%%%%%%%%%%%%%%%%%
% \paragraph{Chapter Include Files.}
%
% The include files are called |cdocsch1.tex| and |cdocsch2.tex|.
%
%\iffalse
%<*samplechap1|samplechap2>
%\fi

% Optional override for |\version| flag:
%    \begin{macrocode}
%%\providecommand{\version}{final}
%    \end{macrocode}

% Include the main document:
%    \begin{macrocode}
\input{childdoc.def}
\childdocof{cdocsamp}
%    \end{macrocode}

%\iffalse
%</samplechap1|samplechap2>
%\fi
%
%\iffalse
%<*samplechap1>
%\fi
% Some text for chapter 1:
%    \begin{macrocode}
\section{one}
some text in chapter one
%    \end{macrocode}

%\iffalse
%</samplechap1>
%\fi
% Some text for chapter 2:
%\iffalse
%<*samplechap2>
%\fi
%    \begin{macrocode}
\section{two}
more text in chapter two
%    \end{macrocode}

%\iffalse
%</samplechap2>
%\fi
%
% %%%%%%%%%%%%%%%%%%%%%%%%%%%%%%%%%%%%%%
% \paragraph{Part Include Files.}
%
% The include files are called |cdocspt3.tex| and |cdocspt4.tex|.
%
%\iffalse
%<*samplepart3|samplepart4>
%\fi

% Optional override for |\version| flag:
%    \begin{macrocode}
%%\providecommand{\version}{final}
%    \end{macrocode}

% Include the main document:
%    \begin{macrocode}
\input{childdoc.def}
\childdocby{cdocsamp}
%    \end{macrocode}

%\iffalse
%</samplepart3|samplepart4>
%\fi
%
%\iffalse
%<*samplepart3>
%\fi
% Some text for part 3:
%    \begin{macrocode}
some text in part three
%    \end{macrocode}

%\iffalse
%</samplepart3>
%\fi
% Some text for part 4:
%\iffalse
%<*samplepart4>
%\fi
%    \begin{macrocode}
more text in part four
%    \end{macrocode}

%\iffalse
%</samplepart4>
%\fi
%
% %%%%%%%%%%%%%%%%%%%%%%%%%%%%%%%%%%%%%%
% \paragraph{Forwarding for a Complete Draft.}
%
% The following forwarding file |cdocsdrf.tex|
% compiles the main document in draft mode:
%\iffalse
%<*sampledraft>
%\fi
%    \begin{macrocode}
\def\version{draft}
\input{childdoc.def}
\childdocforward{cdocsamp}
%    \end{macrocode}

%\iffalse
%</sampledraft>
%\fi
%
% %%%%%%%%%%%%%%%%%%%%%%%%%%%%%%%%%%%%%%
% \paragraph{Forwarding for Final Version of the Chapters.}
%
% The following forwarding files |cdocsfn1.tex| and |cdocsfn2.tex|
% (with identical content)
% compile the final versions of the child documents
% |cdocsch1.tex| and |cdocsch2.tex|, respectively:
%\iffalse
%<*samplefinal>
%\fi
%    \begin{macrocode}
\def\version{final}
\input{childdoc.def}
\childdocforwardprefix[cdocsamp]{cdocsfn}{cdocsch}
%    \end{macrocode}

%\iffalse
%</samplefinal>
%\fi
%
% %%%%%%%%%%%%%%%%%%%%%%%%%%%%%%%%%%%%%%
% \paragraph{Command Line Processing.}
%
% The following three command lines generate the output files
% |cdocscld|, |cdocscl1| and |cdocscl2|
% which should be identical to
% |cdocsdrf|, |cdocsch1| and |cdocsfn2|, respectively:
% \begin{center}
% \begin{tabular}{l}
% |latex -jobname cdocscld \|\\
% |  "\def\version{draft}\input{childdoc.def}\childdocforward{cdocsamp}"|\\
% |latex -jobname cdocscl1 \|\\
% |  "\input{childdoc.def}\childdocforward[cdocsamp]{cdocsch1}"|\\
% |latex -jobname cdocscl2 \|\\
% |  "\def\version{final}\input{childdoc.def}\childdocforward{cdocsch2}"|
% \end{tabular}
% \end{center}
% Note that the trailing backslash on each first line
% merely continues the input to the second line
% (for convenient cut ant paste).
% Furthermore, the command |latex| can be replaced by any
% of its alternative versions such as |pdflatex|.
%
% %%%%%%%%%%%%%%%%%%%%%%%%%%%%%%%%%%%%%%%%%%%%%%%%%%%%%%%%%%%%%%%%%%%%%%%%%%%%%%
% %%%%%%%%%%%%%%%%%%%%%%%%%%%%%%%%%%%%%%%%%%%%%%%%%%%%%%%%%%%%%%%%%%%%%%%%%%%%%%
% \section{Implementation}
%\iffalse
%<*package>
%\fi
%
% This section describes the definitions file |childdoc.def|.

% The definitions cannot be loaded using |\usepackage| or |\RequirePackage|
% which has a mechanism to prevent loading a style file more than once.
% When loading the definitions by means of |\input|
% multiple instances have to be prevented manually:
%\iffalse
%This code needs to be before the `\ProvidesFile' directive
%which is defined at the beginning of this file.
%Therefore it is also placed there and commented out here.
%</package>
%<*discard>
%\fi
%    \begin{macrocode}
\ifdefined\childdocmain\endinput\fi
%    \end{macrocode}
%\iffalse
%</discard>
%<*package>
%\fi
%
% \macro{\ifchilddoc}
% \macro{\ifchilddocmanual}
% The conditional |\ifchilddoc| tells whether a
% child (true) or main (false) document is being compiled.
% The conditional |\ifchilddocmanual| tells whether
% the |\includeonly| mechanism is used (false) or
% the selection of child files must be performed manually (true).
% The definitions initialise to false:
%    \begin{macrocode}
\newif\ifchilddoc
\newif\ifchilddocmanual
%    \end{macrocode}

% \macro{\childdocname}
% \macro{\childdocjob}
% The macro |\childdocname| stores the name of the main document
% to be compiled. The macro |\childdocjob| stores the name of
% the document on which the \LaTeX{} compiler was originally invoked.
% The content of |\jobname| cannot be compared
% to filenames specified in the source due to different catcodes.
% The following code rescans |\jobname|, stores the result
% in |\childdocname| and saves a copy in |\childdocjob|:
%    \begin{macrocode}
\edef\childdocname{\scantokens\expandafter{\jobname\noexpand}}
\let\childdocjob\childdocname
%    \end{macrocode}

% \macro{\childdocdisable}
% The macro |\childdocdisable| prevents the main file
% from being processed more than once.
% At this stage, the main document command |\childdocmain|
% is assumed to be called once again where it should do nothing.
% Any subsequent call to it should prevent
% a secondary processing of the main document
% It overwrites the forwarding commands
% |\childdocof| and |\childdocforward|
% with empty macros to prevent further inclusions of the main document:
%    \begin{macrocode}
\newcommand{\childdocdisable}
{
  \renewcommand{\childdocmain}[1]{\renewcommand{\childdocmain}[1]{\endinput}}
  \renewcommand{\childdocof}[1]{}
  \renewcommand{\childdocby}[2][]{}
  \renewcommand{\childdocforward}[2][]{}
  \renewcommand{\childdocdisable}{}
}
%    \end{macrocode}

% \macro{\childdocmain}
% The macro |\childdocmain| is to be called at the top of the main file
% with nothing or the main filename (without extension) as argument.
% First, it breaks loops.
% If the argument is not empty and does not match |\childdocname|
% (which is set by the first inclusion of |childdoc.def|),
% |\ifchilddoc| is set to true, |\includeonly| is applied to the child file
% and |\jobname| is set to the main file
% (for proper handling of |.aux| files):
%    \begin{macrocode}
\newcommand{\childdocmain}[1]
{
  \childdocdisable\childdocmain{}
  \if?#1?\else
    \begingroup
      \def\childdoctmp{#1}
      \ifx\childdoctmp\childdocname
        \def\childdoctmp{}
      \else
        \def\childdoctmp
        {
          \childdoctrue
          \includeonly{\childdocname}
          \def\childdocjob{#1}
          \def\jobname{#1}
        }
      \fi
      \expandafter
    \endgroup
    \childdoctmp
  \fi
}
%    \end{macrocode}

% \macro{\childdocof}
% The command |\childdocof| redirects
% compilation to the main file |#1|.
%    \begin{macrocode}
\newcommand{\childdocof}[1]
{
  \childdocdisable
  \childdoctrue
  \includeonly{\childdocname}
  \def\jobname{#1}
  \def\childdocjob{#1}
  \input{#1}
}
%    \end{macrocode}

% \macro{\childdocby}
% The command |\childdocby| ....
%    \begin{macrocode}
\newcommand{\childdocby}[2][]
{
  \childdocdisable
  \childdoctrue
  \childdocmanualtrue
  \if?#1?\else
    \def\jobname{#2}
  \fi
  \def\childdocjob{#2}
  \input{#2}
  \endinput
}
%    \end{macrocode}

% \macro{\childdocforward}
% The command |\childdocforward| redirects
% compilation to the main file or
% (if the optional argument is given) a child file.
% Parameters are set as if the main file
% or a child file starting with |\childdocof| was compiled.
% Then compilation is handed over to the main file:
%    \begin{macrocode}
\newcommand{\childdocforward}[2][]
{
  \begingroup
    \if?#1?
      \def\childdoctmp
      {
        \def\childdocname{#2}
        \def\childdocjob{#2}
        \def\jobname{#2}
        \input{#2}
        \endinput
      }
    \else
      \def\childdoctmp
      {
        \childdocdisable
        \def\childdocname{#2}
        \childdoctrue
        \includeonly{#2}
        \def\childdocjob{#1}
        \def\jobname{#1}
        \input{#1}
        \endinput
      }
    \fi
    \expandafter
  \endgroup
  \childdoctmp
}
%    \end{macrocode}

% \macro{\childdocforwardprefix}
% The command |\childdocforwardprefix| redirects
% compilation to the main or a child file by means of a pattern.
% The prefix |#1| in the current filename is replaced by |#2|
% and the suffix of the current filename is kept
% (it is assumed that the filename does not contain the substring `|~~~|'
% which is used as a delimiter).
% Compilation is handed over to the new file by |\childdocforward|:
%    \begin{macrocode}
\newcommand{\childdocforwardprefix}[3][]
{
  \begingroup
    \def\childdocextract #2##1~~~{\def\childdoctmp{\childdocforward[#1]{#3##1}}}
    \expandafter\childdocextract\childdocname~~~
    \expandafter
  \endgroup
  \childdoctmp
}
%    \end{macrocode}

% \macro{\childdoc}
% The deprecated macro |\childdoc| is a legacy version of |\childdocmain|:
%    \begin{macrocode}
\newcommand{\childdoc}{\childdocmain}
%    \end{macrocode}

% \macro{\childdocredirect}
% The deprecated macro |\childdocredirect| is a legacy version
% of |\childdocforward| and |\childdocforwardprefix|:
%    \begin{macrocode}
\newcommand{\childdocredirect}[2][]
{
  \begingroup
    \if?#1?
      \def\childdoctmp{\childdocforward{#2}}
    \else
      \def\childdoctmp{\childdocforwardprefix{#1}{#2}}
    \fi
    \expandafter
  \endgroup
  \childdoctmp
}
%    \end{macrocode}

%\iffalse
%</package>
%\fi
%
\endinput
\childdocforward[cdocsamp]{cdocsch1}"|\\
% |latex -jobname cdocscl2 \|\\
% |  "\def\version{final}% \iffalse
%
% childdoc.dtx Copyright (C) 2017-2018 Niklas Beisert
%
% This work may be distributed and/or modified under the
% conditions of the LaTeX Project Public License, either version 1.3
% of this license or (at your option) any later version.
% The latest version of this license is in
%   http://www.latex-project.org/lppl.txt
% and version 1.3 or later is part of all distributions of LaTeX
% version 2005/12/01 or later.
%
% This work has the LPPL maintenance status `maintained'.
%
% The Current Maintainer of this work is Niklas Beisert.
%
% This work consists of the files childdoc.dtx and childdoc.ins
% and the derived files childdoc.def and cdocsamp.tex with
% cdocsch1.tex, cdocsch2.tex, cdocsdrf.tex, cdocsfn1.tex, cdocsfn2.tex.
%
%<package>\ifdefined\childdocmain\endinput\fi
%<package>\ProvidesFile{childdoc.def}[2018/12/30 v2.0 child document driver]
%<samplemain>\ProvidesFile{cdocsamp.tex}[2018/12/30 v2.0 sample for childdoc]
%<*driver>
%\ProvidesFile{childdoc.drv}[2018/12/30 v2.0 childdoc reference manual file]
\PassOptionsToClass{10pt,a4paper}{article}
\documentclass{ltxdoc}

\usepackage[margin=35mm]{geometry}
\usepackage{hyperref}
\usepackage{hyperxmp}
\usepackage[usenames]{color}

\hypersetup{colorlinks=true}
\hypersetup{pdfstartview=FitH}
\hypersetup{pdfpagemode=UseNone}
\hypersetup{pdfsource={}}
\hypersetup{pdflang={en-UK}}
\hypersetup{pdfcopyright={Copyright 2017-2018 Niklas Beisert.
  This work may be distributed and/or modified under the
  conditions of the LaTeX Project Public License, either version 1.3
  of this license or (at your option) any later version.}}
\hypersetup{pdflicenseurl={http://www.latex-project.org/lppl.txt}}
\hypersetup{pdfcontactaddress={ETH Zurich, ITP, HIT K,
  Wolfgang-Pauli-Strasse 27}}
\hypersetup{pdfcontactpostcode={8093}}
\hypersetup{pdfcontactcity={Zurich}}
\hypersetup{pdfcontactcountry={Switzerland}}
\hypersetup{pdfcontactemail={nbeisert@itp.phys.ethz.ch}}
\hypersetup{pdfcontacturl={http://people.phys.ethz.ch/\xmptilde nbeisert/}}

\newcommand{\secref}[1]{\hyperref[#1]{section \ref*{#1}}}

\parskip1ex
\parindent0pt
\let\olditemize\itemize
\def\itemize{\olditemize\parskip0pt}

\begin{document}

\title{The \textsf{childdoc} Package}
\hypersetup{pdftitle={The childdoc Package}}
\author{Niklas Beisert\\[2ex]
  Institut f\"ur Theoretische Physik\\
  Eidgen\"ossische Technische Hochschule Z\"urich\\
  Wolfgang-Pauli-Strasse 27, 8093 Z\"urich, Switzerland\\[1ex]
  \href{mailto:nbeisert@itp.phys.ethz.ch}
  {\texttt{nbeisert@itp.phys.ethz.ch}}}
\hypersetup{pdfauthor={Niklas Beisert}}
\hypersetup{pdfsubject={Manual for the LaTeX2e Package childdoc}}
\date{30 December 2018, \textsf{v2.0}}
\maketitle

\begin{abstract}\noindent
\textsf{childdoc} is a \LaTeXe{} package
that enables the direct compilation
of document sections included by |\include|
to individual files.
\end{abstract}

\begingroup
\parskip0ex
\tableofcontents
\endgroup

%%%%%%%%%%%%%%%%%%%%%%%%%%%%%%%%%%%%%%%%%%%%%%%%%%%%%%%%%%%%%%%%%%%%%%%%%%%%%%%%
%%%%%%%%%%%%%%%%%%%%%%%%%%%%%%%%%%%%%%%%%%%%%%%%%%%%%%%%%%%%%%%%%%%%%%%%%%%%%%%%
\section{Introduction}

\LaTeX{} provides a mechanism to structure a large document (such as a book)
into a main file and several child files (containing the chapters)
using the |\include| command.
This mechanism is beneficial for documents
which span hundreds of pages in order to
make the source file(s) more manageable.
Moreover, compilation can be restricted to
selected child files by means of the |\includeonly| command.
The latter feature can be used to reduce the compilation time while editing
(this was significantly more useful in the earlier days of \LaTeX{})
or to generate a smaller document which is easier to navigate.
Another application of |\includeonly| is to generate
documents consisting of selected parts of the complete document.

However, there are a few drawbacks of the plain |\include| mechanism:
\begin{itemize}
\item
The child files cannot be compiled on their own,
they can only be compiled via the main file.
A naive editing environment
(such as a text editor with an option
to have the current file processed by \LaTeX)
may require one to switch to the main file before compiling;
attempting to compile the child file produces errors.
\item
The main file must be modified (each time)
to adjust the |\includeonly| command
to the present needs. This easily leaves the main file in a messy state.
\item
The generated document will always carry the filename
of the main document. This is inconvenient if
several child files are to be compiled and
to be kept for distribution.
\end{itemize}

The present package provides a simple interface
to make child files individually compilable by \LaTeX{}.
Compiling a child file then has the same effect as compiling
the main file with an |\includeonly| command
to select the appropriate child.
Moreover the generated document will carry the name of the child
rather than the main file.
This resolves all three above issues.

This feature is meant to make the editing of books,
thesis documents and lecture notes somewhat more convenient.
However, the package can also be used efficiently for
composing a series of documents (such as exercise sheets)
which are typically distributed individually.
It then assists the author in generating the individual documents
(potentially in different versions)
as well as a document containing the collected series.
Another application is in developing style files
or other kinds of included material
where compilation of the style file could redirect
to a sample or test file.

%%%%%%%%%%%%%%%%%%%%%%%%%%%%%%%%%%%%%%%%%%%%%%%%%%%%%%%%%%%%%%%%%%%%%%%%%%%%%%%%
%%%%%%%%%%%%%%%%%%%%%%%%%%%%%%%%%%%%%%%%%%%%%%%%%%%%%%%%%%%%%%%%%%%%%%%%%%%%%%%%
\section{Usage}

First of all, the package \textsf{childdoc} is \emph{not} a standard
\LaTeXe{} |.sty| style file! Therefore it needs to be invoked in
a non-standard way.

%%%%%%%%%%%%%%%%%%%%%%%%%%%%%%%%%%%%%%%%%%%%%%%%%%%%%%%%%%%%%%%%%%%%%%%%%%%%%%%%
\subsection{Included Files}
\label{sec:include}

%%%%%%%%%%%%%%%%%%%%%%%%%%%%%%%%%%%%%%%%
\DescribeMacro{\childdocmain}
To use the package, add the commands
\begin{center}
\begin{tabular}{l}
|\input{childdoc.def}|\\
|\childdocmain{}|\\
\end{tabular}
\end{center}
at the very top of the main \LaTeX{} file,
in particular \emph{before} the |\documentclass| statement!
The argument of |\childdocmain| should be left empty
(but it must be present).

%%%%%%%%%%%%%%%%%%%%%%%%%%%%%%%%%%%%%%%%
\DescribeMacro{\childdocof}
Furthermore, add the commands
\begin{center}
\begin{tabular}{l}
|\input{childdoc.def}|\\
|\childdocof{|\textit{main}|}|\\
\end{tabular}
\end{center}
at the top of every child file \textit{child}
which is included by |\include{|\textit{child}|}|
from within the main file
(or at least for those files to be compiled individually).
The argument \textit{main} must be the filename of the main file.

There are a couple of
considerations in setting up the main and child documents:

%%%%%%%%%%%%%%%%%%%%%%%%%%%%%%%%%%%%%%%%
\paragraph{Restrictions.}

Please note the following restrictions:
\begin{itemize}
\item
|\childdocmain| must be called with one argument \textit{main}
to ensure compatibility with earlier version of the package.
It must either be empty (|\childdocmain{}|)
or precisely match the filename of the main file in which it is specified.
See \secref{sec:detection} for further information.
\item
The filename \textit{main} must be specified without the |.tex| extension.
\item
The filename \textit{main} is case sensitive
(even in case-insensitive file systems)
due to internal string comparison.
\item
The argument \textit{main} should be fully expanded, it cannot be a macro.
\item
Subdirectories and special characters should be avoided in filenames.
\item
The command |\childdocmain{|\textit{main}|}| must be followed by a whitespace.
It should not be followed immediately by another command
or by a comment mark `|%|'.
This is because the \TeX{} parser reads the token immediately following
the argument of |\childdocmain| and puts it
at the beginning of every child section;
however, a white\-space is ignored.
\end{itemize}

%%%%%%%%%%%%%%%%%%%%%%%%%%%%%%%%%%%%%%%%
\paragraph{Content of Main File.}

It is advisable to place all content in the child files included by |\include|.
Any output contained in the main file will appear in all child documents
unless suppressed manually;
it cannot be suppressed automatically by the |\includeonly| directive
and thus should normally be avoided.
A method to include some content in the main file
by means of conditional processing is described in \secref{sec:conditional}.

%%%%%%%%%%%%%%%%%%%%%%%%%%%%%%%%%%%%%%%%
\paragraph{Page Numbering.}

When only a part of the document is compiled,
the appropriate numbering of pages
(as well as other status parameters)
is determined from the |.aux| files.
The latter contain information from previous passes.
However this information needs to propagate through
all intermediate child documents.
Therefore the page numbering in child documents may well
be inconsistent until the complete document is compiled at least once.

A useful (if unconventional) way to always ensure a consistent
page numbering is to restart the numbering in each child document
and denote the pages by `\textit{child}|.|\textit{page}'
where \textit{child} represents the chapter/section number of the child file.
This can be achieved by the command
|\numberwithin{page}{|\textit{child}|}|
of the \textsf{amsmath} package
where \textit{child} can be |chapter| or |section|
depending on the chosen structuring.
Alternatively, one can modify the macro |\thepage| appropriately
and reset the counter |page| at the start of each child file.

%%%%%%%%%%%%%%%%%%%%%%%%%%%%%%%%%%%%%%%%%%%%%%%%%%%%%%%%%%%%%%%%%%%%%%%%%%%%%%%%
\subsection{Conditional Processing}
\label{sec:conditional}

The package provides a mechanism to compile different versions
of a document. To customise the versions further some conditional processing
can come in handy to distinguish which version is being compiled.
The package provides two macros to describe the compilation context:

%%%%%%%%%%%%%%%%%%%%%%%%%%%%%%%%%%%%%%%%
\DescribeMacro{\ifchilddoc}
The conditional |\ifchilddoc| distinguishes between the compilation of
child documents and the main document:
%
\begin{center}
|\ifchilddoc |\textit{child-code}| |[|\||else |\textit{main-code}]| \||fi|
\end{center}

%%%%%%%%%%%%%%%%%%%%%%%%%%%%%%%%%%%%%%%%
\DescribeMacro{\childdocname}
\DescribeMacro{\childdocjob}
The macro |\childdocname| contains the filename (without extension)
of the main or child file being processed.
Note that |\childdocjob| will always contain the name of the main file.

%%%%%%%%%%%%%%%%%%%%%%%%%%%%%%%%%%%%%%%%
\paragraph{Title Page.}

Conditional processing can be used to include a title or banner page
in the main document when proper precautions are taken.
Importantly, the code in the main file should ensure that the page counter
(as well as other status parameters which are stored in the |.aux| files)
takes the same value after the conditional processing.
Otherwise the page numbers may take divergent values
depending on which part is compiled.

For example, a title page could be declared by:
%
\begin{center}
\begin{tabular}{l}
|\ifchilddoc\||else|\\
|\addtocounter{page}{-1}|\\
\textit{code for title page}\\
|\newpage|\\
|\||fi|
\end{tabular}
\end{center}
%
A banner page for the child documents can be generated by:
%
\begin{center}
\begin{tabular}{l}
|\ifchilddoc|\\
|\addtocounter{page}{-1}|\\
\textit{code for banner page}\\
|\newpage|\\
|\||fi|
\end{tabular}
\end{center}
%
Here one could write a message such as:
\begin{center}
|This is the part \childdocname{} of \childdocjob{}.|
\end{center}

%%%%%%%%%%%%%%%%%%%%%%%%%%%%%%%%%%%%%%%%%%%%%%%%%%%%%%%%%%%%%%%%%%%%%%%%%%%%%%%%
\subsection{Flags}
\label{sec:flags}

The package makes it easy to generate different versions
of the main or child documents.
To this end compilation flags can be defined
and assigned different default values.
They will be particularly useful in conjunction
with the forwarding mechanism described in \secref{sec:forward}.

For example, it may be useful to have a flag |\version|
which can be set to |draft| or |final|.
The document source will contain some conditional code
depending on the value of |\version|.
Suppose further, the flag should default to |final| for the main file
and to |draft| for child files
which is a natural assignment for editing the document.
This is achieved by placing the following code
in the preamble of the main document
(below the |\childdocmain| directive):
%
\begin{center}
\begin{tabular}{l}
|\ifchilddoc|\\
|\providecommand{\version}{draft}|\\
|\||else|\\
|\providecommand{\version}{final}|\\
|\||fi|
\end{tabular}
\end{center}
%
The definition by |\providecommand| makes sure
that previous definitions are not overwritten.
Further statements |\providecommand{\version}{...}|
can thus be added before the above code to override it.

For the main file, one might add a line
(between |\childdocmain| and the above block)
%
\begin{center}
|%\ifchilddoc\||else\providecommand{\version}{draft}\||fi|
\end{center}
%
which can be uncommented to produce a draft version.
Likewise one can add a line to the very top of a child file
(above the |\childdocof{|\textit{main}|}| directive)
%
\begin{center}
|%\providecommand{\version}{final}|
\end{center}
%
which can be uncommented to produce the final version of this child document.

%%%%%%%%%%%%%%%%%%%%%%%%%%%%%%%%%%%%%%%%%%%%%%%%%%%%%%%%%%%%%%%%%%%%%%%%%%%%%%%%
\subsection{Forwarding}
\label{sec:forward}

Different versions of the main or child documents
using compilation flags as described in \secref{sec:flags}
can be (permanently) stored in different files
for convenient compilation, viewing and distribution.
To this end, the package defines a command
to pass on compilation to a different file:

%%%%%%%%%%%%%%%%%%%%%%%%%%%%%%%%%%%%%%%%
\DescribeMacro{\childdocforward}
The command |\childdocforward| redirects processing to
another source file:
%
\begin{center}
\begin{tabular}{l}
|\input{childdoc.def}|\\
|\childdocforward[|\textit{main}|]{|\textit{dest}|}|\\
\end{tabular}
\end{center}
%
The argument \textit{dest} is the destination file
(without extension).
It should be the main file or one of the child files.
Note that further \textsf{childdoc} directives
such as |\childdocof| and |\childdocforward|
in the indicated file will be processed in this form.
The optional argument \textit{main}
passes on directly to the main file \textit{main}
while pretending to compile the child \textit{dest}.
This form behaves as if \textit{dest}
issues |\childdocof{|\textit{main}|}| right away,
and no further \textsf{childdoc} directives will be processed.

%%%%%%%%%%%%%%%%%%%%%%%%%%%%%%%%%%%%%%%%
\DescribeMacro{\...prefix}
In the alternative form |\childdocforwardprefix|,
%
\begin{center}
\begin{tabular}{l}
|\input{childdoc.def}|\\
|\childdocforwardprefix[|\textit{main}|]{|\textit{prefix}|}{|\textit{dest}|}|
\end{tabular}
\end{center}
%
the destination file is determined by a pattern
depending on the current file:
To make this work, the current file must be called
`{\textit{prefix}\hspace{0.2em}\textit{suffix}}'
with \textit{prefix} matching precisely the argument.
Processing is then passed on to the file
`{\textit{dest}\hspace{0.2em}\textit{suffix}}'.
Surely, the same effect is achieved by
directly specifying the
argument `{\textit{dest}\hspace{0.2em}\textit{suffix}}'
in the first form.
However, that requires to set up a different file
for each child. With the alternative form of the command
all these files can have exactly the same content
which simplifies setting them up and maintaining them.

For example, the following file |draft.tex|
with a compilation flag |\version| as described in \secref{sec:flags}
compiles the main document as a draft:
%
\begin{center}
\begin{tabular}{l}
|\def\version{draft}|\\
|\input{childdoc.def}|\\
|\childdocforward{|\textit{main}|}|
\end{tabular}
\end{center}
%
Likewise, the following files |final|\textit{nn}|.tex|
compile the final version of the child document
|child|\textit{nn}|.tex|:
%
\begin{center}
\begin{tabular}{l}
|\def\version{final}|\\
|\input{childdoc.def}|\\
|\childdocforwardprefix{final}{child}|
\end{tabular}
\end{center}
%

Note that when several versions of a main file and/or of each child file
are to be generated, it may be convenient to set up a |Makefile| or
shell script to automatise the process.

%%%%%%%%%%%%%%%%%%%%%%%%%%%%%%%%%%%%%%%%%%%%%%%%%%%%%%%%%%%%%%%%%%%%%%%%%%%%%%%%
\subsection{Command Line Processing}
\label{sec:commandline}

The effect of redirection files can also be achieved by invoking
the \LaTeX{} compiler with a more elaborate command line.
Most conveniently this should be done as part
of a shell script or a |Makefile|.

When using \textsf{childdoc} in the main file, the following
command lines effectively perform a redirection
(note that depending on the shell being used,
backslashes may have to be doubled: `|\|' $\to$ `|\\|'):
%
\begin{center}
|... -jobname "|\textit{target}|" |\\|"|[\textit{flags}]%
|\input{childdoc.def}\childdocforward[|\textit{main}|]{|\textit{dest}|}"|
\end{center}
%
Here \textit{target} is the name of the output file,
\textit{main} is the name of the main file
and \textit{dest} is the name of the main or child file to be processed
(all filenames without extensions).
The optional argument \textit{main} can be omitted
if \textit{main} matches \textit{dest}.
Optionally, compilation \textit{flags} can be defined via |\def| commands.
This command line makes the \TeX{} engine believe
it is compiling the file \textit{target}
whose content is specified as the latter parameter.
The provided code then forwards the processing to
\textit{main} or \textit{dest} as described in \secref{sec:forward}.

%%%%%%%%%%%%%%%%%%%%%%%%%%%%%%%%%%%%%%%%%%%%%%%%%%%%%%%%%%%%%%%%%%%%%%%%%%%%%%%%
\subsection{Include by Input}
\label{sec:input}

Including child documents by |\include| has some restrictions by design.
Most notably, the content of a child document always occupies
its own set of pages; pages cannot be shared between child documents.
Usually, this behaviour makes perfect sense
because each child document contain an essential part of the document.
However, in some situations it may be desirable to compose
a document from a collection of parts
without having mandatory page breaks between then.
For this case, the package
provides a mechanism to include parts
by |\input| which can also be processed individually.
However, by construction this mechanism
requires manual handling of the content to be output.

%%%%%%%%%%%%%%%%%%%%%%%%%%%%%%%%%%%%%%%%
\DescribeMacro{\ifchilddocmanual}
The main file should be prepared as usual, see \secref{sec:include}.
However, the document body must make a distinction
between processing of an individual part and of the main document, e.g.:
%
\begin{center}
\begin{tabular}{l}
|\ifchilddocmanual|\\
|\input{\childdocname}|\\
|\||else|\\
\textit{document body with }|\input{|\textit{part}|}|\\
|\||fi|
\end{tabular}
\end{center}
%
The conditional |\ifchilddocmanual| is true whenever
a part to be included by |\input| is being compiled,
and the name of the part is stored in |\childdocname|.

%%%%%%%%%%%%%%%%%%%%%%%%%%%%%%%%%%%%%%%%
\DescribeMacro{\childdocby}
Each part to be included by |\input| should start with:
%
\begin{center}
\begin{tabular}{l}
|\input{childdoc.def}|\\
|\childdocby{|\textit{main}|}|\\
\end{tabular}
\end{center}
%
The directive |\childdocby| is similar to |\childdocof|
described in \secref{sec:include},
but the subsequent selection of content must be done manually.
To that end, both |\ifchilddoc| and |\ifchilddocmanual|
will be true upon processing of a part,
and the name of the part is stored in |\childdocname|.
Note that |\jobname| will be set to the filename of the current part
so that each part receives an individual |.aux| file
that does not interfere with the |.aux| file(s) of the main document.
This behaviour can be altered by the alternative form
|\childdocby[*]{|\textit{main}|}| (with a non-empty optional argument)
which uses the |.aux| file of the main document
by setting |\jobname| to \textit{main}.

%%%%%%%%%%%%%%%%%%%%%%%%%%%%%%%%%%%%%%%%%%%%%%%%%%%%%%%%%%%%%%%%%%%%%%%%%%%%%%%%
\subsection{Driver Development}
\label{sec:driver}

The \textsf{childdoc} mechanism can also be use for the development
of definition files such as \LaTeX{} styles or classes.
This case differs from the above setup with multiple parts
included by |\include| in that no |\includeonly| should be invoked.
This can be achieved by starting the include file
(before |\ProvidesPackage|) with:
%
\begin{center}
\begin{tabular}{l}
|\input{childdoc.def}|\\
|\childdocforward{|\textit{main}|}|\\
\end{tabular}
\end{center}
%
or alternatively with:
%
\begin{center}
\begin{tabular}{l}
|\input{childdoc.def}|\\
|\childdocby{|\textit{main}|}|\\
\end{tabular}
\end{center}
%
Both forms have slightly different effects as described above.
The main file is prepared as usual, see \secref{sec:include}.

%%%%%%%%%%%%%%%%%%%%%%%%%%%%%%%%%%%%%%%%%%%%%%%%%%%%%%%%%%%%%%%%%%%%%%%%%%%%%%%%
\subsection{Legacy Detection}
\label{sec:detection}

The directive |\childdocmain| in the main file can detect
whether the complete document or merely a child is to be compiled
even without using the directive |\childdocof|.
This method is deprecated because it is less robust
and there is no compelling reason to use it;
it is merely provided for backward compatibility
and it may be removed in future versions.

If the detection mechanism is to be used,
it is mandatory to correctly specify
the filename of the main file as the argument of |\childdocmain|:
%
\begin{center}
\begin{tabular}{l}
|\input{childdoc.def}|\\
|\childdocmain{|\textit{main}|}|\\
\end{tabular}
\end{center}
%
If |\jobname| does not match the argument \textit{main} of |\childdocmain|,
it is assumed that |\jobname| points to the child file to be compiled.
When using |\childdocmain| with the main file specified as argument,
it suffices to start a child file
with just |\input{|\textit{main}|}|
without loading of the package and using |\childdocof|.
If instead all processing is done
with the appropriate \textsf{childdoc} directives,
the argument of \textit{main} of |\childdocmain| can be empty.

An alternative version of the command line processing described
in \secref{sec:commandline} using the detection mechanism reads:
%
\begin{center}
|... -jobname "|\textit{target}|" "|[\textit{flags}]%
[|\def\jobname{|\textit{dest}|}|]|\input{|\textit{main}|}"|
\end{center}

%%%%%%%%%%%%%%%%%%%%%%%%%%%%%%%%%%%%%%%%%%%%%%%%%%%%%%%%%%%%%%%%%%%%%%%%%%%%%%%%
\subsection{Manual Code}
\label{sec:manual}

In case one cannot be certain whether the definitions file |childdoc.def|
is installed on the target \TeX{} distribution
and one prefers not to ship it,
it is conceivable to paste a few relevant commands into the sources.

To that end, drop all statements |\input{childdoc.def}|
and perform the replacements as outlined below.
Instead of |\childdocmain{|\textit{main}|}| add the following code
to the top of the main file:
%
\begin{center}
\begin{tabular}{l}
|\||ifdefined\childdocname\endinput\||fi\newif\ifchilddoc|\\
|\edef\childdocname{\scantokens\expandafter{\jobname\noexpand}}|\\
|\def\childdocmain{|\textit{main}|}\||ifx\childdocmain\childdocname\||else|\\
|\childdoctrue\includeonly{\childdocname}\let\jobname\childdocmain\||fi|\\
\end{tabular}
\end{center}
%
Instead of |\childdocof{|\textit{main}|}| just include the main file
at the top of each child file:
%
\begin{center}
|\input{|\textit{main}|}|
\end{center}
%
A simple redirection |\childdocforward{|\textit{dest}|}| is achieved by:
%
\begin{center}
|\def\jobname{|\textit{dest}|}\input{\jobname}|
\end{center}
%
The redirection with prefix
|\childdocforwardprefix[|\textit{prefix}|]{|\textit{dest}|}|
is accomplished by:
%
\begin{center}
\begin{tabular}{l}
|{\edef\jobname{\scantokens\expandafter{\jobname\noexpand}}|\\
|\def\redirectjob |\textit{prefix}|#1~~~{\gdef\jobname{|\textit{dest}|#1}}|\\
|\expandafter\redirectjob\jobname~~~}\input{\jobname}|
\end{tabular}
\end{center}

In an alternative approach,
child documents can be compiled by a specific command line
without additional code or specific definitions:
%
\begin{center}
|... -jobname "|\textit{target}|" "|[\textit{flags}]%
|\includeonly{|\textit{dest}|}\input{|\textit{main}|}"|
\end{center}
%

%%%%%%%%%%%%%%%%%%%%%%%%%%%%%%%%%%%%%%%%%%%%%%%%%%%%%%%%%%%%%%%%%%%%%%%%%%%%%%%%
%%%%%%%%%%%%%%%%%%%%%%%%%%%%%%%%%%%%%%%%%%%%%%%%%%%%%%%%%%%%%%%%%%%%%%%%%%%%%%%%
\section{Information}

%%%%%%%%%%%%%%%%%%%%%%%%%%%%%%%%%%%%%%%%%%%%%%%%%%%%%%%%%%%%%%%%%%%%%%%%%%%%%%%%
\subsection{Copyright}

Copyright \copyright{} 2017--2018 Niklas Beisert

This work may be distributed and/or modified under the
conditions of the \LaTeX{} Project Public License, either version 1.3
of this license or (at your option) any later version.
The latest version of this license is in
  \url{http://www.latex-project.org/lppl.txt}
and version 1.3 or later is part of all distributions of \LaTeX{}
version 2005/12/01 or later.

This work has the LPPL maintenance status `maintained'.

The Current Maintainer of this work is Niklas Beisert.

This work consists of the files |README.txt|, |childdoc.ins| and |childdoc.dtx|
as well as the derived files |childdoc.def|, |cdocsamp.tex|
with |cdocsch1.tex|, |cdocsch2.tex|, |cdocspt3.tex|, |cdocspt4.tex|,
|cdocsdrf.tex|, |cdocsfn1.tex|, |cdocsfn2.tex|
as well as |childdoc.pdf|.

%%%%%%%%%%%%%%%%%%%%%%%%%%%%%%%%%%%%%%%%%%%%%%%%%%%%%%%%%%%%%%%%%%%%%%%%%%%%%%%%
\subsection{Files and Installation}

The package consists of the files:
%
\begin{center}
\begin{tabular}{ll}
    |README.txt|   & readme file \\
    |childdoc.ins| & installation file \\
    |childdoc.dtx| & source file \\
    |childdoc.def| & definition file \\
    |cdocsamp.tex| & sample main file \\
    |cdocsch1.tex| & sample include file \\
    |cdocsch2.tex| & sample include file \\
    |cdocspt3.tex| & sample part file \\
    |cdocspt4.tex| & sample part file \\
    |cdocsdrf.tex| & sample redirection file \\
    |cdocsfn1.tex| & sample redirection file \\
    |cdocsfn2.tex| & sample redirection file \\
    |childdoc.pdf| & manual
\end{tabular}
\end{center}
%
The distribution consists of the files
|README.txt|, |childdoc.ins| and |childdoc.dtx|.
%
\begin{itemize}
\item
Run (pdf)\LaTeX{} on |childdoc.dtx|
to compile the manual |childdoc.pdf| (this file).
\item
Run \LaTeX{} on |childdoc.ins| to create the definitions file |childdoc.def|
and the sample |cdocsamp.tex| with include files
|cdocsch1.tex|, |cdocsch2.tex|, |cdocspt3.tex|, |cdocspt4.tex|,
|cdocsdrf.tex|, |cdocsfn1.tex|, |cdocsfn2.tex|.
Then copy the file |childdoc.def| to an appropriate directory of your \LaTeX{}
distribution, e.g.\ \textit{texmf-root}|/tex/latex/childdoc|.
\end{itemize}

%%%%%%%%%%%%%%%%%%%%%%%%%%%%%%%%%%%%%%%%%%%%%%%%%%%%%%%%%%%%%%%%%%%%%%%%%%%%%%%%
\subsection{Related CTAN Packages}

There are several other packages which offer a similar functionality:
%
\begin{itemize}
\item
The packages
\href{http://ctan.org/pkg/docmute}{\textsf{docmute}},
\href{http://ctan.org/pkg/includex}{\textsf{includex}} and
\href{http://ctan.org/pkg/standalone}{\textsf{standalone}}
provide commands to include only the document body of
a child file thus allowing both files to be compiled individually.
\item
The packages \href{http://ctan.org/pkg/subdocs}{\textsf{subdocs}}
and \href{http://ctan.org/pkg/subfiles}{\textsf{subfiles}}
provide structures in which the main and child documents can be
encapsulated and allowing them to be compiled individually.
The inclusion mechanism is different from the conventional |\include|.
\item
The package \href{http://ctan.org/pkg/combine}{\textsf{combine}}
is an elaborate solution to combine several documents into one.
\end{itemize}
%
See also the CTAN topic \href{http://ctan.org/topic/subdocs}{\textsf{subdocs}}
for further related packages.
The present package differs from the above solutions in that
a document structure constructed with the conventional |\include| mechanism
just needs two extra commands at the top of every file
such that all constituent files can be compiled individually.

%%%%%%%%%%%%%%%%%%%%%%%%%%%%%%%%%%%%%%%%%%%%%%%%%%%%%%%%%%%%%%%%%%%%%%%%%%%%%%%%
%\subsection{Feature Suggestions}
%
%The following is a list of features which may be useful for future
%versions of this package:
%%
%\begin{itemize}
%\item
%\ldots
%\end{itemize}

%%%%%%%%%%%%%%%%%%%%%%%%%%%%%%%%%%%%%%%%%%%%%%%%%%%%%%%%%%%%%%%%%%%%%%%%%%%%%%%%
\subsection{Revision History}

%%%%%%%%%%%%%%%%%%%%%%%%%%%%%%%%%%%%%%%%
\paragraph{v2.0:} 2018/12/30

\begin{itemize}
\item
immediate forward processing
\item
added |\childdocby| mechanism
\item
manual restructured
\end{itemize}

%%%%%%%%%%%%%%%%%%%%%%%%%%%%%%%%%%%%%%%%
\paragraph{v1.6:} 2018/01/17

\begin{itemize}
\item
application for development of include files
\item
corrections to manual
\end{itemize}

%%%%%%%%%%%%%%%%%%%%%%%%%%%%%%%%%%%%%%%%
\paragraph{v1.5:} 2017/05/21

\begin{itemize}
\item
more complete structuring introduced
\item
|\childdocof| introduced
\item
|\childdoc| renamed to |\childdocmain|
\item
|\childredirect| renamed to |\childdocforward| and |\childdocforwardprefix|
and functionality expanded
\end{itemize}

%%%%%%%%%%%%%%%%%%%%%%%%%%%%%%%%%%%%%%%%
\paragraph{v1.0:} 2017/04/27

\begin{itemize}
\item
manual and install package
\item
first version published on CTAN
\end{itemize}

%%%%%%%%%%%%%%%%%%%%%%%%%%%%%%%%%%%%%%%%
\paragraph{v0.6:} 2017/04/26

\begin{itemize}
\item
redirection mechanism added
\end{itemize}

%%%%%%%%%%%%%%%%%%%%%%%%%%%%%%%%%%%%%%%%
\paragraph{v0.5:} 2017/04/26

\begin{itemize}
\item
functionality in definition file
\end{itemize}


%%%%%%%%%%%%%%%%%%%%%%%%%%%%%%%%%%%%%%%%%%%%%%%%%%%%%%%%%%%%%%%%%%%%%%%%%%%%%%%%
%%%%%%%%%%%%%%%%%%%%%%%%%%%%%%%%%%%%%%%%%%%%%%%%%%%%%%%%%%%%%%%%%%%%%%%%%%%%%%%%
%%%%%%%%%%%%%%%%%%%%%%%%%%%%%%%%%%%%%%%%%%%%%%%%%%%%%%%%%%%%%%%%%%%%%%%%%%%%%%%%
\appendix

\settowidth\MacroIndent{\rmfamily\scriptsize 000\ }

 \DocInput{childdoc.dtx}

\end{document}
%</driver>
% \fi
%
% %%%%%%%%%%%%%%%%%%%%%%%%%%%%%%%%%%%%%%%%%%%%%%%%%%%%%%%%%%%%%%%%%%%%%%%%%%%%%%
% %%%%%%%%%%%%%%%%%%%%%%%%%%%%%%%%%%%%%%%%%%%%%%%%%%%%%%%%%%%%%%%%%%%%%%%%%%%%%%
% \section{Sample}
%\iffalse
%<*samplemain>
%\fi
%
% The following presents a sample document
% with two chapters, two parts, a title page,
% a compile flag as well as three forwarding files to set the flag.
% It consists of eight |.tex| files:
% \begin{center}
% \begin{tabular}{ll}
% |cdocsamp.tex|&main file\\
% |cdocsch1.tex|&include file for chapter 1\\
% |cdocsch2.tex|&include file for chapter 2\\
% |cdocspt3.tex|&include file for part 3\\
% |cdocspt4.tex|&include file for part 4\\
% |cdocsdrf.tex|&forwarding file for main file in draft mode\\
% |cdocsfi1.tex|&forwarding file for final version of chapter 1\\
% |cdocsfi2.tex|&forwarding file for final version of chapter 2\\
% \end{tabular}
% \end{center}
% Each of the eight files can be compiled directly by the \LaTeX{} compiler.
%
% %%%%%%%%%%%%%%%%%%%%%%%%%%%%%%%%%%%%%%
% \paragraph{Main File.}
%
% The main file is called |cdocsamp.tex|.
%
% Load the \textsf{childdoc} definitions and
% declare the filename for the main document:
%    \begin{macrocode}
\input{childdoc.def}
\childdocmain{}
%    \end{macrocode}

% Optional override for |\version| flag:
%    \begin{macrocode}
%%\ifchilddoc\else\providecommand{\version}{draft}\fi
%    \end{macrocode}

% Define the default values for the |\version| flag
% (|final| for the main file and |draft| for childs):
%    \begin{macrocode}
\ifchilddoc
\providecommand{\version}{draft}
\else
\providecommand{\version}{final}
\fi
%    \end{macrocode}

% Load the standard document class:
%    \begin{macrocode}
\documentclass[12pt]{article}
%    \end{macrocode}

% Start the document body:
%    \begin{macrocode}
\begin{document}
%    \end{macrocode}

% Declare a title page.
% Print title, part of document being processed and version flag:
%    \begin{macrocode}
\addtocounter{page}{-1}
\begin{center}
{\LARGE\bfseries{}childdoc example\par}
\vspace{1cm}
\ifchilddoc
\ifchilddocmanual part\else chapter\fi:
`\childdocname' of `\childdocjob'\par
\else
main document: `\childdocjob'\par
\fi
version: \version\par
\end{center}
\newpage
%    \end{macrocode}

% Manually include selected file,
% otherwise process as usual:
%    \begin{macrocode}
\ifchilddocmanual
\section*{part `\childdocname'}
\input{\childdocname}
\else
%    \end{macrocode}

% Include the two chapters:
%    \begin{macrocode}
\include{cdocsch1}
\include{cdocsch2}
%    \end{macrocode}

% Include the two parts unless only chapters should be displayed:
%    \begin{macrocode}
\ifchilddoc\else
\section{part three}
\input{cdocspt3}
\section{part four}
\input{cdocspt4}
\fi
%    \end{macrocode}

% Process as usual until here:
%    \begin{macrocode}
\fi
%    \end{macrocode}

% End of document body:
%    \begin{macrocode}
\end{document}
%    \end{macrocode}
%\iffalse
%</samplemain>
%\fi
%
% %%%%%%%%%%%%%%%%%%%%%%%%%%%%%%%%%%%%%%
% \paragraph{Chapter Include Files.}
%
% The include files are called |cdocsch1.tex| and |cdocsch2.tex|.
%
%\iffalse
%<*samplechap1|samplechap2>
%\fi

% Optional override for |\version| flag:
%    \begin{macrocode}
%%\providecommand{\version}{final}
%    \end{macrocode}

% Include the main document:
%    \begin{macrocode}
\input{childdoc.def}
\childdocof{cdocsamp}
%    \end{macrocode}

%\iffalse
%</samplechap1|samplechap2>
%\fi
%
%\iffalse
%<*samplechap1>
%\fi
% Some text for chapter 1:
%    \begin{macrocode}
\section{one}
some text in chapter one
%    \end{macrocode}

%\iffalse
%</samplechap1>
%\fi
% Some text for chapter 2:
%\iffalse
%<*samplechap2>
%\fi
%    \begin{macrocode}
\section{two}
more text in chapter two
%    \end{macrocode}

%\iffalse
%</samplechap2>
%\fi
%
% %%%%%%%%%%%%%%%%%%%%%%%%%%%%%%%%%%%%%%
% \paragraph{Part Include Files.}
%
% The include files are called |cdocspt3.tex| and |cdocspt4.tex|.
%
%\iffalse
%<*samplepart3|samplepart4>
%\fi

% Optional override for |\version| flag:
%    \begin{macrocode}
%%\providecommand{\version}{final}
%    \end{macrocode}

% Include the main document:
%    \begin{macrocode}
\input{childdoc.def}
\childdocby{cdocsamp}
%    \end{macrocode}

%\iffalse
%</samplepart3|samplepart4>
%\fi
%
%\iffalse
%<*samplepart3>
%\fi
% Some text for part 3:
%    \begin{macrocode}
some text in part three
%    \end{macrocode}

%\iffalse
%</samplepart3>
%\fi
% Some text for part 4:
%\iffalse
%<*samplepart4>
%\fi
%    \begin{macrocode}
more text in part four
%    \end{macrocode}

%\iffalse
%</samplepart4>
%\fi
%
% %%%%%%%%%%%%%%%%%%%%%%%%%%%%%%%%%%%%%%
% \paragraph{Forwarding for a Complete Draft.}
%
% The following forwarding file |cdocsdrf.tex|
% compiles the main document in draft mode:
%\iffalse
%<*sampledraft>
%\fi
%    \begin{macrocode}
\def\version{draft}
\input{childdoc.def}
\childdocforward{cdocsamp}
%    \end{macrocode}

%\iffalse
%</sampledraft>
%\fi
%
% %%%%%%%%%%%%%%%%%%%%%%%%%%%%%%%%%%%%%%
% \paragraph{Forwarding for Final Version of the Chapters.}
%
% The following forwarding files |cdocsfn1.tex| and |cdocsfn2.tex|
% (with identical content)
% compile the final versions of the child documents
% |cdocsch1.tex| and |cdocsch2.tex|, respectively:
%\iffalse
%<*samplefinal>
%\fi
%    \begin{macrocode}
\def\version{final}
\input{childdoc.def}
\childdocforwardprefix[cdocsamp]{cdocsfn}{cdocsch}
%    \end{macrocode}

%\iffalse
%</samplefinal>
%\fi
%
% %%%%%%%%%%%%%%%%%%%%%%%%%%%%%%%%%%%%%%
% \paragraph{Command Line Processing.}
%
% The following three command lines generate the output files
% |cdocscld|, |cdocscl1| and |cdocscl2|
% which should be identical to
% |cdocsdrf|, |cdocsch1| and |cdocsfn2|, respectively:
% \begin{center}
% \begin{tabular}{l}
% |latex -jobname cdocscld \|\\
% |  "\def\version{draft}\input{childdoc.def}\childdocforward{cdocsamp}"|\\
% |latex -jobname cdocscl1 \|\\
% |  "\input{childdoc.def}\childdocforward[cdocsamp]{cdocsch1}"|\\
% |latex -jobname cdocscl2 \|\\
% |  "\def\version{final}\input{childdoc.def}\childdocforward{cdocsch2}"|
% \end{tabular}
% \end{center}
% Note that the trailing backslash on each first line
% merely continues the input to the second line
% (for convenient cut ant paste).
% Furthermore, the command |latex| can be replaced by any
% of its alternative versions such as |pdflatex|.
%
% %%%%%%%%%%%%%%%%%%%%%%%%%%%%%%%%%%%%%%%%%%%%%%%%%%%%%%%%%%%%%%%%%%%%%%%%%%%%%%
% %%%%%%%%%%%%%%%%%%%%%%%%%%%%%%%%%%%%%%%%%%%%%%%%%%%%%%%%%%%%%%%%%%%%%%%%%%%%%%
% \section{Implementation}
%\iffalse
%<*package>
%\fi
%
% This section describes the definitions file |childdoc.def|.

% The definitions cannot be loaded using |\usepackage| or |\RequirePackage|
% which has a mechanism to prevent loading a style file more than once.
% When loading the definitions by means of |\input|
% multiple instances have to be prevented manually:
%\iffalse
%This code needs to be before the `\ProvidesFile' directive
%which is defined at the beginning of this file.
%Therefore it is also placed there and commented out here.
%</package>
%<*discard>
%\fi
%    \begin{macrocode}
\ifdefined\childdocmain\endinput\fi
%    \end{macrocode}
%\iffalse
%</discard>
%<*package>
%\fi
%
% \macro{\ifchilddoc}
% \macro{\ifchilddocmanual}
% The conditional |\ifchilddoc| tells whether a
% child (true) or main (false) document is being compiled.
% The conditional |\ifchilddocmanual| tells whether
% the |\includeonly| mechanism is used (false) or
% the selection of child files must be performed manually (true).
% The definitions initialise to false:
%    \begin{macrocode}
\newif\ifchilddoc
\newif\ifchilddocmanual
%    \end{macrocode}

% \macro{\childdocname}
% \macro{\childdocjob}
% The macro |\childdocname| stores the name of the main document
% to be compiled. The macro |\childdocjob| stores the name of
% the document on which the \LaTeX{} compiler was originally invoked.
% The content of |\jobname| cannot be compared
% to filenames specified in the source due to different catcodes.
% The following code rescans |\jobname|, stores the result
% in |\childdocname| and saves a copy in |\childdocjob|:
%    \begin{macrocode}
\edef\childdocname{\scantokens\expandafter{\jobname\noexpand}}
\let\childdocjob\childdocname
%    \end{macrocode}

% \macro{\childdocdisable}
% The macro |\childdocdisable| prevents the main file
% from being processed more than once.
% At this stage, the main document command |\childdocmain|
% is assumed to be called once again where it should do nothing.
% Any subsequent call to it should prevent
% a secondary processing of the main document
% It overwrites the forwarding commands
% |\childdocof| and |\childdocforward|
% with empty macros to prevent further inclusions of the main document:
%    \begin{macrocode}
\newcommand{\childdocdisable}
{
  \renewcommand{\childdocmain}[1]{\renewcommand{\childdocmain}[1]{\endinput}}
  \renewcommand{\childdocof}[1]{}
  \renewcommand{\childdocby}[2][]{}
  \renewcommand{\childdocforward}[2][]{}
  \renewcommand{\childdocdisable}{}
}
%    \end{macrocode}

% \macro{\childdocmain}
% The macro |\childdocmain| is to be called at the top of the main file
% with nothing or the main filename (without extension) as argument.
% First, it breaks loops.
% If the argument is not empty and does not match |\childdocname|
% (which is set by the first inclusion of |childdoc.def|),
% |\ifchilddoc| is set to true, |\includeonly| is applied to the child file
% and |\jobname| is set to the main file
% (for proper handling of |.aux| files):
%    \begin{macrocode}
\newcommand{\childdocmain}[1]
{
  \childdocdisable\childdocmain{}
  \if?#1?\else
    \begingroup
      \def\childdoctmp{#1}
      \ifx\childdoctmp\childdocname
        \def\childdoctmp{}
      \else
        \def\childdoctmp
        {
          \childdoctrue
          \includeonly{\childdocname}
          \def\childdocjob{#1}
          \def\jobname{#1}
        }
      \fi
      \expandafter
    \endgroup
    \childdoctmp
  \fi
}
%    \end{macrocode}

% \macro{\childdocof}
% The command |\childdocof| redirects
% compilation to the main file |#1|.
%    \begin{macrocode}
\newcommand{\childdocof}[1]
{
  \childdocdisable
  \childdoctrue
  \includeonly{\childdocname}
  \def\jobname{#1}
  \def\childdocjob{#1}
  \input{#1}
}
%    \end{macrocode}

% \macro{\childdocby}
% The command |\childdocby| ....
%    \begin{macrocode}
\newcommand{\childdocby}[2][]
{
  \childdocdisable
  \childdoctrue
  \childdocmanualtrue
  \if?#1?\else
    \def\jobname{#2}
  \fi
  \def\childdocjob{#2}
  \input{#2}
  \endinput
}
%    \end{macrocode}

% \macro{\childdocforward}
% The command |\childdocforward| redirects
% compilation to the main file or
% (if the optional argument is given) a child file.
% Parameters are set as if the main file
% or a child file starting with |\childdocof| was compiled.
% Then compilation is handed over to the main file:
%    \begin{macrocode}
\newcommand{\childdocforward}[2][]
{
  \begingroup
    \if?#1?
      \def\childdoctmp
      {
        \def\childdocname{#2}
        \def\childdocjob{#2}
        \def\jobname{#2}
        \input{#2}
        \endinput
      }
    \else
      \def\childdoctmp
      {
        \childdocdisable
        \def\childdocname{#2}
        \childdoctrue
        \includeonly{#2}
        \def\childdocjob{#1}
        \def\jobname{#1}
        \input{#1}
        \endinput
      }
    \fi
    \expandafter
  \endgroup
  \childdoctmp
}
%    \end{macrocode}

% \macro{\childdocforwardprefix}
% The command |\childdocforwardprefix| redirects
% compilation to the main or a child file by means of a pattern.
% The prefix |#1| in the current filename is replaced by |#2|
% and the suffix of the current filename is kept
% (it is assumed that the filename does not contain the substring `|~~~|'
% which is used as a delimiter).
% Compilation is handed over to the new file by |\childdocforward|:
%    \begin{macrocode}
\newcommand{\childdocforwardprefix}[3][]
{
  \begingroup
    \def\childdocextract #2##1~~~{\def\childdoctmp{\childdocforward[#1]{#3##1}}}
    \expandafter\childdocextract\childdocname~~~
    \expandafter
  \endgroup
  \childdoctmp
}
%    \end{macrocode}

% \macro{\childdoc}
% The deprecated macro |\childdoc| is a legacy version of |\childdocmain|:
%    \begin{macrocode}
\newcommand{\childdoc}{\childdocmain}
%    \end{macrocode}

% \macro{\childdocredirect}
% The deprecated macro |\childdocredirect| is a legacy version
% of |\childdocforward| and |\childdocforwardprefix|:
%    \begin{macrocode}
\newcommand{\childdocredirect}[2][]
{
  \begingroup
    \if?#1?
      \def\childdoctmp{\childdocforward{#2}}
    \else
      \def\childdoctmp{\childdocforwardprefix{#1}{#2}}
    \fi
    \expandafter
  \endgroup
  \childdoctmp
}
%    \end{macrocode}

%\iffalse
%</package>
%\fi
%
\endinput
\childdocforward{cdocsch2}"|
% \end{tabular}
% \end{center}
% Note that the trailing backslash on each first line
% merely continues the input to the second line
% (for convenient cut ant paste).
% Furthermore, the command |latex| can be replaced by any
% of its alternative versions such as |pdflatex|.
%
% %%%%%%%%%%%%%%%%%%%%%%%%%%%%%%%%%%%%%%%%%%%%%%%%%%%%%%%%%%%%%%%%%%%%%%%%%%%%%%
% %%%%%%%%%%%%%%%%%%%%%%%%%%%%%%%%%%%%%%%%%%%%%%%%%%%%%%%%%%%%%%%%%%%%%%%%%%%%%%
% \section{Implementation}
%\iffalse
%<*package>
%\fi
%
% This section describes the definitions file |childdoc.def|.

% The definitions cannot be loaded using |\usepackage| or |\RequirePackage|
% which has a mechanism to prevent loading a style file more than once.
% When loading the definitions by means of |\input|
% multiple instances have to be prevented manually:
%\iffalse
%This code needs to be before the `\ProvidesFile' directive
%which is defined at the beginning of this file.
%Therefore it is also placed there and commented out here.
%</package>
%<*discard>
%\fi
%    \begin{macrocode}
\ifdefined\childdocmain\endinput\fi
%    \end{macrocode}
%\iffalse
%</discard>
%<*package>
%\fi
%
% \macro{\ifchilddoc}
% \macro{\ifchilddocmanual}
% The conditional |\ifchilddoc| tells whether a
% child (true) or main (false) document is being compiled.
% The conditional |\ifchilddocmanual| tells whether
% the |\includeonly| mechanism is used (false) or
% the selection of child files must be performed manually (true).
% The definitions initialise to false:
%    \begin{macrocode}
\newif\ifchilddoc
\newif\ifchilddocmanual
%    \end{macrocode}

% \macro{\childdocname}
% \macro{\childdocjob}
% The macro |\childdocname| stores the name of the main document
% to be compiled. The macro |\childdocjob| stores the name of
% the document on which the \LaTeX{} compiler was originally invoked.
% The content of |\jobname| cannot be compared
% to filenames specified in the source due to different catcodes.
% The following code rescans |\jobname|, stores the result
% in |\childdocname| and saves a copy in |\childdocjob|:
%    \begin{macrocode}
\edef\childdocname{\scantokens\expandafter{\jobname\noexpand}}
\let\childdocjob\childdocname
%    \end{macrocode}

% \macro{\childdocdisable}
% The macro |\childdocdisable| prevents the main file
% from being processed more than once.
% At this stage, the main document command |\childdocmain|
% is assumed to be called once again where it should do nothing.
% Any subsequent call to it should prevent
% a secondary processing of the main document
% It overwrites the forwarding commands
% |\childdocof| and |\childdocforward|
% with empty macros to prevent further inclusions of the main document:
%    \begin{macrocode}
\newcommand{\childdocdisable}
{
  \renewcommand{\childdocmain}[1]{\renewcommand{\childdocmain}[1]{\endinput}}
  \renewcommand{\childdocof}[1]{}
  \renewcommand{\childdocby}[2][]{}
  \renewcommand{\childdocforward}[2][]{}
  \renewcommand{\childdocdisable}{}
}
%    \end{macrocode}

% \macro{\childdocmain}
% The macro |\childdocmain| is to be called at the top of the main file
% with nothing or the main filename (without extension) as argument.
% First, it breaks loops.
% If the argument is not empty and does not match |\childdocname|
% (which is set by the first inclusion of |childdoc.def|),
% |\ifchilddoc| is set to true, |\includeonly| is applied to the child file
% and |\jobname| is set to the main file
% (for proper handling of |.aux| files):
%    \begin{macrocode}
\newcommand{\childdocmain}[1]
{
  \childdocdisable\childdocmain{}
  \if?#1?\else
    \begingroup
      \def\childdoctmp{#1}
      \ifx\childdoctmp\childdocname
        \def\childdoctmp{}
      \else
        \def\childdoctmp
        {
          \childdoctrue
          \includeonly{\childdocname}
          \def\childdocjob{#1}
          \def\jobname{#1}
        }
      \fi
      \expandafter
    \endgroup
    \childdoctmp
  \fi
}
%    \end{macrocode}

% \macro{\childdocof}
% The command |\childdocof| redirects
% compilation to the main file |#1|.
%    \begin{macrocode}
\newcommand{\childdocof}[1]
{
  \childdocdisable
  \childdoctrue
  \includeonly{\childdocname}
  \def\jobname{#1}
  \def\childdocjob{#1}
  \input{#1}
}
%    \end{macrocode}

% \macro{\childdocby}
% The command |\childdocby| ....
%    \begin{macrocode}
\newcommand{\childdocby}[2][]
{
  \childdocdisable
  \childdoctrue
  \childdocmanualtrue
  \if?#1?\else
    \def\jobname{#2}
  \fi
  \def\childdocjob{#2}
  \input{#2}
  \endinput
}
%    \end{macrocode}

% \macro{\childdocforward}
% The command |\childdocforward| redirects
% compilation to the main file or
% (if the optional argument is given) a child file.
% Parameters are set as if the main file
% or a child file starting with |\childdocof| was compiled.
% Then compilation is handed over to the main file:
%    \begin{macrocode}
\newcommand{\childdocforward}[2][]
{
  \begingroup
    \if?#1?
      \def\childdoctmp
      {
        \def\childdocname{#2}
        \def\childdocjob{#2}
        \def\jobname{#2}
        \input{#2}
        \endinput
      }
    \else
      \def\childdoctmp
      {
        \childdocdisable
        \def\childdocname{#2}
        \childdoctrue
        \includeonly{#2}
        \def\childdocjob{#1}
        \def\jobname{#1}
        \input{#1}
        \endinput
      }
    \fi
    \expandafter
  \endgroup
  \childdoctmp
}
%    \end{macrocode}

% \macro{\childdocforwardprefix}
% The command |\childdocforwardprefix| redirects
% compilation to the main or a child file by means of a pattern.
% The prefix |#1| in the current filename is replaced by |#2|
% and the suffix of the current filename is kept
% (it is assumed that the filename does not contain the substring `|~~~|'
% which is used as a delimiter).
% Compilation is handed over to the new file by |\childdocforward|:
%    \begin{macrocode}
\newcommand{\childdocforwardprefix}[3][]
{
  \begingroup
    \def\childdocextract #2##1~~~{\def\childdoctmp{\childdocforward[#1]{#3##1}}}
    \expandafter\childdocextract\childdocname~~~
    \expandafter
  \endgroup
  \childdoctmp
}
%    \end{macrocode}

% \macro{\childdoc}
% The deprecated macro |\childdoc| is a legacy version of |\childdocmain|:
%    \begin{macrocode}
\newcommand{\childdoc}{\childdocmain}
%    \end{macrocode}

% \macro{\childdocredirect}
% The deprecated macro |\childdocredirect| is a legacy version
% of |\childdocforward| and |\childdocforwardprefix|:
%    \begin{macrocode}
\newcommand{\childdocredirect}[2][]
{
  \begingroup
    \if?#1?
      \def\childdoctmp{\childdocforward{#2}}
    \else
      \def\childdoctmp{\childdocforwardprefix{#1}{#2}}
    \fi
    \expandafter
  \endgroup
  \childdoctmp
}
%    \end{macrocode}

%\iffalse
%</package>
%\fi
%
\endinput

\childdocof{cdocsamp}
%    \end{macrocode}

%\iffalse
%</samplechap1|samplechap2>
%\fi
%
%\iffalse
%<*samplechap1>
%\fi
% Some text for chapter 1:
%    \begin{macrocode}
\section{one}
some text in chapter one
%    \end{macrocode}

%\iffalse
%</samplechap1>
%\fi
% Some text for chapter 2:
%\iffalse
%<*samplechap2>
%\fi
%    \begin{macrocode}
\section{two}
more text in chapter two
%    \end{macrocode}

%\iffalse
%</samplechap2>
%\fi
%
% %%%%%%%%%%%%%%%%%%%%%%%%%%%%%%%%%%%%%%
% \paragraph{Part Include Files.}
%
% The include files are called |cdocspt3.tex| and |cdocspt4.tex|.
%
%\iffalse
%<*samplepart3|samplepart4>
%\fi

% Optional override for |\version| flag:
%    \begin{macrocode}
%%\providecommand{\version}{final}
%    \end{macrocode}

% Include the main document:
%    \begin{macrocode}
% \iffalse
%
% childdoc.dtx Copyright (C) 2017-2018 Niklas Beisert
%
% This work may be distributed and/or modified under the
% conditions of the LaTeX Project Public License, either version 1.3
% of this license or (at your option) any later version.
% The latest version of this license is in
%   http://www.latex-project.org/lppl.txt
% and version 1.3 or later is part of all distributions of LaTeX
% version 2005/12/01 or later.
%
% This work has the LPPL maintenance status `maintained'.
%
% The Current Maintainer of this work is Niklas Beisert.
%
% This work consists of the files childdoc.dtx and childdoc.ins
% and the derived files childdoc.def and cdocsamp.tex with
% cdocsch1.tex, cdocsch2.tex, cdocsdrf.tex, cdocsfn1.tex, cdocsfn2.tex.
%
%<package>\ifdefined\childdocmain\endinput\fi
%<package>\ProvidesFile{childdoc.def}[2018/12/30 v2.0 child document driver]
%<samplemain>\ProvidesFile{cdocsamp.tex}[2018/12/30 v2.0 sample for childdoc]
%<*driver>
%\ProvidesFile{childdoc.drv}[2018/12/30 v2.0 childdoc reference manual file]
\PassOptionsToClass{10pt,a4paper}{article}
\documentclass{ltxdoc}

\usepackage[margin=35mm]{geometry}
\usepackage{hyperref}
\usepackage{hyperxmp}
\usepackage[usenames]{color}

\hypersetup{colorlinks=true}
\hypersetup{pdfstartview=FitH}
\hypersetup{pdfpagemode=UseNone}
\hypersetup{pdfsource={}}
\hypersetup{pdflang={en-UK}}
\hypersetup{pdfcopyright={Copyright 2017-2018 Niklas Beisert.
  This work may be distributed and/or modified under the
  conditions of the LaTeX Project Public License, either version 1.3
  of this license or (at your option) any later version.}}
\hypersetup{pdflicenseurl={http://www.latex-project.org/lppl.txt}}
\hypersetup{pdfcontactaddress={ETH Zurich, ITP, HIT K,
  Wolfgang-Pauli-Strasse 27}}
\hypersetup{pdfcontactpostcode={8093}}
\hypersetup{pdfcontactcity={Zurich}}
\hypersetup{pdfcontactcountry={Switzerland}}
\hypersetup{pdfcontactemail={nbeisert@itp.phys.ethz.ch}}
\hypersetup{pdfcontacturl={http://people.phys.ethz.ch/\xmptilde nbeisert/}}

\newcommand{\secref}[1]{\hyperref[#1]{section \ref*{#1}}}

\parskip1ex
\parindent0pt
\let\olditemize\itemize
\def\itemize{\olditemize\parskip0pt}

\begin{document}

\title{The \textsf{childdoc} Package}
\hypersetup{pdftitle={The childdoc Package}}
\author{Niklas Beisert\\[2ex]
  Institut f\"ur Theoretische Physik\\
  Eidgen\"ossische Technische Hochschule Z\"urich\\
  Wolfgang-Pauli-Strasse 27, 8093 Z\"urich, Switzerland\\[1ex]
  \href{mailto:nbeisert@itp.phys.ethz.ch}
  {\texttt{nbeisert@itp.phys.ethz.ch}}}
\hypersetup{pdfauthor={Niklas Beisert}}
\hypersetup{pdfsubject={Manual for the LaTeX2e Package childdoc}}
\date{30 December 2018, \textsf{v2.0}}
\maketitle

\begin{abstract}\noindent
\textsf{childdoc} is a \LaTeXe{} package
that enables the direct compilation
of document sections included by |\include|
to individual files.
\end{abstract}

\begingroup
\parskip0ex
\tableofcontents
\endgroup

%%%%%%%%%%%%%%%%%%%%%%%%%%%%%%%%%%%%%%%%%%%%%%%%%%%%%%%%%%%%%%%%%%%%%%%%%%%%%%%%
%%%%%%%%%%%%%%%%%%%%%%%%%%%%%%%%%%%%%%%%%%%%%%%%%%%%%%%%%%%%%%%%%%%%%%%%%%%%%%%%
\section{Introduction}

\LaTeX{} provides a mechanism to structure a large document (such as a book)
into a main file and several child files (containing the chapters)
using the |\include| command.
This mechanism is beneficial for documents
which span hundreds of pages in order to
make the source file(s) more manageable.
Moreover, compilation can be restricted to
selected child files by means of the |\includeonly| command.
The latter feature can be used to reduce the compilation time while editing
(this was significantly more useful in the earlier days of \LaTeX{})
or to generate a smaller document which is easier to navigate.
Another application of |\includeonly| is to generate
documents consisting of selected parts of the complete document.

However, there are a few drawbacks of the plain |\include| mechanism:
\begin{itemize}
\item
The child files cannot be compiled on their own,
they can only be compiled via the main file.
A naive editing environment
(such as a text editor with an option
to have the current file processed by \LaTeX)
may require one to switch to the main file before compiling;
attempting to compile the child file produces errors.
\item
The main file must be modified (each time)
to adjust the |\includeonly| command
to the present needs. This easily leaves the main file in a messy state.
\item
The generated document will always carry the filename
of the main document. This is inconvenient if
several child files are to be compiled and
to be kept for distribution.
\end{itemize}

The present package provides a simple interface
to make child files individually compilable by \LaTeX{}.
Compiling a child file then has the same effect as compiling
the main file with an |\includeonly| command
to select the appropriate child.
Moreover the generated document will carry the name of the child
rather than the main file.
This resolves all three above issues.

This feature is meant to make the editing of books,
thesis documents and lecture notes somewhat more convenient.
However, the package can also be used efficiently for
composing a series of documents (such as exercise sheets)
which are typically distributed individually.
It then assists the author in generating the individual documents
(potentially in different versions)
as well as a document containing the collected series.
Another application is in developing style files
or other kinds of included material
where compilation of the style file could redirect
to a sample or test file.

%%%%%%%%%%%%%%%%%%%%%%%%%%%%%%%%%%%%%%%%%%%%%%%%%%%%%%%%%%%%%%%%%%%%%%%%%%%%%%%%
%%%%%%%%%%%%%%%%%%%%%%%%%%%%%%%%%%%%%%%%%%%%%%%%%%%%%%%%%%%%%%%%%%%%%%%%%%%%%%%%
\section{Usage}

First of all, the package \textsf{childdoc} is \emph{not} a standard
\LaTeXe{} |.sty| style file! Therefore it needs to be invoked in
a non-standard way.

%%%%%%%%%%%%%%%%%%%%%%%%%%%%%%%%%%%%%%%%%%%%%%%%%%%%%%%%%%%%%%%%%%%%%%%%%%%%%%%%
\subsection{Included Files}
\label{sec:include}

%%%%%%%%%%%%%%%%%%%%%%%%%%%%%%%%%%%%%%%%
\DescribeMacro{\childdocmain}
To use the package, add the commands
\begin{center}
\begin{tabular}{l}
|% \iffalse
%
% childdoc.dtx Copyright (C) 2017-2018 Niklas Beisert
%
% This work may be distributed and/or modified under the
% conditions of the LaTeX Project Public License, either version 1.3
% of this license or (at your option) any later version.
% The latest version of this license is in
%   http://www.latex-project.org/lppl.txt
% and version 1.3 or later is part of all distributions of LaTeX
% version 2005/12/01 or later.
%
% This work has the LPPL maintenance status `maintained'.
%
% The Current Maintainer of this work is Niklas Beisert.
%
% This work consists of the files childdoc.dtx and childdoc.ins
% and the derived files childdoc.def and cdocsamp.tex with
% cdocsch1.tex, cdocsch2.tex, cdocsdrf.tex, cdocsfn1.tex, cdocsfn2.tex.
%
%<package>\ifdefined\childdocmain\endinput\fi
%<package>\ProvidesFile{childdoc.def}[2018/12/30 v2.0 child document driver]
%<samplemain>\ProvidesFile{cdocsamp.tex}[2018/12/30 v2.0 sample for childdoc]
%<*driver>
%\ProvidesFile{childdoc.drv}[2018/12/30 v2.0 childdoc reference manual file]
\PassOptionsToClass{10pt,a4paper}{article}
\documentclass{ltxdoc}

\usepackage[margin=35mm]{geometry}
\usepackage{hyperref}
\usepackage{hyperxmp}
\usepackage[usenames]{color}

\hypersetup{colorlinks=true}
\hypersetup{pdfstartview=FitH}
\hypersetup{pdfpagemode=UseNone}
\hypersetup{pdfsource={}}
\hypersetup{pdflang={en-UK}}
\hypersetup{pdfcopyright={Copyright 2017-2018 Niklas Beisert.
  This work may be distributed and/or modified under the
  conditions of the LaTeX Project Public License, either version 1.3
  of this license or (at your option) any later version.}}
\hypersetup{pdflicenseurl={http://www.latex-project.org/lppl.txt}}
\hypersetup{pdfcontactaddress={ETH Zurich, ITP, HIT K,
  Wolfgang-Pauli-Strasse 27}}
\hypersetup{pdfcontactpostcode={8093}}
\hypersetup{pdfcontactcity={Zurich}}
\hypersetup{pdfcontactcountry={Switzerland}}
\hypersetup{pdfcontactemail={nbeisert@itp.phys.ethz.ch}}
\hypersetup{pdfcontacturl={http://people.phys.ethz.ch/\xmptilde nbeisert/}}

\newcommand{\secref}[1]{\hyperref[#1]{section \ref*{#1}}}

\parskip1ex
\parindent0pt
\let\olditemize\itemize
\def\itemize{\olditemize\parskip0pt}

\begin{document}

\title{The \textsf{childdoc} Package}
\hypersetup{pdftitle={The childdoc Package}}
\author{Niklas Beisert\\[2ex]
  Institut f\"ur Theoretische Physik\\
  Eidgen\"ossische Technische Hochschule Z\"urich\\
  Wolfgang-Pauli-Strasse 27, 8093 Z\"urich, Switzerland\\[1ex]
  \href{mailto:nbeisert@itp.phys.ethz.ch}
  {\texttt{nbeisert@itp.phys.ethz.ch}}}
\hypersetup{pdfauthor={Niklas Beisert}}
\hypersetup{pdfsubject={Manual for the LaTeX2e Package childdoc}}
\date{30 December 2018, \textsf{v2.0}}
\maketitle

\begin{abstract}\noindent
\textsf{childdoc} is a \LaTeXe{} package
that enables the direct compilation
of document sections included by |\include|
to individual files.
\end{abstract}

\begingroup
\parskip0ex
\tableofcontents
\endgroup

%%%%%%%%%%%%%%%%%%%%%%%%%%%%%%%%%%%%%%%%%%%%%%%%%%%%%%%%%%%%%%%%%%%%%%%%%%%%%%%%
%%%%%%%%%%%%%%%%%%%%%%%%%%%%%%%%%%%%%%%%%%%%%%%%%%%%%%%%%%%%%%%%%%%%%%%%%%%%%%%%
\section{Introduction}

\LaTeX{} provides a mechanism to structure a large document (such as a book)
into a main file and several child files (containing the chapters)
using the |\include| command.
This mechanism is beneficial for documents
which span hundreds of pages in order to
make the source file(s) more manageable.
Moreover, compilation can be restricted to
selected child files by means of the |\includeonly| command.
The latter feature can be used to reduce the compilation time while editing
(this was significantly more useful in the earlier days of \LaTeX{})
or to generate a smaller document which is easier to navigate.
Another application of |\includeonly| is to generate
documents consisting of selected parts of the complete document.

However, there are a few drawbacks of the plain |\include| mechanism:
\begin{itemize}
\item
The child files cannot be compiled on their own,
they can only be compiled via the main file.
A naive editing environment
(such as a text editor with an option
to have the current file processed by \LaTeX)
may require one to switch to the main file before compiling;
attempting to compile the child file produces errors.
\item
The main file must be modified (each time)
to adjust the |\includeonly| command
to the present needs. This easily leaves the main file in a messy state.
\item
The generated document will always carry the filename
of the main document. This is inconvenient if
several child files are to be compiled and
to be kept for distribution.
\end{itemize}

The present package provides a simple interface
to make child files individually compilable by \LaTeX{}.
Compiling a child file then has the same effect as compiling
the main file with an |\includeonly| command
to select the appropriate child.
Moreover the generated document will carry the name of the child
rather than the main file.
This resolves all three above issues.

This feature is meant to make the editing of books,
thesis documents and lecture notes somewhat more convenient.
However, the package can also be used efficiently for
composing a series of documents (such as exercise sheets)
which are typically distributed individually.
It then assists the author in generating the individual documents
(potentially in different versions)
as well as a document containing the collected series.
Another application is in developing style files
or other kinds of included material
where compilation of the style file could redirect
to a sample or test file.

%%%%%%%%%%%%%%%%%%%%%%%%%%%%%%%%%%%%%%%%%%%%%%%%%%%%%%%%%%%%%%%%%%%%%%%%%%%%%%%%
%%%%%%%%%%%%%%%%%%%%%%%%%%%%%%%%%%%%%%%%%%%%%%%%%%%%%%%%%%%%%%%%%%%%%%%%%%%%%%%%
\section{Usage}

First of all, the package \textsf{childdoc} is \emph{not} a standard
\LaTeXe{} |.sty| style file! Therefore it needs to be invoked in
a non-standard way.

%%%%%%%%%%%%%%%%%%%%%%%%%%%%%%%%%%%%%%%%%%%%%%%%%%%%%%%%%%%%%%%%%%%%%%%%%%%%%%%%
\subsection{Included Files}
\label{sec:include}

%%%%%%%%%%%%%%%%%%%%%%%%%%%%%%%%%%%%%%%%
\DescribeMacro{\childdocmain}
To use the package, add the commands
\begin{center}
\begin{tabular}{l}
|\input{childdoc.def}|\\
|\childdocmain{}|\\
\end{tabular}
\end{center}
at the very top of the main \LaTeX{} file,
in particular \emph{before} the |\documentclass| statement!
The argument of |\childdocmain| should be left empty
(but it must be present).

%%%%%%%%%%%%%%%%%%%%%%%%%%%%%%%%%%%%%%%%
\DescribeMacro{\childdocof}
Furthermore, add the commands
\begin{center}
\begin{tabular}{l}
|\input{childdoc.def}|\\
|\childdocof{|\textit{main}|}|\\
\end{tabular}
\end{center}
at the top of every child file \textit{child}
which is included by |\include{|\textit{child}|}|
from within the main file
(or at least for those files to be compiled individually).
The argument \textit{main} must be the filename of the main file.

There are a couple of
considerations in setting up the main and child documents:

%%%%%%%%%%%%%%%%%%%%%%%%%%%%%%%%%%%%%%%%
\paragraph{Restrictions.}

Please note the following restrictions:
\begin{itemize}
\item
|\childdocmain| must be called with one argument \textit{main}
to ensure compatibility with earlier version of the package.
It must either be empty (|\childdocmain{}|)
or precisely match the filename of the main file in which it is specified.
See \secref{sec:detection} for further information.
\item
The filename \textit{main} must be specified without the |.tex| extension.
\item
The filename \textit{main} is case sensitive
(even in case-insensitive file systems)
due to internal string comparison.
\item
The argument \textit{main} should be fully expanded, it cannot be a macro.
\item
Subdirectories and special characters should be avoided in filenames.
\item
The command |\childdocmain{|\textit{main}|}| must be followed by a whitespace.
It should not be followed immediately by another command
or by a comment mark `|%|'.
This is because the \TeX{} parser reads the token immediately following
the argument of |\childdocmain| and puts it
at the beginning of every child section;
however, a white\-space is ignored.
\end{itemize}

%%%%%%%%%%%%%%%%%%%%%%%%%%%%%%%%%%%%%%%%
\paragraph{Content of Main File.}

It is advisable to place all content in the child files included by |\include|.
Any output contained in the main file will appear in all child documents
unless suppressed manually;
it cannot be suppressed automatically by the |\includeonly| directive
and thus should normally be avoided.
A method to include some content in the main file
by means of conditional processing is described in \secref{sec:conditional}.

%%%%%%%%%%%%%%%%%%%%%%%%%%%%%%%%%%%%%%%%
\paragraph{Page Numbering.}

When only a part of the document is compiled,
the appropriate numbering of pages
(as well as other status parameters)
is determined from the |.aux| files.
The latter contain information from previous passes.
However this information needs to propagate through
all intermediate child documents.
Therefore the page numbering in child documents may well
be inconsistent until the complete document is compiled at least once.

A useful (if unconventional) way to always ensure a consistent
page numbering is to restart the numbering in each child document
and denote the pages by `\textit{child}|.|\textit{page}'
where \textit{child} represents the chapter/section number of the child file.
This can be achieved by the command
|\numberwithin{page}{|\textit{child}|}|
of the \textsf{amsmath} package
where \textit{child} can be |chapter| or |section|
depending on the chosen structuring.
Alternatively, one can modify the macro |\thepage| appropriately
and reset the counter |page| at the start of each child file.

%%%%%%%%%%%%%%%%%%%%%%%%%%%%%%%%%%%%%%%%%%%%%%%%%%%%%%%%%%%%%%%%%%%%%%%%%%%%%%%%
\subsection{Conditional Processing}
\label{sec:conditional}

The package provides a mechanism to compile different versions
of a document. To customise the versions further some conditional processing
can come in handy to distinguish which version is being compiled.
The package provides two macros to describe the compilation context:

%%%%%%%%%%%%%%%%%%%%%%%%%%%%%%%%%%%%%%%%
\DescribeMacro{\ifchilddoc}
The conditional |\ifchilddoc| distinguishes between the compilation of
child documents and the main document:
%
\begin{center}
|\ifchilddoc |\textit{child-code}| |[|\||else |\textit{main-code}]| \||fi|
\end{center}

%%%%%%%%%%%%%%%%%%%%%%%%%%%%%%%%%%%%%%%%
\DescribeMacro{\childdocname}
\DescribeMacro{\childdocjob}
The macro |\childdocname| contains the filename (without extension)
of the main or child file being processed.
Note that |\childdocjob| will always contain the name of the main file.

%%%%%%%%%%%%%%%%%%%%%%%%%%%%%%%%%%%%%%%%
\paragraph{Title Page.}

Conditional processing can be used to include a title or banner page
in the main document when proper precautions are taken.
Importantly, the code in the main file should ensure that the page counter
(as well as other status parameters which are stored in the |.aux| files)
takes the same value after the conditional processing.
Otherwise the page numbers may take divergent values
depending on which part is compiled.

For example, a title page could be declared by:
%
\begin{center}
\begin{tabular}{l}
|\ifchilddoc\||else|\\
|\addtocounter{page}{-1}|\\
\textit{code for title page}\\
|\newpage|\\
|\||fi|
\end{tabular}
\end{center}
%
A banner page for the child documents can be generated by:
%
\begin{center}
\begin{tabular}{l}
|\ifchilddoc|\\
|\addtocounter{page}{-1}|\\
\textit{code for banner page}\\
|\newpage|\\
|\||fi|
\end{tabular}
\end{center}
%
Here one could write a message such as:
\begin{center}
|This is the part \childdocname{} of \childdocjob{}.|
\end{center}

%%%%%%%%%%%%%%%%%%%%%%%%%%%%%%%%%%%%%%%%%%%%%%%%%%%%%%%%%%%%%%%%%%%%%%%%%%%%%%%%
\subsection{Flags}
\label{sec:flags}

The package makes it easy to generate different versions
of the main or child documents.
To this end compilation flags can be defined
and assigned different default values.
They will be particularly useful in conjunction
with the forwarding mechanism described in \secref{sec:forward}.

For example, it may be useful to have a flag |\version|
which can be set to |draft| or |final|.
The document source will contain some conditional code
depending on the value of |\version|.
Suppose further, the flag should default to |final| for the main file
and to |draft| for child files
which is a natural assignment for editing the document.
This is achieved by placing the following code
in the preamble of the main document
(below the |\childdocmain| directive):
%
\begin{center}
\begin{tabular}{l}
|\ifchilddoc|\\
|\providecommand{\version}{draft}|\\
|\||else|\\
|\providecommand{\version}{final}|\\
|\||fi|
\end{tabular}
\end{center}
%
The definition by |\providecommand| makes sure
that previous definitions are not overwritten.
Further statements |\providecommand{\version}{...}|
can thus be added before the above code to override it.

For the main file, one might add a line
(between |\childdocmain| and the above block)
%
\begin{center}
|%\ifchilddoc\||else\providecommand{\version}{draft}\||fi|
\end{center}
%
which can be uncommented to produce a draft version.
Likewise one can add a line to the very top of a child file
(above the |\childdocof{|\textit{main}|}| directive)
%
\begin{center}
|%\providecommand{\version}{final}|
\end{center}
%
which can be uncommented to produce the final version of this child document.

%%%%%%%%%%%%%%%%%%%%%%%%%%%%%%%%%%%%%%%%%%%%%%%%%%%%%%%%%%%%%%%%%%%%%%%%%%%%%%%%
\subsection{Forwarding}
\label{sec:forward}

Different versions of the main or child documents
using compilation flags as described in \secref{sec:flags}
can be (permanently) stored in different files
for convenient compilation, viewing and distribution.
To this end, the package defines a command
to pass on compilation to a different file:

%%%%%%%%%%%%%%%%%%%%%%%%%%%%%%%%%%%%%%%%
\DescribeMacro{\childdocforward}
The command |\childdocforward| redirects processing to
another source file:
%
\begin{center}
\begin{tabular}{l}
|\input{childdoc.def}|\\
|\childdocforward[|\textit{main}|]{|\textit{dest}|}|\\
\end{tabular}
\end{center}
%
The argument \textit{dest} is the destination file
(without extension).
It should be the main file or one of the child files.
Note that further \textsf{childdoc} directives
such as |\childdocof| and |\childdocforward|
in the indicated file will be processed in this form.
The optional argument \textit{main}
passes on directly to the main file \textit{main}
while pretending to compile the child \textit{dest}.
This form behaves as if \textit{dest}
issues |\childdocof{|\textit{main}|}| right away,
and no further \textsf{childdoc} directives will be processed.

%%%%%%%%%%%%%%%%%%%%%%%%%%%%%%%%%%%%%%%%
\DescribeMacro{\...prefix}
In the alternative form |\childdocforwardprefix|,
%
\begin{center}
\begin{tabular}{l}
|\input{childdoc.def}|\\
|\childdocforwardprefix[|\textit{main}|]{|\textit{prefix}|}{|\textit{dest}|}|
\end{tabular}
\end{center}
%
the destination file is determined by a pattern
depending on the current file:
To make this work, the current file must be called
`{\textit{prefix}\hspace{0.2em}\textit{suffix}}'
with \textit{prefix} matching precisely the argument.
Processing is then passed on to the file
`{\textit{dest}\hspace{0.2em}\textit{suffix}}'.
Surely, the same effect is achieved by
directly specifying the
argument `{\textit{dest}\hspace{0.2em}\textit{suffix}}'
in the first form.
However, that requires to set up a different file
for each child. With the alternative form of the command
all these files can have exactly the same content
which simplifies setting them up and maintaining them.

For example, the following file |draft.tex|
with a compilation flag |\version| as described in \secref{sec:flags}
compiles the main document as a draft:
%
\begin{center}
\begin{tabular}{l}
|\def\version{draft}|\\
|\input{childdoc.def}|\\
|\childdocforward{|\textit{main}|}|
\end{tabular}
\end{center}
%
Likewise, the following files |final|\textit{nn}|.tex|
compile the final version of the child document
|child|\textit{nn}|.tex|:
%
\begin{center}
\begin{tabular}{l}
|\def\version{final}|\\
|\input{childdoc.def}|\\
|\childdocforwardprefix{final}{child}|
\end{tabular}
\end{center}
%

Note that when several versions of a main file and/or of each child file
are to be generated, it may be convenient to set up a |Makefile| or
shell script to automatise the process.

%%%%%%%%%%%%%%%%%%%%%%%%%%%%%%%%%%%%%%%%%%%%%%%%%%%%%%%%%%%%%%%%%%%%%%%%%%%%%%%%
\subsection{Command Line Processing}
\label{sec:commandline}

The effect of redirection files can also be achieved by invoking
the \LaTeX{} compiler with a more elaborate command line.
Most conveniently this should be done as part
of a shell script or a |Makefile|.

When using \textsf{childdoc} in the main file, the following
command lines effectively perform a redirection
(note that depending on the shell being used,
backslashes may have to be doubled: `|\|' $\to$ `|\\|'):
%
\begin{center}
|... -jobname "|\textit{target}|" |\\|"|[\textit{flags}]%
|\input{childdoc.def}\childdocforward[|\textit{main}|]{|\textit{dest}|}"|
\end{center}
%
Here \textit{target} is the name of the output file,
\textit{main} is the name of the main file
and \textit{dest} is the name of the main or child file to be processed
(all filenames without extensions).
The optional argument \textit{main} can be omitted
if \textit{main} matches \textit{dest}.
Optionally, compilation \textit{flags} can be defined via |\def| commands.
This command line makes the \TeX{} engine believe
it is compiling the file \textit{target}
whose content is specified as the latter parameter.
The provided code then forwards the processing to
\textit{main} or \textit{dest} as described in \secref{sec:forward}.

%%%%%%%%%%%%%%%%%%%%%%%%%%%%%%%%%%%%%%%%%%%%%%%%%%%%%%%%%%%%%%%%%%%%%%%%%%%%%%%%
\subsection{Include by Input}
\label{sec:input}

Including child documents by |\include| has some restrictions by design.
Most notably, the content of a child document always occupies
its own set of pages; pages cannot be shared between child documents.
Usually, this behaviour makes perfect sense
because each child document contain an essential part of the document.
However, in some situations it may be desirable to compose
a document from a collection of parts
without having mandatory page breaks between then.
For this case, the package
provides a mechanism to include parts
by |\input| which can also be processed individually.
However, by construction this mechanism
requires manual handling of the content to be output.

%%%%%%%%%%%%%%%%%%%%%%%%%%%%%%%%%%%%%%%%
\DescribeMacro{\ifchilddocmanual}
The main file should be prepared as usual, see \secref{sec:include}.
However, the document body must make a distinction
between processing of an individual part and of the main document, e.g.:
%
\begin{center}
\begin{tabular}{l}
|\ifchilddocmanual|\\
|\input{\childdocname}|\\
|\||else|\\
\textit{document body with }|\input{|\textit{part}|}|\\
|\||fi|
\end{tabular}
\end{center}
%
The conditional |\ifchilddocmanual| is true whenever
a part to be included by |\input| is being compiled,
and the name of the part is stored in |\childdocname|.

%%%%%%%%%%%%%%%%%%%%%%%%%%%%%%%%%%%%%%%%
\DescribeMacro{\childdocby}
Each part to be included by |\input| should start with:
%
\begin{center}
\begin{tabular}{l}
|\input{childdoc.def}|\\
|\childdocby{|\textit{main}|}|\\
\end{tabular}
\end{center}
%
The directive |\childdocby| is similar to |\childdocof|
described in \secref{sec:include},
but the subsequent selection of content must be done manually.
To that end, both |\ifchilddoc| and |\ifchilddocmanual|
will be true upon processing of a part,
and the name of the part is stored in |\childdocname|.
Note that |\jobname| will be set to the filename of the current part
so that each part receives an individual |.aux| file
that does not interfere with the |.aux| file(s) of the main document.
This behaviour can be altered by the alternative form
|\childdocby[*]{|\textit{main}|}| (with a non-empty optional argument)
which uses the |.aux| file of the main document
by setting |\jobname| to \textit{main}.

%%%%%%%%%%%%%%%%%%%%%%%%%%%%%%%%%%%%%%%%%%%%%%%%%%%%%%%%%%%%%%%%%%%%%%%%%%%%%%%%
\subsection{Driver Development}
\label{sec:driver}

The \textsf{childdoc} mechanism can also be use for the development
of definition files such as \LaTeX{} styles or classes.
This case differs from the above setup with multiple parts
included by |\include| in that no |\includeonly| should be invoked.
This can be achieved by starting the include file
(before |\ProvidesPackage|) with:
%
\begin{center}
\begin{tabular}{l}
|\input{childdoc.def}|\\
|\childdocforward{|\textit{main}|}|\\
\end{tabular}
\end{center}
%
or alternatively with:
%
\begin{center}
\begin{tabular}{l}
|\input{childdoc.def}|\\
|\childdocby{|\textit{main}|}|\\
\end{tabular}
\end{center}
%
Both forms have slightly different effects as described above.
The main file is prepared as usual, see \secref{sec:include}.

%%%%%%%%%%%%%%%%%%%%%%%%%%%%%%%%%%%%%%%%%%%%%%%%%%%%%%%%%%%%%%%%%%%%%%%%%%%%%%%%
\subsection{Legacy Detection}
\label{sec:detection}

The directive |\childdocmain| in the main file can detect
whether the complete document or merely a child is to be compiled
even without using the directive |\childdocof|.
This method is deprecated because it is less robust
and there is no compelling reason to use it;
it is merely provided for backward compatibility
and it may be removed in future versions.

If the detection mechanism is to be used,
it is mandatory to correctly specify
the filename of the main file as the argument of |\childdocmain|:
%
\begin{center}
\begin{tabular}{l}
|\input{childdoc.def}|\\
|\childdocmain{|\textit{main}|}|\\
\end{tabular}
\end{center}
%
If |\jobname| does not match the argument \textit{main} of |\childdocmain|,
it is assumed that |\jobname| points to the child file to be compiled.
When using |\childdocmain| with the main file specified as argument,
it suffices to start a child file
with just |\input{|\textit{main}|}|
without loading of the package and using |\childdocof|.
If instead all processing is done
with the appropriate \textsf{childdoc} directives,
the argument of \textit{main} of |\childdocmain| can be empty.

An alternative version of the command line processing described
in \secref{sec:commandline} using the detection mechanism reads:
%
\begin{center}
|... -jobname "|\textit{target}|" "|[\textit{flags}]%
[|\def\jobname{|\textit{dest}|}|]|\input{|\textit{main}|}"|
\end{center}

%%%%%%%%%%%%%%%%%%%%%%%%%%%%%%%%%%%%%%%%%%%%%%%%%%%%%%%%%%%%%%%%%%%%%%%%%%%%%%%%
\subsection{Manual Code}
\label{sec:manual}

In case one cannot be certain whether the definitions file |childdoc.def|
is installed on the target \TeX{} distribution
and one prefers not to ship it,
it is conceivable to paste a few relevant commands into the sources.

To that end, drop all statements |\input{childdoc.def}|
and perform the replacements as outlined below.
Instead of |\childdocmain{|\textit{main}|}| add the following code
to the top of the main file:
%
\begin{center}
\begin{tabular}{l}
|\||ifdefined\childdocname\endinput\||fi\newif\ifchilddoc|\\
|\edef\childdocname{\scantokens\expandafter{\jobname\noexpand}}|\\
|\def\childdocmain{|\textit{main}|}\||ifx\childdocmain\childdocname\||else|\\
|\childdoctrue\includeonly{\childdocname}\let\jobname\childdocmain\||fi|\\
\end{tabular}
\end{center}
%
Instead of |\childdocof{|\textit{main}|}| just include the main file
at the top of each child file:
%
\begin{center}
|\input{|\textit{main}|}|
\end{center}
%
A simple redirection |\childdocforward{|\textit{dest}|}| is achieved by:
%
\begin{center}
|\def\jobname{|\textit{dest}|}\input{\jobname}|
\end{center}
%
The redirection with prefix
|\childdocforwardprefix[|\textit{prefix}|]{|\textit{dest}|}|
is accomplished by:
%
\begin{center}
\begin{tabular}{l}
|{\edef\jobname{\scantokens\expandafter{\jobname\noexpand}}|\\
|\def\redirectjob |\textit{prefix}|#1~~~{\gdef\jobname{|\textit{dest}|#1}}|\\
|\expandafter\redirectjob\jobname~~~}\input{\jobname}|
\end{tabular}
\end{center}

In an alternative approach,
child documents can be compiled by a specific command line
without additional code or specific definitions:
%
\begin{center}
|... -jobname "|\textit{target}|" "|[\textit{flags}]%
|\includeonly{|\textit{dest}|}\input{|\textit{main}|}"|
\end{center}
%

%%%%%%%%%%%%%%%%%%%%%%%%%%%%%%%%%%%%%%%%%%%%%%%%%%%%%%%%%%%%%%%%%%%%%%%%%%%%%%%%
%%%%%%%%%%%%%%%%%%%%%%%%%%%%%%%%%%%%%%%%%%%%%%%%%%%%%%%%%%%%%%%%%%%%%%%%%%%%%%%%
\section{Information}

%%%%%%%%%%%%%%%%%%%%%%%%%%%%%%%%%%%%%%%%%%%%%%%%%%%%%%%%%%%%%%%%%%%%%%%%%%%%%%%%
\subsection{Copyright}

Copyright \copyright{} 2017--2018 Niklas Beisert

This work may be distributed and/or modified under the
conditions of the \LaTeX{} Project Public License, either version 1.3
of this license or (at your option) any later version.
The latest version of this license is in
  \url{http://www.latex-project.org/lppl.txt}
and version 1.3 or later is part of all distributions of \LaTeX{}
version 2005/12/01 or later.

This work has the LPPL maintenance status `maintained'.

The Current Maintainer of this work is Niklas Beisert.

This work consists of the files |README.txt|, |childdoc.ins| and |childdoc.dtx|
as well as the derived files |childdoc.def|, |cdocsamp.tex|
with |cdocsch1.tex|, |cdocsch2.tex|, |cdocspt3.tex|, |cdocspt4.tex|,
|cdocsdrf.tex|, |cdocsfn1.tex|, |cdocsfn2.tex|
as well as |childdoc.pdf|.

%%%%%%%%%%%%%%%%%%%%%%%%%%%%%%%%%%%%%%%%%%%%%%%%%%%%%%%%%%%%%%%%%%%%%%%%%%%%%%%%
\subsection{Files and Installation}

The package consists of the files:
%
\begin{center}
\begin{tabular}{ll}
    |README.txt|   & readme file \\
    |childdoc.ins| & installation file \\
    |childdoc.dtx| & source file \\
    |childdoc.def| & definition file \\
    |cdocsamp.tex| & sample main file \\
    |cdocsch1.tex| & sample include file \\
    |cdocsch2.tex| & sample include file \\
    |cdocspt3.tex| & sample part file \\
    |cdocspt4.tex| & sample part file \\
    |cdocsdrf.tex| & sample redirection file \\
    |cdocsfn1.tex| & sample redirection file \\
    |cdocsfn2.tex| & sample redirection file \\
    |childdoc.pdf| & manual
\end{tabular}
\end{center}
%
The distribution consists of the files
|README.txt|, |childdoc.ins| and |childdoc.dtx|.
%
\begin{itemize}
\item
Run (pdf)\LaTeX{} on |childdoc.dtx|
to compile the manual |childdoc.pdf| (this file).
\item
Run \LaTeX{} on |childdoc.ins| to create the definitions file |childdoc.def|
and the sample |cdocsamp.tex| with include files
|cdocsch1.tex|, |cdocsch2.tex|, |cdocspt3.tex|, |cdocspt4.tex|,
|cdocsdrf.tex|, |cdocsfn1.tex|, |cdocsfn2.tex|.
Then copy the file |childdoc.def| to an appropriate directory of your \LaTeX{}
distribution, e.g.\ \textit{texmf-root}|/tex/latex/childdoc|.
\end{itemize}

%%%%%%%%%%%%%%%%%%%%%%%%%%%%%%%%%%%%%%%%%%%%%%%%%%%%%%%%%%%%%%%%%%%%%%%%%%%%%%%%
\subsection{Related CTAN Packages}

There are several other packages which offer a similar functionality:
%
\begin{itemize}
\item
The packages
\href{http://ctan.org/pkg/docmute}{\textsf{docmute}},
\href{http://ctan.org/pkg/includex}{\textsf{includex}} and
\href{http://ctan.org/pkg/standalone}{\textsf{standalone}}
provide commands to include only the document body of
a child file thus allowing both files to be compiled individually.
\item
The packages \href{http://ctan.org/pkg/subdocs}{\textsf{subdocs}}
and \href{http://ctan.org/pkg/subfiles}{\textsf{subfiles}}
provide structures in which the main and child documents can be
encapsulated and allowing them to be compiled individually.
The inclusion mechanism is different from the conventional |\include|.
\item
The package \href{http://ctan.org/pkg/combine}{\textsf{combine}}
is an elaborate solution to combine several documents into one.
\end{itemize}
%
See also the CTAN topic \href{http://ctan.org/topic/subdocs}{\textsf{subdocs}}
for further related packages.
The present package differs from the above solutions in that
a document structure constructed with the conventional |\include| mechanism
just needs two extra commands at the top of every file
such that all constituent files can be compiled individually.

%%%%%%%%%%%%%%%%%%%%%%%%%%%%%%%%%%%%%%%%%%%%%%%%%%%%%%%%%%%%%%%%%%%%%%%%%%%%%%%%
%\subsection{Feature Suggestions}
%
%The following is a list of features which may be useful for future
%versions of this package:
%%
%\begin{itemize}
%\item
%\ldots
%\end{itemize}

%%%%%%%%%%%%%%%%%%%%%%%%%%%%%%%%%%%%%%%%%%%%%%%%%%%%%%%%%%%%%%%%%%%%%%%%%%%%%%%%
\subsection{Revision History}

%%%%%%%%%%%%%%%%%%%%%%%%%%%%%%%%%%%%%%%%
\paragraph{v2.0:} 2018/12/30

\begin{itemize}
\item
immediate forward processing
\item
added |\childdocby| mechanism
\item
manual restructured
\end{itemize}

%%%%%%%%%%%%%%%%%%%%%%%%%%%%%%%%%%%%%%%%
\paragraph{v1.6:} 2018/01/17

\begin{itemize}
\item
application for development of include files
\item
corrections to manual
\end{itemize}

%%%%%%%%%%%%%%%%%%%%%%%%%%%%%%%%%%%%%%%%
\paragraph{v1.5:} 2017/05/21

\begin{itemize}
\item
more complete structuring introduced
\item
|\childdocof| introduced
\item
|\childdoc| renamed to |\childdocmain|
\item
|\childredirect| renamed to |\childdocforward| and |\childdocforwardprefix|
and functionality expanded
\end{itemize}

%%%%%%%%%%%%%%%%%%%%%%%%%%%%%%%%%%%%%%%%
\paragraph{v1.0:} 2017/04/27

\begin{itemize}
\item
manual and install package
\item
first version published on CTAN
\end{itemize}

%%%%%%%%%%%%%%%%%%%%%%%%%%%%%%%%%%%%%%%%
\paragraph{v0.6:} 2017/04/26

\begin{itemize}
\item
redirection mechanism added
\end{itemize}

%%%%%%%%%%%%%%%%%%%%%%%%%%%%%%%%%%%%%%%%
\paragraph{v0.5:} 2017/04/26

\begin{itemize}
\item
functionality in definition file
\end{itemize}


%%%%%%%%%%%%%%%%%%%%%%%%%%%%%%%%%%%%%%%%%%%%%%%%%%%%%%%%%%%%%%%%%%%%%%%%%%%%%%%%
%%%%%%%%%%%%%%%%%%%%%%%%%%%%%%%%%%%%%%%%%%%%%%%%%%%%%%%%%%%%%%%%%%%%%%%%%%%%%%%%
%%%%%%%%%%%%%%%%%%%%%%%%%%%%%%%%%%%%%%%%%%%%%%%%%%%%%%%%%%%%%%%%%%%%%%%%%%%%%%%%
\appendix

\settowidth\MacroIndent{\rmfamily\scriptsize 000\ }

 \DocInput{childdoc.dtx}

\end{document}
%</driver>
% \fi
%
% %%%%%%%%%%%%%%%%%%%%%%%%%%%%%%%%%%%%%%%%%%%%%%%%%%%%%%%%%%%%%%%%%%%%%%%%%%%%%%
% %%%%%%%%%%%%%%%%%%%%%%%%%%%%%%%%%%%%%%%%%%%%%%%%%%%%%%%%%%%%%%%%%%%%%%%%%%%%%%
% \section{Sample}
%\iffalse
%<*samplemain>
%\fi
%
% The following presents a sample document
% with two chapters, two parts, a title page,
% a compile flag as well as three forwarding files to set the flag.
% It consists of eight |.tex| files:
% \begin{center}
% \begin{tabular}{ll}
% |cdocsamp.tex|&main file\\
% |cdocsch1.tex|&include file for chapter 1\\
% |cdocsch2.tex|&include file for chapter 2\\
% |cdocspt3.tex|&include file for part 3\\
% |cdocspt4.tex|&include file for part 4\\
% |cdocsdrf.tex|&forwarding file for main file in draft mode\\
% |cdocsfi1.tex|&forwarding file for final version of chapter 1\\
% |cdocsfi2.tex|&forwarding file for final version of chapter 2\\
% \end{tabular}
% \end{center}
% Each of the eight files can be compiled directly by the \LaTeX{} compiler.
%
% %%%%%%%%%%%%%%%%%%%%%%%%%%%%%%%%%%%%%%
% \paragraph{Main File.}
%
% The main file is called |cdocsamp.tex|.
%
% Load the \textsf{childdoc} definitions and
% declare the filename for the main document:
%    \begin{macrocode}
\input{childdoc.def}
\childdocmain{}
%    \end{macrocode}

% Optional override for |\version| flag:
%    \begin{macrocode}
%%\ifchilddoc\else\providecommand{\version}{draft}\fi
%    \end{macrocode}

% Define the default values for the |\version| flag
% (|final| for the main file and |draft| for childs):
%    \begin{macrocode}
\ifchilddoc
\providecommand{\version}{draft}
\else
\providecommand{\version}{final}
\fi
%    \end{macrocode}

% Load the standard document class:
%    \begin{macrocode}
\documentclass[12pt]{article}
%    \end{macrocode}

% Start the document body:
%    \begin{macrocode}
\begin{document}
%    \end{macrocode}

% Declare a title page.
% Print title, part of document being processed and version flag:
%    \begin{macrocode}
\addtocounter{page}{-1}
\begin{center}
{\LARGE\bfseries{}childdoc example\par}
\vspace{1cm}
\ifchilddoc
\ifchilddocmanual part\else chapter\fi:
`\childdocname' of `\childdocjob'\par
\else
main document: `\childdocjob'\par
\fi
version: \version\par
\end{center}
\newpage
%    \end{macrocode}

% Manually include selected file,
% otherwise process as usual:
%    \begin{macrocode}
\ifchilddocmanual
\section*{part `\childdocname'}
\input{\childdocname}
\else
%    \end{macrocode}

% Include the two chapters:
%    \begin{macrocode}
\include{cdocsch1}
\include{cdocsch2}
%    \end{macrocode}

% Include the two parts unless only chapters should be displayed:
%    \begin{macrocode}
\ifchilddoc\else
\section{part three}
\input{cdocspt3}
\section{part four}
\input{cdocspt4}
\fi
%    \end{macrocode}

% Process as usual until here:
%    \begin{macrocode}
\fi
%    \end{macrocode}

% End of document body:
%    \begin{macrocode}
\end{document}
%    \end{macrocode}
%\iffalse
%</samplemain>
%\fi
%
% %%%%%%%%%%%%%%%%%%%%%%%%%%%%%%%%%%%%%%
% \paragraph{Chapter Include Files.}
%
% The include files are called |cdocsch1.tex| and |cdocsch2.tex|.
%
%\iffalse
%<*samplechap1|samplechap2>
%\fi

% Optional override for |\version| flag:
%    \begin{macrocode}
%%\providecommand{\version}{final}
%    \end{macrocode}

% Include the main document:
%    \begin{macrocode}
\input{childdoc.def}
\childdocof{cdocsamp}
%    \end{macrocode}

%\iffalse
%</samplechap1|samplechap2>
%\fi
%
%\iffalse
%<*samplechap1>
%\fi
% Some text for chapter 1:
%    \begin{macrocode}
\section{one}
some text in chapter one
%    \end{macrocode}

%\iffalse
%</samplechap1>
%\fi
% Some text for chapter 2:
%\iffalse
%<*samplechap2>
%\fi
%    \begin{macrocode}
\section{two}
more text in chapter two
%    \end{macrocode}

%\iffalse
%</samplechap2>
%\fi
%
% %%%%%%%%%%%%%%%%%%%%%%%%%%%%%%%%%%%%%%
% \paragraph{Part Include Files.}
%
% The include files are called |cdocspt3.tex| and |cdocspt4.tex|.
%
%\iffalse
%<*samplepart3|samplepart4>
%\fi

% Optional override for |\version| flag:
%    \begin{macrocode}
%%\providecommand{\version}{final}
%    \end{macrocode}

% Include the main document:
%    \begin{macrocode}
\input{childdoc.def}
\childdocby{cdocsamp}
%    \end{macrocode}

%\iffalse
%</samplepart3|samplepart4>
%\fi
%
%\iffalse
%<*samplepart3>
%\fi
% Some text for part 3:
%    \begin{macrocode}
some text in part three
%    \end{macrocode}

%\iffalse
%</samplepart3>
%\fi
% Some text for part 4:
%\iffalse
%<*samplepart4>
%\fi
%    \begin{macrocode}
more text in part four
%    \end{macrocode}

%\iffalse
%</samplepart4>
%\fi
%
% %%%%%%%%%%%%%%%%%%%%%%%%%%%%%%%%%%%%%%
% \paragraph{Forwarding for a Complete Draft.}
%
% The following forwarding file |cdocsdrf.tex|
% compiles the main document in draft mode:
%\iffalse
%<*sampledraft>
%\fi
%    \begin{macrocode}
\def\version{draft}
\input{childdoc.def}
\childdocforward{cdocsamp}
%    \end{macrocode}

%\iffalse
%</sampledraft>
%\fi
%
% %%%%%%%%%%%%%%%%%%%%%%%%%%%%%%%%%%%%%%
% \paragraph{Forwarding for Final Version of the Chapters.}
%
% The following forwarding files |cdocsfn1.tex| and |cdocsfn2.tex|
% (with identical content)
% compile the final versions of the child documents
% |cdocsch1.tex| and |cdocsch2.tex|, respectively:
%\iffalse
%<*samplefinal>
%\fi
%    \begin{macrocode}
\def\version{final}
\input{childdoc.def}
\childdocforwardprefix[cdocsamp]{cdocsfn}{cdocsch}
%    \end{macrocode}

%\iffalse
%</samplefinal>
%\fi
%
% %%%%%%%%%%%%%%%%%%%%%%%%%%%%%%%%%%%%%%
% \paragraph{Command Line Processing.}
%
% The following three command lines generate the output files
% |cdocscld|, |cdocscl1| and |cdocscl2|
% which should be identical to
% |cdocsdrf|, |cdocsch1| and |cdocsfn2|, respectively:
% \begin{center}
% \begin{tabular}{l}
% |latex -jobname cdocscld \|\\
% |  "\def\version{draft}\input{childdoc.def}\childdocforward{cdocsamp}"|\\
% |latex -jobname cdocscl1 \|\\
% |  "\input{childdoc.def}\childdocforward[cdocsamp]{cdocsch1}"|\\
% |latex -jobname cdocscl2 \|\\
% |  "\def\version{final}\input{childdoc.def}\childdocforward{cdocsch2}"|
% \end{tabular}
% \end{center}
% Note that the trailing backslash on each first line
% merely continues the input to the second line
% (for convenient cut ant paste).
% Furthermore, the command |latex| can be replaced by any
% of its alternative versions such as |pdflatex|.
%
% %%%%%%%%%%%%%%%%%%%%%%%%%%%%%%%%%%%%%%%%%%%%%%%%%%%%%%%%%%%%%%%%%%%%%%%%%%%%%%
% %%%%%%%%%%%%%%%%%%%%%%%%%%%%%%%%%%%%%%%%%%%%%%%%%%%%%%%%%%%%%%%%%%%%%%%%%%%%%%
% \section{Implementation}
%\iffalse
%<*package>
%\fi
%
% This section describes the definitions file |childdoc.def|.

% The definitions cannot be loaded using |\usepackage| or |\RequirePackage|
% which has a mechanism to prevent loading a style file more than once.
% When loading the definitions by means of |\input|
% multiple instances have to be prevented manually:
%\iffalse
%This code needs to be before the `\ProvidesFile' directive
%which is defined at the beginning of this file.
%Therefore it is also placed there and commented out here.
%</package>
%<*discard>
%\fi
%    \begin{macrocode}
\ifdefined\childdocmain\endinput\fi
%    \end{macrocode}
%\iffalse
%</discard>
%<*package>
%\fi
%
% \macro{\ifchilddoc}
% \macro{\ifchilddocmanual}
% The conditional |\ifchilddoc| tells whether a
% child (true) or main (false) document is being compiled.
% The conditional |\ifchilddocmanual| tells whether
% the |\includeonly| mechanism is used (false) or
% the selection of child files must be performed manually (true).
% The definitions initialise to false:
%    \begin{macrocode}
\newif\ifchilddoc
\newif\ifchilddocmanual
%    \end{macrocode}

% \macro{\childdocname}
% \macro{\childdocjob}
% The macro |\childdocname| stores the name of the main document
% to be compiled. The macro |\childdocjob| stores the name of
% the document on which the \LaTeX{} compiler was originally invoked.
% The content of |\jobname| cannot be compared
% to filenames specified in the source due to different catcodes.
% The following code rescans |\jobname|, stores the result
% in |\childdocname| and saves a copy in |\childdocjob|:
%    \begin{macrocode}
\edef\childdocname{\scantokens\expandafter{\jobname\noexpand}}
\let\childdocjob\childdocname
%    \end{macrocode}

% \macro{\childdocdisable}
% The macro |\childdocdisable| prevents the main file
% from being processed more than once.
% At this stage, the main document command |\childdocmain|
% is assumed to be called once again where it should do nothing.
% Any subsequent call to it should prevent
% a secondary processing of the main document
% It overwrites the forwarding commands
% |\childdocof| and |\childdocforward|
% with empty macros to prevent further inclusions of the main document:
%    \begin{macrocode}
\newcommand{\childdocdisable}
{
  \renewcommand{\childdocmain}[1]{\renewcommand{\childdocmain}[1]{\endinput}}
  \renewcommand{\childdocof}[1]{}
  \renewcommand{\childdocby}[2][]{}
  \renewcommand{\childdocforward}[2][]{}
  \renewcommand{\childdocdisable}{}
}
%    \end{macrocode}

% \macro{\childdocmain}
% The macro |\childdocmain| is to be called at the top of the main file
% with nothing or the main filename (without extension) as argument.
% First, it breaks loops.
% If the argument is not empty and does not match |\childdocname|
% (which is set by the first inclusion of |childdoc.def|),
% |\ifchilddoc| is set to true, |\includeonly| is applied to the child file
% and |\jobname| is set to the main file
% (for proper handling of |.aux| files):
%    \begin{macrocode}
\newcommand{\childdocmain}[1]
{
  \childdocdisable\childdocmain{}
  \if?#1?\else
    \begingroup
      \def\childdoctmp{#1}
      \ifx\childdoctmp\childdocname
        \def\childdoctmp{}
      \else
        \def\childdoctmp
        {
          \childdoctrue
          \includeonly{\childdocname}
          \def\childdocjob{#1}
          \def\jobname{#1}
        }
      \fi
      \expandafter
    \endgroup
    \childdoctmp
  \fi
}
%    \end{macrocode}

% \macro{\childdocof}
% The command |\childdocof| redirects
% compilation to the main file |#1|.
%    \begin{macrocode}
\newcommand{\childdocof}[1]
{
  \childdocdisable
  \childdoctrue
  \includeonly{\childdocname}
  \def\jobname{#1}
  \def\childdocjob{#1}
  \input{#1}
}
%    \end{macrocode}

% \macro{\childdocby}
% The command |\childdocby| ....
%    \begin{macrocode}
\newcommand{\childdocby}[2][]
{
  \childdocdisable
  \childdoctrue
  \childdocmanualtrue
  \if?#1?\else
    \def\jobname{#2}
  \fi
  \def\childdocjob{#2}
  \input{#2}
  \endinput
}
%    \end{macrocode}

% \macro{\childdocforward}
% The command |\childdocforward| redirects
% compilation to the main file or
% (if the optional argument is given) a child file.
% Parameters are set as if the main file
% or a child file starting with |\childdocof| was compiled.
% Then compilation is handed over to the main file:
%    \begin{macrocode}
\newcommand{\childdocforward}[2][]
{
  \begingroup
    \if?#1?
      \def\childdoctmp
      {
        \def\childdocname{#2}
        \def\childdocjob{#2}
        \def\jobname{#2}
        \input{#2}
        \endinput
      }
    \else
      \def\childdoctmp
      {
        \childdocdisable
        \def\childdocname{#2}
        \childdoctrue
        \includeonly{#2}
        \def\childdocjob{#1}
        \def\jobname{#1}
        \input{#1}
        \endinput
      }
    \fi
    \expandafter
  \endgroup
  \childdoctmp
}
%    \end{macrocode}

% \macro{\childdocforwardprefix}
% The command |\childdocforwardprefix| redirects
% compilation to the main or a child file by means of a pattern.
% The prefix |#1| in the current filename is replaced by |#2|
% and the suffix of the current filename is kept
% (it is assumed that the filename does not contain the substring `|~~~|'
% which is used as a delimiter).
% Compilation is handed over to the new file by |\childdocforward|:
%    \begin{macrocode}
\newcommand{\childdocforwardprefix}[3][]
{
  \begingroup
    \def\childdocextract #2##1~~~{\def\childdoctmp{\childdocforward[#1]{#3##1}}}
    \expandafter\childdocextract\childdocname~~~
    \expandafter
  \endgroup
  \childdoctmp
}
%    \end{macrocode}

% \macro{\childdoc}
% The deprecated macro |\childdoc| is a legacy version of |\childdocmain|:
%    \begin{macrocode}
\newcommand{\childdoc}{\childdocmain}
%    \end{macrocode}

% \macro{\childdocredirect}
% The deprecated macro |\childdocredirect| is a legacy version
% of |\childdocforward| and |\childdocforwardprefix|:
%    \begin{macrocode}
\newcommand{\childdocredirect}[2][]
{
  \begingroup
    \if?#1?
      \def\childdoctmp{\childdocforward{#2}}
    \else
      \def\childdoctmp{\childdocforwardprefix{#1}{#2}}
    \fi
    \expandafter
  \endgroup
  \childdoctmp
}
%    \end{macrocode}

%\iffalse
%</package>
%\fi
%
\endinput
|\\
|\childdocmain{}|\\
\end{tabular}
\end{center}
at the very top of the main \LaTeX{} file,
in particular \emph{before} the |\documentclass| statement!
The argument of |\childdocmain| should be left empty
(but it must be present).

%%%%%%%%%%%%%%%%%%%%%%%%%%%%%%%%%%%%%%%%
\DescribeMacro{\childdocof}
Furthermore, add the commands
\begin{center}
\begin{tabular}{l}
|% \iffalse
%
% childdoc.dtx Copyright (C) 2017-2018 Niklas Beisert
%
% This work may be distributed and/or modified under the
% conditions of the LaTeX Project Public License, either version 1.3
% of this license or (at your option) any later version.
% The latest version of this license is in
%   http://www.latex-project.org/lppl.txt
% and version 1.3 or later is part of all distributions of LaTeX
% version 2005/12/01 or later.
%
% This work has the LPPL maintenance status `maintained'.
%
% The Current Maintainer of this work is Niklas Beisert.
%
% This work consists of the files childdoc.dtx and childdoc.ins
% and the derived files childdoc.def and cdocsamp.tex with
% cdocsch1.tex, cdocsch2.tex, cdocsdrf.tex, cdocsfn1.tex, cdocsfn2.tex.
%
%<package>\ifdefined\childdocmain\endinput\fi
%<package>\ProvidesFile{childdoc.def}[2018/12/30 v2.0 child document driver]
%<samplemain>\ProvidesFile{cdocsamp.tex}[2018/12/30 v2.0 sample for childdoc]
%<*driver>
%\ProvidesFile{childdoc.drv}[2018/12/30 v2.0 childdoc reference manual file]
\PassOptionsToClass{10pt,a4paper}{article}
\documentclass{ltxdoc}

\usepackage[margin=35mm]{geometry}
\usepackage{hyperref}
\usepackage{hyperxmp}
\usepackage[usenames]{color}

\hypersetup{colorlinks=true}
\hypersetup{pdfstartview=FitH}
\hypersetup{pdfpagemode=UseNone}
\hypersetup{pdfsource={}}
\hypersetup{pdflang={en-UK}}
\hypersetup{pdfcopyright={Copyright 2017-2018 Niklas Beisert.
  This work may be distributed and/or modified under the
  conditions of the LaTeX Project Public License, either version 1.3
  of this license or (at your option) any later version.}}
\hypersetup{pdflicenseurl={http://www.latex-project.org/lppl.txt}}
\hypersetup{pdfcontactaddress={ETH Zurich, ITP, HIT K,
  Wolfgang-Pauli-Strasse 27}}
\hypersetup{pdfcontactpostcode={8093}}
\hypersetup{pdfcontactcity={Zurich}}
\hypersetup{pdfcontactcountry={Switzerland}}
\hypersetup{pdfcontactemail={nbeisert@itp.phys.ethz.ch}}
\hypersetup{pdfcontacturl={http://people.phys.ethz.ch/\xmptilde nbeisert/}}

\newcommand{\secref}[1]{\hyperref[#1]{section \ref*{#1}}}

\parskip1ex
\parindent0pt
\let\olditemize\itemize
\def\itemize{\olditemize\parskip0pt}

\begin{document}

\title{The \textsf{childdoc} Package}
\hypersetup{pdftitle={The childdoc Package}}
\author{Niklas Beisert\\[2ex]
  Institut f\"ur Theoretische Physik\\
  Eidgen\"ossische Technische Hochschule Z\"urich\\
  Wolfgang-Pauli-Strasse 27, 8093 Z\"urich, Switzerland\\[1ex]
  \href{mailto:nbeisert@itp.phys.ethz.ch}
  {\texttt{nbeisert@itp.phys.ethz.ch}}}
\hypersetup{pdfauthor={Niklas Beisert}}
\hypersetup{pdfsubject={Manual for the LaTeX2e Package childdoc}}
\date{30 December 2018, \textsf{v2.0}}
\maketitle

\begin{abstract}\noindent
\textsf{childdoc} is a \LaTeXe{} package
that enables the direct compilation
of document sections included by |\include|
to individual files.
\end{abstract}

\begingroup
\parskip0ex
\tableofcontents
\endgroup

%%%%%%%%%%%%%%%%%%%%%%%%%%%%%%%%%%%%%%%%%%%%%%%%%%%%%%%%%%%%%%%%%%%%%%%%%%%%%%%%
%%%%%%%%%%%%%%%%%%%%%%%%%%%%%%%%%%%%%%%%%%%%%%%%%%%%%%%%%%%%%%%%%%%%%%%%%%%%%%%%
\section{Introduction}

\LaTeX{} provides a mechanism to structure a large document (such as a book)
into a main file and several child files (containing the chapters)
using the |\include| command.
This mechanism is beneficial for documents
which span hundreds of pages in order to
make the source file(s) more manageable.
Moreover, compilation can be restricted to
selected child files by means of the |\includeonly| command.
The latter feature can be used to reduce the compilation time while editing
(this was significantly more useful in the earlier days of \LaTeX{})
or to generate a smaller document which is easier to navigate.
Another application of |\includeonly| is to generate
documents consisting of selected parts of the complete document.

However, there are a few drawbacks of the plain |\include| mechanism:
\begin{itemize}
\item
The child files cannot be compiled on their own,
they can only be compiled via the main file.
A naive editing environment
(such as a text editor with an option
to have the current file processed by \LaTeX)
may require one to switch to the main file before compiling;
attempting to compile the child file produces errors.
\item
The main file must be modified (each time)
to adjust the |\includeonly| command
to the present needs. This easily leaves the main file in a messy state.
\item
The generated document will always carry the filename
of the main document. This is inconvenient if
several child files are to be compiled and
to be kept for distribution.
\end{itemize}

The present package provides a simple interface
to make child files individually compilable by \LaTeX{}.
Compiling a child file then has the same effect as compiling
the main file with an |\includeonly| command
to select the appropriate child.
Moreover the generated document will carry the name of the child
rather than the main file.
This resolves all three above issues.

This feature is meant to make the editing of books,
thesis documents and lecture notes somewhat more convenient.
However, the package can also be used efficiently for
composing a series of documents (such as exercise sheets)
which are typically distributed individually.
It then assists the author in generating the individual documents
(potentially in different versions)
as well as a document containing the collected series.
Another application is in developing style files
or other kinds of included material
where compilation of the style file could redirect
to a sample or test file.

%%%%%%%%%%%%%%%%%%%%%%%%%%%%%%%%%%%%%%%%%%%%%%%%%%%%%%%%%%%%%%%%%%%%%%%%%%%%%%%%
%%%%%%%%%%%%%%%%%%%%%%%%%%%%%%%%%%%%%%%%%%%%%%%%%%%%%%%%%%%%%%%%%%%%%%%%%%%%%%%%
\section{Usage}

First of all, the package \textsf{childdoc} is \emph{not} a standard
\LaTeXe{} |.sty| style file! Therefore it needs to be invoked in
a non-standard way.

%%%%%%%%%%%%%%%%%%%%%%%%%%%%%%%%%%%%%%%%%%%%%%%%%%%%%%%%%%%%%%%%%%%%%%%%%%%%%%%%
\subsection{Included Files}
\label{sec:include}

%%%%%%%%%%%%%%%%%%%%%%%%%%%%%%%%%%%%%%%%
\DescribeMacro{\childdocmain}
To use the package, add the commands
\begin{center}
\begin{tabular}{l}
|\input{childdoc.def}|\\
|\childdocmain{}|\\
\end{tabular}
\end{center}
at the very top of the main \LaTeX{} file,
in particular \emph{before} the |\documentclass| statement!
The argument of |\childdocmain| should be left empty
(but it must be present).

%%%%%%%%%%%%%%%%%%%%%%%%%%%%%%%%%%%%%%%%
\DescribeMacro{\childdocof}
Furthermore, add the commands
\begin{center}
\begin{tabular}{l}
|\input{childdoc.def}|\\
|\childdocof{|\textit{main}|}|\\
\end{tabular}
\end{center}
at the top of every child file \textit{child}
which is included by |\include{|\textit{child}|}|
from within the main file
(or at least for those files to be compiled individually).
The argument \textit{main} must be the filename of the main file.

There are a couple of
considerations in setting up the main and child documents:

%%%%%%%%%%%%%%%%%%%%%%%%%%%%%%%%%%%%%%%%
\paragraph{Restrictions.}

Please note the following restrictions:
\begin{itemize}
\item
|\childdocmain| must be called with one argument \textit{main}
to ensure compatibility with earlier version of the package.
It must either be empty (|\childdocmain{}|)
or precisely match the filename of the main file in which it is specified.
See \secref{sec:detection} for further information.
\item
The filename \textit{main} must be specified without the |.tex| extension.
\item
The filename \textit{main} is case sensitive
(even in case-insensitive file systems)
due to internal string comparison.
\item
The argument \textit{main} should be fully expanded, it cannot be a macro.
\item
Subdirectories and special characters should be avoided in filenames.
\item
The command |\childdocmain{|\textit{main}|}| must be followed by a whitespace.
It should not be followed immediately by another command
or by a comment mark `|%|'.
This is because the \TeX{} parser reads the token immediately following
the argument of |\childdocmain| and puts it
at the beginning of every child section;
however, a white\-space is ignored.
\end{itemize}

%%%%%%%%%%%%%%%%%%%%%%%%%%%%%%%%%%%%%%%%
\paragraph{Content of Main File.}

It is advisable to place all content in the child files included by |\include|.
Any output contained in the main file will appear in all child documents
unless suppressed manually;
it cannot be suppressed automatically by the |\includeonly| directive
and thus should normally be avoided.
A method to include some content in the main file
by means of conditional processing is described in \secref{sec:conditional}.

%%%%%%%%%%%%%%%%%%%%%%%%%%%%%%%%%%%%%%%%
\paragraph{Page Numbering.}

When only a part of the document is compiled,
the appropriate numbering of pages
(as well as other status parameters)
is determined from the |.aux| files.
The latter contain information from previous passes.
However this information needs to propagate through
all intermediate child documents.
Therefore the page numbering in child documents may well
be inconsistent until the complete document is compiled at least once.

A useful (if unconventional) way to always ensure a consistent
page numbering is to restart the numbering in each child document
and denote the pages by `\textit{child}|.|\textit{page}'
where \textit{child} represents the chapter/section number of the child file.
This can be achieved by the command
|\numberwithin{page}{|\textit{child}|}|
of the \textsf{amsmath} package
where \textit{child} can be |chapter| or |section|
depending on the chosen structuring.
Alternatively, one can modify the macro |\thepage| appropriately
and reset the counter |page| at the start of each child file.

%%%%%%%%%%%%%%%%%%%%%%%%%%%%%%%%%%%%%%%%%%%%%%%%%%%%%%%%%%%%%%%%%%%%%%%%%%%%%%%%
\subsection{Conditional Processing}
\label{sec:conditional}

The package provides a mechanism to compile different versions
of a document. To customise the versions further some conditional processing
can come in handy to distinguish which version is being compiled.
The package provides two macros to describe the compilation context:

%%%%%%%%%%%%%%%%%%%%%%%%%%%%%%%%%%%%%%%%
\DescribeMacro{\ifchilddoc}
The conditional |\ifchilddoc| distinguishes between the compilation of
child documents and the main document:
%
\begin{center}
|\ifchilddoc |\textit{child-code}| |[|\||else |\textit{main-code}]| \||fi|
\end{center}

%%%%%%%%%%%%%%%%%%%%%%%%%%%%%%%%%%%%%%%%
\DescribeMacro{\childdocname}
\DescribeMacro{\childdocjob}
The macro |\childdocname| contains the filename (without extension)
of the main or child file being processed.
Note that |\childdocjob| will always contain the name of the main file.

%%%%%%%%%%%%%%%%%%%%%%%%%%%%%%%%%%%%%%%%
\paragraph{Title Page.}

Conditional processing can be used to include a title or banner page
in the main document when proper precautions are taken.
Importantly, the code in the main file should ensure that the page counter
(as well as other status parameters which are stored in the |.aux| files)
takes the same value after the conditional processing.
Otherwise the page numbers may take divergent values
depending on which part is compiled.

For example, a title page could be declared by:
%
\begin{center}
\begin{tabular}{l}
|\ifchilddoc\||else|\\
|\addtocounter{page}{-1}|\\
\textit{code for title page}\\
|\newpage|\\
|\||fi|
\end{tabular}
\end{center}
%
A banner page for the child documents can be generated by:
%
\begin{center}
\begin{tabular}{l}
|\ifchilddoc|\\
|\addtocounter{page}{-1}|\\
\textit{code for banner page}\\
|\newpage|\\
|\||fi|
\end{tabular}
\end{center}
%
Here one could write a message such as:
\begin{center}
|This is the part \childdocname{} of \childdocjob{}.|
\end{center}

%%%%%%%%%%%%%%%%%%%%%%%%%%%%%%%%%%%%%%%%%%%%%%%%%%%%%%%%%%%%%%%%%%%%%%%%%%%%%%%%
\subsection{Flags}
\label{sec:flags}

The package makes it easy to generate different versions
of the main or child documents.
To this end compilation flags can be defined
and assigned different default values.
They will be particularly useful in conjunction
with the forwarding mechanism described in \secref{sec:forward}.

For example, it may be useful to have a flag |\version|
which can be set to |draft| or |final|.
The document source will contain some conditional code
depending on the value of |\version|.
Suppose further, the flag should default to |final| for the main file
and to |draft| for child files
which is a natural assignment for editing the document.
This is achieved by placing the following code
in the preamble of the main document
(below the |\childdocmain| directive):
%
\begin{center}
\begin{tabular}{l}
|\ifchilddoc|\\
|\providecommand{\version}{draft}|\\
|\||else|\\
|\providecommand{\version}{final}|\\
|\||fi|
\end{tabular}
\end{center}
%
The definition by |\providecommand| makes sure
that previous definitions are not overwritten.
Further statements |\providecommand{\version}{...}|
can thus be added before the above code to override it.

For the main file, one might add a line
(between |\childdocmain| and the above block)
%
\begin{center}
|%\ifchilddoc\||else\providecommand{\version}{draft}\||fi|
\end{center}
%
which can be uncommented to produce a draft version.
Likewise one can add a line to the very top of a child file
(above the |\childdocof{|\textit{main}|}| directive)
%
\begin{center}
|%\providecommand{\version}{final}|
\end{center}
%
which can be uncommented to produce the final version of this child document.

%%%%%%%%%%%%%%%%%%%%%%%%%%%%%%%%%%%%%%%%%%%%%%%%%%%%%%%%%%%%%%%%%%%%%%%%%%%%%%%%
\subsection{Forwarding}
\label{sec:forward}

Different versions of the main or child documents
using compilation flags as described in \secref{sec:flags}
can be (permanently) stored in different files
for convenient compilation, viewing and distribution.
To this end, the package defines a command
to pass on compilation to a different file:

%%%%%%%%%%%%%%%%%%%%%%%%%%%%%%%%%%%%%%%%
\DescribeMacro{\childdocforward}
The command |\childdocforward| redirects processing to
another source file:
%
\begin{center}
\begin{tabular}{l}
|\input{childdoc.def}|\\
|\childdocforward[|\textit{main}|]{|\textit{dest}|}|\\
\end{tabular}
\end{center}
%
The argument \textit{dest} is the destination file
(without extension).
It should be the main file or one of the child files.
Note that further \textsf{childdoc} directives
such as |\childdocof| and |\childdocforward|
in the indicated file will be processed in this form.
The optional argument \textit{main}
passes on directly to the main file \textit{main}
while pretending to compile the child \textit{dest}.
This form behaves as if \textit{dest}
issues |\childdocof{|\textit{main}|}| right away,
and no further \textsf{childdoc} directives will be processed.

%%%%%%%%%%%%%%%%%%%%%%%%%%%%%%%%%%%%%%%%
\DescribeMacro{\...prefix}
In the alternative form |\childdocforwardprefix|,
%
\begin{center}
\begin{tabular}{l}
|\input{childdoc.def}|\\
|\childdocforwardprefix[|\textit{main}|]{|\textit{prefix}|}{|\textit{dest}|}|
\end{tabular}
\end{center}
%
the destination file is determined by a pattern
depending on the current file:
To make this work, the current file must be called
`{\textit{prefix}\hspace{0.2em}\textit{suffix}}'
with \textit{prefix} matching precisely the argument.
Processing is then passed on to the file
`{\textit{dest}\hspace{0.2em}\textit{suffix}}'.
Surely, the same effect is achieved by
directly specifying the
argument `{\textit{dest}\hspace{0.2em}\textit{suffix}}'
in the first form.
However, that requires to set up a different file
for each child. With the alternative form of the command
all these files can have exactly the same content
which simplifies setting them up and maintaining them.

For example, the following file |draft.tex|
with a compilation flag |\version| as described in \secref{sec:flags}
compiles the main document as a draft:
%
\begin{center}
\begin{tabular}{l}
|\def\version{draft}|\\
|\input{childdoc.def}|\\
|\childdocforward{|\textit{main}|}|
\end{tabular}
\end{center}
%
Likewise, the following files |final|\textit{nn}|.tex|
compile the final version of the child document
|child|\textit{nn}|.tex|:
%
\begin{center}
\begin{tabular}{l}
|\def\version{final}|\\
|\input{childdoc.def}|\\
|\childdocforwardprefix{final}{child}|
\end{tabular}
\end{center}
%

Note that when several versions of a main file and/or of each child file
are to be generated, it may be convenient to set up a |Makefile| or
shell script to automatise the process.

%%%%%%%%%%%%%%%%%%%%%%%%%%%%%%%%%%%%%%%%%%%%%%%%%%%%%%%%%%%%%%%%%%%%%%%%%%%%%%%%
\subsection{Command Line Processing}
\label{sec:commandline}

The effect of redirection files can also be achieved by invoking
the \LaTeX{} compiler with a more elaborate command line.
Most conveniently this should be done as part
of a shell script or a |Makefile|.

When using \textsf{childdoc} in the main file, the following
command lines effectively perform a redirection
(note that depending on the shell being used,
backslashes may have to be doubled: `|\|' $\to$ `|\\|'):
%
\begin{center}
|... -jobname "|\textit{target}|" |\\|"|[\textit{flags}]%
|\input{childdoc.def}\childdocforward[|\textit{main}|]{|\textit{dest}|}"|
\end{center}
%
Here \textit{target} is the name of the output file,
\textit{main} is the name of the main file
and \textit{dest} is the name of the main or child file to be processed
(all filenames without extensions).
The optional argument \textit{main} can be omitted
if \textit{main} matches \textit{dest}.
Optionally, compilation \textit{flags} can be defined via |\def| commands.
This command line makes the \TeX{} engine believe
it is compiling the file \textit{target}
whose content is specified as the latter parameter.
The provided code then forwards the processing to
\textit{main} or \textit{dest} as described in \secref{sec:forward}.

%%%%%%%%%%%%%%%%%%%%%%%%%%%%%%%%%%%%%%%%%%%%%%%%%%%%%%%%%%%%%%%%%%%%%%%%%%%%%%%%
\subsection{Include by Input}
\label{sec:input}

Including child documents by |\include| has some restrictions by design.
Most notably, the content of a child document always occupies
its own set of pages; pages cannot be shared between child documents.
Usually, this behaviour makes perfect sense
because each child document contain an essential part of the document.
However, in some situations it may be desirable to compose
a document from a collection of parts
without having mandatory page breaks between then.
For this case, the package
provides a mechanism to include parts
by |\input| which can also be processed individually.
However, by construction this mechanism
requires manual handling of the content to be output.

%%%%%%%%%%%%%%%%%%%%%%%%%%%%%%%%%%%%%%%%
\DescribeMacro{\ifchilddocmanual}
The main file should be prepared as usual, see \secref{sec:include}.
However, the document body must make a distinction
between processing of an individual part and of the main document, e.g.:
%
\begin{center}
\begin{tabular}{l}
|\ifchilddocmanual|\\
|\input{\childdocname}|\\
|\||else|\\
\textit{document body with }|\input{|\textit{part}|}|\\
|\||fi|
\end{tabular}
\end{center}
%
The conditional |\ifchilddocmanual| is true whenever
a part to be included by |\input| is being compiled,
and the name of the part is stored in |\childdocname|.

%%%%%%%%%%%%%%%%%%%%%%%%%%%%%%%%%%%%%%%%
\DescribeMacro{\childdocby}
Each part to be included by |\input| should start with:
%
\begin{center}
\begin{tabular}{l}
|\input{childdoc.def}|\\
|\childdocby{|\textit{main}|}|\\
\end{tabular}
\end{center}
%
The directive |\childdocby| is similar to |\childdocof|
described in \secref{sec:include},
but the subsequent selection of content must be done manually.
To that end, both |\ifchilddoc| and |\ifchilddocmanual|
will be true upon processing of a part,
and the name of the part is stored in |\childdocname|.
Note that |\jobname| will be set to the filename of the current part
so that each part receives an individual |.aux| file
that does not interfere with the |.aux| file(s) of the main document.
This behaviour can be altered by the alternative form
|\childdocby[*]{|\textit{main}|}| (with a non-empty optional argument)
which uses the |.aux| file of the main document
by setting |\jobname| to \textit{main}.

%%%%%%%%%%%%%%%%%%%%%%%%%%%%%%%%%%%%%%%%%%%%%%%%%%%%%%%%%%%%%%%%%%%%%%%%%%%%%%%%
\subsection{Driver Development}
\label{sec:driver}

The \textsf{childdoc} mechanism can also be use for the development
of definition files such as \LaTeX{} styles or classes.
This case differs from the above setup with multiple parts
included by |\include| in that no |\includeonly| should be invoked.
This can be achieved by starting the include file
(before |\ProvidesPackage|) with:
%
\begin{center}
\begin{tabular}{l}
|\input{childdoc.def}|\\
|\childdocforward{|\textit{main}|}|\\
\end{tabular}
\end{center}
%
or alternatively with:
%
\begin{center}
\begin{tabular}{l}
|\input{childdoc.def}|\\
|\childdocby{|\textit{main}|}|\\
\end{tabular}
\end{center}
%
Both forms have slightly different effects as described above.
The main file is prepared as usual, see \secref{sec:include}.

%%%%%%%%%%%%%%%%%%%%%%%%%%%%%%%%%%%%%%%%%%%%%%%%%%%%%%%%%%%%%%%%%%%%%%%%%%%%%%%%
\subsection{Legacy Detection}
\label{sec:detection}

The directive |\childdocmain| in the main file can detect
whether the complete document or merely a child is to be compiled
even without using the directive |\childdocof|.
This method is deprecated because it is less robust
and there is no compelling reason to use it;
it is merely provided for backward compatibility
and it may be removed in future versions.

If the detection mechanism is to be used,
it is mandatory to correctly specify
the filename of the main file as the argument of |\childdocmain|:
%
\begin{center}
\begin{tabular}{l}
|\input{childdoc.def}|\\
|\childdocmain{|\textit{main}|}|\\
\end{tabular}
\end{center}
%
If |\jobname| does not match the argument \textit{main} of |\childdocmain|,
it is assumed that |\jobname| points to the child file to be compiled.
When using |\childdocmain| with the main file specified as argument,
it suffices to start a child file
with just |\input{|\textit{main}|}|
without loading of the package and using |\childdocof|.
If instead all processing is done
with the appropriate \textsf{childdoc} directives,
the argument of \textit{main} of |\childdocmain| can be empty.

An alternative version of the command line processing described
in \secref{sec:commandline} using the detection mechanism reads:
%
\begin{center}
|... -jobname "|\textit{target}|" "|[\textit{flags}]%
[|\def\jobname{|\textit{dest}|}|]|\input{|\textit{main}|}"|
\end{center}

%%%%%%%%%%%%%%%%%%%%%%%%%%%%%%%%%%%%%%%%%%%%%%%%%%%%%%%%%%%%%%%%%%%%%%%%%%%%%%%%
\subsection{Manual Code}
\label{sec:manual}

In case one cannot be certain whether the definitions file |childdoc.def|
is installed on the target \TeX{} distribution
and one prefers not to ship it,
it is conceivable to paste a few relevant commands into the sources.

To that end, drop all statements |\input{childdoc.def}|
and perform the replacements as outlined below.
Instead of |\childdocmain{|\textit{main}|}| add the following code
to the top of the main file:
%
\begin{center}
\begin{tabular}{l}
|\||ifdefined\childdocname\endinput\||fi\newif\ifchilddoc|\\
|\edef\childdocname{\scantokens\expandafter{\jobname\noexpand}}|\\
|\def\childdocmain{|\textit{main}|}\||ifx\childdocmain\childdocname\||else|\\
|\childdoctrue\includeonly{\childdocname}\let\jobname\childdocmain\||fi|\\
\end{tabular}
\end{center}
%
Instead of |\childdocof{|\textit{main}|}| just include the main file
at the top of each child file:
%
\begin{center}
|\input{|\textit{main}|}|
\end{center}
%
A simple redirection |\childdocforward{|\textit{dest}|}| is achieved by:
%
\begin{center}
|\def\jobname{|\textit{dest}|}\input{\jobname}|
\end{center}
%
The redirection with prefix
|\childdocforwardprefix[|\textit{prefix}|]{|\textit{dest}|}|
is accomplished by:
%
\begin{center}
\begin{tabular}{l}
|{\edef\jobname{\scantokens\expandafter{\jobname\noexpand}}|\\
|\def\redirectjob |\textit{prefix}|#1~~~{\gdef\jobname{|\textit{dest}|#1}}|\\
|\expandafter\redirectjob\jobname~~~}\input{\jobname}|
\end{tabular}
\end{center}

In an alternative approach,
child documents can be compiled by a specific command line
without additional code or specific definitions:
%
\begin{center}
|... -jobname "|\textit{target}|" "|[\textit{flags}]%
|\includeonly{|\textit{dest}|}\input{|\textit{main}|}"|
\end{center}
%

%%%%%%%%%%%%%%%%%%%%%%%%%%%%%%%%%%%%%%%%%%%%%%%%%%%%%%%%%%%%%%%%%%%%%%%%%%%%%%%%
%%%%%%%%%%%%%%%%%%%%%%%%%%%%%%%%%%%%%%%%%%%%%%%%%%%%%%%%%%%%%%%%%%%%%%%%%%%%%%%%
\section{Information}

%%%%%%%%%%%%%%%%%%%%%%%%%%%%%%%%%%%%%%%%%%%%%%%%%%%%%%%%%%%%%%%%%%%%%%%%%%%%%%%%
\subsection{Copyright}

Copyright \copyright{} 2017--2018 Niklas Beisert

This work may be distributed and/or modified under the
conditions of the \LaTeX{} Project Public License, either version 1.3
of this license or (at your option) any later version.
The latest version of this license is in
  \url{http://www.latex-project.org/lppl.txt}
and version 1.3 or later is part of all distributions of \LaTeX{}
version 2005/12/01 or later.

This work has the LPPL maintenance status `maintained'.

The Current Maintainer of this work is Niklas Beisert.

This work consists of the files |README.txt|, |childdoc.ins| and |childdoc.dtx|
as well as the derived files |childdoc.def|, |cdocsamp.tex|
with |cdocsch1.tex|, |cdocsch2.tex|, |cdocspt3.tex|, |cdocspt4.tex|,
|cdocsdrf.tex|, |cdocsfn1.tex|, |cdocsfn2.tex|
as well as |childdoc.pdf|.

%%%%%%%%%%%%%%%%%%%%%%%%%%%%%%%%%%%%%%%%%%%%%%%%%%%%%%%%%%%%%%%%%%%%%%%%%%%%%%%%
\subsection{Files and Installation}

The package consists of the files:
%
\begin{center}
\begin{tabular}{ll}
    |README.txt|   & readme file \\
    |childdoc.ins| & installation file \\
    |childdoc.dtx| & source file \\
    |childdoc.def| & definition file \\
    |cdocsamp.tex| & sample main file \\
    |cdocsch1.tex| & sample include file \\
    |cdocsch2.tex| & sample include file \\
    |cdocspt3.tex| & sample part file \\
    |cdocspt4.tex| & sample part file \\
    |cdocsdrf.tex| & sample redirection file \\
    |cdocsfn1.tex| & sample redirection file \\
    |cdocsfn2.tex| & sample redirection file \\
    |childdoc.pdf| & manual
\end{tabular}
\end{center}
%
The distribution consists of the files
|README.txt|, |childdoc.ins| and |childdoc.dtx|.
%
\begin{itemize}
\item
Run (pdf)\LaTeX{} on |childdoc.dtx|
to compile the manual |childdoc.pdf| (this file).
\item
Run \LaTeX{} on |childdoc.ins| to create the definitions file |childdoc.def|
and the sample |cdocsamp.tex| with include files
|cdocsch1.tex|, |cdocsch2.tex|, |cdocspt3.tex|, |cdocspt4.tex|,
|cdocsdrf.tex|, |cdocsfn1.tex|, |cdocsfn2.tex|.
Then copy the file |childdoc.def| to an appropriate directory of your \LaTeX{}
distribution, e.g.\ \textit{texmf-root}|/tex/latex/childdoc|.
\end{itemize}

%%%%%%%%%%%%%%%%%%%%%%%%%%%%%%%%%%%%%%%%%%%%%%%%%%%%%%%%%%%%%%%%%%%%%%%%%%%%%%%%
\subsection{Related CTAN Packages}

There are several other packages which offer a similar functionality:
%
\begin{itemize}
\item
The packages
\href{http://ctan.org/pkg/docmute}{\textsf{docmute}},
\href{http://ctan.org/pkg/includex}{\textsf{includex}} and
\href{http://ctan.org/pkg/standalone}{\textsf{standalone}}
provide commands to include only the document body of
a child file thus allowing both files to be compiled individually.
\item
The packages \href{http://ctan.org/pkg/subdocs}{\textsf{subdocs}}
and \href{http://ctan.org/pkg/subfiles}{\textsf{subfiles}}
provide structures in which the main and child documents can be
encapsulated and allowing them to be compiled individually.
The inclusion mechanism is different from the conventional |\include|.
\item
The package \href{http://ctan.org/pkg/combine}{\textsf{combine}}
is an elaborate solution to combine several documents into one.
\end{itemize}
%
See also the CTAN topic \href{http://ctan.org/topic/subdocs}{\textsf{subdocs}}
for further related packages.
The present package differs from the above solutions in that
a document structure constructed with the conventional |\include| mechanism
just needs two extra commands at the top of every file
such that all constituent files can be compiled individually.

%%%%%%%%%%%%%%%%%%%%%%%%%%%%%%%%%%%%%%%%%%%%%%%%%%%%%%%%%%%%%%%%%%%%%%%%%%%%%%%%
%\subsection{Feature Suggestions}
%
%The following is a list of features which may be useful for future
%versions of this package:
%%
%\begin{itemize}
%\item
%\ldots
%\end{itemize}

%%%%%%%%%%%%%%%%%%%%%%%%%%%%%%%%%%%%%%%%%%%%%%%%%%%%%%%%%%%%%%%%%%%%%%%%%%%%%%%%
\subsection{Revision History}

%%%%%%%%%%%%%%%%%%%%%%%%%%%%%%%%%%%%%%%%
\paragraph{v2.0:} 2018/12/30

\begin{itemize}
\item
immediate forward processing
\item
added |\childdocby| mechanism
\item
manual restructured
\end{itemize}

%%%%%%%%%%%%%%%%%%%%%%%%%%%%%%%%%%%%%%%%
\paragraph{v1.6:} 2018/01/17

\begin{itemize}
\item
application for development of include files
\item
corrections to manual
\end{itemize}

%%%%%%%%%%%%%%%%%%%%%%%%%%%%%%%%%%%%%%%%
\paragraph{v1.5:} 2017/05/21

\begin{itemize}
\item
more complete structuring introduced
\item
|\childdocof| introduced
\item
|\childdoc| renamed to |\childdocmain|
\item
|\childredirect| renamed to |\childdocforward| and |\childdocforwardprefix|
and functionality expanded
\end{itemize}

%%%%%%%%%%%%%%%%%%%%%%%%%%%%%%%%%%%%%%%%
\paragraph{v1.0:} 2017/04/27

\begin{itemize}
\item
manual and install package
\item
first version published on CTAN
\end{itemize}

%%%%%%%%%%%%%%%%%%%%%%%%%%%%%%%%%%%%%%%%
\paragraph{v0.6:} 2017/04/26

\begin{itemize}
\item
redirection mechanism added
\end{itemize}

%%%%%%%%%%%%%%%%%%%%%%%%%%%%%%%%%%%%%%%%
\paragraph{v0.5:} 2017/04/26

\begin{itemize}
\item
functionality in definition file
\end{itemize}


%%%%%%%%%%%%%%%%%%%%%%%%%%%%%%%%%%%%%%%%%%%%%%%%%%%%%%%%%%%%%%%%%%%%%%%%%%%%%%%%
%%%%%%%%%%%%%%%%%%%%%%%%%%%%%%%%%%%%%%%%%%%%%%%%%%%%%%%%%%%%%%%%%%%%%%%%%%%%%%%%
%%%%%%%%%%%%%%%%%%%%%%%%%%%%%%%%%%%%%%%%%%%%%%%%%%%%%%%%%%%%%%%%%%%%%%%%%%%%%%%%
\appendix

\settowidth\MacroIndent{\rmfamily\scriptsize 000\ }

 \DocInput{childdoc.dtx}

\end{document}
%</driver>
% \fi
%
% %%%%%%%%%%%%%%%%%%%%%%%%%%%%%%%%%%%%%%%%%%%%%%%%%%%%%%%%%%%%%%%%%%%%%%%%%%%%%%
% %%%%%%%%%%%%%%%%%%%%%%%%%%%%%%%%%%%%%%%%%%%%%%%%%%%%%%%%%%%%%%%%%%%%%%%%%%%%%%
% \section{Sample}
%\iffalse
%<*samplemain>
%\fi
%
% The following presents a sample document
% with two chapters, two parts, a title page,
% a compile flag as well as three forwarding files to set the flag.
% It consists of eight |.tex| files:
% \begin{center}
% \begin{tabular}{ll}
% |cdocsamp.tex|&main file\\
% |cdocsch1.tex|&include file for chapter 1\\
% |cdocsch2.tex|&include file for chapter 2\\
% |cdocspt3.tex|&include file for part 3\\
% |cdocspt4.tex|&include file for part 4\\
% |cdocsdrf.tex|&forwarding file for main file in draft mode\\
% |cdocsfi1.tex|&forwarding file for final version of chapter 1\\
% |cdocsfi2.tex|&forwarding file for final version of chapter 2\\
% \end{tabular}
% \end{center}
% Each of the eight files can be compiled directly by the \LaTeX{} compiler.
%
% %%%%%%%%%%%%%%%%%%%%%%%%%%%%%%%%%%%%%%
% \paragraph{Main File.}
%
% The main file is called |cdocsamp.tex|.
%
% Load the \textsf{childdoc} definitions and
% declare the filename for the main document:
%    \begin{macrocode}
\input{childdoc.def}
\childdocmain{}
%    \end{macrocode}

% Optional override for |\version| flag:
%    \begin{macrocode}
%%\ifchilddoc\else\providecommand{\version}{draft}\fi
%    \end{macrocode}

% Define the default values for the |\version| flag
% (|final| for the main file and |draft| for childs):
%    \begin{macrocode}
\ifchilddoc
\providecommand{\version}{draft}
\else
\providecommand{\version}{final}
\fi
%    \end{macrocode}

% Load the standard document class:
%    \begin{macrocode}
\documentclass[12pt]{article}
%    \end{macrocode}

% Start the document body:
%    \begin{macrocode}
\begin{document}
%    \end{macrocode}

% Declare a title page.
% Print title, part of document being processed and version flag:
%    \begin{macrocode}
\addtocounter{page}{-1}
\begin{center}
{\LARGE\bfseries{}childdoc example\par}
\vspace{1cm}
\ifchilddoc
\ifchilddocmanual part\else chapter\fi:
`\childdocname' of `\childdocjob'\par
\else
main document: `\childdocjob'\par
\fi
version: \version\par
\end{center}
\newpage
%    \end{macrocode}

% Manually include selected file,
% otherwise process as usual:
%    \begin{macrocode}
\ifchilddocmanual
\section*{part `\childdocname'}
\input{\childdocname}
\else
%    \end{macrocode}

% Include the two chapters:
%    \begin{macrocode}
\include{cdocsch1}
\include{cdocsch2}
%    \end{macrocode}

% Include the two parts unless only chapters should be displayed:
%    \begin{macrocode}
\ifchilddoc\else
\section{part three}
\input{cdocspt3}
\section{part four}
\input{cdocspt4}
\fi
%    \end{macrocode}

% Process as usual until here:
%    \begin{macrocode}
\fi
%    \end{macrocode}

% End of document body:
%    \begin{macrocode}
\end{document}
%    \end{macrocode}
%\iffalse
%</samplemain>
%\fi
%
% %%%%%%%%%%%%%%%%%%%%%%%%%%%%%%%%%%%%%%
% \paragraph{Chapter Include Files.}
%
% The include files are called |cdocsch1.tex| and |cdocsch2.tex|.
%
%\iffalse
%<*samplechap1|samplechap2>
%\fi

% Optional override for |\version| flag:
%    \begin{macrocode}
%%\providecommand{\version}{final}
%    \end{macrocode}

% Include the main document:
%    \begin{macrocode}
\input{childdoc.def}
\childdocof{cdocsamp}
%    \end{macrocode}

%\iffalse
%</samplechap1|samplechap2>
%\fi
%
%\iffalse
%<*samplechap1>
%\fi
% Some text for chapter 1:
%    \begin{macrocode}
\section{one}
some text in chapter one
%    \end{macrocode}

%\iffalse
%</samplechap1>
%\fi
% Some text for chapter 2:
%\iffalse
%<*samplechap2>
%\fi
%    \begin{macrocode}
\section{two}
more text in chapter two
%    \end{macrocode}

%\iffalse
%</samplechap2>
%\fi
%
% %%%%%%%%%%%%%%%%%%%%%%%%%%%%%%%%%%%%%%
% \paragraph{Part Include Files.}
%
% The include files are called |cdocspt3.tex| and |cdocspt4.tex|.
%
%\iffalse
%<*samplepart3|samplepart4>
%\fi

% Optional override for |\version| flag:
%    \begin{macrocode}
%%\providecommand{\version}{final}
%    \end{macrocode}

% Include the main document:
%    \begin{macrocode}
\input{childdoc.def}
\childdocby{cdocsamp}
%    \end{macrocode}

%\iffalse
%</samplepart3|samplepart4>
%\fi
%
%\iffalse
%<*samplepart3>
%\fi
% Some text for part 3:
%    \begin{macrocode}
some text in part three
%    \end{macrocode}

%\iffalse
%</samplepart3>
%\fi
% Some text for part 4:
%\iffalse
%<*samplepart4>
%\fi
%    \begin{macrocode}
more text in part four
%    \end{macrocode}

%\iffalse
%</samplepart4>
%\fi
%
% %%%%%%%%%%%%%%%%%%%%%%%%%%%%%%%%%%%%%%
% \paragraph{Forwarding for a Complete Draft.}
%
% The following forwarding file |cdocsdrf.tex|
% compiles the main document in draft mode:
%\iffalse
%<*sampledraft>
%\fi
%    \begin{macrocode}
\def\version{draft}
\input{childdoc.def}
\childdocforward{cdocsamp}
%    \end{macrocode}

%\iffalse
%</sampledraft>
%\fi
%
% %%%%%%%%%%%%%%%%%%%%%%%%%%%%%%%%%%%%%%
% \paragraph{Forwarding for Final Version of the Chapters.}
%
% The following forwarding files |cdocsfn1.tex| and |cdocsfn2.tex|
% (with identical content)
% compile the final versions of the child documents
% |cdocsch1.tex| and |cdocsch2.tex|, respectively:
%\iffalse
%<*samplefinal>
%\fi
%    \begin{macrocode}
\def\version{final}
\input{childdoc.def}
\childdocforwardprefix[cdocsamp]{cdocsfn}{cdocsch}
%    \end{macrocode}

%\iffalse
%</samplefinal>
%\fi
%
% %%%%%%%%%%%%%%%%%%%%%%%%%%%%%%%%%%%%%%
% \paragraph{Command Line Processing.}
%
% The following three command lines generate the output files
% |cdocscld|, |cdocscl1| and |cdocscl2|
% which should be identical to
% |cdocsdrf|, |cdocsch1| and |cdocsfn2|, respectively:
% \begin{center}
% \begin{tabular}{l}
% |latex -jobname cdocscld \|\\
% |  "\def\version{draft}\input{childdoc.def}\childdocforward{cdocsamp}"|\\
% |latex -jobname cdocscl1 \|\\
% |  "\input{childdoc.def}\childdocforward[cdocsamp]{cdocsch1}"|\\
% |latex -jobname cdocscl2 \|\\
% |  "\def\version{final}\input{childdoc.def}\childdocforward{cdocsch2}"|
% \end{tabular}
% \end{center}
% Note that the trailing backslash on each first line
% merely continues the input to the second line
% (for convenient cut ant paste).
% Furthermore, the command |latex| can be replaced by any
% of its alternative versions such as |pdflatex|.
%
% %%%%%%%%%%%%%%%%%%%%%%%%%%%%%%%%%%%%%%%%%%%%%%%%%%%%%%%%%%%%%%%%%%%%%%%%%%%%%%
% %%%%%%%%%%%%%%%%%%%%%%%%%%%%%%%%%%%%%%%%%%%%%%%%%%%%%%%%%%%%%%%%%%%%%%%%%%%%%%
% \section{Implementation}
%\iffalse
%<*package>
%\fi
%
% This section describes the definitions file |childdoc.def|.

% The definitions cannot be loaded using |\usepackage| or |\RequirePackage|
% which has a mechanism to prevent loading a style file more than once.
% When loading the definitions by means of |\input|
% multiple instances have to be prevented manually:
%\iffalse
%This code needs to be before the `\ProvidesFile' directive
%which is defined at the beginning of this file.
%Therefore it is also placed there and commented out here.
%</package>
%<*discard>
%\fi
%    \begin{macrocode}
\ifdefined\childdocmain\endinput\fi
%    \end{macrocode}
%\iffalse
%</discard>
%<*package>
%\fi
%
% \macro{\ifchilddoc}
% \macro{\ifchilddocmanual}
% The conditional |\ifchilddoc| tells whether a
% child (true) or main (false) document is being compiled.
% The conditional |\ifchilddocmanual| tells whether
% the |\includeonly| mechanism is used (false) or
% the selection of child files must be performed manually (true).
% The definitions initialise to false:
%    \begin{macrocode}
\newif\ifchilddoc
\newif\ifchilddocmanual
%    \end{macrocode}

% \macro{\childdocname}
% \macro{\childdocjob}
% The macro |\childdocname| stores the name of the main document
% to be compiled. The macro |\childdocjob| stores the name of
% the document on which the \LaTeX{} compiler was originally invoked.
% The content of |\jobname| cannot be compared
% to filenames specified in the source due to different catcodes.
% The following code rescans |\jobname|, stores the result
% in |\childdocname| and saves a copy in |\childdocjob|:
%    \begin{macrocode}
\edef\childdocname{\scantokens\expandafter{\jobname\noexpand}}
\let\childdocjob\childdocname
%    \end{macrocode}

% \macro{\childdocdisable}
% The macro |\childdocdisable| prevents the main file
% from being processed more than once.
% At this stage, the main document command |\childdocmain|
% is assumed to be called once again where it should do nothing.
% Any subsequent call to it should prevent
% a secondary processing of the main document
% It overwrites the forwarding commands
% |\childdocof| and |\childdocforward|
% with empty macros to prevent further inclusions of the main document:
%    \begin{macrocode}
\newcommand{\childdocdisable}
{
  \renewcommand{\childdocmain}[1]{\renewcommand{\childdocmain}[1]{\endinput}}
  \renewcommand{\childdocof}[1]{}
  \renewcommand{\childdocby}[2][]{}
  \renewcommand{\childdocforward}[2][]{}
  \renewcommand{\childdocdisable}{}
}
%    \end{macrocode}

% \macro{\childdocmain}
% The macro |\childdocmain| is to be called at the top of the main file
% with nothing or the main filename (without extension) as argument.
% First, it breaks loops.
% If the argument is not empty and does not match |\childdocname|
% (which is set by the first inclusion of |childdoc.def|),
% |\ifchilddoc| is set to true, |\includeonly| is applied to the child file
% and |\jobname| is set to the main file
% (for proper handling of |.aux| files):
%    \begin{macrocode}
\newcommand{\childdocmain}[1]
{
  \childdocdisable\childdocmain{}
  \if?#1?\else
    \begingroup
      \def\childdoctmp{#1}
      \ifx\childdoctmp\childdocname
        \def\childdoctmp{}
      \else
        \def\childdoctmp
        {
          \childdoctrue
          \includeonly{\childdocname}
          \def\childdocjob{#1}
          \def\jobname{#1}
        }
      \fi
      \expandafter
    \endgroup
    \childdoctmp
  \fi
}
%    \end{macrocode}

% \macro{\childdocof}
% The command |\childdocof| redirects
% compilation to the main file |#1|.
%    \begin{macrocode}
\newcommand{\childdocof}[1]
{
  \childdocdisable
  \childdoctrue
  \includeonly{\childdocname}
  \def\jobname{#1}
  \def\childdocjob{#1}
  \input{#1}
}
%    \end{macrocode}

% \macro{\childdocby}
% The command |\childdocby| ....
%    \begin{macrocode}
\newcommand{\childdocby}[2][]
{
  \childdocdisable
  \childdoctrue
  \childdocmanualtrue
  \if?#1?\else
    \def\jobname{#2}
  \fi
  \def\childdocjob{#2}
  \input{#2}
  \endinput
}
%    \end{macrocode}

% \macro{\childdocforward}
% The command |\childdocforward| redirects
% compilation to the main file or
% (if the optional argument is given) a child file.
% Parameters are set as if the main file
% or a child file starting with |\childdocof| was compiled.
% Then compilation is handed over to the main file:
%    \begin{macrocode}
\newcommand{\childdocforward}[2][]
{
  \begingroup
    \if?#1?
      \def\childdoctmp
      {
        \def\childdocname{#2}
        \def\childdocjob{#2}
        \def\jobname{#2}
        \input{#2}
        \endinput
      }
    \else
      \def\childdoctmp
      {
        \childdocdisable
        \def\childdocname{#2}
        \childdoctrue
        \includeonly{#2}
        \def\childdocjob{#1}
        \def\jobname{#1}
        \input{#1}
        \endinput
      }
    \fi
    \expandafter
  \endgroup
  \childdoctmp
}
%    \end{macrocode}

% \macro{\childdocforwardprefix}
% The command |\childdocforwardprefix| redirects
% compilation to the main or a child file by means of a pattern.
% The prefix |#1| in the current filename is replaced by |#2|
% and the suffix of the current filename is kept
% (it is assumed that the filename does not contain the substring `|~~~|'
% which is used as a delimiter).
% Compilation is handed over to the new file by |\childdocforward|:
%    \begin{macrocode}
\newcommand{\childdocforwardprefix}[3][]
{
  \begingroup
    \def\childdocextract #2##1~~~{\def\childdoctmp{\childdocforward[#1]{#3##1}}}
    \expandafter\childdocextract\childdocname~~~
    \expandafter
  \endgroup
  \childdoctmp
}
%    \end{macrocode}

% \macro{\childdoc}
% The deprecated macro |\childdoc| is a legacy version of |\childdocmain|:
%    \begin{macrocode}
\newcommand{\childdoc}{\childdocmain}
%    \end{macrocode}

% \macro{\childdocredirect}
% The deprecated macro |\childdocredirect| is a legacy version
% of |\childdocforward| and |\childdocforwardprefix|:
%    \begin{macrocode}
\newcommand{\childdocredirect}[2][]
{
  \begingroup
    \if?#1?
      \def\childdoctmp{\childdocforward{#2}}
    \else
      \def\childdoctmp{\childdocforwardprefix{#1}{#2}}
    \fi
    \expandafter
  \endgroup
  \childdoctmp
}
%    \end{macrocode}

%\iffalse
%</package>
%\fi
%
\endinput
|\\
|\childdocof{|\textit{main}|}|\\
\end{tabular}
\end{center}
at the top of every child file \textit{child}
which is included by |\include{|\textit{child}|}|
from within the main file
(or at least for those files to be compiled individually).
The argument \textit{main} must be the filename of the main file.

There are a couple of
considerations in setting up the main and child documents:

%%%%%%%%%%%%%%%%%%%%%%%%%%%%%%%%%%%%%%%%
\paragraph{Restrictions.}

Please note the following restrictions:
\begin{itemize}
\item
|\childdocmain| must be called with one argument \textit{main}
to ensure compatibility with earlier version of the package.
It must either be empty (|\childdocmain{}|)
or precisely match the filename of the main file in which it is specified.
See \secref{sec:detection} for further information.
\item
The filename \textit{main} must be specified without the |.tex| extension.
\item
The filename \textit{main} is case sensitive
(even in case-insensitive file systems)
due to internal string comparison.
\item
The argument \textit{main} should be fully expanded, it cannot be a macro.
\item
Subdirectories and special characters should be avoided in filenames.
\item
The command |\childdocmain{|\textit{main}|}| must be followed by a whitespace.
It should not be followed immediately by another command
or by a comment mark `|%|'.
This is because the \TeX{} parser reads the token immediately following
the argument of |\childdocmain| and puts it
at the beginning of every child section;
however, a white\-space is ignored.
\end{itemize}

%%%%%%%%%%%%%%%%%%%%%%%%%%%%%%%%%%%%%%%%
\paragraph{Content of Main File.}

It is advisable to place all content in the child files included by |\include|.
Any output contained in the main file will appear in all child documents
unless suppressed manually;
it cannot be suppressed automatically by the |\includeonly| directive
and thus should normally be avoided.
A method to include some content in the main file
by means of conditional processing is described in \secref{sec:conditional}.

%%%%%%%%%%%%%%%%%%%%%%%%%%%%%%%%%%%%%%%%
\paragraph{Page Numbering.}

When only a part of the document is compiled,
the appropriate numbering of pages
(as well as other status parameters)
is determined from the |.aux| files.
The latter contain information from previous passes.
However this information needs to propagate through
all intermediate child documents.
Therefore the page numbering in child documents may well
be inconsistent until the complete document is compiled at least once.

A useful (if unconventional) way to always ensure a consistent
page numbering is to restart the numbering in each child document
and denote the pages by `\textit{child}|.|\textit{page}'
where \textit{child} represents the chapter/section number of the child file.
This can be achieved by the command
|\numberwithin{page}{|\textit{child}|}|
of the \textsf{amsmath} package
where \textit{child} can be |chapter| or |section|
depending on the chosen structuring.
Alternatively, one can modify the macro |\thepage| appropriately
and reset the counter |page| at the start of each child file.

%%%%%%%%%%%%%%%%%%%%%%%%%%%%%%%%%%%%%%%%%%%%%%%%%%%%%%%%%%%%%%%%%%%%%%%%%%%%%%%%
\subsection{Conditional Processing}
\label{sec:conditional}

The package provides a mechanism to compile different versions
of a document. To customise the versions further some conditional processing
can come in handy to distinguish which version is being compiled.
The package provides two macros to describe the compilation context:

%%%%%%%%%%%%%%%%%%%%%%%%%%%%%%%%%%%%%%%%
\DescribeMacro{\ifchilddoc}
The conditional |\ifchilddoc| distinguishes between the compilation of
child documents and the main document:
%
\begin{center}
|\ifchilddoc |\textit{child-code}| |[|\||else |\textit{main-code}]| \||fi|
\end{center}

%%%%%%%%%%%%%%%%%%%%%%%%%%%%%%%%%%%%%%%%
\DescribeMacro{\childdocname}
\DescribeMacro{\childdocjob}
The macro |\childdocname| contains the filename (without extension)
of the main or child file being processed.
Note that |\childdocjob| will always contain the name of the main file.

%%%%%%%%%%%%%%%%%%%%%%%%%%%%%%%%%%%%%%%%
\paragraph{Title Page.}

Conditional processing can be used to include a title or banner page
in the main document when proper precautions are taken.
Importantly, the code in the main file should ensure that the page counter
(as well as other status parameters which are stored in the |.aux| files)
takes the same value after the conditional processing.
Otherwise the page numbers may take divergent values
depending on which part is compiled.

For example, a title page could be declared by:
%
\begin{center}
\begin{tabular}{l}
|\ifchilddoc\||else|\\
|\addtocounter{page}{-1}|\\
\textit{code for title page}\\
|\newpage|\\
|\||fi|
\end{tabular}
\end{center}
%
A banner page for the child documents can be generated by:
%
\begin{center}
\begin{tabular}{l}
|\ifchilddoc|\\
|\addtocounter{page}{-1}|\\
\textit{code for banner page}\\
|\newpage|\\
|\||fi|
\end{tabular}
\end{center}
%
Here one could write a message such as:
\begin{center}
|This is the part \childdocname{} of \childdocjob{}.|
\end{center}

%%%%%%%%%%%%%%%%%%%%%%%%%%%%%%%%%%%%%%%%%%%%%%%%%%%%%%%%%%%%%%%%%%%%%%%%%%%%%%%%
\subsection{Flags}
\label{sec:flags}

The package makes it easy to generate different versions
of the main or child documents.
To this end compilation flags can be defined
and assigned different default values.
They will be particularly useful in conjunction
with the forwarding mechanism described in \secref{sec:forward}.

For example, it may be useful to have a flag |\version|
which can be set to |draft| or |final|.
The document source will contain some conditional code
depending on the value of |\version|.
Suppose further, the flag should default to |final| for the main file
and to |draft| for child files
which is a natural assignment for editing the document.
This is achieved by placing the following code
in the preamble of the main document
(below the |\childdocmain| directive):
%
\begin{center}
\begin{tabular}{l}
|\ifchilddoc|\\
|\providecommand{\version}{draft}|\\
|\||else|\\
|\providecommand{\version}{final}|\\
|\||fi|
\end{tabular}
\end{center}
%
The definition by |\providecommand| makes sure
that previous definitions are not overwritten.
Further statements |\providecommand{\version}{...}|
can thus be added before the above code to override it.

For the main file, one might add a line
(between |\childdocmain| and the above block)
%
\begin{center}
|%\ifchilddoc\||else\providecommand{\version}{draft}\||fi|
\end{center}
%
which can be uncommented to produce a draft version.
Likewise one can add a line to the very top of a child file
(above the |\childdocof{|\textit{main}|}| directive)
%
\begin{center}
|%\providecommand{\version}{final}|
\end{center}
%
which can be uncommented to produce the final version of this child document.

%%%%%%%%%%%%%%%%%%%%%%%%%%%%%%%%%%%%%%%%%%%%%%%%%%%%%%%%%%%%%%%%%%%%%%%%%%%%%%%%
\subsection{Forwarding}
\label{sec:forward}

Different versions of the main or child documents
using compilation flags as described in \secref{sec:flags}
can be (permanently) stored in different files
for convenient compilation, viewing and distribution.
To this end, the package defines a command
to pass on compilation to a different file:

%%%%%%%%%%%%%%%%%%%%%%%%%%%%%%%%%%%%%%%%
\DescribeMacro{\childdocforward}
The command |\childdocforward| redirects processing to
another source file:
%
\begin{center}
\begin{tabular}{l}
|% \iffalse
%
% childdoc.dtx Copyright (C) 2017-2018 Niklas Beisert
%
% This work may be distributed and/or modified under the
% conditions of the LaTeX Project Public License, either version 1.3
% of this license or (at your option) any later version.
% The latest version of this license is in
%   http://www.latex-project.org/lppl.txt
% and version 1.3 or later is part of all distributions of LaTeX
% version 2005/12/01 or later.
%
% This work has the LPPL maintenance status `maintained'.
%
% The Current Maintainer of this work is Niklas Beisert.
%
% This work consists of the files childdoc.dtx and childdoc.ins
% and the derived files childdoc.def and cdocsamp.tex with
% cdocsch1.tex, cdocsch2.tex, cdocsdrf.tex, cdocsfn1.tex, cdocsfn2.tex.
%
%<package>\ifdefined\childdocmain\endinput\fi
%<package>\ProvidesFile{childdoc.def}[2018/12/30 v2.0 child document driver]
%<samplemain>\ProvidesFile{cdocsamp.tex}[2018/12/30 v2.0 sample for childdoc]
%<*driver>
%\ProvidesFile{childdoc.drv}[2018/12/30 v2.0 childdoc reference manual file]
\PassOptionsToClass{10pt,a4paper}{article}
\documentclass{ltxdoc}

\usepackage[margin=35mm]{geometry}
\usepackage{hyperref}
\usepackage{hyperxmp}
\usepackage[usenames]{color}

\hypersetup{colorlinks=true}
\hypersetup{pdfstartview=FitH}
\hypersetup{pdfpagemode=UseNone}
\hypersetup{pdfsource={}}
\hypersetup{pdflang={en-UK}}
\hypersetup{pdfcopyright={Copyright 2017-2018 Niklas Beisert.
  This work may be distributed and/or modified under the
  conditions of the LaTeX Project Public License, either version 1.3
  of this license or (at your option) any later version.}}
\hypersetup{pdflicenseurl={http://www.latex-project.org/lppl.txt}}
\hypersetup{pdfcontactaddress={ETH Zurich, ITP, HIT K,
  Wolfgang-Pauli-Strasse 27}}
\hypersetup{pdfcontactpostcode={8093}}
\hypersetup{pdfcontactcity={Zurich}}
\hypersetup{pdfcontactcountry={Switzerland}}
\hypersetup{pdfcontactemail={nbeisert@itp.phys.ethz.ch}}
\hypersetup{pdfcontacturl={http://people.phys.ethz.ch/\xmptilde nbeisert/}}

\newcommand{\secref}[1]{\hyperref[#1]{section \ref*{#1}}}

\parskip1ex
\parindent0pt
\let\olditemize\itemize
\def\itemize{\olditemize\parskip0pt}

\begin{document}

\title{The \textsf{childdoc} Package}
\hypersetup{pdftitle={The childdoc Package}}
\author{Niklas Beisert\\[2ex]
  Institut f\"ur Theoretische Physik\\
  Eidgen\"ossische Technische Hochschule Z\"urich\\
  Wolfgang-Pauli-Strasse 27, 8093 Z\"urich, Switzerland\\[1ex]
  \href{mailto:nbeisert@itp.phys.ethz.ch}
  {\texttt{nbeisert@itp.phys.ethz.ch}}}
\hypersetup{pdfauthor={Niklas Beisert}}
\hypersetup{pdfsubject={Manual for the LaTeX2e Package childdoc}}
\date{30 December 2018, \textsf{v2.0}}
\maketitle

\begin{abstract}\noindent
\textsf{childdoc} is a \LaTeXe{} package
that enables the direct compilation
of document sections included by |\include|
to individual files.
\end{abstract}

\begingroup
\parskip0ex
\tableofcontents
\endgroup

%%%%%%%%%%%%%%%%%%%%%%%%%%%%%%%%%%%%%%%%%%%%%%%%%%%%%%%%%%%%%%%%%%%%%%%%%%%%%%%%
%%%%%%%%%%%%%%%%%%%%%%%%%%%%%%%%%%%%%%%%%%%%%%%%%%%%%%%%%%%%%%%%%%%%%%%%%%%%%%%%
\section{Introduction}

\LaTeX{} provides a mechanism to structure a large document (such as a book)
into a main file and several child files (containing the chapters)
using the |\include| command.
This mechanism is beneficial for documents
which span hundreds of pages in order to
make the source file(s) more manageable.
Moreover, compilation can be restricted to
selected child files by means of the |\includeonly| command.
The latter feature can be used to reduce the compilation time while editing
(this was significantly more useful in the earlier days of \LaTeX{})
or to generate a smaller document which is easier to navigate.
Another application of |\includeonly| is to generate
documents consisting of selected parts of the complete document.

However, there are a few drawbacks of the plain |\include| mechanism:
\begin{itemize}
\item
The child files cannot be compiled on their own,
they can only be compiled via the main file.
A naive editing environment
(such as a text editor with an option
to have the current file processed by \LaTeX)
may require one to switch to the main file before compiling;
attempting to compile the child file produces errors.
\item
The main file must be modified (each time)
to adjust the |\includeonly| command
to the present needs. This easily leaves the main file in a messy state.
\item
The generated document will always carry the filename
of the main document. This is inconvenient if
several child files are to be compiled and
to be kept for distribution.
\end{itemize}

The present package provides a simple interface
to make child files individually compilable by \LaTeX{}.
Compiling a child file then has the same effect as compiling
the main file with an |\includeonly| command
to select the appropriate child.
Moreover the generated document will carry the name of the child
rather than the main file.
This resolves all three above issues.

This feature is meant to make the editing of books,
thesis documents and lecture notes somewhat more convenient.
However, the package can also be used efficiently for
composing a series of documents (such as exercise sheets)
which are typically distributed individually.
It then assists the author in generating the individual documents
(potentially in different versions)
as well as a document containing the collected series.
Another application is in developing style files
or other kinds of included material
where compilation of the style file could redirect
to a sample or test file.

%%%%%%%%%%%%%%%%%%%%%%%%%%%%%%%%%%%%%%%%%%%%%%%%%%%%%%%%%%%%%%%%%%%%%%%%%%%%%%%%
%%%%%%%%%%%%%%%%%%%%%%%%%%%%%%%%%%%%%%%%%%%%%%%%%%%%%%%%%%%%%%%%%%%%%%%%%%%%%%%%
\section{Usage}

First of all, the package \textsf{childdoc} is \emph{not} a standard
\LaTeXe{} |.sty| style file! Therefore it needs to be invoked in
a non-standard way.

%%%%%%%%%%%%%%%%%%%%%%%%%%%%%%%%%%%%%%%%%%%%%%%%%%%%%%%%%%%%%%%%%%%%%%%%%%%%%%%%
\subsection{Included Files}
\label{sec:include}

%%%%%%%%%%%%%%%%%%%%%%%%%%%%%%%%%%%%%%%%
\DescribeMacro{\childdocmain}
To use the package, add the commands
\begin{center}
\begin{tabular}{l}
|\input{childdoc.def}|\\
|\childdocmain{}|\\
\end{tabular}
\end{center}
at the very top of the main \LaTeX{} file,
in particular \emph{before} the |\documentclass| statement!
The argument of |\childdocmain| should be left empty
(but it must be present).

%%%%%%%%%%%%%%%%%%%%%%%%%%%%%%%%%%%%%%%%
\DescribeMacro{\childdocof}
Furthermore, add the commands
\begin{center}
\begin{tabular}{l}
|\input{childdoc.def}|\\
|\childdocof{|\textit{main}|}|\\
\end{tabular}
\end{center}
at the top of every child file \textit{child}
which is included by |\include{|\textit{child}|}|
from within the main file
(or at least for those files to be compiled individually).
The argument \textit{main} must be the filename of the main file.

There are a couple of
considerations in setting up the main and child documents:

%%%%%%%%%%%%%%%%%%%%%%%%%%%%%%%%%%%%%%%%
\paragraph{Restrictions.}

Please note the following restrictions:
\begin{itemize}
\item
|\childdocmain| must be called with one argument \textit{main}
to ensure compatibility with earlier version of the package.
It must either be empty (|\childdocmain{}|)
or precisely match the filename of the main file in which it is specified.
See \secref{sec:detection} for further information.
\item
The filename \textit{main} must be specified without the |.tex| extension.
\item
The filename \textit{main} is case sensitive
(even in case-insensitive file systems)
due to internal string comparison.
\item
The argument \textit{main} should be fully expanded, it cannot be a macro.
\item
Subdirectories and special characters should be avoided in filenames.
\item
The command |\childdocmain{|\textit{main}|}| must be followed by a whitespace.
It should not be followed immediately by another command
or by a comment mark `|%|'.
This is because the \TeX{} parser reads the token immediately following
the argument of |\childdocmain| and puts it
at the beginning of every child section;
however, a white\-space is ignored.
\end{itemize}

%%%%%%%%%%%%%%%%%%%%%%%%%%%%%%%%%%%%%%%%
\paragraph{Content of Main File.}

It is advisable to place all content in the child files included by |\include|.
Any output contained in the main file will appear in all child documents
unless suppressed manually;
it cannot be suppressed automatically by the |\includeonly| directive
and thus should normally be avoided.
A method to include some content in the main file
by means of conditional processing is described in \secref{sec:conditional}.

%%%%%%%%%%%%%%%%%%%%%%%%%%%%%%%%%%%%%%%%
\paragraph{Page Numbering.}

When only a part of the document is compiled,
the appropriate numbering of pages
(as well as other status parameters)
is determined from the |.aux| files.
The latter contain information from previous passes.
However this information needs to propagate through
all intermediate child documents.
Therefore the page numbering in child documents may well
be inconsistent until the complete document is compiled at least once.

A useful (if unconventional) way to always ensure a consistent
page numbering is to restart the numbering in each child document
and denote the pages by `\textit{child}|.|\textit{page}'
where \textit{child} represents the chapter/section number of the child file.
This can be achieved by the command
|\numberwithin{page}{|\textit{child}|}|
of the \textsf{amsmath} package
where \textit{child} can be |chapter| or |section|
depending on the chosen structuring.
Alternatively, one can modify the macro |\thepage| appropriately
and reset the counter |page| at the start of each child file.

%%%%%%%%%%%%%%%%%%%%%%%%%%%%%%%%%%%%%%%%%%%%%%%%%%%%%%%%%%%%%%%%%%%%%%%%%%%%%%%%
\subsection{Conditional Processing}
\label{sec:conditional}

The package provides a mechanism to compile different versions
of a document. To customise the versions further some conditional processing
can come in handy to distinguish which version is being compiled.
The package provides two macros to describe the compilation context:

%%%%%%%%%%%%%%%%%%%%%%%%%%%%%%%%%%%%%%%%
\DescribeMacro{\ifchilddoc}
The conditional |\ifchilddoc| distinguishes between the compilation of
child documents and the main document:
%
\begin{center}
|\ifchilddoc |\textit{child-code}| |[|\||else |\textit{main-code}]| \||fi|
\end{center}

%%%%%%%%%%%%%%%%%%%%%%%%%%%%%%%%%%%%%%%%
\DescribeMacro{\childdocname}
\DescribeMacro{\childdocjob}
The macro |\childdocname| contains the filename (without extension)
of the main or child file being processed.
Note that |\childdocjob| will always contain the name of the main file.

%%%%%%%%%%%%%%%%%%%%%%%%%%%%%%%%%%%%%%%%
\paragraph{Title Page.}

Conditional processing can be used to include a title or banner page
in the main document when proper precautions are taken.
Importantly, the code in the main file should ensure that the page counter
(as well as other status parameters which are stored in the |.aux| files)
takes the same value after the conditional processing.
Otherwise the page numbers may take divergent values
depending on which part is compiled.

For example, a title page could be declared by:
%
\begin{center}
\begin{tabular}{l}
|\ifchilddoc\||else|\\
|\addtocounter{page}{-1}|\\
\textit{code for title page}\\
|\newpage|\\
|\||fi|
\end{tabular}
\end{center}
%
A banner page for the child documents can be generated by:
%
\begin{center}
\begin{tabular}{l}
|\ifchilddoc|\\
|\addtocounter{page}{-1}|\\
\textit{code for banner page}\\
|\newpage|\\
|\||fi|
\end{tabular}
\end{center}
%
Here one could write a message such as:
\begin{center}
|This is the part \childdocname{} of \childdocjob{}.|
\end{center}

%%%%%%%%%%%%%%%%%%%%%%%%%%%%%%%%%%%%%%%%%%%%%%%%%%%%%%%%%%%%%%%%%%%%%%%%%%%%%%%%
\subsection{Flags}
\label{sec:flags}

The package makes it easy to generate different versions
of the main or child documents.
To this end compilation flags can be defined
and assigned different default values.
They will be particularly useful in conjunction
with the forwarding mechanism described in \secref{sec:forward}.

For example, it may be useful to have a flag |\version|
which can be set to |draft| or |final|.
The document source will contain some conditional code
depending on the value of |\version|.
Suppose further, the flag should default to |final| for the main file
and to |draft| for child files
which is a natural assignment for editing the document.
This is achieved by placing the following code
in the preamble of the main document
(below the |\childdocmain| directive):
%
\begin{center}
\begin{tabular}{l}
|\ifchilddoc|\\
|\providecommand{\version}{draft}|\\
|\||else|\\
|\providecommand{\version}{final}|\\
|\||fi|
\end{tabular}
\end{center}
%
The definition by |\providecommand| makes sure
that previous definitions are not overwritten.
Further statements |\providecommand{\version}{...}|
can thus be added before the above code to override it.

For the main file, one might add a line
(between |\childdocmain| and the above block)
%
\begin{center}
|%\ifchilddoc\||else\providecommand{\version}{draft}\||fi|
\end{center}
%
which can be uncommented to produce a draft version.
Likewise one can add a line to the very top of a child file
(above the |\childdocof{|\textit{main}|}| directive)
%
\begin{center}
|%\providecommand{\version}{final}|
\end{center}
%
which can be uncommented to produce the final version of this child document.

%%%%%%%%%%%%%%%%%%%%%%%%%%%%%%%%%%%%%%%%%%%%%%%%%%%%%%%%%%%%%%%%%%%%%%%%%%%%%%%%
\subsection{Forwarding}
\label{sec:forward}

Different versions of the main or child documents
using compilation flags as described in \secref{sec:flags}
can be (permanently) stored in different files
for convenient compilation, viewing and distribution.
To this end, the package defines a command
to pass on compilation to a different file:

%%%%%%%%%%%%%%%%%%%%%%%%%%%%%%%%%%%%%%%%
\DescribeMacro{\childdocforward}
The command |\childdocforward| redirects processing to
another source file:
%
\begin{center}
\begin{tabular}{l}
|\input{childdoc.def}|\\
|\childdocforward[|\textit{main}|]{|\textit{dest}|}|\\
\end{tabular}
\end{center}
%
The argument \textit{dest} is the destination file
(without extension).
It should be the main file or one of the child files.
Note that further \textsf{childdoc} directives
such as |\childdocof| and |\childdocforward|
in the indicated file will be processed in this form.
The optional argument \textit{main}
passes on directly to the main file \textit{main}
while pretending to compile the child \textit{dest}.
This form behaves as if \textit{dest}
issues |\childdocof{|\textit{main}|}| right away,
and no further \textsf{childdoc} directives will be processed.

%%%%%%%%%%%%%%%%%%%%%%%%%%%%%%%%%%%%%%%%
\DescribeMacro{\...prefix}
In the alternative form |\childdocforwardprefix|,
%
\begin{center}
\begin{tabular}{l}
|\input{childdoc.def}|\\
|\childdocforwardprefix[|\textit{main}|]{|\textit{prefix}|}{|\textit{dest}|}|
\end{tabular}
\end{center}
%
the destination file is determined by a pattern
depending on the current file:
To make this work, the current file must be called
`{\textit{prefix}\hspace{0.2em}\textit{suffix}}'
with \textit{prefix} matching precisely the argument.
Processing is then passed on to the file
`{\textit{dest}\hspace{0.2em}\textit{suffix}}'.
Surely, the same effect is achieved by
directly specifying the
argument `{\textit{dest}\hspace{0.2em}\textit{suffix}}'
in the first form.
However, that requires to set up a different file
for each child. With the alternative form of the command
all these files can have exactly the same content
which simplifies setting them up and maintaining them.

For example, the following file |draft.tex|
with a compilation flag |\version| as described in \secref{sec:flags}
compiles the main document as a draft:
%
\begin{center}
\begin{tabular}{l}
|\def\version{draft}|\\
|\input{childdoc.def}|\\
|\childdocforward{|\textit{main}|}|
\end{tabular}
\end{center}
%
Likewise, the following files |final|\textit{nn}|.tex|
compile the final version of the child document
|child|\textit{nn}|.tex|:
%
\begin{center}
\begin{tabular}{l}
|\def\version{final}|\\
|\input{childdoc.def}|\\
|\childdocforwardprefix{final}{child}|
\end{tabular}
\end{center}
%

Note that when several versions of a main file and/or of each child file
are to be generated, it may be convenient to set up a |Makefile| or
shell script to automatise the process.

%%%%%%%%%%%%%%%%%%%%%%%%%%%%%%%%%%%%%%%%%%%%%%%%%%%%%%%%%%%%%%%%%%%%%%%%%%%%%%%%
\subsection{Command Line Processing}
\label{sec:commandline}

The effect of redirection files can also be achieved by invoking
the \LaTeX{} compiler with a more elaborate command line.
Most conveniently this should be done as part
of a shell script or a |Makefile|.

When using \textsf{childdoc} in the main file, the following
command lines effectively perform a redirection
(note that depending on the shell being used,
backslashes may have to be doubled: `|\|' $\to$ `|\\|'):
%
\begin{center}
|... -jobname "|\textit{target}|" |\\|"|[\textit{flags}]%
|\input{childdoc.def}\childdocforward[|\textit{main}|]{|\textit{dest}|}"|
\end{center}
%
Here \textit{target} is the name of the output file,
\textit{main} is the name of the main file
and \textit{dest} is the name of the main or child file to be processed
(all filenames without extensions).
The optional argument \textit{main} can be omitted
if \textit{main} matches \textit{dest}.
Optionally, compilation \textit{flags} can be defined via |\def| commands.
This command line makes the \TeX{} engine believe
it is compiling the file \textit{target}
whose content is specified as the latter parameter.
The provided code then forwards the processing to
\textit{main} or \textit{dest} as described in \secref{sec:forward}.

%%%%%%%%%%%%%%%%%%%%%%%%%%%%%%%%%%%%%%%%%%%%%%%%%%%%%%%%%%%%%%%%%%%%%%%%%%%%%%%%
\subsection{Include by Input}
\label{sec:input}

Including child documents by |\include| has some restrictions by design.
Most notably, the content of a child document always occupies
its own set of pages; pages cannot be shared between child documents.
Usually, this behaviour makes perfect sense
because each child document contain an essential part of the document.
However, in some situations it may be desirable to compose
a document from a collection of parts
without having mandatory page breaks between then.
For this case, the package
provides a mechanism to include parts
by |\input| which can also be processed individually.
However, by construction this mechanism
requires manual handling of the content to be output.

%%%%%%%%%%%%%%%%%%%%%%%%%%%%%%%%%%%%%%%%
\DescribeMacro{\ifchilddocmanual}
The main file should be prepared as usual, see \secref{sec:include}.
However, the document body must make a distinction
between processing of an individual part and of the main document, e.g.:
%
\begin{center}
\begin{tabular}{l}
|\ifchilddocmanual|\\
|\input{\childdocname}|\\
|\||else|\\
\textit{document body with }|\input{|\textit{part}|}|\\
|\||fi|
\end{tabular}
\end{center}
%
The conditional |\ifchilddocmanual| is true whenever
a part to be included by |\input| is being compiled,
and the name of the part is stored in |\childdocname|.

%%%%%%%%%%%%%%%%%%%%%%%%%%%%%%%%%%%%%%%%
\DescribeMacro{\childdocby}
Each part to be included by |\input| should start with:
%
\begin{center}
\begin{tabular}{l}
|\input{childdoc.def}|\\
|\childdocby{|\textit{main}|}|\\
\end{tabular}
\end{center}
%
The directive |\childdocby| is similar to |\childdocof|
described in \secref{sec:include},
but the subsequent selection of content must be done manually.
To that end, both |\ifchilddoc| and |\ifchilddocmanual|
will be true upon processing of a part,
and the name of the part is stored in |\childdocname|.
Note that |\jobname| will be set to the filename of the current part
so that each part receives an individual |.aux| file
that does not interfere with the |.aux| file(s) of the main document.
This behaviour can be altered by the alternative form
|\childdocby[*]{|\textit{main}|}| (with a non-empty optional argument)
which uses the |.aux| file of the main document
by setting |\jobname| to \textit{main}.

%%%%%%%%%%%%%%%%%%%%%%%%%%%%%%%%%%%%%%%%%%%%%%%%%%%%%%%%%%%%%%%%%%%%%%%%%%%%%%%%
\subsection{Driver Development}
\label{sec:driver}

The \textsf{childdoc} mechanism can also be use for the development
of definition files such as \LaTeX{} styles or classes.
This case differs from the above setup with multiple parts
included by |\include| in that no |\includeonly| should be invoked.
This can be achieved by starting the include file
(before |\ProvidesPackage|) with:
%
\begin{center}
\begin{tabular}{l}
|\input{childdoc.def}|\\
|\childdocforward{|\textit{main}|}|\\
\end{tabular}
\end{center}
%
or alternatively with:
%
\begin{center}
\begin{tabular}{l}
|\input{childdoc.def}|\\
|\childdocby{|\textit{main}|}|\\
\end{tabular}
\end{center}
%
Both forms have slightly different effects as described above.
The main file is prepared as usual, see \secref{sec:include}.

%%%%%%%%%%%%%%%%%%%%%%%%%%%%%%%%%%%%%%%%%%%%%%%%%%%%%%%%%%%%%%%%%%%%%%%%%%%%%%%%
\subsection{Legacy Detection}
\label{sec:detection}

The directive |\childdocmain| in the main file can detect
whether the complete document or merely a child is to be compiled
even without using the directive |\childdocof|.
This method is deprecated because it is less robust
and there is no compelling reason to use it;
it is merely provided for backward compatibility
and it may be removed in future versions.

If the detection mechanism is to be used,
it is mandatory to correctly specify
the filename of the main file as the argument of |\childdocmain|:
%
\begin{center}
\begin{tabular}{l}
|\input{childdoc.def}|\\
|\childdocmain{|\textit{main}|}|\\
\end{tabular}
\end{center}
%
If |\jobname| does not match the argument \textit{main} of |\childdocmain|,
it is assumed that |\jobname| points to the child file to be compiled.
When using |\childdocmain| with the main file specified as argument,
it suffices to start a child file
with just |\input{|\textit{main}|}|
without loading of the package and using |\childdocof|.
If instead all processing is done
with the appropriate \textsf{childdoc} directives,
the argument of \textit{main} of |\childdocmain| can be empty.

An alternative version of the command line processing described
in \secref{sec:commandline} using the detection mechanism reads:
%
\begin{center}
|... -jobname "|\textit{target}|" "|[\textit{flags}]%
[|\def\jobname{|\textit{dest}|}|]|\input{|\textit{main}|}"|
\end{center}

%%%%%%%%%%%%%%%%%%%%%%%%%%%%%%%%%%%%%%%%%%%%%%%%%%%%%%%%%%%%%%%%%%%%%%%%%%%%%%%%
\subsection{Manual Code}
\label{sec:manual}

In case one cannot be certain whether the definitions file |childdoc.def|
is installed on the target \TeX{} distribution
and one prefers not to ship it,
it is conceivable to paste a few relevant commands into the sources.

To that end, drop all statements |\input{childdoc.def}|
and perform the replacements as outlined below.
Instead of |\childdocmain{|\textit{main}|}| add the following code
to the top of the main file:
%
\begin{center}
\begin{tabular}{l}
|\||ifdefined\childdocname\endinput\||fi\newif\ifchilddoc|\\
|\edef\childdocname{\scantokens\expandafter{\jobname\noexpand}}|\\
|\def\childdocmain{|\textit{main}|}\||ifx\childdocmain\childdocname\||else|\\
|\childdoctrue\includeonly{\childdocname}\let\jobname\childdocmain\||fi|\\
\end{tabular}
\end{center}
%
Instead of |\childdocof{|\textit{main}|}| just include the main file
at the top of each child file:
%
\begin{center}
|\input{|\textit{main}|}|
\end{center}
%
A simple redirection |\childdocforward{|\textit{dest}|}| is achieved by:
%
\begin{center}
|\def\jobname{|\textit{dest}|}\input{\jobname}|
\end{center}
%
The redirection with prefix
|\childdocforwardprefix[|\textit{prefix}|]{|\textit{dest}|}|
is accomplished by:
%
\begin{center}
\begin{tabular}{l}
|{\edef\jobname{\scantokens\expandafter{\jobname\noexpand}}|\\
|\def\redirectjob |\textit{prefix}|#1~~~{\gdef\jobname{|\textit{dest}|#1}}|\\
|\expandafter\redirectjob\jobname~~~}\input{\jobname}|
\end{tabular}
\end{center}

In an alternative approach,
child documents can be compiled by a specific command line
without additional code or specific definitions:
%
\begin{center}
|... -jobname "|\textit{target}|" "|[\textit{flags}]%
|\includeonly{|\textit{dest}|}\input{|\textit{main}|}"|
\end{center}
%

%%%%%%%%%%%%%%%%%%%%%%%%%%%%%%%%%%%%%%%%%%%%%%%%%%%%%%%%%%%%%%%%%%%%%%%%%%%%%%%%
%%%%%%%%%%%%%%%%%%%%%%%%%%%%%%%%%%%%%%%%%%%%%%%%%%%%%%%%%%%%%%%%%%%%%%%%%%%%%%%%
\section{Information}

%%%%%%%%%%%%%%%%%%%%%%%%%%%%%%%%%%%%%%%%%%%%%%%%%%%%%%%%%%%%%%%%%%%%%%%%%%%%%%%%
\subsection{Copyright}

Copyright \copyright{} 2017--2018 Niklas Beisert

This work may be distributed and/or modified under the
conditions of the \LaTeX{} Project Public License, either version 1.3
of this license or (at your option) any later version.
The latest version of this license is in
  \url{http://www.latex-project.org/lppl.txt}
and version 1.3 or later is part of all distributions of \LaTeX{}
version 2005/12/01 or later.

This work has the LPPL maintenance status `maintained'.

The Current Maintainer of this work is Niklas Beisert.

This work consists of the files |README.txt|, |childdoc.ins| and |childdoc.dtx|
as well as the derived files |childdoc.def|, |cdocsamp.tex|
with |cdocsch1.tex|, |cdocsch2.tex|, |cdocspt3.tex|, |cdocspt4.tex|,
|cdocsdrf.tex|, |cdocsfn1.tex|, |cdocsfn2.tex|
as well as |childdoc.pdf|.

%%%%%%%%%%%%%%%%%%%%%%%%%%%%%%%%%%%%%%%%%%%%%%%%%%%%%%%%%%%%%%%%%%%%%%%%%%%%%%%%
\subsection{Files and Installation}

The package consists of the files:
%
\begin{center}
\begin{tabular}{ll}
    |README.txt|   & readme file \\
    |childdoc.ins| & installation file \\
    |childdoc.dtx| & source file \\
    |childdoc.def| & definition file \\
    |cdocsamp.tex| & sample main file \\
    |cdocsch1.tex| & sample include file \\
    |cdocsch2.tex| & sample include file \\
    |cdocspt3.tex| & sample part file \\
    |cdocspt4.tex| & sample part file \\
    |cdocsdrf.tex| & sample redirection file \\
    |cdocsfn1.tex| & sample redirection file \\
    |cdocsfn2.tex| & sample redirection file \\
    |childdoc.pdf| & manual
\end{tabular}
\end{center}
%
The distribution consists of the files
|README.txt|, |childdoc.ins| and |childdoc.dtx|.
%
\begin{itemize}
\item
Run (pdf)\LaTeX{} on |childdoc.dtx|
to compile the manual |childdoc.pdf| (this file).
\item
Run \LaTeX{} on |childdoc.ins| to create the definitions file |childdoc.def|
and the sample |cdocsamp.tex| with include files
|cdocsch1.tex|, |cdocsch2.tex|, |cdocspt3.tex|, |cdocspt4.tex|,
|cdocsdrf.tex|, |cdocsfn1.tex|, |cdocsfn2.tex|.
Then copy the file |childdoc.def| to an appropriate directory of your \LaTeX{}
distribution, e.g.\ \textit{texmf-root}|/tex/latex/childdoc|.
\end{itemize}

%%%%%%%%%%%%%%%%%%%%%%%%%%%%%%%%%%%%%%%%%%%%%%%%%%%%%%%%%%%%%%%%%%%%%%%%%%%%%%%%
\subsection{Related CTAN Packages}

There are several other packages which offer a similar functionality:
%
\begin{itemize}
\item
The packages
\href{http://ctan.org/pkg/docmute}{\textsf{docmute}},
\href{http://ctan.org/pkg/includex}{\textsf{includex}} and
\href{http://ctan.org/pkg/standalone}{\textsf{standalone}}
provide commands to include only the document body of
a child file thus allowing both files to be compiled individually.
\item
The packages \href{http://ctan.org/pkg/subdocs}{\textsf{subdocs}}
and \href{http://ctan.org/pkg/subfiles}{\textsf{subfiles}}
provide structures in which the main and child documents can be
encapsulated and allowing them to be compiled individually.
The inclusion mechanism is different from the conventional |\include|.
\item
The package \href{http://ctan.org/pkg/combine}{\textsf{combine}}
is an elaborate solution to combine several documents into one.
\end{itemize}
%
See also the CTAN topic \href{http://ctan.org/topic/subdocs}{\textsf{subdocs}}
for further related packages.
The present package differs from the above solutions in that
a document structure constructed with the conventional |\include| mechanism
just needs two extra commands at the top of every file
such that all constituent files can be compiled individually.

%%%%%%%%%%%%%%%%%%%%%%%%%%%%%%%%%%%%%%%%%%%%%%%%%%%%%%%%%%%%%%%%%%%%%%%%%%%%%%%%
%\subsection{Feature Suggestions}
%
%The following is a list of features which may be useful for future
%versions of this package:
%%
%\begin{itemize}
%\item
%\ldots
%\end{itemize}

%%%%%%%%%%%%%%%%%%%%%%%%%%%%%%%%%%%%%%%%%%%%%%%%%%%%%%%%%%%%%%%%%%%%%%%%%%%%%%%%
\subsection{Revision History}

%%%%%%%%%%%%%%%%%%%%%%%%%%%%%%%%%%%%%%%%
\paragraph{v2.0:} 2018/12/30

\begin{itemize}
\item
immediate forward processing
\item
added |\childdocby| mechanism
\item
manual restructured
\end{itemize}

%%%%%%%%%%%%%%%%%%%%%%%%%%%%%%%%%%%%%%%%
\paragraph{v1.6:} 2018/01/17

\begin{itemize}
\item
application for development of include files
\item
corrections to manual
\end{itemize}

%%%%%%%%%%%%%%%%%%%%%%%%%%%%%%%%%%%%%%%%
\paragraph{v1.5:} 2017/05/21

\begin{itemize}
\item
more complete structuring introduced
\item
|\childdocof| introduced
\item
|\childdoc| renamed to |\childdocmain|
\item
|\childredirect| renamed to |\childdocforward| and |\childdocforwardprefix|
and functionality expanded
\end{itemize}

%%%%%%%%%%%%%%%%%%%%%%%%%%%%%%%%%%%%%%%%
\paragraph{v1.0:} 2017/04/27

\begin{itemize}
\item
manual and install package
\item
first version published on CTAN
\end{itemize}

%%%%%%%%%%%%%%%%%%%%%%%%%%%%%%%%%%%%%%%%
\paragraph{v0.6:} 2017/04/26

\begin{itemize}
\item
redirection mechanism added
\end{itemize}

%%%%%%%%%%%%%%%%%%%%%%%%%%%%%%%%%%%%%%%%
\paragraph{v0.5:} 2017/04/26

\begin{itemize}
\item
functionality in definition file
\end{itemize}


%%%%%%%%%%%%%%%%%%%%%%%%%%%%%%%%%%%%%%%%%%%%%%%%%%%%%%%%%%%%%%%%%%%%%%%%%%%%%%%%
%%%%%%%%%%%%%%%%%%%%%%%%%%%%%%%%%%%%%%%%%%%%%%%%%%%%%%%%%%%%%%%%%%%%%%%%%%%%%%%%
%%%%%%%%%%%%%%%%%%%%%%%%%%%%%%%%%%%%%%%%%%%%%%%%%%%%%%%%%%%%%%%%%%%%%%%%%%%%%%%%
\appendix

\settowidth\MacroIndent{\rmfamily\scriptsize 000\ }

 \DocInput{childdoc.dtx}

\end{document}
%</driver>
% \fi
%
% %%%%%%%%%%%%%%%%%%%%%%%%%%%%%%%%%%%%%%%%%%%%%%%%%%%%%%%%%%%%%%%%%%%%%%%%%%%%%%
% %%%%%%%%%%%%%%%%%%%%%%%%%%%%%%%%%%%%%%%%%%%%%%%%%%%%%%%%%%%%%%%%%%%%%%%%%%%%%%
% \section{Sample}
%\iffalse
%<*samplemain>
%\fi
%
% The following presents a sample document
% with two chapters, two parts, a title page,
% a compile flag as well as three forwarding files to set the flag.
% It consists of eight |.tex| files:
% \begin{center}
% \begin{tabular}{ll}
% |cdocsamp.tex|&main file\\
% |cdocsch1.tex|&include file for chapter 1\\
% |cdocsch2.tex|&include file for chapter 2\\
% |cdocspt3.tex|&include file for part 3\\
% |cdocspt4.tex|&include file for part 4\\
% |cdocsdrf.tex|&forwarding file for main file in draft mode\\
% |cdocsfi1.tex|&forwarding file for final version of chapter 1\\
% |cdocsfi2.tex|&forwarding file for final version of chapter 2\\
% \end{tabular}
% \end{center}
% Each of the eight files can be compiled directly by the \LaTeX{} compiler.
%
% %%%%%%%%%%%%%%%%%%%%%%%%%%%%%%%%%%%%%%
% \paragraph{Main File.}
%
% The main file is called |cdocsamp.tex|.
%
% Load the \textsf{childdoc} definitions and
% declare the filename for the main document:
%    \begin{macrocode}
\input{childdoc.def}
\childdocmain{}
%    \end{macrocode}

% Optional override for |\version| flag:
%    \begin{macrocode}
%%\ifchilddoc\else\providecommand{\version}{draft}\fi
%    \end{macrocode}

% Define the default values for the |\version| flag
% (|final| for the main file and |draft| for childs):
%    \begin{macrocode}
\ifchilddoc
\providecommand{\version}{draft}
\else
\providecommand{\version}{final}
\fi
%    \end{macrocode}

% Load the standard document class:
%    \begin{macrocode}
\documentclass[12pt]{article}
%    \end{macrocode}

% Start the document body:
%    \begin{macrocode}
\begin{document}
%    \end{macrocode}

% Declare a title page.
% Print title, part of document being processed and version flag:
%    \begin{macrocode}
\addtocounter{page}{-1}
\begin{center}
{\LARGE\bfseries{}childdoc example\par}
\vspace{1cm}
\ifchilddoc
\ifchilddocmanual part\else chapter\fi:
`\childdocname' of `\childdocjob'\par
\else
main document: `\childdocjob'\par
\fi
version: \version\par
\end{center}
\newpage
%    \end{macrocode}

% Manually include selected file,
% otherwise process as usual:
%    \begin{macrocode}
\ifchilddocmanual
\section*{part `\childdocname'}
\input{\childdocname}
\else
%    \end{macrocode}

% Include the two chapters:
%    \begin{macrocode}
\include{cdocsch1}
\include{cdocsch2}
%    \end{macrocode}

% Include the two parts unless only chapters should be displayed:
%    \begin{macrocode}
\ifchilddoc\else
\section{part three}
\input{cdocspt3}
\section{part four}
\input{cdocspt4}
\fi
%    \end{macrocode}

% Process as usual until here:
%    \begin{macrocode}
\fi
%    \end{macrocode}

% End of document body:
%    \begin{macrocode}
\end{document}
%    \end{macrocode}
%\iffalse
%</samplemain>
%\fi
%
% %%%%%%%%%%%%%%%%%%%%%%%%%%%%%%%%%%%%%%
% \paragraph{Chapter Include Files.}
%
% The include files are called |cdocsch1.tex| and |cdocsch2.tex|.
%
%\iffalse
%<*samplechap1|samplechap2>
%\fi

% Optional override for |\version| flag:
%    \begin{macrocode}
%%\providecommand{\version}{final}
%    \end{macrocode}

% Include the main document:
%    \begin{macrocode}
\input{childdoc.def}
\childdocof{cdocsamp}
%    \end{macrocode}

%\iffalse
%</samplechap1|samplechap2>
%\fi
%
%\iffalse
%<*samplechap1>
%\fi
% Some text for chapter 1:
%    \begin{macrocode}
\section{one}
some text in chapter one
%    \end{macrocode}

%\iffalse
%</samplechap1>
%\fi
% Some text for chapter 2:
%\iffalse
%<*samplechap2>
%\fi
%    \begin{macrocode}
\section{two}
more text in chapter two
%    \end{macrocode}

%\iffalse
%</samplechap2>
%\fi
%
% %%%%%%%%%%%%%%%%%%%%%%%%%%%%%%%%%%%%%%
% \paragraph{Part Include Files.}
%
% The include files are called |cdocspt3.tex| and |cdocspt4.tex|.
%
%\iffalse
%<*samplepart3|samplepart4>
%\fi

% Optional override for |\version| flag:
%    \begin{macrocode}
%%\providecommand{\version}{final}
%    \end{macrocode}

% Include the main document:
%    \begin{macrocode}
\input{childdoc.def}
\childdocby{cdocsamp}
%    \end{macrocode}

%\iffalse
%</samplepart3|samplepart4>
%\fi
%
%\iffalse
%<*samplepart3>
%\fi
% Some text for part 3:
%    \begin{macrocode}
some text in part three
%    \end{macrocode}

%\iffalse
%</samplepart3>
%\fi
% Some text for part 4:
%\iffalse
%<*samplepart4>
%\fi
%    \begin{macrocode}
more text in part four
%    \end{macrocode}

%\iffalse
%</samplepart4>
%\fi
%
% %%%%%%%%%%%%%%%%%%%%%%%%%%%%%%%%%%%%%%
% \paragraph{Forwarding for a Complete Draft.}
%
% The following forwarding file |cdocsdrf.tex|
% compiles the main document in draft mode:
%\iffalse
%<*sampledraft>
%\fi
%    \begin{macrocode}
\def\version{draft}
\input{childdoc.def}
\childdocforward{cdocsamp}
%    \end{macrocode}

%\iffalse
%</sampledraft>
%\fi
%
% %%%%%%%%%%%%%%%%%%%%%%%%%%%%%%%%%%%%%%
% \paragraph{Forwarding for Final Version of the Chapters.}
%
% The following forwarding files |cdocsfn1.tex| and |cdocsfn2.tex|
% (with identical content)
% compile the final versions of the child documents
% |cdocsch1.tex| and |cdocsch2.tex|, respectively:
%\iffalse
%<*samplefinal>
%\fi
%    \begin{macrocode}
\def\version{final}
\input{childdoc.def}
\childdocforwardprefix[cdocsamp]{cdocsfn}{cdocsch}
%    \end{macrocode}

%\iffalse
%</samplefinal>
%\fi
%
% %%%%%%%%%%%%%%%%%%%%%%%%%%%%%%%%%%%%%%
% \paragraph{Command Line Processing.}
%
% The following three command lines generate the output files
% |cdocscld|, |cdocscl1| and |cdocscl2|
% which should be identical to
% |cdocsdrf|, |cdocsch1| and |cdocsfn2|, respectively:
% \begin{center}
% \begin{tabular}{l}
% |latex -jobname cdocscld \|\\
% |  "\def\version{draft}\input{childdoc.def}\childdocforward{cdocsamp}"|\\
% |latex -jobname cdocscl1 \|\\
% |  "\input{childdoc.def}\childdocforward[cdocsamp]{cdocsch1}"|\\
% |latex -jobname cdocscl2 \|\\
% |  "\def\version{final}\input{childdoc.def}\childdocforward{cdocsch2}"|
% \end{tabular}
% \end{center}
% Note that the trailing backslash on each first line
% merely continues the input to the second line
% (for convenient cut ant paste).
% Furthermore, the command |latex| can be replaced by any
% of its alternative versions such as |pdflatex|.
%
% %%%%%%%%%%%%%%%%%%%%%%%%%%%%%%%%%%%%%%%%%%%%%%%%%%%%%%%%%%%%%%%%%%%%%%%%%%%%%%
% %%%%%%%%%%%%%%%%%%%%%%%%%%%%%%%%%%%%%%%%%%%%%%%%%%%%%%%%%%%%%%%%%%%%%%%%%%%%%%
% \section{Implementation}
%\iffalse
%<*package>
%\fi
%
% This section describes the definitions file |childdoc.def|.

% The definitions cannot be loaded using |\usepackage| or |\RequirePackage|
% which has a mechanism to prevent loading a style file more than once.
% When loading the definitions by means of |\input|
% multiple instances have to be prevented manually:
%\iffalse
%This code needs to be before the `\ProvidesFile' directive
%which is defined at the beginning of this file.
%Therefore it is also placed there and commented out here.
%</package>
%<*discard>
%\fi
%    \begin{macrocode}
\ifdefined\childdocmain\endinput\fi
%    \end{macrocode}
%\iffalse
%</discard>
%<*package>
%\fi
%
% \macro{\ifchilddoc}
% \macro{\ifchilddocmanual}
% The conditional |\ifchilddoc| tells whether a
% child (true) or main (false) document is being compiled.
% The conditional |\ifchilddocmanual| tells whether
% the |\includeonly| mechanism is used (false) or
% the selection of child files must be performed manually (true).
% The definitions initialise to false:
%    \begin{macrocode}
\newif\ifchilddoc
\newif\ifchilddocmanual
%    \end{macrocode}

% \macro{\childdocname}
% \macro{\childdocjob}
% The macro |\childdocname| stores the name of the main document
% to be compiled. The macro |\childdocjob| stores the name of
% the document on which the \LaTeX{} compiler was originally invoked.
% The content of |\jobname| cannot be compared
% to filenames specified in the source due to different catcodes.
% The following code rescans |\jobname|, stores the result
% in |\childdocname| and saves a copy in |\childdocjob|:
%    \begin{macrocode}
\edef\childdocname{\scantokens\expandafter{\jobname\noexpand}}
\let\childdocjob\childdocname
%    \end{macrocode}

% \macro{\childdocdisable}
% The macro |\childdocdisable| prevents the main file
% from being processed more than once.
% At this stage, the main document command |\childdocmain|
% is assumed to be called once again where it should do nothing.
% Any subsequent call to it should prevent
% a secondary processing of the main document
% It overwrites the forwarding commands
% |\childdocof| and |\childdocforward|
% with empty macros to prevent further inclusions of the main document:
%    \begin{macrocode}
\newcommand{\childdocdisable}
{
  \renewcommand{\childdocmain}[1]{\renewcommand{\childdocmain}[1]{\endinput}}
  \renewcommand{\childdocof}[1]{}
  \renewcommand{\childdocby}[2][]{}
  \renewcommand{\childdocforward}[2][]{}
  \renewcommand{\childdocdisable}{}
}
%    \end{macrocode}

% \macro{\childdocmain}
% The macro |\childdocmain| is to be called at the top of the main file
% with nothing or the main filename (without extension) as argument.
% First, it breaks loops.
% If the argument is not empty and does not match |\childdocname|
% (which is set by the first inclusion of |childdoc.def|),
% |\ifchilddoc| is set to true, |\includeonly| is applied to the child file
% and |\jobname| is set to the main file
% (for proper handling of |.aux| files):
%    \begin{macrocode}
\newcommand{\childdocmain}[1]
{
  \childdocdisable\childdocmain{}
  \if?#1?\else
    \begingroup
      \def\childdoctmp{#1}
      \ifx\childdoctmp\childdocname
        \def\childdoctmp{}
      \else
        \def\childdoctmp
        {
          \childdoctrue
          \includeonly{\childdocname}
          \def\childdocjob{#1}
          \def\jobname{#1}
        }
      \fi
      \expandafter
    \endgroup
    \childdoctmp
  \fi
}
%    \end{macrocode}

% \macro{\childdocof}
% The command |\childdocof| redirects
% compilation to the main file |#1|.
%    \begin{macrocode}
\newcommand{\childdocof}[1]
{
  \childdocdisable
  \childdoctrue
  \includeonly{\childdocname}
  \def\jobname{#1}
  \def\childdocjob{#1}
  \input{#1}
}
%    \end{macrocode}

% \macro{\childdocby}
% The command |\childdocby| ....
%    \begin{macrocode}
\newcommand{\childdocby}[2][]
{
  \childdocdisable
  \childdoctrue
  \childdocmanualtrue
  \if?#1?\else
    \def\jobname{#2}
  \fi
  \def\childdocjob{#2}
  \input{#2}
  \endinput
}
%    \end{macrocode}

% \macro{\childdocforward}
% The command |\childdocforward| redirects
% compilation to the main file or
% (if the optional argument is given) a child file.
% Parameters are set as if the main file
% or a child file starting with |\childdocof| was compiled.
% Then compilation is handed over to the main file:
%    \begin{macrocode}
\newcommand{\childdocforward}[2][]
{
  \begingroup
    \if?#1?
      \def\childdoctmp
      {
        \def\childdocname{#2}
        \def\childdocjob{#2}
        \def\jobname{#2}
        \input{#2}
        \endinput
      }
    \else
      \def\childdoctmp
      {
        \childdocdisable
        \def\childdocname{#2}
        \childdoctrue
        \includeonly{#2}
        \def\childdocjob{#1}
        \def\jobname{#1}
        \input{#1}
        \endinput
      }
    \fi
    \expandafter
  \endgroup
  \childdoctmp
}
%    \end{macrocode}

% \macro{\childdocforwardprefix}
% The command |\childdocforwardprefix| redirects
% compilation to the main or a child file by means of a pattern.
% The prefix |#1| in the current filename is replaced by |#2|
% and the suffix of the current filename is kept
% (it is assumed that the filename does not contain the substring `|~~~|'
% which is used as a delimiter).
% Compilation is handed over to the new file by |\childdocforward|:
%    \begin{macrocode}
\newcommand{\childdocforwardprefix}[3][]
{
  \begingroup
    \def\childdocextract #2##1~~~{\def\childdoctmp{\childdocforward[#1]{#3##1}}}
    \expandafter\childdocextract\childdocname~~~
    \expandafter
  \endgroup
  \childdoctmp
}
%    \end{macrocode}

% \macro{\childdoc}
% The deprecated macro |\childdoc| is a legacy version of |\childdocmain|:
%    \begin{macrocode}
\newcommand{\childdoc}{\childdocmain}
%    \end{macrocode}

% \macro{\childdocredirect}
% The deprecated macro |\childdocredirect| is a legacy version
% of |\childdocforward| and |\childdocforwardprefix|:
%    \begin{macrocode}
\newcommand{\childdocredirect}[2][]
{
  \begingroup
    \if?#1?
      \def\childdoctmp{\childdocforward{#2}}
    \else
      \def\childdoctmp{\childdocforwardprefix{#1}{#2}}
    \fi
    \expandafter
  \endgroup
  \childdoctmp
}
%    \end{macrocode}

%\iffalse
%</package>
%\fi
%
\endinput
|\\
|\childdocforward[|\textit{main}|]{|\textit{dest}|}|\\
\end{tabular}
\end{center}
%
The argument \textit{dest} is the destination file
(without extension).
It should be the main file or one of the child files.
Note that further \textsf{childdoc} directives
such as |\childdocof| and |\childdocforward|
in the indicated file will be processed in this form.
The optional argument \textit{main}
passes on directly to the main file \textit{main}
while pretending to compile the child \textit{dest}.
This form behaves as if \textit{dest}
issues |\childdocof{|\textit{main}|}| right away,
and no further \textsf{childdoc} directives will be processed.

%%%%%%%%%%%%%%%%%%%%%%%%%%%%%%%%%%%%%%%%
\DescribeMacro{\...prefix}
In the alternative form |\childdocforwardprefix|,
%
\begin{center}
\begin{tabular}{l}
|% \iffalse
%
% childdoc.dtx Copyright (C) 2017-2018 Niklas Beisert
%
% This work may be distributed and/or modified under the
% conditions of the LaTeX Project Public License, either version 1.3
% of this license or (at your option) any later version.
% The latest version of this license is in
%   http://www.latex-project.org/lppl.txt
% and version 1.3 or later is part of all distributions of LaTeX
% version 2005/12/01 or later.
%
% This work has the LPPL maintenance status `maintained'.
%
% The Current Maintainer of this work is Niklas Beisert.
%
% This work consists of the files childdoc.dtx and childdoc.ins
% and the derived files childdoc.def and cdocsamp.tex with
% cdocsch1.tex, cdocsch2.tex, cdocsdrf.tex, cdocsfn1.tex, cdocsfn2.tex.
%
%<package>\ifdefined\childdocmain\endinput\fi
%<package>\ProvidesFile{childdoc.def}[2018/12/30 v2.0 child document driver]
%<samplemain>\ProvidesFile{cdocsamp.tex}[2018/12/30 v2.0 sample for childdoc]
%<*driver>
%\ProvidesFile{childdoc.drv}[2018/12/30 v2.0 childdoc reference manual file]
\PassOptionsToClass{10pt,a4paper}{article}
\documentclass{ltxdoc}

\usepackage[margin=35mm]{geometry}
\usepackage{hyperref}
\usepackage{hyperxmp}
\usepackage[usenames]{color}

\hypersetup{colorlinks=true}
\hypersetup{pdfstartview=FitH}
\hypersetup{pdfpagemode=UseNone}
\hypersetup{pdfsource={}}
\hypersetup{pdflang={en-UK}}
\hypersetup{pdfcopyright={Copyright 2017-2018 Niklas Beisert.
  This work may be distributed and/or modified under the
  conditions of the LaTeX Project Public License, either version 1.3
  of this license or (at your option) any later version.}}
\hypersetup{pdflicenseurl={http://www.latex-project.org/lppl.txt}}
\hypersetup{pdfcontactaddress={ETH Zurich, ITP, HIT K,
  Wolfgang-Pauli-Strasse 27}}
\hypersetup{pdfcontactpostcode={8093}}
\hypersetup{pdfcontactcity={Zurich}}
\hypersetup{pdfcontactcountry={Switzerland}}
\hypersetup{pdfcontactemail={nbeisert@itp.phys.ethz.ch}}
\hypersetup{pdfcontacturl={http://people.phys.ethz.ch/\xmptilde nbeisert/}}

\newcommand{\secref}[1]{\hyperref[#1]{section \ref*{#1}}}

\parskip1ex
\parindent0pt
\let\olditemize\itemize
\def\itemize{\olditemize\parskip0pt}

\begin{document}

\title{The \textsf{childdoc} Package}
\hypersetup{pdftitle={The childdoc Package}}
\author{Niklas Beisert\\[2ex]
  Institut f\"ur Theoretische Physik\\
  Eidgen\"ossische Technische Hochschule Z\"urich\\
  Wolfgang-Pauli-Strasse 27, 8093 Z\"urich, Switzerland\\[1ex]
  \href{mailto:nbeisert@itp.phys.ethz.ch}
  {\texttt{nbeisert@itp.phys.ethz.ch}}}
\hypersetup{pdfauthor={Niklas Beisert}}
\hypersetup{pdfsubject={Manual for the LaTeX2e Package childdoc}}
\date{30 December 2018, \textsf{v2.0}}
\maketitle

\begin{abstract}\noindent
\textsf{childdoc} is a \LaTeXe{} package
that enables the direct compilation
of document sections included by |\include|
to individual files.
\end{abstract}

\begingroup
\parskip0ex
\tableofcontents
\endgroup

%%%%%%%%%%%%%%%%%%%%%%%%%%%%%%%%%%%%%%%%%%%%%%%%%%%%%%%%%%%%%%%%%%%%%%%%%%%%%%%%
%%%%%%%%%%%%%%%%%%%%%%%%%%%%%%%%%%%%%%%%%%%%%%%%%%%%%%%%%%%%%%%%%%%%%%%%%%%%%%%%
\section{Introduction}

\LaTeX{} provides a mechanism to structure a large document (such as a book)
into a main file and several child files (containing the chapters)
using the |\include| command.
This mechanism is beneficial for documents
which span hundreds of pages in order to
make the source file(s) more manageable.
Moreover, compilation can be restricted to
selected child files by means of the |\includeonly| command.
The latter feature can be used to reduce the compilation time while editing
(this was significantly more useful in the earlier days of \LaTeX{})
or to generate a smaller document which is easier to navigate.
Another application of |\includeonly| is to generate
documents consisting of selected parts of the complete document.

However, there are a few drawbacks of the plain |\include| mechanism:
\begin{itemize}
\item
The child files cannot be compiled on their own,
they can only be compiled via the main file.
A naive editing environment
(such as a text editor with an option
to have the current file processed by \LaTeX)
may require one to switch to the main file before compiling;
attempting to compile the child file produces errors.
\item
The main file must be modified (each time)
to adjust the |\includeonly| command
to the present needs. This easily leaves the main file in a messy state.
\item
The generated document will always carry the filename
of the main document. This is inconvenient if
several child files are to be compiled and
to be kept for distribution.
\end{itemize}

The present package provides a simple interface
to make child files individually compilable by \LaTeX{}.
Compiling a child file then has the same effect as compiling
the main file with an |\includeonly| command
to select the appropriate child.
Moreover the generated document will carry the name of the child
rather than the main file.
This resolves all three above issues.

This feature is meant to make the editing of books,
thesis documents and lecture notes somewhat more convenient.
However, the package can also be used efficiently for
composing a series of documents (such as exercise sheets)
which are typically distributed individually.
It then assists the author in generating the individual documents
(potentially in different versions)
as well as a document containing the collected series.
Another application is in developing style files
or other kinds of included material
where compilation of the style file could redirect
to a sample or test file.

%%%%%%%%%%%%%%%%%%%%%%%%%%%%%%%%%%%%%%%%%%%%%%%%%%%%%%%%%%%%%%%%%%%%%%%%%%%%%%%%
%%%%%%%%%%%%%%%%%%%%%%%%%%%%%%%%%%%%%%%%%%%%%%%%%%%%%%%%%%%%%%%%%%%%%%%%%%%%%%%%
\section{Usage}

First of all, the package \textsf{childdoc} is \emph{not} a standard
\LaTeXe{} |.sty| style file! Therefore it needs to be invoked in
a non-standard way.

%%%%%%%%%%%%%%%%%%%%%%%%%%%%%%%%%%%%%%%%%%%%%%%%%%%%%%%%%%%%%%%%%%%%%%%%%%%%%%%%
\subsection{Included Files}
\label{sec:include}

%%%%%%%%%%%%%%%%%%%%%%%%%%%%%%%%%%%%%%%%
\DescribeMacro{\childdocmain}
To use the package, add the commands
\begin{center}
\begin{tabular}{l}
|\input{childdoc.def}|\\
|\childdocmain{}|\\
\end{tabular}
\end{center}
at the very top of the main \LaTeX{} file,
in particular \emph{before} the |\documentclass| statement!
The argument of |\childdocmain| should be left empty
(but it must be present).

%%%%%%%%%%%%%%%%%%%%%%%%%%%%%%%%%%%%%%%%
\DescribeMacro{\childdocof}
Furthermore, add the commands
\begin{center}
\begin{tabular}{l}
|\input{childdoc.def}|\\
|\childdocof{|\textit{main}|}|\\
\end{tabular}
\end{center}
at the top of every child file \textit{child}
which is included by |\include{|\textit{child}|}|
from within the main file
(or at least for those files to be compiled individually).
The argument \textit{main} must be the filename of the main file.

There are a couple of
considerations in setting up the main and child documents:

%%%%%%%%%%%%%%%%%%%%%%%%%%%%%%%%%%%%%%%%
\paragraph{Restrictions.}

Please note the following restrictions:
\begin{itemize}
\item
|\childdocmain| must be called with one argument \textit{main}
to ensure compatibility with earlier version of the package.
It must either be empty (|\childdocmain{}|)
or precisely match the filename of the main file in which it is specified.
See \secref{sec:detection} for further information.
\item
The filename \textit{main} must be specified without the |.tex| extension.
\item
The filename \textit{main} is case sensitive
(even in case-insensitive file systems)
due to internal string comparison.
\item
The argument \textit{main} should be fully expanded, it cannot be a macro.
\item
Subdirectories and special characters should be avoided in filenames.
\item
The command |\childdocmain{|\textit{main}|}| must be followed by a whitespace.
It should not be followed immediately by another command
or by a comment mark `|%|'.
This is because the \TeX{} parser reads the token immediately following
the argument of |\childdocmain| and puts it
at the beginning of every child section;
however, a white\-space is ignored.
\end{itemize}

%%%%%%%%%%%%%%%%%%%%%%%%%%%%%%%%%%%%%%%%
\paragraph{Content of Main File.}

It is advisable to place all content in the child files included by |\include|.
Any output contained in the main file will appear in all child documents
unless suppressed manually;
it cannot be suppressed automatically by the |\includeonly| directive
and thus should normally be avoided.
A method to include some content in the main file
by means of conditional processing is described in \secref{sec:conditional}.

%%%%%%%%%%%%%%%%%%%%%%%%%%%%%%%%%%%%%%%%
\paragraph{Page Numbering.}

When only a part of the document is compiled,
the appropriate numbering of pages
(as well as other status parameters)
is determined from the |.aux| files.
The latter contain information from previous passes.
However this information needs to propagate through
all intermediate child documents.
Therefore the page numbering in child documents may well
be inconsistent until the complete document is compiled at least once.

A useful (if unconventional) way to always ensure a consistent
page numbering is to restart the numbering in each child document
and denote the pages by `\textit{child}|.|\textit{page}'
where \textit{child} represents the chapter/section number of the child file.
This can be achieved by the command
|\numberwithin{page}{|\textit{child}|}|
of the \textsf{amsmath} package
where \textit{child} can be |chapter| or |section|
depending on the chosen structuring.
Alternatively, one can modify the macro |\thepage| appropriately
and reset the counter |page| at the start of each child file.

%%%%%%%%%%%%%%%%%%%%%%%%%%%%%%%%%%%%%%%%%%%%%%%%%%%%%%%%%%%%%%%%%%%%%%%%%%%%%%%%
\subsection{Conditional Processing}
\label{sec:conditional}

The package provides a mechanism to compile different versions
of a document. To customise the versions further some conditional processing
can come in handy to distinguish which version is being compiled.
The package provides two macros to describe the compilation context:

%%%%%%%%%%%%%%%%%%%%%%%%%%%%%%%%%%%%%%%%
\DescribeMacro{\ifchilddoc}
The conditional |\ifchilddoc| distinguishes between the compilation of
child documents and the main document:
%
\begin{center}
|\ifchilddoc |\textit{child-code}| |[|\||else |\textit{main-code}]| \||fi|
\end{center}

%%%%%%%%%%%%%%%%%%%%%%%%%%%%%%%%%%%%%%%%
\DescribeMacro{\childdocname}
\DescribeMacro{\childdocjob}
The macro |\childdocname| contains the filename (without extension)
of the main or child file being processed.
Note that |\childdocjob| will always contain the name of the main file.

%%%%%%%%%%%%%%%%%%%%%%%%%%%%%%%%%%%%%%%%
\paragraph{Title Page.}

Conditional processing can be used to include a title or banner page
in the main document when proper precautions are taken.
Importantly, the code in the main file should ensure that the page counter
(as well as other status parameters which are stored in the |.aux| files)
takes the same value after the conditional processing.
Otherwise the page numbers may take divergent values
depending on which part is compiled.

For example, a title page could be declared by:
%
\begin{center}
\begin{tabular}{l}
|\ifchilddoc\||else|\\
|\addtocounter{page}{-1}|\\
\textit{code for title page}\\
|\newpage|\\
|\||fi|
\end{tabular}
\end{center}
%
A banner page for the child documents can be generated by:
%
\begin{center}
\begin{tabular}{l}
|\ifchilddoc|\\
|\addtocounter{page}{-1}|\\
\textit{code for banner page}\\
|\newpage|\\
|\||fi|
\end{tabular}
\end{center}
%
Here one could write a message such as:
\begin{center}
|This is the part \childdocname{} of \childdocjob{}.|
\end{center}

%%%%%%%%%%%%%%%%%%%%%%%%%%%%%%%%%%%%%%%%%%%%%%%%%%%%%%%%%%%%%%%%%%%%%%%%%%%%%%%%
\subsection{Flags}
\label{sec:flags}

The package makes it easy to generate different versions
of the main or child documents.
To this end compilation flags can be defined
and assigned different default values.
They will be particularly useful in conjunction
with the forwarding mechanism described in \secref{sec:forward}.

For example, it may be useful to have a flag |\version|
which can be set to |draft| or |final|.
The document source will contain some conditional code
depending on the value of |\version|.
Suppose further, the flag should default to |final| for the main file
and to |draft| for child files
which is a natural assignment for editing the document.
This is achieved by placing the following code
in the preamble of the main document
(below the |\childdocmain| directive):
%
\begin{center}
\begin{tabular}{l}
|\ifchilddoc|\\
|\providecommand{\version}{draft}|\\
|\||else|\\
|\providecommand{\version}{final}|\\
|\||fi|
\end{tabular}
\end{center}
%
The definition by |\providecommand| makes sure
that previous definitions are not overwritten.
Further statements |\providecommand{\version}{...}|
can thus be added before the above code to override it.

For the main file, one might add a line
(between |\childdocmain| and the above block)
%
\begin{center}
|%\ifchilddoc\||else\providecommand{\version}{draft}\||fi|
\end{center}
%
which can be uncommented to produce a draft version.
Likewise one can add a line to the very top of a child file
(above the |\childdocof{|\textit{main}|}| directive)
%
\begin{center}
|%\providecommand{\version}{final}|
\end{center}
%
which can be uncommented to produce the final version of this child document.

%%%%%%%%%%%%%%%%%%%%%%%%%%%%%%%%%%%%%%%%%%%%%%%%%%%%%%%%%%%%%%%%%%%%%%%%%%%%%%%%
\subsection{Forwarding}
\label{sec:forward}

Different versions of the main or child documents
using compilation flags as described in \secref{sec:flags}
can be (permanently) stored in different files
for convenient compilation, viewing and distribution.
To this end, the package defines a command
to pass on compilation to a different file:

%%%%%%%%%%%%%%%%%%%%%%%%%%%%%%%%%%%%%%%%
\DescribeMacro{\childdocforward}
The command |\childdocforward| redirects processing to
another source file:
%
\begin{center}
\begin{tabular}{l}
|\input{childdoc.def}|\\
|\childdocforward[|\textit{main}|]{|\textit{dest}|}|\\
\end{tabular}
\end{center}
%
The argument \textit{dest} is the destination file
(without extension).
It should be the main file or one of the child files.
Note that further \textsf{childdoc} directives
such as |\childdocof| and |\childdocforward|
in the indicated file will be processed in this form.
The optional argument \textit{main}
passes on directly to the main file \textit{main}
while pretending to compile the child \textit{dest}.
This form behaves as if \textit{dest}
issues |\childdocof{|\textit{main}|}| right away,
and no further \textsf{childdoc} directives will be processed.

%%%%%%%%%%%%%%%%%%%%%%%%%%%%%%%%%%%%%%%%
\DescribeMacro{\...prefix}
In the alternative form |\childdocforwardprefix|,
%
\begin{center}
\begin{tabular}{l}
|\input{childdoc.def}|\\
|\childdocforwardprefix[|\textit{main}|]{|\textit{prefix}|}{|\textit{dest}|}|
\end{tabular}
\end{center}
%
the destination file is determined by a pattern
depending on the current file:
To make this work, the current file must be called
`{\textit{prefix}\hspace{0.2em}\textit{suffix}}'
with \textit{prefix} matching precisely the argument.
Processing is then passed on to the file
`{\textit{dest}\hspace{0.2em}\textit{suffix}}'.
Surely, the same effect is achieved by
directly specifying the
argument `{\textit{dest}\hspace{0.2em}\textit{suffix}}'
in the first form.
However, that requires to set up a different file
for each child. With the alternative form of the command
all these files can have exactly the same content
which simplifies setting them up and maintaining them.

For example, the following file |draft.tex|
with a compilation flag |\version| as described in \secref{sec:flags}
compiles the main document as a draft:
%
\begin{center}
\begin{tabular}{l}
|\def\version{draft}|\\
|\input{childdoc.def}|\\
|\childdocforward{|\textit{main}|}|
\end{tabular}
\end{center}
%
Likewise, the following files |final|\textit{nn}|.tex|
compile the final version of the child document
|child|\textit{nn}|.tex|:
%
\begin{center}
\begin{tabular}{l}
|\def\version{final}|\\
|\input{childdoc.def}|\\
|\childdocforwardprefix{final}{child}|
\end{tabular}
\end{center}
%

Note that when several versions of a main file and/or of each child file
are to be generated, it may be convenient to set up a |Makefile| or
shell script to automatise the process.

%%%%%%%%%%%%%%%%%%%%%%%%%%%%%%%%%%%%%%%%%%%%%%%%%%%%%%%%%%%%%%%%%%%%%%%%%%%%%%%%
\subsection{Command Line Processing}
\label{sec:commandline}

The effect of redirection files can also be achieved by invoking
the \LaTeX{} compiler with a more elaborate command line.
Most conveniently this should be done as part
of a shell script or a |Makefile|.

When using \textsf{childdoc} in the main file, the following
command lines effectively perform a redirection
(note that depending on the shell being used,
backslashes may have to be doubled: `|\|' $\to$ `|\\|'):
%
\begin{center}
|... -jobname "|\textit{target}|" |\\|"|[\textit{flags}]%
|\input{childdoc.def}\childdocforward[|\textit{main}|]{|\textit{dest}|}"|
\end{center}
%
Here \textit{target} is the name of the output file,
\textit{main} is the name of the main file
and \textit{dest} is the name of the main or child file to be processed
(all filenames without extensions).
The optional argument \textit{main} can be omitted
if \textit{main} matches \textit{dest}.
Optionally, compilation \textit{flags} can be defined via |\def| commands.
This command line makes the \TeX{} engine believe
it is compiling the file \textit{target}
whose content is specified as the latter parameter.
The provided code then forwards the processing to
\textit{main} or \textit{dest} as described in \secref{sec:forward}.

%%%%%%%%%%%%%%%%%%%%%%%%%%%%%%%%%%%%%%%%%%%%%%%%%%%%%%%%%%%%%%%%%%%%%%%%%%%%%%%%
\subsection{Include by Input}
\label{sec:input}

Including child documents by |\include| has some restrictions by design.
Most notably, the content of a child document always occupies
its own set of pages; pages cannot be shared between child documents.
Usually, this behaviour makes perfect sense
because each child document contain an essential part of the document.
However, in some situations it may be desirable to compose
a document from a collection of parts
without having mandatory page breaks between then.
For this case, the package
provides a mechanism to include parts
by |\input| which can also be processed individually.
However, by construction this mechanism
requires manual handling of the content to be output.

%%%%%%%%%%%%%%%%%%%%%%%%%%%%%%%%%%%%%%%%
\DescribeMacro{\ifchilddocmanual}
The main file should be prepared as usual, see \secref{sec:include}.
However, the document body must make a distinction
between processing of an individual part and of the main document, e.g.:
%
\begin{center}
\begin{tabular}{l}
|\ifchilddocmanual|\\
|\input{\childdocname}|\\
|\||else|\\
\textit{document body with }|\input{|\textit{part}|}|\\
|\||fi|
\end{tabular}
\end{center}
%
The conditional |\ifchilddocmanual| is true whenever
a part to be included by |\input| is being compiled,
and the name of the part is stored in |\childdocname|.

%%%%%%%%%%%%%%%%%%%%%%%%%%%%%%%%%%%%%%%%
\DescribeMacro{\childdocby}
Each part to be included by |\input| should start with:
%
\begin{center}
\begin{tabular}{l}
|\input{childdoc.def}|\\
|\childdocby{|\textit{main}|}|\\
\end{tabular}
\end{center}
%
The directive |\childdocby| is similar to |\childdocof|
described in \secref{sec:include},
but the subsequent selection of content must be done manually.
To that end, both |\ifchilddoc| and |\ifchilddocmanual|
will be true upon processing of a part,
and the name of the part is stored in |\childdocname|.
Note that |\jobname| will be set to the filename of the current part
so that each part receives an individual |.aux| file
that does not interfere with the |.aux| file(s) of the main document.
This behaviour can be altered by the alternative form
|\childdocby[*]{|\textit{main}|}| (with a non-empty optional argument)
which uses the |.aux| file of the main document
by setting |\jobname| to \textit{main}.

%%%%%%%%%%%%%%%%%%%%%%%%%%%%%%%%%%%%%%%%%%%%%%%%%%%%%%%%%%%%%%%%%%%%%%%%%%%%%%%%
\subsection{Driver Development}
\label{sec:driver}

The \textsf{childdoc} mechanism can also be use for the development
of definition files such as \LaTeX{} styles or classes.
This case differs from the above setup with multiple parts
included by |\include| in that no |\includeonly| should be invoked.
This can be achieved by starting the include file
(before |\ProvidesPackage|) with:
%
\begin{center}
\begin{tabular}{l}
|\input{childdoc.def}|\\
|\childdocforward{|\textit{main}|}|\\
\end{tabular}
\end{center}
%
or alternatively with:
%
\begin{center}
\begin{tabular}{l}
|\input{childdoc.def}|\\
|\childdocby{|\textit{main}|}|\\
\end{tabular}
\end{center}
%
Both forms have slightly different effects as described above.
The main file is prepared as usual, see \secref{sec:include}.

%%%%%%%%%%%%%%%%%%%%%%%%%%%%%%%%%%%%%%%%%%%%%%%%%%%%%%%%%%%%%%%%%%%%%%%%%%%%%%%%
\subsection{Legacy Detection}
\label{sec:detection}

The directive |\childdocmain| in the main file can detect
whether the complete document or merely a child is to be compiled
even without using the directive |\childdocof|.
This method is deprecated because it is less robust
and there is no compelling reason to use it;
it is merely provided for backward compatibility
and it may be removed in future versions.

If the detection mechanism is to be used,
it is mandatory to correctly specify
the filename of the main file as the argument of |\childdocmain|:
%
\begin{center}
\begin{tabular}{l}
|\input{childdoc.def}|\\
|\childdocmain{|\textit{main}|}|\\
\end{tabular}
\end{center}
%
If |\jobname| does not match the argument \textit{main} of |\childdocmain|,
it is assumed that |\jobname| points to the child file to be compiled.
When using |\childdocmain| with the main file specified as argument,
it suffices to start a child file
with just |\input{|\textit{main}|}|
without loading of the package and using |\childdocof|.
If instead all processing is done
with the appropriate \textsf{childdoc} directives,
the argument of \textit{main} of |\childdocmain| can be empty.

An alternative version of the command line processing described
in \secref{sec:commandline} using the detection mechanism reads:
%
\begin{center}
|... -jobname "|\textit{target}|" "|[\textit{flags}]%
[|\def\jobname{|\textit{dest}|}|]|\input{|\textit{main}|}"|
\end{center}

%%%%%%%%%%%%%%%%%%%%%%%%%%%%%%%%%%%%%%%%%%%%%%%%%%%%%%%%%%%%%%%%%%%%%%%%%%%%%%%%
\subsection{Manual Code}
\label{sec:manual}

In case one cannot be certain whether the definitions file |childdoc.def|
is installed on the target \TeX{} distribution
and one prefers not to ship it,
it is conceivable to paste a few relevant commands into the sources.

To that end, drop all statements |\input{childdoc.def}|
and perform the replacements as outlined below.
Instead of |\childdocmain{|\textit{main}|}| add the following code
to the top of the main file:
%
\begin{center}
\begin{tabular}{l}
|\||ifdefined\childdocname\endinput\||fi\newif\ifchilddoc|\\
|\edef\childdocname{\scantokens\expandafter{\jobname\noexpand}}|\\
|\def\childdocmain{|\textit{main}|}\||ifx\childdocmain\childdocname\||else|\\
|\childdoctrue\includeonly{\childdocname}\let\jobname\childdocmain\||fi|\\
\end{tabular}
\end{center}
%
Instead of |\childdocof{|\textit{main}|}| just include the main file
at the top of each child file:
%
\begin{center}
|\input{|\textit{main}|}|
\end{center}
%
A simple redirection |\childdocforward{|\textit{dest}|}| is achieved by:
%
\begin{center}
|\def\jobname{|\textit{dest}|}\input{\jobname}|
\end{center}
%
The redirection with prefix
|\childdocforwardprefix[|\textit{prefix}|]{|\textit{dest}|}|
is accomplished by:
%
\begin{center}
\begin{tabular}{l}
|{\edef\jobname{\scantokens\expandafter{\jobname\noexpand}}|\\
|\def\redirectjob |\textit{prefix}|#1~~~{\gdef\jobname{|\textit{dest}|#1}}|\\
|\expandafter\redirectjob\jobname~~~}\input{\jobname}|
\end{tabular}
\end{center}

In an alternative approach,
child documents can be compiled by a specific command line
without additional code or specific definitions:
%
\begin{center}
|... -jobname "|\textit{target}|" "|[\textit{flags}]%
|\includeonly{|\textit{dest}|}\input{|\textit{main}|}"|
\end{center}
%

%%%%%%%%%%%%%%%%%%%%%%%%%%%%%%%%%%%%%%%%%%%%%%%%%%%%%%%%%%%%%%%%%%%%%%%%%%%%%%%%
%%%%%%%%%%%%%%%%%%%%%%%%%%%%%%%%%%%%%%%%%%%%%%%%%%%%%%%%%%%%%%%%%%%%%%%%%%%%%%%%
\section{Information}

%%%%%%%%%%%%%%%%%%%%%%%%%%%%%%%%%%%%%%%%%%%%%%%%%%%%%%%%%%%%%%%%%%%%%%%%%%%%%%%%
\subsection{Copyright}

Copyright \copyright{} 2017--2018 Niklas Beisert

This work may be distributed and/or modified under the
conditions of the \LaTeX{} Project Public License, either version 1.3
of this license or (at your option) any later version.
The latest version of this license is in
  \url{http://www.latex-project.org/lppl.txt}
and version 1.3 or later is part of all distributions of \LaTeX{}
version 2005/12/01 or later.

This work has the LPPL maintenance status `maintained'.

The Current Maintainer of this work is Niklas Beisert.

This work consists of the files |README.txt|, |childdoc.ins| and |childdoc.dtx|
as well as the derived files |childdoc.def|, |cdocsamp.tex|
with |cdocsch1.tex|, |cdocsch2.tex|, |cdocspt3.tex|, |cdocspt4.tex|,
|cdocsdrf.tex|, |cdocsfn1.tex|, |cdocsfn2.tex|
as well as |childdoc.pdf|.

%%%%%%%%%%%%%%%%%%%%%%%%%%%%%%%%%%%%%%%%%%%%%%%%%%%%%%%%%%%%%%%%%%%%%%%%%%%%%%%%
\subsection{Files and Installation}

The package consists of the files:
%
\begin{center}
\begin{tabular}{ll}
    |README.txt|   & readme file \\
    |childdoc.ins| & installation file \\
    |childdoc.dtx| & source file \\
    |childdoc.def| & definition file \\
    |cdocsamp.tex| & sample main file \\
    |cdocsch1.tex| & sample include file \\
    |cdocsch2.tex| & sample include file \\
    |cdocspt3.tex| & sample part file \\
    |cdocspt4.tex| & sample part file \\
    |cdocsdrf.tex| & sample redirection file \\
    |cdocsfn1.tex| & sample redirection file \\
    |cdocsfn2.tex| & sample redirection file \\
    |childdoc.pdf| & manual
\end{tabular}
\end{center}
%
The distribution consists of the files
|README.txt|, |childdoc.ins| and |childdoc.dtx|.
%
\begin{itemize}
\item
Run (pdf)\LaTeX{} on |childdoc.dtx|
to compile the manual |childdoc.pdf| (this file).
\item
Run \LaTeX{} on |childdoc.ins| to create the definitions file |childdoc.def|
and the sample |cdocsamp.tex| with include files
|cdocsch1.tex|, |cdocsch2.tex|, |cdocspt3.tex|, |cdocspt4.tex|,
|cdocsdrf.tex|, |cdocsfn1.tex|, |cdocsfn2.tex|.
Then copy the file |childdoc.def| to an appropriate directory of your \LaTeX{}
distribution, e.g.\ \textit{texmf-root}|/tex/latex/childdoc|.
\end{itemize}

%%%%%%%%%%%%%%%%%%%%%%%%%%%%%%%%%%%%%%%%%%%%%%%%%%%%%%%%%%%%%%%%%%%%%%%%%%%%%%%%
\subsection{Related CTAN Packages}

There are several other packages which offer a similar functionality:
%
\begin{itemize}
\item
The packages
\href{http://ctan.org/pkg/docmute}{\textsf{docmute}},
\href{http://ctan.org/pkg/includex}{\textsf{includex}} and
\href{http://ctan.org/pkg/standalone}{\textsf{standalone}}
provide commands to include only the document body of
a child file thus allowing both files to be compiled individually.
\item
The packages \href{http://ctan.org/pkg/subdocs}{\textsf{subdocs}}
and \href{http://ctan.org/pkg/subfiles}{\textsf{subfiles}}
provide structures in which the main and child documents can be
encapsulated and allowing them to be compiled individually.
The inclusion mechanism is different from the conventional |\include|.
\item
The package \href{http://ctan.org/pkg/combine}{\textsf{combine}}
is an elaborate solution to combine several documents into one.
\end{itemize}
%
See also the CTAN topic \href{http://ctan.org/topic/subdocs}{\textsf{subdocs}}
for further related packages.
The present package differs from the above solutions in that
a document structure constructed with the conventional |\include| mechanism
just needs two extra commands at the top of every file
such that all constituent files can be compiled individually.

%%%%%%%%%%%%%%%%%%%%%%%%%%%%%%%%%%%%%%%%%%%%%%%%%%%%%%%%%%%%%%%%%%%%%%%%%%%%%%%%
%\subsection{Feature Suggestions}
%
%The following is a list of features which may be useful for future
%versions of this package:
%%
%\begin{itemize}
%\item
%\ldots
%\end{itemize}

%%%%%%%%%%%%%%%%%%%%%%%%%%%%%%%%%%%%%%%%%%%%%%%%%%%%%%%%%%%%%%%%%%%%%%%%%%%%%%%%
\subsection{Revision History}

%%%%%%%%%%%%%%%%%%%%%%%%%%%%%%%%%%%%%%%%
\paragraph{v2.0:} 2018/12/30

\begin{itemize}
\item
immediate forward processing
\item
added |\childdocby| mechanism
\item
manual restructured
\end{itemize}

%%%%%%%%%%%%%%%%%%%%%%%%%%%%%%%%%%%%%%%%
\paragraph{v1.6:} 2018/01/17

\begin{itemize}
\item
application for development of include files
\item
corrections to manual
\end{itemize}

%%%%%%%%%%%%%%%%%%%%%%%%%%%%%%%%%%%%%%%%
\paragraph{v1.5:} 2017/05/21

\begin{itemize}
\item
more complete structuring introduced
\item
|\childdocof| introduced
\item
|\childdoc| renamed to |\childdocmain|
\item
|\childredirect| renamed to |\childdocforward| and |\childdocforwardprefix|
and functionality expanded
\end{itemize}

%%%%%%%%%%%%%%%%%%%%%%%%%%%%%%%%%%%%%%%%
\paragraph{v1.0:} 2017/04/27

\begin{itemize}
\item
manual and install package
\item
first version published on CTAN
\end{itemize}

%%%%%%%%%%%%%%%%%%%%%%%%%%%%%%%%%%%%%%%%
\paragraph{v0.6:} 2017/04/26

\begin{itemize}
\item
redirection mechanism added
\end{itemize}

%%%%%%%%%%%%%%%%%%%%%%%%%%%%%%%%%%%%%%%%
\paragraph{v0.5:} 2017/04/26

\begin{itemize}
\item
functionality in definition file
\end{itemize}


%%%%%%%%%%%%%%%%%%%%%%%%%%%%%%%%%%%%%%%%%%%%%%%%%%%%%%%%%%%%%%%%%%%%%%%%%%%%%%%%
%%%%%%%%%%%%%%%%%%%%%%%%%%%%%%%%%%%%%%%%%%%%%%%%%%%%%%%%%%%%%%%%%%%%%%%%%%%%%%%%
%%%%%%%%%%%%%%%%%%%%%%%%%%%%%%%%%%%%%%%%%%%%%%%%%%%%%%%%%%%%%%%%%%%%%%%%%%%%%%%%
\appendix

\settowidth\MacroIndent{\rmfamily\scriptsize 000\ }

 \DocInput{childdoc.dtx}

\end{document}
%</driver>
% \fi
%
% %%%%%%%%%%%%%%%%%%%%%%%%%%%%%%%%%%%%%%%%%%%%%%%%%%%%%%%%%%%%%%%%%%%%%%%%%%%%%%
% %%%%%%%%%%%%%%%%%%%%%%%%%%%%%%%%%%%%%%%%%%%%%%%%%%%%%%%%%%%%%%%%%%%%%%%%%%%%%%
% \section{Sample}
%\iffalse
%<*samplemain>
%\fi
%
% The following presents a sample document
% with two chapters, two parts, a title page,
% a compile flag as well as three forwarding files to set the flag.
% It consists of eight |.tex| files:
% \begin{center}
% \begin{tabular}{ll}
% |cdocsamp.tex|&main file\\
% |cdocsch1.tex|&include file for chapter 1\\
% |cdocsch2.tex|&include file for chapter 2\\
% |cdocspt3.tex|&include file for part 3\\
% |cdocspt4.tex|&include file for part 4\\
% |cdocsdrf.tex|&forwarding file for main file in draft mode\\
% |cdocsfi1.tex|&forwarding file for final version of chapter 1\\
% |cdocsfi2.tex|&forwarding file for final version of chapter 2\\
% \end{tabular}
% \end{center}
% Each of the eight files can be compiled directly by the \LaTeX{} compiler.
%
% %%%%%%%%%%%%%%%%%%%%%%%%%%%%%%%%%%%%%%
% \paragraph{Main File.}
%
% The main file is called |cdocsamp.tex|.
%
% Load the \textsf{childdoc} definitions and
% declare the filename for the main document:
%    \begin{macrocode}
\input{childdoc.def}
\childdocmain{}
%    \end{macrocode}

% Optional override for |\version| flag:
%    \begin{macrocode}
%%\ifchilddoc\else\providecommand{\version}{draft}\fi
%    \end{macrocode}

% Define the default values for the |\version| flag
% (|final| for the main file and |draft| for childs):
%    \begin{macrocode}
\ifchilddoc
\providecommand{\version}{draft}
\else
\providecommand{\version}{final}
\fi
%    \end{macrocode}

% Load the standard document class:
%    \begin{macrocode}
\documentclass[12pt]{article}
%    \end{macrocode}

% Start the document body:
%    \begin{macrocode}
\begin{document}
%    \end{macrocode}

% Declare a title page.
% Print title, part of document being processed and version flag:
%    \begin{macrocode}
\addtocounter{page}{-1}
\begin{center}
{\LARGE\bfseries{}childdoc example\par}
\vspace{1cm}
\ifchilddoc
\ifchilddocmanual part\else chapter\fi:
`\childdocname' of `\childdocjob'\par
\else
main document: `\childdocjob'\par
\fi
version: \version\par
\end{center}
\newpage
%    \end{macrocode}

% Manually include selected file,
% otherwise process as usual:
%    \begin{macrocode}
\ifchilddocmanual
\section*{part `\childdocname'}
\input{\childdocname}
\else
%    \end{macrocode}

% Include the two chapters:
%    \begin{macrocode}
\include{cdocsch1}
\include{cdocsch2}
%    \end{macrocode}

% Include the two parts unless only chapters should be displayed:
%    \begin{macrocode}
\ifchilddoc\else
\section{part three}
\input{cdocspt3}
\section{part four}
\input{cdocspt4}
\fi
%    \end{macrocode}

% Process as usual until here:
%    \begin{macrocode}
\fi
%    \end{macrocode}

% End of document body:
%    \begin{macrocode}
\end{document}
%    \end{macrocode}
%\iffalse
%</samplemain>
%\fi
%
% %%%%%%%%%%%%%%%%%%%%%%%%%%%%%%%%%%%%%%
% \paragraph{Chapter Include Files.}
%
% The include files are called |cdocsch1.tex| and |cdocsch2.tex|.
%
%\iffalse
%<*samplechap1|samplechap2>
%\fi

% Optional override for |\version| flag:
%    \begin{macrocode}
%%\providecommand{\version}{final}
%    \end{macrocode}

% Include the main document:
%    \begin{macrocode}
\input{childdoc.def}
\childdocof{cdocsamp}
%    \end{macrocode}

%\iffalse
%</samplechap1|samplechap2>
%\fi
%
%\iffalse
%<*samplechap1>
%\fi
% Some text for chapter 1:
%    \begin{macrocode}
\section{one}
some text in chapter one
%    \end{macrocode}

%\iffalse
%</samplechap1>
%\fi
% Some text for chapter 2:
%\iffalse
%<*samplechap2>
%\fi
%    \begin{macrocode}
\section{two}
more text in chapter two
%    \end{macrocode}

%\iffalse
%</samplechap2>
%\fi
%
% %%%%%%%%%%%%%%%%%%%%%%%%%%%%%%%%%%%%%%
% \paragraph{Part Include Files.}
%
% The include files are called |cdocspt3.tex| and |cdocspt4.tex|.
%
%\iffalse
%<*samplepart3|samplepart4>
%\fi

% Optional override for |\version| flag:
%    \begin{macrocode}
%%\providecommand{\version}{final}
%    \end{macrocode}

% Include the main document:
%    \begin{macrocode}
\input{childdoc.def}
\childdocby{cdocsamp}
%    \end{macrocode}

%\iffalse
%</samplepart3|samplepart4>
%\fi
%
%\iffalse
%<*samplepart3>
%\fi
% Some text for part 3:
%    \begin{macrocode}
some text in part three
%    \end{macrocode}

%\iffalse
%</samplepart3>
%\fi
% Some text for part 4:
%\iffalse
%<*samplepart4>
%\fi
%    \begin{macrocode}
more text in part four
%    \end{macrocode}

%\iffalse
%</samplepart4>
%\fi
%
% %%%%%%%%%%%%%%%%%%%%%%%%%%%%%%%%%%%%%%
% \paragraph{Forwarding for a Complete Draft.}
%
% The following forwarding file |cdocsdrf.tex|
% compiles the main document in draft mode:
%\iffalse
%<*sampledraft>
%\fi
%    \begin{macrocode}
\def\version{draft}
\input{childdoc.def}
\childdocforward{cdocsamp}
%    \end{macrocode}

%\iffalse
%</sampledraft>
%\fi
%
% %%%%%%%%%%%%%%%%%%%%%%%%%%%%%%%%%%%%%%
% \paragraph{Forwarding for Final Version of the Chapters.}
%
% The following forwarding files |cdocsfn1.tex| and |cdocsfn2.tex|
% (with identical content)
% compile the final versions of the child documents
% |cdocsch1.tex| and |cdocsch2.tex|, respectively:
%\iffalse
%<*samplefinal>
%\fi
%    \begin{macrocode}
\def\version{final}
\input{childdoc.def}
\childdocforwardprefix[cdocsamp]{cdocsfn}{cdocsch}
%    \end{macrocode}

%\iffalse
%</samplefinal>
%\fi
%
% %%%%%%%%%%%%%%%%%%%%%%%%%%%%%%%%%%%%%%
% \paragraph{Command Line Processing.}
%
% The following three command lines generate the output files
% |cdocscld|, |cdocscl1| and |cdocscl2|
% which should be identical to
% |cdocsdrf|, |cdocsch1| and |cdocsfn2|, respectively:
% \begin{center}
% \begin{tabular}{l}
% |latex -jobname cdocscld \|\\
% |  "\def\version{draft}\input{childdoc.def}\childdocforward{cdocsamp}"|\\
% |latex -jobname cdocscl1 \|\\
% |  "\input{childdoc.def}\childdocforward[cdocsamp]{cdocsch1}"|\\
% |latex -jobname cdocscl2 \|\\
% |  "\def\version{final}\input{childdoc.def}\childdocforward{cdocsch2}"|
% \end{tabular}
% \end{center}
% Note that the trailing backslash on each first line
% merely continues the input to the second line
% (for convenient cut ant paste).
% Furthermore, the command |latex| can be replaced by any
% of its alternative versions such as |pdflatex|.
%
% %%%%%%%%%%%%%%%%%%%%%%%%%%%%%%%%%%%%%%%%%%%%%%%%%%%%%%%%%%%%%%%%%%%%%%%%%%%%%%
% %%%%%%%%%%%%%%%%%%%%%%%%%%%%%%%%%%%%%%%%%%%%%%%%%%%%%%%%%%%%%%%%%%%%%%%%%%%%%%
% \section{Implementation}
%\iffalse
%<*package>
%\fi
%
% This section describes the definitions file |childdoc.def|.

% The definitions cannot be loaded using |\usepackage| or |\RequirePackage|
% which has a mechanism to prevent loading a style file more than once.
% When loading the definitions by means of |\input|
% multiple instances have to be prevented manually:
%\iffalse
%This code needs to be before the `\ProvidesFile' directive
%which is defined at the beginning of this file.
%Therefore it is also placed there and commented out here.
%</package>
%<*discard>
%\fi
%    \begin{macrocode}
\ifdefined\childdocmain\endinput\fi
%    \end{macrocode}
%\iffalse
%</discard>
%<*package>
%\fi
%
% \macro{\ifchilddoc}
% \macro{\ifchilddocmanual}
% The conditional |\ifchilddoc| tells whether a
% child (true) or main (false) document is being compiled.
% The conditional |\ifchilddocmanual| tells whether
% the |\includeonly| mechanism is used (false) or
% the selection of child files must be performed manually (true).
% The definitions initialise to false:
%    \begin{macrocode}
\newif\ifchilddoc
\newif\ifchilddocmanual
%    \end{macrocode}

% \macro{\childdocname}
% \macro{\childdocjob}
% The macro |\childdocname| stores the name of the main document
% to be compiled. The macro |\childdocjob| stores the name of
% the document on which the \LaTeX{} compiler was originally invoked.
% The content of |\jobname| cannot be compared
% to filenames specified in the source due to different catcodes.
% The following code rescans |\jobname|, stores the result
% in |\childdocname| and saves a copy in |\childdocjob|:
%    \begin{macrocode}
\edef\childdocname{\scantokens\expandafter{\jobname\noexpand}}
\let\childdocjob\childdocname
%    \end{macrocode}

% \macro{\childdocdisable}
% The macro |\childdocdisable| prevents the main file
% from being processed more than once.
% At this stage, the main document command |\childdocmain|
% is assumed to be called once again where it should do nothing.
% Any subsequent call to it should prevent
% a secondary processing of the main document
% It overwrites the forwarding commands
% |\childdocof| and |\childdocforward|
% with empty macros to prevent further inclusions of the main document:
%    \begin{macrocode}
\newcommand{\childdocdisable}
{
  \renewcommand{\childdocmain}[1]{\renewcommand{\childdocmain}[1]{\endinput}}
  \renewcommand{\childdocof}[1]{}
  \renewcommand{\childdocby}[2][]{}
  \renewcommand{\childdocforward}[2][]{}
  \renewcommand{\childdocdisable}{}
}
%    \end{macrocode}

% \macro{\childdocmain}
% The macro |\childdocmain| is to be called at the top of the main file
% with nothing or the main filename (without extension) as argument.
% First, it breaks loops.
% If the argument is not empty and does not match |\childdocname|
% (which is set by the first inclusion of |childdoc.def|),
% |\ifchilddoc| is set to true, |\includeonly| is applied to the child file
% and |\jobname| is set to the main file
% (for proper handling of |.aux| files):
%    \begin{macrocode}
\newcommand{\childdocmain}[1]
{
  \childdocdisable\childdocmain{}
  \if?#1?\else
    \begingroup
      \def\childdoctmp{#1}
      \ifx\childdoctmp\childdocname
        \def\childdoctmp{}
      \else
        \def\childdoctmp
        {
          \childdoctrue
          \includeonly{\childdocname}
          \def\childdocjob{#1}
          \def\jobname{#1}
        }
      \fi
      \expandafter
    \endgroup
    \childdoctmp
  \fi
}
%    \end{macrocode}

% \macro{\childdocof}
% The command |\childdocof| redirects
% compilation to the main file |#1|.
%    \begin{macrocode}
\newcommand{\childdocof}[1]
{
  \childdocdisable
  \childdoctrue
  \includeonly{\childdocname}
  \def\jobname{#1}
  \def\childdocjob{#1}
  \input{#1}
}
%    \end{macrocode}

% \macro{\childdocby}
% The command |\childdocby| ....
%    \begin{macrocode}
\newcommand{\childdocby}[2][]
{
  \childdocdisable
  \childdoctrue
  \childdocmanualtrue
  \if?#1?\else
    \def\jobname{#2}
  \fi
  \def\childdocjob{#2}
  \input{#2}
  \endinput
}
%    \end{macrocode}

% \macro{\childdocforward}
% The command |\childdocforward| redirects
% compilation to the main file or
% (if the optional argument is given) a child file.
% Parameters are set as if the main file
% or a child file starting with |\childdocof| was compiled.
% Then compilation is handed over to the main file:
%    \begin{macrocode}
\newcommand{\childdocforward}[2][]
{
  \begingroup
    \if?#1?
      \def\childdoctmp
      {
        \def\childdocname{#2}
        \def\childdocjob{#2}
        \def\jobname{#2}
        \input{#2}
        \endinput
      }
    \else
      \def\childdoctmp
      {
        \childdocdisable
        \def\childdocname{#2}
        \childdoctrue
        \includeonly{#2}
        \def\childdocjob{#1}
        \def\jobname{#1}
        \input{#1}
        \endinput
      }
    \fi
    \expandafter
  \endgroup
  \childdoctmp
}
%    \end{macrocode}

% \macro{\childdocforwardprefix}
% The command |\childdocforwardprefix| redirects
% compilation to the main or a child file by means of a pattern.
% The prefix |#1| in the current filename is replaced by |#2|
% and the suffix of the current filename is kept
% (it is assumed that the filename does not contain the substring `|~~~|'
% which is used as a delimiter).
% Compilation is handed over to the new file by |\childdocforward|:
%    \begin{macrocode}
\newcommand{\childdocforwardprefix}[3][]
{
  \begingroup
    \def\childdocextract #2##1~~~{\def\childdoctmp{\childdocforward[#1]{#3##1}}}
    \expandafter\childdocextract\childdocname~~~
    \expandafter
  \endgroup
  \childdoctmp
}
%    \end{macrocode}

% \macro{\childdoc}
% The deprecated macro |\childdoc| is a legacy version of |\childdocmain|:
%    \begin{macrocode}
\newcommand{\childdoc}{\childdocmain}
%    \end{macrocode}

% \macro{\childdocredirect}
% The deprecated macro |\childdocredirect| is a legacy version
% of |\childdocforward| and |\childdocforwardprefix|:
%    \begin{macrocode}
\newcommand{\childdocredirect}[2][]
{
  \begingroup
    \if?#1?
      \def\childdoctmp{\childdocforward{#2}}
    \else
      \def\childdoctmp{\childdocforwardprefix{#1}{#2}}
    \fi
    \expandafter
  \endgroup
  \childdoctmp
}
%    \end{macrocode}

%\iffalse
%</package>
%\fi
%
\endinput
|\\
|\childdocforwardprefix[|\textit{main}|]{|\textit{prefix}|}{|\textit{dest}|}|
\end{tabular}
\end{center}
%
the destination file is determined by a pattern
depending on the current file:
To make this work, the current file must be called
`{\textit{prefix}\hspace{0.2em}\textit{suffix}}'
with \textit{prefix} matching precisely the argument.
Processing is then passed on to the file
`{\textit{dest}\hspace{0.2em}\textit{suffix}}'.
Surely, the same effect is achieved by
directly specifying the
argument `{\textit{dest}\hspace{0.2em}\textit{suffix}}'
in the first form.
However, that requires to set up a different file
for each child. With the alternative form of the command
all these files can have exactly the same content
which simplifies setting them up and maintaining them.

For example, the following file |draft.tex|
with a compilation flag |\version| as described in \secref{sec:flags}
compiles the main document as a draft:
%
\begin{center}
\begin{tabular}{l}
|\def\version{draft}|\\
|% \iffalse
%
% childdoc.dtx Copyright (C) 2017-2018 Niklas Beisert
%
% This work may be distributed and/or modified under the
% conditions of the LaTeX Project Public License, either version 1.3
% of this license or (at your option) any later version.
% The latest version of this license is in
%   http://www.latex-project.org/lppl.txt
% and version 1.3 or later is part of all distributions of LaTeX
% version 2005/12/01 or later.
%
% This work has the LPPL maintenance status `maintained'.
%
% The Current Maintainer of this work is Niklas Beisert.
%
% This work consists of the files childdoc.dtx and childdoc.ins
% and the derived files childdoc.def and cdocsamp.tex with
% cdocsch1.tex, cdocsch2.tex, cdocsdrf.tex, cdocsfn1.tex, cdocsfn2.tex.
%
%<package>\ifdefined\childdocmain\endinput\fi
%<package>\ProvidesFile{childdoc.def}[2018/12/30 v2.0 child document driver]
%<samplemain>\ProvidesFile{cdocsamp.tex}[2018/12/30 v2.0 sample for childdoc]
%<*driver>
%\ProvidesFile{childdoc.drv}[2018/12/30 v2.0 childdoc reference manual file]
\PassOptionsToClass{10pt,a4paper}{article}
\documentclass{ltxdoc}

\usepackage[margin=35mm]{geometry}
\usepackage{hyperref}
\usepackage{hyperxmp}
\usepackage[usenames]{color}

\hypersetup{colorlinks=true}
\hypersetup{pdfstartview=FitH}
\hypersetup{pdfpagemode=UseNone}
\hypersetup{pdfsource={}}
\hypersetup{pdflang={en-UK}}
\hypersetup{pdfcopyright={Copyright 2017-2018 Niklas Beisert.
  This work may be distributed and/or modified under the
  conditions of the LaTeX Project Public License, either version 1.3
  of this license or (at your option) any later version.}}
\hypersetup{pdflicenseurl={http://www.latex-project.org/lppl.txt}}
\hypersetup{pdfcontactaddress={ETH Zurich, ITP, HIT K,
  Wolfgang-Pauli-Strasse 27}}
\hypersetup{pdfcontactpostcode={8093}}
\hypersetup{pdfcontactcity={Zurich}}
\hypersetup{pdfcontactcountry={Switzerland}}
\hypersetup{pdfcontactemail={nbeisert@itp.phys.ethz.ch}}
\hypersetup{pdfcontacturl={http://people.phys.ethz.ch/\xmptilde nbeisert/}}

\newcommand{\secref}[1]{\hyperref[#1]{section \ref*{#1}}}

\parskip1ex
\parindent0pt
\let\olditemize\itemize
\def\itemize{\olditemize\parskip0pt}

\begin{document}

\title{The \textsf{childdoc} Package}
\hypersetup{pdftitle={The childdoc Package}}
\author{Niklas Beisert\\[2ex]
  Institut f\"ur Theoretische Physik\\
  Eidgen\"ossische Technische Hochschule Z\"urich\\
  Wolfgang-Pauli-Strasse 27, 8093 Z\"urich, Switzerland\\[1ex]
  \href{mailto:nbeisert@itp.phys.ethz.ch}
  {\texttt{nbeisert@itp.phys.ethz.ch}}}
\hypersetup{pdfauthor={Niklas Beisert}}
\hypersetup{pdfsubject={Manual for the LaTeX2e Package childdoc}}
\date{30 December 2018, \textsf{v2.0}}
\maketitle

\begin{abstract}\noindent
\textsf{childdoc} is a \LaTeXe{} package
that enables the direct compilation
of document sections included by |\include|
to individual files.
\end{abstract}

\begingroup
\parskip0ex
\tableofcontents
\endgroup

%%%%%%%%%%%%%%%%%%%%%%%%%%%%%%%%%%%%%%%%%%%%%%%%%%%%%%%%%%%%%%%%%%%%%%%%%%%%%%%%
%%%%%%%%%%%%%%%%%%%%%%%%%%%%%%%%%%%%%%%%%%%%%%%%%%%%%%%%%%%%%%%%%%%%%%%%%%%%%%%%
\section{Introduction}

\LaTeX{} provides a mechanism to structure a large document (such as a book)
into a main file and several child files (containing the chapters)
using the |\include| command.
This mechanism is beneficial for documents
which span hundreds of pages in order to
make the source file(s) more manageable.
Moreover, compilation can be restricted to
selected child files by means of the |\includeonly| command.
The latter feature can be used to reduce the compilation time while editing
(this was significantly more useful in the earlier days of \LaTeX{})
or to generate a smaller document which is easier to navigate.
Another application of |\includeonly| is to generate
documents consisting of selected parts of the complete document.

However, there are a few drawbacks of the plain |\include| mechanism:
\begin{itemize}
\item
The child files cannot be compiled on their own,
they can only be compiled via the main file.
A naive editing environment
(such as a text editor with an option
to have the current file processed by \LaTeX)
may require one to switch to the main file before compiling;
attempting to compile the child file produces errors.
\item
The main file must be modified (each time)
to adjust the |\includeonly| command
to the present needs. This easily leaves the main file in a messy state.
\item
The generated document will always carry the filename
of the main document. This is inconvenient if
several child files are to be compiled and
to be kept for distribution.
\end{itemize}

The present package provides a simple interface
to make child files individually compilable by \LaTeX{}.
Compiling a child file then has the same effect as compiling
the main file with an |\includeonly| command
to select the appropriate child.
Moreover the generated document will carry the name of the child
rather than the main file.
This resolves all three above issues.

This feature is meant to make the editing of books,
thesis documents and lecture notes somewhat more convenient.
However, the package can also be used efficiently for
composing a series of documents (such as exercise sheets)
which are typically distributed individually.
It then assists the author in generating the individual documents
(potentially in different versions)
as well as a document containing the collected series.
Another application is in developing style files
or other kinds of included material
where compilation of the style file could redirect
to a sample or test file.

%%%%%%%%%%%%%%%%%%%%%%%%%%%%%%%%%%%%%%%%%%%%%%%%%%%%%%%%%%%%%%%%%%%%%%%%%%%%%%%%
%%%%%%%%%%%%%%%%%%%%%%%%%%%%%%%%%%%%%%%%%%%%%%%%%%%%%%%%%%%%%%%%%%%%%%%%%%%%%%%%
\section{Usage}

First of all, the package \textsf{childdoc} is \emph{not} a standard
\LaTeXe{} |.sty| style file! Therefore it needs to be invoked in
a non-standard way.

%%%%%%%%%%%%%%%%%%%%%%%%%%%%%%%%%%%%%%%%%%%%%%%%%%%%%%%%%%%%%%%%%%%%%%%%%%%%%%%%
\subsection{Included Files}
\label{sec:include}

%%%%%%%%%%%%%%%%%%%%%%%%%%%%%%%%%%%%%%%%
\DescribeMacro{\childdocmain}
To use the package, add the commands
\begin{center}
\begin{tabular}{l}
|\input{childdoc.def}|\\
|\childdocmain{}|\\
\end{tabular}
\end{center}
at the very top of the main \LaTeX{} file,
in particular \emph{before} the |\documentclass| statement!
The argument of |\childdocmain| should be left empty
(but it must be present).

%%%%%%%%%%%%%%%%%%%%%%%%%%%%%%%%%%%%%%%%
\DescribeMacro{\childdocof}
Furthermore, add the commands
\begin{center}
\begin{tabular}{l}
|\input{childdoc.def}|\\
|\childdocof{|\textit{main}|}|\\
\end{tabular}
\end{center}
at the top of every child file \textit{child}
which is included by |\include{|\textit{child}|}|
from within the main file
(or at least for those files to be compiled individually).
The argument \textit{main} must be the filename of the main file.

There are a couple of
considerations in setting up the main and child documents:

%%%%%%%%%%%%%%%%%%%%%%%%%%%%%%%%%%%%%%%%
\paragraph{Restrictions.}

Please note the following restrictions:
\begin{itemize}
\item
|\childdocmain| must be called with one argument \textit{main}
to ensure compatibility with earlier version of the package.
It must either be empty (|\childdocmain{}|)
or precisely match the filename of the main file in which it is specified.
See \secref{sec:detection} for further information.
\item
The filename \textit{main} must be specified without the |.tex| extension.
\item
The filename \textit{main} is case sensitive
(even in case-insensitive file systems)
due to internal string comparison.
\item
The argument \textit{main} should be fully expanded, it cannot be a macro.
\item
Subdirectories and special characters should be avoided in filenames.
\item
The command |\childdocmain{|\textit{main}|}| must be followed by a whitespace.
It should not be followed immediately by another command
or by a comment mark `|%|'.
This is because the \TeX{} parser reads the token immediately following
the argument of |\childdocmain| and puts it
at the beginning of every child section;
however, a white\-space is ignored.
\end{itemize}

%%%%%%%%%%%%%%%%%%%%%%%%%%%%%%%%%%%%%%%%
\paragraph{Content of Main File.}

It is advisable to place all content in the child files included by |\include|.
Any output contained in the main file will appear in all child documents
unless suppressed manually;
it cannot be suppressed automatically by the |\includeonly| directive
and thus should normally be avoided.
A method to include some content in the main file
by means of conditional processing is described in \secref{sec:conditional}.

%%%%%%%%%%%%%%%%%%%%%%%%%%%%%%%%%%%%%%%%
\paragraph{Page Numbering.}

When only a part of the document is compiled,
the appropriate numbering of pages
(as well as other status parameters)
is determined from the |.aux| files.
The latter contain information from previous passes.
However this information needs to propagate through
all intermediate child documents.
Therefore the page numbering in child documents may well
be inconsistent until the complete document is compiled at least once.

A useful (if unconventional) way to always ensure a consistent
page numbering is to restart the numbering in each child document
and denote the pages by `\textit{child}|.|\textit{page}'
where \textit{child} represents the chapter/section number of the child file.
This can be achieved by the command
|\numberwithin{page}{|\textit{child}|}|
of the \textsf{amsmath} package
where \textit{child} can be |chapter| or |section|
depending on the chosen structuring.
Alternatively, one can modify the macro |\thepage| appropriately
and reset the counter |page| at the start of each child file.

%%%%%%%%%%%%%%%%%%%%%%%%%%%%%%%%%%%%%%%%%%%%%%%%%%%%%%%%%%%%%%%%%%%%%%%%%%%%%%%%
\subsection{Conditional Processing}
\label{sec:conditional}

The package provides a mechanism to compile different versions
of a document. To customise the versions further some conditional processing
can come in handy to distinguish which version is being compiled.
The package provides two macros to describe the compilation context:

%%%%%%%%%%%%%%%%%%%%%%%%%%%%%%%%%%%%%%%%
\DescribeMacro{\ifchilddoc}
The conditional |\ifchilddoc| distinguishes between the compilation of
child documents and the main document:
%
\begin{center}
|\ifchilddoc |\textit{child-code}| |[|\||else |\textit{main-code}]| \||fi|
\end{center}

%%%%%%%%%%%%%%%%%%%%%%%%%%%%%%%%%%%%%%%%
\DescribeMacro{\childdocname}
\DescribeMacro{\childdocjob}
The macro |\childdocname| contains the filename (without extension)
of the main or child file being processed.
Note that |\childdocjob| will always contain the name of the main file.

%%%%%%%%%%%%%%%%%%%%%%%%%%%%%%%%%%%%%%%%
\paragraph{Title Page.}

Conditional processing can be used to include a title or banner page
in the main document when proper precautions are taken.
Importantly, the code in the main file should ensure that the page counter
(as well as other status parameters which are stored in the |.aux| files)
takes the same value after the conditional processing.
Otherwise the page numbers may take divergent values
depending on which part is compiled.

For example, a title page could be declared by:
%
\begin{center}
\begin{tabular}{l}
|\ifchilddoc\||else|\\
|\addtocounter{page}{-1}|\\
\textit{code for title page}\\
|\newpage|\\
|\||fi|
\end{tabular}
\end{center}
%
A banner page for the child documents can be generated by:
%
\begin{center}
\begin{tabular}{l}
|\ifchilddoc|\\
|\addtocounter{page}{-1}|\\
\textit{code for banner page}\\
|\newpage|\\
|\||fi|
\end{tabular}
\end{center}
%
Here one could write a message such as:
\begin{center}
|This is the part \childdocname{} of \childdocjob{}.|
\end{center}

%%%%%%%%%%%%%%%%%%%%%%%%%%%%%%%%%%%%%%%%%%%%%%%%%%%%%%%%%%%%%%%%%%%%%%%%%%%%%%%%
\subsection{Flags}
\label{sec:flags}

The package makes it easy to generate different versions
of the main or child documents.
To this end compilation flags can be defined
and assigned different default values.
They will be particularly useful in conjunction
with the forwarding mechanism described in \secref{sec:forward}.

For example, it may be useful to have a flag |\version|
which can be set to |draft| or |final|.
The document source will contain some conditional code
depending on the value of |\version|.
Suppose further, the flag should default to |final| for the main file
and to |draft| for child files
which is a natural assignment for editing the document.
This is achieved by placing the following code
in the preamble of the main document
(below the |\childdocmain| directive):
%
\begin{center}
\begin{tabular}{l}
|\ifchilddoc|\\
|\providecommand{\version}{draft}|\\
|\||else|\\
|\providecommand{\version}{final}|\\
|\||fi|
\end{tabular}
\end{center}
%
The definition by |\providecommand| makes sure
that previous definitions are not overwritten.
Further statements |\providecommand{\version}{...}|
can thus be added before the above code to override it.

For the main file, one might add a line
(between |\childdocmain| and the above block)
%
\begin{center}
|%\ifchilddoc\||else\providecommand{\version}{draft}\||fi|
\end{center}
%
which can be uncommented to produce a draft version.
Likewise one can add a line to the very top of a child file
(above the |\childdocof{|\textit{main}|}| directive)
%
\begin{center}
|%\providecommand{\version}{final}|
\end{center}
%
which can be uncommented to produce the final version of this child document.

%%%%%%%%%%%%%%%%%%%%%%%%%%%%%%%%%%%%%%%%%%%%%%%%%%%%%%%%%%%%%%%%%%%%%%%%%%%%%%%%
\subsection{Forwarding}
\label{sec:forward}

Different versions of the main or child documents
using compilation flags as described in \secref{sec:flags}
can be (permanently) stored in different files
for convenient compilation, viewing and distribution.
To this end, the package defines a command
to pass on compilation to a different file:

%%%%%%%%%%%%%%%%%%%%%%%%%%%%%%%%%%%%%%%%
\DescribeMacro{\childdocforward}
The command |\childdocforward| redirects processing to
another source file:
%
\begin{center}
\begin{tabular}{l}
|\input{childdoc.def}|\\
|\childdocforward[|\textit{main}|]{|\textit{dest}|}|\\
\end{tabular}
\end{center}
%
The argument \textit{dest} is the destination file
(without extension).
It should be the main file or one of the child files.
Note that further \textsf{childdoc} directives
such as |\childdocof| and |\childdocforward|
in the indicated file will be processed in this form.
The optional argument \textit{main}
passes on directly to the main file \textit{main}
while pretending to compile the child \textit{dest}.
This form behaves as if \textit{dest}
issues |\childdocof{|\textit{main}|}| right away,
and no further \textsf{childdoc} directives will be processed.

%%%%%%%%%%%%%%%%%%%%%%%%%%%%%%%%%%%%%%%%
\DescribeMacro{\...prefix}
In the alternative form |\childdocforwardprefix|,
%
\begin{center}
\begin{tabular}{l}
|\input{childdoc.def}|\\
|\childdocforwardprefix[|\textit{main}|]{|\textit{prefix}|}{|\textit{dest}|}|
\end{tabular}
\end{center}
%
the destination file is determined by a pattern
depending on the current file:
To make this work, the current file must be called
`{\textit{prefix}\hspace{0.2em}\textit{suffix}}'
with \textit{prefix} matching precisely the argument.
Processing is then passed on to the file
`{\textit{dest}\hspace{0.2em}\textit{suffix}}'.
Surely, the same effect is achieved by
directly specifying the
argument `{\textit{dest}\hspace{0.2em}\textit{suffix}}'
in the first form.
However, that requires to set up a different file
for each child. With the alternative form of the command
all these files can have exactly the same content
which simplifies setting them up and maintaining them.

For example, the following file |draft.tex|
with a compilation flag |\version| as described in \secref{sec:flags}
compiles the main document as a draft:
%
\begin{center}
\begin{tabular}{l}
|\def\version{draft}|\\
|\input{childdoc.def}|\\
|\childdocforward{|\textit{main}|}|
\end{tabular}
\end{center}
%
Likewise, the following files |final|\textit{nn}|.tex|
compile the final version of the child document
|child|\textit{nn}|.tex|:
%
\begin{center}
\begin{tabular}{l}
|\def\version{final}|\\
|\input{childdoc.def}|\\
|\childdocforwardprefix{final}{child}|
\end{tabular}
\end{center}
%

Note that when several versions of a main file and/or of each child file
are to be generated, it may be convenient to set up a |Makefile| or
shell script to automatise the process.

%%%%%%%%%%%%%%%%%%%%%%%%%%%%%%%%%%%%%%%%%%%%%%%%%%%%%%%%%%%%%%%%%%%%%%%%%%%%%%%%
\subsection{Command Line Processing}
\label{sec:commandline}

The effect of redirection files can also be achieved by invoking
the \LaTeX{} compiler with a more elaborate command line.
Most conveniently this should be done as part
of a shell script or a |Makefile|.

When using \textsf{childdoc} in the main file, the following
command lines effectively perform a redirection
(note that depending on the shell being used,
backslashes may have to be doubled: `|\|' $\to$ `|\\|'):
%
\begin{center}
|... -jobname "|\textit{target}|" |\\|"|[\textit{flags}]%
|\input{childdoc.def}\childdocforward[|\textit{main}|]{|\textit{dest}|}"|
\end{center}
%
Here \textit{target} is the name of the output file,
\textit{main} is the name of the main file
and \textit{dest} is the name of the main or child file to be processed
(all filenames without extensions).
The optional argument \textit{main} can be omitted
if \textit{main} matches \textit{dest}.
Optionally, compilation \textit{flags} can be defined via |\def| commands.
This command line makes the \TeX{} engine believe
it is compiling the file \textit{target}
whose content is specified as the latter parameter.
The provided code then forwards the processing to
\textit{main} or \textit{dest} as described in \secref{sec:forward}.

%%%%%%%%%%%%%%%%%%%%%%%%%%%%%%%%%%%%%%%%%%%%%%%%%%%%%%%%%%%%%%%%%%%%%%%%%%%%%%%%
\subsection{Include by Input}
\label{sec:input}

Including child documents by |\include| has some restrictions by design.
Most notably, the content of a child document always occupies
its own set of pages; pages cannot be shared between child documents.
Usually, this behaviour makes perfect sense
because each child document contain an essential part of the document.
However, in some situations it may be desirable to compose
a document from a collection of parts
without having mandatory page breaks between then.
For this case, the package
provides a mechanism to include parts
by |\input| which can also be processed individually.
However, by construction this mechanism
requires manual handling of the content to be output.

%%%%%%%%%%%%%%%%%%%%%%%%%%%%%%%%%%%%%%%%
\DescribeMacro{\ifchilddocmanual}
The main file should be prepared as usual, see \secref{sec:include}.
However, the document body must make a distinction
between processing of an individual part and of the main document, e.g.:
%
\begin{center}
\begin{tabular}{l}
|\ifchilddocmanual|\\
|\input{\childdocname}|\\
|\||else|\\
\textit{document body with }|\input{|\textit{part}|}|\\
|\||fi|
\end{tabular}
\end{center}
%
The conditional |\ifchilddocmanual| is true whenever
a part to be included by |\input| is being compiled,
and the name of the part is stored in |\childdocname|.

%%%%%%%%%%%%%%%%%%%%%%%%%%%%%%%%%%%%%%%%
\DescribeMacro{\childdocby}
Each part to be included by |\input| should start with:
%
\begin{center}
\begin{tabular}{l}
|\input{childdoc.def}|\\
|\childdocby{|\textit{main}|}|\\
\end{tabular}
\end{center}
%
The directive |\childdocby| is similar to |\childdocof|
described in \secref{sec:include},
but the subsequent selection of content must be done manually.
To that end, both |\ifchilddoc| and |\ifchilddocmanual|
will be true upon processing of a part,
and the name of the part is stored in |\childdocname|.
Note that |\jobname| will be set to the filename of the current part
so that each part receives an individual |.aux| file
that does not interfere with the |.aux| file(s) of the main document.
This behaviour can be altered by the alternative form
|\childdocby[*]{|\textit{main}|}| (with a non-empty optional argument)
which uses the |.aux| file of the main document
by setting |\jobname| to \textit{main}.

%%%%%%%%%%%%%%%%%%%%%%%%%%%%%%%%%%%%%%%%%%%%%%%%%%%%%%%%%%%%%%%%%%%%%%%%%%%%%%%%
\subsection{Driver Development}
\label{sec:driver}

The \textsf{childdoc} mechanism can also be use for the development
of definition files such as \LaTeX{} styles or classes.
This case differs from the above setup with multiple parts
included by |\include| in that no |\includeonly| should be invoked.
This can be achieved by starting the include file
(before |\ProvidesPackage|) with:
%
\begin{center}
\begin{tabular}{l}
|\input{childdoc.def}|\\
|\childdocforward{|\textit{main}|}|\\
\end{tabular}
\end{center}
%
or alternatively with:
%
\begin{center}
\begin{tabular}{l}
|\input{childdoc.def}|\\
|\childdocby{|\textit{main}|}|\\
\end{tabular}
\end{center}
%
Both forms have slightly different effects as described above.
The main file is prepared as usual, see \secref{sec:include}.

%%%%%%%%%%%%%%%%%%%%%%%%%%%%%%%%%%%%%%%%%%%%%%%%%%%%%%%%%%%%%%%%%%%%%%%%%%%%%%%%
\subsection{Legacy Detection}
\label{sec:detection}

The directive |\childdocmain| in the main file can detect
whether the complete document or merely a child is to be compiled
even without using the directive |\childdocof|.
This method is deprecated because it is less robust
and there is no compelling reason to use it;
it is merely provided for backward compatibility
and it may be removed in future versions.

If the detection mechanism is to be used,
it is mandatory to correctly specify
the filename of the main file as the argument of |\childdocmain|:
%
\begin{center}
\begin{tabular}{l}
|\input{childdoc.def}|\\
|\childdocmain{|\textit{main}|}|\\
\end{tabular}
\end{center}
%
If |\jobname| does not match the argument \textit{main} of |\childdocmain|,
it is assumed that |\jobname| points to the child file to be compiled.
When using |\childdocmain| with the main file specified as argument,
it suffices to start a child file
with just |\input{|\textit{main}|}|
without loading of the package and using |\childdocof|.
If instead all processing is done
with the appropriate \textsf{childdoc} directives,
the argument of \textit{main} of |\childdocmain| can be empty.

An alternative version of the command line processing described
in \secref{sec:commandline} using the detection mechanism reads:
%
\begin{center}
|... -jobname "|\textit{target}|" "|[\textit{flags}]%
[|\def\jobname{|\textit{dest}|}|]|\input{|\textit{main}|}"|
\end{center}

%%%%%%%%%%%%%%%%%%%%%%%%%%%%%%%%%%%%%%%%%%%%%%%%%%%%%%%%%%%%%%%%%%%%%%%%%%%%%%%%
\subsection{Manual Code}
\label{sec:manual}

In case one cannot be certain whether the definitions file |childdoc.def|
is installed on the target \TeX{} distribution
and one prefers not to ship it,
it is conceivable to paste a few relevant commands into the sources.

To that end, drop all statements |\input{childdoc.def}|
and perform the replacements as outlined below.
Instead of |\childdocmain{|\textit{main}|}| add the following code
to the top of the main file:
%
\begin{center}
\begin{tabular}{l}
|\||ifdefined\childdocname\endinput\||fi\newif\ifchilddoc|\\
|\edef\childdocname{\scantokens\expandafter{\jobname\noexpand}}|\\
|\def\childdocmain{|\textit{main}|}\||ifx\childdocmain\childdocname\||else|\\
|\childdoctrue\includeonly{\childdocname}\let\jobname\childdocmain\||fi|\\
\end{tabular}
\end{center}
%
Instead of |\childdocof{|\textit{main}|}| just include the main file
at the top of each child file:
%
\begin{center}
|\input{|\textit{main}|}|
\end{center}
%
A simple redirection |\childdocforward{|\textit{dest}|}| is achieved by:
%
\begin{center}
|\def\jobname{|\textit{dest}|}\input{\jobname}|
\end{center}
%
The redirection with prefix
|\childdocforwardprefix[|\textit{prefix}|]{|\textit{dest}|}|
is accomplished by:
%
\begin{center}
\begin{tabular}{l}
|{\edef\jobname{\scantokens\expandafter{\jobname\noexpand}}|\\
|\def\redirectjob |\textit{prefix}|#1~~~{\gdef\jobname{|\textit{dest}|#1}}|\\
|\expandafter\redirectjob\jobname~~~}\input{\jobname}|
\end{tabular}
\end{center}

In an alternative approach,
child documents can be compiled by a specific command line
without additional code or specific definitions:
%
\begin{center}
|... -jobname "|\textit{target}|" "|[\textit{flags}]%
|\includeonly{|\textit{dest}|}\input{|\textit{main}|}"|
\end{center}
%

%%%%%%%%%%%%%%%%%%%%%%%%%%%%%%%%%%%%%%%%%%%%%%%%%%%%%%%%%%%%%%%%%%%%%%%%%%%%%%%%
%%%%%%%%%%%%%%%%%%%%%%%%%%%%%%%%%%%%%%%%%%%%%%%%%%%%%%%%%%%%%%%%%%%%%%%%%%%%%%%%
\section{Information}

%%%%%%%%%%%%%%%%%%%%%%%%%%%%%%%%%%%%%%%%%%%%%%%%%%%%%%%%%%%%%%%%%%%%%%%%%%%%%%%%
\subsection{Copyright}

Copyright \copyright{} 2017--2018 Niklas Beisert

This work may be distributed and/or modified under the
conditions of the \LaTeX{} Project Public License, either version 1.3
of this license or (at your option) any later version.
The latest version of this license is in
  \url{http://www.latex-project.org/lppl.txt}
and version 1.3 or later is part of all distributions of \LaTeX{}
version 2005/12/01 or later.

This work has the LPPL maintenance status `maintained'.

The Current Maintainer of this work is Niklas Beisert.

This work consists of the files |README.txt|, |childdoc.ins| and |childdoc.dtx|
as well as the derived files |childdoc.def|, |cdocsamp.tex|
with |cdocsch1.tex|, |cdocsch2.tex|, |cdocspt3.tex|, |cdocspt4.tex|,
|cdocsdrf.tex|, |cdocsfn1.tex|, |cdocsfn2.tex|
as well as |childdoc.pdf|.

%%%%%%%%%%%%%%%%%%%%%%%%%%%%%%%%%%%%%%%%%%%%%%%%%%%%%%%%%%%%%%%%%%%%%%%%%%%%%%%%
\subsection{Files and Installation}

The package consists of the files:
%
\begin{center}
\begin{tabular}{ll}
    |README.txt|   & readme file \\
    |childdoc.ins| & installation file \\
    |childdoc.dtx| & source file \\
    |childdoc.def| & definition file \\
    |cdocsamp.tex| & sample main file \\
    |cdocsch1.tex| & sample include file \\
    |cdocsch2.tex| & sample include file \\
    |cdocspt3.tex| & sample part file \\
    |cdocspt4.tex| & sample part file \\
    |cdocsdrf.tex| & sample redirection file \\
    |cdocsfn1.tex| & sample redirection file \\
    |cdocsfn2.tex| & sample redirection file \\
    |childdoc.pdf| & manual
\end{tabular}
\end{center}
%
The distribution consists of the files
|README.txt|, |childdoc.ins| and |childdoc.dtx|.
%
\begin{itemize}
\item
Run (pdf)\LaTeX{} on |childdoc.dtx|
to compile the manual |childdoc.pdf| (this file).
\item
Run \LaTeX{} on |childdoc.ins| to create the definitions file |childdoc.def|
and the sample |cdocsamp.tex| with include files
|cdocsch1.tex|, |cdocsch2.tex|, |cdocspt3.tex|, |cdocspt4.tex|,
|cdocsdrf.tex|, |cdocsfn1.tex|, |cdocsfn2.tex|.
Then copy the file |childdoc.def| to an appropriate directory of your \LaTeX{}
distribution, e.g.\ \textit{texmf-root}|/tex/latex/childdoc|.
\end{itemize}

%%%%%%%%%%%%%%%%%%%%%%%%%%%%%%%%%%%%%%%%%%%%%%%%%%%%%%%%%%%%%%%%%%%%%%%%%%%%%%%%
\subsection{Related CTAN Packages}

There are several other packages which offer a similar functionality:
%
\begin{itemize}
\item
The packages
\href{http://ctan.org/pkg/docmute}{\textsf{docmute}},
\href{http://ctan.org/pkg/includex}{\textsf{includex}} and
\href{http://ctan.org/pkg/standalone}{\textsf{standalone}}
provide commands to include only the document body of
a child file thus allowing both files to be compiled individually.
\item
The packages \href{http://ctan.org/pkg/subdocs}{\textsf{subdocs}}
and \href{http://ctan.org/pkg/subfiles}{\textsf{subfiles}}
provide structures in which the main and child documents can be
encapsulated and allowing them to be compiled individually.
The inclusion mechanism is different from the conventional |\include|.
\item
The package \href{http://ctan.org/pkg/combine}{\textsf{combine}}
is an elaborate solution to combine several documents into one.
\end{itemize}
%
See also the CTAN topic \href{http://ctan.org/topic/subdocs}{\textsf{subdocs}}
for further related packages.
The present package differs from the above solutions in that
a document structure constructed with the conventional |\include| mechanism
just needs two extra commands at the top of every file
such that all constituent files can be compiled individually.

%%%%%%%%%%%%%%%%%%%%%%%%%%%%%%%%%%%%%%%%%%%%%%%%%%%%%%%%%%%%%%%%%%%%%%%%%%%%%%%%
%\subsection{Feature Suggestions}
%
%The following is a list of features which may be useful for future
%versions of this package:
%%
%\begin{itemize}
%\item
%\ldots
%\end{itemize}

%%%%%%%%%%%%%%%%%%%%%%%%%%%%%%%%%%%%%%%%%%%%%%%%%%%%%%%%%%%%%%%%%%%%%%%%%%%%%%%%
\subsection{Revision History}

%%%%%%%%%%%%%%%%%%%%%%%%%%%%%%%%%%%%%%%%
\paragraph{v2.0:} 2018/12/30

\begin{itemize}
\item
immediate forward processing
\item
added |\childdocby| mechanism
\item
manual restructured
\end{itemize}

%%%%%%%%%%%%%%%%%%%%%%%%%%%%%%%%%%%%%%%%
\paragraph{v1.6:} 2018/01/17

\begin{itemize}
\item
application for development of include files
\item
corrections to manual
\end{itemize}

%%%%%%%%%%%%%%%%%%%%%%%%%%%%%%%%%%%%%%%%
\paragraph{v1.5:} 2017/05/21

\begin{itemize}
\item
more complete structuring introduced
\item
|\childdocof| introduced
\item
|\childdoc| renamed to |\childdocmain|
\item
|\childredirect| renamed to |\childdocforward| and |\childdocforwardprefix|
and functionality expanded
\end{itemize}

%%%%%%%%%%%%%%%%%%%%%%%%%%%%%%%%%%%%%%%%
\paragraph{v1.0:} 2017/04/27

\begin{itemize}
\item
manual and install package
\item
first version published on CTAN
\end{itemize}

%%%%%%%%%%%%%%%%%%%%%%%%%%%%%%%%%%%%%%%%
\paragraph{v0.6:} 2017/04/26

\begin{itemize}
\item
redirection mechanism added
\end{itemize}

%%%%%%%%%%%%%%%%%%%%%%%%%%%%%%%%%%%%%%%%
\paragraph{v0.5:} 2017/04/26

\begin{itemize}
\item
functionality in definition file
\end{itemize}


%%%%%%%%%%%%%%%%%%%%%%%%%%%%%%%%%%%%%%%%%%%%%%%%%%%%%%%%%%%%%%%%%%%%%%%%%%%%%%%%
%%%%%%%%%%%%%%%%%%%%%%%%%%%%%%%%%%%%%%%%%%%%%%%%%%%%%%%%%%%%%%%%%%%%%%%%%%%%%%%%
%%%%%%%%%%%%%%%%%%%%%%%%%%%%%%%%%%%%%%%%%%%%%%%%%%%%%%%%%%%%%%%%%%%%%%%%%%%%%%%%
\appendix

\settowidth\MacroIndent{\rmfamily\scriptsize 000\ }

 \DocInput{childdoc.dtx}

\end{document}
%</driver>
% \fi
%
% %%%%%%%%%%%%%%%%%%%%%%%%%%%%%%%%%%%%%%%%%%%%%%%%%%%%%%%%%%%%%%%%%%%%%%%%%%%%%%
% %%%%%%%%%%%%%%%%%%%%%%%%%%%%%%%%%%%%%%%%%%%%%%%%%%%%%%%%%%%%%%%%%%%%%%%%%%%%%%
% \section{Sample}
%\iffalse
%<*samplemain>
%\fi
%
% The following presents a sample document
% with two chapters, two parts, a title page,
% a compile flag as well as three forwarding files to set the flag.
% It consists of eight |.tex| files:
% \begin{center}
% \begin{tabular}{ll}
% |cdocsamp.tex|&main file\\
% |cdocsch1.tex|&include file for chapter 1\\
% |cdocsch2.tex|&include file for chapter 2\\
% |cdocspt3.tex|&include file for part 3\\
% |cdocspt4.tex|&include file for part 4\\
% |cdocsdrf.tex|&forwarding file for main file in draft mode\\
% |cdocsfi1.tex|&forwarding file for final version of chapter 1\\
% |cdocsfi2.tex|&forwarding file for final version of chapter 2\\
% \end{tabular}
% \end{center}
% Each of the eight files can be compiled directly by the \LaTeX{} compiler.
%
% %%%%%%%%%%%%%%%%%%%%%%%%%%%%%%%%%%%%%%
% \paragraph{Main File.}
%
% The main file is called |cdocsamp.tex|.
%
% Load the \textsf{childdoc} definitions and
% declare the filename for the main document:
%    \begin{macrocode}
\input{childdoc.def}
\childdocmain{}
%    \end{macrocode}

% Optional override for |\version| flag:
%    \begin{macrocode}
%%\ifchilddoc\else\providecommand{\version}{draft}\fi
%    \end{macrocode}

% Define the default values for the |\version| flag
% (|final| for the main file and |draft| for childs):
%    \begin{macrocode}
\ifchilddoc
\providecommand{\version}{draft}
\else
\providecommand{\version}{final}
\fi
%    \end{macrocode}

% Load the standard document class:
%    \begin{macrocode}
\documentclass[12pt]{article}
%    \end{macrocode}

% Start the document body:
%    \begin{macrocode}
\begin{document}
%    \end{macrocode}

% Declare a title page.
% Print title, part of document being processed and version flag:
%    \begin{macrocode}
\addtocounter{page}{-1}
\begin{center}
{\LARGE\bfseries{}childdoc example\par}
\vspace{1cm}
\ifchilddoc
\ifchilddocmanual part\else chapter\fi:
`\childdocname' of `\childdocjob'\par
\else
main document: `\childdocjob'\par
\fi
version: \version\par
\end{center}
\newpage
%    \end{macrocode}

% Manually include selected file,
% otherwise process as usual:
%    \begin{macrocode}
\ifchilddocmanual
\section*{part `\childdocname'}
\input{\childdocname}
\else
%    \end{macrocode}

% Include the two chapters:
%    \begin{macrocode}
\include{cdocsch1}
\include{cdocsch2}
%    \end{macrocode}

% Include the two parts unless only chapters should be displayed:
%    \begin{macrocode}
\ifchilddoc\else
\section{part three}
\input{cdocspt3}
\section{part four}
\input{cdocspt4}
\fi
%    \end{macrocode}

% Process as usual until here:
%    \begin{macrocode}
\fi
%    \end{macrocode}

% End of document body:
%    \begin{macrocode}
\end{document}
%    \end{macrocode}
%\iffalse
%</samplemain>
%\fi
%
% %%%%%%%%%%%%%%%%%%%%%%%%%%%%%%%%%%%%%%
% \paragraph{Chapter Include Files.}
%
% The include files are called |cdocsch1.tex| and |cdocsch2.tex|.
%
%\iffalse
%<*samplechap1|samplechap2>
%\fi

% Optional override for |\version| flag:
%    \begin{macrocode}
%%\providecommand{\version}{final}
%    \end{macrocode}

% Include the main document:
%    \begin{macrocode}
\input{childdoc.def}
\childdocof{cdocsamp}
%    \end{macrocode}

%\iffalse
%</samplechap1|samplechap2>
%\fi
%
%\iffalse
%<*samplechap1>
%\fi
% Some text for chapter 1:
%    \begin{macrocode}
\section{one}
some text in chapter one
%    \end{macrocode}

%\iffalse
%</samplechap1>
%\fi
% Some text for chapter 2:
%\iffalse
%<*samplechap2>
%\fi
%    \begin{macrocode}
\section{two}
more text in chapter two
%    \end{macrocode}

%\iffalse
%</samplechap2>
%\fi
%
% %%%%%%%%%%%%%%%%%%%%%%%%%%%%%%%%%%%%%%
% \paragraph{Part Include Files.}
%
% The include files are called |cdocspt3.tex| and |cdocspt4.tex|.
%
%\iffalse
%<*samplepart3|samplepart4>
%\fi

% Optional override for |\version| flag:
%    \begin{macrocode}
%%\providecommand{\version}{final}
%    \end{macrocode}

% Include the main document:
%    \begin{macrocode}
\input{childdoc.def}
\childdocby{cdocsamp}
%    \end{macrocode}

%\iffalse
%</samplepart3|samplepart4>
%\fi
%
%\iffalse
%<*samplepart3>
%\fi
% Some text for part 3:
%    \begin{macrocode}
some text in part three
%    \end{macrocode}

%\iffalse
%</samplepart3>
%\fi
% Some text for part 4:
%\iffalse
%<*samplepart4>
%\fi
%    \begin{macrocode}
more text in part four
%    \end{macrocode}

%\iffalse
%</samplepart4>
%\fi
%
% %%%%%%%%%%%%%%%%%%%%%%%%%%%%%%%%%%%%%%
% \paragraph{Forwarding for a Complete Draft.}
%
% The following forwarding file |cdocsdrf.tex|
% compiles the main document in draft mode:
%\iffalse
%<*sampledraft>
%\fi
%    \begin{macrocode}
\def\version{draft}
\input{childdoc.def}
\childdocforward{cdocsamp}
%    \end{macrocode}

%\iffalse
%</sampledraft>
%\fi
%
% %%%%%%%%%%%%%%%%%%%%%%%%%%%%%%%%%%%%%%
% \paragraph{Forwarding for Final Version of the Chapters.}
%
% The following forwarding files |cdocsfn1.tex| and |cdocsfn2.tex|
% (with identical content)
% compile the final versions of the child documents
% |cdocsch1.tex| and |cdocsch2.tex|, respectively:
%\iffalse
%<*samplefinal>
%\fi
%    \begin{macrocode}
\def\version{final}
\input{childdoc.def}
\childdocforwardprefix[cdocsamp]{cdocsfn}{cdocsch}
%    \end{macrocode}

%\iffalse
%</samplefinal>
%\fi
%
% %%%%%%%%%%%%%%%%%%%%%%%%%%%%%%%%%%%%%%
% \paragraph{Command Line Processing.}
%
% The following three command lines generate the output files
% |cdocscld|, |cdocscl1| and |cdocscl2|
% which should be identical to
% |cdocsdrf|, |cdocsch1| and |cdocsfn2|, respectively:
% \begin{center}
% \begin{tabular}{l}
% |latex -jobname cdocscld \|\\
% |  "\def\version{draft}\input{childdoc.def}\childdocforward{cdocsamp}"|\\
% |latex -jobname cdocscl1 \|\\
% |  "\input{childdoc.def}\childdocforward[cdocsamp]{cdocsch1}"|\\
% |latex -jobname cdocscl2 \|\\
% |  "\def\version{final}\input{childdoc.def}\childdocforward{cdocsch2}"|
% \end{tabular}
% \end{center}
% Note that the trailing backslash on each first line
% merely continues the input to the second line
% (for convenient cut ant paste).
% Furthermore, the command |latex| can be replaced by any
% of its alternative versions such as |pdflatex|.
%
% %%%%%%%%%%%%%%%%%%%%%%%%%%%%%%%%%%%%%%%%%%%%%%%%%%%%%%%%%%%%%%%%%%%%%%%%%%%%%%
% %%%%%%%%%%%%%%%%%%%%%%%%%%%%%%%%%%%%%%%%%%%%%%%%%%%%%%%%%%%%%%%%%%%%%%%%%%%%%%
% \section{Implementation}
%\iffalse
%<*package>
%\fi
%
% This section describes the definitions file |childdoc.def|.

% The definitions cannot be loaded using |\usepackage| or |\RequirePackage|
% which has a mechanism to prevent loading a style file more than once.
% When loading the definitions by means of |\input|
% multiple instances have to be prevented manually:
%\iffalse
%This code needs to be before the `\ProvidesFile' directive
%which is defined at the beginning of this file.
%Therefore it is also placed there and commented out here.
%</package>
%<*discard>
%\fi
%    \begin{macrocode}
\ifdefined\childdocmain\endinput\fi
%    \end{macrocode}
%\iffalse
%</discard>
%<*package>
%\fi
%
% \macro{\ifchilddoc}
% \macro{\ifchilddocmanual}
% The conditional |\ifchilddoc| tells whether a
% child (true) or main (false) document is being compiled.
% The conditional |\ifchilddocmanual| tells whether
% the |\includeonly| mechanism is used (false) or
% the selection of child files must be performed manually (true).
% The definitions initialise to false:
%    \begin{macrocode}
\newif\ifchilddoc
\newif\ifchilddocmanual
%    \end{macrocode}

% \macro{\childdocname}
% \macro{\childdocjob}
% The macro |\childdocname| stores the name of the main document
% to be compiled. The macro |\childdocjob| stores the name of
% the document on which the \LaTeX{} compiler was originally invoked.
% The content of |\jobname| cannot be compared
% to filenames specified in the source due to different catcodes.
% The following code rescans |\jobname|, stores the result
% in |\childdocname| and saves a copy in |\childdocjob|:
%    \begin{macrocode}
\edef\childdocname{\scantokens\expandafter{\jobname\noexpand}}
\let\childdocjob\childdocname
%    \end{macrocode}

% \macro{\childdocdisable}
% The macro |\childdocdisable| prevents the main file
% from being processed more than once.
% At this stage, the main document command |\childdocmain|
% is assumed to be called once again where it should do nothing.
% Any subsequent call to it should prevent
% a secondary processing of the main document
% It overwrites the forwarding commands
% |\childdocof| and |\childdocforward|
% with empty macros to prevent further inclusions of the main document:
%    \begin{macrocode}
\newcommand{\childdocdisable}
{
  \renewcommand{\childdocmain}[1]{\renewcommand{\childdocmain}[1]{\endinput}}
  \renewcommand{\childdocof}[1]{}
  \renewcommand{\childdocby}[2][]{}
  \renewcommand{\childdocforward}[2][]{}
  \renewcommand{\childdocdisable}{}
}
%    \end{macrocode}

% \macro{\childdocmain}
% The macro |\childdocmain| is to be called at the top of the main file
% with nothing or the main filename (without extension) as argument.
% First, it breaks loops.
% If the argument is not empty and does not match |\childdocname|
% (which is set by the first inclusion of |childdoc.def|),
% |\ifchilddoc| is set to true, |\includeonly| is applied to the child file
% and |\jobname| is set to the main file
% (for proper handling of |.aux| files):
%    \begin{macrocode}
\newcommand{\childdocmain}[1]
{
  \childdocdisable\childdocmain{}
  \if?#1?\else
    \begingroup
      \def\childdoctmp{#1}
      \ifx\childdoctmp\childdocname
        \def\childdoctmp{}
      \else
        \def\childdoctmp
        {
          \childdoctrue
          \includeonly{\childdocname}
          \def\childdocjob{#1}
          \def\jobname{#1}
        }
      \fi
      \expandafter
    \endgroup
    \childdoctmp
  \fi
}
%    \end{macrocode}

% \macro{\childdocof}
% The command |\childdocof| redirects
% compilation to the main file |#1|.
%    \begin{macrocode}
\newcommand{\childdocof}[1]
{
  \childdocdisable
  \childdoctrue
  \includeonly{\childdocname}
  \def\jobname{#1}
  \def\childdocjob{#1}
  \input{#1}
}
%    \end{macrocode}

% \macro{\childdocby}
% The command |\childdocby| ....
%    \begin{macrocode}
\newcommand{\childdocby}[2][]
{
  \childdocdisable
  \childdoctrue
  \childdocmanualtrue
  \if?#1?\else
    \def\jobname{#2}
  \fi
  \def\childdocjob{#2}
  \input{#2}
  \endinput
}
%    \end{macrocode}

% \macro{\childdocforward}
% The command |\childdocforward| redirects
% compilation to the main file or
% (if the optional argument is given) a child file.
% Parameters are set as if the main file
% or a child file starting with |\childdocof| was compiled.
% Then compilation is handed over to the main file:
%    \begin{macrocode}
\newcommand{\childdocforward}[2][]
{
  \begingroup
    \if?#1?
      \def\childdoctmp
      {
        \def\childdocname{#2}
        \def\childdocjob{#2}
        \def\jobname{#2}
        \input{#2}
        \endinput
      }
    \else
      \def\childdoctmp
      {
        \childdocdisable
        \def\childdocname{#2}
        \childdoctrue
        \includeonly{#2}
        \def\childdocjob{#1}
        \def\jobname{#1}
        \input{#1}
        \endinput
      }
    \fi
    \expandafter
  \endgroup
  \childdoctmp
}
%    \end{macrocode}

% \macro{\childdocforwardprefix}
% The command |\childdocforwardprefix| redirects
% compilation to the main or a child file by means of a pattern.
% The prefix |#1| in the current filename is replaced by |#2|
% and the suffix of the current filename is kept
% (it is assumed that the filename does not contain the substring `|~~~|'
% which is used as a delimiter).
% Compilation is handed over to the new file by |\childdocforward|:
%    \begin{macrocode}
\newcommand{\childdocforwardprefix}[3][]
{
  \begingroup
    \def\childdocextract #2##1~~~{\def\childdoctmp{\childdocforward[#1]{#3##1}}}
    \expandafter\childdocextract\childdocname~~~
    \expandafter
  \endgroup
  \childdoctmp
}
%    \end{macrocode}

% \macro{\childdoc}
% The deprecated macro |\childdoc| is a legacy version of |\childdocmain|:
%    \begin{macrocode}
\newcommand{\childdoc}{\childdocmain}
%    \end{macrocode}

% \macro{\childdocredirect}
% The deprecated macro |\childdocredirect| is a legacy version
% of |\childdocforward| and |\childdocforwardprefix|:
%    \begin{macrocode}
\newcommand{\childdocredirect}[2][]
{
  \begingroup
    \if?#1?
      \def\childdoctmp{\childdocforward{#2}}
    \else
      \def\childdoctmp{\childdocforwardprefix{#1}{#2}}
    \fi
    \expandafter
  \endgroup
  \childdoctmp
}
%    \end{macrocode}

%\iffalse
%</package>
%\fi
%
\endinput
|\\
|\childdocforward{|\textit{main}|}|
\end{tabular}
\end{center}
%
Likewise, the following files |final|\textit{nn}|.tex|
compile the final version of the child document
|child|\textit{nn}|.tex|:
%
\begin{center}
\begin{tabular}{l}
|\def\version{final}|\\
|% \iffalse
%
% childdoc.dtx Copyright (C) 2017-2018 Niklas Beisert
%
% This work may be distributed and/or modified under the
% conditions of the LaTeX Project Public License, either version 1.3
% of this license or (at your option) any later version.
% The latest version of this license is in
%   http://www.latex-project.org/lppl.txt
% and version 1.3 or later is part of all distributions of LaTeX
% version 2005/12/01 or later.
%
% This work has the LPPL maintenance status `maintained'.
%
% The Current Maintainer of this work is Niklas Beisert.
%
% This work consists of the files childdoc.dtx and childdoc.ins
% and the derived files childdoc.def and cdocsamp.tex with
% cdocsch1.tex, cdocsch2.tex, cdocsdrf.tex, cdocsfn1.tex, cdocsfn2.tex.
%
%<package>\ifdefined\childdocmain\endinput\fi
%<package>\ProvidesFile{childdoc.def}[2018/12/30 v2.0 child document driver]
%<samplemain>\ProvidesFile{cdocsamp.tex}[2018/12/30 v2.0 sample for childdoc]
%<*driver>
%\ProvidesFile{childdoc.drv}[2018/12/30 v2.0 childdoc reference manual file]
\PassOptionsToClass{10pt,a4paper}{article}
\documentclass{ltxdoc}

\usepackage[margin=35mm]{geometry}
\usepackage{hyperref}
\usepackage{hyperxmp}
\usepackage[usenames]{color}

\hypersetup{colorlinks=true}
\hypersetup{pdfstartview=FitH}
\hypersetup{pdfpagemode=UseNone}
\hypersetup{pdfsource={}}
\hypersetup{pdflang={en-UK}}
\hypersetup{pdfcopyright={Copyright 2017-2018 Niklas Beisert.
  This work may be distributed and/or modified under the
  conditions of the LaTeX Project Public License, either version 1.3
  of this license or (at your option) any later version.}}
\hypersetup{pdflicenseurl={http://www.latex-project.org/lppl.txt}}
\hypersetup{pdfcontactaddress={ETH Zurich, ITP, HIT K,
  Wolfgang-Pauli-Strasse 27}}
\hypersetup{pdfcontactpostcode={8093}}
\hypersetup{pdfcontactcity={Zurich}}
\hypersetup{pdfcontactcountry={Switzerland}}
\hypersetup{pdfcontactemail={nbeisert@itp.phys.ethz.ch}}
\hypersetup{pdfcontacturl={http://people.phys.ethz.ch/\xmptilde nbeisert/}}

\newcommand{\secref}[1]{\hyperref[#1]{section \ref*{#1}}}

\parskip1ex
\parindent0pt
\let\olditemize\itemize
\def\itemize{\olditemize\parskip0pt}

\begin{document}

\title{The \textsf{childdoc} Package}
\hypersetup{pdftitle={The childdoc Package}}
\author{Niklas Beisert\\[2ex]
  Institut f\"ur Theoretische Physik\\
  Eidgen\"ossische Technische Hochschule Z\"urich\\
  Wolfgang-Pauli-Strasse 27, 8093 Z\"urich, Switzerland\\[1ex]
  \href{mailto:nbeisert@itp.phys.ethz.ch}
  {\texttt{nbeisert@itp.phys.ethz.ch}}}
\hypersetup{pdfauthor={Niklas Beisert}}
\hypersetup{pdfsubject={Manual for the LaTeX2e Package childdoc}}
\date{30 December 2018, \textsf{v2.0}}
\maketitle

\begin{abstract}\noindent
\textsf{childdoc} is a \LaTeXe{} package
that enables the direct compilation
of document sections included by |\include|
to individual files.
\end{abstract}

\begingroup
\parskip0ex
\tableofcontents
\endgroup

%%%%%%%%%%%%%%%%%%%%%%%%%%%%%%%%%%%%%%%%%%%%%%%%%%%%%%%%%%%%%%%%%%%%%%%%%%%%%%%%
%%%%%%%%%%%%%%%%%%%%%%%%%%%%%%%%%%%%%%%%%%%%%%%%%%%%%%%%%%%%%%%%%%%%%%%%%%%%%%%%
\section{Introduction}

\LaTeX{} provides a mechanism to structure a large document (such as a book)
into a main file and several child files (containing the chapters)
using the |\include| command.
This mechanism is beneficial for documents
which span hundreds of pages in order to
make the source file(s) more manageable.
Moreover, compilation can be restricted to
selected child files by means of the |\includeonly| command.
The latter feature can be used to reduce the compilation time while editing
(this was significantly more useful in the earlier days of \LaTeX{})
or to generate a smaller document which is easier to navigate.
Another application of |\includeonly| is to generate
documents consisting of selected parts of the complete document.

However, there are a few drawbacks of the plain |\include| mechanism:
\begin{itemize}
\item
The child files cannot be compiled on their own,
they can only be compiled via the main file.
A naive editing environment
(such as a text editor with an option
to have the current file processed by \LaTeX)
may require one to switch to the main file before compiling;
attempting to compile the child file produces errors.
\item
The main file must be modified (each time)
to adjust the |\includeonly| command
to the present needs. This easily leaves the main file in a messy state.
\item
The generated document will always carry the filename
of the main document. This is inconvenient if
several child files are to be compiled and
to be kept for distribution.
\end{itemize}

The present package provides a simple interface
to make child files individually compilable by \LaTeX{}.
Compiling a child file then has the same effect as compiling
the main file with an |\includeonly| command
to select the appropriate child.
Moreover the generated document will carry the name of the child
rather than the main file.
This resolves all three above issues.

This feature is meant to make the editing of books,
thesis documents and lecture notes somewhat more convenient.
However, the package can also be used efficiently for
composing a series of documents (such as exercise sheets)
which are typically distributed individually.
It then assists the author in generating the individual documents
(potentially in different versions)
as well as a document containing the collected series.
Another application is in developing style files
or other kinds of included material
where compilation of the style file could redirect
to a sample or test file.

%%%%%%%%%%%%%%%%%%%%%%%%%%%%%%%%%%%%%%%%%%%%%%%%%%%%%%%%%%%%%%%%%%%%%%%%%%%%%%%%
%%%%%%%%%%%%%%%%%%%%%%%%%%%%%%%%%%%%%%%%%%%%%%%%%%%%%%%%%%%%%%%%%%%%%%%%%%%%%%%%
\section{Usage}

First of all, the package \textsf{childdoc} is \emph{not} a standard
\LaTeXe{} |.sty| style file! Therefore it needs to be invoked in
a non-standard way.

%%%%%%%%%%%%%%%%%%%%%%%%%%%%%%%%%%%%%%%%%%%%%%%%%%%%%%%%%%%%%%%%%%%%%%%%%%%%%%%%
\subsection{Included Files}
\label{sec:include}

%%%%%%%%%%%%%%%%%%%%%%%%%%%%%%%%%%%%%%%%
\DescribeMacro{\childdocmain}
To use the package, add the commands
\begin{center}
\begin{tabular}{l}
|\input{childdoc.def}|\\
|\childdocmain{}|\\
\end{tabular}
\end{center}
at the very top of the main \LaTeX{} file,
in particular \emph{before} the |\documentclass| statement!
The argument of |\childdocmain| should be left empty
(but it must be present).

%%%%%%%%%%%%%%%%%%%%%%%%%%%%%%%%%%%%%%%%
\DescribeMacro{\childdocof}
Furthermore, add the commands
\begin{center}
\begin{tabular}{l}
|\input{childdoc.def}|\\
|\childdocof{|\textit{main}|}|\\
\end{tabular}
\end{center}
at the top of every child file \textit{child}
which is included by |\include{|\textit{child}|}|
from within the main file
(or at least for those files to be compiled individually).
The argument \textit{main} must be the filename of the main file.

There are a couple of
considerations in setting up the main and child documents:

%%%%%%%%%%%%%%%%%%%%%%%%%%%%%%%%%%%%%%%%
\paragraph{Restrictions.}

Please note the following restrictions:
\begin{itemize}
\item
|\childdocmain| must be called with one argument \textit{main}
to ensure compatibility with earlier version of the package.
It must either be empty (|\childdocmain{}|)
or precisely match the filename of the main file in which it is specified.
See \secref{sec:detection} for further information.
\item
The filename \textit{main} must be specified without the |.tex| extension.
\item
The filename \textit{main} is case sensitive
(even in case-insensitive file systems)
due to internal string comparison.
\item
The argument \textit{main} should be fully expanded, it cannot be a macro.
\item
Subdirectories and special characters should be avoided in filenames.
\item
The command |\childdocmain{|\textit{main}|}| must be followed by a whitespace.
It should not be followed immediately by another command
or by a comment mark `|%|'.
This is because the \TeX{} parser reads the token immediately following
the argument of |\childdocmain| and puts it
at the beginning of every child section;
however, a white\-space is ignored.
\end{itemize}

%%%%%%%%%%%%%%%%%%%%%%%%%%%%%%%%%%%%%%%%
\paragraph{Content of Main File.}

It is advisable to place all content in the child files included by |\include|.
Any output contained in the main file will appear in all child documents
unless suppressed manually;
it cannot be suppressed automatically by the |\includeonly| directive
and thus should normally be avoided.
A method to include some content in the main file
by means of conditional processing is described in \secref{sec:conditional}.

%%%%%%%%%%%%%%%%%%%%%%%%%%%%%%%%%%%%%%%%
\paragraph{Page Numbering.}

When only a part of the document is compiled,
the appropriate numbering of pages
(as well as other status parameters)
is determined from the |.aux| files.
The latter contain information from previous passes.
However this information needs to propagate through
all intermediate child documents.
Therefore the page numbering in child documents may well
be inconsistent until the complete document is compiled at least once.

A useful (if unconventional) way to always ensure a consistent
page numbering is to restart the numbering in each child document
and denote the pages by `\textit{child}|.|\textit{page}'
where \textit{child} represents the chapter/section number of the child file.
This can be achieved by the command
|\numberwithin{page}{|\textit{child}|}|
of the \textsf{amsmath} package
where \textit{child} can be |chapter| or |section|
depending on the chosen structuring.
Alternatively, one can modify the macro |\thepage| appropriately
and reset the counter |page| at the start of each child file.

%%%%%%%%%%%%%%%%%%%%%%%%%%%%%%%%%%%%%%%%%%%%%%%%%%%%%%%%%%%%%%%%%%%%%%%%%%%%%%%%
\subsection{Conditional Processing}
\label{sec:conditional}

The package provides a mechanism to compile different versions
of a document. To customise the versions further some conditional processing
can come in handy to distinguish which version is being compiled.
The package provides two macros to describe the compilation context:

%%%%%%%%%%%%%%%%%%%%%%%%%%%%%%%%%%%%%%%%
\DescribeMacro{\ifchilddoc}
The conditional |\ifchilddoc| distinguishes between the compilation of
child documents and the main document:
%
\begin{center}
|\ifchilddoc |\textit{child-code}| |[|\||else |\textit{main-code}]| \||fi|
\end{center}

%%%%%%%%%%%%%%%%%%%%%%%%%%%%%%%%%%%%%%%%
\DescribeMacro{\childdocname}
\DescribeMacro{\childdocjob}
The macro |\childdocname| contains the filename (without extension)
of the main or child file being processed.
Note that |\childdocjob| will always contain the name of the main file.

%%%%%%%%%%%%%%%%%%%%%%%%%%%%%%%%%%%%%%%%
\paragraph{Title Page.}

Conditional processing can be used to include a title or banner page
in the main document when proper precautions are taken.
Importantly, the code in the main file should ensure that the page counter
(as well as other status parameters which are stored in the |.aux| files)
takes the same value after the conditional processing.
Otherwise the page numbers may take divergent values
depending on which part is compiled.

For example, a title page could be declared by:
%
\begin{center}
\begin{tabular}{l}
|\ifchilddoc\||else|\\
|\addtocounter{page}{-1}|\\
\textit{code for title page}\\
|\newpage|\\
|\||fi|
\end{tabular}
\end{center}
%
A banner page for the child documents can be generated by:
%
\begin{center}
\begin{tabular}{l}
|\ifchilddoc|\\
|\addtocounter{page}{-1}|\\
\textit{code for banner page}\\
|\newpage|\\
|\||fi|
\end{tabular}
\end{center}
%
Here one could write a message such as:
\begin{center}
|This is the part \childdocname{} of \childdocjob{}.|
\end{center}

%%%%%%%%%%%%%%%%%%%%%%%%%%%%%%%%%%%%%%%%%%%%%%%%%%%%%%%%%%%%%%%%%%%%%%%%%%%%%%%%
\subsection{Flags}
\label{sec:flags}

The package makes it easy to generate different versions
of the main or child documents.
To this end compilation flags can be defined
and assigned different default values.
They will be particularly useful in conjunction
with the forwarding mechanism described in \secref{sec:forward}.

For example, it may be useful to have a flag |\version|
which can be set to |draft| or |final|.
The document source will contain some conditional code
depending on the value of |\version|.
Suppose further, the flag should default to |final| for the main file
and to |draft| for child files
which is a natural assignment for editing the document.
This is achieved by placing the following code
in the preamble of the main document
(below the |\childdocmain| directive):
%
\begin{center}
\begin{tabular}{l}
|\ifchilddoc|\\
|\providecommand{\version}{draft}|\\
|\||else|\\
|\providecommand{\version}{final}|\\
|\||fi|
\end{tabular}
\end{center}
%
The definition by |\providecommand| makes sure
that previous definitions are not overwritten.
Further statements |\providecommand{\version}{...}|
can thus be added before the above code to override it.

For the main file, one might add a line
(between |\childdocmain| and the above block)
%
\begin{center}
|%\ifchilddoc\||else\providecommand{\version}{draft}\||fi|
\end{center}
%
which can be uncommented to produce a draft version.
Likewise one can add a line to the very top of a child file
(above the |\childdocof{|\textit{main}|}| directive)
%
\begin{center}
|%\providecommand{\version}{final}|
\end{center}
%
which can be uncommented to produce the final version of this child document.

%%%%%%%%%%%%%%%%%%%%%%%%%%%%%%%%%%%%%%%%%%%%%%%%%%%%%%%%%%%%%%%%%%%%%%%%%%%%%%%%
\subsection{Forwarding}
\label{sec:forward}

Different versions of the main or child documents
using compilation flags as described in \secref{sec:flags}
can be (permanently) stored in different files
for convenient compilation, viewing and distribution.
To this end, the package defines a command
to pass on compilation to a different file:

%%%%%%%%%%%%%%%%%%%%%%%%%%%%%%%%%%%%%%%%
\DescribeMacro{\childdocforward}
The command |\childdocforward| redirects processing to
another source file:
%
\begin{center}
\begin{tabular}{l}
|\input{childdoc.def}|\\
|\childdocforward[|\textit{main}|]{|\textit{dest}|}|\\
\end{tabular}
\end{center}
%
The argument \textit{dest} is the destination file
(without extension).
It should be the main file or one of the child files.
Note that further \textsf{childdoc} directives
such as |\childdocof| and |\childdocforward|
in the indicated file will be processed in this form.
The optional argument \textit{main}
passes on directly to the main file \textit{main}
while pretending to compile the child \textit{dest}.
This form behaves as if \textit{dest}
issues |\childdocof{|\textit{main}|}| right away,
and no further \textsf{childdoc} directives will be processed.

%%%%%%%%%%%%%%%%%%%%%%%%%%%%%%%%%%%%%%%%
\DescribeMacro{\...prefix}
In the alternative form |\childdocforwardprefix|,
%
\begin{center}
\begin{tabular}{l}
|\input{childdoc.def}|\\
|\childdocforwardprefix[|\textit{main}|]{|\textit{prefix}|}{|\textit{dest}|}|
\end{tabular}
\end{center}
%
the destination file is determined by a pattern
depending on the current file:
To make this work, the current file must be called
`{\textit{prefix}\hspace{0.2em}\textit{suffix}}'
with \textit{prefix} matching precisely the argument.
Processing is then passed on to the file
`{\textit{dest}\hspace{0.2em}\textit{suffix}}'.
Surely, the same effect is achieved by
directly specifying the
argument `{\textit{dest}\hspace{0.2em}\textit{suffix}}'
in the first form.
However, that requires to set up a different file
for each child. With the alternative form of the command
all these files can have exactly the same content
which simplifies setting them up and maintaining them.

For example, the following file |draft.tex|
with a compilation flag |\version| as described in \secref{sec:flags}
compiles the main document as a draft:
%
\begin{center}
\begin{tabular}{l}
|\def\version{draft}|\\
|\input{childdoc.def}|\\
|\childdocforward{|\textit{main}|}|
\end{tabular}
\end{center}
%
Likewise, the following files |final|\textit{nn}|.tex|
compile the final version of the child document
|child|\textit{nn}|.tex|:
%
\begin{center}
\begin{tabular}{l}
|\def\version{final}|\\
|\input{childdoc.def}|\\
|\childdocforwardprefix{final}{child}|
\end{tabular}
\end{center}
%

Note that when several versions of a main file and/or of each child file
are to be generated, it may be convenient to set up a |Makefile| or
shell script to automatise the process.

%%%%%%%%%%%%%%%%%%%%%%%%%%%%%%%%%%%%%%%%%%%%%%%%%%%%%%%%%%%%%%%%%%%%%%%%%%%%%%%%
\subsection{Command Line Processing}
\label{sec:commandline}

The effect of redirection files can also be achieved by invoking
the \LaTeX{} compiler with a more elaborate command line.
Most conveniently this should be done as part
of a shell script or a |Makefile|.

When using \textsf{childdoc} in the main file, the following
command lines effectively perform a redirection
(note that depending on the shell being used,
backslashes may have to be doubled: `|\|' $\to$ `|\\|'):
%
\begin{center}
|... -jobname "|\textit{target}|" |\\|"|[\textit{flags}]%
|\input{childdoc.def}\childdocforward[|\textit{main}|]{|\textit{dest}|}"|
\end{center}
%
Here \textit{target} is the name of the output file,
\textit{main} is the name of the main file
and \textit{dest} is the name of the main or child file to be processed
(all filenames without extensions).
The optional argument \textit{main} can be omitted
if \textit{main} matches \textit{dest}.
Optionally, compilation \textit{flags} can be defined via |\def| commands.
This command line makes the \TeX{} engine believe
it is compiling the file \textit{target}
whose content is specified as the latter parameter.
The provided code then forwards the processing to
\textit{main} or \textit{dest} as described in \secref{sec:forward}.

%%%%%%%%%%%%%%%%%%%%%%%%%%%%%%%%%%%%%%%%%%%%%%%%%%%%%%%%%%%%%%%%%%%%%%%%%%%%%%%%
\subsection{Include by Input}
\label{sec:input}

Including child documents by |\include| has some restrictions by design.
Most notably, the content of a child document always occupies
its own set of pages; pages cannot be shared between child documents.
Usually, this behaviour makes perfect sense
because each child document contain an essential part of the document.
However, in some situations it may be desirable to compose
a document from a collection of parts
without having mandatory page breaks between then.
For this case, the package
provides a mechanism to include parts
by |\input| which can also be processed individually.
However, by construction this mechanism
requires manual handling of the content to be output.

%%%%%%%%%%%%%%%%%%%%%%%%%%%%%%%%%%%%%%%%
\DescribeMacro{\ifchilddocmanual}
The main file should be prepared as usual, see \secref{sec:include}.
However, the document body must make a distinction
between processing of an individual part and of the main document, e.g.:
%
\begin{center}
\begin{tabular}{l}
|\ifchilddocmanual|\\
|\input{\childdocname}|\\
|\||else|\\
\textit{document body with }|\input{|\textit{part}|}|\\
|\||fi|
\end{tabular}
\end{center}
%
The conditional |\ifchilddocmanual| is true whenever
a part to be included by |\input| is being compiled,
and the name of the part is stored in |\childdocname|.

%%%%%%%%%%%%%%%%%%%%%%%%%%%%%%%%%%%%%%%%
\DescribeMacro{\childdocby}
Each part to be included by |\input| should start with:
%
\begin{center}
\begin{tabular}{l}
|\input{childdoc.def}|\\
|\childdocby{|\textit{main}|}|\\
\end{tabular}
\end{center}
%
The directive |\childdocby| is similar to |\childdocof|
described in \secref{sec:include},
but the subsequent selection of content must be done manually.
To that end, both |\ifchilddoc| and |\ifchilddocmanual|
will be true upon processing of a part,
and the name of the part is stored in |\childdocname|.
Note that |\jobname| will be set to the filename of the current part
so that each part receives an individual |.aux| file
that does not interfere with the |.aux| file(s) of the main document.
This behaviour can be altered by the alternative form
|\childdocby[*]{|\textit{main}|}| (with a non-empty optional argument)
which uses the |.aux| file of the main document
by setting |\jobname| to \textit{main}.

%%%%%%%%%%%%%%%%%%%%%%%%%%%%%%%%%%%%%%%%%%%%%%%%%%%%%%%%%%%%%%%%%%%%%%%%%%%%%%%%
\subsection{Driver Development}
\label{sec:driver}

The \textsf{childdoc} mechanism can also be use for the development
of definition files such as \LaTeX{} styles or classes.
This case differs from the above setup with multiple parts
included by |\include| in that no |\includeonly| should be invoked.
This can be achieved by starting the include file
(before |\ProvidesPackage|) with:
%
\begin{center}
\begin{tabular}{l}
|\input{childdoc.def}|\\
|\childdocforward{|\textit{main}|}|\\
\end{tabular}
\end{center}
%
or alternatively with:
%
\begin{center}
\begin{tabular}{l}
|\input{childdoc.def}|\\
|\childdocby{|\textit{main}|}|\\
\end{tabular}
\end{center}
%
Both forms have slightly different effects as described above.
The main file is prepared as usual, see \secref{sec:include}.

%%%%%%%%%%%%%%%%%%%%%%%%%%%%%%%%%%%%%%%%%%%%%%%%%%%%%%%%%%%%%%%%%%%%%%%%%%%%%%%%
\subsection{Legacy Detection}
\label{sec:detection}

The directive |\childdocmain| in the main file can detect
whether the complete document or merely a child is to be compiled
even without using the directive |\childdocof|.
This method is deprecated because it is less robust
and there is no compelling reason to use it;
it is merely provided for backward compatibility
and it may be removed in future versions.

If the detection mechanism is to be used,
it is mandatory to correctly specify
the filename of the main file as the argument of |\childdocmain|:
%
\begin{center}
\begin{tabular}{l}
|\input{childdoc.def}|\\
|\childdocmain{|\textit{main}|}|\\
\end{tabular}
\end{center}
%
If |\jobname| does not match the argument \textit{main} of |\childdocmain|,
it is assumed that |\jobname| points to the child file to be compiled.
When using |\childdocmain| with the main file specified as argument,
it suffices to start a child file
with just |\input{|\textit{main}|}|
without loading of the package and using |\childdocof|.
If instead all processing is done
with the appropriate \textsf{childdoc} directives,
the argument of \textit{main} of |\childdocmain| can be empty.

An alternative version of the command line processing described
in \secref{sec:commandline} using the detection mechanism reads:
%
\begin{center}
|... -jobname "|\textit{target}|" "|[\textit{flags}]%
[|\def\jobname{|\textit{dest}|}|]|\input{|\textit{main}|}"|
\end{center}

%%%%%%%%%%%%%%%%%%%%%%%%%%%%%%%%%%%%%%%%%%%%%%%%%%%%%%%%%%%%%%%%%%%%%%%%%%%%%%%%
\subsection{Manual Code}
\label{sec:manual}

In case one cannot be certain whether the definitions file |childdoc.def|
is installed on the target \TeX{} distribution
and one prefers not to ship it,
it is conceivable to paste a few relevant commands into the sources.

To that end, drop all statements |\input{childdoc.def}|
and perform the replacements as outlined below.
Instead of |\childdocmain{|\textit{main}|}| add the following code
to the top of the main file:
%
\begin{center}
\begin{tabular}{l}
|\||ifdefined\childdocname\endinput\||fi\newif\ifchilddoc|\\
|\edef\childdocname{\scantokens\expandafter{\jobname\noexpand}}|\\
|\def\childdocmain{|\textit{main}|}\||ifx\childdocmain\childdocname\||else|\\
|\childdoctrue\includeonly{\childdocname}\let\jobname\childdocmain\||fi|\\
\end{tabular}
\end{center}
%
Instead of |\childdocof{|\textit{main}|}| just include the main file
at the top of each child file:
%
\begin{center}
|\input{|\textit{main}|}|
\end{center}
%
A simple redirection |\childdocforward{|\textit{dest}|}| is achieved by:
%
\begin{center}
|\def\jobname{|\textit{dest}|}\input{\jobname}|
\end{center}
%
The redirection with prefix
|\childdocforwardprefix[|\textit{prefix}|]{|\textit{dest}|}|
is accomplished by:
%
\begin{center}
\begin{tabular}{l}
|{\edef\jobname{\scantokens\expandafter{\jobname\noexpand}}|\\
|\def\redirectjob |\textit{prefix}|#1~~~{\gdef\jobname{|\textit{dest}|#1}}|\\
|\expandafter\redirectjob\jobname~~~}\input{\jobname}|
\end{tabular}
\end{center}

In an alternative approach,
child documents can be compiled by a specific command line
without additional code or specific definitions:
%
\begin{center}
|... -jobname "|\textit{target}|" "|[\textit{flags}]%
|\includeonly{|\textit{dest}|}\input{|\textit{main}|}"|
\end{center}
%

%%%%%%%%%%%%%%%%%%%%%%%%%%%%%%%%%%%%%%%%%%%%%%%%%%%%%%%%%%%%%%%%%%%%%%%%%%%%%%%%
%%%%%%%%%%%%%%%%%%%%%%%%%%%%%%%%%%%%%%%%%%%%%%%%%%%%%%%%%%%%%%%%%%%%%%%%%%%%%%%%
\section{Information}

%%%%%%%%%%%%%%%%%%%%%%%%%%%%%%%%%%%%%%%%%%%%%%%%%%%%%%%%%%%%%%%%%%%%%%%%%%%%%%%%
\subsection{Copyright}

Copyright \copyright{} 2017--2018 Niklas Beisert

This work may be distributed and/or modified under the
conditions of the \LaTeX{} Project Public License, either version 1.3
of this license or (at your option) any later version.
The latest version of this license is in
  \url{http://www.latex-project.org/lppl.txt}
and version 1.3 or later is part of all distributions of \LaTeX{}
version 2005/12/01 or later.

This work has the LPPL maintenance status `maintained'.

The Current Maintainer of this work is Niklas Beisert.

This work consists of the files |README.txt|, |childdoc.ins| and |childdoc.dtx|
as well as the derived files |childdoc.def|, |cdocsamp.tex|
with |cdocsch1.tex|, |cdocsch2.tex|, |cdocspt3.tex|, |cdocspt4.tex|,
|cdocsdrf.tex|, |cdocsfn1.tex|, |cdocsfn2.tex|
as well as |childdoc.pdf|.

%%%%%%%%%%%%%%%%%%%%%%%%%%%%%%%%%%%%%%%%%%%%%%%%%%%%%%%%%%%%%%%%%%%%%%%%%%%%%%%%
\subsection{Files and Installation}

The package consists of the files:
%
\begin{center}
\begin{tabular}{ll}
    |README.txt|   & readme file \\
    |childdoc.ins| & installation file \\
    |childdoc.dtx| & source file \\
    |childdoc.def| & definition file \\
    |cdocsamp.tex| & sample main file \\
    |cdocsch1.tex| & sample include file \\
    |cdocsch2.tex| & sample include file \\
    |cdocspt3.tex| & sample part file \\
    |cdocspt4.tex| & sample part file \\
    |cdocsdrf.tex| & sample redirection file \\
    |cdocsfn1.tex| & sample redirection file \\
    |cdocsfn2.tex| & sample redirection file \\
    |childdoc.pdf| & manual
\end{tabular}
\end{center}
%
The distribution consists of the files
|README.txt|, |childdoc.ins| and |childdoc.dtx|.
%
\begin{itemize}
\item
Run (pdf)\LaTeX{} on |childdoc.dtx|
to compile the manual |childdoc.pdf| (this file).
\item
Run \LaTeX{} on |childdoc.ins| to create the definitions file |childdoc.def|
and the sample |cdocsamp.tex| with include files
|cdocsch1.tex|, |cdocsch2.tex|, |cdocspt3.tex|, |cdocspt4.tex|,
|cdocsdrf.tex|, |cdocsfn1.tex|, |cdocsfn2.tex|.
Then copy the file |childdoc.def| to an appropriate directory of your \LaTeX{}
distribution, e.g.\ \textit{texmf-root}|/tex/latex/childdoc|.
\end{itemize}

%%%%%%%%%%%%%%%%%%%%%%%%%%%%%%%%%%%%%%%%%%%%%%%%%%%%%%%%%%%%%%%%%%%%%%%%%%%%%%%%
\subsection{Related CTAN Packages}

There are several other packages which offer a similar functionality:
%
\begin{itemize}
\item
The packages
\href{http://ctan.org/pkg/docmute}{\textsf{docmute}},
\href{http://ctan.org/pkg/includex}{\textsf{includex}} and
\href{http://ctan.org/pkg/standalone}{\textsf{standalone}}
provide commands to include only the document body of
a child file thus allowing both files to be compiled individually.
\item
The packages \href{http://ctan.org/pkg/subdocs}{\textsf{subdocs}}
and \href{http://ctan.org/pkg/subfiles}{\textsf{subfiles}}
provide structures in which the main and child documents can be
encapsulated and allowing them to be compiled individually.
The inclusion mechanism is different from the conventional |\include|.
\item
The package \href{http://ctan.org/pkg/combine}{\textsf{combine}}
is an elaborate solution to combine several documents into one.
\end{itemize}
%
See also the CTAN topic \href{http://ctan.org/topic/subdocs}{\textsf{subdocs}}
for further related packages.
The present package differs from the above solutions in that
a document structure constructed with the conventional |\include| mechanism
just needs two extra commands at the top of every file
such that all constituent files can be compiled individually.

%%%%%%%%%%%%%%%%%%%%%%%%%%%%%%%%%%%%%%%%%%%%%%%%%%%%%%%%%%%%%%%%%%%%%%%%%%%%%%%%
%\subsection{Feature Suggestions}
%
%The following is a list of features which may be useful for future
%versions of this package:
%%
%\begin{itemize}
%\item
%\ldots
%\end{itemize}

%%%%%%%%%%%%%%%%%%%%%%%%%%%%%%%%%%%%%%%%%%%%%%%%%%%%%%%%%%%%%%%%%%%%%%%%%%%%%%%%
\subsection{Revision History}

%%%%%%%%%%%%%%%%%%%%%%%%%%%%%%%%%%%%%%%%
\paragraph{v2.0:} 2018/12/30

\begin{itemize}
\item
immediate forward processing
\item
added |\childdocby| mechanism
\item
manual restructured
\end{itemize}

%%%%%%%%%%%%%%%%%%%%%%%%%%%%%%%%%%%%%%%%
\paragraph{v1.6:} 2018/01/17

\begin{itemize}
\item
application for development of include files
\item
corrections to manual
\end{itemize}

%%%%%%%%%%%%%%%%%%%%%%%%%%%%%%%%%%%%%%%%
\paragraph{v1.5:} 2017/05/21

\begin{itemize}
\item
more complete structuring introduced
\item
|\childdocof| introduced
\item
|\childdoc| renamed to |\childdocmain|
\item
|\childredirect| renamed to |\childdocforward| and |\childdocforwardprefix|
and functionality expanded
\end{itemize}

%%%%%%%%%%%%%%%%%%%%%%%%%%%%%%%%%%%%%%%%
\paragraph{v1.0:} 2017/04/27

\begin{itemize}
\item
manual and install package
\item
first version published on CTAN
\end{itemize}

%%%%%%%%%%%%%%%%%%%%%%%%%%%%%%%%%%%%%%%%
\paragraph{v0.6:} 2017/04/26

\begin{itemize}
\item
redirection mechanism added
\end{itemize}

%%%%%%%%%%%%%%%%%%%%%%%%%%%%%%%%%%%%%%%%
\paragraph{v0.5:} 2017/04/26

\begin{itemize}
\item
functionality in definition file
\end{itemize}


%%%%%%%%%%%%%%%%%%%%%%%%%%%%%%%%%%%%%%%%%%%%%%%%%%%%%%%%%%%%%%%%%%%%%%%%%%%%%%%%
%%%%%%%%%%%%%%%%%%%%%%%%%%%%%%%%%%%%%%%%%%%%%%%%%%%%%%%%%%%%%%%%%%%%%%%%%%%%%%%%
%%%%%%%%%%%%%%%%%%%%%%%%%%%%%%%%%%%%%%%%%%%%%%%%%%%%%%%%%%%%%%%%%%%%%%%%%%%%%%%%
\appendix

\settowidth\MacroIndent{\rmfamily\scriptsize 000\ }

 \DocInput{childdoc.dtx}

\end{document}
%</driver>
% \fi
%
% %%%%%%%%%%%%%%%%%%%%%%%%%%%%%%%%%%%%%%%%%%%%%%%%%%%%%%%%%%%%%%%%%%%%%%%%%%%%%%
% %%%%%%%%%%%%%%%%%%%%%%%%%%%%%%%%%%%%%%%%%%%%%%%%%%%%%%%%%%%%%%%%%%%%%%%%%%%%%%
% \section{Sample}
%\iffalse
%<*samplemain>
%\fi
%
% The following presents a sample document
% with two chapters, two parts, a title page,
% a compile flag as well as three forwarding files to set the flag.
% It consists of eight |.tex| files:
% \begin{center}
% \begin{tabular}{ll}
% |cdocsamp.tex|&main file\\
% |cdocsch1.tex|&include file for chapter 1\\
% |cdocsch2.tex|&include file for chapter 2\\
% |cdocspt3.tex|&include file for part 3\\
% |cdocspt4.tex|&include file for part 4\\
% |cdocsdrf.tex|&forwarding file for main file in draft mode\\
% |cdocsfi1.tex|&forwarding file for final version of chapter 1\\
% |cdocsfi2.tex|&forwarding file for final version of chapter 2\\
% \end{tabular}
% \end{center}
% Each of the eight files can be compiled directly by the \LaTeX{} compiler.
%
% %%%%%%%%%%%%%%%%%%%%%%%%%%%%%%%%%%%%%%
% \paragraph{Main File.}
%
% The main file is called |cdocsamp.tex|.
%
% Load the \textsf{childdoc} definitions and
% declare the filename for the main document:
%    \begin{macrocode}
\input{childdoc.def}
\childdocmain{}
%    \end{macrocode}

% Optional override for |\version| flag:
%    \begin{macrocode}
%%\ifchilddoc\else\providecommand{\version}{draft}\fi
%    \end{macrocode}

% Define the default values for the |\version| flag
% (|final| for the main file and |draft| for childs):
%    \begin{macrocode}
\ifchilddoc
\providecommand{\version}{draft}
\else
\providecommand{\version}{final}
\fi
%    \end{macrocode}

% Load the standard document class:
%    \begin{macrocode}
\documentclass[12pt]{article}
%    \end{macrocode}

% Start the document body:
%    \begin{macrocode}
\begin{document}
%    \end{macrocode}

% Declare a title page.
% Print title, part of document being processed and version flag:
%    \begin{macrocode}
\addtocounter{page}{-1}
\begin{center}
{\LARGE\bfseries{}childdoc example\par}
\vspace{1cm}
\ifchilddoc
\ifchilddocmanual part\else chapter\fi:
`\childdocname' of `\childdocjob'\par
\else
main document: `\childdocjob'\par
\fi
version: \version\par
\end{center}
\newpage
%    \end{macrocode}

% Manually include selected file,
% otherwise process as usual:
%    \begin{macrocode}
\ifchilddocmanual
\section*{part `\childdocname'}
\input{\childdocname}
\else
%    \end{macrocode}

% Include the two chapters:
%    \begin{macrocode}
\include{cdocsch1}
\include{cdocsch2}
%    \end{macrocode}

% Include the two parts unless only chapters should be displayed:
%    \begin{macrocode}
\ifchilddoc\else
\section{part three}
\input{cdocspt3}
\section{part four}
\input{cdocspt4}
\fi
%    \end{macrocode}

% Process as usual until here:
%    \begin{macrocode}
\fi
%    \end{macrocode}

% End of document body:
%    \begin{macrocode}
\end{document}
%    \end{macrocode}
%\iffalse
%</samplemain>
%\fi
%
% %%%%%%%%%%%%%%%%%%%%%%%%%%%%%%%%%%%%%%
% \paragraph{Chapter Include Files.}
%
% The include files are called |cdocsch1.tex| and |cdocsch2.tex|.
%
%\iffalse
%<*samplechap1|samplechap2>
%\fi

% Optional override for |\version| flag:
%    \begin{macrocode}
%%\providecommand{\version}{final}
%    \end{macrocode}

% Include the main document:
%    \begin{macrocode}
\input{childdoc.def}
\childdocof{cdocsamp}
%    \end{macrocode}

%\iffalse
%</samplechap1|samplechap2>
%\fi
%
%\iffalse
%<*samplechap1>
%\fi
% Some text for chapter 1:
%    \begin{macrocode}
\section{one}
some text in chapter one
%    \end{macrocode}

%\iffalse
%</samplechap1>
%\fi
% Some text for chapter 2:
%\iffalse
%<*samplechap2>
%\fi
%    \begin{macrocode}
\section{two}
more text in chapter two
%    \end{macrocode}

%\iffalse
%</samplechap2>
%\fi
%
% %%%%%%%%%%%%%%%%%%%%%%%%%%%%%%%%%%%%%%
% \paragraph{Part Include Files.}
%
% The include files are called |cdocspt3.tex| and |cdocspt4.tex|.
%
%\iffalse
%<*samplepart3|samplepart4>
%\fi

% Optional override for |\version| flag:
%    \begin{macrocode}
%%\providecommand{\version}{final}
%    \end{macrocode}

% Include the main document:
%    \begin{macrocode}
\input{childdoc.def}
\childdocby{cdocsamp}
%    \end{macrocode}

%\iffalse
%</samplepart3|samplepart4>
%\fi
%
%\iffalse
%<*samplepart3>
%\fi
% Some text for part 3:
%    \begin{macrocode}
some text in part three
%    \end{macrocode}

%\iffalse
%</samplepart3>
%\fi
% Some text for part 4:
%\iffalse
%<*samplepart4>
%\fi
%    \begin{macrocode}
more text in part four
%    \end{macrocode}

%\iffalse
%</samplepart4>
%\fi
%
% %%%%%%%%%%%%%%%%%%%%%%%%%%%%%%%%%%%%%%
% \paragraph{Forwarding for a Complete Draft.}
%
% The following forwarding file |cdocsdrf.tex|
% compiles the main document in draft mode:
%\iffalse
%<*sampledraft>
%\fi
%    \begin{macrocode}
\def\version{draft}
\input{childdoc.def}
\childdocforward{cdocsamp}
%    \end{macrocode}

%\iffalse
%</sampledraft>
%\fi
%
% %%%%%%%%%%%%%%%%%%%%%%%%%%%%%%%%%%%%%%
% \paragraph{Forwarding for Final Version of the Chapters.}
%
% The following forwarding files |cdocsfn1.tex| and |cdocsfn2.tex|
% (with identical content)
% compile the final versions of the child documents
% |cdocsch1.tex| and |cdocsch2.tex|, respectively:
%\iffalse
%<*samplefinal>
%\fi
%    \begin{macrocode}
\def\version{final}
\input{childdoc.def}
\childdocforwardprefix[cdocsamp]{cdocsfn}{cdocsch}
%    \end{macrocode}

%\iffalse
%</samplefinal>
%\fi
%
% %%%%%%%%%%%%%%%%%%%%%%%%%%%%%%%%%%%%%%
% \paragraph{Command Line Processing.}
%
% The following three command lines generate the output files
% |cdocscld|, |cdocscl1| and |cdocscl2|
% which should be identical to
% |cdocsdrf|, |cdocsch1| and |cdocsfn2|, respectively:
% \begin{center}
% \begin{tabular}{l}
% |latex -jobname cdocscld \|\\
% |  "\def\version{draft}\input{childdoc.def}\childdocforward{cdocsamp}"|\\
% |latex -jobname cdocscl1 \|\\
% |  "\input{childdoc.def}\childdocforward[cdocsamp]{cdocsch1}"|\\
% |latex -jobname cdocscl2 \|\\
% |  "\def\version{final}\input{childdoc.def}\childdocforward{cdocsch2}"|
% \end{tabular}
% \end{center}
% Note that the trailing backslash on each first line
% merely continues the input to the second line
% (for convenient cut ant paste).
% Furthermore, the command |latex| can be replaced by any
% of its alternative versions such as |pdflatex|.
%
% %%%%%%%%%%%%%%%%%%%%%%%%%%%%%%%%%%%%%%%%%%%%%%%%%%%%%%%%%%%%%%%%%%%%%%%%%%%%%%
% %%%%%%%%%%%%%%%%%%%%%%%%%%%%%%%%%%%%%%%%%%%%%%%%%%%%%%%%%%%%%%%%%%%%%%%%%%%%%%
% \section{Implementation}
%\iffalse
%<*package>
%\fi
%
% This section describes the definitions file |childdoc.def|.

% The definitions cannot be loaded using |\usepackage| or |\RequirePackage|
% which has a mechanism to prevent loading a style file more than once.
% When loading the definitions by means of |\input|
% multiple instances have to be prevented manually:
%\iffalse
%This code needs to be before the `\ProvidesFile' directive
%which is defined at the beginning of this file.
%Therefore it is also placed there and commented out here.
%</package>
%<*discard>
%\fi
%    \begin{macrocode}
\ifdefined\childdocmain\endinput\fi
%    \end{macrocode}
%\iffalse
%</discard>
%<*package>
%\fi
%
% \macro{\ifchilddoc}
% \macro{\ifchilddocmanual}
% The conditional |\ifchilddoc| tells whether a
% child (true) or main (false) document is being compiled.
% The conditional |\ifchilddocmanual| tells whether
% the |\includeonly| mechanism is used (false) or
% the selection of child files must be performed manually (true).
% The definitions initialise to false:
%    \begin{macrocode}
\newif\ifchilddoc
\newif\ifchilddocmanual
%    \end{macrocode}

% \macro{\childdocname}
% \macro{\childdocjob}
% The macro |\childdocname| stores the name of the main document
% to be compiled. The macro |\childdocjob| stores the name of
% the document on which the \LaTeX{} compiler was originally invoked.
% The content of |\jobname| cannot be compared
% to filenames specified in the source due to different catcodes.
% The following code rescans |\jobname|, stores the result
% in |\childdocname| and saves a copy in |\childdocjob|:
%    \begin{macrocode}
\edef\childdocname{\scantokens\expandafter{\jobname\noexpand}}
\let\childdocjob\childdocname
%    \end{macrocode}

% \macro{\childdocdisable}
% The macro |\childdocdisable| prevents the main file
% from being processed more than once.
% At this stage, the main document command |\childdocmain|
% is assumed to be called once again where it should do nothing.
% Any subsequent call to it should prevent
% a secondary processing of the main document
% It overwrites the forwarding commands
% |\childdocof| and |\childdocforward|
% with empty macros to prevent further inclusions of the main document:
%    \begin{macrocode}
\newcommand{\childdocdisable}
{
  \renewcommand{\childdocmain}[1]{\renewcommand{\childdocmain}[1]{\endinput}}
  \renewcommand{\childdocof}[1]{}
  \renewcommand{\childdocby}[2][]{}
  \renewcommand{\childdocforward}[2][]{}
  \renewcommand{\childdocdisable}{}
}
%    \end{macrocode}

% \macro{\childdocmain}
% The macro |\childdocmain| is to be called at the top of the main file
% with nothing or the main filename (without extension) as argument.
% First, it breaks loops.
% If the argument is not empty and does not match |\childdocname|
% (which is set by the first inclusion of |childdoc.def|),
% |\ifchilddoc| is set to true, |\includeonly| is applied to the child file
% and |\jobname| is set to the main file
% (for proper handling of |.aux| files):
%    \begin{macrocode}
\newcommand{\childdocmain}[1]
{
  \childdocdisable\childdocmain{}
  \if?#1?\else
    \begingroup
      \def\childdoctmp{#1}
      \ifx\childdoctmp\childdocname
        \def\childdoctmp{}
      \else
        \def\childdoctmp
        {
          \childdoctrue
          \includeonly{\childdocname}
          \def\childdocjob{#1}
          \def\jobname{#1}
        }
      \fi
      \expandafter
    \endgroup
    \childdoctmp
  \fi
}
%    \end{macrocode}

% \macro{\childdocof}
% The command |\childdocof| redirects
% compilation to the main file |#1|.
%    \begin{macrocode}
\newcommand{\childdocof}[1]
{
  \childdocdisable
  \childdoctrue
  \includeonly{\childdocname}
  \def\jobname{#1}
  \def\childdocjob{#1}
  \input{#1}
}
%    \end{macrocode}

% \macro{\childdocby}
% The command |\childdocby| ....
%    \begin{macrocode}
\newcommand{\childdocby}[2][]
{
  \childdocdisable
  \childdoctrue
  \childdocmanualtrue
  \if?#1?\else
    \def\jobname{#2}
  \fi
  \def\childdocjob{#2}
  \input{#2}
  \endinput
}
%    \end{macrocode}

% \macro{\childdocforward}
% The command |\childdocforward| redirects
% compilation to the main file or
% (if the optional argument is given) a child file.
% Parameters are set as if the main file
% or a child file starting with |\childdocof| was compiled.
% Then compilation is handed over to the main file:
%    \begin{macrocode}
\newcommand{\childdocforward}[2][]
{
  \begingroup
    \if?#1?
      \def\childdoctmp
      {
        \def\childdocname{#2}
        \def\childdocjob{#2}
        \def\jobname{#2}
        \input{#2}
        \endinput
      }
    \else
      \def\childdoctmp
      {
        \childdocdisable
        \def\childdocname{#2}
        \childdoctrue
        \includeonly{#2}
        \def\childdocjob{#1}
        \def\jobname{#1}
        \input{#1}
        \endinput
      }
    \fi
    \expandafter
  \endgroup
  \childdoctmp
}
%    \end{macrocode}

% \macro{\childdocforwardprefix}
% The command |\childdocforwardprefix| redirects
% compilation to the main or a child file by means of a pattern.
% The prefix |#1| in the current filename is replaced by |#2|
% and the suffix of the current filename is kept
% (it is assumed that the filename does not contain the substring `|~~~|'
% which is used as a delimiter).
% Compilation is handed over to the new file by |\childdocforward|:
%    \begin{macrocode}
\newcommand{\childdocforwardprefix}[3][]
{
  \begingroup
    \def\childdocextract #2##1~~~{\def\childdoctmp{\childdocforward[#1]{#3##1}}}
    \expandafter\childdocextract\childdocname~~~
    \expandafter
  \endgroup
  \childdoctmp
}
%    \end{macrocode}

% \macro{\childdoc}
% The deprecated macro |\childdoc| is a legacy version of |\childdocmain|:
%    \begin{macrocode}
\newcommand{\childdoc}{\childdocmain}
%    \end{macrocode}

% \macro{\childdocredirect}
% The deprecated macro |\childdocredirect| is a legacy version
% of |\childdocforward| and |\childdocforwardprefix|:
%    \begin{macrocode}
\newcommand{\childdocredirect}[2][]
{
  \begingroup
    \if?#1?
      \def\childdoctmp{\childdocforward{#2}}
    \else
      \def\childdoctmp{\childdocforwardprefix{#1}{#2}}
    \fi
    \expandafter
  \endgroup
  \childdoctmp
}
%    \end{macrocode}

%\iffalse
%</package>
%\fi
%
\endinput
|\\
|\childdocforwardprefix{final}{child}|
\end{tabular}
\end{center}
%

Note that when several versions of a main file and/or of each child file
are to be generated, it may be convenient to set up a |Makefile| or
shell script to automatise the process.

%%%%%%%%%%%%%%%%%%%%%%%%%%%%%%%%%%%%%%%%%%%%%%%%%%%%%%%%%%%%%%%%%%%%%%%%%%%%%%%%
\subsection{Command Line Processing}
\label{sec:commandline}

The effect of redirection files can also be achieved by invoking
the \LaTeX{} compiler with a more elaborate command line.
Most conveniently this should be done as part
of a shell script or a |Makefile|.

When using \textsf{childdoc} in the main file, the following
command lines effectively perform a redirection
(note that depending on the shell being used,
backslashes may have to be doubled: `|\|' $\to$ `|\\|'):
%
\begin{center}
|... -jobname "|\textit{target}|" |\\|"|[\textit{flags}]%
|% \iffalse
%
% childdoc.dtx Copyright (C) 2017-2018 Niklas Beisert
%
% This work may be distributed and/or modified under the
% conditions of the LaTeX Project Public License, either version 1.3
% of this license or (at your option) any later version.
% The latest version of this license is in
%   http://www.latex-project.org/lppl.txt
% and version 1.3 or later is part of all distributions of LaTeX
% version 2005/12/01 or later.
%
% This work has the LPPL maintenance status `maintained'.
%
% The Current Maintainer of this work is Niklas Beisert.
%
% This work consists of the files childdoc.dtx and childdoc.ins
% and the derived files childdoc.def and cdocsamp.tex with
% cdocsch1.tex, cdocsch2.tex, cdocsdrf.tex, cdocsfn1.tex, cdocsfn2.tex.
%
%<package>\ifdefined\childdocmain\endinput\fi
%<package>\ProvidesFile{childdoc.def}[2018/12/30 v2.0 child document driver]
%<samplemain>\ProvidesFile{cdocsamp.tex}[2018/12/30 v2.0 sample for childdoc]
%<*driver>
%\ProvidesFile{childdoc.drv}[2018/12/30 v2.0 childdoc reference manual file]
\PassOptionsToClass{10pt,a4paper}{article}
\documentclass{ltxdoc}

\usepackage[margin=35mm]{geometry}
\usepackage{hyperref}
\usepackage{hyperxmp}
\usepackage[usenames]{color}

\hypersetup{colorlinks=true}
\hypersetup{pdfstartview=FitH}
\hypersetup{pdfpagemode=UseNone}
\hypersetup{pdfsource={}}
\hypersetup{pdflang={en-UK}}
\hypersetup{pdfcopyright={Copyright 2017-2018 Niklas Beisert.
  This work may be distributed and/or modified under the
  conditions of the LaTeX Project Public License, either version 1.3
  of this license or (at your option) any later version.}}
\hypersetup{pdflicenseurl={http://www.latex-project.org/lppl.txt}}
\hypersetup{pdfcontactaddress={ETH Zurich, ITP, HIT K,
  Wolfgang-Pauli-Strasse 27}}
\hypersetup{pdfcontactpostcode={8093}}
\hypersetup{pdfcontactcity={Zurich}}
\hypersetup{pdfcontactcountry={Switzerland}}
\hypersetup{pdfcontactemail={nbeisert@itp.phys.ethz.ch}}
\hypersetup{pdfcontacturl={http://people.phys.ethz.ch/\xmptilde nbeisert/}}

\newcommand{\secref}[1]{\hyperref[#1]{section \ref*{#1}}}

\parskip1ex
\parindent0pt
\let\olditemize\itemize
\def\itemize{\olditemize\parskip0pt}

\begin{document}

\title{The \textsf{childdoc} Package}
\hypersetup{pdftitle={The childdoc Package}}
\author{Niklas Beisert\\[2ex]
  Institut f\"ur Theoretische Physik\\
  Eidgen\"ossische Technische Hochschule Z\"urich\\
  Wolfgang-Pauli-Strasse 27, 8093 Z\"urich, Switzerland\\[1ex]
  \href{mailto:nbeisert@itp.phys.ethz.ch}
  {\texttt{nbeisert@itp.phys.ethz.ch}}}
\hypersetup{pdfauthor={Niklas Beisert}}
\hypersetup{pdfsubject={Manual for the LaTeX2e Package childdoc}}
\date{30 December 2018, \textsf{v2.0}}
\maketitle

\begin{abstract}\noindent
\textsf{childdoc} is a \LaTeXe{} package
that enables the direct compilation
of document sections included by |\include|
to individual files.
\end{abstract}

\begingroup
\parskip0ex
\tableofcontents
\endgroup

%%%%%%%%%%%%%%%%%%%%%%%%%%%%%%%%%%%%%%%%%%%%%%%%%%%%%%%%%%%%%%%%%%%%%%%%%%%%%%%%
%%%%%%%%%%%%%%%%%%%%%%%%%%%%%%%%%%%%%%%%%%%%%%%%%%%%%%%%%%%%%%%%%%%%%%%%%%%%%%%%
\section{Introduction}

\LaTeX{} provides a mechanism to structure a large document (such as a book)
into a main file and several child files (containing the chapters)
using the |\include| command.
This mechanism is beneficial for documents
which span hundreds of pages in order to
make the source file(s) more manageable.
Moreover, compilation can be restricted to
selected child files by means of the |\includeonly| command.
The latter feature can be used to reduce the compilation time while editing
(this was significantly more useful in the earlier days of \LaTeX{})
or to generate a smaller document which is easier to navigate.
Another application of |\includeonly| is to generate
documents consisting of selected parts of the complete document.

However, there are a few drawbacks of the plain |\include| mechanism:
\begin{itemize}
\item
The child files cannot be compiled on their own,
they can only be compiled via the main file.
A naive editing environment
(such as a text editor with an option
to have the current file processed by \LaTeX)
may require one to switch to the main file before compiling;
attempting to compile the child file produces errors.
\item
The main file must be modified (each time)
to adjust the |\includeonly| command
to the present needs. This easily leaves the main file in a messy state.
\item
The generated document will always carry the filename
of the main document. This is inconvenient if
several child files are to be compiled and
to be kept for distribution.
\end{itemize}

The present package provides a simple interface
to make child files individually compilable by \LaTeX{}.
Compiling a child file then has the same effect as compiling
the main file with an |\includeonly| command
to select the appropriate child.
Moreover the generated document will carry the name of the child
rather than the main file.
This resolves all three above issues.

This feature is meant to make the editing of books,
thesis documents and lecture notes somewhat more convenient.
However, the package can also be used efficiently for
composing a series of documents (such as exercise sheets)
which are typically distributed individually.
It then assists the author in generating the individual documents
(potentially in different versions)
as well as a document containing the collected series.
Another application is in developing style files
or other kinds of included material
where compilation of the style file could redirect
to a sample or test file.

%%%%%%%%%%%%%%%%%%%%%%%%%%%%%%%%%%%%%%%%%%%%%%%%%%%%%%%%%%%%%%%%%%%%%%%%%%%%%%%%
%%%%%%%%%%%%%%%%%%%%%%%%%%%%%%%%%%%%%%%%%%%%%%%%%%%%%%%%%%%%%%%%%%%%%%%%%%%%%%%%
\section{Usage}

First of all, the package \textsf{childdoc} is \emph{not} a standard
\LaTeXe{} |.sty| style file! Therefore it needs to be invoked in
a non-standard way.

%%%%%%%%%%%%%%%%%%%%%%%%%%%%%%%%%%%%%%%%%%%%%%%%%%%%%%%%%%%%%%%%%%%%%%%%%%%%%%%%
\subsection{Included Files}
\label{sec:include}

%%%%%%%%%%%%%%%%%%%%%%%%%%%%%%%%%%%%%%%%
\DescribeMacro{\childdocmain}
To use the package, add the commands
\begin{center}
\begin{tabular}{l}
|\input{childdoc.def}|\\
|\childdocmain{}|\\
\end{tabular}
\end{center}
at the very top of the main \LaTeX{} file,
in particular \emph{before} the |\documentclass| statement!
The argument of |\childdocmain| should be left empty
(but it must be present).

%%%%%%%%%%%%%%%%%%%%%%%%%%%%%%%%%%%%%%%%
\DescribeMacro{\childdocof}
Furthermore, add the commands
\begin{center}
\begin{tabular}{l}
|\input{childdoc.def}|\\
|\childdocof{|\textit{main}|}|\\
\end{tabular}
\end{center}
at the top of every child file \textit{child}
which is included by |\include{|\textit{child}|}|
from within the main file
(or at least for those files to be compiled individually).
The argument \textit{main} must be the filename of the main file.

There are a couple of
considerations in setting up the main and child documents:

%%%%%%%%%%%%%%%%%%%%%%%%%%%%%%%%%%%%%%%%
\paragraph{Restrictions.}

Please note the following restrictions:
\begin{itemize}
\item
|\childdocmain| must be called with one argument \textit{main}
to ensure compatibility with earlier version of the package.
It must either be empty (|\childdocmain{}|)
or precisely match the filename of the main file in which it is specified.
See \secref{sec:detection} for further information.
\item
The filename \textit{main} must be specified without the |.tex| extension.
\item
The filename \textit{main} is case sensitive
(even in case-insensitive file systems)
due to internal string comparison.
\item
The argument \textit{main} should be fully expanded, it cannot be a macro.
\item
Subdirectories and special characters should be avoided in filenames.
\item
The command |\childdocmain{|\textit{main}|}| must be followed by a whitespace.
It should not be followed immediately by another command
or by a comment mark `|%|'.
This is because the \TeX{} parser reads the token immediately following
the argument of |\childdocmain| and puts it
at the beginning of every child section;
however, a white\-space is ignored.
\end{itemize}

%%%%%%%%%%%%%%%%%%%%%%%%%%%%%%%%%%%%%%%%
\paragraph{Content of Main File.}

It is advisable to place all content in the child files included by |\include|.
Any output contained in the main file will appear in all child documents
unless suppressed manually;
it cannot be suppressed automatically by the |\includeonly| directive
and thus should normally be avoided.
A method to include some content in the main file
by means of conditional processing is described in \secref{sec:conditional}.

%%%%%%%%%%%%%%%%%%%%%%%%%%%%%%%%%%%%%%%%
\paragraph{Page Numbering.}

When only a part of the document is compiled,
the appropriate numbering of pages
(as well as other status parameters)
is determined from the |.aux| files.
The latter contain information from previous passes.
However this information needs to propagate through
all intermediate child documents.
Therefore the page numbering in child documents may well
be inconsistent until the complete document is compiled at least once.

A useful (if unconventional) way to always ensure a consistent
page numbering is to restart the numbering in each child document
and denote the pages by `\textit{child}|.|\textit{page}'
where \textit{child} represents the chapter/section number of the child file.
This can be achieved by the command
|\numberwithin{page}{|\textit{child}|}|
of the \textsf{amsmath} package
where \textit{child} can be |chapter| or |section|
depending on the chosen structuring.
Alternatively, one can modify the macro |\thepage| appropriately
and reset the counter |page| at the start of each child file.

%%%%%%%%%%%%%%%%%%%%%%%%%%%%%%%%%%%%%%%%%%%%%%%%%%%%%%%%%%%%%%%%%%%%%%%%%%%%%%%%
\subsection{Conditional Processing}
\label{sec:conditional}

The package provides a mechanism to compile different versions
of a document. To customise the versions further some conditional processing
can come in handy to distinguish which version is being compiled.
The package provides two macros to describe the compilation context:

%%%%%%%%%%%%%%%%%%%%%%%%%%%%%%%%%%%%%%%%
\DescribeMacro{\ifchilddoc}
The conditional |\ifchilddoc| distinguishes between the compilation of
child documents and the main document:
%
\begin{center}
|\ifchilddoc |\textit{child-code}| |[|\||else |\textit{main-code}]| \||fi|
\end{center}

%%%%%%%%%%%%%%%%%%%%%%%%%%%%%%%%%%%%%%%%
\DescribeMacro{\childdocname}
\DescribeMacro{\childdocjob}
The macro |\childdocname| contains the filename (without extension)
of the main or child file being processed.
Note that |\childdocjob| will always contain the name of the main file.

%%%%%%%%%%%%%%%%%%%%%%%%%%%%%%%%%%%%%%%%
\paragraph{Title Page.}

Conditional processing can be used to include a title or banner page
in the main document when proper precautions are taken.
Importantly, the code in the main file should ensure that the page counter
(as well as other status parameters which are stored in the |.aux| files)
takes the same value after the conditional processing.
Otherwise the page numbers may take divergent values
depending on which part is compiled.

For example, a title page could be declared by:
%
\begin{center}
\begin{tabular}{l}
|\ifchilddoc\||else|\\
|\addtocounter{page}{-1}|\\
\textit{code for title page}\\
|\newpage|\\
|\||fi|
\end{tabular}
\end{center}
%
A banner page for the child documents can be generated by:
%
\begin{center}
\begin{tabular}{l}
|\ifchilddoc|\\
|\addtocounter{page}{-1}|\\
\textit{code for banner page}\\
|\newpage|\\
|\||fi|
\end{tabular}
\end{center}
%
Here one could write a message such as:
\begin{center}
|This is the part \childdocname{} of \childdocjob{}.|
\end{center}

%%%%%%%%%%%%%%%%%%%%%%%%%%%%%%%%%%%%%%%%%%%%%%%%%%%%%%%%%%%%%%%%%%%%%%%%%%%%%%%%
\subsection{Flags}
\label{sec:flags}

The package makes it easy to generate different versions
of the main or child documents.
To this end compilation flags can be defined
and assigned different default values.
They will be particularly useful in conjunction
with the forwarding mechanism described in \secref{sec:forward}.

For example, it may be useful to have a flag |\version|
which can be set to |draft| or |final|.
The document source will contain some conditional code
depending on the value of |\version|.
Suppose further, the flag should default to |final| for the main file
and to |draft| for child files
which is a natural assignment for editing the document.
This is achieved by placing the following code
in the preamble of the main document
(below the |\childdocmain| directive):
%
\begin{center}
\begin{tabular}{l}
|\ifchilddoc|\\
|\providecommand{\version}{draft}|\\
|\||else|\\
|\providecommand{\version}{final}|\\
|\||fi|
\end{tabular}
\end{center}
%
The definition by |\providecommand| makes sure
that previous definitions are not overwritten.
Further statements |\providecommand{\version}{...}|
can thus be added before the above code to override it.

For the main file, one might add a line
(between |\childdocmain| and the above block)
%
\begin{center}
|%\ifchilddoc\||else\providecommand{\version}{draft}\||fi|
\end{center}
%
which can be uncommented to produce a draft version.
Likewise one can add a line to the very top of a child file
(above the |\childdocof{|\textit{main}|}| directive)
%
\begin{center}
|%\providecommand{\version}{final}|
\end{center}
%
which can be uncommented to produce the final version of this child document.

%%%%%%%%%%%%%%%%%%%%%%%%%%%%%%%%%%%%%%%%%%%%%%%%%%%%%%%%%%%%%%%%%%%%%%%%%%%%%%%%
\subsection{Forwarding}
\label{sec:forward}

Different versions of the main or child documents
using compilation flags as described in \secref{sec:flags}
can be (permanently) stored in different files
for convenient compilation, viewing and distribution.
To this end, the package defines a command
to pass on compilation to a different file:

%%%%%%%%%%%%%%%%%%%%%%%%%%%%%%%%%%%%%%%%
\DescribeMacro{\childdocforward}
The command |\childdocforward| redirects processing to
another source file:
%
\begin{center}
\begin{tabular}{l}
|\input{childdoc.def}|\\
|\childdocforward[|\textit{main}|]{|\textit{dest}|}|\\
\end{tabular}
\end{center}
%
The argument \textit{dest} is the destination file
(without extension).
It should be the main file or one of the child files.
Note that further \textsf{childdoc} directives
such as |\childdocof| and |\childdocforward|
in the indicated file will be processed in this form.
The optional argument \textit{main}
passes on directly to the main file \textit{main}
while pretending to compile the child \textit{dest}.
This form behaves as if \textit{dest}
issues |\childdocof{|\textit{main}|}| right away,
and no further \textsf{childdoc} directives will be processed.

%%%%%%%%%%%%%%%%%%%%%%%%%%%%%%%%%%%%%%%%
\DescribeMacro{\...prefix}
In the alternative form |\childdocforwardprefix|,
%
\begin{center}
\begin{tabular}{l}
|\input{childdoc.def}|\\
|\childdocforwardprefix[|\textit{main}|]{|\textit{prefix}|}{|\textit{dest}|}|
\end{tabular}
\end{center}
%
the destination file is determined by a pattern
depending on the current file:
To make this work, the current file must be called
`{\textit{prefix}\hspace{0.2em}\textit{suffix}}'
with \textit{prefix} matching precisely the argument.
Processing is then passed on to the file
`{\textit{dest}\hspace{0.2em}\textit{suffix}}'.
Surely, the same effect is achieved by
directly specifying the
argument `{\textit{dest}\hspace{0.2em}\textit{suffix}}'
in the first form.
However, that requires to set up a different file
for each child. With the alternative form of the command
all these files can have exactly the same content
which simplifies setting them up and maintaining them.

For example, the following file |draft.tex|
with a compilation flag |\version| as described in \secref{sec:flags}
compiles the main document as a draft:
%
\begin{center}
\begin{tabular}{l}
|\def\version{draft}|\\
|\input{childdoc.def}|\\
|\childdocforward{|\textit{main}|}|
\end{tabular}
\end{center}
%
Likewise, the following files |final|\textit{nn}|.tex|
compile the final version of the child document
|child|\textit{nn}|.tex|:
%
\begin{center}
\begin{tabular}{l}
|\def\version{final}|\\
|\input{childdoc.def}|\\
|\childdocforwardprefix{final}{child}|
\end{tabular}
\end{center}
%

Note that when several versions of a main file and/or of each child file
are to be generated, it may be convenient to set up a |Makefile| or
shell script to automatise the process.

%%%%%%%%%%%%%%%%%%%%%%%%%%%%%%%%%%%%%%%%%%%%%%%%%%%%%%%%%%%%%%%%%%%%%%%%%%%%%%%%
\subsection{Command Line Processing}
\label{sec:commandline}

The effect of redirection files can also be achieved by invoking
the \LaTeX{} compiler with a more elaborate command line.
Most conveniently this should be done as part
of a shell script or a |Makefile|.

When using \textsf{childdoc} in the main file, the following
command lines effectively perform a redirection
(note that depending on the shell being used,
backslashes may have to be doubled: `|\|' $\to$ `|\\|'):
%
\begin{center}
|... -jobname "|\textit{target}|" |\\|"|[\textit{flags}]%
|\input{childdoc.def}\childdocforward[|\textit{main}|]{|\textit{dest}|}"|
\end{center}
%
Here \textit{target} is the name of the output file,
\textit{main} is the name of the main file
and \textit{dest} is the name of the main or child file to be processed
(all filenames without extensions).
The optional argument \textit{main} can be omitted
if \textit{main} matches \textit{dest}.
Optionally, compilation \textit{flags} can be defined via |\def| commands.
This command line makes the \TeX{} engine believe
it is compiling the file \textit{target}
whose content is specified as the latter parameter.
The provided code then forwards the processing to
\textit{main} or \textit{dest} as described in \secref{sec:forward}.

%%%%%%%%%%%%%%%%%%%%%%%%%%%%%%%%%%%%%%%%%%%%%%%%%%%%%%%%%%%%%%%%%%%%%%%%%%%%%%%%
\subsection{Include by Input}
\label{sec:input}

Including child documents by |\include| has some restrictions by design.
Most notably, the content of a child document always occupies
its own set of pages; pages cannot be shared between child documents.
Usually, this behaviour makes perfect sense
because each child document contain an essential part of the document.
However, in some situations it may be desirable to compose
a document from a collection of parts
without having mandatory page breaks between then.
For this case, the package
provides a mechanism to include parts
by |\input| which can also be processed individually.
However, by construction this mechanism
requires manual handling of the content to be output.

%%%%%%%%%%%%%%%%%%%%%%%%%%%%%%%%%%%%%%%%
\DescribeMacro{\ifchilddocmanual}
The main file should be prepared as usual, see \secref{sec:include}.
However, the document body must make a distinction
between processing of an individual part and of the main document, e.g.:
%
\begin{center}
\begin{tabular}{l}
|\ifchilddocmanual|\\
|\input{\childdocname}|\\
|\||else|\\
\textit{document body with }|\input{|\textit{part}|}|\\
|\||fi|
\end{tabular}
\end{center}
%
The conditional |\ifchilddocmanual| is true whenever
a part to be included by |\input| is being compiled,
and the name of the part is stored in |\childdocname|.

%%%%%%%%%%%%%%%%%%%%%%%%%%%%%%%%%%%%%%%%
\DescribeMacro{\childdocby}
Each part to be included by |\input| should start with:
%
\begin{center}
\begin{tabular}{l}
|\input{childdoc.def}|\\
|\childdocby{|\textit{main}|}|\\
\end{tabular}
\end{center}
%
The directive |\childdocby| is similar to |\childdocof|
described in \secref{sec:include},
but the subsequent selection of content must be done manually.
To that end, both |\ifchilddoc| and |\ifchilddocmanual|
will be true upon processing of a part,
and the name of the part is stored in |\childdocname|.
Note that |\jobname| will be set to the filename of the current part
so that each part receives an individual |.aux| file
that does not interfere with the |.aux| file(s) of the main document.
This behaviour can be altered by the alternative form
|\childdocby[*]{|\textit{main}|}| (with a non-empty optional argument)
which uses the |.aux| file of the main document
by setting |\jobname| to \textit{main}.

%%%%%%%%%%%%%%%%%%%%%%%%%%%%%%%%%%%%%%%%%%%%%%%%%%%%%%%%%%%%%%%%%%%%%%%%%%%%%%%%
\subsection{Driver Development}
\label{sec:driver}

The \textsf{childdoc} mechanism can also be use for the development
of definition files such as \LaTeX{} styles or classes.
This case differs from the above setup with multiple parts
included by |\include| in that no |\includeonly| should be invoked.
This can be achieved by starting the include file
(before |\ProvidesPackage|) with:
%
\begin{center}
\begin{tabular}{l}
|\input{childdoc.def}|\\
|\childdocforward{|\textit{main}|}|\\
\end{tabular}
\end{center}
%
or alternatively with:
%
\begin{center}
\begin{tabular}{l}
|\input{childdoc.def}|\\
|\childdocby{|\textit{main}|}|\\
\end{tabular}
\end{center}
%
Both forms have slightly different effects as described above.
The main file is prepared as usual, see \secref{sec:include}.

%%%%%%%%%%%%%%%%%%%%%%%%%%%%%%%%%%%%%%%%%%%%%%%%%%%%%%%%%%%%%%%%%%%%%%%%%%%%%%%%
\subsection{Legacy Detection}
\label{sec:detection}

The directive |\childdocmain| in the main file can detect
whether the complete document or merely a child is to be compiled
even without using the directive |\childdocof|.
This method is deprecated because it is less robust
and there is no compelling reason to use it;
it is merely provided for backward compatibility
and it may be removed in future versions.

If the detection mechanism is to be used,
it is mandatory to correctly specify
the filename of the main file as the argument of |\childdocmain|:
%
\begin{center}
\begin{tabular}{l}
|\input{childdoc.def}|\\
|\childdocmain{|\textit{main}|}|\\
\end{tabular}
\end{center}
%
If |\jobname| does not match the argument \textit{main} of |\childdocmain|,
it is assumed that |\jobname| points to the child file to be compiled.
When using |\childdocmain| with the main file specified as argument,
it suffices to start a child file
with just |\input{|\textit{main}|}|
without loading of the package and using |\childdocof|.
If instead all processing is done
with the appropriate \textsf{childdoc} directives,
the argument of \textit{main} of |\childdocmain| can be empty.

An alternative version of the command line processing described
in \secref{sec:commandline} using the detection mechanism reads:
%
\begin{center}
|... -jobname "|\textit{target}|" "|[\textit{flags}]%
[|\def\jobname{|\textit{dest}|}|]|\input{|\textit{main}|}"|
\end{center}

%%%%%%%%%%%%%%%%%%%%%%%%%%%%%%%%%%%%%%%%%%%%%%%%%%%%%%%%%%%%%%%%%%%%%%%%%%%%%%%%
\subsection{Manual Code}
\label{sec:manual}

In case one cannot be certain whether the definitions file |childdoc.def|
is installed on the target \TeX{} distribution
and one prefers not to ship it,
it is conceivable to paste a few relevant commands into the sources.

To that end, drop all statements |\input{childdoc.def}|
and perform the replacements as outlined below.
Instead of |\childdocmain{|\textit{main}|}| add the following code
to the top of the main file:
%
\begin{center}
\begin{tabular}{l}
|\||ifdefined\childdocname\endinput\||fi\newif\ifchilddoc|\\
|\edef\childdocname{\scantokens\expandafter{\jobname\noexpand}}|\\
|\def\childdocmain{|\textit{main}|}\||ifx\childdocmain\childdocname\||else|\\
|\childdoctrue\includeonly{\childdocname}\let\jobname\childdocmain\||fi|\\
\end{tabular}
\end{center}
%
Instead of |\childdocof{|\textit{main}|}| just include the main file
at the top of each child file:
%
\begin{center}
|\input{|\textit{main}|}|
\end{center}
%
A simple redirection |\childdocforward{|\textit{dest}|}| is achieved by:
%
\begin{center}
|\def\jobname{|\textit{dest}|}\input{\jobname}|
\end{center}
%
The redirection with prefix
|\childdocforwardprefix[|\textit{prefix}|]{|\textit{dest}|}|
is accomplished by:
%
\begin{center}
\begin{tabular}{l}
|{\edef\jobname{\scantokens\expandafter{\jobname\noexpand}}|\\
|\def\redirectjob |\textit{prefix}|#1~~~{\gdef\jobname{|\textit{dest}|#1}}|\\
|\expandafter\redirectjob\jobname~~~}\input{\jobname}|
\end{tabular}
\end{center}

In an alternative approach,
child documents can be compiled by a specific command line
without additional code or specific definitions:
%
\begin{center}
|... -jobname "|\textit{target}|" "|[\textit{flags}]%
|\includeonly{|\textit{dest}|}\input{|\textit{main}|}"|
\end{center}
%

%%%%%%%%%%%%%%%%%%%%%%%%%%%%%%%%%%%%%%%%%%%%%%%%%%%%%%%%%%%%%%%%%%%%%%%%%%%%%%%%
%%%%%%%%%%%%%%%%%%%%%%%%%%%%%%%%%%%%%%%%%%%%%%%%%%%%%%%%%%%%%%%%%%%%%%%%%%%%%%%%
\section{Information}

%%%%%%%%%%%%%%%%%%%%%%%%%%%%%%%%%%%%%%%%%%%%%%%%%%%%%%%%%%%%%%%%%%%%%%%%%%%%%%%%
\subsection{Copyright}

Copyright \copyright{} 2017--2018 Niklas Beisert

This work may be distributed and/or modified under the
conditions of the \LaTeX{} Project Public License, either version 1.3
of this license or (at your option) any later version.
The latest version of this license is in
  \url{http://www.latex-project.org/lppl.txt}
and version 1.3 or later is part of all distributions of \LaTeX{}
version 2005/12/01 or later.

This work has the LPPL maintenance status `maintained'.

The Current Maintainer of this work is Niklas Beisert.

This work consists of the files |README.txt|, |childdoc.ins| and |childdoc.dtx|
as well as the derived files |childdoc.def|, |cdocsamp.tex|
with |cdocsch1.tex|, |cdocsch2.tex|, |cdocspt3.tex|, |cdocspt4.tex|,
|cdocsdrf.tex|, |cdocsfn1.tex|, |cdocsfn2.tex|
as well as |childdoc.pdf|.

%%%%%%%%%%%%%%%%%%%%%%%%%%%%%%%%%%%%%%%%%%%%%%%%%%%%%%%%%%%%%%%%%%%%%%%%%%%%%%%%
\subsection{Files and Installation}

The package consists of the files:
%
\begin{center}
\begin{tabular}{ll}
    |README.txt|   & readme file \\
    |childdoc.ins| & installation file \\
    |childdoc.dtx| & source file \\
    |childdoc.def| & definition file \\
    |cdocsamp.tex| & sample main file \\
    |cdocsch1.tex| & sample include file \\
    |cdocsch2.tex| & sample include file \\
    |cdocspt3.tex| & sample part file \\
    |cdocspt4.tex| & sample part file \\
    |cdocsdrf.tex| & sample redirection file \\
    |cdocsfn1.tex| & sample redirection file \\
    |cdocsfn2.tex| & sample redirection file \\
    |childdoc.pdf| & manual
\end{tabular}
\end{center}
%
The distribution consists of the files
|README.txt|, |childdoc.ins| and |childdoc.dtx|.
%
\begin{itemize}
\item
Run (pdf)\LaTeX{} on |childdoc.dtx|
to compile the manual |childdoc.pdf| (this file).
\item
Run \LaTeX{} on |childdoc.ins| to create the definitions file |childdoc.def|
and the sample |cdocsamp.tex| with include files
|cdocsch1.tex|, |cdocsch2.tex|, |cdocspt3.tex|, |cdocspt4.tex|,
|cdocsdrf.tex|, |cdocsfn1.tex|, |cdocsfn2.tex|.
Then copy the file |childdoc.def| to an appropriate directory of your \LaTeX{}
distribution, e.g.\ \textit{texmf-root}|/tex/latex/childdoc|.
\end{itemize}

%%%%%%%%%%%%%%%%%%%%%%%%%%%%%%%%%%%%%%%%%%%%%%%%%%%%%%%%%%%%%%%%%%%%%%%%%%%%%%%%
\subsection{Related CTAN Packages}

There are several other packages which offer a similar functionality:
%
\begin{itemize}
\item
The packages
\href{http://ctan.org/pkg/docmute}{\textsf{docmute}},
\href{http://ctan.org/pkg/includex}{\textsf{includex}} and
\href{http://ctan.org/pkg/standalone}{\textsf{standalone}}
provide commands to include only the document body of
a child file thus allowing both files to be compiled individually.
\item
The packages \href{http://ctan.org/pkg/subdocs}{\textsf{subdocs}}
and \href{http://ctan.org/pkg/subfiles}{\textsf{subfiles}}
provide structures in which the main and child documents can be
encapsulated and allowing them to be compiled individually.
The inclusion mechanism is different from the conventional |\include|.
\item
The package \href{http://ctan.org/pkg/combine}{\textsf{combine}}
is an elaborate solution to combine several documents into one.
\end{itemize}
%
See also the CTAN topic \href{http://ctan.org/topic/subdocs}{\textsf{subdocs}}
for further related packages.
The present package differs from the above solutions in that
a document structure constructed with the conventional |\include| mechanism
just needs two extra commands at the top of every file
such that all constituent files can be compiled individually.

%%%%%%%%%%%%%%%%%%%%%%%%%%%%%%%%%%%%%%%%%%%%%%%%%%%%%%%%%%%%%%%%%%%%%%%%%%%%%%%%
%\subsection{Feature Suggestions}
%
%The following is a list of features which may be useful for future
%versions of this package:
%%
%\begin{itemize}
%\item
%\ldots
%\end{itemize}

%%%%%%%%%%%%%%%%%%%%%%%%%%%%%%%%%%%%%%%%%%%%%%%%%%%%%%%%%%%%%%%%%%%%%%%%%%%%%%%%
\subsection{Revision History}

%%%%%%%%%%%%%%%%%%%%%%%%%%%%%%%%%%%%%%%%
\paragraph{v2.0:} 2018/12/30

\begin{itemize}
\item
immediate forward processing
\item
added |\childdocby| mechanism
\item
manual restructured
\end{itemize}

%%%%%%%%%%%%%%%%%%%%%%%%%%%%%%%%%%%%%%%%
\paragraph{v1.6:} 2018/01/17

\begin{itemize}
\item
application for development of include files
\item
corrections to manual
\end{itemize}

%%%%%%%%%%%%%%%%%%%%%%%%%%%%%%%%%%%%%%%%
\paragraph{v1.5:} 2017/05/21

\begin{itemize}
\item
more complete structuring introduced
\item
|\childdocof| introduced
\item
|\childdoc| renamed to |\childdocmain|
\item
|\childredirect| renamed to |\childdocforward| and |\childdocforwardprefix|
and functionality expanded
\end{itemize}

%%%%%%%%%%%%%%%%%%%%%%%%%%%%%%%%%%%%%%%%
\paragraph{v1.0:} 2017/04/27

\begin{itemize}
\item
manual and install package
\item
first version published on CTAN
\end{itemize}

%%%%%%%%%%%%%%%%%%%%%%%%%%%%%%%%%%%%%%%%
\paragraph{v0.6:} 2017/04/26

\begin{itemize}
\item
redirection mechanism added
\end{itemize}

%%%%%%%%%%%%%%%%%%%%%%%%%%%%%%%%%%%%%%%%
\paragraph{v0.5:} 2017/04/26

\begin{itemize}
\item
functionality in definition file
\end{itemize}


%%%%%%%%%%%%%%%%%%%%%%%%%%%%%%%%%%%%%%%%%%%%%%%%%%%%%%%%%%%%%%%%%%%%%%%%%%%%%%%%
%%%%%%%%%%%%%%%%%%%%%%%%%%%%%%%%%%%%%%%%%%%%%%%%%%%%%%%%%%%%%%%%%%%%%%%%%%%%%%%%
%%%%%%%%%%%%%%%%%%%%%%%%%%%%%%%%%%%%%%%%%%%%%%%%%%%%%%%%%%%%%%%%%%%%%%%%%%%%%%%%
\appendix

\settowidth\MacroIndent{\rmfamily\scriptsize 000\ }

 \DocInput{childdoc.dtx}

\end{document}
%</driver>
% \fi
%
% %%%%%%%%%%%%%%%%%%%%%%%%%%%%%%%%%%%%%%%%%%%%%%%%%%%%%%%%%%%%%%%%%%%%%%%%%%%%%%
% %%%%%%%%%%%%%%%%%%%%%%%%%%%%%%%%%%%%%%%%%%%%%%%%%%%%%%%%%%%%%%%%%%%%%%%%%%%%%%
% \section{Sample}
%\iffalse
%<*samplemain>
%\fi
%
% The following presents a sample document
% with two chapters, two parts, a title page,
% a compile flag as well as three forwarding files to set the flag.
% It consists of eight |.tex| files:
% \begin{center}
% \begin{tabular}{ll}
% |cdocsamp.tex|&main file\\
% |cdocsch1.tex|&include file for chapter 1\\
% |cdocsch2.tex|&include file for chapter 2\\
% |cdocspt3.tex|&include file for part 3\\
% |cdocspt4.tex|&include file for part 4\\
% |cdocsdrf.tex|&forwarding file for main file in draft mode\\
% |cdocsfi1.tex|&forwarding file for final version of chapter 1\\
% |cdocsfi2.tex|&forwarding file for final version of chapter 2\\
% \end{tabular}
% \end{center}
% Each of the eight files can be compiled directly by the \LaTeX{} compiler.
%
% %%%%%%%%%%%%%%%%%%%%%%%%%%%%%%%%%%%%%%
% \paragraph{Main File.}
%
% The main file is called |cdocsamp.tex|.
%
% Load the \textsf{childdoc} definitions and
% declare the filename for the main document:
%    \begin{macrocode}
\input{childdoc.def}
\childdocmain{}
%    \end{macrocode}

% Optional override for |\version| flag:
%    \begin{macrocode}
%%\ifchilddoc\else\providecommand{\version}{draft}\fi
%    \end{macrocode}

% Define the default values for the |\version| flag
% (|final| for the main file and |draft| for childs):
%    \begin{macrocode}
\ifchilddoc
\providecommand{\version}{draft}
\else
\providecommand{\version}{final}
\fi
%    \end{macrocode}

% Load the standard document class:
%    \begin{macrocode}
\documentclass[12pt]{article}
%    \end{macrocode}

% Start the document body:
%    \begin{macrocode}
\begin{document}
%    \end{macrocode}

% Declare a title page.
% Print title, part of document being processed and version flag:
%    \begin{macrocode}
\addtocounter{page}{-1}
\begin{center}
{\LARGE\bfseries{}childdoc example\par}
\vspace{1cm}
\ifchilddoc
\ifchilddocmanual part\else chapter\fi:
`\childdocname' of `\childdocjob'\par
\else
main document: `\childdocjob'\par
\fi
version: \version\par
\end{center}
\newpage
%    \end{macrocode}

% Manually include selected file,
% otherwise process as usual:
%    \begin{macrocode}
\ifchilddocmanual
\section*{part `\childdocname'}
\input{\childdocname}
\else
%    \end{macrocode}

% Include the two chapters:
%    \begin{macrocode}
\include{cdocsch1}
\include{cdocsch2}
%    \end{macrocode}

% Include the two parts unless only chapters should be displayed:
%    \begin{macrocode}
\ifchilddoc\else
\section{part three}
\input{cdocspt3}
\section{part four}
\input{cdocspt4}
\fi
%    \end{macrocode}

% Process as usual until here:
%    \begin{macrocode}
\fi
%    \end{macrocode}

% End of document body:
%    \begin{macrocode}
\end{document}
%    \end{macrocode}
%\iffalse
%</samplemain>
%\fi
%
% %%%%%%%%%%%%%%%%%%%%%%%%%%%%%%%%%%%%%%
% \paragraph{Chapter Include Files.}
%
% The include files are called |cdocsch1.tex| and |cdocsch2.tex|.
%
%\iffalse
%<*samplechap1|samplechap2>
%\fi

% Optional override for |\version| flag:
%    \begin{macrocode}
%%\providecommand{\version}{final}
%    \end{macrocode}

% Include the main document:
%    \begin{macrocode}
\input{childdoc.def}
\childdocof{cdocsamp}
%    \end{macrocode}

%\iffalse
%</samplechap1|samplechap2>
%\fi
%
%\iffalse
%<*samplechap1>
%\fi
% Some text for chapter 1:
%    \begin{macrocode}
\section{one}
some text in chapter one
%    \end{macrocode}

%\iffalse
%</samplechap1>
%\fi
% Some text for chapter 2:
%\iffalse
%<*samplechap2>
%\fi
%    \begin{macrocode}
\section{two}
more text in chapter two
%    \end{macrocode}

%\iffalse
%</samplechap2>
%\fi
%
% %%%%%%%%%%%%%%%%%%%%%%%%%%%%%%%%%%%%%%
% \paragraph{Part Include Files.}
%
% The include files are called |cdocspt3.tex| and |cdocspt4.tex|.
%
%\iffalse
%<*samplepart3|samplepart4>
%\fi

% Optional override for |\version| flag:
%    \begin{macrocode}
%%\providecommand{\version}{final}
%    \end{macrocode}

% Include the main document:
%    \begin{macrocode}
\input{childdoc.def}
\childdocby{cdocsamp}
%    \end{macrocode}

%\iffalse
%</samplepart3|samplepart4>
%\fi
%
%\iffalse
%<*samplepart3>
%\fi
% Some text for part 3:
%    \begin{macrocode}
some text in part three
%    \end{macrocode}

%\iffalse
%</samplepart3>
%\fi
% Some text for part 4:
%\iffalse
%<*samplepart4>
%\fi
%    \begin{macrocode}
more text in part four
%    \end{macrocode}

%\iffalse
%</samplepart4>
%\fi
%
% %%%%%%%%%%%%%%%%%%%%%%%%%%%%%%%%%%%%%%
% \paragraph{Forwarding for a Complete Draft.}
%
% The following forwarding file |cdocsdrf.tex|
% compiles the main document in draft mode:
%\iffalse
%<*sampledraft>
%\fi
%    \begin{macrocode}
\def\version{draft}
\input{childdoc.def}
\childdocforward{cdocsamp}
%    \end{macrocode}

%\iffalse
%</sampledraft>
%\fi
%
% %%%%%%%%%%%%%%%%%%%%%%%%%%%%%%%%%%%%%%
% \paragraph{Forwarding for Final Version of the Chapters.}
%
% The following forwarding files |cdocsfn1.tex| and |cdocsfn2.tex|
% (with identical content)
% compile the final versions of the child documents
% |cdocsch1.tex| and |cdocsch2.tex|, respectively:
%\iffalse
%<*samplefinal>
%\fi
%    \begin{macrocode}
\def\version{final}
\input{childdoc.def}
\childdocforwardprefix[cdocsamp]{cdocsfn}{cdocsch}
%    \end{macrocode}

%\iffalse
%</samplefinal>
%\fi
%
% %%%%%%%%%%%%%%%%%%%%%%%%%%%%%%%%%%%%%%
% \paragraph{Command Line Processing.}
%
% The following three command lines generate the output files
% |cdocscld|, |cdocscl1| and |cdocscl2|
% which should be identical to
% |cdocsdrf|, |cdocsch1| and |cdocsfn2|, respectively:
% \begin{center}
% \begin{tabular}{l}
% |latex -jobname cdocscld \|\\
% |  "\def\version{draft}\input{childdoc.def}\childdocforward{cdocsamp}"|\\
% |latex -jobname cdocscl1 \|\\
% |  "\input{childdoc.def}\childdocforward[cdocsamp]{cdocsch1}"|\\
% |latex -jobname cdocscl2 \|\\
% |  "\def\version{final}\input{childdoc.def}\childdocforward{cdocsch2}"|
% \end{tabular}
% \end{center}
% Note that the trailing backslash on each first line
% merely continues the input to the second line
% (for convenient cut ant paste).
% Furthermore, the command |latex| can be replaced by any
% of its alternative versions such as |pdflatex|.
%
% %%%%%%%%%%%%%%%%%%%%%%%%%%%%%%%%%%%%%%%%%%%%%%%%%%%%%%%%%%%%%%%%%%%%%%%%%%%%%%
% %%%%%%%%%%%%%%%%%%%%%%%%%%%%%%%%%%%%%%%%%%%%%%%%%%%%%%%%%%%%%%%%%%%%%%%%%%%%%%
% \section{Implementation}
%\iffalse
%<*package>
%\fi
%
% This section describes the definitions file |childdoc.def|.

% The definitions cannot be loaded using |\usepackage| or |\RequirePackage|
% which has a mechanism to prevent loading a style file more than once.
% When loading the definitions by means of |\input|
% multiple instances have to be prevented manually:
%\iffalse
%This code needs to be before the `\ProvidesFile' directive
%which is defined at the beginning of this file.
%Therefore it is also placed there and commented out here.
%</package>
%<*discard>
%\fi
%    \begin{macrocode}
\ifdefined\childdocmain\endinput\fi
%    \end{macrocode}
%\iffalse
%</discard>
%<*package>
%\fi
%
% \macro{\ifchilddoc}
% \macro{\ifchilddocmanual}
% The conditional |\ifchilddoc| tells whether a
% child (true) or main (false) document is being compiled.
% The conditional |\ifchilddocmanual| tells whether
% the |\includeonly| mechanism is used (false) or
% the selection of child files must be performed manually (true).
% The definitions initialise to false:
%    \begin{macrocode}
\newif\ifchilddoc
\newif\ifchilddocmanual
%    \end{macrocode}

% \macro{\childdocname}
% \macro{\childdocjob}
% The macro |\childdocname| stores the name of the main document
% to be compiled. The macro |\childdocjob| stores the name of
% the document on which the \LaTeX{} compiler was originally invoked.
% The content of |\jobname| cannot be compared
% to filenames specified in the source due to different catcodes.
% The following code rescans |\jobname|, stores the result
% in |\childdocname| and saves a copy in |\childdocjob|:
%    \begin{macrocode}
\edef\childdocname{\scantokens\expandafter{\jobname\noexpand}}
\let\childdocjob\childdocname
%    \end{macrocode}

% \macro{\childdocdisable}
% The macro |\childdocdisable| prevents the main file
% from being processed more than once.
% At this stage, the main document command |\childdocmain|
% is assumed to be called once again where it should do nothing.
% Any subsequent call to it should prevent
% a secondary processing of the main document
% It overwrites the forwarding commands
% |\childdocof| and |\childdocforward|
% with empty macros to prevent further inclusions of the main document:
%    \begin{macrocode}
\newcommand{\childdocdisable}
{
  \renewcommand{\childdocmain}[1]{\renewcommand{\childdocmain}[1]{\endinput}}
  \renewcommand{\childdocof}[1]{}
  \renewcommand{\childdocby}[2][]{}
  \renewcommand{\childdocforward}[2][]{}
  \renewcommand{\childdocdisable}{}
}
%    \end{macrocode}

% \macro{\childdocmain}
% The macro |\childdocmain| is to be called at the top of the main file
% with nothing or the main filename (without extension) as argument.
% First, it breaks loops.
% If the argument is not empty and does not match |\childdocname|
% (which is set by the first inclusion of |childdoc.def|),
% |\ifchilddoc| is set to true, |\includeonly| is applied to the child file
% and |\jobname| is set to the main file
% (for proper handling of |.aux| files):
%    \begin{macrocode}
\newcommand{\childdocmain}[1]
{
  \childdocdisable\childdocmain{}
  \if?#1?\else
    \begingroup
      \def\childdoctmp{#1}
      \ifx\childdoctmp\childdocname
        \def\childdoctmp{}
      \else
        \def\childdoctmp
        {
          \childdoctrue
          \includeonly{\childdocname}
          \def\childdocjob{#1}
          \def\jobname{#1}
        }
      \fi
      \expandafter
    \endgroup
    \childdoctmp
  \fi
}
%    \end{macrocode}

% \macro{\childdocof}
% The command |\childdocof| redirects
% compilation to the main file |#1|.
%    \begin{macrocode}
\newcommand{\childdocof}[1]
{
  \childdocdisable
  \childdoctrue
  \includeonly{\childdocname}
  \def\jobname{#1}
  \def\childdocjob{#1}
  \input{#1}
}
%    \end{macrocode}

% \macro{\childdocby}
% The command |\childdocby| ....
%    \begin{macrocode}
\newcommand{\childdocby}[2][]
{
  \childdocdisable
  \childdoctrue
  \childdocmanualtrue
  \if?#1?\else
    \def\jobname{#2}
  \fi
  \def\childdocjob{#2}
  \input{#2}
  \endinput
}
%    \end{macrocode}

% \macro{\childdocforward}
% The command |\childdocforward| redirects
% compilation to the main file or
% (if the optional argument is given) a child file.
% Parameters are set as if the main file
% or a child file starting with |\childdocof| was compiled.
% Then compilation is handed over to the main file:
%    \begin{macrocode}
\newcommand{\childdocforward}[2][]
{
  \begingroup
    \if?#1?
      \def\childdoctmp
      {
        \def\childdocname{#2}
        \def\childdocjob{#2}
        \def\jobname{#2}
        \input{#2}
        \endinput
      }
    \else
      \def\childdoctmp
      {
        \childdocdisable
        \def\childdocname{#2}
        \childdoctrue
        \includeonly{#2}
        \def\childdocjob{#1}
        \def\jobname{#1}
        \input{#1}
        \endinput
      }
    \fi
    \expandafter
  \endgroup
  \childdoctmp
}
%    \end{macrocode}

% \macro{\childdocforwardprefix}
% The command |\childdocforwardprefix| redirects
% compilation to the main or a child file by means of a pattern.
% The prefix |#1| in the current filename is replaced by |#2|
% and the suffix of the current filename is kept
% (it is assumed that the filename does not contain the substring `|~~~|'
% which is used as a delimiter).
% Compilation is handed over to the new file by |\childdocforward|:
%    \begin{macrocode}
\newcommand{\childdocforwardprefix}[3][]
{
  \begingroup
    \def\childdocextract #2##1~~~{\def\childdoctmp{\childdocforward[#1]{#3##1}}}
    \expandafter\childdocextract\childdocname~~~
    \expandafter
  \endgroup
  \childdoctmp
}
%    \end{macrocode}

% \macro{\childdoc}
% The deprecated macro |\childdoc| is a legacy version of |\childdocmain|:
%    \begin{macrocode}
\newcommand{\childdoc}{\childdocmain}
%    \end{macrocode}

% \macro{\childdocredirect}
% The deprecated macro |\childdocredirect| is a legacy version
% of |\childdocforward| and |\childdocforwardprefix|:
%    \begin{macrocode}
\newcommand{\childdocredirect}[2][]
{
  \begingroup
    \if?#1?
      \def\childdoctmp{\childdocforward{#2}}
    \else
      \def\childdoctmp{\childdocforwardprefix{#1}{#2}}
    \fi
    \expandafter
  \endgroup
  \childdoctmp
}
%    \end{macrocode}

%\iffalse
%</package>
%\fi
%
\endinput
\childdocforward[|\textit{main}|]{|\textit{dest}|}"|
\end{center}
%
Here \textit{target} is the name of the output file,
\textit{main} is the name of the main file
and \textit{dest} is the name of the main or child file to be processed
(all filenames without extensions).
The optional argument \textit{main} can be omitted
if \textit{main} matches \textit{dest}.
Optionally, compilation \textit{flags} can be defined via |\def| commands.
This command line makes the \TeX{} engine believe
it is compiling the file \textit{target}
whose content is specified as the latter parameter.
The provided code then forwards the processing to
\textit{main} or \textit{dest} as described in \secref{sec:forward}.

%%%%%%%%%%%%%%%%%%%%%%%%%%%%%%%%%%%%%%%%%%%%%%%%%%%%%%%%%%%%%%%%%%%%%%%%%%%%%%%%
\subsection{Include by Input}
\label{sec:input}

Including child documents by |\include| has some restrictions by design.
Most notably, the content of a child document always occupies
its own set of pages; pages cannot be shared between child documents.
Usually, this behaviour makes perfect sense
because each child document contain an essential part of the document.
However, in some situations it may be desirable to compose
a document from a collection of parts
without having mandatory page breaks between then.
For this case, the package
provides a mechanism to include parts
by |\input| which can also be processed individually.
However, by construction this mechanism
requires manual handling of the content to be output.

%%%%%%%%%%%%%%%%%%%%%%%%%%%%%%%%%%%%%%%%
\DescribeMacro{\ifchilddocmanual}
The main file should be prepared as usual, see \secref{sec:include}.
However, the document body must make a distinction
between processing of an individual part and of the main document, e.g.:
%
\begin{center}
\begin{tabular}{l}
|\ifchilddocmanual|\\
|\input{\childdocname}|\\
|\||else|\\
\textit{document body with }|\input{|\textit{part}|}|\\
|\||fi|
\end{tabular}
\end{center}
%
The conditional |\ifchilddocmanual| is true whenever
a part to be included by |\input| is being compiled,
and the name of the part is stored in |\childdocname|.

%%%%%%%%%%%%%%%%%%%%%%%%%%%%%%%%%%%%%%%%
\DescribeMacro{\childdocby}
Each part to be included by |\input| should start with:
%
\begin{center}
\begin{tabular}{l}
|% \iffalse
%
% childdoc.dtx Copyright (C) 2017-2018 Niklas Beisert
%
% This work may be distributed and/or modified under the
% conditions of the LaTeX Project Public License, either version 1.3
% of this license or (at your option) any later version.
% The latest version of this license is in
%   http://www.latex-project.org/lppl.txt
% and version 1.3 or later is part of all distributions of LaTeX
% version 2005/12/01 or later.
%
% This work has the LPPL maintenance status `maintained'.
%
% The Current Maintainer of this work is Niklas Beisert.
%
% This work consists of the files childdoc.dtx and childdoc.ins
% and the derived files childdoc.def and cdocsamp.tex with
% cdocsch1.tex, cdocsch2.tex, cdocsdrf.tex, cdocsfn1.tex, cdocsfn2.tex.
%
%<package>\ifdefined\childdocmain\endinput\fi
%<package>\ProvidesFile{childdoc.def}[2018/12/30 v2.0 child document driver]
%<samplemain>\ProvidesFile{cdocsamp.tex}[2018/12/30 v2.0 sample for childdoc]
%<*driver>
%\ProvidesFile{childdoc.drv}[2018/12/30 v2.0 childdoc reference manual file]
\PassOptionsToClass{10pt,a4paper}{article}
\documentclass{ltxdoc}

\usepackage[margin=35mm]{geometry}
\usepackage{hyperref}
\usepackage{hyperxmp}
\usepackage[usenames]{color}

\hypersetup{colorlinks=true}
\hypersetup{pdfstartview=FitH}
\hypersetup{pdfpagemode=UseNone}
\hypersetup{pdfsource={}}
\hypersetup{pdflang={en-UK}}
\hypersetup{pdfcopyright={Copyright 2017-2018 Niklas Beisert.
  This work may be distributed and/or modified under the
  conditions of the LaTeX Project Public License, either version 1.3
  of this license or (at your option) any later version.}}
\hypersetup{pdflicenseurl={http://www.latex-project.org/lppl.txt}}
\hypersetup{pdfcontactaddress={ETH Zurich, ITP, HIT K,
  Wolfgang-Pauli-Strasse 27}}
\hypersetup{pdfcontactpostcode={8093}}
\hypersetup{pdfcontactcity={Zurich}}
\hypersetup{pdfcontactcountry={Switzerland}}
\hypersetup{pdfcontactemail={nbeisert@itp.phys.ethz.ch}}
\hypersetup{pdfcontacturl={http://people.phys.ethz.ch/\xmptilde nbeisert/}}

\newcommand{\secref}[1]{\hyperref[#1]{section \ref*{#1}}}

\parskip1ex
\parindent0pt
\let\olditemize\itemize
\def\itemize{\olditemize\parskip0pt}

\begin{document}

\title{The \textsf{childdoc} Package}
\hypersetup{pdftitle={The childdoc Package}}
\author{Niklas Beisert\\[2ex]
  Institut f\"ur Theoretische Physik\\
  Eidgen\"ossische Technische Hochschule Z\"urich\\
  Wolfgang-Pauli-Strasse 27, 8093 Z\"urich, Switzerland\\[1ex]
  \href{mailto:nbeisert@itp.phys.ethz.ch}
  {\texttt{nbeisert@itp.phys.ethz.ch}}}
\hypersetup{pdfauthor={Niklas Beisert}}
\hypersetup{pdfsubject={Manual for the LaTeX2e Package childdoc}}
\date{30 December 2018, \textsf{v2.0}}
\maketitle

\begin{abstract}\noindent
\textsf{childdoc} is a \LaTeXe{} package
that enables the direct compilation
of document sections included by |\include|
to individual files.
\end{abstract}

\begingroup
\parskip0ex
\tableofcontents
\endgroup

%%%%%%%%%%%%%%%%%%%%%%%%%%%%%%%%%%%%%%%%%%%%%%%%%%%%%%%%%%%%%%%%%%%%%%%%%%%%%%%%
%%%%%%%%%%%%%%%%%%%%%%%%%%%%%%%%%%%%%%%%%%%%%%%%%%%%%%%%%%%%%%%%%%%%%%%%%%%%%%%%
\section{Introduction}

\LaTeX{} provides a mechanism to structure a large document (such as a book)
into a main file and several child files (containing the chapters)
using the |\include| command.
This mechanism is beneficial for documents
which span hundreds of pages in order to
make the source file(s) more manageable.
Moreover, compilation can be restricted to
selected child files by means of the |\includeonly| command.
The latter feature can be used to reduce the compilation time while editing
(this was significantly more useful in the earlier days of \LaTeX{})
or to generate a smaller document which is easier to navigate.
Another application of |\includeonly| is to generate
documents consisting of selected parts of the complete document.

However, there are a few drawbacks of the plain |\include| mechanism:
\begin{itemize}
\item
The child files cannot be compiled on their own,
they can only be compiled via the main file.
A naive editing environment
(such as a text editor with an option
to have the current file processed by \LaTeX)
may require one to switch to the main file before compiling;
attempting to compile the child file produces errors.
\item
The main file must be modified (each time)
to adjust the |\includeonly| command
to the present needs. This easily leaves the main file in a messy state.
\item
The generated document will always carry the filename
of the main document. This is inconvenient if
several child files are to be compiled and
to be kept for distribution.
\end{itemize}

The present package provides a simple interface
to make child files individually compilable by \LaTeX{}.
Compiling a child file then has the same effect as compiling
the main file with an |\includeonly| command
to select the appropriate child.
Moreover the generated document will carry the name of the child
rather than the main file.
This resolves all three above issues.

This feature is meant to make the editing of books,
thesis documents and lecture notes somewhat more convenient.
However, the package can also be used efficiently for
composing a series of documents (such as exercise sheets)
which are typically distributed individually.
It then assists the author in generating the individual documents
(potentially in different versions)
as well as a document containing the collected series.
Another application is in developing style files
or other kinds of included material
where compilation of the style file could redirect
to a sample or test file.

%%%%%%%%%%%%%%%%%%%%%%%%%%%%%%%%%%%%%%%%%%%%%%%%%%%%%%%%%%%%%%%%%%%%%%%%%%%%%%%%
%%%%%%%%%%%%%%%%%%%%%%%%%%%%%%%%%%%%%%%%%%%%%%%%%%%%%%%%%%%%%%%%%%%%%%%%%%%%%%%%
\section{Usage}

First of all, the package \textsf{childdoc} is \emph{not} a standard
\LaTeXe{} |.sty| style file! Therefore it needs to be invoked in
a non-standard way.

%%%%%%%%%%%%%%%%%%%%%%%%%%%%%%%%%%%%%%%%%%%%%%%%%%%%%%%%%%%%%%%%%%%%%%%%%%%%%%%%
\subsection{Included Files}
\label{sec:include}

%%%%%%%%%%%%%%%%%%%%%%%%%%%%%%%%%%%%%%%%
\DescribeMacro{\childdocmain}
To use the package, add the commands
\begin{center}
\begin{tabular}{l}
|\input{childdoc.def}|\\
|\childdocmain{}|\\
\end{tabular}
\end{center}
at the very top of the main \LaTeX{} file,
in particular \emph{before} the |\documentclass| statement!
The argument of |\childdocmain| should be left empty
(but it must be present).

%%%%%%%%%%%%%%%%%%%%%%%%%%%%%%%%%%%%%%%%
\DescribeMacro{\childdocof}
Furthermore, add the commands
\begin{center}
\begin{tabular}{l}
|\input{childdoc.def}|\\
|\childdocof{|\textit{main}|}|\\
\end{tabular}
\end{center}
at the top of every child file \textit{child}
which is included by |\include{|\textit{child}|}|
from within the main file
(or at least for those files to be compiled individually).
The argument \textit{main} must be the filename of the main file.

There are a couple of
considerations in setting up the main and child documents:

%%%%%%%%%%%%%%%%%%%%%%%%%%%%%%%%%%%%%%%%
\paragraph{Restrictions.}

Please note the following restrictions:
\begin{itemize}
\item
|\childdocmain| must be called with one argument \textit{main}
to ensure compatibility with earlier version of the package.
It must either be empty (|\childdocmain{}|)
or precisely match the filename of the main file in which it is specified.
See \secref{sec:detection} for further information.
\item
The filename \textit{main} must be specified without the |.tex| extension.
\item
The filename \textit{main} is case sensitive
(even in case-insensitive file systems)
due to internal string comparison.
\item
The argument \textit{main} should be fully expanded, it cannot be a macro.
\item
Subdirectories and special characters should be avoided in filenames.
\item
The command |\childdocmain{|\textit{main}|}| must be followed by a whitespace.
It should not be followed immediately by another command
or by a comment mark `|%|'.
This is because the \TeX{} parser reads the token immediately following
the argument of |\childdocmain| and puts it
at the beginning of every child section;
however, a white\-space is ignored.
\end{itemize}

%%%%%%%%%%%%%%%%%%%%%%%%%%%%%%%%%%%%%%%%
\paragraph{Content of Main File.}

It is advisable to place all content in the child files included by |\include|.
Any output contained in the main file will appear in all child documents
unless suppressed manually;
it cannot be suppressed automatically by the |\includeonly| directive
and thus should normally be avoided.
A method to include some content in the main file
by means of conditional processing is described in \secref{sec:conditional}.

%%%%%%%%%%%%%%%%%%%%%%%%%%%%%%%%%%%%%%%%
\paragraph{Page Numbering.}

When only a part of the document is compiled,
the appropriate numbering of pages
(as well as other status parameters)
is determined from the |.aux| files.
The latter contain information from previous passes.
However this information needs to propagate through
all intermediate child documents.
Therefore the page numbering in child documents may well
be inconsistent until the complete document is compiled at least once.

A useful (if unconventional) way to always ensure a consistent
page numbering is to restart the numbering in each child document
and denote the pages by `\textit{child}|.|\textit{page}'
where \textit{child} represents the chapter/section number of the child file.
This can be achieved by the command
|\numberwithin{page}{|\textit{child}|}|
of the \textsf{amsmath} package
where \textit{child} can be |chapter| or |section|
depending on the chosen structuring.
Alternatively, one can modify the macro |\thepage| appropriately
and reset the counter |page| at the start of each child file.

%%%%%%%%%%%%%%%%%%%%%%%%%%%%%%%%%%%%%%%%%%%%%%%%%%%%%%%%%%%%%%%%%%%%%%%%%%%%%%%%
\subsection{Conditional Processing}
\label{sec:conditional}

The package provides a mechanism to compile different versions
of a document. To customise the versions further some conditional processing
can come in handy to distinguish which version is being compiled.
The package provides two macros to describe the compilation context:

%%%%%%%%%%%%%%%%%%%%%%%%%%%%%%%%%%%%%%%%
\DescribeMacro{\ifchilddoc}
The conditional |\ifchilddoc| distinguishes between the compilation of
child documents and the main document:
%
\begin{center}
|\ifchilddoc |\textit{child-code}| |[|\||else |\textit{main-code}]| \||fi|
\end{center}

%%%%%%%%%%%%%%%%%%%%%%%%%%%%%%%%%%%%%%%%
\DescribeMacro{\childdocname}
\DescribeMacro{\childdocjob}
The macro |\childdocname| contains the filename (without extension)
of the main or child file being processed.
Note that |\childdocjob| will always contain the name of the main file.

%%%%%%%%%%%%%%%%%%%%%%%%%%%%%%%%%%%%%%%%
\paragraph{Title Page.}

Conditional processing can be used to include a title or banner page
in the main document when proper precautions are taken.
Importantly, the code in the main file should ensure that the page counter
(as well as other status parameters which are stored in the |.aux| files)
takes the same value after the conditional processing.
Otherwise the page numbers may take divergent values
depending on which part is compiled.

For example, a title page could be declared by:
%
\begin{center}
\begin{tabular}{l}
|\ifchilddoc\||else|\\
|\addtocounter{page}{-1}|\\
\textit{code for title page}\\
|\newpage|\\
|\||fi|
\end{tabular}
\end{center}
%
A banner page for the child documents can be generated by:
%
\begin{center}
\begin{tabular}{l}
|\ifchilddoc|\\
|\addtocounter{page}{-1}|\\
\textit{code for banner page}\\
|\newpage|\\
|\||fi|
\end{tabular}
\end{center}
%
Here one could write a message such as:
\begin{center}
|This is the part \childdocname{} of \childdocjob{}.|
\end{center}

%%%%%%%%%%%%%%%%%%%%%%%%%%%%%%%%%%%%%%%%%%%%%%%%%%%%%%%%%%%%%%%%%%%%%%%%%%%%%%%%
\subsection{Flags}
\label{sec:flags}

The package makes it easy to generate different versions
of the main or child documents.
To this end compilation flags can be defined
and assigned different default values.
They will be particularly useful in conjunction
with the forwarding mechanism described in \secref{sec:forward}.

For example, it may be useful to have a flag |\version|
which can be set to |draft| or |final|.
The document source will contain some conditional code
depending on the value of |\version|.
Suppose further, the flag should default to |final| for the main file
and to |draft| for child files
which is a natural assignment for editing the document.
This is achieved by placing the following code
in the preamble of the main document
(below the |\childdocmain| directive):
%
\begin{center}
\begin{tabular}{l}
|\ifchilddoc|\\
|\providecommand{\version}{draft}|\\
|\||else|\\
|\providecommand{\version}{final}|\\
|\||fi|
\end{tabular}
\end{center}
%
The definition by |\providecommand| makes sure
that previous definitions are not overwritten.
Further statements |\providecommand{\version}{...}|
can thus be added before the above code to override it.

For the main file, one might add a line
(between |\childdocmain| and the above block)
%
\begin{center}
|%\ifchilddoc\||else\providecommand{\version}{draft}\||fi|
\end{center}
%
which can be uncommented to produce a draft version.
Likewise one can add a line to the very top of a child file
(above the |\childdocof{|\textit{main}|}| directive)
%
\begin{center}
|%\providecommand{\version}{final}|
\end{center}
%
which can be uncommented to produce the final version of this child document.

%%%%%%%%%%%%%%%%%%%%%%%%%%%%%%%%%%%%%%%%%%%%%%%%%%%%%%%%%%%%%%%%%%%%%%%%%%%%%%%%
\subsection{Forwarding}
\label{sec:forward}

Different versions of the main or child documents
using compilation flags as described in \secref{sec:flags}
can be (permanently) stored in different files
for convenient compilation, viewing and distribution.
To this end, the package defines a command
to pass on compilation to a different file:

%%%%%%%%%%%%%%%%%%%%%%%%%%%%%%%%%%%%%%%%
\DescribeMacro{\childdocforward}
The command |\childdocforward| redirects processing to
another source file:
%
\begin{center}
\begin{tabular}{l}
|\input{childdoc.def}|\\
|\childdocforward[|\textit{main}|]{|\textit{dest}|}|\\
\end{tabular}
\end{center}
%
The argument \textit{dest} is the destination file
(without extension).
It should be the main file or one of the child files.
Note that further \textsf{childdoc} directives
such as |\childdocof| and |\childdocforward|
in the indicated file will be processed in this form.
The optional argument \textit{main}
passes on directly to the main file \textit{main}
while pretending to compile the child \textit{dest}.
This form behaves as if \textit{dest}
issues |\childdocof{|\textit{main}|}| right away,
and no further \textsf{childdoc} directives will be processed.

%%%%%%%%%%%%%%%%%%%%%%%%%%%%%%%%%%%%%%%%
\DescribeMacro{\...prefix}
In the alternative form |\childdocforwardprefix|,
%
\begin{center}
\begin{tabular}{l}
|\input{childdoc.def}|\\
|\childdocforwardprefix[|\textit{main}|]{|\textit{prefix}|}{|\textit{dest}|}|
\end{tabular}
\end{center}
%
the destination file is determined by a pattern
depending on the current file:
To make this work, the current file must be called
`{\textit{prefix}\hspace{0.2em}\textit{suffix}}'
with \textit{prefix} matching precisely the argument.
Processing is then passed on to the file
`{\textit{dest}\hspace{0.2em}\textit{suffix}}'.
Surely, the same effect is achieved by
directly specifying the
argument `{\textit{dest}\hspace{0.2em}\textit{suffix}}'
in the first form.
However, that requires to set up a different file
for each child. With the alternative form of the command
all these files can have exactly the same content
which simplifies setting them up and maintaining them.

For example, the following file |draft.tex|
with a compilation flag |\version| as described in \secref{sec:flags}
compiles the main document as a draft:
%
\begin{center}
\begin{tabular}{l}
|\def\version{draft}|\\
|\input{childdoc.def}|\\
|\childdocforward{|\textit{main}|}|
\end{tabular}
\end{center}
%
Likewise, the following files |final|\textit{nn}|.tex|
compile the final version of the child document
|child|\textit{nn}|.tex|:
%
\begin{center}
\begin{tabular}{l}
|\def\version{final}|\\
|\input{childdoc.def}|\\
|\childdocforwardprefix{final}{child}|
\end{tabular}
\end{center}
%

Note that when several versions of a main file and/or of each child file
are to be generated, it may be convenient to set up a |Makefile| or
shell script to automatise the process.

%%%%%%%%%%%%%%%%%%%%%%%%%%%%%%%%%%%%%%%%%%%%%%%%%%%%%%%%%%%%%%%%%%%%%%%%%%%%%%%%
\subsection{Command Line Processing}
\label{sec:commandline}

The effect of redirection files can also be achieved by invoking
the \LaTeX{} compiler with a more elaborate command line.
Most conveniently this should be done as part
of a shell script or a |Makefile|.

When using \textsf{childdoc} in the main file, the following
command lines effectively perform a redirection
(note that depending on the shell being used,
backslashes may have to be doubled: `|\|' $\to$ `|\\|'):
%
\begin{center}
|... -jobname "|\textit{target}|" |\\|"|[\textit{flags}]%
|\input{childdoc.def}\childdocforward[|\textit{main}|]{|\textit{dest}|}"|
\end{center}
%
Here \textit{target} is the name of the output file,
\textit{main} is the name of the main file
and \textit{dest} is the name of the main or child file to be processed
(all filenames without extensions).
The optional argument \textit{main} can be omitted
if \textit{main} matches \textit{dest}.
Optionally, compilation \textit{flags} can be defined via |\def| commands.
This command line makes the \TeX{} engine believe
it is compiling the file \textit{target}
whose content is specified as the latter parameter.
The provided code then forwards the processing to
\textit{main} or \textit{dest} as described in \secref{sec:forward}.

%%%%%%%%%%%%%%%%%%%%%%%%%%%%%%%%%%%%%%%%%%%%%%%%%%%%%%%%%%%%%%%%%%%%%%%%%%%%%%%%
\subsection{Include by Input}
\label{sec:input}

Including child documents by |\include| has some restrictions by design.
Most notably, the content of a child document always occupies
its own set of pages; pages cannot be shared between child documents.
Usually, this behaviour makes perfect sense
because each child document contain an essential part of the document.
However, in some situations it may be desirable to compose
a document from a collection of parts
without having mandatory page breaks between then.
For this case, the package
provides a mechanism to include parts
by |\input| which can also be processed individually.
However, by construction this mechanism
requires manual handling of the content to be output.

%%%%%%%%%%%%%%%%%%%%%%%%%%%%%%%%%%%%%%%%
\DescribeMacro{\ifchilddocmanual}
The main file should be prepared as usual, see \secref{sec:include}.
However, the document body must make a distinction
between processing of an individual part and of the main document, e.g.:
%
\begin{center}
\begin{tabular}{l}
|\ifchilddocmanual|\\
|\input{\childdocname}|\\
|\||else|\\
\textit{document body with }|\input{|\textit{part}|}|\\
|\||fi|
\end{tabular}
\end{center}
%
The conditional |\ifchilddocmanual| is true whenever
a part to be included by |\input| is being compiled,
and the name of the part is stored in |\childdocname|.

%%%%%%%%%%%%%%%%%%%%%%%%%%%%%%%%%%%%%%%%
\DescribeMacro{\childdocby}
Each part to be included by |\input| should start with:
%
\begin{center}
\begin{tabular}{l}
|\input{childdoc.def}|\\
|\childdocby{|\textit{main}|}|\\
\end{tabular}
\end{center}
%
The directive |\childdocby| is similar to |\childdocof|
described in \secref{sec:include},
but the subsequent selection of content must be done manually.
To that end, both |\ifchilddoc| and |\ifchilddocmanual|
will be true upon processing of a part,
and the name of the part is stored in |\childdocname|.
Note that |\jobname| will be set to the filename of the current part
so that each part receives an individual |.aux| file
that does not interfere with the |.aux| file(s) of the main document.
This behaviour can be altered by the alternative form
|\childdocby[*]{|\textit{main}|}| (with a non-empty optional argument)
which uses the |.aux| file of the main document
by setting |\jobname| to \textit{main}.

%%%%%%%%%%%%%%%%%%%%%%%%%%%%%%%%%%%%%%%%%%%%%%%%%%%%%%%%%%%%%%%%%%%%%%%%%%%%%%%%
\subsection{Driver Development}
\label{sec:driver}

The \textsf{childdoc} mechanism can also be use for the development
of definition files such as \LaTeX{} styles or classes.
This case differs from the above setup with multiple parts
included by |\include| in that no |\includeonly| should be invoked.
This can be achieved by starting the include file
(before |\ProvidesPackage|) with:
%
\begin{center}
\begin{tabular}{l}
|\input{childdoc.def}|\\
|\childdocforward{|\textit{main}|}|\\
\end{tabular}
\end{center}
%
or alternatively with:
%
\begin{center}
\begin{tabular}{l}
|\input{childdoc.def}|\\
|\childdocby{|\textit{main}|}|\\
\end{tabular}
\end{center}
%
Both forms have slightly different effects as described above.
The main file is prepared as usual, see \secref{sec:include}.

%%%%%%%%%%%%%%%%%%%%%%%%%%%%%%%%%%%%%%%%%%%%%%%%%%%%%%%%%%%%%%%%%%%%%%%%%%%%%%%%
\subsection{Legacy Detection}
\label{sec:detection}

The directive |\childdocmain| in the main file can detect
whether the complete document or merely a child is to be compiled
even without using the directive |\childdocof|.
This method is deprecated because it is less robust
and there is no compelling reason to use it;
it is merely provided for backward compatibility
and it may be removed in future versions.

If the detection mechanism is to be used,
it is mandatory to correctly specify
the filename of the main file as the argument of |\childdocmain|:
%
\begin{center}
\begin{tabular}{l}
|\input{childdoc.def}|\\
|\childdocmain{|\textit{main}|}|\\
\end{tabular}
\end{center}
%
If |\jobname| does not match the argument \textit{main} of |\childdocmain|,
it is assumed that |\jobname| points to the child file to be compiled.
When using |\childdocmain| with the main file specified as argument,
it suffices to start a child file
with just |\input{|\textit{main}|}|
without loading of the package and using |\childdocof|.
If instead all processing is done
with the appropriate \textsf{childdoc} directives,
the argument of \textit{main} of |\childdocmain| can be empty.

An alternative version of the command line processing described
in \secref{sec:commandline} using the detection mechanism reads:
%
\begin{center}
|... -jobname "|\textit{target}|" "|[\textit{flags}]%
[|\def\jobname{|\textit{dest}|}|]|\input{|\textit{main}|}"|
\end{center}

%%%%%%%%%%%%%%%%%%%%%%%%%%%%%%%%%%%%%%%%%%%%%%%%%%%%%%%%%%%%%%%%%%%%%%%%%%%%%%%%
\subsection{Manual Code}
\label{sec:manual}

In case one cannot be certain whether the definitions file |childdoc.def|
is installed on the target \TeX{} distribution
and one prefers not to ship it,
it is conceivable to paste a few relevant commands into the sources.

To that end, drop all statements |\input{childdoc.def}|
and perform the replacements as outlined below.
Instead of |\childdocmain{|\textit{main}|}| add the following code
to the top of the main file:
%
\begin{center}
\begin{tabular}{l}
|\||ifdefined\childdocname\endinput\||fi\newif\ifchilddoc|\\
|\edef\childdocname{\scantokens\expandafter{\jobname\noexpand}}|\\
|\def\childdocmain{|\textit{main}|}\||ifx\childdocmain\childdocname\||else|\\
|\childdoctrue\includeonly{\childdocname}\let\jobname\childdocmain\||fi|\\
\end{tabular}
\end{center}
%
Instead of |\childdocof{|\textit{main}|}| just include the main file
at the top of each child file:
%
\begin{center}
|\input{|\textit{main}|}|
\end{center}
%
A simple redirection |\childdocforward{|\textit{dest}|}| is achieved by:
%
\begin{center}
|\def\jobname{|\textit{dest}|}\input{\jobname}|
\end{center}
%
The redirection with prefix
|\childdocforwardprefix[|\textit{prefix}|]{|\textit{dest}|}|
is accomplished by:
%
\begin{center}
\begin{tabular}{l}
|{\edef\jobname{\scantokens\expandafter{\jobname\noexpand}}|\\
|\def\redirectjob |\textit{prefix}|#1~~~{\gdef\jobname{|\textit{dest}|#1}}|\\
|\expandafter\redirectjob\jobname~~~}\input{\jobname}|
\end{tabular}
\end{center}

In an alternative approach,
child documents can be compiled by a specific command line
without additional code or specific definitions:
%
\begin{center}
|... -jobname "|\textit{target}|" "|[\textit{flags}]%
|\includeonly{|\textit{dest}|}\input{|\textit{main}|}"|
\end{center}
%

%%%%%%%%%%%%%%%%%%%%%%%%%%%%%%%%%%%%%%%%%%%%%%%%%%%%%%%%%%%%%%%%%%%%%%%%%%%%%%%%
%%%%%%%%%%%%%%%%%%%%%%%%%%%%%%%%%%%%%%%%%%%%%%%%%%%%%%%%%%%%%%%%%%%%%%%%%%%%%%%%
\section{Information}

%%%%%%%%%%%%%%%%%%%%%%%%%%%%%%%%%%%%%%%%%%%%%%%%%%%%%%%%%%%%%%%%%%%%%%%%%%%%%%%%
\subsection{Copyright}

Copyright \copyright{} 2017--2018 Niklas Beisert

This work may be distributed and/or modified under the
conditions of the \LaTeX{} Project Public License, either version 1.3
of this license or (at your option) any later version.
The latest version of this license is in
  \url{http://www.latex-project.org/lppl.txt}
and version 1.3 or later is part of all distributions of \LaTeX{}
version 2005/12/01 or later.

This work has the LPPL maintenance status `maintained'.

The Current Maintainer of this work is Niklas Beisert.

This work consists of the files |README.txt|, |childdoc.ins| and |childdoc.dtx|
as well as the derived files |childdoc.def|, |cdocsamp.tex|
with |cdocsch1.tex|, |cdocsch2.tex|, |cdocspt3.tex|, |cdocspt4.tex|,
|cdocsdrf.tex|, |cdocsfn1.tex|, |cdocsfn2.tex|
as well as |childdoc.pdf|.

%%%%%%%%%%%%%%%%%%%%%%%%%%%%%%%%%%%%%%%%%%%%%%%%%%%%%%%%%%%%%%%%%%%%%%%%%%%%%%%%
\subsection{Files and Installation}

The package consists of the files:
%
\begin{center}
\begin{tabular}{ll}
    |README.txt|   & readme file \\
    |childdoc.ins| & installation file \\
    |childdoc.dtx| & source file \\
    |childdoc.def| & definition file \\
    |cdocsamp.tex| & sample main file \\
    |cdocsch1.tex| & sample include file \\
    |cdocsch2.tex| & sample include file \\
    |cdocspt3.tex| & sample part file \\
    |cdocspt4.tex| & sample part file \\
    |cdocsdrf.tex| & sample redirection file \\
    |cdocsfn1.tex| & sample redirection file \\
    |cdocsfn2.tex| & sample redirection file \\
    |childdoc.pdf| & manual
\end{tabular}
\end{center}
%
The distribution consists of the files
|README.txt|, |childdoc.ins| and |childdoc.dtx|.
%
\begin{itemize}
\item
Run (pdf)\LaTeX{} on |childdoc.dtx|
to compile the manual |childdoc.pdf| (this file).
\item
Run \LaTeX{} on |childdoc.ins| to create the definitions file |childdoc.def|
and the sample |cdocsamp.tex| with include files
|cdocsch1.tex|, |cdocsch2.tex|, |cdocspt3.tex|, |cdocspt4.tex|,
|cdocsdrf.tex|, |cdocsfn1.tex|, |cdocsfn2.tex|.
Then copy the file |childdoc.def| to an appropriate directory of your \LaTeX{}
distribution, e.g.\ \textit{texmf-root}|/tex/latex/childdoc|.
\end{itemize}

%%%%%%%%%%%%%%%%%%%%%%%%%%%%%%%%%%%%%%%%%%%%%%%%%%%%%%%%%%%%%%%%%%%%%%%%%%%%%%%%
\subsection{Related CTAN Packages}

There are several other packages which offer a similar functionality:
%
\begin{itemize}
\item
The packages
\href{http://ctan.org/pkg/docmute}{\textsf{docmute}},
\href{http://ctan.org/pkg/includex}{\textsf{includex}} and
\href{http://ctan.org/pkg/standalone}{\textsf{standalone}}
provide commands to include only the document body of
a child file thus allowing both files to be compiled individually.
\item
The packages \href{http://ctan.org/pkg/subdocs}{\textsf{subdocs}}
and \href{http://ctan.org/pkg/subfiles}{\textsf{subfiles}}
provide structures in which the main and child documents can be
encapsulated and allowing them to be compiled individually.
The inclusion mechanism is different from the conventional |\include|.
\item
The package \href{http://ctan.org/pkg/combine}{\textsf{combine}}
is an elaborate solution to combine several documents into one.
\end{itemize}
%
See also the CTAN topic \href{http://ctan.org/topic/subdocs}{\textsf{subdocs}}
for further related packages.
The present package differs from the above solutions in that
a document structure constructed with the conventional |\include| mechanism
just needs two extra commands at the top of every file
such that all constituent files can be compiled individually.

%%%%%%%%%%%%%%%%%%%%%%%%%%%%%%%%%%%%%%%%%%%%%%%%%%%%%%%%%%%%%%%%%%%%%%%%%%%%%%%%
%\subsection{Feature Suggestions}
%
%The following is a list of features which may be useful for future
%versions of this package:
%%
%\begin{itemize}
%\item
%\ldots
%\end{itemize}

%%%%%%%%%%%%%%%%%%%%%%%%%%%%%%%%%%%%%%%%%%%%%%%%%%%%%%%%%%%%%%%%%%%%%%%%%%%%%%%%
\subsection{Revision History}

%%%%%%%%%%%%%%%%%%%%%%%%%%%%%%%%%%%%%%%%
\paragraph{v2.0:} 2018/12/30

\begin{itemize}
\item
immediate forward processing
\item
added |\childdocby| mechanism
\item
manual restructured
\end{itemize}

%%%%%%%%%%%%%%%%%%%%%%%%%%%%%%%%%%%%%%%%
\paragraph{v1.6:} 2018/01/17

\begin{itemize}
\item
application for development of include files
\item
corrections to manual
\end{itemize}

%%%%%%%%%%%%%%%%%%%%%%%%%%%%%%%%%%%%%%%%
\paragraph{v1.5:} 2017/05/21

\begin{itemize}
\item
more complete structuring introduced
\item
|\childdocof| introduced
\item
|\childdoc| renamed to |\childdocmain|
\item
|\childredirect| renamed to |\childdocforward| and |\childdocforwardprefix|
and functionality expanded
\end{itemize}

%%%%%%%%%%%%%%%%%%%%%%%%%%%%%%%%%%%%%%%%
\paragraph{v1.0:} 2017/04/27

\begin{itemize}
\item
manual and install package
\item
first version published on CTAN
\end{itemize}

%%%%%%%%%%%%%%%%%%%%%%%%%%%%%%%%%%%%%%%%
\paragraph{v0.6:} 2017/04/26

\begin{itemize}
\item
redirection mechanism added
\end{itemize}

%%%%%%%%%%%%%%%%%%%%%%%%%%%%%%%%%%%%%%%%
\paragraph{v0.5:} 2017/04/26

\begin{itemize}
\item
functionality in definition file
\end{itemize}


%%%%%%%%%%%%%%%%%%%%%%%%%%%%%%%%%%%%%%%%%%%%%%%%%%%%%%%%%%%%%%%%%%%%%%%%%%%%%%%%
%%%%%%%%%%%%%%%%%%%%%%%%%%%%%%%%%%%%%%%%%%%%%%%%%%%%%%%%%%%%%%%%%%%%%%%%%%%%%%%%
%%%%%%%%%%%%%%%%%%%%%%%%%%%%%%%%%%%%%%%%%%%%%%%%%%%%%%%%%%%%%%%%%%%%%%%%%%%%%%%%
\appendix

\settowidth\MacroIndent{\rmfamily\scriptsize 000\ }

 \DocInput{childdoc.dtx}

\end{document}
%</driver>
% \fi
%
% %%%%%%%%%%%%%%%%%%%%%%%%%%%%%%%%%%%%%%%%%%%%%%%%%%%%%%%%%%%%%%%%%%%%%%%%%%%%%%
% %%%%%%%%%%%%%%%%%%%%%%%%%%%%%%%%%%%%%%%%%%%%%%%%%%%%%%%%%%%%%%%%%%%%%%%%%%%%%%
% \section{Sample}
%\iffalse
%<*samplemain>
%\fi
%
% The following presents a sample document
% with two chapters, two parts, a title page,
% a compile flag as well as three forwarding files to set the flag.
% It consists of eight |.tex| files:
% \begin{center}
% \begin{tabular}{ll}
% |cdocsamp.tex|&main file\\
% |cdocsch1.tex|&include file for chapter 1\\
% |cdocsch2.tex|&include file for chapter 2\\
% |cdocspt3.tex|&include file for part 3\\
% |cdocspt4.tex|&include file for part 4\\
% |cdocsdrf.tex|&forwarding file for main file in draft mode\\
% |cdocsfi1.tex|&forwarding file for final version of chapter 1\\
% |cdocsfi2.tex|&forwarding file for final version of chapter 2\\
% \end{tabular}
% \end{center}
% Each of the eight files can be compiled directly by the \LaTeX{} compiler.
%
% %%%%%%%%%%%%%%%%%%%%%%%%%%%%%%%%%%%%%%
% \paragraph{Main File.}
%
% The main file is called |cdocsamp.tex|.
%
% Load the \textsf{childdoc} definitions and
% declare the filename for the main document:
%    \begin{macrocode}
\input{childdoc.def}
\childdocmain{}
%    \end{macrocode}

% Optional override for |\version| flag:
%    \begin{macrocode}
%%\ifchilddoc\else\providecommand{\version}{draft}\fi
%    \end{macrocode}

% Define the default values for the |\version| flag
% (|final| for the main file and |draft| for childs):
%    \begin{macrocode}
\ifchilddoc
\providecommand{\version}{draft}
\else
\providecommand{\version}{final}
\fi
%    \end{macrocode}

% Load the standard document class:
%    \begin{macrocode}
\documentclass[12pt]{article}
%    \end{macrocode}

% Start the document body:
%    \begin{macrocode}
\begin{document}
%    \end{macrocode}

% Declare a title page.
% Print title, part of document being processed and version flag:
%    \begin{macrocode}
\addtocounter{page}{-1}
\begin{center}
{\LARGE\bfseries{}childdoc example\par}
\vspace{1cm}
\ifchilddoc
\ifchilddocmanual part\else chapter\fi:
`\childdocname' of `\childdocjob'\par
\else
main document: `\childdocjob'\par
\fi
version: \version\par
\end{center}
\newpage
%    \end{macrocode}

% Manually include selected file,
% otherwise process as usual:
%    \begin{macrocode}
\ifchilddocmanual
\section*{part `\childdocname'}
\input{\childdocname}
\else
%    \end{macrocode}

% Include the two chapters:
%    \begin{macrocode}
\include{cdocsch1}
\include{cdocsch2}
%    \end{macrocode}

% Include the two parts unless only chapters should be displayed:
%    \begin{macrocode}
\ifchilddoc\else
\section{part three}
\input{cdocspt3}
\section{part four}
\input{cdocspt4}
\fi
%    \end{macrocode}

% Process as usual until here:
%    \begin{macrocode}
\fi
%    \end{macrocode}

% End of document body:
%    \begin{macrocode}
\end{document}
%    \end{macrocode}
%\iffalse
%</samplemain>
%\fi
%
% %%%%%%%%%%%%%%%%%%%%%%%%%%%%%%%%%%%%%%
% \paragraph{Chapter Include Files.}
%
% The include files are called |cdocsch1.tex| and |cdocsch2.tex|.
%
%\iffalse
%<*samplechap1|samplechap2>
%\fi

% Optional override for |\version| flag:
%    \begin{macrocode}
%%\providecommand{\version}{final}
%    \end{macrocode}

% Include the main document:
%    \begin{macrocode}
\input{childdoc.def}
\childdocof{cdocsamp}
%    \end{macrocode}

%\iffalse
%</samplechap1|samplechap2>
%\fi
%
%\iffalse
%<*samplechap1>
%\fi
% Some text for chapter 1:
%    \begin{macrocode}
\section{one}
some text in chapter one
%    \end{macrocode}

%\iffalse
%</samplechap1>
%\fi
% Some text for chapter 2:
%\iffalse
%<*samplechap2>
%\fi
%    \begin{macrocode}
\section{two}
more text in chapter two
%    \end{macrocode}

%\iffalse
%</samplechap2>
%\fi
%
% %%%%%%%%%%%%%%%%%%%%%%%%%%%%%%%%%%%%%%
% \paragraph{Part Include Files.}
%
% The include files are called |cdocspt3.tex| and |cdocspt4.tex|.
%
%\iffalse
%<*samplepart3|samplepart4>
%\fi

% Optional override for |\version| flag:
%    \begin{macrocode}
%%\providecommand{\version}{final}
%    \end{macrocode}

% Include the main document:
%    \begin{macrocode}
\input{childdoc.def}
\childdocby{cdocsamp}
%    \end{macrocode}

%\iffalse
%</samplepart3|samplepart4>
%\fi
%
%\iffalse
%<*samplepart3>
%\fi
% Some text for part 3:
%    \begin{macrocode}
some text in part three
%    \end{macrocode}

%\iffalse
%</samplepart3>
%\fi
% Some text for part 4:
%\iffalse
%<*samplepart4>
%\fi
%    \begin{macrocode}
more text in part four
%    \end{macrocode}

%\iffalse
%</samplepart4>
%\fi
%
% %%%%%%%%%%%%%%%%%%%%%%%%%%%%%%%%%%%%%%
% \paragraph{Forwarding for a Complete Draft.}
%
% The following forwarding file |cdocsdrf.tex|
% compiles the main document in draft mode:
%\iffalse
%<*sampledraft>
%\fi
%    \begin{macrocode}
\def\version{draft}
\input{childdoc.def}
\childdocforward{cdocsamp}
%    \end{macrocode}

%\iffalse
%</sampledraft>
%\fi
%
% %%%%%%%%%%%%%%%%%%%%%%%%%%%%%%%%%%%%%%
% \paragraph{Forwarding for Final Version of the Chapters.}
%
% The following forwarding files |cdocsfn1.tex| and |cdocsfn2.tex|
% (with identical content)
% compile the final versions of the child documents
% |cdocsch1.tex| and |cdocsch2.tex|, respectively:
%\iffalse
%<*samplefinal>
%\fi
%    \begin{macrocode}
\def\version{final}
\input{childdoc.def}
\childdocforwardprefix[cdocsamp]{cdocsfn}{cdocsch}
%    \end{macrocode}

%\iffalse
%</samplefinal>
%\fi
%
% %%%%%%%%%%%%%%%%%%%%%%%%%%%%%%%%%%%%%%
% \paragraph{Command Line Processing.}
%
% The following three command lines generate the output files
% |cdocscld|, |cdocscl1| and |cdocscl2|
% which should be identical to
% |cdocsdrf|, |cdocsch1| and |cdocsfn2|, respectively:
% \begin{center}
% \begin{tabular}{l}
% |latex -jobname cdocscld \|\\
% |  "\def\version{draft}\input{childdoc.def}\childdocforward{cdocsamp}"|\\
% |latex -jobname cdocscl1 \|\\
% |  "\input{childdoc.def}\childdocforward[cdocsamp]{cdocsch1}"|\\
% |latex -jobname cdocscl2 \|\\
% |  "\def\version{final}\input{childdoc.def}\childdocforward{cdocsch2}"|
% \end{tabular}
% \end{center}
% Note that the trailing backslash on each first line
% merely continues the input to the second line
% (for convenient cut ant paste).
% Furthermore, the command |latex| can be replaced by any
% of its alternative versions such as |pdflatex|.
%
% %%%%%%%%%%%%%%%%%%%%%%%%%%%%%%%%%%%%%%%%%%%%%%%%%%%%%%%%%%%%%%%%%%%%%%%%%%%%%%
% %%%%%%%%%%%%%%%%%%%%%%%%%%%%%%%%%%%%%%%%%%%%%%%%%%%%%%%%%%%%%%%%%%%%%%%%%%%%%%
% \section{Implementation}
%\iffalse
%<*package>
%\fi
%
% This section describes the definitions file |childdoc.def|.

% The definitions cannot be loaded using |\usepackage| or |\RequirePackage|
% which has a mechanism to prevent loading a style file more than once.
% When loading the definitions by means of |\input|
% multiple instances have to be prevented manually:
%\iffalse
%This code needs to be before the `\ProvidesFile' directive
%which is defined at the beginning of this file.
%Therefore it is also placed there and commented out here.
%</package>
%<*discard>
%\fi
%    \begin{macrocode}
\ifdefined\childdocmain\endinput\fi
%    \end{macrocode}
%\iffalse
%</discard>
%<*package>
%\fi
%
% \macro{\ifchilddoc}
% \macro{\ifchilddocmanual}
% The conditional |\ifchilddoc| tells whether a
% child (true) or main (false) document is being compiled.
% The conditional |\ifchilddocmanual| tells whether
% the |\includeonly| mechanism is used (false) or
% the selection of child files must be performed manually (true).
% The definitions initialise to false:
%    \begin{macrocode}
\newif\ifchilddoc
\newif\ifchilddocmanual
%    \end{macrocode}

% \macro{\childdocname}
% \macro{\childdocjob}
% The macro |\childdocname| stores the name of the main document
% to be compiled. The macro |\childdocjob| stores the name of
% the document on which the \LaTeX{} compiler was originally invoked.
% The content of |\jobname| cannot be compared
% to filenames specified in the source due to different catcodes.
% The following code rescans |\jobname|, stores the result
% in |\childdocname| and saves a copy in |\childdocjob|:
%    \begin{macrocode}
\edef\childdocname{\scantokens\expandafter{\jobname\noexpand}}
\let\childdocjob\childdocname
%    \end{macrocode}

% \macro{\childdocdisable}
% The macro |\childdocdisable| prevents the main file
% from being processed more than once.
% At this stage, the main document command |\childdocmain|
% is assumed to be called once again where it should do nothing.
% Any subsequent call to it should prevent
% a secondary processing of the main document
% It overwrites the forwarding commands
% |\childdocof| and |\childdocforward|
% with empty macros to prevent further inclusions of the main document:
%    \begin{macrocode}
\newcommand{\childdocdisable}
{
  \renewcommand{\childdocmain}[1]{\renewcommand{\childdocmain}[1]{\endinput}}
  \renewcommand{\childdocof}[1]{}
  \renewcommand{\childdocby}[2][]{}
  \renewcommand{\childdocforward}[2][]{}
  \renewcommand{\childdocdisable}{}
}
%    \end{macrocode}

% \macro{\childdocmain}
% The macro |\childdocmain| is to be called at the top of the main file
% with nothing or the main filename (without extension) as argument.
% First, it breaks loops.
% If the argument is not empty and does not match |\childdocname|
% (which is set by the first inclusion of |childdoc.def|),
% |\ifchilddoc| is set to true, |\includeonly| is applied to the child file
% and |\jobname| is set to the main file
% (for proper handling of |.aux| files):
%    \begin{macrocode}
\newcommand{\childdocmain}[1]
{
  \childdocdisable\childdocmain{}
  \if?#1?\else
    \begingroup
      \def\childdoctmp{#1}
      \ifx\childdoctmp\childdocname
        \def\childdoctmp{}
      \else
        \def\childdoctmp
        {
          \childdoctrue
          \includeonly{\childdocname}
          \def\childdocjob{#1}
          \def\jobname{#1}
        }
      \fi
      \expandafter
    \endgroup
    \childdoctmp
  \fi
}
%    \end{macrocode}

% \macro{\childdocof}
% The command |\childdocof| redirects
% compilation to the main file |#1|.
%    \begin{macrocode}
\newcommand{\childdocof}[1]
{
  \childdocdisable
  \childdoctrue
  \includeonly{\childdocname}
  \def\jobname{#1}
  \def\childdocjob{#1}
  \input{#1}
}
%    \end{macrocode}

% \macro{\childdocby}
% The command |\childdocby| ....
%    \begin{macrocode}
\newcommand{\childdocby}[2][]
{
  \childdocdisable
  \childdoctrue
  \childdocmanualtrue
  \if?#1?\else
    \def\jobname{#2}
  \fi
  \def\childdocjob{#2}
  \input{#2}
  \endinput
}
%    \end{macrocode}

% \macro{\childdocforward}
% The command |\childdocforward| redirects
% compilation to the main file or
% (if the optional argument is given) a child file.
% Parameters are set as if the main file
% or a child file starting with |\childdocof| was compiled.
% Then compilation is handed over to the main file:
%    \begin{macrocode}
\newcommand{\childdocforward}[2][]
{
  \begingroup
    \if?#1?
      \def\childdoctmp
      {
        \def\childdocname{#2}
        \def\childdocjob{#2}
        \def\jobname{#2}
        \input{#2}
        \endinput
      }
    \else
      \def\childdoctmp
      {
        \childdocdisable
        \def\childdocname{#2}
        \childdoctrue
        \includeonly{#2}
        \def\childdocjob{#1}
        \def\jobname{#1}
        \input{#1}
        \endinput
      }
    \fi
    \expandafter
  \endgroup
  \childdoctmp
}
%    \end{macrocode}

% \macro{\childdocforwardprefix}
% The command |\childdocforwardprefix| redirects
% compilation to the main or a child file by means of a pattern.
% The prefix |#1| in the current filename is replaced by |#2|
% and the suffix of the current filename is kept
% (it is assumed that the filename does not contain the substring `|~~~|'
% which is used as a delimiter).
% Compilation is handed over to the new file by |\childdocforward|:
%    \begin{macrocode}
\newcommand{\childdocforwardprefix}[3][]
{
  \begingroup
    \def\childdocextract #2##1~~~{\def\childdoctmp{\childdocforward[#1]{#3##1}}}
    \expandafter\childdocextract\childdocname~~~
    \expandafter
  \endgroup
  \childdoctmp
}
%    \end{macrocode}

% \macro{\childdoc}
% The deprecated macro |\childdoc| is a legacy version of |\childdocmain|:
%    \begin{macrocode}
\newcommand{\childdoc}{\childdocmain}
%    \end{macrocode}

% \macro{\childdocredirect}
% The deprecated macro |\childdocredirect| is a legacy version
% of |\childdocforward| and |\childdocforwardprefix|:
%    \begin{macrocode}
\newcommand{\childdocredirect}[2][]
{
  \begingroup
    \if?#1?
      \def\childdoctmp{\childdocforward{#2}}
    \else
      \def\childdoctmp{\childdocforwardprefix{#1}{#2}}
    \fi
    \expandafter
  \endgroup
  \childdoctmp
}
%    \end{macrocode}

%\iffalse
%</package>
%\fi
%
\endinput
|\\
|\childdocby{|\textit{main}|}|\\
\end{tabular}
\end{center}
%
The directive |\childdocby| is similar to |\childdocof|
described in \secref{sec:include},
but the subsequent selection of content must be done manually.
To that end, both |\ifchilddoc| and |\ifchilddocmanual|
will be true upon processing of a part,
and the name of the part is stored in |\childdocname|.
Note that |\jobname| will be set to the filename of the current part
so that each part receives an individual |.aux| file
that does not interfere with the |.aux| file(s) of the main document.
This behaviour can be altered by the alternative form
|\childdocby[*]{|\textit{main}|}| (with a non-empty optional argument)
which uses the |.aux| file of the main document
by setting |\jobname| to \textit{main}.

%%%%%%%%%%%%%%%%%%%%%%%%%%%%%%%%%%%%%%%%%%%%%%%%%%%%%%%%%%%%%%%%%%%%%%%%%%%%%%%%
\subsection{Driver Development}
\label{sec:driver}

The \textsf{childdoc} mechanism can also be use for the development
of definition files such as \LaTeX{} styles or classes.
This case differs from the above setup with multiple parts
included by |\include| in that no |\includeonly| should be invoked.
This can be achieved by starting the include file
(before |\ProvidesPackage|) with:
%
\begin{center}
\begin{tabular}{l}
|% \iffalse
%
% childdoc.dtx Copyright (C) 2017-2018 Niklas Beisert
%
% This work may be distributed and/or modified under the
% conditions of the LaTeX Project Public License, either version 1.3
% of this license or (at your option) any later version.
% The latest version of this license is in
%   http://www.latex-project.org/lppl.txt
% and version 1.3 or later is part of all distributions of LaTeX
% version 2005/12/01 or later.
%
% This work has the LPPL maintenance status `maintained'.
%
% The Current Maintainer of this work is Niklas Beisert.
%
% This work consists of the files childdoc.dtx and childdoc.ins
% and the derived files childdoc.def and cdocsamp.tex with
% cdocsch1.tex, cdocsch2.tex, cdocsdrf.tex, cdocsfn1.tex, cdocsfn2.tex.
%
%<package>\ifdefined\childdocmain\endinput\fi
%<package>\ProvidesFile{childdoc.def}[2018/12/30 v2.0 child document driver]
%<samplemain>\ProvidesFile{cdocsamp.tex}[2018/12/30 v2.0 sample for childdoc]
%<*driver>
%\ProvidesFile{childdoc.drv}[2018/12/30 v2.0 childdoc reference manual file]
\PassOptionsToClass{10pt,a4paper}{article}
\documentclass{ltxdoc}

\usepackage[margin=35mm]{geometry}
\usepackage{hyperref}
\usepackage{hyperxmp}
\usepackage[usenames]{color}

\hypersetup{colorlinks=true}
\hypersetup{pdfstartview=FitH}
\hypersetup{pdfpagemode=UseNone}
\hypersetup{pdfsource={}}
\hypersetup{pdflang={en-UK}}
\hypersetup{pdfcopyright={Copyright 2017-2018 Niklas Beisert.
  This work may be distributed and/or modified under the
  conditions of the LaTeX Project Public License, either version 1.3
  of this license or (at your option) any later version.}}
\hypersetup{pdflicenseurl={http://www.latex-project.org/lppl.txt}}
\hypersetup{pdfcontactaddress={ETH Zurich, ITP, HIT K,
  Wolfgang-Pauli-Strasse 27}}
\hypersetup{pdfcontactpostcode={8093}}
\hypersetup{pdfcontactcity={Zurich}}
\hypersetup{pdfcontactcountry={Switzerland}}
\hypersetup{pdfcontactemail={nbeisert@itp.phys.ethz.ch}}
\hypersetup{pdfcontacturl={http://people.phys.ethz.ch/\xmptilde nbeisert/}}

\newcommand{\secref}[1]{\hyperref[#1]{section \ref*{#1}}}

\parskip1ex
\parindent0pt
\let\olditemize\itemize
\def\itemize{\olditemize\parskip0pt}

\begin{document}

\title{The \textsf{childdoc} Package}
\hypersetup{pdftitle={The childdoc Package}}
\author{Niklas Beisert\\[2ex]
  Institut f\"ur Theoretische Physik\\
  Eidgen\"ossische Technische Hochschule Z\"urich\\
  Wolfgang-Pauli-Strasse 27, 8093 Z\"urich, Switzerland\\[1ex]
  \href{mailto:nbeisert@itp.phys.ethz.ch}
  {\texttt{nbeisert@itp.phys.ethz.ch}}}
\hypersetup{pdfauthor={Niklas Beisert}}
\hypersetup{pdfsubject={Manual for the LaTeX2e Package childdoc}}
\date{30 December 2018, \textsf{v2.0}}
\maketitle

\begin{abstract}\noindent
\textsf{childdoc} is a \LaTeXe{} package
that enables the direct compilation
of document sections included by |\include|
to individual files.
\end{abstract}

\begingroup
\parskip0ex
\tableofcontents
\endgroup

%%%%%%%%%%%%%%%%%%%%%%%%%%%%%%%%%%%%%%%%%%%%%%%%%%%%%%%%%%%%%%%%%%%%%%%%%%%%%%%%
%%%%%%%%%%%%%%%%%%%%%%%%%%%%%%%%%%%%%%%%%%%%%%%%%%%%%%%%%%%%%%%%%%%%%%%%%%%%%%%%
\section{Introduction}

\LaTeX{} provides a mechanism to structure a large document (such as a book)
into a main file and several child files (containing the chapters)
using the |\include| command.
This mechanism is beneficial for documents
which span hundreds of pages in order to
make the source file(s) more manageable.
Moreover, compilation can be restricted to
selected child files by means of the |\includeonly| command.
The latter feature can be used to reduce the compilation time while editing
(this was significantly more useful in the earlier days of \LaTeX{})
or to generate a smaller document which is easier to navigate.
Another application of |\includeonly| is to generate
documents consisting of selected parts of the complete document.

However, there are a few drawbacks of the plain |\include| mechanism:
\begin{itemize}
\item
The child files cannot be compiled on their own,
they can only be compiled via the main file.
A naive editing environment
(such as a text editor with an option
to have the current file processed by \LaTeX)
may require one to switch to the main file before compiling;
attempting to compile the child file produces errors.
\item
The main file must be modified (each time)
to adjust the |\includeonly| command
to the present needs. This easily leaves the main file in a messy state.
\item
The generated document will always carry the filename
of the main document. This is inconvenient if
several child files are to be compiled and
to be kept for distribution.
\end{itemize}

The present package provides a simple interface
to make child files individually compilable by \LaTeX{}.
Compiling a child file then has the same effect as compiling
the main file with an |\includeonly| command
to select the appropriate child.
Moreover the generated document will carry the name of the child
rather than the main file.
This resolves all three above issues.

This feature is meant to make the editing of books,
thesis documents and lecture notes somewhat more convenient.
However, the package can also be used efficiently for
composing a series of documents (such as exercise sheets)
which are typically distributed individually.
It then assists the author in generating the individual documents
(potentially in different versions)
as well as a document containing the collected series.
Another application is in developing style files
or other kinds of included material
where compilation of the style file could redirect
to a sample or test file.

%%%%%%%%%%%%%%%%%%%%%%%%%%%%%%%%%%%%%%%%%%%%%%%%%%%%%%%%%%%%%%%%%%%%%%%%%%%%%%%%
%%%%%%%%%%%%%%%%%%%%%%%%%%%%%%%%%%%%%%%%%%%%%%%%%%%%%%%%%%%%%%%%%%%%%%%%%%%%%%%%
\section{Usage}

First of all, the package \textsf{childdoc} is \emph{not} a standard
\LaTeXe{} |.sty| style file! Therefore it needs to be invoked in
a non-standard way.

%%%%%%%%%%%%%%%%%%%%%%%%%%%%%%%%%%%%%%%%%%%%%%%%%%%%%%%%%%%%%%%%%%%%%%%%%%%%%%%%
\subsection{Included Files}
\label{sec:include}

%%%%%%%%%%%%%%%%%%%%%%%%%%%%%%%%%%%%%%%%
\DescribeMacro{\childdocmain}
To use the package, add the commands
\begin{center}
\begin{tabular}{l}
|\input{childdoc.def}|\\
|\childdocmain{}|\\
\end{tabular}
\end{center}
at the very top of the main \LaTeX{} file,
in particular \emph{before} the |\documentclass| statement!
The argument of |\childdocmain| should be left empty
(but it must be present).

%%%%%%%%%%%%%%%%%%%%%%%%%%%%%%%%%%%%%%%%
\DescribeMacro{\childdocof}
Furthermore, add the commands
\begin{center}
\begin{tabular}{l}
|\input{childdoc.def}|\\
|\childdocof{|\textit{main}|}|\\
\end{tabular}
\end{center}
at the top of every child file \textit{child}
which is included by |\include{|\textit{child}|}|
from within the main file
(or at least for those files to be compiled individually).
The argument \textit{main} must be the filename of the main file.

There are a couple of
considerations in setting up the main and child documents:

%%%%%%%%%%%%%%%%%%%%%%%%%%%%%%%%%%%%%%%%
\paragraph{Restrictions.}

Please note the following restrictions:
\begin{itemize}
\item
|\childdocmain| must be called with one argument \textit{main}
to ensure compatibility with earlier version of the package.
It must either be empty (|\childdocmain{}|)
or precisely match the filename of the main file in which it is specified.
See \secref{sec:detection} for further information.
\item
The filename \textit{main} must be specified without the |.tex| extension.
\item
The filename \textit{main} is case sensitive
(even in case-insensitive file systems)
due to internal string comparison.
\item
The argument \textit{main} should be fully expanded, it cannot be a macro.
\item
Subdirectories and special characters should be avoided in filenames.
\item
The command |\childdocmain{|\textit{main}|}| must be followed by a whitespace.
It should not be followed immediately by another command
or by a comment mark `|%|'.
This is because the \TeX{} parser reads the token immediately following
the argument of |\childdocmain| and puts it
at the beginning of every child section;
however, a white\-space is ignored.
\end{itemize}

%%%%%%%%%%%%%%%%%%%%%%%%%%%%%%%%%%%%%%%%
\paragraph{Content of Main File.}

It is advisable to place all content in the child files included by |\include|.
Any output contained in the main file will appear in all child documents
unless suppressed manually;
it cannot be suppressed automatically by the |\includeonly| directive
and thus should normally be avoided.
A method to include some content in the main file
by means of conditional processing is described in \secref{sec:conditional}.

%%%%%%%%%%%%%%%%%%%%%%%%%%%%%%%%%%%%%%%%
\paragraph{Page Numbering.}

When only a part of the document is compiled,
the appropriate numbering of pages
(as well as other status parameters)
is determined from the |.aux| files.
The latter contain information from previous passes.
However this information needs to propagate through
all intermediate child documents.
Therefore the page numbering in child documents may well
be inconsistent until the complete document is compiled at least once.

A useful (if unconventional) way to always ensure a consistent
page numbering is to restart the numbering in each child document
and denote the pages by `\textit{child}|.|\textit{page}'
where \textit{child} represents the chapter/section number of the child file.
This can be achieved by the command
|\numberwithin{page}{|\textit{child}|}|
of the \textsf{amsmath} package
where \textit{child} can be |chapter| or |section|
depending on the chosen structuring.
Alternatively, one can modify the macro |\thepage| appropriately
and reset the counter |page| at the start of each child file.

%%%%%%%%%%%%%%%%%%%%%%%%%%%%%%%%%%%%%%%%%%%%%%%%%%%%%%%%%%%%%%%%%%%%%%%%%%%%%%%%
\subsection{Conditional Processing}
\label{sec:conditional}

The package provides a mechanism to compile different versions
of a document. To customise the versions further some conditional processing
can come in handy to distinguish which version is being compiled.
The package provides two macros to describe the compilation context:

%%%%%%%%%%%%%%%%%%%%%%%%%%%%%%%%%%%%%%%%
\DescribeMacro{\ifchilddoc}
The conditional |\ifchilddoc| distinguishes between the compilation of
child documents and the main document:
%
\begin{center}
|\ifchilddoc |\textit{child-code}| |[|\||else |\textit{main-code}]| \||fi|
\end{center}

%%%%%%%%%%%%%%%%%%%%%%%%%%%%%%%%%%%%%%%%
\DescribeMacro{\childdocname}
\DescribeMacro{\childdocjob}
The macro |\childdocname| contains the filename (without extension)
of the main or child file being processed.
Note that |\childdocjob| will always contain the name of the main file.

%%%%%%%%%%%%%%%%%%%%%%%%%%%%%%%%%%%%%%%%
\paragraph{Title Page.}

Conditional processing can be used to include a title or banner page
in the main document when proper precautions are taken.
Importantly, the code in the main file should ensure that the page counter
(as well as other status parameters which are stored in the |.aux| files)
takes the same value after the conditional processing.
Otherwise the page numbers may take divergent values
depending on which part is compiled.

For example, a title page could be declared by:
%
\begin{center}
\begin{tabular}{l}
|\ifchilddoc\||else|\\
|\addtocounter{page}{-1}|\\
\textit{code for title page}\\
|\newpage|\\
|\||fi|
\end{tabular}
\end{center}
%
A banner page for the child documents can be generated by:
%
\begin{center}
\begin{tabular}{l}
|\ifchilddoc|\\
|\addtocounter{page}{-1}|\\
\textit{code for banner page}\\
|\newpage|\\
|\||fi|
\end{tabular}
\end{center}
%
Here one could write a message such as:
\begin{center}
|This is the part \childdocname{} of \childdocjob{}.|
\end{center}

%%%%%%%%%%%%%%%%%%%%%%%%%%%%%%%%%%%%%%%%%%%%%%%%%%%%%%%%%%%%%%%%%%%%%%%%%%%%%%%%
\subsection{Flags}
\label{sec:flags}

The package makes it easy to generate different versions
of the main or child documents.
To this end compilation flags can be defined
and assigned different default values.
They will be particularly useful in conjunction
with the forwarding mechanism described in \secref{sec:forward}.

For example, it may be useful to have a flag |\version|
which can be set to |draft| or |final|.
The document source will contain some conditional code
depending on the value of |\version|.
Suppose further, the flag should default to |final| for the main file
and to |draft| for child files
which is a natural assignment for editing the document.
This is achieved by placing the following code
in the preamble of the main document
(below the |\childdocmain| directive):
%
\begin{center}
\begin{tabular}{l}
|\ifchilddoc|\\
|\providecommand{\version}{draft}|\\
|\||else|\\
|\providecommand{\version}{final}|\\
|\||fi|
\end{tabular}
\end{center}
%
The definition by |\providecommand| makes sure
that previous definitions are not overwritten.
Further statements |\providecommand{\version}{...}|
can thus be added before the above code to override it.

For the main file, one might add a line
(between |\childdocmain| and the above block)
%
\begin{center}
|%\ifchilddoc\||else\providecommand{\version}{draft}\||fi|
\end{center}
%
which can be uncommented to produce a draft version.
Likewise one can add a line to the very top of a child file
(above the |\childdocof{|\textit{main}|}| directive)
%
\begin{center}
|%\providecommand{\version}{final}|
\end{center}
%
which can be uncommented to produce the final version of this child document.

%%%%%%%%%%%%%%%%%%%%%%%%%%%%%%%%%%%%%%%%%%%%%%%%%%%%%%%%%%%%%%%%%%%%%%%%%%%%%%%%
\subsection{Forwarding}
\label{sec:forward}

Different versions of the main or child documents
using compilation flags as described in \secref{sec:flags}
can be (permanently) stored in different files
for convenient compilation, viewing and distribution.
To this end, the package defines a command
to pass on compilation to a different file:

%%%%%%%%%%%%%%%%%%%%%%%%%%%%%%%%%%%%%%%%
\DescribeMacro{\childdocforward}
The command |\childdocforward| redirects processing to
another source file:
%
\begin{center}
\begin{tabular}{l}
|\input{childdoc.def}|\\
|\childdocforward[|\textit{main}|]{|\textit{dest}|}|\\
\end{tabular}
\end{center}
%
The argument \textit{dest} is the destination file
(without extension).
It should be the main file or one of the child files.
Note that further \textsf{childdoc} directives
such as |\childdocof| and |\childdocforward|
in the indicated file will be processed in this form.
The optional argument \textit{main}
passes on directly to the main file \textit{main}
while pretending to compile the child \textit{dest}.
This form behaves as if \textit{dest}
issues |\childdocof{|\textit{main}|}| right away,
and no further \textsf{childdoc} directives will be processed.

%%%%%%%%%%%%%%%%%%%%%%%%%%%%%%%%%%%%%%%%
\DescribeMacro{\...prefix}
In the alternative form |\childdocforwardprefix|,
%
\begin{center}
\begin{tabular}{l}
|\input{childdoc.def}|\\
|\childdocforwardprefix[|\textit{main}|]{|\textit{prefix}|}{|\textit{dest}|}|
\end{tabular}
\end{center}
%
the destination file is determined by a pattern
depending on the current file:
To make this work, the current file must be called
`{\textit{prefix}\hspace{0.2em}\textit{suffix}}'
with \textit{prefix} matching precisely the argument.
Processing is then passed on to the file
`{\textit{dest}\hspace{0.2em}\textit{suffix}}'.
Surely, the same effect is achieved by
directly specifying the
argument `{\textit{dest}\hspace{0.2em}\textit{suffix}}'
in the first form.
However, that requires to set up a different file
for each child. With the alternative form of the command
all these files can have exactly the same content
which simplifies setting them up and maintaining them.

For example, the following file |draft.tex|
with a compilation flag |\version| as described in \secref{sec:flags}
compiles the main document as a draft:
%
\begin{center}
\begin{tabular}{l}
|\def\version{draft}|\\
|\input{childdoc.def}|\\
|\childdocforward{|\textit{main}|}|
\end{tabular}
\end{center}
%
Likewise, the following files |final|\textit{nn}|.tex|
compile the final version of the child document
|child|\textit{nn}|.tex|:
%
\begin{center}
\begin{tabular}{l}
|\def\version{final}|\\
|\input{childdoc.def}|\\
|\childdocforwardprefix{final}{child}|
\end{tabular}
\end{center}
%

Note that when several versions of a main file and/or of each child file
are to be generated, it may be convenient to set up a |Makefile| or
shell script to automatise the process.

%%%%%%%%%%%%%%%%%%%%%%%%%%%%%%%%%%%%%%%%%%%%%%%%%%%%%%%%%%%%%%%%%%%%%%%%%%%%%%%%
\subsection{Command Line Processing}
\label{sec:commandline}

The effect of redirection files can also be achieved by invoking
the \LaTeX{} compiler with a more elaborate command line.
Most conveniently this should be done as part
of a shell script or a |Makefile|.

When using \textsf{childdoc} in the main file, the following
command lines effectively perform a redirection
(note that depending on the shell being used,
backslashes may have to be doubled: `|\|' $\to$ `|\\|'):
%
\begin{center}
|... -jobname "|\textit{target}|" |\\|"|[\textit{flags}]%
|\input{childdoc.def}\childdocforward[|\textit{main}|]{|\textit{dest}|}"|
\end{center}
%
Here \textit{target} is the name of the output file,
\textit{main} is the name of the main file
and \textit{dest} is the name of the main or child file to be processed
(all filenames without extensions).
The optional argument \textit{main} can be omitted
if \textit{main} matches \textit{dest}.
Optionally, compilation \textit{flags} can be defined via |\def| commands.
This command line makes the \TeX{} engine believe
it is compiling the file \textit{target}
whose content is specified as the latter parameter.
The provided code then forwards the processing to
\textit{main} or \textit{dest} as described in \secref{sec:forward}.

%%%%%%%%%%%%%%%%%%%%%%%%%%%%%%%%%%%%%%%%%%%%%%%%%%%%%%%%%%%%%%%%%%%%%%%%%%%%%%%%
\subsection{Include by Input}
\label{sec:input}

Including child documents by |\include| has some restrictions by design.
Most notably, the content of a child document always occupies
its own set of pages; pages cannot be shared between child documents.
Usually, this behaviour makes perfect sense
because each child document contain an essential part of the document.
However, in some situations it may be desirable to compose
a document from a collection of parts
without having mandatory page breaks between then.
For this case, the package
provides a mechanism to include parts
by |\input| which can also be processed individually.
However, by construction this mechanism
requires manual handling of the content to be output.

%%%%%%%%%%%%%%%%%%%%%%%%%%%%%%%%%%%%%%%%
\DescribeMacro{\ifchilddocmanual}
The main file should be prepared as usual, see \secref{sec:include}.
However, the document body must make a distinction
between processing of an individual part and of the main document, e.g.:
%
\begin{center}
\begin{tabular}{l}
|\ifchilddocmanual|\\
|\input{\childdocname}|\\
|\||else|\\
\textit{document body with }|\input{|\textit{part}|}|\\
|\||fi|
\end{tabular}
\end{center}
%
The conditional |\ifchilddocmanual| is true whenever
a part to be included by |\input| is being compiled,
and the name of the part is stored in |\childdocname|.

%%%%%%%%%%%%%%%%%%%%%%%%%%%%%%%%%%%%%%%%
\DescribeMacro{\childdocby}
Each part to be included by |\input| should start with:
%
\begin{center}
\begin{tabular}{l}
|\input{childdoc.def}|\\
|\childdocby{|\textit{main}|}|\\
\end{tabular}
\end{center}
%
The directive |\childdocby| is similar to |\childdocof|
described in \secref{sec:include},
but the subsequent selection of content must be done manually.
To that end, both |\ifchilddoc| and |\ifchilddocmanual|
will be true upon processing of a part,
and the name of the part is stored in |\childdocname|.
Note that |\jobname| will be set to the filename of the current part
so that each part receives an individual |.aux| file
that does not interfere with the |.aux| file(s) of the main document.
This behaviour can be altered by the alternative form
|\childdocby[*]{|\textit{main}|}| (with a non-empty optional argument)
which uses the |.aux| file of the main document
by setting |\jobname| to \textit{main}.

%%%%%%%%%%%%%%%%%%%%%%%%%%%%%%%%%%%%%%%%%%%%%%%%%%%%%%%%%%%%%%%%%%%%%%%%%%%%%%%%
\subsection{Driver Development}
\label{sec:driver}

The \textsf{childdoc} mechanism can also be use for the development
of definition files such as \LaTeX{} styles or classes.
This case differs from the above setup with multiple parts
included by |\include| in that no |\includeonly| should be invoked.
This can be achieved by starting the include file
(before |\ProvidesPackage|) with:
%
\begin{center}
\begin{tabular}{l}
|\input{childdoc.def}|\\
|\childdocforward{|\textit{main}|}|\\
\end{tabular}
\end{center}
%
or alternatively with:
%
\begin{center}
\begin{tabular}{l}
|\input{childdoc.def}|\\
|\childdocby{|\textit{main}|}|\\
\end{tabular}
\end{center}
%
Both forms have slightly different effects as described above.
The main file is prepared as usual, see \secref{sec:include}.

%%%%%%%%%%%%%%%%%%%%%%%%%%%%%%%%%%%%%%%%%%%%%%%%%%%%%%%%%%%%%%%%%%%%%%%%%%%%%%%%
\subsection{Legacy Detection}
\label{sec:detection}

The directive |\childdocmain| in the main file can detect
whether the complete document or merely a child is to be compiled
even without using the directive |\childdocof|.
This method is deprecated because it is less robust
and there is no compelling reason to use it;
it is merely provided for backward compatibility
and it may be removed in future versions.

If the detection mechanism is to be used,
it is mandatory to correctly specify
the filename of the main file as the argument of |\childdocmain|:
%
\begin{center}
\begin{tabular}{l}
|\input{childdoc.def}|\\
|\childdocmain{|\textit{main}|}|\\
\end{tabular}
\end{center}
%
If |\jobname| does not match the argument \textit{main} of |\childdocmain|,
it is assumed that |\jobname| points to the child file to be compiled.
When using |\childdocmain| with the main file specified as argument,
it suffices to start a child file
with just |\input{|\textit{main}|}|
without loading of the package and using |\childdocof|.
If instead all processing is done
with the appropriate \textsf{childdoc} directives,
the argument of \textit{main} of |\childdocmain| can be empty.

An alternative version of the command line processing described
in \secref{sec:commandline} using the detection mechanism reads:
%
\begin{center}
|... -jobname "|\textit{target}|" "|[\textit{flags}]%
[|\def\jobname{|\textit{dest}|}|]|\input{|\textit{main}|}"|
\end{center}

%%%%%%%%%%%%%%%%%%%%%%%%%%%%%%%%%%%%%%%%%%%%%%%%%%%%%%%%%%%%%%%%%%%%%%%%%%%%%%%%
\subsection{Manual Code}
\label{sec:manual}

In case one cannot be certain whether the definitions file |childdoc.def|
is installed on the target \TeX{} distribution
and one prefers not to ship it,
it is conceivable to paste a few relevant commands into the sources.

To that end, drop all statements |\input{childdoc.def}|
and perform the replacements as outlined below.
Instead of |\childdocmain{|\textit{main}|}| add the following code
to the top of the main file:
%
\begin{center}
\begin{tabular}{l}
|\||ifdefined\childdocname\endinput\||fi\newif\ifchilddoc|\\
|\edef\childdocname{\scantokens\expandafter{\jobname\noexpand}}|\\
|\def\childdocmain{|\textit{main}|}\||ifx\childdocmain\childdocname\||else|\\
|\childdoctrue\includeonly{\childdocname}\let\jobname\childdocmain\||fi|\\
\end{tabular}
\end{center}
%
Instead of |\childdocof{|\textit{main}|}| just include the main file
at the top of each child file:
%
\begin{center}
|\input{|\textit{main}|}|
\end{center}
%
A simple redirection |\childdocforward{|\textit{dest}|}| is achieved by:
%
\begin{center}
|\def\jobname{|\textit{dest}|}\input{\jobname}|
\end{center}
%
The redirection with prefix
|\childdocforwardprefix[|\textit{prefix}|]{|\textit{dest}|}|
is accomplished by:
%
\begin{center}
\begin{tabular}{l}
|{\edef\jobname{\scantokens\expandafter{\jobname\noexpand}}|\\
|\def\redirectjob |\textit{prefix}|#1~~~{\gdef\jobname{|\textit{dest}|#1}}|\\
|\expandafter\redirectjob\jobname~~~}\input{\jobname}|
\end{tabular}
\end{center}

In an alternative approach,
child documents can be compiled by a specific command line
without additional code or specific definitions:
%
\begin{center}
|... -jobname "|\textit{target}|" "|[\textit{flags}]%
|\includeonly{|\textit{dest}|}\input{|\textit{main}|}"|
\end{center}
%

%%%%%%%%%%%%%%%%%%%%%%%%%%%%%%%%%%%%%%%%%%%%%%%%%%%%%%%%%%%%%%%%%%%%%%%%%%%%%%%%
%%%%%%%%%%%%%%%%%%%%%%%%%%%%%%%%%%%%%%%%%%%%%%%%%%%%%%%%%%%%%%%%%%%%%%%%%%%%%%%%
\section{Information}

%%%%%%%%%%%%%%%%%%%%%%%%%%%%%%%%%%%%%%%%%%%%%%%%%%%%%%%%%%%%%%%%%%%%%%%%%%%%%%%%
\subsection{Copyright}

Copyright \copyright{} 2017--2018 Niklas Beisert

This work may be distributed and/or modified under the
conditions of the \LaTeX{} Project Public License, either version 1.3
of this license or (at your option) any later version.
The latest version of this license is in
  \url{http://www.latex-project.org/lppl.txt}
and version 1.3 or later is part of all distributions of \LaTeX{}
version 2005/12/01 or later.

This work has the LPPL maintenance status `maintained'.

The Current Maintainer of this work is Niklas Beisert.

This work consists of the files |README.txt|, |childdoc.ins| and |childdoc.dtx|
as well as the derived files |childdoc.def|, |cdocsamp.tex|
with |cdocsch1.tex|, |cdocsch2.tex|, |cdocspt3.tex|, |cdocspt4.tex|,
|cdocsdrf.tex|, |cdocsfn1.tex|, |cdocsfn2.tex|
as well as |childdoc.pdf|.

%%%%%%%%%%%%%%%%%%%%%%%%%%%%%%%%%%%%%%%%%%%%%%%%%%%%%%%%%%%%%%%%%%%%%%%%%%%%%%%%
\subsection{Files and Installation}

The package consists of the files:
%
\begin{center}
\begin{tabular}{ll}
    |README.txt|   & readme file \\
    |childdoc.ins| & installation file \\
    |childdoc.dtx| & source file \\
    |childdoc.def| & definition file \\
    |cdocsamp.tex| & sample main file \\
    |cdocsch1.tex| & sample include file \\
    |cdocsch2.tex| & sample include file \\
    |cdocspt3.tex| & sample part file \\
    |cdocspt4.tex| & sample part file \\
    |cdocsdrf.tex| & sample redirection file \\
    |cdocsfn1.tex| & sample redirection file \\
    |cdocsfn2.tex| & sample redirection file \\
    |childdoc.pdf| & manual
\end{tabular}
\end{center}
%
The distribution consists of the files
|README.txt|, |childdoc.ins| and |childdoc.dtx|.
%
\begin{itemize}
\item
Run (pdf)\LaTeX{} on |childdoc.dtx|
to compile the manual |childdoc.pdf| (this file).
\item
Run \LaTeX{} on |childdoc.ins| to create the definitions file |childdoc.def|
and the sample |cdocsamp.tex| with include files
|cdocsch1.tex|, |cdocsch2.tex|, |cdocspt3.tex|, |cdocspt4.tex|,
|cdocsdrf.tex|, |cdocsfn1.tex|, |cdocsfn2.tex|.
Then copy the file |childdoc.def| to an appropriate directory of your \LaTeX{}
distribution, e.g.\ \textit{texmf-root}|/tex/latex/childdoc|.
\end{itemize}

%%%%%%%%%%%%%%%%%%%%%%%%%%%%%%%%%%%%%%%%%%%%%%%%%%%%%%%%%%%%%%%%%%%%%%%%%%%%%%%%
\subsection{Related CTAN Packages}

There are several other packages which offer a similar functionality:
%
\begin{itemize}
\item
The packages
\href{http://ctan.org/pkg/docmute}{\textsf{docmute}},
\href{http://ctan.org/pkg/includex}{\textsf{includex}} and
\href{http://ctan.org/pkg/standalone}{\textsf{standalone}}
provide commands to include only the document body of
a child file thus allowing both files to be compiled individually.
\item
The packages \href{http://ctan.org/pkg/subdocs}{\textsf{subdocs}}
and \href{http://ctan.org/pkg/subfiles}{\textsf{subfiles}}
provide structures in which the main and child documents can be
encapsulated and allowing them to be compiled individually.
The inclusion mechanism is different from the conventional |\include|.
\item
The package \href{http://ctan.org/pkg/combine}{\textsf{combine}}
is an elaborate solution to combine several documents into one.
\end{itemize}
%
See also the CTAN topic \href{http://ctan.org/topic/subdocs}{\textsf{subdocs}}
for further related packages.
The present package differs from the above solutions in that
a document structure constructed with the conventional |\include| mechanism
just needs two extra commands at the top of every file
such that all constituent files can be compiled individually.

%%%%%%%%%%%%%%%%%%%%%%%%%%%%%%%%%%%%%%%%%%%%%%%%%%%%%%%%%%%%%%%%%%%%%%%%%%%%%%%%
%\subsection{Feature Suggestions}
%
%The following is a list of features which may be useful for future
%versions of this package:
%%
%\begin{itemize}
%\item
%\ldots
%\end{itemize}

%%%%%%%%%%%%%%%%%%%%%%%%%%%%%%%%%%%%%%%%%%%%%%%%%%%%%%%%%%%%%%%%%%%%%%%%%%%%%%%%
\subsection{Revision History}

%%%%%%%%%%%%%%%%%%%%%%%%%%%%%%%%%%%%%%%%
\paragraph{v2.0:} 2018/12/30

\begin{itemize}
\item
immediate forward processing
\item
added |\childdocby| mechanism
\item
manual restructured
\end{itemize}

%%%%%%%%%%%%%%%%%%%%%%%%%%%%%%%%%%%%%%%%
\paragraph{v1.6:} 2018/01/17

\begin{itemize}
\item
application for development of include files
\item
corrections to manual
\end{itemize}

%%%%%%%%%%%%%%%%%%%%%%%%%%%%%%%%%%%%%%%%
\paragraph{v1.5:} 2017/05/21

\begin{itemize}
\item
more complete structuring introduced
\item
|\childdocof| introduced
\item
|\childdoc| renamed to |\childdocmain|
\item
|\childredirect| renamed to |\childdocforward| and |\childdocforwardprefix|
and functionality expanded
\end{itemize}

%%%%%%%%%%%%%%%%%%%%%%%%%%%%%%%%%%%%%%%%
\paragraph{v1.0:} 2017/04/27

\begin{itemize}
\item
manual and install package
\item
first version published on CTAN
\end{itemize}

%%%%%%%%%%%%%%%%%%%%%%%%%%%%%%%%%%%%%%%%
\paragraph{v0.6:} 2017/04/26

\begin{itemize}
\item
redirection mechanism added
\end{itemize}

%%%%%%%%%%%%%%%%%%%%%%%%%%%%%%%%%%%%%%%%
\paragraph{v0.5:} 2017/04/26

\begin{itemize}
\item
functionality in definition file
\end{itemize}


%%%%%%%%%%%%%%%%%%%%%%%%%%%%%%%%%%%%%%%%%%%%%%%%%%%%%%%%%%%%%%%%%%%%%%%%%%%%%%%%
%%%%%%%%%%%%%%%%%%%%%%%%%%%%%%%%%%%%%%%%%%%%%%%%%%%%%%%%%%%%%%%%%%%%%%%%%%%%%%%%
%%%%%%%%%%%%%%%%%%%%%%%%%%%%%%%%%%%%%%%%%%%%%%%%%%%%%%%%%%%%%%%%%%%%%%%%%%%%%%%%
\appendix

\settowidth\MacroIndent{\rmfamily\scriptsize 000\ }

 \DocInput{childdoc.dtx}

\end{document}
%</driver>
% \fi
%
% %%%%%%%%%%%%%%%%%%%%%%%%%%%%%%%%%%%%%%%%%%%%%%%%%%%%%%%%%%%%%%%%%%%%%%%%%%%%%%
% %%%%%%%%%%%%%%%%%%%%%%%%%%%%%%%%%%%%%%%%%%%%%%%%%%%%%%%%%%%%%%%%%%%%%%%%%%%%%%
% \section{Sample}
%\iffalse
%<*samplemain>
%\fi
%
% The following presents a sample document
% with two chapters, two parts, a title page,
% a compile flag as well as three forwarding files to set the flag.
% It consists of eight |.tex| files:
% \begin{center}
% \begin{tabular}{ll}
% |cdocsamp.tex|&main file\\
% |cdocsch1.tex|&include file for chapter 1\\
% |cdocsch2.tex|&include file for chapter 2\\
% |cdocspt3.tex|&include file for part 3\\
% |cdocspt4.tex|&include file for part 4\\
% |cdocsdrf.tex|&forwarding file for main file in draft mode\\
% |cdocsfi1.tex|&forwarding file for final version of chapter 1\\
% |cdocsfi2.tex|&forwarding file for final version of chapter 2\\
% \end{tabular}
% \end{center}
% Each of the eight files can be compiled directly by the \LaTeX{} compiler.
%
% %%%%%%%%%%%%%%%%%%%%%%%%%%%%%%%%%%%%%%
% \paragraph{Main File.}
%
% The main file is called |cdocsamp.tex|.
%
% Load the \textsf{childdoc} definitions and
% declare the filename for the main document:
%    \begin{macrocode}
\input{childdoc.def}
\childdocmain{}
%    \end{macrocode}

% Optional override for |\version| flag:
%    \begin{macrocode}
%%\ifchilddoc\else\providecommand{\version}{draft}\fi
%    \end{macrocode}

% Define the default values for the |\version| flag
% (|final| for the main file and |draft| for childs):
%    \begin{macrocode}
\ifchilddoc
\providecommand{\version}{draft}
\else
\providecommand{\version}{final}
\fi
%    \end{macrocode}

% Load the standard document class:
%    \begin{macrocode}
\documentclass[12pt]{article}
%    \end{macrocode}

% Start the document body:
%    \begin{macrocode}
\begin{document}
%    \end{macrocode}

% Declare a title page.
% Print title, part of document being processed and version flag:
%    \begin{macrocode}
\addtocounter{page}{-1}
\begin{center}
{\LARGE\bfseries{}childdoc example\par}
\vspace{1cm}
\ifchilddoc
\ifchilddocmanual part\else chapter\fi:
`\childdocname' of `\childdocjob'\par
\else
main document: `\childdocjob'\par
\fi
version: \version\par
\end{center}
\newpage
%    \end{macrocode}

% Manually include selected file,
% otherwise process as usual:
%    \begin{macrocode}
\ifchilddocmanual
\section*{part `\childdocname'}
\input{\childdocname}
\else
%    \end{macrocode}

% Include the two chapters:
%    \begin{macrocode}
\include{cdocsch1}
\include{cdocsch2}
%    \end{macrocode}

% Include the two parts unless only chapters should be displayed:
%    \begin{macrocode}
\ifchilddoc\else
\section{part three}
\input{cdocspt3}
\section{part four}
\input{cdocspt4}
\fi
%    \end{macrocode}

% Process as usual until here:
%    \begin{macrocode}
\fi
%    \end{macrocode}

% End of document body:
%    \begin{macrocode}
\end{document}
%    \end{macrocode}
%\iffalse
%</samplemain>
%\fi
%
% %%%%%%%%%%%%%%%%%%%%%%%%%%%%%%%%%%%%%%
% \paragraph{Chapter Include Files.}
%
% The include files are called |cdocsch1.tex| and |cdocsch2.tex|.
%
%\iffalse
%<*samplechap1|samplechap2>
%\fi

% Optional override for |\version| flag:
%    \begin{macrocode}
%%\providecommand{\version}{final}
%    \end{macrocode}

% Include the main document:
%    \begin{macrocode}
\input{childdoc.def}
\childdocof{cdocsamp}
%    \end{macrocode}

%\iffalse
%</samplechap1|samplechap2>
%\fi
%
%\iffalse
%<*samplechap1>
%\fi
% Some text for chapter 1:
%    \begin{macrocode}
\section{one}
some text in chapter one
%    \end{macrocode}

%\iffalse
%</samplechap1>
%\fi
% Some text for chapter 2:
%\iffalse
%<*samplechap2>
%\fi
%    \begin{macrocode}
\section{two}
more text in chapter two
%    \end{macrocode}

%\iffalse
%</samplechap2>
%\fi
%
% %%%%%%%%%%%%%%%%%%%%%%%%%%%%%%%%%%%%%%
% \paragraph{Part Include Files.}
%
% The include files are called |cdocspt3.tex| and |cdocspt4.tex|.
%
%\iffalse
%<*samplepart3|samplepart4>
%\fi

% Optional override for |\version| flag:
%    \begin{macrocode}
%%\providecommand{\version}{final}
%    \end{macrocode}

% Include the main document:
%    \begin{macrocode}
\input{childdoc.def}
\childdocby{cdocsamp}
%    \end{macrocode}

%\iffalse
%</samplepart3|samplepart4>
%\fi
%
%\iffalse
%<*samplepart3>
%\fi
% Some text for part 3:
%    \begin{macrocode}
some text in part three
%    \end{macrocode}

%\iffalse
%</samplepart3>
%\fi
% Some text for part 4:
%\iffalse
%<*samplepart4>
%\fi
%    \begin{macrocode}
more text in part four
%    \end{macrocode}

%\iffalse
%</samplepart4>
%\fi
%
% %%%%%%%%%%%%%%%%%%%%%%%%%%%%%%%%%%%%%%
% \paragraph{Forwarding for a Complete Draft.}
%
% The following forwarding file |cdocsdrf.tex|
% compiles the main document in draft mode:
%\iffalse
%<*sampledraft>
%\fi
%    \begin{macrocode}
\def\version{draft}
\input{childdoc.def}
\childdocforward{cdocsamp}
%    \end{macrocode}

%\iffalse
%</sampledraft>
%\fi
%
% %%%%%%%%%%%%%%%%%%%%%%%%%%%%%%%%%%%%%%
% \paragraph{Forwarding for Final Version of the Chapters.}
%
% The following forwarding files |cdocsfn1.tex| and |cdocsfn2.tex|
% (with identical content)
% compile the final versions of the child documents
% |cdocsch1.tex| and |cdocsch2.tex|, respectively:
%\iffalse
%<*samplefinal>
%\fi
%    \begin{macrocode}
\def\version{final}
\input{childdoc.def}
\childdocforwardprefix[cdocsamp]{cdocsfn}{cdocsch}
%    \end{macrocode}

%\iffalse
%</samplefinal>
%\fi
%
% %%%%%%%%%%%%%%%%%%%%%%%%%%%%%%%%%%%%%%
% \paragraph{Command Line Processing.}
%
% The following three command lines generate the output files
% |cdocscld|, |cdocscl1| and |cdocscl2|
% which should be identical to
% |cdocsdrf|, |cdocsch1| and |cdocsfn2|, respectively:
% \begin{center}
% \begin{tabular}{l}
% |latex -jobname cdocscld \|\\
% |  "\def\version{draft}\input{childdoc.def}\childdocforward{cdocsamp}"|\\
% |latex -jobname cdocscl1 \|\\
% |  "\input{childdoc.def}\childdocforward[cdocsamp]{cdocsch1}"|\\
% |latex -jobname cdocscl2 \|\\
% |  "\def\version{final}\input{childdoc.def}\childdocforward{cdocsch2}"|
% \end{tabular}
% \end{center}
% Note that the trailing backslash on each first line
% merely continues the input to the second line
% (for convenient cut ant paste).
% Furthermore, the command |latex| can be replaced by any
% of its alternative versions such as |pdflatex|.
%
% %%%%%%%%%%%%%%%%%%%%%%%%%%%%%%%%%%%%%%%%%%%%%%%%%%%%%%%%%%%%%%%%%%%%%%%%%%%%%%
% %%%%%%%%%%%%%%%%%%%%%%%%%%%%%%%%%%%%%%%%%%%%%%%%%%%%%%%%%%%%%%%%%%%%%%%%%%%%%%
% \section{Implementation}
%\iffalse
%<*package>
%\fi
%
% This section describes the definitions file |childdoc.def|.

% The definitions cannot be loaded using |\usepackage| or |\RequirePackage|
% which has a mechanism to prevent loading a style file more than once.
% When loading the definitions by means of |\input|
% multiple instances have to be prevented manually:
%\iffalse
%This code needs to be before the `\ProvidesFile' directive
%which is defined at the beginning of this file.
%Therefore it is also placed there and commented out here.
%</package>
%<*discard>
%\fi
%    \begin{macrocode}
\ifdefined\childdocmain\endinput\fi
%    \end{macrocode}
%\iffalse
%</discard>
%<*package>
%\fi
%
% \macro{\ifchilddoc}
% \macro{\ifchilddocmanual}
% The conditional |\ifchilddoc| tells whether a
% child (true) or main (false) document is being compiled.
% The conditional |\ifchilddocmanual| tells whether
% the |\includeonly| mechanism is used (false) or
% the selection of child files must be performed manually (true).
% The definitions initialise to false:
%    \begin{macrocode}
\newif\ifchilddoc
\newif\ifchilddocmanual
%    \end{macrocode}

% \macro{\childdocname}
% \macro{\childdocjob}
% The macro |\childdocname| stores the name of the main document
% to be compiled. The macro |\childdocjob| stores the name of
% the document on which the \LaTeX{} compiler was originally invoked.
% The content of |\jobname| cannot be compared
% to filenames specified in the source due to different catcodes.
% The following code rescans |\jobname|, stores the result
% in |\childdocname| and saves a copy in |\childdocjob|:
%    \begin{macrocode}
\edef\childdocname{\scantokens\expandafter{\jobname\noexpand}}
\let\childdocjob\childdocname
%    \end{macrocode}

% \macro{\childdocdisable}
% The macro |\childdocdisable| prevents the main file
% from being processed more than once.
% At this stage, the main document command |\childdocmain|
% is assumed to be called once again where it should do nothing.
% Any subsequent call to it should prevent
% a secondary processing of the main document
% It overwrites the forwarding commands
% |\childdocof| and |\childdocforward|
% with empty macros to prevent further inclusions of the main document:
%    \begin{macrocode}
\newcommand{\childdocdisable}
{
  \renewcommand{\childdocmain}[1]{\renewcommand{\childdocmain}[1]{\endinput}}
  \renewcommand{\childdocof}[1]{}
  \renewcommand{\childdocby}[2][]{}
  \renewcommand{\childdocforward}[2][]{}
  \renewcommand{\childdocdisable}{}
}
%    \end{macrocode}

% \macro{\childdocmain}
% The macro |\childdocmain| is to be called at the top of the main file
% with nothing or the main filename (without extension) as argument.
% First, it breaks loops.
% If the argument is not empty and does not match |\childdocname|
% (which is set by the first inclusion of |childdoc.def|),
% |\ifchilddoc| is set to true, |\includeonly| is applied to the child file
% and |\jobname| is set to the main file
% (for proper handling of |.aux| files):
%    \begin{macrocode}
\newcommand{\childdocmain}[1]
{
  \childdocdisable\childdocmain{}
  \if?#1?\else
    \begingroup
      \def\childdoctmp{#1}
      \ifx\childdoctmp\childdocname
        \def\childdoctmp{}
      \else
        \def\childdoctmp
        {
          \childdoctrue
          \includeonly{\childdocname}
          \def\childdocjob{#1}
          \def\jobname{#1}
        }
      \fi
      \expandafter
    \endgroup
    \childdoctmp
  \fi
}
%    \end{macrocode}

% \macro{\childdocof}
% The command |\childdocof| redirects
% compilation to the main file |#1|.
%    \begin{macrocode}
\newcommand{\childdocof}[1]
{
  \childdocdisable
  \childdoctrue
  \includeonly{\childdocname}
  \def\jobname{#1}
  \def\childdocjob{#1}
  \input{#1}
}
%    \end{macrocode}

% \macro{\childdocby}
% The command |\childdocby| ....
%    \begin{macrocode}
\newcommand{\childdocby}[2][]
{
  \childdocdisable
  \childdoctrue
  \childdocmanualtrue
  \if?#1?\else
    \def\jobname{#2}
  \fi
  \def\childdocjob{#2}
  \input{#2}
  \endinput
}
%    \end{macrocode}

% \macro{\childdocforward}
% The command |\childdocforward| redirects
% compilation to the main file or
% (if the optional argument is given) a child file.
% Parameters are set as if the main file
% or a child file starting with |\childdocof| was compiled.
% Then compilation is handed over to the main file:
%    \begin{macrocode}
\newcommand{\childdocforward}[2][]
{
  \begingroup
    \if?#1?
      \def\childdoctmp
      {
        \def\childdocname{#2}
        \def\childdocjob{#2}
        \def\jobname{#2}
        \input{#2}
        \endinput
      }
    \else
      \def\childdoctmp
      {
        \childdocdisable
        \def\childdocname{#2}
        \childdoctrue
        \includeonly{#2}
        \def\childdocjob{#1}
        \def\jobname{#1}
        \input{#1}
        \endinput
      }
    \fi
    \expandafter
  \endgroup
  \childdoctmp
}
%    \end{macrocode}

% \macro{\childdocforwardprefix}
% The command |\childdocforwardprefix| redirects
% compilation to the main or a child file by means of a pattern.
% The prefix |#1| in the current filename is replaced by |#2|
% and the suffix of the current filename is kept
% (it is assumed that the filename does not contain the substring `|~~~|'
% which is used as a delimiter).
% Compilation is handed over to the new file by |\childdocforward|:
%    \begin{macrocode}
\newcommand{\childdocforwardprefix}[3][]
{
  \begingroup
    \def\childdocextract #2##1~~~{\def\childdoctmp{\childdocforward[#1]{#3##1}}}
    \expandafter\childdocextract\childdocname~~~
    \expandafter
  \endgroup
  \childdoctmp
}
%    \end{macrocode}

% \macro{\childdoc}
% The deprecated macro |\childdoc| is a legacy version of |\childdocmain|:
%    \begin{macrocode}
\newcommand{\childdoc}{\childdocmain}
%    \end{macrocode}

% \macro{\childdocredirect}
% The deprecated macro |\childdocredirect| is a legacy version
% of |\childdocforward| and |\childdocforwardprefix|:
%    \begin{macrocode}
\newcommand{\childdocredirect}[2][]
{
  \begingroup
    \if?#1?
      \def\childdoctmp{\childdocforward{#2}}
    \else
      \def\childdoctmp{\childdocforwardprefix{#1}{#2}}
    \fi
    \expandafter
  \endgroup
  \childdoctmp
}
%    \end{macrocode}

%\iffalse
%</package>
%\fi
%
\endinput
|\\
|\childdocforward{|\textit{main}|}|\\
\end{tabular}
\end{center}
%
or alternatively with:
%
\begin{center}
\begin{tabular}{l}
|% \iffalse
%
% childdoc.dtx Copyright (C) 2017-2018 Niklas Beisert
%
% This work may be distributed and/or modified under the
% conditions of the LaTeX Project Public License, either version 1.3
% of this license or (at your option) any later version.
% The latest version of this license is in
%   http://www.latex-project.org/lppl.txt
% and version 1.3 or later is part of all distributions of LaTeX
% version 2005/12/01 or later.
%
% This work has the LPPL maintenance status `maintained'.
%
% The Current Maintainer of this work is Niklas Beisert.
%
% This work consists of the files childdoc.dtx and childdoc.ins
% and the derived files childdoc.def and cdocsamp.tex with
% cdocsch1.tex, cdocsch2.tex, cdocsdrf.tex, cdocsfn1.tex, cdocsfn2.tex.
%
%<package>\ifdefined\childdocmain\endinput\fi
%<package>\ProvidesFile{childdoc.def}[2018/12/30 v2.0 child document driver]
%<samplemain>\ProvidesFile{cdocsamp.tex}[2018/12/30 v2.0 sample for childdoc]
%<*driver>
%\ProvidesFile{childdoc.drv}[2018/12/30 v2.0 childdoc reference manual file]
\PassOptionsToClass{10pt,a4paper}{article}
\documentclass{ltxdoc}

\usepackage[margin=35mm]{geometry}
\usepackage{hyperref}
\usepackage{hyperxmp}
\usepackage[usenames]{color}

\hypersetup{colorlinks=true}
\hypersetup{pdfstartview=FitH}
\hypersetup{pdfpagemode=UseNone}
\hypersetup{pdfsource={}}
\hypersetup{pdflang={en-UK}}
\hypersetup{pdfcopyright={Copyright 2017-2018 Niklas Beisert.
  This work may be distributed and/or modified under the
  conditions of the LaTeX Project Public License, either version 1.3
  of this license or (at your option) any later version.}}
\hypersetup{pdflicenseurl={http://www.latex-project.org/lppl.txt}}
\hypersetup{pdfcontactaddress={ETH Zurich, ITP, HIT K,
  Wolfgang-Pauli-Strasse 27}}
\hypersetup{pdfcontactpostcode={8093}}
\hypersetup{pdfcontactcity={Zurich}}
\hypersetup{pdfcontactcountry={Switzerland}}
\hypersetup{pdfcontactemail={nbeisert@itp.phys.ethz.ch}}
\hypersetup{pdfcontacturl={http://people.phys.ethz.ch/\xmptilde nbeisert/}}

\newcommand{\secref}[1]{\hyperref[#1]{section \ref*{#1}}}

\parskip1ex
\parindent0pt
\let\olditemize\itemize
\def\itemize{\olditemize\parskip0pt}

\begin{document}

\title{The \textsf{childdoc} Package}
\hypersetup{pdftitle={The childdoc Package}}
\author{Niklas Beisert\\[2ex]
  Institut f\"ur Theoretische Physik\\
  Eidgen\"ossische Technische Hochschule Z\"urich\\
  Wolfgang-Pauli-Strasse 27, 8093 Z\"urich, Switzerland\\[1ex]
  \href{mailto:nbeisert@itp.phys.ethz.ch}
  {\texttt{nbeisert@itp.phys.ethz.ch}}}
\hypersetup{pdfauthor={Niklas Beisert}}
\hypersetup{pdfsubject={Manual for the LaTeX2e Package childdoc}}
\date{30 December 2018, \textsf{v2.0}}
\maketitle

\begin{abstract}\noindent
\textsf{childdoc} is a \LaTeXe{} package
that enables the direct compilation
of document sections included by |\include|
to individual files.
\end{abstract}

\begingroup
\parskip0ex
\tableofcontents
\endgroup

%%%%%%%%%%%%%%%%%%%%%%%%%%%%%%%%%%%%%%%%%%%%%%%%%%%%%%%%%%%%%%%%%%%%%%%%%%%%%%%%
%%%%%%%%%%%%%%%%%%%%%%%%%%%%%%%%%%%%%%%%%%%%%%%%%%%%%%%%%%%%%%%%%%%%%%%%%%%%%%%%
\section{Introduction}

\LaTeX{} provides a mechanism to structure a large document (such as a book)
into a main file and several child files (containing the chapters)
using the |\include| command.
This mechanism is beneficial for documents
which span hundreds of pages in order to
make the source file(s) more manageable.
Moreover, compilation can be restricted to
selected child files by means of the |\includeonly| command.
The latter feature can be used to reduce the compilation time while editing
(this was significantly more useful in the earlier days of \LaTeX{})
or to generate a smaller document which is easier to navigate.
Another application of |\includeonly| is to generate
documents consisting of selected parts of the complete document.

However, there are a few drawbacks of the plain |\include| mechanism:
\begin{itemize}
\item
The child files cannot be compiled on their own,
they can only be compiled via the main file.
A naive editing environment
(such as a text editor with an option
to have the current file processed by \LaTeX)
may require one to switch to the main file before compiling;
attempting to compile the child file produces errors.
\item
The main file must be modified (each time)
to adjust the |\includeonly| command
to the present needs. This easily leaves the main file in a messy state.
\item
The generated document will always carry the filename
of the main document. This is inconvenient if
several child files are to be compiled and
to be kept for distribution.
\end{itemize}

The present package provides a simple interface
to make child files individually compilable by \LaTeX{}.
Compiling a child file then has the same effect as compiling
the main file with an |\includeonly| command
to select the appropriate child.
Moreover the generated document will carry the name of the child
rather than the main file.
This resolves all three above issues.

This feature is meant to make the editing of books,
thesis documents and lecture notes somewhat more convenient.
However, the package can also be used efficiently for
composing a series of documents (such as exercise sheets)
which are typically distributed individually.
It then assists the author in generating the individual documents
(potentially in different versions)
as well as a document containing the collected series.
Another application is in developing style files
or other kinds of included material
where compilation of the style file could redirect
to a sample or test file.

%%%%%%%%%%%%%%%%%%%%%%%%%%%%%%%%%%%%%%%%%%%%%%%%%%%%%%%%%%%%%%%%%%%%%%%%%%%%%%%%
%%%%%%%%%%%%%%%%%%%%%%%%%%%%%%%%%%%%%%%%%%%%%%%%%%%%%%%%%%%%%%%%%%%%%%%%%%%%%%%%
\section{Usage}

First of all, the package \textsf{childdoc} is \emph{not} a standard
\LaTeXe{} |.sty| style file! Therefore it needs to be invoked in
a non-standard way.

%%%%%%%%%%%%%%%%%%%%%%%%%%%%%%%%%%%%%%%%%%%%%%%%%%%%%%%%%%%%%%%%%%%%%%%%%%%%%%%%
\subsection{Included Files}
\label{sec:include}

%%%%%%%%%%%%%%%%%%%%%%%%%%%%%%%%%%%%%%%%
\DescribeMacro{\childdocmain}
To use the package, add the commands
\begin{center}
\begin{tabular}{l}
|\input{childdoc.def}|\\
|\childdocmain{}|\\
\end{tabular}
\end{center}
at the very top of the main \LaTeX{} file,
in particular \emph{before} the |\documentclass| statement!
The argument of |\childdocmain| should be left empty
(but it must be present).

%%%%%%%%%%%%%%%%%%%%%%%%%%%%%%%%%%%%%%%%
\DescribeMacro{\childdocof}
Furthermore, add the commands
\begin{center}
\begin{tabular}{l}
|\input{childdoc.def}|\\
|\childdocof{|\textit{main}|}|\\
\end{tabular}
\end{center}
at the top of every child file \textit{child}
which is included by |\include{|\textit{child}|}|
from within the main file
(or at least for those files to be compiled individually).
The argument \textit{main} must be the filename of the main file.

There are a couple of
considerations in setting up the main and child documents:

%%%%%%%%%%%%%%%%%%%%%%%%%%%%%%%%%%%%%%%%
\paragraph{Restrictions.}

Please note the following restrictions:
\begin{itemize}
\item
|\childdocmain| must be called with one argument \textit{main}
to ensure compatibility with earlier version of the package.
It must either be empty (|\childdocmain{}|)
or precisely match the filename of the main file in which it is specified.
See \secref{sec:detection} for further information.
\item
The filename \textit{main} must be specified without the |.tex| extension.
\item
The filename \textit{main} is case sensitive
(even in case-insensitive file systems)
due to internal string comparison.
\item
The argument \textit{main} should be fully expanded, it cannot be a macro.
\item
Subdirectories and special characters should be avoided in filenames.
\item
The command |\childdocmain{|\textit{main}|}| must be followed by a whitespace.
It should not be followed immediately by another command
or by a comment mark `|%|'.
This is because the \TeX{} parser reads the token immediately following
the argument of |\childdocmain| and puts it
at the beginning of every child section;
however, a white\-space is ignored.
\end{itemize}

%%%%%%%%%%%%%%%%%%%%%%%%%%%%%%%%%%%%%%%%
\paragraph{Content of Main File.}

It is advisable to place all content in the child files included by |\include|.
Any output contained in the main file will appear in all child documents
unless suppressed manually;
it cannot be suppressed automatically by the |\includeonly| directive
and thus should normally be avoided.
A method to include some content in the main file
by means of conditional processing is described in \secref{sec:conditional}.

%%%%%%%%%%%%%%%%%%%%%%%%%%%%%%%%%%%%%%%%
\paragraph{Page Numbering.}

When only a part of the document is compiled,
the appropriate numbering of pages
(as well as other status parameters)
is determined from the |.aux| files.
The latter contain information from previous passes.
However this information needs to propagate through
all intermediate child documents.
Therefore the page numbering in child documents may well
be inconsistent until the complete document is compiled at least once.

A useful (if unconventional) way to always ensure a consistent
page numbering is to restart the numbering in each child document
and denote the pages by `\textit{child}|.|\textit{page}'
where \textit{child} represents the chapter/section number of the child file.
This can be achieved by the command
|\numberwithin{page}{|\textit{child}|}|
of the \textsf{amsmath} package
where \textit{child} can be |chapter| or |section|
depending on the chosen structuring.
Alternatively, one can modify the macro |\thepage| appropriately
and reset the counter |page| at the start of each child file.

%%%%%%%%%%%%%%%%%%%%%%%%%%%%%%%%%%%%%%%%%%%%%%%%%%%%%%%%%%%%%%%%%%%%%%%%%%%%%%%%
\subsection{Conditional Processing}
\label{sec:conditional}

The package provides a mechanism to compile different versions
of a document. To customise the versions further some conditional processing
can come in handy to distinguish which version is being compiled.
The package provides two macros to describe the compilation context:

%%%%%%%%%%%%%%%%%%%%%%%%%%%%%%%%%%%%%%%%
\DescribeMacro{\ifchilddoc}
The conditional |\ifchilddoc| distinguishes between the compilation of
child documents and the main document:
%
\begin{center}
|\ifchilddoc |\textit{child-code}| |[|\||else |\textit{main-code}]| \||fi|
\end{center}

%%%%%%%%%%%%%%%%%%%%%%%%%%%%%%%%%%%%%%%%
\DescribeMacro{\childdocname}
\DescribeMacro{\childdocjob}
The macro |\childdocname| contains the filename (without extension)
of the main or child file being processed.
Note that |\childdocjob| will always contain the name of the main file.

%%%%%%%%%%%%%%%%%%%%%%%%%%%%%%%%%%%%%%%%
\paragraph{Title Page.}

Conditional processing can be used to include a title or banner page
in the main document when proper precautions are taken.
Importantly, the code in the main file should ensure that the page counter
(as well as other status parameters which are stored in the |.aux| files)
takes the same value after the conditional processing.
Otherwise the page numbers may take divergent values
depending on which part is compiled.

For example, a title page could be declared by:
%
\begin{center}
\begin{tabular}{l}
|\ifchilddoc\||else|\\
|\addtocounter{page}{-1}|\\
\textit{code for title page}\\
|\newpage|\\
|\||fi|
\end{tabular}
\end{center}
%
A banner page for the child documents can be generated by:
%
\begin{center}
\begin{tabular}{l}
|\ifchilddoc|\\
|\addtocounter{page}{-1}|\\
\textit{code for banner page}\\
|\newpage|\\
|\||fi|
\end{tabular}
\end{center}
%
Here one could write a message such as:
\begin{center}
|This is the part \childdocname{} of \childdocjob{}.|
\end{center}

%%%%%%%%%%%%%%%%%%%%%%%%%%%%%%%%%%%%%%%%%%%%%%%%%%%%%%%%%%%%%%%%%%%%%%%%%%%%%%%%
\subsection{Flags}
\label{sec:flags}

The package makes it easy to generate different versions
of the main or child documents.
To this end compilation flags can be defined
and assigned different default values.
They will be particularly useful in conjunction
with the forwarding mechanism described in \secref{sec:forward}.

For example, it may be useful to have a flag |\version|
which can be set to |draft| or |final|.
The document source will contain some conditional code
depending on the value of |\version|.
Suppose further, the flag should default to |final| for the main file
and to |draft| for child files
which is a natural assignment for editing the document.
This is achieved by placing the following code
in the preamble of the main document
(below the |\childdocmain| directive):
%
\begin{center}
\begin{tabular}{l}
|\ifchilddoc|\\
|\providecommand{\version}{draft}|\\
|\||else|\\
|\providecommand{\version}{final}|\\
|\||fi|
\end{tabular}
\end{center}
%
The definition by |\providecommand| makes sure
that previous definitions are not overwritten.
Further statements |\providecommand{\version}{...}|
can thus be added before the above code to override it.

For the main file, one might add a line
(between |\childdocmain| and the above block)
%
\begin{center}
|%\ifchilddoc\||else\providecommand{\version}{draft}\||fi|
\end{center}
%
which can be uncommented to produce a draft version.
Likewise one can add a line to the very top of a child file
(above the |\childdocof{|\textit{main}|}| directive)
%
\begin{center}
|%\providecommand{\version}{final}|
\end{center}
%
which can be uncommented to produce the final version of this child document.

%%%%%%%%%%%%%%%%%%%%%%%%%%%%%%%%%%%%%%%%%%%%%%%%%%%%%%%%%%%%%%%%%%%%%%%%%%%%%%%%
\subsection{Forwarding}
\label{sec:forward}

Different versions of the main or child documents
using compilation flags as described in \secref{sec:flags}
can be (permanently) stored in different files
for convenient compilation, viewing and distribution.
To this end, the package defines a command
to pass on compilation to a different file:

%%%%%%%%%%%%%%%%%%%%%%%%%%%%%%%%%%%%%%%%
\DescribeMacro{\childdocforward}
The command |\childdocforward| redirects processing to
another source file:
%
\begin{center}
\begin{tabular}{l}
|\input{childdoc.def}|\\
|\childdocforward[|\textit{main}|]{|\textit{dest}|}|\\
\end{tabular}
\end{center}
%
The argument \textit{dest} is the destination file
(without extension).
It should be the main file or one of the child files.
Note that further \textsf{childdoc} directives
such as |\childdocof| and |\childdocforward|
in the indicated file will be processed in this form.
The optional argument \textit{main}
passes on directly to the main file \textit{main}
while pretending to compile the child \textit{dest}.
This form behaves as if \textit{dest}
issues |\childdocof{|\textit{main}|}| right away,
and no further \textsf{childdoc} directives will be processed.

%%%%%%%%%%%%%%%%%%%%%%%%%%%%%%%%%%%%%%%%
\DescribeMacro{\...prefix}
In the alternative form |\childdocforwardprefix|,
%
\begin{center}
\begin{tabular}{l}
|\input{childdoc.def}|\\
|\childdocforwardprefix[|\textit{main}|]{|\textit{prefix}|}{|\textit{dest}|}|
\end{tabular}
\end{center}
%
the destination file is determined by a pattern
depending on the current file:
To make this work, the current file must be called
`{\textit{prefix}\hspace{0.2em}\textit{suffix}}'
with \textit{prefix} matching precisely the argument.
Processing is then passed on to the file
`{\textit{dest}\hspace{0.2em}\textit{suffix}}'.
Surely, the same effect is achieved by
directly specifying the
argument `{\textit{dest}\hspace{0.2em}\textit{suffix}}'
in the first form.
However, that requires to set up a different file
for each child. With the alternative form of the command
all these files can have exactly the same content
which simplifies setting them up and maintaining them.

For example, the following file |draft.tex|
with a compilation flag |\version| as described in \secref{sec:flags}
compiles the main document as a draft:
%
\begin{center}
\begin{tabular}{l}
|\def\version{draft}|\\
|\input{childdoc.def}|\\
|\childdocforward{|\textit{main}|}|
\end{tabular}
\end{center}
%
Likewise, the following files |final|\textit{nn}|.tex|
compile the final version of the child document
|child|\textit{nn}|.tex|:
%
\begin{center}
\begin{tabular}{l}
|\def\version{final}|\\
|\input{childdoc.def}|\\
|\childdocforwardprefix{final}{child}|
\end{tabular}
\end{center}
%

Note that when several versions of a main file and/or of each child file
are to be generated, it may be convenient to set up a |Makefile| or
shell script to automatise the process.

%%%%%%%%%%%%%%%%%%%%%%%%%%%%%%%%%%%%%%%%%%%%%%%%%%%%%%%%%%%%%%%%%%%%%%%%%%%%%%%%
\subsection{Command Line Processing}
\label{sec:commandline}

The effect of redirection files can also be achieved by invoking
the \LaTeX{} compiler with a more elaborate command line.
Most conveniently this should be done as part
of a shell script or a |Makefile|.

When using \textsf{childdoc} in the main file, the following
command lines effectively perform a redirection
(note that depending on the shell being used,
backslashes may have to be doubled: `|\|' $\to$ `|\\|'):
%
\begin{center}
|... -jobname "|\textit{target}|" |\\|"|[\textit{flags}]%
|\input{childdoc.def}\childdocforward[|\textit{main}|]{|\textit{dest}|}"|
\end{center}
%
Here \textit{target} is the name of the output file,
\textit{main} is the name of the main file
and \textit{dest} is the name of the main or child file to be processed
(all filenames without extensions).
The optional argument \textit{main} can be omitted
if \textit{main} matches \textit{dest}.
Optionally, compilation \textit{flags} can be defined via |\def| commands.
This command line makes the \TeX{} engine believe
it is compiling the file \textit{target}
whose content is specified as the latter parameter.
The provided code then forwards the processing to
\textit{main} or \textit{dest} as described in \secref{sec:forward}.

%%%%%%%%%%%%%%%%%%%%%%%%%%%%%%%%%%%%%%%%%%%%%%%%%%%%%%%%%%%%%%%%%%%%%%%%%%%%%%%%
\subsection{Include by Input}
\label{sec:input}

Including child documents by |\include| has some restrictions by design.
Most notably, the content of a child document always occupies
its own set of pages; pages cannot be shared between child documents.
Usually, this behaviour makes perfect sense
because each child document contain an essential part of the document.
However, in some situations it may be desirable to compose
a document from a collection of parts
without having mandatory page breaks between then.
For this case, the package
provides a mechanism to include parts
by |\input| which can also be processed individually.
However, by construction this mechanism
requires manual handling of the content to be output.

%%%%%%%%%%%%%%%%%%%%%%%%%%%%%%%%%%%%%%%%
\DescribeMacro{\ifchilddocmanual}
The main file should be prepared as usual, see \secref{sec:include}.
However, the document body must make a distinction
between processing of an individual part and of the main document, e.g.:
%
\begin{center}
\begin{tabular}{l}
|\ifchilddocmanual|\\
|\input{\childdocname}|\\
|\||else|\\
\textit{document body with }|\input{|\textit{part}|}|\\
|\||fi|
\end{tabular}
\end{center}
%
The conditional |\ifchilddocmanual| is true whenever
a part to be included by |\input| is being compiled,
and the name of the part is stored in |\childdocname|.

%%%%%%%%%%%%%%%%%%%%%%%%%%%%%%%%%%%%%%%%
\DescribeMacro{\childdocby}
Each part to be included by |\input| should start with:
%
\begin{center}
\begin{tabular}{l}
|\input{childdoc.def}|\\
|\childdocby{|\textit{main}|}|\\
\end{tabular}
\end{center}
%
The directive |\childdocby| is similar to |\childdocof|
described in \secref{sec:include},
but the subsequent selection of content must be done manually.
To that end, both |\ifchilddoc| and |\ifchilddocmanual|
will be true upon processing of a part,
and the name of the part is stored in |\childdocname|.
Note that |\jobname| will be set to the filename of the current part
so that each part receives an individual |.aux| file
that does not interfere with the |.aux| file(s) of the main document.
This behaviour can be altered by the alternative form
|\childdocby[*]{|\textit{main}|}| (with a non-empty optional argument)
which uses the |.aux| file of the main document
by setting |\jobname| to \textit{main}.

%%%%%%%%%%%%%%%%%%%%%%%%%%%%%%%%%%%%%%%%%%%%%%%%%%%%%%%%%%%%%%%%%%%%%%%%%%%%%%%%
\subsection{Driver Development}
\label{sec:driver}

The \textsf{childdoc} mechanism can also be use for the development
of definition files such as \LaTeX{} styles or classes.
This case differs from the above setup with multiple parts
included by |\include| in that no |\includeonly| should be invoked.
This can be achieved by starting the include file
(before |\ProvidesPackage|) with:
%
\begin{center}
\begin{tabular}{l}
|\input{childdoc.def}|\\
|\childdocforward{|\textit{main}|}|\\
\end{tabular}
\end{center}
%
or alternatively with:
%
\begin{center}
\begin{tabular}{l}
|\input{childdoc.def}|\\
|\childdocby{|\textit{main}|}|\\
\end{tabular}
\end{center}
%
Both forms have slightly different effects as described above.
The main file is prepared as usual, see \secref{sec:include}.

%%%%%%%%%%%%%%%%%%%%%%%%%%%%%%%%%%%%%%%%%%%%%%%%%%%%%%%%%%%%%%%%%%%%%%%%%%%%%%%%
\subsection{Legacy Detection}
\label{sec:detection}

The directive |\childdocmain| in the main file can detect
whether the complete document or merely a child is to be compiled
even without using the directive |\childdocof|.
This method is deprecated because it is less robust
and there is no compelling reason to use it;
it is merely provided for backward compatibility
and it may be removed in future versions.

If the detection mechanism is to be used,
it is mandatory to correctly specify
the filename of the main file as the argument of |\childdocmain|:
%
\begin{center}
\begin{tabular}{l}
|\input{childdoc.def}|\\
|\childdocmain{|\textit{main}|}|\\
\end{tabular}
\end{center}
%
If |\jobname| does not match the argument \textit{main} of |\childdocmain|,
it is assumed that |\jobname| points to the child file to be compiled.
When using |\childdocmain| with the main file specified as argument,
it suffices to start a child file
with just |\input{|\textit{main}|}|
without loading of the package and using |\childdocof|.
If instead all processing is done
with the appropriate \textsf{childdoc} directives,
the argument of \textit{main} of |\childdocmain| can be empty.

An alternative version of the command line processing described
in \secref{sec:commandline} using the detection mechanism reads:
%
\begin{center}
|... -jobname "|\textit{target}|" "|[\textit{flags}]%
[|\def\jobname{|\textit{dest}|}|]|\input{|\textit{main}|}"|
\end{center}

%%%%%%%%%%%%%%%%%%%%%%%%%%%%%%%%%%%%%%%%%%%%%%%%%%%%%%%%%%%%%%%%%%%%%%%%%%%%%%%%
\subsection{Manual Code}
\label{sec:manual}

In case one cannot be certain whether the definitions file |childdoc.def|
is installed on the target \TeX{} distribution
and one prefers not to ship it,
it is conceivable to paste a few relevant commands into the sources.

To that end, drop all statements |\input{childdoc.def}|
and perform the replacements as outlined below.
Instead of |\childdocmain{|\textit{main}|}| add the following code
to the top of the main file:
%
\begin{center}
\begin{tabular}{l}
|\||ifdefined\childdocname\endinput\||fi\newif\ifchilddoc|\\
|\edef\childdocname{\scantokens\expandafter{\jobname\noexpand}}|\\
|\def\childdocmain{|\textit{main}|}\||ifx\childdocmain\childdocname\||else|\\
|\childdoctrue\includeonly{\childdocname}\let\jobname\childdocmain\||fi|\\
\end{tabular}
\end{center}
%
Instead of |\childdocof{|\textit{main}|}| just include the main file
at the top of each child file:
%
\begin{center}
|\input{|\textit{main}|}|
\end{center}
%
A simple redirection |\childdocforward{|\textit{dest}|}| is achieved by:
%
\begin{center}
|\def\jobname{|\textit{dest}|}\input{\jobname}|
\end{center}
%
The redirection with prefix
|\childdocforwardprefix[|\textit{prefix}|]{|\textit{dest}|}|
is accomplished by:
%
\begin{center}
\begin{tabular}{l}
|{\edef\jobname{\scantokens\expandafter{\jobname\noexpand}}|\\
|\def\redirectjob |\textit{prefix}|#1~~~{\gdef\jobname{|\textit{dest}|#1}}|\\
|\expandafter\redirectjob\jobname~~~}\input{\jobname}|
\end{tabular}
\end{center}

In an alternative approach,
child documents can be compiled by a specific command line
without additional code or specific definitions:
%
\begin{center}
|... -jobname "|\textit{target}|" "|[\textit{flags}]%
|\includeonly{|\textit{dest}|}\input{|\textit{main}|}"|
\end{center}
%

%%%%%%%%%%%%%%%%%%%%%%%%%%%%%%%%%%%%%%%%%%%%%%%%%%%%%%%%%%%%%%%%%%%%%%%%%%%%%%%%
%%%%%%%%%%%%%%%%%%%%%%%%%%%%%%%%%%%%%%%%%%%%%%%%%%%%%%%%%%%%%%%%%%%%%%%%%%%%%%%%
\section{Information}

%%%%%%%%%%%%%%%%%%%%%%%%%%%%%%%%%%%%%%%%%%%%%%%%%%%%%%%%%%%%%%%%%%%%%%%%%%%%%%%%
\subsection{Copyright}

Copyright \copyright{} 2017--2018 Niklas Beisert

This work may be distributed and/or modified under the
conditions of the \LaTeX{} Project Public License, either version 1.3
of this license or (at your option) any later version.
The latest version of this license is in
  \url{http://www.latex-project.org/lppl.txt}
and version 1.3 or later is part of all distributions of \LaTeX{}
version 2005/12/01 or later.

This work has the LPPL maintenance status `maintained'.

The Current Maintainer of this work is Niklas Beisert.

This work consists of the files |README.txt|, |childdoc.ins| and |childdoc.dtx|
as well as the derived files |childdoc.def|, |cdocsamp.tex|
with |cdocsch1.tex|, |cdocsch2.tex|, |cdocspt3.tex|, |cdocspt4.tex|,
|cdocsdrf.tex|, |cdocsfn1.tex|, |cdocsfn2.tex|
as well as |childdoc.pdf|.

%%%%%%%%%%%%%%%%%%%%%%%%%%%%%%%%%%%%%%%%%%%%%%%%%%%%%%%%%%%%%%%%%%%%%%%%%%%%%%%%
\subsection{Files and Installation}

The package consists of the files:
%
\begin{center}
\begin{tabular}{ll}
    |README.txt|   & readme file \\
    |childdoc.ins| & installation file \\
    |childdoc.dtx| & source file \\
    |childdoc.def| & definition file \\
    |cdocsamp.tex| & sample main file \\
    |cdocsch1.tex| & sample include file \\
    |cdocsch2.tex| & sample include file \\
    |cdocspt3.tex| & sample part file \\
    |cdocspt4.tex| & sample part file \\
    |cdocsdrf.tex| & sample redirection file \\
    |cdocsfn1.tex| & sample redirection file \\
    |cdocsfn2.tex| & sample redirection file \\
    |childdoc.pdf| & manual
\end{tabular}
\end{center}
%
The distribution consists of the files
|README.txt|, |childdoc.ins| and |childdoc.dtx|.
%
\begin{itemize}
\item
Run (pdf)\LaTeX{} on |childdoc.dtx|
to compile the manual |childdoc.pdf| (this file).
\item
Run \LaTeX{} on |childdoc.ins| to create the definitions file |childdoc.def|
and the sample |cdocsamp.tex| with include files
|cdocsch1.tex|, |cdocsch2.tex|, |cdocspt3.tex|, |cdocspt4.tex|,
|cdocsdrf.tex|, |cdocsfn1.tex|, |cdocsfn2.tex|.
Then copy the file |childdoc.def| to an appropriate directory of your \LaTeX{}
distribution, e.g.\ \textit{texmf-root}|/tex/latex/childdoc|.
\end{itemize}

%%%%%%%%%%%%%%%%%%%%%%%%%%%%%%%%%%%%%%%%%%%%%%%%%%%%%%%%%%%%%%%%%%%%%%%%%%%%%%%%
\subsection{Related CTAN Packages}

There are several other packages which offer a similar functionality:
%
\begin{itemize}
\item
The packages
\href{http://ctan.org/pkg/docmute}{\textsf{docmute}},
\href{http://ctan.org/pkg/includex}{\textsf{includex}} and
\href{http://ctan.org/pkg/standalone}{\textsf{standalone}}
provide commands to include only the document body of
a child file thus allowing both files to be compiled individually.
\item
The packages \href{http://ctan.org/pkg/subdocs}{\textsf{subdocs}}
and \href{http://ctan.org/pkg/subfiles}{\textsf{subfiles}}
provide structures in which the main and child documents can be
encapsulated and allowing them to be compiled individually.
The inclusion mechanism is different from the conventional |\include|.
\item
The package \href{http://ctan.org/pkg/combine}{\textsf{combine}}
is an elaborate solution to combine several documents into one.
\end{itemize}
%
See also the CTAN topic \href{http://ctan.org/topic/subdocs}{\textsf{subdocs}}
for further related packages.
The present package differs from the above solutions in that
a document structure constructed with the conventional |\include| mechanism
just needs two extra commands at the top of every file
such that all constituent files can be compiled individually.

%%%%%%%%%%%%%%%%%%%%%%%%%%%%%%%%%%%%%%%%%%%%%%%%%%%%%%%%%%%%%%%%%%%%%%%%%%%%%%%%
%\subsection{Feature Suggestions}
%
%The following is a list of features which may be useful for future
%versions of this package:
%%
%\begin{itemize}
%\item
%\ldots
%\end{itemize}

%%%%%%%%%%%%%%%%%%%%%%%%%%%%%%%%%%%%%%%%%%%%%%%%%%%%%%%%%%%%%%%%%%%%%%%%%%%%%%%%
\subsection{Revision History}

%%%%%%%%%%%%%%%%%%%%%%%%%%%%%%%%%%%%%%%%
\paragraph{v2.0:} 2018/12/30

\begin{itemize}
\item
immediate forward processing
\item
added |\childdocby| mechanism
\item
manual restructured
\end{itemize}

%%%%%%%%%%%%%%%%%%%%%%%%%%%%%%%%%%%%%%%%
\paragraph{v1.6:} 2018/01/17

\begin{itemize}
\item
application for development of include files
\item
corrections to manual
\end{itemize}

%%%%%%%%%%%%%%%%%%%%%%%%%%%%%%%%%%%%%%%%
\paragraph{v1.5:} 2017/05/21

\begin{itemize}
\item
more complete structuring introduced
\item
|\childdocof| introduced
\item
|\childdoc| renamed to |\childdocmain|
\item
|\childredirect| renamed to |\childdocforward| and |\childdocforwardprefix|
and functionality expanded
\end{itemize}

%%%%%%%%%%%%%%%%%%%%%%%%%%%%%%%%%%%%%%%%
\paragraph{v1.0:} 2017/04/27

\begin{itemize}
\item
manual and install package
\item
first version published on CTAN
\end{itemize}

%%%%%%%%%%%%%%%%%%%%%%%%%%%%%%%%%%%%%%%%
\paragraph{v0.6:} 2017/04/26

\begin{itemize}
\item
redirection mechanism added
\end{itemize}

%%%%%%%%%%%%%%%%%%%%%%%%%%%%%%%%%%%%%%%%
\paragraph{v0.5:} 2017/04/26

\begin{itemize}
\item
functionality in definition file
\end{itemize}


%%%%%%%%%%%%%%%%%%%%%%%%%%%%%%%%%%%%%%%%%%%%%%%%%%%%%%%%%%%%%%%%%%%%%%%%%%%%%%%%
%%%%%%%%%%%%%%%%%%%%%%%%%%%%%%%%%%%%%%%%%%%%%%%%%%%%%%%%%%%%%%%%%%%%%%%%%%%%%%%%
%%%%%%%%%%%%%%%%%%%%%%%%%%%%%%%%%%%%%%%%%%%%%%%%%%%%%%%%%%%%%%%%%%%%%%%%%%%%%%%%
\appendix

\settowidth\MacroIndent{\rmfamily\scriptsize 000\ }

 \DocInput{childdoc.dtx}

\end{document}
%</driver>
% \fi
%
% %%%%%%%%%%%%%%%%%%%%%%%%%%%%%%%%%%%%%%%%%%%%%%%%%%%%%%%%%%%%%%%%%%%%%%%%%%%%%%
% %%%%%%%%%%%%%%%%%%%%%%%%%%%%%%%%%%%%%%%%%%%%%%%%%%%%%%%%%%%%%%%%%%%%%%%%%%%%%%
% \section{Sample}
%\iffalse
%<*samplemain>
%\fi
%
% The following presents a sample document
% with two chapters, two parts, a title page,
% a compile flag as well as three forwarding files to set the flag.
% It consists of eight |.tex| files:
% \begin{center}
% \begin{tabular}{ll}
% |cdocsamp.tex|&main file\\
% |cdocsch1.tex|&include file for chapter 1\\
% |cdocsch2.tex|&include file for chapter 2\\
% |cdocspt3.tex|&include file for part 3\\
% |cdocspt4.tex|&include file for part 4\\
% |cdocsdrf.tex|&forwarding file for main file in draft mode\\
% |cdocsfi1.tex|&forwarding file for final version of chapter 1\\
% |cdocsfi2.tex|&forwarding file for final version of chapter 2\\
% \end{tabular}
% \end{center}
% Each of the eight files can be compiled directly by the \LaTeX{} compiler.
%
% %%%%%%%%%%%%%%%%%%%%%%%%%%%%%%%%%%%%%%
% \paragraph{Main File.}
%
% The main file is called |cdocsamp.tex|.
%
% Load the \textsf{childdoc} definitions and
% declare the filename for the main document:
%    \begin{macrocode}
\input{childdoc.def}
\childdocmain{}
%    \end{macrocode}

% Optional override for |\version| flag:
%    \begin{macrocode}
%%\ifchilddoc\else\providecommand{\version}{draft}\fi
%    \end{macrocode}

% Define the default values for the |\version| flag
% (|final| for the main file and |draft| for childs):
%    \begin{macrocode}
\ifchilddoc
\providecommand{\version}{draft}
\else
\providecommand{\version}{final}
\fi
%    \end{macrocode}

% Load the standard document class:
%    \begin{macrocode}
\documentclass[12pt]{article}
%    \end{macrocode}

% Start the document body:
%    \begin{macrocode}
\begin{document}
%    \end{macrocode}

% Declare a title page.
% Print title, part of document being processed and version flag:
%    \begin{macrocode}
\addtocounter{page}{-1}
\begin{center}
{\LARGE\bfseries{}childdoc example\par}
\vspace{1cm}
\ifchilddoc
\ifchilddocmanual part\else chapter\fi:
`\childdocname' of `\childdocjob'\par
\else
main document: `\childdocjob'\par
\fi
version: \version\par
\end{center}
\newpage
%    \end{macrocode}

% Manually include selected file,
% otherwise process as usual:
%    \begin{macrocode}
\ifchilddocmanual
\section*{part `\childdocname'}
\input{\childdocname}
\else
%    \end{macrocode}

% Include the two chapters:
%    \begin{macrocode}
\include{cdocsch1}
\include{cdocsch2}
%    \end{macrocode}

% Include the two parts unless only chapters should be displayed:
%    \begin{macrocode}
\ifchilddoc\else
\section{part three}
\input{cdocspt3}
\section{part four}
\input{cdocspt4}
\fi
%    \end{macrocode}

% Process as usual until here:
%    \begin{macrocode}
\fi
%    \end{macrocode}

% End of document body:
%    \begin{macrocode}
\end{document}
%    \end{macrocode}
%\iffalse
%</samplemain>
%\fi
%
% %%%%%%%%%%%%%%%%%%%%%%%%%%%%%%%%%%%%%%
% \paragraph{Chapter Include Files.}
%
% The include files are called |cdocsch1.tex| and |cdocsch2.tex|.
%
%\iffalse
%<*samplechap1|samplechap2>
%\fi

% Optional override for |\version| flag:
%    \begin{macrocode}
%%\providecommand{\version}{final}
%    \end{macrocode}

% Include the main document:
%    \begin{macrocode}
\input{childdoc.def}
\childdocof{cdocsamp}
%    \end{macrocode}

%\iffalse
%</samplechap1|samplechap2>
%\fi
%
%\iffalse
%<*samplechap1>
%\fi
% Some text for chapter 1:
%    \begin{macrocode}
\section{one}
some text in chapter one
%    \end{macrocode}

%\iffalse
%</samplechap1>
%\fi
% Some text for chapter 2:
%\iffalse
%<*samplechap2>
%\fi
%    \begin{macrocode}
\section{two}
more text in chapter two
%    \end{macrocode}

%\iffalse
%</samplechap2>
%\fi
%
% %%%%%%%%%%%%%%%%%%%%%%%%%%%%%%%%%%%%%%
% \paragraph{Part Include Files.}
%
% The include files are called |cdocspt3.tex| and |cdocspt4.tex|.
%
%\iffalse
%<*samplepart3|samplepart4>
%\fi

% Optional override for |\version| flag:
%    \begin{macrocode}
%%\providecommand{\version}{final}
%    \end{macrocode}

% Include the main document:
%    \begin{macrocode}
\input{childdoc.def}
\childdocby{cdocsamp}
%    \end{macrocode}

%\iffalse
%</samplepart3|samplepart4>
%\fi
%
%\iffalse
%<*samplepart3>
%\fi
% Some text for part 3:
%    \begin{macrocode}
some text in part three
%    \end{macrocode}

%\iffalse
%</samplepart3>
%\fi
% Some text for part 4:
%\iffalse
%<*samplepart4>
%\fi
%    \begin{macrocode}
more text in part four
%    \end{macrocode}

%\iffalse
%</samplepart4>
%\fi
%
% %%%%%%%%%%%%%%%%%%%%%%%%%%%%%%%%%%%%%%
% \paragraph{Forwarding for a Complete Draft.}
%
% The following forwarding file |cdocsdrf.tex|
% compiles the main document in draft mode:
%\iffalse
%<*sampledraft>
%\fi
%    \begin{macrocode}
\def\version{draft}
\input{childdoc.def}
\childdocforward{cdocsamp}
%    \end{macrocode}

%\iffalse
%</sampledraft>
%\fi
%
% %%%%%%%%%%%%%%%%%%%%%%%%%%%%%%%%%%%%%%
% \paragraph{Forwarding for Final Version of the Chapters.}
%
% The following forwarding files |cdocsfn1.tex| and |cdocsfn2.tex|
% (with identical content)
% compile the final versions of the child documents
% |cdocsch1.tex| and |cdocsch2.tex|, respectively:
%\iffalse
%<*samplefinal>
%\fi
%    \begin{macrocode}
\def\version{final}
\input{childdoc.def}
\childdocforwardprefix[cdocsamp]{cdocsfn}{cdocsch}
%    \end{macrocode}

%\iffalse
%</samplefinal>
%\fi
%
% %%%%%%%%%%%%%%%%%%%%%%%%%%%%%%%%%%%%%%
% \paragraph{Command Line Processing.}
%
% The following three command lines generate the output files
% |cdocscld|, |cdocscl1| and |cdocscl2|
% which should be identical to
% |cdocsdrf|, |cdocsch1| and |cdocsfn2|, respectively:
% \begin{center}
% \begin{tabular}{l}
% |latex -jobname cdocscld \|\\
% |  "\def\version{draft}\input{childdoc.def}\childdocforward{cdocsamp}"|\\
% |latex -jobname cdocscl1 \|\\
% |  "\input{childdoc.def}\childdocforward[cdocsamp]{cdocsch1}"|\\
% |latex -jobname cdocscl2 \|\\
% |  "\def\version{final}\input{childdoc.def}\childdocforward{cdocsch2}"|
% \end{tabular}
% \end{center}
% Note that the trailing backslash on each first line
% merely continues the input to the second line
% (for convenient cut ant paste).
% Furthermore, the command |latex| can be replaced by any
% of its alternative versions such as |pdflatex|.
%
% %%%%%%%%%%%%%%%%%%%%%%%%%%%%%%%%%%%%%%%%%%%%%%%%%%%%%%%%%%%%%%%%%%%%%%%%%%%%%%
% %%%%%%%%%%%%%%%%%%%%%%%%%%%%%%%%%%%%%%%%%%%%%%%%%%%%%%%%%%%%%%%%%%%%%%%%%%%%%%
% \section{Implementation}
%\iffalse
%<*package>
%\fi
%
% This section describes the definitions file |childdoc.def|.

% The definitions cannot be loaded using |\usepackage| or |\RequirePackage|
% which has a mechanism to prevent loading a style file more than once.
% When loading the definitions by means of |\input|
% multiple instances have to be prevented manually:
%\iffalse
%This code needs to be before the `\ProvidesFile' directive
%which is defined at the beginning of this file.
%Therefore it is also placed there and commented out here.
%</package>
%<*discard>
%\fi
%    \begin{macrocode}
\ifdefined\childdocmain\endinput\fi
%    \end{macrocode}
%\iffalse
%</discard>
%<*package>
%\fi
%
% \macro{\ifchilddoc}
% \macro{\ifchilddocmanual}
% The conditional |\ifchilddoc| tells whether a
% child (true) or main (false) document is being compiled.
% The conditional |\ifchilddocmanual| tells whether
% the |\includeonly| mechanism is used (false) or
% the selection of child files must be performed manually (true).
% The definitions initialise to false:
%    \begin{macrocode}
\newif\ifchilddoc
\newif\ifchilddocmanual
%    \end{macrocode}

% \macro{\childdocname}
% \macro{\childdocjob}
% The macro |\childdocname| stores the name of the main document
% to be compiled. The macro |\childdocjob| stores the name of
% the document on which the \LaTeX{} compiler was originally invoked.
% The content of |\jobname| cannot be compared
% to filenames specified in the source due to different catcodes.
% The following code rescans |\jobname|, stores the result
% in |\childdocname| and saves a copy in |\childdocjob|:
%    \begin{macrocode}
\edef\childdocname{\scantokens\expandafter{\jobname\noexpand}}
\let\childdocjob\childdocname
%    \end{macrocode}

% \macro{\childdocdisable}
% The macro |\childdocdisable| prevents the main file
% from being processed more than once.
% At this stage, the main document command |\childdocmain|
% is assumed to be called once again where it should do nothing.
% Any subsequent call to it should prevent
% a secondary processing of the main document
% It overwrites the forwarding commands
% |\childdocof| and |\childdocforward|
% with empty macros to prevent further inclusions of the main document:
%    \begin{macrocode}
\newcommand{\childdocdisable}
{
  \renewcommand{\childdocmain}[1]{\renewcommand{\childdocmain}[1]{\endinput}}
  \renewcommand{\childdocof}[1]{}
  \renewcommand{\childdocby}[2][]{}
  \renewcommand{\childdocforward}[2][]{}
  \renewcommand{\childdocdisable}{}
}
%    \end{macrocode}

% \macro{\childdocmain}
% The macro |\childdocmain| is to be called at the top of the main file
% with nothing or the main filename (without extension) as argument.
% First, it breaks loops.
% If the argument is not empty and does not match |\childdocname|
% (which is set by the first inclusion of |childdoc.def|),
% |\ifchilddoc| is set to true, |\includeonly| is applied to the child file
% and |\jobname| is set to the main file
% (for proper handling of |.aux| files):
%    \begin{macrocode}
\newcommand{\childdocmain}[1]
{
  \childdocdisable\childdocmain{}
  \if?#1?\else
    \begingroup
      \def\childdoctmp{#1}
      \ifx\childdoctmp\childdocname
        \def\childdoctmp{}
      \else
        \def\childdoctmp
        {
          \childdoctrue
          \includeonly{\childdocname}
          \def\childdocjob{#1}
          \def\jobname{#1}
        }
      \fi
      \expandafter
    \endgroup
    \childdoctmp
  \fi
}
%    \end{macrocode}

% \macro{\childdocof}
% The command |\childdocof| redirects
% compilation to the main file |#1|.
%    \begin{macrocode}
\newcommand{\childdocof}[1]
{
  \childdocdisable
  \childdoctrue
  \includeonly{\childdocname}
  \def\jobname{#1}
  \def\childdocjob{#1}
  \input{#1}
}
%    \end{macrocode}

% \macro{\childdocby}
% The command |\childdocby| ....
%    \begin{macrocode}
\newcommand{\childdocby}[2][]
{
  \childdocdisable
  \childdoctrue
  \childdocmanualtrue
  \if?#1?\else
    \def\jobname{#2}
  \fi
  \def\childdocjob{#2}
  \input{#2}
  \endinput
}
%    \end{macrocode}

% \macro{\childdocforward}
% The command |\childdocforward| redirects
% compilation to the main file or
% (if the optional argument is given) a child file.
% Parameters are set as if the main file
% or a child file starting with |\childdocof| was compiled.
% Then compilation is handed over to the main file:
%    \begin{macrocode}
\newcommand{\childdocforward}[2][]
{
  \begingroup
    \if?#1?
      \def\childdoctmp
      {
        \def\childdocname{#2}
        \def\childdocjob{#2}
        \def\jobname{#2}
        \input{#2}
        \endinput
      }
    \else
      \def\childdoctmp
      {
        \childdocdisable
        \def\childdocname{#2}
        \childdoctrue
        \includeonly{#2}
        \def\childdocjob{#1}
        \def\jobname{#1}
        \input{#1}
        \endinput
      }
    \fi
    \expandafter
  \endgroup
  \childdoctmp
}
%    \end{macrocode}

% \macro{\childdocforwardprefix}
% The command |\childdocforwardprefix| redirects
% compilation to the main or a child file by means of a pattern.
% The prefix |#1| in the current filename is replaced by |#2|
% and the suffix of the current filename is kept
% (it is assumed that the filename does not contain the substring `|~~~|'
% which is used as a delimiter).
% Compilation is handed over to the new file by |\childdocforward|:
%    \begin{macrocode}
\newcommand{\childdocforwardprefix}[3][]
{
  \begingroup
    \def\childdocextract #2##1~~~{\def\childdoctmp{\childdocforward[#1]{#3##1}}}
    \expandafter\childdocextract\childdocname~~~
    \expandafter
  \endgroup
  \childdoctmp
}
%    \end{macrocode}

% \macro{\childdoc}
% The deprecated macro |\childdoc| is a legacy version of |\childdocmain|:
%    \begin{macrocode}
\newcommand{\childdoc}{\childdocmain}
%    \end{macrocode}

% \macro{\childdocredirect}
% The deprecated macro |\childdocredirect| is a legacy version
% of |\childdocforward| and |\childdocforwardprefix|:
%    \begin{macrocode}
\newcommand{\childdocredirect}[2][]
{
  \begingroup
    \if?#1?
      \def\childdoctmp{\childdocforward{#2}}
    \else
      \def\childdoctmp{\childdocforwardprefix{#1}{#2}}
    \fi
    \expandafter
  \endgroup
  \childdoctmp
}
%    \end{macrocode}

%\iffalse
%</package>
%\fi
%
\endinput
|\\
|\childdocby{|\textit{main}|}|\\
\end{tabular}
\end{center}
%
Both forms have slightly different effects as described above.
The main file is prepared as usual, see \secref{sec:include}.

%%%%%%%%%%%%%%%%%%%%%%%%%%%%%%%%%%%%%%%%%%%%%%%%%%%%%%%%%%%%%%%%%%%%%%%%%%%%%%%%
\subsection{Legacy Detection}
\label{sec:detection}

The directive |\childdocmain| in the main file can detect
whether the complete document or merely a child is to be compiled
even without using the directive |\childdocof|.
This method is deprecated because it is less robust
and there is no compelling reason to use it;
it is merely provided for backward compatibility
and it may be removed in future versions.

If the detection mechanism is to be used,
it is mandatory to correctly specify
the filename of the main file as the argument of |\childdocmain|:
%
\begin{center}
\begin{tabular}{l}
|% \iffalse
%
% childdoc.dtx Copyright (C) 2017-2018 Niklas Beisert
%
% This work may be distributed and/or modified under the
% conditions of the LaTeX Project Public License, either version 1.3
% of this license or (at your option) any later version.
% The latest version of this license is in
%   http://www.latex-project.org/lppl.txt
% and version 1.3 or later is part of all distributions of LaTeX
% version 2005/12/01 or later.
%
% This work has the LPPL maintenance status `maintained'.
%
% The Current Maintainer of this work is Niklas Beisert.
%
% This work consists of the files childdoc.dtx and childdoc.ins
% and the derived files childdoc.def and cdocsamp.tex with
% cdocsch1.tex, cdocsch2.tex, cdocsdrf.tex, cdocsfn1.tex, cdocsfn2.tex.
%
%<package>\ifdefined\childdocmain\endinput\fi
%<package>\ProvidesFile{childdoc.def}[2018/12/30 v2.0 child document driver]
%<samplemain>\ProvidesFile{cdocsamp.tex}[2018/12/30 v2.0 sample for childdoc]
%<*driver>
%\ProvidesFile{childdoc.drv}[2018/12/30 v2.0 childdoc reference manual file]
\PassOptionsToClass{10pt,a4paper}{article}
\documentclass{ltxdoc}

\usepackage[margin=35mm]{geometry}
\usepackage{hyperref}
\usepackage{hyperxmp}
\usepackage[usenames]{color}

\hypersetup{colorlinks=true}
\hypersetup{pdfstartview=FitH}
\hypersetup{pdfpagemode=UseNone}
\hypersetup{pdfsource={}}
\hypersetup{pdflang={en-UK}}
\hypersetup{pdfcopyright={Copyright 2017-2018 Niklas Beisert.
  This work may be distributed and/or modified under the
  conditions of the LaTeX Project Public License, either version 1.3
  of this license or (at your option) any later version.}}
\hypersetup{pdflicenseurl={http://www.latex-project.org/lppl.txt}}
\hypersetup{pdfcontactaddress={ETH Zurich, ITP, HIT K,
  Wolfgang-Pauli-Strasse 27}}
\hypersetup{pdfcontactpostcode={8093}}
\hypersetup{pdfcontactcity={Zurich}}
\hypersetup{pdfcontactcountry={Switzerland}}
\hypersetup{pdfcontactemail={nbeisert@itp.phys.ethz.ch}}
\hypersetup{pdfcontacturl={http://people.phys.ethz.ch/\xmptilde nbeisert/}}

\newcommand{\secref}[1]{\hyperref[#1]{section \ref*{#1}}}

\parskip1ex
\parindent0pt
\let\olditemize\itemize
\def\itemize{\olditemize\parskip0pt}

\begin{document}

\title{The \textsf{childdoc} Package}
\hypersetup{pdftitle={The childdoc Package}}
\author{Niklas Beisert\\[2ex]
  Institut f\"ur Theoretische Physik\\
  Eidgen\"ossische Technische Hochschule Z\"urich\\
  Wolfgang-Pauli-Strasse 27, 8093 Z\"urich, Switzerland\\[1ex]
  \href{mailto:nbeisert@itp.phys.ethz.ch}
  {\texttt{nbeisert@itp.phys.ethz.ch}}}
\hypersetup{pdfauthor={Niklas Beisert}}
\hypersetup{pdfsubject={Manual for the LaTeX2e Package childdoc}}
\date{30 December 2018, \textsf{v2.0}}
\maketitle

\begin{abstract}\noindent
\textsf{childdoc} is a \LaTeXe{} package
that enables the direct compilation
of document sections included by |\include|
to individual files.
\end{abstract}

\begingroup
\parskip0ex
\tableofcontents
\endgroup

%%%%%%%%%%%%%%%%%%%%%%%%%%%%%%%%%%%%%%%%%%%%%%%%%%%%%%%%%%%%%%%%%%%%%%%%%%%%%%%%
%%%%%%%%%%%%%%%%%%%%%%%%%%%%%%%%%%%%%%%%%%%%%%%%%%%%%%%%%%%%%%%%%%%%%%%%%%%%%%%%
\section{Introduction}

\LaTeX{} provides a mechanism to structure a large document (such as a book)
into a main file and several child files (containing the chapters)
using the |\include| command.
This mechanism is beneficial for documents
which span hundreds of pages in order to
make the source file(s) more manageable.
Moreover, compilation can be restricted to
selected child files by means of the |\includeonly| command.
The latter feature can be used to reduce the compilation time while editing
(this was significantly more useful in the earlier days of \LaTeX{})
or to generate a smaller document which is easier to navigate.
Another application of |\includeonly| is to generate
documents consisting of selected parts of the complete document.

However, there are a few drawbacks of the plain |\include| mechanism:
\begin{itemize}
\item
The child files cannot be compiled on their own,
they can only be compiled via the main file.
A naive editing environment
(such as a text editor with an option
to have the current file processed by \LaTeX)
may require one to switch to the main file before compiling;
attempting to compile the child file produces errors.
\item
The main file must be modified (each time)
to adjust the |\includeonly| command
to the present needs. This easily leaves the main file in a messy state.
\item
The generated document will always carry the filename
of the main document. This is inconvenient if
several child files are to be compiled and
to be kept for distribution.
\end{itemize}

The present package provides a simple interface
to make child files individually compilable by \LaTeX{}.
Compiling a child file then has the same effect as compiling
the main file with an |\includeonly| command
to select the appropriate child.
Moreover the generated document will carry the name of the child
rather than the main file.
This resolves all three above issues.

This feature is meant to make the editing of books,
thesis documents and lecture notes somewhat more convenient.
However, the package can also be used efficiently for
composing a series of documents (such as exercise sheets)
which are typically distributed individually.
It then assists the author in generating the individual documents
(potentially in different versions)
as well as a document containing the collected series.
Another application is in developing style files
or other kinds of included material
where compilation of the style file could redirect
to a sample or test file.

%%%%%%%%%%%%%%%%%%%%%%%%%%%%%%%%%%%%%%%%%%%%%%%%%%%%%%%%%%%%%%%%%%%%%%%%%%%%%%%%
%%%%%%%%%%%%%%%%%%%%%%%%%%%%%%%%%%%%%%%%%%%%%%%%%%%%%%%%%%%%%%%%%%%%%%%%%%%%%%%%
\section{Usage}

First of all, the package \textsf{childdoc} is \emph{not} a standard
\LaTeXe{} |.sty| style file! Therefore it needs to be invoked in
a non-standard way.

%%%%%%%%%%%%%%%%%%%%%%%%%%%%%%%%%%%%%%%%%%%%%%%%%%%%%%%%%%%%%%%%%%%%%%%%%%%%%%%%
\subsection{Included Files}
\label{sec:include}

%%%%%%%%%%%%%%%%%%%%%%%%%%%%%%%%%%%%%%%%
\DescribeMacro{\childdocmain}
To use the package, add the commands
\begin{center}
\begin{tabular}{l}
|\input{childdoc.def}|\\
|\childdocmain{}|\\
\end{tabular}
\end{center}
at the very top of the main \LaTeX{} file,
in particular \emph{before} the |\documentclass| statement!
The argument of |\childdocmain| should be left empty
(but it must be present).

%%%%%%%%%%%%%%%%%%%%%%%%%%%%%%%%%%%%%%%%
\DescribeMacro{\childdocof}
Furthermore, add the commands
\begin{center}
\begin{tabular}{l}
|\input{childdoc.def}|\\
|\childdocof{|\textit{main}|}|\\
\end{tabular}
\end{center}
at the top of every child file \textit{child}
which is included by |\include{|\textit{child}|}|
from within the main file
(or at least for those files to be compiled individually).
The argument \textit{main} must be the filename of the main file.

There are a couple of
considerations in setting up the main and child documents:

%%%%%%%%%%%%%%%%%%%%%%%%%%%%%%%%%%%%%%%%
\paragraph{Restrictions.}

Please note the following restrictions:
\begin{itemize}
\item
|\childdocmain| must be called with one argument \textit{main}
to ensure compatibility with earlier version of the package.
It must either be empty (|\childdocmain{}|)
or precisely match the filename of the main file in which it is specified.
See \secref{sec:detection} for further information.
\item
The filename \textit{main} must be specified without the |.tex| extension.
\item
The filename \textit{main} is case sensitive
(even in case-insensitive file systems)
due to internal string comparison.
\item
The argument \textit{main} should be fully expanded, it cannot be a macro.
\item
Subdirectories and special characters should be avoided in filenames.
\item
The command |\childdocmain{|\textit{main}|}| must be followed by a whitespace.
It should not be followed immediately by another command
or by a comment mark `|%|'.
This is because the \TeX{} parser reads the token immediately following
the argument of |\childdocmain| and puts it
at the beginning of every child section;
however, a white\-space is ignored.
\end{itemize}

%%%%%%%%%%%%%%%%%%%%%%%%%%%%%%%%%%%%%%%%
\paragraph{Content of Main File.}

It is advisable to place all content in the child files included by |\include|.
Any output contained in the main file will appear in all child documents
unless suppressed manually;
it cannot be suppressed automatically by the |\includeonly| directive
and thus should normally be avoided.
A method to include some content in the main file
by means of conditional processing is described in \secref{sec:conditional}.

%%%%%%%%%%%%%%%%%%%%%%%%%%%%%%%%%%%%%%%%
\paragraph{Page Numbering.}

When only a part of the document is compiled,
the appropriate numbering of pages
(as well as other status parameters)
is determined from the |.aux| files.
The latter contain information from previous passes.
However this information needs to propagate through
all intermediate child documents.
Therefore the page numbering in child documents may well
be inconsistent until the complete document is compiled at least once.

A useful (if unconventional) way to always ensure a consistent
page numbering is to restart the numbering in each child document
and denote the pages by `\textit{child}|.|\textit{page}'
where \textit{child} represents the chapter/section number of the child file.
This can be achieved by the command
|\numberwithin{page}{|\textit{child}|}|
of the \textsf{amsmath} package
where \textit{child} can be |chapter| or |section|
depending on the chosen structuring.
Alternatively, one can modify the macro |\thepage| appropriately
and reset the counter |page| at the start of each child file.

%%%%%%%%%%%%%%%%%%%%%%%%%%%%%%%%%%%%%%%%%%%%%%%%%%%%%%%%%%%%%%%%%%%%%%%%%%%%%%%%
\subsection{Conditional Processing}
\label{sec:conditional}

The package provides a mechanism to compile different versions
of a document. To customise the versions further some conditional processing
can come in handy to distinguish which version is being compiled.
The package provides two macros to describe the compilation context:

%%%%%%%%%%%%%%%%%%%%%%%%%%%%%%%%%%%%%%%%
\DescribeMacro{\ifchilddoc}
The conditional |\ifchilddoc| distinguishes between the compilation of
child documents and the main document:
%
\begin{center}
|\ifchilddoc |\textit{child-code}| |[|\||else |\textit{main-code}]| \||fi|
\end{center}

%%%%%%%%%%%%%%%%%%%%%%%%%%%%%%%%%%%%%%%%
\DescribeMacro{\childdocname}
\DescribeMacro{\childdocjob}
The macro |\childdocname| contains the filename (without extension)
of the main or child file being processed.
Note that |\childdocjob| will always contain the name of the main file.

%%%%%%%%%%%%%%%%%%%%%%%%%%%%%%%%%%%%%%%%
\paragraph{Title Page.}

Conditional processing can be used to include a title or banner page
in the main document when proper precautions are taken.
Importantly, the code in the main file should ensure that the page counter
(as well as other status parameters which are stored in the |.aux| files)
takes the same value after the conditional processing.
Otherwise the page numbers may take divergent values
depending on which part is compiled.

For example, a title page could be declared by:
%
\begin{center}
\begin{tabular}{l}
|\ifchilddoc\||else|\\
|\addtocounter{page}{-1}|\\
\textit{code for title page}\\
|\newpage|\\
|\||fi|
\end{tabular}
\end{center}
%
A banner page for the child documents can be generated by:
%
\begin{center}
\begin{tabular}{l}
|\ifchilddoc|\\
|\addtocounter{page}{-1}|\\
\textit{code for banner page}\\
|\newpage|\\
|\||fi|
\end{tabular}
\end{center}
%
Here one could write a message such as:
\begin{center}
|This is the part \childdocname{} of \childdocjob{}.|
\end{center}

%%%%%%%%%%%%%%%%%%%%%%%%%%%%%%%%%%%%%%%%%%%%%%%%%%%%%%%%%%%%%%%%%%%%%%%%%%%%%%%%
\subsection{Flags}
\label{sec:flags}

The package makes it easy to generate different versions
of the main or child documents.
To this end compilation flags can be defined
and assigned different default values.
They will be particularly useful in conjunction
with the forwarding mechanism described in \secref{sec:forward}.

For example, it may be useful to have a flag |\version|
which can be set to |draft| or |final|.
The document source will contain some conditional code
depending on the value of |\version|.
Suppose further, the flag should default to |final| for the main file
and to |draft| for child files
which is a natural assignment for editing the document.
This is achieved by placing the following code
in the preamble of the main document
(below the |\childdocmain| directive):
%
\begin{center}
\begin{tabular}{l}
|\ifchilddoc|\\
|\providecommand{\version}{draft}|\\
|\||else|\\
|\providecommand{\version}{final}|\\
|\||fi|
\end{tabular}
\end{center}
%
The definition by |\providecommand| makes sure
that previous definitions are not overwritten.
Further statements |\providecommand{\version}{...}|
can thus be added before the above code to override it.

For the main file, one might add a line
(between |\childdocmain| and the above block)
%
\begin{center}
|%\ifchilddoc\||else\providecommand{\version}{draft}\||fi|
\end{center}
%
which can be uncommented to produce a draft version.
Likewise one can add a line to the very top of a child file
(above the |\childdocof{|\textit{main}|}| directive)
%
\begin{center}
|%\providecommand{\version}{final}|
\end{center}
%
which can be uncommented to produce the final version of this child document.

%%%%%%%%%%%%%%%%%%%%%%%%%%%%%%%%%%%%%%%%%%%%%%%%%%%%%%%%%%%%%%%%%%%%%%%%%%%%%%%%
\subsection{Forwarding}
\label{sec:forward}

Different versions of the main or child documents
using compilation flags as described in \secref{sec:flags}
can be (permanently) stored in different files
for convenient compilation, viewing and distribution.
To this end, the package defines a command
to pass on compilation to a different file:

%%%%%%%%%%%%%%%%%%%%%%%%%%%%%%%%%%%%%%%%
\DescribeMacro{\childdocforward}
The command |\childdocforward| redirects processing to
another source file:
%
\begin{center}
\begin{tabular}{l}
|\input{childdoc.def}|\\
|\childdocforward[|\textit{main}|]{|\textit{dest}|}|\\
\end{tabular}
\end{center}
%
The argument \textit{dest} is the destination file
(without extension).
It should be the main file or one of the child files.
Note that further \textsf{childdoc} directives
such as |\childdocof| and |\childdocforward|
in the indicated file will be processed in this form.
The optional argument \textit{main}
passes on directly to the main file \textit{main}
while pretending to compile the child \textit{dest}.
This form behaves as if \textit{dest}
issues |\childdocof{|\textit{main}|}| right away,
and no further \textsf{childdoc} directives will be processed.

%%%%%%%%%%%%%%%%%%%%%%%%%%%%%%%%%%%%%%%%
\DescribeMacro{\...prefix}
In the alternative form |\childdocforwardprefix|,
%
\begin{center}
\begin{tabular}{l}
|\input{childdoc.def}|\\
|\childdocforwardprefix[|\textit{main}|]{|\textit{prefix}|}{|\textit{dest}|}|
\end{tabular}
\end{center}
%
the destination file is determined by a pattern
depending on the current file:
To make this work, the current file must be called
`{\textit{prefix}\hspace{0.2em}\textit{suffix}}'
with \textit{prefix} matching precisely the argument.
Processing is then passed on to the file
`{\textit{dest}\hspace{0.2em}\textit{suffix}}'.
Surely, the same effect is achieved by
directly specifying the
argument `{\textit{dest}\hspace{0.2em}\textit{suffix}}'
in the first form.
However, that requires to set up a different file
for each child. With the alternative form of the command
all these files can have exactly the same content
which simplifies setting them up and maintaining them.

For example, the following file |draft.tex|
with a compilation flag |\version| as described in \secref{sec:flags}
compiles the main document as a draft:
%
\begin{center}
\begin{tabular}{l}
|\def\version{draft}|\\
|\input{childdoc.def}|\\
|\childdocforward{|\textit{main}|}|
\end{tabular}
\end{center}
%
Likewise, the following files |final|\textit{nn}|.tex|
compile the final version of the child document
|child|\textit{nn}|.tex|:
%
\begin{center}
\begin{tabular}{l}
|\def\version{final}|\\
|\input{childdoc.def}|\\
|\childdocforwardprefix{final}{child}|
\end{tabular}
\end{center}
%

Note that when several versions of a main file and/or of each child file
are to be generated, it may be convenient to set up a |Makefile| or
shell script to automatise the process.

%%%%%%%%%%%%%%%%%%%%%%%%%%%%%%%%%%%%%%%%%%%%%%%%%%%%%%%%%%%%%%%%%%%%%%%%%%%%%%%%
\subsection{Command Line Processing}
\label{sec:commandline}

The effect of redirection files can also be achieved by invoking
the \LaTeX{} compiler with a more elaborate command line.
Most conveniently this should be done as part
of a shell script or a |Makefile|.

When using \textsf{childdoc} in the main file, the following
command lines effectively perform a redirection
(note that depending on the shell being used,
backslashes may have to be doubled: `|\|' $\to$ `|\\|'):
%
\begin{center}
|... -jobname "|\textit{target}|" |\\|"|[\textit{flags}]%
|\input{childdoc.def}\childdocforward[|\textit{main}|]{|\textit{dest}|}"|
\end{center}
%
Here \textit{target} is the name of the output file,
\textit{main} is the name of the main file
and \textit{dest} is the name of the main or child file to be processed
(all filenames without extensions).
The optional argument \textit{main} can be omitted
if \textit{main} matches \textit{dest}.
Optionally, compilation \textit{flags} can be defined via |\def| commands.
This command line makes the \TeX{} engine believe
it is compiling the file \textit{target}
whose content is specified as the latter parameter.
The provided code then forwards the processing to
\textit{main} or \textit{dest} as described in \secref{sec:forward}.

%%%%%%%%%%%%%%%%%%%%%%%%%%%%%%%%%%%%%%%%%%%%%%%%%%%%%%%%%%%%%%%%%%%%%%%%%%%%%%%%
\subsection{Include by Input}
\label{sec:input}

Including child documents by |\include| has some restrictions by design.
Most notably, the content of a child document always occupies
its own set of pages; pages cannot be shared between child documents.
Usually, this behaviour makes perfect sense
because each child document contain an essential part of the document.
However, in some situations it may be desirable to compose
a document from a collection of parts
without having mandatory page breaks between then.
For this case, the package
provides a mechanism to include parts
by |\input| which can also be processed individually.
However, by construction this mechanism
requires manual handling of the content to be output.

%%%%%%%%%%%%%%%%%%%%%%%%%%%%%%%%%%%%%%%%
\DescribeMacro{\ifchilddocmanual}
The main file should be prepared as usual, see \secref{sec:include}.
However, the document body must make a distinction
between processing of an individual part and of the main document, e.g.:
%
\begin{center}
\begin{tabular}{l}
|\ifchilddocmanual|\\
|\input{\childdocname}|\\
|\||else|\\
\textit{document body with }|\input{|\textit{part}|}|\\
|\||fi|
\end{tabular}
\end{center}
%
The conditional |\ifchilddocmanual| is true whenever
a part to be included by |\input| is being compiled,
and the name of the part is stored in |\childdocname|.

%%%%%%%%%%%%%%%%%%%%%%%%%%%%%%%%%%%%%%%%
\DescribeMacro{\childdocby}
Each part to be included by |\input| should start with:
%
\begin{center}
\begin{tabular}{l}
|\input{childdoc.def}|\\
|\childdocby{|\textit{main}|}|\\
\end{tabular}
\end{center}
%
The directive |\childdocby| is similar to |\childdocof|
described in \secref{sec:include},
but the subsequent selection of content must be done manually.
To that end, both |\ifchilddoc| and |\ifchilddocmanual|
will be true upon processing of a part,
and the name of the part is stored in |\childdocname|.
Note that |\jobname| will be set to the filename of the current part
so that each part receives an individual |.aux| file
that does not interfere with the |.aux| file(s) of the main document.
This behaviour can be altered by the alternative form
|\childdocby[*]{|\textit{main}|}| (with a non-empty optional argument)
which uses the |.aux| file of the main document
by setting |\jobname| to \textit{main}.

%%%%%%%%%%%%%%%%%%%%%%%%%%%%%%%%%%%%%%%%%%%%%%%%%%%%%%%%%%%%%%%%%%%%%%%%%%%%%%%%
\subsection{Driver Development}
\label{sec:driver}

The \textsf{childdoc} mechanism can also be use for the development
of definition files such as \LaTeX{} styles or classes.
This case differs from the above setup with multiple parts
included by |\include| in that no |\includeonly| should be invoked.
This can be achieved by starting the include file
(before |\ProvidesPackage|) with:
%
\begin{center}
\begin{tabular}{l}
|\input{childdoc.def}|\\
|\childdocforward{|\textit{main}|}|\\
\end{tabular}
\end{center}
%
or alternatively with:
%
\begin{center}
\begin{tabular}{l}
|\input{childdoc.def}|\\
|\childdocby{|\textit{main}|}|\\
\end{tabular}
\end{center}
%
Both forms have slightly different effects as described above.
The main file is prepared as usual, see \secref{sec:include}.

%%%%%%%%%%%%%%%%%%%%%%%%%%%%%%%%%%%%%%%%%%%%%%%%%%%%%%%%%%%%%%%%%%%%%%%%%%%%%%%%
\subsection{Legacy Detection}
\label{sec:detection}

The directive |\childdocmain| in the main file can detect
whether the complete document or merely a child is to be compiled
even without using the directive |\childdocof|.
This method is deprecated because it is less robust
and there is no compelling reason to use it;
it is merely provided for backward compatibility
and it may be removed in future versions.

If the detection mechanism is to be used,
it is mandatory to correctly specify
the filename of the main file as the argument of |\childdocmain|:
%
\begin{center}
\begin{tabular}{l}
|\input{childdoc.def}|\\
|\childdocmain{|\textit{main}|}|\\
\end{tabular}
\end{center}
%
If |\jobname| does not match the argument \textit{main} of |\childdocmain|,
it is assumed that |\jobname| points to the child file to be compiled.
When using |\childdocmain| with the main file specified as argument,
it suffices to start a child file
with just |\input{|\textit{main}|}|
without loading of the package and using |\childdocof|.
If instead all processing is done
with the appropriate \textsf{childdoc} directives,
the argument of \textit{main} of |\childdocmain| can be empty.

An alternative version of the command line processing described
in \secref{sec:commandline} using the detection mechanism reads:
%
\begin{center}
|... -jobname "|\textit{target}|" "|[\textit{flags}]%
[|\def\jobname{|\textit{dest}|}|]|\input{|\textit{main}|}"|
\end{center}

%%%%%%%%%%%%%%%%%%%%%%%%%%%%%%%%%%%%%%%%%%%%%%%%%%%%%%%%%%%%%%%%%%%%%%%%%%%%%%%%
\subsection{Manual Code}
\label{sec:manual}

In case one cannot be certain whether the definitions file |childdoc.def|
is installed on the target \TeX{} distribution
and one prefers not to ship it,
it is conceivable to paste a few relevant commands into the sources.

To that end, drop all statements |\input{childdoc.def}|
and perform the replacements as outlined below.
Instead of |\childdocmain{|\textit{main}|}| add the following code
to the top of the main file:
%
\begin{center}
\begin{tabular}{l}
|\||ifdefined\childdocname\endinput\||fi\newif\ifchilddoc|\\
|\edef\childdocname{\scantokens\expandafter{\jobname\noexpand}}|\\
|\def\childdocmain{|\textit{main}|}\||ifx\childdocmain\childdocname\||else|\\
|\childdoctrue\includeonly{\childdocname}\let\jobname\childdocmain\||fi|\\
\end{tabular}
\end{center}
%
Instead of |\childdocof{|\textit{main}|}| just include the main file
at the top of each child file:
%
\begin{center}
|\input{|\textit{main}|}|
\end{center}
%
A simple redirection |\childdocforward{|\textit{dest}|}| is achieved by:
%
\begin{center}
|\def\jobname{|\textit{dest}|}\input{\jobname}|
\end{center}
%
The redirection with prefix
|\childdocforwardprefix[|\textit{prefix}|]{|\textit{dest}|}|
is accomplished by:
%
\begin{center}
\begin{tabular}{l}
|{\edef\jobname{\scantokens\expandafter{\jobname\noexpand}}|\\
|\def\redirectjob |\textit{prefix}|#1~~~{\gdef\jobname{|\textit{dest}|#1}}|\\
|\expandafter\redirectjob\jobname~~~}\input{\jobname}|
\end{tabular}
\end{center}

In an alternative approach,
child documents can be compiled by a specific command line
without additional code or specific definitions:
%
\begin{center}
|... -jobname "|\textit{target}|" "|[\textit{flags}]%
|\includeonly{|\textit{dest}|}\input{|\textit{main}|}"|
\end{center}
%

%%%%%%%%%%%%%%%%%%%%%%%%%%%%%%%%%%%%%%%%%%%%%%%%%%%%%%%%%%%%%%%%%%%%%%%%%%%%%%%%
%%%%%%%%%%%%%%%%%%%%%%%%%%%%%%%%%%%%%%%%%%%%%%%%%%%%%%%%%%%%%%%%%%%%%%%%%%%%%%%%
\section{Information}

%%%%%%%%%%%%%%%%%%%%%%%%%%%%%%%%%%%%%%%%%%%%%%%%%%%%%%%%%%%%%%%%%%%%%%%%%%%%%%%%
\subsection{Copyright}

Copyright \copyright{} 2017--2018 Niklas Beisert

This work may be distributed and/or modified under the
conditions of the \LaTeX{} Project Public License, either version 1.3
of this license or (at your option) any later version.
The latest version of this license is in
  \url{http://www.latex-project.org/lppl.txt}
and version 1.3 or later is part of all distributions of \LaTeX{}
version 2005/12/01 or later.

This work has the LPPL maintenance status `maintained'.

The Current Maintainer of this work is Niklas Beisert.

This work consists of the files |README.txt|, |childdoc.ins| and |childdoc.dtx|
as well as the derived files |childdoc.def|, |cdocsamp.tex|
with |cdocsch1.tex|, |cdocsch2.tex|, |cdocspt3.tex|, |cdocspt4.tex|,
|cdocsdrf.tex|, |cdocsfn1.tex|, |cdocsfn2.tex|
as well as |childdoc.pdf|.

%%%%%%%%%%%%%%%%%%%%%%%%%%%%%%%%%%%%%%%%%%%%%%%%%%%%%%%%%%%%%%%%%%%%%%%%%%%%%%%%
\subsection{Files and Installation}

The package consists of the files:
%
\begin{center}
\begin{tabular}{ll}
    |README.txt|   & readme file \\
    |childdoc.ins| & installation file \\
    |childdoc.dtx| & source file \\
    |childdoc.def| & definition file \\
    |cdocsamp.tex| & sample main file \\
    |cdocsch1.tex| & sample include file \\
    |cdocsch2.tex| & sample include file \\
    |cdocspt3.tex| & sample part file \\
    |cdocspt4.tex| & sample part file \\
    |cdocsdrf.tex| & sample redirection file \\
    |cdocsfn1.tex| & sample redirection file \\
    |cdocsfn2.tex| & sample redirection file \\
    |childdoc.pdf| & manual
\end{tabular}
\end{center}
%
The distribution consists of the files
|README.txt|, |childdoc.ins| and |childdoc.dtx|.
%
\begin{itemize}
\item
Run (pdf)\LaTeX{} on |childdoc.dtx|
to compile the manual |childdoc.pdf| (this file).
\item
Run \LaTeX{} on |childdoc.ins| to create the definitions file |childdoc.def|
and the sample |cdocsamp.tex| with include files
|cdocsch1.tex|, |cdocsch2.tex|, |cdocspt3.tex|, |cdocspt4.tex|,
|cdocsdrf.tex|, |cdocsfn1.tex|, |cdocsfn2.tex|.
Then copy the file |childdoc.def| to an appropriate directory of your \LaTeX{}
distribution, e.g.\ \textit{texmf-root}|/tex/latex/childdoc|.
\end{itemize}

%%%%%%%%%%%%%%%%%%%%%%%%%%%%%%%%%%%%%%%%%%%%%%%%%%%%%%%%%%%%%%%%%%%%%%%%%%%%%%%%
\subsection{Related CTAN Packages}

There are several other packages which offer a similar functionality:
%
\begin{itemize}
\item
The packages
\href{http://ctan.org/pkg/docmute}{\textsf{docmute}},
\href{http://ctan.org/pkg/includex}{\textsf{includex}} and
\href{http://ctan.org/pkg/standalone}{\textsf{standalone}}
provide commands to include only the document body of
a child file thus allowing both files to be compiled individually.
\item
The packages \href{http://ctan.org/pkg/subdocs}{\textsf{subdocs}}
and \href{http://ctan.org/pkg/subfiles}{\textsf{subfiles}}
provide structures in which the main and child documents can be
encapsulated and allowing them to be compiled individually.
The inclusion mechanism is different from the conventional |\include|.
\item
The package \href{http://ctan.org/pkg/combine}{\textsf{combine}}
is an elaborate solution to combine several documents into one.
\end{itemize}
%
See also the CTAN topic \href{http://ctan.org/topic/subdocs}{\textsf{subdocs}}
for further related packages.
The present package differs from the above solutions in that
a document structure constructed with the conventional |\include| mechanism
just needs two extra commands at the top of every file
such that all constituent files can be compiled individually.

%%%%%%%%%%%%%%%%%%%%%%%%%%%%%%%%%%%%%%%%%%%%%%%%%%%%%%%%%%%%%%%%%%%%%%%%%%%%%%%%
%\subsection{Feature Suggestions}
%
%The following is a list of features which may be useful for future
%versions of this package:
%%
%\begin{itemize}
%\item
%\ldots
%\end{itemize}

%%%%%%%%%%%%%%%%%%%%%%%%%%%%%%%%%%%%%%%%%%%%%%%%%%%%%%%%%%%%%%%%%%%%%%%%%%%%%%%%
\subsection{Revision History}

%%%%%%%%%%%%%%%%%%%%%%%%%%%%%%%%%%%%%%%%
\paragraph{v2.0:} 2018/12/30

\begin{itemize}
\item
immediate forward processing
\item
added |\childdocby| mechanism
\item
manual restructured
\end{itemize}

%%%%%%%%%%%%%%%%%%%%%%%%%%%%%%%%%%%%%%%%
\paragraph{v1.6:} 2018/01/17

\begin{itemize}
\item
application for development of include files
\item
corrections to manual
\end{itemize}

%%%%%%%%%%%%%%%%%%%%%%%%%%%%%%%%%%%%%%%%
\paragraph{v1.5:} 2017/05/21

\begin{itemize}
\item
more complete structuring introduced
\item
|\childdocof| introduced
\item
|\childdoc| renamed to |\childdocmain|
\item
|\childredirect| renamed to |\childdocforward| and |\childdocforwardprefix|
and functionality expanded
\end{itemize}

%%%%%%%%%%%%%%%%%%%%%%%%%%%%%%%%%%%%%%%%
\paragraph{v1.0:} 2017/04/27

\begin{itemize}
\item
manual and install package
\item
first version published on CTAN
\end{itemize}

%%%%%%%%%%%%%%%%%%%%%%%%%%%%%%%%%%%%%%%%
\paragraph{v0.6:} 2017/04/26

\begin{itemize}
\item
redirection mechanism added
\end{itemize}

%%%%%%%%%%%%%%%%%%%%%%%%%%%%%%%%%%%%%%%%
\paragraph{v0.5:} 2017/04/26

\begin{itemize}
\item
functionality in definition file
\end{itemize}


%%%%%%%%%%%%%%%%%%%%%%%%%%%%%%%%%%%%%%%%%%%%%%%%%%%%%%%%%%%%%%%%%%%%%%%%%%%%%%%%
%%%%%%%%%%%%%%%%%%%%%%%%%%%%%%%%%%%%%%%%%%%%%%%%%%%%%%%%%%%%%%%%%%%%%%%%%%%%%%%%
%%%%%%%%%%%%%%%%%%%%%%%%%%%%%%%%%%%%%%%%%%%%%%%%%%%%%%%%%%%%%%%%%%%%%%%%%%%%%%%%
\appendix

\settowidth\MacroIndent{\rmfamily\scriptsize 000\ }

 \DocInput{childdoc.dtx}

\end{document}
%</driver>
% \fi
%
% %%%%%%%%%%%%%%%%%%%%%%%%%%%%%%%%%%%%%%%%%%%%%%%%%%%%%%%%%%%%%%%%%%%%%%%%%%%%%%
% %%%%%%%%%%%%%%%%%%%%%%%%%%%%%%%%%%%%%%%%%%%%%%%%%%%%%%%%%%%%%%%%%%%%%%%%%%%%%%
% \section{Sample}
%\iffalse
%<*samplemain>
%\fi
%
% The following presents a sample document
% with two chapters, two parts, a title page,
% a compile flag as well as three forwarding files to set the flag.
% It consists of eight |.tex| files:
% \begin{center}
% \begin{tabular}{ll}
% |cdocsamp.tex|&main file\\
% |cdocsch1.tex|&include file for chapter 1\\
% |cdocsch2.tex|&include file for chapter 2\\
% |cdocspt3.tex|&include file for part 3\\
% |cdocspt4.tex|&include file for part 4\\
% |cdocsdrf.tex|&forwarding file for main file in draft mode\\
% |cdocsfi1.tex|&forwarding file for final version of chapter 1\\
% |cdocsfi2.tex|&forwarding file for final version of chapter 2\\
% \end{tabular}
% \end{center}
% Each of the eight files can be compiled directly by the \LaTeX{} compiler.
%
% %%%%%%%%%%%%%%%%%%%%%%%%%%%%%%%%%%%%%%
% \paragraph{Main File.}
%
% The main file is called |cdocsamp.tex|.
%
% Load the \textsf{childdoc} definitions and
% declare the filename for the main document:
%    \begin{macrocode}
\input{childdoc.def}
\childdocmain{}
%    \end{macrocode}

% Optional override for |\version| flag:
%    \begin{macrocode}
%%\ifchilddoc\else\providecommand{\version}{draft}\fi
%    \end{macrocode}

% Define the default values for the |\version| flag
% (|final| for the main file and |draft| for childs):
%    \begin{macrocode}
\ifchilddoc
\providecommand{\version}{draft}
\else
\providecommand{\version}{final}
\fi
%    \end{macrocode}

% Load the standard document class:
%    \begin{macrocode}
\documentclass[12pt]{article}
%    \end{macrocode}

% Start the document body:
%    \begin{macrocode}
\begin{document}
%    \end{macrocode}

% Declare a title page.
% Print title, part of document being processed and version flag:
%    \begin{macrocode}
\addtocounter{page}{-1}
\begin{center}
{\LARGE\bfseries{}childdoc example\par}
\vspace{1cm}
\ifchilddoc
\ifchilddocmanual part\else chapter\fi:
`\childdocname' of `\childdocjob'\par
\else
main document: `\childdocjob'\par
\fi
version: \version\par
\end{center}
\newpage
%    \end{macrocode}

% Manually include selected file,
% otherwise process as usual:
%    \begin{macrocode}
\ifchilddocmanual
\section*{part `\childdocname'}
\input{\childdocname}
\else
%    \end{macrocode}

% Include the two chapters:
%    \begin{macrocode}
\include{cdocsch1}
\include{cdocsch2}
%    \end{macrocode}

% Include the two parts unless only chapters should be displayed:
%    \begin{macrocode}
\ifchilddoc\else
\section{part three}
\input{cdocspt3}
\section{part four}
\input{cdocspt4}
\fi
%    \end{macrocode}

% Process as usual until here:
%    \begin{macrocode}
\fi
%    \end{macrocode}

% End of document body:
%    \begin{macrocode}
\end{document}
%    \end{macrocode}
%\iffalse
%</samplemain>
%\fi
%
% %%%%%%%%%%%%%%%%%%%%%%%%%%%%%%%%%%%%%%
% \paragraph{Chapter Include Files.}
%
% The include files are called |cdocsch1.tex| and |cdocsch2.tex|.
%
%\iffalse
%<*samplechap1|samplechap2>
%\fi

% Optional override for |\version| flag:
%    \begin{macrocode}
%%\providecommand{\version}{final}
%    \end{macrocode}

% Include the main document:
%    \begin{macrocode}
\input{childdoc.def}
\childdocof{cdocsamp}
%    \end{macrocode}

%\iffalse
%</samplechap1|samplechap2>
%\fi
%
%\iffalse
%<*samplechap1>
%\fi
% Some text for chapter 1:
%    \begin{macrocode}
\section{one}
some text in chapter one
%    \end{macrocode}

%\iffalse
%</samplechap1>
%\fi
% Some text for chapter 2:
%\iffalse
%<*samplechap2>
%\fi
%    \begin{macrocode}
\section{two}
more text in chapter two
%    \end{macrocode}

%\iffalse
%</samplechap2>
%\fi
%
% %%%%%%%%%%%%%%%%%%%%%%%%%%%%%%%%%%%%%%
% \paragraph{Part Include Files.}
%
% The include files are called |cdocspt3.tex| and |cdocspt4.tex|.
%
%\iffalse
%<*samplepart3|samplepart4>
%\fi

% Optional override for |\version| flag:
%    \begin{macrocode}
%%\providecommand{\version}{final}
%    \end{macrocode}

% Include the main document:
%    \begin{macrocode}
\input{childdoc.def}
\childdocby{cdocsamp}
%    \end{macrocode}

%\iffalse
%</samplepart3|samplepart4>
%\fi
%
%\iffalse
%<*samplepart3>
%\fi
% Some text for part 3:
%    \begin{macrocode}
some text in part three
%    \end{macrocode}

%\iffalse
%</samplepart3>
%\fi
% Some text for part 4:
%\iffalse
%<*samplepart4>
%\fi
%    \begin{macrocode}
more text in part four
%    \end{macrocode}

%\iffalse
%</samplepart4>
%\fi
%
% %%%%%%%%%%%%%%%%%%%%%%%%%%%%%%%%%%%%%%
% \paragraph{Forwarding for a Complete Draft.}
%
% The following forwarding file |cdocsdrf.tex|
% compiles the main document in draft mode:
%\iffalse
%<*sampledraft>
%\fi
%    \begin{macrocode}
\def\version{draft}
\input{childdoc.def}
\childdocforward{cdocsamp}
%    \end{macrocode}

%\iffalse
%</sampledraft>
%\fi
%
% %%%%%%%%%%%%%%%%%%%%%%%%%%%%%%%%%%%%%%
% \paragraph{Forwarding for Final Version of the Chapters.}
%
% The following forwarding files |cdocsfn1.tex| and |cdocsfn2.tex|
% (with identical content)
% compile the final versions of the child documents
% |cdocsch1.tex| and |cdocsch2.tex|, respectively:
%\iffalse
%<*samplefinal>
%\fi
%    \begin{macrocode}
\def\version{final}
\input{childdoc.def}
\childdocforwardprefix[cdocsamp]{cdocsfn}{cdocsch}
%    \end{macrocode}

%\iffalse
%</samplefinal>
%\fi
%
% %%%%%%%%%%%%%%%%%%%%%%%%%%%%%%%%%%%%%%
% \paragraph{Command Line Processing.}
%
% The following three command lines generate the output files
% |cdocscld|, |cdocscl1| and |cdocscl2|
% which should be identical to
% |cdocsdrf|, |cdocsch1| and |cdocsfn2|, respectively:
% \begin{center}
% \begin{tabular}{l}
% |latex -jobname cdocscld \|\\
% |  "\def\version{draft}\input{childdoc.def}\childdocforward{cdocsamp}"|\\
% |latex -jobname cdocscl1 \|\\
% |  "\input{childdoc.def}\childdocforward[cdocsamp]{cdocsch1}"|\\
% |latex -jobname cdocscl2 \|\\
% |  "\def\version{final}\input{childdoc.def}\childdocforward{cdocsch2}"|
% \end{tabular}
% \end{center}
% Note that the trailing backslash on each first line
% merely continues the input to the second line
% (for convenient cut ant paste).
% Furthermore, the command |latex| can be replaced by any
% of its alternative versions such as |pdflatex|.
%
% %%%%%%%%%%%%%%%%%%%%%%%%%%%%%%%%%%%%%%%%%%%%%%%%%%%%%%%%%%%%%%%%%%%%%%%%%%%%%%
% %%%%%%%%%%%%%%%%%%%%%%%%%%%%%%%%%%%%%%%%%%%%%%%%%%%%%%%%%%%%%%%%%%%%%%%%%%%%%%
% \section{Implementation}
%\iffalse
%<*package>
%\fi
%
% This section describes the definitions file |childdoc.def|.

% The definitions cannot be loaded using |\usepackage| or |\RequirePackage|
% which has a mechanism to prevent loading a style file more than once.
% When loading the definitions by means of |\input|
% multiple instances have to be prevented manually:
%\iffalse
%This code needs to be before the `\ProvidesFile' directive
%which is defined at the beginning of this file.
%Therefore it is also placed there and commented out here.
%</package>
%<*discard>
%\fi
%    \begin{macrocode}
\ifdefined\childdocmain\endinput\fi
%    \end{macrocode}
%\iffalse
%</discard>
%<*package>
%\fi
%
% \macro{\ifchilddoc}
% \macro{\ifchilddocmanual}
% The conditional |\ifchilddoc| tells whether a
% child (true) or main (false) document is being compiled.
% The conditional |\ifchilddocmanual| tells whether
% the |\includeonly| mechanism is used (false) or
% the selection of child files must be performed manually (true).
% The definitions initialise to false:
%    \begin{macrocode}
\newif\ifchilddoc
\newif\ifchilddocmanual
%    \end{macrocode}

% \macro{\childdocname}
% \macro{\childdocjob}
% The macro |\childdocname| stores the name of the main document
% to be compiled. The macro |\childdocjob| stores the name of
% the document on which the \LaTeX{} compiler was originally invoked.
% The content of |\jobname| cannot be compared
% to filenames specified in the source due to different catcodes.
% The following code rescans |\jobname|, stores the result
% in |\childdocname| and saves a copy in |\childdocjob|:
%    \begin{macrocode}
\edef\childdocname{\scantokens\expandafter{\jobname\noexpand}}
\let\childdocjob\childdocname
%    \end{macrocode}

% \macro{\childdocdisable}
% The macro |\childdocdisable| prevents the main file
% from being processed more than once.
% At this stage, the main document command |\childdocmain|
% is assumed to be called once again where it should do nothing.
% Any subsequent call to it should prevent
% a secondary processing of the main document
% It overwrites the forwarding commands
% |\childdocof| and |\childdocforward|
% with empty macros to prevent further inclusions of the main document:
%    \begin{macrocode}
\newcommand{\childdocdisable}
{
  \renewcommand{\childdocmain}[1]{\renewcommand{\childdocmain}[1]{\endinput}}
  \renewcommand{\childdocof}[1]{}
  \renewcommand{\childdocby}[2][]{}
  \renewcommand{\childdocforward}[2][]{}
  \renewcommand{\childdocdisable}{}
}
%    \end{macrocode}

% \macro{\childdocmain}
% The macro |\childdocmain| is to be called at the top of the main file
% with nothing or the main filename (without extension) as argument.
% First, it breaks loops.
% If the argument is not empty and does not match |\childdocname|
% (which is set by the first inclusion of |childdoc.def|),
% |\ifchilddoc| is set to true, |\includeonly| is applied to the child file
% and |\jobname| is set to the main file
% (for proper handling of |.aux| files):
%    \begin{macrocode}
\newcommand{\childdocmain}[1]
{
  \childdocdisable\childdocmain{}
  \if?#1?\else
    \begingroup
      \def\childdoctmp{#1}
      \ifx\childdoctmp\childdocname
        \def\childdoctmp{}
      \else
        \def\childdoctmp
        {
          \childdoctrue
          \includeonly{\childdocname}
          \def\childdocjob{#1}
          \def\jobname{#1}
        }
      \fi
      \expandafter
    \endgroup
    \childdoctmp
  \fi
}
%    \end{macrocode}

% \macro{\childdocof}
% The command |\childdocof| redirects
% compilation to the main file |#1|.
%    \begin{macrocode}
\newcommand{\childdocof}[1]
{
  \childdocdisable
  \childdoctrue
  \includeonly{\childdocname}
  \def\jobname{#1}
  \def\childdocjob{#1}
  \input{#1}
}
%    \end{macrocode}

% \macro{\childdocby}
% The command |\childdocby| ....
%    \begin{macrocode}
\newcommand{\childdocby}[2][]
{
  \childdocdisable
  \childdoctrue
  \childdocmanualtrue
  \if?#1?\else
    \def\jobname{#2}
  \fi
  \def\childdocjob{#2}
  \input{#2}
  \endinput
}
%    \end{macrocode}

% \macro{\childdocforward}
% The command |\childdocforward| redirects
% compilation to the main file or
% (if the optional argument is given) a child file.
% Parameters are set as if the main file
% or a child file starting with |\childdocof| was compiled.
% Then compilation is handed over to the main file:
%    \begin{macrocode}
\newcommand{\childdocforward}[2][]
{
  \begingroup
    \if?#1?
      \def\childdoctmp
      {
        \def\childdocname{#2}
        \def\childdocjob{#2}
        \def\jobname{#2}
        \input{#2}
        \endinput
      }
    \else
      \def\childdoctmp
      {
        \childdocdisable
        \def\childdocname{#2}
        \childdoctrue
        \includeonly{#2}
        \def\childdocjob{#1}
        \def\jobname{#1}
        \input{#1}
        \endinput
      }
    \fi
    \expandafter
  \endgroup
  \childdoctmp
}
%    \end{macrocode}

% \macro{\childdocforwardprefix}
% The command |\childdocforwardprefix| redirects
% compilation to the main or a child file by means of a pattern.
% The prefix |#1| in the current filename is replaced by |#2|
% and the suffix of the current filename is kept
% (it is assumed that the filename does not contain the substring `|~~~|'
% which is used as a delimiter).
% Compilation is handed over to the new file by |\childdocforward|:
%    \begin{macrocode}
\newcommand{\childdocforwardprefix}[3][]
{
  \begingroup
    \def\childdocextract #2##1~~~{\def\childdoctmp{\childdocforward[#1]{#3##1}}}
    \expandafter\childdocextract\childdocname~~~
    \expandafter
  \endgroup
  \childdoctmp
}
%    \end{macrocode}

% \macro{\childdoc}
% The deprecated macro |\childdoc| is a legacy version of |\childdocmain|:
%    \begin{macrocode}
\newcommand{\childdoc}{\childdocmain}
%    \end{macrocode}

% \macro{\childdocredirect}
% The deprecated macro |\childdocredirect| is a legacy version
% of |\childdocforward| and |\childdocforwardprefix|:
%    \begin{macrocode}
\newcommand{\childdocredirect}[2][]
{
  \begingroup
    \if?#1?
      \def\childdoctmp{\childdocforward{#2}}
    \else
      \def\childdoctmp{\childdocforwardprefix{#1}{#2}}
    \fi
    \expandafter
  \endgroup
  \childdoctmp
}
%    \end{macrocode}

%\iffalse
%</package>
%\fi
%
\endinput
|\\
|\childdocmain{|\textit{main}|}|\\
\end{tabular}
\end{center}
%
If |\jobname| does not match the argument \textit{main} of |\childdocmain|,
it is assumed that |\jobname| points to the child file to be compiled.
When using |\childdocmain| with the main file specified as argument,
it suffices to start a child file
with just |\input{|\textit{main}|}|
without loading of the package and using |\childdocof|.
If instead all processing is done
with the appropriate \textsf{childdoc} directives,
the argument of \textit{main} of |\childdocmain| can be empty.

An alternative version of the command line processing described
in \secref{sec:commandline} using the detection mechanism reads:
%
\begin{center}
|... -jobname "|\textit{target}|" "|[\textit{flags}]%
[|\def\jobname{|\textit{dest}|}|]|\input{|\textit{main}|}"|
\end{center}

%%%%%%%%%%%%%%%%%%%%%%%%%%%%%%%%%%%%%%%%%%%%%%%%%%%%%%%%%%%%%%%%%%%%%%%%%%%%%%%%
\subsection{Manual Code}
\label{sec:manual}

In case one cannot be certain whether the definitions file |childdoc.def|
is installed on the target \TeX{} distribution
and one prefers not to ship it,
it is conceivable to paste a few relevant commands into the sources.

To that end, drop all statements |% \iffalse
%
% childdoc.dtx Copyright (C) 2017-2018 Niklas Beisert
%
% This work may be distributed and/or modified under the
% conditions of the LaTeX Project Public License, either version 1.3
% of this license or (at your option) any later version.
% The latest version of this license is in
%   http://www.latex-project.org/lppl.txt
% and version 1.3 or later is part of all distributions of LaTeX
% version 2005/12/01 or later.
%
% This work has the LPPL maintenance status `maintained'.
%
% The Current Maintainer of this work is Niklas Beisert.
%
% This work consists of the files childdoc.dtx and childdoc.ins
% and the derived files childdoc.def and cdocsamp.tex with
% cdocsch1.tex, cdocsch2.tex, cdocsdrf.tex, cdocsfn1.tex, cdocsfn2.tex.
%
%<package>\ifdefined\childdocmain\endinput\fi
%<package>\ProvidesFile{childdoc.def}[2018/12/30 v2.0 child document driver]
%<samplemain>\ProvidesFile{cdocsamp.tex}[2018/12/30 v2.0 sample for childdoc]
%<*driver>
%\ProvidesFile{childdoc.drv}[2018/12/30 v2.0 childdoc reference manual file]
\PassOptionsToClass{10pt,a4paper}{article}
\documentclass{ltxdoc}

\usepackage[margin=35mm]{geometry}
\usepackage{hyperref}
\usepackage{hyperxmp}
\usepackage[usenames]{color}

\hypersetup{colorlinks=true}
\hypersetup{pdfstartview=FitH}
\hypersetup{pdfpagemode=UseNone}
\hypersetup{pdfsource={}}
\hypersetup{pdflang={en-UK}}
\hypersetup{pdfcopyright={Copyright 2017-2018 Niklas Beisert.
  This work may be distributed and/or modified under the
  conditions of the LaTeX Project Public License, either version 1.3
  of this license or (at your option) any later version.}}
\hypersetup{pdflicenseurl={http://www.latex-project.org/lppl.txt}}
\hypersetup{pdfcontactaddress={ETH Zurich, ITP, HIT K,
  Wolfgang-Pauli-Strasse 27}}
\hypersetup{pdfcontactpostcode={8093}}
\hypersetup{pdfcontactcity={Zurich}}
\hypersetup{pdfcontactcountry={Switzerland}}
\hypersetup{pdfcontactemail={nbeisert@itp.phys.ethz.ch}}
\hypersetup{pdfcontacturl={http://people.phys.ethz.ch/\xmptilde nbeisert/}}

\newcommand{\secref}[1]{\hyperref[#1]{section \ref*{#1}}}

\parskip1ex
\parindent0pt
\let\olditemize\itemize
\def\itemize{\olditemize\parskip0pt}

\begin{document}

\title{The \textsf{childdoc} Package}
\hypersetup{pdftitle={The childdoc Package}}
\author{Niklas Beisert\\[2ex]
  Institut f\"ur Theoretische Physik\\
  Eidgen\"ossische Technische Hochschule Z\"urich\\
  Wolfgang-Pauli-Strasse 27, 8093 Z\"urich, Switzerland\\[1ex]
  \href{mailto:nbeisert@itp.phys.ethz.ch}
  {\texttt{nbeisert@itp.phys.ethz.ch}}}
\hypersetup{pdfauthor={Niklas Beisert}}
\hypersetup{pdfsubject={Manual for the LaTeX2e Package childdoc}}
\date{30 December 2018, \textsf{v2.0}}
\maketitle

\begin{abstract}\noindent
\textsf{childdoc} is a \LaTeXe{} package
that enables the direct compilation
of document sections included by |\include|
to individual files.
\end{abstract}

\begingroup
\parskip0ex
\tableofcontents
\endgroup

%%%%%%%%%%%%%%%%%%%%%%%%%%%%%%%%%%%%%%%%%%%%%%%%%%%%%%%%%%%%%%%%%%%%%%%%%%%%%%%%
%%%%%%%%%%%%%%%%%%%%%%%%%%%%%%%%%%%%%%%%%%%%%%%%%%%%%%%%%%%%%%%%%%%%%%%%%%%%%%%%
\section{Introduction}

\LaTeX{} provides a mechanism to structure a large document (such as a book)
into a main file and several child files (containing the chapters)
using the |\include| command.
This mechanism is beneficial for documents
which span hundreds of pages in order to
make the source file(s) more manageable.
Moreover, compilation can be restricted to
selected child files by means of the |\includeonly| command.
The latter feature can be used to reduce the compilation time while editing
(this was significantly more useful in the earlier days of \LaTeX{})
or to generate a smaller document which is easier to navigate.
Another application of |\includeonly| is to generate
documents consisting of selected parts of the complete document.

However, there are a few drawbacks of the plain |\include| mechanism:
\begin{itemize}
\item
The child files cannot be compiled on their own,
they can only be compiled via the main file.
A naive editing environment
(such as a text editor with an option
to have the current file processed by \LaTeX)
may require one to switch to the main file before compiling;
attempting to compile the child file produces errors.
\item
The main file must be modified (each time)
to adjust the |\includeonly| command
to the present needs. This easily leaves the main file in a messy state.
\item
The generated document will always carry the filename
of the main document. This is inconvenient if
several child files are to be compiled and
to be kept for distribution.
\end{itemize}

The present package provides a simple interface
to make child files individually compilable by \LaTeX{}.
Compiling a child file then has the same effect as compiling
the main file with an |\includeonly| command
to select the appropriate child.
Moreover the generated document will carry the name of the child
rather than the main file.
This resolves all three above issues.

This feature is meant to make the editing of books,
thesis documents and lecture notes somewhat more convenient.
However, the package can also be used efficiently for
composing a series of documents (such as exercise sheets)
which are typically distributed individually.
It then assists the author in generating the individual documents
(potentially in different versions)
as well as a document containing the collected series.
Another application is in developing style files
or other kinds of included material
where compilation of the style file could redirect
to a sample or test file.

%%%%%%%%%%%%%%%%%%%%%%%%%%%%%%%%%%%%%%%%%%%%%%%%%%%%%%%%%%%%%%%%%%%%%%%%%%%%%%%%
%%%%%%%%%%%%%%%%%%%%%%%%%%%%%%%%%%%%%%%%%%%%%%%%%%%%%%%%%%%%%%%%%%%%%%%%%%%%%%%%
\section{Usage}

First of all, the package \textsf{childdoc} is \emph{not} a standard
\LaTeXe{} |.sty| style file! Therefore it needs to be invoked in
a non-standard way.

%%%%%%%%%%%%%%%%%%%%%%%%%%%%%%%%%%%%%%%%%%%%%%%%%%%%%%%%%%%%%%%%%%%%%%%%%%%%%%%%
\subsection{Included Files}
\label{sec:include}

%%%%%%%%%%%%%%%%%%%%%%%%%%%%%%%%%%%%%%%%
\DescribeMacro{\childdocmain}
To use the package, add the commands
\begin{center}
\begin{tabular}{l}
|\input{childdoc.def}|\\
|\childdocmain{}|\\
\end{tabular}
\end{center}
at the very top of the main \LaTeX{} file,
in particular \emph{before} the |\documentclass| statement!
The argument of |\childdocmain| should be left empty
(but it must be present).

%%%%%%%%%%%%%%%%%%%%%%%%%%%%%%%%%%%%%%%%
\DescribeMacro{\childdocof}
Furthermore, add the commands
\begin{center}
\begin{tabular}{l}
|\input{childdoc.def}|\\
|\childdocof{|\textit{main}|}|\\
\end{tabular}
\end{center}
at the top of every child file \textit{child}
which is included by |\include{|\textit{child}|}|
from within the main file
(or at least for those files to be compiled individually).
The argument \textit{main} must be the filename of the main file.

There are a couple of
considerations in setting up the main and child documents:

%%%%%%%%%%%%%%%%%%%%%%%%%%%%%%%%%%%%%%%%
\paragraph{Restrictions.}

Please note the following restrictions:
\begin{itemize}
\item
|\childdocmain| must be called with one argument \textit{main}
to ensure compatibility with earlier version of the package.
It must either be empty (|\childdocmain{}|)
or precisely match the filename of the main file in which it is specified.
See \secref{sec:detection} for further information.
\item
The filename \textit{main} must be specified without the |.tex| extension.
\item
The filename \textit{main} is case sensitive
(even in case-insensitive file systems)
due to internal string comparison.
\item
The argument \textit{main} should be fully expanded, it cannot be a macro.
\item
Subdirectories and special characters should be avoided in filenames.
\item
The command |\childdocmain{|\textit{main}|}| must be followed by a whitespace.
It should not be followed immediately by another command
or by a comment mark `|%|'.
This is because the \TeX{} parser reads the token immediately following
the argument of |\childdocmain| and puts it
at the beginning of every child section;
however, a white\-space is ignored.
\end{itemize}

%%%%%%%%%%%%%%%%%%%%%%%%%%%%%%%%%%%%%%%%
\paragraph{Content of Main File.}

It is advisable to place all content in the child files included by |\include|.
Any output contained in the main file will appear in all child documents
unless suppressed manually;
it cannot be suppressed automatically by the |\includeonly| directive
and thus should normally be avoided.
A method to include some content in the main file
by means of conditional processing is described in \secref{sec:conditional}.

%%%%%%%%%%%%%%%%%%%%%%%%%%%%%%%%%%%%%%%%
\paragraph{Page Numbering.}

When only a part of the document is compiled,
the appropriate numbering of pages
(as well as other status parameters)
is determined from the |.aux| files.
The latter contain information from previous passes.
However this information needs to propagate through
all intermediate child documents.
Therefore the page numbering in child documents may well
be inconsistent until the complete document is compiled at least once.

A useful (if unconventional) way to always ensure a consistent
page numbering is to restart the numbering in each child document
and denote the pages by `\textit{child}|.|\textit{page}'
where \textit{child} represents the chapter/section number of the child file.
This can be achieved by the command
|\numberwithin{page}{|\textit{child}|}|
of the \textsf{amsmath} package
where \textit{child} can be |chapter| or |section|
depending on the chosen structuring.
Alternatively, one can modify the macro |\thepage| appropriately
and reset the counter |page| at the start of each child file.

%%%%%%%%%%%%%%%%%%%%%%%%%%%%%%%%%%%%%%%%%%%%%%%%%%%%%%%%%%%%%%%%%%%%%%%%%%%%%%%%
\subsection{Conditional Processing}
\label{sec:conditional}

The package provides a mechanism to compile different versions
of a document. To customise the versions further some conditional processing
can come in handy to distinguish which version is being compiled.
The package provides two macros to describe the compilation context:

%%%%%%%%%%%%%%%%%%%%%%%%%%%%%%%%%%%%%%%%
\DescribeMacro{\ifchilddoc}
The conditional |\ifchilddoc| distinguishes between the compilation of
child documents and the main document:
%
\begin{center}
|\ifchilddoc |\textit{child-code}| |[|\||else |\textit{main-code}]| \||fi|
\end{center}

%%%%%%%%%%%%%%%%%%%%%%%%%%%%%%%%%%%%%%%%
\DescribeMacro{\childdocname}
\DescribeMacro{\childdocjob}
The macro |\childdocname| contains the filename (without extension)
of the main or child file being processed.
Note that |\childdocjob| will always contain the name of the main file.

%%%%%%%%%%%%%%%%%%%%%%%%%%%%%%%%%%%%%%%%
\paragraph{Title Page.}

Conditional processing can be used to include a title or banner page
in the main document when proper precautions are taken.
Importantly, the code in the main file should ensure that the page counter
(as well as other status parameters which are stored in the |.aux| files)
takes the same value after the conditional processing.
Otherwise the page numbers may take divergent values
depending on which part is compiled.

For example, a title page could be declared by:
%
\begin{center}
\begin{tabular}{l}
|\ifchilddoc\||else|\\
|\addtocounter{page}{-1}|\\
\textit{code for title page}\\
|\newpage|\\
|\||fi|
\end{tabular}
\end{center}
%
A banner page for the child documents can be generated by:
%
\begin{center}
\begin{tabular}{l}
|\ifchilddoc|\\
|\addtocounter{page}{-1}|\\
\textit{code for banner page}\\
|\newpage|\\
|\||fi|
\end{tabular}
\end{center}
%
Here one could write a message such as:
\begin{center}
|This is the part \childdocname{} of \childdocjob{}.|
\end{center}

%%%%%%%%%%%%%%%%%%%%%%%%%%%%%%%%%%%%%%%%%%%%%%%%%%%%%%%%%%%%%%%%%%%%%%%%%%%%%%%%
\subsection{Flags}
\label{sec:flags}

The package makes it easy to generate different versions
of the main or child documents.
To this end compilation flags can be defined
and assigned different default values.
They will be particularly useful in conjunction
with the forwarding mechanism described in \secref{sec:forward}.

For example, it may be useful to have a flag |\version|
which can be set to |draft| or |final|.
The document source will contain some conditional code
depending on the value of |\version|.
Suppose further, the flag should default to |final| for the main file
and to |draft| for child files
which is a natural assignment for editing the document.
This is achieved by placing the following code
in the preamble of the main document
(below the |\childdocmain| directive):
%
\begin{center}
\begin{tabular}{l}
|\ifchilddoc|\\
|\providecommand{\version}{draft}|\\
|\||else|\\
|\providecommand{\version}{final}|\\
|\||fi|
\end{tabular}
\end{center}
%
The definition by |\providecommand| makes sure
that previous definitions are not overwritten.
Further statements |\providecommand{\version}{...}|
can thus be added before the above code to override it.

For the main file, one might add a line
(between |\childdocmain| and the above block)
%
\begin{center}
|%\ifchilddoc\||else\providecommand{\version}{draft}\||fi|
\end{center}
%
which can be uncommented to produce a draft version.
Likewise one can add a line to the very top of a child file
(above the |\childdocof{|\textit{main}|}| directive)
%
\begin{center}
|%\providecommand{\version}{final}|
\end{center}
%
which can be uncommented to produce the final version of this child document.

%%%%%%%%%%%%%%%%%%%%%%%%%%%%%%%%%%%%%%%%%%%%%%%%%%%%%%%%%%%%%%%%%%%%%%%%%%%%%%%%
\subsection{Forwarding}
\label{sec:forward}

Different versions of the main or child documents
using compilation flags as described in \secref{sec:flags}
can be (permanently) stored in different files
for convenient compilation, viewing and distribution.
To this end, the package defines a command
to pass on compilation to a different file:

%%%%%%%%%%%%%%%%%%%%%%%%%%%%%%%%%%%%%%%%
\DescribeMacro{\childdocforward}
The command |\childdocforward| redirects processing to
another source file:
%
\begin{center}
\begin{tabular}{l}
|\input{childdoc.def}|\\
|\childdocforward[|\textit{main}|]{|\textit{dest}|}|\\
\end{tabular}
\end{center}
%
The argument \textit{dest} is the destination file
(without extension).
It should be the main file or one of the child files.
Note that further \textsf{childdoc} directives
such as |\childdocof| and |\childdocforward|
in the indicated file will be processed in this form.
The optional argument \textit{main}
passes on directly to the main file \textit{main}
while pretending to compile the child \textit{dest}.
This form behaves as if \textit{dest}
issues |\childdocof{|\textit{main}|}| right away,
and no further \textsf{childdoc} directives will be processed.

%%%%%%%%%%%%%%%%%%%%%%%%%%%%%%%%%%%%%%%%
\DescribeMacro{\...prefix}
In the alternative form |\childdocforwardprefix|,
%
\begin{center}
\begin{tabular}{l}
|\input{childdoc.def}|\\
|\childdocforwardprefix[|\textit{main}|]{|\textit{prefix}|}{|\textit{dest}|}|
\end{tabular}
\end{center}
%
the destination file is determined by a pattern
depending on the current file:
To make this work, the current file must be called
`{\textit{prefix}\hspace{0.2em}\textit{suffix}}'
with \textit{prefix} matching precisely the argument.
Processing is then passed on to the file
`{\textit{dest}\hspace{0.2em}\textit{suffix}}'.
Surely, the same effect is achieved by
directly specifying the
argument `{\textit{dest}\hspace{0.2em}\textit{suffix}}'
in the first form.
However, that requires to set up a different file
for each child. With the alternative form of the command
all these files can have exactly the same content
which simplifies setting them up and maintaining them.

For example, the following file |draft.tex|
with a compilation flag |\version| as described in \secref{sec:flags}
compiles the main document as a draft:
%
\begin{center}
\begin{tabular}{l}
|\def\version{draft}|\\
|\input{childdoc.def}|\\
|\childdocforward{|\textit{main}|}|
\end{tabular}
\end{center}
%
Likewise, the following files |final|\textit{nn}|.tex|
compile the final version of the child document
|child|\textit{nn}|.tex|:
%
\begin{center}
\begin{tabular}{l}
|\def\version{final}|\\
|\input{childdoc.def}|\\
|\childdocforwardprefix{final}{child}|
\end{tabular}
\end{center}
%

Note that when several versions of a main file and/or of each child file
are to be generated, it may be convenient to set up a |Makefile| or
shell script to automatise the process.

%%%%%%%%%%%%%%%%%%%%%%%%%%%%%%%%%%%%%%%%%%%%%%%%%%%%%%%%%%%%%%%%%%%%%%%%%%%%%%%%
\subsection{Command Line Processing}
\label{sec:commandline}

The effect of redirection files can also be achieved by invoking
the \LaTeX{} compiler with a more elaborate command line.
Most conveniently this should be done as part
of a shell script or a |Makefile|.

When using \textsf{childdoc} in the main file, the following
command lines effectively perform a redirection
(note that depending on the shell being used,
backslashes may have to be doubled: `|\|' $\to$ `|\\|'):
%
\begin{center}
|... -jobname "|\textit{target}|" |\\|"|[\textit{flags}]%
|\input{childdoc.def}\childdocforward[|\textit{main}|]{|\textit{dest}|}"|
\end{center}
%
Here \textit{target} is the name of the output file,
\textit{main} is the name of the main file
and \textit{dest} is the name of the main or child file to be processed
(all filenames without extensions).
The optional argument \textit{main} can be omitted
if \textit{main} matches \textit{dest}.
Optionally, compilation \textit{flags} can be defined via |\def| commands.
This command line makes the \TeX{} engine believe
it is compiling the file \textit{target}
whose content is specified as the latter parameter.
The provided code then forwards the processing to
\textit{main} or \textit{dest} as described in \secref{sec:forward}.

%%%%%%%%%%%%%%%%%%%%%%%%%%%%%%%%%%%%%%%%%%%%%%%%%%%%%%%%%%%%%%%%%%%%%%%%%%%%%%%%
\subsection{Include by Input}
\label{sec:input}

Including child documents by |\include| has some restrictions by design.
Most notably, the content of a child document always occupies
its own set of pages; pages cannot be shared between child documents.
Usually, this behaviour makes perfect sense
because each child document contain an essential part of the document.
However, in some situations it may be desirable to compose
a document from a collection of parts
without having mandatory page breaks between then.
For this case, the package
provides a mechanism to include parts
by |\input| which can also be processed individually.
However, by construction this mechanism
requires manual handling of the content to be output.

%%%%%%%%%%%%%%%%%%%%%%%%%%%%%%%%%%%%%%%%
\DescribeMacro{\ifchilddocmanual}
The main file should be prepared as usual, see \secref{sec:include}.
However, the document body must make a distinction
between processing of an individual part and of the main document, e.g.:
%
\begin{center}
\begin{tabular}{l}
|\ifchilddocmanual|\\
|\input{\childdocname}|\\
|\||else|\\
\textit{document body with }|\input{|\textit{part}|}|\\
|\||fi|
\end{tabular}
\end{center}
%
The conditional |\ifchilddocmanual| is true whenever
a part to be included by |\input| is being compiled,
and the name of the part is stored in |\childdocname|.

%%%%%%%%%%%%%%%%%%%%%%%%%%%%%%%%%%%%%%%%
\DescribeMacro{\childdocby}
Each part to be included by |\input| should start with:
%
\begin{center}
\begin{tabular}{l}
|\input{childdoc.def}|\\
|\childdocby{|\textit{main}|}|\\
\end{tabular}
\end{center}
%
The directive |\childdocby| is similar to |\childdocof|
described in \secref{sec:include},
but the subsequent selection of content must be done manually.
To that end, both |\ifchilddoc| and |\ifchilddocmanual|
will be true upon processing of a part,
and the name of the part is stored in |\childdocname|.
Note that |\jobname| will be set to the filename of the current part
so that each part receives an individual |.aux| file
that does not interfere with the |.aux| file(s) of the main document.
This behaviour can be altered by the alternative form
|\childdocby[*]{|\textit{main}|}| (with a non-empty optional argument)
which uses the |.aux| file of the main document
by setting |\jobname| to \textit{main}.

%%%%%%%%%%%%%%%%%%%%%%%%%%%%%%%%%%%%%%%%%%%%%%%%%%%%%%%%%%%%%%%%%%%%%%%%%%%%%%%%
\subsection{Driver Development}
\label{sec:driver}

The \textsf{childdoc} mechanism can also be use for the development
of definition files such as \LaTeX{} styles or classes.
This case differs from the above setup with multiple parts
included by |\include| in that no |\includeonly| should be invoked.
This can be achieved by starting the include file
(before |\ProvidesPackage|) with:
%
\begin{center}
\begin{tabular}{l}
|\input{childdoc.def}|\\
|\childdocforward{|\textit{main}|}|\\
\end{tabular}
\end{center}
%
or alternatively with:
%
\begin{center}
\begin{tabular}{l}
|\input{childdoc.def}|\\
|\childdocby{|\textit{main}|}|\\
\end{tabular}
\end{center}
%
Both forms have slightly different effects as described above.
The main file is prepared as usual, see \secref{sec:include}.

%%%%%%%%%%%%%%%%%%%%%%%%%%%%%%%%%%%%%%%%%%%%%%%%%%%%%%%%%%%%%%%%%%%%%%%%%%%%%%%%
\subsection{Legacy Detection}
\label{sec:detection}

The directive |\childdocmain| in the main file can detect
whether the complete document or merely a child is to be compiled
even without using the directive |\childdocof|.
This method is deprecated because it is less robust
and there is no compelling reason to use it;
it is merely provided for backward compatibility
and it may be removed in future versions.

If the detection mechanism is to be used,
it is mandatory to correctly specify
the filename of the main file as the argument of |\childdocmain|:
%
\begin{center}
\begin{tabular}{l}
|\input{childdoc.def}|\\
|\childdocmain{|\textit{main}|}|\\
\end{tabular}
\end{center}
%
If |\jobname| does not match the argument \textit{main} of |\childdocmain|,
it is assumed that |\jobname| points to the child file to be compiled.
When using |\childdocmain| with the main file specified as argument,
it suffices to start a child file
with just |\input{|\textit{main}|}|
without loading of the package and using |\childdocof|.
If instead all processing is done
with the appropriate \textsf{childdoc} directives,
the argument of \textit{main} of |\childdocmain| can be empty.

An alternative version of the command line processing described
in \secref{sec:commandline} using the detection mechanism reads:
%
\begin{center}
|... -jobname "|\textit{target}|" "|[\textit{flags}]%
[|\def\jobname{|\textit{dest}|}|]|\input{|\textit{main}|}"|
\end{center}

%%%%%%%%%%%%%%%%%%%%%%%%%%%%%%%%%%%%%%%%%%%%%%%%%%%%%%%%%%%%%%%%%%%%%%%%%%%%%%%%
\subsection{Manual Code}
\label{sec:manual}

In case one cannot be certain whether the definitions file |childdoc.def|
is installed on the target \TeX{} distribution
and one prefers not to ship it,
it is conceivable to paste a few relevant commands into the sources.

To that end, drop all statements |\input{childdoc.def}|
and perform the replacements as outlined below.
Instead of |\childdocmain{|\textit{main}|}| add the following code
to the top of the main file:
%
\begin{center}
\begin{tabular}{l}
|\||ifdefined\childdocname\endinput\||fi\newif\ifchilddoc|\\
|\edef\childdocname{\scantokens\expandafter{\jobname\noexpand}}|\\
|\def\childdocmain{|\textit{main}|}\||ifx\childdocmain\childdocname\||else|\\
|\childdoctrue\includeonly{\childdocname}\let\jobname\childdocmain\||fi|\\
\end{tabular}
\end{center}
%
Instead of |\childdocof{|\textit{main}|}| just include the main file
at the top of each child file:
%
\begin{center}
|\input{|\textit{main}|}|
\end{center}
%
A simple redirection |\childdocforward{|\textit{dest}|}| is achieved by:
%
\begin{center}
|\def\jobname{|\textit{dest}|}\input{\jobname}|
\end{center}
%
The redirection with prefix
|\childdocforwardprefix[|\textit{prefix}|]{|\textit{dest}|}|
is accomplished by:
%
\begin{center}
\begin{tabular}{l}
|{\edef\jobname{\scantokens\expandafter{\jobname\noexpand}}|\\
|\def\redirectjob |\textit{prefix}|#1~~~{\gdef\jobname{|\textit{dest}|#1}}|\\
|\expandafter\redirectjob\jobname~~~}\input{\jobname}|
\end{tabular}
\end{center}

In an alternative approach,
child documents can be compiled by a specific command line
without additional code or specific definitions:
%
\begin{center}
|... -jobname "|\textit{target}|" "|[\textit{flags}]%
|\includeonly{|\textit{dest}|}\input{|\textit{main}|}"|
\end{center}
%

%%%%%%%%%%%%%%%%%%%%%%%%%%%%%%%%%%%%%%%%%%%%%%%%%%%%%%%%%%%%%%%%%%%%%%%%%%%%%%%%
%%%%%%%%%%%%%%%%%%%%%%%%%%%%%%%%%%%%%%%%%%%%%%%%%%%%%%%%%%%%%%%%%%%%%%%%%%%%%%%%
\section{Information}

%%%%%%%%%%%%%%%%%%%%%%%%%%%%%%%%%%%%%%%%%%%%%%%%%%%%%%%%%%%%%%%%%%%%%%%%%%%%%%%%
\subsection{Copyright}

Copyright \copyright{} 2017--2018 Niklas Beisert

This work may be distributed and/or modified under the
conditions of the \LaTeX{} Project Public License, either version 1.3
of this license or (at your option) any later version.
The latest version of this license is in
  \url{http://www.latex-project.org/lppl.txt}
and version 1.3 or later is part of all distributions of \LaTeX{}
version 2005/12/01 or later.

This work has the LPPL maintenance status `maintained'.

The Current Maintainer of this work is Niklas Beisert.

This work consists of the files |README.txt|, |childdoc.ins| and |childdoc.dtx|
as well as the derived files |childdoc.def|, |cdocsamp.tex|
with |cdocsch1.tex|, |cdocsch2.tex|, |cdocspt3.tex|, |cdocspt4.tex|,
|cdocsdrf.tex|, |cdocsfn1.tex|, |cdocsfn2.tex|
as well as |childdoc.pdf|.

%%%%%%%%%%%%%%%%%%%%%%%%%%%%%%%%%%%%%%%%%%%%%%%%%%%%%%%%%%%%%%%%%%%%%%%%%%%%%%%%
\subsection{Files and Installation}

The package consists of the files:
%
\begin{center}
\begin{tabular}{ll}
    |README.txt|   & readme file \\
    |childdoc.ins| & installation file \\
    |childdoc.dtx| & source file \\
    |childdoc.def| & definition file \\
    |cdocsamp.tex| & sample main file \\
    |cdocsch1.tex| & sample include file \\
    |cdocsch2.tex| & sample include file \\
    |cdocspt3.tex| & sample part file \\
    |cdocspt4.tex| & sample part file \\
    |cdocsdrf.tex| & sample redirection file \\
    |cdocsfn1.tex| & sample redirection file \\
    |cdocsfn2.tex| & sample redirection file \\
    |childdoc.pdf| & manual
\end{tabular}
\end{center}
%
The distribution consists of the files
|README.txt|, |childdoc.ins| and |childdoc.dtx|.
%
\begin{itemize}
\item
Run (pdf)\LaTeX{} on |childdoc.dtx|
to compile the manual |childdoc.pdf| (this file).
\item
Run \LaTeX{} on |childdoc.ins| to create the definitions file |childdoc.def|
and the sample |cdocsamp.tex| with include files
|cdocsch1.tex|, |cdocsch2.tex|, |cdocspt3.tex|, |cdocspt4.tex|,
|cdocsdrf.tex|, |cdocsfn1.tex|, |cdocsfn2.tex|.
Then copy the file |childdoc.def| to an appropriate directory of your \LaTeX{}
distribution, e.g.\ \textit{texmf-root}|/tex/latex/childdoc|.
\end{itemize}

%%%%%%%%%%%%%%%%%%%%%%%%%%%%%%%%%%%%%%%%%%%%%%%%%%%%%%%%%%%%%%%%%%%%%%%%%%%%%%%%
\subsection{Related CTAN Packages}

There are several other packages which offer a similar functionality:
%
\begin{itemize}
\item
The packages
\href{http://ctan.org/pkg/docmute}{\textsf{docmute}},
\href{http://ctan.org/pkg/includex}{\textsf{includex}} and
\href{http://ctan.org/pkg/standalone}{\textsf{standalone}}
provide commands to include only the document body of
a child file thus allowing both files to be compiled individually.
\item
The packages \href{http://ctan.org/pkg/subdocs}{\textsf{subdocs}}
and \href{http://ctan.org/pkg/subfiles}{\textsf{subfiles}}
provide structures in which the main and child documents can be
encapsulated and allowing them to be compiled individually.
The inclusion mechanism is different from the conventional |\include|.
\item
The package \href{http://ctan.org/pkg/combine}{\textsf{combine}}
is an elaborate solution to combine several documents into one.
\end{itemize}
%
See also the CTAN topic \href{http://ctan.org/topic/subdocs}{\textsf{subdocs}}
for further related packages.
The present package differs from the above solutions in that
a document structure constructed with the conventional |\include| mechanism
just needs two extra commands at the top of every file
such that all constituent files can be compiled individually.

%%%%%%%%%%%%%%%%%%%%%%%%%%%%%%%%%%%%%%%%%%%%%%%%%%%%%%%%%%%%%%%%%%%%%%%%%%%%%%%%
%\subsection{Feature Suggestions}
%
%The following is a list of features which may be useful for future
%versions of this package:
%%
%\begin{itemize}
%\item
%\ldots
%\end{itemize}

%%%%%%%%%%%%%%%%%%%%%%%%%%%%%%%%%%%%%%%%%%%%%%%%%%%%%%%%%%%%%%%%%%%%%%%%%%%%%%%%
\subsection{Revision History}

%%%%%%%%%%%%%%%%%%%%%%%%%%%%%%%%%%%%%%%%
\paragraph{v2.0:} 2018/12/30

\begin{itemize}
\item
immediate forward processing
\item
added |\childdocby| mechanism
\item
manual restructured
\end{itemize}

%%%%%%%%%%%%%%%%%%%%%%%%%%%%%%%%%%%%%%%%
\paragraph{v1.6:} 2018/01/17

\begin{itemize}
\item
application for development of include files
\item
corrections to manual
\end{itemize}

%%%%%%%%%%%%%%%%%%%%%%%%%%%%%%%%%%%%%%%%
\paragraph{v1.5:} 2017/05/21

\begin{itemize}
\item
more complete structuring introduced
\item
|\childdocof| introduced
\item
|\childdoc| renamed to |\childdocmain|
\item
|\childredirect| renamed to |\childdocforward| and |\childdocforwardprefix|
and functionality expanded
\end{itemize}

%%%%%%%%%%%%%%%%%%%%%%%%%%%%%%%%%%%%%%%%
\paragraph{v1.0:} 2017/04/27

\begin{itemize}
\item
manual and install package
\item
first version published on CTAN
\end{itemize}

%%%%%%%%%%%%%%%%%%%%%%%%%%%%%%%%%%%%%%%%
\paragraph{v0.6:} 2017/04/26

\begin{itemize}
\item
redirection mechanism added
\end{itemize}

%%%%%%%%%%%%%%%%%%%%%%%%%%%%%%%%%%%%%%%%
\paragraph{v0.5:} 2017/04/26

\begin{itemize}
\item
functionality in definition file
\end{itemize}


%%%%%%%%%%%%%%%%%%%%%%%%%%%%%%%%%%%%%%%%%%%%%%%%%%%%%%%%%%%%%%%%%%%%%%%%%%%%%%%%
%%%%%%%%%%%%%%%%%%%%%%%%%%%%%%%%%%%%%%%%%%%%%%%%%%%%%%%%%%%%%%%%%%%%%%%%%%%%%%%%
%%%%%%%%%%%%%%%%%%%%%%%%%%%%%%%%%%%%%%%%%%%%%%%%%%%%%%%%%%%%%%%%%%%%%%%%%%%%%%%%
\appendix

\settowidth\MacroIndent{\rmfamily\scriptsize 000\ }

 \DocInput{childdoc.dtx}

\end{document}
%</driver>
% \fi
%
% %%%%%%%%%%%%%%%%%%%%%%%%%%%%%%%%%%%%%%%%%%%%%%%%%%%%%%%%%%%%%%%%%%%%%%%%%%%%%%
% %%%%%%%%%%%%%%%%%%%%%%%%%%%%%%%%%%%%%%%%%%%%%%%%%%%%%%%%%%%%%%%%%%%%%%%%%%%%%%
% \section{Sample}
%\iffalse
%<*samplemain>
%\fi
%
% The following presents a sample document
% with two chapters, two parts, a title page,
% a compile flag as well as three forwarding files to set the flag.
% It consists of eight |.tex| files:
% \begin{center}
% \begin{tabular}{ll}
% |cdocsamp.tex|&main file\\
% |cdocsch1.tex|&include file for chapter 1\\
% |cdocsch2.tex|&include file for chapter 2\\
% |cdocspt3.tex|&include file for part 3\\
% |cdocspt4.tex|&include file for part 4\\
% |cdocsdrf.tex|&forwarding file for main file in draft mode\\
% |cdocsfi1.tex|&forwarding file for final version of chapter 1\\
% |cdocsfi2.tex|&forwarding file for final version of chapter 2\\
% \end{tabular}
% \end{center}
% Each of the eight files can be compiled directly by the \LaTeX{} compiler.
%
% %%%%%%%%%%%%%%%%%%%%%%%%%%%%%%%%%%%%%%
% \paragraph{Main File.}
%
% The main file is called |cdocsamp.tex|.
%
% Load the \textsf{childdoc} definitions and
% declare the filename for the main document:
%    \begin{macrocode}
\input{childdoc.def}
\childdocmain{}
%    \end{macrocode}

% Optional override for |\version| flag:
%    \begin{macrocode}
%%\ifchilddoc\else\providecommand{\version}{draft}\fi
%    \end{macrocode}

% Define the default values for the |\version| flag
% (|final| for the main file and |draft| for childs):
%    \begin{macrocode}
\ifchilddoc
\providecommand{\version}{draft}
\else
\providecommand{\version}{final}
\fi
%    \end{macrocode}

% Load the standard document class:
%    \begin{macrocode}
\documentclass[12pt]{article}
%    \end{macrocode}

% Start the document body:
%    \begin{macrocode}
\begin{document}
%    \end{macrocode}

% Declare a title page.
% Print title, part of document being processed and version flag:
%    \begin{macrocode}
\addtocounter{page}{-1}
\begin{center}
{\LARGE\bfseries{}childdoc example\par}
\vspace{1cm}
\ifchilddoc
\ifchilddocmanual part\else chapter\fi:
`\childdocname' of `\childdocjob'\par
\else
main document: `\childdocjob'\par
\fi
version: \version\par
\end{center}
\newpage
%    \end{macrocode}

% Manually include selected file,
% otherwise process as usual:
%    \begin{macrocode}
\ifchilddocmanual
\section*{part `\childdocname'}
\input{\childdocname}
\else
%    \end{macrocode}

% Include the two chapters:
%    \begin{macrocode}
\include{cdocsch1}
\include{cdocsch2}
%    \end{macrocode}

% Include the two parts unless only chapters should be displayed:
%    \begin{macrocode}
\ifchilddoc\else
\section{part three}
\input{cdocspt3}
\section{part four}
\input{cdocspt4}
\fi
%    \end{macrocode}

% Process as usual until here:
%    \begin{macrocode}
\fi
%    \end{macrocode}

% End of document body:
%    \begin{macrocode}
\end{document}
%    \end{macrocode}
%\iffalse
%</samplemain>
%\fi
%
% %%%%%%%%%%%%%%%%%%%%%%%%%%%%%%%%%%%%%%
% \paragraph{Chapter Include Files.}
%
% The include files are called |cdocsch1.tex| and |cdocsch2.tex|.
%
%\iffalse
%<*samplechap1|samplechap2>
%\fi

% Optional override for |\version| flag:
%    \begin{macrocode}
%%\providecommand{\version}{final}
%    \end{macrocode}

% Include the main document:
%    \begin{macrocode}
\input{childdoc.def}
\childdocof{cdocsamp}
%    \end{macrocode}

%\iffalse
%</samplechap1|samplechap2>
%\fi
%
%\iffalse
%<*samplechap1>
%\fi
% Some text for chapter 1:
%    \begin{macrocode}
\section{one}
some text in chapter one
%    \end{macrocode}

%\iffalse
%</samplechap1>
%\fi
% Some text for chapter 2:
%\iffalse
%<*samplechap2>
%\fi
%    \begin{macrocode}
\section{two}
more text in chapter two
%    \end{macrocode}

%\iffalse
%</samplechap2>
%\fi
%
% %%%%%%%%%%%%%%%%%%%%%%%%%%%%%%%%%%%%%%
% \paragraph{Part Include Files.}
%
% The include files are called |cdocspt3.tex| and |cdocspt4.tex|.
%
%\iffalse
%<*samplepart3|samplepart4>
%\fi

% Optional override for |\version| flag:
%    \begin{macrocode}
%%\providecommand{\version}{final}
%    \end{macrocode}

% Include the main document:
%    \begin{macrocode}
\input{childdoc.def}
\childdocby{cdocsamp}
%    \end{macrocode}

%\iffalse
%</samplepart3|samplepart4>
%\fi
%
%\iffalse
%<*samplepart3>
%\fi
% Some text for part 3:
%    \begin{macrocode}
some text in part three
%    \end{macrocode}

%\iffalse
%</samplepart3>
%\fi
% Some text for part 4:
%\iffalse
%<*samplepart4>
%\fi
%    \begin{macrocode}
more text in part four
%    \end{macrocode}

%\iffalse
%</samplepart4>
%\fi
%
% %%%%%%%%%%%%%%%%%%%%%%%%%%%%%%%%%%%%%%
% \paragraph{Forwarding for a Complete Draft.}
%
% The following forwarding file |cdocsdrf.tex|
% compiles the main document in draft mode:
%\iffalse
%<*sampledraft>
%\fi
%    \begin{macrocode}
\def\version{draft}
\input{childdoc.def}
\childdocforward{cdocsamp}
%    \end{macrocode}

%\iffalse
%</sampledraft>
%\fi
%
% %%%%%%%%%%%%%%%%%%%%%%%%%%%%%%%%%%%%%%
% \paragraph{Forwarding for Final Version of the Chapters.}
%
% The following forwarding files |cdocsfn1.tex| and |cdocsfn2.tex|
% (with identical content)
% compile the final versions of the child documents
% |cdocsch1.tex| and |cdocsch2.tex|, respectively:
%\iffalse
%<*samplefinal>
%\fi
%    \begin{macrocode}
\def\version{final}
\input{childdoc.def}
\childdocforwardprefix[cdocsamp]{cdocsfn}{cdocsch}
%    \end{macrocode}

%\iffalse
%</samplefinal>
%\fi
%
% %%%%%%%%%%%%%%%%%%%%%%%%%%%%%%%%%%%%%%
% \paragraph{Command Line Processing.}
%
% The following three command lines generate the output files
% |cdocscld|, |cdocscl1| and |cdocscl2|
% which should be identical to
% |cdocsdrf|, |cdocsch1| and |cdocsfn2|, respectively:
% \begin{center}
% \begin{tabular}{l}
% |latex -jobname cdocscld \|\\
% |  "\def\version{draft}\input{childdoc.def}\childdocforward{cdocsamp}"|\\
% |latex -jobname cdocscl1 \|\\
% |  "\input{childdoc.def}\childdocforward[cdocsamp]{cdocsch1}"|\\
% |latex -jobname cdocscl2 \|\\
% |  "\def\version{final}\input{childdoc.def}\childdocforward{cdocsch2}"|
% \end{tabular}
% \end{center}
% Note that the trailing backslash on each first line
% merely continues the input to the second line
% (for convenient cut ant paste).
% Furthermore, the command |latex| can be replaced by any
% of its alternative versions such as |pdflatex|.
%
% %%%%%%%%%%%%%%%%%%%%%%%%%%%%%%%%%%%%%%%%%%%%%%%%%%%%%%%%%%%%%%%%%%%%%%%%%%%%%%
% %%%%%%%%%%%%%%%%%%%%%%%%%%%%%%%%%%%%%%%%%%%%%%%%%%%%%%%%%%%%%%%%%%%%%%%%%%%%%%
% \section{Implementation}
%\iffalse
%<*package>
%\fi
%
% This section describes the definitions file |childdoc.def|.

% The definitions cannot be loaded using |\usepackage| or |\RequirePackage|
% which has a mechanism to prevent loading a style file more than once.
% When loading the definitions by means of |\input|
% multiple instances have to be prevented manually:
%\iffalse
%This code needs to be before the `\ProvidesFile' directive
%which is defined at the beginning of this file.
%Therefore it is also placed there and commented out here.
%</package>
%<*discard>
%\fi
%    \begin{macrocode}
\ifdefined\childdocmain\endinput\fi
%    \end{macrocode}
%\iffalse
%</discard>
%<*package>
%\fi
%
% \macro{\ifchilddoc}
% \macro{\ifchilddocmanual}
% The conditional |\ifchilddoc| tells whether a
% child (true) or main (false) document is being compiled.
% The conditional |\ifchilddocmanual| tells whether
% the |\includeonly| mechanism is used (false) or
% the selection of child files must be performed manually (true).
% The definitions initialise to false:
%    \begin{macrocode}
\newif\ifchilddoc
\newif\ifchilddocmanual
%    \end{macrocode}

% \macro{\childdocname}
% \macro{\childdocjob}
% The macro |\childdocname| stores the name of the main document
% to be compiled. The macro |\childdocjob| stores the name of
% the document on which the \LaTeX{} compiler was originally invoked.
% The content of |\jobname| cannot be compared
% to filenames specified in the source due to different catcodes.
% The following code rescans |\jobname|, stores the result
% in |\childdocname| and saves a copy in |\childdocjob|:
%    \begin{macrocode}
\edef\childdocname{\scantokens\expandafter{\jobname\noexpand}}
\let\childdocjob\childdocname
%    \end{macrocode}

% \macro{\childdocdisable}
% The macro |\childdocdisable| prevents the main file
% from being processed more than once.
% At this stage, the main document command |\childdocmain|
% is assumed to be called once again where it should do nothing.
% Any subsequent call to it should prevent
% a secondary processing of the main document
% It overwrites the forwarding commands
% |\childdocof| and |\childdocforward|
% with empty macros to prevent further inclusions of the main document:
%    \begin{macrocode}
\newcommand{\childdocdisable}
{
  \renewcommand{\childdocmain}[1]{\renewcommand{\childdocmain}[1]{\endinput}}
  \renewcommand{\childdocof}[1]{}
  \renewcommand{\childdocby}[2][]{}
  \renewcommand{\childdocforward}[2][]{}
  \renewcommand{\childdocdisable}{}
}
%    \end{macrocode}

% \macro{\childdocmain}
% The macro |\childdocmain| is to be called at the top of the main file
% with nothing or the main filename (without extension) as argument.
% First, it breaks loops.
% If the argument is not empty and does not match |\childdocname|
% (which is set by the first inclusion of |childdoc.def|),
% |\ifchilddoc| is set to true, |\includeonly| is applied to the child file
% and |\jobname| is set to the main file
% (for proper handling of |.aux| files):
%    \begin{macrocode}
\newcommand{\childdocmain}[1]
{
  \childdocdisable\childdocmain{}
  \if?#1?\else
    \begingroup
      \def\childdoctmp{#1}
      \ifx\childdoctmp\childdocname
        \def\childdoctmp{}
      \else
        \def\childdoctmp
        {
          \childdoctrue
          \includeonly{\childdocname}
          \def\childdocjob{#1}
          \def\jobname{#1}
        }
      \fi
      \expandafter
    \endgroup
    \childdoctmp
  \fi
}
%    \end{macrocode}

% \macro{\childdocof}
% The command |\childdocof| redirects
% compilation to the main file |#1|.
%    \begin{macrocode}
\newcommand{\childdocof}[1]
{
  \childdocdisable
  \childdoctrue
  \includeonly{\childdocname}
  \def\jobname{#1}
  \def\childdocjob{#1}
  \input{#1}
}
%    \end{macrocode}

% \macro{\childdocby}
% The command |\childdocby| ....
%    \begin{macrocode}
\newcommand{\childdocby}[2][]
{
  \childdocdisable
  \childdoctrue
  \childdocmanualtrue
  \if?#1?\else
    \def\jobname{#2}
  \fi
  \def\childdocjob{#2}
  \input{#2}
  \endinput
}
%    \end{macrocode}

% \macro{\childdocforward}
% The command |\childdocforward| redirects
% compilation to the main file or
% (if the optional argument is given) a child file.
% Parameters are set as if the main file
% or a child file starting with |\childdocof| was compiled.
% Then compilation is handed over to the main file:
%    \begin{macrocode}
\newcommand{\childdocforward}[2][]
{
  \begingroup
    \if?#1?
      \def\childdoctmp
      {
        \def\childdocname{#2}
        \def\childdocjob{#2}
        \def\jobname{#2}
        \input{#2}
        \endinput
      }
    \else
      \def\childdoctmp
      {
        \childdocdisable
        \def\childdocname{#2}
        \childdoctrue
        \includeonly{#2}
        \def\childdocjob{#1}
        \def\jobname{#1}
        \input{#1}
        \endinput
      }
    \fi
    \expandafter
  \endgroup
  \childdoctmp
}
%    \end{macrocode}

% \macro{\childdocforwardprefix}
% The command |\childdocforwardprefix| redirects
% compilation to the main or a child file by means of a pattern.
% The prefix |#1| in the current filename is replaced by |#2|
% and the suffix of the current filename is kept
% (it is assumed that the filename does not contain the substring `|~~~|'
% which is used as a delimiter).
% Compilation is handed over to the new file by |\childdocforward|:
%    \begin{macrocode}
\newcommand{\childdocforwardprefix}[3][]
{
  \begingroup
    \def\childdocextract #2##1~~~{\def\childdoctmp{\childdocforward[#1]{#3##1}}}
    \expandafter\childdocextract\childdocname~~~
    \expandafter
  \endgroup
  \childdoctmp
}
%    \end{macrocode}

% \macro{\childdoc}
% The deprecated macro |\childdoc| is a legacy version of |\childdocmain|:
%    \begin{macrocode}
\newcommand{\childdoc}{\childdocmain}
%    \end{macrocode}

% \macro{\childdocredirect}
% The deprecated macro |\childdocredirect| is a legacy version
% of |\childdocforward| and |\childdocforwardprefix|:
%    \begin{macrocode}
\newcommand{\childdocredirect}[2][]
{
  \begingroup
    \if?#1?
      \def\childdoctmp{\childdocforward{#2}}
    \else
      \def\childdoctmp{\childdocforwardprefix{#1}{#2}}
    \fi
    \expandafter
  \endgroup
  \childdoctmp
}
%    \end{macrocode}

%\iffalse
%</package>
%\fi
%
\endinput
|
and perform the replacements as outlined below.
Instead of |\childdocmain{|\textit{main}|}| add the following code
to the top of the main file:
%
\begin{center}
\begin{tabular}{l}
|\||ifdefined\childdocname\endinput\||fi\newif\ifchilddoc|\\
|\edef\childdocname{\scantokens\expandafter{\jobname\noexpand}}|\\
|\def\childdocmain{|\textit{main}|}\||ifx\childdocmain\childdocname\||else|\\
|\childdoctrue\includeonly{\childdocname}\let\jobname\childdocmain\||fi|\\
\end{tabular}
\end{center}
%
Instead of |\childdocof{|\textit{main}|}| just include the main file
at the top of each child file:
%
\begin{center}
|\input{|\textit{main}|}|
\end{center}
%
A simple redirection |\childdocforward{|\textit{dest}|}| is achieved by:
%
\begin{center}
|\def\jobname{|\textit{dest}|}\input{\jobname}|
\end{center}
%
The redirection with prefix
|\childdocforwardprefix[|\textit{prefix}|]{|\textit{dest}|}|
is accomplished by:
%
\begin{center}
\begin{tabular}{l}
|{\edef\jobname{\scantokens\expandafter{\jobname\noexpand}}|\\
|\def\redirectjob |\textit{prefix}|#1~~~{\gdef\jobname{|\textit{dest}|#1}}|\\
|\expandafter\redirectjob\jobname~~~}\input{\jobname}|
\end{tabular}
\end{center}

In an alternative approach,
child documents can be compiled by a specific command line
without additional code or specific definitions:
%
\begin{center}
|... -jobname "|\textit{target}|" "|[\textit{flags}]%
|\includeonly{|\textit{dest}|}\input{|\textit{main}|}"|
\end{center}
%

%%%%%%%%%%%%%%%%%%%%%%%%%%%%%%%%%%%%%%%%%%%%%%%%%%%%%%%%%%%%%%%%%%%%%%%%%%%%%%%%
%%%%%%%%%%%%%%%%%%%%%%%%%%%%%%%%%%%%%%%%%%%%%%%%%%%%%%%%%%%%%%%%%%%%%%%%%%%%%%%%
\section{Information}

%%%%%%%%%%%%%%%%%%%%%%%%%%%%%%%%%%%%%%%%%%%%%%%%%%%%%%%%%%%%%%%%%%%%%%%%%%%%%%%%
\subsection{Copyright}

Copyright \copyright{} 2017--2018 Niklas Beisert

This work may be distributed and/or modified under the
conditions of the \LaTeX{} Project Public License, either version 1.3
of this license or (at your option) any later version.
The latest version of this license is in
  \url{http://www.latex-project.org/lppl.txt}
and version 1.3 or later is part of all distributions of \LaTeX{}
version 2005/12/01 or later.

This work has the LPPL maintenance status `maintained'.

The Current Maintainer of this work is Niklas Beisert.

This work consists of the files |README.txt|, |childdoc.ins| and |childdoc.dtx|
as well as the derived files |childdoc.def|, |cdocsamp.tex|
with |cdocsch1.tex|, |cdocsch2.tex|, |cdocspt3.tex|, |cdocspt4.tex|,
|cdocsdrf.tex|, |cdocsfn1.tex|, |cdocsfn2.tex|
as well as |childdoc.pdf|.

%%%%%%%%%%%%%%%%%%%%%%%%%%%%%%%%%%%%%%%%%%%%%%%%%%%%%%%%%%%%%%%%%%%%%%%%%%%%%%%%
\subsection{Files and Installation}

The package consists of the files:
%
\begin{center}
\begin{tabular}{ll}
    |README.txt|   & readme file \\
    |childdoc.ins| & installation file \\
    |childdoc.dtx| & source file \\
    |childdoc.def| & definition file \\
    |cdocsamp.tex| & sample main file \\
    |cdocsch1.tex| & sample include file \\
    |cdocsch2.tex| & sample include file \\
    |cdocspt3.tex| & sample part file \\
    |cdocspt4.tex| & sample part file \\
    |cdocsdrf.tex| & sample redirection file \\
    |cdocsfn1.tex| & sample redirection file \\
    |cdocsfn2.tex| & sample redirection file \\
    |childdoc.pdf| & manual
\end{tabular}
\end{center}
%
The distribution consists of the files
|README.txt|, |childdoc.ins| and |childdoc.dtx|.
%
\begin{itemize}
\item
Run (pdf)\LaTeX{} on |childdoc.dtx|
to compile the manual |childdoc.pdf| (this file).
\item
Run \LaTeX{} on |childdoc.ins| to create the definitions file |childdoc.def|
and the sample |cdocsamp.tex| with include files
|cdocsch1.tex|, |cdocsch2.tex|, |cdocspt3.tex|, |cdocspt4.tex|,
|cdocsdrf.tex|, |cdocsfn1.tex|, |cdocsfn2.tex|.
Then copy the file |childdoc.def| to an appropriate directory of your \LaTeX{}
distribution, e.g.\ \textit{texmf-root}|/tex/latex/childdoc|.
\end{itemize}

%%%%%%%%%%%%%%%%%%%%%%%%%%%%%%%%%%%%%%%%%%%%%%%%%%%%%%%%%%%%%%%%%%%%%%%%%%%%%%%%
\subsection{Related CTAN Packages}

There are several other packages which offer a similar functionality:
%
\begin{itemize}
\item
The packages
\href{http://ctan.org/pkg/docmute}{\textsf{docmute}},
\href{http://ctan.org/pkg/includex}{\textsf{includex}} and
\href{http://ctan.org/pkg/standalone}{\textsf{standalone}}
provide commands to include only the document body of
a child file thus allowing both files to be compiled individually.
\item
The packages \href{http://ctan.org/pkg/subdocs}{\textsf{subdocs}}
and \href{http://ctan.org/pkg/subfiles}{\textsf{subfiles}}
provide structures in which the main and child documents can be
encapsulated and allowing them to be compiled individually.
The inclusion mechanism is different from the conventional |\include|.
\item
The package \href{http://ctan.org/pkg/combine}{\textsf{combine}}
is an elaborate solution to combine several documents into one.
\end{itemize}
%
See also the CTAN topic \href{http://ctan.org/topic/subdocs}{\textsf{subdocs}}
for further related packages.
The present package differs from the above solutions in that
a document structure constructed with the conventional |\include| mechanism
just needs two extra commands at the top of every file
such that all constituent files can be compiled individually.

%%%%%%%%%%%%%%%%%%%%%%%%%%%%%%%%%%%%%%%%%%%%%%%%%%%%%%%%%%%%%%%%%%%%%%%%%%%%%%%%
%\subsection{Feature Suggestions}
%
%The following is a list of features which may be useful for future
%versions of this package:
%%
%\begin{itemize}
%\item
%\ldots
%\end{itemize}

%%%%%%%%%%%%%%%%%%%%%%%%%%%%%%%%%%%%%%%%%%%%%%%%%%%%%%%%%%%%%%%%%%%%%%%%%%%%%%%%
\subsection{Revision History}

%%%%%%%%%%%%%%%%%%%%%%%%%%%%%%%%%%%%%%%%
\paragraph{v2.0:} 2018/12/30

\begin{itemize}
\item
immediate forward processing
\item
added |\childdocby| mechanism
\item
manual restructured
\end{itemize}

%%%%%%%%%%%%%%%%%%%%%%%%%%%%%%%%%%%%%%%%
\paragraph{v1.6:} 2018/01/17

\begin{itemize}
\item
application for development of include files
\item
corrections to manual
\end{itemize}

%%%%%%%%%%%%%%%%%%%%%%%%%%%%%%%%%%%%%%%%
\paragraph{v1.5:} 2017/05/21

\begin{itemize}
\item
more complete structuring introduced
\item
|\childdocof| introduced
\item
|\childdoc| renamed to |\childdocmain|
\item
|\childredirect| renamed to |\childdocforward| and |\childdocforwardprefix|
and functionality expanded
\end{itemize}

%%%%%%%%%%%%%%%%%%%%%%%%%%%%%%%%%%%%%%%%
\paragraph{v1.0:} 2017/04/27

\begin{itemize}
\item
manual and install package
\item
first version published on CTAN
\end{itemize}

%%%%%%%%%%%%%%%%%%%%%%%%%%%%%%%%%%%%%%%%
\paragraph{v0.6:} 2017/04/26

\begin{itemize}
\item
redirection mechanism added
\end{itemize}

%%%%%%%%%%%%%%%%%%%%%%%%%%%%%%%%%%%%%%%%
\paragraph{v0.5:} 2017/04/26

\begin{itemize}
\item
functionality in definition file
\end{itemize}


%%%%%%%%%%%%%%%%%%%%%%%%%%%%%%%%%%%%%%%%%%%%%%%%%%%%%%%%%%%%%%%%%%%%%%%%%%%%%%%%
%%%%%%%%%%%%%%%%%%%%%%%%%%%%%%%%%%%%%%%%%%%%%%%%%%%%%%%%%%%%%%%%%%%%%%%%%%%%%%%%
%%%%%%%%%%%%%%%%%%%%%%%%%%%%%%%%%%%%%%%%%%%%%%%%%%%%%%%%%%%%%%%%%%%%%%%%%%%%%%%%
\appendix

\settowidth\MacroIndent{\rmfamily\scriptsize 000\ }

 \DocInput{childdoc.dtx}

\end{document}
%</driver>
% \fi
%
% %%%%%%%%%%%%%%%%%%%%%%%%%%%%%%%%%%%%%%%%%%%%%%%%%%%%%%%%%%%%%%%%%%%%%%%%%%%%%%
% %%%%%%%%%%%%%%%%%%%%%%%%%%%%%%%%%%%%%%%%%%%%%%%%%%%%%%%%%%%%%%%%%%%%%%%%%%%%%%
% \section{Sample}
%\iffalse
%<*samplemain>
%\fi
%
% The following presents a sample document
% with two chapters, two parts, a title page,
% a compile flag as well as three forwarding files to set the flag.
% It consists of eight |.tex| files:
% \begin{center}
% \begin{tabular}{ll}
% |cdocsamp.tex|&main file\\
% |cdocsch1.tex|&include file for chapter 1\\
% |cdocsch2.tex|&include file for chapter 2\\
% |cdocspt3.tex|&include file for part 3\\
% |cdocspt4.tex|&include file for part 4\\
% |cdocsdrf.tex|&forwarding file for main file in draft mode\\
% |cdocsfi1.tex|&forwarding file for final version of chapter 1\\
% |cdocsfi2.tex|&forwarding file for final version of chapter 2\\
% \end{tabular}
% \end{center}
% Each of the eight files can be compiled directly by the \LaTeX{} compiler.
%
% %%%%%%%%%%%%%%%%%%%%%%%%%%%%%%%%%%%%%%
% \paragraph{Main File.}
%
% The main file is called |cdocsamp.tex|.
%
% Load the \textsf{childdoc} definitions and
% declare the filename for the main document:
%    \begin{macrocode}
% \iffalse
%
% childdoc.dtx Copyright (C) 2017-2018 Niklas Beisert
%
% This work may be distributed and/or modified under the
% conditions of the LaTeX Project Public License, either version 1.3
% of this license or (at your option) any later version.
% The latest version of this license is in
%   http://www.latex-project.org/lppl.txt
% and version 1.3 or later is part of all distributions of LaTeX
% version 2005/12/01 or later.
%
% This work has the LPPL maintenance status `maintained'.
%
% The Current Maintainer of this work is Niklas Beisert.
%
% This work consists of the files childdoc.dtx and childdoc.ins
% and the derived files childdoc.def and cdocsamp.tex with
% cdocsch1.tex, cdocsch2.tex, cdocsdrf.tex, cdocsfn1.tex, cdocsfn2.tex.
%
%<package>\ifdefined\childdocmain\endinput\fi
%<package>\ProvidesFile{childdoc.def}[2018/12/30 v2.0 child document driver]
%<samplemain>\ProvidesFile{cdocsamp.tex}[2018/12/30 v2.0 sample for childdoc]
%<*driver>
%\ProvidesFile{childdoc.drv}[2018/12/30 v2.0 childdoc reference manual file]
\PassOptionsToClass{10pt,a4paper}{article}
\documentclass{ltxdoc}

\usepackage[margin=35mm]{geometry}
\usepackage{hyperref}
\usepackage{hyperxmp}
\usepackage[usenames]{color}

\hypersetup{colorlinks=true}
\hypersetup{pdfstartview=FitH}
\hypersetup{pdfpagemode=UseNone}
\hypersetup{pdfsource={}}
\hypersetup{pdflang={en-UK}}
\hypersetup{pdfcopyright={Copyright 2017-2018 Niklas Beisert.
  This work may be distributed and/or modified under the
  conditions of the LaTeX Project Public License, either version 1.3
  of this license or (at your option) any later version.}}
\hypersetup{pdflicenseurl={http://www.latex-project.org/lppl.txt}}
\hypersetup{pdfcontactaddress={ETH Zurich, ITP, HIT K,
  Wolfgang-Pauli-Strasse 27}}
\hypersetup{pdfcontactpostcode={8093}}
\hypersetup{pdfcontactcity={Zurich}}
\hypersetup{pdfcontactcountry={Switzerland}}
\hypersetup{pdfcontactemail={nbeisert@itp.phys.ethz.ch}}
\hypersetup{pdfcontacturl={http://people.phys.ethz.ch/\xmptilde nbeisert/}}

\newcommand{\secref}[1]{\hyperref[#1]{section \ref*{#1}}}

\parskip1ex
\parindent0pt
\let\olditemize\itemize
\def\itemize{\olditemize\parskip0pt}

\begin{document}

\title{The \textsf{childdoc} Package}
\hypersetup{pdftitle={The childdoc Package}}
\author{Niklas Beisert\\[2ex]
  Institut f\"ur Theoretische Physik\\
  Eidgen\"ossische Technische Hochschule Z\"urich\\
  Wolfgang-Pauli-Strasse 27, 8093 Z\"urich, Switzerland\\[1ex]
  \href{mailto:nbeisert@itp.phys.ethz.ch}
  {\texttt{nbeisert@itp.phys.ethz.ch}}}
\hypersetup{pdfauthor={Niklas Beisert}}
\hypersetup{pdfsubject={Manual for the LaTeX2e Package childdoc}}
\date{30 December 2018, \textsf{v2.0}}
\maketitle

\begin{abstract}\noindent
\textsf{childdoc} is a \LaTeXe{} package
that enables the direct compilation
of document sections included by |\include|
to individual files.
\end{abstract}

\begingroup
\parskip0ex
\tableofcontents
\endgroup

%%%%%%%%%%%%%%%%%%%%%%%%%%%%%%%%%%%%%%%%%%%%%%%%%%%%%%%%%%%%%%%%%%%%%%%%%%%%%%%%
%%%%%%%%%%%%%%%%%%%%%%%%%%%%%%%%%%%%%%%%%%%%%%%%%%%%%%%%%%%%%%%%%%%%%%%%%%%%%%%%
\section{Introduction}

\LaTeX{} provides a mechanism to structure a large document (such as a book)
into a main file and several child files (containing the chapters)
using the |\include| command.
This mechanism is beneficial for documents
which span hundreds of pages in order to
make the source file(s) more manageable.
Moreover, compilation can be restricted to
selected child files by means of the |\includeonly| command.
The latter feature can be used to reduce the compilation time while editing
(this was significantly more useful in the earlier days of \LaTeX{})
or to generate a smaller document which is easier to navigate.
Another application of |\includeonly| is to generate
documents consisting of selected parts of the complete document.

However, there are a few drawbacks of the plain |\include| mechanism:
\begin{itemize}
\item
The child files cannot be compiled on their own,
they can only be compiled via the main file.
A naive editing environment
(such as a text editor with an option
to have the current file processed by \LaTeX)
may require one to switch to the main file before compiling;
attempting to compile the child file produces errors.
\item
The main file must be modified (each time)
to adjust the |\includeonly| command
to the present needs. This easily leaves the main file in a messy state.
\item
The generated document will always carry the filename
of the main document. This is inconvenient if
several child files are to be compiled and
to be kept for distribution.
\end{itemize}

The present package provides a simple interface
to make child files individually compilable by \LaTeX{}.
Compiling a child file then has the same effect as compiling
the main file with an |\includeonly| command
to select the appropriate child.
Moreover the generated document will carry the name of the child
rather than the main file.
This resolves all three above issues.

This feature is meant to make the editing of books,
thesis documents and lecture notes somewhat more convenient.
However, the package can also be used efficiently for
composing a series of documents (such as exercise sheets)
which are typically distributed individually.
It then assists the author in generating the individual documents
(potentially in different versions)
as well as a document containing the collected series.
Another application is in developing style files
or other kinds of included material
where compilation of the style file could redirect
to a sample or test file.

%%%%%%%%%%%%%%%%%%%%%%%%%%%%%%%%%%%%%%%%%%%%%%%%%%%%%%%%%%%%%%%%%%%%%%%%%%%%%%%%
%%%%%%%%%%%%%%%%%%%%%%%%%%%%%%%%%%%%%%%%%%%%%%%%%%%%%%%%%%%%%%%%%%%%%%%%%%%%%%%%
\section{Usage}

First of all, the package \textsf{childdoc} is \emph{not} a standard
\LaTeXe{} |.sty| style file! Therefore it needs to be invoked in
a non-standard way.

%%%%%%%%%%%%%%%%%%%%%%%%%%%%%%%%%%%%%%%%%%%%%%%%%%%%%%%%%%%%%%%%%%%%%%%%%%%%%%%%
\subsection{Included Files}
\label{sec:include}

%%%%%%%%%%%%%%%%%%%%%%%%%%%%%%%%%%%%%%%%
\DescribeMacro{\childdocmain}
To use the package, add the commands
\begin{center}
\begin{tabular}{l}
|\input{childdoc.def}|\\
|\childdocmain{}|\\
\end{tabular}
\end{center}
at the very top of the main \LaTeX{} file,
in particular \emph{before} the |\documentclass| statement!
The argument of |\childdocmain| should be left empty
(but it must be present).

%%%%%%%%%%%%%%%%%%%%%%%%%%%%%%%%%%%%%%%%
\DescribeMacro{\childdocof}
Furthermore, add the commands
\begin{center}
\begin{tabular}{l}
|\input{childdoc.def}|\\
|\childdocof{|\textit{main}|}|\\
\end{tabular}
\end{center}
at the top of every child file \textit{child}
which is included by |\include{|\textit{child}|}|
from within the main file
(or at least for those files to be compiled individually).
The argument \textit{main} must be the filename of the main file.

There are a couple of
considerations in setting up the main and child documents:

%%%%%%%%%%%%%%%%%%%%%%%%%%%%%%%%%%%%%%%%
\paragraph{Restrictions.}

Please note the following restrictions:
\begin{itemize}
\item
|\childdocmain| must be called with one argument \textit{main}
to ensure compatibility with earlier version of the package.
It must either be empty (|\childdocmain{}|)
or precisely match the filename of the main file in which it is specified.
See \secref{sec:detection} for further information.
\item
The filename \textit{main} must be specified without the |.tex| extension.
\item
The filename \textit{main} is case sensitive
(even in case-insensitive file systems)
due to internal string comparison.
\item
The argument \textit{main} should be fully expanded, it cannot be a macro.
\item
Subdirectories and special characters should be avoided in filenames.
\item
The command |\childdocmain{|\textit{main}|}| must be followed by a whitespace.
It should not be followed immediately by another command
or by a comment mark `|%|'.
This is because the \TeX{} parser reads the token immediately following
the argument of |\childdocmain| and puts it
at the beginning of every child section;
however, a white\-space is ignored.
\end{itemize}

%%%%%%%%%%%%%%%%%%%%%%%%%%%%%%%%%%%%%%%%
\paragraph{Content of Main File.}

It is advisable to place all content in the child files included by |\include|.
Any output contained in the main file will appear in all child documents
unless suppressed manually;
it cannot be suppressed automatically by the |\includeonly| directive
and thus should normally be avoided.
A method to include some content in the main file
by means of conditional processing is described in \secref{sec:conditional}.

%%%%%%%%%%%%%%%%%%%%%%%%%%%%%%%%%%%%%%%%
\paragraph{Page Numbering.}

When only a part of the document is compiled,
the appropriate numbering of pages
(as well as other status parameters)
is determined from the |.aux| files.
The latter contain information from previous passes.
However this information needs to propagate through
all intermediate child documents.
Therefore the page numbering in child documents may well
be inconsistent until the complete document is compiled at least once.

A useful (if unconventional) way to always ensure a consistent
page numbering is to restart the numbering in each child document
and denote the pages by `\textit{child}|.|\textit{page}'
where \textit{child} represents the chapter/section number of the child file.
This can be achieved by the command
|\numberwithin{page}{|\textit{child}|}|
of the \textsf{amsmath} package
where \textit{child} can be |chapter| or |section|
depending on the chosen structuring.
Alternatively, one can modify the macro |\thepage| appropriately
and reset the counter |page| at the start of each child file.

%%%%%%%%%%%%%%%%%%%%%%%%%%%%%%%%%%%%%%%%%%%%%%%%%%%%%%%%%%%%%%%%%%%%%%%%%%%%%%%%
\subsection{Conditional Processing}
\label{sec:conditional}

The package provides a mechanism to compile different versions
of a document. To customise the versions further some conditional processing
can come in handy to distinguish which version is being compiled.
The package provides two macros to describe the compilation context:

%%%%%%%%%%%%%%%%%%%%%%%%%%%%%%%%%%%%%%%%
\DescribeMacro{\ifchilddoc}
The conditional |\ifchilddoc| distinguishes between the compilation of
child documents and the main document:
%
\begin{center}
|\ifchilddoc |\textit{child-code}| |[|\||else |\textit{main-code}]| \||fi|
\end{center}

%%%%%%%%%%%%%%%%%%%%%%%%%%%%%%%%%%%%%%%%
\DescribeMacro{\childdocname}
\DescribeMacro{\childdocjob}
The macro |\childdocname| contains the filename (without extension)
of the main or child file being processed.
Note that |\childdocjob| will always contain the name of the main file.

%%%%%%%%%%%%%%%%%%%%%%%%%%%%%%%%%%%%%%%%
\paragraph{Title Page.}

Conditional processing can be used to include a title or banner page
in the main document when proper precautions are taken.
Importantly, the code in the main file should ensure that the page counter
(as well as other status parameters which are stored in the |.aux| files)
takes the same value after the conditional processing.
Otherwise the page numbers may take divergent values
depending on which part is compiled.

For example, a title page could be declared by:
%
\begin{center}
\begin{tabular}{l}
|\ifchilddoc\||else|\\
|\addtocounter{page}{-1}|\\
\textit{code for title page}\\
|\newpage|\\
|\||fi|
\end{tabular}
\end{center}
%
A banner page for the child documents can be generated by:
%
\begin{center}
\begin{tabular}{l}
|\ifchilddoc|\\
|\addtocounter{page}{-1}|\\
\textit{code for banner page}\\
|\newpage|\\
|\||fi|
\end{tabular}
\end{center}
%
Here one could write a message such as:
\begin{center}
|This is the part \childdocname{} of \childdocjob{}.|
\end{center}

%%%%%%%%%%%%%%%%%%%%%%%%%%%%%%%%%%%%%%%%%%%%%%%%%%%%%%%%%%%%%%%%%%%%%%%%%%%%%%%%
\subsection{Flags}
\label{sec:flags}

The package makes it easy to generate different versions
of the main or child documents.
To this end compilation flags can be defined
and assigned different default values.
They will be particularly useful in conjunction
with the forwarding mechanism described in \secref{sec:forward}.

For example, it may be useful to have a flag |\version|
which can be set to |draft| or |final|.
The document source will contain some conditional code
depending on the value of |\version|.
Suppose further, the flag should default to |final| for the main file
and to |draft| for child files
which is a natural assignment for editing the document.
This is achieved by placing the following code
in the preamble of the main document
(below the |\childdocmain| directive):
%
\begin{center}
\begin{tabular}{l}
|\ifchilddoc|\\
|\providecommand{\version}{draft}|\\
|\||else|\\
|\providecommand{\version}{final}|\\
|\||fi|
\end{tabular}
\end{center}
%
The definition by |\providecommand| makes sure
that previous definitions are not overwritten.
Further statements |\providecommand{\version}{...}|
can thus be added before the above code to override it.

For the main file, one might add a line
(between |\childdocmain| and the above block)
%
\begin{center}
|%\ifchilddoc\||else\providecommand{\version}{draft}\||fi|
\end{center}
%
which can be uncommented to produce a draft version.
Likewise one can add a line to the very top of a child file
(above the |\childdocof{|\textit{main}|}| directive)
%
\begin{center}
|%\providecommand{\version}{final}|
\end{center}
%
which can be uncommented to produce the final version of this child document.

%%%%%%%%%%%%%%%%%%%%%%%%%%%%%%%%%%%%%%%%%%%%%%%%%%%%%%%%%%%%%%%%%%%%%%%%%%%%%%%%
\subsection{Forwarding}
\label{sec:forward}

Different versions of the main or child documents
using compilation flags as described in \secref{sec:flags}
can be (permanently) stored in different files
for convenient compilation, viewing and distribution.
To this end, the package defines a command
to pass on compilation to a different file:

%%%%%%%%%%%%%%%%%%%%%%%%%%%%%%%%%%%%%%%%
\DescribeMacro{\childdocforward}
The command |\childdocforward| redirects processing to
another source file:
%
\begin{center}
\begin{tabular}{l}
|\input{childdoc.def}|\\
|\childdocforward[|\textit{main}|]{|\textit{dest}|}|\\
\end{tabular}
\end{center}
%
The argument \textit{dest} is the destination file
(without extension).
It should be the main file or one of the child files.
Note that further \textsf{childdoc} directives
such as |\childdocof| and |\childdocforward|
in the indicated file will be processed in this form.
The optional argument \textit{main}
passes on directly to the main file \textit{main}
while pretending to compile the child \textit{dest}.
This form behaves as if \textit{dest}
issues |\childdocof{|\textit{main}|}| right away,
and no further \textsf{childdoc} directives will be processed.

%%%%%%%%%%%%%%%%%%%%%%%%%%%%%%%%%%%%%%%%
\DescribeMacro{\...prefix}
In the alternative form |\childdocforwardprefix|,
%
\begin{center}
\begin{tabular}{l}
|\input{childdoc.def}|\\
|\childdocforwardprefix[|\textit{main}|]{|\textit{prefix}|}{|\textit{dest}|}|
\end{tabular}
\end{center}
%
the destination file is determined by a pattern
depending on the current file:
To make this work, the current file must be called
`{\textit{prefix}\hspace{0.2em}\textit{suffix}}'
with \textit{prefix} matching precisely the argument.
Processing is then passed on to the file
`{\textit{dest}\hspace{0.2em}\textit{suffix}}'.
Surely, the same effect is achieved by
directly specifying the
argument `{\textit{dest}\hspace{0.2em}\textit{suffix}}'
in the first form.
However, that requires to set up a different file
for each child. With the alternative form of the command
all these files can have exactly the same content
which simplifies setting them up and maintaining them.

For example, the following file |draft.tex|
with a compilation flag |\version| as described in \secref{sec:flags}
compiles the main document as a draft:
%
\begin{center}
\begin{tabular}{l}
|\def\version{draft}|\\
|\input{childdoc.def}|\\
|\childdocforward{|\textit{main}|}|
\end{tabular}
\end{center}
%
Likewise, the following files |final|\textit{nn}|.tex|
compile the final version of the child document
|child|\textit{nn}|.tex|:
%
\begin{center}
\begin{tabular}{l}
|\def\version{final}|\\
|\input{childdoc.def}|\\
|\childdocforwardprefix{final}{child}|
\end{tabular}
\end{center}
%

Note that when several versions of a main file and/or of each child file
are to be generated, it may be convenient to set up a |Makefile| or
shell script to automatise the process.

%%%%%%%%%%%%%%%%%%%%%%%%%%%%%%%%%%%%%%%%%%%%%%%%%%%%%%%%%%%%%%%%%%%%%%%%%%%%%%%%
\subsection{Command Line Processing}
\label{sec:commandline}

The effect of redirection files can also be achieved by invoking
the \LaTeX{} compiler with a more elaborate command line.
Most conveniently this should be done as part
of a shell script or a |Makefile|.

When using \textsf{childdoc} in the main file, the following
command lines effectively perform a redirection
(note that depending on the shell being used,
backslashes may have to be doubled: `|\|' $\to$ `|\\|'):
%
\begin{center}
|... -jobname "|\textit{target}|" |\\|"|[\textit{flags}]%
|\input{childdoc.def}\childdocforward[|\textit{main}|]{|\textit{dest}|}"|
\end{center}
%
Here \textit{target} is the name of the output file,
\textit{main} is the name of the main file
and \textit{dest} is the name of the main or child file to be processed
(all filenames without extensions).
The optional argument \textit{main} can be omitted
if \textit{main} matches \textit{dest}.
Optionally, compilation \textit{flags} can be defined via |\def| commands.
This command line makes the \TeX{} engine believe
it is compiling the file \textit{target}
whose content is specified as the latter parameter.
The provided code then forwards the processing to
\textit{main} or \textit{dest} as described in \secref{sec:forward}.

%%%%%%%%%%%%%%%%%%%%%%%%%%%%%%%%%%%%%%%%%%%%%%%%%%%%%%%%%%%%%%%%%%%%%%%%%%%%%%%%
\subsection{Include by Input}
\label{sec:input}

Including child documents by |\include| has some restrictions by design.
Most notably, the content of a child document always occupies
its own set of pages; pages cannot be shared between child documents.
Usually, this behaviour makes perfect sense
because each child document contain an essential part of the document.
However, in some situations it may be desirable to compose
a document from a collection of parts
without having mandatory page breaks between then.
For this case, the package
provides a mechanism to include parts
by |\input| which can also be processed individually.
However, by construction this mechanism
requires manual handling of the content to be output.

%%%%%%%%%%%%%%%%%%%%%%%%%%%%%%%%%%%%%%%%
\DescribeMacro{\ifchilddocmanual}
The main file should be prepared as usual, see \secref{sec:include}.
However, the document body must make a distinction
between processing of an individual part and of the main document, e.g.:
%
\begin{center}
\begin{tabular}{l}
|\ifchilddocmanual|\\
|\input{\childdocname}|\\
|\||else|\\
\textit{document body with }|\input{|\textit{part}|}|\\
|\||fi|
\end{tabular}
\end{center}
%
The conditional |\ifchilddocmanual| is true whenever
a part to be included by |\input| is being compiled,
and the name of the part is stored in |\childdocname|.

%%%%%%%%%%%%%%%%%%%%%%%%%%%%%%%%%%%%%%%%
\DescribeMacro{\childdocby}
Each part to be included by |\input| should start with:
%
\begin{center}
\begin{tabular}{l}
|\input{childdoc.def}|\\
|\childdocby{|\textit{main}|}|\\
\end{tabular}
\end{center}
%
The directive |\childdocby| is similar to |\childdocof|
described in \secref{sec:include},
but the subsequent selection of content must be done manually.
To that end, both |\ifchilddoc| and |\ifchilddocmanual|
will be true upon processing of a part,
and the name of the part is stored in |\childdocname|.
Note that |\jobname| will be set to the filename of the current part
so that each part receives an individual |.aux| file
that does not interfere with the |.aux| file(s) of the main document.
This behaviour can be altered by the alternative form
|\childdocby[*]{|\textit{main}|}| (with a non-empty optional argument)
which uses the |.aux| file of the main document
by setting |\jobname| to \textit{main}.

%%%%%%%%%%%%%%%%%%%%%%%%%%%%%%%%%%%%%%%%%%%%%%%%%%%%%%%%%%%%%%%%%%%%%%%%%%%%%%%%
\subsection{Driver Development}
\label{sec:driver}

The \textsf{childdoc} mechanism can also be use for the development
of definition files such as \LaTeX{} styles or classes.
This case differs from the above setup with multiple parts
included by |\include| in that no |\includeonly| should be invoked.
This can be achieved by starting the include file
(before |\ProvidesPackage|) with:
%
\begin{center}
\begin{tabular}{l}
|\input{childdoc.def}|\\
|\childdocforward{|\textit{main}|}|\\
\end{tabular}
\end{center}
%
or alternatively with:
%
\begin{center}
\begin{tabular}{l}
|\input{childdoc.def}|\\
|\childdocby{|\textit{main}|}|\\
\end{tabular}
\end{center}
%
Both forms have slightly different effects as described above.
The main file is prepared as usual, see \secref{sec:include}.

%%%%%%%%%%%%%%%%%%%%%%%%%%%%%%%%%%%%%%%%%%%%%%%%%%%%%%%%%%%%%%%%%%%%%%%%%%%%%%%%
\subsection{Legacy Detection}
\label{sec:detection}

The directive |\childdocmain| in the main file can detect
whether the complete document or merely a child is to be compiled
even without using the directive |\childdocof|.
This method is deprecated because it is less robust
and there is no compelling reason to use it;
it is merely provided for backward compatibility
and it may be removed in future versions.

If the detection mechanism is to be used,
it is mandatory to correctly specify
the filename of the main file as the argument of |\childdocmain|:
%
\begin{center}
\begin{tabular}{l}
|\input{childdoc.def}|\\
|\childdocmain{|\textit{main}|}|\\
\end{tabular}
\end{center}
%
If |\jobname| does not match the argument \textit{main} of |\childdocmain|,
it is assumed that |\jobname| points to the child file to be compiled.
When using |\childdocmain| with the main file specified as argument,
it suffices to start a child file
with just |\input{|\textit{main}|}|
without loading of the package and using |\childdocof|.
If instead all processing is done
with the appropriate \textsf{childdoc} directives,
the argument of \textit{main} of |\childdocmain| can be empty.

An alternative version of the command line processing described
in \secref{sec:commandline} using the detection mechanism reads:
%
\begin{center}
|... -jobname "|\textit{target}|" "|[\textit{flags}]%
[|\def\jobname{|\textit{dest}|}|]|\input{|\textit{main}|}"|
\end{center}

%%%%%%%%%%%%%%%%%%%%%%%%%%%%%%%%%%%%%%%%%%%%%%%%%%%%%%%%%%%%%%%%%%%%%%%%%%%%%%%%
\subsection{Manual Code}
\label{sec:manual}

In case one cannot be certain whether the definitions file |childdoc.def|
is installed on the target \TeX{} distribution
and one prefers not to ship it,
it is conceivable to paste a few relevant commands into the sources.

To that end, drop all statements |\input{childdoc.def}|
and perform the replacements as outlined below.
Instead of |\childdocmain{|\textit{main}|}| add the following code
to the top of the main file:
%
\begin{center}
\begin{tabular}{l}
|\||ifdefined\childdocname\endinput\||fi\newif\ifchilddoc|\\
|\edef\childdocname{\scantokens\expandafter{\jobname\noexpand}}|\\
|\def\childdocmain{|\textit{main}|}\||ifx\childdocmain\childdocname\||else|\\
|\childdoctrue\includeonly{\childdocname}\let\jobname\childdocmain\||fi|\\
\end{tabular}
\end{center}
%
Instead of |\childdocof{|\textit{main}|}| just include the main file
at the top of each child file:
%
\begin{center}
|\input{|\textit{main}|}|
\end{center}
%
A simple redirection |\childdocforward{|\textit{dest}|}| is achieved by:
%
\begin{center}
|\def\jobname{|\textit{dest}|}\input{\jobname}|
\end{center}
%
The redirection with prefix
|\childdocforwardprefix[|\textit{prefix}|]{|\textit{dest}|}|
is accomplished by:
%
\begin{center}
\begin{tabular}{l}
|{\edef\jobname{\scantokens\expandafter{\jobname\noexpand}}|\\
|\def\redirectjob |\textit{prefix}|#1~~~{\gdef\jobname{|\textit{dest}|#1}}|\\
|\expandafter\redirectjob\jobname~~~}\input{\jobname}|
\end{tabular}
\end{center}

In an alternative approach,
child documents can be compiled by a specific command line
without additional code or specific definitions:
%
\begin{center}
|... -jobname "|\textit{target}|" "|[\textit{flags}]%
|\includeonly{|\textit{dest}|}\input{|\textit{main}|}"|
\end{center}
%

%%%%%%%%%%%%%%%%%%%%%%%%%%%%%%%%%%%%%%%%%%%%%%%%%%%%%%%%%%%%%%%%%%%%%%%%%%%%%%%%
%%%%%%%%%%%%%%%%%%%%%%%%%%%%%%%%%%%%%%%%%%%%%%%%%%%%%%%%%%%%%%%%%%%%%%%%%%%%%%%%
\section{Information}

%%%%%%%%%%%%%%%%%%%%%%%%%%%%%%%%%%%%%%%%%%%%%%%%%%%%%%%%%%%%%%%%%%%%%%%%%%%%%%%%
\subsection{Copyright}

Copyright \copyright{} 2017--2018 Niklas Beisert

This work may be distributed and/or modified under the
conditions of the \LaTeX{} Project Public License, either version 1.3
of this license or (at your option) any later version.
The latest version of this license is in
  \url{http://www.latex-project.org/lppl.txt}
and version 1.3 or later is part of all distributions of \LaTeX{}
version 2005/12/01 or later.

This work has the LPPL maintenance status `maintained'.

The Current Maintainer of this work is Niklas Beisert.

This work consists of the files |README.txt|, |childdoc.ins| and |childdoc.dtx|
as well as the derived files |childdoc.def|, |cdocsamp.tex|
with |cdocsch1.tex|, |cdocsch2.tex|, |cdocspt3.tex|, |cdocspt4.tex|,
|cdocsdrf.tex|, |cdocsfn1.tex|, |cdocsfn2.tex|
as well as |childdoc.pdf|.

%%%%%%%%%%%%%%%%%%%%%%%%%%%%%%%%%%%%%%%%%%%%%%%%%%%%%%%%%%%%%%%%%%%%%%%%%%%%%%%%
\subsection{Files and Installation}

The package consists of the files:
%
\begin{center}
\begin{tabular}{ll}
    |README.txt|   & readme file \\
    |childdoc.ins| & installation file \\
    |childdoc.dtx| & source file \\
    |childdoc.def| & definition file \\
    |cdocsamp.tex| & sample main file \\
    |cdocsch1.tex| & sample include file \\
    |cdocsch2.tex| & sample include file \\
    |cdocspt3.tex| & sample part file \\
    |cdocspt4.tex| & sample part file \\
    |cdocsdrf.tex| & sample redirection file \\
    |cdocsfn1.tex| & sample redirection file \\
    |cdocsfn2.tex| & sample redirection file \\
    |childdoc.pdf| & manual
\end{tabular}
\end{center}
%
The distribution consists of the files
|README.txt|, |childdoc.ins| and |childdoc.dtx|.
%
\begin{itemize}
\item
Run (pdf)\LaTeX{} on |childdoc.dtx|
to compile the manual |childdoc.pdf| (this file).
\item
Run \LaTeX{} on |childdoc.ins| to create the definitions file |childdoc.def|
and the sample |cdocsamp.tex| with include files
|cdocsch1.tex|, |cdocsch2.tex|, |cdocspt3.tex|, |cdocspt4.tex|,
|cdocsdrf.tex|, |cdocsfn1.tex|, |cdocsfn2.tex|.
Then copy the file |childdoc.def| to an appropriate directory of your \LaTeX{}
distribution, e.g.\ \textit{texmf-root}|/tex/latex/childdoc|.
\end{itemize}

%%%%%%%%%%%%%%%%%%%%%%%%%%%%%%%%%%%%%%%%%%%%%%%%%%%%%%%%%%%%%%%%%%%%%%%%%%%%%%%%
\subsection{Related CTAN Packages}

There are several other packages which offer a similar functionality:
%
\begin{itemize}
\item
The packages
\href{http://ctan.org/pkg/docmute}{\textsf{docmute}},
\href{http://ctan.org/pkg/includex}{\textsf{includex}} and
\href{http://ctan.org/pkg/standalone}{\textsf{standalone}}
provide commands to include only the document body of
a child file thus allowing both files to be compiled individually.
\item
The packages \href{http://ctan.org/pkg/subdocs}{\textsf{subdocs}}
and \href{http://ctan.org/pkg/subfiles}{\textsf{subfiles}}
provide structures in which the main and child documents can be
encapsulated and allowing them to be compiled individually.
The inclusion mechanism is different from the conventional |\include|.
\item
The package \href{http://ctan.org/pkg/combine}{\textsf{combine}}
is an elaborate solution to combine several documents into one.
\end{itemize}
%
See also the CTAN topic \href{http://ctan.org/topic/subdocs}{\textsf{subdocs}}
for further related packages.
The present package differs from the above solutions in that
a document structure constructed with the conventional |\include| mechanism
just needs two extra commands at the top of every file
such that all constituent files can be compiled individually.

%%%%%%%%%%%%%%%%%%%%%%%%%%%%%%%%%%%%%%%%%%%%%%%%%%%%%%%%%%%%%%%%%%%%%%%%%%%%%%%%
%\subsection{Feature Suggestions}
%
%The following is a list of features which may be useful for future
%versions of this package:
%%
%\begin{itemize}
%\item
%\ldots
%\end{itemize}

%%%%%%%%%%%%%%%%%%%%%%%%%%%%%%%%%%%%%%%%%%%%%%%%%%%%%%%%%%%%%%%%%%%%%%%%%%%%%%%%
\subsection{Revision History}

%%%%%%%%%%%%%%%%%%%%%%%%%%%%%%%%%%%%%%%%
\paragraph{v2.0:} 2018/12/30

\begin{itemize}
\item
immediate forward processing
\item
added |\childdocby| mechanism
\item
manual restructured
\end{itemize}

%%%%%%%%%%%%%%%%%%%%%%%%%%%%%%%%%%%%%%%%
\paragraph{v1.6:} 2018/01/17

\begin{itemize}
\item
application for development of include files
\item
corrections to manual
\end{itemize}

%%%%%%%%%%%%%%%%%%%%%%%%%%%%%%%%%%%%%%%%
\paragraph{v1.5:} 2017/05/21

\begin{itemize}
\item
more complete structuring introduced
\item
|\childdocof| introduced
\item
|\childdoc| renamed to |\childdocmain|
\item
|\childredirect| renamed to |\childdocforward| and |\childdocforwardprefix|
and functionality expanded
\end{itemize}

%%%%%%%%%%%%%%%%%%%%%%%%%%%%%%%%%%%%%%%%
\paragraph{v1.0:} 2017/04/27

\begin{itemize}
\item
manual and install package
\item
first version published on CTAN
\end{itemize}

%%%%%%%%%%%%%%%%%%%%%%%%%%%%%%%%%%%%%%%%
\paragraph{v0.6:} 2017/04/26

\begin{itemize}
\item
redirection mechanism added
\end{itemize}

%%%%%%%%%%%%%%%%%%%%%%%%%%%%%%%%%%%%%%%%
\paragraph{v0.5:} 2017/04/26

\begin{itemize}
\item
functionality in definition file
\end{itemize}


%%%%%%%%%%%%%%%%%%%%%%%%%%%%%%%%%%%%%%%%%%%%%%%%%%%%%%%%%%%%%%%%%%%%%%%%%%%%%%%%
%%%%%%%%%%%%%%%%%%%%%%%%%%%%%%%%%%%%%%%%%%%%%%%%%%%%%%%%%%%%%%%%%%%%%%%%%%%%%%%%
%%%%%%%%%%%%%%%%%%%%%%%%%%%%%%%%%%%%%%%%%%%%%%%%%%%%%%%%%%%%%%%%%%%%%%%%%%%%%%%%
\appendix

\settowidth\MacroIndent{\rmfamily\scriptsize 000\ }

 \DocInput{childdoc.dtx}

\end{document}
%</driver>
% \fi
%
% %%%%%%%%%%%%%%%%%%%%%%%%%%%%%%%%%%%%%%%%%%%%%%%%%%%%%%%%%%%%%%%%%%%%%%%%%%%%%%
% %%%%%%%%%%%%%%%%%%%%%%%%%%%%%%%%%%%%%%%%%%%%%%%%%%%%%%%%%%%%%%%%%%%%%%%%%%%%%%
% \section{Sample}
%\iffalse
%<*samplemain>
%\fi
%
% The following presents a sample document
% with two chapters, two parts, a title page,
% a compile flag as well as three forwarding files to set the flag.
% It consists of eight |.tex| files:
% \begin{center}
% \begin{tabular}{ll}
% |cdocsamp.tex|&main file\\
% |cdocsch1.tex|&include file for chapter 1\\
% |cdocsch2.tex|&include file for chapter 2\\
% |cdocspt3.tex|&include file for part 3\\
% |cdocspt4.tex|&include file for part 4\\
% |cdocsdrf.tex|&forwarding file for main file in draft mode\\
% |cdocsfi1.tex|&forwarding file for final version of chapter 1\\
% |cdocsfi2.tex|&forwarding file for final version of chapter 2\\
% \end{tabular}
% \end{center}
% Each of the eight files can be compiled directly by the \LaTeX{} compiler.
%
% %%%%%%%%%%%%%%%%%%%%%%%%%%%%%%%%%%%%%%
% \paragraph{Main File.}
%
% The main file is called |cdocsamp.tex|.
%
% Load the \textsf{childdoc} definitions and
% declare the filename for the main document:
%    \begin{macrocode}
\input{childdoc.def}
\childdocmain{}
%    \end{macrocode}

% Optional override for |\version| flag:
%    \begin{macrocode}
%%\ifchilddoc\else\providecommand{\version}{draft}\fi
%    \end{macrocode}

% Define the default values for the |\version| flag
% (|final| for the main file and |draft| for childs):
%    \begin{macrocode}
\ifchilddoc
\providecommand{\version}{draft}
\else
\providecommand{\version}{final}
\fi
%    \end{macrocode}

% Load the standard document class:
%    \begin{macrocode}
\documentclass[12pt]{article}
%    \end{macrocode}

% Start the document body:
%    \begin{macrocode}
\begin{document}
%    \end{macrocode}

% Declare a title page.
% Print title, part of document being processed and version flag:
%    \begin{macrocode}
\addtocounter{page}{-1}
\begin{center}
{\LARGE\bfseries{}childdoc example\par}
\vspace{1cm}
\ifchilddoc
\ifchilddocmanual part\else chapter\fi:
`\childdocname' of `\childdocjob'\par
\else
main document: `\childdocjob'\par
\fi
version: \version\par
\end{center}
\newpage
%    \end{macrocode}

% Manually include selected file,
% otherwise process as usual:
%    \begin{macrocode}
\ifchilddocmanual
\section*{part `\childdocname'}
\input{\childdocname}
\else
%    \end{macrocode}

% Include the two chapters:
%    \begin{macrocode}
\include{cdocsch1}
\include{cdocsch2}
%    \end{macrocode}

% Include the two parts unless only chapters should be displayed:
%    \begin{macrocode}
\ifchilddoc\else
\section{part three}
\input{cdocspt3}
\section{part four}
\input{cdocspt4}
\fi
%    \end{macrocode}

% Process as usual until here:
%    \begin{macrocode}
\fi
%    \end{macrocode}

% End of document body:
%    \begin{macrocode}
\end{document}
%    \end{macrocode}
%\iffalse
%</samplemain>
%\fi
%
% %%%%%%%%%%%%%%%%%%%%%%%%%%%%%%%%%%%%%%
% \paragraph{Chapter Include Files.}
%
% The include files are called |cdocsch1.tex| and |cdocsch2.tex|.
%
%\iffalse
%<*samplechap1|samplechap2>
%\fi

% Optional override for |\version| flag:
%    \begin{macrocode}
%%\providecommand{\version}{final}
%    \end{macrocode}

% Include the main document:
%    \begin{macrocode}
\input{childdoc.def}
\childdocof{cdocsamp}
%    \end{macrocode}

%\iffalse
%</samplechap1|samplechap2>
%\fi
%
%\iffalse
%<*samplechap1>
%\fi
% Some text for chapter 1:
%    \begin{macrocode}
\section{one}
some text in chapter one
%    \end{macrocode}

%\iffalse
%</samplechap1>
%\fi
% Some text for chapter 2:
%\iffalse
%<*samplechap2>
%\fi
%    \begin{macrocode}
\section{two}
more text in chapter two
%    \end{macrocode}

%\iffalse
%</samplechap2>
%\fi
%
% %%%%%%%%%%%%%%%%%%%%%%%%%%%%%%%%%%%%%%
% \paragraph{Part Include Files.}
%
% The include files are called |cdocspt3.tex| and |cdocspt4.tex|.
%
%\iffalse
%<*samplepart3|samplepart4>
%\fi

% Optional override for |\version| flag:
%    \begin{macrocode}
%%\providecommand{\version}{final}
%    \end{macrocode}

% Include the main document:
%    \begin{macrocode}
\input{childdoc.def}
\childdocby{cdocsamp}
%    \end{macrocode}

%\iffalse
%</samplepart3|samplepart4>
%\fi
%
%\iffalse
%<*samplepart3>
%\fi
% Some text for part 3:
%    \begin{macrocode}
some text in part three
%    \end{macrocode}

%\iffalse
%</samplepart3>
%\fi
% Some text for part 4:
%\iffalse
%<*samplepart4>
%\fi
%    \begin{macrocode}
more text in part four
%    \end{macrocode}

%\iffalse
%</samplepart4>
%\fi
%
% %%%%%%%%%%%%%%%%%%%%%%%%%%%%%%%%%%%%%%
% \paragraph{Forwarding for a Complete Draft.}
%
% The following forwarding file |cdocsdrf.tex|
% compiles the main document in draft mode:
%\iffalse
%<*sampledraft>
%\fi
%    \begin{macrocode}
\def\version{draft}
\input{childdoc.def}
\childdocforward{cdocsamp}
%    \end{macrocode}

%\iffalse
%</sampledraft>
%\fi
%
% %%%%%%%%%%%%%%%%%%%%%%%%%%%%%%%%%%%%%%
% \paragraph{Forwarding for Final Version of the Chapters.}
%
% The following forwarding files |cdocsfn1.tex| and |cdocsfn2.tex|
% (with identical content)
% compile the final versions of the child documents
% |cdocsch1.tex| and |cdocsch2.tex|, respectively:
%\iffalse
%<*samplefinal>
%\fi
%    \begin{macrocode}
\def\version{final}
\input{childdoc.def}
\childdocforwardprefix[cdocsamp]{cdocsfn}{cdocsch}
%    \end{macrocode}

%\iffalse
%</samplefinal>
%\fi
%
% %%%%%%%%%%%%%%%%%%%%%%%%%%%%%%%%%%%%%%
% \paragraph{Command Line Processing.}
%
% The following three command lines generate the output files
% |cdocscld|, |cdocscl1| and |cdocscl2|
% which should be identical to
% |cdocsdrf|, |cdocsch1| and |cdocsfn2|, respectively:
% \begin{center}
% \begin{tabular}{l}
% |latex -jobname cdocscld \|\\
% |  "\def\version{draft}\input{childdoc.def}\childdocforward{cdocsamp}"|\\
% |latex -jobname cdocscl1 \|\\
% |  "\input{childdoc.def}\childdocforward[cdocsamp]{cdocsch1}"|\\
% |latex -jobname cdocscl2 \|\\
% |  "\def\version{final}\input{childdoc.def}\childdocforward{cdocsch2}"|
% \end{tabular}
% \end{center}
% Note that the trailing backslash on each first line
% merely continues the input to the second line
% (for convenient cut ant paste).
% Furthermore, the command |latex| can be replaced by any
% of its alternative versions such as |pdflatex|.
%
% %%%%%%%%%%%%%%%%%%%%%%%%%%%%%%%%%%%%%%%%%%%%%%%%%%%%%%%%%%%%%%%%%%%%%%%%%%%%%%
% %%%%%%%%%%%%%%%%%%%%%%%%%%%%%%%%%%%%%%%%%%%%%%%%%%%%%%%%%%%%%%%%%%%%%%%%%%%%%%
% \section{Implementation}
%\iffalse
%<*package>
%\fi
%
% This section describes the definitions file |childdoc.def|.

% The definitions cannot be loaded using |\usepackage| or |\RequirePackage|
% which has a mechanism to prevent loading a style file more than once.
% When loading the definitions by means of |\input|
% multiple instances have to be prevented manually:
%\iffalse
%This code needs to be before the `\ProvidesFile' directive
%which is defined at the beginning of this file.
%Therefore it is also placed there and commented out here.
%</package>
%<*discard>
%\fi
%    \begin{macrocode}
\ifdefined\childdocmain\endinput\fi
%    \end{macrocode}
%\iffalse
%</discard>
%<*package>
%\fi
%
% \macro{\ifchilddoc}
% \macro{\ifchilddocmanual}
% The conditional |\ifchilddoc| tells whether a
% child (true) or main (false) document is being compiled.
% The conditional |\ifchilddocmanual| tells whether
% the |\includeonly| mechanism is used (false) or
% the selection of child files must be performed manually (true).
% The definitions initialise to false:
%    \begin{macrocode}
\newif\ifchilddoc
\newif\ifchilddocmanual
%    \end{macrocode}

% \macro{\childdocname}
% \macro{\childdocjob}
% The macro |\childdocname| stores the name of the main document
% to be compiled. The macro |\childdocjob| stores the name of
% the document on which the \LaTeX{} compiler was originally invoked.
% The content of |\jobname| cannot be compared
% to filenames specified in the source due to different catcodes.
% The following code rescans |\jobname|, stores the result
% in |\childdocname| and saves a copy in |\childdocjob|:
%    \begin{macrocode}
\edef\childdocname{\scantokens\expandafter{\jobname\noexpand}}
\let\childdocjob\childdocname
%    \end{macrocode}

% \macro{\childdocdisable}
% The macro |\childdocdisable| prevents the main file
% from being processed more than once.
% At this stage, the main document command |\childdocmain|
% is assumed to be called once again where it should do nothing.
% Any subsequent call to it should prevent
% a secondary processing of the main document
% It overwrites the forwarding commands
% |\childdocof| and |\childdocforward|
% with empty macros to prevent further inclusions of the main document:
%    \begin{macrocode}
\newcommand{\childdocdisable}
{
  \renewcommand{\childdocmain}[1]{\renewcommand{\childdocmain}[1]{\endinput}}
  \renewcommand{\childdocof}[1]{}
  \renewcommand{\childdocby}[2][]{}
  \renewcommand{\childdocforward}[2][]{}
  \renewcommand{\childdocdisable}{}
}
%    \end{macrocode}

% \macro{\childdocmain}
% The macro |\childdocmain| is to be called at the top of the main file
% with nothing or the main filename (without extension) as argument.
% First, it breaks loops.
% If the argument is not empty and does not match |\childdocname|
% (which is set by the first inclusion of |childdoc.def|),
% |\ifchilddoc| is set to true, |\includeonly| is applied to the child file
% and |\jobname| is set to the main file
% (for proper handling of |.aux| files):
%    \begin{macrocode}
\newcommand{\childdocmain}[1]
{
  \childdocdisable\childdocmain{}
  \if?#1?\else
    \begingroup
      \def\childdoctmp{#1}
      \ifx\childdoctmp\childdocname
        \def\childdoctmp{}
      \else
        \def\childdoctmp
        {
          \childdoctrue
          \includeonly{\childdocname}
          \def\childdocjob{#1}
          \def\jobname{#1}
        }
      \fi
      \expandafter
    \endgroup
    \childdoctmp
  \fi
}
%    \end{macrocode}

% \macro{\childdocof}
% The command |\childdocof| redirects
% compilation to the main file |#1|.
%    \begin{macrocode}
\newcommand{\childdocof}[1]
{
  \childdocdisable
  \childdoctrue
  \includeonly{\childdocname}
  \def\jobname{#1}
  \def\childdocjob{#1}
  \input{#1}
}
%    \end{macrocode}

% \macro{\childdocby}
% The command |\childdocby| ....
%    \begin{macrocode}
\newcommand{\childdocby}[2][]
{
  \childdocdisable
  \childdoctrue
  \childdocmanualtrue
  \if?#1?\else
    \def\jobname{#2}
  \fi
  \def\childdocjob{#2}
  \input{#2}
  \endinput
}
%    \end{macrocode}

% \macro{\childdocforward}
% The command |\childdocforward| redirects
% compilation to the main file or
% (if the optional argument is given) a child file.
% Parameters are set as if the main file
% or a child file starting with |\childdocof| was compiled.
% Then compilation is handed over to the main file:
%    \begin{macrocode}
\newcommand{\childdocforward}[2][]
{
  \begingroup
    \if?#1?
      \def\childdoctmp
      {
        \def\childdocname{#2}
        \def\childdocjob{#2}
        \def\jobname{#2}
        \input{#2}
        \endinput
      }
    \else
      \def\childdoctmp
      {
        \childdocdisable
        \def\childdocname{#2}
        \childdoctrue
        \includeonly{#2}
        \def\childdocjob{#1}
        \def\jobname{#1}
        \input{#1}
        \endinput
      }
    \fi
    \expandafter
  \endgroup
  \childdoctmp
}
%    \end{macrocode}

% \macro{\childdocforwardprefix}
% The command |\childdocforwardprefix| redirects
% compilation to the main or a child file by means of a pattern.
% The prefix |#1| in the current filename is replaced by |#2|
% and the suffix of the current filename is kept
% (it is assumed that the filename does not contain the substring `|~~~|'
% which is used as a delimiter).
% Compilation is handed over to the new file by |\childdocforward|:
%    \begin{macrocode}
\newcommand{\childdocforwardprefix}[3][]
{
  \begingroup
    \def\childdocextract #2##1~~~{\def\childdoctmp{\childdocforward[#1]{#3##1}}}
    \expandafter\childdocextract\childdocname~~~
    \expandafter
  \endgroup
  \childdoctmp
}
%    \end{macrocode}

% \macro{\childdoc}
% The deprecated macro |\childdoc| is a legacy version of |\childdocmain|:
%    \begin{macrocode}
\newcommand{\childdoc}{\childdocmain}
%    \end{macrocode}

% \macro{\childdocredirect}
% The deprecated macro |\childdocredirect| is a legacy version
% of |\childdocforward| and |\childdocforwardprefix|:
%    \begin{macrocode}
\newcommand{\childdocredirect}[2][]
{
  \begingroup
    \if?#1?
      \def\childdoctmp{\childdocforward{#2}}
    \else
      \def\childdoctmp{\childdocforwardprefix{#1}{#2}}
    \fi
    \expandafter
  \endgroup
  \childdoctmp
}
%    \end{macrocode}

%\iffalse
%</package>
%\fi
%
\endinput

\childdocmain{}
%    \end{macrocode}

% Optional override for |\version| flag:
%    \begin{macrocode}
%%\ifchilddoc\else\providecommand{\version}{draft}\fi
%    \end{macrocode}

% Define the default values for the |\version| flag
% (|final| for the main file and |draft| for childs):
%    \begin{macrocode}
\ifchilddoc
\providecommand{\version}{draft}
\else
\providecommand{\version}{final}
\fi
%    \end{macrocode}

% Load the standard document class:
%    \begin{macrocode}
\documentclass[12pt]{article}
%    \end{macrocode}

% Start the document body:
%    \begin{macrocode}
\begin{document}
%    \end{macrocode}

% Declare a title page.
% Print title, part of document being processed and version flag:
%    \begin{macrocode}
\addtocounter{page}{-1}
\begin{center}
{\LARGE\bfseries{}childdoc example\par}
\vspace{1cm}
\ifchilddoc
\ifchilddocmanual part\else chapter\fi:
`\childdocname' of `\childdocjob'\par
\else
main document: `\childdocjob'\par
\fi
version: \version\par
\end{center}
\newpage
%    \end{macrocode}

% Manually include selected file,
% otherwise process as usual:
%    \begin{macrocode}
\ifchilddocmanual
\section*{part `\childdocname'}
\input{\childdocname}
\else
%    \end{macrocode}

% Include the two chapters:
%    \begin{macrocode}
\include{cdocsch1}
\include{cdocsch2}
%    \end{macrocode}

% Include the two parts unless only chapters should be displayed:
%    \begin{macrocode}
\ifchilddoc\else
\section{part three}
\input{cdocspt3}
\section{part four}
\input{cdocspt4}
\fi
%    \end{macrocode}

% Process as usual until here:
%    \begin{macrocode}
\fi
%    \end{macrocode}

% End of document body:
%    \begin{macrocode}
\end{document}
%    \end{macrocode}
%\iffalse
%</samplemain>
%\fi
%
% %%%%%%%%%%%%%%%%%%%%%%%%%%%%%%%%%%%%%%
% \paragraph{Chapter Include Files.}
%
% The include files are called |cdocsch1.tex| and |cdocsch2.tex|.
%
%\iffalse
%<*samplechap1|samplechap2>
%\fi

% Optional override for |\version| flag:
%    \begin{macrocode}
%%\providecommand{\version}{final}
%    \end{macrocode}

% Include the main document:
%    \begin{macrocode}
% \iffalse
%
% childdoc.dtx Copyright (C) 2017-2018 Niklas Beisert
%
% This work may be distributed and/or modified under the
% conditions of the LaTeX Project Public License, either version 1.3
% of this license or (at your option) any later version.
% The latest version of this license is in
%   http://www.latex-project.org/lppl.txt
% and version 1.3 or later is part of all distributions of LaTeX
% version 2005/12/01 or later.
%
% This work has the LPPL maintenance status `maintained'.
%
% The Current Maintainer of this work is Niklas Beisert.
%
% This work consists of the files childdoc.dtx and childdoc.ins
% and the derived files childdoc.def and cdocsamp.tex with
% cdocsch1.tex, cdocsch2.tex, cdocsdrf.tex, cdocsfn1.tex, cdocsfn2.tex.
%
%<package>\ifdefined\childdocmain\endinput\fi
%<package>\ProvidesFile{childdoc.def}[2018/12/30 v2.0 child document driver]
%<samplemain>\ProvidesFile{cdocsamp.tex}[2018/12/30 v2.0 sample for childdoc]
%<*driver>
%\ProvidesFile{childdoc.drv}[2018/12/30 v2.0 childdoc reference manual file]
\PassOptionsToClass{10pt,a4paper}{article}
\documentclass{ltxdoc}

\usepackage[margin=35mm]{geometry}
\usepackage{hyperref}
\usepackage{hyperxmp}
\usepackage[usenames]{color}

\hypersetup{colorlinks=true}
\hypersetup{pdfstartview=FitH}
\hypersetup{pdfpagemode=UseNone}
\hypersetup{pdfsource={}}
\hypersetup{pdflang={en-UK}}
\hypersetup{pdfcopyright={Copyright 2017-2018 Niklas Beisert.
  This work may be distributed and/or modified under the
  conditions of the LaTeX Project Public License, either version 1.3
  of this license or (at your option) any later version.}}
\hypersetup{pdflicenseurl={http://www.latex-project.org/lppl.txt}}
\hypersetup{pdfcontactaddress={ETH Zurich, ITP, HIT K,
  Wolfgang-Pauli-Strasse 27}}
\hypersetup{pdfcontactpostcode={8093}}
\hypersetup{pdfcontactcity={Zurich}}
\hypersetup{pdfcontactcountry={Switzerland}}
\hypersetup{pdfcontactemail={nbeisert@itp.phys.ethz.ch}}
\hypersetup{pdfcontacturl={http://people.phys.ethz.ch/\xmptilde nbeisert/}}

\newcommand{\secref}[1]{\hyperref[#1]{section \ref*{#1}}}

\parskip1ex
\parindent0pt
\let\olditemize\itemize
\def\itemize{\olditemize\parskip0pt}

\begin{document}

\title{The \textsf{childdoc} Package}
\hypersetup{pdftitle={The childdoc Package}}
\author{Niklas Beisert\\[2ex]
  Institut f\"ur Theoretische Physik\\
  Eidgen\"ossische Technische Hochschule Z\"urich\\
  Wolfgang-Pauli-Strasse 27, 8093 Z\"urich, Switzerland\\[1ex]
  \href{mailto:nbeisert@itp.phys.ethz.ch}
  {\texttt{nbeisert@itp.phys.ethz.ch}}}
\hypersetup{pdfauthor={Niklas Beisert}}
\hypersetup{pdfsubject={Manual for the LaTeX2e Package childdoc}}
\date{30 December 2018, \textsf{v2.0}}
\maketitle

\begin{abstract}\noindent
\textsf{childdoc} is a \LaTeXe{} package
that enables the direct compilation
of document sections included by |\include|
to individual files.
\end{abstract}

\begingroup
\parskip0ex
\tableofcontents
\endgroup

%%%%%%%%%%%%%%%%%%%%%%%%%%%%%%%%%%%%%%%%%%%%%%%%%%%%%%%%%%%%%%%%%%%%%%%%%%%%%%%%
%%%%%%%%%%%%%%%%%%%%%%%%%%%%%%%%%%%%%%%%%%%%%%%%%%%%%%%%%%%%%%%%%%%%%%%%%%%%%%%%
\section{Introduction}

\LaTeX{} provides a mechanism to structure a large document (such as a book)
into a main file and several child files (containing the chapters)
using the |\include| command.
This mechanism is beneficial for documents
which span hundreds of pages in order to
make the source file(s) more manageable.
Moreover, compilation can be restricted to
selected child files by means of the |\includeonly| command.
The latter feature can be used to reduce the compilation time while editing
(this was significantly more useful in the earlier days of \LaTeX{})
or to generate a smaller document which is easier to navigate.
Another application of |\includeonly| is to generate
documents consisting of selected parts of the complete document.

However, there are a few drawbacks of the plain |\include| mechanism:
\begin{itemize}
\item
The child files cannot be compiled on their own,
they can only be compiled via the main file.
A naive editing environment
(such as a text editor with an option
to have the current file processed by \LaTeX)
may require one to switch to the main file before compiling;
attempting to compile the child file produces errors.
\item
The main file must be modified (each time)
to adjust the |\includeonly| command
to the present needs. This easily leaves the main file in a messy state.
\item
The generated document will always carry the filename
of the main document. This is inconvenient if
several child files are to be compiled and
to be kept for distribution.
\end{itemize}

The present package provides a simple interface
to make child files individually compilable by \LaTeX{}.
Compiling a child file then has the same effect as compiling
the main file with an |\includeonly| command
to select the appropriate child.
Moreover the generated document will carry the name of the child
rather than the main file.
This resolves all three above issues.

This feature is meant to make the editing of books,
thesis documents and lecture notes somewhat more convenient.
However, the package can also be used efficiently for
composing a series of documents (such as exercise sheets)
which are typically distributed individually.
It then assists the author in generating the individual documents
(potentially in different versions)
as well as a document containing the collected series.
Another application is in developing style files
or other kinds of included material
where compilation of the style file could redirect
to a sample or test file.

%%%%%%%%%%%%%%%%%%%%%%%%%%%%%%%%%%%%%%%%%%%%%%%%%%%%%%%%%%%%%%%%%%%%%%%%%%%%%%%%
%%%%%%%%%%%%%%%%%%%%%%%%%%%%%%%%%%%%%%%%%%%%%%%%%%%%%%%%%%%%%%%%%%%%%%%%%%%%%%%%
\section{Usage}

First of all, the package \textsf{childdoc} is \emph{not} a standard
\LaTeXe{} |.sty| style file! Therefore it needs to be invoked in
a non-standard way.

%%%%%%%%%%%%%%%%%%%%%%%%%%%%%%%%%%%%%%%%%%%%%%%%%%%%%%%%%%%%%%%%%%%%%%%%%%%%%%%%
\subsection{Included Files}
\label{sec:include}

%%%%%%%%%%%%%%%%%%%%%%%%%%%%%%%%%%%%%%%%
\DescribeMacro{\childdocmain}
To use the package, add the commands
\begin{center}
\begin{tabular}{l}
|\input{childdoc.def}|\\
|\childdocmain{}|\\
\end{tabular}
\end{center}
at the very top of the main \LaTeX{} file,
in particular \emph{before} the |\documentclass| statement!
The argument of |\childdocmain| should be left empty
(but it must be present).

%%%%%%%%%%%%%%%%%%%%%%%%%%%%%%%%%%%%%%%%
\DescribeMacro{\childdocof}
Furthermore, add the commands
\begin{center}
\begin{tabular}{l}
|\input{childdoc.def}|\\
|\childdocof{|\textit{main}|}|\\
\end{tabular}
\end{center}
at the top of every child file \textit{child}
which is included by |\include{|\textit{child}|}|
from within the main file
(or at least for those files to be compiled individually).
The argument \textit{main} must be the filename of the main file.

There are a couple of
considerations in setting up the main and child documents:

%%%%%%%%%%%%%%%%%%%%%%%%%%%%%%%%%%%%%%%%
\paragraph{Restrictions.}

Please note the following restrictions:
\begin{itemize}
\item
|\childdocmain| must be called with one argument \textit{main}
to ensure compatibility with earlier version of the package.
It must either be empty (|\childdocmain{}|)
or precisely match the filename of the main file in which it is specified.
See \secref{sec:detection} for further information.
\item
The filename \textit{main} must be specified without the |.tex| extension.
\item
The filename \textit{main} is case sensitive
(even in case-insensitive file systems)
due to internal string comparison.
\item
The argument \textit{main} should be fully expanded, it cannot be a macro.
\item
Subdirectories and special characters should be avoided in filenames.
\item
The command |\childdocmain{|\textit{main}|}| must be followed by a whitespace.
It should not be followed immediately by another command
or by a comment mark `|%|'.
This is because the \TeX{} parser reads the token immediately following
the argument of |\childdocmain| and puts it
at the beginning of every child section;
however, a white\-space is ignored.
\end{itemize}

%%%%%%%%%%%%%%%%%%%%%%%%%%%%%%%%%%%%%%%%
\paragraph{Content of Main File.}

It is advisable to place all content in the child files included by |\include|.
Any output contained in the main file will appear in all child documents
unless suppressed manually;
it cannot be suppressed automatically by the |\includeonly| directive
and thus should normally be avoided.
A method to include some content in the main file
by means of conditional processing is described in \secref{sec:conditional}.

%%%%%%%%%%%%%%%%%%%%%%%%%%%%%%%%%%%%%%%%
\paragraph{Page Numbering.}

When only a part of the document is compiled,
the appropriate numbering of pages
(as well as other status parameters)
is determined from the |.aux| files.
The latter contain information from previous passes.
However this information needs to propagate through
all intermediate child documents.
Therefore the page numbering in child documents may well
be inconsistent until the complete document is compiled at least once.

A useful (if unconventional) way to always ensure a consistent
page numbering is to restart the numbering in each child document
and denote the pages by `\textit{child}|.|\textit{page}'
where \textit{child} represents the chapter/section number of the child file.
This can be achieved by the command
|\numberwithin{page}{|\textit{child}|}|
of the \textsf{amsmath} package
where \textit{child} can be |chapter| or |section|
depending on the chosen structuring.
Alternatively, one can modify the macro |\thepage| appropriately
and reset the counter |page| at the start of each child file.

%%%%%%%%%%%%%%%%%%%%%%%%%%%%%%%%%%%%%%%%%%%%%%%%%%%%%%%%%%%%%%%%%%%%%%%%%%%%%%%%
\subsection{Conditional Processing}
\label{sec:conditional}

The package provides a mechanism to compile different versions
of a document. To customise the versions further some conditional processing
can come in handy to distinguish which version is being compiled.
The package provides two macros to describe the compilation context:

%%%%%%%%%%%%%%%%%%%%%%%%%%%%%%%%%%%%%%%%
\DescribeMacro{\ifchilddoc}
The conditional |\ifchilddoc| distinguishes between the compilation of
child documents and the main document:
%
\begin{center}
|\ifchilddoc |\textit{child-code}| |[|\||else |\textit{main-code}]| \||fi|
\end{center}

%%%%%%%%%%%%%%%%%%%%%%%%%%%%%%%%%%%%%%%%
\DescribeMacro{\childdocname}
\DescribeMacro{\childdocjob}
The macro |\childdocname| contains the filename (without extension)
of the main or child file being processed.
Note that |\childdocjob| will always contain the name of the main file.

%%%%%%%%%%%%%%%%%%%%%%%%%%%%%%%%%%%%%%%%
\paragraph{Title Page.}

Conditional processing can be used to include a title or banner page
in the main document when proper precautions are taken.
Importantly, the code in the main file should ensure that the page counter
(as well as other status parameters which are stored in the |.aux| files)
takes the same value after the conditional processing.
Otherwise the page numbers may take divergent values
depending on which part is compiled.

For example, a title page could be declared by:
%
\begin{center}
\begin{tabular}{l}
|\ifchilddoc\||else|\\
|\addtocounter{page}{-1}|\\
\textit{code for title page}\\
|\newpage|\\
|\||fi|
\end{tabular}
\end{center}
%
A banner page for the child documents can be generated by:
%
\begin{center}
\begin{tabular}{l}
|\ifchilddoc|\\
|\addtocounter{page}{-1}|\\
\textit{code for banner page}\\
|\newpage|\\
|\||fi|
\end{tabular}
\end{center}
%
Here one could write a message such as:
\begin{center}
|This is the part \childdocname{} of \childdocjob{}.|
\end{center}

%%%%%%%%%%%%%%%%%%%%%%%%%%%%%%%%%%%%%%%%%%%%%%%%%%%%%%%%%%%%%%%%%%%%%%%%%%%%%%%%
\subsection{Flags}
\label{sec:flags}

The package makes it easy to generate different versions
of the main or child documents.
To this end compilation flags can be defined
and assigned different default values.
They will be particularly useful in conjunction
with the forwarding mechanism described in \secref{sec:forward}.

For example, it may be useful to have a flag |\version|
which can be set to |draft| or |final|.
The document source will contain some conditional code
depending on the value of |\version|.
Suppose further, the flag should default to |final| for the main file
and to |draft| for child files
which is a natural assignment for editing the document.
This is achieved by placing the following code
in the preamble of the main document
(below the |\childdocmain| directive):
%
\begin{center}
\begin{tabular}{l}
|\ifchilddoc|\\
|\providecommand{\version}{draft}|\\
|\||else|\\
|\providecommand{\version}{final}|\\
|\||fi|
\end{tabular}
\end{center}
%
The definition by |\providecommand| makes sure
that previous definitions are not overwritten.
Further statements |\providecommand{\version}{...}|
can thus be added before the above code to override it.

For the main file, one might add a line
(between |\childdocmain| and the above block)
%
\begin{center}
|%\ifchilddoc\||else\providecommand{\version}{draft}\||fi|
\end{center}
%
which can be uncommented to produce a draft version.
Likewise one can add a line to the very top of a child file
(above the |\childdocof{|\textit{main}|}| directive)
%
\begin{center}
|%\providecommand{\version}{final}|
\end{center}
%
which can be uncommented to produce the final version of this child document.

%%%%%%%%%%%%%%%%%%%%%%%%%%%%%%%%%%%%%%%%%%%%%%%%%%%%%%%%%%%%%%%%%%%%%%%%%%%%%%%%
\subsection{Forwarding}
\label{sec:forward}

Different versions of the main or child documents
using compilation flags as described in \secref{sec:flags}
can be (permanently) stored in different files
for convenient compilation, viewing and distribution.
To this end, the package defines a command
to pass on compilation to a different file:

%%%%%%%%%%%%%%%%%%%%%%%%%%%%%%%%%%%%%%%%
\DescribeMacro{\childdocforward}
The command |\childdocforward| redirects processing to
another source file:
%
\begin{center}
\begin{tabular}{l}
|\input{childdoc.def}|\\
|\childdocforward[|\textit{main}|]{|\textit{dest}|}|\\
\end{tabular}
\end{center}
%
The argument \textit{dest} is the destination file
(without extension).
It should be the main file or one of the child files.
Note that further \textsf{childdoc} directives
such as |\childdocof| and |\childdocforward|
in the indicated file will be processed in this form.
The optional argument \textit{main}
passes on directly to the main file \textit{main}
while pretending to compile the child \textit{dest}.
This form behaves as if \textit{dest}
issues |\childdocof{|\textit{main}|}| right away,
and no further \textsf{childdoc} directives will be processed.

%%%%%%%%%%%%%%%%%%%%%%%%%%%%%%%%%%%%%%%%
\DescribeMacro{\...prefix}
In the alternative form |\childdocforwardprefix|,
%
\begin{center}
\begin{tabular}{l}
|\input{childdoc.def}|\\
|\childdocforwardprefix[|\textit{main}|]{|\textit{prefix}|}{|\textit{dest}|}|
\end{tabular}
\end{center}
%
the destination file is determined by a pattern
depending on the current file:
To make this work, the current file must be called
`{\textit{prefix}\hspace{0.2em}\textit{suffix}}'
with \textit{prefix} matching precisely the argument.
Processing is then passed on to the file
`{\textit{dest}\hspace{0.2em}\textit{suffix}}'.
Surely, the same effect is achieved by
directly specifying the
argument `{\textit{dest}\hspace{0.2em}\textit{suffix}}'
in the first form.
However, that requires to set up a different file
for each child. With the alternative form of the command
all these files can have exactly the same content
which simplifies setting them up and maintaining them.

For example, the following file |draft.tex|
with a compilation flag |\version| as described in \secref{sec:flags}
compiles the main document as a draft:
%
\begin{center}
\begin{tabular}{l}
|\def\version{draft}|\\
|\input{childdoc.def}|\\
|\childdocforward{|\textit{main}|}|
\end{tabular}
\end{center}
%
Likewise, the following files |final|\textit{nn}|.tex|
compile the final version of the child document
|child|\textit{nn}|.tex|:
%
\begin{center}
\begin{tabular}{l}
|\def\version{final}|\\
|\input{childdoc.def}|\\
|\childdocforwardprefix{final}{child}|
\end{tabular}
\end{center}
%

Note that when several versions of a main file and/or of each child file
are to be generated, it may be convenient to set up a |Makefile| or
shell script to automatise the process.

%%%%%%%%%%%%%%%%%%%%%%%%%%%%%%%%%%%%%%%%%%%%%%%%%%%%%%%%%%%%%%%%%%%%%%%%%%%%%%%%
\subsection{Command Line Processing}
\label{sec:commandline}

The effect of redirection files can also be achieved by invoking
the \LaTeX{} compiler with a more elaborate command line.
Most conveniently this should be done as part
of a shell script or a |Makefile|.

When using \textsf{childdoc} in the main file, the following
command lines effectively perform a redirection
(note that depending on the shell being used,
backslashes may have to be doubled: `|\|' $\to$ `|\\|'):
%
\begin{center}
|... -jobname "|\textit{target}|" |\\|"|[\textit{flags}]%
|\input{childdoc.def}\childdocforward[|\textit{main}|]{|\textit{dest}|}"|
\end{center}
%
Here \textit{target} is the name of the output file,
\textit{main} is the name of the main file
and \textit{dest} is the name of the main or child file to be processed
(all filenames without extensions).
The optional argument \textit{main} can be omitted
if \textit{main} matches \textit{dest}.
Optionally, compilation \textit{flags} can be defined via |\def| commands.
This command line makes the \TeX{} engine believe
it is compiling the file \textit{target}
whose content is specified as the latter parameter.
The provided code then forwards the processing to
\textit{main} or \textit{dest} as described in \secref{sec:forward}.

%%%%%%%%%%%%%%%%%%%%%%%%%%%%%%%%%%%%%%%%%%%%%%%%%%%%%%%%%%%%%%%%%%%%%%%%%%%%%%%%
\subsection{Include by Input}
\label{sec:input}

Including child documents by |\include| has some restrictions by design.
Most notably, the content of a child document always occupies
its own set of pages; pages cannot be shared between child documents.
Usually, this behaviour makes perfect sense
because each child document contain an essential part of the document.
However, in some situations it may be desirable to compose
a document from a collection of parts
without having mandatory page breaks between then.
For this case, the package
provides a mechanism to include parts
by |\input| which can also be processed individually.
However, by construction this mechanism
requires manual handling of the content to be output.

%%%%%%%%%%%%%%%%%%%%%%%%%%%%%%%%%%%%%%%%
\DescribeMacro{\ifchilddocmanual}
The main file should be prepared as usual, see \secref{sec:include}.
However, the document body must make a distinction
between processing of an individual part and of the main document, e.g.:
%
\begin{center}
\begin{tabular}{l}
|\ifchilddocmanual|\\
|\input{\childdocname}|\\
|\||else|\\
\textit{document body with }|\input{|\textit{part}|}|\\
|\||fi|
\end{tabular}
\end{center}
%
The conditional |\ifchilddocmanual| is true whenever
a part to be included by |\input| is being compiled,
and the name of the part is stored in |\childdocname|.

%%%%%%%%%%%%%%%%%%%%%%%%%%%%%%%%%%%%%%%%
\DescribeMacro{\childdocby}
Each part to be included by |\input| should start with:
%
\begin{center}
\begin{tabular}{l}
|\input{childdoc.def}|\\
|\childdocby{|\textit{main}|}|\\
\end{tabular}
\end{center}
%
The directive |\childdocby| is similar to |\childdocof|
described in \secref{sec:include},
but the subsequent selection of content must be done manually.
To that end, both |\ifchilddoc| and |\ifchilddocmanual|
will be true upon processing of a part,
and the name of the part is stored in |\childdocname|.
Note that |\jobname| will be set to the filename of the current part
so that each part receives an individual |.aux| file
that does not interfere with the |.aux| file(s) of the main document.
This behaviour can be altered by the alternative form
|\childdocby[*]{|\textit{main}|}| (with a non-empty optional argument)
which uses the |.aux| file of the main document
by setting |\jobname| to \textit{main}.

%%%%%%%%%%%%%%%%%%%%%%%%%%%%%%%%%%%%%%%%%%%%%%%%%%%%%%%%%%%%%%%%%%%%%%%%%%%%%%%%
\subsection{Driver Development}
\label{sec:driver}

The \textsf{childdoc} mechanism can also be use for the development
of definition files such as \LaTeX{} styles or classes.
This case differs from the above setup with multiple parts
included by |\include| in that no |\includeonly| should be invoked.
This can be achieved by starting the include file
(before |\ProvidesPackage|) with:
%
\begin{center}
\begin{tabular}{l}
|\input{childdoc.def}|\\
|\childdocforward{|\textit{main}|}|\\
\end{tabular}
\end{center}
%
or alternatively with:
%
\begin{center}
\begin{tabular}{l}
|\input{childdoc.def}|\\
|\childdocby{|\textit{main}|}|\\
\end{tabular}
\end{center}
%
Both forms have slightly different effects as described above.
The main file is prepared as usual, see \secref{sec:include}.

%%%%%%%%%%%%%%%%%%%%%%%%%%%%%%%%%%%%%%%%%%%%%%%%%%%%%%%%%%%%%%%%%%%%%%%%%%%%%%%%
\subsection{Legacy Detection}
\label{sec:detection}

The directive |\childdocmain| in the main file can detect
whether the complete document or merely a child is to be compiled
even without using the directive |\childdocof|.
This method is deprecated because it is less robust
and there is no compelling reason to use it;
it is merely provided for backward compatibility
and it may be removed in future versions.

If the detection mechanism is to be used,
it is mandatory to correctly specify
the filename of the main file as the argument of |\childdocmain|:
%
\begin{center}
\begin{tabular}{l}
|\input{childdoc.def}|\\
|\childdocmain{|\textit{main}|}|\\
\end{tabular}
\end{center}
%
If |\jobname| does not match the argument \textit{main} of |\childdocmain|,
it is assumed that |\jobname| points to the child file to be compiled.
When using |\childdocmain| with the main file specified as argument,
it suffices to start a child file
with just |\input{|\textit{main}|}|
without loading of the package and using |\childdocof|.
If instead all processing is done
with the appropriate \textsf{childdoc} directives,
the argument of \textit{main} of |\childdocmain| can be empty.

An alternative version of the command line processing described
in \secref{sec:commandline} using the detection mechanism reads:
%
\begin{center}
|... -jobname "|\textit{target}|" "|[\textit{flags}]%
[|\def\jobname{|\textit{dest}|}|]|\input{|\textit{main}|}"|
\end{center}

%%%%%%%%%%%%%%%%%%%%%%%%%%%%%%%%%%%%%%%%%%%%%%%%%%%%%%%%%%%%%%%%%%%%%%%%%%%%%%%%
\subsection{Manual Code}
\label{sec:manual}

In case one cannot be certain whether the definitions file |childdoc.def|
is installed on the target \TeX{} distribution
and one prefers not to ship it,
it is conceivable to paste a few relevant commands into the sources.

To that end, drop all statements |\input{childdoc.def}|
and perform the replacements as outlined below.
Instead of |\childdocmain{|\textit{main}|}| add the following code
to the top of the main file:
%
\begin{center}
\begin{tabular}{l}
|\||ifdefined\childdocname\endinput\||fi\newif\ifchilddoc|\\
|\edef\childdocname{\scantokens\expandafter{\jobname\noexpand}}|\\
|\def\childdocmain{|\textit{main}|}\||ifx\childdocmain\childdocname\||else|\\
|\childdoctrue\includeonly{\childdocname}\let\jobname\childdocmain\||fi|\\
\end{tabular}
\end{center}
%
Instead of |\childdocof{|\textit{main}|}| just include the main file
at the top of each child file:
%
\begin{center}
|\input{|\textit{main}|}|
\end{center}
%
A simple redirection |\childdocforward{|\textit{dest}|}| is achieved by:
%
\begin{center}
|\def\jobname{|\textit{dest}|}\input{\jobname}|
\end{center}
%
The redirection with prefix
|\childdocforwardprefix[|\textit{prefix}|]{|\textit{dest}|}|
is accomplished by:
%
\begin{center}
\begin{tabular}{l}
|{\edef\jobname{\scantokens\expandafter{\jobname\noexpand}}|\\
|\def\redirectjob |\textit{prefix}|#1~~~{\gdef\jobname{|\textit{dest}|#1}}|\\
|\expandafter\redirectjob\jobname~~~}\input{\jobname}|
\end{tabular}
\end{center}

In an alternative approach,
child documents can be compiled by a specific command line
without additional code or specific definitions:
%
\begin{center}
|... -jobname "|\textit{target}|" "|[\textit{flags}]%
|\includeonly{|\textit{dest}|}\input{|\textit{main}|}"|
\end{center}
%

%%%%%%%%%%%%%%%%%%%%%%%%%%%%%%%%%%%%%%%%%%%%%%%%%%%%%%%%%%%%%%%%%%%%%%%%%%%%%%%%
%%%%%%%%%%%%%%%%%%%%%%%%%%%%%%%%%%%%%%%%%%%%%%%%%%%%%%%%%%%%%%%%%%%%%%%%%%%%%%%%
\section{Information}

%%%%%%%%%%%%%%%%%%%%%%%%%%%%%%%%%%%%%%%%%%%%%%%%%%%%%%%%%%%%%%%%%%%%%%%%%%%%%%%%
\subsection{Copyright}

Copyright \copyright{} 2017--2018 Niklas Beisert

This work may be distributed and/or modified under the
conditions of the \LaTeX{} Project Public License, either version 1.3
of this license or (at your option) any later version.
The latest version of this license is in
  \url{http://www.latex-project.org/lppl.txt}
and version 1.3 or later is part of all distributions of \LaTeX{}
version 2005/12/01 or later.

This work has the LPPL maintenance status `maintained'.

The Current Maintainer of this work is Niklas Beisert.

This work consists of the files |README.txt|, |childdoc.ins| and |childdoc.dtx|
as well as the derived files |childdoc.def|, |cdocsamp.tex|
with |cdocsch1.tex|, |cdocsch2.tex|, |cdocspt3.tex|, |cdocspt4.tex|,
|cdocsdrf.tex|, |cdocsfn1.tex|, |cdocsfn2.tex|
as well as |childdoc.pdf|.

%%%%%%%%%%%%%%%%%%%%%%%%%%%%%%%%%%%%%%%%%%%%%%%%%%%%%%%%%%%%%%%%%%%%%%%%%%%%%%%%
\subsection{Files and Installation}

The package consists of the files:
%
\begin{center}
\begin{tabular}{ll}
    |README.txt|   & readme file \\
    |childdoc.ins| & installation file \\
    |childdoc.dtx| & source file \\
    |childdoc.def| & definition file \\
    |cdocsamp.tex| & sample main file \\
    |cdocsch1.tex| & sample include file \\
    |cdocsch2.tex| & sample include file \\
    |cdocspt3.tex| & sample part file \\
    |cdocspt4.tex| & sample part file \\
    |cdocsdrf.tex| & sample redirection file \\
    |cdocsfn1.tex| & sample redirection file \\
    |cdocsfn2.tex| & sample redirection file \\
    |childdoc.pdf| & manual
\end{tabular}
\end{center}
%
The distribution consists of the files
|README.txt|, |childdoc.ins| and |childdoc.dtx|.
%
\begin{itemize}
\item
Run (pdf)\LaTeX{} on |childdoc.dtx|
to compile the manual |childdoc.pdf| (this file).
\item
Run \LaTeX{} on |childdoc.ins| to create the definitions file |childdoc.def|
and the sample |cdocsamp.tex| with include files
|cdocsch1.tex|, |cdocsch2.tex|, |cdocspt3.tex|, |cdocspt4.tex|,
|cdocsdrf.tex|, |cdocsfn1.tex|, |cdocsfn2.tex|.
Then copy the file |childdoc.def| to an appropriate directory of your \LaTeX{}
distribution, e.g.\ \textit{texmf-root}|/tex/latex/childdoc|.
\end{itemize}

%%%%%%%%%%%%%%%%%%%%%%%%%%%%%%%%%%%%%%%%%%%%%%%%%%%%%%%%%%%%%%%%%%%%%%%%%%%%%%%%
\subsection{Related CTAN Packages}

There are several other packages which offer a similar functionality:
%
\begin{itemize}
\item
The packages
\href{http://ctan.org/pkg/docmute}{\textsf{docmute}},
\href{http://ctan.org/pkg/includex}{\textsf{includex}} and
\href{http://ctan.org/pkg/standalone}{\textsf{standalone}}
provide commands to include only the document body of
a child file thus allowing both files to be compiled individually.
\item
The packages \href{http://ctan.org/pkg/subdocs}{\textsf{subdocs}}
and \href{http://ctan.org/pkg/subfiles}{\textsf{subfiles}}
provide structures in which the main and child documents can be
encapsulated and allowing them to be compiled individually.
The inclusion mechanism is different from the conventional |\include|.
\item
The package \href{http://ctan.org/pkg/combine}{\textsf{combine}}
is an elaborate solution to combine several documents into one.
\end{itemize}
%
See also the CTAN topic \href{http://ctan.org/topic/subdocs}{\textsf{subdocs}}
for further related packages.
The present package differs from the above solutions in that
a document structure constructed with the conventional |\include| mechanism
just needs two extra commands at the top of every file
such that all constituent files can be compiled individually.

%%%%%%%%%%%%%%%%%%%%%%%%%%%%%%%%%%%%%%%%%%%%%%%%%%%%%%%%%%%%%%%%%%%%%%%%%%%%%%%%
%\subsection{Feature Suggestions}
%
%The following is a list of features which may be useful for future
%versions of this package:
%%
%\begin{itemize}
%\item
%\ldots
%\end{itemize}

%%%%%%%%%%%%%%%%%%%%%%%%%%%%%%%%%%%%%%%%%%%%%%%%%%%%%%%%%%%%%%%%%%%%%%%%%%%%%%%%
\subsection{Revision History}

%%%%%%%%%%%%%%%%%%%%%%%%%%%%%%%%%%%%%%%%
\paragraph{v2.0:} 2018/12/30

\begin{itemize}
\item
immediate forward processing
\item
added |\childdocby| mechanism
\item
manual restructured
\end{itemize}

%%%%%%%%%%%%%%%%%%%%%%%%%%%%%%%%%%%%%%%%
\paragraph{v1.6:} 2018/01/17

\begin{itemize}
\item
application for development of include files
\item
corrections to manual
\end{itemize}

%%%%%%%%%%%%%%%%%%%%%%%%%%%%%%%%%%%%%%%%
\paragraph{v1.5:} 2017/05/21

\begin{itemize}
\item
more complete structuring introduced
\item
|\childdocof| introduced
\item
|\childdoc| renamed to |\childdocmain|
\item
|\childredirect| renamed to |\childdocforward| and |\childdocforwardprefix|
and functionality expanded
\end{itemize}

%%%%%%%%%%%%%%%%%%%%%%%%%%%%%%%%%%%%%%%%
\paragraph{v1.0:} 2017/04/27

\begin{itemize}
\item
manual and install package
\item
first version published on CTAN
\end{itemize}

%%%%%%%%%%%%%%%%%%%%%%%%%%%%%%%%%%%%%%%%
\paragraph{v0.6:} 2017/04/26

\begin{itemize}
\item
redirection mechanism added
\end{itemize}

%%%%%%%%%%%%%%%%%%%%%%%%%%%%%%%%%%%%%%%%
\paragraph{v0.5:} 2017/04/26

\begin{itemize}
\item
functionality in definition file
\end{itemize}


%%%%%%%%%%%%%%%%%%%%%%%%%%%%%%%%%%%%%%%%%%%%%%%%%%%%%%%%%%%%%%%%%%%%%%%%%%%%%%%%
%%%%%%%%%%%%%%%%%%%%%%%%%%%%%%%%%%%%%%%%%%%%%%%%%%%%%%%%%%%%%%%%%%%%%%%%%%%%%%%%
%%%%%%%%%%%%%%%%%%%%%%%%%%%%%%%%%%%%%%%%%%%%%%%%%%%%%%%%%%%%%%%%%%%%%%%%%%%%%%%%
\appendix

\settowidth\MacroIndent{\rmfamily\scriptsize 000\ }

 \DocInput{childdoc.dtx}

\end{document}
%</driver>
% \fi
%
% %%%%%%%%%%%%%%%%%%%%%%%%%%%%%%%%%%%%%%%%%%%%%%%%%%%%%%%%%%%%%%%%%%%%%%%%%%%%%%
% %%%%%%%%%%%%%%%%%%%%%%%%%%%%%%%%%%%%%%%%%%%%%%%%%%%%%%%%%%%%%%%%%%%%%%%%%%%%%%
% \section{Sample}
%\iffalse
%<*samplemain>
%\fi
%
% The following presents a sample document
% with two chapters, two parts, a title page,
% a compile flag as well as three forwarding files to set the flag.
% It consists of eight |.tex| files:
% \begin{center}
% \begin{tabular}{ll}
% |cdocsamp.tex|&main file\\
% |cdocsch1.tex|&include file for chapter 1\\
% |cdocsch2.tex|&include file for chapter 2\\
% |cdocspt3.tex|&include file for part 3\\
% |cdocspt4.tex|&include file for part 4\\
% |cdocsdrf.tex|&forwarding file for main file in draft mode\\
% |cdocsfi1.tex|&forwarding file for final version of chapter 1\\
% |cdocsfi2.tex|&forwarding file for final version of chapter 2\\
% \end{tabular}
% \end{center}
% Each of the eight files can be compiled directly by the \LaTeX{} compiler.
%
% %%%%%%%%%%%%%%%%%%%%%%%%%%%%%%%%%%%%%%
% \paragraph{Main File.}
%
% The main file is called |cdocsamp.tex|.
%
% Load the \textsf{childdoc} definitions and
% declare the filename for the main document:
%    \begin{macrocode}
\input{childdoc.def}
\childdocmain{}
%    \end{macrocode}

% Optional override for |\version| flag:
%    \begin{macrocode}
%%\ifchilddoc\else\providecommand{\version}{draft}\fi
%    \end{macrocode}

% Define the default values for the |\version| flag
% (|final| for the main file and |draft| for childs):
%    \begin{macrocode}
\ifchilddoc
\providecommand{\version}{draft}
\else
\providecommand{\version}{final}
\fi
%    \end{macrocode}

% Load the standard document class:
%    \begin{macrocode}
\documentclass[12pt]{article}
%    \end{macrocode}

% Start the document body:
%    \begin{macrocode}
\begin{document}
%    \end{macrocode}

% Declare a title page.
% Print title, part of document being processed and version flag:
%    \begin{macrocode}
\addtocounter{page}{-1}
\begin{center}
{\LARGE\bfseries{}childdoc example\par}
\vspace{1cm}
\ifchilddoc
\ifchilddocmanual part\else chapter\fi:
`\childdocname' of `\childdocjob'\par
\else
main document: `\childdocjob'\par
\fi
version: \version\par
\end{center}
\newpage
%    \end{macrocode}

% Manually include selected file,
% otherwise process as usual:
%    \begin{macrocode}
\ifchilddocmanual
\section*{part `\childdocname'}
\input{\childdocname}
\else
%    \end{macrocode}

% Include the two chapters:
%    \begin{macrocode}
\include{cdocsch1}
\include{cdocsch2}
%    \end{macrocode}

% Include the two parts unless only chapters should be displayed:
%    \begin{macrocode}
\ifchilddoc\else
\section{part three}
\input{cdocspt3}
\section{part four}
\input{cdocspt4}
\fi
%    \end{macrocode}

% Process as usual until here:
%    \begin{macrocode}
\fi
%    \end{macrocode}

% End of document body:
%    \begin{macrocode}
\end{document}
%    \end{macrocode}
%\iffalse
%</samplemain>
%\fi
%
% %%%%%%%%%%%%%%%%%%%%%%%%%%%%%%%%%%%%%%
% \paragraph{Chapter Include Files.}
%
% The include files are called |cdocsch1.tex| and |cdocsch2.tex|.
%
%\iffalse
%<*samplechap1|samplechap2>
%\fi

% Optional override for |\version| flag:
%    \begin{macrocode}
%%\providecommand{\version}{final}
%    \end{macrocode}

% Include the main document:
%    \begin{macrocode}
\input{childdoc.def}
\childdocof{cdocsamp}
%    \end{macrocode}

%\iffalse
%</samplechap1|samplechap2>
%\fi
%
%\iffalse
%<*samplechap1>
%\fi
% Some text for chapter 1:
%    \begin{macrocode}
\section{one}
some text in chapter one
%    \end{macrocode}

%\iffalse
%</samplechap1>
%\fi
% Some text for chapter 2:
%\iffalse
%<*samplechap2>
%\fi
%    \begin{macrocode}
\section{two}
more text in chapter two
%    \end{macrocode}

%\iffalse
%</samplechap2>
%\fi
%
% %%%%%%%%%%%%%%%%%%%%%%%%%%%%%%%%%%%%%%
% \paragraph{Part Include Files.}
%
% The include files are called |cdocspt3.tex| and |cdocspt4.tex|.
%
%\iffalse
%<*samplepart3|samplepart4>
%\fi

% Optional override for |\version| flag:
%    \begin{macrocode}
%%\providecommand{\version}{final}
%    \end{macrocode}

% Include the main document:
%    \begin{macrocode}
\input{childdoc.def}
\childdocby{cdocsamp}
%    \end{macrocode}

%\iffalse
%</samplepart3|samplepart4>
%\fi
%
%\iffalse
%<*samplepart3>
%\fi
% Some text for part 3:
%    \begin{macrocode}
some text in part three
%    \end{macrocode}

%\iffalse
%</samplepart3>
%\fi
% Some text for part 4:
%\iffalse
%<*samplepart4>
%\fi
%    \begin{macrocode}
more text in part four
%    \end{macrocode}

%\iffalse
%</samplepart4>
%\fi
%
% %%%%%%%%%%%%%%%%%%%%%%%%%%%%%%%%%%%%%%
% \paragraph{Forwarding for a Complete Draft.}
%
% The following forwarding file |cdocsdrf.tex|
% compiles the main document in draft mode:
%\iffalse
%<*sampledraft>
%\fi
%    \begin{macrocode}
\def\version{draft}
\input{childdoc.def}
\childdocforward{cdocsamp}
%    \end{macrocode}

%\iffalse
%</sampledraft>
%\fi
%
% %%%%%%%%%%%%%%%%%%%%%%%%%%%%%%%%%%%%%%
% \paragraph{Forwarding for Final Version of the Chapters.}
%
% The following forwarding files |cdocsfn1.tex| and |cdocsfn2.tex|
% (with identical content)
% compile the final versions of the child documents
% |cdocsch1.tex| and |cdocsch2.tex|, respectively:
%\iffalse
%<*samplefinal>
%\fi
%    \begin{macrocode}
\def\version{final}
\input{childdoc.def}
\childdocforwardprefix[cdocsamp]{cdocsfn}{cdocsch}
%    \end{macrocode}

%\iffalse
%</samplefinal>
%\fi
%
% %%%%%%%%%%%%%%%%%%%%%%%%%%%%%%%%%%%%%%
% \paragraph{Command Line Processing.}
%
% The following three command lines generate the output files
% |cdocscld|, |cdocscl1| and |cdocscl2|
% which should be identical to
% |cdocsdrf|, |cdocsch1| and |cdocsfn2|, respectively:
% \begin{center}
% \begin{tabular}{l}
% |latex -jobname cdocscld \|\\
% |  "\def\version{draft}\input{childdoc.def}\childdocforward{cdocsamp}"|\\
% |latex -jobname cdocscl1 \|\\
% |  "\input{childdoc.def}\childdocforward[cdocsamp]{cdocsch1}"|\\
% |latex -jobname cdocscl2 \|\\
% |  "\def\version{final}\input{childdoc.def}\childdocforward{cdocsch2}"|
% \end{tabular}
% \end{center}
% Note that the trailing backslash on each first line
% merely continues the input to the second line
% (for convenient cut ant paste).
% Furthermore, the command |latex| can be replaced by any
% of its alternative versions such as |pdflatex|.
%
% %%%%%%%%%%%%%%%%%%%%%%%%%%%%%%%%%%%%%%%%%%%%%%%%%%%%%%%%%%%%%%%%%%%%%%%%%%%%%%
% %%%%%%%%%%%%%%%%%%%%%%%%%%%%%%%%%%%%%%%%%%%%%%%%%%%%%%%%%%%%%%%%%%%%%%%%%%%%%%
% \section{Implementation}
%\iffalse
%<*package>
%\fi
%
% This section describes the definitions file |childdoc.def|.

% The definitions cannot be loaded using |\usepackage| or |\RequirePackage|
% which has a mechanism to prevent loading a style file more than once.
% When loading the definitions by means of |\input|
% multiple instances have to be prevented manually:
%\iffalse
%This code needs to be before the `\ProvidesFile' directive
%which is defined at the beginning of this file.
%Therefore it is also placed there and commented out here.
%</package>
%<*discard>
%\fi
%    \begin{macrocode}
\ifdefined\childdocmain\endinput\fi
%    \end{macrocode}
%\iffalse
%</discard>
%<*package>
%\fi
%
% \macro{\ifchilddoc}
% \macro{\ifchilddocmanual}
% The conditional |\ifchilddoc| tells whether a
% child (true) or main (false) document is being compiled.
% The conditional |\ifchilddocmanual| tells whether
% the |\includeonly| mechanism is used (false) or
% the selection of child files must be performed manually (true).
% The definitions initialise to false:
%    \begin{macrocode}
\newif\ifchilddoc
\newif\ifchilddocmanual
%    \end{macrocode}

% \macro{\childdocname}
% \macro{\childdocjob}
% The macro |\childdocname| stores the name of the main document
% to be compiled. The macro |\childdocjob| stores the name of
% the document on which the \LaTeX{} compiler was originally invoked.
% The content of |\jobname| cannot be compared
% to filenames specified in the source due to different catcodes.
% The following code rescans |\jobname|, stores the result
% in |\childdocname| and saves a copy in |\childdocjob|:
%    \begin{macrocode}
\edef\childdocname{\scantokens\expandafter{\jobname\noexpand}}
\let\childdocjob\childdocname
%    \end{macrocode}

% \macro{\childdocdisable}
% The macro |\childdocdisable| prevents the main file
% from being processed more than once.
% At this stage, the main document command |\childdocmain|
% is assumed to be called once again where it should do nothing.
% Any subsequent call to it should prevent
% a secondary processing of the main document
% It overwrites the forwarding commands
% |\childdocof| and |\childdocforward|
% with empty macros to prevent further inclusions of the main document:
%    \begin{macrocode}
\newcommand{\childdocdisable}
{
  \renewcommand{\childdocmain}[1]{\renewcommand{\childdocmain}[1]{\endinput}}
  \renewcommand{\childdocof}[1]{}
  \renewcommand{\childdocby}[2][]{}
  \renewcommand{\childdocforward}[2][]{}
  \renewcommand{\childdocdisable}{}
}
%    \end{macrocode}

% \macro{\childdocmain}
% The macro |\childdocmain| is to be called at the top of the main file
% with nothing or the main filename (without extension) as argument.
% First, it breaks loops.
% If the argument is not empty and does not match |\childdocname|
% (which is set by the first inclusion of |childdoc.def|),
% |\ifchilddoc| is set to true, |\includeonly| is applied to the child file
% and |\jobname| is set to the main file
% (for proper handling of |.aux| files):
%    \begin{macrocode}
\newcommand{\childdocmain}[1]
{
  \childdocdisable\childdocmain{}
  \if?#1?\else
    \begingroup
      \def\childdoctmp{#1}
      \ifx\childdoctmp\childdocname
        \def\childdoctmp{}
      \else
        \def\childdoctmp
        {
          \childdoctrue
          \includeonly{\childdocname}
          \def\childdocjob{#1}
          \def\jobname{#1}
        }
      \fi
      \expandafter
    \endgroup
    \childdoctmp
  \fi
}
%    \end{macrocode}

% \macro{\childdocof}
% The command |\childdocof| redirects
% compilation to the main file |#1|.
%    \begin{macrocode}
\newcommand{\childdocof}[1]
{
  \childdocdisable
  \childdoctrue
  \includeonly{\childdocname}
  \def\jobname{#1}
  \def\childdocjob{#1}
  \input{#1}
}
%    \end{macrocode}

% \macro{\childdocby}
% The command |\childdocby| ....
%    \begin{macrocode}
\newcommand{\childdocby}[2][]
{
  \childdocdisable
  \childdoctrue
  \childdocmanualtrue
  \if?#1?\else
    \def\jobname{#2}
  \fi
  \def\childdocjob{#2}
  \input{#2}
  \endinput
}
%    \end{macrocode}

% \macro{\childdocforward}
% The command |\childdocforward| redirects
% compilation to the main file or
% (if the optional argument is given) a child file.
% Parameters are set as if the main file
% or a child file starting with |\childdocof| was compiled.
% Then compilation is handed over to the main file:
%    \begin{macrocode}
\newcommand{\childdocforward}[2][]
{
  \begingroup
    \if?#1?
      \def\childdoctmp
      {
        \def\childdocname{#2}
        \def\childdocjob{#2}
        \def\jobname{#2}
        \input{#2}
        \endinput
      }
    \else
      \def\childdoctmp
      {
        \childdocdisable
        \def\childdocname{#2}
        \childdoctrue
        \includeonly{#2}
        \def\childdocjob{#1}
        \def\jobname{#1}
        \input{#1}
        \endinput
      }
    \fi
    \expandafter
  \endgroup
  \childdoctmp
}
%    \end{macrocode}

% \macro{\childdocforwardprefix}
% The command |\childdocforwardprefix| redirects
% compilation to the main or a child file by means of a pattern.
% The prefix |#1| in the current filename is replaced by |#2|
% and the suffix of the current filename is kept
% (it is assumed that the filename does not contain the substring `|~~~|'
% which is used as a delimiter).
% Compilation is handed over to the new file by |\childdocforward|:
%    \begin{macrocode}
\newcommand{\childdocforwardprefix}[3][]
{
  \begingroup
    \def\childdocextract #2##1~~~{\def\childdoctmp{\childdocforward[#1]{#3##1}}}
    \expandafter\childdocextract\childdocname~~~
    \expandafter
  \endgroup
  \childdoctmp
}
%    \end{macrocode}

% \macro{\childdoc}
% The deprecated macro |\childdoc| is a legacy version of |\childdocmain|:
%    \begin{macrocode}
\newcommand{\childdoc}{\childdocmain}
%    \end{macrocode}

% \macro{\childdocredirect}
% The deprecated macro |\childdocredirect| is a legacy version
% of |\childdocforward| and |\childdocforwardprefix|:
%    \begin{macrocode}
\newcommand{\childdocredirect}[2][]
{
  \begingroup
    \if?#1?
      \def\childdoctmp{\childdocforward{#2}}
    \else
      \def\childdoctmp{\childdocforwardprefix{#1}{#2}}
    \fi
    \expandafter
  \endgroup
  \childdoctmp
}
%    \end{macrocode}

%\iffalse
%</package>
%\fi
%
\endinput

\childdocof{cdocsamp}
%    \end{macrocode}

%\iffalse
%</samplechap1|samplechap2>
%\fi
%
%\iffalse
%<*samplechap1>
%\fi
% Some text for chapter 1:
%    \begin{macrocode}
\section{one}
some text in chapter one
%    \end{macrocode}

%\iffalse
%</samplechap1>
%\fi
% Some text for chapter 2:
%\iffalse
%<*samplechap2>
%\fi
%    \begin{macrocode}
\section{two}
more text in chapter two
%    \end{macrocode}

%\iffalse
%</samplechap2>
%\fi
%
% %%%%%%%%%%%%%%%%%%%%%%%%%%%%%%%%%%%%%%
% \paragraph{Part Include Files.}
%
% The include files are called |cdocspt3.tex| and |cdocspt4.tex|.
%
%\iffalse
%<*samplepart3|samplepart4>
%\fi

% Optional override for |\version| flag:
%    \begin{macrocode}
%%\providecommand{\version}{final}
%    \end{macrocode}

% Include the main document:
%    \begin{macrocode}
% \iffalse
%
% childdoc.dtx Copyright (C) 2017-2018 Niklas Beisert
%
% This work may be distributed and/or modified under the
% conditions of the LaTeX Project Public License, either version 1.3
% of this license or (at your option) any later version.
% The latest version of this license is in
%   http://www.latex-project.org/lppl.txt
% and version 1.3 or later is part of all distributions of LaTeX
% version 2005/12/01 or later.
%
% This work has the LPPL maintenance status `maintained'.
%
% The Current Maintainer of this work is Niklas Beisert.
%
% This work consists of the files childdoc.dtx and childdoc.ins
% and the derived files childdoc.def and cdocsamp.tex with
% cdocsch1.tex, cdocsch2.tex, cdocsdrf.tex, cdocsfn1.tex, cdocsfn2.tex.
%
%<package>\ifdefined\childdocmain\endinput\fi
%<package>\ProvidesFile{childdoc.def}[2018/12/30 v2.0 child document driver]
%<samplemain>\ProvidesFile{cdocsamp.tex}[2018/12/30 v2.0 sample for childdoc]
%<*driver>
%\ProvidesFile{childdoc.drv}[2018/12/30 v2.0 childdoc reference manual file]
\PassOptionsToClass{10pt,a4paper}{article}
\documentclass{ltxdoc}

\usepackage[margin=35mm]{geometry}
\usepackage{hyperref}
\usepackage{hyperxmp}
\usepackage[usenames]{color}

\hypersetup{colorlinks=true}
\hypersetup{pdfstartview=FitH}
\hypersetup{pdfpagemode=UseNone}
\hypersetup{pdfsource={}}
\hypersetup{pdflang={en-UK}}
\hypersetup{pdfcopyright={Copyright 2017-2018 Niklas Beisert.
  This work may be distributed and/or modified under the
  conditions of the LaTeX Project Public License, either version 1.3
  of this license or (at your option) any later version.}}
\hypersetup{pdflicenseurl={http://www.latex-project.org/lppl.txt}}
\hypersetup{pdfcontactaddress={ETH Zurich, ITP, HIT K,
  Wolfgang-Pauli-Strasse 27}}
\hypersetup{pdfcontactpostcode={8093}}
\hypersetup{pdfcontactcity={Zurich}}
\hypersetup{pdfcontactcountry={Switzerland}}
\hypersetup{pdfcontactemail={nbeisert@itp.phys.ethz.ch}}
\hypersetup{pdfcontacturl={http://people.phys.ethz.ch/\xmptilde nbeisert/}}

\newcommand{\secref}[1]{\hyperref[#1]{section \ref*{#1}}}

\parskip1ex
\parindent0pt
\let\olditemize\itemize
\def\itemize{\olditemize\parskip0pt}

\begin{document}

\title{The \textsf{childdoc} Package}
\hypersetup{pdftitle={The childdoc Package}}
\author{Niklas Beisert\\[2ex]
  Institut f\"ur Theoretische Physik\\
  Eidgen\"ossische Technische Hochschule Z\"urich\\
  Wolfgang-Pauli-Strasse 27, 8093 Z\"urich, Switzerland\\[1ex]
  \href{mailto:nbeisert@itp.phys.ethz.ch}
  {\texttt{nbeisert@itp.phys.ethz.ch}}}
\hypersetup{pdfauthor={Niklas Beisert}}
\hypersetup{pdfsubject={Manual for the LaTeX2e Package childdoc}}
\date{30 December 2018, \textsf{v2.0}}
\maketitle

\begin{abstract}\noindent
\textsf{childdoc} is a \LaTeXe{} package
that enables the direct compilation
of document sections included by |\include|
to individual files.
\end{abstract}

\begingroup
\parskip0ex
\tableofcontents
\endgroup

%%%%%%%%%%%%%%%%%%%%%%%%%%%%%%%%%%%%%%%%%%%%%%%%%%%%%%%%%%%%%%%%%%%%%%%%%%%%%%%%
%%%%%%%%%%%%%%%%%%%%%%%%%%%%%%%%%%%%%%%%%%%%%%%%%%%%%%%%%%%%%%%%%%%%%%%%%%%%%%%%
\section{Introduction}

\LaTeX{} provides a mechanism to structure a large document (such as a book)
into a main file and several child files (containing the chapters)
using the |\include| command.
This mechanism is beneficial for documents
which span hundreds of pages in order to
make the source file(s) more manageable.
Moreover, compilation can be restricted to
selected child files by means of the |\includeonly| command.
The latter feature can be used to reduce the compilation time while editing
(this was significantly more useful in the earlier days of \LaTeX{})
or to generate a smaller document which is easier to navigate.
Another application of |\includeonly| is to generate
documents consisting of selected parts of the complete document.

However, there are a few drawbacks of the plain |\include| mechanism:
\begin{itemize}
\item
The child files cannot be compiled on their own,
they can only be compiled via the main file.
A naive editing environment
(such as a text editor with an option
to have the current file processed by \LaTeX)
may require one to switch to the main file before compiling;
attempting to compile the child file produces errors.
\item
The main file must be modified (each time)
to adjust the |\includeonly| command
to the present needs. This easily leaves the main file in a messy state.
\item
The generated document will always carry the filename
of the main document. This is inconvenient if
several child files are to be compiled and
to be kept for distribution.
\end{itemize}

The present package provides a simple interface
to make child files individually compilable by \LaTeX{}.
Compiling a child file then has the same effect as compiling
the main file with an |\includeonly| command
to select the appropriate child.
Moreover the generated document will carry the name of the child
rather than the main file.
This resolves all three above issues.

This feature is meant to make the editing of books,
thesis documents and lecture notes somewhat more convenient.
However, the package can also be used efficiently for
composing a series of documents (such as exercise sheets)
which are typically distributed individually.
It then assists the author in generating the individual documents
(potentially in different versions)
as well as a document containing the collected series.
Another application is in developing style files
or other kinds of included material
where compilation of the style file could redirect
to a sample or test file.

%%%%%%%%%%%%%%%%%%%%%%%%%%%%%%%%%%%%%%%%%%%%%%%%%%%%%%%%%%%%%%%%%%%%%%%%%%%%%%%%
%%%%%%%%%%%%%%%%%%%%%%%%%%%%%%%%%%%%%%%%%%%%%%%%%%%%%%%%%%%%%%%%%%%%%%%%%%%%%%%%
\section{Usage}

First of all, the package \textsf{childdoc} is \emph{not} a standard
\LaTeXe{} |.sty| style file! Therefore it needs to be invoked in
a non-standard way.

%%%%%%%%%%%%%%%%%%%%%%%%%%%%%%%%%%%%%%%%%%%%%%%%%%%%%%%%%%%%%%%%%%%%%%%%%%%%%%%%
\subsection{Included Files}
\label{sec:include}

%%%%%%%%%%%%%%%%%%%%%%%%%%%%%%%%%%%%%%%%
\DescribeMacro{\childdocmain}
To use the package, add the commands
\begin{center}
\begin{tabular}{l}
|\input{childdoc.def}|\\
|\childdocmain{}|\\
\end{tabular}
\end{center}
at the very top of the main \LaTeX{} file,
in particular \emph{before} the |\documentclass| statement!
The argument of |\childdocmain| should be left empty
(but it must be present).

%%%%%%%%%%%%%%%%%%%%%%%%%%%%%%%%%%%%%%%%
\DescribeMacro{\childdocof}
Furthermore, add the commands
\begin{center}
\begin{tabular}{l}
|\input{childdoc.def}|\\
|\childdocof{|\textit{main}|}|\\
\end{tabular}
\end{center}
at the top of every child file \textit{child}
which is included by |\include{|\textit{child}|}|
from within the main file
(or at least for those files to be compiled individually).
The argument \textit{main} must be the filename of the main file.

There are a couple of
considerations in setting up the main and child documents:

%%%%%%%%%%%%%%%%%%%%%%%%%%%%%%%%%%%%%%%%
\paragraph{Restrictions.}

Please note the following restrictions:
\begin{itemize}
\item
|\childdocmain| must be called with one argument \textit{main}
to ensure compatibility with earlier version of the package.
It must either be empty (|\childdocmain{}|)
or precisely match the filename of the main file in which it is specified.
See \secref{sec:detection} for further information.
\item
The filename \textit{main} must be specified without the |.tex| extension.
\item
The filename \textit{main} is case sensitive
(even in case-insensitive file systems)
due to internal string comparison.
\item
The argument \textit{main} should be fully expanded, it cannot be a macro.
\item
Subdirectories and special characters should be avoided in filenames.
\item
The command |\childdocmain{|\textit{main}|}| must be followed by a whitespace.
It should not be followed immediately by another command
or by a comment mark `|%|'.
This is because the \TeX{} parser reads the token immediately following
the argument of |\childdocmain| and puts it
at the beginning of every child section;
however, a white\-space is ignored.
\end{itemize}

%%%%%%%%%%%%%%%%%%%%%%%%%%%%%%%%%%%%%%%%
\paragraph{Content of Main File.}

It is advisable to place all content in the child files included by |\include|.
Any output contained in the main file will appear in all child documents
unless suppressed manually;
it cannot be suppressed automatically by the |\includeonly| directive
and thus should normally be avoided.
A method to include some content in the main file
by means of conditional processing is described in \secref{sec:conditional}.

%%%%%%%%%%%%%%%%%%%%%%%%%%%%%%%%%%%%%%%%
\paragraph{Page Numbering.}

When only a part of the document is compiled,
the appropriate numbering of pages
(as well as other status parameters)
is determined from the |.aux| files.
The latter contain information from previous passes.
However this information needs to propagate through
all intermediate child documents.
Therefore the page numbering in child documents may well
be inconsistent until the complete document is compiled at least once.

A useful (if unconventional) way to always ensure a consistent
page numbering is to restart the numbering in each child document
and denote the pages by `\textit{child}|.|\textit{page}'
where \textit{child} represents the chapter/section number of the child file.
This can be achieved by the command
|\numberwithin{page}{|\textit{child}|}|
of the \textsf{amsmath} package
where \textit{child} can be |chapter| or |section|
depending on the chosen structuring.
Alternatively, one can modify the macro |\thepage| appropriately
and reset the counter |page| at the start of each child file.

%%%%%%%%%%%%%%%%%%%%%%%%%%%%%%%%%%%%%%%%%%%%%%%%%%%%%%%%%%%%%%%%%%%%%%%%%%%%%%%%
\subsection{Conditional Processing}
\label{sec:conditional}

The package provides a mechanism to compile different versions
of a document. To customise the versions further some conditional processing
can come in handy to distinguish which version is being compiled.
The package provides two macros to describe the compilation context:

%%%%%%%%%%%%%%%%%%%%%%%%%%%%%%%%%%%%%%%%
\DescribeMacro{\ifchilddoc}
The conditional |\ifchilddoc| distinguishes between the compilation of
child documents and the main document:
%
\begin{center}
|\ifchilddoc |\textit{child-code}| |[|\||else |\textit{main-code}]| \||fi|
\end{center}

%%%%%%%%%%%%%%%%%%%%%%%%%%%%%%%%%%%%%%%%
\DescribeMacro{\childdocname}
\DescribeMacro{\childdocjob}
The macro |\childdocname| contains the filename (without extension)
of the main or child file being processed.
Note that |\childdocjob| will always contain the name of the main file.

%%%%%%%%%%%%%%%%%%%%%%%%%%%%%%%%%%%%%%%%
\paragraph{Title Page.}

Conditional processing can be used to include a title or banner page
in the main document when proper precautions are taken.
Importantly, the code in the main file should ensure that the page counter
(as well as other status parameters which are stored in the |.aux| files)
takes the same value after the conditional processing.
Otherwise the page numbers may take divergent values
depending on which part is compiled.

For example, a title page could be declared by:
%
\begin{center}
\begin{tabular}{l}
|\ifchilddoc\||else|\\
|\addtocounter{page}{-1}|\\
\textit{code for title page}\\
|\newpage|\\
|\||fi|
\end{tabular}
\end{center}
%
A banner page for the child documents can be generated by:
%
\begin{center}
\begin{tabular}{l}
|\ifchilddoc|\\
|\addtocounter{page}{-1}|\\
\textit{code for banner page}\\
|\newpage|\\
|\||fi|
\end{tabular}
\end{center}
%
Here one could write a message such as:
\begin{center}
|This is the part \childdocname{} of \childdocjob{}.|
\end{center}

%%%%%%%%%%%%%%%%%%%%%%%%%%%%%%%%%%%%%%%%%%%%%%%%%%%%%%%%%%%%%%%%%%%%%%%%%%%%%%%%
\subsection{Flags}
\label{sec:flags}

The package makes it easy to generate different versions
of the main or child documents.
To this end compilation flags can be defined
and assigned different default values.
They will be particularly useful in conjunction
with the forwarding mechanism described in \secref{sec:forward}.

For example, it may be useful to have a flag |\version|
which can be set to |draft| or |final|.
The document source will contain some conditional code
depending on the value of |\version|.
Suppose further, the flag should default to |final| for the main file
and to |draft| for child files
which is a natural assignment for editing the document.
This is achieved by placing the following code
in the preamble of the main document
(below the |\childdocmain| directive):
%
\begin{center}
\begin{tabular}{l}
|\ifchilddoc|\\
|\providecommand{\version}{draft}|\\
|\||else|\\
|\providecommand{\version}{final}|\\
|\||fi|
\end{tabular}
\end{center}
%
The definition by |\providecommand| makes sure
that previous definitions are not overwritten.
Further statements |\providecommand{\version}{...}|
can thus be added before the above code to override it.

For the main file, one might add a line
(between |\childdocmain| and the above block)
%
\begin{center}
|%\ifchilddoc\||else\providecommand{\version}{draft}\||fi|
\end{center}
%
which can be uncommented to produce a draft version.
Likewise one can add a line to the very top of a child file
(above the |\childdocof{|\textit{main}|}| directive)
%
\begin{center}
|%\providecommand{\version}{final}|
\end{center}
%
which can be uncommented to produce the final version of this child document.

%%%%%%%%%%%%%%%%%%%%%%%%%%%%%%%%%%%%%%%%%%%%%%%%%%%%%%%%%%%%%%%%%%%%%%%%%%%%%%%%
\subsection{Forwarding}
\label{sec:forward}

Different versions of the main or child documents
using compilation flags as described in \secref{sec:flags}
can be (permanently) stored in different files
for convenient compilation, viewing and distribution.
To this end, the package defines a command
to pass on compilation to a different file:

%%%%%%%%%%%%%%%%%%%%%%%%%%%%%%%%%%%%%%%%
\DescribeMacro{\childdocforward}
The command |\childdocforward| redirects processing to
another source file:
%
\begin{center}
\begin{tabular}{l}
|\input{childdoc.def}|\\
|\childdocforward[|\textit{main}|]{|\textit{dest}|}|\\
\end{tabular}
\end{center}
%
The argument \textit{dest} is the destination file
(without extension).
It should be the main file or one of the child files.
Note that further \textsf{childdoc} directives
such as |\childdocof| and |\childdocforward|
in the indicated file will be processed in this form.
The optional argument \textit{main}
passes on directly to the main file \textit{main}
while pretending to compile the child \textit{dest}.
This form behaves as if \textit{dest}
issues |\childdocof{|\textit{main}|}| right away,
and no further \textsf{childdoc} directives will be processed.

%%%%%%%%%%%%%%%%%%%%%%%%%%%%%%%%%%%%%%%%
\DescribeMacro{\...prefix}
In the alternative form |\childdocforwardprefix|,
%
\begin{center}
\begin{tabular}{l}
|\input{childdoc.def}|\\
|\childdocforwardprefix[|\textit{main}|]{|\textit{prefix}|}{|\textit{dest}|}|
\end{tabular}
\end{center}
%
the destination file is determined by a pattern
depending on the current file:
To make this work, the current file must be called
`{\textit{prefix}\hspace{0.2em}\textit{suffix}}'
with \textit{prefix} matching precisely the argument.
Processing is then passed on to the file
`{\textit{dest}\hspace{0.2em}\textit{suffix}}'.
Surely, the same effect is achieved by
directly specifying the
argument `{\textit{dest}\hspace{0.2em}\textit{suffix}}'
in the first form.
However, that requires to set up a different file
for each child. With the alternative form of the command
all these files can have exactly the same content
which simplifies setting them up and maintaining them.

For example, the following file |draft.tex|
with a compilation flag |\version| as described in \secref{sec:flags}
compiles the main document as a draft:
%
\begin{center}
\begin{tabular}{l}
|\def\version{draft}|\\
|\input{childdoc.def}|\\
|\childdocforward{|\textit{main}|}|
\end{tabular}
\end{center}
%
Likewise, the following files |final|\textit{nn}|.tex|
compile the final version of the child document
|child|\textit{nn}|.tex|:
%
\begin{center}
\begin{tabular}{l}
|\def\version{final}|\\
|\input{childdoc.def}|\\
|\childdocforwardprefix{final}{child}|
\end{tabular}
\end{center}
%

Note that when several versions of a main file and/or of each child file
are to be generated, it may be convenient to set up a |Makefile| or
shell script to automatise the process.

%%%%%%%%%%%%%%%%%%%%%%%%%%%%%%%%%%%%%%%%%%%%%%%%%%%%%%%%%%%%%%%%%%%%%%%%%%%%%%%%
\subsection{Command Line Processing}
\label{sec:commandline}

The effect of redirection files can also be achieved by invoking
the \LaTeX{} compiler with a more elaborate command line.
Most conveniently this should be done as part
of a shell script or a |Makefile|.

When using \textsf{childdoc} in the main file, the following
command lines effectively perform a redirection
(note that depending on the shell being used,
backslashes may have to be doubled: `|\|' $\to$ `|\\|'):
%
\begin{center}
|... -jobname "|\textit{target}|" |\\|"|[\textit{flags}]%
|\input{childdoc.def}\childdocforward[|\textit{main}|]{|\textit{dest}|}"|
\end{center}
%
Here \textit{target} is the name of the output file,
\textit{main} is the name of the main file
and \textit{dest} is the name of the main or child file to be processed
(all filenames without extensions).
The optional argument \textit{main} can be omitted
if \textit{main} matches \textit{dest}.
Optionally, compilation \textit{flags} can be defined via |\def| commands.
This command line makes the \TeX{} engine believe
it is compiling the file \textit{target}
whose content is specified as the latter parameter.
The provided code then forwards the processing to
\textit{main} or \textit{dest} as described in \secref{sec:forward}.

%%%%%%%%%%%%%%%%%%%%%%%%%%%%%%%%%%%%%%%%%%%%%%%%%%%%%%%%%%%%%%%%%%%%%%%%%%%%%%%%
\subsection{Include by Input}
\label{sec:input}

Including child documents by |\include| has some restrictions by design.
Most notably, the content of a child document always occupies
its own set of pages; pages cannot be shared between child documents.
Usually, this behaviour makes perfect sense
because each child document contain an essential part of the document.
However, in some situations it may be desirable to compose
a document from a collection of parts
without having mandatory page breaks between then.
For this case, the package
provides a mechanism to include parts
by |\input| which can also be processed individually.
However, by construction this mechanism
requires manual handling of the content to be output.

%%%%%%%%%%%%%%%%%%%%%%%%%%%%%%%%%%%%%%%%
\DescribeMacro{\ifchilddocmanual}
The main file should be prepared as usual, see \secref{sec:include}.
However, the document body must make a distinction
between processing of an individual part and of the main document, e.g.:
%
\begin{center}
\begin{tabular}{l}
|\ifchilddocmanual|\\
|\input{\childdocname}|\\
|\||else|\\
\textit{document body with }|\input{|\textit{part}|}|\\
|\||fi|
\end{tabular}
\end{center}
%
The conditional |\ifchilddocmanual| is true whenever
a part to be included by |\input| is being compiled,
and the name of the part is stored in |\childdocname|.

%%%%%%%%%%%%%%%%%%%%%%%%%%%%%%%%%%%%%%%%
\DescribeMacro{\childdocby}
Each part to be included by |\input| should start with:
%
\begin{center}
\begin{tabular}{l}
|\input{childdoc.def}|\\
|\childdocby{|\textit{main}|}|\\
\end{tabular}
\end{center}
%
The directive |\childdocby| is similar to |\childdocof|
described in \secref{sec:include},
but the subsequent selection of content must be done manually.
To that end, both |\ifchilddoc| and |\ifchilddocmanual|
will be true upon processing of a part,
and the name of the part is stored in |\childdocname|.
Note that |\jobname| will be set to the filename of the current part
so that each part receives an individual |.aux| file
that does not interfere with the |.aux| file(s) of the main document.
This behaviour can be altered by the alternative form
|\childdocby[*]{|\textit{main}|}| (with a non-empty optional argument)
which uses the |.aux| file of the main document
by setting |\jobname| to \textit{main}.

%%%%%%%%%%%%%%%%%%%%%%%%%%%%%%%%%%%%%%%%%%%%%%%%%%%%%%%%%%%%%%%%%%%%%%%%%%%%%%%%
\subsection{Driver Development}
\label{sec:driver}

The \textsf{childdoc} mechanism can also be use for the development
of definition files such as \LaTeX{} styles or classes.
This case differs from the above setup with multiple parts
included by |\include| in that no |\includeonly| should be invoked.
This can be achieved by starting the include file
(before |\ProvidesPackage|) with:
%
\begin{center}
\begin{tabular}{l}
|\input{childdoc.def}|\\
|\childdocforward{|\textit{main}|}|\\
\end{tabular}
\end{center}
%
or alternatively with:
%
\begin{center}
\begin{tabular}{l}
|\input{childdoc.def}|\\
|\childdocby{|\textit{main}|}|\\
\end{tabular}
\end{center}
%
Both forms have slightly different effects as described above.
The main file is prepared as usual, see \secref{sec:include}.

%%%%%%%%%%%%%%%%%%%%%%%%%%%%%%%%%%%%%%%%%%%%%%%%%%%%%%%%%%%%%%%%%%%%%%%%%%%%%%%%
\subsection{Legacy Detection}
\label{sec:detection}

The directive |\childdocmain| in the main file can detect
whether the complete document or merely a child is to be compiled
even without using the directive |\childdocof|.
This method is deprecated because it is less robust
and there is no compelling reason to use it;
it is merely provided for backward compatibility
and it may be removed in future versions.

If the detection mechanism is to be used,
it is mandatory to correctly specify
the filename of the main file as the argument of |\childdocmain|:
%
\begin{center}
\begin{tabular}{l}
|\input{childdoc.def}|\\
|\childdocmain{|\textit{main}|}|\\
\end{tabular}
\end{center}
%
If |\jobname| does not match the argument \textit{main} of |\childdocmain|,
it is assumed that |\jobname| points to the child file to be compiled.
When using |\childdocmain| with the main file specified as argument,
it suffices to start a child file
with just |\input{|\textit{main}|}|
without loading of the package and using |\childdocof|.
If instead all processing is done
with the appropriate \textsf{childdoc} directives,
the argument of \textit{main} of |\childdocmain| can be empty.

An alternative version of the command line processing described
in \secref{sec:commandline} using the detection mechanism reads:
%
\begin{center}
|... -jobname "|\textit{target}|" "|[\textit{flags}]%
[|\def\jobname{|\textit{dest}|}|]|\input{|\textit{main}|}"|
\end{center}

%%%%%%%%%%%%%%%%%%%%%%%%%%%%%%%%%%%%%%%%%%%%%%%%%%%%%%%%%%%%%%%%%%%%%%%%%%%%%%%%
\subsection{Manual Code}
\label{sec:manual}

In case one cannot be certain whether the definitions file |childdoc.def|
is installed on the target \TeX{} distribution
and one prefers not to ship it,
it is conceivable to paste a few relevant commands into the sources.

To that end, drop all statements |\input{childdoc.def}|
and perform the replacements as outlined below.
Instead of |\childdocmain{|\textit{main}|}| add the following code
to the top of the main file:
%
\begin{center}
\begin{tabular}{l}
|\||ifdefined\childdocname\endinput\||fi\newif\ifchilddoc|\\
|\edef\childdocname{\scantokens\expandafter{\jobname\noexpand}}|\\
|\def\childdocmain{|\textit{main}|}\||ifx\childdocmain\childdocname\||else|\\
|\childdoctrue\includeonly{\childdocname}\let\jobname\childdocmain\||fi|\\
\end{tabular}
\end{center}
%
Instead of |\childdocof{|\textit{main}|}| just include the main file
at the top of each child file:
%
\begin{center}
|\input{|\textit{main}|}|
\end{center}
%
A simple redirection |\childdocforward{|\textit{dest}|}| is achieved by:
%
\begin{center}
|\def\jobname{|\textit{dest}|}\input{\jobname}|
\end{center}
%
The redirection with prefix
|\childdocforwardprefix[|\textit{prefix}|]{|\textit{dest}|}|
is accomplished by:
%
\begin{center}
\begin{tabular}{l}
|{\edef\jobname{\scantokens\expandafter{\jobname\noexpand}}|\\
|\def\redirectjob |\textit{prefix}|#1~~~{\gdef\jobname{|\textit{dest}|#1}}|\\
|\expandafter\redirectjob\jobname~~~}\input{\jobname}|
\end{tabular}
\end{center}

In an alternative approach,
child documents can be compiled by a specific command line
without additional code or specific definitions:
%
\begin{center}
|... -jobname "|\textit{target}|" "|[\textit{flags}]%
|\includeonly{|\textit{dest}|}\input{|\textit{main}|}"|
\end{center}
%

%%%%%%%%%%%%%%%%%%%%%%%%%%%%%%%%%%%%%%%%%%%%%%%%%%%%%%%%%%%%%%%%%%%%%%%%%%%%%%%%
%%%%%%%%%%%%%%%%%%%%%%%%%%%%%%%%%%%%%%%%%%%%%%%%%%%%%%%%%%%%%%%%%%%%%%%%%%%%%%%%
\section{Information}

%%%%%%%%%%%%%%%%%%%%%%%%%%%%%%%%%%%%%%%%%%%%%%%%%%%%%%%%%%%%%%%%%%%%%%%%%%%%%%%%
\subsection{Copyright}

Copyright \copyright{} 2017--2018 Niklas Beisert

This work may be distributed and/or modified under the
conditions of the \LaTeX{} Project Public License, either version 1.3
of this license or (at your option) any later version.
The latest version of this license is in
  \url{http://www.latex-project.org/lppl.txt}
and version 1.3 or later is part of all distributions of \LaTeX{}
version 2005/12/01 or later.

This work has the LPPL maintenance status `maintained'.

The Current Maintainer of this work is Niklas Beisert.

This work consists of the files |README.txt|, |childdoc.ins| and |childdoc.dtx|
as well as the derived files |childdoc.def|, |cdocsamp.tex|
with |cdocsch1.tex|, |cdocsch2.tex|, |cdocspt3.tex|, |cdocspt4.tex|,
|cdocsdrf.tex|, |cdocsfn1.tex|, |cdocsfn2.tex|
as well as |childdoc.pdf|.

%%%%%%%%%%%%%%%%%%%%%%%%%%%%%%%%%%%%%%%%%%%%%%%%%%%%%%%%%%%%%%%%%%%%%%%%%%%%%%%%
\subsection{Files and Installation}

The package consists of the files:
%
\begin{center}
\begin{tabular}{ll}
    |README.txt|   & readme file \\
    |childdoc.ins| & installation file \\
    |childdoc.dtx| & source file \\
    |childdoc.def| & definition file \\
    |cdocsamp.tex| & sample main file \\
    |cdocsch1.tex| & sample include file \\
    |cdocsch2.tex| & sample include file \\
    |cdocspt3.tex| & sample part file \\
    |cdocspt4.tex| & sample part file \\
    |cdocsdrf.tex| & sample redirection file \\
    |cdocsfn1.tex| & sample redirection file \\
    |cdocsfn2.tex| & sample redirection file \\
    |childdoc.pdf| & manual
\end{tabular}
\end{center}
%
The distribution consists of the files
|README.txt|, |childdoc.ins| and |childdoc.dtx|.
%
\begin{itemize}
\item
Run (pdf)\LaTeX{} on |childdoc.dtx|
to compile the manual |childdoc.pdf| (this file).
\item
Run \LaTeX{} on |childdoc.ins| to create the definitions file |childdoc.def|
and the sample |cdocsamp.tex| with include files
|cdocsch1.tex|, |cdocsch2.tex|, |cdocspt3.tex|, |cdocspt4.tex|,
|cdocsdrf.tex|, |cdocsfn1.tex|, |cdocsfn2.tex|.
Then copy the file |childdoc.def| to an appropriate directory of your \LaTeX{}
distribution, e.g.\ \textit{texmf-root}|/tex/latex/childdoc|.
\end{itemize}

%%%%%%%%%%%%%%%%%%%%%%%%%%%%%%%%%%%%%%%%%%%%%%%%%%%%%%%%%%%%%%%%%%%%%%%%%%%%%%%%
\subsection{Related CTAN Packages}

There are several other packages which offer a similar functionality:
%
\begin{itemize}
\item
The packages
\href{http://ctan.org/pkg/docmute}{\textsf{docmute}},
\href{http://ctan.org/pkg/includex}{\textsf{includex}} and
\href{http://ctan.org/pkg/standalone}{\textsf{standalone}}
provide commands to include only the document body of
a child file thus allowing both files to be compiled individually.
\item
The packages \href{http://ctan.org/pkg/subdocs}{\textsf{subdocs}}
and \href{http://ctan.org/pkg/subfiles}{\textsf{subfiles}}
provide structures in which the main and child documents can be
encapsulated and allowing them to be compiled individually.
The inclusion mechanism is different from the conventional |\include|.
\item
The package \href{http://ctan.org/pkg/combine}{\textsf{combine}}
is an elaborate solution to combine several documents into one.
\end{itemize}
%
See also the CTAN topic \href{http://ctan.org/topic/subdocs}{\textsf{subdocs}}
for further related packages.
The present package differs from the above solutions in that
a document structure constructed with the conventional |\include| mechanism
just needs two extra commands at the top of every file
such that all constituent files can be compiled individually.

%%%%%%%%%%%%%%%%%%%%%%%%%%%%%%%%%%%%%%%%%%%%%%%%%%%%%%%%%%%%%%%%%%%%%%%%%%%%%%%%
%\subsection{Feature Suggestions}
%
%The following is a list of features which may be useful for future
%versions of this package:
%%
%\begin{itemize}
%\item
%\ldots
%\end{itemize}

%%%%%%%%%%%%%%%%%%%%%%%%%%%%%%%%%%%%%%%%%%%%%%%%%%%%%%%%%%%%%%%%%%%%%%%%%%%%%%%%
\subsection{Revision History}

%%%%%%%%%%%%%%%%%%%%%%%%%%%%%%%%%%%%%%%%
\paragraph{v2.0:} 2018/12/30

\begin{itemize}
\item
immediate forward processing
\item
added |\childdocby| mechanism
\item
manual restructured
\end{itemize}

%%%%%%%%%%%%%%%%%%%%%%%%%%%%%%%%%%%%%%%%
\paragraph{v1.6:} 2018/01/17

\begin{itemize}
\item
application for development of include files
\item
corrections to manual
\end{itemize}

%%%%%%%%%%%%%%%%%%%%%%%%%%%%%%%%%%%%%%%%
\paragraph{v1.5:} 2017/05/21

\begin{itemize}
\item
more complete structuring introduced
\item
|\childdocof| introduced
\item
|\childdoc| renamed to |\childdocmain|
\item
|\childredirect| renamed to |\childdocforward| and |\childdocforwardprefix|
and functionality expanded
\end{itemize}

%%%%%%%%%%%%%%%%%%%%%%%%%%%%%%%%%%%%%%%%
\paragraph{v1.0:} 2017/04/27

\begin{itemize}
\item
manual and install package
\item
first version published on CTAN
\end{itemize}

%%%%%%%%%%%%%%%%%%%%%%%%%%%%%%%%%%%%%%%%
\paragraph{v0.6:} 2017/04/26

\begin{itemize}
\item
redirection mechanism added
\end{itemize}

%%%%%%%%%%%%%%%%%%%%%%%%%%%%%%%%%%%%%%%%
\paragraph{v0.5:} 2017/04/26

\begin{itemize}
\item
functionality in definition file
\end{itemize}


%%%%%%%%%%%%%%%%%%%%%%%%%%%%%%%%%%%%%%%%%%%%%%%%%%%%%%%%%%%%%%%%%%%%%%%%%%%%%%%%
%%%%%%%%%%%%%%%%%%%%%%%%%%%%%%%%%%%%%%%%%%%%%%%%%%%%%%%%%%%%%%%%%%%%%%%%%%%%%%%%
%%%%%%%%%%%%%%%%%%%%%%%%%%%%%%%%%%%%%%%%%%%%%%%%%%%%%%%%%%%%%%%%%%%%%%%%%%%%%%%%
\appendix

\settowidth\MacroIndent{\rmfamily\scriptsize 000\ }

 \DocInput{childdoc.dtx}

\end{document}
%</driver>
% \fi
%
% %%%%%%%%%%%%%%%%%%%%%%%%%%%%%%%%%%%%%%%%%%%%%%%%%%%%%%%%%%%%%%%%%%%%%%%%%%%%%%
% %%%%%%%%%%%%%%%%%%%%%%%%%%%%%%%%%%%%%%%%%%%%%%%%%%%%%%%%%%%%%%%%%%%%%%%%%%%%%%
% \section{Sample}
%\iffalse
%<*samplemain>
%\fi
%
% The following presents a sample document
% with two chapters, two parts, a title page,
% a compile flag as well as three forwarding files to set the flag.
% It consists of eight |.tex| files:
% \begin{center}
% \begin{tabular}{ll}
% |cdocsamp.tex|&main file\\
% |cdocsch1.tex|&include file for chapter 1\\
% |cdocsch2.tex|&include file for chapter 2\\
% |cdocspt3.tex|&include file for part 3\\
% |cdocspt4.tex|&include file for part 4\\
% |cdocsdrf.tex|&forwarding file for main file in draft mode\\
% |cdocsfi1.tex|&forwarding file for final version of chapter 1\\
% |cdocsfi2.tex|&forwarding file for final version of chapter 2\\
% \end{tabular}
% \end{center}
% Each of the eight files can be compiled directly by the \LaTeX{} compiler.
%
% %%%%%%%%%%%%%%%%%%%%%%%%%%%%%%%%%%%%%%
% \paragraph{Main File.}
%
% The main file is called |cdocsamp.tex|.
%
% Load the \textsf{childdoc} definitions and
% declare the filename for the main document:
%    \begin{macrocode}
\input{childdoc.def}
\childdocmain{}
%    \end{macrocode}

% Optional override for |\version| flag:
%    \begin{macrocode}
%%\ifchilddoc\else\providecommand{\version}{draft}\fi
%    \end{macrocode}

% Define the default values for the |\version| flag
% (|final| for the main file and |draft| for childs):
%    \begin{macrocode}
\ifchilddoc
\providecommand{\version}{draft}
\else
\providecommand{\version}{final}
\fi
%    \end{macrocode}

% Load the standard document class:
%    \begin{macrocode}
\documentclass[12pt]{article}
%    \end{macrocode}

% Start the document body:
%    \begin{macrocode}
\begin{document}
%    \end{macrocode}

% Declare a title page.
% Print title, part of document being processed and version flag:
%    \begin{macrocode}
\addtocounter{page}{-1}
\begin{center}
{\LARGE\bfseries{}childdoc example\par}
\vspace{1cm}
\ifchilddoc
\ifchilddocmanual part\else chapter\fi:
`\childdocname' of `\childdocjob'\par
\else
main document: `\childdocjob'\par
\fi
version: \version\par
\end{center}
\newpage
%    \end{macrocode}

% Manually include selected file,
% otherwise process as usual:
%    \begin{macrocode}
\ifchilddocmanual
\section*{part `\childdocname'}
\input{\childdocname}
\else
%    \end{macrocode}

% Include the two chapters:
%    \begin{macrocode}
\include{cdocsch1}
\include{cdocsch2}
%    \end{macrocode}

% Include the two parts unless only chapters should be displayed:
%    \begin{macrocode}
\ifchilddoc\else
\section{part three}
\input{cdocspt3}
\section{part four}
\input{cdocspt4}
\fi
%    \end{macrocode}

% Process as usual until here:
%    \begin{macrocode}
\fi
%    \end{macrocode}

% End of document body:
%    \begin{macrocode}
\end{document}
%    \end{macrocode}
%\iffalse
%</samplemain>
%\fi
%
% %%%%%%%%%%%%%%%%%%%%%%%%%%%%%%%%%%%%%%
% \paragraph{Chapter Include Files.}
%
% The include files are called |cdocsch1.tex| and |cdocsch2.tex|.
%
%\iffalse
%<*samplechap1|samplechap2>
%\fi

% Optional override for |\version| flag:
%    \begin{macrocode}
%%\providecommand{\version}{final}
%    \end{macrocode}

% Include the main document:
%    \begin{macrocode}
\input{childdoc.def}
\childdocof{cdocsamp}
%    \end{macrocode}

%\iffalse
%</samplechap1|samplechap2>
%\fi
%
%\iffalse
%<*samplechap1>
%\fi
% Some text for chapter 1:
%    \begin{macrocode}
\section{one}
some text in chapter one
%    \end{macrocode}

%\iffalse
%</samplechap1>
%\fi
% Some text for chapter 2:
%\iffalse
%<*samplechap2>
%\fi
%    \begin{macrocode}
\section{two}
more text in chapter two
%    \end{macrocode}

%\iffalse
%</samplechap2>
%\fi
%
% %%%%%%%%%%%%%%%%%%%%%%%%%%%%%%%%%%%%%%
% \paragraph{Part Include Files.}
%
% The include files are called |cdocspt3.tex| and |cdocspt4.tex|.
%
%\iffalse
%<*samplepart3|samplepart4>
%\fi

% Optional override for |\version| flag:
%    \begin{macrocode}
%%\providecommand{\version}{final}
%    \end{macrocode}

% Include the main document:
%    \begin{macrocode}
\input{childdoc.def}
\childdocby{cdocsamp}
%    \end{macrocode}

%\iffalse
%</samplepart3|samplepart4>
%\fi
%
%\iffalse
%<*samplepart3>
%\fi
% Some text for part 3:
%    \begin{macrocode}
some text in part three
%    \end{macrocode}

%\iffalse
%</samplepart3>
%\fi
% Some text for part 4:
%\iffalse
%<*samplepart4>
%\fi
%    \begin{macrocode}
more text in part four
%    \end{macrocode}

%\iffalse
%</samplepart4>
%\fi
%
% %%%%%%%%%%%%%%%%%%%%%%%%%%%%%%%%%%%%%%
% \paragraph{Forwarding for a Complete Draft.}
%
% The following forwarding file |cdocsdrf.tex|
% compiles the main document in draft mode:
%\iffalse
%<*sampledraft>
%\fi
%    \begin{macrocode}
\def\version{draft}
\input{childdoc.def}
\childdocforward{cdocsamp}
%    \end{macrocode}

%\iffalse
%</sampledraft>
%\fi
%
% %%%%%%%%%%%%%%%%%%%%%%%%%%%%%%%%%%%%%%
% \paragraph{Forwarding for Final Version of the Chapters.}
%
% The following forwarding files |cdocsfn1.tex| and |cdocsfn2.tex|
% (with identical content)
% compile the final versions of the child documents
% |cdocsch1.tex| and |cdocsch2.tex|, respectively:
%\iffalse
%<*samplefinal>
%\fi
%    \begin{macrocode}
\def\version{final}
\input{childdoc.def}
\childdocforwardprefix[cdocsamp]{cdocsfn}{cdocsch}
%    \end{macrocode}

%\iffalse
%</samplefinal>
%\fi
%
% %%%%%%%%%%%%%%%%%%%%%%%%%%%%%%%%%%%%%%
% \paragraph{Command Line Processing.}
%
% The following three command lines generate the output files
% |cdocscld|, |cdocscl1| and |cdocscl2|
% which should be identical to
% |cdocsdrf|, |cdocsch1| and |cdocsfn2|, respectively:
% \begin{center}
% \begin{tabular}{l}
% |latex -jobname cdocscld \|\\
% |  "\def\version{draft}\input{childdoc.def}\childdocforward{cdocsamp}"|\\
% |latex -jobname cdocscl1 \|\\
% |  "\input{childdoc.def}\childdocforward[cdocsamp]{cdocsch1}"|\\
% |latex -jobname cdocscl2 \|\\
% |  "\def\version{final}\input{childdoc.def}\childdocforward{cdocsch2}"|
% \end{tabular}
% \end{center}
% Note that the trailing backslash on each first line
% merely continues the input to the second line
% (for convenient cut ant paste).
% Furthermore, the command |latex| can be replaced by any
% of its alternative versions such as |pdflatex|.
%
% %%%%%%%%%%%%%%%%%%%%%%%%%%%%%%%%%%%%%%%%%%%%%%%%%%%%%%%%%%%%%%%%%%%%%%%%%%%%%%
% %%%%%%%%%%%%%%%%%%%%%%%%%%%%%%%%%%%%%%%%%%%%%%%%%%%%%%%%%%%%%%%%%%%%%%%%%%%%%%
% \section{Implementation}
%\iffalse
%<*package>
%\fi
%
% This section describes the definitions file |childdoc.def|.

% The definitions cannot be loaded using |\usepackage| or |\RequirePackage|
% which has a mechanism to prevent loading a style file more than once.
% When loading the definitions by means of |\input|
% multiple instances have to be prevented manually:
%\iffalse
%This code needs to be before the `\ProvidesFile' directive
%which is defined at the beginning of this file.
%Therefore it is also placed there and commented out here.
%</package>
%<*discard>
%\fi
%    \begin{macrocode}
\ifdefined\childdocmain\endinput\fi
%    \end{macrocode}
%\iffalse
%</discard>
%<*package>
%\fi
%
% \macro{\ifchilddoc}
% \macro{\ifchilddocmanual}
% The conditional |\ifchilddoc| tells whether a
% child (true) or main (false) document is being compiled.
% The conditional |\ifchilddocmanual| tells whether
% the |\includeonly| mechanism is used (false) or
% the selection of child files must be performed manually (true).
% The definitions initialise to false:
%    \begin{macrocode}
\newif\ifchilddoc
\newif\ifchilddocmanual
%    \end{macrocode}

% \macro{\childdocname}
% \macro{\childdocjob}
% The macro |\childdocname| stores the name of the main document
% to be compiled. The macro |\childdocjob| stores the name of
% the document on which the \LaTeX{} compiler was originally invoked.
% The content of |\jobname| cannot be compared
% to filenames specified in the source due to different catcodes.
% The following code rescans |\jobname|, stores the result
% in |\childdocname| and saves a copy in |\childdocjob|:
%    \begin{macrocode}
\edef\childdocname{\scantokens\expandafter{\jobname\noexpand}}
\let\childdocjob\childdocname
%    \end{macrocode}

% \macro{\childdocdisable}
% The macro |\childdocdisable| prevents the main file
% from being processed more than once.
% At this stage, the main document command |\childdocmain|
% is assumed to be called once again where it should do nothing.
% Any subsequent call to it should prevent
% a secondary processing of the main document
% It overwrites the forwarding commands
% |\childdocof| and |\childdocforward|
% with empty macros to prevent further inclusions of the main document:
%    \begin{macrocode}
\newcommand{\childdocdisable}
{
  \renewcommand{\childdocmain}[1]{\renewcommand{\childdocmain}[1]{\endinput}}
  \renewcommand{\childdocof}[1]{}
  \renewcommand{\childdocby}[2][]{}
  \renewcommand{\childdocforward}[2][]{}
  \renewcommand{\childdocdisable}{}
}
%    \end{macrocode}

% \macro{\childdocmain}
% The macro |\childdocmain| is to be called at the top of the main file
% with nothing or the main filename (without extension) as argument.
% First, it breaks loops.
% If the argument is not empty and does not match |\childdocname|
% (which is set by the first inclusion of |childdoc.def|),
% |\ifchilddoc| is set to true, |\includeonly| is applied to the child file
% and |\jobname| is set to the main file
% (for proper handling of |.aux| files):
%    \begin{macrocode}
\newcommand{\childdocmain}[1]
{
  \childdocdisable\childdocmain{}
  \if?#1?\else
    \begingroup
      \def\childdoctmp{#1}
      \ifx\childdoctmp\childdocname
        \def\childdoctmp{}
      \else
        \def\childdoctmp
        {
          \childdoctrue
          \includeonly{\childdocname}
          \def\childdocjob{#1}
          \def\jobname{#1}
        }
      \fi
      \expandafter
    \endgroup
    \childdoctmp
  \fi
}
%    \end{macrocode}

% \macro{\childdocof}
% The command |\childdocof| redirects
% compilation to the main file |#1|.
%    \begin{macrocode}
\newcommand{\childdocof}[1]
{
  \childdocdisable
  \childdoctrue
  \includeonly{\childdocname}
  \def\jobname{#1}
  \def\childdocjob{#1}
  \input{#1}
}
%    \end{macrocode}

% \macro{\childdocby}
% The command |\childdocby| ....
%    \begin{macrocode}
\newcommand{\childdocby}[2][]
{
  \childdocdisable
  \childdoctrue
  \childdocmanualtrue
  \if?#1?\else
    \def\jobname{#2}
  \fi
  \def\childdocjob{#2}
  \input{#2}
  \endinput
}
%    \end{macrocode}

% \macro{\childdocforward}
% The command |\childdocforward| redirects
% compilation to the main file or
% (if the optional argument is given) a child file.
% Parameters are set as if the main file
% or a child file starting with |\childdocof| was compiled.
% Then compilation is handed over to the main file:
%    \begin{macrocode}
\newcommand{\childdocforward}[2][]
{
  \begingroup
    \if?#1?
      \def\childdoctmp
      {
        \def\childdocname{#2}
        \def\childdocjob{#2}
        \def\jobname{#2}
        \input{#2}
        \endinput
      }
    \else
      \def\childdoctmp
      {
        \childdocdisable
        \def\childdocname{#2}
        \childdoctrue
        \includeonly{#2}
        \def\childdocjob{#1}
        \def\jobname{#1}
        \input{#1}
        \endinput
      }
    \fi
    \expandafter
  \endgroup
  \childdoctmp
}
%    \end{macrocode}

% \macro{\childdocforwardprefix}
% The command |\childdocforwardprefix| redirects
% compilation to the main or a child file by means of a pattern.
% The prefix |#1| in the current filename is replaced by |#2|
% and the suffix of the current filename is kept
% (it is assumed that the filename does not contain the substring `|~~~|'
% which is used as a delimiter).
% Compilation is handed over to the new file by |\childdocforward|:
%    \begin{macrocode}
\newcommand{\childdocforwardprefix}[3][]
{
  \begingroup
    \def\childdocextract #2##1~~~{\def\childdoctmp{\childdocforward[#1]{#3##1}}}
    \expandafter\childdocextract\childdocname~~~
    \expandafter
  \endgroup
  \childdoctmp
}
%    \end{macrocode}

% \macro{\childdoc}
% The deprecated macro |\childdoc| is a legacy version of |\childdocmain|:
%    \begin{macrocode}
\newcommand{\childdoc}{\childdocmain}
%    \end{macrocode}

% \macro{\childdocredirect}
% The deprecated macro |\childdocredirect| is a legacy version
% of |\childdocforward| and |\childdocforwardprefix|:
%    \begin{macrocode}
\newcommand{\childdocredirect}[2][]
{
  \begingroup
    \if?#1?
      \def\childdoctmp{\childdocforward{#2}}
    \else
      \def\childdoctmp{\childdocforwardprefix{#1}{#2}}
    \fi
    \expandafter
  \endgroup
  \childdoctmp
}
%    \end{macrocode}

%\iffalse
%</package>
%\fi
%
\endinput

\childdocby{cdocsamp}
%    \end{macrocode}

%\iffalse
%</samplepart3|samplepart4>
%\fi
%
%\iffalse
%<*samplepart3>
%\fi
% Some text for part 3:
%    \begin{macrocode}
some text in part three
%    \end{macrocode}

%\iffalse
%</samplepart3>
%\fi
% Some text for part 4:
%\iffalse
%<*samplepart4>
%\fi
%    \begin{macrocode}
more text in part four
%    \end{macrocode}

%\iffalse
%</samplepart4>
%\fi
%
% %%%%%%%%%%%%%%%%%%%%%%%%%%%%%%%%%%%%%%
% \paragraph{Forwarding for a Complete Draft.}
%
% The following forwarding file |cdocsdrf.tex|
% compiles the main document in draft mode:
%\iffalse
%<*sampledraft>
%\fi
%    \begin{macrocode}
\def\version{draft}
% \iffalse
%
% childdoc.dtx Copyright (C) 2017-2018 Niklas Beisert
%
% This work may be distributed and/or modified under the
% conditions of the LaTeX Project Public License, either version 1.3
% of this license or (at your option) any later version.
% The latest version of this license is in
%   http://www.latex-project.org/lppl.txt
% and version 1.3 or later is part of all distributions of LaTeX
% version 2005/12/01 or later.
%
% This work has the LPPL maintenance status `maintained'.
%
% The Current Maintainer of this work is Niklas Beisert.
%
% This work consists of the files childdoc.dtx and childdoc.ins
% and the derived files childdoc.def and cdocsamp.tex with
% cdocsch1.tex, cdocsch2.tex, cdocsdrf.tex, cdocsfn1.tex, cdocsfn2.tex.
%
%<package>\ifdefined\childdocmain\endinput\fi
%<package>\ProvidesFile{childdoc.def}[2018/12/30 v2.0 child document driver]
%<samplemain>\ProvidesFile{cdocsamp.tex}[2018/12/30 v2.0 sample for childdoc]
%<*driver>
%\ProvidesFile{childdoc.drv}[2018/12/30 v2.0 childdoc reference manual file]
\PassOptionsToClass{10pt,a4paper}{article}
\documentclass{ltxdoc}

\usepackage[margin=35mm]{geometry}
\usepackage{hyperref}
\usepackage{hyperxmp}
\usepackage[usenames]{color}

\hypersetup{colorlinks=true}
\hypersetup{pdfstartview=FitH}
\hypersetup{pdfpagemode=UseNone}
\hypersetup{pdfsource={}}
\hypersetup{pdflang={en-UK}}
\hypersetup{pdfcopyright={Copyright 2017-2018 Niklas Beisert.
  This work may be distributed and/or modified under the
  conditions of the LaTeX Project Public License, either version 1.3
  of this license or (at your option) any later version.}}
\hypersetup{pdflicenseurl={http://www.latex-project.org/lppl.txt}}
\hypersetup{pdfcontactaddress={ETH Zurich, ITP, HIT K,
  Wolfgang-Pauli-Strasse 27}}
\hypersetup{pdfcontactpostcode={8093}}
\hypersetup{pdfcontactcity={Zurich}}
\hypersetup{pdfcontactcountry={Switzerland}}
\hypersetup{pdfcontactemail={nbeisert@itp.phys.ethz.ch}}
\hypersetup{pdfcontacturl={http://people.phys.ethz.ch/\xmptilde nbeisert/}}

\newcommand{\secref}[1]{\hyperref[#1]{section \ref*{#1}}}

\parskip1ex
\parindent0pt
\let\olditemize\itemize
\def\itemize{\olditemize\parskip0pt}

\begin{document}

\title{The \textsf{childdoc} Package}
\hypersetup{pdftitle={The childdoc Package}}
\author{Niklas Beisert\\[2ex]
  Institut f\"ur Theoretische Physik\\
  Eidgen\"ossische Technische Hochschule Z\"urich\\
  Wolfgang-Pauli-Strasse 27, 8093 Z\"urich, Switzerland\\[1ex]
  \href{mailto:nbeisert@itp.phys.ethz.ch}
  {\texttt{nbeisert@itp.phys.ethz.ch}}}
\hypersetup{pdfauthor={Niklas Beisert}}
\hypersetup{pdfsubject={Manual for the LaTeX2e Package childdoc}}
\date{30 December 2018, \textsf{v2.0}}
\maketitle

\begin{abstract}\noindent
\textsf{childdoc} is a \LaTeXe{} package
that enables the direct compilation
of document sections included by |\include|
to individual files.
\end{abstract}

\begingroup
\parskip0ex
\tableofcontents
\endgroup

%%%%%%%%%%%%%%%%%%%%%%%%%%%%%%%%%%%%%%%%%%%%%%%%%%%%%%%%%%%%%%%%%%%%%%%%%%%%%%%%
%%%%%%%%%%%%%%%%%%%%%%%%%%%%%%%%%%%%%%%%%%%%%%%%%%%%%%%%%%%%%%%%%%%%%%%%%%%%%%%%
\section{Introduction}

\LaTeX{} provides a mechanism to structure a large document (such as a book)
into a main file and several child files (containing the chapters)
using the |\include| command.
This mechanism is beneficial for documents
which span hundreds of pages in order to
make the source file(s) more manageable.
Moreover, compilation can be restricted to
selected child files by means of the |\includeonly| command.
The latter feature can be used to reduce the compilation time while editing
(this was significantly more useful in the earlier days of \LaTeX{})
or to generate a smaller document which is easier to navigate.
Another application of |\includeonly| is to generate
documents consisting of selected parts of the complete document.

However, there are a few drawbacks of the plain |\include| mechanism:
\begin{itemize}
\item
The child files cannot be compiled on their own,
they can only be compiled via the main file.
A naive editing environment
(such as a text editor with an option
to have the current file processed by \LaTeX)
may require one to switch to the main file before compiling;
attempting to compile the child file produces errors.
\item
The main file must be modified (each time)
to adjust the |\includeonly| command
to the present needs. This easily leaves the main file in a messy state.
\item
The generated document will always carry the filename
of the main document. This is inconvenient if
several child files are to be compiled and
to be kept for distribution.
\end{itemize}

The present package provides a simple interface
to make child files individually compilable by \LaTeX{}.
Compiling a child file then has the same effect as compiling
the main file with an |\includeonly| command
to select the appropriate child.
Moreover the generated document will carry the name of the child
rather than the main file.
This resolves all three above issues.

This feature is meant to make the editing of books,
thesis documents and lecture notes somewhat more convenient.
However, the package can also be used efficiently for
composing a series of documents (such as exercise sheets)
which are typically distributed individually.
It then assists the author in generating the individual documents
(potentially in different versions)
as well as a document containing the collected series.
Another application is in developing style files
or other kinds of included material
where compilation of the style file could redirect
to a sample or test file.

%%%%%%%%%%%%%%%%%%%%%%%%%%%%%%%%%%%%%%%%%%%%%%%%%%%%%%%%%%%%%%%%%%%%%%%%%%%%%%%%
%%%%%%%%%%%%%%%%%%%%%%%%%%%%%%%%%%%%%%%%%%%%%%%%%%%%%%%%%%%%%%%%%%%%%%%%%%%%%%%%
\section{Usage}

First of all, the package \textsf{childdoc} is \emph{not} a standard
\LaTeXe{} |.sty| style file! Therefore it needs to be invoked in
a non-standard way.

%%%%%%%%%%%%%%%%%%%%%%%%%%%%%%%%%%%%%%%%%%%%%%%%%%%%%%%%%%%%%%%%%%%%%%%%%%%%%%%%
\subsection{Included Files}
\label{sec:include}

%%%%%%%%%%%%%%%%%%%%%%%%%%%%%%%%%%%%%%%%
\DescribeMacro{\childdocmain}
To use the package, add the commands
\begin{center}
\begin{tabular}{l}
|\input{childdoc.def}|\\
|\childdocmain{}|\\
\end{tabular}
\end{center}
at the very top of the main \LaTeX{} file,
in particular \emph{before} the |\documentclass| statement!
The argument of |\childdocmain| should be left empty
(but it must be present).

%%%%%%%%%%%%%%%%%%%%%%%%%%%%%%%%%%%%%%%%
\DescribeMacro{\childdocof}
Furthermore, add the commands
\begin{center}
\begin{tabular}{l}
|\input{childdoc.def}|\\
|\childdocof{|\textit{main}|}|\\
\end{tabular}
\end{center}
at the top of every child file \textit{child}
which is included by |\include{|\textit{child}|}|
from within the main file
(or at least for those files to be compiled individually).
The argument \textit{main} must be the filename of the main file.

There are a couple of
considerations in setting up the main and child documents:

%%%%%%%%%%%%%%%%%%%%%%%%%%%%%%%%%%%%%%%%
\paragraph{Restrictions.}

Please note the following restrictions:
\begin{itemize}
\item
|\childdocmain| must be called with one argument \textit{main}
to ensure compatibility with earlier version of the package.
It must either be empty (|\childdocmain{}|)
or precisely match the filename of the main file in which it is specified.
See \secref{sec:detection} for further information.
\item
The filename \textit{main} must be specified without the |.tex| extension.
\item
The filename \textit{main} is case sensitive
(even in case-insensitive file systems)
due to internal string comparison.
\item
The argument \textit{main} should be fully expanded, it cannot be a macro.
\item
Subdirectories and special characters should be avoided in filenames.
\item
The command |\childdocmain{|\textit{main}|}| must be followed by a whitespace.
It should not be followed immediately by another command
or by a comment mark `|%|'.
This is because the \TeX{} parser reads the token immediately following
the argument of |\childdocmain| and puts it
at the beginning of every child section;
however, a white\-space is ignored.
\end{itemize}

%%%%%%%%%%%%%%%%%%%%%%%%%%%%%%%%%%%%%%%%
\paragraph{Content of Main File.}

It is advisable to place all content in the child files included by |\include|.
Any output contained in the main file will appear in all child documents
unless suppressed manually;
it cannot be suppressed automatically by the |\includeonly| directive
and thus should normally be avoided.
A method to include some content in the main file
by means of conditional processing is described in \secref{sec:conditional}.

%%%%%%%%%%%%%%%%%%%%%%%%%%%%%%%%%%%%%%%%
\paragraph{Page Numbering.}

When only a part of the document is compiled,
the appropriate numbering of pages
(as well as other status parameters)
is determined from the |.aux| files.
The latter contain information from previous passes.
However this information needs to propagate through
all intermediate child documents.
Therefore the page numbering in child documents may well
be inconsistent until the complete document is compiled at least once.

A useful (if unconventional) way to always ensure a consistent
page numbering is to restart the numbering in each child document
and denote the pages by `\textit{child}|.|\textit{page}'
where \textit{child} represents the chapter/section number of the child file.
This can be achieved by the command
|\numberwithin{page}{|\textit{child}|}|
of the \textsf{amsmath} package
where \textit{child} can be |chapter| or |section|
depending on the chosen structuring.
Alternatively, one can modify the macro |\thepage| appropriately
and reset the counter |page| at the start of each child file.

%%%%%%%%%%%%%%%%%%%%%%%%%%%%%%%%%%%%%%%%%%%%%%%%%%%%%%%%%%%%%%%%%%%%%%%%%%%%%%%%
\subsection{Conditional Processing}
\label{sec:conditional}

The package provides a mechanism to compile different versions
of a document. To customise the versions further some conditional processing
can come in handy to distinguish which version is being compiled.
The package provides two macros to describe the compilation context:

%%%%%%%%%%%%%%%%%%%%%%%%%%%%%%%%%%%%%%%%
\DescribeMacro{\ifchilddoc}
The conditional |\ifchilddoc| distinguishes between the compilation of
child documents and the main document:
%
\begin{center}
|\ifchilddoc |\textit{child-code}| |[|\||else |\textit{main-code}]| \||fi|
\end{center}

%%%%%%%%%%%%%%%%%%%%%%%%%%%%%%%%%%%%%%%%
\DescribeMacro{\childdocname}
\DescribeMacro{\childdocjob}
The macro |\childdocname| contains the filename (without extension)
of the main or child file being processed.
Note that |\childdocjob| will always contain the name of the main file.

%%%%%%%%%%%%%%%%%%%%%%%%%%%%%%%%%%%%%%%%
\paragraph{Title Page.}

Conditional processing can be used to include a title or banner page
in the main document when proper precautions are taken.
Importantly, the code in the main file should ensure that the page counter
(as well as other status parameters which are stored in the |.aux| files)
takes the same value after the conditional processing.
Otherwise the page numbers may take divergent values
depending on which part is compiled.

For example, a title page could be declared by:
%
\begin{center}
\begin{tabular}{l}
|\ifchilddoc\||else|\\
|\addtocounter{page}{-1}|\\
\textit{code for title page}\\
|\newpage|\\
|\||fi|
\end{tabular}
\end{center}
%
A banner page for the child documents can be generated by:
%
\begin{center}
\begin{tabular}{l}
|\ifchilddoc|\\
|\addtocounter{page}{-1}|\\
\textit{code for banner page}\\
|\newpage|\\
|\||fi|
\end{tabular}
\end{center}
%
Here one could write a message such as:
\begin{center}
|This is the part \childdocname{} of \childdocjob{}.|
\end{center}

%%%%%%%%%%%%%%%%%%%%%%%%%%%%%%%%%%%%%%%%%%%%%%%%%%%%%%%%%%%%%%%%%%%%%%%%%%%%%%%%
\subsection{Flags}
\label{sec:flags}

The package makes it easy to generate different versions
of the main or child documents.
To this end compilation flags can be defined
and assigned different default values.
They will be particularly useful in conjunction
with the forwarding mechanism described in \secref{sec:forward}.

For example, it may be useful to have a flag |\version|
which can be set to |draft| or |final|.
The document source will contain some conditional code
depending on the value of |\version|.
Suppose further, the flag should default to |final| for the main file
and to |draft| for child files
which is a natural assignment for editing the document.
This is achieved by placing the following code
in the preamble of the main document
(below the |\childdocmain| directive):
%
\begin{center}
\begin{tabular}{l}
|\ifchilddoc|\\
|\providecommand{\version}{draft}|\\
|\||else|\\
|\providecommand{\version}{final}|\\
|\||fi|
\end{tabular}
\end{center}
%
The definition by |\providecommand| makes sure
that previous definitions are not overwritten.
Further statements |\providecommand{\version}{...}|
can thus be added before the above code to override it.

For the main file, one might add a line
(between |\childdocmain| and the above block)
%
\begin{center}
|%\ifchilddoc\||else\providecommand{\version}{draft}\||fi|
\end{center}
%
which can be uncommented to produce a draft version.
Likewise one can add a line to the very top of a child file
(above the |\childdocof{|\textit{main}|}| directive)
%
\begin{center}
|%\providecommand{\version}{final}|
\end{center}
%
which can be uncommented to produce the final version of this child document.

%%%%%%%%%%%%%%%%%%%%%%%%%%%%%%%%%%%%%%%%%%%%%%%%%%%%%%%%%%%%%%%%%%%%%%%%%%%%%%%%
\subsection{Forwarding}
\label{sec:forward}

Different versions of the main or child documents
using compilation flags as described in \secref{sec:flags}
can be (permanently) stored in different files
for convenient compilation, viewing and distribution.
To this end, the package defines a command
to pass on compilation to a different file:

%%%%%%%%%%%%%%%%%%%%%%%%%%%%%%%%%%%%%%%%
\DescribeMacro{\childdocforward}
The command |\childdocforward| redirects processing to
another source file:
%
\begin{center}
\begin{tabular}{l}
|\input{childdoc.def}|\\
|\childdocforward[|\textit{main}|]{|\textit{dest}|}|\\
\end{tabular}
\end{center}
%
The argument \textit{dest} is the destination file
(without extension).
It should be the main file or one of the child files.
Note that further \textsf{childdoc} directives
such as |\childdocof| and |\childdocforward|
in the indicated file will be processed in this form.
The optional argument \textit{main}
passes on directly to the main file \textit{main}
while pretending to compile the child \textit{dest}.
This form behaves as if \textit{dest}
issues |\childdocof{|\textit{main}|}| right away,
and no further \textsf{childdoc} directives will be processed.

%%%%%%%%%%%%%%%%%%%%%%%%%%%%%%%%%%%%%%%%
\DescribeMacro{\...prefix}
In the alternative form |\childdocforwardprefix|,
%
\begin{center}
\begin{tabular}{l}
|\input{childdoc.def}|\\
|\childdocforwardprefix[|\textit{main}|]{|\textit{prefix}|}{|\textit{dest}|}|
\end{tabular}
\end{center}
%
the destination file is determined by a pattern
depending on the current file:
To make this work, the current file must be called
`{\textit{prefix}\hspace{0.2em}\textit{suffix}}'
with \textit{prefix} matching precisely the argument.
Processing is then passed on to the file
`{\textit{dest}\hspace{0.2em}\textit{suffix}}'.
Surely, the same effect is achieved by
directly specifying the
argument `{\textit{dest}\hspace{0.2em}\textit{suffix}}'
in the first form.
However, that requires to set up a different file
for each child. With the alternative form of the command
all these files can have exactly the same content
which simplifies setting them up and maintaining them.

For example, the following file |draft.tex|
with a compilation flag |\version| as described in \secref{sec:flags}
compiles the main document as a draft:
%
\begin{center}
\begin{tabular}{l}
|\def\version{draft}|\\
|\input{childdoc.def}|\\
|\childdocforward{|\textit{main}|}|
\end{tabular}
\end{center}
%
Likewise, the following files |final|\textit{nn}|.tex|
compile the final version of the child document
|child|\textit{nn}|.tex|:
%
\begin{center}
\begin{tabular}{l}
|\def\version{final}|\\
|\input{childdoc.def}|\\
|\childdocforwardprefix{final}{child}|
\end{tabular}
\end{center}
%

Note that when several versions of a main file and/or of each child file
are to be generated, it may be convenient to set up a |Makefile| or
shell script to automatise the process.

%%%%%%%%%%%%%%%%%%%%%%%%%%%%%%%%%%%%%%%%%%%%%%%%%%%%%%%%%%%%%%%%%%%%%%%%%%%%%%%%
\subsection{Command Line Processing}
\label{sec:commandline}

The effect of redirection files can also be achieved by invoking
the \LaTeX{} compiler with a more elaborate command line.
Most conveniently this should be done as part
of a shell script or a |Makefile|.

When using \textsf{childdoc} in the main file, the following
command lines effectively perform a redirection
(note that depending on the shell being used,
backslashes may have to be doubled: `|\|' $\to$ `|\\|'):
%
\begin{center}
|... -jobname "|\textit{target}|" |\\|"|[\textit{flags}]%
|\input{childdoc.def}\childdocforward[|\textit{main}|]{|\textit{dest}|}"|
\end{center}
%
Here \textit{target} is the name of the output file,
\textit{main} is the name of the main file
and \textit{dest} is the name of the main or child file to be processed
(all filenames without extensions).
The optional argument \textit{main} can be omitted
if \textit{main} matches \textit{dest}.
Optionally, compilation \textit{flags} can be defined via |\def| commands.
This command line makes the \TeX{} engine believe
it is compiling the file \textit{target}
whose content is specified as the latter parameter.
The provided code then forwards the processing to
\textit{main} or \textit{dest} as described in \secref{sec:forward}.

%%%%%%%%%%%%%%%%%%%%%%%%%%%%%%%%%%%%%%%%%%%%%%%%%%%%%%%%%%%%%%%%%%%%%%%%%%%%%%%%
\subsection{Include by Input}
\label{sec:input}

Including child documents by |\include| has some restrictions by design.
Most notably, the content of a child document always occupies
its own set of pages; pages cannot be shared between child documents.
Usually, this behaviour makes perfect sense
because each child document contain an essential part of the document.
However, in some situations it may be desirable to compose
a document from a collection of parts
without having mandatory page breaks between then.
For this case, the package
provides a mechanism to include parts
by |\input| which can also be processed individually.
However, by construction this mechanism
requires manual handling of the content to be output.

%%%%%%%%%%%%%%%%%%%%%%%%%%%%%%%%%%%%%%%%
\DescribeMacro{\ifchilddocmanual}
The main file should be prepared as usual, see \secref{sec:include}.
However, the document body must make a distinction
between processing of an individual part and of the main document, e.g.:
%
\begin{center}
\begin{tabular}{l}
|\ifchilddocmanual|\\
|\input{\childdocname}|\\
|\||else|\\
\textit{document body with }|\input{|\textit{part}|}|\\
|\||fi|
\end{tabular}
\end{center}
%
The conditional |\ifchilddocmanual| is true whenever
a part to be included by |\input| is being compiled,
and the name of the part is stored in |\childdocname|.

%%%%%%%%%%%%%%%%%%%%%%%%%%%%%%%%%%%%%%%%
\DescribeMacro{\childdocby}
Each part to be included by |\input| should start with:
%
\begin{center}
\begin{tabular}{l}
|\input{childdoc.def}|\\
|\childdocby{|\textit{main}|}|\\
\end{tabular}
\end{center}
%
The directive |\childdocby| is similar to |\childdocof|
described in \secref{sec:include},
but the subsequent selection of content must be done manually.
To that end, both |\ifchilddoc| and |\ifchilddocmanual|
will be true upon processing of a part,
and the name of the part is stored in |\childdocname|.
Note that |\jobname| will be set to the filename of the current part
so that each part receives an individual |.aux| file
that does not interfere with the |.aux| file(s) of the main document.
This behaviour can be altered by the alternative form
|\childdocby[*]{|\textit{main}|}| (with a non-empty optional argument)
which uses the |.aux| file of the main document
by setting |\jobname| to \textit{main}.

%%%%%%%%%%%%%%%%%%%%%%%%%%%%%%%%%%%%%%%%%%%%%%%%%%%%%%%%%%%%%%%%%%%%%%%%%%%%%%%%
\subsection{Driver Development}
\label{sec:driver}

The \textsf{childdoc} mechanism can also be use for the development
of definition files such as \LaTeX{} styles or classes.
This case differs from the above setup with multiple parts
included by |\include| in that no |\includeonly| should be invoked.
This can be achieved by starting the include file
(before |\ProvidesPackage|) with:
%
\begin{center}
\begin{tabular}{l}
|\input{childdoc.def}|\\
|\childdocforward{|\textit{main}|}|\\
\end{tabular}
\end{center}
%
or alternatively with:
%
\begin{center}
\begin{tabular}{l}
|\input{childdoc.def}|\\
|\childdocby{|\textit{main}|}|\\
\end{tabular}
\end{center}
%
Both forms have slightly different effects as described above.
The main file is prepared as usual, see \secref{sec:include}.

%%%%%%%%%%%%%%%%%%%%%%%%%%%%%%%%%%%%%%%%%%%%%%%%%%%%%%%%%%%%%%%%%%%%%%%%%%%%%%%%
\subsection{Legacy Detection}
\label{sec:detection}

The directive |\childdocmain| in the main file can detect
whether the complete document or merely a child is to be compiled
even without using the directive |\childdocof|.
This method is deprecated because it is less robust
and there is no compelling reason to use it;
it is merely provided for backward compatibility
and it may be removed in future versions.

If the detection mechanism is to be used,
it is mandatory to correctly specify
the filename of the main file as the argument of |\childdocmain|:
%
\begin{center}
\begin{tabular}{l}
|\input{childdoc.def}|\\
|\childdocmain{|\textit{main}|}|\\
\end{tabular}
\end{center}
%
If |\jobname| does not match the argument \textit{main} of |\childdocmain|,
it is assumed that |\jobname| points to the child file to be compiled.
When using |\childdocmain| with the main file specified as argument,
it suffices to start a child file
with just |\input{|\textit{main}|}|
without loading of the package and using |\childdocof|.
If instead all processing is done
with the appropriate \textsf{childdoc} directives,
the argument of \textit{main} of |\childdocmain| can be empty.

An alternative version of the command line processing described
in \secref{sec:commandline} using the detection mechanism reads:
%
\begin{center}
|... -jobname "|\textit{target}|" "|[\textit{flags}]%
[|\def\jobname{|\textit{dest}|}|]|\input{|\textit{main}|}"|
\end{center}

%%%%%%%%%%%%%%%%%%%%%%%%%%%%%%%%%%%%%%%%%%%%%%%%%%%%%%%%%%%%%%%%%%%%%%%%%%%%%%%%
\subsection{Manual Code}
\label{sec:manual}

In case one cannot be certain whether the definitions file |childdoc.def|
is installed on the target \TeX{} distribution
and one prefers not to ship it,
it is conceivable to paste a few relevant commands into the sources.

To that end, drop all statements |\input{childdoc.def}|
and perform the replacements as outlined below.
Instead of |\childdocmain{|\textit{main}|}| add the following code
to the top of the main file:
%
\begin{center}
\begin{tabular}{l}
|\||ifdefined\childdocname\endinput\||fi\newif\ifchilddoc|\\
|\edef\childdocname{\scantokens\expandafter{\jobname\noexpand}}|\\
|\def\childdocmain{|\textit{main}|}\||ifx\childdocmain\childdocname\||else|\\
|\childdoctrue\includeonly{\childdocname}\let\jobname\childdocmain\||fi|\\
\end{tabular}
\end{center}
%
Instead of |\childdocof{|\textit{main}|}| just include the main file
at the top of each child file:
%
\begin{center}
|\input{|\textit{main}|}|
\end{center}
%
A simple redirection |\childdocforward{|\textit{dest}|}| is achieved by:
%
\begin{center}
|\def\jobname{|\textit{dest}|}\input{\jobname}|
\end{center}
%
The redirection with prefix
|\childdocforwardprefix[|\textit{prefix}|]{|\textit{dest}|}|
is accomplished by:
%
\begin{center}
\begin{tabular}{l}
|{\edef\jobname{\scantokens\expandafter{\jobname\noexpand}}|\\
|\def\redirectjob |\textit{prefix}|#1~~~{\gdef\jobname{|\textit{dest}|#1}}|\\
|\expandafter\redirectjob\jobname~~~}\input{\jobname}|
\end{tabular}
\end{center}

In an alternative approach,
child documents can be compiled by a specific command line
without additional code or specific definitions:
%
\begin{center}
|... -jobname "|\textit{target}|" "|[\textit{flags}]%
|\includeonly{|\textit{dest}|}\input{|\textit{main}|}"|
\end{center}
%

%%%%%%%%%%%%%%%%%%%%%%%%%%%%%%%%%%%%%%%%%%%%%%%%%%%%%%%%%%%%%%%%%%%%%%%%%%%%%%%%
%%%%%%%%%%%%%%%%%%%%%%%%%%%%%%%%%%%%%%%%%%%%%%%%%%%%%%%%%%%%%%%%%%%%%%%%%%%%%%%%
\section{Information}

%%%%%%%%%%%%%%%%%%%%%%%%%%%%%%%%%%%%%%%%%%%%%%%%%%%%%%%%%%%%%%%%%%%%%%%%%%%%%%%%
\subsection{Copyright}

Copyright \copyright{} 2017--2018 Niklas Beisert

This work may be distributed and/or modified under the
conditions of the \LaTeX{} Project Public License, either version 1.3
of this license or (at your option) any later version.
The latest version of this license is in
  \url{http://www.latex-project.org/lppl.txt}
and version 1.3 or later is part of all distributions of \LaTeX{}
version 2005/12/01 or later.

This work has the LPPL maintenance status `maintained'.

The Current Maintainer of this work is Niklas Beisert.

This work consists of the files |README.txt|, |childdoc.ins| and |childdoc.dtx|
as well as the derived files |childdoc.def|, |cdocsamp.tex|
with |cdocsch1.tex|, |cdocsch2.tex|, |cdocspt3.tex|, |cdocspt4.tex|,
|cdocsdrf.tex|, |cdocsfn1.tex|, |cdocsfn2.tex|
as well as |childdoc.pdf|.

%%%%%%%%%%%%%%%%%%%%%%%%%%%%%%%%%%%%%%%%%%%%%%%%%%%%%%%%%%%%%%%%%%%%%%%%%%%%%%%%
\subsection{Files and Installation}

The package consists of the files:
%
\begin{center}
\begin{tabular}{ll}
    |README.txt|   & readme file \\
    |childdoc.ins| & installation file \\
    |childdoc.dtx| & source file \\
    |childdoc.def| & definition file \\
    |cdocsamp.tex| & sample main file \\
    |cdocsch1.tex| & sample include file \\
    |cdocsch2.tex| & sample include file \\
    |cdocspt3.tex| & sample part file \\
    |cdocspt4.tex| & sample part file \\
    |cdocsdrf.tex| & sample redirection file \\
    |cdocsfn1.tex| & sample redirection file \\
    |cdocsfn2.tex| & sample redirection file \\
    |childdoc.pdf| & manual
\end{tabular}
\end{center}
%
The distribution consists of the files
|README.txt|, |childdoc.ins| and |childdoc.dtx|.
%
\begin{itemize}
\item
Run (pdf)\LaTeX{} on |childdoc.dtx|
to compile the manual |childdoc.pdf| (this file).
\item
Run \LaTeX{} on |childdoc.ins| to create the definitions file |childdoc.def|
and the sample |cdocsamp.tex| with include files
|cdocsch1.tex|, |cdocsch2.tex|, |cdocspt3.tex|, |cdocspt4.tex|,
|cdocsdrf.tex|, |cdocsfn1.tex|, |cdocsfn2.tex|.
Then copy the file |childdoc.def| to an appropriate directory of your \LaTeX{}
distribution, e.g.\ \textit{texmf-root}|/tex/latex/childdoc|.
\end{itemize}

%%%%%%%%%%%%%%%%%%%%%%%%%%%%%%%%%%%%%%%%%%%%%%%%%%%%%%%%%%%%%%%%%%%%%%%%%%%%%%%%
\subsection{Related CTAN Packages}

There are several other packages which offer a similar functionality:
%
\begin{itemize}
\item
The packages
\href{http://ctan.org/pkg/docmute}{\textsf{docmute}},
\href{http://ctan.org/pkg/includex}{\textsf{includex}} and
\href{http://ctan.org/pkg/standalone}{\textsf{standalone}}
provide commands to include only the document body of
a child file thus allowing both files to be compiled individually.
\item
The packages \href{http://ctan.org/pkg/subdocs}{\textsf{subdocs}}
and \href{http://ctan.org/pkg/subfiles}{\textsf{subfiles}}
provide structures in which the main and child documents can be
encapsulated and allowing them to be compiled individually.
The inclusion mechanism is different from the conventional |\include|.
\item
The package \href{http://ctan.org/pkg/combine}{\textsf{combine}}
is an elaborate solution to combine several documents into one.
\end{itemize}
%
See also the CTAN topic \href{http://ctan.org/topic/subdocs}{\textsf{subdocs}}
for further related packages.
The present package differs from the above solutions in that
a document structure constructed with the conventional |\include| mechanism
just needs two extra commands at the top of every file
such that all constituent files can be compiled individually.

%%%%%%%%%%%%%%%%%%%%%%%%%%%%%%%%%%%%%%%%%%%%%%%%%%%%%%%%%%%%%%%%%%%%%%%%%%%%%%%%
%\subsection{Feature Suggestions}
%
%The following is a list of features which may be useful for future
%versions of this package:
%%
%\begin{itemize}
%\item
%\ldots
%\end{itemize}

%%%%%%%%%%%%%%%%%%%%%%%%%%%%%%%%%%%%%%%%%%%%%%%%%%%%%%%%%%%%%%%%%%%%%%%%%%%%%%%%
\subsection{Revision History}

%%%%%%%%%%%%%%%%%%%%%%%%%%%%%%%%%%%%%%%%
\paragraph{v2.0:} 2018/12/30

\begin{itemize}
\item
immediate forward processing
\item
added |\childdocby| mechanism
\item
manual restructured
\end{itemize}

%%%%%%%%%%%%%%%%%%%%%%%%%%%%%%%%%%%%%%%%
\paragraph{v1.6:} 2018/01/17

\begin{itemize}
\item
application for development of include files
\item
corrections to manual
\end{itemize}

%%%%%%%%%%%%%%%%%%%%%%%%%%%%%%%%%%%%%%%%
\paragraph{v1.5:} 2017/05/21

\begin{itemize}
\item
more complete structuring introduced
\item
|\childdocof| introduced
\item
|\childdoc| renamed to |\childdocmain|
\item
|\childredirect| renamed to |\childdocforward| and |\childdocforwardprefix|
and functionality expanded
\end{itemize}

%%%%%%%%%%%%%%%%%%%%%%%%%%%%%%%%%%%%%%%%
\paragraph{v1.0:} 2017/04/27

\begin{itemize}
\item
manual and install package
\item
first version published on CTAN
\end{itemize}

%%%%%%%%%%%%%%%%%%%%%%%%%%%%%%%%%%%%%%%%
\paragraph{v0.6:} 2017/04/26

\begin{itemize}
\item
redirection mechanism added
\end{itemize}

%%%%%%%%%%%%%%%%%%%%%%%%%%%%%%%%%%%%%%%%
\paragraph{v0.5:} 2017/04/26

\begin{itemize}
\item
functionality in definition file
\end{itemize}


%%%%%%%%%%%%%%%%%%%%%%%%%%%%%%%%%%%%%%%%%%%%%%%%%%%%%%%%%%%%%%%%%%%%%%%%%%%%%%%%
%%%%%%%%%%%%%%%%%%%%%%%%%%%%%%%%%%%%%%%%%%%%%%%%%%%%%%%%%%%%%%%%%%%%%%%%%%%%%%%%
%%%%%%%%%%%%%%%%%%%%%%%%%%%%%%%%%%%%%%%%%%%%%%%%%%%%%%%%%%%%%%%%%%%%%%%%%%%%%%%%
\appendix

\settowidth\MacroIndent{\rmfamily\scriptsize 000\ }

 \DocInput{childdoc.dtx}

\end{document}
%</driver>
% \fi
%
% %%%%%%%%%%%%%%%%%%%%%%%%%%%%%%%%%%%%%%%%%%%%%%%%%%%%%%%%%%%%%%%%%%%%%%%%%%%%%%
% %%%%%%%%%%%%%%%%%%%%%%%%%%%%%%%%%%%%%%%%%%%%%%%%%%%%%%%%%%%%%%%%%%%%%%%%%%%%%%
% \section{Sample}
%\iffalse
%<*samplemain>
%\fi
%
% The following presents a sample document
% with two chapters, two parts, a title page,
% a compile flag as well as three forwarding files to set the flag.
% It consists of eight |.tex| files:
% \begin{center}
% \begin{tabular}{ll}
% |cdocsamp.tex|&main file\\
% |cdocsch1.tex|&include file for chapter 1\\
% |cdocsch2.tex|&include file for chapter 2\\
% |cdocspt3.tex|&include file for part 3\\
% |cdocspt4.tex|&include file for part 4\\
% |cdocsdrf.tex|&forwarding file for main file in draft mode\\
% |cdocsfi1.tex|&forwarding file for final version of chapter 1\\
% |cdocsfi2.tex|&forwarding file for final version of chapter 2\\
% \end{tabular}
% \end{center}
% Each of the eight files can be compiled directly by the \LaTeX{} compiler.
%
% %%%%%%%%%%%%%%%%%%%%%%%%%%%%%%%%%%%%%%
% \paragraph{Main File.}
%
% The main file is called |cdocsamp.tex|.
%
% Load the \textsf{childdoc} definitions and
% declare the filename for the main document:
%    \begin{macrocode}
\input{childdoc.def}
\childdocmain{}
%    \end{macrocode}

% Optional override for |\version| flag:
%    \begin{macrocode}
%%\ifchilddoc\else\providecommand{\version}{draft}\fi
%    \end{macrocode}

% Define the default values for the |\version| flag
% (|final| for the main file and |draft| for childs):
%    \begin{macrocode}
\ifchilddoc
\providecommand{\version}{draft}
\else
\providecommand{\version}{final}
\fi
%    \end{macrocode}

% Load the standard document class:
%    \begin{macrocode}
\documentclass[12pt]{article}
%    \end{macrocode}

% Start the document body:
%    \begin{macrocode}
\begin{document}
%    \end{macrocode}

% Declare a title page.
% Print title, part of document being processed and version flag:
%    \begin{macrocode}
\addtocounter{page}{-1}
\begin{center}
{\LARGE\bfseries{}childdoc example\par}
\vspace{1cm}
\ifchilddoc
\ifchilddocmanual part\else chapter\fi:
`\childdocname' of `\childdocjob'\par
\else
main document: `\childdocjob'\par
\fi
version: \version\par
\end{center}
\newpage
%    \end{macrocode}

% Manually include selected file,
% otherwise process as usual:
%    \begin{macrocode}
\ifchilddocmanual
\section*{part `\childdocname'}
\input{\childdocname}
\else
%    \end{macrocode}

% Include the two chapters:
%    \begin{macrocode}
\include{cdocsch1}
\include{cdocsch2}
%    \end{macrocode}

% Include the two parts unless only chapters should be displayed:
%    \begin{macrocode}
\ifchilddoc\else
\section{part three}
\input{cdocspt3}
\section{part four}
\input{cdocspt4}
\fi
%    \end{macrocode}

% Process as usual until here:
%    \begin{macrocode}
\fi
%    \end{macrocode}

% End of document body:
%    \begin{macrocode}
\end{document}
%    \end{macrocode}
%\iffalse
%</samplemain>
%\fi
%
% %%%%%%%%%%%%%%%%%%%%%%%%%%%%%%%%%%%%%%
% \paragraph{Chapter Include Files.}
%
% The include files are called |cdocsch1.tex| and |cdocsch2.tex|.
%
%\iffalse
%<*samplechap1|samplechap2>
%\fi

% Optional override for |\version| flag:
%    \begin{macrocode}
%%\providecommand{\version}{final}
%    \end{macrocode}

% Include the main document:
%    \begin{macrocode}
\input{childdoc.def}
\childdocof{cdocsamp}
%    \end{macrocode}

%\iffalse
%</samplechap1|samplechap2>
%\fi
%
%\iffalse
%<*samplechap1>
%\fi
% Some text for chapter 1:
%    \begin{macrocode}
\section{one}
some text in chapter one
%    \end{macrocode}

%\iffalse
%</samplechap1>
%\fi
% Some text for chapter 2:
%\iffalse
%<*samplechap2>
%\fi
%    \begin{macrocode}
\section{two}
more text in chapter two
%    \end{macrocode}

%\iffalse
%</samplechap2>
%\fi
%
% %%%%%%%%%%%%%%%%%%%%%%%%%%%%%%%%%%%%%%
% \paragraph{Part Include Files.}
%
% The include files are called |cdocspt3.tex| and |cdocspt4.tex|.
%
%\iffalse
%<*samplepart3|samplepart4>
%\fi

% Optional override for |\version| flag:
%    \begin{macrocode}
%%\providecommand{\version}{final}
%    \end{macrocode}

% Include the main document:
%    \begin{macrocode}
\input{childdoc.def}
\childdocby{cdocsamp}
%    \end{macrocode}

%\iffalse
%</samplepart3|samplepart4>
%\fi
%
%\iffalse
%<*samplepart3>
%\fi
% Some text for part 3:
%    \begin{macrocode}
some text in part three
%    \end{macrocode}

%\iffalse
%</samplepart3>
%\fi
% Some text for part 4:
%\iffalse
%<*samplepart4>
%\fi
%    \begin{macrocode}
more text in part four
%    \end{macrocode}

%\iffalse
%</samplepart4>
%\fi
%
% %%%%%%%%%%%%%%%%%%%%%%%%%%%%%%%%%%%%%%
% \paragraph{Forwarding for a Complete Draft.}
%
% The following forwarding file |cdocsdrf.tex|
% compiles the main document in draft mode:
%\iffalse
%<*sampledraft>
%\fi
%    \begin{macrocode}
\def\version{draft}
\input{childdoc.def}
\childdocforward{cdocsamp}
%    \end{macrocode}

%\iffalse
%</sampledraft>
%\fi
%
% %%%%%%%%%%%%%%%%%%%%%%%%%%%%%%%%%%%%%%
% \paragraph{Forwarding for Final Version of the Chapters.}
%
% The following forwarding files |cdocsfn1.tex| and |cdocsfn2.tex|
% (with identical content)
% compile the final versions of the child documents
% |cdocsch1.tex| and |cdocsch2.tex|, respectively:
%\iffalse
%<*samplefinal>
%\fi
%    \begin{macrocode}
\def\version{final}
\input{childdoc.def}
\childdocforwardprefix[cdocsamp]{cdocsfn}{cdocsch}
%    \end{macrocode}

%\iffalse
%</samplefinal>
%\fi
%
% %%%%%%%%%%%%%%%%%%%%%%%%%%%%%%%%%%%%%%
% \paragraph{Command Line Processing.}
%
% The following three command lines generate the output files
% |cdocscld|, |cdocscl1| and |cdocscl2|
% which should be identical to
% |cdocsdrf|, |cdocsch1| and |cdocsfn2|, respectively:
% \begin{center}
% \begin{tabular}{l}
% |latex -jobname cdocscld \|\\
% |  "\def\version{draft}\input{childdoc.def}\childdocforward{cdocsamp}"|\\
% |latex -jobname cdocscl1 \|\\
% |  "\input{childdoc.def}\childdocforward[cdocsamp]{cdocsch1}"|\\
% |latex -jobname cdocscl2 \|\\
% |  "\def\version{final}\input{childdoc.def}\childdocforward{cdocsch2}"|
% \end{tabular}
% \end{center}
% Note that the trailing backslash on each first line
% merely continues the input to the second line
% (for convenient cut ant paste).
% Furthermore, the command |latex| can be replaced by any
% of its alternative versions such as |pdflatex|.
%
% %%%%%%%%%%%%%%%%%%%%%%%%%%%%%%%%%%%%%%%%%%%%%%%%%%%%%%%%%%%%%%%%%%%%%%%%%%%%%%
% %%%%%%%%%%%%%%%%%%%%%%%%%%%%%%%%%%%%%%%%%%%%%%%%%%%%%%%%%%%%%%%%%%%%%%%%%%%%%%
% \section{Implementation}
%\iffalse
%<*package>
%\fi
%
% This section describes the definitions file |childdoc.def|.

% The definitions cannot be loaded using |\usepackage| or |\RequirePackage|
% which has a mechanism to prevent loading a style file more than once.
% When loading the definitions by means of |\input|
% multiple instances have to be prevented manually:
%\iffalse
%This code needs to be before the `\ProvidesFile' directive
%which is defined at the beginning of this file.
%Therefore it is also placed there and commented out here.
%</package>
%<*discard>
%\fi
%    \begin{macrocode}
\ifdefined\childdocmain\endinput\fi
%    \end{macrocode}
%\iffalse
%</discard>
%<*package>
%\fi
%
% \macro{\ifchilddoc}
% \macro{\ifchilddocmanual}
% The conditional |\ifchilddoc| tells whether a
% child (true) or main (false) document is being compiled.
% The conditional |\ifchilddocmanual| tells whether
% the |\includeonly| mechanism is used (false) or
% the selection of child files must be performed manually (true).
% The definitions initialise to false:
%    \begin{macrocode}
\newif\ifchilddoc
\newif\ifchilddocmanual
%    \end{macrocode}

% \macro{\childdocname}
% \macro{\childdocjob}
% The macro |\childdocname| stores the name of the main document
% to be compiled. The macro |\childdocjob| stores the name of
% the document on which the \LaTeX{} compiler was originally invoked.
% The content of |\jobname| cannot be compared
% to filenames specified in the source due to different catcodes.
% The following code rescans |\jobname|, stores the result
% in |\childdocname| and saves a copy in |\childdocjob|:
%    \begin{macrocode}
\edef\childdocname{\scantokens\expandafter{\jobname\noexpand}}
\let\childdocjob\childdocname
%    \end{macrocode}

% \macro{\childdocdisable}
% The macro |\childdocdisable| prevents the main file
% from being processed more than once.
% At this stage, the main document command |\childdocmain|
% is assumed to be called once again where it should do nothing.
% Any subsequent call to it should prevent
% a secondary processing of the main document
% It overwrites the forwarding commands
% |\childdocof| and |\childdocforward|
% with empty macros to prevent further inclusions of the main document:
%    \begin{macrocode}
\newcommand{\childdocdisable}
{
  \renewcommand{\childdocmain}[1]{\renewcommand{\childdocmain}[1]{\endinput}}
  \renewcommand{\childdocof}[1]{}
  \renewcommand{\childdocby}[2][]{}
  \renewcommand{\childdocforward}[2][]{}
  \renewcommand{\childdocdisable}{}
}
%    \end{macrocode}

% \macro{\childdocmain}
% The macro |\childdocmain| is to be called at the top of the main file
% with nothing or the main filename (without extension) as argument.
% First, it breaks loops.
% If the argument is not empty and does not match |\childdocname|
% (which is set by the first inclusion of |childdoc.def|),
% |\ifchilddoc| is set to true, |\includeonly| is applied to the child file
% and |\jobname| is set to the main file
% (for proper handling of |.aux| files):
%    \begin{macrocode}
\newcommand{\childdocmain}[1]
{
  \childdocdisable\childdocmain{}
  \if?#1?\else
    \begingroup
      \def\childdoctmp{#1}
      \ifx\childdoctmp\childdocname
        \def\childdoctmp{}
      \else
        \def\childdoctmp
        {
          \childdoctrue
          \includeonly{\childdocname}
          \def\childdocjob{#1}
          \def\jobname{#1}
        }
      \fi
      \expandafter
    \endgroup
    \childdoctmp
  \fi
}
%    \end{macrocode}

% \macro{\childdocof}
% The command |\childdocof| redirects
% compilation to the main file |#1|.
%    \begin{macrocode}
\newcommand{\childdocof}[1]
{
  \childdocdisable
  \childdoctrue
  \includeonly{\childdocname}
  \def\jobname{#1}
  \def\childdocjob{#1}
  \input{#1}
}
%    \end{macrocode}

% \macro{\childdocby}
% The command |\childdocby| ....
%    \begin{macrocode}
\newcommand{\childdocby}[2][]
{
  \childdocdisable
  \childdoctrue
  \childdocmanualtrue
  \if?#1?\else
    \def\jobname{#2}
  \fi
  \def\childdocjob{#2}
  \input{#2}
  \endinput
}
%    \end{macrocode}

% \macro{\childdocforward}
% The command |\childdocforward| redirects
% compilation to the main file or
% (if the optional argument is given) a child file.
% Parameters are set as if the main file
% or a child file starting with |\childdocof| was compiled.
% Then compilation is handed over to the main file:
%    \begin{macrocode}
\newcommand{\childdocforward}[2][]
{
  \begingroup
    \if?#1?
      \def\childdoctmp
      {
        \def\childdocname{#2}
        \def\childdocjob{#2}
        \def\jobname{#2}
        \input{#2}
        \endinput
      }
    \else
      \def\childdoctmp
      {
        \childdocdisable
        \def\childdocname{#2}
        \childdoctrue
        \includeonly{#2}
        \def\childdocjob{#1}
        \def\jobname{#1}
        \input{#1}
        \endinput
      }
    \fi
    \expandafter
  \endgroup
  \childdoctmp
}
%    \end{macrocode}

% \macro{\childdocforwardprefix}
% The command |\childdocforwardprefix| redirects
% compilation to the main or a child file by means of a pattern.
% The prefix |#1| in the current filename is replaced by |#2|
% and the suffix of the current filename is kept
% (it is assumed that the filename does not contain the substring `|~~~|'
% which is used as a delimiter).
% Compilation is handed over to the new file by |\childdocforward|:
%    \begin{macrocode}
\newcommand{\childdocforwardprefix}[3][]
{
  \begingroup
    \def\childdocextract #2##1~~~{\def\childdoctmp{\childdocforward[#1]{#3##1}}}
    \expandafter\childdocextract\childdocname~~~
    \expandafter
  \endgroup
  \childdoctmp
}
%    \end{macrocode}

% \macro{\childdoc}
% The deprecated macro |\childdoc| is a legacy version of |\childdocmain|:
%    \begin{macrocode}
\newcommand{\childdoc}{\childdocmain}
%    \end{macrocode}

% \macro{\childdocredirect}
% The deprecated macro |\childdocredirect| is a legacy version
% of |\childdocforward| and |\childdocforwardprefix|:
%    \begin{macrocode}
\newcommand{\childdocredirect}[2][]
{
  \begingroup
    \if?#1?
      \def\childdoctmp{\childdocforward{#2}}
    \else
      \def\childdoctmp{\childdocforwardprefix{#1}{#2}}
    \fi
    \expandafter
  \endgroup
  \childdoctmp
}
%    \end{macrocode}

%\iffalse
%</package>
%\fi
%
\endinput

\childdocforward{cdocsamp}
%    \end{macrocode}

%\iffalse
%</sampledraft>
%\fi
%
% %%%%%%%%%%%%%%%%%%%%%%%%%%%%%%%%%%%%%%
% \paragraph{Forwarding for Final Version of the Chapters.}
%
% The following forwarding files |cdocsfn1.tex| and |cdocsfn2.tex|
% (with identical content)
% compile the final versions of the child documents
% |cdocsch1.tex| and |cdocsch2.tex|, respectively:
%\iffalse
%<*samplefinal>
%\fi
%    \begin{macrocode}
\def\version{final}
% \iffalse
%
% childdoc.dtx Copyright (C) 2017-2018 Niklas Beisert
%
% This work may be distributed and/or modified under the
% conditions of the LaTeX Project Public License, either version 1.3
% of this license or (at your option) any later version.
% The latest version of this license is in
%   http://www.latex-project.org/lppl.txt
% and version 1.3 or later is part of all distributions of LaTeX
% version 2005/12/01 or later.
%
% This work has the LPPL maintenance status `maintained'.
%
% The Current Maintainer of this work is Niklas Beisert.
%
% This work consists of the files childdoc.dtx and childdoc.ins
% and the derived files childdoc.def and cdocsamp.tex with
% cdocsch1.tex, cdocsch2.tex, cdocsdrf.tex, cdocsfn1.tex, cdocsfn2.tex.
%
%<package>\ifdefined\childdocmain\endinput\fi
%<package>\ProvidesFile{childdoc.def}[2018/12/30 v2.0 child document driver]
%<samplemain>\ProvidesFile{cdocsamp.tex}[2018/12/30 v2.0 sample for childdoc]
%<*driver>
%\ProvidesFile{childdoc.drv}[2018/12/30 v2.0 childdoc reference manual file]
\PassOptionsToClass{10pt,a4paper}{article}
\documentclass{ltxdoc}

\usepackage[margin=35mm]{geometry}
\usepackage{hyperref}
\usepackage{hyperxmp}
\usepackage[usenames]{color}

\hypersetup{colorlinks=true}
\hypersetup{pdfstartview=FitH}
\hypersetup{pdfpagemode=UseNone}
\hypersetup{pdfsource={}}
\hypersetup{pdflang={en-UK}}
\hypersetup{pdfcopyright={Copyright 2017-2018 Niklas Beisert.
  This work may be distributed and/or modified under the
  conditions of the LaTeX Project Public License, either version 1.3
  of this license or (at your option) any later version.}}
\hypersetup{pdflicenseurl={http://www.latex-project.org/lppl.txt}}
\hypersetup{pdfcontactaddress={ETH Zurich, ITP, HIT K,
  Wolfgang-Pauli-Strasse 27}}
\hypersetup{pdfcontactpostcode={8093}}
\hypersetup{pdfcontactcity={Zurich}}
\hypersetup{pdfcontactcountry={Switzerland}}
\hypersetup{pdfcontactemail={nbeisert@itp.phys.ethz.ch}}
\hypersetup{pdfcontacturl={http://people.phys.ethz.ch/\xmptilde nbeisert/}}

\newcommand{\secref}[1]{\hyperref[#1]{section \ref*{#1}}}

\parskip1ex
\parindent0pt
\let\olditemize\itemize
\def\itemize{\olditemize\parskip0pt}

\begin{document}

\title{The \textsf{childdoc} Package}
\hypersetup{pdftitle={The childdoc Package}}
\author{Niklas Beisert\\[2ex]
  Institut f\"ur Theoretische Physik\\
  Eidgen\"ossische Technische Hochschule Z\"urich\\
  Wolfgang-Pauli-Strasse 27, 8093 Z\"urich, Switzerland\\[1ex]
  \href{mailto:nbeisert@itp.phys.ethz.ch}
  {\texttt{nbeisert@itp.phys.ethz.ch}}}
\hypersetup{pdfauthor={Niklas Beisert}}
\hypersetup{pdfsubject={Manual for the LaTeX2e Package childdoc}}
\date{30 December 2018, \textsf{v2.0}}
\maketitle

\begin{abstract}\noindent
\textsf{childdoc} is a \LaTeXe{} package
that enables the direct compilation
of document sections included by |\include|
to individual files.
\end{abstract}

\begingroup
\parskip0ex
\tableofcontents
\endgroup

%%%%%%%%%%%%%%%%%%%%%%%%%%%%%%%%%%%%%%%%%%%%%%%%%%%%%%%%%%%%%%%%%%%%%%%%%%%%%%%%
%%%%%%%%%%%%%%%%%%%%%%%%%%%%%%%%%%%%%%%%%%%%%%%%%%%%%%%%%%%%%%%%%%%%%%%%%%%%%%%%
\section{Introduction}

\LaTeX{} provides a mechanism to structure a large document (such as a book)
into a main file and several child files (containing the chapters)
using the |\include| command.
This mechanism is beneficial for documents
which span hundreds of pages in order to
make the source file(s) more manageable.
Moreover, compilation can be restricted to
selected child files by means of the |\includeonly| command.
The latter feature can be used to reduce the compilation time while editing
(this was significantly more useful in the earlier days of \LaTeX{})
or to generate a smaller document which is easier to navigate.
Another application of |\includeonly| is to generate
documents consisting of selected parts of the complete document.

However, there are a few drawbacks of the plain |\include| mechanism:
\begin{itemize}
\item
The child files cannot be compiled on their own,
they can only be compiled via the main file.
A naive editing environment
(such as a text editor with an option
to have the current file processed by \LaTeX)
may require one to switch to the main file before compiling;
attempting to compile the child file produces errors.
\item
The main file must be modified (each time)
to adjust the |\includeonly| command
to the present needs. This easily leaves the main file in a messy state.
\item
The generated document will always carry the filename
of the main document. This is inconvenient if
several child files are to be compiled and
to be kept for distribution.
\end{itemize}

The present package provides a simple interface
to make child files individually compilable by \LaTeX{}.
Compiling a child file then has the same effect as compiling
the main file with an |\includeonly| command
to select the appropriate child.
Moreover the generated document will carry the name of the child
rather than the main file.
This resolves all three above issues.

This feature is meant to make the editing of books,
thesis documents and lecture notes somewhat more convenient.
However, the package can also be used efficiently for
composing a series of documents (such as exercise sheets)
which are typically distributed individually.
It then assists the author in generating the individual documents
(potentially in different versions)
as well as a document containing the collected series.
Another application is in developing style files
or other kinds of included material
where compilation of the style file could redirect
to a sample or test file.

%%%%%%%%%%%%%%%%%%%%%%%%%%%%%%%%%%%%%%%%%%%%%%%%%%%%%%%%%%%%%%%%%%%%%%%%%%%%%%%%
%%%%%%%%%%%%%%%%%%%%%%%%%%%%%%%%%%%%%%%%%%%%%%%%%%%%%%%%%%%%%%%%%%%%%%%%%%%%%%%%
\section{Usage}

First of all, the package \textsf{childdoc} is \emph{not} a standard
\LaTeXe{} |.sty| style file! Therefore it needs to be invoked in
a non-standard way.

%%%%%%%%%%%%%%%%%%%%%%%%%%%%%%%%%%%%%%%%%%%%%%%%%%%%%%%%%%%%%%%%%%%%%%%%%%%%%%%%
\subsection{Included Files}
\label{sec:include}

%%%%%%%%%%%%%%%%%%%%%%%%%%%%%%%%%%%%%%%%
\DescribeMacro{\childdocmain}
To use the package, add the commands
\begin{center}
\begin{tabular}{l}
|\input{childdoc.def}|\\
|\childdocmain{}|\\
\end{tabular}
\end{center}
at the very top of the main \LaTeX{} file,
in particular \emph{before} the |\documentclass| statement!
The argument of |\childdocmain| should be left empty
(but it must be present).

%%%%%%%%%%%%%%%%%%%%%%%%%%%%%%%%%%%%%%%%
\DescribeMacro{\childdocof}
Furthermore, add the commands
\begin{center}
\begin{tabular}{l}
|\input{childdoc.def}|\\
|\childdocof{|\textit{main}|}|\\
\end{tabular}
\end{center}
at the top of every child file \textit{child}
which is included by |\include{|\textit{child}|}|
from within the main file
(or at least for those files to be compiled individually).
The argument \textit{main} must be the filename of the main file.

There are a couple of
considerations in setting up the main and child documents:

%%%%%%%%%%%%%%%%%%%%%%%%%%%%%%%%%%%%%%%%
\paragraph{Restrictions.}

Please note the following restrictions:
\begin{itemize}
\item
|\childdocmain| must be called with one argument \textit{main}
to ensure compatibility with earlier version of the package.
It must either be empty (|\childdocmain{}|)
or precisely match the filename of the main file in which it is specified.
See \secref{sec:detection} for further information.
\item
The filename \textit{main} must be specified without the |.tex| extension.
\item
The filename \textit{main} is case sensitive
(even in case-insensitive file systems)
due to internal string comparison.
\item
The argument \textit{main} should be fully expanded, it cannot be a macro.
\item
Subdirectories and special characters should be avoided in filenames.
\item
The command |\childdocmain{|\textit{main}|}| must be followed by a whitespace.
It should not be followed immediately by another command
or by a comment mark `|%|'.
This is because the \TeX{} parser reads the token immediately following
the argument of |\childdocmain| and puts it
at the beginning of every child section;
however, a white\-space is ignored.
\end{itemize}

%%%%%%%%%%%%%%%%%%%%%%%%%%%%%%%%%%%%%%%%
\paragraph{Content of Main File.}

It is advisable to place all content in the child files included by |\include|.
Any output contained in the main file will appear in all child documents
unless suppressed manually;
it cannot be suppressed automatically by the |\includeonly| directive
and thus should normally be avoided.
A method to include some content in the main file
by means of conditional processing is described in \secref{sec:conditional}.

%%%%%%%%%%%%%%%%%%%%%%%%%%%%%%%%%%%%%%%%
\paragraph{Page Numbering.}

When only a part of the document is compiled,
the appropriate numbering of pages
(as well as other status parameters)
is determined from the |.aux| files.
The latter contain information from previous passes.
However this information needs to propagate through
all intermediate child documents.
Therefore the page numbering in child documents may well
be inconsistent until the complete document is compiled at least once.

A useful (if unconventional) way to always ensure a consistent
page numbering is to restart the numbering in each child document
and denote the pages by `\textit{child}|.|\textit{page}'
where \textit{child} represents the chapter/section number of the child file.
This can be achieved by the command
|\numberwithin{page}{|\textit{child}|}|
of the \textsf{amsmath} package
where \textit{child} can be |chapter| or |section|
depending on the chosen structuring.
Alternatively, one can modify the macro |\thepage| appropriately
and reset the counter |page| at the start of each child file.

%%%%%%%%%%%%%%%%%%%%%%%%%%%%%%%%%%%%%%%%%%%%%%%%%%%%%%%%%%%%%%%%%%%%%%%%%%%%%%%%
\subsection{Conditional Processing}
\label{sec:conditional}

The package provides a mechanism to compile different versions
of a document. To customise the versions further some conditional processing
can come in handy to distinguish which version is being compiled.
The package provides two macros to describe the compilation context:

%%%%%%%%%%%%%%%%%%%%%%%%%%%%%%%%%%%%%%%%
\DescribeMacro{\ifchilddoc}
The conditional |\ifchilddoc| distinguishes between the compilation of
child documents and the main document:
%
\begin{center}
|\ifchilddoc |\textit{child-code}| |[|\||else |\textit{main-code}]| \||fi|
\end{center}

%%%%%%%%%%%%%%%%%%%%%%%%%%%%%%%%%%%%%%%%
\DescribeMacro{\childdocname}
\DescribeMacro{\childdocjob}
The macro |\childdocname| contains the filename (without extension)
of the main or child file being processed.
Note that |\childdocjob| will always contain the name of the main file.

%%%%%%%%%%%%%%%%%%%%%%%%%%%%%%%%%%%%%%%%
\paragraph{Title Page.}

Conditional processing can be used to include a title or banner page
in the main document when proper precautions are taken.
Importantly, the code in the main file should ensure that the page counter
(as well as other status parameters which are stored in the |.aux| files)
takes the same value after the conditional processing.
Otherwise the page numbers may take divergent values
depending on which part is compiled.

For example, a title page could be declared by:
%
\begin{center}
\begin{tabular}{l}
|\ifchilddoc\||else|\\
|\addtocounter{page}{-1}|\\
\textit{code for title page}\\
|\newpage|\\
|\||fi|
\end{tabular}
\end{center}
%
A banner page for the child documents can be generated by:
%
\begin{center}
\begin{tabular}{l}
|\ifchilddoc|\\
|\addtocounter{page}{-1}|\\
\textit{code for banner page}\\
|\newpage|\\
|\||fi|
\end{tabular}
\end{center}
%
Here one could write a message such as:
\begin{center}
|This is the part \childdocname{} of \childdocjob{}.|
\end{center}

%%%%%%%%%%%%%%%%%%%%%%%%%%%%%%%%%%%%%%%%%%%%%%%%%%%%%%%%%%%%%%%%%%%%%%%%%%%%%%%%
\subsection{Flags}
\label{sec:flags}

The package makes it easy to generate different versions
of the main or child documents.
To this end compilation flags can be defined
and assigned different default values.
They will be particularly useful in conjunction
with the forwarding mechanism described in \secref{sec:forward}.

For example, it may be useful to have a flag |\version|
which can be set to |draft| or |final|.
The document source will contain some conditional code
depending on the value of |\version|.
Suppose further, the flag should default to |final| for the main file
and to |draft| for child files
which is a natural assignment for editing the document.
This is achieved by placing the following code
in the preamble of the main document
(below the |\childdocmain| directive):
%
\begin{center}
\begin{tabular}{l}
|\ifchilddoc|\\
|\providecommand{\version}{draft}|\\
|\||else|\\
|\providecommand{\version}{final}|\\
|\||fi|
\end{tabular}
\end{center}
%
The definition by |\providecommand| makes sure
that previous definitions are not overwritten.
Further statements |\providecommand{\version}{...}|
can thus be added before the above code to override it.

For the main file, one might add a line
(between |\childdocmain| and the above block)
%
\begin{center}
|%\ifchilddoc\||else\providecommand{\version}{draft}\||fi|
\end{center}
%
which can be uncommented to produce a draft version.
Likewise one can add a line to the very top of a child file
(above the |\childdocof{|\textit{main}|}| directive)
%
\begin{center}
|%\providecommand{\version}{final}|
\end{center}
%
which can be uncommented to produce the final version of this child document.

%%%%%%%%%%%%%%%%%%%%%%%%%%%%%%%%%%%%%%%%%%%%%%%%%%%%%%%%%%%%%%%%%%%%%%%%%%%%%%%%
\subsection{Forwarding}
\label{sec:forward}

Different versions of the main or child documents
using compilation flags as described in \secref{sec:flags}
can be (permanently) stored in different files
for convenient compilation, viewing and distribution.
To this end, the package defines a command
to pass on compilation to a different file:

%%%%%%%%%%%%%%%%%%%%%%%%%%%%%%%%%%%%%%%%
\DescribeMacro{\childdocforward}
The command |\childdocforward| redirects processing to
another source file:
%
\begin{center}
\begin{tabular}{l}
|\input{childdoc.def}|\\
|\childdocforward[|\textit{main}|]{|\textit{dest}|}|\\
\end{tabular}
\end{center}
%
The argument \textit{dest} is the destination file
(without extension).
It should be the main file or one of the child files.
Note that further \textsf{childdoc} directives
such as |\childdocof| and |\childdocforward|
in the indicated file will be processed in this form.
The optional argument \textit{main}
passes on directly to the main file \textit{main}
while pretending to compile the child \textit{dest}.
This form behaves as if \textit{dest}
issues |\childdocof{|\textit{main}|}| right away,
and no further \textsf{childdoc} directives will be processed.

%%%%%%%%%%%%%%%%%%%%%%%%%%%%%%%%%%%%%%%%
\DescribeMacro{\...prefix}
In the alternative form |\childdocforwardprefix|,
%
\begin{center}
\begin{tabular}{l}
|\input{childdoc.def}|\\
|\childdocforwardprefix[|\textit{main}|]{|\textit{prefix}|}{|\textit{dest}|}|
\end{tabular}
\end{center}
%
the destination file is determined by a pattern
depending on the current file:
To make this work, the current file must be called
`{\textit{prefix}\hspace{0.2em}\textit{suffix}}'
with \textit{prefix} matching precisely the argument.
Processing is then passed on to the file
`{\textit{dest}\hspace{0.2em}\textit{suffix}}'.
Surely, the same effect is achieved by
directly specifying the
argument `{\textit{dest}\hspace{0.2em}\textit{suffix}}'
in the first form.
However, that requires to set up a different file
for each child. With the alternative form of the command
all these files can have exactly the same content
which simplifies setting them up and maintaining them.

For example, the following file |draft.tex|
with a compilation flag |\version| as described in \secref{sec:flags}
compiles the main document as a draft:
%
\begin{center}
\begin{tabular}{l}
|\def\version{draft}|\\
|\input{childdoc.def}|\\
|\childdocforward{|\textit{main}|}|
\end{tabular}
\end{center}
%
Likewise, the following files |final|\textit{nn}|.tex|
compile the final version of the child document
|child|\textit{nn}|.tex|:
%
\begin{center}
\begin{tabular}{l}
|\def\version{final}|\\
|\input{childdoc.def}|\\
|\childdocforwardprefix{final}{child}|
\end{tabular}
\end{center}
%

Note that when several versions of a main file and/or of each child file
are to be generated, it may be convenient to set up a |Makefile| or
shell script to automatise the process.

%%%%%%%%%%%%%%%%%%%%%%%%%%%%%%%%%%%%%%%%%%%%%%%%%%%%%%%%%%%%%%%%%%%%%%%%%%%%%%%%
\subsection{Command Line Processing}
\label{sec:commandline}

The effect of redirection files can also be achieved by invoking
the \LaTeX{} compiler with a more elaborate command line.
Most conveniently this should be done as part
of a shell script or a |Makefile|.

When using \textsf{childdoc} in the main file, the following
command lines effectively perform a redirection
(note that depending on the shell being used,
backslashes may have to be doubled: `|\|' $\to$ `|\\|'):
%
\begin{center}
|... -jobname "|\textit{target}|" |\\|"|[\textit{flags}]%
|\input{childdoc.def}\childdocforward[|\textit{main}|]{|\textit{dest}|}"|
\end{center}
%
Here \textit{target} is the name of the output file,
\textit{main} is the name of the main file
and \textit{dest} is the name of the main or child file to be processed
(all filenames without extensions).
The optional argument \textit{main} can be omitted
if \textit{main} matches \textit{dest}.
Optionally, compilation \textit{flags} can be defined via |\def| commands.
This command line makes the \TeX{} engine believe
it is compiling the file \textit{target}
whose content is specified as the latter parameter.
The provided code then forwards the processing to
\textit{main} or \textit{dest} as described in \secref{sec:forward}.

%%%%%%%%%%%%%%%%%%%%%%%%%%%%%%%%%%%%%%%%%%%%%%%%%%%%%%%%%%%%%%%%%%%%%%%%%%%%%%%%
\subsection{Include by Input}
\label{sec:input}

Including child documents by |\include| has some restrictions by design.
Most notably, the content of a child document always occupies
its own set of pages; pages cannot be shared between child documents.
Usually, this behaviour makes perfect sense
because each child document contain an essential part of the document.
However, in some situations it may be desirable to compose
a document from a collection of parts
without having mandatory page breaks between then.
For this case, the package
provides a mechanism to include parts
by |\input| which can also be processed individually.
However, by construction this mechanism
requires manual handling of the content to be output.

%%%%%%%%%%%%%%%%%%%%%%%%%%%%%%%%%%%%%%%%
\DescribeMacro{\ifchilddocmanual}
The main file should be prepared as usual, see \secref{sec:include}.
However, the document body must make a distinction
between processing of an individual part and of the main document, e.g.:
%
\begin{center}
\begin{tabular}{l}
|\ifchilddocmanual|\\
|\input{\childdocname}|\\
|\||else|\\
\textit{document body with }|\input{|\textit{part}|}|\\
|\||fi|
\end{tabular}
\end{center}
%
The conditional |\ifchilddocmanual| is true whenever
a part to be included by |\input| is being compiled,
and the name of the part is stored in |\childdocname|.

%%%%%%%%%%%%%%%%%%%%%%%%%%%%%%%%%%%%%%%%
\DescribeMacro{\childdocby}
Each part to be included by |\input| should start with:
%
\begin{center}
\begin{tabular}{l}
|\input{childdoc.def}|\\
|\childdocby{|\textit{main}|}|\\
\end{tabular}
\end{center}
%
The directive |\childdocby| is similar to |\childdocof|
described in \secref{sec:include},
but the subsequent selection of content must be done manually.
To that end, both |\ifchilddoc| and |\ifchilddocmanual|
will be true upon processing of a part,
and the name of the part is stored in |\childdocname|.
Note that |\jobname| will be set to the filename of the current part
so that each part receives an individual |.aux| file
that does not interfere with the |.aux| file(s) of the main document.
This behaviour can be altered by the alternative form
|\childdocby[*]{|\textit{main}|}| (with a non-empty optional argument)
which uses the |.aux| file of the main document
by setting |\jobname| to \textit{main}.

%%%%%%%%%%%%%%%%%%%%%%%%%%%%%%%%%%%%%%%%%%%%%%%%%%%%%%%%%%%%%%%%%%%%%%%%%%%%%%%%
\subsection{Driver Development}
\label{sec:driver}

The \textsf{childdoc} mechanism can also be use for the development
of definition files such as \LaTeX{} styles or classes.
This case differs from the above setup with multiple parts
included by |\include| in that no |\includeonly| should be invoked.
This can be achieved by starting the include file
(before |\ProvidesPackage|) with:
%
\begin{center}
\begin{tabular}{l}
|\input{childdoc.def}|\\
|\childdocforward{|\textit{main}|}|\\
\end{tabular}
\end{center}
%
or alternatively with:
%
\begin{center}
\begin{tabular}{l}
|\input{childdoc.def}|\\
|\childdocby{|\textit{main}|}|\\
\end{tabular}
\end{center}
%
Both forms have slightly different effects as described above.
The main file is prepared as usual, see \secref{sec:include}.

%%%%%%%%%%%%%%%%%%%%%%%%%%%%%%%%%%%%%%%%%%%%%%%%%%%%%%%%%%%%%%%%%%%%%%%%%%%%%%%%
\subsection{Legacy Detection}
\label{sec:detection}

The directive |\childdocmain| in the main file can detect
whether the complete document or merely a child is to be compiled
even without using the directive |\childdocof|.
This method is deprecated because it is less robust
and there is no compelling reason to use it;
it is merely provided for backward compatibility
and it may be removed in future versions.

If the detection mechanism is to be used,
it is mandatory to correctly specify
the filename of the main file as the argument of |\childdocmain|:
%
\begin{center}
\begin{tabular}{l}
|\input{childdoc.def}|\\
|\childdocmain{|\textit{main}|}|\\
\end{tabular}
\end{center}
%
If |\jobname| does not match the argument \textit{main} of |\childdocmain|,
it is assumed that |\jobname| points to the child file to be compiled.
When using |\childdocmain| with the main file specified as argument,
it suffices to start a child file
with just |\input{|\textit{main}|}|
without loading of the package and using |\childdocof|.
If instead all processing is done
with the appropriate \textsf{childdoc} directives,
the argument of \textit{main} of |\childdocmain| can be empty.

An alternative version of the command line processing described
in \secref{sec:commandline} using the detection mechanism reads:
%
\begin{center}
|... -jobname "|\textit{target}|" "|[\textit{flags}]%
[|\def\jobname{|\textit{dest}|}|]|\input{|\textit{main}|}"|
\end{center}

%%%%%%%%%%%%%%%%%%%%%%%%%%%%%%%%%%%%%%%%%%%%%%%%%%%%%%%%%%%%%%%%%%%%%%%%%%%%%%%%
\subsection{Manual Code}
\label{sec:manual}

In case one cannot be certain whether the definitions file |childdoc.def|
is installed on the target \TeX{} distribution
and one prefers not to ship it,
it is conceivable to paste a few relevant commands into the sources.

To that end, drop all statements |\input{childdoc.def}|
and perform the replacements as outlined below.
Instead of |\childdocmain{|\textit{main}|}| add the following code
to the top of the main file:
%
\begin{center}
\begin{tabular}{l}
|\||ifdefined\childdocname\endinput\||fi\newif\ifchilddoc|\\
|\edef\childdocname{\scantokens\expandafter{\jobname\noexpand}}|\\
|\def\childdocmain{|\textit{main}|}\||ifx\childdocmain\childdocname\||else|\\
|\childdoctrue\includeonly{\childdocname}\let\jobname\childdocmain\||fi|\\
\end{tabular}
\end{center}
%
Instead of |\childdocof{|\textit{main}|}| just include the main file
at the top of each child file:
%
\begin{center}
|\input{|\textit{main}|}|
\end{center}
%
A simple redirection |\childdocforward{|\textit{dest}|}| is achieved by:
%
\begin{center}
|\def\jobname{|\textit{dest}|}\input{\jobname}|
\end{center}
%
The redirection with prefix
|\childdocforwardprefix[|\textit{prefix}|]{|\textit{dest}|}|
is accomplished by:
%
\begin{center}
\begin{tabular}{l}
|{\edef\jobname{\scantokens\expandafter{\jobname\noexpand}}|\\
|\def\redirectjob |\textit{prefix}|#1~~~{\gdef\jobname{|\textit{dest}|#1}}|\\
|\expandafter\redirectjob\jobname~~~}\input{\jobname}|
\end{tabular}
\end{center}

In an alternative approach,
child documents can be compiled by a specific command line
without additional code or specific definitions:
%
\begin{center}
|... -jobname "|\textit{target}|" "|[\textit{flags}]%
|\includeonly{|\textit{dest}|}\input{|\textit{main}|}"|
\end{center}
%

%%%%%%%%%%%%%%%%%%%%%%%%%%%%%%%%%%%%%%%%%%%%%%%%%%%%%%%%%%%%%%%%%%%%%%%%%%%%%%%%
%%%%%%%%%%%%%%%%%%%%%%%%%%%%%%%%%%%%%%%%%%%%%%%%%%%%%%%%%%%%%%%%%%%%%%%%%%%%%%%%
\section{Information}

%%%%%%%%%%%%%%%%%%%%%%%%%%%%%%%%%%%%%%%%%%%%%%%%%%%%%%%%%%%%%%%%%%%%%%%%%%%%%%%%
\subsection{Copyright}

Copyright \copyright{} 2017--2018 Niklas Beisert

This work may be distributed and/or modified under the
conditions of the \LaTeX{} Project Public License, either version 1.3
of this license or (at your option) any later version.
The latest version of this license is in
  \url{http://www.latex-project.org/lppl.txt}
and version 1.3 or later is part of all distributions of \LaTeX{}
version 2005/12/01 or later.

This work has the LPPL maintenance status `maintained'.

The Current Maintainer of this work is Niklas Beisert.

This work consists of the files |README.txt|, |childdoc.ins| and |childdoc.dtx|
as well as the derived files |childdoc.def|, |cdocsamp.tex|
with |cdocsch1.tex|, |cdocsch2.tex|, |cdocspt3.tex|, |cdocspt4.tex|,
|cdocsdrf.tex|, |cdocsfn1.tex|, |cdocsfn2.tex|
as well as |childdoc.pdf|.

%%%%%%%%%%%%%%%%%%%%%%%%%%%%%%%%%%%%%%%%%%%%%%%%%%%%%%%%%%%%%%%%%%%%%%%%%%%%%%%%
\subsection{Files and Installation}

The package consists of the files:
%
\begin{center}
\begin{tabular}{ll}
    |README.txt|   & readme file \\
    |childdoc.ins| & installation file \\
    |childdoc.dtx| & source file \\
    |childdoc.def| & definition file \\
    |cdocsamp.tex| & sample main file \\
    |cdocsch1.tex| & sample include file \\
    |cdocsch2.tex| & sample include file \\
    |cdocspt3.tex| & sample part file \\
    |cdocspt4.tex| & sample part file \\
    |cdocsdrf.tex| & sample redirection file \\
    |cdocsfn1.tex| & sample redirection file \\
    |cdocsfn2.tex| & sample redirection file \\
    |childdoc.pdf| & manual
\end{tabular}
\end{center}
%
The distribution consists of the files
|README.txt|, |childdoc.ins| and |childdoc.dtx|.
%
\begin{itemize}
\item
Run (pdf)\LaTeX{} on |childdoc.dtx|
to compile the manual |childdoc.pdf| (this file).
\item
Run \LaTeX{} on |childdoc.ins| to create the definitions file |childdoc.def|
and the sample |cdocsamp.tex| with include files
|cdocsch1.tex|, |cdocsch2.tex|, |cdocspt3.tex|, |cdocspt4.tex|,
|cdocsdrf.tex|, |cdocsfn1.tex|, |cdocsfn2.tex|.
Then copy the file |childdoc.def| to an appropriate directory of your \LaTeX{}
distribution, e.g.\ \textit{texmf-root}|/tex/latex/childdoc|.
\end{itemize}

%%%%%%%%%%%%%%%%%%%%%%%%%%%%%%%%%%%%%%%%%%%%%%%%%%%%%%%%%%%%%%%%%%%%%%%%%%%%%%%%
\subsection{Related CTAN Packages}

There are several other packages which offer a similar functionality:
%
\begin{itemize}
\item
The packages
\href{http://ctan.org/pkg/docmute}{\textsf{docmute}},
\href{http://ctan.org/pkg/includex}{\textsf{includex}} and
\href{http://ctan.org/pkg/standalone}{\textsf{standalone}}
provide commands to include only the document body of
a child file thus allowing both files to be compiled individually.
\item
The packages \href{http://ctan.org/pkg/subdocs}{\textsf{subdocs}}
and \href{http://ctan.org/pkg/subfiles}{\textsf{subfiles}}
provide structures in which the main and child documents can be
encapsulated and allowing them to be compiled individually.
The inclusion mechanism is different from the conventional |\include|.
\item
The package \href{http://ctan.org/pkg/combine}{\textsf{combine}}
is an elaborate solution to combine several documents into one.
\end{itemize}
%
See also the CTAN topic \href{http://ctan.org/topic/subdocs}{\textsf{subdocs}}
for further related packages.
The present package differs from the above solutions in that
a document structure constructed with the conventional |\include| mechanism
just needs two extra commands at the top of every file
such that all constituent files can be compiled individually.

%%%%%%%%%%%%%%%%%%%%%%%%%%%%%%%%%%%%%%%%%%%%%%%%%%%%%%%%%%%%%%%%%%%%%%%%%%%%%%%%
%\subsection{Feature Suggestions}
%
%The following is a list of features which may be useful for future
%versions of this package:
%%
%\begin{itemize}
%\item
%\ldots
%\end{itemize}

%%%%%%%%%%%%%%%%%%%%%%%%%%%%%%%%%%%%%%%%%%%%%%%%%%%%%%%%%%%%%%%%%%%%%%%%%%%%%%%%
\subsection{Revision History}

%%%%%%%%%%%%%%%%%%%%%%%%%%%%%%%%%%%%%%%%
\paragraph{v2.0:} 2018/12/30

\begin{itemize}
\item
immediate forward processing
\item
added |\childdocby| mechanism
\item
manual restructured
\end{itemize}

%%%%%%%%%%%%%%%%%%%%%%%%%%%%%%%%%%%%%%%%
\paragraph{v1.6:} 2018/01/17

\begin{itemize}
\item
application for development of include files
\item
corrections to manual
\end{itemize}

%%%%%%%%%%%%%%%%%%%%%%%%%%%%%%%%%%%%%%%%
\paragraph{v1.5:} 2017/05/21

\begin{itemize}
\item
more complete structuring introduced
\item
|\childdocof| introduced
\item
|\childdoc| renamed to |\childdocmain|
\item
|\childredirect| renamed to |\childdocforward| and |\childdocforwardprefix|
and functionality expanded
\end{itemize}

%%%%%%%%%%%%%%%%%%%%%%%%%%%%%%%%%%%%%%%%
\paragraph{v1.0:} 2017/04/27

\begin{itemize}
\item
manual and install package
\item
first version published on CTAN
\end{itemize}

%%%%%%%%%%%%%%%%%%%%%%%%%%%%%%%%%%%%%%%%
\paragraph{v0.6:} 2017/04/26

\begin{itemize}
\item
redirection mechanism added
\end{itemize}

%%%%%%%%%%%%%%%%%%%%%%%%%%%%%%%%%%%%%%%%
\paragraph{v0.5:} 2017/04/26

\begin{itemize}
\item
functionality in definition file
\end{itemize}


%%%%%%%%%%%%%%%%%%%%%%%%%%%%%%%%%%%%%%%%%%%%%%%%%%%%%%%%%%%%%%%%%%%%%%%%%%%%%%%%
%%%%%%%%%%%%%%%%%%%%%%%%%%%%%%%%%%%%%%%%%%%%%%%%%%%%%%%%%%%%%%%%%%%%%%%%%%%%%%%%
%%%%%%%%%%%%%%%%%%%%%%%%%%%%%%%%%%%%%%%%%%%%%%%%%%%%%%%%%%%%%%%%%%%%%%%%%%%%%%%%
\appendix

\settowidth\MacroIndent{\rmfamily\scriptsize 000\ }

 \DocInput{childdoc.dtx}

\end{document}
%</driver>
% \fi
%
% %%%%%%%%%%%%%%%%%%%%%%%%%%%%%%%%%%%%%%%%%%%%%%%%%%%%%%%%%%%%%%%%%%%%%%%%%%%%%%
% %%%%%%%%%%%%%%%%%%%%%%%%%%%%%%%%%%%%%%%%%%%%%%%%%%%%%%%%%%%%%%%%%%%%%%%%%%%%%%
% \section{Sample}
%\iffalse
%<*samplemain>
%\fi
%
% The following presents a sample document
% with two chapters, two parts, a title page,
% a compile flag as well as three forwarding files to set the flag.
% It consists of eight |.tex| files:
% \begin{center}
% \begin{tabular}{ll}
% |cdocsamp.tex|&main file\\
% |cdocsch1.tex|&include file for chapter 1\\
% |cdocsch2.tex|&include file for chapter 2\\
% |cdocspt3.tex|&include file for part 3\\
% |cdocspt4.tex|&include file for part 4\\
% |cdocsdrf.tex|&forwarding file for main file in draft mode\\
% |cdocsfi1.tex|&forwarding file for final version of chapter 1\\
% |cdocsfi2.tex|&forwarding file for final version of chapter 2\\
% \end{tabular}
% \end{center}
% Each of the eight files can be compiled directly by the \LaTeX{} compiler.
%
% %%%%%%%%%%%%%%%%%%%%%%%%%%%%%%%%%%%%%%
% \paragraph{Main File.}
%
% The main file is called |cdocsamp.tex|.
%
% Load the \textsf{childdoc} definitions and
% declare the filename for the main document:
%    \begin{macrocode}
\input{childdoc.def}
\childdocmain{}
%    \end{macrocode}

% Optional override for |\version| flag:
%    \begin{macrocode}
%%\ifchilddoc\else\providecommand{\version}{draft}\fi
%    \end{macrocode}

% Define the default values for the |\version| flag
% (|final| for the main file and |draft| for childs):
%    \begin{macrocode}
\ifchilddoc
\providecommand{\version}{draft}
\else
\providecommand{\version}{final}
\fi
%    \end{macrocode}

% Load the standard document class:
%    \begin{macrocode}
\documentclass[12pt]{article}
%    \end{macrocode}

% Start the document body:
%    \begin{macrocode}
\begin{document}
%    \end{macrocode}

% Declare a title page.
% Print title, part of document being processed and version flag:
%    \begin{macrocode}
\addtocounter{page}{-1}
\begin{center}
{\LARGE\bfseries{}childdoc example\par}
\vspace{1cm}
\ifchilddoc
\ifchilddocmanual part\else chapter\fi:
`\childdocname' of `\childdocjob'\par
\else
main document: `\childdocjob'\par
\fi
version: \version\par
\end{center}
\newpage
%    \end{macrocode}

% Manually include selected file,
% otherwise process as usual:
%    \begin{macrocode}
\ifchilddocmanual
\section*{part `\childdocname'}
\input{\childdocname}
\else
%    \end{macrocode}

% Include the two chapters:
%    \begin{macrocode}
\include{cdocsch1}
\include{cdocsch2}
%    \end{macrocode}

% Include the two parts unless only chapters should be displayed:
%    \begin{macrocode}
\ifchilddoc\else
\section{part three}
\input{cdocspt3}
\section{part four}
\input{cdocspt4}
\fi
%    \end{macrocode}

% Process as usual until here:
%    \begin{macrocode}
\fi
%    \end{macrocode}

% End of document body:
%    \begin{macrocode}
\end{document}
%    \end{macrocode}
%\iffalse
%</samplemain>
%\fi
%
% %%%%%%%%%%%%%%%%%%%%%%%%%%%%%%%%%%%%%%
% \paragraph{Chapter Include Files.}
%
% The include files are called |cdocsch1.tex| and |cdocsch2.tex|.
%
%\iffalse
%<*samplechap1|samplechap2>
%\fi

% Optional override for |\version| flag:
%    \begin{macrocode}
%%\providecommand{\version}{final}
%    \end{macrocode}

% Include the main document:
%    \begin{macrocode}
\input{childdoc.def}
\childdocof{cdocsamp}
%    \end{macrocode}

%\iffalse
%</samplechap1|samplechap2>
%\fi
%
%\iffalse
%<*samplechap1>
%\fi
% Some text for chapter 1:
%    \begin{macrocode}
\section{one}
some text in chapter one
%    \end{macrocode}

%\iffalse
%</samplechap1>
%\fi
% Some text for chapter 2:
%\iffalse
%<*samplechap2>
%\fi
%    \begin{macrocode}
\section{two}
more text in chapter two
%    \end{macrocode}

%\iffalse
%</samplechap2>
%\fi
%
% %%%%%%%%%%%%%%%%%%%%%%%%%%%%%%%%%%%%%%
% \paragraph{Part Include Files.}
%
% The include files are called |cdocspt3.tex| and |cdocspt4.tex|.
%
%\iffalse
%<*samplepart3|samplepart4>
%\fi

% Optional override for |\version| flag:
%    \begin{macrocode}
%%\providecommand{\version}{final}
%    \end{macrocode}

% Include the main document:
%    \begin{macrocode}
\input{childdoc.def}
\childdocby{cdocsamp}
%    \end{macrocode}

%\iffalse
%</samplepart3|samplepart4>
%\fi
%
%\iffalse
%<*samplepart3>
%\fi
% Some text for part 3:
%    \begin{macrocode}
some text in part three
%    \end{macrocode}

%\iffalse
%</samplepart3>
%\fi
% Some text for part 4:
%\iffalse
%<*samplepart4>
%\fi
%    \begin{macrocode}
more text in part four
%    \end{macrocode}

%\iffalse
%</samplepart4>
%\fi
%
% %%%%%%%%%%%%%%%%%%%%%%%%%%%%%%%%%%%%%%
% \paragraph{Forwarding for a Complete Draft.}
%
% The following forwarding file |cdocsdrf.tex|
% compiles the main document in draft mode:
%\iffalse
%<*sampledraft>
%\fi
%    \begin{macrocode}
\def\version{draft}
\input{childdoc.def}
\childdocforward{cdocsamp}
%    \end{macrocode}

%\iffalse
%</sampledraft>
%\fi
%
% %%%%%%%%%%%%%%%%%%%%%%%%%%%%%%%%%%%%%%
% \paragraph{Forwarding for Final Version of the Chapters.}
%
% The following forwarding files |cdocsfn1.tex| and |cdocsfn2.tex|
% (with identical content)
% compile the final versions of the child documents
% |cdocsch1.tex| and |cdocsch2.tex|, respectively:
%\iffalse
%<*samplefinal>
%\fi
%    \begin{macrocode}
\def\version{final}
\input{childdoc.def}
\childdocforwardprefix[cdocsamp]{cdocsfn}{cdocsch}
%    \end{macrocode}

%\iffalse
%</samplefinal>
%\fi
%
% %%%%%%%%%%%%%%%%%%%%%%%%%%%%%%%%%%%%%%
% \paragraph{Command Line Processing.}
%
% The following three command lines generate the output files
% |cdocscld|, |cdocscl1| and |cdocscl2|
% which should be identical to
% |cdocsdrf|, |cdocsch1| and |cdocsfn2|, respectively:
% \begin{center}
% \begin{tabular}{l}
% |latex -jobname cdocscld \|\\
% |  "\def\version{draft}\input{childdoc.def}\childdocforward{cdocsamp}"|\\
% |latex -jobname cdocscl1 \|\\
% |  "\input{childdoc.def}\childdocforward[cdocsamp]{cdocsch1}"|\\
% |latex -jobname cdocscl2 \|\\
% |  "\def\version{final}\input{childdoc.def}\childdocforward{cdocsch2}"|
% \end{tabular}
% \end{center}
% Note that the trailing backslash on each first line
% merely continues the input to the second line
% (for convenient cut ant paste).
% Furthermore, the command |latex| can be replaced by any
% of its alternative versions such as |pdflatex|.
%
% %%%%%%%%%%%%%%%%%%%%%%%%%%%%%%%%%%%%%%%%%%%%%%%%%%%%%%%%%%%%%%%%%%%%%%%%%%%%%%
% %%%%%%%%%%%%%%%%%%%%%%%%%%%%%%%%%%%%%%%%%%%%%%%%%%%%%%%%%%%%%%%%%%%%%%%%%%%%%%
% \section{Implementation}
%\iffalse
%<*package>
%\fi
%
% This section describes the definitions file |childdoc.def|.

% The definitions cannot be loaded using |\usepackage| or |\RequirePackage|
% which has a mechanism to prevent loading a style file more than once.
% When loading the definitions by means of |\input|
% multiple instances have to be prevented manually:
%\iffalse
%This code needs to be before the `\ProvidesFile' directive
%which is defined at the beginning of this file.
%Therefore it is also placed there and commented out here.
%</package>
%<*discard>
%\fi
%    \begin{macrocode}
\ifdefined\childdocmain\endinput\fi
%    \end{macrocode}
%\iffalse
%</discard>
%<*package>
%\fi
%
% \macro{\ifchilddoc}
% \macro{\ifchilddocmanual}
% The conditional |\ifchilddoc| tells whether a
% child (true) or main (false) document is being compiled.
% The conditional |\ifchilddocmanual| tells whether
% the |\includeonly| mechanism is used (false) or
% the selection of child files must be performed manually (true).
% The definitions initialise to false:
%    \begin{macrocode}
\newif\ifchilddoc
\newif\ifchilddocmanual
%    \end{macrocode}

% \macro{\childdocname}
% \macro{\childdocjob}
% The macro |\childdocname| stores the name of the main document
% to be compiled. The macro |\childdocjob| stores the name of
% the document on which the \LaTeX{} compiler was originally invoked.
% The content of |\jobname| cannot be compared
% to filenames specified in the source due to different catcodes.
% The following code rescans |\jobname|, stores the result
% in |\childdocname| and saves a copy in |\childdocjob|:
%    \begin{macrocode}
\edef\childdocname{\scantokens\expandafter{\jobname\noexpand}}
\let\childdocjob\childdocname
%    \end{macrocode}

% \macro{\childdocdisable}
% The macro |\childdocdisable| prevents the main file
% from being processed more than once.
% At this stage, the main document command |\childdocmain|
% is assumed to be called once again where it should do nothing.
% Any subsequent call to it should prevent
% a secondary processing of the main document
% It overwrites the forwarding commands
% |\childdocof| and |\childdocforward|
% with empty macros to prevent further inclusions of the main document:
%    \begin{macrocode}
\newcommand{\childdocdisable}
{
  \renewcommand{\childdocmain}[1]{\renewcommand{\childdocmain}[1]{\endinput}}
  \renewcommand{\childdocof}[1]{}
  \renewcommand{\childdocby}[2][]{}
  \renewcommand{\childdocforward}[2][]{}
  \renewcommand{\childdocdisable}{}
}
%    \end{macrocode}

% \macro{\childdocmain}
% The macro |\childdocmain| is to be called at the top of the main file
% with nothing or the main filename (without extension) as argument.
% First, it breaks loops.
% If the argument is not empty and does not match |\childdocname|
% (which is set by the first inclusion of |childdoc.def|),
% |\ifchilddoc| is set to true, |\includeonly| is applied to the child file
% and |\jobname| is set to the main file
% (for proper handling of |.aux| files):
%    \begin{macrocode}
\newcommand{\childdocmain}[1]
{
  \childdocdisable\childdocmain{}
  \if?#1?\else
    \begingroup
      \def\childdoctmp{#1}
      \ifx\childdoctmp\childdocname
        \def\childdoctmp{}
      \else
        \def\childdoctmp
        {
          \childdoctrue
          \includeonly{\childdocname}
          \def\childdocjob{#1}
          \def\jobname{#1}
        }
      \fi
      \expandafter
    \endgroup
    \childdoctmp
  \fi
}
%    \end{macrocode}

% \macro{\childdocof}
% The command |\childdocof| redirects
% compilation to the main file |#1|.
%    \begin{macrocode}
\newcommand{\childdocof}[1]
{
  \childdocdisable
  \childdoctrue
  \includeonly{\childdocname}
  \def\jobname{#1}
  \def\childdocjob{#1}
  \input{#1}
}
%    \end{macrocode}

% \macro{\childdocby}
% The command |\childdocby| ....
%    \begin{macrocode}
\newcommand{\childdocby}[2][]
{
  \childdocdisable
  \childdoctrue
  \childdocmanualtrue
  \if?#1?\else
    \def\jobname{#2}
  \fi
  \def\childdocjob{#2}
  \input{#2}
  \endinput
}
%    \end{macrocode}

% \macro{\childdocforward}
% The command |\childdocforward| redirects
% compilation to the main file or
% (if the optional argument is given) a child file.
% Parameters are set as if the main file
% or a child file starting with |\childdocof| was compiled.
% Then compilation is handed over to the main file:
%    \begin{macrocode}
\newcommand{\childdocforward}[2][]
{
  \begingroup
    \if?#1?
      \def\childdoctmp
      {
        \def\childdocname{#2}
        \def\childdocjob{#2}
        \def\jobname{#2}
        \input{#2}
        \endinput
      }
    \else
      \def\childdoctmp
      {
        \childdocdisable
        \def\childdocname{#2}
        \childdoctrue
        \includeonly{#2}
        \def\childdocjob{#1}
        \def\jobname{#1}
        \input{#1}
        \endinput
      }
    \fi
    \expandafter
  \endgroup
  \childdoctmp
}
%    \end{macrocode}

% \macro{\childdocforwardprefix}
% The command |\childdocforwardprefix| redirects
% compilation to the main or a child file by means of a pattern.
% The prefix |#1| in the current filename is replaced by |#2|
% and the suffix of the current filename is kept
% (it is assumed that the filename does not contain the substring `|~~~|'
% which is used as a delimiter).
% Compilation is handed over to the new file by |\childdocforward|:
%    \begin{macrocode}
\newcommand{\childdocforwardprefix}[3][]
{
  \begingroup
    \def\childdocextract #2##1~~~{\def\childdoctmp{\childdocforward[#1]{#3##1}}}
    \expandafter\childdocextract\childdocname~~~
    \expandafter
  \endgroup
  \childdoctmp
}
%    \end{macrocode}

% \macro{\childdoc}
% The deprecated macro |\childdoc| is a legacy version of |\childdocmain|:
%    \begin{macrocode}
\newcommand{\childdoc}{\childdocmain}
%    \end{macrocode}

% \macro{\childdocredirect}
% The deprecated macro |\childdocredirect| is a legacy version
% of |\childdocforward| and |\childdocforwardprefix|:
%    \begin{macrocode}
\newcommand{\childdocredirect}[2][]
{
  \begingroup
    \if?#1?
      \def\childdoctmp{\childdocforward{#2}}
    \else
      \def\childdoctmp{\childdocforwardprefix{#1}{#2}}
    \fi
    \expandafter
  \endgroup
  \childdoctmp
}
%    \end{macrocode}

%\iffalse
%</package>
%\fi
%
\endinput

\childdocforwardprefix[cdocsamp]{cdocsfn}{cdocsch}
%    \end{macrocode}

%\iffalse
%</samplefinal>
%\fi
%
% %%%%%%%%%%%%%%%%%%%%%%%%%%%%%%%%%%%%%%
% \paragraph{Command Line Processing.}
%
% The following three command lines generate the output files
% |cdocscld|, |cdocscl1| and |cdocscl2|
% which should be identical to
% |cdocsdrf|, |cdocsch1| and |cdocsfn2|, respectively:
% \begin{center}
% \begin{tabular}{l}
% |latex -jobname cdocscld \|\\
% |  "\def\version{draft}% \iffalse
%
% childdoc.dtx Copyright (C) 2017-2018 Niklas Beisert
%
% This work may be distributed and/or modified under the
% conditions of the LaTeX Project Public License, either version 1.3
% of this license or (at your option) any later version.
% The latest version of this license is in
%   http://www.latex-project.org/lppl.txt
% and version 1.3 or later is part of all distributions of LaTeX
% version 2005/12/01 or later.
%
% This work has the LPPL maintenance status `maintained'.
%
% The Current Maintainer of this work is Niklas Beisert.
%
% This work consists of the files childdoc.dtx and childdoc.ins
% and the derived files childdoc.def and cdocsamp.tex with
% cdocsch1.tex, cdocsch2.tex, cdocsdrf.tex, cdocsfn1.tex, cdocsfn2.tex.
%
%<package>\ifdefined\childdocmain\endinput\fi
%<package>\ProvidesFile{childdoc.def}[2018/12/30 v2.0 child document driver]
%<samplemain>\ProvidesFile{cdocsamp.tex}[2018/12/30 v2.0 sample for childdoc]
%<*driver>
%\ProvidesFile{childdoc.drv}[2018/12/30 v2.0 childdoc reference manual file]
\PassOptionsToClass{10pt,a4paper}{article}
\documentclass{ltxdoc}

\usepackage[margin=35mm]{geometry}
\usepackage{hyperref}
\usepackage{hyperxmp}
\usepackage[usenames]{color}

\hypersetup{colorlinks=true}
\hypersetup{pdfstartview=FitH}
\hypersetup{pdfpagemode=UseNone}
\hypersetup{pdfsource={}}
\hypersetup{pdflang={en-UK}}
\hypersetup{pdfcopyright={Copyright 2017-2018 Niklas Beisert.
  This work may be distributed and/or modified under the
  conditions of the LaTeX Project Public License, either version 1.3
  of this license or (at your option) any later version.}}
\hypersetup{pdflicenseurl={http://www.latex-project.org/lppl.txt}}
\hypersetup{pdfcontactaddress={ETH Zurich, ITP, HIT K,
  Wolfgang-Pauli-Strasse 27}}
\hypersetup{pdfcontactpostcode={8093}}
\hypersetup{pdfcontactcity={Zurich}}
\hypersetup{pdfcontactcountry={Switzerland}}
\hypersetup{pdfcontactemail={nbeisert@itp.phys.ethz.ch}}
\hypersetup{pdfcontacturl={http://people.phys.ethz.ch/\xmptilde nbeisert/}}

\newcommand{\secref}[1]{\hyperref[#1]{section \ref*{#1}}}

\parskip1ex
\parindent0pt
\let\olditemize\itemize
\def\itemize{\olditemize\parskip0pt}

\begin{document}

\title{The \textsf{childdoc} Package}
\hypersetup{pdftitle={The childdoc Package}}
\author{Niklas Beisert\\[2ex]
  Institut f\"ur Theoretische Physik\\
  Eidgen\"ossische Technische Hochschule Z\"urich\\
  Wolfgang-Pauli-Strasse 27, 8093 Z\"urich, Switzerland\\[1ex]
  \href{mailto:nbeisert@itp.phys.ethz.ch}
  {\texttt{nbeisert@itp.phys.ethz.ch}}}
\hypersetup{pdfauthor={Niklas Beisert}}
\hypersetup{pdfsubject={Manual for the LaTeX2e Package childdoc}}
\date{30 December 2018, \textsf{v2.0}}
\maketitle

\begin{abstract}\noindent
\textsf{childdoc} is a \LaTeXe{} package
that enables the direct compilation
of document sections included by |\include|
to individual files.
\end{abstract}

\begingroup
\parskip0ex
\tableofcontents
\endgroup

%%%%%%%%%%%%%%%%%%%%%%%%%%%%%%%%%%%%%%%%%%%%%%%%%%%%%%%%%%%%%%%%%%%%%%%%%%%%%%%%
%%%%%%%%%%%%%%%%%%%%%%%%%%%%%%%%%%%%%%%%%%%%%%%%%%%%%%%%%%%%%%%%%%%%%%%%%%%%%%%%
\section{Introduction}

\LaTeX{} provides a mechanism to structure a large document (such as a book)
into a main file and several child files (containing the chapters)
using the |\include| command.
This mechanism is beneficial for documents
which span hundreds of pages in order to
make the source file(s) more manageable.
Moreover, compilation can be restricted to
selected child files by means of the |\includeonly| command.
The latter feature can be used to reduce the compilation time while editing
(this was significantly more useful in the earlier days of \LaTeX{})
or to generate a smaller document which is easier to navigate.
Another application of |\includeonly| is to generate
documents consisting of selected parts of the complete document.

However, there are a few drawbacks of the plain |\include| mechanism:
\begin{itemize}
\item
The child files cannot be compiled on their own,
they can only be compiled via the main file.
A naive editing environment
(such as a text editor with an option
to have the current file processed by \LaTeX)
may require one to switch to the main file before compiling;
attempting to compile the child file produces errors.
\item
The main file must be modified (each time)
to adjust the |\includeonly| command
to the present needs. This easily leaves the main file in a messy state.
\item
The generated document will always carry the filename
of the main document. This is inconvenient if
several child files are to be compiled and
to be kept for distribution.
\end{itemize}

The present package provides a simple interface
to make child files individually compilable by \LaTeX{}.
Compiling a child file then has the same effect as compiling
the main file with an |\includeonly| command
to select the appropriate child.
Moreover the generated document will carry the name of the child
rather than the main file.
This resolves all three above issues.

This feature is meant to make the editing of books,
thesis documents and lecture notes somewhat more convenient.
However, the package can also be used efficiently for
composing a series of documents (such as exercise sheets)
which are typically distributed individually.
It then assists the author in generating the individual documents
(potentially in different versions)
as well as a document containing the collected series.
Another application is in developing style files
or other kinds of included material
where compilation of the style file could redirect
to a sample or test file.

%%%%%%%%%%%%%%%%%%%%%%%%%%%%%%%%%%%%%%%%%%%%%%%%%%%%%%%%%%%%%%%%%%%%%%%%%%%%%%%%
%%%%%%%%%%%%%%%%%%%%%%%%%%%%%%%%%%%%%%%%%%%%%%%%%%%%%%%%%%%%%%%%%%%%%%%%%%%%%%%%
\section{Usage}

First of all, the package \textsf{childdoc} is \emph{not} a standard
\LaTeXe{} |.sty| style file! Therefore it needs to be invoked in
a non-standard way.

%%%%%%%%%%%%%%%%%%%%%%%%%%%%%%%%%%%%%%%%%%%%%%%%%%%%%%%%%%%%%%%%%%%%%%%%%%%%%%%%
\subsection{Included Files}
\label{sec:include}

%%%%%%%%%%%%%%%%%%%%%%%%%%%%%%%%%%%%%%%%
\DescribeMacro{\childdocmain}
To use the package, add the commands
\begin{center}
\begin{tabular}{l}
|\input{childdoc.def}|\\
|\childdocmain{}|\\
\end{tabular}
\end{center}
at the very top of the main \LaTeX{} file,
in particular \emph{before} the |\documentclass| statement!
The argument of |\childdocmain| should be left empty
(but it must be present).

%%%%%%%%%%%%%%%%%%%%%%%%%%%%%%%%%%%%%%%%
\DescribeMacro{\childdocof}
Furthermore, add the commands
\begin{center}
\begin{tabular}{l}
|\input{childdoc.def}|\\
|\childdocof{|\textit{main}|}|\\
\end{tabular}
\end{center}
at the top of every child file \textit{child}
which is included by |\include{|\textit{child}|}|
from within the main file
(or at least for those files to be compiled individually).
The argument \textit{main} must be the filename of the main file.

There are a couple of
considerations in setting up the main and child documents:

%%%%%%%%%%%%%%%%%%%%%%%%%%%%%%%%%%%%%%%%
\paragraph{Restrictions.}

Please note the following restrictions:
\begin{itemize}
\item
|\childdocmain| must be called with one argument \textit{main}
to ensure compatibility with earlier version of the package.
It must either be empty (|\childdocmain{}|)
or precisely match the filename of the main file in which it is specified.
See \secref{sec:detection} for further information.
\item
The filename \textit{main} must be specified without the |.tex| extension.
\item
The filename \textit{main} is case sensitive
(even in case-insensitive file systems)
due to internal string comparison.
\item
The argument \textit{main} should be fully expanded, it cannot be a macro.
\item
Subdirectories and special characters should be avoided in filenames.
\item
The command |\childdocmain{|\textit{main}|}| must be followed by a whitespace.
It should not be followed immediately by another command
or by a comment mark `|%|'.
This is because the \TeX{} parser reads the token immediately following
the argument of |\childdocmain| and puts it
at the beginning of every child section;
however, a white\-space is ignored.
\end{itemize}

%%%%%%%%%%%%%%%%%%%%%%%%%%%%%%%%%%%%%%%%
\paragraph{Content of Main File.}

It is advisable to place all content in the child files included by |\include|.
Any output contained in the main file will appear in all child documents
unless suppressed manually;
it cannot be suppressed automatically by the |\includeonly| directive
and thus should normally be avoided.
A method to include some content in the main file
by means of conditional processing is described in \secref{sec:conditional}.

%%%%%%%%%%%%%%%%%%%%%%%%%%%%%%%%%%%%%%%%
\paragraph{Page Numbering.}

When only a part of the document is compiled,
the appropriate numbering of pages
(as well as other status parameters)
is determined from the |.aux| files.
The latter contain information from previous passes.
However this information needs to propagate through
all intermediate child documents.
Therefore the page numbering in child documents may well
be inconsistent until the complete document is compiled at least once.

A useful (if unconventional) way to always ensure a consistent
page numbering is to restart the numbering in each child document
and denote the pages by `\textit{child}|.|\textit{page}'
where \textit{child} represents the chapter/section number of the child file.
This can be achieved by the command
|\numberwithin{page}{|\textit{child}|}|
of the \textsf{amsmath} package
where \textit{child} can be |chapter| or |section|
depending on the chosen structuring.
Alternatively, one can modify the macro |\thepage| appropriately
and reset the counter |page| at the start of each child file.

%%%%%%%%%%%%%%%%%%%%%%%%%%%%%%%%%%%%%%%%%%%%%%%%%%%%%%%%%%%%%%%%%%%%%%%%%%%%%%%%
\subsection{Conditional Processing}
\label{sec:conditional}

The package provides a mechanism to compile different versions
of a document. To customise the versions further some conditional processing
can come in handy to distinguish which version is being compiled.
The package provides two macros to describe the compilation context:

%%%%%%%%%%%%%%%%%%%%%%%%%%%%%%%%%%%%%%%%
\DescribeMacro{\ifchilddoc}
The conditional |\ifchilddoc| distinguishes between the compilation of
child documents and the main document:
%
\begin{center}
|\ifchilddoc |\textit{child-code}| |[|\||else |\textit{main-code}]| \||fi|
\end{center}

%%%%%%%%%%%%%%%%%%%%%%%%%%%%%%%%%%%%%%%%
\DescribeMacro{\childdocname}
\DescribeMacro{\childdocjob}
The macro |\childdocname| contains the filename (without extension)
of the main or child file being processed.
Note that |\childdocjob| will always contain the name of the main file.

%%%%%%%%%%%%%%%%%%%%%%%%%%%%%%%%%%%%%%%%
\paragraph{Title Page.}

Conditional processing can be used to include a title or banner page
in the main document when proper precautions are taken.
Importantly, the code in the main file should ensure that the page counter
(as well as other status parameters which are stored in the |.aux| files)
takes the same value after the conditional processing.
Otherwise the page numbers may take divergent values
depending on which part is compiled.

For example, a title page could be declared by:
%
\begin{center}
\begin{tabular}{l}
|\ifchilddoc\||else|\\
|\addtocounter{page}{-1}|\\
\textit{code for title page}\\
|\newpage|\\
|\||fi|
\end{tabular}
\end{center}
%
A banner page for the child documents can be generated by:
%
\begin{center}
\begin{tabular}{l}
|\ifchilddoc|\\
|\addtocounter{page}{-1}|\\
\textit{code for banner page}\\
|\newpage|\\
|\||fi|
\end{tabular}
\end{center}
%
Here one could write a message such as:
\begin{center}
|This is the part \childdocname{} of \childdocjob{}.|
\end{center}

%%%%%%%%%%%%%%%%%%%%%%%%%%%%%%%%%%%%%%%%%%%%%%%%%%%%%%%%%%%%%%%%%%%%%%%%%%%%%%%%
\subsection{Flags}
\label{sec:flags}

The package makes it easy to generate different versions
of the main or child documents.
To this end compilation flags can be defined
and assigned different default values.
They will be particularly useful in conjunction
with the forwarding mechanism described in \secref{sec:forward}.

For example, it may be useful to have a flag |\version|
which can be set to |draft| or |final|.
The document source will contain some conditional code
depending on the value of |\version|.
Suppose further, the flag should default to |final| for the main file
and to |draft| for child files
which is a natural assignment for editing the document.
This is achieved by placing the following code
in the preamble of the main document
(below the |\childdocmain| directive):
%
\begin{center}
\begin{tabular}{l}
|\ifchilddoc|\\
|\providecommand{\version}{draft}|\\
|\||else|\\
|\providecommand{\version}{final}|\\
|\||fi|
\end{tabular}
\end{center}
%
The definition by |\providecommand| makes sure
that previous definitions are not overwritten.
Further statements |\providecommand{\version}{...}|
can thus be added before the above code to override it.

For the main file, one might add a line
(between |\childdocmain| and the above block)
%
\begin{center}
|%\ifchilddoc\||else\providecommand{\version}{draft}\||fi|
\end{center}
%
which can be uncommented to produce a draft version.
Likewise one can add a line to the very top of a child file
(above the |\childdocof{|\textit{main}|}| directive)
%
\begin{center}
|%\providecommand{\version}{final}|
\end{center}
%
which can be uncommented to produce the final version of this child document.

%%%%%%%%%%%%%%%%%%%%%%%%%%%%%%%%%%%%%%%%%%%%%%%%%%%%%%%%%%%%%%%%%%%%%%%%%%%%%%%%
\subsection{Forwarding}
\label{sec:forward}

Different versions of the main or child documents
using compilation flags as described in \secref{sec:flags}
can be (permanently) stored in different files
for convenient compilation, viewing and distribution.
To this end, the package defines a command
to pass on compilation to a different file:

%%%%%%%%%%%%%%%%%%%%%%%%%%%%%%%%%%%%%%%%
\DescribeMacro{\childdocforward}
The command |\childdocforward| redirects processing to
another source file:
%
\begin{center}
\begin{tabular}{l}
|\input{childdoc.def}|\\
|\childdocforward[|\textit{main}|]{|\textit{dest}|}|\\
\end{tabular}
\end{center}
%
The argument \textit{dest} is the destination file
(without extension).
It should be the main file or one of the child files.
Note that further \textsf{childdoc} directives
such as |\childdocof| and |\childdocforward|
in the indicated file will be processed in this form.
The optional argument \textit{main}
passes on directly to the main file \textit{main}
while pretending to compile the child \textit{dest}.
This form behaves as if \textit{dest}
issues |\childdocof{|\textit{main}|}| right away,
and no further \textsf{childdoc} directives will be processed.

%%%%%%%%%%%%%%%%%%%%%%%%%%%%%%%%%%%%%%%%
\DescribeMacro{\...prefix}
In the alternative form |\childdocforwardprefix|,
%
\begin{center}
\begin{tabular}{l}
|\input{childdoc.def}|\\
|\childdocforwardprefix[|\textit{main}|]{|\textit{prefix}|}{|\textit{dest}|}|
\end{tabular}
\end{center}
%
the destination file is determined by a pattern
depending on the current file:
To make this work, the current file must be called
`{\textit{prefix}\hspace{0.2em}\textit{suffix}}'
with \textit{prefix} matching precisely the argument.
Processing is then passed on to the file
`{\textit{dest}\hspace{0.2em}\textit{suffix}}'.
Surely, the same effect is achieved by
directly specifying the
argument `{\textit{dest}\hspace{0.2em}\textit{suffix}}'
in the first form.
However, that requires to set up a different file
for each child. With the alternative form of the command
all these files can have exactly the same content
which simplifies setting them up and maintaining them.

For example, the following file |draft.tex|
with a compilation flag |\version| as described in \secref{sec:flags}
compiles the main document as a draft:
%
\begin{center}
\begin{tabular}{l}
|\def\version{draft}|\\
|\input{childdoc.def}|\\
|\childdocforward{|\textit{main}|}|
\end{tabular}
\end{center}
%
Likewise, the following files |final|\textit{nn}|.tex|
compile the final version of the child document
|child|\textit{nn}|.tex|:
%
\begin{center}
\begin{tabular}{l}
|\def\version{final}|\\
|\input{childdoc.def}|\\
|\childdocforwardprefix{final}{child}|
\end{tabular}
\end{center}
%

Note that when several versions of a main file and/or of each child file
are to be generated, it may be convenient to set up a |Makefile| or
shell script to automatise the process.

%%%%%%%%%%%%%%%%%%%%%%%%%%%%%%%%%%%%%%%%%%%%%%%%%%%%%%%%%%%%%%%%%%%%%%%%%%%%%%%%
\subsection{Command Line Processing}
\label{sec:commandline}

The effect of redirection files can also be achieved by invoking
the \LaTeX{} compiler with a more elaborate command line.
Most conveniently this should be done as part
of a shell script or a |Makefile|.

When using \textsf{childdoc} in the main file, the following
command lines effectively perform a redirection
(note that depending on the shell being used,
backslashes may have to be doubled: `|\|' $\to$ `|\\|'):
%
\begin{center}
|... -jobname "|\textit{target}|" |\\|"|[\textit{flags}]%
|\input{childdoc.def}\childdocforward[|\textit{main}|]{|\textit{dest}|}"|
\end{center}
%
Here \textit{target} is the name of the output file,
\textit{main} is the name of the main file
and \textit{dest} is the name of the main or child file to be processed
(all filenames without extensions).
The optional argument \textit{main} can be omitted
if \textit{main} matches \textit{dest}.
Optionally, compilation \textit{flags} can be defined via |\def| commands.
This command line makes the \TeX{} engine believe
it is compiling the file \textit{target}
whose content is specified as the latter parameter.
The provided code then forwards the processing to
\textit{main} or \textit{dest} as described in \secref{sec:forward}.

%%%%%%%%%%%%%%%%%%%%%%%%%%%%%%%%%%%%%%%%%%%%%%%%%%%%%%%%%%%%%%%%%%%%%%%%%%%%%%%%
\subsection{Include by Input}
\label{sec:input}

Including child documents by |\include| has some restrictions by design.
Most notably, the content of a child document always occupies
its own set of pages; pages cannot be shared between child documents.
Usually, this behaviour makes perfect sense
because each child document contain an essential part of the document.
However, in some situations it may be desirable to compose
a document from a collection of parts
without having mandatory page breaks between then.
For this case, the package
provides a mechanism to include parts
by |\input| which can also be processed individually.
However, by construction this mechanism
requires manual handling of the content to be output.

%%%%%%%%%%%%%%%%%%%%%%%%%%%%%%%%%%%%%%%%
\DescribeMacro{\ifchilddocmanual}
The main file should be prepared as usual, see \secref{sec:include}.
However, the document body must make a distinction
between processing of an individual part and of the main document, e.g.:
%
\begin{center}
\begin{tabular}{l}
|\ifchilddocmanual|\\
|\input{\childdocname}|\\
|\||else|\\
\textit{document body with }|\input{|\textit{part}|}|\\
|\||fi|
\end{tabular}
\end{center}
%
The conditional |\ifchilddocmanual| is true whenever
a part to be included by |\input| is being compiled,
and the name of the part is stored in |\childdocname|.

%%%%%%%%%%%%%%%%%%%%%%%%%%%%%%%%%%%%%%%%
\DescribeMacro{\childdocby}
Each part to be included by |\input| should start with:
%
\begin{center}
\begin{tabular}{l}
|\input{childdoc.def}|\\
|\childdocby{|\textit{main}|}|\\
\end{tabular}
\end{center}
%
The directive |\childdocby| is similar to |\childdocof|
described in \secref{sec:include},
but the subsequent selection of content must be done manually.
To that end, both |\ifchilddoc| and |\ifchilddocmanual|
will be true upon processing of a part,
and the name of the part is stored in |\childdocname|.
Note that |\jobname| will be set to the filename of the current part
so that each part receives an individual |.aux| file
that does not interfere with the |.aux| file(s) of the main document.
This behaviour can be altered by the alternative form
|\childdocby[*]{|\textit{main}|}| (with a non-empty optional argument)
which uses the |.aux| file of the main document
by setting |\jobname| to \textit{main}.

%%%%%%%%%%%%%%%%%%%%%%%%%%%%%%%%%%%%%%%%%%%%%%%%%%%%%%%%%%%%%%%%%%%%%%%%%%%%%%%%
\subsection{Driver Development}
\label{sec:driver}

The \textsf{childdoc} mechanism can also be use for the development
of definition files such as \LaTeX{} styles or classes.
This case differs from the above setup with multiple parts
included by |\include| in that no |\includeonly| should be invoked.
This can be achieved by starting the include file
(before |\ProvidesPackage|) with:
%
\begin{center}
\begin{tabular}{l}
|\input{childdoc.def}|\\
|\childdocforward{|\textit{main}|}|\\
\end{tabular}
\end{center}
%
or alternatively with:
%
\begin{center}
\begin{tabular}{l}
|\input{childdoc.def}|\\
|\childdocby{|\textit{main}|}|\\
\end{tabular}
\end{center}
%
Both forms have slightly different effects as described above.
The main file is prepared as usual, see \secref{sec:include}.

%%%%%%%%%%%%%%%%%%%%%%%%%%%%%%%%%%%%%%%%%%%%%%%%%%%%%%%%%%%%%%%%%%%%%%%%%%%%%%%%
\subsection{Legacy Detection}
\label{sec:detection}

The directive |\childdocmain| in the main file can detect
whether the complete document or merely a child is to be compiled
even without using the directive |\childdocof|.
This method is deprecated because it is less robust
and there is no compelling reason to use it;
it is merely provided for backward compatibility
and it may be removed in future versions.

If the detection mechanism is to be used,
it is mandatory to correctly specify
the filename of the main file as the argument of |\childdocmain|:
%
\begin{center}
\begin{tabular}{l}
|\input{childdoc.def}|\\
|\childdocmain{|\textit{main}|}|\\
\end{tabular}
\end{center}
%
If |\jobname| does not match the argument \textit{main} of |\childdocmain|,
it is assumed that |\jobname| points to the child file to be compiled.
When using |\childdocmain| with the main file specified as argument,
it suffices to start a child file
with just |\input{|\textit{main}|}|
without loading of the package and using |\childdocof|.
If instead all processing is done
with the appropriate \textsf{childdoc} directives,
the argument of \textit{main} of |\childdocmain| can be empty.

An alternative version of the command line processing described
in \secref{sec:commandline} using the detection mechanism reads:
%
\begin{center}
|... -jobname "|\textit{target}|" "|[\textit{flags}]%
[|\def\jobname{|\textit{dest}|}|]|\input{|\textit{main}|}"|
\end{center}

%%%%%%%%%%%%%%%%%%%%%%%%%%%%%%%%%%%%%%%%%%%%%%%%%%%%%%%%%%%%%%%%%%%%%%%%%%%%%%%%
\subsection{Manual Code}
\label{sec:manual}

In case one cannot be certain whether the definitions file |childdoc.def|
is installed on the target \TeX{} distribution
and one prefers not to ship it,
it is conceivable to paste a few relevant commands into the sources.

To that end, drop all statements |\input{childdoc.def}|
and perform the replacements as outlined below.
Instead of |\childdocmain{|\textit{main}|}| add the following code
to the top of the main file:
%
\begin{center}
\begin{tabular}{l}
|\||ifdefined\childdocname\endinput\||fi\newif\ifchilddoc|\\
|\edef\childdocname{\scantokens\expandafter{\jobname\noexpand}}|\\
|\def\childdocmain{|\textit{main}|}\||ifx\childdocmain\childdocname\||else|\\
|\childdoctrue\includeonly{\childdocname}\let\jobname\childdocmain\||fi|\\
\end{tabular}
\end{center}
%
Instead of |\childdocof{|\textit{main}|}| just include the main file
at the top of each child file:
%
\begin{center}
|\input{|\textit{main}|}|
\end{center}
%
A simple redirection |\childdocforward{|\textit{dest}|}| is achieved by:
%
\begin{center}
|\def\jobname{|\textit{dest}|}\input{\jobname}|
\end{center}
%
The redirection with prefix
|\childdocforwardprefix[|\textit{prefix}|]{|\textit{dest}|}|
is accomplished by:
%
\begin{center}
\begin{tabular}{l}
|{\edef\jobname{\scantokens\expandafter{\jobname\noexpand}}|\\
|\def\redirectjob |\textit{prefix}|#1~~~{\gdef\jobname{|\textit{dest}|#1}}|\\
|\expandafter\redirectjob\jobname~~~}\input{\jobname}|
\end{tabular}
\end{center}

In an alternative approach,
child documents can be compiled by a specific command line
without additional code or specific definitions:
%
\begin{center}
|... -jobname "|\textit{target}|" "|[\textit{flags}]%
|\includeonly{|\textit{dest}|}\input{|\textit{main}|}"|
\end{center}
%

%%%%%%%%%%%%%%%%%%%%%%%%%%%%%%%%%%%%%%%%%%%%%%%%%%%%%%%%%%%%%%%%%%%%%%%%%%%%%%%%
%%%%%%%%%%%%%%%%%%%%%%%%%%%%%%%%%%%%%%%%%%%%%%%%%%%%%%%%%%%%%%%%%%%%%%%%%%%%%%%%
\section{Information}

%%%%%%%%%%%%%%%%%%%%%%%%%%%%%%%%%%%%%%%%%%%%%%%%%%%%%%%%%%%%%%%%%%%%%%%%%%%%%%%%
\subsection{Copyright}

Copyright \copyright{} 2017--2018 Niklas Beisert

This work may be distributed and/or modified under the
conditions of the \LaTeX{} Project Public License, either version 1.3
of this license or (at your option) any later version.
The latest version of this license is in
  \url{http://www.latex-project.org/lppl.txt}
and version 1.3 or later is part of all distributions of \LaTeX{}
version 2005/12/01 or later.

This work has the LPPL maintenance status `maintained'.

The Current Maintainer of this work is Niklas Beisert.

This work consists of the files |README.txt|, |childdoc.ins| and |childdoc.dtx|
as well as the derived files |childdoc.def|, |cdocsamp.tex|
with |cdocsch1.tex|, |cdocsch2.tex|, |cdocspt3.tex|, |cdocspt4.tex|,
|cdocsdrf.tex|, |cdocsfn1.tex|, |cdocsfn2.tex|
as well as |childdoc.pdf|.

%%%%%%%%%%%%%%%%%%%%%%%%%%%%%%%%%%%%%%%%%%%%%%%%%%%%%%%%%%%%%%%%%%%%%%%%%%%%%%%%
\subsection{Files and Installation}

The package consists of the files:
%
\begin{center}
\begin{tabular}{ll}
    |README.txt|   & readme file \\
    |childdoc.ins| & installation file \\
    |childdoc.dtx| & source file \\
    |childdoc.def| & definition file \\
    |cdocsamp.tex| & sample main file \\
    |cdocsch1.tex| & sample include file \\
    |cdocsch2.tex| & sample include file \\
    |cdocspt3.tex| & sample part file \\
    |cdocspt4.tex| & sample part file \\
    |cdocsdrf.tex| & sample redirection file \\
    |cdocsfn1.tex| & sample redirection file \\
    |cdocsfn2.tex| & sample redirection file \\
    |childdoc.pdf| & manual
\end{tabular}
\end{center}
%
The distribution consists of the files
|README.txt|, |childdoc.ins| and |childdoc.dtx|.
%
\begin{itemize}
\item
Run (pdf)\LaTeX{} on |childdoc.dtx|
to compile the manual |childdoc.pdf| (this file).
\item
Run \LaTeX{} on |childdoc.ins| to create the definitions file |childdoc.def|
and the sample |cdocsamp.tex| with include files
|cdocsch1.tex|, |cdocsch2.tex|, |cdocspt3.tex|, |cdocspt4.tex|,
|cdocsdrf.tex|, |cdocsfn1.tex|, |cdocsfn2.tex|.
Then copy the file |childdoc.def| to an appropriate directory of your \LaTeX{}
distribution, e.g.\ \textit{texmf-root}|/tex/latex/childdoc|.
\end{itemize}

%%%%%%%%%%%%%%%%%%%%%%%%%%%%%%%%%%%%%%%%%%%%%%%%%%%%%%%%%%%%%%%%%%%%%%%%%%%%%%%%
\subsection{Related CTAN Packages}

There are several other packages which offer a similar functionality:
%
\begin{itemize}
\item
The packages
\href{http://ctan.org/pkg/docmute}{\textsf{docmute}},
\href{http://ctan.org/pkg/includex}{\textsf{includex}} and
\href{http://ctan.org/pkg/standalone}{\textsf{standalone}}
provide commands to include only the document body of
a child file thus allowing both files to be compiled individually.
\item
The packages \href{http://ctan.org/pkg/subdocs}{\textsf{subdocs}}
and \href{http://ctan.org/pkg/subfiles}{\textsf{subfiles}}
provide structures in which the main and child documents can be
encapsulated and allowing them to be compiled individually.
The inclusion mechanism is different from the conventional |\include|.
\item
The package \href{http://ctan.org/pkg/combine}{\textsf{combine}}
is an elaborate solution to combine several documents into one.
\end{itemize}
%
See also the CTAN topic \href{http://ctan.org/topic/subdocs}{\textsf{subdocs}}
for further related packages.
The present package differs from the above solutions in that
a document structure constructed with the conventional |\include| mechanism
just needs two extra commands at the top of every file
such that all constituent files can be compiled individually.

%%%%%%%%%%%%%%%%%%%%%%%%%%%%%%%%%%%%%%%%%%%%%%%%%%%%%%%%%%%%%%%%%%%%%%%%%%%%%%%%
%\subsection{Feature Suggestions}
%
%The following is a list of features which may be useful for future
%versions of this package:
%%
%\begin{itemize}
%\item
%\ldots
%\end{itemize}

%%%%%%%%%%%%%%%%%%%%%%%%%%%%%%%%%%%%%%%%%%%%%%%%%%%%%%%%%%%%%%%%%%%%%%%%%%%%%%%%
\subsection{Revision History}

%%%%%%%%%%%%%%%%%%%%%%%%%%%%%%%%%%%%%%%%
\paragraph{v2.0:} 2018/12/30

\begin{itemize}
\item
immediate forward processing
\item
added |\childdocby| mechanism
\item
manual restructured
\end{itemize}

%%%%%%%%%%%%%%%%%%%%%%%%%%%%%%%%%%%%%%%%
\paragraph{v1.6:} 2018/01/17

\begin{itemize}
\item
application for development of include files
\item
corrections to manual
\end{itemize}

%%%%%%%%%%%%%%%%%%%%%%%%%%%%%%%%%%%%%%%%
\paragraph{v1.5:} 2017/05/21

\begin{itemize}
\item
more complete structuring introduced
\item
|\childdocof| introduced
\item
|\childdoc| renamed to |\childdocmain|
\item
|\childredirect| renamed to |\childdocforward| and |\childdocforwardprefix|
and functionality expanded
\end{itemize}

%%%%%%%%%%%%%%%%%%%%%%%%%%%%%%%%%%%%%%%%
\paragraph{v1.0:} 2017/04/27

\begin{itemize}
\item
manual and install package
\item
first version published on CTAN
\end{itemize}

%%%%%%%%%%%%%%%%%%%%%%%%%%%%%%%%%%%%%%%%
\paragraph{v0.6:} 2017/04/26

\begin{itemize}
\item
redirection mechanism added
\end{itemize}

%%%%%%%%%%%%%%%%%%%%%%%%%%%%%%%%%%%%%%%%
\paragraph{v0.5:} 2017/04/26

\begin{itemize}
\item
functionality in definition file
\end{itemize}


%%%%%%%%%%%%%%%%%%%%%%%%%%%%%%%%%%%%%%%%%%%%%%%%%%%%%%%%%%%%%%%%%%%%%%%%%%%%%%%%
%%%%%%%%%%%%%%%%%%%%%%%%%%%%%%%%%%%%%%%%%%%%%%%%%%%%%%%%%%%%%%%%%%%%%%%%%%%%%%%%
%%%%%%%%%%%%%%%%%%%%%%%%%%%%%%%%%%%%%%%%%%%%%%%%%%%%%%%%%%%%%%%%%%%%%%%%%%%%%%%%
\appendix

\settowidth\MacroIndent{\rmfamily\scriptsize 000\ }

 \DocInput{childdoc.dtx}

\end{document}
%</driver>
% \fi
%
% %%%%%%%%%%%%%%%%%%%%%%%%%%%%%%%%%%%%%%%%%%%%%%%%%%%%%%%%%%%%%%%%%%%%%%%%%%%%%%
% %%%%%%%%%%%%%%%%%%%%%%%%%%%%%%%%%%%%%%%%%%%%%%%%%%%%%%%%%%%%%%%%%%%%%%%%%%%%%%
% \section{Sample}
%\iffalse
%<*samplemain>
%\fi
%
% The following presents a sample document
% with two chapters, two parts, a title page,
% a compile flag as well as three forwarding files to set the flag.
% It consists of eight |.tex| files:
% \begin{center}
% \begin{tabular}{ll}
% |cdocsamp.tex|&main file\\
% |cdocsch1.tex|&include file for chapter 1\\
% |cdocsch2.tex|&include file for chapter 2\\
% |cdocspt3.tex|&include file for part 3\\
% |cdocspt4.tex|&include file for part 4\\
% |cdocsdrf.tex|&forwarding file for main file in draft mode\\
% |cdocsfi1.tex|&forwarding file for final version of chapter 1\\
% |cdocsfi2.tex|&forwarding file for final version of chapter 2\\
% \end{tabular}
% \end{center}
% Each of the eight files can be compiled directly by the \LaTeX{} compiler.
%
% %%%%%%%%%%%%%%%%%%%%%%%%%%%%%%%%%%%%%%
% \paragraph{Main File.}
%
% The main file is called |cdocsamp.tex|.
%
% Load the \textsf{childdoc} definitions and
% declare the filename for the main document:
%    \begin{macrocode}
\input{childdoc.def}
\childdocmain{}
%    \end{macrocode}

% Optional override for |\version| flag:
%    \begin{macrocode}
%%\ifchilddoc\else\providecommand{\version}{draft}\fi
%    \end{macrocode}

% Define the default values for the |\version| flag
% (|final| for the main file and |draft| for childs):
%    \begin{macrocode}
\ifchilddoc
\providecommand{\version}{draft}
\else
\providecommand{\version}{final}
\fi
%    \end{macrocode}

% Load the standard document class:
%    \begin{macrocode}
\documentclass[12pt]{article}
%    \end{macrocode}

% Start the document body:
%    \begin{macrocode}
\begin{document}
%    \end{macrocode}

% Declare a title page.
% Print title, part of document being processed and version flag:
%    \begin{macrocode}
\addtocounter{page}{-1}
\begin{center}
{\LARGE\bfseries{}childdoc example\par}
\vspace{1cm}
\ifchilddoc
\ifchilddocmanual part\else chapter\fi:
`\childdocname' of `\childdocjob'\par
\else
main document: `\childdocjob'\par
\fi
version: \version\par
\end{center}
\newpage
%    \end{macrocode}

% Manually include selected file,
% otherwise process as usual:
%    \begin{macrocode}
\ifchilddocmanual
\section*{part `\childdocname'}
\input{\childdocname}
\else
%    \end{macrocode}

% Include the two chapters:
%    \begin{macrocode}
\include{cdocsch1}
\include{cdocsch2}
%    \end{macrocode}

% Include the two parts unless only chapters should be displayed:
%    \begin{macrocode}
\ifchilddoc\else
\section{part three}
\input{cdocspt3}
\section{part four}
\input{cdocspt4}
\fi
%    \end{macrocode}

% Process as usual until here:
%    \begin{macrocode}
\fi
%    \end{macrocode}

% End of document body:
%    \begin{macrocode}
\end{document}
%    \end{macrocode}
%\iffalse
%</samplemain>
%\fi
%
% %%%%%%%%%%%%%%%%%%%%%%%%%%%%%%%%%%%%%%
% \paragraph{Chapter Include Files.}
%
% The include files are called |cdocsch1.tex| and |cdocsch2.tex|.
%
%\iffalse
%<*samplechap1|samplechap2>
%\fi

% Optional override for |\version| flag:
%    \begin{macrocode}
%%\providecommand{\version}{final}
%    \end{macrocode}

% Include the main document:
%    \begin{macrocode}
\input{childdoc.def}
\childdocof{cdocsamp}
%    \end{macrocode}

%\iffalse
%</samplechap1|samplechap2>
%\fi
%
%\iffalse
%<*samplechap1>
%\fi
% Some text for chapter 1:
%    \begin{macrocode}
\section{one}
some text in chapter one
%    \end{macrocode}

%\iffalse
%</samplechap1>
%\fi
% Some text for chapter 2:
%\iffalse
%<*samplechap2>
%\fi
%    \begin{macrocode}
\section{two}
more text in chapter two
%    \end{macrocode}

%\iffalse
%</samplechap2>
%\fi
%
% %%%%%%%%%%%%%%%%%%%%%%%%%%%%%%%%%%%%%%
% \paragraph{Part Include Files.}
%
% The include files are called |cdocspt3.tex| and |cdocspt4.tex|.
%
%\iffalse
%<*samplepart3|samplepart4>
%\fi

% Optional override for |\version| flag:
%    \begin{macrocode}
%%\providecommand{\version}{final}
%    \end{macrocode}

% Include the main document:
%    \begin{macrocode}
\input{childdoc.def}
\childdocby{cdocsamp}
%    \end{macrocode}

%\iffalse
%</samplepart3|samplepart4>
%\fi
%
%\iffalse
%<*samplepart3>
%\fi
% Some text for part 3:
%    \begin{macrocode}
some text in part three
%    \end{macrocode}

%\iffalse
%</samplepart3>
%\fi
% Some text for part 4:
%\iffalse
%<*samplepart4>
%\fi
%    \begin{macrocode}
more text in part four
%    \end{macrocode}

%\iffalse
%</samplepart4>
%\fi
%
% %%%%%%%%%%%%%%%%%%%%%%%%%%%%%%%%%%%%%%
% \paragraph{Forwarding for a Complete Draft.}
%
% The following forwarding file |cdocsdrf.tex|
% compiles the main document in draft mode:
%\iffalse
%<*sampledraft>
%\fi
%    \begin{macrocode}
\def\version{draft}
\input{childdoc.def}
\childdocforward{cdocsamp}
%    \end{macrocode}

%\iffalse
%</sampledraft>
%\fi
%
% %%%%%%%%%%%%%%%%%%%%%%%%%%%%%%%%%%%%%%
% \paragraph{Forwarding for Final Version of the Chapters.}
%
% The following forwarding files |cdocsfn1.tex| and |cdocsfn2.tex|
% (with identical content)
% compile the final versions of the child documents
% |cdocsch1.tex| and |cdocsch2.tex|, respectively:
%\iffalse
%<*samplefinal>
%\fi
%    \begin{macrocode}
\def\version{final}
\input{childdoc.def}
\childdocforwardprefix[cdocsamp]{cdocsfn}{cdocsch}
%    \end{macrocode}

%\iffalse
%</samplefinal>
%\fi
%
% %%%%%%%%%%%%%%%%%%%%%%%%%%%%%%%%%%%%%%
% \paragraph{Command Line Processing.}
%
% The following three command lines generate the output files
% |cdocscld|, |cdocscl1| and |cdocscl2|
% which should be identical to
% |cdocsdrf|, |cdocsch1| and |cdocsfn2|, respectively:
% \begin{center}
% \begin{tabular}{l}
% |latex -jobname cdocscld \|\\
% |  "\def\version{draft}\input{childdoc.def}\childdocforward{cdocsamp}"|\\
% |latex -jobname cdocscl1 \|\\
% |  "\input{childdoc.def}\childdocforward[cdocsamp]{cdocsch1}"|\\
% |latex -jobname cdocscl2 \|\\
% |  "\def\version{final}\input{childdoc.def}\childdocforward{cdocsch2}"|
% \end{tabular}
% \end{center}
% Note that the trailing backslash on each first line
% merely continues the input to the second line
% (for convenient cut ant paste).
% Furthermore, the command |latex| can be replaced by any
% of its alternative versions such as |pdflatex|.
%
% %%%%%%%%%%%%%%%%%%%%%%%%%%%%%%%%%%%%%%%%%%%%%%%%%%%%%%%%%%%%%%%%%%%%%%%%%%%%%%
% %%%%%%%%%%%%%%%%%%%%%%%%%%%%%%%%%%%%%%%%%%%%%%%%%%%%%%%%%%%%%%%%%%%%%%%%%%%%%%
% \section{Implementation}
%\iffalse
%<*package>
%\fi
%
% This section describes the definitions file |childdoc.def|.

% The definitions cannot be loaded using |\usepackage| or |\RequirePackage|
% which has a mechanism to prevent loading a style file more than once.
% When loading the definitions by means of |\input|
% multiple instances have to be prevented manually:
%\iffalse
%This code needs to be before the `\ProvidesFile' directive
%which is defined at the beginning of this file.
%Therefore it is also placed there and commented out here.
%</package>
%<*discard>
%\fi
%    \begin{macrocode}
\ifdefined\childdocmain\endinput\fi
%    \end{macrocode}
%\iffalse
%</discard>
%<*package>
%\fi
%
% \macro{\ifchilddoc}
% \macro{\ifchilddocmanual}
% The conditional |\ifchilddoc| tells whether a
% child (true) or main (false) document is being compiled.
% The conditional |\ifchilddocmanual| tells whether
% the |\includeonly| mechanism is used (false) or
% the selection of child files must be performed manually (true).
% The definitions initialise to false:
%    \begin{macrocode}
\newif\ifchilddoc
\newif\ifchilddocmanual
%    \end{macrocode}

% \macro{\childdocname}
% \macro{\childdocjob}
% The macro |\childdocname| stores the name of the main document
% to be compiled. The macro |\childdocjob| stores the name of
% the document on which the \LaTeX{} compiler was originally invoked.
% The content of |\jobname| cannot be compared
% to filenames specified in the source due to different catcodes.
% The following code rescans |\jobname|, stores the result
% in |\childdocname| and saves a copy in |\childdocjob|:
%    \begin{macrocode}
\edef\childdocname{\scantokens\expandafter{\jobname\noexpand}}
\let\childdocjob\childdocname
%    \end{macrocode}

% \macro{\childdocdisable}
% The macro |\childdocdisable| prevents the main file
% from being processed more than once.
% At this stage, the main document command |\childdocmain|
% is assumed to be called once again where it should do nothing.
% Any subsequent call to it should prevent
% a secondary processing of the main document
% It overwrites the forwarding commands
% |\childdocof| and |\childdocforward|
% with empty macros to prevent further inclusions of the main document:
%    \begin{macrocode}
\newcommand{\childdocdisable}
{
  \renewcommand{\childdocmain}[1]{\renewcommand{\childdocmain}[1]{\endinput}}
  \renewcommand{\childdocof}[1]{}
  \renewcommand{\childdocby}[2][]{}
  \renewcommand{\childdocforward}[2][]{}
  \renewcommand{\childdocdisable}{}
}
%    \end{macrocode}

% \macro{\childdocmain}
% The macro |\childdocmain| is to be called at the top of the main file
% with nothing or the main filename (without extension) as argument.
% First, it breaks loops.
% If the argument is not empty and does not match |\childdocname|
% (which is set by the first inclusion of |childdoc.def|),
% |\ifchilddoc| is set to true, |\includeonly| is applied to the child file
% and |\jobname| is set to the main file
% (for proper handling of |.aux| files):
%    \begin{macrocode}
\newcommand{\childdocmain}[1]
{
  \childdocdisable\childdocmain{}
  \if?#1?\else
    \begingroup
      \def\childdoctmp{#1}
      \ifx\childdoctmp\childdocname
        \def\childdoctmp{}
      \else
        \def\childdoctmp
        {
          \childdoctrue
          \includeonly{\childdocname}
          \def\childdocjob{#1}
          \def\jobname{#1}
        }
      \fi
      \expandafter
    \endgroup
    \childdoctmp
  \fi
}
%    \end{macrocode}

% \macro{\childdocof}
% The command |\childdocof| redirects
% compilation to the main file |#1|.
%    \begin{macrocode}
\newcommand{\childdocof}[1]
{
  \childdocdisable
  \childdoctrue
  \includeonly{\childdocname}
  \def\jobname{#1}
  \def\childdocjob{#1}
  \input{#1}
}
%    \end{macrocode}

% \macro{\childdocby}
% The command |\childdocby| ....
%    \begin{macrocode}
\newcommand{\childdocby}[2][]
{
  \childdocdisable
  \childdoctrue
  \childdocmanualtrue
  \if?#1?\else
    \def\jobname{#2}
  \fi
  \def\childdocjob{#2}
  \input{#2}
  \endinput
}
%    \end{macrocode}

% \macro{\childdocforward}
% The command |\childdocforward| redirects
% compilation to the main file or
% (if the optional argument is given) a child file.
% Parameters are set as if the main file
% or a child file starting with |\childdocof| was compiled.
% Then compilation is handed over to the main file:
%    \begin{macrocode}
\newcommand{\childdocforward}[2][]
{
  \begingroup
    \if?#1?
      \def\childdoctmp
      {
        \def\childdocname{#2}
        \def\childdocjob{#2}
        \def\jobname{#2}
        \input{#2}
        \endinput
      }
    \else
      \def\childdoctmp
      {
        \childdocdisable
        \def\childdocname{#2}
        \childdoctrue
        \includeonly{#2}
        \def\childdocjob{#1}
        \def\jobname{#1}
        \input{#1}
        \endinput
      }
    \fi
    \expandafter
  \endgroup
  \childdoctmp
}
%    \end{macrocode}

% \macro{\childdocforwardprefix}
% The command |\childdocforwardprefix| redirects
% compilation to the main or a child file by means of a pattern.
% The prefix |#1| in the current filename is replaced by |#2|
% and the suffix of the current filename is kept
% (it is assumed that the filename does not contain the substring `|~~~|'
% which is used as a delimiter).
% Compilation is handed over to the new file by |\childdocforward|:
%    \begin{macrocode}
\newcommand{\childdocforwardprefix}[3][]
{
  \begingroup
    \def\childdocextract #2##1~~~{\def\childdoctmp{\childdocforward[#1]{#3##1}}}
    \expandafter\childdocextract\childdocname~~~
    \expandafter
  \endgroup
  \childdoctmp
}
%    \end{macrocode}

% \macro{\childdoc}
% The deprecated macro |\childdoc| is a legacy version of |\childdocmain|:
%    \begin{macrocode}
\newcommand{\childdoc}{\childdocmain}
%    \end{macrocode}

% \macro{\childdocredirect}
% The deprecated macro |\childdocredirect| is a legacy version
% of |\childdocforward| and |\childdocforwardprefix|:
%    \begin{macrocode}
\newcommand{\childdocredirect}[2][]
{
  \begingroup
    \if?#1?
      \def\childdoctmp{\childdocforward{#2}}
    \else
      \def\childdoctmp{\childdocforwardprefix{#1}{#2}}
    \fi
    \expandafter
  \endgroup
  \childdoctmp
}
%    \end{macrocode}

%\iffalse
%</package>
%\fi
%
\endinput
\childdocforward{cdocsamp}"|\\
% |latex -jobname cdocscl1 \|\\
% |  "% \iffalse
%
% childdoc.dtx Copyright (C) 2017-2018 Niklas Beisert
%
% This work may be distributed and/or modified under the
% conditions of the LaTeX Project Public License, either version 1.3
% of this license or (at your option) any later version.
% The latest version of this license is in
%   http://www.latex-project.org/lppl.txt
% and version 1.3 or later is part of all distributions of LaTeX
% version 2005/12/01 or later.
%
% This work has the LPPL maintenance status `maintained'.
%
% The Current Maintainer of this work is Niklas Beisert.
%
% This work consists of the files childdoc.dtx and childdoc.ins
% and the derived files childdoc.def and cdocsamp.tex with
% cdocsch1.tex, cdocsch2.tex, cdocsdrf.tex, cdocsfn1.tex, cdocsfn2.tex.
%
%<package>\ifdefined\childdocmain\endinput\fi
%<package>\ProvidesFile{childdoc.def}[2018/12/30 v2.0 child document driver]
%<samplemain>\ProvidesFile{cdocsamp.tex}[2018/12/30 v2.0 sample for childdoc]
%<*driver>
%\ProvidesFile{childdoc.drv}[2018/12/30 v2.0 childdoc reference manual file]
\PassOptionsToClass{10pt,a4paper}{article}
\documentclass{ltxdoc}

\usepackage[margin=35mm]{geometry}
\usepackage{hyperref}
\usepackage{hyperxmp}
\usepackage[usenames]{color}

\hypersetup{colorlinks=true}
\hypersetup{pdfstartview=FitH}
\hypersetup{pdfpagemode=UseNone}
\hypersetup{pdfsource={}}
\hypersetup{pdflang={en-UK}}
\hypersetup{pdfcopyright={Copyright 2017-2018 Niklas Beisert.
  This work may be distributed and/or modified under the
  conditions of the LaTeX Project Public License, either version 1.3
  of this license or (at your option) any later version.}}
\hypersetup{pdflicenseurl={http://www.latex-project.org/lppl.txt}}
\hypersetup{pdfcontactaddress={ETH Zurich, ITP, HIT K,
  Wolfgang-Pauli-Strasse 27}}
\hypersetup{pdfcontactpostcode={8093}}
\hypersetup{pdfcontactcity={Zurich}}
\hypersetup{pdfcontactcountry={Switzerland}}
\hypersetup{pdfcontactemail={nbeisert@itp.phys.ethz.ch}}
\hypersetup{pdfcontacturl={http://people.phys.ethz.ch/\xmptilde nbeisert/}}

\newcommand{\secref}[1]{\hyperref[#1]{section \ref*{#1}}}

\parskip1ex
\parindent0pt
\let\olditemize\itemize
\def\itemize{\olditemize\parskip0pt}

\begin{document}

\title{The \textsf{childdoc} Package}
\hypersetup{pdftitle={The childdoc Package}}
\author{Niklas Beisert\\[2ex]
  Institut f\"ur Theoretische Physik\\
  Eidgen\"ossische Technische Hochschule Z\"urich\\
  Wolfgang-Pauli-Strasse 27, 8093 Z\"urich, Switzerland\\[1ex]
  \href{mailto:nbeisert@itp.phys.ethz.ch}
  {\texttt{nbeisert@itp.phys.ethz.ch}}}
\hypersetup{pdfauthor={Niklas Beisert}}
\hypersetup{pdfsubject={Manual for the LaTeX2e Package childdoc}}
\date{30 December 2018, \textsf{v2.0}}
\maketitle

\begin{abstract}\noindent
\textsf{childdoc} is a \LaTeXe{} package
that enables the direct compilation
of document sections included by |\include|
to individual files.
\end{abstract}

\begingroup
\parskip0ex
\tableofcontents
\endgroup

%%%%%%%%%%%%%%%%%%%%%%%%%%%%%%%%%%%%%%%%%%%%%%%%%%%%%%%%%%%%%%%%%%%%%%%%%%%%%%%%
%%%%%%%%%%%%%%%%%%%%%%%%%%%%%%%%%%%%%%%%%%%%%%%%%%%%%%%%%%%%%%%%%%%%%%%%%%%%%%%%
\section{Introduction}

\LaTeX{} provides a mechanism to structure a large document (such as a book)
into a main file and several child files (containing the chapters)
using the |\include| command.
This mechanism is beneficial for documents
which span hundreds of pages in order to
make the source file(s) more manageable.
Moreover, compilation can be restricted to
selected child files by means of the |\includeonly| command.
The latter feature can be used to reduce the compilation time while editing
(this was significantly more useful in the earlier days of \LaTeX{})
or to generate a smaller document which is easier to navigate.
Another application of |\includeonly| is to generate
documents consisting of selected parts of the complete document.

However, there are a few drawbacks of the plain |\include| mechanism:
\begin{itemize}
\item
The child files cannot be compiled on their own,
they can only be compiled via the main file.
A naive editing environment
(such as a text editor with an option
to have the current file processed by \LaTeX)
may require one to switch to the main file before compiling;
attempting to compile the child file produces errors.
\item
The main file must be modified (each time)
to adjust the |\includeonly| command
to the present needs. This easily leaves the main file in a messy state.
\item
The generated document will always carry the filename
of the main document. This is inconvenient if
several child files are to be compiled and
to be kept for distribution.
\end{itemize}

The present package provides a simple interface
to make child files individually compilable by \LaTeX{}.
Compiling a child file then has the same effect as compiling
the main file with an |\includeonly| command
to select the appropriate child.
Moreover the generated document will carry the name of the child
rather than the main file.
This resolves all three above issues.

This feature is meant to make the editing of books,
thesis documents and lecture notes somewhat more convenient.
However, the package can also be used efficiently for
composing a series of documents (such as exercise sheets)
which are typically distributed individually.
It then assists the author in generating the individual documents
(potentially in different versions)
as well as a document containing the collected series.
Another application is in developing style files
or other kinds of included material
where compilation of the style file could redirect
to a sample or test file.

%%%%%%%%%%%%%%%%%%%%%%%%%%%%%%%%%%%%%%%%%%%%%%%%%%%%%%%%%%%%%%%%%%%%%%%%%%%%%%%%
%%%%%%%%%%%%%%%%%%%%%%%%%%%%%%%%%%%%%%%%%%%%%%%%%%%%%%%%%%%%%%%%%%%%%%%%%%%%%%%%
\section{Usage}

First of all, the package \textsf{childdoc} is \emph{not} a standard
\LaTeXe{} |.sty| style file! Therefore it needs to be invoked in
a non-standard way.

%%%%%%%%%%%%%%%%%%%%%%%%%%%%%%%%%%%%%%%%%%%%%%%%%%%%%%%%%%%%%%%%%%%%%%%%%%%%%%%%
\subsection{Included Files}
\label{sec:include}

%%%%%%%%%%%%%%%%%%%%%%%%%%%%%%%%%%%%%%%%
\DescribeMacro{\childdocmain}
To use the package, add the commands
\begin{center}
\begin{tabular}{l}
|\input{childdoc.def}|\\
|\childdocmain{}|\\
\end{tabular}
\end{center}
at the very top of the main \LaTeX{} file,
in particular \emph{before} the |\documentclass| statement!
The argument of |\childdocmain| should be left empty
(but it must be present).

%%%%%%%%%%%%%%%%%%%%%%%%%%%%%%%%%%%%%%%%
\DescribeMacro{\childdocof}
Furthermore, add the commands
\begin{center}
\begin{tabular}{l}
|\input{childdoc.def}|\\
|\childdocof{|\textit{main}|}|\\
\end{tabular}
\end{center}
at the top of every child file \textit{child}
which is included by |\include{|\textit{child}|}|
from within the main file
(or at least for those files to be compiled individually).
The argument \textit{main} must be the filename of the main file.

There are a couple of
considerations in setting up the main and child documents:

%%%%%%%%%%%%%%%%%%%%%%%%%%%%%%%%%%%%%%%%
\paragraph{Restrictions.}

Please note the following restrictions:
\begin{itemize}
\item
|\childdocmain| must be called with one argument \textit{main}
to ensure compatibility with earlier version of the package.
It must either be empty (|\childdocmain{}|)
or precisely match the filename of the main file in which it is specified.
See \secref{sec:detection} for further information.
\item
The filename \textit{main} must be specified without the |.tex| extension.
\item
The filename \textit{main} is case sensitive
(even in case-insensitive file systems)
due to internal string comparison.
\item
The argument \textit{main} should be fully expanded, it cannot be a macro.
\item
Subdirectories and special characters should be avoided in filenames.
\item
The command |\childdocmain{|\textit{main}|}| must be followed by a whitespace.
It should not be followed immediately by another command
or by a comment mark `|%|'.
This is because the \TeX{} parser reads the token immediately following
the argument of |\childdocmain| and puts it
at the beginning of every child section;
however, a white\-space is ignored.
\end{itemize}

%%%%%%%%%%%%%%%%%%%%%%%%%%%%%%%%%%%%%%%%
\paragraph{Content of Main File.}

It is advisable to place all content in the child files included by |\include|.
Any output contained in the main file will appear in all child documents
unless suppressed manually;
it cannot be suppressed automatically by the |\includeonly| directive
and thus should normally be avoided.
A method to include some content in the main file
by means of conditional processing is described in \secref{sec:conditional}.

%%%%%%%%%%%%%%%%%%%%%%%%%%%%%%%%%%%%%%%%
\paragraph{Page Numbering.}

When only a part of the document is compiled,
the appropriate numbering of pages
(as well as other status parameters)
is determined from the |.aux| files.
The latter contain information from previous passes.
However this information needs to propagate through
all intermediate child documents.
Therefore the page numbering in child documents may well
be inconsistent until the complete document is compiled at least once.

A useful (if unconventional) way to always ensure a consistent
page numbering is to restart the numbering in each child document
and denote the pages by `\textit{child}|.|\textit{page}'
where \textit{child} represents the chapter/section number of the child file.
This can be achieved by the command
|\numberwithin{page}{|\textit{child}|}|
of the \textsf{amsmath} package
where \textit{child} can be |chapter| or |section|
depending on the chosen structuring.
Alternatively, one can modify the macro |\thepage| appropriately
and reset the counter |page| at the start of each child file.

%%%%%%%%%%%%%%%%%%%%%%%%%%%%%%%%%%%%%%%%%%%%%%%%%%%%%%%%%%%%%%%%%%%%%%%%%%%%%%%%
\subsection{Conditional Processing}
\label{sec:conditional}

The package provides a mechanism to compile different versions
of a document. To customise the versions further some conditional processing
can come in handy to distinguish which version is being compiled.
The package provides two macros to describe the compilation context:

%%%%%%%%%%%%%%%%%%%%%%%%%%%%%%%%%%%%%%%%
\DescribeMacro{\ifchilddoc}
The conditional |\ifchilddoc| distinguishes between the compilation of
child documents and the main document:
%
\begin{center}
|\ifchilddoc |\textit{child-code}| |[|\||else |\textit{main-code}]| \||fi|
\end{center}

%%%%%%%%%%%%%%%%%%%%%%%%%%%%%%%%%%%%%%%%
\DescribeMacro{\childdocname}
\DescribeMacro{\childdocjob}
The macro |\childdocname| contains the filename (without extension)
of the main or child file being processed.
Note that |\childdocjob| will always contain the name of the main file.

%%%%%%%%%%%%%%%%%%%%%%%%%%%%%%%%%%%%%%%%
\paragraph{Title Page.}

Conditional processing can be used to include a title or banner page
in the main document when proper precautions are taken.
Importantly, the code in the main file should ensure that the page counter
(as well as other status parameters which are stored in the |.aux| files)
takes the same value after the conditional processing.
Otherwise the page numbers may take divergent values
depending on which part is compiled.

For example, a title page could be declared by:
%
\begin{center}
\begin{tabular}{l}
|\ifchilddoc\||else|\\
|\addtocounter{page}{-1}|\\
\textit{code for title page}\\
|\newpage|\\
|\||fi|
\end{tabular}
\end{center}
%
A banner page for the child documents can be generated by:
%
\begin{center}
\begin{tabular}{l}
|\ifchilddoc|\\
|\addtocounter{page}{-1}|\\
\textit{code for banner page}\\
|\newpage|\\
|\||fi|
\end{tabular}
\end{center}
%
Here one could write a message such as:
\begin{center}
|This is the part \childdocname{} of \childdocjob{}.|
\end{center}

%%%%%%%%%%%%%%%%%%%%%%%%%%%%%%%%%%%%%%%%%%%%%%%%%%%%%%%%%%%%%%%%%%%%%%%%%%%%%%%%
\subsection{Flags}
\label{sec:flags}

The package makes it easy to generate different versions
of the main or child documents.
To this end compilation flags can be defined
and assigned different default values.
They will be particularly useful in conjunction
with the forwarding mechanism described in \secref{sec:forward}.

For example, it may be useful to have a flag |\version|
which can be set to |draft| or |final|.
The document source will contain some conditional code
depending on the value of |\version|.
Suppose further, the flag should default to |final| for the main file
and to |draft| for child files
which is a natural assignment for editing the document.
This is achieved by placing the following code
in the preamble of the main document
(below the |\childdocmain| directive):
%
\begin{center}
\begin{tabular}{l}
|\ifchilddoc|\\
|\providecommand{\version}{draft}|\\
|\||else|\\
|\providecommand{\version}{final}|\\
|\||fi|
\end{tabular}
\end{center}
%
The definition by |\providecommand| makes sure
that previous definitions are not overwritten.
Further statements |\providecommand{\version}{...}|
can thus be added before the above code to override it.

For the main file, one might add a line
(between |\childdocmain| and the above block)
%
\begin{center}
|%\ifchilddoc\||else\providecommand{\version}{draft}\||fi|
\end{center}
%
which can be uncommented to produce a draft version.
Likewise one can add a line to the very top of a child file
(above the |\childdocof{|\textit{main}|}| directive)
%
\begin{center}
|%\providecommand{\version}{final}|
\end{center}
%
which can be uncommented to produce the final version of this child document.

%%%%%%%%%%%%%%%%%%%%%%%%%%%%%%%%%%%%%%%%%%%%%%%%%%%%%%%%%%%%%%%%%%%%%%%%%%%%%%%%
\subsection{Forwarding}
\label{sec:forward}

Different versions of the main or child documents
using compilation flags as described in \secref{sec:flags}
can be (permanently) stored in different files
for convenient compilation, viewing and distribution.
To this end, the package defines a command
to pass on compilation to a different file:

%%%%%%%%%%%%%%%%%%%%%%%%%%%%%%%%%%%%%%%%
\DescribeMacro{\childdocforward}
The command |\childdocforward| redirects processing to
another source file:
%
\begin{center}
\begin{tabular}{l}
|\input{childdoc.def}|\\
|\childdocforward[|\textit{main}|]{|\textit{dest}|}|\\
\end{tabular}
\end{center}
%
The argument \textit{dest} is the destination file
(without extension).
It should be the main file or one of the child files.
Note that further \textsf{childdoc} directives
such as |\childdocof| and |\childdocforward|
in the indicated file will be processed in this form.
The optional argument \textit{main}
passes on directly to the main file \textit{main}
while pretending to compile the child \textit{dest}.
This form behaves as if \textit{dest}
issues |\childdocof{|\textit{main}|}| right away,
and no further \textsf{childdoc} directives will be processed.

%%%%%%%%%%%%%%%%%%%%%%%%%%%%%%%%%%%%%%%%
\DescribeMacro{\...prefix}
In the alternative form |\childdocforwardprefix|,
%
\begin{center}
\begin{tabular}{l}
|\input{childdoc.def}|\\
|\childdocforwardprefix[|\textit{main}|]{|\textit{prefix}|}{|\textit{dest}|}|
\end{tabular}
\end{center}
%
the destination file is determined by a pattern
depending on the current file:
To make this work, the current file must be called
`{\textit{prefix}\hspace{0.2em}\textit{suffix}}'
with \textit{prefix} matching precisely the argument.
Processing is then passed on to the file
`{\textit{dest}\hspace{0.2em}\textit{suffix}}'.
Surely, the same effect is achieved by
directly specifying the
argument `{\textit{dest}\hspace{0.2em}\textit{suffix}}'
in the first form.
However, that requires to set up a different file
for each child. With the alternative form of the command
all these files can have exactly the same content
which simplifies setting them up and maintaining them.

For example, the following file |draft.tex|
with a compilation flag |\version| as described in \secref{sec:flags}
compiles the main document as a draft:
%
\begin{center}
\begin{tabular}{l}
|\def\version{draft}|\\
|\input{childdoc.def}|\\
|\childdocforward{|\textit{main}|}|
\end{tabular}
\end{center}
%
Likewise, the following files |final|\textit{nn}|.tex|
compile the final version of the child document
|child|\textit{nn}|.tex|:
%
\begin{center}
\begin{tabular}{l}
|\def\version{final}|\\
|\input{childdoc.def}|\\
|\childdocforwardprefix{final}{child}|
\end{tabular}
\end{center}
%

Note that when several versions of a main file and/or of each child file
are to be generated, it may be convenient to set up a |Makefile| or
shell script to automatise the process.

%%%%%%%%%%%%%%%%%%%%%%%%%%%%%%%%%%%%%%%%%%%%%%%%%%%%%%%%%%%%%%%%%%%%%%%%%%%%%%%%
\subsection{Command Line Processing}
\label{sec:commandline}

The effect of redirection files can also be achieved by invoking
the \LaTeX{} compiler with a more elaborate command line.
Most conveniently this should be done as part
of a shell script or a |Makefile|.

When using \textsf{childdoc} in the main file, the following
command lines effectively perform a redirection
(note that depending on the shell being used,
backslashes may have to be doubled: `|\|' $\to$ `|\\|'):
%
\begin{center}
|... -jobname "|\textit{target}|" |\\|"|[\textit{flags}]%
|\input{childdoc.def}\childdocforward[|\textit{main}|]{|\textit{dest}|}"|
\end{center}
%
Here \textit{target} is the name of the output file,
\textit{main} is the name of the main file
and \textit{dest} is the name of the main or child file to be processed
(all filenames without extensions).
The optional argument \textit{main} can be omitted
if \textit{main} matches \textit{dest}.
Optionally, compilation \textit{flags} can be defined via |\def| commands.
This command line makes the \TeX{} engine believe
it is compiling the file \textit{target}
whose content is specified as the latter parameter.
The provided code then forwards the processing to
\textit{main} or \textit{dest} as described in \secref{sec:forward}.

%%%%%%%%%%%%%%%%%%%%%%%%%%%%%%%%%%%%%%%%%%%%%%%%%%%%%%%%%%%%%%%%%%%%%%%%%%%%%%%%
\subsection{Include by Input}
\label{sec:input}

Including child documents by |\include| has some restrictions by design.
Most notably, the content of a child document always occupies
its own set of pages; pages cannot be shared between child documents.
Usually, this behaviour makes perfect sense
because each child document contain an essential part of the document.
However, in some situations it may be desirable to compose
a document from a collection of parts
without having mandatory page breaks between then.
For this case, the package
provides a mechanism to include parts
by |\input| which can also be processed individually.
However, by construction this mechanism
requires manual handling of the content to be output.

%%%%%%%%%%%%%%%%%%%%%%%%%%%%%%%%%%%%%%%%
\DescribeMacro{\ifchilddocmanual}
The main file should be prepared as usual, see \secref{sec:include}.
However, the document body must make a distinction
between processing of an individual part and of the main document, e.g.:
%
\begin{center}
\begin{tabular}{l}
|\ifchilddocmanual|\\
|\input{\childdocname}|\\
|\||else|\\
\textit{document body with }|\input{|\textit{part}|}|\\
|\||fi|
\end{tabular}
\end{center}
%
The conditional |\ifchilddocmanual| is true whenever
a part to be included by |\input| is being compiled,
and the name of the part is stored in |\childdocname|.

%%%%%%%%%%%%%%%%%%%%%%%%%%%%%%%%%%%%%%%%
\DescribeMacro{\childdocby}
Each part to be included by |\input| should start with:
%
\begin{center}
\begin{tabular}{l}
|\input{childdoc.def}|\\
|\childdocby{|\textit{main}|}|\\
\end{tabular}
\end{center}
%
The directive |\childdocby| is similar to |\childdocof|
described in \secref{sec:include},
but the subsequent selection of content must be done manually.
To that end, both |\ifchilddoc| and |\ifchilddocmanual|
will be true upon processing of a part,
and the name of the part is stored in |\childdocname|.
Note that |\jobname| will be set to the filename of the current part
so that each part receives an individual |.aux| file
that does not interfere with the |.aux| file(s) of the main document.
This behaviour can be altered by the alternative form
|\childdocby[*]{|\textit{main}|}| (with a non-empty optional argument)
which uses the |.aux| file of the main document
by setting |\jobname| to \textit{main}.

%%%%%%%%%%%%%%%%%%%%%%%%%%%%%%%%%%%%%%%%%%%%%%%%%%%%%%%%%%%%%%%%%%%%%%%%%%%%%%%%
\subsection{Driver Development}
\label{sec:driver}

The \textsf{childdoc} mechanism can also be use for the development
of definition files such as \LaTeX{} styles or classes.
This case differs from the above setup with multiple parts
included by |\include| in that no |\includeonly| should be invoked.
This can be achieved by starting the include file
(before |\ProvidesPackage|) with:
%
\begin{center}
\begin{tabular}{l}
|\input{childdoc.def}|\\
|\childdocforward{|\textit{main}|}|\\
\end{tabular}
\end{center}
%
or alternatively with:
%
\begin{center}
\begin{tabular}{l}
|\input{childdoc.def}|\\
|\childdocby{|\textit{main}|}|\\
\end{tabular}
\end{center}
%
Both forms have slightly different effects as described above.
The main file is prepared as usual, see \secref{sec:include}.

%%%%%%%%%%%%%%%%%%%%%%%%%%%%%%%%%%%%%%%%%%%%%%%%%%%%%%%%%%%%%%%%%%%%%%%%%%%%%%%%
\subsection{Legacy Detection}
\label{sec:detection}

The directive |\childdocmain| in the main file can detect
whether the complete document or merely a child is to be compiled
even without using the directive |\childdocof|.
This method is deprecated because it is less robust
and there is no compelling reason to use it;
it is merely provided for backward compatibility
and it may be removed in future versions.

If the detection mechanism is to be used,
it is mandatory to correctly specify
the filename of the main file as the argument of |\childdocmain|:
%
\begin{center}
\begin{tabular}{l}
|\input{childdoc.def}|\\
|\childdocmain{|\textit{main}|}|\\
\end{tabular}
\end{center}
%
If |\jobname| does not match the argument \textit{main} of |\childdocmain|,
it is assumed that |\jobname| points to the child file to be compiled.
When using |\childdocmain| with the main file specified as argument,
it suffices to start a child file
with just |\input{|\textit{main}|}|
without loading of the package and using |\childdocof|.
If instead all processing is done
with the appropriate \textsf{childdoc} directives,
the argument of \textit{main} of |\childdocmain| can be empty.

An alternative version of the command line processing described
in \secref{sec:commandline} using the detection mechanism reads:
%
\begin{center}
|... -jobname "|\textit{target}|" "|[\textit{flags}]%
[|\def\jobname{|\textit{dest}|}|]|\input{|\textit{main}|}"|
\end{center}

%%%%%%%%%%%%%%%%%%%%%%%%%%%%%%%%%%%%%%%%%%%%%%%%%%%%%%%%%%%%%%%%%%%%%%%%%%%%%%%%
\subsection{Manual Code}
\label{sec:manual}

In case one cannot be certain whether the definitions file |childdoc.def|
is installed on the target \TeX{} distribution
and one prefers not to ship it,
it is conceivable to paste a few relevant commands into the sources.

To that end, drop all statements |\input{childdoc.def}|
and perform the replacements as outlined below.
Instead of |\childdocmain{|\textit{main}|}| add the following code
to the top of the main file:
%
\begin{center}
\begin{tabular}{l}
|\||ifdefined\childdocname\endinput\||fi\newif\ifchilddoc|\\
|\edef\childdocname{\scantokens\expandafter{\jobname\noexpand}}|\\
|\def\childdocmain{|\textit{main}|}\||ifx\childdocmain\childdocname\||else|\\
|\childdoctrue\includeonly{\childdocname}\let\jobname\childdocmain\||fi|\\
\end{tabular}
\end{center}
%
Instead of |\childdocof{|\textit{main}|}| just include the main file
at the top of each child file:
%
\begin{center}
|\input{|\textit{main}|}|
\end{center}
%
A simple redirection |\childdocforward{|\textit{dest}|}| is achieved by:
%
\begin{center}
|\def\jobname{|\textit{dest}|}\input{\jobname}|
\end{center}
%
The redirection with prefix
|\childdocforwardprefix[|\textit{prefix}|]{|\textit{dest}|}|
is accomplished by:
%
\begin{center}
\begin{tabular}{l}
|{\edef\jobname{\scantokens\expandafter{\jobname\noexpand}}|\\
|\def\redirectjob |\textit{prefix}|#1~~~{\gdef\jobname{|\textit{dest}|#1}}|\\
|\expandafter\redirectjob\jobname~~~}\input{\jobname}|
\end{tabular}
\end{center}

In an alternative approach,
child documents can be compiled by a specific command line
without additional code or specific definitions:
%
\begin{center}
|... -jobname "|\textit{target}|" "|[\textit{flags}]%
|\includeonly{|\textit{dest}|}\input{|\textit{main}|}"|
\end{center}
%

%%%%%%%%%%%%%%%%%%%%%%%%%%%%%%%%%%%%%%%%%%%%%%%%%%%%%%%%%%%%%%%%%%%%%%%%%%%%%%%%
%%%%%%%%%%%%%%%%%%%%%%%%%%%%%%%%%%%%%%%%%%%%%%%%%%%%%%%%%%%%%%%%%%%%%%%%%%%%%%%%
\section{Information}

%%%%%%%%%%%%%%%%%%%%%%%%%%%%%%%%%%%%%%%%%%%%%%%%%%%%%%%%%%%%%%%%%%%%%%%%%%%%%%%%
\subsection{Copyright}

Copyright \copyright{} 2017--2018 Niklas Beisert

This work may be distributed and/or modified under the
conditions of the \LaTeX{} Project Public License, either version 1.3
of this license or (at your option) any later version.
The latest version of this license is in
  \url{http://www.latex-project.org/lppl.txt}
and version 1.3 or later is part of all distributions of \LaTeX{}
version 2005/12/01 or later.

This work has the LPPL maintenance status `maintained'.

The Current Maintainer of this work is Niklas Beisert.

This work consists of the files |README.txt|, |childdoc.ins| and |childdoc.dtx|
as well as the derived files |childdoc.def|, |cdocsamp.tex|
with |cdocsch1.tex|, |cdocsch2.tex|, |cdocspt3.tex|, |cdocspt4.tex|,
|cdocsdrf.tex|, |cdocsfn1.tex|, |cdocsfn2.tex|
as well as |childdoc.pdf|.

%%%%%%%%%%%%%%%%%%%%%%%%%%%%%%%%%%%%%%%%%%%%%%%%%%%%%%%%%%%%%%%%%%%%%%%%%%%%%%%%
\subsection{Files and Installation}

The package consists of the files:
%
\begin{center}
\begin{tabular}{ll}
    |README.txt|   & readme file \\
    |childdoc.ins| & installation file \\
    |childdoc.dtx| & source file \\
    |childdoc.def| & definition file \\
    |cdocsamp.tex| & sample main file \\
    |cdocsch1.tex| & sample include file \\
    |cdocsch2.tex| & sample include file \\
    |cdocspt3.tex| & sample part file \\
    |cdocspt4.tex| & sample part file \\
    |cdocsdrf.tex| & sample redirection file \\
    |cdocsfn1.tex| & sample redirection file \\
    |cdocsfn2.tex| & sample redirection file \\
    |childdoc.pdf| & manual
\end{tabular}
\end{center}
%
The distribution consists of the files
|README.txt|, |childdoc.ins| and |childdoc.dtx|.
%
\begin{itemize}
\item
Run (pdf)\LaTeX{} on |childdoc.dtx|
to compile the manual |childdoc.pdf| (this file).
\item
Run \LaTeX{} on |childdoc.ins| to create the definitions file |childdoc.def|
and the sample |cdocsamp.tex| with include files
|cdocsch1.tex|, |cdocsch2.tex|, |cdocspt3.tex|, |cdocspt4.tex|,
|cdocsdrf.tex|, |cdocsfn1.tex|, |cdocsfn2.tex|.
Then copy the file |childdoc.def| to an appropriate directory of your \LaTeX{}
distribution, e.g.\ \textit{texmf-root}|/tex/latex/childdoc|.
\end{itemize}

%%%%%%%%%%%%%%%%%%%%%%%%%%%%%%%%%%%%%%%%%%%%%%%%%%%%%%%%%%%%%%%%%%%%%%%%%%%%%%%%
\subsection{Related CTAN Packages}

There are several other packages which offer a similar functionality:
%
\begin{itemize}
\item
The packages
\href{http://ctan.org/pkg/docmute}{\textsf{docmute}},
\href{http://ctan.org/pkg/includex}{\textsf{includex}} and
\href{http://ctan.org/pkg/standalone}{\textsf{standalone}}
provide commands to include only the document body of
a child file thus allowing both files to be compiled individually.
\item
The packages \href{http://ctan.org/pkg/subdocs}{\textsf{subdocs}}
and \href{http://ctan.org/pkg/subfiles}{\textsf{subfiles}}
provide structures in which the main and child documents can be
encapsulated and allowing them to be compiled individually.
The inclusion mechanism is different from the conventional |\include|.
\item
The package \href{http://ctan.org/pkg/combine}{\textsf{combine}}
is an elaborate solution to combine several documents into one.
\end{itemize}
%
See also the CTAN topic \href{http://ctan.org/topic/subdocs}{\textsf{subdocs}}
for further related packages.
The present package differs from the above solutions in that
a document structure constructed with the conventional |\include| mechanism
just needs two extra commands at the top of every file
such that all constituent files can be compiled individually.

%%%%%%%%%%%%%%%%%%%%%%%%%%%%%%%%%%%%%%%%%%%%%%%%%%%%%%%%%%%%%%%%%%%%%%%%%%%%%%%%
%\subsection{Feature Suggestions}
%
%The following is a list of features which may be useful for future
%versions of this package:
%%
%\begin{itemize}
%\item
%\ldots
%\end{itemize}

%%%%%%%%%%%%%%%%%%%%%%%%%%%%%%%%%%%%%%%%%%%%%%%%%%%%%%%%%%%%%%%%%%%%%%%%%%%%%%%%
\subsection{Revision History}

%%%%%%%%%%%%%%%%%%%%%%%%%%%%%%%%%%%%%%%%
\paragraph{v2.0:} 2018/12/30

\begin{itemize}
\item
immediate forward processing
\item
added |\childdocby| mechanism
\item
manual restructured
\end{itemize}

%%%%%%%%%%%%%%%%%%%%%%%%%%%%%%%%%%%%%%%%
\paragraph{v1.6:} 2018/01/17

\begin{itemize}
\item
application for development of include files
\item
corrections to manual
\end{itemize}

%%%%%%%%%%%%%%%%%%%%%%%%%%%%%%%%%%%%%%%%
\paragraph{v1.5:} 2017/05/21

\begin{itemize}
\item
more complete structuring introduced
\item
|\childdocof| introduced
\item
|\childdoc| renamed to |\childdocmain|
\item
|\childredirect| renamed to |\childdocforward| and |\childdocforwardprefix|
and functionality expanded
\end{itemize}

%%%%%%%%%%%%%%%%%%%%%%%%%%%%%%%%%%%%%%%%
\paragraph{v1.0:} 2017/04/27

\begin{itemize}
\item
manual and install package
\item
first version published on CTAN
\end{itemize}

%%%%%%%%%%%%%%%%%%%%%%%%%%%%%%%%%%%%%%%%
\paragraph{v0.6:} 2017/04/26

\begin{itemize}
\item
redirection mechanism added
\end{itemize}

%%%%%%%%%%%%%%%%%%%%%%%%%%%%%%%%%%%%%%%%
\paragraph{v0.5:} 2017/04/26

\begin{itemize}
\item
functionality in definition file
\end{itemize}


%%%%%%%%%%%%%%%%%%%%%%%%%%%%%%%%%%%%%%%%%%%%%%%%%%%%%%%%%%%%%%%%%%%%%%%%%%%%%%%%
%%%%%%%%%%%%%%%%%%%%%%%%%%%%%%%%%%%%%%%%%%%%%%%%%%%%%%%%%%%%%%%%%%%%%%%%%%%%%%%%
%%%%%%%%%%%%%%%%%%%%%%%%%%%%%%%%%%%%%%%%%%%%%%%%%%%%%%%%%%%%%%%%%%%%%%%%%%%%%%%%
\appendix

\settowidth\MacroIndent{\rmfamily\scriptsize 000\ }

 \DocInput{childdoc.dtx}

\end{document}
%</driver>
% \fi
%
% %%%%%%%%%%%%%%%%%%%%%%%%%%%%%%%%%%%%%%%%%%%%%%%%%%%%%%%%%%%%%%%%%%%%%%%%%%%%%%
% %%%%%%%%%%%%%%%%%%%%%%%%%%%%%%%%%%%%%%%%%%%%%%%%%%%%%%%%%%%%%%%%%%%%%%%%%%%%%%
% \section{Sample}
%\iffalse
%<*samplemain>
%\fi
%
% The following presents a sample document
% with two chapters, two parts, a title page,
% a compile flag as well as three forwarding files to set the flag.
% It consists of eight |.tex| files:
% \begin{center}
% \begin{tabular}{ll}
% |cdocsamp.tex|&main file\\
% |cdocsch1.tex|&include file for chapter 1\\
% |cdocsch2.tex|&include file for chapter 2\\
% |cdocspt3.tex|&include file for part 3\\
% |cdocspt4.tex|&include file for part 4\\
% |cdocsdrf.tex|&forwarding file for main file in draft mode\\
% |cdocsfi1.tex|&forwarding file for final version of chapter 1\\
% |cdocsfi2.tex|&forwarding file for final version of chapter 2\\
% \end{tabular}
% \end{center}
% Each of the eight files can be compiled directly by the \LaTeX{} compiler.
%
% %%%%%%%%%%%%%%%%%%%%%%%%%%%%%%%%%%%%%%
% \paragraph{Main File.}
%
% The main file is called |cdocsamp.tex|.
%
% Load the \textsf{childdoc} definitions and
% declare the filename for the main document:
%    \begin{macrocode}
\input{childdoc.def}
\childdocmain{}
%    \end{macrocode}

% Optional override for |\version| flag:
%    \begin{macrocode}
%%\ifchilddoc\else\providecommand{\version}{draft}\fi
%    \end{macrocode}

% Define the default values for the |\version| flag
% (|final| for the main file and |draft| for childs):
%    \begin{macrocode}
\ifchilddoc
\providecommand{\version}{draft}
\else
\providecommand{\version}{final}
\fi
%    \end{macrocode}

% Load the standard document class:
%    \begin{macrocode}
\documentclass[12pt]{article}
%    \end{macrocode}

% Start the document body:
%    \begin{macrocode}
\begin{document}
%    \end{macrocode}

% Declare a title page.
% Print title, part of document being processed and version flag:
%    \begin{macrocode}
\addtocounter{page}{-1}
\begin{center}
{\LARGE\bfseries{}childdoc example\par}
\vspace{1cm}
\ifchilddoc
\ifchilddocmanual part\else chapter\fi:
`\childdocname' of `\childdocjob'\par
\else
main document: `\childdocjob'\par
\fi
version: \version\par
\end{center}
\newpage
%    \end{macrocode}

% Manually include selected file,
% otherwise process as usual:
%    \begin{macrocode}
\ifchilddocmanual
\section*{part `\childdocname'}
\input{\childdocname}
\else
%    \end{macrocode}

% Include the two chapters:
%    \begin{macrocode}
\include{cdocsch1}
\include{cdocsch2}
%    \end{macrocode}

% Include the two parts unless only chapters should be displayed:
%    \begin{macrocode}
\ifchilddoc\else
\section{part three}
\input{cdocspt3}
\section{part four}
\input{cdocspt4}
\fi
%    \end{macrocode}

% Process as usual until here:
%    \begin{macrocode}
\fi
%    \end{macrocode}

% End of document body:
%    \begin{macrocode}
\end{document}
%    \end{macrocode}
%\iffalse
%</samplemain>
%\fi
%
% %%%%%%%%%%%%%%%%%%%%%%%%%%%%%%%%%%%%%%
% \paragraph{Chapter Include Files.}
%
% The include files are called |cdocsch1.tex| and |cdocsch2.tex|.
%
%\iffalse
%<*samplechap1|samplechap2>
%\fi

% Optional override for |\version| flag:
%    \begin{macrocode}
%%\providecommand{\version}{final}
%    \end{macrocode}

% Include the main document:
%    \begin{macrocode}
\input{childdoc.def}
\childdocof{cdocsamp}
%    \end{macrocode}

%\iffalse
%</samplechap1|samplechap2>
%\fi
%
%\iffalse
%<*samplechap1>
%\fi
% Some text for chapter 1:
%    \begin{macrocode}
\section{one}
some text in chapter one
%    \end{macrocode}

%\iffalse
%</samplechap1>
%\fi
% Some text for chapter 2:
%\iffalse
%<*samplechap2>
%\fi
%    \begin{macrocode}
\section{two}
more text in chapter two
%    \end{macrocode}

%\iffalse
%</samplechap2>
%\fi
%
% %%%%%%%%%%%%%%%%%%%%%%%%%%%%%%%%%%%%%%
% \paragraph{Part Include Files.}
%
% The include files are called |cdocspt3.tex| and |cdocspt4.tex|.
%
%\iffalse
%<*samplepart3|samplepart4>
%\fi

% Optional override for |\version| flag:
%    \begin{macrocode}
%%\providecommand{\version}{final}
%    \end{macrocode}

% Include the main document:
%    \begin{macrocode}
\input{childdoc.def}
\childdocby{cdocsamp}
%    \end{macrocode}

%\iffalse
%</samplepart3|samplepart4>
%\fi
%
%\iffalse
%<*samplepart3>
%\fi
% Some text for part 3:
%    \begin{macrocode}
some text in part three
%    \end{macrocode}

%\iffalse
%</samplepart3>
%\fi
% Some text for part 4:
%\iffalse
%<*samplepart4>
%\fi
%    \begin{macrocode}
more text in part four
%    \end{macrocode}

%\iffalse
%</samplepart4>
%\fi
%
% %%%%%%%%%%%%%%%%%%%%%%%%%%%%%%%%%%%%%%
% \paragraph{Forwarding for a Complete Draft.}
%
% The following forwarding file |cdocsdrf.tex|
% compiles the main document in draft mode:
%\iffalse
%<*sampledraft>
%\fi
%    \begin{macrocode}
\def\version{draft}
\input{childdoc.def}
\childdocforward{cdocsamp}
%    \end{macrocode}

%\iffalse
%</sampledraft>
%\fi
%
% %%%%%%%%%%%%%%%%%%%%%%%%%%%%%%%%%%%%%%
% \paragraph{Forwarding for Final Version of the Chapters.}
%
% The following forwarding files |cdocsfn1.tex| and |cdocsfn2.tex|
% (with identical content)
% compile the final versions of the child documents
% |cdocsch1.tex| and |cdocsch2.tex|, respectively:
%\iffalse
%<*samplefinal>
%\fi
%    \begin{macrocode}
\def\version{final}
\input{childdoc.def}
\childdocforwardprefix[cdocsamp]{cdocsfn}{cdocsch}
%    \end{macrocode}

%\iffalse
%</samplefinal>
%\fi
%
% %%%%%%%%%%%%%%%%%%%%%%%%%%%%%%%%%%%%%%
% \paragraph{Command Line Processing.}
%
% The following three command lines generate the output files
% |cdocscld|, |cdocscl1| and |cdocscl2|
% which should be identical to
% |cdocsdrf|, |cdocsch1| and |cdocsfn2|, respectively:
% \begin{center}
% \begin{tabular}{l}
% |latex -jobname cdocscld \|\\
% |  "\def\version{draft}\input{childdoc.def}\childdocforward{cdocsamp}"|\\
% |latex -jobname cdocscl1 \|\\
% |  "\input{childdoc.def}\childdocforward[cdocsamp]{cdocsch1}"|\\
% |latex -jobname cdocscl2 \|\\
% |  "\def\version{final}\input{childdoc.def}\childdocforward{cdocsch2}"|
% \end{tabular}
% \end{center}
% Note that the trailing backslash on each first line
% merely continues the input to the second line
% (for convenient cut ant paste).
% Furthermore, the command |latex| can be replaced by any
% of its alternative versions such as |pdflatex|.
%
% %%%%%%%%%%%%%%%%%%%%%%%%%%%%%%%%%%%%%%%%%%%%%%%%%%%%%%%%%%%%%%%%%%%%%%%%%%%%%%
% %%%%%%%%%%%%%%%%%%%%%%%%%%%%%%%%%%%%%%%%%%%%%%%%%%%%%%%%%%%%%%%%%%%%%%%%%%%%%%
% \section{Implementation}
%\iffalse
%<*package>
%\fi
%
% This section describes the definitions file |childdoc.def|.

% The definitions cannot be loaded using |\usepackage| or |\RequirePackage|
% which has a mechanism to prevent loading a style file more than once.
% When loading the definitions by means of |\input|
% multiple instances have to be prevented manually:
%\iffalse
%This code needs to be before the `\ProvidesFile' directive
%which is defined at the beginning of this file.
%Therefore it is also placed there and commented out here.
%</package>
%<*discard>
%\fi
%    \begin{macrocode}
\ifdefined\childdocmain\endinput\fi
%    \end{macrocode}
%\iffalse
%</discard>
%<*package>
%\fi
%
% \macro{\ifchilddoc}
% \macro{\ifchilddocmanual}
% The conditional |\ifchilddoc| tells whether a
% child (true) or main (false) document is being compiled.
% The conditional |\ifchilddocmanual| tells whether
% the |\includeonly| mechanism is used (false) or
% the selection of child files must be performed manually (true).
% The definitions initialise to false:
%    \begin{macrocode}
\newif\ifchilddoc
\newif\ifchilddocmanual
%    \end{macrocode}

% \macro{\childdocname}
% \macro{\childdocjob}
% The macro |\childdocname| stores the name of the main document
% to be compiled. The macro |\childdocjob| stores the name of
% the document on which the \LaTeX{} compiler was originally invoked.
% The content of |\jobname| cannot be compared
% to filenames specified in the source due to different catcodes.
% The following code rescans |\jobname|, stores the result
% in |\childdocname| and saves a copy in |\childdocjob|:
%    \begin{macrocode}
\edef\childdocname{\scantokens\expandafter{\jobname\noexpand}}
\let\childdocjob\childdocname
%    \end{macrocode}

% \macro{\childdocdisable}
% The macro |\childdocdisable| prevents the main file
% from being processed more than once.
% At this stage, the main document command |\childdocmain|
% is assumed to be called once again where it should do nothing.
% Any subsequent call to it should prevent
% a secondary processing of the main document
% It overwrites the forwarding commands
% |\childdocof| and |\childdocforward|
% with empty macros to prevent further inclusions of the main document:
%    \begin{macrocode}
\newcommand{\childdocdisable}
{
  \renewcommand{\childdocmain}[1]{\renewcommand{\childdocmain}[1]{\endinput}}
  \renewcommand{\childdocof}[1]{}
  \renewcommand{\childdocby}[2][]{}
  \renewcommand{\childdocforward}[2][]{}
  \renewcommand{\childdocdisable}{}
}
%    \end{macrocode}

% \macro{\childdocmain}
% The macro |\childdocmain| is to be called at the top of the main file
% with nothing or the main filename (without extension) as argument.
% First, it breaks loops.
% If the argument is not empty and does not match |\childdocname|
% (which is set by the first inclusion of |childdoc.def|),
% |\ifchilddoc| is set to true, |\includeonly| is applied to the child file
% and |\jobname| is set to the main file
% (for proper handling of |.aux| files):
%    \begin{macrocode}
\newcommand{\childdocmain}[1]
{
  \childdocdisable\childdocmain{}
  \if?#1?\else
    \begingroup
      \def\childdoctmp{#1}
      \ifx\childdoctmp\childdocname
        \def\childdoctmp{}
      \else
        \def\childdoctmp
        {
          \childdoctrue
          \includeonly{\childdocname}
          \def\childdocjob{#1}
          \def\jobname{#1}
        }
      \fi
      \expandafter
    \endgroup
    \childdoctmp
  \fi
}
%    \end{macrocode}

% \macro{\childdocof}
% The command |\childdocof| redirects
% compilation to the main file |#1|.
%    \begin{macrocode}
\newcommand{\childdocof}[1]
{
  \childdocdisable
  \childdoctrue
  \includeonly{\childdocname}
  \def\jobname{#1}
  \def\childdocjob{#1}
  \input{#1}
}
%    \end{macrocode}

% \macro{\childdocby}
% The command |\childdocby| ....
%    \begin{macrocode}
\newcommand{\childdocby}[2][]
{
  \childdocdisable
  \childdoctrue
  \childdocmanualtrue
  \if?#1?\else
    \def\jobname{#2}
  \fi
  \def\childdocjob{#2}
  \input{#2}
  \endinput
}
%    \end{macrocode}

% \macro{\childdocforward}
% The command |\childdocforward| redirects
% compilation to the main file or
% (if the optional argument is given) a child file.
% Parameters are set as if the main file
% or a child file starting with |\childdocof| was compiled.
% Then compilation is handed over to the main file:
%    \begin{macrocode}
\newcommand{\childdocforward}[2][]
{
  \begingroup
    \if?#1?
      \def\childdoctmp
      {
        \def\childdocname{#2}
        \def\childdocjob{#2}
        \def\jobname{#2}
        \input{#2}
        \endinput
      }
    \else
      \def\childdoctmp
      {
        \childdocdisable
        \def\childdocname{#2}
        \childdoctrue
        \includeonly{#2}
        \def\childdocjob{#1}
        \def\jobname{#1}
        \input{#1}
        \endinput
      }
    \fi
    \expandafter
  \endgroup
  \childdoctmp
}
%    \end{macrocode}

% \macro{\childdocforwardprefix}
% The command |\childdocforwardprefix| redirects
% compilation to the main or a child file by means of a pattern.
% The prefix |#1| in the current filename is replaced by |#2|
% and the suffix of the current filename is kept
% (it is assumed that the filename does not contain the substring `|~~~|'
% which is used as a delimiter).
% Compilation is handed over to the new file by |\childdocforward|:
%    \begin{macrocode}
\newcommand{\childdocforwardprefix}[3][]
{
  \begingroup
    \def\childdocextract #2##1~~~{\def\childdoctmp{\childdocforward[#1]{#3##1}}}
    \expandafter\childdocextract\childdocname~~~
    \expandafter
  \endgroup
  \childdoctmp
}
%    \end{macrocode}

% \macro{\childdoc}
% The deprecated macro |\childdoc| is a legacy version of |\childdocmain|:
%    \begin{macrocode}
\newcommand{\childdoc}{\childdocmain}
%    \end{macrocode}

% \macro{\childdocredirect}
% The deprecated macro |\childdocredirect| is a legacy version
% of |\childdocforward| and |\childdocforwardprefix|:
%    \begin{macrocode}
\newcommand{\childdocredirect}[2][]
{
  \begingroup
    \if?#1?
      \def\childdoctmp{\childdocforward{#2}}
    \else
      \def\childdoctmp{\childdocforwardprefix{#1}{#2}}
    \fi
    \expandafter
  \endgroup
  \childdoctmp
}
%    \end{macrocode}

%\iffalse
%</package>
%\fi
%
\endinput
\childdocforward[cdocsamp]{cdocsch1}"|\\
% |latex -jobname cdocscl2 \|\\
% |  "\def\version{final}% \iffalse
%
% childdoc.dtx Copyright (C) 2017-2018 Niklas Beisert
%
% This work may be distributed and/or modified under the
% conditions of the LaTeX Project Public License, either version 1.3
% of this license or (at your option) any later version.
% The latest version of this license is in
%   http://www.latex-project.org/lppl.txt
% and version 1.3 or later is part of all distributions of LaTeX
% version 2005/12/01 or later.
%
% This work has the LPPL maintenance status `maintained'.
%
% The Current Maintainer of this work is Niklas Beisert.
%
% This work consists of the files childdoc.dtx and childdoc.ins
% and the derived files childdoc.def and cdocsamp.tex with
% cdocsch1.tex, cdocsch2.tex, cdocsdrf.tex, cdocsfn1.tex, cdocsfn2.tex.
%
%<package>\ifdefined\childdocmain\endinput\fi
%<package>\ProvidesFile{childdoc.def}[2018/12/30 v2.0 child document driver]
%<samplemain>\ProvidesFile{cdocsamp.tex}[2018/12/30 v2.0 sample for childdoc]
%<*driver>
%\ProvidesFile{childdoc.drv}[2018/12/30 v2.0 childdoc reference manual file]
\PassOptionsToClass{10pt,a4paper}{article}
\documentclass{ltxdoc}

\usepackage[margin=35mm]{geometry}
\usepackage{hyperref}
\usepackage{hyperxmp}
\usepackage[usenames]{color}

\hypersetup{colorlinks=true}
\hypersetup{pdfstartview=FitH}
\hypersetup{pdfpagemode=UseNone}
\hypersetup{pdfsource={}}
\hypersetup{pdflang={en-UK}}
\hypersetup{pdfcopyright={Copyright 2017-2018 Niklas Beisert.
  This work may be distributed and/or modified under the
  conditions of the LaTeX Project Public License, either version 1.3
  of this license or (at your option) any later version.}}
\hypersetup{pdflicenseurl={http://www.latex-project.org/lppl.txt}}
\hypersetup{pdfcontactaddress={ETH Zurich, ITP, HIT K,
  Wolfgang-Pauli-Strasse 27}}
\hypersetup{pdfcontactpostcode={8093}}
\hypersetup{pdfcontactcity={Zurich}}
\hypersetup{pdfcontactcountry={Switzerland}}
\hypersetup{pdfcontactemail={nbeisert@itp.phys.ethz.ch}}
\hypersetup{pdfcontacturl={http://people.phys.ethz.ch/\xmptilde nbeisert/}}

\newcommand{\secref}[1]{\hyperref[#1]{section \ref*{#1}}}

\parskip1ex
\parindent0pt
\let\olditemize\itemize
\def\itemize{\olditemize\parskip0pt}

\begin{document}

\title{The \textsf{childdoc} Package}
\hypersetup{pdftitle={The childdoc Package}}
\author{Niklas Beisert\\[2ex]
  Institut f\"ur Theoretische Physik\\
  Eidgen\"ossische Technische Hochschule Z\"urich\\
  Wolfgang-Pauli-Strasse 27, 8093 Z\"urich, Switzerland\\[1ex]
  \href{mailto:nbeisert@itp.phys.ethz.ch}
  {\texttt{nbeisert@itp.phys.ethz.ch}}}
\hypersetup{pdfauthor={Niklas Beisert}}
\hypersetup{pdfsubject={Manual for the LaTeX2e Package childdoc}}
\date{30 December 2018, \textsf{v2.0}}
\maketitle

\begin{abstract}\noindent
\textsf{childdoc} is a \LaTeXe{} package
that enables the direct compilation
of document sections included by |\include|
to individual files.
\end{abstract}

\begingroup
\parskip0ex
\tableofcontents
\endgroup

%%%%%%%%%%%%%%%%%%%%%%%%%%%%%%%%%%%%%%%%%%%%%%%%%%%%%%%%%%%%%%%%%%%%%%%%%%%%%%%%
%%%%%%%%%%%%%%%%%%%%%%%%%%%%%%%%%%%%%%%%%%%%%%%%%%%%%%%%%%%%%%%%%%%%%%%%%%%%%%%%
\section{Introduction}

\LaTeX{} provides a mechanism to structure a large document (such as a book)
into a main file and several child files (containing the chapters)
using the |\include| command.
This mechanism is beneficial for documents
which span hundreds of pages in order to
make the source file(s) more manageable.
Moreover, compilation can be restricted to
selected child files by means of the |\includeonly| command.
The latter feature can be used to reduce the compilation time while editing
(this was significantly more useful in the earlier days of \LaTeX{})
or to generate a smaller document which is easier to navigate.
Another application of |\includeonly| is to generate
documents consisting of selected parts of the complete document.

However, there are a few drawbacks of the plain |\include| mechanism:
\begin{itemize}
\item
The child files cannot be compiled on their own,
they can only be compiled via the main file.
A naive editing environment
(such as a text editor with an option
to have the current file processed by \LaTeX)
may require one to switch to the main file before compiling;
attempting to compile the child file produces errors.
\item
The main file must be modified (each time)
to adjust the |\includeonly| command
to the present needs. This easily leaves the main file in a messy state.
\item
The generated document will always carry the filename
of the main document. This is inconvenient if
several child files are to be compiled and
to be kept for distribution.
\end{itemize}

The present package provides a simple interface
to make child files individually compilable by \LaTeX{}.
Compiling a child file then has the same effect as compiling
the main file with an |\includeonly| command
to select the appropriate child.
Moreover the generated document will carry the name of the child
rather than the main file.
This resolves all three above issues.

This feature is meant to make the editing of books,
thesis documents and lecture notes somewhat more convenient.
However, the package can also be used efficiently for
composing a series of documents (such as exercise sheets)
which are typically distributed individually.
It then assists the author in generating the individual documents
(potentially in different versions)
as well as a document containing the collected series.
Another application is in developing style files
or other kinds of included material
where compilation of the style file could redirect
to a sample or test file.

%%%%%%%%%%%%%%%%%%%%%%%%%%%%%%%%%%%%%%%%%%%%%%%%%%%%%%%%%%%%%%%%%%%%%%%%%%%%%%%%
%%%%%%%%%%%%%%%%%%%%%%%%%%%%%%%%%%%%%%%%%%%%%%%%%%%%%%%%%%%%%%%%%%%%%%%%%%%%%%%%
\section{Usage}

First of all, the package \textsf{childdoc} is \emph{not} a standard
\LaTeXe{} |.sty| style file! Therefore it needs to be invoked in
a non-standard way.

%%%%%%%%%%%%%%%%%%%%%%%%%%%%%%%%%%%%%%%%%%%%%%%%%%%%%%%%%%%%%%%%%%%%%%%%%%%%%%%%
\subsection{Included Files}
\label{sec:include}

%%%%%%%%%%%%%%%%%%%%%%%%%%%%%%%%%%%%%%%%
\DescribeMacro{\childdocmain}
To use the package, add the commands
\begin{center}
\begin{tabular}{l}
|\input{childdoc.def}|\\
|\childdocmain{}|\\
\end{tabular}
\end{center}
at the very top of the main \LaTeX{} file,
in particular \emph{before} the |\documentclass| statement!
The argument of |\childdocmain| should be left empty
(but it must be present).

%%%%%%%%%%%%%%%%%%%%%%%%%%%%%%%%%%%%%%%%
\DescribeMacro{\childdocof}
Furthermore, add the commands
\begin{center}
\begin{tabular}{l}
|\input{childdoc.def}|\\
|\childdocof{|\textit{main}|}|\\
\end{tabular}
\end{center}
at the top of every child file \textit{child}
which is included by |\include{|\textit{child}|}|
from within the main file
(or at least for those files to be compiled individually).
The argument \textit{main} must be the filename of the main file.

There are a couple of
considerations in setting up the main and child documents:

%%%%%%%%%%%%%%%%%%%%%%%%%%%%%%%%%%%%%%%%
\paragraph{Restrictions.}

Please note the following restrictions:
\begin{itemize}
\item
|\childdocmain| must be called with one argument \textit{main}
to ensure compatibility with earlier version of the package.
It must either be empty (|\childdocmain{}|)
or precisely match the filename of the main file in which it is specified.
See \secref{sec:detection} for further information.
\item
The filename \textit{main} must be specified without the |.tex| extension.
\item
The filename \textit{main} is case sensitive
(even in case-insensitive file systems)
due to internal string comparison.
\item
The argument \textit{main} should be fully expanded, it cannot be a macro.
\item
Subdirectories and special characters should be avoided in filenames.
\item
The command |\childdocmain{|\textit{main}|}| must be followed by a whitespace.
It should not be followed immediately by another command
or by a comment mark `|%|'.
This is because the \TeX{} parser reads the token immediately following
the argument of |\childdocmain| and puts it
at the beginning of every child section;
however, a white\-space is ignored.
\end{itemize}

%%%%%%%%%%%%%%%%%%%%%%%%%%%%%%%%%%%%%%%%
\paragraph{Content of Main File.}

It is advisable to place all content in the child files included by |\include|.
Any output contained in the main file will appear in all child documents
unless suppressed manually;
it cannot be suppressed automatically by the |\includeonly| directive
and thus should normally be avoided.
A method to include some content in the main file
by means of conditional processing is described in \secref{sec:conditional}.

%%%%%%%%%%%%%%%%%%%%%%%%%%%%%%%%%%%%%%%%
\paragraph{Page Numbering.}

When only a part of the document is compiled,
the appropriate numbering of pages
(as well as other status parameters)
is determined from the |.aux| files.
The latter contain information from previous passes.
However this information needs to propagate through
all intermediate child documents.
Therefore the page numbering in child documents may well
be inconsistent until the complete document is compiled at least once.

A useful (if unconventional) way to always ensure a consistent
page numbering is to restart the numbering in each child document
and denote the pages by `\textit{child}|.|\textit{page}'
where \textit{child} represents the chapter/section number of the child file.
This can be achieved by the command
|\numberwithin{page}{|\textit{child}|}|
of the \textsf{amsmath} package
where \textit{child} can be |chapter| or |section|
depending on the chosen structuring.
Alternatively, one can modify the macro |\thepage| appropriately
and reset the counter |page| at the start of each child file.

%%%%%%%%%%%%%%%%%%%%%%%%%%%%%%%%%%%%%%%%%%%%%%%%%%%%%%%%%%%%%%%%%%%%%%%%%%%%%%%%
\subsection{Conditional Processing}
\label{sec:conditional}

The package provides a mechanism to compile different versions
of a document. To customise the versions further some conditional processing
can come in handy to distinguish which version is being compiled.
The package provides two macros to describe the compilation context:

%%%%%%%%%%%%%%%%%%%%%%%%%%%%%%%%%%%%%%%%
\DescribeMacro{\ifchilddoc}
The conditional |\ifchilddoc| distinguishes between the compilation of
child documents and the main document:
%
\begin{center}
|\ifchilddoc |\textit{child-code}| |[|\||else |\textit{main-code}]| \||fi|
\end{center}

%%%%%%%%%%%%%%%%%%%%%%%%%%%%%%%%%%%%%%%%
\DescribeMacro{\childdocname}
\DescribeMacro{\childdocjob}
The macro |\childdocname| contains the filename (without extension)
of the main or child file being processed.
Note that |\childdocjob| will always contain the name of the main file.

%%%%%%%%%%%%%%%%%%%%%%%%%%%%%%%%%%%%%%%%
\paragraph{Title Page.}

Conditional processing can be used to include a title or banner page
in the main document when proper precautions are taken.
Importantly, the code in the main file should ensure that the page counter
(as well as other status parameters which are stored in the |.aux| files)
takes the same value after the conditional processing.
Otherwise the page numbers may take divergent values
depending on which part is compiled.

For example, a title page could be declared by:
%
\begin{center}
\begin{tabular}{l}
|\ifchilddoc\||else|\\
|\addtocounter{page}{-1}|\\
\textit{code for title page}\\
|\newpage|\\
|\||fi|
\end{tabular}
\end{center}
%
A banner page for the child documents can be generated by:
%
\begin{center}
\begin{tabular}{l}
|\ifchilddoc|\\
|\addtocounter{page}{-1}|\\
\textit{code for banner page}\\
|\newpage|\\
|\||fi|
\end{tabular}
\end{center}
%
Here one could write a message such as:
\begin{center}
|This is the part \childdocname{} of \childdocjob{}.|
\end{center}

%%%%%%%%%%%%%%%%%%%%%%%%%%%%%%%%%%%%%%%%%%%%%%%%%%%%%%%%%%%%%%%%%%%%%%%%%%%%%%%%
\subsection{Flags}
\label{sec:flags}

The package makes it easy to generate different versions
of the main or child documents.
To this end compilation flags can be defined
and assigned different default values.
They will be particularly useful in conjunction
with the forwarding mechanism described in \secref{sec:forward}.

For example, it may be useful to have a flag |\version|
which can be set to |draft| or |final|.
The document source will contain some conditional code
depending on the value of |\version|.
Suppose further, the flag should default to |final| for the main file
and to |draft| for child files
which is a natural assignment for editing the document.
This is achieved by placing the following code
in the preamble of the main document
(below the |\childdocmain| directive):
%
\begin{center}
\begin{tabular}{l}
|\ifchilddoc|\\
|\providecommand{\version}{draft}|\\
|\||else|\\
|\providecommand{\version}{final}|\\
|\||fi|
\end{tabular}
\end{center}
%
The definition by |\providecommand| makes sure
that previous definitions are not overwritten.
Further statements |\providecommand{\version}{...}|
can thus be added before the above code to override it.

For the main file, one might add a line
(between |\childdocmain| and the above block)
%
\begin{center}
|%\ifchilddoc\||else\providecommand{\version}{draft}\||fi|
\end{center}
%
which can be uncommented to produce a draft version.
Likewise one can add a line to the very top of a child file
(above the |\childdocof{|\textit{main}|}| directive)
%
\begin{center}
|%\providecommand{\version}{final}|
\end{center}
%
which can be uncommented to produce the final version of this child document.

%%%%%%%%%%%%%%%%%%%%%%%%%%%%%%%%%%%%%%%%%%%%%%%%%%%%%%%%%%%%%%%%%%%%%%%%%%%%%%%%
\subsection{Forwarding}
\label{sec:forward}

Different versions of the main or child documents
using compilation flags as described in \secref{sec:flags}
can be (permanently) stored in different files
for convenient compilation, viewing and distribution.
To this end, the package defines a command
to pass on compilation to a different file:

%%%%%%%%%%%%%%%%%%%%%%%%%%%%%%%%%%%%%%%%
\DescribeMacro{\childdocforward}
The command |\childdocforward| redirects processing to
another source file:
%
\begin{center}
\begin{tabular}{l}
|\input{childdoc.def}|\\
|\childdocforward[|\textit{main}|]{|\textit{dest}|}|\\
\end{tabular}
\end{center}
%
The argument \textit{dest} is the destination file
(without extension).
It should be the main file or one of the child files.
Note that further \textsf{childdoc} directives
such as |\childdocof| and |\childdocforward|
in the indicated file will be processed in this form.
The optional argument \textit{main}
passes on directly to the main file \textit{main}
while pretending to compile the child \textit{dest}.
This form behaves as if \textit{dest}
issues |\childdocof{|\textit{main}|}| right away,
and no further \textsf{childdoc} directives will be processed.

%%%%%%%%%%%%%%%%%%%%%%%%%%%%%%%%%%%%%%%%
\DescribeMacro{\...prefix}
In the alternative form |\childdocforwardprefix|,
%
\begin{center}
\begin{tabular}{l}
|\input{childdoc.def}|\\
|\childdocforwardprefix[|\textit{main}|]{|\textit{prefix}|}{|\textit{dest}|}|
\end{tabular}
\end{center}
%
the destination file is determined by a pattern
depending on the current file:
To make this work, the current file must be called
`{\textit{prefix}\hspace{0.2em}\textit{suffix}}'
with \textit{prefix} matching precisely the argument.
Processing is then passed on to the file
`{\textit{dest}\hspace{0.2em}\textit{suffix}}'.
Surely, the same effect is achieved by
directly specifying the
argument `{\textit{dest}\hspace{0.2em}\textit{suffix}}'
in the first form.
However, that requires to set up a different file
for each child. With the alternative form of the command
all these files can have exactly the same content
which simplifies setting them up and maintaining them.

For example, the following file |draft.tex|
with a compilation flag |\version| as described in \secref{sec:flags}
compiles the main document as a draft:
%
\begin{center}
\begin{tabular}{l}
|\def\version{draft}|\\
|\input{childdoc.def}|\\
|\childdocforward{|\textit{main}|}|
\end{tabular}
\end{center}
%
Likewise, the following files |final|\textit{nn}|.tex|
compile the final version of the child document
|child|\textit{nn}|.tex|:
%
\begin{center}
\begin{tabular}{l}
|\def\version{final}|\\
|\input{childdoc.def}|\\
|\childdocforwardprefix{final}{child}|
\end{tabular}
\end{center}
%

Note that when several versions of a main file and/or of each child file
are to be generated, it may be convenient to set up a |Makefile| or
shell script to automatise the process.

%%%%%%%%%%%%%%%%%%%%%%%%%%%%%%%%%%%%%%%%%%%%%%%%%%%%%%%%%%%%%%%%%%%%%%%%%%%%%%%%
\subsection{Command Line Processing}
\label{sec:commandline}

The effect of redirection files can also be achieved by invoking
the \LaTeX{} compiler with a more elaborate command line.
Most conveniently this should be done as part
of a shell script or a |Makefile|.

When using \textsf{childdoc} in the main file, the following
command lines effectively perform a redirection
(note that depending on the shell being used,
backslashes may have to be doubled: `|\|' $\to$ `|\\|'):
%
\begin{center}
|... -jobname "|\textit{target}|" |\\|"|[\textit{flags}]%
|\input{childdoc.def}\childdocforward[|\textit{main}|]{|\textit{dest}|}"|
\end{center}
%
Here \textit{target} is the name of the output file,
\textit{main} is the name of the main file
and \textit{dest} is the name of the main or child file to be processed
(all filenames without extensions).
The optional argument \textit{main} can be omitted
if \textit{main} matches \textit{dest}.
Optionally, compilation \textit{flags} can be defined via |\def| commands.
This command line makes the \TeX{} engine believe
it is compiling the file \textit{target}
whose content is specified as the latter parameter.
The provided code then forwards the processing to
\textit{main} or \textit{dest} as described in \secref{sec:forward}.

%%%%%%%%%%%%%%%%%%%%%%%%%%%%%%%%%%%%%%%%%%%%%%%%%%%%%%%%%%%%%%%%%%%%%%%%%%%%%%%%
\subsection{Include by Input}
\label{sec:input}

Including child documents by |\include| has some restrictions by design.
Most notably, the content of a child document always occupies
its own set of pages; pages cannot be shared between child documents.
Usually, this behaviour makes perfect sense
because each child document contain an essential part of the document.
However, in some situations it may be desirable to compose
a document from a collection of parts
without having mandatory page breaks between then.
For this case, the package
provides a mechanism to include parts
by |\input| which can also be processed individually.
However, by construction this mechanism
requires manual handling of the content to be output.

%%%%%%%%%%%%%%%%%%%%%%%%%%%%%%%%%%%%%%%%
\DescribeMacro{\ifchilddocmanual}
The main file should be prepared as usual, see \secref{sec:include}.
However, the document body must make a distinction
between processing of an individual part and of the main document, e.g.:
%
\begin{center}
\begin{tabular}{l}
|\ifchilddocmanual|\\
|\input{\childdocname}|\\
|\||else|\\
\textit{document body with }|\input{|\textit{part}|}|\\
|\||fi|
\end{tabular}
\end{center}
%
The conditional |\ifchilddocmanual| is true whenever
a part to be included by |\input| is being compiled,
and the name of the part is stored in |\childdocname|.

%%%%%%%%%%%%%%%%%%%%%%%%%%%%%%%%%%%%%%%%
\DescribeMacro{\childdocby}
Each part to be included by |\input| should start with:
%
\begin{center}
\begin{tabular}{l}
|\input{childdoc.def}|\\
|\childdocby{|\textit{main}|}|\\
\end{tabular}
\end{center}
%
The directive |\childdocby| is similar to |\childdocof|
described in \secref{sec:include},
but the subsequent selection of content must be done manually.
To that end, both |\ifchilddoc| and |\ifchilddocmanual|
will be true upon processing of a part,
and the name of the part is stored in |\childdocname|.
Note that |\jobname| will be set to the filename of the current part
so that each part receives an individual |.aux| file
that does not interfere with the |.aux| file(s) of the main document.
This behaviour can be altered by the alternative form
|\childdocby[*]{|\textit{main}|}| (with a non-empty optional argument)
which uses the |.aux| file of the main document
by setting |\jobname| to \textit{main}.

%%%%%%%%%%%%%%%%%%%%%%%%%%%%%%%%%%%%%%%%%%%%%%%%%%%%%%%%%%%%%%%%%%%%%%%%%%%%%%%%
\subsection{Driver Development}
\label{sec:driver}

The \textsf{childdoc} mechanism can also be use for the development
of definition files such as \LaTeX{} styles or classes.
This case differs from the above setup with multiple parts
included by |\include| in that no |\includeonly| should be invoked.
This can be achieved by starting the include file
(before |\ProvidesPackage|) with:
%
\begin{center}
\begin{tabular}{l}
|\input{childdoc.def}|\\
|\childdocforward{|\textit{main}|}|\\
\end{tabular}
\end{center}
%
or alternatively with:
%
\begin{center}
\begin{tabular}{l}
|\input{childdoc.def}|\\
|\childdocby{|\textit{main}|}|\\
\end{tabular}
\end{center}
%
Both forms have slightly different effects as described above.
The main file is prepared as usual, see \secref{sec:include}.

%%%%%%%%%%%%%%%%%%%%%%%%%%%%%%%%%%%%%%%%%%%%%%%%%%%%%%%%%%%%%%%%%%%%%%%%%%%%%%%%
\subsection{Legacy Detection}
\label{sec:detection}

The directive |\childdocmain| in the main file can detect
whether the complete document or merely a child is to be compiled
even without using the directive |\childdocof|.
This method is deprecated because it is less robust
and there is no compelling reason to use it;
it is merely provided for backward compatibility
and it may be removed in future versions.

If the detection mechanism is to be used,
it is mandatory to correctly specify
the filename of the main file as the argument of |\childdocmain|:
%
\begin{center}
\begin{tabular}{l}
|\input{childdoc.def}|\\
|\childdocmain{|\textit{main}|}|\\
\end{tabular}
\end{center}
%
If |\jobname| does not match the argument \textit{main} of |\childdocmain|,
it is assumed that |\jobname| points to the child file to be compiled.
When using |\childdocmain| with the main file specified as argument,
it suffices to start a child file
with just |\input{|\textit{main}|}|
without loading of the package and using |\childdocof|.
If instead all processing is done
with the appropriate \textsf{childdoc} directives,
the argument of \textit{main} of |\childdocmain| can be empty.

An alternative version of the command line processing described
in \secref{sec:commandline} using the detection mechanism reads:
%
\begin{center}
|... -jobname "|\textit{target}|" "|[\textit{flags}]%
[|\def\jobname{|\textit{dest}|}|]|\input{|\textit{main}|}"|
\end{center}

%%%%%%%%%%%%%%%%%%%%%%%%%%%%%%%%%%%%%%%%%%%%%%%%%%%%%%%%%%%%%%%%%%%%%%%%%%%%%%%%
\subsection{Manual Code}
\label{sec:manual}

In case one cannot be certain whether the definitions file |childdoc.def|
is installed on the target \TeX{} distribution
and one prefers not to ship it,
it is conceivable to paste a few relevant commands into the sources.

To that end, drop all statements |\input{childdoc.def}|
and perform the replacements as outlined below.
Instead of |\childdocmain{|\textit{main}|}| add the following code
to the top of the main file:
%
\begin{center}
\begin{tabular}{l}
|\||ifdefined\childdocname\endinput\||fi\newif\ifchilddoc|\\
|\edef\childdocname{\scantokens\expandafter{\jobname\noexpand}}|\\
|\def\childdocmain{|\textit{main}|}\||ifx\childdocmain\childdocname\||else|\\
|\childdoctrue\includeonly{\childdocname}\let\jobname\childdocmain\||fi|\\
\end{tabular}
\end{center}
%
Instead of |\childdocof{|\textit{main}|}| just include the main file
at the top of each child file:
%
\begin{center}
|\input{|\textit{main}|}|
\end{center}
%
A simple redirection |\childdocforward{|\textit{dest}|}| is achieved by:
%
\begin{center}
|\def\jobname{|\textit{dest}|}\input{\jobname}|
\end{center}
%
The redirection with prefix
|\childdocforwardprefix[|\textit{prefix}|]{|\textit{dest}|}|
is accomplished by:
%
\begin{center}
\begin{tabular}{l}
|{\edef\jobname{\scantokens\expandafter{\jobname\noexpand}}|\\
|\def\redirectjob |\textit{prefix}|#1~~~{\gdef\jobname{|\textit{dest}|#1}}|\\
|\expandafter\redirectjob\jobname~~~}\input{\jobname}|
\end{tabular}
\end{center}

In an alternative approach,
child documents can be compiled by a specific command line
without additional code or specific definitions:
%
\begin{center}
|... -jobname "|\textit{target}|" "|[\textit{flags}]%
|\includeonly{|\textit{dest}|}\input{|\textit{main}|}"|
\end{center}
%

%%%%%%%%%%%%%%%%%%%%%%%%%%%%%%%%%%%%%%%%%%%%%%%%%%%%%%%%%%%%%%%%%%%%%%%%%%%%%%%%
%%%%%%%%%%%%%%%%%%%%%%%%%%%%%%%%%%%%%%%%%%%%%%%%%%%%%%%%%%%%%%%%%%%%%%%%%%%%%%%%
\section{Information}

%%%%%%%%%%%%%%%%%%%%%%%%%%%%%%%%%%%%%%%%%%%%%%%%%%%%%%%%%%%%%%%%%%%%%%%%%%%%%%%%
\subsection{Copyright}

Copyright \copyright{} 2017--2018 Niklas Beisert

This work may be distributed and/or modified under the
conditions of the \LaTeX{} Project Public License, either version 1.3
of this license or (at your option) any later version.
The latest version of this license is in
  \url{http://www.latex-project.org/lppl.txt}
and version 1.3 or later is part of all distributions of \LaTeX{}
version 2005/12/01 or later.

This work has the LPPL maintenance status `maintained'.

The Current Maintainer of this work is Niklas Beisert.

This work consists of the files |README.txt|, |childdoc.ins| and |childdoc.dtx|
as well as the derived files |childdoc.def|, |cdocsamp.tex|
with |cdocsch1.tex|, |cdocsch2.tex|, |cdocspt3.tex|, |cdocspt4.tex|,
|cdocsdrf.tex|, |cdocsfn1.tex|, |cdocsfn2.tex|
as well as |childdoc.pdf|.

%%%%%%%%%%%%%%%%%%%%%%%%%%%%%%%%%%%%%%%%%%%%%%%%%%%%%%%%%%%%%%%%%%%%%%%%%%%%%%%%
\subsection{Files and Installation}

The package consists of the files:
%
\begin{center}
\begin{tabular}{ll}
    |README.txt|   & readme file \\
    |childdoc.ins| & installation file \\
    |childdoc.dtx| & source file \\
    |childdoc.def| & definition file \\
    |cdocsamp.tex| & sample main file \\
    |cdocsch1.tex| & sample include file \\
    |cdocsch2.tex| & sample include file \\
    |cdocspt3.tex| & sample part file \\
    |cdocspt4.tex| & sample part file \\
    |cdocsdrf.tex| & sample redirection file \\
    |cdocsfn1.tex| & sample redirection file \\
    |cdocsfn2.tex| & sample redirection file \\
    |childdoc.pdf| & manual
\end{tabular}
\end{center}
%
The distribution consists of the files
|README.txt|, |childdoc.ins| and |childdoc.dtx|.
%
\begin{itemize}
\item
Run (pdf)\LaTeX{} on |childdoc.dtx|
to compile the manual |childdoc.pdf| (this file).
\item
Run \LaTeX{} on |childdoc.ins| to create the definitions file |childdoc.def|
and the sample |cdocsamp.tex| with include files
|cdocsch1.tex|, |cdocsch2.tex|, |cdocspt3.tex|, |cdocspt4.tex|,
|cdocsdrf.tex|, |cdocsfn1.tex|, |cdocsfn2.tex|.
Then copy the file |childdoc.def| to an appropriate directory of your \LaTeX{}
distribution, e.g.\ \textit{texmf-root}|/tex/latex/childdoc|.
\end{itemize}

%%%%%%%%%%%%%%%%%%%%%%%%%%%%%%%%%%%%%%%%%%%%%%%%%%%%%%%%%%%%%%%%%%%%%%%%%%%%%%%%
\subsection{Related CTAN Packages}

There are several other packages which offer a similar functionality:
%
\begin{itemize}
\item
The packages
\href{http://ctan.org/pkg/docmute}{\textsf{docmute}},
\href{http://ctan.org/pkg/includex}{\textsf{includex}} and
\href{http://ctan.org/pkg/standalone}{\textsf{standalone}}
provide commands to include only the document body of
a child file thus allowing both files to be compiled individually.
\item
The packages \href{http://ctan.org/pkg/subdocs}{\textsf{subdocs}}
and \href{http://ctan.org/pkg/subfiles}{\textsf{subfiles}}
provide structures in which the main and child documents can be
encapsulated and allowing them to be compiled individually.
The inclusion mechanism is different from the conventional |\include|.
\item
The package \href{http://ctan.org/pkg/combine}{\textsf{combine}}
is an elaborate solution to combine several documents into one.
\end{itemize}
%
See also the CTAN topic \href{http://ctan.org/topic/subdocs}{\textsf{subdocs}}
for further related packages.
The present package differs from the above solutions in that
a document structure constructed with the conventional |\include| mechanism
just needs two extra commands at the top of every file
such that all constituent files can be compiled individually.

%%%%%%%%%%%%%%%%%%%%%%%%%%%%%%%%%%%%%%%%%%%%%%%%%%%%%%%%%%%%%%%%%%%%%%%%%%%%%%%%
%\subsection{Feature Suggestions}
%
%The following is a list of features which may be useful for future
%versions of this package:
%%
%\begin{itemize}
%\item
%\ldots
%\end{itemize}

%%%%%%%%%%%%%%%%%%%%%%%%%%%%%%%%%%%%%%%%%%%%%%%%%%%%%%%%%%%%%%%%%%%%%%%%%%%%%%%%
\subsection{Revision History}

%%%%%%%%%%%%%%%%%%%%%%%%%%%%%%%%%%%%%%%%
\paragraph{v2.0:} 2018/12/30

\begin{itemize}
\item
immediate forward processing
\item
added |\childdocby| mechanism
\item
manual restructured
\end{itemize}

%%%%%%%%%%%%%%%%%%%%%%%%%%%%%%%%%%%%%%%%
\paragraph{v1.6:} 2018/01/17

\begin{itemize}
\item
application for development of include files
\item
corrections to manual
\end{itemize}

%%%%%%%%%%%%%%%%%%%%%%%%%%%%%%%%%%%%%%%%
\paragraph{v1.5:} 2017/05/21

\begin{itemize}
\item
more complete structuring introduced
\item
|\childdocof| introduced
\item
|\childdoc| renamed to |\childdocmain|
\item
|\childredirect| renamed to |\childdocforward| and |\childdocforwardprefix|
and functionality expanded
\end{itemize}

%%%%%%%%%%%%%%%%%%%%%%%%%%%%%%%%%%%%%%%%
\paragraph{v1.0:} 2017/04/27

\begin{itemize}
\item
manual and install package
\item
first version published on CTAN
\end{itemize}

%%%%%%%%%%%%%%%%%%%%%%%%%%%%%%%%%%%%%%%%
\paragraph{v0.6:} 2017/04/26

\begin{itemize}
\item
redirection mechanism added
\end{itemize}

%%%%%%%%%%%%%%%%%%%%%%%%%%%%%%%%%%%%%%%%
\paragraph{v0.5:} 2017/04/26

\begin{itemize}
\item
functionality in definition file
\end{itemize}


%%%%%%%%%%%%%%%%%%%%%%%%%%%%%%%%%%%%%%%%%%%%%%%%%%%%%%%%%%%%%%%%%%%%%%%%%%%%%%%%
%%%%%%%%%%%%%%%%%%%%%%%%%%%%%%%%%%%%%%%%%%%%%%%%%%%%%%%%%%%%%%%%%%%%%%%%%%%%%%%%
%%%%%%%%%%%%%%%%%%%%%%%%%%%%%%%%%%%%%%%%%%%%%%%%%%%%%%%%%%%%%%%%%%%%%%%%%%%%%%%%
\appendix

\settowidth\MacroIndent{\rmfamily\scriptsize 000\ }

 \DocInput{childdoc.dtx}

\end{document}
%</driver>
% \fi
%
% %%%%%%%%%%%%%%%%%%%%%%%%%%%%%%%%%%%%%%%%%%%%%%%%%%%%%%%%%%%%%%%%%%%%%%%%%%%%%%
% %%%%%%%%%%%%%%%%%%%%%%%%%%%%%%%%%%%%%%%%%%%%%%%%%%%%%%%%%%%%%%%%%%%%%%%%%%%%%%
% \section{Sample}
%\iffalse
%<*samplemain>
%\fi
%
% The following presents a sample document
% with two chapters, two parts, a title page,
% a compile flag as well as three forwarding files to set the flag.
% It consists of eight |.tex| files:
% \begin{center}
% \begin{tabular}{ll}
% |cdocsamp.tex|&main file\\
% |cdocsch1.tex|&include file for chapter 1\\
% |cdocsch2.tex|&include file for chapter 2\\
% |cdocspt3.tex|&include file for part 3\\
% |cdocspt4.tex|&include file for part 4\\
% |cdocsdrf.tex|&forwarding file for main file in draft mode\\
% |cdocsfi1.tex|&forwarding file for final version of chapter 1\\
% |cdocsfi2.tex|&forwarding file for final version of chapter 2\\
% \end{tabular}
% \end{center}
% Each of the eight files can be compiled directly by the \LaTeX{} compiler.
%
% %%%%%%%%%%%%%%%%%%%%%%%%%%%%%%%%%%%%%%
% \paragraph{Main File.}
%
% The main file is called |cdocsamp.tex|.
%
% Load the \textsf{childdoc} definitions and
% declare the filename for the main document:
%    \begin{macrocode}
\input{childdoc.def}
\childdocmain{}
%    \end{macrocode}

% Optional override for |\version| flag:
%    \begin{macrocode}
%%\ifchilddoc\else\providecommand{\version}{draft}\fi
%    \end{macrocode}

% Define the default values for the |\version| flag
% (|final| for the main file and |draft| for childs):
%    \begin{macrocode}
\ifchilddoc
\providecommand{\version}{draft}
\else
\providecommand{\version}{final}
\fi
%    \end{macrocode}

% Load the standard document class:
%    \begin{macrocode}
\documentclass[12pt]{article}
%    \end{macrocode}

% Start the document body:
%    \begin{macrocode}
\begin{document}
%    \end{macrocode}

% Declare a title page.
% Print title, part of document being processed and version flag:
%    \begin{macrocode}
\addtocounter{page}{-1}
\begin{center}
{\LARGE\bfseries{}childdoc example\par}
\vspace{1cm}
\ifchilddoc
\ifchilddocmanual part\else chapter\fi:
`\childdocname' of `\childdocjob'\par
\else
main document: `\childdocjob'\par
\fi
version: \version\par
\end{center}
\newpage
%    \end{macrocode}

% Manually include selected file,
% otherwise process as usual:
%    \begin{macrocode}
\ifchilddocmanual
\section*{part `\childdocname'}
\input{\childdocname}
\else
%    \end{macrocode}

% Include the two chapters:
%    \begin{macrocode}
\include{cdocsch1}
\include{cdocsch2}
%    \end{macrocode}

% Include the two parts unless only chapters should be displayed:
%    \begin{macrocode}
\ifchilddoc\else
\section{part three}
\input{cdocspt3}
\section{part four}
\input{cdocspt4}
\fi
%    \end{macrocode}

% Process as usual until here:
%    \begin{macrocode}
\fi
%    \end{macrocode}

% End of document body:
%    \begin{macrocode}
\end{document}
%    \end{macrocode}
%\iffalse
%</samplemain>
%\fi
%
% %%%%%%%%%%%%%%%%%%%%%%%%%%%%%%%%%%%%%%
% \paragraph{Chapter Include Files.}
%
% The include files are called |cdocsch1.tex| and |cdocsch2.tex|.
%
%\iffalse
%<*samplechap1|samplechap2>
%\fi

% Optional override for |\version| flag:
%    \begin{macrocode}
%%\providecommand{\version}{final}
%    \end{macrocode}

% Include the main document:
%    \begin{macrocode}
\input{childdoc.def}
\childdocof{cdocsamp}
%    \end{macrocode}

%\iffalse
%</samplechap1|samplechap2>
%\fi
%
%\iffalse
%<*samplechap1>
%\fi
% Some text for chapter 1:
%    \begin{macrocode}
\section{one}
some text in chapter one
%    \end{macrocode}

%\iffalse
%</samplechap1>
%\fi
% Some text for chapter 2:
%\iffalse
%<*samplechap2>
%\fi
%    \begin{macrocode}
\section{two}
more text in chapter two
%    \end{macrocode}

%\iffalse
%</samplechap2>
%\fi
%
% %%%%%%%%%%%%%%%%%%%%%%%%%%%%%%%%%%%%%%
% \paragraph{Part Include Files.}
%
% The include files are called |cdocspt3.tex| and |cdocspt4.tex|.
%
%\iffalse
%<*samplepart3|samplepart4>
%\fi

% Optional override for |\version| flag:
%    \begin{macrocode}
%%\providecommand{\version}{final}
%    \end{macrocode}

% Include the main document:
%    \begin{macrocode}
\input{childdoc.def}
\childdocby{cdocsamp}
%    \end{macrocode}

%\iffalse
%</samplepart3|samplepart4>
%\fi
%
%\iffalse
%<*samplepart3>
%\fi
% Some text for part 3:
%    \begin{macrocode}
some text in part three
%    \end{macrocode}

%\iffalse
%</samplepart3>
%\fi
% Some text for part 4:
%\iffalse
%<*samplepart4>
%\fi
%    \begin{macrocode}
more text in part four
%    \end{macrocode}

%\iffalse
%</samplepart4>
%\fi
%
% %%%%%%%%%%%%%%%%%%%%%%%%%%%%%%%%%%%%%%
% \paragraph{Forwarding for a Complete Draft.}
%
% The following forwarding file |cdocsdrf.tex|
% compiles the main document in draft mode:
%\iffalse
%<*sampledraft>
%\fi
%    \begin{macrocode}
\def\version{draft}
\input{childdoc.def}
\childdocforward{cdocsamp}
%    \end{macrocode}

%\iffalse
%</sampledraft>
%\fi
%
% %%%%%%%%%%%%%%%%%%%%%%%%%%%%%%%%%%%%%%
% \paragraph{Forwarding for Final Version of the Chapters.}
%
% The following forwarding files |cdocsfn1.tex| and |cdocsfn2.tex|
% (with identical content)
% compile the final versions of the child documents
% |cdocsch1.tex| and |cdocsch2.tex|, respectively:
%\iffalse
%<*samplefinal>
%\fi
%    \begin{macrocode}
\def\version{final}
\input{childdoc.def}
\childdocforwardprefix[cdocsamp]{cdocsfn}{cdocsch}
%    \end{macrocode}

%\iffalse
%</samplefinal>
%\fi
%
% %%%%%%%%%%%%%%%%%%%%%%%%%%%%%%%%%%%%%%
% \paragraph{Command Line Processing.}
%
% The following three command lines generate the output files
% |cdocscld|, |cdocscl1| and |cdocscl2|
% which should be identical to
% |cdocsdrf|, |cdocsch1| and |cdocsfn2|, respectively:
% \begin{center}
% \begin{tabular}{l}
% |latex -jobname cdocscld \|\\
% |  "\def\version{draft}\input{childdoc.def}\childdocforward{cdocsamp}"|\\
% |latex -jobname cdocscl1 \|\\
% |  "\input{childdoc.def}\childdocforward[cdocsamp]{cdocsch1}"|\\
% |latex -jobname cdocscl2 \|\\
% |  "\def\version{final}\input{childdoc.def}\childdocforward{cdocsch2}"|
% \end{tabular}
% \end{center}
% Note that the trailing backslash on each first line
% merely continues the input to the second line
% (for convenient cut ant paste).
% Furthermore, the command |latex| can be replaced by any
% of its alternative versions such as |pdflatex|.
%
% %%%%%%%%%%%%%%%%%%%%%%%%%%%%%%%%%%%%%%%%%%%%%%%%%%%%%%%%%%%%%%%%%%%%%%%%%%%%%%
% %%%%%%%%%%%%%%%%%%%%%%%%%%%%%%%%%%%%%%%%%%%%%%%%%%%%%%%%%%%%%%%%%%%%%%%%%%%%%%
% \section{Implementation}
%\iffalse
%<*package>
%\fi
%
% This section describes the definitions file |childdoc.def|.

% The definitions cannot be loaded using |\usepackage| or |\RequirePackage|
% which has a mechanism to prevent loading a style file more than once.
% When loading the definitions by means of |\input|
% multiple instances have to be prevented manually:
%\iffalse
%This code needs to be before the `\ProvidesFile' directive
%which is defined at the beginning of this file.
%Therefore it is also placed there and commented out here.
%</package>
%<*discard>
%\fi
%    \begin{macrocode}
\ifdefined\childdocmain\endinput\fi
%    \end{macrocode}
%\iffalse
%</discard>
%<*package>
%\fi
%
% \macro{\ifchilddoc}
% \macro{\ifchilddocmanual}
% The conditional |\ifchilddoc| tells whether a
% child (true) or main (false) document is being compiled.
% The conditional |\ifchilddocmanual| tells whether
% the |\includeonly| mechanism is used (false) or
% the selection of child files must be performed manually (true).
% The definitions initialise to false:
%    \begin{macrocode}
\newif\ifchilddoc
\newif\ifchilddocmanual
%    \end{macrocode}

% \macro{\childdocname}
% \macro{\childdocjob}
% The macro |\childdocname| stores the name of the main document
% to be compiled. The macro |\childdocjob| stores the name of
% the document on which the \LaTeX{} compiler was originally invoked.
% The content of |\jobname| cannot be compared
% to filenames specified in the source due to different catcodes.
% The following code rescans |\jobname|, stores the result
% in |\childdocname| and saves a copy in |\childdocjob|:
%    \begin{macrocode}
\edef\childdocname{\scantokens\expandafter{\jobname\noexpand}}
\let\childdocjob\childdocname
%    \end{macrocode}

% \macro{\childdocdisable}
% The macro |\childdocdisable| prevents the main file
% from being processed more than once.
% At this stage, the main document command |\childdocmain|
% is assumed to be called once again where it should do nothing.
% Any subsequent call to it should prevent
% a secondary processing of the main document
% It overwrites the forwarding commands
% |\childdocof| and |\childdocforward|
% with empty macros to prevent further inclusions of the main document:
%    \begin{macrocode}
\newcommand{\childdocdisable}
{
  \renewcommand{\childdocmain}[1]{\renewcommand{\childdocmain}[1]{\endinput}}
  \renewcommand{\childdocof}[1]{}
  \renewcommand{\childdocby}[2][]{}
  \renewcommand{\childdocforward}[2][]{}
  \renewcommand{\childdocdisable}{}
}
%    \end{macrocode}

% \macro{\childdocmain}
% The macro |\childdocmain| is to be called at the top of the main file
% with nothing or the main filename (without extension) as argument.
% First, it breaks loops.
% If the argument is not empty and does not match |\childdocname|
% (which is set by the first inclusion of |childdoc.def|),
% |\ifchilddoc| is set to true, |\includeonly| is applied to the child file
% and |\jobname| is set to the main file
% (for proper handling of |.aux| files):
%    \begin{macrocode}
\newcommand{\childdocmain}[1]
{
  \childdocdisable\childdocmain{}
  \if?#1?\else
    \begingroup
      \def\childdoctmp{#1}
      \ifx\childdoctmp\childdocname
        \def\childdoctmp{}
      \else
        \def\childdoctmp
        {
          \childdoctrue
          \includeonly{\childdocname}
          \def\childdocjob{#1}
          \def\jobname{#1}
        }
      \fi
      \expandafter
    \endgroup
    \childdoctmp
  \fi
}
%    \end{macrocode}

% \macro{\childdocof}
% The command |\childdocof| redirects
% compilation to the main file |#1|.
%    \begin{macrocode}
\newcommand{\childdocof}[1]
{
  \childdocdisable
  \childdoctrue
  \includeonly{\childdocname}
  \def\jobname{#1}
  \def\childdocjob{#1}
  \input{#1}
}
%    \end{macrocode}

% \macro{\childdocby}
% The command |\childdocby| ....
%    \begin{macrocode}
\newcommand{\childdocby}[2][]
{
  \childdocdisable
  \childdoctrue
  \childdocmanualtrue
  \if?#1?\else
    \def\jobname{#2}
  \fi
  \def\childdocjob{#2}
  \input{#2}
  \endinput
}
%    \end{macrocode}

% \macro{\childdocforward}
% The command |\childdocforward| redirects
% compilation to the main file or
% (if the optional argument is given) a child file.
% Parameters are set as if the main file
% or a child file starting with |\childdocof| was compiled.
% Then compilation is handed over to the main file:
%    \begin{macrocode}
\newcommand{\childdocforward}[2][]
{
  \begingroup
    \if?#1?
      \def\childdoctmp
      {
        \def\childdocname{#2}
        \def\childdocjob{#2}
        \def\jobname{#2}
        \input{#2}
        \endinput
      }
    \else
      \def\childdoctmp
      {
        \childdocdisable
        \def\childdocname{#2}
        \childdoctrue
        \includeonly{#2}
        \def\childdocjob{#1}
        \def\jobname{#1}
        \input{#1}
        \endinput
      }
    \fi
    \expandafter
  \endgroup
  \childdoctmp
}
%    \end{macrocode}

% \macro{\childdocforwardprefix}
% The command |\childdocforwardprefix| redirects
% compilation to the main or a child file by means of a pattern.
% The prefix |#1| in the current filename is replaced by |#2|
% and the suffix of the current filename is kept
% (it is assumed that the filename does not contain the substring `|~~~|'
% which is used as a delimiter).
% Compilation is handed over to the new file by |\childdocforward|:
%    \begin{macrocode}
\newcommand{\childdocforwardprefix}[3][]
{
  \begingroup
    \def\childdocextract #2##1~~~{\def\childdoctmp{\childdocforward[#1]{#3##1}}}
    \expandafter\childdocextract\childdocname~~~
    \expandafter
  \endgroup
  \childdoctmp
}
%    \end{macrocode}

% \macro{\childdoc}
% The deprecated macro |\childdoc| is a legacy version of |\childdocmain|:
%    \begin{macrocode}
\newcommand{\childdoc}{\childdocmain}
%    \end{macrocode}

% \macro{\childdocredirect}
% The deprecated macro |\childdocredirect| is a legacy version
% of |\childdocforward| and |\childdocforwardprefix|:
%    \begin{macrocode}
\newcommand{\childdocredirect}[2][]
{
  \begingroup
    \if?#1?
      \def\childdoctmp{\childdocforward{#2}}
    \else
      \def\childdoctmp{\childdocforwardprefix{#1}{#2}}
    \fi
    \expandafter
  \endgroup
  \childdoctmp
}
%    \end{macrocode}

%\iffalse
%</package>
%\fi
%
\endinput
\childdocforward{cdocsch2}"|
% \end{tabular}
% \end{center}
% Note that the trailing backslash on each first line
% merely continues the input to the second line
% (for convenient cut ant paste).
% Furthermore, the command |latex| can be replaced by any
% of its alternative versions such as |pdflatex|.
%
% %%%%%%%%%%%%%%%%%%%%%%%%%%%%%%%%%%%%%%%%%%%%%%%%%%%%%%%%%%%%%%%%%%%%%%%%%%%%%%
% %%%%%%%%%%%%%%%%%%%%%%%%%%%%%%%%%%%%%%%%%%%%%%%%%%%%%%%%%%%%%%%%%%%%%%%%%%%%%%
% \section{Implementation}
%\iffalse
%<*package>
%\fi
%
% This section describes the definitions file |childdoc.def|.

% The definitions cannot be loaded using |\usepackage| or |\RequirePackage|
% which has a mechanism to prevent loading a style file more than once.
% When loading the definitions by means of |\input|
% multiple instances have to be prevented manually:
%\iffalse
%This code needs to be before the `\ProvidesFile' directive
%which is defined at the beginning of this file.
%Therefore it is also placed there and commented out here.
%</package>
%<*discard>
%\fi
%    \begin{macrocode}
\ifdefined\childdocmain\endinput\fi
%    \end{macrocode}
%\iffalse
%</discard>
%<*package>
%\fi
%
% \macro{\ifchilddoc}
% \macro{\ifchilddocmanual}
% The conditional |\ifchilddoc| tells whether a
% child (true) or main (false) document is being compiled.
% The conditional |\ifchilddocmanual| tells whether
% the |\includeonly| mechanism is used (false) or
% the selection of child files must be performed manually (true).
% The definitions initialise to false:
%    \begin{macrocode}
\newif\ifchilddoc
\newif\ifchilddocmanual
%    \end{macrocode}

% \macro{\childdocname}
% \macro{\childdocjob}
% The macro |\childdocname| stores the name of the main document
% to be compiled. The macro |\childdocjob| stores the name of
% the document on which the \LaTeX{} compiler was originally invoked.
% The content of |\jobname| cannot be compared
% to filenames specified in the source due to different catcodes.
% The following code rescans |\jobname|, stores the result
% in |\childdocname| and saves a copy in |\childdocjob|:
%    \begin{macrocode}
\edef\childdocname{\scantokens\expandafter{\jobname\noexpand}}
\let\childdocjob\childdocname
%    \end{macrocode}

% \macro{\childdocdisable}
% The macro |\childdocdisable| prevents the main file
% from being processed more than once.
% At this stage, the main document command |\childdocmain|
% is assumed to be called once again where it should do nothing.
% Any subsequent call to it should prevent
% a secondary processing of the main document
% It overwrites the forwarding commands
% |\childdocof| and |\childdocforward|
% with empty macros to prevent further inclusions of the main document:
%    \begin{macrocode}
\newcommand{\childdocdisable}
{
  \renewcommand{\childdocmain}[1]{\renewcommand{\childdocmain}[1]{\endinput}}
  \renewcommand{\childdocof}[1]{}
  \renewcommand{\childdocby}[2][]{}
  \renewcommand{\childdocforward}[2][]{}
  \renewcommand{\childdocdisable}{}
}
%    \end{macrocode}

% \macro{\childdocmain}
% The macro |\childdocmain| is to be called at the top of the main file
% with nothing or the main filename (without extension) as argument.
% First, it breaks loops.
% If the argument is not empty and does not match |\childdocname|
% (which is set by the first inclusion of |childdoc.def|),
% |\ifchilddoc| is set to true, |\includeonly| is applied to the child file
% and |\jobname| is set to the main file
% (for proper handling of |.aux| files):
%    \begin{macrocode}
\newcommand{\childdocmain}[1]
{
  \childdocdisable\childdocmain{}
  \if?#1?\else
    \begingroup
      \def\childdoctmp{#1}
      \ifx\childdoctmp\childdocname
        \def\childdoctmp{}
      \else
        \def\childdoctmp
        {
          \childdoctrue
          \includeonly{\childdocname}
          \def\childdocjob{#1}
          \def\jobname{#1}
        }
      \fi
      \expandafter
    \endgroup
    \childdoctmp
  \fi
}
%    \end{macrocode}

% \macro{\childdocof}
% The command |\childdocof| redirects
% compilation to the main file |#1|.
%    \begin{macrocode}
\newcommand{\childdocof}[1]
{
  \childdocdisable
  \childdoctrue
  \includeonly{\childdocname}
  \def\jobname{#1}
  \def\childdocjob{#1}
  \input{#1}
}
%    \end{macrocode}

% \macro{\childdocby}
% The command |\childdocby| ....
%    \begin{macrocode}
\newcommand{\childdocby}[2][]
{
  \childdocdisable
  \childdoctrue
  \childdocmanualtrue
  \if?#1?\else
    \def\jobname{#2}
  \fi
  \def\childdocjob{#2}
  \input{#2}
  \endinput
}
%    \end{macrocode}

% \macro{\childdocforward}
% The command |\childdocforward| redirects
% compilation to the main file or
% (if the optional argument is given) a child file.
% Parameters are set as if the main file
% or a child file starting with |\childdocof| was compiled.
% Then compilation is handed over to the main file:
%    \begin{macrocode}
\newcommand{\childdocforward}[2][]
{
  \begingroup
    \if?#1?
      \def\childdoctmp
      {
        \def\childdocname{#2}
        \def\childdocjob{#2}
        \def\jobname{#2}
        \input{#2}
        \endinput
      }
    \else
      \def\childdoctmp
      {
        \childdocdisable
        \def\childdocname{#2}
        \childdoctrue
        \includeonly{#2}
        \def\childdocjob{#1}
        \def\jobname{#1}
        \input{#1}
        \endinput
      }
    \fi
    \expandafter
  \endgroup
  \childdoctmp
}
%    \end{macrocode}

% \macro{\childdocforwardprefix}
% The command |\childdocforwardprefix| redirects
% compilation to the main or a child file by means of a pattern.
% The prefix |#1| in the current filename is replaced by |#2|
% and the suffix of the current filename is kept
% (it is assumed that the filename does not contain the substring `|~~~|'
% which is used as a delimiter).
% Compilation is handed over to the new file by |\childdocforward|:
%    \begin{macrocode}
\newcommand{\childdocforwardprefix}[3][]
{
  \begingroup
    \def\childdocextract #2##1~~~{\def\childdoctmp{\childdocforward[#1]{#3##1}}}
    \expandafter\childdocextract\childdocname~~~
    \expandafter
  \endgroup
  \childdoctmp
}
%    \end{macrocode}

% \macro{\childdoc}
% The deprecated macro |\childdoc| is a legacy version of |\childdocmain|:
%    \begin{macrocode}
\newcommand{\childdoc}{\childdocmain}
%    \end{macrocode}

% \macro{\childdocredirect}
% The deprecated macro |\childdocredirect| is a legacy version
% of |\childdocforward| and |\childdocforwardprefix|:
%    \begin{macrocode}
\newcommand{\childdocredirect}[2][]
{
  \begingroup
    \if?#1?
      \def\childdoctmp{\childdocforward{#2}}
    \else
      \def\childdoctmp{\childdocforwardprefix{#1}{#2}}
    \fi
    \expandafter
  \endgroup
  \childdoctmp
}
%    \end{macrocode}

%\iffalse
%</package>
%\fi
%
\endinput

\childdocby{cdocsamp}
%    \end{macrocode}

%\iffalse
%</samplepart3|samplepart4>
%\fi
%
%\iffalse
%<*samplepart3>
%\fi
% Some text for part 3:
%    \begin{macrocode}
some text in part three
%    \end{macrocode}

%\iffalse
%</samplepart3>
%\fi
% Some text for part 4:
%\iffalse
%<*samplepart4>
%\fi
%    \begin{macrocode}
more text in part four
%    \end{macrocode}

%\iffalse
%</samplepart4>
%\fi
%
% %%%%%%%%%%%%%%%%%%%%%%%%%%%%%%%%%%%%%%
% \paragraph{Forwarding for a Complete Draft.}
%
% The following forwarding file |cdocsdrf.tex|
% compiles the main document in draft mode:
%\iffalse
%<*sampledraft>
%\fi
%    \begin{macrocode}
\def\version{draft}
% \iffalse
%
% childdoc.dtx Copyright (C) 2017-2018 Niklas Beisert
%
% This work may be distributed and/or modified under the
% conditions of the LaTeX Project Public License, either version 1.3
% of this license or (at your option) any later version.
% The latest version of this license is in
%   http://www.latex-project.org/lppl.txt
% and version 1.3 or later is part of all distributions of LaTeX
% version 2005/12/01 or later.
%
% This work has the LPPL maintenance status `maintained'.
%
% The Current Maintainer of this work is Niklas Beisert.
%
% This work consists of the files childdoc.dtx and childdoc.ins
% and the derived files childdoc.def and cdocsamp.tex with
% cdocsch1.tex, cdocsch2.tex, cdocsdrf.tex, cdocsfn1.tex, cdocsfn2.tex.
%
%<package>\ifdefined\childdocmain\endinput\fi
%<package>\ProvidesFile{childdoc.def}[2018/12/30 v2.0 child document driver]
%<samplemain>\ProvidesFile{cdocsamp.tex}[2018/12/30 v2.0 sample for childdoc]
%<*driver>
%\ProvidesFile{childdoc.drv}[2018/12/30 v2.0 childdoc reference manual file]
\PassOptionsToClass{10pt,a4paper}{article}
\documentclass{ltxdoc}

\usepackage[margin=35mm]{geometry}
\usepackage{hyperref}
\usepackage{hyperxmp}
\usepackage[usenames]{color}

\hypersetup{colorlinks=true}
\hypersetup{pdfstartview=FitH}
\hypersetup{pdfpagemode=UseNone}
\hypersetup{pdfsource={}}
\hypersetup{pdflang={en-UK}}
\hypersetup{pdfcopyright={Copyright 2017-2018 Niklas Beisert.
  This work may be distributed and/or modified under the
  conditions of the LaTeX Project Public License, either version 1.3
  of this license or (at your option) any later version.}}
\hypersetup{pdflicenseurl={http://www.latex-project.org/lppl.txt}}
\hypersetup{pdfcontactaddress={ETH Zurich, ITP, HIT K,
  Wolfgang-Pauli-Strasse 27}}
\hypersetup{pdfcontactpostcode={8093}}
\hypersetup{pdfcontactcity={Zurich}}
\hypersetup{pdfcontactcountry={Switzerland}}
\hypersetup{pdfcontactemail={nbeisert@itp.phys.ethz.ch}}
\hypersetup{pdfcontacturl={http://people.phys.ethz.ch/\xmptilde nbeisert/}}

\newcommand{\secref}[1]{\hyperref[#1]{section \ref*{#1}}}

\parskip1ex
\parindent0pt
\let\olditemize\itemize
\def\itemize{\olditemize\parskip0pt}

\begin{document}

\title{The \textsf{childdoc} Package}
\hypersetup{pdftitle={The childdoc Package}}
\author{Niklas Beisert\\[2ex]
  Institut f\"ur Theoretische Physik\\
  Eidgen\"ossische Technische Hochschule Z\"urich\\
  Wolfgang-Pauli-Strasse 27, 8093 Z\"urich, Switzerland\\[1ex]
  \href{mailto:nbeisert@itp.phys.ethz.ch}
  {\texttt{nbeisert@itp.phys.ethz.ch}}}
\hypersetup{pdfauthor={Niklas Beisert}}
\hypersetup{pdfsubject={Manual for the LaTeX2e Package childdoc}}
\date{30 December 2018, \textsf{v2.0}}
\maketitle

\begin{abstract}\noindent
\textsf{childdoc} is a \LaTeXe{} package
that enables the direct compilation
of document sections included by |\include|
to individual files.
\end{abstract}

\begingroup
\parskip0ex
\tableofcontents
\endgroup

%%%%%%%%%%%%%%%%%%%%%%%%%%%%%%%%%%%%%%%%%%%%%%%%%%%%%%%%%%%%%%%%%%%%%%%%%%%%%%%%
%%%%%%%%%%%%%%%%%%%%%%%%%%%%%%%%%%%%%%%%%%%%%%%%%%%%%%%%%%%%%%%%%%%%%%%%%%%%%%%%
\section{Introduction}

\LaTeX{} provides a mechanism to structure a large document (such as a book)
into a main file and several child files (containing the chapters)
using the |\include| command.
This mechanism is beneficial for documents
which span hundreds of pages in order to
make the source file(s) more manageable.
Moreover, compilation can be restricted to
selected child files by means of the |\includeonly| command.
The latter feature can be used to reduce the compilation time while editing
(this was significantly more useful in the earlier days of \LaTeX{})
or to generate a smaller document which is easier to navigate.
Another application of |\includeonly| is to generate
documents consisting of selected parts of the complete document.

However, there are a few drawbacks of the plain |\include| mechanism:
\begin{itemize}
\item
The child files cannot be compiled on their own,
they can only be compiled via the main file.
A naive editing environment
(such as a text editor with an option
to have the current file processed by \LaTeX)
may require one to switch to the main file before compiling;
attempting to compile the child file produces errors.
\item
The main file must be modified (each time)
to adjust the |\includeonly| command
to the present needs. This easily leaves the main file in a messy state.
\item
The generated document will always carry the filename
of the main document. This is inconvenient if
several child files are to be compiled and
to be kept for distribution.
\end{itemize}

The present package provides a simple interface
to make child files individually compilable by \LaTeX{}.
Compiling a child file then has the same effect as compiling
the main file with an |\includeonly| command
to select the appropriate child.
Moreover the generated document will carry the name of the child
rather than the main file.
This resolves all three above issues.

This feature is meant to make the editing of books,
thesis documents and lecture notes somewhat more convenient.
However, the package can also be used efficiently for
composing a series of documents (such as exercise sheets)
which are typically distributed individually.
It then assists the author in generating the individual documents
(potentially in different versions)
as well as a document containing the collected series.
Another application is in developing style files
or other kinds of included material
where compilation of the style file could redirect
to a sample or test file.

%%%%%%%%%%%%%%%%%%%%%%%%%%%%%%%%%%%%%%%%%%%%%%%%%%%%%%%%%%%%%%%%%%%%%%%%%%%%%%%%
%%%%%%%%%%%%%%%%%%%%%%%%%%%%%%%%%%%%%%%%%%%%%%%%%%%%%%%%%%%%%%%%%%%%%%%%%%%%%%%%
\section{Usage}

First of all, the package \textsf{childdoc} is \emph{not} a standard
\LaTeXe{} |.sty| style file! Therefore it needs to be invoked in
a non-standard way.

%%%%%%%%%%%%%%%%%%%%%%%%%%%%%%%%%%%%%%%%%%%%%%%%%%%%%%%%%%%%%%%%%%%%%%%%%%%%%%%%
\subsection{Included Files}
\label{sec:include}

%%%%%%%%%%%%%%%%%%%%%%%%%%%%%%%%%%%%%%%%
\DescribeMacro{\childdocmain}
To use the package, add the commands
\begin{center}
\begin{tabular}{l}
|% \iffalse
%
% childdoc.dtx Copyright (C) 2017-2018 Niklas Beisert
%
% This work may be distributed and/or modified under the
% conditions of the LaTeX Project Public License, either version 1.3
% of this license or (at your option) any later version.
% The latest version of this license is in
%   http://www.latex-project.org/lppl.txt
% and version 1.3 or later is part of all distributions of LaTeX
% version 2005/12/01 or later.
%
% This work has the LPPL maintenance status `maintained'.
%
% The Current Maintainer of this work is Niklas Beisert.
%
% This work consists of the files childdoc.dtx and childdoc.ins
% and the derived files childdoc.def and cdocsamp.tex with
% cdocsch1.tex, cdocsch2.tex, cdocsdrf.tex, cdocsfn1.tex, cdocsfn2.tex.
%
%<package>\ifdefined\childdocmain\endinput\fi
%<package>\ProvidesFile{childdoc.def}[2018/12/30 v2.0 child document driver]
%<samplemain>\ProvidesFile{cdocsamp.tex}[2018/12/30 v2.0 sample for childdoc]
%<*driver>
%\ProvidesFile{childdoc.drv}[2018/12/30 v2.0 childdoc reference manual file]
\PassOptionsToClass{10pt,a4paper}{article}
\documentclass{ltxdoc}

\usepackage[margin=35mm]{geometry}
\usepackage{hyperref}
\usepackage{hyperxmp}
\usepackage[usenames]{color}

\hypersetup{colorlinks=true}
\hypersetup{pdfstartview=FitH}
\hypersetup{pdfpagemode=UseNone}
\hypersetup{pdfsource={}}
\hypersetup{pdflang={en-UK}}
\hypersetup{pdfcopyright={Copyright 2017-2018 Niklas Beisert.
  This work may be distributed and/or modified under the
  conditions of the LaTeX Project Public License, either version 1.3
  of this license or (at your option) any later version.}}
\hypersetup{pdflicenseurl={http://www.latex-project.org/lppl.txt}}
\hypersetup{pdfcontactaddress={ETH Zurich, ITP, HIT K,
  Wolfgang-Pauli-Strasse 27}}
\hypersetup{pdfcontactpostcode={8093}}
\hypersetup{pdfcontactcity={Zurich}}
\hypersetup{pdfcontactcountry={Switzerland}}
\hypersetup{pdfcontactemail={nbeisert@itp.phys.ethz.ch}}
\hypersetup{pdfcontacturl={http://people.phys.ethz.ch/\xmptilde nbeisert/}}

\newcommand{\secref}[1]{\hyperref[#1]{section \ref*{#1}}}

\parskip1ex
\parindent0pt
\let\olditemize\itemize
\def\itemize{\olditemize\parskip0pt}

\begin{document}

\title{The \textsf{childdoc} Package}
\hypersetup{pdftitle={The childdoc Package}}
\author{Niklas Beisert\\[2ex]
  Institut f\"ur Theoretische Physik\\
  Eidgen\"ossische Technische Hochschule Z\"urich\\
  Wolfgang-Pauli-Strasse 27, 8093 Z\"urich, Switzerland\\[1ex]
  \href{mailto:nbeisert@itp.phys.ethz.ch}
  {\texttt{nbeisert@itp.phys.ethz.ch}}}
\hypersetup{pdfauthor={Niklas Beisert}}
\hypersetup{pdfsubject={Manual for the LaTeX2e Package childdoc}}
\date{30 December 2018, \textsf{v2.0}}
\maketitle

\begin{abstract}\noindent
\textsf{childdoc} is a \LaTeXe{} package
that enables the direct compilation
of document sections included by |\include|
to individual files.
\end{abstract}

\begingroup
\parskip0ex
\tableofcontents
\endgroup

%%%%%%%%%%%%%%%%%%%%%%%%%%%%%%%%%%%%%%%%%%%%%%%%%%%%%%%%%%%%%%%%%%%%%%%%%%%%%%%%
%%%%%%%%%%%%%%%%%%%%%%%%%%%%%%%%%%%%%%%%%%%%%%%%%%%%%%%%%%%%%%%%%%%%%%%%%%%%%%%%
\section{Introduction}

\LaTeX{} provides a mechanism to structure a large document (such as a book)
into a main file and several child files (containing the chapters)
using the |\include| command.
This mechanism is beneficial for documents
which span hundreds of pages in order to
make the source file(s) more manageable.
Moreover, compilation can be restricted to
selected child files by means of the |\includeonly| command.
The latter feature can be used to reduce the compilation time while editing
(this was significantly more useful in the earlier days of \LaTeX{})
or to generate a smaller document which is easier to navigate.
Another application of |\includeonly| is to generate
documents consisting of selected parts of the complete document.

However, there are a few drawbacks of the plain |\include| mechanism:
\begin{itemize}
\item
The child files cannot be compiled on their own,
they can only be compiled via the main file.
A naive editing environment
(such as a text editor with an option
to have the current file processed by \LaTeX)
may require one to switch to the main file before compiling;
attempting to compile the child file produces errors.
\item
The main file must be modified (each time)
to adjust the |\includeonly| command
to the present needs. This easily leaves the main file in a messy state.
\item
The generated document will always carry the filename
of the main document. This is inconvenient if
several child files are to be compiled and
to be kept for distribution.
\end{itemize}

The present package provides a simple interface
to make child files individually compilable by \LaTeX{}.
Compiling a child file then has the same effect as compiling
the main file with an |\includeonly| command
to select the appropriate child.
Moreover the generated document will carry the name of the child
rather than the main file.
This resolves all three above issues.

This feature is meant to make the editing of books,
thesis documents and lecture notes somewhat more convenient.
However, the package can also be used efficiently for
composing a series of documents (such as exercise sheets)
which are typically distributed individually.
It then assists the author in generating the individual documents
(potentially in different versions)
as well as a document containing the collected series.
Another application is in developing style files
or other kinds of included material
where compilation of the style file could redirect
to a sample or test file.

%%%%%%%%%%%%%%%%%%%%%%%%%%%%%%%%%%%%%%%%%%%%%%%%%%%%%%%%%%%%%%%%%%%%%%%%%%%%%%%%
%%%%%%%%%%%%%%%%%%%%%%%%%%%%%%%%%%%%%%%%%%%%%%%%%%%%%%%%%%%%%%%%%%%%%%%%%%%%%%%%
\section{Usage}

First of all, the package \textsf{childdoc} is \emph{not} a standard
\LaTeXe{} |.sty| style file! Therefore it needs to be invoked in
a non-standard way.

%%%%%%%%%%%%%%%%%%%%%%%%%%%%%%%%%%%%%%%%%%%%%%%%%%%%%%%%%%%%%%%%%%%%%%%%%%%%%%%%
\subsection{Included Files}
\label{sec:include}

%%%%%%%%%%%%%%%%%%%%%%%%%%%%%%%%%%%%%%%%
\DescribeMacro{\childdocmain}
To use the package, add the commands
\begin{center}
\begin{tabular}{l}
|\input{childdoc.def}|\\
|\childdocmain{}|\\
\end{tabular}
\end{center}
at the very top of the main \LaTeX{} file,
in particular \emph{before} the |\documentclass| statement!
The argument of |\childdocmain| should be left empty
(but it must be present).

%%%%%%%%%%%%%%%%%%%%%%%%%%%%%%%%%%%%%%%%
\DescribeMacro{\childdocof}
Furthermore, add the commands
\begin{center}
\begin{tabular}{l}
|\input{childdoc.def}|\\
|\childdocof{|\textit{main}|}|\\
\end{tabular}
\end{center}
at the top of every child file \textit{child}
which is included by |\include{|\textit{child}|}|
from within the main file
(or at least for those files to be compiled individually).
The argument \textit{main} must be the filename of the main file.

There are a couple of
considerations in setting up the main and child documents:

%%%%%%%%%%%%%%%%%%%%%%%%%%%%%%%%%%%%%%%%
\paragraph{Restrictions.}

Please note the following restrictions:
\begin{itemize}
\item
|\childdocmain| must be called with one argument \textit{main}
to ensure compatibility with earlier version of the package.
It must either be empty (|\childdocmain{}|)
or precisely match the filename of the main file in which it is specified.
See \secref{sec:detection} for further information.
\item
The filename \textit{main} must be specified without the |.tex| extension.
\item
The filename \textit{main} is case sensitive
(even in case-insensitive file systems)
due to internal string comparison.
\item
The argument \textit{main} should be fully expanded, it cannot be a macro.
\item
Subdirectories and special characters should be avoided in filenames.
\item
The command |\childdocmain{|\textit{main}|}| must be followed by a whitespace.
It should not be followed immediately by another command
or by a comment mark `|%|'.
This is because the \TeX{} parser reads the token immediately following
the argument of |\childdocmain| and puts it
at the beginning of every child section;
however, a white\-space is ignored.
\end{itemize}

%%%%%%%%%%%%%%%%%%%%%%%%%%%%%%%%%%%%%%%%
\paragraph{Content of Main File.}

It is advisable to place all content in the child files included by |\include|.
Any output contained in the main file will appear in all child documents
unless suppressed manually;
it cannot be suppressed automatically by the |\includeonly| directive
and thus should normally be avoided.
A method to include some content in the main file
by means of conditional processing is described in \secref{sec:conditional}.

%%%%%%%%%%%%%%%%%%%%%%%%%%%%%%%%%%%%%%%%
\paragraph{Page Numbering.}

When only a part of the document is compiled,
the appropriate numbering of pages
(as well as other status parameters)
is determined from the |.aux| files.
The latter contain information from previous passes.
However this information needs to propagate through
all intermediate child documents.
Therefore the page numbering in child documents may well
be inconsistent until the complete document is compiled at least once.

A useful (if unconventional) way to always ensure a consistent
page numbering is to restart the numbering in each child document
and denote the pages by `\textit{child}|.|\textit{page}'
where \textit{child} represents the chapter/section number of the child file.
This can be achieved by the command
|\numberwithin{page}{|\textit{child}|}|
of the \textsf{amsmath} package
where \textit{child} can be |chapter| or |section|
depending on the chosen structuring.
Alternatively, one can modify the macro |\thepage| appropriately
and reset the counter |page| at the start of each child file.

%%%%%%%%%%%%%%%%%%%%%%%%%%%%%%%%%%%%%%%%%%%%%%%%%%%%%%%%%%%%%%%%%%%%%%%%%%%%%%%%
\subsection{Conditional Processing}
\label{sec:conditional}

The package provides a mechanism to compile different versions
of a document. To customise the versions further some conditional processing
can come in handy to distinguish which version is being compiled.
The package provides two macros to describe the compilation context:

%%%%%%%%%%%%%%%%%%%%%%%%%%%%%%%%%%%%%%%%
\DescribeMacro{\ifchilddoc}
The conditional |\ifchilddoc| distinguishes between the compilation of
child documents and the main document:
%
\begin{center}
|\ifchilddoc |\textit{child-code}| |[|\||else |\textit{main-code}]| \||fi|
\end{center}

%%%%%%%%%%%%%%%%%%%%%%%%%%%%%%%%%%%%%%%%
\DescribeMacro{\childdocname}
\DescribeMacro{\childdocjob}
The macro |\childdocname| contains the filename (without extension)
of the main or child file being processed.
Note that |\childdocjob| will always contain the name of the main file.

%%%%%%%%%%%%%%%%%%%%%%%%%%%%%%%%%%%%%%%%
\paragraph{Title Page.}

Conditional processing can be used to include a title or banner page
in the main document when proper precautions are taken.
Importantly, the code in the main file should ensure that the page counter
(as well as other status parameters which are stored in the |.aux| files)
takes the same value after the conditional processing.
Otherwise the page numbers may take divergent values
depending on which part is compiled.

For example, a title page could be declared by:
%
\begin{center}
\begin{tabular}{l}
|\ifchilddoc\||else|\\
|\addtocounter{page}{-1}|\\
\textit{code for title page}\\
|\newpage|\\
|\||fi|
\end{tabular}
\end{center}
%
A banner page for the child documents can be generated by:
%
\begin{center}
\begin{tabular}{l}
|\ifchilddoc|\\
|\addtocounter{page}{-1}|\\
\textit{code for banner page}\\
|\newpage|\\
|\||fi|
\end{tabular}
\end{center}
%
Here one could write a message such as:
\begin{center}
|This is the part \childdocname{} of \childdocjob{}.|
\end{center}

%%%%%%%%%%%%%%%%%%%%%%%%%%%%%%%%%%%%%%%%%%%%%%%%%%%%%%%%%%%%%%%%%%%%%%%%%%%%%%%%
\subsection{Flags}
\label{sec:flags}

The package makes it easy to generate different versions
of the main or child documents.
To this end compilation flags can be defined
and assigned different default values.
They will be particularly useful in conjunction
with the forwarding mechanism described in \secref{sec:forward}.

For example, it may be useful to have a flag |\version|
which can be set to |draft| or |final|.
The document source will contain some conditional code
depending on the value of |\version|.
Suppose further, the flag should default to |final| for the main file
and to |draft| for child files
which is a natural assignment for editing the document.
This is achieved by placing the following code
in the preamble of the main document
(below the |\childdocmain| directive):
%
\begin{center}
\begin{tabular}{l}
|\ifchilddoc|\\
|\providecommand{\version}{draft}|\\
|\||else|\\
|\providecommand{\version}{final}|\\
|\||fi|
\end{tabular}
\end{center}
%
The definition by |\providecommand| makes sure
that previous definitions are not overwritten.
Further statements |\providecommand{\version}{...}|
can thus be added before the above code to override it.

For the main file, one might add a line
(between |\childdocmain| and the above block)
%
\begin{center}
|%\ifchilddoc\||else\providecommand{\version}{draft}\||fi|
\end{center}
%
which can be uncommented to produce a draft version.
Likewise one can add a line to the very top of a child file
(above the |\childdocof{|\textit{main}|}| directive)
%
\begin{center}
|%\providecommand{\version}{final}|
\end{center}
%
which can be uncommented to produce the final version of this child document.

%%%%%%%%%%%%%%%%%%%%%%%%%%%%%%%%%%%%%%%%%%%%%%%%%%%%%%%%%%%%%%%%%%%%%%%%%%%%%%%%
\subsection{Forwarding}
\label{sec:forward}

Different versions of the main or child documents
using compilation flags as described in \secref{sec:flags}
can be (permanently) stored in different files
for convenient compilation, viewing and distribution.
To this end, the package defines a command
to pass on compilation to a different file:

%%%%%%%%%%%%%%%%%%%%%%%%%%%%%%%%%%%%%%%%
\DescribeMacro{\childdocforward}
The command |\childdocforward| redirects processing to
another source file:
%
\begin{center}
\begin{tabular}{l}
|\input{childdoc.def}|\\
|\childdocforward[|\textit{main}|]{|\textit{dest}|}|\\
\end{tabular}
\end{center}
%
The argument \textit{dest} is the destination file
(without extension).
It should be the main file or one of the child files.
Note that further \textsf{childdoc} directives
such as |\childdocof| and |\childdocforward|
in the indicated file will be processed in this form.
The optional argument \textit{main}
passes on directly to the main file \textit{main}
while pretending to compile the child \textit{dest}.
This form behaves as if \textit{dest}
issues |\childdocof{|\textit{main}|}| right away,
and no further \textsf{childdoc} directives will be processed.

%%%%%%%%%%%%%%%%%%%%%%%%%%%%%%%%%%%%%%%%
\DescribeMacro{\...prefix}
In the alternative form |\childdocforwardprefix|,
%
\begin{center}
\begin{tabular}{l}
|\input{childdoc.def}|\\
|\childdocforwardprefix[|\textit{main}|]{|\textit{prefix}|}{|\textit{dest}|}|
\end{tabular}
\end{center}
%
the destination file is determined by a pattern
depending on the current file:
To make this work, the current file must be called
`{\textit{prefix}\hspace{0.2em}\textit{suffix}}'
with \textit{prefix} matching precisely the argument.
Processing is then passed on to the file
`{\textit{dest}\hspace{0.2em}\textit{suffix}}'.
Surely, the same effect is achieved by
directly specifying the
argument `{\textit{dest}\hspace{0.2em}\textit{suffix}}'
in the first form.
However, that requires to set up a different file
for each child. With the alternative form of the command
all these files can have exactly the same content
which simplifies setting them up and maintaining them.

For example, the following file |draft.tex|
with a compilation flag |\version| as described in \secref{sec:flags}
compiles the main document as a draft:
%
\begin{center}
\begin{tabular}{l}
|\def\version{draft}|\\
|\input{childdoc.def}|\\
|\childdocforward{|\textit{main}|}|
\end{tabular}
\end{center}
%
Likewise, the following files |final|\textit{nn}|.tex|
compile the final version of the child document
|child|\textit{nn}|.tex|:
%
\begin{center}
\begin{tabular}{l}
|\def\version{final}|\\
|\input{childdoc.def}|\\
|\childdocforwardprefix{final}{child}|
\end{tabular}
\end{center}
%

Note that when several versions of a main file and/or of each child file
are to be generated, it may be convenient to set up a |Makefile| or
shell script to automatise the process.

%%%%%%%%%%%%%%%%%%%%%%%%%%%%%%%%%%%%%%%%%%%%%%%%%%%%%%%%%%%%%%%%%%%%%%%%%%%%%%%%
\subsection{Command Line Processing}
\label{sec:commandline}

The effect of redirection files can also be achieved by invoking
the \LaTeX{} compiler with a more elaborate command line.
Most conveniently this should be done as part
of a shell script or a |Makefile|.

When using \textsf{childdoc} in the main file, the following
command lines effectively perform a redirection
(note that depending on the shell being used,
backslashes may have to be doubled: `|\|' $\to$ `|\\|'):
%
\begin{center}
|... -jobname "|\textit{target}|" |\\|"|[\textit{flags}]%
|\input{childdoc.def}\childdocforward[|\textit{main}|]{|\textit{dest}|}"|
\end{center}
%
Here \textit{target} is the name of the output file,
\textit{main} is the name of the main file
and \textit{dest} is the name of the main or child file to be processed
(all filenames without extensions).
The optional argument \textit{main} can be omitted
if \textit{main} matches \textit{dest}.
Optionally, compilation \textit{flags} can be defined via |\def| commands.
This command line makes the \TeX{} engine believe
it is compiling the file \textit{target}
whose content is specified as the latter parameter.
The provided code then forwards the processing to
\textit{main} or \textit{dest} as described in \secref{sec:forward}.

%%%%%%%%%%%%%%%%%%%%%%%%%%%%%%%%%%%%%%%%%%%%%%%%%%%%%%%%%%%%%%%%%%%%%%%%%%%%%%%%
\subsection{Include by Input}
\label{sec:input}

Including child documents by |\include| has some restrictions by design.
Most notably, the content of a child document always occupies
its own set of pages; pages cannot be shared between child documents.
Usually, this behaviour makes perfect sense
because each child document contain an essential part of the document.
However, in some situations it may be desirable to compose
a document from a collection of parts
without having mandatory page breaks between then.
For this case, the package
provides a mechanism to include parts
by |\input| which can also be processed individually.
However, by construction this mechanism
requires manual handling of the content to be output.

%%%%%%%%%%%%%%%%%%%%%%%%%%%%%%%%%%%%%%%%
\DescribeMacro{\ifchilddocmanual}
The main file should be prepared as usual, see \secref{sec:include}.
However, the document body must make a distinction
between processing of an individual part and of the main document, e.g.:
%
\begin{center}
\begin{tabular}{l}
|\ifchilddocmanual|\\
|\input{\childdocname}|\\
|\||else|\\
\textit{document body with }|\input{|\textit{part}|}|\\
|\||fi|
\end{tabular}
\end{center}
%
The conditional |\ifchilddocmanual| is true whenever
a part to be included by |\input| is being compiled,
and the name of the part is stored in |\childdocname|.

%%%%%%%%%%%%%%%%%%%%%%%%%%%%%%%%%%%%%%%%
\DescribeMacro{\childdocby}
Each part to be included by |\input| should start with:
%
\begin{center}
\begin{tabular}{l}
|\input{childdoc.def}|\\
|\childdocby{|\textit{main}|}|\\
\end{tabular}
\end{center}
%
The directive |\childdocby| is similar to |\childdocof|
described in \secref{sec:include},
but the subsequent selection of content must be done manually.
To that end, both |\ifchilddoc| and |\ifchilddocmanual|
will be true upon processing of a part,
and the name of the part is stored in |\childdocname|.
Note that |\jobname| will be set to the filename of the current part
so that each part receives an individual |.aux| file
that does not interfere with the |.aux| file(s) of the main document.
This behaviour can be altered by the alternative form
|\childdocby[*]{|\textit{main}|}| (with a non-empty optional argument)
which uses the |.aux| file of the main document
by setting |\jobname| to \textit{main}.

%%%%%%%%%%%%%%%%%%%%%%%%%%%%%%%%%%%%%%%%%%%%%%%%%%%%%%%%%%%%%%%%%%%%%%%%%%%%%%%%
\subsection{Driver Development}
\label{sec:driver}

The \textsf{childdoc} mechanism can also be use for the development
of definition files such as \LaTeX{} styles or classes.
This case differs from the above setup with multiple parts
included by |\include| in that no |\includeonly| should be invoked.
This can be achieved by starting the include file
(before |\ProvidesPackage|) with:
%
\begin{center}
\begin{tabular}{l}
|\input{childdoc.def}|\\
|\childdocforward{|\textit{main}|}|\\
\end{tabular}
\end{center}
%
or alternatively with:
%
\begin{center}
\begin{tabular}{l}
|\input{childdoc.def}|\\
|\childdocby{|\textit{main}|}|\\
\end{tabular}
\end{center}
%
Both forms have slightly different effects as described above.
The main file is prepared as usual, see \secref{sec:include}.

%%%%%%%%%%%%%%%%%%%%%%%%%%%%%%%%%%%%%%%%%%%%%%%%%%%%%%%%%%%%%%%%%%%%%%%%%%%%%%%%
\subsection{Legacy Detection}
\label{sec:detection}

The directive |\childdocmain| in the main file can detect
whether the complete document or merely a child is to be compiled
even without using the directive |\childdocof|.
This method is deprecated because it is less robust
and there is no compelling reason to use it;
it is merely provided for backward compatibility
and it may be removed in future versions.

If the detection mechanism is to be used,
it is mandatory to correctly specify
the filename of the main file as the argument of |\childdocmain|:
%
\begin{center}
\begin{tabular}{l}
|\input{childdoc.def}|\\
|\childdocmain{|\textit{main}|}|\\
\end{tabular}
\end{center}
%
If |\jobname| does not match the argument \textit{main} of |\childdocmain|,
it is assumed that |\jobname| points to the child file to be compiled.
When using |\childdocmain| with the main file specified as argument,
it suffices to start a child file
with just |\input{|\textit{main}|}|
without loading of the package and using |\childdocof|.
If instead all processing is done
with the appropriate \textsf{childdoc} directives,
the argument of \textit{main} of |\childdocmain| can be empty.

An alternative version of the command line processing described
in \secref{sec:commandline} using the detection mechanism reads:
%
\begin{center}
|... -jobname "|\textit{target}|" "|[\textit{flags}]%
[|\def\jobname{|\textit{dest}|}|]|\input{|\textit{main}|}"|
\end{center}

%%%%%%%%%%%%%%%%%%%%%%%%%%%%%%%%%%%%%%%%%%%%%%%%%%%%%%%%%%%%%%%%%%%%%%%%%%%%%%%%
\subsection{Manual Code}
\label{sec:manual}

In case one cannot be certain whether the definitions file |childdoc.def|
is installed on the target \TeX{} distribution
and one prefers not to ship it,
it is conceivable to paste a few relevant commands into the sources.

To that end, drop all statements |\input{childdoc.def}|
and perform the replacements as outlined below.
Instead of |\childdocmain{|\textit{main}|}| add the following code
to the top of the main file:
%
\begin{center}
\begin{tabular}{l}
|\||ifdefined\childdocname\endinput\||fi\newif\ifchilddoc|\\
|\edef\childdocname{\scantokens\expandafter{\jobname\noexpand}}|\\
|\def\childdocmain{|\textit{main}|}\||ifx\childdocmain\childdocname\||else|\\
|\childdoctrue\includeonly{\childdocname}\let\jobname\childdocmain\||fi|\\
\end{tabular}
\end{center}
%
Instead of |\childdocof{|\textit{main}|}| just include the main file
at the top of each child file:
%
\begin{center}
|\input{|\textit{main}|}|
\end{center}
%
A simple redirection |\childdocforward{|\textit{dest}|}| is achieved by:
%
\begin{center}
|\def\jobname{|\textit{dest}|}\input{\jobname}|
\end{center}
%
The redirection with prefix
|\childdocforwardprefix[|\textit{prefix}|]{|\textit{dest}|}|
is accomplished by:
%
\begin{center}
\begin{tabular}{l}
|{\edef\jobname{\scantokens\expandafter{\jobname\noexpand}}|\\
|\def\redirectjob |\textit{prefix}|#1~~~{\gdef\jobname{|\textit{dest}|#1}}|\\
|\expandafter\redirectjob\jobname~~~}\input{\jobname}|
\end{tabular}
\end{center}

In an alternative approach,
child documents can be compiled by a specific command line
without additional code or specific definitions:
%
\begin{center}
|... -jobname "|\textit{target}|" "|[\textit{flags}]%
|\includeonly{|\textit{dest}|}\input{|\textit{main}|}"|
\end{center}
%

%%%%%%%%%%%%%%%%%%%%%%%%%%%%%%%%%%%%%%%%%%%%%%%%%%%%%%%%%%%%%%%%%%%%%%%%%%%%%%%%
%%%%%%%%%%%%%%%%%%%%%%%%%%%%%%%%%%%%%%%%%%%%%%%%%%%%%%%%%%%%%%%%%%%%%%%%%%%%%%%%
\section{Information}

%%%%%%%%%%%%%%%%%%%%%%%%%%%%%%%%%%%%%%%%%%%%%%%%%%%%%%%%%%%%%%%%%%%%%%%%%%%%%%%%
\subsection{Copyright}

Copyright \copyright{} 2017--2018 Niklas Beisert

This work may be distributed and/or modified under the
conditions of the \LaTeX{} Project Public License, either version 1.3
of this license or (at your option) any later version.
The latest version of this license is in
  \url{http://www.latex-project.org/lppl.txt}
and version 1.3 or later is part of all distributions of \LaTeX{}
version 2005/12/01 or later.

This work has the LPPL maintenance status `maintained'.

The Current Maintainer of this work is Niklas Beisert.

This work consists of the files |README.txt|, |childdoc.ins| and |childdoc.dtx|
as well as the derived files |childdoc.def|, |cdocsamp.tex|
with |cdocsch1.tex|, |cdocsch2.tex|, |cdocspt3.tex|, |cdocspt4.tex|,
|cdocsdrf.tex|, |cdocsfn1.tex|, |cdocsfn2.tex|
as well as |childdoc.pdf|.

%%%%%%%%%%%%%%%%%%%%%%%%%%%%%%%%%%%%%%%%%%%%%%%%%%%%%%%%%%%%%%%%%%%%%%%%%%%%%%%%
\subsection{Files and Installation}

The package consists of the files:
%
\begin{center}
\begin{tabular}{ll}
    |README.txt|   & readme file \\
    |childdoc.ins| & installation file \\
    |childdoc.dtx| & source file \\
    |childdoc.def| & definition file \\
    |cdocsamp.tex| & sample main file \\
    |cdocsch1.tex| & sample include file \\
    |cdocsch2.tex| & sample include file \\
    |cdocspt3.tex| & sample part file \\
    |cdocspt4.tex| & sample part file \\
    |cdocsdrf.tex| & sample redirection file \\
    |cdocsfn1.tex| & sample redirection file \\
    |cdocsfn2.tex| & sample redirection file \\
    |childdoc.pdf| & manual
\end{tabular}
\end{center}
%
The distribution consists of the files
|README.txt|, |childdoc.ins| and |childdoc.dtx|.
%
\begin{itemize}
\item
Run (pdf)\LaTeX{} on |childdoc.dtx|
to compile the manual |childdoc.pdf| (this file).
\item
Run \LaTeX{} on |childdoc.ins| to create the definitions file |childdoc.def|
and the sample |cdocsamp.tex| with include files
|cdocsch1.tex|, |cdocsch2.tex|, |cdocspt3.tex|, |cdocspt4.tex|,
|cdocsdrf.tex|, |cdocsfn1.tex|, |cdocsfn2.tex|.
Then copy the file |childdoc.def| to an appropriate directory of your \LaTeX{}
distribution, e.g.\ \textit{texmf-root}|/tex/latex/childdoc|.
\end{itemize}

%%%%%%%%%%%%%%%%%%%%%%%%%%%%%%%%%%%%%%%%%%%%%%%%%%%%%%%%%%%%%%%%%%%%%%%%%%%%%%%%
\subsection{Related CTAN Packages}

There are several other packages which offer a similar functionality:
%
\begin{itemize}
\item
The packages
\href{http://ctan.org/pkg/docmute}{\textsf{docmute}},
\href{http://ctan.org/pkg/includex}{\textsf{includex}} and
\href{http://ctan.org/pkg/standalone}{\textsf{standalone}}
provide commands to include only the document body of
a child file thus allowing both files to be compiled individually.
\item
The packages \href{http://ctan.org/pkg/subdocs}{\textsf{subdocs}}
and \href{http://ctan.org/pkg/subfiles}{\textsf{subfiles}}
provide structures in which the main and child documents can be
encapsulated and allowing them to be compiled individually.
The inclusion mechanism is different from the conventional |\include|.
\item
The package \href{http://ctan.org/pkg/combine}{\textsf{combine}}
is an elaborate solution to combine several documents into one.
\end{itemize}
%
See also the CTAN topic \href{http://ctan.org/topic/subdocs}{\textsf{subdocs}}
for further related packages.
The present package differs from the above solutions in that
a document structure constructed with the conventional |\include| mechanism
just needs two extra commands at the top of every file
such that all constituent files can be compiled individually.

%%%%%%%%%%%%%%%%%%%%%%%%%%%%%%%%%%%%%%%%%%%%%%%%%%%%%%%%%%%%%%%%%%%%%%%%%%%%%%%%
%\subsection{Feature Suggestions}
%
%The following is a list of features which may be useful for future
%versions of this package:
%%
%\begin{itemize}
%\item
%\ldots
%\end{itemize}

%%%%%%%%%%%%%%%%%%%%%%%%%%%%%%%%%%%%%%%%%%%%%%%%%%%%%%%%%%%%%%%%%%%%%%%%%%%%%%%%
\subsection{Revision History}

%%%%%%%%%%%%%%%%%%%%%%%%%%%%%%%%%%%%%%%%
\paragraph{v2.0:} 2018/12/30

\begin{itemize}
\item
immediate forward processing
\item
added |\childdocby| mechanism
\item
manual restructured
\end{itemize}

%%%%%%%%%%%%%%%%%%%%%%%%%%%%%%%%%%%%%%%%
\paragraph{v1.6:} 2018/01/17

\begin{itemize}
\item
application for development of include files
\item
corrections to manual
\end{itemize}

%%%%%%%%%%%%%%%%%%%%%%%%%%%%%%%%%%%%%%%%
\paragraph{v1.5:} 2017/05/21

\begin{itemize}
\item
more complete structuring introduced
\item
|\childdocof| introduced
\item
|\childdoc| renamed to |\childdocmain|
\item
|\childredirect| renamed to |\childdocforward| and |\childdocforwardprefix|
and functionality expanded
\end{itemize}

%%%%%%%%%%%%%%%%%%%%%%%%%%%%%%%%%%%%%%%%
\paragraph{v1.0:} 2017/04/27

\begin{itemize}
\item
manual and install package
\item
first version published on CTAN
\end{itemize}

%%%%%%%%%%%%%%%%%%%%%%%%%%%%%%%%%%%%%%%%
\paragraph{v0.6:} 2017/04/26

\begin{itemize}
\item
redirection mechanism added
\end{itemize}

%%%%%%%%%%%%%%%%%%%%%%%%%%%%%%%%%%%%%%%%
\paragraph{v0.5:} 2017/04/26

\begin{itemize}
\item
functionality in definition file
\end{itemize}


%%%%%%%%%%%%%%%%%%%%%%%%%%%%%%%%%%%%%%%%%%%%%%%%%%%%%%%%%%%%%%%%%%%%%%%%%%%%%%%%
%%%%%%%%%%%%%%%%%%%%%%%%%%%%%%%%%%%%%%%%%%%%%%%%%%%%%%%%%%%%%%%%%%%%%%%%%%%%%%%%
%%%%%%%%%%%%%%%%%%%%%%%%%%%%%%%%%%%%%%%%%%%%%%%%%%%%%%%%%%%%%%%%%%%%%%%%%%%%%%%%
\appendix

\settowidth\MacroIndent{\rmfamily\scriptsize 000\ }

 \DocInput{childdoc.dtx}

\end{document}
%</driver>
% \fi
%
% %%%%%%%%%%%%%%%%%%%%%%%%%%%%%%%%%%%%%%%%%%%%%%%%%%%%%%%%%%%%%%%%%%%%%%%%%%%%%%
% %%%%%%%%%%%%%%%%%%%%%%%%%%%%%%%%%%%%%%%%%%%%%%%%%%%%%%%%%%%%%%%%%%%%%%%%%%%%%%
% \section{Sample}
%\iffalse
%<*samplemain>
%\fi
%
% The following presents a sample document
% with two chapters, two parts, a title page,
% a compile flag as well as three forwarding files to set the flag.
% It consists of eight |.tex| files:
% \begin{center}
% \begin{tabular}{ll}
% |cdocsamp.tex|&main file\\
% |cdocsch1.tex|&include file for chapter 1\\
% |cdocsch2.tex|&include file for chapter 2\\
% |cdocspt3.tex|&include file for part 3\\
% |cdocspt4.tex|&include file for part 4\\
% |cdocsdrf.tex|&forwarding file for main file in draft mode\\
% |cdocsfi1.tex|&forwarding file for final version of chapter 1\\
% |cdocsfi2.tex|&forwarding file for final version of chapter 2\\
% \end{tabular}
% \end{center}
% Each of the eight files can be compiled directly by the \LaTeX{} compiler.
%
% %%%%%%%%%%%%%%%%%%%%%%%%%%%%%%%%%%%%%%
% \paragraph{Main File.}
%
% The main file is called |cdocsamp.tex|.
%
% Load the \textsf{childdoc} definitions and
% declare the filename for the main document:
%    \begin{macrocode}
\input{childdoc.def}
\childdocmain{}
%    \end{macrocode}

% Optional override for |\version| flag:
%    \begin{macrocode}
%%\ifchilddoc\else\providecommand{\version}{draft}\fi
%    \end{macrocode}

% Define the default values for the |\version| flag
% (|final| for the main file and |draft| for childs):
%    \begin{macrocode}
\ifchilddoc
\providecommand{\version}{draft}
\else
\providecommand{\version}{final}
\fi
%    \end{macrocode}

% Load the standard document class:
%    \begin{macrocode}
\documentclass[12pt]{article}
%    \end{macrocode}

% Start the document body:
%    \begin{macrocode}
\begin{document}
%    \end{macrocode}

% Declare a title page.
% Print title, part of document being processed and version flag:
%    \begin{macrocode}
\addtocounter{page}{-1}
\begin{center}
{\LARGE\bfseries{}childdoc example\par}
\vspace{1cm}
\ifchilddoc
\ifchilddocmanual part\else chapter\fi:
`\childdocname' of `\childdocjob'\par
\else
main document: `\childdocjob'\par
\fi
version: \version\par
\end{center}
\newpage
%    \end{macrocode}

% Manually include selected file,
% otherwise process as usual:
%    \begin{macrocode}
\ifchilddocmanual
\section*{part `\childdocname'}
\input{\childdocname}
\else
%    \end{macrocode}

% Include the two chapters:
%    \begin{macrocode}
\include{cdocsch1}
\include{cdocsch2}
%    \end{macrocode}

% Include the two parts unless only chapters should be displayed:
%    \begin{macrocode}
\ifchilddoc\else
\section{part three}
\input{cdocspt3}
\section{part four}
\input{cdocspt4}
\fi
%    \end{macrocode}

% Process as usual until here:
%    \begin{macrocode}
\fi
%    \end{macrocode}

% End of document body:
%    \begin{macrocode}
\end{document}
%    \end{macrocode}
%\iffalse
%</samplemain>
%\fi
%
% %%%%%%%%%%%%%%%%%%%%%%%%%%%%%%%%%%%%%%
% \paragraph{Chapter Include Files.}
%
% The include files are called |cdocsch1.tex| and |cdocsch2.tex|.
%
%\iffalse
%<*samplechap1|samplechap2>
%\fi

% Optional override for |\version| flag:
%    \begin{macrocode}
%%\providecommand{\version}{final}
%    \end{macrocode}

% Include the main document:
%    \begin{macrocode}
\input{childdoc.def}
\childdocof{cdocsamp}
%    \end{macrocode}

%\iffalse
%</samplechap1|samplechap2>
%\fi
%
%\iffalse
%<*samplechap1>
%\fi
% Some text for chapter 1:
%    \begin{macrocode}
\section{one}
some text in chapter one
%    \end{macrocode}

%\iffalse
%</samplechap1>
%\fi
% Some text for chapter 2:
%\iffalse
%<*samplechap2>
%\fi
%    \begin{macrocode}
\section{two}
more text in chapter two
%    \end{macrocode}

%\iffalse
%</samplechap2>
%\fi
%
% %%%%%%%%%%%%%%%%%%%%%%%%%%%%%%%%%%%%%%
% \paragraph{Part Include Files.}
%
% The include files are called |cdocspt3.tex| and |cdocspt4.tex|.
%
%\iffalse
%<*samplepart3|samplepart4>
%\fi

% Optional override for |\version| flag:
%    \begin{macrocode}
%%\providecommand{\version}{final}
%    \end{macrocode}

% Include the main document:
%    \begin{macrocode}
\input{childdoc.def}
\childdocby{cdocsamp}
%    \end{macrocode}

%\iffalse
%</samplepart3|samplepart4>
%\fi
%
%\iffalse
%<*samplepart3>
%\fi
% Some text for part 3:
%    \begin{macrocode}
some text in part three
%    \end{macrocode}

%\iffalse
%</samplepart3>
%\fi
% Some text for part 4:
%\iffalse
%<*samplepart4>
%\fi
%    \begin{macrocode}
more text in part four
%    \end{macrocode}

%\iffalse
%</samplepart4>
%\fi
%
% %%%%%%%%%%%%%%%%%%%%%%%%%%%%%%%%%%%%%%
% \paragraph{Forwarding for a Complete Draft.}
%
% The following forwarding file |cdocsdrf.tex|
% compiles the main document in draft mode:
%\iffalse
%<*sampledraft>
%\fi
%    \begin{macrocode}
\def\version{draft}
\input{childdoc.def}
\childdocforward{cdocsamp}
%    \end{macrocode}

%\iffalse
%</sampledraft>
%\fi
%
% %%%%%%%%%%%%%%%%%%%%%%%%%%%%%%%%%%%%%%
% \paragraph{Forwarding for Final Version of the Chapters.}
%
% The following forwarding files |cdocsfn1.tex| and |cdocsfn2.tex|
% (with identical content)
% compile the final versions of the child documents
% |cdocsch1.tex| and |cdocsch2.tex|, respectively:
%\iffalse
%<*samplefinal>
%\fi
%    \begin{macrocode}
\def\version{final}
\input{childdoc.def}
\childdocforwardprefix[cdocsamp]{cdocsfn}{cdocsch}
%    \end{macrocode}

%\iffalse
%</samplefinal>
%\fi
%
% %%%%%%%%%%%%%%%%%%%%%%%%%%%%%%%%%%%%%%
% \paragraph{Command Line Processing.}
%
% The following three command lines generate the output files
% |cdocscld|, |cdocscl1| and |cdocscl2|
% which should be identical to
% |cdocsdrf|, |cdocsch1| and |cdocsfn2|, respectively:
% \begin{center}
% \begin{tabular}{l}
% |latex -jobname cdocscld \|\\
% |  "\def\version{draft}\input{childdoc.def}\childdocforward{cdocsamp}"|\\
% |latex -jobname cdocscl1 \|\\
% |  "\input{childdoc.def}\childdocforward[cdocsamp]{cdocsch1}"|\\
% |latex -jobname cdocscl2 \|\\
% |  "\def\version{final}\input{childdoc.def}\childdocforward{cdocsch2}"|
% \end{tabular}
% \end{center}
% Note that the trailing backslash on each first line
% merely continues the input to the second line
% (for convenient cut ant paste).
% Furthermore, the command |latex| can be replaced by any
% of its alternative versions such as |pdflatex|.
%
% %%%%%%%%%%%%%%%%%%%%%%%%%%%%%%%%%%%%%%%%%%%%%%%%%%%%%%%%%%%%%%%%%%%%%%%%%%%%%%
% %%%%%%%%%%%%%%%%%%%%%%%%%%%%%%%%%%%%%%%%%%%%%%%%%%%%%%%%%%%%%%%%%%%%%%%%%%%%%%
% \section{Implementation}
%\iffalse
%<*package>
%\fi
%
% This section describes the definitions file |childdoc.def|.

% The definitions cannot be loaded using |\usepackage| or |\RequirePackage|
% which has a mechanism to prevent loading a style file more than once.
% When loading the definitions by means of |\input|
% multiple instances have to be prevented manually:
%\iffalse
%This code needs to be before the `\ProvidesFile' directive
%which is defined at the beginning of this file.
%Therefore it is also placed there and commented out here.
%</package>
%<*discard>
%\fi
%    \begin{macrocode}
\ifdefined\childdocmain\endinput\fi
%    \end{macrocode}
%\iffalse
%</discard>
%<*package>
%\fi
%
% \macro{\ifchilddoc}
% \macro{\ifchilddocmanual}
% The conditional |\ifchilddoc| tells whether a
% child (true) or main (false) document is being compiled.
% The conditional |\ifchilddocmanual| tells whether
% the |\includeonly| mechanism is used (false) or
% the selection of child files must be performed manually (true).
% The definitions initialise to false:
%    \begin{macrocode}
\newif\ifchilddoc
\newif\ifchilddocmanual
%    \end{macrocode}

% \macro{\childdocname}
% \macro{\childdocjob}
% The macro |\childdocname| stores the name of the main document
% to be compiled. The macro |\childdocjob| stores the name of
% the document on which the \LaTeX{} compiler was originally invoked.
% The content of |\jobname| cannot be compared
% to filenames specified in the source due to different catcodes.
% The following code rescans |\jobname|, stores the result
% in |\childdocname| and saves a copy in |\childdocjob|:
%    \begin{macrocode}
\edef\childdocname{\scantokens\expandafter{\jobname\noexpand}}
\let\childdocjob\childdocname
%    \end{macrocode}

% \macro{\childdocdisable}
% The macro |\childdocdisable| prevents the main file
% from being processed more than once.
% At this stage, the main document command |\childdocmain|
% is assumed to be called once again where it should do nothing.
% Any subsequent call to it should prevent
% a secondary processing of the main document
% It overwrites the forwarding commands
% |\childdocof| and |\childdocforward|
% with empty macros to prevent further inclusions of the main document:
%    \begin{macrocode}
\newcommand{\childdocdisable}
{
  \renewcommand{\childdocmain}[1]{\renewcommand{\childdocmain}[1]{\endinput}}
  \renewcommand{\childdocof}[1]{}
  \renewcommand{\childdocby}[2][]{}
  \renewcommand{\childdocforward}[2][]{}
  \renewcommand{\childdocdisable}{}
}
%    \end{macrocode}

% \macro{\childdocmain}
% The macro |\childdocmain| is to be called at the top of the main file
% with nothing or the main filename (without extension) as argument.
% First, it breaks loops.
% If the argument is not empty and does not match |\childdocname|
% (which is set by the first inclusion of |childdoc.def|),
% |\ifchilddoc| is set to true, |\includeonly| is applied to the child file
% and |\jobname| is set to the main file
% (for proper handling of |.aux| files):
%    \begin{macrocode}
\newcommand{\childdocmain}[1]
{
  \childdocdisable\childdocmain{}
  \if?#1?\else
    \begingroup
      \def\childdoctmp{#1}
      \ifx\childdoctmp\childdocname
        \def\childdoctmp{}
      \else
        \def\childdoctmp
        {
          \childdoctrue
          \includeonly{\childdocname}
          \def\childdocjob{#1}
          \def\jobname{#1}
        }
      \fi
      \expandafter
    \endgroup
    \childdoctmp
  \fi
}
%    \end{macrocode}

% \macro{\childdocof}
% The command |\childdocof| redirects
% compilation to the main file |#1|.
%    \begin{macrocode}
\newcommand{\childdocof}[1]
{
  \childdocdisable
  \childdoctrue
  \includeonly{\childdocname}
  \def\jobname{#1}
  \def\childdocjob{#1}
  \input{#1}
}
%    \end{macrocode}

% \macro{\childdocby}
% The command |\childdocby| ....
%    \begin{macrocode}
\newcommand{\childdocby}[2][]
{
  \childdocdisable
  \childdoctrue
  \childdocmanualtrue
  \if?#1?\else
    \def\jobname{#2}
  \fi
  \def\childdocjob{#2}
  \input{#2}
  \endinput
}
%    \end{macrocode}

% \macro{\childdocforward}
% The command |\childdocforward| redirects
% compilation to the main file or
% (if the optional argument is given) a child file.
% Parameters are set as if the main file
% or a child file starting with |\childdocof| was compiled.
% Then compilation is handed over to the main file:
%    \begin{macrocode}
\newcommand{\childdocforward}[2][]
{
  \begingroup
    \if?#1?
      \def\childdoctmp
      {
        \def\childdocname{#2}
        \def\childdocjob{#2}
        \def\jobname{#2}
        \input{#2}
        \endinput
      }
    \else
      \def\childdoctmp
      {
        \childdocdisable
        \def\childdocname{#2}
        \childdoctrue
        \includeonly{#2}
        \def\childdocjob{#1}
        \def\jobname{#1}
        \input{#1}
        \endinput
      }
    \fi
    \expandafter
  \endgroup
  \childdoctmp
}
%    \end{macrocode}

% \macro{\childdocforwardprefix}
% The command |\childdocforwardprefix| redirects
% compilation to the main or a child file by means of a pattern.
% The prefix |#1| in the current filename is replaced by |#2|
% and the suffix of the current filename is kept
% (it is assumed that the filename does not contain the substring `|~~~|'
% which is used as a delimiter).
% Compilation is handed over to the new file by |\childdocforward|:
%    \begin{macrocode}
\newcommand{\childdocforwardprefix}[3][]
{
  \begingroup
    \def\childdocextract #2##1~~~{\def\childdoctmp{\childdocforward[#1]{#3##1}}}
    \expandafter\childdocextract\childdocname~~~
    \expandafter
  \endgroup
  \childdoctmp
}
%    \end{macrocode}

% \macro{\childdoc}
% The deprecated macro |\childdoc| is a legacy version of |\childdocmain|:
%    \begin{macrocode}
\newcommand{\childdoc}{\childdocmain}
%    \end{macrocode}

% \macro{\childdocredirect}
% The deprecated macro |\childdocredirect| is a legacy version
% of |\childdocforward| and |\childdocforwardprefix|:
%    \begin{macrocode}
\newcommand{\childdocredirect}[2][]
{
  \begingroup
    \if?#1?
      \def\childdoctmp{\childdocforward{#2}}
    \else
      \def\childdoctmp{\childdocforwardprefix{#1}{#2}}
    \fi
    \expandafter
  \endgroup
  \childdoctmp
}
%    \end{macrocode}

%\iffalse
%</package>
%\fi
%
\endinput
|\\
|\childdocmain{}|\\
\end{tabular}
\end{center}
at the very top of the main \LaTeX{} file,
in particular \emph{before} the |\documentclass| statement!
The argument of |\childdocmain| should be left empty
(but it must be present).

%%%%%%%%%%%%%%%%%%%%%%%%%%%%%%%%%%%%%%%%
\DescribeMacro{\childdocof}
Furthermore, add the commands
\begin{center}
\begin{tabular}{l}
|% \iffalse
%
% childdoc.dtx Copyright (C) 2017-2018 Niklas Beisert
%
% This work may be distributed and/or modified under the
% conditions of the LaTeX Project Public License, either version 1.3
% of this license or (at your option) any later version.
% The latest version of this license is in
%   http://www.latex-project.org/lppl.txt
% and version 1.3 or later is part of all distributions of LaTeX
% version 2005/12/01 or later.
%
% This work has the LPPL maintenance status `maintained'.
%
% The Current Maintainer of this work is Niklas Beisert.
%
% This work consists of the files childdoc.dtx and childdoc.ins
% and the derived files childdoc.def and cdocsamp.tex with
% cdocsch1.tex, cdocsch2.tex, cdocsdrf.tex, cdocsfn1.tex, cdocsfn2.tex.
%
%<package>\ifdefined\childdocmain\endinput\fi
%<package>\ProvidesFile{childdoc.def}[2018/12/30 v2.0 child document driver]
%<samplemain>\ProvidesFile{cdocsamp.tex}[2018/12/30 v2.0 sample for childdoc]
%<*driver>
%\ProvidesFile{childdoc.drv}[2018/12/30 v2.0 childdoc reference manual file]
\PassOptionsToClass{10pt,a4paper}{article}
\documentclass{ltxdoc}

\usepackage[margin=35mm]{geometry}
\usepackage{hyperref}
\usepackage{hyperxmp}
\usepackage[usenames]{color}

\hypersetup{colorlinks=true}
\hypersetup{pdfstartview=FitH}
\hypersetup{pdfpagemode=UseNone}
\hypersetup{pdfsource={}}
\hypersetup{pdflang={en-UK}}
\hypersetup{pdfcopyright={Copyright 2017-2018 Niklas Beisert.
  This work may be distributed and/or modified under the
  conditions of the LaTeX Project Public License, either version 1.3
  of this license or (at your option) any later version.}}
\hypersetup{pdflicenseurl={http://www.latex-project.org/lppl.txt}}
\hypersetup{pdfcontactaddress={ETH Zurich, ITP, HIT K,
  Wolfgang-Pauli-Strasse 27}}
\hypersetup{pdfcontactpostcode={8093}}
\hypersetup{pdfcontactcity={Zurich}}
\hypersetup{pdfcontactcountry={Switzerland}}
\hypersetup{pdfcontactemail={nbeisert@itp.phys.ethz.ch}}
\hypersetup{pdfcontacturl={http://people.phys.ethz.ch/\xmptilde nbeisert/}}

\newcommand{\secref}[1]{\hyperref[#1]{section \ref*{#1}}}

\parskip1ex
\parindent0pt
\let\olditemize\itemize
\def\itemize{\olditemize\parskip0pt}

\begin{document}

\title{The \textsf{childdoc} Package}
\hypersetup{pdftitle={The childdoc Package}}
\author{Niklas Beisert\\[2ex]
  Institut f\"ur Theoretische Physik\\
  Eidgen\"ossische Technische Hochschule Z\"urich\\
  Wolfgang-Pauli-Strasse 27, 8093 Z\"urich, Switzerland\\[1ex]
  \href{mailto:nbeisert@itp.phys.ethz.ch}
  {\texttt{nbeisert@itp.phys.ethz.ch}}}
\hypersetup{pdfauthor={Niklas Beisert}}
\hypersetup{pdfsubject={Manual for the LaTeX2e Package childdoc}}
\date{30 December 2018, \textsf{v2.0}}
\maketitle

\begin{abstract}\noindent
\textsf{childdoc} is a \LaTeXe{} package
that enables the direct compilation
of document sections included by |\include|
to individual files.
\end{abstract}

\begingroup
\parskip0ex
\tableofcontents
\endgroup

%%%%%%%%%%%%%%%%%%%%%%%%%%%%%%%%%%%%%%%%%%%%%%%%%%%%%%%%%%%%%%%%%%%%%%%%%%%%%%%%
%%%%%%%%%%%%%%%%%%%%%%%%%%%%%%%%%%%%%%%%%%%%%%%%%%%%%%%%%%%%%%%%%%%%%%%%%%%%%%%%
\section{Introduction}

\LaTeX{} provides a mechanism to structure a large document (such as a book)
into a main file and several child files (containing the chapters)
using the |\include| command.
This mechanism is beneficial for documents
which span hundreds of pages in order to
make the source file(s) more manageable.
Moreover, compilation can be restricted to
selected child files by means of the |\includeonly| command.
The latter feature can be used to reduce the compilation time while editing
(this was significantly more useful in the earlier days of \LaTeX{})
or to generate a smaller document which is easier to navigate.
Another application of |\includeonly| is to generate
documents consisting of selected parts of the complete document.

However, there are a few drawbacks of the plain |\include| mechanism:
\begin{itemize}
\item
The child files cannot be compiled on their own,
they can only be compiled via the main file.
A naive editing environment
(such as a text editor with an option
to have the current file processed by \LaTeX)
may require one to switch to the main file before compiling;
attempting to compile the child file produces errors.
\item
The main file must be modified (each time)
to adjust the |\includeonly| command
to the present needs. This easily leaves the main file in a messy state.
\item
The generated document will always carry the filename
of the main document. This is inconvenient if
several child files are to be compiled and
to be kept for distribution.
\end{itemize}

The present package provides a simple interface
to make child files individually compilable by \LaTeX{}.
Compiling a child file then has the same effect as compiling
the main file with an |\includeonly| command
to select the appropriate child.
Moreover the generated document will carry the name of the child
rather than the main file.
This resolves all three above issues.

This feature is meant to make the editing of books,
thesis documents and lecture notes somewhat more convenient.
However, the package can also be used efficiently for
composing a series of documents (such as exercise sheets)
which are typically distributed individually.
It then assists the author in generating the individual documents
(potentially in different versions)
as well as a document containing the collected series.
Another application is in developing style files
or other kinds of included material
where compilation of the style file could redirect
to a sample or test file.

%%%%%%%%%%%%%%%%%%%%%%%%%%%%%%%%%%%%%%%%%%%%%%%%%%%%%%%%%%%%%%%%%%%%%%%%%%%%%%%%
%%%%%%%%%%%%%%%%%%%%%%%%%%%%%%%%%%%%%%%%%%%%%%%%%%%%%%%%%%%%%%%%%%%%%%%%%%%%%%%%
\section{Usage}

First of all, the package \textsf{childdoc} is \emph{not} a standard
\LaTeXe{} |.sty| style file! Therefore it needs to be invoked in
a non-standard way.

%%%%%%%%%%%%%%%%%%%%%%%%%%%%%%%%%%%%%%%%%%%%%%%%%%%%%%%%%%%%%%%%%%%%%%%%%%%%%%%%
\subsection{Included Files}
\label{sec:include}

%%%%%%%%%%%%%%%%%%%%%%%%%%%%%%%%%%%%%%%%
\DescribeMacro{\childdocmain}
To use the package, add the commands
\begin{center}
\begin{tabular}{l}
|\input{childdoc.def}|\\
|\childdocmain{}|\\
\end{tabular}
\end{center}
at the very top of the main \LaTeX{} file,
in particular \emph{before} the |\documentclass| statement!
The argument of |\childdocmain| should be left empty
(but it must be present).

%%%%%%%%%%%%%%%%%%%%%%%%%%%%%%%%%%%%%%%%
\DescribeMacro{\childdocof}
Furthermore, add the commands
\begin{center}
\begin{tabular}{l}
|\input{childdoc.def}|\\
|\childdocof{|\textit{main}|}|\\
\end{tabular}
\end{center}
at the top of every child file \textit{child}
which is included by |\include{|\textit{child}|}|
from within the main file
(or at least for those files to be compiled individually).
The argument \textit{main} must be the filename of the main file.

There are a couple of
considerations in setting up the main and child documents:

%%%%%%%%%%%%%%%%%%%%%%%%%%%%%%%%%%%%%%%%
\paragraph{Restrictions.}

Please note the following restrictions:
\begin{itemize}
\item
|\childdocmain| must be called with one argument \textit{main}
to ensure compatibility with earlier version of the package.
It must either be empty (|\childdocmain{}|)
or precisely match the filename of the main file in which it is specified.
See \secref{sec:detection} for further information.
\item
The filename \textit{main} must be specified without the |.tex| extension.
\item
The filename \textit{main} is case sensitive
(even in case-insensitive file systems)
due to internal string comparison.
\item
The argument \textit{main} should be fully expanded, it cannot be a macro.
\item
Subdirectories and special characters should be avoided in filenames.
\item
The command |\childdocmain{|\textit{main}|}| must be followed by a whitespace.
It should not be followed immediately by another command
or by a comment mark `|%|'.
This is because the \TeX{} parser reads the token immediately following
the argument of |\childdocmain| and puts it
at the beginning of every child section;
however, a white\-space is ignored.
\end{itemize}

%%%%%%%%%%%%%%%%%%%%%%%%%%%%%%%%%%%%%%%%
\paragraph{Content of Main File.}

It is advisable to place all content in the child files included by |\include|.
Any output contained in the main file will appear in all child documents
unless suppressed manually;
it cannot be suppressed automatically by the |\includeonly| directive
and thus should normally be avoided.
A method to include some content in the main file
by means of conditional processing is described in \secref{sec:conditional}.

%%%%%%%%%%%%%%%%%%%%%%%%%%%%%%%%%%%%%%%%
\paragraph{Page Numbering.}

When only a part of the document is compiled,
the appropriate numbering of pages
(as well as other status parameters)
is determined from the |.aux| files.
The latter contain information from previous passes.
However this information needs to propagate through
all intermediate child documents.
Therefore the page numbering in child documents may well
be inconsistent until the complete document is compiled at least once.

A useful (if unconventional) way to always ensure a consistent
page numbering is to restart the numbering in each child document
and denote the pages by `\textit{child}|.|\textit{page}'
where \textit{child} represents the chapter/section number of the child file.
This can be achieved by the command
|\numberwithin{page}{|\textit{child}|}|
of the \textsf{amsmath} package
where \textit{child} can be |chapter| or |section|
depending on the chosen structuring.
Alternatively, one can modify the macro |\thepage| appropriately
and reset the counter |page| at the start of each child file.

%%%%%%%%%%%%%%%%%%%%%%%%%%%%%%%%%%%%%%%%%%%%%%%%%%%%%%%%%%%%%%%%%%%%%%%%%%%%%%%%
\subsection{Conditional Processing}
\label{sec:conditional}

The package provides a mechanism to compile different versions
of a document. To customise the versions further some conditional processing
can come in handy to distinguish which version is being compiled.
The package provides two macros to describe the compilation context:

%%%%%%%%%%%%%%%%%%%%%%%%%%%%%%%%%%%%%%%%
\DescribeMacro{\ifchilddoc}
The conditional |\ifchilddoc| distinguishes between the compilation of
child documents and the main document:
%
\begin{center}
|\ifchilddoc |\textit{child-code}| |[|\||else |\textit{main-code}]| \||fi|
\end{center}

%%%%%%%%%%%%%%%%%%%%%%%%%%%%%%%%%%%%%%%%
\DescribeMacro{\childdocname}
\DescribeMacro{\childdocjob}
The macro |\childdocname| contains the filename (without extension)
of the main or child file being processed.
Note that |\childdocjob| will always contain the name of the main file.

%%%%%%%%%%%%%%%%%%%%%%%%%%%%%%%%%%%%%%%%
\paragraph{Title Page.}

Conditional processing can be used to include a title or banner page
in the main document when proper precautions are taken.
Importantly, the code in the main file should ensure that the page counter
(as well as other status parameters which are stored in the |.aux| files)
takes the same value after the conditional processing.
Otherwise the page numbers may take divergent values
depending on which part is compiled.

For example, a title page could be declared by:
%
\begin{center}
\begin{tabular}{l}
|\ifchilddoc\||else|\\
|\addtocounter{page}{-1}|\\
\textit{code for title page}\\
|\newpage|\\
|\||fi|
\end{tabular}
\end{center}
%
A banner page for the child documents can be generated by:
%
\begin{center}
\begin{tabular}{l}
|\ifchilddoc|\\
|\addtocounter{page}{-1}|\\
\textit{code for banner page}\\
|\newpage|\\
|\||fi|
\end{tabular}
\end{center}
%
Here one could write a message such as:
\begin{center}
|This is the part \childdocname{} of \childdocjob{}.|
\end{center}

%%%%%%%%%%%%%%%%%%%%%%%%%%%%%%%%%%%%%%%%%%%%%%%%%%%%%%%%%%%%%%%%%%%%%%%%%%%%%%%%
\subsection{Flags}
\label{sec:flags}

The package makes it easy to generate different versions
of the main or child documents.
To this end compilation flags can be defined
and assigned different default values.
They will be particularly useful in conjunction
with the forwarding mechanism described in \secref{sec:forward}.

For example, it may be useful to have a flag |\version|
which can be set to |draft| or |final|.
The document source will contain some conditional code
depending on the value of |\version|.
Suppose further, the flag should default to |final| for the main file
and to |draft| for child files
which is a natural assignment for editing the document.
This is achieved by placing the following code
in the preamble of the main document
(below the |\childdocmain| directive):
%
\begin{center}
\begin{tabular}{l}
|\ifchilddoc|\\
|\providecommand{\version}{draft}|\\
|\||else|\\
|\providecommand{\version}{final}|\\
|\||fi|
\end{tabular}
\end{center}
%
The definition by |\providecommand| makes sure
that previous definitions are not overwritten.
Further statements |\providecommand{\version}{...}|
can thus be added before the above code to override it.

For the main file, one might add a line
(between |\childdocmain| and the above block)
%
\begin{center}
|%\ifchilddoc\||else\providecommand{\version}{draft}\||fi|
\end{center}
%
which can be uncommented to produce a draft version.
Likewise one can add a line to the very top of a child file
(above the |\childdocof{|\textit{main}|}| directive)
%
\begin{center}
|%\providecommand{\version}{final}|
\end{center}
%
which can be uncommented to produce the final version of this child document.

%%%%%%%%%%%%%%%%%%%%%%%%%%%%%%%%%%%%%%%%%%%%%%%%%%%%%%%%%%%%%%%%%%%%%%%%%%%%%%%%
\subsection{Forwarding}
\label{sec:forward}

Different versions of the main or child documents
using compilation flags as described in \secref{sec:flags}
can be (permanently) stored in different files
for convenient compilation, viewing and distribution.
To this end, the package defines a command
to pass on compilation to a different file:

%%%%%%%%%%%%%%%%%%%%%%%%%%%%%%%%%%%%%%%%
\DescribeMacro{\childdocforward}
The command |\childdocforward| redirects processing to
another source file:
%
\begin{center}
\begin{tabular}{l}
|\input{childdoc.def}|\\
|\childdocforward[|\textit{main}|]{|\textit{dest}|}|\\
\end{tabular}
\end{center}
%
The argument \textit{dest} is the destination file
(without extension).
It should be the main file or one of the child files.
Note that further \textsf{childdoc} directives
such as |\childdocof| and |\childdocforward|
in the indicated file will be processed in this form.
The optional argument \textit{main}
passes on directly to the main file \textit{main}
while pretending to compile the child \textit{dest}.
This form behaves as if \textit{dest}
issues |\childdocof{|\textit{main}|}| right away,
and no further \textsf{childdoc} directives will be processed.

%%%%%%%%%%%%%%%%%%%%%%%%%%%%%%%%%%%%%%%%
\DescribeMacro{\...prefix}
In the alternative form |\childdocforwardprefix|,
%
\begin{center}
\begin{tabular}{l}
|\input{childdoc.def}|\\
|\childdocforwardprefix[|\textit{main}|]{|\textit{prefix}|}{|\textit{dest}|}|
\end{tabular}
\end{center}
%
the destination file is determined by a pattern
depending on the current file:
To make this work, the current file must be called
`{\textit{prefix}\hspace{0.2em}\textit{suffix}}'
with \textit{prefix} matching precisely the argument.
Processing is then passed on to the file
`{\textit{dest}\hspace{0.2em}\textit{suffix}}'.
Surely, the same effect is achieved by
directly specifying the
argument `{\textit{dest}\hspace{0.2em}\textit{suffix}}'
in the first form.
However, that requires to set up a different file
for each child. With the alternative form of the command
all these files can have exactly the same content
which simplifies setting them up and maintaining them.

For example, the following file |draft.tex|
with a compilation flag |\version| as described in \secref{sec:flags}
compiles the main document as a draft:
%
\begin{center}
\begin{tabular}{l}
|\def\version{draft}|\\
|\input{childdoc.def}|\\
|\childdocforward{|\textit{main}|}|
\end{tabular}
\end{center}
%
Likewise, the following files |final|\textit{nn}|.tex|
compile the final version of the child document
|child|\textit{nn}|.tex|:
%
\begin{center}
\begin{tabular}{l}
|\def\version{final}|\\
|\input{childdoc.def}|\\
|\childdocforwardprefix{final}{child}|
\end{tabular}
\end{center}
%

Note that when several versions of a main file and/or of each child file
are to be generated, it may be convenient to set up a |Makefile| or
shell script to automatise the process.

%%%%%%%%%%%%%%%%%%%%%%%%%%%%%%%%%%%%%%%%%%%%%%%%%%%%%%%%%%%%%%%%%%%%%%%%%%%%%%%%
\subsection{Command Line Processing}
\label{sec:commandline}

The effect of redirection files can also be achieved by invoking
the \LaTeX{} compiler with a more elaborate command line.
Most conveniently this should be done as part
of a shell script or a |Makefile|.

When using \textsf{childdoc} in the main file, the following
command lines effectively perform a redirection
(note that depending on the shell being used,
backslashes may have to be doubled: `|\|' $\to$ `|\\|'):
%
\begin{center}
|... -jobname "|\textit{target}|" |\\|"|[\textit{flags}]%
|\input{childdoc.def}\childdocforward[|\textit{main}|]{|\textit{dest}|}"|
\end{center}
%
Here \textit{target} is the name of the output file,
\textit{main} is the name of the main file
and \textit{dest} is the name of the main or child file to be processed
(all filenames without extensions).
The optional argument \textit{main} can be omitted
if \textit{main} matches \textit{dest}.
Optionally, compilation \textit{flags} can be defined via |\def| commands.
This command line makes the \TeX{} engine believe
it is compiling the file \textit{target}
whose content is specified as the latter parameter.
The provided code then forwards the processing to
\textit{main} or \textit{dest} as described in \secref{sec:forward}.

%%%%%%%%%%%%%%%%%%%%%%%%%%%%%%%%%%%%%%%%%%%%%%%%%%%%%%%%%%%%%%%%%%%%%%%%%%%%%%%%
\subsection{Include by Input}
\label{sec:input}

Including child documents by |\include| has some restrictions by design.
Most notably, the content of a child document always occupies
its own set of pages; pages cannot be shared between child documents.
Usually, this behaviour makes perfect sense
because each child document contain an essential part of the document.
However, in some situations it may be desirable to compose
a document from a collection of parts
without having mandatory page breaks between then.
For this case, the package
provides a mechanism to include parts
by |\input| which can also be processed individually.
However, by construction this mechanism
requires manual handling of the content to be output.

%%%%%%%%%%%%%%%%%%%%%%%%%%%%%%%%%%%%%%%%
\DescribeMacro{\ifchilddocmanual}
The main file should be prepared as usual, see \secref{sec:include}.
However, the document body must make a distinction
between processing of an individual part and of the main document, e.g.:
%
\begin{center}
\begin{tabular}{l}
|\ifchilddocmanual|\\
|\input{\childdocname}|\\
|\||else|\\
\textit{document body with }|\input{|\textit{part}|}|\\
|\||fi|
\end{tabular}
\end{center}
%
The conditional |\ifchilddocmanual| is true whenever
a part to be included by |\input| is being compiled,
and the name of the part is stored in |\childdocname|.

%%%%%%%%%%%%%%%%%%%%%%%%%%%%%%%%%%%%%%%%
\DescribeMacro{\childdocby}
Each part to be included by |\input| should start with:
%
\begin{center}
\begin{tabular}{l}
|\input{childdoc.def}|\\
|\childdocby{|\textit{main}|}|\\
\end{tabular}
\end{center}
%
The directive |\childdocby| is similar to |\childdocof|
described in \secref{sec:include},
but the subsequent selection of content must be done manually.
To that end, both |\ifchilddoc| and |\ifchilddocmanual|
will be true upon processing of a part,
and the name of the part is stored in |\childdocname|.
Note that |\jobname| will be set to the filename of the current part
so that each part receives an individual |.aux| file
that does not interfere with the |.aux| file(s) of the main document.
This behaviour can be altered by the alternative form
|\childdocby[*]{|\textit{main}|}| (with a non-empty optional argument)
which uses the |.aux| file of the main document
by setting |\jobname| to \textit{main}.

%%%%%%%%%%%%%%%%%%%%%%%%%%%%%%%%%%%%%%%%%%%%%%%%%%%%%%%%%%%%%%%%%%%%%%%%%%%%%%%%
\subsection{Driver Development}
\label{sec:driver}

The \textsf{childdoc} mechanism can also be use for the development
of definition files such as \LaTeX{} styles or classes.
This case differs from the above setup with multiple parts
included by |\include| in that no |\includeonly| should be invoked.
This can be achieved by starting the include file
(before |\ProvidesPackage|) with:
%
\begin{center}
\begin{tabular}{l}
|\input{childdoc.def}|\\
|\childdocforward{|\textit{main}|}|\\
\end{tabular}
\end{center}
%
or alternatively with:
%
\begin{center}
\begin{tabular}{l}
|\input{childdoc.def}|\\
|\childdocby{|\textit{main}|}|\\
\end{tabular}
\end{center}
%
Both forms have slightly different effects as described above.
The main file is prepared as usual, see \secref{sec:include}.

%%%%%%%%%%%%%%%%%%%%%%%%%%%%%%%%%%%%%%%%%%%%%%%%%%%%%%%%%%%%%%%%%%%%%%%%%%%%%%%%
\subsection{Legacy Detection}
\label{sec:detection}

The directive |\childdocmain| in the main file can detect
whether the complete document or merely a child is to be compiled
even without using the directive |\childdocof|.
This method is deprecated because it is less robust
and there is no compelling reason to use it;
it is merely provided for backward compatibility
and it may be removed in future versions.

If the detection mechanism is to be used,
it is mandatory to correctly specify
the filename of the main file as the argument of |\childdocmain|:
%
\begin{center}
\begin{tabular}{l}
|\input{childdoc.def}|\\
|\childdocmain{|\textit{main}|}|\\
\end{tabular}
\end{center}
%
If |\jobname| does not match the argument \textit{main} of |\childdocmain|,
it is assumed that |\jobname| points to the child file to be compiled.
When using |\childdocmain| with the main file specified as argument,
it suffices to start a child file
with just |\input{|\textit{main}|}|
without loading of the package and using |\childdocof|.
If instead all processing is done
with the appropriate \textsf{childdoc} directives,
the argument of \textit{main} of |\childdocmain| can be empty.

An alternative version of the command line processing described
in \secref{sec:commandline} using the detection mechanism reads:
%
\begin{center}
|... -jobname "|\textit{target}|" "|[\textit{flags}]%
[|\def\jobname{|\textit{dest}|}|]|\input{|\textit{main}|}"|
\end{center}

%%%%%%%%%%%%%%%%%%%%%%%%%%%%%%%%%%%%%%%%%%%%%%%%%%%%%%%%%%%%%%%%%%%%%%%%%%%%%%%%
\subsection{Manual Code}
\label{sec:manual}

In case one cannot be certain whether the definitions file |childdoc.def|
is installed on the target \TeX{} distribution
and one prefers not to ship it,
it is conceivable to paste a few relevant commands into the sources.

To that end, drop all statements |\input{childdoc.def}|
and perform the replacements as outlined below.
Instead of |\childdocmain{|\textit{main}|}| add the following code
to the top of the main file:
%
\begin{center}
\begin{tabular}{l}
|\||ifdefined\childdocname\endinput\||fi\newif\ifchilddoc|\\
|\edef\childdocname{\scantokens\expandafter{\jobname\noexpand}}|\\
|\def\childdocmain{|\textit{main}|}\||ifx\childdocmain\childdocname\||else|\\
|\childdoctrue\includeonly{\childdocname}\let\jobname\childdocmain\||fi|\\
\end{tabular}
\end{center}
%
Instead of |\childdocof{|\textit{main}|}| just include the main file
at the top of each child file:
%
\begin{center}
|\input{|\textit{main}|}|
\end{center}
%
A simple redirection |\childdocforward{|\textit{dest}|}| is achieved by:
%
\begin{center}
|\def\jobname{|\textit{dest}|}\input{\jobname}|
\end{center}
%
The redirection with prefix
|\childdocforwardprefix[|\textit{prefix}|]{|\textit{dest}|}|
is accomplished by:
%
\begin{center}
\begin{tabular}{l}
|{\edef\jobname{\scantokens\expandafter{\jobname\noexpand}}|\\
|\def\redirectjob |\textit{prefix}|#1~~~{\gdef\jobname{|\textit{dest}|#1}}|\\
|\expandafter\redirectjob\jobname~~~}\input{\jobname}|
\end{tabular}
\end{center}

In an alternative approach,
child documents can be compiled by a specific command line
without additional code or specific definitions:
%
\begin{center}
|... -jobname "|\textit{target}|" "|[\textit{flags}]%
|\includeonly{|\textit{dest}|}\input{|\textit{main}|}"|
\end{center}
%

%%%%%%%%%%%%%%%%%%%%%%%%%%%%%%%%%%%%%%%%%%%%%%%%%%%%%%%%%%%%%%%%%%%%%%%%%%%%%%%%
%%%%%%%%%%%%%%%%%%%%%%%%%%%%%%%%%%%%%%%%%%%%%%%%%%%%%%%%%%%%%%%%%%%%%%%%%%%%%%%%
\section{Information}

%%%%%%%%%%%%%%%%%%%%%%%%%%%%%%%%%%%%%%%%%%%%%%%%%%%%%%%%%%%%%%%%%%%%%%%%%%%%%%%%
\subsection{Copyright}

Copyright \copyright{} 2017--2018 Niklas Beisert

This work may be distributed and/or modified under the
conditions of the \LaTeX{} Project Public License, either version 1.3
of this license or (at your option) any later version.
The latest version of this license is in
  \url{http://www.latex-project.org/lppl.txt}
and version 1.3 or later is part of all distributions of \LaTeX{}
version 2005/12/01 or later.

This work has the LPPL maintenance status `maintained'.

The Current Maintainer of this work is Niklas Beisert.

This work consists of the files |README.txt|, |childdoc.ins| and |childdoc.dtx|
as well as the derived files |childdoc.def|, |cdocsamp.tex|
with |cdocsch1.tex|, |cdocsch2.tex|, |cdocspt3.tex|, |cdocspt4.tex|,
|cdocsdrf.tex|, |cdocsfn1.tex|, |cdocsfn2.tex|
as well as |childdoc.pdf|.

%%%%%%%%%%%%%%%%%%%%%%%%%%%%%%%%%%%%%%%%%%%%%%%%%%%%%%%%%%%%%%%%%%%%%%%%%%%%%%%%
\subsection{Files and Installation}

The package consists of the files:
%
\begin{center}
\begin{tabular}{ll}
    |README.txt|   & readme file \\
    |childdoc.ins| & installation file \\
    |childdoc.dtx| & source file \\
    |childdoc.def| & definition file \\
    |cdocsamp.tex| & sample main file \\
    |cdocsch1.tex| & sample include file \\
    |cdocsch2.tex| & sample include file \\
    |cdocspt3.tex| & sample part file \\
    |cdocspt4.tex| & sample part file \\
    |cdocsdrf.tex| & sample redirection file \\
    |cdocsfn1.tex| & sample redirection file \\
    |cdocsfn2.tex| & sample redirection file \\
    |childdoc.pdf| & manual
\end{tabular}
\end{center}
%
The distribution consists of the files
|README.txt|, |childdoc.ins| and |childdoc.dtx|.
%
\begin{itemize}
\item
Run (pdf)\LaTeX{} on |childdoc.dtx|
to compile the manual |childdoc.pdf| (this file).
\item
Run \LaTeX{} on |childdoc.ins| to create the definitions file |childdoc.def|
and the sample |cdocsamp.tex| with include files
|cdocsch1.tex|, |cdocsch2.tex|, |cdocspt3.tex|, |cdocspt4.tex|,
|cdocsdrf.tex|, |cdocsfn1.tex|, |cdocsfn2.tex|.
Then copy the file |childdoc.def| to an appropriate directory of your \LaTeX{}
distribution, e.g.\ \textit{texmf-root}|/tex/latex/childdoc|.
\end{itemize}

%%%%%%%%%%%%%%%%%%%%%%%%%%%%%%%%%%%%%%%%%%%%%%%%%%%%%%%%%%%%%%%%%%%%%%%%%%%%%%%%
\subsection{Related CTAN Packages}

There are several other packages which offer a similar functionality:
%
\begin{itemize}
\item
The packages
\href{http://ctan.org/pkg/docmute}{\textsf{docmute}},
\href{http://ctan.org/pkg/includex}{\textsf{includex}} and
\href{http://ctan.org/pkg/standalone}{\textsf{standalone}}
provide commands to include only the document body of
a child file thus allowing both files to be compiled individually.
\item
The packages \href{http://ctan.org/pkg/subdocs}{\textsf{subdocs}}
and \href{http://ctan.org/pkg/subfiles}{\textsf{subfiles}}
provide structures in which the main and child documents can be
encapsulated and allowing them to be compiled individually.
The inclusion mechanism is different from the conventional |\include|.
\item
The package \href{http://ctan.org/pkg/combine}{\textsf{combine}}
is an elaborate solution to combine several documents into one.
\end{itemize}
%
See also the CTAN topic \href{http://ctan.org/topic/subdocs}{\textsf{subdocs}}
for further related packages.
The present package differs from the above solutions in that
a document structure constructed with the conventional |\include| mechanism
just needs two extra commands at the top of every file
such that all constituent files can be compiled individually.

%%%%%%%%%%%%%%%%%%%%%%%%%%%%%%%%%%%%%%%%%%%%%%%%%%%%%%%%%%%%%%%%%%%%%%%%%%%%%%%%
%\subsection{Feature Suggestions}
%
%The following is a list of features which may be useful for future
%versions of this package:
%%
%\begin{itemize}
%\item
%\ldots
%\end{itemize}

%%%%%%%%%%%%%%%%%%%%%%%%%%%%%%%%%%%%%%%%%%%%%%%%%%%%%%%%%%%%%%%%%%%%%%%%%%%%%%%%
\subsection{Revision History}

%%%%%%%%%%%%%%%%%%%%%%%%%%%%%%%%%%%%%%%%
\paragraph{v2.0:} 2018/12/30

\begin{itemize}
\item
immediate forward processing
\item
added |\childdocby| mechanism
\item
manual restructured
\end{itemize}

%%%%%%%%%%%%%%%%%%%%%%%%%%%%%%%%%%%%%%%%
\paragraph{v1.6:} 2018/01/17

\begin{itemize}
\item
application for development of include files
\item
corrections to manual
\end{itemize}

%%%%%%%%%%%%%%%%%%%%%%%%%%%%%%%%%%%%%%%%
\paragraph{v1.5:} 2017/05/21

\begin{itemize}
\item
more complete structuring introduced
\item
|\childdocof| introduced
\item
|\childdoc| renamed to |\childdocmain|
\item
|\childredirect| renamed to |\childdocforward| and |\childdocforwardprefix|
and functionality expanded
\end{itemize}

%%%%%%%%%%%%%%%%%%%%%%%%%%%%%%%%%%%%%%%%
\paragraph{v1.0:} 2017/04/27

\begin{itemize}
\item
manual and install package
\item
first version published on CTAN
\end{itemize}

%%%%%%%%%%%%%%%%%%%%%%%%%%%%%%%%%%%%%%%%
\paragraph{v0.6:} 2017/04/26

\begin{itemize}
\item
redirection mechanism added
\end{itemize}

%%%%%%%%%%%%%%%%%%%%%%%%%%%%%%%%%%%%%%%%
\paragraph{v0.5:} 2017/04/26

\begin{itemize}
\item
functionality in definition file
\end{itemize}


%%%%%%%%%%%%%%%%%%%%%%%%%%%%%%%%%%%%%%%%%%%%%%%%%%%%%%%%%%%%%%%%%%%%%%%%%%%%%%%%
%%%%%%%%%%%%%%%%%%%%%%%%%%%%%%%%%%%%%%%%%%%%%%%%%%%%%%%%%%%%%%%%%%%%%%%%%%%%%%%%
%%%%%%%%%%%%%%%%%%%%%%%%%%%%%%%%%%%%%%%%%%%%%%%%%%%%%%%%%%%%%%%%%%%%%%%%%%%%%%%%
\appendix

\settowidth\MacroIndent{\rmfamily\scriptsize 000\ }

 \DocInput{childdoc.dtx}

\end{document}
%</driver>
% \fi
%
% %%%%%%%%%%%%%%%%%%%%%%%%%%%%%%%%%%%%%%%%%%%%%%%%%%%%%%%%%%%%%%%%%%%%%%%%%%%%%%
% %%%%%%%%%%%%%%%%%%%%%%%%%%%%%%%%%%%%%%%%%%%%%%%%%%%%%%%%%%%%%%%%%%%%%%%%%%%%%%
% \section{Sample}
%\iffalse
%<*samplemain>
%\fi
%
% The following presents a sample document
% with two chapters, two parts, a title page,
% a compile flag as well as three forwarding files to set the flag.
% It consists of eight |.tex| files:
% \begin{center}
% \begin{tabular}{ll}
% |cdocsamp.tex|&main file\\
% |cdocsch1.tex|&include file for chapter 1\\
% |cdocsch2.tex|&include file for chapter 2\\
% |cdocspt3.tex|&include file for part 3\\
% |cdocspt4.tex|&include file for part 4\\
% |cdocsdrf.tex|&forwarding file for main file in draft mode\\
% |cdocsfi1.tex|&forwarding file for final version of chapter 1\\
% |cdocsfi2.tex|&forwarding file for final version of chapter 2\\
% \end{tabular}
% \end{center}
% Each of the eight files can be compiled directly by the \LaTeX{} compiler.
%
% %%%%%%%%%%%%%%%%%%%%%%%%%%%%%%%%%%%%%%
% \paragraph{Main File.}
%
% The main file is called |cdocsamp.tex|.
%
% Load the \textsf{childdoc} definitions and
% declare the filename for the main document:
%    \begin{macrocode}
\input{childdoc.def}
\childdocmain{}
%    \end{macrocode}

% Optional override for |\version| flag:
%    \begin{macrocode}
%%\ifchilddoc\else\providecommand{\version}{draft}\fi
%    \end{macrocode}

% Define the default values for the |\version| flag
% (|final| for the main file and |draft| for childs):
%    \begin{macrocode}
\ifchilddoc
\providecommand{\version}{draft}
\else
\providecommand{\version}{final}
\fi
%    \end{macrocode}

% Load the standard document class:
%    \begin{macrocode}
\documentclass[12pt]{article}
%    \end{macrocode}

% Start the document body:
%    \begin{macrocode}
\begin{document}
%    \end{macrocode}

% Declare a title page.
% Print title, part of document being processed and version flag:
%    \begin{macrocode}
\addtocounter{page}{-1}
\begin{center}
{\LARGE\bfseries{}childdoc example\par}
\vspace{1cm}
\ifchilddoc
\ifchilddocmanual part\else chapter\fi:
`\childdocname' of `\childdocjob'\par
\else
main document: `\childdocjob'\par
\fi
version: \version\par
\end{center}
\newpage
%    \end{macrocode}

% Manually include selected file,
% otherwise process as usual:
%    \begin{macrocode}
\ifchilddocmanual
\section*{part `\childdocname'}
\input{\childdocname}
\else
%    \end{macrocode}

% Include the two chapters:
%    \begin{macrocode}
\include{cdocsch1}
\include{cdocsch2}
%    \end{macrocode}

% Include the two parts unless only chapters should be displayed:
%    \begin{macrocode}
\ifchilddoc\else
\section{part three}
\input{cdocspt3}
\section{part four}
\input{cdocspt4}
\fi
%    \end{macrocode}

% Process as usual until here:
%    \begin{macrocode}
\fi
%    \end{macrocode}

% End of document body:
%    \begin{macrocode}
\end{document}
%    \end{macrocode}
%\iffalse
%</samplemain>
%\fi
%
% %%%%%%%%%%%%%%%%%%%%%%%%%%%%%%%%%%%%%%
% \paragraph{Chapter Include Files.}
%
% The include files are called |cdocsch1.tex| and |cdocsch2.tex|.
%
%\iffalse
%<*samplechap1|samplechap2>
%\fi

% Optional override for |\version| flag:
%    \begin{macrocode}
%%\providecommand{\version}{final}
%    \end{macrocode}

% Include the main document:
%    \begin{macrocode}
\input{childdoc.def}
\childdocof{cdocsamp}
%    \end{macrocode}

%\iffalse
%</samplechap1|samplechap2>
%\fi
%
%\iffalse
%<*samplechap1>
%\fi
% Some text for chapter 1:
%    \begin{macrocode}
\section{one}
some text in chapter one
%    \end{macrocode}

%\iffalse
%</samplechap1>
%\fi
% Some text for chapter 2:
%\iffalse
%<*samplechap2>
%\fi
%    \begin{macrocode}
\section{two}
more text in chapter two
%    \end{macrocode}

%\iffalse
%</samplechap2>
%\fi
%
% %%%%%%%%%%%%%%%%%%%%%%%%%%%%%%%%%%%%%%
% \paragraph{Part Include Files.}
%
% The include files are called |cdocspt3.tex| and |cdocspt4.tex|.
%
%\iffalse
%<*samplepart3|samplepart4>
%\fi

% Optional override for |\version| flag:
%    \begin{macrocode}
%%\providecommand{\version}{final}
%    \end{macrocode}

% Include the main document:
%    \begin{macrocode}
\input{childdoc.def}
\childdocby{cdocsamp}
%    \end{macrocode}

%\iffalse
%</samplepart3|samplepart4>
%\fi
%
%\iffalse
%<*samplepart3>
%\fi
% Some text for part 3:
%    \begin{macrocode}
some text in part three
%    \end{macrocode}

%\iffalse
%</samplepart3>
%\fi
% Some text for part 4:
%\iffalse
%<*samplepart4>
%\fi
%    \begin{macrocode}
more text in part four
%    \end{macrocode}

%\iffalse
%</samplepart4>
%\fi
%
% %%%%%%%%%%%%%%%%%%%%%%%%%%%%%%%%%%%%%%
% \paragraph{Forwarding for a Complete Draft.}
%
% The following forwarding file |cdocsdrf.tex|
% compiles the main document in draft mode:
%\iffalse
%<*sampledraft>
%\fi
%    \begin{macrocode}
\def\version{draft}
\input{childdoc.def}
\childdocforward{cdocsamp}
%    \end{macrocode}

%\iffalse
%</sampledraft>
%\fi
%
% %%%%%%%%%%%%%%%%%%%%%%%%%%%%%%%%%%%%%%
% \paragraph{Forwarding for Final Version of the Chapters.}
%
% The following forwarding files |cdocsfn1.tex| and |cdocsfn2.tex|
% (with identical content)
% compile the final versions of the child documents
% |cdocsch1.tex| and |cdocsch2.tex|, respectively:
%\iffalse
%<*samplefinal>
%\fi
%    \begin{macrocode}
\def\version{final}
\input{childdoc.def}
\childdocforwardprefix[cdocsamp]{cdocsfn}{cdocsch}
%    \end{macrocode}

%\iffalse
%</samplefinal>
%\fi
%
% %%%%%%%%%%%%%%%%%%%%%%%%%%%%%%%%%%%%%%
% \paragraph{Command Line Processing.}
%
% The following three command lines generate the output files
% |cdocscld|, |cdocscl1| and |cdocscl2|
% which should be identical to
% |cdocsdrf|, |cdocsch1| and |cdocsfn2|, respectively:
% \begin{center}
% \begin{tabular}{l}
% |latex -jobname cdocscld \|\\
% |  "\def\version{draft}\input{childdoc.def}\childdocforward{cdocsamp}"|\\
% |latex -jobname cdocscl1 \|\\
% |  "\input{childdoc.def}\childdocforward[cdocsamp]{cdocsch1}"|\\
% |latex -jobname cdocscl2 \|\\
% |  "\def\version{final}\input{childdoc.def}\childdocforward{cdocsch2}"|
% \end{tabular}
% \end{center}
% Note that the trailing backslash on each first line
% merely continues the input to the second line
% (for convenient cut ant paste).
% Furthermore, the command |latex| can be replaced by any
% of its alternative versions such as |pdflatex|.
%
% %%%%%%%%%%%%%%%%%%%%%%%%%%%%%%%%%%%%%%%%%%%%%%%%%%%%%%%%%%%%%%%%%%%%%%%%%%%%%%
% %%%%%%%%%%%%%%%%%%%%%%%%%%%%%%%%%%%%%%%%%%%%%%%%%%%%%%%%%%%%%%%%%%%%%%%%%%%%%%
% \section{Implementation}
%\iffalse
%<*package>
%\fi
%
% This section describes the definitions file |childdoc.def|.

% The definitions cannot be loaded using |\usepackage| or |\RequirePackage|
% which has a mechanism to prevent loading a style file more than once.
% When loading the definitions by means of |\input|
% multiple instances have to be prevented manually:
%\iffalse
%This code needs to be before the `\ProvidesFile' directive
%which is defined at the beginning of this file.
%Therefore it is also placed there and commented out here.
%</package>
%<*discard>
%\fi
%    \begin{macrocode}
\ifdefined\childdocmain\endinput\fi
%    \end{macrocode}
%\iffalse
%</discard>
%<*package>
%\fi
%
% \macro{\ifchilddoc}
% \macro{\ifchilddocmanual}
% The conditional |\ifchilddoc| tells whether a
% child (true) or main (false) document is being compiled.
% The conditional |\ifchilddocmanual| tells whether
% the |\includeonly| mechanism is used (false) or
% the selection of child files must be performed manually (true).
% The definitions initialise to false:
%    \begin{macrocode}
\newif\ifchilddoc
\newif\ifchilddocmanual
%    \end{macrocode}

% \macro{\childdocname}
% \macro{\childdocjob}
% The macro |\childdocname| stores the name of the main document
% to be compiled. The macro |\childdocjob| stores the name of
% the document on which the \LaTeX{} compiler was originally invoked.
% The content of |\jobname| cannot be compared
% to filenames specified in the source due to different catcodes.
% The following code rescans |\jobname|, stores the result
% in |\childdocname| and saves a copy in |\childdocjob|:
%    \begin{macrocode}
\edef\childdocname{\scantokens\expandafter{\jobname\noexpand}}
\let\childdocjob\childdocname
%    \end{macrocode}

% \macro{\childdocdisable}
% The macro |\childdocdisable| prevents the main file
% from being processed more than once.
% At this stage, the main document command |\childdocmain|
% is assumed to be called once again where it should do nothing.
% Any subsequent call to it should prevent
% a secondary processing of the main document
% It overwrites the forwarding commands
% |\childdocof| and |\childdocforward|
% with empty macros to prevent further inclusions of the main document:
%    \begin{macrocode}
\newcommand{\childdocdisable}
{
  \renewcommand{\childdocmain}[1]{\renewcommand{\childdocmain}[1]{\endinput}}
  \renewcommand{\childdocof}[1]{}
  \renewcommand{\childdocby}[2][]{}
  \renewcommand{\childdocforward}[2][]{}
  \renewcommand{\childdocdisable}{}
}
%    \end{macrocode}

% \macro{\childdocmain}
% The macro |\childdocmain| is to be called at the top of the main file
% with nothing or the main filename (without extension) as argument.
% First, it breaks loops.
% If the argument is not empty and does not match |\childdocname|
% (which is set by the first inclusion of |childdoc.def|),
% |\ifchilddoc| is set to true, |\includeonly| is applied to the child file
% and |\jobname| is set to the main file
% (for proper handling of |.aux| files):
%    \begin{macrocode}
\newcommand{\childdocmain}[1]
{
  \childdocdisable\childdocmain{}
  \if?#1?\else
    \begingroup
      \def\childdoctmp{#1}
      \ifx\childdoctmp\childdocname
        \def\childdoctmp{}
      \else
        \def\childdoctmp
        {
          \childdoctrue
          \includeonly{\childdocname}
          \def\childdocjob{#1}
          \def\jobname{#1}
        }
      \fi
      \expandafter
    \endgroup
    \childdoctmp
  \fi
}
%    \end{macrocode}

% \macro{\childdocof}
% The command |\childdocof| redirects
% compilation to the main file |#1|.
%    \begin{macrocode}
\newcommand{\childdocof}[1]
{
  \childdocdisable
  \childdoctrue
  \includeonly{\childdocname}
  \def\jobname{#1}
  \def\childdocjob{#1}
  \input{#1}
}
%    \end{macrocode}

% \macro{\childdocby}
% The command |\childdocby| ....
%    \begin{macrocode}
\newcommand{\childdocby}[2][]
{
  \childdocdisable
  \childdoctrue
  \childdocmanualtrue
  \if?#1?\else
    \def\jobname{#2}
  \fi
  \def\childdocjob{#2}
  \input{#2}
  \endinput
}
%    \end{macrocode}

% \macro{\childdocforward}
% The command |\childdocforward| redirects
% compilation to the main file or
% (if the optional argument is given) a child file.
% Parameters are set as if the main file
% or a child file starting with |\childdocof| was compiled.
% Then compilation is handed over to the main file:
%    \begin{macrocode}
\newcommand{\childdocforward}[2][]
{
  \begingroup
    \if?#1?
      \def\childdoctmp
      {
        \def\childdocname{#2}
        \def\childdocjob{#2}
        \def\jobname{#2}
        \input{#2}
        \endinput
      }
    \else
      \def\childdoctmp
      {
        \childdocdisable
        \def\childdocname{#2}
        \childdoctrue
        \includeonly{#2}
        \def\childdocjob{#1}
        \def\jobname{#1}
        \input{#1}
        \endinput
      }
    \fi
    \expandafter
  \endgroup
  \childdoctmp
}
%    \end{macrocode}

% \macro{\childdocforwardprefix}
% The command |\childdocforwardprefix| redirects
% compilation to the main or a child file by means of a pattern.
% The prefix |#1| in the current filename is replaced by |#2|
% and the suffix of the current filename is kept
% (it is assumed that the filename does not contain the substring `|~~~|'
% which is used as a delimiter).
% Compilation is handed over to the new file by |\childdocforward|:
%    \begin{macrocode}
\newcommand{\childdocforwardprefix}[3][]
{
  \begingroup
    \def\childdocextract #2##1~~~{\def\childdoctmp{\childdocforward[#1]{#3##1}}}
    \expandafter\childdocextract\childdocname~~~
    \expandafter
  \endgroup
  \childdoctmp
}
%    \end{macrocode}

% \macro{\childdoc}
% The deprecated macro |\childdoc| is a legacy version of |\childdocmain|:
%    \begin{macrocode}
\newcommand{\childdoc}{\childdocmain}
%    \end{macrocode}

% \macro{\childdocredirect}
% The deprecated macro |\childdocredirect| is a legacy version
% of |\childdocforward| and |\childdocforwardprefix|:
%    \begin{macrocode}
\newcommand{\childdocredirect}[2][]
{
  \begingroup
    \if?#1?
      \def\childdoctmp{\childdocforward{#2}}
    \else
      \def\childdoctmp{\childdocforwardprefix{#1}{#2}}
    \fi
    \expandafter
  \endgroup
  \childdoctmp
}
%    \end{macrocode}

%\iffalse
%</package>
%\fi
%
\endinput
|\\
|\childdocof{|\textit{main}|}|\\
\end{tabular}
\end{center}
at the top of every child file \textit{child}
which is included by |\include{|\textit{child}|}|
from within the main file
(or at least for those files to be compiled individually).
The argument \textit{main} must be the filename of the main file.

There are a couple of
considerations in setting up the main and child documents:

%%%%%%%%%%%%%%%%%%%%%%%%%%%%%%%%%%%%%%%%
\paragraph{Restrictions.}

Please note the following restrictions:
\begin{itemize}
\item
|\childdocmain| must be called with one argument \textit{main}
to ensure compatibility with earlier version of the package.
It must either be empty (|\childdocmain{}|)
or precisely match the filename of the main file in which it is specified.
See \secref{sec:detection} for further information.
\item
The filename \textit{main} must be specified without the |.tex| extension.
\item
The filename \textit{main} is case sensitive
(even in case-insensitive file systems)
due to internal string comparison.
\item
The argument \textit{main} should be fully expanded, it cannot be a macro.
\item
Subdirectories and special characters should be avoided in filenames.
\item
The command |\childdocmain{|\textit{main}|}| must be followed by a whitespace.
It should not be followed immediately by another command
or by a comment mark `|%|'.
This is because the \TeX{} parser reads the token immediately following
the argument of |\childdocmain| and puts it
at the beginning of every child section;
however, a white\-space is ignored.
\end{itemize}

%%%%%%%%%%%%%%%%%%%%%%%%%%%%%%%%%%%%%%%%
\paragraph{Content of Main File.}

It is advisable to place all content in the child files included by |\include|.
Any output contained in the main file will appear in all child documents
unless suppressed manually;
it cannot be suppressed automatically by the |\includeonly| directive
and thus should normally be avoided.
A method to include some content in the main file
by means of conditional processing is described in \secref{sec:conditional}.

%%%%%%%%%%%%%%%%%%%%%%%%%%%%%%%%%%%%%%%%
\paragraph{Page Numbering.}

When only a part of the document is compiled,
the appropriate numbering of pages
(as well as other status parameters)
is determined from the |.aux| files.
The latter contain information from previous passes.
However this information needs to propagate through
all intermediate child documents.
Therefore the page numbering in child documents may well
be inconsistent until the complete document is compiled at least once.

A useful (if unconventional) way to always ensure a consistent
page numbering is to restart the numbering in each child document
and denote the pages by `\textit{child}|.|\textit{page}'
where \textit{child} represents the chapter/section number of the child file.
This can be achieved by the command
|\numberwithin{page}{|\textit{child}|}|
of the \textsf{amsmath} package
where \textit{child} can be |chapter| or |section|
depending on the chosen structuring.
Alternatively, one can modify the macro |\thepage| appropriately
and reset the counter |page| at the start of each child file.

%%%%%%%%%%%%%%%%%%%%%%%%%%%%%%%%%%%%%%%%%%%%%%%%%%%%%%%%%%%%%%%%%%%%%%%%%%%%%%%%
\subsection{Conditional Processing}
\label{sec:conditional}

The package provides a mechanism to compile different versions
of a document. To customise the versions further some conditional processing
can come in handy to distinguish which version is being compiled.
The package provides two macros to describe the compilation context:

%%%%%%%%%%%%%%%%%%%%%%%%%%%%%%%%%%%%%%%%
\DescribeMacro{\ifchilddoc}
The conditional |\ifchilddoc| distinguishes between the compilation of
child documents and the main document:
%
\begin{center}
|\ifchilddoc |\textit{child-code}| |[|\||else |\textit{main-code}]| \||fi|
\end{center}

%%%%%%%%%%%%%%%%%%%%%%%%%%%%%%%%%%%%%%%%
\DescribeMacro{\childdocname}
\DescribeMacro{\childdocjob}
The macro |\childdocname| contains the filename (without extension)
of the main or child file being processed.
Note that |\childdocjob| will always contain the name of the main file.

%%%%%%%%%%%%%%%%%%%%%%%%%%%%%%%%%%%%%%%%
\paragraph{Title Page.}

Conditional processing can be used to include a title or banner page
in the main document when proper precautions are taken.
Importantly, the code in the main file should ensure that the page counter
(as well as other status parameters which are stored in the |.aux| files)
takes the same value after the conditional processing.
Otherwise the page numbers may take divergent values
depending on which part is compiled.

For example, a title page could be declared by:
%
\begin{center}
\begin{tabular}{l}
|\ifchilddoc\||else|\\
|\addtocounter{page}{-1}|\\
\textit{code for title page}\\
|\newpage|\\
|\||fi|
\end{tabular}
\end{center}
%
A banner page for the child documents can be generated by:
%
\begin{center}
\begin{tabular}{l}
|\ifchilddoc|\\
|\addtocounter{page}{-1}|\\
\textit{code for banner page}\\
|\newpage|\\
|\||fi|
\end{tabular}
\end{center}
%
Here one could write a message such as:
\begin{center}
|This is the part \childdocname{} of \childdocjob{}.|
\end{center}

%%%%%%%%%%%%%%%%%%%%%%%%%%%%%%%%%%%%%%%%%%%%%%%%%%%%%%%%%%%%%%%%%%%%%%%%%%%%%%%%
\subsection{Flags}
\label{sec:flags}

The package makes it easy to generate different versions
of the main or child documents.
To this end compilation flags can be defined
and assigned different default values.
They will be particularly useful in conjunction
with the forwarding mechanism described in \secref{sec:forward}.

For example, it may be useful to have a flag |\version|
which can be set to |draft| or |final|.
The document source will contain some conditional code
depending on the value of |\version|.
Suppose further, the flag should default to |final| for the main file
and to |draft| for child files
which is a natural assignment for editing the document.
This is achieved by placing the following code
in the preamble of the main document
(below the |\childdocmain| directive):
%
\begin{center}
\begin{tabular}{l}
|\ifchilddoc|\\
|\providecommand{\version}{draft}|\\
|\||else|\\
|\providecommand{\version}{final}|\\
|\||fi|
\end{tabular}
\end{center}
%
The definition by |\providecommand| makes sure
that previous definitions are not overwritten.
Further statements |\providecommand{\version}{...}|
can thus be added before the above code to override it.

For the main file, one might add a line
(between |\childdocmain| and the above block)
%
\begin{center}
|%\ifchilddoc\||else\providecommand{\version}{draft}\||fi|
\end{center}
%
which can be uncommented to produce a draft version.
Likewise one can add a line to the very top of a child file
(above the |\childdocof{|\textit{main}|}| directive)
%
\begin{center}
|%\providecommand{\version}{final}|
\end{center}
%
which can be uncommented to produce the final version of this child document.

%%%%%%%%%%%%%%%%%%%%%%%%%%%%%%%%%%%%%%%%%%%%%%%%%%%%%%%%%%%%%%%%%%%%%%%%%%%%%%%%
\subsection{Forwarding}
\label{sec:forward}

Different versions of the main or child documents
using compilation flags as described in \secref{sec:flags}
can be (permanently) stored in different files
for convenient compilation, viewing and distribution.
To this end, the package defines a command
to pass on compilation to a different file:

%%%%%%%%%%%%%%%%%%%%%%%%%%%%%%%%%%%%%%%%
\DescribeMacro{\childdocforward}
The command |\childdocforward| redirects processing to
another source file:
%
\begin{center}
\begin{tabular}{l}
|% \iffalse
%
% childdoc.dtx Copyright (C) 2017-2018 Niklas Beisert
%
% This work may be distributed and/or modified under the
% conditions of the LaTeX Project Public License, either version 1.3
% of this license or (at your option) any later version.
% The latest version of this license is in
%   http://www.latex-project.org/lppl.txt
% and version 1.3 or later is part of all distributions of LaTeX
% version 2005/12/01 or later.
%
% This work has the LPPL maintenance status `maintained'.
%
% The Current Maintainer of this work is Niklas Beisert.
%
% This work consists of the files childdoc.dtx and childdoc.ins
% and the derived files childdoc.def and cdocsamp.tex with
% cdocsch1.tex, cdocsch2.tex, cdocsdrf.tex, cdocsfn1.tex, cdocsfn2.tex.
%
%<package>\ifdefined\childdocmain\endinput\fi
%<package>\ProvidesFile{childdoc.def}[2018/12/30 v2.0 child document driver]
%<samplemain>\ProvidesFile{cdocsamp.tex}[2018/12/30 v2.0 sample for childdoc]
%<*driver>
%\ProvidesFile{childdoc.drv}[2018/12/30 v2.0 childdoc reference manual file]
\PassOptionsToClass{10pt,a4paper}{article}
\documentclass{ltxdoc}

\usepackage[margin=35mm]{geometry}
\usepackage{hyperref}
\usepackage{hyperxmp}
\usepackage[usenames]{color}

\hypersetup{colorlinks=true}
\hypersetup{pdfstartview=FitH}
\hypersetup{pdfpagemode=UseNone}
\hypersetup{pdfsource={}}
\hypersetup{pdflang={en-UK}}
\hypersetup{pdfcopyright={Copyright 2017-2018 Niklas Beisert.
  This work may be distributed and/or modified under the
  conditions of the LaTeX Project Public License, either version 1.3
  of this license or (at your option) any later version.}}
\hypersetup{pdflicenseurl={http://www.latex-project.org/lppl.txt}}
\hypersetup{pdfcontactaddress={ETH Zurich, ITP, HIT K,
  Wolfgang-Pauli-Strasse 27}}
\hypersetup{pdfcontactpostcode={8093}}
\hypersetup{pdfcontactcity={Zurich}}
\hypersetup{pdfcontactcountry={Switzerland}}
\hypersetup{pdfcontactemail={nbeisert@itp.phys.ethz.ch}}
\hypersetup{pdfcontacturl={http://people.phys.ethz.ch/\xmptilde nbeisert/}}

\newcommand{\secref}[1]{\hyperref[#1]{section \ref*{#1}}}

\parskip1ex
\parindent0pt
\let\olditemize\itemize
\def\itemize{\olditemize\parskip0pt}

\begin{document}

\title{The \textsf{childdoc} Package}
\hypersetup{pdftitle={The childdoc Package}}
\author{Niklas Beisert\\[2ex]
  Institut f\"ur Theoretische Physik\\
  Eidgen\"ossische Technische Hochschule Z\"urich\\
  Wolfgang-Pauli-Strasse 27, 8093 Z\"urich, Switzerland\\[1ex]
  \href{mailto:nbeisert@itp.phys.ethz.ch}
  {\texttt{nbeisert@itp.phys.ethz.ch}}}
\hypersetup{pdfauthor={Niklas Beisert}}
\hypersetup{pdfsubject={Manual for the LaTeX2e Package childdoc}}
\date{30 December 2018, \textsf{v2.0}}
\maketitle

\begin{abstract}\noindent
\textsf{childdoc} is a \LaTeXe{} package
that enables the direct compilation
of document sections included by |\include|
to individual files.
\end{abstract}

\begingroup
\parskip0ex
\tableofcontents
\endgroup

%%%%%%%%%%%%%%%%%%%%%%%%%%%%%%%%%%%%%%%%%%%%%%%%%%%%%%%%%%%%%%%%%%%%%%%%%%%%%%%%
%%%%%%%%%%%%%%%%%%%%%%%%%%%%%%%%%%%%%%%%%%%%%%%%%%%%%%%%%%%%%%%%%%%%%%%%%%%%%%%%
\section{Introduction}

\LaTeX{} provides a mechanism to structure a large document (such as a book)
into a main file and several child files (containing the chapters)
using the |\include| command.
This mechanism is beneficial for documents
which span hundreds of pages in order to
make the source file(s) more manageable.
Moreover, compilation can be restricted to
selected child files by means of the |\includeonly| command.
The latter feature can be used to reduce the compilation time while editing
(this was significantly more useful in the earlier days of \LaTeX{})
or to generate a smaller document which is easier to navigate.
Another application of |\includeonly| is to generate
documents consisting of selected parts of the complete document.

However, there are a few drawbacks of the plain |\include| mechanism:
\begin{itemize}
\item
The child files cannot be compiled on their own,
they can only be compiled via the main file.
A naive editing environment
(such as a text editor with an option
to have the current file processed by \LaTeX)
may require one to switch to the main file before compiling;
attempting to compile the child file produces errors.
\item
The main file must be modified (each time)
to adjust the |\includeonly| command
to the present needs. This easily leaves the main file in a messy state.
\item
The generated document will always carry the filename
of the main document. This is inconvenient if
several child files are to be compiled and
to be kept for distribution.
\end{itemize}

The present package provides a simple interface
to make child files individually compilable by \LaTeX{}.
Compiling a child file then has the same effect as compiling
the main file with an |\includeonly| command
to select the appropriate child.
Moreover the generated document will carry the name of the child
rather than the main file.
This resolves all three above issues.

This feature is meant to make the editing of books,
thesis documents and lecture notes somewhat more convenient.
However, the package can also be used efficiently for
composing a series of documents (such as exercise sheets)
which are typically distributed individually.
It then assists the author in generating the individual documents
(potentially in different versions)
as well as a document containing the collected series.
Another application is in developing style files
or other kinds of included material
where compilation of the style file could redirect
to a sample or test file.

%%%%%%%%%%%%%%%%%%%%%%%%%%%%%%%%%%%%%%%%%%%%%%%%%%%%%%%%%%%%%%%%%%%%%%%%%%%%%%%%
%%%%%%%%%%%%%%%%%%%%%%%%%%%%%%%%%%%%%%%%%%%%%%%%%%%%%%%%%%%%%%%%%%%%%%%%%%%%%%%%
\section{Usage}

First of all, the package \textsf{childdoc} is \emph{not} a standard
\LaTeXe{} |.sty| style file! Therefore it needs to be invoked in
a non-standard way.

%%%%%%%%%%%%%%%%%%%%%%%%%%%%%%%%%%%%%%%%%%%%%%%%%%%%%%%%%%%%%%%%%%%%%%%%%%%%%%%%
\subsection{Included Files}
\label{sec:include}

%%%%%%%%%%%%%%%%%%%%%%%%%%%%%%%%%%%%%%%%
\DescribeMacro{\childdocmain}
To use the package, add the commands
\begin{center}
\begin{tabular}{l}
|\input{childdoc.def}|\\
|\childdocmain{}|\\
\end{tabular}
\end{center}
at the very top of the main \LaTeX{} file,
in particular \emph{before} the |\documentclass| statement!
The argument of |\childdocmain| should be left empty
(but it must be present).

%%%%%%%%%%%%%%%%%%%%%%%%%%%%%%%%%%%%%%%%
\DescribeMacro{\childdocof}
Furthermore, add the commands
\begin{center}
\begin{tabular}{l}
|\input{childdoc.def}|\\
|\childdocof{|\textit{main}|}|\\
\end{tabular}
\end{center}
at the top of every child file \textit{child}
which is included by |\include{|\textit{child}|}|
from within the main file
(or at least for those files to be compiled individually).
The argument \textit{main} must be the filename of the main file.

There are a couple of
considerations in setting up the main and child documents:

%%%%%%%%%%%%%%%%%%%%%%%%%%%%%%%%%%%%%%%%
\paragraph{Restrictions.}

Please note the following restrictions:
\begin{itemize}
\item
|\childdocmain| must be called with one argument \textit{main}
to ensure compatibility with earlier version of the package.
It must either be empty (|\childdocmain{}|)
or precisely match the filename of the main file in which it is specified.
See \secref{sec:detection} for further information.
\item
The filename \textit{main} must be specified without the |.tex| extension.
\item
The filename \textit{main} is case sensitive
(even in case-insensitive file systems)
due to internal string comparison.
\item
The argument \textit{main} should be fully expanded, it cannot be a macro.
\item
Subdirectories and special characters should be avoided in filenames.
\item
The command |\childdocmain{|\textit{main}|}| must be followed by a whitespace.
It should not be followed immediately by another command
or by a comment mark `|%|'.
This is because the \TeX{} parser reads the token immediately following
the argument of |\childdocmain| and puts it
at the beginning of every child section;
however, a white\-space is ignored.
\end{itemize}

%%%%%%%%%%%%%%%%%%%%%%%%%%%%%%%%%%%%%%%%
\paragraph{Content of Main File.}

It is advisable to place all content in the child files included by |\include|.
Any output contained in the main file will appear in all child documents
unless suppressed manually;
it cannot be suppressed automatically by the |\includeonly| directive
and thus should normally be avoided.
A method to include some content in the main file
by means of conditional processing is described in \secref{sec:conditional}.

%%%%%%%%%%%%%%%%%%%%%%%%%%%%%%%%%%%%%%%%
\paragraph{Page Numbering.}

When only a part of the document is compiled,
the appropriate numbering of pages
(as well as other status parameters)
is determined from the |.aux| files.
The latter contain information from previous passes.
However this information needs to propagate through
all intermediate child documents.
Therefore the page numbering in child documents may well
be inconsistent until the complete document is compiled at least once.

A useful (if unconventional) way to always ensure a consistent
page numbering is to restart the numbering in each child document
and denote the pages by `\textit{child}|.|\textit{page}'
where \textit{child} represents the chapter/section number of the child file.
This can be achieved by the command
|\numberwithin{page}{|\textit{child}|}|
of the \textsf{amsmath} package
where \textit{child} can be |chapter| or |section|
depending on the chosen structuring.
Alternatively, one can modify the macro |\thepage| appropriately
and reset the counter |page| at the start of each child file.

%%%%%%%%%%%%%%%%%%%%%%%%%%%%%%%%%%%%%%%%%%%%%%%%%%%%%%%%%%%%%%%%%%%%%%%%%%%%%%%%
\subsection{Conditional Processing}
\label{sec:conditional}

The package provides a mechanism to compile different versions
of a document. To customise the versions further some conditional processing
can come in handy to distinguish which version is being compiled.
The package provides two macros to describe the compilation context:

%%%%%%%%%%%%%%%%%%%%%%%%%%%%%%%%%%%%%%%%
\DescribeMacro{\ifchilddoc}
The conditional |\ifchilddoc| distinguishes between the compilation of
child documents and the main document:
%
\begin{center}
|\ifchilddoc |\textit{child-code}| |[|\||else |\textit{main-code}]| \||fi|
\end{center}

%%%%%%%%%%%%%%%%%%%%%%%%%%%%%%%%%%%%%%%%
\DescribeMacro{\childdocname}
\DescribeMacro{\childdocjob}
The macro |\childdocname| contains the filename (without extension)
of the main or child file being processed.
Note that |\childdocjob| will always contain the name of the main file.

%%%%%%%%%%%%%%%%%%%%%%%%%%%%%%%%%%%%%%%%
\paragraph{Title Page.}

Conditional processing can be used to include a title or banner page
in the main document when proper precautions are taken.
Importantly, the code in the main file should ensure that the page counter
(as well as other status parameters which are stored in the |.aux| files)
takes the same value after the conditional processing.
Otherwise the page numbers may take divergent values
depending on which part is compiled.

For example, a title page could be declared by:
%
\begin{center}
\begin{tabular}{l}
|\ifchilddoc\||else|\\
|\addtocounter{page}{-1}|\\
\textit{code for title page}\\
|\newpage|\\
|\||fi|
\end{tabular}
\end{center}
%
A banner page for the child documents can be generated by:
%
\begin{center}
\begin{tabular}{l}
|\ifchilddoc|\\
|\addtocounter{page}{-1}|\\
\textit{code for banner page}\\
|\newpage|\\
|\||fi|
\end{tabular}
\end{center}
%
Here one could write a message such as:
\begin{center}
|This is the part \childdocname{} of \childdocjob{}.|
\end{center}

%%%%%%%%%%%%%%%%%%%%%%%%%%%%%%%%%%%%%%%%%%%%%%%%%%%%%%%%%%%%%%%%%%%%%%%%%%%%%%%%
\subsection{Flags}
\label{sec:flags}

The package makes it easy to generate different versions
of the main or child documents.
To this end compilation flags can be defined
and assigned different default values.
They will be particularly useful in conjunction
with the forwarding mechanism described in \secref{sec:forward}.

For example, it may be useful to have a flag |\version|
which can be set to |draft| or |final|.
The document source will contain some conditional code
depending on the value of |\version|.
Suppose further, the flag should default to |final| for the main file
and to |draft| for child files
which is a natural assignment for editing the document.
This is achieved by placing the following code
in the preamble of the main document
(below the |\childdocmain| directive):
%
\begin{center}
\begin{tabular}{l}
|\ifchilddoc|\\
|\providecommand{\version}{draft}|\\
|\||else|\\
|\providecommand{\version}{final}|\\
|\||fi|
\end{tabular}
\end{center}
%
The definition by |\providecommand| makes sure
that previous definitions are not overwritten.
Further statements |\providecommand{\version}{...}|
can thus be added before the above code to override it.

For the main file, one might add a line
(between |\childdocmain| and the above block)
%
\begin{center}
|%\ifchilddoc\||else\providecommand{\version}{draft}\||fi|
\end{center}
%
which can be uncommented to produce a draft version.
Likewise one can add a line to the very top of a child file
(above the |\childdocof{|\textit{main}|}| directive)
%
\begin{center}
|%\providecommand{\version}{final}|
\end{center}
%
which can be uncommented to produce the final version of this child document.

%%%%%%%%%%%%%%%%%%%%%%%%%%%%%%%%%%%%%%%%%%%%%%%%%%%%%%%%%%%%%%%%%%%%%%%%%%%%%%%%
\subsection{Forwarding}
\label{sec:forward}

Different versions of the main or child documents
using compilation flags as described in \secref{sec:flags}
can be (permanently) stored in different files
for convenient compilation, viewing and distribution.
To this end, the package defines a command
to pass on compilation to a different file:

%%%%%%%%%%%%%%%%%%%%%%%%%%%%%%%%%%%%%%%%
\DescribeMacro{\childdocforward}
The command |\childdocforward| redirects processing to
another source file:
%
\begin{center}
\begin{tabular}{l}
|\input{childdoc.def}|\\
|\childdocforward[|\textit{main}|]{|\textit{dest}|}|\\
\end{tabular}
\end{center}
%
The argument \textit{dest} is the destination file
(without extension).
It should be the main file or one of the child files.
Note that further \textsf{childdoc} directives
such as |\childdocof| and |\childdocforward|
in the indicated file will be processed in this form.
The optional argument \textit{main}
passes on directly to the main file \textit{main}
while pretending to compile the child \textit{dest}.
This form behaves as if \textit{dest}
issues |\childdocof{|\textit{main}|}| right away,
and no further \textsf{childdoc} directives will be processed.

%%%%%%%%%%%%%%%%%%%%%%%%%%%%%%%%%%%%%%%%
\DescribeMacro{\...prefix}
In the alternative form |\childdocforwardprefix|,
%
\begin{center}
\begin{tabular}{l}
|\input{childdoc.def}|\\
|\childdocforwardprefix[|\textit{main}|]{|\textit{prefix}|}{|\textit{dest}|}|
\end{tabular}
\end{center}
%
the destination file is determined by a pattern
depending on the current file:
To make this work, the current file must be called
`{\textit{prefix}\hspace{0.2em}\textit{suffix}}'
with \textit{prefix} matching precisely the argument.
Processing is then passed on to the file
`{\textit{dest}\hspace{0.2em}\textit{suffix}}'.
Surely, the same effect is achieved by
directly specifying the
argument `{\textit{dest}\hspace{0.2em}\textit{suffix}}'
in the first form.
However, that requires to set up a different file
for each child. With the alternative form of the command
all these files can have exactly the same content
which simplifies setting them up and maintaining them.

For example, the following file |draft.tex|
with a compilation flag |\version| as described in \secref{sec:flags}
compiles the main document as a draft:
%
\begin{center}
\begin{tabular}{l}
|\def\version{draft}|\\
|\input{childdoc.def}|\\
|\childdocforward{|\textit{main}|}|
\end{tabular}
\end{center}
%
Likewise, the following files |final|\textit{nn}|.tex|
compile the final version of the child document
|child|\textit{nn}|.tex|:
%
\begin{center}
\begin{tabular}{l}
|\def\version{final}|\\
|\input{childdoc.def}|\\
|\childdocforwardprefix{final}{child}|
\end{tabular}
\end{center}
%

Note that when several versions of a main file and/or of each child file
are to be generated, it may be convenient to set up a |Makefile| or
shell script to automatise the process.

%%%%%%%%%%%%%%%%%%%%%%%%%%%%%%%%%%%%%%%%%%%%%%%%%%%%%%%%%%%%%%%%%%%%%%%%%%%%%%%%
\subsection{Command Line Processing}
\label{sec:commandline}

The effect of redirection files can also be achieved by invoking
the \LaTeX{} compiler with a more elaborate command line.
Most conveniently this should be done as part
of a shell script or a |Makefile|.

When using \textsf{childdoc} in the main file, the following
command lines effectively perform a redirection
(note that depending on the shell being used,
backslashes may have to be doubled: `|\|' $\to$ `|\\|'):
%
\begin{center}
|... -jobname "|\textit{target}|" |\\|"|[\textit{flags}]%
|\input{childdoc.def}\childdocforward[|\textit{main}|]{|\textit{dest}|}"|
\end{center}
%
Here \textit{target} is the name of the output file,
\textit{main} is the name of the main file
and \textit{dest} is the name of the main or child file to be processed
(all filenames without extensions).
The optional argument \textit{main} can be omitted
if \textit{main} matches \textit{dest}.
Optionally, compilation \textit{flags} can be defined via |\def| commands.
This command line makes the \TeX{} engine believe
it is compiling the file \textit{target}
whose content is specified as the latter parameter.
The provided code then forwards the processing to
\textit{main} or \textit{dest} as described in \secref{sec:forward}.

%%%%%%%%%%%%%%%%%%%%%%%%%%%%%%%%%%%%%%%%%%%%%%%%%%%%%%%%%%%%%%%%%%%%%%%%%%%%%%%%
\subsection{Include by Input}
\label{sec:input}

Including child documents by |\include| has some restrictions by design.
Most notably, the content of a child document always occupies
its own set of pages; pages cannot be shared between child documents.
Usually, this behaviour makes perfect sense
because each child document contain an essential part of the document.
However, in some situations it may be desirable to compose
a document from a collection of parts
without having mandatory page breaks between then.
For this case, the package
provides a mechanism to include parts
by |\input| which can also be processed individually.
However, by construction this mechanism
requires manual handling of the content to be output.

%%%%%%%%%%%%%%%%%%%%%%%%%%%%%%%%%%%%%%%%
\DescribeMacro{\ifchilddocmanual}
The main file should be prepared as usual, see \secref{sec:include}.
However, the document body must make a distinction
between processing of an individual part and of the main document, e.g.:
%
\begin{center}
\begin{tabular}{l}
|\ifchilddocmanual|\\
|\input{\childdocname}|\\
|\||else|\\
\textit{document body with }|\input{|\textit{part}|}|\\
|\||fi|
\end{tabular}
\end{center}
%
The conditional |\ifchilddocmanual| is true whenever
a part to be included by |\input| is being compiled,
and the name of the part is stored in |\childdocname|.

%%%%%%%%%%%%%%%%%%%%%%%%%%%%%%%%%%%%%%%%
\DescribeMacro{\childdocby}
Each part to be included by |\input| should start with:
%
\begin{center}
\begin{tabular}{l}
|\input{childdoc.def}|\\
|\childdocby{|\textit{main}|}|\\
\end{tabular}
\end{center}
%
The directive |\childdocby| is similar to |\childdocof|
described in \secref{sec:include},
but the subsequent selection of content must be done manually.
To that end, both |\ifchilddoc| and |\ifchilddocmanual|
will be true upon processing of a part,
and the name of the part is stored in |\childdocname|.
Note that |\jobname| will be set to the filename of the current part
so that each part receives an individual |.aux| file
that does not interfere with the |.aux| file(s) of the main document.
This behaviour can be altered by the alternative form
|\childdocby[*]{|\textit{main}|}| (with a non-empty optional argument)
which uses the |.aux| file of the main document
by setting |\jobname| to \textit{main}.

%%%%%%%%%%%%%%%%%%%%%%%%%%%%%%%%%%%%%%%%%%%%%%%%%%%%%%%%%%%%%%%%%%%%%%%%%%%%%%%%
\subsection{Driver Development}
\label{sec:driver}

The \textsf{childdoc} mechanism can also be use for the development
of definition files such as \LaTeX{} styles or classes.
This case differs from the above setup with multiple parts
included by |\include| in that no |\includeonly| should be invoked.
This can be achieved by starting the include file
(before |\ProvidesPackage|) with:
%
\begin{center}
\begin{tabular}{l}
|\input{childdoc.def}|\\
|\childdocforward{|\textit{main}|}|\\
\end{tabular}
\end{center}
%
or alternatively with:
%
\begin{center}
\begin{tabular}{l}
|\input{childdoc.def}|\\
|\childdocby{|\textit{main}|}|\\
\end{tabular}
\end{center}
%
Both forms have slightly different effects as described above.
The main file is prepared as usual, see \secref{sec:include}.

%%%%%%%%%%%%%%%%%%%%%%%%%%%%%%%%%%%%%%%%%%%%%%%%%%%%%%%%%%%%%%%%%%%%%%%%%%%%%%%%
\subsection{Legacy Detection}
\label{sec:detection}

The directive |\childdocmain| in the main file can detect
whether the complete document or merely a child is to be compiled
even without using the directive |\childdocof|.
This method is deprecated because it is less robust
and there is no compelling reason to use it;
it is merely provided for backward compatibility
and it may be removed in future versions.

If the detection mechanism is to be used,
it is mandatory to correctly specify
the filename of the main file as the argument of |\childdocmain|:
%
\begin{center}
\begin{tabular}{l}
|\input{childdoc.def}|\\
|\childdocmain{|\textit{main}|}|\\
\end{tabular}
\end{center}
%
If |\jobname| does not match the argument \textit{main} of |\childdocmain|,
it is assumed that |\jobname| points to the child file to be compiled.
When using |\childdocmain| with the main file specified as argument,
it suffices to start a child file
with just |\input{|\textit{main}|}|
without loading of the package and using |\childdocof|.
If instead all processing is done
with the appropriate \textsf{childdoc} directives,
the argument of \textit{main} of |\childdocmain| can be empty.

An alternative version of the command line processing described
in \secref{sec:commandline} using the detection mechanism reads:
%
\begin{center}
|... -jobname "|\textit{target}|" "|[\textit{flags}]%
[|\def\jobname{|\textit{dest}|}|]|\input{|\textit{main}|}"|
\end{center}

%%%%%%%%%%%%%%%%%%%%%%%%%%%%%%%%%%%%%%%%%%%%%%%%%%%%%%%%%%%%%%%%%%%%%%%%%%%%%%%%
\subsection{Manual Code}
\label{sec:manual}

In case one cannot be certain whether the definitions file |childdoc.def|
is installed on the target \TeX{} distribution
and one prefers not to ship it,
it is conceivable to paste a few relevant commands into the sources.

To that end, drop all statements |\input{childdoc.def}|
and perform the replacements as outlined below.
Instead of |\childdocmain{|\textit{main}|}| add the following code
to the top of the main file:
%
\begin{center}
\begin{tabular}{l}
|\||ifdefined\childdocname\endinput\||fi\newif\ifchilddoc|\\
|\edef\childdocname{\scantokens\expandafter{\jobname\noexpand}}|\\
|\def\childdocmain{|\textit{main}|}\||ifx\childdocmain\childdocname\||else|\\
|\childdoctrue\includeonly{\childdocname}\let\jobname\childdocmain\||fi|\\
\end{tabular}
\end{center}
%
Instead of |\childdocof{|\textit{main}|}| just include the main file
at the top of each child file:
%
\begin{center}
|\input{|\textit{main}|}|
\end{center}
%
A simple redirection |\childdocforward{|\textit{dest}|}| is achieved by:
%
\begin{center}
|\def\jobname{|\textit{dest}|}\input{\jobname}|
\end{center}
%
The redirection with prefix
|\childdocforwardprefix[|\textit{prefix}|]{|\textit{dest}|}|
is accomplished by:
%
\begin{center}
\begin{tabular}{l}
|{\edef\jobname{\scantokens\expandafter{\jobname\noexpand}}|\\
|\def\redirectjob |\textit{prefix}|#1~~~{\gdef\jobname{|\textit{dest}|#1}}|\\
|\expandafter\redirectjob\jobname~~~}\input{\jobname}|
\end{tabular}
\end{center}

In an alternative approach,
child documents can be compiled by a specific command line
without additional code or specific definitions:
%
\begin{center}
|... -jobname "|\textit{target}|" "|[\textit{flags}]%
|\includeonly{|\textit{dest}|}\input{|\textit{main}|}"|
\end{center}
%

%%%%%%%%%%%%%%%%%%%%%%%%%%%%%%%%%%%%%%%%%%%%%%%%%%%%%%%%%%%%%%%%%%%%%%%%%%%%%%%%
%%%%%%%%%%%%%%%%%%%%%%%%%%%%%%%%%%%%%%%%%%%%%%%%%%%%%%%%%%%%%%%%%%%%%%%%%%%%%%%%
\section{Information}

%%%%%%%%%%%%%%%%%%%%%%%%%%%%%%%%%%%%%%%%%%%%%%%%%%%%%%%%%%%%%%%%%%%%%%%%%%%%%%%%
\subsection{Copyright}

Copyright \copyright{} 2017--2018 Niklas Beisert

This work may be distributed and/or modified under the
conditions of the \LaTeX{} Project Public License, either version 1.3
of this license or (at your option) any later version.
The latest version of this license is in
  \url{http://www.latex-project.org/lppl.txt}
and version 1.3 or later is part of all distributions of \LaTeX{}
version 2005/12/01 or later.

This work has the LPPL maintenance status `maintained'.

The Current Maintainer of this work is Niklas Beisert.

This work consists of the files |README.txt|, |childdoc.ins| and |childdoc.dtx|
as well as the derived files |childdoc.def|, |cdocsamp.tex|
with |cdocsch1.tex|, |cdocsch2.tex|, |cdocspt3.tex|, |cdocspt4.tex|,
|cdocsdrf.tex|, |cdocsfn1.tex|, |cdocsfn2.tex|
as well as |childdoc.pdf|.

%%%%%%%%%%%%%%%%%%%%%%%%%%%%%%%%%%%%%%%%%%%%%%%%%%%%%%%%%%%%%%%%%%%%%%%%%%%%%%%%
\subsection{Files and Installation}

The package consists of the files:
%
\begin{center}
\begin{tabular}{ll}
    |README.txt|   & readme file \\
    |childdoc.ins| & installation file \\
    |childdoc.dtx| & source file \\
    |childdoc.def| & definition file \\
    |cdocsamp.tex| & sample main file \\
    |cdocsch1.tex| & sample include file \\
    |cdocsch2.tex| & sample include file \\
    |cdocspt3.tex| & sample part file \\
    |cdocspt4.tex| & sample part file \\
    |cdocsdrf.tex| & sample redirection file \\
    |cdocsfn1.tex| & sample redirection file \\
    |cdocsfn2.tex| & sample redirection file \\
    |childdoc.pdf| & manual
\end{tabular}
\end{center}
%
The distribution consists of the files
|README.txt|, |childdoc.ins| and |childdoc.dtx|.
%
\begin{itemize}
\item
Run (pdf)\LaTeX{} on |childdoc.dtx|
to compile the manual |childdoc.pdf| (this file).
\item
Run \LaTeX{} on |childdoc.ins| to create the definitions file |childdoc.def|
and the sample |cdocsamp.tex| with include files
|cdocsch1.tex|, |cdocsch2.tex|, |cdocspt3.tex|, |cdocspt4.tex|,
|cdocsdrf.tex|, |cdocsfn1.tex|, |cdocsfn2.tex|.
Then copy the file |childdoc.def| to an appropriate directory of your \LaTeX{}
distribution, e.g.\ \textit{texmf-root}|/tex/latex/childdoc|.
\end{itemize}

%%%%%%%%%%%%%%%%%%%%%%%%%%%%%%%%%%%%%%%%%%%%%%%%%%%%%%%%%%%%%%%%%%%%%%%%%%%%%%%%
\subsection{Related CTAN Packages}

There are several other packages which offer a similar functionality:
%
\begin{itemize}
\item
The packages
\href{http://ctan.org/pkg/docmute}{\textsf{docmute}},
\href{http://ctan.org/pkg/includex}{\textsf{includex}} and
\href{http://ctan.org/pkg/standalone}{\textsf{standalone}}
provide commands to include only the document body of
a child file thus allowing both files to be compiled individually.
\item
The packages \href{http://ctan.org/pkg/subdocs}{\textsf{subdocs}}
and \href{http://ctan.org/pkg/subfiles}{\textsf{subfiles}}
provide structures in which the main and child documents can be
encapsulated and allowing them to be compiled individually.
The inclusion mechanism is different from the conventional |\include|.
\item
The package \href{http://ctan.org/pkg/combine}{\textsf{combine}}
is an elaborate solution to combine several documents into one.
\end{itemize}
%
See also the CTAN topic \href{http://ctan.org/topic/subdocs}{\textsf{subdocs}}
for further related packages.
The present package differs from the above solutions in that
a document structure constructed with the conventional |\include| mechanism
just needs two extra commands at the top of every file
such that all constituent files can be compiled individually.

%%%%%%%%%%%%%%%%%%%%%%%%%%%%%%%%%%%%%%%%%%%%%%%%%%%%%%%%%%%%%%%%%%%%%%%%%%%%%%%%
%\subsection{Feature Suggestions}
%
%The following is a list of features which may be useful for future
%versions of this package:
%%
%\begin{itemize}
%\item
%\ldots
%\end{itemize}

%%%%%%%%%%%%%%%%%%%%%%%%%%%%%%%%%%%%%%%%%%%%%%%%%%%%%%%%%%%%%%%%%%%%%%%%%%%%%%%%
\subsection{Revision History}

%%%%%%%%%%%%%%%%%%%%%%%%%%%%%%%%%%%%%%%%
\paragraph{v2.0:} 2018/12/30

\begin{itemize}
\item
immediate forward processing
\item
added |\childdocby| mechanism
\item
manual restructured
\end{itemize}

%%%%%%%%%%%%%%%%%%%%%%%%%%%%%%%%%%%%%%%%
\paragraph{v1.6:} 2018/01/17

\begin{itemize}
\item
application for development of include files
\item
corrections to manual
\end{itemize}

%%%%%%%%%%%%%%%%%%%%%%%%%%%%%%%%%%%%%%%%
\paragraph{v1.5:} 2017/05/21

\begin{itemize}
\item
more complete structuring introduced
\item
|\childdocof| introduced
\item
|\childdoc| renamed to |\childdocmain|
\item
|\childredirect| renamed to |\childdocforward| and |\childdocforwardprefix|
and functionality expanded
\end{itemize}

%%%%%%%%%%%%%%%%%%%%%%%%%%%%%%%%%%%%%%%%
\paragraph{v1.0:} 2017/04/27

\begin{itemize}
\item
manual and install package
\item
first version published on CTAN
\end{itemize}

%%%%%%%%%%%%%%%%%%%%%%%%%%%%%%%%%%%%%%%%
\paragraph{v0.6:} 2017/04/26

\begin{itemize}
\item
redirection mechanism added
\end{itemize}

%%%%%%%%%%%%%%%%%%%%%%%%%%%%%%%%%%%%%%%%
\paragraph{v0.5:} 2017/04/26

\begin{itemize}
\item
functionality in definition file
\end{itemize}


%%%%%%%%%%%%%%%%%%%%%%%%%%%%%%%%%%%%%%%%%%%%%%%%%%%%%%%%%%%%%%%%%%%%%%%%%%%%%%%%
%%%%%%%%%%%%%%%%%%%%%%%%%%%%%%%%%%%%%%%%%%%%%%%%%%%%%%%%%%%%%%%%%%%%%%%%%%%%%%%%
%%%%%%%%%%%%%%%%%%%%%%%%%%%%%%%%%%%%%%%%%%%%%%%%%%%%%%%%%%%%%%%%%%%%%%%%%%%%%%%%
\appendix

\settowidth\MacroIndent{\rmfamily\scriptsize 000\ }

 \DocInput{childdoc.dtx}

\end{document}
%</driver>
% \fi
%
% %%%%%%%%%%%%%%%%%%%%%%%%%%%%%%%%%%%%%%%%%%%%%%%%%%%%%%%%%%%%%%%%%%%%%%%%%%%%%%
% %%%%%%%%%%%%%%%%%%%%%%%%%%%%%%%%%%%%%%%%%%%%%%%%%%%%%%%%%%%%%%%%%%%%%%%%%%%%%%
% \section{Sample}
%\iffalse
%<*samplemain>
%\fi
%
% The following presents a sample document
% with two chapters, two parts, a title page,
% a compile flag as well as three forwarding files to set the flag.
% It consists of eight |.tex| files:
% \begin{center}
% \begin{tabular}{ll}
% |cdocsamp.tex|&main file\\
% |cdocsch1.tex|&include file for chapter 1\\
% |cdocsch2.tex|&include file for chapter 2\\
% |cdocspt3.tex|&include file for part 3\\
% |cdocspt4.tex|&include file for part 4\\
% |cdocsdrf.tex|&forwarding file for main file in draft mode\\
% |cdocsfi1.tex|&forwarding file for final version of chapter 1\\
% |cdocsfi2.tex|&forwarding file for final version of chapter 2\\
% \end{tabular}
% \end{center}
% Each of the eight files can be compiled directly by the \LaTeX{} compiler.
%
% %%%%%%%%%%%%%%%%%%%%%%%%%%%%%%%%%%%%%%
% \paragraph{Main File.}
%
% The main file is called |cdocsamp.tex|.
%
% Load the \textsf{childdoc} definitions and
% declare the filename for the main document:
%    \begin{macrocode}
\input{childdoc.def}
\childdocmain{}
%    \end{macrocode}

% Optional override for |\version| flag:
%    \begin{macrocode}
%%\ifchilddoc\else\providecommand{\version}{draft}\fi
%    \end{macrocode}

% Define the default values for the |\version| flag
% (|final| for the main file and |draft| for childs):
%    \begin{macrocode}
\ifchilddoc
\providecommand{\version}{draft}
\else
\providecommand{\version}{final}
\fi
%    \end{macrocode}

% Load the standard document class:
%    \begin{macrocode}
\documentclass[12pt]{article}
%    \end{macrocode}

% Start the document body:
%    \begin{macrocode}
\begin{document}
%    \end{macrocode}

% Declare a title page.
% Print title, part of document being processed and version flag:
%    \begin{macrocode}
\addtocounter{page}{-1}
\begin{center}
{\LARGE\bfseries{}childdoc example\par}
\vspace{1cm}
\ifchilddoc
\ifchilddocmanual part\else chapter\fi:
`\childdocname' of `\childdocjob'\par
\else
main document: `\childdocjob'\par
\fi
version: \version\par
\end{center}
\newpage
%    \end{macrocode}

% Manually include selected file,
% otherwise process as usual:
%    \begin{macrocode}
\ifchilddocmanual
\section*{part `\childdocname'}
\input{\childdocname}
\else
%    \end{macrocode}

% Include the two chapters:
%    \begin{macrocode}
\include{cdocsch1}
\include{cdocsch2}
%    \end{macrocode}

% Include the two parts unless only chapters should be displayed:
%    \begin{macrocode}
\ifchilddoc\else
\section{part three}
\input{cdocspt3}
\section{part four}
\input{cdocspt4}
\fi
%    \end{macrocode}

% Process as usual until here:
%    \begin{macrocode}
\fi
%    \end{macrocode}

% End of document body:
%    \begin{macrocode}
\end{document}
%    \end{macrocode}
%\iffalse
%</samplemain>
%\fi
%
% %%%%%%%%%%%%%%%%%%%%%%%%%%%%%%%%%%%%%%
% \paragraph{Chapter Include Files.}
%
% The include files are called |cdocsch1.tex| and |cdocsch2.tex|.
%
%\iffalse
%<*samplechap1|samplechap2>
%\fi

% Optional override for |\version| flag:
%    \begin{macrocode}
%%\providecommand{\version}{final}
%    \end{macrocode}

% Include the main document:
%    \begin{macrocode}
\input{childdoc.def}
\childdocof{cdocsamp}
%    \end{macrocode}

%\iffalse
%</samplechap1|samplechap2>
%\fi
%
%\iffalse
%<*samplechap1>
%\fi
% Some text for chapter 1:
%    \begin{macrocode}
\section{one}
some text in chapter one
%    \end{macrocode}

%\iffalse
%</samplechap1>
%\fi
% Some text for chapter 2:
%\iffalse
%<*samplechap2>
%\fi
%    \begin{macrocode}
\section{two}
more text in chapter two
%    \end{macrocode}

%\iffalse
%</samplechap2>
%\fi
%
% %%%%%%%%%%%%%%%%%%%%%%%%%%%%%%%%%%%%%%
% \paragraph{Part Include Files.}
%
% The include files are called |cdocspt3.tex| and |cdocspt4.tex|.
%
%\iffalse
%<*samplepart3|samplepart4>
%\fi

% Optional override for |\version| flag:
%    \begin{macrocode}
%%\providecommand{\version}{final}
%    \end{macrocode}

% Include the main document:
%    \begin{macrocode}
\input{childdoc.def}
\childdocby{cdocsamp}
%    \end{macrocode}

%\iffalse
%</samplepart3|samplepart4>
%\fi
%
%\iffalse
%<*samplepart3>
%\fi
% Some text for part 3:
%    \begin{macrocode}
some text in part three
%    \end{macrocode}

%\iffalse
%</samplepart3>
%\fi
% Some text for part 4:
%\iffalse
%<*samplepart4>
%\fi
%    \begin{macrocode}
more text in part four
%    \end{macrocode}

%\iffalse
%</samplepart4>
%\fi
%
% %%%%%%%%%%%%%%%%%%%%%%%%%%%%%%%%%%%%%%
% \paragraph{Forwarding for a Complete Draft.}
%
% The following forwarding file |cdocsdrf.tex|
% compiles the main document in draft mode:
%\iffalse
%<*sampledraft>
%\fi
%    \begin{macrocode}
\def\version{draft}
\input{childdoc.def}
\childdocforward{cdocsamp}
%    \end{macrocode}

%\iffalse
%</sampledraft>
%\fi
%
% %%%%%%%%%%%%%%%%%%%%%%%%%%%%%%%%%%%%%%
% \paragraph{Forwarding for Final Version of the Chapters.}
%
% The following forwarding files |cdocsfn1.tex| and |cdocsfn2.tex|
% (with identical content)
% compile the final versions of the child documents
% |cdocsch1.tex| and |cdocsch2.tex|, respectively:
%\iffalse
%<*samplefinal>
%\fi
%    \begin{macrocode}
\def\version{final}
\input{childdoc.def}
\childdocforwardprefix[cdocsamp]{cdocsfn}{cdocsch}
%    \end{macrocode}

%\iffalse
%</samplefinal>
%\fi
%
% %%%%%%%%%%%%%%%%%%%%%%%%%%%%%%%%%%%%%%
% \paragraph{Command Line Processing.}
%
% The following three command lines generate the output files
% |cdocscld|, |cdocscl1| and |cdocscl2|
% which should be identical to
% |cdocsdrf|, |cdocsch1| and |cdocsfn2|, respectively:
% \begin{center}
% \begin{tabular}{l}
% |latex -jobname cdocscld \|\\
% |  "\def\version{draft}\input{childdoc.def}\childdocforward{cdocsamp}"|\\
% |latex -jobname cdocscl1 \|\\
% |  "\input{childdoc.def}\childdocforward[cdocsamp]{cdocsch1}"|\\
% |latex -jobname cdocscl2 \|\\
% |  "\def\version{final}\input{childdoc.def}\childdocforward{cdocsch2}"|
% \end{tabular}
% \end{center}
% Note that the trailing backslash on each first line
% merely continues the input to the second line
% (for convenient cut ant paste).
% Furthermore, the command |latex| can be replaced by any
% of its alternative versions such as |pdflatex|.
%
% %%%%%%%%%%%%%%%%%%%%%%%%%%%%%%%%%%%%%%%%%%%%%%%%%%%%%%%%%%%%%%%%%%%%%%%%%%%%%%
% %%%%%%%%%%%%%%%%%%%%%%%%%%%%%%%%%%%%%%%%%%%%%%%%%%%%%%%%%%%%%%%%%%%%%%%%%%%%%%
% \section{Implementation}
%\iffalse
%<*package>
%\fi
%
% This section describes the definitions file |childdoc.def|.

% The definitions cannot be loaded using |\usepackage| or |\RequirePackage|
% which has a mechanism to prevent loading a style file more than once.
% When loading the definitions by means of |\input|
% multiple instances have to be prevented manually:
%\iffalse
%This code needs to be before the `\ProvidesFile' directive
%which is defined at the beginning of this file.
%Therefore it is also placed there and commented out here.
%</package>
%<*discard>
%\fi
%    \begin{macrocode}
\ifdefined\childdocmain\endinput\fi
%    \end{macrocode}
%\iffalse
%</discard>
%<*package>
%\fi
%
% \macro{\ifchilddoc}
% \macro{\ifchilddocmanual}
% The conditional |\ifchilddoc| tells whether a
% child (true) or main (false) document is being compiled.
% The conditional |\ifchilddocmanual| tells whether
% the |\includeonly| mechanism is used (false) or
% the selection of child files must be performed manually (true).
% The definitions initialise to false:
%    \begin{macrocode}
\newif\ifchilddoc
\newif\ifchilddocmanual
%    \end{macrocode}

% \macro{\childdocname}
% \macro{\childdocjob}
% The macro |\childdocname| stores the name of the main document
% to be compiled. The macro |\childdocjob| stores the name of
% the document on which the \LaTeX{} compiler was originally invoked.
% The content of |\jobname| cannot be compared
% to filenames specified in the source due to different catcodes.
% The following code rescans |\jobname|, stores the result
% in |\childdocname| and saves a copy in |\childdocjob|:
%    \begin{macrocode}
\edef\childdocname{\scantokens\expandafter{\jobname\noexpand}}
\let\childdocjob\childdocname
%    \end{macrocode}

% \macro{\childdocdisable}
% The macro |\childdocdisable| prevents the main file
% from being processed more than once.
% At this stage, the main document command |\childdocmain|
% is assumed to be called once again where it should do nothing.
% Any subsequent call to it should prevent
% a secondary processing of the main document
% It overwrites the forwarding commands
% |\childdocof| and |\childdocforward|
% with empty macros to prevent further inclusions of the main document:
%    \begin{macrocode}
\newcommand{\childdocdisable}
{
  \renewcommand{\childdocmain}[1]{\renewcommand{\childdocmain}[1]{\endinput}}
  \renewcommand{\childdocof}[1]{}
  \renewcommand{\childdocby}[2][]{}
  \renewcommand{\childdocforward}[2][]{}
  \renewcommand{\childdocdisable}{}
}
%    \end{macrocode}

% \macro{\childdocmain}
% The macro |\childdocmain| is to be called at the top of the main file
% with nothing or the main filename (without extension) as argument.
% First, it breaks loops.
% If the argument is not empty and does not match |\childdocname|
% (which is set by the first inclusion of |childdoc.def|),
% |\ifchilddoc| is set to true, |\includeonly| is applied to the child file
% and |\jobname| is set to the main file
% (for proper handling of |.aux| files):
%    \begin{macrocode}
\newcommand{\childdocmain}[1]
{
  \childdocdisable\childdocmain{}
  \if?#1?\else
    \begingroup
      \def\childdoctmp{#1}
      \ifx\childdoctmp\childdocname
        \def\childdoctmp{}
      \else
        \def\childdoctmp
        {
          \childdoctrue
          \includeonly{\childdocname}
          \def\childdocjob{#1}
          \def\jobname{#1}
        }
      \fi
      \expandafter
    \endgroup
    \childdoctmp
  \fi
}
%    \end{macrocode}

% \macro{\childdocof}
% The command |\childdocof| redirects
% compilation to the main file |#1|.
%    \begin{macrocode}
\newcommand{\childdocof}[1]
{
  \childdocdisable
  \childdoctrue
  \includeonly{\childdocname}
  \def\jobname{#1}
  \def\childdocjob{#1}
  \input{#1}
}
%    \end{macrocode}

% \macro{\childdocby}
% The command |\childdocby| ....
%    \begin{macrocode}
\newcommand{\childdocby}[2][]
{
  \childdocdisable
  \childdoctrue
  \childdocmanualtrue
  \if?#1?\else
    \def\jobname{#2}
  \fi
  \def\childdocjob{#2}
  \input{#2}
  \endinput
}
%    \end{macrocode}

% \macro{\childdocforward}
% The command |\childdocforward| redirects
% compilation to the main file or
% (if the optional argument is given) a child file.
% Parameters are set as if the main file
% or a child file starting with |\childdocof| was compiled.
% Then compilation is handed over to the main file:
%    \begin{macrocode}
\newcommand{\childdocforward}[2][]
{
  \begingroup
    \if?#1?
      \def\childdoctmp
      {
        \def\childdocname{#2}
        \def\childdocjob{#2}
        \def\jobname{#2}
        \input{#2}
        \endinput
      }
    \else
      \def\childdoctmp
      {
        \childdocdisable
        \def\childdocname{#2}
        \childdoctrue
        \includeonly{#2}
        \def\childdocjob{#1}
        \def\jobname{#1}
        \input{#1}
        \endinput
      }
    \fi
    \expandafter
  \endgroup
  \childdoctmp
}
%    \end{macrocode}

% \macro{\childdocforwardprefix}
% The command |\childdocforwardprefix| redirects
% compilation to the main or a child file by means of a pattern.
% The prefix |#1| in the current filename is replaced by |#2|
% and the suffix of the current filename is kept
% (it is assumed that the filename does not contain the substring `|~~~|'
% which is used as a delimiter).
% Compilation is handed over to the new file by |\childdocforward|:
%    \begin{macrocode}
\newcommand{\childdocforwardprefix}[3][]
{
  \begingroup
    \def\childdocextract #2##1~~~{\def\childdoctmp{\childdocforward[#1]{#3##1}}}
    \expandafter\childdocextract\childdocname~~~
    \expandafter
  \endgroup
  \childdoctmp
}
%    \end{macrocode}

% \macro{\childdoc}
% The deprecated macro |\childdoc| is a legacy version of |\childdocmain|:
%    \begin{macrocode}
\newcommand{\childdoc}{\childdocmain}
%    \end{macrocode}

% \macro{\childdocredirect}
% The deprecated macro |\childdocredirect| is a legacy version
% of |\childdocforward| and |\childdocforwardprefix|:
%    \begin{macrocode}
\newcommand{\childdocredirect}[2][]
{
  \begingroup
    \if?#1?
      \def\childdoctmp{\childdocforward{#2}}
    \else
      \def\childdoctmp{\childdocforwardprefix{#1}{#2}}
    \fi
    \expandafter
  \endgroup
  \childdoctmp
}
%    \end{macrocode}

%\iffalse
%</package>
%\fi
%
\endinput
|\\
|\childdocforward[|\textit{main}|]{|\textit{dest}|}|\\
\end{tabular}
\end{center}
%
The argument \textit{dest} is the destination file
(without extension).
It should be the main file or one of the child files.
Note that further \textsf{childdoc} directives
such as |\childdocof| and |\childdocforward|
in the indicated file will be processed in this form.
The optional argument \textit{main}
passes on directly to the main file \textit{main}
while pretending to compile the child \textit{dest}.
This form behaves as if \textit{dest}
issues |\childdocof{|\textit{main}|}| right away,
and no further \textsf{childdoc} directives will be processed.

%%%%%%%%%%%%%%%%%%%%%%%%%%%%%%%%%%%%%%%%
\DescribeMacro{\...prefix}
In the alternative form |\childdocforwardprefix|,
%
\begin{center}
\begin{tabular}{l}
|% \iffalse
%
% childdoc.dtx Copyright (C) 2017-2018 Niklas Beisert
%
% This work may be distributed and/or modified under the
% conditions of the LaTeX Project Public License, either version 1.3
% of this license or (at your option) any later version.
% The latest version of this license is in
%   http://www.latex-project.org/lppl.txt
% and version 1.3 or later is part of all distributions of LaTeX
% version 2005/12/01 or later.
%
% This work has the LPPL maintenance status `maintained'.
%
% The Current Maintainer of this work is Niklas Beisert.
%
% This work consists of the files childdoc.dtx and childdoc.ins
% and the derived files childdoc.def and cdocsamp.tex with
% cdocsch1.tex, cdocsch2.tex, cdocsdrf.tex, cdocsfn1.tex, cdocsfn2.tex.
%
%<package>\ifdefined\childdocmain\endinput\fi
%<package>\ProvidesFile{childdoc.def}[2018/12/30 v2.0 child document driver]
%<samplemain>\ProvidesFile{cdocsamp.tex}[2018/12/30 v2.0 sample for childdoc]
%<*driver>
%\ProvidesFile{childdoc.drv}[2018/12/30 v2.0 childdoc reference manual file]
\PassOptionsToClass{10pt,a4paper}{article}
\documentclass{ltxdoc}

\usepackage[margin=35mm]{geometry}
\usepackage{hyperref}
\usepackage{hyperxmp}
\usepackage[usenames]{color}

\hypersetup{colorlinks=true}
\hypersetup{pdfstartview=FitH}
\hypersetup{pdfpagemode=UseNone}
\hypersetup{pdfsource={}}
\hypersetup{pdflang={en-UK}}
\hypersetup{pdfcopyright={Copyright 2017-2018 Niklas Beisert.
  This work may be distributed and/or modified under the
  conditions of the LaTeX Project Public License, either version 1.3
  of this license or (at your option) any later version.}}
\hypersetup{pdflicenseurl={http://www.latex-project.org/lppl.txt}}
\hypersetup{pdfcontactaddress={ETH Zurich, ITP, HIT K,
  Wolfgang-Pauli-Strasse 27}}
\hypersetup{pdfcontactpostcode={8093}}
\hypersetup{pdfcontactcity={Zurich}}
\hypersetup{pdfcontactcountry={Switzerland}}
\hypersetup{pdfcontactemail={nbeisert@itp.phys.ethz.ch}}
\hypersetup{pdfcontacturl={http://people.phys.ethz.ch/\xmptilde nbeisert/}}

\newcommand{\secref}[1]{\hyperref[#1]{section \ref*{#1}}}

\parskip1ex
\parindent0pt
\let\olditemize\itemize
\def\itemize{\olditemize\parskip0pt}

\begin{document}

\title{The \textsf{childdoc} Package}
\hypersetup{pdftitle={The childdoc Package}}
\author{Niklas Beisert\\[2ex]
  Institut f\"ur Theoretische Physik\\
  Eidgen\"ossische Technische Hochschule Z\"urich\\
  Wolfgang-Pauli-Strasse 27, 8093 Z\"urich, Switzerland\\[1ex]
  \href{mailto:nbeisert@itp.phys.ethz.ch}
  {\texttt{nbeisert@itp.phys.ethz.ch}}}
\hypersetup{pdfauthor={Niklas Beisert}}
\hypersetup{pdfsubject={Manual for the LaTeX2e Package childdoc}}
\date{30 December 2018, \textsf{v2.0}}
\maketitle

\begin{abstract}\noindent
\textsf{childdoc} is a \LaTeXe{} package
that enables the direct compilation
of document sections included by |\include|
to individual files.
\end{abstract}

\begingroup
\parskip0ex
\tableofcontents
\endgroup

%%%%%%%%%%%%%%%%%%%%%%%%%%%%%%%%%%%%%%%%%%%%%%%%%%%%%%%%%%%%%%%%%%%%%%%%%%%%%%%%
%%%%%%%%%%%%%%%%%%%%%%%%%%%%%%%%%%%%%%%%%%%%%%%%%%%%%%%%%%%%%%%%%%%%%%%%%%%%%%%%
\section{Introduction}

\LaTeX{} provides a mechanism to structure a large document (such as a book)
into a main file and several child files (containing the chapters)
using the |\include| command.
This mechanism is beneficial for documents
which span hundreds of pages in order to
make the source file(s) more manageable.
Moreover, compilation can be restricted to
selected child files by means of the |\includeonly| command.
The latter feature can be used to reduce the compilation time while editing
(this was significantly more useful in the earlier days of \LaTeX{})
or to generate a smaller document which is easier to navigate.
Another application of |\includeonly| is to generate
documents consisting of selected parts of the complete document.

However, there are a few drawbacks of the plain |\include| mechanism:
\begin{itemize}
\item
The child files cannot be compiled on their own,
they can only be compiled via the main file.
A naive editing environment
(such as a text editor with an option
to have the current file processed by \LaTeX)
may require one to switch to the main file before compiling;
attempting to compile the child file produces errors.
\item
The main file must be modified (each time)
to adjust the |\includeonly| command
to the present needs. This easily leaves the main file in a messy state.
\item
The generated document will always carry the filename
of the main document. This is inconvenient if
several child files are to be compiled and
to be kept for distribution.
\end{itemize}

The present package provides a simple interface
to make child files individually compilable by \LaTeX{}.
Compiling a child file then has the same effect as compiling
the main file with an |\includeonly| command
to select the appropriate child.
Moreover the generated document will carry the name of the child
rather than the main file.
This resolves all three above issues.

This feature is meant to make the editing of books,
thesis documents and lecture notes somewhat more convenient.
However, the package can also be used efficiently for
composing a series of documents (such as exercise sheets)
which are typically distributed individually.
It then assists the author in generating the individual documents
(potentially in different versions)
as well as a document containing the collected series.
Another application is in developing style files
or other kinds of included material
where compilation of the style file could redirect
to a sample or test file.

%%%%%%%%%%%%%%%%%%%%%%%%%%%%%%%%%%%%%%%%%%%%%%%%%%%%%%%%%%%%%%%%%%%%%%%%%%%%%%%%
%%%%%%%%%%%%%%%%%%%%%%%%%%%%%%%%%%%%%%%%%%%%%%%%%%%%%%%%%%%%%%%%%%%%%%%%%%%%%%%%
\section{Usage}

First of all, the package \textsf{childdoc} is \emph{not} a standard
\LaTeXe{} |.sty| style file! Therefore it needs to be invoked in
a non-standard way.

%%%%%%%%%%%%%%%%%%%%%%%%%%%%%%%%%%%%%%%%%%%%%%%%%%%%%%%%%%%%%%%%%%%%%%%%%%%%%%%%
\subsection{Included Files}
\label{sec:include}

%%%%%%%%%%%%%%%%%%%%%%%%%%%%%%%%%%%%%%%%
\DescribeMacro{\childdocmain}
To use the package, add the commands
\begin{center}
\begin{tabular}{l}
|\input{childdoc.def}|\\
|\childdocmain{}|\\
\end{tabular}
\end{center}
at the very top of the main \LaTeX{} file,
in particular \emph{before} the |\documentclass| statement!
The argument of |\childdocmain| should be left empty
(but it must be present).

%%%%%%%%%%%%%%%%%%%%%%%%%%%%%%%%%%%%%%%%
\DescribeMacro{\childdocof}
Furthermore, add the commands
\begin{center}
\begin{tabular}{l}
|\input{childdoc.def}|\\
|\childdocof{|\textit{main}|}|\\
\end{tabular}
\end{center}
at the top of every child file \textit{child}
which is included by |\include{|\textit{child}|}|
from within the main file
(or at least for those files to be compiled individually).
The argument \textit{main} must be the filename of the main file.

There are a couple of
considerations in setting up the main and child documents:

%%%%%%%%%%%%%%%%%%%%%%%%%%%%%%%%%%%%%%%%
\paragraph{Restrictions.}

Please note the following restrictions:
\begin{itemize}
\item
|\childdocmain| must be called with one argument \textit{main}
to ensure compatibility with earlier version of the package.
It must either be empty (|\childdocmain{}|)
or precisely match the filename of the main file in which it is specified.
See \secref{sec:detection} for further information.
\item
The filename \textit{main} must be specified without the |.tex| extension.
\item
The filename \textit{main} is case sensitive
(even in case-insensitive file systems)
due to internal string comparison.
\item
The argument \textit{main} should be fully expanded, it cannot be a macro.
\item
Subdirectories and special characters should be avoided in filenames.
\item
The command |\childdocmain{|\textit{main}|}| must be followed by a whitespace.
It should not be followed immediately by another command
or by a comment mark `|%|'.
This is because the \TeX{} parser reads the token immediately following
the argument of |\childdocmain| and puts it
at the beginning of every child section;
however, a white\-space is ignored.
\end{itemize}

%%%%%%%%%%%%%%%%%%%%%%%%%%%%%%%%%%%%%%%%
\paragraph{Content of Main File.}

It is advisable to place all content in the child files included by |\include|.
Any output contained in the main file will appear in all child documents
unless suppressed manually;
it cannot be suppressed automatically by the |\includeonly| directive
and thus should normally be avoided.
A method to include some content in the main file
by means of conditional processing is described in \secref{sec:conditional}.

%%%%%%%%%%%%%%%%%%%%%%%%%%%%%%%%%%%%%%%%
\paragraph{Page Numbering.}

When only a part of the document is compiled,
the appropriate numbering of pages
(as well as other status parameters)
is determined from the |.aux| files.
The latter contain information from previous passes.
However this information needs to propagate through
all intermediate child documents.
Therefore the page numbering in child documents may well
be inconsistent until the complete document is compiled at least once.

A useful (if unconventional) way to always ensure a consistent
page numbering is to restart the numbering in each child document
and denote the pages by `\textit{child}|.|\textit{page}'
where \textit{child} represents the chapter/section number of the child file.
This can be achieved by the command
|\numberwithin{page}{|\textit{child}|}|
of the \textsf{amsmath} package
where \textit{child} can be |chapter| or |section|
depending on the chosen structuring.
Alternatively, one can modify the macro |\thepage| appropriately
and reset the counter |page| at the start of each child file.

%%%%%%%%%%%%%%%%%%%%%%%%%%%%%%%%%%%%%%%%%%%%%%%%%%%%%%%%%%%%%%%%%%%%%%%%%%%%%%%%
\subsection{Conditional Processing}
\label{sec:conditional}

The package provides a mechanism to compile different versions
of a document. To customise the versions further some conditional processing
can come in handy to distinguish which version is being compiled.
The package provides two macros to describe the compilation context:

%%%%%%%%%%%%%%%%%%%%%%%%%%%%%%%%%%%%%%%%
\DescribeMacro{\ifchilddoc}
The conditional |\ifchilddoc| distinguishes between the compilation of
child documents and the main document:
%
\begin{center}
|\ifchilddoc |\textit{child-code}| |[|\||else |\textit{main-code}]| \||fi|
\end{center}

%%%%%%%%%%%%%%%%%%%%%%%%%%%%%%%%%%%%%%%%
\DescribeMacro{\childdocname}
\DescribeMacro{\childdocjob}
The macro |\childdocname| contains the filename (without extension)
of the main or child file being processed.
Note that |\childdocjob| will always contain the name of the main file.

%%%%%%%%%%%%%%%%%%%%%%%%%%%%%%%%%%%%%%%%
\paragraph{Title Page.}

Conditional processing can be used to include a title or banner page
in the main document when proper precautions are taken.
Importantly, the code in the main file should ensure that the page counter
(as well as other status parameters which are stored in the |.aux| files)
takes the same value after the conditional processing.
Otherwise the page numbers may take divergent values
depending on which part is compiled.

For example, a title page could be declared by:
%
\begin{center}
\begin{tabular}{l}
|\ifchilddoc\||else|\\
|\addtocounter{page}{-1}|\\
\textit{code for title page}\\
|\newpage|\\
|\||fi|
\end{tabular}
\end{center}
%
A banner page for the child documents can be generated by:
%
\begin{center}
\begin{tabular}{l}
|\ifchilddoc|\\
|\addtocounter{page}{-1}|\\
\textit{code for banner page}\\
|\newpage|\\
|\||fi|
\end{tabular}
\end{center}
%
Here one could write a message such as:
\begin{center}
|This is the part \childdocname{} of \childdocjob{}.|
\end{center}

%%%%%%%%%%%%%%%%%%%%%%%%%%%%%%%%%%%%%%%%%%%%%%%%%%%%%%%%%%%%%%%%%%%%%%%%%%%%%%%%
\subsection{Flags}
\label{sec:flags}

The package makes it easy to generate different versions
of the main or child documents.
To this end compilation flags can be defined
and assigned different default values.
They will be particularly useful in conjunction
with the forwarding mechanism described in \secref{sec:forward}.

For example, it may be useful to have a flag |\version|
which can be set to |draft| or |final|.
The document source will contain some conditional code
depending on the value of |\version|.
Suppose further, the flag should default to |final| for the main file
and to |draft| for child files
which is a natural assignment for editing the document.
This is achieved by placing the following code
in the preamble of the main document
(below the |\childdocmain| directive):
%
\begin{center}
\begin{tabular}{l}
|\ifchilddoc|\\
|\providecommand{\version}{draft}|\\
|\||else|\\
|\providecommand{\version}{final}|\\
|\||fi|
\end{tabular}
\end{center}
%
The definition by |\providecommand| makes sure
that previous definitions are not overwritten.
Further statements |\providecommand{\version}{...}|
can thus be added before the above code to override it.

For the main file, one might add a line
(between |\childdocmain| and the above block)
%
\begin{center}
|%\ifchilddoc\||else\providecommand{\version}{draft}\||fi|
\end{center}
%
which can be uncommented to produce a draft version.
Likewise one can add a line to the very top of a child file
(above the |\childdocof{|\textit{main}|}| directive)
%
\begin{center}
|%\providecommand{\version}{final}|
\end{center}
%
which can be uncommented to produce the final version of this child document.

%%%%%%%%%%%%%%%%%%%%%%%%%%%%%%%%%%%%%%%%%%%%%%%%%%%%%%%%%%%%%%%%%%%%%%%%%%%%%%%%
\subsection{Forwarding}
\label{sec:forward}

Different versions of the main or child documents
using compilation flags as described in \secref{sec:flags}
can be (permanently) stored in different files
for convenient compilation, viewing and distribution.
To this end, the package defines a command
to pass on compilation to a different file:

%%%%%%%%%%%%%%%%%%%%%%%%%%%%%%%%%%%%%%%%
\DescribeMacro{\childdocforward}
The command |\childdocforward| redirects processing to
another source file:
%
\begin{center}
\begin{tabular}{l}
|\input{childdoc.def}|\\
|\childdocforward[|\textit{main}|]{|\textit{dest}|}|\\
\end{tabular}
\end{center}
%
The argument \textit{dest} is the destination file
(without extension).
It should be the main file or one of the child files.
Note that further \textsf{childdoc} directives
such as |\childdocof| and |\childdocforward|
in the indicated file will be processed in this form.
The optional argument \textit{main}
passes on directly to the main file \textit{main}
while pretending to compile the child \textit{dest}.
This form behaves as if \textit{dest}
issues |\childdocof{|\textit{main}|}| right away,
and no further \textsf{childdoc} directives will be processed.

%%%%%%%%%%%%%%%%%%%%%%%%%%%%%%%%%%%%%%%%
\DescribeMacro{\...prefix}
In the alternative form |\childdocforwardprefix|,
%
\begin{center}
\begin{tabular}{l}
|\input{childdoc.def}|\\
|\childdocforwardprefix[|\textit{main}|]{|\textit{prefix}|}{|\textit{dest}|}|
\end{tabular}
\end{center}
%
the destination file is determined by a pattern
depending on the current file:
To make this work, the current file must be called
`{\textit{prefix}\hspace{0.2em}\textit{suffix}}'
with \textit{prefix} matching precisely the argument.
Processing is then passed on to the file
`{\textit{dest}\hspace{0.2em}\textit{suffix}}'.
Surely, the same effect is achieved by
directly specifying the
argument `{\textit{dest}\hspace{0.2em}\textit{suffix}}'
in the first form.
However, that requires to set up a different file
for each child. With the alternative form of the command
all these files can have exactly the same content
which simplifies setting them up and maintaining them.

For example, the following file |draft.tex|
with a compilation flag |\version| as described in \secref{sec:flags}
compiles the main document as a draft:
%
\begin{center}
\begin{tabular}{l}
|\def\version{draft}|\\
|\input{childdoc.def}|\\
|\childdocforward{|\textit{main}|}|
\end{tabular}
\end{center}
%
Likewise, the following files |final|\textit{nn}|.tex|
compile the final version of the child document
|child|\textit{nn}|.tex|:
%
\begin{center}
\begin{tabular}{l}
|\def\version{final}|\\
|\input{childdoc.def}|\\
|\childdocforwardprefix{final}{child}|
\end{tabular}
\end{center}
%

Note that when several versions of a main file and/or of each child file
are to be generated, it may be convenient to set up a |Makefile| or
shell script to automatise the process.

%%%%%%%%%%%%%%%%%%%%%%%%%%%%%%%%%%%%%%%%%%%%%%%%%%%%%%%%%%%%%%%%%%%%%%%%%%%%%%%%
\subsection{Command Line Processing}
\label{sec:commandline}

The effect of redirection files can also be achieved by invoking
the \LaTeX{} compiler with a more elaborate command line.
Most conveniently this should be done as part
of a shell script or a |Makefile|.

When using \textsf{childdoc} in the main file, the following
command lines effectively perform a redirection
(note that depending on the shell being used,
backslashes may have to be doubled: `|\|' $\to$ `|\\|'):
%
\begin{center}
|... -jobname "|\textit{target}|" |\\|"|[\textit{flags}]%
|\input{childdoc.def}\childdocforward[|\textit{main}|]{|\textit{dest}|}"|
\end{center}
%
Here \textit{target} is the name of the output file,
\textit{main} is the name of the main file
and \textit{dest} is the name of the main or child file to be processed
(all filenames without extensions).
The optional argument \textit{main} can be omitted
if \textit{main} matches \textit{dest}.
Optionally, compilation \textit{flags} can be defined via |\def| commands.
This command line makes the \TeX{} engine believe
it is compiling the file \textit{target}
whose content is specified as the latter parameter.
The provided code then forwards the processing to
\textit{main} or \textit{dest} as described in \secref{sec:forward}.

%%%%%%%%%%%%%%%%%%%%%%%%%%%%%%%%%%%%%%%%%%%%%%%%%%%%%%%%%%%%%%%%%%%%%%%%%%%%%%%%
\subsection{Include by Input}
\label{sec:input}

Including child documents by |\include| has some restrictions by design.
Most notably, the content of a child document always occupies
its own set of pages; pages cannot be shared between child documents.
Usually, this behaviour makes perfect sense
because each child document contain an essential part of the document.
However, in some situations it may be desirable to compose
a document from a collection of parts
without having mandatory page breaks between then.
For this case, the package
provides a mechanism to include parts
by |\input| which can also be processed individually.
However, by construction this mechanism
requires manual handling of the content to be output.

%%%%%%%%%%%%%%%%%%%%%%%%%%%%%%%%%%%%%%%%
\DescribeMacro{\ifchilddocmanual}
The main file should be prepared as usual, see \secref{sec:include}.
However, the document body must make a distinction
between processing of an individual part and of the main document, e.g.:
%
\begin{center}
\begin{tabular}{l}
|\ifchilddocmanual|\\
|\input{\childdocname}|\\
|\||else|\\
\textit{document body with }|\input{|\textit{part}|}|\\
|\||fi|
\end{tabular}
\end{center}
%
The conditional |\ifchilddocmanual| is true whenever
a part to be included by |\input| is being compiled,
and the name of the part is stored in |\childdocname|.

%%%%%%%%%%%%%%%%%%%%%%%%%%%%%%%%%%%%%%%%
\DescribeMacro{\childdocby}
Each part to be included by |\input| should start with:
%
\begin{center}
\begin{tabular}{l}
|\input{childdoc.def}|\\
|\childdocby{|\textit{main}|}|\\
\end{tabular}
\end{center}
%
The directive |\childdocby| is similar to |\childdocof|
described in \secref{sec:include},
but the subsequent selection of content must be done manually.
To that end, both |\ifchilddoc| and |\ifchilddocmanual|
will be true upon processing of a part,
and the name of the part is stored in |\childdocname|.
Note that |\jobname| will be set to the filename of the current part
so that each part receives an individual |.aux| file
that does not interfere with the |.aux| file(s) of the main document.
This behaviour can be altered by the alternative form
|\childdocby[*]{|\textit{main}|}| (with a non-empty optional argument)
which uses the |.aux| file of the main document
by setting |\jobname| to \textit{main}.

%%%%%%%%%%%%%%%%%%%%%%%%%%%%%%%%%%%%%%%%%%%%%%%%%%%%%%%%%%%%%%%%%%%%%%%%%%%%%%%%
\subsection{Driver Development}
\label{sec:driver}

The \textsf{childdoc} mechanism can also be use for the development
of definition files such as \LaTeX{} styles or classes.
This case differs from the above setup with multiple parts
included by |\include| in that no |\includeonly| should be invoked.
This can be achieved by starting the include file
(before |\ProvidesPackage|) with:
%
\begin{center}
\begin{tabular}{l}
|\input{childdoc.def}|\\
|\childdocforward{|\textit{main}|}|\\
\end{tabular}
\end{center}
%
or alternatively with:
%
\begin{center}
\begin{tabular}{l}
|\input{childdoc.def}|\\
|\childdocby{|\textit{main}|}|\\
\end{tabular}
\end{center}
%
Both forms have slightly different effects as described above.
The main file is prepared as usual, see \secref{sec:include}.

%%%%%%%%%%%%%%%%%%%%%%%%%%%%%%%%%%%%%%%%%%%%%%%%%%%%%%%%%%%%%%%%%%%%%%%%%%%%%%%%
\subsection{Legacy Detection}
\label{sec:detection}

The directive |\childdocmain| in the main file can detect
whether the complete document or merely a child is to be compiled
even without using the directive |\childdocof|.
This method is deprecated because it is less robust
and there is no compelling reason to use it;
it is merely provided for backward compatibility
and it may be removed in future versions.

If the detection mechanism is to be used,
it is mandatory to correctly specify
the filename of the main file as the argument of |\childdocmain|:
%
\begin{center}
\begin{tabular}{l}
|\input{childdoc.def}|\\
|\childdocmain{|\textit{main}|}|\\
\end{tabular}
\end{center}
%
If |\jobname| does not match the argument \textit{main} of |\childdocmain|,
it is assumed that |\jobname| points to the child file to be compiled.
When using |\childdocmain| with the main file specified as argument,
it suffices to start a child file
with just |\input{|\textit{main}|}|
without loading of the package and using |\childdocof|.
If instead all processing is done
with the appropriate \textsf{childdoc} directives,
the argument of \textit{main} of |\childdocmain| can be empty.

An alternative version of the command line processing described
in \secref{sec:commandline} using the detection mechanism reads:
%
\begin{center}
|... -jobname "|\textit{target}|" "|[\textit{flags}]%
[|\def\jobname{|\textit{dest}|}|]|\input{|\textit{main}|}"|
\end{center}

%%%%%%%%%%%%%%%%%%%%%%%%%%%%%%%%%%%%%%%%%%%%%%%%%%%%%%%%%%%%%%%%%%%%%%%%%%%%%%%%
\subsection{Manual Code}
\label{sec:manual}

In case one cannot be certain whether the definitions file |childdoc.def|
is installed on the target \TeX{} distribution
and one prefers not to ship it,
it is conceivable to paste a few relevant commands into the sources.

To that end, drop all statements |\input{childdoc.def}|
and perform the replacements as outlined below.
Instead of |\childdocmain{|\textit{main}|}| add the following code
to the top of the main file:
%
\begin{center}
\begin{tabular}{l}
|\||ifdefined\childdocname\endinput\||fi\newif\ifchilddoc|\\
|\edef\childdocname{\scantokens\expandafter{\jobname\noexpand}}|\\
|\def\childdocmain{|\textit{main}|}\||ifx\childdocmain\childdocname\||else|\\
|\childdoctrue\includeonly{\childdocname}\let\jobname\childdocmain\||fi|\\
\end{tabular}
\end{center}
%
Instead of |\childdocof{|\textit{main}|}| just include the main file
at the top of each child file:
%
\begin{center}
|\input{|\textit{main}|}|
\end{center}
%
A simple redirection |\childdocforward{|\textit{dest}|}| is achieved by:
%
\begin{center}
|\def\jobname{|\textit{dest}|}\input{\jobname}|
\end{center}
%
The redirection with prefix
|\childdocforwardprefix[|\textit{prefix}|]{|\textit{dest}|}|
is accomplished by:
%
\begin{center}
\begin{tabular}{l}
|{\edef\jobname{\scantokens\expandafter{\jobname\noexpand}}|\\
|\def\redirectjob |\textit{prefix}|#1~~~{\gdef\jobname{|\textit{dest}|#1}}|\\
|\expandafter\redirectjob\jobname~~~}\input{\jobname}|
\end{tabular}
\end{center}

In an alternative approach,
child documents can be compiled by a specific command line
without additional code or specific definitions:
%
\begin{center}
|... -jobname "|\textit{target}|" "|[\textit{flags}]%
|\includeonly{|\textit{dest}|}\input{|\textit{main}|}"|
\end{center}
%

%%%%%%%%%%%%%%%%%%%%%%%%%%%%%%%%%%%%%%%%%%%%%%%%%%%%%%%%%%%%%%%%%%%%%%%%%%%%%%%%
%%%%%%%%%%%%%%%%%%%%%%%%%%%%%%%%%%%%%%%%%%%%%%%%%%%%%%%%%%%%%%%%%%%%%%%%%%%%%%%%
\section{Information}

%%%%%%%%%%%%%%%%%%%%%%%%%%%%%%%%%%%%%%%%%%%%%%%%%%%%%%%%%%%%%%%%%%%%%%%%%%%%%%%%
\subsection{Copyright}

Copyright \copyright{} 2017--2018 Niklas Beisert

This work may be distributed and/or modified under the
conditions of the \LaTeX{} Project Public License, either version 1.3
of this license or (at your option) any later version.
The latest version of this license is in
  \url{http://www.latex-project.org/lppl.txt}
and version 1.3 or later is part of all distributions of \LaTeX{}
version 2005/12/01 or later.

This work has the LPPL maintenance status `maintained'.

The Current Maintainer of this work is Niklas Beisert.

This work consists of the files |README.txt|, |childdoc.ins| and |childdoc.dtx|
as well as the derived files |childdoc.def|, |cdocsamp.tex|
with |cdocsch1.tex|, |cdocsch2.tex|, |cdocspt3.tex|, |cdocspt4.tex|,
|cdocsdrf.tex|, |cdocsfn1.tex|, |cdocsfn2.tex|
as well as |childdoc.pdf|.

%%%%%%%%%%%%%%%%%%%%%%%%%%%%%%%%%%%%%%%%%%%%%%%%%%%%%%%%%%%%%%%%%%%%%%%%%%%%%%%%
\subsection{Files and Installation}

The package consists of the files:
%
\begin{center}
\begin{tabular}{ll}
    |README.txt|   & readme file \\
    |childdoc.ins| & installation file \\
    |childdoc.dtx| & source file \\
    |childdoc.def| & definition file \\
    |cdocsamp.tex| & sample main file \\
    |cdocsch1.tex| & sample include file \\
    |cdocsch2.tex| & sample include file \\
    |cdocspt3.tex| & sample part file \\
    |cdocspt4.tex| & sample part file \\
    |cdocsdrf.tex| & sample redirection file \\
    |cdocsfn1.tex| & sample redirection file \\
    |cdocsfn2.tex| & sample redirection file \\
    |childdoc.pdf| & manual
\end{tabular}
\end{center}
%
The distribution consists of the files
|README.txt|, |childdoc.ins| and |childdoc.dtx|.
%
\begin{itemize}
\item
Run (pdf)\LaTeX{} on |childdoc.dtx|
to compile the manual |childdoc.pdf| (this file).
\item
Run \LaTeX{} on |childdoc.ins| to create the definitions file |childdoc.def|
and the sample |cdocsamp.tex| with include files
|cdocsch1.tex|, |cdocsch2.tex|, |cdocspt3.tex|, |cdocspt4.tex|,
|cdocsdrf.tex|, |cdocsfn1.tex|, |cdocsfn2.tex|.
Then copy the file |childdoc.def| to an appropriate directory of your \LaTeX{}
distribution, e.g.\ \textit{texmf-root}|/tex/latex/childdoc|.
\end{itemize}

%%%%%%%%%%%%%%%%%%%%%%%%%%%%%%%%%%%%%%%%%%%%%%%%%%%%%%%%%%%%%%%%%%%%%%%%%%%%%%%%
\subsection{Related CTAN Packages}

There are several other packages which offer a similar functionality:
%
\begin{itemize}
\item
The packages
\href{http://ctan.org/pkg/docmute}{\textsf{docmute}},
\href{http://ctan.org/pkg/includex}{\textsf{includex}} and
\href{http://ctan.org/pkg/standalone}{\textsf{standalone}}
provide commands to include only the document body of
a child file thus allowing both files to be compiled individually.
\item
The packages \href{http://ctan.org/pkg/subdocs}{\textsf{subdocs}}
and \href{http://ctan.org/pkg/subfiles}{\textsf{subfiles}}
provide structures in which the main and child documents can be
encapsulated and allowing them to be compiled individually.
The inclusion mechanism is different from the conventional |\include|.
\item
The package \href{http://ctan.org/pkg/combine}{\textsf{combine}}
is an elaborate solution to combine several documents into one.
\end{itemize}
%
See also the CTAN topic \href{http://ctan.org/topic/subdocs}{\textsf{subdocs}}
for further related packages.
The present package differs from the above solutions in that
a document structure constructed with the conventional |\include| mechanism
just needs two extra commands at the top of every file
such that all constituent files can be compiled individually.

%%%%%%%%%%%%%%%%%%%%%%%%%%%%%%%%%%%%%%%%%%%%%%%%%%%%%%%%%%%%%%%%%%%%%%%%%%%%%%%%
%\subsection{Feature Suggestions}
%
%The following is a list of features which may be useful for future
%versions of this package:
%%
%\begin{itemize}
%\item
%\ldots
%\end{itemize}

%%%%%%%%%%%%%%%%%%%%%%%%%%%%%%%%%%%%%%%%%%%%%%%%%%%%%%%%%%%%%%%%%%%%%%%%%%%%%%%%
\subsection{Revision History}

%%%%%%%%%%%%%%%%%%%%%%%%%%%%%%%%%%%%%%%%
\paragraph{v2.0:} 2018/12/30

\begin{itemize}
\item
immediate forward processing
\item
added |\childdocby| mechanism
\item
manual restructured
\end{itemize}

%%%%%%%%%%%%%%%%%%%%%%%%%%%%%%%%%%%%%%%%
\paragraph{v1.6:} 2018/01/17

\begin{itemize}
\item
application for development of include files
\item
corrections to manual
\end{itemize}

%%%%%%%%%%%%%%%%%%%%%%%%%%%%%%%%%%%%%%%%
\paragraph{v1.5:} 2017/05/21

\begin{itemize}
\item
more complete structuring introduced
\item
|\childdocof| introduced
\item
|\childdoc| renamed to |\childdocmain|
\item
|\childredirect| renamed to |\childdocforward| and |\childdocforwardprefix|
and functionality expanded
\end{itemize}

%%%%%%%%%%%%%%%%%%%%%%%%%%%%%%%%%%%%%%%%
\paragraph{v1.0:} 2017/04/27

\begin{itemize}
\item
manual and install package
\item
first version published on CTAN
\end{itemize}

%%%%%%%%%%%%%%%%%%%%%%%%%%%%%%%%%%%%%%%%
\paragraph{v0.6:} 2017/04/26

\begin{itemize}
\item
redirection mechanism added
\end{itemize}

%%%%%%%%%%%%%%%%%%%%%%%%%%%%%%%%%%%%%%%%
\paragraph{v0.5:} 2017/04/26

\begin{itemize}
\item
functionality in definition file
\end{itemize}


%%%%%%%%%%%%%%%%%%%%%%%%%%%%%%%%%%%%%%%%%%%%%%%%%%%%%%%%%%%%%%%%%%%%%%%%%%%%%%%%
%%%%%%%%%%%%%%%%%%%%%%%%%%%%%%%%%%%%%%%%%%%%%%%%%%%%%%%%%%%%%%%%%%%%%%%%%%%%%%%%
%%%%%%%%%%%%%%%%%%%%%%%%%%%%%%%%%%%%%%%%%%%%%%%%%%%%%%%%%%%%%%%%%%%%%%%%%%%%%%%%
\appendix

\settowidth\MacroIndent{\rmfamily\scriptsize 000\ }

 \DocInput{childdoc.dtx}

\end{document}
%</driver>
% \fi
%
% %%%%%%%%%%%%%%%%%%%%%%%%%%%%%%%%%%%%%%%%%%%%%%%%%%%%%%%%%%%%%%%%%%%%%%%%%%%%%%
% %%%%%%%%%%%%%%%%%%%%%%%%%%%%%%%%%%%%%%%%%%%%%%%%%%%%%%%%%%%%%%%%%%%%%%%%%%%%%%
% \section{Sample}
%\iffalse
%<*samplemain>
%\fi
%
% The following presents a sample document
% with two chapters, two parts, a title page,
% a compile flag as well as three forwarding files to set the flag.
% It consists of eight |.tex| files:
% \begin{center}
% \begin{tabular}{ll}
% |cdocsamp.tex|&main file\\
% |cdocsch1.tex|&include file for chapter 1\\
% |cdocsch2.tex|&include file for chapter 2\\
% |cdocspt3.tex|&include file for part 3\\
% |cdocspt4.tex|&include file for part 4\\
% |cdocsdrf.tex|&forwarding file for main file in draft mode\\
% |cdocsfi1.tex|&forwarding file for final version of chapter 1\\
% |cdocsfi2.tex|&forwarding file for final version of chapter 2\\
% \end{tabular}
% \end{center}
% Each of the eight files can be compiled directly by the \LaTeX{} compiler.
%
% %%%%%%%%%%%%%%%%%%%%%%%%%%%%%%%%%%%%%%
% \paragraph{Main File.}
%
% The main file is called |cdocsamp.tex|.
%
% Load the \textsf{childdoc} definitions and
% declare the filename for the main document:
%    \begin{macrocode}
\input{childdoc.def}
\childdocmain{}
%    \end{macrocode}

% Optional override for |\version| flag:
%    \begin{macrocode}
%%\ifchilddoc\else\providecommand{\version}{draft}\fi
%    \end{macrocode}

% Define the default values for the |\version| flag
% (|final| for the main file and |draft| for childs):
%    \begin{macrocode}
\ifchilddoc
\providecommand{\version}{draft}
\else
\providecommand{\version}{final}
\fi
%    \end{macrocode}

% Load the standard document class:
%    \begin{macrocode}
\documentclass[12pt]{article}
%    \end{macrocode}

% Start the document body:
%    \begin{macrocode}
\begin{document}
%    \end{macrocode}

% Declare a title page.
% Print title, part of document being processed and version flag:
%    \begin{macrocode}
\addtocounter{page}{-1}
\begin{center}
{\LARGE\bfseries{}childdoc example\par}
\vspace{1cm}
\ifchilddoc
\ifchilddocmanual part\else chapter\fi:
`\childdocname' of `\childdocjob'\par
\else
main document: `\childdocjob'\par
\fi
version: \version\par
\end{center}
\newpage
%    \end{macrocode}

% Manually include selected file,
% otherwise process as usual:
%    \begin{macrocode}
\ifchilddocmanual
\section*{part `\childdocname'}
\input{\childdocname}
\else
%    \end{macrocode}

% Include the two chapters:
%    \begin{macrocode}
\include{cdocsch1}
\include{cdocsch2}
%    \end{macrocode}

% Include the two parts unless only chapters should be displayed:
%    \begin{macrocode}
\ifchilddoc\else
\section{part three}
\input{cdocspt3}
\section{part four}
\input{cdocspt4}
\fi
%    \end{macrocode}

% Process as usual until here:
%    \begin{macrocode}
\fi
%    \end{macrocode}

% End of document body:
%    \begin{macrocode}
\end{document}
%    \end{macrocode}
%\iffalse
%</samplemain>
%\fi
%
% %%%%%%%%%%%%%%%%%%%%%%%%%%%%%%%%%%%%%%
% \paragraph{Chapter Include Files.}
%
% The include files are called |cdocsch1.tex| and |cdocsch2.tex|.
%
%\iffalse
%<*samplechap1|samplechap2>
%\fi

% Optional override for |\version| flag:
%    \begin{macrocode}
%%\providecommand{\version}{final}
%    \end{macrocode}

% Include the main document:
%    \begin{macrocode}
\input{childdoc.def}
\childdocof{cdocsamp}
%    \end{macrocode}

%\iffalse
%</samplechap1|samplechap2>
%\fi
%
%\iffalse
%<*samplechap1>
%\fi
% Some text for chapter 1:
%    \begin{macrocode}
\section{one}
some text in chapter one
%    \end{macrocode}

%\iffalse
%</samplechap1>
%\fi
% Some text for chapter 2:
%\iffalse
%<*samplechap2>
%\fi
%    \begin{macrocode}
\section{two}
more text in chapter two
%    \end{macrocode}

%\iffalse
%</samplechap2>
%\fi
%
% %%%%%%%%%%%%%%%%%%%%%%%%%%%%%%%%%%%%%%
% \paragraph{Part Include Files.}
%
% The include files are called |cdocspt3.tex| and |cdocspt4.tex|.
%
%\iffalse
%<*samplepart3|samplepart4>
%\fi

% Optional override for |\version| flag:
%    \begin{macrocode}
%%\providecommand{\version}{final}
%    \end{macrocode}

% Include the main document:
%    \begin{macrocode}
\input{childdoc.def}
\childdocby{cdocsamp}
%    \end{macrocode}

%\iffalse
%</samplepart3|samplepart4>
%\fi
%
%\iffalse
%<*samplepart3>
%\fi
% Some text for part 3:
%    \begin{macrocode}
some text in part three
%    \end{macrocode}

%\iffalse
%</samplepart3>
%\fi
% Some text for part 4:
%\iffalse
%<*samplepart4>
%\fi
%    \begin{macrocode}
more text in part four
%    \end{macrocode}

%\iffalse
%</samplepart4>
%\fi
%
% %%%%%%%%%%%%%%%%%%%%%%%%%%%%%%%%%%%%%%
% \paragraph{Forwarding for a Complete Draft.}
%
% The following forwarding file |cdocsdrf.tex|
% compiles the main document in draft mode:
%\iffalse
%<*sampledraft>
%\fi
%    \begin{macrocode}
\def\version{draft}
\input{childdoc.def}
\childdocforward{cdocsamp}
%    \end{macrocode}

%\iffalse
%</sampledraft>
%\fi
%
% %%%%%%%%%%%%%%%%%%%%%%%%%%%%%%%%%%%%%%
% \paragraph{Forwarding for Final Version of the Chapters.}
%
% The following forwarding files |cdocsfn1.tex| and |cdocsfn2.tex|
% (with identical content)
% compile the final versions of the child documents
% |cdocsch1.tex| and |cdocsch2.tex|, respectively:
%\iffalse
%<*samplefinal>
%\fi
%    \begin{macrocode}
\def\version{final}
\input{childdoc.def}
\childdocforwardprefix[cdocsamp]{cdocsfn}{cdocsch}
%    \end{macrocode}

%\iffalse
%</samplefinal>
%\fi
%
% %%%%%%%%%%%%%%%%%%%%%%%%%%%%%%%%%%%%%%
% \paragraph{Command Line Processing.}
%
% The following three command lines generate the output files
% |cdocscld|, |cdocscl1| and |cdocscl2|
% which should be identical to
% |cdocsdrf|, |cdocsch1| and |cdocsfn2|, respectively:
% \begin{center}
% \begin{tabular}{l}
% |latex -jobname cdocscld \|\\
% |  "\def\version{draft}\input{childdoc.def}\childdocforward{cdocsamp}"|\\
% |latex -jobname cdocscl1 \|\\
% |  "\input{childdoc.def}\childdocforward[cdocsamp]{cdocsch1}"|\\
% |latex -jobname cdocscl2 \|\\
% |  "\def\version{final}\input{childdoc.def}\childdocforward{cdocsch2}"|
% \end{tabular}
% \end{center}
% Note that the trailing backslash on each first line
% merely continues the input to the second line
% (for convenient cut ant paste).
% Furthermore, the command |latex| can be replaced by any
% of its alternative versions such as |pdflatex|.
%
% %%%%%%%%%%%%%%%%%%%%%%%%%%%%%%%%%%%%%%%%%%%%%%%%%%%%%%%%%%%%%%%%%%%%%%%%%%%%%%
% %%%%%%%%%%%%%%%%%%%%%%%%%%%%%%%%%%%%%%%%%%%%%%%%%%%%%%%%%%%%%%%%%%%%%%%%%%%%%%
% \section{Implementation}
%\iffalse
%<*package>
%\fi
%
% This section describes the definitions file |childdoc.def|.

% The definitions cannot be loaded using |\usepackage| or |\RequirePackage|
% which has a mechanism to prevent loading a style file more than once.
% When loading the definitions by means of |\input|
% multiple instances have to be prevented manually:
%\iffalse
%This code needs to be before the `\ProvidesFile' directive
%which is defined at the beginning of this file.
%Therefore it is also placed there and commented out here.
%</package>
%<*discard>
%\fi
%    \begin{macrocode}
\ifdefined\childdocmain\endinput\fi
%    \end{macrocode}
%\iffalse
%</discard>
%<*package>
%\fi
%
% \macro{\ifchilddoc}
% \macro{\ifchilddocmanual}
% The conditional |\ifchilddoc| tells whether a
% child (true) or main (false) document is being compiled.
% The conditional |\ifchilddocmanual| tells whether
% the |\includeonly| mechanism is used (false) or
% the selection of child files must be performed manually (true).
% The definitions initialise to false:
%    \begin{macrocode}
\newif\ifchilddoc
\newif\ifchilddocmanual
%    \end{macrocode}

% \macro{\childdocname}
% \macro{\childdocjob}
% The macro |\childdocname| stores the name of the main document
% to be compiled. The macro |\childdocjob| stores the name of
% the document on which the \LaTeX{} compiler was originally invoked.
% The content of |\jobname| cannot be compared
% to filenames specified in the source due to different catcodes.
% The following code rescans |\jobname|, stores the result
% in |\childdocname| and saves a copy in |\childdocjob|:
%    \begin{macrocode}
\edef\childdocname{\scantokens\expandafter{\jobname\noexpand}}
\let\childdocjob\childdocname
%    \end{macrocode}

% \macro{\childdocdisable}
% The macro |\childdocdisable| prevents the main file
% from being processed more than once.
% At this stage, the main document command |\childdocmain|
% is assumed to be called once again where it should do nothing.
% Any subsequent call to it should prevent
% a secondary processing of the main document
% It overwrites the forwarding commands
% |\childdocof| and |\childdocforward|
% with empty macros to prevent further inclusions of the main document:
%    \begin{macrocode}
\newcommand{\childdocdisable}
{
  \renewcommand{\childdocmain}[1]{\renewcommand{\childdocmain}[1]{\endinput}}
  \renewcommand{\childdocof}[1]{}
  \renewcommand{\childdocby}[2][]{}
  \renewcommand{\childdocforward}[2][]{}
  \renewcommand{\childdocdisable}{}
}
%    \end{macrocode}

% \macro{\childdocmain}
% The macro |\childdocmain| is to be called at the top of the main file
% with nothing or the main filename (without extension) as argument.
% First, it breaks loops.
% If the argument is not empty and does not match |\childdocname|
% (which is set by the first inclusion of |childdoc.def|),
% |\ifchilddoc| is set to true, |\includeonly| is applied to the child file
% and |\jobname| is set to the main file
% (for proper handling of |.aux| files):
%    \begin{macrocode}
\newcommand{\childdocmain}[1]
{
  \childdocdisable\childdocmain{}
  \if?#1?\else
    \begingroup
      \def\childdoctmp{#1}
      \ifx\childdoctmp\childdocname
        \def\childdoctmp{}
      \else
        \def\childdoctmp
        {
          \childdoctrue
          \includeonly{\childdocname}
          \def\childdocjob{#1}
          \def\jobname{#1}
        }
      \fi
      \expandafter
    \endgroup
    \childdoctmp
  \fi
}
%    \end{macrocode}

% \macro{\childdocof}
% The command |\childdocof| redirects
% compilation to the main file |#1|.
%    \begin{macrocode}
\newcommand{\childdocof}[1]
{
  \childdocdisable
  \childdoctrue
  \includeonly{\childdocname}
  \def\jobname{#1}
  \def\childdocjob{#1}
  \input{#1}
}
%    \end{macrocode}

% \macro{\childdocby}
% The command |\childdocby| ....
%    \begin{macrocode}
\newcommand{\childdocby}[2][]
{
  \childdocdisable
  \childdoctrue
  \childdocmanualtrue
  \if?#1?\else
    \def\jobname{#2}
  \fi
  \def\childdocjob{#2}
  \input{#2}
  \endinput
}
%    \end{macrocode}

% \macro{\childdocforward}
% The command |\childdocforward| redirects
% compilation to the main file or
% (if the optional argument is given) a child file.
% Parameters are set as if the main file
% or a child file starting with |\childdocof| was compiled.
% Then compilation is handed over to the main file:
%    \begin{macrocode}
\newcommand{\childdocforward}[2][]
{
  \begingroup
    \if?#1?
      \def\childdoctmp
      {
        \def\childdocname{#2}
        \def\childdocjob{#2}
        \def\jobname{#2}
        \input{#2}
        \endinput
      }
    \else
      \def\childdoctmp
      {
        \childdocdisable
        \def\childdocname{#2}
        \childdoctrue
        \includeonly{#2}
        \def\childdocjob{#1}
        \def\jobname{#1}
        \input{#1}
        \endinput
      }
    \fi
    \expandafter
  \endgroup
  \childdoctmp
}
%    \end{macrocode}

% \macro{\childdocforwardprefix}
% The command |\childdocforwardprefix| redirects
% compilation to the main or a child file by means of a pattern.
% The prefix |#1| in the current filename is replaced by |#2|
% and the suffix of the current filename is kept
% (it is assumed that the filename does not contain the substring `|~~~|'
% which is used as a delimiter).
% Compilation is handed over to the new file by |\childdocforward|:
%    \begin{macrocode}
\newcommand{\childdocforwardprefix}[3][]
{
  \begingroup
    \def\childdocextract #2##1~~~{\def\childdoctmp{\childdocforward[#1]{#3##1}}}
    \expandafter\childdocextract\childdocname~~~
    \expandafter
  \endgroup
  \childdoctmp
}
%    \end{macrocode}

% \macro{\childdoc}
% The deprecated macro |\childdoc| is a legacy version of |\childdocmain|:
%    \begin{macrocode}
\newcommand{\childdoc}{\childdocmain}
%    \end{macrocode}

% \macro{\childdocredirect}
% The deprecated macro |\childdocredirect| is a legacy version
% of |\childdocforward| and |\childdocforwardprefix|:
%    \begin{macrocode}
\newcommand{\childdocredirect}[2][]
{
  \begingroup
    \if?#1?
      \def\childdoctmp{\childdocforward{#2}}
    \else
      \def\childdoctmp{\childdocforwardprefix{#1}{#2}}
    \fi
    \expandafter
  \endgroup
  \childdoctmp
}
%    \end{macrocode}

%\iffalse
%</package>
%\fi
%
\endinput
|\\
|\childdocforwardprefix[|\textit{main}|]{|\textit{prefix}|}{|\textit{dest}|}|
\end{tabular}
\end{center}
%
the destination file is determined by a pattern
depending on the current file:
To make this work, the current file must be called
`{\textit{prefix}\hspace{0.2em}\textit{suffix}}'
with \textit{prefix} matching precisely the argument.
Processing is then passed on to the file
`{\textit{dest}\hspace{0.2em}\textit{suffix}}'.
Surely, the same effect is achieved by
directly specifying the
argument `{\textit{dest}\hspace{0.2em}\textit{suffix}}'
in the first form.
However, that requires to set up a different file
for each child. With the alternative form of the command
all these files can have exactly the same content
which simplifies setting them up and maintaining them.

For example, the following file |draft.tex|
with a compilation flag |\version| as described in \secref{sec:flags}
compiles the main document as a draft:
%
\begin{center}
\begin{tabular}{l}
|\def\version{draft}|\\
|% \iffalse
%
% childdoc.dtx Copyright (C) 2017-2018 Niklas Beisert
%
% This work may be distributed and/or modified under the
% conditions of the LaTeX Project Public License, either version 1.3
% of this license or (at your option) any later version.
% The latest version of this license is in
%   http://www.latex-project.org/lppl.txt
% and version 1.3 or later is part of all distributions of LaTeX
% version 2005/12/01 or later.
%
% This work has the LPPL maintenance status `maintained'.
%
% The Current Maintainer of this work is Niklas Beisert.
%
% This work consists of the files childdoc.dtx and childdoc.ins
% and the derived files childdoc.def and cdocsamp.tex with
% cdocsch1.tex, cdocsch2.tex, cdocsdrf.tex, cdocsfn1.tex, cdocsfn2.tex.
%
%<package>\ifdefined\childdocmain\endinput\fi
%<package>\ProvidesFile{childdoc.def}[2018/12/30 v2.0 child document driver]
%<samplemain>\ProvidesFile{cdocsamp.tex}[2018/12/30 v2.0 sample for childdoc]
%<*driver>
%\ProvidesFile{childdoc.drv}[2018/12/30 v2.0 childdoc reference manual file]
\PassOptionsToClass{10pt,a4paper}{article}
\documentclass{ltxdoc}

\usepackage[margin=35mm]{geometry}
\usepackage{hyperref}
\usepackage{hyperxmp}
\usepackage[usenames]{color}

\hypersetup{colorlinks=true}
\hypersetup{pdfstartview=FitH}
\hypersetup{pdfpagemode=UseNone}
\hypersetup{pdfsource={}}
\hypersetup{pdflang={en-UK}}
\hypersetup{pdfcopyright={Copyright 2017-2018 Niklas Beisert.
  This work may be distributed and/or modified under the
  conditions of the LaTeX Project Public License, either version 1.3
  of this license or (at your option) any later version.}}
\hypersetup{pdflicenseurl={http://www.latex-project.org/lppl.txt}}
\hypersetup{pdfcontactaddress={ETH Zurich, ITP, HIT K,
  Wolfgang-Pauli-Strasse 27}}
\hypersetup{pdfcontactpostcode={8093}}
\hypersetup{pdfcontactcity={Zurich}}
\hypersetup{pdfcontactcountry={Switzerland}}
\hypersetup{pdfcontactemail={nbeisert@itp.phys.ethz.ch}}
\hypersetup{pdfcontacturl={http://people.phys.ethz.ch/\xmptilde nbeisert/}}

\newcommand{\secref}[1]{\hyperref[#1]{section \ref*{#1}}}

\parskip1ex
\parindent0pt
\let\olditemize\itemize
\def\itemize{\olditemize\parskip0pt}

\begin{document}

\title{The \textsf{childdoc} Package}
\hypersetup{pdftitle={The childdoc Package}}
\author{Niklas Beisert\\[2ex]
  Institut f\"ur Theoretische Physik\\
  Eidgen\"ossische Technische Hochschule Z\"urich\\
  Wolfgang-Pauli-Strasse 27, 8093 Z\"urich, Switzerland\\[1ex]
  \href{mailto:nbeisert@itp.phys.ethz.ch}
  {\texttt{nbeisert@itp.phys.ethz.ch}}}
\hypersetup{pdfauthor={Niklas Beisert}}
\hypersetup{pdfsubject={Manual for the LaTeX2e Package childdoc}}
\date{30 December 2018, \textsf{v2.0}}
\maketitle

\begin{abstract}\noindent
\textsf{childdoc} is a \LaTeXe{} package
that enables the direct compilation
of document sections included by |\include|
to individual files.
\end{abstract}

\begingroup
\parskip0ex
\tableofcontents
\endgroup

%%%%%%%%%%%%%%%%%%%%%%%%%%%%%%%%%%%%%%%%%%%%%%%%%%%%%%%%%%%%%%%%%%%%%%%%%%%%%%%%
%%%%%%%%%%%%%%%%%%%%%%%%%%%%%%%%%%%%%%%%%%%%%%%%%%%%%%%%%%%%%%%%%%%%%%%%%%%%%%%%
\section{Introduction}

\LaTeX{} provides a mechanism to structure a large document (such as a book)
into a main file and several child files (containing the chapters)
using the |\include| command.
This mechanism is beneficial for documents
which span hundreds of pages in order to
make the source file(s) more manageable.
Moreover, compilation can be restricted to
selected child files by means of the |\includeonly| command.
The latter feature can be used to reduce the compilation time while editing
(this was significantly more useful in the earlier days of \LaTeX{})
or to generate a smaller document which is easier to navigate.
Another application of |\includeonly| is to generate
documents consisting of selected parts of the complete document.

However, there are a few drawbacks of the plain |\include| mechanism:
\begin{itemize}
\item
The child files cannot be compiled on their own,
they can only be compiled via the main file.
A naive editing environment
(such as a text editor with an option
to have the current file processed by \LaTeX)
may require one to switch to the main file before compiling;
attempting to compile the child file produces errors.
\item
The main file must be modified (each time)
to adjust the |\includeonly| command
to the present needs. This easily leaves the main file in a messy state.
\item
The generated document will always carry the filename
of the main document. This is inconvenient if
several child files are to be compiled and
to be kept for distribution.
\end{itemize}

The present package provides a simple interface
to make child files individually compilable by \LaTeX{}.
Compiling a child file then has the same effect as compiling
the main file with an |\includeonly| command
to select the appropriate child.
Moreover the generated document will carry the name of the child
rather than the main file.
This resolves all three above issues.

This feature is meant to make the editing of books,
thesis documents and lecture notes somewhat more convenient.
However, the package can also be used efficiently for
composing a series of documents (such as exercise sheets)
which are typically distributed individually.
It then assists the author in generating the individual documents
(potentially in different versions)
as well as a document containing the collected series.
Another application is in developing style files
or other kinds of included material
where compilation of the style file could redirect
to a sample or test file.

%%%%%%%%%%%%%%%%%%%%%%%%%%%%%%%%%%%%%%%%%%%%%%%%%%%%%%%%%%%%%%%%%%%%%%%%%%%%%%%%
%%%%%%%%%%%%%%%%%%%%%%%%%%%%%%%%%%%%%%%%%%%%%%%%%%%%%%%%%%%%%%%%%%%%%%%%%%%%%%%%
\section{Usage}

First of all, the package \textsf{childdoc} is \emph{not} a standard
\LaTeXe{} |.sty| style file! Therefore it needs to be invoked in
a non-standard way.

%%%%%%%%%%%%%%%%%%%%%%%%%%%%%%%%%%%%%%%%%%%%%%%%%%%%%%%%%%%%%%%%%%%%%%%%%%%%%%%%
\subsection{Included Files}
\label{sec:include}

%%%%%%%%%%%%%%%%%%%%%%%%%%%%%%%%%%%%%%%%
\DescribeMacro{\childdocmain}
To use the package, add the commands
\begin{center}
\begin{tabular}{l}
|\input{childdoc.def}|\\
|\childdocmain{}|\\
\end{tabular}
\end{center}
at the very top of the main \LaTeX{} file,
in particular \emph{before} the |\documentclass| statement!
The argument of |\childdocmain| should be left empty
(but it must be present).

%%%%%%%%%%%%%%%%%%%%%%%%%%%%%%%%%%%%%%%%
\DescribeMacro{\childdocof}
Furthermore, add the commands
\begin{center}
\begin{tabular}{l}
|\input{childdoc.def}|\\
|\childdocof{|\textit{main}|}|\\
\end{tabular}
\end{center}
at the top of every child file \textit{child}
which is included by |\include{|\textit{child}|}|
from within the main file
(or at least for those files to be compiled individually).
The argument \textit{main} must be the filename of the main file.

There are a couple of
considerations in setting up the main and child documents:

%%%%%%%%%%%%%%%%%%%%%%%%%%%%%%%%%%%%%%%%
\paragraph{Restrictions.}

Please note the following restrictions:
\begin{itemize}
\item
|\childdocmain| must be called with one argument \textit{main}
to ensure compatibility with earlier version of the package.
It must either be empty (|\childdocmain{}|)
or precisely match the filename of the main file in which it is specified.
See \secref{sec:detection} for further information.
\item
The filename \textit{main} must be specified without the |.tex| extension.
\item
The filename \textit{main} is case sensitive
(even in case-insensitive file systems)
due to internal string comparison.
\item
The argument \textit{main} should be fully expanded, it cannot be a macro.
\item
Subdirectories and special characters should be avoided in filenames.
\item
The command |\childdocmain{|\textit{main}|}| must be followed by a whitespace.
It should not be followed immediately by another command
or by a comment mark `|%|'.
This is because the \TeX{} parser reads the token immediately following
the argument of |\childdocmain| and puts it
at the beginning of every child section;
however, a white\-space is ignored.
\end{itemize}

%%%%%%%%%%%%%%%%%%%%%%%%%%%%%%%%%%%%%%%%
\paragraph{Content of Main File.}

It is advisable to place all content in the child files included by |\include|.
Any output contained in the main file will appear in all child documents
unless suppressed manually;
it cannot be suppressed automatically by the |\includeonly| directive
and thus should normally be avoided.
A method to include some content in the main file
by means of conditional processing is described in \secref{sec:conditional}.

%%%%%%%%%%%%%%%%%%%%%%%%%%%%%%%%%%%%%%%%
\paragraph{Page Numbering.}

When only a part of the document is compiled,
the appropriate numbering of pages
(as well as other status parameters)
is determined from the |.aux| files.
The latter contain information from previous passes.
However this information needs to propagate through
all intermediate child documents.
Therefore the page numbering in child documents may well
be inconsistent until the complete document is compiled at least once.

A useful (if unconventional) way to always ensure a consistent
page numbering is to restart the numbering in each child document
and denote the pages by `\textit{child}|.|\textit{page}'
where \textit{child} represents the chapter/section number of the child file.
This can be achieved by the command
|\numberwithin{page}{|\textit{child}|}|
of the \textsf{amsmath} package
where \textit{child} can be |chapter| or |section|
depending on the chosen structuring.
Alternatively, one can modify the macro |\thepage| appropriately
and reset the counter |page| at the start of each child file.

%%%%%%%%%%%%%%%%%%%%%%%%%%%%%%%%%%%%%%%%%%%%%%%%%%%%%%%%%%%%%%%%%%%%%%%%%%%%%%%%
\subsection{Conditional Processing}
\label{sec:conditional}

The package provides a mechanism to compile different versions
of a document. To customise the versions further some conditional processing
can come in handy to distinguish which version is being compiled.
The package provides two macros to describe the compilation context:

%%%%%%%%%%%%%%%%%%%%%%%%%%%%%%%%%%%%%%%%
\DescribeMacro{\ifchilddoc}
The conditional |\ifchilddoc| distinguishes between the compilation of
child documents and the main document:
%
\begin{center}
|\ifchilddoc |\textit{child-code}| |[|\||else |\textit{main-code}]| \||fi|
\end{center}

%%%%%%%%%%%%%%%%%%%%%%%%%%%%%%%%%%%%%%%%
\DescribeMacro{\childdocname}
\DescribeMacro{\childdocjob}
The macro |\childdocname| contains the filename (without extension)
of the main or child file being processed.
Note that |\childdocjob| will always contain the name of the main file.

%%%%%%%%%%%%%%%%%%%%%%%%%%%%%%%%%%%%%%%%
\paragraph{Title Page.}

Conditional processing can be used to include a title or banner page
in the main document when proper precautions are taken.
Importantly, the code in the main file should ensure that the page counter
(as well as other status parameters which are stored in the |.aux| files)
takes the same value after the conditional processing.
Otherwise the page numbers may take divergent values
depending on which part is compiled.

For example, a title page could be declared by:
%
\begin{center}
\begin{tabular}{l}
|\ifchilddoc\||else|\\
|\addtocounter{page}{-1}|\\
\textit{code for title page}\\
|\newpage|\\
|\||fi|
\end{tabular}
\end{center}
%
A banner page for the child documents can be generated by:
%
\begin{center}
\begin{tabular}{l}
|\ifchilddoc|\\
|\addtocounter{page}{-1}|\\
\textit{code for banner page}\\
|\newpage|\\
|\||fi|
\end{tabular}
\end{center}
%
Here one could write a message such as:
\begin{center}
|This is the part \childdocname{} of \childdocjob{}.|
\end{center}

%%%%%%%%%%%%%%%%%%%%%%%%%%%%%%%%%%%%%%%%%%%%%%%%%%%%%%%%%%%%%%%%%%%%%%%%%%%%%%%%
\subsection{Flags}
\label{sec:flags}

The package makes it easy to generate different versions
of the main or child documents.
To this end compilation flags can be defined
and assigned different default values.
They will be particularly useful in conjunction
with the forwarding mechanism described in \secref{sec:forward}.

For example, it may be useful to have a flag |\version|
which can be set to |draft| or |final|.
The document source will contain some conditional code
depending on the value of |\version|.
Suppose further, the flag should default to |final| for the main file
and to |draft| for child files
which is a natural assignment for editing the document.
This is achieved by placing the following code
in the preamble of the main document
(below the |\childdocmain| directive):
%
\begin{center}
\begin{tabular}{l}
|\ifchilddoc|\\
|\providecommand{\version}{draft}|\\
|\||else|\\
|\providecommand{\version}{final}|\\
|\||fi|
\end{tabular}
\end{center}
%
The definition by |\providecommand| makes sure
that previous definitions are not overwritten.
Further statements |\providecommand{\version}{...}|
can thus be added before the above code to override it.

For the main file, one might add a line
(between |\childdocmain| and the above block)
%
\begin{center}
|%\ifchilddoc\||else\providecommand{\version}{draft}\||fi|
\end{center}
%
which can be uncommented to produce a draft version.
Likewise one can add a line to the very top of a child file
(above the |\childdocof{|\textit{main}|}| directive)
%
\begin{center}
|%\providecommand{\version}{final}|
\end{center}
%
which can be uncommented to produce the final version of this child document.

%%%%%%%%%%%%%%%%%%%%%%%%%%%%%%%%%%%%%%%%%%%%%%%%%%%%%%%%%%%%%%%%%%%%%%%%%%%%%%%%
\subsection{Forwarding}
\label{sec:forward}

Different versions of the main or child documents
using compilation flags as described in \secref{sec:flags}
can be (permanently) stored in different files
for convenient compilation, viewing and distribution.
To this end, the package defines a command
to pass on compilation to a different file:

%%%%%%%%%%%%%%%%%%%%%%%%%%%%%%%%%%%%%%%%
\DescribeMacro{\childdocforward}
The command |\childdocforward| redirects processing to
another source file:
%
\begin{center}
\begin{tabular}{l}
|\input{childdoc.def}|\\
|\childdocforward[|\textit{main}|]{|\textit{dest}|}|\\
\end{tabular}
\end{center}
%
The argument \textit{dest} is the destination file
(without extension).
It should be the main file or one of the child files.
Note that further \textsf{childdoc} directives
such as |\childdocof| and |\childdocforward|
in the indicated file will be processed in this form.
The optional argument \textit{main}
passes on directly to the main file \textit{main}
while pretending to compile the child \textit{dest}.
This form behaves as if \textit{dest}
issues |\childdocof{|\textit{main}|}| right away,
and no further \textsf{childdoc} directives will be processed.

%%%%%%%%%%%%%%%%%%%%%%%%%%%%%%%%%%%%%%%%
\DescribeMacro{\...prefix}
In the alternative form |\childdocforwardprefix|,
%
\begin{center}
\begin{tabular}{l}
|\input{childdoc.def}|\\
|\childdocforwardprefix[|\textit{main}|]{|\textit{prefix}|}{|\textit{dest}|}|
\end{tabular}
\end{center}
%
the destination file is determined by a pattern
depending on the current file:
To make this work, the current file must be called
`{\textit{prefix}\hspace{0.2em}\textit{suffix}}'
with \textit{prefix} matching precisely the argument.
Processing is then passed on to the file
`{\textit{dest}\hspace{0.2em}\textit{suffix}}'.
Surely, the same effect is achieved by
directly specifying the
argument `{\textit{dest}\hspace{0.2em}\textit{suffix}}'
in the first form.
However, that requires to set up a different file
for each child. With the alternative form of the command
all these files can have exactly the same content
which simplifies setting them up and maintaining them.

For example, the following file |draft.tex|
with a compilation flag |\version| as described in \secref{sec:flags}
compiles the main document as a draft:
%
\begin{center}
\begin{tabular}{l}
|\def\version{draft}|\\
|\input{childdoc.def}|\\
|\childdocforward{|\textit{main}|}|
\end{tabular}
\end{center}
%
Likewise, the following files |final|\textit{nn}|.tex|
compile the final version of the child document
|child|\textit{nn}|.tex|:
%
\begin{center}
\begin{tabular}{l}
|\def\version{final}|\\
|\input{childdoc.def}|\\
|\childdocforwardprefix{final}{child}|
\end{tabular}
\end{center}
%

Note that when several versions of a main file and/or of each child file
are to be generated, it may be convenient to set up a |Makefile| or
shell script to automatise the process.

%%%%%%%%%%%%%%%%%%%%%%%%%%%%%%%%%%%%%%%%%%%%%%%%%%%%%%%%%%%%%%%%%%%%%%%%%%%%%%%%
\subsection{Command Line Processing}
\label{sec:commandline}

The effect of redirection files can also be achieved by invoking
the \LaTeX{} compiler with a more elaborate command line.
Most conveniently this should be done as part
of a shell script or a |Makefile|.

When using \textsf{childdoc} in the main file, the following
command lines effectively perform a redirection
(note that depending on the shell being used,
backslashes may have to be doubled: `|\|' $\to$ `|\\|'):
%
\begin{center}
|... -jobname "|\textit{target}|" |\\|"|[\textit{flags}]%
|\input{childdoc.def}\childdocforward[|\textit{main}|]{|\textit{dest}|}"|
\end{center}
%
Here \textit{target} is the name of the output file,
\textit{main} is the name of the main file
and \textit{dest} is the name of the main or child file to be processed
(all filenames without extensions).
The optional argument \textit{main} can be omitted
if \textit{main} matches \textit{dest}.
Optionally, compilation \textit{flags} can be defined via |\def| commands.
This command line makes the \TeX{} engine believe
it is compiling the file \textit{target}
whose content is specified as the latter parameter.
The provided code then forwards the processing to
\textit{main} or \textit{dest} as described in \secref{sec:forward}.

%%%%%%%%%%%%%%%%%%%%%%%%%%%%%%%%%%%%%%%%%%%%%%%%%%%%%%%%%%%%%%%%%%%%%%%%%%%%%%%%
\subsection{Include by Input}
\label{sec:input}

Including child documents by |\include| has some restrictions by design.
Most notably, the content of a child document always occupies
its own set of pages; pages cannot be shared between child documents.
Usually, this behaviour makes perfect sense
because each child document contain an essential part of the document.
However, in some situations it may be desirable to compose
a document from a collection of parts
without having mandatory page breaks between then.
For this case, the package
provides a mechanism to include parts
by |\input| which can also be processed individually.
However, by construction this mechanism
requires manual handling of the content to be output.

%%%%%%%%%%%%%%%%%%%%%%%%%%%%%%%%%%%%%%%%
\DescribeMacro{\ifchilddocmanual}
The main file should be prepared as usual, see \secref{sec:include}.
However, the document body must make a distinction
between processing of an individual part and of the main document, e.g.:
%
\begin{center}
\begin{tabular}{l}
|\ifchilddocmanual|\\
|\input{\childdocname}|\\
|\||else|\\
\textit{document body with }|\input{|\textit{part}|}|\\
|\||fi|
\end{tabular}
\end{center}
%
The conditional |\ifchilddocmanual| is true whenever
a part to be included by |\input| is being compiled,
and the name of the part is stored in |\childdocname|.

%%%%%%%%%%%%%%%%%%%%%%%%%%%%%%%%%%%%%%%%
\DescribeMacro{\childdocby}
Each part to be included by |\input| should start with:
%
\begin{center}
\begin{tabular}{l}
|\input{childdoc.def}|\\
|\childdocby{|\textit{main}|}|\\
\end{tabular}
\end{center}
%
The directive |\childdocby| is similar to |\childdocof|
described in \secref{sec:include},
but the subsequent selection of content must be done manually.
To that end, both |\ifchilddoc| and |\ifchilddocmanual|
will be true upon processing of a part,
and the name of the part is stored in |\childdocname|.
Note that |\jobname| will be set to the filename of the current part
so that each part receives an individual |.aux| file
that does not interfere with the |.aux| file(s) of the main document.
This behaviour can be altered by the alternative form
|\childdocby[*]{|\textit{main}|}| (with a non-empty optional argument)
which uses the |.aux| file of the main document
by setting |\jobname| to \textit{main}.

%%%%%%%%%%%%%%%%%%%%%%%%%%%%%%%%%%%%%%%%%%%%%%%%%%%%%%%%%%%%%%%%%%%%%%%%%%%%%%%%
\subsection{Driver Development}
\label{sec:driver}

The \textsf{childdoc} mechanism can also be use for the development
of definition files such as \LaTeX{} styles or classes.
This case differs from the above setup with multiple parts
included by |\include| in that no |\includeonly| should be invoked.
This can be achieved by starting the include file
(before |\ProvidesPackage|) with:
%
\begin{center}
\begin{tabular}{l}
|\input{childdoc.def}|\\
|\childdocforward{|\textit{main}|}|\\
\end{tabular}
\end{center}
%
or alternatively with:
%
\begin{center}
\begin{tabular}{l}
|\input{childdoc.def}|\\
|\childdocby{|\textit{main}|}|\\
\end{tabular}
\end{center}
%
Both forms have slightly different effects as described above.
The main file is prepared as usual, see \secref{sec:include}.

%%%%%%%%%%%%%%%%%%%%%%%%%%%%%%%%%%%%%%%%%%%%%%%%%%%%%%%%%%%%%%%%%%%%%%%%%%%%%%%%
\subsection{Legacy Detection}
\label{sec:detection}

The directive |\childdocmain| in the main file can detect
whether the complete document or merely a child is to be compiled
even without using the directive |\childdocof|.
This method is deprecated because it is less robust
and there is no compelling reason to use it;
it is merely provided for backward compatibility
and it may be removed in future versions.

If the detection mechanism is to be used,
it is mandatory to correctly specify
the filename of the main file as the argument of |\childdocmain|:
%
\begin{center}
\begin{tabular}{l}
|\input{childdoc.def}|\\
|\childdocmain{|\textit{main}|}|\\
\end{tabular}
\end{center}
%
If |\jobname| does not match the argument \textit{main} of |\childdocmain|,
it is assumed that |\jobname| points to the child file to be compiled.
When using |\childdocmain| with the main file specified as argument,
it suffices to start a child file
with just |\input{|\textit{main}|}|
without loading of the package and using |\childdocof|.
If instead all processing is done
with the appropriate \textsf{childdoc} directives,
the argument of \textit{main} of |\childdocmain| can be empty.

An alternative version of the command line processing described
in \secref{sec:commandline} using the detection mechanism reads:
%
\begin{center}
|... -jobname "|\textit{target}|" "|[\textit{flags}]%
[|\def\jobname{|\textit{dest}|}|]|\input{|\textit{main}|}"|
\end{center}

%%%%%%%%%%%%%%%%%%%%%%%%%%%%%%%%%%%%%%%%%%%%%%%%%%%%%%%%%%%%%%%%%%%%%%%%%%%%%%%%
\subsection{Manual Code}
\label{sec:manual}

In case one cannot be certain whether the definitions file |childdoc.def|
is installed on the target \TeX{} distribution
and one prefers not to ship it,
it is conceivable to paste a few relevant commands into the sources.

To that end, drop all statements |\input{childdoc.def}|
and perform the replacements as outlined below.
Instead of |\childdocmain{|\textit{main}|}| add the following code
to the top of the main file:
%
\begin{center}
\begin{tabular}{l}
|\||ifdefined\childdocname\endinput\||fi\newif\ifchilddoc|\\
|\edef\childdocname{\scantokens\expandafter{\jobname\noexpand}}|\\
|\def\childdocmain{|\textit{main}|}\||ifx\childdocmain\childdocname\||else|\\
|\childdoctrue\includeonly{\childdocname}\let\jobname\childdocmain\||fi|\\
\end{tabular}
\end{center}
%
Instead of |\childdocof{|\textit{main}|}| just include the main file
at the top of each child file:
%
\begin{center}
|\input{|\textit{main}|}|
\end{center}
%
A simple redirection |\childdocforward{|\textit{dest}|}| is achieved by:
%
\begin{center}
|\def\jobname{|\textit{dest}|}\input{\jobname}|
\end{center}
%
The redirection with prefix
|\childdocforwardprefix[|\textit{prefix}|]{|\textit{dest}|}|
is accomplished by:
%
\begin{center}
\begin{tabular}{l}
|{\edef\jobname{\scantokens\expandafter{\jobname\noexpand}}|\\
|\def\redirectjob |\textit{prefix}|#1~~~{\gdef\jobname{|\textit{dest}|#1}}|\\
|\expandafter\redirectjob\jobname~~~}\input{\jobname}|
\end{tabular}
\end{center}

In an alternative approach,
child documents can be compiled by a specific command line
without additional code or specific definitions:
%
\begin{center}
|... -jobname "|\textit{target}|" "|[\textit{flags}]%
|\includeonly{|\textit{dest}|}\input{|\textit{main}|}"|
\end{center}
%

%%%%%%%%%%%%%%%%%%%%%%%%%%%%%%%%%%%%%%%%%%%%%%%%%%%%%%%%%%%%%%%%%%%%%%%%%%%%%%%%
%%%%%%%%%%%%%%%%%%%%%%%%%%%%%%%%%%%%%%%%%%%%%%%%%%%%%%%%%%%%%%%%%%%%%%%%%%%%%%%%
\section{Information}

%%%%%%%%%%%%%%%%%%%%%%%%%%%%%%%%%%%%%%%%%%%%%%%%%%%%%%%%%%%%%%%%%%%%%%%%%%%%%%%%
\subsection{Copyright}

Copyright \copyright{} 2017--2018 Niklas Beisert

This work may be distributed and/or modified under the
conditions of the \LaTeX{} Project Public License, either version 1.3
of this license or (at your option) any later version.
The latest version of this license is in
  \url{http://www.latex-project.org/lppl.txt}
and version 1.3 or later is part of all distributions of \LaTeX{}
version 2005/12/01 or later.

This work has the LPPL maintenance status `maintained'.

The Current Maintainer of this work is Niklas Beisert.

This work consists of the files |README.txt|, |childdoc.ins| and |childdoc.dtx|
as well as the derived files |childdoc.def|, |cdocsamp.tex|
with |cdocsch1.tex|, |cdocsch2.tex|, |cdocspt3.tex|, |cdocspt4.tex|,
|cdocsdrf.tex|, |cdocsfn1.tex|, |cdocsfn2.tex|
as well as |childdoc.pdf|.

%%%%%%%%%%%%%%%%%%%%%%%%%%%%%%%%%%%%%%%%%%%%%%%%%%%%%%%%%%%%%%%%%%%%%%%%%%%%%%%%
\subsection{Files and Installation}

The package consists of the files:
%
\begin{center}
\begin{tabular}{ll}
    |README.txt|   & readme file \\
    |childdoc.ins| & installation file \\
    |childdoc.dtx| & source file \\
    |childdoc.def| & definition file \\
    |cdocsamp.tex| & sample main file \\
    |cdocsch1.tex| & sample include file \\
    |cdocsch2.tex| & sample include file \\
    |cdocspt3.tex| & sample part file \\
    |cdocspt4.tex| & sample part file \\
    |cdocsdrf.tex| & sample redirection file \\
    |cdocsfn1.tex| & sample redirection file \\
    |cdocsfn2.tex| & sample redirection file \\
    |childdoc.pdf| & manual
\end{tabular}
\end{center}
%
The distribution consists of the files
|README.txt|, |childdoc.ins| and |childdoc.dtx|.
%
\begin{itemize}
\item
Run (pdf)\LaTeX{} on |childdoc.dtx|
to compile the manual |childdoc.pdf| (this file).
\item
Run \LaTeX{} on |childdoc.ins| to create the definitions file |childdoc.def|
and the sample |cdocsamp.tex| with include files
|cdocsch1.tex|, |cdocsch2.tex|, |cdocspt3.tex|, |cdocspt4.tex|,
|cdocsdrf.tex|, |cdocsfn1.tex|, |cdocsfn2.tex|.
Then copy the file |childdoc.def| to an appropriate directory of your \LaTeX{}
distribution, e.g.\ \textit{texmf-root}|/tex/latex/childdoc|.
\end{itemize}

%%%%%%%%%%%%%%%%%%%%%%%%%%%%%%%%%%%%%%%%%%%%%%%%%%%%%%%%%%%%%%%%%%%%%%%%%%%%%%%%
\subsection{Related CTAN Packages}

There are several other packages which offer a similar functionality:
%
\begin{itemize}
\item
The packages
\href{http://ctan.org/pkg/docmute}{\textsf{docmute}},
\href{http://ctan.org/pkg/includex}{\textsf{includex}} and
\href{http://ctan.org/pkg/standalone}{\textsf{standalone}}
provide commands to include only the document body of
a child file thus allowing both files to be compiled individually.
\item
The packages \href{http://ctan.org/pkg/subdocs}{\textsf{subdocs}}
and \href{http://ctan.org/pkg/subfiles}{\textsf{subfiles}}
provide structures in which the main and child documents can be
encapsulated and allowing them to be compiled individually.
The inclusion mechanism is different from the conventional |\include|.
\item
The package \href{http://ctan.org/pkg/combine}{\textsf{combine}}
is an elaborate solution to combine several documents into one.
\end{itemize}
%
See also the CTAN topic \href{http://ctan.org/topic/subdocs}{\textsf{subdocs}}
for further related packages.
The present package differs from the above solutions in that
a document structure constructed with the conventional |\include| mechanism
just needs two extra commands at the top of every file
such that all constituent files can be compiled individually.

%%%%%%%%%%%%%%%%%%%%%%%%%%%%%%%%%%%%%%%%%%%%%%%%%%%%%%%%%%%%%%%%%%%%%%%%%%%%%%%%
%\subsection{Feature Suggestions}
%
%The following is a list of features which may be useful for future
%versions of this package:
%%
%\begin{itemize}
%\item
%\ldots
%\end{itemize}

%%%%%%%%%%%%%%%%%%%%%%%%%%%%%%%%%%%%%%%%%%%%%%%%%%%%%%%%%%%%%%%%%%%%%%%%%%%%%%%%
\subsection{Revision History}

%%%%%%%%%%%%%%%%%%%%%%%%%%%%%%%%%%%%%%%%
\paragraph{v2.0:} 2018/12/30

\begin{itemize}
\item
immediate forward processing
\item
added |\childdocby| mechanism
\item
manual restructured
\end{itemize}

%%%%%%%%%%%%%%%%%%%%%%%%%%%%%%%%%%%%%%%%
\paragraph{v1.6:} 2018/01/17

\begin{itemize}
\item
application for development of include files
\item
corrections to manual
\end{itemize}

%%%%%%%%%%%%%%%%%%%%%%%%%%%%%%%%%%%%%%%%
\paragraph{v1.5:} 2017/05/21

\begin{itemize}
\item
more complete structuring introduced
\item
|\childdocof| introduced
\item
|\childdoc| renamed to |\childdocmain|
\item
|\childredirect| renamed to |\childdocforward| and |\childdocforwardprefix|
and functionality expanded
\end{itemize}

%%%%%%%%%%%%%%%%%%%%%%%%%%%%%%%%%%%%%%%%
\paragraph{v1.0:} 2017/04/27

\begin{itemize}
\item
manual and install package
\item
first version published on CTAN
\end{itemize}

%%%%%%%%%%%%%%%%%%%%%%%%%%%%%%%%%%%%%%%%
\paragraph{v0.6:} 2017/04/26

\begin{itemize}
\item
redirection mechanism added
\end{itemize}

%%%%%%%%%%%%%%%%%%%%%%%%%%%%%%%%%%%%%%%%
\paragraph{v0.5:} 2017/04/26

\begin{itemize}
\item
functionality in definition file
\end{itemize}


%%%%%%%%%%%%%%%%%%%%%%%%%%%%%%%%%%%%%%%%%%%%%%%%%%%%%%%%%%%%%%%%%%%%%%%%%%%%%%%%
%%%%%%%%%%%%%%%%%%%%%%%%%%%%%%%%%%%%%%%%%%%%%%%%%%%%%%%%%%%%%%%%%%%%%%%%%%%%%%%%
%%%%%%%%%%%%%%%%%%%%%%%%%%%%%%%%%%%%%%%%%%%%%%%%%%%%%%%%%%%%%%%%%%%%%%%%%%%%%%%%
\appendix

\settowidth\MacroIndent{\rmfamily\scriptsize 000\ }

 \DocInput{childdoc.dtx}

\end{document}
%</driver>
% \fi
%
% %%%%%%%%%%%%%%%%%%%%%%%%%%%%%%%%%%%%%%%%%%%%%%%%%%%%%%%%%%%%%%%%%%%%%%%%%%%%%%
% %%%%%%%%%%%%%%%%%%%%%%%%%%%%%%%%%%%%%%%%%%%%%%%%%%%%%%%%%%%%%%%%%%%%%%%%%%%%%%
% \section{Sample}
%\iffalse
%<*samplemain>
%\fi
%
% The following presents a sample document
% with two chapters, two parts, a title page,
% a compile flag as well as three forwarding files to set the flag.
% It consists of eight |.tex| files:
% \begin{center}
% \begin{tabular}{ll}
% |cdocsamp.tex|&main file\\
% |cdocsch1.tex|&include file for chapter 1\\
% |cdocsch2.tex|&include file for chapter 2\\
% |cdocspt3.tex|&include file for part 3\\
% |cdocspt4.tex|&include file for part 4\\
% |cdocsdrf.tex|&forwarding file for main file in draft mode\\
% |cdocsfi1.tex|&forwarding file for final version of chapter 1\\
% |cdocsfi2.tex|&forwarding file for final version of chapter 2\\
% \end{tabular}
% \end{center}
% Each of the eight files can be compiled directly by the \LaTeX{} compiler.
%
% %%%%%%%%%%%%%%%%%%%%%%%%%%%%%%%%%%%%%%
% \paragraph{Main File.}
%
% The main file is called |cdocsamp.tex|.
%
% Load the \textsf{childdoc} definitions and
% declare the filename for the main document:
%    \begin{macrocode}
\input{childdoc.def}
\childdocmain{}
%    \end{macrocode}

% Optional override for |\version| flag:
%    \begin{macrocode}
%%\ifchilddoc\else\providecommand{\version}{draft}\fi
%    \end{macrocode}

% Define the default values for the |\version| flag
% (|final| for the main file and |draft| for childs):
%    \begin{macrocode}
\ifchilddoc
\providecommand{\version}{draft}
\else
\providecommand{\version}{final}
\fi
%    \end{macrocode}

% Load the standard document class:
%    \begin{macrocode}
\documentclass[12pt]{article}
%    \end{macrocode}

% Start the document body:
%    \begin{macrocode}
\begin{document}
%    \end{macrocode}

% Declare a title page.
% Print title, part of document being processed and version flag:
%    \begin{macrocode}
\addtocounter{page}{-1}
\begin{center}
{\LARGE\bfseries{}childdoc example\par}
\vspace{1cm}
\ifchilddoc
\ifchilddocmanual part\else chapter\fi:
`\childdocname' of `\childdocjob'\par
\else
main document: `\childdocjob'\par
\fi
version: \version\par
\end{center}
\newpage
%    \end{macrocode}

% Manually include selected file,
% otherwise process as usual:
%    \begin{macrocode}
\ifchilddocmanual
\section*{part `\childdocname'}
\input{\childdocname}
\else
%    \end{macrocode}

% Include the two chapters:
%    \begin{macrocode}
\include{cdocsch1}
\include{cdocsch2}
%    \end{macrocode}

% Include the two parts unless only chapters should be displayed:
%    \begin{macrocode}
\ifchilddoc\else
\section{part three}
\input{cdocspt3}
\section{part four}
\input{cdocspt4}
\fi
%    \end{macrocode}

% Process as usual until here:
%    \begin{macrocode}
\fi
%    \end{macrocode}

% End of document body:
%    \begin{macrocode}
\end{document}
%    \end{macrocode}
%\iffalse
%</samplemain>
%\fi
%
% %%%%%%%%%%%%%%%%%%%%%%%%%%%%%%%%%%%%%%
% \paragraph{Chapter Include Files.}
%
% The include files are called |cdocsch1.tex| and |cdocsch2.tex|.
%
%\iffalse
%<*samplechap1|samplechap2>
%\fi

% Optional override for |\version| flag:
%    \begin{macrocode}
%%\providecommand{\version}{final}
%    \end{macrocode}

% Include the main document:
%    \begin{macrocode}
\input{childdoc.def}
\childdocof{cdocsamp}
%    \end{macrocode}

%\iffalse
%</samplechap1|samplechap2>
%\fi
%
%\iffalse
%<*samplechap1>
%\fi
% Some text for chapter 1:
%    \begin{macrocode}
\section{one}
some text in chapter one
%    \end{macrocode}

%\iffalse
%</samplechap1>
%\fi
% Some text for chapter 2:
%\iffalse
%<*samplechap2>
%\fi
%    \begin{macrocode}
\section{two}
more text in chapter two
%    \end{macrocode}

%\iffalse
%</samplechap2>
%\fi
%
% %%%%%%%%%%%%%%%%%%%%%%%%%%%%%%%%%%%%%%
% \paragraph{Part Include Files.}
%
% The include files are called |cdocspt3.tex| and |cdocspt4.tex|.
%
%\iffalse
%<*samplepart3|samplepart4>
%\fi

% Optional override for |\version| flag:
%    \begin{macrocode}
%%\providecommand{\version}{final}
%    \end{macrocode}

% Include the main document:
%    \begin{macrocode}
\input{childdoc.def}
\childdocby{cdocsamp}
%    \end{macrocode}

%\iffalse
%</samplepart3|samplepart4>
%\fi
%
%\iffalse
%<*samplepart3>
%\fi
% Some text for part 3:
%    \begin{macrocode}
some text in part three
%    \end{macrocode}

%\iffalse
%</samplepart3>
%\fi
% Some text for part 4:
%\iffalse
%<*samplepart4>
%\fi
%    \begin{macrocode}
more text in part four
%    \end{macrocode}

%\iffalse
%</samplepart4>
%\fi
%
% %%%%%%%%%%%%%%%%%%%%%%%%%%%%%%%%%%%%%%
% \paragraph{Forwarding for a Complete Draft.}
%
% The following forwarding file |cdocsdrf.tex|
% compiles the main document in draft mode:
%\iffalse
%<*sampledraft>
%\fi
%    \begin{macrocode}
\def\version{draft}
\input{childdoc.def}
\childdocforward{cdocsamp}
%    \end{macrocode}

%\iffalse
%</sampledraft>
%\fi
%
% %%%%%%%%%%%%%%%%%%%%%%%%%%%%%%%%%%%%%%
% \paragraph{Forwarding for Final Version of the Chapters.}
%
% The following forwarding files |cdocsfn1.tex| and |cdocsfn2.tex|
% (with identical content)
% compile the final versions of the child documents
% |cdocsch1.tex| and |cdocsch2.tex|, respectively:
%\iffalse
%<*samplefinal>
%\fi
%    \begin{macrocode}
\def\version{final}
\input{childdoc.def}
\childdocforwardprefix[cdocsamp]{cdocsfn}{cdocsch}
%    \end{macrocode}

%\iffalse
%</samplefinal>
%\fi
%
% %%%%%%%%%%%%%%%%%%%%%%%%%%%%%%%%%%%%%%
% \paragraph{Command Line Processing.}
%
% The following three command lines generate the output files
% |cdocscld|, |cdocscl1| and |cdocscl2|
% which should be identical to
% |cdocsdrf|, |cdocsch1| and |cdocsfn2|, respectively:
% \begin{center}
% \begin{tabular}{l}
% |latex -jobname cdocscld \|\\
% |  "\def\version{draft}\input{childdoc.def}\childdocforward{cdocsamp}"|\\
% |latex -jobname cdocscl1 \|\\
% |  "\input{childdoc.def}\childdocforward[cdocsamp]{cdocsch1}"|\\
% |latex -jobname cdocscl2 \|\\
% |  "\def\version{final}\input{childdoc.def}\childdocforward{cdocsch2}"|
% \end{tabular}
% \end{center}
% Note that the trailing backslash on each first line
% merely continues the input to the second line
% (for convenient cut ant paste).
% Furthermore, the command |latex| can be replaced by any
% of its alternative versions such as |pdflatex|.
%
% %%%%%%%%%%%%%%%%%%%%%%%%%%%%%%%%%%%%%%%%%%%%%%%%%%%%%%%%%%%%%%%%%%%%%%%%%%%%%%
% %%%%%%%%%%%%%%%%%%%%%%%%%%%%%%%%%%%%%%%%%%%%%%%%%%%%%%%%%%%%%%%%%%%%%%%%%%%%%%
% \section{Implementation}
%\iffalse
%<*package>
%\fi
%
% This section describes the definitions file |childdoc.def|.

% The definitions cannot be loaded using |\usepackage| or |\RequirePackage|
% which has a mechanism to prevent loading a style file more than once.
% When loading the definitions by means of |\input|
% multiple instances have to be prevented manually:
%\iffalse
%This code needs to be before the `\ProvidesFile' directive
%which is defined at the beginning of this file.
%Therefore it is also placed there and commented out here.
%</package>
%<*discard>
%\fi
%    \begin{macrocode}
\ifdefined\childdocmain\endinput\fi
%    \end{macrocode}
%\iffalse
%</discard>
%<*package>
%\fi
%
% \macro{\ifchilddoc}
% \macro{\ifchilddocmanual}
% The conditional |\ifchilddoc| tells whether a
% child (true) or main (false) document is being compiled.
% The conditional |\ifchilddocmanual| tells whether
% the |\includeonly| mechanism is used (false) or
% the selection of child files must be performed manually (true).
% The definitions initialise to false:
%    \begin{macrocode}
\newif\ifchilddoc
\newif\ifchilddocmanual
%    \end{macrocode}

% \macro{\childdocname}
% \macro{\childdocjob}
% The macro |\childdocname| stores the name of the main document
% to be compiled. The macro |\childdocjob| stores the name of
% the document on which the \LaTeX{} compiler was originally invoked.
% The content of |\jobname| cannot be compared
% to filenames specified in the source due to different catcodes.
% The following code rescans |\jobname|, stores the result
% in |\childdocname| and saves a copy in |\childdocjob|:
%    \begin{macrocode}
\edef\childdocname{\scantokens\expandafter{\jobname\noexpand}}
\let\childdocjob\childdocname
%    \end{macrocode}

% \macro{\childdocdisable}
% The macro |\childdocdisable| prevents the main file
% from being processed more than once.
% At this stage, the main document command |\childdocmain|
% is assumed to be called once again where it should do nothing.
% Any subsequent call to it should prevent
% a secondary processing of the main document
% It overwrites the forwarding commands
% |\childdocof| and |\childdocforward|
% with empty macros to prevent further inclusions of the main document:
%    \begin{macrocode}
\newcommand{\childdocdisable}
{
  \renewcommand{\childdocmain}[1]{\renewcommand{\childdocmain}[1]{\endinput}}
  \renewcommand{\childdocof}[1]{}
  \renewcommand{\childdocby}[2][]{}
  \renewcommand{\childdocforward}[2][]{}
  \renewcommand{\childdocdisable}{}
}
%    \end{macrocode}

% \macro{\childdocmain}
% The macro |\childdocmain| is to be called at the top of the main file
% with nothing or the main filename (without extension) as argument.
% First, it breaks loops.
% If the argument is not empty and does not match |\childdocname|
% (which is set by the first inclusion of |childdoc.def|),
% |\ifchilddoc| is set to true, |\includeonly| is applied to the child file
% and |\jobname| is set to the main file
% (for proper handling of |.aux| files):
%    \begin{macrocode}
\newcommand{\childdocmain}[1]
{
  \childdocdisable\childdocmain{}
  \if?#1?\else
    \begingroup
      \def\childdoctmp{#1}
      \ifx\childdoctmp\childdocname
        \def\childdoctmp{}
      \else
        \def\childdoctmp
        {
          \childdoctrue
          \includeonly{\childdocname}
          \def\childdocjob{#1}
          \def\jobname{#1}
        }
      \fi
      \expandafter
    \endgroup
    \childdoctmp
  \fi
}
%    \end{macrocode}

% \macro{\childdocof}
% The command |\childdocof| redirects
% compilation to the main file |#1|.
%    \begin{macrocode}
\newcommand{\childdocof}[1]
{
  \childdocdisable
  \childdoctrue
  \includeonly{\childdocname}
  \def\jobname{#1}
  \def\childdocjob{#1}
  \input{#1}
}
%    \end{macrocode}

% \macro{\childdocby}
% The command |\childdocby| ....
%    \begin{macrocode}
\newcommand{\childdocby}[2][]
{
  \childdocdisable
  \childdoctrue
  \childdocmanualtrue
  \if?#1?\else
    \def\jobname{#2}
  \fi
  \def\childdocjob{#2}
  \input{#2}
  \endinput
}
%    \end{macrocode}

% \macro{\childdocforward}
% The command |\childdocforward| redirects
% compilation to the main file or
% (if the optional argument is given) a child file.
% Parameters are set as if the main file
% or a child file starting with |\childdocof| was compiled.
% Then compilation is handed over to the main file:
%    \begin{macrocode}
\newcommand{\childdocforward}[2][]
{
  \begingroup
    \if?#1?
      \def\childdoctmp
      {
        \def\childdocname{#2}
        \def\childdocjob{#2}
        \def\jobname{#2}
        \input{#2}
        \endinput
      }
    \else
      \def\childdoctmp
      {
        \childdocdisable
        \def\childdocname{#2}
        \childdoctrue
        \includeonly{#2}
        \def\childdocjob{#1}
        \def\jobname{#1}
        \input{#1}
        \endinput
      }
    \fi
    \expandafter
  \endgroup
  \childdoctmp
}
%    \end{macrocode}

% \macro{\childdocforwardprefix}
% The command |\childdocforwardprefix| redirects
% compilation to the main or a child file by means of a pattern.
% The prefix |#1| in the current filename is replaced by |#2|
% and the suffix of the current filename is kept
% (it is assumed that the filename does not contain the substring `|~~~|'
% which is used as a delimiter).
% Compilation is handed over to the new file by |\childdocforward|:
%    \begin{macrocode}
\newcommand{\childdocforwardprefix}[3][]
{
  \begingroup
    \def\childdocextract #2##1~~~{\def\childdoctmp{\childdocforward[#1]{#3##1}}}
    \expandafter\childdocextract\childdocname~~~
    \expandafter
  \endgroup
  \childdoctmp
}
%    \end{macrocode}

% \macro{\childdoc}
% The deprecated macro |\childdoc| is a legacy version of |\childdocmain|:
%    \begin{macrocode}
\newcommand{\childdoc}{\childdocmain}
%    \end{macrocode}

% \macro{\childdocredirect}
% The deprecated macro |\childdocredirect| is a legacy version
% of |\childdocforward| and |\childdocforwardprefix|:
%    \begin{macrocode}
\newcommand{\childdocredirect}[2][]
{
  \begingroup
    \if?#1?
      \def\childdoctmp{\childdocforward{#2}}
    \else
      \def\childdoctmp{\childdocforwardprefix{#1}{#2}}
    \fi
    \expandafter
  \endgroup
  \childdoctmp
}
%    \end{macrocode}

%\iffalse
%</package>
%\fi
%
\endinput
|\\
|\childdocforward{|\textit{main}|}|
\end{tabular}
\end{center}
%
Likewise, the following files |final|\textit{nn}|.tex|
compile the final version of the child document
|child|\textit{nn}|.tex|:
%
\begin{center}
\begin{tabular}{l}
|\def\version{final}|\\
|% \iffalse
%
% childdoc.dtx Copyright (C) 2017-2018 Niklas Beisert
%
% This work may be distributed and/or modified under the
% conditions of the LaTeX Project Public License, either version 1.3
% of this license or (at your option) any later version.
% The latest version of this license is in
%   http://www.latex-project.org/lppl.txt
% and version 1.3 or later is part of all distributions of LaTeX
% version 2005/12/01 or later.
%
% This work has the LPPL maintenance status `maintained'.
%
% The Current Maintainer of this work is Niklas Beisert.
%
% This work consists of the files childdoc.dtx and childdoc.ins
% and the derived files childdoc.def and cdocsamp.tex with
% cdocsch1.tex, cdocsch2.tex, cdocsdrf.tex, cdocsfn1.tex, cdocsfn2.tex.
%
%<package>\ifdefined\childdocmain\endinput\fi
%<package>\ProvidesFile{childdoc.def}[2018/12/30 v2.0 child document driver]
%<samplemain>\ProvidesFile{cdocsamp.tex}[2018/12/30 v2.0 sample for childdoc]
%<*driver>
%\ProvidesFile{childdoc.drv}[2018/12/30 v2.0 childdoc reference manual file]
\PassOptionsToClass{10pt,a4paper}{article}
\documentclass{ltxdoc}

\usepackage[margin=35mm]{geometry}
\usepackage{hyperref}
\usepackage{hyperxmp}
\usepackage[usenames]{color}

\hypersetup{colorlinks=true}
\hypersetup{pdfstartview=FitH}
\hypersetup{pdfpagemode=UseNone}
\hypersetup{pdfsource={}}
\hypersetup{pdflang={en-UK}}
\hypersetup{pdfcopyright={Copyright 2017-2018 Niklas Beisert.
  This work may be distributed and/or modified under the
  conditions of the LaTeX Project Public License, either version 1.3
  of this license or (at your option) any later version.}}
\hypersetup{pdflicenseurl={http://www.latex-project.org/lppl.txt}}
\hypersetup{pdfcontactaddress={ETH Zurich, ITP, HIT K,
  Wolfgang-Pauli-Strasse 27}}
\hypersetup{pdfcontactpostcode={8093}}
\hypersetup{pdfcontactcity={Zurich}}
\hypersetup{pdfcontactcountry={Switzerland}}
\hypersetup{pdfcontactemail={nbeisert@itp.phys.ethz.ch}}
\hypersetup{pdfcontacturl={http://people.phys.ethz.ch/\xmptilde nbeisert/}}

\newcommand{\secref}[1]{\hyperref[#1]{section \ref*{#1}}}

\parskip1ex
\parindent0pt
\let\olditemize\itemize
\def\itemize{\olditemize\parskip0pt}

\begin{document}

\title{The \textsf{childdoc} Package}
\hypersetup{pdftitle={The childdoc Package}}
\author{Niklas Beisert\\[2ex]
  Institut f\"ur Theoretische Physik\\
  Eidgen\"ossische Technische Hochschule Z\"urich\\
  Wolfgang-Pauli-Strasse 27, 8093 Z\"urich, Switzerland\\[1ex]
  \href{mailto:nbeisert@itp.phys.ethz.ch}
  {\texttt{nbeisert@itp.phys.ethz.ch}}}
\hypersetup{pdfauthor={Niklas Beisert}}
\hypersetup{pdfsubject={Manual for the LaTeX2e Package childdoc}}
\date{30 December 2018, \textsf{v2.0}}
\maketitle

\begin{abstract}\noindent
\textsf{childdoc} is a \LaTeXe{} package
that enables the direct compilation
of document sections included by |\include|
to individual files.
\end{abstract}

\begingroup
\parskip0ex
\tableofcontents
\endgroup

%%%%%%%%%%%%%%%%%%%%%%%%%%%%%%%%%%%%%%%%%%%%%%%%%%%%%%%%%%%%%%%%%%%%%%%%%%%%%%%%
%%%%%%%%%%%%%%%%%%%%%%%%%%%%%%%%%%%%%%%%%%%%%%%%%%%%%%%%%%%%%%%%%%%%%%%%%%%%%%%%
\section{Introduction}

\LaTeX{} provides a mechanism to structure a large document (such as a book)
into a main file and several child files (containing the chapters)
using the |\include| command.
This mechanism is beneficial for documents
which span hundreds of pages in order to
make the source file(s) more manageable.
Moreover, compilation can be restricted to
selected child files by means of the |\includeonly| command.
The latter feature can be used to reduce the compilation time while editing
(this was significantly more useful in the earlier days of \LaTeX{})
or to generate a smaller document which is easier to navigate.
Another application of |\includeonly| is to generate
documents consisting of selected parts of the complete document.

However, there are a few drawbacks of the plain |\include| mechanism:
\begin{itemize}
\item
The child files cannot be compiled on their own,
they can only be compiled via the main file.
A naive editing environment
(such as a text editor with an option
to have the current file processed by \LaTeX)
may require one to switch to the main file before compiling;
attempting to compile the child file produces errors.
\item
The main file must be modified (each time)
to adjust the |\includeonly| command
to the present needs. This easily leaves the main file in a messy state.
\item
The generated document will always carry the filename
of the main document. This is inconvenient if
several child files are to be compiled and
to be kept for distribution.
\end{itemize}

The present package provides a simple interface
to make child files individually compilable by \LaTeX{}.
Compiling a child file then has the same effect as compiling
the main file with an |\includeonly| command
to select the appropriate child.
Moreover the generated document will carry the name of the child
rather than the main file.
This resolves all three above issues.

This feature is meant to make the editing of books,
thesis documents and lecture notes somewhat more convenient.
However, the package can also be used efficiently for
composing a series of documents (such as exercise sheets)
which are typically distributed individually.
It then assists the author in generating the individual documents
(potentially in different versions)
as well as a document containing the collected series.
Another application is in developing style files
or other kinds of included material
where compilation of the style file could redirect
to a sample or test file.

%%%%%%%%%%%%%%%%%%%%%%%%%%%%%%%%%%%%%%%%%%%%%%%%%%%%%%%%%%%%%%%%%%%%%%%%%%%%%%%%
%%%%%%%%%%%%%%%%%%%%%%%%%%%%%%%%%%%%%%%%%%%%%%%%%%%%%%%%%%%%%%%%%%%%%%%%%%%%%%%%
\section{Usage}

First of all, the package \textsf{childdoc} is \emph{not} a standard
\LaTeXe{} |.sty| style file! Therefore it needs to be invoked in
a non-standard way.

%%%%%%%%%%%%%%%%%%%%%%%%%%%%%%%%%%%%%%%%%%%%%%%%%%%%%%%%%%%%%%%%%%%%%%%%%%%%%%%%
\subsection{Included Files}
\label{sec:include}

%%%%%%%%%%%%%%%%%%%%%%%%%%%%%%%%%%%%%%%%
\DescribeMacro{\childdocmain}
To use the package, add the commands
\begin{center}
\begin{tabular}{l}
|\input{childdoc.def}|\\
|\childdocmain{}|\\
\end{tabular}
\end{center}
at the very top of the main \LaTeX{} file,
in particular \emph{before} the |\documentclass| statement!
The argument of |\childdocmain| should be left empty
(but it must be present).

%%%%%%%%%%%%%%%%%%%%%%%%%%%%%%%%%%%%%%%%
\DescribeMacro{\childdocof}
Furthermore, add the commands
\begin{center}
\begin{tabular}{l}
|\input{childdoc.def}|\\
|\childdocof{|\textit{main}|}|\\
\end{tabular}
\end{center}
at the top of every child file \textit{child}
which is included by |\include{|\textit{child}|}|
from within the main file
(or at least for those files to be compiled individually).
The argument \textit{main} must be the filename of the main file.

There are a couple of
considerations in setting up the main and child documents:

%%%%%%%%%%%%%%%%%%%%%%%%%%%%%%%%%%%%%%%%
\paragraph{Restrictions.}

Please note the following restrictions:
\begin{itemize}
\item
|\childdocmain| must be called with one argument \textit{main}
to ensure compatibility with earlier version of the package.
It must either be empty (|\childdocmain{}|)
or precisely match the filename of the main file in which it is specified.
See \secref{sec:detection} for further information.
\item
The filename \textit{main} must be specified without the |.tex| extension.
\item
The filename \textit{main} is case sensitive
(even in case-insensitive file systems)
due to internal string comparison.
\item
The argument \textit{main} should be fully expanded, it cannot be a macro.
\item
Subdirectories and special characters should be avoided in filenames.
\item
The command |\childdocmain{|\textit{main}|}| must be followed by a whitespace.
It should not be followed immediately by another command
or by a comment mark `|%|'.
This is because the \TeX{} parser reads the token immediately following
the argument of |\childdocmain| and puts it
at the beginning of every child section;
however, a white\-space is ignored.
\end{itemize}

%%%%%%%%%%%%%%%%%%%%%%%%%%%%%%%%%%%%%%%%
\paragraph{Content of Main File.}

It is advisable to place all content in the child files included by |\include|.
Any output contained in the main file will appear in all child documents
unless suppressed manually;
it cannot be suppressed automatically by the |\includeonly| directive
and thus should normally be avoided.
A method to include some content in the main file
by means of conditional processing is described in \secref{sec:conditional}.

%%%%%%%%%%%%%%%%%%%%%%%%%%%%%%%%%%%%%%%%
\paragraph{Page Numbering.}

When only a part of the document is compiled,
the appropriate numbering of pages
(as well as other status parameters)
is determined from the |.aux| files.
The latter contain information from previous passes.
However this information needs to propagate through
all intermediate child documents.
Therefore the page numbering in child documents may well
be inconsistent until the complete document is compiled at least once.

A useful (if unconventional) way to always ensure a consistent
page numbering is to restart the numbering in each child document
and denote the pages by `\textit{child}|.|\textit{page}'
where \textit{child} represents the chapter/section number of the child file.
This can be achieved by the command
|\numberwithin{page}{|\textit{child}|}|
of the \textsf{amsmath} package
where \textit{child} can be |chapter| or |section|
depending on the chosen structuring.
Alternatively, one can modify the macro |\thepage| appropriately
and reset the counter |page| at the start of each child file.

%%%%%%%%%%%%%%%%%%%%%%%%%%%%%%%%%%%%%%%%%%%%%%%%%%%%%%%%%%%%%%%%%%%%%%%%%%%%%%%%
\subsection{Conditional Processing}
\label{sec:conditional}

The package provides a mechanism to compile different versions
of a document. To customise the versions further some conditional processing
can come in handy to distinguish which version is being compiled.
The package provides two macros to describe the compilation context:

%%%%%%%%%%%%%%%%%%%%%%%%%%%%%%%%%%%%%%%%
\DescribeMacro{\ifchilddoc}
The conditional |\ifchilddoc| distinguishes between the compilation of
child documents and the main document:
%
\begin{center}
|\ifchilddoc |\textit{child-code}| |[|\||else |\textit{main-code}]| \||fi|
\end{center}

%%%%%%%%%%%%%%%%%%%%%%%%%%%%%%%%%%%%%%%%
\DescribeMacro{\childdocname}
\DescribeMacro{\childdocjob}
The macro |\childdocname| contains the filename (without extension)
of the main or child file being processed.
Note that |\childdocjob| will always contain the name of the main file.

%%%%%%%%%%%%%%%%%%%%%%%%%%%%%%%%%%%%%%%%
\paragraph{Title Page.}

Conditional processing can be used to include a title or banner page
in the main document when proper precautions are taken.
Importantly, the code in the main file should ensure that the page counter
(as well as other status parameters which are stored in the |.aux| files)
takes the same value after the conditional processing.
Otherwise the page numbers may take divergent values
depending on which part is compiled.

For example, a title page could be declared by:
%
\begin{center}
\begin{tabular}{l}
|\ifchilddoc\||else|\\
|\addtocounter{page}{-1}|\\
\textit{code for title page}\\
|\newpage|\\
|\||fi|
\end{tabular}
\end{center}
%
A banner page for the child documents can be generated by:
%
\begin{center}
\begin{tabular}{l}
|\ifchilddoc|\\
|\addtocounter{page}{-1}|\\
\textit{code for banner page}\\
|\newpage|\\
|\||fi|
\end{tabular}
\end{center}
%
Here one could write a message such as:
\begin{center}
|This is the part \childdocname{} of \childdocjob{}.|
\end{center}

%%%%%%%%%%%%%%%%%%%%%%%%%%%%%%%%%%%%%%%%%%%%%%%%%%%%%%%%%%%%%%%%%%%%%%%%%%%%%%%%
\subsection{Flags}
\label{sec:flags}

The package makes it easy to generate different versions
of the main or child documents.
To this end compilation flags can be defined
and assigned different default values.
They will be particularly useful in conjunction
with the forwarding mechanism described in \secref{sec:forward}.

For example, it may be useful to have a flag |\version|
which can be set to |draft| or |final|.
The document source will contain some conditional code
depending on the value of |\version|.
Suppose further, the flag should default to |final| for the main file
and to |draft| for child files
which is a natural assignment for editing the document.
This is achieved by placing the following code
in the preamble of the main document
(below the |\childdocmain| directive):
%
\begin{center}
\begin{tabular}{l}
|\ifchilddoc|\\
|\providecommand{\version}{draft}|\\
|\||else|\\
|\providecommand{\version}{final}|\\
|\||fi|
\end{tabular}
\end{center}
%
The definition by |\providecommand| makes sure
that previous definitions are not overwritten.
Further statements |\providecommand{\version}{...}|
can thus be added before the above code to override it.

For the main file, one might add a line
(between |\childdocmain| and the above block)
%
\begin{center}
|%\ifchilddoc\||else\providecommand{\version}{draft}\||fi|
\end{center}
%
which can be uncommented to produce a draft version.
Likewise one can add a line to the very top of a child file
(above the |\childdocof{|\textit{main}|}| directive)
%
\begin{center}
|%\providecommand{\version}{final}|
\end{center}
%
which can be uncommented to produce the final version of this child document.

%%%%%%%%%%%%%%%%%%%%%%%%%%%%%%%%%%%%%%%%%%%%%%%%%%%%%%%%%%%%%%%%%%%%%%%%%%%%%%%%
\subsection{Forwarding}
\label{sec:forward}

Different versions of the main or child documents
using compilation flags as described in \secref{sec:flags}
can be (permanently) stored in different files
for convenient compilation, viewing and distribution.
To this end, the package defines a command
to pass on compilation to a different file:

%%%%%%%%%%%%%%%%%%%%%%%%%%%%%%%%%%%%%%%%
\DescribeMacro{\childdocforward}
The command |\childdocforward| redirects processing to
another source file:
%
\begin{center}
\begin{tabular}{l}
|\input{childdoc.def}|\\
|\childdocforward[|\textit{main}|]{|\textit{dest}|}|\\
\end{tabular}
\end{center}
%
The argument \textit{dest} is the destination file
(without extension).
It should be the main file or one of the child files.
Note that further \textsf{childdoc} directives
such as |\childdocof| and |\childdocforward|
in the indicated file will be processed in this form.
The optional argument \textit{main}
passes on directly to the main file \textit{main}
while pretending to compile the child \textit{dest}.
This form behaves as if \textit{dest}
issues |\childdocof{|\textit{main}|}| right away,
and no further \textsf{childdoc} directives will be processed.

%%%%%%%%%%%%%%%%%%%%%%%%%%%%%%%%%%%%%%%%
\DescribeMacro{\...prefix}
In the alternative form |\childdocforwardprefix|,
%
\begin{center}
\begin{tabular}{l}
|\input{childdoc.def}|\\
|\childdocforwardprefix[|\textit{main}|]{|\textit{prefix}|}{|\textit{dest}|}|
\end{tabular}
\end{center}
%
the destination file is determined by a pattern
depending on the current file:
To make this work, the current file must be called
`{\textit{prefix}\hspace{0.2em}\textit{suffix}}'
with \textit{prefix} matching precisely the argument.
Processing is then passed on to the file
`{\textit{dest}\hspace{0.2em}\textit{suffix}}'.
Surely, the same effect is achieved by
directly specifying the
argument `{\textit{dest}\hspace{0.2em}\textit{suffix}}'
in the first form.
However, that requires to set up a different file
for each child. With the alternative form of the command
all these files can have exactly the same content
which simplifies setting them up and maintaining them.

For example, the following file |draft.tex|
with a compilation flag |\version| as described in \secref{sec:flags}
compiles the main document as a draft:
%
\begin{center}
\begin{tabular}{l}
|\def\version{draft}|\\
|\input{childdoc.def}|\\
|\childdocforward{|\textit{main}|}|
\end{tabular}
\end{center}
%
Likewise, the following files |final|\textit{nn}|.tex|
compile the final version of the child document
|child|\textit{nn}|.tex|:
%
\begin{center}
\begin{tabular}{l}
|\def\version{final}|\\
|\input{childdoc.def}|\\
|\childdocforwardprefix{final}{child}|
\end{tabular}
\end{center}
%

Note that when several versions of a main file and/or of each child file
are to be generated, it may be convenient to set up a |Makefile| or
shell script to automatise the process.

%%%%%%%%%%%%%%%%%%%%%%%%%%%%%%%%%%%%%%%%%%%%%%%%%%%%%%%%%%%%%%%%%%%%%%%%%%%%%%%%
\subsection{Command Line Processing}
\label{sec:commandline}

The effect of redirection files can also be achieved by invoking
the \LaTeX{} compiler with a more elaborate command line.
Most conveniently this should be done as part
of a shell script or a |Makefile|.

When using \textsf{childdoc} in the main file, the following
command lines effectively perform a redirection
(note that depending on the shell being used,
backslashes may have to be doubled: `|\|' $\to$ `|\\|'):
%
\begin{center}
|... -jobname "|\textit{target}|" |\\|"|[\textit{flags}]%
|\input{childdoc.def}\childdocforward[|\textit{main}|]{|\textit{dest}|}"|
\end{center}
%
Here \textit{target} is the name of the output file,
\textit{main} is the name of the main file
and \textit{dest} is the name of the main or child file to be processed
(all filenames without extensions).
The optional argument \textit{main} can be omitted
if \textit{main} matches \textit{dest}.
Optionally, compilation \textit{flags} can be defined via |\def| commands.
This command line makes the \TeX{} engine believe
it is compiling the file \textit{target}
whose content is specified as the latter parameter.
The provided code then forwards the processing to
\textit{main} or \textit{dest} as described in \secref{sec:forward}.

%%%%%%%%%%%%%%%%%%%%%%%%%%%%%%%%%%%%%%%%%%%%%%%%%%%%%%%%%%%%%%%%%%%%%%%%%%%%%%%%
\subsection{Include by Input}
\label{sec:input}

Including child documents by |\include| has some restrictions by design.
Most notably, the content of a child document always occupies
its own set of pages; pages cannot be shared between child documents.
Usually, this behaviour makes perfect sense
because each child document contain an essential part of the document.
However, in some situations it may be desirable to compose
a document from a collection of parts
without having mandatory page breaks between then.
For this case, the package
provides a mechanism to include parts
by |\input| which can also be processed individually.
However, by construction this mechanism
requires manual handling of the content to be output.

%%%%%%%%%%%%%%%%%%%%%%%%%%%%%%%%%%%%%%%%
\DescribeMacro{\ifchilddocmanual}
The main file should be prepared as usual, see \secref{sec:include}.
However, the document body must make a distinction
between processing of an individual part and of the main document, e.g.:
%
\begin{center}
\begin{tabular}{l}
|\ifchilddocmanual|\\
|\input{\childdocname}|\\
|\||else|\\
\textit{document body with }|\input{|\textit{part}|}|\\
|\||fi|
\end{tabular}
\end{center}
%
The conditional |\ifchilddocmanual| is true whenever
a part to be included by |\input| is being compiled,
and the name of the part is stored in |\childdocname|.

%%%%%%%%%%%%%%%%%%%%%%%%%%%%%%%%%%%%%%%%
\DescribeMacro{\childdocby}
Each part to be included by |\input| should start with:
%
\begin{center}
\begin{tabular}{l}
|\input{childdoc.def}|\\
|\childdocby{|\textit{main}|}|\\
\end{tabular}
\end{center}
%
The directive |\childdocby| is similar to |\childdocof|
described in \secref{sec:include},
but the subsequent selection of content must be done manually.
To that end, both |\ifchilddoc| and |\ifchilddocmanual|
will be true upon processing of a part,
and the name of the part is stored in |\childdocname|.
Note that |\jobname| will be set to the filename of the current part
so that each part receives an individual |.aux| file
that does not interfere with the |.aux| file(s) of the main document.
This behaviour can be altered by the alternative form
|\childdocby[*]{|\textit{main}|}| (with a non-empty optional argument)
which uses the |.aux| file of the main document
by setting |\jobname| to \textit{main}.

%%%%%%%%%%%%%%%%%%%%%%%%%%%%%%%%%%%%%%%%%%%%%%%%%%%%%%%%%%%%%%%%%%%%%%%%%%%%%%%%
\subsection{Driver Development}
\label{sec:driver}

The \textsf{childdoc} mechanism can also be use for the development
of definition files such as \LaTeX{} styles or classes.
This case differs from the above setup with multiple parts
included by |\include| in that no |\includeonly| should be invoked.
This can be achieved by starting the include file
(before |\ProvidesPackage|) with:
%
\begin{center}
\begin{tabular}{l}
|\input{childdoc.def}|\\
|\childdocforward{|\textit{main}|}|\\
\end{tabular}
\end{center}
%
or alternatively with:
%
\begin{center}
\begin{tabular}{l}
|\input{childdoc.def}|\\
|\childdocby{|\textit{main}|}|\\
\end{tabular}
\end{center}
%
Both forms have slightly different effects as described above.
The main file is prepared as usual, see \secref{sec:include}.

%%%%%%%%%%%%%%%%%%%%%%%%%%%%%%%%%%%%%%%%%%%%%%%%%%%%%%%%%%%%%%%%%%%%%%%%%%%%%%%%
\subsection{Legacy Detection}
\label{sec:detection}

The directive |\childdocmain| in the main file can detect
whether the complete document or merely a child is to be compiled
even without using the directive |\childdocof|.
This method is deprecated because it is less robust
and there is no compelling reason to use it;
it is merely provided for backward compatibility
and it may be removed in future versions.

If the detection mechanism is to be used,
it is mandatory to correctly specify
the filename of the main file as the argument of |\childdocmain|:
%
\begin{center}
\begin{tabular}{l}
|\input{childdoc.def}|\\
|\childdocmain{|\textit{main}|}|\\
\end{tabular}
\end{center}
%
If |\jobname| does not match the argument \textit{main} of |\childdocmain|,
it is assumed that |\jobname| points to the child file to be compiled.
When using |\childdocmain| with the main file specified as argument,
it suffices to start a child file
with just |\input{|\textit{main}|}|
without loading of the package and using |\childdocof|.
If instead all processing is done
with the appropriate \textsf{childdoc} directives,
the argument of \textit{main} of |\childdocmain| can be empty.

An alternative version of the command line processing described
in \secref{sec:commandline} using the detection mechanism reads:
%
\begin{center}
|... -jobname "|\textit{target}|" "|[\textit{flags}]%
[|\def\jobname{|\textit{dest}|}|]|\input{|\textit{main}|}"|
\end{center}

%%%%%%%%%%%%%%%%%%%%%%%%%%%%%%%%%%%%%%%%%%%%%%%%%%%%%%%%%%%%%%%%%%%%%%%%%%%%%%%%
\subsection{Manual Code}
\label{sec:manual}

In case one cannot be certain whether the definitions file |childdoc.def|
is installed on the target \TeX{} distribution
and one prefers not to ship it,
it is conceivable to paste a few relevant commands into the sources.

To that end, drop all statements |\input{childdoc.def}|
and perform the replacements as outlined below.
Instead of |\childdocmain{|\textit{main}|}| add the following code
to the top of the main file:
%
\begin{center}
\begin{tabular}{l}
|\||ifdefined\childdocname\endinput\||fi\newif\ifchilddoc|\\
|\edef\childdocname{\scantokens\expandafter{\jobname\noexpand}}|\\
|\def\childdocmain{|\textit{main}|}\||ifx\childdocmain\childdocname\||else|\\
|\childdoctrue\includeonly{\childdocname}\let\jobname\childdocmain\||fi|\\
\end{tabular}
\end{center}
%
Instead of |\childdocof{|\textit{main}|}| just include the main file
at the top of each child file:
%
\begin{center}
|\input{|\textit{main}|}|
\end{center}
%
A simple redirection |\childdocforward{|\textit{dest}|}| is achieved by:
%
\begin{center}
|\def\jobname{|\textit{dest}|}\input{\jobname}|
\end{center}
%
The redirection with prefix
|\childdocforwardprefix[|\textit{prefix}|]{|\textit{dest}|}|
is accomplished by:
%
\begin{center}
\begin{tabular}{l}
|{\edef\jobname{\scantokens\expandafter{\jobname\noexpand}}|\\
|\def\redirectjob |\textit{prefix}|#1~~~{\gdef\jobname{|\textit{dest}|#1}}|\\
|\expandafter\redirectjob\jobname~~~}\input{\jobname}|
\end{tabular}
\end{center}

In an alternative approach,
child documents can be compiled by a specific command line
without additional code or specific definitions:
%
\begin{center}
|... -jobname "|\textit{target}|" "|[\textit{flags}]%
|\includeonly{|\textit{dest}|}\input{|\textit{main}|}"|
\end{center}
%

%%%%%%%%%%%%%%%%%%%%%%%%%%%%%%%%%%%%%%%%%%%%%%%%%%%%%%%%%%%%%%%%%%%%%%%%%%%%%%%%
%%%%%%%%%%%%%%%%%%%%%%%%%%%%%%%%%%%%%%%%%%%%%%%%%%%%%%%%%%%%%%%%%%%%%%%%%%%%%%%%
\section{Information}

%%%%%%%%%%%%%%%%%%%%%%%%%%%%%%%%%%%%%%%%%%%%%%%%%%%%%%%%%%%%%%%%%%%%%%%%%%%%%%%%
\subsection{Copyright}

Copyright \copyright{} 2017--2018 Niklas Beisert

This work may be distributed and/or modified under the
conditions of the \LaTeX{} Project Public License, either version 1.3
of this license or (at your option) any later version.
The latest version of this license is in
  \url{http://www.latex-project.org/lppl.txt}
and version 1.3 or later is part of all distributions of \LaTeX{}
version 2005/12/01 or later.

This work has the LPPL maintenance status `maintained'.

The Current Maintainer of this work is Niklas Beisert.

This work consists of the files |README.txt|, |childdoc.ins| and |childdoc.dtx|
as well as the derived files |childdoc.def|, |cdocsamp.tex|
with |cdocsch1.tex|, |cdocsch2.tex|, |cdocspt3.tex|, |cdocspt4.tex|,
|cdocsdrf.tex|, |cdocsfn1.tex|, |cdocsfn2.tex|
as well as |childdoc.pdf|.

%%%%%%%%%%%%%%%%%%%%%%%%%%%%%%%%%%%%%%%%%%%%%%%%%%%%%%%%%%%%%%%%%%%%%%%%%%%%%%%%
\subsection{Files and Installation}

The package consists of the files:
%
\begin{center}
\begin{tabular}{ll}
    |README.txt|   & readme file \\
    |childdoc.ins| & installation file \\
    |childdoc.dtx| & source file \\
    |childdoc.def| & definition file \\
    |cdocsamp.tex| & sample main file \\
    |cdocsch1.tex| & sample include file \\
    |cdocsch2.tex| & sample include file \\
    |cdocspt3.tex| & sample part file \\
    |cdocspt4.tex| & sample part file \\
    |cdocsdrf.tex| & sample redirection file \\
    |cdocsfn1.tex| & sample redirection file \\
    |cdocsfn2.tex| & sample redirection file \\
    |childdoc.pdf| & manual
\end{tabular}
\end{center}
%
The distribution consists of the files
|README.txt|, |childdoc.ins| and |childdoc.dtx|.
%
\begin{itemize}
\item
Run (pdf)\LaTeX{} on |childdoc.dtx|
to compile the manual |childdoc.pdf| (this file).
\item
Run \LaTeX{} on |childdoc.ins| to create the definitions file |childdoc.def|
and the sample |cdocsamp.tex| with include files
|cdocsch1.tex|, |cdocsch2.tex|, |cdocspt3.tex|, |cdocspt4.tex|,
|cdocsdrf.tex|, |cdocsfn1.tex|, |cdocsfn2.tex|.
Then copy the file |childdoc.def| to an appropriate directory of your \LaTeX{}
distribution, e.g.\ \textit{texmf-root}|/tex/latex/childdoc|.
\end{itemize}

%%%%%%%%%%%%%%%%%%%%%%%%%%%%%%%%%%%%%%%%%%%%%%%%%%%%%%%%%%%%%%%%%%%%%%%%%%%%%%%%
\subsection{Related CTAN Packages}

There are several other packages which offer a similar functionality:
%
\begin{itemize}
\item
The packages
\href{http://ctan.org/pkg/docmute}{\textsf{docmute}},
\href{http://ctan.org/pkg/includex}{\textsf{includex}} and
\href{http://ctan.org/pkg/standalone}{\textsf{standalone}}
provide commands to include only the document body of
a child file thus allowing both files to be compiled individually.
\item
The packages \href{http://ctan.org/pkg/subdocs}{\textsf{subdocs}}
and \href{http://ctan.org/pkg/subfiles}{\textsf{subfiles}}
provide structures in which the main and child documents can be
encapsulated and allowing them to be compiled individually.
The inclusion mechanism is different from the conventional |\include|.
\item
The package \href{http://ctan.org/pkg/combine}{\textsf{combine}}
is an elaborate solution to combine several documents into one.
\end{itemize}
%
See also the CTAN topic \href{http://ctan.org/topic/subdocs}{\textsf{subdocs}}
for further related packages.
The present package differs from the above solutions in that
a document structure constructed with the conventional |\include| mechanism
just needs two extra commands at the top of every file
such that all constituent files can be compiled individually.

%%%%%%%%%%%%%%%%%%%%%%%%%%%%%%%%%%%%%%%%%%%%%%%%%%%%%%%%%%%%%%%%%%%%%%%%%%%%%%%%
%\subsection{Feature Suggestions}
%
%The following is a list of features which may be useful for future
%versions of this package:
%%
%\begin{itemize}
%\item
%\ldots
%\end{itemize}

%%%%%%%%%%%%%%%%%%%%%%%%%%%%%%%%%%%%%%%%%%%%%%%%%%%%%%%%%%%%%%%%%%%%%%%%%%%%%%%%
\subsection{Revision History}

%%%%%%%%%%%%%%%%%%%%%%%%%%%%%%%%%%%%%%%%
\paragraph{v2.0:} 2018/12/30

\begin{itemize}
\item
immediate forward processing
\item
added |\childdocby| mechanism
\item
manual restructured
\end{itemize}

%%%%%%%%%%%%%%%%%%%%%%%%%%%%%%%%%%%%%%%%
\paragraph{v1.6:} 2018/01/17

\begin{itemize}
\item
application for development of include files
\item
corrections to manual
\end{itemize}

%%%%%%%%%%%%%%%%%%%%%%%%%%%%%%%%%%%%%%%%
\paragraph{v1.5:} 2017/05/21

\begin{itemize}
\item
more complete structuring introduced
\item
|\childdocof| introduced
\item
|\childdoc| renamed to |\childdocmain|
\item
|\childredirect| renamed to |\childdocforward| and |\childdocforwardprefix|
and functionality expanded
\end{itemize}

%%%%%%%%%%%%%%%%%%%%%%%%%%%%%%%%%%%%%%%%
\paragraph{v1.0:} 2017/04/27

\begin{itemize}
\item
manual and install package
\item
first version published on CTAN
\end{itemize}

%%%%%%%%%%%%%%%%%%%%%%%%%%%%%%%%%%%%%%%%
\paragraph{v0.6:} 2017/04/26

\begin{itemize}
\item
redirection mechanism added
\end{itemize}

%%%%%%%%%%%%%%%%%%%%%%%%%%%%%%%%%%%%%%%%
\paragraph{v0.5:} 2017/04/26

\begin{itemize}
\item
functionality in definition file
\end{itemize}


%%%%%%%%%%%%%%%%%%%%%%%%%%%%%%%%%%%%%%%%%%%%%%%%%%%%%%%%%%%%%%%%%%%%%%%%%%%%%%%%
%%%%%%%%%%%%%%%%%%%%%%%%%%%%%%%%%%%%%%%%%%%%%%%%%%%%%%%%%%%%%%%%%%%%%%%%%%%%%%%%
%%%%%%%%%%%%%%%%%%%%%%%%%%%%%%%%%%%%%%%%%%%%%%%%%%%%%%%%%%%%%%%%%%%%%%%%%%%%%%%%
\appendix

\settowidth\MacroIndent{\rmfamily\scriptsize 000\ }

 \DocInput{childdoc.dtx}

\end{document}
%</driver>
% \fi
%
% %%%%%%%%%%%%%%%%%%%%%%%%%%%%%%%%%%%%%%%%%%%%%%%%%%%%%%%%%%%%%%%%%%%%%%%%%%%%%%
% %%%%%%%%%%%%%%%%%%%%%%%%%%%%%%%%%%%%%%%%%%%%%%%%%%%%%%%%%%%%%%%%%%%%%%%%%%%%%%
% \section{Sample}
%\iffalse
%<*samplemain>
%\fi
%
% The following presents a sample document
% with two chapters, two parts, a title page,
% a compile flag as well as three forwarding files to set the flag.
% It consists of eight |.tex| files:
% \begin{center}
% \begin{tabular}{ll}
% |cdocsamp.tex|&main file\\
% |cdocsch1.tex|&include file for chapter 1\\
% |cdocsch2.tex|&include file for chapter 2\\
% |cdocspt3.tex|&include file for part 3\\
% |cdocspt4.tex|&include file for part 4\\
% |cdocsdrf.tex|&forwarding file for main file in draft mode\\
% |cdocsfi1.tex|&forwarding file for final version of chapter 1\\
% |cdocsfi2.tex|&forwarding file for final version of chapter 2\\
% \end{tabular}
% \end{center}
% Each of the eight files can be compiled directly by the \LaTeX{} compiler.
%
% %%%%%%%%%%%%%%%%%%%%%%%%%%%%%%%%%%%%%%
% \paragraph{Main File.}
%
% The main file is called |cdocsamp.tex|.
%
% Load the \textsf{childdoc} definitions and
% declare the filename for the main document:
%    \begin{macrocode}
\input{childdoc.def}
\childdocmain{}
%    \end{macrocode}

% Optional override for |\version| flag:
%    \begin{macrocode}
%%\ifchilddoc\else\providecommand{\version}{draft}\fi
%    \end{macrocode}

% Define the default values for the |\version| flag
% (|final| for the main file and |draft| for childs):
%    \begin{macrocode}
\ifchilddoc
\providecommand{\version}{draft}
\else
\providecommand{\version}{final}
\fi
%    \end{macrocode}

% Load the standard document class:
%    \begin{macrocode}
\documentclass[12pt]{article}
%    \end{macrocode}

% Start the document body:
%    \begin{macrocode}
\begin{document}
%    \end{macrocode}

% Declare a title page.
% Print title, part of document being processed and version flag:
%    \begin{macrocode}
\addtocounter{page}{-1}
\begin{center}
{\LARGE\bfseries{}childdoc example\par}
\vspace{1cm}
\ifchilddoc
\ifchilddocmanual part\else chapter\fi:
`\childdocname' of `\childdocjob'\par
\else
main document: `\childdocjob'\par
\fi
version: \version\par
\end{center}
\newpage
%    \end{macrocode}

% Manually include selected file,
% otherwise process as usual:
%    \begin{macrocode}
\ifchilddocmanual
\section*{part `\childdocname'}
\input{\childdocname}
\else
%    \end{macrocode}

% Include the two chapters:
%    \begin{macrocode}
\include{cdocsch1}
\include{cdocsch2}
%    \end{macrocode}

% Include the two parts unless only chapters should be displayed:
%    \begin{macrocode}
\ifchilddoc\else
\section{part three}
\input{cdocspt3}
\section{part four}
\input{cdocspt4}
\fi
%    \end{macrocode}

% Process as usual until here:
%    \begin{macrocode}
\fi
%    \end{macrocode}

% End of document body:
%    \begin{macrocode}
\end{document}
%    \end{macrocode}
%\iffalse
%</samplemain>
%\fi
%
% %%%%%%%%%%%%%%%%%%%%%%%%%%%%%%%%%%%%%%
% \paragraph{Chapter Include Files.}
%
% The include files are called |cdocsch1.tex| and |cdocsch2.tex|.
%
%\iffalse
%<*samplechap1|samplechap2>
%\fi

% Optional override for |\version| flag:
%    \begin{macrocode}
%%\providecommand{\version}{final}
%    \end{macrocode}

% Include the main document:
%    \begin{macrocode}
\input{childdoc.def}
\childdocof{cdocsamp}
%    \end{macrocode}

%\iffalse
%</samplechap1|samplechap2>
%\fi
%
%\iffalse
%<*samplechap1>
%\fi
% Some text for chapter 1:
%    \begin{macrocode}
\section{one}
some text in chapter one
%    \end{macrocode}

%\iffalse
%</samplechap1>
%\fi
% Some text for chapter 2:
%\iffalse
%<*samplechap2>
%\fi
%    \begin{macrocode}
\section{two}
more text in chapter two
%    \end{macrocode}

%\iffalse
%</samplechap2>
%\fi
%
% %%%%%%%%%%%%%%%%%%%%%%%%%%%%%%%%%%%%%%
% \paragraph{Part Include Files.}
%
% The include files are called |cdocspt3.tex| and |cdocspt4.tex|.
%
%\iffalse
%<*samplepart3|samplepart4>
%\fi

% Optional override for |\version| flag:
%    \begin{macrocode}
%%\providecommand{\version}{final}
%    \end{macrocode}

% Include the main document:
%    \begin{macrocode}
\input{childdoc.def}
\childdocby{cdocsamp}
%    \end{macrocode}

%\iffalse
%</samplepart3|samplepart4>
%\fi
%
%\iffalse
%<*samplepart3>
%\fi
% Some text for part 3:
%    \begin{macrocode}
some text in part three
%    \end{macrocode}

%\iffalse
%</samplepart3>
%\fi
% Some text for part 4:
%\iffalse
%<*samplepart4>
%\fi
%    \begin{macrocode}
more text in part four
%    \end{macrocode}

%\iffalse
%</samplepart4>
%\fi
%
% %%%%%%%%%%%%%%%%%%%%%%%%%%%%%%%%%%%%%%
% \paragraph{Forwarding for a Complete Draft.}
%
% The following forwarding file |cdocsdrf.tex|
% compiles the main document in draft mode:
%\iffalse
%<*sampledraft>
%\fi
%    \begin{macrocode}
\def\version{draft}
\input{childdoc.def}
\childdocforward{cdocsamp}
%    \end{macrocode}

%\iffalse
%</sampledraft>
%\fi
%
% %%%%%%%%%%%%%%%%%%%%%%%%%%%%%%%%%%%%%%
% \paragraph{Forwarding for Final Version of the Chapters.}
%
% The following forwarding files |cdocsfn1.tex| and |cdocsfn2.tex|
% (with identical content)
% compile the final versions of the child documents
% |cdocsch1.tex| and |cdocsch2.tex|, respectively:
%\iffalse
%<*samplefinal>
%\fi
%    \begin{macrocode}
\def\version{final}
\input{childdoc.def}
\childdocforwardprefix[cdocsamp]{cdocsfn}{cdocsch}
%    \end{macrocode}

%\iffalse
%</samplefinal>
%\fi
%
% %%%%%%%%%%%%%%%%%%%%%%%%%%%%%%%%%%%%%%
% \paragraph{Command Line Processing.}
%
% The following three command lines generate the output files
% |cdocscld|, |cdocscl1| and |cdocscl2|
% which should be identical to
% |cdocsdrf|, |cdocsch1| and |cdocsfn2|, respectively:
% \begin{center}
% \begin{tabular}{l}
% |latex -jobname cdocscld \|\\
% |  "\def\version{draft}\input{childdoc.def}\childdocforward{cdocsamp}"|\\
% |latex -jobname cdocscl1 \|\\
% |  "\input{childdoc.def}\childdocforward[cdocsamp]{cdocsch1}"|\\
% |latex -jobname cdocscl2 \|\\
% |  "\def\version{final}\input{childdoc.def}\childdocforward{cdocsch2}"|
% \end{tabular}
% \end{center}
% Note that the trailing backslash on each first line
% merely continues the input to the second line
% (for convenient cut ant paste).
% Furthermore, the command |latex| can be replaced by any
% of its alternative versions such as |pdflatex|.
%
% %%%%%%%%%%%%%%%%%%%%%%%%%%%%%%%%%%%%%%%%%%%%%%%%%%%%%%%%%%%%%%%%%%%%%%%%%%%%%%
% %%%%%%%%%%%%%%%%%%%%%%%%%%%%%%%%%%%%%%%%%%%%%%%%%%%%%%%%%%%%%%%%%%%%%%%%%%%%%%
% \section{Implementation}
%\iffalse
%<*package>
%\fi
%
% This section describes the definitions file |childdoc.def|.

% The definitions cannot be loaded using |\usepackage| or |\RequirePackage|
% which has a mechanism to prevent loading a style file more than once.
% When loading the definitions by means of |\input|
% multiple instances have to be prevented manually:
%\iffalse
%This code needs to be before the `\ProvidesFile' directive
%which is defined at the beginning of this file.
%Therefore it is also placed there and commented out here.
%</package>
%<*discard>
%\fi
%    \begin{macrocode}
\ifdefined\childdocmain\endinput\fi
%    \end{macrocode}
%\iffalse
%</discard>
%<*package>
%\fi
%
% \macro{\ifchilddoc}
% \macro{\ifchilddocmanual}
% The conditional |\ifchilddoc| tells whether a
% child (true) or main (false) document is being compiled.
% The conditional |\ifchilddocmanual| tells whether
% the |\includeonly| mechanism is used (false) or
% the selection of child files must be performed manually (true).
% The definitions initialise to false:
%    \begin{macrocode}
\newif\ifchilddoc
\newif\ifchilddocmanual
%    \end{macrocode}

% \macro{\childdocname}
% \macro{\childdocjob}
% The macro |\childdocname| stores the name of the main document
% to be compiled. The macro |\childdocjob| stores the name of
% the document on which the \LaTeX{} compiler was originally invoked.
% The content of |\jobname| cannot be compared
% to filenames specified in the source due to different catcodes.
% The following code rescans |\jobname|, stores the result
% in |\childdocname| and saves a copy in |\childdocjob|:
%    \begin{macrocode}
\edef\childdocname{\scantokens\expandafter{\jobname\noexpand}}
\let\childdocjob\childdocname
%    \end{macrocode}

% \macro{\childdocdisable}
% The macro |\childdocdisable| prevents the main file
% from being processed more than once.
% At this stage, the main document command |\childdocmain|
% is assumed to be called once again where it should do nothing.
% Any subsequent call to it should prevent
% a secondary processing of the main document
% It overwrites the forwarding commands
% |\childdocof| and |\childdocforward|
% with empty macros to prevent further inclusions of the main document:
%    \begin{macrocode}
\newcommand{\childdocdisable}
{
  \renewcommand{\childdocmain}[1]{\renewcommand{\childdocmain}[1]{\endinput}}
  \renewcommand{\childdocof}[1]{}
  \renewcommand{\childdocby}[2][]{}
  \renewcommand{\childdocforward}[2][]{}
  \renewcommand{\childdocdisable}{}
}
%    \end{macrocode}

% \macro{\childdocmain}
% The macro |\childdocmain| is to be called at the top of the main file
% with nothing or the main filename (without extension) as argument.
% First, it breaks loops.
% If the argument is not empty and does not match |\childdocname|
% (which is set by the first inclusion of |childdoc.def|),
% |\ifchilddoc| is set to true, |\includeonly| is applied to the child file
% and |\jobname| is set to the main file
% (for proper handling of |.aux| files):
%    \begin{macrocode}
\newcommand{\childdocmain}[1]
{
  \childdocdisable\childdocmain{}
  \if?#1?\else
    \begingroup
      \def\childdoctmp{#1}
      \ifx\childdoctmp\childdocname
        \def\childdoctmp{}
      \else
        \def\childdoctmp
        {
          \childdoctrue
          \includeonly{\childdocname}
          \def\childdocjob{#1}
          \def\jobname{#1}
        }
      \fi
      \expandafter
    \endgroup
    \childdoctmp
  \fi
}
%    \end{macrocode}

% \macro{\childdocof}
% The command |\childdocof| redirects
% compilation to the main file |#1|.
%    \begin{macrocode}
\newcommand{\childdocof}[1]
{
  \childdocdisable
  \childdoctrue
  \includeonly{\childdocname}
  \def\jobname{#1}
  \def\childdocjob{#1}
  \input{#1}
}
%    \end{macrocode}

% \macro{\childdocby}
% The command |\childdocby| ....
%    \begin{macrocode}
\newcommand{\childdocby}[2][]
{
  \childdocdisable
  \childdoctrue
  \childdocmanualtrue
  \if?#1?\else
    \def\jobname{#2}
  \fi
  \def\childdocjob{#2}
  \input{#2}
  \endinput
}
%    \end{macrocode}

% \macro{\childdocforward}
% The command |\childdocforward| redirects
% compilation to the main file or
% (if the optional argument is given) a child file.
% Parameters are set as if the main file
% or a child file starting with |\childdocof| was compiled.
% Then compilation is handed over to the main file:
%    \begin{macrocode}
\newcommand{\childdocforward}[2][]
{
  \begingroup
    \if?#1?
      \def\childdoctmp
      {
        \def\childdocname{#2}
        \def\childdocjob{#2}
        \def\jobname{#2}
        \input{#2}
        \endinput
      }
    \else
      \def\childdoctmp
      {
        \childdocdisable
        \def\childdocname{#2}
        \childdoctrue
        \includeonly{#2}
        \def\childdocjob{#1}
        \def\jobname{#1}
        \input{#1}
        \endinput
      }
    \fi
    \expandafter
  \endgroup
  \childdoctmp
}
%    \end{macrocode}

% \macro{\childdocforwardprefix}
% The command |\childdocforwardprefix| redirects
% compilation to the main or a child file by means of a pattern.
% The prefix |#1| in the current filename is replaced by |#2|
% and the suffix of the current filename is kept
% (it is assumed that the filename does not contain the substring `|~~~|'
% which is used as a delimiter).
% Compilation is handed over to the new file by |\childdocforward|:
%    \begin{macrocode}
\newcommand{\childdocforwardprefix}[3][]
{
  \begingroup
    \def\childdocextract #2##1~~~{\def\childdoctmp{\childdocforward[#1]{#3##1}}}
    \expandafter\childdocextract\childdocname~~~
    \expandafter
  \endgroup
  \childdoctmp
}
%    \end{macrocode}

% \macro{\childdoc}
% The deprecated macro |\childdoc| is a legacy version of |\childdocmain|:
%    \begin{macrocode}
\newcommand{\childdoc}{\childdocmain}
%    \end{macrocode}

% \macro{\childdocredirect}
% The deprecated macro |\childdocredirect| is a legacy version
% of |\childdocforward| and |\childdocforwardprefix|:
%    \begin{macrocode}
\newcommand{\childdocredirect}[2][]
{
  \begingroup
    \if?#1?
      \def\childdoctmp{\childdocforward{#2}}
    \else
      \def\childdoctmp{\childdocforwardprefix{#1}{#2}}
    \fi
    \expandafter
  \endgroup
  \childdoctmp
}
%    \end{macrocode}

%\iffalse
%</package>
%\fi
%
\endinput
|\\
|\childdocforwardprefix{final}{child}|
\end{tabular}
\end{center}
%

Note that when several versions of a main file and/or of each child file
are to be generated, it may be convenient to set up a |Makefile| or
shell script to automatise the process.

%%%%%%%%%%%%%%%%%%%%%%%%%%%%%%%%%%%%%%%%%%%%%%%%%%%%%%%%%%%%%%%%%%%%%%%%%%%%%%%%
\subsection{Command Line Processing}
\label{sec:commandline}

The effect of redirection files can also be achieved by invoking
the \LaTeX{} compiler with a more elaborate command line.
Most conveniently this should be done as part
of a shell script or a |Makefile|.

When using \textsf{childdoc} in the main file, the following
command lines effectively perform a redirection
(note that depending on the shell being used,
backslashes may have to be doubled: `|\|' $\to$ `|\\|'):
%
\begin{center}
|... -jobname "|\textit{target}|" |\\|"|[\textit{flags}]%
|% \iffalse
%
% childdoc.dtx Copyright (C) 2017-2018 Niklas Beisert
%
% This work may be distributed and/or modified under the
% conditions of the LaTeX Project Public License, either version 1.3
% of this license or (at your option) any later version.
% The latest version of this license is in
%   http://www.latex-project.org/lppl.txt
% and version 1.3 or later is part of all distributions of LaTeX
% version 2005/12/01 or later.
%
% This work has the LPPL maintenance status `maintained'.
%
% The Current Maintainer of this work is Niklas Beisert.
%
% This work consists of the files childdoc.dtx and childdoc.ins
% and the derived files childdoc.def and cdocsamp.tex with
% cdocsch1.tex, cdocsch2.tex, cdocsdrf.tex, cdocsfn1.tex, cdocsfn2.tex.
%
%<package>\ifdefined\childdocmain\endinput\fi
%<package>\ProvidesFile{childdoc.def}[2018/12/30 v2.0 child document driver]
%<samplemain>\ProvidesFile{cdocsamp.tex}[2018/12/30 v2.0 sample for childdoc]
%<*driver>
%\ProvidesFile{childdoc.drv}[2018/12/30 v2.0 childdoc reference manual file]
\PassOptionsToClass{10pt,a4paper}{article}
\documentclass{ltxdoc}

\usepackage[margin=35mm]{geometry}
\usepackage{hyperref}
\usepackage{hyperxmp}
\usepackage[usenames]{color}

\hypersetup{colorlinks=true}
\hypersetup{pdfstartview=FitH}
\hypersetup{pdfpagemode=UseNone}
\hypersetup{pdfsource={}}
\hypersetup{pdflang={en-UK}}
\hypersetup{pdfcopyright={Copyright 2017-2018 Niklas Beisert.
  This work may be distributed and/or modified under the
  conditions of the LaTeX Project Public License, either version 1.3
  of this license or (at your option) any later version.}}
\hypersetup{pdflicenseurl={http://www.latex-project.org/lppl.txt}}
\hypersetup{pdfcontactaddress={ETH Zurich, ITP, HIT K,
  Wolfgang-Pauli-Strasse 27}}
\hypersetup{pdfcontactpostcode={8093}}
\hypersetup{pdfcontactcity={Zurich}}
\hypersetup{pdfcontactcountry={Switzerland}}
\hypersetup{pdfcontactemail={nbeisert@itp.phys.ethz.ch}}
\hypersetup{pdfcontacturl={http://people.phys.ethz.ch/\xmptilde nbeisert/}}

\newcommand{\secref}[1]{\hyperref[#1]{section \ref*{#1}}}

\parskip1ex
\parindent0pt
\let\olditemize\itemize
\def\itemize{\olditemize\parskip0pt}

\begin{document}

\title{The \textsf{childdoc} Package}
\hypersetup{pdftitle={The childdoc Package}}
\author{Niklas Beisert\\[2ex]
  Institut f\"ur Theoretische Physik\\
  Eidgen\"ossische Technische Hochschule Z\"urich\\
  Wolfgang-Pauli-Strasse 27, 8093 Z\"urich, Switzerland\\[1ex]
  \href{mailto:nbeisert@itp.phys.ethz.ch}
  {\texttt{nbeisert@itp.phys.ethz.ch}}}
\hypersetup{pdfauthor={Niklas Beisert}}
\hypersetup{pdfsubject={Manual for the LaTeX2e Package childdoc}}
\date{30 December 2018, \textsf{v2.0}}
\maketitle

\begin{abstract}\noindent
\textsf{childdoc} is a \LaTeXe{} package
that enables the direct compilation
of document sections included by |\include|
to individual files.
\end{abstract}

\begingroup
\parskip0ex
\tableofcontents
\endgroup

%%%%%%%%%%%%%%%%%%%%%%%%%%%%%%%%%%%%%%%%%%%%%%%%%%%%%%%%%%%%%%%%%%%%%%%%%%%%%%%%
%%%%%%%%%%%%%%%%%%%%%%%%%%%%%%%%%%%%%%%%%%%%%%%%%%%%%%%%%%%%%%%%%%%%%%%%%%%%%%%%
\section{Introduction}

\LaTeX{} provides a mechanism to structure a large document (such as a book)
into a main file and several child files (containing the chapters)
using the |\include| command.
This mechanism is beneficial for documents
which span hundreds of pages in order to
make the source file(s) more manageable.
Moreover, compilation can be restricted to
selected child files by means of the |\includeonly| command.
The latter feature can be used to reduce the compilation time while editing
(this was significantly more useful in the earlier days of \LaTeX{})
or to generate a smaller document which is easier to navigate.
Another application of |\includeonly| is to generate
documents consisting of selected parts of the complete document.

However, there are a few drawbacks of the plain |\include| mechanism:
\begin{itemize}
\item
The child files cannot be compiled on their own,
they can only be compiled via the main file.
A naive editing environment
(such as a text editor with an option
to have the current file processed by \LaTeX)
may require one to switch to the main file before compiling;
attempting to compile the child file produces errors.
\item
The main file must be modified (each time)
to adjust the |\includeonly| command
to the present needs. This easily leaves the main file in a messy state.
\item
The generated document will always carry the filename
of the main document. This is inconvenient if
several child files are to be compiled and
to be kept for distribution.
\end{itemize}

The present package provides a simple interface
to make child files individually compilable by \LaTeX{}.
Compiling a child file then has the same effect as compiling
the main file with an |\includeonly| command
to select the appropriate child.
Moreover the generated document will carry the name of the child
rather than the main file.
This resolves all three above issues.

This feature is meant to make the editing of books,
thesis documents and lecture notes somewhat more convenient.
However, the package can also be used efficiently for
composing a series of documents (such as exercise sheets)
which are typically distributed individually.
It then assists the author in generating the individual documents
(potentially in different versions)
as well as a document containing the collected series.
Another application is in developing style files
or other kinds of included material
where compilation of the style file could redirect
to a sample or test file.

%%%%%%%%%%%%%%%%%%%%%%%%%%%%%%%%%%%%%%%%%%%%%%%%%%%%%%%%%%%%%%%%%%%%%%%%%%%%%%%%
%%%%%%%%%%%%%%%%%%%%%%%%%%%%%%%%%%%%%%%%%%%%%%%%%%%%%%%%%%%%%%%%%%%%%%%%%%%%%%%%
\section{Usage}

First of all, the package \textsf{childdoc} is \emph{not} a standard
\LaTeXe{} |.sty| style file! Therefore it needs to be invoked in
a non-standard way.

%%%%%%%%%%%%%%%%%%%%%%%%%%%%%%%%%%%%%%%%%%%%%%%%%%%%%%%%%%%%%%%%%%%%%%%%%%%%%%%%
\subsection{Included Files}
\label{sec:include}

%%%%%%%%%%%%%%%%%%%%%%%%%%%%%%%%%%%%%%%%
\DescribeMacro{\childdocmain}
To use the package, add the commands
\begin{center}
\begin{tabular}{l}
|\input{childdoc.def}|\\
|\childdocmain{}|\\
\end{tabular}
\end{center}
at the very top of the main \LaTeX{} file,
in particular \emph{before} the |\documentclass| statement!
The argument of |\childdocmain| should be left empty
(but it must be present).

%%%%%%%%%%%%%%%%%%%%%%%%%%%%%%%%%%%%%%%%
\DescribeMacro{\childdocof}
Furthermore, add the commands
\begin{center}
\begin{tabular}{l}
|\input{childdoc.def}|\\
|\childdocof{|\textit{main}|}|\\
\end{tabular}
\end{center}
at the top of every child file \textit{child}
which is included by |\include{|\textit{child}|}|
from within the main file
(or at least for those files to be compiled individually).
The argument \textit{main} must be the filename of the main file.

There are a couple of
considerations in setting up the main and child documents:

%%%%%%%%%%%%%%%%%%%%%%%%%%%%%%%%%%%%%%%%
\paragraph{Restrictions.}

Please note the following restrictions:
\begin{itemize}
\item
|\childdocmain| must be called with one argument \textit{main}
to ensure compatibility with earlier version of the package.
It must either be empty (|\childdocmain{}|)
or precisely match the filename of the main file in which it is specified.
See \secref{sec:detection} for further information.
\item
The filename \textit{main} must be specified without the |.tex| extension.
\item
The filename \textit{main} is case sensitive
(even in case-insensitive file systems)
due to internal string comparison.
\item
The argument \textit{main} should be fully expanded, it cannot be a macro.
\item
Subdirectories and special characters should be avoided in filenames.
\item
The command |\childdocmain{|\textit{main}|}| must be followed by a whitespace.
It should not be followed immediately by another command
or by a comment mark `|%|'.
This is because the \TeX{} parser reads the token immediately following
the argument of |\childdocmain| and puts it
at the beginning of every child section;
however, a white\-space is ignored.
\end{itemize}

%%%%%%%%%%%%%%%%%%%%%%%%%%%%%%%%%%%%%%%%
\paragraph{Content of Main File.}

It is advisable to place all content in the child files included by |\include|.
Any output contained in the main file will appear in all child documents
unless suppressed manually;
it cannot be suppressed automatically by the |\includeonly| directive
and thus should normally be avoided.
A method to include some content in the main file
by means of conditional processing is described in \secref{sec:conditional}.

%%%%%%%%%%%%%%%%%%%%%%%%%%%%%%%%%%%%%%%%
\paragraph{Page Numbering.}

When only a part of the document is compiled,
the appropriate numbering of pages
(as well as other status parameters)
is determined from the |.aux| files.
The latter contain information from previous passes.
However this information needs to propagate through
all intermediate child documents.
Therefore the page numbering in child documents may well
be inconsistent until the complete document is compiled at least once.

A useful (if unconventional) way to always ensure a consistent
page numbering is to restart the numbering in each child document
and denote the pages by `\textit{child}|.|\textit{page}'
where \textit{child} represents the chapter/section number of the child file.
This can be achieved by the command
|\numberwithin{page}{|\textit{child}|}|
of the \textsf{amsmath} package
where \textit{child} can be |chapter| or |section|
depending on the chosen structuring.
Alternatively, one can modify the macro |\thepage| appropriately
and reset the counter |page| at the start of each child file.

%%%%%%%%%%%%%%%%%%%%%%%%%%%%%%%%%%%%%%%%%%%%%%%%%%%%%%%%%%%%%%%%%%%%%%%%%%%%%%%%
\subsection{Conditional Processing}
\label{sec:conditional}

The package provides a mechanism to compile different versions
of a document. To customise the versions further some conditional processing
can come in handy to distinguish which version is being compiled.
The package provides two macros to describe the compilation context:

%%%%%%%%%%%%%%%%%%%%%%%%%%%%%%%%%%%%%%%%
\DescribeMacro{\ifchilddoc}
The conditional |\ifchilddoc| distinguishes between the compilation of
child documents and the main document:
%
\begin{center}
|\ifchilddoc |\textit{child-code}| |[|\||else |\textit{main-code}]| \||fi|
\end{center}

%%%%%%%%%%%%%%%%%%%%%%%%%%%%%%%%%%%%%%%%
\DescribeMacro{\childdocname}
\DescribeMacro{\childdocjob}
The macro |\childdocname| contains the filename (without extension)
of the main or child file being processed.
Note that |\childdocjob| will always contain the name of the main file.

%%%%%%%%%%%%%%%%%%%%%%%%%%%%%%%%%%%%%%%%
\paragraph{Title Page.}

Conditional processing can be used to include a title or banner page
in the main document when proper precautions are taken.
Importantly, the code in the main file should ensure that the page counter
(as well as other status parameters which are stored in the |.aux| files)
takes the same value after the conditional processing.
Otherwise the page numbers may take divergent values
depending on which part is compiled.

For example, a title page could be declared by:
%
\begin{center}
\begin{tabular}{l}
|\ifchilddoc\||else|\\
|\addtocounter{page}{-1}|\\
\textit{code for title page}\\
|\newpage|\\
|\||fi|
\end{tabular}
\end{center}
%
A banner page for the child documents can be generated by:
%
\begin{center}
\begin{tabular}{l}
|\ifchilddoc|\\
|\addtocounter{page}{-1}|\\
\textit{code for banner page}\\
|\newpage|\\
|\||fi|
\end{tabular}
\end{center}
%
Here one could write a message such as:
\begin{center}
|This is the part \childdocname{} of \childdocjob{}.|
\end{center}

%%%%%%%%%%%%%%%%%%%%%%%%%%%%%%%%%%%%%%%%%%%%%%%%%%%%%%%%%%%%%%%%%%%%%%%%%%%%%%%%
\subsection{Flags}
\label{sec:flags}

The package makes it easy to generate different versions
of the main or child documents.
To this end compilation flags can be defined
and assigned different default values.
They will be particularly useful in conjunction
with the forwarding mechanism described in \secref{sec:forward}.

For example, it may be useful to have a flag |\version|
which can be set to |draft| or |final|.
The document source will contain some conditional code
depending on the value of |\version|.
Suppose further, the flag should default to |final| for the main file
and to |draft| for child files
which is a natural assignment for editing the document.
This is achieved by placing the following code
in the preamble of the main document
(below the |\childdocmain| directive):
%
\begin{center}
\begin{tabular}{l}
|\ifchilddoc|\\
|\providecommand{\version}{draft}|\\
|\||else|\\
|\providecommand{\version}{final}|\\
|\||fi|
\end{tabular}
\end{center}
%
The definition by |\providecommand| makes sure
that previous definitions are not overwritten.
Further statements |\providecommand{\version}{...}|
can thus be added before the above code to override it.

For the main file, one might add a line
(between |\childdocmain| and the above block)
%
\begin{center}
|%\ifchilddoc\||else\providecommand{\version}{draft}\||fi|
\end{center}
%
which can be uncommented to produce a draft version.
Likewise one can add a line to the very top of a child file
(above the |\childdocof{|\textit{main}|}| directive)
%
\begin{center}
|%\providecommand{\version}{final}|
\end{center}
%
which can be uncommented to produce the final version of this child document.

%%%%%%%%%%%%%%%%%%%%%%%%%%%%%%%%%%%%%%%%%%%%%%%%%%%%%%%%%%%%%%%%%%%%%%%%%%%%%%%%
\subsection{Forwarding}
\label{sec:forward}

Different versions of the main or child documents
using compilation flags as described in \secref{sec:flags}
can be (permanently) stored in different files
for convenient compilation, viewing and distribution.
To this end, the package defines a command
to pass on compilation to a different file:

%%%%%%%%%%%%%%%%%%%%%%%%%%%%%%%%%%%%%%%%
\DescribeMacro{\childdocforward}
The command |\childdocforward| redirects processing to
another source file:
%
\begin{center}
\begin{tabular}{l}
|\input{childdoc.def}|\\
|\childdocforward[|\textit{main}|]{|\textit{dest}|}|\\
\end{tabular}
\end{center}
%
The argument \textit{dest} is the destination file
(without extension).
It should be the main file or one of the child files.
Note that further \textsf{childdoc} directives
such as |\childdocof| and |\childdocforward|
in the indicated file will be processed in this form.
The optional argument \textit{main}
passes on directly to the main file \textit{main}
while pretending to compile the child \textit{dest}.
This form behaves as if \textit{dest}
issues |\childdocof{|\textit{main}|}| right away,
and no further \textsf{childdoc} directives will be processed.

%%%%%%%%%%%%%%%%%%%%%%%%%%%%%%%%%%%%%%%%
\DescribeMacro{\...prefix}
In the alternative form |\childdocforwardprefix|,
%
\begin{center}
\begin{tabular}{l}
|\input{childdoc.def}|\\
|\childdocforwardprefix[|\textit{main}|]{|\textit{prefix}|}{|\textit{dest}|}|
\end{tabular}
\end{center}
%
the destination file is determined by a pattern
depending on the current file:
To make this work, the current file must be called
`{\textit{prefix}\hspace{0.2em}\textit{suffix}}'
with \textit{prefix} matching precisely the argument.
Processing is then passed on to the file
`{\textit{dest}\hspace{0.2em}\textit{suffix}}'.
Surely, the same effect is achieved by
directly specifying the
argument `{\textit{dest}\hspace{0.2em}\textit{suffix}}'
in the first form.
However, that requires to set up a different file
for each child. With the alternative form of the command
all these files can have exactly the same content
which simplifies setting them up and maintaining them.

For example, the following file |draft.tex|
with a compilation flag |\version| as described in \secref{sec:flags}
compiles the main document as a draft:
%
\begin{center}
\begin{tabular}{l}
|\def\version{draft}|\\
|\input{childdoc.def}|\\
|\childdocforward{|\textit{main}|}|
\end{tabular}
\end{center}
%
Likewise, the following files |final|\textit{nn}|.tex|
compile the final version of the child document
|child|\textit{nn}|.tex|:
%
\begin{center}
\begin{tabular}{l}
|\def\version{final}|\\
|\input{childdoc.def}|\\
|\childdocforwardprefix{final}{child}|
\end{tabular}
\end{center}
%

Note that when several versions of a main file and/or of each child file
are to be generated, it may be convenient to set up a |Makefile| or
shell script to automatise the process.

%%%%%%%%%%%%%%%%%%%%%%%%%%%%%%%%%%%%%%%%%%%%%%%%%%%%%%%%%%%%%%%%%%%%%%%%%%%%%%%%
\subsection{Command Line Processing}
\label{sec:commandline}

The effect of redirection files can also be achieved by invoking
the \LaTeX{} compiler with a more elaborate command line.
Most conveniently this should be done as part
of a shell script or a |Makefile|.

When using \textsf{childdoc} in the main file, the following
command lines effectively perform a redirection
(note that depending on the shell being used,
backslashes may have to be doubled: `|\|' $\to$ `|\\|'):
%
\begin{center}
|... -jobname "|\textit{target}|" |\\|"|[\textit{flags}]%
|\input{childdoc.def}\childdocforward[|\textit{main}|]{|\textit{dest}|}"|
\end{center}
%
Here \textit{target} is the name of the output file,
\textit{main} is the name of the main file
and \textit{dest} is the name of the main or child file to be processed
(all filenames without extensions).
The optional argument \textit{main} can be omitted
if \textit{main} matches \textit{dest}.
Optionally, compilation \textit{flags} can be defined via |\def| commands.
This command line makes the \TeX{} engine believe
it is compiling the file \textit{target}
whose content is specified as the latter parameter.
The provided code then forwards the processing to
\textit{main} or \textit{dest} as described in \secref{sec:forward}.

%%%%%%%%%%%%%%%%%%%%%%%%%%%%%%%%%%%%%%%%%%%%%%%%%%%%%%%%%%%%%%%%%%%%%%%%%%%%%%%%
\subsection{Include by Input}
\label{sec:input}

Including child documents by |\include| has some restrictions by design.
Most notably, the content of a child document always occupies
its own set of pages; pages cannot be shared between child documents.
Usually, this behaviour makes perfect sense
because each child document contain an essential part of the document.
However, in some situations it may be desirable to compose
a document from a collection of parts
without having mandatory page breaks between then.
For this case, the package
provides a mechanism to include parts
by |\input| which can also be processed individually.
However, by construction this mechanism
requires manual handling of the content to be output.

%%%%%%%%%%%%%%%%%%%%%%%%%%%%%%%%%%%%%%%%
\DescribeMacro{\ifchilddocmanual}
The main file should be prepared as usual, see \secref{sec:include}.
However, the document body must make a distinction
between processing of an individual part and of the main document, e.g.:
%
\begin{center}
\begin{tabular}{l}
|\ifchilddocmanual|\\
|\input{\childdocname}|\\
|\||else|\\
\textit{document body with }|\input{|\textit{part}|}|\\
|\||fi|
\end{tabular}
\end{center}
%
The conditional |\ifchilddocmanual| is true whenever
a part to be included by |\input| is being compiled,
and the name of the part is stored in |\childdocname|.

%%%%%%%%%%%%%%%%%%%%%%%%%%%%%%%%%%%%%%%%
\DescribeMacro{\childdocby}
Each part to be included by |\input| should start with:
%
\begin{center}
\begin{tabular}{l}
|\input{childdoc.def}|\\
|\childdocby{|\textit{main}|}|\\
\end{tabular}
\end{center}
%
The directive |\childdocby| is similar to |\childdocof|
described in \secref{sec:include},
but the subsequent selection of content must be done manually.
To that end, both |\ifchilddoc| and |\ifchilddocmanual|
will be true upon processing of a part,
and the name of the part is stored in |\childdocname|.
Note that |\jobname| will be set to the filename of the current part
so that each part receives an individual |.aux| file
that does not interfere with the |.aux| file(s) of the main document.
This behaviour can be altered by the alternative form
|\childdocby[*]{|\textit{main}|}| (with a non-empty optional argument)
which uses the |.aux| file of the main document
by setting |\jobname| to \textit{main}.

%%%%%%%%%%%%%%%%%%%%%%%%%%%%%%%%%%%%%%%%%%%%%%%%%%%%%%%%%%%%%%%%%%%%%%%%%%%%%%%%
\subsection{Driver Development}
\label{sec:driver}

The \textsf{childdoc} mechanism can also be use for the development
of definition files such as \LaTeX{} styles or classes.
This case differs from the above setup with multiple parts
included by |\include| in that no |\includeonly| should be invoked.
This can be achieved by starting the include file
(before |\ProvidesPackage|) with:
%
\begin{center}
\begin{tabular}{l}
|\input{childdoc.def}|\\
|\childdocforward{|\textit{main}|}|\\
\end{tabular}
\end{center}
%
or alternatively with:
%
\begin{center}
\begin{tabular}{l}
|\input{childdoc.def}|\\
|\childdocby{|\textit{main}|}|\\
\end{tabular}
\end{center}
%
Both forms have slightly different effects as described above.
The main file is prepared as usual, see \secref{sec:include}.

%%%%%%%%%%%%%%%%%%%%%%%%%%%%%%%%%%%%%%%%%%%%%%%%%%%%%%%%%%%%%%%%%%%%%%%%%%%%%%%%
\subsection{Legacy Detection}
\label{sec:detection}

The directive |\childdocmain| in the main file can detect
whether the complete document or merely a child is to be compiled
even without using the directive |\childdocof|.
This method is deprecated because it is less robust
and there is no compelling reason to use it;
it is merely provided for backward compatibility
and it may be removed in future versions.

If the detection mechanism is to be used,
it is mandatory to correctly specify
the filename of the main file as the argument of |\childdocmain|:
%
\begin{center}
\begin{tabular}{l}
|\input{childdoc.def}|\\
|\childdocmain{|\textit{main}|}|\\
\end{tabular}
\end{center}
%
If |\jobname| does not match the argument \textit{main} of |\childdocmain|,
it is assumed that |\jobname| points to the child file to be compiled.
When using |\childdocmain| with the main file specified as argument,
it suffices to start a child file
with just |\input{|\textit{main}|}|
without loading of the package and using |\childdocof|.
If instead all processing is done
with the appropriate \textsf{childdoc} directives,
the argument of \textit{main} of |\childdocmain| can be empty.

An alternative version of the command line processing described
in \secref{sec:commandline} using the detection mechanism reads:
%
\begin{center}
|... -jobname "|\textit{target}|" "|[\textit{flags}]%
[|\def\jobname{|\textit{dest}|}|]|\input{|\textit{main}|}"|
\end{center}

%%%%%%%%%%%%%%%%%%%%%%%%%%%%%%%%%%%%%%%%%%%%%%%%%%%%%%%%%%%%%%%%%%%%%%%%%%%%%%%%
\subsection{Manual Code}
\label{sec:manual}

In case one cannot be certain whether the definitions file |childdoc.def|
is installed on the target \TeX{} distribution
and one prefers not to ship it,
it is conceivable to paste a few relevant commands into the sources.

To that end, drop all statements |\input{childdoc.def}|
and perform the replacements as outlined below.
Instead of |\childdocmain{|\textit{main}|}| add the following code
to the top of the main file:
%
\begin{center}
\begin{tabular}{l}
|\||ifdefined\childdocname\endinput\||fi\newif\ifchilddoc|\\
|\edef\childdocname{\scantokens\expandafter{\jobname\noexpand}}|\\
|\def\childdocmain{|\textit{main}|}\||ifx\childdocmain\childdocname\||else|\\
|\childdoctrue\includeonly{\childdocname}\let\jobname\childdocmain\||fi|\\
\end{tabular}
\end{center}
%
Instead of |\childdocof{|\textit{main}|}| just include the main file
at the top of each child file:
%
\begin{center}
|\input{|\textit{main}|}|
\end{center}
%
A simple redirection |\childdocforward{|\textit{dest}|}| is achieved by:
%
\begin{center}
|\def\jobname{|\textit{dest}|}\input{\jobname}|
\end{center}
%
The redirection with prefix
|\childdocforwardprefix[|\textit{prefix}|]{|\textit{dest}|}|
is accomplished by:
%
\begin{center}
\begin{tabular}{l}
|{\edef\jobname{\scantokens\expandafter{\jobname\noexpand}}|\\
|\def\redirectjob |\textit{prefix}|#1~~~{\gdef\jobname{|\textit{dest}|#1}}|\\
|\expandafter\redirectjob\jobname~~~}\input{\jobname}|
\end{tabular}
\end{center}

In an alternative approach,
child documents can be compiled by a specific command line
without additional code or specific definitions:
%
\begin{center}
|... -jobname "|\textit{target}|" "|[\textit{flags}]%
|\includeonly{|\textit{dest}|}\input{|\textit{main}|}"|
\end{center}
%

%%%%%%%%%%%%%%%%%%%%%%%%%%%%%%%%%%%%%%%%%%%%%%%%%%%%%%%%%%%%%%%%%%%%%%%%%%%%%%%%
%%%%%%%%%%%%%%%%%%%%%%%%%%%%%%%%%%%%%%%%%%%%%%%%%%%%%%%%%%%%%%%%%%%%%%%%%%%%%%%%
\section{Information}

%%%%%%%%%%%%%%%%%%%%%%%%%%%%%%%%%%%%%%%%%%%%%%%%%%%%%%%%%%%%%%%%%%%%%%%%%%%%%%%%
\subsection{Copyright}

Copyright \copyright{} 2017--2018 Niklas Beisert

This work may be distributed and/or modified under the
conditions of the \LaTeX{} Project Public License, either version 1.3
of this license or (at your option) any later version.
The latest version of this license is in
  \url{http://www.latex-project.org/lppl.txt}
and version 1.3 or later is part of all distributions of \LaTeX{}
version 2005/12/01 or later.

This work has the LPPL maintenance status `maintained'.

The Current Maintainer of this work is Niklas Beisert.

This work consists of the files |README.txt|, |childdoc.ins| and |childdoc.dtx|
as well as the derived files |childdoc.def|, |cdocsamp.tex|
with |cdocsch1.tex|, |cdocsch2.tex|, |cdocspt3.tex|, |cdocspt4.tex|,
|cdocsdrf.tex|, |cdocsfn1.tex|, |cdocsfn2.tex|
as well as |childdoc.pdf|.

%%%%%%%%%%%%%%%%%%%%%%%%%%%%%%%%%%%%%%%%%%%%%%%%%%%%%%%%%%%%%%%%%%%%%%%%%%%%%%%%
\subsection{Files and Installation}

The package consists of the files:
%
\begin{center}
\begin{tabular}{ll}
    |README.txt|   & readme file \\
    |childdoc.ins| & installation file \\
    |childdoc.dtx| & source file \\
    |childdoc.def| & definition file \\
    |cdocsamp.tex| & sample main file \\
    |cdocsch1.tex| & sample include file \\
    |cdocsch2.tex| & sample include file \\
    |cdocspt3.tex| & sample part file \\
    |cdocspt4.tex| & sample part file \\
    |cdocsdrf.tex| & sample redirection file \\
    |cdocsfn1.tex| & sample redirection file \\
    |cdocsfn2.tex| & sample redirection file \\
    |childdoc.pdf| & manual
\end{tabular}
\end{center}
%
The distribution consists of the files
|README.txt|, |childdoc.ins| and |childdoc.dtx|.
%
\begin{itemize}
\item
Run (pdf)\LaTeX{} on |childdoc.dtx|
to compile the manual |childdoc.pdf| (this file).
\item
Run \LaTeX{} on |childdoc.ins| to create the definitions file |childdoc.def|
and the sample |cdocsamp.tex| with include files
|cdocsch1.tex|, |cdocsch2.tex|, |cdocspt3.tex|, |cdocspt4.tex|,
|cdocsdrf.tex|, |cdocsfn1.tex|, |cdocsfn2.tex|.
Then copy the file |childdoc.def| to an appropriate directory of your \LaTeX{}
distribution, e.g.\ \textit{texmf-root}|/tex/latex/childdoc|.
\end{itemize}

%%%%%%%%%%%%%%%%%%%%%%%%%%%%%%%%%%%%%%%%%%%%%%%%%%%%%%%%%%%%%%%%%%%%%%%%%%%%%%%%
\subsection{Related CTAN Packages}

There are several other packages which offer a similar functionality:
%
\begin{itemize}
\item
The packages
\href{http://ctan.org/pkg/docmute}{\textsf{docmute}},
\href{http://ctan.org/pkg/includex}{\textsf{includex}} and
\href{http://ctan.org/pkg/standalone}{\textsf{standalone}}
provide commands to include only the document body of
a child file thus allowing both files to be compiled individually.
\item
The packages \href{http://ctan.org/pkg/subdocs}{\textsf{subdocs}}
and \href{http://ctan.org/pkg/subfiles}{\textsf{subfiles}}
provide structures in which the main and child documents can be
encapsulated and allowing them to be compiled individually.
The inclusion mechanism is different from the conventional |\include|.
\item
The package \href{http://ctan.org/pkg/combine}{\textsf{combine}}
is an elaborate solution to combine several documents into one.
\end{itemize}
%
See also the CTAN topic \href{http://ctan.org/topic/subdocs}{\textsf{subdocs}}
for further related packages.
The present package differs from the above solutions in that
a document structure constructed with the conventional |\include| mechanism
just needs two extra commands at the top of every file
such that all constituent files can be compiled individually.

%%%%%%%%%%%%%%%%%%%%%%%%%%%%%%%%%%%%%%%%%%%%%%%%%%%%%%%%%%%%%%%%%%%%%%%%%%%%%%%%
%\subsection{Feature Suggestions}
%
%The following is a list of features which may be useful for future
%versions of this package:
%%
%\begin{itemize}
%\item
%\ldots
%\end{itemize}

%%%%%%%%%%%%%%%%%%%%%%%%%%%%%%%%%%%%%%%%%%%%%%%%%%%%%%%%%%%%%%%%%%%%%%%%%%%%%%%%
\subsection{Revision History}

%%%%%%%%%%%%%%%%%%%%%%%%%%%%%%%%%%%%%%%%
\paragraph{v2.0:} 2018/12/30

\begin{itemize}
\item
immediate forward processing
\item
added |\childdocby| mechanism
\item
manual restructured
\end{itemize}

%%%%%%%%%%%%%%%%%%%%%%%%%%%%%%%%%%%%%%%%
\paragraph{v1.6:} 2018/01/17

\begin{itemize}
\item
application for development of include files
\item
corrections to manual
\end{itemize}

%%%%%%%%%%%%%%%%%%%%%%%%%%%%%%%%%%%%%%%%
\paragraph{v1.5:} 2017/05/21

\begin{itemize}
\item
more complete structuring introduced
\item
|\childdocof| introduced
\item
|\childdoc| renamed to |\childdocmain|
\item
|\childredirect| renamed to |\childdocforward| and |\childdocforwardprefix|
and functionality expanded
\end{itemize}

%%%%%%%%%%%%%%%%%%%%%%%%%%%%%%%%%%%%%%%%
\paragraph{v1.0:} 2017/04/27

\begin{itemize}
\item
manual and install package
\item
first version published on CTAN
\end{itemize}

%%%%%%%%%%%%%%%%%%%%%%%%%%%%%%%%%%%%%%%%
\paragraph{v0.6:} 2017/04/26

\begin{itemize}
\item
redirection mechanism added
\end{itemize}

%%%%%%%%%%%%%%%%%%%%%%%%%%%%%%%%%%%%%%%%
\paragraph{v0.5:} 2017/04/26

\begin{itemize}
\item
functionality in definition file
\end{itemize}


%%%%%%%%%%%%%%%%%%%%%%%%%%%%%%%%%%%%%%%%%%%%%%%%%%%%%%%%%%%%%%%%%%%%%%%%%%%%%%%%
%%%%%%%%%%%%%%%%%%%%%%%%%%%%%%%%%%%%%%%%%%%%%%%%%%%%%%%%%%%%%%%%%%%%%%%%%%%%%%%%
%%%%%%%%%%%%%%%%%%%%%%%%%%%%%%%%%%%%%%%%%%%%%%%%%%%%%%%%%%%%%%%%%%%%%%%%%%%%%%%%
\appendix

\settowidth\MacroIndent{\rmfamily\scriptsize 000\ }

 \DocInput{childdoc.dtx}

\end{document}
%</driver>
% \fi
%
% %%%%%%%%%%%%%%%%%%%%%%%%%%%%%%%%%%%%%%%%%%%%%%%%%%%%%%%%%%%%%%%%%%%%%%%%%%%%%%
% %%%%%%%%%%%%%%%%%%%%%%%%%%%%%%%%%%%%%%%%%%%%%%%%%%%%%%%%%%%%%%%%%%%%%%%%%%%%%%
% \section{Sample}
%\iffalse
%<*samplemain>
%\fi
%
% The following presents a sample document
% with two chapters, two parts, a title page,
% a compile flag as well as three forwarding files to set the flag.
% It consists of eight |.tex| files:
% \begin{center}
% \begin{tabular}{ll}
% |cdocsamp.tex|&main file\\
% |cdocsch1.tex|&include file for chapter 1\\
% |cdocsch2.tex|&include file for chapter 2\\
% |cdocspt3.tex|&include file for part 3\\
% |cdocspt4.tex|&include file for part 4\\
% |cdocsdrf.tex|&forwarding file for main file in draft mode\\
% |cdocsfi1.tex|&forwarding file for final version of chapter 1\\
% |cdocsfi2.tex|&forwarding file for final version of chapter 2\\
% \end{tabular}
% \end{center}
% Each of the eight files can be compiled directly by the \LaTeX{} compiler.
%
% %%%%%%%%%%%%%%%%%%%%%%%%%%%%%%%%%%%%%%
% \paragraph{Main File.}
%
% The main file is called |cdocsamp.tex|.
%
% Load the \textsf{childdoc} definitions and
% declare the filename for the main document:
%    \begin{macrocode}
\input{childdoc.def}
\childdocmain{}
%    \end{macrocode}

% Optional override for |\version| flag:
%    \begin{macrocode}
%%\ifchilddoc\else\providecommand{\version}{draft}\fi
%    \end{macrocode}

% Define the default values for the |\version| flag
% (|final| for the main file and |draft| for childs):
%    \begin{macrocode}
\ifchilddoc
\providecommand{\version}{draft}
\else
\providecommand{\version}{final}
\fi
%    \end{macrocode}

% Load the standard document class:
%    \begin{macrocode}
\documentclass[12pt]{article}
%    \end{macrocode}

% Start the document body:
%    \begin{macrocode}
\begin{document}
%    \end{macrocode}

% Declare a title page.
% Print title, part of document being processed and version flag:
%    \begin{macrocode}
\addtocounter{page}{-1}
\begin{center}
{\LARGE\bfseries{}childdoc example\par}
\vspace{1cm}
\ifchilddoc
\ifchilddocmanual part\else chapter\fi:
`\childdocname' of `\childdocjob'\par
\else
main document: `\childdocjob'\par
\fi
version: \version\par
\end{center}
\newpage
%    \end{macrocode}

% Manually include selected file,
% otherwise process as usual:
%    \begin{macrocode}
\ifchilddocmanual
\section*{part `\childdocname'}
\input{\childdocname}
\else
%    \end{macrocode}

% Include the two chapters:
%    \begin{macrocode}
\include{cdocsch1}
\include{cdocsch2}
%    \end{macrocode}

% Include the two parts unless only chapters should be displayed:
%    \begin{macrocode}
\ifchilddoc\else
\section{part three}
\input{cdocspt3}
\section{part four}
\input{cdocspt4}
\fi
%    \end{macrocode}

% Process as usual until here:
%    \begin{macrocode}
\fi
%    \end{macrocode}

% End of document body:
%    \begin{macrocode}
\end{document}
%    \end{macrocode}
%\iffalse
%</samplemain>
%\fi
%
% %%%%%%%%%%%%%%%%%%%%%%%%%%%%%%%%%%%%%%
% \paragraph{Chapter Include Files.}
%
% The include files are called |cdocsch1.tex| and |cdocsch2.tex|.
%
%\iffalse
%<*samplechap1|samplechap2>
%\fi

% Optional override for |\version| flag:
%    \begin{macrocode}
%%\providecommand{\version}{final}
%    \end{macrocode}

% Include the main document:
%    \begin{macrocode}
\input{childdoc.def}
\childdocof{cdocsamp}
%    \end{macrocode}

%\iffalse
%</samplechap1|samplechap2>
%\fi
%
%\iffalse
%<*samplechap1>
%\fi
% Some text for chapter 1:
%    \begin{macrocode}
\section{one}
some text in chapter one
%    \end{macrocode}

%\iffalse
%</samplechap1>
%\fi
% Some text for chapter 2:
%\iffalse
%<*samplechap2>
%\fi
%    \begin{macrocode}
\section{two}
more text in chapter two
%    \end{macrocode}

%\iffalse
%</samplechap2>
%\fi
%
% %%%%%%%%%%%%%%%%%%%%%%%%%%%%%%%%%%%%%%
% \paragraph{Part Include Files.}
%
% The include files are called |cdocspt3.tex| and |cdocspt4.tex|.
%
%\iffalse
%<*samplepart3|samplepart4>
%\fi

% Optional override for |\version| flag:
%    \begin{macrocode}
%%\providecommand{\version}{final}
%    \end{macrocode}

% Include the main document:
%    \begin{macrocode}
\input{childdoc.def}
\childdocby{cdocsamp}
%    \end{macrocode}

%\iffalse
%</samplepart3|samplepart4>
%\fi
%
%\iffalse
%<*samplepart3>
%\fi
% Some text for part 3:
%    \begin{macrocode}
some text in part three
%    \end{macrocode}

%\iffalse
%</samplepart3>
%\fi
% Some text for part 4:
%\iffalse
%<*samplepart4>
%\fi
%    \begin{macrocode}
more text in part four
%    \end{macrocode}

%\iffalse
%</samplepart4>
%\fi
%
% %%%%%%%%%%%%%%%%%%%%%%%%%%%%%%%%%%%%%%
% \paragraph{Forwarding for a Complete Draft.}
%
% The following forwarding file |cdocsdrf.tex|
% compiles the main document in draft mode:
%\iffalse
%<*sampledraft>
%\fi
%    \begin{macrocode}
\def\version{draft}
\input{childdoc.def}
\childdocforward{cdocsamp}
%    \end{macrocode}

%\iffalse
%</sampledraft>
%\fi
%
% %%%%%%%%%%%%%%%%%%%%%%%%%%%%%%%%%%%%%%
% \paragraph{Forwarding for Final Version of the Chapters.}
%
% The following forwarding files |cdocsfn1.tex| and |cdocsfn2.tex|
% (with identical content)
% compile the final versions of the child documents
% |cdocsch1.tex| and |cdocsch2.tex|, respectively:
%\iffalse
%<*samplefinal>
%\fi
%    \begin{macrocode}
\def\version{final}
\input{childdoc.def}
\childdocforwardprefix[cdocsamp]{cdocsfn}{cdocsch}
%    \end{macrocode}

%\iffalse
%</samplefinal>
%\fi
%
% %%%%%%%%%%%%%%%%%%%%%%%%%%%%%%%%%%%%%%
% \paragraph{Command Line Processing.}
%
% The following three command lines generate the output files
% |cdocscld|, |cdocscl1| and |cdocscl2|
% which should be identical to
% |cdocsdrf|, |cdocsch1| and |cdocsfn2|, respectively:
% \begin{center}
% \begin{tabular}{l}
% |latex -jobname cdocscld \|\\
% |  "\def\version{draft}\input{childdoc.def}\childdocforward{cdocsamp}"|\\
% |latex -jobname cdocscl1 \|\\
% |  "\input{childdoc.def}\childdocforward[cdocsamp]{cdocsch1}"|\\
% |latex -jobname cdocscl2 \|\\
% |  "\def\version{final}\input{childdoc.def}\childdocforward{cdocsch2}"|
% \end{tabular}
% \end{center}
% Note that the trailing backslash on each first line
% merely continues the input to the second line
% (for convenient cut ant paste).
% Furthermore, the command |latex| can be replaced by any
% of its alternative versions such as |pdflatex|.
%
% %%%%%%%%%%%%%%%%%%%%%%%%%%%%%%%%%%%%%%%%%%%%%%%%%%%%%%%%%%%%%%%%%%%%%%%%%%%%%%
% %%%%%%%%%%%%%%%%%%%%%%%%%%%%%%%%%%%%%%%%%%%%%%%%%%%%%%%%%%%%%%%%%%%%%%%%%%%%%%
% \section{Implementation}
%\iffalse
%<*package>
%\fi
%
% This section describes the definitions file |childdoc.def|.

% The definitions cannot be loaded using |\usepackage| or |\RequirePackage|
% which has a mechanism to prevent loading a style file more than once.
% When loading the definitions by means of |\input|
% multiple instances have to be prevented manually:
%\iffalse
%This code needs to be before the `\ProvidesFile' directive
%which is defined at the beginning of this file.
%Therefore it is also placed there and commented out here.
%</package>
%<*discard>
%\fi
%    \begin{macrocode}
\ifdefined\childdocmain\endinput\fi
%    \end{macrocode}
%\iffalse
%</discard>
%<*package>
%\fi
%
% \macro{\ifchilddoc}
% \macro{\ifchilddocmanual}
% The conditional |\ifchilddoc| tells whether a
% child (true) or main (false) document is being compiled.
% The conditional |\ifchilddocmanual| tells whether
% the |\includeonly| mechanism is used (false) or
% the selection of child files must be performed manually (true).
% The definitions initialise to false:
%    \begin{macrocode}
\newif\ifchilddoc
\newif\ifchilddocmanual
%    \end{macrocode}

% \macro{\childdocname}
% \macro{\childdocjob}
% The macro |\childdocname| stores the name of the main document
% to be compiled. The macro |\childdocjob| stores the name of
% the document on which the \LaTeX{} compiler was originally invoked.
% The content of |\jobname| cannot be compared
% to filenames specified in the source due to different catcodes.
% The following code rescans |\jobname|, stores the result
% in |\childdocname| and saves a copy in |\childdocjob|:
%    \begin{macrocode}
\edef\childdocname{\scantokens\expandafter{\jobname\noexpand}}
\let\childdocjob\childdocname
%    \end{macrocode}

% \macro{\childdocdisable}
% The macro |\childdocdisable| prevents the main file
% from being processed more than once.
% At this stage, the main document command |\childdocmain|
% is assumed to be called once again where it should do nothing.
% Any subsequent call to it should prevent
% a secondary processing of the main document
% It overwrites the forwarding commands
% |\childdocof| and |\childdocforward|
% with empty macros to prevent further inclusions of the main document:
%    \begin{macrocode}
\newcommand{\childdocdisable}
{
  \renewcommand{\childdocmain}[1]{\renewcommand{\childdocmain}[1]{\endinput}}
  \renewcommand{\childdocof}[1]{}
  \renewcommand{\childdocby}[2][]{}
  \renewcommand{\childdocforward}[2][]{}
  \renewcommand{\childdocdisable}{}
}
%    \end{macrocode}

% \macro{\childdocmain}
% The macro |\childdocmain| is to be called at the top of the main file
% with nothing or the main filename (without extension) as argument.
% First, it breaks loops.
% If the argument is not empty and does not match |\childdocname|
% (which is set by the first inclusion of |childdoc.def|),
% |\ifchilddoc| is set to true, |\includeonly| is applied to the child file
% and |\jobname| is set to the main file
% (for proper handling of |.aux| files):
%    \begin{macrocode}
\newcommand{\childdocmain}[1]
{
  \childdocdisable\childdocmain{}
  \if?#1?\else
    \begingroup
      \def\childdoctmp{#1}
      \ifx\childdoctmp\childdocname
        \def\childdoctmp{}
      \else
        \def\childdoctmp
        {
          \childdoctrue
          \includeonly{\childdocname}
          \def\childdocjob{#1}
          \def\jobname{#1}
        }
      \fi
      \expandafter
    \endgroup
    \childdoctmp
  \fi
}
%    \end{macrocode}

% \macro{\childdocof}
% The command |\childdocof| redirects
% compilation to the main file |#1|.
%    \begin{macrocode}
\newcommand{\childdocof}[1]
{
  \childdocdisable
  \childdoctrue
  \includeonly{\childdocname}
  \def\jobname{#1}
  \def\childdocjob{#1}
  \input{#1}
}
%    \end{macrocode}

% \macro{\childdocby}
% The command |\childdocby| ....
%    \begin{macrocode}
\newcommand{\childdocby}[2][]
{
  \childdocdisable
  \childdoctrue
  \childdocmanualtrue
  \if?#1?\else
    \def\jobname{#2}
  \fi
  \def\childdocjob{#2}
  \input{#2}
  \endinput
}
%    \end{macrocode}

% \macro{\childdocforward}
% The command |\childdocforward| redirects
% compilation to the main file or
% (if the optional argument is given) a child file.
% Parameters are set as if the main file
% or a child file starting with |\childdocof| was compiled.
% Then compilation is handed over to the main file:
%    \begin{macrocode}
\newcommand{\childdocforward}[2][]
{
  \begingroup
    \if?#1?
      \def\childdoctmp
      {
        \def\childdocname{#2}
        \def\childdocjob{#2}
        \def\jobname{#2}
        \input{#2}
        \endinput
      }
    \else
      \def\childdoctmp
      {
        \childdocdisable
        \def\childdocname{#2}
        \childdoctrue
        \includeonly{#2}
        \def\childdocjob{#1}
        \def\jobname{#1}
        \input{#1}
        \endinput
      }
    \fi
    \expandafter
  \endgroup
  \childdoctmp
}
%    \end{macrocode}

% \macro{\childdocforwardprefix}
% The command |\childdocforwardprefix| redirects
% compilation to the main or a child file by means of a pattern.
% The prefix |#1| in the current filename is replaced by |#2|
% and the suffix of the current filename is kept
% (it is assumed that the filename does not contain the substring `|~~~|'
% which is used as a delimiter).
% Compilation is handed over to the new file by |\childdocforward|:
%    \begin{macrocode}
\newcommand{\childdocforwardprefix}[3][]
{
  \begingroup
    \def\childdocextract #2##1~~~{\def\childdoctmp{\childdocforward[#1]{#3##1}}}
    \expandafter\childdocextract\childdocname~~~
    \expandafter
  \endgroup
  \childdoctmp
}
%    \end{macrocode}

% \macro{\childdoc}
% The deprecated macro |\childdoc| is a legacy version of |\childdocmain|:
%    \begin{macrocode}
\newcommand{\childdoc}{\childdocmain}
%    \end{macrocode}

% \macro{\childdocredirect}
% The deprecated macro |\childdocredirect| is a legacy version
% of |\childdocforward| and |\childdocforwardprefix|:
%    \begin{macrocode}
\newcommand{\childdocredirect}[2][]
{
  \begingroup
    \if?#1?
      \def\childdoctmp{\childdocforward{#2}}
    \else
      \def\childdoctmp{\childdocforwardprefix{#1}{#2}}
    \fi
    \expandafter
  \endgroup
  \childdoctmp
}
%    \end{macrocode}

%\iffalse
%</package>
%\fi
%
\endinput
\childdocforward[|\textit{main}|]{|\textit{dest}|}"|
\end{center}
%
Here \textit{target} is the name of the output file,
\textit{main} is the name of the main file
and \textit{dest} is the name of the main or child file to be processed
(all filenames without extensions).
The optional argument \textit{main} can be omitted
if \textit{main} matches \textit{dest}.
Optionally, compilation \textit{flags} can be defined via |\def| commands.
This command line makes the \TeX{} engine believe
it is compiling the file \textit{target}
whose content is specified as the latter parameter.
The provided code then forwards the processing to
\textit{main} or \textit{dest} as described in \secref{sec:forward}.

%%%%%%%%%%%%%%%%%%%%%%%%%%%%%%%%%%%%%%%%%%%%%%%%%%%%%%%%%%%%%%%%%%%%%%%%%%%%%%%%
\subsection{Include by Input}
\label{sec:input}

Including child documents by |\include| has some restrictions by design.
Most notably, the content of a child document always occupies
its own set of pages; pages cannot be shared between child documents.
Usually, this behaviour makes perfect sense
because each child document contain an essential part of the document.
However, in some situations it may be desirable to compose
a document from a collection of parts
without having mandatory page breaks between then.
For this case, the package
provides a mechanism to include parts
by |\input| which can also be processed individually.
However, by construction this mechanism
requires manual handling of the content to be output.

%%%%%%%%%%%%%%%%%%%%%%%%%%%%%%%%%%%%%%%%
\DescribeMacro{\ifchilddocmanual}
The main file should be prepared as usual, see \secref{sec:include}.
However, the document body must make a distinction
between processing of an individual part and of the main document, e.g.:
%
\begin{center}
\begin{tabular}{l}
|\ifchilddocmanual|\\
|\input{\childdocname}|\\
|\||else|\\
\textit{document body with }|\input{|\textit{part}|}|\\
|\||fi|
\end{tabular}
\end{center}
%
The conditional |\ifchilddocmanual| is true whenever
a part to be included by |\input| is being compiled,
and the name of the part is stored in |\childdocname|.

%%%%%%%%%%%%%%%%%%%%%%%%%%%%%%%%%%%%%%%%
\DescribeMacro{\childdocby}
Each part to be included by |\input| should start with:
%
\begin{center}
\begin{tabular}{l}
|% \iffalse
%
% childdoc.dtx Copyright (C) 2017-2018 Niklas Beisert
%
% This work may be distributed and/or modified under the
% conditions of the LaTeX Project Public License, either version 1.3
% of this license or (at your option) any later version.
% The latest version of this license is in
%   http://www.latex-project.org/lppl.txt
% and version 1.3 or later is part of all distributions of LaTeX
% version 2005/12/01 or later.
%
% This work has the LPPL maintenance status `maintained'.
%
% The Current Maintainer of this work is Niklas Beisert.
%
% This work consists of the files childdoc.dtx and childdoc.ins
% and the derived files childdoc.def and cdocsamp.tex with
% cdocsch1.tex, cdocsch2.tex, cdocsdrf.tex, cdocsfn1.tex, cdocsfn2.tex.
%
%<package>\ifdefined\childdocmain\endinput\fi
%<package>\ProvidesFile{childdoc.def}[2018/12/30 v2.0 child document driver]
%<samplemain>\ProvidesFile{cdocsamp.tex}[2018/12/30 v2.0 sample for childdoc]
%<*driver>
%\ProvidesFile{childdoc.drv}[2018/12/30 v2.0 childdoc reference manual file]
\PassOptionsToClass{10pt,a4paper}{article}
\documentclass{ltxdoc}

\usepackage[margin=35mm]{geometry}
\usepackage{hyperref}
\usepackage{hyperxmp}
\usepackage[usenames]{color}

\hypersetup{colorlinks=true}
\hypersetup{pdfstartview=FitH}
\hypersetup{pdfpagemode=UseNone}
\hypersetup{pdfsource={}}
\hypersetup{pdflang={en-UK}}
\hypersetup{pdfcopyright={Copyright 2017-2018 Niklas Beisert.
  This work may be distributed and/or modified under the
  conditions of the LaTeX Project Public License, either version 1.3
  of this license or (at your option) any later version.}}
\hypersetup{pdflicenseurl={http://www.latex-project.org/lppl.txt}}
\hypersetup{pdfcontactaddress={ETH Zurich, ITP, HIT K,
  Wolfgang-Pauli-Strasse 27}}
\hypersetup{pdfcontactpostcode={8093}}
\hypersetup{pdfcontactcity={Zurich}}
\hypersetup{pdfcontactcountry={Switzerland}}
\hypersetup{pdfcontactemail={nbeisert@itp.phys.ethz.ch}}
\hypersetup{pdfcontacturl={http://people.phys.ethz.ch/\xmptilde nbeisert/}}

\newcommand{\secref}[1]{\hyperref[#1]{section \ref*{#1}}}

\parskip1ex
\parindent0pt
\let\olditemize\itemize
\def\itemize{\olditemize\parskip0pt}

\begin{document}

\title{The \textsf{childdoc} Package}
\hypersetup{pdftitle={The childdoc Package}}
\author{Niklas Beisert\\[2ex]
  Institut f\"ur Theoretische Physik\\
  Eidgen\"ossische Technische Hochschule Z\"urich\\
  Wolfgang-Pauli-Strasse 27, 8093 Z\"urich, Switzerland\\[1ex]
  \href{mailto:nbeisert@itp.phys.ethz.ch}
  {\texttt{nbeisert@itp.phys.ethz.ch}}}
\hypersetup{pdfauthor={Niklas Beisert}}
\hypersetup{pdfsubject={Manual for the LaTeX2e Package childdoc}}
\date{30 December 2018, \textsf{v2.0}}
\maketitle

\begin{abstract}\noindent
\textsf{childdoc} is a \LaTeXe{} package
that enables the direct compilation
of document sections included by |\include|
to individual files.
\end{abstract}

\begingroup
\parskip0ex
\tableofcontents
\endgroup

%%%%%%%%%%%%%%%%%%%%%%%%%%%%%%%%%%%%%%%%%%%%%%%%%%%%%%%%%%%%%%%%%%%%%%%%%%%%%%%%
%%%%%%%%%%%%%%%%%%%%%%%%%%%%%%%%%%%%%%%%%%%%%%%%%%%%%%%%%%%%%%%%%%%%%%%%%%%%%%%%
\section{Introduction}

\LaTeX{} provides a mechanism to structure a large document (such as a book)
into a main file and several child files (containing the chapters)
using the |\include| command.
This mechanism is beneficial for documents
which span hundreds of pages in order to
make the source file(s) more manageable.
Moreover, compilation can be restricted to
selected child files by means of the |\includeonly| command.
The latter feature can be used to reduce the compilation time while editing
(this was significantly more useful in the earlier days of \LaTeX{})
or to generate a smaller document which is easier to navigate.
Another application of |\includeonly| is to generate
documents consisting of selected parts of the complete document.

However, there are a few drawbacks of the plain |\include| mechanism:
\begin{itemize}
\item
The child files cannot be compiled on their own,
they can only be compiled via the main file.
A naive editing environment
(such as a text editor with an option
to have the current file processed by \LaTeX)
may require one to switch to the main file before compiling;
attempting to compile the child file produces errors.
\item
The main file must be modified (each time)
to adjust the |\includeonly| command
to the present needs. This easily leaves the main file in a messy state.
\item
The generated document will always carry the filename
of the main document. This is inconvenient if
several child files are to be compiled and
to be kept for distribution.
\end{itemize}

The present package provides a simple interface
to make child files individually compilable by \LaTeX{}.
Compiling a child file then has the same effect as compiling
the main file with an |\includeonly| command
to select the appropriate child.
Moreover the generated document will carry the name of the child
rather than the main file.
This resolves all three above issues.

This feature is meant to make the editing of books,
thesis documents and lecture notes somewhat more convenient.
However, the package can also be used efficiently for
composing a series of documents (such as exercise sheets)
which are typically distributed individually.
It then assists the author in generating the individual documents
(potentially in different versions)
as well as a document containing the collected series.
Another application is in developing style files
or other kinds of included material
where compilation of the style file could redirect
to a sample or test file.

%%%%%%%%%%%%%%%%%%%%%%%%%%%%%%%%%%%%%%%%%%%%%%%%%%%%%%%%%%%%%%%%%%%%%%%%%%%%%%%%
%%%%%%%%%%%%%%%%%%%%%%%%%%%%%%%%%%%%%%%%%%%%%%%%%%%%%%%%%%%%%%%%%%%%%%%%%%%%%%%%
\section{Usage}

First of all, the package \textsf{childdoc} is \emph{not} a standard
\LaTeXe{} |.sty| style file! Therefore it needs to be invoked in
a non-standard way.

%%%%%%%%%%%%%%%%%%%%%%%%%%%%%%%%%%%%%%%%%%%%%%%%%%%%%%%%%%%%%%%%%%%%%%%%%%%%%%%%
\subsection{Included Files}
\label{sec:include}

%%%%%%%%%%%%%%%%%%%%%%%%%%%%%%%%%%%%%%%%
\DescribeMacro{\childdocmain}
To use the package, add the commands
\begin{center}
\begin{tabular}{l}
|\input{childdoc.def}|\\
|\childdocmain{}|\\
\end{tabular}
\end{center}
at the very top of the main \LaTeX{} file,
in particular \emph{before} the |\documentclass| statement!
The argument of |\childdocmain| should be left empty
(but it must be present).

%%%%%%%%%%%%%%%%%%%%%%%%%%%%%%%%%%%%%%%%
\DescribeMacro{\childdocof}
Furthermore, add the commands
\begin{center}
\begin{tabular}{l}
|\input{childdoc.def}|\\
|\childdocof{|\textit{main}|}|\\
\end{tabular}
\end{center}
at the top of every child file \textit{child}
which is included by |\include{|\textit{child}|}|
from within the main file
(or at least for those files to be compiled individually).
The argument \textit{main} must be the filename of the main file.

There are a couple of
considerations in setting up the main and child documents:

%%%%%%%%%%%%%%%%%%%%%%%%%%%%%%%%%%%%%%%%
\paragraph{Restrictions.}

Please note the following restrictions:
\begin{itemize}
\item
|\childdocmain| must be called with one argument \textit{main}
to ensure compatibility with earlier version of the package.
It must either be empty (|\childdocmain{}|)
or precisely match the filename of the main file in which it is specified.
See \secref{sec:detection} for further information.
\item
The filename \textit{main} must be specified without the |.tex| extension.
\item
The filename \textit{main} is case sensitive
(even in case-insensitive file systems)
due to internal string comparison.
\item
The argument \textit{main} should be fully expanded, it cannot be a macro.
\item
Subdirectories and special characters should be avoided in filenames.
\item
The command |\childdocmain{|\textit{main}|}| must be followed by a whitespace.
It should not be followed immediately by another command
or by a comment mark `|%|'.
This is because the \TeX{} parser reads the token immediately following
the argument of |\childdocmain| and puts it
at the beginning of every child section;
however, a white\-space is ignored.
\end{itemize}

%%%%%%%%%%%%%%%%%%%%%%%%%%%%%%%%%%%%%%%%
\paragraph{Content of Main File.}

It is advisable to place all content in the child files included by |\include|.
Any output contained in the main file will appear in all child documents
unless suppressed manually;
it cannot be suppressed automatically by the |\includeonly| directive
and thus should normally be avoided.
A method to include some content in the main file
by means of conditional processing is described in \secref{sec:conditional}.

%%%%%%%%%%%%%%%%%%%%%%%%%%%%%%%%%%%%%%%%
\paragraph{Page Numbering.}

When only a part of the document is compiled,
the appropriate numbering of pages
(as well as other status parameters)
is determined from the |.aux| files.
The latter contain information from previous passes.
However this information needs to propagate through
all intermediate child documents.
Therefore the page numbering in child documents may well
be inconsistent until the complete document is compiled at least once.

A useful (if unconventional) way to always ensure a consistent
page numbering is to restart the numbering in each child document
and denote the pages by `\textit{child}|.|\textit{page}'
where \textit{child} represents the chapter/section number of the child file.
This can be achieved by the command
|\numberwithin{page}{|\textit{child}|}|
of the \textsf{amsmath} package
where \textit{child} can be |chapter| or |section|
depending on the chosen structuring.
Alternatively, one can modify the macro |\thepage| appropriately
and reset the counter |page| at the start of each child file.

%%%%%%%%%%%%%%%%%%%%%%%%%%%%%%%%%%%%%%%%%%%%%%%%%%%%%%%%%%%%%%%%%%%%%%%%%%%%%%%%
\subsection{Conditional Processing}
\label{sec:conditional}

The package provides a mechanism to compile different versions
of a document. To customise the versions further some conditional processing
can come in handy to distinguish which version is being compiled.
The package provides two macros to describe the compilation context:

%%%%%%%%%%%%%%%%%%%%%%%%%%%%%%%%%%%%%%%%
\DescribeMacro{\ifchilddoc}
The conditional |\ifchilddoc| distinguishes between the compilation of
child documents and the main document:
%
\begin{center}
|\ifchilddoc |\textit{child-code}| |[|\||else |\textit{main-code}]| \||fi|
\end{center}

%%%%%%%%%%%%%%%%%%%%%%%%%%%%%%%%%%%%%%%%
\DescribeMacro{\childdocname}
\DescribeMacro{\childdocjob}
The macro |\childdocname| contains the filename (without extension)
of the main or child file being processed.
Note that |\childdocjob| will always contain the name of the main file.

%%%%%%%%%%%%%%%%%%%%%%%%%%%%%%%%%%%%%%%%
\paragraph{Title Page.}

Conditional processing can be used to include a title or banner page
in the main document when proper precautions are taken.
Importantly, the code in the main file should ensure that the page counter
(as well as other status parameters which are stored in the |.aux| files)
takes the same value after the conditional processing.
Otherwise the page numbers may take divergent values
depending on which part is compiled.

For example, a title page could be declared by:
%
\begin{center}
\begin{tabular}{l}
|\ifchilddoc\||else|\\
|\addtocounter{page}{-1}|\\
\textit{code for title page}\\
|\newpage|\\
|\||fi|
\end{tabular}
\end{center}
%
A banner page for the child documents can be generated by:
%
\begin{center}
\begin{tabular}{l}
|\ifchilddoc|\\
|\addtocounter{page}{-1}|\\
\textit{code for banner page}\\
|\newpage|\\
|\||fi|
\end{tabular}
\end{center}
%
Here one could write a message such as:
\begin{center}
|This is the part \childdocname{} of \childdocjob{}.|
\end{center}

%%%%%%%%%%%%%%%%%%%%%%%%%%%%%%%%%%%%%%%%%%%%%%%%%%%%%%%%%%%%%%%%%%%%%%%%%%%%%%%%
\subsection{Flags}
\label{sec:flags}

The package makes it easy to generate different versions
of the main or child documents.
To this end compilation flags can be defined
and assigned different default values.
They will be particularly useful in conjunction
with the forwarding mechanism described in \secref{sec:forward}.

For example, it may be useful to have a flag |\version|
which can be set to |draft| or |final|.
The document source will contain some conditional code
depending on the value of |\version|.
Suppose further, the flag should default to |final| for the main file
and to |draft| for child files
which is a natural assignment for editing the document.
This is achieved by placing the following code
in the preamble of the main document
(below the |\childdocmain| directive):
%
\begin{center}
\begin{tabular}{l}
|\ifchilddoc|\\
|\providecommand{\version}{draft}|\\
|\||else|\\
|\providecommand{\version}{final}|\\
|\||fi|
\end{tabular}
\end{center}
%
The definition by |\providecommand| makes sure
that previous definitions are not overwritten.
Further statements |\providecommand{\version}{...}|
can thus be added before the above code to override it.

For the main file, one might add a line
(between |\childdocmain| and the above block)
%
\begin{center}
|%\ifchilddoc\||else\providecommand{\version}{draft}\||fi|
\end{center}
%
which can be uncommented to produce a draft version.
Likewise one can add a line to the very top of a child file
(above the |\childdocof{|\textit{main}|}| directive)
%
\begin{center}
|%\providecommand{\version}{final}|
\end{center}
%
which can be uncommented to produce the final version of this child document.

%%%%%%%%%%%%%%%%%%%%%%%%%%%%%%%%%%%%%%%%%%%%%%%%%%%%%%%%%%%%%%%%%%%%%%%%%%%%%%%%
\subsection{Forwarding}
\label{sec:forward}

Different versions of the main or child documents
using compilation flags as described in \secref{sec:flags}
can be (permanently) stored in different files
for convenient compilation, viewing and distribution.
To this end, the package defines a command
to pass on compilation to a different file:

%%%%%%%%%%%%%%%%%%%%%%%%%%%%%%%%%%%%%%%%
\DescribeMacro{\childdocforward}
The command |\childdocforward| redirects processing to
another source file:
%
\begin{center}
\begin{tabular}{l}
|\input{childdoc.def}|\\
|\childdocforward[|\textit{main}|]{|\textit{dest}|}|\\
\end{tabular}
\end{center}
%
The argument \textit{dest} is the destination file
(without extension).
It should be the main file or one of the child files.
Note that further \textsf{childdoc} directives
such as |\childdocof| and |\childdocforward|
in the indicated file will be processed in this form.
The optional argument \textit{main}
passes on directly to the main file \textit{main}
while pretending to compile the child \textit{dest}.
This form behaves as if \textit{dest}
issues |\childdocof{|\textit{main}|}| right away,
and no further \textsf{childdoc} directives will be processed.

%%%%%%%%%%%%%%%%%%%%%%%%%%%%%%%%%%%%%%%%
\DescribeMacro{\...prefix}
In the alternative form |\childdocforwardprefix|,
%
\begin{center}
\begin{tabular}{l}
|\input{childdoc.def}|\\
|\childdocforwardprefix[|\textit{main}|]{|\textit{prefix}|}{|\textit{dest}|}|
\end{tabular}
\end{center}
%
the destination file is determined by a pattern
depending on the current file:
To make this work, the current file must be called
`{\textit{prefix}\hspace{0.2em}\textit{suffix}}'
with \textit{prefix} matching precisely the argument.
Processing is then passed on to the file
`{\textit{dest}\hspace{0.2em}\textit{suffix}}'.
Surely, the same effect is achieved by
directly specifying the
argument `{\textit{dest}\hspace{0.2em}\textit{suffix}}'
in the first form.
However, that requires to set up a different file
for each child. With the alternative form of the command
all these files can have exactly the same content
which simplifies setting them up and maintaining them.

For example, the following file |draft.tex|
with a compilation flag |\version| as described in \secref{sec:flags}
compiles the main document as a draft:
%
\begin{center}
\begin{tabular}{l}
|\def\version{draft}|\\
|\input{childdoc.def}|\\
|\childdocforward{|\textit{main}|}|
\end{tabular}
\end{center}
%
Likewise, the following files |final|\textit{nn}|.tex|
compile the final version of the child document
|child|\textit{nn}|.tex|:
%
\begin{center}
\begin{tabular}{l}
|\def\version{final}|\\
|\input{childdoc.def}|\\
|\childdocforwardprefix{final}{child}|
\end{tabular}
\end{center}
%

Note that when several versions of a main file and/or of each child file
are to be generated, it may be convenient to set up a |Makefile| or
shell script to automatise the process.

%%%%%%%%%%%%%%%%%%%%%%%%%%%%%%%%%%%%%%%%%%%%%%%%%%%%%%%%%%%%%%%%%%%%%%%%%%%%%%%%
\subsection{Command Line Processing}
\label{sec:commandline}

The effect of redirection files can also be achieved by invoking
the \LaTeX{} compiler with a more elaborate command line.
Most conveniently this should be done as part
of a shell script or a |Makefile|.

When using \textsf{childdoc} in the main file, the following
command lines effectively perform a redirection
(note that depending on the shell being used,
backslashes may have to be doubled: `|\|' $\to$ `|\\|'):
%
\begin{center}
|... -jobname "|\textit{target}|" |\\|"|[\textit{flags}]%
|\input{childdoc.def}\childdocforward[|\textit{main}|]{|\textit{dest}|}"|
\end{center}
%
Here \textit{target} is the name of the output file,
\textit{main} is the name of the main file
and \textit{dest} is the name of the main or child file to be processed
(all filenames without extensions).
The optional argument \textit{main} can be omitted
if \textit{main} matches \textit{dest}.
Optionally, compilation \textit{flags} can be defined via |\def| commands.
This command line makes the \TeX{} engine believe
it is compiling the file \textit{target}
whose content is specified as the latter parameter.
The provided code then forwards the processing to
\textit{main} or \textit{dest} as described in \secref{sec:forward}.

%%%%%%%%%%%%%%%%%%%%%%%%%%%%%%%%%%%%%%%%%%%%%%%%%%%%%%%%%%%%%%%%%%%%%%%%%%%%%%%%
\subsection{Include by Input}
\label{sec:input}

Including child documents by |\include| has some restrictions by design.
Most notably, the content of a child document always occupies
its own set of pages; pages cannot be shared between child documents.
Usually, this behaviour makes perfect sense
because each child document contain an essential part of the document.
However, in some situations it may be desirable to compose
a document from a collection of parts
without having mandatory page breaks between then.
For this case, the package
provides a mechanism to include parts
by |\input| which can also be processed individually.
However, by construction this mechanism
requires manual handling of the content to be output.

%%%%%%%%%%%%%%%%%%%%%%%%%%%%%%%%%%%%%%%%
\DescribeMacro{\ifchilddocmanual}
The main file should be prepared as usual, see \secref{sec:include}.
However, the document body must make a distinction
between processing of an individual part and of the main document, e.g.:
%
\begin{center}
\begin{tabular}{l}
|\ifchilddocmanual|\\
|\input{\childdocname}|\\
|\||else|\\
\textit{document body with }|\input{|\textit{part}|}|\\
|\||fi|
\end{tabular}
\end{center}
%
The conditional |\ifchilddocmanual| is true whenever
a part to be included by |\input| is being compiled,
and the name of the part is stored in |\childdocname|.

%%%%%%%%%%%%%%%%%%%%%%%%%%%%%%%%%%%%%%%%
\DescribeMacro{\childdocby}
Each part to be included by |\input| should start with:
%
\begin{center}
\begin{tabular}{l}
|\input{childdoc.def}|\\
|\childdocby{|\textit{main}|}|\\
\end{tabular}
\end{center}
%
The directive |\childdocby| is similar to |\childdocof|
described in \secref{sec:include},
but the subsequent selection of content must be done manually.
To that end, both |\ifchilddoc| and |\ifchilddocmanual|
will be true upon processing of a part,
and the name of the part is stored in |\childdocname|.
Note that |\jobname| will be set to the filename of the current part
so that each part receives an individual |.aux| file
that does not interfere with the |.aux| file(s) of the main document.
This behaviour can be altered by the alternative form
|\childdocby[*]{|\textit{main}|}| (with a non-empty optional argument)
which uses the |.aux| file of the main document
by setting |\jobname| to \textit{main}.

%%%%%%%%%%%%%%%%%%%%%%%%%%%%%%%%%%%%%%%%%%%%%%%%%%%%%%%%%%%%%%%%%%%%%%%%%%%%%%%%
\subsection{Driver Development}
\label{sec:driver}

The \textsf{childdoc} mechanism can also be use for the development
of definition files such as \LaTeX{} styles or classes.
This case differs from the above setup with multiple parts
included by |\include| in that no |\includeonly| should be invoked.
This can be achieved by starting the include file
(before |\ProvidesPackage|) with:
%
\begin{center}
\begin{tabular}{l}
|\input{childdoc.def}|\\
|\childdocforward{|\textit{main}|}|\\
\end{tabular}
\end{center}
%
or alternatively with:
%
\begin{center}
\begin{tabular}{l}
|\input{childdoc.def}|\\
|\childdocby{|\textit{main}|}|\\
\end{tabular}
\end{center}
%
Both forms have slightly different effects as described above.
The main file is prepared as usual, see \secref{sec:include}.

%%%%%%%%%%%%%%%%%%%%%%%%%%%%%%%%%%%%%%%%%%%%%%%%%%%%%%%%%%%%%%%%%%%%%%%%%%%%%%%%
\subsection{Legacy Detection}
\label{sec:detection}

The directive |\childdocmain| in the main file can detect
whether the complete document or merely a child is to be compiled
even without using the directive |\childdocof|.
This method is deprecated because it is less robust
and there is no compelling reason to use it;
it is merely provided for backward compatibility
and it may be removed in future versions.

If the detection mechanism is to be used,
it is mandatory to correctly specify
the filename of the main file as the argument of |\childdocmain|:
%
\begin{center}
\begin{tabular}{l}
|\input{childdoc.def}|\\
|\childdocmain{|\textit{main}|}|\\
\end{tabular}
\end{center}
%
If |\jobname| does not match the argument \textit{main} of |\childdocmain|,
it is assumed that |\jobname| points to the child file to be compiled.
When using |\childdocmain| with the main file specified as argument,
it suffices to start a child file
with just |\input{|\textit{main}|}|
without loading of the package and using |\childdocof|.
If instead all processing is done
with the appropriate \textsf{childdoc} directives,
the argument of \textit{main} of |\childdocmain| can be empty.

An alternative version of the command line processing described
in \secref{sec:commandline} using the detection mechanism reads:
%
\begin{center}
|... -jobname "|\textit{target}|" "|[\textit{flags}]%
[|\def\jobname{|\textit{dest}|}|]|\input{|\textit{main}|}"|
\end{center}

%%%%%%%%%%%%%%%%%%%%%%%%%%%%%%%%%%%%%%%%%%%%%%%%%%%%%%%%%%%%%%%%%%%%%%%%%%%%%%%%
\subsection{Manual Code}
\label{sec:manual}

In case one cannot be certain whether the definitions file |childdoc.def|
is installed on the target \TeX{} distribution
and one prefers not to ship it,
it is conceivable to paste a few relevant commands into the sources.

To that end, drop all statements |\input{childdoc.def}|
and perform the replacements as outlined below.
Instead of |\childdocmain{|\textit{main}|}| add the following code
to the top of the main file:
%
\begin{center}
\begin{tabular}{l}
|\||ifdefined\childdocname\endinput\||fi\newif\ifchilddoc|\\
|\edef\childdocname{\scantokens\expandafter{\jobname\noexpand}}|\\
|\def\childdocmain{|\textit{main}|}\||ifx\childdocmain\childdocname\||else|\\
|\childdoctrue\includeonly{\childdocname}\let\jobname\childdocmain\||fi|\\
\end{tabular}
\end{center}
%
Instead of |\childdocof{|\textit{main}|}| just include the main file
at the top of each child file:
%
\begin{center}
|\input{|\textit{main}|}|
\end{center}
%
A simple redirection |\childdocforward{|\textit{dest}|}| is achieved by:
%
\begin{center}
|\def\jobname{|\textit{dest}|}\input{\jobname}|
\end{center}
%
The redirection with prefix
|\childdocforwardprefix[|\textit{prefix}|]{|\textit{dest}|}|
is accomplished by:
%
\begin{center}
\begin{tabular}{l}
|{\edef\jobname{\scantokens\expandafter{\jobname\noexpand}}|\\
|\def\redirectjob |\textit{prefix}|#1~~~{\gdef\jobname{|\textit{dest}|#1}}|\\
|\expandafter\redirectjob\jobname~~~}\input{\jobname}|
\end{tabular}
\end{center}

In an alternative approach,
child documents can be compiled by a specific command line
without additional code or specific definitions:
%
\begin{center}
|... -jobname "|\textit{target}|" "|[\textit{flags}]%
|\includeonly{|\textit{dest}|}\input{|\textit{main}|}"|
\end{center}
%

%%%%%%%%%%%%%%%%%%%%%%%%%%%%%%%%%%%%%%%%%%%%%%%%%%%%%%%%%%%%%%%%%%%%%%%%%%%%%%%%
%%%%%%%%%%%%%%%%%%%%%%%%%%%%%%%%%%%%%%%%%%%%%%%%%%%%%%%%%%%%%%%%%%%%%%%%%%%%%%%%
\section{Information}

%%%%%%%%%%%%%%%%%%%%%%%%%%%%%%%%%%%%%%%%%%%%%%%%%%%%%%%%%%%%%%%%%%%%%%%%%%%%%%%%
\subsection{Copyright}

Copyright \copyright{} 2017--2018 Niklas Beisert

This work may be distributed and/or modified under the
conditions of the \LaTeX{} Project Public License, either version 1.3
of this license or (at your option) any later version.
The latest version of this license is in
  \url{http://www.latex-project.org/lppl.txt}
and version 1.3 or later is part of all distributions of \LaTeX{}
version 2005/12/01 or later.

This work has the LPPL maintenance status `maintained'.

The Current Maintainer of this work is Niklas Beisert.

This work consists of the files |README.txt|, |childdoc.ins| and |childdoc.dtx|
as well as the derived files |childdoc.def|, |cdocsamp.tex|
with |cdocsch1.tex|, |cdocsch2.tex|, |cdocspt3.tex|, |cdocspt4.tex|,
|cdocsdrf.tex|, |cdocsfn1.tex|, |cdocsfn2.tex|
as well as |childdoc.pdf|.

%%%%%%%%%%%%%%%%%%%%%%%%%%%%%%%%%%%%%%%%%%%%%%%%%%%%%%%%%%%%%%%%%%%%%%%%%%%%%%%%
\subsection{Files and Installation}

The package consists of the files:
%
\begin{center}
\begin{tabular}{ll}
    |README.txt|   & readme file \\
    |childdoc.ins| & installation file \\
    |childdoc.dtx| & source file \\
    |childdoc.def| & definition file \\
    |cdocsamp.tex| & sample main file \\
    |cdocsch1.tex| & sample include file \\
    |cdocsch2.tex| & sample include file \\
    |cdocspt3.tex| & sample part file \\
    |cdocspt4.tex| & sample part file \\
    |cdocsdrf.tex| & sample redirection file \\
    |cdocsfn1.tex| & sample redirection file \\
    |cdocsfn2.tex| & sample redirection file \\
    |childdoc.pdf| & manual
\end{tabular}
\end{center}
%
The distribution consists of the files
|README.txt|, |childdoc.ins| and |childdoc.dtx|.
%
\begin{itemize}
\item
Run (pdf)\LaTeX{} on |childdoc.dtx|
to compile the manual |childdoc.pdf| (this file).
\item
Run \LaTeX{} on |childdoc.ins| to create the definitions file |childdoc.def|
and the sample |cdocsamp.tex| with include files
|cdocsch1.tex|, |cdocsch2.tex|, |cdocspt3.tex|, |cdocspt4.tex|,
|cdocsdrf.tex|, |cdocsfn1.tex|, |cdocsfn2.tex|.
Then copy the file |childdoc.def| to an appropriate directory of your \LaTeX{}
distribution, e.g.\ \textit{texmf-root}|/tex/latex/childdoc|.
\end{itemize}

%%%%%%%%%%%%%%%%%%%%%%%%%%%%%%%%%%%%%%%%%%%%%%%%%%%%%%%%%%%%%%%%%%%%%%%%%%%%%%%%
\subsection{Related CTAN Packages}

There are several other packages which offer a similar functionality:
%
\begin{itemize}
\item
The packages
\href{http://ctan.org/pkg/docmute}{\textsf{docmute}},
\href{http://ctan.org/pkg/includex}{\textsf{includex}} and
\href{http://ctan.org/pkg/standalone}{\textsf{standalone}}
provide commands to include only the document body of
a child file thus allowing both files to be compiled individually.
\item
The packages \href{http://ctan.org/pkg/subdocs}{\textsf{subdocs}}
and \href{http://ctan.org/pkg/subfiles}{\textsf{subfiles}}
provide structures in which the main and child documents can be
encapsulated and allowing them to be compiled individually.
The inclusion mechanism is different from the conventional |\include|.
\item
The package \href{http://ctan.org/pkg/combine}{\textsf{combine}}
is an elaborate solution to combine several documents into one.
\end{itemize}
%
See also the CTAN topic \href{http://ctan.org/topic/subdocs}{\textsf{subdocs}}
for further related packages.
The present package differs from the above solutions in that
a document structure constructed with the conventional |\include| mechanism
just needs two extra commands at the top of every file
such that all constituent files can be compiled individually.

%%%%%%%%%%%%%%%%%%%%%%%%%%%%%%%%%%%%%%%%%%%%%%%%%%%%%%%%%%%%%%%%%%%%%%%%%%%%%%%%
%\subsection{Feature Suggestions}
%
%The following is a list of features which may be useful for future
%versions of this package:
%%
%\begin{itemize}
%\item
%\ldots
%\end{itemize}

%%%%%%%%%%%%%%%%%%%%%%%%%%%%%%%%%%%%%%%%%%%%%%%%%%%%%%%%%%%%%%%%%%%%%%%%%%%%%%%%
\subsection{Revision History}

%%%%%%%%%%%%%%%%%%%%%%%%%%%%%%%%%%%%%%%%
\paragraph{v2.0:} 2018/12/30

\begin{itemize}
\item
immediate forward processing
\item
added |\childdocby| mechanism
\item
manual restructured
\end{itemize}

%%%%%%%%%%%%%%%%%%%%%%%%%%%%%%%%%%%%%%%%
\paragraph{v1.6:} 2018/01/17

\begin{itemize}
\item
application for development of include files
\item
corrections to manual
\end{itemize}

%%%%%%%%%%%%%%%%%%%%%%%%%%%%%%%%%%%%%%%%
\paragraph{v1.5:} 2017/05/21

\begin{itemize}
\item
more complete structuring introduced
\item
|\childdocof| introduced
\item
|\childdoc| renamed to |\childdocmain|
\item
|\childredirect| renamed to |\childdocforward| and |\childdocforwardprefix|
and functionality expanded
\end{itemize}

%%%%%%%%%%%%%%%%%%%%%%%%%%%%%%%%%%%%%%%%
\paragraph{v1.0:} 2017/04/27

\begin{itemize}
\item
manual and install package
\item
first version published on CTAN
\end{itemize}

%%%%%%%%%%%%%%%%%%%%%%%%%%%%%%%%%%%%%%%%
\paragraph{v0.6:} 2017/04/26

\begin{itemize}
\item
redirection mechanism added
\end{itemize}

%%%%%%%%%%%%%%%%%%%%%%%%%%%%%%%%%%%%%%%%
\paragraph{v0.5:} 2017/04/26

\begin{itemize}
\item
functionality in definition file
\end{itemize}


%%%%%%%%%%%%%%%%%%%%%%%%%%%%%%%%%%%%%%%%%%%%%%%%%%%%%%%%%%%%%%%%%%%%%%%%%%%%%%%%
%%%%%%%%%%%%%%%%%%%%%%%%%%%%%%%%%%%%%%%%%%%%%%%%%%%%%%%%%%%%%%%%%%%%%%%%%%%%%%%%
%%%%%%%%%%%%%%%%%%%%%%%%%%%%%%%%%%%%%%%%%%%%%%%%%%%%%%%%%%%%%%%%%%%%%%%%%%%%%%%%
\appendix

\settowidth\MacroIndent{\rmfamily\scriptsize 000\ }

 \DocInput{childdoc.dtx}

\end{document}
%</driver>
% \fi
%
% %%%%%%%%%%%%%%%%%%%%%%%%%%%%%%%%%%%%%%%%%%%%%%%%%%%%%%%%%%%%%%%%%%%%%%%%%%%%%%
% %%%%%%%%%%%%%%%%%%%%%%%%%%%%%%%%%%%%%%%%%%%%%%%%%%%%%%%%%%%%%%%%%%%%%%%%%%%%%%
% \section{Sample}
%\iffalse
%<*samplemain>
%\fi
%
% The following presents a sample document
% with two chapters, two parts, a title page,
% a compile flag as well as three forwarding files to set the flag.
% It consists of eight |.tex| files:
% \begin{center}
% \begin{tabular}{ll}
% |cdocsamp.tex|&main file\\
% |cdocsch1.tex|&include file for chapter 1\\
% |cdocsch2.tex|&include file for chapter 2\\
% |cdocspt3.tex|&include file for part 3\\
% |cdocspt4.tex|&include file for part 4\\
% |cdocsdrf.tex|&forwarding file for main file in draft mode\\
% |cdocsfi1.tex|&forwarding file for final version of chapter 1\\
% |cdocsfi2.tex|&forwarding file for final version of chapter 2\\
% \end{tabular}
% \end{center}
% Each of the eight files can be compiled directly by the \LaTeX{} compiler.
%
% %%%%%%%%%%%%%%%%%%%%%%%%%%%%%%%%%%%%%%
% \paragraph{Main File.}
%
% The main file is called |cdocsamp.tex|.
%
% Load the \textsf{childdoc} definitions and
% declare the filename for the main document:
%    \begin{macrocode}
\input{childdoc.def}
\childdocmain{}
%    \end{macrocode}

% Optional override for |\version| flag:
%    \begin{macrocode}
%%\ifchilddoc\else\providecommand{\version}{draft}\fi
%    \end{macrocode}

% Define the default values for the |\version| flag
% (|final| for the main file and |draft| for childs):
%    \begin{macrocode}
\ifchilddoc
\providecommand{\version}{draft}
\else
\providecommand{\version}{final}
\fi
%    \end{macrocode}

% Load the standard document class:
%    \begin{macrocode}
\documentclass[12pt]{article}
%    \end{macrocode}

% Start the document body:
%    \begin{macrocode}
\begin{document}
%    \end{macrocode}

% Declare a title page.
% Print title, part of document being processed and version flag:
%    \begin{macrocode}
\addtocounter{page}{-1}
\begin{center}
{\LARGE\bfseries{}childdoc example\par}
\vspace{1cm}
\ifchilddoc
\ifchilddocmanual part\else chapter\fi:
`\childdocname' of `\childdocjob'\par
\else
main document: `\childdocjob'\par
\fi
version: \version\par
\end{center}
\newpage
%    \end{macrocode}

% Manually include selected file,
% otherwise process as usual:
%    \begin{macrocode}
\ifchilddocmanual
\section*{part `\childdocname'}
\input{\childdocname}
\else
%    \end{macrocode}

% Include the two chapters:
%    \begin{macrocode}
\include{cdocsch1}
\include{cdocsch2}
%    \end{macrocode}

% Include the two parts unless only chapters should be displayed:
%    \begin{macrocode}
\ifchilddoc\else
\section{part three}
\input{cdocspt3}
\section{part four}
\input{cdocspt4}
\fi
%    \end{macrocode}

% Process as usual until here:
%    \begin{macrocode}
\fi
%    \end{macrocode}

% End of document body:
%    \begin{macrocode}
\end{document}
%    \end{macrocode}
%\iffalse
%</samplemain>
%\fi
%
% %%%%%%%%%%%%%%%%%%%%%%%%%%%%%%%%%%%%%%
% \paragraph{Chapter Include Files.}
%
% The include files are called |cdocsch1.tex| and |cdocsch2.tex|.
%
%\iffalse
%<*samplechap1|samplechap2>
%\fi

% Optional override for |\version| flag:
%    \begin{macrocode}
%%\providecommand{\version}{final}
%    \end{macrocode}

% Include the main document:
%    \begin{macrocode}
\input{childdoc.def}
\childdocof{cdocsamp}
%    \end{macrocode}

%\iffalse
%</samplechap1|samplechap2>
%\fi
%
%\iffalse
%<*samplechap1>
%\fi
% Some text for chapter 1:
%    \begin{macrocode}
\section{one}
some text in chapter one
%    \end{macrocode}

%\iffalse
%</samplechap1>
%\fi
% Some text for chapter 2:
%\iffalse
%<*samplechap2>
%\fi
%    \begin{macrocode}
\section{two}
more text in chapter two
%    \end{macrocode}

%\iffalse
%</samplechap2>
%\fi
%
% %%%%%%%%%%%%%%%%%%%%%%%%%%%%%%%%%%%%%%
% \paragraph{Part Include Files.}
%
% The include files are called |cdocspt3.tex| and |cdocspt4.tex|.
%
%\iffalse
%<*samplepart3|samplepart4>
%\fi

% Optional override for |\version| flag:
%    \begin{macrocode}
%%\providecommand{\version}{final}
%    \end{macrocode}

% Include the main document:
%    \begin{macrocode}
\input{childdoc.def}
\childdocby{cdocsamp}
%    \end{macrocode}

%\iffalse
%</samplepart3|samplepart4>
%\fi
%
%\iffalse
%<*samplepart3>
%\fi
% Some text for part 3:
%    \begin{macrocode}
some text in part three
%    \end{macrocode}

%\iffalse
%</samplepart3>
%\fi
% Some text for part 4:
%\iffalse
%<*samplepart4>
%\fi
%    \begin{macrocode}
more text in part four
%    \end{macrocode}

%\iffalse
%</samplepart4>
%\fi
%
% %%%%%%%%%%%%%%%%%%%%%%%%%%%%%%%%%%%%%%
% \paragraph{Forwarding for a Complete Draft.}
%
% The following forwarding file |cdocsdrf.tex|
% compiles the main document in draft mode:
%\iffalse
%<*sampledraft>
%\fi
%    \begin{macrocode}
\def\version{draft}
\input{childdoc.def}
\childdocforward{cdocsamp}
%    \end{macrocode}

%\iffalse
%</sampledraft>
%\fi
%
% %%%%%%%%%%%%%%%%%%%%%%%%%%%%%%%%%%%%%%
% \paragraph{Forwarding for Final Version of the Chapters.}
%
% The following forwarding files |cdocsfn1.tex| and |cdocsfn2.tex|
% (with identical content)
% compile the final versions of the child documents
% |cdocsch1.tex| and |cdocsch2.tex|, respectively:
%\iffalse
%<*samplefinal>
%\fi
%    \begin{macrocode}
\def\version{final}
\input{childdoc.def}
\childdocforwardprefix[cdocsamp]{cdocsfn}{cdocsch}
%    \end{macrocode}

%\iffalse
%</samplefinal>
%\fi
%
% %%%%%%%%%%%%%%%%%%%%%%%%%%%%%%%%%%%%%%
% \paragraph{Command Line Processing.}
%
% The following three command lines generate the output files
% |cdocscld|, |cdocscl1| and |cdocscl2|
% which should be identical to
% |cdocsdrf|, |cdocsch1| and |cdocsfn2|, respectively:
% \begin{center}
% \begin{tabular}{l}
% |latex -jobname cdocscld \|\\
% |  "\def\version{draft}\input{childdoc.def}\childdocforward{cdocsamp}"|\\
% |latex -jobname cdocscl1 \|\\
% |  "\input{childdoc.def}\childdocforward[cdocsamp]{cdocsch1}"|\\
% |latex -jobname cdocscl2 \|\\
% |  "\def\version{final}\input{childdoc.def}\childdocforward{cdocsch2}"|
% \end{tabular}
% \end{center}
% Note that the trailing backslash on each first line
% merely continues the input to the second line
% (for convenient cut ant paste).
% Furthermore, the command |latex| can be replaced by any
% of its alternative versions such as |pdflatex|.
%
% %%%%%%%%%%%%%%%%%%%%%%%%%%%%%%%%%%%%%%%%%%%%%%%%%%%%%%%%%%%%%%%%%%%%%%%%%%%%%%
% %%%%%%%%%%%%%%%%%%%%%%%%%%%%%%%%%%%%%%%%%%%%%%%%%%%%%%%%%%%%%%%%%%%%%%%%%%%%%%
% \section{Implementation}
%\iffalse
%<*package>
%\fi
%
% This section describes the definitions file |childdoc.def|.

% The definitions cannot be loaded using |\usepackage| or |\RequirePackage|
% which has a mechanism to prevent loading a style file more than once.
% When loading the definitions by means of |\input|
% multiple instances have to be prevented manually:
%\iffalse
%This code needs to be before the `\ProvidesFile' directive
%which is defined at the beginning of this file.
%Therefore it is also placed there and commented out here.
%</package>
%<*discard>
%\fi
%    \begin{macrocode}
\ifdefined\childdocmain\endinput\fi
%    \end{macrocode}
%\iffalse
%</discard>
%<*package>
%\fi
%
% \macro{\ifchilddoc}
% \macro{\ifchilddocmanual}
% The conditional |\ifchilddoc| tells whether a
% child (true) or main (false) document is being compiled.
% The conditional |\ifchilddocmanual| tells whether
% the |\includeonly| mechanism is used (false) or
% the selection of child files must be performed manually (true).
% The definitions initialise to false:
%    \begin{macrocode}
\newif\ifchilddoc
\newif\ifchilddocmanual
%    \end{macrocode}

% \macro{\childdocname}
% \macro{\childdocjob}
% The macro |\childdocname| stores the name of the main document
% to be compiled. The macro |\childdocjob| stores the name of
% the document on which the \LaTeX{} compiler was originally invoked.
% The content of |\jobname| cannot be compared
% to filenames specified in the source due to different catcodes.
% The following code rescans |\jobname|, stores the result
% in |\childdocname| and saves a copy in |\childdocjob|:
%    \begin{macrocode}
\edef\childdocname{\scantokens\expandafter{\jobname\noexpand}}
\let\childdocjob\childdocname
%    \end{macrocode}

% \macro{\childdocdisable}
% The macro |\childdocdisable| prevents the main file
% from being processed more than once.
% At this stage, the main document command |\childdocmain|
% is assumed to be called once again where it should do nothing.
% Any subsequent call to it should prevent
% a secondary processing of the main document
% It overwrites the forwarding commands
% |\childdocof| and |\childdocforward|
% with empty macros to prevent further inclusions of the main document:
%    \begin{macrocode}
\newcommand{\childdocdisable}
{
  \renewcommand{\childdocmain}[1]{\renewcommand{\childdocmain}[1]{\endinput}}
  \renewcommand{\childdocof}[1]{}
  \renewcommand{\childdocby}[2][]{}
  \renewcommand{\childdocforward}[2][]{}
  \renewcommand{\childdocdisable}{}
}
%    \end{macrocode}

% \macro{\childdocmain}
% The macro |\childdocmain| is to be called at the top of the main file
% with nothing or the main filename (without extension) as argument.
% First, it breaks loops.
% If the argument is not empty and does not match |\childdocname|
% (which is set by the first inclusion of |childdoc.def|),
% |\ifchilddoc| is set to true, |\includeonly| is applied to the child file
% and |\jobname| is set to the main file
% (for proper handling of |.aux| files):
%    \begin{macrocode}
\newcommand{\childdocmain}[1]
{
  \childdocdisable\childdocmain{}
  \if?#1?\else
    \begingroup
      \def\childdoctmp{#1}
      \ifx\childdoctmp\childdocname
        \def\childdoctmp{}
      \else
        \def\childdoctmp
        {
          \childdoctrue
          \includeonly{\childdocname}
          \def\childdocjob{#1}
          \def\jobname{#1}
        }
      \fi
      \expandafter
    \endgroup
    \childdoctmp
  \fi
}
%    \end{macrocode}

% \macro{\childdocof}
% The command |\childdocof| redirects
% compilation to the main file |#1|.
%    \begin{macrocode}
\newcommand{\childdocof}[1]
{
  \childdocdisable
  \childdoctrue
  \includeonly{\childdocname}
  \def\jobname{#1}
  \def\childdocjob{#1}
  \input{#1}
}
%    \end{macrocode}

% \macro{\childdocby}
% The command |\childdocby| ....
%    \begin{macrocode}
\newcommand{\childdocby}[2][]
{
  \childdocdisable
  \childdoctrue
  \childdocmanualtrue
  \if?#1?\else
    \def\jobname{#2}
  \fi
  \def\childdocjob{#2}
  \input{#2}
  \endinput
}
%    \end{macrocode}

% \macro{\childdocforward}
% The command |\childdocforward| redirects
% compilation to the main file or
% (if the optional argument is given) a child file.
% Parameters are set as if the main file
% or a child file starting with |\childdocof| was compiled.
% Then compilation is handed over to the main file:
%    \begin{macrocode}
\newcommand{\childdocforward}[2][]
{
  \begingroup
    \if?#1?
      \def\childdoctmp
      {
        \def\childdocname{#2}
        \def\childdocjob{#2}
        \def\jobname{#2}
        \input{#2}
        \endinput
      }
    \else
      \def\childdoctmp
      {
        \childdocdisable
        \def\childdocname{#2}
        \childdoctrue
        \includeonly{#2}
        \def\childdocjob{#1}
        \def\jobname{#1}
        \input{#1}
        \endinput
      }
    \fi
    \expandafter
  \endgroup
  \childdoctmp
}
%    \end{macrocode}

% \macro{\childdocforwardprefix}
% The command |\childdocforwardprefix| redirects
% compilation to the main or a child file by means of a pattern.
% The prefix |#1| in the current filename is replaced by |#2|
% and the suffix of the current filename is kept
% (it is assumed that the filename does not contain the substring `|~~~|'
% which is used as a delimiter).
% Compilation is handed over to the new file by |\childdocforward|:
%    \begin{macrocode}
\newcommand{\childdocforwardprefix}[3][]
{
  \begingroup
    \def\childdocextract #2##1~~~{\def\childdoctmp{\childdocforward[#1]{#3##1}}}
    \expandafter\childdocextract\childdocname~~~
    \expandafter
  \endgroup
  \childdoctmp
}
%    \end{macrocode}

% \macro{\childdoc}
% The deprecated macro |\childdoc| is a legacy version of |\childdocmain|:
%    \begin{macrocode}
\newcommand{\childdoc}{\childdocmain}
%    \end{macrocode}

% \macro{\childdocredirect}
% The deprecated macro |\childdocredirect| is a legacy version
% of |\childdocforward| and |\childdocforwardprefix|:
%    \begin{macrocode}
\newcommand{\childdocredirect}[2][]
{
  \begingroup
    \if?#1?
      \def\childdoctmp{\childdocforward{#2}}
    \else
      \def\childdoctmp{\childdocforwardprefix{#1}{#2}}
    \fi
    \expandafter
  \endgroup
  \childdoctmp
}
%    \end{macrocode}

%\iffalse
%</package>
%\fi
%
\endinput
|\\
|\childdocby{|\textit{main}|}|\\
\end{tabular}
\end{center}
%
The directive |\childdocby| is similar to |\childdocof|
described in \secref{sec:include},
but the subsequent selection of content must be done manually.
To that end, both |\ifchilddoc| and |\ifchilddocmanual|
will be true upon processing of a part,
and the name of the part is stored in |\childdocname|.
Note that |\jobname| will be set to the filename of the current part
so that each part receives an individual |.aux| file
that does not interfere with the |.aux| file(s) of the main document.
This behaviour can be altered by the alternative form
|\childdocby[*]{|\textit{main}|}| (with a non-empty optional argument)
which uses the |.aux| file of the main document
by setting |\jobname| to \textit{main}.

%%%%%%%%%%%%%%%%%%%%%%%%%%%%%%%%%%%%%%%%%%%%%%%%%%%%%%%%%%%%%%%%%%%%%%%%%%%%%%%%
\subsection{Driver Development}
\label{sec:driver}

The \textsf{childdoc} mechanism can also be use for the development
of definition files such as \LaTeX{} styles or classes.
This case differs from the above setup with multiple parts
included by |\include| in that no |\includeonly| should be invoked.
This can be achieved by starting the include file
(before |\ProvidesPackage|) with:
%
\begin{center}
\begin{tabular}{l}
|% \iffalse
%
% childdoc.dtx Copyright (C) 2017-2018 Niklas Beisert
%
% This work may be distributed and/or modified under the
% conditions of the LaTeX Project Public License, either version 1.3
% of this license or (at your option) any later version.
% The latest version of this license is in
%   http://www.latex-project.org/lppl.txt
% and version 1.3 or later is part of all distributions of LaTeX
% version 2005/12/01 or later.
%
% This work has the LPPL maintenance status `maintained'.
%
% The Current Maintainer of this work is Niklas Beisert.
%
% This work consists of the files childdoc.dtx and childdoc.ins
% and the derived files childdoc.def and cdocsamp.tex with
% cdocsch1.tex, cdocsch2.tex, cdocsdrf.tex, cdocsfn1.tex, cdocsfn2.tex.
%
%<package>\ifdefined\childdocmain\endinput\fi
%<package>\ProvidesFile{childdoc.def}[2018/12/30 v2.0 child document driver]
%<samplemain>\ProvidesFile{cdocsamp.tex}[2018/12/30 v2.0 sample for childdoc]
%<*driver>
%\ProvidesFile{childdoc.drv}[2018/12/30 v2.0 childdoc reference manual file]
\PassOptionsToClass{10pt,a4paper}{article}
\documentclass{ltxdoc}

\usepackage[margin=35mm]{geometry}
\usepackage{hyperref}
\usepackage{hyperxmp}
\usepackage[usenames]{color}

\hypersetup{colorlinks=true}
\hypersetup{pdfstartview=FitH}
\hypersetup{pdfpagemode=UseNone}
\hypersetup{pdfsource={}}
\hypersetup{pdflang={en-UK}}
\hypersetup{pdfcopyright={Copyright 2017-2018 Niklas Beisert.
  This work may be distributed and/or modified under the
  conditions of the LaTeX Project Public License, either version 1.3
  of this license or (at your option) any later version.}}
\hypersetup{pdflicenseurl={http://www.latex-project.org/lppl.txt}}
\hypersetup{pdfcontactaddress={ETH Zurich, ITP, HIT K,
  Wolfgang-Pauli-Strasse 27}}
\hypersetup{pdfcontactpostcode={8093}}
\hypersetup{pdfcontactcity={Zurich}}
\hypersetup{pdfcontactcountry={Switzerland}}
\hypersetup{pdfcontactemail={nbeisert@itp.phys.ethz.ch}}
\hypersetup{pdfcontacturl={http://people.phys.ethz.ch/\xmptilde nbeisert/}}

\newcommand{\secref}[1]{\hyperref[#1]{section \ref*{#1}}}

\parskip1ex
\parindent0pt
\let\olditemize\itemize
\def\itemize{\olditemize\parskip0pt}

\begin{document}

\title{The \textsf{childdoc} Package}
\hypersetup{pdftitle={The childdoc Package}}
\author{Niklas Beisert\\[2ex]
  Institut f\"ur Theoretische Physik\\
  Eidgen\"ossische Technische Hochschule Z\"urich\\
  Wolfgang-Pauli-Strasse 27, 8093 Z\"urich, Switzerland\\[1ex]
  \href{mailto:nbeisert@itp.phys.ethz.ch}
  {\texttt{nbeisert@itp.phys.ethz.ch}}}
\hypersetup{pdfauthor={Niklas Beisert}}
\hypersetup{pdfsubject={Manual for the LaTeX2e Package childdoc}}
\date{30 December 2018, \textsf{v2.0}}
\maketitle

\begin{abstract}\noindent
\textsf{childdoc} is a \LaTeXe{} package
that enables the direct compilation
of document sections included by |\include|
to individual files.
\end{abstract}

\begingroup
\parskip0ex
\tableofcontents
\endgroup

%%%%%%%%%%%%%%%%%%%%%%%%%%%%%%%%%%%%%%%%%%%%%%%%%%%%%%%%%%%%%%%%%%%%%%%%%%%%%%%%
%%%%%%%%%%%%%%%%%%%%%%%%%%%%%%%%%%%%%%%%%%%%%%%%%%%%%%%%%%%%%%%%%%%%%%%%%%%%%%%%
\section{Introduction}

\LaTeX{} provides a mechanism to structure a large document (such as a book)
into a main file and several child files (containing the chapters)
using the |\include| command.
This mechanism is beneficial for documents
which span hundreds of pages in order to
make the source file(s) more manageable.
Moreover, compilation can be restricted to
selected child files by means of the |\includeonly| command.
The latter feature can be used to reduce the compilation time while editing
(this was significantly more useful in the earlier days of \LaTeX{})
or to generate a smaller document which is easier to navigate.
Another application of |\includeonly| is to generate
documents consisting of selected parts of the complete document.

However, there are a few drawbacks of the plain |\include| mechanism:
\begin{itemize}
\item
The child files cannot be compiled on their own,
they can only be compiled via the main file.
A naive editing environment
(such as a text editor with an option
to have the current file processed by \LaTeX)
may require one to switch to the main file before compiling;
attempting to compile the child file produces errors.
\item
The main file must be modified (each time)
to adjust the |\includeonly| command
to the present needs. This easily leaves the main file in a messy state.
\item
The generated document will always carry the filename
of the main document. This is inconvenient if
several child files are to be compiled and
to be kept for distribution.
\end{itemize}

The present package provides a simple interface
to make child files individually compilable by \LaTeX{}.
Compiling a child file then has the same effect as compiling
the main file with an |\includeonly| command
to select the appropriate child.
Moreover the generated document will carry the name of the child
rather than the main file.
This resolves all three above issues.

This feature is meant to make the editing of books,
thesis documents and lecture notes somewhat more convenient.
However, the package can also be used efficiently for
composing a series of documents (such as exercise sheets)
which are typically distributed individually.
It then assists the author in generating the individual documents
(potentially in different versions)
as well as a document containing the collected series.
Another application is in developing style files
or other kinds of included material
where compilation of the style file could redirect
to a sample or test file.

%%%%%%%%%%%%%%%%%%%%%%%%%%%%%%%%%%%%%%%%%%%%%%%%%%%%%%%%%%%%%%%%%%%%%%%%%%%%%%%%
%%%%%%%%%%%%%%%%%%%%%%%%%%%%%%%%%%%%%%%%%%%%%%%%%%%%%%%%%%%%%%%%%%%%%%%%%%%%%%%%
\section{Usage}

First of all, the package \textsf{childdoc} is \emph{not} a standard
\LaTeXe{} |.sty| style file! Therefore it needs to be invoked in
a non-standard way.

%%%%%%%%%%%%%%%%%%%%%%%%%%%%%%%%%%%%%%%%%%%%%%%%%%%%%%%%%%%%%%%%%%%%%%%%%%%%%%%%
\subsection{Included Files}
\label{sec:include}

%%%%%%%%%%%%%%%%%%%%%%%%%%%%%%%%%%%%%%%%
\DescribeMacro{\childdocmain}
To use the package, add the commands
\begin{center}
\begin{tabular}{l}
|\input{childdoc.def}|\\
|\childdocmain{}|\\
\end{tabular}
\end{center}
at the very top of the main \LaTeX{} file,
in particular \emph{before} the |\documentclass| statement!
The argument of |\childdocmain| should be left empty
(but it must be present).

%%%%%%%%%%%%%%%%%%%%%%%%%%%%%%%%%%%%%%%%
\DescribeMacro{\childdocof}
Furthermore, add the commands
\begin{center}
\begin{tabular}{l}
|\input{childdoc.def}|\\
|\childdocof{|\textit{main}|}|\\
\end{tabular}
\end{center}
at the top of every child file \textit{child}
which is included by |\include{|\textit{child}|}|
from within the main file
(or at least for those files to be compiled individually).
The argument \textit{main} must be the filename of the main file.

There are a couple of
considerations in setting up the main and child documents:

%%%%%%%%%%%%%%%%%%%%%%%%%%%%%%%%%%%%%%%%
\paragraph{Restrictions.}

Please note the following restrictions:
\begin{itemize}
\item
|\childdocmain| must be called with one argument \textit{main}
to ensure compatibility with earlier version of the package.
It must either be empty (|\childdocmain{}|)
or precisely match the filename of the main file in which it is specified.
See \secref{sec:detection} for further information.
\item
The filename \textit{main} must be specified without the |.tex| extension.
\item
The filename \textit{main} is case sensitive
(even in case-insensitive file systems)
due to internal string comparison.
\item
The argument \textit{main} should be fully expanded, it cannot be a macro.
\item
Subdirectories and special characters should be avoided in filenames.
\item
The command |\childdocmain{|\textit{main}|}| must be followed by a whitespace.
It should not be followed immediately by another command
or by a comment mark `|%|'.
This is because the \TeX{} parser reads the token immediately following
the argument of |\childdocmain| and puts it
at the beginning of every child section;
however, a white\-space is ignored.
\end{itemize}

%%%%%%%%%%%%%%%%%%%%%%%%%%%%%%%%%%%%%%%%
\paragraph{Content of Main File.}

It is advisable to place all content in the child files included by |\include|.
Any output contained in the main file will appear in all child documents
unless suppressed manually;
it cannot be suppressed automatically by the |\includeonly| directive
and thus should normally be avoided.
A method to include some content in the main file
by means of conditional processing is described in \secref{sec:conditional}.

%%%%%%%%%%%%%%%%%%%%%%%%%%%%%%%%%%%%%%%%
\paragraph{Page Numbering.}

When only a part of the document is compiled,
the appropriate numbering of pages
(as well as other status parameters)
is determined from the |.aux| files.
The latter contain information from previous passes.
However this information needs to propagate through
all intermediate child documents.
Therefore the page numbering in child documents may well
be inconsistent until the complete document is compiled at least once.

A useful (if unconventional) way to always ensure a consistent
page numbering is to restart the numbering in each child document
and denote the pages by `\textit{child}|.|\textit{page}'
where \textit{child} represents the chapter/section number of the child file.
This can be achieved by the command
|\numberwithin{page}{|\textit{child}|}|
of the \textsf{amsmath} package
where \textit{child} can be |chapter| or |section|
depending on the chosen structuring.
Alternatively, one can modify the macro |\thepage| appropriately
and reset the counter |page| at the start of each child file.

%%%%%%%%%%%%%%%%%%%%%%%%%%%%%%%%%%%%%%%%%%%%%%%%%%%%%%%%%%%%%%%%%%%%%%%%%%%%%%%%
\subsection{Conditional Processing}
\label{sec:conditional}

The package provides a mechanism to compile different versions
of a document. To customise the versions further some conditional processing
can come in handy to distinguish which version is being compiled.
The package provides two macros to describe the compilation context:

%%%%%%%%%%%%%%%%%%%%%%%%%%%%%%%%%%%%%%%%
\DescribeMacro{\ifchilddoc}
The conditional |\ifchilddoc| distinguishes between the compilation of
child documents and the main document:
%
\begin{center}
|\ifchilddoc |\textit{child-code}| |[|\||else |\textit{main-code}]| \||fi|
\end{center}

%%%%%%%%%%%%%%%%%%%%%%%%%%%%%%%%%%%%%%%%
\DescribeMacro{\childdocname}
\DescribeMacro{\childdocjob}
The macro |\childdocname| contains the filename (without extension)
of the main or child file being processed.
Note that |\childdocjob| will always contain the name of the main file.

%%%%%%%%%%%%%%%%%%%%%%%%%%%%%%%%%%%%%%%%
\paragraph{Title Page.}

Conditional processing can be used to include a title or banner page
in the main document when proper precautions are taken.
Importantly, the code in the main file should ensure that the page counter
(as well as other status parameters which are stored in the |.aux| files)
takes the same value after the conditional processing.
Otherwise the page numbers may take divergent values
depending on which part is compiled.

For example, a title page could be declared by:
%
\begin{center}
\begin{tabular}{l}
|\ifchilddoc\||else|\\
|\addtocounter{page}{-1}|\\
\textit{code for title page}\\
|\newpage|\\
|\||fi|
\end{tabular}
\end{center}
%
A banner page for the child documents can be generated by:
%
\begin{center}
\begin{tabular}{l}
|\ifchilddoc|\\
|\addtocounter{page}{-1}|\\
\textit{code for banner page}\\
|\newpage|\\
|\||fi|
\end{tabular}
\end{center}
%
Here one could write a message such as:
\begin{center}
|This is the part \childdocname{} of \childdocjob{}.|
\end{center}

%%%%%%%%%%%%%%%%%%%%%%%%%%%%%%%%%%%%%%%%%%%%%%%%%%%%%%%%%%%%%%%%%%%%%%%%%%%%%%%%
\subsection{Flags}
\label{sec:flags}

The package makes it easy to generate different versions
of the main or child documents.
To this end compilation flags can be defined
and assigned different default values.
They will be particularly useful in conjunction
with the forwarding mechanism described in \secref{sec:forward}.

For example, it may be useful to have a flag |\version|
which can be set to |draft| or |final|.
The document source will contain some conditional code
depending on the value of |\version|.
Suppose further, the flag should default to |final| for the main file
and to |draft| for child files
which is a natural assignment for editing the document.
This is achieved by placing the following code
in the preamble of the main document
(below the |\childdocmain| directive):
%
\begin{center}
\begin{tabular}{l}
|\ifchilddoc|\\
|\providecommand{\version}{draft}|\\
|\||else|\\
|\providecommand{\version}{final}|\\
|\||fi|
\end{tabular}
\end{center}
%
The definition by |\providecommand| makes sure
that previous definitions are not overwritten.
Further statements |\providecommand{\version}{...}|
can thus be added before the above code to override it.

For the main file, one might add a line
(between |\childdocmain| and the above block)
%
\begin{center}
|%\ifchilddoc\||else\providecommand{\version}{draft}\||fi|
\end{center}
%
which can be uncommented to produce a draft version.
Likewise one can add a line to the very top of a child file
(above the |\childdocof{|\textit{main}|}| directive)
%
\begin{center}
|%\providecommand{\version}{final}|
\end{center}
%
which can be uncommented to produce the final version of this child document.

%%%%%%%%%%%%%%%%%%%%%%%%%%%%%%%%%%%%%%%%%%%%%%%%%%%%%%%%%%%%%%%%%%%%%%%%%%%%%%%%
\subsection{Forwarding}
\label{sec:forward}

Different versions of the main or child documents
using compilation flags as described in \secref{sec:flags}
can be (permanently) stored in different files
for convenient compilation, viewing and distribution.
To this end, the package defines a command
to pass on compilation to a different file:

%%%%%%%%%%%%%%%%%%%%%%%%%%%%%%%%%%%%%%%%
\DescribeMacro{\childdocforward}
The command |\childdocforward| redirects processing to
another source file:
%
\begin{center}
\begin{tabular}{l}
|\input{childdoc.def}|\\
|\childdocforward[|\textit{main}|]{|\textit{dest}|}|\\
\end{tabular}
\end{center}
%
The argument \textit{dest} is the destination file
(without extension).
It should be the main file or one of the child files.
Note that further \textsf{childdoc} directives
such as |\childdocof| and |\childdocforward|
in the indicated file will be processed in this form.
The optional argument \textit{main}
passes on directly to the main file \textit{main}
while pretending to compile the child \textit{dest}.
This form behaves as if \textit{dest}
issues |\childdocof{|\textit{main}|}| right away,
and no further \textsf{childdoc} directives will be processed.

%%%%%%%%%%%%%%%%%%%%%%%%%%%%%%%%%%%%%%%%
\DescribeMacro{\...prefix}
In the alternative form |\childdocforwardprefix|,
%
\begin{center}
\begin{tabular}{l}
|\input{childdoc.def}|\\
|\childdocforwardprefix[|\textit{main}|]{|\textit{prefix}|}{|\textit{dest}|}|
\end{tabular}
\end{center}
%
the destination file is determined by a pattern
depending on the current file:
To make this work, the current file must be called
`{\textit{prefix}\hspace{0.2em}\textit{suffix}}'
with \textit{prefix} matching precisely the argument.
Processing is then passed on to the file
`{\textit{dest}\hspace{0.2em}\textit{suffix}}'.
Surely, the same effect is achieved by
directly specifying the
argument `{\textit{dest}\hspace{0.2em}\textit{suffix}}'
in the first form.
However, that requires to set up a different file
for each child. With the alternative form of the command
all these files can have exactly the same content
which simplifies setting them up and maintaining them.

For example, the following file |draft.tex|
with a compilation flag |\version| as described in \secref{sec:flags}
compiles the main document as a draft:
%
\begin{center}
\begin{tabular}{l}
|\def\version{draft}|\\
|\input{childdoc.def}|\\
|\childdocforward{|\textit{main}|}|
\end{tabular}
\end{center}
%
Likewise, the following files |final|\textit{nn}|.tex|
compile the final version of the child document
|child|\textit{nn}|.tex|:
%
\begin{center}
\begin{tabular}{l}
|\def\version{final}|\\
|\input{childdoc.def}|\\
|\childdocforwardprefix{final}{child}|
\end{tabular}
\end{center}
%

Note that when several versions of a main file and/or of each child file
are to be generated, it may be convenient to set up a |Makefile| or
shell script to automatise the process.

%%%%%%%%%%%%%%%%%%%%%%%%%%%%%%%%%%%%%%%%%%%%%%%%%%%%%%%%%%%%%%%%%%%%%%%%%%%%%%%%
\subsection{Command Line Processing}
\label{sec:commandline}

The effect of redirection files can also be achieved by invoking
the \LaTeX{} compiler with a more elaborate command line.
Most conveniently this should be done as part
of a shell script or a |Makefile|.

When using \textsf{childdoc} in the main file, the following
command lines effectively perform a redirection
(note that depending on the shell being used,
backslashes may have to be doubled: `|\|' $\to$ `|\\|'):
%
\begin{center}
|... -jobname "|\textit{target}|" |\\|"|[\textit{flags}]%
|\input{childdoc.def}\childdocforward[|\textit{main}|]{|\textit{dest}|}"|
\end{center}
%
Here \textit{target} is the name of the output file,
\textit{main} is the name of the main file
and \textit{dest} is the name of the main or child file to be processed
(all filenames without extensions).
The optional argument \textit{main} can be omitted
if \textit{main} matches \textit{dest}.
Optionally, compilation \textit{flags} can be defined via |\def| commands.
This command line makes the \TeX{} engine believe
it is compiling the file \textit{target}
whose content is specified as the latter parameter.
The provided code then forwards the processing to
\textit{main} or \textit{dest} as described in \secref{sec:forward}.

%%%%%%%%%%%%%%%%%%%%%%%%%%%%%%%%%%%%%%%%%%%%%%%%%%%%%%%%%%%%%%%%%%%%%%%%%%%%%%%%
\subsection{Include by Input}
\label{sec:input}

Including child documents by |\include| has some restrictions by design.
Most notably, the content of a child document always occupies
its own set of pages; pages cannot be shared between child documents.
Usually, this behaviour makes perfect sense
because each child document contain an essential part of the document.
However, in some situations it may be desirable to compose
a document from a collection of parts
without having mandatory page breaks between then.
For this case, the package
provides a mechanism to include parts
by |\input| which can also be processed individually.
However, by construction this mechanism
requires manual handling of the content to be output.

%%%%%%%%%%%%%%%%%%%%%%%%%%%%%%%%%%%%%%%%
\DescribeMacro{\ifchilddocmanual}
The main file should be prepared as usual, see \secref{sec:include}.
However, the document body must make a distinction
between processing of an individual part and of the main document, e.g.:
%
\begin{center}
\begin{tabular}{l}
|\ifchilddocmanual|\\
|\input{\childdocname}|\\
|\||else|\\
\textit{document body with }|\input{|\textit{part}|}|\\
|\||fi|
\end{tabular}
\end{center}
%
The conditional |\ifchilddocmanual| is true whenever
a part to be included by |\input| is being compiled,
and the name of the part is stored in |\childdocname|.

%%%%%%%%%%%%%%%%%%%%%%%%%%%%%%%%%%%%%%%%
\DescribeMacro{\childdocby}
Each part to be included by |\input| should start with:
%
\begin{center}
\begin{tabular}{l}
|\input{childdoc.def}|\\
|\childdocby{|\textit{main}|}|\\
\end{tabular}
\end{center}
%
The directive |\childdocby| is similar to |\childdocof|
described in \secref{sec:include},
but the subsequent selection of content must be done manually.
To that end, both |\ifchilddoc| and |\ifchilddocmanual|
will be true upon processing of a part,
and the name of the part is stored in |\childdocname|.
Note that |\jobname| will be set to the filename of the current part
so that each part receives an individual |.aux| file
that does not interfere with the |.aux| file(s) of the main document.
This behaviour can be altered by the alternative form
|\childdocby[*]{|\textit{main}|}| (with a non-empty optional argument)
which uses the |.aux| file of the main document
by setting |\jobname| to \textit{main}.

%%%%%%%%%%%%%%%%%%%%%%%%%%%%%%%%%%%%%%%%%%%%%%%%%%%%%%%%%%%%%%%%%%%%%%%%%%%%%%%%
\subsection{Driver Development}
\label{sec:driver}

The \textsf{childdoc} mechanism can also be use for the development
of definition files such as \LaTeX{} styles or classes.
This case differs from the above setup with multiple parts
included by |\include| in that no |\includeonly| should be invoked.
This can be achieved by starting the include file
(before |\ProvidesPackage|) with:
%
\begin{center}
\begin{tabular}{l}
|\input{childdoc.def}|\\
|\childdocforward{|\textit{main}|}|\\
\end{tabular}
\end{center}
%
or alternatively with:
%
\begin{center}
\begin{tabular}{l}
|\input{childdoc.def}|\\
|\childdocby{|\textit{main}|}|\\
\end{tabular}
\end{center}
%
Both forms have slightly different effects as described above.
The main file is prepared as usual, see \secref{sec:include}.

%%%%%%%%%%%%%%%%%%%%%%%%%%%%%%%%%%%%%%%%%%%%%%%%%%%%%%%%%%%%%%%%%%%%%%%%%%%%%%%%
\subsection{Legacy Detection}
\label{sec:detection}

The directive |\childdocmain| in the main file can detect
whether the complete document or merely a child is to be compiled
even without using the directive |\childdocof|.
This method is deprecated because it is less robust
and there is no compelling reason to use it;
it is merely provided for backward compatibility
and it may be removed in future versions.

If the detection mechanism is to be used,
it is mandatory to correctly specify
the filename of the main file as the argument of |\childdocmain|:
%
\begin{center}
\begin{tabular}{l}
|\input{childdoc.def}|\\
|\childdocmain{|\textit{main}|}|\\
\end{tabular}
\end{center}
%
If |\jobname| does not match the argument \textit{main} of |\childdocmain|,
it is assumed that |\jobname| points to the child file to be compiled.
When using |\childdocmain| with the main file specified as argument,
it suffices to start a child file
with just |\input{|\textit{main}|}|
without loading of the package and using |\childdocof|.
If instead all processing is done
with the appropriate \textsf{childdoc} directives,
the argument of \textit{main} of |\childdocmain| can be empty.

An alternative version of the command line processing described
in \secref{sec:commandline} using the detection mechanism reads:
%
\begin{center}
|... -jobname "|\textit{target}|" "|[\textit{flags}]%
[|\def\jobname{|\textit{dest}|}|]|\input{|\textit{main}|}"|
\end{center}

%%%%%%%%%%%%%%%%%%%%%%%%%%%%%%%%%%%%%%%%%%%%%%%%%%%%%%%%%%%%%%%%%%%%%%%%%%%%%%%%
\subsection{Manual Code}
\label{sec:manual}

In case one cannot be certain whether the definitions file |childdoc.def|
is installed on the target \TeX{} distribution
and one prefers not to ship it,
it is conceivable to paste a few relevant commands into the sources.

To that end, drop all statements |\input{childdoc.def}|
and perform the replacements as outlined below.
Instead of |\childdocmain{|\textit{main}|}| add the following code
to the top of the main file:
%
\begin{center}
\begin{tabular}{l}
|\||ifdefined\childdocname\endinput\||fi\newif\ifchilddoc|\\
|\edef\childdocname{\scantokens\expandafter{\jobname\noexpand}}|\\
|\def\childdocmain{|\textit{main}|}\||ifx\childdocmain\childdocname\||else|\\
|\childdoctrue\includeonly{\childdocname}\let\jobname\childdocmain\||fi|\\
\end{tabular}
\end{center}
%
Instead of |\childdocof{|\textit{main}|}| just include the main file
at the top of each child file:
%
\begin{center}
|\input{|\textit{main}|}|
\end{center}
%
A simple redirection |\childdocforward{|\textit{dest}|}| is achieved by:
%
\begin{center}
|\def\jobname{|\textit{dest}|}\input{\jobname}|
\end{center}
%
The redirection with prefix
|\childdocforwardprefix[|\textit{prefix}|]{|\textit{dest}|}|
is accomplished by:
%
\begin{center}
\begin{tabular}{l}
|{\edef\jobname{\scantokens\expandafter{\jobname\noexpand}}|\\
|\def\redirectjob |\textit{prefix}|#1~~~{\gdef\jobname{|\textit{dest}|#1}}|\\
|\expandafter\redirectjob\jobname~~~}\input{\jobname}|
\end{tabular}
\end{center}

In an alternative approach,
child documents can be compiled by a specific command line
without additional code or specific definitions:
%
\begin{center}
|... -jobname "|\textit{target}|" "|[\textit{flags}]%
|\includeonly{|\textit{dest}|}\input{|\textit{main}|}"|
\end{center}
%

%%%%%%%%%%%%%%%%%%%%%%%%%%%%%%%%%%%%%%%%%%%%%%%%%%%%%%%%%%%%%%%%%%%%%%%%%%%%%%%%
%%%%%%%%%%%%%%%%%%%%%%%%%%%%%%%%%%%%%%%%%%%%%%%%%%%%%%%%%%%%%%%%%%%%%%%%%%%%%%%%
\section{Information}

%%%%%%%%%%%%%%%%%%%%%%%%%%%%%%%%%%%%%%%%%%%%%%%%%%%%%%%%%%%%%%%%%%%%%%%%%%%%%%%%
\subsection{Copyright}

Copyright \copyright{} 2017--2018 Niklas Beisert

This work may be distributed and/or modified under the
conditions of the \LaTeX{} Project Public License, either version 1.3
of this license or (at your option) any later version.
The latest version of this license is in
  \url{http://www.latex-project.org/lppl.txt}
and version 1.3 or later is part of all distributions of \LaTeX{}
version 2005/12/01 or later.

This work has the LPPL maintenance status `maintained'.

The Current Maintainer of this work is Niklas Beisert.

This work consists of the files |README.txt|, |childdoc.ins| and |childdoc.dtx|
as well as the derived files |childdoc.def|, |cdocsamp.tex|
with |cdocsch1.tex|, |cdocsch2.tex|, |cdocspt3.tex|, |cdocspt4.tex|,
|cdocsdrf.tex|, |cdocsfn1.tex|, |cdocsfn2.tex|
as well as |childdoc.pdf|.

%%%%%%%%%%%%%%%%%%%%%%%%%%%%%%%%%%%%%%%%%%%%%%%%%%%%%%%%%%%%%%%%%%%%%%%%%%%%%%%%
\subsection{Files and Installation}

The package consists of the files:
%
\begin{center}
\begin{tabular}{ll}
    |README.txt|   & readme file \\
    |childdoc.ins| & installation file \\
    |childdoc.dtx| & source file \\
    |childdoc.def| & definition file \\
    |cdocsamp.tex| & sample main file \\
    |cdocsch1.tex| & sample include file \\
    |cdocsch2.tex| & sample include file \\
    |cdocspt3.tex| & sample part file \\
    |cdocspt4.tex| & sample part file \\
    |cdocsdrf.tex| & sample redirection file \\
    |cdocsfn1.tex| & sample redirection file \\
    |cdocsfn2.tex| & sample redirection file \\
    |childdoc.pdf| & manual
\end{tabular}
\end{center}
%
The distribution consists of the files
|README.txt|, |childdoc.ins| and |childdoc.dtx|.
%
\begin{itemize}
\item
Run (pdf)\LaTeX{} on |childdoc.dtx|
to compile the manual |childdoc.pdf| (this file).
\item
Run \LaTeX{} on |childdoc.ins| to create the definitions file |childdoc.def|
and the sample |cdocsamp.tex| with include files
|cdocsch1.tex|, |cdocsch2.tex|, |cdocspt3.tex|, |cdocspt4.tex|,
|cdocsdrf.tex|, |cdocsfn1.tex|, |cdocsfn2.tex|.
Then copy the file |childdoc.def| to an appropriate directory of your \LaTeX{}
distribution, e.g.\ \textit{texmf-root}|/tex/latex/childdoc|.
\end{itemize}

%%%%%%%%%%%%%%%%%%%%%%%%%%%%%%%%%%%%%%%%%%%%%%%%%%%%%%%%%%%%%%%%%%%%%%%%%%%%%%%%
\subsection{Related CTAN Packages}

There are several other packages which offer a similar functionality:
%
\begin{itemize}
\item
The packages
\href{http://ctan.org/pkg/docmute}{\textsf{docmute}},
\href{http://ctan.org/pkg/includex}{\textsf{includex}} and
\href{http://ctan.org/pkg/standalone}{\textsf{standalone}}
provide commands to include only the document body of
a child file thus allowing both files to be compiled individually.
\item
The packages \href{http://ctan.org/pkg/subdocs}{\textsf{subdocs}}
and \href{http://ctan.org/pkg/subfiles}{\textsf{subfiles}}
provide structures in which the main and child documents can be
encapsulated and allowing them to be compiled individually.
The inclusion mechanism is different from the conventional |\include|.
\item
The package \href{http://ctan.org/pkg/combine}{\textsf{combine}}
is an elaborate solution to combine several documents into one.
\end{itemize}
%
See also the CTAN topic \href{http://ctan.org/topic/subdocs}{\textsf{subdocs}}
for further related packages.
The present package differs from the above solutions in that
a document structure constructed with the conventional |\include| mechanism
just needs two extra commands at the top of every file
such that all constituent files can be compiled individually.

%%%%%%%%%%%%%%%%%%%%%%%%%%%%%%%%%%%%%%%%%%%%%%%%%%%%%%%%%%%%%%%%%%%%%%%%%%%%%%%%
%\subsection{Feature Suggestions}
%
%The following is a list of features which may be useful for future
%versions of this package:
%%
%\begin{itemize}
%\item
%\ldots
%\end{itemize}

%%%%%%%%%%%%%%%%%%%%%%%%%%%%%%%%%%%%%%%%%%%%%%%%%%%%%%%%%%%%%%%%%%%%%%%%%%%%%%%%
\subsection{Revision History}

%%%%%%%%%%%%%%%%%%%%%%%%%%%%%%%%%%%%%%%%
\paragraph{v2.0:} 2018/12/30

\begin{itemize}
\item
immediate forward processing
\item
added |\childdocby| mechanism
\item
manual restructured
\end{itemize}

%%%%%%%%%%%%%%%%%%%%%%%%%%%%%%%%%%%%%%%%
\paragraph{v1.6:} 2018/01/17

\begin{itemize}
\item
application for development of include files
\item
corrections to manual
\end{itemize}

%%%%%%%%%%%%%%%%%%%%%%%%%%%%%%%%%%%%%%%%
\paragraph{v1.5:} 2017/05/21

\begin{itemize}
\item
more complete structuring introduced
\item
|\childdocof| introduced
\item
|\childdoc| renamed to |\childdocmain|
\item
|\childredirect| renamed to |\childdocforward| and |\childdocforwardprefix|
and functionality expanded
\end{itemize}

%%%%%%%%%%%%%%%%%%%%%%%%%%%%%%%%%%%%%%%%
\paragraph{v1.0:} 2017/04/27

\begin{itemize}
\item
manual and install package
\item
first version published on CTAN
\end{itemize}

%%%%%%%%%%%%%%%%%%%%%%%%%%%%%%%%%%%%%%%%
\paragraph{v0.6:} 2017/04/26

\begin{itemize}
\item
redirection mechanism added
\end{itemize}

%%%%%%%%%%%%%%%%%%%%%%%%%%%%%%%%%%%%%%%%
\paragraph{v0.5:} 2017/04/26

\begin{itemize}
\item
functionality in definition file
\end{itemize}


%%%%%%%%%%%%%%%%%%%%%%%%%%%%%%%%%%%%%%%%%%%%%%%%%%%%%%%%%%%%%%%%%%%%%%%%%%%%%%%%
%%%%%%%%%%%%%%%%%%%%%%%%%%%%%%%%%%%%%%%%%%%%%%%%%%%%%%%%%%%%%%%%%%%%%%%%%%%%%%%%
%%%%%%%%%%%%%%%%%%%%%%%%%%%%%%%%%%%%%%%%%%%%%%%%%%%%%%%%%%%%%%%%%%%%%%%%%%%%%%%%
\appendix

\settowidth\MacroIndent{\rmfamily\scriptsize 000\ }

 \DocInput{childdoc.dtx}

\end{document}
%</driver>
% \fi
%
% %%%%%%%%%%%%%%%%%%%%%%%%%%%%%%%%%%%%%%%%%%%%%%%%%%%%%%%%%%%%%%%%%%%%%%%%%%%%%%
% %%%%%%%%%%%%%%%%%%%%%%%%%%%%%%%%%%%%%%%%%%%%%%%%%%%%%%%%%%%%%%%%%%%%%%%%%%%%%%
% \section{Sample}
%\iffalse
%<*samplemain>
%\fi
%
% The following presents a sample document
% with two chapters, two parts, a title page,
% a compile flag as well as three forwarding files to set the flag.
% It consists of eight |.tex| files:
% \begin{center}
% \begin{tabular}{ll}
% |cdocsamp.tex|&main file\\
% |cdocsch1.tex|&include file for chapter 1\\
% |cdocsch2.tex|&include file for chapter 2\\
% |cdocspt3.tex|&include file for part 3\\
% |cdocspt4.tex|&include file for part 4\\
% |cdocsdrf.tex|&forwarding file for main file in draft mode\\
% |cdocsfi1.tex|&forwarding file for final version of chapter 1\\
% |cdocsfi2.tex|&forwarding file for final version of chapter 2\\
% \end{tabular}
% \end{center}
% Each of the eight files can be compiled directly by the \LaTeX{} compiler.
%
% %%%%%%%%%%%%%%%%%%%%%%%%%%%%%%%%%%%%%%
% \paragraph{Main File.}
%
% The main file is called |cdocsamp.tex|.
%
% Load the \textsf{childdoc} definitions and
% declare the filename for the main document:
%    \begin{macrocode}
\input{childdoc.def}
\childdocmain{}
%    \end{macrocode}

% Optional override for |\version| flag:
%    \begin{macrocode}
%%\ifchilddoc\else\providecommand{\version}{draft}\fi
%    \end{macrocode}

% Define the default values for the |\version| flag
% (|final| for the main file and |draft| for childs):
%    \begin{macrocode}
\ifchilddoc
\providecommand{\version}{draft}
\else
\providecommand{\version}{final}
\fi
%    \end{macrocode}

% Load the standard document class:
%    \begin{macrocode}
\documentclass[12pt]{article}
%    \end{macrocode}

% Start the document body:
%    \begin{macrocode}
\begin{document}
%    \end{macrocode}

% Declare a title page.
% Print title, part of document being processed and version flag:
%    \begin{macrocode}
\addtocounter{page}{-1}
\begin{center}
{\LARGE\bfseries{}childdoc example\par}
\vspace{1cm}
\ifchilddoc
\ifchilddocmanual part\else chapter\fi:
`\childdocname' of `\childdocjob'\par
\else
main document: `\childdocjob'\par
\fi
version: \version\par
\end{center}
\newpage
%    \end{macrocode}

% Manually include selected file,
% otherwise process as usual:
%    \begin{macrocode}
\ifchilddocmanual
\section*{part `\childdocname'}
\input{\childdocname}
\else
%    \end{macrocode}

% Include the two chapters:
%    \begin{macrocode}
\include{cdocsch1}
\include{cdocsch2}
%    \end{macrocode}

% Include the two parts unless only chapters should be displayed:
%    \begin{macrocode}
\ifchilddoc\else
\section{part three}
\input{cdocspt3}
\section{part four}
\input{cdocspt4}
\fi
%    \end{macrocode}

% Process as usual until here:
%    \begin{macrocode}
\fi
%    \end{macrocode}

% End of document body:
%    \begin{macrocode}
\end{document}
%    \end{macrocode}
%\iffalse
%</samplemain>
%\fi
%
% %%%%%%%%%%%%%%%%%%%%%%%%%%%%%%%%%%%%%%
% \paragraph{Chapter Include Files.}
%
% The include files are called |cdocsch1.tex| and |cdocsch2.tex|.
%
%\iffalse
%<*samplechap1|samplechap2>
%\fi

% Optional override for |\version| flag:
%    \begin{macrocode}
%%\providecommand{\version}{final}
%    \end{macrocode}

% Include the main document:
%    \begin{macrocode}
\input{childdoc.def}
\childdocof{cdocsamp}
%    \end{macrocode}

%\iffalse
%</samplechap1|samplechap2>
%\fi
%
%\iffalse
%<*samplechap1>
%\fi
% Some text for chapter 1:
%    \begin{macrocode}
\section{one}
some text in chapter one
%    \end{macrocode}

%\iffalse
%</samplechap1>
%\fi
% Some text for chapter 2:
%\iffalse
%<*samplechap2>
%\fi
%    \begin{macrocode}
\section{two}
more text in chapter two
%    \end{macrocode}

%\iffalse
%</samplechap2>
%\fi
%
% %%%%%%%%%%%%%%%%%%%%%%%%%%%%%%%%%%%%%%
% \paragraph{Part Include Files.}
%
% The include files are called |cdocspt3.tex| and |cdocspt4.tex|.
%
%\iffalse
%<*samplepart3|samplepart4>
%\fi

% Optional override for |\version| flag:
%    \begin{macrocode}
%%\providecommand{\version}{final}
%    \end{macrocode}

% Include the main document:
%    \begin{macrocode}
\input{childdoc.def}
\childdocby{cdocsamp}
%    \end{macrocode}

%\iffalse
%</samplepart3|samplepart4>
%\fi
%
%\iffalse
%<*samplepart3>
%\fi
% Some text for part 3:
%    \begin{macrocode}
some text in part three
%    \end{macrocode}

%\iffalse
%</samplepart3>
%\fi
% Some text for part 4:
%\iffalse
%<*samplepart4>
%\fi
%    \begin{macrocode}
more text in part four
%    \end{macrocode}

%\iffalse
%</samplepart4>
%\fi
%
% %%%%%%%%%%%%%%%%%%%%%%%%%%%%%%%%%%%%%%
% \paragraph{Forwarding for a Complete Draft.}
%
% The following forwarding file |cdocsdrf.tex|
% compiles the main document in draft mode:
%\iffalse
%<*sampledraft>
%\fi
%    \begin{macrocode}
\def\version{draft}
\input{childdoc.def}
\childdocforward{cdocsamp}
%    \end{macrocode}

%\iffalse
%</sampledraft>
%\fi
%
% %%%%%%%%%%%%%%%%%%%%%%%%%%%%%%%%%%%%%%
% \paragraph{Forwarding for Final Version of the Chapters.}
%
% The following forwarding files |cdocsfn1.tex| and |cdocsfn2.tex|
% (with identical content)
% compile the final versions of the child documents
% |cdocsch1.tex| and |cdocsch2.tex|, respectively:
%\iffalse
%<*samplefinal>
%\fi
%    \begin{macrocode}
\def\version{final}
\input{childdoc.def}
\childdocforwardprefix[cdocsamp]{cdocsfn}{cdocsch}
%    \end{macrocode}

%\iffalse
%</samplefinal>
%\fi
%
% %%%%%%%%%%%%%%%%%%%%%%%%%%%%%%%%%%%%%%
% \paragraph{Command Line Processing.}
%
% The following three command lines generate the output files
% |cdocscld|, |cdocscl1| and |cdocscl2|
% which should be identical to
% |cdocsdrf|, |cdocsch1| and |cdocsfn2|, respectively:
% \begin{center}
% \begin{tabular}{l}
% |latex -jobname cdocscld \|\\
% |  "\def\version{draft}\input{childdoc.def}\childdocforward{cdocsamp}"|\\
% |latex -jobname cdocscl1 \|\\
% |  "\input{childdoc.def}\childdocforward[cdocsamp]{cdocsch1}"|\\
% |latex -jobname cdocscl2 \|\\
% |  "\def\version{final}\input{childdoc.def}\childdocforward{cdocsch2}"|
% \end{tabular}
% \end{center}
% Note that the trailing backslash on each first line
% merely continues the input to the second line
% (for convenient cut ant paste).
% Furthermore, the command |latex| can be replaced by any
% of its alternative versions such as |pdflatex|.
%
% %%%%%%%%%%%%%%%%%%%%%%%%%%%%%%%%%%%%%%%%%%%%%%%%%%%%%%%%%%%%%%%%%%%%%%%%%%%%%%
% %%%%%%%%%%%%%%%%%%%%%%%%%%%%%%%%%%%%%%%%%%%%%%%%%%%%%%%%%%%%%%%%%%%%%%%%%%%%%%
% \section{Implementation}
%\iffalse
%<*package>
%\fi
%
% This section describes the definitions file |childdoc.def|.

% The definitions cannot be loaded using |\usepackage| or |\RequirePackage|
% which has a mechanism to prevent loading a style file more than once.
% When loading the definitions by means of |\input|
% multiple instances have to be prevented manually:
%\iffalse
%This code needs to be before the `\ProvidesFile' directive
%which is defined at the beginning of this file.
%Therefore it is also placed there and commented out here.
%</package>
%<*discard>
%\fi
%    \begin{macrocode}
\ifdefined\childdocmain\endinput\fi
%    \end{macrocode}
%\iffalse
%</discard>
%<*package>
%\fi
%
% \macro{\ifchilddoc}
% \macro{\ifchilddocmanual}
% The conditional |\ifchilddoc| tells whether a
% child (true) or main (false) document is being compiled.
% The conditional |\ifchilddocmanual| tells whether
% the |\includeonly| mechanism is used (false) or
% the selection of child files must be performed manually (true).
% The definitions initialise to false:
%    \begin{macrocode}
\newif\ifchilddoc
\newif\ifchilddocmanual
%    \end{macrocode}

% \macro{\childdocname}
% \macro{\childdocjob}
% The macro |\childdocname| stores the name of the main document
% to be compiled. The macro |\childdocjob| stores the name of
% the document on which the \LaTeX{} compiler was originally invoked.
% The content of |\jobname| cannot be compared
% to filenames specified in the source due to different catcodes.
% The following code rescans |\jobname|, stores the result
% in |\childdocname| and saves a copy in |\childdocjob|:
%    \begin{macrocode}
\edef\childdocname{\scantokens\expandafter{\jobname\noexpand}}
\let\childdocjob\childdocname
%    \end{macrocode}

% \macro{\childdocdisable}
% The macro |\childdocdisable| prevents the main file
% from being processed more than once.
% At this stage, the main document command |\childdocmain|
% is assumed to be called once again where it should do nothing.
% Any subsequent call to it should prevent
% a secondary processing of the main document
% It overwrites the forwarding commands
% |\childdocof| and |\childdocforward|
% with empty macros to prevent further inclusions of the main document:
%    \begin{macrocode}
\newcommand{\childdocdisable}
{
  \renewcommand{\childdocmain}[1]{\renewcommand{\childdocmain}[1]{\endinput}}
  \renewcommand{\childdocof}[1]{}
  \renewcommand{\childdocby}[2][]{}
  \renewcommand{\childdocforward}[2][]{}
  \renewcommand{\childdocdisable}{}
}
%    \end{macrocode}

% \macro{\childdocmain}
% The macro |\childdocmain| is to be called at the top of the main file
% with nothing or the main filename (without extension) as argument.
% First, it breaks loops.
% If the argument is not empty and does not match |\childdocname|
% (which is set by the first inclusion of |childdoc.def|),
% |\ifchilddoc| is set to true, |\includeonly| is applied to the child file
% and |\jobname| is set to the main file
% (for proper handling of |.aux| files):
%    \begin{macrocode}
\newcommand{\childdocmain}[1]
{
  \childdocdisable\childdocmain{}
  \if?#1?\else
    \begingroup
      \def\childdoctmp{#1}
      \ifx\childdoctmp\childdocname
        \def\childdoctmp{}
      \else
        \def\childdoctmp
        {
          \childdoctrue
          \includeonly{\childdocname}
          \def\childdocjob{#1}
          \def\jobname{#1}
        }
      \fi
      \expandafter
    \endgroup
    \childdoctmp
  \fi
}
%    \end{macrocode}

% \macro{\childdocof}
% The command |\childdocof| redirects
% compilation to the main file |#1|.
%    \begin{macrocode}
\newcommand{\childdocof}[1]
{
  \childdocdisable
  \childdoctrue
  \includeonly{\childdocname}
  \def\jobname{#1}
  \def\childdocjob{#1}
  \input{#1}
}
%    \end{macrocode}

% \macro{\childdocby}
% The command |\childdocby| ....
%    \begin{macrocode}
\newcommand{\childdocby}[2][]
{
  \childdocdisable
  \childdoctrue
  \childdocmanualtrue
  \if?#1?\else
    \def\jobname{#2}
  \fi
  \def\childdocjob{#2}
  \input{#2}
  \endinput
}
%    \end{macrocode}

% \macro{\childdocforward}
% The command |\childdocforward| redirects
% compilation to the main file or
% (if the optional argument is given) a child file.
% Parameters are set as if the main file
% or a child file starting with |\childdocof| was compiled.
% Then compilation is handed over to the main file:
%    \begin{macrocode}
\newcommand{\childdocforward}[2][]
{
  \begingroup
    \if?#1?
      \def\childdoctmp
      {
        \def\childdocname{#2}
        \def\childdocjob{#2}
        \def\jobname{#2}
        \input{#2}
        \endinput
      }
    \else
      \def\childdoctmp
      {
        \childdocdisable
        \def\childdocname{#2}
        \childdoctrue
        \includeonly{#2}
        \def\childdocjob{#1}
        \def\jobname{#1}
        \input{#1}
        \endinput
      }
    \fi
    \expandafter
  \endgroup
  \childdoctmp
}
%    \end{macrocode}

% \macro{\childdocforwardprefix}
% The command |\childdocforwardprefix| redirects
% compilation to the main or a child file by means of a pattern.
% The prefix |#1| in the current filename is replaced by |#2|
% and the suffix of the current filename is kept
% (it is assumed that the filename does not contain the substring `|~~~|'
% which is used as a delimiter).
% Compilation is handed over to the new file by |\childdocforward|:
%    \begin{macrocode}
\newcommand{\childdocforwardprefix}[3][]
{
  \begingroup
    \def\childdocextract #2##1~~~{\def\childdoctmp{\childdocforward[#1]{#3##1}}}
    \expandafter\childdocextract\childdocname~~~
    \expandafter
  \endgroup
  \childdoctmp
}
%    \end{macrocode}

% \macro{\childdoc}
% The deprecated macro |\childdoc| is a legacy version of |\childdocmain|:
%    \begin{macrocode}
\newcommand{\childdoc}{\childdocmain}
%    \end{macrocode}

% \macro{\childdocredirect}
% The deprecated macro |\childdocredirect| is a legacy version
% of |\childdocforward| and |\childdocforwardprefix|:
%    \begin{macrocode}
\newcommand{\childdocredirect}[2][]
{
  \begingroup
    \if?#1?
      \def\childdoctmp{\childdocforward{#2}}
    \else
      \def\childdoctmp{\childdocforwardprefix{#1}{#2}}
    \fi
    \expandafter
  \endgroup
  \childdoctmp
}
%    \end{macrocode}

%\iffalse
%</package>
%\fi
%
\endinput
|\\
|\childdocforward{|\textit{main}|}|\\
\end{tabular}
\end{center}
%
or alternatively with:
%
\begin{center}
\begin{tabular}{l}
|% \iffalse
%
% childdoc.dtx Copyright (C) 2017-2018 Niklas Beisert
%
% This work may be distributed and/or modified under the
% conditions of the LaTeX Project Public License, either version 1.3
% of this license or (at your option) any later version.
% The latest version of this license is in
%   http://www.latex-project.org/lppl.txt
% and version 1.3 or later is part of all distributions of LaTeX
% version 2005/12/01 or later.
%
% This work has the LPPL maintenance status `maintained'.
%
% The Current Maintainer of this work is Niklas Beisert.
%
% This work consists of the files childdoc.dtx and childdoc.ins
% and the derived files childdoc.def and cdocsamp.tex with
% cdocsch1.tex, cdocsch2.tex, cdocsdrf.tex, cdocsfn1.tex, cdocsfn2.tex.
%
%<package>\ifdefined\childdocmain\endinput\fi
%<package>\ProvidesFile{childdoc.def}[2018/12/30 v2.0 child document driver]
%<samplemain>\ProvidesFile{cdocsamp.tex}[2018/12/30 v2.0 sample for childdoc]
%<*driver>
%\ProvidesFile{childdoc.drv}[2018/12/30 v2.0 childdoc reference manual file]
\PassOptionsToClass{10pt,a4paper}{article}
\documentclass{ltxdoc}

\usepackage[margin=35mm]{geometry}
\usepackage{hyperref}
\usepackage{hyperxmp}
\usepackage[usenames]{color}

\hypersetup{colorlinks=true}
\hypersetup{pdfstartview=FitH}
\hypersetup{pdfpagemode=UseNone}
\hypersetup{pdfsource={}}
\hypersetup{pdflang={en-UK}}
\hypersetup{pdfcopyright={Copyright 2017-2018 Niklas Beisert.
  This work may be distributed and/or modified under the
  conditions of the LaTeX Project Public License, either version 1.3
  of this license or (at your option) any later version.}}
\hypersetup{pdflicenseurl={http://www.latex-project.org/lppl.txt}}
\hypersetup{pdfcontactaddress={ETH Zurich, ITP, HIT K,
  Wolfgang-Pauli-Strasse 27}}
\hypersetup{pdfcontactpostcode={8093}}
\hypersetup{pdfcontactcity={Zurich}}
\hypersetup{pdfcontactcountry={Switzerland}}
\hypersetup{pdfcontactemail={nbeisert@itp.phys.ethz.ch}}
\hypersetup{pdfcontacturl={http://people.phys.ethz.ch/\xmptilde nbeisert/}}

\newcommand{\secref}[1]{\hyperref[#1]{section \ref*{#1}}}

\parskip1ex
\parindent0pt
\let\olditemize\itemize
\def\itemize{\olditemize\parskip0pt}

\begin{document}

\title{The \textsf{childdoc} Package}
\hypersetup{pdftitle={The childdoc Package}}
\author{Niklas Beisert\\[2ex]
  Institut f\"ur Theoretische Physik\\
  Eidgen\"ossische Technische Hochschule Z\"urich\\
  Wolfgang-Pauli-Strasse 27, 8093 Z\"urich, Switzerland\\[1ex]
  \href{mailto:nbeisert@itp.phys.ethz.ch}
  {\texttt{nbeisert@itp.phys.ethz.ch}}}
\hypersetup{pdfauthor={Niklas Beisert}}
\hypersetup{pdfsubject={Manual for the LaTeX2e Package childdoc}}
\date{30 December 2018, \textsf{v2.0}}
\maketitle

\begin{abstract}\noindent
\textsf{childdoc} is a \LaTeXe{} package
that enables the direct compilation
of document sections included by |\include|
to individual files.
\end{abstract}

\begingroup
\parskip0ex
\tableofcontents
\endgroup

%%%%%%%%%%%%%%%%%%%%%%%%%%%%%%%%%%%%%%%%%%%%%%%%%%%%%%%%%%%%%%%%%%%%%%%%%%%%%%%%
%%%%%%%%%%%%%%%%%%%%%%%%%%%%%%%%%%%%%%%%%%%%%%%%%%%%%%%%%%%%%%%%%%%%%%%%%%%%%%%%
\section{Introduction}

\LaTeX{} provides a mechanism to structure a large document (such as a book)
into a main file and several child files (containing the chapters)
using the |\include| command.
This mechanism is beneficial for documents
which span hundreds of pages in order to
make the source file(s) more manageable.
Moreover, compilation can be restricted to
selected child files by means of the |\includeonly| command.
The latter feature can be used to reduce the compilation time while editing
(this was significantly more useful in the earlier days of \LaTeX{})
or to generate a smaller document which is easier to navigate.
Another application of |\includeonly| is to generate
documents consisting of selected parts of the complete document.

However, there are a few drawbacks of the plain |\include| mechanism:
\begin{itemize}
\item
The child files cannot be compiled on their own,
they can only be compiled via the main file.
A naive editing environment
(such as a text editor with an option
to have the current file processed by \LaTeX)
may require one to switch to the main file before compiling;
attempting to compile the child file produces errors.
\item
The main file must be modified (each time)
to adjust the |\includeonly| command
to the present needs. This easily leaves the main file in a messy state.
\item
The generated document will always carry the filename
of the main document. This is inconvenient if
several child files are to be compiled and
to be kept for distribution.
\end{itemize}

The present package provides a simple interface
to make child files individually compilable by \LaTeX{}.
Compiling a child file then has the same effect as compiling
the main file with an |\includeonly| command
to select the appropriate child.
Moreover the generated document will carry the name of the child
rather than the main file.
This resolves all three above issues.

This feature is meant to make the editing of books,
thesis documents and lecture notes somewhat more convenient.
However, the package can also be used efficiently for
composing a series of documents (such as exercise sheets)
which are typically distributed individually.
It then assists the author in generating the individual documents
(potentially in different versions)
as well as a document containing the collected series.
Another application is in developing style files
or other kinds of included material
where compilation of the style file could redirect
to a sample or test file.

%%%%%%%%%%%%%%%%%%%%%%%%%%%%%%%%%%%%%%%%%%%%%%%%%%%%%%%%%%%%%%%%%%%%%%%%%%%%%%%%
%%%%%%%%%%%%%%%%%%%%%%%%%%%%%%%%%%%%%%%%%%%%%%%%%%%%%%%%%%%%%%%%%%%%%%%%%%%%%%%%
\section{Usage}

First of all, the package \textsf{childdoc} is \emph{not} a standard
\LaTeXe{} |.sty| style file! Therefore it needs to be invoked in
a non-standard way.

%%%%%%%%%%%%%%%%%%%%%%%%%%%%%%%%%%%%%%%%%%%%%%%%%%%%%%%%%%%%%%%%%%%%%%%%%%%%%%%%
\subsection{Included Files}
\label{sec:include}

%%%%%%%%%%%%%%%%%%%%%%%%%%%%%%%%%%%%%%%%
\DescribeMacro{\childdocmain}
To use the package, add the commands
\begin{center}
\begin{tabular}{l}
|\input{childdoc.def}|\\
|\childdocmain{}|\\
\end{tabular}
\end{center}
at the very top of the main \LaTeX{} file,
in particular \emph{before} the |\documentclass| statement!
The argument of |\childdocmain| should be left empty
(but it must be present).

%%%%%%%%%%%%%%%%%%%%%%%%%%%%%%%%%%%%%%%%
\DescribeMacro{\childdocof}
Furthermore, add the commands
\begin{center}
\begin{tabular}{l}
|\input{childdoc.def}|\\
|\childdocof{|\textit{main}|}|\\
\end{tabular}
\end{center}
at the top of every child file \textit{child}
which is included by |\include{|\textit{child}|}|
from within the main file
(or at least for those files to be compiled individually).
The argument \textit{main} must be the filename of the main file.

There are a couple of
considerations in setting up the main and child documents:

%%%%%%%%%%%%%%%%%%%%%%%%%%%%%%%%%%%%%%%%
\paragraph{Restrictions.}

Please note the following restrictions:
\begin{itemize}
\item
|\childdocmain| must be called with one argument \textit{main}
to ensure compatibility with earlier version of the package.
It must either be empty (|\childdocmain{}|)
or precisely match the filename of the main file in which it is specified.
See \secref{sec:detection} for further information.
\item
The filename \textit{main} must be specified without the |.tex| extension.
\item
The filename \textit{main} is case sensitive
(even in case-insensitive file systems)
due to internal string comparison.
\item
The argument \textit{main} should be fully expanded, it cannot be a macro.
\item
Subdirectories and special characters should be avoided in filenames.
\item
The command |\childdocmain{|\textit{main}|}| must be followed by a whitespace.
It should not be followed immediately by another command
or by a comment mark `|%|'.
This is because the \TeX{} parser reads the token immediately following
the argument of |\childdocmain| and puts it
at the beginning of every child section;
however, a white\-space is ignored.
\end{itemize}

%%%%%%%%%%%%%%%%%%%%%%%%%%%%%%%%%%%%%%%%
\paragraph{Content of Main File.}

It is advisable to place all content in the child files included by |\include|.
Any output contained in the main file will appear in all child documents
unless suppressed manually;
it cannot be suppressed automatically by the |\includeonly| directive
and thus should normally be avoided.
A method to include some content in the main file
by means of conditional processing is described in \secref{sec:conditional}.

%%%%%%%%%%%%%%%%%%%%%%%%%%%%%%%%%%%%%%%%
\paragraph{Page Numbering.}

When only a part of the document is compiled,
the appropriate numbering of pages
(as well as other status parameters)
is determined from the |.aux| files.
The latter contain information from previous passes.
However this information needs to propagate through
all intermediate child documents.
Therefore the page numbering in child documents may well
be inconsistent until the complete document is compiled at least once.

A useful (if unconventional) way to always ensure a consistent
page numbering is to restart the numbering in each child document
and denote the pages by `\textit{child}|.|\textit{page}'
where \textit{child} represents the chapter/section number of the child file.
This can be achieved by the command
|\numberwithin{page}{|\textit{child}|}|
of the \textsf{amsmath} package
where \textit{child} can be |chapter| or |section|
depending on the chosen structuring.
Alternatively, one can modify the macro |\thepage| appropriately
and reset the counter |page| at the start of each child file.

%%%%%%%%%%%%%%%%%%%%%%%%%%%%%%%%%%%%%%%%%%%%%%%%%%%%%%%%%%%%%%%%%%%%%%%%%%%%%%%%
\subsection{Conditional Processing}
\label{sec:conditional}

The package provides a mechanism to compile different versions
of a document. To customise the versions further some conditional processing
can come in handy to distinguish which version is being compiled.
The package provides two macros to describe the compilation context:

%%%%%%%%%%%%%%%%%%%%%%%%%%%%%%%%%%%%%%%%
\DescribeMacro{\ifchilddoc}
The conditional |\ifchilddoc| distinguishes between the compilation of
child documents and the main document:
%
\begin{center}
|\ifchilddoc |\textit{child-code}| |[|\||else |\textit{main-code}]| \||fi|
\end{center}

%%%%%%%%%%%%%%%%%%%%%%%%%%%%%%%%%%%%%%%%
\DescribeMacro{\childdocname}
\DescribeMacro{\childdocjob}
The macro |\childdocname| contains the filename (without extension)
of the main or child file being processed.
Note that |\childdocjob| will always contain the name of the main file.

%%%%%%%%%%%%%%%%%%%%%%%%%%%%%%%%%%%%%%%%
\paragraph{Title Page.}

Conditional processing can be used to include a title or banner page
in the main document when proper precautions are taken.
Importantly, the code in the main file should ensure that the page counter
(as well as other status parameters which are stored in the |.aux| files)
takes the same value after the conditional processing.
Otherwise the page numbers may take divergent values
depending on which part is compiled.

For example, a title page could be declared by:
%
\begin{center}
\begin{tabular}{l}
|\ifchilddoc\||else|\\
|\addtocounter{page}{-1}|\\
\textit{code for title page}\\
|\newpage|\\
|\||fi|
\end{tabular}
\end{center}
%
A banner page for the child documents can be generated by:
%
\begin{center}
\begin{tabular}{l}
|\ifchilddoc|\\
|\addtocounter{page}{-1}|\\
\textit{code for banner page}\\
|\newpage|\\
|\||fi|
\end{tabular}
\end{center}
%
Here one could write a message such as:
\begin{center}
|This is the part \childdocname{} of \childdocjob{}.|
\end{center}

%%%%%%%%%%%%%%%%%%%%%%%%%%%%%%%%%%%%%%%%%%%%%%%%%%%%%%%%%%%%%%%%%%%%%%%%%%%%%%%%
\subsection{Flags}
\label{sec:flags}

The package makes it easy to generate different versions
of the main or child documents.
To this end compilation flags can be defined
and assigned different default values.
They will be particularly useful in conjunction
with the forwarding mechanism described in \secref{sec:forward}.

For example, it may be useful to have a flag |\version|
which can be set to |draft| or |final|.
The document source will contain some conditional code
depending on the value of |\version|.
Suppose further, the flag should default to |final| for the main file
and to |draft| for child files
which is a natural assignment for editing the document.
This is achieved by placing the following code
in the preamble of the main document
(below the |\childdocmain| directive):
%
\begin{center}
\begin{tabular}{l}
|\ifchilddoc|\\
|\providecommand{\version}{draft}|\\
|\||else|\\
|\providecommand{\version}{final}|\\
|\||fi|
\end{tabular}
\end{center}
%
The definition by |\providecommand| makes sure
that previous definitions are not overwritten.
Further statements |\providecommand{\version}{...}|
can thus be added before the above code to override it.

For the main file, one might add a line
(between |\childdocmain| and the above block)
%
\begin{center}
|%\ifchilddoc\||else\providecommand{\version}{draft}\||fi|
\end{center}
%
which can be uncommented to produce a draft version.
Likewise one can add a line to the very top of a child file
(above the |\childdocof{|\textit{main}|}| directive)
%
\begin{center}
|%\providecommand{\version}{final}|
\end{center}
%
which can be uncommented to produce the final version of this child document.

%%%%%%%%%%%%%%%%%%%%%%%%%%%%%%%%%%%%%%%%%%%%%%%%%%%%%%%%%%%%%%%%%%%%%%%%%%%%%%%%
\subsection{Forwarding}
\label{sec:forward}

Different versions of the main or child documents
using compilation flags as described in \secref{sec:flags}
can be (permanently) stored in different files
for convenient compilation, viewing and distribution.
To this end, the package defines a command
to pass on compilation to a different file:

%%%%%%%%%%%%%%%%%%%%%%%%%%%%%%%%%%%%%%%%
\DescribeMacro{\childdocforward}
The command |\childdocforward| redirects processing to
another source file:
%
\begin{center}
\begin{tabular}{l}
|\input{childdoc.def}|\\
|\childdocforward[|\textit{main}|]{|\textit{dest}|}|\\
\end{tabular}
\end{center}
%
The argument \textit{dest} is the destination file
(without extension).
It should be the main file or one of the child files.
Note that further \textsf{childdoc} directives
such as |\childdocof| and |\childdocforward|
in the indicated file will be processed in this form.
The optional argument \textit{main}
passes on directly to the main file \textit{main}
while pretending to compile the child \textit{dest}.
This form behaves as if \textit{dest}
issues |\childdocof{|\textit{main}|}| right away,
and no further \textsf{childdoc} directives will be processed.

%%%%%%%%%%%%%%%%%%%%%%%%%%%%%%%%%%%%%%%%
\DescribeMacro{\...prefix}
In the alternative form |\childdocforwardprefix|,
%
\begin{center}
\begin{tabular}{l}
|\input{childdoc.def}|\\
|\childdocforwardprefix[|\textit{main}|]{|\textit{prefix}|}{|\textit{dest}|}|
\end{tabular}
\end{center}
%
the destination file is determined by a pattern
depending on the current file:
To make this work, the current file must be called
`{\textit{prefix}\hspace{0.2em}\textit{suffix}}'
with \textit{prefix} matching precisely the argument.
Processing is then passed on to the file
`{\textit{dest}\hspace{0.2em}\textit{suffix}}'.
Surely, the same effect is achieved by
directly specifying the
argument `{\textit{dest}\hspace{0.2em}\textit{suffix}}'
in the first form.
However, that requires to set up a different file
for each child. With the alternative form of the command
all these files can have exactly the same content
which simplifies setting them up and maintaining them.

For example, the following file |draft.tex|
with a compilation flag |\version| as described in \secref{sec:flags}
compiles the main document as a draft:
%
\begin{center}
\begin{tabular}{l}
|\def\version{draft}|\\
|\input{childdoc.def}|\\
|\childdocforward{|\textit{main}|}|
\end{tabular}
\end{center}
%
Likewise, the following files |final|\textit{nn}|.tex|
compile the final version of the child document
|child|\textit{nn}|.tex|:
%
\begin{center}
\begin{tabular}{l}
|\def\version{final}|\\
|\input{childdoc.def}|\\
|\childdocforwardprefix{final}{child}|
\end{tabular}
\end{center}
%

Note that when several versions of a main file and/or of each child file
are to be generated, it may be convenient to set up a |Makefile| or
shell script to automatise the process.

%%%%%%%%%%%%%%%%%%%%%%%%%%%%%%%%%%%%%%%%%%%%%%%%%%%%%%%%%%%%%%%%%%%%%%%%%%%%%%%%
\subsection{Command Line Processing}
\label{sec:commandline}

The effect of redirection files can also be achieved by invoking
the \LaTeX{} compiler with a more elaborate command line.
Most conveniently this should be done as part
of a shell script or a |Makefile|.

When using \textsf{childdoc} in the main file, the following
command lines effectively perform a redirection
(note that depending on the shell being used,
backslashes may have to be doubled: `|\|' $\to$ `|\\|'):
%
\begin{center}
|... -jobname "|\textit{target}|" |\\|"|[\textit{flags}]%
|\input{childdoc.def}\childdocforward[|\textit{main}|]{|\textit{dest}|}"|
\end{center}
%
Here \textit{target} is the name of the output file,
\textit{main} is the name of the main file
and \textit{dest} is the name of the main or child file to be processed
(all filenames without extensions).
The optional argument \textit{main} can be omitted
if \textit{main} matches \textit{dest}.
Optionally, compilation \textit{flags} can be defined via |\def| commands.
This command line makes the \TeX{} engine believe
it is compiling the file \textit{target}
whose content is specified as the latter parameter.
The provided code then forwards the processing to
\textit{main} or \textit{dest} as described in \secref{sec:forward}.

%%%%%%%%%%%%%%%%%%%%%%%%%%%%%%%%%%%%%%%%%%%%%%%%%%%%%%%%%%%%%%%%%%%%%%%%%%%%%%%%
\subsection{Include by Input}
\label{sec:input}

Including child documents by |\include| has some restrictions by design.
Most notably, the content of a child document always occupies
its own set of pages; pages cannot be shared between child documents.
Usually, this behaviour makes perfect sense
because each child document contain an essential part of the document.
However, in some situations it may be desirable to compose
a document from a collection of parts
without having mandatory page breaks between then.
For this case, the package
provides a mechanism to include parts
by |\input| which can also be processed individually.
However, by construction this mechanism
requires manual handling of the content to be output.

%%%%%%%%%%%%%%%%%%%%%%%%%%%%%%%%%%%%%%%%
\DescribeMacro{\ifchilddocmanual}
The main file should be prepared as usual, see \secref{sec:include}.
However, the document body must make a distinction
between processing of an individual part and of the main document, e.g.:
%
\begin{center}
\begin{tabular}{l}
|\ifchilddocmanual|\\
|\input{\childdocname}|\\
|\||else|\\
\textit{document body with }|\input{|\textit{part}|}|\\
|\||fi|
\end{tabular}
\end{center}
%
The conditional |\ifchilddocmanual| is true whenever
a part to be included by |\input| is being compiled,
and the name of the part is stored in |\childdocname|.

%%%%%%%%%%%%%%%%%%%%%%%%%%%%%%%%%%%%%%%%
\DescribeMacro{\childdocby}
Each part to be included by |\input| should start with:
%
\begin{center}
\begin{tabular}{l}
|\input{childdoc.def}|\\
|\childdocby{|\textit{main}|}|\\
\end{tabular}
\end{center}
%
The directive |\childdocby| is similar to |\childdocof|
described in \secref{sec:include},
but the subsequent selection of content must be done manually.
To that end, both |\ifchilddoc| and |\ifchilddocmanual|
will be true upon processing of a part,
and the name of the part is stored in |\childdocname|.
Note that |\jobname| will be set to the filename of the current part
so that each part receives an individual |.aux| file
that does not interfere with the |.aux| file(s) of the main document.
This behaviour can be altered by the alternative form
|\childdocby[*]{|\textit{main}|}| (with a non-empty optional argument)
which uses the |.aux| file of the main document
by setting |\jobname| to \textit{main}.

%%%%%%%%%%%%%%%%%%%%%%%%%%%%%%%%%%%%%%%%%%%%%%%%%%%%%%%%%%%%%%%%%%%%%%%%%%%%%%%%
\subsection{Driver Development}
\label{sec:driver}

The \textsf{childdoc} mechanism can also be use for the development
of definition files such as \LaTeX{} styles or classes.
This case differs from the above setup with multiple parts
included by |\include| in that no |\includeonly| should be invoked.
This can be achieved by starting the include file
(before |\ProvidesPackage|) with:
%
\begin{center}
\begin{tabular}{l}
|\input{childdoc.def}|\\
|\childdocforward{|\textit{main}|}|\\
\end{tabular}
\end{center}
%
or alternatively with:
%
\begin{center}
\begin{tabular}{l}
|\input{childdoc.def}|\\
|\childdocby{|\textit{main}|}|\\
\end{tabular}
\end{center}
%
Both forms have slightly different effects as described above.
The main file is prepared as usual, see \secref{sec:include}.

%%%%%%%%%%%%%%%%%%%%%%%%%%%%%%%%%%%%%%%%%%%%%%%%%%%%%%%%%%%%%%%%%%%%%%%%%%%%%%%%
\subsection{Legacy Detection}
\label{sec:detection}

The directive |\childdocmain| in the main file can detect
whether the complete document or merely a child is to be compiled
even without using the directive |\childdocof|.
This method is deprecated because it is less robust
and there is no compelling reason to use it;
it is merely provided for backward compatibility
and it may be removed in future versions.

If the detection mechanism is to be used,
it is mandatory to correctly specify
the filename of the main file as the argument of |\childdocmain|:
%
\begin{center}
\begin{tabular}{l}
|\input{childdoc.def}|\\
|\childdocmain{|\textit{main}|}|\\
\end{tabular}
\end{center}
%
If |\jobname| does not match the argument \textit{main} of |\childdocmain|,
it is assumed that |\jobname| points to the child file to be compiled.
When using |\childdocmain| with the main file specified as argument,
it suffices to start a child file
with just |\input{|\textit{main}|}|
without loading of the package and using |\childdocof|.
If instead all processing is done
with the appropriate \textsf{childdoc} directives,
the argument of \textit{main} of |\childdocmain| can be empty.

An alternative version of the command line processing described
in \secref{sec:commandline} using the detection mechanism reads:
%
\begin{center}
|... -jobname "|\textit{target}|" "|[\textit{flags}]%
[|\def\jobname{|\textit{dest}|}|]|\input{|\textit{main}|}"|
\end{center}

%%%%%%%%%%%%%%%%%%%%%%%%%%%%%%%%%%%%%%%%%%%%%%%%%%%%%%%%%%%%%%%%%%%%%%%%%%%%%%%%
\subsection{Manual Code}
\label{sec:manual}

In case one cannot be certain whether the definitions file |childdoc.def|
is installed on the target \TeX{} distribution
and one prefers not to ship it,
it is conceivable to paste a few relevant commands into the sources.

To that end, drop all statements |\input{childdoc.def}|
and perform the replacements as outlined below.
Instead of |\childdocmain{|\textit{main}|}| add the following code
to the top of the main file:
%
\begin{center}
\begin{tabular}{l}
|\||ifdefined\childdocname\endinput\||fi\newif\ifchilddoc|\\
|\edef\childdocname{\scantokens\expandafter{\jobname\noexpand}}|\\
|\def\childdocmain{|\textit{main}|}\||ifx\childdocmain\childdocname\||else|\\
|\childdoctrue\includeonly{\childdocname}\let\jobname\childdocmain\||fi|\\
\end{tabular}
\end{center}
%
Instead of |\childdocof{|\textit{main}|}| just include the main file
at the top of each child file:
%
\begin{center}
|\input{|\textit{main}|}|
\end{center}
%
A simple redirection |\childdocforward{|\textit{dest}|}| is achieved by:
%
\begin{center}
|\def\jobname{|\textit{dest}|}\input{\jobname}|
\end{center}
%
The redirection with prefix
|\childdocforwardprefix[|\textit{prefix}|]{|\textit{dest}|}|
is accomplished by:
%
\begin{center}
\begin{tabular}{l}
|{\edef\jobname{\scantokens\expandafter{\jobname\noexpand}}|\\
|\def\redirectjob |\textit{prefix}|#1~~~{\gdef\jobname{|\textit{dest}|#1}}|\\
|\expandafter\redirectjob\jobname~~~}\input{\jobname}|
\end{tabular}
\end{center}

In an alternative approach,
child documents can be compiled by a specific command line
without additional code or specific definitions:
%
\begin{center}
|... -jobname "|\textit{target}|" "|[\textit{flags}]%
|\includeonly{|\textit{dest}|}\input{|\textit{main}|}"|
\end{center}
%

%%%%%%%%%%%%%%%%%%%%%%%%%%%%%%%%%%%%%%%%%%%%%%%%%%%%%%%%%%%%%%%%%%%%%%%%%%%%%%%%
%%%%%%%%%%%%%%%%%%%%%%%%%%%%%%%%%%%%%%%%%%%%%%%%%%%%%%%%%%%%%%%%%%%%%%%%%%%%%%%%
\section{Information}

%%%%%%%%%%%%%%%%%%%%%%%%%%%%%%%%%%%%%%%%%%%%%%%%%%%%%%%%%%%%%%%%%%%%%%%%%%%%%%%%
\subsection{Copyright}

Copyright \copyright{} 2017--2018 Niklas Beisert

This work may be distributed and/or modified under the
conditions of the \LaTeX{} Project Public License, either version 1.3
of this license or (at your option) any later version.
The latest version of this license is in
  \url{http://www.latex-project.org/lppl.txt}
and version 1.3 or later is part of all distributions of \LaTeX{}
version 2005/12/01 or later.

This work has the LPPL maintenance status `maintained'.

The Current Maintainer of this work is Niklas Beisert.

This work consists of the files |README.txt|, |childdoc.ins| and |childdoc.dtx|
as well as the derived files |childdoc.def|, |cdocsamp.tex|
with |cdocsch1.tex|, |cdocsch2.tex|, |cdocspt3.tex|, |cdocspt4.tex|,
|cdocsdrf.tex|, |cdocsfn1.tex|, |cdocsfn2.tex|
as well as |childdoc.pdf|.

%%%%%%%%%%%%%%%%%%%%%%%%%%%%%%%%%%%%%%%%%%%%%%%%%%%%%%%%%%%%%%%%%%%%%%%%%%%%%%%%
\subsection{Files and Installation}

The package consists of the files:
%
\begin{center}
\begin{tabular}{ll}
    |README.txt|   & readme file \\
    |childdoc.ins| & installation file \\
    |childdoc.dtx| & source file \\
    |childdoc.def| & definition file \\
    |cdocsamp.tex| & sample main file \\
    |cdocsch1.tex| & sample include file \\
    |cdocsch2.tex| & sample include file \\
    |cdocspt3.tex| & sample part file \\
    |cdocspt4.tex| & sample part file \\
    |cdocsdrf.tex| & sample redirection file \\
    |cdocsfn1.tex| & sample redirection file \\
    |cdocsfn2.tex| & sample redirection file \\
    |childdoc.pdf| & manual
\end{tabular}
\end{center}
%
The distribution consists of the files
|README.txt|, |childdoc.ins| and |childdoc.dtx|.
%
\begin{itemize}
\item
Run (pdf)\LaTeX{} on |childdoc.dtx|
to compile the manual |childdoc.pdf| (this file).
\item
Run \LaTeX{} on |childdoc.ins| to create the definitions file |childdoc.def|
and the sample |cdocsamp.tex| with include files
|cdocsch1.tex|, |cdocsch2.tex|, |cdocspt3.tex|, |cdocspt4.tex|,
|cdocsdrf.tex|, |cdocsfn1.tex|, |cdocsfn2.tex|.
Then copy the file |childdoc.def| to an appropriate directory of your \LaTeX{}
distribution, e.g.\ \textit{texmf-root}|/tex/latex/childdoc|.
\end{itemize}

%%%%%%%%%%%%%%%%%%%%%%%%%%%%%%%%%%%%%%%%%%%%%%%%%%%%%%%%%%%%%%%%%%%%%%%%%%%%%%%%
\subsection{Related CTAN Packages}

There are several other packages which offer a similar functionality:
%
\begin{itemize}
\item
The packages
\href{http://ctan.org/pkg/docmute}{\textsf{docmute}},
\href{http://ctan.org/pkg/includex}{\textsf{includex}} and
\href{http://ctan.org/pkg/standalone}{\textsf{standalone}}
provide commands to include only the document body of
a child file thus allowing both files to be compiled individually.
\item
The packages \href{http://ctan.org/pkg/subdocs}{\textsf{subdocs}}
and \href{http://ctan.org/pkg/subfiles}{\textsf{subfiles}}
provide structures in which the main and child documents can be
encapsulated and allowing them to be compiled individually.
The inclusion mechanism is different from the conventional |\include|.
\item
The package \href{http://ctan.org/pkg/combine}{\textsf{combine}}
is an elaborate solution to combine several documents into one.
\end{itemize}
%
See also the CTAN topic \href{http://ctan.org/topic/subdocs}{\textsf{subdocs}}
for further related packages.
The present package differs from the above solutions in that
a document structure constructed with the conventional |\include| mechanism
just needs two extra commands at the top of every file
such that all constituent files can be compiled individually.

%%%%%%%%%%%%%%%%%%%%%%%%%%%%%%%%%%%%%%%%%%%%%%%%%%%%%%%%%%%%%%%%%%%%%%%%%%%%%%%%
%\subsection{Feature Suggestions}
%
%The following is a list of features which may be useful for future
%versions of this package:
%%
%\begin{itemize}
%\item
%\ldots
%\end{itemize}

%%%%%%%%%%%%%%%%%%%%%%%%%%%%%%%%%%%%%%%%%%%%%%%%%%%%%%%%%%%%%%%%%%%%%%%%%%%%%%%%
\subsection{Revision History}

%%%%%%%%%%%%%%%%%%%%%%%%%%%%%%%%%%%%%%%%
\paragraph{v2.0:} 2018/12/30

\begin{itemize}
\item
immediate forward processing
\item
added |\childdocby| mechanism
\item
manual restructured
\end{itemize}

%%%%%%%%%%%%%%%%%%%%%%%%%%%%%%%%%%%%%%%%
\paragraph{v1.6:} 2018/01/17

\begin{itemize}
\item
application for development of include files
\item
corrections to manual
\end{itemize}

%%%%%%%%%%%%%%%%%%%%%%%%%%%%%%%%%%%%%%%%
\paragraph{v1.5:} 2017/05/21

\begin{itemize}
\item
more complete structuring introduced
\item
|\childdocof| introduced
\item
|\childdoc| renamed to |\childdocmain|
\item
|\childredirect| renamed to |\childdocforward| and |\childdocforwardprefix|
and functionality expanded
\end{itemize}

%%%%%%%%%%%%%%%%%%%%%%%%%%%%%%%%%%%%%%%%
\paragraph{v1.0:} 2017/04/27

\begin{itemize}
\item
manual and install package
\item
first version published on CTAN
\end{itemize}

%%%%%%%%%%%%%%%%%%%%%%%%%%%%%%%%%%%%%%%%
\paragraph{v0.6:} 2017/04/26

\begin{itemize}
\item
redirection mechanism added
\end{itemize}

%%%%%%%%%%%%%%%%%%%%%%%%%%%%%%%%%%%%%%%%
\paragraph{v0.5:} 2017/04/26

\begin{itemize}
\item
functionality in definition file
\end{itemize}


%%%%%%%%%%%%%%%%%%%%%%%%%%%%%%%%%%%%%%%%%%%%%%%%%%%%%%%%%%%%%%%%%%%%%%%%%%%%%%%%
%%%%%%%%%%%%%%%%%%%%%%%%%%%%%%%%%%%%%%%%%%%%%%%%%%%%%%%%%%%%%%%%%%%%%%%%%%%%%%%%
%%%%%%%%%%%%%%%%%%%%%%%%%%%%%%%%%%%%%%%%%%%%%%%%%%%%%%%%%%%%%%%%%%%%%%%%%%%%%%%%
\appendix

\settowidth\MacroIndent{\rmfamily\scriptsize 000\ }

 \DocInput{childdoc.dtx}

\end{document}
%</driver>
% \fi
%
% %%%%%%%%%%%%%%%%%%%%%%%%%%%%%%%%%%%%%%%%%%%%%%%%%%%%%%%%%%%%%%%%%%%%%%%%%%%%%%
% %%%%%%%%%%%%%%%%%%%%%%%%%%%%%%%%%%%%%%%%%%%%%%%%%%%%%%%%%%%%%%%%%%%%%%%%%%%%%%
% \section{Sample}
%\iffalse
%<*samplemain>
%\fi
%
% The following presents a sample document
% with two chapters, two parts, a title page,
% a compile flag as well as three forwarding files to set the flag.
% It consists of eight |.tex| files:
% \begin{center}
% \begin{tabular}{ll}
% |cdocsamp.tex|&main file\\
% |cdocsch1.tex|&include file for chapter 1\\
% |cdocsch2.tex|&include file for chapter 2\\
% |cdocspt3.tex|&include file for part 3\\
% |cdocspt4.tex|&include file for part 4\\
% |cdocsdrf.tex|&forwarding file for main file in draft mode\\
% |cdocsfi1.tex|&forwarding file for final version of chapter 1\\
% |cdocsfi2.tex|&forwarding file for final version of chapter 2\\
% \end{tabular}
% \end{center}
% Each of the eight files can be compiled directly by the \LaTeX{} compiler.
%
% %%%%%%%%%%%%%%%%%%%%%%%%%%%%%%%%%%%%%%
% \paragraph{Main File.}
%
% The main file is called |cdocsamp.tex|.
%
% Load the \textsf{childdoc} definitions and
% declare the filename for the main document:
%    \begin{macrocode}
\input{childdoc.def}
\childdocmain{}
%    \end{macrocode}

% Optional override for |\version| flag:
%    \begin{macrocode}
%%\ifchilddoc\else\providecommand{\version}{draft}\fi
%    \end{macrocode}

% Define the default values for the |\version| flag
% (|final| for the main file and |draft| for childs):
%    \begin{macrocode}
\ifchilddoc
\providecommand{\version}{draft}
\else
\providecommand{\version}{final}
\fi
%    \end{macrocode}

% Load the standard document class:
%    \begin{macrocode}
\documentclass[12pt]{article}
%    \end{macrocode}

% Start the document body:
%    \begin{macrocode}
\begin{document}
%    \end{macrocode}

% Declare a title page.
% Print title, part of document being processed and version flag:
%    \begin{macrocode}
\addtocounter{page}{-1}
\begin{center}
{\LARGE\bfseries{}childdoc example\par}
\vspace{1cm}
\ifchilddoc
\ifchilddocmanual part\else chapter\fi:
`\childdocname' of `\childdocjob'\par
\else
main document: `\childdocjob'\par
\fi
version: \version\par
\end{center}
\newpage
%    \end{macrocode}

% Manually include selected file,
% otherwise process as usual:
%    \begin{macrocode}
\ifchilddocmanual
\section*{part `\childdocname'}
\input{\childdocname}
\else
%    \end{macrocode}

% Include the two chapters:
%    \begin{macrocode}
\include{cdocsch1}
\include{cdocsch2}
%    \end{macrocode}

% Include the two parts unless only chapters should be displayed:
%    \begin{macrocode}
\ifchilddoc\else
\section{part three}
\input{cdocspt3}
\section{part four}
\input{cdocspt4}
\fi
%    \end{macrocode}

% Process as usual until here:
%    \begin{macrocode}
\fi
%    \end{macrocode}

% End of document body:
%    \begin{macrocode}
\end{document}
%    \end{macrocode}
%\iffalse
%</samplemain>
%\fi
%
% %%%%%%%%%%%%%%%%%%%%%%%%%%%%%%%%%%%%%%
% \paragraph{Chapter Include Files.}
%
% The include files are called |cdocsch1.tex| and |cdocsch2.tex|.
%
%\iffalse
%<*samplechap1|samplechap2>
%\fi

% Optional override for |\version| flag:
%    \begin{macrocode}
%%\providecommand{\version}{final}
%    \end{macrocode}

% Include the main document:
%    \begin{macrocode}
\input{childdoc.def}
\childdocof{cdocsamp}
%    \end{macrocode}

%\iffalse
%</samplechap1|samplechap2>
%\fi
%
%\iffalse
%<*samplechap1>
%\fi
% Some text for chapter 1:
%    \begin{macrocode}
\section{one}
some text in chapter one
%    \end{macrocode}

%\iffalse
%</samplechap1>
%\fi
% Some text for chapter 2:
%\iffalse
%<*samplechap2>
%\fi
%    \begin{macrocode}
\section{two}
more text in chapter two
%    \end{macrocode}

%\iffalse
%</samplechap2>
%\fi
%
% %%%%%%%%%%%%%%%%%%%%%%%%%%%%%%%%%%%%%%
% \paragraph{Part Include Files.}
%
% The include files are called |cdocspt3.tex| and |cdocspt4.tex|.
%
%\iffalse
%<*samplepart3|samplepart4>
%\fi

% Optional override for |\version| flag:
%    \begin{macrocode}
%%\providecommand{\version}{final}
%    \end{macrocode}

% Include the main document:
%    \begin{macrocode}
\input{childdoc.def}
\childdocby{cdocsamp}
%    \end{macrocode}

%\iffalse
%</samplepart3|samplepart4>
%\fi
%
%\iffalse
%<*samplepart3>
%\fi
% Some text for part 3:
%    \begin{macrocode}
some text in part three
%    \end{macrocode}

%\iffalse
%</samplepart3>
%\fi
% Some text for part 4:
%\iffalse
%<*samplepart4>
%\fi
%    \begin{macrocode}
more text in part four
%    \end{macrocode}

%\iffalse
%</samplepart4>
%\fi
%
% %%%%%%%%%%%%%%%%%%%%%%%%%%%%%%%%%%%%%%
% \paragraph{Forwarding for a Complete Draft.}
%
% The following forwarding file |cdocsdrf.tex|
% compiles the main document in draft mode:
%\iffalse
%<*sampledraft>
%\fi
%    \begin{macrocode}
\def\version{draft}
\input{childdoc.def}
\childdocforward{cdocsamp}
%    \end{macrocode}

%\iffalse
%</sampledraft>
%\fi
%
% %%%%%%%%%%%%%%%%%%%%%%%%%%%%%%%%%%%%%%
% \paragraph{Forwarding for Final Version of the Chapters.}
%
% The following forwarding files |cdocsfn1.tex| and |cdocsfn2.tex|
% (with identical content)
% compile the final versions of the child documents
% |cdocsch1.tex| and |cdocsch2.tex|, respectively:
%\iffalse
%<*samplefinal>
%\fi
%    \begin{macrocode}
\def\version{final}
\input{childdoc.def}
\childdocforwardprefix[cdocsamp]{cdocsfn}{cdocsch}
%    \end{macrocode}

%\iffalse
%</samplefinal>
%\fi
%
% %%%%%%%%%%%%%%%%%%%%%%%%%%%%%%%%%%%%%%
% \paragraph{Command Line Processing.}
%
% The following three command lines generate the output files
% |cdocscld|, |cdocscl1| and |cdocscl2|
% which should be identical to
% |cdocsdrf|, |cdocsch1| and |cdocsfn2|, respectively:
% \begin{center}
% \begin{tabular}{l}
% |latex -jobname cdocscld \|\\
% |  "\def\version{draft}\input{childdoc.def}\childdocforward{cdocsamp}"|\\
% |latex -jobname cdocscl1 \|\\
% |  "\input{childdoc.def}\childdocforward[cdocsamp]{cdocsch1}"|\\
% |latex -jobname cdocscl2 \|\\
% |  "\def\version{final}\input{childdoc.def}\childdocforward{cdocsch2}"|
% \end{tabular}
% \end{center}
% Note that the trailing backslash on each first line
% merely continues the input to the second line
% (for convenient cut ant paste).
% Furthermore, the command |latex| can be replaced by any
% of its alternative versions such as |pdflatex|.
%
% %%%%%%%%%%%%%%%%%%%%%%%%%%%%%%%%%%%%%%%%%%%%%%%%%%%%%%%%%%%%%%%%%%%%%%%%%%%%%%
% %%%%%%%%%%%%%%%%%%%%%%%%%%%%%%%%%%%%%%%%%%%%%%%%%%%%%%%%%%%%%%%%%%%%%%%%%%%%%%
% \section{Implementation}
%\iffalse
%<*package>
%\fi
%
% This section describes the definitions file |childdoc.def|.

% The definitions cannot be loaded using |\usepackage| or |\RequirePackage|
% which has a mechanism to prevent loading a style file more than once.
% When loading the definitions by means of |\input|
% multiple instances have to be prevented manually:
%\iffalse
%This code needs to be before the `\ProvidesFile' directive
%which is defined at the beginning of this file.
%Therefore it is also placed there and commented out here.
%</package>
%<*discard>
%\fi
%    \begin{macrocode}
\ifdefined\childdocmain\endinput\fi
%    \end{macrocode}
%\iffalse
%</discard>
%<*package>
%\fi
%
% \macro{\ifchilddoc}
% \macro{\ifchilddocmanual}
% The conditional |\ifchilddoc| tells whether a
% child (true) or main (false) document is being compiled.
% The conditional |\ifchilddocmanual| tells whether
% the |\includeonly| mechanism is used (false) or
% the selection of child files must be performed manually (true).
% The definitions initialise to false:
%    \begin{macrocode}
\newif\ifchilddoc
\newif\ifchilddocmanual
%    \end{macrocode}

% \macro{\childdocname}
% \macro{\childdocjob}
% The macro |\childdocname| stores the name of the main document
% to be compiled. The macro |\childdocjob| stores the name of
% the document on which the \LaTeX{} compiler was originally invoked.
% The content of |\jobname| cannot be compared
% to filenames specified in the source due to different catcodes.
% The following code rescans |\jobname|, stores the result
% in |\childdocname| and saves a copy in |\childdocjob|:
%    \begin{macrocode}
\edef\childdocname{\scantokens\expandafter{\jobname\noexpand}}
\let\childdocjob\childdocname
%    \end{macrocode}

% \macro{\childdocdisable}
% The macro |\childdocdisable| prevents the main file
% from being processed more than once.
% At this stage, the main document command |\childdocmain|
% is assumed to be called once again where it should do nothing.
% Any subsequent call to it should prevent
% a secondary processing of the main document
% It overwrites the forwarding commands
% |\childdocof| and |\childdocforward|
% with empty macros to prevent further inclusions of the main document:
%    \begin{macrocode}
\newcommand{\childdocdisable}
{
  \renewcommand{\childdocmain}[1]{\renewcommand{\childdocmain}[1]{\endinput}}
  \renewcommand{\childdocof}[1]{}
  \renewcommand{\childdocby}[2][]{}
  \renewcommand{\childdocforward}[2][]{}
  \renewcommand{\childdocdisable}{}
}
%    \end{macrocode}

% \macro{\childdocmain}
% The macro |\childdocmain| is to be called at the top of the main file
% with nothing or the main filename (without extension) as argument.
% First, it breaks loops.
% If the argument is not empty and does not match |\childdocname|
% (which is set by the first inclusion of |childdoc.def|),
% |\ifchilddoc| is set to true, |\includeonly| is applied to the child file
% and |\jobname| is set to the main file
% (for proper handling of |.aux| files):
%    \begin{macrocode}
\newcommand{\childdocmain}[1]
{
  \childdocdisable\childdocmain{}
  \if?#1?\else
    \begingroup
      \def\childdoctmp{#1}
      \ifx\childdoctmp\childdocname
        \def\childdoctmp{}
      \else
        \def\childdoctmp
        {
          \childdoctrue
          \includeonly{\childdocname}
          \def\childdocjob{#1}
          \def\jobname{#1}
        }
      \fi
      \expandafter
    \endgroup
    \childdoctmp
  \fi
}
%    \end{macrocode}

% \macro{\childdocof}
% The command |\childdocof| redirects
% compilation to the main file |#1|.
%    \begin{macrocode}
\newcommand{\childdocof}[1]
{
  \childdocdisable
  \childdoctrue
  \includeonly{\childdocname}
  \def\jobname{#1}
  \def\childdocjob{#1}
  \input{#1}
}
%    \end{macrocode}

% \macro{\childdocby}
% The command |\childdocby| ....
%    \begin{macrocode}
\newcommand{\childdocby}[2][]
{
  \childdocdisable
  \childdoctrue
  \childdocmanualtrue
  \if?#1?\else
    \def\jobname{#2}
  \fi
  \def\childdocjob{#2}
  \input{#2}
  \endinput
}
%    \end{macrocode}

% \macro{\childdocforward}
% The command |\childdocforward| redirects
% compilation to the main file or
% (if the optional argument is given) a child file.
% Parameters are set as if the main file
% or a child file starting with |\childdocof| was compiled.
% Then compilation is handed over to the main file:
%    \begin{macrocode}
\newcommand{\childdocforward}[2][]
{
  \begingroup
    \if?#1?
      \def\childdoctmp
      {
        \def\childdocname{#2}
        \def\childdocjob{#2}
        \def\jobname{#2}
        \input{#2}
        \endinput
      }
    \else
      \def\childdoctmp
      {
        \childdocdisable
        \def\childdocname{#2}
        \childdoctrue
        \includeonly{#2}
        \def\childdocjob{#1}
        \def\jobname{#1}
        \input{#1}
        \endinput
      }
    \fi
    \expandafter
  \endgroup
  \childdoctmp
}
%    \end{macrocode}

% \macro{\childdocforwardprefix}
% The command |\childdocforwardprefix| redirects
% compilation to the main or a child file by means of a pattern.
% The prefix |#1| in the current filename is replaced by |#2|
% and the suffix of the current filename is kept
% (it is assumed that the filename does not contain the substring `|~~~|'
% which is used as a delimiter).
% Compilation is handed over to the new file by |\childdocforward|:
%    \begin{macrocode}
\newcommand{\childdocforwardprefix}[3][]
{
  \begingroup
    \def\childdocextract #2##1~~~{\def\childdoctmp{\childdocforward[#1]{#3##1}}}
    \expandafter\childdocextract\childdocname~~~
    \expandafter
  \endgroup
  \childdoctmp
}
%    \end{macrocode}

% \macro{\childdoc}
% The deprecated macro |\childdoc| is a legacy version of |\childdocmain|:
%    \begin{macrocode}
\newcommand{\childdoc}{\childdocmain}
%    \end{macrocode}

% \macro{\childdocredirect}
% The deprecated macro |\childdocredirect| is a legacy version
% of |\childdocforward| and |\childdocforwardprefix|:
%    \begin{macrocode}
\newcommand{\childdocredirect}[2][]
{
  \begingroup
    \if?#1?
      \def\childdoctmp{\childdocforward{#2}}
    \else
      \def\childdoctmp{\childdocforwardprefix{#1}{#2}}
    \fi
    \expandafter
  \endgroup
  \childdoctmp
}
%    \end{macrocode}

%\iffalse
%</package>
%\fi
%
\endinput
|\\
|\childdocby{|\textit{main}|}|\\
\end{tabular}
\end{center}
%
Both forms have slightly different effects as described above.
The main file is prepared as usual, see \secref{sec:include}.

%%%%%%%%%%%%%%%%%%%%%%%%%%%%%%%%%%%%%%%%%%%%%%%%%%%%%%%%%%%%%%%%%%%%%%%%%%%%%%%%
\subsection{Legacy Detection}
\label{sec:detection}

The directive |\childdocmain| in the main file can detect
whether the complete document or merely a child is to be compiled
even without using the directive |\childdocof|.
This method is deprecated because it is less robust
and there is no compelling reason to use it;
it is merely provided for backward compatibility
and it may be removed in future versions.

If the detection mechanism is to be used,
it is mandatory to correctly specify
the filename of the main file as the argument of |\childdocmain|:
%
\begin{center}
\begin{tabular}{l}
|% \iffalse
%
% childdoc.dtx Copyright (C) 2017-2018 Niklas Beisert
%
% This work may be distributed and/or modified under the
% conditions of the LaTeX Project Public License, either version 1.3
% of this license or (at your option) any later version.
% The latest version of this license is in
%   http://www.latex-project.org/lppl.txt
% and version 1.3 or later is part of all distributions of LaTeX
% version 2005/12/01 or later.
%
% This work has the LPPL maintenance status `maintained'.
%
% The Current Maintainer of this work is Niklas Beisert.
%
% This work consists of the files childdoc.dtx and childdoc.ins
% and the derived files childdoc.def and cdocsamp.tex with
% cdocsch1.tex, cdocsch2.tex, cdocsdrf.tex, cdocsfn1.tex, cdocsfn2.tex.
%
%<package>\ifdefined\childdocmain\endinput\fi
%<package>\ProvidesFile{childdoc.def}[2018/12/30 v2.0 child document driver]
%<samplemain>\ProvidesFile{cdocsamp.tex}[2018/12/30 v2.0 sample for childdoc]
%<*driver>
%\ProvidesFile{childdoc.drv}[2018/12/30 v2.0 childdoc reference manual file]
\PassOptionsToClass{10pt,a4paper}{article}
\documentclass{ltxdoc}

\usepackage[margin=35mm]{geometry}
\usepackage{hyperref}
\usepackage{hyperxmp}
\usepackage[usenames]{color}

\hypersetup{colorlinks=true}
\hypersetup{pdfstartview=FitH}
\hypersetup{pdfpagemode=UseNone}
\hypersetup{pdfsource={}}
\hypersetup{pdflang={en-UK}}
\hypersetup{pdfcopyright={Copyright 2017-2018 Niklas Beisert.
  This work may be distributed and/or modified under the
  conditions of the LaTeX Project Public License, either version 1.3
  of this license or (at your option) any later version.}}
\hypersetup{pdflicenseurl={http://www.latex-project.org/lppl.txt}}
\hypersetup{pdfcontactaddress={ETH Zurich, ITP, HIT K,
  Wolfgang-Pauli-Strasse 27}}
\hypersetup{pdfcontactpostcode={8093}}
\hypersetup{pdfcontactcity={Zurich}}
\hypersetup{pdfcontactcountry={Switzerland}}
\hypersetup{pdfcontactemail={nbeisert@itp.phys.ethz.ch}}
\hypersetup{pdfcontacturl={http://people.phys.ethz.ch/\xmptilde nbeisert/}}

\newcommand{\secref}[1]{\hyperref[#1]{section \ref*{#1}}}

\parskip1ex
\parindent0pt
\let\olditemize\itemize
\def\itemize{\olditemize\parskip0pt}

\begin{document}

\title{The \textsf{childdoc} Package}
\hypersetup{pdftitle={The childdoc Package}}
\author{Niklas Beisert\\[2ex]
  Institut f\"ur Theoretische Physik\\
  Eidgen\"ossische Technische Hochschule Z\"urich\\
  Wolfgang-Pauli-Strasse 27, 8093 Z\"urich, Switzerland\\[1ex]
  \href{mailto:nbeisert@itp.phys.ethz.ch}
  {\texttt{nbeisert@itp.phys.ethz.ch}}}
\hypersetup{pdfauthor={Niklas Beisert}}
\hypersetup{pdfsubject={Manual for the LaTeX2e Package childdoc}}
\date{30 December 2018, \textsf{v2.0}}
\maketitle

\begin{abstract}\noindent
\textsf{childdoc} is a \LaTeXe{} package
that enables the direct compilation
of document sections included by |\include|
to individual files.
\end{abstract}

\begingroup
\parskip0ex
\tableofcontents
\endgroup

%%%%%%%%%%%%%%%%%%%%%%%%%%%%%%%%%%%%%%%%%%%%%%%%%%%%%%%%%%%%%%%%%%%%%%%%%%%%%%%%
%%%%%%%%%%%%%%%%%%%%%%%%%%%%%%%%%%%%%%%%%%%%%%%%%%%%%%%%%%%%%%%%%%%%%%%%%%%%%%%%
\section{Introduction}

\LaTeX{} provides a mechanism to structure a large document (such as a book)
into a main file and several child files (containing the chapters)
using the |\include| command.
This mechanism is beneficial for documents
which span hundreds of pages in order to
make the source file(s) more manageable.
Moreover, compilation can be restricted to
selected child files by means of the |\includeonly| command.
The latter feature can be used to reduce the compilation time while editing
(this was significantly more useful in the earlier days of \LaTeX{})
or to generate a smaller document which is easier to navigate.
Another application of |\includeonly| is to generate
documents consisting of selected parts of the complete document.

However, there are a few drawbacks of the plain |\include| mechanism:
\begin{itemize}
\item
The child files cannot be compiled on their own,
they can only be compiled via the main file.
A naive editing environment
(such as a text editor with an option
to have the current file processed by \LaTeX)
may require one to switch to the main file before compiling;
attempting to compile the child file produces errors.
\item
The main file must be modified (each time)
to adjust the |\includeonly| command
to the present needs. This easily leaves the main file in a messy state.
\item
The generated document will always carry the filename
of the main document. This is inconvenient if
several child files are to be compiled and
to be kept for distribution.
\end{itemize}

The present package provides a simple interface
to make child files individually compilable by \LaTeX{}.
Compiling a child file then has the same effect as compiling
the main file with an |\includeonly| command
to select the appropriate child.
Moreover the generated document will carry the name of the child
rather than the main file.
This resolves all three above issues.

This feature is meant to make the editing of books,
thesis documents and lecture notes somewhat more convenient.
However, the package can also be used efficiently for
composing a series of documents (such as exercise sheets)
which are typically distributed individually.
It then assists the author in generating the individual documents
(potentially in different versions)
as well as a document containing the collected series.
Another application is in developing style files
or other kinds of included material
where compilation of the style file could redirect
to a sample or test file.

%%%%%%%%%%%%%%%%%%%%%%%%%%%%%%%%%%%%%%%%%%%%%%%%%%%%%%%%%%%%%%%%%%%%%%%%%%%%%%%%
%%%%%%%%%%%%%%%%%%%%%%%%%%%%%%%%%%%%%%%%%%%%%%%%%%%%%%%%%%%%%%%%%%%%%%%%%%%%%%%%
\section{Usage}

First of all, the package \textsf{childdoc} is \emph{not} a standard
\LaTeXe{} |.sty| style file! Therefore it needs to be invoked in
a non-standard way.

%%%%%%%%%%%%%%%%%%%%%%%%%%%%%%%%%%%%%%%%%%%%%%%%%%%%%%%%%%%%%%%%%%%%%%%%%%%%%%%%
\subsection{Included Files}
\label{sec:include}

%%%%%%%%%%%%%%%%%%%%%%%%%%%%%%%%%%%%%%%%
\DescribeMacro{\childdocmain}
To use the package, add the commands
\begin{center}
\begin{tabular}{l}
|\input{childdoc.def}|\\
|\childdocmain{}|\\
\end{tabular}
\end{center}
at the very top of the main \LaTeX{} file,
in particular \emph{before} the |\documentclass| statement!
The argument of |\childdocmain| should be left empty
(but it must be present).

%%%%%%%%%%%%%%%%%%%%%%%%%%%%%%%%%%%%%%%%
\DescribeMacro{\childdocof}
Furthermore, add the commands
\begin{center}
\begin{tabular}{l}
|\input{childdoc.def}|\\
|\childdocof{|\textit{main}|}|\\
\end{tabular}
\end{center}
at the top of every child file \textit{child}
which is included by |\include{|\textit{child}|}|
from within the main file
(or at least for those files to be compiled individually).
The argument \textit{main} must be the filename of the main file.

There are a couple of
considerations in setting up the main and child documents:

%%%%%%%%%%%%%%%%%%%%%%%%%%%%%%%%%%%%%%%%
\paragraph{Restrictions.}

Please note the following restrictions:
\begin{itemize}
\item
|\childdocmain| must be called with one argument \textit{main}
to ensure compatibility with earlier version of the package.
It must either be empty (|\childdocmain{}|)
or precisely match the filename of the main file in which it is specified.
See \secref{sec:detection} for further information.
\item
The filename \textit{main} must be specified without the |.tex| extension.
\item
The filename \textit{main} is case sensitive
(even in case-insensitive file systems)
due to internal string comparison.
\item
The argument \textit{main} should be fully expanded, it cannot be a macro.
\item
Subdirectories and special characters should be avoided in filenames.
\item
The command |\childdocmain{|\textit{main}|}| must be followed by a whitespace.
It should not be followed immediately by another command
or by a comment mark `|%|'.
This is because the \TeX{} parser reads the token immediately following
the argument of |\childdocmain| and puts it
at the beginning of every child section;
however, a white\-space is ignored.
\end{itemize}

%%%%%%%%%%%%%%%%%%%%%%%%%%%%%%%%%%%%%%%%
\paragraph{Content of Main File.}

It is advisable to place all content in the child files included by |\include|.
Any output contained in the main file will appear in all child documents
unless suppressed manually;
it cannot be suppressed automatically by the |\includeonly| directive
and thus should normally be avoided.
A method to include some content in the main file
by means of conditional processing is described in \secref{sec:conditional}.

%%%%%%%%%%%%%%%%%%%%%%%%%%%%%%%%%%%%%%%%
\paragraph{Page Numbering.}

When only a part of the document is compiled,
the appropriate numbering of pages
(as well as other status parameters)
is determined from the |.aux| files.
The latter contain information from previous passes.
However this information needs to propagate through
all intermediate child documents.
Therefore the page numbering in child documents may well
be inconsistent until the complete document is compiled at least once.

A useful (if unconventional) way to always ensure a consistent
page numbering is to restart the numbering in each child document
and denote the pages by `\textit{child}|.|\textit{page}'
where \textit{child} represents the chapter/section number of the child file.
This can be achieved by the command
|\numberwithin{page}{|\textit{child}|}|
of the \textsf{amsmath} package
where \textit{child} can be |chapter| or |section|
depending on the chosen structuring.
Alternatively, one can modify the macro |\thepage| appropriately
and reset the counter |page| at the start of each child file.

%%%%%%%%%%%%%%%%%%%%%%%%%%%%%%%%%%%%%%%%%%%%%%%%%%%%%%%%%%%%%%%%%%%%%%%%%%%%%%%%
\subsection{Conditional Processing}
\label{sec:conditional}

The package provides a mechanism to compile different versions
of a document. To customise the versions further some conditional processing
can come in handy to distinguish which version is being compiled.
The package provides two macros to describe the compilation context:

%%%%%%%%%%%%%%%%%%%%%%%%%%%%%%%%%%%%%%%%
\DescribeMacro{\ifchilddoc}
The conditional |\ifchilddoc| distinguishes between the compilation of
child documents and the main document:
%
\begin{center}
|\ifchilddoc |\textit{child-code}| |[|\||else |\textit{main-code}]| \||fi|
\end{center}

%%%%%%%%%%%%%%%%%%%%%%%%%%%%%%%%%%%%%%%%
\DescribeMacro{\childdocname}
\DescribeMacro{\childdocjob}
The macro |\childdocname| contains the filename (without extension)
of the main or child file being processed.
Note that |\childdocjob| will always contain the name of the main file.

%%%%%%%%%%%%%%%%%%%%%%%%%%%%%%%%%%%%%%%%
\paragraph{Title Page.}

Conditional processing can be used to include a title or banner page
in the main document when proper precautions are taken.
Importantly, the code in the main file should ensure that the page counter
(as well as other status parameters which are stored in the |.aux| files)
takes the same value after the conditional processing.
Otherwise the page numbers may take divergent values
depending on which part is compiled.

For example, a title page could be declared by:
%
\begin{center}
\begin{tabular}{l}
|\ifchilddoc\||else|\\
|\addtocounter{page}{-1}|\\
\textit{code for title page}\\
|\newpage|\\
|\||fi|
\end{tabular}
\end{center}
%
A banner page for the child documents can be generated by:
%
\begin{center}
\begin{tabular}{l}
|\ifchilddoc|\\
|\addtocounter{page}{-1}|\\
\textit{code for banner page}\\
|\newpage|\\
|\||fi|
\end{tabular}
\end{center}
%
Here one could write a message such as:
\begin{center}
|This is the part \childdocname{} of \childdocjob{}.|
\end{center}

%%%%%%%%%%%%%%%%%%%%%%%%%%%%%%%%%%%%%%%%%%%%%%%%%%%%%%%%%%%%%%%%%%%%%%%%%%%%%%%%
\subsection{Flags}
\label{sec:flags}

The package makes it easy to generate different versions
of the main or child documents.
To this end compilation flags can be defined
and assigned different default values.
They will be particularly useful in conjunction
with the forwarding mechanism described in \secref{sec:forward}.

For example, it may be useful to have a flag |\version|
which can be set to |draft| or |final|.
The document source will contain some conditional code
depending on the value of |\version|.
Suppose further, the flag should default to |final| for the main file
and to |draft| for child files
which is a natural assignment for editing the document.
This is achieved by placing the following code
in the preamble of the main document
(below the |\childdocmain| directive):
%
\begin{center}
\begin{tabular}{l}
|\ifchilddoc|\\
|\providecommand{\version}{draft}|\\
|\||else|\\
|\providecommand{\version}{final}|\\
|\||fi|
\end{tabular}
\end{center}
%
The definition by |\providecommand| makes sure
that previous definitions are not overwritten.
Further statements |\providecommand{\version}{...}|
can thus be added before the above code to override it.

For the main file, one might add a line
(between |\childdocmain| and the above block)
%
\begin{center}
|%\ifchilddoc\||else\providecommand{\version}{draft}\||fi|
\end{center}
%
which can be uncommented to produce a draft version.
Likewise one can add a line to the very top of a child file
(above the |\childdocof{|\textit{main}|}| directive)
%
\begin{center}
|%\providecommand{\version}{final}|
\end{center}
%
which can be uncommented to produce the final version of this child document.

%%%%%%%%%%%%%%%%%%%%%%%%%%%%%%%%%%%%%%%%%%%%%%%%%%%%%%%%%%%%%%%%%%%%%%%%%%%%%%%%
\subsection{Forwarding}
\label{sec:forward}

Different versions of the main or child documents
using compilation flags as described in \secref{sec:flags}
can be (permanently) stored in different files
for convenient compilation, viewing and distribution.
To this end, the package defines a command
to pass on compilation to a different file:

%%%%%%%%%%%%%%%%%%%%%%%%%%%%%%%%%%%%%%%%
\DescribeMacro{\childdocforward}
The command |\childdocforward| redirects processing to
another source file:
%
\begin{center}
\begin{tabular}{l}
|\input{childdoc.def}|\\
|\childdocforward[|\textit{main}|]{|\textit{dest}|}|\\
\end{tabular}
\end{center}
%
The argument \textit{dest} is the destination file
(without extension).
It should be the main file or one of the child files.
Note that further \textsf{childdoc} directives
such as |\childdocof| and |\childdocforward|
in the indicated file will be processed in this form.
The optional argument \textit{main}
passes on directly to the main file \textit{main}
while pretending to compile the child \textit{dest}.
This form behaves as if \textit{dest}
issues |\childdocof{|\textit{main}|}| right away,
and no further \textsf{childdoc} directives will be processed.

%%%%%%%%%%%%%%%%%%%%%%%%%%%%%%%%%%%%%%%%
\DescribeMacro{\...prefix}
In the alternative form |\childdocforwardprefix|,
%
\begin{center}
\begin{tabular}{l}
|\input{childdoc.def}|\\
|\childdocforwardprefix[|\textit{main}|]{|\textit{prefix}|}{|\textit{dest}|}|
\end{tabular}
\end{center}
%
the destination file is determined by a pattern
depending on the current file:
To make this work, the current file must be called
`{\textit{prefix}\hspace{0.2em}\textit{suffix}}'
with \textit{prefix} matching precisely the argument.
Processing is then passed on to the file
`{\textit{dest}\hspace{0.2em}\textit{suffix}}'.
Surely, the same effect is achieved by
directly specifying the
argument `{\textit{dest}\hspace{0.2em}\textit{suffix}}'
in the first form.
However, that requires to set up a different file
for each child. With the alternative form of the command
all these files can have exactly the same content
which simplifies setting them up and maintaining them.

For example, the following file |draft.tex|
with a compilation flag |\version| as described in \secref{sec:flags}
compiles the main document as a draft:
%
\begin{center}
\begin{tabular}{l}
|\def\version{draft}|\\
|\input{childdoc.def}|\\
|\childdocforward{|\textit{main}|}|
\end{tabular}
\end{center}
%
Likewise, the following files |final|\textit{nn}|.tex|
compile the final version of the child document
|child|\textit{nn}|.tex|:
%
\begin{center}
\begin{tabular}{l}
|\def\version{final}|\\
|\input{childdoc.def}|\\
|\childdocforwardprefix{final}{child}|
\end{tabular}
\end{center}
%

Note that when several versions of a main file and/or of each child file
are to be generated, it may be convenient to set up a |Makefile| or
shell script to automatise the process.

%%%%%%%%%%%%%%%%%%%%%%%%%%%%%%%%%%%%%%%%%%%%%%%%%%%%%%%%%%%%%%%%%%%%%%%%%%%%%%%%
\subsection{Command Line Processing}
\label{sec:commandline}

The effect of redirection files can also be achieved by invoking
the \LaTeX{} compiler with a more elaborate command line.
Most conveniently this should be done as part
of a shell script or a |Makefile|.

When using \textsf{childdoc} in the main file, the following
command lines effectively perform a redirection
(note that depending on the shell being used,
backslashes may have to be doubled: `|\|' $\to$ `|\\|'):
%
\begin{center}
|... -jobname "|\textit{target}|" |\\|"|[\textit{flags}]%
|\input{childdoc.def}\childdocforward[|\textit{main}|]{|\textit{dest}|}"|
\end{center}
%
Here \textit{target} is the name of the output file,
\textit{main} is the name of the main file
and \textit{dest} is the name of the main or child file to be processed
(all filenames without extensions).
The optional argument \textit{main} can be omitted
if \textit{main} matches \textit{dest}.
Optionally, compilation \textit{flags} can be defined via |\def| commands.
This command line makes the \TeX{} engine believe
it is compiling the file \textit{target}
whose content is specified as the latter parameter.
The provided code then forwards the processing to
\textit{main} or \textit{dest} as described in \secref{sec:forward}.

%%%%%%%%%%%%%%%%%%%%%%%%%%%%%%%%%%%%%%%%%%%%%%%%%%%%%%%%%%%%%%%%%%%%%%%%%%%%%%%%
\subsection{Include by Input}
\label{sec:input}

Including child documents by |\include| has some restrictions by design.
Most notably, the content of a child document always occupies
its own set of pages; pages cannot be shared between child documents.
Usually, this behaviour makes perfect sense
because each child document contain an essential part of the document.
However, in some situations it may be desirable to compose
a document from a collection of parts
without having mandatory page breaks between then.
For this case, the package
provides a mechanism to include parts
by |\input| which can also be processed individually.
However, by construction this mechanism
requires manual handling of the content to be output.

%%%%%%%%%%%%%%%%%%%%%%%%%%%%%%%%%%%%%%%%
\DescribeMacro{\ifchilddocmanual}
The main file should be prepared as usual, see \secref{sec:include}.
However, the document body must make a distinction
between processing of an individual part and of the main document, e.g.:
%
\begin{center}
\begin{tabular}{l}
|\ifchilddocmanual|\\
|\input{\childdocname}|\\
|\||else|\\
\textit{document body with }|\input{|\textit{part}|}|\\
|\||fi|
\end{tabular}
\end{center}
%
The conditional |\ifchilddocmanual| is true whenever
a part to be included by |\input| is being compiled,
and the name of the part is stored in |\childdocname|.

%%%%%%%%%%%%%%%%%%%%%%%%%%%%%%%%%%%%%%%%
\DescribeMacro{\childdocby}
Each part to be included by |\input| should start with:
%
\begin{center}
\begin{tabular}{l}
|\input{childdoc.def}|\\
|\childdocby{|\textit{main}|}|\\
\end{tabular}
\end{center}
%
The directive |\childdocby| is similar to |\childdocof|
described in \secref{sec:include},
but the subsequent selection of content must be done manually.
To that end, both |\ifchilddoc| and |\ifchilddocmanual|
will be true upon processing of a part,
and the name of the part is stored in |\childdocname|.
Note that |\jobname| will be set to the filename of the current part
so that each part receives an individual |.aux| file
that does not interfere with the |.aux| file(s) of the main document.
This behaviour can be altered by the alternative form
|\childdocby[*]{|\textit{main}|}| (with a non-empty optional argument)
which uses the |.aux| file of the main document
by setting |\jobname| to \textit{main}.

%%%%%%%%%%%%%%%%%%%%%%%%%%%%%%%%%%%%%%%%%%%%%%%%%%%%%%%%%%%%%%%%%%%%%%%%%%%%%%%%
\subsection{Driver Development}
\label{sec:driver}

The \textsf{childdoc} mechanism can also be use for the development
of definition files such as \LaTeX{} styles or classes.
This case differs from the above setup with multiple parts
included by |\include| in that no |\includeonly| should be invoked.
This can be achieved by starting the include file
(before |\ProvidesPackage|) with:
%
\begin{center}
\begin{tabular}{l}
|\input{childdoc.def}|\\
|\childdocforward{|\textit{main}|}|\\
\end{tabular}
\end{center}
%
or alternatively with:
%
\begin{center}
\begin{tabular}{l}
|\input{childdoc.def}|\\
|\childdocby{|\textit{main}|}|\\
\end{tabular}
\end{center}
%
Both forms have slightly different effects as described above.
The main file is prepared as usual, see \secref{sec:include}.

%%%%%%%%%%%%%%%%%%%%%%%%%%%%%%%%%%%%%%%%%%%%%%%%%%%%%%%%%%%%%%%%%%%%%%%%%%%%%%%%
\subsection{Legacy Detection}
\label{sec:detection}

The directive |\childdocmain| in the main file can detect
whether the complete document or merely a child is to be compiled
even without using the directive |\childdocof|.
This method is deprecated because it is less robust
and there is no compelling reason to use it;
it is merely provided for backward compatibility
and it may be removed in future versions.

If the detection mechanism is to be used,
it is mandatory to correctly specify
the filename of the main file as the argument of |\childdocmain|:
%
\begin{center}
\begin{tabular}{l}
|\input{childdoc.def}|\\
|\childdocmain{|\textit{main}|}|\\
\end{tabular}
\end{center}
%
If |\jobname| does not match the argument \textit{main} of |\childdocmain|,
it is assumed that |\jobname| points to the child file to be compiled.
When using |\childdocmain| with the main file specified as argument,
it suffices to start a child file
with just |\input{|\textit{main}|}|
without loading of the package and using |\childdocof|.
If instead all processing is done
with the appropriate \textsf{childdoc} directives,
the argument of \textit{main} of |\childdocmain| can be empty.

An alternative version of the command line processing described
in \secref{sec:commandline} using the detection mechanism reads:
%
\begin{center}
|... -jobname "|\textit{target}|" "|[\textit{flags}]%
[|\def\jobname{|\textit{dest}|}|]|\input{|\textit{main}|}"|
\end{center}

%%%%%%%%%%%%%%%%%%%%%%%%%%%%%%%%%%%%%%%%%%%%%%%%%%%%%%%%%%%%%%%%%%%%%%%%%%%%%%%%
\subsection{Manual Code}
\label{sec:manual}

In case one cannot be certain whether the definitions file |childdoc.def|
is installed on the target \TeX{} distribution
and one prefers not to ship it,
it is conceivable to paste a few relevant commands into the sources.

To that end, drop all statements |\input{childdoc.def}|
and perform the replacements as outlined below.
Instead of |\childdocmain{|\textit{main}|}| add the following code
to the top of the main file:
%
\begin{center}
\begin{tabular}{l}
|\||ifdefined\childdocname\endinput\||fi\newif\ifchilddoc|\\
|\edef\childdocname{\scantokens\expandafter{\jobname\noexpand}}|\\
|\def\childdocmain{|\textit{main}|}\||ifx\childdocmain\childdocname\||else|\\
|\childdoctrue\includeonly{\childdocname}\let\jobname\childdocmain\||fi|\\
\end{tabular}
\end{center}
%
Instead of |\childdocof{|\textit{main}|}| just include the main file
at the top of each child file:
%
\begin{center}
|\input{|\textit{main}|}|
\end{center}
%
A simple redirection |\childdocforward{|\textit{dest}|}| is achieved by:
%
\begin{center}
|\def\jobname{|\textit{dest}|}\input{\jobname}|
\end{center}
%
The redirection with prefix
|\childdocforwardprefix[|\textit{prefix}|]{|\textit{dest}|}|
is accomplished by:
%
\begin{center}
\begin{tabular}{l}
|{\edef\jobname{\scantokens\expandafter{\jobname\noexpand}}|\\
|\def\redirectjob |\textit{prefix}|#1~~~{\gdef\jobname{|\textit{dest}|#1}}|\\
|\expandafter\redirectjob\jobname~~~}\input{\jobname}|
\end{tabular}
\end{center}

In an alternative approach,
child documents can be compiled by a specific command line
without additional code or specific definitions:
%
\begin{center}
|... -jobname "|\textit{target}|" "|[\textit{flags}]%
|\includeonly{|\textit{dest}|}\input{|\textit{main}|}"|
\end{center}
%

%%%%%%%%%%%%%%%%%%%%%%%%%%%%%%%%%%%%%%%%%%%%%%%%%%%%%%%%%%%%%%%%%%%%%%%%%%%%%%%%
%%%%%%%%%%%%%%%%%%%%%%%%%%%%%%%%%%%%%%%%%%%%%%%%%%%%%%%%%%%%%%%%%%%%%%%%%%%%%%%%
\section{Information}

%%%%%%%%%%%%%%%%%%%%%%%%%%%%%%%%%%%%%%%%%%%%%%%%%%%%%%%%%%%%%%%%%%%%%%%%%%%%%%%%
\subsection{Copyright}

Copyright \copyright{} 2017--2018 Niklas Beisert

This work may be distributed and/or modified under the
conditions of the \LaTeX{} Project Public License, either version 1.3
of this license or (at your option) any later version.
The latest version of this license is in
  \url{http://www.latex-project.org/lppl.txt}
and version 1.3 or later is part of all distributions of \LaTeX{}
version 2005/12/01 or later.

This work has the LPPL maintenance status `maintained'.

The Current Maintainer of this work is Niklas Beisert.

This work consists of the files |README.txt|, |childdoc.ins| and |childdoc.dtx|
as well as the derived files |childdoc.def|, |cdocsamp.tex|
with |cdocsch1.tex|, |cdocsch2.tex|, |cdocspt3.tex|, |cdocspt4.tex|,
|cdocsdrf.tex|, |cdocsfn1.tex|, |cdocsfn2.tex|
as well as |childdoc.pdf|.

%%%%%%%%%%%%%%%%%%%%%%%%%%%%%%%%%%%%%%%%%%%%%%%%%%%%%%%%%%%%%%%%%%%%%%%%%%%%%%%%
\subsection{Files and Installation}

The package consists of the files:
%
\begin{center}
\begin{tabular}{ll}
    |README.txt|   & readme file \\
    |childdoc.ins| & installation file \\
    |childdoc.dtx| & source file \\
    |childdoc.def| & definition file \\
    |cdocsamp.tex| & sample main file \\
    |cdocsch1.tex| & sample include file \\
    |cdocsch2.tex| & sample include file \\
    |cdocspt3.tex| & sample part file \\
    |cdocspt4.tex| & sample part file \\
    |cdocsdrf.tex| & sample redirection file \\
    |cdocsfn1.tex| & sample redirection file \\
    |cdocsfn2.tex| & sample redirection file \\
    |childdoc.pdf| & manual
\end{tabular}
\end{center}
%
The distribution consists of the files
|README.txt|, |childdoc.ins| and |childdoc.dtx|.
%
\begin{itemize}
\item
Run (pdf)\LaTeX{} on |childdoc.dtx|
to compile the manual |childdoc.pdf| (this file).
\item
Run \LaTeX{} on |childdoc.ins| to create the definitions file |childdoc.def|
and the sample |cdocsamp.tex| with include files
|cdocsch1.tex|, |cdocsch2.tex|, |cdocspt3.tex|, |cdocspt4.tex|,
|cdocsdrf.tex|, |cdocsfn1.tex|, |cdocsfn2.tex|.
Then copy the file |childdoc.def| to an appropriate directory of your \LaTeX{}
distribution, e.g.\ \textit{texmf-root}|/tex/latex/childdoc|.
\end{itemize}

%%%%%%%%%%%%%%%%%%%%%%%%%%%%%%%%%%%%%%%%%%%%%%%%%%%%%%%%%%%%%%%%%%%%%%%%%%%%%%%%
\subsection{Related CTAN Packages}

There are several other packages which offer a similar functionality:
%
\begin{itemize}
\item
The packages
\href{http://ctan.org/pkg/docmute}{\textsf{docmute}},
\href{http://ctan.org/pkg/includex}{\textsf{includex}} and
\href{http://ctan.org/pkg/standalone}{\textsf{standalone}}
provide commands to include only the document body of
a child file thus allowing both files to be compiled individually.
\item
The packages \href{http://ctan.org/pkg/subdocs}{\textsf{subdocs}}
and \href{http://ctan.org/pkg/subfiles}{\textsf{subfiles}}
provide structures in which the main and child documents can be
encapsulated and allowing them to be compiled individually.
The inclusion mechanism is different from the conventional |\include|.
\item
The package \href{http://ctan.org/pkg/combine}{\textsf{combine}}
is an elaborate solution to combine several documents into one.
\end{itemize}
%
See also the CTAN topic \href{http://ctan.org/topic/subdocs}{\textsf{subdocs}}
for further related packages.
The present package differs from the above solutions in that
a document structure constructed with the conventional |\include| mechanism
just needs two extra commands at the top of every file
such that all constituent files can be compiled individually.

%%%%%%%%%%%%%%%%%%%%%%%%%%%%%%%%%%%%%%%%%%%%%%%%%%%%%%%%%%%%%%%%%%%%%%%%%%%%%%%%
%\subsection{Feature Suggestions}
%
%The following is a list of features which may be useful for future
%versions of this package:
%%
%\begin{itemize}
%\item
%\ldots
%\end{itemize}

%%%%%%%%%%%%%%%%%%%%%%%%%%%%%%%%%%%%%%%%%%%%%%%%%%%%%%%%%%%%%%%%%%%%%%%%%%%%%%%%
\subsection{Revision History}

%%%%%%%%%%%%%%%%%%%%%%%%%%%%%%%%%%%%%%%%
\paragraph{v2.0:} 2018/12/30

\begin{itemize}
\item
immediate forward processing
\item
added |\childdocby| mechanism
\item
manual restructured
\end{itemize}

%%%%%%%%%%%%%%%%%%%%%%%%%%%%%%%%%%%%%%%%
\paragraph{v1.6:} 2018/01/17

\begin{itemize}
\item
application for development of include files
\item
corrections to manual
\end{itemize}

%%%%%%%%%%%%%%%%%%%%%%%%%%%%%%%%%%%%%%%%
\paragraph{v1.5:} 2017/05/21

\begin{itemize}
\item
more complete structuring introduced
\item
|\childdocof| introduced
\item
|\childdoc| renamed to |\childdocmain|
\item
|\childredirect| renamed to |\childdocforward| and |\childdocforwardprefix|
and functionality expanded
\end{itemize}

%%%%%%%%%%%%%%%%%%%%%%%%%%%%%%%%%%%%%%%%
\paragraph{v1.0:} 2017/04/27

\begin{itemize}
\item
manual and install package
\item
first version published on CTAN
\end{itemize}

%%%%%%%%%%%%%%%%%%%%%%%%%%%%%%%%%%%%%%%%
\paragraph{v0.6:} 2017/04/26

\begin{itemize}
\item
redirection mechanism added
\end{itemize}

%%%%%%%%%%%%%%%%%%%%%%%%%%%%%%%%%%%%%%%%
\paragraph{v0.5:} 2017/04/26

\begin{itemize}
\item
functionality in definition file
\end{itemize}


%%%%%%%%%%%%%%%%%%%%%%%%%%%%%%%%%%%%%%%%%%%%%%%%%%%%%%%%%%%%%%%%%%%%%%%%%%%%%%%%
%%%%%%%%%%%%%%%%%%%%%%%%%%%%%%%%%%%%%%%%%%%%%%%%%%%%%%%%%%%%%%%%%%%%%%%%%%%%%%%%
%%%%%%%%%%%%%%%%%%%%%%%%%%%%%%%%%%%%%%%%%%%%%%%%%%%%%%%%%%%%%%%%%%%%%%%%%%%%%%%%
\appendix

\settowidth\MacroIndent{\rmfamily\scriptsize 000\ }

 \DocInput{childdoc.dtx}

\end{document}
%</driver>
% \fi
%
% %%%%%%%%%%%%%%%%%%%%%%%%%%%%%%%%%%%%%%%%%%%%%%%%%%%%%%%%%%%%%%%%%%%%%%%%%%%%%%
% %%%%%%%%%%%%%%%%%%%%%%%%%%%%%%%%%%%%%%%%%%%%%%%%%%%%%%%%%%%%%%%%%%%%%%%%%%%%%%
% \section{Sample}
%\iffalse
%<*samplemain>
%\fi
%
% The following presents a sample document
% with two chapters, two parts, a title page,
% a compile flag as well as three forwarding files to set the flag.
% It consists of eight |.tex| files:
% \begin{center}
% \begin{tabular}{ll}
% |cdocsamp.tex|&main file\\
% |cdocsch1.tex|&include file for chapter 1\\
% |cdocsch2.tex|&include file for chapter 2\\
% |cdocspt3.tex|&include file for part 3\\
% |cdocspt4.tex|&include file for part 4\\
% |cdocsdrf.tex|&forwarding file for main file in draft mode\\
% |cdocsfi1.tex|&forwarding file for final version of chapter 1\\
% |cdocsfi2.tex|&forwarding file for final version of chapter 2\\
% \end{tabular}
% \end{center}
% Each of the eight files can be compiled directly by the \LaTeX{} compiler.
%
% %%%%%%%%%%%%%%%%%%%%%%%%%%%%%%%%%%%%%%
% \paragraph{Main File.}
%
% The main file is called |cdocsamp.tex|.
%
% Load the \textsf{childdoc} definitions and
% declare the filename for the main document:
%    \begin{macrocode}
\input{childdoc.def}
\childdocmain{}
%    \end{macrocode}

% Optional override for |\version| flag:
%    \begin{macrocode}
%%\ifchilddoc\else\providecommand{\version}{draft}\fi
%    \end{macrocode}

% Define the default values for the |\version| flag
% (|final| for the main file and |draft| for childs):
%    \begin{macrocode}
\ifchilddoc
\providecommand{\version}{draft}
\else
\providecommand{\version}{final}
\fi
%    \end{macrocode}

% Load the standard document class:
%    \begin{macrocode}
\documentclass[12pt]{article}
%    \end{macrocode}

% Start the document body:
%    \begin{macrocode}
\begin{document}
%    \end{macrocode}

% Declare a title page.
% Print title, part of document being processed and version flag:
%    \begin{macrocode}
\addtocounter{page}{-1}
\begin{center}
{\LARGE\bfseries{}childdoc example\par}
\vspace{1cm}
\ifchilddoc
\ifchilddocmanual part\else chapter\fi:
`\childdocname' of `\childdocjob'\par
\else
main document: `\childdocjob'\par
\fi
version: \version\par
\end{center}
\newpage
%    \end{macrocode}

% Manually include selected file,
% otherwise process as usual:
%    \begin{macrocode}
\ifchilddocmanual
\section*{part `\childdocname'}
\input{\childdocname}
\else
%    \end{macrocode}

% Include the two chapters:
%    \begin{macrocode}
\include{cdocsch1}
\include{cdocsch2}
%    \end{macrocode}

% Include the two parts unless only chapters should be displayed:
%    \begin{macrocode}
\ifchilddoc\else
\section{part three}
\input{cdocspt3}
\section{part four}
\input{cdocspt4}
\fi
%    \end{macrocode}

% Process as usual until here:
%    \begin{macrocode}
\fi
%    \end{macrocode}

% End of document body:
%    \begin{macrocode}
\end{document}
%    \end{macrocode}
%\iffalse
%</samplemain>
%\fi
%
% %%%%%%%%%%%%%%%%%%%%%%%%%%%%%%%%%%%%%%
% \paragraph{Chapter Include Files.}
%
% The include files are called |cdocsch1.tex| and |cdocsch2.tex|.
%
%\iffalse
%<*samplechap1|samplechap2>
%\fi

% Optional override for |\version| flag:
%    \begin{macrocode}
%%\providecommand{\version}{final}
%    \end{macrocode}

% Include the main document:
%    \begin{macrocode}
\input{childdoc.def}
\childdocof{cdocsamp}
%    \end{macrocode}

%\iffalse
%</samplechap1|samplechap2>
%\fi
%
%\iffalse
%<*samplechap1>
%\fi
% Some text for chapter 1:
%    \begin{macrocode}
\section{one}
some text in chapter one
%    \end{macrocode}

%\iffalse
%</samplechap1>
%\fi
% Some text for chapter 2:
%\iffalse
%<*samplechap2>
%\fi
%    \begin{macrocode}
\section{two}
more text in chapter two
%    \end{macrocode}

%\iffalse
%</samplechap2>
%\fi
%
% %%%%%%%%%%%%%%%%%%%%%%%%%%%%%%%%%%%%%%
% \paragraph{Part Include Files.}
%
% The include files are called |cdocspt3.tex| and |cdocspt4.tex|.
%
%\iffalse
%<*samplepart3|samplepart4>
%\fi

% Optional override for |\version| flag:
%    \begin{macrocode}
%%\providecommand{\version}{final}
%    \end{macrocode}

% Include the main document:
%    \begin{macrocode}
\input{childdoc.def}
\childdocby{cdocsamp}
%    \end{macrocode}

%\iffalse
%</samplepart3|samplepart4>
%\fi
%
%\iffalse
%<*samplepart3>
%\fi
% Some text for part 3:
%    \begin{macrocode}
some text in part three
%    \end{macrocode}

%\iffalse
%</samplepart3>
%\fi
% Some text for part 4:
%\iffalse
%<*samplepart4>
%\fi
%    \begin{macrocode}
more text in part four
%    \end{macrocode}

%\iffalse
%</samplepart4>
%\fi
%
% %%%%%%%%%%%%%%%%%%%%%%%%%%%%%%%%%%%%%%
% \paragraph{Forwarding for a Complete Draft.}
%
% The following forwarding file |cdocsdrf.tex|
% compiles the main document in draft mode:
%\iffalse
%<*sampledraft>
%\fi
%    \begin{macrocode}
\def\version{draft}
\input{childdoc.def}
\childdocforward{cdocsamp}
%    \end{macrocode}

%\iffalse
%</sampledraft>
%\fi
%
% %%%%%%%%%%%%%%%%%%%%%%%%%%%%%%%%%%%%%%
% \paragraph{Forwarding for Final Version of the Chapters.}
%
% The following forwarding files |cdocsfn1.tex| and |cdocsfn2.tex|
% (with identical content)
% compile the final versions of the child documents
% |cdocsch1.tex| and |cdocsch2.tex|, respectively:
%\iffalse
%<*samplefinal>
%\fi
%    \begin{macrocode}
\def\version{final}
\input{childdoc.def}
\childdocforwardprefix[cdocsamp]{cdocsfn}{cdocsch}
%    \end{macrocode}

%\iffalse
%</samplefinal>
%\fi
%
% %%%%%%%%%%%%%%%%%%%%%%%%%%%%%%%%%%%%%%
% \paragraph{Command Line Processing.}
%
% The following three command lines generate the output files
% |cdocscld|, |cdocscl1| and |cdocscl2|
% which should be identical to
% |cdocsdrf|, |cdocsch1| and |cdocsfn2|, respectively:
% \begin{center}
% \begin{tabular}{l}
% |latex -jobname cdocscld \|\\
% |  "\def\version{draft}\input{childdoc.def}\childdocforward{cdocsamp}"|\\
% |latex -jobname cdocscl1 \|\\
% |  "\input{childdoc.def}\childdocforward[cdocsamp]{cdocsch1}"|\\
% |latex -jobname cdocscl2 \|\\
% |  "\def\version{final}\input{childdoc.def}\childdocforward{cdocsch2}"|
% \end{tabular}
% \end{center}
% Note that the trailing backslash on each first line
% merely continues the input to the second line
% (for convenient cut ant paste).
% Furthermore, the command |latex| can be replaced by any
% of its alternative versions such as |pdflatex|.
%
% %%%%%%%%%%%%%%%%%%%%%%%%%%%%%%%%%%%%%%%%%%%%%%%%%%%%%%%%%%%%%%%%%%%%%%%%%%%%%%
% %%%%%%%%%%%%%%%%%%%%%%%%%%%%%%%%%%%%%%%%%%%%%%%%%%%%%%%%%%%%%%%%%%%%%%%%%%%%%%
% \section{Implementation}
%\iffalse
%<*package>
%\fi
%
% This section describes the definitions file |childdoc.def|.

% The definitions cannot be loaded using |\usepackage| or |\RequirePackage|
% which has a mechanism to prevent loading a style file more than once.
% When loading the definitions by means of |\input|
% multiple instances have to be prevented manually:
%\iffalse
%This code needs to be before the `\ProvidesFile' directive
%which is defined at the beginning of this file.
%Therefore it is also placed there and commented out here.
%</package>
%<*discard>
%\fi
%    \begin{macrocode}
\ifdefined\childdocmain\endinput\fi
%    \end{macrocode}
%\iffalse
%</discard>
%<*package>
%\fi
%
% \macro{\ifchilddoc}
% \macro{\ifchilddocmanual}
% The conditional |\ifchilddoc| tells whether a
% child (true) or main (false) document is being compiled.
% The conditional |\ifchilddocmanual| tells whether
% the |\includeonly| mechanism is used (false) or
% the selection of child files must be performed manually (true).
% The definitions initialise to false:
%    \begin{macrocode}
\newif\ifchilddoc
\newif\ifchilddocmanual
%    \end{macrocode}

% \macro{\childdocname}
% \macro{\childdocjob}
% The macro |\childdocname| stores the name of the main document
% to be compiled. The macro |\childdocjob| stores the name of
% the document on which the \LaTeX{} compiler was originally invoked.
% The content of |\jobname| cannot be compared
% to filenames specified in the source due to different catcodes.
% The following code rescans |\jobname|, stores the result
% in |\childdocname| and saves a copy in |\childdocjob|:
%    \begin{macrocode}
\edef\childdocname{\scantokens\expandafter{\jobname\noexpand}}
\let\childdocjob\childdocname
%    \end{macrocode}

% \macro{\childdocdisable}
% The macro |\childdocdisable| prevents the main file
% from being processed more than once.
% At this stage, the main document command |\childdocmain|
% is assumed to be called once again where it should do nothing.
% Any subsequent call to it should prevent
% a secondary processing of the main document
% It overwrites the forwarding commands
% |\childdocof| and |\childdocforward|
% with empty macros to prevent further inclusions of the main document:
%    \begin{macrocode}
\newcommand{\childdocdisable}
{
  \renewcommand{\childdocmain}[1]{\renewcommand{\childdocmain}[1]{\endinput}}
  \renewcommand{\childdocof}[1]{}
  \renewcommand{\childdocby}[2][]{}
  \renewcommand{\childdocforward}[2][]{}
  \renewcommand{\childdocdisable}{}
}
%    \end{macrocode}

% \macro{\childdocmain}
% The macro |\childdocmain| is to be called at the top of the main file
% with nothing or the main filename (without extension) as argument.
% First, it breaks loops.
% If the argument is not empty and does not match |\childdocname|
% (which is set by the first inclusion of |childdoc.def|),
% |\ifchilddoc| is set to true, |\includeonly| is applied to the child file
% and |\jobname| is set to the main file
% (for proper handling of |.aux| files):
%    \begin{macrocode}
\newcommand{\childdocmain}[1]
{
  \childdocdisable\childdocmain{}
  \if?#1?\else
    \begingroup
      \def\childdoctmp{#1}
      \ifx\childdoctmp\childdocname
        \def\childdoctmp{}
      \else
        \def\childdoctmp
        {
          \childdoctrue
          \includeonly{\childdocname}
          \def\childdocjob{#1}
          \def\jobname{#1}
        }
      \fi
      \expandafter
    \endgroup
    \childdoctmp
  \fi
}
%    \end{macrocode}

% \macro{\childdocof}
% The command |\childdocof| redirects
% compilation to the main file |#1|.
%    \begin{macrocode}
\newcommand{\childdocof}[1]
{
  \childdocdisable
  \childdoctrue
  \includeonly{\childdocname}
  \def\jobname{#1}
  \def\childdocjob{#1}
  \input{#1}
}
%    \end{macrocode}

% \macro{\childdocby}
% The command |\childdocby| ....
%    \begin{macrocode}
\newcommand{\childdocby}[2][]
{
  \childdocdisable
  \childdoctrue
  \childdocmanualtrue
  \if?#1?\else
    \def\jobname{#2}
  \fi
  \def\childdocjob{#2}
  \input{#2}
  \endinput
}
%    \end{macrocode}

% \macro{\childdocforward}
% The command |\childdocforward| redirects
% compilation to the main file or
% (if the optional argument is given) a child file.
% Parameters are set as if the main file
% or a child file starting with |\childdocof| was compiled.
% Then compilation is handed over to the main file:
%    \begin{macrocode}
\newcommand{\childdocforward}[2][]
{
  \begingroup
    \if?#1?
      \def\childdoctmp
      {
        \def\childdocname{#2}
        \def\childdocjob{#2}
        \def\jobname{#2}
        \input{#2}
        \endinput
      }
    \else
      \def\childdoctmp
      {
        \childdocdisable
        \def\childdocname{#2}
        \childdoctrue
        \includeonly{#2}
        \def\childdocjob{#1}
        \def\jobname{#1}
        \input{#1}
        \endinput
      }
    \fi
    \expandafter
  \endgroup
  \childdoctmp
}
%    \end{macrocode}

% \macro{\childdocforwardprefix}
% The command |\childdocforwardprefix| redirects
% compilation to the main or a child file by means of a pattern.
% The prefix |#1| in the current filename is replaced by |#2|
% and the suffix of the current filename is kept
% (it is assumed that the filename does not contain the substring `|~~~|'
% which is used as a delimiter).
% Compilation is handed over to the new file by |\childdocforward|:
%    \begin{macrocode}
\newcommand{\childdocforwardprefix}[3][]
{
  \begingroup
    \def\childdocextract #2##1~~~{\def\childdoctmp{\childdocforward[#1]{#3##1}}}
    \expandafter\childdocextract\childdocname~~~
    \expandafter
  \endgroup
  \childdoctmp
}
%    \end{macrocode}

% \macro{\childdoc}
% The deprecated macro |\childdoc| is a legacy version of |\childdocmain|:
%    \begin{macrocode}
\newcommand{\childdoc}{\childdocmain}
%    \end{macrocode}

% \macro{\childdocredirect}
% The deprecated macro |\childdocredirect| is a legacy version
% of |\childdocforward| and |\childdocforwardprefix|:
%    \begin{macrocode}
\newcommand{\childdocredirect}[2][]
{
  \begingroup
    \if?#1?
      \def\childdoctmp{\childdocforward{#2}}
    \else
      \def\childdoctmp{\childdocforwardprefix{#1}{#2}}
    \fi
    \expandafter
  \endgroup
  \childdoctmp
}
%    \end{macrocode}

%\iffalse
%</package>
%\fi
%
\endinput
|\\
|\childdocmain{|\textit{main}|}|\\
\end{tabular}
\end{center}
%
If |\jobname| does not match the argument \textit{main} of |\childdocmain|,
it is assumed that |\jobname| points to the child file to be compiled.
When using |\childdocmain| with the main file specified as argument,
it suffices to start a child file
with just |\input{|\textit{main}|}|
without loading of the package and using |\childdocof|.
If instead all processing is done
with the appropriate \textsf{childdoc} directives,
the argument of \textit{main} of |\childdocmain| can be empty.

An alternative version of the command line processing described
in \secref{sec:commandline} using the detection mechanism reads:
%
\begin{center}
|... -jobname "|\textit{target}|" "|[\textit{flags}]%
[|\def\jobname{|\textit{dest}|}|]|\input{|\textit{main}|}"|
\end{center}

%%%%%%%%%%%%%%%%%%%%%%%%%%%%%%%%%%%%%%%%%%%%%%%%%%%%%%%%%%%%%%%%%%%%%%%%%%%%%%%%
\subsection{Manual Code}
\label{sec:manual}

In case one cannot be certain whether the definitions file |childdoc.def|
is installed on the target \TeX{} distribution
and one prefers not to ship it,
it is conceivable to paste a few relevant commands into the sources.

To that end, drop all statements |% \iffalse
%
% childdoc.dtx Copyright (C) 2017-2018 Niklas Beisert
%
% This work may be distributed and/or modified under the
% conditions of the LaTeX Project Public License, either version 1.3
% of this license or (at your option) any later version.
% The latest version of this license is in
%   http://www.latex-project.org/lppl.txt
% and version 1.3 or later is part of all distributions of LaTeX
% version 2005/12/01 or later.
%
% This work has the LPPL maintenance status `maintained'.
%
% The Current Maintainer of this work is Niklas Beisert.
%
% This work consists of the files childdoc.dtx and childdoc.ins
% and the derived files childdoc.def and cdocsamp.tex with
% cdocsch1.tex, cdocsch2.tex, cdocsdrf.tex, cdocsfn1.tex, cdocsfn2.tex.
%
%<package>\ifdefined\childdocmain\endinput\fi
%<package>\ProvidesFile{childdoc.def}[2018/12/30 v2.0 child document driver]
%<samplemain>\ProvidesFile{cdocsamp.tex}[2018/12/30 v2.0 sample for childdoc]
%<*driver>
%\ProvidesFile{childdoc.drv}[2018/12/30 v2.0 childdoc reference manual file]
\PassOptionsToClass{10pt,a4paper}{article}
\documentclass{ltxdoc}

\usepackage[margin=35mm]{geometry}
\usepackage{hyperref}
\usepackage{hyperxmp}
\usepackage[usenames]{color}

\hypersetup{colorlinks=true}
\hypersetup{pdfstartview=FitH}
\hypersetup{pdfpagemode=UseNone}
\hypersetup{pdfsource={}}
\hypersetup{pdflang={en-UK}}
\hypersetup{pdfcopyright={Copyright 2017-2018 Niklas Beisert.
  This work may be distributed and/or modified under the
  conditions of the LaTeX Project Public License, either version 1.3
  of this license or (at your option) any later version.}}
\hypersetup{pdflicenseurl={http://www.latex-project.org/lppl.txt}}
\hypersetup{pdfcontactaddress={ETH Zurich, ITP, HIT K,
  Wolfgang-Pauli-Strasse 27}}
\hypersetup{pdfcontactpostcode={8093}}
\hypersetup{pdfcontactcity={Zurich}}
\hypersetup{pdfcontactcountry={Switzerland}}
\hypersetup{pdfcontactemail={nbeisert@itp.phys.ethz.ch}}
\hypersetup{pdfcontacturl={http://people.phys.ethz.ch/\xmptilde nbeisert/}}

\newcommand{\secref}[1]{\hyperref[#1]{section \ref*{#1}}}

\parskip1ex
\parindent0pt
\let\olditemize\itemize
\def\itemize{\olditemize\parskip0pt}

\begin{document}

\title{The \textsf{childdoc} Package}
\hypersetup{pdftitle={The childdoc Package}}
\author{Niklas Beisert\\[2ex]
  Institut f\"ur Theoretische Physik\\
  Eidgen\"ossische Technische Hochschule Z\"urich\\
  Wolfgang-Pauli-Strasse 27, 8093 Z\"urich, Switzerland\\[1ex]
  \href{mailto:nbeisert@itp.phys.ethz.ch}
  {\texttt{nbeisert@itp.phys.ethz.ch}}}
\hypersetup{pdfauthor={Niklas Beisert}}
\hypersetup{pdfsubject={Manual for the LaTeX2e Package childdoc}}
\date{30 December 2018, \textsf{v2.0}}
\maketitle

\begin{abstract}\noindent
\textsf{childdoc} is a \LaTeXe{} package
that enables the direct compilation
of document sections included by |\include|
to individual files.
\end{abstract}

\begingroup
\parskip0ex
\tableofcontents
\endgroup

%%%%%%%%%%%%%%%%%%%%%%%%%%%%%%%%%%%%%%%%%%%%%%%%%%%%%%%%%%%%%%%%%%%%%%%%%%%%%%%%
%%%%%%%%%%%%%%%%%%%%%%%%%%%%%%%%%%%%%%%%%%%%%%%%%%%%%%%%%%%%%%%%%%%%%%%%%%%%%%%%
\section{Introduction}

\LaTeX{} provides a mechanism to structure a large document (such as a book)
into a main file and several child files (containing the chapters)
using the |\include| command.
This mechanism is beneficial for documents
which span hundreds of pages in order to
make the source file(s) more manageable.
Moreover, compilation can be restricted to
selected child files by means of the |\includeonly| command.
The latter feature can be used to reduce the compilation time while editing
(this was significantly more useful in the earlier days of \LaTeX{})
or to generate a smaller document which is easier to navigate.
Another application of |\includeonly| is to generate
documents consisting of selected parts of the complete document.

However, there are a few drawbacks of the plain |\include| mechanism:
\begin{itemize}
\item
The child files cannot be compiled on their own,
they can only be compiled via the main file.
A naive editing environment
(such as a text editor with an option
to have the current file processed by \LaTeX)
may require one to switch to the main file before compiling;
attempting to compile the child file produces errors.
\item
The main file must be modified (each time)
to adjust the |\includeonly| command
to the present needs. This easily leaves the main file in a messy state.
\item
The generated document will always carry the filename
of the main document. This is inconvenient if
several child files are to be compiled and
to be kept for distribution.
\end{itemize}

The present package provides a simple interface
to make child files individually compilable by \LaTeX{}.
Compiling a child file then has the same effect as compiling
the main file with an |\includeonly| command
to select the appropriate child.
Moreover the generated document will carry the name of the child
rather than the main file.
This resolves all three above issues.

This feature is meant to make the editing of books,
thesis documents and lecture notes somewhat more convenient.
However, the package can also be used efficiently for
composing a series of documents (such as exercise sheets)
which are typically distributed individually.
It then assists the author in generating the individual documents
(potentially in different versions)
as well as a document containing the collected series.
Another application is in developing style files
or other kinds of included material
where compilation of the style file could redirect
to a sample or test file.

%%%%%%%%%%%%%%%%%%%%%%%%%%%%%%%%%%%%%%%%%%%%%%%%%%%%%%%%%%%%%%%%%%%%%%%%%%%%%%%%
%%%%%%%%%%%%%%%%%%%%%%%%%%%%%%%%%%%%%%%%%%%%%%%%%%%%%%%%%%%%%%%%%%%%%%%%%%%%%%%%
\section{Usage}

First of all, the package \textsf{childdoc} is \emph{not} a standard
\LaTeXe{} |.sty| style file! Therefore it needs to be invoked in
a non-standard way.

%%%%%%%%%%%%%%%%%%%%%%%%%%%%%%%%%%%%%%%%%%%%%%%%%%%%%%%%%%%%%%%%%%%%%%%%%%%%%%%%
\subsection{Included Files}
\label{sec:include}

%%%%%%%%%%%%%%%%%%%%%%%%%%%%%%%%%%%%%%%%
\DescribeMacro{\childdocmain}
To use the package, add the commands
\begin{center}
\begin{tabular}{l}
|\input{childdoc.def}|\\
|\childdocmain{}|\\
\end{tabular}
\end{center}
at the very top of the main \LaTeX{} file,
in particular \emph{before} the |\documentclass| statement!
The argument of |\childdocmain| should be left empty
(but it must be present).

%%%%%%%%%%%%%%%%%%%%%%%%%%%%%%%%%%%%%%%%
\DescribeMacro{\childdocof}
Furthermore, add the commands
\begin{center}
\begin{tabular}{l}
|\input{childdoc.def}|\\
|\childdocof{|\textit{main}|}|\\
\end{tabular}
\end{center}
at the top of every child file \textit{child}
which is included by |\include{|\textit{child}|}|
from within the main file
(or at least for those files to be compiled individually).
The argument \textit{main} must be the filename of the main file.

There are a couple of
considerations in setting up the main and child documents:

%%%%%%%%%%%%%%%%%%%%%%%%%%%%%%%%%%%%%%%%
\paragraph{Restrictions.}

Please note the following restrictions:
\begin{itemize}
\item
|\childdocmain| must be called with one argument \textit{main}
to ensure compatibility with earlier version of the package.
It must either be empty (|\childdocmain{}|)
or precisely match the filename of the main file in which it is specified.
See \secref{sec:detection} for further information.
\item
The filename \textit{main} must be specified without the |.tex| extension.
\item
The filename \textit{main} is case sensitive
(even in case-insensitive file systems)
due to internal string comparison.
\item
The argument \textit{main} should be fully expanded, it cannot be a macro.
\item
Subdirectories and special characters should be avoided in filenames.
\item
The command |\childdocmain{|\textit{main}|}| must be followed by a whitespace.
It should not be followed immediately by another command
or by a comment mark `|%|'.
This is because the \TeX{} parser reads the token immediately following
the argument of |\childdocmain| and puts it
at the beginning of every child section;
however, a white\-space is ignored.
\end{itemize}

%%%%%%%%%%%%%%%%%%%%%%%%%%%%%%%%%%%%%%%%
\paragraph{Content of Main File.}

It is advisable to place all content in the child files included by |\include|.
Any output contained in the main file will appear in all child documents
unless suppressed manually;
it cannot be suppressed automatically by the |\includeonly| directive
and thus should normally be avoided.
A method to include some content in the main file
by means of conditional processing is described in \secref{sec:conditional}.

%%%%%%%%%%%%%%%%%%%%%%%%%%%%%%%%%%%%%%%%
\paragraph{Page Numbering.}

When only a part of the document is compiled,
the appropriate numbering of pages
(as well as other status parameters)
is determined from the |.aux| files.
The latter contain information from previous passes.
However this information needs to propagate through
all intermediate child documents.
Therefore the page numbering in child documents may well
be inconsistent until the complete document is compiled at least once.

A useful (if unconventional) way to always ensure a consistent
page numbering is to restart the numbering in each child document
and denote the pages by `\textit{child}|.|\textit{page}'
where \textit{child} represents the chapter/section number of the child file.
This can be achieved by the command
|\numberwithin{page}{|\textit{child}|}|
of the \textsf{amsmath} package
where \textit{child} can be |chapter| or |section|
depending on the chosen structuring.
Alternatively, one can modify the macro |\thepage| appropriately
and reset the counter |page| at the start of each child file.

%%%%%%%%%%%%%%%%%%%%%%%%%%%%%%%%%%%%%%%%%%%%%%%%%%%%%%%%%%%%%%%%%%%%%%%%%%%%%%%%
\subsection{Conditional Processing}
\label{sec:conditional}

The package provides a mechanism to compile different versions
of a document. To customise the versions further some conditional processing
can come in handy to distinguish which version is being compiled.
The package provides two macros to describe the compilation context:

%%%%%%%%%%%%%%%%%%%%%%%%%%%%%%%%%%%%%%%%
\DescribeMacro{\ifchilddoc}
The conditional |\ifchilddoc| distinguishes between the compilation of
child documents and the main document:
%
\begin{center}
|\ifchilddoc |\textit{child-code}| |[|\||else |\textit{main-code}]| \||fi|
\end{center}

%%%%%%%%%%%%%%%%%%%%%%%%%%%%%%%%%%%%%%%%
\DescribeMacro{\childdocname}
\DescribeMacro{\childdocjob}
The macro |\childdocname| contains the filename (without extension)
of the main or child file being processed.
Note that |\childdocjob| will always contain the name of the main file.

%%%%%%%%%%%%%%%%%%%%%%%%%%%%%%%%%%%%%%%%
\paragraph{Title Page.}

Conditional processing can be used to include a title or banner page
in the main document when proper precautions are taken.
Importantly, the code in the main file should ensure that the page counter
(as well as other status parameters which are stored in the |.aux| files)
takes the same value after the conditional processing.
Otherwise the page numbers may take divergent values
depending on which part is compiled.

For example, a title page could be declared by:
%
\begin{center}
\begin{tabular}{l}
|\ifchilddoc\||else|\\
|\addtocounter{page}{-1}|\\
\textit{code for title page}\\
|\newpage|\\
|\||fi|
\end{tabular}
\end{center}
%
A banner page for the child documents can be generated by:
%
\begin{center}
\begin{tabular}{l}
|\ifchilddoc|\\
|\addtocounter{page}{-1}|\\
\textit{code for banner page}\\
|\newpage|\\
|\||fi|
\end{tabular}
\end{center}
%
Here one could write a message such as:
\begin{center}
|This is the part \childdocname{} of \childdocjob{}.|
\end{center}

%%%%%%%%%%%%%%%%%%%%%%%%%%%%%%%%%%%%%%%%%%%%%%%%%%%%%%%%%%%%%%%%%%%%%%%%%%%%%%%%
\subsection{Flags}
\label{sec:flags}

The package makes it easy to generate different versions
of the main or child documents.
To this end compilation flags can be defined
and assigned different default values.
They will be particularly useful in conjunction
with the forwarding mechanism described in \secref{sec:forward}.

For example, it may be useful to have a flag |\version|
which can be set to |draft| or |final|.
The document source will contain some conditional code
depending on the value of |\version|.
Suppose further, the flag should default to |final| for the main file
and to |draft| for child files
which is a natural assignment for editing the document.
This is achieved by placing the following code
in the preamble of the main document
(below the |\childdocmain| directive):
%
\begin{center}
\begin{tabular}{l}
|\ifchilddoc|\\
|\providecommand{\version}{draft}|\\
|\||else|\\
|\providecommand{\version}{final}|\\
|\||fi|
\end{tabular}
\end{center}
%
The definition by |\providecommand| makes sure
that previous definitions are not overwritten.
Further statements |\providecommand{\version}{...}|
can thus be added before the above code to override it.

For the main file, one might add a line
(between |\childdocmain| and the above block)
%
\begin{center}
|%\ifchilddoc\||else\providecommand{\version}{draft}\||fi|
\end{center}
%
which can be uncommented to produce a draft version.
Likewise one can add a line to the very top of a child file
(above the |\childdocof{|\textit{main}|}| directive)
%
\begin{center}
|%\providecommand{\version}{final}|
\end{center}
%
which can be uncommented to produce the final version of this child document.

%%%%%%%%%%%%%%%%%%%%%%%%%%%%%%%%%%%%%%%%%%%%%%%%%%%%%%%%%%%%%%%%%%%%%%%%%%%%%%%%
\subsection{Forwarding}
\label{sec:forward}

Different versions of the main or child documents
using compilation flags as described in \secref{sec:flags}
can be (permanently) stored in different files
for convenient compilation, viewing and distribution.
To this end, the package defines a command
to pass on compilation to a different file:

%%%%%%%%%%%%%%%%%%%%%%%%%%%%%%%%%%%%%%%%
\DescribeMacro{\childdocforward}
The command |\childdocforward| redirects processing to
another source file:
%
\begin{center}
\begin{tabular}{l}
|\input{childdoc.def}|\\
|\childdocforward[|\textit{main}|]{|\textit{dest}|}|\\
\end{tabular}
\end{center}
%
The argument \textit{dest} is the destination file
(without extension).
It should be the main file or one of the child files.
Note that further \textsf{childdoc} directives
such as |\childdocof| and |\childdocforward|
in the indicated file will be processed in this form.
The optional argument \textit{main}
passes on directly to the main file \textit{main}
while pretending to compile the child \textit{dest}.
This form behaves as if \textit{dest}
issues |\childdocof{|\textit{main}|}| right away,
and no further \textsf{childdoc} directives will be processed.

%%%%%%%%%%%%%%%%%%%%%%%%%%%%%%%%%%%%%%%%
\DescribeMacro{\...prefix}
In the alternative form |\childdocforwardprefix|,
%
\begin{center}
\begin{tabular}{l}
|\input{childdoc.def}|\\
|\childdocforwardprefix[|\textit{main}|]{|\textit{prefix}|}{|\textit{dest}|}|
\end{tabular}
\end{center}
%
the destination file is determined by a pattern
depending on the current file:
To make this work, the current file must be called
`{\textit{prefix}\hspace{0.2em}\textit{suffix}}'
with \textit{prefix} matching precisely the argument.
Processing is then passed on to the file
`{\textit{dest}\hspace{0.2em}\textit{suffix}}'.
Surely, the same effect is achieved by
directly specifying the
argument `{\textit{dest}\hspace{0.2em}\textit{suffix}}'
in the first form.
However, that requires to set up a different file
for each child. With the alternative form of the command
all these files can have exactly the same content
which simplifies setting them up and maintaining them.

For example, the following file |draft.tex|
with a compilation flag |\version| as described in \secref{sec:flags}
compiles the main document as a draft:
%
\begin{center}
\begin{tabular}{l}
|\def\version{draft}|\\
|\input{childdoc.def}|\\
|\childdocforward{|\textit{main}|}|
\end{tabular}
\end{center}
%
Likewise, the following files |final|\textit{nn}|.tex|
compile the final version of the child document
|child|\textit{nn}|.tex|:
%
\begin{center}
\begin{tabular}{l}
|\def\version{final}|\\
|\input{childdoc.def}|\\
|\childdocforwardprefix{final}{child}|
\end{tabular}
\end{center}
%

Note that when several versions of a main file and/or of each child file
are to be generated, it may be convenient to set up a |Makefile| or
shell script to automatise the process.

%%%%%%%%%%%%%%%%%%%%%%%%%%%%%%%%%%%%%%%%%%%%%%%%%%%%%%%%%%%%%%%%%%%%%%%%%%%%%%%%
\subsection{Command Line Processing}
\label{sec:commandline}

The effect of redirection files can also be achieved by invoking
the \LaTeX{} compiler with a more elaborate command line.
Most conveniently this should be done as part
of a shell script or a |Makefile|.

When using \textsf{childdoc} in the main file, the following
command lines effectively perform a redirection
(note that depending on the shell being used,
backslashes may have to be doubled: `|\|' $\to$ `|\\|'):
%
\begin{center}
|... -jobname "|\textit{target}|" |\\|"|[\textit{flags}]%
|\input{childdoc.def}\childdocforward[|\textit{main}|]{|\textit{dest}|}"|
\end{center}
%
Here \textit{target} is the name of the output file,
\textit{main} is the name of the main file
and \textit{dest} is the name of the main or child file to be processed
(all filenames without extensions).
The optional argument \textit{main} can be omitted
if \textit{main} matches \textit{dest}.
Optionally, compilation \textit{flags} can be defined via |\def| commands.
This command line makes the \TeX{} engine believe
it is compiling the file \textit{target}
whose content is specified as the latter parameter.
The provided code then forwards the processing to
\textit{main} or \textit{dest} as described in \secref{sec:forward}.

%%%%%%%%%%%%%%%%%%%%%%%%%%%%%%%%%%%%%%%%%%%%%%%%%%%%%%%%%%%%%%%%%%%%%%%%%%%%%%%%
\subsection{Include by Input}
\label{sec:input}

Including child documents by |\include| has some restrictions by design.
Most notably, the content of a child document always occupies
its own set of pages; pages cannot be shared between child documents.
Usually, this behaviour makes perfect sense
because each child document contain an essential part of the document.
However, in some situations it may be desirable to compose
a document from a collection of parts
without having mandatory page breaks between then.
For this case, the package
provides a mechanism to include parts
by |\input| which can also be processed individually.
However, by construction this mechanism
requires manual handling of the content to be output.

%%%%%%%%%%%%%%%%%%%%%%%%%%%%%%%%%%%%%%%%
\DescribeMacro{\ifchilddocmanual}
The main file should be prepared as usual, see \secref{sec:include}.
However, the document body must make a distinction
between processing of an individual part and of the main document, e.g.:
%
\begin{center}
\begin{tabular}{l}
|\ifchilddocmanual|\\
|\input{\childdocname}|\\
|\||else|\\
\textit{document body with }|\input{|\textit{part}|}|\\
|\||fi|
\end{tabular}
\end{center}
%
The conditional |\ifchilddocmanual| is true whenever
a part to be included by |\input| is being compiled,
and the name of the part is stored in |\childdocname|.

%%%%%%%%%%%%%%%%%%%%%%%%%%%%%%%%%%%%%%%%
\DescribeMacro{\childdocby}
Each part to be included by |\input| should start with:
%
\begin{center}
\begin{tabular}{l}
|\input{childdoc.def}|\\
|\childdocby{|\textit{main}|}|\\
\end{tabular}
\end{center}
%
The directive |\childdocby| is similar to |\childdocof|
described in \secref{sec:include},
but the subsequent selection of content must be done manually.
To that end, both |\ifchilddoc| and |\ifchilddocmanual|
will be true upon processing of a part,
and the name of the part is stored in |\childdocname|.
Note that |\jobname| will be set to the filename of the current part
so that each part receives an individual |.aux| file
that does not interfere with the |.aux| file(s) of the main document.
This behaviour can be altered by the alternative form
|\childdocby[*]{|\textit{main}|}| (with a non-empty optional argument)
which uses the |.aux| file of the main document
by setting |\jobname| to \textit{main}.

%%%%%%%%%%%%%%%%%%%%%%%%%%%%%%%%%%%%%%%%%%%%%%%%%%%%%%%%%%%%%%%%%%%%%%%%%%%%%%%%
\subsection{Driver Development}
\label{sec:driver}

The \textsf{childdoc} mechanism can also be use for the development
of definition files such as \LaTeX{} styles or classes.
This case differs from the above setup with multiple parts
included by |\include| in that no |\includeonly| should be invoked.
This can be achieved by starting the include file
(before |\ProvidesPackage|) with:
%
\begin{center}
\begin{tabular}{l}
|\input{childdoc.def}|\\
|\childdocforward{|\textit{main}|}|\\
\end{tabular}
\end{center}
%
or alternatively with:
%
\begin{center}
\begin{tabular}{l}
|\input{childdoc.def}|\\
|\childdocby{|\textit{main}|}|\\
\end{tabular}
\end{center}
%
Both forms have slightly different effects as described above.
The main file is prepared as usual, see \secref{sec:include}.

%%%%%%%%%%%%%%%%%%%%%%%%%%%%%%%%%%%%%%%%%%%%%%%%%%%%%%%%%%%%%%%%%%%%%%%%%%%%%%%%
\subsection{Legacy Detection}
\label{sec:detection}

The directive |\childdocmain| in the main file can detect
whether the complete document or merely a child is to be compiled
even without using the directive |\childdocof|.
This method is deprecated because it is less robust
and there is no compelling reason to use it;
it is merely provided for backward compatibility
and it may be removed in future versions.

If the detection mechanism is to be used,
it is mandatory to correctly specify
the filename of the main file as the argument of |\childdocmain|:
%
\begin{center}
\begin{tabular}{l}
|\input{childdoc.def}|\\
|\childdocmain{|\textit{main}|}|\\
\end{tabular}
\end{center}
%
If |\jobname| does not match the argument \textit{main} of |\childdocmain|,
it is assumed that |\jobname| points to the child file to be compiled.
When using |\childdocmain| with the main file specified as argument,
it suffices to start a child file
with just |\input{|\textit{main}|}|
without loading of the package and using |\childdocof|.
If instead all processing is done
with the appropriate \textsf{childdoc} directives,
the argument of \textit{main} of |\childdocmain| can be empty.

An alternative version of the command line processing described
in \secref{sec:commandline} using the detection mechanism reads:
%
\begin{center}
|... -jobname "|\textit{target}|" "|[\textit{flags}]%
[|\def\jobname{|\textit{dest}|}|]|\input{|\textit{main}|}"|
\end{center}

%%%%%%%%%%%%%%%%%%%%%%%%%%%%%%%%%%%%%%%%%%%%%%%%%%%%%%%%%%%%%%%%%%%%%%%%%%%%%%%%
\subsection{Manual Code}
\label{sec:manual}

In case one cannot be certain whether the definitions file |childdoc.def|
is installed on the target \TeX{} distribution
and one prefers not to ship it,
it is conceivable to paste a few relevant commands into the sources.

To that end, drop all statements |\input{childdoc.def}|
and perform the replacements as outlined below.
Instead of |\childdocmain{|\textit{main}|}| add the following code
to the top of the main file:
%
\begin{center}
\begin{tabular}{l}
|\||ifdefined\childdocname\endinput\||fi\newif\ifchilddoc|\\
|\edef\childdocname{\scantokens\expandafter{\jobname\noexpand}}|\\
|\def\childdocmain{|\textit{main}|}\||ifx\childdocmain\childdocname\||else|\\
|\childdoctrue\includeonly{\childdocname}\let\jobname\childdocmain\||fi|\\
\end{tabular}
\end{center}
%
Instead of |\childdocof{|\textit{main}|}| just include the main file
at the top of each child file:
%
\begin{center}
|\input{|\textit{main}|}|
\end{center}
%
A simple redirection |\childdocforward{|\textit{dest}|}| is achieved by:
%
\begin{center}
|\def\jobname{|\textit{dest}|}\input{\jobname}|
\end{center}
%
The redirection with prefix
|\childdocforwardprefix[|\textit{prefix}|]{|\textit{dest}|}|
is accomplished by:
%
\begin{center}
\begin{tabular}{l}
|{\edef\jobname{\scantokens\expandafter{\jobname\noexpand}}|\\
|\def\redirectjob |\textit{prefix}|#1~~~{\gdef\jobname{|\textit{dest}|#1}}|\\
|\expandafter\redirectjob\jobname~~~}\input{\jobname}|
\end{tabular}
\end{center}

In an alternative approach,
child documents can be compiled by a specific command line
without additional code or specific definitions:
%
\begin{center}
|... -jobname "|\textit{target}|" "|[\textit{flags}]%
|\includeonly{|\textit{dest}|}\input{|\textit{main}|}"|
\end{center}
%

%%%%%%%%%%%%%%%%%%%%%%%%%%%%%%%%%%%%%%%%%%%%%%%%%%%%%%%%%%%%%%%%%%%%%%%%%%%%%%%%
%%%%%%%%%%%%%%%%%%%%%%%%%%%%%%%%%%%%%%%%%%%%%%%%%%%%%%%%%%%%%%%%%%%%%%%%%%%%%%%%
\section{Information}

%%%%%%%%%%%%%%%%%%%%%%%%%%%%%%%%%%%%%%%%%%%%%%%%%%%%%%%%%%%%%%%%%%%%%%%%%%%%%%%%
\subsection{Copyright}

Copyright \copyright{} 2017--2018 Niklas Beisert

This work may be distributed and/or modified under the
conditions of the \LaTeX{} Project Public License, either version 1.3
of this license or (at your option) any later version.
The latest version of this license is in
  \url{http://www.latex-project.org/lppl.txt}
and version 1.3 or later is part of all distributions of \LaTeX{}
version 2005/12/01 or later.

This work has the LPPL maintenance status `maintained'.

The Current Maintainer of this work is Niklas Beisert.

This work consists of the files |README.txt|, |childdoc.ins| and |childdoc.dtx|
as well as the derived files |childdoc.def|, |cdocsamp.tex|
with |cdocsch1.tex|, |cdocsch2.tex|, |cdocspt3.tex|, |cdocspt4.tex|,
|cdocsdrf.tex|, |cdocsfn1.tex|, |cdocsfn2.tex|
as well as |childdoc.pdf|.

%%%%%%%%%%%%%%%%%%%%%%%%%%%%%%%%%%%%%%%%%%%%%%%%%%%%%%%%%%%%%%%%%%%%%%%%%%%%%%%%
\subsection{Files and Installation}

The package consists of the files:
%
\begin{center}
\begin{tabular}{ll}
    |README.txt|   & readme file \\
    |childdoc.ins| & installation file \\
    |childdoc.dtx| & source file \\
    |childdoc.def| & definition file \\
    |cdocsamp.tex| & sample main file \\
    |cdocsch1.tex| & sample include file \\
    |cdocsch2.tex| & sample include file \\
    |cdocspt3.tex| & sample part file \\
    |cdocspt4.tex| & sample part file \\
    |cdocsdrf.tex| & sample redirection file \\
    |cdocsfn1.tex| & sample redirection file \\
    |cdocsfn2.tex| & sample redirection file \\
    |childdoc.pdf| & manual
\end{tabular}
\end{center}
%
The distribution consists of the files
|README.txt|, |childdoc.ins| and |childdoc.dtx|.
%
\begin{itemize}
\item
Run (pdf)\LaTeX{} on |childdoc.dtx|
to compile the manual |childdoc.pdf| (this file).
\item
Run \LaTeX{} on |childdoc.ins| to create the definitions file |childdoc.def|
and the sample |cdocsamp.tex| with include files
|cdocsch1.tex|, |cdocsch2.tex|, |cdocspt3.tex|, |cdocspt4.tex|,
|cdocsdrf.tex|, |cdocsfn1.tex|, |cdocsfn2.tex|.
Then copy the file |childdoc.def| to an appropriate directory of your \LaTeX{}
distribution, e.g.\ \textit{texmf-root}|/tex/latex/childdoc|.
\end{itemize}

%%%%%%%%%%%%%%%%%%%%%%%%%%%%%%%%%%%%%%%%%%%%%%%%%%%%%%%%%%%%%%%%%%%%%%%%%%%%%%%%
\subsection{Related CTAN Packages}

There are several other packages which offer a similar functionality:
%
\begin{itemize}
\item
The packages
\href{http://ctan.org/pkg/docmute}{\textsf{docmute}},
\href{http://ctan.org/pkg/includex}{\textsf{includex}} and
\href{http://ctan.org/pkg/standalone}{\textsf{standalone}}
provide commands to include only the document body of
a child file thus allowing both files to be compiled individually.
\item
The packages \href{http://ctan.org/pkg/subdocs}{\textsf{subdocs}}
and \href{http://ctan.org/pkg/subfiles}{\textsf{subfiles}}
provide structures in which the main and child documents can be
encapsulated and allowing them to be compiled individually.
The inclusion mechanism is different from the conventional |\include|.
\item
The package \href{http://ctan.org/pkg/combine}{\textsf{combine}}
is an elaborate solution to combine several documents into one.
\end{itemize}
%
See also the CTAN topic \href{http://ctan.org/topic/subdocs}{\textsf{subdocs}}
for further related packages.
The present package differs from the above solutions in that
a document structure constructed with the conventional |\include| mechanism
just needs two extra commands at the top of every file
such that all constituent files can be compiled individually.

%%%%%%%%%%%%%%%%%%%%%%%%%%%%%%%%%%%%%%%%%%%%%%%%%%%%%%%%%%%%%%%%%%%%%%%%%%%%%%%%
%\subsection{Feature Suggestions}
%
%The following is a list of features which may be useful for future
%versions of this package:
%%
%\begin{itemize}
%\item
%\ldots
%\end{itemize}

%%%%%%%%%%%%%%%%%%%%%%%%%%%%%%%%%%%%%%%%%%%%%%%%%%%%%%%%%%%%%%%%%%%%%%%%%%%%%%%%
\subsection{Revision History}

%%%%%%%%%%%%%%%%%%%%%%%%%%%%%%%%%%%%%%%%
\paragraph{v2.0:} 2018/12/30

\begin{itemize}
\item
immediate forward processing
\item
added |\childdocby| mechanism
\item
manual restructured
\end{itemize}

%%%%%%%%%%%%%%%%%%%%%%%%%%%%%%%%%%%%%%%%
\paragraph{v1.6:} 2018/01/17

\begin{itemize}
\item
application for development of include files
\item
corrections to manual
\end{itemize}

%%%%%%%%%%%%%%%%%%%%%%%%%%%%%%%%%%%%%%%%
\paragraph{v1.5:} 2017/05/21

\begin{itemize}
\item
more complete structuring introduced
\item
|\childdocof| introduced
\item
|\childdoc| renamed to |\childdocmain|
\item
|\childredirect| renamed to |\childdocforward| and |\childdocforwardprefix|
and functionality expanded
\end{itemize}

%%%%%%%%%%%%%%%%%%%%%%%%%%%%%%%%%%%%%%%%
\paragraph{v1.0:} 2017/04/27

\begin{itemize}
\item
manual and install package
\item
first version published on CTAN
\end{itemize}

%%%%%%%%%%%%%%%%%%%%%%%%%%%%%%%%%%%%%%%%
\paragraph{v0.6:} 2017/04/26

\begin{itemize}
\item
redirection mechanism added
\end{itemize}

%%%%%%%%%%%%%%%%%%%%%%%%%%%%%%%%%%%%%%%%
\paragraph{v0.5:} 2017/04/26

\begin{itemize}
\item
functionality in definition file
\end{itemize}


%%%%%%%%%%%%%%%%%%%%%%%%%%%%%%%%%%%%%%%%%%%%%%%%%%%%%%%%%%%%%%%%%%%%%%%%%%%%%%%%
%%%%%%%%%%%%%%%%%%%%%%%%%%%%%%%%%%%%%%%%%%%%%%%%%%%%%%%%%%%%%%%%%%%%%%%%%%%%%%%%
%%%%%%%%%%%%%%%%%%%%%%%%%%%%%%%%%%%%%%%%%%%%%%%%%%%%%%%%%%%%%%%%%%%%%%%%%%%%%%%%
\appendix

\settowidth\MacroIndent{\rmfamily\scriptsize 000\ }

 \DocInput{childdoc.dtx}

\end{document}
%</driver>
% \fi
%
% %%%%%%%%%%%%%%%%%%%%%%%%%%%%%%%%%%%%%%%%%%%%%%%%%%%%%%%%%%%%%%%%%%%%%%%%%%%%%%
% %%%%%%%%%%%%%%%%%%%%%%%%%%%%%%%%%%%%%%%%%%%%%%%%%%%%%%%%%%%%%%%%%%%%%%%%%%%%%%
% \section{Sample}
%\iffalse
%<*samplemain>
%\fi
%
% The following presents a sample document
% with two chapters, two parts, a title page,
% a compile flag as well as three forwarding files to set the flag.
% It consists of eight |.tex| files:
% \begin{center}
% \begin{tabular}{ll}
% |cdocsamp.tex|&main file\\
% |cdocsch1.tex|&include file for chapter 1\\
% |cdocsch2.tex|&include file for chapter 2\\
% |cdocspt3.tex|&include file for part 3\\
% |cdocspt4.tex|&include file for part 4\\
% |cdocsdrf.tex|&forwarding file for main file in draft mode\\
% |cdocsfi1.tex|&forwarding file for final version of chapter 1\\
% |cdocsfi2.tex|&forwarding file for final version of chapter 2\\
% \end{tabular}
% \end{center}
% Each of the eight files can be compiled directly by the \LaTeX{} compiler.
%
% %%%%%%%%%%%%%%%%%%%%%%%%%%%%%%%%%%%%%%
% \paragraph{Main File.}
%
% The main file is called |cdocsamp.tex|.
%
% Load the \textsf{childdoc} definitions and
% declare the filename for the main document:
%    \begin{macrocode}
\input{childdoc.def}
\childdocmain{}
%    \end{macrocode}

% Optional override for |\version| flag:
%    \begin{macrocode}
%%\ifchilddoc\else\providecommand{\version}{draft}\fi
%    \end{macrocode}

% Define the default values for the |\version| flag
% (|final| for the main file and |draft| for childs):
%    \begin{macrocode}
\ifchilddoc
\providecommand{\version}{draft}
\else
\providecommand{\version}{final}
\fi
%    \end{macrocode}

% Load the standard document class:
%    \begin{macrocode}
\documentclass[12pt]{article}
%    \end{macrocode}

% Start the document body:
%    \begin{macrocode}
\begin{document}
%    \end{macrocode}

% Declare a title page.
% Print title, part of document being processed and version flag:
%    \begin{macrocode}
\addtocounter{page}{-1}
\begin{center}
{\LARGE\bfseries{}childdoc example\par}
\vspace{1cm}
\ifchilddoc
\ifchilddocmanual part\else chapter\fi:
`\childdocname' of `\childdocjob'\par
\else
main document: `\childdocjob'\par
\fi
version: \version\par
\end{center}
\newpage
%    \end{macrocode}

% Manually include selected file,
% otherwise process as usual:
%    \begin{macrocode}
\ifchilddocmanual
\section*{part `\childdocname'}
\input{\childdocname}
\else
%    \end{macrocode}

% Include the two chapters:
%    \begin{macrocode}
\include{cdocsch1}
\include{cdocsch2}
%    \end{macrocode}

% Include the two parts unless only chapters should be displayed:
%    \begin{macrocode}
\ifchilddoc\else
\section{part three}
\input{cdocspt3}
\section{part four}
\input{cdocspt4}
\fi
%    \end{macrocode}

% Process as usual until here:
%    \begin{macrocode}
\fi
%    \end{macrocode}

% End of document body:
%    \begin{macrocode}
\end{document}
%    \end{macrocode}
%\iffalse
%</samplemain>
%\fi
%
% %%%%%%%%%%%%%%%%%%%%%%%%%%%%%%%%%%%%%%
% \paragraph{Chapter Include Files.}
%
% The include files are called |cdocsch1.tex| and |cdocsch2.tex|.
%
%\iffalse
%<*samplechap1|samplechap2>
%\fi

% Optional override for |\version| flag:
%    \begin{macrocode}
%%\providecommand{\version}{final}
%    \end{macrocode}

% Include the main document:
%    \begin{macrocode}
\input{childdoc.def}
\childdocof{cdocsamp}
%    \end{macrocode}

%\iffalse
%</samplechap1|samplechap2>
%\fi
%
%\iffalse
%<*samplechap1>
%\fi
% Some text for chapter 1:
%    \begin{macrocode}
\section{one}
some text in chapter one
%    \end{macrocode}

%\iffalse
%</samplechap1>
%\fi
% Some text for chapter 2:
%\iffalse
%<*samplechap2>
%\fi
%    \begin{macrocode}
\section{two}
more text in chapter two
%    \end{macrocode}

%\iffalse
%</samplechap2>
%\fi
%
% %%%%%%%%%%%%%%%%%%%%%%%%%%%%%%%%%%%%%%
% \paragraph{Part Include Files.}
%
% The include files are called |cdocspt3.tex| and |cdocspt4.tex|.
%
%\iffalse
%<*samplepart3|samplepart4>
%\fi

% Optional override for |\version| flag:
%    \begin{macrocode}
%%\providecommand{\version}{final}
%    \end{macrocode}

% Include the main document:
%    \begin{macrocode}
\input{childdoc.def}
\childdocby{cdocsamp}
%    \end{macrocode}

%\iffalse
%</samplepart3|samplepart4>
%\fi
%
%\iffalse
%<*samplepart3>
%\fi
% Some text for part 3:
%    \begin{macrocode}
some text in part three
%    \end{macrocode}

%\iffalse
%</samplepart3>
%\fi
% Some text for part 4:
%\iffalse
%<*samplepart4>
%\fi
%    \begin{macrocode}
more text in part four
%    \end{macrocode}

%\iffalse
%</samplepart4>
%\fi
%
% %%%%%%%%%%%%%%%%%%%%%%%%%%%%%%%%%%%%%%
% \paragraph{Forwarding for a Complete Draft.}
%
% The following forwarding file |cdocsdrf.tex|
% compiles the main document in draft mode:
%\iffalse
%<*sampledraft>
%\fi
%    \begin{macrocode}
\def\version{draft}
\input{childdoc.def}
\childdocforward{cdocsamp}
%    \end{macrocode}

%\iffalse
%</sampledraft>
%\fi
%
% %%%%%%%%%%%%%%%%%%%%%%%%%%%%%%%%%%%%%%
% \paragraph{Forwarding for Final Version of the Chapters.}
%
% The following forwarding files |cdocsfn1.tex| and |cdocsfn2.tex|
% (with identical content)
% compile the final versions of the child documents
% |cdocsch1.tex| and |cdocsch2.tex|, respectively:
%\iffalse
%<*samplefinal>
%\fi
%    \begin{macrocode}
\def\version{final}
\input{childdoc.def}
\childdocforwardprefix[cdocsamp]{cdocsfn}{cdocsch}
%    \end{macrocode}

%\iffalse
%</samplefinal>
%\fi
%
% %%%%%%%%%%%%%%%%%%%%%%%%%%%%%%%%%%%%%%
% \paragraph{Command Line Processing.}
%
% The following three command lines generate the output files
% |cdocscld|, |cdocscl1| and |cdocscl2|
% which should be identical to
% |cdocsdrf|, |cdocsch1| and |cdocsfn2|, respectively:
% \begin{center}
% \begin{tabular}{l}
% |latex -jobname cdocscld \|\\
% |  "\def\version{draft}\input{childdoc.def}\childdocforward{cdocsamp}"|\\
% |latex -jobname cdocscl1 \|\\
% |  "\input{childdoc.def}\childdocforward[cdocsamp]{cdocsch1}"|\\
% |latex -jobname cdocscl2 \|\\
% |  "\def\version{final}\input{childdoc.def}\childdocforward{cdocsch2}"|
% \end{tabular}
% \end{center}
% Note that the trailing backslash on each first line
% merely continues the input to the second line
% (for convenient cut ant paste).
% Furthermore, the command |latex| can be replaced by any
% of its alternative versions such as |pdflatex|.
%
% %%%%%%%%%%%%%%%%%%%%%%%%%%%%%%%%%%%%%%%%%%%%%%%%%%%%%%%%%%%%%%%%%%%%%%%%%%%%%%
% %%%%%%%%%%%%%%%%%%%%%%%%%%%%%%%%%%%%%%%%%%%%%%%%%%%%%%%%%%%%%%%%%%%%%%%%%%%%%%
% \section{Implementation}
%\iffalse
%<*package>
%\fi
%
% This section describes the definitions file |childdoc.def|.

% The definitions cannot be loaded using |\usepackage| or |\RequirePackage|
% which has a mechanism to prevent loading a style file more than once.
% When loading the definitions by means of |\input|
% multiple instances have to be prevented manually:
%\iffalse
%This code needs to be before the `\ProvidesFile' directive
%which is defined at the beginning of this file.
%Therefore it is also placed there and commented out here.
%</package>
%<*discard>
%\fi
%    \begin{macrocode}
\ifdefined\childdocmain\endinput\fi
%    \end{macrocode}
%\iffalse
%</discard>
%<*package>
%\fi
%
% \macro{\ifchilddoc}
% \macro{\ifchilddocmanual}
% The conditional |\ifchilddoc| tells whether a
% child (true) or main (false) document is being compiled.
% The conditional |\ifchilddocmanual| tells whether
% the |\includeonly| mechanism is used (false) or
% the selection of child files must be performed manually (true).
% The definitions initialise to false:
%    \begin{macrocode}
\newif\ifchilddoc
\newif\ifchilddocmanual
%    \end{macrocode}

% \macro{\childdocname}
% \macro{\childdocjob}
% The macro |\childdocname| stores the name of the main document
% to be compiled. The macro |\childdocjob| stores the name of
% the document on which the \LaTeX{} compiler was originally invoked.
% The content of |\jobname| cannot be compared
% to filenames specified in the source due to different catcodes.
% The following code rescans |\jobname|, stores the result
% in |\childdocname| and saves a copy in |\childdocjob|:
%    \begin{macrocode}
\edef\childdocname{\scantokens\expandafter{\jobname\noexpand}}
\let\childdocjob\childdocname
%    \end{macrocode}

% \macro{\childdocdisable}
% The macro |\childdocdisable| prevents the main file
% from being processed more than once.
% At this stage, the main document command |\childdocmain|
% is assumed to be called once again where it should do nothing.
% Any subsequent call to it should prevent
% a secondary processing of the main document
% It overwrites the forwarding commands
% |\childdocof| and |\childdocforward|
% with empty macros to prevent further inclusions of the main document:
%    \begin{macrocode}
\newcommand{\childdocdisable}
{
  \renewcommand{\childdocmain}[1]{\renewcommand{\childdocmain}[1]{\endinput}}
  \renewcommand{\childdocof}[1]{}
  \renewcommand{\childdocby}[2][]{}
  \renewcommand{\childdocforward}[2][]{}
  \renewcommand{\childdocdisable}{}
}
%    \end{macrocode}

% \macro{\childdocmain}
% The macro |\childdocmain| is to be called at the top of the main file
% with nothing or the main filename (without extension) as argument.
% First, it breaks loops.
% If the argument is not empty and does not match |\childdocname|
% (which is set by the first inclusion of |childdoc.def|),
% |\ifchilddoc| is set to true, |\includeonly| is applied to the child file
% and |\jobname| is set to the main file
% (for proper handling of |.aux| files):
%    \begin{macrocode}
\newcommand{\childdocmain}[1]
{
  \childdocdisable\childdocmain{}
  \if?#1?\else
    \begingroup
      \def\childdoctmp{#1}
      \ifx\childdoctmp\childdocname
        \def\childdoctmp{}
      \else
        \def\childdoctmp
        {
          \childdoctrue
          \includeonly{\childdocname}
          \def\childdocjob{#1}
          \def\jobname{#1}
        }
      \fi
      \expandafter
    \endgroup
    \childdoctmp
  \fi
}
%    \end{macrocode}

% \macro{\childdocof}
% The command |\childdocof| redirects
% compilation to the main file |#1|.
%    \begin{macrocode}
\newcommand{\childdocof}[1]
{
  \childdocdisable
  \childdoctrue
  \includeonly{\childdocname}
  \def\jobname{#1}
  \def\childdocjob{#1}
  \input{#1}
}
%    \end{macrocode}

% \macro{\childdocby}
% The command |\childdocby| ....
%    \begin{macrocode}
\newcommand{\childdocby}[2][]
{
  \childdocdisable
  \childdoctrue
  \childdocmanualtrue
  \if?#1?\else
    \def\jobname{#2}
  \fi
  \def\childdocjob{#2}
  \input{#2}
  \endinput
}
%    \end{macrocode}

% \macro{\childdocforward}
% The command |\childdocforward| redirects
% compilation to the main file or
% (if the optional argument is given) a child file.
% Parameters are set as if the main file
% or a child file starting with |\childdocof| was compiled.
% Then compilation is handed over to the main file:
%    \begin{macrocode}
\newcommand{\childdocforward}[2][]
{
  \begingroup
    \if?#1?
      \def\childdoctmp
      {
        \def\childdocname{#2}
        \def\childdocjob{#2}
        \def\jobname{#2}
        \input{#2}
        \endinput
      }
    \else
      \def\childdoctmp
      {
        \childdocdisable
        \def\childdocname{#2}
        \childdoctrue
        \includeonly{#2}
        \def\childdocjob{#1}
        \def\jobname{#1}
        \input{#1}
        \endinput
      }
    \fi
    \expandafter
  \endgroup
  \childdoctmp
}
%    \end{macrocode}

% \macro{\childdocforwardprefix}
% The command |\childdocforwardprefix| redirects
% compilation to the main or a child file by means of a pattern.
% The prefix |#1| in the current filename is replaced by |#2|
% and the suffix of the current filename is kept
% (it is assumed that the filename does not contain the substring `|~~~|'
% which is used as a delimiter).
% Compilation is handed over to the new file by |\childdocforward|:
%    \begin{macrocode}
\newcommand{\childdocforwardprefix}[3][]
{
  \begingroup
    \def\childdocextract #2##1~~~{\def\childdoctmp{\childdocforward[#1]{#3##1}}}
    \expandafter\childdocextract\childdocname~~~
    \expandafter
  \endgroup
  \childdoctmp
}
%    \end{macrocode}

% \macro{\childdoc}
% The deprecated macro |\childdoc| is a legacy version of |\childdocmain|:
%    \begin{macrocode}
\newcommand{\childdoc}{\childdocmain}
%    \end{macrocode}

% \macro{\childdocredirect}
% The deprecated macro |\childdocredirect| is a legacy version
% of |\childdocforward| and |\childdocforwardprefix|:
%    \begin{macrocode}
\newcommand{\childdocredirect}[2][]
{
  \begingroup
    \if?#1?
      \def\childdoctmp{\childdocforward{#2}}
    \else
      \def\childdoctmp{\childdocforwardprefix{#1}{#2}}
    \fi
    \expandafter
  \endgroup
  \childdoctmp
}
%    \end{macrocode}

%\iffalse
%</package>
%\fi
%
\endinput
|
and perform the replacements as outlined below.
Instead of |\childdocmain{|\textit{main}|}| add the following code
to the top of the main file:
%
\begin{center}
\begin{tabular}{l}
|\||ifdefined\childdocname\endinput\||fi\newif\ifchilddoc|\\
|\edef\childdocname{\scantokens\expandafter{\jobname\noexpand}}|\\
|\def\childdocmain{|\textit{main}|}\||ifx\childdocmain\childdocname\||else|\\
|\childdoctrue\includeonly{\childdocname}\let\jobname\childdocmain\||fi|\\
\end{tabular}
\end{center}
%
Instead of |\childdocof{|\textit{main}|}| just include the main file
at the top of each child file:
%
\begin{center}
|\input{|\textit{main}|}|
\end{center}
%
A simple redirection |\childdocforward{|\textit{dest}|}| is achieved by:
%
\begin{center}
|\def\jobname{|\textit{dest}|}\input{\jobname}|
\end{center}
%
The redirection with prefix
|\childdocforwardprefix[|\textit{prefix}|]{|\textit{dest}|}|
is accomplished by:
%
\begin{center}
\begin{tabular}{l}
|{\edef\jobname{\scantokens\expandafter{\jobname\noexpand}}|\\
|\def\redirectjob |\textit{prefix}|#1~~~{\gdef\jobname{|\textit{dest}|#1}}|\\
|\expandafter\redirectjob\jobname~~~}\input{\jobname}|
\end{tabular}
\end{center}

In an alternative approach,
child documents can be compiled by a specific command line
without additional code or specific definitions:
%
\begin{center}
|... -jobname "|\textit{target}|" "|[\textit{flags}]%
|\includeonly{|\textit{dest}|}\input{|\textit{main}|}"|
\end{center}
%

%%%%%%%%%%%%%%%%%%%%%%%%%%%%%%%%%%%%%%%%%%%%%%%%%%%%%%%%%%%%%%%%%%%%%%%%%%%%%%%%
%%%%%%%%%%%%%%%%%%%%%%%%%%%%%%%%%%%%%%%%%%%%%%%%%%%%%%%%%%%%%%%%%%%%%%%%%%%%%%%%
\section{Information}

%%%%%%%%%%%%%%%%%%%%%%%%%%%%%%%%%%%%%%%%%%%%%%%%%%%%%%%%%%%%%%%%%%%%%%%%%%%%%%%%
\subsection{Copyright}

Copyright \copyright{} 2017--2018 Niklas Beisert

This work may be distributed and/or modified under the
conditions of the \LaTeX{} Project Public License, either version 1.3
of this license or (at your option) any later version.
The latest version of this license is in
  \url{http://www.latex-project.org/lppl.txt}
and version 1.3 or later is part of all distributions of \LaTeX{}
version 2005/12/01 or later.

This work has the LPPL maintenance status `maintained'.

The Current Maintainer of this work is Niklas Beisert.

This work consists of the files |README.txt|, |childdoc.ins| and |childdoc.dtx|
as well as the derived files |childdoc.def|, |cdocsamp.tex|
with |cdocsch1.tex|, |cdocsch2.tex|, |cdocspt3.tex|, |cdocspt4.tex|,
|cdocsdrf.tex|, |cdocsfn1.tex|, |cdocsfn2.tex|
as well as |childdoc.pdf|.

%%%%%%%%%%%%%%%%%%%%%%%%%%%%%%%%%%%%%%%%%%%%%%%%%%%%%%%%%%%%%%%%%%%%%%%%%%%%%%%%
\subsection{Files and Installation}

The package consists of the files:
%
\begin{center}
\begin{tabular}{ll}
    |README.txt|   & readme file \\
    |childdoc.ins| & installation file \\
    |childdoc.dtx| & source file \\
    |childdoc.def| & definition file \\
    |cdocsamp.tex| & sample main file \\
    |cdocsch1.tex| & sample include file \\
    |cdocsch2.tex| & sample include file \\
    |cdocspt3.tex| & sample part file \\
    |cdocspt4.tex| & sample part file \\
    |cdocsdrf.tex| & sample redirection file \\
    |cdocsfn1.tex| & sample redirection file \\
    |cdocsfn2.tex| & sample redirection file \\
    |childdoc.pdf| & manual
\end{tabular}
\end{center}
%
The distribution consists of the files
|README.txt|, |childdoc.ins| and |childdoc.dtx|.
%
\begin{itemize}
\item
Run (pdf)\LaTeX{} on |childdoc.dtx|
to compile the manual |childdoc.pdf| (this file).
\item
Run \LaTeX{} on |childdoc.ins| to create the definitions file |childdoc.def|
and the sample |cdocsamp.tex| with include files
|cdocsch1.tex|, |cdocsch2.tex|, |cdocspt3.tex|, |cdocspt4.tex|,
|cdocsdrf.tex|, |cdocsfn1.tex|, |cdocsfn2.tex|.
Then copy the file |childdoc.def| to an appropriate directory of your \LaTeX{}
distribution, e.g.\ \textit{texmf-root}|/tex/latex/childdoc|.
\end{itemize}

%%%%%%%%%%%%%%%%%%%%%%%%%%%%%%%%%%%%%%%%%%%%%%%%%%%%%%%%%%%%%%%%%%%%%%%%%%%%%%%%
\subsection{Related CTAN Packages}

There are several other packages which offer a similar functionality:
%
\begin{itemize}
\item
The packages
\href{http://ctan.org/pkg/docmute}{\textsf{docmute}},
\href{http://ctan.org/pkg/includex}{\textsf{includex}} and
\href{http://ctan.org/pkg/standalone}{\textsf{standalone}}
provide commands to include only the document body of
a child file thus allowing both files to be compiled individually.
\item
The packages \href{http://ctan.org/pkg/subdocs}{\textsf{subdocs}}
and \href{http://ctan.org/pkg/subfiles}{\textsf{subfiles}}
provide structures in which the main and child documents can be
encapsulated and allowing them to be compiled individually.
The inclusion mechanism is different from the conventional |\include|.
\item
The package \href{http://ctan.org/pkg/combine}{\textsf{combine}}
is an elaborate solution to combine several documents into one.
\end{itemize}
%
See also the CTAN topic \href{http://ctan.org/topic/subdocs}{\textsf{subdocs}}
for further related packages.
The present package differs from the above solutions in that
a document structure constructed with the conventional |\include| mechanism
just needs two extra commands at the top of every file
such that all constituent files can be compiled individually.

%%%%%%%%%%%%%%%%%%%%%%%%%%%%%%%%%%%%%%%%%%%%%%%%%%%%%%%%%%%%%%%%%%%%%%%%%%%%%%%%
%\subsection{Feature Suggestions}
%
%The following is a list of features which may be useful for future
%versions of this package:
%%
%\begin{itemize}
%\item
%\ldots
%\end{itemize}

%%%%%%%%%%%%%%%%%%%%%%%%%%%%%%%%%%%%%%%%%%%%%%%%%%%%%%%%%%%%%%%%%%%%%%%%%%%%%%%%
\subsection{Revision History}

%%%%%%%%%%%%%%%%%%%%%%%%%%%%%%%%%%%%%%%%
\paragraph{v2.0:} 2018/12/30

\begin{itemize}
\item
immediate forward processing
\item
added |\childdocby| mechanism
\item
manual restructured
\end{itemize}

%%%%%%%%%%%%%%%%%%%%%%%%%%%%%%%%%%%%%%%%
\paragraph{v1.6:} 2018/01/17

\begin{itemize}
\item
application for development of include files
\item
corrections to manual
\end{itemize}

%%%%%%%%%%%%%%%%%%%%%%%%%%%%%%%%%%%%%%%%
\paragraph{v1.5:} 2017/05/21

\begin{itemize}
\item
more complete structuring introduced
\item
|\childdocof| introduced
\item
|\childdoc| renamed to |\childdocmain|
\item
|\childredirect| renamed to |\childdocforward| and |\childdocforwardprefix|
and functionality expanded
\end{itemize}

%%%%%%%%%%%%%%%%%%%%%%%%%%%%%%%%%%%%%%%%
\paragraph{v1.0:} 2017/04/27

\begin{itemize}
\item
manual and install package
\item
first version published on CTAN
\end{itemize}

%%%%%%%%%%%%%%%%%%%%%%%%%%%%%%%%%%%%%%%%
\paragraph{v0.6:} 2017/04/26

\begin{itemize}
\item
redirection mechanism added
\end{itemize}

%%%%%%%%%%%%%%%%%%%%%%%%%%%%%%%%%%%%%%%%
\paragraph{v0.5:} 2017/04/26

\begin{itemize}
\item
functionality in definition file
\end{itemize}


%%%%%%%%%%%%%%%%%%%%%%%%%%%%%%%%%%%%%%%%%%%%%%%%%%%%%%%%%%%%%%%%%%%%%%%%%%%%%%%%
%%%%%%%%%%%%%%%%%%%%%%%%%%%%%%%%%%%%%%%%%%%%%%%%%%%%%%%%%%%%%%%%%%%%%%%%%%%%%%%%
%%%%%%%%%%%%%%%%%%%%%%%%%%%%%%%%%%%%%%%%%%%%%%%%%%%%%%%%%%%%%%%%%%%%%%%%%%%%%%%%
\appendix

\settowidth\MacroIndent{\rmfamily\scriptsize 000\ }

 \DocInput{childdoc.dtx}

\end{document}
%</driver>
% \fi
%
% %%%%%%%%%%%%%%%%%%%%%%%%%%%%%%%%%%%%%%%%%%%%%%%%%%%%%%%%%%%%%%%%%%%%%%%%%%%%%%
% %%%%%%%%%%%%%%%%%%%%%%%%%%%%%%%%%%%%%%%%%%%%%%%%%%%%%%%%%%%%%%%%%%%%%%%%%%%%%%
% \section{Sample}
%\iffalse
%<*samplemain>
%\fi
%
% The following presents a sample document
% with two chapters, two parts, a title page,
% a compile flag as well as three forwarding files to set the flag.
% It consists of eight |.tex| files:
% \begin{center}
% \begin{tabular}{ll}
% |cdocsamp.tex|&main file\\
% |cdocsch1.tex|&include file for chapter 1\\
% |cdocsch2.tex|&include file for chapter 2\\
% |cdocspt3.tex|&include file for part 3\\
% |cdocspt4.tex|&include file for part 4\\
% |cdocsdrf.tex|&forwarding file for main file in draft mode\\
% |cdocsfi1.tex|&forwarding file for final version of chapter 1\\
% |cdocsfi2.tex|&forwarding file for final version of chapter 2\\
% \end{tabular}
% \end{center}
% Each of the eight files can be compiled directly by the \LaTeX{} compiler.
%
% %%%%%%%%%%%%%%%%%%%%%%%%%%%%%%%%%%%%%%
% \paragraph{Main File.}
%
% The main file is called |cdocsamp.tex|.
%
% Load the \textsf{childdoc} definitions and
% declare the filename for the main document:
%    \begin{macrocode}
% \iffalse
%
% childdoc.dtx Copyright (C) 2017-2018 Niklas Beisert
%
% This work may be distributed and/or modified under the
% conditions of the LaTeX Project Public License, either version 1.3
% of this license or (at your option) any later version.
% The latest version of this license is in
%   http://www.latex-project.org/lppl.txt
% and version 1.3 or later is part of all distributions of LaTeX
% version 2005/12/01 or later.
%
% This work has the LPPL maintenance status `maintained'.
%
% The Current Maintainer of this work is Niklas Beisert.
%
% This work consists of the files childdoc.dtx and childdoc.ins
% and the derived files childdoc.def and cdocsamp.tex with
% cdocsch1.tex, cdocsch2.tex, cdocsdrf.tex, cdocsfn1.tex, cdocsfn2.tex.
%
%<package>\ifdefined\childdocmain\endinput\fi
%<package>\ProvidesFile{childdoc.def}[2018/12/30 v2.0 child document driver]
%<samplemain>\ProvidesFile{cdocsamp.tex}[2018/12/30 v2.0 sample for childdoc]
%<*driver>
%\ProvidesFile{childdoc.drv}[2018/12/30 v2.0 childdoc reference manual file]
\PassOptionsToClass{10pt,a4paper}{article}
\documentclass{ltxdoc}

\usepackage[margin=35mm]{geometry}
\usepackage{hyperref}
\usepackage{hyperxmp}
\usepackage[usenames]{color}

\hypersetup{colorlinks=true}
\hypersetup{pdfstartview=FitH}
\hypersetup{pdfpagemode=UseNone}
\hypersetup{pdfsource={}}
\hypersetup{pdflang={en-UK}}
\hypersetup{pdfcopyright={Copyright 2017-2018 Niklas Beisert.
  This work may be distributed and/or modified under the
  conditions of the LaTeX Project Public License, either version 1.3
  of this license or (at your option) any later version.}}
\hypersetup{pdflicenseurl={http://www.latex-project.org/lppl.txt}}
\hypersetup{pdfcontactaddress={ETH Zurich, ITP, HIT K,
  Wolfgang-Pauli-Strasse 27}}
\hypersetup{pdfcontactpostcode={8093}}
\hypersetup{pdfcontactcity={Zurich}}
\hypersetup{pdfcontactcountry={Switzerland}}
\hypersetup{pdfcontactemail={nbeisert@itp.phys.ethz.ch}}
\hypersetup{pdfcontacturl={http://people.phys.ethz.ch/\xmptilde nbeisert/}}

\newcommand{\secref}[1]{\hyperref[#1]{section \ref*{#1}}}

\parskip1ex
\parindent0pt
\let\olditemize\itemize
\def\itemize{\olditemize\parskip0pt}

\begin{document}

\title{The \textsf{childdoc} Package}
\hypersetup{pdftitle={The childdoc Package}}
\author{Niklas Beisert\\[2ex]
  Institut f\"ur Theoretische Physik\\
  Eidgen\"ossische Technische Hochschule Z\"urich\\
  Wolfgang-Pauli-Strasse 27, 8093 Z\"urich, Switzerland\\[1ex]
  \href{mailto:nbeisert@itp.phys.ethz.ch}
  {\texttt{nbeisert@itp.phys.ethz.ch}}}
\hypersetup{pdfauthor={Niklas Beisert}}
\hypersetup{pdfsubject={Manual for the LaTeX2e Package childdoc}}
\date{30 December 2018, \textsf{v2.0}}
\maketitle

\begin{abstract}\noindent
\textsf{childdoc} is a \LaTeXe{} package
that enables the direct compilation
of document sections included by |\include|
to individual files.
\end{abstract}

\begingroup
\parskip0ex
\tableofcontents
\endgroup

%%%%%%%%%%%%%%%%%%%%%%%%%%%%%%%%%%%%%%%%%%%%%%%%%%%%%%%%%%%%%%%%%%%%%%%%%%%%%%%%
%%%%%%%%%%%%%%%%%%%%%%%%%%%%%%%%%%%%%%%%%%%%%%%%%%%%%%%%%%%%%%%%%%%%%%%%%%%%%%%%
\section{Introduction}

\LaTeX{} provides a mechanism to structure a large document (such as a book)
into a main file and several child files (containing the chapters)
using the |\include| command.
This mechanism is beneficial for documents
which span hundreds of pages in order to
make the source file(s) more manageable.
Moreover, compilation can be restricted to
selected child files by means of the |\includeonly| command.
The latter feature can be used to reduce the compilation time while editing
(this was significantly more useful in the earlier days of \LaTeX{})
or to generate a smaller document which is easier to navigate.
Another application of |\includeonly| is to generate
documents consisting of selected parts of the complete document.

However, there are a few drawbacks of the plain |\include| mechanism:
\begin{itemize}
\item
The child files cannot be compiled on their own,
they can only be compiled via the main file.
A naive editing environment
(such as a text editor with an option
to have the current file processed by \LaTeX)
may require one to switch to the main file before compiling;
attempting to compile the child file produces errors.
\item
The main file must be modified (each time)
to adjust the |\includeonly| command
to the present needs. This easily leaves the main file in a messy state.
\item
The generated document will always carry the filename
of the main document. This is inconvenient if
several child files are to be compiled and
to be kept for distribution.
\end{itemize}

The present package provides a simple interface
to make child files individually compilable by \LaTeX{}.
Compiling a child file then has the same effect as compiling
the main file with an |\includeonly| command
to select the appropriate child.
Moreover the generated document will carry the name of the child
rather than the main file.
This resolves all three above issues.

This feature is meant to make the editing of books,
thesis documents and lecture notes somewhat more convenient.
However, the package can also be used efficiently for
composing a series of documents (such as exercise sheets)
which are typically distributed individually.
It then assists the author in generating the individual documents
(potentially in different versions)
as well as a document containing the collected series.
Another application is in developing style files
or other kinds of included material
where compilation of the style file could redirect
to a sample or test file.

%%%%%%%%%%%%%%%%%%%%%%%%%%%%%%%%%%%%%%%%%%%%%%%%%%%%%%%%%%%%%%%%%%%%%%%%%%%%%%%%
%%%%%%%%%%%%%%%%%%%%%%%%%%%%%%%%%%%%%%%%%%%%%%%%%%%%%%%%%%%%%%%%%%%%%%%%%%%%%%%%
\section{Usage}

First of all, the package \textsf{childdoc} is \emph{not} a standard
\LaTeXe{} |.sty| style file! Therefore it needs to be invoked in
a non-standard way.

%%%%%%%%%%%%%%%%%%%%%%%%%%%%%%%%%%%%%%%%%%%%%%%%%%%%%%%%%%%%%%%%%%%%%%%%%%%%%%%%
\subsection{Included Files}
\label{sec:include}

%%%%%%%%%%%%%%%%%%%%%%%%%%%%%%%%%%%%%%%%
\DescribeMacro{\childdocmain}
To use the package, add the commands
\begin{center}
\begin{tabular}{l}
|\input{childdoc.def}|\\
|\childdocmain{}|\\
\end{tabular}
\end{center}
at the very top of the main \LaTeX{} file,
in particular \emph{before} the |\documentclass| statement!
The argument of |\childdocmain| should be left empty
(but it must be present).

%%%%%%%%%%%%%%%%%%%%%%%%%%%%%%%%%%%%%%%%
\DescribeMacro{\childdocof}
Furthermore, add the commands
\begin{center}
\begin{tabular}{l}
|\input{childdoc.def}|\\
|\childdocof{|\textit{main}|}|\\
\end{tabular}
\end{center}
at the top of every child file \textit{child}
which is included by |\include{|\textit{child}|}|
from within the main file
(or at least for those files to be compiled individually).
The argument \textit{main} must be the filename of the main file.

There are a couple of
considerations in setting up the main and child documents:

%%%%%%%%%%%%%%%%%%%%%%%%%%%%%%%%%%%%%%%%
\paragraph{Restrictions.}

Please note the following restrictions:
\begin{itemize}
\item
|\childdocmain| must be called with one argument \textit{main}
to ensure compatibility with earlier version of the package.
It must either be empty (|\childdocmain{}|)
or precisely match the filename of the main file in which it is specified.
See \secref{sec:detection} for further information.
\item
The filename \textit{main} must be specified without the |.tex| extension.
\item
The filename \textit{main} is case sensitive
(even in case-insensitive file systems)
due to internal string comparison.
\item
The argument \textit{main} should be fully expanded, it cannot be a macro.
\item
Subdirectories and special characters should be avoided in filenames.
\item
The command |\childdocmain{|\textit{main}|}| must be followed by a whitespace.
It should not be followed immediately by another command
or by a comment mark `|%|'.
This is because the \TeX{} parser reads the token immediately following
the argument of |\childdocmain| and puts it
at the beginning of every child section;
however, a white\-space is ignored.
\end{itemize}

%%%%%%%%%%%%%%%%%%%%%%%%%%%%%%%%%%%%%%%%
\paragraph{Content of Main File.}

It is advisable to place all content in the child files included by |\include|.
Any output contained in the main file will appear in all child documents
unless suppressed manually;
it cannot be suppressed automatically by the |\includeonly| directive
and thus should normally be avoided.
A method to include some content in the main file
by means of conditional processing is described in \secref{sec:conditional}.

%%%%%%%%%%%%%%%%%%%%%%%%%%%%%%%%%%%%%%%%
\paragraph{Page Numbering.}

When only a part of the document is compiled,
the appropriate numbering of pages
(as well as other status parameters)
is determined from the |.aux| files.
The latter contain information from previous passes.
However this information needs to propagate through
all intermediate child documents.
Therefore the page numbering in child documents may well
be inconsistent until the complete document is compiled at least once.

A useful (if unconventional) way to always ensure a consistent
page numbering is to restart the numbering in each child document
and denote the pages by `\textit{child}|.|\textit{page}'
where \textit{child} represents the chapter/section number of the child file.
This can be achieved by the command
|\numberwithin{page}{|\textit{child}|}|
of the \textsf{amsmath} package
where \textit{child} can be |chapter| or |section|
depending on the chosen structuring.
Alternatively, one can modify the macro |\thepage| appropriately
and reset the counter |page| at the start of each child file.

%%%%%%%%%%%%%%%%%%%%%%%%%%%%%%%%%%%%%%%%%%%%%%%%%%%%%%%%%%%%%%%%%%%%%%%%%%%%%%%%
\subsection{Conditional Processing}
\label{sec:conditional}

The package provides a mechanism to compile different versions
of a document. To customise the versions further some conditional processing
can come in handy to distinguish which version is being compiled.
The package provides two macros to describe the compilation context:

%%%%%%%%%%%%%%%%%%%%%%%%%%%%%%%%%%%%%%%%
\DescribeMacro{\ifchilddoc}
The conditional |\ifchilddoc| distinguishes between the compilation of
child documents and the main document:
%
\begin{center}
|\ifchilddoc |\textit{child-code}| |[|\||else |\textit{main-code}]| \||fi|
\end{center}

%%%%%%%%%%%%%%%%%%%%%%%%%%%%%%%%%%%%%%%%
\DescribeMacro{\childdocname}
\DescribeMacro{\childdocjob}
The macro |\childdocname| contains the filename (without extension)
of the main or child file being processed.
Note that |\childdocjob| will always contain the name of the main file.

%%%%%%%%%%%%%%%%%%%%%%%%%%%%%%%%%%%%%%%%
\paragraph{Title Page.}

Conditional processing can be used to include a title or banner page
in the main document when proper precautions are taken.
Importantly, the code in the main file should ensure that the page counter
(as well as other status parameters which are stored in the |.aux| files)
takes the same value after the conditional processing.
Otherwise the page numbers may take divergent values
depending on which part is compiled.

For example, a title page could be declared by:
%
\begin{center}
\begin{tabular}{l}
|\ifchilddoc\||else|\\
|\addtocounter{page}{-1}|\\
\textit{code for title page}\\
|\newpage|\\
|\||fi|
\end{tabular}
\end{center}
%
A banner page for the child documents can be generated by:
%
\begin{center}
\begin{tabular}{l}
|\ifchilddoc|\\
|\addtocounter{page}{-1}|\\
\textit{code for banner page}\\
|\newpage|\\
|\||fi|
\end{tabular}
\end{center}
%
Here one could write a message such as:
\begin{center}
|This is the part \childdocname{} of \childdocjob{}.|
\end{center}

%%%%%%%%%%%%%%%%%%%%%%%%%%%%%%%%%%%%%%%%%%%%%%%%%%%%%%%%%%%%%%%%%%%%%%%%%%%%%%%%
\subsection{Flags}
\label{sec:flags}

The package makes it easy to generate different versions
of the main or child documents.
To this end compilation flags can be defined
and assigned different default values.
They will be particularly useful in conjunction
with the forwarding mechanism described in \secref{sec:forward}.

For example, it may be useful to have a flag |\version|
which can be set to |draft| or |final|.
The document source will contain some conditional code
depending on the value of |\version|.
Suppose further, the flag should default to |final| for the main file
and to |draft| for child files
which is a natural assignment for editing the document.
This is achieved by placing the following code
in the preamble of the main document
(below the |\childdocmain| directive):
%
\begin{center}
\begin{tabular}{l}
|\ifchilddoc|\\
|\providecommand{\version}{draft}|\\
|\||else|\\
|\providecommand{\version}{final}|\\
|\||fi|
\end{tabular}
\end{center}
%
The definition by |\providecommand| makes sure
that previous definitions are not overwritten.
Further statements |\providecommand{\version}{...}|
can thus be added before the above code to override it.

For the main file, one might add a line
(between |\childdocmain| and the above block)
%
\begin{center}
|%\ifchilddoc\||else\providecommand{\version}{draft}\||fi|
\end{center}
%
which can be uncommented to produce a draft version.
Likewise one can add a line to the very top of a child file
(above the |\childdocof{|\textit{main}|}| directive)
%
\begin{center}
|%\providecommand{\version}{final}|
\end{center}
%
which can be uncommented to produce the final version of this child document.

%%%%%%%%%%%%%%%%%%%%%%%%%%%%%%%%%%%%%%%%%%%%%%%%%%%%%%%%%%%%%%%%%%%%%%%%%%%%%%%%
\subsection{Forwarding}
\label{sec:forward}

Different versions of the main or child documents
using compilation flags as described in \secref{sec:flags}
can be (permanently) stored in different files
for convenient compilation, viewing and distribution.
To this end, the package defines a command
to pass on compilation to a different file:

%%%%%%%%%%%%%%%%%%%%%%%%%%%%%%%%%%%%%%%%
\DescribeMacro{\childdocforward}
The command |\childdocforward| redirects processing to
another source file:
%
\begin{center}
\begin{tabular}{l}
|\input{childdoc.def}|\\
|\childdocforward[|\textit{main}|]{|\textit{dest}|}|\\
\end{tabular}
\end{center}
%
The argument \textit{dest} is the destination file
(without extension).
It should be the main file or one of the child files.
Note that further \textsf{childdoc} directives
such as |\childdocof| and |\childdocforward|
in the indicated file will be processed in this form.
The optional argument \textit{main}
passes on directly to the main file \textit{main}
while pretending to compile the child \textit{dest}.
This form behaves as if \textit{dest}
issues |\childdocof{|\textit{main}|}| right away,
and no further \textsf{childdoc} directives will be processed.

%%%%%%%%%%%%%%%%%%%%%%%%%%%%%%%%%%%%%%%%
\DescribeMacro{\...prefix}
In the alternative form |\childdocforwardprefix|,
%
\begin{center}
\begin{tabular}{l}
|\input{childdoc.def}|\\
|\childdocforwardprefix[|\textit{main}|]{|\textit{prefix}|}{|\textit{dest}|}|
\end{tabular}
\end{center}
%
the destination file is determined by a pattern
depending on the current file:
To make this work, the current file must be called
`{\textit{prefix}\hspace{0.2em}\textit{suffix}}'
with \textit{prefix} matching precisely the argument.
Processing is then passed on to the file
`{\textit{dest}\hspace{0.2em}\textit{suffix}}'.
Surely, the same effect is achieved by
directly specifying the
argument `{\textit{dest}\hspace{0.2em}\textit{suffix}}'
in the first form.
However, that requires to set up a different file
for each child. With the alternative form of the command
all these files can have exactly the same content
which simplifies setting them up and maintaining them.

For example, the following file |draft.tex|
with a compilation flag |\version| as described in \secref{sec:flags}
compiles the main document as a draft:
%
\begin{center}
\begin{tabular}{l}
|\def\version{draft}|\\
|\input{childdoc.def}|\\
|\childdocforward{|\textit{main}|}|
\end{tabular}
\end{center}
%
Likewise, the following files |final|\textit{nn}|.tex|
compile the final version of the child document
|child|\textit{nn}|.tex|:
%
\begin{center}
\begin{tabular}{l}
|\def\version{final}|\\
|\input{childdoc.def}|\\
|\childdocforwardprefix{final}{child}|
\end{tabular}
\end{center}
%

Note that when several versions of a main file and/or of each child file
are to be generated, it may be convenient to set up a |Makefile| or
shell script to automatise the process.

%%%%%%%%%%%%%%%%%%%%%%%%%%%%%%%%%%%%%%%%%%%%%%%%%%%%%%%%%%%%%%%%%%%%%%%%%%%%%%%%
\subsection{Command Line Processing}
\label{sec:commandline}

The effect of redirection files can also be achieved by invoking
the \LaTeX{} compiler with a more elaborate command line.
Most conveniently this should be done as part
of a shell script or a |Makefile|.

When using \textsf{childdoc} in the main file, the following
command lines effectively perform a redirection
(note that depending on the shell being used,
backslashes may have to be doubled: `|\|' $\to$ `|\\|'):
%
\begin{center}
|... -jobname "|\textit{target}|" |\\|"|[\textit{flags}]%
|\input{childdoc.def}\childdocforward[|\textit{main}|]{|\textit{dest}|}"|
\end{center}
%
Here \textit{target} is the name of the output file,
\textit{main} is the name of the main file
and \textit{dest} is the name of the main or child file to be processed
(all filenames without extensions).
The optional argument \textit{main} can be omitted
if \textit{main} matches \textit{dest}.
Optionally, compilation \textit{flags} can be defined via |\def| commands.
This command line makes the \TeX{} engine believe
it is compiling the file \textit{target}
whose content is specified as the latter parameter.
The provided code then forwards the processing to
\textit{main} or \textit{dest} as described in \secref{sec:forward}.

%%%%%%%%%%%%%%%%%%%%%%%%%%%%%%%%%%%%%%%%%%%%%%%%%%%%%%%%%%%%%%%%%%%%%%%%%%%%%%%%
\subsection{Include by Input}
\label{sec:input}

Including child documents by |\include| has some restrictions by design.
Most notably, the content of a child document always occupies
its own set of pages; pages cannot be shared between child documents.
Usually, this behaviour makes perfect sense
because each child document contain an essential part of the document.
However, in some situations it may be desirable to compose
a document from a collection of parts
without having mandatory page breaks between then.
For this case, the package
provides a mechanism to include parts
by |\input| which can also be processed individually.
However, by construction this mechanism
requires manual handling of the content to be output.

%%%%%%%%%%%%%%%%%%%%%%%%%%%%%%%%%%%%%%%%
\DescribeMacro{\ifchilddocmanual}
The main file should be prepared as usual, see \secref{sec:include}.
However, the document body must make a distinction
between processing of an individual part and of the main document, e.g.:
%
\begin{center}
\begin{tabular}{l}
|\ifchilddocmanual|\\
|\input{\childdocname}|\\
|\||else|\\
\textit{document body with }|\input{|\textit{part}|}|\\
|\||fi|
\end{tabular}
\end{center}
%
The conditional |\ifchilddocmanual| is true whenever
a part to be included by |\input| is being compiled,
and the name of the part is stored in |\childdocname|.

%%%%%%%%%%%%%%%%%%%%%%%%%%%%%%%%%%%%%%%%
\DescribeMacro{\childdocby}
Each part to be included by |\input| should start with:
%
\begin{center}
\begin{tabular}{l}
|\input{childdoc.def}|\\
|\childdocby{|\textit{main}|}|\\
\end{tabular}
\end{center}
%
The directive |\childdocby| is similar to |\childdocof|
described in \secref{sec:include},
but the subsequent selection of content must be done manually.
To that end, both |\ifchilddoc| and |\ifchilddocmanual|
will be true upon processing of a part,
and the name of the part is stored in |\childdocname|.
Note that |\jobname| will be set to the filename of the current part
so that each part receives an individual |.aux| file
that does not interfere with the |.aux| file(s) of the main document.
This behaviour can be altered by the alternative form
|\childdocby[*]{|\textit{main}|}| (with a non-empty optional argument)
which uses the |.aux| file of the main document
by setting |\jobname| to \textit{main}.

%%%%%%%%%%%%%%%%%%%%%%%%%%%%%%%%%%%%%%%%%%%%%%%%%%%%%%%%%%%%%%%%%%%%%%%%%%%%%%%%
\subsection{Driver Development}
\label{sec:driver}

The \textsf{childdoc} mechanism can also be use for the development
of definition files such as \LaTeX{} styles or classes.
This case differs from the above setup with multiple parts
included by |\include| in that no |\includeonly| should be invoked.
This can be achieved by starting the include file
(before |\ProvidesPackage|) with:
%
\begin{center}
\begin{tabular}{l}
|\input{childdoc.def}|\\
|\childdocforward{|\textit{main}|}|\\
\end{tabular}
\end{center}
%
or alternatively with:
%
\begin{center}
\begin{tabular}{l}
|\input{childdoc.def}|\\
|\childdocby{|\textit{main}|}|\\
\end{tabular}
\end{center}
%
Both forms have slightly different effects as described above.
The main file is prepared as usual, see \secref{sec:include}.

%%%%%%%%%%%%%%%%%%%%%%%%%%%%%%%%%%%%%%%%%%%%%%%%%%%%%%%%%%%%%%%%%%%%%%%%%%%%%%%%
\subsection{Legacy Detection}
\label{sec:detection}

The directive |\childdocmain| in the main file can detect
whether the complete document or merely a child is to be compiled
even without using the directive |\childdocof|.
This method is deprecated because it is less robust
and there is no compelling reason to use it;
it is merely provided for backward compatibility
and it may be removed in future versions.

If the detection mechanism is to be used,
it is mandatory to correctly specify
the filename of the main file as the argument of |\childdocmain|:
%
\begin{center}
\begin{tabular}{l}
|\input{childdoc.def}|\\
|\childdocmain{|\textit{main}|}|\\
\end{tabular}
\end{center}
%
If |\jobname| does not match the argument \textit{main} of |\childdocmain|,
it is assumed that |\jobname| points to the child file to be compiled.
When using |\childdocmain| with the main file specified as argument,
it suffices to start a child file
with just |\input{|\textit{main}|}|
without loading of the package and using |\childdocof|.
If instead all processing is done
with the appropriate \textsf{childdoc} directives,
the argument of \textit{main} of |\childdocmain| can be empty.

An alternative version of the command line processing described
in \secref{sec:commandline} using the detection mechanism reads:
%
\begin{center}
|... -jobname "|\textit{target}|" "|[\textit{flags}]%
[|\def\jobname{|\textit{dest}|}|]|\input{|\textit{main}|}"|
\end{center}

%%%%%%%%%%%%%%%%%%%%%%%%%%%%%%%%%%%%%%%%%%%%%%%%%%%%%%%%%%%%%%%%%%%%%%%%%%%%%%%%
\subsection{Manual Code}
\label{sec:manual}

In case one cannot be certain whether the definitions file |childdoc.def|
is installed on the target \TeX{} distribution
and one prefers not to ship it,
it is conceivable to paste a few relevant commands into the sources.

To that end, drop all statements |\input{childdoc.def}|
and perform the replacements as outlined below.
Instead of |\childdocmain{|\textit{main}|}| add the following code
to the top of the main file:
%
\begin{center}
\begin{tabular}{l}
|\||ifdefined\childdocname\endinput\||fi\newif\ifchilddoc|\\
|\edef\childdocname{\scantokens\expandafter{\jobname\noexpand}}|\\
|\def\childdocmain{|\textit{main}|}\||ifx\childdocmain\childdocname\||else|\\
|\childdoctrue\includeonly{\childdocname}\let\jobname\childdocmain\||fi|\\
\end{tabular}
\end{center}
%
Instead of |\childdocof{|\textit{main}|}| just include the main file
at the top of each child file:
%
\begin{center}
|\input{|\textit{main}|}|
\end{center}
%
A simple redirection |\childdocforward{|\textit{dest}|}| is achieved by:
%
\begin{center}
|\def\jobname{|\textit{dest}|}\input{\jobname}|
\end{center}
%
The redirection with prefix
|\childdocforwardprefix[|\textit{prefix}|]{|\textit{dest}|}|
is accomplished by:
%
\begin{center}
\begin{tabular}{l}
|{\edef\jobname{\scantokens\expandafter{\jobname\noexpand}}|\\
|\def\redirectjob |\textit{prefix}|#1~~~{\gdef\jobname{|\textit{dest}|#1}}|\\
|\expandafter\redirectjob\jobname~~~}\input{\jobname}|
\end{tabular}
\end{center}

In an alternative approach,
child documents can be compiled by a specific command line
without additional code or specific definitions:
%
\begin{center}
|... -jobname "|\textit{target}|" "|[\textit{flags}]%
|\includeonly{|\textit{dest}|}\input{|\textit{main}|}"|
\end{center}
%

%%%%%%%%%%%%%%%%%%%%%%%%%%%%%%%%%%%%%%%%%%%%%%%%%%%%%%%%%%%%%%%%%%%%%%%%%%%%%%%%
%%%%%%%%%%%%%%%%%%%%%%%%%%%%%%%%%%%%%%%%%%%%%%%%%%%%%%%%%%%%%%%%%%%%%%%%%%%%%%%%
\section{Information}

%%%%%%%%%%%%%%%%%%%%%%%%%%%%%%%%%%%%%%%%%%%%%%%%%%%%%%%%%%%%%%%%%%%%%%%%%%%%%%%%
\subsection{Copyright}

Copyright \copyright{} 2017--2018 Niklas Beisert

This work may be distributed and/or modified under the
conditions of the \LaTeX{} Project Public License, either version 1.3
of this license or (at your option) any later version.
The latest version of this license is in
  \url{http://www.latex-project.org/lppl.txt}
and version 1.3 or later is part of all distributions of \LaTeX{}
version 2005/12/01 or later.

This work has the LPPL maintenance status `maintained'.

The Current Maintainer of this work is Niklas Beisert.

This work consists of the files |README.txt|, |childdoc.ins| and |childdoc.dtx|
as well as the derived files |childdoc.def|, |cdocsamp.tex|
with |cdocsch1.tex|, |cdocsch2.tex|, |cdocspt3.tex|, |cdocspt4.tex|,
|cdocsdrf.tex|, |cdocsfn1.tex|, |cdocsfn2.tex|
as well as |childdoc.pdf|.

%%%%%%%%%%%%%%%%%%%%%%%%%%%%%%%%%%%%%%%%%%%%%%%%%%%%%%%%%%%%%%%%%%%%%%%%%%%%%%%%
\subsection{Files and Installation}

The package consists of the files:
%
\begin{center}
\begin{tabular}{ll}
    |README.txt|   & readme file \\
    |childdoc.ins| & installation file \\
    |childdoc.dtx| & source file \\
    |childdoc.def| & definition file \\
    |cdocsamp.tex| & sample main file \\
    |cdocsch1.tex| & sample include file \\
    |cdocsch2.tex| & sample include file \\
    |cdocspt3.tex| & sample part file \\
    |cdocspt4.tex| & sample part file \\
    |cdocsdrf.tex| & sample redirection file \\
    |cdocsfn1.tex| & sample redirection file \\
    |cdocsfn2.tex| & sample redirection file \\
    |childdoc.pdf| & manual
\end{tabular}
\end{center}
%
The distribution consists of the files
|README.txt|, |childdoc.ins| and |childdoc.dtx|.
%
\begin{itemize}
\item
Run (pdf)\LaTeX{} on |childdoc.dtx|
to compile the manual |childdoc.pdf| (this file).
\item
Run \LaTeX{} on |childdoc.ins| to create the definitions file |childdoc.def|
and the sample |cdocsamp.tex| with include files
|cdocsch1.tex|, |cdocsch2.tex|, |cdocspt3.tex|, |cdocspt4.tex|,
|cdocsdrf.tex|, |cdocsfn1.tex|, |cdocsfn2.tex|.
Then copy the file |childdoc.def| to an appropriate directory of your \LaTeX{}
distribution, e.g.\ \textit{texmf-root}|/tex/latex/childdoc|.
\end{itemize}

%%%%%%%%%%%%%%%%%%%%%%%%%%%%%%%%%%%%%%%%%%%%%%%%%%%%%%%%%%%%%%%%%%%%%%%%%%%%%%%%
\subsection{Related CTAN Packages}

There are several other packages which offer a similar functionality:
%
\begin{itemize}
\item
The packages
\href{http://ctan.org/pkg/docmute}{\textsf{docmute}},
\href{http://ctan.org/pkg/includex}{\textsf{includex}} and
\href{http://ctan.org/pkg/standalone}{\textsf{standalone}}
provide commands to include only the document body of
a child file thus allowing both files to be compiled individually.
\item
The packages \href{http://ctan.org/pkg/subdocs}{\textsf{subdocs}}
and \href{http://ctan.org/pkg/subfiles}{\textsf{subfiles}}
provide structures in which the main and child documents can be
encapsulated and allowing them to be compiled individually.
The inclusion mechanism is different from the conventional |\include|.
\item
The package \href{http://ctan.org/pkg/combine}{\textsf{combine}}
is an elaborate solution to combine several documents into one.
\end{itemize}
%
See also the CTAN topic \href{http://ctan.org/topic/subdocs}{\textsf{subdocs}}
for further related packages.
The present package differs from the above solutions in that
a document structure constructed with the conventional |\include| mechanism
just needs two extra commands at the top of every file
such that all constituent files can be compiled individually.

%%%%%%%%%%%%%%%%%%%%%%%%%%%%%%%%%%%%%%%%%%%%%%%%%%%%%%%%%%%%%%%%%%%%%%%%%%%%%%%%
%\subsection{Feature Suggestions}
%
%The following is a list of features which may be useful for future
%versions of this package:
%%
%\begin{itemize}
%\item
%\ldots
%\end{itemize}

%%%%%%%%%%%%%%%%%%%%%%%%%%%%%%%%%%%%%%%%%%%%%%%%%%%%%%%%%%%%%%%%%%%%%%%%%%%%%%%%
\subsection{Revision History}

%%%%%%%%%%%%%%%%%%%%%%%%%%%%%%%%%%%%%%%%
\paragraph{v2.0:} 2018/12/30

\begin{itemize}
\item
immediate forward processing
\item
added |\childdocby| mechanism
\item
manual restructured
\end{itemize}

%%%%%%%%%%%%%%%%%%%%%%%%%%%%%%%%%%%%%%%%
\paragraph{v1.6:} 2018/01/17

\begin{itemize}
\item
application for development of include files
\item
corrections to manual
\end{itemize}

%%%%%%%%%%%%%%%%%%%%%%%%%%%%%%%%%%%%%%%%
\paragraph{v1.5:} 2017/05/21

\begin{itemize}
\item
more complete structuring introduced
\item
|\childdocof| introduced
\item
|\childdoc| renamed to |\childdocmain|
\item
|\childredirect| renamed to |\childdocforward| and |\childdocforwardprefix|
and functionality expanded
\end{itemize}

%%%%%%%%%%%%%%%%%%%%%%%%%%%%%%%%%%%%%%%%
\paragraph{v1.0:} 2017/04/27

\begin{itemize}
\item
manual and install package
\item
first version published on CTAN
\end{itemize}

%%%%%%%%%%%%%%%%%%%%%%%%%%%%%%%%%%%%%%%%
\paragraph{v0.6:} 2017/04/26

\begin{itemize}
\item
redirection mechanism added
\end{itemize}

%%%%%%%%%%%%%%%%%%%%%%%%%%%%%%%%%%%%%%%%
\paragraph{v0.5:} 2017/04/26

\begin{itemize}
\item
functionality in definition file
\end{itemize}


%%%%%%%%%%%%%%%%%%%%%%%%%%%%%%%%%%%%%%%%%%%%%%%%%%%%%%%%%%%%%%%%%%%%%%%%%%%%%%%%
%%%%%%%%%%%%%%%%%%%%%%%%%%%%%%%%%%%%%%%%%%%%%%%%%%%%%%%%%%%%%%%%%%%%%%%%%%%%%%%%
%%%%%%%%%%%%%%%%%%%%%%%%%%%%%%%%%%%%%%%%%%%%%%%%%%%%%%%%%%%%%%%%%%%%%%%%%%%%%%%%
\appendix

\settowidth\MacroIndent{\rmfamily\scriptsize 000\ }

 \DocInput{childdoc.dtx}

\end{document}
%</driver>
% \fi
%
% %%%%%%%%%%%%%%%%%%%%%%%%%%%%%%%%%%%%%%%%%%%%%%%%%%%%%%%%%%%%%%%%%%%%%%%%%%%%%%
% %%%%%%%%%%%%%%%%%%%%%%%%%%%%%%%%%%%%%%%%%%%%%%%%%%%%%%%%%%%%%%%%%%%%%%%%%%%%%%
% \section{Sample}
%\iffalse
%<*samplemain>
%\fi
%
% The following presents a sample document
% with two chapters, two parts, a title page,
% a compile flag as well as three forwarding files to set the flag.
% It consists of eight |.tex| files:
% \begin{center}
% \begin{tabular}{ll}
% |cdocsamp.tex|&main file\\
% |cdocsch1.tex|&include file for chapter 1\\
% |cdocsch2.tex|&include file for chapter 2\\
% |cdocspt3.tex|&include file for part 3\\
% |cdocspt4.tex|&include file for part 4\\
% |cdocsdrf.tex|&forwarding file for main file in draft mode\\
% |cdocsfi1.tex|&forwarding file for final version of chapter 1\\
% |cdocsfi2.tex|&forwarding file for final version of chapter 2\\
% \end{tabular}
% \end{center}
% Each of the eight files can be compiled directly by the \LaTeX{} compiler.
%
% %%%%%%%%%%%%%%%%%%%%%%%%%%%%%%%%%%%%%%
% \paragraph{Main File.}
%
% The main file is called |cdocsamp.tex|.
%
% Load the \textsf{childdoc} definitions and
% declare the filename for the main document:
%    \begin{macrocode}
\input{childdoc.def}
\childdocmain{}
%    \end{macrocode}

% Optional override for |\version| flag:
%    \begin{macrocode}
%%\ifchilddoc\else\providecommand{\version}{draft}\fi
%    \end{macrocode}

% Define the default values for the |\version| flag
% (|final| for the main file and |draft| for childs):
%    \begin{macrocode}
\ifchilddoc
\providecommand{\version}{draft}
\else
\providecommand{\version}{final}
\fi
%    \end{macrocode}

% Load the standard document class:
%    \begin{macrocode}
\documentclass[12pt]{article}
%    \end{macrocode}

% Start the document body:
%    \begin{macrocode}
\begin{document}
%    \end{macrocode}

% Declare a title page.
% Print title, part of document being processed and version flag:
%    \begin{macrocode}
\addtocounter{page}{-1}
\begin{center}
{\LARGE\bfseries{}childdoc example\par}
\vspace{1cm}
\ifchilddoc
\ifchilddocmanual part\else chapter\fi:
`\childdocname' of `\childdocjob'\par
\else
main document: `\childdocjob'\par
\fi
version: \version\par
\end{center}
\newpage
%    \end{macrocode}

% Manually include selected file,
% otherwise process as usual:
%    \begin{macrocode}
\ifchilddocmanual
\section*{part `\childdocname'}
\input{\childdocname}
\else
%    \end{macrocode}

% Include the two chapters:
%    \begin{macrocode}
\include{cdocsch1}
\include{cdocsch2}
%    \end{macrocode}

% Include the two parts unless only chapters should be displayed:
%    \begin{macrocode}
\ifchilddoc\else
\section{part three}
\input{cdocspt3}
\section{part four}
\input{cdocspt4}
\fi
%    \end{macrocode}

% Process as usual until here:
%    \begin{macrocode}
\fi
%    \end{macrocode}

% End of document body:
%    \begin{macrocode}
\end{document}
%    \end{macrocode}
%\iffalse
%</samplemain>
%\fi
%
% %%%%%%%%%%%%%%%%%%%%%%%%%%%%%%%%%%%%%%
% \paragraph{Chapter Include Files.}
%
% The include files are called |cdocsch1.tex| and |cdocsch2.tex|.
%
%\iffalse
%<*samplechap1|samplechap2>
%\fi

% Optional override for |\version| flag:
%    \begin{macrocode}
%%\providecommand{\version}{final}
%    \end{macrocode}

% Include the main document:
%    \begin{macrocode}
\input{childdoc.def}
\childdocof{cdocsamp}
%    \end{macrocode}

%\iffalse
%</samplechap1|samplechap2>
%\fi
%
%\iffalse
%<*samplechap1>
%\fi
% Some text for chapter 1:
%    \begin{macrocode}
\section{one}
some text in chapter one
%    \end{macrocode}

%\iffalse
%</samplechap1>
%\fi
% Some text for chapter 2:
%\iffalse
%<*samplechap2>
%\fi
%    \begin{macrocode}
\section{two}
more text in chapter two
%    \end{macrocode}

%\iffalse
%</samplechap2>
%\fi
%
% %%%%%%%%%%%%%%%%%%%%%%%%%%%%%%%%%%%%%%
% \paragraph{Part Include Files.}
%
% The include files are called |cdocspt3.tex| and |cdocspt4.tex|.
%
%\iffalse
%<*samplepart3|samplepart4>
%\fi

% Optional override for |\version| flag:
%    \begin{macrocode}
%%\providecommand{\version}{final}
%    \end{macrocode}

% Include the main document:
%    \begin{macrocode}
\input{childdoc.def}
\childdocby{cdocsamp}
%    \end{macrocode}

%\iffalse
%</samplepart3|samplepart4>
%\fi
%
%\iffalse
%<*samplepart3>
%\fi
% Some text for part 3:
%    \begin{macrocode}
some text in part three
%    \end{macrocode}

%\iffalse
%</samplepart3>
%\fi
% Some text for part 4:
%\iffalse
%<*samplepart4>
%\fi
%    \begin{macrocode}
more text in part four
%    \end{macrocode}

%\iffalse
%</samplepart4>
%\fi
%
% %%%%%%%%%%%%%%%%%%%%%%%%%%%%%%%%%%%%%%
% \paragraph{Forwarding for a Complete Draft.}
%
% The following forwarding file |cdocsdrf.tex|
% compiles the main document in draft mode:
%\iffalse
%<*sampledraft>
%\fi
%    \begin{macrocode}
\def\version{draft}
\input{childdoc.def}
\childdocforward{cdocsamp}
%    \end{macrocode}

%\iffalse
%</sampledraft>
%\fi
%
% %%%%%%%%%%%%%%%%%%%%%%%%%%%%%%%%%%%%%%
% \paragraph{Forwarding for Final Version of the Chapters.}
%
% The following forwarding files |cdocsfn1.tex| and |cdocsfn2.tex|
% (with identical content)
% compile the final versions of the child documents
% |cdocsch1.tex| and |cdocsch2.tex|, respectively:
%\iffalse
%<*samplefinal>
%\fi
%    \begin{macrocode}
\def\version{final}
\input{childdoc.def}
\childdocforwardprefix[cdocsamp]{cdocsfn}{cdocsch}
%    \end{macrocode}

%\iffalse
%</samplefinal>
%\fi
%
% %%%%%%%%%%%%%%%%%%%%%%%%%%%%%%%%%%%%%%
% \paragraph{Command Line Processing.}
%
% The following three command lines generate the output files
% |cdocscld|, |cdocscl1| and |cdocscl2|
% which should be identical to
% |cdocsdrf|, |cdocsch1| and |cdocsfn2|, respectively:
% \begin{center}
% \begin{tabular}{l}
% |latex -jobname cdocscld \|\\
% |  "\def\version{draft}\input{childdoc.def}\childdocforward{cdocsamp}"|\\
% |latex -jobname cdocscl1 \|\\
% |  "\input{childdoc.def}\childdocforward[cdocsamp]{cdocsch1}"|\\
% |latex -jobname cdocscl2 \|\\
% |  "\def\version{final}\input{childdoc.def}\childdocforward{cdocsch2}"|
% \end{tabular}
% \end{center}
% Note that the trailing backslash on each first line
% merely continues the input to the second line
% (for convenient cut ant paste).
% Furthermore, the command |latex| can be replaced by any
% of its alternative versions such as |pdflatex|.
%
% %%%%%%%%%%%%%%%%%%%%%%%%%%%%%%%%%%%%%%%%%%%%%%%%%%%%%%%%%%%%%%%%%%%%%%%%%%%%%%
% %%%%%%%%%%%%%%%%%%%%%%%%%%%%%%%%%%%%%%%%%%%%%%%%%%%%%%%%%%%%%%%%%%%%%%%%%%%%%%
% \section{Implementation}
%\iffalse
%<*package>
%\fi
%
% This section describes the definitions file |childdoc.def|.

% The definitions cannot be loaded using |\usepackage| or |\RequirePackage|
% which has a mechanism to prevent loading a style file more than once.
% When loading the definitions by means of |\input|
% multiple instances have to be prevented manually:
%\iffalse
%This code needs to be before the `\ProvidesFile' directive
%which is defined at the beginning of this file.
%Therefore it is also placed there and commented out here.
%</package>
%<*discard>
%\fi
%    \begin{macrocode}
\ifdefined\childdocmain\endinput\fi
%    \end{macrocode}
%\iffalse
%</discard>
%<*package>
%\fi
%
% \macro{\ifchilddoc}
% \macro{\ifchilddocmanual}
% The conditional |\ifchilddoc| tells whether a
% child (true) or main (false) document is being compiled.
% The conditional |\ifchilddocmanual| tells whether
% the |\includeonly| mechanism is used (false) or
% the selection of child files must be performed manually (true).
% The definitions initialise to false:
%    \begin{macrocode}
\newif\ifchilddoc
\newif\ifchilddocmanual
%    \end{macrocode}

% \macro{\childdocname}
% \macro{\childdocjob}
% The macro |\childdocname| stores the name of the main document
% to be compiled. The macro |\childdocjob| stores the name of
% the document on which the \LaTeX{} compiler was originally invoked.
% The content of |\jobname| cannot be compared
% to filenames specified in the source due to different catcodes.
% The following code rescans |\jobname|, stores the result
% in |\childdocname| and saves a copy in |\childdocjob|:
%    \begin{macrocode}
\edef\childdocname{\scantokens\expandafter{\jobname\noexpand}}
\let\childdocjob\childdocname
%    \end{macrocode}

% \macro{\childdocdisable}
% The macro |\childdocdisable| prevents the main file
% from being processed more than once.
% At this stage, the main document command |\childdocmain|
% is assumed to be called once again where it should do nothing.
% Any subsequent call to it should prevent
% a secondary processing of the main document
% It overwrites the forwarding commands
% |\childdocof| and |\childdocforward|
% with empty macros to prevent further inclusions of the main document:
%    \begin{macrocode}
\newcommand{\childdocdisable}
{
  \renewcommand{\childdocmain}[1]{\renewcommand{\childdocmain}[1]{\endinput}}
  \renewcommand{\childdocof}[1]{}
  \renewcommand{\childdocby}[2][]{}
  \renewcommand{\childdocforward}[2][]{}
  \renewcommand{\childdocdisable}{}
}
%    \end{macrocode}

% \macro{\childdocmain}
% The macro |\childdocmain| is to be called at the top of the main file
% with nothing or the main filename (without extension) as argument.
% First, it breaks loops.
% If the argument is not empty and does not match |\childdocname|
% (which is set by the first inclusion of |childdoc.def|),
% |\ifchilddoc| is set to true, |\includeonly| is applied to the child file
% and |\jobname| is set to the main file
% (for proper handling of |.aux| files):
%    \begin{macrocode}
\newcommand{\childdocmain}[1]
{
  \childdocdisable\childdocmain{}
  \if?#1?\else
    \begingroup
      \def\childdoctmp{#1}
      \ifx\childdoctmp\childdocname
        \def\childdoctmp{}
      \else
        \def\childdoctmp
        {
          \childdoctrue
          \includeonly{\childdocname}
          \def\childdocjob{#1}
          \def\jobname{#1}
        }
      \fi
      \expandafter
    \endgroup
    \childdoctmp
  \fi
}
%    \end{macrocode}

% \macro{\childdocof}
% The command |\childdocof| redirects
% compilation to the main file |#1|.
%    \begin{macrocode}
\newcommand{\childdocof}[1]
{
  \childdocdisable
  \childdoctrue
  \includeonly{\childdocname}
  \def\jobname{#1}
  \def\childdocjob{#1}
  \input{#1}
}
%    \end{macrocode}

% \macro{\childdocby}
% The command |\childdocby| ....
%    \begin{macrocode}
\newcommand{\childdocby}[2][]
{
  \childdocdisable
  \childdoctrue
  \childdocmanualtrue
  \if?#1?\else
    \def\jobname{#2}
  \fi
  \def\childdocjob{#2}
  \input{#2}
  \endinput
}
%    \end{macrocode}

% \macro{\childdocforward}
% The command |\childdocforward| redirects
% compilation to the main file or
% (if the optional argument is given) a child file.
% Parameters are set as if the main file
% or a child file starting with |\childdocof| was compiled.
% Then compilation is handed over to the main file:
%    \begin{macrocode}
\newcommand{\childdocforward}[2][]
{
  \begingroup
    \if?#1?
      \def\childdoctmp
      {
        \def\childdocname{#2}
        \def\childdocjob{#2}
        \def\jobname{#2}
        \input{#2}
        \endinput
      }
    \else
      \def\childdoctmp
      {
        \childdocdisable
        \def\childdocname{#2}
        \childdoctrue
        \includeonly{#2}
        \def\childdocjob{#1}
        \def\jobname{#1}
        \input{#1}
        \endinput
      }
    \fi
    \expandafter
  \endgroup
  \childdoctmp
}
%    \end{macrocode}

% \macro{\childdocforwardprefix}
% The command |\childdocforwardprefix| redirects
% compilation to the main or a child file by means of a pattern.
% The prefix |#1| in the current filename is replaced by |#2|
% and the suffix of the current filename is kept
% (it is assumed that the filename does not contain the substring `|~~~|'
% which is used as a delimiter).
% Compilation is handed over to the new file by |\childdocforward|:
%    \begin{macrocode}
\newcommand{\childdocforwardprefix}[3][]
{
  \begingroup
    \def\childdocextract #2##1~~~{\def\childdoctmp{\childdocforward[#1]{#3##1}}}
    \expandafter\childdocextract\childdocname~~~
    \expandafter
  \endgroup
  \childdoctmp
}
%    \end{macrocode}

% \macro{\childdoc}
% The deprecated macro |\childdoc| is a legacy version of |\childdocmain|:
%    \begin{macrocode}
\newcommand{\childdoc}{\childdocmain}
%    \end{macrocode}

% \macro{\childdocredirect}
% The deprecated macro |\childdocredirect| is a legacy version
% of |\childdocforward| and |\childdocforwardprefix|:
%    \begin{macrocode}
\newcommand{\childdocredirect}[2][]
{
  \begingroup
    \if?#1?
      \def\childdoctmp{\childdocforward{#2}}
    \else
      \def\childdoctmp{\childdocforwardprefix{#1}{#2}}
    \fi
    \expandafter
  \endgroup
  \childdoctmp
}
%    \end{macrocode}

%\iffalse
%</package>
%\fi
%
\endinput

\childdocmain{}
%    \end{macrocode}

% Optional override for |\version| flag:
%    \begin{macrocode}
%%\ifchilddoc\else\providecommand{\version}{draft}\fi
%    \end{macrocode}

% Define the default values for the |\version| flag
% (|final| for the main file and |draft| for childs):
%    \begin{macrocode}
\ifchilddoc
\providecommand{\version}{draft}
\else
\providecommand{\version}{final}
\fi
%    \end{macrocode}

% Load the standard document class:
%    \begin{macrocode}
\documentclass[12pt]{article}
%    \end{macrocode}

% Start the document body:
%    \begin{macrocode}
\begin{document}
%    \end{macrocode}

% Declare a title page.
% Print title, part of document being processed and version flag:
%    \begin{macrocode}
\addtocounter{page}{-1}
\begin{center}
{\LARGE\bfseries{}childdoc example\par}
\vspace{1cm}
\ifchilddoc
\ifchilddocmanual part\else chapter\fi:
`\childdocname' of `\childdocjob'\par
\else
main document: `\childdocjob'\par
\fi
version: \version\par
\end{center}
\newpage
%    \end{macrocode}

% Manually include selected file,
% otherwise process as usual:
%    \begin{macrocode}
\ifchilddocmanual
\section*{part `\childdocname'}
\input{\childdocname}
\else
%    \end{macrocode}

% Include the two chapters:
%    \begin{macrocode}
\include{cdocsch1}
\include{cdocsch2}
%    \end{macrocode}

% Include the two parts unless only chapters should be displayed:
%    \begin{macrocode}
\ifchilddoc\else
\section{part three}
\input{cdocspt3}
\section{part four}
\input{cdocspt4}
\fi
%    \end{macrocode}

% Process as usual until here:
%    \begin{macrocode}
\fi
%    \end{macrocode}

% End of document body:
%    \begin{macrocode}
\end{document}
%    \end{macrocode}
%\iffalse
%</samplemain>
%\fi
%
% %%%%%%%%%%%%%%%%%%%%%%%%%%%%%%%%%%%%%%
% \paragraph{Chapter Include Files.}
%
% The include files are called |cdocsch1.tex| and |cdocsch2.tex|.
%
%\iffalse
%<*samplechap1|samplechap2>
%\fi

% Optional override for |\version| flag:
%    \begin{macrocode}
%%\providecommand{\version}{final}
%    \end{macrocode}

% Include the main document:
%    \begin{macrocode}
% \iffalse
%
% childdoc.dtx Copyright (C) 2017-2018 Niklas Beisert
%
% This work may be distributed and/or modified under the
% conditions of the LaTeX Project Public License, either version 1.3
% of this license or (at your option) any later version.
% The latest version of this license is in
%   http://www.latex-project.org/lppl.txt
% and version 1.3 or later is part of all distributions of LaTeX
% version 2005/12/01 or later.
%
% This work has the LPPL maintenance status `maintained'.
%
% The Current Maintainer of this work is Niklas Beisert.
%
% This work consists of the files childdoc.dtx and childdoc.ins
% and the derived files childdoc.def and cdocsamp.tex with
% cdocsch1.tex, cdocsch2.tex, cdocsdrf.tex, cdocsfn1.tex, cdocsfn2.tex.
%
%<package>\ifdefined\childdocmain\endinput\fi
%<package>\ProvidesFile{childdoc.def}[2018/12/30 v2.0 child document driver]
%<samplemain>\ProvidesFile{cdocsamp.tex}[2018/12/30 v2.0 sample for childdoc]
%<*driver>
%\ProvidesFile{childdoc.drv}[2018/12/30 v2.0 childdoc reference manual file]
\PassOptionsToClass{10pt,a4paper}{article}
\documentclass{ltxdoc}

\usepackage[margin=35mm]{geometry}
\usepackage{hyperref}
\usepackage{hyperxmp}
\usepackage[usenames]{color}

\hypersetup{colorlinks=true}
\hypersetup{pdfstartview=FitH}
\hypersetup{pdfpagemode=UseNone}
\hypersetup{pdfsource={}}
\hypersetup{pdflang={en-UK}}
\hypersetup{pdfcopyright={Copyright 2017-2018 Niklas Beisert.
  This work may be distributed and/or modified under the
  conditions of the LaTeX Project Public License, either version 1.3
  of this license or (at your option) any later version.}}
\hypersetup{pdflicenseurl={http://www.latex-project.org/lppl.txt}}
\hypersetup{pdfcontactaddress={ETH Zurich, ITP, HIT K,
  Wolfgang-Pauli-Strasse 27}}
\hypersetup{pdfcontactpostcode={8093}}
\hypersetup{pdfcontactcity={Zurich}}
\hypersetup{pdfcontactcountry={Switzerland}}
\hypersetup{pdfcontactemail={nbeisert@itp.phys.ethz.ch}}
\hypersetup{pdfcontacturl={http://people.phys.ethz.ch/\xmptilde nbeisert/}}

\newcommand{\secref}[1]{\hyperref[#1]{section \ref*{#1}}}

\parskip1ex
\parindent0pt
\let\olditemize\itemize
\def\itemize{\olditemize\parskip0pt}

\begin{document}

\title{The \textsf{childdoc} Package}
\hypersetup{pdftitle={The childdoc Package}}
\author{Niklas Beisert\\[2ex]
  Institut f\"ur Theoretische Physik\\
  Eidgen\"ossische Technische Hochschule Z\"urich\\
  Wolfgang-Pauli-Strasse 27, 8093 Z\"urich, Switzerland\\[1ex]
  \href{mailto:nbeisert@itp.phys.ethz.ch}
  {\texttt{nbeisert@itp.phys.ethz.ch}}}
\hypersetup{pdfauthor={Niklas Beisert}}
\hypersetup{pdfsubject={Manual for the LaTeX2e Package childdoc}}
\date{30 December 2018, \textsf{v2.0}}
\maketitle

\begin{abstract}\noindent
\textsf{childdoc} is a \LaTeXe{} package
that enables the direct compilation
of document sections included by |\include|
to individual files.
\end{abstract}

\begingroup
\parskip0ex
\tableofcontents
\endgroup

%%%%%%%%%%%%%%%%%%%%%%%%%%%%%%%%%%%%%%%%%%%%%%%%%%%%%%%%%%%%%%%%%%%%%%%%%%%%%%%%
%%%%%%%%%%%%%%%%%%%%%%%%%%%%%%%%%%%%%%%%%%%%%%%%%%%%%%%%%%%%%%%%%%%%%%%%%%%%%%%%
\section{Introduction}

\LaTeX{} provides a mechanism to structure a large document (such as a book)
into a main file and several child files (containing the chapters)
using the |\include| command.
This mechanism is beneficial for documents
which span hundreds of pages in order to
make the source file(s) more manageable.
Moreover, compilation can be restricted to
selected child files by means of the |\includeonly| command.
The latter feature can be used to reduce the compilation time while editing
(this was significantly more useful in the earlier days of \LaTeX{})
or to generate a smaller document which is easier to navigate.
Another application of |\includeonly| is to generate
documents consisting of selected parts of the complete document.

However, there are a few drawbacks of the plain |\include| mechanism:
\begin{itemize}
\item
The child files cannot be compiled on their own,
they can only be compiled via the main file.
A naive editing environment
(such as a text editor with an option
to have the current file processed by \LaTeX)
may require one to switch to the main file before compiling;
attempting to compile the child file produces errors.
\item
The main file must be modified (each time)
to adjust the |\includeonly| command
to the present needs. This easily leaves the main file in a messy state.
\item
The generated document will always carry the filename
of the main document. This is inconvenient if
several child files are to be compiled and
to be kept for distribution.
\end{itemize}

The present package provides a simple interface
to make child files individually compilable by \LaTeX{}.
Compiling a child file then has the same effect as compiling
the main file with an |\includeonly| command
to select the appropriate child.
Moreover the generated document will carry the name of the child
rather than the main file.
This resolves all three above issues.

This feature is meant to make the editing of books,
thesis documents and lecture notes somewhat more convenient.
However, the package can also be used efficiently for
composing a series of documents (such as exercise sheets)
which are typically distributed individually.
It then assists the author in generating the individual documents
(potentially in different versions)
as well as a document containing the collected series.
Another application is in developing style files
or other kinds of included material
where compilation of the style file could redirect
to a sample or test file.

%%%%%%%%%%%%%%%%%%%%%%%%%%%%%%%%%%%%%%%%%%%%%%%%%%%%%%%%%%%%%%%%%%%%%%%%%%%%%%%%
%%%%%%%%%%%%%%%%%%%%%%%%%%%%%%%%%%%%%%%%%%%%%%%%%%%%%%%%%%%%%%%%%%%%%%%%%%%%%%%%
\section{Usage}

First of all, the package \textsf{childdoc} is \emph{not} a standard
\LaTeXe{} |.sty| style file! Therefore it needs to be invoked in
a non-standard way.

%%%%%%%%%%%%%%%%%%%%%%%%%%%%%%%%%%%%%%%%%%%%%%%%%%%%%%%%%%%%%%%%%%%%%%%%%%%%%%%%
\subsection{Included Files}
\label{sec:include}

%%%%%%%%%%%%%%%%%%%%%%%%%%%%%%%%%%%%%%%%
\DescribeMacro{\childdocmain}
To use the package, add the commands
\begin{center}
\begin{tabular}{l}
|\input{childdoc.def}|\\
|\childdocmain{}|\\
\end{tabular}
\end{center}
at the very top of the main \LaTeX{} file,
in particular \emph{before} the |\documentclass| statement!
The argument of |\childdocmain| should be left empty
(but it must be present).

%%%%%%%%%%%%%%%%%%%%%%%%%%%%%%%%%%%%%%%%
\DescribeMacro{\childdocof}
Furthermore, add the commands
\begin{center}
\begin{tabular}{l}
|\input{childdoc.def}|\\
|\childdocof{|\textit{main}|}|\\
\end{tabular}
\end{center}
at the top of every child file \textit{child}
which is included by |\include{|\textit{child}|}|
from within the main file
(or at least for those files to be compiled individually).
The argument \textit{main} must be the filename of the main file.

There are a couple of
considerations in setting up the main and child documents:

%%%%%%%%%%%%%%%%%%%%%%%%%%%%%%%%%%%%%%%%
\paragraph{Restrictions.}

Please note the following restrictions:
\begin{itemize}
\item
|\childdocmain| must be called with one argument \textit{main}
to ensure compatibility with earlier version of the package.
It must either be empty (|\childdocmain{}|)
or precisely match the filename of the main file in which it is specified.
See \secref{sec:detection} for further information.
\item
The filename \textit{main} must be specified without the |.tex| extension.
\item
The filename \textit{main} is case sensitive
(even in case-insensitive file systems)
due to internal string comparison.
\item
The argument \textit{main} should be fully expanded, it cannot be a macro.
\item
Subdirectories and special characters should be avoided in filenames.
\item
The command |\childdocmain{|\textit{main}|}| must be followed by a whitespace.
It should not be followed immediately by another command
or by a comment mark `|%|'.
This is because the \TeX{} parser reads the token immediately following
the argument of |\childdocmain| and puts it
at the beginning of every child section;
however, a white\-space is ignored.
\end{itemize}

%%%%%%%%%%%%%%%%%%%%%%%%%%%%%%%%%%%%%%%%
\paragraph{Content of Main File.}

It is advisable to place all content in the child files included by |\include|.
Any output contained in the main file will appear in all child documents
unless suppressed manually;
it cannot be suppressed automatically by the |\includeonly| directive
and thus should normally be avoided.
A method to include some content in the main file
by means of conditional processing is described in \secref{sec:conditional}.

%%%%%%%%%%%%%%%%%%%%%%%%%%%%%%%%%%%%%%%%
\paragraph{Page Numbering.}

When only a part of the document is compiled,
the appropriate numbering of pages
(as well as other status parameters)
is determined from the |.aux| files.
The latter contain information from previous passes.
However this information needs to propagate through
all intermediate child documents.
Therefore the page numbering in child documents may well
be inconsistent until the complete document is compiled at least once.

A useful (if unconventional) way to always ensure a consistent
page numbering is to restart the numbering in each child document
and denote the pages by `\textit{child}|.|\textit{page}'
where \textit{child} represents the chapter/section number of the child file.
This can be achieved by the command
|\numberwithin{page}{|\textit{child}|}|
of the \textsf{amsmath} package
where \textit{child} can be |chapter| or |section|
depending on the chosen structuring.
Alternatively, one can modify the macro |\thepage| appropriately
and reset the counter |page| at the start of each child file.

%%%%%%%%%%%%%%%%%%%%%%%%%%%%%%%%%%%%%%%%%%%%%%%%%%%%%%%%%%%%%%%%%%%%%%%%%%%%%%%%
\subsection{Conditional Processing}
\label{sec:conditional}

The package provides a mechanism to compile different versions
of a document. To customise the versions further some conditional processing
can come in handy to distinguish which version is being compiled.
The package provides two macros to describe the compilation context:

%%%%%%%%%%%%%%%%%%%%%%%%%%%%%%%%%%%%%%%%
\DescribeMacro{\ifchilddoc}
The conditional |\ifchilddoc| distinguishes between the compilation of
child documents and the main document:
%
\begin{center}
|\ifchilddoc |\textit{child-code}| |[|\||else |\textit{main-code}]| \||fi|
\end{center}

%%%%%%%%%%%%%%%%%%%%%%%%%%%%%%%%%%%%%%%%
\DescribeMacro{\childdocname}
\DescribeMacro{\childdocjob}
The macro |\childdocname| contains the filename (without extension)
of the main or child file being processed.
Note that |\childdocjob| will always contain the name of the main file.

%%%%%%%%%%%%%%%%%%%%%%%%%%%%%%%%%%%%%%%%
\paragraph{Title Page.}

Conditional processing can be used to include a title or banner page
in the main document when proper precautions are taken.
Importantly, the code in the main file should ensure that the page counter
(as well as other status parameters which are stored in the |.aux| files)
takes the same value after the conditional processing.
Otherwise the page numbers may take divergent values
depending on which part is compiled.

For example, a title page could be declared by:
%
\begin{center}
\begin{tabular}{l}
|\ifchilddoc\||else|\\
|\addtocounter{page}{-1}|\\
\textit{code for title page}\\
|\newpage|\\
|\||fi|
\end{tabular}
\end{center}
%
A banner page for the child documents can be generated by:
%
\begin{center}
\begin{tabular}{l}
|\ifchilddoc|\\
|\addtocounter{page}{-1}|\\
\textit{code for banner page}\\
|\newpage|\\
|\||fi|
\end{tabular}
\end{center}
%
Here one could write a message such as:
\begin{center}
|This is the part \childdocname{} of \childdocjob{}.|
\end{center}

%%%%%%%%%%%%%%%%%%%%%%%%%%%%%%%%%%%%%%%%%%%%%%%%%%%%%%%%%%%%%%%%%%%%%%%%%%%%%%%%
\subsection{Flags}
\label{sec:flags}

The package makes it easy to generate different versions
of the main or child documents.
To this end compilation flags can be defined
and assigned different default values.
They will be particularly useful in conjunction
with the forwarding mechanism described in \secref{sec:forward}.

For example, it may be useful to have a flag |\version|
which can be set to |draft| or |final|.
The document source will contain some conditional code
depending on the value of |\version|.
Suppose further, the flag should default to |final| for the main file
and to |draft| for child files
which is a natural assignment for editing the document.
This is achieved by placing the following code
in the preamble of the main document
(below the |\childdocmain| directive):
%
\begin{center}
\begin{tabular}{l}
|\ifchilddoc|\\
|\providecommand{\version}{draft}|\\
|\||else|\\
|\providecommand{\version}{final}|\\
|\||fi|
\end{tabular}
\end{center}
%
The definition by |\providecommand| makes sure
that previous definitions are not overwritten.
Further statements |\providecommand{\version}{...}|
can thus be added before the above code to override it.

For the main file, one might add a line
(between |\childdocmain| and the above block)
%
\begin{center}
|%\ifchilddoc\||else\providecommand{\version}{draft}\||fi|
\end{center}
%
which can be uncommented to produce a draft version.
Likewise one can add a line to the very top of a child file
(above the |\childdocof{|\textit{main}|}| directive)
%
\begin{center}
|%\providecommand{\version}{final}|
\end{center}
%
which can be uncommented to produce the final version of this child document.

%%%%%%%%%%%%%%%%%%%%%%%%%%%%%%%%%%%%%%%%%%%%%%%%%%%%%%%%%%%%%%%%%%%%%%%%%%%%%%%%
\subsection{Forwarding}
\label{sec:forward}

Different versions of the main or child documents
using compilation flags as described in \secref{sec:flags}
can be (permanently) stored in different files
for convenient compilation, viewing and distribution.
To this end, the package defines a command
to pass on compilation to a different file:

%%%%%%%%%%%%%%%%%%%%%%%%%%%%%%%%%%%%%%%%
\DescribeMacro{\childdocforward}
The command |\childdocforward| redirects processing to
another source file:
%
\begin{center}
\begin{tabular}{l}
|\input{childdoc.def}|\\
|\childdocforward[|\textit{main}|]{|\textit{dest}|}|\\
\end{tabular}
\end{center}
%
The argument \textit{dest} is the destination file
(without extension).
It should be the main file or one of the child files.
Note that further \textsf{childdoc} directives
such as |\childdocof| and |\childdocforward|
in the indicated file will be processed in this form.
The optional argument \textit{main}
passes on directly to the main file \textit{main}
while pretending to compile the child \textit{dest}.
This form behaves as if \textit{dest}
issues |\childdocof{|\textit{main}|}| right away,
and no further \textsf{childdoc} directives will be processed.

%%%%%%%%%%%%%%%%%%%%%%%%%%%%%%%%%%%%%%%%
\DescribeMacro{\...prefix}
In the alternative form |\childdocforwardprefix|,
%
\begin{center}
\begin{tabular}{l}
|\input{childdoc.def}|\\
|\childdocforwardprefix[|\textit{main}|]{|\textit{prefix}|}{|\textit{dest}|}|
\end{tabular}
\end{center}
%
the destination file is determined by a pattern
depending on the current file:
To make this work, the current file must be called
`{\textit{prefix}\hspace{0.2em}\textit{suffix}}'
with \textit{prefix} matching precisely the argument.
Processing is then passed on to the file
`{\textit{dest}\hspace{0.2em}\textit{suffix}}'.
Surely, the same effect is achieved by
directly specifying the
argument `{\textit{dest}\hspace{0.2em}\textit{suffix}}'
in the first form.
However, that requires to set up a different file
for each child. With the alternative form of the command
all these files can have exactly the same content
which simplifies setting them up and maintaining them.

For example, the following file |draft.tex|
with a compilation flag |\version| as described in \secref{sec:flags}
compiles the main document as a draft:
%
\begin{center}
\begin{tabular}{l}
|\def\version{draft}|\\
|\input{childdoc.def}|\\
|\childdocforward{|\textit{main}|}|
\end{tabular}
\end{center}
%
Likewise, the following files |final|\textit{nn}|.tex|
compile the final version of the child document
|child|\textit{nn}|.tex|:
%
\begin{center}
\begin{tabular}{l}
|\def\version{final}|\\
|\input{childdoc.def}|\\
|\childdocforwardprefix{final}{child}|
\end{tabular}
\end{center}
%

Note that when several versions of a main file and/or of each child file
are to be generated, it may be convenient to set up a |Makefile| or
shell script to automatise the process.

%%%%%%%%%%%%%%%%%%%%%%%%%%%%%%%%%%%%%%%%%%%%%%%%%%%%%%%%%%%%%%%%%%%%%%%%%%%%%%%%
\subsection{Command Line Processing}
\label{sec:commandline}

The effect of redirection files can also be achieved by invoking
the \LaTeX{} compiler with a more elaborate command line.
Most conveniently this should be done as part
of a shell script or a |Makefile|.

When using \textsf{childdoc} in the main file, the following
command lines effectively perform a redirection
(note that depending on the shell being used,
backslashes may have to be doubled: `|\|' $\to$ `|\\|'):
%
\begin{center}
|... -jobname "|\textit{target}|" |\\|"|[\textit{flags}]%
|\input{childdoc.def}\childdocforward[|\textit{main}|]{|\textit{dest}|}"|
\end{center}
%
Here \textit{target} is the name of the output file,
\textit{main} is the name of the main file
and \textit{dest} is the name of the main or child file to be processed
(all filenames without extensions).
The optional argument \textit{main} can be omitted
if \textit{main} matches \textit{dest}.
Optionally, compilation \textit{flags} can be defined via |\def| commands.
This command line makes the \TeX{} engine believe
it is compiling the file \textit{target}
whose content is specified as the latter parameter.
The provided code then forwards the processing to
\textit{main} or \textit{dest} as described in \secref{sec:forward}.

%%%%%%%%%%%%%%%%%%%%%%%%%%%%%%%%%%%%%%%%%%%%%%%%%%%%%%%%%%%%%%%%%%%%%%%%%%%%%%%%
\subsection{Include by Input}
\label{sec:input}

Including child documents by |\include| has some restrictions by design.
Most notably, the content of a child document always occupies
its own set of pages; pages cannot be shared between child documents.
Usually, this behaviour makes perfect sense
because each child document contain an essential part of the document.
However, in some situations it may be desirable to compose
a document from a collection of parts
without having mandatory page breaks between then.
For this case, the package
provides a mechanism to include parts
by |\input| which can also be processed individually.
However, by construction this mechanism
requires manual handling of the content to be output.

%%%%%%%%%%%%%%%%%%%%%%%%%%%%%%%%%%%%%%%%
\DescribeMacro{\ifchilddocmanual}
The main file should be prepared as usual, see \secref{sec:include}.
However, the document body must make a distinction
between processing of an individual part and of the main document, e.g.:
%
\begin{center}
\begin{tabular}{l}
|\ifchilddocmanual|\\
|\input{\childdocname}|\\
|\||else|\\
\textit{document body with }|\input{|\textit{part}|}|\\
|\||fi|
\end{tabular}
\end{center}
%
The conditional |\ifchilddocmanual| is true whenever
a part to be included by |\input| is being compiled,
and the name of the part is stored in |\childdocname|.

%%%%%%%%%%%%%%%%%%%%%%%%%%%%%%%%%%%%%%%%
\DescribeMacro{\childdocby}
Each part to be included by |\input| should start with:
%
\begin{center}
\begin{tabular}{l}
|\input{childdoc.def}|\\
|\childdocby{|\textit{main}|}|\\
\end{tabular}
\end{center}
%
The directive |\childdocby| is similar to |\childdocof|
described in \secref{sec:include},
but the subsequent selection of content must be done manually.
To that end, both |\ifchilddoc| and |\ifchilddocmanual|
will be true upon processing of a part,
and the name of the part is stored in |\childdocname|.
Note that |\jobname| will be set to the filename of the current part
so that each part receives an individual |.aux| file
that does not interfere with the |.aux| file(s) of the main document.
This behaviour can be altered by the alternative form
|\childdocby[*]{|\textit{main}|}| (with a non-empty optional argument)
which uses the |.aux| file of the main document
by setting |\jobname| to \textit{main}.

%%%%%%%%%%%%%%%%%%%%%%%%%%%%%%%%%%%%%%%%%%%%%%%%%%%%%%%%%%%%%%%%%%%%%%%%%%%%%%%%
\subsection{Driver Development}
\label{sec:driver}

The \textsf{childdoc} mechanism can also be use for the development
of definition files such as \LaTeX{} styles or classes.
This case differs from the above setup with multiple parts
included by |\include| in that no |\includeonly| should be invoked.
This can be achieved by starting the include file
(before |\ProvidesPackage|) with:
%
\begin{center}
\begin{tabular}{l}
|\input{childdoc.def}|\\
|\childdocforward{|\textit{main}|}|\\
\end{tabular}
\end{center}
%
or alternatively with:
%
\begin{center}
\begin{tabular}{l}
|\input{childdoc.def}|\\
|\childdocby{|\textit{main}|}|\\
\end{tabular}
\end{center}
%
Both forms have slightly different effects as described above.
The main file is prepared as usual, see \secref{sec:include}.

%%%%%%%%%%%%%%%%%%%%%%%%%%%%%%%%%%%%%%%%%%%%%%%%%%%%%%%%%%%%%%%%%%%%%%%%%%%%%%%%
\subsection{Legacy Detection}
\label{sec:detection}

The directive |\childdocmain| in the main file can detect
whether the complete document or merely a child is to be compiled
even without using the directive |\childdocof|.
This method is deprecated because it is less robust
and there is no compelling reason to use it;
it is merely provided for backward compatibility
and it may be removed in future versions.

If the detection mechanism is to be used,
it is mandatory to correctly specify
the filename of the main file as the argument of |\childdocmain|:
%
\begin{center}
\begin{tabular}{l}
|\input{childdoc.def}|\\
|\childdocmain{|\textit{main}|}|\\
\end{tabular}
\end{center}
%
If |\jobname| does not match the argument \textit{main} of |\childdocmain|,
it is assumed that |\jobname| points to the child file to be compiled.
When using |\childdocmain| with the main file specified as argument,
it suffices to start a child file
with just |\input{|\textit{main}|}|
without loading of the package and using |\childdocof|.
If instead all processing is done
with the appropriate \textsf{childdoc} directives,
the argument of \textit{main} of |\childdocmain| can be empty.

An alternative version of the command line processing described
in \secref{sec:commandline} using the detection mechanism reads:
%
\begin{center}
|... -jobname "|\textit{target}|" "|[\textit{flags}]%
[|\def\jobname{|\textit{dest}|}|]|\input{|\textit{main}|}"|
\end{center}

%%%%%%%%%%%%%%%%%%%%%%%%%%%%%%%%%%%%%%%%%%%%%%%%%%%%%%%%%%%%%%%%%%%%%%%%%%%%%%%%
\subsection{Manual Code}
\label{sec:manual}

In case one cannot be certain whether the definitions file |childdoc.def|
is installed on the target \TeX{} distribution
and one prefers not to ship it,
it is conceivable to paste a few relevant commands into the sources.

To that end, drop all statements |\input{childdoc.def}|
and perform the replacements as outlined below.
Instead of |\childdocmain{|\textit{main}|}| add the following code
to the top of the main file:
%
\begin{center}
\begin{tabular}{l}
|\||ifdefined\childdocname\endinput\||fi\newif\ifchilddoc|\\
|\edef\childdocname{\scantokens\expandafter{\jobname\noexpand}}|\\
|\def\childdocmain{|\textit{main}|}\||ifx\childdocmain\childdocname\||else|\\
|\childdoctrue\includeonly{\childdocname}\let\jobname\childdocmain\||fi|\\
\end{tabular}
\end{center}
%
Instead of |\childdocof{|\textit{main}|}| just include the main file
at the top of each child file:
%
\begin{center}
|\input{|\textit{main}|}|
\end{center}
%
A simple redirection |\childdocforward{|\textit{dest}|}| is achieved by:
%
\begin{center}
|\def\jobname{|\textit{dest}|}\input{\jobname}|
\end{center}
%
The redirection with prefix
|\childdocforwardprefix[|\textit{prefix}|]{|\textit{dest}|}|
is accomplished by:
%
\begin{center}
\begin{tabular}{l}
|{\edef\jobname{\scantokens\expandafter{\jobname\noexpand}}|\\
|\def\redirectjob |\textit{prefix}|#1~~~{\gdef\jobname{|\textit{dest}|#1}}|\\
|\expandafter\redirectjob\jobname~~~}\input{\jobname}|
\end{tabular}
\end{center}

In an alternative approach,
child documents can be compiled by a specific command line
without additional code or specific definitions:
%
\begin{center}
|... -jobname "|\textit{target}|" "|[\textit{flags}]%
|\includeonly{|\textit{dest}|}\input{|\textit{main}|}"|
\end{center}
%

%%%%%%%%%%%%%%%%%%%%%%%%%%%%%%%%%%%%%%%%%%%%%%%%%%%%%%%%%%%%%%%%%%%%%%%%%%%%%%%%
%%%%%%%%%%%%%%%%%%%%%%%%%%%%%%%%%%%%%%%%%%%%%%%%%%%%%%%%%%%%%%%%%%%%%%%%%%%%%%%%
\section{Information}

%%%%%%%%%%%%%%%%%%%%%%%%%%%%%%%%%%%%%%%%%%%%%%%%%%%%%%%%%%%%%%%%%%%%%%%%%%%%%%%%
\subsection{Copyright}

Copyright \copyright{} 2017--2018 Niklas Beisert

This work may be distributed and/or modified under the
conditions of the \LaTeX{} Project Public License, either version 1.3
of this license or (at your option) any later version.
The latest version of this license is in
  \url{http://www.latex-project.org/lppl.txt}
and version 1.3 or later is part of all distributions of \LaTeX{}
version 2005/12/01 or later.

This work has the LPPL maintenance status `maintained'.

The Current Maintainer of this work is Niklas Beisert.

This work consists of the files |README.txt|, |childdoc.ins| and |childdoc.dtx|
as well as the derived files |childdoc.def|, |cdocsamp.tex|
with |cdocsch1.tex|, |cdocsch2.tex|, |cdocspt3.tex|, |cdocspt4.tex|,
|cdocsdrf.tex|, |cdocsfn1.tex|, |cdocsfn2.tex|
as well as |childdoc.pdf|.

%%%%%%%%%%%%%%%%%%%%%%%%%%%%%%%%%%%%%%%%%%%%%%%%%%%%%%%%%%%%%%%%%%%%%%%%%%%%%%%%
\subsection{Files and Installation}

The package consists of the files:
%
\begin{center}
\begin{tabular}{ll}
    |README.txt|   & readme file \\
    |childdoc.ins| & installation file \\
    |childdoc.dtx| & source file \\
    |childdoc.def| & definition file \\
    |cdocsamp.tex| & sample main file \\
    |cdocsch1.tex| & sample include file \\
    |cdocsch2.tex| & sample include file \\
    |cdocspt3.tex| & sample part file \\
    |cdocspt4.tex| & sample part file \\
    |cdocsdrf.tex| & sample redirection file \\
    |cdocsfn1.tex| & sample redirection file \\
    |cdocsfn2.tex| & sample redirection file \\
    |childdoc.pdf| & manual
\end{tabular}
\end{center}
%
The distribution consists of the files
|README.txt|, |childdoc.ins| and |childdoc.dtx|.
%
\begin{itemize}
\item
Run (pdf)\LaTeX{} on |childdoc.dtx|
to compile the manual |childdoc.pdf| (this file).
\item
Run \LaTeX{} on |childdoc.ins| to create the definitions file |childdoc.def|
and the sample |cdocsamp.tex| with include files
|cdocsch1.tex|, |cdocsch2.tex|, |cdocspt3.tex|, |cdocspt4.tex|,
|cdocsdrf.tex|, |cdocsfn1.tex|, |cdocsfn2.tex|.
Then copy the file |childdoc.def| to an appropriate directory of your \LaTeX{}
distribution, e.g.\ \textit{texmf-root}|/tex/latex/childdoc|.
\end{itemize}

%%%%%%%%%%%%%%%%%%%%%%%%%%%%%%%%%%%%%%%%%%%%%%%%%%%%%%%%%%%%%%%%%%%%%%%%%%%%%%%%
\subsection{Related CTAN Packages}

There are several other packages which offer a similar functionality:
%
\begin{itemize}
\item
The packages
\href{http://ctan.org/pkg/docmute}{\textsf{docmute}},
\href{http://ctan.org/pkg/includex}{\textsf{includex}} and
\href{http://ctan.org/pkg/standalone}{\textsf{standalone}}
provide commands to include only the document body of
a child file thus allowing both files to be compiled individually.
\item
The packages \href{http://ctan.org/pkg/subdocs}{\textsf{subdocs}}
and \href{http://ctan.org/pkg/subfiles}{\textsf{subfiles}}
provide structures in which the main and child documents can be
encapsulated and allowing them to be compiled individually.
The inclusion mechanism is different from the conventional |\include|.
\item
The package \href{http://ctan.org/pkg/combine}{\textsf{combine}}
is an elaborate solution to combine several documents into one.
\end{itemize}
%
See also the CTAN topic \href{http://ctan.org/topic/subdocs}{\textsf{subdocs}}
for further related packages.
The present package differs from the above solutions in that
a document structure constructed with the conventional |\include| mechanism
just needs two extra commands at the top of every file
such that all constituent files can be compiled individually.

%%%%%%%%%%%%%%%%%%%%%%%%%%%%%%%%%%%%%%%%%%%%%%%%%%%%%%%%%%%%%%%%%%%%%%%%%%%%%%%%
%\subsection{Feature Suggestions}
%
%The following is a list of features which may be useful for future
%versions of this package:
%%
%\begin{itemize}
%\item
%\ldots
%\end{itemize}

%%%%%%%%%%%%%%%%%%%%%%%%%%%%%%%%%%%%%%%%%%%%%%%%%%%%%%%%%%%%%%%%%%%%%%%%%%%%%%%%
\subsection{Revision History}

%%%%%%%%%%%%%%%%%%%%%%%%%%%%%%%%%%%%%%%%
\paragraph{v2.0:} 2018/12/30

\begin{itemize}
\item
immediate forward processing
\item
added |\childdocby| mechanism
\item
manual restructured
\end{itemize}

%%%%%%%%%%%%%%%%%%%%%%%%%%%%%%%%%%%%%%%%
\paragraph{v1.6:} 2018/01/17

\begin{itemize}
\item
application for development of include files
\item
corrections to manual
\end{itemize}

%%%%%%%%%%%%%%%%%%%%%%%%%%%%%%%%%%%%%%%%
\paragraph{v1.5:} 2017/05/21

\begin{itemize}
\item
more complete structuring introduced
\item
|\childdocof| introduced
\item
|\childdoc| renamed to |\childdocmain|
\item
|\childredirect| renamed to |\childdocforward| and |\childdocforwardprefix|
and functionality expanded
\end{itemize}

%%%%%%%%%%%%%%%%%%%%%%%%%%%%%%%%%%%%%%%%
\paragraph{v1.0:} 2017/04/27

\begin{itemize}
\item
manual and install package
\item
first version published on CTAN
\end{itemize}

%%%%%%%%%%%%%%%%%%%%%%%%%%%%%%%%%%%%%%%%
\paragraph{v0.6:} 2017/04/26

\begin{itemize}
\item
redirection mechanism added
\end{itemize}

%%%%%%%%%%%%%%%%%%%%%%%%%%%%%%%%%%%%%%%%
\paragraph{v0.5:} 2017/04/26

\begin{itemize}
\item
functionality in definition file
\end{itemize}


%%%%%%%%%%%%%%%%%%%%%%%%%%%%%%%%%%%%%%%%%%%%%%%%%%%%%%%%%%%%%%%%%%%%%%%%%%%%%%%%
%%%%%%%%%%%%%%%%%%%%%%%%%%%%%%%%%%%%%%%%%%%%%%%%%%%%%%%%%%%%%%%%%%%%%%%%%%%%%%%%
%%%%%%%%%%%%%%%%%%%%%%%%%%%%%%%%%%%%%%%%%%%%%%%%%%%%%%%%%%%%%%%%%%%%%%%%%%%%%%%%
\appendix

\settowidth\MacroIndent{\rmfamily\scriptsize 000\ }

 \DocInput{childdoc.dtx}

\end{document}
%</driver>
% \fi
%
% %%%%%%%%%%%%%%%%%%%%%%%%%%%%%%%%%%%%%%%%%%%%%%%%%%%%%%%%%%%%%%%%%%%%%%%%%%%%%%
% %%%%%%%%%%%%%%%%%%%%%%%%%%%%%%%%%%%%%%%%%%%%%%%%%%%%%%%%%%%%%%%%%%%%%%%%%%%%%%
% \section{Sample}
%\iffalse
%<*samplemain>
%\fi
%
% The following presents a sample document
% with two chapters, two parts, a title page,
% a compile flag as well as three forwarding files to set the flag.
% It consists of eight |.tex| files:
% \begin{center}
% \begin{tabular}{ll}
% |cdocsamp.tex|&main file\\
% |cdocsch1.tex|&include file for chapter 1\\
% |cdocsch2.tex|&include file for chapter 2\\
% |cdocspt3.tex|&include file for part 3\\
% |cdocspt4.tex|&include file for part 4\\
% |cdocsdrf.tex|&forwarding file for main file in draft mode\\
% |cdocsfi1.tex|&forwarding file for final version of chapter 1\\
% |cdocsfi2.tex|&forwarding file for final version of chapter 2\\
% \end{tabular}
% \end{center}
% Each of the eight files can be compiled directly by the \LaTeX{} compiler.
%
% %%%%%%%%%%%%%%%%%%%%%%%%%%%%%%%%%%%%%%
% \paragraph{Main File.}
%
% The main file is called |cdocsamp.tex|.
%
% Load the \textsf{childdoc} definitions and
% declare the filename for the main document:
%    \begin{macrocode}
\input{childdoc.def}
\childdocmain{}
%    \end{macrocode}

% Optional override for |\version| flag:
%    \begin{macrocode}
%%\ifchilddoc\else\providecommand{\version}{draft}\fi
%    \end{macrocode}

% Define the default values for the |\version| flag
% (|final| for the main file and |draft| for childs):
%    \begin{macrocode}
\ifchilddoc
\providecommand{\version}{draft}
\else
\providecommand{\version}{final}
\fi
%    \end{macrocode}

% Load the standard document class:
%    \begin{macrocode}
\documentclass[12pt]{article}
%    \end{macrocode}

% Start the document body:
%    \begin{macrocode}
\begin{document}
%    \end{macrocode}

% Declare a title page.
% Print title, part of document being processed and version flag:
%    \begin{macrocode}
\addtocounter{page}{-1}
\begin{center}
{\LARGE\bfseries{}childdoc example\par}
\vspace{1cm}
\ifchilddoc
\ifchilddocmanual part\else chapter\fi:
`\childdocname' of `\childdocjob'\par
\else
main document: `\childdocjob'\par
\fi
version: \version\par
\end{center}
\newpage
%    \end{macrocode}

% Manually include selected file,
% otherwise process as usual:
%    \begin{macrocode}
\ifchilddocmanual
\section*{part `\childdocname'}
\input{\childdocname}
\else
%    \end{macrocode}

% Include the two chapters:
%    \begin{macrocode}
\include{cdocsch1}
\include{cdocsch2}
%    \end{macrocode}

% Include the two parts unless only chapters should be displayed:
%    \begin{macrocode}
\ifchilddoc\else
\section{part three}
\input{cdocspt3}
\section{part four}
\input{cdocspt4}
\fi
%    \end{macrocode}

% Process as usual until here:
%    \begin{macrocode}
\fi
%    \end{macrocode}

% End of document body:
%    \begin{macrocode}
\end{document}
%    \end{macrocode}
%\iffalse
%</samplemain>
%\fi
%
% %%%%%%%%%%%%%%%%%%%%%%%%%%%%%%%%%%%%%%
% \paragraph{Chapter Include Files.}
%
% The include files are called |cdocsch1.tex| and |cdocsch2.tex|.
%
%\iffalse
%<*samplechap1|samplechap2>
%\fi

% Optional override for |\version| flag:
%    \begin{macrocode}
%%\providecommand{\version}{final}
%    \end{macrocode}

% Include the main document:
%    \begin{macrocode}
\input{childdoc.def}
\childdocof{cdocsamp}
%    \end{macrocode}

%\iffalse
%</samplechap1|samplechap2>
%\fi
%
%\iffalse
%<*samplechap1>
%\fi
% Some text for chapter 1:
%    \begin{macrocode}
\section{one}
some text in chapter one
%    \end{macrocode}

%\iffalse
%</samplechap1>
%\fi
% Some text for chapter 2:
%\iffalse
%<*samplechap2>
%\fi
%    \begin{macrocode}
\section{two}
more text in chapter two
%    \end{macrocode}

%\iffalse
%</samplechap2>
%\fi
%
% %%%%%%%%%%%%%%%%%%%%%%%%%%%%%%%%%%%%%%
% \paragraph{Part Include Files.}
%
% The include files are called |cdocspt3.tex| and |cdocspt4.tex|.
%
%\iffalse
%<*samplepart3|samplepart4>
%\fi

% Optional override for |\version| flag:
%    \begin{macrocode}
%%\providecommand{\version}{final}
%    \end{macrocode}

% Include the main document:
%    \begin{macrocode}
\input{childdoc.def}
\childdocby{cdocsamp}
%    \end{macrocode}

%\iffalse
%</samplepart3|samplepart4>
%\fi
%
%\iffalse
%<*samplepart3>
%\fi
% Some text for part 3:
%    \begin{macrocode}
some text in part three
%    \end{macrocode}

%\iffalse
%</samplepart3>
%\fi
% Some text for part 4:
%\iffalse
%<*samplepart4>
%\fi
%    \begin{macrocode}
more text in part four
%    \end{macrocode}

%\iffalse
%</samplepart4>
%\fi
%
% %%%%%%%%%%%%%%%%%%%%%%%%%%%%%%%%%%%%%%
% \paragraph{Forwarding for a Complete Draft.}
%
% The following forwarding file |cdocsdrf.tex|
% compiles the main document in draft mode:
%\iffalse
%<*sampledraft>
%\fi
%    \begin{macrocode}
\def\version{draft}
\input{childdoc.def}
\childdocforward{cdocsamp}
%    \end{macrocode}

%\iffalse
%</sampledraft>
%\fi
%
% %%%%%%%%%%%%%%%%%%%%%%%%%%%%%%%%%%%%%%
% \paragraph{Forwarding for Final Version of the Chapters.}
%
% The following forwarding files |cdocsfn1.tex| and |cdocsfn2.tex|
% (with identical content)
% compile the final versions of the child documents
% |cdocsch1.tex| and |cdocsch2.tex|, respectively:
%\iffalse
%<*samplefinal>
%\fi
%    \begin{macrocode}
\def\version{final}
\input{childdoc.def}
\childdocforwardprefix[cdocsamp]{cdocsfn}{cdocsch}
%    \end{macrocode}

%\iffalse
%</samplefinal>
%\fi
%
% %%%%%%%%%%%%%%%%%%%%%%%%%%%%%%%%%%%%%%
% \paragraph{Command Line Processing.}
%
% The following three command lines generate the output files
% |cdocscld|, |cdocscl1| and |cdocscl2|
% which should be identical to
% |cdocsdrf|, |cdocsch1| and |cdocsfn2|, respectively:
% \begin{center}
% \begin{tabular}{l}
% |latex -jobname cdocscld \|\\
% |  "\def\version{draft}\input{childdoc.def}\childdocforward{cdocsamp}"|\\
% |latex -jobname cdocscl1 \|\\
% |  "\input{childdoc.def}\childdocforward[cdocsamp]{cdocsch1}"|\\
% |latex -jobname cdocscl2 \|\\
% |  "\def\version{final}\input{childdoc.def}\childdocforward{cdocsch2}"|
% \end{tabular}
% \end{center}
% Note that the trailing backslash on each first line
% merely continues the input to the second line
% (for convenient cut ant paste).
% Furthermore, the command |latex| can be replaced by any
% of its alternative versions such as |pdflatex|.
%
% %%%%%%%%%%%%%%%%%%%%%%%%%%%%%%%%%%%%%%%%%%%%%%%%%%%%%%%%%%%%%%%%%%%%%%%%%%%%%%
% %%%%%%%%%%%%%%%%%%%%%%%%%%%%%%%%%%%%%%%%%%%%%%%%%%%%%%%%%%%%%%%%%%%%%%%%%%%%%%
% \section{Implementation}
%\iffalse
%<*package>
%\fi
%
% This section describes the definitions file |childdoc.def|.

% The definitions cannot be loaded using |\usepackage| or |\RequirePackage|
% which has a mechanism to prevent loading a style file more than once.
% When loading the definitions by means of |\input|
% multiple instances have to be prevented manually:
%\iffalse
%This code needs to be before the `\ProvidesFile' directive
%which is defined at the beginning of this file.
%Therefore it is also placed there and commented out here.
%</package>
%<*discard>
%\fi
%    \begin{macrocode}
\ifdefined\childdocmain\endinput\fi
%    \end{macrocode}
%\iffalse
%</discard>
%<*package>
%\fi
%
% \macro{\ifchilddoc}
% \macro{\ifchilddocmanual}
% The conditional |\ifchilddoc| tells whether a
% child (true) or main (false) document is being compiled.
% The conditional |\ifchilddocmanual| tells whether
% the |\includeonly| mechanism is used (false) or
% the selection of child files must be performed manually (true).
% The definitions initialise to false:
%    \begin{macrocode}
\newif\ifchilddoc
\newif\ifchilddocmanual
%    \end{macrocode}

% \macro{\childdocname}
% \macro{\childdocjob}
% The macro |\childdocname| stores the name of the main document
% to be compiled. The macro |\childdocjob| stores the name of
% the document on which the \LaTeX{} compiler was originally invoked.
% The content of |\jobname| cannot be compared
% to filenames specified in the source due to different catcodes.
% The following code rescans |\jobname|, stores the result
% in |\childdocname| and saves a copy in |\childdocjob|:
%    \begin{macrocode}
\edef\childdocname{\scantokens\expandafter{\jobname\noexpand}}
\let\childdocjob\childdocname
%    \end{macrocode}

% \macro{\childdocdisable}
% The macro |\childdocdisable| prevents the main file
% from being processed more than once.
% At this stage, the main document command |\childdocmain|
% is assumed to be called once again where it should do nothing.
% Any subsequent call to it should prevent
% a secondary processing of the main document
% It overwrites the forwarding commands
% |\childdocof| and |\childdocforward|
% with empty macros to prevent further inclusions of the main document:
%    \begin{macrocode}
\newcommand{\childdocdisable}
{
  \renewcommand{\childdocmain}[1]{\renewcommand{\childdocmain}[1]{\endinput}}
  \renewcommand{\childdocof}[1]{}
  \renewcommand{\childdocby}[2][]{}
  \renewcommand{\childdocforward}[2][]{}
  \renewcommand{\childdocdisable}{}
}
%    \end{macrocode}

% \macro{\childdocmain}
% The macro |\childdocmain| is to be called at the top of the main file
% with nothing or the main filename (without extension) as argument.
% First, it breaks loops.
% If the argument is not empty and does not match |\childdocname|
% (which is set by the first inclusion of |childdoc.def|),
% |\ifchilddoc| is set to true, |\includeonly| is applied to the child file
% and |\jobname| is set to the main file
% (for proper handling of |.aux| files):
%    \begin{macrocode}
\newcommand{\childdocmain}[1]
{
  \childdocdisable\childdocmain{}
  \if?#1?\else
    \begingroup
      \def\childdoctmp{#1}
      \ifx\childdoctmp\childdocname
        \def\childdoctmp{}
      \else
        \def\childdoctmp
        {
          \childdoctrue
          \includeonly{\childdocname}
          \def\childdocjob{#1}
          \def\jobname{#1}
        }
      \fi
      \expandafter
    \endgroup
    \childdoctmp
  \fi
}
%    \end{macrocode}

% \macro{\childdocof}
% The command |\childdocof| redirects
% compilation to the main file |#1|.
%    \begin{macrocode}
\newcommand{\childdocof}[1]
{
  \childdocdisable
  \childdoctrue
  \includeonly{\childdocname}
  \def\jobname{#1}
  \def\childdocjob{#1}
  \input{#1}
}
%    \end{macrocode}

% \macro{\childdocby}
% The command |\childdocby| ....
%    \begin{macrocode}
\newcommand{\childdocby}[2][]
{
  \childdocdisable
  \childdoctrue
  \childdocmanualtrue
  \if?#1?\else
    \def\jobname{#2}
  \fi
  \def\childdocjob{#2}
  \input{#2}
  \endinput
}
%    \end{macrocode}

% \macro{\childdocforward}
% The command |\childdocforward| redirects
% compilation to the main file or
% (if the optional argument is given) a child file.
% Parameters are set as if the main file
% or a child file starting with |\childdocof| was compiled.
% Then compilation is handed over to the main file:
%    \begin{macrocode}
\newcommand{\childdocforward}[2][]
{
  \begingroup
    \if?#1?
      \def\childdoctmp
      {
        \def\childdocname{#2}
        \def\childdocjob{#2}
        \def\jobname{#2}
        \input{#2}
        \endinput
      }
    \else
      \def\childdoctmp
      {
        \childdocdisable
        \def\childdocname{#2}
        \childdoctrue
        \includeonly{#2}
        \def\childdocjob{#1}
        \def\jobname{#1}
        \input{#1}
        \endinput
      }
    \fi
    \expandafter
  \endgroup
  \childdoctmp
}
%    \end{macrocode}

% \macro{\childdocforwardprefix}
% The command |\childdocforwardprefix| redirects
% compilation to the main or a child file by means of a pattern.
% The prefix |#1| in the current filename is replaced by |#2|
% and the suffix of the current filename is kept
% (it is assumed that the filename does not contain the substring `|~~~|'
% which is used as a delimiter).
% Compilation is handed over to the new file by |\childdocforward|:
%    \begin{macrocode}
\newcommand{\childdocforwardprefix}[3][]
{
  \begingroup
    \def\childdocextract #2##1~~~{\def\childdoctmp{\childdocforward[#1]{#3##1}}}
    \expandafter\childdocextract\childdocname~~~
    \expandafter
  \endgroup
  \childdoctmp
}
%    \end{macrocode}

% \macro{\childdoc}
% The deprecated macro |\childdoc| is a legacy version of |\childdocmain|:
%    \begin{macrocode}
\newcommand{\childdoc}{\childdocmain}
%    \end{macrocode}

% \macro{\childdocredirect}
% The deprecated macro |\childdocredirect| is a legacy version
% of |\childdocforward| and |\childdocforwardprefix|:
%    \begin{macrocode}
\newcommand{\childdocredirect}[2][]
{
  \begingroup
    \if?#1?
      \def\childdoctmp{\childdocforward{#2}}
    \else
      \def\childdoctmp{\childdocforwardprefix{#1}{#2}}
    \fi
    \expandafter
  \endgroup
  \childdoctmp
}
%    \end{macrocode}

%\iffalse
%</package>
%\fi
%
\endinput

\childdocof{cdocsamp}
%    \end{macrocode}

%\iffalse
%</samplechap1|samplechap2>
%\fi
%
%\iffalse
%<*samplechap1>
%\fi
% Some text for chapter 1:
%    \begin{macrocode}
\section{one}
some text in chapter one
%    \end{macrocode}

%\iffalse
%</samplechap1>
%\fi
% Some text for chapter 2:
%\iffalse
%<*samplechap2>
%\fi
%    \begin{macrocode}
\section{two}
more text in chapter two
%    \end{macrocode}

%\iffalse
%</samplechap2>
%\fi
%
% %%%%%%%%%%%%%%%%%%%%%%%%%%%%%%%%%%%%%%
% \paragraph{Part Include Files.}
%
% The include files are called |cdocspt3.tex| and |cdocspt4.tex|.
%
%\iffalse
%<*samplepart3|samplepart4>
%\fi

% Optional override for |\version| flag:
%    \begin{macrocode}
%%\providecommand{\version}{final}
%    \end{macrocode}

% Include the main document:
%    \begin{macrocode}
% \iffalse
%
% childdoc.dtx Copyright (C) 2017-2018 Niklas Beisert
%
% This work may be distributed and/or modified under the
% conditions of the LaTeX Project Public License, either version 1.3
% of this license or (at your option) any later version.
% The latest version of this license is in
%   http://www.latex-project.org/lppl.txt
% and version 1.3 or later is part of all distributions of LaTeX
% version 2005/12/01 or later.
%
% This work has the LPPL maintenance status `maintained'.
%
% The Current Maintainer of this work is Niklas Beisert.
%
% This work consists of the files childdoc.dtx and childdoc.ins
% and the derived files childdoc.def and cdocsamp.tex with
% cdocsch1.tex, cdocsch2.tex, cdocsdrf.tex, cdocsfn1.tex, cdocsfn2.tex.
%
%<package>\ifdefined\childdocmain\endinput\fi
%<package>\ProvidesFile{childdoc.def}[2018/12/30 v2.0 child document driver]
%<samplemain>\ProvidesFile{cdocsamp.tex}[2018/12/30 v2.0 sample for childdoc]
%<*driver>
%\ProvidesFile{childdoc.drv}[2018/12/30 v2.0 childdoc reference manual file]
\PassOptionsToClass{10pt,a4paper}{article}
\documentclass{ltxdoc}

\usepackage[margin=35mm]{geometry}
\usepackage{hyperref}
\usepackage{hyperxmp}
\usepackage[usenames]{color}

\hypersetup{colorlinks=true}
\hypersetup{pdfstartview=FitH}
\hypersetup{pdfpagemode=UseNone}
\hypersetup{pdfsource={}}
\hypersetup{pdflang={en-UK}}
\hypersetup{pdfcopyright={Copyright 2017-2018 Niklas Beisert.
  This work may be distributed and/or modified under the
  conditions of the LaTeX Project Public License, either version 1.3
  of this license or (at your option) any later version.}}
\hypersetup{pdflicenseurl={http://www.latex-project.org/lppl.txt}}
\hypersetup{pdfcontactaddress={ETH Zurich, ITP, HIT K,
  Wolfgang-Pauli-Strasse 27}}
\hypersetup{pdfcontactpostcode={8093}}
\hypersetup{pdfcontactcity={Zurich}}
\hypersetup{pdfcontactcountry={Switzerland}}
\hypersetup{pdfcontactemail={nbeisert@itp.phys.ethz.ch}}
\hypersetup{pdfcontacturl={http://people.phys.ethz.ch/\xmptilde nbeisert/}}

\newcommand{\secref}[1]{\hyperref[#1]{section \ref*{#1}}}

\parskip1ex
\parindent0pt
\let\olditemize\itemize
\def\itemize{\olditemize\parskip0pt}

\begin{document}

\title{The \textsf{childdoc} Package}
\hypersetup{pdftitle={The childdoc Package}}
\author{Niklas Beisert\\[2ex]
  Institut f\"ur Theoretische Physik\\
  Eidgen\"ossische Technische Hochschule Z\"urich\\
  Wolfgang-Pauli-Strasse 27, 8093 Z\"urich, Switzerland\\[1ex]
  \href{mailto:nbeisert@itp.phys.ethz.ch}
  {\texttt{nbeisert@itp.phys.ethz.ch}}}
\hypersetup{pdfauthor={Niklas Beisert}}
\hypersetup{pdfsubject={Manual for the LaTeX2e Package childdoc}}
\date{30 December 2018, \textsf{v2.0}}
\maketitle

\begin{abstract}\noindent
\textsf{childdoc} is a \LaTeXe{} package
that enables the direct compilation
of document sections included by |\include|
to individual files.
\end{abstract}

\begingroup
\parskip0ex
\tableofcontents
\endgroup

%%%%%%%%%%%%%%%%%%%%%%%%%%%%%%%%%%%%%%%%%%%%%%%%%%%%%%%%%%%%%%%%%%%%%%%%%%%%%%%%
%%%%%%%%%%%%%%%%%%%%%%%%%%%%%%%%%%%%%%%%%%%%%%%%%%%%%%%%%%%%%%%%%%%%%%%%%%%%%%%%
\section{Introduction}

\LaTeX{} provides a mechanism to structure a large document (such as a book)
into a main file and several child files (containing the chapters)
using the |\include| command.
This mechanism is beneficial for documents
which span hundreds of pages in order to
make the source file(s) more manageable.
Moreover, compilation can be restricted to
selected child files by means of the |\includeonly| command.
The latter feature can be used to reduce the compilation time while editing
(this was significantly more useful in the earlier days of \LaTeX{})
or to generate a smaller document which is easier to navigate.
Another application of |\includeonly| is to generate
documents consisting of selected parts of the complete document.

However, there are a few drawbacks of the plain |\include| mechanism:
\begin{itemize}
\item
The child files cannot be compiled on their own,
they can only be compiled via the main file.
A naive editing environment
(such as a text editor with an option
to have the current file processed by \LaTeX)
may require one to switch to the main file before compiling;
attempting to compile the child file produces errors.
\item
The main file must be modified (each time)
to adjust the |\includeonly| command
to the present needs. This easily leaves the main file in a messy state.
\item
The generated document will always carry the filename
of the main document. This is inconvenient if
several child files are to be compiled and
to be kept for distribution.
\end{itemize}

The present package provides a simple interface
to make child files individually compilable by \LaTeX{}.
Compiling a child file then has the same effect as compiling
the main file with an |\includeonly| command
to select the appropriate child.
Moreover the generated document will carry the name of the child
rather than the main file.
This resolves all three above issues.

This feature is meant to make the editing of books,
thesis documents and lecture notes somewhat more convenient.
However, the package can also be used efficiently for
composing a series of documents (such as exercise sheets)
which are typically distributed individually.
It then assists the author in generating the individual documents
(potentially in different versions)
as well as a document containing the collected series.
Another application is in developing style files
or other kinds of included material
where compilation of the style file could redirect
to a sample or test file.

%%%%%%%%%%%%%%%%%%%%%%%%%%%%%%%%%%%%%%%%%%%%%%%%%%%%%%%%%%%%%%%%%%%%%%%%%%%%%%%%
%%%%%%%%%%%%%%%%%%%%%%%%%%%%%%%%%%%%%%%%%%%%%%%%%%%%%%%%%%%%%%%%%%%%%%%%%%%%%%%%
\section{Usage}

First of all, the package \textsf{childdoc} is \emph{not} a standard
\LaTeXe{} |.sty| style file! Therefore it needs to be invoked in
a non-standard way.

%%%%%%%%%%%%%%%%%%%%%%%%%%%%%%%%%%%%%%%%%%%%%%%%%%%%%%%%%%%%%%%%%%%%%%%%%%%%%%%%
\subsection{Included Files}
\label{sec:include}

%%%%%%%%%%%%%%%%%%%%%%%%%%%%%%%%%%%%%%%%
\DescribeMacro{\childdocmain}
To use the package, add the commands
\begin{center}
\begin{tabular}{l}
|\input{childdoc.def}|\\
|\childdocmain{}|\\
\end{tabular}
\end{center}
at the very top of the main \LaTeX{} file,
in particular \emph{before} the |\documentclass| statement!
The argument of |\childdocmain| should be left empty
(but it must be present).

%%%%%%%%%%%%%%%%%%%%%%%%%%%%%%%%%%%%%%%%
\DescribeMacro{\childdocof}
Furthermore, add the commands
\begin{center}
\begin{tabular}{l}
|\input{childdoc.def}|\\
|\childdocof{|\textit{main}|}|\\
\end{tabular}
\end{center}
at the top of every child file \textit{child}
which is included by |\include{|\textit{child}|}|
from within the main file
(or at least for those files to be compiled individually).
The argument \textit{main} must be the filename of the main file.

There are a couple of
considerations in setting up the main and child documents:

%%%%%%%%%%%%%%%%%%%%%%%%%%%%%%%%%%%%%%%%
\paragraph{Restrictions.}

Please note the following restrictions:
\begin{itemize}
\item
|\childdocmain| must be called with one argument \textit{main}
to ensure compatibility with earlier version of the package.
It must either be empty (|\childdocmain{}|)
or precisely match the filename of the main file in which it is specified.
See \secref{sec:detection} for further information.
\item
The filename \textit{main} must be specified without the |.tex| extension.
\item
The filename \textit{main} is case sensitive
(even in case-insensitive file systems)
due to internal string comparison.
\item
The argument \textit{main} should be fully expanded, it cannot be a macro.
\item
Subdirectories and special characters should be avoided in filenames.
\item
The command |\childdocmain{|\textit{main}|}| must be followed by a whitespace.
It should not be followed immediately by another command
or by a comment mark `|%|'.
This is because the \TeX{} parser reads the token immediately following
the argument of |\childdocmain| and puts it
at the beginning of every child section;
however, a white\-space is ignored.
\end{itemize}

%%%%%%%%%%%%%%%%%%%%%%%%%%%%%%%%%%%%%%%%
\paragraph{Content of Main File.}

It is advisable to place all content in the child files included by |\include|.
Any output contained in the main file will appear in all child documents
unless suppressed manually;
it cannot be suppressed automatically by the |\includeonly| directive
and thus should normally be avoided.
A method to include some content in the main file
by means of conditional processing is described in \secref{sec:conditional}.

%%%%%%%%%%%%%%%%%%%%%%%%%%%%%%%%%%%%%%%%
\paragraph{Page Numbering.}

When only a part of the document is compiled,
the appropriate numbering of pages
(as well as other status parameters)
is determined from the |.aux| files.
The latter contain information from previous passes.
However this information needs to propagate through
all intermediate child documents.
Therefore the page numbering in child documents may well
be inconsistent until the complete document is compiled at least once.

A useful (if unconventional) way to always ensure a consistent
page numbering is to restart the numbering in each child document
and denote the pages by `\textit{child}|.|\textit{page}'
where \textit{child} represents the chapter/section number of the child file.
This can be achieved by the command
|\numberwithin{page}{|\textit{child}|}|
of the \textsf{amsmath} package
where \textit{child} can be |chapter| or |section|
depending on the chosen structuring.
Alternatively, one can modify the macro |\thepage| appropriately
and reset the counter |page| at the start of each child file.

%%%%%%%%%%%%%%%%%%%%%%%%%%%%%%%%%%%%%%%%%%%%%%%%%%%%%%%%%%%%%%%%%%%%%%%%%%%%%%%%
\subsection{Conditional Processing}
\label{sec:conditional}

The package provides a mechanism to compile different versions
of a document. To customise the versions further some conditional processing
can come in handy to distinguish which version is being compiled.
The package provides two macros to describe the compilation context:

%%%%%%%%%%%%%%%%%%%%%%%%%%%%%%%%%%%%%%%%
\DescribeMacro{\ifchilddoc}
The conditional |\ifchilddoc| distinguishes between the compilation of
child documents and the main document:
%
\begin{center}
|\ifchilddoc |\textit{child-code}| |[|\||else |\textit{main-code}]| \||fi|
\end{center}

%%%%%%%%%%%%%%%%%%%%%%%%%%%%%%%%%%%%%%%%
\DescribeMacro{\childdocname}
\DescribeMacro{\childdocjob}
The macro |\childdocname| contains the filename (without extension)
of the main or child file being processed.
Note that |\childdocjob| will always contain the name of the main file.

%%%%%%%%%%%%%%%%%%%%%%%%%%%%%%%%%%%%%%%%
\paragraph{Title Page.}

Conditional processing can be used to include a title or banner page
in the main document when proper precautions are taken.
Importantly, the code in the main file should ensure that the page counter
(as well as other status parameters which are stored in the |.aux| files)
takes the same value after the conditional processing.
Otherwise the page numbers may take divergent values
depending on which part is compiled.

For example, a title page could be declared by:
%
\begin{center}
\begin{tabular}{l}
|\ifchilddoc\||else|\\
|\addtocounter{page}{-1}|\\
\textit{code for title page}\\
|\newpage|\\
|\||fi|
\end{tabular}
\end{center}
%
A banner page for the child documents can be generated by:
%
\begin{center}
\begin{tabular}{l}
|\ifchilddoc|\\
|\addtocounter{page}{-1}|\\
\textit{code for banner page}\\
|\newpage|\\
|\||fi|
\end{tabular}
\end{center}
%
Here one could write a message such as:
\begin{center}
|This is the part \childdocname{} of \childdocjob{}.|
\end{center}

%%%%%%%%%%%%%%%%%%%%%%%%%%%%%%%%%%%%%%%%%%%%%%%%%%%%%%%%%%%%%%%%%%%%%%%%%%%%%%%%
\subsection{Flags}
\label{sec:flags}

The package makes it easy to generate different versions
of the main or child documents.
To this end compilation flags can be defined
and assigned different default values.
They will be particularly useful in conjunction
with the forwarding mechanism described in \secref{sec:forward}.

For example, it may be useful to have a flag |\version|
which can be set to |draft| or |final|.
The document source will contain some conditional code
depending on the value of |\version|.
Suppose further, the flag should default to |final| for the main file
and to |draft| for child files
which is a natural assignment for editing the document.
This is achieved by placing the following code
in the preamble of the main document
(below the |\childdocmain| directive):
%
\begin{center}
\begin{tabular}{l}
|\ifchilddoc|\\
|\providecommand{\version}{draft}|\\
|\||else|\\
|\providecommand{\version}{final}|\\
|\||fi|
\end{tabular}
\end{center}
%
The definition by |\providecommand| makes sure
that previous definitions are not overwritten.
Further statements |\providecommand{\version}{...}|
can thus be added before the above code to override it.

For the main file, one might add a line
(between |\childdocmain| and the above block)
%
\begin{center}
|%\ifchilddoc\||else\providecommand{\version}{draft}\||fi|
\end{center}
%
which can be uncommented to produce a draft version.
Likewise one can add a line to the very top of a child file
(above the |\childdocof{|\textit{main}|}| directive)
%
\begin{center}
|%\providecommand{\version}{final}|
\end{center}
%
which can be uncommented to produce the final version of this child document.

%%%%%%%%%%%%%%%%%%%%%%%%%%%%%%%%%%%%%%%%%%%%%%%%%%%%%%%%%%%%%%%%%%%%%%%%%%%%%%%%
\subsection{Forwarding}
\label{sec:forward}

Different versions of the main or child documents
using compilation flags as described in \secref{sec:flags}
can be (permanently) stored in different files
for convenient compilation, viewing and distribution.
To this end, the package defines a command
to pass on compilation to a different file:

%%%%%%%%%%%%%%%%%%%%%%%%%%%%%%%%%%%%%%%%
\DescribeMacro{\childdocforward}
The command |\childdocforward| redirects processing to
another source file:
%
\begin{center}
\begin{tabular}{l}
|\input{childdoc.def}|\\
|\childdocforward[|\textit{main}|]{|\textit{dest}|}|\\
\end{tabular}
\end{center}
%
The argument \textit{dest} is the destination file
(without extension).
It should be the main file or one of the child files.
Note that further \textsf{childdoc} directives
such as |\childdocof| and |\childdocforward|
in the indicated file will be processed in this form.
The optional argument \textit{main}
passes on directly to the main file \textit{main}
while pretending to compile the child \textit{dest}.
This form behaves as if \textit{dest}
issues |\childdocof{|\textit{main}|}| right away,
and no further \textsf{childdoc} directives will be processed.

%%%%%%%%%%%%%%%%%%%%%%%%%%%%%%%%%%%%%%%%
\DescribeMacro{\...prefix}
In the alternative form |\childdocforwardprefix|,
%
\begin{center}
\begin{tabular}{l}
|\input{childdoc.def}|\\
|\childdocforwardprefix[|\textit{main}|]{|\textit{prefix}|}{|\textit{dest}|}|
\end{tabular}
\end{center}
%
the destination file is determined by a pattern
depending on the current file:
To make this work, the current file must be called
`{\textit{prefix}\hspace{0.2em}\textit{suffix}}'
with \textit{prefix} matching precisely the argument.
Processing is then passed on to the file
`{\textit{dest}\hspace{0.2em}\textit{suffix}}'.
Surely, the same effect is achieved by
directly specifying the
argument `{\textit{dest}\hspace{0.2em}\textit{suffix}}'
in the first form.
However, that requires to set up a different file
for each child. With the alternative form of the command
all these files can have exactly the same content
which simplifies setting them up and maintaining them.

For example, the following file |draft.tex|
with a compilation flag |\version| as described in \secref{sec:flags}
compiles the main document as a draft:
%
\begin{center}
\begin{tabular}{l}
|\def\version{draft}|\\
|\input{childdoc.def}|\\
|\childdocforward{|\textit{main}|}|
\end{tabular}
\end{center}
%
Likewise, the following files |final|\textit{nn}|.tex|
compile the final version of the child document
|child|\textit{nn}|.tex|:
%
\begin{center}
\begin{tabular}{l}
|\def\version{final}|\\
|\input{childdoc.def}|\\
|\childdocforwardprefix{final}{child}|
\end{tabular}
\end{center}
%

Note that when several versions of a main file and/or of each child file
are to be generated, it may be convenient to set up a |Makefile| or
shell script to automatise the process.

%%%%%%%%%%%%%%%%%%%%%%%%%%%%%%%%%%%%%%%%%%%%%%%%%%%%%%%%%%%%%%%%%%%%%%%%%%%%%%%%
\subsection{Command Line Processing}
\label{sec:commandline}

The effect of redirection files can also be achieved by invoking
the \LaTeX{} compiler with a more elaborate command line.
Most conveniently this should be done as part
of a shell script or a |Makefile|.

When using \textsf{childdoc} in the main file, the following
command lines effectively perform a redirection
(note that depending on the shell being used,
backslashes may have to be doubled: `|\|' $\to$ `|\\|'):
%
\begin{center}
|... -jobname "|\textit{target}|" |\\|"|[\textit{flags}]%
|\input{childdoc.def}\childdocforward[|\textit{main}|]{|\textit{dest}|}"|
\end{center}
%
Here \textit{target} is the name of the output file,
\textit{main} is the name of the main file
and \textit{dest} is the name of the main or child file to be processed
(all filenames without extensions).
The optional argument \textit{main} can be omitted
if \textit{main} matches \textit{dest}.
Optionally, compilation \textit{flags} can be defined via |\def| commands.
This command line makes the \TeX{} engine believe
it is compiling the file \textit{target}
whose content is specified as the latter parameter.
The provided code then forwards the processing to
\textit{main} or \textit{dest} as described in \secref{sec:forward}.

%%%%%%%%%%%%%%%%%%%%%%%%%%%%%%%%%%%%%%%%%%%%%%%%%%%%%%%%%%%%%%%%%%%%%%%%%%%%%%%%
\subsection{Include by Input}
\label{sec:input}

Including child documents by |\include| has some restrictions by design.
Most notably, the content of a child document always occupies
its own set of pages; pages cannot be shared between child documents.
Usually, this behaviour makes perfect sense
because each child document contain an essential part of the document.
However, in some situations it may be desirable to compose
a document from a collection of parts
without having mandatory page breaks between then.
For this case, the package
provides a mechanism to include parts
by |\input| which can also be processed individually.
However, by construction this mechanism
requires manual handling of the content to be output.

%%%%%%%%%%%%%%%%%%%%%%%%%%%%%%%%%%%%%%%%
\DescribeMacro{\ifchilddocmanual}
The main file should be prepared as usual, see \secref{sec:include}.
However, the document body must make a distinction
between processing of an individual part and of the main document, e.g.:
%
\begin{center}
\begin{tabular}{l}
|\ifchilddocmanual|\\
|\input{\childdocname}|\\
|\||else|\\
\textit{document body with }|\input{|\textit{part}|}|\\
|\||fi|
\end{tabular}
\end{center}
%
The conditional |\ifchilddocmanual| is true whenever
a part to be included by |\input| is being compiled,
and the name of the part is stored in |\childdocname|.

%%%%%%%%%%%%%%%%%%%%%%%%%%%%%%%%%%%%%%%%
\DescribeMacro{\childdocby}
Each part to be included by |\input| should start with:
%
\begin{center}
\begin{tabular}{l}
|\input{childdoc.def}|\\
|\childdocby{|\textit{main}|}|\\
\end{tabular}
\end{center}
%
The directive |\childdocby| is similar to |\childdocof|
described in \secref{sec:include},
but the subsequent selection of content must be done manually.
To that end, both |\ifchilddoc| and |\ifchilddocmanual|
will be true upon processing of a part,
and the name of the part is stored in |\childdocname|.
Note that |\jobname| will be set to the filename of the current part
so that each part receives an individual |.aux| file
that does not interfere with the |.aux| file(s) of the main document.
This behaviour can be altered by the alternative form
|\childdocby[*]{|\textit{main}|}| (with a non-empty optional argument)
which uses the |.aux| file of the main document
by setting |\jobname| to \textit{main}.

%%%%%%%%%%%%%%%%%%%%%%%%%%%%%%%%%%%%%%%%%%%%%%%%%%%%%%%%%%%%%%%%%%%%%%%%%%%%%%%%
\subsection{Driver Development}
\label{sec:driver}

The \textsf{childdoc} mechanism can also be use for the development
of definition files such as \LaTeX{} styles or classes.
This case differs from the above setup with multiple parts
included by |\include| in that no |\includeonly| should be invoked.
This can be achieved by starting the include file
(before |\ProvidesPackage|) with:
%
\begin{center}
\begin{tabular}{l}
|\input{childdoc.def}|\\
|\childdocforward{|\textit{main}|}|\\
\end{tabular}
\end{center}
%
or alternatively with:
%
\begin{center}
\begin{tabular}{l}
|\input{childdoc.def}|\\
|\childdocby{|\textit{main}|}|\\
\end{tabular}
\end{center}
%
Both forms have slightly different effects as described above.
The main file is prepared as usual, see \secref{sec:include}.

%%%%%%%%%%%%%%%%%%%%%%%%%%%%%%%%%%%%%%%%%%%%%%%%%%%%%%%%%%%%%%%%%%%%%%%%%%%%%%%%
\subsection{Legacy Detection}
\label{sec:detection}

The directive |\childdocmain| in the main file can detect
whether the complete document or merely a child is to be compiled
even without using the directive |\childdocof|.
This method is deprecated because it is less robust
and there is no compelling reason to use it;
it is merely provided for backward compatibility
and it may be removed in future versions.

If the detection mechanism is to be used,
it is mandatory to correctly specify
the filename of the main file as the argument of |\childdocmain|:
%
\begin{center}
\begin{tabular}{l}
|\input{childdoc.def}|\\
|\childdocmain{|\textit{main}|}|\\
\end{tabular}
\end{center}
%
If |\jobname| does not match the argument \textit{main} of |\childdocmain|,
it is assumed that |\jobname| points to the child file to be compiled.
When using |\childdocmain| with the main file specified as argument,
it suffices to start a child file
with just |\input{|\textit{main}|}|
without loading of the package and using |\childdocof|.
If instead all processing is done
with the appropriate \textsf{childdoc} directives,
the argument of \textit{main} of |\childdocmain| can be empty.

An alternative version of the command line processing described
in \secref{sec:commandline} using the detection mechanism reads:
%
\begin{center}
|... -jobname "|\textit{target}|" "|[\textit{flags}]%
[|\def\jobname{|\textit{dest}|}|]|\input{|\textit{main}|}"|
\end{center}

%%%%%%%%%%%%%%%%%%%%%%%%%%%%%%%%%%%%%%%%%%%%%%%%%%%%%%%%%%%%%%%%%%%%%%%%%%%%%%%%
\subsection{Manual Code}
\label{sec:manual}

In case one cannot be certain whether the definitions file |childdoc.def|
is installed on the target \TeX{} distribution
and one prefers not to ship it,
it is conceivable to paste a few relevant commands into the sources.

To that end, drop all statements |\input{childdoc.def}|
and perform the replacements as outlined below.
Instead of |\childdocmain{|\textit{main}|}| add the following code
to the top of the main file:
%
\begin{center}
\begin{tabular}{l}
|\||ifdefined\childdocname\endinput\||fi\newif\ifchilddoc|\\
|\edef\childdocname{\scantokens\expandafter{\jobname\noexpand}}|\\
|\def\childdocmain{|\textit{main}|}\||ifx\childdocmain\childdocname\||else|\\
|\childdoctrue\includeonly{\childdocname}\let\jobname\childdocmain\||fi|\\
\end{tabular}
\end{center}
%
Instead of |\childdocof{|\textit{main}|}| just include the main file
at the top of each child file:
%
\begin{center}
|\input{|\textit{main}|}|
\end{center}
%
A simple redirection |\childdocforward{|\textit{dest}|}| is achieved by:
%
\begin{center}
|\def\jobname{|\textit{dest}|}\input{\jobname}|
\end{center}
%
The redirection with prefix
|\childdocforwardprefix[|\textit{prefix}|]{|\textit{dest}|}|
is accomplished by:
%
\begin{center}
\begin{tabular}{l}
|{\edef\jobname{\scantokens\expandafter{\jobname\noexpand}}|\\
|\def\redirectjob |\textit{prefix}|#1~~~{\gdef\jobname{|\textit{dest}|#1}}|\\
|\expandafter\redirectjob\jobname~~~}\input{\jobname}|
\end{tabular}
\end{center}

In an alternative approach,
child documents can be compiled by a specific command line
without additional code or specific definitions:
%
\begin{center}
|... -jobname "|\textit{target}|" "|[\textit{flags}]%
|\includeonly{|\textit{dest}|}\input{|\textit{main}|}"|
\end{center}
%

%%%%%%%%%%%%%%%%%%%%%%%%%%%%%%%%%%%%%%%%%%%%%%%%%%%%%%%%%%%%%%%%%%%%%%%%%%%%%%%%
%%%%%%%%%%%%%%%%%%%%%%%%%%%%%%%%%%%%%%%%%%%%%%%%%%%%%%%%%%%%%%%%%%%%%%%%%%%%%%%%
\section{Information}

%%%%%%%%%%%%%%%%%%%%%%%%%%%%%%%%%%%%%%%%%%%%%%%%%%%%%%%%%%%%%%%%%%%%%%%%%%%%%%%%
\subsection{Copyright}

Copyright \copyright{} 2017--2018 Niklas Beisert

This work may be distributed and/or modified under the
conditions of the \LaTeX{} Project Public License, either version 1.3
of this license or (at your option) any later version.
The latest version of this license is in
  \url{http://www.latex-project.org/lppl.txt}
and version 1.3 or later is part of all distributions of \LaTeX{}
version 2005/12/01 or later.

This work has the LPPL maintenance status `maintained'.

The Current Maintainer of this work is Niklas Beisert.

This work consists of the files |README.txt|, |childdoc.ins| and |childdoc.dtx|
as well as the derived files |childdoc.def|, |cdocsamp.tex|
with |cdocsch1.tex|, |cdocsch2.tex|, |cdocspt3.tex|, |cdocspt4.tex|,
|cdocsdrf.tex|, |cdocsfn1.tex|, |cdocsfn2.tex|
as well as |childdoc.pdf|.

%%%%%%%%%%%%%%%%%%%%%%%%%%%%%%%%%%%%%%%%%%%%%%%%%%%%%%%%%%%%%%%%%%%%%%%%%%%%%%%%
\subsection{Files and Installation}

The package consists of the files:
%
\begin{center}
\begin{tabular}{ll}
    |README.txt|   & readme file \\
    |childdoc.ins| & installation file \\
    |childdoc.dtx| & source file \\
    |childdoc.def| & definition file \\
    |cdocsamp.tex| & sample main file \\
    |cdocsch1.tex| & sample include file \\
    |cdocsch2.tex| & sample include file \\
    |cdocspt3.tex| & sample part file \\
    |cdocspt4.tex| & sample part file \\
    |cdocsdrf.tex| & sample redirection file \\
    |cdocsfn1.tex| & sample redirection file \\
    |cdocsfn2.tex| & sample redirection file \\
    |childdoc.pdf| & manual
\end{tabular}
\end{center}
%
The distribution consists of the files
|README.txt|, |childdoc.ins| and |childdoc.dtx|.
%
\begin{itemize}
\item
Run (pdf)\LaTeX{} on |childdoc.dtx|
to compile the manual |childdoc.pdf| (this file).
\item
Run \LaTeX{} on |childdoc.ins| to create the definitions file |childdoc.def|
and the sample |cdocsamp.tex| with include files
|cdocsch1.tex|, |cdocsch2.tex|, |cdocspt3.tex|, |cdocspt4.tex|,
|cdocsdrf.tex|, |cdocsfn1.tex|, |cdocsfn2.tex|.
Then copy the file |childdoc.def| to an appropriate directory of your \LaTeX{}
distribution, e.g.\ \textit{texmf-root}|/tex/latex/childdoc|.
\end{itemize}

%%%%%%%%%%%%%%%%%%%%%%%%%%%%%%%%%%%%%%%%%%%%%%%%%%%%%%%%%%%%%%%%%%%%%%%%%%%%%%%%
\subsection{Related CTAN Packages}

There are several other packages which offer a similar functionality:
%
\begin{itemize}
\item
The packages
\href{http://ctan.org/pkg/docmute}{\textsf{docmute}},
\href{http://ctan.org/pkg/includex}{\textsf{includex}} and
\href{http://ctan.org/pkg/standalone}{\textsf{standalone}}
provide commands to include only the document body of
a child file thus allowing both files to be compiled individually.
\item
The packages \href{http://ctan.org/pkg/subdocs}{\textsf{subdocs}}
and \href{http://ctan.org/pkg/subfiles}{\textsf{subfiles}}
provide structures in which the main and child documents can be
encapsulated and allowing them to be compiled individually.
The inclusion mechanism is different from the conventional |\include|.
\item
The package \href{http://ctan.org/pkg/combine}{\textsf{combine}}
is an elaborate solution to combine several documents into one.
\end{itemize}
%
See also the CTAN topic \href{http://ctan.org/topic/subdocs}{\textsf{subdocs}}
for further related packages.
The present package differs from the above solutions in that
a document structure constructed with the conventional |\include| mechanism
just needs two extra commands at the top of every file
such that all constituent files can be compiled individually.

%%%%%%%%%%%%%%%%%%%%%%%%%%%%%%%%%%%%%%%%%%%%%%%%%%%%%%%%%%%%%%%%%%%%%%%%%%%%%%%%
%\subsection{Feature Suggestions}
%
%The following is a list of features which may be useful for future
%versions of this package:
%%
%\begin{itemize}
%\item
%\ldots
%\end{itemize}

%%%%%%%%%%%%%%%%%%%%%%%%%%%%%%%%%%%%%%%%%%%%%%%%%%%%%%%%%%%%%%%%%%%%%%%%%%%%%%%%
\subsection{Revision History}

%%%%%%%%%%%%%%%%%%%%%%%%%%%%%%%%%%%%%%%%
\paragraph{v2.0:} 2018/12/30

\begin{itemize}
\item
immediate forward processing
\item
added |\childdocby| mechanism
\item
manual restructured
\end{itemize}

%%%%%%%%%%%%%%%%%%%%%%%%%%%%%%%%%%%%%%%%
\paragraph{v1.6:} 2018/01/17

\begin{itemize}
\item
application for development of include files
\item
corrections to manual
\end{itemize}

%%%%%%%%%%%%%%%%%%%%%%%%%%%%%%%%%%%%%%%%
\paragraph{v1.5:} 2017/05/21

\begin{itemize}
\item
more complete structuring introduced
\item
|\childdocof| introduced
\item
|\childdoc| renamed to |\childdocmain|
\item
|\childredirect| renamed to |\childdocforward| and |\childdocforwardprefix|
and functionality expanded
\end{itemize}

%%%%%%%%%%%%%%%%%%%%%%%%%%%%%%%%%%%%%%%%
\paragraph{v1.0:} 2017/04/27

\begin{itemize}
\item
manual and install package
\item
first version published on CTAN
\end{itemize}

%%%%%%%%%%%%%%%%%%%%%%%%%%%%%%%%%%%%%%%%
\paragraph{v0.6:} 2017/04/26

\begin{itemize}
\item
redirection mechanism added
\end{itemize}

%%%%%%%%%%%%%%%%%%%%%%%%%%%%%%%%%%%%%%%%
\paragraph{v0.5:} 2017/04/26

\begin{itemize}
\item
functionality in definition file
\end{itemize}


%%%%%%%%%%%%%%%%%%%%%%%%%%%%%%%%%%%%%%%%%%%%%%%%%%%%%%%%%%%%%%%%%%%%%%%%%%%%%%%%
%%%%%%%%%%%%%%%%%%%%%%%%%%%%%%%%%%%%%%%%%%%%%%%%%%%%%%%%%%%%%%%%%%%%%%%%%%%%%%%%
%%%%%%%%%%%%%%%%%%%%%%%%%%%%%%%%%%%%%%%%%%%%%%%%%%%%%%%%%%%%%%%%%%%%%%%%%%%%%%%%
\appendix

\settowidth\MacroIndent{\rmfamily\scriptsize 000\ }

 \DocInput{childdoc.dtx}

\end{document}
%</driver>
% \fi
%
% %%%%%%%%%%%%%%%%%%%%%%%%%%%%%%%%%%%%%%%%%%%%%%%%%%%%%%%%%%%%%%%%%%%%%%%%%%%%%%
% %%%%%%%%%%%%%%%%%%%%%%%%%%%%%%%%%%%%%%%%%%%%%%%%%%%%%%%%%%%%%%%%%%%%%%%%%%%%%%
% \section{Sample}
%\iffalse
%<*samplemain>
%\fi
%
% The following presents a sample document
% with two chapters, two parts, a title page,
% a compile flag as well as three forwarding files to set the flag.
% It consists of eight |.tex| files:
% \begin{center}
% \begin{tabular}{ll}
% |cdocsamp.tex|&main file\\
% |cdocsch1.tex|&include file for chapter 1\\
% |cdocsch2.tex|&include file for chapter 2\\
% |cdocspt3.tex|&include file for part 3\\
% |cdocspt4.tex|&include file for part 4\\
% |cdocsdrf.tex|&forwarding file for main file in draft mode\\
% |cdocsfi1.tex|&forwarding file for final version of chapter 1\\
% |cdocsfi2.tex|&forwarding file for final version of chapter 2\\
% \end{tabular}
% \end{center}
% Each of the eight files can be compiled directly by the \LaTeX{} compiler.
%
% %%%%%%%%%%%%%%%%%%%%%%%%%%%%%%%%%%%%%%
% \paragraph{Main File.}
%
% The main file is called |cdocsamp.tex|.
%
% Load the \textsf{childdoc} definitions and
% declare the filename for the main document:
%    \begin{macrocode}
\input{childdoc.def}
\childdocmain{}
%    \end{macrocode}

% Optional override for |\version| flag:
%    \begin{macrocode}
%%\ifchilddoc\else\providecommand{\version}{draft}\fi
%    \end{macrocode}

% Define the default values for the |\version| flag
% (|final| for the main file and |draft| for childs):
%    \begin{macrocode}
\ifchilddoc
\providecommand{\version}{draft}
\else
\providecommand{\version}{final}
\fi
%    \end{macrocode}

% Load the standard document class:
%    \begin{macrocode}
\documentclass[12pt]{article}
%    \end{macrocode}

% Start the document body:
%    \begin{macrocode}
\begin{document}
%    \end{macrocode}

% Declare a title page.
% Print title, part of document being processed and version flag:
%    \begin{macrocode}
\addtocounter{page}{-1}
\begin{center}
{\LARGE\bfseries{}childdoc example\par}
\vspace{1cm}
\ifchilddoc
\ifchilddocmanual part\else chapter\fi:
`\childdocname' of `\childdocjob'\par
\else
main document: `\childdocjob'\par
\fi
version: \version\par
\end{center}
\newpage
%    \end{macrocode}

% Manually include selected file,
% otherwise process as usual:
%    \begin{macrocode}
\ifchilddocmanual
\section*{part `\childdocname'}
\input{\childdocname}
\else
%    \end{macrocode}

% Include the two chapters:
%    \begin{macrocode}
\include{cdocsch1}
\include{cdocsch2}
%    \end{macrocode}

% Include the two parts unless only chapters should be displayed:
%    \begin{macrocode}
\ifchilddoc\else
\section{part three}
\input{cdocspt3}
\section{part four}
\input{cdocspt4}
\fi
%    \end{macrocode}

% Process as usual until here:
%    \begin{macrocode}
\fi
%    \end{macrocode}

% End of document body:
%    \begin{macrocode}
\end{document}
%    \end{macrocode}
%\iffalse
%</samplemain>
%\fi
%
% %%%%%%%%%%%%%%%%%%%%%%%%%%%%%%%%%%%%%%
% \paragraph{Chapter Include Files.}
%
% The include files are called |cdocsch1.tex| and |cdocsch2.tex|.
%
%\iffalse
%<*samplechap1|samplechap2>
%\fi

% Optional override for |\version| flag:
%    \begin{macrocode}
%%\providecommand{\version}{final}
%    \end{macrocode}

% Include the main document:
%    \begin{macrocode}
\input{childdoc.def}
\childdocof{cdocsamp}
%    \end{macrocode}

%\iffalse
%</samplechap1|samplechap2>
%\fi
%
%\iffalse
%<*samplechap1>
%\fi
% Some text for chapter 1:
%    \begin{macrocode}
\section{one}
some text in chapter one
%    \end{macrocode}

%\iffalse
%</samplechap1>
%\fi
% Some text for chapter 2:
%\iffalse
%<*samplechap2>
%\fi
%    \begin{macrocode}
\section{two}
more text in chapter two
%    \end{macrocode}

%\iffalse
%</samplechap2>
%\fi
%
% %%%%%%%%%%%%%%%%%%%%%%%%%%%%%%%%%%%%%%
% \paragraph{Part Include Files.}
%
% The include files are called |cdocspt3.tex| and |cdocspt4.tex|.
%
%\iffalse
%<*samplepart3|samplepart4>
%\fi

% Optional override for |\version| flag:
%    \begin{macrocode}
%%\providecommand{\version}{final}
%    \end{macrocode}

% Include the main document:
%    \begin{macrocode}
\input{childdoc.def}
\childdocby{cdocsamp}
%    \end{macrocode}

%\iffalse
%</samplepart3|samplepart4>
%\fi
%
%\iffalse
%<*samplepart3>
%\fi
% Some text for part 3:
%    \begin{macrocode}
some text in part three
%    \end{macrocode}

%\iffalse
%</samplepart3>
%\fi
% Some text for part 4:
%\iffalse
%<*samplepart4>
%\fi
%    \begin{macrocode}
more text in part four
%    \end{macrocode}

%\iffalse
%</samplepart4>
%\fi
%
% %%%%%%%%%%%%%%%%%%%%%%%%%%%%%%%%%%%%%%
% \paragraph{Forwarding for a Complete Draft.}
%
% The following forwarding file |cdocsdrf.tex|
% compiles the main document in draft mode:
%\iffalse
%<*sampledraft>
%\fi
%    \begin{macrocode}
\def\version{draft}
\input{childdoc.def}
\childdocforward{cdocsamp}
%    \end{macrocode}

%\iffalse
%</sampledraft>
%\fi
%
% %%%%%%%%%%%%%%%%%%%%%%%%%%%%%%%%%%%%%%
% \paragraph{Forwarding for Final Version of the Chapters.}
%
% The following forwarding files |cdocsfn1.tex| and |cdocsfn2.tex|
% (with identical content)
% compile the final versions of the child documents
% |cdocsch1.tex| and |cdocsch2.tex|, respectively:
%\iffalse
%<*samplefinal>
%\fi
%    \begin{macrocode}
\def\version{final}
\input{childdoc.def}
\childdocforwardprefix[cdocsamp]{cdocsfn}{cdocsch}
%    \end{macrocode}

%\iffalse
%</samplefinal>
%\fi
%
% %%%%%%%%%%%%%%%%%%%%%%%%%%%%%%%%%%%%%%
% \paragraph{Command Line Processing.}
%
% The following three command lines generate the output files
% |cdocscld|, |cdocscl1| and |cdocscl2|
% which should be identical to
% |cdocsdrf|, |cdocsch1| and |cdocsfn2|, respectively:
% \begin{center}
% \begin{tabular}{l}
% |latex -jobname cdocscld \|\\
% |  "\def\version{draft}\input{childdoc.def}\childdocforward{cdocsamp}"|\\
% |latex -jobname cdocscl1 \|\\
% |  "\input{childdoc.def}\childdocforward[cdocsamp]{cdocsch1}"|\\
% |latex -jobname cdocscl2 \|\\
% |  "\def\version{final}\input{childdoc.def}\childdocforward{cdocsch2}"|
% \end{tabular}
% \end{center}
% Note that the trailing backslash on each first line
% merely continues the input to the second line
% (for convenient cut ant paste).
% Furthermore, the command |latex| can be replaced by any
% of its alternative versions such as |pdflatex|.
%
% %%%%%%%%%%%%%%%%%%%%%%%%%%%%%%%%%%%%%%%%%%%%%%%%%%%%%%%%%%%%%%%%%%%%%%%%%%%%%%
% %%%%%%%%%%%%%%%%%%%%%%%%%%%%%%%%%%%%%%%%%%%%%%%%%%%%%%%%%%%%%%%%%%%%%%%%%%%%%%
% \section{Implementation}
%\iffalse
%<*package>
%\fi
%
% This section describes the definitions file |childdoc.def|.

% The definitions cannot be loaded using |\usepackage| or |\RequirePackage|
% which has a mechanism to prevent loading a style file more than once.
% When loading the definitions by means of |\input|
% multiple instances have to be prevented manually:
%\iffalse
%This code needs to be before the `\ProvidesFile' directive
%which is defined at the beginning of this file.
%Therefore it is also placed there and commented out here.
%</package>
%<*discard>
%\fi
%    \begin{macrocode}
\ifdefined\childdocmain\endinput\fi
%    \end{macrocode}
%\iffalse
%</discard>
%<*package>
%\fi
%
% \macro{\ifchilddoc}
% \macro{\ifchilddocmanual}
% The conditional |\ifchilddoc| tells whether a
% child (true) or main (false) document is being compiled.
% The conditional |\ifchilddocmanual| tells whether
% the |\includeonly| mechanism is used (false) or
% the selection of child files must be performed manually (true).
% The definitions initialise to false:
%    \begin{macrocode}
\newif\ifchilddoc
\newif\ifchilddocmanual
%    \end{macrocode}

% \macro{\childdocname}
% \macro{\childdocjob}
% The macro |\childdocname| stores the name of the main document
% to be compiled. The macro |\childdocjob| stores the name of
% the document on which the \LaTeX{} compiler was originally invoked.
% The content of |\jobname| cannot be compared
% to filenames specified in the source due to different catcodes.
% The following code rescans |\jobname|, stores the result
% in |\childdocname| and saves a copy in |\childdocjob|:
%    \begin{macrocode}
\edef\childdocname{\scantokens\expandafter{\jobname\noexpand}}
\let\childdocjob\childdocname
%    \end{macrocode}

% \macro{\childdocdisable}
% The macro |\childdocdisable| prevents the main file
% from being processed more than once.
% At this stage, the main document command |\childdocmain|
% is assumed to be called once again where it should do nothing.
% Any subsequent call to it should prevent
% a secondary processing of the main document
% It overwrites the forwarding commands
% |\childdocof| and |\childdocforward|
% with empty macros to prevent further inclusions of the main document:
%    \begin{macrocode}
\newcommand{\childdocdisable}
{
  \renewcommand{\childdocmain}[1]{\renewcommand{\childdocmain}[1]{\endinput}}
  \renewcommand{\childdocof}[1]{}
  \renewcommand{\childdocby}[2][]{}
  \renewcommand{\childdocforward}[2][]{}
  \renewcommand{\childdocdisable}{}
}
%    \end{macrocode}

% \macro{\childdocmain}
% The macro |\childdocmain| is to be called at the top of the main file
% with nothing or the main filename (without extension) as argument.
% First, it breaks loops.
% If the argument is not empty and does not match |\childdocname|
% (which is set by the first inclusion of |childdoc.def|),
% |\ifchilddoc| is set to true, |\includeonly| is applied to the child file
% and |\jobname| is set to the main file
% (for proper handling of |.aux| files):
%    \begin{macrocode}
\newcommand{\childdocmain}[1]
{
  \childdocdisable\childdocmain{}
  \if?#1?\else
    \begingroup
      \def\childdoctmp{#1}
      \ifx\childdoctmp\childdocname
        \def\childdoctmp{}
      \else
        \def\childdoctmp
        {
          \childdoctrue
          \includeonly{\childdocname}
          \def\childdocjob{#1}
          \def\jobname{#1}
        }
      \fi
      \expandafter
    \endgroup
    \childdoctmp
  \fi
}
%    \end{macrocode}

% \macro{\childdocof}
% The command |\childdocof| redirects
% compilation to the main file |#1|.
%    \begin{macrocode}
\newcommand{\childdocof}[1]
{
  \childdocdisable
  \childdoctrue
  \includeonly{\childdocname}
  \def\jobname{#1}
  \def\childdocjob{#1}
  \input{#1}
}
%    \end{macrocode}

% \macro{\childdocby}
% The command |\childdocby| ....
%    \begin{macrocode}
\newcommand{\childdocby}[2][]
{
  \childdocdisable
  \childdoctrue
  \childdocmanualtrue
  \if?#1?\else
    \def\jobname{#2}
  \fi
  \def\childdocjob{#2}
  \input{#2}
  \endinput
}
%    \end{macrocode}

% \macro{\childdocforward}
% The command |\childdocforward| redirects
% compilation to the main file or
% (if the optional argument is given) a child file.
% Parameters are set as if the main file
% or a child file starting with |\childdocof| was compiled.
% Then compilation is handed over to the main file:
%    \begin{macrocode}
\newcommand{\childdocforward}[2][]
{
  \begingroup
    \if?#1?
      \def\childdoctmp
      {
        \def\childdocname{#2}
        \def\childdocjob{#2}
        \def\jobname{#2}
        \input{#2}
        \endinput
      }
    \else
      \def\childdoctmp
      {
        \childdocdisable
        \def\childdocname{#2}
        \childdoctrue
        \includeonly{#2}
        \def\childdocjob{#1}
        \def\jobname{#1}
        \input{#1}
        \endinput
      }
    \fi
    \expandafter
  \endgroup
  \childdoctmp
}
%    \end{macrocode}

% \macro{\childdocforwardprefix}
% The command |\childdocforwardprefix| redirects
% compilation to the main or a child file by means of a pattern.
% The prefix |#1| in the current filename is replaced by |#2|
% and the suffix of the current filename is kept
% (it is assumed that the filename does not contain the substring `|~~~|'
% which is used as a delimiter).
% Compilation is handed over to the new file by |\childdocforward|:
%    \begin{macrocode}
\newcommand{\childdocforwardprefix}[3][]
{
  \begingroup
    \def\childdocextract #2##1~~~{\def\childdoctmp{\childdocforward[#1]{#3##1}}}
    \expandafter\childdocextract\childdocname~~~
    \expandafter
  \endgroup
  \childdoctmp
}
%    \end{macrocode}

% \macro{\childdoc}
% The deprecated macro |\childdoc| is a legacy version of |\childdocmain|:
%    \begin{macrocode}
\newcommand{\childdoc}{\childdocmain}
%    \end{macrocode}

% \macro{\childdocredirect}
% The deprecated macro |\childdocredirect| is a legacy version
% of |\childdocforward| and |\childdocforwardprefix|:
%    \begin{macrocode}
\newcommand{\childdocredirect}[2][]
{
  \begingroup
    \if?#1?
      \def\childdoctmp{\childdocforward{#2}}
    \else
      \def\childdoctmp{\childdocforwardprefix{#1}{#2}}
    \fi
    \expandafter
  \endgroup
  \childdoctmp
}
%    \end{macrocode}

%\iffalse
%</package>
%\fi
%
\endinput

\childdocby{cdocsamp}
%    \end{macrocode}

%\iffalse
%</samplepart3|samplepart4>
%\fi
%
%\iffalse
%<*samplepart3>
%\fi
% Some text for part 3:
%    \begin{macrocode}
some text in part three
%    \end{macrocode}

%\iffalse
%</samplepart3>
%\fi
% Some text for part 4:
%\iffalse
%<*samplepart4>
%\fi
%    \begin{macrocode}
more text in part four
%    \end{macrocode}

%\iffalse
%</samplepart4>
%\fi
%
% %%%%%%%%%%%%%%%%%%%%%%%%%%%%%%%%%%%%%%
% \paragraph{Forwarding for a Complete Draft.}
%
% The following forwarding file |cdocsdrf.tex|
% compiles the main document in draft mode:
%\iffalse
%<*sampledraft>
%\fi
%    \begin{macrocode}
\def\version{draft}
% \iffalse
%
% childdoc.dtx Copyright (C) 2017-2018 Niklas Beisert
%
% This work may be distributed and/or modified under the
% conditions of the LaTeX Project Public License, either version 1.3
% of this license or (at your option) any later version.
% The latest version of this license is in
%   http://www.latex-project.org/lppl.txt
% and version 1.3 or later is part of all distributions of LaTeX
% version 2005/12/01 or later.
%
% This work has the LPPL maintenance status `maintained'.
%
% The Current Maintainer of this work is Niklas Beisert.
%
% This work consists of the files childdoc.dtx and childdoc.ins
% and the derived files childdoc.def and cdocsamp.tex with
% cdocsch1.tex, cdocsch2.tex, cdocsdrf.tex, cdocsfn1.tex, cdocsfn2.tex.
%
%<package>\ifdefined\childdocmain\endinput\fi
%<package>\ProvidesFile{childdoc.def}[2018/12/30 v2.0 child document driver]
%<samplemain>\ProvidesFile{cdocsamp.tex}[2018/12/30 v2.0 sample for childdoc]
%<*driver>
%\ProvidesFile{childdoc.drv}[2018/12/30 v2.0 childdoc reference manual file]
\PassOptionsToClass{10pt,a4paper}{article}
\documentclass{ltxdoc}

\usepackage[margin=35mm]{geometry}
\usepackage{hyperref}
\usepackage{hyperxmp}
\usepackage[usenames]{color}

\hypersetup{colorlinks=true}
\hypersetup{pdfstartview=FitH}
\hypersetup{pdfpagemode=UseNone}
\hypersetup{pdfsource={}}
\hypersetup{pdflang={en-UK}}
\hypersetup{pdfcopyright={Copyright 2017-2018 Niklas Beisert.
  This work may be distributed and/or modified under the
  conditions of the LaTeX Project Public License, either version 1.3
  of this license or (at your option) any later version.}}
\hypersetup{pdflicenseurl={http://www.latex-project.org/lppl.txt}}
\hypersetup{pdfcontactaddress={ETH Zurich, ITP, HIT K,
  Wolfgang-Pauli-Strasse 27}}
\hypersetup{pdfcontactpostcode={8093}}
\hypersetup{pdfcontactcity={Zurich}}
\hypersetup{pdfcontactcountry={Switzerland}}
\hypersetup{pdfcontactemail={nbeisert@itp.phys.ethz.ch}}
\hypersetup{pdfcontacturl={http://people.phys.ethz.ch/\xmptilde nbeisert/}}

\newcommand{\secref}[1]{\hyperref[#1]{section \ref*{#1}}}

\parskip1ex
\parindent0pt
\let\olditemize\itemize
\def\itemize{\olditemize\parskip0pt}

\begin{document}

\title{The \textsf{childdoc} Package}
\hypersetup{pdftitle={The childdoc Package}}
\author{Niklas Beisert\\[2ex]
  Institut f\"ur Theoretische Physik\\
  Eidgen\"ossische Technische Hochschule Z\"urich\\
  Wolfgang-Pauli-Strasse 27, 8093 Z\"urich, Switzerland\\[1ex]
  \href{mailto:nbeisert@itp.phys.ethz.ch}
  {\texttt{nbeisert@itp.phys.ethz.ch}}}
\hypersetup{pdfauthor={Niklas Beisert}}
\hypersetup{pdfsubject={Manual for the LaTeX2e Package childdoc}}
\date{30 December 2018, \textsf{v2.0}}
\maketitle

\begin{abstract}\noindent
\textsf{childdoc} is a \LaTeXe{} package
that enables the direct compilation
of document sections included by |\include|
to individual files.
\end{abstract}

\begingroup
\parskip0ex
\tableofcontents
\endgroup

%%%%%%%%%%%%%%%%%%%%%%%%%%%%%%%%%%%%%%%%%%%%%%%%%%%%%%%%%%%%%%%%%%%%%%%%%%%%%%%%
%%%%%%%%%%%%%%%%%%%%%%%%%%%%%%%%%%%%%%%%%%%%%%%%%%%%%%%%%%%%%%%%%%%%%%%%%%%%%%%%
\section{Introduction}

\LaTeX{} provides a mechanism to structure a large document (such as a book)
into a main file and several child files (containing the chapters)
using the |\include| command.
This mechanism is beneficial for documents
which span hundreds of pages in order to
make the source file(s) more manageable.
Moreover, compilation can be restricted to
selected child files by means of the |\includeonly| command.
The latter feature can be used to reduce the compilation time while editing
(this was significantly more useful in the earlier days of \LaTeX{})
or to generate a smaller document which is easier to navigate.
Another application of |\includeonly| is to generate
documents consisting of selected parts of the complete document.

However, there are a few drawbacks of the plain |\include| mechanism:
\begin{itemize}
\item
The child files cannot be compiled on their own,
they can only be compiled via the main file.
A naive editing environment
(such as a text editor with an option
to have the current file processed by \LaTeX)
may require one to switch to the main file before compiling;
attempting to compile the child file produces errors.
\item
The main file must be modified (each time)
to adjust the |\includeonly| command
to the present needs. This easily leaves the main file in a messy state.
\item
The generated document will always carry the filename
of the main document. This is inconvenient if
several child files are to be compiled and
to be kept for distribution.
\end{itemize}

The present package provides a simple interface
to make child files individually compilable by \LaTeX{}.
Compiling a child file then has the same effect as compiling
the main file with an |\includeonly| command
to select the appropriate child.
Moreover the generated document will carry the name of the child
rather than the main file.
This resolves all three above issues.

This feature is meant to make the editing of books,
thesis documents and lecture notes somewhat more convenient.
However, the package can also be used efficiently for
composing a series of documents (such as exercise sheets)
which are typically distributed individually.
It then assists the author in generating the individual documents
(potentially in different versions)
as well as a document containing the collected series.
Another application is in developing style files
or other kinds of included material
where compilation of the style file could redirect
to a sample or test file.

%%%%%%%%%%%%%%%%%%%%%%%%%%%%%%%%%%%%%%%%%%%%%%%%%%%%%%%%%%%%%%%%%%%%%%%%%%%%%%%%
%%%%%%%%%%%%%%%%%%%%%%%%%%%%%%%%%%%%%%%%%%%%%%%%%%%%%%%%%%%%%%%%%%%%%%%%%%%%%%%%
\section{Usage}

First of all, the package \textsf{childdoc} is \emph{not} a standard
\LaTeXe{} |.sty| style file! Therefore it needs to be invoked in
a non-standard way.

%%%%%%%%%%%%%%%%%%%%%%%%%%%%%%%%%%%%%%%%%%%%%%%%%%%%%%%%%%%%%%%%%%%%%%%%%%%%%%%%
\subsection{Included Files}
\label{sec:include}

%%%%%%%%%%%%%%%%%%%%%%%%%%%%%%%%%%%%%%%%
\DescribeMacro{\childdocmain}
To use the package, add the commands
\begin{center}
\begin{tabular}{l}
|\input{childdoc.def}|\\
|\childdocmain{}|\\
\end{tabular}
\end{center}
at the very top of the main \LaTeX{} file,
in particular \emph{before} the |\documentclass| statement!
The argument of |\childdocmain| should be left empty
(but it must be present).

%%%%%%%%%%%%%%%%%%%%%%%%%%%%%%%%%%%%%%%%
\DescribeMacro{\childdocof}
Furthermore, add the commands
\begin{center}
\begin{tabular}{l}
|\input{childdoc.def}|\\
|\childdocof{|\textit{main}|}|\\
\end{tabular}
\end{center}
at the top of every child file \textit{child}
which is included by |\include{|\textit{child}|}|
from within the main file
(or at least for those files to be compiled individually).
The argument \textit{main} must be the filename of the main file.

There are a couple of
considerations in setting up the main and child documents:

%%%%%%%%%%%%%%%%%%%%%%%%%%%%%%%%%%%%%%%%
\paragraph{Restrictions.}

Please note the following restrictions:
\begin{itemize}
\item
|\childdocmain| must be called with one argument \textit{main}
to ensure compatibility with earlier version of the package.
It must either be empty (|\childdocmain{}|)
or precisely match the filename of the main file in which it is specified.
See \secref{sec:detection} for further information.
\item
The filename \textit{main} must be specified without the |.tex| extension.
\item
The filename \textit{main} is case sensitive
(even in case-insensitive file systems)
due to internal string comparison.
\item
The argument \textit{main} should be fully expanded, it cannot be a macro.
\item
Subdirectories and special characters should be avoided in filenames.
\item
The command |\childdocmain{|\textit{main}|}| must be followed by a whitespace.
It should not be followed immediately by another command
or by a comment mark `|%|'.
This is because the \TeX{} parser reads the token immediately following
the argument of |\childdocmain| and puts it
at the beginning of every child section;
however, a white\-space is ignored.
\end{itemize}

%%%%%%%%%%%%%%%%%%%%%%%%%%%%%%%%%%%%%%%%
\paragraph{Content of Main File.}

It is advisable to place all content in the child files included by |\include|.
Any output contained in the main file will appear in all child documents
unless suppressed manually;
it cannot be suppressed automatically by the |\includeonly| directive
and thus should normally be avoided.
A method to include some content in the main file
by means of conditional processing is described in \secref{sec:conditional}.

%%%%%%%%%%%%%%%%%%%%%%%%%%%%%%%%%%%%%%%%
\paragraph{Page Numbering.}

When only a part of the document is compiled,
the appropriate numbering of pages
(as well as other status parameters)
is determined from the |.aux| files.
The latter contain information from previous passes.
However this information needs to propagate through
all intermediate child documents.
Therefore the page numbering in child documents may well
be inconsistent until the complete document is compiled at least once.

A useful (if unconventional) way to always ensure a consistent
page numbering is to restart the numbering in each child document
and denote the pages by `\textit{child}|.|\textit{page}'
where \textit{child} represents the chapter/section number of the child file.
This can be achieved by the command
|\numberwithin{page}{|\textit{child}|}|
of the \textsf{amsmath} package
where \textit{child} can be |chapter| or |section|
depending on the chosen structuring.
Alternatively, one can modify the macro |\thepage| appropriately
and reset the counter |page| at the start of each child file.

%%%%%%%%%%%%%%%%%%%%%%%%%%%%%%%%%%%%%%%%%%%%%%%%%%%%%%%%%%%%%%%%%%%%%%%%%%%%%%%%
\subsection{Conditional Processing}
\label{sec:conditional}

The package provides a mechanism to compile different versions
of a document. To customise the versions further some conditional processing
can come in handy to distinguish which version is being compiled.
The package provides two macros to describe the compilation context:

%%%%%%%%%%%%%%%%%%%%%%%%%%%%%%%%%%%%%%%%
\DescribeMacro{\ifchilddoc}
The conditional |\ifchilddoc| distinguishes between the compilation of
child documents and the main document:
%
\begin{center}
|\ifchilddoc |\textit{child-code}| |[|\||else |\textit{main-code}]| \||fi|
\end{center}

%%%%%%%%%%%%%%%%%%%%%%%%%%%%%%%%%%%%%%%%
\DescribeMacro{\childdocname}
\DescribeMacro{\childdocjob}
The macro |\childdocname| contains the filename (without extension)
of the main or child file being processed.
Note that |\childdocjob| will always contain the name of the main file.

%%%%%%%%%%%%%%%%%%%%%%%%%%%%%%%%%%%%%%%%
\paragraph{Title Page.}

Conditional processing can be used to include a title or banner page
in the main document when proper precautions are taken.
Importantly, the code in the main file should ensure that the page counter
(as well as other status parameters which are stored in the |.aux| files)
takes the same value after the conditional processing.
Otherwise the page numbers may take divergent values
depending on which part is compiled.

For example, a title page could be declared by:
%
\begin{center}
\begin{tabular}{l}
|\ifchilddoc\||else|\\
|\addtocounter{page}{-1}|\\
\textit{code for title page}\\
|\newpage|\\
|\||fi|
\end{tabular}
\end{center}
%
A banner page for the child documents can be generated by:
%
\begin{center}
\begin{tabular}{l}
|\ifchilddoc|\\
|\addtocounter{page}{-1}|\\
\textit{code for banner page}\\
|\newpage|\\
|\||fi|
\end{tabular}
\end{center}
%
Here one could write a message such as:
\begin{center}
|This is the part \childdocname{} of \childdocjob{}.|
\end{center}

%%%%%%%%%%%%%%%%%%%%%%%%%%%%%%%%%%%%%%%%%%%%%%%%%%%%%%%%%%%%%%%%%%%%%%%%%%%%%%%%
\subsection{Flags}
\label{sec:flags}

The package makes it easy to generate different versions
of the main or child documents.
To this end compilation flags can be defined
and assigned different default values.
They will be particularly useful in conjunction
with the forwarding mechanism described in \secref{sec:forward}.

For example, it may be useful to have a flag |\version|
which can be set to |draft| or |final|.
The document source will contain some conditional code
depending on the value of |\version|.
Suppose further, the flag should default to |final| for the main file
and to |draft| for child files
which is a natural assignment for editing the document.
This is achieved by placing the following code
in the preamble of the main document
(below the |\childdocmain| directive):
%
\begin{center}
\begin{tabular}{l}
|\ifchilddoc|\\
|\providecommand{\version}{draft}|\\
|\||else|\\
|\providecommand{\version}{final}|\\
|\||fi|
\end{tabular}
\end{center}
%
The definition by |\providecommand| makes sure
that previous definitions are not overwritten.
Further statements |\providecommand{\version}{...}|
can thus be added before the above code to override it.

For the main file, one might add a line
(between |\childdocmain| and the above block)
%
\begin{center}
|%\ifchilddoc\||else\providecommand{\version}{draft}\||fi|
\end{center}
%
which can be uncommented to produce a draft version.
Likewise one can add a line to the very top of a child file
(above the |\childdocof{|\textit{main}|}| directive)
%
\begin{center}
|%\providecommand{\version}{final}|
\end{center}
%
which can be uncommented to produce the final version of this child document.

%%%%%%%%%%%%%%%%%%%%%%%%%%%%%%%%%%%%%%%%%%%%%%%%%%%%%%%%%%%%%%%%%%%%%%%%%%%%%%%%
\subsection{Forwarding}
\label{sec:forward}

Different versions of the main or child documents
using compilation flags as described in \secref{sec:flags}
can be (permanently) stored in different files
for convenient compilation, viewing and distribution.
To this end, the package defines a command
to pass on compilation to a different file:

%%%%%%%%%%%%%%%%%%%%%%%%%%%%%%%%%%%%%%%%
\DescribeMacro{\childdocforward}
The command |\childdocforward| redirects processing to
another source file:
%
\begin{center}
\begin{tabular}{l}
|\input{childdoc.def}|\\
|\childdocforward[|\textit{main}|]{|\textit{dest}|}|\\
\end{tabular}
\end{center}
%
The argument \textit{dest} is the destination file
(without extension).
It should be the main file or one of the child files.
Note that further \textsf{childdoc} directives
such as |\childdocof| and |\childdocforward|
in the indicated file will be processed in this form.
The optional argument \textit{main}
passes on directly to the main file \textit{main}
while pretending to compile the child \textit{dest}.
This form behaves as if \textit{dest}
issues |\childdocof{|\textit{main}|}| right away,
and no further \textsf{childdoc} directives will be processed.

%%%%%%%%%%%%%%%%%%%%%%%%%%%%%%%%%%%%%%%%
\DescribeMacro{\...prefix}
In the alternative form |\childdocforwardprefix|,
%
\begin{center}
\begin{tabular}{l}
|\input{childdoc.def}|\\
|\childdocforwardprefix[|\textit{main}|]{|\textit{prefix}|}{|\textit{dest}|}|
\end{tabular}
\end{center}
%
the destination file is determined by a pattern
depending on the current file:
To make this work, the current file must be called
`{\textit{prefix}\hspace{0.2em}\textit{suffix}}'
with \textit{prefix} matching precisely the argument.
Processing is then passed on to the file
`{\textit{dest}\hspace{0.2em}\textit{suffix}}'.
Surely, the same effect is achieved by
directly specifying the
argument `{\textit{dest}\hspace{0.2em}\textit{suffix}}'
in the first form.
However, that requires to set up a different file
for each child. With the alternative form of the command
all these files can have exactly the same content
which simplifies setting them up and maintaining them.

For example, the following file |draft.tex|
with a compilation flag |\version| as described in \secref{sec:flags}
compiles the main document as a draft:
%
\begin{center}
\begin{tabular}{l}
|\def\version{draft}|\\
|\input{childdoc.def}|\\
|\childdocforward{|\textit{main}|}|
\end{tabular}
\end{center}
%
Likewise, the following files |final|\textit{nn}|.tex|
compile the final version of the child document
|child|\textit{nn}|.tex|:
%
\begin{center}
\begin{tabular}{l}
|\def\version{final}|\\
|\input{childdoc.def}|\\
|\childdocforwardprefix{final}{child}|
\end{tabular}
\end{center}
%

Note that when several versions of a main file and/or of each child file
are to be generated, it may be convenient to set up a |Makefile| or
shell script to automatise the process.

%%%%%%%%%%%%%%%%%%%%%%%%%%%%%%%%%%%%%%%%%%%%%%%%%%%%%%%%%%%%%%%%%%%%%%%%%%%%%%%%
\subsection{Command Line Processing}
\label{sec:commandline}

The effect of redirection files can also be achieved by invoking
the \LaTeX{} compiler with a more elaborate command line.
Most conveniently this should be done as part
of a shell script or a |Makefile|.

When using \textsf{childdoc} in the main file, the following
command lines effectively perform a redirection
(note that depending on the shell being used,
backslashes may have to be doubled: `|\|' $\to$ `|\\|'):
%
\begin{center}
|... -jobname "|\textit{target}|" |\\|"|[\textit{flags}]%
|\input{childdoc.def}\childdocforward[|\textit{main}|]{|\textit{dest}|}"|
\end{center}
%
Here \textit{target} is the name of the output file,
\textit{main} is the name of the main file
and \textit{dest} is the name of the main or child file to be processed
(all filenames without extensions).
The optional argument \textit{main} can be omitted
if \textit{main} matches \textit{dest}.
Optionally, compilation \textit{flags} can be defined via |\def| commands.
This command line makes the \TeX{} engine believe
it is compiling the file \textit{target}
whose content is specified as the latter parameter.
The provided code then forwards the processing to
\textit{main} or \textit{dest} as described in \secref{sec:forward}.

%%%%%%%%%%%%%%%%%%%%%%%%%%%%%%%%%%%%%%%%%%%%%%%%%%%%%%%%%%%%%%%%%%%%%%%%%%%%%%%%
\subsection{Include by Input}
\label{sec:input}

Including child documents by |\include| has some restrictions by design.
Most notably, the content of a child document always occupies
its own set of pages; pages cannot be shared between child documents.
Usually, this behaviour makes perfect sense
because each child document contain an essential part of the document.
However, in some situations it may be desirable to compose
a document from a collection of parts
without having mandatory page breaks between then.
For this case, the package
provides a mechanism to include parts
by |\input| which can also be processed individually.
However, by construction this mechanism
requires manual handling of the content to be output.

%%%%%%%%%%%%%%%%%%%%%%%%%%%%%%%%%%%%%%%%
\DescribeMacro{\ifchilddocmanual}
The main file should be prepared as usual, see \secref{sec:include}.
However, the document body must make a distinction
between processing of an individual part and of the main document, e.g.:
%
\begin{center}
\begin{tabular}{l}
|\ifchilddocmanual|\\
|\input{\childdocname}|\\
|\||else|\\
\textit{document body with }|\input{|\textit{part}|}|\\
|\||fi|
\end{tabular}
\end{center}
%
The conditional |\ifchilddocmanual| is true whenever
a part to be included by |\input| is being compiled,
and the name of the part is stored in |\childdocname|.

%%%%%%%%%%%%%%%%%%%%%%%%%%%%%%%%%%%%%%%%
\DescribeMacro{\childdocby}
Each part to be included by |\input| should start with:
%
\begin{center}
\begin{tabular}{l}
|\input{childdoc.def}|\\
|\childdocby{|\textit{main}|}|\\
\end{tabular}
\end{center}
%
The directive |\childdocby| is similar to |\childdocof|
described in \secref{sec:include},
but the subsequent selection of content must be done manually.
To that end, both |\ifchilddoc| and |\ifchilddocmanual|
will be true upon processing of a part,
and the name of the part is stored in |\childdocname|.
Note that |\jobname| will be set to the filename of the current part
so that each part receives an individual |.aux| file
that does not interfere with the |.aux| file(s) of the main document.
This behaviour can be altered by the alternative form
|\childdocby[*]{|\textit{main}|}| (with a non-empty optional argument)
which uses the |.aux| file of the main document
by setting |\jobname| to \textit{main}.

%%%%%%%%%%%%%%%%%%%%%%%%%%%%%%%%%%%%%%%%%%%%%%%%%%%%%%%%%%%%%%%%%%%%%%%%%%%%%%%%
\subsection{Driver Development}
\label{sec:driver}

The \textsf{childdoc} mechanism can also be use for the development
of definition files such as \LaTeX{} styles or classes.
This case differs from the above setup with multiple parts
included by |\include| in that no |\includeonly| should be invoked.
This can be achieved by starting the include file
(before |\ProvidesPackage|) with:
%
\begin{center}
\begin{tabular}{l}
|\input{childdoc.def}|\\
|\childdocforward{|\textit{main}|}|\\
\end{tabular}
\end{center}
%
or alternatively with:
%
\begin{center}
\begin{tabular}{l}
|\input{childdoc.def}|\\
|\childdocby{|\textit{main}|}|\\
\end{tabular}
\end{center}
%
Both forms have slightly different effects as described above.
The main file is prepared as usual, see \secref{sec:include}.

%%%%%%%%%%%%%%%%%%%%%%%%%%%%%%%%%%%%%%%%%%%%%%%%%%%%%%%%%%%%%%%%%%%%%%%%%%%%%%%%
\subsection{Legacy Detection}
\label{sec:detection}

The directive |\childdocmain| in the main file can detect
whether the complete document or merely a child is to be compiled
even without using the directive |\childdocof|.
This method is deprecated because it is less robust
and there is no compelling reason to use it;
it is merely provided for backward compatibility
and it may be removed in future versions.

If the detection mechanism is to be used,
it is mandatory to correctly specify
the filename of the main file as the argument of |\childdocmain|:
%
\begin{center}
\begin{tabular}{l}
|\input{childdoc.def}|\\
|\childdocmain{|\textit{main}|}|\\
\end{tabular}
\end{center}
%
If |\jobname| does not match the argument \textit{main} of |\childdocmain|,
it is assumed that |\jobname| points to the child file to be compiled.
When using |\childdocmain| with the main file specified as argument,
it suffices to start a child file
with just |\input{|\textit{main}|}|
without loading of the package and using |\childdocof|.
If instead all processing is done
with the appropriate \textsf{childdoc} directives,
the argument of \textit{main} of |\childdocmain| can be empty.

An alternative version of the command line processing described
in \secref{sec:commandline} using the detection mechanism reads:
%
\begin{center}
|... -jobname "|\textit{target}|" "|[\textit{flags}]%
[|\def\jobname{|\textit{dest}|}|]|\input{|\textit{main}|}"|
\end{center}

%%%%%%%%%%%%%%%%%%%%%%%%%%%%%%%%%%%%%%%%%%%%%%%%%%%%%%%%%%%%%%%%%%%%%%%%%%%%%%%%
\subsection{Manual Code}
\label{sec:manual}

In case one cannot be certain whether the definitions file |childdoc.def|
is installed on the target \TeX{} distribution
and one prefers not to ship it,
it is conceivable to paste a few relevant commands into the sources.

To that end, drop all statements |\input{childdoc.def}|
and perform the replacements as outlined below.
Instead of |\childdocmain{|\textit{main}|}| add the following code
to the top of the main file:
%
\begin{center}
\begin{tabular}{l}
|\||ifdefined\childdocname\endinput\||fi\newif\ifchilddoc|\\
|\edef\childdocname{\scantokens\expandafter{\jobname\noexpand}}|\\
|\def\childdocmain{|\textit{main}|}\||ifx\childdocmain\childdocname\||else|\\
|\childdoctrue\includeonly{\childdocname}\let\jobname\childdocmain\||fi|\\
\end{tabular}
\end{center}
%
Instead of |\childdocof{|\textit{main}|}| just include the main file
at the top of each child file:
%
\begin{center}
|\input{|\textit{main}|}|
\end{center}
%
A simple redirection |\childdocforward{|\textit{dest}|}| is achieved by:
%
\begin{center}
|\def\jobname{|\textit{dest}|}\input{\jobname}|
\end{center}
%
The redirection with prefix
|\childdocforwardprefix[|\textit{prefix}|]{|\textit{dest}|}|
is accomplished by:
%
\begin{center}
\begin{tabular}{l}
|{\edef\jobname{\scantokens\expandafter{\jobname\noexpand}}|\\
|\def\redirectjob |\textit{prefix}|#1~~~{\gdef\jobname{|\textit{dest}|#1}}|\\
|\expandafter\redirectjob\jobname~~~}\input{\jobname}|
\end{tabular}
\end{center}

In an alternative approach,
child documents can be compiled by a specific command line
without additional code or specific definitions:
%
\begin{center}
|... -jobname "|\textit{target}|" "|[\textit{flags}]%
|\includeonly{|\textit{dest}|}\input{|\textit{main}|}"|
\end{center}
%

%%%%%%%%%%%%%%%%%%%%%%%%%%%%%%%%%%%%%%%%%%%%%%%%%%%%%%%%%%%%%%%%%%%%%%%%%%%%%%%%
%%%%%%%%%%%%%%%%%%%%%%%%%%%%%%%%%%%%%%%%%%%%%%%%%%%%%%%%%%%%%%%%%%%%%%%%%%%%%%%%
\section{Information}

%%%%%%%%%%%%%%%%%%%%%%%%%%%%%%%%%%%%%%%%%%%%%%%%%%%%%%%%%%%%%%%%%%%%%%%%%%%%%%%%
\subsection{Copyright}

Copyright \copyright{} 2017--2018 Niklas Beisert

This work may be distributed and/or modified under the
conditions of the \LaTeX{} Project Public License, either version 1.3
of this license or (at your option) any later version.
The latest version of this license is in
  \url{http://www.latex-project.org/lppl.txt}
and version 1.3 or later is part of all distributions of \LaTeX{}
version 2005/12/01 or later.

This work has the LPPL maintenance status `maintained'.

The Current Maintainer of this work is Niklas Beisert.

This work consists of the files |README.txt|, |childdoc.ins| and |childdoc.dtx|
as well as the derived files |childdoc.def|, |cdocsamp.tex|
with |cdocsch1.tex|, |cdocsch2.tex|, |cdocspt3.tex|, |cdocspt4.tex|,
|cdocsdrf.tex|, |cdocsfn1.tex|, |cdocsfn2.tex|
as well as |childdoc.pdf|.

%%%%%%%%%%%%%%%%%%%%%%%%%%%%%%%%%%%%%%%%%%%%%%%%%%%%%%%%%%%%%%%%%%%%%%%%%%%%%%%%
\subsection{Files and Installation}

The package consists of the files:
%
\begin{center}
\begin{tabular}{ll}
    |README.txt|   & readme file \\
    |childdoc.ins| & installation file \\
    |childdoc.dtx| & source file \\
    |childdoc.def| & definition file \\
    |cdocsamp.tex| & sample main file \\
    |cdocsch1.tex| & sample include file \\
    |cdocsch2.tex| & sample include file \\
    |cdocspt3.tex| & sample part file \\
    |cdocspt4.tex| & sample part file \\
    |cdocsdrf.tex| & sample redirection file \\
    |cdocsfn1.tex| & sample redirection file \\
    |cdocsfn2.tex| & sample redirection file \\
    |childdoc.pdf| & manual
\end{tabular}
\end{center}
%
The distribution consists of the files
|README.txt|, |childdoc.ins| and |childdoc.dtx|.
%
\begin{itemize}
\item
Run (pdf)\LaTeX{} on |childdoc.dtx|
to compile the manual |childdoc.pdf| (this file).
\item
Run \LaTeX{} on |childdoc.ins| to create the definitions file |childdoc.def|
and the sample |cdocsamp.tex| with include files
|cdocsch1.tex|, |cdocsch2.tex|, |cdocspt3.tex|, |cdocspt4.tex|,
|cdocsdrf.tex|, |cdocsfn1.tex|, |cdocsfn2.tex|.
Then copy the file |childdoc.def| to an appropriate directory of your \LaTeX{}
distribution, e.g.\ \textit{texmf-root}|/tex/latex/childdoc|.
\end{itemize}

%%%%%%%%%%%%%%%%%%%%%%%%%%%%%%%%%%%%%%%%%%%%%%%%%%%%%%%%%%%%%%%%%%%%%%%%%%%%%%%%
\subsection{Related CTAN Packages}

There are several other packages which offer a similar functionality:
%
\begin{itemize}
\item
The packages
\href{http://ctan.org/pkg/docmute}{\textsf{docmute}},
\href{http://ctan.org/pkg/includex}{\textsf{includex}} and
\href{http://ctan.org/pkg/standalone}{\textsf{standalone}}
provide commands to include only the document body of
a child file thus allowing both files to be compiled individually.
\item
The packages \href{http://ctan.org/pkg/subdocs}{\textsf{subdocs}}
and \href{http://ctan.org/pkg/subfiles}{\textsf{subfiles}}
provide structures in which the main and child documents can be
encapsulated and allowing them to be compiled individually.
The inclusion mechanism is different from the conventional |\include|.
\item
The package \href{http://ctan.org/pkg/combine}{\textsf{combine}}
is an elaborate solution to combine several documents into one.
\end{itemize}
%
See also the CTAN topic \href{http://ctan.org/topic/subdocs}{\textsf{subdocs}}
for further related packages.
The present package differs from the above solutions in that
a document structure constructed with the conventional |\include| mechanism
just needs two extra commands at the top of every file
such that all constituent files can be compiled individually.

%%%%%%%%%%%%%%%%%%%%%%%%%%%%%%%%%%%%%%%%%%%%%%%%%%%%%%%%%%%%%%%%%%%%%%%%%%%%%%%%
%\subsection{Feature Suggestions}
%
%The following is a list of features which may be useful for future
%versions of this package:
%%
%\begin{itemize}
%\item
%\ldots
%\end{itemize}

%%%%%%%%%%%%%%%%%%%%%%%%%%%%%%%%%%%%%%%%%%%%%%%%%%%%%%%%%%%%%%%%%%%%%%%%%%%%%%%%
\subsection{Revision History}

%%%%%%%%%%%%%%%%%%%%%%%%%%%%%%%%%%%%%%%%
\paragraph{v2.0:} 2018/12/30

\begin{itemize}
\item
immediate forward processing
\item
added |\childdocby| mechanism
\item
manual restructured
\end{itemize}

%%%%%%%%%%%%%%%%%%%%%%%%%%%%%%%%%%%%%%%%
\paragraph{v1.6:} 2018/01/17

\begin{itemize}
\item
application for development of include files
\item
corrections to manual
\end{itemize}

%%%%%%%%%%%%%%%%%%%%%%%%%%%%%%%%%%%%%%%%
\paragraph{v1.5:} 2017/05/21

\begin{itemize}
\item
more complete structuring introduced
\item
|\childdocof| introduced
\item
|\childdoc| renamed to |\childdocmain|
\item
|\childredirect| renamed to |\childdocforward| and |\childdocforwardprefix|
and functionality expanded
\end{itemize}

%%%%%%%%%%%%%%%%%%%%%%%%%%%%%%%%%%%%%%%%
\paragraph{v1.0:} 2017/04/27

\begin{itemize}
\item
manual and install package
\item
first version published on CTAN
\end{itemize}

%%%%%%%%%%%%%%%%%%%%%%%%%%%%%%%%%%%%%%%%
\paragraph{v0.6:} 2017/04/26

\begin{itemize}
\item
redirection mechanism added
\end{itemize}

%%%%%%%%%%%%%%%%%%%%%%%%%%%%%%%%%%%%%%%%
\paragraph{v0.5:} 2017/04/26

\begin{itemize}
\item
functionality in definition file
\end{itemize}


%%%%%%%%%%%%%%%%%%%%%%%%%%%%%%%%%%%%%%%%%%%%%%%%%%%%%%%%%%%%%%%%%%%%%%%%%%%%%%%%
%%%%%%%%%%%%%%%%%%%%%%%%%%%%%%%%%%%%%%%%%%%%%%%%%%%%%%%%%%%%%%%%%%%%%%%%%%%%%%%%
%%%%%%%%%%%%%%%%%%%%%%%%%%%%%%%%%%%%%%%%%%%%%%%%%%%%%%%%%%%%%%%%%%%%%%%%%%%%%%%%
\appendix

\settowidth\MacroIndent{\rmfamily\scriptsize 000\ }

 \DocInput{childdoc.dtx}

\end{document}
%</driver>
% \fi
%
% %%%%%%%%%%%%%%%%%%%%%%%%%%%%%%%%%%%%%%%%%%%%%%%%%%%%%%%%%%%%%%%%%%%%%%%%%%%%%%
% %%%%%%%%%%%%%%%%%%%%%%%%%%%%%%%%%%%%%%%%%%%%%%%%%%%%%%%%%%%%%%%%%%%%%%%%%%%%%%
% \section{Sample}
%\iffalse
%<*samplemain>
%\fi
%
% The following presents a sample document
% with two chapters, two parts, a title page,
% a compile flag as well as three forwarding files to set the flag.
% It consists of eight |.tex| files:
% \begin{center}
% \begin{tabular}{ll}
% |cdocsamp.tex|&main file\\
% |cdocsch1.tex|&include file for chapter 1\\
% |cdocsch2.tex|&include file for chapter 2\\
% |cdocspt3.tex|&include file for part 3\\
% |cdocspt4.tex|&include file for part 4\\
% |cdocsdrf.tex|&forwarding file for main file in draft mode\\
% |cdocsfi1.tex|&forwarding file for final version of chapter 1\\
% |cdocsfi2.tex|&forwarding file for final version of chapter 2\\
% \end{tabular}
% \end{center}
% Each of the eight files can be compiled directly by the \LaTeX{} compiler.
%
% %%%%%%%%%%%%%%%%%%%%%%%%%%%%%%%%%%%%%%
% \paragraph{Main File.}
%
% The main file is called |cdocsamp.tex|.
%
% Load the \textsf{childdoc} definitions and
% declare the filename for the main document:
%    \begin{macrocode}
\input{childdoc.def}
\childdocmain{}
%    \end{macrocode}

% Optional override for |\version| flag:
%    \begin{macrocode}
%%\ifchilddoc\else\providecommand{\version}{draft}\fi
%    \end{macrocode}

% Define the default values for the |\version| flag
% (|final| for the main file and |draft| for childs):
%    \begin{macrocode}
\ifchilddoc
\providecommand{\version}{draft}
\else
\providecommand{\version}{final}
\fi
%    \end{macrocode}

% Load the standard document class:
%    \begin{macrocode}
\documentclass[12pt]{article}
%    \end{macrocode}

% Start the document body:
%    \begin{macrocode}
\begin{document}
%    \end{macrocode}

% Declare a title page.
% Print title, part of document being processed and version flag:
%    \begin{macrocode}
\addtocounter{page}{-1}
\begin{center}
{\LARGE\bfseries{}childdoc example\par}
\vspace{1cm}
\ifchilddoc
\ifchilddocmanual part\else chapter\fi:
`\childdocname' of `\childdocjob'\par
\else
main document: `\childdocjob'\par
\fi
version: \version\par
\end{center}
\newpage
%    \end{macrocode}

% Manually include selected file,
% otherwise process as usual:
%    \begin{macrocode}
\ifchilddocmanual
\section*{part `\childdocname'}
\input{\childdocname}
\else
%    \end{macrocode}

% Include the two chapters:
%    \begin{macrocode}
\include{cdocsch1}
\include{cdocsch2}
%    \end{macrocode}

% Include the two parts unless only chapters should be displayed:
%    \begin{macrocode}
\ifchilddoc\else
\section{part three}
\input{cdocspt3}
\section{part four}
\input{cdocspt4}
\fi
%    \end{macrocode}

% Process as usual until here:
%    \begin{macrocode}
\fi
%    \end{macrocode}

% End of document body:
%    \begin{macrocode}
\end{document}
%    \end{macrocode}
%\iffalse
%</samplemain>
%\fi
%
% %%%%%%%%%%%%%%%%%%%%%%%%%%%%%%%%%%%%%%
% \paragraph{Chapter Include Files.}
%
% The include files are called |cdocsch1.tex| and |cdocsch2.tex|.
%
%\iffalse
%<*samplechap1|samplechap2>
%\fi

% Optional override for |\version| flag:
%    \begin{macrocode}
%%\providecommand{\version}{final}
%    \end{macrocode}

% Include the main document:
%    \begin{macrocode}
\input{childdoc.def}
\childdocof{cdocsamp}
%    \end{macrocode}

%\iffalse
%</samplechap1|samplechap2>
%\fi
%
%\iffalse
%<*samplechap1>
%\fi
% Some text for chapter 1:
%    \begin{macrocode}
\section{one}
some text in chapter one
%    \end{macrocode}

%\iffalse
%</samplechap1>
%\fi
% Some text for chapter 2:
%\iffalse
%<*samplechap2>
%\fi
%    \begin{macrocode}
\section{two}
more text in chapter two
%    \end{macrocode}

%\iffalse
%</samplechap2>
%\fi
%
% %%%%%%%%%%%%%%%%%%%%%%%%%%%%%%%%%%%%%%
% \paragraph{Part Include Files.}
%
% The include files are called |cdocspt3.tex| and |cdocspt4.tex|.
%
%\iffalse
%<*samplepart3|samplepart4>
%\fi

% Optional override for |\version| flag:
%    \begin{macrocode}
%%\providecommand{\version}{final}
%    \end{macrocode}

% Include the main document:
%    \begin{macrocode}
\input{childdoc.def}
\childdocby{cdocsamp}
%    \end{macrocode}

%\iffalse
%</samplepart3|samplepart4>
%\fi
%
%\iffalse
%<*samplepart3>
%\fi
% Some text for part 3:
%    \begin{macrocode}
some text in part three
%    \end{macrocode}

%\iffalse
%</samplepart3>
%\fi
% Some text for part 4:
%\iffalse
%<*samplepart4>
%\fi
%    \begin{macrocode}
more text in part four
%    \end{macrocode}

%\iffalse
%</samplepart4>
%\fi
%
% %%%%%%%%%%%%%%%%%%%%%%%%%%%%%%%%%%%%%%
% \paragraph{Forwarding for a Complete Draft.}
%
% The following forwarding file |cdocsdrf.tex|
% compiles the main document in draft mode:
%\iffalse
%<*sampledraft>
%\fi
%    \begin{macrocode}
\def\version{draft}
\input{childdoc.def}
\childdocforward{cdocsamp}
%    \end{macrocode}

%\iffalse
%</sampledraft>
%\fi
%
% %%%%%%%%%%%%%%%%%%%%%%%%%%%%%%%%%%%%%%
% \paragraph{Forwarding for Final Version of the Chapters.}
%
% The following forwarding files |cdocsfn1.tex| and |cdocsfn2.tex|
% (with identical content)
% compile the final versions of the child documents
% |cdocsch1.tex| and |cdocsch2.tex|, respectively:
%\iffalse
%<*samplefinal>
%\fi
%    \begin{macrocode}
\def\version{final}
\input{childdoc.def}
\childdocforwardprefix[cdocsamp]{cdocsfn}{cdocsch}
%    \end{macrocode}

%\iffalse
%</samplefinal>
%\fi
%
% %%%%%%%%%%%%%%%%%%%%%%%%%%%%%%%%%%%%%%
% \paragraph{Command Line Processing.}
%
% The following three command lines generate the output files
% |cdocscld|, |cdocscl1| and |cdocscl2|
% which should be identical to
% |cdocsdrf|, |cdocsch1| and |cdocsfn2|, respectively:
% \begin{center}
% \begin{tabular}{l}
% |latex -jobname cdocscld \|\\
% |  "\def\version{draft}\input{childdoc.def}\childdocforward{cdocsamp}"|\\
% |latex -jobname cdocscl1 \|\\
% |  "\input{childdoc.def}\childdocforward[cdocsamp]{cdocsch1}"|\\
% |latex -jobname cdocscl2 \|\\
% |  "\def\version{final}\input{childdoc.def}\childdocforward{cdocsch2}"|
% \end{tabular}
% \end{center}
% Note that the trailing backslash on each first line
% merely continues the input to the second line
% (for convenient cut ant paste).
% Furthermore, the command |latex| can be replaced by any
% of its alternative versions such as |pdflatex|.
%
% %%%%%%%%%%%%%%%%%%%%%%%%%%%%%%%%%%%%%%%%%%%%%%%%%%%%%%%%%%%%%%%%%%%%%%%%%%%%%%
% %%%%%%%%%%%%%%%%%%%%%%%%%%%%%%%%%%%%%%%%%%%%%%%%%%%%%%%%%%%%%%%%%%%%%%%%%%%%%%
% \section{Implementation}
%\iffalse
%<*package>
%\fi
%
% This section describes the definitions file |childdoc.def|.

% The definitions cannot be loaded using |\usepackage| or |\RequirePackage|
% which has a mechanism to prevent loading a style file more than once.
% When loading the definitions by means of |\input|
% multiple instances have to be prevented manually:
%\iffalse
%This code needs to be before the `\ProvidesFile' directive
%which is defined at the beginning of this file.
%Therefore it is also placed there and commented out here.
%</package>
%<*discard>
%\fi
%    \begin{macrocode}
\ifdefined\childdocmain\endinput\fi
%    \end{macrocode}
%\iffalse
%</discard>
%<*package>
%\fi
%
% \macro{\ifchilddoc}
% \macro{\ifchilddocmanual}
% The conditional |\ifchilddoc| tells whether a
% child (true) or main (false) document is being compiled.
% The conditional |\ifchilddocmanual| tells whether
% the |\includeonly| mechanism is used (false) or
% the selection of child files must be performed manually (true).
% The definitions initialise to false:
%    \begin{macrocode}
\newif\ifchilddoc
\newif\ifchilddocmanual
%    \end{macrocode}

% \macro{\childdocname}
% \macro{\childdocjob}
% The macro |\childdocname| stores the name of the main document
% to be compiled. The macro |\childdocjob| stores the name of
% the document on which the \LaTeX{} compiler was originally invoked.
% The content of |\jobname| cannot be compared
% to filenames specified in the source due to different catcodes.
% The following code rescans |\jobname|, stores the result
% in |\childdocname| and saves a copy in |\childdocjob|:
%    \begin{macrocode}
\edef\childdocname{\scantokens\expandafter{\jobname\noexpand}}
\let\childdocjob\childdocname
%    \end{macrocode}

% \macro{\childdocdisable}
% The macro |\childdocdisable| prevents the main file
% from being processed more than once.
% At this stage, the main document command |\childdocmain|
% is assumed to be called once again where it should do nothing.
% Any subsequent call to it should prevent
% a secondary processing of the main document
% It overwrites the forwarding commands
% |\childdocof| and |\childdocforward|
% with empty macros to prevent further inclusions of the main document:
%    \begin{macrocode}
\newcommand{\childdocdisable}
{
  \renewcommand{\childdocmain}[1]{\renewcommand{\childdocmain}[1]{\endinput}}
  \renewcommand{\childdocof}[1]{}
  \renewcommand{\childdocby}[2][]{}
  \renewcommand{\childdocforward}[2][]{}
  \renewcommand{\childdocdisable}{}
}
%    \end{macrocode}

% \macro{\childdocmain}
% The macro |\childdocmain| is to be called at the top of the main file
% with nothing or the main filename (without extension) as argument.
% First, it breaks loops.
% If the argument is not empty and does not match |\childdocname|
% (which is set by the first inclusion of |childdoc.def|),
% |\ifchilddoc| is set to true, |\includeonly| is applied to the child file
% and |\jobname| is set to the main file
% (for proper handling of |.aux| files):
%    \begin{macrocode}
\newcommand{\childdocmain}[1]
{
  \childdocdisable\childdocmain{}
  \if?#1?\else
    \begingroup
      \def\childdoctmp{#1}
      \ifx\childdoctmp\childdocname
        \def\childdoctmp{}
      \else
        \def\childdoctmp
        {
          \childdoctrue
          \includeonly{\childdocname}
          \def\childdocjob{#1}
          \def\jobname{#1}
        }
      \fi
      \expandafter
    \endgroup
    \childdoctmp
  \fi
}
%    \end{macrocode}

% \macro{\childdocof}
% The command |\childdocof| redirects
% compilation to the main file |#1|.
%    \begin{macrocode}
\newcommand{\childdocof}[1]
{
  \childdocdisable
  \childdoctrue
  \includeonly{\childdocname}
  \def\jobname{#1}
  \def\childdocjob{#1}
  \input{#1}
}
%    \end{macrocode}

% \macro{\childdocby}
% The command |\childdocby| ....
%    \begin{macrocode}
\newcommand{\childdocby}[2][]
{
  \childdocdisable
  \childdoctrue
  \childdocmanualtrue
  \if?#1?\else
    \def\jobname{#2}
  \fi
  \def\childdocjob{#2}
  \input{#2}
  \endinput
}
%    \end{macrocode}

% \macro{\childdocforward}
% The command |\childdocforward| redirects
% compilation to the main file or
% (if the optional argument is given) a child file.
% Parameters are set as if the main file
% or a child file starting with |\childdocof| was compiled.
% Then compilation is handed over to the main file:
%    \begin{macrocode}
\newcommand{\childdocforward}[2][]
{
  \begingroup
    \if?#1?
      \def\childdoctmp
      {
        \def\childdocname{#2}
        \def\childdocjob{#2}
        \def\jobname{#2}
        \input{#2}
        \endinput
      }
    \else
      \def\childdoctmp
      {
        \childdocdisable
        \def\childdocname{#2}
        \childdoctrue
        \includeonly{#2}
        \def\childdocjob{#1}
        \def\jobname{#1}
        \input{#1}
        \endinput
      }
    \fi
    \expandafter
  \endgroup
  \childdoctmp
}
%    \end{macrocode}

% \macro{\childdocforwardprefix}
% The command |\childdocforwardprefix| redirects
% compilation to the main or a child file by means of a pattern.
% The prefix |#1| in the current filename is replaced by |#2|
% and the suffix of the current filename is kept
% (it is assumed that the filename does not contain the substring `|~~~|'
% which is used as a delimiter).
% Compilation is handed over to the new file by |\childdocforward|:
%    \begin{macrocode}
\newcommand{\childdocforwardprefix}[3][]
{
  \begingroup
    \def\childdocextract #2##1~~~{\def\childdoctmp{\childdocforward[#1]{#3##1}}}
    \expandafter\childdocextract\childdocname~~~
    \expandafter
  \endgroup
  \childdoctmp
}
%    \end{macrocode}

% \macro{\childdoc}
% The deprecated macro |\childdoc| is a legacy version of |\childdocmain|:
%    \begin{macrocode}
\newcommand{\childdoc}{\childdocmain}
%    \end{macrocode}

% \macro{\childdocredirect}
% The deprecated macro |\childdocredirect| is a legacy version
% of |\childdocforward| and |\childdocforwardprefix|:
%    \begin{macrocode}
\newcommand{\childdocredirect}[2][]
{
  \begingroup
    \if?#1?
      \def\childdoctmp{\childdocforward{#2}}
    \else
      \def\childdoctmp{\childdocforwardprefix{#1}{#2}}
    \fi
    \expandafter
  \endgroup
  \childdoctmp
}
%    \end{macrocode}

%\iffalse
%</package>
%\fi
%
\endinput

\childdocforward{cdocsamp}
%    \end{macrocode}

%\iffalse
%</sampledraft>
%\fi
%
% %%%%%%%%%%%%%%%%%%%%%%%%%%%%%%%%%%%%%%
% \paragraph{Forwarding for Final Version of the Chapters.}
%
% The following forwarding files |cdocsfn1.tex| and |cdocsfn2.tex|
% (with identical content)
% compile the final versions of the child documents
% |cdocsch1.tex| and |cdocsch2.tex|, respectively:
%\iffalse
%<*samplefinal>
%\fi
%    \begin{macrocode}
\def\version{final}
% \iffalse
%
% childdoc.dtx Copyright (C) 2017-2018 Niklas Beisert
%
% This work may be distributed and/or modified under the
% conditions of the LaTeX Project Public License, either version 1.3
% of this license or (at your option) any later version.
% The latest version of this license is in
%   http://www.latex-project.org/lppl.txt
% and version 1.3 or later is part of all distributions of LaTeX
% version 2005/12/01 or later.
%
% This work has the LPPL maintenance status `maintained'.
%
% The Current Maintainer of this work is Niklas Beisert.
%
% This work consists of the files childdoc.dtx and childdoc.ins
% and the derived files childdoc.def and cdocsamp.tex with
% cdocsch1.tex, cdocsch2.tex, cdocsdrf.tex, cdocsfn1.tex, cdocsfn2.tex.
%
%<package>\ifdefined\childdocmain\endinput\fi
%<package>\ProvidesFile{childdoc.def}[2018/12/30 v2.0 child document driver]
%<samplemain>\ProvidesFile{cdocsamp.tex}[2018/12/30 v2.0 sample for childdoc]
%<*driver>
%\ProvidesFile{childdoc.drv}[2018/12/30 v2.0 childdoc reference manual file]
\PassOptionsToClass{10pt,a4paper}{article}
\documentclass{ltxdoc}

\usepackage[margin=35mm]{geometry}
\usepackage{hyperref}
\usepackage{hyperxmp}
\usepackage[usenames]{color}

\hypersetup{colorlinks=true}
\hypersetup{pdfstartview=FitH}
\hypersetup{pdfpagemode=UseNone}
\hypersetup{pdfsource={}}
\hypersetup{pdflang={en-UK}}
\hypersetup{pdfcopyright={Copyright 2017-2018 Niklas Beisert.
  This work may be distributed and/or modified under the
  conditions of the LaTeX Project Public License, either version 1.3
  of this license or (at your option) any later version.}}
\hypersetup{pdflicenseurl={http://www.latex-project.org/lppl.txt}}
\hypersetup{pdfcontactaddress={ETH Zurich, ITP, HIT K,
  Wolfgang-Pauli-Strasse 27}}
\hypersetup{pdfcontactpostcode={8093}}
\hypersetup{pdfcontactcity={Zurich}}
\hypersetup{pdfcontactcountry={Switzerland}}
\hypersetup{pdfcontactemail={nbeisert@itp.phys.ethz.ch}}
\hypersetup{pdfcontacturl={http://people.phys.ethz.ch/\xmptilde nbeisert/}}

\newcommand{\secref}[1]{\hyperref[#1]{section \ref*{#1}}}

\parskip1ex
\parindent0pt
\let\olditemize\itemize
\def\itemize{\olditemize\parskip0pt}

\begin{document}

\title{The \textsf{childdoc} Package}
\hypersetup{pdftitle={The childdoc Package}}
\author{Niklas Beisert\\[2ex]
  Institut f\"ur Theoretische Physik\\
  Eidgen\"ossische Technische Hochschule Z\"urich\\
  Wolfgang-Pauli-Strasse 27, 8093 Z\"urich, Switzerland\\[1ex]
  \href{mailto:nbeisert@itp.phys.ethz.ch}
  {\texttt{nbeisert@itp.phys.ethz.ch}}}
\hypersetup{pdfauthor={Niklas Beisert}}
\hypersetup{pdfsubject={Manual for the LaTeX2e Package childdoc}}
\date{30 December 2018, \textsf{v2.0}}
\maketitle

\begin{abstract}\noindent
\textsf{childdoc} is a \LaTeXe{} package
that enables the direct compilation
of document sections included by |\include|
to individual files.
\end{abstract}

\begingroup
\parskip0ex
\tableofcontents
\endgroup

%%%%%%%%%%%%%%%%%%%%%%%%%%%%%%%%%%%%%%%%%%%%%%%%%%%%%%%%%%%%%%%%%%%%%%%%%%%%%%%%
%%%%%%%%%%%%%%%%%%%%%%%%%%%%%%%%%%%%%%%%%%%%%%%%%%%%%%%%%%%%%%%%%%%%%%%%%%%%%%%%
\section{Introduction}

\LaTeX{} provides a mechanism to structure a large document (such as a book)
into a main file and several child files (containing the chapters)
using the |\include| command.
This mechanism is beneficial for documents
which span hundreds of pages in order to
make the source file(s) more manageable.
Moreover, compilation can be restricted to
selected child files by means of the |\includeonly| command.
The latter feature can be used to reduce the compilation time while editing
(this was significantly more useful in the earlier days of \LaTeX{})
or to generate a smaller document which is easier to navigate.
Another application of |\includeonly| is to generate
documents consisting of selected parts of the complete document.

However, there are a few drawbacks of the plain |\include| mechanism:
\begin{itemize}
\item
The child files cannot be compiled on their own,
they can only be compiled via the main file.
A naive editing environment
(such as a text editor with an option
to have the current file processed by \LaTeX)
may require one to switch to the main file before compiling;
attempting to compile the child file produces errors.
\item
The main file must be modified (each time)
to adjust the |\includeonly| command
to the present needs. This easily leaves the main file in a messy state.
\item
The generated document will always carry the filename
of the main document. This is inconvenient if
several child files are to be compiled and
to be kept for distribution.
\end{itemize}

The present package provides a simple interface
to make child files individually compilable by \LaTeX{}.
Compiling a child file then has the same effect as compiling
the main file with an |\includeonly| command
to select the appropriate child.
Moreover the generated document will carry the name of the child
rather than the main file.
This resolves all three above issues.

This feature is meant to make the editing of books,
thesis documents and lecture notes somewhat more convenient.
However, the package can also be used efficiently for
composing a series of documents (such as exercise sheets)
which are typically distributed individually.
It then assists the author in generating the individual documents
(potentially in different versions)
as well as a document containing the collected series.
Another application is in developing style files
or other kinds of included material
where compilation of the style file could redirect
to a sample or test file.

%%%%%%%%%%%%%%%%%%%%%%%%%%%%%%%%%%%%%%%%%%%%%%%%%%%%%%%%%%%%%%%%%%%%%%%%%%%%%%%%
%%%%%%%%%%%%%%%%%%%%%%%%%%%%%%%%%%%%%%%%%%%%%%%%%%%%%%%%%%%%%%%%%%%%%%%%%%%%%%%%
\section{Usage}

First of all, the package \textsf{childdoc} is \emph{not} a standard
\LaTeXe{} |.sty| style file! Therefore it needs to be invoked in
a non-standard way.

%%%%%%%%%%%%%%%%%%%%%%%%%%%%%%%%%%%%%%%%%%%%%%%%%%%%%%%%%%%%%%%%%%%%%%%%%%%%%%%%
\subsection{Included Files}
\label{sec:include}

%%%%%%%%%%%%%%%%%%%%%%%%%%%%%%%%%%%%%%%%
\DescribeMacro{\childdocmain}
To use the package, add the commands
\begin{center}
\begin{tabular}{l}
|\input{childdoc.def}|\\
|\childdocmain{}|\\
\end{tabular}
\end{center}
at the very top of the main \LaTeX{} file,
in particular \emph{before} the |\documentclass| statement!
The argument of |\childdocmain| should be left empty
(but it must be present).

%%%%%%%%%%%%%%%%%%%%%%%%%%%%%%%%%%%%%%%%
\DescribeMacro{\childdocof}
Furthermore, add the commands
\begin{center}
\begin{tabular}{l}
|\input{childdoc.def}|\\
|\childdocof{|\textit{main}|}|\\
\end{tabular}
\end{center}
at the top of every child file \textit{child}
which is included by |\include{|\textit{child}|}|
from within the main file
(or at least for those files to be compiled individually).
The argument \textit{main} must be the filename of the main file.

There are a couple of
considerations in setting up the main and child documents:

%%%%%%%%%%%%%%%%%%%%%%%%%%%%%%%%%%%%%%%%
\paragraph{Restrictions.}

Please note the following restrictions:
\begin{itemize}
\item
|\childdocmain| must be called with one argument \textit{main}
to ensure compatibility with earlier version of the package.
It must either be empty (|\childdocmain{}|)
or precisely match the filename of the main file in which it is specified.
See \secref{sec:detection} for further information.
\item
The filename \textit{main} must be specified without the |.tex| extension.
\item
The filename \textit{main} is case sensitive
(even in case-insensitive file systems)
due to internal string comparison.
\item
The argument \textit{main} should be fully expanded, it cannot be a macro.
\item
Subdirectories and special characters should be avoided in filenames.
\item
The command |\childdocmain{|\textit{main}|}| must be followed by a whitespace.
It should not be followed immediately by another command
or by a comment mark `|%|'.
This is because the \TeX{} parser reads the token immediately following
the argument of |\childdocmain| and puts it
at the beginning of every child section;
however, a white\-space is ignored.
\end{itemize}

%%%%%%%%%%%%%%%%%%%%%%%%%%%%%%%%%%%%%%%%
\paragraph{Content of Main File.}

It is advisable to place all content in the child files included by |\include|.
Any output contained in the main file will appear in all child documents
unless suppressed manually;
it cannot be suppressed automatically by the |\includeonly| directive
and thus should normally be avoided.
A method to include some content in the main file
by means of conditional processing is described in \secref{sec:conditional}.

%%%%%%%%%%%%%%%%%%%%%%%%%%%%%%%%%%%%%%%%
\paragraph{Page Numbering.}

When only a part of the document is compiled,
the appropriate numbering of pages
(as well as other status parameters)
is determined from the |.aux| files.
The latter contain information from previous passes.
However this information needs to propagate through
all intermediate child documents.
Therefore the page numbering in child documents may well
be inconsistent until the complete document is compiled at least once.

A useful (if unconventional) way to always ensure a consistent
page numbering is to restart the numbering in each child document
and denote the pages by `\textit{child}|.|\textit{page}'
where \textit{child} represents the chapter/section number of the child file.
This can be achieved by the command
|\numberwithin{page}{|\textit{child}|}|
of the \textsf{amsmath} package
where \textit{child} can be |chapter| or |section|
depending on the chosen structuring.
Alternatively, one can modify the macro |\thepage| appropriately
and reset the counter |page| at the start of each child file.

%%%%%%%%%%%%%%%%%%%%%%%%%%%%%%%%%%%%%%%%%%%%%%%%%%%%%%%%%%%%%%%%%%%%%%%%%%%%%%%%
\subsection{Conditional Processing}
\label{sec:conditional}

The package provides a mechanism to compile different versions
of a document. To customise the versions further some conditional processing
can come in handy to distinguish which version is being compiled.
The package provides two macros to describe the compilation context:

%%%%%%%%%%%%%%%%%%%%%%%%%%%%%%%%%%%%%%%%
\DescribeMacro{\ifchilddoc}
The conditional |\ifchilddoc| distinguishes between the compilation of
child documents and the main document:
%
\begin{center}
|\ifchilddoc |\textit{child-code}| |[|\||else |\textit{main-code}]| \||fi|
\end{center}

%%%%%%%%%%%%%%%%%%%%%%%%%%%%%%%%%%%%%%%%
\DescribeMacro{\childdocname}
\DescribeMacro{\childdocjob}
The macro |\childdocname| contains the filename (without extension)
of the main or child file being processed.
Note that |\childdocjob| will always contain the name of the main file.

%%%%%%%%%%%%%%%%%%%%%%%%%%%%%%%%%%%%%%%%
\paragraph{Title Page.}

Conditional processing can be used to include a title or banner page
in the main document when proper precautions are taken.
Importantly, the code in the main file should ensure that the page counter
(as well as other status parameters which are stored in the |.aux| files)
takes the same value after the conditional processing.
Otherwise the page numbers may take divergent values
depending on which part is compiled.

For example, a title page could be declared by:
%
\begin{center}
\begin{tabular}{l}
|\ifchilddoc\||else|\\
|\addtocounter{page}{-1}|\\
\textit{code for title page}\\
|\newpage|\\
|\||fi|
\end{tabular}
\end{center}
%
A banner page for the child documents can be generated by:
%
\begin{center}
\begin{tabular}{l}
|\ifchilddoc|\\
|\addtocounter{page}{-1}|\\
\textit{code for banner page}\\
|\newpage|\\
|\||fi|
\end{tabular}
\end{center}
%
Here one could write a message such as:
\begin{center}
|This is the part \childdocname{} of \childdocjob{}.|
\end{center}

%%%%%%%%%%%%%%%%%%%%%%%%%%%%%%%%%%%%%%%%%%%%%%%%%%%%%%%%%%%%%%%%%%%%%%%%%%%%%%%%
\subsection{Flags}
\label{sec:flags}

The package makes it easy to generate different versions
of the main or child documents.
To this end compilation flags can be defined
and assigned different default values.
They will be particularly useful in conjunction
with the forwarding mechanism described in \secref{sec:forward}.

For example, it may be useful to have a flag |\version|
which can be set to |draft| or |final|.
The document source will contain some conditional code
depending on the value of |\version|.
Suppose further, the flag should default to |final| for the main file
and to |draft| for child files
which is a natural assignment for editing the document.
This is achieved by placing the following code
in the preamble of the main document
(below the |\childdocmain| directive):
%
\begin{center}
\begin{tabular}{l}
|\ifchilddoc|\\
|\providecommand{\version}{draft}|\\
|\||else|\\
|\providecommand{\version}{final}|\\
|\||fi|
\end{tabular}
\end{center}
%
The definition by |\providecommand| makes sure
that previous definitions are not overwritten.
Further statements |\providecommand{\version}{...}|
can thus be added before the above code to override it.

For the main file, one might add a line
(between |\childdocmain| and the above block)
%
\begin{center}
|%\ifchilddoc\||else\providecommand{\version}{draft}\||fi|
\end{center}
%
which can be uncommented to produce a draft version.
Likewise one can add a line to the very top of a child file
(above the |\childdocof{|\textit{main}|}| directive)
%
\begin{center}
|%\providecommand{\version}{final}|
\end{center}
%
which can be uncommented to produce the final version of this child document.

%%%%%%%%%%%%%%%%%%%%%%%%%%%%%%%%%%%%%%%%%%%%%%%%%%%%%%%%%%%%%%%%%%%%%%%%%%%%%%%%
\subsection{Forwarding}
\label{sec:forward}

Different versions of the main or child documents
using compilation flags as described in \secref{sec:flags}
can be (permanently) stored in different files
for convenient compilation, viewing and distribution.
To this end, the package defines a command
to pass on compilation to a different file:

%%%%%%%%%%%%%%%%%%%%%%%%%%%%%%%%%%%%%%%%
\DescribeMacro{\childdocforward}
The command |\childdocforward| redirects processing to
another source file:
%
\begin{center}
\begin{tabular}{l}
|\input{childdoc.def}|\\
|\childdocforward[|\textit{main}|]{|\textit{dest}|}|\\
\end{tabular}
\end{center}
%
The argument \textit{dest} is the destination file
(without extension).
It should be the main file or one of the child files.
Note that further \textsf{childdoc} directives
such as |\childdocof| and |\childdocforward|
in the indicated file will be processed in this form.
The optional argument \textit{main}
passes on directly to the main file \textit{main}
while pretending to compile the child \textit{dest}.
This form behaves as if \textit{dest}
issues |\childdocof{|\textit{main}|}| right away,
and no further \textsf{childdoc} directives will be processed.

%%%%%%%%%%%%%%%%%%%%%%%%%%%%%%%%%%%%%%%%
\DescribeMacro{\...prefix}
In the alternative form |\childdocforwardprefix|,
%
\begin{center}
\begin{tabular}{l}
|\input{childdoc.def}|\\
|\childdocforwardprefix[|\textit{main}|]{|\textit{prefix}|}{|\textit{dest}|}|
\end{tabular}
\end{center}
%
the destination file is determined by a pattern
depending on the current file:
To make this work, the current file must be called
`{\textit{prefix}\hspace{0.2em}\textit{suffix}}'
with \textit{prefix} matching precisely the argument.
Processing is then passed on to the file
`{\textit{dest}\hspace{0.2em}\textit{suffix}}'.
Surely, the same effect is achieved by
directly specifying the
argument `{\textit{dest}\hspace{0.2em}\textit{suffix}}'
in the first form.
However, that requires to set up a different file
for each child. With the alternative form of the command
all these files can have exactly the same content
which simplifies setting them up and maintaining them.

For example, the following file |draft.tex|
with a compilation flag |\version| as described in \secref{sec:flags}
compiles the main document as a draft:
%
\begin{center}
\begin{tabular}{l}
|\def\version{draft}|\\
|\input{childdoc.def}|\\
|\childdocforward{|\textit{main}|}|
\end{tabular}
\end{center}
%
Likewise, the following files |final|\textit{nn}|.tex|
compile the final version of the child document
|child|\textit{nn}|.tex|:
%
\begin{center}
\begin{tabular}{l}
|\def\version{final}|\\
|\input{childdoc.def}|\\
|\childdocforwardprefix{final}{child}|
\end{tabular}
\end{center}
%

Note that when several versions of a main file and/or of each child file
are to be generated, it may be convenient to set up a |Makefile| or
shell script to automatise the process.

%%%%%%%%%%%%%%%%%%%%%%%%%%%%%%%%%%%%%%%%%%%%%%%%%%%%%%%%%%%%%%%%%%%%%%%%%%%%%%%%
\subsection{Command Line Processing}
\label{sec:commandline}

The effect of redirection files can also be achieved by invoking
the \LaTeX{} compiler with a more elaborate command line.
Most conveniently this should be done as part
of a shell script or a |Makefile|.

When using \textsf{childdoc} in the main file, the following
command lines effectively perform a redirection
(note that depending on the shell being used,
backslashes may have to be doubled: `|\|' $\to$ `|\\|'):
%
\begin{center}
|... -jobname "|\textit{target}|" |\\|"|[\textit{flags}]%
|\input{childdoc.def}\childdocforward[|\textit{main}|]{|\textit{dest}|}"|
\end{center}
%
Here \textit{target} is the name of the output file,
\textit{main} is the name of the main file
and \textit{dest} is the name of the main or child file to be processed
(all filenames without extensions).
The optional argument \textit{main} can be omitted
if \textit{main} matches \textit{dest}.
Optionally, compilation \textit{flags} can be defined via |\def| commands.
This command line makes the \TeX{} engine believe
it is compiling the file \textit{target}
whose content is specified as the latter parameter.
The provided code then forwards the processing to
\textit{main} or \textit{dest} as described in \secref{sec:forward}.

%%%%%%%%%%%%%%%%%%%%%%%%%%%%%%%%%%%%%%%%%%%%%%%%%%%%%%%%%%%%%%%%%%%%%%%%%%%%%%%%
\subsection{Include by Input}
\label{sec:input}

Including child documents by |\include| has some restrictions by design.
Most notably, the content of a child document always occupies
its own set of pages; pages cannot be shared between child documents.
Usually, this behaviour makes perfect sense
because each child document contain an essential part of the document.
However, in some situations it may be desirable to compose
a document from a collection of parts
without having mandatory page breaks between then.
For this case, the package
provides a mechanism to include parts
by |\input| which can also be processed individually.
However, by construction this mechanism
requires manual handling of the content to be output.

%%%%%%%%%%%%%%%%%%%%%%%%%%%%%%%%%%%%%%%%
\DescribeMacro{\ifchilddocmanual}
The main file should be prepared as usual, see \secref{sec:include}.
However, the document body must make a distinction
between processing of an individual part and of the main document, e.g.:
%
\begin{center}
\begin{tabular}{l}
|\ifchilddocmanual|\\
|\input{\childdocname}|\\
|\||else|\\
\textit{document body with }|\input{|\textit{part}|}|\\
|\||fi|
\end{tabular}
\end{center}
%
The conditional |\ifchilddocmanual| is true whenever
a part to be included by |\input| is being compiled,
and the name of the part is stored in |\childdocname|.

%%%%%%%%%%%%%%%%%%%%%%%%%%%%%%%%%%%%%%%%
\DescribeMacro{\childdocby}
Each part to be included by |\input| should start with:
%
\begin{center}
\begin{tabular}{l}
|\input{childdoc.def}|\\
|\childdocby{|\textit{main}|}|\\
\end{tabular}
\end{center}
%
The directive |\childdocby| is similar to |\childdocof|
described in \secref{sec:include},
but the subsequent selection of content must be done manually.
To that end, both |\ifchilddoc| and |\ifchilddocmanual|
will be true upon processing of a part,
and the name of the part is stored in |\childdocname|.
Note that |\jobname| will be set to the filename of the current part
so that each part receives an individual |.aux| file
that does not interfere with the |.aux| file(s) of the main document.
This behaviour can be altered by the alternative form
|\childdocby[*]{|\textit{main}|}| (with a non-empty optional argument)
which uses the |.aux| file of the main document
by setting |\jobname| to \textit{main}.

%%%%%%%%%%%%%%%%%%%%%%%%%%%%%%%%%%%%%%%%%%%%%%%%%%%%%%%%%%%%%%%%%%%%%%%%%%%%%%%%
\subsection{Driver Development}
\label{sec:driver}

The \textsf{childdoc} mechanism can also be use for the development
of definition files such as \LaTeX{} styles or classes.
This case differs from the above setup with multiple parts
included by |\include| in that no |\includeonly| should be invoked.
This can be achieved by starting the include file
(before |\ProvidesPackage|) with:
%
\begin{center}
\begin{tabular}{l}
|\input{childdoc.def}|\\
|\childdocforward{|\textit{main}|}|\\
\end{tabular}
\end{center}
%
or alternatively with:
%
\begin{center}
\begin{tabular}{l}
|\input{childdoc.def}|\\
|\childdocby{|\textit{main}|}|\\
\end{tabular}
\end{center}
%
Both forms have slightly different effects as described above.
The main file is prepared as usual, see \secref{sec:include}.

%%%%%%%%%%%%%%%%%%%%%%%%%%%%%%%%%%%%%%%%%%%%%%%%%%%%%%%%%%%%%%%%%%%%%%%%%%%%%%%%
\subsection{Legacy Detection}
\label{sec:detection}

The directive |\childdocmain| in the main file can detect
whether the complete document or merely a child is to be compiled
even without using the directive |\childdocof|.
This method is deprecated because it is less robust
and there is no compelling reason to use it;
it is merely provided for backward compatibility
and it may be removed in future versions.

If the detection mechanism is to be used,
it is mandatory to correctly specify
the filename of the main file as the argument of |\childdocmain|:
%
\begin{center}
\begin{tabular}{l}
|\input{childdoc.def}|\\
|\childdocmain{|\textit{main}|}|\\
\end{tabular}
\end{center}
%
If |\jobname| does not match the argument \textit{main} of |\childdocmain|,
it is assumed that |\jobname| points to the child file to be compiled.
When using |\childdocmain| with the main file specified as argument,
it suffices to start a child file
with just |\input{|\textit{main}|}|
without loading of the package and using |\childdocof|.
If instead all processing is done
with the appropriate \textsf{childdoc} directives,
the argument of \textit{main} of |\childdocmain| can be empty.

An alternative version of the command line processing described
in \secref{sec:commandline} using the detection mechanism reads:
%
\begin{center}
|... -jobname "|\textit{target}|" "|[\textit{flags}]%
[|\def\jobname{|\textit{dest}|}|]|\input{|\textit{main}|}"|
\end{center}

%%%%%%%%%%%%%%%%%%%%%%%%%%%%%%%%%%%%%%%%%%%%%%%%%%%%%%%%%%%%%%%%%%%%%%%%%%%%%%%%
\subsection{Manual Code}
\label{sec:manual}

In case one cannot be certain whether the definitions file |childdoc.def|
is installed on the target \TeX{} distribution
and one prefers not to ship it,
it is conceivable to paste a few relevant commands into the sources.

To that end, drop all statements |\input{childdoc.def}|
and perform the replacements as outlined below.
Instead of |\childdocmain{|\textit{main}|}| add the following code
to the top of the main file:
%
\begin{center}
\begin{tabular}{l}
|\||ifdefined\childdocname\endinput\||fi\newif\ifchilddoc|\\
|\edef\childdocname{\scantokens\expandafter{\jobname\noexpand}}|\\
|\def\childdocmain{|\textit{main}|}\||ifx\childdocmain\childdocname\||else|\\
|\childdoctrue\includeonly{\childdocname}\let\jobname\childdocmain\||fi|\\
\end{tabular}
\end{center}
%
Instead of |\childdocof{|\textit{main}|}| just include the main file
at the top of each child file:
%
\begin{center}
|\input{|\textit{main}|}|
\end{center}
%
A simple redirection |\childdocforward{|\textit{dest}|}| is achieved by:
%
\begin{center}
|\def\jobname{|\textit{dest}|}\input{\jobname}|
\end{center}
%
The redirection with prefix
|\childdocforwardprefix[|\textit{prefix}|]{|\textit{dest}|}|
is accomplished by:
%
\begin{center}
\begin{tabular}{l}
|{\edef\jobname{\scantokens\expandafter{\jobname\noexpand}}|\\
|\def\redirectjob |\textit{prefix}|#1~~~{\gdef\jobname{|\textit{dest}|#1}}|\\
|\expandafter\redirectjob\jobname~~~}\input{\jobname}|
\end{tabular}
\end{center}

In an alternative approach,
child documents can be compiled by a specific command line
without additional code or specific definitions:
%
\begin{center}
|... -jobname "|\textit{target}|" "|[\textit{flags}]%
|\includeonly{|\textit{dest}|}\input{|\textit{main}|}"|
\end{center}
%

%%%%%%%%%%%%%%%%%%%%%%%%%%%%%%%%%%%%%%%%%%%%%%%%%%%%%%%%%%%%%%%%%%%%%%%%%%%%%%%%
%%%%%%%%%%%%%%%%%%%%%%%%%%%%%%%%%%%%%%%%%%%%%%%%%%%%%%%%%%%%%%%%%%%%%%%%%%%%%%%%
\section{Information}

%%%%%%%%%%%%%%%%%%%%%%%%%%%%%%%%%%%%%%%%%%%%%%%%%%%%%%%%%%%%%%%%%%%%%%%%%%%%%%%%
\subsection{Copyright}

Copyright \copyright{} 2017--2018 Niklas Beisert

This work may be distributed and/or modified under the
conditions of the \LaTeX{} Project Public License, either version 1.3
of this license or (at your option) any later version.
The latest version of this license is in
  \url{http://www.latex-project.org/lppl.txt}
and version 1.3 or later is part of all distributions of \LaTeX{}
version 2005/12/01 or later.

This work has the LPPL maintenance status `maintained'.

The Current Maintainer of this work is Niklas Beisert.

This work consists of the files |README.txt|, |childdoc.ins| and |childdoc.dtx|
as well as the derived files |childdoc.def|, |cdocsamp.tex|
with |cdocsch1.tex|, |cdocsch2.tex|, |cdocspt3.tex|, |cdocspt4.tex|,
|cdocsdrf.tex|, |cdocsfn1.tex|, |cdocsfn2.tex|
as well as |childdoc.pdf|.

%%%%%%%%%%%%%%%%%%%%%%%%%%%%%%%%%%%%%%%%%%%%%%%%%%%%%%%%%%%%%%%%%%%%%%%%%%%%%%%%
\subsection{Files and Installation}

The package consists of the files:
%
\begin{center}
\begin{tabular}{ll}
    |README.txt|   & readme file \\
    |childdoc.ins| & installation file \\
    |childdoc.dtx| & source file \\
    |childdoc.def| & definition file \\
    |cdocsamp.tex| & sample main file \\
    |cdocsch1.tex| & sample include file \\
    |cdocsch2.tex| & sample include file \\
    |cdocspt3.tex| & sample part file \\
    |cdocspt4.tex| & sample part file \\
    |cdocsdrf.tex| & sample redirection file \\
    |cdocsfn1.tex| & sample redirection file \\
    |cdocsfn2.tex| & sample redirection file \\
    |childdoc.pdf| & manual
\end{tabular}
\end{center}
%
The distribution consists of the files
|README.txt|, |childdoc.ins| and |childdoc.dtx|.
%
\begin{itemize}
\item
Run (pdf)\LaTeX{} on |childdoc.dtx|
to compile the manual |childdoc.pdf| (this file).
\item
Run \LaTeX{} on |childdoc.ins| to create the definitions file |childdoc.def|
and the sample |cdocsamp.tex| with include files
|cdocsch1.tex|, |cdocsch2.tex|, |cdocspt3.tex|, |cdocspt4.tex|,
|cdocsdrf.tex|, |cdocsfn1.tex|, |cdocsfn2.tex|.
Then copy the file |childdoc.def| to an appropriate directory of your \LaTeX{}
distribution, e.g.\ \textit{texmf-root}|/tex/latex/childdoc|.
\end{itemize}

%%%%%%%%%%%%%%%%%%%%%%%%%%%%%%%%%%%%%%%%%%%%%%%%%%%%%%%%%%%%%%%%%%%%%%%%%%%%%%%%
\subsection{Related CTAN Packages}

There are several other packages which offer a similar functionality:
%
\begin{itemize}
\item
The packages
\href{http://ctan.org/pkg/docmute}{\textsf{docmute}},
\href{http://ctan.org/pkg/includex}{\textsf{includex}} and
\href{http://ctan.org/pkg/standalone}{\textsf{standalone}}
provide commands to include only the document body of
a child file thus allowing both files to be compiled individually.
\item
The packages \href{http://ctan.org/pkg/subdocs}{\textsf{subdocs}}
and \href{http://ctan.org/pkg/subfiles}{\textsf{subfiles}}
provide structures in which the main and child documents can be
encapsulated and allowing them to be compiled individually.
The inclusion mechanism is different from the conventional |\include|.
\item
The package \href{http://ctan.org/pkg/combine}{\textsf{combine}}
is an elaborate solution to combine several documents into one.
\end{itemize}
%
See also the CTAN topic \href{http://ctan.org/topic/subdocs}{\textsf{subdocs}}
for further related packages.
The present package differs from the above solutions in that
a document structure constructed with the conventional |\include| mechanism
just needs two extra commands at the top of every file
such that all constituent files can be compiled individually.

%%%%%%%%%%%%%%%%%%%%%%%%%%%%%%%%%%%%%%%%%%%%%%%%%%%%%%%%%%%%%%%%%%%%%%%%%%%%%%%%
%\subsection{Feature Suggestions}
%
%The following is a list of features which may be useful for future
%versions of this package:
%%
%\begin{itemize}
%\item
%\ldots
%\end{itemize}

%%%%%%%%%%%%%%%%%%%%%%%%%%%%%%%%%%%%%%%%%%%%%%%%%%%%%%%%%%%%%%%%%%%%%%%%%%%%%%%%
\subsection{Revision History}

%%%%%%%%%%%%%%%%%%%%%%%%%%%%%%%%%%%%%%%%
\paragraph{v2.0:} 2018/12/30

\begin{itemize}
\item
immediate forward processing
\item
added |\childdocby| mechanism
\item
manual restructured
\end{itemize}

%%%%%%%%%%%%%%%%%%%%%%%%%%%%%%%%%%%%%%%%
\paragraph{v1.6:} 2018/01/17

\begin{itemize}
\item
application for development of include files
\item
corrections to manual
\end{itemize}

%%%%%%%%%%%%%%%%%%%%%%%%%%%%%%%%%%%%%%%%
\paragraph{v1.5:} 2017/05/21

\begin{itemize}
\item
more complete structuring introduced
\item
|\childdocof| introduced
\item
|\childdoc| renamed to |\childdocmain|
\item
|\childredirect| renamed to |\childdocforward| and |\childdocforwardprefix|
and functionality expanded
\end{itemize}

%%%%%%%%%%%%%%%%%%%%%%%%%%%%%%%%%%%%%%%%
\paragraph{v1.0:} 2017/04/27

\begin{itemize}
\item
manual and install package
\item
first version published on CTAN
\end{itemize}

%%%%%%%%%%%%%%%%%%%%%%%%%%%%%%%%%%%%%%%%
\paragraph{v0.6:} 2017/04/26

\begin{itemize}
\item
redirection mechanism added
\end{itemize}

%%%%%%%%%%%%%%%%%%%%%%%%%%%%%%%%%%%%%%%%
\paragraph{v0.5:} 2017/04/26

\begin{itemize}
\item
functionality in definition file
\end{itemize}


%%%%%%%%%%%%%%%%%%%%%%%%%%%%%%%%%%%%%%%%%%%%%%%%%%%%%%%%%%%%%%%%%%%%%%%%%%%%%%%%
%%%%%%%%%%%%%%%%%%%%%%%%%%%%%%%%%%%%%%%%%%%%%%%%%%%%%%%%%%%%%%%%%%%%%%%%%%%%%%%%
%%%%%%%%%%%%%%%%%%%%%%%%%%%%%%%%%%%%%%%%%%%%%%%%%%%%%%%%%%%%%%%%%%%%%%%%%%%%%%%%
\appendix

\settowidth\MacroIndent{\rmfamily\scriptsize 000\ }

 \DocInput{childdoc.dtx}

\end{document}
%</driver>
% \fi
%
% %%%%%%%%%%%%%%%%%%%%%%%%%%%%%%%%%%%%%%%%%%%%%%%%%%%%%%%%%%%%%%%%%%%%%%%%%%%%%%
% %%%%%%%%%%%%%%%%%%%%%%%%%%%%%%%%%%%%%%%%%%%%%%%%%%%%%%%%%%%%%%%%%%%%%%%%%%%%%%
% \section{Sample}
%\iffalse
%<*samplemain>
%\fi
%
% The following presents a sample document
% with two chapters, two parts, a title page,
% a compile flag as well as three forwarding files to set the flag.
% It consists of eight |.tex| files:
% \begin{center}
% \begin{tabular}{ll}
% |cdocsamp.tex|&main file\\
% |cdocsch1.tex|&include file for chapter 1\\
% |cdocsch2.tex|&include file for chapter 2\\
% |cdocspt3.tex|&include file for part 3\\
% |cdocspt4.tex|&include file for part 4\\
% |cdocsdrf.tex|&forwarding file for main file in draft mode\\
% |cdocsfi1.tex|&forwarding file for final version of chapter 1\\
% |cdocsfi2.tex|&forwarding file for final version of chapter 2\\
% \end{tabular}
% \end{center}
% Each of the eight files can be compiled directly by the \LaTeX{} compiler.
%
% %%%%%%%%%%%%%%%%%%%%%%%%%%%%%%%%%%%%%%
% \paragraph{Main File.}
%
% The main file is called |cdocsamp.tex|.
%
% Load the \textsf{childdoc} definitions and
% declare the filename for the main document:
%    \begin{macrocode}
\input{childdoc.def}
\childdocmain{}
%    \end{macrocode}

% Optional override for |\version| flag:
%    \begin{macrocode}
%%\ifchilddoc\else\providecommand{\version}{draft}\fi
%    \end{macrocode}

% Define the default values for the |\version| flag
% (|final| for the main file and |draft| for childs):
%    \begin{macrocode}
\ifchilddoc
\providecommand{\version}{draft}
\else
\providecommand{\version}{final}
\fi
%    \end{macrocode}

% Load the standard document class:
%    \begin{macrocode}
\documentclass[12pt]{article}
%    \end{macrocode}

% Start the document body:
%    \begin{macrocode}
\begin{document}
%    \end{macrocode}

% Declare a title page.
% Print title, part of document being processed and version flag:
%    \begin{macrocode}
\addtocounter{page}{-1}
\begin{center}
{\LARGE\bfseries{}childdoc example\par}
\vspace{1cm}
\ifchilddoc
\ifchilddocmanual part\else chapter\fi:
`\childdocname' of `\childdocjob'\par
\else
main document: `\childdocjob'\par
\fi
version: \version\par
\end{center}
\newpage
%    \end{macrocode}

% Manually include selected file,
% otherwise process as usual:
%    \begin{macrocode}
\ifchilddocmanual
\section*{part `\childdocname'}
\input{\childdocname}
\else
%    \end{macrocode}

% Include the two chapters:
%    \begin{macrocode}
\include{cdocsch1}
\include{cdocsch2}
%    \end{macrocode}

% Include the two parts unless only chapters should be displayed:
%    \begin{macrocode}
\ifchilddoc\else
\section{part three}
\input{cdocspt3}
\section{part four}
\input{cdocspt4}
\fi
%    \end{macrocode}

% Process as usual until here:
%    \begin{macrocode}
\fi
%    \end{macrocode}

% End of document body:
%    \begin{macrocode}
\end{document}
%    \end{macrocode}
%\iffalse
%</samplemain>
%\fi
%
% %%%%%%%%%%%%%%%%%%%%%%%%%%%%%%%%%%%%%%
% \paragraph{Chapter Include Files.}
%
% The include files are called |cdocsch1.tex| and |cdocsch2.tex|.
%
%\iffalse
%<*samplechap1|samplechap2>
%\fi

% Optional override for |\version| flag:
%    \begin{macrocode}
%%\providecommand{\version}{final}
%    \end{macrocode}

% Include the main document:
%    \begin{macrocode}
\input{childdoc.def}
\childdocof{cdocsamp}
%    \end{macrocode}

%\iffalse
%</samplechap1|samplechap2>
%\fi
%
%\iffalse
%<*samplechap1>
%\fi
% Some text for chapter 1:
%    \begin{macrocode}
\section{one}
some text in chapter one
%    \end{macrocode}

%\iffalse
%</samplechap1>
%\fi
% Some text for chapter 2:
%\iffalse
%<*samplechap2>
%\fi
%    \begin{macrocode}
\section{two}
more text in chapter two
%    \end{macrocode}

%\iffalse
%</samplechap2>
%\fi
%
% %%%%%%%%%%%%%%%%%%%%%%%%%%%%%%%%%%%%%%
% \paragraph{Part Include Files.}
%
% The include files are called |cdocspt3.tex| and |cdocspt4.tex|.
%
%\iffalse
%<*samplepart3|samplepart4>
%\fi

% Optional override for |\version| flag:
%    \begin{macrocode}
%%\providecommand{\version}{final}
%    \end{macrocode}

% Include the main document:
%    \begin{macrocode}
\input{childdoc.def}
\childdocby{cdocsamp}
%    \end{macrocode}

%\iffalse
%</samplepart3|samplepart4>
%\fi
%
%\iffalse
%<*samplepart3>
%\fi
% Some text for part 3:
%    \begin{macrocode}
some text in part three
%    \end{macrocode}

%\iffalse
%</samplepart3>
%\fi
% Some text for part 4:
%\iffalse
%<*samplepart4>
%\fi
%    \begin{macrocode}
more text in part four
%    \end{macrocode}

%\iffalse
%</samplepart4>
%\fi
%
% %%%%%%%%%%%%%%%%%%%%%%%%%%%%%%%%%%%%%%
% \paragraph{Forwarding for a Complete Draft.}
%
% The following forwarding file |cdocsdrf.tex|
% compiles the main document in draft mode:
%\iffalse
%<*sampledraft>
%\fi
%    \begin{macrocode}
\def\version{draft}
\input{childdoc.def}
\childdocforward{cdocsamp}
%    \end{macrocode}

%\iffalse
%</sampledraft>
%\fi
%
% %%%%%%%%%%%%%%%%%%%%%%%%%%%%%%%%%%%%%%
% \paragraph{Forwarding for Final Version of the Chapters.}
%
% The following forwarding files |cdocsfn1.tex| and |cdocsfn2.tex|
% (with identical content)
% compile the final versions of the child documents
% |cdocsch1.tex| and |cdocsch2.tex|, respectively:
%\iffalse
%<*samplefinal>
%\fi
%    \begin{macrocode}
\def\version{final}
\input{childdoc.def}
\childdocforwardprefix[cdocsamp]{cdocsfn}{cdocsch}
%    \end{macrocode}

%\iffalse
%</samplefinal>
%\fi
%
% %%%%%%%%%%%%%%%%%%%%%%%%%%%%%%%%%%%%%%
% \paragraph{Command Line Processing.}
%
% The following three command lines generate the output files
% |cdocscld|, |cdocscl1| and |cdocscl2|
% which should be identical to
% |cdocsdrf|, |cdocsch1| and |cdocsfn2|, respectively:
% \begin{center}
% \begin{tabular}{l}
% |latex -jobname cdocscld \|\\
% |  "\def\version{draft}\input{childdoc.def}\childdocforward{cdocsamp}"|\\
% |latex -jobname cdocscl1 \|\\
% |  "\input{childdoc.def}\childdocforward[cdocsamp]{cdocsch1}"|\\
% |latex -jobname cdocscl2 \|\\
% |  "\def\version{final}\input{childdoc.def}\childdocforward{cdocsch2}"|
% \end{tabular}
% \end{center}
% Note that the trailing backslash on each first line
% merely continues the input to the second line
% (for convenient cut ant paste).
% Furthermore, the command |latex| can be replaced by any
% of its alternative versions such as |pdflatex|.
%
% %%%%%%%%%%%%%%%%%%%%%%%%%%%%%%%%%%%%%%%%%%%%%%%%%%%%%%%%%%%%%%%%%%%%%%%%%%%%%%
% %%%%%%%%%%%%%%%%%%%%%%%%%%%%%%%%%%%%%%%%%%%%%%%%%%%%%%%%%%%%%%%%%%%%%%%%%%%%%%
% \section{Implementation}
%\iffalse
%<*package>
%\fi
%
% This section describes the definitions file |childdoc.def|.

% The definitions cannot be loaded using |\usepackage| or |\RequirePackage|
% which has a mechanism to prevent loading a style file more than once.
% When loading the definitions by means of |\input|
% multiple instances have to be prevented manually:
%\iffalse
%This code needs to be before the `\ProvidesFile' directive
%which is defined at the beginning of this file.
%Therefore it is also placed there and commented out here.
%</package>
%<*discard>
%\fi
%    \begin{macrocode}
\ifdefined\childdocmain\endinput\fi
%    \end{macrocode}
%\iffalse
%</discard>
%<*package>
%\fi
%
% \macro{\ifchilddoc}
% \macro{\ifchilddocmanual}
% The conditional |\ifchilddoc| tells whether a
% child (true) or main (false) document is being compiled.
% The conditional |\ifchilddocmanual| tells whether
% the |\includeonly| mechanism is used (false) or
% the selection of child files must be performed manually (true).
% The definitions initialise to false:
%    \begin{macrocode}
\newif\ifchilddoc
\newif\ifchilddocmanual
%    \end{macrocode}

% \macro{\childdocname}
% \macro{\childdocjob}
% The macro |\childdocname| stores the name of the main document
% to be compiled. The macro |\childdocjob| stores the name of
% the document on which the \LaTeX{} compiler was originally invoked.
% The content of |\jobname| cannot be compared
% to filenames specified in the source due to different catcodes.
% The following code rescans |\jobname|, stores the result
% in |\childdocname| and saves a copy in |\childdocjob|:
%    \begin{macrocode}
\edef\childdocname{\scantokens\expandafter{\jobname\noexpand}}
\let\childdocjob\childdocname
%    \end{macrocode}

% \macro{\childdocdisable}
% The macro |\childdocdisable| prevents the main file
% from being processed more than once.
% At this stage, the main document command |\childdocmain|
% is assumed to be called once again where it should do nothing.
% Any subsequent call to it should prevent
% a secondary processing of the main document
% It overwrites the forwarding commands
% |\childdocof| and |\childdocforward|
% with empty macros to prevent further inclusions of the main document:
%    \begin{macrocode}
\newcommand{\childdocdisable}
{
  \renewcommand{\childdocmain}[1]{\renewcommand{\childdocmain}[1]{\endinput}}
  \renewcommand{\childdocof}[1]{}
  \renewcommand{\childdocby}[2][]{}
  \renewcommand{\childdocforward}[2][]{}
  \renewcommand{\childdocdisable}{}
}
%    \end{macrocode}

% \macro{\childdocmain}
% The macro |\childdocmain| is to be called at the top of the main file
% with nothing or the main filename (without extension) as argument.
% First, it breaks loops.
% If the argument is not empty and does not match |\childdocname|
% (which is set by the first inclusion of |childdoc.def|),
% |\ifchilddoc| is set to true, |\includeonly| is applied to the child file
% and |\jobname| is set to the main file
% (for proper handling of |.aux| files):
%    \begin{macrocode}
\newcommand{\childdocmain}[1]
{
  \childdocdisable\childdocmain{}
  \if?#1?\else
    \begingroup
      \def\childdoctmp{#1}
      \ifx\childdoctmp\childdocname
        \def\childdoctmp{}
      \else
        \def\childdoctmp
        {
          \childdoctrue
          \includeonly{\childdocname}
          \def\childdocjob{#1}
          \def\jobname{#1}
        }
      \fi
      \expandafter
    \endgroup
    \childdoctmp
  \fi
}
%    \end{macrocode}

% \macro{\childdocof}
% The command |\childdocof| redirects
% compilation to the main file |#1|.
%    \begin{macrocode}
\newcommand{\childdocof}[1]
{
  \childdocdisable
  \childdoctrue
  \includeonly{\childdocname}
  \def\jobname{#1}
  \def\childdocjob{#1}
  \input{#1}
}
%    \end{macrocode}

% \macro{\childdocby}
% The command |\childdocby| ....
%    \begin{macrocode}
\newcommand{\childdocby}[2][]
{
  \childdocdisable
  \childdoctrue
  \childdocmanualtrue
  \if?#1?\else
    \def\jobname{#2}
  \fi
  \def\childdocjob{#2}
  \input{#2}
  \endinput
}
%    \end{macrocode}

% \macro{\childdocforward}
% The command |\childdocforward| redirects
% compilation to the main file or
% (if the optional argument is given) a child file.
% Parameters are set as if the main file
% or a child file starting with |\childdocof| was compiled.
% Then compilation is handed over to the main file:
%    \begin{macrocode}
\newcommand{\childdocforward}[2][]
{
  \begingroup
    \if?#1?
      \def\childdoctmp
      {
        \def\childdocname{#2}
        \def\childdocjob{#2}
        \def\jobname{#2}
        \input{#2}
        \endinput
      }
    \else
      \def\childdoctmp
      {
        \childdocdisable
        \def\childdocname{#2}
        \childdoctrue
        \includeonly{#2}
        \def\childdocjob{#1}
        \def\jobname{#1}
        \input{#1}
        \endinput
      }
    \fi
    \expandafter
  \endgroup
  \childdoctmp
}
%    \end{macrocode}

% \macro{\childdocforwardprefix}
% The command |\childdocforwardprefix| redirects
% compilation to the main or a child file by means of a pattern.
% The prefix |#1| in the current filename is replaced by |#2|
% and the suffix of the current filename is kept
% (it is assumed that the filename does not contain the substring `|~~~|'
% which is used as a delimiter).
% Compilation is handed over to the new file by |\childdocforward|:
%    \begin{macrocode}
\newcommand{\childdocforwardprefix}[3][]
{
  \begingroup
    \def\childdocextract #2##1~~~{\def\childdoctmp{\childdocforward[#1]{#3##1}}}
    \expandafter\childdocextract\childdocname~~~
    \expandafter
  \endgroup
  \childdoctmp
}
%    \end{macrocode}

% \macro{\childdoc}
% The deprecated macro |\childdoc| is a legacy version of |\childdocmain|:
%    \begin{macrocode}
\newcommand{\childdoc}{\childdocmain}
%    \end{macrocode}

% \macro{\childdocredirect}
% The deprecated macro |\childdocredirect| is a legacy version
% of |\childdocforward| and |\childdocforwardprefix|:
%    \begin{macrocode}
\newcommand{\childdocredirect}[2][]
{
  \begingroup
    \if?#1?
      \def\childdoctmp{\childdocforward{#2}}
    \else
      \def\childdoctmp{\childdocforwardprefix{#1}{#2}}
    \fi
    \expandafter
  \endgroup
  \childdoctmp
}
%    \end{macrocode}

%\iffalse
%</package>
%\fi
%
\endinput

\childdocforwardprefix[cdocsamp]{cdocsfn}{cdocsch}
%    \end{macrocode}

%\iffalse
%</samplefinal>
%\fi
%
% %%%%%%%%%%%%%%%%%%%%%%%%%%%%%%%%%%%%%%
% \paragraph{Command Line Processing.}
%
% The following three command lines generate the output files
% |cdocscld|, |cdocscl1| and |cdocscl2|
% which should be identical to
% |cdocsdrf|, |cdocsch1| and |cdocsfn2|, respectively:
% \begin{center}
% \begin{tabular}{l}
% |latex -jobname cdocscld \|\\
% |  "\def\version{draft}% \iffalse
%
% childdoc.dtx Copyright (C) 2017-2018 Niklas Beisert
%
% This work may be distributed and/or modified under the
% conditions of the LaTeX Project Public License, either version 1.3
% of this license or (at your option) any later version.
% The latest version of this license is in
%   http://www.latex-project.org/lppl.txt
% and version 1.3 or later is part of all distributions of LaTeX
% version 2005/12/01 or later.
%
% This work has the LPPL maintenance status `maintained'.
%
% The Current Maintainer of this work is Niklas Beisert.
%
% This work consists of the files childdoc.dtx and childdoc.ins
% and the derived files childdoc.def and cdocsamp.tex with
% cdocsch1.tex, cdocsch2.tex, cdocsdrf.tex, cdocsfn1.tex, cdocsfn2.tex.
%
%<package>\ifdefined\childdocmain\endinput\fi
%<package>\ProvidesFile{childdoc.def}[2018/12/30 v2.0 child document driver]
%<samplemain>\ProvidesFile{cdocsamp.tex}[2018/12/30 v2.0 sample for childdoc]
%<*driver>
%\ProvidesFile{childdoc.drv}[2018/12/30 v2.0 childdoc reference manual file]
\PassOptionsToClass{10pt,a4paper}{article}
\documentclass{ltxdoc}

\usepackage[margin=35mm]{geometry}
\usepackage{hyperref}
\usepackage{hyperxmp}
\usepackage[usenames]{color}

\hypersetup{colorlinks=true}
\hypersetup{pdfstartview=FitH}
\hypersetup{pdfpagemode=UseNone}
\hypersetup{pdfsource={}}
\hypersetup{pdflang={en-UK}}
\hypersetup{pdfcopyright={Copyright 2017-2018 Niklas Beisert.
  This work may be distributed and/or modified under the
  conditions of the LaTeX Project Public License, either version 1.3
  of this license or (at your option) any later version.}}
\hypersetup{pdflicenseurl={http://www.latex-project.org/lppl.txt}}
\hypersetup{pdfcontactaddress={ETH Zurich, ITP, HIT K,
  Wolfgang-Pauli-Strasse 27}}
\hypersetup{pdfcontactpostcode={8093}}
\hypersetup{pdfcontactcity={Zurich}}
\hypersetup{pdfcontactcountry={Switzerland}}
\hypersetup{pdfcontactemail={nbeisert@itp.phys.ethz.ch}}
\hypersetup{pdfcontacturl={http://people.phys.ethz.ch/\xmptilde nbeisert/}}

\newcommand{\secref}[1]{\hyperref[#1]{section \ref*{#1}}}

\parskip1ex
\parindent0pt
\let\olditemize\itemize
\def\itemize{\olditemize\parskip0pt}

\begin{document}

\title{The \textsf{childdoc} Package}
\hypersetup{pdftitle={The childdoc Package}}
\author{Niklas Beisert\\[2ex]
  Institut f\"ur Theoretische Physik\\
  Eidgen\"ossische Technische Hochschule Z\"urich\\
  Wolfgang-Pauli-Strasse 27, 8093 Z\"urich, Switzerland\\[1ex]
  \href{mailto:nbeisert@itp.phys.ethz.ch}
  {\texttt{nbeisert@itp.phys.ethz.ch}}}
\hypersetup{pdfauthor={Niklas Beisert}}
\hypersetup{pdfsubject={Manual for the LaTeX2e Package childdoc}}
\date{30 December 2018, \textsf{v2.0}}
\maketitle

\begin{abstract}\noindent
\textsf{childdoc} is a \LaTeXe{} package
that enables the direct compilation
of document sections included by |\include|
to individual files.
\end{abstract}

\begingroup
\parskip0ex
\tableofcontents
\endgroup

%%%%%%%%%%%%%%%%%%%%%%%%%%%%%%%%%%%%%%%%%%%%%%%%%%%%%%%%%%%%%%%%%%%%%%%%%%%%%%%%
%%%%%%%%%%%%%%%%%%%%%%%%%%%%%%%%%%%%%%%%%%%%%%%%%%%%%%%%%%%%%%%%%%%%%%%%%%%%%%%%
\section{Introduction}

\LaTeX{} provides a mechanism to structure a large document (such as a book)
into a main file and several child files (containing the chapters)
using the |\include| command.
This mechanism is beneficial for documents
which span hundreds of pages in order to
make the source file(s) more manageable.
Moreover, compilation can be restricted to
selected child files by means of the |\includeonly| command.
The latter feature can be used to reduce the compilation time while editing
(this was significantly more useful in the earlier days of \LaTeX{})
or to generate a smaller document which is easier to navigate.
Another application of |\includeonly| is to generate
documents consisting of selected parts of the complete document.

However, there are a few drawbacks of the plain |\include| mechanism:
\begin{itemize}
\item
The child files cannot be compiled on their own,
they can only be compiled via the main file.
A naive editing environment
(such as a text editor with an option
to have the current file processed by \LaTeX)
may require one to switch to the main file before compiling;
attempting to compile the child file produces errors.
\item
The main file must be modified (each time)
to adjust the |\includeonly| command
to the present needs. This easily leaves the main file in a messy state.
\item
The generated document will always carry the filename
of the main document. This is inconvenient if
several child files are to be compiled and
to be kept for distribution.
\end{itemize}

The present package provides a simple interface
to make child files individually compilable by \LaTeX{}.
Compiling a child file then has the same effect as compiling
the main file with an |\includeonly| command
to select the appropriate child.
Moreover the generated document will carry the name of the child
rather than the main file.
This resolves all three above issues.

This feature is meant to make the editing of books,
thesis documents and lecture notes somewhat more convenient.
However, the package can also be used efficiently for
composing a series of documents (such as exercise sheets)
which are typically distributed individually.
It then assists the author in generating the individual documents
(potentially in different versions)
as well as a document containing the collected series.
Another application is in developing style files
or other kinds of included material
where compilation of the style file could redirect
to a sample or test file.

%%%%%%%%%%%%%%%%%%%%%%%%%%%%%%%%%%%%%%%%%%%%%%%%%%%%%%%%%%%%%%%%%%%%%%%%%%%%%%%%
%%%%%%%%%%%%%%%%%%%%%%%%%%%%%%%%%%%%%%%%%%%%%%%%%%%%%%%%%%%%%%%%%%%%%%%%%%%%%%%%
\section{Usage}

First of all, the package \textsf{childdoc} is \emph{not} a standard
\LaTeXe{} |.sty| style file! Therefore it needs to be invoked in
a non-standard way.

%%%%%%%%%%%%%%%%%%%%%%%%%%%%%%%%%%%%%%%%%%%%%%%%%%%%%%%%%%%%%%%%%%%%%%%%%%%%%%%%
\subsection{Included Files}
\label{sec:include}

%%%%%%%%%%%%%%%%%%%%%%%%%%%%%%%%%%%%%%%%
\DescribeMacro{\childdocmain}
To use the package, add the commands
\begin{center}
\begin{tabular}{l}
|\input{childdoc.def}|\\
|\childdocmain{}|\\
\end{tabular}
\end{center}
at the very top of the main \LaTeX{} file,
in particular \emph{before} the |\documentclass| statement!
The argument of |\childdocmain| should be left empty
(but it must be present).

%%%%%%%%%%%%%%%%%%%%%%%%%%%%%%%%%%%%%%%%
\DescribeMacro{\childdocof}
Furthermore, add the commands
\begin{center}
\begin{tabular}{l}
|\input{childdoc.def}|\\
|\childdocof{|\textit{main}|}|\\
\end{tabular}
\end{center}
at the top of every child file \textit{child}
which is included by |\include{|\textit{child}|}|
from within the main file
(or at least for those files to be compiled individually).
The argument \textit{main} must be the filename of the main file.

There are a couple of
considerations in setting up the main and child documents:

%%%%%%%%%%%%%%%%%%%%%%%%%%%%%%%%%%%%%%%%
\paragraph{Restrictions.}

Please note the following restrictions:
\begin{itemize}
\item
|\childdocmain| must be called with one argument \textit{main}
to ensure compatibility with earlier version of the package.
It must either be empty (|\childdocmain{}|)
or precisely match the filename of the main file in which it is specified.
See \secref{sec:detection} for further information.
\item
The filename \textit{main} must be specified without the |.tex| extension.
\item
The filename \textit{main} is case sensitive
(even in case-insensitive file systems)
due to internal string comparison.
\item
The argument \textit{main} should be fully expanded, it cannot be a macro.
\item
Subdirectories and special characters should be avoided in filenames.
\item
The command |\childdocmain{|\textit{main}|}| must be followed by a whitespace.
It should not be followed immediately by another command
or by a comment mark `|%|'.
This is because the \TeX{} parser reads the token immediately following
the argument of |\childdocmain| and puts it
at the beginning of every child section;
however, a white\-space is ignored.
\end{itemize}

%%%%%%%%%%%%%%%%%%%%%%%%%%%%%%%%%%%%%%%%
\paragraph{Content of Main File.}

It is advisable to place all content in the child files included by |\include|.
Any output contained in the main file will appear in all child documents
unless suppressed manually;
it cannot be suppressed automatically by the |\includeonly| directive
and thus should normally be avoided.
A method to include some content in the main file
by means of conditional processing is described in \secref{sec:conditional}.

%%%%%%%%%%%%%%%%%%%%%%%%%%%%%%%%%%%%%%%%
\paragraph{Page Numbering.}

When only a part of the document is compiled,
the appropriate numbering of pages
(as well as other status parameters)
is determined from the |.aux| files.
The latter contain information from previous passes.
However this information needs to propagate through
all intermediate child documents.
Therefore the page numbering in child documents may well
be inconsistent until the complete document is compiled at least once.

A useful (if unconventional) way to always ensure a consistent
page numbering is to restart the numbering in each child document
and denote the pages by `\textit{child}|.|\textit{page}'
where \textit{child} represents the chapter/section number of the child file.
This can be achieved by the command
|\numberwithin{page}{|\textit{child}|}|
of the \textsf{amsmath} package
where \textit{child} can be |chapter| or |section|
depending on the chosen structuring.
Alternatively, one can modify the macro |\thepage| appropriately
and reset the counter |page| at the start of each child file.

%%%%%%%%%%%%%%%%%%%%%%%%%%%%%%%%%%%%%%%%%%%%%%%%%%%%%%%%%%%%%%%%%%%%%%%%%%%%%%%%
\subsection{Conditional Processing}
\label{sec:conditional}

The package provides a mechanism to compile different versions
of a document. To customise the versions further some conditional processing
can come in handy to distinguish which version is being compiled.
The package provides two macros to describe the compilation context:

%%%%%%%%%%%%%%%%%%%%%%%%%%%%%%%%%%%%%%%%
\DescribeMacro{\ifchilddoc}
The conditional |\ifchilddoc| distinguishes between the compilation of
child documents and the main document:
%
\begin{center}
|\ifchilddoc |\textit{child-code}| |[|\||else |\textit{main-code}]| \||fi|
\end{center}

%%%%%%%%%%%%%%%%%%%%%%%%%%%%%%%%%%%%%%%%
\DescribeMacro{\childdocname}
\DescribeMacro{\childdocjob}
The macro |\childdocname| contains the filename (without extension)
of the main or child file being processed.
Note that |\childdocjob| will always contain the name of the main file.

%%%%%%%%%%%%%%%%%%%%%%%%%%%%%%%%%%%%%%%%
\paragraph{Title Page.}

Conditional processing can be used to include a title or banner page
in the main document when proper precautions are taken.
Importantly, the code in the main file should ensure that the page counter
(as well as other status parameters which are stored in the |.aux| files)
takes the same value after the conditional processing.
Otherwise the page numbers may take divergent values
depending on which part is compiled.

For example, a title page could be declared by:
%
\begin{center}
\begin{tabular}{l}
|\ifchilddoc\||else|\\
|\addtocounter{page}{-1}|\\
\textit{code for title page}\\
|\newpage|\\
|\||fi|
\end{tabular}
\end{center}
%
A banner page for the child documents can be generated by:
%
\begin{center}
\begin{tabular}{l}
|\ifchilddoc|\\
|\addtocounter{page}{-1}|\\
\textit{code for banner page}\\
|\newpage|\\
|\||fi|
\end{tabular}
\end{center}
%
Here one could write a message such as:
\begin{center}
|This is the part \childdocname{} of \childdocjob{}.|
\end{center}

%%%%%%%%%%%%%%%%%%%%%%%%%%%%%%%%%%%%%%%%%%%%%%%%%%%%%%%%%%%%%%%%%%%%%%%%%%%%%%%%
\subsection{Flags}
\label{sec:flags}

The package makes it easy to generate different versions
of the main or child documents.
To this end compilation flags can be defined
and assigned different default values.
They will be particularly useful in conjunction
with the forwarding mechanism described in \secref{sec:forward}.

For example, it may be useful to have a flag |\version|
which can be set to |draft| or |final|.
The document source will contain some conditional code
depending on the value of |\version|.
Suppose further, the flag should default to |final| for the main file
and to |draft| for child files
which is a natural assignment for editing the document.
This is achieved by placing the following code
in the preamble of the main document
(below the |\childdocmain| directive):
%
\begin{center}
\begin{tabular}{l}
|\ifchilddoc|\\
|\providecommand{\version}{draft}|\\
|\||else|\\
|\providecommand{\version}{final}|\\
|\||fi|
\end{tabular}
\end{center}
%
The definition by |\providecommand| makes sure
that previous definitions are not overwritten.
Further statements |\providecommand{\version}{...}|
can thus be added before the above code to override it.

For the main file, one might add a line
(between |\childdocmain| and the above block)
%
\begin{center}
|%\ifchilddoc\||else\providecommand{\version}{draft}\||fi|
\end{center}
%
which can be uncommented to produce a draft version.
Likewise one can add a line to the very top of a child file
(above the |\childdocof{|\textit{main}|}| directive)
%
\begin{center}
|%\providecommand{\version}{final}|
\end{center}
%
which can be uncommented to produce the final version of this child document.

%%%%%%%%%%%%%%%%%%%%%%%%%%%%%%%%%%%%%%%%%%%%%%%%%%%%%%%%%%%%%%%%%%%%%%%%%%%%%%%%
\subsection{Forwarding}
\label{sec:forward}

Different versions of the main or child documents
using compilation flags as described in \secref{sec:flags}
can be (permanently) stored in different files
for convenient compilation, viewing and distribution.
To this end, the package defines a command
to pass on compilation to a different file:

%%%%%%%%%%%%%%%%%%%%%%%%%%%%%%%%%%%%%%%%
\DescribeMacro{\childdocforward}
The command |\childdocforward| redirects processing to
another source file:
%
\begin{center}
\begin{tabular}{l}
|\input{childdoc.def}|\\
|\childdocforward[|\textit{main}|]{|\textit{dest}|}|\\
\end{tabular}
\end{center}
%
The argument \textit{dest} is the destination file
(without extension).
It should be the main file or one of the child files.
Note that further \textsf{childdoc} directives
such as |\childdocof| and |\childdocforward|
in the indicated file will be processed in this form.
The optional argument \textit{main}
passes on directly to the main file \textit{main}
while pretending to compile the child \textit{dest}.
This form behaves as if \textit{dest}
issues |\childdocof{|\textit{main}|}| right away,
and no further \textsf{childdoc} directives will be processed.

%%%%%%%%%%%%%%%%%%%%%%%%%%%%%%%%%%%%%%%%
\DescribeMacro{\...prefix}
In the alternative form |\childdocforwardprefix|,
%
\begin{center}
\begin{tabular}{l}
|\input{childdoc.def}|\\
|\childdocforwardprefix[|\textit{main}|]{|\textit{prefix}|}{|\textit{dest}|}|
\end{tabular}
\end{center}
%
the destination file is determined by a pattern
depending on the current file:
To make this work, the current file must be called
`{\textit{prefix}\hspace{0.2em}\textit{suffix}}'
with \textit{prefix} matching precisely the argument.
Processing is then passed on to the file
`{\textit{dest}\hspace{0.2em}\textit{suffix}}'.
Surely, the same effect is achieved by
directly specifying the
argument `{\textit{dest}\hspace{0.2em}\textit{suffix}}'
in the first form.
However, that requires to set up a different file
for each child. With the alternative form of the command
all these files can have exactly the same content
which simplifies setting them up and maintaining them.

For example, the following file |draft.tex|
with a compilation flag |\version| as described in \secref{sec:flags}
compiles the main document as a draft:
%
\begin{center}
\begin{tabular}{l}
|\def\version{draft}|\\
|\input{childdoc.def}|\\
|\childdocforward{|\textit{main}|}|
\end{tabular}
\end{center}
%
Likewise, the following files |final|\textit{nn}|.tex|
compile the final version of the child document
|child|\textit{nn}|.tex|:
%
\begin{center}
\begin{tabular}{l}
|\def\version{final}|\\
|\input{childdoc.def}|\\
|\childdocforwardprefix{final}{child}|
\end{tabular}
\end{center}
%

Note that when several versions of a main file and/or of each child file
are to be generated, it may be convenient to set up a |Makefile| or
shell script to automatise the process.

%%%%%%%%%%%%%%%%%%%%%%%%%%%%%%%%%%%%%%%%%%%%%%%%%%%%%%%%%%%%%%%%%%%%%%%%%%%%%%%%
\subsection{Command Line Processing}
\label{sec:commandline}

The effect of redirection files can also be achieved by invoking
the \LaTeX{} compiler with a more elaborate command line.
Most conveniently this should be done as part
of a shell script or a |Makefile|.

When using \textsf{childdoc} in the main file, the following
command lines effectively perform a redirection
(note that depending on the shell being used,
backslashes may have to be doubled: `|\|' $\to$ `|\\|'):
%
\begin{center}
|... -jobname "|\textit{target}|" |\\|"|[\textit{flags}]%
|\input{childdoc.def}\childdocforward[|\textit{main}|]{|\textit{dest}|}"|
\end{center}
%
Here \textit{target} is the name of the output file,
\textit{main} is the name of the main file
and \textit{dest} is the name of the main or child file to be processed
(all filenames without extensions).
The optional argument \textit{main} can be omitted
if \textit{main} matches \textit{dest}.
Optionally, compilation \textit{flags} can be defined via |\def| commands.
This command line makes the \TeX{} engine believe
it is compiling the file \textit{target}
whose content is specified as the latter parameter.
The provided code then forwards the processing to
\textit{main} or \textit{dest} as described in \secref{sec:forward}.

%%%%%%%%%%%%%%%%%%%%%%%%%%%%%%%%%%%%%%%%%%%%%%%%%%%%%%%%%%%%%%%%%%%%%%%%%%%%%%%%
\subsection{Include by Input}
\label{sec:input}

Including child documents by |\include| has some restrictions by design.
Most notably, the content of a child document always occupies
its own set of pages; pages cannot be shared between child documents.
Usually, this behaviour makes perfect sense
because each child document contain an essential part of the document.
However, in some situations it may be desirable to compose
a document from a collection of parts
without having mandatory page breaks between then.
For this case, the package
provides a mechanism to include parts
by |\input| which can also be processed individually.
However, by construction this mechanism
requires manual handling of the content to be output.

%%%%%%%%%%%%%%%%%%%%%%%%%%%%%%%%%%%%%%%%
\DescribeMacro{\ifchilddocmanual}
The main file should be prepared as usual, see \secref{sec:include}.
However, the document body must make a distinction
between processing of an individual part and of the main document, e.g.:
%
\begin{center}
\begin{tabular}{l}
|\ifchilddocmanual|\\
|\input{\childdocname}|\\
|\||else|\\
\textit{document body with }|\input{|\textit{part}|}|\\
|\||fi|
\end{tabular}
\end{center}
%
The conditional |\ifchilddocmanual| is true whenever
a part to be included by |\input| is being compiled,
and the name of the part is stored in |\childdocname|.

%%%%%%%%%%%%%%%%%%%%%%%%%%%%%%%%%%%%%%%%
\DescribeMacro{\childdocby}
Each part to be included by |\input| should start with:
%
\begin{center}
\begin{tabular}{l}
|\input{childdoc.def}|\\
|\childdocby{|\textit{main}|}|\\
\end{tabular}
\end{center}
%
The directive |\childdocby| is similar to |\childdocof|
described in \secref{sec:include},
but the subsequent selection of content must be done manually.
To that end, both |\ifchilddoc| and |\ifchilddocmanual|
will be true upon processing of a part,
and the name of the part is stored in |\childdocname|.
Note that |\jobname| will be set to the filename of the current part
so that each part receives an individual |.aux| file
that does not interfere with the |.aux| file(s) of the main document.
This behaviour can be altered by the alternative form
|\childdocby[*]{|\textit{main}|}| (with a non-empty optional argument)
which uses the |.aux| file of the main document
by setting |\jobname| to \textit{main}.

%%%%%%%%%%%%%%%%%%%%%%%%%%%%%%%%%%%%%%%%%%%%%%%%%%%%%%%%%%%%%%%%%%%%%%%%%%%%%%%%
\subsection{Driver Development}
\label{sec:driver}

The \textsf{childdoc} mechanism can also be use for the development
of definition files such as \LaTeX{} styles or classes.
This case differs from the above setup with multiple parts
included by |\include| in that no |\includeonly| should be invoked.
This can be achieved by starting the include file
(before |\ProvidesPackage|) with:
%
\begin{center}
\begin{tabular}{l}
|\input{childdoc.def}|\\
|\childdocforward{|\textit{main}|}|\\
\end{tabular}
\end{center}
%
or alternatively with:
%
\begin{center}
\begin{tabular}{l}
|\input{childdoc.def}|\\
|\childdocby{|\textit{main}|}|\\
\end{tabular}
\end{center}
%
Both forms have slightly different effects as described above.
The main file is prepared as usual, see \secref{sec:include}.

%%%%%%%%%%%%%%%%%%%%%%%%%%%%%%%%%%%%%%%%%%%%%%%%%%%%%%%%%%%%%%%%%%%%%%%%%%%%%%%%
\subsection{Legacy Detection}
\label{sec:detection}

The directive |\childdocmain| in the main file can detect
whether the complete document or merely a child is to be compiled
even without using the directive |\childdocof|.
This method is deprecated because it is less robust
and there is no compelling reason to use it;
it is merely provided for backward compatibility
and it may be removed in future versions.

If the detection mechanism is to be used,
it is mandatory to correctly specify
the filename of the main file as the argument of |\childdocmain|:
%
\begin{center}
\begin{tabular}{l}
|\input{childdoc.def}|\\
|\childdocmain{|\textit{main}|}|\\
\end{tabular}
\end{center}
%
If |\jobname| does not match the argument \textit{main} of |\childdocmain|,
it is assumed that |\jobname| points to the child file to be compiled.
When using |\childdocmain| with the main file specified as argument,
it suffices to start a child file
with just |\input{|\textit{main}|}|
without loading of the package and using |\childdocof|.
If instead all processing is done
with the appropriate \textsf{childdoc} directives,
the argument of \textit{main} of |\childdocmain| can be empty.

An alternative version of the command line processing described
in \secref{sec:commandline} using the detection mechanism reads:
%
\begin{center}
|... -jobname "|\textit{target}|" "|[\textit{flags}]%
[|\def\jobname{|\textit{dest}|}|]|\input{|\textit{main}|}"|
\end{center}

%%%%%%%%%%%%%%%%%%%%%%%%%%%%%%%%%%%%%%%%%%%%%%%%%%%%%%%%%%%%%%%%%%%%%%%%%%%%%%%%
\subsection{Manual Code}
\label{sec:manual}

In case one cannot be certain whether the definitions file |childdoc.def|
is installed on the target \TeX{} distribution
and one prefers not to ship it,
it is conceivable to paste a few relevant commands into the sources.

To that end, drop all statements |\input{childdoc.def}|
and perform the replacements as outlined below.
Instead of |\childdocmain{|\textit{main}|}| add the following code
to the top of the main file:
%
\begin{center}
\begin{tabular}{l}
|\||ifdefined\childdocname\endinput\||fi\newif\ifchilddoc|\\
|\edef\childdocname{\scantokens\expandafter{\jobname\noexpand}}|\\
|\def\childdocmain{|\textit{main}|}\||ifx\childdocmain\childdocname\||else|\\
|\childdoctrue\includeonly{\childdocname}\let\jobname\childdocmain\||fi|\\
\end{tabular}
\end{center}
%
Instead of |\childdocof{|\textit{main}|}| just include the main file
at the top of each child file:
%
\begin{center}
|\input{|\textit{main}|}|
\end{center}
%
A simple redirection |\childdocforward{|\textit{dest}|}| is achieved by:
%
\begin{center}
|\def\jobname{|\textit{dest}|}\input{\jobname}|
\end{center}
%
The redirection with prefix
|\childdocforwardprefix[|\textit{prefix}|]{|\textit{dest}|}|
is accomplished by:
%
\begin{center}
\begin{tabular}{l}
|{\edef\jobname{\scantokens\expandafter{\jobname\noexpand}}|\\
|\def\redirectjob |\textit{prefix}|#1~~~{\gdef\jobname{|\textit{dest}|#1}}|\\
|\expandafter\redirectjob\jobname~~~}\input{\jobname}|
\end{tabular}
\end{center}

In an alternative approach,
child documents can be compiled by a specific command line
without additional code or specific definitions:
%
\begin{center}
|... -jobname "|\textit{target}|" "|[\textit{flags}]%
|\includeonly{|\textit{dest}|}\input{|\textit{main}|}"|
\end{center}
%

%%%%%%%%%%%%%%%%%%%%%%%%%%%%%%%%%%%%%%%%%%%%%%%%%%%%%%%%%%%%%%%%%%%%%%%%%%%%%%%%
%%%%%%%%%%%%%%%%%%%%%%%%%%%%%%%%%%%%%%%%%%%%%%%%%%%%%%%%%%%%%%%%%%%%%%%%%%%%%%%%
\section{Information}

%%%%%%%%%%%%%%%%%%%%%%%%%%%%%%%%%%%%%%%%%%%%%%%%%%%%%%%%%%%%%%%%%%%%%%%%%%%%%%%%
\subsection{Copyright}

Copyright \copyright{} 2017--2018 Niklas Beisert

This work may be distributed and/or modified under the
conditions of the \LaTeX{} Project Public License, either version 1.3
of this license or (at your option) any later version.
The latest version of this license is in
  \url{http://www.latex-project.org/lppl.txt}
and version 1.3 or later is part of all distributions of \LaTeX{}
version 2005/12/01 or later.

This work has the LPPL maintenance status `maintained'.

The Current Maintainer of this work is Niklas Beisert.

This work consists of the files |README.txt|, |childdoc.ins| and |childdoc.dtx|
as well as the derived files |childdoc.def|, |cdocsamp.tex|
with |cdocsch1.tex|, |cdocsch2.tex|, |cdocspt3.tex|, |cdocspt4.tex|,
|cdocsdrf.tex|, |cdocsfn1.tex|, |cdocsfn2.tex|
as well as |childdoc.pdf|.

%%%%%%%%%%%%%%%%%%%%%%%%%%%%%%%%%%%%%%%%%%%%%%%%%%%%%%%%%%%%%%%%%%%%%%%%%%%%%%%%
\subsection{Files and Installation}

The package consists of the files:
%
\begin{center}
\begin{tabular}{ll}
    |README.txt|   & readme file \\
    |childdoc.ins| & installation file \\
    |childdoc.dtx| & source file \\
    |childdoc.def| & definition file \\
    |cdocsamp.tex| & sample main file \\
    |cdocsch1.tex| & sample include file \\
    |cdocsch2.tex| & sample include file \\
    |cdocspt3.tex| & sample part file \\
    |cdocspt4.tex| & sample part file \\
    |cdocsdrf.tex| & sample redirection file \\
    |cdocsfn1.tex| & sample redirection file \\
    |cdocsfn2.tex| & sample redirection file \\
    |childdoc.pdf| & manual
\end{tabular}
\end{center}
%
The distribution consists of the files
|README.txt|, |childdoc.ins| and |childdoc.dtx|.
%
\begin{itemize}
\item
Run (pdf)\LaTeX{} on |childdoc.dtx|
to compile the manual |childdoc.pdf| (this file).
\item
Run \LaTeX{} on |childdoc.ins| to create the definitions file |childdoc.def|
and the sample |cdocsamp.tex| with include files
|cdocsch1.tex|, |cdocsch2.tex|, |cdocspt3.tex|, |cdocspt4.tex|,
|cdocsdrf.tex|, |cdocsfn1.tex|, |cdocsfn2.tex|.
Then copy the file |childdoc.def| to an appropriate directory of your \LaTeX{}
distribution, e.g.\ \textit{texmf-root}|/tex/latex/childdoc|.
\end{itemize}

%%%%%%%%%%%%%%%%%%%%%%%%%%%%%%%%%%%%%%%%%%%%%%%%%%%%%%%%%%%%%%%%%%%%%%%%%%%%%%%%
\subsection{Related CTAN Packages}

There are several other packages which offer a similar functionality:
%
\begin{itemize}
\item
The packages
\href{http://ctan.org/pkg/docmute}{\textsf{docmute}},
\href{http://ctan.org/pkg/includex}{\textsf{includex}} and
\href{http://ctan.org/pkg/standalone}{\textsf{standalone}}
provide commands to include only the document body of
a child file thus allowing both files to be compiled individually.
\item
The packages \href{http://ctan.org/pkg/subdocs}{\textsf{subdocs}}
and \href{http://ctan.org/pkg/subfiles}{\textsf{subfiles}}
provide structures in which the main and child documents can be
encapsulated and allowing them to be compiled individually.
The inclusion mechanism is different from the conventional |\include|.
\item
The package \href{http://ctan.org/pkg/combine}{\textsf{combine}}
is an elaborate solution to combine several documents into one.
\end{itemize}
%
See also the CTAN topic \href{http://ctan.org/topic/subdocs}{\textsf{subdocs}}
for further related packages.
The present package differs from the above solutions in that
a document structure constructed with the conventional |\include| mechanism
just needs two extra commands at the top of every file
such that all constituent files can be compiled individually.

%%%%%%%%%%%%%%%%%%%%%%%%%%%%%%%%%%%%%%%%%%%%%%%%%%%%%%%%%%%%%%%%%%%%%%%%%%%%%%%%
%\subsection{Feature Suggestions}
%
%The following is a list of features which may be useful for future
%versions of this package:
%%
%\begin{itemize}
%\item
%\ldots
%\end{itemize}

%%%%%%%%%%%%%%%%%%%%%%%%%%%%%%%%%%%%%%%%%%%%%%%%%%%%%%%%%%%%%%%%%%%%%%%%%%%%%%%%
\subsection{Revision History}

%%%%%%%%%%%%%%%%%%%%%%%%%%%%%%%%%%%%%%%%
\paragraph{v2.0:} 2018/12/30

\begin{itemize}
\item
immediate forward processing
\item
added |\childdocby| mechanism
\item
manual restructured
\end{itemize}

%%%%%%%%%%%%%%%%%%%%%%%%%%%%%%%%%%%%%%%%
\paragraph{v1.6:} 2018/01/17

\begin{itemize}
\item
application for development of include files
\item
corrections to manual
\end{itemize}

%%%%%%%%%%%%%%%%%%%%%%%%%%%%%%%%%%%%%%%%
\paragraph{v1.5:} 2017/05/21

\begin{itemize}
\item
more complete structuring introduced
\item
|\childdocof| introduced
\item
|\childdoc| renamed to |\childdocmain|
\item
|\childredirect| renamed to |\childdocforward| and |\childdocforwardprefix|
and functionality expanded
\end{itemize}

%%%%%%%%%%%%%%%%%%%%%%%%%%%%%%%%%%%%%%%%
\paragraph{v1.0:} 2017/04/27

\begin{itemize}
\item
manual and install package
\item
first version published on CTAN
\end{itemize}

%%%%%%%%%%%%%%%%%%%%%%%%%%%%%%%%%%%%%%%%
\paragraph{v0.6:} 2017/04/26

\begin{itemize}
\item
redirection mechanism added
\end{itemize}

%%%%%%%%%%%%%%%%%%%%%%%%%%%%%%%%%%%%%%%%
\paragraph{v0.5:} 2017/04/26

\begin{itemize}
\item
functionality in definition file
\end{itemize}


%%%%%%%%%%%%%%%%%%%%%%%%%%%%%%%%%%%%%%%%%%%%%%%%%%%%%%%%%%%%%%%%%%%%%%%%%%%%%%%%
%%%%%%%%%%%%%%%%%%%%%%%%%%%%%%%%%%%%%%%%%%%%%%%%%%%%%%%%%%%%%%%%%%%%%%%%%%%%%%%%
%%%%%%%%%%%%%%%%%%%%%%%%%%%%%%%%%%%%%%%%%%%%%%%%%%%%%%%%%%%%%%%%%%%%%%%%%%%%%%%%
\appendix

\settowidth\MacroIndent{\rmfamily\scriptsize 000\ }

 \DocInput{childdoc.dtx}

\end{document}
%</driver>
% \fi
%
% %%%%%%%%%%%%%%%%%%%%%%%%%%%%%%%%%%%%%%%%%%%%%%%%%%%%%%%%%%%%%%%%%%%%%%%%%%%%%%
% %%%%%%%%%%%%%%%%%%%%%%%%%%%%%%%%%%%%%%%%%%%%%%%%%%%%%%%%%%%%%%%%%%%%%%%%%%%%%%
% \section{Sample}
%\iffalse
%<*samplemain>
%\fi
%
% The following presents a sample document
% with two chapters, two parts, a title page,
% a compile flag as well as three forwarding files to set the flag.
% It consists of eight |.tex| files:
% \begin{center}
% \begin{tabular}{ll}
% |cdocsamp.tex|&main file\\
% |cdocsch1.tex|&include file for chapter 1\\
% |cdocsch2.tex|&include file for chapter 2\\
% |cdocspt3.tex|&include file for part 3\\
% |cdocspt4.tex|&include file for part 4\\
% |cdocsdrf.tex|&forwarding file for main file in draft mode\\
% |cdocsfi1.tex|&forwarding file for final version of chapter 1\\
% |cdocsfi2.tex|&forwarding file for final version of chapter 2\\
% \end{tabular}
% \end{center}
% Each of the eight files can be compiled directly by the \LaTeX{} compiler.
%
% %%%%%%%%%%%%%%%%%%%%%%%%%%%%%%%%%%%%%%
% \paragraph{Main File.}
%
% The main file is called |cdocsamp.tex|.
%
% Load the \textsf{childdoc} definitions and
% declare the filename for the main document:
%    \begin{macrocode}
\input{childdoc.def}
\childdocmain{}
%    \end{macrocode}

% Optional override for |\version| flag:
%    \begin{macrocode}
%%\ifchilddoc\else\providecommand{\version}{draft}\fi
%    \end{macrocode}

% Define the default values for the |\version| flag
% (|final| for the main file and |draft| for childs):
%    \begin{macrocode}
\ifchilddoc
\providecommand{\version}{draft}
\else
\providecommand{\version}{final}
\fi
%    \end{macrocode}

% Load the standard document class:
%    \begin{macrocode}
\documentclass[12pt]{article}
%    \end{macrocode}

% Start the document body:
%    \begin{macrocode}
\begin{document}
%    \end{macrocode}

% Declare a title page.
% Print title, part of document being processed and version flag:
%    \begin{macrocode}
\addtocounter{page}{-1}
\begin{center}
{\LARGE\bfseries{}childdoc example\par}
\vspace{1cm}
\ifchilddoc
\ifchilddocmanual part\else chapter\fi:
`\childdocname' of `\childdocjob'\par
\else
main document: `\childdocjob'\par
\fi
version: \version\par
\end{center}
\newpage
%    \end{macrocode}

% Manually include selected file,
% otherwise process as usual:
%    \begin{macrocode}
\ifchilddocmanual
\section*{part `\childdocname'}
\input{\childdocname}
\else
%    \end{macrocode}

% Include the two chapters:
%    \begin{macrocode}
\include{cdocsch1}
\include{cdocsch2}
%    \end{macrocode}

% Include the two parts unless only chapters should be displayed:
%    \begin{macrocode}
\ifchilddoc\else
\section{part three}
\input{cdocspt3}
\section{part four}
\input{cdocspt4}
\fi
%    \end{macrocode}

% Process as usual until here:
%    \begin{macrocode}
\fi
%    \end{macrocode}

% End of document body:
%    \begin{macrocode}
\end{document}
%    \end{macrocode}
%\iffalse
%</samplemain>
%\fi
%
% %%%%%%%%%%%%%%%%%%%%%%%%%%%%%%%%%%%%%%
% \paragraph{Chapter Include Files.}
%
% The include files are called |cdocsch1.tex| and |cdocsch2.tex|.
%
%\iffalse
%<*samplechap1|samplechap2>
%\fi

% Optional override for |\version| flag:
%    \begin{macrocode}
%%\providecommand{\version}{final}
%    \end{macrocode}

% Include the main document:
%    \begin{macrocode}
\input{childdoc.def}
\childdocof{cdocsamp}
%    \end{macrocode}

%\iffalse
%</samplechap1|samplechap2>
%\fi
%
%\iffalse
%<*samplechap1>
%\fi
% Some text for chapter 1:
%    \begin{macrocode}
\section{one}
some text in chapter one
%    \end{macrocode}

%\iffalse
%</samplechap1>
%\fi
% Some text for chapter 2:
%\iffalse
%<*samplechap2>
%\fi
%    \begin{macrocode}
\section{two}
more text in chapter two
%    \end{macrocode}

%\iffalse
%</samplechap2>
%\fi
%
% %%%%%%%%%%%%%%%%%%%%%%%%%%%%%%%%%%%%%%
% \paragraph{Part Include Files.}
%
% The include files are called |cdocspt3.tex| and |cdocspt4.tex|.
%
%\iffalse
%<*samplepart3|samplepart4>
%\fi

% Optional override for |\version| flag:
%    \begin{macrocode}
%%\providecommand{\version}{final}
%    \end{macrocode}

% Include the main document:
%    \begin{macrocode}
\input{childdoc.def}
\childdocby{cdocsamp}
%    \end{macrocode}

%\iffalse
%</samplepart3|samplepart4>
%\fi
%
%\iffalse
%<*samplepart3>
%\fi
% Some text for part 3:
%    \begin{macrocode}
some text in part three
%    \end{macrocode}

%\iffalse
%</samplepart3>
%\fi
% Some text for part 4:
%\iffalse
%<*samplepart4>
%\fi
%    \begin{macrocode}
more text in part four
%    \end{macrocode}

%\iffalse
%</samplepart4>
%\fi
%
% %%%%%%%%%%%%%%%%%%%%%%%%%%%%%%%%%%%%%%
% \paragraph{Forwarding for a Complete Draft.}
%
% The following forwarding file |cdocsdrf.tex|
% compiles the main document in draft mode:
%\iffalse
%<*sampledraft>
%\fi
%    \begin{macrocode}
\def\version{draft}
\input{childdoc.def}
\childdocforward{cdocsamp}
%    \end{macrocode}

%\iffalse
%</sampledraft>
%\fi
%
% %%%%%%%%%%%%%%%%%%%%%%%%%%%%%%%%%%%%%%
% \paragraph{Forwarding for Final Version of the Chapters.}
%
% The following forwarding files |cdocsfn1.tex| and |cdocsfn2.tex|
% (with identical content)
% compile the final versions of the child documents
% |cdocsch1.tex| and |cdocsch2.tex|, respectively:
%\iffalse
%<*samplefinal>
%\fi
%    \begin{macrocode}
\def\version{final}
\input{childdoc.def}
\childdocforwardprefix[cdocsamp]{cdocsfn}{cdocsch}
%    \end{macrocode}

%\iffalse
%</samplefinal>
%\fi
%
% %%%%%%%%%%%%%%%%%%%%%%%%%%%%%%%%%%%%%%
% \paragraph{Command Line Processing.}
%
% The following three command lines generate the output files
% |cdocscld|, |cdocscl1| and |cdocscl2|
% which should be identical to
% |cdocsdrf|, |cdocsch1| and |cdocsfn2|, respectively:
% \begin{center}
% \begin{tabular}{l}
% |latex -jobname cdocscld \|\\
% |  "\def\version{draft}\input{childdoc.def}\childdocforward{cdocsamp}"|\\
% |latex -jobname cdocscl1 \|\\
% |  "\input{childdoc.def}\childdocforward[cdocsamp]{cdocsch1}"|\\
% |latex -jobname cdocscl2 \|\\
% |  "\def\version{final}\input{childdoc.def}\childdocforward{cdocsch2}"|
% \end{tabular}
% \end{center}
% Note that the trailing backslash on each first line
% merely continues the input to the second line
% (for convenient cut ant paste).
% Furthermore, the command |latex| can be replaced by any
% of its alternative versions such as |pdflatex|.
%
% %%%%%%%%%%%%%%%%%%%%%%%%%%%%%%%%%%%%%%%%%%%%%%%%%%%%%%%%%%%%%%%%%%%%%%%%%%%%%%
% %%%%%%%%%%%%%%%%%%%%%%%%%%%%%%%%%%%%%%%%%%%%%%%%%%%%%%%%%%%%%%%%%%%%%%%%%%%%%%
% \section{Implementation}
%\iffalse
%<*package>
%\fi
%
% This section describes the definitions file |childdoc.def|.

% The definitions cannot be loaded using |\usepackage| or |\RequirePackage|
% which has a mechanism to prevent loading a style file more than once.
% When loading the definitions by means of |\input|
% multiple instances have to be prevented manually:
%\iffalse
%This code needs to be before the `\ProvidesFile' directive
%which is defined at the beginning of this file.
%Therefore it is also placed there and commented out here.
%</package>
%<*discard>
%\fi
%    \begin{macrocode}
\ifdefined\childdocmain\endinput\fi
%    \end{macrocode}
%\iffalse
%</discard>
%<*package>
%\fi
%
% \macro{\ifchilddoc}
% \macro{\ifchilddocmanual}
% The conditional |\ifchilddoc| tells whether a
% child (true) or main (false) document is being compiled.
% The conditional |\ifchilddocmanual| tells whether
% the |\includeonly| mechanism is used (false) or
% the selection of child files must be performed manually (true).
% The definitions initialise to false:
%    \begin{macrocode}
\newif\ifchilddoc
\newif\ifchilddocmanual
%    \end{macrocode}

% \macro{\childdocname}
% \macro{\childdocjob}
% The macro |\childdocname| stores the name of the main document
% to be compiled. The macro |\childdocjob| stores the name of
% the document on which the \LaTeX{} compiler was originally invoked.
% The content of |\jobname| cannot be compared
% to filenames specified in the source due to different catcodes.
% The following code rescans |\jobname|, stores the result
% in |\childdocname| and saves a copy in |\childdocjob|:
%    \begin{macrocode}
\edef\childdocname{\scantokens\expandafter{\jobname\noexpand}}
\let\childdocjob\childdocname
%    \end{macrocode}

% \macro{\childdocdisable}
% The macro |\childdocdisable| prevents the main file
% from being processed more than once.
% At this stage, the main document command |\childdocmain|
% is assumed to be called once again where it should do nothing.
% Any subsequent call to it should prevent
% a secondary processing of the main document
% It overwrites the forwarding commands
% |\childdocof| and |\childdocforward|
% with empty macros to prevent further inclusions of the main document:
%    \begin{macrocode}
\newcommand{\childdocdisable}
{
  \renewcommand{\childdocmain}[1]{\renewcommand{\childdocmain}[1]{\endinput}}
  \renewcommand{\childdocof}[1]{}
  \renewcommand{\childdocby}[2][]{}
  \renewcommand{\childdocforward}[2][]{}
  \renewcommand{\childdocdisable}{}
}
%    \end{macrocode}

% \macro{\childdocmain}
% The macro |\childdocmain| is to be called at the top of the main file
% with nothing or the main filename (without extension) as argument.
% First, it breaks loops.
% If the argument is not empty and does not match |\childdocname|
% (which is set by the first inclusion of |childdoc.def|),
% |\ifchilddoc| is set to true, |\includeonly| is applied to the child file
% and |\jobname| is set to the main file
% (for proper handling of |.aux| files):
%    \begin{macrocode}
\newcommand{\childdocmain}[1]
{
  \childdocdisable\childdocmain{}
  \if?#1?\else
    \begingroup
      \def\childdoctmp{#1}
      \ifx\childdoctmp\childdocname
        \def\childdoctmp{}
      \else
        \def\childdoctmp
        {
          \childdoctrue
          \includeonly{\childdocname}
          \def\childdocjob{#1}
          \def\jobname{#1}
        }
      \fi
      \expandafter
    \endgroup
    \childdoctmp
  \fi
}
%    \end{macrocode}

% \macro{\childdocof}
% The command |\childdocof| redirects
% compilation to the main file |#1|.
%    \begin{macrocode}
\newcommand{\childdocof}[1]
{
  \childdocdisable
  \childdoctrue
  \includeonly{\childdocname}
  \def\jobname{#1}
  \def\childdocjob{#1}
  \input{#1}
}
%    \end{macrocode}

% \macro{\childdocby}
% The command |\childdocby| ....
%    \begin{macrocode}
\newcommand{\childdocby}[2][]
{
  \childdocdisable
  \childdoctrue
  \childdocmanualtrue
  \if?#1?\else
    \def\jobname{#2}
  \fi
  \def\childdocjob{#2}
  \input{#2}
  \endinput
}
%    \end{macrocode}

% \macro{\childdocforward}
% The command |\childdocforward| redirects
% compilation to the main file or
% (if the optional argument is given) a child file.
% Parameters are set as if the main file
% or a child file starting with |\childdocof| was compiled.
% Then compilation is handed over to the main file:
%    \begin{macrocode}
\newcommand{\childdocforward}[2][]
{
  \begingroup
    \if?#1?
      \def\childdoctmp
      {
        \def\childdocname{#2}
        \def\childdocjob{#2}
        \def\jobname{#2}
        \input{#2}
        \endinput
      }
    \else
      \def\childdoctmp
      {
        \childdocdisable
        \def\childdocname{#2}
        \childdoctrue
        \includeonly{#2}
        \def\childdocjob{#1}
        \def\jobname{#1}
        \input{#1}
        \endinput
      }
    \fi
    \expandafter
  \endgroup
  \childdoctmp
}
%    \end{macrocode}

% \macro{\childdocforwardprefix}
% The command |\childdocforwardprefix| redirects
% compilation to the main or a child file by means of a pattern.
% The prefix |#1| in the current filename is replaced by |#2|
% and the suffix of the current filename is kept
% (it is assumed that the filename does not contain the substring `|~~~|'
% which is used as a delimiter).
% Compilation is handed over to the new file by |\childdocforward|:
%    \begin{macrocode}
\newcommand{\childdocforwardprefix}[3][]
{
  \begingroup
    \def\childdocextract #2##1~~~{\def\childdoctmp{\childdocforward[#1]{#3##1}}}
    \expandafter\childdocextract\childdocname~~~
    \expandafter
  \endgroup
  \childdoctmp
}
%    \end{macrocode}

% \macro{\childdoc}
% The deprecated macro |\childdoc| is a legacy version of |\childdocmain|:
%    \begin{macrocode}
\newcommand{\childdoc}{\childdocmain}
%    \end{macrocode}

% \macro{\childdocredirect}
% The deprecated macro |\childdocredirect| is a legacy version
% of |\childdocforward| and |\childdocforwardprefix|:
%    \begin{macrocode}
\newcommand{\childdocredirect}[2][]
{
  \begingroup
    \if?#1?
      \def\childdoctmp{\childdocforward{#2}}
    \else
      \def\childdoctmp{\childdocforwardprefix{#1}{#2}}
    \fi
    \expandafter
  \endgroup
  \childdoctmp
}
%    \end{macrocode}

%\iffalse
%</package>
%\fi
%
\endinput
\childdocforward{cdocsamp}"|\\
% |latex -jobname cdocscl1 \|\\
% |  "% \iffalse
%
% childdoc.dtx Copyright (C) 2017-2018 Niklas Beisert
%
% This work may be distributed and/or modified under the
% conditions of the LaTeX Project Public License, either version 1.3
% of this license or (at your option) any later version.
% The latest version of this license is in
%   http://www.latex-project.org/lppl.txt
% and version 1.3 or later is part of all distributions of LaTeX
% version 2005/12/01 or later.
%
% This work has the LPPL maintenance status `maintained'.
%
% The Current Maintainer of this work is Niklas Beisert.
%
% This work consists of the files childdoc.dtx and childdoc.ins
% and the derived files childdoc.def and cdocsamp.tex with
% cdocsch1.tex, cdocsch2.tex, cdocsdrf.tex, cdocsfn1.tex, cdocsfn2.tex.
%
%<package>\ifdefined\childdocmain\endinput\fi
%<package>\ProvidesFile{childdoc.def}[2018/12/30 v2.0 child document driver]
%<samplemain>\ProvidesFile{cdocsamp.tex}[2018/12/30 v2.0 sample for childdoc]
%<*driver>
%\ProvidesFile{childdoc.drv}[2018/12/30 v2.0 childdoc reference manual file]
\PassOptionsToClass{10pt,a4paper}{article}
\documentclass{ltxdoc}

\usepackage[margin=35mm]{geometry}
\usepackage{hyperref}
\usepackage{hyperxmp}
\usepackage[usenames]{color}

\hypersetup{colorlinks=true}
\hypersetup{pdfstartview=FitH}
\hypersetup{pdfpagemode=UseNone}
\hypersetup{pdfsource={}}
\hypersetup{pdflang={en-UK}}
\hypersetup{pdfcopyright={Copyright 2017-2018 Niklas Beisert.
  This work may be distributed and/or modified under the
  conditions of the LaTeX Project Public License, either version 1.3
  of this license or (at your option) any later version.}}
\hypersetup{pdflicenseurl={http://www.latex-project.org/lppl.txt}}
\hypersetup{pdfcontactaddress={ETH Zurich, ITP, HIT K,
  Wolfgang-Pauli-Strasse 27}}
\hypersetup{pdfcontactpostcode={8093}}
\hypersetup{pdfcontactcity={Zurich}}
\hypersetup{pdfcontactcountry={Switzerland}}
\hypersetup{pdfcontactemail={nbeisert@itp.phys.ethz.ch}}
\hypersetup{pdfcontacturl={http://people.phys.ethz.ch/\xmptilde nbeisert/}}

\newcommand{\secref}[1]{\hyperref[#1]{section \ref*{#1}}}

\parskip1ex
\parindent0pt
\let\olditemize\itemize
\def\itemize{\olditemize\parskip0pt}

\begin{document}

\title{The \textsf{childdoc} Package}
\hypersetup{pdftitle={The childdoc Package}}
\author{Niklas Beisert\\[2ex]
  Institut f\"ur Theoretische Physik\\
  Eidgen\"ossische Technische Hochschule Z\"urich\\
  Wolfgang-Pauli-Strasse 27, 8093 Z\"urich, Switzerland\\[1ex]
  \href{mailto:nbeisert@itp.phys.ethz.ch}
  {\texttt{nbeisert@itp.phys.ethz.ch}}}
\hypersetup{pdfauthor={Niklas Beisert}}
\hypersetup{pdfsubject={Manual for the LaTeX2e Package childdoc}}
\date{30 December 2018, \textsf{v2.0}}
\maketitle

\begin{abstract}\noindent
\textsf{childdoc} is a \LaTeXe{} package
that enables the direct compilation
of document sections included by |\include|
to individual files.
\end{abstract}

\begingroup
\parskip0ex
\tableofcontents
\endgroup

%%%%%%%%%%%%%%%%%%%%%%%%%%%%%%%%%%%%%%%%%%%%%%%%%%%%%%%%%%%%%%%%%%%%%%%%%%%%%%%%
%%%%%%%%%%%%%%%%%%%%%%%%%%%%%%%%%%%%%%%%%%%%%%%%%%%%%%%%%%%%%%%%%%%%%%%%%%%%%%%%
\section{Introduction}

\LaTeX{} provides a mechanism to structure a large document (such as a book)
into a main file and several child files (containing the chapters)
using the |\include| command.
This mechanism is beneficial for documents
which span hundreds of pages in order to
make the source file(s) more manageable.
Moreover, compilation can be restricted to
selected child files by means of the |\includeonly| command.
The latter feature can be used to reduce the compilation time while editing
(this was significantly more useful in the earlier days of \LaTeX{})
or to generate a smaller document which is easier to navigate.
Another application of |\includeonly| is to generate
documents consisting of selected parts of the complete document.

However, there are a few drawbacks of the plain |\include| mechanism:
\begin{itemize}
\item
The child files cannot be compiled on their own,
they can only be compiled via the main file.
A naive editing environment
(such as a text editor with an option
to have the current file processed by \LaTeX)
may require one to switch to the main file before compiling;
attempting to compile the child file produces errors.
\item
The main file must be modified (each time)
to adjust the |\includeonly| command
to the present needs. This easily leaves the main file in a messy state.
\item
The generated document will always carry the filename
of the main document. This is inconvenient if
several child files are to be compiled and
to be kept for distribution.
\end{itemize}

The present package provides a simple interface
to make child files individually compilable by \LaTeX{}.
Compiling a child file then has the same effect as compiling
the main file with an |\includeonly| command
to select the appropriate child.
Moreover the generated document will carry the name of the child
rather than the main file.
This resolves all three above issues.

This feature is meant to make the editing of books,
thesis documents and lecture notes somewhat more convenient.
However, the package can also be used efficiently for
composing a series of documents (such as exercise sheets)
which are typically distributed individually.
It then assists the author in generating the individual documents
(potentially in different versions)
as well as a document containing the collected series.
Another application is in developing style files
or other kinds of included material
where compilation of the style file could redirect
to a sample or test file.

%%%%%%%%%%%%%%%%%%%%%%%%%%%%%%%%%%%%%%%%%%%%%%%%%%%%%%%%%%%%%%%%%%%%%%%%%%%%%%%%
%%%%%%%%%%%%%%%%%%%%%%%%%%%%%%%%%%%%%%%%%%%%%%%%%%%%%%%%%%%%%%%%%%%%%%%%%%%%%%%%
\section{Usage}

First of all, the package \textsf{childdoc} is \emph{not} a standard
\LaTeXe{} |.sty| style file! Therefore it needs to be invoked in
a non-standard way.

%%%%%%%%%%%%%%%%%%%%%%%%%%%%%%%%%%%%%%%%%%%%%%%%%%%%%%%%%%%%%%%%%%%%%%%%%%%%%%%%
\subsection{Included Files}
\label{sec:include}

%%%%%%%%%%%%%%%%%%%%%%%%%%%%%%%%%%%%%%%%
\DescribeMacro{\childdocmain}
To use the package, add the commands
\begin{center}
\begin{tabular}{l}
|\input{childdoc.def}|\\
|\childdocmain{}|\\
\end{tabular}
\end{center}
at the very top of the main \LaTeX{} file,
in particular \emph{before} the |\documentclass| statement!
The argument of |\childdocmain| should be left empty
(but it must be present).

%%%%%%%%%%%%%%%%%%%%%%%%%%%%%%%%%%%%%%%%
\DescribeMacro{\childdocof}
Furthermore, add the commands
\begin{center}
\begin{tabular}{l}
|\input{childdoc.def}|\\
|\childdocof{|\textit{main}|}|\\
\end{tabular}
\end{center}
at the top of every child file \textit{child}
which is included by |\include{|\textit{child}|}|
from within the main file
(or at least for those files to be compiled individually).
The argument \textit{main} must be the filename of the main file.

There are a couple of
considerations in setting up the main and child documents:

%%%%%%%%%%%%%%%%%%%%%%%%%%%%%%%%%%%%%%%%
\paragraph{Restrictions.}

Please note the following restrictions:
\begin{itemize}
\item
|\childdocmain| must be called with one argument \textit{main}
to ensure compatibility with earlier version of the package.
It must either be empty (|\childdocmain{}|)
or precisely match the filename of the main file in which it is specified.
See \secref{sec:detection} for further information.
\item
The filename \textit{main} must be specified without the |.tex| extension.
\item
The filename \textit{main} is case sensitive
(even in case-insensitive file systems)
due to internal string comparison.
\item
The argument \textit{main} should be fully expanded, it cannot be a macro.
\item
Subdirectories and special characters should be avoided in filenames.
\item
The command |\childdocmain{|\textit{main}|}| must be followed by a whitespace.
It should not be followed immediately by another command
or by a comment mark `|%|'.
This is because the \TeX{} parser reads the token immediately following
the argument of |\childdocmain| and puts it
at the beginning of every child section;
however, a white\-space is ignored.
\end{itemize}

%%%%%%%%%%%%%%%%%%%%%%%%%%%%%%%%%%%%%%%%
\paragraph{Content of Main File.}

It is advisable to place all content in the child files included by |\include|.
Any output contained in the main file will appear in all child documents
unless suppressed manually;
it cannot be suppressed automatically by the |\includeonly| directive
and thus should normally be avoided.
A method to include some content in the main file
by means of conditional processing is described in \secref{sec:conditional}.

%%%%%%%%%%%%%%%%%%%%%%%%%%%%%%%%%%%%%%%%
\paragraph{Page Numbering.}

When only a part of the document is compiled,
the appropriate numbering of pages
(as well as other status parameters)
is determined from the |.aux| files.
The latter contain information from previous passes.
However this information needs to propagate through
all intermediate child documents.
Therefore the page numbering in child documents may well
be inconsistent until the complete document is compiled at least once.

A useful (if unconventional) way to always ensure a consistent
page numbering is to restart the numbering in each child document
and denote the pages by `\textit{child}|.|\textit{page}'
where \textit{child} represents the chapter/section number of the child file.
This can be achieved by the command
|\numberwithin{page}{|\textit{child}|}|
of the \textsf{amsmath} package
where \textit{child} can be |chapter| or |section|
depending on the chosen structuring.
Alternatively, one can modify the macro |\thepage| appropriately
and reset the counter |page| at the start of each child file.

%%%%%%%%%%%%%%%%%%%%%%%%%%%%%%%%%%%%%%%%%%%%%%%%%%%%%%%%%%%%%%%%%%%%%%%%%%%%%%%%
\subsection{Conditional Processing}
\label{sec:conditional}

The package provides a mechanism to compile different versions
of a document. To customise the versions further some conditional processing
can come in handy to distinguish which version is being compiled.
The package provides two macros to describe the compilation context:

%%%%%%%%%%%%%%%%%%%%%%%%%%%%%%%%%%%%%%%%
\DescribeMacro{\ifchilddoc}
The conditional |\ifchilddoc| distinguishes between the compilation of
child documents and the main document:
%
\begin{center}
|\ifchilddoc |\textit{child-code}| |[|\||else |\textit{main-code}]| \||fi|
\end{center}

%%%%%%%%%%%%%%%%%%%%%%%%%%%%%%%%%%%%%%%%
\DescribeMacro{\childdocname}
\DescribeMacro{\childdocjob}
The macro |\childdocname| contains the filename (without extension)
of the main or child file being processed.
Note that |\childdocjob| will always contain the name of the main file.

%%%%%%%%%%%%%%%%%%%%%%%%%%%%%%%%%%%%%%%%
\paragraph{Title Page.}

Conditional processing can be used to include a title or banner page
in the main document when proper precautions are taken.
Importantly, the code in the main file should ensure that the page counter
(as well as other status parameters which are stored in the |.aux| files)
takes the same value after the conditional processing.
Otherwise the page numbers may take divergent values
depending on which part is compiled.

For example, a title page could be declared by:
%
\begin{center}
\begin{tabular}{l}
|\ifchilddoc\||else|\\
|\addtocounter{page}{-1}|\\
\textit{code for title page}\\
|\newpage|\\
|\||fi|
\end{tabular}
\end{center}
%
A banner page for the child documents can be generated by:
%
\begin{center}
\begin{tabular}{l}
|\ifchilddoc|\\
|\addtocounter{page}{-1}|\\
\textit{code for banner page}\\
|\newpage|\\
|\||fi|
\end{tabular}
\end{center}
%
Here one could write a message such as:
\begin{center}
|This is the part \childdocname{} of \childdocjob{}.|
\end{center}

%%%%%%%%%%%%%%%%%%%%%%%%%%%%%%%%%%%%%%%%%%%%%%%%%%%%%%%%%%%%%%%%%%%%%%%%%%%%%%%%
\subsection{Flags}
\label{sec:flags}

The package makes it easy to generate different versions
of the main or child documents.
To this end compilation flags can be defined
and assigned different default values.
They will be particularly useful in conjunction
with the forwarding mechanism described in \secref{sec:forward}.

For example, it may be useful to have a flag |\version|
which can be set to |draft| or |final|.
The document source will contain some conditional code
depending on the value of |\version|.
Suppose further, the flag should default to |final| for the main file
and to |draft| for child files
which is a natural assignment for editing the document.
This is achieved by placing the following code
in the preamble of the main document
(below the |\childdocmain| directive):
%
\begin{center}
\begin{tabular}{l}
|\ifchilddoc|\\
|\providecommand{\version}{draft}|\\
|\||else|\\
|\providecommand{\version}{final}|\\
|\||fi|
\end{tabular}
\end{center}
%
The definition by |\providecommand| makes sure
that previous definitions are not overwritten.
Further statements |\providecommand{\version}{...}|
can thus be added before the above code to override it.

For the main file, one might add a line
(between |\childdocmain| and the above block)
%
\begin{center}
|%\ifchilddoc\||else\providecommand{\version}{draft}\||fi|
\end{center}
%
which can be uncommented to produce a draft version.
Likewise one can add a line to the very top of a child file
(above the |\childdocof{|\textit{main}|}| directive)
%
\begin{center}
|%\providecommand{\version}{final}|
\end{center}
%
which can be uncommented to produce the final version of this child document.

%%%%%%%%%%%%%%%%%%%%%%%%%%%%%%%%%%%%%%%%%%%%%%%%%%%%%%%%%%%%%%%%%%%%%%%%%%%%%%%%
\subsection{Forwarding}
\label{sec:forward}

Different versions of the main or child documents
using compilation flags as described in \secref{sec:flags}
can be (permanently) stored in different files
for convenient compilation, viewing and distribution.
To this end, the package defines a command
to pass on compilation to a different file:

%%%%%%%%%%%%%%%%%%%%%%%%%%%%%%%%%%%%%%%%
\DescribeMacro{\childdocforward}
The command |\childdocforward| redirects processing to
another source file:
%
\begin{center}
\begin{tabular}{l}
|\input{childdoc.def}|\\
|\childdocforward[|\textit{main}|]{|\textit{dest}|}|\\
\end{tabular}
\end{center}
%
The argument \textit{dest} is the destination file
(without extension).
It should be the main file or one of the child files.
Note that further \textsf{childdoc} directives
such as |\childdocof| and |\childdocforward|
in the indicated file will be processed in this form.
The optional argument \textit{main}
passes on directly to the main file \textit{main}
while pretending to compile the child \textit{dest}.
This form behaves as if \textit{dest}
issues |\childdocof{|\textit{main}|}| right away,
and no further \textsf{childdoc} directives will be processed.

%%%%%%%%%%%%%%%%%%%%%%%%%%%%%%%%%%%%%%%%
\DescribeMacro{\...prefix}
In the alternative form |\childdocforwardprefix|,
%
\begin{center}
\begin{tabular}{l}
|\input{childdoc.def}|\\
|\childdocforwardprefix[|\textit{main}|]{|\textit{prefix}|}{|\textit{dest}|}|
\end{tabular}
\end{center}
%
the destination file is determined by a pattern
depending on the current file:
To make this work, the current file must be called
`{\textit{prefix}\hspace{0.2em}\textit{suffix}}'
with \textit{prefix} matching precisely the argument.
Processing is then passed on to the file
`{\textit{dest}\hspace{0.2em}\textit{suffix}}'.
Surely, the same effect is achieved by
directly specifying the
argument `{\textit{dest}\hspace{0.2em}\textit{suffix}}'
in the first form.
However, that requires to set up a different file
for each child. With the alternative form of the command
all these files can have exactly the same content
which simplifies setting them up and maintaining them.

For example, the following file |draft.tex|
with a compilation flag |\version| as described in \secref{sec:flags}
compiles the main document as a draft:
%
\begin{center}
\begin{tabular}{l}
|\def\version{draft}|\\
|\input{childdoc.def}|\\
|\childdocforward{|\textit{main}|}|
\end{tabular}
\end{center}
%
Likewise, the following files |final|\textit{nn}|.tex|
compile the final version of the child document
|child|\textit{nn}|.tex|:
%
\begin{center}
\begin{tabular}{l}
|\def\version{final}|\\
|\input{childdoc.def}|\\
|\childdocforwardprefix{final}{child}|
\end{tabular}
\end{center}
%

Note that when several versions of a main file and/or of each child file
are to be generated, it may be convenient to set up a |Makefile| or
shell script to automatise the process.

%%%%%%%%%%%%%%%%%%%%%%%%%%%%%%%%%%%%%%%%%%%%%%%%%%%%%%%%%%%%%%%%%%%%%%%%%%%%%%%%
\subsection{Command Line Processing}
\label{sec:commandline}

The effect of redirection files can also be achieved by invoking
the \LaTeX{} compiler with a more elaborate command line.
Most conveniently this should be done as part
of a shell script or a |Makefile|.

When using \textsf{childdoc} in the main file, the following
command lines effectively perform a redirection
(note that depending on the shell being used,
backslashes may have to be doubled: `|\|' $\to$ `|\\|'):
%
\begin{center}
|... -jobname "|\textit{target}|" |\\|"|[\textit{flags}]%
|\input{childdoc.def}\childdocforward[|\textit{main}|]{|\textit{dest}|}"|
\end{center}
%
Here \textit{target} is the name of the output file,
\textit{main} is the name of the main file
and \textit{dest} is the name of the main or child file to be processed
(all filenames without extensions).
The optional argument \textit{main} can be omitted
if \textit{main} matches \textit{dest}.
Optionally, compilation \textit{flags} can be defined via |\def| commands.
This command line makes the \TeX{} engine believe
it is compiling the file \textit{target}
whose content is specified as the latter parameter.
The provided code then forwards the processing to
\textit{main} or \textit{dest} as described in \secref{sec:forward}.

%%%%%%%%%%%%%%%%%%%%%%%%%%%%%%%%%%%%%%%%%%%%%%%%%%%%%%%%%%%%%%%%%%%%%%%%%%%%%%%%
\subsection{Include by Input}
\label{sec:input}

Including child documents by |\include| has some restrictions by design.
Most notably, the content of a child document always occupies
its own set of pages; pages cannot be shared between child documents.
Usually, this behaviour makes perfect sense
because each child document contain an essential part of the document.
However, in some situations it may be desirable to compose
a document from a collection of parts
without having mandatory page breaks between then.
For this case, the package
provides a mechanism to include parts
by |\input| which can also be processed individually.
However, by construction this mechanism
requires manual handling of the content to be output.

%%%%%%%%%%%%%%%%%%%%%%%%%%%%%%%%%%%%%%%%
\DescribeMacro{\ifchilddocmanual}
The main file should be prepared as usual, see \secref{sec:include}.
However, the document body must make a distinction
between processing of an individual part and of the main document, e.g.:
%
\begin{center}
\begin{tabular}{l}
|\ifchilddocmanual|\\
|\input{\childdocname}|\\
|\||else|\\
\textit{document body with }|\input{|\textit{part}|}|\\
|\||fi|
\end{tabular}
\end{center}
%
The conditional |\ifchilddocmanual| is true whenever
a part to be included by |\input| is being compiled,
and the name of the part is stored in |\childdocname|.

%%%%%%%%%%%%%%%%%%%%%%%%%%%%%%%%%%%%%%%%
\DescribeMacro{\childdocby}
Each part to be included by |\input| should start with:
%
\begin{center}
\begin{tabular}{l}
|\input{childdoc.def}|\\
|\childdocby{|\textit{main}|}|\\
\end{tabular}
\end{center}
%
The directive |\childdocby| is similar to |\childdocof|
described in \secref{sec:include},
but the subsequent selection of content must be done manually.
To that end, both |\ifchilddoc| and |\ifchilddocmanual|
will be true upon processing of a part,
and the name of the part is stored in |\childdocname|.
Note that |\jobname| will be set to the filename of the current part
so that each part receives an individual |.aux| file
that does not interfere with the |.aux| file(s) of the main document.
This behaviour can be altered by the alternative form
|\childdocby[*]{|\textit{main}|}| (with a non-empty optional argument)
which uses the |.aux| file of the main document
by setting |\jobname| to \textit{main}.

%%%%%%%%%%%%%%%%%%%%%%%%%%%%%%%%%%%%%%%%%%%%%%%%%%%%%%%%%%%%%%%%%%%%%%%%%%%%%%%%
\subsection{Driver Development}
\label{sec:driver}

The \textsf{childdoc} mechanism can also be use for the development
of definition files such as \LaTeX{} styles or classes.
This case differs from the above setup with multiple parts
included by |\include| in that no |\includeonly| should be invoked.
This can be achieved by starting the include file
(before |\ProvidesPackage|) with:
%
\begin{center}
\begin{tabular}{l}
|\input{childdoc.def}|\\
|\childdocforward{|\textit{main}|}|\\
\end{tabular}
\end{center}
%
or alternatively with:
%
\begin{center}
\begin{tabular}{l}
|\input{childdoc.def}|\\
|\childdocby{|\textit{main}|}|\\
\end{tabular}
\end{center}
%
Both forms have slightly different effects as described above.
The main file is prepared as usual, see \secref{sec:include}.

%%%%%%%%%%%%%%%%%%%%%%%%%%%%%%%%%%%%%%%%%%%%%%%%%%%%%%%%%%%%%%%%%%%%%%%%%%%%%%%%
\subsection{Legacy Detection}
\label{sec:detection}

The directive |\childdocmain| in the main file can detect
whether the complete document or merely a child is to be compiled
even without using the directive |\childdocof|.
This method is deprecated because it is less robust
and there is no compelling reason to use it;
it is merely provided for backward compatibility
and it may be removed in future versions.

If the detection mechanism is to be used,
it is mandatory to correctly specify
the filename of the main file as the argument of |\childdocmain|:
%
\begin{center}
\begin{tabular}{l}
|\input{childdoc.def}|\\
|\childdocmain{|\textit{main}|}|\\
\end{tabular}
\end{center}
%
If |\jobname| does not match the argument \textit{main} of |\childdocmain|,
it is assumed that |\jobname| points to the child file to be compiled.
When using |\childdocmain| with the main file specified as argument,
it suffices to start a child file
with just |\input{|\textit{main}|}|
without loading of the package and using |\childdocof|.
If instead all processing is done
with the appropriate \textsf{childdoc} directives,
the argument of \textit{main} of |\childdocmain| can be empty.

An alternative version of the command line processing described
in \secref{sec:commandline} using the detection mechanism reads:
%
\begin{center}
|... -jobname "|\textit{target}|" "|[\textit{flags}]%
[|\def\jobname{|\textit{dest}|}|]|\input{|\textit{main}|}"|
\end{center}

%%%%%%%%%%%%%%%%%%%%%%%%%%%%%%%%%%%%%%%%%%%%%%%%%%%%%%%%%%%%%%%%%%%%%%%%%%%%%%%%
\subsection{Manual Code}
\label{sec:manual}

In case one cannot be certain whether the definitions file |childdoc.def|
is installed on the target \TeX{} distribution
and one prefers not to ship it,
it is conceivable to paste a few relevant commands into the sources.

To that end, drop all statements |\input{childdoc.def}|
and perform the replacements as outlined below.
Instead of |\childdocmain{|\textit{main}|}| add the following code
to the top of the main file:
%
\begin{center}
\begin{tabular}{l}
|\||ifdefined\childdocname\endinput\||fi\newif\ifchilddoc|\\
|\edef\childdocname{\scantokens\expandafter{\jobname\noexpand}}|\\
|\def\childdocmain{|\textit{main}|}\||ifx\childdocmain\childdocname\||else|\\
|\childdoctrue\includeonly{\childdocname}\let\jobname\childdocmain\||fi|\\
\end{tabular}
\end{center}
%
Instead of |\childdocof{|\textit{main}|}| just include the main file
at the top of each child file:
%
\begin{center}
|\input{|\textit{main}|}|
\end{center}
%
A simple redirection |\childdocforward{|\textit{dest}|}| is achieved by:
%
\begin{center}
|\def\jobname{|\textit{dest}|}\input{\jobname}|
\end{center}
%
The redirection with prefix
|\childdocforwardprefix[|\textit{prefix}|]{|\textit{dest}|}|
is accomplished by:
%
\begin{center}
\begin{tabular}{l}
|{\edef\jobname{\scantokens\expandafter{\jobname\noexpand}}|\\
|\def\redirectjob |\textit{prefix}|#1~~~{\gdef\jobname{|\textit{dest}|#1}}|\\
|\expandafter\redirectjob\jobname~~~}\input{\jobname}|
\end{tabular}
\end{center}

In an alternative approach,
child documents can be compiled by a specific command line
without additional code or specific definitions:
%
\begin{center}
|... -jobname "|\textit{target}|" "|[\textit{flags}]%
|\includeonly{|\textit{dest}|}\input{|\textit{main}|}"|
\end{center}
%

%%%%%%%%%%%%%%%%%%%%%%%%%%%%%%%%%%%%%%%%%%%%%%%%%%%%%%%%%%%%%%%%%%%%%%%%%%%%%%%%
%%%%%%%%%%%%%%%%%%%%%%%%%%%%%%%%%%%%%%%%%%%%%%%%%%%%%%%%%%%%%%%%%%%%%%%%%%%%%%%%
\section{Information}

%%%%%%%%%%%%%%%%%%%%%%%%%%%%%%%%%%%%%%%%%%%%%%%%%%%%%%%%%%%%%%%%%%%%%%%%%%%%%%%%
\subsection{Copyright}

Copyright \copyright{} 2017--2018 Niklas Beisert

This work may be distributed and/or modified under the
conditions of the \LaTeX{} Project Public License, either version 1.3
of this license or (at your option) any later version.
The latest version of this license is in
  \url{http://www.latex-project.org/lppl.txt}
and version 1.3 or later is part of all distributions of \LaTeX{}
version 2005/12/01 or later.

This work has the LPPL maintenance status `maintained'.

The Current Maintainer of this work is Niklas Beisert.

This work consists of the files |README.txt|, |childdoc.ins| and |childdoc.dtx|
as well as the derived files |childdoc.def|, |cdocsamp.tex|
with |cdocsch1.tex|, |cdocsch2.tex|, |cdocspt3.tex|, |cdocspt4.tex|,
|cdocsdrf.tex|, |cdocsfn1.tex|, |cdocsfn2.tex|
as well as |childdoc.pdf|.

%%%%%%%%%%%%%%%%%%%%%%%%%%%%%%%%%%%%%%%%%%%%%%%%%%%%%%%%%%%%%%%%%%%%%%%%%%%%%%%%
\subsection{Files and Installation}

The package consists of the files:
%
\begin{center}
\begin{tabular}{ll}
    |README.txt|   & readme file \\
    |childdoc.ins| & installation file \\
    |childdoc.dtx| & source file \\
    |childdoc.def| & definition file \\
    |cdocsamp.tex| & sample main file \\
    |cdocsch1.tex| & sample include file \\
    |cdocsch2.tex| & sample include file \\
    |cdocspt3.tex| & sample part file \\
    |cdocspt4.tex| & sample part file \\
    |cdocsdrf.tex| & sample redirection file \\
    |cdocsfn1.tex| & sample redirection file \\
    |cdocsfn2.tex| & sample redirection file \\
    |childdoc.pdf| & manual
\end{tabular}
\end{center}
%
The distribution consists of the files
|README.txt|, |childdoc.ins| and |childdoc.dtx|.
%
\begin{itemize}
\item
Run (pdf)\LaTeX{} on |childdoc.dtx|
to compile the manual |childdoc.pdf| (this file).
\item
Run \LaTeX{} on |childdoc.ins| to create the definitions file |childdoc.def|
and the sample |cdocsamp.tex| with include files
|cdocsch1.tex|, |cdocsch2.tex|, |cdocspt3.tex|, |cdocspt4.tex|,
|cdocsdrf.tex|, |cdocsfn1.tex|, |cdocsfn2.tex|.
Then copy the file |childdoc.def| to an appropriate directory of your \LaTeX{}
distribution, e.g.\ \textit{texmf-root}|/tex/latex/childdoc|.
\end{itemize}

%%%%%%%%%%%%%%%%%%%%%%%%%%%%%%%%%%%%%%%%%%%%%%%%%%%%%%%%%%%%%%%%%%%%%%%%%%%%%%%%
\subsection{Related CTAN Packages}

There are several other packages which offer a similar functionality:
%
\begin{itemize}
\item
The packages
\href{http://ctan.org/pkg/docmute}{\textsf{docmute}},
\href{http://ctan.org/pkg/includex}{\textsf{includex}} and
\href{http://ctan.org/pkg/standalone}{\textsf{standalone}}
provide commands to include only the document body of
a child file thus allowing both files to be compiled individually.
\item
The packages \href{http://ctan.org/pkg/subdocs}{\textsf{subdocs}}
and \href{http://ctan.org/pkg/subfiles}{\textsf{subfiles}}
provide structures in which the main and child documents can be
encapsulated and allowing them to be compiled individually.
The inclusion mechanism is different from the conventional |\include|.
\item
The package \href{http://ctan.org/pkg/combine}{\textsf{combine}}
is an elaborate solution to combine several documents into one.
\end{itemize}
%
See also the CTAN topic \href{http://ctan.org/topic/subdocs}{\textsf{subdocs}}
for further related packages.
The present package differs from the above solutions in that
a document structure constructed with the conventional |\include| mechanism
just needs two extra commands at the top of every file
such that all constituent files can be compiled individually.

%%%%%%%%%%%%%%%%%%%%%%%%%%%%%%%%%%%%%%%%%%%%%%%%%%%%%%%%%%%%%%%%%%%%%%%%%%%%%%%%
%\subsection{Feature Suggestions}
%
%The following is a list of features which may be useful for future
%versions of this package:
%%
%\begin{itemize}
%\item
%\ldots
%\end{itemize}

%%%%%%%%%%%%%%%%%%%%%%%%%%%%%%%%%%%%%%%%%%%%%%%%%%%%%%%%%%%%%%%%%%%%%%%%%%%%%%%%
\subsection{Revision History}

%%%%%%%%%%%%%%%%%%%%%%%%%%%%%%%%%%%%%%%%
\paragraph{v2.0:} 2018/12/30

\begin{itemize}
\item
immediate forward processing
\item
added |\childdocby| mechanism
\item
manual restructured
\end{itemize}

%%%%%%%%%%%%%%%%%%%%%%%%%%%%%%%%%%%%%%%%
\paragraph{v1.6:} 2018/01/17

\begin{itemize}
\item
application for development of include files
\item
corrections to manual
\end{itemize}

%%%%%%%%%%%%%%%%%%%%%%%%%%%%%%%%%%%%%%%%
\paragraph{v1.5:} 2017/05/21

\begin{itemize}
\item
more complete structuring introduced
\item
|\childdocof| introduced
\item
|\childdoc| renamed to |\childdocmain|
\item
|\childredirect| renamed to |\childdocforward| and |\childdocforwardprefix|
and functionality expanded
\end{itemize}

%%%%%%%%%%%%%%%%%%%%%%%%%%%%%%%%%%%%%%%%
\paragraph{v1.0:} 2017/04/27

\begin{itemize}
\item
manual and install package
\item
first version published on CTAN
\end{itemize}

%%%%%%%%%%%%%%%%%%%%%%%%%%%%%%%%%%%%%%%%
\paragraph{v0.6:} 2017/04/26

\begin{itemize}
\item
redirection mechanism added
\end{itemize}

%%%%%%%%%%%%%%%%%%%%%%%%%%%%%%%%%%%%%%%%
\paragraph{v0.5:} 2017/04/26

\begin{itemize}
\item
functionality in definition file
\end{itemize}


%%%%%%%%%%%%%%%%%%%%%%%%%%%%%%%%%%%%%%%%%%%%%%%%%%%%%%%%%%%%%%%%%%%%%%%%%%%%%%%%
%%%%%%%%%%%%%%%%%%%%%%%%%%%%%%%%%%%%%%%%%%%%%%%%%%%%%%%%%%%%%%%%%%%%%%%%%%%%%%%%
%%%%%%%%%%%%%%%%%%%%%%%%%%%%%%%%%%%%%%%%%%%%%%%%%%%%%%%%%%%%%%%%%%%%%%%%%%%%%%%%
\appendix

\settowidth\MacroIndent{\rmfamily\scriptsize 000\ }

 \DocInput{childdoc.dtx}

\end{document}
%</driver>
% \fi
%
% %%%%%%%%%%%%%%%%%%%%%%%%%%%%%%%%%%%%%%%%%%%%%%%%%%%%%%%%%%%%%%%%%%%%%%%%%%%%%%
% %%%%%%%%%%%%%%%%%%%%%%%%%%%%%%%%%%%%%%%%%%%%%%%%%%%%%%%%%%%%%%%%%%%%%%%%%%%%%%
% \section{Sample}
%\iffalse
%<*samplemain>
%\fi
%
% The following presents a sample document
% with two chapters, two parts, a title page,
% a compile flag as well as three forwarding files to set the flag.
% It consists of eight |.tex| files:
% \begin{center}
% \begin{tabular}{ll}
% |cdocsamp.tex|&main file\\
% |cdocsch1.tex|&include file for chapter 1\\
% |cdocsch2.tex|&include file for chapter 2\\
% |cdocspt3.tex|&include file for part 3\\
% |cdocspt4.tex|&include file for part 4\\
% |cdocsdrf.tex|&forwarding file for main file in draft mode\\
% |cdocsfi1.tex|&forwarding file for final version of chapter 1\\
% |cdocsfi2.tex|&forwarding file for final version of chapter 2\\
% \end{tabular}
% \end{center}
% Each of the eight files can be compiled directly by the \LaTeX{} compiler.
%
% %%%%%%%%%%%%%%%%%%%%%%%%%%%%%%%%%%%%%%
% \paragraph{Main File.}
%
% The main file is called |cdocsamp.tex|.
%
% Load the \textsf{childdoc} definitions and
% declare the filename for the main document:
%    \begin{macrocode}
\input{childdoc.def}
\childdocmain{}
%    \end{macrocode}

% Optional override for |\version| flag:
%    \begin{macrocode}
%%\ifchilddoc\else\providecommand{\version}{draft}\fi
%    \end{macrocode}

% Define the default values for the |\version| flag
% (|final| for the main file and |draft| for childs):
%    \begin{macrocode}
\ifchilddoc
\providecommand{\version}{draft}
\else
\providecommand{\version}{final}
\fi
%    \end{macrocode}

% Load the standard document class:
%    \begin{macrocode}
\documentclass[12pt]{article}
%    \end{macrocode}

% Start the document body:
%    \begin{macrocode}
\begin{document}
%    \end{macrocode}

% Declare a title page.
% Print title, part of document being processed and version flag:
%    \begin{macrocode}
\addtocounter{page}{-1}
\begin{center}
{\LARGE\bfseries{}childdoc example\par}
\vspace{1cm}
\ifchilddoc
\ifchilddocmanual part\else chapter\fi:
`\childdocname' of `\childdocjob'\par
\else
main document: `\childdocjob'\par
\fi
version: \version\par
\end{center}
\newpage
%    \end{macrocode}

% Manually include selected file,
% otherwise process as usual:
%    \begin{macrocode}
\ifchilddocmanual
\section*{part `\childdocname'}
\input{\childdocname}
\else
%    \end{macrocode}

% Include the two chapters:
%    \begin{macrocode}
\include{cdocsch1}
\include{cdocsch2}
%    \end{macrocode}

% Include the two parts unless only chapters should be displayed:
%    \begin{macrocode}
\ifchilddoc\else
\section{part three}
\input{cdocspt3}
\section{part four}
\input{cdocspt4}
\fi
%    \end{macrocode}

% Process as usual until here:
%    \begin{macrocode}
\fi
%    \end{macrocode}

% End of document body:
%    \begin{macrocode}
\end{document}
%    \end{macrocode}
%\iffalse
%</samplemain>
%\fi
%
% %%%%%%%%%%%%%%%%%%%%%%%%%%%%%%%%%%%%%%
% \paragraph{Chapter Include Files.}
%
% The include files are called |cdocsch1.tex| and |cdocsch2.tex|.
%
%\iffalse
%<*samplechap1|samplechap2>
%\fi

% Optional override for |\version| flag:
%    \begin{macrocode}
%%\providecommand{\version}{final}
%    \end{macrocode}

% Include the main document:
%    \begin{macrocode}
\input{childdoc.def}
\childdocof{cdocsamp}
%    \end{macrocode}

%\iffalse
%</samplechap1|samplechap2>
%\fi
%
%\iffalse
%<*samplechap1>
%\fi
% Some text for chapter 1:
%    \begin{macrocode}
\section{one}
some text in chapter one
%    \end{macrocode}

%\iffalse
%</samplechap1>
%\fi
% Some text for chapter 2:
%\iffalse
%<*samplechap2>
%\fi
%    \begin{macrocode}
\section{two}
more text in chapter two
%    \end{macrocode}

%\iffalse
%</samplechap2>
%\fi
%
% %%%%%%%%%%%%%%%%%%%%%%%%%%%%%%%%%%%%%%
% \paragraph{Part Include Files.}
%
% The include files are called |cdocspt3.tex| and |cdocspt4.tex|.
%
%\iffalse
%<*samplepart3|samplepart4>
%\fi

% Optional override for |\version| flag:
%    \begin{macrocode}
%%\providecommand{\version}{final}
%    \end{macrocode}

% Include the main document:
%    \begin{macrocode}
\input{childdoc.def}
\childdocby{cdocsamp}
%    \end{macrocode}

%\iffalse
%</samplepart3|samplepart4>
%\fi
%
%\iffalse
%<*samplepart3>
%\fi
% Some text for part 3:
%    \begin{macrocode}
some text in part three
%    \end{macrocode}

%\iffalse
%</samplepart3>
%\fi
% Some text for part 4:
%\iffalse
%<*samplepart4>
%\fi
%    \begin{macrocode}
more text in part four
%    \end{macrocode}

%\iffalse
%</samplepart4>
%\fi
%
% %%%%%%%%%%%%%%%%%%%%%%%%%%%%%%%%%%%%%%
% \paragraph{Forwarding for a Complete Draft.}
%
% The following forwarding file |cdocsdrf.tex|
% compiles the main document in draft mode:
%\iffalse
%<*sampledraft>
%\fi
%    \begin{macrocode}
\def\version{draft}
\input{childdoc.def}
\childdocforward{cdocsamp}
%    \end{macrocode}

%\iffalse
%</sampledraft>
%\fi
%
% %%%%%%%%%%%%%%%%%%%%%%%%%%%%%%%%%%%%%%
% \paragraph{Forwarding for Final Version of the Chapters.}
%
% The following forwarding files |cdocsfn1.tex| and |cdocsfn2.tex|
% (with identical content)
% compile the final versions of the child documents
% |cdocsch1.tex| and |cdocsch2.tex|, respectively:
%\iffalse
%<*samplefinal>
%\fi
%    \begin{macrocode}
\def\version{final}
\input{childdoc.def}
\childdocforwardprefix[cdocsamp]{cdocsfn}{cdocsch}
%    \end{macrocode}

%\iffalse
%</samplefinal>
%\fi
%
% %%%%%%%%%%%%%%%%%%%%%%%%%%%%%%%%%%%%%%
% \paragraph{Command Line Processing.}
%
% The following three command lines generate the output files
% |cdocscld|, |cdocscl1| and |cdocscl2|
% which should be identical to
% |cdocsdrf|, |cdocsch1| and |cdocsfn2|, respectively:
% \begin{center}
% \begin{tabular}{l}
% |latex -jobname cdocscld \|\\
% |  "\def\version{draft}\input{childdoc.def}\childdocforward{cdocsamp}"|\\
% |latex -jobname cdocscl1 \|\\
% |  "\input{childdoc.def}\childdocforward[cdocsamp]{cdocsch1}"|\\
% |latex -jobname cdocscl2 \|\\
% |  "\def\version{final}\input{childdoc.def}\childdocforward{cdocsch2}"|
% \end{tabular}
% \end{center}
% Note that the trailing backslash on each first line
% merely continues the input to the second line
% (for convenient cut ant paste).
% Furthermore, the command |latex| can be replaced by any
% of its alternative versions such as |pdflatex|.
%
% %%%%%%%%%%%%%%%%%%%%%%%%%%%%%%%%%%%%%%%%%%%%%%%%%%%%%%%%%%%%%%%%%%%%%%%%%%%%%%
% %%%%%%%%%%%%%%%%%%%%%%%%%%%%%%%%%%%%%%%%%%%%%%%%%%%%%%%%%%%%%%%%%%%%%%%%%%%%%%
% \section{Implementation}
%\iffalse
%<*package>
%\fi
%
% This section describes the definitions file |childdoc.def|.

% The definitions cannot be loaded using |\usepackage| or |\RequirePackage|
% which has a mechanism to prevent loading a style file more than once.
% When loading the definitions by means of |\input|
% multiple instances have to be prevented manually:
%\iffalse
%This code needs to be before the `\ProvidesFile' directive
%which is defined at the beginning of this file.
%Therefore it is also placed there and commented out here.
%</package>
%<*discard>
%\fi
%    \begin{macrocode}
\ifdefined\childdocmain\endinput\fi
%    \end{macrocode}
%\iffalse
%</discard>
%<*package>
%\fi
%
% \macro{\ifchilddoc}
% \macro{\ifchilddocmanual}
% The conditional |\ifchilddoc| tells whether a
% child (true) or main (false) document is being compiled.
% The conditional |\ifchilddocmanual| tells whether
% the |\includeonly| mechanism is used (false) or
% the selection of child files must be performed manually (true).
% The definitions initialise to false:
%    \begin{macrocode}
\newif\ifchilddoc
\newif\ifchilddocmanual
%    \end{macrocode}

% \macro{\childdocname}
% \macro{\childdocjob}
% The macro |\childdocname| stores the name of the main document
% to be compiled. The macro |\childdocjob| stores the name of
% the document on which the \LaTeX{} compiler was originally invoked.
% The content of |\jobname| cannot be compared
% to filenames specified in the source due to different catcodes.
% The following code rescans |\jobname|, stores the result
% in |\childdocname| and saves a copy in |\childdocjob|:
%    \begin{macrocode}
\edef\childdocname{\scantokens\expandafter{\jobname\noexpand}}
\let\childdocjob\childdocname
%    \end{macrocode}

% \macro{\childdocdisable}
% The macro |\childdocdisable| prevents the main file
% from being processed more than once.
% At this stage, the main document command |\childdocmain|
% is assumed to be called once again where it should do nothing.
% Any subsequent call to it should prevent
% a secondary processing of the main document
% It overwrites the forwarding commands
% |\childdocof| and |\childdocforward|
% with empty macros to prevent further inclusions of the main document:
%    \begin{macrocode}
\newcommand{\childdocdisable}
{
  \renewcommand{\childdocmain}[1]{\renewcommand{\childdocmain}[1]{\endinput}}
  \renewcommand{\childdocof}[1]{}
  \renewcommand{\childdocby}[2][]{}
  \renewcommand{\childdocforward}[2][]{}
  \renewcommand{\childdocdisable}{}
}
%    \end{macrocode}

% \macro{\childdocmain}
% The macro |\childdocmain| is to be called at the top of the main file
% with nothing or the main filename (without extension) as argument.
% First, it breaks loops.
% If the argument is not empty and does not match |\childdocname|
% (which is set by the first inclusion of |childdoc.def|),
% |\ifchilddoc| is set to true, |\includeonly| is applied to the child file
% and |\jobname| is set to the main file
% (for proper handling of |.aux| files):
%    \begin{macrocode}
\newcommand{\childdocmain}[1]
{
  \childdocdisable\childdocmain{}
  \if?#1?\else
    \begingroup
      \def\childdoctmp{#1}
      \ifx\childdoctmp\childdocname
        \def\childdoctmp{}
      \else
        \def\childdoctmp
        {
          \childdoctrue
          \includeonly{\childdocname}
          \def\childdocjob{#1}
          \def\jobname{#1}
        }
      \fi
      \expandafter
    \endgroup
    \childdoctmp
  \fi
}
%    \end{macrocode}

% \macro{\childdocof}
% The command |\childdocof| redirects
% compilation to the main file |#1|.
%    \begin{macrocode}
\newcommand{\childdocof}[1]
{
  \childdocdisable
  \childdoctrue
  \includeonly{\childdocname}
  \def\jobname{#1}
  \def\childdocjob{#1}
  \input{#1}
}
%    \end{macrocode}

% \macro{\childdocby}
% The command |\childdocby| ....
%    \begin{macrocode}
\newcommand{\childdocby}[2][]
{
  \childdocdisable
  \childdoctrue
  \childdocmanualtrue
  \if?#1?\else
    \def\jobname{#2}
  \fi
  \def\childdocjob{#2}
  \input{#2}
  \endinput
}
%    \end{macrocode}

% \macro{\childdocforward}
% The command |\childdocforward| redirects
% compilation to the main file or
% (if the optional argument is given) a child file.
% Parameters are set as if the main file
% or a child file starting with |\childdocof| was compiled.
% Then compilation is handed over to the main file:
%    \begin{macrocode}
\newcommand{\childdocforward}[2][]
{
  \begingroup
    \if?#1?
      \def\childdoctmp
      {
        \def\childdocname{#2}
        \def\childdocjob{#2}
        \def\jobname{#2}
        \input{#2}
        \endinput
      }
    \else
      \def\childdoctmp
      {
        \childdocdisable
        \def\childdocname{#2}
        \childdoctrue
        \includeonly{#2}
        \def\childdocjob{#1}
        \def\jobname{#1}
        \input{#1}
        \endinput
      }
    \fi
    \expandafter
  \endgroup
  \childdoctmp
}
%    \end{macrocode}

% \macro{\childdocforwardprefix}
% The command |\childdocforwardprefix| redirects
% compilation to the main or a child file by means of a pattern.
% The prefix |#1| in the current filename is replaced by |#2|
% and the suffix of the current filename is kept
% (it is assumed that the filename does not contain the substring `|~~~|'
% which is used as a delimiter).
% Compilation is handed over to the new file by |\childdocforward|:
%    \begin{macrocode}
\newcommand{\childdocforwardprefix}[3][]
{
  \begingroup
    \def\childdocextract #2##1~~~{\def\childdoctmp{\childdocforward[#1]{#3##1}}}
    \expandafter\childdocextract\childdocname~~~
    \expandafter
  \endgroup
  \childdoctmp
}
%    \end{macrocode}

% \macro{\childdoc}
% The deprecated macro |\childdoc| is a legacy version of |\childdocmain|:
%    \begin{macrocode}
\newcommand{\childdoc}{\childdocmain}
%    \end{macrocode}

% \macro{\childdocredirect}
% The deprecated macro |\childdocredirect| is a legacy version
% of |\childdocforward| and |\childdocforwardprefix|:
%    \begin{macrocode}
\newcommand{\childdocredirect}[2][]
{
  \begingroup
    \if?#1?
      \def\childdoctmp{\childdocforward{#2}}
    \else
      \def\childdoctmp{\childdocforwardprefix{#1}{#2}}
    \fi
    \expandafter
  \endgroup
  \childdoctmp
}
%    \end{macrocode}

%\iffalse
%</package>
%\fi
%
\endinput
\childdocforward[cdocsamp]{cdocsch1}"|\\
% |latex -jobname cdocscl2 \|\\
% |  "\def\version{final}% \iffalse
%
% childdoc.dtx Copyright (C) 2017-2018 Niklas Beisert
%
% This work may be distributed and/or modified under the
% conditions of the LaTeX Project Public License, either version 1.3
% of this license or (at your option) any later version.
% The latest version of this license is in
%   http://www.latex-project.org/lppl.txt
% and version 1.3 or later is part of all distributions of LaTeX
% version 2005/12/01 or later.
%
% This work has the LPPL maintenance status `maintained'.
%
% The Current Maintainer of this work is Niklas Beisert.
%
% This work consists of the files childdoc.dtx and childdoc.ins
% and the derived files childdoc.def and cdocsamp.tex with
% cdocsch1.tex, cdocsch2.tex, cdocsdrf.tex, cdocsfn1.tex, cdocsfn2.tex.
%
%<package>\ifdefined\childdocmain\endinput\fi
%<package>\ProvidesFile{childdoc.def}[2018/12/30 v2.0 child document driver]
%<samplemain>\ProvidesFile{cdocsamp.tex}[2018/12/30 v2.0 sample for childdoc]
%<*driver>
%\ProvidesFile{childdoc.drv}[2018/12/30 v2.0 childdoc reference manual file]
\PassOptionsToClass{10pt,a4paper}{article}
\documentclass{ltxdoc}

\usepackage[margin=35mm]{geometry}
\usepackage{hyperref}
\usepackage{hyperxmp}
\usepackage[usenames]{color}

\hypersetup{colorlinks=true}
\hypersetup{pdfstartview=FitH}
\hypersetup{pdfpagemode=UseNone}
\hypersetup{pdfsource={}}
\hypersetup{pdflang={en-UK}}
\hypersetup{pdfcopyright={Copyright 2017-2018 Niklas Beisert.
  This work may be distributed and/or modified under the
  conditions of the LaTeX Project Public License, either version 1.3
  of this license or (at your option) any later version.}}
\hypersetup{pdflicenseurl={http://www.latex-project.org/lppl.txt}}
\hypersetup{pdfcontactaddress={ETH Zurich, ITP, HIT K,
  Wolfgang-Pauli-Strasse 27}}
\hypersetup{pdfcontactpostcode={8093}}
\hypersetup{pdfcontactcity={Zurich}}
\hypersetup{pdfcontactcountry={Switzerland}}
\hypersetup{pdfcontactemail={nbeisert@itp.phys.ethz.ch}}
\hypersetup{pdfcontacturl={http://people.phys.ethz.ch/\xmptilde nbeisert/}}

\newcommand{\secref}[1]{\hyperref[#1]{section \ref*{#1}}}

\parskip1ex
\parindent0pt
\let\olditemize\itemize
\def\itemize{\olditemize\parskip0pt}

\begin{document}

\title{The \textsf{childdoc} Package}
\hypersetup{pdftitle={The childdoc Package}}
\author{Niklas Beisert\\[2ex]
  Institut f\"ur Theoretische Physik\\
  Eidgen\"ossische Technische Hochschule Z\"urich\\
  Wolfgang-Pauli-Strasse 27, 8093 Z\"urich, Switzerland\\[1ex]
  \href{mailto:nbeisert@itp.phys.ethz.ch}
  {\texttt{nbeisert@itp.phys.ethz.ch}}}
\hypersetup{pdfauthor={Niklas Beisert}}
\hypersetup{pdfsubject={Manual for the LaTeX2e Package childdoc}}
\date{30 December 2018, \textsf{v2.0}}
\maketitle

\begin{abstract}\noindent
\textsf{childdoc} is a \LaTeXe{} package
that enables the direct compilation
of document sections included by |\include|
to individual files.
\end{abstract}

\begingroup
\parskip0ex
\tableofcontents
\endgroup

%%%%%%%%%%%%%%%%%%%%%%%%%%%%%%%%%%%%%%%%%%%%%%%%%%%%%%%%%%%%%%%%%%%%%%%%%%%%%%%%
%%%%%%%%%%%%%%%%%%%%%%%%%%%%%%%%%%%%%%%%%%%%%%%%%%%%%%%%%%%%%%%%%%%%%%%%%%%%%%%%
\section{Introduction}

\LaTeX{} provides a mechanism to structure a large document (such as a book)
into a main file and several child files (containing the chapters)
using the |\include| command.
This mechanism is beneficial for documents
which span hundreds of pages in order to
make the source file(s) more manageable.
Moreover, compilation can be restricted to
selected child files by means of the |\includeonly| command.
The latter feature can be used to reduce the compilation time while editing
(this was significantly more useful in the earlier days of \LaTeX{})
or to generate a smaller document which is easier to navigate.
Another application of |\includeonly| is to generate
documents consisting of selected parts of the complete document.

However, there are a few drawbacks of the plain |\include| mechanism:
\begin{itemize}
\item
The child files cannot be compiled on their own,
they can only be compiled via the main file.
A naive editing environment
(such as a text editor with an option
to have the current file processed by \LaTeX)
may require one to switch to the main file before compiling;
attempting to compile the child file produces errors.
\item
The main file must be modified (each time)
to adjust the |\includeonly| command
to the present needs. This easily leaves the main file in a messy state.
\item
The generated document will always carry the filename
of the main document. This is inconvenient if
several child files are to be compiled and
to be kept for distribution.
\end{itemize}

The present package provides a simple interface
to make child files individually compilable by \LaTeX{}.
Compiling a child file then has the same effect as compiling
the main file with an |\includeonly| command
to select the appropriate child.
Moreover the generated document will carry the name of the child
rather than the main file.
This resolves all three above issues.

This feature is meant to make the editing of books,
thesis documents and lecture notes somewhat more convenient.
However, the package can also be used efficiently for
composing a series of documents (such as exercise sheets)
which are typically distributed individually.
It then assists the author in generating the individual documents
(potentially in different versions)
as well as a document containing the collected series.
Another application is in developing style files
or other kinds of included material
where compilation of the style file could redirect
to a sample or test file.

%%%%%%%%%%%%%%%%%%%%%%%%%%%%%%%%%%%%%%%%%%%%%%%%%%%%%%%%%%%%%%%%%%%%%%%%%%%%%%%%
%%%%%%%%%%%%%%%%%%%%%%%%%%%%%%%%%%%%%%%%%%%%%%%%%%%%%%%%%%%%%%%%%%%%%%%%%%%%%%%%
\section{Usage}

First of all, the package \textsf{childdoc} is \emph{not} a standard
\LaTeXe{} |.sty| style file! Therefore it needs to be invoked in
a non-standard way.

%%%%%%%%%%%%%%%%%%%%%%%%%%%%%%%%%%%%%%%%%%%%%%%%%%%%%%%%%%%%%%%%%%%%%%%%%%%%%%%%
\subsection{Included Files}
\label{sec:include}

%%%%%%%%%%%%%%%%%%%%%%%%%%%%%%%%%%%%%%%%
\DescribeMacro{\childdocmain}
To use the package, add the commands
\begin{center}
\begin{tabular}{l}
|\input{childdoc.def}|\\
|\childdocmain{}|\\
\end{tabular}
\end{center}
at the very top of the main \LaTeX{} file,
in particular \emph{before} the |\documentclass| statement!
The argument of |\childdocmain| should be left empty
(but it must be present).

%%%%%%%%%%%%%%%%%%%%%%%%%%%%%%%%%%%%%%%%
\DescribeMacro{\childdocof}
Furthermore, add the commands
\begin{center}
\begin{tabular}{l}
|\input{childdoc.def}|\\
|\childdocof{|\textit{main}|}|\\
\end{tabular}
\end{center}
at the top of every child file \textit{child}
which is included by |\include{|\textit{child}|}|
from within the main file
(or at least for those files to be compiled individually).
The argument \textit{main} must be the filename of the main file.

There are a couple of
considerations in setting up the main and child documents:

%%%%%%%%%%%%%%%%%%%%%%%%%%%%%%%%%%%%%%%%
\paragraph{Restrictions.}

Please note the following restrictions:
\begin{itemize}
\item
|\childdocmain| must be called with one argument \textit{main}
to ensure compatibility with earlier version of the package.
It must either be empty (|\childdocmain{}|)
or precisely match the filename of the main file in which it is specified.
See \secref{sec:detection} for further information.
\item
The filename \textit{main} must be specified without the |.tex| extension.
\item
The filename \textit{main} is case sensitive
(even in case-insensitive file systems)
due to internal string comparison.
\item
The argument \textit{main} should be fully expanded, it cannot be a macro.
\item
Subdirectories and special characters should be avoided in filenames.
\item
The command |\childdocmain{|\textit{main}|}| must be followed by a whitespace.
It should not be followed immediately by another command
or by a comment mark `|%|'.
This is because the \TeX{} parser reads the token immediately following
the argument of |\childdocmain| and puts it
at the beginning of every child section;
however, a white\-space is ignored.
\end{itemize}

%%%%%%%%%%%%%%%%%%%%%%%%%%%%%%%%%%%%%%%%
\paragraph{Content of Main File.}

It is advisable to place all content in the child files included by |\include|.
Any output contained in the main file will appear in all child documents
unless suppressed manually;
it cannot be suppressed automatically by the |\includeonly| directive
and thus should normally be avoided.
A method to include some content in the main file
by means of conditional processing is described in \secref{sec:conditional}.

%%%%%%%%%%%%%%%%%%%%%%%%%%%%%%%%%%%%%%%%
\paragraph{Page Numbering.}

When only a part of the document is compiled,
the appropriate numbering of pages
(as well as other status parameters)
is determined from the |.aux| files.
The latter contain information from previous passes.
However this information needs to propagate through
all intermediate child documents.
Therefore the page numbering in child documents may well
be inconsistent until the complete document is compiled at least once.

A useful (if unconventional) way to always ensure a consistent
page numbering is to restart the numbering in each child document
and denote the pages by `\textit{child}|.|\textit{page}'
where \textit{child} represents the chapter/section number of the child file.
This can be achieved by the command
|\numberwithin{page}{|\textit{child}|}|
of the \textsf{amsmath} package
where \textit{child} can be |chapter| or |section|
depending on the chosen structuring.
Alternatively, one can modify the macro |\thepage| appropriately
and reset the counter |page| at the start of each child file.

%%%%%%%%%%%%%%%%%%%%%%%%%%%%%%%%%%%%%%%%%%%%%%%%%%%%%%%%%%%%%%%%%%%%%%%%%%%%%%%%
\subsection{Conditional Processing}
\label{sec:conditional}

The package provides a mechanism to compile different versions
of a document. To customise the versions further some conditional processing
can come in handy to distinguish which version is being compiled.
The package provides two macros to describe the compilation context:

%%%%%%%%%%%%%%%%%%%%%%%%%%%%%%%%%%%%%%%%
\DescribeMacro{\ifchilddoc}
The conditional |\ifchilddoc| distinguishes between the compilation of
child documents and the main document:
%
\begin{center}
|\ifchilddoc |\textit{child-code}| |[|\||else |\textit{main-code}]| \||fi|
\end{center}

%%%%%%%%%%%%%%%%%%%%%%%%%%%%%%%%%%%%%%%%
\DescribeMacro{\childdocname}
\DescribeMacro{\childdocjob}
The macro |\childdocname| contains the filename (without extension)
of the main or child file being processed.
Note that |\childdocjob| will always contain the name of the main file.

%%%%%%%%%%%%%%%%%%%%%%%%%%%%%%%%%%%%%%%%
\paragraph{Title Page.}

Conditional processing can be used to include a title or banner page
in the main document when proper precautions are taken.
Importantly, the code in the main file should ensure that the page counter
(as well as other status parameters which are stored in the |.aux| files)
takes the same value after the conditional processing.
Otherwise the page numbers may take divergent values
depending on which part is compiled.

For example, a title page could be declared by:
%
\begin{center}
\begin{tabular}{l}
|\ifchilddoc\||else|\\
|\addtocounter{page}{-1}|\\
\textit{code for title page}\\
|\newpage|\\
|\||fi|
\end{tabular}
\end{center}
%
A banner page for the child documents can be generated by:
%
\begin{center}
\begin{tabular}{l}
|\ifchilddoc|\\
|\addtocounter{page}{-1}|\\
\textit{code for banner page}\\
|\newpage|\\
|\||fi|
\end{tabular}
\end{center}
%
Here one could write a message such as:
\begin{center}
|This is the part \childdocname{} of \childdocjob{}.|
\end{center}

%%%%%%%%%%%%%%%%%%%%%%%%%%%%%%%%%%%%%%%%%%%%%%%%%%%%%%%%%%%%%%%%%%%%%%%%%%%%%%%%
\subsection{Flags}
\label{sec:flags}

The package makes it easy to generate different versions
of the main or child documents.
To this end compilation flags can be defined
and assigned different default values.
They will be particularly useful in conjunction
with the forwarding mechanism described in \secref{sec:forward}.

For example, it may be useful to have a flag |\version|
which can be set to |draft| or |final|.
The document source will contain some conditional code
depending on the value of |\version|.
Suppose further, the flag should default to |final| for the main file
and to |draft| for child files
which is a natural assignment for editing the document.
This is achieved by placing the following code
in the preamble of the main document
(below the |\childdocmain| directive):
%
\begin{center}
\begin{tabular}{l}
|\ifchilddoc|\\
|\providecommand{\version}{draft}|\\
|\||else|\\
|\providecommand{\version}{final}|\\
|\||fi|
\end{tabular}
\end{center}
%
The definition by |\providecommand| makes sure
that previous definitions are not overwritten.
Further statements |\providecommand{\version}{...}|
can thus be added before the above code to override it.

For the main file, one might add a line
(between |\childdocmain| and the above block)
%
\begin{center}
|%\ifchilddoc\||else\providecommand{\version}{draft}\||fi|
\end{center}
%
which can be uncommented to produce a draft version.
Likewise one can add a line to the very top of a child file
(above the |\childdocof{|\textit{main}|}| directive)
%
\begin{center}
|%\providecommand{\version}{final}|
\end{center}
%
which can be uncommented to produce the final version of this child document.

%%%%%%%%%%%%%%%%%%%%%%%%%%%%%%%%%%%%%%%%%%%%%%%%%%%%%%%%%%%%%%%%%%%%%%%%%%%%%%%%
\subsection{Forwarding}
\label{sec:forward}

Different versions of the main or child documents
using compilation flags as described in \secref{sec:flags}
can be (permanently) stored in different files
for convenient compilation, viewing and distribution.
To this end, the package defines a command
to pass on compilation to a different file:

%%%%%%%%%%%%%%%%%%%%%%%%%%%%%%%%%%%%%%%%
\DescribeMacro{\childdocforward}
The command |\childdocforward| redirects processing to
another source file:
%
\begin{center}
\begin{tabular}{l}
|\input{childdoc.def}|\\
|\childdocforward[|\textit{main}|]{|\textit{dest}|}|\\
\end{tabular}
\end{center}
%
The argument \textit{dest} is the destination file
(without extension).
It should be the main file or one of the child files.
Note that further \textsf{childdoc} directives
such as |\childdocof| and |\childdocforward|
in the indicated file will be processed in this form.
The optional argument \textit{main}
passes on directly to the main file \textit{main}
while pretending to compile the child \textit{dest}.
This form behaves as if \textit{dest}
issues |\childdocof{|\textit{main}|}| right away,
and no further \textsf{childdoc} directives will be processed.

%%%%%%%%%%%%%%%%%%%%%%%%%%%%%%%%%%%%%%%%
\DescribeMacro{\...prefix}
In the alternative form |\childdocforwardprefix|,
%
\begin{center}
\begin{tabular}{l}
|\input{childdoc.def}|\\
|\childdocforwardprefix[|\textit{main}|]{|\textit{prefix}|}{|\textit{dest}|}|
\end{tabular}
\end{center}
%
the destination file is determined by a pattern
depending on the current file:
To make this work, the current file must be called
`{\textit{prefix}\hspace{0.2em}\textit{suffix}}'
with \textit{prefix} matching precisely the argument.
Processing is then passed on to the file
`{\textit{dest}\hspace{0.2em}\textit{suffix}}'.
Surely, the same effect is achieved by
directly specifying the
argument `{\textit{dest}\hspace{0.2em}\textit{suffix}}'
in the first form.
However, that requires to set up a different file
for each child. With the alternative form of the command
all these files can have exactly the same content
which simplifies setting them up and maintaining them.

For example, the following file |draft.tex|
with a compilation flag |\version| as described in \secref{sec:flags}
compiles the main document as a draft:
%
\begin{center}
\begin{tabular}{l}
|\def\version{draft}|\\
|\input{childdoc.def}|\\
|\childdocforward{|\textit{main}|}|
\end{tabular}
\end{center}
%
Likewise, the following files |final|\textit{nn}|.tex|
compile the final version of the child document
|child|\textit{nn}|.tex|:
%
\begin{center}
\begin{tabular}{l}
|\def\version{final}|\\
|\input{childdoc.def}|\\
|\childdocforwardprefix{final}{child}|
\end{tabular}
\end{center}
%

Note that when several versions of a main file and/or of each child file
are to be generated, it may be convenient to set up a |Makefile| or
shell script to automatise the process.

%%%%%%%%%%%%%%%%%%%%%%%%%%%%%%%%%%%%%%%%%%%%%%%%%%%%%%%%%%%%%%%%%%%%%%%%%%%%%%%%
\subsection{Command Line Processing}
\label{sec:commandline}

The effect of redirection files can also be achieved by invoking
the \LaTeX{} compiler with a more elaborate command line.
Most conveniently this should be done as part
of a shell script or a |Makefile|.

When using \textsf{childdoc} in the main file, the following
command lines effectively perform a redirection
(note that depending on the shell being used,
backslashes may have to be doubled: `|\|' $\to$ `|\\|'):
%
\begin{center}
|... -jobname "|\textit{target}|" |\\|"|[\textit{flags}]%
|\input{childdoc.def}\childdocforward[|\textit{main}|]{|\textit{dest}|}"|
\end{center}
%
Here \textit{target} is the name of the output file,
\textit{main} is the name of the main file
and \textit{dest} is the name of the main or child file to be processed
(all filenames without extensions).
The optional argument \textit{main} can be omitted
if \textit{main} matches \textit{dest}.
Optionally, compilation \textit{flags} can be defined via |\def| commands.
This command line makes the \TeX{} engine believe
it is compiling the file \textit{target}
whose content is specified as the latter parameter.
The provided code then forwards the processing to
\textit{main} or \textit{dest} as described in \secref{sec:forward}.

%%%%%%%%%%%%%%%%%%%%%%%%%%%%%%%%%%%%%%%%%%%%%%%%%%%%%%%%%%%%%%%%%%%%%%%%%%%%%%%%
\subsection{Include by Input}
\label{sec:input}

Including child documents by |\include| has some restrictions by design.
Most notably, the content of a child document always occupies
its own set of pages; pages cannot be shared between child documents.
Usually, this behaviour makes perfect sense
because each child document contain an essential part of the document.
However, in some situations it may be desirable to compose
a document from a collection of parts
without having mandatory page breaks between then.
For this case, the package
provides a mechanism to include parts
by |\input| which can also be processed individually.
However, by construction this mechanism
requires manual handling of the content to be output.

%%%%%%%%%%%%%%%%%%%%%%%%%%%%%%%%%%%%%%%%
\DescribeMacro{\ifchilddocmanual}
The main file should be prepared as usual, see \secref{sec:include}.
However, the document body must make a distinction
between processing of an individual part and of the main document, e.g.:
%
\begin{center}
\begin{tabular}{l}
|\ifchilddocmanual|\\
|\input{\childdocname}|\\
|\||else|\\
\textit{document body with }|\input{|\textit{part}|}|\\
|\||fi|
\end{tabular}
\end{center}
%
The conditional |\ifchilddocmanual| is true whenever
a part to be included by |\input| is being compiled,
and the name of the part is stored in |\childdocname|.

%%%%%%%%%%%%%%%%%%%%%%%%%%%%%%%%%%%%%%%%
\DescribeMacro{\childdocby}
Each part to be included by |\input| should start with:
%
\begin{center}
\begin{tabular}{l}
|\input{childdoc.def}|\\
|\childdocby{|\textit{main}|}|\\
\end{tabular}
\end{center}
%
The directive |\childdocby| is similar to |\childdocof|
described in \secref{sec:include},
but the subsequent selection of content must be done manually.
To that end, both |\ifchilddoc| and |\ifchilddocmanual|
will be true upon processing of a part,
and the name of the part is stored in |\childdocname|.
Note that |\jobname| will be set to the filename of the current part
so that each part receives an individual |.aux| file
that does not interfere with the |.aux| file(s) of the main document.
This behaviour can be altered by the alternative form
|\childdocby[*]{|\textit{main}|}| (with a non-empty optional argument)
which uses the |.aux| file of the main document
by setting |\jobname| to \textit{main}.

%%%%%%%%%%%%%%%%%%%%%%%%%%%%%%%%%%%%%%%%%%%%%%%%%%%%%%%%%%%%%%%%%%%%%%%%%%%%%%%%
\subsection{Driver Development}
\label{sec:driver}

The \textsf{childdoc} mechanism can also be use for the development
of definition files such as \LaTeX{} styles or classes.
This case differs from the above setup with multiple parts
included by |\include| in that no |\includeonly| should be invoked.
This can be achieved by starting the include file
(before |\ProvidesPackage|) with:
%
\begin{center}
\begin{tabular}{l}
|\input{childdoc.def}|\\
|\childdocforward{|\textit{main}|}|\\
\end{tabular}
\end{center}
%
or alternatively with:
%
\begin{center}
\begin{tabular}{l}
|\input{childdoc.def}|\\
|\childdocby{|\textit{main}|}|\\
\end{tabular}
\end{center}
%
Both forms have slightly different effects as described above.
The main file is prepared as usual, see \secref{sec:include}.

%%%%%%%%%%%%%%%%%%%%%%%%%%%%%%%%%%%%%%%%%%%%%%%%%%%%%%%%%%%%%%%%%%%%%%%%%%%%%%%%
\subsection{Legacy Detection}
\label{sec:detection}

The directive |\childdocmain| in the main file can detect
whether the complete document or merely a child is to be compiled
even without using the directive |\childdocof|.
This method is deprecated because it is less robust
and there is no compelling reason to use it;
it is merely provided for backward compatibility
and it may be removed in future versions.

If the detection mechanism is to be used,
it is mandatory to correctly specify
the filename of the main file as the argument of |\childdocmain|:
%
\begin{center}
\begin{tabular}{l}
|\input{childdoc.def}|\\
|\childdocmain{|\textit{main}|}|\\
\end{tabular}
\end{center}
%
If |\jobname| does not match the argument \textit{main} of |\childdocmain|,
it is assumed that |\jobname| points to the child file to be compiled.
When using |\childdocmain| with the main file specified as argument,
it suffices to start a child file
with just |\input{|\textit{main}|}|
without loading of the package and using |\childdocof|.
If instead all processing is done
with the appropriate \textsf{childdoc} directives,
the argument of \textit{main} of |\childdocmain| can be empty.

An alternative version of the command line processing described
in \secref{sec:commandline} using the detection mechanism reads:
%
\begin{center}
|... -jobname "|\textit{target}|" "|[\textit{flags}]%
[|\def\jobname{|\textit{dest}|}|]|\input{|\textit{main}|}"|
\end{center}

%%%%%%%%%%%%%%%%%%%%%%%%%%%%%%%%%%%%%%%%%%%%%%%%%%%%%%%%%%%%%%%%%%%%%%%%%%%%%%%%
\subsection{Manual Code}
\label{sec:manual}

In case one cannot be certain whether the definitions file |childdoc.def|
is installed on the target \TeX{} distribution
and one prefers not to ship it,
it is conceivable to paste a few relevant commands into the sources.

To that end, drop all statements |\input{childdoc.def}|
and perform the replacements as outlined below.
Instead of |\childdocmain{|\textit{main}|}| add the following code
to the top of the main file:
%
\begin{center}
\begin{tabular}{l}
|\||ifdefined\childdocname\endinput\||fi\newif\ifchilddoc|\\
|\edef\childdocname{\scantokens\expandafter{\jobname\noexpand}}|\\
|\def\childdocmain{|\textit{main}|}\||ifx\childdocmain\childdocname\||else|\\
|\childdoctrue\includeonly{\childdocname}\let\jobname\childdocmain\||fi|\\
\end{tabular}
\end{center}
%
Instead of |\childdocof{|\textit{main}|}| just include the main file
at the top of each child file:
%
\begin{center}
|\input{|\textit{main}|}|
\end{center}
%
A simple redirection |\childdocforward{|\textit{dest}|}| is achieved by:
%
\begin{center}
|\def\jobname{|\textit{dest}|}\input{\jobname}|
\end{center}
%
The redirection with prefix
|\childdocforwardprefix[|\textit{prefix}|]{|\textit{dest}|}|
is accomplished by:
%
\begin{center}
\begin{tabular}{l}
|{\edef\jobname{\scantokens\expandafter{\jobname\noexpand}}|\\
|\def\redirectjob |\textit{prefix}|#1~~~{\gdef\jobname{|\textit{dest}|#1}}|\\
|\expandafter\redirectjob\jobname~~~}\input{\jobname}|
\end{tabular}
\end{center}

In an alternative approach,
child documents can be compiled by a specific command line
without additional code or specific definitions:
%
\begin{center}
|... -jobname "|\textit{target}|" "|[\textit{flags}]%
|\includeonly{|\textit{dest}|}\input{|\textit{main}|}"|
\end{center}
%

%%%%%%%%%%%%%%%%%%%%%%%%%%%%%%%%%%%%%%%%%%%%%%%%%%%%%%%%%%%%%%%%%%%%%%%%%%%%%%%%
%%%%%%%%%%%%%%%%%%%%%%%%%%%%%%%%%%%%%%%%%%%%%%%%%%%%%%%%%%%%%%%%%%%%%%%%%%%%%%%%
\section{Information}

%%%%%%%%%%%%%%%%%%%%%%%%%%%%%%%%%%%%%%%%%%%%%%%%%%%%%%%%%%%%%%%%%%%%%%%%%%%%%%%%
\subsection{Copyright}

Copyright \copyright{} 2017--2018 Niklas Beisert

This work may be distributed and/or modified under the
conditions of the \LaTeX{} Project Public License, either version 1.3
of this license or (at your option) any later version.
The latest version of this license is in
  \url{http://www.latex-project.org/lppl.txt}
and version 1.3 or later is part of all distributions of \LaTeX{}
version 2005/12/01 or later.

This work has the LPPL maintenance status `maintained'.

The Current Maintainer of this work is Niklas Beisert.

This work consists of the files |README.txt|, |childdoc.ins| and |childdoc.dtx|
as well as the derived files |childdoc.def|, |cdocsamp.tex|
with |cdocsch1.tex|, |cdocsch2.tex|, |cdocspt3.tex|, |cdocspt4.tex|,
|cdocsdrf.tex|, |cdocsfn1.tex|, |cdocsfn2.tex|
as well as |childdoc.pdf|.

%%%%%%%%%%%%%%%%%%%%%%%%%%%%%%%%%%%%%%%%%%%%%%%%%%%%%%%%%%%%%%%%%%%%%%%%%%%%%%%%
\subsection{Files and Installation}

The package consists of the files:
%
\begin{center}
\begin{tabular}{ll}
    |README.txt|   & readme file \\
    |childdoc.ins| & installation file \\
    |childdoc.dtx| & source file \\
    |childdoc.def| & definition file \\
    |cdocsamp.tex| & sample main file \\
    |cdocsch1.tex| & sample include file \\
    |cdocsch2.tex| & sample include file \\
    |cdocspt3.tex| & sample part file \\
    |cdocspt4.tex| & sample part file \\
    |cdocsdrf.tex| & sample redirection file \\
    |cdocsfn1.tex| & sample redirection file \\
    |cdocsfn2.tex| & sample redirection file \\
    |childdoc.pdf| & manual
\end{tabular}
\end{center}
%
The distribution consists of the files
|README.txt|, |childdoc.ins| and |childdoc.dtx|.
%
\begin{itemize}
\item
Run (pdf)\LaTeX{} on |childdoc.dtx|
to compile the manual |childdoc.pdf| (this file).
\item
Run \LaTeX{} on |childdoc.ins| to create the definitions file |childdoc.def|
and the sample |cdocsamp.tex| with include files
|cdocsch1.tex|, |cdocsch2.tex|, |cdocspt3.tex|, |cdocspt4.tex|,
|cdocsdrf.tex|, |cdocsfn1.tex|, |cdocsfn2.tex|.
Then copy the file |childdoc.def| to an appropriate directory of your \LaTeX{}
distribution, e.g.\ \textit{texmf-root}|/tex/latex/childdoc|.
\end{itemize}

%%%%%%%%%%%%%%%%%%%%%%%%%%%%%%%%%%%%%%%%%%%%%%%%%%%%%%%%%%%%%%%%%%%%%%%%%%%%%%%%
\subsection{Related CTAN Packages}

There are several other packages which offer a similar functionality:
%
\begin{itemize}
\item
The packages
\href{http://ctan.org/pkg/docmute}{\textsf{docmute}},
\href{http://ctan.org/pkg/includex}{\textsf{includex}} and
\href{http://ctan.org/pkg/standalone}{\textsf{standalone}}
provide commands to include only the document body of
a child file thus allowing both files to be compiled individually.
\item
The packages \href{http://ctan.org/pkg/subdocs}{\textsf{subdocs}}
and \href{http://ctan.org/pkg/subfiles}{\textsf{subfiles}}
provide structures in which the main and child documents can be
encapsulated and allowing them to be compiled individually.
The inclusion mechanism is different from the conventional |\include|.
\item
The package \href{http://ctan.org/pkg/combine}{\textsf{combine}}
is an elaborate solution to combine several documents into one.
\end{itemize}
%
See also the CTAN topic \href{http://ctan.org/topic/subdocs}{\textsf{subdocs}}
for further related packages.
The present package differs from the above solutions in that
a document structure constructed with the conventional |\include| mechanism
just needs two extra commands at the top of every file
such that all constituent files can be compiled individually.

%%%%%%%%%%%%%%%%%%%%%%%%%%%%%%%%%%%%%%%%%%%%%%%%%%%%%%%%%%%%%%%%%%%%%%%%%%%%%%%%
%\subsection{Feature Suggestions}
%
%The following is a list of features which may be useful for future
%versions of this package:
%%
%\begin{itemize}
%\item
%\ldots
%\end{itemize}

%%%%%%%%%%%%%%%%%%%%%%%%%%%%%%%%%%%%%%%%%%%%%%%%%%%%%%%%%%%%%%%%%%%%%%%%%%%%%%%%
\subsection{Revision History}

%%%%%%%%%%%%%%%%%%%%%%%%%%%%%%%%%%%%%%%%
\paragraph{v2.0:} 2018/12/30

\begin{itemize}
\item
immediate forward processing
\item
added |\childdocby| mechanism
\item
manual restructured
\end{itemize}

%%%%%%%%%%%%%%%%%%%%%%%%%%%%%%%%%%%%%%%%
\paragraph{v1.6:} 2018/01/17

\begin{itemize}
\item
application for development of include files
\item
corrections to manual
\end{itemize}

%%%%%%%%%%%%%%%%%%%%%%%%%%%%%%%%%%%%%%%%
\paragraph{v1.5:} 2017/05/21

\begin{itemize}
\item
more complete structuring introduced
\item
|\childdocof| introduced
\item
|\childdoc| renamed to |\childdocmain|
\item
|\childredirect| renamed to |\childdocforward| and |\childdocforwardprefix|
and functionality expanded
\end{itemize}

%%%%%%%%%%%%%%%%%%%%%%%%%%%%%%%%%%%%%%%%
\paragraph{v1.0:} 2017/04/27

\begin{itemize}
\item
manual and install package
\item
first version published on CTAN
\end{itemize}

%%%%%%%%%%%%%%%%%%%%%%%%%%%%%%%%%%%%%%%%
\paragraph{v0.6:} 2017/04/26

\begin{itemize}
\item
redirection mechanism added
\end{itemize}

%%%%%%%%%%%%%%%%%%%%%%%%%%%%%%%%%%%%%%%%
\paragraph{v0.5:} 2017/04/26

\begin{itemize}
\item
functionality in definition file
\end{itemize}


%%%%%%%%%%%%%%%%%%%%%%%%%%%%%%%%%%%%%%%%%%%%%%%%%%%%%%%%%%%%%%%%%%%%%%%%%%%%%%%%
%%%%%%%%%%%%%%%%%%%%%%%%%%%%%%%%%%%%%%%%%%%%%%%%%%%%%%%%%%%%%%%%%%%%%%%%%%%%%%%%
%%%%%%%%%%%%%%%%%%%%%%%%%%%%%%%%%%%%%%%%%%%%%%%%%%%%%%%%%%%%%%%%%%%%%%%%%%%%%%%%
\appendix

\settowidth\MacroIndent{\rmfamily\scriptsize 000\ }

 \DocInput{childdoc.dtx}

\end{document}
%</driver>
% \fi
%
% %%%%%%%%%%%%%%%%%%%%%%%%%%%%%%%%%%%%%%%%%%%%%%%%%%%%%%%%%%%%%%%%%%%%%%%%%%%%%%
% %%%%%%%%%%%%%%%%%%%%%%%%%%%%%%%%%%%%%%%%%%%%%%%%%%%%%%%%%%%%%%%%%%%%%%%%%%%%%%
% \section{Sample}
%\iffalse
%<*samplemain>
%\fi
%
% The following presents a sample document
% with two chapters, two parts, a title page,
% a compile flag as well as three forwarding files to set the flag.
% It consists of eight |.tex| files:
% \begin{center}
% \begin{tabular}{ll}
% |cdocsamp.tex|&main file\\
% |cdocsch1.tex|&include file for chapter 1\\
% |cdocsch2.tex|&include file for chapter 2\\
% |cdocspt3.tex|&include file for part 3\\
% |cdocspt4.tex|&include file for part 4\\
% |cdocsdrf.tex|&forwarding file for main file in draft mode\\
% |cdocsfi1.tex|&forwarding file for final version of chapter 1\\
% |cdocsfi2.tex|&forwarding file for final version of chapter 2\\
% \end{tabular}
% \end{center}
% Each of the eight files can be compiled directly by the \LaTeX{} compiler.
%
% %%%%%%%%%%%%%%%%%%%%%%%%%%%%%%%%%%%%%%
% \paragraph{Main File.}
%
% The main file is called |cdocsamp.tex|.
%
% Load the \textsf{childdoc} definitions and
% declare the filename for the main document:
%    \begin{macrocode}
\input{childdoc.def}
\childdocmain{}
%    \end{macrocode}

% Optional override for |\version| flag:
%    \begin{macrocode}
%%\ifchilddoc\else\providecommand{\version}{draft}\fi
%    \end{macrocode}

% Define the default values for the |\version| flag
% (|final| for the main file and |draft| for childs):
%    \begin{macrocode}
\ifchilddoc
\providecommand{\version}{draft}
\else
\providecommand{\version}{final}
\fi
%    \end{macrocode}

% Load the standard document class:
%    \begin{macrocode}
\documentclass[12pt]{article}
%    \end{macrocode}

% Start the document body:
%    \begin{macrocode}
\begin{document}
%    \end{macrocode}

% Declare a title page.
% Print title, part of document being processed and version flag:
%    \begin{macrocode}
\addtocounter{page}{-1}
\begin{center}
{\LARGE\bfseries{}childdoc example\par}
\vspace{1cm}
\ifchilddoc
\ifchilddocmanual part\else chapter\fi:
`\childdocname' of `\childdocjob'\par
\else
main document: `\childdocjob'\par
\fi
version: \version\par
\end{center}
\newpage
%    \end{macrocode}

% Manually include selected file,
% otherwise process as usual:
%    \begin{macrocode}
\ifchilddocmanual
\section*{part `\childdocname'}
\input{\childdocname}
\else
%    \end{macrocode}

% Include the two chapters:
%    \begin{macrocode}
\include{cdocsch1}
\include{cdocsch2}
%    \end{macrocode}

% Include the two parts unless only chapters should be displayed:
%    \begin{macrocode}
\ifchilddoc\else
\section{part three}
\input{cdocspt3}
\section{part four}
\input{cdocspt4}
\fi
%    \end{macrocode}

% Process as usual until here:
%    \begin{macrocode}
\fi
%    \end{macrocode}

% End of document body:
%    \begin{macrocode}
\end{document}
%    \end{macrocode}
%\iffalse
%</samplemain>
%\fi
%
% %%%%%%%%%%%%%%%%%%%%%%%%%%%%%%%%%%%%%%
% \paragraph{Chapter Include Files.}
%
% The include files are called |cdocsch1.tex| and |cdocsch2.tex|.
%
%\iffalse
%<*samplechap1|samplechap2>
%\fi

% Optional override for |\version| flag:
%    \begin{macrocode}
%%\providecommand{\version}{final}
%    \end{macrocode}

% Include the main document:
%    \begin{macrocode}
\input{childdoc.def}
\childdocof{cdocsamp}
%    \end{macrocode}

%\iffalse
%</samplechap1|samplechap2>
%\fi
%
%\iffalse
%<*samplechap1>
%\fi
% Some text for chapter 1:
%    \begin{macrocode}
\section{one}
some text in chapter one
%    \end{macrocode}

%\iffalse
%</samplechap1>
%\fi
% Some text for chapter 2:
%\iffalse
%<*samplechap2>
%\fi
%    \begin{macrocode}
\section{two}
more text in chapter two
%    \end{macrocode}

%\iffalse
%</samplechap2>
%\fi
%
% %%%%%%%%%%%%%%%%%%%%%%%%%%%%%%%%%%%%%%
% \paragraph{Part Include Files.}
%
% The include files are called |cdocspt3.tex| and |cdocspt4.tex|.
%
%\iffalse
%<*samplepart3|samplepart4>
%\fi

% Optional override for |\version| flag:
%    \begin{macrocode}
%%\providecommand{\version}{final}
%    \end{macrocode}

% Include the main document:
%    \begin{macrocode}
\input{childdoc.def}
\childdocby{cdocsamp}
%    \end{macrocode}

%\iffalse
%</samplepart3|samplepart4>
%\fi
%
%\iffalse
%<*samplepart3>
%\fi
% Some text for part 3:
%    \begin{macrocode}
some text in part three
%    \end{macrocode}

%\iffalse
%</samplepart3>
%\fi
% Some text for part 4:
%\iffalse
%<*samplepart4>
%\fi
%    \begin{macrocode}
more text in part four
%    \end{macrocode}

%\iffalse
%</samplepart4>
%\fi
%
% %%%%%%%%%%%%%%%%%%%%%%%%%%%%%%%%%%%%%%
% \paragraph{Forwarding for a Complete Draft.}
%
% The following forwarding file |cdocsdrf.tex|
% compiles the main document in draft mode:
%\iffalse
%<*sampledraft>
%\fi
%    \begin{macrocode}
\def\version{draft}
\input{childdoc.def}
\childdocforward{cdocsamp}
%    \end{macrocode}

%\iffalse
%</sampledraft>
%\fi
%
% %%%%%%%%%%%%%%%%%%%%%%%%%%%%%%%%%%%%%%
% \paragraph{Forwarding for Final Version of the Chapters.}
%
% The following forwarding files |cdocsfn1.tex| and |cdocsfn2.tex|
% (with identical content)
% compile the final versions of the child documents
% |cdocsch1.tex| and |cdocsch2.tex|, respectively:
%\iffalse
%<*samplefinal>
%\fi
%    \begin{macrocode}
\def\version{final}
\input{childdoc.def}
\childdocforwardprefix[cdocsamp]{cdocsfn}{cdocsch}
%    \end{macrocode}

%\iffalse
%</samplefinal>
%\fi
%
% %%%%%%%%%%%%%%%%%%%%%%%%%%%%%%%%%%%%%%
% \paragraph{Command Line Processing.}
%
% The following three command lines generate the output files
% |cdocscld|, |cdocscl1| and |cdocscl2|
% which should be identical to
% |cdocsdrf|, |cdocsch1| and |cdocsfn2|, respectively:
% \begin{center}
% \begin{tabular}{l}
% |latex -jobname cdocscld \|\\
% |  "\def\version{draft}\input{childdoc.def}\childdocforward{cdocsamp}"|\\
% |latex -jobname cdocscl1 \|\\
% |  "\input{childdoc.def}\childdocforward[cdocsamp]{cdocsch1}"|\\
% |latex -jobname cdocscl2 \|\\
% |  "\def\version{final}\input{childdoc.def}\childdocforward{cdocsch2}"|
% \end{tabular}
% \end{center}
% Note that the trailing backslash on each first line
% merely continues the input to the second line
% (for convenient cut ant paste).
% Furthermore, the command |latex| can be replaced by any
% of its alternative versions such as |pdflatex|.
%
% %%%%%%%%%%%%%%%%%%%%%%%%%%%%%%%%%%%%%%%%%%%%%%%%%%%%%%%%%%%%%%%%%%%%%%%%%%%%%%
% %%%%%%%%%%%%%%%%%%%%%%%%%%%%%%%%%%%%%%%%%%%%%%%%%%%%%%%%%%%%%%%%%%%%%%%%%%%%%%
% \section{Implementation}
%\iffalse
%<*package>
%\fi
%
% This section describes the definitions file |childdoc.def|.

% The definitions cannot be loaded using |\usepackage| or |\RequirePackage|
% which has a mechanism to prevent loading a style file more than once.
% When loading the definitions by means of |\input|
% multiple instances have to be prevented manually:
%\iffalse
%This code needs to be before the `\ProvidesFile' directive
%which is defined at the beginning of this file.
%Therefore it is also placed there and commented out here.
%</package>
%<*discard>
%\fi
%    \begin{macrocode}
\ifdefined\childdocmain\endinput\fi
%    \end{macrocode}
%\iffalse
%</discard>
%<*package>
%\fi
%
% \macro{\ifchilddoc}
% \macro{\ifchilddocmanual}
% The conditional |\ifchilddoc| tells whether a
% child (true) or main (false) document is being compiled.
% The conditional |\ifchilddocmanual| tells whether
% the |\includeonly| mechanism is used (false) or
% the selection of child files must be performed manually (true).
% The definitions initialise to false:
%    \begin{macrocode}
\newif\ifchilddoc
\newif\ifchilddocmanual
%    \end{macrocode}

% \macro{\childdocname}
% \macro{\childdocjob}
% The macro |\childdocname| stores the name of the main document
% to be compiled. The macro |\childdocjob| stores the name of
% the document on which the \LaTeX{} compiler was originally invoked.
% The content of |\jobname| cannot be compared
% to filenames specified in the source due to different catcodes.
% The following code rescans |\jobname|, stores the result
% in |\childdocname| and saves a copy in |\childdocjob|:
%    \begin{macrocode}
\edef\childdocname{\scantokens\expandafter{\jobname\noexpand}}
\let\childdocjob\childdocname
%    \end{macrocode}

% \macro{\childdocdisable}
% The macro |\childdocdisable| prevents the main file
% from being processed more than once.
% At this stage, the main document command |\childdocmain|
% is assumed to be called once again where it should do nothing.
% Any subsequent call to it should prevent
% a secondary processing of the main document
% It overwrites the forwarding commands
% |\childdocof| and |\childdocforward|
% with empty macros to prevent further inclusions of the main document:
%    \begin{macrocode}
\newcommand{\childdocdisable}
{
  \renewcommand{\childdocmain}[1]{\renewcommand{\childdocmain}[1]{\endinput}}
  \renewcommand{\childdocof}[1]{}
  \renewcommand{\childdocby}[2][]{}
  \renewcommand{\childdocforward}[2][]{}
  \renewcommand{\childdocdisable}{}
}
%    \end{macrocode}

% \macro{\childdocmain}
% The macro |\childdocmain| is to be called at the top of the main file
% with nothing or the main filename (without extension) as argument.
% First, it breaks loops.
% If the argument is not empty and does not match |\childdocname|
% (which is set by the first inclusion of |childdoc.def|),
% |\ifchilddoc| is set to true, |\includeonly| is applied to the child file
% and |\jobname| is set to the main file
% (for proper handling of |.aux| files):
%    \begin{macrocode}
\newcommand{\childdocmain}[1]
{
  \childdocdisable\childdocmain{}
  \if?#1?\else
    \begingroup
      \def\childdoctmp{#1}
      \ifx\childdoctmp\childdocname
        \def\childdoctmp{}
      \else
        \def\childdoctmp
        {
          \childdoctrue
          \includeonly{\childdocname}
          \def\childdocjob{#1}
          \def\jobname{#1}
        }
      \fi
      \expandafter
    \endgroup
    \childdoctmp
  \fi
}
%    \end{macrocode}

% \macro{\childdocof}
% The command |\childdocof| redirects
% compilation to the main file |#1|.
%    \begin{macrocode}
\newcommand{\childdocof}[1]
{
  \childdocdisable
  \childdoctrue
  \includeonly{\childdocname}
  \def\jobname{#1}
  \def\childdocjob{#1}
  \input{#1}
}
%    \end{macrocode}

% \macro{\childdocby}
% The command |\childdocby| ....
%    \begin{macrocode}
\newcommand{\childdocby}[2][]
{
  \childdocdisable
  \childdoctrue
  \childdocmanualtrue
  \if?#1?\else
    \def\jobname{#2}
  \fi
  \def\childdocjob{#2}
  \input{#2}
  \endinput
}
%    \end{macrocode}

% \macro{\childdocforward}
% The command |\childdocforward| redirects
% compilation to the main file or
% (if the optional argument is given) a child file.
% Parameters are set as if the main file
% or a child file starting with |\childdocof| was compiled.
% Then compilation is handed over to the main file:
%    \begin{macrocode}
\newcommand{\childdocforward}[2][]
{
  \begingroup
    \if?#1?
      \def\childdoctmp
      {
        \def\childdocname{#2}
        \def\childdocjob{#2}
        \def\jobname{#2}
        \input{#2}
        \endinput
      }
    \else
      \def\childdoctmp
      {
        \childdocdisable
        \def\childdocname{#2}
        \childdoctrue
        \includeonly{#2}
        \def\childdocjob{#1}
        \def\jobname{#1}
        \input{#1}
        \endinput
      }
    \fi
    \expandafter
  \endgroup
  \childdoctmp
}
%    \end{macrocode}

% \macro{\childdocforwardprefix}
% The command |\childdocforwardprefix| redirects
% compilation to the main or a child file by means of a pattern.
% The prefix |#1| in the current filename is replaced by |#2|
% and the suffix of the current filename is kept
% (it is assumed that the filename does not contain the substring `|~~~|'
% which is used as a delimiter).
% Compilation is handed over to the new file by |\childdocforward|:
%    \begin{macrocode}
\newcommand{\childdocforwardprefix}[3][]
{
  \begingroup
    \def\childdocextract #2##1~~~{\def\childdoctmp{\childdocforward[#1]{#3##1}}}
    \expandafter\childdocextract\childdocname~~~
    \expandafter
  \endgroup
  \childdoctmp
}
%    \end{macrocode}

% \macro{\childdoc}
% The deprecated macro |\childdoc| is a legacy version of |\childdocmain|:
%    \begin{macrocode}
\newcommand{\childdoc}{\childdocmain}
%    \end{macrocode}

% \macro{\childdocredirect}
% The deprecated macro |\childdocredirect| is a legacy version
% of |\childdocforward| and |\childdocforwardprefix|:
%    \begin{macrocode}
\newcommand{\childdocredirect}[2][]
{
  \begingroup
    \if?#1?
      \def\childdoctmp{\childdocforward{#2}}
    \else
      \def\childdoctmp{\childdocforwardprefix{#1}{#2}}
    \fi
    \expandafter
  \endgroup
  \childdoctmp
}
%    \end{macrocode}

%\iffalse
%</package>
%\fi
%
\endinput
\childdocforward{cdocsch2}"|
% \end{tabular}
% \end{center}
% Note that the trailing backslash on each first line
% merely continues the input to the second line
% (for convenient cut ant paste).
% Furthermore, the command |latex| can be replaced by any
% of its alternative versions such as |pdflatex|.
%
% %%%%%%%%%%%%%%%%%%%%%%%%%%%%%%%%%%%%%%%%%%%%%%%%%%%%%%%%%%%%%%%%%%%%%%%%%%%%%%
% %%%%%%%%%%%%%%%%%%%%%%%%%%%%%%%%%%%%%%%%%%%%%%%%%%%%%%%%%%%%%%%%%%%%%%%%%%%%%%
% \section{Implementation}
%\iffalse
%<*package>
%\fi
%
% This section describes the definitions file |childdoc.def|.

% The definitions cannot be loaded using |\usepackage| or |\RequirePackage|
% which has a mechanism to prevent loading a style file more than once.
% When loading the definitions by means of |\input|
% multiple instances have to be prevented manually:
%\iffalse
%This code needs to be before the `\ProvidesFile' directive
%which is defined at the beginning of this file.
%Therefore it is also placed there and commented out here.
%</package>
%<*discard>
%\fi
%    \begin{macrocode}
\ifdefined\childdocmain\endinput\fi
%    \end{macrocode}
%\iffalse
%</discard>
%<*package>
%\fi
%
% \macro{\ifchilddoc}
% \macro{\ifchilddocmanual}
% The conditional |\ifchilddoc| tells whether a
% child (true) or main (false) document is being compiled.
% The conditional |\ifchilddocmanual| tells whether
% the |\includeonly| mechanism is used (false) or
% the selection of child files must be performed manually (true).
% The definitions initialise to false:
%    \begin{macrocode}
\newif\ifchilddoc
\newif\ifchilddocmanual
%    \end{macrocode}

% \macro{\childdocname}
% \macro{\childdocjob}
% The macro |\childdocname| stores the name of the main document
% to be compiled. The macro |\childdocjob| stores the name of
% the document on which the \LaTeX{} compiler was originally invoked.
% The content of |\jobname| cannot be compared
% to filenames specified in the source due to different catcodes.
% The following code rescans |\jobname|, stores the result
% in |\childdocname| and saves a copy in |\childdocjob|:
%    \begin{macrocode}
\edef\childdocname{\scantokens\expandafter{\jobname\noexpand}}
\let\childdocjob\childdocname
%    \end{macrocode}

% \macro{\childdocdisable}
% The macro |\childdocdisable| prevents the main file
% from being processed more than once.
% At this stage, the main document command |\childdocmain|
% is assumed to be called once again where it should do nothing.
% Any subsequent call to it should prevent
% a secondary processing of the main document
% It overwrites the forwarding commands
% |\childdocof| and |\childdocforward|
% with empty macros to prevent further inclusions of the main document:
%    \begin{macrocode}
\newcommand{\childdocdisable}
{
  \renewcommand{\childdocmain}[1]{\renewcommand{\childdocmain}[1]{\endinput}}
  \renewcommand{\childdocof}[1]{}
  \renewcommand{\childdocby}[2][]{}
  \renewcommand{\childdocforward}[2][]{}
  \renewcommand{\childdocdisable}{}
}
%    \end{macrocode}

% \macro{\childdocmain}
% The macro |\childdocmain| is to be called at the top of the main file
% with nothing or the main filename (without extension) as argument.
% First, it breaks loops.
% If the argument is not empty and does not match |\childdocname|
% (which is set by the first inclusion of |childdoc.def|),
% |\ifchilddoc| is set to true, |\includeonly| is applied to the child file
% and |\jobname| is set to the main file
% (for proper handling of |.aux| files):
%    \begin{macrocode}
\newcommand{\childdocmain}[1]
{
  \childdocdisable\childdocmain{}
  \if?#1?\else
    \begingroup
      \def\childdoctmp{#1}
      \ifx\childdoctmp\childdocname
        \def\childdoctmp{}
      \else
        \def\childdoctmp
        {
          \childdoctrue
          \includeonly{\childdocname}
          \def\childdocjob{#1}
          \def\jobname{#1}
        }
      \fi
      \expandafter
    \endgroup
    \childdoctmp
  \fi
}
%    \end{macrocode}

% \macro{\childdocof}
% The command |\childdocof| redirects
% compilation to the main file |#1|.
%    \begin{macrocode}
\newcommand{\childdocof}[1]
{
  \childdocdisable
  \childdoctrue
  \includeonly{\childdocname}
  \def\jobname{#1}
  \def\childdocjob{#1}
  \input{#1}
}
%    \end{macrocode}

% \macro{\childdocby}
% The command |\childdocby| ....
%    \begin{macrocode}
\newcommand{\childdocby}[2][]
{
  \childdocdisable
  \childdoctrue
  \childdocmanualtrue
  \if?#1?\else
    \def\jobname{#2}
  \fi
  \def\childdocjob{#2}
  \input{#2}
  \endinput
}
%    \end{macrocode}

% \macro{\childdocforward}
% The command |\childdocforward| redirects
% compilation to the main file or
% (if the optional argument is given) a child file.
% Parameters are set as if the main file
% or a child file starting with |\childdocof| was compiled.
% Then compilation is handed over to the main file:
%    \begin{macrocode}
\newcommand{\childdocforward}[2][]
{
  \begingroup
    \if?#1?
      \def\childdoctmp
      {
        \def\childdocname{#2}
        \def\childdocjob{#2}
        \def\jobname{#2}
        \input{#2}
        \endinput
      }
    \else
      \def\childdoctmp
      {
        \childdocdisable
        \def\childdocname{#2}
        \childdoctrue
        \includeonly{#2}
        \def\childdocjob{#1}
        \def\jobname{#1}
        \input{#1}
        \endinput
      }
    \fi
    \expandafter
  \endgroup
  \childdoctmp
}
%    \end{macrocode}

% \macro{\childdocforwardprefix}
% The command |\childdocforwardprefix| redirects
% compilation to the main or a child file by means of a pattern.
% The prefix |#1| in the current filename is replaced by |#2|
% and the suffix of the current filename is kept
% (it is assumed that the filename does not contain the substring `|~~~|'
% which is used as a delimiter).
% Compilation is handed over to the new file by |\childdocforward|:
%    \begin{macrocode}
\newcommand{\childdocforwardprefix}[3][]
{
  \begingroup
    \def\childdocextract #2##1~~~{\def\childdoctmp{\childdocforward[#1]{#3##1}}}
    \expandafter\childdocextract\childdocname~~~
    \expandafter
  \endgroup
  \childdoctmp
}
%    \end{macrocode}

% \macro{\childdoc}
% The deprecated macro |\childdoc| is a legacy version of |\childdocmain|:
%    \begin{macrocode}
\newcommand{\childdoc}{\childdocmain}
%    \end{macrocode}

% \macro{\childdocredirect}
% The deprecated macro |\childdocredirect| is a legacy version
% of |\childdocforward| and |\childdocforwardprefix|:
%    \begin{macrocode}
\newcommand{\childdocredirect}[2][]
{
  \begingroup
    \if?#1?
      \def\childdoctmp{\childdocforward{#2}}
    \else
      \def\childdoctmp{\childdocforwardprefix{#1}{#2}}
    \fi
    \expandafter
  \endgroup
  \childdoctmp
}
%    \end{macrocode}

%\iffalse
%</package>
%\fi
%
\endinput

\childdocforward{cdocsamp}
%    \end{macrocode}

%\iffalse
%</sampledraft>
%\fi
%
% %%%%%%%%%%%%%%%%%%%%%%%%%%%%%%%%%%%%%%
% \paragraph{Forwarding for Final Version of the Chapters.}
%
% The following forwarding files |cdocsfn1.tex| and |cdocsfn2.tex|
% (with identical content)
% compile the final versions of the child documents
% |cdocsch1.tex| and |cdocsch2.tex|, respectively:
%\iffalse
%<*samplefinal>
%\fi
%    \begin{macrocode}
\def\version{final}
% \iffalse
%
% childdoc.dtx Copyright (C) 2017-2018 Niklas Beisert
%
% This work may be distributed and/or modified under the
% conditions of the LaTeX Project Public License, either version 1.3
% of this license or (at your option) any later version.
% The latest version of this license is in
%   http://www.latex-project.org/lppl.txt
% and version 1.3 or later is part of all distributions of LaTeX
% version 2005/12/01 or later.
%
% This work has the LPPL maintenance status `maintained'.
%
% The Current Maintainer of this work is Niklas Beisert.
%
% This work consists of the files childdoc.dtx and childdoc.ins
% and the derived files childdoc.def and cdocsamp.tex with
% cdocsch1.tex, cdocsch2.tex, cdocsdrf.tex, cdocsfn1.tex, cdocsfn2.tex.
%
%<package>\ifdefined\childdocmain\endinput\fi
%<package>\ProvidesFile{childdoc.def}[2018/12/30 v2.0 child document driver]
%<samplemain>\ProvidesFile{cdocsamp.tex}[2018/12/30 v2.0 sample for childdoc]
%<*driver>
%\ProvidesFile{childdoc.drv}[2018/12/30 v2.0 childdoc reference manual file]
\PassOptionsToClass{10pt,a4paper}{article}
\documentclass{ltxdoc}

\usepackage[margin=35mm]{geometry}
\usepackage{hyperref}
\usepackage{hyperxmp}
\usepackage[usenames]{color}

\hypersetup{colorlinks=true}
\hypersetup{pdfstartview=FitH}
\hypersetup{pdfpagemode=UseNone}
\hypersetup{pdfsource={}}
\hypersetup{pdflang={en-UK}}
\hypersetup{pdfcopyright={Copyright 2017-2018 Niklas Beisert.
  This work may be distributed and/or modified under the
  conditions of the LaTeX Project Public License, either version 1.3
  of this license or (at your option) any later version.}}
\hypersetup{pdflicenseurl={http://www.latex-project.org/lppl.txt}}
\hypersetup{pdfcontactaddress={ETH Zurich, ITP, HIT K,
  Wolfgang-Pauli-Strasse 27}}
\hypersetup{pdfcontactpostcode={8093}}
\hypersetup{pdfcontactcity={Zurich}}
\hypersetup{pdfcontactcountry={Switzerland}}
\hypersetup{pdfcontactemail={nbeisert@itp.phys.ethz.ch}}
\hypersetup{pdfcontacturl={http://people.phys.ethz.ch/\xmptilde nbeisert/}}

\newcommand{\secref}[1]{\hyperref[#1]{section \ref*{#1}}}

\parskip1ex
\parindent0pt
\let\olditemize\itemize
\def\itemize{\olditemize\parskip0pt}

\begin{document}

\title{The \textsf{childdoc} Package}
\hypersetup{pdftitle={The childdoc Package}}
\author{Niklas Beisert\\[2ex]
  Institut f\"ur Theoretische Physik\\
  Eidgen\"ossische Technische Hochschule Z\"urich\\
  Wolfgang-Pauli-Strasse 27, 8093 Z\"urich, Switzerland\\[1ex]
  \href{mailto:nbeisert@itp.phys.ethz.ch}
  {\texttt{nbeisert@itp.phys.ethz.ch}}}
\hypersetup{pdfauthor={Niklas Beisert}}
\hypersetup{pdfsubject={Manual for the LaTeX2e Package childdoc}}
\date{30 December 2018, \textsf{v2.0}}
\maketitle

\begin{abstract}\noindent
\textsf{childdoc} is a \LaTeXe{} package
that enables the direct compilation
of document sections included by |\include|
to individual files.
\end{abstract}

\begingroup
\parskip0ex
\tableofcontents
\endgroup

%%%%%%%%%%%%%%%%%%%%%%%%%%%%%%%%%%%%%%%%%%%%%%%%%%%%%%%%%%%%%%%%%%%%%%%%%%%%%%%%
%%%%%%%%%%%%%%%%%%%%%%%%%%%%%%%%%%%%%%%%%%%%%%%%%%%%%%%%%%%%%%%%%%%%%%%%%%%%%%%%
\section{Introduction}

\LaTeX{} provides a mechanism to structure a large document (such as a book)
into a main file and several child files (containing the chapters)
using the |\include| command.
This mechanism is beneficial for documents
which span hundreds of pages in order to
make the source file(s) more manageable.
Moreover, compilation can be restricted to
selected child files by means of the |\includeonly| command.
The latter feature can be used to reduce the compilation time while editing
(this was significantly more useful in the earlier days of \LaTeX{})
or to generate a smaller document which is easier to navigate.
Another application of |\includeonly| is to generate
documents consisting of selected parts of the complete document.

However, there are a few drawbacks of the plain |\include| mechanism:
\begin{itemize}
\item
The child files cannot be compiled on their own,
they can only be compiled via the main file.
A naive editing environment
(such as a text editor with an option
to have the current file processed by \LaTeX)
may require one to switch to the main file before compiling;
attempting to compile the child file produces errors.
\item
The main file must be modified (each time)
to adjust the |\includeonly| command
to the present needs. This easily leaves the main file in a messy state.
\item
The generated document will always carry the filename
of the main document. This is inconvenient if
several child files are to be compiled and
to be kept for distribution.
\end{itemize}

The present package provides a simple interface
to make child files individually compilable by \LaTeX{}.
Compiling a child file then has the same effect as compiling
the main file with an |\includeonly| command
to select the appropriate child.
Moreover the generated document will carry the name of the child
rather than the main file.
This resolves all three above issues.

This feature is meant to make the editing of books,
thesis documents and lecture notes somewhat more convenient.
However, the package can also be used efficiently for
composing a series of documents (such as exercise sheets)
which are typically distributed individually.
It then assists the author in generating the individual documents
(potentially in different versions)
as well as a document containing the collected series.
Another application is in developing style files
or other kinds of included material
where compilation of the style file could redirect
to a sample or test file.

%%%%%%%%%%%%%%%%%%%%%%%%%%%%%%%%%%%%%%%%%%%%%%%%%%%%%%%%%%%%%%%%%%%%%%%%%%%%%%%%
%%%%%%%%%%%%%%%%%%%%%%%%%%%%%%%%%%%%%%%%%%%%%%%%%%%%%%%%%%%%%%%%%%%%%%%%%%%%%%%%
\section{Usage}

First of all, the package \textsf{childdoc} is \emph{not} a standard
\LaTeXe{} |.sty| style file! Therefore it needs to be invoked in
a non-standard way.

%%%%%%%%%%%%%%%%%%%%%%%%%%%%%%%%%%%%%%%%%%%%%%%%%%%%%%%%%%%%%%%%%%%%%%%%%%%%%%%%
\subsection{Included Files}
\label{sec:include}

%%%%%%%%%%%%%%%%%%%%%%%%%%%%%%%%%%%%%%%%
\DescribeMacro{\childdocmain}
To use the package, add the commands
\begin{center}
\begin{tabular}{l}
|% \iffalse
%
% childdoc.dtx Copyright (C) 2017-2018 Niklas Beisert
%
% This work may be distributed and/or modified under the
% conditions of the LaTeX Project Public License, either version 1.3
% of this license or (at your option) any later version.
% The latest version of this license is in
%   http://www.latex-project.org/lppl.txt
% and version 1.3 or later is part of all distributions of LaTeX
% version 2005/12/01 or later.
%
% This work has the LPPL maintenance status `maintained'.
%
% The Current Maintainer of this work is Niklas Beisert.
%
% This work consists of the files childdoc.dtx and childdoc.ins
% and the derived files childdoc.def and cdocsamp.tex with
% cdocsch1.tex, cdocsch2.tex, cdocsdrf.tex, cdocsfn1.tex, cdocsfn2.tex.
%
%<package>\ifdefined\childdocmain\endinput\fi
%<package>\ProvidesFile{childdoc.def}[2018/12/30 v2.0 child document driver]
%<samplemain>\ProvidesFile{cdocsamp.tex}[2018/12/30 v2.0 sample for childdoc]
%<*driver>
%\ProvidesFile{childdoc.drv}[2018/12/30 v2.0 childdoc reference manual file]
\PassOptionsToClass{10pt,a4paper}{article}
\documentclass{ltxdoc}

\usepackage[margin=35mm]{geometry}
\usepackage{hyperref}
\usepackage{hyperxmp}
\usepackage[usenames]{color}

\hypersetup{colorlinks=true}
\hypersetup{pdfstartview=FitH}
\hypersetup{pdfpagemode=UseNone}
\hypersetup{pdfsource={}}
\hypersetup{pdflang={en-UK}}
\hypersetup{pdfcopyright={Copyright 2017-2018 Niklas Beisert.
  This work may be distributed and/or modified under the
  conditions of the LaTeX Project Public License, either version 1.3
  of this license or (at your option) any later version.}}
\hypersetup{pdflicenseurl={http://www.latex-project.org/lppl.txt}}
\hypersetup{pdfcontactaddress={ETH Zurich, ITP, HIT K,
  Wolfgang-Pauli-Strasse 27}}
\hypersetup{pdfcontactpostcode={8093}}
\hypersetup{pdfcontactcity={Zurich}}
\hypersetup{pdfcontactcountry={Switzerland}}
\hypersetup{pdfcontactemail={nbeisert@itp.phys.ethz.ch}}
\hypersetup{pdfcontacturl={http://people.phys.ethz.ch/\xmptilde nbeisert/}}

\newcommand{\secref}[1]{\hyperref[#1]{section \ref*{#1}}}

\parskip1ex
\parindent0pt
\let\olditemize\itemize
\def\itemize{\olditemize\parskip0pt}

\begin{document}

\title{The \textsf{childdoc} Package}
\hypersetup{pdftitle={The childdoc Package}}
\author{Niklas Beisert\\[2ex]
  Institut f\"ur Theoretische Physik\\
  Eidgen\"ossische Technische Hochschule Z\"urich\\
  Wolfgang-Pauli-Strasse 27, 8093 Z\"urich, Switzerland\\[1ex]
  \href{mailto:nbeisert@itp.phys.ethz.ch}
  {\texttt{nbeisert@itp.phys.ethz.ch}}}
\hypersetup{pdfauthor={Niklas Beisert}}
\hypersetup{pdfsubject={Manual for the LaTeX2e Package childdoc}}
\date{30 December 2018, \textsf{v2.0}}
\maketitle

\begin{abstract}\noindent
\textsf{childdoc} is a \LaTeXe{} package
that enables the direct compilation
of document sections included by |\include|
to individual files.
\end{abstract}

\begingroup
\parskip0ex
\tableofcontents
\endgroup

%%%%%%%%%%%%%%%%%%%%%%%%%%%%%%%%%%%%%%%%%%%%%%%%%%%%%%%%%%%%%%%%%%%%%%%%%%%%%%%%
%%%%%%%%%%%%%%%%%%%%%%%%%%%%%%%%%%%%%%%%%%%%%%%%%%%%%%%%%%%%%%%%%%%%%%%%%%%%%%%%
\section{Introduction}

\LaTeX{} provides a mechanism to structure a large document (such as a book)
into a main file and several child files (containing the chapters)
using the |\include| command.
This mechanism is beneficial for documents
which span hundreds of pages in order to
make the source file(s) more manageable.
Moreover, compilation can be restricted to
selected child files by means of the |\includeonly| command.
The latter feature can be used to reduce the compilation time while editing
(this was significantly more useful in the earlier days of \LaTeX{})
or to generate a smaller document which is easier to navigate.
Another application of |\includeonly| is to generate
documents consisting of selected parts of the complete document.

However, there are a few drawbacks of the plain |\include| mechanism:
\begin{itemize}
\item
The child files cannot be compiled on their own,
they can only be compiled via the main file.
A naive editing environment
(such as a text editor with an option
to have the current file processed by \LaTeX)
may require one to switch to the main file before compiling;
attempting to compile the child file produces errors.
\item
The main file must be modified (each time)
to adjust the |\includeonly| command
to the present needs. This easily leaves the main file in a messy state.
\item
The generated document will always carry the filename
of the main document. This is inconvenient if
several child files are to be compiled and
to be kept for distribution.
\end{itemize}

The present package provides a simple interface
to make child files individually compilable by \LaTeX{}.
Compiling a child file then has the same effect as compiling
the main file with an |\includeonly| command
to select the appropriate child.
Moreover the generated document will carry the name of the child
rather than the main file.
This resolves all three above issues.

This feature is meant to make the editing of books,
thesis documents and lecture notes somewhat more convenient.
However, the package can also be used efficiently for
composing a series of documents (such as exercise sheets)
which are typically distributed individually.
It then assists the author in generating the individual documents
(potentially in different versions)
as well as a document containing the collected series.
Another application is in developing style files
or other kinds of included material
where compilation of the style file could redirect
to a sample or test file.

%%%%%%%%%%%%%%%%%%%%%%%%%%%%%%%%%%%%%%%%%%%%%%%%%%%%%%%%%%%%%%%%%%%%%%%%%%%%%%%%
%%%%%%%%%%%%%%%%%%%%%%%%%%%%%%%%%%%%%%%%%%%%%%%%%%%%%%%%%%%%%%%%%%%%%%%%%%%%%%%%
\section{Usage}

First of all, the package \textsf{childdoc} is \emph{not} a standard
\LaTeXe{} |.sty| style file! Therefore it needs to be invoked in
a non-standard way.

%%%%%%%%%%%%%%%%%%%%%%%%%%%%%%%%%%%%%%%%%%%%%%%%%%%%%%%%%%%%%%%%%%%%%%%%%%%%%%%%
\subsection{Included Files}
\label{sec:include}

%%%%%%%%%%%%%%%%%%%%%%%%%%%%%%%%%%%%%%%%
\DescribeMacro{\childdocmain}
To use the package, add the commands
\begin{center}
\begin{tabular}{l}
|\input{childdoc.def}|\\
|\childdocmain{}|\\
\end{tabular}
\end{center}
at the very top of the main \LaTeX{} file,
in particular \emph{before} the |\documentclass| statement!
The argument of |\childdocmain| should be left empty
(but it must be present).

%%%%%%%%%%%%%%%%%%%%%%%%%%%%%%%%%%%%%%%%
\DescribeMacro{\childdocof}
Furthermore, add the commands
\begin{center}
\begin{tabular}{l}
|\input{childdoc.def}|\\
|\childdocof{|\textit{main}|}|\\
\end{tabular}
\end{center}
at the top of every child file \textit{child}
which is included by |\include{|\textit{child}|}|
from within the main file
(or at least for those files to be compiled individually).
The argument \textit{main} must be the filename of the main file.

There are a couple of
considerations in setting up the main and child documents:

%%%%%%%%%%%%%%%%%%%%%%%%%%%%%%%%%%%%%%%%
\paragraph{Restrictions.}

Please note the following restrictions:
\begin{itemize}
\item
|\childdocmain| must be called with one argument \textit{main}
to ensure compatibility with earlier version of the package.
It must either be empty (|\childdocmain{}|)
or precisely match the filename of the main file in which it is specified.
See \secref{sec:detection} for further information.
\item
The filename \textit{main} must be specified without the |.tex| extension.
\item
The filename \textit{main} is case sensitive
(even in case-insensitive file systems)
due to internal string comparison.
\item
The argument \textit{main} should be fully expanded, it cannot be a macro.
\item
Subdirectories and special characters should be avoided in filenames.
\item
The command |\childdocmain{|\textit{main}|}| must be followed by a whitespace.
It should not be followed immediately by another command
or by a comment mark `|%|'.
This is because the \TeX{} parser reads the token immediately following
the argument of |\childdocmain| and puts it
at the beginning of every child section;
however, a white\-space is ignored.
\end{itemize}

%%%%%%%%%%%%%%%%%%%%%%%%%%%%%%%%%%%%%%%%
\paragraph{Content of Main File.}

It is advisable to place all content in the child files included by |\include|.
Any output contained in the main file will appear in all child documents
unless suppressed manually;
it cannot be suppressed automatically by the |\includeonly| directive
and thus should normally be avoided.
A method to include some content in the main file
by means of conditional processing is described in \secref{sec:conditional}.

%%%%%%%%%%%%%%%%%%%%%%%%%%%%%%%%%%%%%%%%
\paragraph{Page Numbering.}

When only a part of the document is compiled,
the appropriate numbering of pages
(as well as other status parameters)
is determined from the |.aux| files.
The latter contain information from previous passes.
However this information needs to propagate through
all intermediate child documents.
Therefore the page numbering in child documents may well
be inconsistent until the complete document is compiled at least once.

A useful (if unconventional) way to always ensure a consistent
page numbering is to restart the numbering in each child document
and denote the pages by `\textit{child}|.|\textit{page}'
where \textit{child} represents the chapter/section number of the child file.
This can be achieved by the command
|\numberwithin{page}{|\textit{child}|}|
of the \textsf{amsmath} package
where \textit{child} can be |chapter| or |section|
depending on the chosen structuring.
Alternatively, one can modify the macro |\thepage| appropriately
and reset the counter |page| at the start of each child file.

%%%%%%%%%%%%%%%%%%%%%%%%%%%%%%%%%%%%%%%%%%%%%%%%%%%%%%%%%%%%%%%%%%%%%%%%%%%%%%%%
\subsection{Conditional Processing}
\label{sec:conditional}

The package provides a mechanism to compile different versions
of a document. To customise the versions further some conditional processing
can come in handy to distinguish which version is being compiled.
The package provides two macros to describe the compilation context:

%%%%%%%%%%%%%%%%%%%%%%%%%%%%%%%%%%%%%%%%
\DescribeMacro{\ifchilddoc}
The conditional |\ifchilddoc| distinguishes between the compilation of
child documents and the main document:
%
\begin{center}
|\ifchilddoc |\textit{child-code}| |[|\||else |\textit{main-code}]| \||fi|
\end{center}

%%%%%%%%%%%%%%%%%%%%%%%%%%%%%%%%%%%%%%%%
\DescribeMacro{\childdocname}
\DescribeMacro{\childdocjob}
The macro |\childdocname| contains the filename (without extension)
of the main or child file being processed.
Note that |\childdocjob| will always contain the name of the main file.

%%%%%%%%%%%%%%%%%%%%%%%%%%%%%%%%%%%%%%%%
\paragraph{Title Page.}

Conditional processing can be used to include a title or banner page
in the main document when proper precautions are taken.
Importantly, the code in the main file should ensure that the page counter
(as well as other status parameters which are stored in the |.aux| files)
takes the same value after the conditional processing.
Otherwise the page numbers may take divergent values
depending on which part is compiled.

For example, a title page could be declared by:
%
\begin{center}
\begin{tabular}{l}
|\ifchilddoc\||else|\\
|\addtocounter{page}{-1}|\\
\textit{code for title page}\\
|\newpage|\\
|\||fi|
\end{tabular}
\end{center}
%
A banner page for the child documents can be generated by:
%
\begin{center}
\begin{tabular}{l}
|\ifchilddoc|\\
|\addtocounter{page}{-1}|\\
\textit{code for banner page}\\
|\newpage|\\
|\||fi|
\end{tabular}
\end{center}
%
Here one could write a message such as:
\begin{center}
|This is the part \childdocname{} of \childdocjob{}.|
\end{center}

%%%%%%%%%%%%%%%%%%%%%%%%%%%%%%%%%%%%%%%%%%%%%%%%%%%%%%%%%%%%%%%%%%%%%%%%%%%%%%%%
\subsection{Flags}
\label{sec:flags}

The package makes it easy to generate different versions
of the main or child documents.
To this end compilation flags can be defined
and assigned different default values.
They will be particularly useful in conjunction
with the forwarding mechanism described in \secref{sec:forward}.

For example, it may be useful to have a flag |\version|
which can be set to |draft| or |final|.
The document source will contain some conditional code
depending on the value of |\version|.
Suppose further, the flag should default to |final| for the main file
and to |draft| for child files
which is a natural assignment for editing the document.
This is achieved by placing the following code
in the preamble of the main document
(below the |\childdocmain| directive):
%
\begin{center}
\begin{tabular}{l}
|\ifchilddoc|\\
|\providecommand{\version}{draft}|\\
|\||else|\\
|\providecommand{\version}{final}|\\
|\||fi|
\end{tabular}
\end{center}
%
The definition by |\providecommand| makes sure
that previous definitions are not overwritten.
Further statements |\providecommand{\version}{...}|
can thus be added before the above code to override it.

For the main file, one might add a line
(between |\childdocmain| and the above block)
%
\begin{center}
|%\ifchilddoc\||else\providecommand{\version}{draft}\||fi|
\end{center}
%
which can be uncommented to produce a draft version.
Likewise one can add a line to the very top of a child file
(above the |\childdocof{|\textit{main}|}| directive)
%
\begin{center}
|%\providecommand{\version}{final}|
\end{center}
%
which can be uncommented to produce the final version of this child document.

%%%%%%%%%%%%%%%%%%%%%%%%%%%%%%%%%%%%%%%%%%%%%%%%%%%%%%%%%%%%%%%%%%%%%%%%%%%%%%%%
\subsection{Forwarding}
\label{sec:forward}

Different versions of the main or child documents
using compilation flags as described in \secref{sec:flags}
can be (permanently) stored in different files
for convenient compilation, viewing and distribution.
To this end, the package defines a command
to pass on compilation to a different file:

%%%%%%%%%%%%%%%%%%%%%%%%%%%%%%%%%%%%%%%%
\DescribeMacro{\childdocforward}
The command |\childdocforward| redirects processing to
another source file:
%
\begin{center}
\begin{tabular}{l}
|\input{childdoc.def}|\\
|\childdocforward[|\textit{main}|]{|\textit{dest}|}|\\
\end{tabular}
\end{center}
%
The argument \textit{dest} is the destination file
(without extension).
It should be the main file or one of the child files.
Note that further \textsf{childdoc} directives
such as |\childdocof| and |\childdocforward|
in the indicated file will be processed in this form.
The optional argument \textit{main}
passes on directly to the main file \textit{main}
while pretending to compile the child \textit{dest}.
This form behaves as if \textit{dest}
issues |\childdocof{|\textit{main}|}| right away,
and no further \textsf{childdoc} directives will be processed.

%%%%%%%%%%%%%%%%%%%%%%%%%%%%%%%%%%%%%%%%
\DescribeMacro{\...prefix}
In the alternative form |\childdocforwardprefix|,
%
\begin{center}
\begin{tabular}{l}
|\input{childdoc.def}|\\
|\childdocforwardprefix[|\textit{main}|]{|\textit{prefix}|}{|\textit{dest}|}|
\end{tabular}
\end{center}
%
the destination file is determined by a pattern
depending on the current file:
To make this work, the current file must be called
`{\textit{prefix}\hspace{0.2em}\textit{suffix}}'
with \textit{prefix} matching precisely the argument.
Processing is then passed on to the file
`{\textit{dest}\hspace{0.2em}\textit{suffix}}'.
Surely, the same effect is achieved by
directly specifying the
argument `{\textit{dest}\hspace{0.2em}\textit{suffix}}'
in the first form.
However, that requires to set up a different file
for each child. With the alternative form of the command
all these files can have exactly the same content
which simplifies setting them up and maintaining them.

For example, the following file |draft.tex|
with a compilation flag |\version| as described in \secref{sec:flags}
compiles the main document as a draft:
%
\begin{center}
\begin{tabular}{l}
|\def\version{draft}|\\
|\input{childdoc.def}|\\
|\childdocforward{|\textit{main}|}|
\end{tabular}
\end{center}
%
Likewise, the following files |final|\textit{nn}|.tex|
compile the final version of the child document
|child|\textit{nn}|.tex|:
%
\begin{center}
\begin{tabular}{l}
|\def\version{final}|\\
|\input{childdoc.def}|\\
|\childdocforwardprefix{final}{child}|
\end{tabular}
\end{center}
%

Note that when several versions of a main file and/or of each child file
are to be generated, it may be convenient to set up a |Makefile| or
shell script to automatise the process.

%%%%%%%%%%%%%%%%%%%%%%%%%%%%%%%%%%%%%%%%%%%%%%%%%%%%%%%%%%%%%%%%%%%%%%%%%%%%%%%%
\subsection{Command Line Processing}
\label{sec:commandline}

The effect of redirection files can also be achieved by invoking
the \LaTeX{} compiler with a more elaborate command line.
Most conveniently this should be done as part
of a shell script or a |Makefile|.

When using \textsf{childdoc} in the main file, the following
command lines effectively perform a redirection
(note that depending on the shell being used,
backslashes may have to be doubled: `|\|' $\to$ `|\\|'):
%
\begin{center}
|... -jobname "|\textit{target}|" |\\|"|[\textit{flags}]%
|\input{childdoc.def}\childdocforward[|\textit{main}|]{|\textit{dest}|}"|
\end{center}
%
Here \textit{target} is the name of the output file,
\textit{main} is the name of the main file
and \textit{dest} is the name of the main or child file to be processed
(all filenames without extensions).
The optional argument \textit{main} can be omitted
if \textit{main} matches \textit{dest}.
Optionally, compilation \textit{flags} can be defined via |\def| commands.
This command line makes the \TeX{} engine believe
it is compiling the file \textit{target}
whose content is specified as the latter parameter.
The provided code then forwards the processing to
\textit{main} or \textit{dest} as described in \secref{sec:forward}.

%%%%%%%%%%%%%%%%%%%%%%%%%%%%%%%%%%%%%%%%%%%%%%%%%%%%%%%%%%%%%%%%%%%%%%%%%%%%%%%%
\subsection{Include by Input}
\label{sec:input}

Including child documents by |\include| has some restrictions by design.
Most notably, the content of a child document always occupies
its own set of pages; pages cannot be shared between child documents.
Usually, this behaviour makes perfect sense
because each child document contain an essential part of the document.
However, in some situations it may be desirable to compose
a document from a collection of parts
without having mandatory page breaks between then.
For this case, the package
provides a mechanism to include parts
by |\input| which can also be processed individually.
However, by construction this mechanism
requires manual handling of the content to be output.

%%%%%%%%%%%%%%%%%%%%%%%%%%%%%%%%%%%%%%%%
\DescribeMacro{\ifchilddocmanual}
The main file should be prepared as usual, see \secref{sec:include}.
However, the document body must make a distinction
between processing of an individual part and of the main document, e.g.:
%
\begin{center}
\begin{tabular}{l}
|\ifchilddocmanual|\\
|\input{\childdocname}|\\
|\||else|\\
\textit{document body with }|\input{|\textit{part}|}|\\
|\||fi|
\end{tabular}
\end{center}
%
The conditional |\ifchilddocmanual| is true whenever
a part to be included by |\input| is being compiled,
and the name of the part is stored in |\childdocname|.

%%%%%%%%%%%%%%%%%%%%%%%%%%%%%%%%%%%%%%%%
\DescribeMacro{\childdocby}
Each part to be included by |\input| should start with:
%
\begin{center}
\begin{tabular}{l}
|\input{childdoc.def}|\\
|\childdocby{|\textit{main}|}|\\
\end{tabular}
\end{center}
%
The directive |\childdocby| is similar to |\childdocof|
described in \secref{sec:include},
but the subsequent selection of content must be done manually.
To that end, both |\ifchilddoc| and |\ifchilddocmanual|
will be true upon processing of a part,
and the name of the part is stored in |\childdocname|.
Note that |\jobname| will be set to the filename of the current part
so that each part receives an individual |.aux| file
that does not interfere with the |.aux| file(s) of the main document.
This behaviour can be altered by the alternative form
|\childdocby[*]{|\textit{main}|}| (with a non-empty optional argument)
which uses the |.aux| file of the main document
by setting |\jobname| to \textit{main}.

%%%%%%%%%%%%%%%%%%%%%%%%%%%%%%%%%%%%%%%%%%%%%%%%%%%%%%%%%%%%%%%%%%%%%%%%%%%%%%%%
\subsection{Driver Development}
\label{sec:driver}

The \textsf{childdoc} mechanism can also be use for the development
of definition files such as \LaTeX{} styles or classes.
This case differs from the above setup with multiple parts
included by |\include| in that no |\includeonly| should be invoked.
This can be achieved by starting the include file
(before |\ProvidesPackage|) with:
%
\begin{center}
\begin{tabular}{l}
|\input{childdoc.def}|\\
|\childdocforward{|\textit{main}|}|\\
\end{tabular}
\end{center}
%
or alternatively with:
%
\begin{center}
\begin{tabular}{l}
|\input{childdoc.def}|\\
|\childdocby{|\textit{main}|}|\\
\end{tabular}
\end{center}
%
Both forms have slightly different effects as described above.
The main file is prepared as usual, see \secref{sec:include}.

%%%%%%%%%%%%%%%%%%%%%%%%%%%%%%%%%%%%%%%%%%%%%%%%%%%%%%%%%%%%%%%%%%%%%%%%%%%%%%%%
\subsection{Legacy Detection}
\label{sec:detection}

The directive |\childdocmain| in the main file can detect
whether the complete document or merely a child is to be compiled
even without using the directive |\childdocof|.
This method is deprecated because it is less robust
and there is no compelling reason to use it;
it is merely provided for backward compatibility
and it may be removed in future versions.

If the detection mechanism is to be used,
it is mandatory to correctly specify
the filename of the main file as the argument of |\childdocmain|:
%
\begin{center}
\begin{tabular}{l}
|\input{childdoc.def}|\\
|\childdocmain{|\textit{main}|}|\\
\end{tabular}
\end{center}
%
If |\jobname| does not match the argument \textit{main} of |\childdocmain|,
it is assumed that |\jobname| points to the child file to be compiled.
When using |\childdocmain| with the main file specified as argument,
it suffices to start a child file
with just |\input{|\textit{main}|}|
without loading of the package and using |\childdocof|.
If instead all processing is done
with the appropriate \textsf{childdoc} directives,
the argument of \textit{main} of |\childdocmain| can be empty.

An alternative version of the command line processing described
in \secref{sec:commandline} using the detection mechanism reads:
%
\begin{center}
|... -jobname "|\textit{target}|" "|[\textit{flags}]%
[|\def\jobname{|\textit{dest}|}|]|\input{|\textit{main}|}"|
\end{center}

%%%%%%%%%%%%%%%%%%%%%%%%%%%%%%%%%%%%%%%%%%%%%%%%%%%%%%%%%%%%%%%%%%%%%%%%%%%%%%%%
\subsection{Manual Code}
\label{sec:manual}

In case one cannot be certain whether the definitions file |childdoc.def|
is installed on the target \TeX{} distribution
and one prefers not to ship it,
it is conceivable to paste a few relevant commands into the sources.

To that end, drop all statements |\input{childdoc.def}|
and perform the replacements as outlined below.
Instead of |\childdocmain{|\textit{main}|}| add the following code
to the top of the main file:
%
\begin{center}
\begin{tabular}{l}
|\||ifdefined\childdocname\endinput\||fi\newif\ifchilddoc|\\
|\edef\childdocname{\scantokens\expandafter{\jobname\noexpand}}|\\
|\def\childdocmain{|\textit{main}|}\||ifx\childdocmain\childdocname\||else|\\
|\childdoctrue\includeonly{\childdocname}\let\jobname\childdocmain\||fi|\\
\end{tabular}
\end{center}
%
Instead of |\childdocof{|\textit{main}|}| just include the main file
at the top of each child file:
%
\begin{center}
|\input{|\textit{main}|}|
\end{center}
%
A simple redirection |\childdocforward{|\textit{dest}|}| is achieved by:
%
\begin{center}
|\def\jobname{|\textit{dest}|}\input{\jobname}|
\end{center}
%
The redirection with prefix
|\childdocforwardprefix[|\textit{prefix}|]{|\textit{dest}|}|
is accomplished by:
%
\begin{center}
\begin{tabular}{l}
|{\edef\jobname{\scantokens\expandafter{\jobname\noexpand}}|\\
|\def\redirectjob |\textit{prefix}|#1~~~{\gdef\jobname{|\textit{dest}|#1}}|\\
|\expandafter\redirectjob\jobname~~~}\input{\jobname}|
\end{tabular}
\end{center}

In an alternative approach,
child documents can be compiled by a specific command line
without additional code or specific definitions:
%
\begin{center}
|... -jobname "|\textit{target}|" "|[\textit{flags}]%
|\includeonly{|\textit{dest}|}\input{|\textit{main}|}"|
\end{center}
%

%%%%%%%%%%%%%%%%%%%%%%%%%%%%%%%%%%%%%%%%%%%%%%%%%%%%%%%%%%%%%%%%%%%%%%%%%%%%%%%%
%%%%%%%%%%%%%%%%%%%%%%%%%%%%%%%%%%%%%%%%%%%%%%%%%%%%%%%%%%%%%%%%%%%%%%%%%%%%%%%%
\section{Information}

%%%%%%%%%%%%%%%%%%%%%%%%%%%%%%%%%%%%%%%%%%%%%%%%%%%%%%%%%%%%%%%%%%%%%%%%%%%%%%%%
\subsection{Copyright}

Copyright \copyright{} 2017--2018 Niklas Beisert

This work may be distributed and/or modified under the
conditions of the \LaTeX{} Project Public License, either version 1.3
of this license or (at your option) any later version.
The latest version of this license is in
  \url{http://www.latex-project.org/lppl.txt}
and version 1.3 or later is part of all distributions of \LaTeX{}
version 2005/12/01 or later.

This work has the LPPL maintenance status `maintained'.

The Current Maintainer of this work is Niklas Beisert.

This work consists of the files |README.txt|, |childdoc.ins| and |childdoc.dtx|
as well as the derived files |childdoc.def|, |cdocsamp.tex|
with |cdocsch1.tex|, |cdocsch2.tex|, |cdocspt3.tex|, |cdocspt4.tex|,
|cdocsdrf.tex|, |cdocsfn1.tex|, |cdocsfn2.tex|
as well as |childdoc.pdf|.

%%%%%%%%%%%%%%%%%%%%%%%%%%%%%%%%%%%%%%%%%%%%%%%%%%%%%%%%%%%%%%%%%%%%%%%%%%%%%%%%
\subsection{Files and Installation}

The package consists of the files:
%
\begin{center}
\begin{tabular}{ll}
    |README.txt|   & readme file \\
    |childdoc.ins| & installation file \\
    |childdoc.dtx| & source file \\
    |childdoc.def| & definition file \\
    |cdocsamp.tex| & sample main file \\
    |cdocsch1.tex| & sample include file \\
    |cdocsch2.tex| & sample include file \\
    |cdocspt3.tex| & sample part file \\
    |cdocspt4.tex| & sample part file \\
    |cdocsdrf.tex| & sample redirection file \\
    |cdocsfn1.tex| & sample redirection file \\
    |cdocsfn2.tex| & sample redirection file \\
    |childdoc.pdf| & manual
\end{tabular}
\end{center}
%
The distribution consists of the files
|README.txt|, |childdoc.ins| and |childdoc.dtx|.
%
\begin{itemize}
\item
Run (pdf)\LaTeX{} on |childdoc.dtx|
to compile the manual |childdoc.pdf| (this file).
\item
Run \LaTeX{} on |childdoc.ins| to create the definitions file |childdoc.def|
and the sample |cdocsamp.tex| with include files
|cdocsch1.tex|, |cdocsch2.tex|, |cdocspt3.tex|, |cdocspt4.tex|,
|cdocsdrf.tex|, |cdocsfn1.tex|, |cdocsfn2.tex|.
Then copy the file |childdoc.def| to an appropriate directory of your \LaTeX{}
distribution, e.g.\ \textit{texmf-root}|/tex/latex/childdoc|.
\end{itemize}

%%%%%%%%%%%%%%%%%%%%%%%%%%%%%%%%%%%%%%%%%%%%%%%%%%%%%%%%%%%%%%%%%%%%%%%%%%%%%%%%
\subsection{Related CTAN Packages}

There are several other packages which offer a similar functionality:
%
\begin{itemize}
\item
The packages
\href{http://ctan.org/pkg/docmute}{\textsf{docmute}},
\href{http://ctan.org/pkg/includex}{\textsf{includex}} and
\href{http://ctan.org/pkg/standalone}{\textsf{standalone}}
provide commands to include only the document body of
a child file thus allowing both files to be compiled individually.
\item
The packages \href{http://ctan.org/pkg/subdocs}{\textsf{subdocs}}
and \href{http://ctan.org/pkg/subfiles}{\textsf{subfiles}}
provide structures in which the main and child documents can be
encapsulated and allowing them to be compiled individually.
The inclusion mechanism is different from the conventional |\include|.
\item
The package \href{http://ctan.org/pkg/combine}{\textsf{combine}}
is an elaborate solution to combine several documents into one.
\end{itemize}
%
See also the CTAN topic \href{http://ctan.org/topic/subdocs}{\textsf{subdocs}}
for further related packages.
The present package differs from the above solutions in that
a document structure constructed with the conventional |\include| mechanism
just needs two extra commands at the top of every file
such that all constituent files can be compiled individually.

%%%%%%%%%%%%%%%%%%%%%%%%%%%%%%%%%%%%%%%%%%%%%%%%%%%%%%%%%%%%%%%%%%%%%%%%%%%%%%%%
%\subsection{Feature Suggestions}
%
%The following is a list of features which may be useful for future
%versions of this package:
%%
%\begin{itemize}
%\item
%\ldots
%\end{itemize}

%%%%%%%%%%%%%%%%%%%%%%%%%%%%%%%%%%%%%%%%%%%%%%%%%%%%%%%%%%%%%%%%%%%%%%%%%%%%%%%%
\subsection{Revision History}

%%%%%%%%%%%%%%%%%%%%%%%%%%%%%%%%%%%%%%%%
\paragraph{v2.0:} 2018/12/30

\begin{itemize}
\item
immediate forward processing
\item
added |\childdocby| mechanism
\item
manual restructured
\end{itemize}

%%%%%%%%%%%%%%%%%%%%%%%%%%%%%%%%%%%%%%%%
\paragraph{v1.6:} 2018/01/17

\begin{itemize}
\item
application for development of include files
\item
corrections to manual
\end{itemize}

%%%%%%%%%%%%%%%%%%%%%%%%%%%%%%%%%%%%%%%%
\paragraph{v1.5:} 2017/05/21

\begin{itemize}
\item
more complete structuring introduced
\item
|\childdocof| introduced
\item
|\childdoc| renamed to |\childdocmain|
\item
|\childredirect| renamed to |\childdocforward| and |\childdocforwardprefix|
and functionality expanded
\end{itemize}

%%%%%%%%%%%%%%%%%%%%%%%%%%%%%%%%%%%%%%%%
\paragraph{v1.0:} 2017/04/27

\begin{itemize}
\item
manual and install package
\item
first version published on CTAN
\end{itemize}

%%%%%%%%%%%%%%%%%%%%%%%%%%%%%%%%%%%%%%%%
\paragraph{v0.6:} 2017/04/26

\begin{itemize}
\item
redirection mechanism added
\end{itemize}

%%%%%%%%%%%%%%%%%%%%%%%%%%%%%%%%%%%%%%%%
\paragraph{v0.5:} 2017/04/26

\begin{itemize}
\item
functionality in definition file
\end{itemize}


%%%%%%%%%%%%%%%%%%%%%%%%%%%%%%%%%%%%%%%%%%%%%%%%%%%%%%%%%%%%%%%%%%%%%%%%%%%%%%%%
%%%%%%%%%%%%%%%%%%%%%%%%%%%%%%%%%%%%%%%%%%%%%%%%%%%%%%%%%%%%%%%%%%%%%%%%%%%%%%%%
%%%%%%%%%%%%%%%%%%%%%%%%%%%%%%%%%%%%%%%%%%%%%%%%%%%%%%%%%%%%%%%%%%%%%%%%%%%%%%%%
\appendix

\settowidth\MacroIndent{\rmfamily\scriptsize 000\ }

 \DocInput{childdoc.dtx}

\end{document}
%</driver>
% \fi
%
% %%%%%%%%%%%%%%%%%%%%%%%%%%%%%%%%%%%%%%%%%%%%%%%%%%%%%%%%%%%%%%%%%%%%%%%%%%%%%%
% %%%%%%%%%%%%%%%%%%%%%%%%%%%%%%%%%%%%%%%%%%%%%%%%%%%%%%%%%%%%%%%%%%%%%%%%%%%%%%
% \section{Sample}
%\iffalse
%<*samplemain>
%\fi
%
% The following presents a sample document
% with two chapters, two parts, a title page,
% a compile flag as well as three forwarding files to set the flag.
% It consists of eight |.tex| files:
% \begin{center}
% \begin{tabular}{ll}
% |cdocsamp.tex|&main file\\
% |cdocsch1.tex|&include file for chapter 1\\
% |cdocsch2.tex|&include file for chapter 2\\
% |cdocspt3.tex|&include file for part 3\\
% |cdocspt4.tex|&include file for part 4\\
% |cdocsdrf.tex|&forwarding file for main file in draft mode\\
% |cdocsfi1.tex|&forwarding file for final version of chapter 1\\
% |cdocsfi2.tex|&forwarding file for final version of chapter 2\\
% \end{tabular}
% \end{center}
% Each of the eight files can be compiled directly by the \LaTeX{} compiler.
%
% %%%%%%%%%%%%%%%%%%%%%%%%%%%%%%%%%%%%%%
% \paragraph{Main File.}
%
% The main file is called |cdocsamp.tex|.
%
% Load the \textsf{childdoc} definitions and
% declare the filename for the main document:
%    \begin{macrocode}
\input{childdoc.def}
\childdocmain{}
%    \end{macrocode}

% Optional override for |\version| flag:
%    \begin{macrocode}
%%\ifchilddoc\else\providecommand{\version}{draft}\fi
%    \end{macrocode}

% Define the default values for the |\version| flag
% (|final| for the main file and |draft| for childs):
%    \begin{macrocode}
\ifchilddoc
\providecommand{\version}{draft}
\else
\providecommand{\version}{final}
\fi
%    \end{macrocode}

% Load the standard document class:
%    \begin{macrocode}
\documentclass[12pt]{article}
%    \end{macrocode}

% Start the document body:
%    \begin{macrocode}
\begin{document}
%    \end{macrocode}

% Declare a title page.
% Print title, part of document being processed and version flag:
%    \begin{macrocode}
\addtocounter{page}{-1}
\begin{center}
{\LARGE\bfseries{}childdoc example\par}
\vspace{1cm}
\ifchilddoc
\ifchilddocmanual part\else chapter\fi:
`\childdocname' of `\childdocjob'\par
\else
main document: `\childdocjob'\par
\fi
version: \version\par
\end{center}
\newpage
%    \end{macrocode}

% Manually include selected file,
% otherwise process as usual:
%    \begin{macrocode}
\ifchilddocmanual
\section*{part `\childdocname'}
\input{\childdocname}
\else
%    \end{macrocode}

% Include the two chapters:
%    \begin{macrocode}
\include{cdocsch1}
\include{cdocsch2}
%    \end{macrocode}

% Include the two parts unless only chapters should be displayed:
%    \begin{macrocode}
\ifchilddoc\else
\section{part three}
\input{cdocspt3}
\section{part four}
\input{cdocspt4}
\fi
%    \end{macrocode}

% Process as usual until here:
%    \begin{macrocode}
\fi
%    \end{macrocode}

% End of document body:
%    \begin{macrocode}
\end{document}
%    \end{macrocode}
%\iffalse
%</samplemain>
%\fi
%
% %%%%%%%%%%%%%%%%%%%%%%%%%%%%%%%%%%%%%%
% \paragraph{Chapter Include Files.}
%
% The include files are called |cdocsch1.tex| and |cdocsch2.tex|.
%
%\iffalse
%<*samplechap1|samplechap2>
%\fi

% Optional override for |\version| flag:
%    \begin{macrocode}
%%\providecommand{\version}{final}
%    \end{macrocode}

% Include the main document:
%    \begin{macrocode}
\input{childdoc.def}
\childdocof{cdocsamp}
%    \end{macrocode}

%\iffalse
%</samplechap1|samplechap2>
%\fi
%
%\iffalse
%<*samplechap1>
%\fi
% Some text for chapter 1:
%    \begin{macrocode}
\section{one}
some text in chapter one
%    \end{macrocode}

%\iffalse
%</samplechap1>
%\fi
% Some text for chapter 2:
%\iffalse
%<*samplechap2>
%\fi
%    \begin{macrocode}
\section{two}
more text in chapter two
%    \end{macrocode}

%\iffalse
%</samplechap2>
%\fi
%
% %%%%%%%%%%%%%%%%%%%%%%%%%%%%%%%%%%%%%%
% \paragraph{Part Include Files.}
%
% The include files are called |cdocspt3.tex| and |cdocspt4.tex|.
%
%\iffalse
%<*samplepart3|samplepart4>
%\fi

% Optional override for |\version| flag:
%    \begin{macrocode}
%%\providecommand{\version}{final}
%    \end{macrocode}

% Include the main document:
%    \begin{macrocode}
\input{childdoc.def}
\childdocby{cdocsamp}
%    \end{macrocode}

%\iffalse
%</samplepart3|samplepart4>
%\fi
%
%\iffalse
%<*samplepart3>
%\fi
% Some text for part 3:
%    \begin{macrocode}
some text in part three
%    \end{macrocode}

%\iffalse
%</samplepart3>
%\fi
% Some text for part 4:
%\iffalse
%<*samplepart4>
%\fi
%    \begin{macrocode}
more text in part four
%    \end{macrocode}

%\iffalse
%</samplepart4>
%\fi
%
% %%%%%%%%%%%%%%%%%%%%%%%%%%%%%%%%%%%%%%
% \paragraph{Forwarding for a Complete Draft.}
%
% The following forwarding file |cdocsdrf.tex|
% compiles the main document in draft mode:
%\iffalse
%<*sampledraft>
%\fi
%    \begin{macrocode}
\def\version{draft}
\input{childdoc.def}
\childdocforward{cdocsamp}
%    \end{macrocode}

%\iffalse
%</sampledraft>
%\fi
%
% %%%%%%%%%%%%%%%%%%%%%%%%%%%%%%%%%%%%%%
% \paragraph{Forwarding for Final Version of the Chapters.}
%
% The following forwarding files |cdocsfn1.tex| and |cdocsfn2.tex|
% (with identical content)
% compile the final versions of the child documents
% |cdocsch1.tex| and |cdocsch2.tex|, respectively:
%\iffalse
%<*samplefinal>
%\fi
%    \begin{macrocode}
\def\version{final}
\input{childdoc.def}
\childdocforwardprefix[cdocsamp]{cdocsfn}{cdocsch}
%    \end{macrocode}

%\iffalse
%</samplefinal>
%\fi
%
% %%%%%%%%%%%%%%%%%%%%%%%%%%%%%%%%%%%%%%
% \paragraph{Command Line Processing.}
%
% The following three command lines generate the output files
% |cdocscld|, |cdocscl1| and |cdocscl2|
% which should be identical to
% |cdocsdrf|, |cdocsch1| and |cdocsfn2|, respectively:
% \begin{center}
% \begin{tabular}{l}
% |latex -jobname cdocscld \|\\
% |  "\def\version{draft}\input{childdoc.def}\childdocforward{cdocsamp}"|\\
% |latex -jobname cdocscl1 \|\\
% |  "\input{childdoc.def}\childdocforward[cdocsamp]{cdocsch1}"|\\
% |latex -jobname cdocscl2 \|\\
% |  "\def\version{final}\input{childdoc.def}\childdocforward{cdocsch2}"|
% \end{tabular}
% \end{center}
% Note that the trailing backslash on each first line
% merely continues the input to the second line
% (for convenient cut ant paste).
% Furthermore, the command |latex| can be replaced by any
% of its alternative versions such as |pdflatex|.
%
% %%%%%%%%%%%%%%%%%%%%%%%%%%%%%%%%%%%%%%%%%%%%%%%%%%%%%%%%%%%%%%%%%%%%%%%%%%%%%%
% %%%%%%%%%%%%%%%%%%%%%%%%%%%%%%%%%%%%%%%%%%%%%%%%%%%%%%%%%%%%%%%%%%%%%%%%%%%%%%
% \section{Implementation}
%\iffalse
%<*package>
%\fi
%
% This section describes the definitions file |childdoc.def|.

% The definitions cannot be loaded using |\usepackage| or |\RequirePackage|
% which has a mechanism to prevent loading a style file more than once.
% When loading the definitions by means of |\input|
% multiple instances have to be prevented manually:
%\iffalse
%This code needs to be before the `\ProvidesFile' directive
%which is defined at the beginning of this file.
%Therefore it is also placed there and commented out here.
%</package>
%<*discard>
%\fi
%    \begin{macrocode}
\ifdefined\childdocmain\endinput\fi
%    \end{macrocode}
%\iffalse
%</discard>
%<*package>
%\fi
%
% \macro{\ifchilddoc}
% \macro{\ifchilddocmanual}
% The conditional |\ifchilddoc| tells whether a
% child (true) or main (false) document is being compiled.
% The conditional |\ifchilddocmanual| tells whether
% the |\includeonly| mechanism is used (false) or
% the selection of child files must be performed manually (true).
% The definitions initialise to false:
%    \begin{macrocode}
\newif\ifchilddoc
\newif\ifchilddocmanual
%    \end{macrocode}

% \macro{\childdocname}
% \macro{\childdocjob}
% The macro |\childdocname| stores the name of the main document
% to be compiled. The macro |\childdocjob| stores the name of
% the document on which the \LaTeX{} compiler was originally invoked.
% The content of |\jobname| cannot be compared
% to filenames specified in the source due to different catcodes.
% The following code rescans |\jobname|, stores the result
% in |\childdocname| and saves a copy in |\childdocjob|:
%    \begin{macrocode}
\edef\childdocname{\scantokens\expandafter{\jobname\noexpand}}
\let\childdocjob\childdocname
%    \end{macrocode}

% \macro{\childdocdisable}
% The macro |\childdocdisable| prevents the main file
% from being processed more than once.
% At this stage, the main document command |\childdocmain|
% is assumed to be called once again where it should do nothing.
% Any subsequent call to it should prevent
% a secondary processing of the main document
% It overwrites the forwarding commands
% |\childdocof| and |\childdocforward|
% with empty macros to prevent further inclusions of the main document:
%    \begin{macrocode}
\newcommand{\childdocdisable}
{
  \renewcommand{\childdocmain}[1]{\renewcommand{\childdocmain}[1]{\endinput}}
  \renewcommand{\childdocof}[1]{}
  \renewcommand{\childdocby}[2][]{}
  \renewcommand{\childdocforward}[2][]{}
  \renewcommand{\childdocdisable}{}
}
%    \end{macrocode}

% \macro{\childdocmain}
% The macro |\childdocmain| is to be called at the top of the main file
% with nothing or the main filename (without extension) as argument.
% First, it breaks loops.
% If the argument is not empty and does not match |\childdocname|
% (which is set by the first inclusion of |childdoc.def|),
% |\ifchilddoc| is set to true, |\includeonly| is applied to the child file
% and |\jobname| is set to the main file
% (for proper handling of |.aux| files):
%    \begin{macrocode}
\newcommand{\childdocmain}[1]
{
  \childdocdisable\childdocmain{}
  \if?#1?\else
    \begingroup
      \def\childdoctmp{#1}
      \ifx\childdoctmp\childdocname
        \def\childdoctmp{}
      \else
        \def\childdoctmp
        {
          \childdoctrue
          \includeonly{\childdocname}
          \def\childdocjob{#1}
          \def\jobname{#1}
        }
      \fi
      \expandafter
    \endgroup
    \childdoctmp
  \fi
}
%    \end{macrocode}

% \macro{\childdocof}
% The command |\childdocof| redirects
% compilation to the main file |#1|.
%    \begin{macrocode}
\newcommand{\childdocof}[1]
{
  \childdocdisable
  \childdoctrue
  \includeonly{\childdocname}
  \def\jobname{#1}
  \def\childdocjob{#1}
  \input{#1}
}
%    \end{macrocode}

% \macro{\childdocby}
% The command |\childdocby| ....
%    \begin{macrocode}
\newcommand{\childdocby}[2][]
{
  \childdocdisable
  \childdoctrue
  \childdocmanualtrue
  \if?#1?\else
    \def\jobname{#2}
  \fi
  \def\childdocjob{#2}
  \input{#2}
  \endinput
}
%    \end{macrocode}

% \macro{\childdocforward}
% The command |\childdocforward| redirects
% compilation to the main file or
% (if the optional argument is given) a child file.
% Parameters are set as if the main file
% or a child file starting with |\childdocof| was compiled.
% Then compilation is handed over to the main file:
%    \begin{macrocode}
\newcommand{\childdocforward}[2][]
{
  \begingroup
    \if?#1?
      \def\childdoctmp
      {
        \def\childdocname{#2}
        \def\childdocjob{#2}
        \def\jobname{#2}
        \input{#2}
        \endinput
      }
    \else
      \def\childdoctmp
      {
        \childdocdisable
        \def\childdocname{#2}
        \childdoctrue
        \includeonly{#2}
        \def\childdocjob{#1}
        \def\jobname{#1}
        \input{#1}
        \endinput
      }
    \fi
    \expandafter
  \endgroup
  \childdoctmp
}
%    \end{macrocode}

% \macro{\childdocforwardprefix}
% The command |\childdocforwardprefix| redirects
% compilation to the main or a child file by means of a pattern.
% The prefix |#1| in the current filename is replaced by |#2|
% and the suffix of the current filename is kept
% (it is assumed that the filename does not contain the substring `|~~~|'
% which is used as a delimiter).
% Compilation is handed over to the new file by |\childdocforward|:
%    \begin{macrocode}
\newcommand{\childdocforwardprefix}[3][]
{
  \begingroup
    \def\childdocextract #2##1~~~{\def\childdoctmp{\childdocforward[#1]{#3##1}}}
    \expandafter\childdocextract\childdocname~~~
    \expandafter
  \endgroup
  \childdoctmp
}
%    \end{macrocode}

% \macro{\childdoc}
% The deprecated macro |\childdoc| is a legacy version of |\childdocmain|:
%    \begin{macrocode}
\newcommand{\childdoc}{\childdocmain}
%    \end{macrocode}

% \macro{\childdocredirect}
% The deprecated macro |\childdocredirect| is a legacy version
% of |\childdocforward| and |\childdocforwardprefix|:
%    \begin{macrocode}
\newcommand{\childdocredirect}[2][]
{
  \begingroup
    \if?#1?
      \def\childdoctmp{\childdocforward{#2}}
    \else
      \def\childdoctmp{\childdocforwardprefix{#1}{#2}}
    \fi
    \expandafter
  \endgroup
  \childdoctmp
}
%    \end{macrocode}

%\iffalse
%</package>
%\fi
%
\endinput
|\\
|\childdocmain{}|\\
\end{tabular}
\end{center}
at the very top of the main \LaTeX{} file,
in particular \emph{before} the |\documentclass| statement!
The argument of |\childdocmain| should be left empty
(but it must be present).

%%%%%%%%%%%%%%%%%%%%%%%%%%%%%%%%%%%%%%%%
\DescribeMacro{\childdocof}
Furthermore, add the commands
\begin{center}
\begin{tabular}{l}
|% \iffalse
%
% childdoc.dtx Copyright (C) 2017-2018 Niklas Beisert
%
% This work may be distributed and/or modified under the
% conditions of the LaTeX Project Public License, either version 1.3
% of this license or (at your option) any later version.
% The latest version of this license is in
%   http://www.latex-project.org/lppl.txt
% and version 1.3 or later is part of all distributions of LaTeX
% version 2005/12/01 or later.
%
% This work has the LPPL maintenance status `maintained'.
%
% The Current Maintainer of this work is Niklas Beisert.
%
% This work consists of the files childdoc.dtx and childdoc.ins
% and the derived files childdoc.def and cdocsamp.tex with
% cdocsch1.tex, cdocsch2.tex, cdocsdrf.tex, cdocsfn1.tex, cdocsfn2.tex.
%
%<package>\ifdefined\childdocmain\endinput\fi
%<package>\ProvidesFile{childdoc.def}[2018/12/30 v2.0 child document driver]
%<samplemain>\ProvidesFile{cdocsamp.tex}[2018/12/30 v2.0 sample for childdoc]
%<*driver>
%\ProvidesFile{childdoc.drv}[2018/12/30 v2.0 childdoc reference manual file]
\PassOptionsToClass{10pt,a4paper}{article}
\documentclass{ltxdoc}

\usepackage[margin=35mm]{geometry}
\usepackage{hyperref}
\usepackage{hyperxmp}
\usepackage[usenames]{color}

\hypersetup{colorlinks=true}
\hypersetup{pdfstartview=FitH}
\hypersetup{pdfpagemode=UseNone}
\hypersetup{pdfsource={}}
\hypersetup{pdflang={en-UK}}
\hypersetup{pdfcopyright={Copyright 2017-2018 Niklas Beisert.
  This work may be distributed and/or modified under the
  conditions of the LaTeX Project Public License, either version 1.3
  of this license or (at your option) any later version.}}
\hypersetup{pdflicenseurl={http://www.latex-project.org/lppl.txt}}
\hypersetup{pdfcontactaddress={ETH Zurich, ITP, HIT K,
  Wolfgang-Pauli-Strasse 27}}
\hypersetup{pdfcontactpostcode={8093}}
\hypersetup{pdfcontactcity={Zurich}}
\hypersetup{pdfcontactcountry={Switzerland}}
\hypersetup{pdfcontactemail={nbeisert@itp.phys.ethz.ch}}
\hypersetup{pdfcontacturl={http://people.phys.ethz.ch/\xmptilde nbeisert/}}

\newcommand{\secref}[1]{\hyperref[#1]{section \ref*{#1}}}

\parskip1ex
\parindent0pt
\let\olditemize\itemize
\def\itemize{\olditemize\parskip0pt}

\begin{document}

\title{The \textsf{childdoc} Package}
\hypersetup{pdftitle={The childdoc Package}}
\author{Niklas Beisert\\[2ex]
  Institut f\"ur Theoretische Physik\\
  Eidgen\"ossische Technische Hochschule Z\"urich\\
  Wolfgang-Pauli-Strasse 27, 8093 Z\"urich, Switzerland\\[1ex]
  \href{mailto:nbeisert@itp.phys.ethz.ch}
  {\texttt{nbeisert@itp.phys.ethz.ch}}}
\hypersetup{pdfauthor={Niklas Beisert}}
\hypersetup{pdfsubject={Manual for the LaTeX2e Package childdoc}}
\date{30 December 2018, \textsf{v2.0}}
\maketitle

\begin{abstract}\noindent
\textsf{childdoc} is a \LaTeXe{} package
that enables the direct compilation
of document sections included by |\include|
to individual files.
\end{abstract}

\begingroup
\parskip0ex
\tableofcontents
\endgroup

%%%%%%%%%%%%%%%%%%%%%%%%%%%%%%%%%%%%%%%%%%%%%%%%%%%%%%%%%%%%%%%%%%%%%%%%%%%%%%%%
%%%%%%%%%%%%%%%%%%%%%%%%%%%%%%%%%%%%%%%%%%%%%%%%%%%%%%%%%%%%%%%%%%%%%%%%%%%%%%%%
\section{Introduction}

\LaTeX{} provides a mechanism to structure a large document (such as a book)
into a main file and several child files (containing the chapters)
using the |\include| command.
This mechanism is beneficial for documents
which span hundreds of pages in order to
make the source file(s) more manageable.
Moreover, compilation can be restricted to
selected child files by means of the |\includeonly| command.
The latter feature can be used to reduce the compilation time while editing
(this was significantly more useful in the earlier days of \LaTeX{})
or to generate a smaller document which is easier to navigate.
Another application of |\includeonly| is to generate
documents consisting of selected parts of the complete document.

However, there are a few drawbacks of the plain |\include| mechanism:
\begin{itemize}
\item
The child files cannot be compiled on their own,
they can only be compiled via the main file.
A naive editing environment
(such as a text editor with an option
to have the current file processed by \LaTeX)
may require one to switch to the main file before compiling;
attempting to compile the child file produces errors.
\item
The main file must be modified (each time)
to adjust the |\includeonly| command
to the present needs. This easily leaves the main file in a messy state.
\item
The generated document will always carry the filename
of the main document. This is inconvenient if
several child files are to be compiled and
to be kept for distribution.
\end{itemize}

The present package provides a simple interface
to make child files individually compilable by \LaTeX{}.
Compiling a child file then has the same effect as compiling
the main file with an |\includeonly| command
to select the appropriate child.
Moreover the generated document will carry the name of the child
rather than the main file.
This resolves all three above issues.

This feature is meant to make the editing of books,
thesis documents and lecture notes somewhat more convenient.
However, the package can also be used efficiently for
composing a series of documents (such as exercise sheets)
which are typically distributed individually.
It then assists the author in generating the individual documents
(potentially in different versions)
as well as a document containing the collected series.
Another application is in developing style files
or other kinds of included material
where compilation of the style file could redirect
to a sample or test file.

%%%%%%%%%%%%%%%%%%%%%%%%%%%%%%%%%%%%%%%%%%%%%%%%%%%%%%%%%%%%%%%%%%%%%%%%%%%%%%%%
%%%%%%%%%%%%%%%%%%%%%%%%%%%%%%%%%%%%%%%%%%%%%%%%%%%%%%%%%%%%%%%%%%%%%%%%%%%%%%%%
\section{Usage}

First of all, the package \textsf{childdoc} is \emph{not} a standard
\LaTeXe{} |.sty| style file! Therefore it needs to be invoked in
a non-standard way.

%%%%%%%%%%%%%%%%%%%%%%%%%%%%%%%%%%%%%%%%%%%%%%%%%%%%%%%%%%%%%%%%%%%%%%%%%%%%%%%%
\subsection{Included Files}
\label{sec:include}

%%%%%%%%%%%%%%%%%%%%%%%%%%%%%%%%%%%%%%%%
\DescribeMacro{\childdocmain}
To use the package, add the commands
\begin{center}
\begin{tabular}{l}
|\input{childdoc.def}|\\
|\childdocmain{}|\\
\end{tabular}
\end{center}
at the very top of the main \LaTeX{} file,
in particular \emph{before} the |\documentclass| statement!
The argument of |\childdocmain| should be left empty
(but it must be present).

%%%%%%%%%%%%%%%%%%%%%%%%%%%%%%%%%%%%%%%%
\DescribeMacro{\childdocof}
Furthermore, add the commands
\begin{center}
\begin{tabular}{l}
|\input{childdoc.def}|\\
|\childdocof{|\textit{main}|}|\\
\end{tabular}
\end{center}
at the top of every child file \textit{child}
which is included by |\include{|\textit{child}|}|
from within the main file
(or at least for those files to be compiled individually).
The argument \textit{main} must be the filename of the main file.

There are a couple of
considerations in setting up the main and child documents:

%%%%%%%%%%%%%%%%%%%%%%%%%%%%%%%%%%%%%%%%
\paragraph{Restrictions.}

Please note the following restrictions:
\begin{itemize}
\item
|\childdocmain| must be called with one argument \textit{main}
to ensure compatibility with earlier version of the package.
It must either be empty (|\childdocmain{}|)
or precisely match the filename of the main file in which it is specified.
See \secref{sec:detection} for further information.
\item
The filename \textit{main} must be specified without the |.tex| extension.
\item
The filename \textit{main} is case sensitive
(even in case-insensitive file systems)
due to internal string comparison.
\item
The argument \textit{main} should be fully expanded, it cannot be a macro.
\item
Subdirectories and special characters should be avoided in filenames.
\item
The command |\childdocmain{|\textit{main}|}| must be followed by a whitespace.
It should not be followed immediately by another command
or by a comment mark `|%|'.
This is because the \TeX{} parser reads the token immediately following
the argument of |\childdocmain| and puts it
at the beginning of every child section;
however, a white\-space is ignored.
\end{itemize}

%%%%%%%%%%%%%%%%%%%%%%%%%%%%%%%%%%%%%%%%
\paragraph{Content of Main File.}

It is advisable to place all content in the child files included by |\include|.
Any output contained in the main file will appear in all child documents
unless suppressed manually;
it cannot be suppressed automatically by the |\includeonly| directive
and thus should normally be avoided.
A method to include some content in the main file
by means of conditional processing is described in \secref{sec:conditional}.

%%%%%%%%%%%%%%%%%%%%%%%%%%%%%%%%%%%%%%%%
\paragraph{Page Numbering.}

When only a part of the document is compiled,
the appropriate numbering of pages
(as well as other status parameters)
is determined from the |.aux| files.
The latter contain information from previous passes.
However this information needs to propagate through
all intermediate child documents.
Therefore the page numbering in child documents may well
be inconsistent until the complete document is compiled at least once.

A useful (if unconventional) way to always ensure a consistent
page numbering is to restart the numbering in each child document
and denote the pages by `\textit{child}|.|\textit{page}'
where \textit{child} represents the chapter/section number of the child file.
This can be achieved by the command
|\numberwithin{page}{|\textit{child}|}|
of the \textsf{amsmath} package
where \textit{child} can be |chapter| or |section|
depending on the chosen structuring.
Alternatively, one can modify the macro |\thepage| appropriately
and reset the counter |page| at the start of each child file.

%%%%%%%%%%%%%%%%%%%%%%%%%%%%%%%%%%%%%%%%%%%%%%%%%%%%%%%%%%%%%%%%%%%%%%%%%%%%%%%%
\subsection{Conditional Processing}
\label{sec:conditional}

The package provides a mechanism to compile different versions
of a document. To customise the versions further some conditional processing
can come in handy to distinguish which version is being compiled.
The package provides two macros to describe the compilation context:

%%%%%%%%%%%%%%%%%%%%%%%%%%%%%%%%%%%%%%%%
\DescribeMacro{\ifchilddoc}
The conditional |\ifchilddoc| distinguishes between the compilation of
child documents and the main document:
%
\begin{center}
|\ifchilddoc |\textit{child-code}| |[|\||else |\textit{main-code}]| \||fi|
\end{center}

%%%%%%%%%%%%%%%%%%%%%%%%%%%%%%%%%%%%%%%%
\DescribeMacro{\childdocname}
\DescribeMacro{\childdocjob}
The macro |\childdocname| contains the filename (without extension)
of the main or child file being processed.
Note that |\childdocjob| will always contain the name of the main file.

%%%%%%%%%%%%%%%%%%%%%%%%%%%%%%%%%%%%%%%%
\paragraph{Title Page.}

Conditional processing can be used to include a title or banner page
in the main document when proper precautions are taken.
Importantly, the code in the main file should ensure that the page counter
(as well as other status parameters which are stored in the |.aux| files)
takes the same value after the conditional processing.
Otherwise the page numbers may take divergent values
depending on which part is compiled.

For example, a title page could be declared by:
%
\begin{center}
\begin{tabular}{l}
|\ifchilddoc\||else|\\
|\addtocounter{page}{-1}|\\
\textit{code for title page}\\
|\newpage|\\
|\||fi|
\end{tabular}
\end{center}
%
A banner page for the child documents can be generated by:
%
\begin{center}
\begin{tabular}{l}
|\ifchilddoc|\\
|\addtocounter{page}{-1}|\\
\textit{code for banner page}\\
|\newpage|\\
|\||fi|
\end{tabular}
\end{center}
%
Here one could write a message such as:
\begin{center}
|This is the part \childdocname{} of \childdocjob{}.|
\end{center}

%%%%%%%%%%%%%%%%%%%%%%%%%%%%%%%%%%%%%%%%%%%%%%%%%%%%%%%%%%%%%%%%%%%%%%%%%%%%%%%%
\subsection{Flags}
\label{sec:flags}

The package makes it easy to generate different versions
of the main or child documents.
To this end compilation flags can be defined
and assigned different default values.
They will be particularly useful in conjunction
with the forwarding mechanism described in \secref{sec:forward}.

For example, it may be useful to have a flag |\version|
which can be set to |draft| or |final|.
The document source will contain some conditional code
depending on the value of |\version|.
Suppose further, the flag should default to |final| for the main file
and to |draft| for child files
which is a natural assignment for editing the document.
This is achieved by placing the following code
in the preamble of the main document
(below the |\childdocmain| directive):
%
\begin{center}
\begin{tabular}{l}
|\ifchilddoc|\\
|\providecommand{\version}{draft}|\\
|\||else|\\
|\providecommand{\version}{final}|\\
|\||fi|
\end{tabular}
\end{center}
%
The definition by |\providecommand| makes sure
that previous definitions are not overwritten.
Further statements |\providecommand{\version}{...}|
can thus be added before the above code to override it.

For the main file, one might add a line
(between |\childdocmain| and the above block)
%
\begin{center}
|%\ifchilddoc\||else\providecommand{\version}{draft}\||fi|
\end{center}
%
which can be uncommented to produce a draft version.
Likewise one can add a line to the very top of a child file
(above the |\childdocof{|\textit{main}|}| directive)
%
\begin{center}
|%\providecommand{\version}{final}|
\end{center}
%
which can be uncommented to produce the final version of this child document.

%%%%%%%%%%%%%%%%%%%%%%%%%%%%%%%%%%%%%%%%%%%%%%%%%%%%%%%%%%%%%%%%%%%%%%%%%%%%%%%%
\subsection{Forwarding}
\label{sec:forward}

Different versions of the main or child documents
using compilation flags as described in \secref{sec:flags}
can be (permanently) stored in different files
for convenient compilation, viewing and distribution.
To this end, the package defines a command
to pass on compilation to a different file:

%%%%%%%%%%%%%%%%%%%%%%%%%%%%%%%%%%%%%%%%
\DescribeMacro{\childdocforward}
The command |\childdocforward| redirects processing to
another source file:
%
\begin{center}
\begin{tabular}{l}
|\input{childdoc.def}|\\
|\childdocforward[|\textit{main}|]{|\textit{dest}|}|\\
\end{tabular}
\end{center}
%
The argument \textit{dest} is the destination file
(without extension).
It should be the main file or one of the child files.
Note that further \textsf{childdoc} directives
such as |\childdocof| and |\childdocforward|
in the indicated file will be processed in this form.
The optional argument \textit{main}
passes on directly to the main file \textit{main}
while pretending to compile the child \textit{dest}.
This form behaves as if \textit{dest}
issues |\childdocof{|\textit{main}|}| right away,
and no further \textsf{childdoc} directives will be processed.

%%%%%%%%%%%%%%%%%%%%%%%%%%%%%%%%%%%%%%%%
\DescribeMacro{\...prefix}
In the alternative form |\childdocforwardprefix|,
%
\begin{center}
\begin{tabular}{l}
|\input{childdoc.def}|\\
|\childdocforwardprefix[|\textit{main}|]{|\textit{prefix}|}{|\textit{dest}|}|
\end{tabular}
\end{center}
%
the destination file is determined by a pattern
depending on the current file:
To make this work, the current file must be called
`{\textit{prefix}\hspace{0.2em}\textit{suffix}}'
with \textit{prefix} matching precisely the argument.
Processing is then passed on to the file
`{\textit{dest}\hspace{0.2em}\textit{suffix}}'.
Surely, the same effect is achieved by
directly specifying the
argument `{\textit{dest}\hspace{0.2em}\textit{suffix}}'
in the first form.
However, that requires to set up a different file
for each child. With the alternative form of the command
all these files can have exactly the same content
which simplifies setting them up and maintaining them.

For example, the following file |draft.tex|
with a compilation flag |\version| as described in \secref{sec:flags}
compiles the main document as a draft:
%
\begin{center}
\begin{tabular}{l}
|\def\version{draft}|\\
|\input{childdoc.def}|\\
|\childdocforward{|\textit{main}|}|
\end{tabular}
\end{center}
%
Likewise, the following files |final|\textit{nn}|.tex|
compile the final version of the child document
|child|\textit{nn}|.tex|:
%
\begin{center}
\begin{tabular}{l}
|\def\version{final}|\\
|\input{childdoc.def}|\\
|\childdocforwardprefix{final}{child}|
\end{tabular}
\end{center}
%

Note that when several versions of a main file and/or of each child file
are to be generated, it may be convenient to set up a |Makefile| or
shell script to automatise the process.

%%%%%%%%%%%%%%%%%%%%%%%%%%%%%%%%%%%%%%%%%%%%%%%%%%%%%%%%%%%%%%%%%%%%%%%%%%%%%%%%
\subsection{Command Line Processing}
\label{sec:commandline}

The effect of redirection files can also be achieved by invoking
the \LaTeX{} compiler with a more elaborate command line.
Most conveniently this should be done as part
of a shell script or a |Makefile|.

When using \textsf{childdoc} in the main file, the following
command lines effectively perform a redirection
(note that depending on the shell being used,
backslashes may have to be doubled: `|\|' $\to$ `|\\|'):
%
\begin{center}
|... -jobname "|\textit{target}|" |\\|"|[\textit{flags}]%
|\input{childdoc.def}\childdocforward[|\textit{main}|]{|\textit{dest}|}"|
\end{center}
%
Here \textit{target} is the name of the output file,
\textit{main} is the name of the main file
and \textit{dest} is the name of the main or child file to be processed
(all filenames without extensions).
The optional argument \textit{main} can be omitted
if \textit{main} matches \textit{dest}.
Optionally, compilation \textit{flags} can be defined via |\def| commands.
This command line makes the \TeX{} engine believe
it is compiling the file \textit{target}
whose content is specified as the latter parameter.
The provided code then forwards the processing to
\textit{main} or \textit{dest} as described in \secref{sec:forward}.

%%%%%%%%%%%%%%%%%%%%%%%%%%%%%%%%%%%%%%%%%%%%%%%%%%%%%%%%%%%%%%%%%%%%%%%%%%%%%%%%
\subsection{Include by Input}
\label{sec:input}

Including child documents by |\include| has some restrictions by design.
Most notably, the content of a child document always occupies
its own set of pages; pages cannot be shared between child documents.
Usually, this behaviour makes perfect sense
because each child document contain an essential part of the document.
However, in some situations it may be desirable to compose
a document from a collection of parts
without having mandatory page breaks between then.
For this case, the package
provides a mechanism to include parts
by |\input| which can also be processed individually.
However, by construction this mechanism
requires manual handling of the content to be output.

%%%%%%%%%%%%%%%%%%%%%%%%%%%%%%%%%%%%%%%%
\DescribeMacro{\ifchilddocmanual}
The main file should be prepared as usual, see \secref{sec:include}.
However, the document body must make a distinction
between processing of an individual part and of the main document, e.g.:
%
\begin{center}
\begin{tabular}{l}
|\ifchilddocmanual|\\
|\input{\childdocname}|\\
|\||else|\\
\textit{document body with }|\input{|\textit{part}|}|\\
|\||fi|
\end{tabular}
\end{center}
%
The conditional |\ifchilddocmanual| is true whenever
a part to be included by |\input| is being compiled,
and the name of the part is stored in |\childdocname|.

%%%%%%%%%%%%%%%%%%%%%%%%%%%%%%%%%%%%%%%%
\DescribeMacro{\childdocby}
Each part to be included by |\input| should start with:
%
\begin{center}
\begin{tabular}{l}
|\input{childdoc.def}|\\
|\childdocby{|\textit{main}|}|\\
\end{tabular}
\end{center}
%
The directive |\childdocby| is similar to |\childdocof|
described in \secref{sec:include},
but the subsequent selection of content must be done manually.
To that end, both |\ifchilddoc| and |\ifchilddocmanual|
will be true upon processing of a part,
and the name of the part is stored in |\childdocname|.
Note that |\jobname| will be set to the filename of the current part
so that each part receives an individual |.aux| file
that does not interfere with the |.aux| file(s) of the main document.
This behaviour can be altered by the alternative form
|\childdocby[*]{|\textit{main}|}| (with a non-empty optional argument)
which uses the |.aux| file of the main document
by setting |\jobname| to \textit{main}.

%%%%%%%%%%%%%%%%%%%%%%%%%%%%%%%%%%%%%%%%%%%%%%%%%%%%%%%%%%%%%%%%%%%%%%%%%%%%%%%%
\subsection{Driver Development}
\label{sec:driver}

The \textsf{childdoc} mechanism can also be use for the development
of definition files such as \LaTeX{} styles or classes.
This case differs from the above setup with multiple parts
included by |\include| in that no |\includeonly| should be invoked.
This can be achieved by starting the include file
(before |\ProvidesPackage|) with:
%
\begin{center}
\begin{tabular}{l}
|\input{childdoc.def}|\\
|\childdocforward{|\textit{main}|}|\\
\end{tabular}
\end{center}
%
or alternatively with:
%
\begin{center}
\begin{tabular}{l}
|\input{childdoc.def}|\\
|\childdocby{|\textit{main}|}|\\
\end{tabular}
\end{center}
%
Both forms have slightly different effects as described above.
The main file is prepared as usual, see \secref{sec:include}.

%%%%%%%%%%%%%%%%%%%%%%%%%%%%%%%%%%%%%%%%%%%%%%%%%%%%%%%%%%%%%%%%%%%%%%%%%%%%%%%%
\subsection{Legacy Detection}
\label{sec:detection}

The directive |\childdocmain| in the main file can detect
whether the complete document or merely a child is to be compiled
even without using the directive |\childdocof|.
This method is deprecated because it is less robust
and there is no compelling reason to use it;
it is merely provided for backward compatibility
and it may be removed in future versions.

If the detection mechanism is to be used,
it is mandatory to correctly specify
the filename of the main file as the argument of |\childdocmain|:
%
\begin{center}
\begin{tabular}{l}
|\input{childdoc.def}|\\
|\childdocmain{|\textit{main}|}|\\
\end{tabular}
\end{center}
%
If |\jobname| does not match the argument \textit{main} of |\childdocmain|,
it is assumed that |\jobname| points to the child file to be compiled.
When using |\childdocmain| with the main file specified as argument,
it suffices to start a child file
with just |\input{|\textit{main}|}|
without loading of the package and using |\childdocof|.
If instead all processing is done
with the appropriate \textsf{childdoc} directives,
the argument of \textit{main} of |\childdocmain| can be empty.

An alternative version of the command line processing described
in \secref{sec:commandline} using the detection mechanism reads:
%
\begin{center}
|... -jobname "|\textit{target}|" "|[\textit{flags}]%
[|\def\jobname{|\textit{dest}|}|]|\input{|\textit{main}|}"|
\end{center}

%%%%%%%%%%%%%%%%%%%%%%%%%%%%%%%%%%%%%%%%%%%%%%%%%%%%%%%%%%%%%%%%%%%%%%%%%%%%%%%%
\subsection{Manual Code}
\label{sec:manual}

In case one cannot be certain whether the definitions file |childdoc.def|
is installed on the target \TeX{} distribution
and one prefers not to ship it,
it is conceivable to paste a few relevant commands into the sources.

To that end, drop all statements |\input{childdoc.def}|
and perform the replacements as outlined below.
Instead of |\childdocmain{|\textit{main}|}| add the following code
to the top of the main file:
%
\begin{center}
\begin{tabular}{l}
|\||ifdefined\childdocname\endinput\||fi\newif\ifchilddoc|\\
|\edef\childdocname{\scantokens\expandafter{\jobname\noexpand}}|\\
|\def\childdocmain{|\textit{main}|}\||ifx\childdocmain\childdocname\||else|\\
|\childdoctrue\includeonly{\childdocname}\let\jobname\childdocmain\||fi|\\
\end{tabular}
\end{center}
%
Instead of |\childdocof{|\textit{main}|}| just include the main file
at the top of each child file:
%
\begin{center}
|\input{|\textit{main}|}|
\end{center}
%
A simple redirection |\childdocforward{|\textit{dest}|}| is achieved by:
%
\begin{center}
|\def\jobname{|\textit{dest}|}\input{\jobname}|
\end{center}
%
The redirection with prefix
|\childdocforwardprefix[|\textit{prefix}|]{|\textit{dest}|}|
is accomplished by:
%
\begin{center}
\begin{tabular}{l}
|{\edef\jobname{\scantokens\expandafter{\jobname\noexpand}}|\\
|\def\redirectjob |\textit{prefix}|#1~~~{\gdef\jobname{|\textit{dest}|#1}}|\\
|\expandafter\redirectjob\jobname~~~}\input{\jobname}|
\end{tabular}
\end{center}

In an alternative approach,
child documents can be compiled by a specific command line
without additional code or specific definitions:
%
\begin{center}
|... -jobname "|\textit{target}|" "|[\textit{flags}]%
|\includeonly{|\textit{dest}|}\input{|\textit{main}|}"|
\end{center}
%

%%%%%%%%%%%%%%%%%%%%%%%%%%%%%%%%%%%%%%%%%%%%%%%%%%%%%%%%%%%%%%%%%%%%%%%%%%%%%%%%
%%%%%%%%%%%%%%%%%%%%%%%%%%%%%%%%%%%%%%%%%%%%%%%%%%%%%%%%%%%%%%%%%%%%%%%%%%%%%%%%
\section{Information}

%%%%%%%%%%%%%%%%%%%%%%%%%%%%%%%%%%%%%%%%%%%%%%%%%%%%%%%%%%%%%%%%%%%%%%%%%%%%%%%%
\subsection{Copyright}

Copyright \copyright{} 2017--2018 Niklas Beisert

This work may be distributed and/or modified under the
conditions of the \LaTeX{} Project Public License, either version 1.3
of this license or (at your option) any later version.
The latest version of this license is in
  \url{http://www.latex-project.org/lppl.txt}
and version 1.3 or later is part of all distributions of \LaTeX{}
version 2005/12/01 or later.

This work has the LPPL maintenance status `maintained'.

The Current Maintainer of this work is Niklas Beisert.

This work consists of the files |README.txt|, |childdoc.ins| and |childdoc.dtx|
as well as the derived files |childdoc.def|, |cdocsamp.tex|
with |cdocsch1.tex|, |cdocsch2.tex|, |cdocspt3.tex|, |cdocspt4.tex|,
|cdocsdrf.tex|, |cdocsfn1.tex|, |cdocsfn2.tex|
as well as |childdoc.pdf|.

%%%%%%%%%%%%%%%%%%%%%%%%%%%%%%%%%%%%%%%%%%%%%%%%%%%%%%%%%%%%%%%%%%%%%%%%%%%%%%%%
\subsection{Files and Installation}

The package consists of the files:
%
\begin{center}
\begin{tabular}{ll}
    |README.txt|   & readme file \\
    |childdoc.ins| & installation file \\
    |childdoc.dtx| & source file \\
    |childdoc.def| & definition file \\
    |cdocsamp.tex| & sample main file \\
    |cdocsch1.tex| & sample include file \\
    |cdocsch2.tex| & sample include file \\
    |cdocspt3.tex| & sample part file \\
    |cdocspt4.tex| & sample part file \\
    |cdocsdrf.tex| & sample redirection file \\
    |cdocsfn1.tex| & sample redirection file \\
    |cdocsfn2.tex| & sample redirection file \\
    |childdoc.pdf| & manual
\end{tabular}
\end{center}
%
The distribution consists of the files
|README.txt|, |childdoc.ins| and |childdoc.dtx|.
%
\begin{itemize}
\item
Run (pdf)\LaTeX{} on |childdoc.dtx|
to compile the manual |childdoc.pdf| (this file).
\item
Run \LaTeX{} on |childdoc.ins| to create the definitions file |childdoc.def|
and the sample |cdocsamp.tex| with include files
|cdocsch1.tex|, |cdocsch2.tex|, |cdocspt3.tex|, |cdocspt4.tex|,
|cdocsdrf.tex|, |cdocsfn1.tex|, |cdocsfn2.tex|.
Then copy the file |childdoc.def| to an appropriate directory of your \LaTeX{}
distribution, e.g.\ \textit{texmf-root}|/tex/latex/childdoc|.
\end{itemize}

%%%%%%%%%%%%%%%%%%%%%%%%%%%%%%%%%%%%%%%%%%%%%%%%%%%%%%%%%%%%%%%%%%%%%%%%%%%%%%%%
\subsection{Related CTAN Packages}

There are several other packages which offer a similar functionality:
%
\begin{itemize}
\item
The packages
\href{http://ctan.org/pkg/docmute}{\textsf{docmute}},
\href{http://ctan.org/pkg/includex}{\textsf{includex}} and
\href{http://ctan.org/pkg/standalone}{\textsf{standalone}}
provide commands to include only the document body of
a child file thus allowing both files to be compiled individually.
\item
The packages \href{http://ctan.org/pkg/subdocs}{\textsf{subdocs}}
and \href{http://ctan.org/pkg/subfiles}{\textsf{subfiles}}
provide structures in which the main and child documents can be
encapsulated and allowing them to be compiled individually.
The inclusion mechanism is different from the conventional |\include|.
\item
The package \href{http://ctan.org/pkg/combine}{\textsf{combine}}
is an elaborate solution to combine several documents into one.
\end{itemize}
%
See also the CTAN topic \href{http://ctan.org/topic/subdocs}{\textsf{subdocs}}
for further related packages.
The present package differs from the above solutions in that
a document structure constructed with the conventional |\include| mechanism
just needs two extra commands at the top of every file
such that all constituent files can be compiled individually.

%%%%%%%%%%%%%%%%%%%%%%%%%%%%%%%%%%%%%%%%%%%%%%%%%%%%%%%%%%%%%%%%%%%%%%%%%%%%%%%%
%\subsection{Feature Suggestions}
%
%The following is a list of features which may be useful for future
%versions of this package:
%%
%\begin{itemize}
%\item
%\ldots
%\end{itemize}

%%%%%%%%%%%%%%%%%%%%%%%%%%%%%%%%%%%%%%%%%%%%%%%%%%%%%%%%%%%%%%%%%%%%%%%%%%%%%%%%
\subsection{Revision History}

%%%%%%%%%%%%%%%%%%%%%%%%%%%%%%%%%%%%%%%%
\paragraph{v2.0:} 2018/12/30

\begin{itemize}
\item
immediate forward processing
\item
added |\childdocby| mechanism
\item
manual restructured
\end{itemize}

%%%%%%%%%%%%%%%%%%%%%%%%%%%%%%%%%%%%%%%%
\paragraph{v1.6:} 2018/01/17

\begin{itemize}
\item
application for development of include files
\item
corrections to manual
\end{itemize}

%%%%%%%%%%%%%%%%%%%%%%%%%%%%%%%%%%%%%%%%
\paragraph{v1.5:} 2017/05/21

\begin{itemize}
\item
more complete structuring introduced
\item
|\childdocof| introduced
\item
|\childdoc| renamed to |\childdocmain|
\item
|\childredirect| renamed to |\childdocforward| and |\childdocforwardprefix|
and functionality expanded
\end{itemize}

%%%%%%%%%%%%%%%%%%%%%%%%%%%%%%%%%%%%%%%%
\paragraph{v1.0:} 2017/04/27

\begin{itemize}
\item
manual and install package
\item
first version published on CTAN
\end{itemize}

%%%%%%%%%%%%%%%%%%%%%%%%%%%%%%%%%%%%%%%%
\paragraph{v0.6:} 2017/04/26

\begin{itemize}
\item
redirection mechanism added
\end{itemize}

%%%%%%%%%%%%%%%%%%%%%%%%%%%%%%%%%%%%%%%%
\paragraph{v0.5:} 2017/04/26

\begin{itemize}
\item
functionality in definition file
\end{itemize}


%%%%%%%%%%%%%%%%%%%%%%%%%%%%%%%%%%%%%%%%%%%%%%%%%%%%%%%%%%%%%%%%%%%%%%%%%%%%%%%%
%%%%%%%%%%%%%%%%%%%%%%%%%%%%%%%%%%%%%%%%%%%%%%%%%%%%%%%%%%%%%%%%%%%%%%%%%%%%%%%%
%%%%%%%%%%%%%%%%%%%%%%%%%%%%%%%%%%%%%%%%%%%%%%%%%%%%%%%%%%%%%%%%%%%%%%%%%%%%%%%%
\appendix

\settowidth\MacroIndent{\rmfamily\scriptsize 000\ }

 \DocInput{childdoc.dtx}

\end{document}
%</driver>
% \fi
%
% %%%%%%%%%%%%%%%%%%%%%%%%%%%%%%%%%%%%%%%%%%%%%%%%%%%%%%%%%%%%%%%%%%%%%%%%%%%%%%
% %%%%%%%%%%%%%%%%%%%%%%%%%%%%%%%%%%%%%%%%%%%%%%%%%%%%%%%%%%%%%%%%%%%%%%%%%%%%%%
% \section{Sample}
%\iffalse
%<*samplemain>
%\fi
%
% The following presents a sample document
% with two chapters, two parts, a title page,
% a compile flag as well as three forwarding files to set the flag.
% It consists of eight |.tex| files:
% \begin{center}
% \begin{tabular}{ll}
% |cdocsamp.tex|&main file\\
% |cdocsch1.tex|&include file for chapter 1\\
% |cdocsch2.tex|&include file for chapter 2\\
% |cdocspt3.tex|&include file for part 3\\
% |cdocspt4.tex|&include file for part 4\\
% |cdocsdrf.tex|&forwarding file for main file in draft mode\\
% |cdocsfi1.tex|&forwarding file for final version of chapter 1\\
% |cdocsfi2.tex|&forwarding file for final version of chapter 2\\
% \end{tabular}
% \end{center}
% Each of the eight files can be compiled directly by the \LaTeX{} compiler.
%
% %%%%%%%%%%%%%%%%%%%%%%%%%%%%%%%%%%%%%%
% \paragraph{Main File.}
%
% The main file is called |cdocsamp.tex|.
%
% Load the \textsf{childdoc} definitions and
% declare the filename for the main document:
%    \begin{macrocode}
\input{childdoc.def}
\childdocmain{}
%    \end{macrocode}

% Optional override for |\version| flag:
%    \begin{macrocode}
%%\ifchilddoc\else\providecommand{\version}{draft}\fi
%    \end{macrocode}

% Define the default values for the |\version| flag
% (|final| for the main file and |draft| for childs):
%    \begin{macrocode}
\ifchilddoc
\providecommand{\version}{draft}
\else
\providecommand{\version}{final}
\fi
%    \end{macrocode}

% Load the standard document class:
%    \begin{macrocode}
\documentclass[12pt]{article}
%    \end{macrocode}

% Start the document body:
%    \begin{macrocode}
\begin{document}
%    \end{macrocode}

% Declare a title page.
% Print title, part of document being processed and version flag:
%    \begin{macrocode}
\addtocounter{page}{-1}
\begin{center}
{\LARGE\bfseries{}childdoc example\par}
\vspace{1cm}
\ifchilddoc
\ifchilddocmanual part\else chapter\fi:
`\childdocname' of `\childdocjob'\par
\else
main document: `\childdocjob'\par
\fi
version: \version\par
\end{center}
\newpage
%    \end{macrocode}

% Manually include selected file,
% otherwise process as usual:
%    \begin{macrocode}
\ifchilddocmanual
\section*{part `\childdocname'}
\input{\childdocname}
\else
%    \end{macrocode}

% Include the two chapters:
%    \begin{macrocode}
\include{cdocsch1}
\include{cdocsch2}
%    \end{macrocode}

% Include the two parts unless only chapters should be displayed:
%    \begin{macrocode}
\ifchilddoc\else
\section{part three}
\input{cdocspt3}
\section{part four}
\input{cdocspt4}
\fi
%    \end{macrocode}

% Process as usual until here:
%    \begin{macrocode}
\fi
%    \end{macrocode}

% End of document body:
%    \begin{macrocode}
\end{document}
%    \end{macrocode}
%\iffalse
%</samplemain>
%\fi
%
% %%%%%%%%%%%%%%%%%%%%%%%%%%%%%%%%%%%%%%
% \paragraph{Chapter Include Files.}
%
% The include files are called |cdocsch1.tex| and |cdocsch2.tex|.
%
%\iffalse
%<*samplechap1|samplechap2>
%\fi

% Optional override for |\version| flag:
%    \begin{macrocode}
%%\providecommand{\version}{final}
%    \end{macrocode}

% Include the main document:
%    \begin{macrocode}
\input{childdoc.def}
\childdocof{cdocsamp}
%    \end{macrocode}

%\iffalse
%</samplechap1|samplechap2>
%\fi
%
%\iffalse
%<*samplechap1>
%\fi
% Some text for chapter 1:
%    \begin{macrocode}
\section{one}
some text in chapter one
%    \end{macrocode}

%\iffalse
%</samplechap1>
%\fi
% Some text for chapter 2:
%\iffalse
%<*samplechap2>
%\fi
%    \begin{macrocode}
\section{two}
more text in chapter two
%    \end{macrocode}

%\iffalse
%</samplechap2>
%\fi
%
% %%%%%%%%%%%%%%%%%%%%%%%%%%%%%%%%%%%%%%
% \paragraph{Part Include Files.}
%
% The include files are called |cdocspt3.tex| and |cdocspt4.tex|.
%
%\iffalse
%<*samplepart3|samplepart4>
%\fi

% Optional override for |\version| flag:
%    \begin{macrocode}
%%\providecommand{\version}{final}
%    \end{macrocode}

% Include the main document:
%    \begin{macrocode}
\input{childdoc.def}
\childdocby{cdocsamp}
%    \end{macrocode}

%\iffalse
%</samplepart3|samplepart4>
%\fi
%
%\iffalse
%<*samplepart3>
%\fi
% Some text for part 3:
%    \begin{macrocode}
some text in part three
%    \end{macrocode}

%\iffalse
%</samplepart3>
%\fi
% Some text for part 4:
%\iffalse
%<*samplepart4>
%\fi
%    \begin{macrocode}
more text in part four
%    \end{macrocode}

%\iffalse
%</samplepart4>
%\fi
%
% %%%%%%%%%%%%%%%%%%%%%%%%%%%%%%%%%%%%%%
% \paragraph{Forwarding for a Complete Draft.}
%
% The following forwarding file |cdocsdrf.tex|
% compiles the main document in draft mode:
%\iffalse
%<*sampledraft>
%\fi
%    \begin{macrocode}
\def\version{draft}
\input{childdoc.def}
\childdocforward{cdocsamp}
%    \end{macrocode}

%\iffalse
%</sampledraft>
%\fi
%
% %%%%%%%%%%%%%%%%%%%%%%%%%%%%%%%%%%%%%%
% \paragraph{Forwarding for Final Version of the Chapters.}
%
% The following forwarding files |cdocsfn1.tex| and |cdocsfn2.tex|
% (with identical content)
% compile the final versions of the child documents
% |cdocsch1.tex| and |cdocsch2.tex|, respectively:
%\iffalse
%<*samplefinal>
%\fi
%    \begin{macrocode}
\def\version{final}
\input{childdoc.def}
\childdocforwardprefix[cdocsamp]{cdocsfn}{cdocsch}
%    \end{macrocode}

%\iffalse
%</samplefinal>
%\fi
%
% %%%%%%%%%%%%%%%%%%%%%%%%%%%%%%%%%%%%%%
% \paragraph{Command Line Processing.}
%
% The following three command lines generate the output files
% |cdocscld|, |cdocscl1| and |cdocscl2|
% which should be identical to
% |cdocsdrf|, |cdocsch1| and |cdocsfn2|, respectively:
% \begin{center}
% \begin{tabular}{l}
% |latex -jobname cdocscld \|\\
% |  "\def\version{draft}\input{childdoc.def}\childdocforward{cdocsamp}"|\\
% |latex -jobname cdocscl1 \|\\
% |  "\input{childdoc.def}\childdocforward[cdocsamp]{cdocsch1}"|\\
% |latex -jobname cdocscl2 \|\\
% |  "\def\version{final}\input{childdoc.def}\childdocforward{cdocsch2}"|
% \end{tabular}
% \end{center}
% Note that the trailing backslash on each first line
% merely continues the input to the second line
% (for convenient cut ant paste).
% Furthermore, the command |latex| can be replaced by any
% of its alternative versions such as |pdflatex|.
%
% %%%%%%%%%%%%%%%%%%%%%%%%%%%%%%%%%%%%%%%%%%%%%%%%%%%%%%%%%%%%%%%%%%%%%%%%%%%%%%
% %%%%%%%%%%%%%%%%%%%%%%%%%%%%%%%%%%%%%%%%%%%%%%%%%%%%%%%%%%%%%%%%%%%%%%%%%%%%%%
% \section{Implementation}
%\iffalse
%<*package>
%\fi
%
% This section describes the definitions file |childdoc.def|.

% The definitions cannot be loaded using |\usepackage| or |\RequirePackage|
% which has a mechanism to prevent loading a style file more than once.
% When loading the definitions by means of |\input|
% multiple instances have to be prevented manually:
%\iffalse
%This code needs to be before the `\ProvidesFile' directive
%which is defined at the beginning of this file.
%Therefore it is also placed there and commented out here.
%</package>
%<*discard>
%\fi
%    \begin{macrocode}
\ifdefined\childdocmain\endinput\fi
%    \end{macrocode}
%\iffalse
%</discard>
%<*package>
%\fi
%
% \macro{\ifchilddoc}
% \macro{\ifchilddocmanual}
% The conditional |\ifchilddoc| tells whether a
% child (true) or main (false) document is being compiled.
% The conditional |\ifchilddocmanual| tells whether
% the |\includeonly| mechanism is used (false) or
% the selection of child files must be performed manually (true).
% The definitions initialise to false:
%    \begin{macrocode}
\newif\ifchilddoc
\newif\ifchilddocmanual
%    \end{macrocode}

% \macro{\childdocname}
% \macro{\childdocjob}
% The macro |\childdocname| stores the name of the main document
% to be compiled. The macro |\childdocjob| stores the name of
% the document on which the \LaTeX{} compiler was originally invoked.
% The content of |\jobname| cannot be compared
% to filenames specified in the source due to different catcodes.
% The following code rescans |\jobname|, stores the result
% in |\childdocname| and saves a copy in |\childdocjob|:
%    \begin{macrocode}
\edef\childdocname{\scantokens\expandafter{\jobname\noexpand}}
\let\childdocjob\childdocname
%    \end{macrocode}

% \macro{\childdocdisable}
% The macro |\childdocdisable| prevents the main file
% from being processed more than once.
% At this stage, the main document command |\childdocmain|
% is assumed to be called once again where it should do nothing.
% Any subsequent call to it should prevent
% a secondary processing of the main document
% It overwrites the forwarding commands
% |\childdocof| and |\childdocforward|
% with empty macros to prevent further inclusions of the main document:
%    \begin{macrocode}
\newcommand{\childdocdisable}
{
  \renewcommand{\childdocmain}[1]{\renewcommand{\childdocmain}[1]{\endinput}}
  \renewcommand{\childdocof}[1]{}
  \renewcommand{\childdocby}[2][]{}
  \renewcommand{\childdocforward}[2][]{}
  \renewcommand{\childdocdisable}{}
}
%    \end{macrocode}

% \macro{\childdocmain}
% The macro |\childdocmain| is to be called at the top of the main file
% with nothing or the main filename (without extension) as argument.
% First, it breaks loops.
% If the argument is not empty and does not match |\childdocname|
% (which is set by the first inclusion of |childdoc.def|),
% |\ifchilddoc| is set to true, |\includeonly| is applied to the child file
% and |\jobname| is set to the main file
% (for proper handling of |.aux| files):
%    \begin{macrocode}
\newcommand{\childdocmain}[1]
{
  \childdocdisable\childdocmain{}
  \if?#1?\else
    \begingroup
      \def\childdoctmp{#1}
      \ifx\childdoctmp\childdocname
        \def\childdoctmp{}
      \else
        \def\childdoctmp
        {
          \childdoctrue
          \includeonly{\childdocname}
          \def\childdocjob{#1}
          \def\jobname{#1}
        }
      \fi
      \expandafter
    \endgroup
    \childdoctmp
  \fi
}
%    \end{macrocode}

% \macro{\childdocof}
% The command |\childdocof| redirects
% compilation to the main file |#1|.
%    \begin{macrocode}
\newcommand{\childdocof}[1]
{
  \childdocdisable
  \childdoctrue
  \includeonly{\childdocname}
  \def\jobname{#1}
  \def\childdocjob{#1}
  \input{#1}
}
%    \end{macrocode}

% \macro{\childdocby}
% The command |\childdocby| ....
%    \begin{macrocode}
\newcommand{\childdocby}[2][]
{
  \childdocdisable
  \childdoctrue
  \childdocmanualtrue
  \if?#1?\else
    \def\jobname{#2}
  \fi
  \def\childdocjob{#2}
  \input{#2}
  \endinput
}
%    \end{macrocode}

% \macro{\childdocforward}
% The command |\childdocforward| redirects
% compilation to the main file or
% (if the optional argument is given) a child file.
% Parameters are set as if the main file
% or a child file starting with |\childdocof| was compiled.
% Then compilation is handed over to the main file:
%    \begin{macrocode}
\newcommand{\childdocforward}[2][]
{
  \begingroup
    \if?#1?
      \def\childdoctmp
      {
        \def\childdocname{#2}
        \def\childdocjob{#2}
        \def\jobname{#2}
        \input{#2}
        \endinput
      }
    \else
      \def\childdoctmp
      {
        \childdocdisable
        \def\childdocname{#2}
        \childdoctrue
        \includeonly{#2}
        \def\childdocjob{#1}
        \def\jobname{#1}
        \input{#1}
        \endinput
      }
    \fi
    \expandafter
  \endgroup
  \childdoctmp
}
%    \end{macrocode}

% \macro{\childdocforwardprefix}
% The command |\childdocforwardprefix| redirects
% compilation to the main or a child file by means of a pattern.
% The prefix |#1| in the current filename is replaced by |#2|
% and the suffix of the current filename is kept
% (it is assumed that the filename does not contain the substring `|~~~|'
% which is used as a delimiter).
% Compilation is handed over to the new file by |\childdocforward|:
%    \begin{macrocode}
\newcommand{\childdocforwardprefix}[3][]
{
  \begingroup
    \def\childdocextract #2##1~~~{\def\childdoctmp{\childdocforward[#1]{#3##1}}}
    \expandafter\childdocextract\childdocname~~~
    \expandafter
  \endgroup
  \childdoctmp
}
%    \end{macrocode}

% \macro{\childdoc}
% The deprecated macro |\childdoc| is a legacy version of |\childdocmain|:
%    \begin{macrocode}
\newcommand{\childdoc}{\childdocmain}
%    \end{macrocode}

% \macro{\childdocredirect}
% The deprecated macro |\childdocredirect| is a legacy version
% of |\childdocforward| and |\childdocforwardprefix|:
%    \begin{macrocode}
\newcommand{\childdocredirect}[2][]
{
  \begingroup
    \if?#1?
      \def\childdoctmp{\childdocforward{#2}}
    \else
      \def\childdoctmp{\childdocforwardprefix{#1}{#2}}
    \fi
    \expandafter
  \endgroup
  \childdoctmp
}
%    \end{macrocode}

%\iffalse
%</package>
%\fi
%
\endinput
|\\
|\childdocof{|\textit{main}|}|\\
\end{tabular}
\end{center}
at the top of every child file \textit{child}
which is included by |\include{|\textit{child}|}|
from within the main file
(or at least for those files to be compiled individually).
The argument \textit{main} must be the filename of the main file.

There are a couple of
considerations in setting up the main and child documents:

%%%%%%%%%%%%%%%%%%%%%%%%%%%%%%%%%%%%%%%%
\paragraph{Restrictions.}

Please note the following restrictions:
\begin{itemize}
\item
|\childdocmain| must be called with one argument \textit{main}
to ensure compatibility with earlier version of the package.
It must either be empty (|\childdocmain{}|)
or precisely match the filename of the main file in which it is specified.
See \secref{sec:detection} for further information.
\item
The filename \textit{main} must be specified without the |.tex| extension.
\item
The filename \textit{main} is case sensitive
(even in case-insensitive file systems)
due to internal string comparison.
\item
The argument \textit{main} should be fully expanded, it cannot be a macro.
\item
Subdirectories and special characters should be avoided in filenames.
\item
The command |\childdocmain{|\textit{main}|}| must be followed by a whitespace.
It should not be followed immediately by another command
or by a comment mark `|%|'.
This is because the \TeX{} parser reads the token immediately following
the argument of |\childdocmain| and puts it
at the beginning of every child section;
however, a white\-space is ignored.
\end{itemize}

%%%%%%%%%%%%%%%%%%%%%%%%%%%%%%%%%%%%%%%%
\paragraph{Content of Main File.}

It is advisable to place all content in the child files included by |\include|.
Any output contained in the main file will appear in all child documents
unless suppressed manually;
it cannot be suppressed automatically by the |\includeonly| directive
and thus should normally be avoided.
A method to include some content in the main file
by means of conditional processing is described in \secref{sec:conditional}.

%%%%%%%%%%%%%%%%%%%%%%%%%%%%%%%%%%%%%%%%
\paragraph{Page Numbering.}

When only a part of the document is compiled,
the appropriate numbering of pages
(as well as other status parameters)
is determined from the |.aux| files.
The latter contain information from previous passes.
However this information needs to propagate through
all intermediate child documents.
Therefore the page numbering in child documents may well
be inconsistent until the complete document is compiled at least once.

A useful (if unconventional) way to always ensure a consistent
page numbering is to restart the numbering in each child document
and denote the pages by `\textit{child}|.|\textit{page}'
where \textit{child} represents the chapter/section number of the child file.
This can be achieved by the command
|\numberwithin{page}{|\textit{child}|}|
of the \textsf{amsmath} package
where \textit{child} can be |chapter| or |section|
depending on the chosen structuring.
Alternatively, one can modify the macro |\thepage| appropriately
and reset the counter |page| at the start of each child file.

%%%%%%%%%%%%%%%%%%%%%%%%%%%%%%%%%%%%%%%%%%%%%%%%%%%%%%%%%%%%%%%%%%%%%%%%%%%%%%%%
\subsection{Conditional Processing}
\label{sec:conditional}

The package provides a mechanism to compile different versions
of a document. To customise the versions further some conditional processing
can come in handy to distinguish which version is being compiled.
The package provides two macros to describe the compilation context:

%%%%%%%%%%%%%%%%%%%%%%%%%%%%%%%%%%%%%%%%
\DescribeMacro{\ifchilddoc}
The conditional |\ifchilddoc| distinguishes between the compilation of
child documents and the main document:
%
\begin{center}
|\ifchilddoc |\textit{child-code}| |[|\||else |\textit{main-code}]| \||fi|
\end{center}

%%%%%%%%%%%%%%%%%%%%%%%%%%%%%%%%%%%%%%%%
\DescribeMacro{\childdocname}
\DescribeMacro{\childdocjob}
The macro |\childdocname| contains the filename (without extension)
of the main or child file being processed.
Note that |\childdocjob| will always contain the name of the main file.

%%%%%%%%%%%%%%%%%%%%%%%%%%%%%%%%%%%%%%%%
\paragraph{Title Page.}

Conditional processing can be used to include a title or banner page
in the main document when proper precautions are taken.
Importantly, the code in the main file should ensure that the page counter
(as well as other status parameters which are stored in the |.aux| files)
takes the same value after the conditional processing.
Otherwise the page numbers may take divergent values
depending on which part is compiled.

For example, a title page could be declared by:
%
\begin{center}
\begin{tabular}{l}
|\ifchilddoc\||else|\\
|\addtocounter{page}{-1}|\\
\textit{code for title page}\\
|\newpage|\\
|\||fi|
\end{tabular}
\end{center}
%
A banner page for the child documents can be generated by:
%
\begin{center}
\begin{tabular}{l}
|\ifchilddoc|\\
|\addtocounter{page}{-1}|\\
\textit{code for banner page}\\
|\newpage|\\
|\||fi|
\end{tabular}
\end{center}
%
Here one could write a message such as:
\begin{center}
|This is the part \childdocname{} of \childdocjob{}.|
\end{center}

%%%%%%%%%%%%%%%%%%%%%%%%%%%%%%%%%%%%%%%%%%%%%%%%%%%%%%%%%%%%%%%%%%%%%%%%%%%%%%%%
\subsection{Flags}
\label{sec:flags}

The package makes it easy to generate different versions
of the main or child documents.
To this end compilation flags can be defined
and assigned different default values.
They will be particularly useful in conjunction
with the forwarding mechanism described in \secref{sec:forward}.

For example, it may be useful to have a flag |\version|
which can be set to |draft| or |final|.
The document source will contain some conditional code
depending on the value of |\version|.
Suppose further, the flag should default to |final| for the main file
and to |draft| for child files
which is a natural assignment for editing the document.
This is achieved by placing the following code
in the preamble of the main document
(below the |\childdocmain| directive):
%
\begin{center}
\begin{tabular}{l}
|\ifchilddoc|\\
|\providecommand{\version}{draft}|\\
|\||else|\\
|\providecommand{\version}{final}|\\
|\||fi|
\end{tabular}
\end{center}
%
The definition by |\providecommand| makes sure
that previous definitions are not overwritten.
Further statements |\providecommand{\version}{...}|
can thus be added before the above code to override it.

For the main file, one might add a line
(between |\childdocmain| and the above block)
%
\begin{center}
|%\ifchilddoc\||else\providecommand{\version}{draft}\||fi|
\end{center}
%
which can be uncommented to produce a draft version.
Likewise one can add a line to the very top of a child file
(above the |\childdocof{|\textit{main}|}| directive)
%
\begin{center}
|%\providecommand{\version}{final}|
\end{center}
%
which can be uncommented to produce the final version of this child document.

%%%%%%%%%%%%%%%%%%%%%%%%%%%%%%%%%%%%%%%%%%%%%%%%%%%%%%%%%%%%%%%%%%%%%%%%%%%%%%%%
\subsection{Forwarding}
\label{sec:forward}

Different versions of the main or child documents
using compilation flags as described in \secref{sec:flags}
can be (permanently) stored in different files
for convenient compilation, viewing and distribution.
To this end, the package defines a command
to pass on compilation to a different file:

%%%%%%%%%%%%%%%%%%%%%%%%%%%%%%%%%%%%%%%%
\DescribeMacro{\childdocforward}
The command |\childdocforward| redirects processing to
another source file:
%
\begin{center}
\begin{tabular}{l}
|% \iffalse
%
% childdoc.dtx Copyright (C) 2017-2018 Niklas Beisert
%
% This work may be distributed and/or modified under the
% conditions of the LaTeX Project Public License, either version 1.3
% of this license or (at your option) any later version.
% The latest version of this license is in
%   http://www.latex-project.org/lppl.txt
% and version 1.3 or later is part of all distributions of LaTeX
% version 2005/12/01 or later.
%
% This work has the LPPL maintenance status `maintained'.
%
% The Current Maintainer of this work is Niklas Beisert.
%
% This work consists of the files childdoc.dtx and childdoc.ins
% and the derived files childdoc.def and cdocsamp.tex with
% cdocsch1.tex, cdocsch2.tex, cdocsdrf.tex, cdocsfn1.tex, cdocsfn2.tex.
%
%<package>\ifdefined\childdocmain\endinput\fi
%<package>\ProvidesFile{childdoc.def}[2018/12/30 v2.0 child document driver]
%<samplemain>\ProvidesFile{cdocsamp.tex}[2018/12/30 v2.0 sample for childdoc]
%<*driver>
%\ProvidesFile{childdoc.drv}[2018/12/30 v2.0 childdoc reference manual file]
\PassOptionsToClass{10pt,a4paper}{article}
\documentclass{ltxdoc}

\usepackage[margin=35mm]{geometry}
\usepackage{hyperref}
\usepackage{hyperxmp}
\usepackage[usenames]{color}

\hypersetup{colorlinks=true}
\hypersetup{pdfstartview=FitH}
\hypersetup{pdfpagemode=UseNone}
\hypersetup{pdfsource={}}
\hypersetup{pdflang={en-UK}}
\hypersetup{pdfcopyright={Copyright 2017-2018 Niklas Beisert.
  This work may be distributed and/or modified under the
  conditions of the LaTeX Project Public License, either version 1.3
  of this license or (at your option) any later version.}}
\hypersetup{pdflicenseurl={http://www.latex-project.org/lppl.txt}}
\hypersetup{pdfcontactaddress={ETH Zurich, ITP, HIT K,
  Wolfgang-Pauli-Strasse 27}}
\hypersetup{pdfcontactpostcode={8093}}
\hypersetup{pdfcontactcity={Zurich}}
\hypersetup{pdfcontactcountry={Switzerland}}
\hypersetup{pdfcontactemail={nbeisert@itp.phys.ethz.ch}}
\hypersetup{pdfcontacturl={http://people.phys.ethz.ch/\xmptilde nbeisert/}}

\newcommand{\secref}[1]{\hyperref[#1]{section \ref*{#1}}}

\parskip1ex
\parindent0pt
\let\olditemize\itemize
\def\itemize{\olditemize\parskip0pt}

\begin{document}

\title{The \textsf{childdoc} Package}
\hypersetup{pdftitle={The childdoc Package}}
\author{Niklas Beisert\\[2ex]
  Institut f\"ur Theoretische Physik\\
  Eidgen\"ossische Technische Hochschule Z\"urich\\
  Wolfgang-Pauli-Strasse 27, 8093 Z\"urich, Switzerland\\[1ex]
  \href{mailto:nbeisert@itp.phys.ethz.ch}
  {\texttt{nbeisert@itp.phys.ethz.ch}}}
\hypersetup{pdfauthor={Niklas Beisert}}
\hypersetup{pdfsubject={Manual for the LaTeX2e Package childdoc}}
\date{30 December 2018, \textsf{v2.0}}
\maketitle

\begin{abstract}\noindent
\textsf{childdoc} is a \LaTeXe{} package
that enables the direct compilation
of document sections included by |\include|
to individual files.
\end{abstract}

\begingroup
\parskip0ex
\tableofcontents
\endgroup

%%%%%%%%%%%%%%%%%%%%%%%%%%%%%%%%%%%%%%%%%%%%%%%%%%%%%%%%%%%%%%%%%%%%%%%%%%%%%%%%
%%%%%%%%%%%%%%%%%%%%%%%%%%%%%%%%%%%%%%%%%%%%%%%%%%%%%%%%%%%%%%%%%%%%%%%%%%%%%%%%
\section{Introduction}

\LaTeX{} provides a mechanism to structure a large document (such as a book)
into a main file and several child files (containing the chapters)
using the |\include| command.
This mechanism is beneficial for documents
which span hundreds of pages in order to
make the source file(s) more manageable.
Moreover, compilation can be restricted to
selected child files by means of the |\includeonly| command.
The latter feature can be used to reduce the compilation time while editing
(this was significantly more useful in the earlier days of \LaTeX{})
or to generate a smaller document which is easier to navigate.
Another application of |\includeonly| is to generate
documents consisting of selected parts of the complete document.

However, there are a few drawbacks of the plain |\include| mechanism:
\begin{itemize}
\item
The child files cannot be compiled on their own,
they can only be compiled via the main file.
A naive editing environment
(such as a text editor with an option
to have the current file processed by \LaTeX)
may require one to switch to the main file before compiling;
attempting to compile the child file produces errors.
\item
The main file must be modified (each time)
to adjust the |\includeonly| command
to the present needs. This easily leaves the main file in a messy state.
\item
The generated document will always carry the filename
of the main document. This is inconvenient if
several child files are to be compiled and
to be kept for distribution.
\end{itemize}

The present package provides a simple interface
to make child files individually compilable by \LaTeX{}.
Compiling a child file then has the same effect as compiling
the main file with an |\includeonly| command
to select the appropriate child.
Moreover the generated document will carry the name of the child
rather than the main file.
This resolves all three above issues.

This feature is meant to make the editing of books,
thesis documents and lecture notes somewhat more convenient.
However, the package can also be used efficiently for
composing a series of documents (such as exercise sheets)
which are typically distributed individually.
It then assists the author in generating the individual documents
(potentially in different versions)
as well as a document containing the collected series.
Another application is in developing style files
or other kinds of included material
where compilation of the style file could redirect
to a sample or test file.

%%%%%%%%%%%%%%%%%%%%%%%%%%%%%%%%%%%%%%%%%%%%%%%%%%%%%%%%%%%%%%%%%%%%%%%%%%%%%%%%
%%%%%%%%%%%%%%%%%%%%%%%%%%%%%%%%%%%%%%%%%%%%%%%%%%%%%%%%%%%%%%%%%%%%%%%%%%%%%%%%
\section{Usage}

First of all, the package \textsf{childdoc} is \emph{not} a standard
\LaTeXe{} |.sty| style file! Therefore it needs to be invoked in
a non-standard way.

%%%%%%%%%%%%%%%%%%%%%%%%%%%%%%%%%%%%%%%%%%%%%%%%%%%%%%%%%%%%%%%%%%%%%%%%%%%%%%%%
\subsection{Included Files}
\label{sec:include}

%%%%%%%%%%%%%%%%%%%%%%%%%%%%%%%%%%%%%%%%
\DescribeMacro{\childdocmain}
To use the package, add the commands
\begin{center}
\begin{tabular}{l}
|\input{childdoc.def}|\\
|\childdocmain{}|\\
\end{tabular}
\end{center}
at the very top of the main \LaTeX{} file,
in particular \emph{before} the |\documentclass| statement!
The argument of |\childdocmain| should be left empty
(but it must be present).

%%%%%%%%%%%%%%%%%%%%%%%%%%%%%%%%%%%%%%%%
\DescribeMacro{\childdocof}
Furthermore, add the commands
\begin{center}
\begin{tabular}{l}
|\input{childdoc.def}|\\
|\childdocof{|\textit{main}|}|\\
\end{tabular}
\end{center}
at the top of every child file \textit{child}
which is included by |\include{|\textit{child}|}|
from within the main file
(or at least for those files to be compiled individually).
The argument \textit{main} must be the filename of the main file.

There are a couple of
considerations in setting up the main and child documents:

%%%%%%%%%%%%%%%%%%%%%%%%%%%%%%%%%%%%%%%%
\paragraph{Restrictions.}

Please note the following restrictions:
\begin{itemize}
\item
|\childdocmain| must be called with one argument \textit{main}
to ensure compatibility with earlier version of the package.
It must either be empty (|\childdocmain{}|)
or precisely match the filename of the main file in which it is specified.
See \secref{sec:detection} for further information.
\item
The filename \textit{main} must be specified without the |.tex| extension.
\item
The filename \textit{main} is case sensitive
(even in case-insensitive file systems)
due to internal string comparison.
\item
The argument \textit{main} should be fully expanded, it cannot be a macro.
\item
Subdirectories and special characters should be avoided in filenames.
\item
The command |\childdocmain{|\textit{main}|}| must be followed by a whitespace.
It should not be followed immediately by another command
or by a comment mark `|%|'.
This is because the \TeX{} parser reads the token immediately following
the argument of |\childdocmain| and puts it
at the beginning of every child section;
however, a white\-space is ignored.
\end{itemize}

%%%%%%%%%%%%%%%%%%%%%%%%%%%%%%%%%%%%%%%%
\paragraph{Content of Main File.}

It is advisable to place all content in the child files included by |\include|.
Any output contained in the main file will appear in all child documents
unless suppressed manually;
it cannot be suppressed automatically by the |\includeonly| directive
and thus should normally be avoided.
A method to include some content in the main file
by means of conditional processing is described in \secref{sec:conditional}.

%%%%%%%%%%%%%%%%%%%%%%%%%%%%%%%%%%%%%%%%
\paragraph{Page Numbering.}

When only a part of the document is compiled,
the appropriate numbering of pages
(as well as other status parameters)
is determined from the |.aux| files.
The latter contain information from previous passes.
However this information needs to propagate through
all intermediate child documents.
Therefore the page numbering in child documents may well
be inconsistent until the complete document is compiled at least once.

A useful (if unconventional) way to always ensure a consistent
page numbering is to restart the numbering in each child document
and denote the pages by `\textit{child}|.|\textit{page}'
where \textit{child} represents the chapter/section number of the child file.
This can be achieved by the command
|\numberwithin{page}{|\textit{child}|}|
of the \textsf{amsmath} package
where \textit{child} can be |chapter| or |section|
depending on the chosen structuring.
Alternatively, one can modify the macro |\thepage| appropriately
and reset the counter |page| at the start of each child file.

%%%%%%%%%%%%%%%%%%%%%%%%%%%%%%%%%%%%%%%%%%%%%%%%%%%%%%%%%%%%%%%%%%%%%%%%%%%%%%%%
\subsection{Conditional Processing}
\label{sec:conditional}

The package provides a mechanism to compile different versions
of a document. To customise the versions further some conditional processing
can come in handy to distinguish which version is being compiled.
The package provides two macros to describe the compilation context:

%%%%%%%%%%%%%%%%%%%%%%%%%%%%%%%%%%%%%%%%
\DescribeMacro{\ifchilddoc}
The conditional |\ifchilddoc| distinguishes between the compilation of
child documents and the main document:
%
\begin{center}
|\ifchilddoc |\textit{child-code}| |[|\||else |\textit{main-code}]| \||fi|
\end{center}

%%%%%%%%%%%%%%%%%%%%%%%%%%%%%%%%%%%%%%%%
\DescribeMacro{\childdocname}
\DescribeMacro{\childdocjob}
The macro |\childdocname| contains the filename (without extension)
of the main or child file being processed.
Note that |\childdocjob| will always contain the name of the main file.

%%%%%%%%%%%%%%%%%%%%%%%%%%%%%%%%%%%%%%%%
\paragraph{Title Page.}

Conditional processing can be used to include a title or banner page
in the main document when proper precautions are taken.
Importantly, the code in the main file should ensure that the page counter
(as well as other status parameters which are stored in the |.aux| files)
takes the same value after the conditional processing.
Otherwise the page numbers may take divergent values
depending on which part is compiled.

For example, a title page could be declared by:
%
\begin{center}
\begin{tabular}{l}
|\ifchilddoc\||else|\\
|\addtocounter{page}{-1}|\\
\textit{code for title page}\\
|\newpage|\\
|\||fi|
\end{tabular}
\end{center}
%
A banner page for the child documents can be generated by:
%
\begin{center}
\begin{tabular}{l}
|\ifchilddoc|\\
|\addtocounter{page}{-1}|\\
\textit{code for banner page}\\
|\newpage|\\
|\||fi|
\end{tabular}
\end{center}
%
Here one could write a message such as:
\begin{center}
|This is the part \childdocname{} of \childdocjob{}.|
\end{center}

%%%%%%%%%%%%%%%%%%%%%%%%%%%%%%%%%%%%%%%%%%%%%%%%%%%%%%%%%%%%%%%%%%%%%%%%%%%%%%%%
\subsection{Flags}
\label{sec:flags}

The package makes it easy to generate different versions
of the main or child documents.
To this end compilation flags can be defined
and assigned different default values.
They will be particularly useful in conjunction
with the forwarding mechanism described in \secref{sec:forward}.

For example, it may be useful to have a flag |\version|
which can be set to |draft| or |final|.
The document source will contain some conditional code
depending on the value of |\version|.
Suppose further, the flag should default to |final| for the main file
and to |draft| for child files
which is a natural assignment for editing the document.
This is achieved by placing the following code
in the preamble of the main document
(below the |\childdocmain| directive):
%
\begin{center}
\begin{tabular}{l}
|\ifchilddoc|\\
|\providecommand{\version}{draft}|\\
|\||else|\\
|\providecommand{\version}{final}|\\
|\||fi|
\end{tabular}
\end{center}
%
The definition by |\providecommand| makes sure
that previous definitions are not overwritten.
Further statements |\providecommand{\version}{...}|
can thus be added before the above code to override it.

For the main file, one might add a line
(between |\childdocmain| and the above block)
%
\begin{center}
|%\ifchilddoc\||else\providecommand{\version}{draft}\||fi|
\end{center}
%
which can be uncommented to produce a draft version.
Likewise one can add a line to the very top of a child file
(above the |\childdocof{|\textit{main}|}| directive)
%
\begin{center}
|%\providecommand{\version}{final}|
\end{center}
%
which can be uncommented to produce the final version of this child document.

%%%%%%%%%%%%%%%%%%%%%%%%%%%%%%%%%%%%%%%%%%%%%%%%%%%%%%%%%%%%%%%%%%%%%%%%%%%%%%%%
\subsection{Forwarding}
\label{sec:forward}

Different versions of the main or child documents
using compilation flags as described in \secref{sec:flags}
can be (permanently) stored in different files
for convenient compilation, viewing and distribution.
To this end, the package defines a command
to pass on compilation to a different file:

%%%%%%%%%%%%%%%%%%%%%%%%%%%%%%%%%%%%%%%%
\DescribeMacro{\childdocforward}
The command |\childdocforward| redirects processing to
another source file:
%
\begin{center}
\begin{tabular}{l}
|\input{childdoc.def}|\\
|\childdocforward[|\textit{main}|]{|\textit{dest}|}|\\
\end{tabular}
\end{center}
%
The argument \textit{dest} is the destination file
(without extension).
It should be the main file or one of the child files.
Note that further \textsf{childdoc} directives
such as |\childdocof| and |\childdocforward|
in the indicated file will be processed in this form.
The optional argument \textit{main}
passes on directly to the main file \textit{main}
while pretending to compile the child \textit{dest}.
This form behaves as if \textit{dest}
issues |\childdocof{|\textit{main}|}| right away,
and no further \textsf{childdoc} directives will be processed.

%%%%%%%%%%%%%%%%%%%%%%%%%%%%%%%%%%%%%%%%
\DescribeMacro{\...prefix}
In the alternative form |\childdocforwardprefix|,
%
\begin{center}
\begin{tabular}{l}
|\input{childdoc.def}|\\
|\childdocforwardprefix[|\textit{main}|]{|\textit{prefix}|}{|\textit{dest}|}|
\end{tabular}
\end{center}
%
the destination file is determined by a pattern
depending on the current file:
To make this work, the current file must be called
`{\textit{prefix}\hspace{0.2em}\textit{suffix}}'
with \textit{prefix} matching precisely the argument.
Processing is then passed on to the file
`{\textit{dest}\hspace{0.2em}\textit{suffix}}'.
Surely, the same effect is achieved by
directly specifying the
argument `{\textit{dest}\hspace{0.2em}\textit{suffix}}'
in the first form.
However, that requires to set up a different file
for each child. With the alternative form of the command
all these files can have exactly the same content
which simplifies setting them up and maintaining them.

For example, the following file |draft.tex|
with a compilation flag |\version| as described in \secref{sec:flags}
compiles the main document as a draft:
%
\begin{center}
\begin{tabular}{l}
|\def\version{draft}|\\
|\input{childdoc.def}|\\
|\childdocforward{|\textit{main}|}|
\end{tabular}
\end{center}
%
Likewise, the following files |final|\textit{nn}|.tex|
compile the final version of the child document
|child|\textit{nn}|.tex|:
%
\begin{center}
\begin{tabular}{l}
|\def\version{final}|\\
|\input{childdoc.def}|\\
|\childdocforwardprefix{final}{child}|
\end{tabular}
\end{center}
%

Note that when several versions of a main file and/or of each child file
are to be generated, it may be convenient to set up a |Makefile| or
shell script to automatise the process.

%%%%%%%%%%%%%%%%%%%%%%%%%%%%%%%%%%%%%%%%%%%%%%%%%%%%%%%%%%%%%%%%%%%%%%%%%%%%%%%%
\subsection{Command Line Processing}
\label{sec:commandline}

The effect of redirection files can also be achieved by invoking
the \LaTeX{} compiler with a more elaborate command line.
Most conveniently this should be done as part
of a shell script or a |Makefile|.

When using \textsf{childdoc} in the main file, the following
command lines effectively perform a redirection
(note that depending on the shell being used,
backslashes may have to be doubled: `|\|' $\to$ `|\\|'):
%
\begin{center}
|... -jobname "|\textit{target}|" |\\|"|[\textit{flags}]%
|\input{childdoc.def}\childdocforward[|\textit{main}|]{|\textit{dest}|}"|
\end{center}
%
Here \textit{target} is the name of the output file,
\textit{main} is the name of the main file
and \textit{dest} is the name of the main or child file to be processed
(all filenames without extensions).
The optional argument \textit{main} can be omitted
if \textit{main} matches \textit{dest}.
Optionally, compilation \textit{flags} can be defined via |\def| commands.
This command line makes the \TeX{} engine believe
it is compiling the file \textit{target}
whose content is specified as the latter parameter.
The provided code then forwards the processing to
\textit{main} or \textit{dest} as described in \secref{sec:forward}.

%%%%%%%%%%%%%%%%%%%%%%%%%%%%%%%%%%%%%%%%%%%%%%%%%%%%%%%%%%%%%%%%%%%%%%%%%%%%%%%%
\subsection{Include by Input}
\label{sec:input}

Including child documents by |\include| has some restrictions by design.
Most notably, the content of a child document always occupies
its own set of pages; pages cannot be shared between child documents.
Usually, this behaviour makes perfect sense
because each child document contain an essential part of the document.
However, in some situations it may be desirable to compose
a document from a collection of parts
without having mandatory page breaks between then.
For this case, the package
provides a mechanism to include parts
by |\input| which can also be processed individually.
However, by construction this mechanism
requires manual handling of the content to be output.

%%%%%%%%%%%%%%%%%%%%%%%%%%%%%%%%%%%%%%%%
\DescribeMacro{\ifchilddocmanual}
The main file should be prepared as usual, see \secref{sec:include}.
However, the document body must make a distinction
between processing of an individual part and of the main document, e.g.:
%
\begin{center}
\begin{tabular}{l}
|\ifchilddocmanual|\\
|\input{\childdocname}|\\
|\||else|\\
\textit{document body with }|\input{|\textit{part}|}|\\
|\||fi|
\end{tabular}
\end{center}
%
The conditional |\ifchilddocmanual| is true whenever
a part to be included by |\input| is being compiled,
and the name of the part is stored in |\childdocname|.

%%%%%%%%%%%%%%%%%%%%%%%%%%%%%%%%%%%%%%%%
\DescribeMacro{\childdocby}
Each part to be included by |\input| should start with:
%
\begin{center}
\begin{tabular}{l}
|\input{childdoc.def}|\\
|\childdocby{|\textit{main}|}|\\
\end{tabular}
\end{center}
%
The directive |\childdocby| is similar to |\childdocof|
described in \secref{sec:include},
but the subsequent selection of content must be done manually.
To that end, both |\ifchilddoc| and |\ifchilddocmanual|
will be true upon processing of a part,
and the name of the part is stored in |\childdocname|.
Note that |\jobname| will be set to the filename of the current part
so that each part receives an individual |.aux| file
that does not interfere with the |.aux| file(s) of the main document.
This behaviour can be altered by the alternative form
|\childdocby[*]{|\textit{main}|}| (with a non-empty optional argument)
which uses the |.aux| file of the main document
by setting |\jobname| to \textit{main}.

%%%%%%%%%%%%%%%%%%%%%%%%%%%%%%%%%%%%%%%%%%%%%%%%%%%%%%%%%%%%%%%%%%%%%%%%%%%%%%%%
\subsection{Driver Development}
\label{sec:driver}

The \textsf{childdoc} mechanism can also be use for the development
of definition files such as \LaTeX{} styles or classes.
This case differs from the above setup with multiple parts
included by |\include| in that no |\includeonly| should be invoked.
This can be achieved by starting the include file
(before |\ProvidesPackage|) with:
%
\begin{center}
\begin{tabular}{l}
|\input{childdoc.def}|\\
|\childdocforward{|\textit{main}|}|\\
\end{tabular}
\end{center}
%
or alternatively with:
%
\begin{center}
\begin{tabular}{l}
|\input{childdoc.def}|\\
|\childdocby{|\textit{main}|}|\\
\end{tabular}
\end{center}
%
Both forms have slightly different effects as described above.
The main file is prepared as usual, see \secref{sec:include}.

%%%%%%%%%%%%%%%%%%%%%%%%%%%%%%%%%%%%%%%%%%%%%%%%%%%%%%%%%%%%%%%%%%%%%%%%%%%%%%%%
\subsection{Legacy Detection}
\label{sec:detection}

The directive |\childdocmain| in the main file can detect
whether the complete document or merely a child is to be compiled
even without using the directive |\childdocof|.
This method is deprecated because it is less robust
and there is no compelling reason to use it;
it is merely provided for backward compatibility
and it may be removed in future versions.

If the detection mechanism is to be used,
it is mandatory to correctly specify
the filename of the main file as the argument of |\childdocmain|:
%
\begin{center}
\begin{tabular}{l}
|\input{childdoc.def}|\\
|\childdocmain{|\textit{main}|}|\\
\end{tabular}
\end{center}
%
If |\jobname| does not match the argument \textit{main} of |\childdocmain|,
it is assumed that |\jobname| points to the child file to be compiled.
When using |\childdocmain| with the main file specified as argument,
it suffices to start a child file
with just |\input{|\textit{main}|}|
without loading of the package and using |\childdocof|.
If instead all processing is done
with the appropriate \textsf{childdoc} directives,
the argument of \textit{main} of |\childdocmain| can be empty.

An alternative version of the command line processing described
in \secref{sec:commandline} using the detection mechanism reads:
%
\begin{center}
|... -jobname "|\textit{target}|" "|[\textit{flags}]%
[|\def\jobname{|\textit{dest}|}|]|\input{|\textit{main}|}"|
\end{center}

%%%%%%%%%%%%%%%%%%%%%%%%%%%%%%%%%%%%%%%%%%%%%%%%%%%%%%%%%%%%%%%%%%%%%%%%%%%%%%%%
\subsection{Manual Code}
\label{sec:manual}

In case one cannot be certain whether the definitions file |childdoc.def|
is installed on the target \TeX{} distribution
and one prefers not to ship it,
it is conceivable to paste a few relevant commands into the sources.

To that end, drop all statements |\input{childdoc.def}|
and perform the replacements as outlined below.
Instead of |\childdocmain{|\textit{main}|}| add the following code
to the top of the main file:
%
\begin{center}
\begin{tabular}{l}
|\||ifdefined\childdocname\endinput\||fi\newif\ifchilddoc|\\
|\edef\childdocname{\scantokens\expandafter{\jobname\noexpand}}|\\
|\def\childdocmain{|\textit{main}|}\||ifx\childdocmain\childdocname\||else|\\
|\childdoctrue\includeonly{\childdocname}\let\jobname\childdocmain\||fi|\\
\end{tabular}
\end{center}
%
Instead of |\childdocof{|\textit{main}|}| just include the main file
at the top of each child file:
%
\begin{center}
|\input{|\textit{main}|}|
\end{center}
%
A simple redirection |\childdocforward{|\textit{dest}|}| is achieved by:
%
\begin{center}
|\def\jobname{|\textit{dest}|}\input{\jobname}|
\end{center}
%
The redirection with prefix
|\childdocforwardprefix[|\textit{prefix}|]{|\textit{dest}|}|
is accomplished by:
%
\begin{center}
\begin{tabular}{l}
|{\edef\jobname{\scantokens\expandafter{\jobname\noexpand}}|\\
|\def\redirectjob |\textit{prefix}|#1~~~{\gdef\jobname{|\textit{dest}|#1}}|\\
|\expandafter\redirectjob\jobname~~~}\input{\jobname}|
\end{tabular}
\end{center}

In an alternative approach,
child documents can be compiled by a specific command line
without additional code or specific definitions:
%
\begin{center}
|... -jobname "|\textit{target}|" "|[\textit{flags}]%
|\includeonly{|\textit{dest}|}\input{|\textit{main}|}"|
\end{center}
%

%%%%%%%%%%%%%%%%%%%%%%%%%%%%%%%%%%%%%%%%%%%%%%%%%%%%%%%%%%%%%%%%%%%%%%%%%%%%%%%%
%%%%%%%%%%%%%%%%%%%%%%%%%%%%%%%%%%%%%%%%%%%%%%%%%%%%%%%%%%%%%%%%%%%%%%%%%%%%%%%%
\section{Information}

%%%%%%%%%%%%%%%%%%%%%%%%%%%%%%%%%%%%%%%%%%%%%%%%%%%%%%%%%%%%%%%%%%%%%%%%%%%%%%%%
\subsection{Copyright}

Copyright \copyright{} 2017--2018 Niklas Beisert

This work may be distributed and/or modified under the
conditions of the \LaTeX{} Project Public License, either version 1.3
of this license or (at your option) any later version.
The latest version of this license is in
  \url{http://www.latex-project.org/lppl.txt}
and version 1.3 or later is part of all distributions of \LaTeX{}
version 2005/12/01 or later.

This work has the LPPL maintenance status `maintained'.

The Current Maintainer of this work is Niklas Beisert.

This work consists of the files |README.txt|, |childdoc.ins| and |childdoc.dtx|
as well as the derived files |childdoc.def|, |cdocsamp.tex|
with |cdocsch1.tex|, |cdocsch2.tex|, |cdocspt3.tex|, |cdocspt4.tex|,
|cdocsdrf.tex|, |cdocsfn1.tex|, |cdocsfn2.tex|
as well as |childdoc.pdf|.

%%%%%%%%%%%%%%%%%%%%%%%%%%%%%%%%%%%%%%%%%%%%%%%%%%%%%%%%%%%%%%%%%%%%%%%%%%%%%%%%
\subsection{Files and Installation}

The package consists of the files:
%
\begin{center}
\begin{tabular}{ll}
    |README.txt|   & readme file \\
    |childdoc.ins| & installation file \\
    |childdoc.dtx| & source file \\
    |childdoc.def| & definition file \\
    |cdocsamp.tex| & sample main file \\
    |cdocsch1.tex| & sample include file \\
    |cdocsch2.tex| & sample include file \\
    |cdocspt3.tex| & sample part file \\
    |cdocspt4.tex| & sample part file \\
    |cdocsdrf.tex| & sample redirection file \\
    |cdocsfn1.tex| & sample redirection file \\
    |cdocsfn2.tex| & sample redirection file \\
    |childdoc.pdf| & manual
\end{tabular}
\end{center}
%
The distribution consists of the files
|README.txt|, |childdoc.ins| and |childdoc.dtx|.
%
\begin{itemize}
\item
Run (pdf)\LaTeX{} on |childdoc.dtx|
to compile the manual |childdoc.pdf| (this file).
\item
Run \LaTeX{} on |childdoc.ins| to create the definitions file |childdoc.def|
and the sample |cdocsamp.tex| with include files
|cdocsch1.tex|, |cdocsch2.tex|, |cdocspt3.tex|, |cdocspt4.tex|,
|cdocsdrf.tex|, |cdocsfn1.tex|, |cdocsfn2.tex|.
Then copy the file |childdoc.def| to an appropriate directory of your \LaTeX{}
distribution, e.g.\ \textit{texmf-root}|/tex/latex/childdoc|.
\end{itemize}

%%%%%%%%%%%%%%%%%%%%%%%%%%%%%%%%%%%%%%%%%%%%%%%%%%%%%%%%%%%%%%%%%%%%%%%%%%%%%%%%
\subsection{Related CTAN Packages}

There are several other packages which offer a similar functionality:
%
\begin{itemize}
\item
The packages
\href{http://ctan.org/pkg/docmute}{\textsf{docmute}},
\href{http://ctan.org/pkg/includex}{\textsf{includex}} and
\href{http://ctan.org/pkg/standalone}{\textsf{standalone}}
provide commands to include only the document body of
a child file thus allowing both files to be compiled individually.
\item
The packages \href{http://ctan.org/pkg/subdocs}{\textsf{subdocs}}
and \href{http://ctan.org/pkg/subfiles}{\textsf{subfiles}}
provide structures in which the main and child documents can be
encapsulated and allowing them to be compiled individually.
The inclusion mechanism is different from the conventional |\include|.
\item
The package \href{http://ctan.org/pkg/combine}{\textsf{combine}}
is an elaborate solution to combine several documents into one.
\end{itemize}
%
See also the CTAN topic \href{http://ctan.org/topic/subdocs}{\textsf{subdocs}}
for further related packages.
The present package differs from the above solutions in that
a document structure constructed with the conventional |\include| mechanism
just needs two extra commands at the top of every file
such that all constituent files can be compiled individually.

%%%%%%%%%%%%%%%%%%%%%%%%%%%%%%%%%%%%%%%%%%%%%%%%%%%%%%%%%%%%%%%%%%%%%%%%%%%%%%%%
%\subsection{Feature Suggestions}
%
%The following is a list of features which may be useful for future
%versions of this package:
%%
%\begin{itemize}
%\item
%\ldots
%\end{itemize}

%%%%%%%%%%%%%%%%%%%%%%%%%%%%%%%%%%%%%%%%%%%%%%%%%%%%%%%%%%%%%%%%%%%%%%%%%%%%%%%%
\subsection{Revision History}

%%%%%%%%%%%%%%%%%%%%%%%%%%%%%%%%%%%%%%%%
\paragraph{v2.0:} 2018/12/30

\begin{itemize}
\item
immediate forward processing
\item
added |\childdocby| mechanism
\item
manual restructured
\end{itemize}

%%%%%%%%%%%%%%%%%%%%%%%%%%%%%%%%%%%%%%%%
\paragraph{v1.6:} 2018/01/17

\begin{itemize}
\item
application for development of include files
\item
corrections to manual
\end{itemize}

%%%%%%%%%%%%%%%%%%%%%%%%%%%%%%%%%%%%%%%%
\paragraph{v1.5:} 2017/05/21

\begin{itemize}
\item
more complete structuring introduced
\item
|\childdocof| introduced
\item
|\childdoc| renamed to |\childdocmain|
\item
|\childredirect| renamed to |\childdocforward| and |\childdocforwardprefix|
and functionality expanded
\end{itemize}

%%%%%%%%%%%%%%%%%%%%%%%%%%%%%%%%%%%%%%%%
\paragraph{v1.0:} 2017/04/27

\begin{itemize}
\item
manual and install package
\item
first version published on CTAN
\end{itemize}

%%%%%%%%%%%%%%%%%%%%%%%%%%%%%%%%%%%%%%%%
\paragraph{v0.6:} 2017/04/26

\begin{itemize}
\item
redirection mechanism added
\end{itemize}

%%%%%%%%%%%%%%%%%%%%%%%%%%%%%%%%%%%%%%%%
\paragraph{v0.5:} 2017/04/26

\begin{itemize}
\item
functionality in definition file
\end{itemize}


%%%%%%%%%%%%%%%%%%%%%%%%%%%%%%%%%%%%%%%%%%%%%%%%%%%%%%%%%%%%%%%%%%%%%%%%%%%%%%%%
%%%%%%%%%%%%%%%%%%%%%%%%%%%%%%%%%%%%%%%%%%%%%%%%%%%%%%%%%%%%%%%%%%%%%%%%%%%%%%%%
%%%%%%%%%%%%%%%%%%%%%%%%%%%%%%%%%%%%%%%%%%%%%%%%%%%%%%%%%%%%%%%%%%%%%%%%%%%%%%%%
\appendix

\settowidth\MacroIndent{\rmfamily\scriptsize 000\ }

 \DocInput{childdoc.dtx}

\end{document}
%</driver>
% \fi
%
% %%%%%%%%%%%%%%%%%%%%%%%%%%%%%%%%%%%%%%%%%%%%%%%%%%%%%%%%%%%%%%%%%%%%%%%%%%%%%%
% %%%%%%%%%%%%%%%%%%%%%%%%%%%%%%%%%%%%%%%%%%%%%%%%%%%%%%%%%%%%%%%%%%%%%%%%%%%%%%
% \section{Sample}
%\iffalse
%<*samplemain>
%\fi
%
% The following presents a sample document
% with two chapters, two parts, a title page,
% a compile flag as well as three forwarding files to set the flag.
% It consists of eight |.tex| files:
% \begin{center}
% \begin{tabular}{ll}
% |cdocsamp.tex|&main file\\
% |cdocsch1.tex|&include file for chapter 1\\
% |cdocsch2.tex|&include file for chapter 2\\
% |cdocspt3.tex|&include file for part 3\\
% |cdocspt4.tex|&include file for part 4\\
% |cdocsdrf.tex|&forwarding file for main file in draft mode\\
% |cdocsfi1.tex|&forwarding file for final version of chapter 1\\
% |cdocsfi2.tex|&forwarding file for final version of chapter 2\\
% \end{tabular}
% \end{center}
% Each of the eight files can be compiled directly by the \LaTeX{} compiler.
%
% %%%%%%%%%%%%%%%%%%%%%%%%%%%%%%%%%%%%%%
% \paragraph{Main File.}
%
% The main file is called |cdocsamp.tex|.
%
% Load the \textsf{childdoc} definitions and
% declare the filename for the main document:
%    \begin{macrocode}
\input{childdoc.def}
\childdocmain{}
%    \end{macrocode}

% Optional override for |\version| flag:
%    \begin{macrocode}
%%\ifchilddoc\else\providecommand{\version}{draft}\fi
%    \end{macrocode}

% Define the default values for the |\version| flag
% (|final| for the main file and |draft| for childs):
%    \begin{macrocode}
\ifchilddoc
\providecommand{\version}{draft}
\else
\providecommand{\version}{final}
\fi
%    \end{macrocode}

% Load the standard document class:
%    \begin{macrocode}
\documentclass[12pt]{article}
%    \end{macrocode}

% Start the document body:
%    \begin{macrocode}
\begin{document}
%    \end{macrocode}

% Declare a title page.
% Print title, part of document being processed and version flag:
%    \begin{macrocode}
\addtocounter{page}{-1}
\begin{center}
{\LARGE\bfseries{}childdoc example\par}
\vspace{1cm}
\ifchilddoc
\ifchilddocmanual part\else chapter\fi:
`\childdocname' of `\childdocjob'\par
\else
main document: `\childdocjob'\par
\fi
version: \version\par
\end{center}
\newpage
%    \end{macrocode}

% Manually include selected file,
% otherwise process as usual:
%    \begin{macrocode}
\ifchilddocmanual
\section*{part `\childdocname'}
\input{\childdocname}
\else
%    \end{macrocode}

% Include the two chapters:
%    \begin{macrocode}
\include{cdocsch1}
\include{cdocsch2}
%    \end{macrocode}

% Include the two parts unless only chapters should be displayed:
%    \begin{macrocode}
\ifchilddoc\else
\section{part three}
\input{cdocspt3}
\section{part four}
\input{cdocspt4}
\fi
%    \end{macrocode}

% Process as usual until here:
%    \begin{macrocode}
\fi
%    \end{macrocode}

% End of document body:
%    \begin{macrocode}
\end{document}
%    \end{macrocode}
%\iffalse
%</samplemain>
%\fi
%
% %%%%%%%%%%%%%%%%%%%%%%%%%%%%%%%%%%%%%%
% \paragraph{Chapter Include Files.}
%
% The include files are called |cdocsch1.tex| and |cdocsch2.tex|.
%
%\iffalse
%<*samplechap1|samplechap2>
%\fi

% Optional override for |\version| flag:
%    \begin{macrocode}
%%\providecommand{\version}{final}
%    \end{macrocode}

% Include the main document:
%    \begin{macrocode}
\input{childdoc.def}
\childdocof{cdocsamp}
%    \end{macrocode}

%\iffalse
%</samplechap1|samplechap2>
%\fi
%
%\iffalse
%<*samplechap1>
%\fi
% Some text for chapter 1:
%    \begin{macrocode}
\section{one}
some text in chapter one
%    \end{macrocode}

%\iffalse
%</samplechap1>
%\fi
% Some text for chapter 2:
%\iffalse
%<*samplechap2>
%\fi
%    \begin{macrocode}
\section{two}
more text in chapter two
%    \end{macrocode}

%\iffalse
%</samplechap2>
%\fi
%
% %%%%%%%%%%%%%%%%%%%%%%%%%%%%%%%%%%%%%%
% \paragraph{Part Include Files.}
%
% The include files are called |cdocspt3.tex| and |cdocspt4.tex|.
%
%\iffalse
%<*samplepart3|samplepart4>
%\fi

% Optional override for |\version| flag:
%    \begin{macrocode}
%%\providecommand{\version}{final}
%    \end{macrocode}

% Include the main document:
%    \begin{macrocode}
\input{childdoc.def}
\childdocby{cdocsamp}
%    \end{macrocode}

%\iffalse
%</samplepart3|samplepart4>
%\fi
%
%\iffalse
%<*samplepart3>
%\fi
% Some text for part 3:
%    \begin{macrocode}
some text in part three
%    \end{macrocode}

%\iffalse
%</samplepart3>
%\fi
% Some text for part 4:
%\iffalse
%<*samplepart4>
%\fi
%    \begin{macrocode}
more text in part four
%    \end{macrocode}

%\iffalse
%</samplepart4>
%\fi
%
% %%%%%%%%%%%%%%%%%%%%%%%%%%%%%%%%%%%%%%
% \paragraph{Forwarding for a Complete Draft.}
%
% The following forwarding file |cdocsdrf.tex|
% compiles the main document in draft mode:
%\iffalse
%<*sampledraft>
%\fi
%    \begin{macrocode}
\def\version{draft}
\input{childdoc.def}
\childdocforward{cdocsamp}
%    \end{macrocode}

%\iffalse
%</sampledraft>
%\fi
%
% %%%%%%%%%%%%%%%%%%%%%%%%%%%%%%%%%%%%%%
% \paragraph{Forwarding for Final Version of the Chapters.}
%
% The following forwarding files |cdocsfn1.tex| and |cdocsfn2.tex|
% (with identical content)
% compile the final versions of the child documents
% |cdocsch1.tex| and |cdocsch2.tex|, respectively:
%\iffalse
%<*samplefinal>
%\fi
%    \begin{macrocode}
\def\version{final}
\input{childdoc.def}
\childdocforwardprefix[cdocsamp]{cdocsfn}{cdocsch}
%    \end{macrocode}

%\iffalse
%</samplefinal>
%\fi
%
% %%%%%%%%%%%%%%%%%%%%%%%%%%%%%%%%%%%%%%
% \paragraph{Command Line Processing.}
%
% The following three command lines generate the output files
% |cdocscld|, |cdocscl1| and |cdocscl2|
% which should be identical to
% |cdocsdrf|, |cdocsch1| and |cdocsfn2|, respectively:
% \begin{center}
% \begin{tabular}{l}
% |latex -jobname cdocscld \|\\
% |  "\def\version{draft}\input{childdoc.def}\childdocforward{cdocsamp}"|\\
% |latex -jobname cdocscl1 \|\\
% |  "\input{childdoc.def}\childdocforward[cdocsamp]{cdocsch1}"|\\
% |latex -jobname cdocscl2 \|\\
% |  "\def\version{final}\input{childdoc.def}\childdocforward{cdocsch2}"|
% \end{tabular}
% \end{center}
% Note that the trailing backslash on each first line
% merely continues the input to the second line
% (for convenient cut ant paste).
% Furthermore, the command |latex| can be replaced by any
% of its alternative versions such as |pdflatex|.
%
% %%%%%%%%%%%%%%%%%%%%%%%%%%%%%%%%%%%%%%%%%%%%%%%%%%%%%%%%%%%%%%%%%%%%%%%%%%%%%%
% %%%%%%%%%%%%%%%%%%%%%%%%%%%%%%%%%%%%%%%%%%%%%%%%%%%%%%%%%%%%%%%%%%%%%%%%%%%%%%
% \section{Implementation}
%\iffalse
%<*package>
%\fi
%
% This section describes the definitions file |childdoc.def|.

% The definitions cannot be loaded using |\usepackage| or |\RequirePackage|
% which has a mechanism to prevent loading a style file more than once.
% When loading the definitions by means of |\input|
% multiple instances have to be prevented manually:
%\iffalse
%This code needs to be before the `\ProvidesFile' directive
%which is defined at the beginning of this file.
%Therefore it is also placed there and commented out here.
%</package>
%<*discard>
%\fi
%    \begin{macrocode}
\ifdefined\childdocmain\endinput\fi
%    \end{macrocode}
%\iffalse
%</discard>
%<*package>
%\fi
%
% \macro{\ifchilddoc}
% \macro{\ifchilddocmanual}
% The conditional |\ifchilddoc| tells whether a
% child (true) or main (false) document is being compiled.
% The conditional |\ifchilddocmanual| tells whether
% the |\includeonly| mechanism is used (false) or
% the selection of child files must be performed manually (true).
% The definitions initialise to false:
%    \begin{macrocode}
\newif\ifchilddoc
\newif\ifchilddocmanual
%    \end{macrocode}

% \macro{\childdocname}
% \macro{\childdocjob}
% The macro |\childdocname| stores the name of the main document
% to be compiled. The macro |\childdocjob| stores the name of
% the document on which the \LaTeX{} compiler was originally invoked.
% The content of |\jobname| cannot be compared
% to filenames specified in the source due to different catcodes.
% The following code rescans |\jobname|, stores the result
% in |\childdocname| and saves a copy in |\childdocjob|:
%    \begin{macrocode}
\edef\childdocname{\scantokens\expandafter{\jobname\noexpand}}
\let\childdocjob\childdocname
%    \end{macrocode}

% \macro{\childdocdisable}
% The macro |\childdocdisable| prevents the main file
% from being processed more than once.
% At this stage, the main document command |\childdocmain|
% is assumed to be called once again where it should do nothing.
% Any subsequent call to it should prevent
% a secondary processing of the main document
% It overwrites the forwarding commands
% |\childdocof| and |\childdocforward|
% with empty macros to prevent further inclusions of the main document:
%    \begin{macrocode}
\newcommand{\childdocdisable}
{
  \renewcommand{\childdocmain}[1]{\renewcommand{\childdocmain}[1]{\endinput}}
  \renewcommand{\childdocof}[1]{}
  \renewcommand{\childdocby}[2][]{}
  \renewcommand{\childdocforward}[2][]{}
  \renewcommand{\childdocdisable}{}
}
%    \end{macrocode}

% \macro{\childdocmain}
% The macro |\childdocmain| is to be called at the top of the main file
% with nothing or the main filename (without extension) as argument.
% First, it breaks loops.
% If the argument is not empty and does not match |\childdocname|
% (which is set by the first inclusion of |childdoc.def|),
% |\ifchilddoc| is set to true, |\includeonly| is applied to the child file
% and |\jobname| is set to the main file
% (for proper handling of |.aux| files):
%    \begin{macrocode}
\newcommand{\childdocmain}[1]
{
  \childdocdisable\childdocmain{}
  \if?#1?\else
    \begingroup
      \def\childdoctmp{#1}
      \ifx\childdoctmp\childdocname
        \def\childdoctmp{}
      \else
        \def\childdoctmp
        {
          \childdoctrue
          \includeonly{\childdocname}
          \def\childdocjob{#1}
          \def\jobname{#1}
        }
      \fi
      \expandafter
    \endgroup
    \childdoctmp
  \fi
}
%    \end{macrocode}

% \macro{\childdocof}
% The command |\childdocof| redirects
% compilation to the main file |#1|.
%    \begin{macrocode}
\newcommand{\childdocof}[1]
{
  \childdocdisable
  \childdoctrue
  \includeonly{\childdocname}
  \def\jobname{#1}
  \def\childdocjob{#1}
  \input{#1}
}
%    \end{macrocode}

% \macro{\childdocby}
% The command |\childdocby| ....
%    \begin{macrocode}
\newcommand{\childdocby}[2][]
{
  \childdocdisable
  \childdoctrue
  \childdocmanualtrue
  \if?#1?\else
    \def\jobname{#2}
  \fi
  \def\childdocjob{#2}
  \input{#2}
  \endinput
}
%    \end{macrocode}

% \macro{\childdocforward}
% The command |\childdocforward| redirects
% compilation to the main file or
% (if the optional argument is given) a child file.
% Parameters are set as if the main file
% or a child file starting with |\childdocof| was compiled.
% Then compilation is handed over to the main file:
%    \begin{macrocode}
\newcommand{\childdocforward}[2][]
{
  \begingroup
    \if?#1?
      \def\childdoctmp
      {
        \def\childdocname{#2}
        \def\childdocjob{#2}
        \def\jobname{#2}
        \input{#2}
        \endinput
      }
    \else
      \def\childdoctmp
      {
        \childdocdisable
        \def\childdocname{#2}
        \childdoctrue
        \includeonly{#2}
        \def\childdocjob{#1}
        \def\jobname{#1}
        \input{#1}
        \endinput
      }
    \fi
    \expandafter
  \endgroup
  \childdoctmp
}
%    \end{macrocode}

% \macro{\childdocforwardprefix}
% The command |\childdocforwardprefix| redirects
% compilation to the main or a child file by means of a pattern.
% The prefix |#1| in the current filename is replaced by |#2|
% and the suffix of the current filename is kept
% (it is assumed that the filename does not contain the substring `|~~~|'
% which is used as a delimiter).
% Compilation is handed over to the new file by |\childdocforward|:
%    \begin{macrocode}
\newcommand{\childdocforwardprefix}[3][]
{
  \begingroup
    \def\childdocextract #2##1~~~{\def\childdoctmp{\childdocforward[#1]{#3##1}}}
    \expandafter\childdocextract\childdocname~~~
    \expandafter
  \endgroup
  \childdoctmp
}
%    \end{macrocode}

% \macro{\childdoc}
% The deprecated macro |\childdoc| is a legacy version of |\childdocmain|:
%    \begin{macrocode}
\newcommand{\childdoc}{\childdocmain}
%    \end{macrocode}

% \macro{\childdocredirect}
% The deprecated macro |\childdocredirect| is a legacy version
% of |\childdocforward| and |\childdocforwardprefix|:
%    \begin{macrocode}
\newcommand{\childdocredirect}[2][]
{
  \begingroup
    \if?#1?
      \def\childdoctmp{\childdocforward{#2}}
    \else
      \def\childdoctmp{\childdocforwardprefix{#1}{#2}}
    \fi
    \expandafter
  \endgroup
  \childdoctmp
}
%    \end{macrocode}

%\iffalse
%</package>
%\fi
%
\endinput
|\\
|\childdocforward[|\textit{main}|]{|\textit{dest}|}|\\
\end{tabular}
\end{center}
%
The argument \textit{dest} is the destination file
(without extension).
It should be the main file or one of the child files.
Note that further \textsf{childdoc} directives
such as |\childdocof| and |\childdocforward|
in the indicated file will be processed in this form.
The optional argument \textit{main}
passes on directly to the main file \textit{main}
while pretending to compile the child \textit{dest}.
This form behaves as if \textit{dest}
issues |\childdocof{|\textit{main}|}| right away,
and no further \textsf{childdoc} directives will be processed.

%%%%%%%%%%%%%%%%%%%%%%%%%%%%%%%%%%%%%%%%
\DescribeMacro{\...prefix}
In the alternative form |\childdocforwardprefix|,
%
\begin{center}
\begin{tabular}{l}
|% \iffalse
%
% childdoc.dtx Copyright (C) 2017-2018 Niklas Beisert
%
% This work may be distributed and/or modified under the
% conditions of the LaTeX Project Public License, either version 1.3
% of this license or (at your option) any later version.
% The latest version of this license is in
%   http://www.latex-project.org/lppl.txt
% and version 1.3 or later is part of all distributions of LaTeX
% version 2005/12/01 or later.
%
% This work has the LPPL maintenance status `maintained'.
%
% The Current Maintainer of this work is Niklas Beisert.
%
% This work consists of the files childdoc.dtx and childdoc.ins
% and the derived files childdoc.def and cdocsamp.tex with
% cdocsch1.tex, cdocsch2.tex, cdocsdrf.tex, cdocsfn1.tex, cdocsfn2.tex.
%
%<package>\ifdefined\childdocmain\endinput\fi
%<package>\ProvidesFile{childdoc.def}[2018/12/30 v2.0 child document driver]
%<samplemain>\ProvidesFile{cdocsamp.tex}[2018/12/30 v2.0 sample for childdoc]
%<*driver>
%\ProvidesFile{childdoc.drv}[2018/12/30 v2.0 childdoc reference manual file]
\PassOptionsToClass{10pt,a4paper}{article}
\documentclass{ltxdoc}

\usepackage[margin=35mm]{geometry}
\usepackage{hyperref}
\usepackage{hyperxmp}
\usepackage[usenames]{color}

\hypersetup{colorlinks=true}
\hypersetup{pdfstartview=FitH}
\hypersetup{pdfpagemode=UseNone}
\hypersetup{pdfsource={}}
\hypersetup{pdflang={en-UK}}
\hypersetup{pdfcopyright={Copyright 2017-2018 Niklas Beisert.
  This work may be distributed and/or modified under the
  conditions of the LaTeX Project Public License, either version 1.3
  of this license or (at your option) any later version.}}
\hypersetup{pdflicenseurl={http://www.latex-project.org/lppl.txt}}
\hypersetup{pdfcontactaddress={ETH Zurich, ITP, HIT K,
  Wolfgang-Pauli-Strasse 27}}
\hypersetup{pdfcontactpostcode={8093}}
\hypersetup{pdfcontactcity={Zurich}}
\hypersetup{pdfcontactcountry={Switzerland}}
\hypersetup{pdfcontactemail={nbeisert@itp.phys.ethz.ch}}
\hypersetup{pdfcontacturl={http://people.phys.ethz.ch/\xmptilde nbeisert/}}

\newcommand{\secref}[1]{\hyperref[#1]{section \ref*{#1}}}

\parskip1ex
\parindent0pt
\let\olditemize\itemize
\def\itemize{\olditemize\parskip0pt}

\begin{document}

\title{The \textsf{childdoc} Package}
\hypersetup{pdftitle={The childdoc Package}}
\author{Niklas Beisert\\[2ex]
  Institut f\"ur Theoretische Physik\\
  Eidgen\"ossische Technische Hochschule Z\"urich\\
  Wolfgang-Pauli-Strasse 27, 8093 Z\"urich, Switzerland\\[1ex]
  \href{mailto:nbeisert@itp.phys.ethz.ch}
  {\texttt{nbeisert@itp.phys.ethz.ch}}}
\hypersetup{pdfauthor={Niklas Beisert}}
\hypersetup{pdfsubject={Manual for the LaTeX2e Package childdoc}}
\date{30 December 2018, \textsf{v2.0}}
\maketitle

\begin{abstract}\noindent
\textsf{childdoc} is a \LaTeXe{} package
that enables the direct compilation
of document sections included by |\include|
to individual files.
\end{abstract}

\begingroup
\parskip0ex
\tableofcontents
\endgroup

%%%%%%%%%%%%%%%%%%%%%%%%%%%%%%%%%%%%%%%%%%%%%%%%%%%%%%%%%%%%%%%%%%%%%%%%%%%%%%%%
%%%%%%%%%%%%%%%%%%%%%%%%%%%%%%%%%%%%%%%%%%%%%%%%%%%%%%%%%%%%%%%%%%%%%%%%%%%%%%%%
\section{Introduction}

\LaTeX{} provides a mechanism to structure a large document (such as a book)
into a main file and several child files (containing the chapters)
using the |\include| command.
This mechanism is beneficial for documents
which span hundreds of pages in order to
make the source file(s) more manageable.
Moreover, compilation can be restricted to
selected child files by means of the |\includeonly| command.
The latter feature can be used to reduce the compilation time while editing
(this was significantly more useful in the earlier days of \LaTeX{})
or to generate a smaller document which is easier to navigate.
Another application of |\includeonly| is to generate
documents consisting of selected parts of the complete document.

However, there are a few drawbacks of the plain |\include| mechanism:
\begin{itemize}
\item
The child files cannot be compiled on their own,
they can only be compiled via the main file.
A naive editing environment
(such as a text editor with an option
to have the current file processed by \LaTeX)
may require one to switch to the main file before compiling;
attempting to compile the child file produces errors.
\item
The main file must be modified (each time)
to adjust the |\includeonly| command
to the present needs. This easily leaves the main file in a messy state.
\item
The generated document will always carry the filename
of the main document. This is inconvenient if
several child files are to be compiled and
to be kept for distribution.
\end{itemize}

The present package provides a simple interface
to make child files individually compilable by \LaTeX{}.
Compiling a child file then has the same effect as compiling
the main file with an |\includeonly| command
to select the appropriate child.
Moreover the generated document will carry the name of the child
rather than the main file.
This resolves all three above issues.

This feature is meant to make the editing of books,
thesis documents and lecture notes somewhat more convenient.
However, the package can also be used efficiently for
composing a series of documents (such as exercise sheets)
which are typically distributed individually.
It then assists the author in generating the individual documents
(potentially in different versions)
as well as a document containing the collected series.
Another application is in developing style files
or other kinds of included material
where compilation of the style file could redirect
to a sample or test file.

%%%%%%%%%%%%%%%%%%%%%%%%%%%%%%%%%%%%%%%%%%%%%%%%%%%%%%%%%%%%%%%%%%%%%%%%%%%%%%%%
%%%%%%%%%%%%%%%%%%%%%%%%%%%%%%%%%%%%%%%%%%%%%%%%%%%%%%%%%%%%%%%%%%%%%%%%%%%%%%%%
\section{Usage}

First of all, the package \textsf{childdoc} is \emph{not} a standard
\LaTeXe{} |.sty| style file! Therefore it needs to be invoked in
a non-standard way.

%%%%%%%%%%%%%%%%%%%%%%%%%%%%%%%%%%%%%%%%%%%%%%%%%%%%%%%%%%%%%%%%%%%%%%%%%%%%%%%%
\subsection{Included Files}
\label{sec:include}

%%%%%%%%%%%%%%%%%%%%%%%%%%%%%%%%%%%%%%%%
\DescribeMacro{\childdocmain}
To use the package, add the commands
\begin{center}
\begin{tabular}{l}
|\input{childdoc.def}|\\
|\childdocmain{}|\\
\end{tabular}
\end{center}
at the very top of the main \LaTeX{} file,
in particular \emph{before} the |\documentclass| statement!
The argument of |\childdocmain| should be left empty
(but it must be present).

%%%%%%%%%%%%%%%%%%%%%%%%%%%%%%%%%%%%%%%%
\DescribeMacro{\childdocof}
Furthermore, add the commands
\begin{center}
\begin{tabular}{l}
|\input{childdoc.def}|\\
|\childdocof{|\textit{main}|}|\\
\end{tabular}
\end{center}
at the top of every child file \textit{child}
which is included by |\include{|\textit{child}|}|
from within the main file
(or at least for those files to be compiled individually).
The argument \textit{main} must be the filename of the main file.

There are a couple of
considerations in setting up the main and child documents:

%%%%%%%%%%%%%%%%%%%%%%%%%%%%%%%%%%%%%%%%
\paragraph{Restrictions.}

Please note the following restrictions:
\begin{itemize}
\item
|\childdocmain| must be called with one argument \textit{main}
to ensure compatibility with earlier version of the package.
It must either be empty (|\childdocmain{}|)
or precisely match the filename of the main file in which it is specified.
See \secref{sec:detection} for further information.
\item
The filename \textit{main} must be specified without the |.tex| extension.
\item
The filename \textit{main} is case sensitive
(even in case-insensitive file systems)
due to internal string comparison.
\item
The argument \textit{main} should be fully expanded, it cannot be a macro.
\item
Subdirectories and special characters should be avoided in filenames.
\item
The command |\childdocmain{|\textit{main}|}| must be followed by a whitespace.
It should not be followed immediately by another command
or by a comment mark `|%|'.
This is because the \TeX{} parser reads the token immediately following
the argument of |\childdocmain| and puts it
at the beginning of every child section;
however, a white\-space is ignored.
\end{itemize}

%%%%%%%%%%%%%%%%%%%%%%%%%%%%%%%%%%%%%%%%
\paragraph{Content of Main File.}

It is advisable to place all content in the child files included by |\include|.
Any output contained in the main file will appear in all child documents
unless suppressed manually;
it cannot be suppressed automatically by the |\includeonly| directive
and thus should normally be avoided.
A method to include some content in the main file
by means of conditional processing is described in \secref{sec:conditional}.

%%%%%%%%%%%%%%%%%%%%%%%%%%%%%%%%%%%%%%%%
\paragraph{Page Numbering.}

When only a part of the document is compiled,
the appropriate numbering of pages
(as well as other status parameters)
is determined from the |.aux| files.
The latter contain information from previous passes.
However this information needs to propagate through
all intermediate child documents.
Therefore the page numbering in child documents may well
be inconsistent until the complete document is compiled at least once.

A useful (if unconventional) way to always ensure a consistent
page numbering is to restart the numbering in each child document
and denote the pages by `\textit{child}|.|\textit{page}'
where \textit{child} represents the chapter/section number of the child file.
This can be achieved by the command
|\numberwithin{page}{|\textit{child}|}|
of the \textsf{amsmath} package
where \textit{child} can be |chapter| or |section|
depending on the chosen structuring.
Alternatively, one can modify the macro |\thepage| appropriately
and reset the counter |page| at the start of each child file.

%%%%%%%%%%%%%%%%%%%%%%%%%%%%%%%%%%%%%%%%%%%%%%%%%%%%%%%%%%%%%%%%%%%%%%%%%%%%%%%%
\subsection{Conditional Processing}
\label{sec:conditional}

The package provides a mechanism to compile different versions
of a document. To customise the versions further some conditional processing
can come in handy to distinguish which version is being compiled.
The package provides two macros to describe the compilation context:

%%%%%%%%%%%%%%%%%%%%%%%%%%%%%%%%%%%%%%%%
\DescribeMacro{\ifchilddoc}
The conditional |\ifchilddoc| distinguishes between the compilation of
child documents and the main document:
%
\begin{center}
|\ifchilddoc |\textit{child-code}| |[|\||else |\textit{main-code}]| \||fi|
\end{center}

%%%%%%%%%%%%%%%%%%%%%%%%%%%%%%%%%%%%%%%%
\DescribeMacro{\childdocname}
\DescribeMacro{\childdocjob}
The macro |\childdocname| contains the filename (without extension)
of the main or child file being processed.
Note that |\childdocjob| will always contain the name of the main file.

%%%%%%%%%%%%%%%%%%%%%%%%%%%%%%%%%%%%%%%%
\paragraph{Title Page.}

Conditional processing can be used to include a title or banner page
in the main document when proper precautions are taken.
Importantly, the code in the main file should ensure that the page counter
(as well as other status parameters which are stored in the |.aux| files)
takes the same value after the conditional processing.
Otherwise the page numbers may take divergent values
depending on which part is compiled.

For example, a title page could be declared by:
%
\begin{center}
\begin{tabular}{l}
|\ifchilddoc\||else|\\
|\addtocounter{page}{-1}|\\
\textit{code for title page}\\
|\newpage|\\
|\||fi|
\end{tabular}
\end{center}
%
A banner page for the child documents can be generated by:
%
\begin{center}
\begin{tabular}{l}
|\ifchilddoc|\\
|\addtocounter{page}{-1}|\\
\textit{code for banner page}\\
|\newpage|\\
|\||fi|
\end{tabular}
\end{center}
%
Here one could write a message such as:
\begin{center}
|This is the part \childdocname{} of \childdocjob{}.|
\end{center}

%%%%%%%%%%%%%%%%%%%%%%%%%%%%%%%%%%%%%%%%%%%%%%%%%%%%%%%%%%%%%%%%%%%%%%%%%%%%%%%%
\subsection{Flags}
\label{sec:flags}

The package makes it easy to generate different versions
of the main or child documents.
To this end compilation flags can be defined
and assigned different default values.
They will be particularly useful in conjunction
with the forwarding mechanism described in \secref{sec:forward}.

For example, it may be useful to have a flag |\version|
which can be set to |draft| or |final|.
The document source will contain some conditional code
depending on the value of |\version|.
Suppose further, the flag should default to |final| for the main file
and to |draft| for child files
which is a natural assignment for editing the document.
This is achieved by placing the following code
in the preamble of the main document
(below the |\childdocmain| directive):
%
\begin{center}
\begin{tabular}{l}
|\ifchilddoc|\\
|\providecommand{\version}{draft}|\\
|\||else|\\
|\providecommand{\version}{final}|\\
|\||fi|
\end{tabular}
\end{center}
%
The definition by |\providecommand| makes sure
that previous definitions are not overwritten.
Further statements |\providecommand{\version}{...}|
can thus be added before the above code to override it.

For the main file, one might add a line
(between |\childdocmain| and the above block)
%
\begin{center}
|%\ifchilddoc\||else\providecommand{\version}{draft}\||fi|
\end{center}
%
which can be uncommented to produce a draft version.
Likewise one can add a line to the very top of a child file
(above the |\childdocof{|\textit{main}|}| directive)
%
\begin{center}
|%\providecommand{\version}{final}|
\end{center}
%
which can be uncommented to produce the final version of this child document.

%%%%%%%%%%%%%%%%%%%%%%%%%%%%%%%%%%%%%%%%%%%%%%%%%%%%%%%%%%%%%%%%%%%%%%%%%%%%%%%%
\subsection{Forwarding}
\label{sec:forward}

Different versions of the main or child documents
using compilation flags as described in \secref{sec:flags}
can be (permanently) stored in different files
for convenient compilation, viewing and distribution.
To this end, the package defines a command
to pass on compilation to a different file:

%%%%%%%%%%%%%%%%%%%%%%%%%%%%%%%%%%%%%%%%
\DescribeMacro{\childdocforward}
The command |\childdocforward| redirects processing to
another source file:
%
\begin{center}
\begin{tabular}{l}
|\input{childdoc.def}|\\
|\childdocforward[|\textit{main}|]{|\textit{dest}|}|\\
\end{tabular}
\end{center}
%
The argument \textit{dest} is the destination file
(without extension).
It should be the main file or one of the child files.
Note that further \textsf{childdoc} directives
such as |\childdocof| and |\childdocforward|
in the indicated file will be processed in this form.
The optional argument \textit{main}
passes on directly to the main file \textit{main}
while pretending to compile the child \textit{dest}.
This form behaves as if \textit{dest}
issues |\childdocof{|\textit{main}|}| right away,
and no further \textsf{childdoc} directives will be processed.

%%%%%%%%%%%%%%%%%%%%%%%%%%%%%%%%%%%%%%%%
\DescribeMacro{\...prefix}
In the alternative form |\childdocforwardprefix|,
%
\begin{center}
\begin{tabular}{l}
|\input{childdoc.def}|\\
|\childdocforwardprefix[|\textit{main}|]{|\textit{prefix}|}{|\textit{dest}|}|
\end{tabular}
\end{center}
%
the destination file is determined by a pattern
depending on the current file:
To make this work, the current file must be called
`{\textit{prefix}\hspace{0.2em}\textit{suffix}}'
with \textit{prefix} matching precisely the argument.
Processing is then passed on to the file
`{\textit{dest}\hspace{0.2em}\textit{suffix}}'.
Surely, the same effect is achieved by
directly specifying the
argument `{\textit{dest}\hspace{0.2em}\textit{suffix}}'
in the first form.
However, that requires to set up a different file
for each child. With the alternative form of the command
all these files can have exactly the same content
which simplifies setting them up and maintaining them.

For example, the following file |draft.tex|
with a compilation flag |\version| as described in \secref{sec:flags}
compiles the main document as a draft:
%
\begin{center}
\begin{tabular}{l}
|\def\version{draft}|\\
|\input{childdoc.def}|\\
|\childdocforward{|\textit{main}|}|
\end{tabular}
\end{center}
%
Likewise, the following files |final|\textit{nn}|.tex|
compile the final version of the child document
|child|\textit{nn}|.tex|:
%
\begin{center}
\begin{tabular}{l}
|\def\version{final}|\\
|\input{childdoc.def}|\\
|\childdocforwardprefix{final}{child}|
\end{tabular}
\end{center}
%

Note that when several versions of a main file and/or of each child file
are to be generated, it may be convenient to set up a |Makefile| or
shell script to automatise the process.

%%%%%%%%%%%%%%%%%%%%%%%%%%%%%%%%%%%%%%%%%%%%%%%%%%%%%%%%%%%%%%%%%%%%%%%%%%%%%%%%
\subsection{Command Line Processing}
\label{sec:commandline}

The effect of redirection files can also be achieved by invoking
the \LaTeX{} compiler with a more elaborate command line.
Most conveniently this should be done as part
of a shell script or a |Makefile|.

When using \textsf{childdoc} in the main file, the following
command lines effectively perform a redirection
(note that depending on the shell being used,
backslashes may have to be doubled: `|\|' $\to$ `|\\|'):
%
\begin{center}
|... -jobname "|\textit{target}|" |\\|"|[\textit{flags}]%
|\input{childdoc.def}\childdocforward[|\textit{main}|]{|\textit{dest}|}"|
\end{center}
%
Here \textit{target} is the name of the output file,
\textit{main} is the name of the main file
and \textit{dest} is the name of the main or child file to be processed
(all filenames without extensions).
The optional argument \textit{main} can be omitted
if \textit{main} matches \textit{dest}.
Optionally, compilation \textit{flags} can be defined via |\def| commands.
This command line makes the \TeX{} engine believe
it is compiling the file \textit{target}
whose content is specified as the latter parameter.
The provided code then forwards the processing to
\textit{main} or \textit{dest} as described in \secref{sec:forward}.

%%%%%%%%%%%%%%%%%%%%%%%%%%%%%%%%%%%%%%%%%%%%%%%%%%%%%%%%%%%%%%%%%%%%%%%%%%%%%%%%
\subsection{Include by Input}
\label{sec:input}

Including child documents by |\include| has some restrictions by design.
Most notably, the content of a child document always occupies
its own set of pages; pages cannot be shared between child documents.
Usually, this behaviour makes perfect sense
because each child document contain an essential part of the document.
However, in some situations it may be desirable to compose
a document from a collection of parts
without having mandatory page breaks between then.
For this case, the package
provides a mechanism to include parts
by |\input| which can also be processed individually.
However, by construction this mechanism
requires manual handling of the content to be output.

%%%%%%%%%%%%%%%%%%%%%%%%%%%%%%%%%%%%%%%%
\DescribeMacro{\ifchilddocmanual}
The main file should be prepared as usual, see \secref{sec:include}.
However, the document body must make a distinction
between processing of an individual part and of the main document, e.g.:
%
\begin{center}
\begin{tabular}{l}
|\ifchilddocmanual|\\
|\input{\childdocname}|\\
|\||else|\\
\textit{document body with }|\input{|\textit{part}|}|\\
|\||fi|
\end{tabular}
\end{center}
%
The conditional |\ifchilddocmanual| is true whenever
a part to be included by |\input| is being compiled,
and the name of the part is stored in |\childdocname|.

%%%%%%%%%%%%%%%%%%%%%%%%%%%%%%%%%%%%%%%%
\DescribeMacro{\childdocby}
Each part to be included by |\input| should start with:
%
\begin{center}
\begin{tabular}{l}
|\input{childdoc.def}|\\
|\childdocby{|\textit{main}|}|\\
\end{tabular}
\end{center}
%
The directive |\childdocby| is similar to |\childdocof|
described in \secref{sec:include},
but the subsequent selection of content must be done manually.
To that end, both |\ifchilddoc| and |\ifchilddocmanual|
will be true upon processing of a part,
and the name of the part is stored in |\childdocname|.
Note that |\jobname| will be set to the filename of the current part
so that each part receives an individual |.aux| file
that does not interfere with the |.aux| file(s) of the main document.
This behaviour can be altered by the alternative form
|\childdocby[*]{|\textit{main}|}| (with a non-empty optional argument)
which uses the |.aux| file of the main document
by setting |\jobname| to \textit{main}.

%%%%%%%%%%%%%%%%%%%%%%%%%%%%%%%%%%%%%%%%%%%%%%%%%%%%%%%%%%%%%%%%%%%%%%%%%%%%%%%%
\subsection{Driver Development}
\label{sec:driver}

The \textsf{childdoc} mechanism can also be use for the development
of definition files such as \LaTeX{} styles or classes.
This case differs from the above setup with multiple parts
included by |\include| in that no |\includeonly| should be invoked.
This can be achieved by starting the include file
(before |\ProvidesPackage|) with:
%
\begin{center}
\begin{tabular}{l}
|\input{childdoc.def}|\\
|\childdocforward{|\textit{main}|}|\\
\end{tabular}
\end{center}
%
or alternatively with:
%
\begin{center}
\begin{tabular}{l}
|\input{childdoc.def}|\\
|\childdocby{|\textit{main}|}|\\
\end{tabular}
\end{center}
%
Both forms have slightly different effects as described above.
The main file is prepared as usual, see \secref{sec:include}.

%%%%%%%%%%%%%%%%%%%%%%%%%%%%%%%%%%%%%%%%%%%%%%%%%%%%%%%%%%%%%%%%%%%%%%%%%%%%%%%%
\subsection{Legacy Detection}
\label{sec:detection}

The directive |\childdocmain| in the main file can detect
whether the complete document or merely a child is to be compiled
even without using the directive |\childdocof|.
This method is deprecated because it is less robust
and there is no compelling reason to use it;
it is merely provided for backward compatibility
and it may be removed in future versions.

If the detection mechanism is to be used,
it is mandatory to correctly specify
the filename of the main file as the argument of |\childdocmain|:
%
\begin{center}
\begin{tabular}{l}
|\input{childdoc.def}|\\
|\childdocmain{|\textit{main}|}|\\
\end{tabular}
\end{center}
%
If |\jobname| does not match the argument \textit{main} of |\childdocmain|,
it is assumed that |\jobname| points to the child file to be compiled.
When using |\childdocmain| with the main file specified as argument,
it suffices to start a child file
with just |\input{|\textit{main}|}|
without loading of the package and using |\childdocof|.
If instead all processing is done
with the appropriate \textsf{childdoc} directives,
the argument of \textit{main} of |\childdocmain| can be empty.

An alternative version of the command line processing described
in \secref{sec:commandline} using the detection mechanism reads:
%
\begin{center}
|... -jobname "|\textit{target}|" "|[\textit{flags}]%
[|\def\jobname{|\textit{dest}|}|]|\input{|\textit{main}|}"|
\end{center}

%%%%%%%%%%%%%%%%%%%%%%%%%%%%%%%%%%%%%%%%%%%%%%%%%%%%%%%%%%%%%%%%%%%%%%%%%%%%%%%%
\subsection{Manual Code}
\label{sec:manual}

In case one cannot be certain whether the definitions file |childdoc.def|
is installed on the target \TeX{} distribution
and one prefers not to ship it,
it is conceivable to paste a few relevant commands into the sources.

To that end, drop all statements |\input{childdoc.def}|
and perform the replacements as outlined below.
Instead of |\childdocmain{|\textit{main}|}| add the following code
to the top of the main file:
%
\begin{center}
\begin{tabular}{l}
|\||ifdefined\childdocname\endinput\||fi\newif\ifchilddoc|\\
|\edef\childdocname{\scantokens\expandafter{\jobname\noexpand}}|\\
|\def\childdocmain{|\textit{main}|}\||ifx\childdocmain\childdocname\||else|\\
|\childdoctrue\includeonly{\childdocname}\let\jobname\childdocmain\||fi|\\
\end{tabular}
\end{center}
%
Instead of |\childdocof{|\textit{main}|}| just include the main file
at the top of each child file:
%
\begin{center}
|\input{|\textit{main}|}|
\end{center}
%
A simple redirection |\childdocforward{|\textit{dest}|}| is achieved by:
%
\begin{center}
|\def\jobname{|\textit{dest}|}\input{\jobname}|
\end{center}
%
The redirection with prefix
|\childdocforwardprefix[|\textit{prefix}|]{|\textit{dest}|}|
is accomplished by:
%
\begin{center}
\begin{tabular}{l}
|{\edef\jobname{\scantokens\expandafter{\jobname\noexpand}}|\\
|\def\redirectjob |\textit{prefix}|#1~~~{\gdef\jobname{|\textit{dest}|#1}}|\\
|\expandafter\redirectjob\jobname~~~}\input{\jobname}|
\end{tabular}
\end{center}

In an alternative approach,
child documents can be compiled by a specific command line
without additional code or specific definitions:
%
\begin{center}
|... -jobname "|\textit{target}|" "|[\textit{flags}]%
|\includeonly{|\textit{dest}|}\input{|\textit{main}|}"|
\end{center}
%

%%%%%%%%%%%%%%%%%%%%%%%%%%%%%%%%%%%%%%%%%%%%%%%%%%%%%%%%%%%%%%%%%%%%%%%%%%%%%%%%
%%%%%%%%%%%%%%%%%%%%%%%%%%%%%%%%%%%%%%%%%%%%%%%%%%%%%%%%%%%%%%%%%%%%%%%%%%%%%%%%
\section{Information}

%%%%%%%%%%%%%%%%%%%%%%%%%%%%%%%%%%%%%%%%%%%%%%%%%%%%%%%%%%%%%%%%%%%%%%%%%%%%%%%%
\subsection{Copyright}

Copyright \copyright{} 2017--2018 Niklas Beisert

This work may be distributed and/or modified under the
conditions of the \LaTeX{} Project Public License, either version 1.3
of this license or (at your option) any later version.
The latest version of this license is in
  \url{http://www.latex-project.org/lppl.txt}
and version 1.3 or later is part of all distributions of \LaTeX{}
version 2005/12/01 or later.

This work has the LPPL maintenance status `maintained'.

The Current Maintainer of this work is Niklas Beisert.

This work consists of the files |README.txt|, |childdoc.ins| and |childdoc.dtx|
as well as the derived files |childdoc.def|, |cdocsamp.tex|
with |cdocsch1.tex|, |cdocsch2.tex|, |cdocspt3.tex|, |cdocspt4.tex|,
|cdocsdrf.tex|, |cdocsfn1.tex|, |cdocsfn2.tex|
as well as |childdoc.pdf|.

%%%%%%%%%%%%%%%%%%%%%%%%%%%%%%%%%%%%%%%%%%%%%%%%%%%%%%%%%%%%%%%%%%%%%%%%%%%%%%%%
\subsection{Files and Installation}

The package consists of the files:
%
\begin{center}
\begin{tabular}{ll}
    |README.txt|   & readme file \\
    |childdoc.ins| & installation file \\
    |childdoc.dtx| & source file \\
    |childdoc.def| & definition file \\
    |cdocsamp.tex| & sample main file \\
    |cdocsch1.tex| & sample include file \\
    |cdocsch2.tex| & sample include file \\
    |cdocspt3.tex| & sample part file \\
    |cdocspt4.tex| & sample part file \\
    |cdocsdrf.tex| & sample redirection file \\
    |cdocsfn1.tex| & sample redirection file \\
    |cdocsfn2.tex| & sample redirection file \\
    |childdoc.pdf| & manual
\end{tabular}
\end{center}
%
The distribution consists of the files
|README.txt|, |childdoc.ins| and |childdoc.dtx|.
%
\begin{itemize}
\item
Run (pdf)\LaTeX{} on |childdoc.dtx|
to compile the manual |childdoc.pdf| (this file).
\item
Run \LaTeX{} on |childdoc.ins| to create the definitions file |childdoc.def|
and the sample |cdocsamp.tex| with include files
|cdocsch1.tex|, |cdocsch2.tex|, |cdocspt3.tex|, |cdocspt4.tex|,
|cdocsdrf.tex|, |cdocsfn1.tex|, |cdocsfn2.tex|.
Then copy the file |childdoc.def| to an appropriate directory of your \LaTeX{}
distribution, e.g.\ \textit{texmf-root}|/tex/latex/childdoc|.
\end{itemize}

%%%%%%%%%%%%%%%%%%%%%%%%%%%%%%%%%%%%%%%%%%%%%%%%%%%%%%%%%%%%%%%%%%%%%%%%%%%%%%%%
\subsection{Related CTAN Packages}

There are several other packages which offer a similar functionality:
%
\begin{itemize}
\item
The packages
\href{http://ctan.org/pkg/docmute}{\textsf{docmute}},
\href{http://ctan.org/pkg/includex}{\textsf{includex}} and
\href{http://ctan.org/pkg/standalone}{\textsf{standalone}}
provide commands to include only the document body of
a child file thus allowing both files to be compiled individually.
\item
The packages \href{http://ctan.org/pkg/subdocs}{\textsf{subdocs}}
and \href{http://ctan.org/pkg/subfiles}{\textsf{subfiles}}
provide structures in which the main and child documents can be
encapsulated and allowing them to be compiled individually.
The inclusion mechanism is different from the conventional |\include|.
\item
The package \href{http://ctan.org/pkg/combine}{\textsf{combine}}
is an elaborate solution to combine several documents into one.
\end{itemize}
%
See also the CTAN topic \href{http://ctan.org/topic/subdocs}{\textsf{subdocs}}
for further related packages.
The present package differs from the above solutions in that
a document structure constructed with the conventional |\include| mechanism
just needs two extra commands at the top of every file
such that all constituent files can be compiled individually.

%%%%%%%%%%%%%%%%%%%%%%%%%%%%%%%%%%%%%%%%%%%%%%%%%%%%%%%%%%%%%%%%%%%%%%%%%%%%%%%%
%\subsection{Feature Suggestions}
%
%The following is a list of features which may be useful for future
%versions of this package:
%%
%\begin{itemize}
%\item
%\ldots
%\end{itemize}

%%%%%%%%%%%%%%%%%%%%%%%%%%%%%%%%%%%%%%%%%%%%%%%%%%%%%%%%%%%%%%%%%%%%%%%%%%%%%%%%
\subsection{Revision History}

%%%%%%%%%%%%%%%%%%%%%%%%%%%%%%%%%%%%%%%%
\paragraph{v2.0:} 2018/12/30

\begin{itemize}
\item
immediate forward processing
\item
added |\childdocby| mechanism
\item
manual restructured
\end{itemize}

%%%%%%%%%%%%%%%%%%%%%%%%%%%%%%%%%%%%%%%%
\paragraph{v1.6:} 2018/01/17

\begin{itemize}
\item
application for development of include files
\item
corrections to manual
\end{itemize}

%%%%%%%%%%%%%%%%%%%%%%%%%%%%%%%%%%%%%%%%
\paragraph{v1.5:} 2017/05/21

\begin{itemize}
\item
more complete structuring introduced
\item
|\childdocof| introduced
\item
|\childdoc| renamed to |\childdocmain|
\item
|\childredirect| renamed to |\childdocforward| and |\childdocforwardprefix|
and functionality expanded
\end{itemize}

%%%%%%%%%%%%%%%%%%%%%%%%%%%%%%%%%%%%%%%%
\paragraph{v1.0:} 2017/04/27

\begin{itemize}
\item
manual and install package
\item
first version published on CTAN
\end{itemize}

%%%%%%%%%%%%%%%%%%%%%%%%%%%%%%%%%%%%%%%%
\paragraph{v0.6:} 2017/04/26

\begin{itemize}
\item
redirection mechanism added
\end{itemize}

%%%%%%%%%%%%%%%%%%%%%%%%%%%%%%%%%%%%%%%%
\paragraph{v0.5:} 2017/04/26

\begin{itemize}
\item
functionality in definition file
\end{itemize}


%%%%%%%%%%%%%%%%%%%%%%%%%%%%%%%%%%%%%%%%%%%%%%%%%%%%%%%%%%%%%%%%%%%%%%%%%%%%%%%%
%%%%%%%%%%%%%%%%%%%%%%%%%%%%%%%%%%%%%%%%%%%%%%%%%%%%%%%%%%%%%%%%%%%%%%%%%%%%%%%%
%%%%%%%%%%%%%%%%%%%%%%%%%%%%%%%%%%%%%%%%%%%%%%%%%%%%%%%%%%%%%%%%%%%%%%%%%%%%%%%%
\appendix

\settowidth\MacroIndent{\rmfamily\scriptsize 000\ }

 \DocInput{childdoc.dtx}

\end{document}
%</driver>
% \fi
%
% %%%%%%%%%%%%%%%%%%%%%%%%%%%%%%%%%%%%%%%%%%%%%%%%%%%%%%%%%%%%%%%%%%%%%%%%%%%%%%
% %%%%%%%%%%%%%%%%%%%%%%%%%%%%%%%%%%%%%%%%%%%%%%%%%%%%%%%%%%%%%%%%%%%%%%%%%%%%%%
% \section{Sample}
%\iffalse
%<*samplemain>
%\fi
%
% The following presents a sample document
% with two chapters, two parts, a title page,
% a compile flag as well as three forwarding files to set the flag.
% It consists of eight |.tex| files:
% \begin{center}
% \begin{tabular}{ll}
% |cdocsamp.tex|&main file\\
% |cdocsch1.tex|&include file for chapter 1\\
% |cdocsch2.tex|&include file for chapter 2\\
% |cdocspt3.tex|&include file for part 3\\
% |cdocspt4.tex|&include file for part 4\\
% |cdocsdrf.tex|&forwarding file for main file in draft mode\\
% |cdocsfi1.tex|&forwarding file for final version of chapter 1\\
% |cdocsfi2.tex|&forwarding file for final version of chapter 2\\
% \end{tabular}
% \end{center}
% Each of the eight files can be compiled directly by the \LaTeX{} compiler.
%
% %%%%%%%%%%%%%%%%%%%%%%%%%%%%%%%%%%%%%%
% \paragraph{Main File.}
%
% The main file is called |cdocsamp.tex|.
%
% Load the \textsf{childdoc} definitions and
% declare the filename for the main document:
%    \begin{macrocode}
\input{childdoc.def}
\childdocmain{}
%    \end{macrocode}

% Optional override for |\version| flag:
%    \begin{macrocode}
%%\ifchilddoc\else\providecommand{\version}{draft}\fi
%    \end{macrocode}

% Define the default values for the |\version| flag
% (|final| for the main file and |draft| for childs):
%    \begin{macrocode}
\ifchilddoc
\providecommand{\version}{draft}
\else
\providecommand{\version}{final}
\fi
%    \end{macrocode}

% Load the standard document class:
%    \begin{macrocode}
\documentclass[12pt]{article}
%    \end{macrocode}

% Start the document body:
%    \begin{macrocode}
\begin{document}
%    \end{macrocode}

% Declare a title page.
% Print title, part of document being processed and version flag:
%    \begin{macrocode}
\addtocounter{page}{-1}
\begin{center}
{\LARGE\bfseries{}childdoc example\par}
\vspace{1cm}
\ifchilddoc
\ifchilddocmanual part\else chapter\fi:
`\childdocname' of `\childdocjob'\par
\else
main document: `\childdocjob'\par
\fi
version: \version\par
\end{center}
\newpage
%    \end{macrocode}

% Manually include selected file,
% otherwise process as usual:
%    \begin{macrocode}
\ifchilddocmanual
\section*{part `\childdocname'}
\input{\childdocname}
\else
%    \end{macrocode}

% Include the two chapters:
%    \begin{macrocode}
\include{cdocsch1}
\include{cdocsch2}
%    \end{macrocode}

% Include the two parts unless only chapters should be displayed:
%    \begin{macrocode}
\ifchilddoc\else
\section{part three}
\input{cdocspt3}
\section{part four}
\input{cdocspt4}
\fi
%    \end{macrocode}

% Process as usual until here:
%    \begin{macrocode}
\fi
%    \end{macrocode}

% End of document body:
%    \begin{macrocode}
\end{document}
%    \end{macrocode}
%\iffalse
%</samplemain>
%\fi
%
% %%%%%%%%%%%%%%%%%%%%%%%%%%%%%%%%%%%%%%
% \paragraph{Chapter Include Files.}
%
% The include files are called |cdocsch1.tex| and |cdocsch2.tex|.
%
%\iffalse
%<*samplechap1|samplechap2>
%\fi

% Optional override for |\version| flag:
%    \begin{macrocode}
%%\providecommand{\version}{final}
%    \end{macrocode}

% Include the main document:
%    \begin{macrocode}
\input{childdoc.def}
\childdocof{cdocsamp}
%    \end{macrocode}

%\iffalse
%</samplechap1|samplechap2>
%\fi
%
%\iffalse
%<*samplechap1>
%\fi
% Some text for chapter 1:
%    \begin{macrocode}
\section{one}
some text in chapter one
%    \end{macrocode}

%\iffalse
%</samplechap1>
%\fi
% Some text for chapter 2:
%\iffalse
%<*samplechap2>
%\fi
%    \begin{macrocode}
\section{two}
more text in chapter two
%    \end{macrocode}

%\iffalse
%</samplechap2>
%\fi
%
% %%%%%%%%%%%%%%%%%%%%%%%%%%%%%%%%%%%%%%
% \paragraph{Part Include Files.}
%
% The include files are called |cdocspt3.tex| and |cdocspt4.tex|.
%
%\iffalse
%<*samplepart3|samplepart4>
%\fi

% Optional override for |\version| flag:
%    \begin{macrocode}
%%\providecommand{\version}{final}
%    \end{macrocode}

% Include the main document:
%    \begin{macrocode}
\input{childdoc.def}
\childdocby{cdocsamp}
%    \end{macrocode}

%\iffalse
%</samplepart3|samplepart4>
%\fi
%
%\iffalse
%<*samplepart3>
%\fi
% Some text for part 3:
%    \begin{macrocode}
some text in part three
%    \end{macrocode}

%\iffalse
%</samplepart3>
%\fi
% Some text for part 4:
%\iffalse
%<*samplepart4>
%\fi
%    \begin{macrocode}
more text in part four
%    \end{macrocode}

%\iffalse
%</samplepart4>
%\fi
%
% %%%%%%%%%%%%%%%%%%%%%%%%%%%%%%%%%%%%%%
% \paragraph{Forwarding for a Complete Draft.}
%
% The following forwarding file |cdocsdrf.tex|
% compiles the main document in draft mode:
%\iffalse
%<*sampledraft>
%\fi
%    \begin{macrocode}
\def\version{draft}
\input{childdoc.def}
\childdocforward{cdocsamp}
%    \end{macrocode}

%\iffalse
%</sampledraft>
%\fi
%
% %%%%%%%%%%%%%%%%%%%%%%%%%%%%%%%%%%%%%%
% \paragraph{Forwarding for Final Version of the Chapters.}
%
% The following forwarding files |cdocsfn1.tex| and |cdocsfn2.tex|
% (with identical content)
% compile the final versions of the child documents
% |cdocsch1.tex| and |cdocsch2.tex|, respectively:
%\iffalse
%<*samplefinal>
%\fi
%    \begin{macrocode}
\def\version{final}
\input{childdoc.def}
\childdocforwardprefix[cdocsamp]{cdocsfn}{cdocsch}
%    \end{macrocode}

%\iffalse
%</samplefinal>
%\fi
%
% %%%%%%%%%%%%%%%%%%%%%%%%%%%%%%%%%%%%%%
% \paragraph{Command Line Processing.}
%
% The following three command lines generate the output files
% |cdocscld|, |cdocscl1| and |cdocscl2|
% which should be identical to
% |cdocsdrf|, |cdocsch1| and |cdocsfn2|, respectively:
% \begin{center}
% \begin{tabular}{l}
% |latex -jobname cdocscld \|\\
% |  "\def\version{draft}\input{childdoc.def}\childdocforward{cdocsamp}"|\\
% |latex -jobname cdocscl1 \|\\
% |  "\input{childdoc.def}\childdocforward[cdocsamp]{cdocsch1}"|\\
% |latex -jobname cdocscl2 \|\\
% |  "\def\version{final}\input{childdoc.def}\childdocforward{cdocsch2}"|
% \end{tabular}
% \end{center}
% Note that the trailing backslash on each first line
% merely continues the input to the second line
% (for convenient cut ant paste).
% Furthermore, the command |latex| can be replaced by any
% of its alternative versions such as |pdflatex|.
%
% %%%%%%%%%%%%%%%%%%%%%%%%%%%%%%%%%%%%%%%%%%%%%%%%%%%%%%%%%%%%%%%%%%%%%%%%%%%%%%
% %%%%%%%%%%%%%%%%%%%%%%%%%%%%%%%%%%%%%%%%%%%%%%%%%%%%%%%%%%%%%%%%%%%%%%%%%%%%%%
% \section{Implementation}
%\iffalse
%<*package>
%\fi
%
% This section describes the definitions file |childdoc.def|.

% The definitions cannot be loaded using |\usepackage| or |\RequirePackage|
% which has a mechanism to prevent loading a style file more than once.
% When loading the definitions by means of |\input|
% multiple instances have to be prevented manually:
%\iffalse
%This code needs to be before the `\ProvidesFile' directive
%which is defined at the beginning of this file.
%Therefore it is also placed there and commented out here.
%</package>
%<*discard>
%\fi
%    \begin{macrocode}
\ifdefined\childdocmain\endinput\fi
%    \end{macrocode}
%\iffalse
%</discard>
%<*package>
%\fi
%
% \macro{\ifchilddoc}
% \macro{\ifchilddocmanual}
% The conditional |\ifchilddoc| tells whether a
% child (true) or main (false) document is being compiled.
% The conditional |\ifchilddocmanual| tells whether
% the |\includeonly| mechanism is used (false) or
% the selection of child files must be performed manually (true).
% The definitions initialise to false:
%    \begin{macrocode}
\newif\ifchilddoc
\newif\ifchilddocmanual
%    \end{macrocode}

% \macro{\childdocname}
% \macro{\childdocjob}
% The macro |\childdocname| stores the name of the main document
% to be compiled. The macro |\childdocjob| stores the name of
% the document on which the \LaTeX{} compiler was originally invoked.
% The content of |\jobname| cannot be compared
% to filenames specified in the source due to different catcodes.
% The following code rescans |\jobname|, stores the result
% in |\childdocname| and saves a copy in |\childdocjob|:
%    \begin{macrocode}
\edef\childdocname{\scantokens\expandafter{\jobname\noexpand}}
\let\childdocjob\childdocname
%    \end{macrocode}

% \macro{\childdocdisable}
% The macro |\childdocdisable| prevents the main file
% from being processed more than once.
% At this stage, the main document command |\childdocmain|
% is assumed to be called once again where it should do nothing.
% Any subsequent call to it should prevent
% a secondary processing of the main document
% It overwrites the forwarding commands
% |\childdocof| and |\childdocforward|
% with empty macros to prevent further inclusions of the main document:
%    \begin{macrocode}
\newcommand{\childdocdisable}
{
  \renewcommand{\childdocmain}[1]{\renewcommand{\childdocmain}[1]{\endinput}}
  \renewcommand{\childdocof}[1]{}
  \renewcommand{\childdocby}[2][]{}
  \renewcommand{\childdocforward}[2][]{}
  \renewcommand{\childdocdisable}{}
}
%    \end{macrocode}

% \macro{\childdocmain}
% The macro |\childdocmain| is to be called at the top of the main file
% with nothing or the main filename (without extension) as argument.
% First, it breaks loops.
% If the argument is not empty and does not match |\childdocname|
% (which is set by the first inclusion of |childdoc.def|),
% |\ifchilddoc| is set to true, |\includeonly| is applied to the child file
% and |\jobname| is set to the main file
% (for proper handling of |.aux| files):
%    \begin{macrocode}
\newcommand{\childdocmain}[1]
{
  \childdocdisable\childdocmain{}
  \if?#1?\else
    \begingroup
      \def\childdoctmp{#1}
      \ifx\childdoctmp\childdocname
        \def\childdoctmp{}
      \else
        \def\childdoctmp
        {
          \childdoctrue
          \includeonly{\childdocname}
          \def\childdocjob{#1}
          \def\jobname{#1}
        }
      \fi
      \expandafter
    \endgroup
    \childdoctmp
  \fi
}
%    \end{macrocode}

% \macro{\childdocof}
% The command |\childdocof| redirects
% compilation to the main file |#1|.
%    \begin{macrocode}
\newcommand{\childdocof}[1]
{
  \childdocdisable
  \childdoctrue
  \includeonly{\childdocname}
  \def\jobname{#1}
  \def\childdocjob{#1}
  \input{#1}
}
%    \end{macrocode}

% \macro{\childdocby}
% The command |\childdocby| ....
%    \begin{macrocode}
\newcommand{\childdocby}[2][]
{
  \childdocdisable
  \childdoctrue
  \childdocmanualtrue
  \if?#1?\else
    \def\jobname{#2}
  \fi
  \def\childdocjob{#2}
  \input{#2}
  \endinput
}
%    \end{macrocode}

% \macro{\childdocforward}
% The command |\childdocforward| redirects
% compilation to the main file or
% (if the optional argument is given) a child file.
% Parameters are set as if the main file
% or a child file starting with |\childdocof| was compiled.
% Then compilation is handed over to the main file:
%    \begin{macrocode}
\newcommand{\childdocforward}[2][]
{
  \begingroup
    \if?#1?
      \def\childdoctmp
      {
        \def\childdocname{#2}
        \def\childdocjob{#2}
        \def\jobname{#2}
        \input{#2}
        \endinput
      }
    \else
      \def\childdoctmp
      {
        \childdocdisable
        \def\childdocname{#2}
        \childdoctrue
        \includeonly{#2}
        \def\childdocjob{#1}
        \def\jobname{#1}
        \input{#1}
        \endinput
      }
    \fi
    \expandafter
  \endgroup
  \childdoctmp
}
%    \end{macrocode}

% \macro{\childdocforwardprefix}
% The command |\childdocforwardprefix| redirects
% compilation to the main or a child file by means of a pattern.
% The prefix |#1| in the current filename is replaced by |#2|
% and the suffix of the current filename is kept
% (it is assumed that the filename does not contain the substring `|~~~|'
% which is used as a delimiter).
% Compilation is handed over to the new file by |\childdocforward|:
%    \begin{macrocode}
\newcommand{\childdocforwardprefix}[3][]
{
  \begingroup
    \def\childdocextract #2##1~~~{\def\childdoctmp{\childdocforward[#1]{#3##1}}}
    \expandafter\childdocextract\childdocname~~~
    \expandafter
  \endgroup
  \childdoctmp
}
%    \end{macrocode}

% \macro{\childdoc}
% The deprecated macro |\childdoc| is a legacy version of |\childdocmain|:
%    \begin{macrocode}
\newcommand{\childdoc}{\childdocmain}
%    \end{macrocode}

% \macro{\childdocredirect}
% The deprecated macro |\childdocredirect| is a legacy version
% of |\childdocforward| and |\childdocforwardprefix|:
%    \begin{macrocode}
\newcommand{\childdocredirect}[2][]
{
  \begingroup
    \if?#1?
      \def\childdoctmp{\childdocforward{#2}}
    \else
      \def\childdoctmp{\childdocforwardprefix{#1}{#2}}
    \fi
    \expandafter
  \endgroup
  \childdoctmp
}
%    \end{macrocode}

%\iffalse
%</package>
%\fi
%
\endinput
|\\
|\childdocforwardprefix[|\textit{main}|]{|\textit{prefix}|}{|\textit{dest}|}|
\end{tabular}
\end{center}
%
the destination file is determined by a pattern
depending on the current file:
To make this work, the current file must be called
`{\textit{prefix}\hspace{0.2em}\textit{suffix}}'
with \textit{prefix} matching precisely the argument.
Processing is then passed on to the file
`{\textit{dest}\hspace{0.2em}\textit{suffix}}'.
Surely, the same effect is achieved by
directly specifying the
argument `{\textit{dest}\hspace{0.2em}\textit{suffix}}'
in the first form.
However, that requires to set up a different file
for each child. With the alternative form of the command
all these files can have exactly the same content
which simplifies setting them up and maintaining them.

For example, the following file |draft.tex|
with a compilation flag |\version| as described in \secref{sec:flags}
compiles the main document as a draft:
%
\begin{center}
\begin{tabular}{l}
|\def\version{draft}|\\
|% \iffalse
%
% childdoc.dtx Copyright (C) 2017-2018 Niklas Beisert
%
% This work may be distributed and/or modified under the
% conditions of the LaTeX Project Public License, either version 1.3
% of this license or (at your option) any later version.
% The latest version of this license is in
%   http://www.latex-project.org/lppl.txt
% and version 1.3 or later is part of all distributions of LaTeX
% version 2005/12/01 or later.
%
% This work has the LPPL maintenance status `maintained'.
%
% The Current Maintainer of this work is Niklas Beisert.
%
% This work consists of the files childdoc.dtx and childdoc.ins
% and the derived files childdoc.def and cdocsamp.tex with
% cdocsch1.tex, cdocsch2.tex, cdocsdrf.tex, cdocsfn1.tex, cdocsfn2.tex.
%
%<package>\ifdefined\childdocmain\endinput\fi
%<package>\ProvidesFile{childdoc.def}[2018/12/30 v2.0 child document driver]
%<samplemain>\ProvidesFile{cdocsamp.tex}[2018/12/30 v2.0 sample for childdoc]
%<*driver>
%\ProvidesFile{childdoc.drv}[2018/12/30 v2.0 childdoc reference manual file]
\PassOptionsToClass{10pt,a4paper}{article}
\documentclass{ltxdoc}

\usepackage[margin=35mm]{geometry}
\usepackage{hyperref}
\usepackage{hyperxmp}
\usepackage[usenames]{color}

\hypersetup{colorlinks=true}
\hypersetup{pdfstartview=FitH}
\hypersetup{pdfpagemode=UseNone}
\hypersetup{pdfsource={}}
\hypersetup{pdflang={en-UK}}
\hypersetup{pdfcopyright={Copyright 2017-2018 Niklas Beisert.
  This work may be distributed and/or modified under the
  conditions of the LaTeX Project Public License, either version 1.3
  of this license or (at your option) any later version.}}
\hypersetup{pdflicenseurl={http://www.latex-project.org/lppl.txt}}
\hypersetup{pdfcontactaddress={ETH Zurich, ITP, HIT K,
  Wolfgang-Pauli-Strasse 27}}
\hypersetup{pdfcontactpostcode={8093}}
\hypersetup{pdfcontactcity={Zurich}}
\hypersetup{pdfcontactcountry={Switzerland}}
\hypersetup{pdfcontactemail={nbeisert@itp.phys.ethz.ch}}
\hypersetup{pdfcontacturl={http://people.phys.ethz.ch/\xmptilde nbeisert/}}

\newcommand{\secref}[1]{\hyperref[#1]{section \ref*{#1}}}

\parskip1ex
\parindent0pt
\let\olditemize\itemize
\def\itemize{\olditemize\parskip0pt}

\begin{document}

\title{The \textsf{childdoc} Package}
\hypersetup{pdftitle={The childdoc Package}}
\author{Niklas Beisert\\[2ex]
  Institut f\"ur Theoretische Physik\\
  Eidgen\"ossische Technische Hochschule Z\"urich\\
  Wolfgang-Pauli-Strasse 27, 8093 Z\"urich, Switzerland\\[1ex]
  \href{mailto:nbeisert@itp.phys.ethz.ch}
  {\texttt{nbeisert@itp.phys.ethz.ch}}}
\hypersetup{pdfauthor={Niklas Beisert}}
\hypersetup{pdfsubject={Manual for the LaTeX2e Package childdoc}}
\date{30 December 2018, \textsf{v2.0}}
\maketitle

\begin{abstract}\noindent
\textsf{childdoc} is a \LaTeXe{} package
that enables the direct compilation
of document sections included by |\include|
to individual files.
\end{abstract}

\begingroup
\parskip0ex
\tableofcontents
\endgroup

%%%%%%%%%%%%%%%%%%%%%%%%%%%%%%%%%%%%%%%%%%%%%%%%%%%%%%%%%%%%%%%%%%%%%%%%%%%%%%%%
%%%%%%%%%%%%%%%%%%%%%%%%%%%%%%%%%%%%%%%%%%%%%%%%%%%%%%%%%%%%%%%%%%%%%%%%%%%%%%%%
\section{Introduction}

\LaTeX{} provides a mechanism to structure a large document (such as a book)
into a main file and several child files (containing the chapters)
using the |\include| command.
This mechanism is beneficial for documents
which span hundreds of pages in order to
make the source file(s) more manageable.
Moreover, compilation can be restricted to
selected child files by means of the |\includeonly| command.
The latter feature can be used to reduce the compilation time while editing
(this was significantly more useful in the earlier days of \LaTeX{})
or to generate a smaller document which is easier to navigate.
Another application of |\includeonly| is to generate
documents consisting of selected parts of the complete document.

However, there are a few drawbacks of the plain |\include| mechanism:
\begin{itemize}
\item
The child files cannot be compiled on their own,
they can only be compiled via the main file.
A naive editing environment
(such as a text editor with an option
to have the current file processed by \LaTeX)
may require one to switch to the main file before compiling;
attempting to compile the child file produces errors.
\item
The main file must be modified (each time)
to adjust the |\includeonly| command
to the present needs. This easily leaves the main file in a messy state.
\item
The generated document will always carry the filename
of the main document. This is inconvenient if
several child files are to be compiled and
to be kept for distribution.
\end{itemize}

The present package provides a simple interface
to make child files individually compilable by \LaTeX{}.
Compiling a child file then has the same effect as compiling
the main file with an |\includeonly| command
to select the appropriate child.
Moreover the generated document will carry the name of the child
rather than the main file.
This resolves all three above issues.

This feature is meant to make the editing of books,
thesis documents and lecture notes somewhat more convenient.
However, the package can also be used efficiently for
composing a series of documents (such as exercise sheets)
which are typically distributed individually.
It then assists the author in generating the individual documents
(potentially in different versions)
as well as a document containing the collected series.
Another application is in developing style files
or other kinds of included material
where compilation of the style file could redirect
to a sample or test file.

%%%%%%%%%%%%%%%%%%%%%%%%%%%%%%%%%%%%%%%%%%%%%%%%%%%%%%%%%%%%%%%%%%%%%%%%%%%%%%%%
%%%%%%%%%%%%%%%%%%%%%%%%%%%%%%%%%%%%%%%%%%%%%%%%%%%%%%%%%%%%%%%%%%%%%%%%%%%%%%%%
\section{Usage}

First of all, the package \textsf{childdoc} is \emph{not} a standard
\LaTeXe{} |.sty| style file! Therefore it needs to be invoked in
a non-standard way.

%%%%%%%%%%%%%%%%%%%%%%%%%%%%%%%%%%%%%%%%%%%%%%%%%%%%%%%%%%%%%%%%%%%%%%%%%%%%%%%%
\subsection{Included Files}
\label{sec:include}

%%%%%%%%%%%%%%%%%%%%%%%%%%%%%%%%%%%%%%%%
\DescribeMacro{\childdocmain}
To use the package, add the commands
\begin{center}
\begin{tabular}{l}
|\input{childdoc.def}|\\
|\childdocmain{}|\\
\end{tabular}
\end{center}
at the very top of the main \LaTeX{} file,
in particular \emph{before} the |\documentclass| statement!
The argument of |\childdocmain| should be left empty
(but it must be present).

%%%%%%%%%%%%%%%%%%%%%%%%%%%%%%%%%%%%%%%%
\DescribeMacro{\childdocof}
Furthermore, add the commands
\begin{center}
\begin{tabular}{l}
|\input{childdoc.def}|\\
|\childdocof{|\textit{main}|}|\\
\end{tabular}
\end{center}
at the top of every child file \textit{child}
which is included by |\include{|\textit{child}|}|
from within the main file
(or at least for those files to be compiled individually).
The argument \textit{main} must be the filename of the main file.

There are a couple of
considerations in setting up the main and child documents:

%%%%%%%%%%%%%%%%%%%%%%%%%%%%%%%%%%%%%%%%
\paragraph{Restrictions.}

Please note the following restrictions:
\begin{itemize}
\item
|\childdocmain| must be called with one argument \textit{main}
to ensure compatibility with earlier version of the package.
It must either be empty (|\childdocmain{}|)
or precisely match the filename of the main file in which it is specified.
See \secref{sec:detection} for further information.
\item
The filename \textit{main} must be specified without the |.tex| extension.
\item
The filename \textit{main} is case sensitive
(even in case-insensitive file systems)
due to internal string comparison.
\item
The argument \textit{main} should be fully expanded, it cannot be a macro.
\item
Subdirectories and special characters should be avoided in filenames.
\item
The command |\childdocmain{|\textit{main}|}| must be followed by a whitespace.
It should not be followed immediately by another command
or by a comment mark `|%|'.
This is because the \TeX{} parser reads the token immediately following
the argument of |\childdocmain| and puts it
at the beginning of every child section;
however, a white\-space is ignored.
\end{itemize}

%%%%%%%%%%%%%%%%%%%%%%%%%%%%%%%%%%%%%%%%
\paragraph{Content of Main File.}

It is advisable to place all content in the child files included by |\include|.
Any output contained in the main file will appear in all child documents
unless suppressed manually;
it cannot be suppressed automatically by the |\includeonly| directive
and thus should normally be avoided.
A method to include some content in the main file
by means of conditional processing is described in \secref{sec:conditional}.

%%%%%%%%%%%%%%%%%%%%%%%%%%%%%%%%%%%%%%%%
\paragraph{Page Numbering.}

When only a part of the document is compiled,
the appropriate numbering of pages
(as well as other status parameters)
is determined from the |.aux| files.
The latter contain information from previous passes.
However this information needs to propagate through
all intermediate child documents.
Therefore the page numbering in child documents may well
be inconsistent until the complete document is compiled at least once.

A useful (if unconventional) way to always ensure a consistent
page numbering is to restart the numbering in each child document
and denote the pages by `\textit{child}|.|\textit{page}'
where \textit{child} represents the chapter/section number of the child file.
This can be achieved by the command
|\numberwithin{page}{|\textit{child}|}|
of the \textsf{amsmath} package
where \textit{child} can be |chapter| or |section|
depending on the chosen structuring.
Alternatively, one can modify the macro |\thepage| appropriately
and reset the counter |page| at the start of each child file.

%%%%%%%%%%%%%%%%%%%%%%%%%%%%%%%%%%%%%%%%%%%%%%%%%%%%%%%%%%%%%%%%%%%%%%%%%%%%%%%%
\subsection{Conditional Processing}
\label{sec:conditional}

The package provides a mechanism to compile different versions
of a document. To customise the versions further some conditional processing
can come in handy to distinguish which version is being compiled.
The package provides two macros to describe the compilation context:

%%%%%%%%%%%%%%%%%%%%%%%%%%%%%%%%%%%%%%%%
\DescribeMacro{\ifchilddoc}
The conditional |\ifchilddoc| distinguishes between the compilation of
child documents and the main document:
%
\begin{center}
|\ifchilddoc |\textit{child-code}| |[|\||else |\textit{main-code}]| \||fi|
\end{center}

%%%%%%%%%%%%%%%%%%%%%%%%%%%%%%%%%%%%%%%%
\DescribeMacro{\childdocname}
\DescribeMacro{\childdocjob}
The macro |\childdocname| contains the filename (without extension)
of the main or child file being processed.
Note that |\childdocjob| will always contain the name of the main file.

%%%%%%%%%%%%%%%%%%%%%%%%%%%%%%%%%%%%%%%%
\paragraph{Title Page.}

Conditional processing can be used to include a title or banner page
in the main document when proper precautions are taken.
Importantly, the code in the main file should ensure that the page counter
(as well as other status parameters which are stored in the |.aux| files)
takes the same value after the conditional processing.
Otherwise the page numbers may take divergent values
depending on which part is compiled.

For example, a title page could be declared by:
%
\begin{center}
\begin{tabular}{l}
|\ifchilddoc\||else|\\
|\addtocounter{page}{-1}|\\
\textit{code for title page}\\
|\newpage|\\
|\||fi|
\end{tabular}
\end{center}
%
A banner page for the child documents can be generated by:
%
\begin{center}
\begin{tabular}{l}
|\ifchilddoc|\\
|\addtocounter{page}{-1}|\\
\textit{code for banner page}\\
|\newpage|\\
|\||fi|
\end{tabular}
\end{center}
%
Here one could write a message such as:
\begin{center}
|This is the part \childdocname{} of \childdocjob{}.|
\end{center}

%%%%%%%%%%%%%%%%%%%%%%%%%%%%%%%%%%%%%%%%%%%%%%%%%%%%%%%%%%%%%%%%%%%%%%%%%%%%%%%%
\subsection{Flags}
\label{sec:flags}

The package makes it easy to generate different versions
of the main or child documents.
To this end compilation flags can be defined
and assigned different default values.
They will be particularly useful in conjunction
with the forwarding mechanism described in \secref{sec:forward}.

For example, it may be useful to have a flag |\version|
which can be set to |draft| or |final|.
The document source will contain some conditional code
depending on the value of |\version|.
Suppose further, the flag should default to |final| for the main file
and to |draft| for child files
which is a natural assignment for editing the document.
This is achieved by placing the following code
in the preamble of the main document
(below the |\childdocmain| directive):
%
\begin{center}
\begin{tabular}{l}
|\ifchilddoc|\\
|\providecommand{\version}{draft}|\\
|\||else|\\
|\providecommand{\version}{final}|\\
|\||fi|
\end{tabular}
\end{center}
%
The definition by |\providecommand| makes sure
that previous definitions are not overwritten.
Further statements |\providecommand{\version}{...}|
can thus be added before the above code to override it.

For the main file, one might add a line
(between |\childdocmain| and the above block)
%
\begin{center}
|%\ifchilddoc\||else\providecommand{\version}{draft}\||fi|
\end{center}
%
which can be uncommented to produce a draft version.
Likewise one can add a line to the very top of a child file
(above the |\childdocof{|\textit{main}|}| directive)
%
\begin{center}
|%\providecommand{\version}{final}|
\end{center}
%
which can be uncommented to produce the final version of this child document.

%%%%%%%%%%%%%%%%%%%%%%%%%%%%%%%%%%%%%%%%%%%%%%%%%%%%%%%%%%%%%%%%%%%%%%%%%%%%%%%%
\subsection{Forwarding}
\label{sec:forward}

Different versions of the main or child documents
using compilation flags as described in \secref{sec:flags}
can be (permanently) stored in different files
for convenient compilation, viewing and distribution.
To this end, the package defines a command
to pass on compilation to a different file:

%%%%%%%%%%%%%%%%%%%%%%%%%%%%%%%%%%%%%%%%
\DescribeMacro{\childdocforward}
The command |\childdocforward| redirects processing to
another source file:
%
\begin{center}
\begin{tabular}{l}
|\input{childdoc.def}|\\
|\childdocforward[|\textit{main}|]{|\textit{dest}|}|\\
\end{tabular}
\end{center}
%
The argument \textit{dest} is the destination file
(without extension).
It should be the main file or one of the child files.
Note that further \textsf{childdoc} directives
such as |\childdocof| and |\childdocforward|
in the indicated file will be processed in this form.
The optional argument \textit{main}
passes on directly to the main file \textit{main}
while pretending to compile the child \textit{dest}.
This form behaves as if \textit{dest}
issues |\childdocof{|\textit{main}|}| right away,
and no further \textsf{childdoc} directives will be processed.

%%%%%%%%%%%%%%%%%%%%%%%%%%%%%%%%%%%%%%%%
\DescribeMacro{\...prefix}
In the alternative form |\childdocforwardprefix|,
%
\begin{center}
\begin{tabular}{l}
|\input{childdoc.def}|\\
|\childdocforwardprefix[|\textit{main}|]{|\textit{prefix}|}{|\textit{dest}|}|
\end{tabular}
\end{center}
%
the destination file is determined by a pattern
depending on the current file:
To make this work, the current file must be called
`{\textit{prefix}\hspace{0.2em}\textit{suffix}}'
with \textit{prefix} matching precisely the argument.
Processing is then passed on to the file
`{\textit{dest}\hspace{0.2em}\textit{suffix}}'.
Surely, the same effect is achieved by
directly specifying the
argument `{\textit{dest}\hspace{0.2em}\textit{suffix}}'
in the first form.
However, that requires to set up a different file
for each child. With the alternative form of the command
all these files can have exactly the same content
which simplifies setting them up and maintaining them.

For example, the following file |draft.tex|
with a compilation flag |\version| as described in \secref{sec:flags}
compiles the main document as a draft:
%
\begin{center}
\begin{tabular}{l}
|\def\version{draft}|\\
|\input{childdoc.def}|\\
|\childdocforward{|\textit{main}|}|
\end{tabular}
\end{center}
%
Likewise, the following files |final|\textit{nn}|.tex|
compile the final version of the child document
|child|\textit{nn}|.tex|:
%
\begin{center}
\begin{tabular}{l}
|\def\version{final}|\\
|\input{childdoc.def}|\\
|\childdocforwardprefix{final}{child}|
\end{tabular}
\end{center}
%

Note that when several versions of a main file and/or of each child file
are to be generated, it may be convenient to set up a |Makefile| or
shell script to automatise the process.

%%%%%%%%%%%%%%%%%%%%%%%%%%%%%%%%%%%%%%%%%%%%%%%%%%%%%%%%%%%%%%%%%%%%%%%%%%%%%%%%
\subsection{Command Line Processing}
\label{sec:commandline}

The effect of redirection files can also be achieved by invoking
the \LaTeX{} compiler with a more elaborate command line.
Most conveniently this should be done as part
of a shell script or a |Makefile|.

When using \textsf{childdoc} in the main file, the following
command lines effectively perform a redirection
(note that depending on the shell being used,
backslashes may have to be doubled: `|\|' $\to$ `|\\|'):
%
\begin{center}
|... -jobname "|\textit{target}|" |\\|"|[\textit{flags}]%
|\input{childdoc.def}\childdocforward[|\textit{main}|]{|\textit{dest}|}"|
\end{center}
%
Here \textit{target} is the name of the output file,
\textit{main} is the name of the main file
and \textit{dest} is the name of the main or child file to be processed
(all filenames without extensions).
The optional argument \textit{main} can be omitted
if \textit{main} matches \textit{dest}.
Optionally, compilation \textit{flags} can be defined via |\def| commands.
This command line makes the \TeX{} engine believe
it is compiling the file \textit{target}
whose content is specified as the latter parameter.
The provided code then forwards the processing to
\textit{main} or \textit{dest} as described in \secref{sec:forward}.

%%%%%%%%%%%%%%%%%%%%%%%%%%%%%%%%%%%%%%%%%%%%%%%%%%%%%%%%%%%%%%%%%%%%%%%%%%%%%%%%
\subsection{Include by Input}
\label{sec:input}

Including child documents by |\include| has some restrictions by design.
Most notably, the content of a child document always occupies
its own set of pages; pages cannot be shared between child documents.
Usually, this behaviour makes perfect sense
because each child document contain an essential part of the document.
However, in some situations it may be desirable to compose
a document from a collection of parts
without having mandatory page breaks between then.
For this case, the package
provides a mechanism to include parts
by |\input| which can also be processed individually.
However, by construction this mechanism
requires manual handling of the content to be output.

%%%%%%%%%%%%%%%%%%%%%%%%%%%%%%%%%%%%%%%%
\DescribeMacro{\ifchilddocmanual}
The main file should be prepared as usual, see \secref{sec:include}.
However, the document body must make a distinction
between processing of an individual part and of the main document, e.g.:
%
\begin{center}
\begin{tabular}{l}
|\ifchilddocmanual|\\
|\input{\childdocname}|\\
|\||else|\\
\textit{document body with }|\input{|\textit{part}|}|\\
|\||fi|
\end{tabular}
\end{center}
%
The conditional |\ifchilddocmanual| is true whenever
a part to be included by |\input| is being compiled,
and the name of the part is stored in |\childdocname|.

%%%%%%%%%%%%%%%%%%%%%%%%%%%%%%%%%%%%%%%%
\DescribeMacro{\childdocby}
Each part to be included by |\input| should start with:
%
\begin{center}
\begin{tabular}{l}
|\input{childdoc.def}|\\
|\childdocby{|\textit{main}|}|\\
\end{tabular}
\end{center}
%
The directive |\childdocby| is similar to |\childdocof|
described in \secref{sec:include},
but the subsequent selection of content must be done manually.
To that end, both |\ifchilddoc| and |\ifchilddocmanual|
will be true upon processing of a part,
and the name of the part is stored in |\childdocname|.
Note that |\jobname| will be set to the filename of the current part
so that each part receives an individual |.aux| file
that does not interfere with the |.aux| file(s) of the main document.
This behaviour can be altered by the alternative form
|\childdocby[*]{|\textit{main}|}| (with a non-empty optional argument)
which uses the |.aux| file of the main document
by setting |\jobname| to \textit{main}.

%%%%%%%%%%%%%%%%%%%%%%%%%%%%%%%%%%%%%%%%%%%%%%%%%%%%%%%%%%%%%%%%%%%%%%%%%%%%%%%%
\subsection{Driver Development}
\label{sec:driver}

The \textsf{childdoc} mechanism can also be use for the development
of definition files such as \LaTeX{} styles or classes.
This case differs from the above setup with multiple parts
included by |\include| in that no |\includeonly| should be invoked.
This can be achieved by starting the include file
(before |\ProvidesPackage|) with:
%
\begin{center}
\begin{tabular}{l}
|\input{childdoc.def}|\\
|\childdocforward{|\textit{main}|}|\\
\end{tabular}
\end{center}
%
or alternatively with:
%
\begin{center}
\begin{tabular}{l}
|\input{childdoc.def}|\\
|\childdocby{|\textit{main}|}|\\
\end{tabular}
\end{center}
%
Both forms have slightly different effects as described above.
The main file is prepared as usual, see \secref{sec:include}.

%%%%%%%%%%%%%%%%%%%%%%%%%%%%%%%%%%%%%%%%%%%%%%%%%%%%%%%%%%%%%%%%%%%%%%%%%%%%%%%%
\subsection{Legacy Detection}
\label{sec:detection}

The directive |\childdocmain| in the main file can detect
whether the complete document or merely a child is to be compiled
even without using the directive |\childdocof|.
This method is deprecated because it is less robust
and there is no compelling reason to use it;
it is merely provided for backward compatibility
and it may be removed in future versions.

If the detection mechanism is to be used,
it is mandatory to correctly specify
the filename of the main file as the argument of |\childdocmain|:
%
\begin{center}
\begin{tabular}{l}
|\input{childdoc.def}|\\
|\childdocmain{|\textit{main}|}|\\
\end{tabular}
\end{center}
%
If |\jobname| does not match the argument \textit{main} of |\childdocmain|,
it is assumed that |\jobname| points to the child file to be compiled.
When using |\childdocmain| with the main file specified as argument,
it suffices to start a child file
with just |\input{|\textit{main}|}|
without loading of the package and using |\childdocof|.
If instead all processing is done
with the appropriate \textsf{childdoc} directives,
the argument of \textit{main} of |\childdocmain| can be empty.

An alternative version of the command line processing described
in \secref{sec:commandline} using the detection mechanism reads:
%
\begin{center}
|... -jobname "|\textit{target}|" "|[\textit{flags}]%
[|\def\jobname{|\textit{dest}|}|]|\input{|\textit{main}|}"|
\end{center}

%%%%%%%%%%%%%%%%%%%%%%%%%%%%%%%%%%%%%%%%%%%%%%%%%%%%%%%%%%%%%%%%%%%%%%%%%%%%%%%%
\subsection{Manual Code}
\label{sec:manual}

In case one cannot be certain whether the definitions file |childdoc.def|
is installed on the target \TeX{} distribution
and one prefers not to ship it,
it is conceivable to paste a few relevant commands into the sources.

To that end, drop all statements |\input{childdoc.def}|
and perform the replacements as outlined below.
Instead of |\childdocmain{|\textit{main}|}| add the following code
to the top of the main file:
%
\begin{center}
\begin{tabular}{l}
|\||ifdefined\childdocname\endinput\||fi\newif\ifchilddoc|\\
|\edef\childdocname{\scantokens\expandafter{\jobname\noexpand}}|\\
|\def\childdocmain{|\textit{main}|}\||ifx\childdocmain\childdocname\||else|\\
|\childdoctrue\includeonly{\childdocname}\let\jobname\childdocmain\||fi|\\
\end{tabular}
\end{center}
%
Instead of |\childdocof{|\textit{main}|}| just include the main file
at the top of each child file:
%
\begin{center}
|\input{|\textit{main}|}|
\end{center}
%
A simple redirection |\childdocforward{|\textit{dest}|}| is achieved by:
%
\begin{center}
|\def\jobname{|\textit{dest}|}\input{\jobname}|
\end{center}
%
The redirection with prefix
|\childdocforwardprefix[|\textit{prefix}|]{|\textit{dest}|}|
is accomplished by:
%
\begin{center}
\begin{tabular}{l}
|{\edef\jobname{\scantokens\expandafter{\jobname\noexpand}}|\\
|\def\redirectjob |\textit{prefix}|#1~~~{\gdef\jobname{|\textit{dest}|#1}}|\\
|\expandafter\redirectjob\jobname~~~}\input{\jobname}|
\end{tabular}
\end{center}

In an alternative approach,
child documents can be compiled by a specific command line
without additional code or specific definitions:
%
\begin{center}
|... -jobname "|\textit{target}|" "|[\textit{flags}]%
|\includeonly{|\textit{dest}|}\input{|\textit{main}|}"|
\end{center}
%

%%%%%%%%%%%%%%%%%%%%%%%%%%%%%%%%%%%%%%%%%%%%%%%%%%%%%%%%%%%%%%%%%%%%%%%%%%%%%%%%
%%%%%%%%%%%%%%%%%%%%%%%%%%%%%%%%%%%%%%%%%%%%%%%%%%%%%%%%%%%%%%%%%%%%%%%%%%%%%%%%
\section{Information}

%%%%%%%%%%%%%%%%%%%%%%%%%%%%%%%%%%%%%%%%%%%%%%%%%%%%%%%%%%%%%%%%%%%%%%%%%%%%%%%%
\subsection{Copyright}

Copyright \copyright{} 2017--2018 Niklas Beisert

This work may be distributed and/or modified under the
conditions of the \LaTeX{} Project Public License, either version 1.3
of this license or (at your option) any later version.
The latest version of this license is in
  \url{http://www.latex-project.org/lppl.txt}
and version 1.3 or later is part of all distributions of \LaTeX{}
version 2005/12/01 or later.

This work has the LPPL maintenance status `maintained'.

The Current Maintainer of this work is Niklas Beisert.

This work consists of the files |README.txt|, |childdoc.ins| and |childdoc.dtx|
as well as the derived files |childdoc.def|, |cdocsamp.tex|
with |cdocsch1.tex|, |cdocsch2.tex|, |cdocspt3.tex|, |cdocspt4.tex|,
|cdocsdrf.tex|, |cdocsfn1.tex|, |cdocsfn2.tex|
as well as |childdoc.pdf|.

%%%%%%%%%%%%%%%%%%%%%%%%%%%%%%%%%%%%%%%%%%%%%%%%%%%%%%%%%%%%%%%%%%%%%%%%%%%%%%%%
\subsection{Files and Installation}

The package consists of the files:
%
\begin{center}
\begin{tabular}{ll}
    |README.txt|   & readme file \\
    |childdoc.ins| & installation file \\
    |childdoc.dtx| & source file \\
    |childdoc.def| & definition file \\
    |cdocsamp.tex| & sample main file \\
    |cdocsch1.tex| & sample include file \\
    |cdocsch2.tex| & sample include file \\
    |cdocspt3.tex| & sample part file \\
    |cdocspt4.tex| & sample part file \\
    |cdocsdrf.tex| & sample redirection file \\
    |cdocsfn1.tex| & sample redirection file \\
    |cdocsfn2.tex| & sample redirection file \\
    |childdoc.pdf| & manual
\end{tabular}
\end{center}
%
The distribution consists of the files
|README.txt|, |childdoc.ins| and |childdoc.dtx|.
%
\begin{itemize}
\item
Run (pdf)\LaTeX{} on |childdoc.dtx|
to compile the manual |childdoc.pdf| (this file).
\item
Run \LaTeX{} on |childdoc.ins| to create the definitions file |childdoc.def|
and the sample |cdocsamp.tex| with include files
|cdocsch1.tex|, |cdocsch2.tex|, |cdocspt3.tex|, |cdocspt4.tex|,
|cdocsdrf.tex|, |cdocsfn1.tex|, |cdocsfn2.tex|.
Then copy the file |childdoc.def| to an appropriate directory of your \LaTeX{}
distribution, e.g.\ \textit{texmf-root}|/tex/latex/childdoc|.
\end{itemize}

%%%%%%%%%%%%%%%%%%%%%%%%%%%%%%%%%%%%%%%%%%%%%%%%%%%%%%%%%%%%%%%%%%%%%%%%%%%%%%%%
\subsection{Related CTAN Packages}

There are several other packages which offer a similar functionality:
%
\begin{itemize}
\item
The packages
\href{http://ctan.org/pkg/docmute}{\textsf{docmute}},
\href{http://ctan.org/pkg/includex}{\textsf{includex}} and
\href{http://ctan.org/pkg/standalone}{\textsf{standalone}}
provide commands to include only the document body of
a child file thus allowing both files to be compiled individually.
\item
The packages \href{http://ctan.org/pkg/subdocs}{\textsf{subdocs}}
and \href{http://ctan.org/pkg/subfiles}{\textsf{subfiles}}
provide structures in which the main and child documents can be
encapsulated and allowing them to be compiled individually.
The inclusion mechanism is different from the conventional |\include|.
\item
The package \href{http://ctan.org/pkg/combine}{\textsf{combine}}
is an elaborate solution to combine several documents into one.
\end{itemize}
%
See also the CTAN topic \href{http://ctan.org/topic/subdocs}{\textsf{subdocs}}
for further related packages.
The present package differs from the above solutions in that
a document structure constructed with the conventional |\include| mechanism
just needs two extra commands at the top of every file
such that all constituent files can be compiled individually.

%%%%%%%%%%%%%%%%%%%%%%%%%%%%%%%%%%%%%%%%%%%%%%%%%%%%%%%%%%%%%%%%%%%%%%%%%%%%%%%%
%\subsection{Feature Suggestions}
%
%The following is a list of features which may be useful for future
%versions of this package:
%%
%\begin{itemize}
%\item
%\ldots
%\end{itemize}

%%%%%%%%%%%%%%%%%%%%%%%%%%%%%%%%%%%%%%%%%%%%%%%%%%%%%%%%%%%%%%%%%%%%%%%%%%%%%%%%
\subsection{Revision History}

%%%%%%%%%%%%%%%%%%%%%%%%%%%%%%%%%%%%%%%%
\paragraph{v2.0:} 2018/12/30

\begin{itemize}
\item
immediate forward processing
\item
added |\childdocby| mechanism
\item
manual restructured
\end{itemize}

%%%%%%%%%%%%%%%%%%%%%%%%%%%%%%%%%%%%%%%%
\paragraph{v1.6:} 2018/01/17

\begin{itemize}
\item
application for development of include files
\item
corrections to manual
\end{itemize}

%%%%%%%%%%%%%%%%%%%%%%%%%%%%%%%%%%%%%%%%
\paragraph{v1.5:} 2017/05/21

\begin{itemize}
\item
more complete structuring introduced
\item
|\childdocof| introduced
\item
|\childdoc| renamed to |\childdocmain|
\item
|\childredirect| renamed to |\childdocforward| and |\childdocforwardprefix|
and functionality expanded
\end{itemize}

%%%%%%%%%%%%%%%%%%%%%%%%%%%%%%%%%%%%%%%%
\paragraph{v1.0:} 2017/04/27

\begin{itemize}
\item
manual and install package
\item
first version published on CTAN
\end{itemize}

%%%%%%%%%%%%%%%%%%%%%%%%%%%%%%%%%%%%%%%%
\paragraph{v0.6:} 2017/04/26

\begin{itemize}
\item
redirection mechanism added
\end{itemize}

%%%%%%%%%%%%%%%%%%%%%%%%%%%%%%%%%%%%%%%%
\paragraph{v0.5:} 2017/04/26

\begin{itemize}
\item
functionality in definition file
\end{itemize}


%%%%%%%%%%%%%%%%%%%%%%%%%%%%%%%%%%%%%%%%%%%%%%%%%%%%%%%%%%%%%%%%%%%%%%%%%%%%%%%%
%%%%%%%%%%%%%%%%%%%%%%%%%%%%%%%%%%%%%%%%%%%%%%%%%%%%%%%%%%%%%%%%%%%%%%%%%%%%%%%%
%%%%%%%%%%%%%%%%%%%%%%%%%%%%%%%%%%%%%%%%%%%%%%%%%%%%%%%%%%%%%%%%%%%%%%%%%%%%%%%%
\appendix

\settowidth\MacroIndent{\rmfamily\scriptsize 000\ }

 \DocInput{childdoc.dtx}

\end{document}
%</driver>
% \fi
%
% %%%%%%%%%%%%%%%%%%%%%%%%%%%%%%%%%%%%%%%%%%%%%%%%%%%%%%%%%%%%%%%%%%%%%%%%%%%%%%
% %%%%%%%%%%%%%%%%%%%%%%%%%%%%%%%%%%%%%%%%%%%%%%%%%%%%%%%%%%%%%%%%%%%%%%%%%%%%%%
% \section{Sample}
%\iffalse
%<*samplemain>
%\fi
%
% The following presents a sample document
% with two chapters, two parts, a title page,
% a compile flag as well as three forwarding files to set the flag.
% It consists of eight |.tex| files:
% \begin{center}
% \begin{tabular}{ll}
% |cdocsamp.tex|&main file\\
% |cdocsch1.tex|&include file for chapter 1\\
% |cdocsch2.tex|&include file for chapter 2\\
% |cdocspt3.tex|&include file for part 3\\
% |cdocspt4.tex|&include file for part 4\\
% |cdocsdrf.tex|&forwarding file for main file in draft mode\\
% |cdocsfi1.tex|&forwarding file for final version of chapter 1\\
% |cdocsfi2.tex|&forwarding file for final version of chapter 2\\
% \end{tabular}
% \end{center}
% Each of the eight files can be compiled directly by the \LaTeX{} compiler.
%
% %%%%%%%%%%%%%%%%%%%%%%%%%%%%%%%%%%%%%%
% \paragraph{Main File.}
%
% The main file is called |cdocsamp.tex|.
%
% Load the \textsf{childdoc} definitions and
% declare the filename for the main document:
%    \begin{macrocode}
\input{childdoc.def}
\childdocmain{}
%    \end{macrocode}

% Optional override for |\version| flag:
%    \begin{macrocode}
%%\ifchilddoc\else\providecommand{\version}{draft}\fi
%    \end{macrocode}

% Define the default values for the |\version| flag
% (|final| for the main file and |draft| for childs):
%    \begin{macrocode}
\ifchilddoc
\providecommand{\version}{draft}
\else
\providecommand{\version}{final}
\fi
%    \end{macrocode}

% Load the standard document class:
%    \begin{macrocode}
\documentclass[12pt]{article}
%    \end{macrocode}

% Start the document body:
%    \begin{macrocode}
\begin{document}
%    \end{macrocode}

% Declare a title page.
% Print title, part of document being processed and version flag:
%    \begin{macrocode}
\addtocounter{page}{-1}
\begin{center}
{\LARGE\bfseries{}childdoc example\par}
\vspace{1cm}
\ifchilddoc
\ifchilddocmanual part\else chapter\fi:
`\childdocname' of `\childdocjob'\par
\else
main document: `\childdocjob'\par
\fi
version: \version\par
\end{center}
\newpage
%    \end{macrocode}

% Manually include selected file,
% otherwise process as usual:
%    \begin{macrocode}
\ifchilddocmanual
\section*{part `\childdocname'}
\input{\childdocname}
\else
%    \end{macrocode}

% Include the two chapters:
%    \begin{macrocode}
\include{cdocsch1}
\include{cdocsch2}
%    \end{macrocode}

% Include the two parts unless only chapters should be displayed:
%    \begin{macrocode}
\ifchilddoc\else
\section{part three}
\input{cdocspt3}
\section{part four}
\input{cdocspt4}
\fi
%    \end{macrocode}

% Process as usual until here:
%    \begin{macrocode}
\fi
%    \end{macrocode}

% End of document body:
%    \begin{macrocode}
\end{document}
%    \end{macrocode}
%\iffalse
%</samplemain>
%\fi
%
% %%%%%%%%%%%%%%%%%%%%%%%%%%%%%%%%%%%%%%
% \paragraph{Chapter Include Files.}
%
% The include files are called |cdocsch1.tex| and |cdocsch2.tex|.
%
%\iffalse
%<*samplechap1|samplechap2>
%\fi

% Optional override for |\version| flag:
%    \begin{macrocode}
%%\providecommand{\version}{final}
%    \end{macrocode}

% Include the main document:
%    \begin{macrocode}
\input{childdoc.def}
\childdocof{cdocsamp}
%    \end{macrocode}

%\iffalse
%</samplechap1|samplechap2>
%\fi
%
%\iffalse
%<*samplechap1>
%\fi
% Some text for chapter 1:
%    \begin{macrocode}
\section{one}
some text in chapter one
%    \end{macrocode}

%\iffalse
%</samplechap1>
%\fi
% Some text for chapter 2:
%\iffalse
%<*samplechap2>
%\fi
%    \begin{macrocode}
\section{two}
more text in chapter two
%    \end{macrocode}

%\iffalse
%</samplechap2>
%\fi
%
% %%%%%%%%%%%%%%%%%%%%%%%%%%%%%%%%%%%%%%
% \paragraph{Part Include Files.}
%
% The include files are called |cdocspt3.tex| and |cdocspt4.tex|.
%
%\iffalse
%<*samplepart3|samplepart4>
%\fi

% Optional override for |\version| flag:
%    \begin{macrocode}
%%\providecommand{\version}{final}
%    \end{macrocode}

% Include the main document:
%    \begin{macrocode}
\input{childdoc.def}
\childdocby{cdocsamp}
%    \end{macrocode}

%\iffalse
%</samplepart3|samplepart4>
%\fi
%
%\iffalse
%<*samplepart3>
%\fi
% Some text for part 3:
%    \begin{macrocode}
some text in part three
%    \end{macrocode}

%\iffalse
%</samplepart3>
%\fi
% Some text for part 4:
%\iffalse
%<*samplepart4>
%\fi
%    \begin{macrocode}
more text in part four
%    \end{macrocode}

%\iffalse
%</samplepart4>
%\fi
%
% %%%%%%%%%%%%%%%%%%%%%%%%%%%%%%%%%%%%%%
% \paragraph{Forwarding for a Complete Draft.}
%
% The following forwarding file |cdocsdrf.tex|
% compiles the main document in draft mode:
%\iffalse
%<*sampledraft>
%\fi
%    \begin{macrocode}
\def\version{draft}
\input{childdoc.def}
\childdocforward{cdocsamp}
%    \end{macrocode}

%\iffalse
%</sampledraft>
%\fi
%
% %%%%%%%%%%%%%%%%%%%%%%%%%%%%%%%%%%%%%%
% \paragraph{Forwarding for Final Version of the Chapters.}
%
% The following forwarding files |cdocsfn1.tex| and |cdocsfn2.tex|
% (with identical content)
% compile the final versions of the child documents
% |cdocsch1.tex| and |cdocsch2.tex|, respectively:
%\iffalse
%<*samplefinal>
%\fi
%    \begin{macrocode}
\def\version{final}
\input{childdoc.def}
\childdocforwardprefix[cdocsamp]{cdocsfn}{cdocsch}
%    \end{macrocode}

%\iffalse
%</samplefinal>
%\fi
%
% %%%%%%%%%%%%%%%%%%%%%%%%%%%%%%%%%%%%%%
% \paragraph{Command Line Processing.}
%
% The following three command lines generate the output files
% |cdocscld|, |cdocscl1| and |cdocscl2|
% which should be identical to
% |cdocsdrf|, |cdocsch1| and |cdocsfn2|, respectively:
% \begin{center}
% \begin{tabular}{l}
% |latex -jobname cdocscld \|\\
% |  "\def\version{draft}\input{childdoc.def}\childdocforward{cdocsamp}"|\\
% |latex -jobname cdocscl1 \|\\
% |  "\input{childdoc.def}\childdocforward[cdocsamp]{cdocsch1}"|\\
% |latex -jobname cdocscl2 \|\\
% |  "\def\version{final}\input{childdoc.def}\childdocforward{cdocsch2}"|
% \end{tabular}
% \end{center}
% Note that the trailing backslash on each first line
% merely continues the input to the second line
% (for convenient cut ant paste).
% Furthermore, the command |latex| can be replaced by any
% of its alternative versions such as |pdflatex|.
%
% %%%%%%%%%%%%%%%%%%%%%%%%%%%%%%%%%%%%%%%%%%%%%%%%%%%%%%%%%%%%%%%%%%%%%%%%%%%%%%
% %%%%%%%%%%%%%%%%%%%%%%%%%%%%%%%%%%%%%%%%%%%%%%%%%%%%%%%%%%%%%%%%%%%%%%%%%%%%%%
% \section{Implementation}
%\iffalse
%<*package>
%\fi
%
% This section describes the definitions file |childdoc.def|.

% The definitions cannot be loaded using |\usepackage| or |\RequirePackage|
% which has a mechanism to prevent loading a style file more than once.
% When loading the definitions by means of |\input|
% multiple instances have to be prevented manually:
%\iffalse
%This code needs to be before the `\ProvidesFile' directive
%which is defined at the beginning of this file.
%Therefore it is also placed there and commented out here.
%</package>
%<*discard>
%\fi
%    \begin{macrocode}
\ifdefined\childdocmain\endinput\fi
%    \end{macrocode}
%\iffalse
%</discard>
%<*package>
%\fi
%
% \macro{\ifchilddoc}
% \macro{\ifchilddocmanual}
% The conditional |\ifchilddoc| tells whether a
% child (true) or main (false) document is being compiled.
% The conditional |\ifchilddocmanual| tells whether
% the |\includeonly| mechanism is used (false) or
% the selection of child files must be performed manually (true).
% The definitions initialise to false:
%    \begin{macrocode}
\newif\ifchilddoc
\newif\ifchilddocmanual
%    \end{macrocode}

% \macro{\childdocname}
% \macro{\childdocjob}
% The macro |\childdocname| stores the name of the main document
% to be compiled. The macro |\childdocjob| stores the name of
% the document on which the \LaTeX{} compiler was originally invoked.
% The content of |\jobname| cannot be compared
% to filenames specified in the source due to different catcodes.
% The following code rescans |\jobname|, stores the result
% in |\childdocname| and saves a copy in |\childdocjob|:
%    \begin{macrocode}
\edef\childdocname{\scantokens\expandafter{\jobname\noexpand}}
\let\childdocjob\childdocname
%    \end{macrocode}

% \macro{\childdocdisable}
% The macro |\childdocdisable| prevents the main file
% from being processed more than once.
% At this stage, the main document command |\childdocmain|
% is assumed to be called once again where it should do nothing.
% Any subsequent call to it should prevent
% a secondary processing of the main document
% It overwrites the forwarding commands
% |\childdocof| and |\childdocforward|
% with empty macros to prevent further inclusions of the main document:
%    \begin{macrocode}
\newcommand{\childdocdisable}
{
  \renewcommand{\childdocmain}[1]{\renewcommand{\childdocmain}[1]{\endinput}}
  \renewcommand{\childdocof}[1]{}
  \renewcommand{\childdocby}[2][]{}
  \renewcommand{\childdocforward}[2][]{}
  \renewcommand{\childdocdisable}{}
}
%    \end{macrocode}

% \macro{\childdocmain}
% The macro |\childdocmain| is to be called at the top of the main file
% with nothing or the main filename (without extension) as argument.
% First, it breaks loops.
% If the argument is not empty and does not match |\childdocname|
% (which is set by the first inclusion of |childdoc.def|),
% |\ifchilddoc| is set to true, |\includeonly| is applied to the child file
% and |\jobname| is set to the main file
% (for proper handling of |.aux| files):
%    \begin{macrocode}
\newcommand{\childdocmain}[1]
{
  \childdocdisable\childdocmain{}
  \if?#1?\else
    \begingroup
      \def\childdoctmp{#1}
      \ifx\childdoctmp\childdocname
        \def\childdoctmp{}
      \else
        \def\childdoctmp
        {
          \childdoctrue
          \includeonly{\childdocname}
          \def\childdocjob{#1}
          \def\jobname{#1}
        }
      \fi
      \expandafter
    \endgroup
    \childdoctmp
  \fi
}
%    \end{macrocode}

% \macro{\childdocof}
% The command |\childdocof| redirects
% compilation to the main file |#1|.
%    \begin{macrocode}
\newcommand{\childdocof}[1]
{
  \childdocdisable
  \childdoctrue
  \includeonly{\childdocname}
  \def\jobname{#1}
  \def\childdocjob{#1}
  \input{#1}
}
%    \end{macrocode}

% \macro{\childdocby}
% The command |\childdocby| ....
%    \begin{macrocode}
\newcommand{\childdocby}[2][]
{
  \childdocdisable
  \childdoctrue
  \childdocmanualtrue
  \if?#1?\else
    \def\jobname{#2}
  \fi
  \def\childdocjob{#2}
  \input{#2}
  \endinput
}
%    \end{macrocode}

% \macro{\childdocforward}
% The command |\childdocforward| redirects
% compilation to the main file or
% (if the optional argument is given) a child file.
% Parameters are set as if the main file
% or a child file starting with |\childdocof| was compiled.
% Then compilation is handed over to the main file:
%    \begin{macrocode}
\newcommand{\childdocforward}[2][]
{
  \begingroup
    \if?#1?
      \def\childdoctmp
      {
        \def\childdocname{#2}
        \def\childdocjob{#2}
        \def\jobname{#2}
        \input{#2}
        \endinput
      }
    \else
      \def\childdoctmp
      {
        \childdocdisable
        \def\childdocname{#2}
        \childdoctrue
        \includeonly{#2}
        \def\childdocjob{#1}
        \def\jobname{#1}
        \input{#1}
        \endinput
      }
    \fi
    \expandafter
  \endgroup
  \childdoctmp
}
%    \end{macrocode}

% \macro{\childdocforwardprefix}
% The command |\childdocforwardprefix| redirects
% compilation to the main or a child file by means of a pattern.
% The prefix |#1| in the current filename is replaced by |#2|
% and the suffix of the current filename is kept
% (it is assumed that the filename does not contain the substring `|~~~|'
% which is used as a delimiter).
% Compilation is handed over to the new file by |\childdocforward|:
%    \begin{macrocode}
\newcommand{\childdocforwardprefix}[3][]
{
  \begingroup
    \def\childdocextract #2##1~~~{\def\childdoctmp{\childdocforward[#1]{#3##1}}}
    \expandafter\childdocextract\childdocname~~~
    \expandafter
  \endgroup
  \childdoctmp
}
%    \end{macrocode}

% \macro{\childdoc}
% The deprecated macro |\childdoc| is a legacy version of |\childdocmain|:
%    \begin{macrocode}
\newcommand{\childdoc}{\childdocmain}
%    \end{macrocode}

% \macro{\childdocredirect}
% The deprecated macro |\childdocredirect| is a legacy version
% of |\childdocforward| and |\childdocforwardprefix|:
%    \begin{macrocode}
\newcommand{\childdocredirect}[2][]
{
  \begingroup
    \if?#1?
      \def\childdoctmp{\childdocforward{#2}}
    \else
      \def\childdoctmp{\childdocforwardprefix{#1}{#2}}
    \fi
    \expandafter
  \endgroup
  \childdoctmp
}
%    \end{macrocode}

%\iffalse
%</package>
%\fi
%
\endinput
|\\
|\childdocforward{|\textit{main}|}|
\end{tabular}
\end{center}
%
Likewise, the following files |final|\textit{nn}|.tex|
compile the final version of the child document
|child|\textit{nn}|.tex|:
%
\begin{center}
\begin{tabular}{l}
|\def\version{final}|\\
|% \iffalse
%
% childdoc.dtx Copyright (C) 2017-2018 Niklas Beisert
%
% This work may be distributed and/or modified under the
% conditions of the LaTeX Project Public License, either version 1.3
% of this license or (at your option) any later version.
% The latest version of this license is in
%   http://www.latex-project.org/lppl.txt
% and version 1.3 or later is part of all distributions of LaTeX
% version 2005/12/01 or later.
%
% This work has the LPPL maintenance status `maintained'.
%
% The Current Maintainer of this work is Niklas Beisert.
%
% This work consists of the files childdoc.dtx and childdoc.ins
% and the derived files childdoc.def and cdocsamp.tex with
% cdocsch1.tex, cdocsch2.tex, cdocsdrf.tex, cdocsfn1.tex, cdocsfn2.tex.
%
%<package>\ifdefined\childdocmain\endinput\fi
%<package>\ProvidesFile{childdoc.def}[2018/12/30 v2.0 child document driver]
%<samplemain>\ProvidesFile{cdocsamp.tex}[2018/12/30 v2.0 sample for childdoc]
%<*driver>
%\ProvidesFile{childdoc.drv}[2018/12/30 v2.0 childdoc reference manual file]
\PassOptionsToClass{10pt,a4paper}{article}
\documentclass{ltxdoc}

\usepackage[margin=35mm]{geometry}
\usepackage{hyperref}
\usepackage{hyperxmp}
\usepackage[usenames]{color}

\hypersetup{colorlinks=true}
\hypersetup{pdfstartview=FitH}
\hypersetup{pdfpagemode=UseNone}
\hypersetup{pdfsource={}}
\hypersetup{pdflang={en-UK}}
\hypersetup{pdfcopyright={Copyright 2017-2018 Niklas Beisert.
  This work may be distributed and/or modified under the
  conditions of the LaTeX Project Public License, either version 1.3
  of this license or (at your option) any later version.}}
\hypersetup{pdflicenseurl={http://www.latex-project.org/lppl.txt}}
\hypersetup{pdfcontactaddress={ETH Zurich, ITP, HIT K,
  Wolfgang-Pauli-Strasse 27}}
\hypersetup{pdfcontactpostcode={8093}}
\hypersetup{pdfcontactcity={Zurich}}
\hypersetup{pdfcontactcountry={Switzerland}}
\hypersetup{pdfcontactemail={nbeisert@itp.phys.ethz.ch}}
\hypersetup{pdfcontacturl={http://people.phys.ethz.ch/\xmptilde nbeisert/}}

\newcommand{\secref}[1]{\hyperref[#1]{section \ref*{#1}}}

\parskip1ex
\parindent0pt
\let\olditemize\itemize
\def\itemize{\olditemize\parskip0pt}

\begin{document}

\title{The \textsf{childdoc} Package}
\hypersetup{pdftitle={The childdoc Package}}
\author{Niklas Beisert\\[2ex]
  Institut f\"ur Theoretische Physik\\
  Eidgen\"ossische Technische Hochschule Z\"urich\\
  Wolfgang-Pauli-Strasse 27, 8093 Z\"urich, Switzerland\\[1ex]
  \href{mailto:nbeisert@itp.phys.ethz.ch}
  {\texttt{nbeisert@itp.phys.ethz.ch}}}
\hypersetup{pdfauthor={Niklas Beisert}}
\hypersetup{pdfsubject={Manual for the LaTeX2e Package childdoc}}
\date{30 December 2018, \textsf{v2.0}}
\maketitle

\begin{abstract}\noindent
\textsf{childdoc} is a \LaTeXe{} package
that enables the direct compilation
of document sections included by |\include|
to individual files.
\end{abstract}

\begingroup
\parskip0ex
\tableofcontents
\endgroup

%%%%%%%%%%%%%%%%%%%%%%%%%%%%%%%%%%%%%%%%%%%%%%%%%%%%%%%%%%%%%%%%%%%%%%%%%%%%%%%%
%%%%%%%%%%%%%%%%%%%%%%%%%%%%%%%%%%%%%%%%%%%%%%%%%%%%%%%%%%%%%%%%%%%%%%%%%%%%%%%%
\section{Introduction}

\LaTeX{} provides a mechanism to structure a large document (such as a book)
into a main file and several child files (containing the chapters)
using the |\include| command.
This mechanism is beneficial for documents
which span hundreds of pages in order to
make the source file(s) more manageable.
Moreover, compilation can be restricted to
selected child files by means of the |\includeonly| command.
The latter feature can be used to reduce the compilation time while editing
(this was significantly more useful in the earlier days of \LaTeX{})
or to generate a smaller document which is easier to navigate.
Another application of |\includeonly| is to generate
documents consisting of selected parts of the complete document.

However, there are a few drawbacks of the plain |\include| mechanism:
\begin{itemize}
\item
The child files cannot be compiled on their own,
they can only be compiled via the main file.
A naive editing environment
(such as a text editor with an option
to have the current file processed by \LaTeX)
may require one to switch to the main file before compiling;
attempting to compile the child file produces errors.
\item
The main file must be modified (each time)
to adjust the |\includeonly| command
to the present needs. This easily leaves the main file in a messy state.
\item
The generated document will always carry the filename
of the main document. This is inconvenient if
several child files are to be compiled and
to be kept for distribution.
\end{itemize}

The present package provides a simple interface
to make child files individually compilable by \LaTeX{}.
Compiling a child file then has the same effect as compiling
the main file with an |\includeonly| command
to select the appropriate child.
Moreover the generated document will carry the name of the child
rather than the main file.
This resolves all three above issues.

This feature is meant to make the editing of books,
thesis documents and lecture notes somewhat more convenient.
However, the package can also be used efficiently for
composing a series of documents (such as exercise sheets)
which are typically distributed individually.
It then assists the author in generating the individual documents
(potentially in different versions)
as well as a document containing the collected series.
Another application is in developing style files
or other kinds of included material
where compilation of the style file could redirect
to a sample or test file.

%%%%%%%%%%%%%%%%%%%%%%%%%%%%%%%%%%%%%%%%%%%%%%%%%%%%%%%%%%%%%%%%%%%%%%%%%%%%%%%%
%%%%%%%%%%%%%%%%%%%%%%%%%%%%%%%%%%%%%%%%%%%%%%%%%%%%%%%%%%%%%%%%%%%%%%%%%%%%%%%%
\section{Usage}

First of all, the package \textsf{childdoc} is \emph{not} a standard
\LaTeXe{} |.sty| style file! Therefore it needs to be invoked in
a non-standard way.

%%%%%%%%%%%%%%%%%%%%%%%%%%%%%%%%%%%%%%%%%%%%%%%%%%%%%%%%%%%%%%%%%%%%%%%%%%%%%%%%
\subsection{Included Files}
\label{sec:include}

%%%%%%%%%%%%%%%%%%%%%%%%%%%%%%%%%%%%%%%%
\DescribeMacro{\childdocmain}
To use the package, add the commands
\begin{center}
\begin{tabular}{l}
|\input{childdoc.def}|\\
|\childdocmain{}|\\
\end{tabular}
\end{center}
at the very top of the main \LaTeX{} file,
in particular \emph{before} the |\documentclass| statement!
The argument of |\childdocmain| should be left empty
(but it must be present).

%%%%%%%%%%%%%%%%%%%%%%%%%%%%%%%%%%%%%%%%
\DescribeMacro{\childdocof}
Furthermore, add the commands
\begin{center}
\begin{tabular}{l}
|\input{childdoc.def}|\\
|\childdocof{|\textit{main}|}|\\
\end{tabular}
\end{center}
at the top of every child file \textit{child}
which is included by |\include{|\textit{child}|}|
from within the main file
(or at least for those files to be compiled individually).
The argument \textit{main} must be the filename of the main file.

There are a couple of
considerations in setting up the main and child documents:

%%%%%%%%%%%%%%%%%%%%%%%%%%%%%%%%%%%%%%%%
\paragraph{Restrictions.}

Please note the following restrictions:
\begin{itemize}
\item
|\childdocmain| must be called with one argument \textit{main}
to ensure compatibility with earlier version of the package.
It must either be empty (|\childdocmain{}|)
or precisely match the filename of the main file in which it is specified.
See \secref{sec:detection} for further information.
\item
The filename \textit{main} must be specified without the |.tex| extension.
\item
The filename \textit{main} is case sensitive
(even in case-insensitive file systems)
due to internal string comparison.
\item
The argument \textit{main} should be fully expanded, it cannot be a macro.
\item
Subdirectories and special characters should be avoided in filenames.
\item
The command |\childdocmain{|\textit{main}|}| must be followed by a whitespace.
It should not be followed immediately by another command
or by a comment mark `|%|'.
This is because the \TeX{} parser reads the token immediately following
the argument of |\childdocmain| and puts it
at the beginning of every child section;
however, a white\-space is ignored.
\end{itemize}

%%%%%%%%%%%%%%%%%%%%%%%%%%%%%%%%%%%%%%%%
\paragraph{Content of Main File.}

It is advisable to place all content in the child files included by |\include|.
Any output contained in the main file will appear in all child documents
unless suppressed manually;
it cannot be suppressed automatically by the |\includeonly| directive
and thus should normally be avoided.
A method to include some content in the main file
by means of conditional processing is described in \secref{sec:conditional}.

%%%%%%%%%%%%%%%%%%%%%%%%%%%%%%%%%%%%%%%%
\paragraph{Page Numbering.}

When only a part of the document is compiled,
the appropriate numbering of pages
(as well as other status parameters)
is determined from the |.aux| files.
The latter contain information from previous passes.
However this information needs to propagate through
all intermediate child documents.
Therefore the page numbering in child documents may well
be inconsistent until the complete document is compiled at least once.

A useful (if unconventional) way to always ensure a consistent
page numbering is to restart the numbering in each child document
and denote the pages by `\textit{child}|.|\textit{page}'
where \textit{child} represents the chapter/section number of the child file.
This can be achieved by the command
|\numberwithin{page}{|\textit{child}|}|
of the \textsf{amsmath} package
where \textit{child} can be |chapter| or |section|
depending on the chosen structuring.
Alternatively, one can modify the macro |\thepage| appropriately
and reset the counter |page| at the start of each child file.

%%%%%%%%%%%%%%%%%%%%%%%%%%%%%%%%%%%%%%%%%%%%%%%%%%%%%%%%%%%%%%%%%%%%%%%%%%%%%%%%
\subsection{Conditional Processing}
\label{sec:conditional}

The package provides a mechanism to compile different versions
of a document. To customise the versions further some conditional processing
can come in handy to distinguish which version is being compiled.
The package provides two macros to describe the compilation context:

%%%%%%%%%%%%%%%%%%%%%%%%%%%%%%%%%%%%%%%%
\DescribeMacro{\ifchilddoc}
The conditional |\ifchilddoc| distinguishes between the compilation of
child documents and the main document:
%
\begin{center}
|\ifchilddoc |\textit{child-code}| |[|\||else |\textit{main-code}]| \||fi|
\end{center}

%%%%%%%%%%%%%%%%%%%%%%%%%%%%%%%%%%%%%%%%
\DescribeMacro{\childdocname}
\DescribeMacro{\childdocjob}
The macro |\childdocname| contains the filename (without extension)
of the main or child file being processed.
Note that |\childdocjob| will always contain the name of the main file.

%%%%%%%%%%%%%%%%%%%%%%%%%%%%%%%%%%%%%%%%
\paragraph{Title Page.}

Conditional processing can be used to include a title or banner page
in the main document when proper precautions are taken.
Importantly, the code in the main file should ensure that the page counter
(as well as other status parameters which are stored in the |.aux| files)
takes the same value after the conditional processing.
Otherwise the page numbers may take divergent values
depending on which part is compiled.

For example, a title page could be declared by:
%
\begin{center}
\begin{tabular}{l}
|\ifchilddoc\||else|\\
|\addtocounter{page}{-1}|\\
\textit{code for title page}\\
|\newpage|\\
|\||fi|
\end{tabular}
\end{center}
%
A banner page for the child documents can be generated by:
%
\begin{center}
\begin{tabular}{l}
|\ifchilddoc|\\
|\addtocounter{page}{-1}|\\
\textit{code for banner page}\\
|\newpage|\\
|\||fi|
\end{tabular}
\end{center}
%
Here one could write a message such as:
\begin{center}
|This is the part \childdocname{} of \childdocjob{}.|
\end{center}

%%%%%%%%%%%%%%%%%%%%%%%%%%%%%%%%%%%%%%%%%%%%%%%%%%%%%%%%%%%%%%%%%%%%%%%%%%%%%%%%
\subsection{Flags}
\label{sec:flags}

The package makes it easy to generate different versions
of the main or child documents.
To this end compilation flags can be defined
and assigned different default values.
They will be particularly useful in conjunction
with the forwarding mechanism described in \secref{sec:forward}.

For example, it may be useful to have a flag |\version|
which can be set to |draft| or |final|.
The document source will contain some conditional code
depending on the value of |\version|.
Suppose further, the flag should default to |final| for the main file
and to |draft| for child files
which is a natural assignment for editing the document.
This is achieved by placing the following code
in the preamble of the main document
(below the |\childdocmain| directive):
%
\begin{center}
\begin{tabular}{l}
|\ifchilddoc|\\
|\providecommand{\version}{draft}|\\
|\||else|\\
|\providecommand{\version}{final}|\\
|\||fi|
\end{tabular}
\end{center}
%
The definition by |\providecommand| makes sure
that previous definitions are not overwritten.
Further statements |\providecommand{\version}{...}|
can thus be added before the above code to override it.

For the main file, one might add a line
(between |\childdocmain| and the above block)
%
\begin{center}
|%\ifchilddoc\||else\providecommand{\version}{draft}\||fi|
\end{center}
%
which can be uncommented to produce a draft version.
Likewise one can add a line to the very top of a child file
(above the |\childdocof{|\textit{main}|}| directive)
%
\begin{center}
|%\providecommand{\version}{final}|
\end{center}
%
which can be uncommented to produce the final version of this child document.

%%%%%%%%%%%%%%%%%%%%%%%%%%%%%%%%%%%%%%%%%%%%%%%%%%%%%%%%%%%%%%%%%%%%%%%%%%%%%%%%
\subsection{Forwarding}
\label{sec:forward}

Different versions of the main or child documents
using compilation flags as described in \secref{sec:flags}
can be (permanently) stored in different files
for convenient compilation, viewing and distribution.
To this end, the package defines a command
to pass on compilation to a different file:

%%%%%%%%%%%%%%%%%%%%%%%%%%%%%%%%%%%%%%%%
\DescribeMacro{\childdocforward}
The command |\childdocforward| redirects processing to
another source file:
%
\begin{center}
\begin{tabular}{l}
|\input{childdoc.def}|\\
|\childdocforward[|\textit{main}|]{|\textit{dest}|}|\\
\end{tabular}
\end{center}
%
The argument \textit{dest} is the destination file
(without extension).
It should be the main file or one of the child files.
Note that further \textsf{childdoc} directives
such as |\childdocof| and |\childdocforward|
in the indicated file will be processed in this form.
The optional argument \textit{main}
passes on directly to the main file \textit{main}
while pretending to compile the child \textit{dest}.
This form behaves as if \textit{dest}
issues |\childdocof{|\textit{main}|}| right away,
and no further \textsf{childdoc} directives will be processed.

%%%%%%%%%%%%%%%%%%%%%%%%%%%%%%%%%%%%%%%%
\DescribeMacro{\...prefix}
In the alternative form |\childdocforwardprefix|,
%
\begin{center}
\begin{tabular}{l}
|\input{childdoc.def}|\\
|\childdocforwardprefix[|\textit{main}|]{|\textit{prefix}|}{|\textit{dest}|}|
\end{tabular}
\end{center}
%
the destination file is determined by a pattern
depending on the current file:
To make this work, the current file must be called
`{\textit{prefix}\hspace{0.2em}\textit{suffix}}'
with \textit{prefix} matching precisely the argument.
Processing is then passed on to the file
`{\textit{dest}\hspace{0.2em}\textit{suffix}}'.
Surely, the same effect is achieved by
directly specifying the
argument `{\textit{dest}\hspace{0.2em}\textit{suffix}}'
in the first form.
However, that requires to set up a different file
for each child. With the alternative form of the command
all these files can have exactly the same content
which simplifies setting them up and maintaining them.

For example, the following file |draft.tex|
with a compilation flag |\version| as described in \secref{sec:flags}
compiles the main document as a draft:
%
\begin{center}
\begin{tabular}{l}
|\def\version{draft}|\\
|\input{childdoc.def}|\\
|\childdocforward{|\textit{main}|}|
\end{tabular}
\end{center}
%
Likewise, the following files |final|\textit{nn}|.tex|
compile the final version of the child document
|child|\textit{nn}|.tex|:
%
\begin{center}
\begin{tabular}{l}
|\def\version{final}|\\
|\input{childdoc.def}|\\
|\childdocforwardprefix{final}{child}|
\end{tabular}
\end{center}
%

Note that when several versions of a main file and/or of each child file
are to be generated, it may be convenient to set up a |Makefile| or
shell script to automatise the process.

%%%%%%%%%%%%%%%%%%%%%%%%%%%%%%%%%%%%%%%%%%%%%%%%%%%%%%%%%%%%%%%%%%%%%%%%%%%%%%%%
\subsection{Command Line Processing}
\label{sec:commandline}

The effect of redirection files can also be achieved by invoking
the \LaTeX{} compiler with a more elaborate command line.
Most conveniently this should be done as part
of a shell script or a |Makefile|.

When using \textsf{childdoc} in the main file, the following
command lines effectively perform a redirection
(note that depending on the shell being used,
backslashes may have to be doubled: `|\|' $\to$ `|\\|'):
%
\begin{center}
|... -jobname "|\textit{target}|" |\\|"|[\textit{flags}]%
|\input{childdoc.def}\childdocforward[|\textit{main}|]{|\textit{dest}|}"|
\end{center}
%
Here \textit{target} is the name of the output file,
\textit{main} is the name of the main file
and \textit{dest} is the name of the main or child file to be processed
(all filenames without extensions).
The optional argument \textit{main} can be omitted
if \textit{main} matches \textit{dest}.
Optionally, compilation \textit{flags} can be defined via |\def| commands.
This command line makes the \TeX{} engine believe
it is compiling the file \textit{target}
whose content is specified as the latter parameter.
The provided code then forwards the processing to
\textit{main} or \textit{dest} as described in \secref{sec:forward}.

%%%%%%%%%%%%%%%%%%%%%%%%%%%%%%%%%%%%%%%%%%%%%%%%%%%%%%%%%%%%%%%%%%%%%%%%%%%%%%%%
\subsection{Include by Input}
\label{sec:input}

Including child documents by |\include| has some restrictions by design.
Most notably, the content of a child document always occupies
its own set of pages; pages cannot be shared between child documents.
Usually, this behaviour makes perfect sense
because each child document contain an essential part of the document.
However, in some situations it may be desirable to compose
a document from a collection of parts
without having mandatory page breaks between then.
For this case, the package
provides a mechanism to include parts
by |\input| which can also be processed individually.
However, by construction this mechanism
requires manual handling of the content to be output.

%%%%%%%%%%%%%%%%%%%%%%%%%%%%%%%%%%%%%%%%
\DescribeMacro{\ifchilddocmanual}
The main file should be prepared as usual, see \secref{sec:include}.
However, the document body must make a distinction
between processing of an individual part and of the main document, e.g.:
%
\begin{center}
\begin{tabular}{l}
|\ifchilddocmanual|\\
|\input{\childdocname}|\\
|\||else|\\
\textit{document body with }|\input{|\textit{part}|}|\\
|\||fi|
\end{tabular}
\end{center}
%
The conditional |\ifchilddocmanual| is true whenever
a part to be included by |\input| is being compiled,
and the name of the part is stored in |\childdocname|.

%%%%%%%%%%%%%%%%%%%%%%%%%%%%%%%%%%%%%%%%
\DescribeMacro{\childdocby}
Each part to be included by |\input| should start with:
%
\begin{center}
\begin{tabular}{l}
|\input{childdoc.def}|\\
|\childdocby{|\textit{main}|}|\\
\end{tabular}
\end{center}
%
The directive |\childdocby| is similar to |\childdocof|
described in \secref{sec:include},
but the subsequent selection of content must be done manually.
To that end, both |\ifchilddoc| and |\ifchilddocmanual|
will be true upon processing of a part,
and the name of the part is stored in |\childdocname|.
Note that |\jobname| will be set to the filename of the current part
so that each part receives an individual |.aux| file
that does not interfere with the |.aux| file(s) of the main document.
This behaviour can be altered by the alternative form
|\childdocby[*]{|\textit{main}|}| (with a non-empty optional argument)
which uses the |.aux| file of the main document
by setting |\jobname| to \textit{main}.

%%%%%%%%%%%%%%%%%%%%%%%%%%%%%%%%%%%%%%%%%%%%%%%%%%%%%%%%%%%%%%%%%%%%%%%%%%%%%%%%
\subsection{Driver Development}
\label{sec:driver}

The \textsf{childdoc} mechanism can also be use for the development
of definition files such as \LaTeX{} styles or classes.
This case differs from the above setup with multiple parts
included by |\include| in that no |\includeonly| should be invoked.
This can be achieved by starting the include file
(before |\ProvidesPackage|) with:
%
\begin{center}
\begin{tabular}{l}
|\input{childdoc.def}|\\
|\childdocforward{|\textit{main}|}|\\
\end{tabular}
\end{center}
%
or alternatively with:
%
\begin{center}
\begin{tabular}{l}
|\input{childdoc.def}|\\
|\childdocby{|\textit{main}|}|\\
\end{tabular}
\end{center}
%
Both forms have slightly different effects as described above.
The main file is prepared as usual, see \secref{sec:include}.

%%%%%%%%%%%%%%%%%%%%%%%%%%%%%%%%%%%%%%%%%%%%%%%%%%%%%%%%%%%%%%%%%%%%%%%%%%%%%%%%
\subsection{Legacy Detection}
\label{sec:detection}

The directive |\childdocmain| in the main file can detect
whether the complete document or merely a child is to be compiled
even without using the directive |\childdocof|.
This method is deprecated because it is less robust
and there is no compelling reason to use it;
it is merely provided for backward compatibility
and it may be removed in future versions.

If the detection mechanism is to be used,
it is mandatory to correctly specify
the filename of the main file as the argument of |\childdocmain|:
%
\begin{center}
\begin{tabular}{l}
|\input{childdoc.def}|\\
|\childdocmain{|\textit{main}|}|\\
\end{tabular}
\end{center}
%
If |\jobname| does not match the argument \textit{main} of |\childdocmain|,
it is assumed that |\jobname| points to the child file to be compiled.
When using |\childdocmain| with the main file specified as argument,
it suffices to start a child file
with just |\input{|\textit{main}|}|
without loading of the package and using |\childdocof|.
If instead all processing is done
with the appropriate \textsf{childdoc} directives,
the argument of \textit{main} of |\childdocmain| can be empty.

An alternative version of the command line processing described
in \secref{sec:commandline} using the detection mechanism reads:
%
\begin{center}
|... -jobname "|\textit{target}|" "|[\textit{flags}]%
[|\def\jobname{|\textit{dest}|}|]|\input{|\textit{main}|}"|
\end{center}

%%%%%%%%%%%%%%%%%%%%%%%%%%%%%%%%%%%%%%%%%%%%%%%%%%%%%%%%%%%%%%%%%%%%%%%%%%%%%%%%
\subsection{Manual Code}
\label{sec:manual}

In case one cannot be certain whether the definitions file |childdoc.def|
is installed on the target \TeX{} distribution
and one prefers not to ship it,
it is conceivable to paste a few relevant commands into the sources.

To that end, drop all statements |\input{childdoc.def}|
and perform the replacements as outlined below.
Instead of |\childdocmain{|\textit{main}|}| add the following code
to the top of the main file:
%
\begin{center}
\begin{tabular}{l}
|\||ifdefined\childdocname\endinput\||fi\newif\ifchilddoc|\\
|\edef\childdocname{\scantokens\expandafter{\jobname\noexpand}}|\\
|\def\childdocmain{|\textit{main}|}\||ifx\childdocmain\childdocname\||else|\\
|\childdoctrue\includeonly{\childdocname}\let\jobname\childdocmain\||fi|\\
\end{tabular}
\end{center}
%
Instead of |\childdocof{|\textit{main}|}| just include the main file
at the top of each child file:
%
\begin{center}
|\input{|\textit{main}|}|
\end{center}
%
A simple redirection |\childdocforward{|\textit{dest}|}| is achieved by:
%
\begin{center}
|\def\jobname{|\textit{dest}|}\input{\jobname}|
\end{center}
%
The redirection with prefix
|\childdocforwardprefix[|\textit{prefix}|]{|\textit{dest}|}|
is accomplished by:
%
\begin{center}
\begin{tabular}{l}
|{\edef\jobname{\scantokens\expandafter{\jobname\noexpand}}|\\
|\def\redirectjob |\textit{prefix}|#1~~~{\gdef\jobname{|\textit{dest}|#1}}|\\
|\expandafter\redirectjob\jobname~~~}\input{\jobname}|
\end{tabular}
\end{center}

In an alternative approach,
child documents can be compiled by a specific command line
without additional code or specific definitions:
%
\begin{center}
|... -jobname "|\textit{target}|" "|[\textit{flags}]%
|\includeonly{|\textit{dest}|}\input{|\textit{main}|}"|
\end{center}
%

%%%%%%%%%%%%%%%%%%%%%%%%%%%%%%%%%%%%%%%%%%%%%%%%%%%%%%%%%%%%%%%%%%%%%%%%%%%%%%%%
%%%%%%%%%%%%%%%%%%%%%%%%%%%%%%%%%%%%%%%%%%%%%%%%%%%%%%%%%%%%%%%%%%%%%%%%%%%%%%%%
\section{Information}

%%%%%%%%%%%%%%%%%%%%%%%%%%%%%%%%%%%%%%%%%%%%%%%%%%%%%%%%%%%%%%%%%%%%%%%%%%%%%%%%
\subsection{Copyright}

Copyright \copyright{} 2017--2018 Niklas Beisert

This work may be distributed and/or modified under the
conditions of the \LaTeX{} Project Public License, either version 1.3
of this license or (at your option) any later version.
The latest version of this license is in
  \url{http://www.latex-project.org/lppl.txt}
and version 1.3 or later is part of all distributions of \LaTeX{}
version 2005/12/01 or later.

This work has the LPPL maintenance status `maintained'.

The Current Maintainer of this work is Niklas Beisert.

This work consists of the files |README.txt|, |childdoc.ins| and |childdoc.dtx|
as well as the derived files |childdoc.def|, |cdocsamp.tex|
with |cdocsch1.tex|, |cdocsch2.tex|, |cdocspt3.tex|, |cdocspt4.tex|,
|cdocsdrf.tex|, |cdocsfn1.tex|, |cdocsfn2.tex|
as well as |childdoc.pdf|.

%%%%%%%%%%%%%%%%%%%%%%%%%%%%%%%%%%%%%%%%%%%%%%%%%%%%%%%%%%%%%%%%%%%%%%%%%%%%%%%%
\subsection{Files and Installation}

The package consists of the files:
%
\begin{center}
\begin{tabular}{ll}
    |README.txt|   & readme file \\
    |childdoc.ins| & installation file \\
    |childdoc.dtx| & source file \\
    |childdoc.def| & definition file \\
    |cdocsamp.tex| & sample main file \\
    |cdocsch1.tex| & sample include file \\
    |cdocsch2.tex| & sample include file \\
    |cdocspt3.tex| & sample part file \\
    |cdocspt4.tex| & sample part file \\
    |cdocsdrf.tex| & sample redirection file \\
    |cdocsfn1.tex| & sample redirection file \\
    |cdocsfn2.tex| & sample redirection file \\
    |childdoc.pdf| & manual
\end{tabular}
\end{center}
%
The distribution consists of the files
|README.txt|, |childdoc.ins| and |childdoc.dtx|.
%
\begin{itemize}
\item
Run (pdf)\LaTeX{} on |childdoc.dtx|
to compile the manual |childdoc.pdf| (this file).
\item
Run \LaTeX{} on |childdoc.ins| to create the definitions file |childdoc.def|
and the sample |cdocsamp.tex| with include files
|cdocsch1.tex|, |cdocsch2.tex|, |cdocspt3.tex|, |cdocspt4.tex|,
|cdocsdrf.tex|, |cdocsfn1.tex|, |cdocsfn2.tex|.
Then copy the file |childdoc.def| to an appropriate directory of your \LaTeX{}
distribution, e.g.\ \textit{texmf-root}|/tex/latex/childdoc|.
\end{itemize}

%%%%%%%%%%%%%%%%%%%%%%%%%%%%%%%%%%%%%%%%%%%%%%%%%%%%%%%%%%%%%%%%%%%%%%%%%%%%%%%%
\subsection{Related CTAN Packages}

There are several other packages which offer a similar functionality:
%
\begin{itemize}
\item
The packages
\href{http://ctan.org/pkg/docmute}{\textsf{docmute}},
\href{http://ctan.org/pkg/includex}{\textsf{includex}} and
\href{http://ctan.org/pkg/standalone}{\textsf{standalone}}
provide commands to include only the document body of
a child file thus allowing both files to be compiled individually.
\item
The packages \href{http://ctan.org/pkg/subdocs}{\textsf{subdocs}}
and \href{http://ctan.org/pkg/subfiles}{\textsf{subfiles}}
provide structures in which the main and child documents can be
encapsulated and allowing them to be compiled individually.
The inclusion mechanism is different from the conventional |\include|.
\item
The package \href{http://ctan.org/pkg/combine}{\textsf{combine}}
is an elaborate solution to combine several documents into one.
\end{itemize}
%
See also the CTAN topic \href{http://ctan.org/topic/subdocs}{\textsf{subdocs}}
for further related packages.
The present package differs from the above solutions in that
a document structure constructed with the conventional |\include| mechanism
just needs two extra commands at the top of every file
such that all constituent files can be compiled individually.

%%%%%%%%%%%%%%%%%%%%%%%%%%%%%%%%%%%%%%%%%%%%%%%%%%%%%%%%%%%%%%%%%%%%%%%%%%%%%%%%
%\subsection{Feature Suggestions}
%
%The following is a list of features which may be useful for future
%versions of this package:
%%
%\begin{itemize}
%\item
%\ldots
%\end{itemize}

%%%%%%%%%%%%%%%%%%%%%%%%%%%%%%%%%%%%%%%%%%%%%%%%%%%%%%%%%%%%%%%%%%%%%%%%%%%%%%%%
\subsection{Revision History}

%%%%%%%%%%%%%%%%%%%%%%%%%%%%%%%%%%%%%%%%
\paragraph{v2.0:} 2018/12/30

\begin{itemize}
\item
immediate forward processing
\item
added |\childdocby| mechanism
\item
manual restructured
\end{itemize}

%%%%%%%%%%%%%%%%%%%%%%%%%%%%%%%%%%%%%%%%
\paragraph{v1.6:} 2018/01/17

\begin{itemize}
\item
application for development of include files
\item
corrections to manual
\end{itemize}

%%%%%%%%%%%%%%%%%%%%%%%%%%%%%%%%%%%%%%%%
\paragraph{v1.5:} 2017/05/21

\begin{itemize}
\item
more complete structuring introduced
\item
|\childdocof| introduced
\item
|\childdoc| renamed to |\childdocmain|
\item
|\childredirect| renamed to |\childdocforward| and |\childdocforwardprefix|
and functionality expanded
\end{itemize}

%%%%%%%%%%%%%%%%%%%%%%%%%%%%%%%%%%%%%%%%
\paragraph{v1.0:} 2017/04/27

\begin{itemize}
\item
manual and install package
\item
first version published on CTAN
\end{itemize}

%%%%%%%%%%%%%%%%%%%%%%%%%%%%%%%%%%%%%%%%
\paragraph{v0.6:} 2017/04/26

\begin{itemize}
\item
redirection mechanism added
\end{itemize}

%%%%%%%%%%%%%%%%%%%%%%%%%%%%%%%%%%%%%%%%
\paragraph{v0.5:} 2017/04/26

\begin{itemize}
\item
functionality in definition file
\end{itemize}


%%%%%%%%%%%%%%%%%%%%%%%%%%%%%%%%%%%%%%%%%%%%%%%%%%%%%%%%%%%%%%%%%%%%%%%%%%%%%%%%
%%%%%%%%%%%%%%%%%%%%%%%%%%%%%%%%%%%%%%%%%%%%%%%%%%%%%%%%%%%%%%%%%%%%%%%%%%%%%%%%
%%%%%%%%%%%%%%%%%%%%%%%%%%%%%%%%%%%%%%%%%%%%%%%%%%%%%%%%%%%%%%%%%%%%%%%%%%%%%%%%
\appendix

\settowidth\MacroIndent{\rmfamily\scriptsize 000\ }

 \DocInput{childdoc.dtx}

\end{document}
%</driver>
% \fi
%
% %%%%%%%%%%%%%%%%%%%%%%%%%%%%%%%%%%%%%%%%%%%%%%%%%%%%%%%%%%%%%%%%%%%%%%%%%%%%%%
% %%%%%%%%%%%%%%%%%%%%%%%%%%%%%%%%%%%%%%%%%%%%%%%%%%%%%%%%%%%%%%%%%%%%%%%%%%%%%%
% \section{Sample}
%\iffalse
%<*samplemain>
%\fi
%
% The following presents a sample document
% with two chapters, two parts, a title page,
% a compile flag as well as three forwarding files to set the flag.
% It consists of eight |.tex| files:
% \begin{center}
% \begin{tabular}{ll}
% |cdocsamp.tex|&main file\\
% |cdocsch1.tex|&include file for chapter 1\\
% |cdocsch2.tex|&include file for chapter 2\\
% |cdocspt3.tex|&include file for part 3\\
% |cdocspt4.tex|&include file for part 4\\
% |cdocsdrf.tex|&forwarding file for main file in draft mode\\
% |cdocsfi1.tex|&forwarding file for final version of chapter 1\\
% |cdocsfi2.tex|&forwarding file for final version of chapter 2\\
% \end{tabular}
% \end{center}
% Each of the eight files can be compiled directly by the \LaTeX{} compiler.
%
% %%%%%%%%%%%%%%%%%%%%%%%%%%%%%%%%%%%%%%
% \paragraph{Main File.}
%
% The main file is called |cdocsamp.tex|.
%
% Load the \textsf{childdoc} definitions and
% declare the filename for the main document:
%    \begin{macrocode}
\input{childdoc.def}
\childdocmain{}
%    \end{macrocode}

% Optional override for |\version| flag:
%    \begin{macrocode}
%%\ifchilddoc\else\providecommand{\version}{draft}\fi
%    \end{macrocode}

% Define the default values for the |\version| flag
% (|final| for the main file and |draft| for childs):
%    \begin{macrocode}
\ifchilddoc
\providecommand{\version}{draft}
\else
\providecommand{\version}{final}
\fi
%    \end{macrocode}

% Load the standard document class:
%    \begin{macrocode}
\documentclass[12pt]{article}
%    \end{macrocode}

% Start the document body:
%    \begin{macrocode}
\begin{document}
%    \end{macrocode}

% Declare a title page.
% Print title, part of document being processed and version flag:
%    \begin{macrocode}
\addtocounter{page}{-1}
\begin{center}
{\LARGE\bfseries{}childdoc example\par}
\vspace{1cm}
\ifchilddoc
\ifchilddocmanual part\else chapter\fi:
`\childdocname' of `\childdocjob'\par
\else
main document: `\childdocjob'\par
\fi
version: \version\par
\end{center}
\newpage
%    \end{macrocode}

% Manually include selected file,
% otherwise process as usual:
%    \begin{macrocode}
\ifchilddocmanual
\section*{part `\childdocname'}
\input{\childdocname}
\else
%    \end{macrocode}

% Include the two chapters:
%    \begin{macrocode}
\include{cdocsch1}
\include{cdocsch2}
%    \end{macrocode}

% Include the two parts unless only chapters should be displayed:
%    \begin{macrocode}
\ifchilddoc\else
\section{part three}
\input{cdocspt3}
\section{part four}
\input{cdocspt4}
\fi
%    \end{macrocode}

% Process as usual until here:
%    \begin{macrocode}
\fi
%    \end{macrocode}

% End of document body:
%    \begin{macrocode}
\end{document}
%    \end{macrocode}
%\iffalse
%</samplemain>
%\fi
%
% %%%%%%%%%%%%%%%%%%%%%%%%%%%%%%%%%%%%%%
% \paragraph{Chapter Include Files.}
%
% The include files are called |cdocsch1.tex| and |cdocsch2.tex|.
%
%\iffalse
%<*samplechap1|samplechap2>
%\fi

% Optional override for |\version| flag:
%    \begin{macrocode}
%%\providecommand{\version}{final}
%    \end{macrocode}

% Include the main document:
%    \begin{macrocode}
\input{childdoc.def}
\childdocof{cdocsamp}
%    \end{macrocode}

%\iffalse
%</samplechap1|samplechap2>
%\fi
%
%\iffalse
%<*samplechap1>
%\fi
% Some text for chapter 1:
%    \begin{macrocode}
\section{one}
some text in chapter one
%    \end{macrocode}

%\iffalse
%</samplechap1>
%\fi
% Some text for chapter 2:
%\iffalse
%<*samplechap2>
%\fi
%    \begin{macrocode}
\section{two}
more text in chapter two
%    \end{macrocode}

%\iffalse
%</samplechap2>
%\fi
%
% %%%%%%%%%%%%%%%%%%%%%%%%%%%%%%%%%%%%%%
% \paragraph{Part Include Files.}
%
% The include files are called |cdocspt3.tex| and |cdocspt4.tex|.
%
%\iffalse
%<*samplepart3|samplepart4>
%\fi

% Optional override for |\version| flag:
%    \begin{macrocode}
%%\providecommand{\version}{final}
%    \end{macrocode}

% Include the main document:
%    \begin{macrocode}
\input{childdoc.def}
\childdocby{cdocsamp}
%    \end{macrocode}

%\iffalse
%</samplepart3|samplepart4>
%\fi
%
%\iffalse
%<*samplepart3>
%\fi
% Some text for part 3:
%    \begin{macrocode}
some text in part three
%    \end{macrocode}

%\iffalse
%</samplepart3>
%\fi
% Some text for part 4:
%\iffalse
%<*samplepart4>
%\fi
%    \begin{macrocode}
more text in part four
%    \end{macrocode}

%\iffalse
%</samplepart4>
%\fi
%
% %%%%%%%%%%%%%%%%%%%%%%%%%%%%%%%%%%%%%%
% \paragraph{Forwarding for a Complete Draft.}
%
% The following forwarding file |cdocsdrf.tex|
% compiles the main document in draft mode:
%\iffalse
%<*sampledraft>
%\fi
%    \begin{macrocode}
\def\version{draft}
\input{childdoc.def}
\childdocforward{cdocsamp}
%    \end{macrocode}

%\iffalse
%</sampledraft>
%\fi
%
% %%%%%%%%%%%%%%%%%%%%%%%%%%%%%%%%%%%%%%
% \paragraph{Forwarding for Final Version of the Chapters.}
%
% The following forwarding files |cdocsfn1.tex| and |cdocsfn2.tex|
% (with identical content)
% compile the final versions of the child documents
% |cdocsch1.tex| and |cdocsch2.tex|, respectively:
%\iffalse
%<*samplefinal>
%\fi
%    \begin{macrocode}
\def\version{final}
\input{childdoc.def}
\childdocforwardprefix[cdocsamp]{cdocsfn}{cdocsch}
%    \end{macrocode}

%\iffalse
%</samplefinal>
%\fi
%
% %%%%%%%%%%%%%%%%%%%%%%%%%%%%%%%%%%%%%%
% \paragraph{Command Line Processing.}
%
% The following three command lines generate the output files
% |cdocscld|, |cdocscl1| and |cdocscl2|
% which should be identical to
% |cdocsdrf|, |cdocsch1| and |cdocsfn2|, respectively:
% \begin{center}
% \begin{tabular}{l}
% |latex -jobname cdocscld \|\\
% |  "\def\version{draft}\input{childdoc.def}\childdocforward{cdocsamp}"|\\
% |latex -jobname cdocscl1 \|\\
% |  "\input{childdoc.def}\childdocforward[cdocsamp]{cdocsch1}"|\\
% |latex -jobname cdocscl2 \|\\
% |  "\def\version{final}\input{childdoc.def}\childdocforward{cdocsch2}"|
% \end{tabular}
% \end{center}
% Note that the trailing backslash on each first line
% merely continues the input to the second line
% (for convenient cut ant paste).
% Furthermore, the command |latex| can be replaced by any
% of its alternative versions such as |pdflatex|.
%
% %%%%%%%%%%%%%%%%%%%%%%%%%%%%%%%%%%%%%%%%%%%%%%%%%%%%%%%%%%%%%%%%%%%%%%%%%%%%%%
% %%%%%%%%%%%%%%%%%%%%%%%%%%%%%%%%%%%%%%%%%%%%%%%%%%%%%%%%%%%%%%%%%%%%%%%%%%%%%%
% \section{Implementation}
%\iffalse
%<*package>
%\fi
%
% This section describes the definitions file |childdoc.def|.

% The definitions cannot be loaded using |\usepackage| or |\RequirePackage|
% which has a mechanism to prevent loading a style file more than once.
% When loading the definitions by means of |\input|
% multiple instances have to be prevented manually:
%\iffalse
%This code needs to be before the `\ProvidesFile' directive
%which is defined at the beginning of this file.
%Therefore it is also placed there and commented out here.
%</package>
%<*discard>
%\fi
%    \begin{macrocode}
\ifdefined\childdocmain\endinput\fi
%    \end{macrocode}
%\iffalse
%</discard>
%<*package>
%\fi
%
% \macro{\ifchilddoc}
% \macro{\ifchilddocmanual}
% The conditional |\ifchilddoc| tells whether a
% child (true) or main (false) document is being compiled.
% The conditional |\ifchilddocmanual| tells whether
% the |\includeonly| mechanism is used (false) or
% the selection of child files must be performed manually (true).
% The definitions initialise to false:
%    \begin{macrocode}
\newif\ifchilddoc
\newif\ifchilddocmanual
%    \end{macrocode}

% \macro{\childdocname}
% \macro{\childdocjob}
% The macro |\childdocname| stores the name of the main document
% to be compiled. The macro |\childdocjob| stores the name of
% the document on which the \LaTeX{} compiler was originally invoked.
% The content of |\jobname| cannot be compared
% to filenames specified in the source due to different catcodes.
% The following code rescans |\jobname|, stores the result
% in |\childdocname| and saves a copy in |\childdocjob|:
%    \begin{macrocode}
\edef\childdocname{\scantokens\expandafter{\jobname\noexpand}}
\let\childdocjob\childdocname
%    \end{macrocode}

% \macro{\childdocdisable}
% The macro |\childdocdisable| prevents the main file
% from being processed more than once.
% At this stage, the main document command |\childdocmain|
% is assumed to be called once again where it should do nothing.
% Any subsequent call to it should prevent
% a secondary processing of the main document
% It overwrites the forwarding commands
% |\childdocof| and |\childdocforward|
% with empty macros to prevent further inclusions of the main document:
%    \begin{macrocode}
\newcommand{\childdocdisable}
{
  \renewcommand{\childdocmain}[1]{\renewcommand{\childdocmain}[1]{\endinput}}
  \renewcommand{\childdocof}[1]{}
  \renewcommand{\childdocby}[2][]{}
  \renewcommand{\childdocforward}[2][]{}
  \renewcommand{\childdocdisable}{}
}
%    \end{macrocode}

% \macro{\childdocmain}
% The macro |\childdocmain| is to be called at the top of the main file
% with nothing or the main filename (without extension) as argument.
% First, it breaks loops.
% If the argument is not empty and does not match |\childdocname|
% (which is set by the first inclusion of |childdoc.def|),
% |\ifchilddoc| is set to true, |\includeonly| is applied to the child file
% and |\jobname| is set to the main file
% (for proper handling of |.aux| files):
%    \begin{macrocode}
\newcommand{\childdocmain}[1]
{
  \childdocdisable\childdocmain{}
  \if?#1?\else
    \begingroup
      \def\childdoctmp{#1}
      \ifx\childdoctmp\childdocname
        \def\childdoctmp{}
      \else
        \def\childdoctmp
        {
          \childdoctrue
          \includeonly{\childdocname}
          \def\childdocjob{#1}
          \def\jobname{#1}
        }
      \fi
      \expandafter
    \endgroup
    \childdoctmp
  \fi
}
%    \end{macrocode}

% \macro{\childdocof}
% The command |\childdocof| redirects
% compilation to the main file |#1|.
%    \begin{macrocode}
\newcommand{\childdocof}[1]
{
  \childdocdisable
  \childdoctrue
  \includeonly{\childdocname}
  \def\jobname{#1}
  \def\childdocjob{#1}
  \input{#1}
}
%    \end{macrocode}

% \macro{\childdocby}
% The command |\childdocby| ....
%    \begin{macrocode}
\newcommand{\childdocby}[2][]
{
  \childdocdisable
  \childdoctrue
  \childdocmanualtrue
  \if?#1?\else
    \def\jobname{#2}
  \fi
  \def\childdocjob{#2}
  \input{#2}
  \endinput
}
%    \end{macrocode}

% \macro{\childdocforward}
% The command |\childdocforward| redirects
% compilation to the main file or
% (if the optional argument is given) a child file.
% Parameters are set as if the main file
% or a child file starting with |\childdocof| was compiled.
% Then compilation is handed over to the main file:
%    \begin{macrocode}
\newcommand{\childdocforward}[2][]
{
  \begingroup
    \if?#1?
      \def\childdoctmp
      {
        \def\childdocname{#2}
        \def\childdocjob{#2}
        \def\jobname{#2}
        \input{#2}
        \endinput
      }
    \else
      \def\childdoctmp
      {
        \childdocdisable
        \def\childdocname{#2}
        \childdoctrue
        \includeonly{#2}
        \def\childdocjob{#1}
        \def\jobname{#1}
        \input{#1}
        \endinput
      }
    \fi
    \expandafter
  \endgroup
  \childdoctmp
}
%    \end{macrocode}

% \macro{\childdocforwardprefix}
% The command |\childdocforwardprefix| redirects
% compilation to the main or a child file by means of a pattern.
% The prefix |#1| in the current filename is replaced by |#2|
% and the suffix of the current filename is kept
% (it is assumed that the filename does not contain the substring `|~~~|'
% which is used as a delimiter).
% Compilation is handed over to the new file by |\childdocforward|:
%    \begin{macrocode}
\newcommand{\childdocforwardprefix}[3][]
{
  \begingroup
    \def\childdocextract #2##1~~~{\def\childdoctmp{\childdocforward[#1]{#3##1}}}
    \expandafter\childdocextract\childdocname~~~
    \expandafter
  \endgroup
  \childdoctmp
}
%    \end{macrocode}

% \macro{\childdoc}
% The deprecated macro |\childdoc| is a legacy version of |\childdocmain|:
%    \begin{macrocode}
\newcommand{\childdoc}{\childdocmain}
%    \end{macrocode}

% \macro{\childdocredirect}
% The deprecated macro |\childdocredirect| is a legacy version
% of |\childdocforward| and |\childdocforwardprefix|:
%    \begin{macrocode}
\newcommand{\childdocredirect}[2][]
{
  \begingroup
    \if?#1?
      \def\childdoctmp{\childdocforward{#2}}
    \else
      \def\childdoctmp{\childdocforwardprefix{#1}{#2}}
    \fi
    \expandafter
  \endgroup
  \childdoctmp
}
%    \end{macrocode}

%\iffalse
%</package>
%\fi
%
\endinput
|\\
|\childdocforwardprefix{final}{child}|
\end{tabular}
\end{center}
%

Note that when several versions of a main file and/or of each child file
are to be generated, it may be convenient to set up a |Makefile| or
shell script to automatise the process.

%%%%%%%%%%%%%%%%%%%%%%%%%%%%%%%%%%%%%%%%%%%%%%%%%%%%%%%%%%%%%%%%%%%%%%%%%%%%%%%%
\subsection{Command Line Processing}
\label{sec:commandline}

The effect of redirection files can also be achieved by invoking
the \LaTeX{} compiler with a more elaborate command line.
Most conveniently this should be done as part
of a shell script or a |Makefile|.

When using \textsf{childdoc} in the main file, the following
command lines effectively perform a redirection
(note that depending on the shell being used,
backslashes may have to be doubled: `|\|' $\to$ `|\\|'):
%
\begin{center}
|... -jobname "|\textit{target}|" |\\|"|[\textit{flags}]%
|% \iffalse
%
% childdoc.dtx Copyright (C) 2017-2018 Niklas Beisert
%
% This work may be distributed and/or modified under the
% conditions of the LaTeX Project Public License, either version 1.3
% of this license or (at your option) any later version.
% The latest version of this license is in
%   http://www.latex-project.org/lppl.txt
% and version 1.3 or later is part of all distributions of LaTeX
% version 2005/12/01 or later.
%
% This work has the LPPL maintenance status `maintained'.
%
% The Current Maintainer of this work is Niklas Beisert.
%
% This work consists of the files childdoc.dtx and childdoc.ins
% and the derived files childdoc.def and cdocsamp.tex with
% cdocsch1.tex, cdocsch2.tex, cdocsdrf.tex, cdocsfn1.tex, cdocsfn2.tex.
%
%<package>\ifdefined\childdocmain\endinput\fi
%<package>\ProvidesFile{childdoc.def}[2018/12/30 v2.0 child document driver]
%<samplemain>\ProvidesFile{cdocsamp.tex}[2018/12/30 v2.0 sample for childdoc]
%<*driver>
%\ProvidesFile{childdoc.drv}[2018/12/30 v2.0 childdoc reference manual file]
\PassOptionsToClass{10pt,a4paper}{article}
\documentclass{ltxdoc}

\usepackage[margin=35mm]{geometry}
\usepackage{hyperref}
\usepackage{hyperxmp}
\usepackage[usenames]{color}

\hypersetup{colorlinks=true}
\hypersetup{pdfstartview=FitH}
\hypersetup{pdfpagemode=UseNone}
\hypersetup{pdfsource={}}
\hypersetup{pdflang={en-UK}}
\hypersetup{pdfcopyright={Copyright 2017-2018 Niklas Beisert.
  This work may be distributed and/or modified under the
  conditions of the LaTeX Project Public License, either version 1.3
  of this license or (at your option) any later version.}}
\hypersetup{pdflicenseurl={http://www.latex-project.org/lppl.txt}}
\hypersetup{pdfcontactaddress={ETH Zurich, ITP, HIT K,
  Wolfgang-Pauli-Strasse 27}}
\hypersetup{pdfcontactpostcode={8093}}
\hypersetup{pdfcontactcity={Zurich}}
\hypersetup{pdfcontactcountry={Switzerland}}
\hypersetup{pdfcontactemail={nbeisert@itp.phys.ethz.ch}}
\hypersetup{pdfcontacturl={http://people.phys.ethz.ch/\xmptilde nbeisert/}}

\newcommand{\secref}[1]{\hyperref[#1]{section \ref*{#1}}}

\parskip1ex
\parindent0pt
\let\olditemize\itemize
\def\itemize{\olditemize\parskip0pt}

\begin{document}

\title{The \textsf{childdoc} Package}
\hypersetup{pdftitle={The childdoc Package}}
\author{Niklas Beisert\\[2ex]
  Institut f\"ur Theoretische Physik\\
  Eidgen\"ossische Technische Hochschule Z\"urich\\
  Wolfgang-Pauli-Strasse 27, 8093 Z\"urich, Switzerland\\[1ex]
  \href{mailto:nbeisert@itp.phys.ethz.ch}
  {\texttt{nbeisert@itp.phys.ethz.ch}}}
\hypersetup{pdfauthor={Niklas Beisert}}
\hypersetup{pdfsubject={Manual for the LaTeX2e Package childdoc}}
\date{30 December 2018, \textsf{v2.0}}
\maketitle

\begin{abstract}\noindent
\textsf{childdoc} is a \LaTeXe{} package
that enables the direct compilation
of document sections included by |\include|
to individual files.
\end{abstract}

\begingroup
\parskip0ex
\tableofcontents
\endgroup

%%%%%%%%%%%%%%%%%%%%%%%%%%%%%%%%%%%%%%%%%%%%%%%%%%%%%%%%%%%%%%%%%%%%%%%%%%%%%%%%
%%%%%%%%%%%%%%%%%%%%%%%%%%%%%%%%%%%%%%%%%%%%%%%%%%%%%%%%%%%%%%%%%%%%%%%%%%%%%%%%
\section{Introduction}

\LaTeX{} provides a mechanism to structure a large document (such as a book)
into a main file and several child files (containing the chapters)
using the |\include| command.
This mechanism is beneficial for documents
which span hundreds of pages in order to
make the source file(s) more manageable.
Moreover, compilation can be restricted to
selected child files by means of the |\includeonly| command.
The latter feature can be used to reduce the compilation time while editing
(this was significantly more useful in the earlier days of \LaTeX{})
or to generate a smaller document which is easier to navigate.
Another application of |\includeonly| is to generate
documents consisting of selected parts of the complete document.

However, there are a few drawbacks of the plain |\include| mechanism:
\begin{itemize}
\item
The child files cannot be compiled on their own,
they can only be compiled via the main file.
A naive editing environment
(such as a text editor with an option
to have the current file processed by \LaTeX)
may require one to switch to the main file before compiling;
attempting to compile the child file produces errors.
\item
The main file must be modified (each time)
to adjust the |\includeonly| command
to the present needs. This easily leaves the main file in a messy state.
\item
The generated document will always carry the filename
of the main document. This is inconvenient if
several child files are to be compiled and
to be kept for distribution.
\end{itemize}

The present package provides a simple interface
to make child files individually compilable by \LaTeX{}.
Compiling a child file then has the same effect as compiling
the main file with an |\includeonly| command
to select the appropriate child.
Moreover the generated document will carry the name of the child
rather than the main file.
This resolves all three above issues.

This feature is meant to make the editing of books,
thesis documents and lecture notes somewhat more convenient.
However, the package can also be used efficiently for
composing a series of documents (such as exercise sheets)
which are typically distributed individually.
It then assists the author in generating the individual documents
(potentially in different versions)
as well as a document containing the collected series.
Another application is in developing style files
or other kinds of included material
where compilation of the style file could redirect
to a sample or test file.

%%%%%%%%%%%%%%%%%%%%%%%%%%%%%%%%%%%%%%%%%%%%%%%%%%%%%%%%%%%%%%%%%%%%%%%%%%%%%%%%
%%%%%%%%%%%%%%%%%%%%%%%%%%%%%%%%%%%%%%%%%%%%%%%%%%%%%%%%%%%%%%%%%%%%%%%%%%%%%%%%
\section{Usage}

First of all, the package \textsf{childdoc} is \emph{not} a standard
\LaTeXe{} |.sty| style file! Therefore it needs to be invoked in
a non-standard way.

%%%%%%%%%%%%%%%%%%%%%%%%%%%%%%%%%%%%%%%%%%%%%%%%%%%%%%%%%%%%%%%%%%%%%%%%%%%%%%%%
\subsection{Included Files}
\label{sec:include}

%%%%%%%%%%%%%%%%%%%%%%%%%%%%%%%%%%%%%%%%
\DescribeMacro{\childdocmain}
To use the package, add the commands
\begin{center}
\begin{tabular}{l}
|\input{childdoc.def}|\\
|\childdocmain{}|\\
\end{tabular}
\end{center}
at the very top of the main \LaTeX{} file,
in particular \emph{before} the |\documentclass| statement!
The argument of |\childdocmain| should be left empty
(but it must be present).

%%%%%%%%%%%%%%%%%%%%%%%%%%%%%%%%%%%%%%%%
\DescribeMacro{\childdocof}
Furthermore, add the commands
\begin{center}
\begin{tabular}{l}
|\input{childdoc.def}|\\
|\childdocof{|\textit{main}|}|\\
\end{tabular}
\end{center}
at the top of every child file \textit{child}
which is included by |\include{|\textit{child}|}|
from within the main file
(or at least for those files to be compiled individually).
The argument \textit{main} must be the filename of the main file.

There are a couple of
considerations in setting up the main and child documents:

%%%%%%%%%%%%%%%%%%%%%%%%%%%%%%%%%%%%%%%%
\paragraph{Restrictions.}

Please note the following restrictions:
\begin{itemize}
\item
|\childdocmain| must be called with one argument \textit{main}
to ensure compatibility with earlier version of the package.
It must either be empty (|\childdocmain{}|)
or precisely match the filename of the main file in which it is specified.
See \secref{sec:detection} for further information.
\item
The filename \textit{main} must be specified without the |.tex| extension.
\item
The filename \textit{main} is case sensitive
(even in case-insensitive file systems)
due to internal string comparison.
\item
The argument \textit{main} should be fully expanded, it cannot be a macro.
\item
Subdirectories and special characters should be avoided in filenames.
\item
The command |\childdocmain{|\textit{main}|}| must be followed by a whitespace.
It should not be followed immediately by another command
or by a comment mark `|%|'.
This is because the \TeX{} parser reads the token immediately following
the argument of |\childdocmain| and puts it
at the beginning of every child section;
however, a white\-space is ignored.
\end{itemize}

%%%%%%%%%%%%%%%%%%%%%%%%%%%%%%%%%%%%%%%%
\paragraph{Content of Main File.}

It is advisable to place all content in the child files included by |\include|.
Any output contained in the main file will appear in all child documents
unless suppressed manually;
it cannot be suppressed automatically by the |\includeonly| directive
and thus should normally be avoided.
A method to include some content in the main file
by means of conditional processing is described in \secref{sec:conditional}.

%%%%%%%%%%%%%%%%%%%%%%%%%%%%%%%%%%%%%%%%
\paragraph{Page Numbering.}

When only a part of the document is compiled,
the appropriate numbering of pages
(as well as other status parameters)
is determined from the |.aux| files.
The latter contain information from previous passes.
However this information needs to propagate through
all intermediate child documents.
Therefore the page numbering in child documents may well
be inconsistent until the complete document is compiled at least once.

A useful (if unconventional) way to always ensure a consistent
page numbering is to restart the numbering in each child document
and denote the pages by `\textit{child}|.|\textit{page}'
where \textit{child} represents the chapter/section number of the child file.
This can be achieved by the command
|\numberwithin{page}{|\textit{child}|}|
of the \textsf{amsmath} package
where \textit{child} can be |chapter| or |section|
depending on the chosen structuring.
Alternatively, one can modify the macro |\thepage| appropriately
and reset the counter |page| at the start of each child file.

%%%%%%%%%%%%%%%%%%%%%%%%%%%%%%%%%%%%%%%%%%%%%%%%%%%%%%%%%%%%%%%%%%%%%%%%%%%%%%%%
\subsection{Conditional Processing}
\label{sec:conditional}

The package provides a mechanism to compile different versions
of a document. To customise the versions further some conditional processing
can come in handy to distinguish which version is being compiled.
The package provides two macros to describe the compilation context:

%%%%%%%%%%%%%%%%%%%%%%%%%%%%%%%%%%%%%%%%
\DescribeMacro{\ifchilddoc}
The conditional |\ifchilddoc| distinguishes between the compilation of
child documents and the main document:
%
\begin{center}
|\ifchilddoc |\textit{child-code}| |[|\||else |\textit{main-code}]| \||fi|
\end{center}

%%%%%%%%%%%%%%%%%%%%%%%%%%%%%%%%%%%%%%%%
\DescribeMacro{\childdocname}
\DescribeMacro{\childdocjob}
The macro |\childdocname| contains the filename (without extension)
of the main or child file being processed.
Note that |\childdocjob| will always contain the name of the main file.

%%%%%%%%%%%%%%%%%%%%%%%%%%%%%%%%%%%%%%%%
\paragraph{Title Page.}

Conditional processing can be used to include a title or banner page
in the main document when proper precautions are taken.
Importantly, the code in the main file should ensure that the page counter
(as well as other status parameters which are stored in the |.aux| files)
takes the same value after the conditional processing.
Otherwise the page numbers may take divergent values
depending on which part is compiled.

For example, a title page could be declared by:
%
\begin{center}
\begin{tabular}{l}
|\ifchilddoc\||else|\\
|\addtocounter{page}{-1}|\\
\textit{code for title page}\\
|\newpage|\\
|\||fi|
\end{tabular}
\end{center}
%
A banner page for the child documents can be generated by:
%
\begin{center}
\begin{tabular}{l}
|\ifchilddoc|\\
|\addtocounter{page}{-1}|\\
\textit{code for banner page}\\
|\newpage|\\
|\||fi|
\end{tabular}
\end{center}
%
Here one could write a message such as:
\begin{center}
|This is the part \childdocname{} of \childdocjob{}.|
\end{center}

%%%%%%%%%%%%%%%%%%%%%%%%%%%%%%%%%%%%%%%%%%%%%%%%%%%%%%%%%%%%%%%%%%%%%%%%%%%%%%%%
\subsection{Flags}
\label{sec:flags}

The package makes it easy to generate different versions
of the main or child documents.
To this end compilation flags can be defined
and assigned different default values.
They will be particularly useful in conjunction
with the forwarding mechanism described in \secref{sec:forward}.

For example, it may be useful to have a flag |\version|
which can be set to |draft| or |final|.
The document source will contain some conditional code
depending on the value of |\version|.
Suppose further, the flag should default to |final| for the main file
and to |draft| for child files
which is a natural assignment for editing the document.
This is achieved by placing the following code
in the preamble of the main document
(below the |\childdocmain| directive):
%
\begin{center}
\begin{tabular}{l}
|\ifchilddoc|\\
|\providecommand{\version}{draft}|\\
|\||else|\\
|\providecommand{\version}{final}|\\
|\||fi|
\end{tabular}
\end{center}
%
The definition by |\providecommand| makes sure
that previous definitions are not overwritten.
Further statements |\providecommand{\version}{...}|
can thus be added before the above code to override it.

For the main file, one might add a line
(between |\childdocmain| and the above block)
%
\begin{center}
|%\ifchilddoc\||else\providecommand{\version}{draft}\||fi|
\end{center}
%
which can be uncommented to produce a draft version.
Likewise one can add a line to the very top of a child file
(above the |\childdocof{|\textit{main}|}| directive)
%
\begin{center}
|%\providecommand{\version}{final}|
\end{center}
%
which can be uncommented to produce the final version of this child document.

%%%%%%%%%%%%%%%%%%%%%%%%%%%%%%%%%%%%%%%%%%%%%%%%%%%%%%%%%%%%%%%%%%%%%%%%%%%%%%%%
\subsection{Forwarding}
\label{sec:forward}

Different versions of the main or child documents
using compilation flags as described in \secref{sec:flags}
can be (permanently) stored in different files
for convenient compilation, viewing and distribution.
To this end, the package defines a command
to pass on compilation to a different file:

%%%%%%%%%%%%%%%%%%%%%%%%%%%%%%%%%%%%%%%%
\DescribeMacro{\childdocforward}
The command |\childdocforward| redirects processing to
another source file:
%
\begin{center}
\begin{tabular}{l}
|\input{childdoc.def}|\\
|\childdocforward[|\textit{main}|]{|\textit{dest}|}|\\
\end{tabular}
\end{center}
%
The argument \textit{dest} is the destination file
(without extension).
It should be the main file or one of the child files.
Note that further \textsf{childdoc} directives
such as |\childdocof| and |\childdocforward|
in the indicated file will be processed in this form.
The optional argument \textit{main}
passes on directly to the main file \textit{main}
while pretending to compile the child \textit{dest}.
This form behaves as if \textit{dest}
issues |\childdocof{|\textit{main}|}| right away,
and no further \textsf{childdoc} directives will be processed.

%%%%%%%%%%%%%%%%%%%%%%%%%%%%%%%%%%%%%%%%
\DescribeMacro{\...prefix}
In the alternative form |\childdocforwardprefix|,
%
\begin{center}
\begin{tabular}{l}
|\input{childdoc.def}|\\
|\childdocforwardprefix[|\textit{main}|]{|\textit{prefix}|}{|\textit{dest}|}|
\end{tabular}
\end{center}
%
the destination file is determined by a pattern
depending on the current file:
To make this work, the current file must be called
`{\textit{prefix}\hspace{0.2em}\textit{suffix}}'
with \textit{prefix} matching precisely the argument.
Processing is then passed on to the file
`{\textit{dest}\hspace{0.2em}\textit{suffix}}'.
Surely, the same effect is achieved by
directly specifying the
argument `{\textit{dest}\hspace{0.2em}\textit{suffix}}'
in the first form.
However, that requires to set up a different file
for each child. With the alternative form of the command
all these files can have exactly the same content
which simplifies setting them up and maintaining them.

For example, the following file |draft.tex|
with a compilation flag |\version| as described in \secref{sec:flags}
compiles the main document as a draft:
%
\begin{center}
\begin{tabular}{l}
|\def\version{draft}|\\
|\input{childdoc.def}|\\
|\childdocforward{|\textit{main}|}|
\end{tabular}
\end{center}
%
Likewise, the following files |final|\textit{nn}|.tex|
compile the final version of the child document
|child|\textit{nn}|.tex|:
%
\begin{center}
\begin{tabular}{l}
|\def\version{final}|\\
|\input{childdoc.def}|\\
|\childdocforwardprefix{final}{child}|
\end{tabular}
\end{center}
%

Note that when several versions of a main file and/or of each child file
are to be generated, it may be convenient to set up a |Makefile| or
shell script to automatise the process.

%%%%%%%%%%%%%%%%%%%%%%%%%%%%%%%%%%%%%%%%%%%%%%%%%%%%%%%%%%%%%%%%%%%%%%%%%%%%%%%%
\subsection{Command Line Processing}
\label{sec:commandline}

The effect of redirection files can also be achieved by invoking
the \LaTeX{} compiler with a more elaborate command line.
Most conveniently this should be done as part
of a shell script or a |Makefile|.

When using \textsf{childdoc} in the main file, the following
command lines effectively perform a redirection
(note that depending on the shell being used,
backslashes may have to be doubled: `|\|' $\to$ `|\\|'):
%
\begin{center}
|... -jobname "|\textit{target}|" |\\|"|[\textit{flags}]%
|\input{childdoc.def}\childdocforward[|\textit{main}|]{|\textit{dest}|}"|
\end{center}
%
Here \textit{target} is the name of the output file,
\textit{main} is the name of the main file
and \textit{dest} is the name of the main or child file to be processed
(all filenames without extensions).
The optional argument \textit{main} can be omitted
if \textit{main} matches \textit{dest}.
Optionally, compilation \textit{flags} can be defined via |\def| commands.
This command line makes the \TeX{} engine believe
it is compiling the file \textit{target}
whose content is specified as the latter parameter.
The provided code then forwards the processing to
\textit{main} or \textit{dest} as described in \secref{sec:forward}.

%%%%%%%%%%%%%%%%%%%%%%%%%%%%%%%%%%%%%%%%%%%%%%%%%%%%%%%%%%%%%%%%%%%%%%%%%%%%%%%%
\subsection{Include by Input}
\label{sec:input}

Including child documents by |\include| has some restrictions by design.
Most notably, the content of a child document always occupies
its own set of pages; pages cannot be shared between child documents.
Usually, this behaviour makes perfect sense
because each child document contain an essential part of the document.
However, in some situations it may be desirable to compose
a document from a collection of parts
without having mandatory page breaks between then.
For this case, the package
provides a mechanism to include parts
by |\input| which can also be processed individually.
However, by construction this mechanism
requires manual handling of the content to be output.

%%%%%%%%%%%%%%%%%%%%%%%%%%%%%%%%%%%%%%%%
\DescribeMacro{\ifchilddocmanual}
The main file should be prepared as usual, see \secref{sec:include}.
However, the document body must make a distinction
between processing of an individual part and of the main document, e.g.:
%
\begin{center}
\begin{tabular}{l}
|\ifchilddocmanual|\\
|\input{\childdocname}|\\
|\||else|\\
\textit{document body with }|\input{|\textit{part}|}|\\
|\||fi|
\end{tabular}
\end{center}
%
The conditional |\ifchilddocmanual| is true whenever
a part to be included by |\input| is being compiled,
and the name of the part is stored in |\childdocname|.

%%%%%%%%%%%%%%%%%%%%%%%%%%%%%%%%%%%%%%%%
\DescribeMacro{\childdocby}
Each part to be included by |\input| should start with:
%
\begin{center}
\begin{tabular}{l}
|\input{childdoc.def}|\\
|\childdocby{|\textit{main}|}|\\
\end{tabular}
\end{center}
%
The directive |\childdocby| is similar to |\childdocof|
described in \secref{sec:include},
but the subsequent selection of content must be done manually.
To that end, both |\ifchilddoc| and |\ifchilddocmanual|
will be true upon processing of a part,
and the name of the part is stored in |\childdocname|.
Note that |\jobname| will be set to the filename of the current part
so that each part receives an individual |.aux| file
that does not interfere with the |.aux| file(s) of the main document.
This behaviour can be altered by the alternative form
|\childdocby[*]{|\textit{main}|}| (with a non-empty optional argument)
which uses the |.aux| file of the main document
by setting |\jobname| to \textit{main}.

%%%%%%%%%%%%%%%%%%%%%%%%%%%%%%%%%%%%%%%%%%%%%%%%%%%%%%%%%%%%%%%%%%%%%%%%%%%%%%%%
\subsection{Driver Development}
\label{sec:driver}

The \textsf{childdoc} mechanism can also be use for the development
of definition files such as \LaTeX{} styles or classes.
This case differs from the above setup with multiple parts
included by |\include| in that no |\includeonly| should be invoked.
This can be achieved by starting the include file
(before |\ProvidesPackage|) with:
%
\begin{center}
\begin{tabular}{l}
|\input{childdoc.def}|\\
|\childdocforward{|\textit{main}|}|\\
\end{tabular}
\end{center}
%
or alternatively with:
%
\begin{center}
\begin{tabular}{l}
|\input{childdoc.def}|\\
|\childdocby{|\textit{main}|}|\\
\end{tabular}
\end{center}
%
Both forms have slightly different effects as described above.
The main file is prepared as usual, see \secref{sec:include}.

%%%%%%%%%%%%%%%%%%%%%%%%%%%%%%%%%%%%%%%%%%%%%%%%%%%%%%%%%%%%%%%%%%%%%%%%%%%%%%%%
\subsection{Legacy Detection}
\label{sec:detection}

The directive |\childdocmain| in the main file can detect
whether the complete document or merely a child is to be compiled
even without using the directive |\childdocof|.
This method is deprecated because it is less robust
and there is no compelling reason to use it;
it is merely provided for backward compatibility
and it may be removed in future versions.

If the detection mechanism is to be used,
it is mandatory to correctly specify
the filename of the main file as the argument of |\childdocmain|:
%
\begin{center}
\begin{tabular}{l}
|\input{childdoc.def}|\\
|\childdocmain{|\textit{main}|}|\\
\end{tabular}
\end{center}
%
If |\jobname| does not match the argument \textit{main} of |\childdocmain|,
it is assumed that |\jobname| points to the child file to be compiled.
When using |\childdocmain| with the main file specified as argument,
it suffices to start a child file
with just |\input{|\textit{main}|}|
without loading of the package and using |\childdocof|.
If instead all processing is done
with the appropriate \textsf{childdoc} directives,
the argument of \textit{main} of |\childdocmain| can be empty.

An alternative version of the command line processing described
in \secref{sec:commandline} using the detection mechanism reads:
%
\begin{center}
|... -jobname "|\textit{target}|" "|[\textit{flags}]%
[|\def\jobname{|\textit{dest}|}|]|\input{|\textit{main}|}"|
\end{center}

%%%%%%%%%%%%%%%%%%%%%%%%%%%%%%%%%%%%%%%%%%%%%%%%%%%%%%%%%%%%%%%%%%%%%%%%%%%%%%%%
\subsection{Manual Code}
\label{sec:manual}

In case one cannot be certain whether the definitions file |childdoc.def|
is installed on the target \TeX{} distribution
and one prefers not to ship it,
it is conceivable to paste a few relevant commands into the sources.

To that end, drop all statements |\input{childdoc.def}|
and perform the replacements as outlined below.
Instead of |\childdocmain{|\textit{main}|}| add the following code
to the top of the main file:
%
\begin{center}
\begin{tabular}{l}
|\||ifdefined\childdocname\endinput\||fi\newif\ifchilddoc|\\
|\edef\childdocname{\scantokens\expandafter{\jobname\noexpand}}|\\
|\def\childdocmain{|\textit{main}|}\||ifx\childdocmain\childdocname\||else|\\
|\childdoctrue\includeonly{\childdocname}\let\jobname\childdocmain\||fi|\\
\end{tabular}
\end{center}
%
Instead of |\childdocof{|\textit{main}|}| just include the main file
at the top of each child file:
%
\begin{center}
|\input{|\textit{main}|}|
\end{center}
%
A simple redirection |\childdocforward{|\textit{dest}|}| is achieved by:
%
\begin{center}
|\def\jobname{|\textit{dest}|}\input{\jobname}|
\end{center}
%
The redirection with prefix
|\childdocforwardprefix[|\textit{prefix}|]{|\textit{dest}|}|
is accomplished by:
%
\begin{center}
\begin{tabular}{l}
|{\edef\jobname{\scantokens\expandafter{\jobname\noexpand}}|\\
|\def\redirectjob |\textit{prefix}|#1~~~{\gdef\jobname{|\textit{dest}|#1}}|\\
|\expandafter\redirectjob\jobname~~~}\input{\jobname}|
\end{tabular}
\end{center}

In an alternative approach,
child documents can be compiled by a specific command line
without additional code or specific definitions:
%
\begin{center}
|... -jobname "|\textit{target}|" "|[\textit{flags}]%
|\includeonly{|\textit{dest}|}\input{|\textit{main}|}"|
\end{center}
%

%%%%%%%%%%%%%%%%%%%%%%%%%%%%%%%%%%%%%%%%%%%%%%%%%%%%%%%%%%%%%%%%%%%%%%%%%%%%%%%%
%%%%%%%%%%%%%%%%%%%%%%%%%%%%%%%%%%%%%%%%%%%%%%%%%%%%%%%%%%%%%%%%%%%%%%%%%%%%%%%%
\section{Information}

%%%%%%%%%%%%%%%%%%%%%%%%%%%%%%%%%%%%%%%%%%%%%%%%%%%%%%%%%%%%%%%%%%%%%%%%%%%%%%%%
\subsection{Copyright}

Copyright \copyright{} 2017--2018 Niklas Beisert

This work may be distributed and/or modified under the
conditions of the \LaTeX{} Project Public License, either version 1.3
of this license or (at your option) any later version.
The latest version of this license is in
  \url{http://www.latex-project.org/lppl.txt}
and version 1.3 or later is part of all distributions of \LaTeX{}
version 2005/12/01 or later.

This work has the LPPL maintenance status `maintained'.

The Current Maintainer of this work is Niklas Beisert.

This work consists of the files |README.txt|, |childdoc.ins| and |childdoc.dtx|
as well as the derived files |childdoc.def|, |cdocsamp.tex|
with |cdocsch1.tex|, |cdocsch2.tex|, |cdocspt3.tex|, |cdocspt4.tex|,
|cdocsdrf.tex|, |cdocsfn1.tex|, |cdocsfn2.tex|
as well as |childdoc.pdf|.

%%%%%%%%%%%%%%%%%%%%%%%%%%%%%%%%%%%%%%%%%%%%%%%%%%%%%%%%%%%%%%%%%%%%%%%%%%%%%%%%
\subsection{Files and Installation}

The package consists of the files:
%
\begin{center}
\begin{tabular}{ll}
    |README.txt|   & readme file \\
    |childdoc.ins| & installation file \\
    |childdoc.dtx| & source file \\
    |childdoc.def| & definition file \\
    |cdocsamp.tex| & sample main file \\
    |cdocsch1.tex| & sample include file \\
    |cdocsch2.tex| & sample include file \\
    |cdocspt3.tex| & sample part file \\
    |cdocspt4.tex| & sample part file \\
    |cdocsdrf.tex| & sample redirection file \\
    |cdocsfn1.tex| & sample redirection file \\
    |cdocsfn2.tex| & sample redirection file \\
    |childdoc.pdf| & manual
\end{tabular}
\end{center}
%
The distribution consists of the files
|README.txt|, |childdoc.ins| and |childdoc.dtx|.
%
\begin{itemize}
\item
Run (pdf)\LaTeX{} on |childdoc.dtx|
to compile the manual |childdoc.pdf| (this file).
\item
Run \LaTeX{} on |childdoc.ins| to create the definitions file |childdoc.def|
and the sample |cdocsamp.tex| with include files
|cdocsch1.tex|, |cdocsch2.tex|, |cdocspt3.tex|, |cdocspt4.tex|,
|cdocsdrf.tex|, |cdocsfn1.tex|, |cdocsfn2.tex|.
Then copy the file |childdoc.def| to an appropriate directory of your \LaTeX{}
distribution, e.g.\ \textit{texmf-root}|/tex/latex/childdoc|.
\end{itemize}

%%%%%%%%%%%%%%%%%%%%%%%%%%%%%%%%%%%%%%%%%%%%%%%%%%%%%%%%%%%%%%%%%%%%%%%%%%%%%%%%
\subsection{Related CTAN Packages}

There are several other packages which offer a similar functionality:
%
\begin{itemize}
\item
The packages
\href{http://ctan.org/pkg/docmute}{\textsf{docmute}},
\href{http://ctan.org/pkg/includex}{\textsf{includex}} and
\href{http://ctan.org/pkg/standalone}{\textsf{standalone}}
provide commands to include only the document body of
a child file thus allowing both files to be compiled individually.
\item
The packages \href{http://ctan.org/pkg/subdocs}{\textsf{subdocs}}
and \href{http://ctan.org/pkg/subfiles}{\textsf{subfiles}}
provide structures in which the main and child documents can be
encapsulated and allowing them to be compiled individually.
The inclusion mechanism is different from the conventional |\include|.
\item
The package \href{http://ctan.org/pkg/combine}{\textsf{combine}}
is an elaborate solution to combine several documents into one.
\end{itemize}
%
See also the CTAN topic \href{http://ctan.org/topic/subdocs}{\textsf{subdocs}}
for further related packages.
The present package differs from the above solutions in that
a document structure constructed with the conventional |\include| mechanism
just needs two extra commands at the top of every file
such that all constituent files can be compiled individually.

%%%%%%%%%%%%%%%%%%%%%%%%%%%%%%%%%%%%%%%%%%%%%%%%%%%%%%%%%%%%%%%%%%%%%%%%%%%%%%%%
%\subsection{Feature Suggestions}
%
%The following is a list of features which may be useful for future
%versions of this package:
%%
%\begin{itemize}
%\item
%\ldots
%\end{itemize}

%%%%%%%%%%%%%%%%%%%%%%%%%%%%%%%%%%%%%%%%%%%%%%%%%%%%%%%%%%%%%%%%%%%%%%%%%%%%%%%%
\subsection{Revision History}

%%%%%%%%%%%%%%%%%%%%%%%%%%%%%%%%%%%%%%%%
\paragraph{v2.0:} 2018/12/30

\begin{itemize}
\item
immediate forward processing
\item
added |\childdocby| mechanism
\item
manual restructured
\end{itemize}

%%%%%%%%%%%%%%%%%%%%%%%%%%%%%%%%%%%%%%%%
\paragraph{v1.6:} 2018/01/17

\begin{itemize}
\item
application for development of include files
\item
corrections to manual
\end{itemize}

%%%%%%%%%%%%%%%%%%%%%%%%%%%%%%%%%%%%%%%%
\paragraph{v1.5:} 2017/05/21

\begin{itemize}
\item
more complete structuring introduced
\item
|\childdocof| introduced
\item
|\childdoc| renamed to |\childdocmain|
\item
|\childredirect| renamed to |\childdocforward| and |\childdocforwardprefix|
and functionality expanded
\end{itemize}

%%%%%%%%%%%%%%%%%%%%%%%%%%%%%%%%%%%%%%%%
\paragraph{v1.0:} 2017/04/27

\begin{itemize}
\item
manual and install package
\item
first version published on CTAN
\end{itemize}

%%%%%%%%%%%%%%%%%%%%%%%%%%%%%%%%%%%%%%%%
\paragraph{v0.6:} 2017/04/26

\begin{itemize}
\item
redirection mechanism added
\end{itemize}

%%%%%%%%%%%%%%%%%%%%%%%%%%%%%%%%%%%%%%%%
\paragraph{v0.5:} 2017/04/26

\begin{itemize}
\item
functionality in definition file
\end{itemize}


%%%%%%%%%%%%%%%%%%%%%%%%%%%%%%%%%%%%%%%%%%%%%%%%%%%%%%%%%%%%%%%%%%%%%%%%%%%%%%%%
%%%%%%%%%%%%%%%%%%%%%%%%%%%%%%%%%%%%%%%%%%%%%%%%%%%%%%%%%%%%%%%%%%%%%%%%%%%%%%%%
%%%%%%%%%%%%%%%%%%%%%%%%%%%%%%%%%%%%%%%%%%%%%%%%%%%%%%%%%%%%%%%%%%%%%%%%%%%%%%%%
\appendix

\settowidth\MacroIndent{\rmfamily\scriptsize 000\ }

 \DocInput{childdoc.dtx}

\end{document}
%</driver>
% \fi
%
% %%%%%%%%%%%%%%%%%%%%%%%%%%%%%%%%%%%%%%%%%%%%%%%%%%%%%%%%%%%%%%%%%%%%%%%%%%%%%%
% %%%%%%%%%%%%%%%%%%%%%%%%%%%%%%%%%%%%%%%%%%%%%%%%%%%%%%%%%%%%%%%%%%%%%%%%%%%%%%
% \section{Sample}
%\iffalse
%<*samplemain>
%\fi
%
% The following presents a sample document
% with two chapters, two parts, a title page,
% a compile flag as well as three forwarding files to set the flag.
% It consists of eight |.tex| files:
% \begin{center}
% \begin{tabular}{ll}
% |cdocsamp.tex|&main file\\
% |cdocsch1.tex|&include file for chapter 1\\
% |cdocsch2.tex|&include file for chapter 2\\
% |cdocspt3.tex|&include file for part 3\\
% |cdocspt4.tex|&include file for part 4\\
% |cdocsdrf.tex|&forwarding file for main file in draft mode\\
% |cdocsfi1.tex|&forwarding file for final version of chapter 1\\
% |cdocsfi2.tex|&forwarding file for final version of chapter 2\\
% \end{tabular}
% \end{center}
% Each of the eight files can be compiled directly by the \LaTeX{} compiler.
%
% %%%%%%%%%%%%%%%%%%%%%%%%%%%%%%%%%%%%%%
% \paragraph{Main File.}
%
% The main file is called |cdocsamp.tex|.
%
% Load the \textsf{childdoc} definitions and
% declare the filename for the main document:
%    \begin{macrocode}
\input{childdoc.def}
\childdocmain{}
%    \end{macrocode}

% Optional override for |\version| flag:
%    \begin{macrocode}
%%\ifchilddoc\else\providecommand{\version}{draft}\fi
%    \end{macrocode}

% Define the default values for the |\version| flag
% (|final| for the main file and |draft| for childs):
%    \begin{macrocode}
\ifchilddoc
\providecommand{\version}{draft}
\else
\providecommand{\version}{final}
\fi
%    \end{macrocode}

% Load the standard document class:
%    \begin{macrocode}
\documentclass[12pt]{article}
%    \end{macrocode}

% Start the document body:
%    \begin{macrocode}
\begin{document}
%    \end{macrocode}

% Declare a title page.
% Print title, part of document being processed and version flag:
%    \begin{macrocode}
\addtocounter{page}{-1}
\begin{center}
{\LARGE\bfseries{}childdoc example\par}
\vspace{1cm}
\ifchilddoc
\ifchilddocmanual part\else chapter\fi:
`\childdocname' of `\childdocjob'\par
\else
main document: `\childdocjob'\par
\fi
version: \version\par
\end{center}
\newpage
%    \end{macrocode}

% Manually include selected file,
% otherwise process as usual:
%    \begin{macrocode}
\ifchilddocmanual
\section*{part `\childdocname'}
\input{\childdocname}
\else
%    \end{macrocode}

% Include the two chapters:
%    \begin{macrocode}
\include{cdocsch1}
\include{cdocsch2}
%    \end{macrocode}

% Include the two parts unless only chapters should be displayed:
%    \begin{macrocode}
\ifchilddoc\else
\section{part three}
\input{cdocspt3}
\section{part four}
\input{cdocspt4}
\fi
%    \end{macrocode}

% Process as usual until here:
%    \begin{macrocode}
\fi
%    \end{macrocode}

% End of document body:
%    \begin{macrocode}
\end{document}
%    \end{macrocode}
%\iffalse
%</samplemain>
%\fi
%
% %%%%%%%%%%%%%%%%%%%%%%%%%%%%%%%%%%%%%%
% \paragraph{Chapter Include Files.}
%
% The include files are called |cdocsch1.tex| and |cdocsch2.tex|.
%
%\iffalse
%<*samplechap1|samplechap2>
%\fi

% Optional override for |\version| flag:
%    \begin{macrocode}
%%\providecommand{\version}{final}
%    \end{macrocode}

% Include the main document:
%    \begin{macrocode}
\input{childdoc.def}
\childdocof{cdocsamp}
%    \end{macrocode}

%\iffalse
%</samplechap1|samplechap2>
%\fi
%
%\iffalse
%<*samplechap1>
%\fi
% Some text for chapter 1:
%    \begin{macrocode}
\section{one}
some text in chapter one
%    \end{macrocode}

%\iffalse
%</samplechap1>
%\fi
% Some text for chapter 2:
%\iffalse
%<*samplechap2>
%\fi
%    \begin{macrocode}
\section{two}
more text in chapter two
%    \end{macrocode}

%\iffalse
%</samplechap2>
%\fi
%
% %%%%%%%%%%%%%%%%%%%%%%%%%%%%%%%%%%%%%%
% \paragraph{Part Include Files.}
%
% The include files are called |cdocspt3.tex| and |cdocspt4.tex|.
%
%\iffalse
%<*samplepart3|samplepart4>
%\fi

% Optional override for |\version| flag:
%    \begin{macrocode}
%%\providecommand{\version}{final}
%    \end{macrocode}

% Include the main document:
%    \begin{macrocode}
\input{childdoc.def}
\childdocby{cdocsamp}
%    \end{macrocode}

%\iffalse
%</samplepart3|samplepart4>
%\fi
%
%\iffalse
%<*samplepart3>
%\fi
% Some text for part 3:
%    \begin{macrocode}
some text in part three
%    \end{macrocode}

%\iffalse
%</samplepart3>
%\fi
% Some text for part 4:
%\iffalse
%<*samplepart4>
%\fi
%    \begin{macrocode}
more text in part four
%    \end{macrocode}

%\iffalse
%</samplepart4>
%\fi
%
% %%%%%%%%%%%%%%%%%%%%%%%%%%%%%%%%%%%%%%
% \paragraph{Forwarding for a Complete Draft.}
%
% The following forwarding file |cdocsdrf.tex|
% compiles the main document in draft mode:
%\iffalse
%<*sampledraft>
%\fi
%    \begin{macrocode}
\def\version{draft}
\input{childdoc.def}
\childdocforward{cdocsamp}
%    \end{macrocode}

%\iffalse
%</sampledraft>
%\fi
%
% %%%%%%%%%%%%%%%%%%%%%%%%%%%%%%%%%%%%%%
% \paragraph{Forwarding for Final Version of the Chapters.}
%
% The following forwarding files |cdocsfn1.tex| and |cdocsfn2.tex|
% (with identical content)
% compile the final versions of the child documents
% |cdocsch1.tex| and |cdocsch2.tex|, respectively:
%\iffalse
%<*samplefinal>
%\fi
%    \begin{macrocode}
\def\version{final}
\input{childdoc.def}
\childdocforwardprefix[cdocsamp]{cdocsfn}{cdocsch}
%    \end{macrocode}

%\iffalse
%</samplefinal>
%\fi
%
% %%%%%%%%%%%%%%%%%%%%%%%%%%%%%%%%%%%%%%
% \paragraph{Command Line Processing.}
%
% The following three command lines generate the output files
% |cdocscld|, |cdocscl1| and |cdocscl2|
% which should be identical to
% |cdocsdrf|, |cdocsch1| and |cdocsfn2|, respectively:
% \begin{center}
% \begin{tabular}{l}
% |latex -jobname cdocscld \|\\
% |  "\def\version{draft}\input{childdoc.def}\childdocforward{cdocsamp}"|\\
% |latex -jobname cdocscl1 \|\\
% |  "\input{childdoc.def}\childdocforward[cdocsamp]{cdocsch1}"|\\
% |latex -jobname cdocscl2 \|\\
% |  "\def\version{final}\input{childdoc.def}\childdocforward{cdocsch2}"|
% \end{tabular}
% \end{center}
% Note that the trailing backslash on each first line
% merely continues the input to the second line
% (for convenient cut ant paste).
% Furthermore, the command |latex| can be replaced by any
% of its alternative versions such as |pdflatex|.
%
% %%%%%%%%%%%%%%%%%%%%%%%%%%%%%%%%%%%%%%%%%%%%%%%%%%%%%%%%%%%%%%%%%%%%%%%%%%%%%%
% %%%%%%%%%%%%%%%%%%%%%%%%%%%%%%%%%%%%%%%%%%%%%%%%%%%%%%%%%%%%%%%%%%%%%%%%%%%%%%
% \section{Implementation}
%\iffalse
%<*package>
%\fi
%
% This section describes the definitions file |childdoc.def|.

% The definitions cannot be loaded using |\usepackage| or |\RequirePackage|
% which has a mechanism to prevent loading a style file more than once.
% When loading the definitions by means of |\input|
% multiple instances have to be prevented manually:
%\iffalse
%This code needs to be before the `\ProvidesFile' directive
%which is defined at the beginning of this file.
%Therefore it is also placed there and commented out here.
%</package>
%<*discard>
%\fi
%    \begin{macrocode}
\ifdefined\childdocmain\endinput\fi
%    \end{macrocode}
%\iffalse
%</discard>
%<*package>
%\fi
%
% \macro{\ifchilddoc}
% \macro{\ifchilddocmanual}
% The conditional |\ifchilddoc| tells whether a
% child (true) or main (false) document is being compiled.
% The conditional |\ifchilddocmanual| tells whether
% the |\includeonly| mechanism is used (false) or
% the selection of child files must be performed manually (true).
% The definitions initialise to false:
%    \begin{macrocode}
\newif\ifchilddoc
\newif\ifchilddocmanual
%    \end{macrocode}

% \macro{\childdocname}
% \macro{\childdocjob}
% The macro |\childdocname| stores the name of the main document
% to be compiled. The macro |\childdocjob| stores the name of
% the document on which the \LaTeX{} compiler was originally invoked.
% The content of |\jobname| cannot be compared
% to filenames specified in the source due to different catcodes.
% The following code rescans |\jobname|, stores the result
% in |\childdocname| and saves a copy in |\childdocjob|:
%    \begin{macrocode}
\edef\childdocname{\scantokens\expandafter{\jobname\noexpand}}
\let\childdocjob\childdocname
%    \end{macrocode}

% \macro{\childdocdisable}
% The macro |\childdocdisable| prevents the main file
% from being processed more than once.
% At this stage, the main document command |\childdocmain|
% is assumed to be called once again where it should do nothing.
% Any subsequent call to it should prevent
% a secondary processing of the main document
% It overwrites the forwarding commands
% |\childdocof| and |\childdocforward|
% with empty macros to prevent further inclusions of the main document:
%    \begin{macrocode}
\newcommand{\childdocdisable}
{
  \renewcommand{\childdocmain}[1]{\renewcommand{\childdocmain}[1]{\endinput}}
  \renewcommand{\childdocof}[1]{}
  \renewcommand{\childdocby}[2][]{}
  \renewcommand{\childdocforward}[2][]{}
  \renewcommand{\childdocdisable}{}
}
%    \end{macrocode}

% \macro{\childdocmain}
% The macro |\childdocmain| is to be called at the top of the main file
% with nothing or the main filename (without extension) as argument.
% First, it breaks loops.
% If the argument is not empty and does not match |\childdocname|
% (which is set by the first inclusion of |childdoc.def|),
% |\ifchilddoc| is set to true, |\includeonly| is applied to the child file
% and |\jobname| is set to the main file
% (for proper handling of |.aux| files):
%    \begin{macrocode}
\newcommand{\childdocmain}[1]
{
  \childdocdisable\childdocmain{}
  \if?#1?\else
    \begingroup
      \def\childdoctmp{#1}
      \ifx\childdoctmp\childdocname
        \def\childdoctmp{}
      \else
        \def\childdoctmp
        {
          \childdoctrue
          \includeonly{\childdocname}
          \def\childdocjob{#1}
          \def\jobname{#1}
        }
      \fi
      \expandafter
    \endgroup
    \childdoctmp
  \fi
}
%    \end{macrocode}

% \macro{\childdocof}
% The command |\childdocof| redirects
% compilation to the main file |#1|.
%    \begin{macrocode}
\newcommand{\childdocof}[1]
{
  \childdocdisable
  \childdoctrue
  \includeonly{\childdocname}
  \def\jobname{#1}
  \def\childdocjob{#1}
  \input{#1}
}
%    \end{macrocode}

% \macro{\childdocby}
% The command |\childdocby| ....
%    \begin{macrocode}
\newcommand{\childdocby}[2][]
{
  \childdocdisable
  \childdoctrue
  \childdocmanualtrue
  \if?#1?\else
    \def\jobname{#2}
  \fi
  \def\childdocjob{#2}
  \input{#2}
  \endinput
}
%    \end{macrocode}

% \macro{\childdocforward}
% The command |\childdocforward| redirects
% compilation to the main file or
% (if the optional argument is given) a child file.
% Parameters are set as if the main file
% or a child file starting with |\childdocof| was compiled.
% Then compilation is handed over to the main file:
%    \begin{macrocode}
\newcommand{\childdocforward}[2][]
{
  \begingroup
    \if?#1?
      \def\childdoctmp
      {
        \def\childdocname{#2}
        \def\childdocjob{#2}
        \def\jobname{#2}
        \input{#2}
        \endinput
      }
    \else
      \def\childdoctmp
      {
        \childdocdisable
        \def\childdocname{#2}
        \childdoctrue
        \includeonly{#2}
        \def\childdocjob{#1}
        \def\jobname{#1}
        \input{#1}
        \endinput
      }
    \fi
    \expandafter
  \endgroup
  \childdoctmp
}
%    \end{macrocode}

% \macro{\childdocforwardprefix}
% The command |\childdocforwardprefix| redirects
% compilation to the main or a child file by means of a pattern.
% The prefix |#1| in the current filename is replaced by |#2|
% and the suffix of the current filename is kept
% (it is assumed that the filename does not contain the substring `|~~~|'
% which is used as a delimiter).
% Compilation is handed over to the new file by |\childdocforward|:
%    \begin{macrocode}
\newcommand{\childdocforwardprefix}[3][]
{
  \begingroup
    \def\childdocextract #2##1~~~{\def\childdoctmp{\childdocforward[#1]{#3##1}}}
    \expandafter\childdocextract\childdocname~~~
    \expandafter
  \endgroup
  \childdoctmp
}
%    \end{macrocode}

% \macro{\childdoc}
% The deprecated macro |\childdoc| is a legacy version of |\childdocmain|:
%    \begin{macrocode}
\newcommand{\childdoc}{\childdocmain}
%    \end{macrocode}

% \macro{\childdocredirect}
% The deprecated macro |\childdocredirect| is a legacy version
% of |\childdocforward| and |\childdocforwardprefix|:
%    \begin{macrocode}
\newcommand{\childdocredirect}[2][]
{
  \begingroup
    \if?#1?
      \def\childdoctmp{\childdocforward{#2}}
    \else
      \def\childdoctmp{\childdocforwardprefix{#1}{#2}}
    \fi
    \expandafter
  \endgroup
  \childdoctmp
}
%    \end{macrocode}

%\iffalse
%</package>
%\fi
%
\endinput
\childdocforward[|\textit{main}|]{|\textit{dest}|}"|
\end{center}
%
Here \textit{target} is the name of the output file,
\textit{main} is the name of the main file
and \textit{dest} is the name of the main or child file to be processed
(all filenames without extensions).
The optional argument \textit{main} can be omitted
if \textit{main} matches \textit{dest}.
Optionally, compilation \textit{flags} can be defined via |\def| commands.
This command line makes the \TeX{} engine believe
it is compiling the file \textit{target}
whose content is specified as the latter parameter.
The provided code then forwards the processing to
\textit{main} or \textit{dest} as described in \secref{sec:forward}.

%%%%%%%%%%%%%%%%%%%%%%%%%%%%%%%%%%%%%%%%%%%%%%%%%%%%%%%%%%%%%%%%%%%%%%%%%%%%%%%%
\subsection{Include by Input}
\label{sec:input}

Including child documents by |\include| has some restrictions by design.
Most notably, the content of a child document always occupies
its own set of pages; pages cannot be shared between child documents.
Usually, this behaviour makes perfect sense
because each child document contain an essential part of the document.
However, in some situations it may be desirable to compose
a document from a collection of parts
without having mandatory page breaks between then.
For this case, the package
provides a mechanism to include parts
by |\input| which can also be processed individually.
However, by construction this mechanism
requires manual handling of the content to be output.

%%%%%%%%%%%%%%%%%%%%%%%%%%%%%%%%%%%%%%%%
\DescribeMacro{\ifchilddocmanual}
The main file should be prepared as usual, see \secref{sec:include}.
However, the document body must make a distinction
between processing of an individual part and of the main document, e.g.:
%
\begin{center}
\begin{tabular}{l}
|\ifchilddocmanual|\\
|\input{\childdocname}|\\
|\||else|\\
\textit{document body with }|\input{|\textit{part}|}|\\
|\||fi|
\end{tabular}
\end{center}
%
The conditional |\ifchilddocmanual| is true whenever
a part to be included by |\input| is being compiled,
and the name of the part is stored in |\childdocname|.

%%%%%%%%%%%%%%%%%%%%%%%%%%%%%%%%%%%%%%%%
\DescribeMacro{\childdocby}
Each part to be included by |\input| should start with:
%
\begin{center}
\begin{tabular}{l}
|% \iffalse
%
% childdoc.dtx Copyright (C) 2017-2018 Niklas Beisert
%
% This work may be distributed and/or modified under the
% conditions of the LaTeX Project Public License, either version 1.3
% of this license or (at your option) any later version.
% The latest version of this license is in
%   http://www.latex-project.org/lppl.txt
% and version 1.3 or later is part of all distributions of LaTeX
% version 2005/12/01 or later.
%
% This work has the LPPL maintenance status `maintained'.
%
% The Current Maintainer of this work is Niklas Beisert.
%
% This work consists of the files childdoc.dtx and childdoc.ins
% and the derived files childdoc.def and cdocsamp.tex with
% cdocsch1.tex, cdocsch2.tex, cdocsdrf.tex, cdocsfn1.tex, cdocsfn2.tex.
%
%<package>\ifdefined\childdocmain\endinput\fi
%<package>\ProvidesFile{childdoc.def}[2018/12/30 v2.0 child document driver]
%<samplemain>\ProvidesFile{cdocsamp.tex}[2018/12/30 v2.0 sample for childdoc]
%<*driver>
%\ProvidesFile{childdoc.drv}[2018/12/30 v2.0 childdoc reference manual file]
\PassOptionsToClass{10pt,a4paper}{article}
\documentclass{ltxdoc}

\usepackage[margin=35mm]{geometry}
\usepackage{hyperref}
\usepackage{hyperxmp}
\usepackage[usenames]{color}

\hypersetup{colorlinks=true}
\hypersetup{pdfstartview=FitH}
\hypersetup{pdfpagemode=UseNone}
\hypersetup{pdfsource={}}
\hypersetup{pdflang={en-UK}}
\hypersetup{pdfcopyright={Copyright 2017-2018 Niklas Beisert.
  This work may be distributed and/or modified under the
  conditions of the LaTeX Project Public License, either version 1.3
  of this license or (at your option) any later version.}}
\hypersetup{pdflicenseurl={http://www.latex-project.org/lppl.txt}}
\hypersetup{pdfcontactaddress={ETH Zurich, ITP, HIT K,
  Wolfgang-Pauli-Strasse 27}}
\hypersetup{pdfcontactpostcode={8093}}
\hypersetup{pdfcontactcity={Zurich}}
\hypersetup{pdfcontactcountry={Switzerland}}
\hypersetup{pdfcontactemail={nbeisert@itp.phys.ethz.ch}}
\hypersetup{pdfcontacturl={http://people.phys.ethz.ch/\xmptilde nbeisert/}}

\newcommand{\secref}[1]{\hyperref[#1]{section \ref*{#1}}}

\parskip1ex
\parindent0pt
\let\olditemize\itemize
\def\itemize{\olditemize\parskip0pt}

\begin{document}

\title{The \textsf{childdoc} Package}
\hypersetup{pdftitle={The childdoc Package}}
\author{Niklas Beisert\\[2ex]
  Institut f\"ur Theoretische Physik\\
  Eidgen\"ossische Technische Hochschule Z\"urich\\
  Wolfgang-Pauli-Strasse 27, 8093 Z\"urich, Switzerland\\[1ex]
  \href{mailto:nbeisert@itp.phys.ethz.ch}
  {\texttt{nbeisert@itp.phys.ethz.ch}}}
\hypersetup{pdfauthor={Niklas Beisert}}
\hypersetup{pdfsubject={Manual for the LaTeX2e Package childdoc}}
\date{30 December 2018, \textsf{v2.0}}
\maketitle

\begin{abstract}\noindent
\textsf{childdoc} is a \LaTeXe{} package
that enables the direct compilation
of document sections included by |\include|
to individual files.
\end{abstract}

\begingroup
\parskip0ex
\tableofcontents
\endgroup

%%%%%%%%%%%%%%%%%%%%%%%%%%%%%%%%%%%%%%%%%%%%%%%%%%%%%%%%%%%%%%%%%%%%%%%%%%%%%%%%
%%%%%%%%%%%%%%%%%%%%%%%%%%%%%%%%%%%%%%%%%%%%%%%%%%%%%%%%%%%%%%%%%%%%%%%%%%%%%%%%
\section{Introduction}

\LaTeX{} provides a mechanism to structure a large document (such as a book)
into a main file and several child files (containing the chapters)
using the |\include| command.
This mechanism is beneficial for documents
which span hundreds of pages in order to
make the source file(s) more manageable.
Moreover, compilation can be restricted to
selected child files by means of the |\includeonly| command.
The latter feature can be used to reduce the compilation time while editing
(this was significantly more useful in the earlier days of \LaTeX{})
or to generate a smaller document which is easier to navigate.
Another application of |\includeonly| is to generate
documents consisting of selected parts of the complete document.

However, there are a few drawbacks of the plain |\include| mechanism:
\begin{itemize}
\item
The child files cannot be compiled on their own,
they can only be compiled via the main file.
A naive editing environment
(such as a text editor with an option
to have the current file processed by \LaTeX)
may require one to switch to the main file before compiling;
attempting to compile the child file produces errors.
\item
The main file must be modified (each time)
to adjust the |\includeonly| command
to the present needs. This easily leaves the main file in a messy state.
\item
The generated document will always carry the filename
of the main document. This is inconvenient if
several child files are to be compiled and
to be kept for distribution.
\end{itemize}

The present package provides a simple interface
to make child files individually compilable by \LaTeX{}.
Compiling a child file then has the same effect as compiling
the main file with an |\includeonly| command
to select the appropriate child.
Moreover the generated document will carry the name of the child
rather than the main file.
This resolves all three above issues.

This feature is meant to make the editing of books,
thesis documents and lecture notes somewhat more convenient.
However, the package can also be used efficiently for
composing a series of documents (such as exercise sheets)
which are typically distributed individually.
It then assists the author in generating the individual documents
(potentially in different versions)
as well as a document containing the collected series.
Another application is in developing style files
or other kinds of included material
where compilation of the style file could redirect
to a sample or test file.

%%%%%%%%%%%%%%%%%%%%%%%%%%%%%%%%%%%%%%%%%%%%%%%%%%%%%%%%%%%%%%%%%%%%%%%%%%%%%%%%
%%%%%%%%%%%%%%%%%%%%%%%%%%%%%%%%%%%%%%%%%%%%%%%%%%%%%%%%%%%%%%%%%%%%%%%%%%%%%%%%
\section{Usage}

First of all, the package \textsf{childdoc} is \emph{not} a standard
\LaTeXe{} |.sty| style file! Therefore it needs to be invoked in
a non-standard way.

%%%%%%%%%%%%%%%%%%%%%%%%%%%%%%%%%%%%%%%%%%%%%%%%%%%%%%%%%%%%%%%%%%%%%%%%%%%%%%%%
\subsection{Included Files}
\label{sec:include}

%%%%%%%%%%%%%%%%%%%%%%%%%%%%%%%%%%%%%%%%
\DescribeMacro{\childdocmain}
To use the package, add the commands
\begin{center}
\begin{tabular}{l}
|\input{childdoc.def}|\\
|\childdocmain{}|\\
\end{tabular}
\end{center}
at the very top of the main \LaTeX{} file,
in particular \emph{before} the |\documentclass| statement!
The argument of |\childdocmain| should be left empty
(but it must be present).

%%%%%%%%%%%%%%%%%%%%%%%%%%%%%%%%%%%%%%%%
\DescribeMacro{\childdocof}
Furthermore, add the commands
\begin{center}
\begin{tabular}{l}
|\input{childdoc.def}|\\
|\childdocof{|\textit{main}|}|\\
\end{tabular}
\end{center}
at the top of every child file \textit{child}
which is included by |\include{|\textit{child}|}|
from within the main file
(or at least for those files to be compiled individually).
The argument \textit{main} must be the filename of the main file.

There are a couple of
considerations in setting up the main and child documents:

%%%%%%%%%%%%%%%%%%%%%%%%%%%%%%%%%%%%%%%%
\paragraph{Restrictions.}

Please note the following restrictions:
\begin{itemize}
\item
|\childdocmain| must be called with one argument \textit{main}
to ensure compatibility with earlier version of the package.
It must either be empty (|\childdocmain{}|)
or precisely match the filename of the main file in which it is specified.
See \secref{sec:detection} for further information.
\item
The filename \textit{main} must be specified without the |.tex| extension.
\item
The filename \textit{main} is case sensitive
(even in case-insensitive file systems)
due to internal string comparison.
\item
The argument \textit{main} should be fully expanded, it cannot be a macro.
\item
Subdirectories and special characters should be avoided in filenames.
\item
The command |\childdocmain{|\textit{main}|}| must be followed by a whitespace.
It should not be followed immediately by another command
or by a comment mark `|%|'.
This is because the \TeX{} parser reads the token immediately following
the argument of |\childdocmain| and puts it
at the beginning of every child section;
however, a white\-space is ignored.
\end{itemize}

%%%%%%%%%%%%%%%%%%%%%%%%%%%%%%%%%%%%%%%%
\paragraph{Content of Main File.}

It is advisable to place all content in the child files included by |\include|.
Any output contained in the main file will appear in all child documents
unless suppressed manually;
it cannot be suppressed automatically by the |\includeonly| directive
and thus should normally be avoided.
A method to include some content in the main file
by means of conditional processing is described in \secref{sec:conditional}.

%%%%%%%%%%%%%%%%%%%%%%%%%%%%%%%%%%%%%%%%
\paragraph{Page Numbering.}

When only a part of the document is compiled,
the appropriate numbering of pages
(as well as other status parameters)
is determined from the |.aux| files.
The latter contain information from previous passes.
However this information needs to propagate through
all intermediate child documents.
Therefore the page numbering in child documents may well
be inconsistent until the complete document is compiled at least once.

A useful (if unconventional) way to always ensure a consistent
page numbering is to restart the numbering in each child document
and denote the pages by `\textit{child}|.|\textit{page}'
where \textit{child} represents the chapter/section number of the child file.
This can be achieved by the command
|\numberwithin{page}{|\textit{child}|}|
of the \textsf{amsmath} package
where \textit{child} can be |chapter| or |section|
depending on the chosen structuring.
Alternatively, one can modify the macro |\thepage| appropriately
and reset the counter |page| at the start of each child file.

%%%%%%%%%%%%%%%%%%%%%%%%%%%%%%%%%%%%%%%%%%%%%%%%%%%%%%%%%%%%%%%%%%%%%%%%%%%%%%%%
\subsection{Conditional Processing}
\label{sec:conditional}

The package provides a mechanism to compile different versions
of a document. To customise the versions further some conditional processing
can come in handy to distinguish which version is being compiled.
The package provides two macros to describe the compilation context:

%%%%%%%%%%%%%%%%%%%%%%%%%%%%%%%%%%%%%%%%
\DescribeMacro{\ifchilddoc}
The conditional |\ifchilddoc| distinguishes between the compilation of
child documents and the main document:
%
\begin{center}
|\ifchilddoc |\textit{child-code}| |[|\||else |\textit{main-code}]| \||fi|
\end{center}

%%%%%%%%%%%%%%%%%%%%%%%%%%%%%%%%%%%%%%%%
\DescribeMacro{\childdocname}
\DescribeMacro{\childdocjob}
The macro |\childdocname| contains the filename (without extension)
of the main or child file being processed.
Note that |\childdocjob| will always contain the name of the main file.

%%%%%%%%%%%%%%%%%%%%%%%%%%%%%%%%%%%%%%%%
\paragraph{Title Page.}

Conditional processing can be used to include a title or banner page
in the main document when proper precautions are taken.
Importantly, the code in the main file should ensure that the page counter
(as well as other status parameters which are stored in the |.aux| files)
takes the same value after the conditional processing.
Otherwise the page numbers may take divergent values
depending on which part is compiled.

For example, a title page could be declared by:
%
\begin{center}
\begin{tabular}{l}
|\ifchilddoc\||else|\\
|\addtocounter{page}{-1}|\\
\textit{code for title page}\\
|\newpage|\\
|\||fi|
\end{tabular}
\end{center}
%
A banner page for the child documents can be generated by:
%
\begin{center}
\begin{tabular}{l}
|\ifchilddoc|\\
|\addtocounter{page}{-1}|\\
\textit{code for banner page}\\
|\newpage|\\
|\||fi|
\end{tabular}
\end{center}
%
Here one could write a message such as:
\begin{center}
|This is the part \childdocname{} of \childdocjob{}.|
\end{center}

%%%%%%%%%%%%%%%%%%%%%%%%%%%%%%%%%%%%%%%%%%%%%%%%%%%%%%%%%%%%%%%%%%%%%%%%%%%%%%%%
\subsection{Flags}
\label{sec:flags}

The package makes it easy to generate different versions
of the main or child documents.
To this end compilation flags can be defined
and assigned different default values.
They will be particularly useful in conjunction
with the forwarding mechanism described in \secref{sec:forward}.

For example, it may be useful to have a flag |\version|
which can be set to |draft| or |final|.
The document source will contain some conditional code
depending on the value of |\version|.
Suppose further, the flag should default to |final| for the main file
and to |draft| for child files
which is a natural assignment for editing the document.
This is achieved by placing the following code
in the preamble of the main document
(below the |\childdocmain| directive):
%
\begin{center}
\begin{tabular}{l}
|\ifchilddoc|\\
|\providecommand{\version}{draft}|\\
|\||else|\\
|\providecommand{\version}{final}|\\
|\||fi|
\end{tabular}
\end{center}
%
The definition by |\providecommand| makes sure
that previous definitions are not overwritten.
Further statements |\providecommand{\version}{...}|
can thus be added before the above code to override it.

For the main file, one might add a line
(between |\childdocmain| and the above block)
%
\begin{center}
|%\ifchilddoc\||else\providecommand{\version}{draft}\||fi|
\end{center}
%
which can be uncommented to produce a draft version.
Likewise one can add a line to the very top of a child file
(above the |\childdocof{|\textit{main}|}| directive)
%
\begin{center}
|%\providecommand{\version}{final}|
\end{center}
%
which can be uncommented to produce the final version of this child document.

%%%%%%%%%%%%%%%%%%%%%%%%%%%%%%%%%%%%%%%%%%%%%%%%%%%%%%%%%%%%%%%%%%%%%%%%%%%%%%%%
\subsection{Forwarding}
\label{sec:forward}

Different versions of the main or child documents
using compilation flags as described in \secref{sec:flags}
can be (permanently) stored in different files
for convenient compilation, viewing and distribution.
To this end, the package defines a command
to pass on compilation to a different file:

%%%%%%%%%%%%%%%%%%%%%%%%%%%%%%%%%%%%%%%%
\DescribeMacro{\childdocforward}
The command |\childdocforward| redirects processing to
another source file:
%
\begin{center}
\begin{tabular}{l}
|\input{childdoc.def}|\\
|\childdocforward[|\textit{main}|]{|\textit{dest}|}|\\
\end{tabular}
\end{center}
%
The argument \textit{dest} is the destination file
(without extension).
It should be the main file or one of the child files.
Note that further \textsf{childdoc} directives
such as |\childdocof| and |\childdocforward|
in the indicated file will be processed in this form.
The optional argument \textit{main}
passes on directly to the main file \textit{main}
while pretending to compile the child \textit{dest}.
This form behaves as if \textit{dest}
issues |\childdocof{|\textit{main}|}| right away,
and no further \textsf{childdoc} directives will be processed.

%%%%%%%%%%%%%%%%%%%%%%%%%%%%%%%%%%%%%%%%
\DescribeMacro{\...prefix}
In the alternative form |\childdocforwardprefix|,
%
\begin{center}
\begin{tabular}{l}
|\input{childdoc.def}|\\
|\childdocforwardprefix[|\textit{main}|]{|\textit{prefix}|}{|\textit{dest}|}|
\end{tabular}
\end{center}
%
the destination file is determined by a pattern
depending on the current file:
To make this work, the current file must be called
`{\textit{prefix}\hspace{0.2em}\textit{suffix}}'
with \textit{prefix} matching precisely the argument.
Processing is then passed on to the file
`{\textit{dest}\hspace{0.2em}\textit{suffix}}'.
Surely, the same effect is achieved by
directly specifying the
argument `{\textit{dest}\hspace{0.2em}\textit{suffix}}'
in the first form.
However, that requires to set up a different file
for each child. With the alternative form of the command
all these files can have exactly the same content
which simplifies setting them up and maintaining them.

For example, the following file |draft.tex|
with a compilation flag |\version| as described in \secref{sec:flags}
compiles the main document as a draft:
%
\begin{center}
\begin{tabular}{l}
|\def\version{draft}|\\
|\input{childdoc.def}|\\
|\childdocforward{|\textit{main}|}|
\end{tabular}
\end{center}
%
Likewise, the following files |final|\textit{nn}|.tex|
compile the final version of the child document
|child|\textit{nn}|.tex|:
%
\begin{center}
\begin{tabular}{l}
|\def\version{final}|\\
|\input{childdoc.def}|\\
|\childdocforwardprefix{final}{child}|
\end{tabular}
\end{center}
%

Note that when several versions of a main file and/or of each child file
are to be generated, it may be convenient to set up a |Makefile| or
shell script to automatise the process.

%%%%%%%%%%%%%%%%%%%%%%%%%%%%%%%%%%%%%%%%%%%%%%%%%%%%%%%%%%%%%%%%%%%%%%%%%%%%%%%%
\subsection{Command Line Processing}
\label{sec:commandline}

The effect of redirection files can also be achieved by invoking
the \LaTeX{} compiler with a more elaborate command line.
Most conveniently this should be done as part
of a shell script or a |Makefile|.

When using \textsf{childdoc} in the main file, the following
command lines effectively perform a redirection
(note that depending on the shell being used,
backslashes may have to be doubled: `|\|' $\to$ `|\\|'):
%
\begin{center}
|... -jobname "|\textit{target}|" |\\|"|[\textit{flags}]%
|\input{childdoc.def}\childdocforward[|\textit{main}|]{|\textit{dest}|}"|
\end{center}
%
Here \textit{target} is the name of the output file,
\textit{main} is the name of the main file
and \textit{dest} is the name of the main or child file to be processed
(all filenames without extensions).
The optional argument \textit{main} can be omitted
if \textit{main} matches \textit{dest}.
Optionally, compilation \textit{flags} can be defined via |\def| commands.
This command line makes the \TeX{} engine believe
it is compiling the file \textit{target}
whose content is specified as the latter parameter.
The provided code then forwards the processing to
\textit{main} or \textit{dest} as described in \secref{sec:forward}.

%%%%%%%%%%%%%%%%%%%%%%%%%%%%%%%%%%%%%%%%%%%%%%%%%%%%%%%%%%%%%%%%%%%%%%%%%%%%%%%%
\subsection{Include by Input}
\label{sec:input}

Including child documents by |\include| has some restrictions by design.
Most notably, the content of a child document always occupies
its own set of pages; pages cannot be shared between child documents.
Usually, this behaviour makes perfect sense
because each child document contain an essential part of the document.
However, in some situations it may be desirable to compose
a document from a collection of parts
without having mandatory page breaks between then.
For this case, the package
provides a mechanism to include parts
by |\input| which can also be processed individually.
However, by construction this mechanism
requires manual handling of the content to be output.

%%%%%%%%%%%%%%%%%%%%%%%%%%%%%%%%%%%%%%%%
\DescribeMacro{\ifchilddocmanual}
The main file should be prepared as usual, see \secref{sec:include}.
However, the document body must make a distinction
between processing of an individual part and of the main document, e.g.:
%
\begin{center}
\begin{tabular}{l}
|\ifchilddocmanual|\\
|\input{\childdocname}|\\
|\||else|\\
\textit{document body with }|\input{|\textit{part}|}|\\
|\||fi|
\end{tabular}
\end{center}
%
The conditional |\ifchilddocmanual| is true whenever
a part to be included by |\input| is being compiled,
and the name of the part is stored in |\childdocname|.

%%%%%%%%%%%%%%%%%%%%%%%%%%%%%%%%%%%%%%%%
\DescribeMacro{\childdocby}
Each part to be included by |\input| should start with:
%
\begin{center}
\begin{tabular}{l}
|\input{childdoc.def}|\\
|\childdocby{|\textit{main}|}|\\
\end{tabular}
\end{center}
%
The directive |\childdocby| is similar to |\childdocof|
described in \secref{sec:include},
but the subsequent selection of content must be done manually.
To that end, both |\ifchilddoc| and |\ifchilddocmanual|
will be true upon processing of a part,
and the name of the part is stored in |\childdocname|.
Note that |\jobname| will be set to the filename of the current part
so that each part receives an individual |.aux| file
that does not interfere with the |.aux| file(s) of the main document.
This behaviour can be altered by the alternative form
|\childdocby[*]{|\textit{main}|}| (with a non-empty optional argument)
which uses the |.aux| file of the main document
by setting |\jobname| to \textit{main}.

%%%%%%%%%%%%%%%%%%%%%%%%%%%%%%%%%%%%%%%%%%%%%%%%%%%%%%%%%%%%%%%%%%%%%%%%%%%%%%%%
\subsection{Driver Development}
\label{sec:driver}

The \textsf{childdoc} mechanism can also be use for the development
of definition files such as \LaTeX{} styles or classes.
This case differs from the above setup with multiple parts
included by |\include| in that no |\includeonly| should be invoked.
This can be achieved by starting the include file
(before |\ProvidesPackage|) with:
%
\begin{center}
\begin{tabular}{l}
|\input{childdoc.def}|\\
|\childdocforward{|\textit{main}|}|\\
\end{tabular}
\end{center}
%
or alternatively with:
%
\begin{center}
\begin{tabular}{l}
|\input{childdoc.def}|\\
|\childdocby{|\textit{main}|}|\\
\end{tabular}
\end{center}
%
Both forms have slightly different effects as described above.
The main file is prepared as usual, see \secref{sec:include}.

%%%%%%%%%%%%%%%%%%%%%%%%%%%%%%%%%%%%%%%%%%%%%%%%%%%%%%%%%%%%%%%%%%%%%%%%%%%%%%%%
\subsection{Legacy Detection}
\label{sec:detection}

The directive |\childdocmain| in the main file can detect
whether the complete document or merely a child is to be compiled
even without using the directive |\childdocof|.
This method is deprecated because it is less robust
and there is no compelling reason to use it;
it is merely provided for backward compatibility
and it may be removed in future versions.

If the detection mechanism is to be used,
it is mandatory to correctly specify
the filename of the main file as the argument of |\childdocmain|:
%
\begin{center}
\begin{tabular}{l}
|\input{childdoc.def}|\\
|\childdocmain{|\textit{main}|}|\\
\end{tabular}
\end{center}
%
If |\jobname| does not match the argument \textit{main} of |\childdocmain|,
it is assumed that |\jobname| points to the child file to be compiled.
When using |\childdocmain| with the main file specified as argument,
it suffices to start a child file
with just |\input{|\textit{main}|}|
without loading of the package and using |\childdocof|.
If instead all processing is done
with the appropriate \textsf{childdoc} directives,
the argument of \textit{main} of |\childdocmain| can be empty.

An alternative version of the command line processing described
in \secref{sec:commandline} using the detection mechanism reads:
%
\begin{center}
|... -jobname "|\textit{target}|" "|[\textit{flags}]%
[|\def\jobname{|\textit{dest}|}|]|\input{|\textit{main}|}"|
\end{center}

%%%%%%%%%%%%%%%%%%%%%%%%%%%%%%%%%%%%%%%%%%%%%%%%%%%%%%%%%%%%%%%%%%%%%%%%%%%%%%%%
\subsection{Manual Code}
\label{sec:manual}

In case one cannot be certain whether the definitions file |childdoc.def|
is installed on the target \TeX{} distribution
and one prefers not to ship it,
it is conceivable to paste a few relevant commands into the sources.

To that end, drop all statements |\input{childdoc.def}|
and perform the replacements as outlined below.
Instead of |\childdocmain{|\textit{main}|}| add the following code
to the top of the main file:
%
\begin{center}
\begin{tabular}{l}
|\||ifdefined\childdocname\endinput\||fi\newif\ifchilddoc|\\
|\edef\childdocname{\scantokens\expandafter{\jobname\noexpand}}|\\
|\def\childdocmain{|\textit{main}|}\||ifx\childdocmain\childdocname\||else|\\
|\childdoctrue\includeonly{\childdocname}\let\jobname\childdocmain\||fi|\\
\end{tabular}
\end{center}
%
Instead of |\childdocof{|\textit{main}|}| just include the main file
at the top of each child file:
%
\begin{center}
|\input{|\textit{main}|}|
\end{center}
%
A simple redirection |\childdocforward{|\textit{dest}|}| is achieved by:
%
\begin{center}
|\def\jobname{|\textit{dest}|}\input{\jobname}|
\end{center}
%
The redirection with prefix
|\childdocforwardprefix[|\textit{prefix}|]{|\textit{dest}|}|
is accomplished by:
%
\begin{center}
\begin{tabular}{l}
|{\edef\jobname{\scantokens\expandafter{\jobname\noexpand}}|\\
|\def\redirectjob |\textit{prefix}|#1~~~{\gdef\jobname{|\textit{dest}|#1}}|\\
|\expandafter\redirectjob\jobname~~~}\input{\jobname}|
\end{tabular}
\end{center}

In an alternative approach,
child documents can be compiled by a specific command line
without additional code or specific definitions:
%
\begin{center}
|... -jobname "|\textit{target}|" "|[\textit{flags}]%
|\includeonly{|\textit{dest}|}\input{|\textit{main}|}"|
\end{center}
%

%%%%%%%%%%%%%%%%%%%%%%%%%%%%%%%%%%%%%%%%%%%%%%%%%%%%%%%%%%%%%%%%%%%%%%%%%%%%%%%%
%%%%%%%%%%%%%%%%%%%%%%%%%%%%%%%%%%%%%%%%%%%%%%%%%%%%%%%%%%%%%%%%%%%%%%%%%%%%%%%%
\section{Information}

%%%%%%%%%%%%%%%%%%%%%%%%%%%%%%%%%%%%%%%%%%%%%%%%%%%%%%%%%%%%%%%%%%%%%%%%%%%%%%%%
\subsection{Copyright}

Copyright \copyright{} 2017--2018 Niklas Beisert

This work may be distributed and/or modified under the
conditions of the \LaTeX{} Project Public License, either version 1.3
of this license or (at your option) any later version.
The latest version of this license is in
  \url{http://www.latex-project.org/lppl.txt}
and version 1.3 or later is part of all distributions of \LaTeX{}
version 2005/12/01 or later.

This work has the LPPL maintenance status `maintained'.

The Current Maintainer of this work is Niklas Beisert.

This work consists of the files |README.txt|, |childdoc.ins| and |childdoc.dtx|
as well as the derived files |childdoc.def|, |cdocsamp.tex|
with |cdocsch1.tex|, |cdocsch2.tex|, |cdocspt3.tex|, |cdocspt4.tex|,
|cdocsdrf.tex|, |cdocsfn1.tex|, |cdocsfn2.tex|
as well as |childdoc.pdf|.

%%%%%%%%%%%%%%%%%%%%%%%%%%%%%%%%%%%%%%%%%%%%%%%%%%%%%%%%%%%%%%%%%%%%%%%%%%%%%%%%
\subsection{Files and Installation}

The package consists of the files:
%
\begin{center}
\begin{tabular}{ll}
    |README.txt|   & readme file \\
    |childdoc.ins| & installation file \\
    |childdoc.dtx| & source file \\
    |childdoc.def| & definition file \\
    |cdocsamp.tex| & sample main file \\
    |cdocsch1.tex| & sample include file \\
    |cdocsch2.tex| & sample include file \\
    |cdocspt3.tex| & sample part file \\
    |cdocspt4.tex| & sample part file \\
    |cdocsdrf.tex| & sample redirection file \\
    |cdocsfn1.tex| & sample redirection file \\
    |cdocsfn2.tex| & sample redirection file \\
    |childdoc.pdf| & manual
\end{tabular}
\end{center}
%
The distribution consists of the files
|README.txt|, |childdoc.ins| and |childdoc.dtx|.
%
\begin{itemize}
\item
Run (pdf)\LaTeX{} on |childdoc.dtx|
to compile the manual |childdoc.pdf| (this file).
\item
Run \LaTeX{} on |childdoc.ins| to create the definitions file |childdoc.def|
and the sample |cdocsamp.tex| with include files
|cdocsch1.tex|, |cdocsch2.tex|, |cdocspt3.tex|, |cdocspt4.tex|,
|cdocsdrf.tex|, |cdocsfn1.tex|, |cdocsfn2.tex|.
Then copy the file |childdoc.def| to an appropriate directory of your \LaTeX{}
distribution, e.g.\ \textit{texmf-root}|/tex/latex/childdoc|.
\end{itemize}

%%%%%%%%%%%%%%%%%%%%%%%%%%%%%%%%%%%%%%%%%%%%%%%%%%%%%%%%%%%%%%%%%%%%%%%%%%%%%%%%
\subsection{Related CTAN Packages}

There are several other packages which offer a similar functionality:
%
\begin{itemize}
\item
The packages
\href{http://ctan.org/pkg/docmute}{\textsf{docmute}},
\href{http://ctan.org/pkg/includex}{\textsf{includex}} and
\href{http://ctan.org/pkg/standalone}{\textsf{standalone}}
provide commands to include only the document body of
a child file thus allowing both files to be compiled individually.
\item
The packages \href{http://ctan.org/pkg/subdocs}{\textsf{subdocs}}
and \href{http://ctan.org/pkg/subfiles}{\textsf{subfiles}}
provide structures in which the main and child documents can be
encapsulated and allowing them to be compiled individually.
The inclusion mechanism is different from the conventional |\include|.
\item
The package \href{http://ctan.org/pkg/combine}{\textsf{combine}}
is an elaborate solution to combine several documents into one.
\end{itemize}
%
See also the CTAN topic \href{http://ctan.org/topic/subdocs}{\textsf{subdocs}}
for further related packages.
The present package differs from the above solutions in that
a document structure constructed with the conventional |\include| mechanism
just needs two extra commands at the top of every file
such that all constituent files can be compiled individually.

%%%%%%%%%%%%%%%%%%%%%%%%%%%%%%%%%%%%%%%%%%%%%%%%%%%%%%%%%%%%%%%%%%%%%%%%%%%%%%%%
%\subsection{Feature Suggestions}
%
%The following is a list of features which may be useful for future
%versions of this package:
%%
%\begin{itemize}
%\item
%\ldots
%\end{itemize}

%%%%%%%%%%%%%%%%%%%%%%%%%%%%%%%%%%%%%%%%%%%%%%%%%%%%%%%%%%%%%%%%%%%%%%%%%%%%%%%%
\subsection{Revision History}

%%%%%%%%%%%%%%%%%%%%%%%%%%%%%%%%%%%%%%%%
\paragraph{v2.0:} 2018/12/30

\begin{itemize}
\item
immediate forward processing
\item
added |\childdocby| mechanism
\item
manual restructured
\end{itemize}

%%%%%%%%%%%%%%%%%%%%%%%%%%%%%%%%%%%%%%%%
\paragraph{v1.6:} 2018/01/17

\begin{itemize}
\item
application for development of include files
\item
corrections to manual
\end{itemize}

%%%%%%%%%%%%%%%%%%%%%%%%%%%%%%%%%%%%%%%%
\paragraph{v1.5:} 2017/05/21

\begin{itemize}
\item
more complete structuring introduced
\item
|\childdocof| introduced
\item
|\childdoc| renamed to |\childdocmain|
\item
|\childredirect| renamed to |\childdocforward| and |\childdocforwardprefix|
and functionality expanded
\end{itemize}

%%%%%%%%%%%%%%%%%%%%%%%%%%%%%%%%%%%%%%%%
\paragraph{v1.0:} 2017/04/27

\begin{itemize}
\item
manual and install package
\item
first version published on CTAN
\end{itemize}

%%%%%%%%%%%%%%%%%%%%%%%%%%%%%%%%%%%%%%%%
\paragraph{v0.6:} 2017/04/26

\begin{itemize}
\item
redirection mechanism added
\end{itemize}

%%%%%%%%%%%%%%%%%%%%%%%%%%%%%%%%%%%%%%%%
\paragraph{v0.5:} 2017/04/26

\begin{itemize}
\item
functionality in definition file
\end{itemize}


%%%%%%%%%%%%%%%%%%%%%%%%%%%%%%%%%%%%%%%%%%%%%%%%%%%%%%%%%%%%%%%%%%%%%%%%%%%%%%%%
%%%%%%%%%%%%%%%%%%%%%%%%%%%%%%%%%%%%%%%%%%%%%%%%%%%%%%%%%%%%%%%%%%%%%%%%%%%%%%%%
%%%%%%%%%%%%%%%%%%%%%%%%%%%%%%%%%%%%%%%%%%%%%%%%%%%%%%%%%%%%%%%%%%%%%%%%%%%%%%%%
\appendix

\settowidth\MacroIndent{\rmfamily\scriptsize 000\ }

 \DocInput{childdoc.dtx}

\end{document}
%</driver>
% \fi
%
% %%%%%%%%%%%%%%%%%%%%%%%%%%%%%%%%%%%%%%%%%%%%%%%%%%%%%%%%%%%%%%%%%%%%%%%%%%%%%%
% %%%%%%%%%%%%%%%%%%%%%%%%%%%%%%%%%%%%%%%%%%%%%%%%%%%%%%%%%%%%%%%%%%%%%%%%%%%%%%
% \section{Sample}
%\iffalse
%<*samplemain>
%\fi
%
% The following presents a sample document
% with two chapters, two parts, a title page,
% a compile flag as well as three forwarding files to set the flag.
% It consists of eight |.tex| files:
% \begin{center}
% \begin{tabular}{ll}
% |cdocsamp.tex|&main file\\
% |cdocsch1.tex|&include file for chapter 1\\
% |cdocsch2.tex|&include file for chapter 2\\
% |cdocspt3.tex|&include file for part 3\\
% |cdocspt4.tex|&include file for part 4\\
% |cdocsdrf.tex|&forwarding file for main file in draft mode\\
% |cdocsfi1.tex|&forwarding file for final version of chapter 1\\
% |cdocsfi2.tex|&forwarding file for final version of chapter 2\\
% \end{tabular}
% \end{center}
% Each of the eight files can be compiled directly by the \LaTeX{} compiler.
%
% %%%%%%%%%%%%%%%%%%%%%%%%%%%%%%%%%%%%%%
% \paragraph{Main File.}
%
% The main file is called |cdocsamp.tex|.
%
% Load the \textsf{childdoc} definitions and
% declare the filename for the main document:
%    \begin{macrocode}
\input{childdoc.def}
\childdocmain{}
%    \end{macrocode}

% Optional override for |\version| flag:
%    \begin{macrocode}
%%\ifchilddoc\else\providecommand{\version}{draft}\fi
%    \end{macrocode}

% Define the default values for the |\version| flag
% (|final| for the main file and |draft| for childs):
%    \begin{macrocode}
\ifchilddoc
\providecommand{\version}{draft}
\else
\providecommand{\version}{final}
\fi
%    \end{macrocode}

% Load the standard document class:
%    \begin{macrocode}
\documentclass[12pt]{article}
%    \end{macrocode}

% Start the document body:
%    \begin{macrocode}
\begin{document}
%    \end{macrocode}

% Declare a title page.
% Print title, part of document being processed and version flag:
%    \begin{macrocode}
\addtocounter{page}{-1}
\begin{center}
{\LARGE\bfseries{}childdoc example\par}
\vspace{1cm}
\ifchilddoc
\ifchilddocmanual part\else chapter\fi:
`\childdocname' of `\childdocjob'\par
\else
main document: `\childdocjob'\par
\fi
version: \version\par
\end{center}
\newpage
%    \end{macrocode}

% Manually include selected file,
% otherwise process as usual:
%    \begin{macrocode}
\ifchilddocmanual
\section*{part `\childdocname'}
\input{\childdocname}
\else
%    \end{macrocode}

% Include the two chapters:
%    \begin{macrocode}
\include{cdocsch1}
\include{cdocsch2}
%    \end{macrocode}

% Include the two parts unless only chapters should be displayed:
%    \begin{macrocode}
\ifchilddoc\else
\section{part three}
\input{cdocspt3}
\section{part four}
\input{cdocspt4}
\fi
%    \end{macrocode}

% Process as usual until here:
%    \begin{macrocode}
\fi
%    \end{macrocode}

% End of document body:
%    \begin{macrocode}
\end{document}
%    \end{macrocode}
%\iffalse
%</samplemain>
%\fi
%
% %%%%%%%%%%%%%%%%%%%%%%%%%%%%%%%%%%%%%%
% \paragraph{Chapter Include Files.}
%
% The include files are called |cdocsch1.tex| and |cdocsch2.tex|.
%
%\iffalse
%<*samplechap1|samplechap2>
%\fi

% Optional override for |\version| flag:
%    \begin{macrocode}
%%\providecommand{\version}{final}
%    \end{macrocode}

% Include the main document:
%    \begin{macrocode}
\input{childdoc.def}
\childdocof{cdocsamp}
%    \end{macrocode}

%\iffalse
%</samplechap1|samplechap2>
%\fi
%
%\iffalse
%<*samplechap1>
%\fi
% Some text for chapter 1:
%    \begin{macrocode}
\section{one}
some text in chapter one
%    \end{macrocode}

%\iffalse
%</samplechap1>
%\fi
% Some text for chapter 2:
%\iffalse
%<*samplechap2>
%\fi
%    \begin{macrocode}
\section{two}
more text in chapter two
%    \end{macrocode}

%\iffalse
%</samplechap2>
%\fi
%
% %%%%%%%%%%%%%%%%%%%%%%%%%%%%%%%%%%%%%%
% \paragraph{Part Include Files.}
%
% The include files are called |cdocspt3.tex| and |cdocspt4.tex|.
%
%\iffalse
%<*samplepart3|samplepart4>
%\fi

% Optional override for |\version| flag:
%    \begin{macrocode}
%%\providecommand{\version}{final}
%    \end{macrocode}

% Include the main document:
%    \begin{macrocode}
\input{childdoc.def}
\childdocby{cdocsamp}
%    \end{macrocode}

%\iffalse
%</samplepart3|samplepart4>
%\fi
%
%\iffalse
%<*samplepart3>
%\fi
% Some text for part 3:
%    \begin{macrocode}
some text in part three
%    \end{macrocode}

%\iffalse
%</samplepart3>
%\fi
% Some text for part 4:
%\iffalse
%<*samplepart4>
%\fi
%    \begin{macrocode}
more text in part four
%    \end{macrocode}

%\iffalse
%</samplepart4>
%\fi
%
% %%%%%%%%%%%%%%%%%%%%%%%%%%%%%%%%%%%%%%
% \paragraph{Forwarding for a Complete Draft.}
%
% The following forwarding file |cdocsdrf.tex|
% compiles the main document in draft mode:
%\iffalse
%<*sampledraft>
%\fi
%    \begin{macrocode}
\def\version{draft}
\input{childdoc.def}
\childdocforward{cdocsamp}
%    \end{macrocode}

%\iffalse
%</sampledraft>
%\fi
%
% %%%%%%%%%%%%%%%%%%%%%%%%%%%%%%%%%%%%%%
% \paragraph{Forwarding for Final Version of the Chapters.}
%
% The following forwarding files |cdocsfn1.tex| and |cdocsfn2.tex|
% (with identical content)
% compile the final versions of the child documents
% |cdocsch1.tex| and |cdocsch2.tex|, respectively:
%\iffalse
%<*samplefinal>
%\fi
%    \begin{macrocode}
\def\version{final}
\input{childdoc.def}
\childdocforwardprefix[cdocsamp]{cdocsfn}{cdocsch}
%    \end{macrocode}

%\iffalse
%</samplefinal>
%\fi
%
% %%%%%%%%%%%%%%%%%%%%%%%%%%%%%%%%%%%%%%
% \paragraph{Command Line Processing.}
%
% The following three command lines generate the output files
% |cdocscld|, |cdocscl1| and |cdocscl2|
% which should be identical to
% |cdocsdrf|, |cdocsch1| and |cdocsfn2|, respectively:
% \begin{center}
% \begin{tabular}{l}
% |latex -jobname cdocscld \|\\
% |  "\def\version{draft}\input{childdoc.def}\childdocforward{cdocsamp}"|\\
% |latex -jobname cdocscl1 \|\\
% |  "\input{childdoc.def}\childdocforward[cdocsamp]{cdocsch1}"|\\
% |latex -jobname cdocscl2 \|\\
% |  "\def\version{final}\input{childdoc.def}\childdocforward{cdocsch2}"|
% \end{tabular}
% \end{center}
% Note that the trailing backslash on each first line
% merely continues the input to the second line
% (for convenient cut ant paste).
% Furthermore, the command |latex| can be replaced by any
% of its alternative versions such as |pdflatex|.
%
% %%%%%%%%%%%%%%%%%%%%%%%%%%%%%%%%%%%%%%%%%%%%%%%%%%%%%%%%%%%%%%%%%%%%%%%%%%%%%%
% %%%%%%%%%%%%%%%%%%%%%%%%%%%%%%%%%%%%%%%%%%%%%%%%%%%%%%%%%%%%%%%%%%%%%%%%%%%%%%
% \section{Implementation}
%\iffalse
%<*package>
%\fi
%
% This section describes the definitions file |childdoc.def|.

% The definitions cannot be loaded using |\usepackage| or |\RequirePackage|
% which has a mechanism to prevent loading a style file more than once.
% When loading the definitions by means of |\input|
% multiple instances have to be prevented manually:
%\iffalse
%This code needs to be before the `\ProvidesFile' directive
%which is defined at the beginning of this file.
%Therefore it is also placed there and commented out here.
%</package>
%<*discard>
%\fi
%    \begin{macrocode}
\ifdefined\childdocmain\endinput\fi
%    \end{macrocode}
%\iffalse
%</discard>
%<*package>
%\fi
%
% \macro{\ifchilddoc}
% \macro{\ifchilddocmanual}
% The conditional |\ifchilddoc| tells whether a
% child (true) or main (false) document is being compiled.
% The conditional |\ifchilddocmanual| tells whether
% the |\includeonly| mechanism is used (false) or
% the selection of child files must be performed manually (true).
% The definitions initialise to false:
%    \begin{macrocode}
\newif\ifchilddoc
\newif\ifchilddocmanual
%    \end{macrocode}

% \macro{\childdocname}
% \macro{\childdocjob}
% The macro |\childdocname| stores the name of the main document
% to be compiled. The macro |\childdocjob| stores the name of
% the document on which the \LaTeX{} compiler was originally invoked.
% The content of |\jobname| cannot be compared
% to filenames specified in the source due to different catcodes.
% The following code rescans |\jobname|, stores the result
% in |\childdocname| and saves a copy in |\childdocjob|:
%    \begin{macrocode}
\edef\childdocname{\scantokens\expandafter{\jobname\noexpand}}
\let\childdocjob\childdocname
%    \end{macrocode}

% \macro{\childdocdisable}
% The macro |\childdocdisable| prevents the main file
% from being processed more than once.
% At this stage, the main document command |\childdocmain|
% is assumed to be called once again where it should do nothing.
% Any subsequent call to it should prevent
% a secondary processing of the main document
% It overwrites the forwarding commands
% |\childdocof| and |\childdocforward|
% with empty macros to prevent further inclusions of the main document:
%    \begin{macrocode}
\newcommand{\childdocdisable}
{
  \renewcommand{\childdocmain}[1]{\renewcommand{\childdocmain}[1]{\endinput}}
  \renewcommand{\childdocof}[1]{}
  \renewcommand{\childdocby}[2][]{}
  \renewcommand{\childdocforward}[2][]{}
  \renewcommand{\childdocdisable}{}
}
%    \end{macrocode}

% \macro{\childdocmain}
% The macro |\childdocmain| is to be called at the top of the main file
% with nothing or the main filename (without extension) as argument.
% First, it breaks loops.
% If the argument is not empty and does not match |\childdocname|
% (which is set by the first inclusion of |childdoc.def|),
% |\ifchilddoc| is set to true, |\includeonly| is applied to the child file
% and |\jobname| is set to the main file
% (for proper handling of |.aux| files):
%    \begin{macrocode}
\newcommand{\childdocmain}[1]
{
  \childdocdisable\childdocmain{}
  \if?#1?\else
    \begingroup
      \def\childdoctmp{#1}
      \ifx\childdoctmp\childdocname
        \def\childdoctmp{}
      \else
        \def\childdoctmp
        {
          \childdoctrue
          \includeonly{\childdocname}
          \def\childdocjob{#1}
          \def\jobname{#1}
        }
      \fi
      \expandafter
    \endgroup
    \childdoctmp
  \fi
}
%    \end{macrocode}

% \macro{\childdocof}
% The command |\childdocof| redirects
% compilation to the main file |#1|.
%    \begin{macrocode}
\newcommand{\childdocof}[1]
{
  \childdocdisable
  \childdoctrue
  \includeonly{\childdocname}
  \def\jobname{#1}
  \def\childdocjob{#1}
  \input{#1}
}
%    \end{macrocode}

% \macro{\childdocby}
% The command |\childdocby| ....
%    \begin{macrocode}
\newcommand{\childdocby}[2][]
{
  \childdocdisable
  \childdoctrue
  \childdocmanualtrue
  \if?#1?\else
    \def\jobname{#2}
  \fi
  \def\childdocjob{#2}
  \input{#2}
  \endinput
}
%    \end{macrocode}

% \macro{\childdocforward}
% The command |\childdocforward| redirects
% compilation to the main file or
% (if the optional argument is given) a child file.
% Parameters are set as if the main file
% or a child file starting with |\childdocof| was compiled.
% Then compilation is handed over to the main file:
%    \begin{macrocode}
\newcommand{\childdocforward}[2][]
{
  \begingroup
    \if?#1?
      \def\childdoctmp
      {
        \def\childdocname{#2}
        \def\childdocjob{#2}
        \def\jobname{#2}
        \input{#2}
        \endinput
      }
    \else
      \def\childdoctmp
      {
        \childdocdisable
        \def\childdocname{#2}
        \childdoctrue
        \includeonly{#2}
        \def\childdocjob{#1}
        \def\jobname{#1}
        \input{#1}
        \endinput
      }
    \fi
    \expandafter
  \endgroup
  \childdoctmp
}
%    \end{macrocode}

% \macro{\childdocforwardprefix}
% The command |\childdocforwardprefix| redirects
% compilation to the main or a child file by means of a pattern.
% The prefix |#1| in the current filename is replaced by |#2|
% and the suffix of the current filename is kept
% (it is assumed that the filename does not contain the substring `|~~~|'
% which is used as a delimiter).
% Compilation is handed over to the new file by |\childdocforward|:
%    \begin{macrocode}
\newcommand{\childdocforwardprefix}[3][]
{
  \begingroup
    \def\childdocextract #2##1~~~{\def\childdoctmp{\childdocforward[#1]{#3##1}}}
    \expandafter\childdocextract\childdocname~~~
    \expandafter
  \endgroup
  \childdoctmp
}
%    \end{macrocode}

% \macro{\childdoc}
% The deprecated macro |\childdoc| is a legacy version of |\childdocmain|:
%    \begin{macrocode}
\newcommand{\childdoc}{\childdocmain}
%    \end{macrocode}

% \macro{\childdocredirect}
% The deprecated macro |\childdocredirect| is a legacy version
% of |\childdocforward| and |\childdocforwardprefix|:
%    \begin{macrocode}
\newcommand{\childdocredirect}[2][]
{
  \begingroup
    \if?#1?
      \def\childdoctmp{\childdocforward{#2}}
    \else
      \def\childdoctmp{\childdocforwardprefix{#1}{#2}}
    \fi
    \expandafter
  \endgroup
  \childdoctmp
}
%    \end{macrocode}

%\iffalse
%</package>
%\fi
%
\endinput
|\\
|\childdocby{|\textit{main}|}|\\
\end{tabular}
\end{center}
%
The directive |\childdocby| is similar to |\childdocof|
described in \secref{sec:include},
but the subsequent selection of content must be done manually.
To that end, both |\ifchilddoc| and |\ifchilddocmanual|
will be true upon processing of a part,
and the name of the part is stored in |\childdocname|.
Note that |\jobname| will be set to the filename of the current part
so that each part receives an individual |.aux| file
that does not interfere with the |.aux| file(s) of the main document.
This behaviour can be altered by the alternative form
|\childdocby[*]{|\textit{main}|}| (with a non-empty optional argument)
which uses the |.aux| file of the main document
by setting |\jobname| to \textit{main}.

%%%%%%%%%%%%%%%%%%%%%%%%%%%%%%%%%%%%%%%%%%%%%%%%%%%%%%%%%%%%%%%%%%%%%%%%%%%%%%%%
\subsection{Driver Development}
\label{sec:driver}

The \textsf{childdoc} mechanism can also be use for the development
of definition files such as \LaTeX{} styles or classes.
This case differs from the above setup with multiple parts
included by |\include| in that no |\includeonly| should be invoked.
This can be achieved by starting the include file
(before |\ProvidesPackage|) with:
%
\begin{center}
\begin{tabular}{l}
|% \iffalse
%
% childdoc.dtx Copyright (C) 2017-2018 Niklas Beisert
%
% This work may be distributed and/or modified under the
% conditions of the LaTeX Project Public License, either version 1.3
% of this license or (at your option) any later version.
% The latest version of this license is in
%   http://www.latex-project.org/lppl.txt
% and version 1.3 or later is part of all distributions of LaTeX
% version 2005/12/01 or later.
%
% This work has the LPPL maintenance status `maintained'.
%
% The Current Maintainer of this work is Niklas Beisert.
%
% This work consists of the files childdoc.dtx and childdoc.ins
% and the derived files childdoc.def and cdocsamp.tex with
% cdocsch1.tex, cdocsch2.tex, cdocsdrf.tex, cdocsfn1.tex, cdocsfn2.tex.
%
%<package>\ifdefined\childdocmain\endinput\fi
%<package>\ProvidesFile{childdoc.def}[2018/12/30 v2.0 child document driver]
%<samplemain>\ProvidesFile{cdocsamp.tex}[2018/12/30 v2.0 sample for childdoc]
%<*driver>
%\ProvidesFile{childdoc.drv}[2018/12/30 v2.0 childdoc reference manual file]
\PassOptionsToClass{10pt,a4paper}{article}
\documentclass{ltxdoc}

\usepackage[margin=35mm]{geometry}
\usepackage{hyperref}
\usepackage{hyperxmp}
\usepackage[usenames]{color}

\hypersetup{colorlinks=true}
\hypersetup{pdfstartview=FitH}
\hypersetup{pdfpagemode=UseNone}
\hypersetup{pdfsource={}}
\hypersetup{pdflang={en-UK}}
\hypersetup{pdfcopyright={Copyright 2017-2018 Niklas Beisert.
  This work may be distributed and/or modified under the
  conditions of the LaTeX Project Public License, either version 1.3
  of this license or (at your option) any later version.}}
\hypersetup{pdflicenseurl={http://www.latex-project.org/lppl.txt}}
\hypersetup{pdfcontactaddress={ETH Zurich, ITP, HIT K,
  Wolfgang-Pauli-Strasse 27}}
\hypersetup{pdfcontactpostcode={8093}}
\hypersetup{pdfcontactcity={Zurich}}
\hypersetup{pdfcontactcountry={Switzerland}}
\hypersetup{pdfcontactemail={nbeisert@itp.phys.ethz.ch}}
\hypersetup{pdfcontacturl={http://people.phys.ethz.ch/\xmptilde nbeisert/}}

\newcommand{\secref}[1]{\hyperref[#1]{section \ref*{#1}}}

\parskip1ex
\parindent0pt
\let\olditemize\itemize
\def\itemize{\olditemize\parskip0pt}

\begin{document}

\title{The \textsf{childdoc} Package}
\hypersetup{pdftitle={The childdoc Package}}
\author{Niklas Beisert\\[2ex]
  Institut f\"ur Theoretische Physik\\
  Eidgen\"ossische Technische Hochschule Z\"urich\\
  Wolfgang-Pauli-Strasse 27, 8093 Z\"urich, Switzerland\\[1ex]
  \href{mailto:nbeisert@itp.phys.ethz.ch}
  {\texttt{nbeisert@itp.phys.ethz.ch}}}
\hypersetup{pdfauthor={Niklas Beisert}}
\hypersetup{pdfsubject={Manual for the LaTeX2e Package childdoc}}
\date{30 December 2018, \textsf{v2.0}}
\maketitle

\begin{abstract}\noindent
\textsf{childdoc} is a \LaTeXe{} package
that enables the direct compilation
of document sections included by |\include|
to individual files.
\end{abstract}

\begingroup
\parskip0ex
\tableofcontents
\endgroup

%%%%%%%%%%%%%%%%%%%%%%%%%%%%%%%%%%%%%%%%%%%%%%%%%%%%%%%%%%%%%%%%%%%%%%%%%%%%%%%%
%%%%%%%%%%%%%%%%%%%%%%%%%%%%%%%%%%%%%%%%%%%%%%%%%%%%%%%%%%%%%%%%%%%%%%%%%%%%%%%%
\section{Introduction}

\LaTeX{} provides a mechanism to structure a large document (such as a book)
into a main file and several child files (containing the chapters)
using the |\include| command.
This mechanism is beneficial for documents
which span hundreds of pages in order to
make the source file(s) more manageable.
Moreover, compilation can be restricted to
selected child files by means of the |\includeonly| command.
The latter feature can be used to reduce the compilation time while editing
(this was significantly more useful in the earlier days of \LaTeX{})
or to generate a smaller document which is easier to navigate.
Another application of |\includeonly| is to generate
documents consisting of selected parts of the complete document.

However, there are a few drawbacks of the plain |\include| mechanism:
\begin{itemize}
\item
The child files cannot be compiled on their own,
they can only be compiled via the main file.
A naive editing environment
(such as a text editor with an option
to have the current file processed by \LaTeX)
may require one to switch to the main file before compiling;
attempting to compile the child file produces errors.
\item
The main file must be modified (each time)
to adjust the |\includeonly| command
to the present needs. This easily leaves the main file in a messy state.
\item
The generated document will always carry the filename
of the main document. This is inconvenient if
several child files are to be compiled and
to be kept for distribution.
\end{itemize}

The present package provides a simple interface
to make child files individually compilable by \LaTeX{}.
Compiling a child file then has the same effect as compiling
the main file with an |\includeonly| command
to select the appropriate child.
Moreover the generated document will carry the name of the child
rather than the main file.
This resolves all three above issues.

This feature is meant to make the editing of books,
thesis documents and lecture notes somewhat more convenient.
However, the package can also be used efficiently for
composing a series of documents (such as exercise sheets)
which are typically distributed individually.
It then assists the author in generating the individual documents
(potentially in different versions)
as well as a document containing the collected series.
Another application is in developing style files
or other kinds of included material
where compilation of the style file could redirect
to a sample or test file.

%%%%%%%%%%%%%%%%%%%%%%%%%%%%%%%%%%%%%%%%%%%%%%%%%%%%%%%%%%%%%%%%%%%%%%%%%%%%%%%%
%%%%%%%%%%%%%%%%%%%%%%%%%%%%%%%%%%%%%%%%%%%%%%%%%%%%%%%%%%%%%%%%%%%%%%%%%%%%%%%%
\section{Usage}

First of all, the package \textsf{childdoc} is \emph{not} a standard
\LaTeXe{} |.sty| style file! Therefore it needs to be invoked in
a non-standard way.

%%%%%%%%%%%%%%%%%%%%%%%%%%%%%%%%%%%%%%%%%%%%%%%%%%%%%%%%%%%%%%%%%%%%%%%%%%%%%%%%
\subsection{Included Files}
\label{sec:include}

%%%%%%%%%%%%%%%%%%%%%%%%%%%%%%%%%%%%%%%%
\DescribeMacro{\childdocmain}
To use the package, add the commands
\begin{center}
\begin{tabular}{l}
|\input{childdoc.def}|\\
|\childdocmain{}|\\
\end{tabular}
\end{center}
at the very top of the main \LaTeX{} file,
in particular \emph{before} the |\documentclass| statement!
The argument of |\childdocmain| should be left empty
(but it must be present).

%%%%%%%%%%%%%%%%%%%%%%%%%%%%%%%%%%%%%%%%
\DescribeMacro{\childdocof}
Furthermore, add the commands
\begin{center}
\begin{tabular}{l}
|\input{childdoc.def}|\\
|\childdocof{|\textit{main}|}|\\
\end{tabular}
\end{center}
at the top of every child file \textit{child}
which is included by |\include{|\textit{child}|}|
from within the main file
(or at least for those files to be compiled individually).
The argument \textit{main} must be the filename of the main file.

There are a couple of
considerations in setting up the main and child documents:

%%%%%%%%%%%%%%%%%%%%%%%%%%%%%%%%%%%%%%%%
\paragraph{Restrictions.}

Please note the following restrictions:
\begin{itemize}
\item
|\childdocmain| must be called with one argument \textit{main}
to ensure compatibility with earlier version of the package.
It must either be empty (|\childdocmain{}|)
or precisely match the filename of the main file in which it is specified.
See \secref{sec:detection} for further information.
\item
The filename \textit{main} must be specified without the |.tex| extension.
\item
The filename \textit{main} is case sensitive
(even in case-insensitive file systems)
due to internal string comparison.
\item
The argument \textit{main} should be fully expanded, it cannot be a macro.
\item
Subdirectories and special characters should be avoided in filenames.
\item
The command |\childdocmain{|\textit{main}|}| must be followed by a whitespace.
It should not be followed immediately by another command
or by a comment mark `|%|'.
This is because the \TeX{} parser reads the token immediately following
the argument of |\childdocmain| and puts it
at the beginning of every child section;
however, a white\-space is ignored.
\end{itemize}

%%%%%%%%%%%%%%%%%%%%%%%%%%%%%%%%%%%%%%%%
\paragraph{Content of Main File.}

It is advisable to place all content in the child files included by |\include|.
Any output contained in the main file will appear in all child documents
unless suppressed manually;
it cannot be suppressed automatically by the |\includeonly| directive
and thus should normally be avoided.
A method to include some content in the main file
by means of conditional processing is described in \secref{sec:conditional}.

%%%%%%%%%%%%%%%%%%%%%%%%%%%%%%%%%%%%%%%%
\paragraph{Page Numbering.}

When only a part of the document is compiled,
the appropriate numbering of pages
(as well as other status parameters)
is determined from the |.aux| files.
The latter contain information from previous passes.
However this information needs to propagate through
all intermediate child documents.
Therefore the page numbering in child documents may well
be inconsistent until the complete document is compiled at least once.

A useful (if unconventional) way to always ensure a consistent
page numbering is to restart the numbering in each child document
and denote the pages by `\textit{child}|.|\textit{page}'
where \textit{child} represents the chapter/section number of the child file.
This can be achieved by the command
|\numberwithin{page}{|\textit{child}|}|
of the \textsf{amsmath} package
where \textit{child} can be |chapter| or |section|
depending on the chosen structuring.
Alternatively, one can modify the macro |\thepage| appropriately
and reset the counter |page| at the start of each child file.

%%%%%%%%%%%%%%%%%%%%%%%%%%%%%%%%%%%%%%%%%%%%%%%%%%%%%%%%%%%%%%%%%%%%%%%%%%%%%%%%
\subsection{Conditional Processing}
\label{sec:conditional}

The package provides a mechanism to compile different versions
of a document. To customise the versions further some conditional processing
can come in handy to distinguish which version is being compiled.
The package provides two macros to describe the compilation context:

%%%%%%%%%%%%%%%%%%%%%%%%%%%%%%%%%%%%%%%%
\DescribeMacro{\ifchilddoc}
The conditional |\ifchilddoc| distinguishes between the compilation of
child documents and the main document:
%
\begin{center}
|\ifchilddoc |\textit{child-code}| |[|\||else |\textit{main-code}]| \||fi|
\end{center}

%%%%%%%%%%%%%%%%%%%%%%%%%%%%%%%%%%%%%%%%
\DescribeMacro{\childdocname}
\DescribeMacro{\childdocjob}
The macro |\childdocname| contains the filename (without extension)
of the main or child file being processed.
Note that |\childdocjob| will always contain the name of the main file.

%%%%%%%%%%%%%%%%%%%%%%%%%%%%%%%%%%%%%%%%
\paragraph{Title Page.}

Conditional processing can be used to include a title or banner page
in the main document when proper precautions are taken.
Importantly, the code in the main file should ensure that the page counter
(as well as other status parameters which are stored in the |.aux| files)
takes the same value after the conditional processing.
Otherwise the page numbers may take divergent values
depending on which part is compiled.

For example, a title page could be declared by:
%
\begin{center}
\begin{tabular}{l}
|\ifchilddoc\||else|\\
|\addtocounter{page}{-1}|\\
\textit{code for title page}\\
|\newpage|\\
|\||fi|
\end{tabular}
\end{center}
%
A banner page for the child documents can be generated by:
%
\begin{center}
\begin{tabular}{l}
|\ifchilddoc|\\
|\addtocounter{page}{-1}|\\
\textit{code for banner page}\\
|\newpage|\\
|\||fi|
\end{tabular}
\end{center}
%
Here one could write a message such as:
\begin{center}
|This is the part \childdocname{} of \childdocjob{}.|
\end{center}

%%%%%%%%%%%%%%%%%%%%%%%%%%%%%%%%%%%%%%%%%%%%%%%%%%%%%%%%%%%%%%%%%%%%%%%%%%%%%%%%
\subsection{Flags}
\label{sec:flags}

The package makes it easy to generate different versions
of the main or child documents.
To this end compilation flags can be defined
and assigned different default values.
They will be particularly useful in conjunction
with the forwarding mechanism described in \secref{sec:forward}.

For example, it may be useful to have a flag |\version|
which can be set to |draft| or |final|.
The document source will contain some conditional code
depending on the value of |\version|.
Suppose further, the flag should default to |final| for the main file
and to |draft| for child files
which is a natural assignment for editing the document.
This is achieved by placing the following code
in the preamble of the main document
(below the |\childdocmain| directive):
%
\begin{center}
\begin{tabular}{l}
|\ifchilddoc|\\
|\providecommand{\version}{draft}|\\
|\||else|\\
|\providecommand{\version}{final}|\\
|\||fi|
\end{tabular}
\end{center}
%
The definition by |\providecommand| makes sure
that previous definitions are not overwritten.
Further statements |\providecommand{\version}{...}|
can thus be added before the above code to override it.

For the main file, one might add a line
(between |\childdocmain| and the above block)
%
\begin{center}
|%\ifchilddoc\||else\providecommand{\version}{draft}\||fi|
\end{center}
%
which can be uncommented to produce a draft version.
Likewise one can add a line to the very top of a child file
(above the |\childdocof{|\textit{main}|}| directive)
%
\begin{center}
|%\providecommand{\version}{final}|
\end{center}
%
which can be uncommented to produce the final version of this child document.

%%%%%%%%%%%%%%%%%%%%%%%%%%%%%%%%%%%%%%%%%%%%%%%%%%%%%%%%%%%%%%%%%%%%%%%%%%%%%%%%
\subsection{Forwarding}
\label{sec:forward}

Different versions of the main or child documents
using compilation flags as described in \secref{sec:flags}
can be (permanently) stored in different files
for convenient compilation, viewing and distribution.
To this end, the package defines a command
to pass on compilation to a different file:

%%%%%%%%%%%%%%%%%%%%%%%%%%%%%%%%%%%%%%%%
\DescribeMacro{\childdocforward}
The command |\childdocforward| redirects processing to
another source file:
%
\begin{center}
\begin{tabular}{l}
|\input{childdoc.def}|\\
|\childdocforward[|\textit{main}|]{|\textit{dest}|}|\\
\end{tabular}
\end{center}
%
The argument \textit{dest} is the destination file
(without extension).
It should be the main file or one of the child files.
Note that further \textsf{childdoc} directives
such as |\childdocof| and |\childdocforward|
in the indicated file will be processed in this form.
The optional argument \textit{main}
passes on directly to the main file \textit{main}
while pretending to compile the child \textit{dest}.
This form behaves as if \textit{dest}
issues |\childdocof{|\textit{main}|}| right away,
and no further \textsf{childdoc} directives will be processed.

%%%%%%%%%%%%%%%%%%%%%%%%%%%%%%%%%%%%%%%%
\DescribeMacro{\...prefix}
In the alternative form |\childdocforwardprefix|,
%
\begin{center}
\begin{tabular}{l}
|\input{childdoc.def}|\\
|\childdocforwardprefix[|\textit{main}|]{|\textit{prefix}|}{|\textit{dest}|}|
\end{tabular}
\end{center}
%
the destination file is determined by a pattern
depending on the current file:
To make this work, the current file must be called
`{\textit{prefix}\hspace{0.2em}\textit{suffix}}'
with \textit{prefix} matching precisely the argument.
Processing is then passed on to the file
`{\textit{dest}\hspace{0.2em}\textit{suffix}}'.
Surely, the same effect is achieved by
directly specifying the
argument `{\textit{dest}\hspace{0.2em}\textit{suffix}}'
in the first form.
However, that requires to set up a different file
for each child. With the alternative form of the command
all these files can have exactly the same content
which simplifies setting them up and maintaining them.

For example, the following file |draft.tex|
with a compilation flag |\version| as described in \secref{sec:flags}
compiles the main document as a draft:
%
\begin{center}
\begin{tabular}{l}
|\def\version{draft}|\\
|\input{childdoc.def}|\\
|\childdocforward{|\textit{main}|}|
\end{tabular}
\end{center}
%
Likewise, the following files |final|\textit{nn}|.tex|
compile the final version of the child document
|child|\textit{nn}|.tex|:
%
\begin{center}
\begin{tabular}{l}
|\def\version{final}|\\
|\input{childdoc.def}|\\
|\childdocforwardprefix{final}{child}|
\end{tabular}
\end{center}
%

Note that when several versions of a main file and/or of each child file
are to be generated, it may be convenient to set up a |Makefile| or
shell script to automatise the process.

%%%%%%%%%%%%%%%%%%%%%%%%%%%%%%%%%%%%%%%%%%%%%%%%%%%%%%%%%%%%%%%%%%%%%%%%%%%%%%%%
\subsection{Command Line Processing}
\label{sec:commandline}

The effect of redirection files can also be achieved by invoking
the \LaTeX{} compiler with a more elaborate command line.
Most conveniently this should be done as part
of a shell script or a |Makefile|.

When using \textsf{childdoc} in the main file, the following
command lines effectively perform a redirection
(note that depending on the shell being used,
backslashes may have to be doubled: `|\|' $\to$ `|\\|'):
%
\begin{center}
|... -jobname "|\textit{target}|" |\\|"|[\textit{flags}]%
|\input{childdoc.def}\childdocforward[|\textit{main}|]{|\textit{dest}|}"|
\end{center}
%
Here \textit{target} is the name of the output file,
\textit{main} is the name of the main file
and \textit{dest} is the name of the main or child file to be processed
(all filenames without extensions).
The optional argument \textit{main} can be omitted
if \textit{main} matches \textit{dest}.
Optionally, compilation \textit{flags} can be defined via |\def| commands.
This command line makes the \TeX{} engine believe
it is compiling the file \textit{target}
whose content is specified as the latter parameter.
The provided code then forwards the processing to
\textit{main} or \textit{dest} as described in \secref{sec:forward}.

%%%%%%%%%%%%%%%%%%%%%%%%%%%%%%%%%%%%%%%%%%%%%%%%%%%%%%%%%%%%%%%%%%%%%%%%%%%%%%%%
\subsection{Include by Input}
\label{sec:input}

Including child documents by |\include| has some restrictions by design.
Most notably, the content of a child document always occupies
its own set of pages; pages cannot be shared between child documents.
Usually, this behaviour makes perfect sense
because each child document contain an essential part of the document.
However, in some situations it may be desirable to compose
a document from a collection of parts
without having mandatory page breaks between then.
For this case, the package
provides a mechanism to include parts
by |\input| which can also be processed individually.
However, by construction this mechanism
requires manual handling of the content to be output.

%%%%%%%%%%%%%%%%%%%%%%%%%%%%%%%%%%%%%%%%
\DescribeMacro{\ifchilddocmanual}
The main file should be prepared as usual, see \secref{sec:include}.
However, the document body must make a distinction
between processing of an individual part and of the main document, e.g.:
%
\begin{center}
\begin{tabular}{l}
|\ifchilddocmanual|\\
|\input{\childdocname}|\\
|\||else|\\
\textit{document body with }|\input{|\textit{part}|}|\\
|\||fi|
\end{tabular}
\end{center}
%
The conditional |\ifchilddocmanual| is true whenever
a part to be included by |\input| is being compiled,
and the name of the part is stored in |\childdocname|.

%%%%%%%%%%%%%%%%%%%%%%%%%%%%%%%%%%%%%%%%
\DescribeMacro{\childdocby}
Each part to be included by |\input| should start with:
%
\begin{center}
\begin{tabular}{l}
|\input{childdoc.def}|\\
|\childdocby{|\textit{main}|}|\\
\end{tabular}
\end{center}
%
The directive |\childdocby| is similar to |\childdocof|
described in \secref{sec:include},
but the subsequent selection of content must be done manually.
To that end, both |\ifchilddoc| and |\ifchilddocmanual|
will be true upon processing of a part,
and the name of the part is stored in |\childdocname|.
Note that |\jobname| will be set to the filename of the current part
so that each part receives an individual |.aux| file
that does not interfere with the |.aux| file(s) of the main document.
This behaviour can be altered by the alternative form
|\childdocby[*]{|\textit{main}|}| (with a non-empty optional argument)
which uses the |.aux| file of the main document
by setting |\jobname| to \textit{main}.

%%%%%%%%%%%%%%%%%%%%%%%%%%%%%%%%%%%%%%%%%%%%%%%%%%%%%%%%%%%%%%%%%%%%%%%%%%%%%%%%
\subsection{Driver Development}
\label{sec:driver}

The \textsf{childdoc} mechanism can also be use for the development
of definition files such as \LaTeX{} styles or classes.
This case differs from the above setup with multiple parts
included by |\include| in that no |\includeonly| should be invoked.
This can be achieved by starting the include file
(before |\ProvidesPackage|) with:
%
\begin{center}
\begin{tabular}{l}
|\input{childdoc.def}|\\
|\childdocforward{|\textit{main}|}|\\
\end{tabular}
\end{center}
%
or alternatively with:
%
\begin{center}
\begin{tabular}{l}
|\input{childdoc.def}|\\
|\childdocby{|\textit{main}|}|\\
\end{tabular}
\end{center}
%
Both forms have slightly different effects as described above.
The main file is prepared as usual, see \secref{sec:include}.

%%%%%%%%%%%%%%%%%%%%%%%%%%%%%%%%%%%%%%%%%%%%%%%%%%%%%%%%%%%%%%%%%%%%%%%%%%%%%%%%
\subsection{Legacy Detection}
\label{sec:detection}

The directive |\childdocmain| in the main file can detect
whether the complete document or merely a child is to be compiled
even without using the directive |\childdocof|.
This method is deprecated because it is less robust
and there is no compelling reason to use it;
it is merely provided for backward compatibility
and it may be removed in future versions.

If the detection mechanism is to be used,
it is mandatory to correctly specify
the filename of the main file as the argument of |\childdocmain|:
%
\begin{center}
\begin{tabular}{l}
|\input{childdoc.def}|\\
|\childdocmain{|\textit{main}|}|\\
\end{tabular}
\end{center}
%
If |\jobname| does not match the argument \textit{main} of |\childdocmain|,
it is assumed that |\jobname| points to the child file to be compiled.
When using |\childdocmain| with the main file specified as argument,
it suffices to start a child file
with just |\input{|\textit{main}|}|
without loading of the package and using |\childdocof|.
If instead all processing is done
with the appropriate \textsf{childdoc} directives,
the argument of \textit{main} of |\childdocmain| can be empty.

An alternative version of the command line processing described
in \secref{sec:commandline} using the detection mechanism reads:
%
\begin{center}
|... -jobname "|\textit{target}|" "|[\textit{flags}]%
[|\def\jobname{|\textit{dest}|}|]|\input{|\textit{main}|}"|
\end{center}

%%%%%%%%%%%%%%%%%%%%%%%%%%%%%%%%%%%%%%%%%%%%%%%%%%%%%%%%%%%%%%%%%%%%%%%%%%%%%%%%
\subsection{Manual Code}
\label{sec:manual}

In case one cannot be certain whether the definitions file |childdoc.def|
is installed on the target \TeX{} distribution
and one prefers not to ship it,
it is conceivable to paste a few relevant commands into the sources.

To that end, drop all statements |\input{childdoc.def}|
and perform the replacements as outlined below.
Instead of |\childdocmain{|\textit{main}|}| add the following code
to the top of the main file:
%
\begin{center}
\begin{tabular}{l}
|\||ifdefined\childdocname\endinput\||fi\newif\ifchilddoc|\\
|\edef\childdocname{\scantokens\expandafter{\jobname\noexpand}}|\\
|\def\childdocmain{|\textit{main}|}\||ifx\childdocmain\childdocname\||else|\\
|\childdoctrue\includeonly{\childdocname}\let\jobname\childdocmain\||fi|\\
\end{tabular}
\end{center}
%
Instead of |\childdocof{|\textit{main}|}| just include the main file
at the top of each child file:
%
\begin{center}
|\input{|\textit{main}|}|
\end{center}
%
A simple redirection |\childdocforward{|\textit{dest}|}| is achieved by:
%
\begin{center}
|\def\jobname{|\textit{dest}|}\input{\jobname}|
\end{center}
%
The redirection with prefix
|\childdocforwardprefix[|\textit{prefix}|]{|\textit{dest}|}|
is accomplished by:
%
\begin{center}
\begin{tabular}{l}
|{\edef\jobname{\scantokens\expandafter{\jobname\noexpand}}|\\
|\def\redirectjob |\textit{prefix}|#1~~~{\gdef\jobname{|\textit{dest}|#1}}|\\
|\expandafter\redirectjob\jobname~~~}\input{\jobname}|
\end{tabular}
\end{center}

In an alternative approach,
child documents can be compiled by a specific command line
without additional code or specific definitions:
%
\begin{center}
|... -jobname "|\textit{target}|" "|[\textit{flags}]%
|\includeonly{|\textit{dest}|}\input{|\textit{main}|}"|
\end{center}
%

%%%%%%%%%%%%%%%%%%%%%%%%%%%%%%%%%%%%%%%%%%%%%%%%%%%%%%%%%%%%%%%%%%%%%%%%%%%%%%%%
%%%%%%%%%%%%%%%%%%%%%%%%%%%%%%%%%%%%%%%%%%%%%%%%%%%%%%%%%%%%%%%%%%%%%%%%%%%%%%%%
\section{Information}

%%%%%%%%%%%%%%%%%%%%%%%%%%%%%%%%%%%%%%%%%%%%%%%%%%%%%%%%%%%%%%%%%%%%%%%%%%%%%%%%
\subsection{Copyright}

Copyright \copyright{} 2017--2018 Niklas Beisert

This work may be distributed and/or modified under the
conditions of the \LaTeX{} Project Public License, either version 1.3
of this license or (at your option) any later version.
The latest version of this license is in
  \url{http://www.latex-project.org/lppl.txt}
and version 1.3 or later is part of all distributions of \LaTeX{}
version 2005/12/01 or later.

This work has the LPPL maintenance status `maintained'.

The Current Maintainer of this work is Niklas Beisert.

This work consists of the files |README.txt|, |childdoc.ins| and |childdoc.dtx|
as well as the derived files |childdoc.def|, |cdocsamp.tex|
with |cdocsch1.tex|, |cdocsch2.tex|, |cdocspt3.tex|, |cdocspt4.tex|,
|cdocsdrf.tex|, |cdocsfn1.tex|, |cdocsfn2.tex|
as well as |childdoc.pdf|.

%%%%%%%%%%%%%%%%%%%%%%%%%%%%%%%%%%%%%%%%%%%%%%%%%%%%%%%%%%%%%%%%%%%%%%%%%%%%%%%%
\subsection{Files and Installation}

The package consists of the files:
%
\begin{center}
\begin{tabular}{ll}
    |README.txt|   & readme file \\
    |childdoc.ins| & installation file \\
    |childdoc.dtx| & source file \\
    |childdoc.def| & definition file \\
    |cdocsamp.tex| & sample main file \\
    |cdocsch1.tex| & sample include file \\
    |cdocsch2.tex| & sample include file \\
    |cdocspt3.tex| & sample part file \\
    |cdocspt4.tex| & sample part file \\
    |cdocsdrf.tex| & sample redirection file \\
    |cdocsfn1.tex| & sample redirection file \\
    |cdocsfn2.tex| & sample redirection file \\
    |childdoc.pdf| & manual
\end{tabular}
\end{center}
%
The distribution consists of the files
|README.txt|, |childdoc.ins| and |childdoc.dtx|.
%
\begin{itemize}
\item
Run (pdf)\LaTeX{} on |childdoc.dtx|
to compile the manual |childdoc.pdf| (this file).
\item
Run \LaTeX{} on |childdoc.ins| to create the definitions file |childdoc.def|
and the sample |cdocsamp.tex| with include files
|cdocsch1.tex|, |cdocsch2.tex|, |cdocspt3.tex|, |cdocspt4.tex|,
|cdocsdrf.tex|, |cdocsfn1.tex|, |cdocsfn2.tex|.
Then copy the file |childdoc.def| to an appropriate directory of your \LaTeX{}
distribution, e.g.\ \textit{texmf-root}|/tex/latex/childdoc|.
\end{itemize}

%%%%%%%%%%%%%%%%%%%%%%%%%%%%%%%%%%%%%%%%%%%%%%%%%%%%%%%%%%%%%%%%%%%%%%%%%%%%%%%%
\subsection{Related CTAN Packages}

There are several other packages which offer a similar functionality:
%
\begin{itemize}
\item
The packages
\href{http://ctan.org/pkg/docmute}{\textsf{docmute}},
\href{http://ctan.org/pkg/includex}{\textsf{includex}} and
\href{http://ctan.org/pkg/standalone}{\textsf{standalone}}
provide commands to include only the document body of
a child file thus allowing both files to be compiled individually.
\item
The packages \href{http://ctan.org/pkg/subdocs}{\textsf{subdocs}}
and \href{http://ctan.org/pkg/subfiles}{\textsf{subfiles}}
provide structures in which the main and child documents can be
encapsulated and allowing them to be compiled individually.
The inclusion mechanism is different from the conventional |\include|.
\item
The package \href{http://ctan.org/pkg/combine}{\textsf{combine}}
is an elaborate solution to combine several documents into one.
\end{itemize}
%
See also the CTAN topic \href{http://ctan.org/topic/subdocs}{\textsf{subdocs}}
for further related packages.
The present package differs from the above solutions in that
a document structure constructed with the conventional |\include| mechanism
just needs two extra commands at the top of every file
such that all constituent files can be compiled individually.

%%%%%%%%%%%%%%%%%%%%%%%%%%%%%%%%%%%%%%%%%%%%%%%%%%%%%%%%%%%%%%%%%%%%%%%%%%%%%%%%
%\subsection{Feature Suggestions}
%
%The following is a list of features which may be useful for future
%versions of this package:
%%
%\begin{itemize}
%\item
%\ldots
%\end{itemize}

%%%%%%%%%%%%%%%%%%%%%%%%%%%%%%%%%%%%%%%%%%%%%%%%%%%%%%%%%%%%%%%%%%%%%%%%%%%%%%%%
\subsection{Revision History}

%%%%%%%%%%%%%%%%%%%%%%%%%%%%%%%%%%%%%%%%
\paragraph{v2.0:} 2018/12/30

\begin{itemize}
\item
immediate forward processing
\item
added |\childdocby| mechanism
\item
manual restructured
\end{itemize}

%%%%%%%%%%%%%%%%%%%%%%%%%%%%%%%%%%%%%%%%
\paragraph{v1.6:} 2018/01/17

\begin{itemize}
\item
application for development of include files
\item
corrections to manual
\end{itemize}

%%%%%%%%%%%%%%%%%%%%%%%%%%%%%%%%%%%%%%%%
\paragraph{v1.5:} 2017/05/21

\begin{itemize}
\item
more complete structuring introduced
\item
|\childdocof| introduced
\item
|\childdoc| renamed to |\childdocmain|
\item
|\childredirect| renamed to |\childdocforward| and |\childdocforwardprefix|
and functionality expanded
\end{itemize}

%%%%%%%%%%%%%%%%%%%%%%%%%%%%%%%%%%%%%%%%
\paragraph{v1.0:} 2017/04/27

\begin{itemize}
\item
manual and install package
\item
first version published on CTAN
\end{itemize}

%%%%%%%%%%%%%%%%%%%%%%%%%%%%%%%%%%%%%%%%
\paragraph{v0.6:} 2017/04/26

\begin{itemize}
\item
redirection mechanism added
\end{itemize}

%%%%%%%%%%%%%%%%%%%%%%%%%%%%%%%%%%%%%%%%
\paragraph{v0.5:} 2017/04/26

\begin{itemize}
\item
functionality in definition file
\end{itemize}


%%%%%%%%%%%%%%%%%%%%%%%%%%%%%%%%%%%%%%%%%%%%%%%%%%%%%%%%%%%%%%%%%%%%%%%%%%%%%%%%
%%%%%%%%%%%%%%%%%%%%%%%%%%%%%%%%%%%%%%%%%%%%%%%%%%%%%%%%%%%%%%%%%%%%%%%%%%%%%%%%
%%%%%%%%%%%%%%%%%%%%%%%%%%%%%%%%%%%%%%%%%%%%%%%%%%%%%%%%%%%%%%%%%%%%%%%%%%%%%%%%
\appendix

\settowidth\MacroIndent{\rmfamily\scriptsize 000\ }

 \DocInput{childdoc.dtx}

\end{document}
%</driver>
% \fi
%
% %%%%%%%%%%%%%%%%%%%%%%%%%%%%%%%%%%%%%%%%%%%%%%%%%%%%%%%%%%%%%%%%%%%%%%%%%%%%%%
% %%%%%%%%%%%%%%%%%%%%%%%%%%%%%%%%%%%%%%%%%%%%%%%%%%%%%%%%%%%%%%%%%%%%%%%%%%%%%%
% \section{Sample}
%\iffalse
%<*samplemain>
%\fi
%
% The following presents a sample document
% with two chapters, two parts, a title page,
% a compile flag as well as three forwarding files to set the flag.
% It consists of eight |.tex| files:
% \begin{center}
% \begin{tabular}{ll}
% |cdocsamp.tex|&main file\\
% |cdocsch1.tex|&include file for chapter 1\\
% |cdocsch2.tex|&include file for chapter 2\\
% |cdocspt3.tex|&include file for part 3\\
% |cdocspt4.tex|&include file for part 4\\
% |cdocsdrf.tex|&forwarding file for main file in draft mode\\
% |cdocsfi1.tex|&forwarding file for final version of chapter 1\\
% |cdocsfi2.tex|&forwarding file for final version of chapter 2\\
% \end{tabular}
% \end{center}
% Each of the eight files can be compiled directly by the \LaTeX{} compiler.
%
% %%%%%%%%%%%%%%%%%%%%%%%%%%%%%%%%%%%%%%
% \paragraph{Main File.}
%
% The main file is called |cdocsamp.tex|.
%
% Load the \textsf{childdoc} definitions and
% declare the filename for the main document:
%    \begin{macrocode}
\input{childdoc.def}
\childdocmain{}
%    \end{macrocode}

% Optional override for |\version| flag:
%    \begin{macrocode}
%%\ifchilddoc\else\providecommand{\version}{draft}\fi
%    \end{macrocode}

% Define the default values for the |\version| flag
% (|final| for the main file and |draft| for childs):
%    \begin{macrocode}
\ifchilddoc
\providecommand{\version}{draft}
\else
\providecommand{\version}{final}
\fi
%    \end{macrocode}

% Load the standard document class:
%    \begin{macrocode}
\documentclass[12pt]{article}
%    \end{macrocode}

% Start the document body:
%    \begin{macrocode}
\begin{document}
%    \end{macrocode}

% Declare a title page.
% Print title, part of document being processed and version flag:
%    \begin{macrocode}
\addtocounter{page}{-1}
\begin{center}
{\LARGE\bfseries{}childdoc example\par}
\vspace{1cm}
\ifchilddoc
\ifchilddocmanual part\else chapter\fi:
`\childdocname' of `\childdocjob'\par
\else
main document: `\childdocjob'\par
\fi
version: \version\par
\end{center}
\newpage
%    \end{macrocode}

% Manually include selected file,
% otherwise process as usual:
%    \begin{macrocode}
\ifchilddocmanual
\section*{part `\childdocname'}
\input{\childdocname}
\else
%    \end{macrocode}

% Include the two chapters:
%    \begin{macrocode}
\include{cdocsch1}
\include{cdocsch2}
%    \end{macrocode}

% Include the two parts unless only chapters should be displayed:
%    \begin{macrocode}
\ifchilddoc\else
\section{part three}
\input{cdocspt3}
\section{part four}
\input{cdocspt4}
\fi
%    \end{macrocode}

% Process as usual until here:
%    \begin{macrocode}
\fi
%    \end{macrocode}

% End of document body:
%    \begin{macrocode}
\end{document}
%    \end{macrocode}
%\iffalse
%</samplemain>
%\fi
%
% %%%%%%%%%%%%%%%%%%%%%%%%%%%%%%%%%%%%%%
% \paragraph{Chapter Include Files.}
%
% The include files are called |cdocsch1.tex| and |cdocsch2.tex|.
%
%\iffalse
%<*samplechap1|samplechap2>
%\fi

% Optional override for |\version| flag:
%    \begin{macrocode}
%%\providecommand{\version}{final}
%    \end{macrocode}

% Include the main document:
%    \begin{macrocode}
\input{childdoc.def}
\childdocof{cdocsamp}
%    \end{macrocode}

%\iffalse
%</samplechap1|samplechap2>
%\fi
%
%\iffalse
%<*samplechap1>
%\fi
% Some text for chapter 1:
%    \begin{macrocode}
\section{one}
some text in chapter one
%    \end{macrocode}

%\iffalse
%</samplechap1>
%\fi
% Some text for chapter 2:
%\iffalse
%<*samplechap2>
%\fi
%    \begin{macrocode}
\section{two}
more text in chapter two
%    \end{macrocode}

%\iffalse
%</samplechap2>
%\fi
%
% %%%%%%%%%%%%%%%%%%%%%%%%%%%%%%%%%%%%%%
% \paragraph{Part Include Files.}
%
% The include files are called |cdocspt3.tex| and |cdocspt4.tex|.
%
%\iffalse
%<*samplepart3|samplepart4>
%\fi

% Optional override for |\version| flag:
%    \begin{macrocode}
%%\providecommand{\version}{final}
%    \end{macrocode}

% Include the main document:
%    \begin{macrocode}
\input{childdoc.def}
\childdocby{cdocsamp}
%    \end{macrocode}

%\iffalse
%</samplepart3|samplepart4>
%\fi
%
%\iffalse
%<*samplepart3>
%\fi
% Some text for part 3:
%    \begin{macrocode}
some text in part three
%    \end{macrocode}

%\iffalse
%</samplepart3>
%\fi
% Some text for part 4:
%\iffalse
%<*samplepart4>
%\fi
%    \begin{macrocode}
more text in part four
%    \end{macrocode}

%\iffalse
%</samplepart4>
%\fi
%
% %%%%%%%%%%%%%%%%%%%%%%%%%%%%%%%%%%%%%%
% \paragraph{Forwarding for a Complete Draft.}
%
% The following forwarding file |cdocsdrf.tex|
% compiles the main document in draft mode:
%\iffalse
%<*sampledraft>
%\fi
%    \begin{macrocode}
\def\version{draft}
\input{childdoc.def}
\childdocforward{cdocsamp}
%    \end{macrocode}

%\iffalse
%</sampledraft>
%\fi
%
% %%%%%%%%%%%%%%%%%%%%%%%%%%%%%%%%%%%%%%
% \paragraph{Forwarding for Final Version of the Chapters.}
%
% The following forwarding files |cdocsfn1.tex| and |cdocsfn2.tex|
% (with identical content)
% compile the final versions of the child documents
% |cdocsch1.tex| and |cdocsch2.tex|, respectively:
%\iffalse
%<*samplefinal>
%\fi
%    \begin{macrocode}
\def\version{final}
\input{childdoc.def}
\childdocforwardprefix[cdocsamp]{cdocsfn}{cdocsch}
%    \end{macrocode}

%\iffalse
%</samplefinal>
%\fi
%
% %%%%%%%%%%%%%%%%%%%%%%%%%%%%%%%%%%%%%%
% \paragraph{Command Line Processing.}
%
% The following three command lines generate the output files
% |cdocscld|, |cdocscl1| and |cdocscl2|
% which should be identical to
% |cdocsdrf|, |cdocsch1| and |cdocsfn2|, respectively:
% \begin{center}
% \begin{tabular}{l}
% |latex -jobname cdocscld \|\\
% |  "\def\version{draft}\input{childdoc.def}\childdocforward{cdocsamp}"|\\
% |latex -jobname cdocscl1 \|\\
% |  "\input{childdoc.def}\childdocforward[cdocsamp]{cdocsch1}"|\\
% |latex -jobname cdocscl2 \|\\
% |  "\def\version{final}\input{childdoc.def}\childdocforward{cdocsch2}"|
% \end{tabular}
% \end{center}
% Note that the trailing backslash on each first line
% merely continues the input to the second line
% (for convenient cut ant paste).
% Furthermore, the command |latex| can be replaced by any
% of its alternative versions such as |pdflatex|.
%
% %%%%%%%%%%%%%%%%%%%%%%%%%%%%%%%%%%%%%%%%%%%%%%%%%%%%%%%%%%%%%%%%%%%%%%%%%%%%%%
% %%%%%%%%%%%%%%%%%%%%%%%%%%%%%%%%%%%%%%%%%%%%%%%%%%%%%%%%%%%%%%%%%%%%%%%%%%%%%%
% \section{Implementation}
%\iffalse
%<*package>
%\fi
%
% This section describes the definitions file |childdoc.def|.

% The definitions cannot be loaded using |\usepackage| or |\RequirePackage|
% which has a mechanism to prevent loading a style file more than once.
% When loading the definitions by means of |\input|
% multiple instances have to be prevented manually:
%\iffalse
%This code needs to be before the `\ProvidesFile' directive
%which is defined at the beginning of this file.
%Therefore it is also placed there and commented out here.
%</package>
%<*discard>
%\fi
%    \begin{macrocode}
\ifdefined\childdocmain\endinput\fi
%    \end{macrocode}
%\iffalse
%</discard>
%<*package>
%\fi
%
% \macro{\ifchilddoc}
% \macro{\ifchilddocmanual}
% The conditional |\ifchilddoc| tells whether a
% child (true) or main (false) document is being compiled.
% The conditional |\ifchilddocmanual| tells whether
% the |\includeonly| mechanism is used (false) or
% the selection of child files must be performed manually (true).
% The definitions initialise to false:
%    \begin{macrocode}
\newif\ifchilddoc
\newif\ifchilddocmanual
%    \end{macrocode}

% \macro{\childdocname}
% \macro{\childdocjob}
% The macro |\childdocname| stores the name of the main document
% to be compiled. The macro |\childdocjob| stores the name of
% the document on which the \LaTeX{} compiler was originally invoked.
% The content of |\jobname| cannot be compared
% to filenames specified in the source due to different catcodes.
% The following code rescans |\jobname|, stores the result
% in |\childdocname| and saves a copy in |\childdocjob|:
%    \begin{macrocode}
\edef\childdocname{\scantokens\expandafter{\jobname\noexpand}}
\let\childdocjob\childdocname
%    \end{macrocode}

% \macro{\childdocdisable}
% The macro |\childdocdisable| prevents the main file
% from being processed more than once.
% At this stage, the main document command |\childdocmain|
% is assumed to be called once again where it should do nothing.
% Any subsequent call to it should prevent
% a secondary processing of the main document
% It overwrites the forwarding commands
% |\childdocof| and |\childdocforward|
% with empty macros to prevent further inclusions of the main document:
%    \begin{macrocode}
\newcommand{\childdocdisable}
{
  \renewcommand{\childdocmain}[1]{\renewcommand{\childdocmain}[1]{\endinput}}
  \renewcommand{\childdocof}[1]{}
  \renewcommand{\childdocby}[2][]{}
  \renewcommand{\childdocforward}[2][]{}
  \renewcommand{\childdocdisable}{}
}
%    \end{macrocode}

% \macro{\childdocmain}
% The macro |\childdocmain| is to be called at the top of the main file
% with nothing or the main filename (without extension) as argument.
% First, it breaks loops.
% If the argument is not empty and does not match |\childdocname|
% (which is set by the first inclusion of |childdoc.def|),
% |\ifchilddoc| is set to true, |\includeonly| is applied to the child file
% and |\jobname| is set to the main file
% (for proper handling of |.aux| files):
%    \begin{macrocode}
\newcommand{\childdocmain}[1]
{
  \childdocdisable\childdocmain{}
  \if?#1?\else
    \begingroup
      \def\childdoctmp{#1}
      \ifx\childdoctmp\childdocname
        \def\childdoctmp{}
      \else
        \def\childdoctmp
        {
          \childdoctrue
          \includeonly{\childdocname}
          \def\childdocjob{#1}
          \def\jobname{#1}
        }
      \fi
      \expandafter
    \endgroup
    \childdoctmp
  \fi
}
%    \end{macrocode}

% \macro{\childdocof}
% The command |\childdocof| redirects
% compilation to the main file |#1|.
%    \begin{macrocode}
\newcommand{\childdocof}[1]
{
  \childdocdisable
  \childdoctrue
  \includeonly{\childdocname}
  \def\jobname{#1}
  \def\childdocjob{#1}
  \input{#1}
}
%    \end{macrocode}

% \macro{\childdocby}
% The command |\childdocby| ....
%    \begin{macrocode}
\newcommand{\childdocby}[2][]
{
  \childdocdisable
  \childdoctrue
  \childdocmanualtrue
  \if?#1?\else
    \def\jobname{#2}
  \fi
  \def\childdocjob{#2}
  \input{#2}
  \endinput
}
%    \end{macrocode}

% \macro{\childdocforward}
% The command |\childdocforward| redirects
% compilation to the main file or
% (if the optional argument is given) a child file.
% Parameters are set as if the main file
% or a child file starting with |\childdocof| was compiled.
% Then compilation is handed over to the main file:
%    \begin{macrocode}
\newcommand{\childdocforward}[2][]
{
  \begingroup
    \if?#1?
      \def\childdoctmp
      {
        \def\childdocname{#2}
        \def\childdocjob{#2}
        \def\jobname{#2}
        \input{#2}
        \endinput
      }
    \else
      \def\childdoctmp
      {
        \childdocdisable
        \def\childdocname{#2}
        \childdoctrue
        \includeonly{#2}
        \def\childdocjob{#1}
        \def\jobname{#1}
        \input{#1}
        \endinput
      }
    \fi
    \expandafter
  \endgroup
  \childdoctmp
}
%    \end{macrocode}

% \macro{\childdocforwardprefix}
% The command |\childdocforwardprefix| redirects
% compilation to the main or a child file by means of a pattern.
% The prefix |#1| in the current filename is replaced by |#2|
% and the suffix of the current filename is kept
% (it is assumed that the filename does not contain the substring `|~~~|'
% which is used as a delimiter).
% Compilation is handed over to the new file by |\childdocforward|:
%    \begin{macrocode}
\newcommand{\childdocforwardprefix}[3][]
{
  \begingroup
    \def\childdocextract #2##1~~~{\def\childdoctmp{\childdocforward[#1]{#3##1}}}
    \expandafter\childdocextract\childdocname~~~
    \expandafter
  \endgroup
  \childdoctmp
}
%    \end{macrocode}

% \macro{\childdoc}
% The deprecated macro |\childdoc| is a legacy version of |\childdocmain|:
%    \begin{macrocode}
\newcommand{\childdoc}{\childdocmain}
%    \end{macrocode}

% \macro{\childdocredirect}
% The deprecated macro |\childdocredirect| is a legacy version
% of |\childdocforward| and |\childdocforwardprefix|:
%    \begin{macrocode}
\newcommand{\childdocredirect}[2][]
{
  \begingroup
    \if?#1?
      \def\childdoctmp{\childdocforward{#2}}
    \else
      \def\childdoctmp{\childdocforwardprefix{#1}{#2}}
    \fi
    \expandafter
  \endgroup
  \childdoctmp
}
%    \end{macrocode}

%\iffalse
%</package>
%\fi
%
\endinput
|\\
|\childdocforward{|\textit{main}|}|\\
\end{tabular}
\end{center}
%
or alternatively with:
%
\begin{center}
\begin{tabular}{l}
|% \iffalse
%
% childdoc.dtx Copyright (C) 2017-2018 Niklas Beisert
%
% This work may be distributed and/or modified under the
% conditions of the LaTeX Project Public License, either version 1.3
% of this license or (at your option) any later version.
% The latest version of this license is in
%   http://www.latex-project.org/lppl.txt
% and version 1.3 or later is part of all distributions of LaTeX
% version 2005/12/01 or later.
%
% This work has the LPPL maintenance status `maintained'.
%
% The Current Maintainer of this work is Niklas Beisert.
%
% This work consists of the files childdoc.dtx and childdoc.ins
% and the derived files childdoc.def and cdocsamp.tex with
% cdocsch1.tex, cdocsch2.tex, cdocsdrf.tex, cdocsfn1.tex, cdocsfn2.tex.
%
%<package>\ifdefined\childdocmain\endinput\fi
%<package>\ProvidesFile{childdoc.def}[2018/12/30 v2.0 child document driver]
%<samplemain>\ProvidesFile{cdocsamp.tex}[2018/12/30 v2.0 sample for childdoc]
%<*driver>
%\ProvidesFile{childdoc.drv}[2018/12/30 v2.0 childdoc reference manual file]
\PassOptionsToClass{10pt,a4paper}{article}
\documentclass{ltxdoc}

\usepackage[margin=35mm]{geometry}
\usepackage{hyperref}
\usepackage{hyperxmp}
\usepackage[usenames]{color}

\hypersetup{colorlinks=true}
\hypersetup{pdfstartview=FitH}
\hypersetup{pdfpagemode=UseNone}
\hypersetup{pdfsource={}}
\hypersetup{pdflang={en-UK}}
\hypersetup{pdfcopyright={Copyright 2017-2018 Niklas Beisert.
  This work may be distributed and/or modified under the
  conditions of the LaTeX Project Public License, either version 1.3
  of this license or (at your option) any later version.}}
\hypersetup{pdflicenseurl={http://www.latex-project.org/lppl.txt}}
\hypersetup{pdfcontactaddress={ETH Zurich, ITP, HIT K,
  Wolfgang-Pauli-Strasse 27}}
\hypersetup{pdfcontactpostcode={8093}}
\hypersetup{pdfcontactcity={Zurich}}
\hypersetup{pdfcontactcountry={Switzerland}}
\hypersetup{pdfcontactemail={nbeisert@itp.phys.ethz.ch}}
\hypersetup{pdfcontacturl={http://people.phys.ethz.ch/\xmptilde nbeisert/}}

\newcommand{\secref}[1]{\hyperref[#1]{section \ref*{#1}}}

\parskip1ex
\parindent0pt
\let\olditemize\itemize
\def\itemize{\olditemize\parskip0pt}

\begin{document}

\title{The \textsf{childdoc} Package}
\hypersetup{pdftitle={The childdoc Package}}
\author{Niklas Beisert\\[2ex]
  Institut f\"ur Theoretische Physik\\
  Eidgen\"ossische Technische Hochschule Z\"urich\\
  Wolfgang-Pauli-Strasse 27, 8093 Z\"urich, Switzerland\\[1ex]
  \href{mailto:nbeisert@itp.phys.ethz.ch}
  {\texttt{nbeisert@itp.phys.ethz.ch}}}
\hypersetup{pdfauthor={Niklas Beisert}}
\hypersetup{pdfsubject={Manual for the LaTeX2e Package childdoc}}
\date{30 December 2018, \textsf{v2.0}}
\maketitle

\begin{abstract}\noindent
\textsf{childdoc} is a \LaTeXe{} package
that enables the direct compilation
of document sections included by |\include|
to individual files.
\end{abstract}

\begingroup
\parskip0ex
\tableofcontents
\endgroup

%%%%%%%%%%%%%%%%%%%%%%%%%%%%%%%%%%%%%%%%%%%%%%%%%%%%%%%%%%%%%%%%%%%%%%%%%%%%%%%%
%%%%%%%%%%%%%%%%%%%%%%%%%%%%%%%%%%%%%%%%%%%%%%%%%%%%%%%%%%%%%%%%%%%%%%%%%%%%%%%%
\section{Introduction}

\LaTeX{} provides a mechanism to structure a large document (such as a book)
into a main file and several child files (containing the chapters)
using the |\include| command.
This mechanism is beneficial for documents
which span hundreds of pages in order to
make the source file(s) more manageable.
Moreover, compilation can be restricted to
selected child files by means of the |\includeonly| command.
The latter feature can be used to reduce the compilation time while editing
(this was significantly more useful in the earlier days of \LaTeX{})
or to generate a smaller document which is easier to navigate.
Another application of |\includeonly| is to generate
documents consisting of selected parts of the complete document.

However, there are a few drawbacks of the plain |\include| mechanism:
\begin{itemize}
\item
The child files cannot be compiled on their own,
they can only be compiled via the main file.
A naive editing environment
(such as a text editor with an option
to have the current file processed by \LaTeX)
may require one to switch to the main file before compiling;
attempting to compile the child file produces errors.
\item
The main file must be modified (each time)
to adjust the |\includeonly| command
to the present needs. This easily leaves the main file in a messy state.
\item
The generated document will always carry the filename
of the main document. This is inconvenient if
several child files are to be compiled and
to be kept for distribution.
\end{itemize}

The present package provides a simple interface
to make child files individually compilable by \LaTeX{}.
Compiling a child file then has the same effect as compiling
the main file with an |\includeonly| command
to select the appropriate child.
Moreover the generated document will carry the name of the child
rather than the main file.
This resolves all three above issues.

This feature is meant to make the editing of books,
thesis documents and lecture notes somewhat more convenient.
However, the package can also be used efficiently for
composing a series of documents (such as exercise sheets)
which are typically distributed individually.
It then assists the author in generating the individual documents
(potentially in different versions)
as well as a document containing the collected series.
Another application is in developing style files
or other kinds of included material
where compilation of the style file could redirect
to a sample or test file.

%%%%%%%%%%%%%%%%%%%%%%%%%%%%%%%%%%%%%%%%%%%%%%%%%%%%%%%%%%%%%%%%%%%%%%%%%%%%%%%%
%%%%%%%%%%%%%%%%%%%%%%%%%%%%%%%%%%%%%%%%%%%%%%%%%%%%%%%%%%%%%%%%%%%%%%%%%%%%%%%%
\section{Usage}

First of all, the package \textsf{childdoc} is \emph{not} a standard
\LaTeXe{} |.sty| style file! Therefore it needs to be invoked in
a non-standard way.

%%%%%%%%%%%%%%%%%%%%%%%%%%%%%%%%%%%%%%%%%%%%%%%%%%%%%%%%%%%%%%%%%%%%%%%%%%%%%%%%
\subsection{Included Files}
\label{sec:include}

%%%%%%%%%%%%%%%%%%%%%%%%%%%%%%%%%%%%%%%%
\DescribeMacro{\childdocmain}
To use the package, add the commands
\begin{center}
\begin{tabular}{l}
|\input{childdoc.def}|\\
|\childdocmain{}|\\
\end{tabular}
\end{center}
at the very top of the main \LaTeX{} file,
in particular \emph{before} the |\documentclass| statement!
The argument of |\childdocmain| should be left empty
(but it must be present).

%%%%%%%%%%%%%%%%%%%%%%%%%%%%%%%%%%%%%%%%
\DescribeMacro{\childdocof}
Furthermore, add the commands
\begin{center}
\begin{tabular}{l}
|\input{childdoc.def}|\\
|\childdocof{|\textit{main}|}|\\
\end{tabular}
\end{center}
at the top of every child file \textit{child}
which is included by |\include{|\textit{child}|}|
from within the main file
(or at least for those files to be compiled individually).
The argument \textit{main} must be the filename of the main file.

There are a couple of
considerations in setting up the main and child documents:

%%%%%%%%%%%%%%%%%%%%%%%%%%%%%%%%%%%%%%%%
\paragraph{Restrictions.}

Please note the following restrictions:
\begin{itemize}
\item
|\childdocmain| must be called with one argument \textit{main}
to ensure compatibility with earlier version of the package.
It must either be empty (|\childdocmain{}|)
or precisely match the filename of the main file in which it is specified.
See \secref{sec:detection} for further information.
\item
The filename \textit{main} must be specified without the |.tex| extension.
\item
The filename \textit{main} is case sensitive
(even in case-insensitive file systems)
due to internal string comparison.
\item
The argument \textit{main} should be fully expanded, it cannot be a macro.
\item
Subdirectories and special characters should be avoided in filenames.
\item
The command |\childdocmain{|\textit{main}|}| must be followed by a whitespace.
It should not be followed immediately by another command
or by a comment mark `|%|'.
This is because the \TeX{} parser reads the token immediately following
the argument of |\childdocmain| and puts it
at the beginning of every child section;
however, a white\-space is ignored.
\end{itemize}

%%%%%%%%%%%%%%%%%%%%%%%%%%%%%%%%%%%%%%%%
\paragraph{Content of Main File.}

It is advisable to place all content in the child files included by |\include|.
Any output contained in the main file will appear in all child documents
unless suppressed manually;
it cannot be suppressed automatically by the |\includeonly| directive
and thus should normally be avoided.
A method to include some content in the main file
by means of conditional processing is described in \secref{sec:conditional}.

%%%%%%%%%%%%%%%%%%%%%%%%%%%%%%%%%%%%%%%%
\paragraph{Page Numbering.}

When only a part of the document is compiled,
the appropriate numbering of pages
(as well as other status parameters)
is determined from the |.aux| files.
The latter contain information from previous passes.
However this information needs to propagate through
all intermediate child documents.
Therefore the page numbering in child documents may well
be inconsistent until the complete document is compiled at least once.

A useful (if unconventional) way to always ensure a consistent
page numbering is to restart the numbering in each child document
and denote the pages by `\textit{child}|.|\textit{page}'
where \textit{child} represents the chapter/section number of the child file.
This can be achieved by the command
|\numberwithin{page}{|\textit{child}|}|
of the \textsf{amsmath} package
where \textit{child} can be |chapter| or |section|
depending on the chosen structuring.
Alternatively, one can modify the macro |\thepage| appropriately
and reset the counter |page| at the start of each child file.

%%%%%%%%%%%%%%%%%%%%%%%%%%%%%%%%%%%%%%%%%%%%%%%%%%%%%%%%%%%%%%%%%%%%%%%%%%%%%%%%
\subsection{Conditional Processing}
\label{sec:conditional}

The package provides a mechanism to compile different versions
of a document. To customise the versions further some conditional processing
can come in handy to distinguish which version is being compiled.
The package provides two macros to describe the compilation context:

%%%%%%%%%%%%%%%%%%%%%%%%%%%%%%%%%%%%%%%%
\DescribeMacro{\ifchilddoc}
The conditional |\ifchilddoc| distinguishes between the compilation of
child documents and the main document:
%
\begin{center}
|\ifchilddoc |\textit{child-code}| |[|\||else |\textit{main-code}]| \||fi|
\end{center}

%%%%%%%%%%%%%%%%%%%%%%%%%%%%%%%%%%%%%%%%
\DescribeMacro{\childdocname}
\DescribeMacro{\childdocjob}
The macro |\childdocname| contains the filename (without extension)
of the main or child file being processed.
Note that |\childdocjob| will always contain the name of the main file.

%%%%%%%%%%%%%%%%%%%%%%%%%%%%%%%%%%%%%%%%
\paragraph{Title Page.}

Conditional processing can be used to include a title or banner page
in the main document when proper precautions are taken.
Importantly, the code in the main file should ensure that the page counter
(as well as other status parameters which are stored in the |.aux| files)
takes the same value after the conditional processing.
Otherwise the page numbers may take divergent values
depending on which part is compiled.

For example, a title page could be declared by:
%
\begin{center}
\begin{tabular}{l}
|\ifchilddoc\||else|\\
|\addtocounter{page}{-1}|\\
\textit{code for title page}\\
|\newpage|\\
|\||fi|
\end{tabular}
\end{center}
%
A banner page for the child documents can be generated by:
%
\begin{center}
\begin{tabular}{l}
|\ifchilddoc|\\
|\addtocounter{page}{-1}|\\
\textit{code for banner page}\\
|\newpage|\\
|\||fi|
\end{tabular}
\end{center}
%
Here one could write a message such as:
\begin{center}
|This is the part \childdocname{} of \childdocjob{}.|
\end{center}

%%%%%%%%%%%%%%%%%%%%%%%%%%%%%%%%%%%%%%%%%%%%%%%%%%%%%%%%%%%%%%%%%%%%%%%%%%%%%%%%
\subsection{Flags}
\label{sec:flags}

The package makes it easy to generate different versions
of the main or child documents.
To this end compilation flags can be defined
and assigned different default values.
They will be particularly useful in conjunction
with the forwarding mechanism described in \secref{sec:forward}.

For example, it may be useful to have a flag |\version|
which can be set to |draft| or |final|.
The document source will contain some conditional code
depending on the value of |\version|.
Suppose further, the flag should default to |final| for the main file
and to |draft| for child files
which is a natural assignment for editing the document.
This is achieved by placing the following code
in the preamble of the main document
(below the |\childdocmain| directive):
%
\begin{center}
\begin{tabular}{l}
|\ifchilddoc|\\
|\providecommand{\version}{draft}|\\
|\||else|\\
|\providecommand{\version}{final}|\\
|\||fi|
\end{tabular}
\end{center}
%
The definition by |\providecommand| makes sure
that previous definitions are not overwritten.
Further statements |\providecommand{\version}{...}|
can thus be added before the above code to override it.

For the main file, one might add a line
(between |\childdocmain| and the above block)
%
\begin{center}
|%\ifchilddoc\||else\providecommand{\version}{draft}\||fi|
\end{center}
%
which can be uncommented to produce a draft version.
Likewise one can add a line to the very top of a child file
(above the |\childdocof{|\textit{main}|}| directive)
%
\begin{center}
|%\providecommand{\version}{final}|
\end{center}
%
which can be uncommented to produce the final version of this child document.

%%%%%%%%%%%%%%%%%%%%%%%%%%%%%%%%%%%%%%%%%%%%%%%%%%%%%%%%%%%%%%%%%%%%%%%%%%%%%%%%
\subsection{Forwarding}
\label{sec:forward}

Different versions of the main or child documents
using compilation flags as described in \secref{sec:flags}
can be (permanently) stored in different files
for convenient compilation, viewing and distribution.
To this end, the package defines a command
to pass on compilation to a different file:

%%%%%%%%%%%%%%%%%%%%%%%%%%%%%%%%%%%%%%%%
\DescribeMacro{\childdocforward}
The command |\childdocforward| redirects processing to
another source file:
%
\begin{center}
\begin{tabular}{l}
|\input{childdoc.def}|\\
|\childdocforward[|\textit{main}|]{|\textit{dest}|}|\\
\end{tabular}
\end{center}
%
The argument \textit{dest} is the destination file
(without extension).
It should be the main file or one of the child files.
Note that further \textsf{childdoc} directives
such as |\childdocof| and |\childdocforward|
in the indicated file will be processed in this form.
The optional argument \textit{main}
passes on directly to the main file \textit{main}
while pretending to compile the child \textit{dest}.
This form behaves as if \textit{dest}
issues |\childdocof{|\textit{main}|}| right away,
and no further \textsf{childdoc} directives will be processed.

%%%%%%%%%%%%%%%%%%%%%%%%%%%%%%%%%%%%%%%%
\DescribeMacro{\...prefix}
In the alternative form |\childdocforwardprefix|,
%
\begin{center}
\begin{tabular}{l}
|\input{childdoc.def}|\\
|\childdocforwardprefix[|\textit{main}|]{|\textit{prefix}|}{|\textit{dest}|}|
\end{tabular}
\end{center}
%
the destination file is determined by a pattern
depending on the current file:
To make this work, the current file must be called
`{\textit{prefix}\hspace{0.2em}\textit{suffix}}'
with \textit{prefix} matching precisely the argument.
Processing is then passed on to the file
`{\textit{dest}\hspace{0.2em}\textit{suffix}}'.
Surely, the same effect is achieved by
directly specifying the
argument `{\textit{dest}\hspace{0.2em}\textit{suffix}}'
in the first form.
However, that requires to set up a different file
for each child. With the alternative form of the command
all these files can have exactly the same content
which simplifies setting them up and maintaining them.

For example, the following file |draft.tex|
with a compilation flag |\version| as described in \secref{sec:flags}
compiles the main document as a draft:
%
\begin{center}
\begin{tabular}{l}
|\def\version{draft}|\\
|\input{childdoc.def}|\\
|\childdocforward{|\textit{main}|}|
\end{tabular}
\end{center}
%
Likewise, the following files |final|\textit{nn}|.tex|
compile the final version of the child document
|child|\textit{nn}|.tex|:
%
\begin{center}
\begin{tabular}{l}
|\def\version{final}|\\
|\input{childdoc.def}|\\
|\childdocforwardprefix{final}{child}|
\end{tabular}
\end{center}
%

Note that when several versions of a main file and/or of each child file
are to be generated, it may be convenient to set up a |Makefile| or
shell script to automatise the process.

%%%%%%%%%%%%%%%%%%%%%%%%%%%%%%%%%%%%%%%%%%%%%%%%%%%%%%%%%%%%%%%%%%%%%%%%%%%%%%%%
\subsection{Command Line Processing}
\label{sec:commandline}

The effect of redirection files can also be achieved by invoking
the \LaTeX{} compiler with a more elaborate command line.
Most conveniently this should be done as part
of a shell script or a |Makefile|.

When using \textsf{childdoc} in the main file, the following
command lines effectively perform a redirection
(note that depending on the shell being used,
backslashes may have to be doubled: `|\|' $\to$ `|\\|'):
%
\begin{center}
|... -jobname "|\textit{target}|" |\\|"|[\textit{flags}]%
|\input{childdoc.def}\childdocforward[|\textit{main}|]{|\textit{dest}|}"|
\end{center}
%
Here \textit{target} is the name of the output file,
\textit{main} is the name of the main file
and \textit{dest} is the name of the main or child file to be processed
(all filenames without extensions).
The optional argument \textit{main} can be omitted
if \textit{main} matches \textit{dest}.
Optionally, compilation \textit{flags} can be defined via |\def| commands.
This command line makes the \TeX{} engine believe
it is compiling the file \textit{target}
whose content is specified as the latter parameter.
The provided code then forwards the processing to
\textit{main} or \textit{dest} as described in \secref{sec:forward}.

%%%%%%%%%%%%%%%%%%%%%%%%%%%%%%%%%%%%%%%%%%%%%%%%%%%%%%%%%%%%%%%%%%%%%%%%%%%%%%%%
\subsection{Include by Input}
\label{sec:input}

Including child documents by |\include| has some restrictions by design.
Most notably, the content of a child document always occupies
its own set of pages; pages cannot be shared between child documents.
Usually, this behaviour makes perfect sense
because each child document contain an essential part of the document.
However, in some situations it may be desirable to compose
a document from a collection of parts
without having mandatory page breaks between then.
For this case, the package
provides a mechanism to include parts
by |\input| which can also be processed individually.
However, by construction this mechanism
requires manual handling of the content to be output.

%%%%%%%%%%%%%%%%%%%%%%%%%%%%%%%%%%%%%%%%
\DescribeMacro{\ifchilddocmanual}
The main file should be prepared as usual, see \secref{sec:include}.
However, the document body must make a distinction
between processing of an individual part and of the main document, e.g.:
%
\begin{center}
\begin{tabular}{l}
|\ifchilddocmanual|\\
|\input{\childdocname}|\\
|\||else|\\
\textit{document body with }|\input{|\textit{part}|}|\\
|\||fi|
\end{tabular}
\end{center}
%
The conditional |\ifchilddocmanual| is true whenever
a part to be included by |\input| is being compiled,
and the name of the part is stored in |\childdocname|.

%%%%%%%%%%%%%%%%%%%%%%%%%%%%%%%%%%%%%%%%
\DescribeMacro{\childdocby}
Each part to be included by |\input| should start with:
%
\begin{center}
\begin{tabular}{l}
|\input{childdoc.def}|\\
|\childdocby{|\textit{main}|}|\\
\end{tabular}
\end{center}
%
The directive |\childdocby| is similar to |\childdocof|
described in \secref{sec:include},
but the subsequent selection of content must be done manually.
To that end, both |\ifchilddoc| and |\ifchilddocmanual|
will be true upon processing of a part,
and the name of the part is stored in |\childdocname|.
Note that |\jobname| will be set to the filename of the current part
so that each part receives an individual |.aux| file
that does not interfere with the |.aux| file(s) of the main document.
This behaviour can be altered by the alternative form
|\childdocby[*]{|\textit{main}|}| (with a non-empty optional argument)
which uses the |.aux| file of the main document
by setting |\jobname| to \textit{main}.

%%%%%%%%%%%%%%%%%%%%%%%%%%%%%%%%%%%%%%%%%%%%%%%%%%%%%%%%%%%%%%%%%%%%%%%%%%%%%%%%
\subsection{Driver Development}
\label{sec:driver}

The \textsf{childdoc} mechanism can also be use for the development
of definition files such as \LaTeX{} styles or classes.
This case differs from the above setup with multiple parts
included by |\include| in that no |\includeonly| should be invoked.
This can be achieved by starting the include file
(before |\ProvidesPackage|) with:
%
\begin{center}
\begin{tabular}{l}
|\input{childdoc.def}|\\
|\childdocforward{|\textit{main}|}|\\
\end{tabular}
\end{center}
%
or alternatively with:
%
\begin{center}
\begin{tabular}{l}
|\input{childdoc.def}|\\
|\childdocby{|\textit{main}|}|\\
\end{tabular}
\end{center}
%
Both forms have slightly different effects as described above.
The main file is prepared as usual, see \secref{sec:include}.

%%%%%%%%%%%%%%%%%%%%%%%%%%%%%%%%%%%%%%%%%%%%%%%%%%%%%%%%%%%%%%%%%%%%%%%%%%%%%%%%
\subsection{Legacy Detection}
\label{sec:detection}

The directive |\childdocmain| in the main file can detect
whether the complete document or merely a child is to be compiled
even without using the directive |\childdocof|.
This method is deprecated because it is less robust
and there is no compelling reason to use it;
it is merely provided for backward compatibility
and it may be removed in future versions.

If the detection mechanism is to be used,
it is mandatory to correctly specify
the filename of the main file as the argument of |\childdocmain|:
%
\begin{center}
\begin{tabular}{l}
|\input{childdoc.def}|\\
|\childdocmain{|\textit{main}|}|\\
\end{tabular}
\end{center}
%
If |\jobname| does not match the argument \textit{main} of |\childdocmain|,
it is assumed that |\jobname| points to the child file to be compiled.
When using |\childdocmain| with the main file specified as argument,
it suffices to start a child file
with just |\input{|\textit{main}|}|
without loading of the package and using |\childdocof|.
If instead all processing is done
with the appropriate \textsf{childdoc} directives,
the argument of \textit{main} of |\childdocmain| can be empty.

An alternative version of the command line processing described
in \secref{sec:commandline} using the detection mechanism reads:
%
\begin{center}
|... -jobname "|\textit{target}|" "|[\textit{flags}]%
[|\def\jobname{|\textit{dest}|}|]|\input{|\textit{main}|}"|
\end{center}

%%%%%%%%%%%%%%%%%%%%%%%%%%%%%%%%%%%%%%%%%%%%%%%%%%%%%%%%%%%%%%%%%%%%%%%%%%%%%%%%
\subsection{Manual Code}
\label{sec:manual}

In case one cannot be certain whether the definitions file |childdoc.def|
is installed on the target \TeX{} distribution
and one prefers not to ship it,
it is conceivable to paste a few relevant commands into the sources.

To that end, drop all statements |\input{childdoc.def}|
and perform the replacements as outlined below.
Instead of |\childdocmain{|\textit{main}|}| add the following code
to the top of the main file:
%
\begin{center}
\begin{tabular}{l}
|\||ifdefined\childdocname\endinput\||fi\newif\ifchilddoc|\\
|\edef\childdocname{\scantokens\expandafter{\jobname\noexpand}}|\\
|\def\childdocmain{|\textit{main}|}\||ifx\childdocmain\childdocname\||else|\\
|\childdoctrue\includeonly{\childdocname}\let\jobname\childdocmain\||fi|\\
\end{tabular}
\end{center}
%
Instead of |\childdocof{|\textit{main}|}| just include the main file
at the top of each child file:
%
\begin{center}
|\input{|\textit{main}|}|
\end{center}
%
A simple redirection |\childdocforward{|\textit{dest}|}| is achieved by:
%
\begin{center}
|\def\jobname{|\textit{dest}|}\input{\jobname}|
\end{center}
%
The redirection with prefix
|\childdocforwardprefix[|\textit{prefix}|]{|\textit{dest}|}|
is accomplished by:
%
\begin{center}
\begin{tabular}{l}
|{\edef\jobname{\scantokens\expandafter{\jobname\noexpand}}|\\
|\def\redirectjob |\textit{prefix}|#1~~~{\gdef\jobname{|\textit{dest}|#1}}|\\
|\expandafter\redirectjob\jobname~~~}\input{\jobname}|
\end{tabular}
\end{center}

In an alternative approach,
child documents can be compiled by a specific command line
without additional code or specific definitions:
%
\begin{center}
|... -jobname "|\textit{target}|" "|[\textit{flags}]%
|\includeonly{|\textit{dest}|}\input{|\textit{main}|}"|
\end{center}
%

%%%%%%%%%%%%%%%%%%%%%%%%%%%%%%%%%%%%%%%%%%%%%%%%%%%%%%%%%%%%%%%%%%%%%%%%%%%%%%%%
%%%%%%%%%%%%%%%%%%%%%%%%%%%%%%%%%%%%%%%%%%%%%%%%%%%%%%%%%%%%%%%%%%%%%%%%%%%%%%%%
\section{Information}

%%%%%%%%%%%%%%%%%%%%%%%%%%%%%%%%%%%%%%%%%%%%%%%%%%%%%%%%%%%%%%%%%%%%%%%%%%%%%%%%
\subsection{Copyright}

Copyright \copyright{} 2017--2018 Niklas Beisert

This work may be distributed and/or modified under the
conditions of the \LaTeX{} Project Public License, either version 1.3
of this license or (at your option) any later version.
The latest version of this license is in
  \url{http://www.latex-project.org/lppl.txt}
and version 1.3 or later is part of all distributions of \LaTeX{}
version 2005/12/01 or later.

This work has the LPPL maintenance status `maintained'.

The Current Maintainer of this work is Niklas Beisert.

This work consists of the files |README.txt|, |childdoc.ins| and |childdoc.dtx|
as well as the derived files |childdoc.def|, |cdocsamp.tex|
with |cdocsch1.tex|, |cdocsch2.tex|, |cdocspt3.tex|, |cdocspt4.tex|,
|cdocsdrf.tex|, |cdocsfn1.tex|, |cdocsfn2.tex|
as well as |childdoc.pdf|.

%%%%%%%%%%%%%%%%%%%%%%%%%%%%%%%%%%%%%%%%%%%%%%%%%%%%%%%%%%%%%%%%%%%%%%%%%%%%%%%%
\subsection{Files and Installation}

The package consists of the files:
%
\begin{center}
\begin{tabular}{ll}
    |README.txt|   & readme file \\
    |childdoc.ins| & installation file \\
    |childdoc.dtx| & source file \\
    |childdoc.def| & definition file \\
    |cdocsamp.tex| & sample main file \\
    |cdocsch1.tex| & sample include file \\
    |cdocsch2.tex| & sample include file \\
    |cdocspt3.tex| & sample part file \\
    |cdocspt4.tex| & sample part file \\
    |cdocsdrf.tex| & sample redirection file \\
    |cdocsfn1.tex| & sample redirection file \\
    |cdocsfn2.tex| & sample redirection file \\
    |childdoc.pdf| & manual
\end{tabular}
\end{center}
%
The distribution consists of the files
|README.txt|, |childdoc.ins| and |childdoc.dtx|.
%
\begin{itemize}
\item
Run (pdf)\LaTeX{} on |childdoc.dtx|
to compile the manual |childdoc.pdf| (this file).
\item
Run \LaTeX{} on |childdoc.ins| to create the definitions file |childdoc.def|
and the sample |cdocsamp.tex| with include files
|cdocsch1.tex|, |cdocsch2.tex|, |cdocspt3.tex|, |cdocspt4.tex|,
|cdocsdrf.tex|, |cdocsfn1.tex|, |cdocsfn2.tex|.
Then copy the file |childdoc.def| to an appropriate directory of your \LaTeX{}
distribution, e.g.\ \textit{texmf-root}|/tex/latex/childdoc|.
\end{itemize}

%%%%%%%%%%%%%%%%%%%%%%%%%%%%%%%%%%%%%%%%%%%%%%%%%%%%%%%%%%%%%%%%%%%%%%%%%%%%%%%%
\subsection{Related CTAN Packages}

There are several other packages which offer a similar functionality:
%
\begin{itemize}
\item
The packages
\href{http://ctan.org/pkg/docmute}{\textsf{docmute}},
\href{http://ctan.org/pkg/includex}{\textsf{includex}} and
\href{http://ctan.org/pkg/standalone}{\textsf{standalone}}
provide commands to include only the document body of
a child file thus allowing both files to be compiled individually.
\item
The packages \href{http://ctan.org/pkg/subdocs}{\textsf{subdocs}}
and \href{http://ctan.org/pkg/subfiles}{\textsf{subfiles}}
provide structures in which the main and child documents can be
encapsulated and allowing them to be compiled individually.
The inclusion mechanism is different from the conventional |\include|.
\item
The package \href{http://ctan.org/pkg/combine}{\textsf{combine}}
is an elaborate solution to combine several documents into one.
\end{itemize}
%
See also the CTAN topic \href{http://ctan.org/topic/subdocs}{\textsf{subdocs}}
for further related packages.
The present package differs from the above solutions in that
a document structure constructed with the conventional |\include| mechanism
just needs two extra commands at the top of every file
such that all constituent files can be compiled individually.

%%%%%%%%%%%%%%%%%%%%%%%%%%%%%%%%%%%%%%%%%%%%%%%%%%%%%%%%%%%%%%%%%%%%%%%%%%%%%%%%
%\subsection{Feature Suggestions}
%
%The following is a list of features which may be useful for future
%versions of this package:
%%
%\begin{itemize}
%\item
%\ldots
%\end{itemize}

%%%%%%%%%%%%%%%%%%%%%%%%%%%%%%%%%%%%%%%%%%%%%%%%%%%%%%%%%%%%%%%%%%%%%%%%%%%%%%%%
\subsection{Revision History}

%%%%%%%%%%%%%%%%%%%%%%%%%%%%%%%%%%%%%%%%
\paragraph{v2.0:} 2018/12/30

\begin{itemize}
\item
immediate forward processing
\item
added |\childdocby| mechanism
\item
manual restructured
\end{itemize}

%%%%%%%%%%%%%%%%%%%%%%%%%%%%%%%%%%%%%%%%
\paragraph{v1.6:} 2018/01/17

\begin{itemize}
\item
application for development of include files
\item
corrections to manual
\end{itemize}

%%%%%%%%%%%%%%%%%%%%%%%%%%%%%%%%%%%%%%%%
\paragraph{v1.5:} 2017/05/21

\begin{itemize}
\item
more complete structuring introduced
\item
|\childdocof| introduced
\item
|\childdoc| renamed to |\childdocmain|
\item
|\childredirect| renamed to |\childdocforward| and |\childdocforwardprefix|
and functionality expanded
\end{itemize}

%%%%%%%%%%%%%%%%%%%%%%%%%%%%%%%%%%%%%%%%
\paragraph{v1.0:} 2017/04/27

\begin{itemize}
\item
manual and install package
\item
first version published on CTAN
\end{itemize}

%%%%%%%%%%%%%%%%%%%%%%%%%%%%%%%%%%%%%%%%
\paragraph{v0.6:} 2017/04/26

\begin{itemize}
\item
redirection mechanism added
\end{itemize}

%%%%%%%%%%%%%%%%%%%%%%%%%%%%%%%%%%%%%%%%
\paragraph{v0.5:} 2017/04/26

\begin{itemize}
\item
functionality in definition file
\end{itemize}


%%%%%%%%%%%%%%%%%%%%%%%%%%%%%%%%%%%%%%%%%%%%%%%%%%%%%%%%%%%%%%%%%%%%%%%%%%%%%%%%
%%%%%%%%%%%%%%%%%%%%%%%%%%%%%%%%%%%%%%%%%%%%%%%%%%%%%%%%%%%%%%%%%%%%%%%%%%%%%%%%
%%%%%%%%%%%%%%%%%%%%%%%%%%%%%%%%%%%%%%%%%%%%%%%%%%%%%%%%%%%%%%%%%%%%%%%%%%%%%%%%
\appendix

\settowidth\MacroIndent{\rmfamily\scriptsize 000\ }

 \DocInput{childdoc.dtx}

\end{document}
%</driver>
% \fi
%
% %%%%%%%%%%%%%%%%%%%%%%%%%%%%%%%%%%%%%%%%%%%%%%%%%%%%%%%%%%%%%%%%%%%%%%%%%%%%%%
% %%%%%%%%%%%%%%%%%%%%%%%%%%%%%%%%%%%%%%%%%%%%%%%%%%%%%%%%%%%%%%%%%%%%%%%%%%%%%%
% \section{Sample}
%\iffalse
%<*samplemain>
%\fi
%
% The following presents a sample document
% with two chapters, two parts, a title page,
% a compile flag as well as three forwarding files to set the flag.
% It consists of eight |.tex| files:
% \begin{center}
% \begin{tabular}{ll}
% |cdocsamp.tex|&main file\\
% |cdocsch1.tex|&include file for chapter 1\\
% |cdocsch2.tex|&include file for chapter 2\\
% |cdocspt3.tex|&include file for part 3\\
% |cdocspt4.tex|&include file for part 4\\
% |cdocsdrf.tex|&forwarding file for main file in draft mode\\
% |cdocsfi1.tex|&forwarding file for final version of chapter 1\\
% |cdocsfi2.tex|&forwarding file for final version of chapter 2\\
% \end{tabular}
% \end{center}
% Each of the eight files can be compiled directly by the \LaTeX{} compiler.
%
% %%%%%%%%%%%%%%%%%%%%%%%%%%%%%%%%%%%%%%
% \paragraph{Main File.}
%
% The main file is called |cdocsamp.tex|.
%
% Load the \textsf{childdoc} definitions and
% declare the filename for the main document:
%    \begin{macrocode}
\input{childdoc.def}
\childdocmain{}
%    \end{macrocode}

% Optional override for |\version| flag:
%    \begin{macrocode}
%%\ifchilddoc\else\providecommand{\version}{draft}\fi
%    \end{macrocode}

% Define the default values for the |\version| flag
% (|final| for the main file and |draft| for childs):
%    \begin{macrocode}
\ifchilddoc
\providecommand{\version}{draft}
\else
\providecommand{\version}{final}
\fi
%    \end{macrocode}

% Load the standard document class:
%    \begin{macrocode}
\documentclass[12pt]{article}
%    \end{macrocode}

% Start the document body:
%    \begin{macrocode}
\begin{document}
%    \end{macrocode}

% Declare a title page.
% Print title, part of document being processed and version flag:
%    \begin{macrocode}
\addtocounter{page}{-1}
\begin{center}
{\LARGE\bfseries{}childdoc example\par}
\vspace{1cm}
\ifchilddoc
\ifchilddocmanual part\else chapter\fi:
`\childdocname' of `\childdocjob'\par
\else
main document: `\childdocjob'\par
\fi
version: \version\par
\end{center}
\newpage
%    \end{macrocode}

% Manually include selected file,
% otherwise process as usual:
%    \begin{macrocode}
\ifchilddocmanual
\section*{part `\childdocname'}
\input{\childdocname}
\else
%    \end{macrocode}

% Include the two chapters:
%    \begin{macrocode}
\include{cdocsch1}
\include{cdocsch2}
%    \end{macrocode}

% Include the two parts unless only chapters should be displayed:
%    \begin{macrocode}
\ifchilddoc\else
\section{part three}
\input{cdocspt3}
\section{part four}
\input{cdocspt4}
\fi
%    \end{macrocode}

% Process as usual until here:
%    \begin{macrocode}
\fi
%    \end{macrocode}

% End of document body:
%    \begin{macrocode}
\end{document}
%    \end{macrocode}
%\iffalse
%</samplemain>
%\fi
%
% %%%%%%%%%%%%%%%%%%%%%%%%%%%%%%%%%%%%%%
% \paragraph{Chapter Include Files.}
%
% The include files are called |cdocsch1.tex| and |cdocsch2.tex|.
%
%\iffalse
%<*samplechap1|samplechap2>
%\fi

% Optional override for |\version| flag:
%    \begin{macrocode}
%%\providecommand{\version}{final}
%    \end{macrocode}

% Include the main document:
%    \begin{macrocode}
\input{childdoc.def}
\childdocof{cdocsamp}
%    \end{macrocode}

%\iffalse
%</samplechap1|samplechap2>
%\fi
%
%\iffalse
%<*samplechap1>
%\fi
% Some text for chapter 1:
%    \begin{macrocode}
\section{one}
some text in chapter one
%    \end{macrocode}

%\iffalse
%</samplechap1>
%\fi
% Some text for chapter 2:
%\iffalse
%<*samplechap2>
%\fi
%    \begin{macrocode}
\section{two}
more text in chapter two
%    \end{macrocode}

%\iffalse
%</samplechap2>
%\fi
%
% %%%%%%%%%%%%%%%%%%%%%%%%%%%%%%%%%%%%%%
% \paragraph{Part Include Files.}
%
% The include files are called |cdocspt3.tex| and |cdocspt4.tex|.
%
%\iffalse
%<*samplepart3|samplepart4>
%\fi

% Optional override for |\version| flag:
%    \begin{macrocode}
%%\providecommand{\version}{final}
%    \end{macrocode}

% Include the main document:
%    \begin{macrocode}
\input{childdoc.def}
\childdocby{cdocsamp}
%    \end{macrocode}

%\iffalse
%</samplepart3|samplepart4>
%\fi
%
%\iffalse
%<*samplepart3>
%\fi
% Some text for part 3:
%    \begin{macrocode}
some text in part three
%    \end{macrocode}

%\iffalse
%</samplepart3>
%\fi
% Some text for part 4:
%\iffalse
%<*samplepart4>
%\fi
%    \begin{macrocode}
more text in part four
%    \end{macrocode}

%\iffalse
%</samplepart4>
%\fi
%
% %%%%%%%%%%%%%%%%%%%%%%%%%%%%%%%%%%%%%%
% \paragraph{Forwarding for a Complete Draft.}
%
% The following forwarding file |cdocsdrf.tex|
% compiles the main document in draft mode:
%\iffalse
%<*sampledraft>
%\fi
%    \begin{macrocode}
\def\version{draft}
\input{childdoc.def}
\childdocforward{cdocsamp}
%    \end{macrocode}

%\iffalse
%</sampledraft>
%\fi
%
% %%%%%%%%%%%%%%%%%%%%%%%%%%%%%%%%%%%%%%
% \paragraph{Forwarding for Final Version of the Chapters.}
%
% The following forwarding files |cdocsfn1.tex| and |cdocsfn2.tex|
% (with identical content)
% compile the final versions of the child documents
% |cdocsch1.tex| and |cdocsch2.tex|, respectively:
%\iffalse
%<*samplefinal>
%\fi
%    \begin{macrocode}
\def\version{final}
\input{childdoc.def}
\childdocforwardprefix[cdocsamp]{cdocsfn}{cdocsch}
%    \end{macrocode}

%\iffalse
%</samplefinal>
%\fi
%
% %%%%%%%%%%%%%%%%%%%%%%%%%%%%%%%%%%%%%%
% \paragraph{Command Line Processing.}
%
% The following three command lines generate the output files
% |cdocscld|, |cdocscl1| and |cdocscl2|
% which should be identical to
% |cdocsdrf|, |cdocsch1| and |cdocsfn2|, respectively:
% \begin{center}
% \begin{tabular}{l}
% |latex -jobname cdocscld \|\\
% |  "\def\version{draft}\input{childdoc.def}\childdocforward{cdocsamp}"|\\
% |latex -jobname cdocscl1 \|\\
% |  "\input{childdoc.def}\childdocforward[cdocsamp]{cdocsch1}"|\\
% |latex -jobname cdocscl2 \|\\
% |  "\def\version{final}\input{childdoc.def}\childdocforward{cdocsch2}"|
% \end{tabular}
% \end{center}
% Note that the trailing backslash on each first line
% merely continues the input to the second line
% (for convenient cut ant paste).
% Furthermore, the command |latex| can be replaced by any
% of its alternative versions such as |pdflatex|.
%
% %%%%%%%%%%%%%%%%%%%%%%%%%%%%%%%%%%%%%%%%%%%%%%%%%%%%%%%%%%%%%%%%%%%%%%%%%%%%%%
% %%%%%%%%%%%%%%%%%%%%%%%%%%%%%%%%%%%%%%%%%%%%%%%%%%%%%%%%%%%%%%%%%%%%%%%%%%%%%%
% \section{Implementation}
%\iffalse
%<*package>
%\fi
%
% This section describes the definitions file |childdoc.def|.

% The definitions cannot be loaded using |\usepackage| or |\RequirePackage|
% which has a mechanism to prevent loading a style file more than once.
% When loading the definitions by means of |\input|
% multiple instances have to be prevented manually:
%\iffalse
%This code needs to be before the `\ProvidesFile' directive
%which is defined at the beginning of this file.
%Therefore it is also placed there and commented out here.
%</package>
%<*discard>
%\fi
%    \begin{macrocode}
\ifdefined\childdocmain\endinput\fi
%    \end{macrocode}
%\iffalse
%</discard>
%<*package>
%\fi
%
% \macro{\ifchilddoc}
% \macro{\ifchilddocmanual}
% The conditional |\ifchilddoc| tells whether a
% child (true) or main (false) document is being compiled.
% The conditional |\ifchilddocmanual| tells whether
% the |\includeonly| mechanism is used (false) or
% the selection of child files must be performed manually (true).
% The definitions initialise to false:
%    \begin{macrocode}
\newif\ifchilddoc
\newif\ifchilddocmanual
%    \end{macrocode}

% \macro{\childdocname}
% \macro{\childdocjob}
% The macro |\childdocname| stores the name of the main document
% to be compiled. The macro |\childdocjob| stores the name of
% the document on which the \LaTeX{} compiler was originally invoked.
% The content of |\jobname| cannot be compared
% to filenames specified in the source due to different catcodes.
% The following code rescans |\jobname|, stores the result
% in |\childdocname| and saves a copy in |\childdocjob|:
%    \begin{macrocode}
\edef\childdocname{\scantokens\expandafter{\jobname\noexpand}}
\let\childdocjob\childdocname
%    \end{macrocode}

% \macro{\childdocdisable}
% The macro |\childdocdisable| prevents the main file
% from being processed more than once.
% At this stage, the main document command |\childdocmain|
% is assumed to be called once again where it should do nothing.
% Any subsequent call to it should prevent
% a secondary processing of the main document
% It overwrites the forwarding commands
% |\childdocof| and |\childdocforward|
% with empty macros to prevent further inclusions of the main document:
%    \begin{macrocode}
\newcommand{\childdocdisable}
{
  \renewcommand{\childdocmain}[1]{\renewcommand{\childdocmain}[1]{\endinput}}
  \renewcommand{\childdocof}[1]{}
  \renewcommand{\childdocby}[2][]{}
  \renewcommand{\childdocforward}[2][]{}
  \renewcommand{\childdocdisable}{}
}
%    \end{macrocode}

% \macro{\childdocmain}
% The macro |\childdocmain| is to be called at the top of the main file
% with nothing or the main filename (without extension) as argument.
% First, it breaks loops.
% If the argument is not empty and does not match |\childdocname|
% (which is set by the first inclusion of |childdoc.def|),
% |\ifchilddoc| is set to true, |\includeonly| is applied to the child file
% and |\jobname| is set to the main file
% (for proper handling of |.aux| files):
%    \begin{macrocode}
\newcommand{\childdocmain}[1]
{
  \childdocdisable\childdocmain{}
  \if?#1?\else
    \begingroup
      \def\childdoctmp{#1}
      \ifx\childdoctmp\childdocname
        \def\childdoctmp{}
      \else
        \def\childdoctmp
        {
          \childdoctrue
          \includeonly{\childdocname}
          \def\childdocjob{#1}
          \def\jobname{#1}
        }
      \fi
      \expandafter
    \endgroup
    \childdoctmp
  \fi
}
%    \end{macrocode}

% \macro{\childdocof}
% The command |\childdocof| redirects
% compilation to the main file |#1|.
%    \begin{macrocode}
\newcommand{\childdocof}[1]
{
  \childdocdisable
  \childdoctrue
  \includeonly{\childdocname}
  \def\jobname{#1}
  \def\childdocjob{#1}
  \input{#1}
}
%    \end{macrocode}

% \macro{\childdocby}
% The command |\childdocby| ....
%    \begin{macrocode}
\newcommand{\childdocby}[2][]
{
  \childdocdisable
  \childdoctrue
  \childdocmanualtrue
  \if?#1?\else
    \def\jobname{#2}
  \fi
  \def\childdocjob{#2}
  \input{#2}
  \endinput
}
%    \end{macrocode}

% \macro{\childdocforward}
% The command |\childdocforward| redirects
% compilation to the main file or
% (if the optional argument is given) a child file.
% Parameters are set as if the main file
% or a child file starting with |\childdocof| was compiled.
% Then compilation is handed over to the main file:
%    \begin{macrocode}
\newcommand{\childdocforward}[2][]
{
  \begingroup
    \if?#1?
      \def\childdoctmp
      {
        \def\childdocname{#2}
        \def\childdocjob{#2}
        \def\jobname{#2}
        \input{#2}
        \endinput
      }
    \else
      \def\childdoctmp
      {
        \childdocdisable
        \def\childdocname{#2}
        \childdoctrue
        \includeonly{#2}
        \def\childdocjob{#1}
        \def\jobname{#1}
        \input{#1}
        \endinput
      }
    \fi
    \expandafter
  \endgroup
  \childdoctmp
}
%    \end{macrocode}

% \macro{\childdocforwardprefix}
% The command |\childdocforwardprefix| redirects
% compilation to the main or a child file by means of a pattern.
% The prefix |#1| in the current filename is replaced by |#2|
% and the suffix of the current filename is kept
% (it is assumed that the filename does not contain the substring `|~~~|'
% which is used as a delimiter).
% Compilation is handed over to the new file by |\childdocforward|:
%    \begin{macrocode}
\newcommand{\childdocforwardprefix}[3][]
{
  \begingroup
    \def\childdocextract #2##1~~~{\def\childdoctmp{\childdocforward[#1]{#3##1}}}
    \expandafter\childdocextract\childdocname~~~
    \expandafter
  \endgroup
  \childdoctmp
}
%    \end{macrocode}

% \macro{\childdoc}
% The deprecated macro |\childdoc| is a legacy version of |\childdocmain|:
%    \begin{macrocode}
\newcommand{\childdoc}{\childdocmain}
%    \end{macrocode}

% \macro{\childdocredirect}
% The deprecated macro |\childdocredirect| is a legacy version
% of |\childdocforward| and |\childdocforwardprefix|:
%    \begin{macrocode}
\newcommand{\childdocredirect}[2][]
{
  \begingroup
    \if?#1?
      \def\childdoctmp{\childdocforward{#2}}
    \else
      \def\childdoctmp{\childdocforwardprefix{#1}{#2}}
    \fi
    \expandafter
  \endgroup
  \childdoctmp
}
%    \end{macrocode}

%\iffalse
%</package>
%\fi
%
\endinput
|\\
|\childdocby{|\textit{main}|}|\\
\end{tabular}
\end{center}
%
Both forms have slightly different effects as described above.
The main file is prepared as usual, see \secref{sec:include}.

%%%%%%%%%%%%%%%%%%%%%%%%%%%%%%%%%%%%%%%%%%%%%%%%%%%%%%%%%%%%%%%%%%%%%%%%%%%%%%%%
\subsection{Legacy Detection}
\label{sec:detection}

The directive |\childdocmain| in the main file can detect
whether the complete document or merely a child is to be compiled
even without using the directive |\childdocof|.
This method is deprecated because it is less robust
and there is no compelling reason to use it;
it is merely provided for backward compatibility
and it may be removed in future versions.

If the detection mechanism is to be used,
it is mandatory to correctly specify
the filename of the main file as the argument of |\childdocmain|:
%
\begin{center}
\begin{tabular}{l}
|% \iffalse
%
% childdoc.dtx Copyright (C) 2017-2018 Niklas Beisert
%
% This work may be distributed and/or modified under the
% conditions of the LaTeX Project Public License, either version 1.3
% of this license or (at your option) any later version.
% The latest version of this license is in
%   http://www.latex-project.org/lppl.txt
% and version 1.3 or later is part of all distributions of LaTeX
% version 2005/12/01 or later.
%
% This work has the LPPL maintenance status `maintained'.
%
% The Current Maintainer of this work is Niklas Beisert.
%
% This work consists of the files childdoc.dtx and childdoc.ins
% and the derived files childdoc.def and cdocsamp.tex with
% cdocsch1.tex, cdocsch2.tex, cdocsdrf.tex, cdocsfn1.tex, cdocsfn2.tex.
%
%<package>\ifdefined\childdocmain\endinput\fi
%<package>\ProvidesFile{childdoc.def}[2018/12/30 v2.0 child document driver]
%<samplemain>\ProvidesFile{cdocsamp.tex}[2018/12/30 v2.0 sample for childdoc]
%<*driver>
%\ProvidesFile{childdoc.drv}[2018/12/30 v2.0 childdoc reference manual file]
\PassOptionsToClass{10pt,a4paper}{article}
\documentclass{ltxdoc}

\usepackage[margin=35mm]{geometry}
\usepackage{hyperref}
\usepackage{hyperxmp}
\usepackage[usenames]{color}

\hypersetup{colorlinks=true}
\hypersetup{pdfstartview=FitH}
\hypersetup{pdfpagemode=UseNone}
\hypersetup{pdfsource={}}
\hypersetup{pdflang={en-UK}}
\hypersetup{pdfcopyright={Copyright 2017-2018 Niklas Beisert.
  This work may be distributed and/or modified under the
  conditions of the LaTeX Project Public License, either version 1.3
  of this license or (at your option) any later version.}}
\hypersetup{pdflicenseurl={http://www.latex-project.org/lppl.txt}}
\hypersetup{pdfcontactaddress={ETH Zurich, ITP, HIT K,
  Wolfgang-Pauli-Strasse 27}}
\hypersetup{pdfcontactpostcode={8093}}
\hypersetup{pdfcontactcity={Zurich}}
\hypersetup{pdfcontactcountry={Switzerland}}
\hypersetup{pdfcontactemail={nbeisert@itp.phys.ethz.ch}}
\hypersetup{pdfcontacturl={http://people.phys.ethz.ch/\xmptilde nbeisert/}}

\newcommand{\secref}[1]{\hyperref[#1]{section \ref*{#1}}}

\parskip1ex
\parindent0pt
\let\olditemize\itemize
\def\itemize{\olditemize\parskip0pt}

\begin{document}

\title{The \textsf{childdoc} Package}
\hypersetup{pdftitle={The childdoc Package}}
\author{Niklas Beisert\\[2ex]
  Institut f\"ur Theoretische Physik\\
  Eidgen\"ossische Technische Hochschule Z\"urich\\
  Wolfgang-Pauli-Strasse 27, 8093 Z\"urich, Switzerland\\[1ex]
  \href{mailto:nbeisert@itp.phys.ethz.ch}
  {\texttt{nbeisert@itp.phys.ethz.ch}}}
\hypersetup{pdfauthor={Niklas Beisert}}
\hypersetup{pdfsubject={Manual for the LaTeX2e Package childdoc}}
\date{30 December 2018, \textsf{v2.0}}
\maketitle

\begin{abstract}\noindent
\textsf{childdoc} is a \LaTeXe{} package
that enables the direct compilation
of document sections included by |\include|
to individual files.
\end{abstract}

\begingroup
\parskip0ex
\tableofcontents
\endgroup

%%%%%%%%%%%%%%%%%%%%%%%%%%%%%%%%%%%%%%%%%%%%%%%%%%%%%%%%%%%%%%%%%%%%%%%%%%%%%%%%
%%%%%%%%%%%%%%%%%%%%%%%%%%%%%%%%%%%%%%%%%%%%%%%%%%%%%%%%%%%%%%%%%%%%%%%%%%%%%%%%
\section{Introduction}

\LaTeX{} provides a mechanism to structure a large document (such as a book)
into a main file and several child files (containing the chapters)
using the |\include| command.
This mechanism is beneficial for documents
which span hundreds of pages in order to
make the source file(s) more manageable.
Moreover, compilation can be restricted to
selected child files by means of the |\includeonly| command.
The latter feature can be used to reduce the compilation time while editing
(this was significantly more useful in the earlier days of \LaTeX{})
or to generate a smaller document which is easier to navigate.
Another application of |\includeonly| is to generate
documents consisting of selected parts of the complete document.

However, there are a few drawbacks of the plain |\include| mechanism:
\begin{itemize}
\item
The child files cannot be compiled on their own,
they can only be compiled via the main file.
A naive editing environment
(such as a text editor with an option
to have the current file processed by \LaTeX)
may require one to switch to the main file before compiling;
attempting to compile the child file produces errors.
\item
The main file must be modified (each time)
to adjust the |\includeonly| command
to the present needs. This easily leaves the main file in a messy state.
\item
The generated document will always carry the filename
of the main document. This is inconvenient if
several child files are to be compiled and
to be kept for distribution.
\end{itemize}

The present package provides a simple interface
to make child files individually compilable by \LaTeX{}.
Compiling a child file then has the same effect as compiling
the main file with an |\includeonly| command
to select the appropriate child.
Moreover the generated document will carry the name of the child
rather than the main file.
This resolves all three above issues.

This feature is meant to make the editing of books,
thesis documents and lecture notes somewhat more convenient.
However, the package can also be used efficiently for
composing a series of documents (such as exercise sheets)
which are typically distributed individually.
It then assists the author in generating the individual documents
(potentially in different versions)
as well as a document containing the collected series.
Another application is in developing style files
or other kinds of included material
where compilation of the style file could redirect
to a sample or test file.

%%%%%%%%%%%%%%%%%%%%%%%%%%%%%%%%%%%%%%%%%%%%%%%%%%%%%%%%%%%%%%%%%%%%%%%%%%%%%%%%
%%%%%%%%%%%%%%%%%%%%%%%%%%%%%%%%%%%%%%%%%%%%%%%%%%%%%%%%%%%%%%%%%%%%%%%%%%%%%%%%
\section{Usage}

First of all, the package \textsf{childdoc} is \emph{not} a standard
\LaTeXe{} |.sty| style file! Therefore it needs to be invoked in
a non-standard way.

%%%%%%%%%%%%%%%%%%%%%%%%%%%%%%%%%%%%%%%%%%%%%%%%%%%%%%%%%%%%%%%%%%%%%%%%%%%%%%%%
\subsection{Included Files}
\label{sec:include}

%%%%%%%%%%%%%%%%%%%%%%%%%%%%%%%%%%%%%%%%
\DescribeMacro{\childdocmain}
To use the package, add the commands
\begin{center}
\begin{tabular}{l}
|\input{childdoc.def}|\\
|\childdocmain{}|\\
\end{tabular}
\end{center}
at the very top of the main \LaTeX{} file,
in particular \emph{before} the |\documentclass| statement!
The argument of |\childdocmain| should be left empty
(but it must be present).

%%%%%%%%%%%%%%%%%%%%%%%%%%%%%%%%%%%%%%%%
\DescribeMacro{\childdocof}
Furthermore, add the commands
\begin{center}
\begin{tabular}{l}
|\input{childdoc.def}|\\
|\childdocof{|\textit{main}|}|\\
\end{tabular}
\end{center}
at the top of every child file \textit{child}
which is included by |\include{|\textit{child}|}|
from within the main file
(or at least for those files to be compiled individually).
The argument \textit{main} must be the filename of the main file.

There are a couple of
considerations in setting up the main and child documents:

%%%%%%%%%%%%%%%%%%%%%%%%%%%%%%%%%%%%%%%%
\paragraph{Restrictions.}

Please note the following restrictions:
\begin{itemize}
\item
|\childdocmain| must be called with one argument \textit{main}
to ensure compatibility with earlier version of the package.
It must either be empty (|\childdocmain{}|)
or precisely match the filename of the main file in which it is specified.
See \secref{sec:detection} for further information.
\item
The filename \textit{main} must be specified without the |.tex| extension.
\item
The filename \textit{main} is case sensitive
(even in case-insensitive file systems)
due to internal string comparison.
\item
The argument \textit{main} should be fully expanded, it cannot be a macro.
\item
Subdirectories and special characters should be avoided in filenames.
\item
The command |\childdocmain{|\textit{main}|}| must be followed by a whitespace.
It should not be followed immediately by another command
or by a comment mark `|%|'.
This is because the \TeX{} parser reads the token immediately following
the argument of |\childdocmain| and puts it
at the beginning of every child section;
however, a white\-space is ignored.
\end{itemize}

%%%%%%%%%%%%%%%%%%%%%%%%%%%%%%%%%%%%%%%%
\paragraph{Content of Main File.}

It is advisable to place all content in the child files included by |\include|.
Any output contained in the main file will appear in all child documents
unless suppressed manually;
it cannot be suppressed automatically by the |\includeonly| directive
and thus should normally be avoided.
A method to include some content in the main file
by means of conditional processing is described in \secref{sec:conditional}.

%%%%%%%%%%%%%%%%%%%%%%%%%%%%%%%%%%%%%%%%
\paragraph{Page Numbering.}

When only a part of the document is compiled,
the appropriate numbering of pages
(as well as other status parameters)
is determined from the |.aux| files.
The latter contain information from previous passes.
However this information needs to propagate through
all intermediate child documents.
Therefore the page numbering in child documents may well
be inconsistent until the complete document is compiled at least once.

A useful (if unconventional) way to always ensure a consistent
page numbering is to restart the numbering in each child document
and denote the pages by `\textit{child}|.|\textit{page}'
where \textit{child} represents the chapter/section number of the child file.
This can be achieved by the command
|\numberwithin{page}{|\textit{child}|}|
of the \textsf{amsmath} package
where \textit{child} can be |chapter| or |section|
depending on the chosen structuring.
Alternatively, one can modify the macro |\thepage| appropriately
and reset the counter |page| at the start of each child file.

%%%%%%%%%%%%%%%%%%%%%%%%%%%%%%%%%%%%%%%%%%%%%%%%%%%%%%%%%%%%%%%%%%%%%%%%%%%%%%%%
\subsection{Conditional Processing}
\label{sec:conditional}

The package provides a mechanism to compile different versions
of a document. To customise the versions further some conditional processing
can come in handy to distinguish which version is being compiled.
The package provides two macros to describe the compilation context:

%%%%%%%%%%%%%%%%%%%%%%%%%%%%%%%%%%%%%%%%
\DescribeMacro{\ifchilddoc}
The conditional |\ifchilddoc| distinguishes between the compilation of
child documents and the main document:
%
\begin{center}
|\ifchilddoc |\textit{child-code}| |[|\||else |\textit{main-code}]| \||fi|
\end{center}

%%%%%%%%%%%%%%%%%%%%%%%%%%%%%%%%%%%%%%%%
\DescribeMacro{\childdocname}
\DescribeMacro{\childdocjob}
The macro |\childdocname| contains the filename (without extension)
of the main or child file being processed.
Note that |\childdocjob| will always contain the name of the main file.

%%%%%%%%%%%%%%%%%%%%%%%%%%%%%%%%%%%%%%%%
\paragraph{Title Page.}

Conditional processing can be used to include a title or banner page
in the main document when proper precautions are taken.
Importantly, the code in the main file should ensure that the page counter
(as well as other status parameters which are stored in the |.aux| files)
takes the same value after the conditional processing.
Otherwise the page numbers may take divergent values
depending on which part is compiled.

For example, a title page could be declared by:
%
\begin{center}
\begin{tabular}{l}
|\ifchilddoc\||else|\\
|\addtocounter{page}{-1}|\\
\textit{code for title page}\\
|\newpage|\\
|\||fi|
\end{tabular}
\end{center}
%
A banner page for the child documents can be generated by:
%
\begin{center}
\begin{tabular}{l}
|\ifchilddoc|\\
|\addtocounter{page}{-1}|\\
\textit{code for banner page}\\
|\newpage|\\
|\||fi|
\end{tabular}
\end{center}
%
Here one could write a message such as:
\begin{center}
|This is the part \childdocname{} of \childdocjob{}.|
\end{center}

%%%%%%%%%%%%%%%%%%%%%%%%%%%%%%%%%%%%%%%%%%%%%%%%%%%%%%%%%%%%%%%%%%%%%%%%%%%%%%%%
\subsection{Flags}
\label{sec:flags}

The package makes it easy to generate different versions
of the main or child documents.
To this end compilation flags can be defined
and assigned different default values.
They will be particularly useful in conjunction
with the forwarding mechanism described in \secref{sec:forward}.

For example, it may be useful to have a flag |\version|
which can be set to |draft| or |final|.
The document source will contain some conditional code
depending on the value of |\version|.
Suppose further, the flag should default to |final| for the main file
and to |draft| for child files
which is a natural assignment for editing the document.
This is achieved by placing the following code
in the preamble of the main document
(below the |\childdocmain| directive):
%
\begin{center}
\begin{tabular}{l}
|\ifchilddoc|\\
|\providecommand{\version}{draft}|\\
|\||else|\\
|\providecommand{\version}{final}|\\
|\||fi|
\end{tabular}
\end{center}
%
The definition by |\providecommand| makes sure
that previous definitions are not overwritten.
Further statements |\providecommand{\version}{...}|
can thus be added before the above code to override it.

For the main file, one might add a line
(between |\childdocmain| and the above block)
%
\begin{center}
|%\ifchilddoc\||else\providecommand{\version}{draft}\||fi|
\end{center}
%
which can be uncommented to produce a draft version.
Likewise one can add a line to the very top of a child file
(above the |\childdocof{|\textit{main}|}| directive)
%
\begin{center}
|%\providecommand{\version}{final}|
\end{center}
%
which can be uncommented to produce the final version of this child document.

%%%%%%%%%%%%%%%%%%%%%%%%%%%%%%%%%%%%%%%%%%%%%%%%%%%%%%%%%%%%%%%%%%%%%%%%%%%%%%%%
\subsection{Forwarding}
\label{sec:forward}

Different versions of the main or child documents
using compilation flags as described in \secref{sec:flags}
can be (permanently) stored in different files
for convenient compilation, viewing and distribution.
To this end, the package defines a command
to pass on compilation to a different file:

%%%%%%%%%%%%%%%%%%%%%%%%%%%%%%%%%%%%%%%%
\DescribeMacro{\childdocforward}
The command |\childdocforward| redirects processing to
another source file:
%
\begin{center}
\begin{tabular}{l}
|\input{childdoc.def}|\\
|\childdocforward[|\textit{main}|]{|\textit{dest}|}|\\
\end{tabular}
\end{center}
%
The argument \textit{dest} is the destination file
(without extension).
It should be the main file or one of the child files.
Note that further \textsf{childdoc} directives
such as |\childdocof| and |\childdocforward|
in the indicated file will be processed in this form.
The optional argument \textit{main}
passes on directly to the main file \textit{main}
while pretending to compile the child \textit{dest}.
This form behaves as if \textit{dest}
issues |\childdocof{|\textit{main}|}| right away,
and no further \textsf{childdoc} directives will be processed.

%%%%%%%%%%%%%%%%%%%%%%%%%%%%%%%%%%%%%%%%
\DescribeMacro{\...prefix}
In the alternative form |\childdocforwardprefix|,
%
\begin{center}
\begin{tabular}{l}
|\input{childdoc.def}|\\
|\childdocforwardprefix[|\textit{main}|]{|\textit{prefix}|}{|\textit{dest}|}|
\end{tabular}
\end{center}
%
the destination file is determined by a pattern
depending on the current file:
To make this work, the current file must be called
`{\textit{prefix}\hspace{0.2em}\textit{suffix}}'
with \textit{prefix} matching precisely the argument.
Processing is then passed on to the file
`{\textit{dest}\hspace{0.2em}\textit{suffix}}'.
Surely, the same effect is achieved by
directly specifying the
argument `{\textit{dest}\hspace{0.2em}\textit{suffix}}'
in the first form.
However, that requires to set up a different file
for each child. With the alternative form of the command
all these files can have exactly the same content
which simplifies setting them up and maintaining them.

For example, the following file |draft.tex|
with a compilation flag |\version| as described in \secref{sec:flags}
compiles the main document as a draft:
%
\begin{center}
\begin{tabular}{l}
|\def\version{draft}|\\
|\input{childdoc.def}|\\
|\childdocforward{|\textit{main}|}|
\end{tabular}
\end{center}
%
Likewise, the following files |final|\textit{nn}|.tex|
compile the final version of the child document
|child|\textit{nn}|.tex|:
%
\begin{center}
\begin{tabular}{l}
|\def\version{final}|\\
|\input{childdoc.def}|\\
|\childdocforwardprefix{final}{child}|
\end{tabular}
\end{center}
%

Note that when several versions of a main file and/or of each child file
are to be generated, it may be convenient to set up a |Makefile| or
shell script to automatise the process.

%%%%%%%%%%%%%%%%%%%%%%%%%%%%%%%%%%%%%%%%%%%%%%%%%%%%%%%%%%%%%%%%%%%%%%%%%%%%%%%%
\subsection{Command Line Processing}
\label{sec:commandline}

The effect of redirection files can also be achieved by invoking
the \LaTeX{} compiler with a more elaborate command line.
Most conveniently this should be done as part
of a shell script or a |Makefile|.

When using \textsf{childdoc} in the main file, the following
command lines effectively perform a redirection
(note that depending on the shell being used,
backslashes may have to be doubled: `|\|' $\to$ `|\\|'):
%
\begin{center}
|... -jobname "|\textit{target}|" |\\|"|[\textit{flags}]%
|\input{childdoc.def}\childdocforward[|\textit{main}|]{|\textit{dest}|}"|
\end{center}
%
Here \textit{target} is the name of the output file,
\textit{main} is the name of the main file
and \textit{dest} is the name of the main or child file to be processed
(all filenames without extensions).
The optional argument \textit{main} can be omitted
if \textit{main} matches \textit{dest}.
Optionally, compilation \textit{flags} can be defined via |\def| commands.
This command line makes the \TeX{} engine believe
it is compiling the file \textit{target}
whose content is specified as the latter parameter.
The provided code then forwards the processing to
\textit{main} or \textit{dest} as described in \secref{sec:forward}.

%%%%%%%%%%%%%%%%%%%%%%%%%%%%%%%%%%%%%%%%%%%%%%%%%%%%%%%%%%%%%%%%%%%%%%%%%%%%%%%%
\subsection{Include by Input}
\label{sec:input}

Including child documents by |\include| has some restrictions by design.
Most notably, the content of a child document always occupies
its own set of pages; pages cannot be shared between child documents.
Usually, this behaviour makes perfect sense
because each child document contain an essential part of the document.
However, in some situations it may be desirable to compose
a document from a collection of parts
without having mandatory page breaks between then.
For this case, the package
provides a mechanism to include parts
by |\input| which can also be processed individually.
However, by construction this mechanism
requires manual handling of the content to be output.

%%%%%%%%%%%%%%%%%%%%%%%%%%%%%%%%%%%%%%%%
\DescribeMacro{\ifchilddocmanual}
The main file should be prepared as usual, see \secref{sec:include}.
However, the document body must make a distinction
between processing of an individual part and of the main document, e.g.:
%
\begin{center}
\begin{tabular}{l}
|\ifchilddocmanual|\\
|\input{\childdocname}|\\
|\||else|\\
\textit{document body with }|\input{|\textit{part}|}|\\
|\||fi|
\end{tabular}
\end{center}
%
The conditional |\ifchilddocmanual| is true whenever
a part to be included by |\input| is being compiled,
and the name of the part is stored in |\childdocname|.

%%%%%%%%%%%%%%%%%%%%%%%%%%%%%%%%%%%%%%%%
\DescribeMacro{\childdocby}
Each part to be included by |\input| should start with:
%
\begin{center}
\begin{tabular}{l}
|\input{childdoc.def}|\\
|\childdocby{|\textit{main}|}|\\
\end{tabular}
\end{center}
%
The directive |\childdocby| is similar to |\childdocof|
described in \secref{sec:include},
but the subsequent selection of content must be done manually.
To that end, both |\ifchilddoc| and |\ifchilddocmanual|
will be true upon processing of a part,
and the name of the part is stored in |\childdocname|.
Note that |\jobname| will be set to the filename of the current part
so that each part receives an individual |.aux| file
that does not interfere with the |.aux| file(s) of the main document.
This behaviour can be altered by the alternative form
|\childdocby[*]{|\textit{main}|}| (with a non-empty optional argument)
which uses the |.aux| file of the main document
by setting |\jobname| to \textit{main}.

%%%%%%%%%%%%%%%%%%%%%%%%%%%%%%%%%%%%%%%%%%%%%%%%%%%%%%%%%%%%%%%%%%%%%%%%%%%%%%%%
\subsection{Driver Development}
\label{sec:driver}

The \textsf{childdoc} mechanism can also be use for the development
of definition files such as \LaTeX{} styles or classes.
This case differs from the above setup with multiple parts
included by |\include| in that no |\includeonly| should be invoked.
This can be achieved by starting the include file
(before |\ProvidesPackage|) with:
%
\begin{center}
\begin{tabular}{l}
|\input{childdoc.def}|\\
|\childdocforward{|\textit{main}|}|\\
\end{tabular}
\end{center}
%
or alternatively with:
%
\begin{center}
\begin{tabular}{l}
|\input{childdoc.def}|\\
|\childdocby{|\textit{main}|}|\\
\end{tabular}
\end{center}
%
Both forms have slightly different effects as described above.
The main file is prepared as usual, see \secref{sec:include}.

%%%%%%%%%%%%%%%%%%%%%%%%%%%%%%%%%%%%%%%%%%%%%%%%%%%%%%%%%%%%%%%%%%%%%%%%%%%%%%%%
\subsection{Legacy Detection}
\label{sec:detection}

The directive |\childdocmain| in the main file can detect
whether the complete document or merely a child is to be compiled
even without using the directive |\childdocof|.
This method is deprecated because it is less robust
and there is no compelling reason to use it;
it is merely provided for backward compatibility
and it may be removed in future versions.

If the detection mechanism is to be used,
it is mandatory to correctly specify
the filename of the main file as the argument of |\childdocmain|:
%
\begin{center}
\begin{tabular}{l}
|\input{childdoc.def}|\\
|\childdocmain{|\textit{main}|}|\\
\end{tabular}
\end{center}
%
If |\jobname| does not match the argument \textit{main} of |\childdocmain|,
it is assumed that |\jobname| points to the child file to be compiled.
When using |\childdocmain| with the main file specified as argument,
it suffices to start a child file
with just |\input{|\textit{main}|}|
without loading of the package and using |\childdocof|.
If instead all processing is done
with the appropriate \textsf{childdoc} directives,
the argument of \textit{main} of |\childdocmain| can be empty.

An alternative version of the command line processing described
in \secref{sec:commandline} using the detection mechanism reads:
%
\begin{center}
|... -jobname "|\textit{target}|" "|[\textit{flags}]%
[|\def\jobname{|\textit{dest}|}|]|\input{|\textit{main}|}"|
\end{center}

%%%%%%%%%%%%%%%%%%%%%%%%%%%%%%%%%%%%%%%%%%%%%%%%%%%%%%%%%%%%%%%%%%%%%%%%%%%%%%%%
\subsection{Manual Code}
\label{sec:manual}

In case one cannot be certain whether the definitions file |childdoc.def|
is installed on the target \TeX{} distribution
and one prefers not to ship it,
it is conceivable to paste a few relevant commands into the sources.

To that end, drop all statements |\input{childdoc.def}|
and perform the replacements as outlined below.
Instead of |\childdocmain{|\textit{main}|}| add the following code
to the top of the main file:
%
\begin{center}
\begin{tabular}{l}
|\||ifdefined\childdocname\endinput\||fi\newif\ifchilddoc|\\
|\edef\childdocname{\scantokens\expandafter{\jobname\noexpand}}|\\
|\def\childdocmain{|\textit{main}|}\||ifx\childdocmain\childdocname\||else|\\
|\childdoctrue\includeonly{\childdocname}\let\jobname\childdocmain\||fi|\\
\end{tabular}
\end{center}
%
Instead of |\childdocof{|\textit{main}|}| just include the main file
at the top of each child file:
%
\begin{center}
|\input{|\textit{main}|}|
\end{center}
%
A simple redirection |\childdocforward{|\textit{dest}|}| is achieved by:
%
\begin{center}
|\def\jobname{|\textit{dest}|}\input{\jobname}|
\end{center}
%
The redirection with prefix
|\childdocforwardprefix[|\textit{prefix}|]{|\textit{dest}|}|
is accomplished by:
%
\begin{center}
\begin{tabular}{l}
|{\edef\jobname{\scantokens\expandafter{\jobname\noexpand}}|\\
|\def\redirectjob |\textit{prefix}|#1~~~{\gdef\jobname{|\textit{dest}|#1}}|\\
|\expandafter\redirectjob\jobname~~~}\input{\jobname}|
\end{tabular}
\end{center}

In an alternative approach,
child documents can be compiled by a specific command line
without additional code or specific definitions:
%
\begin{center}
|... -jobname "|\textit{target}|" "|[\textit{flags}]%
|\includeonly{|\textit{dest}|}\input{|\textit{main}|}"|
\end{center}
%

%%%%%%%%%%%%%%%%%%%%%%%%%%%%%%%%%%%%%%%%%%%%%%%%%%%%%%%%%%%%%%%%%%%%%%%%%%%%%%%%
%%%%%%%%%%%%%%%%%%%%%%%%%%%%%%%%%%%%%%%%%%%%%%%%%%%%%%%%%%%%%%%%%%%%%%%%%%%%%%%%
\section{Information}

%%%%%%%%%%%%%%%%%%%%%%%%%%%%%%%%%%%%%%%%%%%%%%%%%%%%%%%%%%%%%%%%%%%%%%%%%%%%%%%%
\subsection{Copyright}

Copyright \copyright{} 2017--2018 Niklas Beisert

This work may be distributed and/or modified under the
conditions of the \LaTeX{} Project Public License, either version 1.3
of this license or (at your option) any later version.
The latest version of this license is in
  \url{http://www.latex-project.org/lppl.txt}
and version 1.3 or later is part of all distributions of \LaTeX{}
version 2005/12/01 or later.

This work has the LPPL maintenance status `maintained'.

The Current Maintainer of this work is Niklas Beisert.

This work consists of the files |README.txt|, |childdoc.ins| and |childdoc.dtx|
as well as the derived files |childdoc.def|, |cdocsamp.tex|
with |cdocsch1.tex|, |cdocsch2.tex|, |cdocspt3.tex|, |cdocspt4.tex|,
|cdocsdrf.tex|, |cdocsfn1.tex|, |cdocsfn2.tex|
as well as |childdoc.pdf|.

%%%%%%%%%%%%%%%%%%%%%%%%%%%%%%%%%%%%%%%%%%%%%%%%%%%%%%%%%%%%%%%%%%%%%%%%%%%%%%%%
\subsection{Files and Installation}

The package consists of the files:
%
\begin{center}
\begin{tabular}{ll}
    |README.txt|   & readme file \\
    |childdoc.ins| & installation file \\
    |childdoc.dtx| & source file \\
    |childdoc.def| & definition file \\
    |cdocsamp.tex| & sample main file \\
    |cdocsch1.tex| & sample include file \\
    |cdocsch2.tex| & sample include file \\
    |cdocspt3.tex| & sample part file \\
    |cdocspt4.tex| & sample part file \\
    |cdocsdrf.tex| & sample redirection file \\
    |cdocsfn1.tex| & sample redirection file \\
    |cdocsfn2.tex| & sample redirection file \\
    |childdoc.pdf| & manual
\end{tabular}
\end{center}
%
The distribution consists of the files
|README.txt|, |childdoc.ins| and |childdoc.dtx|.
%
\begin{itemize}
\item
Run (pdf)\LaTeX{} on |childdoc.dtx|
to compile the manual |childdoc.pdf| (this file).
\item
Run \LaTeX{} on |childdoc.ins| to create the definitions file |childdoc.def|
and the sample |cdocsamp.tex| with include files
|cdocsch1.tex|, |cdocsch2.tex|, |cdocspt3.tex|, |cdocspt4.tex|,
|cdocsdrf.tex|, |cdocsfn1.tex|, |cdocsfn2.tex|.
Then copy the file |childdoc.def| to an appropriate directory of your \LaTeX{}
distribution, e.g.\ \textit{texmf-root}|/tex/latex/childdoc|.
\end{itemize}

%%%%%%%%%%%%%%%%%%%%%%%%%%%%%%%%%%%%%%%%%%%%%%%%%%%%%%%%%%%%%%%%%%%%%%%%%%%%%%%%
\subsection{Related CTAN Packages}

There are several other packages which offer a similar functionality:
%
\begin{itemize}
\item
The packages
\href{http://ctan.org/pkg/docmute}{\textsf{docmute}},
\href{http://ctan.org/pkg/includex}{\textsf{includex}} and
\href{http://ctan.org/pkg/standalone}{\textsf{standalone}}
provide commands to include only the document body of
a child file thus allowing both files to be compiled individually.
\item
The packages \href{http://ctan.org/pkg/subdocs}{\textsf{subdocs}}
and \href{http://ctan.org/pkg/subfiles}{\textsf{subfiles}}
provide structures in which the main and child documents can be
encapsulated and allowing them to be compiled individually.
The inclusion mechanism is different from the conventional |\include|.
\item
The package \href{http://ctan.org/pkg/combine}{\textsf{combine}}
is an elaborate solution to combine several documents into one.
\end{itemize}
%
See also the CTAN topic \href{http://ctan.org/topic/subdocs}{\textsf{subdocs}}
for further related packages.
The present package differs from the above solutions in that
a document structure constructed with the conventional |\include| mechanism
just needs two extra commands at the top of every file
such that all constituent files can be compiled individually.

%%%%%%%%%%%%%%%%%%%%%%%%%%%%%%%%%%%%%%%%%%%%%%%%%%%%%%%%%%%%%%%%%%%%%%%%%%%%%%%%
%\subsection{Feature Suggestions}
%
%The following is a list of features which may be useful for future
%versions of this package:
%%
%\begin{itemize}
%\item
%\ldots
%\end{itemize}

%%%%%%%%%%%%%%%%%%%%%%%%%%%%%%%%%%%%%%%%%%%%%%%%%%%%%%%%%%%%%%%%%%%%%%%%%%%%%%%%
\subsection{Revision History}

%%%%%%%%%%%%%%%%%%%%%%%%%%%%%%%%%%%%%%%%
\paragraph{v2.0:} 2018/12/30

\begin{itemize}
\item
immediate forward processing
\item
added |\childdocby| mechanism
\item
manual restructured
\end{itemize}

%%%%%%%%%%%%%%%%%%%%%%%%%%%%%%%%%%%%%%%%
\paragraph{v1.6:} 2018/01/17

\begin{itemize}
\item
application for development of include files
\item
corrections to manual
\end{itemize}

%%%%%%%%%%%%%%%%%%%%%%%%%%%%%%%%%%%%%%%%
\paragraph{v1.5:} 2017/05/21

\begin{itemize}
\item
more complete structuring introduced
\item
|\childdocof| introduced
\item
|\childdoc| renamed to |\childdocmain|
\item
|\childredirect| renamed to |\childdocforward| and |\childdocforwardprefix|
and functionality expanded
\end{itemize}

%%%%%%%%%%%%%%%%%%%%%%%%%%%%%%%%%%%%%%%%
\paragraph{v1.0:} 2017/04/27

\begin{itemize}
\item
manual and install package
\item
first version published on CTAN
\end{itemize}

%%%%%%%%%%%%%%%%%%%%%%%%%%%%%%%%%%%%%%%%
\paragraph{v0.6:} 2017/04/26

\begin{itemize}
\item
redirection mechanism added
\end{itemize}

%%%%%%%%%%%%%%%%%%%%%%%%%%%%%%%%%%%%%%%%
\paragraph{v0.5:} 2017/04/26

\begin{itemize}
\item
functionality in definition file
\end{itemize}


%%%%%%%%%%%%%%%%%%%%%%%%%%%%%%%%%%%%%%%%%%%%%%%%%%%%%%%%%%%%%%%%%%%%%%%%%%%%%%%%
%%%%%%%%%%%%%%%%%%%%%%%%%%%%%%%%%%%%%%%%%%%%%%%%%%%%%%%%%%%%%%%%%%%%%%%%%%%%%%%%
%%%%%%%%%%%%%%%%%%%%%%%%%%%%%%%%%%%%%%%%%%%%%%%%%%%%%%%%%%%%%%%%%%%%%%%%%%%%%%%%
\appendix

\settowidth\MacroIndent{\rmfamily\scriptsize 000\ }

 \DocInput{childdoc.dtx}

\end{document}
%</driver>
% \fi
%
% %%%%%%%%%%%%%%%%%%%%%%%%%%%%%%%%%%%%%%%%%%%%%%%%%%%%%%%%%%%%%%%%%%%%%%%%%%%%%%
% %%%%%%%%%%%%%%%%%%%%%%%%%%%%%%%%%%%%%%%%%%%%%%%%%%%%%%%%%%%%%%%%%%%%%%%%%%%%%%
% \section{Sample}
%\iffalse
%<*samplemain>
%\fi
%
% The following presents a sample document
% with two chapters, two parts, a title page,
% a compile flag as well as three forwarding files to set the flag.
% It consists of eight |.tex| files:
% \begin{center}
% \begin{tabular}{ll}
% |cdocsamp.tex|&main file\\
% |cdocsch1.tex|&include file for chapter 1\\
% |cdocsch2.tex|&include file for chapter 2\\
% |cdocspt3.tex|&include file for part 3\\
% |cdocspt4.tex|&include file for part 4\\
% |cdocsdrf.tex|&forwarding file for main file in draft mode\\
% |cdocsfi1.tex|&forwarding file for final version of chapter 1\\
% |cdocsfi2.tex|&forwarding file for final version of chapter 2\\
% \end{tabular}
% \end{center}
% Each of the eight files can be compiled directly by the \LaTeX{} compiler.
%
% %%%%%%%%%%%%%%%%%%%%%%%%%%%%%%%%%%%%%%
% \paragraph{Main File.}
%
% The main file is called |cdocsamp.tex|.
%
% Load the \textsf{childdoc} definitions and
% declare the filename for the main document:
%    \begin{macrocode}
\input{childdoc.def}
\childdocmain{}
%    \end{macrocode}

% Optional override for |\version| flag:
%    \begin{macrocode}
%%\ifchilddoc\else\providecommand{\version}{draft}\fi
%    \end{macrocode}

% Define the default values for the |\version| flag
% (|final| for the main file and |draft| for childs):
%    \begin{macrocode}
\ifchilddoc
\providecommand{\version}{draft}
\else
\providecommand{\version}{final}
\fi
%    \end{macrocode}

% Load the standard document class:
%    \begin{macrocode}
\documentclass[12pt]{article}
%    \end{macrocode}

% Start the document body:
%    \begin{macrocode}
\begin{document}
%    \end{macrocode}

% Declare a title page.
% Print title, part of document being processed and version flag:
%    \begin{macrocode}
\addtocounter{page}{-1}
\begin{center}
{\LARGE\bfseries{}childdoc example\par}
\vspace{1cm}
\ifchilddoc
\ifchilddocmanual part\else chapter\fi:
`\childdocname' of `\childdocjob'\par
\else
main document: `\childdocjob'\par
\fi
version: \version\par
\end{center}
\newpage
%    \end{macrocode}

% Manually include selected file,
% otherwise process as usual:
%    \begin{macrocode}
\ifchilddocmanual
\section*{part `\childdocname'}
\input{\childdocname}
\else
%    \end{macrocode}

% Include the two chapters:
%    \begin{macrocode}
\include{cdocsch1}
\include{cdocsch2}
%    \end{macrocode}

% Include the two parts unless only chapters should be displayed:
%    \begin{macrocode}
\ifchilddoc\else
\section{part three}
\input{cdocspt3}
\section{part four}
\input{cdocspt4}
\fi
%    \end{macrocode}

% Process as usual until here:
%    \begin{macrocode}
\fi
%    \end{macrocode}

% End of document body:
%    \begin{macrocode}
\end{document}
%    \end{macrocode}
%\iffalse
%</samplemain>
%\fi
%
% %%%%%%%%%%%%%%%%%%%%%%%%%%%%%%%%%%%%%%
% \paragraph{Chapter Include Files.}
%
% The include files are called |cdocsch1.tex| and |cdocsch2.tex|.
%
%\iffalse
%<*samplechap1|samplechap2>
%\fi

% Optional override for |\version| flag:
%    \begin{macrocode}
%%\providecommand{\version}{final}
%    \end{macrocode}

% Include the main document:
%    \begin{macrocode}
\input{childdoc.def}
\childdocof{cdocsamp}
%    \end{macrocode}

%\iffalse
%</samplechap1|samplechap2>
%\fi
%
%\iffalse
%<*samplechap1>
%\fi
% Some text for chapter 1:
%    \begin{macrocode}
\section{one}
some text in chapter one
%    \end{macrocode}

%\iffalse
%</samplechap1>
%\fi
% Some text for chapter 2:
%\iffalse
%<*samplechap2>
%\fi
%    \begin{macrocode}
\section{two}
more text in chapter two
%    \end{macrocode}

%\iffalse
%</samplechap2>
%\fi
%
% %%%%%%%%%%%%%%%%%%%%%%%%%%%%%%%%%%%%%%
% \paragraph{Part Include Files.}
%
% The include files are called |cdocspt3.tex| and |cdocspt4.tex|.
%
%\iffalse
%<*samplepart3|samplepart4>
%\fi

% Optional override for |\version| flag:
%    \begin{macrocode}
%%\providecommand{\version}{final}
%    \end{macrocode}

% Include the main document:
%    \begin{macrocode}
\input{childdoc.def}
\childdocby{cdocsamp}
%    \end{macrocode}

%\iffalse
%</samplepart3|samplepart4>
%\fi
%
%\iffalse
%<*samplepart3>
%\fi
% Some text for part 3:
%    \begin{macrocode}
some text in part three
%    \end{macrocode}

%\iffalse
%</samplepart3>
%\fi
% Some text for part 4:
%\iffalse
%<*samplepart4>
%\fi
%    \begin{macrocode}
more text in part four
%    \end{macrocode}

%\iffalse
%</samplepart4>
%\fi
%
% %%%%%%%%%%%%%%%%%%%%%%%%%%%%%%%%%%%%%%
% \paragraph{Forwarding for a Complete Draft.}
%
% The following forwarding file |cdocsdrf.tex|
% compiles the main document in draft mode:
%\iffalse
%<*sampledraft>
%\fi
%    \begin{macrocode}
\def\version{draft}
\input{childdoc.def}
\childdocforward{cdocsamp}
%    \end{macrocode}

%\iffalse
%</sampledraft>
%\fi
%
% %%%%%%%%%%%%%%%%%%%%%%%%%%%%%%%%%%%%%%
% \paragraph{Forwarding for Final Version of the Chapters.}
%
% The following forwarding files |cdocsfn1.tex| and |cdocsfn2.tex|
% (with identical content)
% compile the final versions of the child documents
% |cdocsch1.tex| and |cdocsch2.tex|, respectively:
%\iffalse
%<*samplefinal>
%\fi
%    \begin{macrocode}
\def\version{final}
\input{childdoc.def}
\childdocforwardprefix[cdocsamp]{cdocsfn}{cdocsch}
%    \end{macrocode}

%\iffalse
%</samplefinal>
%\fi
%
% %%%%%%%%%%%%%%%%%%%%%%%%%%%%%%%%%%%%%%
% \paragraph{Command Line Processing.}
%
% The following three command lines generate the output files
% |cdocscld|, |cdocscl1| and |cdocscl2|
% which should be identical to
% |cdocsdrf|, |cdocsch1| and |cdocsfn2|, respectively:
% \begin{center}
% \begin{tabular}{l}
% |latex -jobname cdocscld \|\\
% |  "\def\version{draft}\input{childdoc.def}\childdocforward{cdocsamp}"|\\
% |latex -jobname cdocscl1 \|\\
% |  "\input{childdoc.def}\childdocforward[cdocsamp]{cdocsch1}"|\\
% |latex -jobname cdocscl2 \|\\
% |  "\def\version{final}\input{childdoc.def}\childdocforward{cdocsch2}"|
% \end{tabular}
% \end{center}
% Note that the trailing backslash on each first line
% merely continues the input to the second line
% (for convenient cut ant paste).
% Furthermore, the command |latex| can be replaced by any
% of its alternative versions such as |pdflatex|.
%
% %%%%%%%%%%%%%%%%%%%%%%%%%%%%%%%%%%%%%%%%%%%%%%%%%%%%%%%%%%%%%%%%%%%%%%%%%%%%%%
% %%%%%%%%%%%%%%%%%%%%%%%%%%%%%%%%%%%%%%%%%%%%%%%%%%%%%%%%%%%%%%%%%%%%%%%%%%%%%%
% \section{Implementation}
%\iffalse
%<*package>
%\fi
%
% This section describes the definitions file |childdoc.def|.

% The definitions cannot be loaded using |\usepackage| or |\RequirePackage|
% which has a mechanism to prevent loading a style file more than once.
% When loading the definitions by means of |\input|
% multiple instances have to be prevented manually:
%\iffalse
%This code needs to be before the `\ProvidesFile' directive
%which is defined at the beginning of this file.
%Therefore it is also placed there and commented out here.
%</package>
%<*discard>
%\fi
%    \begin{macrocode}
\ifdefined\childdocmain\endinput\fi
%    \end{macrocode}
%\iffalse
%</discard>
%<*package>
%\fi
%
% \macro{\ifchilddoc}
% \macro{\ifchilddocmanual}
% The conditional |\ifchilddoc| tells whether a
% child (true) or main (false) document is being compiled.
% The conditional |\ifchilddocmanual| tells whether
% the |\includeonly| mechanism is used (false) or
% the selection of child files must be performed manually (true).
% The definitions initialise to false:
%    \begin{macrocode}
\newif\ifchilddoc
\newif\ifchilddocmanual
%    \end{macrocode}

% \macro{\childdocname}
% \macro{\childdocjob}
% The macro |\childdocname| stores the name of the main document
% to be compiled. The macro |\childdocjob| stores the name of
% the document on which the \LaTeX{} compiler was originally invoked.
% The content of |\jobname| cannot be compared
% to filenames specified in the source due to different catcodes.
% The following code rescans |\jobname|, stores the result
% in |\childdocname| and saves a copy in |\childdocjob|:
%    \begin{macrocode}
\edef\childdocname{\scantokens\expandafter{\jobname\noexpand}}
\let\childdocjob\childdocname
%    \end{macrocode}

% \macro{\childdocdisable}
% The macro |\childdocdisable| prevents the main file
% from being processed more than once.
% At this stage, the main document command |\childdocmain|
% is assumed to be called once again where it should do nothing.
% Any subsequent call to it should prevent
% a secondary processing of the main document
% It overwrites the forwarding commands
% |\childdocof| and |\childdocforward|
% with empty macros to prevent further inclusions of the main document:
%    \begin{macrocode}
\newcommand{\childdocdisable}
{
  \renewcommand{\childdocmain}[1]{\renewcommand{\childdocmain}[1]{\endinput}}
  \renewcommand{\childdocof}[1]{}
  \renewcommand{\childdocby}[2][]{}
  \renewcommand{\childdocforward}[2][]{}
  \renewcommand{\childdocdisable}{}
}
%    \end{macrocode}

% \macro{\childdocmain}
% The macro |\childdocmain| is to be called at the top of the main file
% with nothing or the main filename (without extension) as argument.
% First, it breaks loops.
% If the argument is not empty and does not match |\childdocname|
% (which is set by the first inclusion of |childdoc.def|),
% |\ifchilddoc| is set to true, |\includeonly| is applied to the child file
% and |\jobname| is set to the main file
% (for proper handling of |.aux| files):
%    \begin{macrocode}
\newcommand{\childdocmain}[1]
{
  \childdocdisable\childdocmain{}
  \if?#1?\else
    \begingroup
      \def\childdoctmp{#1}
      \ifx\childdoctmp\childdocname
        \def\childdoctmp{}
      \else
        \def\childdoctmp
        {
          \childdoctrue
          \includeonly{\childdocname}
          \def\childdocjob{#1}
          \def\jobname{#1}
        }
      \fi
      \expandafter
    \endgroup
    \childdoctmp
  \fi
}
%    \end{macrocode}

% \macro{\childdocof}
% The command |\childdocof| redirects
% compilation to the main file |#1|.
%    \begin{macrocode}
\newcommand{\childdocof}[1]
{
  \childdocdisable
  \childdoctrue
  \includeonly{\childdocname}
  \def\jobname{#1}
  \def\childdocjob{#1}
  \input{#1}
}
%    \end{macrocode}

% \macro{\childdocby}
% The command |\childdocby| ....
%    \begin{macrocode}
\newcommand{\childdocby}[2][]
{
  \childdocdisable
  \childdoctrue
  \childdocmanualtrue
  \if?#1?\else
    \def\jobname{#2}
  \fi
  \def\childdocjob{#2}
  \input{#2}
  \endinput
}
%    \end{macrocode}

% \macro{\childdocforward}
% The command |\childdocforward| redirects
% compilation to the main file or
% (if the optional argument is given) a child file.
% Parameters are set as if the main file
% or a child file starting with |\childdocof| was compiled.
% Then compilation is handed over to the main file:
%    \begin{macrocode}
\newcommand{\childdocforward}[2][]
{
  \begingroup
    \if?#1?
      \def\childdoctmp
      {
        \def\childdocname{#2}
        \def\childdocjob{#2}
        \def\jobname{#2}
        \input{#2}
        \endinput
      }
    \else
      \def\childdoctmp
      {
        \childdocdisable
        \def\childdocname{#2}
        \childdoctrue
        \includeonly{#2}
        \def\childdocjob{#1}
        \def\jobname{#1}
        \input{#1}
        \endinput
      }
    \fi
    \expandafter
  \endgroup
  \childdoctmp
}
%    \end{macrocode}

% \macro{\childdocforwardprefix}
% The command |\childdocforwardprefix| redirects
% compilation to the main or a child file by means of a pattern.
% The prefix |#1| in the current filename is replaced by |#2|
% and the suffix of the current filename is kept
% (it is assumed that the filename does not contain the substring `|~~~|'
% which is used as a delimiter).
% Compilation is handed over to the new file by |\childdocforward|:
%    \begin{macrocode}
\newcommand{\childdocforwardprefix}[3][]
{
  \begingroup
    \def\childdocextract #2##1~~~{\def\childdoctmp{\childdocforward[#1]{#3##1}}}
    \expandafter\childdocextract\childdocname~~~
    \expandafter
  \endgroup
  \childdoctmp
}
%    \end{macrocode}

% \macro{\childdoc}
% The deprecated macro |\childdoc| is a legacy version of |\childdocmain|:
%    \begin{macrocode}
\newcommand{\childdoc}{\childdocmain}
%    \end{macrocode}

% \macro{\childdocredirect}
% The deprecated macro |\childdocredirect| is a legacy version
% of |\childdocforward| and |\childdocforwardprefix|:
%    \begin{macrocode}
\newcommand{\childdocredirect}[2][]
{
  \begingroup
    \if?#1?
      \def\childdoctmp{\childdocforward{#2}}
    \else
      \def\childdoctmp{\childdocforwardprefix{#1}{#2}}
    \fi
    \expandafter
  \endgroup
  \childdoctmp
}
%    \end{macrocode}

%\iffalse
%</package>
%\fi
%
\endinput
|\\
|\childdocmain{|\textit{main}|}|\\
\end{tabular}
\end{center}
%
If |\jobname| does not match the argument \textit{main} of |\childdocmain|,
it is assumed that |\jobname| points to the child file to be compiled.
When using |\childdocmain| with the main file specified as argument,
it suffices to start a child file
with just |\input{|\textit{main}|}|
without loading of the package and using |\childdocof|.
If instead all processing is done
with the appropriate \textsf{childdoc} directives,
the argument of \textit{main} of |\childdocmain| can be empty.

An alternative version of the command line processing described
in \secref{sec:commandline} using the detection mechanism reads:
%
\begin{center}
|... -jobname "|\textit{target}|" "|[\textit{flags}]%
[|\def\jobname{|\textit{dest}|}|]|\input{|\textit{main}|}"|
\end{center}

%%%%%%%%%%%%%%%%%%%%%%%%%%%%%%%%%%%%%%%%%%%%%%%%%%%%%%%%%%%%%%%%%%%%%%%%%%%%%%%%
\subsection{Manual Code}
\label{sec:manual}

In case one cannot be certain whether the definitions file |childdoc.def|
is installed on the target \TeX{} distribution
and one prefers not to ship it,
it is conceivable to paste a few relevant commands into the sources.

To that end, drop all statements |% \iffalse
%
% childdoc.dtx Copyright (C) 2017-2018 Niklas Beisert
%
% This work may be distributed and/or modified under the
% conditions of the LaTeX Project Public License, either version 1.3
% of this license or (at your option) any later version.
% The latest version of this license is in
%   http://www.latex-project.org/lppl.txt
% and version 1.3 or later is part of all distributions of LaTeX
% version 2005/12/01 or later.
%
% This work has the LPPL maintenance status `maintained'.
%
% The Current Maintainer of this work is Niklas Beisert.
%
% This work consists of the files childdoc.dtx and childdoc.ins
% and the derived files childdoc.def and cdocsamp.tex with
% cdocsch1.tex, cdocsch2.tex, cdocsdrf.tex, cdocsfn1.tex, cdocsfn2.tex.
%
%<package>\ifdefined\childdocmain\endinput\fi
%<package>\ProvidesFile{childdoc.def}[2018/12/30 v2.0 child document driver]
%<samplemain>\ProvidesFile{cdocsamp.tex}[2018/12/30 v2.0 sample for childdoc]
%<*driver>
%\ProvidesFile{childdoc.drv}[2018/12/30 v2.0 childdoc reference manual file]
\PassOptionsToClass{10pt,a4paper}{article}
\documentclass{ltxdoc}

\usepackage[margin=35mm]{geometry}
\usepackage{hyperref}
\usepackage{hyperxmp}
\usepackage[usenames]{color}

\hypersetup{colorlinks=true}
\hypersetup{pdfstartview=FitH}
\hypersetup{pdfpagemode=UseNone}
\hypersetup{pdfsource={}}
\hypersetup{pdflang={en-UK}}
\hypersetup{pdfcopyright={Copyright 2017-2018 Niklas Beisert.
  This work may be distributed and/or modified under the
  conditions of the LaTeX Project Public License, either version 1.3
  of this license or (at your option) any later version.}}
\hypersetup{pdflicenseurl={http://www.latex-project.org/lppl.txt}}
\hypersetup{pdfcontactaddress={ETH Zurich, ITP, HIT K,
  Wolfgang-Pauli-Strasse 27}}
\hypersetup{pdfcontactpostcode={8093}}
\hypersetup{pdfcontactcity={Zurich}}
\hypersetup{pdfcontactcountry={Switzerland}}
\hypersetup{pdfcontactemail={nbeisert@itp.phys.ethz.ch}}
\hypersetup{pdfcontacturl={http://people.phys.ethz.ch/\xmptilde nbeisert/}}

\newcommand{\secref}[1]{\hyperref[#1]{section \ref*{#1}}}

\parskip1ex
\parindent0pt
\let\olditemize\itemize
\def\itemize{\olditemize\parskip0pt}

\begin{document}

\title{The \textsf{childdoc} Package}
\hypersetup{pdftitle={The childdoc Package}}
\author{Niklas Beisert\\[2ex]
  Institut f\"ur Theoretische Physik\\
  Eidgen\"ossische Technische Hochschule Z\"urich\\
  Wolfgang-Pauli-Strasse 27, 8093 Z\"urich, Switzerland\\[1ex]
  \href{mailto:nbeisert@itp.phys.ethz.ch}
  {\texttt{nbeisert@itp.phys.ethz.ch}}}
\hypersetup{pdfauthor={Niklas Beisert}}
\hypersetup{pdfsubject={Manual for the LaTeX2e Package childdoc}}
\date{30 December 2018, \textsf{v2.0}}
\maketitle

\begin{abstract}\noindent
\textsf{childdoc} is a \LaTeXe{} package
that enables the direct compilation
of document sections included by |\include|
to individual files.
\end{abstract}

\begingroup
\parskip0ex
\tableofcontents
\endgroup

%%%%%%%%%%%%%%%%%%%%%%%%%%%%%%%%%%%%%%%%%%%%%%%%%%%%%%%%%%%%%%%%%%%%%%%%%%%%%%%%
%%%%%%%%%%%%%%%%%%%%%%%%%%%%%%%%%%%%%%%%%%%%%%%%%%%%%%%%%%%%%%%%%%%%%%%%%%%%%%%%
\section{Introduction}

\LaTeX{} provides a mechanism to structure a large document (such as a book)
into a main file and several child files (containing the chapters)
using the |\include| command.
This mechanism is beneficial for documents
which span hundreds of pages in order to
make the source file(s) more manageable.
Moreover, compilation can be restricted to
selected child files by means of the |\includeonly| command.
The latter feature can be used to reduce the compilation time while editing
(this was significantly more useful in the earlier days of \LaTeX{})
or to generate a smaller document which is easier to navigate.
Another application of |\includeonly| is to generate
documents consisting of selected parts of the complete document.

However, there are a few drawbacks of the plain |\include| mechanism:
\begin{itemize}
\item
The child files cannot be compiled on their own,
they can only be compiled via the main file.
A naive editing environment
(such as a text editor with an option
to have the current file processed by \LaTeX)
may require one to switch to the main file before compiling;
attempting to compile the child file produces errors.
\item
The main file must be modified (each time)
to adjust the |\includeonly| command
to the present needs. This easily leaves the main file in a messy state.
\item
The generated document will always carry the filename
of the main document. This is inconvenient if
several child files are to be compiled and
to be kept for distribution.
\end{itemize}

The present package provides a simple interface
to make child files individually compilable by \LaTeX{}.
Compiling a child file then has the same effect as compiling
the main file with an |\includeonly| command
to select the appropriate child.
Moreover the generated document will carry the name of the child
rather than the main file.
This resolves all three above issues.

This feature is meant to make the editing of books,
thesis documents and lecture notes somewhat more convenient.
However, the package can also be used efficiently for
composing a series of documents (such as exercise sheets)
which are typically distributed individually.
It then assists the author in generating the individual documents
(potentially in different versions)
as well as a document containing the collected series.
Another application is in developing style files
or other kinds of included material
where compilation of the style file could redirect
to a sample or test file.

%%%%%%%%%%%%%%%%%%%%%%%%%%%%%%%%%%%%%%%%%%%%%%%%%%%%%%%%%%%%%%%%%%%%%%%%%%%%%%%%
%%%%%%%%%%%%%%%%%%%%%%%%%%%%%%%%%%%%%%%%%%%%%%%%%%%%%%%%%%%%%%%%%%%%%%%%%%%%%%%%
\section{Usage}

First of all, the package \textsf{childdoc} is \emph{not} a standard
\LaTeXe{} |.sty| style file! Therefore it needs to be invoked in
a non-standard way.

%%%%%%%%%%%%%%%%%%%%%%%%%%%%%%%%%%%%%%%%%%%%%%%%%%%%%%%%%%%%%%%%%%%%%%%%%%%%%%%%
\subsection{Included Files}
\label{sec:include}

%%%%%%%%%%%%%%%%%%%%%%%%%%%%%%%%%%%%%%%%
\DescribeMacro{\childdocmain}
To use the package, add the commands
\begin{center}
\begin{tabular}{l}
|\input{childdoc.def}|\\
|\childdocmain{}|\\
\end{tabular}
\end{center}
at the very top of the main \LaTeX{} file,
in particular \emph{before} the |\documentclass| statement!
The argument of |\childdocmain| should be left empty
(but it must be present).

%%%%%%%%%%%%%%%%%%%%%%%%%%%%%%%%%%%%%%%%
\DescribeMacro{\childdocof}
Furthermore, add the commands
\begin{center}
\begin{tabular}{l}
|\input{childdoc.def}|\\
|\childdocof{|\textit{main}|}|\\
\end{tabular}
\end{center}
at the top of every child file \textit{child}
which is included by |\include{|\textit{child}|}|
from within the main file
(or at least for those files to be compiled individually).
The argument \textit{main} must be the filename of the main file.

There are a couple of
considerations in setting up the main and child documents:

%%%%%%%%%%%%%%%%%%%%%%%%%%%%%%%%%%%%%%%%
\paragraph{Restrictions.}

Please note the following restrictions:
\begin{itemize}
\item
|\childdocmain| must be called with one argument \textit{main}
to ensure compatibility with earlier version of the package.
It must either be empty (|\childdocmain{}|)
or precisely match the filename of the main file in which it is specified.
See \secref{sec:detection} for further information.
\item
The filename \textit{main} must be specified without the |.tex| extension.
\item
The filename \textit{main} is case sensitive
(even in case-insensitive file systems)
due to internal string comparison.
\item
The argument \textit{main} should be fully expanded, it cannot be a macro.
\item
Subdirectories and special characters should be avoided in filenames.
\item
The command |\childdocmain{|\textit{main}|}| must be followed by a whitespace.
It should not be followed immediately by another command
or by a comment mark `|%|'.
This is because the \TeX{} parser reads the token immediately following
the argument of |\childdocmain| and puts it
at the beginning of every child section;
however, a white\-space is ignored.
\end{itemize}

%%%%%%%%%%%%%%%%%%%%%%%%%%%%%%%%%%%%%%%%
\paragraph{Content of Main File.}

It is advisable to place all content in the child files included by |\include|.
Any output contained in the main file will appear in all child documents
unless suppressed manually;
it cannot be suppressed automatically by the |\includeonly| directive
and thus should normally be avoided.
A method to include some content in the main file
by means of conditional processing is described in \secref{sec:conditional}.

%%%%%%%%%%%%%%%%%%%%%%%%%%%%%%%%%%%%%%%%
\paragraph{Page Numbering.}

When only a part of the document is compiled,
the appropriate numbering of pages
(as well as other status parameters)
is determined from the |.aux| files.
The latter contain information from previous passes.
However this information needs to propagate through
all intermediate child documents.
Therefore the page numbering in child documents may well
be inconsistent until the complete document is compiled at least once.

A useful (if unconventional) way to always ensure a consistent
page numbering is to restart the numbering in each child document
and denote the pages by `\textit{child}|.|\textit{page}'
where \textit{child} represents the chapter/section number of the child file.
This can be achieved by the command
|\numberwithin{page}{|\textit{child}|}|
of the \textsf{amsmath} package
where \textit{child} can be |chapter| or |section|
depending on the chosen structuring.
Alternatively, one can modify the macro |\thepage| appropriately
and reset the counter |page| at the start of each child file.

%%%%%%%%%%%%%%%%%%%%%%%%%%%%%%%%%%%%%%%%%%%%%%%%%%%%%%%%%%%%%%%%%%%%%%%%%%%%%%%%
\subsection{Conditional Processing}
\label{sec:conditional}

The package provides a mechanism to compile different versions
of a document. To customise the versions further some conditional processing
can come in handy to distinguish which version is being compiled.
The package provides two macros to describe the compilation context:

%%%%%%%%%%%%%%%%%%%%%%%%%%%%%%%%%%%%%%%%
\DescribeMacro{\ifchilddoc}
The conditional |\ifchilddoc| distinguishes between the compilation of
child documents and the main document:
%
\begin{center}
|\ifchilddoc |\textit{child-code}| |[|\||else |\textit{main-code}]| \||fi|
\end{center}

%%%%%%%%%%%%%%%%%%%%%%%%%%%%%%%%%%%%%%%%
\DescribeMacro{\childdocname}
\DescribeMacro{\childdocjob}
The macro |\childdocname| contains the filename (without extension)
of the main or child file being processed.
Note that |\childdocjob| will always contain the name of the main file.

%%%%%%%%%%%%%%%%%%%%%%%%%%%%%%%%%%%%%%%%
\paragraph{Title Page.}

Conditional processing can be used to include a title or banner page
in the main document when proper precautions are taken.
Importantly, the code in the main file should ensure that the page counter
(as well as other status parameters which are stored in the |.aux| files)
takes the same value after the conditional processing.
Otherwise the page numbers may take divergent values
depending on which part is compiled.

For example, a title page could be declared by:
%
\begin{center}
\begin{tabular}{l}
|\ifchilddoc\||else|\\
|\addtocounter{page}{-1}|\\
\textit{code for title page}\\
|\newpage|\\
|\||fi|
\end{tabular}
\end{center}
%
A banner page for the child documents can be generated by:
%
\begin{center}
\begin{tabular}{l}
|\ifchilddoc|\\
|\addtocounter{page}{-1}|\\
\textit{code for banner page}\\
|\newpage|\\
|\||fi|
\end{tabular}
\end{center}
%
Here one could write a message such as:
\begin{center}
|This is the part \childdocname{} of \childdocjob{}.|
\end{center}

%%%%%%%%%%%%%%%%%%%%%%%%%%%%%%%%%%%%%%%%%%%%%%%%%%%%%%%%%%%%%%%%%%%%%%%%%%%%%%%%
\subsection{Flags}
\label{sec:flags}

The package makes it easy to generate different versions
of the main or child documents.
To this end compilation flags can be defined
and assigned different default values.
They will be particularly useful in conjunction
with the forwarding mechanism described in \secref{sec:forward}.

For example, it may be useful to have a flag |\version|
which can be set to |draft| or |final|.
The document source will contain some conditional code
depending on the value of |\version|.
Suppose further, the flag should default to |final| for the main file
and to |draft| for child files
which is a natural assignment for editing the document.
This is achieved by placing the following code
in the preamble of the main document
(below the |\childdocmain| directive):
%
\begin{center}
\begin{tabular}{l}
|\ifchilddoc|\\
|\providecommand{\version}{draft}|\\
|\||else|\\
|\providecommand{\version}{final}|\\
|\||fi|
\end{tabular}
\end{center}
%
The definition by |\providecommand| makes sure
that previous definitions are not overwritten.
Further statements |\providecommand{\version}{...}|
can thus be added before the above code to override it.

For the main file, one might add a line
(between |\childdocmain| and the above block)
%
\begin{center}
|%\ifchilddoc\||else\providecommand{\version}{draft}\||fi|
\end{center}
%
which can be uncommented to produce a draft version.
Likewise one can add a line to the very top of a child file
(above the |\childdocof{|\textit{main}|}| directive)
%
\begin{center}
|%\providecommand{\version}{final}|
\end{center}
%
which can be uncommented to produce the final version of this child document.

%%%%%%%%%%%%%%%%%%%%%%%%%%%%%%%%%%%%%%%%%%%%%%%%%%%%%%%%%%%%%%%%%%%%%%%%%%%%%%%%
\subsection{Forwarding}
\label{sec:forward}

Different versions of the main or child documents
using compilation flags as described in \secref{sec:flags}
can be (permanently) stored in different files
for convenient compilation, viewing and distribution.
To this end, the package defines a command
to pass on compilation to a different file:

%%%%%%%%%%%%%%%%%%%%%%%%%%%%%%%%%%%%%%%%
\DescribeMacro{\childdocforward}
The command |\childdocforward| redirects processing to
another source file:
%
\begin{center}
\begin{tabular}{l}
|\input{childdoc.def}|\\
|\childdocforward[|\textit{main}|]{|\textit{dest}|}|\\
\end{tabular}
\end{center}
%
The argument \textit{dest} is the destination file
(without extension).
It should be the main file or one of the child files.
Note that further \textsf{childdoc} directives
such as |\childdocof| and |\childdocforward|
in the indicated file will be processed in this form.
The optional argument \textit{main}
passes on directly to the main file \textit{main}
while pretending to compile the child \textit{dest}.
This form behaves as if \textit{dest}
issues |\childdocof{|\textit{main}|}| right away,
and no further \textsf{childdoc} directives will be processed.

%%%%%%%%%%%%%%%%%%%%%%%%%%%%%%%%%%%%%%%%
\DescribeMacro{\...prefix}
In the alternative form |\childdocforwardprefix|,
%
\begin{center}
\begin{tabular}{l}
|\input{childdoc.def}|\\
|\childdocforwardprefix[|\textit{main}|]{|\textit{prefix}|}{|\textit{dest}|}|
\end{tabular}
\end{center}
%
the destination file is determined by a pattern
depending on the current file:
To make this work, the current file must be called
`{\textit{prefix}\hspace{0.2em}\textit{suffix}}'
with \textit{prefix} matching precisely the argument.
Processing is then passed on to the file
`{\textit{dest}\hspace{0.2em}\textit{suffix}}'.
Surely, the same effect is achieved by
directly specifying the
argument `{\textit{dest}\hspace{0.2em}\textit{suffix}}'
in the first form.
However, that requires to set up a different file
for each child. With the alternative form of the command
all these files can have exactly the same content
which simplifies setting them up and maintaining them.

For example, the following file |draft.tex|
with a compilation flag |\version| as described in \secref{sec:flags}
compiles the main document as a draft:
%
\begin{center}
\begin{tabular}{l}
|\def\version{draft}|\\
|\input{childdoc.def}|\\
|\childdocforward{|\textit{main}|}|
\end{tabular}
\end{center}
%
Likewise, the following files |final|\textit{nn}|.tex|
compile the final version of the child document
|child|\textit{nn}|.tex|:
%
\begin{center}
\begin{tabular}{l}
|\def\version{final}|\\
|\input{childdoc.def}|\\
|\childdocforwardprefix{final}{child}|
\end{tabular}
\end{center}
%

Note that when several versions of a main file and/or of each child file
are to be generated, it may be convenient to set up a |Makefile| or
shell script to automatise the process.

%%%%%%%%%%%%%%%%%%%%%%%%%%%%%%%%%%%%%%%%%%%%%%%%%%%%%%%%%%%%%%%%%%%%%%%%%%%%%%%%
\subsection{Command Line Processing}
\label{sec:commandline}

The effect of redirection files can also be achieved by invoking
the \LaTeX{} compiler with a more elaborate command line.
Most conveniently this should be done as part
of a shell script or a |Makefile|.

When using \textsf{childdoc} in the main file, the following
command lines effectively perform a redirection
(note that depending on the shell being used,
backslashes may have to be doubled: `|\|' $\to$ `|\\|'):
%
\begin{center}
|... -jobname "|\textit{target}|" |\\|"|[\textit{flags}]%
|\input{childdoc.def}\childdocforward[|\textit{main}|]{|\textit{dest}|}"|
\end{center}
%
Here \textit{target} is the name of the output file,
\textit{main} is the name of the main file
and \textit{dest} is the name of the main or child file to be processed
(all filenames without extensions).
The optional argument \textit{main} can be omitted
if \textit{main} matches \textit{dest}.
Optionally, compilation \textit{flags} can be defined via |\def| commands.
This command line makes the \TeX{} engine believe
it is compiling the file \textit{target}
whose content is specified as the latter parameter.
The provided code then forwards the processing to
\textit{main} or \textit{dest} as described in \secref{sec:forward}.

%%%%%%%%%%%%%%%%%%%%%%%%%%%%%%%%%%%%%%%%%%%%%%%%%%%%%%%%%%%%%%%%%%%%%%%%%%%%%%%%
\subsection{Include by Input}
\label{sec:input}

Including child documents by |\include| has some restrictions by design.
Most notably, the content of a child document always occupies
its own set of pages; pages cannot be shared between child documents.
Usually, this behaviour makes perfect sense
because each child document contain an essential part of the document.
However, in some situations it may be desirable to compose
a document from a collection of parts
without having mandatory page breaks between then.
For this case, the package
provides a mechanism to include parts
by |\input| which can also be processed individually.
However, by construction this mechanism
requires manual handling of the content to be output.

%%%%%%%%%%%%%%%%%%%%%%%%%%%%%%%%%%%%%%%%
\DescribeMacro{\ifchilddocmanual}
The main file should be prepared as usual, see \secref{sec:include}.
However, the document body must make a distinction
between processing of an individual part and of the main document, e.g.:
%
\begin{center}
\begin{tabular}{l}
|\ifchilddocmanual|\\
|\input{\childdocname}|\\
|\||else|\\
\textit{document body with }|\input{|\textit{part}|}|\\
|\||fi|
\end{tabular}
\end{center}
%
The conditional |\ifchilddocmanual| is true whenever
a part to be included by |\input| is being compiled,
and the name of the part is stored in |\childdocname|.

%%%%%%%%%%%%%%%%%%%%%%%%%%%%%%%%%%%%%%%%
\DescribeMacro{\childdocby}
Each part to be included by |\input| should start with:
%
\begin{center}
\begin{tabular}{l}
|\input{childdoc.def}|\\
|\childdocby{|\textit{main}|}|\\
\end{tabular}
\end{center}
%
The directive |\childdocby| is similar to |\childdocof|
described in \secref{sec:include},
but the subsequent selection of content must be done manually.
To that end, both |\ifchilddoc| and |\ifchilddocmanual|
will be true upon processing of a part,
and the name of the part is stored in |\childdocname|.
Note that |\jobname| will be set to the filename of the current part
so that each part receives an individual |.aux| file
that does not interfere with the |.aux| file(s) of the main document.
This behaviour can be altered by the alternative form
|\childdocby[*]{|\textit{main}|}| (with a non-empty optional argument)
which uses the |.aux| file of the main document
by setting |\jobname| to \textit{main}.

%%%%%%%%%%%%%%%%%%%%%%%%%%%%%%%%%%%%%%%%%%%%%%%%%%%%%%%%%%%%%%%%%%%%%%%%%%%%%%%%
\subsection{Driver Development}
\label{sec:driver}

The \textsf{childdoc} mechanism can also be use for the development
of definition files such as \LaTeX{} styles or classes.
This case differs from the above setup with multiple parts
included by |\include| in that no |\includeonly| should be invoked.
This can be achieved by starting the include file
(before |\ProvidesPackage|) with:
%
\begin{center}
\begin{tabular}{l}
|\input{childdoc.def}|\\
|\childdocforward{|\textit{main}|}|\\
\end{tabular}
\end{center}
%
or alternatively with:
%
\begin{center}
\begin{tabular}{l}
|\input{childdoc.def}|\\
|\childdocby{|\textit{main}|}|\\
\end{tabular}
\end{center}
%
Both forms have slightly different effects as described above.
The main file is prepared as usual, see \secref{sec:include}.

%%%%%%%%%%%%%%%%%%%%%%%%%%%%%%%%%%%%%%%%%%%%%%%%%%%%%%%%%%%%%%%%%%%%%%%%%%%%%%%%
\subsection{Legacy Detection}
\label{sec:detection}

The directive |\childdocmain| in the main file can detect
whether the complete document or merely a child is to be compiled
even without using the directive |\childdocof|.
This method is deprecated because it is less robust
and there is no compelling reason to use it;
it is merely provided for backward compatibility
and it may be removed in future versions.

If the detection mechanism is to be used,
it is mandatory to correctly specify
the filename of the main file as the argument of |\childdocmain|:
%
\begin{center}
\begin{tabular}{l}
|\input{childdoc.def}|\\
|\childdocmain{|\textit{main}|}|\\
\end{tabular}
\end{center}
%
If |\jobname| does not match the argument \textit{main} of |\childdocmain|,
it is assumed that |\jobname| points to the child file to be compiled.
When using |\childdocmain| with the main file specified as argument,
it suffices to start a child file
with just |\input{|\textit{main}|}|
without loading of the package and using |\childdocof|.
If instead all processing is done
with the appropriate \textsf{childdoc} directives,
the argument of \textit{main} of |\childdocmain| can be empty.

An alternative version of the command line processing described
in \secref{sec:commandline} using the detection mechanism reads:
%
\begin{center}
|... -jobname "|\textit{target}|" "|[\textit{flags}]%
[|\def\jobname{|\textit{dest}|}|]|\input{|\textit{main}|}"|
\end{center}

%%%%%%%%%%%%%%%%%%%%%%%%%%%%%%%%%%%%%%%%%%%%%%%%%%%%%%%%%%%%%%%%%%%%%%%%%%%%%%%%
\subsection{Manual Code}
\label{sec:manual}

In case one cannot be certain whether the definitions file |childdoc.def|
is installed on the target \TeX{} distribution
and one prefers not to ship it,
it is conceivable to paste a few relevant commands into the sources.

To that end, drop all statements |\input{childdoc.def}|
and perform the replacements as outlined below.
Instead of |\childdocmain{|\textit{main}|}| add the following code
to the top of the main file:
%
\begin{center}
\begin{tabular}{l}
|\||ifdefined\childdocname\endinput\||fi\newif\ifchilddoc|\\
|\edef\childdocname{\scantokens\expandafter{\jobname\noexpand}}|\\
|\def\childdocmain{|\textit{main}|}\||ifx\childdocmain\childdocname\||else|\\
|\childdoctrue\includeonly{\childdocname}\let\jobname\childdocmain\||fi|\\
\end{tabular}
\end{center}
%
Instead of |\childdocof{|\textit{main}|}| just include the main file
at the top of each child file:
%
\begin{center}
|\input{|\textit{main}|}|
\end{center}
%
A simple redirection |\childdocforward{|\textit{dest}|}| is achieved by:
%
\begin{center}
|\def\jobname{|\textit{dest}|}\input{\jobname}|
\end{center}
%
The redirection with prefix
|\childdocforwardprefix[|\textit{prefix}|]{|\textit{dest}|}|
is accomplished by:
%
\begin{center}
\begin{tabular}{l}
|{\edef\jobname{\scantokens\expandafter{\jobname\noexpand}}|\\
|\def\redirectjob |\textit{prefix}|#1~~~{\gdef\jobname{|\textit{dest}|#1}}|\\
|\expandafter\redirectjob\jobname~~~}\input{\jobname}|
\end{tabular}
\end{center}

In an alternative approach,
child documents can be compiled by a specific command line
without additional code or specific definitions:
%
\begin{center}
|... -jobname "|\textit{target}|" "|[\textit{flags}]%
|\includeonly{|\textit{dest}|}\input{|\textit{main}|}"|
\end{center}
%

%%%%%%%%%%%%%%%%%%%%%%%%%%%%%%%%%%%%%%%%%%%%%%%%%%%%%%%%%%%%%%%%%%%%%%%%%%%%%%%%
%%%%%%%%%%%%%%%%%%%%%%%%%%%%%%%%%%%%%%%%%%%%%%%%%%%%%%%%%%%%%%%%%%%%%%%%%%%%%%%%
\section{Information}

%%%%%%%%%%%%%%%%%%%%%%%%%%%%%%%%%%%%%%%%%%%%%%%%%%%%%%%%%%%%%%%%%%%%%%%%%%%%%%%%
\subsection{Copyright}

Copyright \copyright{} 2017--2018 Niklas Beisert

This work may be distributed and/or modified under the
conditions of the \LaTeX{} Project Public License, either version 1.3
of this license or (at your option) any later version.
The latest version of this license is in
  \url{http://www.latex-project.org/lppl.txt}
and version 1.3 or later is part of all distributions of \LaTeX{}
version 2005/12/01 or later.

This work has the LPPL maintenance status `maintained'.

The Current Maintainer of this work is Niklas Beisert.

This work consists of the files |README.txt|, |childdoc.ins| and |childdoc.dtx|
as well as the derived files |childdoc.def|, |cdocsamp.tex|
with |cdocsch1.tex|, |cdocsch2.tex|, |cdocspt3.tex|, |cdocspt4.tex|,
|cdocsdrf.tex|, |cdocsfn1.tex|, |cdocsfn2.tex|
as well as |childdoc.pdf|.

%%%%%%%%%%%%%%%%%%%%%%%%%%%%%%%%%%%%%%%%%%%%%%%%%%%%%%%%%%%%%%%%%%%%%%%%%%%%%%%%
\subsection{Files and Installation}

The package consists of the files:
%
\begin{center}
\begin{tabular}{ll}
    |README.txt|   & readme file \\
    |childdoc.ins| & installation file \\
    |childdoc.dtx| & source file \\
    |childdoc.def| & definition file \\
    |cdocsamp.tex| & sample main file \\
    |cdocsch1.tex| & sample include file \\
    |cdocsch2.tex| & sample include file \\
    |cdocspt3.tex| & sample part file \\
    |cdocspt4.tex| & sample part file \\
    |cdocsdrf.tex| & sample redirection file \\
    |cdocsfn1.tex| & sample redirection file \\
    |cdocsfn2.tex| & sample redirection file \\
    |childdoc.pdf| & manual
\end{tabular}
\end{center}
%
The distribution consists of the files
|README.txt|, |childdoc.ins| and |childdoc.dtx|.
%
\begin{itemize}
\item
Run (pdf)\LaTeX{} on |childdoc.dtx|
to compile the manual |childdoc.pdf| (this file).
\item
Run \LaTeX{} on |childdoc.ins| to create the definitions file |childdoc.def|
and the sample |cdocsamp.tex| with include files
|cdocsch1.tex|, |cdocsch2.tex|, |cdocspt3.tex|, |cdocspt4.tex|,
|cdocsdrf.tex|, |cdocsfn1.tex|, |cdocsfn2.tex|.
Then copy the file |childdoc.def| to an appropriate directory of your \LaTeX{}
distribution, e.g.\ \textit{texmf-root}|/tex/latex/childdoc|.
\end{itemize}

%%%%%%%%%%%%%%%%%%%%%%%%%%%%%%%%%%%%%%%%%%%%%%%%%%%%%%%%%%%%%%%%%%%%%%%%%%%%%%%%
\subsection{Related CTAN Packages}

There are several other packages which offer a similar functionality:
%
\begin{itemize}
\item
The packages
\href{http://ctan.org/pkg/docmute}{\textsf{docmute}},
\href{http://ctan.org/pkg/includex}{\textsf{includex}} and
\href{http://ctan.org/pkg/standalone}{\textsf{standalone}}
provide commands to include only the document body of
a child file thus allowing both files to be compiled individually.
\item
The packages \href{http://ctan.org/pkg/subdocs}{\textsf{subdocs}}
and \href{http://ctan.org/pkg/subfiles}{\textsf{subfiles}}
provide structures in which the main and child documents can be
encapsulated and allowing them to be compiled individually.
The inclusion mechanism is different from the conventional |\include|.
\item
The package \href{http://ctan.org/pkg/combine}{\textsf{combine}}
is an elaborate solution to combine several documents into one.
\end{itemize}
%
See also the CTAN topic \href{http://ctan.org/topic/subdocs}{\textsf{subdocs}}
for further related packages.
The present package differs from the above solutions in that
a document structure constructed with the conventional |\include| mechanism
just needs two extra commands at the top of every file
such that all constituent files can be compiled individually.

%%%%%%%%%%%%%%%%%%%%%%%%%%%%%%%%%%%%%%%%%%%%%%%%%%%%%%%%%%%%%%%%%%%%%%%%%%%%%%%%
%\subsection{Feature Suggestions}
%
%The following is a list of features which may be useful for future
%versions of this package:
%%
%\begin{itemize}
%\item
%\ldots
%\end{itemize}

%%%%%%%%%%%%%%%%%%%%%%%%%%%%%%%%%%%%%%%%%%%%%%%%%%%%%%%%%%%%%%%%%%%%%%%%%%%%%%%%
\subsection{Revision History}

%%%%%%%%%%%%%%%%%%%%%%%%%%%%%%%%%%%%%%%%
\paragraph{v2.0:} 2018/12/30

\begin{itemize}
\item
immediate forward processing
\item
added |\childdocby| mechanism
\item
manual restructured
\end{itemize}

%%%%%%%%%%%%%%%%%%%%%%%%%%%%%%%%%%%%%%%%
\paragraph{v1.6:} 2018/01/17

\begin{itemize}
\item
application for development of include files
\item
corrections to manual
\end{itemize}

%%%%%%%%%%%%%%%%%%%%%%%%%%%%%%%%%%%%%%%%
\paragraph{v1.5:} 2017/05/21

\begin{itemize}
\item
more complete structuring introduced
\item
|\childdocof| introduced
\item
|\childdoc| renamed to |\childdocmain|
\item
|\childredirect| renamed to |\childdocforward| and |\childdocforwardprefix|
and functionality expanded
\end{itemize}

%%%%%%%%%%%%%%%%%%%%%%%%%%%%%%%%%%%%%%%%
\paragraph{v1.0:} 2017/04/27

\begin{itemize}
\item
manual and install package
\item
first version published on CTAN
\end{itemize}

%%%%%%%%%%%%%%%%%%%%%%%%%%%%%%%%%%%%%%%%
\paragraph{v0.6:} 2017/04/26

\begin{itemize}
\item
redirection mechanism added
\end{itemize}

%%%%%%%%%%%%%%%%%%%%%%%%%%%%%%%%%%%%%%%%
\paragraph{v0.5:} 2017/04/26

\begin{itemize}
\item
functionality in definition file
\end{itemize}


%%%%%%%%%%%%%%%%%%%%%%%%%%%%%%%%%%%%%%%%%%%%%%%%%%%%%%%%%%%%%%%%%%%%%%%%%%%%%%%%
%%%%%%%%%%%%%%%%%%%%%%%%%%%%%%%%%%%%%%%%%%%%%%%%%%%%%%%%%%%%%%%%%%%%%%%%%%%%%%%%
%%%%%%%%%%%%%%%%%%%%%%%%%%%%%%%%%%%%%%%%%%%%%%%%%%%%%%%%%%%%%%%%%%%%%%%%%%%%%%%%
\appendix

\settowidth\MacroIndent{\rmfamily\scriptsize 000\ }

 \DocInput{childdoc.dtx}

\end{document}
%</driver>
% \fi
%
% %%%%%%%%%%%%%%%%%%%%%%%%%%%%%%%%%%%%%%%%%%%%%%%%%%%%%%%%%%%%%%%%%%%%%%%%%%%%%%
% %%%%%%%%%%%%%%%%%%%%%%%%%%%%%%%%%%%%%%%%%%%%%%%%%%%%%%%%%%%%%%%%%%%%%%%%%%%%%%
% \section{Sample}
%\iffalse
%<*samplemain>
%\fi
%
% The following presents a sample document
% with two chapters, two parts, a title page,
% a compile flag as well as three forwarding files to set the flag.
% It consists of eight |.tex| files:
% \begin{center}
% \begin{tabular}{ll}
% |cdocsamp.tex|&main file\\
% |cdocsch1.tex|&include file for chapter 1\\
% |cdocsch2.tex|&include file for chapter 2\\
% |cdocspt3.tex|&include file for part 3\\
% |cdocspt4.tex|&include file for part 4\\
% |cdocsdrf.tex|&forwarding file for main file in draft mode\\
% |cdocsfi1.tex|&forwarding file for final version of chapter 1\\
% |cdocsfi2.tex|&forwarding file for final version of chapter 2\\
% \end{tabular}
% \end{center}
% Each of the eight files can be compiled directly by the \LaTeX{} compiler.
%
% %%%%%%%%%%%%%%%%%%%%%%%%%%%%%%%%%%%%%%
% \paragraph{Main File.}
%
% The main file is called |cdocsamp.tex|.
%
% Load the \textsf{childdoc} definitions and
% declare the filename for the main document:
%    \begin{macrocode}
\input{childdoc.def}
\childdocmain{}
%    \end{macrocode}

% Optional override for |\version| flag:
%    \begin{macrocode}
%%\ifchilddoc\else\providecommand{\version}{draft}\fi
%    \end{macrocode}

% Define the default values for the |\version| flag
% (|final| for the main file and |draft| for childs):
%    \begin{macrocode}
\ifchilddoc
\providecommand{\version}{draft}
\else
\providecommand{\version}{final}
\fi
%    \end{macrocode}

% Load the standard document class:
%    \begin{macrocode}
\documentclass[12pt]{article}
%    \end{macrocode}

% Start the document body:
%    \begin{macrocode}
\begin{document}
%    \end{macrocode}

% Declare a title page.
% Print title, part of document being processed and version flag:
%    \begin{macrocode}
\addtocounter{page}{-1}
\begin{center}
{\LARGE\bfseries{}childdoc example\par}
\vspace{1cm}
\ifchilddoc
\ifchilddocmanual part\else chapter\fi:
`\childdocname' of `\childdocjob'\par
\else
main document: `\childdocjob'\par
\fi
version: \version\par
\end{center}
\newpage
%    \end{macrocode}

% Manually include selected file,
% otherwise process as usual:
%    \begin{macrocode}
\ifchilddocmanual
\section*{part `\childdocname'}
\input{\childdocname}
\else
%    \end{macrocode}

% Include the two chapters:
%    \begin{macrocode}
\include{cdocsch1}
\include{cdocsch2}
%    \end{macrocode}

% Include the two parts unless only chapters should be displayed:
%    \begin{macrocode}
\ifchilddoc\else
\section{part three}
\input{cdocspt3}
\section{part four}
\input{cdocspt4}
\fi
%    \end{macrocode}

% Process as usual until here:
%    \begin{macrocode}
\fi
%    \end{macrocode}

% End of document body:
%    \begin{macrocode}
\end{document}
%    \end{macrocode}
%\iffalse
%</samplemain>
%\fi
%
% %%%%%%%%%%%%%%%%%%%%%%%%%%%%%%%%%%%%%%
% \paragraph{Chapter Include Files.}
%
% The include files are called |cdocsch1.tex| and |cdocsch2.tex|.
%
%\iffalse
%<*samplechap1|samplechap2>
%\fi

% Optional override for |\version| flag:
%    \begin{macrocode}
%%\providecommand{\version}{final}
%    \end{macrocode}

% Include the main document:
%    \begin{macrocode}
\input{childdoc.def}
\childdocof{cdocsamp}
%    \end{macrocode}

%\iffalse
%</samplechap1|samplechap2>
%\fi
%
%\iffalse
%<*samplechap1>
%\fi
% Some text for chapter 1:
%    \begin{macrocode}
\section{one}
some text in chapter one
%    \end{macrocode}

%\iffalse
%</samplechap1>
%\fi
% Some text for chapter 2:
%\iffalse
%<*samplechap2>
%\fi
%    \begin{macrocode}
\section{two}
more text in chapter two
%    \end{macrocode}

%\iffalse
%</samplechap2>
%\fi
%
% %%%%%%%%%%%%%%%%%%%%%%%%%%%%%%%%%%%%%%
% \paragraph{Part Include Files.}
%
% The include files are called |cdocspt3.tex| and |cdocspt4.tex|.
%
%\iffalse
%<*samplepart3|samplepart4>
%\fi

% Optional override for |\version| flag:
%    \begin{macrocode}
%%\providecommand{\version}{final}
%    \end{macrocode}

% Include the main document:
%    \begin{macrocode}
\input{childdoc.def}
\childdocby{cdocsamp}
%    \end{macrocode}

%\iffalse
%</samplepart3|samplepart4>
%\fi
%
%\iffalse
%<*samplepart3>
%\fi
% Some text for part 3:
%    \begin{macrocode}
some text in part three
%    \end{macrocode}

%\iffalse
%</samplepart3>
%\fi
% Some text for part 4:
%\iffalse
%<*samplepart4>
%\fi
%    \begin{macrocode}
more text in part four
%    \end{macrocode}

%\iffalse
%</samplepart4>
%\fi
%
% %%%%%%%%%%%%%%%%%%%%%%%%%%%%%%%%%%%%%%
% \paragraph{Forwarding for a Complete Draft.}
%
% The following forwarding file |cdocsdrf.tex|
% compiles the main document in draft mode:
%\iffalse
%<*sampledraft>
%\fi
%    \begin{macrocode}
\def\version{draft}
\input{childdoc.def}
\childdocforward{cdocsamp}
%    \end{macrocode}

%\iffalse
%</sampledraft>
%\fi
%
% %%%%%%%%%%%%%%%%%%%%%%%%%%%%%%%%%%%%%%
% \paragraph{Forwarding for Final Version of the Chapters.}
%
% The following forwarding files |cdocsfn1.tex| and |cdocsfn2.tex|
% (with identical content)
% compile the final versions of the child documents
% |cdocsch1.tex| and |cdocsch2.tex|, respectively:
%\iffalse
%<*samplefinal>
%\fi
%    \begin{macrocode}
\def\version{final}
\input{childdoc.def}
\childdocforwardprefix[cdocsamp]{cdocsfn}{cdocsch}
%    \end{macrocode}

%\iffalse
%</samplefinal>
%\fi
%
% %%%%%%%%%%%%%%%%%%%%%%%%%%%%%%%%%%%%%%
% \paragraph{Command Line Processing.}
%
% The following three command lines generate the output files
% |cdocscld|, |cdocscl1| and |cdocscl2|
% which should be identical to
% |cdocsdrf|, |cdocsch1| and |cdocsfn2|, respectively:
% \begin{center}
% \begin{tabular}{l}
% |latex -jobname cdocscld \|\\
% |  "\def\version{draft}\input{childdoc.def}\childdocforward{cdocsamp}"|\\
% |latex -jobname cdocscl1 \|\\
% |  "\input{childdoc.def}\childdocforward[cdocsamp]{cdocsch1}"|\\
% |latex -jobname cdocscl2 \|\\
% |  "\def\version{final}\input{childdoc.def}\childdocforward{cdocsch2}"|
% \end{tabular}
% \end{center}
% Note that the trailing backslash on each first line
% merely continues the input to the second line
% (for convenient cut ant paste).
% Furthermore, the command |latex| can be replaced by any
% of its alternative versions such as |pdflatex|.
%
% %%%%%%%%%%%%%%%%%%%%%%%%%%%%%%%%%%%%%%%%%%%%%%%%%%%%%%%%%%%%%%%%%%%%%%%%%%%%%%
% %%%%%%%%%%%%%%%%%%%%%%%%%%%%%%%%%%%%%%%%%%%%%%%%%%%%%%%%%%%%%%%%%%%%%%%%%%%%%%
% \section{Implementation}
%\iffalse
%<*package>
%\fi
%
% This section describes the definitions file |childdoc.def|.

% The definitions cannot be loaded using |\usepackage| or |\RequirePackage|
% which has a mechanism to prevent loading a style file more than once.
% When loading the definitions by means of |\input|
% multiple instances have to be prevented manually:
%\iffalse
%This code needs to be before the `\ProvidesFile' directive
%which is defined at the beginning of this file.
%Therefore it is also placed there and commented out here.
%</package>
%<*discard>
%\fi
%    \begin{macrocode}
\ifdefined\childdocmain\endinput\fi
%    \end{macrocode}
%\iffalse
%</discard>
%<*package>
%\fi
%
% \macro{\ifchilddoc}
% \macro{\ifchilddocmanual}
% The conditional |\ifchilddoc| tells whether a
% child (true) or main (false) document is being compiled.
% The conditional |\ifchilddocmanual| tells whether
% the |\includeonly| mechanism is used (false) or
% the selection of child files must be performed manually (true).
% The definitions initialise to false:
%    \begin{macrocode}
\newif\ifchilddoc
\newif\ifchilddocmanual
%    \end{macrocode}

% \macro{\childdocname}
% \macro{\childdocjob}
% The macro |\childdocname| stores the name of the main document
% to be compiled. The macro |\childdocjob| stores the name of
% the document on which the \LaTeX{} compiler was originally invoked.
% The content of |\jobname| cannot be compared
% to filenames specified in the source due to different catcodes.
% The following code rescans |\jobname|, stores the result
% in |\childdocname| and saves a copy in |\childdocjob|:
%    \begin{macrocode}
\edef\childdocname{\scantokens\expandafter{\jobname\noexpand}}
\let\childdocjob\childdocname
%    \end{macrocode}

% \macro{\childdocdisable}
% The macro |\childdocdisable| prevents the main file
% from being processed more than once.
% At this stage, the main document command |\childdocmain|
% is assumed to be called once again where it should do nothing.
% Any subsequent call to it should prevent
% a secondary processing of the main document
% It overwrites the forwarding commands
% |\childdocof| and |\childdocforward|
% with empty macros to prevent further inclusions of the main document:
%    \begin{macrocode}
\newcommand{\childdocdisable}
{
  \renewcommand{\childdocmain}[1]{\renewcommand{\childdocmain}[1]{\endinput}}
  \renewcommand{\childdocof}[1]{}
  \renewcommand{\childdocby}[2][]{}
  \renewcommand{\childdocforward}[2][]{}
  \renewcommand{\childdocdisable}{}
}
%    \end{macrocode}

% \macro{\childdocmain}
% The macro |\childdocmain| is to be called at the top of the main file
% with nothing or the main filename (without extension) as argument.
% First, it breaks loops.
% If the argument is not empty and does not match |\childdocname|
% (which is set by the first inclusion of |childdoc.def|),
% |\ifchilddoc| is set to true, |\includeonly| is applied to the child file
% and |\jobname| is set to the main file
% (for proper handling of |.aux| files):
%    \begin{macrocode}
\newcommand{\childdocmain}[1]
{
  \childdocdisable\childdocmain{}
  \if?#1?\else
    \begingroup
      \def\childdoctmp{#1}
      \ifx\childdoctmp\childdocname
        \def\childdoctmp{}
      \else
        \def\childdoctmp
        {
          \childdoctrue
          \includeonly{\childdocname}
          \def\childdocjob{#1}
          \def\jobname{#1}
        }
      \fi
      \expandafter
    \endgroup
    \childdoctmp
  \fi
}
%    \end{macrocode}

% \macro{\childdocof}
% The command |\childdocof| redirects
% compilation to the main file |#1|.
%    \begin{macrocode}
\newcommand{\childdocof}[1]
{
  \childdocdisable
  \childdoctrue
  \includeonly{\childdocname}
  \def\jobname{#1}
  \def\childdocjob{#1}
  \input{#1}
}
%    \end{macrocode}

% \macro{\childdocby}
% The command |\childdocby| ....
%    \begin{macrocode}
\newcommand{\childdocby}[2][]
{
  \childdocdisable
  \childdoctrue
  \childdocmanualtrue
  \if?#1?\else
    \def\jobname{#2}
  \fi
  \def\childdocjob{#2}
  \input{#2}
  \endinput
}
%    \end{macrocode}

% \macro{\childdocforward}
% The command |\childdocforward| redirects
% compilation to the main file or
% (if the optional argument is given) a child file.
% Parameters are set as if the main file
% or a child file starting with |\childdocof| was compiled.
% Then compilation is handed over to the main file:
%    \begin{macrocode}
\newcommand{\childdocforward}[2][]
{
  \begingroup
    \if?#1?
      \def\childdoctmp
      {
        \def\childdocname{#2}
        \def\childdocjob{#2}
        \def\jobname{#2}
        \input{#2}
        \endinput
      }
    \else
      \def\childdoctmp
      {
        \childdocdisable
        \def\childdocname{#2}
        \childdoctrue
        \includeonly{#2}
        \def\childdocjob{#1}
        \def\jobname{#1}
        \input{#1}
        \endinput
      }
    \fi
    \expandafter
  \endgroup
  \childdoctmp
}
%    \end{macrocode}

% \macro{\childdocforwardprefix}
% The command |\childdocforwardprefix| redirects
% compilation to the main or a child file by means of a pattern.
% The prefix |#1| in the current filename is replaced by |#2|
% and the suffix of the current filename is kept
% (it is assumed that the filename does not contain the substring `|~~~|'
% which is used as a delimiter).
% Compilation is handed over to the new file by |\childdocforward|:
%    \begin{macrocode}
\newcommand{\childdocforwardprefix}[3][]
{
  \begingroup
    \def\childdocextract #2##1~~~{\def\childdoctmp{\childdocforward[#1]{#3##1}}}
    \expandafter\childdocextract\childdocname~~~
    \expandafter
  \endgroup
  \childdoctmp
}
%    \end{macrocode}

% \macro{\childdoc}
% The deprecated macro |\childdoc| is a legacy version of |\childdocmain|:
%    \begin{macrocode}
\newcommand{\childdoc}{\childdocmain}
%    \end{macrocode}

% \macro{\childdocredirect}
% The deprecated macro |\childdocredirect| is a legacy version
% of |\childdocforward| and |\childdocforwardprefix|:
%    \begin{macrocode}
\newcommand{\childdocredirect}[2][]
{
  \begingroup
    \if?#1?
      \def\childdoctmp{\childdocforward{#2}}
    \else
      \def\childdoctmp{\childdocforwardprefix{#1}{#2}}
    \fi
    \expandafter
  \endgroup
  \childdoctmp
}
%    \end{macrocode}

%\iffalse
%</package>
%\fi
%
\endinput
|
and perform the replacements as outlined below.
Instead of |\childdocmain{|\textit{main}|}| add the following code
to the top of the main file:
%
\begin{center}
\begin{tabular}{l}
|\||ifdefined\childdocname\endinput\||fi\newif\ifchilddoc|\\
|\edef\childdocname{\scantokens\expandafter{\jobname\noexpand}}|\\
|\def\childdocmain{|\textit{main}|}\||ifx\childdocmain\childdocname\||else|\\
|\childdoctrue\includeonly{\childdocname}\let\jobname\childdocmain\||fi|\\
\end{tabular}
\end{center}
%
Instead of |\childdocof{|\textit{main}|}| just include the main file
at the top of each child file:
%
\begin{center}
|\input{|\textit{main}|}|
\end{center}
%
A simple redirection |\childdocforward{|\textit{dest}|}| is achieved by:
%
\begin{center}
|\def\jobname{|\textit{dest}|}\input{\jobname}|
\end{center}
%
The redirection with prefix
|\childdocforwardprefix[|\textit{prefix}|]{|\textit{dest}|}|
is accomplished by:
%
\begin{center}
\begin{tabular}{l}
|{\edef\jobname{\scantokens\expandafter{\jobname\noexpand}}|\\
|\def\redirectjob |\textit{prefix}|#1~~~{\gdef\jobname{|\textit{dest}|#1}}|\\
|\expandafter\redirectjob\jobname~~~}\input{\jobname}|
\end{tabular}
\end{center}

In an alternative approach,
child documents can be compiled by a specific command line
without additional code or specific definitions:
%
\begin{center}
|... -jobname "|\textit{target}|" "|[\textit{flags}]%
|\includeonly{|\textit{dest}|}\input{|\textit{main}|}"|
\end{center}
%

%%%%%%%%%%%%%%%%%%%%%%%%%%%%%%%%%%%%%%%%%%%%%%%%%%%%%%%%%%%%%%%%%%%%%%%%%%%%%%%%
%%%%%%%%%%%%%%%%%%%%%%%%%%%%%%%%%%%%%%%%%%%%%%%%%%%%%%%%%%%%%%%%%%%%%%%%%%%%%%%%
\section{Information}

%%%%%%%%%%%%%%%%%%%%%%%%%%%%%%%%%%%%%%%%%%%%%%%%%%%%%%%%%%%%%%%%%%%%%%%%%%%%%%%%
\subsection{Copyright}

Copyright \copyright{} 2017--2018 Niklas Beisert

This work may be distributed and/or modified under the
conditions of the \LaTeX{} Project Public License, either version 1.3
of this license or (at your option) any later version.
The latest version of this license is in
  \url{http://www.latex-project.org/lppl.txt}
and version 1.3 or later is part of all distributions of \LaTeX{}
version 2005/12/01 or later.

This work has the LPPL maintenance status `maintained'.

The Current Maintainer of this work is Niklas Beisert.

This work consists of the files |README.txt|, |childdoc.ins| and |childdoc.dtx|
as well as the derived files |childdoc.def|, |cdocsamp.tex|
with |cdocsch1.tex|, |cdocsch2.tex|, |cdocspt3.tex|, |cdocspt4.tex|,
|cdocsdrf.tex|, |cdocsfn1.tex|, |cdocsfn2.tex|
as well as |childdoc.pdf|.

%%%%%%%%%%%%%%%%%%%%%%%%%%%%%%%%%%%%%%%%%%%%%%%%%%%%%%%%%%%%%%%%%%%%%%%%%%%%%%%%
\subsection{Files and Installation}

The package consists of the files:
%
\begin{center}
\begin{tabular}{ll}
    |README.txt|   & readme file \\
    |childdoc.ins| & installation file \\
    |childdoc.dtx| & source file \\
    |childdoc.def| & definition file \\
    |cdocsamp.tex| & sample main file \\
    |cdocsch1.tex| & sample include file \\
    |cdocsch2.tex| & sample include file \\
    |cdocspt3.tex| & sample part file \\
    |cdocspt4.tex| & sample part file \\
    |cdocsdrf.tex| & sample redirection file \\
    |cdocsfn1.tex| & sample redirection file \\
    |cdocsfn2.tex| & sample redirection file \\
    |childdoc.pdf| & manual
\end{tabular}
\end{center}
%
The distribution consists of the files
|README.txt|, |childdoc.ins| and |childdoc.dtx|.
%
\begin{itemize}
\item
Run (pdf)\LaTeX{} on |childdoc.dtx|
to compile the manual |childdoc.pdf| (this file).
\item
Run \LaTeX{} on |childdoc.ins| to create the definitions file |childdoc.def|
and the sample |cdocsamp.tex| with include files
|cdocsch1.tex|, |cdocsch2.tex|, |cdocspt3.tex|, |cdocspt4.tex|,
|cdocsdrf.tex|, |cdocsfn1.tex|, |cdocsfn2.tex|.
Then copy the file |childdoc.def| to an appropriate directory of your \LaTeX{}
distribution, e.g.\ \textit{texmf-root}|/tex/latex/childdoc|.
\end{itemize}

%%%%%%%%%%%%%%%%%%%%%%%%%%%%%%%%%%%%%%%%%%%%%%%%%%%%%%%%%%%%%%%%%%%%%%%%%%%%%%%%
\subsection{Related CTAN Packages}

There are several other packages which offer a similar functionality:
%
\begin{itemize}
\item
The packages
\href{http://ctan.org/pkg/docmute}{\textsf{docmute}},
\href{http://ctan.org/pkg/includex}{\textsf{includex}} and
\href{http://ctan.org/pkg/standalone}{\textsf{standalone}}
provide commands to include only the document body of
a child file thus allowing both files to be compiled individually.
\item
The packages \href{http://ctan.org/pkg/subdocs}{\textsf{subdocs}}
and \href{http://ctan.org/pkg/subfiles}{\textsf{subfiles}}
provide structures in which the main and child documents can be
encapsulated and allowing them to be compiled individually.
The inclusion mechanism is different from the conventional |\include|.
\item
The package \href{http://ctan.org/pkg/combine}{\textsf{combine}}
is an elaborate solution to combine several documents into one.
\end{itemize}
%
See also the CTAN topic \href{http://ctan.org/topic/subdocs}{\textsf{subdocs}}
for further related packages.
The present package differs from the above solutions in that
a document structure constructed with the conventional |\include| mechanism
just needs two extra commands at the top of every file
such that all constituent files can be compiled individually.

%%%%%%%%%%%%%%%%%%%%%%%%%%%%%%%%%%%%%%%%%%%%%%%%%%%%%%%%%%%%%%%%%%%%%%%%%%%%%%%%
%\subsection{Feature Suggestions}
%
%The following is a list of features which may be useful for future
%versions of this package:
%%
%\begin{itemize}
%\item
%\ldots
%\end{itemize}

%%%%%%%%%%%%%%%%%%%%%%%%%%%%%%%%%%%%%%%%%%%%%%%%%%%%%%%%%%%%%%%%%%%%%%%%%%%%%%%%
\subsection{Revision History}

%%%%%%%%%%%%%%%%%%%%%%%%%%%%%%%%%%%%%%%%
\paragraph{v2.0:} 2018/12/30

\begin{itemize}
\item
immediate forward processing
\item
added |\childdocby| mechanism
\item
manual restructured
\end{itemize}

%%%%%%%%%%%%%%%%%%%%%%%%%%%%%%%%%%%%%%%%
\paragraph{v1.6:} 2018/01/17

\begin{itemize}
\item
application for development of include files
\item
corrections to manual
\end{itemize}

%%%%%%%%%%%%%%%%%%%%%%%%%%%%%%%%%%%%%%%%
\paragraph{v1.5:} 2017/05/21

\begin{itemize}
\item
more complete structuring introduced
\item
|\childdocof| introduced
\item
|\childdoc| renamed to |\childdocmain|
\item
|\childredirect| renamed to |\childdocforward| and |\childdocforwardprefix|
and functionality expanded
\end{itemize}

%%%%%%%%%%%%%%%%%%%%%%%%%%%%%%%%%%%%%%%%
\paragraph{v1.0:} 2017/04/27

\begin{itemize}
\item
manual and install package
\item
first version published on CTAN
\end{itemize}

%%%%%%%%%%%%%%%%%%%%%%%%%%%%%%%%%%%%%%%%
\paragraph{v0.6:} 2017/04/26

\begin{itemize}
\item
redirection mechanism added
\end{itemize}

%%%%%%%%%%%%%%%%%%%%%%%%%%%%%%%%%%%%%%%%
\paragraph{v0.5:} 2017/04/26

\begin{itemize}
\item
functionality in definition file
\end{itemize}


%%%%%%%%%%%%%%%%%%%%%%%%%%%%%%%%%%%%%%%%%%%%%%%%%%%%%%%%%%%%%%%%%%%%%%%%%%%%%%%%
%%%%%%%%%%%%%%%%%%%%%%%%%%%%%%%%%%%%%%%%%%%%%%%%%%%%%%%%%%%%%%%%%%%%%%%%%%%%%%%%
%%%%%%%%%%%%%%%%%%%%%%%%%%%%%%%%%%%%%%%%%%%%%%%%%%%%%%%%%%%%%%%%%%%%%%%%%%%%%%%%
\appendix

\settowidth\MacroIndent{\rmfamily\scriptsize 000\ }

 \DocInput{childdoc.dtx}

\end{document}
%</driver>
% \fi
%
% %%%%%%%%%%%%%%%%%%%%%%%%%%%%%%%%%%%%%%%%%%%%%%%%%%%%%%%%%%%%%%%%%%%%%%%%%%%%%%
% %%%%%%%%%%%%%%%%%%%%%%%%%%%%%%%%%%%%%%%%%%%%%%%%%%%%%%%%%%%%%%%%%%%%%%%%%%%%%%
% \section{Sample}
%\iffalse
%<*samplemain>
%\fi
%
% The following presents a sample document
% with two chapters, two parts, a title page,
% a compile flag as well as three forwarding files to set the flag.
% It consists of eight |.tex| files:
% \begin{center}
% \begin{tabular}{ll}
% |cdocsamp.tex|&main file\\
% |cdocsch1.tex|&include file for chapter 1\\
% |cdocsch2.tex|&include file for chapter 2\\
% |cdocspt3.tex|&include file for part 3\\
% |cdocspt4.tex|&include file for part 4\\
% |cdocsdrf.tex|&forwarding file for main file in draft mode\\
% |cdocsfi1.tex|&forwarding file for final version of chapter 1\\
% |cdocsfi2.tex|&forwarding file for final version of chapter 2\\
% \end{tabular}
% \end{center}
% Each of the eight files can be compiled directly by the \LaTeX{} compiler.
%
% %%%%%%%%%%%%%%%%%%%%%%%%%%%%%%%%%%%%%%
% \paragraph{Main File.}
%
% The main file is called |cdocsamp.tex|.
%
% Load the \textsf{childdoc} definitions and
% declare the filename for the main document:
%    \begin{macrocode}
% \iffalse
%
% childdoc.dtx Copyright (C) 2017-2018 Niklas Beisert
%
% This work may be distributed and/or modified under the
% conditions of the LaTeX Project Public License, either version 1.3
% of this license or (at your option) any later version.
% The latest version of this license is in
%   http://www.latex-project.org/lppl.txt
% and version 1.3 or later is part of all distributions of LaTeX
% version 2005/12/01 or later.
%
% This work has the LPPL maintenance status `maintained'.
%
% The Current Maintainer of this work is Niklas Beisert.
%
% This work consists of the files childdoc.dtx and childdoc.ins
% and the derived files childdoc.def and cdocsamp.tex with
% cdocsch1.tex, cdocsch2.tex, cdocsdrf.tex, cdocsfn1.tex, cdocsfn2.tex.
%
%<package>\ifdefined\childdocmain\endinput\fi
%<package>\ProvidesFile{childdoc.def}[2018/12/30 v2.0 child document driver]
%<samplemain>\ProvidesFile{cdocsamp.tex}[2018/12/30 v2.0 sample for childdoc]
%<*driver>
%\ProvidesFile{childdoc.drv}[2018/12/30 v2.0 childdoc reference manual file]
\PassOptionsToClass{10pt,a4paper}{article}
\documentclass{ltxdoc}

\usepackage[margin=35mm]{geometry}
\usepackage{hyperref}
\usepackage{hyperxmp}
\usepackage[usenames]{color}

\hypersetup{colorlinks=true}
\hypersetup{pdfstartview=FitH}
\hypersetup{pdfpagemode=UseNone}
\hypersetup{pdfsource={}}
\hypersetup{pdflang={en-UK}}
\hypersetup{pdfcopyright={Copyright 2017-2018 Niklas Beisert.
  This work may be distributed and/or modified under the
  conditions of the LaTeX Project Public License, either version 1.3
  of this license or (at your option) any later version.}}
\hypersetup{pdflicenseurl={http://www.latex-project.org/lppl.txt}}
\hypersetup{pdfcontactaddress={ETH Zurich, ITP, HIT K,
  Wolfgang-Pauli-Strasse 27}}
\hypersetup{pdfcontactpostcode={8093}}
\hypersetup{pdfcontactcity={Zurich}}
\hypersetup{pdfcontactcountry={Switzerland}}
\hypersetup{pdfcontactemail={nbeisert@itp.phys.ethz.ch}}
\hypersetup{pdfcontacturl={http://people.phys.ethz.ch/\xmptilde nbeisert/}}

\newcommand{\secref}[1]{\hyperref[#1]{section \ref*{#1}}}

\parskip1ex
\parindent0pt
\let\olditemize\itemize
\def\itemize{\olditemize\parskip0pt}

\begin{document}

\title{The \textsf{childdoc} Package}
\hypersetup{pdftitle={The childdoc Package}}
\author{Niklas Beisert\\[2ex]
  Institut f\"ur Theoretische Physik\\
  Eidgen\"ossische Technische Hochschule Z\"urich\\
  Wolfgang-Pauli-Strasse 27, 8093 Z\"urich, Switzerland\\[1ex]
  \href{mailto:nbeisert@itp.phys.ethz.ch}
  {\texttt{nbeisert@itp.phys.ethz.ch}}}
\hypersetup{pdfauthor={Niklas Beisert}}
\hypersetup{pdfsubject={Manual for the LaTeX2e Package childdoc}}
\date{30 December 2018, \textsf{v2.0}}
\maketitle

\begin{abstract}\noindent
\textsf{childdoc} is a \LaTeXe{} package
that enables the direct compilation
of document sections included by |\include|
to individual files.
\end{abstract}

\begingroup
\parskip0ex
\tableofcontents
\endgroup

%%%%%%%%%%%%%%%%%%%%%%%%%%%%%%%%%%%%%%%%%%%%%%%%%%%%%%%%%%%%%%%%%%%%%%%%%%%%%%%%
%%%%%%%%%%%%%%%%%%%%%%%%%%%%%%%%%%%%%%%%%%%%%%%%%%%%%%%%%%%%%%%%%%%%%%%%%%%%%%%%
\section{Introduction}

\LaTeX{} provides a mechanism to structure a large document (such as a book)
into a main file and several child files (containing the chapters)
using the |\include| command.
This mechanism is beneficial for documents
which span hundreds of pages in order to
make the source file(s) more manageable.
Moreover, compilation can be restricted to
selected child files by means of the |\includeonly| command.
The latter feature can be used to reduce the compilation time while editing
(this was significantly more useful in the earlier days of \LaTeX{})
or to generate a smaller document which is easier to navigate.
Another application of |\includeonly| is to generate
documents consisting of selected parts of the complete document.

However, there are a few drawbacks of the plain |\include| mechanism:
\begin{itemize}
\item
The child files cannot be compiled on their own,
they can only be compiled via the main file.
A naive editing environment
(such as a text editor with an option
to have the current file processed by \LaTeX)
may require one to switch to the main file before compiling;
attempting to compile the child file produces errors.
\item
The main file must be modified (each time)
to adjust the |\includeonly| command
to the present needs. This easily leaves the main file in a messy state.
\item
The generated document will always carry the filename
of the main document. This is inconvenient if
several child files are to be compiled and
to be kept for distribution.
\end{itemize}

The present package provides a simple interface
to make child files individually compilable by \LaTeX{}.
Compiling a child file then has the same effect as compiling
the main file with an |\includeonly| command
to select the appropriate child.
Moreover the generated document will carry the name of the child
rather than the main file.
This resolves all three above issues.

This feature is meant to make the editing of books,
thesis documents and lecture notes somewhat more convenient.
However, the package can also be used efficiently for
composing a series of documents (such as exercise sheets)
which are typically distributed individually.
It then assists the author in generating the individual documents
(potentially in different versions)
as well as a document containing the collected series.
Another application is in developing style files
or other kinds of included material
where compilation of the style file could redirect
to a sample or test file.

%%%%%%%%%%%%%%%%%%%%%%%%%%%%%%%%%%%%%%%%%%%%%%%%%%%%%%%%%%%%%%%%%%%%%%%%%%%%%%%%
%%%%%%%%%%%%%%%%%%%%%%%%%%%%%%%%%%%%%%%%%%%%%%%%%%%%%%%%%%%%%%%%%%%%%%%%%%%%%%%%
\section{Usage}

First of all, the package \textsf{childdoc} is \emph{not} a standard
\LaTeXe{} |.sty| style file! Therefore it needs to be invoked in
a non-standard way.

%%%%%%%%%%%%%%%%%%%%%%%%%%%%%%%%%%%%%%%%%%%%%%%%%%%%%%%%%%%%%%%%%%%%%%%%%%%%%%%%
\subsection{Included Files}
\label{sec:include}

%%%%%%%%%%%%%%%%%%%%%%%%%%%%%%%%%%%%%%%%
\DescribeMacro{\childdocmain}
To use the package, add the commands
\begin{center}
\begin{tabular}{l}
|\input{childdoc.def}|\\
|\childdocmain{}|\\
\end{tabular}
\end{center}
at the very top of the main \LaTeX{} file,
in particular \emph{before} the |\documentclass| statement!
The argument of |\childdocmain| should be left empty
(but it must be present).

%%%%%%%%%%%%%%%%%%%%%%%%%%%%%%%%%%%%%%%%
\DescribeMacro{\childdocof}
Furthermore, add the commands
\begin{center}
\begin{tabular}{l}
|\input{childdoc.def}|\\
|\childdocof{|\textit{main}|}|\\
\end{tabular}
\end{center}
at the top of every child file \textit{child}
which is included by |\include{|\textit{child}|}|
from within the main file
(or at least for those files to be compiled individually).
The argument \textit{main} must be the filename of the main file.

There are a couple of
considerations in setting up the main and child documents:

%%%%%%%%%%%%%%%%%%%%%%%%%%%%%%%%%%%%%%%%
\paragraph{Restrictions.}

Please note the following restrictions:
\begin{itemize}
\item
|\childdocmain| must be called with one argument \textit{main}
to ensure compatibility with earlier version of the package.
It must either be empty (|\childdocmain{}|)
or precisely match the filename of the main file in which it is specified.
See \secref{sec:detection} for further information.
\item
The filename \textit{main} must be specified without the |.tex| extension.
\item
The filename \textit{main} is case sensitive
(even in case-insensitive file systems)
due to internal string comparison.
\item
The argument \textit{main} should be fully expanded, it cannot be a macro.
\item
Subdirectories and special characters should be avoided in filenames.
\item
The command |\childdocmain{|\textit{main}|}| must be followed by a whitespace.
It should not be followed immediately by another command
or by a comment mark `|%|'.
This is because the \TeX{} parser reads the token immediately following
the argument of |\childdocmain| and puts it
at the beginning of every child section;
however, a white\-space is ignored.
\end{itemize}

%%%%%%%%%%%%%%%%%%%%%%%%%%%%%%%%%%%%%%%%
\paragraph{Content of Main File.}

It is advisable to place all content in the child files included by |\include|.
Any output contained in the main file will appear in all child documents
unless suppressed manually;
it cannot be suppressed automatically by the |\includeonly| directive
and thus should normally be avoided.
A method to include some content in the main file
by means of conditional processing is described in \secref{sec:conditional}.

%%%%%%%%%%%%%%%%%%%%%%%%%%%%%%%%%%%%%%%%
\paragraph{Page Numbering.}

When only a part of the document is compiled,
the appropriate numbering of pages
(as well as other status parameters)
is determined from the |.aux| files.
The latter contain information from previous passes.
However this information needs to propagate through
all intermediate child documents.
Therefore the page numbering in child documents may well
be inconsistent until the complete document is compiled at least once.

A useful (if unconventional) way to always ensure a consistent
page numbering is to restart the numbering in each child document
and denote the pages by `\textit{child}|.|\textit{page}'
where \textit{child} represents the chapter/section number of the child file.
This can be achieved by the command
|\numberwithin{page}{|\textit{child}|}|
of the \textsf{amsmath} package
where \textit{child} can be |chapter| or |section|
depending on the chosen structuring.
Alternatively, one can modify the macro |\thepage| appropriately
and reset the counter |page| at the start of each child file.

%%%%%%%%%%%%%%%%%%%%%%%%%%%%%%%%%%%%%%%%%%%%%%%%%%%%%%%%%%%%%%%%%%%%%%%%%%%%%%%%
\subsection{Conditional Processing}
\label{sec:conditional}

The package provides a mechanism to compile different versions
of a document. To customise the versions further some conditional processing
can come in handy to distinguish which version is being compiled.
The package provides two macros to describe the compilation context:

%%%%%%%%%%%%%%%%%%%%%%%%%%%%%%%%%%%%%%%%
\DescribeMacro{\ifchilddoc}
The conditional |\ifchilddoc| distinguishes between the compilation of
child documents and the main document:
%
\begin{center}
|\ifchilddoc |\textit{child-code}| |[|\||else |\textit{main-code}]| \||fi|
\end{center}

%%%%%%%%%%%%%%%%%%%%%%%%%%%%%%%%%%%%%%%%
\DescribeMacro{\childdocname}
\DescribeMacro{\childdocjob}
The macro |\childdocname| contains the filename (without extension)
of the main or child file being processed.
Note that |\childdocjob| will always contain the name of the main file.

%%%%%%%%%%%%%%%%%%%%%%%%%%%%%%%%%%%%%%%%
\paragraph{Title Page.}

Conditional processing can be used to include a title or banner page
in the main document when proper precautions are taken.
Importantly, the code in the main file should ensure that the page counter
(as well as other status parameters which are stored in the |.aux| files)
takes the same value after the conditional processing.
Otherwise the page numbers may take divergent values
depending on which part is compiled.

For example, a title page could be declared by:
%
\begin{center}
\begin{tabular}{l}
|\ifchilddoc\||else|\\
|\addtocounter{page}{-1}|\\
\textit{code for title page}\\
|\newpage|\\
|\||fi|
\end{tabular}
\end{center}
%
A banner page for the child documents can be generated by:
%
\begin{center}
\begin{tabular}{l}
|\ifchilddoc|\\
|\addtocounter{page}{-1}|\\
\textit{code for banner page}\\
|\newpage|\\
|\||fi|
\end{tabular}
\end{center}
%
Here one could write a message such as:
\begin{center}
|This is the part \childdocname{} of \childdocjob{}.|
\end{center}

%%%%%%%%%%%%%%%%%%%%%%%%%%%%%%%%%%%%%%%%%%%%%%%%%%%%%%%%%%%%%%%%%%%%%%%%%%%%%%%%
\subsection{Flags}
\label{sec:flags}

The package makes it easy to generate different versions
of the main or child documents.
To this end compilation flags can be defined
and assigned different default values.
They will be particularly useful in conjunction
with the forwarding mechanism described in \secref{sec:forward}.

For example, it may be useful to have a flag |\version|
which can be set to |draft| or |final|.
The document source will contain some conditional code
depending on the value of |\version|.
Suppose further, the flag should default to |final| for the main file
and to |draft| for child files
which is a natural assignment for editing the document.
This is achieved by placing the following code
in the preamble of the main document
(below the |\childdocmain| directive):
%
\begin{center}
\begin{tabular}{l}
|\ifchilddoc|\\
|\providecommand{\version}{draft}|\\
|\||else|\\
|\providecommand{\version}{final}|\\
|\||fi|
\end{tabular}
\end{center}
%
The definition by |\providecommand| makes sure
that previous definitions are not overwritten.
Further statements |\providecommand{\version}{...}|
can thus be added before the above code to override it.

For the main file, one might add a line
(between |\childdocmain| and the above block)
%
\begin{center}
|%\ifchilddoc\||else\providecommand{\version}{draft}\||fi|
\end{center}
%
which can be uncommented to produce a draft version.
Likewise one can add a line to the very top of a child file
(above the |\childdocof{|\textit{main}|}| directive)
%
\begin{center}
|%\providecommand{\version}{final}|
\end{center}
%
which can be uncommented to produce the final version of this child document.

%%%%%%%%%%%%%%%%%%%%%%%%%%%%%%%%%%%%%%%%%%%%%%%%%%%%%%%%%%%%%%%%%%%%%%%%%%%%%%%%
\subsection{Forwarding}
\label{sec:forward}

Different versions of the main or child documents
using compilation flags as described in \secref{sec:flags}
can be (permanently) stored in different files
for convenient compilation, viewing and distribution.
To this end, the package defines a command
to pass on compilation to a different file:

%%%%%%%%%%%%%%%%%%%%%%%%%%%%%%%%%%%%%%%%
\DescribeMacro{\childdocforward}
The command |\childdocforward| redirects processing to
another source file:
%
\begin{center}
\begin{tabular}{l}
|\input{childdoc.def}|\\
|\childdocforward[|\textit{main}|]{|\textit{dest}|}|\\
\end{tabular}
\end{center}
%
The argument \textit{dest} is the destination file
(without extension).
It should be the main file or one of the child files.
Note that further \textsf{childdoc} directives
such as |\childdocof| and |\childdocforward|
in the indicated file will be processed in this form.
The optional argument \textit{main}
passes on directly to the main file \textit{main}
while pretending to compile the child \textit{dest}.
This form behaves as if \textit{dest}
issues |\childdocof{|\textit{main}|}| right away,
and no further \textsf{childdoc} directives will be processed.

%%%%%%%%%%%%%%%%%%%%%%%%%%%%%%%%%%%%%%%%
\DescribeMacro{\...prefix}
In the alternative form |\childdocforwardprefix|,
%
\begin{center}
\begin{tabular}{l}
|\input{childdoc.def}|\\
|\childdocforwardprefix[|\textit{main}|]{|\textit{prefix}|}{|\textit{dest}|}|
\end{tabular}
\end{center}
%
the destination file is determined by a pattern
depending on the current file:
To make this work, the current file must be called
`{\textit{prefix}\hspace{0.2em}\textit{suffix}}'
with \textit{prefix} matching precisely the argument.
Processing is then passed on to the file
`{\textit{dest}\hspace{0.2em}\textit{suffix}}'.
Surely, the same effect is achieved by
directly specifying the
argument `{\textit{dest}\hspace{0.2em}\textit{suffix}}'
in the first form.
However, that requires to set up a different file
for each child. With the alternative form of the command
all these files can have exactly the same content
which simplifies setting them up and maintaining them.

For example, the following file |draft.tex|
with a compilation flag |\version| as described in \secref{sec:flags}
compiles the main document as a draft:
%
\begin{center}
\begin{tabular}{l}
|\def\version{draft}|\\
|\input{childdoc.def}|\\
|\childdocforward{|\textit{main}|}|
\end{tabular}
\end{center}
%
Likewise, the following files |final|\textit{nn}|.tex|
compile the final version of the child document
|child|\textit{nn}|.tex|:
%
\begin{center}
\begin{tabular}{l}
|\def\version{final}|\\
|\input{childdoc.def}|\\
|\childdocforwardprefix{final}{child}|
\end{tabular}
\end{center}
%

Note that when several versions of a main file and/or of each child file
are to be generated, it may be convenient to set up a |Makefile| or
shell script to automatise the process.

%%%%%%%%%%%%%%%%%%%%%%%%%%%%%%%%%%%%%%%%%%%%%%%%%%%%%%%%%%%%%%%%%%%%%%%%%%%%%%%%
\subsection{Command Line Processing}
\label{sec:commandline}

The effect of redirection files can also be achieved by invoking
the \LaTeX{} compiler with a more elaborate command line.
Most conveniently this should be done as part
of a shell script or a |Makefile|.

When using \textsf{childdoc} in the main file, the following
command lines effectively perform a redirection
(note that depending on the shell being used,
backslashes may have to be doubled: `|\|' $\to$ `|\\|'):
%
\begin{center}
|... -jobname "|\textit{target}|" |\\|"|[\textit{flags}]%
|\input{childdoc.def}\childdocforward[|\textit{main}|]{|\textit{dest}|}"|
\end{center}
%
Here \textit{target} is the name of the output file,
\textit{main} is the name of the main file
and \textit{dest} is the name of the main or child file to be processed
(all filenames without extensions).
The optional argument \textit{main} can be omitted
if \textit{main} matches \textit{dest}.
Optionally, compilation \textit{flags} can be defined via |\def| commands.
This command line makes the \TeX{} engine believe
it is compiling the file \textit{target}
whose content is specified as the latter parameter.
The provided code then forwards the processing to
\textit{main} or \textit{dest} as described in \secref{sec:forward}.

%%%%%%%%%%%%%%%%%%%%%%%%%%%%%%%%%%%%%%%%%%%%%%%%%%%%%%%%%%%%%%%%%%%%%%%%%%%%%%%%
\subsection{Include by Input}
\label{sec:input}

Including child documents by |\include| has some restrictions by design.
Most notably, the content of a child document always occupies
its own set of pages; pages cannot be shared between child documents.
Usually, this behaviour makes perfect sense
because each child document contain an essential part of the document.
However, in some situations it may be desirable to compose
a document from a collection of parts
without having mandatory page breaks between then.
For this case, the package
provides a mechanism to include parts
by |\input| which can also be processed individually.
However, by construction this mechanism
requires manual handling of the content to be output.

%%%%%%%%%%%%%%%%%%%%%%%%%%%%%%%%%%%%%%%%
\DescribeMacro{\ifchilddocmanual}
The main file should be prepared as usual, see \secref{sec:include}.
However, the document body must make a distinction
between processing of an individual part and of the main document, e.g.:
%
\begin{center}
\begin{tabular}{l}
|\ifchilddocmanual|\\
|\input{\childdocname}|\\
|\||else|\\
\textit{document body with }|\input{|\textit{part}|}|\\
|\||fi|
\end{tabular}
\end{center}
%
The conditional |\ifchilddocmanual| is true whenever
a part to be included by |\input| is being compiled,
and the name of the part is stored in |\childdocname|.

%%%%%%%%%%%%%%%%%%%%%%%%%%%%%%%%%%%%%%%%
\DescribeMacro{\childdocby}
Each part to be included by |\input| should start with:
%
\begin{center}
\begin{tabular}{l}
|\input{childdoc.def}|\\
|\childdocby{|\textit{main}|}|\\
\end{tabular}
\end{center}
%
The directive |\childdocby| is similar to |\childdocof|
described in \secref{sec:include},
but the subsequent selection of content must be done manually.
To that end, both |\ifchilddoc| and |\ifchilddocmanual|
will be true upon processing of a part,
and the name of the part is stored in |\childdocname|.
Note that |\jobname| will be set to the filename of the current part
so that each part receives an individual |.aux| file
that does not interfere with the |.aux| file(s) of the main document.
This behaviour can be altered by the alternative form
|\childdocby[*]{|\textit{main}|}| (with a non-empty optional argument)
which uses the |.aux| file of the main document
by setting |\jobname| to \textit{main}.

%%%%%%%%%%%%%%%%%%%%%%%%%%%%%%%%%%%%%%%%%%%%%%%%%%%%%%%%%%%%%%%%%%%%%%%%%%%%%%%%
\subsection{Driver Development}
\label{sec:driver}

The \textsf{childdoc} mechanism can also be use for the development
of definition files such as \LaTeX{} styles or classes.
This case differs from the above setup with multiple parts
included by |\include| in that no |\includeonly| should be invoked.
This can be achieved by starting the include file
(before |\ProvidesPackage|) with:
%
\begin{center}
\begin{tabular}{l}
|\input{childdoc.def}|\\
|\childdocforward{|\textit{main}|}|\\
\end{tabular}
\end{center}
%
or alternatively with:
%
\begin{center}
\begin{tabular}{l}
|\input{childdoc.def}|\\
|\childdocby{|\textit{main}|}|\\
\end{tabular}
\end{center}
%
Both forms have slightly different effects as described above.
The main file is prepared as usual, see \secref{sec:include}.

%%%%%%%%%%%%%%%%%%%%%%%%%%%%%%%%%%%%%%%%%%%%%%%%%%%%%%%%%%%%%%%%%%%%%%%%%%%%%%%%
\subsection{Legacy Detection}
\label{sec:detection}

The directive |\childdocmain| in the main file can detect
whether the complete document or merely a child is to be compiled
even without using the directive |\childdocof|.
This method is deprecated because it is less robust
and there is no compelling reason to use it;
it is merely provided for backward compatibility
and it may be removed in future versions.

If the detection mechanism is to be used,
it is mandatory to correctly specify
the filename of the main file as the argument of |\childdocmain|:
%
\begin{center}
\begin{tabular}{l}
|\input{childdoc.def}|\\
|\childdocmain{|\textit{main}|}|\\
\end{tabular}
\end{center}
%
If |\jobname| does not match the argument \textit{main} of |\childdocmain|,
it is assumed that |\jobname| points to the child file to be compiled.
When using |\childdocmain| with the main file specified as argument,
it suffices to start a child file
with just |\input{|\textit{main}|}|
without loading of the package and using |\childdocof|.
If instead all processing is done
with the appropriate \textsf{childdoc} directives,
the argument of \textit{main} of |\childdocmain| can be empty.

An alternative version of the command line processing described
in \secref{sec:commandline} using the detection mechanism reads:
%
\begin{center}
|... -jobname "|\textit{target}|" "|[\textit{flags}]%
[|\def\jobname{|\textit{dest}|}|]|\input{|\textit{main}|}"|
\end{center}

%%%%%%%%%%%%%%%%%%%%%%%%%%%%%%%%%%%%%%%%%%%%%%%%%%%%%%%%%%%%%%%%%%%%%%%%%%%%%%%%
\subsection{Manual Code}
\label{sec:manual}

In case one cannot be certain whether the definitions file |childdoc.def|
is installed on the target \TeX{} distribution
and one prefers not to ship it,
it is conceivable to paste a few relevant commands into the sources.

To that end, drop all statements |\input{childdoc.def}|
and perform the replacements as outlined below.
Instead of |\childdocmain{|\textit{main}|}| add the following code
to the top of the main file:
%
\begin{center}
\begin{tabular}{l}
|\||ifdefined\childdocname\endinput\||fi\newif\ifchilddoc|\\
|\edef\childdocname{\scantokens\expandafter{\jobname\noexpand}}|\\
|\def\childdocmain{|\textit{main}|}\||ifx\childdocmain\childdocname\||else|\\
|\childdoctrue\includeonly{\childdocname}\let\jobname\childdocmain\||fi|\\
\end{tabular}
\end{center}
%
Instead of |\childdocof{|\textit{main}|}| just include the main file
at the top of each child file:
%
\begin{center}
|\input{|\textit{main}|}|
\end{center}
%
A simple redirection |\childdocforward{|\textit{dest}|}| is achieved by:
%
\begin{center}
|\def\jobname{|\textit{dest}|}\input{\jobname}|
\end{center}
%
The redirection with prefix
|\childdocforwardprefix[|\textit{prefix}|]{|\textit{dest}|}|
is accomplished by:
%
\begin{center}
\begin{tabular}{l}
|{\edef\jobname{\scantokens\expandafter{\jobname\noexpand}}|\\
|\def\redirectjob |\textit{prefix}|#1~~~{\gdef\jobname{|\textit{dest}|#1}}|\\
|\expandafter\redirectjob\jobname~~~}\input{\jobname}|
\end{tabular}
\end{center}

In an alternative approach,
child documents can be compiled by a specific command line
without additional code or specific definitions:
%
\begin{center}
|... -jobname "|\textit{target}|" "|[\textit{flags}]%
|\includeonly{|\textit{dest}|}\input{|\textit{main}|}"|
\end{center}
%

%%%%%%%%%%%%%%%%%%%%%%%%%%%%%%%%%%%%%%%%%%%%%%%%%%%%%%%%%%%%%%%%%%%%%%%%%%%%%%%%
%%%%%%%%%%%%%%%%%%%%%%%%%%%%%%%%%%%%%%%%%%%%%%%%%%%%%%%%%%%%%%%%%%%%%%%%%%%%%%%%
\section{Information}

%%%%%%%%%%%%%%%%%%%%%%%%%%%%%%%%%%%%%%%%%%%%%%%%%%%%%%%%%%%%%%%%%%%%%%%%%%%%%%%%
\subsection{Copyright}

Copyright \copyright{} 2017--2018 Niklas Beisert

This work may be distributed and/or modified under the
conditions of the \LaTeX{} Project Public License, either version 1.3
of this license or (at your option) any later version.
The latest version of this license is in
  \url{http://www.latex-project.org/lppl.txt}
and version 1.3 or later is part of all distributions of \LaTeX{}
version 2005/12/01 or later.

This work has the LPPL maintenance status `maintained'.

The Current Maintainer of this work is Niklas Beisert.

This work consists of the files |README.txt|, |childdoc.ins| and |childdoc.dtx|
as well as the derived files |childdoc.def|, |cdocsamp.tex|
with |cdocsch1.tex|, |cdocsch2.tex|, |cdocspt3.tex|, |cdocspt4.tex|,
|cdocsdrf.tex|, |cdocsfn1.tex|, |cdocsfn2.tex|
as well as |childdoc.pdf|.

%%%%%%%%%%%%%%%%%%%%%%%%%%%%%%%%%%%%%%%%%%%%%%%%%%%%%%%%%%%%%%%%%%%%%%%%%%%%%%%%
\subsection{Files and Installation}

The package consists of the files:
%
\begin{center}
\begin{tabular}{ll}
    |README.txt|   & readme file \\
    |childdoc.ins| & installation file \\
    |childdoc.dtx| & source file \\
    |childdoc.def| & definition file \\
    |cdocsamp.tex| & sample main file \\
    |cdocsch1.tex| & sample include file \\
    |cdocsch2.tex| & sample include file \\
    |cdocspt3.tex| & sample part file \\
    |cdocspt4.tex| & sample part file \\
    |cdocsdrf.tex| & sample redirection file \\
    |cdocsfn1.tex| & sample redirection file \\
    |cdocsfn2.tex| & sample redirection file \\
    |childdoc.pdf| & manual
\end{tabular}
\end{center}
%
The distribution consists of the files
|README.txt|, |childdoc.ins| and |childdoc.dtx|.
%
\begin{itemize}
\item
Run (pdf)\LaTeX{} on |childdoc.dtx|
to compile the manual |childdoc.pdf| (this file).
\item
Run \LaTeX{} on |childdoc.ins| to create the definitions file |childdoc.def|
and the sample |cdocsamp.tex| with include files
|cdocsch1.tex|, |cdocsch2.tex|, |cdocspt3.tex|, |cdocspt4.tex|,
|cdocsdrf.tex|, |cdocsfn1.tex|, |cdocsfn2.tex|.
Then copy the file |childdoc.def| to an appropriate directory of your \LaTeX{}
distribution, e.g.\ \textit{texmf-root}|/tex/latex/childdoc|.
\end{itemize}

%%%%%%%%%%%%%%%%%%%%%%%%%%%%%%%%%%%%%%%%%%%%%%%%%%%%%%%%%%%%%%%%%%%%%%%%%%%%%%%%
\subsection{Related CTAN Packages}

There are several other packages which offer a similar functionality:
%
\begin{itemize}
\item
The packages
\href{http://ctan.org/pkg/docmute}{\textsf{docmute}},
\href{http://ctan.org/pkg/includex}{\textsf{includex}} and
\href{http://ctan.org/pkg/standalone}{\textsf{standalone}}
provide commands to include only the document body of
a child file thus allowing both files to be compiled individually.
\item
The packages \href{http://ctan.org/pkg/subdocs}{\textsf{subdocs}}
and \href{http://ctan.org/pkg/subfiles}{\textsf{subfiles}}
provide structures in which the main and child documents can be
encapsulated and allowing them to be compiled individually.
The inclusion mechanism is different from the conventional |\include|.
\item
The package \href{http://ctan.org/pkg/combine}{\textsf{combine}}
is an elaborate solution to combine several documents into one.
\end{itemize}
%
See also the CTAN topic \href{http://ctan.org/topic/subdocs}{\textsf{subdocs}}
for further related packages.
The present package differs from the above solutions in that
a document structure constructed with the conventional |\include| mechanism
just needs two extra commands at the top of every file
such that all constituent files can be compiled individually.

%%%%%%%%%%%%%%%%%%%%%%%%%%%%%%%%%%%%%%%%%%%%%%%%%%%%%%%%%%%%%%%%%%%%%%%%%%%%%%%%
%\subsection{Feature Suggestions}
%
%The following is a list of features which may be useful for future
%versions of this package:
%%
%\begin{itemize}
%\item
%\ldots
%\end{itemize}

%%%%%%%%%%%%%%%%%%%%%%%%%%%%%%%%%%%%%%%%%%%%%%%%%%%%%%%%%%%%%%%%%%%%%%%%%%%%%%%%
\subsection{Revision History}

%%%%%%%%%%%%%%%%%%%%%%%%%%%%%%%%%%%%%%%%
\paragraph{v2.0:} 2018/12/30

\begin{itemize}
\item
immediate forward processing
\item
added |\childdocby| mechanism
\item
manual restructured
\end{itemize}

%%%%%%%%%%%%%%%%%%%%%%%%%%%%%%%%%%%%%%%%
\paragraph{v1.6:} 2018/01/17

\begin{itemize}
\item
application for development of include files
\item
corrections to manual
\end{itemize}

%%%%%%%%%%%%%%%%%%%%%%%%%%%%%%%%%%%%%%%%
\paragraph{v1.5:} 2017/05/21

\begin{itemize}
\item
more complete structuring introduced
\item
|\childdocof| introduced
\item
|\childdoc| renamed to |\childdocmain|
\item
|\childredirect| renamed to |\childdocforward| and |\childdocforwardprefix|
and functionality expanded
\end{itemize}

%%%%%%%%%%%%%%%%%%%%%%%%%%%%%%%%%%%%%%%%
\paragraph{v1.0:} 2017/04/27

\begin{itemize}
\item
manual and install package
\item
first version published on CTAN
\end{itemize}

%%%%%%%%%%%%%%%%%%%%%%%%%%%%%%%%%%%%%%%%
\paragraph{v0.6:} 2017/04/26

\begin{itemize}
\item
redirection mechanism added
\end{itemize}

%%%%%%%%%%%%%%%%%%%%%%%%%%%%%%%%%%%%%%%%
\paragraph{v0.5:} 2017/04/26

\begin{itemize}
\item
functionality in definition file
\end{itemize}


%%%%%%%%%%%%%%%%%%%%%%%%%%%%%%%%%%%%%%%%%%%%%%%%%%%%%%%%%%%%%%%%%%%%%%%%%%%%%%%%
%%%%%%%%%%%%%%%%%%%%%%%%%%%%%%%%%%%%%%%%%%%%%%%%%%%%%%%%%%%%%%%%%%%%%%%%%%%%%%%%
%%%%%%%%%%%%%%%%%%%%%%%%%%%%%%%%%%%%%%%%%%%%%%%%%%%%%%%%%%%%%%%%%%%%%%%%%%%%%%%%
\appendix

\settowidth\MacroIndent{\rmfamily\scriptsize 000\ }

 \DocInput{childdoc.dtx}

\end{document}
%</driver>
% \fi
%
% %%%%%%%%%%%%%%%%%%%%%%%%%%%%%%%%%%%%%%%%%%%%%%%%%%%%%%%%%%%%%%%%%%%%%%%%%%%%%%
% %%%%%%%%%%%%%%%%%%%%%%%%%%%%%%%%%%%%%%%%%%%%%%%%%%%%%%%%%%%%%%%%%%%%%%%%%%%%%%
% \section{Sample}
%\iffalse
%<*samplemain>
%\fi
%
% The following presents a sample document
% with two chapters, two parts, a title page,
% a compile flag as well as three forwarding files to set the flag.
% It consists of eight |.tex| files:
% \begin{center}
% \begin{tabular}{ll}
% |cdocsamp.tex|&main file\\
% |cdocsch1.tex|&include file for chapter 1\\
% |cdocsch2.tex|&include file for chapter 2\\
% |cdocspt3.tex|&include file for part 3\\
% |cdocspt4.tex|&include file for part 4\\
% |cdocsdrf.tex|&forwarding file for main file in draft mode\\
% |cdocsfi1.tex|&forwarding file for final version of chapter 1\\
% |cdocsfi2.tex|&forwarding file for final version of chapter 2\\
% \end{tabular}
% \end{center}
% Each of the eight files can be compiled directly by the \LaTeX{} compiler.
%
% %%%%%%%%%%%%%%%%%%%%%%%%%%%%%%%%%%%%%%
% \paragraph{Main File.}
%
% The main file is called |cdocsamp.tex|.
%
% Load the \textsf{childdoc} definitions and
% declare the filename for the main document:
%    \begin{macrocode}
\input{childdoc.def}
\childdocmain{}
%    \end{macrocode}

% Optional override for |\version| flag:
%    \begin{macrocode}
%%\ifchilddoc\else\providecommand{\version}{draft}\fi
%    \end{macrocode}

% Define the default values for the |\version| flag
% (|final| for the main file and |draft| for childs):
%    \begin{macrocode}
\ifchilddoc
\providecommand{\version}{draft}
\else
\providecommand{\version}{final}
\fi
%    \end{macrocode}

% Load the standard document class:
%    \begin{macrocode}
\documentclass[12pt]{article}
%    \end{macrocode}

% Start the document body:
%    \begin{macrocode}
\begin{document}
%    \end{macrocode}

% Declare a title page.
% Print title, part of document being processed and version flag:
%    \begin{macrocode}
\addtocounter{page}{-1}
\begin{center}
{\LARGE\bfseries{}childdoc example\par}
\vspace{1cm}
\ifchilddoc
\ifchilddocmanual part\else chapter\fi:
`\childdocname' of `\childdocjob'\par
\else
main document: `\childdocjob'\par
\fi
version: \version\par
\end{center}
\newpage
%    \end{macrocode}

% Manually include selected file,
% otherwise process as usual:
%    \begin{macrocode}
\ifchilddocmanual
\section*{part `\childdocname'}
\input{\childdocname}
\else
%    \end{macrocode}

% Include the two chapters:
%    \begin{macrocode}
\include{cdocsch1}
\include{cdocsch2}
%    \end{macrocode}

% Include the two parts unless only chapters should be displayed:
%    \begin{macrocode}
\ifchilddoc\else
\section{part three}
\input{cdocspt3}
\section{part four}
\input{cdocspt4}
\fi
%    \end{macrocode}

% Process as usual until here:
%    \begin{macrocode}
\fi
%    \end{macrocode}

% End of document body:
%    \begin{macrocode}
\end{document}
%    \end{macrocode}
%\iffalse
%</samplemain>
%\fi
%
% %%%%%%%%%%%%%%%%%%%%%%%%%%%%%%%%%%%%%%
% \paragraph{Chapter Include Files.}
%
% The include files are called |cdocsch1.tex| and |cdocsch2.tex|.
%
%\iffalse
%<*samplechap1|samplechap2>
%\fi

% Optional override for |\version| flag:
%    \begin{macrocode}
%%\providecommand{\version}{final}
%    \end{macrocode}

% Include the main document:
%    \begin{macrocode}
\input{childdoc.def}
\childdocof{cdocsamp}
%    \end{macrocode}

%\iffalse
%</samplechap1|samplechap2>
%\fi
%
%\iffalse
%<*samplechap1>
%\fi
% Some text for chapter 1:
%    \begin{macrocode}
\section{one}
some text in chapter one
%    \end{macrocode}

%\iffalse
%</samplechap1>
%\fi
% Some text for chapter 2:
%\iffalse
%<*samplechap2>
%\fi
%    \begin{macrocode}
\section{two}
more text in chapter two
%    \end{macrocode}

%\iffalse
%</samplechap2>
%\fi
%
% %%%%%%%%%%%%%%%%%%%%%%%%%%%%%%%%%%%%%%
% \paragraph{Part Include Files.}
%
% The include files are called |cdocspt3.tex| and |cdocspt4.tex|.
%
%\iffalse
%<*samplepart3|samplepart4>
%\fi

% Optional override for |\version| flag:
%    \begin{macrocode}
%%\providecommand{\version}{final}
%    \end{macrocode}

% Include the main document:
%    \begin{macrocode}
\input{childdoc.def}
\childdocby{cdocsamp}
%    \end{macrocode}

%\iffalse
%</samplepart3|samplepart4>
%\fi
%
%\iffalse
%<*samplepart3>
%\fi
% Some text for part 3:
%    \begin{macrocode}
some text in part three
%    \end{macrocode}

%\iffalse
%</samplepart3>
%\fi
% Some text for part 4:
%\iffalse
%<*samplepart4>
%\fi
%    \begin{macrocode}
more text in part four
%    \end{macrocode}

%\iffalse
%</samplepart4>
%\fi
%
% %%%%%%%%%%%%%%%%%%%%%%%%%%%%%%%%%%%%%%
% \paragraph{Forwarding for a Complete Draft.}
%
% The following forwarding file |cdocsdrf.tex|
% compiles the main document in draft mode:
%\iffalse
%<*sampledraft>
%\fi
%    \begin{macrocode}
\def\version{draft}
\input{childdoc.def}
\childdocforward{cdocsamp}
%    \end{macrocode}

%\iffalse
%</sampledraft>
%\fi
%
% %%%%%%%%%%%%%%%%%%%%%%%%%%%%%%%%%%%%%%
% \paragraph{Forwarding for Final Version of the Chapters.}
%
% The following forwarding files |cdocsfn1.tex| and |cdocsfn2.tex|
% (with identical content)
% compile the final versions of the child documents
% |cdocsch1.tex| and |cdocsch2.tex|, respectively:
%\iffalse
%<*samplefinal>
%\fi
%    \begin{macrocode}
\def\version{final}
\input{childdoc.def}
\childdocforwardprefix[cdocsamp]{cdocsfn}{cdocsch}
%    \end{macrocode}

%\iffalse
%</samplefinal>
%\fi
%
% %%%%%%%%%%%%%%%%%%%%%%%%%%%%%%%%%%%%%%
% \paragraph{Command Line Processing.}
%
% The following three command lines generate the output files
% |cdocscld|, |cdocscl1| and |cdocscl2|
% which should be identical to
% |cdocsdrf|, |cdocsch1| and |cdocsfn2|, respectively:
% \begin{center}
% \begin{tabular}{l}
% |latex -jobname cdocscld \|\\
% |  "\def\version{draft}\input{childdoc.def}\childdocforward{cdocsamp}"|\\
% |latex -jobname cdocscl1 \|\\
% |  "\input{childdoc.def}\childdocforward[cdocsamp]{cdocsch1}"|\\
% |latex -jobname cdocscl2 \|\\
% |  "\def\version{final}\input{childdoc.def}\childdocforward{cdocsch2}"|
% \end{tabular}
% \end{center}
% Note that the trailing backslash on each first line
% merely continues the input to the second line
% (for convenient cut ant paste).
% Furthermore, the command |latex| can be replaced by any
% of its alternative versions such as |pdflatex|.
%
% %%%%%%%%%%%%%%%%%%%%%%%%%%%%%%%%%%%%%%%%%%%%%%%%%%%%%%%%%%%%%%%%%%%%%%%%%%%%%%
% %%%%%%%%%%%%%%%%%%%%%%%%%%%%%%%%%%%%%%%%%%%%%%%%%%%%%%%%%%%%%%%%%%%%%%%%%%%%%%
% \section{Implementation}
%\iffalse
%<*package>
%\fi
%
% This section describes the definitions file |childdoc.def|.

% The definitions cannot be loaded using |\usepackage| or |\RequirePackage|
% which has a mechanism to prevent loading a style file more than once.
% When loading the definitions by means of |\input|
% multiple instances have to be prevented manually:
%\iffalse
%This code needs to be before the `\ProvidesFile' directive
%which is defined at the beginning of this file.
%Therefore it is also placed there and commented out here.
%</package>
%<*discard>
%\fi
%    \begin{macrocode}
\ifdefined\childdocmain\endinput\fi
%    \end{macrocode}
%\iffalse
%</discard>
%<*package>
%\fi
%
% \macro{\ifchilddoc}
% \macro{\ifchilddocmanual}
% The conditional |\ifchilddoc| tells whether a
% child (true) or main (false) document is being compiled.
% The conditional |\ifchilddocmanual| tells whether
% the |\includeonly| mechanism is used (false) or
% the selection of child files must be performed manually (true).
% The definitions initialise to false:
%    \begin{macrocode}
\newif\ifchilddoc
\newif\ifchilddocmanual
%    \end{macrocode}

% \macro{\childdocname}
% \macro{\childdocjob}
% The macro |\childdocname| stores the name of the main document
% to be compiled. The macro |\childdocjob| stores the name of
% the document on which the \LaTeX{} compiler was originally invoked.
% The content of |\jobname| cannot be compared
% to filenames specified in the source due to different catcodes.
% The following code rescans |\jobname|, stores the result
% in |\childdocname| and saves a copy in |\childdocjob|:
%    \begin{macrocode}
\edef\childdocname{\scantokens\expandafter{\jobname\noexpand}}
\let\childdocjob\childdocname
%    \end{macrocode}

% \macro{\childdocdisable}
% The macro |\childdocdisable| prevents the main file
% from being processed more than once.
% At this stage, the main document command |\childdocmain|
% is assumed to be called once again where it should do nothing.
% Any subsequent call to it should prevent
% a secondary processing of the main document
% It overwrites the forwarding commands
% |\childdocof| and |\childdocforward|
% with empty macros to prevent further inclusions of the main document:
%    \begin{macrocode}
\newcommand{\childdocdisable}
{
  \renewcommand{\childdocmain}[1]{\renewcommand{\childdocmain}[1]{\endinput}}
  \renewcommand{\childdocof}[1]{}
  \renewcommand{\childdocby}[2][]{}
  \renewcommand{\childdocforward}[2][]{}
  \renewcommand{\childdocdisable}{}
}
%    \end{macrocode}

% \macro{\childdocmain}
% The macro |\childdocmain| is to be called at the top of the main file
% with nothing or the main filename (without extension) as argument.
% First, it breaks loops.
% If the argument is not empty and does not match |\childdocname|
% (which is set by the first inclusion of |childdoc.def|),
% |\ifchilddoc| is set to true, |\includeonly| is applied to the child file
% and |\jobname| is set to the main file
% (for proper handling of |.aux| files):
%    \begin{macrocode}
\newcommand{\childdocmain}[1]
{
  \childdocdisable\childdocmain{}
  \if?#1?\else
    \begingroup
      \def\childdoctmp{#1}
      \ifx\childdoctmp\childdocname
        \def\childdoctmp{}
      \else
        \def\childdoctmp
        {
          \childdoctrue
          \includeonly{\childdocname}
          \def\childdocjob{#1}
          \def\jobname{#1}
        }
      \fi
      \expandafter
    \endgroup
    \childdoctmp
  \fi
}
%    \end{macrocode}

% \macro{\childdocof}
% The command |\childdocof| redirects
% compilation to the main file |#1|.
%    \begin{macrocode}
\newcommand{\childdocof}[1]
{
  \childdocdisable
  \childdoctrue
  \includeonly{\childdocname}
  \def\jobname{#1}
  \def\childdocjob{#1}
  \input{#1}
}
%    \end{macrocode}

% \macro{\childdocby}
% The command |\childdocby| ....
%    \begin{macrocode}
\newcommand{\childdocby}[2][]
{
  \childdocdisable
  \childdoctrue
  \childdocmanualtrue
  \if?#1?\else
    \def\jobname{#2}
  \fi
  \def\childdocjob{#2}
  \input{#2}
  \endinput
}
%    \end{macrocode}

% \macro{\childdocforward}
% The command |\childdocforward| redirects
% compilation to the main file or
% (if the optional argument is given) a child file.
% Parameters are set as if the main file
% or a child file starting with |\childdocof| was compiled.
% Then compilation is handed over to the main file:
%    \begin{macrocode}
\newcommand{\childdocforward}[2][]
{
  \begingroup
    \if?#1?
      \def\childdoctmp
      {
        \def\childdocname{#2}
        \def\childdocjob{#2}
        \def\jobname{#2}
        \input{#2}
        \endinput
      }
    \else
      \def\childdoctmp
      {
        \childdocdisable
        \def\childdocname{#2}
        \childdoctrue
        \includeonly{#2}
        \def\childdocjob{#1}
        \def\jobname{#1}
        \input{#1}
        \endinput
      }
    \fi
    \expandafter
  \endgroup
  \childdoctmp
}
%    \end{macrocode}

% \macro{\childdocforwardprefix}
% The command |\childdocforwardprefix| redirects
% compilation to the main or a child file by means of a pattern.
% The prefix |#1| in the current filename is replaced by |#2|
% and the suffix of the current filename is kept
% (it is assumed that the filename does not contain the substring `|~~~|'
% which is used as a delimiter).
% Compilation is handed over to the new file by |\childdocforward|:
%    \begin{macrocode}
\newcommand{\childdocforwardprefix}[3][]
{
  \begingroup
    \def\childdocextract #2##1~~~{\def\childdoctmp{\childdocforward[#1]{#3##1}}}
    \expandafter\childdocextract\childdocname~~~
    \expandafter
  \endgroup
  \childdoctmp
}
%    \end{macrocode}

% \macro{\childdoc}
% The deprecated macro |\childdoc| is a legacy version of |\childdocmain|:
%    \begin{macrocode}
\newcommand{\childdoc}{\childdocmain}
%    \end{macrocode}

% \macro{\childdocredirect}
% The deprecated macro |\childdocredirect| is a legacy version
% of |\childdocforward| and |\childdocforwardprefix|:
%    \begin{macrocode}
\newcommand{\childdocredirect}[2][]
{
  \begingroup
    \if?#1?
      \def\childdoctmp{\childdocforward{#2}}
    \else
      \def\childdoctmp{\childdocforwardprefix{#1}{#2}}
    \fi
    \expandafter
  \endgroup
  \childdoctmp
}
%    \end{macrocode}

%\iffalse
%</package>
%\fi
%
\endinput

\childdocmain{}
%    \end{macrocode}

% Optional override for |\version| flag:
%    \begin{macrocode}
%%\ifchilddoc\else\providecommand{\version}{draft}\fi
%    \end{macrocode}

% Define the default values for the |\version| flag
% (|final| for the main file and |draft| for childs):
%    \begin{macrocode}
\ifchilddoc
\providecommand{\version}{draft}
\else
\providecommand{\version}{final}
\fi
%    \end{macrocode}

% Load the standard document class:
%    \begin{macrocode}
\documentclass[12pt]{article}
%    \end{macrocode}

% Start the document body:
%    \begin{macrocode}
\begin{document}
%    \end{macrocode}

% Declare a title page.
% Print title, part of document being processed and version flag:
%    \begin{macrocode}
\addtocounter{page}{-1}
\begin{center}
{\LARGE\bfseries{}childdoc example\par}
\vspace{1cm}
\ifchilddoc
\ifchilddocmanual part\else chapter\fi:
`\childdocname' of `\childdocjob'\par
\else
main document: `\childdocjob'\par
\fi
version: \version\par
\end{center}
\newpage
%    \end{macrocode}

% Manually include selected file,
% otherwise process as usual:
%    \begin{macrocode}
\ifchilddocmanual
\section*{part `\childdocname'}
\input{\childdocname}
\else
%    \end{macrocode}

% Include the two chapters:
%    \begin{macrocode}
\include{cdocsch1}
\include{cdocsch2}
%    \end{macrocode}

% Include the two parts unless only chapters should be displayed:
%    \begin{macrocode}
\ifchilddoc\else
\section{part three}
\input{cdocspt3}
\section{part four}
\input{cdocspt4}
\fi
%    \end{macrocode}

% Process as usual until here:
%    \begin{macrocode}
\fi
%    \end{macrocode}

% End of document body:
%    \begin{macrocode}
\end{document}
%    \end{macrocode}
%\iffalse
%</samplemain>
%\fi
%
% %%%%%%%%%%%%%%%%%%%%%%%%%%%%%%%%%%%%%%
% \paragraph{Chapter Include Files.}
%
% The include files are called |cdocsch1.tex| and |cdocsch2.tex|.
%
%\iffalse
%<*samplechap1|samplechap2>
%\fi

% Optional override for |\version| flag:
%    \begin{macrocode}
%%\providecommand{\version}{final}
%    \end{macrocode}

% Include the main document:
%    \begin{macrocode}
% \iffalse
%
% childdoc.dtx Copyright (C) 2017-2018 Niklas Beisert
%
% This work may be distributed and/or modified under the
% conditions of the LaTeX Project Public License, either version 1.3
% of this license or (at your option) any later version.
% The latest version of this license is in
%   http://www.latex-project.org/lppl.txt
% and version 1.3 or later is part of all distributions of LaTeX
% version 2005/12/01 or later.
%
% This work has the LPPL maintenance status `maintained'.
%
% The Current Maintainer of this work is Niklas Beisert.
%
% This work consists of the files childdoc.dtx and childdoc.ins
% and the derived files childdoc.def and cdocsamp.tex with
% cdocsch1.tex, cdocsch2.tex, cdocsdrf.tex, cdocsfn1.tex, cdocsfn2.tex.
%
%<package>\ifdefined\childdocmain\endinput\fi
%<package>\ProvidesFile{childdoc.def}[2018/12/30 v2.0 child document driver]
%<samplemain>\ProvidesFile{cdocsamp.tex}[2018/12/30 v2.0 sample for childdoc]
%<*driver>
%\ProvidesFile{childdoc.drv}[2018/12/30 v2.0 childdoc reference manual file]
\PassOptionsToClass{10pt,a4paper}{article}
\documentclass{ltxdoc}

\usepackage[margin=35mm]{geometry}
\usepackage{hyperref}
\usepackage{hyperxmp}
\usepackage[usenames]{color}

\hypersetup{colorlinks=true}
\hypersetup{pdfstartview=FitH}
\hypersetup{pdfpagemode=UseNone}
\hypersetup{pdfsource={}}
\hypersetup{pdflang={en-UK}}
\hypersetup{pdfcopyright={Copyright 2017-2018 Niklas Beisert.
  This work may be distributed and/or modified under the
  conditions of the LaTeX Project Public License, either version 1.3
  of this license or (at your option) any later version.}}
\hypersetup{pdflicenseurl={http://www.latex-project.org/lppl.txt}}
\hypersetup{pdfcontactaddress={ETH Zurich, ITP, HIT K,
  Wolfgang-Pauli-Strasse 27}}
\hypersetup{pdfcontactpostcode={8093}}
\hypersetup{pdfcontactcity={Zurich}}
\hypersetup{pdfcontactcountry={Switzerland}}
\hypersetup{pdfcontactemail={nbeisert@itp.phys.ethz.ch}}
\hypersetup{pdfcontacturl={http://people.phys.ethz.ch/\xmptilde nbeisert/}}

\newcommand{\secref}[1]{\hyperref[#1]{section \ref*{#1}}}

\parskip1ex
\parindent0pt
\let\olditemize\itemize
\def\itemize{\olditemize\parskip0pt}

\begin{document}

\title{The \textsf{childdoc} Package}
\hypersetup{pdftitle={The childdoc Package}}
\author{Niklas Beisert\\[2ex]
  Institut f\"ur Theoretische Physik\\
  Eidgen\"ossische Technische Hochschule Z\"urich\\
  Wolfgang-Pauli-Strasse 27, 8093 Z\"urich, Switzerland\\[1ex]
  \href{mailto:nbeisert@itp.phys.ethz.ch}
  {\texttt{nbeisert@itp.phys.ethz.ch}}}
\hypersetup{pdfauthor={Niklas Beisert}}
\hypersetup{pdfsubject={Manual for the LaTeX2e Package childdoc}}
\date{30 December 2018, \textsf{v2.0}}
\maketitle

\begin{abstract}\noindent
\textsf{childdoc} is a \LaTeXe{} package
that enables the direct compilation
of document sections included by |\include|
to individual files.
\end{abstract}

\begingroup
\parskip0ex
\tableofcontents
\endgroup

%%%%%%%%%%%%%%%%%%%%%%%%%%%%%%%%%%%%%%%%%%%%%%%%%%%%%%%%%%%%%%%%%%%%%%%%%%%%%%%%
%%%%%%%%%%%%%%%%%%%%%%%%%%%%%%%%%%%%%%%%%%%%%%%%%%%%%%%%%%%%%%%%%%%%%%%%%%%%%%%%
\section{Introduction}

\LaTeX{} provides a mechanism to structure a large document (such as a book)
into a main file and several child files (containing the chapters)
using the |\include| command.
This mechanism is beneficial for documents
which span hundreds of pages in order to
make the source file(s) more manageable.
Moreover, compilation can be restricted to
selected child files by means of the |\includeonly| command.
The latter feature can be used to reduce the compilation time while editing
(this was significantly more useful in the earlier days of \LaTeX{})
or to generate a smaller document which is easier to navigate.
Another application of |\includeonly| is to generate
documents consisting of selected parts of the complete document.

However, there are a few drawbacks of the plain |\include| mechanism:
\begin{itemize}
\item
The child files cannot be compiled on their own,
they can only be compiled via the main file.
A naive editing environment
(such as a text editor with an option
to have the current file processed by \LaTeX)
may require one to switch to the main file before compiling;
attempting to compile the child file produces errors.
\item
The main file must be modified (each time)
to adjust the |\includeonly| command
to the present needs. This easily leaves the main file in a messy state.
\item
The generated document will always carry the filename
of the main document. This is inconvenient if
several child files are to be compiled and
to be kept for distribution.
\end{itemize}

The present package provides a simple interface
to make child files individually compilable by \LaTeX{}.
Compiling a child file then has the same effect as compiling
the main file with an |\includeonly| command
to select the appropriate child.
Moreover the generated document will carry the name of the child
rather than the main file.
This resolves all three above issues.

This feature is meant to make the editing of books,
thesis documents and lecture notes somewhat more convenient.
However, the package can also be used efficiently for
composing a series of documents (such as exercise sheets)
which are typically distributed individually.
It then assists the author in generating the individual documents
(potentially in different versions)
as well as a document containing the collected series.
Another application is in developing style files
or other kinds of included material
where compilation of the style file could redirect
to a sample or test file.

%%%%%%%%%%%%%%%%%%%%%%%%%%%%%%%%%%%%%%%%%%%%%%%%%%%%%%%%%%%%%%%%%%%%%%%%%%%%%%%%
%%%%%%%%%%%%%%%%%%%%%%%%%%%%%%%%%%%%%%%%%%%%%%%%%%%%%%%%%%%%%%%%%%%%%%%%%%%%%%%%
\section{Usage}

First of all, the package \textsf{childdoc} is \emph{not} a standard
\LaTeXe{} |.sty| style file! Therefore it needs to be invoked in
a non-standard way.

%%%%%%%%%%%%%%%%%%%%%%%%%%%%%%%%%%%%%%%%%%%%%%%%%%%%%%%%%%%%%%%%%%%%%%%%%%%%%%%%
\subsection{Included Files}
\label{sec:include}

%%%%%%%%%%%%%%%%%%%%%%%%%%%%%%%%%%%%%%%%
\DescribeMacro{\childdocmain}
To use the package, add the commands
\begin{center}
\begin{tabular}{l}
|\input{childdoc.def}|\\
|\childdocmain{}|\\
\end{tabular}
\end{center}
at the very top of the main \LaTeX{} file,
in particular \emph{before} the |\documentclass| statement!
The argument of |\childdocmain| should be left empty
(but it must be present).

%%%%%%%%%%%%%%%%%%%%%%%%%%%%%%%%%%%%%%%%
\DescribeMacro{\childdocof}
Furthermore, add the commands
\begin{center}
\begin{tabular}{l}
|\input{childdoc.def}|\\
|\childdocof{|\textit{main}|}|\\
\end{tabular}
\end{center}
at the top of every child file \textit{child}
which is included by |\include{|\textit{child}|}|
from within the main file
(or at least for those files to be compiled individually).
The argument \textit{main} must be the filename of the main file.

There are a couple of
considerations in setting up the main and child documents:

%%%%%%%%%%%%%%%%%%%%%%%%%%%%%%%%%%%%%%%%
\paragraph{Restrictions.}

Please note the following restrictions:
\begin{itemize}
\item
|\childdocmain| must be called with one argument \textit{main}
to ensure compatibility with earlier version of the package.
It must either be empty (|\childdocmain{}|)
or precisely match the filename of the main file in which it is specified.
See \secref{sec:detection} for further information.
\item
The filename \textit{main} must be specified without the |.tex| extension.
\item
The filename \textit{main} is case sensitive
(even in case-insensitive file systems)
due to internal string comparison.
\item
The argument \textit{main} should be fully expanded, it cannot be a macro.
\item
Subdirectories and special characters should be avoided in filenames.
\item
The command |\childdocmain{|\textit{main}|}| must be followed by a whitespace.
It should not be followed immediately by another command
or by a comment mark `|%|'.
This is because the \TeX{} parser reads the token immediately following
the argument of |\childdocmain| and puts it
at the beginning of every child section;
however, a white\-space is ignored.
\end{itemize}

%%%%%%%%%%%%%%%%%%%%%%%%%%%%%%%%%%%%%%%%
\paragraph{Content of Main File.}

It is advisable to place all content in the child files included by |\include|.
Any output contained in the main file will appear in all child documents
unless suppressed manually;
it cannot be suppressed automatically by the |\includeonly| directive
and thus should normally be avoided.
A method to include some content in the main file
by means of conditional processing is described in \secref{sec:conditional}.

%%%%%%%%%%%%%%%%%%%%%%%%%%%%%%%%%%%%%%%%
\paragraph{Page Numbering.}

When only a part of the document is compiled,
the appropriate numbering of pages
(as well as other status parameters)
is determined from the |.aux| files.
The latter contain information from previous passes.
However this information needs to propagate through
all intermediate child documents.
Therefore the page numbering in child documents may well
be inconsistent until the complete document is compiled at least once.

A useful (if unconventional) way to always ensure a consistent
page numbering is to restart the numbering in each child document
and denote the pages by `\textit{child}|.|\textit{page}'
where \textit{child} represents the chapter/section number of the child file.
This can be achieved by the command
|\numberwithin{page}{|\textit{child}|}|
of the \textsf{amsmath} package
where \textit{child} can be |chapter| or |section|
depending on the chosen structuring.
Alternatively, one can modify the macro |\thepage| appropriately
and reset the counter |page| at the start of each child file.

%%%%%%%%%%%%%%%%%%%%%%%%%%%%%%%%%%%%%%%%%%%%%%%%%%%%%%%%%%%%%%%%%%%%%%%%%%%%%%%%
\subsection{Conditional Processing}
\label{sec:conditional}

The package provides a mechanism to compile different versions
of a document. To customise the versions further some conditional processing
can come in handy to distinguish which version is being compiled.
The package provides two macros to describe the compilation context:

%%%%%%%%%%%%%%%%%%%%%%%%%%%%%%%%%%%%%%%%
\DescribeMacro{\ifchilddoc}
The conditional |\ifchilddoc| distinguishes between the compilation of
child documents and the main document:
%
\begin{center}
|\ifchilddoc |\textit{child-code}| |[|\||else |\textit{main-code}]| \||fi|
\end{center}

%%%%%%%%%%%%%%%%%%%%%%%%%%%%%%%%%%%%%%%%
\DescribeMacro{\childdocname}
\DescribeMacro{\childdocjob}
The macro |\childdocname| contains the filename (without extension)
of the main or child file being processed.
Note that |\childdocjob| will always contain the name of the main file.

%%%%%%%%%%%%%%%%%%%%%%%%%%%%%%%%%%%%%%%%
\paragraph{Title Page.}

Conditional processing can be used to include a title or banner page
in the main document when proper precautions are taken.
Importantly, the code in the main file should ensure that the page counter
(as well as other status parameters which are stored in the |.aux| files)
takes the same value after the conditional processing.
Otherwise the page numbers may take divergent values
depending on which part is compiled.

For example, a title page could be declared by:
%
\begin{center}
\begin{tabular}{l}
|\ifchilddoc\||else|\\
|\addtocounter{page}{-1}|\\
\textit{code for title page}\\
|\newpage|\\
|\||fi|
\end{tabular}
\end{center}
%
A banner page for the child documents can be generated by:
%
\begin{center}
\begin{tabular}{l}
|\ifchilddoc|\\
|\addtocounter{page}{-1}|\\
\textit{code for banner page}\\
|\newpage|\\
|\||fi|
\end{tabular}
\end{center}
%
Here one could write a message such as:
\begin{center}
|This is the part \childdocname{} of \childdocjob{}.|
\end{center}

%%%%%%%%%%%%%%%%%%%%%%%%%%%%%%%%%%%%%%%%%%%%%%%%%%%%%%%%%%%%%%%%%%%%%%%%%%%%%%%%
\subsection{Flags}
\label{sec:flags}

The package makes it easy to generate different versions
of the main or child documents.
To this end compilation flags can be defined
and assigned different default values.
They will be particularly useful in conjunction
with the forwarding mechanism described in \secref{sec:forward}.

For example, it may be useful to have a flag |\version|
which can be set to |draft| or |final|.
The document source will contain some conditional code
depending on the value of |\version|.
Suppose further, the flag should default to |final| for the main file
and to |draft| for child files
which is a natural assignment for editing the document.
This is achieved by placing the following code
in the preamble of the main document
(below the |\childdocmain| directive):
%
\begin{center}
\begin{tabular}{l}
|\ifchilddoc|\\
|\providecommand{\version}{draft}|\\
|\||else|\\
|\providecommand{\version}{final}|\\
|\||fi|
\end{tabular}
\end{center}
%
The definition by |\providecommand| makes sure
that previous definitions are not overwritten.
Further statements |\providecommand{\version}{...}|
can thus be added before the above code to override it.

For the main file, one might add a line
(between |\childdocmain| and the above block)
%
\begin{center}
|%\ifchilddoc\||else\providecommand{\version}{draft}\||fi|
\end{center}
%
which can be uncommented to produce a draft version.
Likewise one can add a line to the very top of a child file
(above the |\childdocof{|\textit{main}|}| directive)
%
\begin{center}
|%\providecommand{\version}{final}|
\end{center}
%
which can be uncommented to produce the final version of this child document.

%%%%%%%%%%%%%%%%%%%%%%%%%%%%%%%%%%%%%%%%%%%%%%%%%%%%%%%%%%%%%%%%%%%%%%%%%%%%%%%%
\subsection{Forwarding}
\label{sec:forward}

Different versions of the main or child documents
using compilation flags as described in \secref{sec:flags}
can be (permanently) stored in different files
for convenient compilation, viewing and distribution.
To this end, the package defines a command
to pass on compilation to a different file:

%%%%%%%%%%%%%%%%%%%%%%%%%%%%%%%%%%%%%%%%
\DescribeMacro{\childdocforward}
The command |\childdocforward| redirects processing to
another source file:
%
\begin{center}
\begin{tabular}{l}
|\input{childdoc.def}|\\
|\childdocforward[|\textit{main}|]{|\textit{dest}|}|\\
\end{tabular}
\end{center}
%
The argument \textit{dest} is the destination file
(without extension).
It should be the main file or one of the child files.
Note that further \textsf{childdoc} directives
such as |\childdocof| and |\childdocforward|
in the indicated file will be processed in this form.
The optional argument \textit{main}
passes on directly to the main file \textit{main}
while pretending to compile the child \textit{dest}.
This form behaves as if \textit{dest}
issues |\childdocof{|\textit{main}|}| right away,
and no further \textsf{childdoc} directives will be processed.

%%%%%%%%%%%%%%%%%%%%%%%%%%%%%%%%%%%%%%%%
\DescribeMacro{\...prefix}
In the alternative form |\childdocforwardprefix|,
%
\begin{center}
\begin{tabular}{l}
|\input{childdoc.def}|\\
|\childdocforwardprefix[|\textit{main}|]{|\textit{prefix}|}{|\textit{dest}|}|
\end{tabular}
\end{center}
%
the destination file is determined by a pattern
depending on the current file:
To make this work, the current file must be called
`{\textit{prefix}\hspace{0.2em}\textit{suffix}}'
with \textit{prefix} matching precisely the argument.
Processing is then passed on to the file
`{\textit{dest}\hspace{0.2em}\textit{suffix}}'.
Surely, the same effect is achieved by
directly specifying the
argument `{\textit{dest}\hspace{0.2em}\textit{suffix}}'
in the first form.
However, that requires to set up a different file
for each child. With the alternative form of the command
all these files can have exactly the same content
which simplifies setting them up and maintaining them.

For example, the following file |draft.tex|
with a compilation flag |\version| as described in \secref{sec:flags}
compiles the main document as a draft:
%
\begin{center}
\begin{tabular}{l}
|\def\version{draft}|\\
|\input{childdoc.def}|\\
|\childdocforward{|\textit{main}|}|
\end{tabular}
\end{center}
%
Likewise, the following files |final|\textit{nn}|.tex|
compile the final version of the child document
|child|\textit{nn}|.tex|:
%
\begin{center}
\begin{tabular}{l}
|\def\version{final}|\\
|\input{childdoc.def}|\\
|\childdocforwardprefix{final}{child}|
\end{tabular}
\end{center}
%

Note that when several versions of a main file and/or of each child file
are to be generated, it may be convenient to set up a |Makefile| or
shell script to automatise the process.

%%%%%%%%%%%%%%%%%%%%%%%%%%%%%%%%%%%%%%%%%%%%%%%%%%%%%%%%%%%%%%%%%%%%%%%%%%%%%%%%
\subsection{Command Line Processing}
\label{sec:commandline}

The effect of redirection files can also be achieved by invoking
the \LaTeX{} compiler with a more elaborate command line.
Most conveniently this should be done as part
of a shell script or a |Makefile|.

When using \textsf{childdoc} in the main file, the following
command lines effectively perform a redirection
(note that depending on the shell being used,
backslashes may have to be doubled: `|\|' $\to$ `|\\|'):
%
\begin{center}
|... -jobname "|\textit{target}|" |\\|"|[\textit{flags}]%
|\input{childdoc.def}\childdocforward[|\textit{main}|]{|\textit{dest}|}"|
\end{center}
%
Here \textit{target} is the name of the output file,
\textit{main} is the name of the main file
and \textit{dest} is the name of the main or child file to be processed
(all filenames without extensions).
The optional argument \textit{main} can be omitted
if \textit{main} matches \textit{dest}.
Optionally, compilation \textit{flags} can be defined via |\def| commands.
This command line makes the \TeX{} engine believe
it is compiling the file \textit{target}
whose content is specified as the latter parameter.
The provided code then forwards the processing to
\textit{main} or \textit{dest} as described in \secref{sec:forward}.

%%%%%%%%%%%%%%%%%%%%%%%%%%%%%%%%%%%%%%%%%%%%%%%%%%%%%%%%%%%%%%%%%%%%%%%%%%%%%%%%
\subsection{Include by Input}
\label{sec:input}

Including child documents by |\include| has some restrictions by design.
Most notably, the content of a child document always occupies
its own set of pages; pages cannot be shared between child documents.
Usually, this behaviour makes perfect sense
because each child document contain an essential part of the document.
However, in some situations it may be desirable to compose
a document from a collection of parts
without having mandatory page breaks between then.
For this case, the package
provides a mechanism to include parts
by |\input| which can also be processed individually.
However, by construction this mechanism
requires manual handling of the content to be output.

%%%%%%%%%%%%%%%%%%%%%%%%%%%%%%%%%%%%%%%%
\DescribeMacro{\ifchilddocmanual}
The main file should be prepared as usual, see \secref{sec:include}.
However, the document body must make a distinction
between processing of an individual part and of the main document, e.g.:
%
\begin{center}
\begin{tabular}{l}
|\ifchilddocmanual|\\
|\input{\childdocname}|\\
|\||else|\\
\textit{document body with }|\input{|\textit{part}|}|\\
|\||fi|
\end{tabular}
\end{center}
%
The conditional |\ifchilddocmanual| is true whenever
a part to be included by |\input| is being compiled,
and the name of the part is stored in |\childdocname|.

%%%%%%%%%%%%%%%%%%%%%%%%%%%%%%%%%%%%%%%%
\DescribeMacro{\childdocby}
Each part to be included by |\input| should start with:
%
\begin{center}
\begin{tabular}{l}
|\input{childdoc.def}|\\
|\childdocby{|\textit{main}|}|\\
\end{tabular}
\end{center}
%
The directive |\childdocby| is similar to |\childdocof|
described in \secref{sec:include},
but the subsequent selection of content must be done manually.
To that end, both |\ifchilddoc| and |\ifchilddocmanual|
will be true upon processing of a part,
and the name of the part is stored in |\childdocname|.
Note that |\jobname| will be set to the filename of the current part
so that each part receives an individual |.aux| file
that does not interfere with the |.aux| file(s) of the main document.
This behaviour can be altered by the alternative form
|\childdocby[*]{|\textit{main}|}| (with a non-empty optional argument)
which uses the |.aux| file of the main document
by setting |\jobname| to \textit{main}.

%%%%%%%%%%%%%%%%%%%%%%%%%%%%%%%%%%%%%%%%%%%%%%%%%%%%%%%%%%%%%%%%%%%%%%%%%%%%%%%%
\subsection{Driver Development}
\label{sec:driver}

The \textsf{childdoc} mechanism can also be use for the development
of definition files such as \LaTeX{} styles or classes.
This case differs from the above setup with multiple parts
included by |\include| in that no |\includeonly| should be invoked.
This can be achieved by starting the include file
(before |\ProvidesPackage|) with:
%
\begin{center}
\begin{tabular}{l}
|\input{childdoc.def}|\\
|\childdocforward{|\textit{main}|}|\\
\end{tabular}
\end{center}
%
or alternatively with:
%
\begin{center}
\begin{tabular}{l}
|\input{childdoc.def}|\\
|\childdocby{|\textit{main}|}|\\
\end{tabular}
\end{center}
%
Both forms have slightly different effects as described above.
The main file is prepared as usual, see \secref{sec:include}.

%%%%%%%%%%%%%%%%%%%%%%%%%%%%%%%%%%%%%%%%%%%%%%%%%%%%%%%%%%%%%%%%%%%%%%%%%%%%%%%%
\subsection{Legacy Detection}
\label{sec:detection}

The directive |\childdocmain| in the main file can detect
whether the complete document or merely a child is to be compiled
even without using the directive |\childdocof|.
This method is deprecated because it is less robust
and there is no compelling reason to use it;
it is merely provided for backward compatibility
and it may be removed in future versions.

If the detection mechanism is to be used,
it is mandatory to correctly specify
the filename of the main file as the argument of |\childdocmain|:
%
\begin{center}
\begin{tabular}{l}
|\input{childdoc.def}|\\
|\childdocmain{|\textit{main}|}|\\
\end{tabular}
\end{center}
%
If |\jobname| does not match the argument \textit{main} of |\childdocmain|,
it is assumed that |\jobname| points to the child file to be compiled.
When using |\childdocmain| with the main file specified as argument,
it suffices to start a child file
with just |\input{|\textit{main}|}|
without loading of the package and using |\childdocof|.
If instead all processing is done
with the appropriate \textsf{childdoc} directives,
the argument of \textit{main} of |\childdocmain| can be empty.

An alternative version of the command line processing described
in \secref{sec:commandline} using the detection mechanism reads:
%
\begin{center}
|... -jobname "|\textit{target}|" "|[\textit{flags}]%
[|\def\jobname{|\textit{dest}|}|]|\input{|\textit{main}|}"|
\end{center}

%%%%%%%%%%%%%%%%%%%%%%%%%%%%%%%%%%%%%%%%%%%%%%%%%%%%%%%%%%%%%%%%%%%%%%%%%%%%%%%%
\subsection{Manual Code}
\label{sec:manual}

In case one cannot be certain whether the definitions file |childdoc.def|
is installed on the target \TeX{} distribution
and one prefers not to ship it,
it is conceivable to paste a few relevant commands into the sources.

To that end, drop all statements |\input{childdoc.def}|
and perform the replacements as outlined below.
Instead of |\childdocmain{|\textit{main}|}| add the following code
to the top of the main file:
%
\begin{center}
\begin{tabular}{l}
|\||ifdefined\childdocname\endinput\||fi\newif\ifchilddoc|\\
|\edef\childdocname{\scantokens\expandafter{\jobname\noexpand}}|\\
|\def\childdocmain{|\textit{main}|}\||ifx\childdocmain\childdocname\||else|\\
|\childdoctrue\includeonly{\childdocname}\let\jobname\childdocmain\||fi|\\
\end{tabular}
\end{center}
%
Instead of |\childdocof{|\textit{main}|}| just include the main file
at the top of each child file:
%
\begin{center}
|\input{|\textit{main}|}|
\end{center}
%
A simple redirection |\childdocforward{|\textit{dest}|}| is achieved by:
%
\begin{center}
|\def\jobname{|\textit{dest}|}\input{\jobname}|
\end{center}
%
The redirection with prefix
|\childdocforwardprefix[|\textit{prefix}|]{|\textit{dest}|}|
is accomplished by:
%
\begin{center}
\begin{tabular}{l}
|{\edef\jobname{\scantokens\expandafter{\jobname\noexpand}}|\\
|\def\redirectjob |\textit{prefix}|#1~~~{\gdef\jobname{|\textit{dest}|#1}}|\\
|\expandafter\redirectjob\jobname~~~}\input{\jobname}|
\end{tabular}
\end{center}

In an alternative approach,
child documents can be compiled by a specific command line
without additional code or specific definitions:
%
\begin{center}
|... -jobname "|\textit{target}|" "|[\textit{flags}]%
|\includeonly{|\textit{dest}|}\input{|\textit{main}|}"|
\end{center}
%

%%%%%%%%%%%%%%%%%%%%%%%%%%%%%%%%%%%%%%%%%%%%%%%%%%%%%%%%%%%%%%%%%%%%%%%%%%%%%%%%
%%%%%%%%%%%%%%%%%%%%%%%%%%%%%%%%%%%%%%%%%%%%%%%%%%%%%%%%%%%%%%%%%%%%%%%%%%%%%%%%
\section{Information}

%%%%%%%%%%%%%%%%%%%%%%%%%%%%%%%%%%%%%%%%%%%%%%%%%%%%%%%%%%%%%%%%%%%%%%%%%%%%%%%%
\subsection{Copyright}

Copyright \copyright{} 2017--2018 Niklas Beisert

This work may be distributed and/or modified under the
conditions of the \LaTeX{} Project Public License, either version 1.3
of this license or (at your option) any later version.
The latest version of this license is in
  \url{http://www.latex-project.org/lppl.txt}
and version 1.3 or later is part of all distributions of \LaTeX{}
version 2005/12/01 or later.

This work has the LPPL maintenance status `maintained'.

The Current Maintainer of this work is Niklas Beisert.

This work consists of the files |README.txt|, |childdoc.ins| and |childdoc.dtx|
as well as the derived files |childdoc.def|, |cdocsamp.tex|
with |cdocsch1.tex|, |cdocsch2.tex|, |cdocspt3.tex|, |cdocspt4.tex|,
|cdocsdrf.tex|, |cdocsfn1.tex|, |cdocsfn2.tex|
as well as |childdoc.pdf|.

%%%%%%%%%%%%%%%%%%%%%%%%%%%%%%%%%%%%%%%%%%%%%%%%%%%%%%%%%%%%%%%%%%%%%%%%%%%%%%%%
\subsection{Files and Installation}

The package consists of the files:
%
\begin{center}
\begin{tabular}{ll}
    |README.txt|   & readme file \\
    |childdoc.ins| & installation file \\
    |childdoc.dtx| & source file \\
    |childdoc.def| & definition file \\
    |cdocsamp.tex| & sample main file \\
    |cdocsch1.tex| & sample include file \\
    |cdocsch2.tex| & sample include file \\
    |cdocspt3.tex| & sample part file \\
    |cdocspt4.tex| & sample part file \\
    |cdocsdrf.tex| & sample redirection file \\
    |cdocsfn1.tex| & sample redirection file \\
    |cdocsfn2.tex| & sample redirection file \\
    |childdoc.pdf| & manual
\end{tabular}
\end{center}
%
The distribution consists of the files
|README.txt|, |childdoc.ins| and |childdoc.dtx|.
%
\begin{itemize}
\item
Run (pdf)\LaTeX{} on |childdoc.dtx|
to compile the manual |childdoc.pdf| (this file).
\item
Run \LaTeX{} on |childdoc.ins| to create the definitions file |childdoc.def|
and the sample |cdocsamp.tex| with include files
|cdocsch1.tex|, |cdocsch2.tex|, |cdocspt3.tex|, |cdocspt4.tex|,
|cdocsdrf.tex|, |cdocsfn1.tex|, |cdocsfn2.tex|.
Then copy the file |childdoc.def| to an appropriate directory of your \LaTeX{}
distribution, e.g.\ \textit{texmf-root}|/tex/latex/childdoc|.
\end{itemize}

%%%%%%%%%%%%%%%%%%%%%%%%%%%%%%%%%%%%%%%%%%%%%%%%%%%%%%%%%%%%%%%%%%%%%%%%%%%%%%%%
\subsection{Related CTAN Packages}

There are several other packages which offer a similar functionality:
%
\begin{itemize}
\item
The packages
\href{http://ctan.org/pkg/docmute}{\textsf{docmute}},
\href{http://ctan.org/pkg/includex}{\textsf{includex}} and
\href{http://ctan.org/pkg/standalone}{\textsf{standalone}}
provide commands to include only the document body of
a child file thus allowing both files to be compiled individually.
\item
The packages \href{http://ctan.org/pkg/subdocs}{\textsf{subdocs}}
and \href{http://ctan.org/pkg/subfiles}{\textsf{subfiles}}
provide structures in which the main and child documents can be
encapsulated and allowing them to be compiled individually.
The inclusion mechanism is different from the conventional |\include|.
\item
The package \href{http://ctan.org/pkg/combine}{\textsf{combine}}
is an elaborate solution to combine several documents into one.
\end{itemize}
%
See also the CTAN topic \href{http://ctan.org/topic/subdocs}{\textsf{subdocs}}
for further related packages.
The present package differs from the above solutions in that
a document structure constructed with the conventional |\include| mechanism
just needs two extra commands at the top of every file
such that all constituent files can be compiled individually.

%%%%%%%%%%%%%%%%%%%%%%%%%%%%%%%%%%%%%%%%%%%%%%%%%%%%%%%%%%%%%%%%%%%%%%%%%%%%%%%%
%\subsection{Feature Suggestions}
%
%The following is a list of features which may be useful for future
%versions of this package:
%%
%\begin{itemize}
%\item
%\ldots
%\end{itemize}

%%%%%%%%%%%%%%%%%%%%%%%%%%%%%%%%%%%%%%%%%%%%%%%%%%%%%%%%%%%%%%%%%%%%%%%%%%%%%%%%
\subsection{Revision History}

%%%%%%%%%%%%%%%%%%%%%%%%%%%%%%%%%%%%%%%%
\paragraph{v2.0:} 2018/12/30

\begin{itemize}
\item
immediate forward processing
\item
added |\childdocby| mechanism
\item
manual restructured
\end{itemize}

%%%%%%%%%%%%%%%%%%%%%%%%%%%%%%%%%%%%%%%%
\paragraph{v1.6:} 2018/01/17

\begin{itemize}
\item
application for development of include files
\item
corrections to manual
\end{itemize}

%%%%%%%%%%%%%%%%%%%%%%%%%%%%%%%%%%%%%%%%
\paragraph{v1.5:} 2017/05/21

\begin{itemize}
\item
more complete structuring introduced
\item
|\childdocof| introduced
\item
|\childdoc| renamed to |\childdocmain|
\item
|\childredirect| renamed to |\childdocforward| and |\childdocforwardprefix|
and functionality expanded
\end{itemize}

%%%%%%%%%%%%%%%%%%%%%%%%%%%%%%%%%%%%%%%%
\paragraph{v1.0:} 2017/04/27

\begin{itemize}
\item
manual and install package
\item
first version published on CTAN
\end{itemize}

%%%%%%%%%%%%%%%%%%%%%%%%%%%%%%%%%%%%%%%%
\paragraph{v0.6:} 2017/04/26

\begin{itemize}
\item
redirection mechanism added
\end{itemize}

%%%%%%%%%%%%%%%%%%%%%%%%%%%%%%%%%%%%%%%%
\paragraph{v0.5:} 2017/04/26

\begin{itemize}
\item
functionality in definition file
\end{itemize}


%%%%%%%%%%%%%%%%%%%%%%%%%%%%%%%%%%%%%%%%%%%%%%%%%%%%%%%%%%%%%%%%%%%%%%%%%%%%%%%%
%%%%%%%%%%%%%%%%%%%%%%%%%%%%%%%%%%%%%%%%%%%%%%%%%%%%%%%%%%%%%%%%%%%%%%%%%%%%%%%%
%%%%%%%%%%%%%%%%%%%%%%%%%%%%%%%%%%%%%%%%%%%%%%%%%%%%%%%%%%%%%%%%%%%%%%%%%%%%%%%%
\appendix

\settowidth\MacroIndent{\rmfamily\scriptsize 000\ }

 \DocInput{childdoc.dtx}

\end{document}
%</driver>
% \fi
%
% %%%%%%%%%%%%%%%%%%%%%%%%%%%%%%%%%%%%%%%%%%%%%%%%%%%%%%%%%%%%%%%%%%%%%%%%%%%%%%
% %%%%%%%%%%%%%%%%%%%%%%%%%%%%%%%%%%%%%%%%%%%%%%%%%%%%%%%%%%%%%%%%%%%%%%%%%%%%%%
% \section{Sample}
%\iffalse
%<*samplemain>
%\fi
%
% The following presents a sample document
% with two chapters, two parts, a title page,
% a compile flag as well as three forwarding files to set the flag.
% It consists of eight |.tex| files:
% \begin{center}
% \begin{tabular}{ll}
% |cdocsamp.tex|&main file\\
% |cdocsch1.tex|&include file for chapter 1\\
% |cdocsch2.tex|&include file for chapter 2\\
% |cdocspt3.tex|&include file for part 3\\
% |cdocspt4.tex|&include file for part 4\\
% |cdocsdrf.tex|&forwarding file for main file in draft mode\\
% |cdocsfi1.tex|&forwarding file for final version of chapter 1\\
% |cdocsfi2.tex|&forwarding file for final version of chapter 2\\
% \end{tabular}
% \end{center}
% Each of the eight files can be compiled directly by the \LaTeX{} compiler.
%
% %%%%%%%%%%%%%%%%%%%%%%%%%%%%%%%%%%%%%%
% \paragraph{Main File.}
%
% The main file is called |cdocsamp.tex|.
%
% Load the \textsf{childdoc} definitions and
% declare the filename for the main document:
%    \begin{macrocode}
\input{childdoc.def}
\childdocmain{}
%    \end{macrocode}

% Optional override for |\version| flag:
%    \begin{macrocode}
%%\ifchilddoc\else\providecommand{\version}{draft}\fi
%    \end{macrocode}

% Define the default values for the |\version| flag
% (|final| for the main file and |draft| for childs):
%    \begin{macrocode}
\ifchilddoc
\providecommand{\version}{draft}
\else
\providecommand{\version}{final}
\fi
%    \end{macrocode}

% Load the standard document class:
%    \begin{macrocode}
\documentclass[12pt]{article}
%    \end{macrocode}

% Start the document body:
%    \begin{macrocode}
\begin{document}
%    \end{macrocode}

% Declare a title page.
% Print title, part of document being processed and version flag:
%    \begin{macrocode}
\addtocounter{page}{-1}
\begin{center}
{\LARGE\bfseries{}childdoc example\par}
\vspace{1cm}
\ifchilddoc
\ifchilddocmanual part\else chapter\fi:
`\childdocname' of `\childdocjob'\par
\else
main document: `\childdocjob'\par
\fi
version: \version\par
\end{center}
\newpage
%    \end{macrocode}

% Manually include selected file,
% otherwise process as usual:
%    \begin{macrocode}
\ifchilddocmanual
\section*{part `\childdocname'}
\input{\childdocname}
\else
%    \end{macrocode}

% Include the two chapters:
%    \begin{macrocode}
\include{cdocsch1}
\include{cdocsch2}
%    \end{macrocode}

% Include the two parts unless only chapters should be displayed:
%    \begin{macrocode}
\ifchilddoc\else
\section{part three}
\input{cdocspt3}
\section{part four}
\input{cdocspt4}
\fi
%    \end{macrocode}

% Process as usual until here:
%    \begin{macrocode}
\fi
%    \end{macrocode}

% End of document body:
%    \begin{macrocode}
\end{document}
%    \end{macrocode}
%\iffalse
%</samplemain>
%\fi
%
% %%%%%%%%%%%%%%%%%%%%%%%%%%%%%%%%%%%%%%
% \paragraph{Chapter Include Files.}
%
% The include files are called |cdocsch1.tex| and |cdocsch2.tex|.
%
%\iffalse
%<*samplechap1|samplechap2>
%\fi

% Optional override for |\version| flag:
%    \begin{macrocode}
%%\providecommand{\version}{final}
%    \end{macrocode}

% Include the main document:
%    \begin{macrocode}
\input{childdoc.def}
\childdocof{cdocsamp}
%    \end{macrocode}

%\iffalse
%</samplechap1|samplechap2>
%\fi
%
%\iffalse
%<*samplechap1>
%\fi
% Some text for chapter 1:
%    \begin{macrocode}
\section{one}
some text in chapter one
%    \end{macrocode}

%\iffalse
%</samplechap1>
%\fi
% Some text for chapter 2:
%\iffalse
%<*samplechap2>
%\fi
%    \begin{macrocode}
\section{two}
more text in chapter two
%    \end{macrocode}

%\iffalse
%</samplechap2>
%\fi
%
% %%%%%%%%%%%%%%%%%%%%%%%%%%%%%%%%%%%%%%
% \paragraph{Part Include Files.}
%
% The include files are called |cdocspt3.tex| and |cdocspt4.tex|.
%
%\iffalse
%<*samplepart3|samplepart4>
%\fi

% Optional override for |\version| flag:
%    \begin{macrocode}
%%\providecommand{\version}{final}
%    \end{macrocode}

% Include the main document:
%    \begin{macrocode}
\input{childdoc.def}
\childdocby{cdocsamp}
%    \end{macrocode}

%\iffalse
%</samplepart3|samplepart4>
%\fi
%
%\iffalse
%<*samplepart3>
%\fi
% Some text for part 3:
%    \begin{macrocode}
some text in part three
%    \end{macrocode}

%\iffalse
%</samplepart3>
%\fi
% Some text for part 4:
%\iffalse
%<*samplepart4>
%\fi
%    \begin{macrocode}
more text in part four
%    \end{macrocode}

%\iffalse
%</samplepart4>
%\fi
%
% %%%%%%%%%%%%%%%%%%%%%%%%%%%%%%%%%%%%%%
% \paragraph{Forwarding for a Complete Draft.}
%
% The following forwarding file |cdocsdrf.tex|
% compiles the main document in draft mode:
%\iffalse
%<*sampledraft>
%\fi
%    \begin{macrocode}
\def\version{draft}
\input{childdoc.def}
\childdocforward{cdocsamp}
%    \end{macrocode}

%\iffalse
%</sampledraft>
%\fi
%
% %%%%%%%%%%%%%%%%%%%%%%%%%%%%%%%%%%%%%%
% \paragraph{Forwarding for Final Version of the Chapters.}
%
% The following forwarding files |cdocsfn1.tex| and |cdocsfn2.tex|
% (with identical content)
% compile the final versions of the child documents
% |cdocsch1.tex| and |cdocsch2.tex|, respectively:
%\iffalse
%<*samplefinal>
%\fi
%    \begin{macrocode}
\def\version{final}
\input{childdoc.def}
\childdocforwardprefix[cdocsamp]{cdocsfn}{cdocsch}
%    \end{macrocode}

%\iffalse
%</samplefinal>
%\fi
%
% %%%%%%%%%%%%%%%%%%%%%%%%%%%%%%%%%%%%%%
% \paragraph{Command Line Processing.}
%
% The following three command lines generate the output files
% |cdocscld|, |cdocscl1| and |cdocscl2|
% which should be identical to
% |cdocsdrf|, |cdocsch1| and |cdocsfn2|, respectively:
% \begin{center}
% \begin{tabular}{l}
% |latex -jobname cdocscld \|\\
% |  "\def\version{draft}\input{childdoc.def}\childdocforward{cdocsamp}"|\\
% |latex -jobname cdocscl1 \|\\
% |  "\input{childdoc.def}\childdocforward[cdocsamp]{cdocsch1}"|\\
% |latex -jobname cdocscl2 \|\\
% |  "\def\version{final}\input{childdoc.def}\childdocforward{cdocsch2}"|
% \end{tabular}
% \end{center}
% Note that the trailing backslash on each first line
% merely continues the input to the second line
% (for convenient cut ant paste).
% Furthermore, the command |latex| can be replaced by any
% of its alternative versions such as |pdflatex|.
%
% %%%%%%%%%%%%%%%%%%%%%%%%%%%%%%%%%%%%%%%%%%%%%%%%%%%%%%%%%%%%%%%%%%%%%%%%%%%%%%
% %%%%%%%%%%%%%%%%%%%%%%%%%%%%%%%%%%%%%%%%%%%%%%%%%%%%%%%%%%%%%%%%%%%%%%%%%%%%%%
% \section{Implementation}
%\iffalse
%<*package>
%\fi
%
% This section describes the definitions file |childdoc.def|.

% The definitions cannot be loaded using |\usepackage| or |\RequirePackage|
% which has a mechanism to prevent loading a style file more than once.
% When loading the definitions by means of |\input|
% multiple instances have to be prevented manually:
%\iffalse
%This code needs to be before the `\ProvidesFile' directive
%which is defined at the beginning of this file.
%Therefore it is also placed there and commented out here.
%</package>
%<*discard>
%\fi
%    \begin{macrocode}
\ifdefined\childdocmain\endinput\fi
%    \end{macrocode}
%\iffalse
%</discard>
%<*package>
%\fi
%
% \macro{\ifchilddoc}
% \macro{\ifchilddocmanual}
% The conditional |\ifchilddoc| tells whether a
% child (true) or main (false) document is being compiled.
% The conditional |\ifchilddocmanual| tells whether
% the |\includeonly| mechanism is used (false) or
% the selection of child files must be performed manually (true).
% The definitions initialise to false:
%    \begin{macrocode}
\newif\ifchilddoc
\newif\ifchilddocmanual
%    \end{macrocode}

% \macro{\childdocname}
% \macro{\childdocjob}
% The macro |\childdocname| stores the name of the main document
% to be compiled. The macro |\childdocjob| stores the name of
% the document on which the \LaTeX{} compiler was originally invoked.
% The content of |\jobname| cannot be compared
% to filenames specified in the source due to different catcodes.
% The following code rescans |\jobname|, stores the result
% in |\childdocname| and saves a copy in |\childdocjob|:
%    \begin{macrocode}
\edef\childdocname{\scantokens\expandafter{\jobname\noexpand}}
\let\childdocjob\childdocname
%    \end{macrocode}

% \macro{\childdocdisable}
% The macro |\childdocdisable| prevents the main file
% from being processed more than once.
% At this stage, the main document command |\childdocmain|
% is assumed to be called once again where it should do nothing.
% Any subsequent call to it should prevent
% a secondary processing of the main document
% It overwrites the forwarding commands
% |\childdocof| and |\childdocforward|
% with empty macros to prevent further inclusions of the main document:
%    \begin{macrocode}
\newcommand{\childdocdisable}
{
  \renewcommand{\childdocmain}[1]{\renewcommand{\childdocmain}[1]{\endinput}}
  \renewcommand{\childdocof}[1]{}
  \renewcommand{\childdocby}[2][]{}
  \renewcommand{\childdocforward}[2][]{}
  \renewcommand{\childdocdisable}{}
}
%    \end{macrocode}

% \macro{\childdocmain}
% The macro |\childdocmain| is to be called at the top of the main file
% with nothing or the main filename (without extension) as argument.
% First, it breaks loops.
% If the argument is not empty and does not match |\childdocname|
% (which is set by the first inclusion of |childdoc.def|),
% |\ifchilddoc| is set to true, |\includeonly| is applied to the child file
% and |\jobname| is set to the main file
% (for proper handling of |.aux| files):
%    \begin{macrocode}
\newcommand{\childdocmain}[1]
{
  \childdocdisable\childdocmain{}
  \if?#1?\else
    \begingroup
      \def\childdoctmp{#1}
      \ifx\childdoctmp\childdocname
        \def\childdoctmp{}
      \else
        \def\childdoctmp
        {
          \childdoctrue
          \includeonly{\childdocname}
          \def\childdocjob{#1}
          \def\jobname{#1}
        }
      \fi
      \expandafter
    \endgroup
    \childdoctmp
  \fi
}
%    \end{macrocode}

% \macro{\childdocof}
% The command |\childdocof| redirects
% compilation to the main file |#1|.
%    \begin{macrocode}
\newcommand{\childdocof}[1]
{
  \childdocdisable
  \childdoctrue
  \includeonly{\childdocname}
  \def\jobname{#1}
  \def\childdocjob{#1}
  \input{#1}
}
%    \end{macrocode}

% \macro{\childdocby}
% The command |\childdocby| ....
%    \begin{macrocode}
\newcommand{\childdocby}[2][]
{
  \childdocdisable
  \childdoctrue
  \childdocmanualtrue
  \if?#1?\else
    \def\jobname{#2}
  \fi
  \def\childdocjob{#2}
  \input{#2}
  \endinput
}
%    \end{macrocode}

% \macro{\childdocforward}
% The command |\childdocforward| redirects
% compilation to the main file or
% (if the optional argument is given) a child file.
% Parameters are set as if the main file
% or a child file starting with |\childdocof| was compiled.
% Then compilation is handed over to the main file:
%    \begin{macrocode}
\newcommand{\childdocforward}[2][]
{
  \begingroup
    \if?#1?
      \def\childdoctmp
      {
        \def\childdocname{#2}
        \def\childdocjob{#2}
        \def\jobname{#2}
        \input{#2}
        \endinput
      }
    \else
      \def\childdoctmp
      {
        \childdocdisable
        \def\childdocname{#2}
        \childdoctrue
        \includeonly{#2}
        \def\childdocjob{#1}
        \def\jobname{#1}
        \input{#1}
        \endinput
      }
    \fi
    \expandafter
  \endgroup
  \childdoctmp
}
%    \end{macrocode}

% \macro{\childdocforwardprefix}
% The command |\childdocforwardprefix| redirects
% compilation to the main or a child file by means of a pattern.
% The prefix |#1| in the current filename is replaced by |#2|
% and the suffix of the current filename is kept
% (it is assumed that the filename does not contain the substring `|~~~|'
% which is used as a delimiter).
% Compilation is handed over to the new file by |\childdocforward|:
%    \begin{macrocode}
\newcommand{\childdocforwardprefix}[3][]
{
  \begingroup
    \def\childdocextract #2##1~~~{\def\childdoctmp{\childdocforward[#1]{#3##1}}}
    \expandafter\childdocextract\childdocname~~~
    \expandafter
  \endgroup
  \childdoctmp
}
%    \end{macrocode}

% \macro{\childdoc}
% The deprecated macro |\childdoc| is a legacy version of |\childdocmain|:
%    \begin{macrocode}
\newcommand{\childdoc}{\childdocmain}
%    \end{macrocode}

% \macro{\childdocredirect}
% The deprecated macro |\childdocredirect| is a legacy version
% of |\childdocforward| and |\childdocforwardprefix|:
%    \begin{macrocode}
\newcommand{\childdocredirect}[2][]
{
  \begingroup
    \if?#1?
      \def\childdoctmp{\childdocforward{#2}}
    \else
      \def\childdoctmp{\childdocforwardprefix{#1}{#2}}
    \fi
    \expandafter
  \endgroup
  \childdoctmp
}
%    \end{macrocode}

%\iffalse
%</package>
%\fi
%
\endinput

\childdocof{cdocsamp}
%    \end{macrocode}

%\iffalse
%</samplechap1|samplechap2>
%\fi
%
%\iffalse
%<*samplechap1>
%\fi
% Some text for chapter 1:
%    \begin{macrocode}
\section{one}
some text in chapter one
%    \end{macrocode}

%\iffalse
%</samplechap1>
%\fi
% Some text for chapter 2:
%\iffalse
%<*samplechap2>
%\fi
%    \begin{macrocode}
\section{two}
more text in chapter two
%    \end{macrocode}

%\iffalse
%</samplechap2>
%\fi
%
% %%%%%%%%%%%%%%%%%%%%%%%%%%%%%%%%%%%%%%
% \paragraph{Part Include Files.}
%
% The include files are called |cdocspt3.tex| and |cdocspt4.tex|.
%
%\iffalse
%<*samplepart3|samplepart4>
%\fi

% Optional override for |\version| flag:
%    \begin{macrocode}
%%\providecommand{\version}{final}
%    \end{macrocode}

% Include the main document:
%    \begin{macrocode}
% \iffalse
%
% childdoc.dtx Copyright (C) 2017-2018 Niklas Beisert
%
% This work may be distributed and/or modified under the
% conditions of the LaTeX Project Public License, either version 1.3
% of this license or (at your option) any later version.
% The latest version of this license is in
%   http://www.latex-project.org/lppl.txt
% and version 1.3 or later is part of all distributions of LaTeX
% version 2005/12/01 or later.
%
% This work has the LPPL maintenance status `maintained'.
%
% The Current Maintainer of this work is Niklas Beisert.
%
% This work consists of the files childdoc.dtx and childdoc.ins
% and the derived files childdoc.def and cdocsamp.tex with
% cdocsch1.tex, cdocsch2.tex, cdocsdrf.tex, cdocsfn1.tex, cdocsfn2.tex.
%
%<package>\ifdefined\childdocmain\endinput\fi
%<package>\ProvidesFile{childdoc.def}[2018/12/30 v2.0 child document driver]
%<samplemain>\ProvidesFile{cdocsamp.tex}[2018/12/30 v2.0 sample for childdoc]
%<*driver>
%\ProvidesFile{childdoc.drv}[2018/12/30 v2.0 childdoc reference manual file]
\PassOptionsToClass{10pt,a4paper}{article}
\documentclass{ltxdoc}

\usepackage[margin=35mm]{geometry}
\usepackage{hyperref}
\usepackage{hyperxmp}
\usepackage[usenames]{color}

\hypersetup{colorlinks=true}
\hypersetup{pdfstartview=FitH}
\hypersetup{pdfpagemode=UseNone}
\hypersetup{pdfsource={}}
\hypersetup{pdflang={en-UK}}
\hypersetup{pdfcopyright={Copyright 2017-2018 Niklas Beisert.
  This work may be distributed and/or modified under the
  conditions of the LaTeX Project Public License, either version 1.3
  of this license or (at your option) any later version.}}
\hypersetup{pdflicenseurl={http://www.latex-project.org/lppl.txt}}
\hypersetup{pdfcontactaddress={ETH Zurich, ITP, HIT K,
  Wolfgang-Pauli-Strasse 27}}
\hypersetup{pdfcontactpostcode={8093}}
\hypersetup{pdfcontactcity={Zurich}}
\hypersetup{pdfcontactcountry={Switzerland}}
\hypersetup{pdfcontactemail={nbeisert@itp.phys.ethz.ch}}
\hypersetup{pdfcontacturl={http://people.phys.ethz.ch/\xmptilde nbeisert/}}

\newcommand{\secref}[1]{\hyperref[#1]{section \ref*{#1}}}

\parskip1ex
\parindent0pt
\let\olditemize\itemize
\def\itemize{\olditemize\parskip0pt}

\begin{document}

\title{The \textsf{childdoc} Package}
\hypersetup{pdftitle={The childdoc Package}}
\author{Niklas Beisert\\[2ex]
  Institut f\"ur Theoretische Physik\\
  Eidgen\"ossische Technische Hochschule Z\"urich\\
  Wolfgang-Pauli-Strasse 27, 8093 Z\"urich, Switzerland\\[1ex]
  \href{mailto:nbeisert@itp.phys.ethz.ch}
  {\texttt{nbeisert@itp.phys.ethz.ch}}}
\hypersetup{pdfauthor={Niklas Beisert}}
\hypersetup{pdfsubject={Manual for the LaTeX2e Package childdoc}}
\date{30 December 2018, \textsf{v2.0}}
\maketitle

\begin{abstract}\noindent
\textsf{childdoc} is a \LaTeXe{} package
that enables the direct compilation
of document sections included by |\include|
to individual files.
\end{abstract}

\begingroup
\parskip0ex
\tableofcontents
\endgroup

%%%%%%%%%%%%%%%%%%%%%%%%%%%%%%%%%%%%%%%%%%%%%%%%%%%%%%%%%%%%%%%%%%%%%%%%%%%%%%%%
%%%%%%%%%%%%%%%%%%%%%%%%%%%%%%%%%%%%%%%%%%%%%%%%%%%%%%%%%%%%%%%%%%%%%%%%%%%%%%%%
\section{Introduction}

\LaTeX{} provides a mechanism to structure a large document (such as a book)
into a main file and several child files (containing the chapters)
using the |\include| command.
This mechanism is beneficial for documents
which span hundreds of pages in order to
make the source file(s) more manageable.
Moreover, compilation can be restricted to
selected child files by means of the |\includeonly| command.
The latter feature can be used to reduce the compilation time while editing
(this was significantly more useful in the earlier days of \LaTeX{})
or to generate a smaller document which is easier to navigate.
Another application of |\includeonly| is to generate
documents consisting of selected parts of the complete document.

However, there are a few drawbacks of the plain |\include| mechanism:
\begin{itemize}
\item
The child files cannot be compiled on their own,
they can only be compiled via the main file.
A naive editing environment
(such as a text editor with an option
to have the current file processed by \LaTeX)
may require one to switch to the main file before compiling;
attempting to compile the child file produces errors.
\item
The main file must be modified (each time)
to adjust the |\includeonly| command
to the present needs. This easily leaves the main file in a messy state.
\item
The generated document will always carry the filename
of the main document. This is inconvenient if
several child files are to be compiled and
to be kept for distribution.
\end{itemize}

The present package provides a simple interface
to make child files individually compilable by \LaTeX{}.
Compiling a child file then has the same effect as compiling
the main file with an |\includeonly| command
to select the appropriate child.
Moreover the generated document will carry the name of the child
rather than the main file.
This resolves all three above issues.

This feature is meant to make the editing of books,
thesis documents and lecture notes somewhat more convenient.
However, the package can also be used efficiently for
composing a series of documents (such as exercise sheets)
which are typically distributed individually.
It then assists the author in generating the individual documents
(potentially in different versions)
as well as a document containing the collected series.
Another application is in developing style files
or other kinds of included material
where compilation of the style file could redirect
to a sample or test file.

%%%%%%%%%%%%%%%%%%%%%%%%%%%%%%%%%%%%%%%%%%%%%%%%%%%%%%%%%%%%%%%%%%%%%%%%%%%%%%%%
%%%%%%%%%%%%%%%%%%%%%%%%%%%%%%%%%%%%%%%%%%%%%%%%%%%%%%%%%%%%%%%%%%%%%%%%%%%%%%%%
\section{Usage}

First of all, the package \textsf{childdoc} is \emph{not} a standard
\LaTeXe{} |.sty| style file! Therefore it needs to be invoked in
a non-standard way.

%%%%%%%%%%%%%%%%%%%%%%%%%%%%%%%%%%%%%%%%%%%%%%%%%%%%%%%%%%%%%%%%%%%%%%%%%%%%%%%%
\subsection{Included Files}
\label{sec:include}

%%%%%%%%%%%%%%%%%%%%%%%%%%%%%%%%%%%%%%%%
\DescribeMacro{\childdocmain}
To use the package, add the commands
\begin{center}
\begin{tabular}{l}
|\input{childdoc.def}|\\
|\childdocmain{}|\\
\end{tabular}
\end{center}
at the very top of the main \LaTeX{} file,
in particular \emph{before} the |\documentclass| statement!
The argument of |\childdocmain| should be left empty
(but it must be present).

%%%%%%%%%%%%%%%%%%%%%%%%%%%%%%%%%%%%%%%%
\DescribeMacro{\childdocof}
Furthermore, add the commands
\begin{center}
\begin{tabular}{l}
|\input{childdoc.def}|\\
|\childdocof{|\textit{main}|}|\\
\end{tabular}
\end{center}
at the top of every child file \textit{child}
which is included by |\include{|\textit{child}|}|
from within the main file
(or at least for those files to be compiled individually).
The argument \textit{main} must be the filename of the main file.

There are a couple of
considerations in setting up the main and child documents:

%%%%%%%%%%%%%%%%%%%%%%%%%%%%%%%%%%%%%%%%
\paragraph{Restrictions.}

Please note the following restrictions:
\begin{itemize}
\item
|\childdocmain| must be called with one argument \textit{main}
to ensure compatibility with earlier version of the package.
It must either be empty (|\childdocmain{}|)
or precisely match the filename of the main file in which it is specified.
See \secref{sec:detection} for further information.
\item
The filename \textit{main} must be specified without the |.tex| extension.
\item
The filename \textit{main} is case sensitive
(even in case-insensitive file systems)
due to internal string comparison.
\item
The argument \textit{main} should be fully expanded, it cannot be a macro.
\item
Subdirectories and special characters should be avoided in filenames.
\item
The command |\childdocmain{|\textit{main}|}| must be followed by a whitespace.
It should not be followed immediately by another command
or by a comment mark `|%|'.
This is because the \TeX{} parser reads the token immediately following
the argument of |\childdocmain| and puts it
at the beginning of every child section;
however, a white\-space is ignored.
\end{itemize}

%%%%%%%%%%%%%%%%%%%%%%%%%%%%%%%%%%%%%%%%
\paragraph{Content of Main File.}

It is advisable to place all content in the child files included by |\include|.
Any output contained in the main file will appear in all child documents
unless suppressed manually;
it cannot be suppressed automatically by the |\includeonly| directive
and thus should normally be avoided.
A method to include some content in the main file
by means of conditional processing is described in \secref{sec:conditional}.

%%%%%%%%%%%%%%%%%%%%%%%%%%%%%%%%%%%%%%%%
\paragraph{Page Numbering.}

When only a part of the document is compiled,
the appropriate numbering of pages
(as well as other status parameters)
is determined from the |.aux| files.
The latter contain information from previous passes.
However this information needs to propagate through
all intermediate child documents.
Therefore the page numbering in child documents may well
be inconsistent until the complete document is compiled at least once.

A useful (if unconventional) way to always ensure a consistent
page numbering is to restart the numbering in each child document
and denote the pages by `\textit{child}|.|\textit{page}'
where \textit{child} represents the chapter/section number of the child file.
This can be achieved by the command
|\numberwithin{page}{|\textit{child}|}|
of the \textsf{amsmath} package
where \textit{child} can be |chapter| or |section|
depending on the chosen structuring.
Alternatively, one can modify the macro |\thepage| appropriately
and reset the counter |page| at the start of each child file.

%%%%%%%%%%%%%%%%%%%%%%%%%%%%%%%%%%%%%%%%%%%%%%%%%%%%%%%%%%%%%%%%%%%%%%%%%%%%%%%%
\subsection{Conditional Processing}
\label{sec:conditional}

The package provides a mechanism to compile different versions
of a document. To customise the versions further some conditional processing
can come in handy to distinguish which version is being compiled.
The package provides two macros to describe the compilation context:

%%%%%%%%%%%%%%%%%%%%%%%%%%%%%%%%%%%%%%%%
\DescribeMacro{\ifchilddoc}
The conditional |\ifchilddoc| distinguishes between the compilation of
child documents and the main document:
%
\begin{center}
|\ifchilddoc |\textit{child-code}| |[|\||else |\textit{main-code}]| \||fi|
\end{center}

%%%%%%%%%%%%%%%%%%%%%%%%%%%%%%%%%%%%%%%%
\DescribeMacro{\childdocname}
\DescribeMacro{\childdocjob}
The macro |\childdocname| contains the filename (without extension)
of the main or child file being processed.
Note that |\childdocjob| will always contain the name of the main file.

%%%%%%%%%%%%%%%%%%%%%%%%%%%%%%%%%%%%%%%%
\paragraph{Title Page.}

Conditional processing can be used to include a title or banner page
in the main document when proper precautions are taken.
Importantly, the code in the main file should ensure that the page counter
(as well as other status parameters which are stored in the |.aux| files)
takes the same value after the conditional processing.
Otherwise the page numbers may take divergent values
depending on which part is compiled.

For example, a title page could be declared by:
%
\begin{center}
\begin{tabular}{l}
|\ifchilddoc\||else|\\
|\addtocounter{page}{-1}|\\
\textit{code for title page}\\
|\newpage|\\
|\||fi|
\end{tabular}
\end{center}
%
A banner page for the child documents can be generated by:
%
\begin{center}
\begin{tabular}{l}
|\ifchilddoc|\\
|\addtocounter{page}{-1}|\\
\textit{code for banner page}\\
|\newpage|\\
|\||fi|
\end{tabular}
\end{center}
%
Here one could write a message such as:
\begin{center}
|This is the part \childdocname{} of \childdocjob{}.|
\end{center}

%%%%%%%%%%%%%%%%%%%%%%%%%%%%%%%%%%%%%%%%%%%%%%%%%%%%%%%%%%%%%%%%%%%%%%%%%%%%%%%%
\subsection{Flags}
\label{sec:flags}

The package makes it easy to generate different versions
of the main or child documents.
To this end compilation flags can be defined
and assigned different default values.
They will be particularly useful in conjunction
with the forwarding mechanism described in \secref{sec:forward}.

For example, it may be useful to have a flag |\version|
which can be set to |draft| or |final|.
The document source will contain some conditional code
depending on the value of |\version|.
Suppose further, the flag should default to |final| for the main file
and to |draft| for child files
which is a natural assignment for editing the document.
This is achieved by placing the following code
in the preamble of the main document
(below the |\childdocmain| directive):
%
\begin{center}
\begin{tabular}{l}
|\ifchilddoc|\\
|\providecommand{\version}{draft}|\\
|\||else|\\
|\providecommand{\version}{final}|\\
|\||fi|
\end{tabular}
\end{center}
%
The definition by |\providecommand| makes sure
that previous definitions are not overwritten.
Further statements |\providecommand{\version}{...}|
can thus be added before the above code to override it.

For the main file, one might add a line
(between |\childdocmain| and the above block)
%
\begin{center}
|%\ifchilddoc\||else\providecommand{\version}{draft}\||fi|
\end{center}
%
which can be uncommented to produce a draft version.
Likewise one can add a line to the very top of a child file
(above the |\childdocof{|\textit{main}|}| directive)
%
\begin{center}
|%\providecommand{\version}{final}|
\end{center}
%
which can be uncommented to produce the final version of this child document.

%%%%%%%%%%%%%%%%%%%%%%%%%%%%%%%%%%%%%%%%%%%%%%%%%%%%%%%%%%%%%%%%%%%%%%%%%%%%%%%%
\subsection{Forwarding}
\label{sec:forward}

Different versions of the main or child documents
using compilation flags as described in \secref{sec:flags}
can be (permanently) stored in different files
for convenient compilation, viewing and distribution.
To this end, the package defines a command
to pass on compilation to a different file:

%%%%%%%%%%%%%%%%%%%%%%%%%%%%%%%%%%%%%%%%
\DescribeMacro{\childdocforward}
The command |\childdocforward| redirects processing to
another source file:
%
\begin{center}
\begin{tabular}{l}
|\input{childdoc.def}|\\
|\childdocforward[|\textit{main}|]{|\textit{dest}|}|\\
\end{tabular}
\end{center}
%
The argument \textit{dest} is the destination file
(without extension).
It should be the main file or one of the child files.
Note that further \textsf{childdoc} directives
such as |\childdocof| and |\childdocforward|
in the indicated file will be processed in this form.
The optional argument \textit{main}
passes on directly to the main file \textit{main}
while pretending to compile the child \textit{dest}.
This form behaves as if \textit{dest}
issues |\childdocof{|\textit{main}|}| right away,
and no further \textsf{childdoc} directives will be processed.

%%%%%%%%%%%%%%%%%%%%%%%%%%%%%%%%%%%%%%%%
\DescribeMacro{\...prefix}
In the alternative form |\childdocforwardprefix|,
%
\begin{center}
\begin{tabular}{l}
|\input{childdoc.def}|\\
|\childdocforwardprefix[|\textit{main}|]{|\textit{prefix}|}{|\textit{dest}|}|
\end{tabular}
\end{center}
%
the destination file is determined by a pattern
depending on the current file:
To make this work, the current file must be called
`{\textit{prefix}\hspace{0.2em}\textit{suffix}}'
with \textit{prefix} matching precisely the argument.
Processing is then passed on to the file
`{\textit{dest}\hspace{0.2em}\textit{suffix}}'.
Surely, the same effect is achieved by
directly specifying the
argument `{\textit{dest}\hspace{0.2em}\textit{suffix}}'
in the first form.
However, that requires to set up a different file
for each child. With the alternative form of the command
all these files can have exactly the same content
which simplifies setting them up and maintaining them.

For example, the following file |draft.tex|
with a compilation flag |\version| as described in \secref{sec:flags}
compiles the main document as a draft:
%
\begin{center}
\begin{tabular}{l}
|\def\version{draft}|\\
|\input{childdoc.def}|\\
|\childdocforward{|\textit{main}|}|
\end{tabular}
\end{center}
%
Likewise, the following files |final|\textit{nn}|.tex|
compile the final version of the child document
|child|\textit{nn}|.tex|:
%
\begin{center}
\begin{tabular}{l}
|\def\version{final}|\\
|\input{childdoc.def}|\\
|\childdocforwardprefix{final}{child}|
\end{tabular}
\end{center}
%

Note that when several versions of a main file and/or of each child file
are to be generated, it may be convenient to set up a |Makefile| or
shell script to automatise the process.

%%%%%%%%%%%%%%%%%%%%%%%%%%%%%%%%%%%%%%%%%%%%%%%%%%%%%%%%%%%%%%%%%%%%%%%%%%%%%%%%
\subsection{Command Line Processing}
\label{sec:commandline}

The effect of redirection files can also be achieved by invoking
the \LaTeX{} compiler with a more elaborate command line.
Most conveniently this should be done as part
of a shell script or a |Makefile|.

When using \textsf{childdoc} in the main file, the following
command lines effectively perform a redirection
(note that depending on the shell being used,
backslashes may have to be doubled: `|\|' $\to$ `|\\|'):
%
\begin{center}
|... -jobname "|\textit{target}|" |\\|"|[\textit{flags}]%
|\input{childdoc.def}\childdocforward[|\textit{main}|]{|\textit{dest}|}"|
\end{center}
%
Here \textit{target} is the name of the output file,
\textit{main} is the name of the main file
and \textit{dest} is the name of the main or child file to be processed
(all filenames without extensions).
The optional argument \textit{main} can be omitted
if \textit{main} matches \textit{dest}.
Optionally, compilation \textit{flags} can be defined via |\def| commands.
This command line makes the \TeX{} engine believe
it is compiling the file \textit{target}
whose content is specified as the latter parameter.
The provided code then forwards the processing to
\textit{main} or \textit{dest} as described in \secref{sec:forward}.

%%%%%%%%%%%%%%%%%%%%%%%%%%%%%%%%%%%%%%%%%%%%%%%%%%%%%%%%%%%%%%%%%%%%%%%%%%%%%%%%
\subsection{Include by Input}
\label{sec:input}

Including child documents by |\include| has some restrictions by design.
Most notably, the content of a child document always occupies
its own set of pages; pages cannot be shared between child documents.
Usually, this behaviour makes perfect sense
because each child document contain an essential part of the document.
However, in some situations it may be desirable to compose
a document from a collection of parts
without having mandatory page breaks between then.
For this case, the package
provides a mechanism to include parts
by |\input| which can also be processed individually.
However, by construction this mechanism
requires manual handling of the content to be output.

%%%%%%%%%%%%%%%%%%%%%%%%%%%%%%%%%%%%%%%%
\DescribeMacro{\ifchilddocmanual}
The main file should be prepared as usual, see \secref{sec:include}.
However, the document body must make a distinction
between processing of an individual part and of the main document, e.g.:
%
\begin{center}
\begin{tabular}{l}
|\ifchilddocmanual|\\
|\input{\childdocname}|\\
|\||else|\\
\textit{document body with }|\input{|\textit{part}|}|\\
|\||fi|
\end{tabular}
\end{center}
%
The conditional |\ifchilddocmanual| is true whenever
a part to be included by |\input| is being compiled,
and the name of the part is stored in |\childdocname|.

%%%%%%%%%%%%%%%%%%%%%%%%%%%%%%%%%%%%%%%%
\DescribeMacro{\childdocby}
Each part to be included by |\input| should start with:
%
\begin{center}
\begin{tabular}{l}
|\input{childdoc.def}|\\
|\childdocby{|\textit{main}|}|\\
\end{tabular}
\end{center}
%
The directive |\childdocby| is similar to |\childdocof|
described in \secref{sec:include},
but the subsequent selection of content must be done manually.
To that end, both |\ifchilddoc| and |\ifchilddocmanual|
will be true upon processing of a part,
and the name of the part is stored in |\childdocname|.
Note that |\jobname| will be set to the filename of the current part
so that each part receives an individual |.aux| file
that does not interfere with the |.aux| file(s) of the main document.
This behaviour can be altered by the alternative form
|\childdocby[*]{|\textit{main}|}| (with a non-empty optional argument)
which uses the |.aux| file of the main document
by setting |\jobname| to \textit{main}.

%%%%%%%%%%%%%%%%%%%%%%%%%%%%%%%%%%%%%%%%%%%%%%%%%%%%%%%%%%%%%%%%%%%%%%%%%%%%%%%%
\subsection{Driver Development}
\label{sec:driver}

The \textsf{childdoc} mechanism can also be use for the development
of definition files such as \LaTeX{} styles or classes.
This case differs from the above setup with multiple parts
included by |\include| in that no |\includeonly| should be invoked.
This can be achieved by starting the include file
(before |\ProvidesPackage|) with:
%
\begin{center}
\begin{tabular}{l}
|\input{childdoc.def}|\\
|\childdocforward{|\textit{main}|}|\\
\end{tabular}
\end{center}
%
or alternatively with:
%
\begin{center}
\begin{tabular}{l}
|\input{childdoc.def}|\\
|\childdocby{|\textit{main}|}|\\
\end{tabular}
\end{center}
%
Both forms have slightly different effects as described above.
The main file is prepared as usual, see \secref{sec:include}.

%%%%%%%%%%%%%%%%%%%%%%%%%%%%%%%%%%%%%%%%%%%%%%%%%%%%%%%%%%%%%%%%%%%%%%%%%%%%%%%%
\subsection{Legacy Detection}
\label{sec:detection}

The directive |\childdocmain| in the main file can detect
whether the complete document or merely a child is to be compiled
even without using the directive |\childdocof|.
This method is deprecated because it is less robust
and there is no compelling reason to use it;
it is merely provided for backward compatibility
and it may be removed in future versions.

If the detection mechanism is to be used,
it is mandatory to correctly specify
the filename of the main file as the argument of |\childdocmain|:
%
\begin{center}
\begin{tabular}{l}
|\input{childdoc.def}|\\
|\childdocmain{|\textit{main}|}|\\
\end{tabular}
\end{center}
%
If |\jobname| does not match the argument \textit{main} of |\childdocmain|,
it is assumed that |\jobname| points to the child file to be compiled.
When using |\childdocmain| with the main file specified as argument,
it suffices to start a child file
with just |\input{|\textit{main}|}|
without loading of the package and using |\childdocof|.
If instead all processing is done
with the appropriate \textsf{childdoc} directives,
the argument of \textit{main} of |\childdocmain| can be empty.

An alternative version of the command line processing described
in \secref{sec:commandline} using the detection mechanism reads:
%
\begin{center}
|... -jobname "|\textit{target}|" "|[\textit{flags}]%
[|\def\jobname{|\textit{dest}|}|]|\input{|\textit{main}|}"|
\end{center}

%%%%%%%%%%%%%%%%%%%%%%%%%%%%%%%%%%%%%%%%%%%%%%%%%%%%%%%%%%%%%%%%%%%%%%%%%%%%%%%%
\subsection{Manual Code}
\label{sec:manual}

In case one cannot be certain whether the definitions file |childdoc.def|
is installed on the target \TeX{} distribution
and one prefers not to ship it,
it is conceivable to paste a few relevant commands into the sources.

To that end, drop all statements |\input{childdoc.def}|
and perform the replacements as outlined below.
Instead of |\childdocmain{|\textit{main}|}| add the following code
to the top of the main file:
%
\begin{center}
\begin{tabular}{l}
|\||ifdefined\childdocname\endinput\||fi\newif\ifchilddoc|\\
|\edef\childdocname{\scantokens\expandafter{\jobname\noexpand}}|\\
|\def\childdocmain{|\textit{main}|}\||ifx\childdocmain\childdocname\||else|\\
|\childdoctrue\includeonly{\childdocname}\let\jobname\childdocmain\||fi|\\
\end{tabular}
\end{center}
%
Instead of |\childdocof{|\textit{main}|}| just include the main file
at the top of each child file:
%
\begin{center}
|\input{|\textit{main}|}|
\end{center}
%
A simple redirection |\childdocforward{|\textit{dest}|}| is achieved by:
%
\begin{center}
|\def\jobname{|\textit{dest}|}\input{\jobname}|
\end{center}
%
The redirection with prefix
|\childdocforwardprefix[|\textit{prefix}|]{|\textit{dest}|}|
is accomplished by:
%
\begin{center}
\begin{tabular}{l}
|{\edef\jobname{\scantokens\expandafter{\jobname\noexpand}}|\\
|\def\redirectjob |\textit{prefix}|#1~~~{\gdef\jobname{|\textit{dest}|#1}}|\\
|\expandafter\redirectjob\jobname~~~}\input{\jobname}|
\end{tabular}
\end{center}

In an alternative approach,
child documents can be compiled by a specific command line
without additional code or specific definitions:
%
\begin{center}
|... -jobname "|\textit{target}|" "|[\textit{flags}]%
|\includeonly{|\textit{dest}|}\input{|\textit{main}|}"|
\end{center}
%

%%%%%%%%%%%%%%%%%%%%%%%%%%%%%%%%%%%%%%%%%%%%%%%%%%%%%%%%%%%%%%%%%%%%%%%%%%%%%%%%
%%%%%%%%%%%%%%%%%%%%%%%%%%%%%%%%%%%%%%%%%%%%%%%%%%%%%%%%%%%%%%%%%%%%%%%%%%%%%%%%
\section{Information}

%%%%%%%%%%%%%%%%%%%%%%%%%%%%%%%%%%%%%%%%%%%%%%%%%%%%%%%%%%%%%%%%%%%%%%%%%%%%%%%%
\subsection{Copyright}

Copyright \copyright{} 2017--2018 Niklas Beisert

This work may be distributed and/or modified under the
conditions of the \LaTeX{} Project Public License, either version 1.3
of this license or (at your option) any later version.
The latest version of this license is in
  \url{http://www.latex-project.org/lppl.txt}
and version 1.3 or later is part of all distributions of \LaTeX{}
version 2005/12/01 or later.

This work has the LPPL maintenance status `maintained'.

The Current Maintainer of this work is Niklas Beisert.

This work consists of the files |README.txt|, |childdoc.ins| and |childdoc.dtx|
as well as the derived files |childdoc.def|, |cdocsamp.tex|
with |cdocsch1.tex|, |cdocsch2.tex|, |cdocspt3.tex|, |cdocspt4.tex|,
|cdocsdrf.tex|, |cdocsfn1.tex|, |cdocsfn2.tex|
as well as |childdoc.pdf|.

%%%%%%%%%%%%%%%%%%%%%%%%%%%%%%%%%%%%%%%%%%%%%%%%%%%%%%%%%%%%%%%%%%%%%%%%%%%%%%%%
\subsection{Files and Installation}

The package consists of the files:
%
\begin{center}
\begin{tabular}{ll}
    |README.txt|   & readme file \\
    |childdoc.ins| & installation file \\
    |childdoc.dtx| & source file \\
    |childdoc.def| & definition file \\
    |cdocsamp.tex| & sample main file \\
    |cdocsch1.tex| & sample include file \\
    |cdocsch2.tex| & sample include file \\
    |cdocspt3.tex| & sample part file \\
    |cdocspt4.tex| & sample part file \\
    |cdocsdrf.tex| & sample redirection file \\
    |cdocsfn1.tex| & sample redirection file \\
    |cdocsfn2.tex| & sample redirection file \\
    |childdoc.pdf| & manual
\end{tabular}
\end{center}
%
The distribution consists of the files
|README.txt|, |childdoc.ins| and |childdoc.dtx|.
%
\begin{itemize}
\item
Run (pdf)\LaTeX{} on |childdoc.dtx|
to compile the manual |childdoc.pdf| (this file).
\item
Run \LaTeX{} on |childdoc.ins| to create the definitions file |childdoc.def|
and the sample |cdocsamp.tex| with include files
|cdocsch1.tex|, |cdocsch2.tex|, |cdocspt3.tex|, |cdocspt4.tex|,
|cdocsdrf.tex|, |cdocsfn1.tex|, |cdocsfn2.tex|.
Then copy the file |childdoc.def| to an appropriate directory of your \LaTeX{}
distribution, e.g.\ \textit{texmf-root}|/tex/latex/childdoc|.
\end{itemize}

%%%%%%%%%%%%%%%%%%%%%%%%%%%%%%%%%%%%%%%%%%%%%%%%%%%%%%%%%%%%%%%%%%%%%%%%%%%%%%%%
\subsection{Related CTAN Packages}

There are several other packages which offer a similar functionality:
%
\begin{itemize}
\item
The packages
\href{http://ctan.org/pkg/docmute}{\textsf{docmute}},
\href{http://ctan.org/pkg/includex}{\textsf{includex}} and
\href{http://ctan.org/pkg/standalone}{\textsf{standalone}}
provide commands to include only the document body of
a child file thus allowing both files to be compiled individually.
\item
The packages \href{http://ctan.org/pkg/subdocs}{\textsf{subdocs}}
and \href{http://ctan.org/pkg/subfiles}{\textsf{subfiles}}
provide structures in which the main and child documents can be
encapsulated and allowing them to be compiled individually.
The inclusion mechanism is different from the conventional |\include|.
\item
The package \href{http://ctan.org/pkg/combine}{\textsf{combine}}
is an elaborate solution to combine several documents into one.
\end{itemize}
%
See also the CTAN topic \href{http://ctan.org/topic/subdocs}{\textsf{subdocs}}
for further related packages.
The present package differs from the above solutions in that
a document structure constructed with the conventional |\include| mechanism
just needs two extra commands at the top of every file
such that all constituent files can be compiled individually.

%%%%%%%%%%%%%%%%%%%%%%%%%%%%%%%%%%%%%%%%%%%%%%%%%%%%%%%%%%%%%%%%%%%%%%%%%%%%%%%%
%\subsection{Feature Suggestions}
%
%The following is a list of features which may be useful for future
%versions of this package:
%%
%\begin{itemize}
%\item
%\ldots
%\end{itemize}

%%%%%%%%%%%%%%%%%%%%%%%%%%%%%%%%%%%%%%%%%%%%%%%%%%%%%%%%%%%%%%%%%%%%%%%%%%%%%%%%
\subsection{Revision History}

%%%%%%%%%%%%%%%%%%%%%%%%%%%%%%%%%%%%%%%%
\paragraph{v2.0:} 2018/12/30

\begin{itemize}
\item
immediate forward processing
\item
added |\childdocby| mechanism
\item
manual restructured
\end{itemize}

%%%%%%%%%%%%%%%%%%%%%%%%%%%%%%%%%%%%%%%%
\paragraph{v1.6:} 2018/01/17

\begin{itemize}
\item
application for development of include files
\item
corrections to manual
\end{itemize}

%%%%%%%%%%%%%%%%%%%%%%%%%%%%%%%%%%%%%%%%
\paragraph{v1.5:} 2017/05/21

\begin{itemize}
\item
more complete structuring introduced
\item
|\childdocof| introduced
\item
|\childdoc| renamed to |\childdocmain|
\item
|\childredirect| renamed to |\childdocforward| and |\childdocforwardprefix|
and functionality expanded
\end{itemize}

%%%%%%%%%%%%%%%%%%%%%%%%%%%%%%%%%%%%%%%%
\paragraph{v1.0:} 2017/04/27

\begin{itemize}
\item
manual and install package
\item
first version published on CTAN
\end{itemize}

%%%%%%%%%%%%%%%%%%%%%%%%%%%%%%%%%%%%%%%%
\paragraph{v0.6:} 2017/04/26

\begin{itemize}
\item
redirection mechanism added
\end{itemize}

%%%%%%%%%%%%%%%%%%%%%%%%%%%%%%%%%%%%%%%%
\paragraph{v0.5:} 2017/04/26

\begin{itemize}
\item
functionality in definition file
\end{itemize}


%%%%%%%%%%%%%%%%%%%%%%%%%%%%%%%%%%%%%%%%%%%%%%%%%%%%%%%%%%%%%%%%%%%%%%%%%%%%%%%%
%%%%%%%%%%%%%%%%%%%%%%%%%%%%%%%%%%%%%%%%%%%%%%%%%%%%%%%%%%%%%%%%%%%%%%%%%%%%%%%%
%%%%%%%%%%%%%%%%%%%%%%%%%%%%%%%%%%%%%%%%%%%%%%%%%%%%%%%%%%%%%%%%%%%%%%%%%%%%%%%%
\appendix

\settowidth\MacroIndent{\rmfamily\scriptsize 000\ }

 \DocInput{childdoc.dtx}

\end{document}
%</driver>
% \fi
%
% %%%%%%%%%%%%%%%%%%%%%%%%%%%%%%%%%%%%%%%%%%%%%%%%%%%%%%%%%%%%%%%%%%%%%%%%%%%%%%
% %%%%%%%%%%%%%%%%%%%%%%%%%%%%%%%%%%%%%%%%%%%%%%%%%%%%%%%%%%%%%%%%%%%%%%%%%%%%%%
% \section{Sample}
%\iffalse
%<*samplemain>
%\fi
%
% The following presents a sample document
% with two chapters, two parts, a title page,
% a compile flag as well as three forwarding files to set the flag.
% It consists of eight |.tex| files:
% \begin{center}
% \begin{tabular}{ll}
% |cdocsamp.tex|&main file\\
% |cdocsch1.tex|&include file for chapter 1\\
% |cdocsch2.tex|&include file for chapter 2\\
% |cdocspt3.tex|&include file for part 3\\
% |cdocspt4.tex|&include file for part 4\\
% |cdocsdrf.tex|&forwarding file for main file in draft mode\\
% |cdocsfi1.tex|&forwarding file for final version of chapter 1\\
% |cdocsfi2.tex|&forwarding file for final version of chapter 2\\
% \end{tabular}
% \end{center}
% Each of the eight files can be compiled directly by the \LaTeX{} compiler.
%
% %%%%%%%%%%%%%%%%%%%%%%%%%%%%%%%%%%%%%%
% \paragraph{Main File.}
%
% The main file is called |cdocsamp.tex|.
%
% Load the \textsf{childdoc} definitions and
% declare the filename for the main document:
%    \begin{macrocode}
\input{childdoc.def}
\childdocmain{}
%    \end{macrocode}

% Optional override for |\version| flag:
%    \begin{macrocode}
%%\ifchilddoc\else\providecommand{\version}{draft}\fi
%    \end{macrocode}

% Define the default values for the |\version| flag
% (|final| for the main file and |draft| for childs):
%    \begin{macrocode}
\ifchilddoc
\providecommand{\version}{draft}
\else
\providecommand{\version}{final}
\fi
%    \end{macrocode}

% Load the standard document class:
%    \begin{macrocode}
\documentclass[12pt]{article}
%    \end{macrocode}

% Start the document body:
%    \begin{macrocode}
\begin{document}
%    \end{macrocode}

% Declare a title page.
% Print title, part of document being processed and version flag:
%    \begin{macrocode}
\addtocounter{page}{-1}
\begin{center}
{\LARGE\bfseries{}childdoc example\par}
\vspace{1cm}
\ifchilddoc
\ifchilddocmanual part\else chapter\fi:
`\childdocname' of `\childdocjob'\par
\else
main document: `\childdocjob'\par
\fi
version: \version\par
\end{center}
\newpage
%    \end{macrocode}

% Manually include selected file,
% otherwise process as usual:
%    \begin{macrocode}
\ifchilddocmanual
\section*{part `\childdocname'}
\input{\childdocname}
\else
%    \end{macrocode}

% Include the two chapters:
%    \begin{macrocode}
\include{cdocsch1}
\include{cdocsch2}
%    \end{macrocode}

% Include the two parts unless only chapters should be displayed:
%    \begin{macrocode}
\ifchilddoc\else
\section{part three}
\input{cdocspt3}
\section{part four}
\input{cdocspt4}
\fi
%    \end{macrocode}

% Process as usual until here:
%    \begin{macrocode}
\fi
%    \end{macrocode}

% End of document body:
%    \begin{macrocode}
\end{document}
%    \end{macrocode}
%\iffalse
%</samplemain>
%\fi
%
% %%%%%%%%%%%%%%%%%%%%%%%%%%%%%%%%%%%%%%
% \paragraph{Chapter Include Files.}
%
% The include files are called |cdocsch1.tex| and |cdocsch2.tex|.
%
%\iffalse
%<*samplechap1|samplechap2>
%\fi

% Optional override for |\version| flag:
%    \begin{macrocode}
%%\providecommand{\version}{final}
%    \end{macrocode}

% Include the main document:
%    \begin{macrocode}
\input{childdoc.def}
\childdocof{cdocsamp}
%    \end{macrocode}

%\iffalse
%</samplechap1|samplechap2>
%\fi
%
%\iffalse
%<*samplechap1>
%\fi
% Some text for chapter 1:
%    \begin{macrocode}
\section{one}
some text in chapter one
%    \end{macrocode}

%\iffalse
%</samplechap1>
%\fi
% Some text for chapter 2:
%\iffalse
%<*samplechap2>
%\fi
%    \begin{macrocode}
\section{two}
more text in chapter two
%    \end{macrocode}

%\iffalse
%</samplechap2>
%\fi
%
% %%%%%%%%%%%%%%%%%%%%%%%%%%%%%%%%%%%%%%
% \paragraph{Part Include Files.}
%
% The include files are called |cdocspt3.tex| and |cdocspt4.tex|.
%
%\iffalse
%<*samplepart3|samplepart4>
%\fi

% Optional override for |\version| flag:
%    \begin{macrocode}
%%\providecommand{\version}{final}
%    \end{macrocode}

% Include the main document:
%    \begin{macrocode}
\input{childdoc.def}
\childdocby{cdocsamp}
%    \end{macrocode}

%\iffalse
%</samplepart3|samplepart4>
%\fi
%
%\iffalse
%<*samplepart3>
%\fi
% Some text for part 3:
%    \begin{macrocode}
some text in part three
%    \end{macrocode}

%\iffalse
%</samplepart3>
%\fi
% Some text for part 4:
%\iffalse
%<*samplepart4>
%\fi
%    \begin{macrocode}
more text in part four
%    \end{macrocode}

%\iffalse
%</samplepart4>
%\fi
%
% %%%%%%%%%%%%%%%%%%%%%%%%%%%%%%%%%%%%%%
% \paragraph{Forwarding for a Complete Draft.}
%
% The following forwarding file |cdocsdrf.tex|
% compiles the main document in draft mode:
%\iffalse
%<*sampledraft>
%\fi
%    \begin{macrocode}
\def\version{draft}
\input{childdoc.def}
\childdocforward{cdocsamp}
%    \end{macrocode}

%\iffalse
%</sampledraft>
%\fi
%
% %%%%%%%%%%%%%%%%%%%%%%%%%%%%%%%%%%%%%%
% \paragraph{Forwarding for Final Version of the Chapters.}
%
% The following forwarding files |cdocsfn1.tex| and |cdocsfn2.tex|
% (with identical content)
% compile the final versions of the child documents
% |cdocsch1.tex| and |cdocsch2.tex|, respectively:
%\iffalse
%<*samplefinal>
%\fi
%    \begin{macrocode}
\def\version{final}
\input{childdoc.def}
\childdocforwardprefix[cdocsamp]{cdocsfn}{cdocsch}
%    \end{macrocode}

%\iffalse
%</samplefinal>
%\fi
%
% %%%%%%%%%%%%%%%%%%%%%%%%%%%%%%%%%%%%%%
% \paragraph{Command Line Processing.}
%
% The following three command lines generate the output files
% |cdocscld|, |cdocscl1| and |cdocscl2|
% which should be identical to
% |cdocsdrf|, |cdocsch1| and |cdocsfn2|, respectively:
% \begin{center}
% \begin{tabular}{l}
% |latex -jobname cdocscld \|\\
% |  "\def\version{draft}\input{childdoc.def}\childdocforward{cdocsamp}"|\\
% |latex -jobname cdocscl1 \|\\
% |  "\input{childdoc.def}\childdocforward[cdocsamp]{cdocsch1}"|\\
% |latex -jobname cdocscl2 \|\\
% |  "\def\version{final}\input{childdoc.def}\childdocforward{cdocsch2}"|
% \end{tabular}
% \end{center}
% Note that the trailing backslash on each first line
% merely continues the input to the second line
% (for convenient cut ant paste).
% Furthermore, the command |latex| can be replaced by any
% of its alternative versions such as |pdflatex|.
%
% %%%%%%%%%%%%%%%%%%%%%%%%%%%%%%%%%%%%%%%%%%%%%%%%%%%%%%%%%%%%%%%%%%%%%%%%%%%%%%
% %%%%%%%%%%%%%%%%%%%%%%%%%%%%%%%%%%%%%%%%%%%%%%%%%%%%%%%%%%%%%%%%%%%%%%%%%%%%%%
% \section{Implementation}
%\iffalse
%<*package>
%\fi
%
% This section describes the definitions file |childdoc.def|.

% The definitions cannot be loaded using |\usepackage| or |\RequirePackage|
% which has a mechanism to prevent loading a style file more than once.
% When loading the definitions by means of |\input|
% multiple instances have to be prevented manually:
%\iffalse
%This code needs to be before the `\ProvidesFile' directive
%which is defined at the beginning of this file.
%Therefore it is also placed there and commented out here.
%</package>
%<*discard>
%\fi
%    \begin{macrocode}
\ifdefined\childdocmain\endinput\fi
%    \end{macrocode}
%\iffalse
%</discard>
%<*package>
%\fi
%
% \macro{\ifchilddoc}
% \macro{\ifchilddocmanual}
% The conditional |\ifchilddoc| tells whether a
% child (true) or main (false) document is being compiled.
% The conditional |\ifchilddocmanual| tells whether
% the |\includeonly| mechanism is used (false) or
% the selection of child files must be performed manually (true).
% The definitions initialise to false:
%    \begin{macrocode}
\newif\ifchilddoc
\newif\ifchilddocmanual
%    \end{macrocode}

% \macro{\childdocname}
% \macro{\childdocjob}
% The macro |\childdocname| stores the name of the main document
% to be compiled. The macro |\childdocjob| stores the name of
% the document on which the \LaTeX{} compiler was originally invoked.
% The content of |\jobname| cannot be compared
% to filenames specified in the source due to different catcodes.
% The following code rescans |\jobname|, stores the result
% in |\childdocname| and saves a copy in |\childdocjob|:
%    \begin{macrocode}
\edef\childdocname{\scantokens\expandafter{\jobname\noexpand}}
\let\childdocjob\childdocname
%    \end{macrocode}

% \macro{\childdocdisable}
% The macro |\childdocdisable| prevents the main file
% from being processed more than once.
% At this stage, the main document command |\childdocmain|
% is assumed to be called once again where it should do nothing.
% Any subsequent call to it should prevent
% a secondary processing of the main document
% It overwrites the forwarding commands
% |\childdocof| and |\childdocforward|
% with empty macros to prevent further inclusions of the main document:
%    \begin{macrocode}
\newcommand{\childdocdisable}
{
  \renewcommand{\childdocmain}[1]{\renewcommand{\childdocmain}[1]{\endinput}}
  \renewcommand{\childdocof}[1]{}
  \renewcommand{\childdocby}[2][]{}
  \renewcommand{\childdocforward}[2][]{}
  \renewcommand{\childdocdisable}{}
}
%    \end{macrocode}

% \macro{\childdocmain}
% The macro |\childdocmain| is to be called at the top of the main file
% with nothing or the main filename (without extension) as argument.
% First, it breaks loops.
% If the argument is not empty and does not match |\childdocname|
% (which is set by the first inclusion of |childdoc.def|),
% |\ifchilddoc| is set to true, |\includeonly| is applied to the child file
% and |\jobname| is set to the main file
% (for proper handling of |.aux| files):
%    \begin{macrocode}
\newcommand{\childdocmain}[1]
{
  \childdocdisable\childdocmain{}
  \if?#1?\else
    \begingroup
      \def\childdoctmp{#1}
      \ifx\childdoctmp\childdocname
        \def\childdoctmp{}
      \else
        \def\childdoctmp
        {
          \childdoctrue
          \includeonly{\childdocname}
          \def\childdocjob{#1}
          \def\jobname{#1}
        }
      \fi
      \expandafter
    \endgroup
    \childdoctmp
  \fi
}
%    \end{macrocode}

% \macro{\childdocof}
% The command |\childdocof| redirects
% compilation to the main file |#1|.
%    \begin{macrocode}
\newcommand{\childdocof}[1]
{
  \childdocdisable
  \childdoctrue
  \includeonly{\childdocname}
  \def\jobname{#1}
  \def\childdocjob{#1}
  \input{#1}
}
%    \end{macrocode}

% \macro{\childdocby}
% The command |\childdocby| ....
%    \begin{macrocode}
\newcommand{\childdocby}[2][]
{
  \childdocdisable
  \childdoctrue
  \childdocmanualtrue
  \if?#1?\else
    \def\jobname{#2}
  \fi
  \def\childdocjob{#2}
  \input{#2}
  \endinput
}
%    \end{macrocode}

% \macro{\childdocforward}
% The command |\childdocforward| redirects
% compilation to the main file or
% (if the optional argument is given) a child file.
% Parameters are set as if the main file
% or a child file starting with |\childdocof| was compiled.
% Then compilation is handed over to the main file:
%    \begin{macrocode}
\newcommand{\childdocforward}[2][]
{
  \begingroup
    \if?#1?
      \def\childdoctmp
      {
        \def\childdocname{#2}
        \def\childdocjob{#2}
        \def\jobname{#2}
        \input{#2}
        \endinput
      }
    \else
      \def\childdoctmp
      {
        \childdocdisable
        \def\childdocname{#2}
        \childdoctrue
        \includeonly{#2}
        \def\childdocjob{#1}
        \def\jobname{#1}
        \input{#1}
        \endinput
      }
    \fi
    \expandafter
  \endgroup
  \childdoctmp
}
%    \end{macrocode}

% \macro{\childdocforwardprefix}
% The command |\childdocforwardprefix| redirects
% compilation to the main or a child file by means of a pattern.
% The prefix |#1| in the current filename is replaced by |#2|
% and the suffix of the current filename is kept
% (it is assumed that the filename does not contain the substring `|~~~|'
% which is used as a delimiter).
% Compilation is handed over to the new file by |\childdocforward|:
%    \begin{macrocode}
\newcommand{\childdocforwardprefix}[3][]
{
  \begingroup
    \def\childdocextract #2##1~~~{\def\childdoctmp{\childdocforward[#1]{#3##1}}}
    \expandafter\childdocextract\childdocname~~~
    \expandafter
  \endgroup
  \childdoctmp
}
%    \end{macrocode}

% \macro{\childdoc}
% The deprecated macro |\childdoc| is a legacy version of |\childdocmain|:
%    \begin{macrocode}
\newcommand{\childdoc}{\childdocmain}
%    \end{macrocode}

% \macro{\childdocredirect}
% The deprecated macro |\childdocredirect| is a legacy version
% of |\childdocforward| and |\childdocforwardprefix|:
%    \begin{macrocode}
\newcommand{\childdocredirect}[2][]
{
  \begingroup
    \if?#1?
      \def\childdoctmp{\childdocforward{#2}}
    \else
      \def\childdoctmp{\childdocforwardprefix{#1}{#2}}
    \fi
    \expandafter
  \endgroup
  \childdoctmp
}
%    \end{macrocode}

%\iffalse
%</package>
%\fi
%
\endinput

\childdocby{cdocsamp}
%    \end{macrocode}

%\iffalse
%</samplepart3|samplepart4>
%\fi
%
%\iffalse
%<*samplepart3>
%\fi
% Some text for part 3:
%    \begin{macrocode}
some text in part three
%    \end{macrocode}

%\iffalse
%</samplepart3>
%\fi
% Some text for part 4:
%\iffalse
%<*samplepart4>
%\fi
%    \begin{macrocode}
more text in part four
%    \end{macrocode}

%\iffalse
%</samplepart4>
%\fi
%
% %%%%%%%%%%%%%%%%%%%%%%%%%%%%%%%%%%%%%%
% \paragraph{Forwarding for a Complete Draft.}
%
% The following forwarding file |cdocsdrf.tex|
% compiles the main document in draft mode:
%\iffalse
%<*sampledraft>
%\fi
%    \begin{macrocode}
\def\version{draft}
% \iffalse
%
% childdoc.dtx Copyright (C) 2017-2018 Niklas Beisert
%
% This work may be distributed and/or modified under the
% conditions of the LaTeX Project Public License, either version 1.3
% of this license or (at your option) any later version.
% The latest version of this license is in
%   http://www.latex-project.org/lppl.txt
% and version 1.3 or later is part of all distributions of LaTeX
% version 2005/12/01 or later.
%
% This work has the LPPL maintenance status `maintained'.
%
% The Current Maintainer of this work is Niklas Beisert.
%
% This work consists of the files childdoc.dtx and childdoc.ins
% and the derived files childdoc.def and cdocsamp.tex with
% cdocsch1.tex, cdocsch2.tex, cdocsdrf.tex, cdocsfn1.tex, cdocsfn2.tex.
%
%<package>\ifdefined\childdocmain\endinput\fi
%<package>\ProvidesFile{childdoc.def}[2018/12/30 v2.0 child document driver]
%<samplemain>\ProvidesFile{cdocsamp.tex}[2018/12/30 v2.0 sample for childdoc]
%<*driver>
%\ProvidesFile{childdoc.drv}[2018/12/30 v2.0 childdoc reference manual file]
\PassOptionsToClass{10pt,a4paper}{article}
\documentclass{ltxdoc}

\usepackage[margin=35mm]{geometry}
\usepackage{hyperref}
\usepackage{hyperxmp}
\usepackage[usenames]{color}

\hypersetup{colorlinks=true}
\hypersetup{pdfstartview=FitH}
\hypersetup{pdfpagemode=UseNone}
\hypersetup{pdfsource={}}
\hypersetup{pdflang={en-UK}}
\hypersetup{pdfcopyright={Copyright 2017-2018 Niklas Beisert.
  This work may be distributed and/or modified under the
  conditions of the LaTeX Project Public License, either version 1.3
  of this license or (at your option) any later version.}}
\hypersetup{pdflicenseurl={http://www.latex-project.org/lppl.txt}}
\hypersetup{pdfcontactaddress={ETH Zurich, ITP, HIT K,
  Wolfgang-Pauli-Strasse 27}}
\hypersetup{pdfcontactpostcode={8093}}
\hypersetup{pdfcontactcity={Zurich}}
\hypersetup{pdfcontactcountry={Switzerland}}
\hypersetup{pdfcontactemail={nbeisert@itp.phys.ethz.ch}}
\hypersetup{pdfcontacturl={http://people.phys.ethz.ch/\xmptilde nbeisert/}}

\newcommand{\secref}[1]{\hyperref[#1]{section \ref*{#1}}}

\parskip1ex
\parindent0pt
\let\olditemize\itemize
\def\itemize{\olditemize\parskip0pt}

\begin{document}

\title{The \textsf{childdoc} Package}
\hypersetup{pdftitle={The childdoc Package}}
\author{Niklas Beisert\\[2ex]
  Institut f\"ur Theoretische Physik\\
  Eidgen\"ossische Technische Hochschule Z\"urich\\
  Wolfgang-Pauli-Strasse 27, 8093 Z\"urich, Switzerland\\[1ex]
  \href{mailto:nbeisert@itp.phys.ethz.ch}
  {\texttt{nbeisert@itp.phys.ethz.ch}}}
\hypersetup{pdfauthor={Niklas Beisert}}
\hypersetup{pdfsubject={Manual for the LaTeX2e Package childdoc}}
\date{30 December 2018, \textsf{v2.0}}
\maketitle

\begin{abstract}\noindent
\textsf{childdoc} is a \LaTeXe{} package
that enables the direct compilation
of document sections included by |\include|
to individual files.
\end{abstract}

\begingroup
\parskip0ex
\tableofcontents
\endgroup

%%%%%%%%%%%%%%%%%%%%%%%%%%%%%%%%%%%%%%%%%%%%%%%%%%%%%%%%%%%%%%%%%%%%%%%%%%%%%%%%
%%%%%%%%%%%%%%%%%%%%%%%%%%%%%%%%%%%%%%%%%%%%%%%%%%%%%%%%%%%%%%%%%%%%%%%%%%%%%%%%
\section{Introduction}

\LaTeX{} provides a mechanism to structure a large document (such as a book)
into a main file and several child files (containing the chapters)
using the |\include| command.
This mechanism is beneficial for documents
which span hundreds of pages in order to
make the source file(s) more manageable.
Moreover, compilation can be restricted to
selected child files by means of the |\includeonly| command.
The latter feature can be used to reduce the compilation time while editing
(this was significantly more useful in the earlier days of \LaTeX{})
or to generate a smaller document which is easier to navigate.
Another application of |\includeonly| is to generate
documents consisting of selected parts of the complete document.

However, there are a few drawbacks of the plain |\include| mechanism:
\begin{itemize}
\item
The child files cannot be compiled on their own,
they can only be compiled via the main file.
A naive editing environment
(such as a text editor with an option
to have the current file processed by \LaTeX)
may require one to switch to the main file before compiling;
attempting to compile the child file produces errors.
\item
The main file must be modified (each time)
to adjust the |\includeonly| command
to the present needs. This easily leaves the main file in a messy state.
\item
The generated document will always carry the filename
of the main document. This is inconvenient if
several child files are to be compiled and
to be kept for distribution.
\end{itemize}

The present package provides a simple interface
to make child files individually compilable by \LaTeX{}.
Compiling a child file then has the same effect as compiling
the main file with an |\includeonly| command
to select the appropriate child.
Moreover the generated document will carry the name of the child
rather than the main file.
This resolves all three above issues.

This feature is meant to make the editing of books,
thesis documents and lecture notes somewhat more convenient.
However, the package can also be used efficiently for
composing a series of documents (such as exercise sheets)
which are typically distributed individually.
It then assists the author in generating the individual documents
(potentially in different versions)
as well as a document containing the collected series.
Another application is in developing style files
or other kinds of included material
where compilation of the style file could redirect
to a sample or test file.

%%%%%%%%%%%%%%%%%%%%%%%%%%%%%%%%%%%%%%%%%%%%%%%%%%%%%%%%%%%%%%%%%%%%%%%%%%%%%%%%
%%%%%%%%%%%%%%%%%%%%%%%%%%%%%%%%%%%%%%%%%%%%%%%%%%%%%%%%%%%%%%%%%%%%%%%%%%%%%%%%
\section{Usage}

First of all, the package \textsf{childdoc} is \emph{not} a standard
\LaTeXe{} |.sty| style file! Therefore it needs to be invoked in
a non-standard way.

%%%%%%%%%%%%%%%%%%%%%%%%%%%%%%%%%%%%%%%%%%%%%%%%%%%%%%%%%%%%%%%%%%%%%%%%%%%%%%%%
\subsection{Included Files}
\label{sec:include}

%%%%%%%%%%%%%%%%%%%%%%%%%%%%%%%%%%%%%%%%
\DescribeMacro{\childdocmain}
To use the package, add the commands
\begin{center}
\begin{tabular}{l}
|\input{childdoc.def}|\\
|\childdocmain{}|\\
\end{tabular}
\end{center}
at the very top of the main \LaTeX{} file,
in particular \emph{before} the |\documentclass| statement!
The argument of |\childdocmain| should be left empty
(but it must be present).

%%%%%%%%%%%%%%%%%%%%%%%%%%%%%%%%%%%%%%%%
\DescribeMacro{\childdocof}
Furthermore, add the commands
\begin{center}
\begin{tabular}{l}
|\input{childdoc.def}|\\
|\childdocof{|\textit{main}|}|\\
\end{tabular}
\end{center}
at the top of every child file \textit{child}
which is included by |\include{|\textit{child}|}|
from within the main file
(or at least for those files to be compiled individually).
The argument \textit{main} must be the filename of the main file.

There are a couple of
considerations in setting up the main and child documents:

%%%%%%%%%%%%%%%%%%%%%%%%%%%%%%%%%%%%%%%%
\paragraph{Restrictions.}

Please note the following restrictions:
\begin{itemize}
\item
|\childdocmain| must be called with one argument \textit{main}
to ensure compatibility with earlier version of the package.
It must either be empty (|\childdocmain{}|)
or precisely match the filename of the main file in which it is specified.
See \secref{sec:detection} for further information.
\item
The filename \textit{main} must be specified without the |.tex| extension.
\item
The filename \textit{main} is case sensitive
(even in case-insensitive file systems)
due to internal string comparison.
\item
The argument \textit{main} should be fully expanded, it cannot be a macro.
\item
Subdirectories and special characters should be avoided in filenames.
\item
The command |\childdocmain{|\textit{main}|}| must be followed by a whitespace.
It should not be followed immediately by another command
or by a comment mark `|%|'.
This is because the \TeX{} parser reads the token immediately following
the argument of |\childdocmain| and puts it
at the beginning of every child section;
however, a white\-space is ignored.
\end{itemize}

%%%%%%%%%%%%%%%%%%%%%%%%%%%%%%%%%%%%%%%%
\paragraph{Content of Main File.}

It is advisable to place all content in the child files included by |\include|.
Any output contained in the main file will appear in all child documents
unless suppressed manually;
it cannot be suppressed automatically by the |\includeonly| directive
and thus should normally be avoided.
A method to include some content in the main file
by means of conditional processing is described in \secref{sec:conditional}.

%%%%%%%%%%%%%%%%%%%%%%%%%%%%%%%%%%%%%%%%
\paragraph{Page Numbering.}

When only a part of the document is compiled,
the appropriate numbering of pages
(as well as other status parameters)
is determined from the |.aux| files.
The latter contain information from previous passes.
However this information needs to propagate through
all intermediate child documents.
Therefore the page numbering in child documents may well
be inconsistent until the complete document is compiled at least once.

A useful (if unconventional) way to always ensure a consistent
page numbering is to restart the numbering in each child document
and denote the pages by `\textit{child}|.|\textit{page}'
where \textit{child} represents the chapter/section number of the child file.
This can be achieved by the command
|\numberwithin{page}{|\textit{child}|}|
of the \textsf{amsmath} package
where \textit{child} can be |chapter| or |section|
depending on the chosen structuring.
Alternatively, one can modify the macro |\thepage| appropriately
and reset the counter |page| at the start of each child file.

%%%%%%%%%%%%%%%%%%%%%%%%%%%%%%%%%%%%%%%%%%%%%%%%%%%%%%%%%%%%%%%%%%%%%%%%%%%%%%%%
\subsection{Conditional Processing}
\label{sec:conditional}

The package provides a mechanism to compile different versions
of a document. To customise the versions further some conditional processing
can come in handy to distinguish which version is being compiled.
The package provides two macros to describe the compilation context:

%%%%%%%%%%%%%%%%%%%%%%%%%%%%%%%%%%%%%%%%
\DescribeMacro{\ifchilddoc}
The conditional |\ifchilddoc| distinguishes between the compilation of
child documents and the main document:
%
\begin{center}
|\ifchilddoc |\textit{child-code}| |[|\||else |\textit{main-code}]| \||fi|
\end{center}

%%%%%%%%%%%%%%%%%%%%%%%%%%%%%%%%%%%%%%%%
\DescribeMacro{\childdocname}
\DescribeMacro{\childdocjob}
The macro |\childdocname| contains the filename (without extension)
of the main or child file being processed.
Note that |\childdocjob| will always contain the name of the main file.

%%%%%%%%%%%%%%%%%%%%%%%%%%%%%%%%%%%%%%%%
\paragraph{Title Page.}

Conditional processing can be used to include a title or banner page
in the main document when proper precautions are taken.
Importantly, the code in the main file should ensure that the page counter
(as well as other status parameters which are stored in the |.aux| files)
takes the same value after the conditional processing.
Otherwise the page numbers may take divergent values
depending on which part is compiled.

For example, a title page could be declared by:
%
\begin{center}
\begin{tabular}{l}
|\ifchilddoc\||else|\\
|\addtocounter{page}{-1}|\\
\textit{code for title page}\\
|\newpage|\\
|\||fi|
\end{tabular}
\end{center}
%
A banner page for the child documents can be generated by:
%
\begin{center}
\begin{tabular}{l}
|\ifchilddoc|\\
|\addtocounter{page}{-1}|\\
\textit{code for banner page}\\
|\newpage|\\
|\||fi|
\end{tabular}
\end{center}
%
Here one could write a message such as:
\begin{center}
|This is the part \childdocname{} of \childdocjob{}.|
\end{center}

%%%%%%%%%%%%%%%%%%%%%%%%%%%%%%%%%%%%%%%%%%%%%%%%%%%%%%%%%%%%%%%%%%%%%%%%%%%%%%%%
\subsection{Flags}
\label{sec:flags}

The package makes it easy to generate different versions
of the main or child documents.
To this end compilation flags can be defined
and assigned different default values.
They will be particularly useful in conjunction
with the forwarding mechanism described in \secref{sec:forward}.

For example, it may be useful to have a flag |\version|
which can be set to |draft| or |final|.
The document source will contain some conditional code
depending on the value of |\version|.
Suppose further, the flag should default to |final| for the main file
and to |draft| for child files
which is a natural assignment for editing the document.
This is achieved by placing the following code
in the preamble of the main document
(below the |\childdocmain| directive):
%
\begin{center}
\begin{tabular}{l}
|\ifchilddoc|\\
|\providecommand{\version}{draft}|\\
|\||else|\\
|\providecommand{\version}{final}|\\
|\||fi|
\end{tabular}
\end{center}
%
The definition by |\providecommand| makes sure
that previous definitions are not overwritten.
Further statements |\providecommand{\version}{...}|
can thus be added before the above code to override it.

For the main file, one might add a line
(between |\childdocmain| and the above block)
%
\begin{center}
|%\ifchilddoc\||else\providecommand{\version}{draft}\||fi|
\end{center}
%
which can be uncommented to produce a draft version.
Likewise one can add a line to the very top of a child file
(above the |\childdocof{|\textit{main}|}| directive)
%
\begin{center}
|%\providecommand{\version}{final}|
\end{center}
%
which can be uncommented to produce the final version of this child document.

%%%%%%%%%%%%%%%%%%%%%%%%%%%%%%%%%%%%%%%%%%%%%%%%%%%%%%%%%%%%%%%%%%%%%%%%%%%%%%%%
\subsection{Forwarding}
\label{sec:forward}

Different versions of the main or child documents
using compilation flags as described in \secref{sec:flags}
can be (permanently) stored in different files
for convenient compilation, viewing and distribution.
To this end, the package defines a command
to pass on compilation to a different file:

%%%%%%%%%%%%%%%%%%%%%%%%%%%%%%%%%%%%%%%%
\DescribeMacro{\childdocforward}
The command |\childdocforward| redirects processing to
another source file:
%
\begin{center}
\begin{tabular}{l}
|\input{childdoc.def}|\\
|\childdocforward[|\textit{main}|]{|\textit{dest}|}|\\
\end{tabular}
\end{center}
%
The argument \textit{dest} is the destination file
(without extension).
It should be the main file or one of the child files.
Note that further \textsf{childdoc} directives
such as |\childdocof| and |\childdocforward|
in the indicated file will be processed in this form.
The optional argument \textit{main}
passes on directly to the main file \textit{main}
while pretending to compile the child \textit{dest}.
This form behaves as if \textit{dest}
issues |\childdocof{|\textit{main}|}| right away,
and no further \textsf{childdoc} directives will be processed.

%%%%%%%%%%%%%%%%%%%%%%%%%%%%%%%%%%%%%%%%
\DescribeMacro{\...prefix}
In the alternative form |\childdocforwardprefix|,
%
\begin{center}
\begin{tabular}{l}
|\input{childdoc.def}|\\
|\childdocforwardprefix[|\textit{main}|]{|\textit{prefix}|}{|\textit{dest}|}|
\end{tabular}
\end{center}
%
the destination file is determined by a pattern
depending on the current file:
To make this work, the current file must be called
`{\textit{prefix}\hspace{0.2em}\textit{suffix}}'
with \textit{prefix} matching precisely the argument.
Processing is then passed on to the file
`{\textit{dest}\hspace{0.2em}\textit{suffix}}'.
Surely, the same effect is achieved by
directly specifying the
argument `{\textit{dest}\hspace{0.2em}\textit{suffix}}'
in the first form.
However, that requires to set up a different file
for each child. With the alternative form of the command
all these files can have exactly the same content
which simplifies setting them up and maintaining them.

For example, the following file |draft.tex|
with a compilation flag |\version| as described in \secref{sec:flags}
compiles the main document as a draft:
%
\begin{center}
\begin{tabular}{l}
|\def\version{draft}|\\
|\input{childdoc.def}|\\
|\childdocforward{|\textit{main}|}|
\end{tabular}
\end{center}
%
Likewise, the following files |final|\textit{nn}|.tex|
compile the final version of the child document
|child|\textit{nn}|.tex|:
%
\begin{center}
\begin{tabular}{l}
|\def\version{final}|\\
|\input{childdoc.def}|\\
|\childdocforwardprefix{final}{child}|
\end{tabular}
\end{center}
%

Note that when several versions of a main file and/or of each child file
are to be generated, it may be convenient to set up a |Makefile| or
shell script to automatise the process.

%%%%%%%%%%%%%%%%%%%%%%%%%%%%%%%%%%%%%%%%%%%%%%%%%%%%%%%%%%%%%%%%%%%%%%%%%%%%%%%%
\subsection{Command Line Processing}
\label{sec:commandline}

The effect of redirection files can also be achieved by invoking
the \LaTeX{} compiler with a more elaborate command line.
Most conveniently this should be done as part
of a shell script or a |Makefile|.

When using \textsf{childdoc} in the main file, the following
command lines effectively perform a redirection
(note that depending on the shell being used,
backslashes may have to be doubled: `|\|' $\to$ `|\\|'):
%
\begin{center}
|... -jobname "|\textit{target}|" |\\|"|[\textit{flags}]%
|\input{childdoc.def}\childdocforward[|\textit{main}|]{|\textit{dest}|}"|
\end{center}
%
Here \textit{target} is the name of the output file,
\textit{main} is the name of the main file
and \textit{dest} is the name of the main or child file to be processed
(all filenames without extensions).
The optional argument \textit{main} can be omitted
if \textit{main} matches \textit{dest}.
Optionally, compilation \textit{flags} can be defined via |\def| commands.
This command line makes the \TeX{} engine believe
it is compiling the file \textit{target}
whose content is specified as the latter parameter.
The provided code then forwards the processing to
\textit{main} or \textit{dest} as described in \secref{sec:forward}.

%%%%%%%%%%%%%%%%%%%%%%%%%%%%%%%%%%%%%%%%%%%%%%%%%%%%%%%%%%%%%%%%%%%%%%%%%%%%%%%%
\subsection{Include by Input}
\label{sec:input}

Including child documents by |\include| has some restrictions by design.
Most notably, the content of a child document always occupies
its own set of pages; pages cannot be shared between child documents.
Usually, this behaviour makes perfect sense
because each child document contain an essential part of the document.
However, in some situations it may be desirable to compose
a document from a collection of parts
without having mandatory page breaks between then.
For this case, the package
provides a mechanism to include parts
by |\input| which can also be processed individually.
However, by construction this mechanism
requires manual handling of the content to be output.

%%%%%%%%%%%%%%%%%%%%%%%%%%%%%%%%%%%%%%%%
\DescribeMacro{\ifchilddocmanual}
The main file should be prepared as usual, see \secref{sec:include}.
However, the document body must make a distinction
between processing of an individual part and of the main document, e.g.:
%
\begin{center}
\begin{tabular}{l}
|\ifchilddocmanual|\\
|\input{\childdocname}|\\
|\||else|\\
\textit{document body with }|\input{|\textit{part}|}|\\
|\||fi|
\end{tabular}
\end{center}
%
The conditional |\ifchilddocmanual| is true whenever
a part to be included by |\input| is being compiled,
and the name of the part is stored in |\childdocname|.

%%%%%%%%%%%%%%%%%%%%%%%%%%%%%%%%%%%%%%%%
\DescribeMacro{\childdocby}
Each part to be included by |\input| should start with:
%
\begin{center}
\begin{tabular}{l}
|\input{childdoc.def}|\\
|\childdocby{|\textit{main}|}|\\
\end{tabular}
\end{center}
%
The directive |\childdocby| is similar to |\childdocof|
described in \secref{sec:include},
but the subsequent selection of content must be done manually.
To that end, both |\ifchilddoc| and |\ifchilddocmanual|
will be true upon processing of a part,
and the name of the part is stored in |\childdocname|.
Note that |\jobname| will be set to the filename of the current part
so that each part receives an individual |.aux| file
that does not interfere with the |.aux| file(s) of the main document.
This behaviour can be altered by the alternative form
|\childdocby[*]{|\textit{main}|}| (with a non-empty optional argument)
which uses the |.aux| file of the main document
by setting |\jobname| to \textit{main}.

%%%%%%%%%%%%%%%%%%%%%%%%%%%%%%%%%%%%%%%%%%%%%%%%%%%%%%%%%%%%%%%%%%%%%%%%%%%%%%%%
\subsection{Driver Development}
\label{sec:driver}

The \textsf{childdoc} mechanism can also be use for the development
of definition files such as \LaTeX{} styles or classes.
This case differs from the above setup with multiple parts
included by |\include| in that no |\includeonly| should be invoked.
This can be achieved by starting the include file
(before |\ProvidesPackage|) with:
%
\begin{center}
\begin{tabular}{l}
|\input{childdoc.def}|\\
|\childdocforward{|\textit{main}|}|\\
\end{tabular}
\end{center}
%
or alternatively with:
%
\begin{center}
\begin{tabular}{l}
|\input{childdoc.def}|\\
|\childdocby{|\textit{main}|}|\\
\end{tabular}
\end{center}
%
Both forms have slightly different effects as described above.
The main file is prepared as usual, see \secref{sec:include}.

%%%%%%%%%%%%%%%%%%%%%%%%%%%%%%%%%%%%%%%%%%%%%%%%%%%%%%%%%%%%%%%%%%%%%%%%%%%%%%%%
\subsection{Legacy Detection}
\label{sec:detection}

The directive |\childdocmain| in the main file can detect
whether the complete document or merely a child is to be compiled
even without using the directive |\childdocof|.
This method is deprecated because it is less robust
and there is no compelling reason to use it;
it is merely provided for backward compatibility
and it may be removed in future versions.

If the detection mechanism is to be used,
it is mandatory to correctly specify
the filename of the main file as the argument of |\childdocmain|:
%
\begin{center}
\begin{tabular}{l}
|\input{childdoc.def}|\\
|\childdocmain{|\textit{main}|}|\\
\end{tabular}
\end{center}
%
If |\jobname| does not match the argument \textit{main} of |\childdocmain|,
it is assumed that |\jobname| points to the child file to be compiled.
When using |\childdocmain| with the main file specified as argument,
it suffices to start a child file
with just |\input{|\textit{main}|}|
without loading of the package and using |\childdocof|.
If instead all processing is done
with the appropriate \textsf{childdoc} directives,
the argument of \textit{main} of |\childdocmain| can be empty.

An alternative version of the command line processing described
in \secref{sec:commandline} using the detection mechanism reads:
%
\begin{center}
|... -jobname "|\textit{target}|" "|[\textit{flags}]%
[|\def\jobname{|\textit{dest}|}|]|\input{|\textit{main}|}"|
\end{center}

%%%%%%%%%%%%%%%%%%%%%%%%%%%%%%%%%%%%%%%%%%%%%%%%%%%%%%%%%%%%%%%%%%%%%%%%%%%%%%%%
\subsection{Manual Code}
\label{sec:manual}

In case one cannot be certain whether the definitions file |childdoc.def|
is installed on the target \TeX{} distribution
and one prefers not to ship it,
it is conceivable to paste a few relevant commands into the sources.

To that end, drop all statements |\input{childdoc.def}|
and perform the replacements as outlined below.
Instead of |\childdocmain{|\textit{main}|}| add the following code
to the top of the main file:
%
\begin{center}
\begin{tabular}{l}
|\||ifdefined\childdocname\endinput\||fi\newif\ifchilddoc|\\
|\edef\childdocname{\scantokens\expandafter{\jobname\noexpand}}|\\
|\def\childdocmain{|\textit{main}|}\||ifx\childdocmain\childdocname\||else|\\
|\childdoctrue\includeonly{\childdocname}\let\jobname\childdocmain\||fi|\\
\end{tabular}
\end{center}
%
Instead of |\childdocof{|\textit{main}|}| just include the main file
at the top of each child file:
%
\begin{center}
|\input{|\textit{main}|}|
\end{center}
%
A simple redirection |\childdocforward{|\textit{dest}|}| is achieved by:
%
\begin{center}
|\def\jobname{|\textit{dest}|}\input{\jobname}|
\end{center}
%
The redirection with prefix
|\childdocforwardprefix[|\textit{prefix}|]{|\textit{dest}|}|
is accomplished by:
%
\begin{center}
\begin{tabular}{l}
|{\edef\jobname{\scantokens\expandafter{\jobname\noexpand}}|\\
|\def\redirectjob |\textit{prefix}|#1~~~{\gdef\jobname{|\textit{dest}|#1}}|\\
|\expandafter\redirectjob\jobname~~~}\input{\jobname}|
\end{tabular}
\end{center}

In an alternative approach,
child documents can be compiled by a specific command line
without additional code or specific definitions:
%
\begin{center}
|... -jobname "|\textit{target}|" "|[\textit{flags}]%
|\includeonly{|\textit{dest}|}\input{|\textit{main}|}"|
\end{center}
%

%%%%%%%%%%%%%%%%%%%%%%%%%%%%%%%%%%%%%%%%%%%%%%%%%%%%%%%%%%%%%%%%%%%%%%%%%%%%%%%%
%%%%%%%%%%%%%%%%%%%%%%%%%%%%%%%%%%%%%%%%%%%%%%%%%%%%%%%%%%%%%%%%%%%%%%%%%%%%%%%%
\section{Information}

%%%%%%%%%%%%%%%%%%%%%%%%%%%%%%%%%%%%%%%%%%%%%%%%%%%%%%%%%%%%%%%%%%%%%%%%%%%%%%%%
\subsection{Copyright}

Copyright \copyright{} 2017--2018 Niklas Beisert

This work may be distributed and/or modified under the
conditions of the \LaTeX{} Project Public License, either version 1.3
of this license or (at your option) any later version.
The latest version of this license is in
  \url{http://www.latex-project.org/lppl.txt}
and version 1.3 or later is part of all distributions of \LaTeX{}
version 2005/12/01 or later.

This work has the LPPL maintenance status `maintained'.

The Current Maintainer of this work is Niklas Beisert.

This work consists of the files |README.txt|, |childdoc.ins| and |childdoc.dtx|
as well as the derived files |childdoc.def|, |cdocsamp.tex|
with |cdocsch1.tex|, |cdocsch2.tex|, |cdocspt3.tex|, |cdocspt4.tex|,
|cdocsdrf.tex|, |cdocsfn1.tex|, |cdocsfn2.tex|
as well as |childdoc.pdf|.

%%%%%%%%%%%%%%%%%%%%%%%%%%%%%%%%%%%%%%%%%%%%%%%%%%%%%%%%%%%%%%%%%%%%%%%%%%%%%%%%
\subsection{Files and Installation}

The package consists of the files:
%
\begin{center}
\begin{tabular}{ll}
    |README.txt|   & readme file \\
    |childdoc.ins| & installation file \\
    |childdoc.dtx| & source file \\
    |childdoc.def| & definition file \\
    |cdocsamp.tex| & sample main file \\
    |cdocsch1.tex| & sample include file \\
    |cdocsch2.tex| & sample include file \\
    |cdocspt3.tex| & sample part file \\
    |cdocspt4.tex| & sample part file \\
    |cdocsdrf.tex| & sample redirection file \\
    |cdocsfn1.tex| & sample redirection file \\
    |cdocsfn2.tex| & sample redirection file \\
    |childdoc.pdf| & manual
\end{tabular}
\end{center}
%
The distribution consists of the files
|README.txt|, |childdoc.ins| and |childdoc.dtx|.
%
\begin{itemize}
\item
Run (pdf)\LaTeX{} on |childdoc.dtx|
to compile the manual |childdoc.pdf| (this file).
\item
Run \LaTeX{} on |childdoc.ins| to create the definitions file |childdoc.def|
and the sample |cdocsamp.tex| with include files
|cdocsch1.tex|, |cdocsch2.tex|, |cdocspt3.tex|, |cdocspt4.tex|,
|cdocsdrf.tex|, |cdocsfn1.tex|, |cdocsfn2.tex|.
Then copy the file |childdoc.def| to an appropriate directory of your \LaTeX{}
distribution, e.g.\ \textit{texmf-root}|/tex/latex/childdoc|.
\end{itemize}

%%%%%%%%%%%%%%%%%%%%%%%%%%%%%%%%%%%%%%%%%%%%%%%%%%%%%%%%%%%%%%%%%%%%%%%%%%%%%%%%
\subsection{Related CTAN Packages}

There are several other packages which offer a similar functionality:
%
\begin{itemize}
\item
The packages
\href{http://ctan.org/pkg/docmute}{\textsf{docmute}},
\href{http://ctan.org/pkg/includex}{\textsf{includex}} and
\href{http://ctan.org/pkg/standalone}{\textsf{standalone}}
provide commands to include only the document body of
a child file thus allowing both files to be compiled individually.
\item
The packages \href{http://ctan.org/pkg/subdocs}{\textsf{subdocs}}
and \href{http://ctan.org/pkg/subfiles}{\textsf{subfiles}}
provide structures in which the main and child documents can be
encapsulated and allowing them to be compiled individually.
The inclusion mechanism is different from the conventional |\include|.
\item
The package \href{http://ctan.org/pkg/combine}{\textsf{combine}}
is an elaborate solution to combine several documents into one.
\end{itemize}
%
See also the CTAN topic \href{http://ctan.org/topic/subdocs}{\textsf{subdocs}}
for further related packages.
The present package differs from the above solutions in that
a document structure constructed with the conventional |\include| mechanism
just needs two extra commands at the top of every file
such that all constituent files can be compiled individually.

%%%%%%%%%%%%%%%%%%%%%%%%%%%%%%%%%%%%%%%%%%%%%%%%%%%%%%%%%%%%%%%%%%%%%%%%%%%%%%%%
%\subsection{Feature Suggestions}
%
%The following is a list of features which may be useful for future
%versions of this package:
%%
%\begin{itemize}
%\item
%\ldots
%\end{itemize}

%%%%%%%%%%%%%%%%%%%%%%%%%%%%%%%%%%%%%%%%%%%%%%%%%%%%%%%%%%%%%%%%%%%%%%%%%%%%%%%%
\subsection{Revision History}

%%%%%%%%%%%%%%%%%%%%%%%%%%%%%%%%%%%%%%%%
\paragraph{v2.0:} 2018/12/30

\begin{itemize}
\item
immediate forward processing
\item
added |\childdocby| mechanism
\item
manual restructured
\end{itemize}

%%%%%%%%%%%%%%%%%%%%%%%%%%%%%%%%%%%%%%%%
\paragraph{v1.6:} 2018/01/17

\begin{itemize}
\item
application for development of include files
\item
corrections to manual
\end{itemize}

%%%%%%%%%%%%%%%%%%%%%%%%%%%%%%%%%%%%%%%%
\paragraph{v1.5:} 2017/05/21

\begin{itemize}
\item
more complete structuring introduced
\item
|\childdocof| introduced
\item
|\childdoc| renamed to |\childdocmain|
\item
|\childredirect| renamed to |\childdocforward| and |\childdocforwardprefix|
and functionality expanded
\end{itemize}

%%%%%%%%%%%%%%%%%%%%%%%%%%%%%%%%%%%%%%%%
\paragraph{v1.0:} 2017/04/27

\begin{itemize}
\item
manual and install package
\item
first version published on CTAN
\end{itemize}

%%%%%%%%%%%%%%%%%%%%%%%%%%%%%%%%%%%%%%%%
\paragraph{v0.6:} 2017/04/26

\begin{itemize}
\item
redirection mechanism added
\end{itemize}

%%%%%%%%%%%%%%%%%%%%%%%%%%%%%%%%%%%%%%%%
\paragraph{v0.5:} 2017/04/26

\begin{itemize}
\item
functionality in definition file
\end{itemize}


%%%%%%%%%%%%%%%%%%%%%%%%%%%%%%%%%%%%%%%%%%%%%%%%%%%%%%%%%%%%%%%%%%%%%%%%%%%%%%%%
%%%%%%%%%%%%%%%%%%%%%%%%%%%%%%%%%%%%%%%%%%%%%%%%%%%%%%%%%%%%%%%%%%%%%%%%%%%%%%%%
%%%%%%%%%%%%%%%%%%%%%%%%%%%%%%%%%%%%%%%%%%%%%%%%%%%%%%%%%%%%%%%%%%%%%%%%%%%%%%%%
\appendix

\settowidth\MacroIndent{\rmfamily\scriptsize 000\ }

 \DocInput{childdoc.dtx}

\end{document}
%</driver>
% \fi
%
% %%%%%%%%%%%%%%%%%%%%%%%%%%%%%%%%%%%%%%%%%%%%%%%%%%%%%%%%%%%%%%%%%%%%%%%%%%%%%%
% %%%%%%%%%%%%%%%%%%%%%%%%%%%%%%%%%%%%%%%%%%%%%%%%%%%%%%%%%%%%%%%%%%%%%%%%%%%%%%
% \section{Sample}
%\iffalse
%<*samplemain>
%\fi
%
% The following presents a sample document
% with two chapters, two parts, a title page,
% a compile flag as well as three forwarding files to set the flag.
% It consists of eight |.tex| files:
% \begin{center}
% \begin{tabular}{ll}
% |cdocsamp.tex|&main file\\
% |cdocsch1.tex|&include file for chapter 1\\
% |cdocsch2.tex|&include file for chapter 2\\
% |cdocspt3.tex|&include file for part 3\\
% |cdocspt4.tex|&include file for part 4\\
% |cdocsdrf.tex|&forwarding file for main file in draft mode\\
% |cdocsfi1.tex|&forwarding file for final version of chapter 1\\
% |cdocsfi2.tex|&forwarding file for final version of chapter 2\\
% \end{tabular}
% \end{center}
% Each of the eight files can be compiled directly by the \LaTeX{} compiler.
%
% %%%%%%%%%%%%%%%%%%%%%%%%%%%%%%%%%%%%%%
% \paragraph{Main File.}
%
% The main file is called |cdocsamp.tex|.
%
% Load the \textsf{childdoc} definitions and
% declare the filename for the main document:
%    \begin{macrocode}
\input{childdoc.def}
\childdocmain{}
%    \end{macrocode}

% Optional override for |\version| flag:
%    \begin{macrocode}
%%\ifchilddoc\else\providecommand{\version}{draft}\fi
%    \end{macrocode}

% Define the default values for the |\version| flag
% (|final| for the main file and |draft| for childs):
%    \begin{macrocode}
\ifchilddoc
\providecommand{\version}{draft}
\else
\providecommand{\version}{final}
\fi
%    \end{macrocode}

% Load the standard document class:
%    \begin{macrocode}
\documentclass[12pt]{article}
%    \end{macrocode}

% Start the document body:
%    \begin{macrocode}
\begin{document}
%    \end{macrocode}

% Declare a title page.
% Print title, part of document being processed and version flag:
%    \begin{macrocode}
\addtocounter{page}{-1}
\begin{center}
{\LARGE\bfseries{}childdoc example\par}
\vspace{1cm}
\ifchilddoc
\ifchilddocmanual part\else chapter\fi:
`\childdocname' of `\childdocjob'\par
\else
main document: `\childdocjob'\par
\fi
version: \version\par
\end{center}
\newpage
%    \end{macrocode}

% Manually include selected file,
% otherwise process as usual:
%    \begin{macrocode}
\ifchilddocmanual
\section*{part `\childdocname'}
\input{\childdocname}
\else
%    \end{macrocode}

% Include the two chapters:
%    \begin{macrocode}
\include{cdocsch1}
\include{cdocsch2}
%    \end{macrocode}

% Include the two parts unless only chapters should be displayed:
%    \begin{macrocode}
\ifchilddoc\else
\section{part three}
\input{cdocspt3}
\section{part four}
\input{cdocspt4}
\fi
%    \end{macrocode}

% Process as usual until here:
%    \begin{macrocode}
\fi
%    \end{macrocode}

% End of document body:
%    \begin{macrocode}
\end{document}
%    \end{macrocode}
%\iffalse
%</samplemain>
%\fi
%
% %%%%%%%%%%%%%%%%%%%%%%%%%%%%%%%%%%%%%%
% \paragraph{Chapter Include Files.}
%
% The include files are called |cdocsch1.tex| and |cdocsch2.tex|.
%
%\iffalse
%<*samplechap1|samplechap2>
%\fi

% Optional override for |\version| flag:
%    \begin{macrocode}
%%\providecommand{\version}{final}
%    \end{macrocode}

% Include the main document:
%    \begin{macrocode}
\input{childdoc.def}
\childdocof{cdocsamp}
%    \end{macrocode}

%\iffalse
%</samplechap1|samplechap2>
%\fi
%
%\iffalse
%<*samplechap1>
%\fi
% Some text for chapter 1:
%    \begin{macrocode}
\section{one}
some text in chapter one
%    \end{macrocode}

%\iffalse
%</samplechap1>
%\fi
% Some text for chapter 2:
%\iffalse
%<*samplechap2>
%\fi
%    \begin{macrocode}
\section{two}
more text in chapter two
%    \end{macrocode}

%\iffalse
%</samplechap2>
%\fi
%
% %%%%%%%%%%%%%%%%%%%%%%%%%%%%%%%%%%%%%%
% \paragraph{Part Include Files.}
%
% The include files are called |cdocspt3.tex| and |cdocspt4.tex|.
%
%\iffalse
%<*samplepart3|samplepart4>
%\fi

% Optional override for |\version| flag:
%    \begin{macrocode}
%%\providecommand{\version}{final}
%    \end{macrocode}

% Include the main document:
%    \begin{macrocode}
\input{childdoc.def}
\childdocby{cdocsamp}
%    \end{macrocode}

%\iffalse
%</samplepart3|samplepart4>
%\fi
%
%\iffalse
%<*samplepart3>
%\fi
% Some text for part 3:
%    \begin{macrocode}
some text in part three
%    \end{macrocode}

%\iffalse
%</samplepart3>
%\fi
% Some text for part 4:
%\iffalse
%<*samplepart4>
%\fi
%    \begin{macrocode}
more text in part four
%    \end{macrocode}

%\iffalse
%</samplepart4>
%\fi
%
% %%%%%%%%%%%%%%%%%%%%%%%%%%%%%%%%%%%%%%
% \paragraph{Forwarding for a Complete Draft.}
%
% The following forwarding file |cdocsdrf.tex|
% compiles the main document in draft mode:
%\iffalse
%<*sampledraft>
%\fi
%    \begin{macrocode}
\def\version{draft}
\input{childdoc.def}
\childdocforward{cdocsamp}
%    \end{macrocode}

%\iffalse
%</sampledraft>
%\fi
%
% %%%%%%%%%%%%%%%%%%%%%%%%%%%%%%%%%%%%%%
% \paragraph{Forwarding for Final Version of the Chapters.}
%
% The following forwarding files |cdocsfn1.tex| and |cdocsfn2.tex|
% (with identical content)
% compile the final versions of the child documents
% |cdocsch1.tex| and |cdocsch2.tex|, respectively:
%\iffalse
%<*samplefinal>
%\fi
%    \begin{macrocode}
\def\version{final}
\input{childdoc.def}
\childdocforwardprefix[cdocsamp]{cdocsfn}{cdocsch}
%    \end{macrocode}

%\iffalse
%</samplefinal>
%\fi
%
% %%%%%%%%%%%%%%%%%%%%%%%%%%%%%%%%%%%%%%
% \paragraph{Command Line Processing.}
%
% The following three command lines generate the output files
% |cdocscld|, |cdocscl1| and |cdocscl2|
% which should be identical to
% |cdocsdrf|, |cdocsch1| and |cdocsfn2|, respectively:
% \begin{center}
% \begin{tabular}{l}
% |latex -jobname cdocscld \|\\
% |  "\def\version{draft}\input{childdoc.def}\childdocforward{cdocsamp}"|\\
% |latex -jobname cdocscl1 \|\\
% |  "\input{childdoc.def}\childdocforward[cdocsamp]{cdocsch1}"|\\
% |latex -jobname cdocscl2 \|\\
% |  "\def\version{final}\input{childdoc.def}\childdocforward{cdocsch2}"|
% \end{tabular}
% \end{center}
% Note that the trailing backslash on each first line
% merely continues the input to the second line
% (for convenient cut ant paste).
% Furthermore, the command |latex| can be replaced by any
% of its alternative versions such as |pdflatex|.
%
% %%%%%%%%%%%%%%%%%%%%%%%%%%%%%%%%%%%%%%%%%%%%%%%%%%%%%%%%%%%%%%%%%%%%%%%%%%%%%%
% %%%%%%%%%%%%%%%%%%%%%%%%%%%%%%%%%%%%%%%%%%%%%%%%%%%%%%%%%%%%%%%%%%%%%%%%%%%%%%
% \section{Implementation}
%\iffalse
%<*package>
%\fi
%
% This section describes the definitions file |childdoc.def|.

% The definitions cannot be loaded using |\usepackage| or |\RequirePackage|
% which has a mechanism to prevent loading a style file more than once.
% When loading the definitions by means of |\input|
% multiple instances have to be prevented manually:
%\iffalse
%This code needs to be before the `\ProvidesFile' directive
%which is defined at the beginning of this file.
%Therefore it is also placed there and commented out here.
%</package>
%<*discard>
%\fi
%    \begin{macrocode}
\ifdefined\childdocmain\endinput\fi
%    \end{macrocode}
%\iffalse
%</discard>
%<*package>
%\fi
%
% \macro{\ifchilddoc}
% \macro{\ifchilddocmanual}
% The conditional |\ifchilddoc| tells whether a
% child (true) or main (false) document is being compiled.
% The conditional |\ifchilddocmanual| tells whether
% the |\includeonly| mechanism is used (false) or
% the selection of child files must be performed manually (true).
% The definitions initialise to false:
%    \begin{macrocode}
\newif\ifchilddoc
\newif\ifchilddocmanual
%    \end{macrocode}

% \macro{\childdocname}
% \macro{\childdocjob}
% The macro |\childdocname| stores the name of the main document
% to be compiled. The macro |\childdocjob| stores the name of
% the document on which the \LaTeX{} compiler was originally invoked.
% The content of |\jobname| cannot be compared
% to filenames specified in the source due to different catcodes.
% The following code rescans |\jobname|, stores the result
% in |\childdocname| and saves a copy in |\childdocjob|:
%    \begin{macrocode}
\edef\childdocname{\scantokens\expandafter{\jobname\noexpand}}
\let\childdocjob\childdocname
%    \end{macrocode}

% \macro{\childdocdisable}
% The macro |\childdocdisable| prevents the main file
% from being processed more than once.
% At this stage, the main document command |\childdocmain|
% is assumed to be called once again where it should do nothing.
% Any subsequent call to it should prevent
% a secondary processing of the main document
% It overwrites the forwarding commands
% |\childdocof| and |\childdocforward|
% with empty macros to prevent further inclusions of the main document:
%    \begin{macrocode}
\newcommand{\childdocdisable}
{
  \renewcommand{\childdocmain}[1]{\renewcommand{\childdocmain}[1]{\endinput}}
  \renewcommand{\childdocof}[1]{}
  \renewcommand{\childdocby}[2][]{}
  \renewcommand{\childdocforward}[2][]{}
  \renewcommand{\childdocdisable}{}
}
%    \end{macrocode}

% \macro{\childdocmain}
% The macro |\childdocmain| is to be called at the top of the main file
% with nothing or the main filename (without extension) as argument.
% First, it breaks loops.
% If the argument is not empty and does not match |\childdocname|
% (which is set by the first inclusion of |childdoc.def|),
% |\ifchilddoc| is set to true, |\includeonly| is applied to the child file
% and |\jobname| is set to the main file
% (for proper handling of |.aux| files):
%    \begin{macrocode}
\newcommand{\childdocmain}[1]
{
  \childdocdisable\childdocmain{}
  \if?#1?\else
    \begingroup
      \def\childdoctmp{#1}
      \ifx\childdoctmp\childdocname
        \def\childdoctmp{}
      \else
        \def\childdoctmp
        {
          \childdoctrue
          \includeonly{\childdocname}
          \def\childdocjob{#1}
          \def\jobname{#1}
        }
      \fi
      \expandafter
    \endgroup
    \childdoctmp
  \fi
}
%    \end{macrocode}

% \macro{\childdocof}
% The command |\childdocof| redirects
% compilation to the main file |#1|.
%    \begin{macrocode}
\newcommand{\childdocof}[1]
{
  \childdocdisable
  \childdoctrue
  \includeonly{\childdocname}
  \def\jobname{#1}
  \def\childdocjob{#1}
  \input{#1}
}
%    \end{macrocode}

% \macro{\childdocby}
% The command |\childdocby| ....
%    \begin{macrocode}
\newcommand{\childdocby}[2][]
{
  \childdocdisable
  \childdoctrue
  \childdocmanualtrue
  \if?#1?\else
    \def\jobname{#2}
  \fi
  \def\childdocjob{#2}
  \input{#2}
  \endinput
}
%    \end{macrocode}

% \macro{\childdocforward}
% The command |\childdocforward| redirects
% compilation to the main file or
% (if the optional argument is given) a child file.
% Parameters are set as if the main file
% or a child file starting with |\childdocof| was compiled.
% Then compilation is handed over to the main file:
%    \begin{macrocode}
\newcommand{\childdocforward}[2][]
{
  \begingroup
    \if?#1?
      \def\childdoctmp
      {
        \def\childdocname{#2}
        \def\childdocjob{#2}
        \def\jobname{#2}
        \input{#2}
        \endinput
      }
    \else
      \def\childdoctmp
      {
        \childdocdisable
        \def\childdocname{#2}
        \childdoctrue
        \includeonly{#2}
        \def\childdocjob{#1}
        \def\jobname{#1}
        \input{#1}
        \endinput
      }
    \fi
    \expandafter
  \endgroup
  \childdoctmp
}
%    \end{macrocode}

% \macro{\childdocforwardprefix}
% The command |\childdocforwardprefix| redirects
% compilation to the main or a child file by means of a pattern.
% The prefix |#1| in the current filename is replaced by |#2|
% and the suffix of the current filename is kept
% (it is assumed that the filename does not contain the substring `|~~~|'
% which is used as a delimiter).
% Compilation is handed over to the new file by |\childdocforward|:
%    \begin{macrocode}
\newcommand{\childdocforwardprefix}[3][]
{
  \begingroup
    \def\childdocextract #2##1~~~{\def\childdoctmp{\childdocforward[#1]{#3##1}}}
    \expandafter\childdocextract\childdocname~~~
    \expandafter
  \endgroup
  \childdoctmp
}
%    \end{macrocode}

% \macro{\childdoc}
% The deprecated macro |\childdoc| is a legacy version of |\childdocmain|:
%    \begin{macrocode}
\newcommand{\childdoc}{\childdocmain}
%    \end{macrocode}

% \macro{\childdocredirect}
% The deprecated macro |\childdocredirect| is a legacy version
% of |\childdocforward| and |\childdocforwardprefix|:
%    \begin{macrocode}
\newcommand{\childdocredirect}[2][]
{
  \begingroup
    \if?#1?
      \def\childdoctmp{\childdocforward{#2}}
    \else
      \def\childdoctmp{\childdocforwardprefix{#1}{#2}}
    \fi
    \expandafter
  \endgroup
  \childdoctmp
}
%    \end{macrocode}

%\iffalse
%</package>
%\fi
%
\endinput

\childdocforward{cdocsamp}
%    \end{macrocode}

%\iffalse
%</sampledraft>
%\fi
%
% %%%%%%%%%%%%%%%%%%%%%%%%%%%%%%%%%%%%%%
% \paragraph{Forwarding for Final Version of the Chapters.}
%
% The following forwarding files |cdocsfn1.tex| and |cdocsfn2.tex|
% (with identical content)
% compile the final versions of the child documents
% |cdocsch1.tex| and |cdocsch2.tex|, respectively:
%\iffalse
%<*samplefinal>
%\fi
%    \begin{macrocode}
\def\version{final}
% \iffalse
%
% childdoc.dtx Copyright (C) 2017-2018 Niklas Beisert
%
% This work may be distributed and/or modified under the
% conditions of the LaTeX Project Public License, either version 1.3
% of this license or (at your option) any later version.
% The latest version of this license is in
%   http://www.latex-project.org/lppl.txt
% and version 1.3 or later is part of all distributions of LaTeX
% version 2005/12/01 or later.
%
% This work has the LPPL maintenance status `maintained'.
%
% The Current Maintainer of this work is Niklas Beisert.
%
% This work consists of the files childdoc.dtx and childdoc.ins
% and the derived files childdoc.def and cdocsamp.tex with
% cdocsch1.tex, cdocsch2.tex, cdocsdrf.tex, cdocsfn1.tex, cdocsfn2.tex.
%
%<package>\ifdefined\childdocmain\endinput\fi
%<package>\ProvidesFile{childdoc.def}[2018/12/30 v2.0 child document driver]
%<samplemain>\ProvidesFile{cdocsamp.tex}[2018/12/30 v2.0 sample for childdoc]
%<*driver>
%\ProvidesFile{childdoc.drv}[2018/12/30 v2.0 childdoc reference manual file]
\PassOptionsToClass{10pt,a4paper}{article}
\documentclass{ltxdoc}

\usepackage[margin=35mm]{geometry}
\usepackage{hyperref}
\usepackage{hyperxmp}
\usepackage[usenames]{color}

\hypersetup{colorlinks=true}
\hypersetup{pdfstartview=FitH}
\hypersetup{pdfpagemode=UseNone}
\hypersetup{pdfsource={}}
\hypersetup{pdflang={en-UK}}
\hypersetup{pdfcopyright={Copyright 2017-2018 Niklas Beisert.
  This work may be distributed and/or modified under the
  conditions of the LaTeX Project Public License, either version 1.3
  of this license or (at your option) any later version.}}
\hypersetup{pdflicenseurl={http://www.latex-project.org/lppl.txt}}
\hypersetup{pdfcontactaddress={ETH Zurich, ITP, HIT K,
  Wolfgang-Pauli-Strasse 27}}
\hypersetup{pdfcontactpostcode={8093}}
\hypersetup{pdfcontactcity={Zurich}}
\hypersetup{pdfcontactcountry={Switzerland}}
\hypersetup{pdfcontactemail={nbeisert@itp.phys.ethz.ch}}
\hypersetup{pdfcontacturl={http://people.phys.ethz.ch/\xmptilde nbeisert/}}

\newcommand{\secref}[1]{\hyperref[#1]{section \ref*{#1}}}

\parskip1ex
\parindent0pt
\let\olditemize\itemize
\def\itemize{\olditemize\parskip0pt}

\begin{document}

\title{The \textsf{childdoc} Package}
\hypersetup{pdftitle={The childdoc Package}}
\author{Niklas Beisert\\[2ex]
  Institut f\"ur Theoretische Physik\\
  Eidgen\"ossische Technische Hochschule Z\"urich\\
  Wolfgang-Pauli-Strasse 27, 8093 Z\"urich, Switzerland\\[1ex]
  \href{mailto:nbeisert@itp.phys.ethz.ch}
  {\texttt{nbeisert@itp.phys.ethz.ch}}}
\hypersetup{pdfauthor={Niklas Beisert}}
\hypersetup{pdfsubject={Manual for the LaTeX2e Package childdoc}}
\date{30 December 2018, \textsf{v2.0}}
\maketitle

\begin{abstract}\noindent
\textsf{childdoc} is a \LaTeXe{} package
that enables the direct compilation
of document sections included by |\include|
to individual files.
\end{abstract}

\begingroup
\parskip0ex
\tableofcontents
\endgroup

%%%%%%%%%%%%%%%%%%%%%%%%%%%%%%%%%%%%%%%%%%%%%%%%%%%%%%%%%%%%%%%%%%%%%%%%%%%%%%%%
%%%%%%%%%%%%%%%%%%%%%%%%%%%%%%%%%%%%%%%%%%%%%%%%%%%%%%%%%%%%%%%%%%%%%%%%%%%%%%%%
\section{Introduction}

\LaTeX{} provides a mechanism to structure a large document (such as a book)
into a main file and several child files (containing the chapters)
using the |\include| command.
This mechanism is beneficial for documents
which span hundreds of pages in order to
make the source file(s) more manageable.
Moreover, compilation can be restricted to
selected child files by means of the |\includeonly| command.
The latter feature can be used to reduce the compilation time while editing
(this was significantly more useful in the earlier days of \LaTeX{})
or to generate a smaller document which is easier to navigate.
Another application of |\includeonly| is to generate
documents consisting of selected parts of the complete document.

However, there are a few drawbacks of the plain |\include| mechanism:
\begin{itemize}
\item
The child files cannot be compiled on their own,
they can only be compiled via the main file.
A naive editing environment
(such as a text editor with an option
to have the current file processed by \LaTeX)
may require one to switch to the main file before compiling;
attempting to compile the child file produces errors.
\item
The main file must be modified (each time)
to adjust the |\includeonly| command
to the present needs. This easily leaves the main file in a messy state.
\item
The generated document will always carry the filename
of the main document. This is inconvenient if
several child files are to be compiled and
to be kept for distribution.
\end{itemize}

The present package provides a simple interface
to make child files individually compilable by \LaTeX{}.
Compiling a child file then has the same effect as compiling
the main file with an |\includeonly| command
to select the appropriate child.
Moreover the generated document will carry the name of the child
rather than the main file.
This resolves all three above issues.

This feature is meant to make the editing of books,
thesis documents and lecture notes somewhat more convenient.
However, the package can also be used efficiently for
composing a series of documents (such as exercise sheets)
which are typically distributed individually.
It then assists the author in generating the individual documents
(potentially in different versions)
as well as a document containing the collected series.
Another application is in developing style files
or other kinds of included material
where compilation of the style file could redirect
to a sample or test file.

%%%%%%%%%%%%%%%%%%%%%%%%%%%%%%%%%%%%%%%%%%%%%%%%%%%%%%%%%%%%%%%%%%%%%%%%%%%%%%%%
%%%%%%%%%%%%%%%%%%%%%%%%%%%%%%%%%%%%%%%%%%%%%%%%%%%%%%%%%%%%%%%%%%%%%%%%%%%%%%%%
\section{Usage}

First of all, the package \textsf{childdoc} is \emph{not} a standard
\LaTeXe{} |.sty| style file! Therefore it needs to be invoked in
a non-standard way.

%%%%%%%%%%%%%%%%%%%%%%%%%%%%%%%%%%%%%%%%%%%%%%%%%%%%%%%%%%%%%%%%%%%%%%%%%%%%%%%%
\subsection{Included Files}
\label{sec:include}

%%%%%%%%%%%%%%%%%%%%%%%%%%%%%%%%%%%%%%%%
\DescribeMacro{\childdocmain}
To use the package, add the commands
\begin{center}
\begin{tabular}{l}
|\input{childdoc.def}|\\
|\childdocmain{}|\\
\end{tabular}
\end{center}
at the very top of the main \LaTeX{} file,
in particular \emph{before} the |\documentclass| statement!
The argument of |\childdocmain| should be left empty
(but it must be present).

%%%%%%%%%%%%%%%%%%%%%%%%%%%%%%%%%%%%%%%%
\DescribeMacro{\childdocof}
Furthermore, add the commands
\begin{center}
\begin{tabular}{l}
|\input{childdoc.def}|\\
|\childdocof{|\textit{main}|}|\\
\end{tabular}
\end{center}
at the top of every child file \textit{child}
which is included by |\include{|\textit{child}|}|
from within the main file
(or at least for those files to be compiled individually).
The argument \textit{main} must be the filename of the main file.

There are a couple of
considerations in setting up the main and child documents:

%%%%%%%%%%%%%%%%%%%%%%%%%%%%%%%%%%%%%%%%
\paragraph{Restrictions.}

Please note the following restrictions:
\begin{itemize}
\item
|\childdocmain| must be called with one argument \textit{main}
to ensure compatibility with earlier version of the package.
It must either be empty (|\childdocmain{}|)
or precisely match the filename of the main file in which it is specified.
See \secref{sec:detection} for further information.
\item
The filename \textit{main} must be specified without the |.tex| extension.
\item
The filename \textit{main} is case sensitive
(even in case-insensitive file systems)
due to internal string comparison.
\item
The argument \textit{main} should be fully expanded, it cannot be a macro.
\item
Subdirectories and special characters should be avoided in filenames.
\item
The command |\childdocmain{|\textit{main}|}| must be followed by a whitespace.
It should not be followed immediately by another command
or by a comment mark `|%|'.
This is because the \TeX{} parser reads the token immediately following
the argument of |\childdocmain| and puts it
at the beginning of every child section;
however, a white\-space is ignored.
\end{itemize}

%%%%%%%%%%%%%%%%%%%%%%%%%%%%%%%%%%%%%%%%
\paragraph{Content of Main File.}

It is advisable to place all content in the child files included by |\include|.
Any output contained in the main file will appear in all child documents
unless suppressed manually;
it cannot be suppressed automatically by the |\includeonly| directive
and thus should normally be avoided.
A method to include some content in the main file
by means of conditional processing is described in \secref{sec:conditional}.

%%%%%%%%%%%%%%%%%%%%%%%%%%%%%%%%%%%%%%%%
\paragraph{Page Numbering.}

When only a part of the document is compiled,
the appropriate numbering of pages
(as well as other status parameters)
is determined from the |.aux| files.
The latter contain information from previous passes.
However this information needs to propagate through
all intermediate child documents.
Therefore the page numbering in child documents may well
be inconsistent until the complete document is compiled at least once.

A useful (if unconventional) way to always ensure a consistent
page numbering is to restart the numbering in each child document
and denote the pages by `\textit{child}|.|\textit{page}'
where \textit{child} represents the chapter/section number of the child file.
This can be achieved by the command
|\numberwithin{page}{|\textit{child}|}|
of the \textsf{amsmath} package
where \textit{child} can be |chapter| or |section|
depending on the chosen structuring.
Alternatively, one can modify the macro |\thepage| appropriately
and reset the counter |page| at the start of each child file.

%%%%%%%%%%%%%%%%%%%%%%%%%%%%%%%%%%%%%%%%%%%%%%%%%%%%%%%%%%%%%%%%%%%%%%%%%%%%%%%%
\subsection{Conditional Processing}
\label{sec:conditional}

The package provides a mechanism to compile different versions
of a document. To customise the versions further some conditional processing
can come in handy to distinguish which version is being compiled.
The package provides two macros to describe the compilation context:

%%%%%%%%%%%%%%%%%%%%%%%%%%%%%%%%%%%%%%%%
\DescribeMacro{\ifchilddoc}
The conditional |\ifchilddoc| distinguishes between the compilation of
child documents and the main document:
%
\begin{center}
|\ifchilddoc |\textit{child-code}| |[|\||else |\textit{main-code}]| \||fi|
\end{center}

%%%%%%%%%%%%%%%%%%%%%%%%%%%%%%%%%%%%%%%%
\DescribeMacro{\childdocname}
\DescribeMacro{\childdocjob}
The macro |\childdocname| contains the filename (without extension)
of the main or child file being processed.
Note that |\childdocjob| will always contain the name of the main file.

%%%%%%%%%%%%%%%%%%%%%%%%%%%%%%%%%%%%%%%%
\paragraph{Title Page.}

Conditional processing can be used to include a title or banner page
in the main document when proper precautions are taken.
Importantly, the code in the main file should ensure that the page counter
(as well as other status parameters which are stored in the |.aux| files)
takes the same value after the conditional processing.
Otherwise the page numbers may take divergent values
depending on which part is compiled.

For example, a title page could be declared by:
%
\begin{center}
\begin{tabular}{l}
|\ifchilddoc\||else|\\
|\addtocounter{page}{-1}|\\
\textit{code for title page}\\
|\newpage|\\
|\||fi|
\end{tabular}
\end{center}
%
A banner page for the child documents can be generated by:
%
\begin{center}
\begin{tabular}{l}
|\ifchilddoc|\\
|\addtocounter{page}{-1}|\\
\textit{code for banner page}\\
|\newpage|\\
|\||fi|
\end{tabular}
\end{center}
%
Here one could write a message such as:
\begin{center}
|This is the part \childdocname{} of \childdocjob{}.|
\end{center}

%%%%%%%%%%%%%%%%%%%%%%%%%%%%%%%%%%%%%%%%%%%%%%%%%%%%%%%%%%%%%%%%%%%%%%%%%%%%%%%%
\subsection{Flags}
\label{sec:flags}

The package makes it easy to generate different versions
of the main or child documents.
To this end compilation flags can be defined
and assigned different default values.
They will be particularly useful in conjunction
with the forwarding mechanism described in \secref{sec:forward}.

For example, it may be useful to have a flag |\version|
which can be set to |draft| or |final|.
The document source will contain some conditional code
depending on the value of |\version|.
Suppose further, the flag should default to |final| for the main file
and to |draft| for child files
which is a natural assignment for editing the document.
This is achieved by placing the following code
in the preamble of the main document
(below the |\childdocmain| directive):
%
\begin{center}
\begin{tabular}{l}
|\ifchilddoc|\\
|\providecommand{\version}{draft}|\\
|\||else|\\
|\providecommand{\version}{final}|\\
|\||fi|
\end{tabular}
\end{center}
%
The definition by |\providecommand| makes sure
that previous definitions are not overwritten.
Further statements |\providecommand{\version}{...}|
can thus be added before the above code to override it.

For the main file, one might add a line
(between |\childdocmain| and the above block)
%
\begin{center}
|%\ifchilddoc\||else\providecommand{\version}{draft}\||fi|
\end{center}
%
which can be uncommented to produce a draft version.
Likewise one can add a line to the very top of a child file
(above the |\childdocof{|\textit{main}|}| directive)
%
\begin{center}
|%\providecommand{\version}{final}|
\end{center}
%
which can be uncommented to produce the final version of this child document.

%%%%%%%%%%%%%%%%%%%%%%%%%%%%%%%%%%%%%%%%%%%%%%%%%%%%%%%%%%%%%%%%%%%%%%%%%%%%%%%%
\subsection{Forwarding}
\label{sec:forward}

Different versions of the main or child documents
using compilation flags as described in \secref{sec:flags}
can be (permanently) stored in different files
for convenient compilation, viewing and distribution.
To this end, the package defines a command
to pass on compilation to a different file:

%%%%%%%%%%%%%%%%%%%%%%%%%%%%%%%%%%%%%%%%
\DescribeMacro{\childdocforward}
The command |\childdocforward| redirects processing to
another source file:
%
\begin{center}
\begin{tabular}{l}
|\input{childdoc.def}|\\
|\childdocforward[|\textit{main}|]{|\textit{dest}|}|\\
\end{tabular}
\end{center}
%
The argument \textit{dest} is the destination file
(without extension).
It should be the main file or one of the child files.
Note that further \textsf{childdoc} directives
such as |\childdocof| and |\childdocforward|
in the indicated file will be processed in this form.
The optional argument \textit{main}
passes on directly to the main file \textit{main}
while pretending to compile the child \textit{dest}.
This form behaves as if \textit{dest}
issues |\childdocof{|\textit{main}|}| right away,
and no further \textsf{childdoc} directives will be processed.

%%%%%%%%%%%%%%%%%%%%%%%%%%%%%%%%%%%%%%%%
\DescribeMacro{\...prefix}
In the alternative form |\childdocforwardprefix|,
%
\begin{center}
\begin{tabular}{l}
|\input{childdoc.def}|\\
|\childdocforwardprefix[|\textit{main}|]{|\textit{prefix}|}{|\textit{dest}|}|
\end{tabular}
\end{center}
%
the destination file is determined by a pattern
depending on the current file:
To make this work, the current file must be called
`{\textit{prefix}\hspace{0.2em}\textit{suffix}}'
with \textit{prefix} matching precisely the argument.
Processing is then passed on to the file
`{\textit{dest}\hspace{0.2em}\textit{suffix}}'.
Surely, the same effect is achieved by
directly specifying the
argument `{\textit{dest}\hspace{0.2em}\textit{suffix}}'
in the first form.
However, that requires to set up a different file
for each child. With the alternative form of the command
all these files can have exactly the same content
which simplifies setting them up and maintaining them.

For example, the following file |draft.tex|
with a compilation flag |\version| as described in \secref{sec:flags}
compiles the main document as a draft:
%
\begin{center}
\begin{tabular}{l}
|\def\version{draft}|\\
|\input{childdoc.def}|\\
|\childdocforward{|\textit{main}|}|
\end{tabular}
\end{center}
%
Likewise, the following files |final|\textit{nn}|.tex|
compile the final version of the child document
|child|\textit{nn}|.tex|:
%
\begin{center}
\begin{tabular}{l}
|\def\version{final}|\\
|\input{childdoc.def}|\\
|\childdocforwardprefix{final}{child}|
\end{tabular}
\end{center}
%

Note that when several versions of a main file and/or of each child file
are to be generated, it may be convenient to set up a |Makefile| or
shell script to automatise the process.

%%%%%%%%%%%%%%%%%%%%%%%%%%%%%%%%%%%%%%%%%%%%%%%%%%%%%%%%%%%%%%%%%%%%%%%%%%%%%%%%
\subsection{Command Line Processing}
\label{sec:commandline}

The effect of redirection files can also be achieved by invoking
the \LaTeX{} compiler with a more elaborate command line.
Most conveniently this should be done as part
of a shell script or a |Makefile|.

When using \textsf{childdoc} in the main file, the following
command lines effectively perform a redirection
(note that depending on the shell being used,
backslashes may have to be doubled: `|\|' $\to$ `|\\|'):
%
\begin{center}
|... -jobname "|\textit{target}|" |\\|"|[\textit{flags}]%
|\input{childdoc.def}\childdocforward[|\textit{main}|]{|\textit{dest}|}"|
\end{center}
%
Here \textit{target} is the name of the output file,
\textit{main} is the name of the main file
and \textit{dest} is the name of the main or child file to be processed
(all filenames without extensions).
The optional argument \textit{main} can be omitted
if \textit{main} matches \textit{dest}.
Optionally, compilation \textit{flags} can be defined via |\def| commands.
This command line makes the \TeX{} engine believe
it is compiling the file \textit{target}
whose content is specified as the latter parameter.
The provided code then forwards the processing to
\textit{main} or \textit{dest} as described in \secref{sec:forward}.

%%%%%%%%%%%%%%%%%%%%%%%%%%%%%%%%%%%%%%%%%%%%%%%%%%%%%%%%%%%%%%%%%%%%%%%%%%%%%%%%
\subsection{Include by Input}
\label{sec:input}

Including child documents by |\include| has some restrictions by design.
Most notably, the content of a child document always occupies
its own set of pages; pages cannot be shared between child documents.
Usually, this behaviour makes perfect sense
because each child document contain an essential part of the document.
However, in some situations it may be desirable to compose
a document from a collection of parts
without having mandatory page breaks between then.
For this case, the package
provides a mechanism to include parts
by |\input| which can also be processed individually.
However, by construction this mechanism
requires manual handling of the content to be output.

%%%%%%%%%%%%%%%%%%%%%%%%%%%%%%%%%%%%%%%%
\DescribeMacro{\ifchilddocmanual}
The main file should be prepared as usual, see \secref{sec:include}.
However, the document body must make a distinction
between processing of an individual part and of the main document, e.g.:
%
\begin{center}
\begin{tabular}{l}
|\ifchilddocmanual|\\
|\input{\childdocname}|\\
|\||else|\\
\textit{document body with }|\input{|\textit{part}|}|\\
|\||fi|
\end{tabular}
\end{center}
%
The conditional |\ifchilddocmanual| is true whenever
a part to be included by |\input| is being compiled,
and the name of the part is stored in |\childdocname|.

%%%%%%%%%%%%%%%%%%%%%%%%%%%%%%%%%%%%%%%%
\DescribeMacro{\childdocby}
Each part to be included by |\input| should start with:
%
\begin{center}
\begin{tabular}{l}
|\input{childdoc.def}|\\
|\childdocby{|\textit{main}|}|\\
\end{tabular}
\end{center}
%
The directive |\childdocby| is similar to |\childdocof|
described in \secref{sec:include},
but the subsequent selection of content must be done manually.
To that end, both |\ifchilddoc| and |\ifchilddocmanual|
will be true upon processing of a part,
and the name of the part is stored in |\childdocname|.
Note that |\jobname| will be set to the filename of the current part
so that each part receives an individual |.aux| file
that does not interfere with the |.aux| file(s) of the main document.
This behaviour can be altered by the alternative form
|\childdocby[*]{|\textit{main}|}| (with a non-empty optional argument)
which uses the |.aux| file of the main document
by setting |\jobname| to \textit{main}.

%%%%%%%%%%%%%%%%%%%%%%%%%%%%%%%%%%%%%%%%%%%%%%%%%%%%%%%%%%%%%%%%%%%%%%%%%%%%%%%%
\subsection{Driver Development}
\label{sec:driver}

The \textsf{childdoc} mechanism can also be use for the development
of definition files such as \LaTeX{} styles or classes.
This case differs from the above setup with multiple parts
included by |\include| in that no |\includeonly| should be invoked.
This can be achieved by starting the include file
(before |\ProvidesPackage|) with:
%
\begin{center}
\begin{tabular}{l}
|\input{childdoc.def}|\\
|\childdocforward{|\textit{main}|}|\\
\end{tabular}
\end{center}
%
or alternatively with:
%
\begin{center}
\begin{tabular}{l}
|\input{childdoc.def}|\\
|\childdocby{|\textit{main}|}|\\
\end{tabular}
\end{center}
%
Both forms have slightly different effects as described above.
The main file is prepared as usual, see \secref{sec:include}.

%%%%%%%%%%%%%%%%%%%%%%%%%%%%%%%%%%%%%%%%%%%%%%%%%%%%%%%%%%%%%%%%%%%%%%%%%%%%%%%%
\subsection{Legacy Detection}
\label{sec:detection}

The directive |\childdocmain| in the main file can detect
whether the complete document or merely a child is to be compiled
even without using the directive |\childdocof|.
This method is deprecated because it is less robust
and there is no compelling reason to use it;
it is merely provided for backward compatibility
and it may be removed in future versions.

If the detection mechanism is to be used,
it is mandatory to correctly specify
the filename of the main file as the argument of |\childdocmain|:
%
\begin{center}
\begin{tabular}{l}
|\input{childdoc.def}|\\
|\childdocmain{|\textit{main}|}|\\
\end{tabular}
\end{center}
%
If |\jobname| does not match the argument \textit{main} of |\childdocmain|,
it is assumed that |\jobname| points to the child file to be compiled.
When using |\childdocmain| with the main file specified as argument,
it suffices to start a child file
with just |\input{|\textit{main}|}|
without loading of the package and using |\childdocof|.
If instead all processing is done
with the appropriate \textsf{childdoc} directives,
the argument of \textit{main} of |\childdocmain| can be empty.

An alternative version of the command line processing described
in \secref{sec:commandline} using the detection mechanism reads:
%
\begin{center}
|... -jobname "|\textit{target}|" "|[\textit{flags}]%
[|\def\jobname{|\textit{dest}|}|]|\input{|\textit{main}|}"|
\end{center}

%%%%%%%%%%%%%%%%%%%%%%%%%%%%%%%%%%%%%%%%%%%%%%%%%%%%%%%%%%%%%%%%%%%%%%%%%%%%%%%%
\subsection{Manual Code}
\label{sec:manual}

In case one cannot be certain whether the definitions file |childdoc.def|
is installed on the target \TeX{} distribution
and one prefers not to ship it,
it is conceivable to paste a few relevant commands into the sources.

To that end, drop all statements |\input{childdoc.def}|
and perform the replacements as outlined below.
Instead of |\childdocmain{|\textit{main}|}| add the following code
to the top of the main file:
%
\begin{center}
\begin{tabular}{l}
|\||ifdefined\childdocname\endinput\||fi\newif\ifchilddoc|\\
|\edef\childdocname{\scantokens\expandafter{\jobname\noexpand}}|\\
|\def\childdocmain{|\textit{main}|}\||ifx\childdocmain\childdocname\||else|\\
|\childdoctrue\includeonly{\childdocname}\let\jobname\childdocmain\||fi|\\
\end{tabular}
\end{center}
%
Instead of |\childdocof{|\textit{main}|}| just include the main file
at the top of each child file:
%
\begin{center}
|\input{|\textit{main}|}|
\end{center}
%
A simple redirection |\childdocforward{|\textit{dest}|}| is achieved by:
%
\begin{center}
|\def\jobname{|\textit{dest}|}\input{\jobname}|
\end{center}
%
The redirection with prefix
|\childdocforwardprefix[|\textit{prefix}|]{|\textit{dest}|}|
is accomplished by:
%
\begin{center}
\begin{tabular}{l}
|{\edef\jobname{\scantokens\expandafter{\jobname\noexpand}}|\\
|\def\redirectjob |\textit{prefix}|#1~~~{\gdef\jobname{|\textit{dest}|#1}}|\\
|\expandafter\redirectjob\jobname~~~}\input{\jobname}|
\end{tabular}
\end{center}

In an alternative approach,
child documents can be compiled by a specific command line
without additional code or specific definitions:
%
\begin{center}
|... -jobname "|\textit{target}|" "|[\textit{flags}]%
|\includeonly{|\textit{dest}|}\input{|\textit{main}|}"|
\end{center}
%

%%%%%%%%%%%%%%%%%%%%%%%%%%%%%%%%%%%%%%%%%%%%%%%%%%%%%%%%%%%%%%%%%%%%%%%%%%%%%%%%
%%%%%%%%%%%%%%%%%%%%%%%%%%%%%%%%%%%%%%%%%%%%%%%%%%%%%%%%%%%%%%%%%%%%%%%%%%%%%%%%
\section{Information}

%%%%%%%%%%%%%%%%%%%%%%%%%%%%%%%%%%%%%%%%%%%%%%%%%%%%%%%%%%%%%%%%%%%%%%%%%%%%%%%%
\subsection{Copyright}

Copyright \copyright{} 2017--2018 Niklas Beisert

This work may be distributed and/or modified under the
conditions of the \LaTeX{} Project Public License, either version 1.3
of this license or (at your option) any later version.
The latest version of this license is in
  \url{http://www.latex-project.org/lppl.txt}
and version 1.3 or later is part of all distributions of \LaTeX{}
version 2005/12/01 or later.

This work has the LPPL maintenance status `maintained'.

The Current Maintainer of this work is Niklas Beisert.

This work consists of the files |README.txt|, |childdoc.ins| and |childdoc.dtx|
as well as the derived files |childdoc.def|, |cdocsamp.tex|
with |cdocsch1.tex|, |cdocsch2.tex|, |cdocspt3.tex|, |cdocspt4.tex|,
|cdocsdrf.tex|, |cdocsfn1.tex|, |cdocsfn2.tex|
as well as |childdoc.pdf|.

%%%%%%%%%%%%%%%%%%%%%%%%%%%%%%%%%%%%%%%%%%%%%%%%%%%%%%%%%%%%%%%%%%%%%%%%%%%%%%%%
\subsection{Files and Installation}

The package consists of the files:
%
\begin{center}
\begin{tabular}{ll}
    |README.txt|   & readme file \\
    |childdoc.ins| & installation file \\
    |childdoc.dtx| & source file \\
    |childdoc.def| & definition file \\
    |cdocsamp.tex| & sample main file \\
    |cdocsch1.tex| & sample include file \\
    |cdocsch2.tex| & sample include file \\
    |cdocspt3.tex| & sample part file \\
    |cdocspt4.tex| & sample part file \\
    |cdocsdrf.tex| & sample redirection file \\
    |cdocsfn1.tex| & sample redirection file \\
    |cdocsfn2.tex| & sample redirection file \\
    |childdoc.pdf| & manual
\end{tabular}
\end{center}
%
The distribution consists of the files
|README.txt|, |childdoc.ins| and |childdoc.dtx|.
%
\begin{itemize}
\item
Run (pdf)\LaTeX{} on |childdoc.dtx|
to compile the manual |childdoc.pdf| (this file).
\item
Run \LaTeX{} on |childdoc.ins| to create the definitions file |childdoc.def|
and the sample |cdocsamp.tex| with include files
|cdocsch1.tex|, |cdocsch2.tex|, |cdocspt3.tex|, |cdocspt4.tex|,
|cdocsdrf.tex|, |cdocsfn1.tex|, |cdocsfn2.tex|.
Then copy the file |childdoc.def| to an appropriate directory of your \LaTeX{}
distribution, e.g.\ \textit{texmf-root}|/tex/latex/childdoc|.
\end{itemize}

%%%%%%%%%%%%%%%%%%%%%%%%%%%%%%%%%%%%%%%%%%%%%%%%%%%%%%%%%%%%%%%%%%%%%%%%%%%%%%%%
\subsection{Related CTAN Packages}

There are several other packages which offer a similar functionality:
%
\begin{itemize}
\item
The packages
\href{http://ctan.org/pkg/docmute}{\textsf{docmute}},
\href{http://ctan.org/pkg/includex}{\textsf{includex}} and
\href{http://ctan.org/pkg/standalone}{\textsf{standalone}}
provide commands to include only the document body of
a child file thus allowing both files to be compiled individually.
\item
The packages \href{http://ctan.org/pkg/subdocs}{\textsf{subdocs}}
and \href{http://ctan.org/pkg/subfiles}{\textsf{subfiles}}
provide structures in which the main and child documents can be
encapsulated and allowing them to be compiled individually.
The inclusion mechanism is different from the conventional |\include|.
\item
The package \href{http://ctan.org/pkg/combine}{\textsf{combine}}
is an elaborate solution to combine several documents into one.
\end{itemize}
%
See also the CTAN topic \href{http://ctan.org/topic/subdocs}{\textsf{subdocs}}
for further related packages.
The present package differs from the above solutions in that
a document structure constructed with the conventional |\include| mechanism
just needs two extra commands at the top of every file
such that all constituent files can be compiled individually.

%%%%%%%%%%%%%%%%%%%%%%%%%%%%%%%%%%%%%%%%%%%%%%%%%%%%%%%%%%%%%%%%%%%%%%%%%%%%%%%%
%\subsection{Feature Suggestions}
%
%The following is a list of features which may be useful for future
%versions of this package:
%%
%\begin{itemize}
%\item
%\ldots
%\end{itemize}

%%%%%%%%%%%%%%%%%%%%%%%%%%%%%%%%%%%%%%%%%%%%%%%%%%%%%%%%%%%%%%%%%%%%%%%%%%%%%%%%
\subsection{Revision History}

%%%%%%%%%%%%%%%%%%%%%%%%%%%%%%%%%%%%%%%%
\paragraph{v2.0:} 2018/12/30

\begin{itemize}
\item
immediate forward processing
\item
added |\childdocby| mechanism
\item
manual restructured
\end{itemize}

%%%%%%%%%%%%%%%%%%%%%%%%%%%%%%%%%%%%%%%%
\paragraph{v1.6:} 2018/01/17

\begin{itemize}
\item
application for development of include files
\item
corrections to manual
\end{itemize}

%%%%%%%%%%%%%%%%%%%%%%%%%%%%%%%%%%%%%%%%
\paragraph{v1.5:} 2017/05/21

\begin{itemize}
\item
more complete structuring introduced
\item
|\childdocof| introduced
\item
|\childdoc| renamed to |\childdocmain|
\item
|\childredirect| renamed to |\childdocforward| and |\childdocforwardprefix|
and functionality expanded
\end{itemize}

%%%%%%%%%%%%%%%%%%%%%%%%%%%%%%%%%%%%%%%%
\paragraph{v1.0:} 2017/04/27

\begin{itemize}
\item
manual and install package
\item
first version published on CTAN
\end{itemize}

%%%%%%%%%%%%%%%%%%%%%%%%%%%%%%%%%%%%%%%%
\paragraph{v0.6:} 2017/04/26

\begin{itemize}
\item
redirection mechanism added
\end{itemize}

%%%%%%%%%%%%%%%%%%%%%%%%%%%%%%%%%%%%%%%%
\paragraph{v0.5:} 2017/04/26

\begin{itemize}
\item
functionality in definition file
\end{itemize}


%%%%%%%%%%%%%%%%%%%%%%%%%%%%%%%%%%%%%%%%%%%%%%%%%%%%%%%%%%%%%%%%%%%%%%%%%%%%%%%%
%%%%%%%%%%%%%%%%%%%%%%%%%%%%%%%%%%%%%%%%%%%%%%%%%%%%%%%%%%%%%%%%%%%%%%%%%%%%%%%%
%%%%%%%%%%%%%%%%%%%%%%%%%%%%%%%%%%%%%%%%%%%%%%%%%%%%%%%%%%%%%%%%%%%%%%%%%%%%%%%%
\appendix

\settowidth\MacroIndent{\rmfamily\scriptsize 000\ }

 \DocInput{childdoc.dtx}

\end{document}
%</driver>
% \fi
%
% %%%%%%%%%%%%%%%%%%%%%%%%%%%%%%%%%%%%%%%%%%%%%%%%%%%%%%%%%%%%%%%%%%%%%%%%%%%%%%
% %%%%%%%%%%%%%%%%%%%%%%%%%%%%%%%%%%%%%%%%%%%%%%%%%%%%%%%%%%%%%%%%%%%%%%%%%%%%%%
% \section{Sample}
%\iffalse
%<*samplemain>
%\fi
%
% The following presents a sample document
% with two chapters, two parts, a title page,
% a compile flag as well as three forwarding files to set the flag.
% It consists of eight |.tex| files:
% \begin{center}
% \begin{tabular}{ll}
% |cdocsamp.tex|&main file\\
% |cdocsch1.tex|&include file for chapter 1\\
% |cdocsch2.tex|&include file for chapter 2\\
% |cdocspt3.tex|&include file for part 3\\
% |cdocspt4.tex|&include file for part 4\\
% |cdocsdrf.tex|&forwarding file for main file in draft mode\\
% |cdocsfi1.tex|&forwarding file for final version of chapter 1\\
% |cdocsfi2.tex|&forwarding file for final version of chapter 2\\
% \end{tabular}
% \end{center}
% Each of the eight files can be compiled directly by the \LaTeX{} compiler.
%
% %%%%%%%%%%%%%%%%%%%%%%%%%%%%%%%%%%%%%%
% \paragraph{Main File.}
%
% The main file is called |cdocsamp.tex|.
%
% Load the \textsf{childdoc} definitions and
% declare the filename for the main document:
%    \begin{macrocode}
\input{childdoc.def}
\childdocmain{}
%    \end{macrocode}

% Optional override for |\version| flag:
%    \begin{macrocode}
%%\ifchilddoc\else\providecommand{\version}{draft}\fi
%    \end{macrocode}

% Define the default values for the |\version| flag
% (|final| for the main file and |draft| for childs):
%    \begin{macrocode}
\ifchilddoc
\providecommand{\version}{draft}
\else
\providecommand{\version}{final}
\fi
%    \end{macrocode}

% Load the standard document class:
%    \begin{macrocode}
\documentclass[12pt]{article}
%    \end{macrocode}

% Start the document body:
%    \begin{macrocode}
\begin{document}
%    \end{macrocode}

% Declare a title page.
% Print title, part of document being processed and version flag:
%    \begin{macrocode}
\addtocounter{page}{-1}
\begin{center}
{\LARGE\bfseries{}childdoc example\par}
\vspace{1cm}
\ifchilddoc
\ifchilddocmanual part\else chapter\fi:
`\childdocname' of `\childdocjob'\par
\else
main document: `\childdocjob'\par
\fi
version: \version\par
\end{center}
\newpage
%    \end{macrocode}

% Manually include selected file,
% otherwise process as usual:
%    \begin{macrocode}
\ifchilddocmanual
\section*{part `\childdocname'}
\input{\childdocname}
\else
%    \end{macrocode}

% Include the two chapters:
%    \begin{macrocode}
\include{cdocsch1}
\include{cdocsch2}
%    \end{macrocode}

% Include the two parts unless only chapters should be displayed:
%    \begin{macrocode}
\ifchilddoc\else
\section{part three}
\input{cdocspt3}
\section{part four}
\input{cdocspt4}
\fi
%    \end{macrocode}

% Process as usual until here:
%    \begin{macrocode}
\fi
%    \end{macrocode}

% End of document body:
%    \begin{macrocode}
\end{document}
%    \end{macrocode}
%\iffalse
%</samplemain>
%\fi
%
% %%%%%%%%%%%%%%%%%%%%%%%%%%%%%%%%%%%%%%
% \paragraph{Chapter Include Files.}
%
% The include files are called |cdocsch1.tex| and |cdocsch2.tex|.
%
%\iffalse
%<*samplechap1|samplechap2>
%\fi

% Optional override for |\version| flag:
%    \begin{macrocode}
%%\providecommand{\version}{final}
%    \end{macrocode}

% Include the main document:
%    \begin{macrocode}
\input{childdoc.def}
\childdocof{cdocsamp}
%    \end{macrocode}

%\iffalse
%</samplechap1|samplechap2>
%\fi
%
%\iffalse
%<*samplechap1>
%\fi
% Some text for chapter 1:
%    \begin{macrocode}
\section{one}
some text in chapter one
%    \end{macrocode}

%\iffalse
%</samplechap1>
%\fi
% Some text for chapter 2:
%\iffalse
%<*samplechap2>
%\fi
%    \begin{macrocode}
\section{two}
more text in chapter two
%    \end{macrocode}

%\iffalse
%</samplechap2>
%\fi
%
% %%%%%%%%%%%%%%%%%%%%%%%%%%%%%%%%%%%%%%
% \paragraph{Part Include Files.}
%
% The include files are called |cdocspt3.tex| and |cdocspt4.tex|.
%
%\iffalse
%<*samplepart3|samplepart4>
%\fi

% Optional override for |\version| flag:
%    \begin{macrocode}
%%\providecommand{\version}{final}
%    \end{macrocode}

% Include the main document:
%    \begin{macrocode}
\input{childdoc.def}
\childdocby{cdocsamp}
%    \end{macrocode}

%\iffalse
%</samplepart3|samplepart4>
%\fi
%
%\iffalse
%<*samplepart3>
%\fi
% Some text for part 3:
%    \begin{macrocode}
some text in part three
%    \end{macrocode}

%\iffalse
%</samplepart3>
%\fi
% Some text for part 4:
%\iffalse
%<*samplepart4>
%\fi
%    \begin{macrocode}
more text in part four
%    \end{macrocode}

%\iffalse
%</samplepart4>
%\fi
%
% %%%%%%%%%%%%%%%%%%%%%%%%%%%%%%%%%%%%%%
% \paragraph{Forwarding for a Complete Draft.}
%
% The following forwarding file |cdocsdrf.tex|
% compiles the main document in draft mode:
%\iffalse
%<*sampledraft>
%\fi
%    \begin{macrocode}
\def\version{draft}
\input{childdoc.def}
\childdocforward{cdocsamp}
%    \end{macrocode}

%\iffalse
%</sampledraft>
%\fi
%
% %%%%%%%%%%%%%%%%%%%%%%%%%%%%%%%%%%%%%%
% \paragraph{Forwarding for Final Version of the Chapters.}
%
% The following forwarding files |cdocsfn1.tex| and |cdocsfn2.tex|
% (with identical content)
% compile the final versions of the child documents
% |cdocsch1.tex| and |cdocsch2.tex|, respectively:
%\iffalse
%<*samplefinal>
%\fi
%    \begin{macrocode}
\def\version{final}
\input{childdoc.def}
\childdocforwardprefix[cdocsamp]{cdocsfn}{cdocsch}
%    \end{macrocode}

%\iffalse
%</samplefinal>
%\fi
%
% %%%%%%%%%%%%%%%%%%%%%%%%%%%%%%%%%%%%%%
% \paragraph{Command Line Processing.}
%
% The following three command lines generate the output files
% |cdocscld|, |cdocscl1| and |cdocscl2|
% which should be identical to
% |cdocsdrf|, |cdocsch1| and |cdocsfn2|, respectively:
% \begin{center}
% \begin{tabular}{l}
% |latex -jobname cdocscld \|\\
% |  "\def\version{draft}\input{childdoc.def}\childdocforward{cdocsamp}"|\\
% |latex -jobname cdocscl1 \|\\
% |  "\input{childdoc.def}\childdocforward[cdocsamp]{cdocsch1}"|\\
% |latex -jobname cdocscl2 \|\\
% |  "\def\version{final}\input{childdoc.def}\childdocforward{cdocsch2}"|
% \end{tabular}
% \end{center}
% Note that the trailing backslash on each first line
% merely continues the input to the second line
% (for convenient cut ant paste).
% Furthermore, the command |latex| can be replaced by any
% of its alternative versions such as |pdflatex|.
%
% %%%%%%%%%%%%%%%%%%%%%%%%%%%%%%%%%%%%%%%%%%%%%%%%%%%%%%%%%%%%%%%%%%%%%%%%%%%%%%
% %%%%%%%%%%%%%%%%%%%%%%%%%%%%%%%%%%%%%%%%%%%%%%%%%%%%%%%%%%%%%%%%%%%%%%%%%%%%%%
% \section{Implementation}
%\iffalse
%<*package>
%\fi
%
% This section describes the definitions file |childdoc.def|.

% The definitions cannot be loaded using |\usepackage| or |\RequirePackage|
% which has a mechanism to prevent loading a style file more than once.
% When loading the definitions by means of |\input|
% multiple instances have to be prevented manually:
%\iffalse
%This code needs to be before the `\ProvidesFile' directive
%which is defined at the beginning of this file.
%Therefore it is also placed there and commented out here.
%</package>
%<*discard>
%\fi
%    \begin{macrocode}
\ifdefined\childdocmain\endinput\fi
%    \end{macrocode}
%\iffalse
%</discard>
%<*package>
%\fi
%
% \macro{\ifchilddoc}
% \macro{\ifchilddocmanual}
% The conditional |\ifchilddoc| tells whether a
% child (true) or main (false) document is being compiled.
% The conditional |\ifchilddocmanual| tells whether
% the |\includeonly| mechanism is used (false) or
% the selection of child files must be performed manually (true).
% The definitions initialise to false:
%    \begin{macrocode}
\newif\ifchilddoc
\newif\ifchilddocmanual
%    \end{macrocode}

% \macro{\childdocname}
% \macro{\childdocjob}
% The macro |\childdocname| stores the name of the main document
% to be compiled. The macro |\childdocjob| stores the name of
% the document on which the \LaTeX{} compiler was originally invoked.
% The content of |\jobname| cannot be compared
% to filenames specified in the source due to different catcodes.
% The following code rescans |\jobname|, stores the result
% in |\childdocname| and saves a copy in |\childdocjob|:
%    \begin{macrocode}
\edef\childdocname{\scantokens\expandafter{\jobname\noexpand}}
\let\childdocjob\childdocname
%    \end{macrocode}

% \macro{\childdocdisable}
% The macro |\childdocdisable| prevents the main file
% from being processed more than once.
% At this stage, the main document command |\childdocmain|
% is assumed to be called once again where it should do nothing.
% Any subsequent call to it should prevent
% a secondary processing of the main document
% It overwrites the forwarding commands
% |\childdocof| and |\childdocforward|
% with empty macros to prevent further inclusions of the main document:
%    \begin{macrocode}
\newcommand{\childdocdisable}
{
  \renewcommand{\childdocmain}[1]{\renewcommand{\childdocmain}[1]{\endinput}}
  \renewcommand{\childdocof}[1]{}
  \renewcommand{\childdocby}[2][]{}
  \renewcommand{\childdocforward}[2][]{}
  \renewcommand{\childdocdisable}{}
}
%    \end{macrocode}

% \macro{\childdocmain}
% The macro |\childdocmain| is to be called at the top of the main file
% with nothing or the main filename (without extension) as argument.
% First, it breaks loops.
% If the argument is not empty and does not match |\childdocname|
% (which is set by the first inclusion of |childdoc.def|),
% |\ifchilddoc| is set to true, |\includeonly| is applied to the child file
% and |\jobname| is set to the main file
% (for proper handling of |.aux| files):
%    \begin{macrocode}
\newcommand{\childdocmain}[1]
{
  \childdocdisable\childdocmain{}
  \if?#1?\else
    \begingroup
      \def\childdoctmp{#1}
      \ifx\childdoctmp\childdocname
        \def\childdoctmp{}
      \else
        \def\childdoctmp
        {
          \childdoctrue
          \includeonly{\childdocname}
          \def\childdocjob{#1}
          \def\jobname{#1}
        }
      \fi
      \expandafter
    \endgroup
    \childdoctmp
  \fi
}
%    \end{macrocode}

% \macro{\childdocof}
% The command |\childdocof| redirects
% compilation to the main file |#1|.
%    \begin{macrocode}
\newcommand{\childdocof}[1]
{
  \childdocdisable
  \childdoctrue
  \includeonly{\childdocname}
  \def\jobname{#1}
  \def\childdocjob{#1}
  \input{#1}
}
%    \end{macrocode}

% \macro{\childdocby}
% The command |\childdocby| ....
%    \begin{macrocode}
\newcommand{\childdocby}[2][]
{
  \childdocdisable
  \childdoctrue
  \childdocmanualtrue
  \if?#1?\else
    \def\jobname{#2}
  \fi
  \def\childdocjob{#2}
  \input{#2}
  \endinput
}
%    \end{macrocode}

% \macro{\childdocforward}
% The command |\childdocforward| redirects
% compilation to the main file or
% (if the optional argument is given) a child file.
% Parameters are set as if the main file
% or a child file starting with |\childdocof| was compiled.
% Then compilation is handed over to the main file:
%    \begin{macrocode}
\newcommand{\childdocforward}[2][]
{
  \begingroup
    \if?#1?
      \def\childdoctmp
      {
        \def\childdocname{#2}
        \def\childdocjob{#2}
        \def\jobname{#2}
        \input{#2}
        \endinput
      }
    \else
      \def\childdoctmp
      {
        \childdocdisable
        \def\childdocname{#2}
        \childdoctrue
        \includeonly{#2}
        \def\childdocjob{#1}
        \def\jobname{#1}
        \input{#1}
        \endinput
      }
    \fi
    \expandafter
  \endgroup
  \childdoctmp
}
%    \end{macrocode}

% \macro{\childdocforwardprefix}
% The command |\childdocforwardprefix| redirects
% compilation to the main or a child file by means of a pattern.
% The prefix |#1| in the current filename is replaced by |#2|
% and the suffix of the current filename is kept
% (it is assumed that the filename does not contain the substring `|~~~|'
% which is used as a delimiter).
% Compilation is handed over to the new file by |\childdocforward|:
%    \begin{macrocode}
\newcommand{\childdocforwardprefix}[3][]
{
  \begingroup
    \def\childdocextract #2##1~~~{\def\childdoctmp{\childdocforward[#1]{#3##1}}}
    \expandafter\childdocextract\childdocname~~~
    \expandafter
  \endgroup
  \childdoctmp
}
%    \end{macrocode}

% \macro{\childdoc}
% The deprecated macro |\childdoc| is a legacy version of |\childdocmain|:
%    \begin{macrocode}
\newcommand{\childdoc}{\childdocmain}
%    \end{macrocode}

% \macro{\childdocredirect}
% The deprecated macro |\childdocredirect| is a legacy version
% of |\childdocforward| and |\childdocforwardprefix|:
%    \begin{macrocode}
\newcommand{\childdocredirect}[2][]
{
  \begingroup
    \if?#1?
      \def\childdoctmp{\childdocforward{#2}}
    \else
      \def\childdoctmp{\childdocforwardprefix{#1}{#2}}
    \fi
    \expandafter
  \endgroup
  \childdoctmp
}
%    \end{macrocode}

%\iffalse
%</package>
%\fi
%
\endinput

\childdocforwardprefix[cdocsamp]{cdocsfn}{cdocsch}
%    \end{macrocode}

%\iffalse
%</samplefinal>
%\fi
%
% %%%%%%%%%%%%%%%%%%%%%%%%%%%%%%%%%%%%%%
% \paragraph{Command Line Processing.}
%
% The following three command lines generate the output files
% |cdocscld|, |cdocscl1| and |cdocscl2|
% which should be identical to
% |cdocsdrf|, |cdocsch1| and |cdocsfn2|, respectively:
% \begin{center}
% \begin{tabular}{l}
% |latex -jobname cdocscld \|\\
% |  "\def\version{draft}% \iffalse
%
% childdoc.dtx Copyright (C) 2017-2018 Niklas Beisert
%
% This work may be distributed and/or modified under the
% conditions of the LaTeX Project Public License, either version 1.3
% of this license or (at your option) any later version.
% The latest version of this license is in
%   http://www.latex-project.org/lppl.txt
% and version 1.3 or later is part of all distributions of LaTeX
% version 2005/12/01 or later.
%
% This work has the LPPL maintenance status `maintained'.
%
% The Current Maintainer of this work is Niklas Beisert.
%
% This work consists of the files childdoc.dtx and childdoc.ins
% and the derived files childdoc.def and cdocsamp.tex with
% cdocsch1.tex, cdocsch2.tex, cdocsdrf.tex, cdocsfn1.tex, cdocsfn2.tex.
%
%<package>\ifdefined\childdocmain\endinput\fi
%<package>\ProvidesFile{childdoc.def}[2018/12/30 v2.0 child document driver]
%<samplemain>\ProvidesFile{cdocsamp.tex}[2018/12/30 v2.0 sample for childdoc]
%<*driver>
%\ProvidesFile{childdoc.drv}[2018/12/30 v2.0 childdoc reference manual file]
\PassOptionsToClass{10pt,a4paper}{article}
\documentclass{ltxdoc}

\usepackage[margin=35mm]{geometry}
\usepackage{hyperref}
\usepackage{hyperxmp}
\usepackage[usenames]{color}

\hypersetup{colorlinks=true}
\hypersetup{pdfstartview=FitH}
\hypersetup{pdfpagemode=UseNone}
\hypersetup{pdfsource={}}
\hypersetup{pdflang={en-UK}}
\hypersetup{pdfcopyright={Copyright 2017-2018 Niklas Beisert.
  This work may be distributed and/or modified under the
  conditions of the LaTeX Project Public License, either version 1.3
  of this license or (at your option) any later version.}}
\hypersetup{pdflicenseurl={http://www.latex-project.org/lppl.txt}}
\hypersetup{pdfcontactaddress={ETH Zurich, ITP, HIT K,
  Wolfgang-Pauli-Strasse 27}}
\hypersetup{pdfcontactpostcode={8093}}
\hypersetup{pdfcontactcity={Zurich}}
\hypersetup{pdfcontactcountry={Switzerland}}
\hypersetup{pdfcontactemail={nbeisert@itp.phys.ethz.ch}}
\hypersetup{pdfcontacturl={http://people.phys.ethz.ch/\xmptilde nbeisert/}}

\newcommand{\secref}[1]{\hyperref[#1]{section \ref*{#1}}}

\parskip1ex
\parindent0pt
\let\olditemize\itemize
\def\itemize{\olditemize\parskip0pt}

\begin{document}

\title{The \textsf{childdoc} Package}
\hypersetup{pdftitle={The childdoc Package}}
\author{Niklas Beisert\\[2ex]
  Institut f\"ur Theoretische Physik\\
  Eidgen\"ossische Technische Hochschule Z\"urich\\
  Wolfgang-Pauli-Strasse 27, 8093 Z\"urich, Switzerland\\[1ex]
  \href{mailto:nbeisert@itp.phys.ethz.ch}
  {\texttt{nbeisert@itp.phys.ethz.ch}}}
\hypersetup{pdfauthor={Niklas Beisert}}
\hypersetup{pdfsubject={Manual for the LaTeX2e Package childdoc}}
\date{30 December 2018, \textsf{v2.0}}
\maketitle

\begin{abstract}\noindent
\textsf{childdoc} is a \LaTeXe{} package
that enables the direct compilation
of document sections included by |\include|
to individual files.
\end{abstract}

\begingroup
\parskip0ex
\tableofcontents
\endgroup

%%%%%%%%%%%%%%%%%%%%%%%%%%%%%%%%%%%%%%%%%%%%%%%%%%%%%%%%%%%%%%%%%%%%%%%%%%%%%%%%
%%%%%%%%%%%%%%%%%%%%%%%%%%%%%%%%%%%%%%%%%%%%%%%%%%%%%%%%%%%%%%%%%%%%%%%%%%%%%%%%
\section{Introduction}

\LaTeX{} provides a mechanism to structure a large document (such as a book)
into a main file and several child files (containing the chapters)
using the |\include| command.
This mechanism is beneficial for documents
which span hundreds of pages in order to
make the source file(s) more manageable.
Moreover, compilation can be restricted to
selected child files by means of the |\includeonly| command.
The latter feature can be used to reduce the compilation time while editing
(this was significantly more useful in the earlier days of \LaTeX{})
or to generate a smaller document which is easier to navigate.
Another application of |\includeonly| is to generate
documents consisting of selected parts of the complete document.

However, there are a few drawbacks of the plain |\include| mechanism:
\begin{itemize}
\item
The child files cannot be compiled on their own,
they can only be compiled via the main file.
A naive editing environment
(such as a text editor with an option
to have the current file processed by \LaTeX)
may require one to switch to the main file before compiling;
attempting to compile the child file produces errors.
\item
The main file must be modified (each time)
to adjust the |\includeonly| command
to the present needs. This easily leaves the main file in a messy state.
\item
The generated document will always carry the filename
of the main document. This is inconvenient if
several child files are to be compiled and
to be kept for distribution.
\end{itemize}

The present package provides a simple interface
to make child files individually compilable by \LaTeX{}.
Compiling a child file then has the same effect as compiling
the main file with an |\includeonly| command
to select the appropriate child.
Moreover the generated document will carry the name of the child
rather than the main file.
This resolves all three above issues.

This feature is meant to make the editing of books,
thesis documents and lecture notes somewhat more convenient.
However, the package can also be used efficiently for
composing a series of documents (such as exercise sheets)
which are typically distributed individually.
It then assists the author in generating the individual documents
(potentially in different versions)
as well as a document containing the collected series.
Another application is in developing style files
or other kinds of included material
where compilation of the style file could redirect
to a sample or test file.

%%%%%%%%%%%%%%%%%%%%%%%%%%%%%%%%%%%%%%%%%%%%%%%%%%%%%%%%%%%%%%%%%%%%%%%%%%%%%%%%
%%%%%%%%%%%%%%%%%%%%%%%%%%%%%%%%%%%%%%%%%%%%%%%%%%%%%%%%%%%%%%%%%%%%%%%%%%%%%%%%
\section{Usage}

First of all, the package \textsf{childdoc} is \emph{not} a standard
\LaTeXe{} |.sty| style file! Therefore it needs to be invoked in
a non-standard way.

%%%%%%%%%%%%%%%%%%%%%%%%%%%%%%%%%%%%%%%%%%%%%%%%%%%%%%%%%%%%%%%%%%%%%%%%%%%%%%%%
\subsection{Included Files}
\label{sec:include}

%%%%%%%%%%%%%%%%%%%%%%%%%%%%%%%%%%%%%%%%
\DescribeMacro{\childdocmain}
To use the package, add the commands
\begin{center}
\begin{tabular}{l}
|\input{childdoc.def}|\\
|\childdocmain{}|\\
\end{tabular}
\end{center}
at the very top of the main \LaTeX{} file,
in particular \emph{before} the |\documentclass| statement!
The argument of |\childdocmain| should be left empty
(but it must be present).

%%%%%%%%%%%%%%%%%%%%%%%%%%%%%%%%%%%%%%%%
\DescribeMacro{\childdocof}
Furthermore, add the commands
\begin{center}
\begin{tabular}{l}
|\input{childdoc.def}|\\
|\childdocof{|\textit{main}|}|\\
\end{tabular}
\end{center}
at the top of every child file \textit{child}
which is included by |\include{|\textit{child}|}|
from within the main file
(or at least for those files to be compiled individually).
The argument \textit{main} must be the filename of the main file.

There are a couple of
considerations in setting up the main and child documents:

%%%%%%%%%%%%%%%%%%%%%%%%%%%%%%%%%%%%%%%%
\paragraph{Restrictions.}

Please note the following restrictions:
\begin{itemize}
\item
|\childdocmain| must be called with one argument \textit{main}
to ensure compatibility with earlier version of the package.
It must either be empty (|\childdocmain{}|)
or precisely match the filename of the main file in which it is specified.
See \secref{sec:detection} for further information.
\item
The filename \textit{main} must be specified without the |.tex| extension.
\item
The filename \textit{main} is case sensitive
(even in case-insensitive file systems)
due to internal string comparison.
\item
The argument \textit{main} should be fully expanded, it cannot be a macro.
\item
Subdirectories and special characters should be avoided in filenames.
\item
The command |\childdocmain{|\textit{main}|}| must be followed by a whitespace.
It should not be followed immediately by another command
or by a comment mark `|%|'.
This is because the \TeX{} parser reads the token immediately following
the argument of |\childdocmain| and puts it
at the beginning of every child section;
however, a white\-space is ignored.
\end{itemize}

%%%%%%%%%%%%%%%%%%%%%%%%%%%%%%%%%%%%%%%%
\paragraph{Content of Main File.}

It is advisable to place all content in the child files included by |\include|.
Any output contained in the main file will appear in all child documents
unless suppressed manually;
it cannot be suppressed automatically by the |\includeonly| directive
and thus should normally be avoided.
A method to include some content in the main file
by means of conditional processing is described in \secref{sec:conditional}.

%%%%%%%%%%%%%%%%%%%%%%%%%%%%%%%%%%%%%%%%
\paragraph{Page Numbering.}

When only a part of the document is compiled,
the appropriate numbering of pages
(as well as other status parameters)
is determined from the |.aux| files.
The latter contain information from previous passes.
However this information needs to propagate through
all intermediate child documents.
Therefore the page numbering in child documents may well
be inconsistent until the complete document is compiled at least once.

A useful (if unconventional) way to always ensure a consistent
page numbering is to restart the numbering in each child document
and denote the pages by `\textit{child}|.|\textit{page}'
where \textit{child} represents the chapter/section number of the child file.
This can be achieved by the command
|\numberwithin{page}{|\textit{child}|}|
of the \textsf{amsmath} package
where \textit{child} can be |chapter| or |section|
depending on the chosen structuring.
Alternatively, one can modify the macro |\thepage| appropriately
and reset the counter |page| at the start of each child file.

%%%%%%%%%%%%%%%%%%%%%%%%%%%%%%%%%%%%%%%%%%%%%%%%%%%%%%%%%%%%%%%%%%%%%%%%%%%%%%%%
\subsection{Conditional Processing}
\label{sec:conditional}

The package provides a mechanism to compile different versions
of a document. To customise the versions further some conditional processing
can come in handy to distinguish which version is being compiled.
The package provides two macros to describe the compilation context:

%%%%%%%%%%%%%%%%%%%%%%%%%%%%%%%%%%%%%%%%
\DescribeMacro{\ifchilddoc}
The conditional |\ifchilddoc| distinguishes between the compilation of
child documents and the main document:
%
\begin{center}
|\ifchilddoc |\textit{child-code}| |[|\||else |\textit{main-code}]| \||fi|
\end{center}

%%%%%%%%%%%%%%%%%%%%%%%%%%%%%%%%%%%%%%%%
\DescribeMacro{\childdocname}
\DescribeMacro{\childdocjob}
The macro |\childdocname| contains the filename (without extension)
of the main or child file being processed.
Note that |\childdocjob| will always contain the name of the main file.

%%%%%%%%%%%%%%%%%%%%%%%%%%%%%%%%%%%%%%%%
\paragraph{Title Page.}

Conditional processing can be used to include a title or banner page
in the main document when proper precautions are taken.
Importantly, the code in the main file should ensure that the page counter
(as well as other status parameters which are stored in the |.aux| files)
takes the same value after the conditional processing.
Otherwise the page numbers may take divergent values
depending on which part is compiled.

For example, a title page could be declared by:
%
\begin{center}
\begin{tabular}{l}
|\ifchilddoc\||else|\\
|\addtocounter{page}{-1}|\\
\textit{code for title page}\\
|\newpage|\\
|\||fi|
\end{tabular}
\end{center}
%
A banner page for the child documents can be generated by:
%
\begin{center}
\begin{tabular}{l}
|\ifchilddoc|\\
|\addtocounter{page}{-1}|\\
\textit{code for banner page}\\
|\newpage|\\
|\||fi|
\end{tabular}
\end{center}
%
Here one could write a message such as:
\begin{center}
|This is the part \childdocname{} of \childdocjob{}.|
\end{center}

%%%%%%%%%%%%%%%%%%%%%%%%%%%%%%%%%%%%%%%%%%%%%%%%%%%%%%%%%%%%%%%%%%%%%%%%%%%%%%%%
\subsection{Flags}
\label{sec:flags}

The package makes it easy to generate different versions
of the main or child documents.
To this end compilation flags can be defined
and assigned different default values.
They will be particularly useful in conjunction
with the forwarding mechanism described in \secref{sec:forward}.

For example, it may be useful to have a flag |\version|
which can be set to |draft| or |final|.
The document source will contain some conditional code
depending on the value of |\version|.
Suppose further, the flag should default to |final| for the main file
and to |draft| for child files
which is a natural assignment for editing the document.
This is achieved by placing the following code
in the preamble of the main document
(below the |\childdocmain| directive):
%
\begin{center}
\begin{tabular}{l}
|\ifchilddoc|\\
|\providecommand{\version}{draft}|\\
|\||else|\\
|\providecommand{\version}{final}|\\
|\||fi|
\end{tabular}
\end{center}
%
The definition by |\providecommand| makes sure
that previous definitions are not overwritten.
Further statements |\providecommand{\version}{...}|
can thus be added before the above code to override it.

For the main file, one might add a line
(between |\childdocmain| and the above block)
%
\begin{center}
|%\ifchilddoc\||else\providecommand{\version}{draft}\||fi|
\end{center}
%
which can be uncommented to produce a draft version.
Likewise one can add a line to the very top of a child file
(above the |\childdocof{|\textit{main}|}| directive)
%
\begin{center}
|%\providecommand{\version}{final}|
\end{center}
%
which can be uncommented to produce the final version of this child document.

%%%%%%%%%%%%%%%%%%%%%%%%%%%%%%%%%%%%%%%%%%%%%%%%%%%%%%%%%%%%%%%%%%%%%%%%%%%%%%%%
\subsection{Forwarding}
\label{sec:forward}

Different versions of the main or child documents
using compilation flags as described in \secref{sec:flags}
can be (permanently) stored in different files
for convenient compilation, viewing and distribution.
To this end, the package defines a command
to pass on compilation to a different file:

%%%%%%%%%%%%%%%%%%%%%%%%%%%%%%%%%%%%%%%%
\DescribeMacro{\childdocforward}
The command |\childdocforward| redirects processing to
another source file:
%
\begin{center}
\begin{tabular}{l}
|\input{childdoc.def}|\\
|\childdocforward[|\textit{main}|]{|\textit{dest}|}|\\
\end{tabular}
\end{center}
%
The argument \textit{dest} is the destination file
(without extension).
It should be the main file or one of the child files.
Note that further \textsf{childdoc} directives
such as |\childdocof| and |\childdocforward|
in the indicated file will be processed in this form.
The optional argument \textit{main}
passes on directly to the main file \textit{main}
while pretending to compile the child \textit{dest}.
This form behaves as if \textit{dest}
issues |\childdocof{|\textit{main}|}| right away,
and no further \textsf{childdoc} directives will be processed.

%%%%%%%%%%%%%%%%%%%%%%%%%%%%%%%%%%%%%%%%
\DescribeMacro{\...prefix}
In the alternative form |\childdocforwardprefix|,
%
\begin{center}
\begin{tabular}{l}
|\input{childdoc.def}|\\
|\childdocforwardprefix[|\textit{main}|]{|\textit{prefix}|}{|\textit{dest}|}|
\end{tabular}
\end{center}
%
the destination file is determined by a pattern
depending on the current file:
To make this work, the current file must be called
`{\textit{prefix}\hspace{0.2em}\textit{suffix}}'
with \textit{prefix} matching precisely the argument.
Processing is then passed on to the file
`{\textit{dest}\hspace{0.2em}\textit{suffix}}'.
Surely, the same effect is achieved by
directly specifying the
argument `{\textit{dest}\hspace{0.2em}\textit{suffix}}'
in the first form.
However, that requires to set up a different file
for each child. With the alternative form of the command
all these files can have exactly the same content
which simplifies setting them up and maintaining them.

For example, the following file |draft.tex|
with a compilation flag |\version| as described in \secref{sec:flags}
compiles the main document as a draft:
%
\begin{center}
\begin{tabular}{l}
|\def\version{draft}|\\
|\input{childdoc.def}|\\
|\childdocforward{|\textit{main}|}|
\end{tabular}
\end{center}
%
Likewise, the following files |final|\textit{nn}|.tex|
compile the final version of the child document
|child|\textit{nn}|.tex|:
%
\begin{center}
\begin{tabular}{l}
|\def\version{final}|\\
|\input{childdoc.def}|\\
|\childdocforwardprefix{final}{child}|
\end{tabular}
\end{center}
%

Note that when several versions of a main file and/or of each child file
are to be generated, it may be convenient to set up a |Makefile| or
shell script to automatise the process.

%%%%%%%%%%%%%%%%%%%%%%%%%%%%%%%%%%%%%%%%%%%%%%%%%%%%%%%%%%%%%%%%%%%%%%%%%%%%%%%%
\subsection{Command Line Processing}
\label{sec:commandline}

The effect of redirection files can also be achieved by invoking
the \LaTeX{} compiler with a more elaborate command line.
Most conveniently this should be done as part
of a shell script or a |Makefile|.

When using \textsf{childdoc} in the main file, the following
command lines effectively perform a redirection
(note that depending on the shell being used,
backslashes may have to be doubled: `|\|' $\to$ `|\\|'):
%
\begin{center}
|... -jobname "|\textit{target}|" |\\|"|[\textit{flags}]%
|\input{childdoc.def}\childdocforward[|\textit{main}|]{|\textit{dest}|}"|
\end{center}
%
Here \textit{target} is the name of the output file,
\textit{main} is the name of the main file
and \textit{dest} is the name of the main or child file to be processed
(all filenames without extensions).
The optional argument \textit{main} can be omitted
if \textit{main} matches \textit{dest}.
Optionally, compilation \textit{flags} can be defined via |\def| commands.
This command line makes the \TeX{} engine believe
it is compiling the file \textit{target}
whose content is specified as the latter parameter.
The provided code then forwards the processing to
\textit{main} or \textit{dest} as described in \secref{sec:forward}.

%%%%%%%%%%%%%%%%%%%%%%%%%%%%%%%%%%%%%%%%%%%%%%%%%%%%%%%%%%%%%%%%%%%%%%%%%%%%%%%%
\subsection{Include by Input}
\label{sec:input}

Including child documents by |\include| has some restrictions by design.
Most notably, the content of a child document always occupies
its own set of pages; pages cannot be shared between child documents.
Usually, this behaviour makes perfect sense
because each child document contain an essential part of the document.
However, in some situations it may be desirable to compose
a document from a collection of parts
without having mandatory page breaks between then.
For this case, the package
provides a mechanism to include parts
by |\input| which can also be processed individually.
However, by construction this mechanism
requires manual handling of the content to be output.

%%%%%%%%%%%%%%%%%%%%%%%%%%%%%%%%%%%%%%%%
\DescribeMacro{\ifchilddocmanual}
The main file should be prepared as usual, see \secref{sec:include}.
However, the document body must make a distinction
between processing of an individual part and of the main document, e.g.:
%
\begin{center}
\begin{tabular}{l}
|\ifchilddocmanual|\\
|\input{\childdocname}|\\
|\||else|\\
\textit{document body with }|\input{|\textit{part}|}|\\
|\||fi|
\end{tabular}
\end{center}
%
The conditional |\ifchilddocmanual| is true whenever
a part to be included by |\input| is being compiled,
and the name of the part is stored in |\childdocname|.

%%%%%%%%%%%%%%%%%%%%%%%%%%%%%%%%%%%%%%%%
\DescribeMacro{\childdocby}
Each part to be included by |\input| should start with:
%
\begin{center}
\begin{tabular}{l}
|\input{childdoc.def}|\\
|\childdocby{|\textit{main}|}|\\
\end{tabular}
\end{center}
%
The directive |\childdocby| is similar to |\childdocof|
described in \secref{sec:include},
but the subsequent selection of content must be done manually.
To that end, both |\ifchilddoc| and |\ifchilddocmanual|
will be true upon processing of a part,
and the name of the part is stored in |\childdocname|.
Note that |\jobname| will be set to the filename of the current part
so that each part receives an individual |.aux| file
that does not interfere with the |.aux| file(s) of the main document.
This behaviour can be altered by the alternative form
|\childdocby[*]{|\textit{main}|}| (with a non-empty optional argument)
which uses the |.aux| file of the main document
by setting |\jobname| to \textit{main}.

%%%%%%%%%%%%%%%%%%%%%%%%%%%%%%%%%%%%%%%%%%%%%%%%%%%%%%%%%%%%%%%%%%%%%%%%%%%%%%%%
\subsection{Driver Development}
\label{sec:driver}

The \textsf{childdoc} mechanism can also be use for the development
of definition files such as \LaTeX{} styles or classes.
This case differs from the above setup with multiple parts
included by |\include| in that no |\includeonly| should be invoked.
This can be achieved by starting the include file
(before |\ProvidesPackage|) with:
%
\begin{center}
\begin{tabular}{l}
|\input{childdoc.def}|\\
|\childdocforward{|\textit{main}|}|\\
\end{tabular}
\end{center}
%
or alternatively with:
%
\begin{center}
\begin{tabular}{l}
|\input{childdoc.def}|\\
|\childdocby{|\textit{main}|}|\\
\end{tabular}
\end{center}
%
Both forms have slightly different effects as described above.
The main file is prepared as usual, see \secref{sec:include}.

%%%%%%%%%%%%%%%%%%%%%%%%%%%%%%%%%%%%%%%%%%%%%%%%%%%%%%%%%%%%%%%%%%%%%%%%%%%%%%%%
\subsection{Legacy Detection}
\label{sec:detection}

The directive |\childdocmain| in the main file can detect
whether the complete document or merely a child is to be compiled
even without using the directive |\childdocof|.
This method is deprecated because it is less robust
and there is no compelling reason to use it;
it is merely provided for backward compatibility
and it may be removed in future versions.

If the detection mechanism is to be used,
it is mandatory to correctly specify
the filename of the main file as the argument of |\childdocmain|:
%
\begin{center}
\begin{tabular}{l}
|\input{childdoc.def}|\\
|\childdocmain{|\textit{main}|}|\\
\end{tabular}
\end{center}
%
If |\jobname| does not match the argument \textit{main} of |\childdocmain|,
it is assumed that |\jobname| points to the child file to be compiled.
When using |\childdocmain| with the main file specified as argument,
it suffices to start a child file
with just |\input{|\textit{main}|}|
without loading of the package and using |\childdocof|.
If instead all processing is done
with the appropriate \textsf{childdoc} directives,
the argument of \textit{main} of |\childdocmain| can be empty.

An alternative version of the command line processing described
in \secref{sec:commandline} using the detection mechanism reads:
%
\begin{center}
|... -jobname "|\textit{target}|" "|[\textit{flags}]%
[|\def\jobname{|\textit{dest}|}|]|\input{|\textit{main}|}"|
\end{center}

%%%%%%%%%%%%%%%%%%%%%%%%%%%%%%%%%%%%%%%%%%%%%%%%%%%%%%%%%%%%%%%%%%%%%%%%%%%%%%%%
\subsection{Manual Code}
\label{sec:manual}

In case one cannot be certain whether the definitions file |childdoc.def|
is installed on the target \TeX{} distribution
and one prefers not to ship it,
it is conceivable to paste a few relevant commands into the sources.

To that end, drop all statements |\input{childdoc.def}|
and perform the replacements as outlined below.
Instead of |\childdocmain{|\textit{main}|}| add the following code
to the top of the main file:
%
\begin{center}
\begin{tabular}{l}
|\||ifdefined\childdocname\endinput\||fi\newif\ifchilddoc|\\
|\edef\childdocname{\scantokens\expandafter{\jobname\noexpand}}|\\
|\def\childdocmain{|\textit{main}|}\||ifx\childdocmain\childdocname\||else|\\
|\childdoctrue\includeonly{\childdocname}\let\jobname\childdocmain\||fi|\\
\end{tabular}
\end{center}
%
Instead of |\childdocof{|\textit{main}|}| just include the main file
at the top of each child file:
%
\begin{center}
|\input{|\textit{main}|}|
\end{center}
%
A simple redirection |\childdocforward{|\textit{dest}|}| is achieved by:
%
\begin{center}
|\def\jobname{|\textit{dest}|}\input{\jobname}|
\end{center}
%
The redirection with prefix
|\childdocforwardprefix[|\textit{prefix}|]{|\textit{dest}|}|
is accomplished by:
%
\begin{center}
\begin{tabular}{l}
|{\edef\jobname{\scantokens\expandafter{\jobname\noexpand}}|\\
|\def\redirectjob |\textit{prefix}|#1~~~{\gdef\jobname{|\textit{dest}|#1}}|\\
|\expandafter\redirectjob\jobname~~~}\input{\jobname}|
\end{tabular}
\end{center}

In an alternative approach,
child documents can be compiled by a specific command line
without additional code or specific definitions:
%
\begin{center}
|... -jobname "|\textit{target}|" "|[\textit{flags}]%
|\includeonly{|\textit{dest}|}\input{|\textit{main}|}"|
\end{center}
%

%%%%%%%%%%%%%%%%%%%%%%%%%%%%%%%%%%%%%%%%%%%%%%%%%%%%%%%%%%%%%%%%%%%%%%%%%%%%%%%%
%%%%%%%%%%%%%%%%%%%%%%%%%%%%%%%%%%%%%%%%%%%%%%%%%%%%%%%%%%%%%%%%%%%%%%%%%%%%%%%%
\section{Information}

%%%%%%%%%%%%%%%%%%%%%%%%%%%%%%%%%%%%%%%%%%%%%%%%%%%%%%%%%%%%%%%%%%%%%%%%%%%%%%%%
\subsection{Copyright}

Copyright \copyright{} 2017--2018 Niklas Beisert

This work may be distributed and/or modified under the
conditions of the \LaTeX{} Project Public License, either version 1.3
of this license or (at your option) any later version.
The latest version of this license is in
  \url{http://www.latex-project.org/lppl.txt}
and version 1.3 or later is part of all distributions of \LaTeX{}
version 2005/12/01 or later.

This work has the LPPL maintenance status `maintained'.

The Current Maintainer of this work is Niklas Beisert.

This work consists of the files |README.txt|, |childdoc.ins| and |childdoc.dtx|
as well as the derived files |childdoc.def|, |cdocsamp.tex|
with |cdocsch1.tex|, |cdocsch2.tex|, |cdocspt3.tex|, |cdocspt4.tex|,
|cdocsdrf.tex|, |cdocsfn1.tex|, |cdocsfn2.tex|
as well as |childdoc.pdf|.

%%%%%%%%%%%%%%%%%%%%%%%%%%%%%%%%%%%%%%%%%%%%%%%%%%%%%%%%%%%%%%%%%%%%%%%%%%%%%%%%
\subsection{Files and Installation}

The package consists of the files:
%
\begin{center}
\begin{tabular}{ll}
    |README.txt|   & readme file \\
    |childdoc.ins| & installation file \\
    |childdoc.dtx| & source file \\
    |childdoc.def| & definition file \\
    |cdocsamp.tex| & sample main file \\
    |cdocsch1.tex| & sample include file \\
    |cdocsch2.tex| & sample include file \\
    |cdocspt3.tex| & sample part file \\
    |cdocspt4.tex| & sample part file \\
    |cdocsdrf.tex| & sample redirection file \\
    |cdocsfn1.tex| & sample redirection file \\
    |cdocsfn2.tex| & sample redirection file \\
    |childdoc.pdf| & manual
\end{tabular}
\end{center}
%
The distribution consists of the files
|README.txt|, |childdoc.ins| and |childdoc.dtx|.
%
\begin{itemize}
\item
Run (pdf)\LaTeX{} on |childdoc.dtx|
to compile the manual |childdoc.pdf| (this file).
\item
Run \LaTeX{} on |childdoc.ins| to create the definitions file |childdoc.def|
and the sample |cdocsamp.tex| with include files
|cdocsch1.tex|, |cdocsch2.tex|, |cdocspt3.tex|, |cdocspt4.tex|,
|cdocsdrf.tex|, |cdocsfn1.tex|, |cdocsfn2.tex|.
Then copy the file |childdoc.def| to an appropriate directory of your \LaTeX{}
distribution, e.g.\ \textit{texmf-root}|/tex/latex/childdoc|.
\end{itemize}

%%%%%%%%%%%%%%%%%%%%%%%%%%%%%%%%%%%%%%%%%%%%%%%%%%%%%%%%%%%%%%%%%%%%%%%%%%%%%%%%
\subsection{Related CTAN Packages}

There are several other packages which offer a similar functionality:
%
\begin{itemize}
\item
The packages
\href{http://ctan.org/pkg/docmute}{\textsf{docmute}},
\href{http://ctan.org/pkg/includex}{\textsf{includex}} and
\href{http://ctan.org/pkg/standalone}{\textsf{standalone}}
provide commands to include only the document body of
a child file thus allowing both files to be compiled individually.
\item
The packages \href{http://ctan.org/pkg/subdocs}{\textsf{subdocs}}
and \href{http://ctan.org/pkg/subfiles}{\textsf{subfiles}}
provide structures in which the main and child documents can be
encapsulated and allowing them to be compiled individually.
The inclusion mechanism is different from the conventional |\include|.
\item
The package \href{http://ctan.org/pkg/combine}{\textsf{combine}}
is an elaborate solution to combine several documents into one.
\end{itemize}
%
See also the CTAN topic \href{http://ctan.org/topic/subdocs}{\textsf{subdocs}}
for further related packages.
The present package differs from the above solutions in that
a document structure constructed with the conventional |\include| mechanism
just needs two extra commands at the top of every file
such that all constituent files can be compiled individually.

%%%%%%%%%%%%%%%%%%%%%%%%%%%%%%%%%%%%%%%%%%%%%%%%%%%%%%%%%%%%%%%%%%%%%%%%%%%%%%%%
%\subsection{Feature Suggestions}
%
%The following is a list of features which may be useful for future
%versions of this package:
%%
%\begin{itemize}
%\item
%\ldots
%\end{itemize}

%%%%%%%%%%%%%%%%%%%%%%%%%%%%%%%%%%%%%%%%%%%%%%%%%%%%%%%%%%%%%%%%%%%%%%%%%%%%%%%%
\subsection{Revision History}

%%%%%%%%%%%%%%%%%%%%%%%%%%%%%%%%%%%%%%%%
\paragraph{v2.0:} 2018/12/30

\begin{itemize}
\item
immediate forward processing
\item
added |\childdocby| mechanism
\item
manual restructured
\end{itemize}

%%%%%%%%%%%%%%%%%%%%%%%%%%%%%%%%%%%%%%%%
\paragraph{v1.6:} 2018/01/17

\begin{itemize}
\item
application for development of include files
\item
corrections to manual
\end{itemize}

%%%%%%%%%%%%%%%%%%%%%%%%%%%%%%%%%%%%%%%%
\paragraph{v1.5:} 2017/05/21

\begin{itemize}
\item
more complete structuring introduced
\item
|\childdocof| introduced
\item
|\childdoc| renamed to |\childdocmain|
\item
|\childredirect| renamed to |\childdocforward| and |\childdocforwardprefix|
and functionality expanded
\end{itemize}

%%%%%%%%%%%%%%%%%%%%%%%%%%%%%%%%%%%%%%%%
\paragraph{v1.0:} 2017/04/27

\begin{itemize}
\item
manual and install package
\item
first version published on CTAN
\end{itemize}

%%%%%%%%%%%%%%%%%%%%%%%%%%%%%%%%%%%%%%%%
\paragraph{v0.6:} 2017/04/26

\begin{itemize}
\item
redirection mechanism added
\end{itemize}

%%%%%%%%%%%%%%%%%%%%%%%%%%%%%%%%%%%%%%%%
\paragraph{v0.5:} 2017/04/26

\begin{itemize}
\item
functionality in definition file
\end{itemize}


%%%%%%%%%%%%%%%%%%%%%%%%%%%%%%%%%%%%%%%%%%%%%%%%%%%%%%%%%%%%%%%%%%%%%%%%%%%%%%%%
%%%%%%%%%%%%%%%%%%%%%%%%%%%%%%%%%%%%%%%%%%%%%%%%%%%%%%%%%%%%%%%%%%%%%%%%%%%%%%%%
%%%%%%%%%%%%%%%%%%%%%%%%%%%%%%%%%%%%%%%%%%%%%%%%%%%%%%%%%%%%%%%%%%%%%%%%%%%%%%%%
\appendix

\settowidth\MacroIndent{\rmfamily\scriptsize 000\ }

 \DocInput{childdoc.dtx}

\end{document}
%</driver>
% \fi
%
% %%%%%%%%%%%%%%%%%%%%%%%%%%%%%%%%%%%%%%%%%%%%%%%%%%%%%%%%%%%%%%%%%%%%%%%%%%%%%%
% %%%%%%%%%%%%%%%%%%%%%%%%%%%%%%%%%%%%%%%%%%%%%%%%%%%%%%%%%%%%%%%%%%%%%%%%%%%%%%
% \section{Sample}
%\iffalse
%<*samplemain>
%\fi
%
% The following presents a sample document
% with two chapters, two parts, a title page,
% a compile flag as well as three forwarding files to set the flag.
% It consists of eight |.tex| files:
% \begin{center}
% \begin{tabular}{ll}
% |cdocsamp.tex|&main file\\
% |cdocsch1.tex|&include file for chapter 1\\
% |cdocsch2.tex|&include file for chapter 2\\
% |cdocspt3.tex|&include file for part 3\\
% |cdocspt4.tex|&include file for part 4\\
% |cdocsdrf.tex|&forwarding file for main file in draft mode\\
% |cdocsfi1.tex|&forwarding file for final version of chapter 1\\
% |cdocsfi2.tex|&forwarding file for final version of chapter 2\\
% \end{tabular}
% \end{center}
% Each of the eight files can be compiled directly by the \LaTeX{} compiler.
%
% %%%%%%%%%%%%%%%%%%%%%%%%%%%%%%%%%%%%%%
% \paragraph{Main File.}
%
% The main file is called |cdocsamp.tex|.
%
% Load the \textsf{childdoc} definitions and
% declare the filename for the main document:
%    \begin{macrocode}
\input{childdoc.def}
\childdocmain{}
%    \end{macrocode}

% Optional override for |\version| flag:
%    \begin{macrocode}
%%\ifchilddoc\else\providecommand{\version}{draft}\fi
%    \end{macrocode}

% Define the default values for the |\version| flag
% (|final| for the main file and |draft| for childs):
%    \begin{macrocode}
\ifchilddoc
\providecommand{\version}{draft}
\else
\providecommand{\version}{final}
\fi
%    \end{macrocode}

% Load the standard document class:
%    \begin{macrocode}
\documentclass[12pt]{article}
%    \end{macrocode}

% Start the document body:
%    \begin{macrocode}
\begin{document}
%    \end{macrocode}

% Declare a title page.
% Print title, part of document being processed and version flag:
%    \begin{macrocode}
\addtocounter{page}{-1}
\begin{center}
{\LARGE\bfseries{}childdoc example\par}
\vspace{1cm}
\ifchilddoc
\ifchilddocmanual part\else chapter\fi:
`\childdocname' of `\childdocjob'\par
\else
main document: `\childdocjob'\par
\fi
version: \version\par
\end{center}
\newpage
%    \end{macrocode}

% Manually include selected file,
% otherwise process as usual:
%    \begin{macrocode}
\ifchilddocmanual
\section*{part `\childdocname'}
\input{\childdocname}
\else
%    \end{macrocode}

% Include the two chapters:
%    \begin{macrocode}
\include{cdocsch1}
\include{cdocsch2}
%    \end{macrocode}

% Include the two parts unless only chapters should be displayed:
%    \begin{macrocode}
\ifchilddoc\else
\section{part three}
\input{cdocspt3}
\section{part four}
\input{cdocspt4}
\fi
%    \end{macrocode}

% Process as usual until here:
%    \begin{macrocode}
\fi
%    \end{macrocode}

% End of document body:
%    \begin{macrocode}
\end{document}
%    \end{macrocode}
%\iffalse
%</samplemain>
%\fi
%
% %%%%%%%%%%%%%%%%%%%%%%%%%%%%%%%%%%%%%%
% \paragraph{Chapter Include Files.}
%
% The include files are called |cdocsch1.tex| and |cdocsch2.tex|.
%
%\iffalse
%<*samplechap1|samplechap2>
%\fi

% Optional override for |\version| flag:
%    \begin{macrocode}
%%\providecommand{\version}{final}
%    \end{macrocode}

% Include the main document:
%    \begin{macrocode}
\input{childdoc.def}
\childdocof{cdocsamp}
%    \end{macrocode}

%\iffalse
%</samplechap1|samplechap2>
%\fi
%
%\iffalse
%<*samplechap1>
%\fi
% Some text for chapter 1:
%    \begin{macrocode}
\section{one}
some text in chapter one
%    \end{macrocode}

%\iffalse
%</samplechap1>
%\fi
% Some text for chapter 2:
%\iffalse
%<*samplechap2>
%\fi
%    \begin{macrocode}
\section{two}
more text in chapter two
%    \end{macrocode}

%\iffalse
%</samplechap2>
%\fi
%
% %%%%%%%%%%%%%%%%%%%%%%%%%%%%%%%%%%%%%%
% \paragraph{Part Include Files.}
%
% The include files are called |cdocspt3.tex| and |cdocspt4.tex|.
%
%\iffalse
%<*samplepart3|samplepart4>
%\fi

% Optional override for |\version| flag:
%    \begin{macrocode}
%%\providecommand{\version}{final}
%    \end{macrocode}

% Include the main document:
%    \begin{macrocode}
\input{childdoc.def}
\childdocby{cdocsamp}
%    \end{macrocode}

%\iffalse
%</samplepart3|samplepart4>
%\fi
%
%\iffalse
%<*samplepart3>
%\fi
% Some text for part 3:
%    \begin{macrocode}
some text in part three
%    \end{macrocode}

%\iffalse
%</samplepart3>
%\fi
% Some text for part 4:
%\iffalse
%<*samplepart4>
%\fi
%    \begin{macrocode}
more text in part four
%    \end{macrocode}

%\iffalse
%</samplepart4>
%\fi
%
% %%%%%%%%%%%%%%%%%%%%%%%%%%%%%%%%%%%%%%
% \paragraph{Forwarding for a Complete Draft.}
%
% The following forwarding file |cdocsdrf.tex|
% compiles the main document in draft mode:
%\iffalse
%<*sampledraft>
%\fi
%    \begin{macrocode}
\def\version{draft}
\input{childdoc.def}
\childdocforward{cdocsamp}
%    \end{macrocode}

%\iffalse
%</sampledraft>
%\fi
%
% %%%%%%%%%%%%%%%%%%%%%%%%%%%%%%%%%%%%%%
% \paragraph{Forwarding for Final Version of the Chapters.}
%
% The following forwarding files |cdocsfn1.tex| and |cdocsfn2.tex|
% (with identical content)
% compile the final versions of the child documents
% |cdocsch1.tex| and |cdocsch2.tex|, respectively:
%\iffalse
%<*samplefinal>
%\fi
%    \begin{macrocode}
\def\version{final}
\input{childdoc.def}
\childdocforwardprefix[cdocsamp]{cdocsfn}{cdocsch}
%    \end{macrocode}

%\iffalse
%</samplefinal>
%\fi
%
% %%%%%%%%%%%%%%%%%%%%%%%%%%%%%%%%%%%%%%
% \paragraph{Command Line Processing.}
%
% The following three command lines generate the output files
% |cdocscld|, |cdocscl1| and |cdocscl2|
% which should be identical to
% |cdocsdrf|, |cdocsch1| and |cdocsfn2|, respectively:
% \begin{center}
% \begin{tabular}{l}
% |latex -jobname cdocscld \|\\
% |  "\def\version{draft}\input{childdoc.def}\childdocforward{cdocsamp}"|\\
% |latex -jobname cdocscl1 \|\\
% |  "\input{childdoc.def}\childdocforward[cdocsamp]{cdocsch1}"|\\
% |latex -jobname cdocscl2 \|\\
% |  "\def\version{final}\input{childdoc.def}\childdocforward{cdocsch2}"|
% \end{tabular}
% \end{center}
% Note that the trailing backslash on each first line
% merely continues the input to the second line
% (for convenient cut ant paste).
% Furthermore, the command |latex| can be replaced by any
% of its alternative versions such as |pdflatex|.
%
% %%%%%%%%%%%%%%%%%%%%%%%%%%%%%%%%%%%%%%%%%%%%%%%%%%%%%%%%%%%%%%%%%%%%%%%%%%%%%%
% %%%%%%%%%%%%%%%%%%%%%%%%%%%%%%%%%%%%%%%%%%%%%%%%%%%%%%%%%%%%%%%%%%%%%%%%%%%%%%
% \section{Implementation}
%\iffalse
%<*package>
%\fi
%
% This section describes the definitions file |childdoc.def|.

% The definitions cannot be loaded using |\usepackage| or |\RequirePackage|
% which has a mechanism to prevent loading a style file more than once.
% When loading the definitions by means of |\input|
% multiple instances have to be prevented manually:
%\iffalse
%This code needs to be before the `\ProvidesFile' directive
%which is defined at the beginning of this file.
%Therefore it is also placed there and commented out here.
%</package>
%<*discard>
%\fi
%    \begin{macrocode}
\ifdefined\childdocmain\endinput\fi
%    \end{macrocode}
%\iffalse
%</discard>
%<*package>
%\fi
%
% \macro{\ifchilddoc}
% \macro{\ifchilddocmanual}
% The conditional |\ifchilddoc| tells whether a
% child (true) or main (false) document is being compiled.
% The conditional |\ifchilddocmanual| tells whether
% the |\includeonly| mechanism is used (false) or
% the selection of child files must be performed manually (true).
% The definitions initialise to false:
%    \begin{macrocode}
\newif\ifchilddoc
\newif\ifchilddocmanual
%    \end{macrocode}

% \macro{\childdocname}
% \macro{\childdocjob}
% The macro |\childdocname| stores the name of the main document
% to be compiled. The macro |\childdocjob| stores the name of
% the document on which the \LaTeX{} compiler was originally invoked.
% The content of |\jobname| cannot be compared
% to filenames specified in the source due to different catcodes.
% The following code rescans |\jobname|, stores the result
% in |\childdocname| and saves a copy in |\childdocjob|:
%    \begin{macrocode}
\edef\childdocname{\scantokens\expandafter{\jobname\noexpand}}
\let\childdocjob\childdocname
%    \end{macrocode}

% \macro{\childdocdisable}
% The macro |\childdocdisable| prevents the main file
% from being processed more than once.
% At this stage, the main document command |\childdocmain|
% is assumed to be called once again where it should do nothing.
% Any subsequent call to it should prevent
% a secondary processing of the main document
% It overwrites the forwarding commands
% |\childdocof| and |\childdocforward|
% with empty macros to prevent further inclusions of the main document:
%    \begin{macrocode}
\newcommand{\childdocdisable}
{
  \renewcommand{\childdocmain}[1]{\renewcommand{\childdocmain}[1]{\endinput}}
  \renewcommand{\childdocof}[1]{}
  \renewcommand{\childdocby}[2][]{}
  \renewcommand{\childdocforward}[2][]{}
  \renewcommand{\childdocdisable}{}
}
%    \end{macrocode}

% \macro{\childdocmain}
% The macro |\childdocmain| is to be called at the top of the main file
% with nothing or the main filename (without extension) as argument.
% First, it breaks loops.
% If the argument is not empty and does not match |\childdocname|
% (which is set by the first inclusion of |childdoc.def|),
% |\ifchilddoc| is set to true, |\includeonly| is applied to the child file
% and |\jobname| is set to the main file
% (for proper handling of |.aux| files):
%    \begin{macrocode}
\newcommand{\childdocmain}[1]
{
  \childdocdisable\childdocmain{}
  \if?#1?\else
    \begingroup
      \def\childdoctmp{#1}
      \ifx\childdoctmp\childdocname
        \def\childdoctmp{}
      \else
        \def\childdoctmp
        {
          \childdoctrue
          \includeonly{\childdocname}
          \def\childdocjob{#1}
          \def\jobname{#1}
        }
      \fi
      \expandafter
    \endgroup
    \childdoctmp
  \fi
}
%    \end{macrocode}

% \macro{\childdocof}
% The command |\childdocof| redirects
% compilation to the main file |#1|.
%    \begin{macrocode}
\newcommand{\childdocof}[1]
{
  \childdocdisable
  \childdoctrue
  \includeonly{\childdocname}
  \def\jobname{#1}
  \def\childdocjob{#1}
  \input{#1}
}
%    \end{macrocode}

% \macro{\childdocby}
% The command |\childdocby| ....
%    \begin{macrocode}
\newcommand{\childdocby}[2][]
{
  \childdocdisable
  \childdoctrue
  \childdocmanualtrue
  \if?#1?\else
    \def\jobname{#2}
  \fi
  \def\childdocjob{#2}
  \input{#2}
  \endinput
}
%    \end{macrocode}

% \macro{\childdocforward}
% The command |\childdocforward| redirects
% compilation to the main file or
% (if the optional argument is given) a child file.
% Parameters are set as if the main file
% or a child file starting with |\childdocof| was compiled.
% Then compilation is handed over to the main file:
%    \begin{macrocode}
\newcommand{\childdocforward}[2][]
{
  \begingroup
    \if?#1?
      \def\childdoctmp
      {
        \def\childdocname{#2}
        \def\childdocjob{#2}
        \def\jobname{#2}
        \input{#2}
        \endinput
      }
    \else
      \def\childdoctmp
      {
        \childdocdisable
        \def\childdocname{#2}
        \childdoctrue
        \includeonly{#2}
        \def\childdocjob{#1}
        \def\jobname{#1}
        \input{#1}
        \endinput
      }
    \fi
    \expandafter
  \endgroup
  \childdoctmp
}
%    \end{macrocode}

% \macro{\childdocforwardprefix}
% The command |\childdocforwardprefix| redirects
% compilation to the main or a child file by means of a pattern.
% The prefix |#1| in the current filename is replaced by |#2|
% and the suffix of the current filename is kept
% (it is assumed that the filename does not contain the substring `|~~~|'
% which is used as a delimiter).
% Compilation is handed over to the new file by |\childdocforward|:
%    \begin{macrocode}
\newcommand{\childdocforwardprefix}[3][]
{
  \begingroup
    \def\childdocextract #2##1~~~{\def\childdoctmp{\childdocforward[#1]{#3##1}}}
    \expandafter\childdocextract\childdocname~~~
    \expandafter
  \endgroup
  \childdoctmp
}
%    \end{macrocode}

% \macro{\childdoc}
% The deprecated macro |\childdoc| is a legacy version of |\childdocmain|:
%    \begin{macrocode}
\newcommand{\childdoc}{\childdocmain}
%    \end{macrocode}

% \macro{\childdocredirect}
% The deprecated macro |\childdocredirect| is a legacy version
% of |\childdocforward| and |\childdocforwardprefix|:
%    \begin{macrocode}
\newcommand{\childdocredirect}[2][]
{
  \begingroup
    \if?#1?
      \def\childdoctmp{\childdocforward{#2}}
    \else
      \def\childdoctmp{\childdocforwardprefix{#1}{#2}}
    \fi
    \expandafter
  \endgroup
  \childdoctmp
}
%    \end{macrocode}

%\iffalse
%</package>
%\fi
%
\endinput
\childdocforward{cdocsamp}"|\\
% |latex -jobname cdocscl1 \|\\
% |  "% \iffalse
%
% childdoc.dtx Copyright (C) 2017-2018 Niklas Beisert
%
% This work may be distributed and/or modified under the
% conditions of the LaTeX Project Public License, either version 1.3
% of this license or (at your option) any later version.
% The latest version of this license is in
%   http://www.latex-project.org/lppl.txt
% and version 1.3 or later is part of all distributions of LaTeX
% version 2005/12/01 or later.
%
% This work has the LPPL maintenance status `maintained'.
%
% The Current Maintainer of this work is Niklas Beisert.
%
% This work consists of the files childdoc.dtx and childdoc.ins
% and the derived files childdoc.def and cdocsamp.tex with
% cdocsch1.tex, cdocsch2.tex, cdocsdrf.tex, cdocsfn1.tex, cdocsfn2.tex.
%
%<package>\ifdefined\childdocmain\endinput\fi
%<package>\ProvidesFile{childdoc.def}[2018/12/30 v2.0 child document driver]
%<samplemain>\ProvidesFile{cdocsamp.tex}[2018/12/30 v2.0 sample for childdoc]
%<*driver>
%\ProvidesFile{childdoc.drv}[2018/12/30 v2.0 childdoc reference manual file]
\PassOptionsToClass{10pt,a4paper}{article}
\documentclass{ltxdoc}

\usepackage[margin=35mm]{geometry}
\usepackage{hyperref}
\usepackage{hyperxmp}
\usepackage[usenames]{color}

\hypersetup{colorlinks=true}
\hypersetup{pdfstartview=FitH}
\hypersetup{pdfpagemode=UseNone}
\hypersetup{pdfsource={}}
\hypersetup{pdflang={en-UK}}
\hypersetup{pdfcopyright={Copyright 2017-2018 Niklas Beisert.
  This work may be distributed and/or modified under the
  conditions of the LaTeX Project Public License, either version 1.3
  of this license or (at your option) any later version.}}
\hypersetup{pdflicenseurl={http://www.latex-project.org/lppl.txt}}
\hypersetup{pdfcontactaddress={ETH Zurich, ITP, HIT K,
  Wolfgang-Pauli-Strasse 27}}
\hypersetup{pdfcontactpostcode={8093}}
\hypersetup{pdfcontactcity={Zurich}}
\hypersetup{pdfcontactcountry={Switzerland}}
\hypersetup{pdfcontactemail={nbeisert@itp.phys.ethz.ch}}
\hypersetup{pdfcontacturl={http://people.phys.ethz.ch/\xmptilde nbeisert/}}

\newcommand{\secref}[1]{\hyperref[#1]{section \ref*{#1}}}

\parskip1ex
\parindent0pt
\let\olditemize\itemize
\def\itemize{\olditemize\parskip0pt}

\begin{document}

\title{The \textsf{childdoc} Package}
\hypersetup{pdftitle={The childdoc Package}}
\author{Niklas Beisert\\[2ex]
  Institut f\"ur Theoretische Physik\\
  Eidgen\"ossische Technische Hochschule Z\"urich\\
  Wolfgang-Pauli-Strasse 27, 8093 Z\"urich, Switzerland\\[1ex]
  \href{mailto:nbeisert@itp.phys.ethz.ch}
  {\texttt{nbeisert@itp.phys.ethz.ch}}}
\hypersetup{pdfauthor={Niklas Beisert}}
\hypersetup{pdfsubject={Manual for the LaTeX2e Package childdoc}}
\date{30 December 2018, \textsf{v2.0}}
\maketitle

\begin{abstract}\noindent
\textsf{childdoc} is a \LaTeXe{} package
that enables the direct compilation
of document sections included by |\include|
to individual files.
\end{abstract}

\begingroup
\parskip0ex
\tableofcontents
\endgroup

%%%%%%%%%%%%%%%%%%%%%%%%%%%%%%%%%%%%%%%%%%%%%%%%%%%%%%%%%%%%%%%%%%%%%%%%%%%%%%%%
%%%%%%%%%%%%%%%%%%%%%%%%%%%%%%%%%%%%%%%%%%%%%%%%%%%%%%%%%%%%%%%%%%%%%%%%%%%%%%%%
\section{Introduction}

\LaTeX{} provides a mechanism to structure a large document (such as a book)
into a main file and several child files (containing the chapters)
using the |\include| command.
This mechanism is beneficial for documents
which span hundreds of pages in order to
make the source file(s) more manageable.
Moreover, compilation can be restricted to
selected child files by means of the |\includeonly| command.
The latter feature can be used to reduce the compilation time while editing
(this was significantly more useful in the earlier days of \LaTeX{})
or to generate a smaller document which is easier to navigate.
Another application of |\includeonly| is to generate
documents consisting of selected parts of the complete document.

However, there are a few drawbacks of the plain |\include| mechanism:
\begin{itemize}
\item
The child files cannot be compiled on their own,
they can only be compiled via the main file.
A naive editing environment
(such as a text editor with an option
to have the current file processed by \LaTeX)
may require one to switch to the main file before compiling;
attempting to compile the child file produces errors.
\item
The main file must be modified (each time)
to adjust the |\includeonly| command
to the present needs. This easily leaves the main file in a messy state.
\item
The generated document will always carry the filename
of the main document. This is inconvenient if
several child files are to be compiled and
to be kept for distribution.
\end{itemize}

The present package provides a simple interface
to make child files individually compilable by \LaTeX{}.
Compiling a child file then has the same effect as compiling
the main file with an |\includeonly| command
to select the appropriate child.
Moreover the generated document will carry the name of the child
rather than the main file.
This resolves all three above issues.

This feature is meant to make the editing of books,
thesis documents and lecture notes somewhat more convenient.
However, the package can also be used efficiently for
composing a series of documents (such as exercise sheets)
which are typically distributed individually.
It then assists the author in generating the individual documents
(potentially in different versions)
as well as a document containing the collected series.
Another application is in developing style files
or other kinds of included material
where compilation of the style file could redirect
to a sample or test file.

%%%%%%%%%%%%%%%%%%%%%%%%%%%%%%%%%%%%%%%%%%%%%%%%%%%%%%%%%%%%%%%%%%%%%%%%%%%%%%%%
%%%%%%%%%%%%%%%%%%%%%%%%%%%%%%%%%%%%%%%%%%%%%%%%%%%%%%%%%%%%%%%%%%%%%%%%%%%%%%%%
\section{Usage}

First of all, the package \textsf{childdoc} is \emph{not} a standard
\LaTeXe{} |.sty| style file! Therefore it needs to be invoked in
a non-standard way.

%%%%%%%%%%%%%%%%%%%%%%%%%%%%%%%%%%%%%%%%%%%%%%%%%%%%%%%%%%%%%%%%%%%%%%%%%%%%%%%%
\subsection{Included Files}
\label{sec:include}

%%%%%%%%%%%%%%%%%%%%%%%%%%%%%%%%%%%%%%%%
\DescribeMacro{\childdocmain}
To use the package, add the commands
\begin{center}
\begin{tabular}{l}
|\input{childdoc.def}|\\
|\childdocmain{}|\\
\end{tabular}
\end{center}
at the very top of the main \LaTeX{} file,
in particular \emph{before} the |\documentclass| statement!
The argument of |\childdocmain| should be left empty
(but it must be present).

%%%%%%%%%%%%%%%%%%%%%%%%%%%%%%%%%%%%%%%%
\DescribeMacro{\childdocof}
Furthermore, add the commands
\begin{center}
\begin{tabular}{l}
|\input{childdoc.def}|\\
|\childdocof{|\textit{main}|}|\\
\end{tabular}
\end{center}
at the top of every child file \textit{child}
which is included by |\include{|\textit{child}|}|
from within the main file
(or at least for those files to be compiled individually).
The argument \textit{main} must be the filename of the main file.

There are a couple of
considerations in setting up the main and child documents:

%%%%%%%%%%%%%%%%%%%%%%%%%%%%%%%%%%%%%%%%
\paragraph{Restrictions.}

Please note the following restrictions:
\begin{itemize}
\item
|\childdocmain| must be called with one argument \textit{main}
to ensure compatibility with earlier version of the package.
It must either be empty (|\childdocmain{}|)
or precisely match the filename of the main file in which it is specified.
See \secref{sec:detection} for further information.
\item
The filename \textit{main} must be specified without the |.tex| extension.
\item
The filename \textit{main} is case sensitive
(even in case-insensitive file systems)
due to internal string comparison.
\item
The argument \textit{main} should be fully expanded, it cannot be a macro.
\item
Subdirectories and special characters should be avoided in filenames.
\item
The command |\childdocmain{|\textit{main}|}| must be followed by a whitespace.
It should not be followed immediately by another command
or by a comment mark `|%|'.
This is because the \TeX{} parser reads the token immediately following
the argument of |\childdocmain| and puts it
at the beginning of every child section;
however, a white\-space is ignored.
\end{itemize}

%%%%%%%%%%%%%%%%%%%%%%%%%%%%%%%%%%%%%%%%
\paragraph{Content of Main File.}

It is advisable to place all content in the child files included by |\include|.
Any output contained in the main file will appear in all child documents
unless suppressed manually;
it cannot be suppressed automatically by the |\includeonly| directive
and thus should normally be avoided.
A method to include some content in the main file
by means of conditional processing is described in \secref{sec:conditional}.

%%%%%%%%%%%%%%%%%%%%%%%%%%%%%%%%%%%%%%%%
\paragraph{Page Numbering.}

When only a part of the document is compiled,
the appropriate numbering of pages
(as well as other status parameters)
is determined from the |.aux| files.
The latter contain information from previous passes.
However this information needs to propagate through
all intermediate child documents.
Therefore the page numbering in child documents may well
be inconsistent until the complete document is compiled at least once.

A useful (if unconventional) way to always ensure a consistent
page numbering is to restart the numbering in each child document
and denote the pages by `\textit{child}|.|\textit{page}'
where \textit{child} represents the chapter/section number of the child file.
This can be achieved by the command
|\numberwithin{page}{|\textit{child}|}|
of the \textsf{amsmath} package
where \textit{child} can be |chapter| or |section|
depending on the chosen structuring.
Alternatively, one can modify the macro |\thepage| appropriately
and reset the counter |page| at the start of each child file.

%%%%%%%%%%%%%%%%%%%%%%%%%%%%%%%%%%%%%%%%%%%%%%%%%%%%%%%%%%%%%%%%%%%%%%%%%%%%%%%%
\subsection{Conditional Processing}
\label{sec:conditional}

The package provides a mechanism to compile different versions
of a document. To customise the versions further some conditional processing
can come in handy to distinguish which version is being compiled.
The package provides two macros to describe the compilation context:

%%%%%%%%%%%%%%%%%%%%%%%%%%%%%%%%%%%%%%%%
\DescribeMacro{\ifchilddoc}
The conditional |\ifchilddoc| distinguishes between the compilation of
child documents and the main document:
%
\begin{center}
|\ifchilddoc |\textit{child-code}| |[|\||else |\textit{main-code}]| \||fi|
\end{center}

%%%%%%%%%%%%%%%%%%%%%%%%%%%%%%%%%%%%%%%%
\DescribeMacro{\childdocname}
\DescribeMacro{\childdocjob}
The macro |\childdocname| contains the filename (without extension)
of the main or child file being processed.
Note that |\childdocjob| will always contain the name of the main file.

%%%%%%%%%%%%%%%%%%%%%%%%%%%%%%%%%%%%%%%%
\paragraph{Title Page.}

Conditional processing can be used to include a title or banner page
in the main document when proper precautions are taken.
Importantly, the code in the main file should ensure that the page counter
(as well as other status parameters which are stored in the |.aux| files)
takes the same value after the conditional processing.
Otherwise the page numbers may take divergent values
depending on which part is compiled.

For example, a title page could be declared by:
%
\begin{center}
\begin{tabular}{l}
|\ifchilddoc\||else|\\
|\addtocounter{page}{-1}|\\
\textit{code for title page}\\
|\newpage|\\
|\||fi|
\end{tabular}
\end{center}
%
A banner page for the child documents can be generated by:
%
\begin{center}
\begin{tabular}{l}
|\ifchilddoc|\\
|\addtocounter{page}{-1}|\\
\textit{code for banner page}\\
|\newpage|\\
|\||fi|
\end{tabular}
\end{center}
%
Here one could write a message such as:
\begin{center}
|This is the part \childdocname{} of \childdocjob{}.|
\end{center}

%%%%%%%%%%%%%%%%%%%%%%%%%%%%%%%%%%%%%%%%%%%%%%%%%%%%%%%%%%%%%%%%%%%%%%%%%%%%%%%%
\subsection{Flags}
\label{sec:flags}

The package makes it easy to generate different versions
of the main or child documents.
To this end compilation flags can be defined
and assigned different default values.
They will be particularly useful in conjunction
with the forwarding mechanism described in \secref{sec:forward}.

For example, it may be useful to have a flag |\version|
which can be set to |draft| or |final|.
The document source will contain some conditional code
depending on the value of |\version|.
Suppose further, the flag should default to |final| for the main file
and to |draft| for child files
which is a natural assignment for editing the document.
This is achieved by placing the following code
in the preamble of the main document
(below the |\childdocmain| directive):
%
\begin{center}
\begin{tabular}{l}
|\ifchilddoc|\\
|\providecommand{\version}{draft}|\\
|\||else|\\
|\providecommand{\version}{final}|\\
|\||fi|
\end{tabular}
\end{center}
%
The definition by |\providecommand| makes sure
that previous definitions are not overwritten.
Further statements |\providecommand{\version}{...}|
can thus be added before the above code to override it.

For the main file, one might add a line
(between |\childdocmain| and the above block)
%
\begin{center}
|%\ifchilddoc\||else\providecommand{\version}{draft}\||fi|
\end{center}
%
which can be uncommented to produce a draft version.
Likewise one can add a line to the very top of a child file
(above the |\childdocof{|\textit{main}|}| directive)
%
\begin{center}
|%\providecommand{\version}{final}|
\end{center}
%
which can be uncommented to produce the final version of this child document.

%%%%%%%%%%%%%%%%%%%%%%%%%%%%%%%%%%%%%%%%%%%%%%%%%%%%%%%%%%%%%%%%%%%%%%%%%%%%%%%%
\subsection{Forwarding}
\label{sec:forward}

Different versions of the main or child documents
using compilation flags as described in \secref{sec:flags}
can be (permanently) stored in different files
for convenient compilation, viewing and distribution.
To this end, the package defines a command
to pass on compilation to a different file:

%%%%%%%%%%%%%%%%%%%%%%%%%%%%%%%%%%%%%%%%
\DescribeMacro{\childdocforward}
The command |\childdocforward| redirects processing to
another source file:
%
\begin{center}
\begin{tabular}{l}
|\input{childdoc.def}|\\
|\childdocforward[|\textit{main}|]{|\textit{dest}|}|\\
\end{tabular}
\end{center}
%
The argument \textit{dest} is the destination file
(without extension).
It should be the main file or one of the child files.
Note that further \textsf{childdoc} directives
such as |\childdocof| and |\childdocforward|
in the indicated file will be processed in this form.
The optional argument \textit{main}
passes on directly to the main file \textit{main}
while pretending to compile the child \textit{dest}.
This form behaves as if \textit{dest}
issues |\childdocof{|\textit{main}|}| right away,
and no further \textsf{childdoc} directives will be processed.

%%%%%%%%%%%%%%%%%%%%%%%%%%%%%%%%%%%%%%%%
\DescribeMacro{\...prefix}
In the alternative form |\childdocforwardprefix|,
%
\begin{center}
\begin{tabular}{l}
|\input{childdoc.def}|\\
|\childdocforwardprefix[|\textit{main}|]{|\textit{prefix}|}{|\textit{dest}|}|
\end{tabular}
\end{center}
%
the destination file is determined by a pattern
depending on the current file:
To make this work, the current file must be called
`{\textit{prefix}\hspace{0.2em}\textit{suffix}}'
with \textit{prefix} matching precisely the argument.
Processing is then passed on to the file
`{\textit{dest}\hspace{0.2em}\textit{suffix}}'.
Surely, the same effect is achieved by
directly specifying the
argument `{\textit{dest}\hspace{0.2em}\textit{suffix}}'
in the first form.
However, that requires to set up a different file
for each child. With the alternative form of the command
all these files can have exactly the same content
which simplifies setting them up and maintaining them.

For example, the following file |draft.tex|
with a compilation flag |\version| as described in \secref{sec:flags}
compiles the main document as a draft:
%
\begin{center}
\begin{tabular}{l}
|\def\version{draft}|\\
|\input{childdoc.def}|\\
|\childdocforward{|\textit{main}|}|
\end{tabular}
\end{center}
%
Likewise, the following files |final|\textit{nn}|.tex|
compile the final version of the child document
|child|\textit{nn}|.tex|:
%
\begin{center}
\begin{tabular}{l}
|\def\version{final}|\\
|\input{childdoc.def}|\\
|\childdocforwardprefix{final}{child}|
\end{tabular}
\end{center}
%

Note that when several versions of a main file and/or of each child file
are to be generated, it may be convenient to set up a |Makefile| or
shell script to automatise the process.

%%%%%%%%%%%%%%%%%%%%%%%%%%%%%%%%%%%%%%%%%%%%%%%%%%%%%%%%%%%%%%%%%%%%%%%%%%%%%%%%
\subsection{Command Line Processing}
\label{sec:commandline}

The effect of redirection files can also be achieved by invoking
the \LaTeX{} compiler with a more elaborate command line.
Most conveniently this should be done as part
of a shell script or a |Makefile|.

When using \textsf{childdoc} in the main file, the following
command lines effectively perform a redirection
(note that depending on the shell being used,
backslashes may have to be doubled: `|\|' $\to$ `|\\|'):
%
\begin{center}
|... -jobname "|\textit{target}|" |\\|"|[\textit{flags}]%
|\input{childdoc.def}\childdocforward[|\textit{main}|]{|\textit{dest}|}"|
\end{center}
%
Here \textit{target} is the name of the output file,
\textit{main} is the name of the main file
and \textit{dest} is the name of the main or child file to be processed
(all filenames without extensions).
The optional argument \textit{main} can be omitted
if \textit{main} matches \textit{dest}.
Optionally, compilation \textit{flags} can be defined via |\def| commands.
This command line makes the \TeX{} engine believe
it is compiling the file \textit{target}
whose content is specified as the latter parameter.
The provided code then forwards the processing to
\textit{main} or \textit{dest} as described in \secref{sec:forward}.

%%%%%%%%%%%%%%%%%%%%%%%%%%%%%%%%%%%%%%%%%%%%%%%%%%%%%%%%%%%%%%%%%%%%%%%%%%%%%%%%
\subsection{Include by Input}
\label{sec:input}

Including child documents by |\include| has some restrictions by design.
Most notably, the content of a child document always occupies
its own set of pages; pages cannot be shared between child documents.
Usually, this behaviour makes perfect sense
because each child document contain an essential part of the document.
However, in some situations it may be desirable to compose
a document from a collection of parts
without having mandatory page breaks between then.
For this case, the package
provides a mechanism to include parts
by |\input| which can also be processed individually.
However, by construction this mechanism
requires manual handling of the content to be output.

%%%%%%%%%%%%%%%%%%%%%%%%%%%%%%%%%%%%%%%%
\DescribeMacro{\ifchilddocmanual}
The main file should be prepared as usual, see \secref{sec:include}.
However, the document body must make a distinction
between processing of an individual part and of the main document, e.g.:
%
\begin{center}
\begin{tabular}{l}
|\ifchilddocmanual|\\
|\input{\childdocname}|\\
|\||else|\\
\textit{document body with }|\input{|\textit{part}|}|\\
|\||fi|
\end{tabular}
\end{center}
%
The conditional |\ifchilddocmanual| is true whenever
a part to be included by |\input| is being compiled,
and the name of the part is stored in |\childdocname|.

%%%%%%%%%%%%%%%%%%%%%%%%%%%%%%%%%%%%%%%%
\DescribeMacro{\childdocby}
Each part to be included by |\input| should start with:
%
\begin{center}
\begin{tabular}{l}
|\input{childdoc.def}|\\
|\childdocby{|\textit{main}|}|\\
\end{tabular}
\end{center}
%
The directive |\childdocby| is similar to |\childdocof|
described in \secref{sec:include},
but the subsequent selection of content must be done manually.
To that end, both |\ifchilddoc| and |\ifchilddocmanual|
will be true upon processing of a part,
and the name of the part is stored in |\childdocname|.
Note that |\jobname| will be set to the filename of the current part
so that each part receives an individual |.aux| file
that does not interfere with the |.aux| file(s) of the main document.
This behaviour can be altered by the alternative form
|\childdocby[*]{|\textit{main}|}| (with a non-empty optional argument)
which uses the |.aux| file of the main document
by setting |\jobname| to \textit{main}.

%%%%%%%%%%%%%%%%%%%%%%%%%%%%%%%%%%%%%%%%%%%%%%%%%%%%%%%%%%%%%%%%%%%%%%%%%%%%%%%%
\subsection{Driver Development}
\label{sec:driver}

The \textsf{childdoc} mechanism can also be use for the development
of definition files such as \LaTeX{} styles or classes.
This case differs from the above setup with multiple parts
included by |\include| in that no |\includeonly| should be invoked.
This can be achieved by starting the include file
(before |\ProvidesPackage|) with:
%
\begin{center}
\begin{tabular}{l}
|\input{childdoc.def}|\\
|\childdocforward{|\textit{main}|}|\\
\end{tabular}
\end{center}
%
or alternatively with:
%
\begin{center}
\begin{tabular}{l}
|\input{childdoc.def}|\\
|\childdocby{|\textit{main}|}|\\
\end{tabular}
\end{center}
%
Both forms have slightly different effects as described above.
The main file is prepared as usual, see \secref{sec:include}.

%%%%%%%%%%%%%%%%%%%%%%%%%%%%%%%%%%%%%%%%%%%%%%%%%%%%%%%%%%%%%%%%%%%%%%%%%%%%%%%%
\subsection{Legacy Detection}
\label{sec:detection}

The directive |\childdocmain| in the main file can detect
whether the complete document or merely a child is to be compiled
even without using the directive |\childdocof|.
This method is deprecated because it is less robust
and there is no compelling reason to use it;
it is merely provided for backward compatibility
and it may be removed in future versions.

If the detection mechanism is to be used,
it is mandatory to correctly specify
the filename of the main file as the argument of |\childdocmain|:
%
\begin{center}
\begin{tabular}{l}
|\input{childdoc.def}|\\
|\childdocmain{|\textit{main}|}|\\
\end{tabular}
\end{center}
%
If |\jobname| does not match the argument \textit{main} of |\childdocmain|,
it is assumed that |\jobname| points to the child file to be compiled.
When using |\childdocmain| with the main file specified as argument,
it suffices to start a child file
with just |\input{|\textit{main}|}|
without loading of the package and using |\childdocof|.
If instead all processing is done
with the appropriate \textsf{childdoc} directives,
the argument of \textit{main} of |\childdocmain| can be empty.

An alternative version of the command line processing described
in \secref{sec:commandline} using the detection mechanism reads:
%
\begin{center}
|... -jobname "|\textit{target}|" "|[\textit{flags}]%
[|\def\jobname{|\textit{dest}|}|]|\input{|\textit{main}|}"|
\end{center}

%%%%%%%%%%%%%%%%%%%%%%%%%%%%%%%%%%%%%%%%%%%%%%%%%%%%%%%%%%%%%%%%%%%%%%%%%%%%%%%%
\subsection{Manual Code}
\label{sec:manual}

In case one cannot be certain whether the definitions file |childdoc.def|
is installed on the target \TeX{} distribution
and one prefers not to ship it,
it is conceivable to paste a few relevant commands into the sources.

To that end, drop all statements |\input{childdoc.def}|
and perform the replacements as outlined below.
Instead of |\childdocmain{|\textit{main}|}| add the following code
to the top of the main file:
%
\begin{center}
\begin{tabular}{l}
|\||ifdefined\childdocname\endinput\||fi\newif\ifchilddoc|\\
|\edef\childdocname{\scantokens\expandafter{\jobname\noexpand}}|\\
|\def\childdocmain{|\textit{main}|}\||ifx\childdocmain\childdocname\||else|\\
|\childdoctrue\includeonly{\childdocname}\let\jobname\childdocmain\||fi|\\
\end{tabular}
\end{center}
%
Instead of |\childdocof{|\textit{main}|}| just include the main file
at the top of each child file:
%
\begin{center}
|\input{|\textit{main}|}|
\end{center}
%
A simple redirection |\childdocforward{|\textit{dest}|}| is achieved by:
%
\begin{center}
|\def\jobname{|\textit{dest}|}\input{\jobname}|
\end{center}
%
The redirection with prefix
|\childdocforwardprefix[|\textit{prefix}|]{|\textit{dest}|}|
is accomplished by:
%
\begin{center}
\begin{tabular}{l}
|{\edef\jobname{\scantokens\expandafter{\jobname\noexpand}}|\\
|\def\redirectjob |\textit{prefix}|#1~~~{\gdef\jobname{|\textit{dest}|#1}}|\\
|\expandafter\redirectjob\jobname~~~}\input{\jobname}|
\end{tabular}
\end{center}

In an alternative approach,
child documents can be compiled by a specific command line
without additional code or specific definitions:
%
\begin{center}
|... -jobname "|\textit{target}|" "|[\textit{flags}]%
|\includeonly{|\textit{dest}|}\input{|\textit{main}|}"|
\end{center}
%

%%%%%%%%%%%%%%%%%%%%%%%%%%%%%%%%%%%%%%%%%%%%%%%%%%%%%%%%%%%%%%%%%%%%%%%%%%%%%%%%
%%%%%%%%%%%%%%%%%%%%%%%%%%%%%%%%%%%%%%%%%%%%%%%%%%%%%%%%%%%%%%%%%%%%%%%%%%%%%%%%
\section{Information}

%%%%%%%%%%%%%%%%%%%%%%%%%%%%%%%%%%%%%%%%%%%%%%%%%%%%%%%%%%%%%%%%%%%%%%%%%%%%%%%%
\subsection{Copyright}

Copyright \copyright{} 2017--2018 Niklas Beisert

This work may be distributed and/or modified under the
conditions of the \LaTeX{} Project Public License, either version 1.3
of this license or (at your option) any later version.
The latest version of this license is in
  \url{http://www.latex-project.org/lppl.txt}
and version 1.3 or later is part of all distributions of \LaTeX{}
version 2005/12/01 or later.

This work has the LPPL maintenance status `maintained'.

The Current Maintainer of this work is Niklas Beisert.

This work consists of the files |README.txt|, |childdoc.ins| and |childdoc.dtx|
as well as the derived files |childdoc.def|, |cdocsamp.tex|
with |cdocsch1.tex|, |cdocsch2.tex|, |cdocspt3.tex|, |cdocspt4.tex|,
|cdocsdrf.tex|, |cdocsfn1.tex|, |cdocsfn2.tex|
as well as |childdoc.pdf|.

%%%%%%%%%%%%%%%%%%%%%%%%%%%%%%%%%%%%%%%%%%%%%%%%%%%%%%%%%%%%%%%%%%%%%%%%%%%%%%%%
\subsection{Files and Installation}

The package consists of the files:
%
\begin{center}
\begin{tabular}{ll}
    |README.txt|   & readme file \\
    |childdoc.ins| & installation file \\
    |childdoc.dtx| & source file \\
    |childdoc.def| & definition file \\
    |cdocsamp.tex| & sample main file \\
    |cdocsch1.tex| & sample include file \\
    |cdocsch2.tex| & sample include file \\
    |cdocspt3.tex| & sample part file \\
    |cdocspt4.tex| & sample part file \\
    |cdocsdrf.tex| & sample redirection file \\
    |cdocsfn1.tex| & sample redirection file \\
    |cdocsfn2.tex| & sample redirection file \\
    |childdoc.pdf| & manual
\end{tabular}
\end{center}
%
The distribution consists of the files
|README.txt|, |childdoc.ins| and |childdoc.dtx|.
%
\begin{itemize}
\item
Run (pdf)\LaTeX{} on |childdoc.dtx|
to compile the manual |childdoc.pdf| (this file).
\item
Run \LaTeX{} on |childdoc.ins| to create the definitions file |childdoc.def|
and the sample |cdocsamp.tex| with include files
|cdocsch1.tex|, |cdocsch2.tex|, |cdocspt3.tex|, |cdocspt4.tex|,
|cdocsdrf.tex|, |cdocsfn1.tex|, |cdocsfn2.tex|.
Then copy the file |childdoc.def| to an appropriate directory of your \LaTeX{}
distribution, e.g.\ \textit{texmf-root}|/tex/latex/childdoc|.
\end{itemize}

%%%%%%%%%%%%%%%%%%%%%%%%%%%%%%%%%%%%%%%%%%%%%%%%%%%%%%%%%%%%%%%%%%%%%%%%%%%%%%%%
\subsection{Related CTAN Packages}

There are several other packages which offer a similar functionality:
%
\begin{itemize}
\item
The packages
\href{http://ctan.org/pkg/docmute}{\textsf{docmute}},
\href{http://ctan.org/pkg/includex}{\textsf{includex}} and
\href{http://ctan.org/pkg/standalone}{\textsf{standalone}}
provide commands to include only the document body of
a child file thus allowing both files to be compiled individually.
\item
The packages \href{http://ctan.org/pkg/subdocs}{\textsf{subdocs}}
and \href{http://ctan.org/pkg/subfiles}{\textsf{subfiles}}
provide structures in which the main and child documents can be
encapsulated and allowing them to be compiled individually.
The inclusion mechanism is different from the conventional |\include|.
\item
The package \href{http://ctan.org/pkg/combine}{\textsf{combine}}
is an elaborate solution to combine several documents into one.
\end{itemize}
%
See also the CTAN topic \href{http://ctan.org/topic/subdocs}{\textsf{subdocs}}
for further related packages.
The present package differs from the above solutions in that
a document structure constructed with the conventional |\include| mechanism
just needs two extra commands at the top of every file
such that all constituent files can be compiled individually.

%%%%%%%%%%%%%%%%%%%%%%%%%%%%%%%%%%%%%%%%%%%%%%%%%%%%%%%%%%%%%%%%%%%%%%%%%%%%%%%%
%\subsection{Feature Suggestions}
%
%The following is a list of features which may be useful for future
%versions of this package:
%%
%\begin{itemize}
%\item
%\ldots
%\end{itemize}

%%%%%%%%%%%%%%%%%%%%%%%%%%%%%%%%%%%%%%%%%%%%%%%%%%%%%%%%%%%%%%%%%%%%%%%%%%%%%%%%
\subsection{Revision History}

%%%%%%%%%%%%%%%%%%%%%%%%%%%%%%%%%%%%%%%%
\paragraph{v2.0:} 2018/12/30

\begin{itemize}
\item
immediate forward processing
\item
added |\childdocby| mechanism
\item
manual restructured
\end{itemize}

%%%%%%%%%%%%%%%%%%%%%%%%%%%%%%%%%%%%%%%%
\paragraph{v1.6:} 2018/01/17

\begin{itemize}
\item
application for development of include files
\item
corrections to manual
\end{itemize}

%%%%%%%%%%%%%%%%%%%%%%%%%%%%%%%%%%%%%%%%
\paragraph{v1.5:} 2017/05/21

\begin{itemize}
\item
more complete structuring introduced
\item
|\childdocof| introduced
\item
|\childdoc| renamed to |\childdocmain|
\item
|\childredirect| renamed to |\childdocforward| and |\childdocforwardprefix|
and functionality expanded
\end{itemize}

%%%%%%%%%%%%%%%%%%%%%%%%%%%%%%%%%%%%%%%%
\paragraph{v1.0:} 2017/04/27

\begin{itemize}
\item
manual and install package
\item
first version published on CTAN
\end{itemize}

%%%%%%%%%%%%%%%%%%%%%%%%%%%%%%%%%%%%%%%%
\paragraph{v0.6:} 2017/04/26

\begin{itemize}
\item
redirection mechanism added
\end{itemize}

%%%%%%%%%%%%%%%%%%%%%%%%%%%%%%%%%%%%%%%%
\paragraph{v0.5:} 2017/04/26

\begin{itemize}
\item
functionality in definition file
\end{itemize}


%%%%%%%%%%%%%%%%%%%%%%%%%%%%%%%%%%%%%%%%%%%%%%%%%%%%%%%%%%%%%%%%%%%%%%%%%%%%%%%%
%%%%%%%%%%%%%%%%%%%%%%%%%%%%%%%%%%%%%%%%%%%%%%%%%%%%%%%%%%%%%%%%%%%%%%%%%%%%%%%%
%%%%%%%%%%%%%%%%%%%%%%%%%%%%%%%%%%%%%%%%%%%%%%%%%%%%%%%%%%%%%%%%%%%%%%%%%%%%%%%%
\appendix

\settowidth\MacroIndent{\rmfamily\scriptsize 000\ }

 \DocInput{childdoc.dtx}

\end{document}
%</driver>
% \fi
%
% %%%%%%%%%%%%%%%%%%%%%%%%%%%%%%%%%%%%%%%%%%%%%%%%%%%%%%%%%%%%%%%%%%%%%%%%%%%%%%
% %%%%%%%%%%%%%%%%%%%%%%%%%%%%%%%%%%%%%%%%%%%%%%%%%%%%%%%%%%%%%%%%%%%%%%%%%%%%%%
% \section{Sample}
%\iffalse
%<*samplemain>
%\fi
%
% The following presents a sample document
% with two chapters, two parts, a title page,
% a compile flag as well as three forwarding files to set the flag.
% It consists of eight |.tex| files:
% \begin{center}
% \begin{tabular}{ll}
% |cdocsamp.tex|&main file\\
% |cdocsch1.tex|&include file for chapter 1\\
% |cdocsch2.tex|&include file for chapter 2\\
% |cdocspt3.tex|&include file for part 3\\
% |cdocspt4.tex|&include file for part 4\\
% |cdocsdrf.tex|&forwarding file for main file in draft mode\\
% |cdocsfi1.tex|&forwarding file for final version of chapter 1\\
% |cdocsfi2.tex|&forwarding file for final version of chapter 2\\
% \end{tabular}
% \end{center}
% Each of the eight files can be compiled directly by the \LaTeX{} compiler.
%
% %%%%%%%%%%%%%%%%%%%%%%%%%%%%%%%%%%%%%%
% \paragraph{Main File.}
%
% The main file is called |cdocsamp.tex|.
%
% Load the \textsf{childdoc} definitions and
% declare the filename for the main document:
%    \begin{macrocode}
\input{childdoc.def}
\childdocmain{}
%    \end{macrocode}

% Optional override for |\version| flag:
%    \begin{macrocode}
%%\ifchilddoc\else\providecommand{\version}{draft}\fi
%    \end{macrocode}

% Define the default values for the |\version| flag
% (|final| for the main file and |draft| for childs):
%    \begin{macrocode}
\ifchilddoc
\providecommand{\version}{draft}
\else
\providecommand{\version}{final}
\fi
%    \end{macrocode}

% Load the standard document class:
%    \begin{macrocode}
\documentclass[12pt]{article}
%    \end{macrocode}

% Start the document body:
%    \begin{macrocode}
\begin{document}
%    \end{macrocode}

% Declare a title page.
% Print title, part of document being processed and version flag:
%    \begin{macrocode}
\addtocounter{page}{-1}
\begin{center}
{\LARGE\bfseries{}childdoc example\par}
\vspace{1cm}
\ifchilddoc
\ifchilddocmanual part\else chapter\fi:
`\childdocname' of `\childdocjob'\par
\else
main document: `\childdocjob'\par
\fi
version: \version\par
\end{center}
\newpage
%    \end{macrocode}

% Manually include selected file,
% otherwise process as usual:
%    \begin{macrocode}
\ifchilddocmanual
\section*{part `\childdocname'}
\input{\childdocname}
\else
%    \end{macrocode}

% Include the two chapters:
%    \begin{macrocode}
\include{cdocsch1}
\include{cdocsch2}
%    \end{macrocode}

% Include the two parts unless only chapters should be displayed:
%    \begin{macrocode}
\ifchilddoc\else
\section{part three}
\input{cdocspt3}
\section{part four}
\input{cdocspt4}
\fi
%    \end{macrocode}

% Process as usual until here:
%    \begin{macrocode}
\fi
%    \end{macrocode}

% End of document body:
%    \begin{macrocode}
\end{document}
%    \end{macrocode}
%\iffalse
%</samplemain>
%\fi
%
% %%%%%%%%%%%%%%%%%%%%%%%%%%%%%%%%%%%%%%
% \paragraph{Chapter Include Files.}
%
% The include files are called |cdocsch1.tex| and |cdocsch2.tex|.
%
%\iffalse
%<*samplechap1|samplechap2>
%\fi

% Optional override for |\version| flag:
%    \begin{macrocode}
%%\providecommand{\version}{final}
%    \end{macrocode}

% Include the main document:
%    \begin{macrocode}
\input{childdoc.def}
\childdocof{cdocsamp}
%    \end{macrocode}

%\iffalse
%</samplechap1|samplechap2>
%\fi
%
%\iffalse
%<*samplechap1>
%\fi
% Some text for chapter 1:
%    \begin{macrocode}
\section{one}
some text in chapter one
%    \end{macrocode}

%\iffalse
%</samplechap1>
%\fi
% Some text for chapter 2:
%\iffalse
%<*samplechap2>
%\fi
%    \begin{macrocode}
\section{two}
more text in chapter two
%    \end{macrocode}

%\iffalse
%</samplechap2>
%\fi
%
% %%%%%%%%%%%%%%%%%%%%%%%%%%%%%%%%%%%%%%
% \paragraph{Part Include Files.}
%
% The include files are called |cdocspt3.tex| and |cdocspt4.tex|.
%
%\iffalse
%<*samplepart3|samplepart4>
%\fi

% Optional override for |\version| flag:
%    \begin{macrocode}
%%\providecommand{\version}{final}
%    \end{macrocode}

% Include the main document:
%    \begin{macrocode}
\input{childdoc.def}
\childdocby{cdocsamp}
%    \end{macrocode}

%\iffalse
%</samplepart3|samplepart4>
%\fi
%
%\iffalse
%<*samplepart3>
%\fi
% Some text for part 3:
%    \begin{macrocode}
some text in part three
%    \end{macrocode}

%\iffalse
%</samplepart3>
%\fi
% Some text for part 4:
%\iffalse
%<*samplepart4>
%\fi
%    \begin{macrocode}
more text in part four
%    \end{macrocode}

%\iffalse
%</samplepart4>
%\fi
%
% %%%%%%%%%%%%%%%%%%%%%%%%%%%%%%%%%%%%%%
% \paragraph{Forwarding for a Complete Draft.}
%
% The following forwarding file |cdocsdrf.tex|
% compiles the main document in draft mode:
%\iffalse
%<*sampledraft>
%\fi
%    \begin{macrocode}
\def\version{draft}
\input{childdoc.def}
\childdocforward{cdocsamp}
%    \end{macrocode}

%\iffalse
%</sampledraft>
%\fi
%
% %%%%%%%%%%%%%%%%%%%%%%%%%%%%%%%%%%%%%%
% \paragraph{Forwarding for Final Version of the Chapters.}
%
% The following forwarding files |cdocsfn1.tex| and |cdocsfn2.tex|
% (with identical content)
% compile the final versions of the child documents
% |cdocsch1.tex| and |cdocsch2.tex|, respectively:
%\iffalse
%<*samplefinal>
%\fi
%    \begin{macrocode}
\def\version{final}
\input{childdoc.def}
\childdocforwardprefix[cdocsamp]{cdocsfn}{cdocsch}
%    \end{macrocode}

%\iffalse
%</samplefinal>
%\fi
%
% %%%%%%%%%%%%%%%%%%%%%%%%%%%%%%%%%%%%%%
% \paragraph{Command Line Processing.}
%
% The following three command lines generate the output files
% |cdocscld|, |cdocscl1| and |cdocscl2|
% which should be identical to
% |cdocsdrf|, |cdocsch1| and |cdocsfn2|, respectively:
% \begin{center}
% \begin{tabular}{l}
% |latex -jobname cdocscld \|\\
% |  "\def\version{draft}\input{childdoc.def}\childdocforward{cdocsamp}"|\\
% |latex -jobname cdocscl1 \|\\
% |  "\input{childdoc.def}\childdocforward[cdocsamp]{cdocsch1}"|\\
% |latex -jobname cdocscl2 \|\\
% |  "\def\version{final}\input{childdoc.def}\childdocforward{cdocsch2}"|
% \end{tabular}
% \end{center}
% Note that the trailing backslash on each first line
% merely continues the input to the second line
% (for convenient cut ant paste).
% Furthermore, the command |latex| can be replaced by any
% of its alternative versions such as |pdflatex|.
%
% %%%%%%%%%%%%%%%%%%%%%%%%%%%%%%%%%%%%%%%%%%%%%%%%%%%%%%%%%%%%%%%%%%%%%%%%%%%%%%
% %%%%%%%%%%%%%%%%%%%%%%%%%%%%%%%%%%%%%%%%%%%%%%%%%%%%%%%%%%%%%%%%%%%%%%%%%%%%%%
% \section{Implementation}
%\iffalse
%<*package>
%\fi
%
% This section describes the definitions file |childdoc.def|.

% The definitions cannot be loaded using |\usepackage| or |\RequirePackage|
% which has a mechanism to prevent loading a style file more than once.
% When loading the definitions by means of |\input|
% multiple instances have to be prevented manually:
%\iffalse
%This code needs to be before the `\ProvidesFile' directive
%which is defined at the beginning of this file.
%Therefore it is also placed there and commented out here.
%</package>
%<*discard>
%\fi
%    \begin{macrocode}
\ifdefined\childdocmain\endinput\fi
%    \end{macrocode}
%\iffalse
%</discard>
%<*package>
%\fi
%
% \macro{\ifchilddoc}
% \macro{\ifchilddocmanual}
% The conditional |\ifchilddoc| tells whether a
% child (true) or main (false) document is being compiled.
% The conditional |\ifchilddocmanual| tells whether
% the |\includeonly| mechanism is used (false) or
% the selection of child files must be performed manually (true).
% The definitions initialise to false:
%    \begin{macrocode}
\newif\ifchilddoc
\newif\ifchilddocmanual
%    \end{macrocode}

% \macro{\childdocname}
% \macro{\childdocjob}
% The macro |\childdocname| stores the name of the main document
% to be compiled. The macro |\childdocjob| stores the name of
% the document on which the \LaTeX{} compiler was originally invoked.
% The content of |\jobname| cannot be compared
% to filenames specified in the source due to different catcodes.
% The following code rescans |\jobname|, stores the result
% in |\childdocname| and saves a copy in |\childdocjob|:
%    \begin{macrocode}
\edef\childdocname{\scantokens\expandafter{\jobname\noexpand}}
\let\childdocjob\childdocname
%    \end{macrocode}

% \macro{\childdocdisable}
% The macro |\childdocdisable| prevents the main file
% from being processed more than once.
% At this stage, the main document command |\childdocmain|
% is assumed to be called once again where it should do nothing.
% Any subsequent call to it should prevent
% a secondary processing of the main document
% It overwrites the forwarding commands
% |\childdocof| and |\childdocforward|
% with empty macros to prevent further inclusions of the main document:
%    \begin{macrocode}
\newcommand{\childdocdisable}
{
  \renewcommand{\childdocmain}[1]{\renewcommand{\childdocmain}[1]{\endinput}}
  \renewcommand{\childdocof}[1]{}
  \renewcommand{\childdocby}[2][]{}
  \renewcommand{\childdocforward}[2][]{}
  \renewcommand{\childdocdisable}{}
}
%    \end{macrocode}

% \macro{\childdocmain}
% The macro |\childdocmain| is to be called at the top of the main file
% with nothing or the main filename (without extension) as argument.
% First, it breaks loops.
% If the argument is not empty and does not match |\childdocname|
% (which is set by the first inclusion of |childdoc.def|),
% |\ifchilddoc| is set to true, |\includeonly| is applied to the child file
% and |\jobname| is set to the main file
% (for proper handling of |.aux| files):
%    \begin{macrocode}
\newcommand{\childdocmain}[1]
{
  \childdocdisable\childdocmain{}
  \if?#1?\else
    \begingroup
      \def\childdoctmp{#1}
      \ifx\childdoctmp\childdocname
        \def\childdoctmp{}
      \else
        \def\childdoctmp
        {
          \childdoctrue
          \includeonly{\childdocname}
          \def\childdocjob{#1}
          \def\jobname{#1}
        }
      \fi
      \expandafter
    \endgroup
    \childdoctmp
  \fi
}
%    \end{macrocode}

% \macro{\childdocof}
% The command |\childdocof| redirects
% compilation to the main file |#1|.
%    \begin{macrocode}
\newcommand{\childdocof}[1]
{
  \childdocdisable
  \childdoctrue
  \includeonly{\childdocname}
  \def\jobname{#1}
  \def\childdocjob{#1}
  \input{#1}
}
%    \end{macrocode}

% \macro{\childdocby}
% The command |\childdocby| ....
%    \begin{macrocode}
\newcommand{\childdocby}[2][]
{
  \childdocdisable
  \childdoctrue
  \childdocmanualtrue
  \if?#1?\else
    \def\jobname{#2}
  \fi
  \def\childdocjob{#2}
  \input{#2}
  \endinput
}
%    \end{macrocode}

% \macro{\childdocforward}
% The command |\childdocforward| redirects
% compilation to the main file or
% (if the optional argument is given) a child file.
% Parameters are set as if the main file
% or a child file starting with |\childdocof| was compiled.
% Then compilation is handed over to the main file:
%    \begin{macrocode}
\newcommand{\childdocforward}[2][]
{
  \begingroup
    \if?#1?
      \def\childdoctmp
      {
        \def\childdocname{#2}
        \def\childdocjob{#2}
        \def\jobname{#2}
        \input{#2}
        \endinput
      }
    \else
      \def\childdoctmp
      {
        \childdocdisable
        \def\childdocname{#2}
        \childdoctrue
        \includeonly{#2}
        \def\childdocjob{#1}
        \def\jobname{#1}
        \input{#1}
        \endinput
      }
    \fi
    \expandafter
  \endgroup
  \childdoctmp
}
%    \end{macrocode}

% \macro{\childdocforwardprefix}
% The command |\childdocforwardprefix| redirects
% compilation to the main or a child file by means of a pattern.
% The prefix |#1| in the current filename is replaced by |#2|
% and the suffix of the current filename is kept
% (it is assumed that the filename does not contain the substring `|~~~|'
% which is used as a delimiter).
% Compilation is handed over to the new file by |\childdocforward|:
%    \begin{macrocode}
\newcommand{\childdocforwardprefix}[3][]
{
  \begingroup
    \def\childdocextract #2##1~~~{\def\childdoctmp{\childdocforward[#1]{#3##1}}}
    \expandafter\childdocextract\childdocname~~~
    \expandafter
  \endgroup
  \childdoctmp
}
%    \end{macrocode}

% \macro{\childdoc}
% The deprecated macro |\childdoc| is a legacy version of |\childdocmain|:
%    \begin{macrocode}
\newcommand{\childdoc}{\childdocmain}
%    \end{macrocode}

% \macro{\childdocredirect}
% The deprecated macro |\childdocredirect| is a legacy version
% of |\childdocforward| and |\childdocforwardprefix|:
%    \begin{macrocode}
\newcommand{\childdocredirect}[2][]
{
  \begingroup
    \if?#1?
      \def\childdoctmp{\childdocforward{#2}}
    \else
      \def\childdoctmp{\childdocforwardprefix{#1}{#2}}
    \fi
    \expandafter
  \endgroup
  \childdoctmp
}
%    \end{macrocode}

%\iffalse
%</package>
%\fi
%
\endinput
\childdocforward[cdocsamp]{cdocsch1}"|\\
% |latex -jobname cdocscl2 \|\\
% |  "\def\version{final}% \iffalse
%
% childdoc.dtx Copyright (C) 2017-2018 Niklas Beisert
%
% This work may be distributed and/or modified under the
% conditions of the LaTeX Project Public License, either version 1.3
% of this license or (at your option) any later version.
% The latest version of this license is in
%   http://www.latex-project.org/lppl.txt
% and version 1.3 or later is part of all distributions of LaTeX
% version 2005/12/01 or later.
%
% This work has the LPPL maintenance status `maintained'.
%
% The Current Maintainer of this work is Niklas Beisert.
%
% This work consists of the files childdoc.dtx and childdoc.ins
% and the derived files childdoc.def and cdocsamp.tex with
% cdocsch1.tex, cdocsch2.tex, cdocsdrf.tex, cdocsfn1.tex, cdocsfn2.tex.
%
%<package>\ifdefined\childdocmain\endinput\fi
%<package>\ProvidesFile{childdoc.def}[2018/12/30 v2.0 child document driver]
%<samplemain>\ProvidesFile{cdocsamp.tex}[2018/12/30 v2.0 sample for childdoc]
%<*driver>
%\ProvidesFile{childdoc.drv}[2018/12/30 v2.0 childdoc reference manual file]
\PassOptionsToClass{10pt,a4paper}{article}
\documentclass{ltxdoc}

\usepackage[margin=35mm]{geometry}
\usepackage{hyperref}
\usepackage{hyperxmp}
\usepackage[usenames]{color}

\hypersetup{colorlinks=true}
\hypersetup{pdfstartview=FitH}
\hypersetup{pdfpagemode=UseNone}
\hypersetup{pdfsource={}}
\hypersetup{pdflang={en-UK}}
\hypersetup{pdfcopyright={Copyright 2017-2018 Niklas Beisert.
  This work may be distributed and/or modified under the
  conditions of the LaTeX Project Public License, either version 1.3
  of this license or (at your option) any later version.}}
\hypersetup{pdflicenseurl={http://www.latex-project.org/lppl.txt}}
\hypersetup{pdfcontactaddress={ETH Zurich, ITP, HIT K,
  Wolfgang-Pauli-Strasse 27}}
\hypersetup{pdfcontactpostcode={8093}}
\hypersetup{pdfcontactcity={Zurich}}
\hypersetup{pdfcontactcountry={Switzerland}}
\hypersetup{pdfcontactemail={nbeisert@itp.phys.ethz.ch}}
\hypersetup{pdfcontacturl={http://people.phys.ethz.ch/\xmptilde nbeisert/}}

\newcommand{\secref}[1]{\hyperref[#1]{section \ref*{#1}}}

\parskip1ex
\parindent0pt
\let\olditemize\itemize
\def\itemize{\olditemize\parskip0pt}

\begin{document}

\title{The \textsf{childdoc} Package}
\hypersetup{pdftitle={The childdoc Package}}
\author{Niklas Beisert\\[2ex]
  Institut f\"ur Theoretische Physik\\
  Eidgen\"ossische Technische Hochschule Z\"urich\\
  Wolfgang-Pauli-Strasse 27, 8093 Z\"urich, Switzerland\\[1ex]
  \href{mailto:nbeisert@itp.phys.ethz.ch}
  {\texttt{nbeisert@itp.phys.ethz.ch}}}
\hypersetup{pdfauthor={Niklas Beisert}}
\hypersetup{pdfsubject={Manual for the LaTeX2e Package childdoc}}
\date{30 December 2018, \textsf{v2.0}}
\maketitle

\begin{abstract}\noindent
\textsf{childdoc} is a \LaTeXe{} package
that enables the direct compilation
of document sections included by |\include|
to individual files.
\end{abstract}

\begingroup
\parskip0ex
\tableofcontents
\endgroup

%%%%%%%%%%%%%%%%%%%%%%%%%%%%%%%%%%%%%%%%%%%%%%%%%%%%%%%%%%%%%%%%%%%%%%%%%%%%%%%%
%%%%%%%%%%%%%%%%%%%%%%%%%%%%%%%%%%%%%%%%%%%%%%%%%%%%%%%%%%%%%%%%%%%%%%%%%%%%%%%%
\section{Introduction}

\LaTeX{} provides a mechanism to structure a large document (such as a book)
into a main file and several child files (containing the chapters)
using the |\include| command.
This mechanism is beneficial for documents
which span hundreds of pages in order to
make the source file(s) more manageable.
Moreover, compilation can be restricted to
selected child files by means of the |\includeonly| command.
The latter feature can be used to reduce the compilation time while editing
(this was significantly more useful in the earlier days of \LaTeX{})
or to generate a smaller document which is easier to navigate.
Another application of |\includeonly| is to generate
documents consisting of selected parts of the complete document.

However, there are a few drawbacks of the plain |\include| mechanism:
\begin{itemize}
\item
The child files cannot be compiled on their own,
they can only be compiled via the main file.
A naive editing environment
(such as a text editor with an option
to have the current file processed by \LaTeX)
may require one to switch to the main file before compiling;
attempting to compile the child file produces errors.
\item
The main file must be modified (each time)
to adjust the |\includeonly| command
to the present needs. This easily leaves the main file in a messy state.
\item
The generated document will always carry the filename
of the main document. This is inconvenient if
several child files are to be compiled and
to be kept for distribution.
\end{itemize}

The present package provides a simple interface
to make child files individually compilable by \LaTeX{}.
Compiling a child file then has the same effect as compiling
the main file with an |\includeonly| command
to select the appropriate child.
Moreover the generated document will carry the name of the child
rather than the main file.
This resolves all three above issues.

This feature is meant to make the editing of books,
thesis documents and lecture notes somewhat more convenient.
However, the package can also be used efficiently for
composing a series of documents (such as exercise sheets)
which are typically distributed individually.
It then assists the author in generating the individual documents
(potentially in different versions)
as well as a document containing the collected series.
Another application is in developing style files
or other kinds of included material
where compilation of the style file could redirect
to a sample or test file.

%%%%%%%%%%%%%%%%%%%%%%%%%%%%%%%%%%%%%%%%%%%%%%%%%%%%%%%%%%%%%%%%%%%%%%%%%%%%%%%%
%%%%%%%%%%%%%%%%%%%%%%%%%%%%%%%%%%%%%%%%%%%%%%%%%%%%%%%%%%%%%%%%%%%%%%%%%%%%%%%%
\section{Usage}

First of all, the package \textsf{childdoc} is \emph{not} a standard
\LaTeXe{} |.sty| style file! Therefore it needs to be invoked in
a non-standard way.

%%%%%%%%%%%%%%%%%%%%%%%%%%%%%%%%%%%%%%%%%%%%%%%%%%%%%%%%%%%%%%%%%%%%%%%%%%%%%%%%
\subsection{Included Files}
\label{sec:include}

%%%%%%%%%%%%%%%%%%%%%%%%%%%%%%%%%%%%%%%%
\DescribeMacro{\childdocmain}
To use the package, add the commands
\begin{center}
\begin{tabular}{l}
|\input{childdoc.def}|\\
|\childdocmain{}|\\
\end{tabular}
\end{center}
at the very top of the main \LaTeX{} file,
in particular \emph{before} the |\documentclass| statement!
The argument of |\childdocmain| should be left empty
(but it must be present).

%%%%%%%%%%%%%%%%%%%%%%%%%%%%%%%%%%%%%%%%
\DescribeMacro{\childdocof}
Furthermore, add the commands
\begin{center}
\begin{tabular}{l}
|\input{childdoc.def}|\\
|\childdocof{|\textit{main}|}|\\
\end{tabular}
\end{center}
at the top of every child file \textit{child}
which is included by |\include{|\textit{child}|}|
from within the main file
(or at least for those files to be compiled individually).
The argument \textit{main} must be the filename of the main file.

There are a couple of
considerations in setting up the main and child documents:

%%%%%%%%%%%%%%%%%%%%%%%%%%%%%%%%%%%%%%%%
\paragraph{Restrictions.}

Please note the following restrictions:
\begin{itemize}
\item
|\childdocmain| must be called with one argument \textit{main}
to ensure compatibility with earlier version of the package.
It must either be empty (|\childdocmain{}|)
or precisely match the filename of the main file in which it is specified.
See \secref{sec:detection} for further information.
\item
The filename \textit{main} must be specified without the |.tex| extension.
\item
The filename \textit{main} is case sensitive
(even in case-insensitive file systems)
due to internal string comparison.
\item
The argument \textit{main} should be fully expanded, it cannot be a macro.
\item
Subdirectories and special characters should be avoided in filenames.
\item
The command |\childdocmain{|\textit{main}|}| must be followed by a whitespace.
It should not be followed immediately by another command
or by a comment mark `|%|'.
This is because the \TeX{} parser reads the token immediately following
the argument of |\childdocmain| and puts it
at the beginning of every child section;
however, a white\-space is ignored.
\end{itemize}

%%%%%%%%%%%%%%%%%%%%%%%%%%%%%%%%%%%%%%%%
\paragraph{Content of Main File.}

It is advisable to place all content in the child files included by |\include|.
Any output contained in the main file will appear in all child documents
unless suppressed manually;
it cannot be suppressed automatically by the |\includeonly| directive
and thus should normally be avoided.
A method to include some content in the main file
by means of conditional processing is described in \secref{sec:conditional}.

%%%%%%%%%%%%%%%%%%%%%%%%%%%%%%%%%%%%%%%%
\paragraph{Page Numbering.}

When only a part of the document is compiled,
the appropriate numbering of pages
(as well as other status parameters)
is determined from the |.aux| files.
The latter contain information from previous passes.
However this information needs to propagate through
all intermediate child documents.
Therefore the page numbering in child documents may well
be inconsistent until the complete document is compiled at least once.

A useful (if unconventional) way to always ensure a consistent
page numbering is to restart the numbering in each child document
and denote the pages by `\textit{child}|.|\textit{page}'
where \textit{child} represents the chapter/section number of the child file.
This can be achieved by the command
|\numberwithin{page}{|\textit{child}|}|
of the \textsf{amsmath} package
where \textit{child} can be |chapter| or |section|
depending on the chosen structuring.
Alternatively, one can modify the macro |\thepage| appropriately
and reset the counter |page| at the start of each child file.

%%%%%%%%%%%%%%%%%%%%%%%%%%%%%%%%%%%%%%%%%%%%%%%%%%%%%%%%%%%%%%%%%%%%%%%%%%%%%%%%
\subsection{Conditional Processing}
\label{sec:conditional}

The package provides a mechanism to compile different versions
of a document. To customise the versions further some conditional processing
can come in handy to distinguish which version is being compiled.
The package provides two macros to describe the compilation context:

%%%%%%%%%%%%%%%%%%%%%%%%%%%%%%%%%%%%%%%%
\DescribeMacro{\ifchilddoc}
The conditional |\ifchilddoc| distinguishes between the compilation of
child documents and the main document:
%
\begin{center}
|\ifchilddoc |\textit{child-code}| |[|\||else |\textit{main-code}]| \||fi|
\end{center}

%%%%%%%%%%%%%%%%%%%%%%%%%%%%%%%%%%%%%%%%
\DescribeMacro{\childdocname}
\DescribeMacro{\childdocjob}
The macro |\childdocname| contains the filename (without extension)
of the main or child file being processed.
Note that |\childdocjob| will always contain the name of the main file.

%%%%%%%%%%%%%%%%%%%%%%%%%%%%%%%%%%%%%%%%
\paragraph{Title Page.}

Conditional processing can be used to include a title or banner page
in the main document when proper precautions are taken.
Importantly, the code in the main file should ensure that the page counter
(as well as other status parameters which are stored in the |.aux| files)
takes the same value after the conditional processing.
Otherwise the page numbers may take divergent values
depending on which part is compiled.

For example, a title page could be declared by:
%
\begin{center}
\begin{tabular}{l}
|\ifchilddoc\||else|\\
|\addtocounter{page}{-1}|\\
\textit{code for title page}\\
|\newpage|\\
|\||fi|
\end{tabular}
\end{center}
%
A banner page for the child documents can be generated by:
%
\begin{center}
\begin{tabular}{l}
|\ifchilddoc|\\
|\addtocounter{page}{-1}|\\
\textit{code for banner page}\\
|\newpage|\\
|\||fi|
\end{tabular}
\end{center}
%
Here one could write a message such as:
\begin{center}
|This is the part \childdocname{} of \childdocjob{}.|
\end{center}

%%%%%%%%%%%%%%%%%%%%%%%%%%%%%%%%%%%%%%%%%%%%%%%%%%%%%%%%%%%%%%%%%%%%%%%%%%%%%%%%
\subsection{Flags}
\label{sec:flags}

The package makes it easy to generate different versions
of the main or child documents.
To this end compilation flags can be defined
and assigned different default values.
They will be particularly useful in conjunction
with the forwarding mechanism described in \secref{sec:forward}.

For example, it may be useful to have a flag |\version|
which can be set to |draft| or |final|.
The document source will contain some conditional code
depending on the value of |\version|.
Suppose further, the flag should default to |final| for the main file
and to |draft| for child files
which is a natural assignment for editing the document.
This is achieved by placing the following code
in the preamble of the main document
(below the |\childdocmain| directive):
%
\begin{center}
\begin{tabular}{l}
|\ifchilddoc|\\
|\providecommand{\version}{draft}|\\
|\||else|\\
|\providecommand{\version}{final}|\\
|\||fi|
\end{tabular}
\end{center}
%
The definition by |\providecommand| makes sure
that previous definitions are not overwritten.
Further statements |\providecommand{\version}{...}|
can thus be added before the above code to override it.

For the main file, one might add a line
(between |\childdocmain| and the above block)
%
\begin{center}
|%\ifchilddoc\||else\providecommand{\version}{draft}\||fi|
\end{center}
%
which can be uncommented to produce a draft version.
Likewise one can add a line to the very top of a child file
(above the |\childdocof{|\textit{main}|}| directive)
%
\begin{center}
|%\providecommand{\version}{final}|
\end{center}
%
which can be uncommented to produce the final version of this child document.

%%%%%%%%%%%%%%%%%%%%%%%%%%%%%%%%%%%%%%%%%%%%%%%%%%%%%%%%%%%%%%%%%%%%%%%%%%%%%%%%
\subsection{Forwarding}
\label{sec:forward}

Different versions of the main or child documents
using compilation flags as described in \secref{sec:flags}
can be (permanently) stored in different files
for convenient compilation, viewing and distribution.
To this end, the package defines a command
to pass on compilation to a different file:

%%%%%%%%%%%%%%%%%%%%%%%%%%%%%%%%%%%%%%%%
\DescribeMacro{\childdocforward}
The command |\childdocforward| redirects processing to
another source file:
%
\begin{center}
\begin{tabular}{l}
|\input{childdoc.def}|\\
|\childdocforward[|\textit{main}|]{|\textit{dest}|}|\\
\end{tabular}
\end{center}
%
The argument \textit{dest} is the destination file
(without extension).
It should be the main file or one of the child files.
Note that further \textsf{childdoc} directives
such as |\childdocof| and |\childdocforward|
in the indicated file will be processed in this form.
The optional argument \textit{main}
passes on directly to the main file \textit{main}
while pretending to compile the child \textit{dest}.
This form behaves as if \textit{dest}
issues |\childdocof{|\textit{main}|}| right away,
and no further \textsf{childdoc} directives will be processed.

%%%%%%%%%%%%%%%%%%%%%%%%%%%%%%%%%%%%%%%%
\DescribeMacro{\...prefix}
In the alternative form |\childdocforwardprefix|,
%
\begin{center}
\begin{tabular}{l}
|\input{childdoc.def}|\\
|\childdocforwardprefix[|\textit{main}|]{|\textit{prefix}|}{|\textit{dest}|}|
\end{tabular}
\end{center}
%
the destination file is determined by a pattern
depending on the current file:
To make this work, the current file must be called
`{\textit{prefix}\hspace{0.2em}\textit{suffix}}'
with \textit{prefix} matching precisely the argument.
Processing is then passed on to the file
`{\textit{dest}\hspace{0.2em}\textit{suffix}}'.
Surely, the same effect is achieved by
directly specifying the
argument `{\textit{dest}\hspace{0.2em}\textit{suffix}}'
in the first form.
However, that requires to set up a different file
for each child. With the alternative form of the command
all these files can have exactly the same content
which simplifies setting them up and maintaining them.

For example, the following file |draft.tex|
with a compilation flag |\version| as described in \secref{sec:flags}
compiles the main document as a draft:
%
\begin{center}
\begin{tabular}{l}
|\def\version{draft}|\\
|\input{childdoc.def}|\\
|\childdocforward{|\textit{main}|}|
\end{tabular}
\end{center}
%
Likewise, the following files |final|\textit{nn}|.tex|
compile the final version of the child document
|child|\textit{nn}|.tex|:
%
\begin{center}
\begin{tabular}{l}
|\def\version{final}|\\
|\input{childdoc.def}|\\
|\childdocforwardprefix{final}{child}|
\end{tabular}
\end{center}
%

Note that when several versions of a main file and/or of each child file
are to be generated, it may be convenient to set up a |Makefile| or
shell script to automatise the process.

%%%%%%%%%%%%%%%%%%%%%%%%%%%%%%%%%%%%%%%%%%%%%%%%%%%%%%%%%%%%%%%%%%%%%%%%%%%%%%%%
\subsection{Command Line Processing}
\label{sec:commandline}

The effect of redirection files can also be achieved by invoking
the \LaTeX{} compiler with a more elaborate command line.
Most conveniently this should be done as part
of a shell script or a |Makefile|.

When using \textsf{childdoc} in the main file, the following
command lines effectively perform a redirection
(note that depending on the shell being used,
backslashes may have to be doubled: `|\|' $\to$ `|\\|'):
%
\begin{center}
|... -jobname "|\textit{target}|" |\\|"|[\textit{flags}]%
|\input{childdoc.def}\childdocforward[|\textit{main}|]{|\textit{dest}|}"|
\end{center}
%
Here \textit{target} is the name of the output file,
\textit{main} is the name of the main file
and \textit{dest} is the name of the main or child file to be processed
(all filenames without extensions).
The optional argument \textit{main} can be omitted
if \textit{main} matches \textit{dest}.
Optionally, compilation \textit{flags} can be defined via |\def| commands.
This command line makes the \TeX{} engine believe
it is compiling the file \textit{target}
whose content is specified as the latter parameter.
The provided code then forwards the processing to
\textit{main} or \textit{dest} as described in \secref{sec:forward}.

%%%%%%%%%%%%%%%%%%%%%%%%%%%%%%%%%%%%%%%%%%%%%%%%%%%%%%%%%%%%%%%%%%%%%%%%%%%%%%%%
\subsection{Include by Input}
\label{sec:input}

Including child documents by |\include| has some restrictions by design.
Most notably, the content of a child document always occupies
its own set of pages; pages cannot be shared between child documents.
Usually, this behaviour makes perfect sense
because each child document contain an essential part of the document.
However, in some situations it may be desirable to compose
a document from a collection of parts
without having mandatory page breaks between then.
For this case, the package
provides a mechanism to include parts
by |\input| which can also be processed individually.
However, by construction this mechanism
requires manual handling of the content to be output.

%%%%%%%%%%%%%%%%%%%%%%%%%%%%%%%%%%%%%%%%
\DescribeMacro{\ifchilddocmanual}
The main file should be prepared as usual, see \secref{sec:include}.
However, the document body must make a distinction
between processing of an individual part and of the main document, e.g.:
%
\begin{center}
\begin{tabular}{l}
|\ifchilddocmanual|\\
|\input{\childdocname}|\\
|\||else|\\
\textit{document body with }|\input{|\textit{part}|}|\\
|\||fi|
\end{tabular}
\end{center}
%
The conditional |\ifchilddocmanual| is true whenever
a part to be included by |\input| is being compiled,
and the name of the part is stored in |\childdocname|.

%%%%%%%%%%%%%%%%%%%%%%%%%%%%%%%%%%%%%%%%
\DescribeMacro{\childdocby}
Each part to be included by |\input| should start with:
%
\begin{center}
\begin{tabular}{l}
|\input{childdoc.def}|\\
|\childdocby{|\textit{main}|}|\\
\end{tabular}
\end{center}
%
The directive |\childdocby| is similar to |\childdocof|
described in \secref{sec:include},
but the subsequent selection of content must be done manually.
To that end, both |\ifchilddoc| and |\ifchilddocmanual|
will be true upon processing of a part,
and the name of the part is stored in |\childdocname|.
Note that |\jobname| will be set to the filename of the current part
so that each part receives an individual |.aux| file
that does not interfere with the |.aux| file(s) of the main document.
This behaviour can be altered by the alternative form
|\childdocby[*]{|\textit{main}|}| (with a non-empty optional argument)
which uses the |.aux| file of the main document
by setting |\jobname| to \textit{main}.

%%%%%%%%%%%%%%%%%%%%%%%%%%%%%%%%%%%%%%%%%%%%%%%%%%%%%%%%%%%%%%%%%%%%%%%%%%%%%%%%
\subsection{Driver Development}
\label{sec:driver}

The \textsf{childdoc} mechanism can also be use for the development
of definition files such as \LaTeX{} styles or classes.
This case differs from the above setup with multiple parts
included by |\include| in that no |\includeonly| should be invoked.
This can be achieved by starting the include file
(before |\ProvidesPackage|) with:
%
\begin{center}
\begin{tabular}{l}
|\input{childdoc.def}|\\
|\childdocforward{|\textit{main}|}|\\
\end{tabular}
\end{center}
%
or alternatively with:
%
\begin{center}
\begin{tabular}{l}
|\input{childdoc.def}|\\
|\childdocby{|\textit{main}|}|\\
\end{tabular}
\end{center}
%
Both forms have slightly different effects as described above.
The main file is prepared as usual, see \secref{sec:include}.

%%%%%%%%%%%%%%%%%%%%%%%%%%%%%%%%%%%%%%%%%%%%%%%%%%%%%%%%%%%%%%%%%%%%%%%%%%%%%%%%
\subsection{Legacy Detection}
\label{sec:detection}

The directive |\childdocmain| in the main file can detect
whether the complete document or merely a child is to be compiled
even without using the directive |\childdocof|.
This method is deprecated because it is less robust
and there is no compelling reason to use it;
it is merely provided for backward compatibility
and it may be removed in future versions.

If the detection mechanism is to be used,
it is mandatory to correctly specify
the filename of the main file as the argument of |\childdocmain|:
%
\begin{center}
\begin{tabular}{l}
|\input{childdoc.def}|\\
|\childdocmain{|\textit{main}|}|\\
\end{tabular}
\end{center}
%
If |\jobname| does not match the argument \textit{main} of |\childdocmain|,
it is assumed that |\jobname| points to the child file to be compiled.
When using |\childdocmain| with the main file specified as argument,
it suffices to start a child file
with just |\input{|\textit{main}|}|
without loading of the package and using |\childdocof|.
If instead all processing is done
with the appropriate \textsf{childdoc} directives,
the argument of \textit{main} of |\childdocmain| can be empty.

An alternative version of the command line processing described
in \secref{sec:commandline} using the detection mechanism reads:
%
\begin{center}
|... -jobname "|\textit{target}|" "|[\textit{flags}]%
[|\def\jobname{|\textit{dest}|}|]|\input{|\textit{main}|}"|
\end{center}

%%%%%%%%%%%%%%%%%%%%%%%%%%%%%%%%%%%%%%%%%%%%%%%%%%%%%%%%%%%%%%%%%%%%%%%%%%%%%%%%
\subsection{Manual Code}
\label{sec:manual}

In case one cannot be certain whether the definitions file |childdoc.def|
is installed on the target \TeX{} distribution
and one prefers not to ship it,
it is conceivable to paste a few relevant commands into the sources.

To that end, drop all statements |\input{childdoc.def}|
and perform the replacements as outlined below.
Instead of |\childdocmain{|\textit{main}|}| add the following code
to the top of the main file:
%
\begin{center}
\begin{tabular}{l}
|\||ifdefined\childdocname\endinput\||fi\newif\ifchilddoc|\\
|\edef\childdocname{\scantokens\expandafter{\jobname\noexpand}}|\\
|\def\childdocmain{|\textit{main}|}\||ifx\childdocmain\childdocname\||else|\\
|\childdoctrue\includeonly{\childdocname}\let\jobname\childdocmain\||fi|\\
\end{tabular}
\end{center}
%
Instead of |\childdocof{|\textit{main}|}| just include the main file
at the top of each child file:
%
\begin{center}
|\input{|\textit{main}|}|
\end{center}
%
A simple redirection |\childdocforward{|\textit{dest}|}| is achieved by:
%
\begin{center}
|\def\jobname{|\textit{dest}|}\input{\jobname}|
\end{center}
%
The redirection with prefix
|\childdocforwardprefix[|\textit{prefix}|]{|\textit{dest}|}|
is accomplished by:
%
\begin{center}
\begin{tabular}{l}
|{\edef\jobname{\scantokens\expandafter{\jobname\noexpand}}|\\
|\def\redirectjob |\textit{prefix}|#1~~~{\gdef\jobname{|\textit{dest}|#1}}|\\
|\expandafter\redirectjob\jobname~~~}\input{\jobname}|
\end{tabular}
\end{center}

In an alternative approach,
child documents can be compiled by a specific command line
without additional code or specific definitions:
%
\begin{center}
|... -jobname "|\textit{target}|" "|[\textit{flags}]%
|\includeonly{|\textit{dest}|}\input{|\textit{main}|}"|
\end{center}
%

%%%%%%%%%%%%%%%%%%%%%%%%%%%%%%%%%%%%%%%%%%%%%%%%%%%%%%%%%%%%%%%%%%%%%%%%%%%%%%%%
%%%%%%%%%%%%%%%%%%%%%%%%%%%%%%%%%%%%%%%%%%%%%%%%%%%%%%%%%%%%%%%%%%%%%%%%%%%%%%%%
\section{Information}

%%%%%%%%%%%%%%%%%%%%%%%%%%%%%%%%%%%%%%%%%%%%%%%%%%%%%%%%%%%%%%%%%%%%%%%%%%%%%%%%
\subsection{Copyright}

Copyright \copyright{} 2017--2018 Niklas Beisert

This work may be distributed and/or modified under the
conditions of the \LaTeX{} Project Public License, either version 1.3
of this license or (at your option) any later version.
The latest version of this license is in
  \url{http://www.latex-project.org/lppl.txt}
and version 1.3 or later is part of all distributions of \LaTeX{}
version 2005/12/01 or later.

This work has the LPPL maintenance status `maintained'.

The Current Maintainer of this work is Niklas Beisert.

This work consists of the files |README.txt|, |childdoc.ins| and |childdoc.dtx|
as well as the derived files |childdoc.def|, |cdocsamp.tex|
with |cdocsch1.tex|, |cdocsch2.tex|, |cdocspt3.tex|, |cdocspt4.tex|,
|cdocsdrf.tex|, |cdocsfn1.tex|, |cdocsfn2.tex|
as well as |childdoc.pdf|.

%%%%%%%%%%%%%%%%%%%%%%%%%%%%%%%%%%%%%%%%%%%%%%%%%%%%%%%%%%%%%%%%%%%%%%%%%%%%%%%%
\subsection{Files and Installation}

The package consists of the files:
%
\begin{center}
\begin{tabular}{ll}
    |README.txt|   & readme file \\
    |childdoc.ins| & installation file \\
    |childdoc.dtx| & source file \\
    |childdoc.def| & definition file \\
    |cdocsamp.tex| & sample main file \\
    |cdocsch1.tex| & sample include file \\
    |cdocsch2.tex| & sample include file \\
    |cdocspt3.tex| & sample part file \\
    |cdocspt4.tex| & sample part file \\
    |cdocsdrf.tex| & sample redirection file \\
    |cdocsfn1.tex| & sample redirection file \\
    |cdocsfn2.tex| & sample redirection file \\
    |childdoc.pdf| & manual
\end{tabular}
\end{center}
%
The distribution consists of the files
|README.txt|, |childdoc.ins| and |childdoc.dtx|.
%
\begin{itemize}
\item
Run (pdf)\LaTeX{} on |childdoc.dtx|
to compile the manual |childdoc.pdf| (this file).
\item
Run \LaTeX{} on |childdoc.ins| to create the definitions file |childdoc.def|
and the sample |cdocsamp.tex| with include files
|cdocsch1.tex|, |cdocsch2.tex|, |cdocspt3.tex|, |cdocspt4.tex|,
|cdocsdrf.tex|, |cdocsfn1.tex|, |cdocsfn2.tex|.
Then copy the file |childdoc.def| to an appropriate directory of your \LaTeX{}
distribution, e.g.\ \textit{texmf-root}|/tex/latex/childdoc|.
\end{itemize}

%%%%%%%%%%%%%%%%%%%%%%%%%%%%%%%%%%%%%%%%%%%%%%%%%%%%%%%%%%%%%%%%%%%%%%%%%%%%%%%%
\subsection{Related CTAN Packages}

There are several other packages which offer a similar functionality:
%
\begin{itemize}
\item
The packages
\href{http://ctan.org/pkg/docmute}{\textsf{docmute}},
\href{http://ctan.org/pkg/includex}{\textsf{includex}} and
\href{http://ctan.org/pkg/standalone}{\textsf{standalone}}
provide commands to include only the document body of
a child file thus allowing both files to be compiled individually.
\item
The packages \href{http://ctan.org/pkg/subdocs}{\textsf{subdocs}}
and \href{http://ctan.org/pkg/subfiles}{\textsf{subfiles}}
provide structures in which the main and child documents can be
encapsulated and allowing them to be compiled individually.
The inclusion mechanism is different from the conventional |\include|.
\item
The package \href{http://ctan.org/pkg/combine}{\textsf{combine}}
is an elaborate solution to combine several documents into one.
\end{itemize}
%
See also the CTAN topic \href{http://ctan.org/topic/subdocs}{\textsf{subdocs}}
for further related packages.
The present package differs from the above solutions in that
a document structure constructed with the conventional |\include| mechanism
just needs two extra commands at the top of every file
such that all constituent files can be compiled individually.

%%%%%%%%%%%%%%%%%%%%%%%%%%%%%%%%%%%%%%%%%%%%%%%%%%%%%%%%%%%%%%%%%%%%%%%%%%%%%%%%
%\subsection{Feature Suggestions}
%
%The following is a list of features which may be useful for future
%versions of this package:
%%
%\begin{itemize}
%\item
%\ldots
%\end{itemize}

%%%%%%%%%%%%%%%%%%%%%%%%%%%%%%%%%%%%%%%%%%%%%%%%%%%%%%%%%%%%%%%%%%%%%%%%%%%%%%%%
\subsection{Revision History}

%%%%%%%%%%%%%%%%%%%%%%%%%%%%%%%%%%%%%%%%
\paragraph{v2.0:} 2018/12/30

\begin{itemize}
\item
immediate forward processing
\item
added |\childdocby| mechanism
\item
manual restructured
\end{itemize}

%%%%%%%%%%%%%%%%%%%%%%%%%%%%%%%%%%%%%%%%
\paragraph{v1.6:} 2018/01/17

\begin{itemize}
\item
application for development of include files
\item
corrections to manual
\end{itemize}

%%%%%%%%%%%%%%%%%%%%%%%%%%%%%%%%%%%%%%%%
\paragraph{v1.5:} 2017/05/21

\begin{itemize}
\item
more complete structuring introduced
\item
|\childdocof| introduced
\item
|\childdoc| renamed to |\childdocmain|
\item
|\childredirect| renamed to |\childdocforward| and |\childdocforwardprefix|
and functionality expanded
\end{itemize}

%%%%%%%%%%%%%%%%%%%%%%%%%%%%%%%%%%%%%%%%
\paragraph{v1.0:} 2017/04/27

\begin{itemize}
\item
manual and install package
\item
first version published on CTAN
\end{itemize}

%%%%%%%%%%%%%%%%%%%%%%%%%%%%%%%%%%%%%%%%
\paragraph{v0.6:} 2017/04/26

\begin{itemize}
\item
redirection mechanism added
\end{itemize}

%%%%%%%%%%%%%%%%%%%%%%%%%%%%%%%%%%%%%%%%
\paragraph{v0.5:} 2017/04/26

\begin{itemize}
\item
functionality in definition file
\end{itemize}


%%%%%%%%%%%%%%%%%%%%%%%%%%%%%%%%%%%%%%%%%%%%%%%%%%%%%%%%%%%%%%%%%%%%%%%%%%%%%%%%
%%%%%%%%%%%%%%%%%%%%%%%%%%%%%%%%%%%%%%%%%%%%%%%%%%%%%%%%%%%%%%%%%%%%%%%%%%%%%%%%
%%%%%%%%%%%%%%%%%%%%%%%%%%%%%%%%%%%%%%%%%%%%%%%%%%%%%%%%%%%%%%%%%%%%%%%%%%%%%%%%
\appendix

\settowidth\MacroIndent{\rmfamily\scriptsize 000\ }

 \DocInput{childdoc.dtx}

\end{document}
%</driver>
% \fi
%
% %%%%%%%%%%%%%%%%%%%%%%%%%%%%%%%%%%%%%%%%%%%%%%%%%%%%%%%%%%%%%%%%%%%%%%%%%%%%%%
% %%%%%%%%%%%%%%%%%%%%%%%%%%%%%%%%%%%%%%%%%%%%%%%%%%%%%%%%%%%%%%%%%%%%%%%%%%%%%%
% \section{Sample}
%\iffalse
%<*samplemain>
%\fi
%
% The following presents a sample document
% with two chapters, two parts, a title page,
% a compile flag as well as three forwarding files to set the flag.
% It consists of eight |.tex| files:
% \begin{center}
% \begin{tabular}{ll}
% |cdocsamp.tex|&main file\\
% |cdocsch1.tex|&include file for chapter 1\\
% |cdocsch2.tex|&include file for chapter 2\\
% |cdocspt3.tex|&include file for part 3\\
% |cdocspt4.tex|&include file for part 4\\
% |cdocsdrf.tex|&forwarding file for main file in draft mode\\
% |cdocsfi1.tex|&forwarding file for final version of chapter 1\\
% |cdocsfi2.tex|&forwarding file for final version of chapter 2\\
% \end{tabular}
% \end{center}
% Each of the eight files can be compiled directly by the \LaTeX{} compiler.
%
% %%%%%%%%%%%%%%%%%%%%%%%%%%%%%%%%%%%%%%
% \paragraph{Main File.}
%
% The main file is called |cdocsamp.tex|.
%
% Load the \textsf{childdoc} definitions and
% declare the filename for the main document:
%    \begin{macrocode}
\input{childdoc.def}
\childdocmain{}
%    \end{macrocode}

% Optional override for |\version| flag:
%    \begin{macrocode}
%%\ifchilddoc\else\providecommand{\version}{draft}\fi
%    \end{macrocode}

% Define the default values for the |\version| flag
% (|final| for the main file and |draft| for childs):
%    \begin{macrocode}
\ifchilddoc
\providecommand{\version}{draft}
\else
\providecommand{\version}{final}
\fi
%    \end{macrocode}

% Load the standard document class:
%    \begin{macrocode}
\documentclass[12pt]{article}
%    \end{macrocode}

% Start the document body:
%    \begin{macrocode}
\begin{document}
%    \end{macrocode}

% Declare a title page.
% Print title, part of document being processed and version flag:
%    \begin{macrocode}
\addtocounter{page}{-1}
\begin{center}
{\LARGE\bfseries{}childdoc example\par}
\vspace{1cm}
\ifchilddoc
\ifchilddocmanual part\else chapter\fi:
`\childdocname' of `\childdocjob'\par
\else
main document: `\childdocjob'\par
\fi
version: \version\par
\end{center}
\newpage
%    \end{macrocode}

% Manually include selected file,
% otherwise process as usual:
%    \begin{macrocode}
\ifchilddocmanual
\section*{part `\childdocname'}
\input{\childdocname}
\else
%    \end{macrocode}

% Include the two chapters:
%    \begin{macrocode}
\include{cdocsch1}
\include{cdocsch2}
%    \end{macrocode}

% Include the two parts unless only chapters should be displayed:
%    \begin{macrocode}
\ifchilddoc\else
\section{part three}
\input{cdocspt3}
\section{part four}
\input{cdocspt4}
\fi
%    \end{macrocode}

% Process as usual until here:
%    \begin{macrocode}
\fi
%    \end{macrocode}

% End of document body:
%    \begin{macrocode}
\end{document}
%    \end{macrocode}
%\iffalse
%</samplemain>
%\fi
%
% %%%%%%%%%%%%%%%%%%%%%%%%%%%%%%%%%%%%%%
% \paragraph{Chapter Include Files.}
%
% The include files are called |cdocsch1.tex| and |cdocsch2.tex|.
%
%\iffalse
%<*samplechap1|samplechap2>
%\fi

% Optional override for |\version| flag:
%    \begin{macrocode}
%%\providecommand{\version}{final}
%    \end{macrocode}

% Include the main document:
%    \begin{macrocode}
\input{childdoc.def}
\childdocof{cdocsamp}
%    \end{macrocode}

%\iffalse
%</samplechap1|samplechap2>
%\fi
%
%\iffalse
%<*samplechap1>
%\fi
% Some text for chapter 1:
%    \begin{macrocode}
\section{one}
some text in chapter one
%    \end{macrocode}

%\iffalse
%</samplechap1>
%\fi
% Some text for chapter 2:
%\iffalse
%<*samplechap2>
%\fi
%    \begin{macrocode}
\section{two}
more text in chapter two
%    \end{macrocode}

%\iffalse
%</samplechap2>
%\fi
%
% %%%%%%%%%%%%%%%%%%%%%%%%%%%%%%%%%%%%%%
% \paragraph{Part Include Files.}
%
% The include files are called |cdocspt3.tex| and |cdocspt4.tex|.
%
%\iffalse
%<*samplepart3|samplepart4>
%\fi

% Optional override for |\version| flag:
%    \begin{macrocode}
%%\providecommand{\version}{final}
%    \end{macrocode}

% Include the main document:
%    \begin{macrocode}
\input{childdoc.def}
\childdocby{cdocsamp}
%    \end{macrocode}

%\iffalse
%</samplepart3|samplepart4>
%\fi
%
%\iffalse
%<*samplepart3>
%\fi
% Some text for part 3:
%    \begin{macrocode}
some text in part three
%    \end{macrocode}

%\iffalse
%</samplepart3>
%\fi
% Some text for part 4:
%\iffalse
%<*samplepart4>
%\fi
%    \begin{macrocode}
more text in part four
%    \end{macrocode}

%\iffalse
%</samplepart4>
%\fi
%
% %%%%%%%%%%%%%%%%%%%%%%%%%%%%%%%%%%%%%%
% \paragraph{Forwarding for a Complete Draft.}
%
% The following forwarding file |cdocsdrf.tex|
% compiles the main document in draft mode:
%\iffalse
%<*sampledraft>
%\fi
%    \begin{macrocode}
\def\version{draft}
\input{childdoc.def}
\childdocforward{cdocsamp}
%    \end{macrocode}

%\iffalse
%</sampledraft>
%\fi
%
% %%%%%%%%%%%%%%%%%%%%%%%%%%%%%%%%%%%%%%
% \paragraph{Forwarding for Final Version of the Chapters.}
%
% The following forwarding files |cdocsfn1.tex| and |cdocsfn2.tex|
% (with identical content)
% compile the final versions of the child documents
% |cdocsch1.tex| and |cdocsch2.tex|, respectively:
%\iffalse
%<*samplefinal>
%\fi
%    \begin{macrocode}
\def\version{final}
\input{childdoc.def}
\childdocforwardprefix[cdocsamp]{cdocsfn}{cdocsch}
%    \end{macrocode}

%\iffalse
%</samplefinal>
%\fi
%
% %%%%%%%%%%%%%%%%%%%%%%%%%%%%%%%%%%%%%%
% \paragraph{Command Line Processing.}
%
% The following three command lines generate the output files
% |cdocscld|, |cdocscl1| and |cdocscl2|
% which should be identical to
% |cdocsdrf|, |cdocsch1| and |cdocsfn2|, respectively:
% \begin{center}
% \begin{tabular}{l}
% |latex -jobname cdocscld \|\\
% |  "\def\version{draft}\input{childdoc.def}\childdocforward{cdocsamp}"|\\
% |latex -jobname cdocscl1 \|\\
% |  "\input{childdoc.def}\childdocforward[cdocsamp]{cdocsch1}"|\\
% |latex -jobname cdocscl2 \|\\
% |  "\def\version{final}\input{childdoc.def}\childdocforward{cdocsch2}"|
% \end{tabular}
% \end{center}
% Note that the trailing backslash on each first line
% merely continues the input to the second line
% (for convenient cut ant paste).
% Furthermore, the command |latex| can be replaced by any
% of its alternative versions such as |pdflatex|.
%
% %%%%%%%%%%%%%%%%%%%%%%%%%%%%%%%%%%%%%%%%%%%%%%%%%%%%%%%%%%%%%%%%%%%%%%%%%%%%%%
% %%%%%%%%%%%%%%%%%%%%%%%%%%%%%%%%%%%%%%%%%%%%%%%%%%%%%%%%%%%%%%%%%%%%%%%%%%%%%%
% \section{Implementation}
%\iffalse
%<*package>
%\fi
%
% This section describes the definitions file |childdoc.def|.

% The definitions cannot be loaded using |\usepackage| or |\RequirePackage|
% which has a mechanism to prevent loading a style file more than once.
% When loading the definitions by means of |\input|
% multiple instances have to be prevented manually:
%\iffalse
%This code needs to be before the `\ProvidesFile' directive
%which is defined at the beginning of this file.
%Therefore it is also placed there and commented out here.
%</package>
%<*discard>
%\fi
%    \begin{macrocode}
\ifdefined\childdocmain\endinput\fi
%    \end{macrocode}
%\iffalse
%</discard>
%<*package>
%\fi
%
% \macro{\ifchilddoc}
% \macro{\ifchilddocmanual}
% The conditional |\ifchilddoc| tells whether a
% child (true) or main (false) document is being compiled.
% The conditional |\ifchilddocmanual| tells whether
% the |\includeonly| mechanism is used (false) or
% the selection of child files must be performed manually (true).
% The definitions initialise to false:
%    \begin{macrocode}
\newif\ifchilddoc
\newif\ifchilddocmanual
%    \end{macrocode}

% \macro{\childdocname}
% \macro{\childdocjob}
% The macro |\childdocname| stores the name of the main document
% to be compiled. The macro |\childdocjob| stores the name of
% the document on which the \LaTeX{} compiler was originally invoked.
% The content of |\jobname| cannot be compared
% to filenames specified in the source due to different catcodes.
% The following code rescans |\jobname|, stores the result
% in |\childdocname| and saves a copy in |\childdocjob|:
%    \begin{macrocode}
\edef\childdocname{\scantokens\expandafter{\jobname\noexpand}}
\let\childdocjob\childdocname
%    \end{macrocode}

% \macro{\childdocdisable}
% The macro |\childdocdisable| prevents the main file
% from being processed more than once.
% At this stage, the main document command |\childdocmain|
% is assumed to be called once again where it should do nothing.
% Any subsequent call to it should prevent
% a secondary processing of the main document
% It overwrites the forwarding commands
% |\childdocof| and |\childdocforward|
% with empty macros to prevent further inclusions of the main document:
%    \begin{macrocode}
\newcommand{\childdocdisable}
{
  \renewcommand{\childdocmain}[1]{\renewcommand{\childdocmain}[1]{\endinput}}
  \renewcommand{\childdocof}[1]{}
  \renewcommand{\childdocby}[2][]{}
  \renewcommand{\childdocforward}[2][]{}
  \renewcommand{\childdocdisable}{}
}
%    \end{macrocode}

% \macro{\childdocmain}
% The macro |\childdocmain| is to be called at the top of the main file
% with nothing or the main filename (without extension) as argument.
% First, it breaks loops.
% If the argument is not empty and does not match |\childdocname|
% (which is set by the first inclusion of |childdoc.def|),
% |\ifchilddoc| is set to true, |\includeonly| is applied to the child file
% and |\jobname| is set to the main file
% (for proper handling of |.aux| files):
%    \begin{macrocode}
\newcommand{\childdocmain}[1]
{
  \childdocdisable\childdocmain{}
  \if?#1?\else
    \begingroup
      \def\childdoctmp{#1}
      \ifx\childdoctmp\childdocname
        \def\childdoctmp{}
      \else
        \def\childdoctmp
        {
          \childdoctrue
          \includeonly{\childdocname}
          \def\childdocjob{#1}
          \def\jobname{#1}
        }
      \fi
      \expandafter
    \endgroup
    \childdoctmp
  \fi
}
%    \end{macrocode}

% \macro{\childdocof}
% The command |\childdocof| redirects
% compilation to the main file |#1|.
%    \begin{macrocode}
\newcommand{\childdocof}[1]
{
  \childdocdisable
  \childdoctrue
  \includeonly{\childdocname}
  \def\jobname{#1}
  \def\childdocjob{#1}
  \input{#1}
}
%    \end{macrocode}

% \macro{\childdocby}
% The command |\childdocby| ....
%    \begin{macrocode}
\newcommand{\childdocby}[2][]
{
  \childdocdisable
  \childdoctrue
  \childdocmanualtrue
  \if?#1?\else
    \def\jobname{#2}
  \fi
  \def\childdocjob{#2}
  \input{#2}
  \endinput
}
%    \end{macrocode}

% \macro{\childdocforward}
% The command |\childdocforward| redirects
% compilation to the main file or
% (if the optional argument is given) a child file.
% Parameters are set as if the main file
% or a child file starting with |\childdocof| was compiled.
% Then compilation is handed over to the main file:
%    \begin{macrocode}
\newcommand{\childdocforward}[2][]
{
  \begingroup
    \if?#1?
      \def\childdoctmp
      {
        \def\childdocname{#2}
        \def\childdocjob{#2}
        \def\jobname{#2}
        \input{#2}
        \endinput
      }
    \else
      \def\childdoctmp
      {
        \childdocdisable
        \def\childdocname{#2}
        \childdoctrue
        \includeonly{#2}
        \def\childdocjob{#1}
        \def\jobname{#1}
        \input{#1}
        \endinput
      }
    \fi
    \expandafter
  \endgroup
  \childdoctmp
}
%    \end{macrocode}

% \macro{\childdocforwardprefix}
% The command |\childdocforwardprefix| redirects
% compilation to the main or a child file by means of a pattern.
% The prefix |#1| in the current filename is replaced by |#2|
% and the suffix of the current filename is kept
% (it is assumed that the filename does not contain the substring `|~~~|'
% which is used as a delimiter).
% Compilation is handed over to the new file by |\childdocforward|:
%    \begin{macrocode}
\newcommand{\childdocforwardprefix}[3][]
{
  \begingroup
    \def\childdocextract #2##1~~~{\def\childdoctmp{\childdocforward[#1]{#3##1}}}
    \expandafter\childdocextract\childdocname~~~
    \expandafter
  \endgroup
  \childdoctmp
}
%    \end{macrocode}

% \macro{\childdoc}
% The deprecated macro |\childdoc| is a legacy version of |\childdocmain|:
%    \begin{macrocode}
\newcommand{\childdoc}{\childdocmain}
%    \end{macrocode}

% \macro{\childdocredirect}
% The deprecated macro |\childdocredirect| is a legacy version
% of |\childdocforward| and |\childdocforwardprefix|:
%    \begin{macrocode}
\newcommand{\childdocredirect}[2][]
{
  \begingroup
    \if?#1?
      \def\childdoctmp{\childdocforward{#2}}
    \else
      \def\childdoctmp{\childdocforwardprefix{#1}{#2}}
    \fi
    \expandafter
  \endgroup
  \childdoctmp
}
%    \end{macrocode}

%\iffalse
%</package>
%\fi
%
\endinput
\childdocforward{cdocsch2}"|
% \end{tabular}
% \end{center}
% Note that the trailing backslash on each first line
% merely continues the input to the second line
% (for convenient cut ant paste).
% Furthermore, the command |latex| can be replaced by any
% of its alternative versions such as |pdflatex|.
%
% %%%%%%%%%%%%%%%%%%%%%%%%%%%%%%%%%%%%%%%%%%%%%%%%%%%%%%%%%%%%%%%%%%%%%%%%%%%%%%
% %%%%%%%%%%%%%%%%%%%%%%%%%%%%%%%%%%%%%%%%%%%%%%%%%%%%%%%%%%%%%%%%%%%%%%%%%%%%%%
% \section{Implementation}
%\iffalse
%<*package>
%\fi
%
% This section describes the definitions file |childdoc.def|.

% The definitions cannot be loaded using |\usepackage| or |\RequirePackage|
% which has a mechanism to prevent loading a style file more than once.
% When loading the definitions by means of |\input|
% multiple instances have to be prevented manually:
%\iffalse
%This code needs to be before the `\ProvidesFile' directive
%which is defined at the beginning of this file.
%Therefore it is also placed there and commented out here.
%</package>
%<*discard>
%\fi
%    \begin{macrocode}
\ifdefined\childdocmain\endinput\fi
%    \end{macrocode}
%\iffalse
%</discard>
%<*package>
%\fi
%
% \macro{\ifchilddoc}
% \macro{\ifchilddocmanual}
% The conditional |\ifchilddoc| tells whether a
% child (true) or main (false) document is being compiled.
% The conditional |\ifchilddocmanual| tells whether
% the |\includeonly| mechanism is used (false) or
% the selection of child files must be performed manually (true).
% The definitions initialise to false:
%    \begin{macrocode}
\newif\ifchilddoc
\newif\ifchilddocmanual
%    \end{macrocode}

% \macro{\childdocname}
% \macro{\childdocjob}
% The macro |\childdocname| stores the name of the main document
% to be compiled. The macro |\childdocjob| stores the name of
% the document on which the \LaTeX{} compiler was originally invoked.
% The content of |\jobname| cannot be compared
% to filenames specified in the source due to different catcodes.
% The following code rescans |\jobname|, stores the result
% in |\childdocname| and saves a copy in |\childdocjob|:
%    \begin{macrocode}
\edef\childdocname{\scantokens\expandafter{\jobname\noexpand}}
\let\childdocjob\childdocname
%    \end{macrocode}

% \macro{\childdocdisable}
% The macro |\childdocdisable| prevents the main file
% from being processed more than once.
% At this stage, the main document command |\childdocmain|
% is assumed to be called once again where it should do nothing.
% Any subsequent call to it should prevent
% a secondary processing of the main document
% It overwrites the forwarding commands
% |\childdocof| and |\childdocforward|
% with empty macros to prevent further inclusions of the main document:
%    \begin{macrocode}
\newcommand{\childdocdisable}
{
  \renewcommand{\childdocmain}[1]{\renewcommand{\childdocmain}[1]{\endinput}}
  \renewcommand{\childdocof}[1]{}
  \renewcommand{\childdocby}[2][]{}
  \renewcommand{\childdocforward}[2][]{}
  \renewcommand{\childdocdisable}{}
}
%    \end{macrocode}

% \macro{\childdocmain}
% The macro |\childdocmain| is to be called at the top of the main file
% with nothing or the main filename (without extension) as argument.
% First, it breaks loops.
% If the argument is not empty and does not match |\childdocname|
% (which is set by the first inclusion of |childdoc.def|),
% |\ifchilddoc| is set to true, |\includeonly| is applied to the child file
% and |\jobname| is set to the main file
% (for proper handling of |.aux| files):
%    \begin{macrocode}
\newcommand{\childdocmain}[1]
{
  \childdocdisable\childdocmain{}
  \if?#1?\else
    \begingroup
      \def\childdoctmp{#1}
      \ifx\childdoctmp\childdocname
        \def\childdoctmp{}
      \else
        \def\childdoctmp
        {
          \childdoctrue
          \includeonly{\childdocname}
          \def\childdocjob{#1}
          \def\jobname{#1}
        }
      \fi
      \expandafter
    \endgroup
    \childdoctmp
  \fi
}
%    \end{macrocode}

% \macro{\childdocof}
% The command |\childdocof| redirects
% compilation to the main file |#1|.
%    \begin{macrocode}
\newcommand{\childdocof}[1]
{
  \childdocdisable
  \childdoctrue
  \includeonly{\childdocname}
  \def\jobname{#1}
  \def\childdocjob{#1}
  \input{#1}
}
%    \end{macrocode}

% \macro{\childdocby}
% The command |\childdocby| ....
%    \begin{macrocode}
\newcommand{\childdocby}[2][]
{
  \childdocdisable
  \childdoctrue
  \childdocmanualtrue
  \if?#1?\else
    \def\jobname{#2}
  \fi
  \def\childdocjob{#2}
  \input{#2}
  \endinput
}
%    \end{macrocode}

% \macro{\childdocforward}
% The command |\childdocforward| redirects
% compilation to the main file or
% (if the optional argument is given) a child file.
% Parameters are set as if the main file
% or a child file starting with |\childdocof| was compiled.
% Then compilation is handed over to the main file:
%    \begin{macrocode}
\newcommand{\childdocforward}[2][]
{
  \begingroup
    \if?#1?
      \def\childdoctmp
      {
        \def\childdocname{#2}
        \def\childdocjob{#2}
        \def\jobname{#2}
        \input{#2}
        \endinput
      }
    \else
      \def\childdoctmp
      {
        \childdocdisable
        \def\childdocname{#2}
        \childdoctrue
        \includeonly{#2}
        \def\childdocjob{#1}
        \def\jobname{#1}
        \input{#1}
        \endinput
      }
    \fi
    \expandafter
  \endgroup
  \childdoctmp
}
%    \end{macrocode}

% \macro{\childdocforwardprefix}
% The command |\childdocforwardprefix| redirects
% compilation to the main or a child file by means of a pattern.
% The prefix |#1| in the current filename is replaced by |#2|
% and the suffix of the current filename is kept
% (it is assumed that the filename does not contain the substring `|~~~|'
% which is used as a delimiter).
% Compilation is handed over to the new file by |\childdocforward|:
%    \begin{macrocode}
\newcommand{\childdocforwardprefix}[3][]
{
  \begingroup
    \def\childdocextract #2##1~~~{\def\childdoctmp{\childdocforward[#1]{#3##1}}}
    \expandafter\childdocextract\childdocname~~~
    \expandafter
  \endgroup
  \childdoctmp
}
%    \end{macrocode}

% \macro{\childdoc}
% The deprecated macro |\childdoc| is a legacy version of |\childdocmain|:
%    \begin{macrocode}
\newcommand{\childdoc}{\childdocmain}
%    \end{macrocode}

% \macro{\childdocredirect}
% The deprecated macro |\childdocredirect| is a legacy version
% of |\childdocforward| and |\childdocforwardprefix|:
%    \begin{macrocode}
\newcommand{\childdocredirect}[2][]
{
  \begingroup
    \if?#1?
      \def\childdoctmp{\childdocforward{#2}}
    \else
      \def\childdoctmp{\childdocforwardprefix{#1}{#2}}
    \fi
    \expandafter
  \endgroup
  \childdoctmp
}
%    \end{macrocode}

%\iffalse
%</package>
%\fi
%
\endinput

\childdocforwardprefix[cdocsamp]{cdocsfn}{cdocsch}
%    \end{macrocode}

%\iffalse
%</samplefinal>
%\fi
%
% %%%%%%%%%%%%%%%%%%%%%%%%%%%%%%%%%%%%%%
% \paragraph{Command Line Processing.}
%
% The following three command lines generate the output files
% |cdocscld|, |cdocscl1| and |cdocscl2|
% which should be identical to
% |cdocsdrf|, |cdocsch1| and |cdocsfn2|, respectively:
% \begin{center}
% \begin{tabular}{l}
% |latex -jobname cdocscld \|\\
% |  "\def\version{draft}% \iffalse
%
% childdoc.dtx Copyright (C) 2017-2018 Niklas Beisert
%
% This work may be distributed and/or modified under the
% conditions of the LaTeX Project Public License, either version 1.3
% of this license or (at your option) any later version.
% The latest version of this license is in
%   http://www.latex-project.org/lppl.txt
% and version 1.3 or later is part of all distributions of LaTeX
% version 2005/12/01 or later.
%
% This work has the LPPL maintenance status `maintained'.
%
% The Current Maintainer of this work is Niklas Beisert.
%
% This work consists of the files childdoc.dtx and childdoc.ins
% and the derived files childdoc.def and cdocsamp.tex with
% cdocsch1.tex, cdocsch2.tex, cdocsdrf.tex, cdocsfn1.tex, cdocsfn2.tex.
%
%<package>\ifdefined\childdocmain\endinput\fi
%<package>\ProvidesFile{childdoc.def}[2018/12/30 v2.0 child document driver]
%<samplemain>\ProvidesFile{cdocsamp.tex}[2018/12/30 v2.0 sample for childdoc]
%<*driver>
%\ProvidesFile{childdoc.drv}[2018/12/30 v2.0 childdoc reference manual file]
\PassOptionsToClass{10pt,a4paper}{article}
\documentclass{ltxdoc}

\usepackage[margin=35mm]{geometry}
\usepackage{hyperref}
\usepackage{hyperxmp}
\usepackage[usenames]{color}

\hypersetup{colorlinks=true}
\hypersetup{pdfstartview=FitH}
\hypersetup{pdfpagemode=UseNone}
\hypersetup{pdfsource={}}
\hypersetup{pdflang={en-UK}}
\hypersetup{pdfcopyright={Copyright 2017-2018 Niklas Beisert.
  This work may be distributed and/or modified under the
  conditions of the LaTeX Project Public License, either version 1.3
  of this license or (at your option) any later version.}}
\hypersetup{pdflicenseurl={http://www.latex-project.org/lppl.txt}}
\hypersetup{pdfcontactaddress={ETH Zurich, ITP, HIT K,
  Wolfgang-Pauli-Strasse 27}}
\hypersetup{pdfcontactpostcode={8093}}
\hypersetup{pdfcontactcity={Zurich}}
\hypersetup{pdfcontactcountry={Switzerland}}
\hypersetup{pdfcontactemail={nbeisert@itp.phys.ethz.ch}}
\hypersetup{pdfcontacturl={http://people.phys.ethz.ch/\xmptilde nbeisert/}}

\newcommand{\secref}[1]{\hyperref[#1]{section \ref*{#1}}}

\parskip1ex
\parindent0pt
\let\olditemize\itemize
\def\itemize{\olditemize\parskip0pt}

\begin{document}

\title{The \textsf{childdoc} Package}
\hypersetup{pdftitle={The childdoc Package}}
\author{Niklas Beisert\\[2ex]
  Institut f\"ur Theoretische Physik\\
  Eidgen\"ossische Technische Hochschule Z\"urich\\
  Wolfgang-Pauli-Strasse 27, 8093 Z\"urich, Switzerland\\[1ex]
  \href{mailto:nbeisert@itp.phys.ethz.ch}
  {\texttt{nbeisert@itp.phys.ethz.ch}}}
\hypersetup{pdfauthor={Niklas Beisert}}
\hypersetup{pdfsubject={Manual for the LaTeX2e Package childdoc}}
\date{30 December 2018, \textsf{v2.0}}
\maketitle

\begin{abstract}\noindent
\textsf{childdoc} is a \LaTeXe{} package
that enables the direct compilation
of document sections included by |\include|
to individual files.
\end{abstract}

\begingroup
\parskip0ex
\tableofcontents
\endgroup

%%%%%%%%%%%%%%%%%%%%%%%%%%%%%%%%%%%%%%%%%%%%%%%%%%%%%%%%%%%%%%%%%%%%%%%%%%%%%%%%
%%%%%%%%%%%%%%%%%%%%%%%%%%%%%%%%%%%%%%%%%%%%%%%%%%%%%%%%%%%%%%%%%%%%%%%%%%%%%%%%
\section{Introduction}

\LaTeX{} provides a mechanism to structure a large document (such as a book)
into a main file and several child files (containing the chapters)
using the |\include| command.
This mechanism is beneficial for documents
which span hundreds of pages in order to
make the source file(s) more manageable.
Moreover, compilation can be restricted to
selected child files by means of the |\includeonly| command.
The latter feature can be used to reduce the compilation time while editing
(this was significantly more useful in the earlier days of \LaTeX{})
or to generate a smaller document which is easier to navigate.
Another application of |\includeonly| is to generate
documents consisting of selected parts of the complete document.

However, there are a few drawbacks of the plain |\include| mechanism:
\begin{itemize}
\item
The child files cannot be compiled on their own,
they can only be compiled via the main file.
A naive editing environment
(such as a text editor with an option
to have the current file processed by \LaTeX)
may require one to switch to the main file before compiling;
attempting to compile the child file produces errors.
\item
The main file must be modified (each time)
to adjust the |\includeonly| command
to the present needs. This easily leaves the main file in a messy state.
\item
The generated document will always carry the filename
of the main document. This is inconvenient if
several child files are to be compiled and
to be kept for distribution.
\end{itemize}

The present package provides a simple interface
to make child files individually compilable by \LaTeX{}.
Compiling a child file then has the same effect as compiling
the main file with an |\includeonly| command
to select the appropriate child.
Moreover the generated document will carry the name of the child
rather than the main file.
This resolves all three above issues.

This feature is meant to make the editing of books,
thesis documents and lecture notes somewhat more convenient.
However, the package can also be used efficiently for
composing a series of documents (such as exercise sheets)
which are typically distributed individually.
It then assists the author in generating the individual documents
(potentially in different versions)
as well as a document containing the collected series.
Another application is in developing style files
or other kinds of included material
where compilation of the style file could redirect
to a sample or test file.

%%%%%%%%%%%%%%%%%%%%%%%%%%%%%%%%%%%%%%%%%%%%%%%%%%%%%%%%%%%%%%%%%%%%%%%%%%%%%%%%
%%%%%%%%%%%%%%%%%%%%%%%%%%%%%%%%%%%%%%%%%%%%%%%%%%%%%%%%%%%%%%%%%%%%%%%%%%%%%%%%
\section{Usage}

First of all, the package \textsf{childdoc} is \emph{not} a standard
\LaTeXe{} |.sty| style file! Therefore it needs to be invoked in
a non-standard way.

%%%%%%%%%%%%%%%%%%%%%%%%%%%%%%%%%%%%%%%%%%%%%%%%%%%%%%%%%%%%%%%%%%%%%%%%%%%%%%%%
\subsection{Included Files}
\label{sec:include}

%%%%%%%%%%%%%%%%%%%%%%%%%%%%%%%%%%%%%%%%
\DescribeMacro{\childdocmain}
To use the package, add the commands
\begin{center}
\begin{tabular}{l}
|% \iffalse
%
% childdoc.dtx Copyright (C) 2017-2018 Niklas Beisert
%
% This work may be distributed and/or modified under the
% conditions of the LaTeX Project Public License, either version 1.3
% of this license or (at your option) any later version.
% The latest version of this license is in
%   http://www.latex-project.org/lppl.txt
% and version 1.3 or later is part of all distributions of LaTeX
% version 2005/12/01 or later.
%
% This work has the LPPL maintenance status `maintained'.
%
% The Current Maintainer of this work is Niklas Beisert.
%
% This work consists of the files childdoc.dtx and childdoc.ins
% and the derived files childdoc.def and cdocsamp.tex with
% cdocsch1.tex, cdocsch2.tex, cdocsdrf.tex, cdocsfn1.tex, cdocsfn2.tex.
%
%<package>\ifdefined\childdocmain\endinput\fi
%<package>\ProvidesFile{childdoc.def}[2018/12/30 v2.0 child document driver]
%<samplemain>\ProvidesFile{cdocsamp.tex}[2018/12/30 v2.0 sample for childdoc]
%<*driver>
%\ProvidesFile{childdoc.drv}[2018/12/30 v2.0 childdoc reference manual file]
\PassOptionsToClass{10pt,a4paper}{article}
\documentclass{ltxdoc}

\usepackage[margin=35mm]{geometry}
\usepackage{hyperref}
\usepackage{hyperxmp}
\usepackage[usenames]{color}

\hypersetup{colorlinks=true}
\hypersetup{pdfstartview=FitH}
\hypersetup{pdfpagemode=UseNone}
\hypersetup{pdfsource={}}
\hypersetup{pdflang={en-UK}}
\hypersetup{pdfcopyright={Copyright 2017-2018 Niklas Beisert.
  This work may be distributed and/or modified under the
  conditions of the LaTeX Project Public License, either version 1.3
  of this license or (at your option) any later version.}}
\hypersetup{pdflicenseurl={http://www.latex-project.org/lppl.txt}}
\hypersetup{pdfcontactaddress={ETH Zurich, ITP, HIT K,
  Wolfgang-Pauli-Strasse 27}}
\hypersetup{pdfcontactpostcode={8093}}
\hypersetup{pdfcontactcity={Zurich}}
\hypersetup{pdfcontactcountry={Switzerland}}
\hypersetup{pdfcontactemail={nbeisert@itp.phys.ethz.ch}}
\hypersetup{pdfcontacturl={http://people.phys.ethz.ch/\xmptilde nbeisert/}}

\newcommand{\secref}[1]{\hyperref[#1]{section \ref*{#1}}}

\parskip1ex
\parindent0pt
\let\olditemize\itemize
\def\itemize{\olditemize\parskip0pt}

\begin{document}

\title{The \textsf{childdoc} Package}
\hypersetup{pdftitle={The childdoc Package}}
\author{Niklas Beisert\\[2ex]
  Institut f\"ur Theoretische Physik\\
  Eidgen\"ossische Technische Hochschule Z\"urich\\
  Wolfgang-Pauli-Strasse 27, 8093 Z\"urich, Switzerland\\[1ex]
  \href{mailto:nbeisert@itp.phys.ethz.ch}
  {\texttt{nbeisert@itp.phys.ethz.ch}}}
\hypersetup{pdfauthor={Niklas Beisert}}
\hypersetup{pdfsubject={Manual for the LaTeX2e Package childdoc}}
\date{30 December 2018, \textsf{v2.0}}
\maketitle

\begin{abstract}\noindent
\textsf{childdoc} is a \LaTeXe{} package
that enables the direct compilation
of document sections included by |\include|
to individual files.
\end{abstract}

\begingroup
\parskip0ex
\tableofcontents
\endgroup

%%%%%%%%%%%%%%%%%%%%%%%%%%%%%%%%%%%%%%%%%%%%%%%%%%%%%%%%%%%%%%%%%%%%%%%%%%%%%%%%
%%%%%%%%%%%%%%%%%%%%%%%%%%%%%%%%%%%%%%%%%%%%%%%%%%%%%%%%%%%%%%%%%%%%%%%%%%%%%%%%
\section{Introduction}

\LaTeX{} provides a mechanism to structure a large document (such as a book)
into a main file and several child files (containing the chapters)
using the |\include| command.
This mechanism is beneficial for documents
which span hundreds of pages in order to
make the source file(s) more manageable.
Moreover, compilation can be restricted to
selected child files by means of the |\includeonly| command.
The latter feature can be used to reduce the compilation time while editing
(this was significantly more useful in the earlier days of \LaTeX{})
or to generate a smaller document which is easier to navigate.
Another application of |\includeonly| is to generate
documents consisting of selected parts of the complete document.

However, there are a few drawbacks of the plain |\include| mechanism:
\begin{itemize}
\item
The child files cannot be compiled on their own,
they can only be compiled via the main file.
A naive editing environment
(such as a text editor with an option
to have the current file processed by \LaTeX)
may require one to switch to the main file before compiling;
attempting to compile the child file produces errors.
\item
The main file must be modified (each time)
to adjust the |\includeonly| command
to the present needs. This easily leaves the main file in a messy state.
\item
The generated document will always carry the filename
of the main document. This is inconvenient if
several child files are to be compiled and
to be kept for distribution.
\end{itemize}

The present package provides a simple interface
to make child files individually compilable by \LaTeX{}.
Compiling a child file then has the same effect as compiling
the main file with an |\includeonly| command
to select the appropriate child.
Moreover the generated document will carry the name of the child
rather than the main file.
This resolves all three above issues.

This feature is meant to make the editing of books,
thesis documents and lecture notes somewhat more convenient.
However, the package can also be used efficiently for
composing a series of documents (such as exercise sheets)
which are typically distributed individually.
It then assists the author in generating the individual documents
(potentially in different versions)
as well as a document containing the collected series.
Another application is in developing style files
or other kinds of included material
where compilation of the style file could redirect
to a sample or test file.

%%%%%%%%%%%%%%%%%%%%%%%%%%%%%%%%%%%%%%%%%%%%%%%%%%%%%%%%%%%%%%%%%%%%%%%%%%%%%%%%
%%%%%%%%%%%%%%%%%%%%%%%%%%%%%%%%%%%%%%%%%%%%%%%%%%%%%%%%%%%%%%%%%%%%%%%%%%%%%%%%
\section{Usage}

First of all, the package \textsf{childdoc} is \emph{not} a standard
\LaTeXe{} |.sty| style file! Therefore it needs to be invoked in
a non-standard way.

%%%%%%%%%%%%%%%%%%%%%%%%%%%%%%%%%%%%%%%%%%%%%%%%%%%%%%%%%%%%%%%%%%%%%%%%%%%%%%%%
\subsection{Included Files}
\label{sec:include}

%%%%%%%%%%%%%%%%%%%%%%%%%%%%%%%%%%%%%%%%
\DescribeMacro{\childdocmain}
To use the package, add the commands
\begin{center}
\begin{tabular}{l}
|\input{childdoc.def}|\\
|\childdocmain{}|\\
\end{tabular}
\end{center}
at the very top of the main \LaTeX{} file,
in particular \emph{before} the |\documentclass| statement!
The argument of |\childdocmain| should be left empty
(but it must be present).

%%%%%%%%%%%%%%%%%%%%%%%%%%%%%%%%%%%%%%%%
\DescribeMacro{\childdocof}
Furthermore, add the commands
\begin{center}
\begin{tabular}{l}
|\input{childdoc.def}|\\
|\childdocof{|\textit{main}|}|\\
\end{tabular}
\end{center}
at the top of every child file \textit{child}
which is included by |\include{|\textit{child}|}|
from within the main file
(or at least for those files to be compiled individually).
The argument \textit{main} must be the filename of the main file.

There are a couple of
considerations in setting up the main and child documents:

%%%%%%%%%%%%%%%%%%%%%%%%%%%%%%%%%%%%%%%%
\paragraph{Restrictions.}

Please note the following restrictions:
\begin{itemize}
\item
|\childdocmain| must be called with one argument \textit{main}
to ensure compatibility with earlier version of the package.
It must either be empty (|\childdocmain{}|)
or precisely match the filename of the main file in which it is specified.
See \secref{sec:detection} for further information.
\item
The filename \textit{main} must be specified without the |.tex| extension.
\item
The filename \textit{main} is case sensitive
(even in case-insensitive file systems)
due to internal string comparison.
\item
The argument \textit{main} should be fully expanded, it cannot be a macro.
\item
Subdirectories and special characters should be avoided in filenames.
\item
The command |\childdocmain{|\textit{main}|}| must be followed by a whitespace.
It should not be followed immediately by another command
or by a comment mark `|%|'.
This is because the \TeX{} parser reads the token immediately following
the argument of |\childdocmain| and puts it
at the beginning of every child section;
however, a white\-space is ignored.
\end{itemize}

%%%%%%%%%%%%%%%%%%%%%%%%%%%%%%%%%%%%%%%%
\paragraph{Content of Main File.}

It is advisable to place all content in the child files included by |\include|.
Any output contained in the main file will appear in all child documents
unless suppressed manually;
it cannot be suppressed automatically by the |\includeonly| directive
and thus should normally be avoided.
A method to include some content in the main file
by means of conditional processing is described in \secref{sec:conditional}.

%%%%%%%%%%%%%%%%%%%%%%%%%%%%%%%%%%%%%%%%
\paragraph{Page Numbering.}

When only a part of the document is compiled,
the appropriate numbering of pages
(as well as other status parameters)
is determined from the |.aux| files.
The latter contain information from previous passes.
However this information needs to propagate through
all intermediate child documents.
Therefore the page numbering in child documents may well
be inconsistent until the complete document is compiled at least once.

A useful (if unconventional) way to always ensure a consistent
page numbering is to restart the numbering in each child document
and denote the pages by `\textit{child}|.|\textit{page}'
where \textit{child} represents the chapter/section number of the child file.
This can be achieved by the command
|\numberwithin{page}{|\textit{child}|}|
of the \textsf{amsmath} package
where \textit{child} can be |chapter| or |section|
depending on the chosen structuring.
Alternatively, one can modify the macro |\thepage| appropriately
and reset the counter |page| at the start of each child file.

%%%%%%%%%%%%%%%%%%%%%%%%%%%%%%%%%%%%%%%%%%%%%%%%%%%%%%%%%%%%%%%%%%%%%%%%%%%%%%%%
\subsection{Conditional Processing}
\label{sec:conditional}

The package provides a mechanism to compile different versions
of a document. To customise the versions further some conditional processing
can come in handy to distinguish which version is being compiled.
The package provides two macros to describe the compilation context:

%%%%%%%%%%%%%%%%%%%%%%%%%%%%%%%%%%%%%%%%
\DescribeMacro{\ifchilddoc}
The conditional |\ifchilddoc| distinguishes between the compilation of
child documents and the main document:
%
\begin{center}
|\ifchilddoc |\textit{child-code}| |[|\||else |\textit{main-code}]| \||fi|
\end{center}

%%%%%%%%%%%%%%%%%%%%%%%%%%%%%%%%%%%%%%%%
\DescribeMacro{\childdocname}
\DescribeMacro{\childdocjob}
The macro |\childdocname| contains the filename (without extension)
of the main or child file being processed.
Note that |\childdocjob| will always contain the name of the main file.

%%%%%%%%%%%%%%%%%%%%%%%%%%%%%%%%%%%%%%%%
\paragraph{Title Page.}

Conditional processing can be used to include a title or banner page
in the main document when proper precautions are taken.
Importantly, the code in the main file should ensure that the page counter
(as well as other status parameters which are stored in the |.aux| files)
takes the same value after the conditional processing.
Otherwise the page numbers may take divergent values
depending on which part is compiled.

For example, a title page could be declared by:
%
\begin{center}
\begin{tabular}{l}
|\ifchilddoc\||else|\\
|\addtocounter{page}{-1}|\\
\textit{code for title page}\\
|\newpage|\\
|\||fi|
\end{tabular}
\end{center}
%
A banner page for the child documents can be generated by:
%
\begin{center}
\begin{tabular}{l}
|\ifchilddoc|\\
|\addtocounter{page}{-1}|\\
\textit{code for banner page}\\
|\newpage|\\
|\||fi|
\end{tabular}
\end{center}
%
Here one could write a message such as:
\begin{center}
|This is the part \childdocname{} of \childdocjob{}.|
\end{center}

%%%%%%%%%%%%%%%%%%%%%%%%%%%%%%%%%%%%%%%%%%%%%%%%%%%%%%%%%%%%%%%%%%%%%%%%%%%%%%%%
\subsection{Flags}
\label{sec:flags}

The package makes it easy to generate different versions
of the main or child documents.
To this end compilation flags can be defined
and assigned different default values.
They will be particularly useful in conjunction
with the forwarding mechanism described in \secref{sec:forward}.

For example, it may be useful to have a flag |\version|
which can be set to |draft| or |final|.
The document source will contain some conditional code
depending on the value of |\version|.
Suppose further, the flag should default to |final| for the main file
and to |draft| for child files
which is a natural assignment for editing the document.
This is achieved by placing the following code
in the preamble of the main document
(below the |\childdocmain| directive):
%
\begin{center}
\begin{tabular}{l}
|\ifchilddoc|\\
|\providecommand{\version}{draft}|\\
|\||else|\\
|\providecommand{\version}{final}|\\
|\||fi|
\end{tabular}
\end{center}
%
The definition by |\providecommand| makes sure
that previous definitions are not overwritten.
Further statements |\providecommand{\version}{...}|
can thus be added before the above code to override it.

For the main file, one might add a line
(between |\childdocmain| and the above block)
%
\begin{center}
|%\ifchilddoc\||else\providecommand{\version}{draft}\||fi|
\end{center}
%
which can be uncommented to produce a draft version.
Likewise one can add a line to the very top of a child file
(above the |\childdocof{|\textit{main}|}| directive)
%
\begin{center}
|%\providecommand{\version}{final}|
\end{center}
%
which can be uncommented to produce the final version of this child document.

%%%%%%%%%%%%%%%%%%%%%%%%%%%%%%%%%%%%%%%%%%%%%%%%%%%%%%%%%%%%%%%%%%%%%%%%%%%%%%%%
\subsection{Forwarding}
\label{sec:forward}

Different versions of the main or child documents
using compilation flags as described in \secref{sec:flags}
can be (permanently) stored in different files
for convenient compilation, viewing and distribution.
To this end, the package defines a command
to pass on compilation to a different file:

%%%%%%%%%%%%%%%%%%%%%%%%%%%%%%%%%%%%%%%%
\DescribeMacro{\childdocforward}
The command |\childdocforward| redirects processing to
another source file:
%
\begin{center}
\begin{tabular}{l}
|\input{childdoc.def}|\\
|\childdocforward[|\textit{main}|]{|\textit{dest}|}|\\
\end{tabular}
\end{center}
%
The argument \textit{dest} is the destination file
(without extension).
It should be the main file or one of the child files.
Note that further \textsf{childdoc} directives
such as |\childdocof| and |\childdocforward|
in the indicated file will be processed in this form.
The optional argument \textit{main}
passes on directly to the main file \textit{main}
while pretending to compile the child \textit{dest}.
This form behaves as if \textit{dest}
issues |\childdocof{|\textit{main}|}| right away,
and no further \textsf{childdoc} directives will be processed.

%%%%%%%%%%%%%%%%%%%%%%%%%%%%%%%%%%%%%%%%
\DescribeMacro{\...prefix}
In the alternative form |\childdocforwardprefix|,
%
\begin{center}
\begin{tabular}{l}
|\input{childdoc.def}|\\
|\childdocforwardprefix[|\textit{main}|]{|\textit{prefix}|}{|\textit{dest}|}|
\end{tabular}
\end{center}
%
the destination file is determined by a pattern
depending on the current file:
To make this work, the current file must be called
`{\textit{prefix}\hspace{0.2em}\textit{suffix}}'
with \textit{prefix} matching precisely the argument.
Processing is then passed on to the file
`{\textit{dest}\hspace{0.2em}\textit{suffix}}'.
Surely, the same effect is achieved by
directly specifying the
argument `{\textit{dest}\hspace{0.2em}\textit{suffix}}'
in the first form.
However, that requires to set up a different file
for each child. With the alternative form of the command
all these files can have exactly the same content
which simplifies setting them up and maintaining them.

For example, the following file |draft.tex|
with a compilation flag |\version| as described in \secref{sec:flags}
compiles the main document as a draft:
%
\begin{center}
\begin{tabular}{l}
|\def\version{draft}|\\
|\input{childdoc.def}|\\
|\childdocforward{|\textit{main}|}|
\end{tabular}
\end{center}
%
Likewise, the following files |final|\textit{nn}|.tex|
compile the final version of the child document
|child|\textit{nn}|.tex|:
%
\begin{center}
\begin{tabular}{l}
|\def\version{final}|\\
|\input{childdoc.def}|\\
|\childdocforwardprefix{final}{child}|
\end{tabular}
\end{center}
%

Note that when several versions of a main file and/or of each child file
are to be generated, it may be convenient to set up a |Makefile| or
shell script to automatise the process.

%%%%%%%%%%%%%%%%%%%%%%%%%%%%%%%%%%%%%%%%%%%%%%%%%%%%%%%%%%%%%%%%%%%%%%%%%%%%%%%%
\subsection{Command Line Processing}
\label{sec:commandline}

The effect of redirection files can also be achieved by invoking
the \LaTeX{} compiler with a more elaborate command line.
Most conveniently this should be done as part
of a shell script or a |Makefile|.

When using \textsf{childdoc} in the main file, the following
command lines effectively perform a redirection
(note that depending on the shell being used,
backslashes may have to be doubled: `|\|' $\to$ `|\\|'):
%
\begin{center}
|... -jobname "|\textit{target}|" |\\|"|[\textit{flags}]%
|\input{childdoc.def}\childdocforward[|\textit{main}|]{|\textit{dest}|}"|
\end{center}
%
Here \textit{target} is the name of the output file,
\textit{main} is the name of the main file
and \textit{dest} is the name of the main or child file to be processed
(all filenames without extensions).
The optional argument \textit{main} can be omitted
if \textit{main} matches \textit{dest}.
Optionally, compilation \textit{flags} can be defined via |\def| commands.
This command line makes the \TeX{} engine believe
it is compiling the file \textit{target}
whose content is specified as the latter parameter.
The provided code then forwards the processing to
\textit{main} or \textit{dest} as described in \secref{sec:forward}.

%%%%%%%%%%%%%%%%%%%%%%%%%%%%%%%%%%%%%%%%%%%%%%%%%%%%%%%%%%%%%%%%%%%%%%%%%%%%%%%%
\subsection{Include by Input}
\label{sec:input}

Including child documents by |\include| has some restrictions by design.
Most notably, the content of a child document always occupies
its own set of pages; pages cannot be shared between child documents.
Usually, this behaviour makes perfect sense
because each child document contain an essential part of the document.
However, in some situations it may be desirable to compose
a document from a collection of parts
without having mandatory page breaks between then.
For this case, the package
provides a mechanism to include parts
by |\input| which can also be processed individually.
However, by construction this mechanism
requires manual handling of the content to be output.

%%%%%%%%%%%%%%%%%%%%%%%%%%%%%%%%%%%%%%%%
\DescribeMacro{\ifchilddocmanual}
The main file should be prepared as usual, see \secref{sec:include}.
However, the document body must make a distinction
between processing of an individual part and of the main document, e.g.:
%
\begin{center}
\begin{tabular}{l}
|\ifchilddocmanual|\\
|\input{\childdocname}|\\
|\||else|\\
\textit{document body with }|\input{|\textit{part}|}|\\
|\||fi|
\end{tabular}
\end{center}
%
The conditional |\ifchilddocmanual| is true whenever
a part to be included by |\input| is being compiled,
and the name of the part is stored in |\childdocname|.

%%%%%%%%%%%%%%%%%%%%%%%%%%%%%%%%%%%%%%%%
\DescribeMacro{\childdocby}
Each part to be included by |\input| should start with:
%
\begin{center}
\begin{tabular}{l}
|\input{childdoc.def}|\\
|\childdocby{|\textit{main}|}|\\
\end{tabular}
\end{center}
%
The directive |\childdocby| is similar to |\childdocof|
described in \secref{sec:include},
but the subsequent selection of content must be done manually.
To that end, both |\ifchilddoc| and |\ifchilddocmanual|
will be true upon processing of a part,
and the name of the part is stored in |\childdocname|.
Note that |\jobname| will be set to the filename of the current part
so that each part receives an individual |.aux| file
that does not interfere with the |.aux| file(s) of the main document.
This behaviour can be altered by the alternative form
|\childdocby[*]{|\textit{main}|}| (with a non-empty optional argument)
which uses the |.aux| file of the main document
by setting |\jobname| to \textit{main}.

%%%%%%%%%%%%%%%%%%%%%%%%%%%%%%%%%%%%%%%%%%%%%%%%%%%%%%%%%%%%%%%%%%%%%%%%%%%%%%%%
\subsection{Driver Development}
\label{sec:driver}

The \textsf{childdoc} mechanism can also be use for the development
of definition files such as \LaTeX{} styles or classes.
This case differs from the above setup with multiple parts
included by |\include| in that no |\includeonly| should be invoked.
This can be achieved by starting the include file
(before |\ProvidesPackage|) with:
%
\begin{center}
\begin{tabular}{l}
|\input{childdoc.def}|\\
|\childdocforward{|\textit{main}|}|\\
\end{tabular}
\end{center}
%
or alternatively with:
%
\begin{center}
\begin{tabular}{l}
|\input{childdoc.def}|\\
|\childdocby{|\textit{main}|}|\\
\end{tabular}
\end{center}
%
Both forms have slightly different effects as described above.
The main file is prepared as usual, see \secref{sec:include}.

%%%%%%%%%%%%%%%%%%%%%%%%%%%%%%%%%%%%%%%%%%%%%%%%%%%%%%%%%%%%%%%%%%%%%%%%%%%%%%%%
\subsection{Legacy Detection}
\label{sec:detection}

The directive |\childdocmain| in the main file can detect
whether the complete document or merely a child is to be compiled
even without using the directive |\childdocof|.
This method is deprecated because it is less robust
and there is no compelling reason to use it;
it is merely provided for backward compatibility
and it may be removed in future versions.

If the detection mechanism is to be used,
it is mandatory to correctly specify
the filename of the main file as the argument of |\childdocmain|:
%
\begin{center}
\begin{tabular}{l}
|\input{childdoc.def}|\\
|\childdocmain{|\textit{main}|}|\\
\end{tabular}
\end{center}
%
If |\jobname| does not match the argument \textit{main} of |\childdocmain|,
it is assumed that |\jobname| points to the child file to be compiled.
When using |\childdocmain| with the main file specified as argument,
it suffices to start a child file
with just |\input{|\textit{main}|}|
without loading of the package and using |\childdocof|.
If instead all processing is done
with the appropriate \textsf{childdoc} directives,
the argument of \textit{main} of |\childdocmain| can be empty.

An alternative version of the command line processing described
in \secref{sec:commandline} using the detection mechanism reads:
%
\begin{center}
|... -jobname "|\textit{target}|" "|[\textit{flags}]%
[|\def\jobname{|\textit{dest}|}|]|\input{|\textit{main}|}"|
\end{center}

%%%%%%%%%%%%%%%%%%%%%%%%%%%%%%%%%%%%%%%%%%%%%%%%%%%%%%%%%%%%%%%%%%%%%%%%%%%%%%%%
\subsection{Manual Code}
\label{sec:manual}

In case one cannot be certain whether the definitions file |childdoc.def|
is installed on the target \TeX{} distribution
and one prefers not to ship it,
it is conceivable to paste a few relevant commands into the sources.

To that end, drop all statements |\input{childdoc.def}|
and perform the replacements as outlined below.
Instead of |\childdocmain{|\textit{main}|}| add the following code
to the top of the main file:
%
\begin{center}
\begin{tabular}{l}
|\||ifdefined\childdocname\endinput\||fi\newif\ifchilddoc|\\
|\edef\childdocname{\scantokens\expandafter{\jobname\noexpand}}|\\
|\def\childdocmain{|\textit{main}|}\||ifx\childdocmain\childdocname\||else|\\
|\childdoctrue\includeonly{\childdocname}\let\jobname\childdocmain\||fi|\\
\end{tabular}
\end{center}
%
Instead of |\childdocof{|\textit{main}|}| just include the main file
at the top of each child file:
%
\begin{center}
|\input{|\textit{main}|}|
\end{center}
%
A simple redirection |\childdocforward{|\textit{dest}|}| is achieved by:
%
\begin{center}
|\def\jobname{|\textit{dest}|}\input{\jobname}|
\end{center}
%
The redirection with prefix
|\childdocforwardprefix[|\textit{prefix}|]{|\textit{dest}|}|
is accomplished by:
%
\begin{center}
\begin{tabular}{l}
|{\edef\jobname{\scantokens\expandafter{\jobname\noexpand}}|\\
|\def\redirectjob |\textit{prefix}|#1~~~{\gdef\jobname{|\textit{dest}|#1}}|\\
|\expandafter\redirectjob\jobname~~~}\input{\jobname}|
\end{tabular}
\end{center}

In an alternative approach,
child documents can be compiled by a specific command line
without additional code or specific definitions:
%
\begin{center}
|... -jobname "|\textit{target}|" "|[\textit{flags}]%
|\includeonly{|\textit{dest}|}\input{|\textit{main}|}"|
\end{center}
%

%%%%%%%%%%%%%%%%%%%%%%%%%%%%%%%%%%%%%%%%%%%%%%%%%%%%%%%%%%%%%%%%%%%%%%%%%%%%%%%%
%%%%%%%%%%%%%%%%%%%%%%%%%%%%%%%%%%%%%%%%%%%%%%%%%%%%%%%%%%%%%%%%%%%%%%%%%%%%%%%%
\section{Information}

%%%%%%%%%%%%%%%%%%%%%%%%%%%%%%%%%%%%%%%%%%%%%%%%%%%%%%%%%%%%%%%%%%%%%%%%%%%%%%%%
\subsection{Copyright}

Copyright \copyright{} 2017--2018 Niklas Beisert

This work may be distributed and/or modified under the
conditions of the \LaTeX{} Project Public License, either version 1.3
of this license or (at your option) any later version.
The latest version of this license is in
  \url{http://www.latex-project.org/lppl.txt}
and version 1.3 or later is part of all distributions of \LaTeX{}
version 2005/12/01 or later.

This work has the LPPL maintenance status `maintained'.

The Current Maintainer of this work is Niklas Beisert.

This work consists of the files |README.txt|, |childdoc.ins| and |childdoc.dtx|
as well as the derived files |childdoc.def|, |cdocsamp.tex|
with |cdocsch1.tex|, |cdocsch2.tex|, |cdocspt3.tex|, |cdocspt4.tex|,
|cdocsdrf.tex|, |cdocsfn1.tex|, |cdocsfn2.tex|
as well as |childdoc.pdf|.

%%%%%%%%%%%%%%%%%%%%%%%%%%%%%%%%%%%%%%%%%%%%%%%%%%%%%%%%%%%%%%%%%%%%%%%%%%%%%%%%
\subsection{Files and Installation}

The package consists of the files:
%
\begin{center}
\begin{tabular}{ll}
    |README.txt|   & readme file \\
    |childdoc.ins| & installation file \\
    |childdoc.dtx| & source file \\
    |childdoc.def| & definition file \\
    |cdocsamp.tex| & sample main file \\
    |cdocsch1.tex| & sample include file \\
    |cdocsch2.tex| & sample include file \\
    |cdocspt3.tex| & sample part file \\
    |cdocspt4.tex| & sample part file \\
    |cdocsdrf.tex| & sample redirection file \\
    |cdocsfn1.tex| & sample redirection file \\
    |cdocsfn2.tex| & sample redirection file \\
    |childdoc.pdf| & manual
\end{tabular}
\end{center}
%
The distribution consists of the files
|README.txt|, |childdoc.ins| and |childdoc.dtx|.
%
\begin{itemize}
\item
Run (pdf)\LaTeX{} on |childdoc.dtx|
to compile the manual |childdoc.pdf| (this file).
\item
Run \LaTeX{} on |childdoc.ins| to create the definitions file |childdoc.def|
and the sample |cdocsamp.tex| with include files
|cdocsch1.tex|, |cdocsch2.tex|, |cdocspt3.tex|, |cdocspt4.tex|,
|cdocsdrf.tex|, |cdocsfn1.tex|, |cdocsfn2.tex|.
Then copy the file |childdoc.def| to an appropriate directory of your \LaTeX{}
distribution, e.g.\ \textit{texmf-root}|/tex/latex/childdoc|.
\end{itemize}

%%%%%%%%%%%%%%%%%%%%%%%%%%%%%%%%%%%%%%%%%%%%%%%%%%%%%%%%%%%%%%%%%%%%%%%%%%%%%%%%
\subsection{Related CTAN Packages}

There are several other packages which offer a similar functionality:
%
\begin{itemize}
\item
The packages
\href{http://ctan.org/pkg/docmute}{\textsf{docmute}},
\href{http://ctan.org/pkg/includex}{\textsf{includex}} and
\href{http://ctan.org/pkg/standalone}{\textsf{standalone}}
provide commands to include only the document body of
a child file thus allowing both files to be compiled individually.
\item
The packages \href{http://ctan.org/pkg/subdocs}{\textsf{subdocs}}
and \href{http://ctan.org/pkg/subfiles}{\textsf{subfiles}}
provide structures in which the main and child documents can be
encapsulated and allowing them to be compiled individually.
The inclusion mechanism is different from the conventional |\include|.
\item
The package \href{http://ctan.org/pkg/combine}{\textsf{combine}}
is an elaborate solution to combine several documents into one.
\end{itemize}
%
See also the CTAN topic \href{http://ctan.org/topic/subdocs}{\textsf{subdocs}}
for further related packages.
The present package differs from the above solutions in that
a document structure constructed with the conventional |\include| mechanism
just needs two extra commands at the top of every file
such that all constituent files can be compiled individually.

%%%%%%%%%%%%%%%%%%%%%%%%%%%%%%%%%%%%%%%%%%%%%%%%%%%%%%%%%%%%%%%%%%%%%%%%%%%%%%%%
%\subsection{Feature Suggestions}
%
%The following is a list of features which may be useful for future
%versions of this package:
%%
%\begin{itemize}
%\item
%\ldots
%\end{itemize}

%%%%%%%%%%%%%%%%%%%%%%%%%%%%%%%%%%%%%%%%%%%%%%%%%%%%%%%%%%%%%%%%%%%%%%%%%%%%%%%%
\subsection{Revision History}

%%%%%%%%%%%%%%%%%%%%%%%%%%%%%%%%%%%%%%%%
\paragraph{v2.0:} 2018/12/30

\begin{itemize}
\item
immediate forward processing
\item
added |\childdocby| mechanism
\item
manual restructured
\end{itemize}

%%%%%%%%%%%%%%%%%%%%%%%%%%%%%%%%%%%%%%%%
\paragraph{v1.6:} 2018/01/17

\begin{itemize}
\item
application for development of include files
\item
corrections to manual
\end{itemize}

%%%%%%%%%%%%%%%%%%%%%%%%%%%%%%%%%%%%%%%%
\paragraph{v1.5:} 2017/05/21

\begin{itemize}
\item
more complete structuring introduced
\item
|\childdocof| introduced
\item
|\childdoc| renamed to |\childdocmain|
\item
|\childredirect| renamed to |\childdocforward| and |\childdocforwardprefix|
and functionality expanded
\end{itemize}

%%%%%%%%%%%%%%%%%%%%%%%%%%%%%%%%%%%%%%%%
\paragraph{v1.0:} 2017/04/27

\begin{itemize}
\item
manual and install package
\item
first version published on CTAN
\end{itemize}

%%%%%%%%%%%%%%%%%%%%%%%%%%%%%%%%%%%%%%%%
\paragraph{v0.6:} 2017/04/26

\begin{itemize}
\item
redirection mechanism added
\end{itemize}

%%%%%%%%%%%%%%%%%%%%%%%%%%%%%%%%%%%%%%%%
\paragraph{v0.5:} 2017/04/26

\begin{itemize}
\item
functionality in definition file
\end{itemize}


%%%%%%%%%%%%%%%%%%%%%%%%%%%%%%%%%%%%%%%%%%%%%%%%%%%%%%%%%%%%%%%%%%%%%%%%%%%%%%%%
%%%%%%%%%%%%%%%%%%%%%%%%%%%%%%%%%%%%%%%%%%%%%%%%%%%%%%%%%%%%%%%%%%%%%%%%%%%%%%%%
%%%%%%%%%%%%%%%%%%%%%%%%%%%%%%%%%%%%%%%%%%%%%%%%%%%%%%%%%%%%%%%%%%%%%%%%%%%%%%%%
\appendix

\settowidth\MacroIndent{\rmfamily\scriptsize 000\ }

 \DocInput{childdoc.dtx}

\end{document}
%</driver>
% \fi
%
% %%%%%%%%%%%%%%%%%%%%%%%%%%%%%%%%%%%%%%%%%%%%%%%%%%%%%%%%%%%%%%%%%%%%%%%%%%%%%%
% %%%%%%%%%%%%%%%%%%%%%%%%%%%%%%%%%%%%%%%%%%%%%%%%%%%%%%%%%%%%%%%%%%%%%%%%%%%%%%
% \section{Sample}
%\iffalse
%<*samplemain>
%\fi
%
% The following presents a sample document
% with two chapters, two parts, a title page,
% a compile flag as well as three forwarding files to set the flag.
% It consists of eight |.tex| files:
% \begin{center}
% \begin{tabular}{ll}
% |cdocsamp.tex|&main file\\
% |cdocsch1.tex|&include file for chapter 1\\
% |cdocsch2.tex|&include file for chapter 2\\
% |cdocspt3.tex|&include file for part 3\\
% |cdocspt4.tex|&include file for part 4\\
% |cdocsdrf.tex|&forwarding file for main file in draft mode\\
% |cdocsfi1.tex|&forwarding file for final version of chapter 1\\
% |cdocsfi2.tex|&forwarding file for final version of chapter 2\\
% \end{tabular}
% \end{center}
% Each of the eight files can be compiled directly by the \LaTeX{} compiler.
%
% %%%%%%%%%%%%%%%%%%%%%%%%%%%%%%%%%%%%%%
% \paragraph{Main File.}
%
% The main file is called |cdocsamp.tex|.
%
% Load the \textsf{childdoc} definitions and
% declare the filename for the main document:
%    \begin{macrocode}
\input{childdoc.def}
\childdocmain{}
%    \end{macrocode}

% Optional override for |\version| flag:
%    \begin{macrocode}
%%\ifchilddoc\else\providecommand{\version}{draft}\fi
%    \end{macrocode}

% Define the default values for the |\version| flag
% (|final| for the main file and |draft| for childs):
%    \begin{macrocode}
\ifchilddoc
\providecommand{\version}{draft}
\else
\providecommand{\version}{final}
\fi
%    \end{macrocode}

% Load the standard document class:
%    \begin{macrocode}
\documentclass[12pt]{article}
%    \end{macrocode}

% Start the document body:
%    \begin{macrocode}
\begin{document}
%    \end{macrocode}

% Declare a title page.
% Print title, part of document being processed and version flag:
%    \begin{macrocode}
\addtocounter{page}{-1}
\begin{center}
{\LARGE\bfseries{}childdoc example\par}
\vspace{1cm}
\ifchilddoc
\ifchilddocmanual part\else chapter\fi:
`\childdocname' of `\childdocjob'\par
\else
main document: `\childdocjob'\par
\fi
version: \version\par
\end{center}
\newpage
%    \end{macrocode}

% Manually include selected file,
% otherwise process as usual:
%    \begin{macrocode}
\ifchilddocmanual
\section*{part `\childdocname'}
\input{\childdocname}
\else
%    \end{macrocode}

% Include the two chapters:
%    \begin{macrocode}
\include{cdocsch1}
\include{cdocsch2}
%    \end{macrocode}

% Include the two parts unless only chapters should be displayed:
%    \begin{macrocode}
\ifchilddoc\else
\section{part three}
\input{cdocspt3}
\section{part four}
\input{cdocspt4}
\fi
%    \end{macrocode}

% Process as usual until here:
%    \begin{macrocode}
\fi
%    \end{macrocode}

% End of document body:
%    \begin{macrocode}
\end{document}
%    \end{macrocode}
%\iffalse
%</samplemain>
%\fi
%
% %%%%%%%%%%%%%%%%%%%%%%%%%%%%%%%%%%%%%%
% \paragraph{Chapter Include Files.}
%
% The include files are called |cdocsch1.tex| and |cdocsch2.tex|.
%
%\iffalse
%<*samplechap1|samplechap2>
%\fi

% Optional override for |\version| flag:
%    \begin{macrocode}
%%\providecommand{\version}{final}
%    \end{macrocode}

% Include the main document:
%    \begin{macrocode}
\input{childdoc.def}
\childdocof{cdocsamp}
%    \end{macrocode}

%\iffalse
%</samplechap1|samplechap2>
%\fi
%
%\iffalse
%<*samplechap1>
%\fi
% Some text for chapter 1:
%    \begin{macrocode}
\section{one}
some text in chapter one
%    \end{macrocode}

%\iffalse
%</samplechap1>
%\fi
% Some text for chapter 2:
%\iffalse
%<*samplechap2>
%\fi
%    \begin{macrocode}
\section{two}
more text in chapter two
%    \end{macrocode}

%\iffalse
%</samplechap2>
%\fi
%
% %%%%%%%%%%%%%%%%%%%%%%%%%%%%%%%%%%%%%%
% \paragraph{Part Include Files.}
%
% The include files are called |cdocspt3.tex| and |cdocspt4.tex|.
%
%\iffalse
%<*samplepart3|samplepart4>
%\fi

% Optional override for |\version| flag:
%    \begin{macrocode}
%%\providecommand{\version}{final}
%    \end{macrocode}

% Include the main document:
%    \begin{macrocode}
\input{childdoc.def}
\childdocby{cdocsamp}
%    \end{macrocode}

%\iffalse
%</samplepart3|samplepart4>
%\fi
%
%\iffalse
%<*samplepart3>
%\fi
% Some text for part 3:
%    \begin{macrocode}
some text in part three
%    \end{macrocode}

%\iffalse
%</samplepart3>
%\fi
% Some text for part 4:
%\iffalse
%<*samplepart4>
%\fi
%    \begin{macrocode}
more text in part four
%    \end{macrocode}

%\iffalse
%</samplepart4>
%\fi
%
% %%%%%%%%%%%%%%%%%%%%%%%%%%%%%%%%%%%%%%
% \paragraph{Forwarding for a Complete Draft.}
%
% The following forwarding file |cdocsdrf.tex|
% compiles the main document in draft mode:
%\iffalse
%<*sampledraft>
%\fi
%    \begin{macrocode}
\def\version{draft}
\input{childdoc.def}
\childdocforward{cdocsamp}
%    \end{macrocode}

%\iffalse
%</sampledraft>
%\fi
%
% %%%%%%%%%%%%%%%%%%%%%%%%%%%%%%%%%%%%%%
% \paragraph{Forwarding for Final Version of the Chapters.}
%
% The following forwarding files |cdocsfn1.tex| and |cdocsfn2.tex|
% (with identical content)
% compile the final versions of the child documents
% |cdocsch1.tex| and |cdocsch2.tex|, respectively:
%\iffalse
%<*samplefinal>
%\fi
%    \begin{macrocode}
\def\version{final}
\input{childdoc.def}
\childdocforwardprefix[cdocsamp]{cdocsfn}{cdocsch}
%    \end{macrocode}

%\iffalse
%</samplefinal>
%\fi
%
% %%%%%%%%%%%%%%%%%%%%%%%%%%%%%%%%%%%%%%
% \paragraph{Command Line Processing.}
%
% The following three command lines generate the output files
% |cdocscld|, |cdocscl1| and |cdocscl2|
% which should be identical to
% |cdocsdrf|, |cdocsch1| and |cdocsfn2|, respectively:
% \begin{center}
% \begin{tabular}{l}
% |latex -jobname cdocscld \|\\
% |  "\def\version{draft}\input{childdoc.def}\childdocforward{cdocsamp}"|\\
% |latex -jobname cdocscl1 \|\\
% |  "\input{childdoc.def}\childdocforward[cdocsamp]{cdocsch1}"|\\
% |latex -jobname cdocscl2 \|\\
% |  "\def\version{final}\input{childdoc.def}\childdocforward{cdocsch2}"|
% \end{tabular}
% \end{center}
% Note that the trailing backslash on each first line
% merely continues the input to the second line
% (for convenient cut ant paste).
% Furthermore, the command |latex| can be replaced by any
% of its alternative versions such as |pdflatex|.
%
% %%%%%%%%%%%%%%%%%%%%%%%%%%%%%%%%%%%%%%%%%%%%%%%%%%%%%%%%%%%%%%%%%%%%%%%%%%%%%%
% %%%%%%%%%%%%%%%%%%%%%%%%%%%%%%%%%%%%%%%%%%%%%%%%%%%%%%%%%%%%%%%%%%%%%%%%%%%%%%
% \section{Implementation}
%\iffalse
%<*package>
%\fi
%
% This section describes the definitions file |childdoc.def|.

% The definitions cannot be loaded using |\usepackage| or |\RequirePackage|
% which has a mechanism to prevent loading a style file more than once.
% When loading the definitions by means of |\input|
% multiple instances have to be prevented manually:
%\iffalse
%This code needs to be before the `\ProvidesFile' directive
%which is defined at the beginning of this file.
%Therefore it is also placed there and commented out here.
%</package>
%<*discard>
%\fi
%    \begin{macrocode}
\ifdefined\childdocmain\endinput\fi
%    \end{macrocode}
%\iffalse
%</discard>
%<*package>
%\fi
%
% \macro{\ifchilddoc}
% \macro{\ifchilddocmanual}
% The conditional |\ifchilddoc| tells whether a
% child (true) or main (false) document is being compiled.
% The conditional |\ifchilddocmanual| tells whether
% the |\includeonly| mechanism is used (false) or
% the selection of child files must be performed manually (true).
% The definitions initialise to false:
%    \begin{macrocode}
\newif\ifchilddoc
\newif\ifchilddocmanual
%    \end{macrocode}

% \macro{\childdocname}
% \macro{\childdocjob}
% The macro |\childdocname| stores the name of the main document
% to be compiled. The macro |\childdocjob| stores the name of
% the document on which the \LaTeX{} compiler was originally invoked.
% The content of |\jobname| cannot be compared
% to filenames specified in the source due to different catcodes.
% The following code rescans |\jobname|, stores the result
% in |\childdocname| and saves a copy in |\childdocjob|:
%    \begin{macrocode}
\edef\childdocname{\scantokens\expandafter{\jobname\noexpand}}
\let\childdocjob\childdocname
%    \end{macrocode}

% \macro{\childdocdisable}
% The macro |\childdocdisable| prevents the main file
% from being processed more than once.
% At this stage, the main document command |\childdocmain|
% is assumed to be called once again where it should do nothing.
% Any subsequent call to it should prevent
% a secondary processing of the main document
% It overwrites the forwarding commands
% |\childdocof| and |\childdocforward|
% with empty macros to prevent further inclusions of the main document:
%    \begin{macrocode}
\newcommand{\childdocdisable}
{
  \renewcommand{\childdocmain}[1]{\renewcommand{\childdocmain}[1]{\endinput}}
  \renewcommand{\childdocof}[1]{}
  \renewcommand{\childdocby}[2][]{}
  \renewcommand{\childdocforward}[2][]{}
  \renewcommand{\childdocdisable}{}
}
%    \end{macrocode}

% \macro{\childdocmain}
% The macro |\childdocmain| is to be called at the top of the main file
% with nothing or the main filename (without extension) as argument.
% First, it breaks loops.
% If the argument is not empty and does not match |\childdocname|
% (which is set by the first inclusion of |childdoc.def|),
% |\ifchilddoc| is set to true, |\includeonly| is applied to the child file
% and |\jobname| is set to the main file
% (for proper handling of |.aux| files):
%    \begin{macrocode}
\newcommand{\childdocmain}[1]
{
  \childdocdisable\childdocmain{}
  \if?#1?\else
    \begingroup
      \def\childdoctmp{#1}
      \ifx\childdoctmp\childdocname
        \def\childdoctmp{}
      \else
        \def\childdoctmp
        {
          \childdoctrue
          \includeonly{\childdocname}
          \def\childdocjob{#1}
          \def\jobname{#1}
        }
      \fi
      \expandafter
    \endgroup
    \childdoctmp
  \fi
}
%    \end{macrocode}

% \macro{\childdocof}
% The command |\childdocof| redirects
% compilation to the main file |#1|.
%    \begin{macrocode}
\newcommand{\childdocof}[1]
{
  \childdocdisable
  \childdoctrue
  \includeonly{\childdocname}
  \def\jobname{#1}
  \def\childdocjob{#1}
  \input{#1}
}
%    \end{macrocode}

% \macro{\childdocby}
% The command |\childdocby| ....
%    \begin{macrocode}
\newcommand{\childdocby}[2][]
{
  \childdocdisable
  \childdoctrue
  \childdocmanualtrue
  \if?#1?\else
    \def\jobname{#2}
  \fi
  \def\childdocjob{#2}
  \input{#2}
  \endinput
}
%    \end{macrocode}

% \macro{\childdocforward}
% The command |\childdocforward| redirects
% compilation to the main file or
% (if the optional argument is given) a child file.
% Parameters are set as if the main file
% or a child file starting with |\childdocof| was compiled.
% Then compilation is handed over to the main file:
%    \begin{macrocode}
\newcommand{\childdocforward}[2][]
{
  \begingroup
    \if?#1?
      \def\childdoctmp
      {
        \def\childdocname{#2}
        \def\childdocjob{#2}
        \def\jobname{#2}
        \input{#2}
        \endinput
      }
    \else
      \def\childdoctmp
      {
        \childdocdisable
        \def\childdocname{#2}
        \childdoctrue
        \includeonly{#2}
        \def\childdocjob{#1}
        \def\jobname{#1}
        \input{#1}
        \endinput
      }
    \fi
    \expandafter
  \endgroup
  \childdoctmp
}
%    \end{macrocode}

% \macro{\childdocforwardprefix}
% The command |\childdocforwardprefix| redirects
% compilation to the main or a child file by means of a pattern.
% The prefix |#1| in the current filename is replaced by |#2|
% and the suffix of the current filename is kept
% (it is assumed that the filename does not contain the substring `|~~~|'
% which is used as a delimiter).
% Compilation is handed over to the new file by |\childdocforward|:
%    \begin{macrocode}
\newcommand{\childdocforwardprefix}[3][]
{
  \begingroup
    \def\childdocextract #2##1~~~{\def\childdoctmp{\childdocforward[#1]{#3##1}}}
    \expandafter\childdocextract\childdocname~~~
    \expandafter
  \endgroup
  \childdoctmp
}
%    \end{macrocode}

% \macro{\childdoc}
% The deprecated macro |\childdoc| is a legacy version of |\childdocmain|:
%    \begin{macrocode}
\newcommand{\childdoc}{\childdocmain}
%    \end{macrocode}

% \macro{\childdocredirect}
% The deprecated macro |\childdocredirect| is a legacy version
% of |\childdocforward| and |\childdocforwardprefix|:
%    \begin{macrocode}
\newcommand{\childdocredirect}[2][]
{
  \begingroup
    \if?#1?
      \def\childdoctmp{\childdocforward{#2}}
    \else
      \def\childdoctmp{\childdocforwardprefix{#1}{#2}}
    \fi
    \expandafter
  \endgroup
  \childdoctmp
}
%    \end{macrocode}

%\iffalse
%</package>
%\fi
%
\endinput
|\\
|\childdocmain{}|\\
\end{tabular}
\end{center}
at the very top of the main \LaTeX{} file,
in particular \emph{before} the |\documentclass| statement!
The argument of |\childdocmain| should be left empty
(but it must be present).

%%%%%%%%%%%%%%%%%%%%%%%%%%%%%%%%%%%%%%%%
\DescribeMacro{\childdocof}
Furthermore, add the commands
\begin{center}
\begin{tabular}{l}
|% \iffalse
%
% childdoc.dtx Copyright (C) 2017-2018 Niklas Beisert
%
% This work may be distributed and/or modified under the
% conditions of the LaTeX Project Public License, either version 1.3
% of this license or (at your option) any later version.
% The latest version of this license is in
%   http://www.latex-project.org/lppl.txt
% and version 1.3 or later is part of all distributions of LaTeX
% version 2005/12/01 or later.
%
% This work has the LPPL maintenance status `maintained'.
%
% The Current Maintainer of this work is Niklas Beisert.
%
% This work consists of the files childdoc.dtx and childdoc.ins
% and the derived files childdoc.def and cdocsamp.tex with
% cdocsch1.tex, cdocsch2.tex, cdocsdrf.tex, cdocsfn1.tex, cdocsfn2.tex.
%
%<package>\ifdefined\childdocmain\endinput\fi
%<package>\ProvidesFile{childdoc.def}[2018/12/30 v2.0 child document driver]
%<samplemain>\ProvidesFile{cdocsamp.tex}[2018/12/30 v2.0 sample for childdoc]
%<*driver>
%\ProvidesFile{childdoc.drv}[2018/12/30 v2.0 childdoc reference manual file]
\PassOptionsToClass{10pt,a4paper}{article}
\documentclass{ltxdoc}

\usepackage[margin=35mm]{geometry}
\usepackage{hyperref}
\usepackage{hyperxmp}
\usepackage[usenames]{color}

\hypersetup{colorlinks=true}
\hypersetup{pdfstartview=FitH}
\hypersetup{pdfpagemode=UseNone}
\hypersetup{pdfsource={}}
\hypersetup{pdflang={en-UK}}
\hypersetup{pdfcopyright={Copyright 2017-2018 Niklas Beisert.
  This work may be distributed and/or modified under the
  conditions of the LaTeX Project Public License, either version 1.3
  of this license or (at your option) any later version.}}
\hypersetup{pdflicenseurl={http://www.latex-project.org/lppl.txt}}
\hypersetup{pdfcontactaddress={ETH Zurich, ITP, HIT K,
  Wolfgang-Pauli-Strasse 27}}
\hypersetup{pdfcontactpostcode={8093}}
\hypersetup{pdfcontactcity={Zurich}}
\hypersetup{pdfcontactcountry={Switzerland}}
\hypersetup{pdfcontactemail={nbeisert@itp.phys.ethz.ch}}
\hypersetup{pdfcontacturl={http://people.phys.ethz.ch/\xmptilde nbeisert/}}

\newcommand{\secref}[1]{\hyperref[#1]{section \ref*{#1}}}

\parskip1ex
\parindent0pt
\let\olditemize\itemize
\def\itemize{\olditemize\parskip0pt}

\begin{document}

\title{The \textsf{childdoc} Package}
\hypersetup{pdftitle={The childdoc Package}}
\author{Niklas Beisert\\[2ex]
  Institut f\"ur Theoretische Physik\\
  Eidgen\"ossische Technische Hochschule Z\"urich\\
  Wolfgang-Pauli-Strasse 27, 8093 Z\"urich, Switzerland\\[1ex]
  \href{mailto:nbeisert@itp.phys.ethz.ch}
  {\texttt{nbeisert@itp.phys.ethz.ch}}}
\hypersetup{pdfauthor={Niklas Beisert}}
\hypersetup{pdfsubject={Manual for the LaTeX2e Package childdoc}}
\date{30 December 2018, \textsf{v2.0}}
\maketitle

\begin{abstract}\noindent
\textsf{childdoc} is a \LaTeXe{} package
that enables the direct compilation
of document sections included by |\include|
to individual files.
\end{abstract}

\begingroup
\parskip0ex
\tableofcontents
\endgroup

%%%%%%%%%%%%%%%%%%%%%%%%%%%%%%%%%%%%%%%%%%%%%%%%%%%%%%%%%%%%%%%%%%%%%%%%%%%%%%%%
%%%%%%%%%%%%%%%%%%%%%%%%%%%%%%%%%%%%%%%%%%%%%%%%%%%%%%%%%%%%%%%%%%%%%%%%%%%%%%%%
\section{Introduction}

\LaTeX{} provides a mechanism to structure a large document (such as a book)
into a main file and several child files (containing the chapters)
using the |\include| command.
This mechanism is beneficial for documents
which span hundreds of pages in order to
make the source file(s) more manageable.
Moreover, compilation can be restricted to
selected child files by means of the |\includeonly| command.
The latter feature can be used to reduce the compilation time while editing
(this was significantly more useful in the earlier days of \LaTeX{})
or to generate a smaller document which is easier to navigate.
Another application of |\includeonly| is to generate
documents consisting of selected parts of the complete document.

However, there are a few drawbacks of the plain |\include| mechanism:
\begin{itemize}
\item
The child files cannot be compiled on their own,
they can only be compiled via the main file.
A naive editing environment
(such as a text editor with an option
to have the current file processed by \LaTeX)
may require one to switch to the main file before compiling;
attempting to compile the child file produces errors.
\item
The main file must be modified (each time)
to adjust the |\includeonly| command
to the present needs. This easily leaves the main file in a messy state.
\item
The generated document will always carry the filename
of the main document. This is inconvenient if
several child files are to be compiled and
to be kept for distribution.
\end{itemize}

The present package provides a simple interface
to make child files individually compilable by \LaTeX{}.
Compiling a child file then has the same effect as compiling
the main file with an |\includeonly| command
to select the appropriate child.
Moreover the generated document will carry the name of the child
rather than the main file.
This resolves all three above issues.

This feature is meant to make the editing of books,
thesis documents and lecture notes somewhat more convenient.
However, the package can also be used efficiently for
composing a series of documents (such as exercise sheets)
which are typically distributed individually.
It then assists the author in generating the individual documents
(potentially in different versions)
as well as a document containing the collected series.
Another application is in developing style files
or other kinds of included material
where compilation of the style file could redirect
to a sample or test file.

%%%%%%%%%%%%%%%%%%%%%%%%%%%%%%%%%%%%%%%%%%%%%%%%%%%%%%%%%%%%%%%%%%%%%%%%%%%%%%%%
%%%%%%%%%%%%%%%%%%%%%%%%%%%%%%%%%%%%%%%%%%%%%%%%%%%%%%%%%%%%%%%%%%%%%%%%%%%%%%%%
\section{Usage}

First of all, the package \textsf{childdoc} is \emph{not} a standard
\LaTeXe{} |.sty| style file! Therefore it needs to be invoked in
a non-standard way.

%%%%%%%%%%%%%%%%%%%%%%%%%%%%%%%%%%%%%%%%%%%%%%%%%%%%%%%%%%%%%%%%%%%%%%%%%%%%%%%%
\subsection{Included Files}
\label{sec:include}

%%%%%%%%%%%%%%%%%%%%%%%%%%%%%%%%%%%%%%%%
\DescribeMacro{\childdocmain}
To use the package, add the commands
\begin{center}
\begin{tabular}{l}
|\input{childdoc.def}|\\
|\childdocmain{}|\\
\end{tabular}
\end{center}
at the very top of the main \LaTeX{} file,
in particular \emph{before} the |\documentclass| statement!
The argument of |\childdocmain| should be left empty
(but it must be present).

%%%%%%%%%%%%%%%%%%%%%%%%%%%%%%%%%%%%%%%%
\DescribeMacro{\childdocof}
Furthermore, add the commands
\begin{center}
\begin{tabular}{l}
|\input{childdoc.def}|\\
|\childdocof{|\textit{main}|}|\\
\end{tabular}
\end{center}
at the top of every child file \textit{child}
which is included by |\include{|\textit{child}|}|
from within the main file
(or at least for those files to be compiled individually).
The argument \textit{main} must be the filename of the main file.

There are a couple of
considerations in setting up the main and child documents:

%%%%%%%%%%%%%%%%%%%%%%%%%%%%%%%%%%%%%%%%
\paragraph{Restrictions.}

Please note the following restrictions:
\begin{itemize}
\item
|\childdocmain| must be called with one argument \textit{main}
to ensure compatibility with earlier version of the package.
It must either be empty (|\childdocmain{}|)
or precisely match the filename of the main file in which it is specified.
See \secref{sec:detection} for further information.
\item
The filename \textit{main} must be specified without the |.tex| extension.
\item
The filename \textit{main} is case sensitive
(even in case-insensitive file systems)
due to internal string comparison.
\item
The argument \textit{main} should be fully expanded, it cannot be a macro.
\item
Subdirectories and special characters should be avoided in filenames.
\item
The command |\childdocmain{|\textit{main}|}| must be followed by a whitespace.
It should not be followed immediately by another command
or by a comment mark `|%|'.
This is because the \TeX{} parser reads the token immediately following
the argument of |\childdocmain| and puts it
at the beginning of every child section;
however, a white\-space is ignored.
\end{itemize}

%%%%%%%%%%%%%%%%%%%%%%%%%%%%%%%%%%%%%%%%
\paragraph{Content of Main File.}

It is advisable to place all content in the child files included by |\include|.
Any output contained in the main file will appear in all child documents
unless suppressed manually;
it cannot be suppressed automatically by the |\includeonly| directive
and thus should normally be avoided.
A method to include some content in the main file
by means of conditional processing is described in \secref{sec:conditional}.

%%%%%%%%%%%%%%%%%%%%%%%%%%%%%%%%%%%%%%%%
\paragraph{Page Numbering.}

When only a part of the document is compiled,
the appropriate numbering of pages
(as well as other status parameters)
is determined from the |.aux| files.
The latter contain information from previous passes.
However this information needs to propagate through
all intermediate child documents.
Therefore the page numbering in child documents may well
be inconsistent until the complete document is compiled at least once.

A useful (if unconventional) way to always ensure a consistent
page numbering is to restart the numbering in each child document
and denote the pages by `\textit{child}|.|\textit{page}'
where \textit{child} represents the chapter/section number of the child file.
This can be achieved by the command
|\numberwithin{page}{|\textit{child}|}|
of the \textsf{amsmath} package
where \textit{child} can be |chapter| or |section|
depending on the chosen structuring.
Alternatively, one can modify the macro |\thepage| appropriately
and reset the counter |page| at the start of each child file.

%%%%%%%%%%%%%%%%%%%%%%%%%%%%%%%%%%%%%%%%%%%%%%%%%%%%%%%%%%%%%%%%%%%%%%%%%%%%%%%%
\subsection{Conditional Processing}
\label{sec:conditional}

The package provides a mechanism to compile different versions
of a document. To customise the versions further some conditional processing
can come in handy to distinguish which version is being compiled.
The package provides two macros to describe the compilation context:

%%%%%%%%%%%%%%%%%%%%%%%%%%%%%%%%%%%%%%%%
\DescribeMacro{\ifchilddoc}
The conditional |\ifchilddoc| distinguishes between the compilation of
child documents and the main document:
%
\begin{center}
|\ifchilddoc |\textit{child-code}| |[|\||else |\textit{main-code}]| \||fi|
\end{center}

%%%%%%%%%%%%%%%%%%%%%%%%%%%%%%%%%%%%%%%%
\DescribeMacro{\childdocname}
\DescribeMacro{\childdocjob}
The macro |\childdocname| contains the filename (without extension)
of the main or child file being processed.
Note that |\childdocjob| will always contain the name of the main file.

%%%%%%%%%%%%%%%%%%%%%%%%%%%%%%%%%%%%%%%%
\paragraph{Title Page.}

Conditional processing can be used to include a title or banner page
in the main document when proper precautions are taken.
Importantly, the code in the main file should ensure that the page counter
(as well as other status parameters which are stored in the |.aux| files)
takes the same value after the conditional processing.
Otherwise the page numbers may take divergent values
depending on which part is compiled.

For example, a title page could be declared by:
%
\begin{center}
\begin{tabular}{l}
|\ifchilddoc\||else|\\
|\addtocounter{page}{-1}|\\
\textit{code for title page}\\
|\newpage|\\
|\||fi|
\end{tabular}
\end{center}
%
A banner page for the child documents can be generated by:
%
\begin{center}
\begin{tabular}{l}
|\ifchilddoc|\\
|\addtocounter{page}{-1}|\\
\textit{code for banner page}\\
|\newpage|\\
|\||fi|
\end{tabular}
\end{center}
%
Here one could write a message such as:
\begin{center}
|This is the part \childdocname{} of \childdocjob{}.|
\end{center}

%%%%%%%%%%%%%%%%%%%%%%%%%%%%%%%%%%%%%%%%%%%%%%%%%%%%%%%%%%%%%%%%%%%%%%%%%%%%%%%%
\subsection{Flags}
\label{sec:flags}

The package makes it easy to generate different versions
of the main or child documents.
To this end compilation flags can be defined
and assigned different default values.
They will be particularly useful in conjunction
with the forwarding mechanism described in \secref{sec:forward}.

For example, it may be useful to have a flag |\version|
which can be set to |draft| or |final|.
The document source will contain some conditional code
depending on the value of |\version|.
Suppose further, the flag should default to |final| for the main file
and to |draft| for child files
which is a natural assignment for editing the document.
This is achieved by placing the following code
in the preamble of the main document
(below the |\childdocmain| directive):
%
\begin{center}
\begin{tabular}{l}
|\ifchilddoc|\\
|\providecommand{\version}{draft}|\\
|\||else|\\
|\providecommand{\version}{final}|\\
|\||fi|
\end{tabular}
\end{center}
%
The definition by |\providecommand| makes sure
that previous definitions are not overwritten.
Further statements |\providecommand{\version}{...}|
can thus be added before the above code to override it.

For the main file, one might add a line
(between |\childdocmain| and the above block)
%
\begin{center}
|%\ifchilddoc\||else\providecommand{\version}{draft}\||fi|
\end{center}
%
which can be uncommented to produce a draft version.
Likewise one can add a line to the very top of a child file
(above the |\childdocof{|\textit{main}|}| directive)
%
\begin{center}
|%\providecommand{\version}{final}|
\end{center}
%
which can be uncommented to produce the final version of this child document.

%%%%%%%%%%%%%%%%%%%%%%%%%%%%%%%%%%%%%%%%%%%%%%%%%%%%%%%%%%%%%%%%%%%%%%%%%%%%%%%%
\subsection{Forwarding}
\label{sec:forward}

Different versions of the main or child documents
using compilation flags as described in \secref{sec:flags}
can be (permanently) stored in different files
for convenient compilation, viewing and distribution.
To this end, the package defines a command
to pass on compilation to a different file:

%%%%%%%%%%%%%%%%%%%%%%%%%%%%%%%%%%%%%%%%
\DescribeMacro{\childdocforward}
The command |\childdocforward| redirects processing to
another source file:
%
\begin{center}
\begin{tabular}{l}
|\input{childdoc.def}|\\
|\childdocforward[|\textit{main}|]{|\textit{dest}|}|\\
\end{tabular}
\end{center}
%
The argument \textit{dest} is the destination file
(without extension).
It should be the main file or one of the child files.
Note that further \textsf{childdoc} directives
such as |\childdocof| and |\childdocforward|
in the indicated file will be processed in this form.
The optional argument \textit{main}
passes on directly to the main file \textit{main}
while pretending to compile the child \textit{dest}.
This form behaves as if \textit{dest}
issues |\childdocof{|\textit{main}|}| right away,
and no further \textsf{childdoc} directives will be processed.

%%%%%%%%%%%%%%%%%%%%%%%%%%%%%%%%%%%%%%%%
\DescribeMacro{\...prefix}
In the alternative form |\childdocforwardprefix|,
%
\begin{center}
\begin{tabular}{l}
|\input{childdoc.def}|\\
|\childdocforwardprefix[|\textit{main}|]{|\textit{prefix}|}{|\textit{dest}|}|
\end{tabular}
\end{center}
%
the destination file is determined by a pattern
depending on the current file:
To make this work, the current file must be called
`{\textit{prefix}\hspace{0.2em}\textit{suffix}}'
with \textit{prefix} matching precisely the argument.
Processing is then passed on to the file
`{\textit{dest}\hspace{0.2em}\textit{suffix}}'.
Surely, the same effect is achieved by
directly specifying the
argument `{\textit{dest}\hspace{0.2em}\textit{suffix}}'
in the first form.
However, that requires to set up a different file
for each child. With the alternative form of the command
all these files can have exactly the same content
which simplifies setting them up and maintaining them.

For example, the following file |draft.tex|
with a compilation flag |\version| as described in \secref{sec:flags}
compiles the main document as a draft:
%
\begin{center}
\begin{tabular}{l}
|\def\version{draft}|\\
|\input{childdoc.def}|\\
|\childdocforward{|\textit{main}|}|
\end{tabular}
\end{center}
%
Likewise, the following files |final|\textit{nn}|.tex|
compile the final version of the child document
|child|\textit{nn}|.tex|:
%
\begin{center}
\begin{tabular}{l}
|\def\version{final}|\\
|\input{childdoc.def}|\\
|\childdocforwardprefix{final}{child}|
\end{tabular}
\end{center}
%

Note that when several versions of a main file and/or of each child file
are to be generated, it may be convenient to set up a |Makefile| or
shell script to automatise the process.

%%%%%%%%%%%%%%%%%%%%%%%%%%%%%%%%%%%%%%%%%%%%%%%%%%%%%%%%%%%%%%%%%%%%%%%%%%%%%%%%
\subsection{Command Line Processing}
\label{sec:commandline}

The effect of redirection files can also be achieved by invoking
the \LaTeX{} compiler with a more elaborate command line.
Most conveniently this should be done as part
of a shell script or a |Makefile|.

When using \textsf{childdoc} in the main file, the following
command lines effectively perform a redirection
(note that depending on the shell being used,
backslashes may have to be doubled: `|\|' $\to$ `|\\|'):
%
\begin{center}
|... -jobname "|\textit{target}|" |\\|"|[\textit{flags}]%
|\input{childdoc.def}\childdocforward[|\textit{main}|]{|\textit{dest}|}"|
\end{center}
%
Here \textit{target} is the name of the output file,
\textit{main} is the name of the main file
and \textit{dest} is the name of the main or child file to be processed
(all filenames without extensions).
The optional argument \textit{main} can be omitted
if \textit{main} matches \textit{dest}.
Optionally, compilation \textit{flags} can be defined via |\def| commands.
This command line makes the \TeX{} engine believe
it is compiling the file \textit{target}
whose content is specified as the latter parameter.
The provided code then forwards the processing to
\textit{main} or \textit{dest} as described in \secref{sec:forward}.

%%%%%%%%%%%%%%%%%%%%%%%%%%%%%%%%%%%%%%%%%%%%%%%%%%%%%%%%%%%%%%%%%%%%%%%%%%%%%%%%
\subsection{Include by Input}
\label{sec:input}

Including child documents by |\include| has some restrictions by design.
Most notably, the content of a child document always occupies
its own set of pages; pages cannot be shared between child documents.
Usually, this behaviour makes perfect sense
because each child document contain an essential part of the document.
However, in some situations it may be desirable to compose
a document from a collection of parts
without having mandatory page breaks between then.
For this case, the package
provides a mechanism to include parts
by |\input| which can also be processed individually.
However, by construction this mechanism
requires manual handling of the content to be output.

%%%%%%%%%%%%%%%%%%%%%%%%%%%%%%%%%%%%%%%%
\DescribeMacro{\ifchilddocmanual}
The main file should be prepared as usual, see \secref{sec:include}.
However, the document body must make a distinction
between processing of an individual part and of the main document, e.g.:
%
\begin{center}
\begin{tabular}{l}
|\ifchilddocmanual|\\
|\input{\childdocname}|\\
|\||else|\\
\textit{document body with }|\input{|\textit{part}|}|\\
|\||fi|
\end{tabular}
\end{center}
%
The conditional |\ifchilddocmanual| is true whenever
a part to be included by |\input| is being compiled,
and the name of the part is stored in |\childdocname|.

%%%%%%%%%%%%%%%%%%%%%%%%%%%%%%%%%%%%%%%%
\DescribeMacro{\childdocby}
Each part to be included by |\input| should start with:
%
\begin{center}
\begin{tabular}{l}
|\input{childdoc.def}|\\
|\childdocby{|\textit{main}|}|\\
\end{tabular}
\end{center}
%
The directive |\childdocby| is similar to |\childdocof|
described in \secref{sec:include},
but the subsequent selection of content must be done manually.
To that end, both |\ifchilddoc| and |\ifchilddocmanual|
will be true upon processing of a part,
and the name of the part is stored in |\childdocname|.
Note that |\jobname| will be set to the filename of the current part
so that each part receives an individual |.aux| file
that does not interfere with the |.aux| file(s) of the main document.
This behaviour can be altered by the alternative form
|\childdocby[*]{|\textit{main}|}| (with a non-empty optional argument)
which uses the |.aux| file of the main document
by setting |\jobname| to \textit{main}.

%%%%%%%%%%%%%%%%%%%%%%%%%%%%%%%%%%%%%%%%%%%%%%%%%%%%%%%%%%%%%%%%%%%%%%%%%%%%%%%%
\subsection{Driver Development}
\label{sec:driver}

The \textsf{childdoc} mechanism can also be use for the development
of definition files such as \LaTeX{} styles or classes.
This case differs from the above setup with multiple parts
included by |\include| in that no |\includeonly| should be invoked.
This can be achieved by starting the include file
(before |\ProvidesPackage|) with:
%
\begin{center}
\begin{tabular}{l}
|\input{childdoc.def}|\\
|\childdocforward{|\textit{main}|}|\\
\end{tabular}
\end{center}
%
or alternatively with:
%
\begin{center}
\begin{tabular}{l}
|\input{childdoc.def}|\\
|\childdocby{|\textit{main}|}|\\
\end{tabular}
\end{center}
%
Both forms have slightly different effects as described above.
The main file is prepared as usual, see \secref{sec:include}.

%%%%%%%%%%%%%%%%%%%%%%%%%%%%%%%%%%%%%%%%%%%%%%%%%%%%%%%%%%%%%%%%%%%%%%%%%%%%%%%%
\subsection{Legacy Detection}
\label{sec:detection}

The directive |\childdocmain| in the main file can detect
whether the complete document or merely a child is to be compiled
even without using the directive |\childdocof|.
This method is deprecated because it is less robust
and there is no compelling reason to use it;
it is merely provided for backward compatibility
and it may be removed in future versions.

If the detection mechanism is to be used,
it is mandatory to correctly specify
the filename of the main file as the argument of |\childdocmain|:
%
\begin{center}
\begin{tabular}{l}
|\input{childdoc.def}|\\
|\childdocmain{|\textit{main}|}|\\
\end{tabular}
\end{center}
%
If |\jobname| does not match the argument \textit{main} of |\childdocmain|,
it is assumed that |\jobname| points to the child file to be compiled.
When using |\childdocmain| with the main file specified as argument,
it suffices to start a child file
with just |\input{|\textit{main}|}|
without loading of the package and using |\childdocof|.
If instead all processing is done
with the appropriate \textsf{childdoc} directives,
the argument of \textit{main} of |\childdocmain| can be empty.

An alternative version of the command line processing described
in \secref{sec:commandline} using the detection mechanism reads:
%
\begin{center}
|... -jobname "|\textit{target}|" "|[\textit{flags}]%
[|\def\jobname{|\textit{dest}|}|]|\input{|\textit{main}|}"|
\end{center}

%%%%%%%%%%%%%%%%%%%%%%%%%%%%%%%%%%%%%%%%%%%%%%%%%%%%%%%%%%%%%%%%%%%%%%%%%%%%%%%%
\subsection{Manual Code}
\label{sec:manual}

In case one cannot be certain whether the definitions file |childdoc.def|
is installed on the target \TeX{} distribution
and one prefers not to ship it,
it is conceivable to paste a few relevant commands into the sources.

To that end, drop all statements |\input{childdoc.def}|
and perform the replacements as outlined below.
Instead of |\childdocmain{|\textit{main}|}| add the following code
to the top of the main file:
%
\begin{center}
\begin{tabular}{l}
|\||ifdefined\childdocname\endinput\||fi\newif\ifchilddoc|\\
|\edef\childdocname{\scantokens\expandafter{\jobname\noexpand}}|\\
|\def\childdocmain{|\textit{main}|}\||ifx\childdocmain\childdocname\||else|\\
|\childdoctrue\includeonly{\childdocname}\let\jobname\childdocmain\||fi|\\
\end{tabular}
\end{center}
%
Instead of |\childdocof{|\textit{main}|}| just include the main file
at the top of each child file:
%
\begin{center}
|\input{|\textit{main}|}|
\end{center}
%
A simple redirection |\childdocforward{|\textit{dest}|}| is achieved by:
%
\begin{center}
|\def\jobname{|\textit{dest}|}\input{\jobname}|
\end{center}
%
The redirection with prefix
|\childdocforwardprefix[|\textit{prefix}|]{|\textit{dest}|}|
is accomplished by:
%
\begin{center}
\begin{tabular}{l}
|{\edef\jobname{\scantokens\expandafter{\jobname\noexpand}}|\\
|\def\redirectjob |\textit{prefix}|#1~~~{\gdef\jobname{|\textit{dest}|#1}}|\\
|\expandafter\redirectjob\jobname~~~}\input{\jobname}|
\end{tabular}
\end{center}

In an alternative approach,
child documents can be compiled by a specific command line
without additional code or specific definitions:
%
\begin{center}
|... -jobname "|\textit{target}|" "|[\textit{flags}]%
|\includeonly{|\textit{dest}|}\input{|\textit{main}|}"|
\end{center}
%

%%%%%%%%%%%%%%%%%%%%%%%%%%%%%%%%%%%%%%%%%%%%%%%%%%%%%%%%%%%%%%%%%%%%%%%%%%%%%%%%
%%%%%%%%%%%%%%%%%%%%%%%%%%%%%%%%%%%%%%%%%%%%%%%%%%%%%%%%%%%%%%%%%%%%%%%%%%%%%%%%
\section{Information}

%%%%%%%%%%%%%%%%%%%%%%%%%%%%%%%%%%%%%%%%%%%%%%%%%%%%%%%%%%%%%%%%%%%%%%%%%%%%%%%%
\subsection{Copyright}

Copyright \copyright{} 2017--2018 Niklas Beisert

This work may be distributed and/or modified under the
conditions of the \LaTeX{} Project Public License, either version 1.3
of this license or (at your option) any later version.
The latest version of this license is in
  \url{http://www.latex-project.org/lppl.txt}
and version 1.3 or later is part of all distributions of \LaTeX{}
version 2005/12/01 or later.

This work has the LPPL maintenance status `maintained'.

The Current Maintainer of this work is Niklas Beisert.

This work consists of the files |README.txt|, |childdoc.ins| and |childdoc.dtx|
as well as the derived files |childdoc.def|, |cdocsamp.tex|
with |cdocsch1.tex|, |cdocsch2.tex|, |cdocspt3.tex|, |cdocspt4.tex|,
|cdocsdrf.tex|, |cdocsfn1.tex|, |cdocsfn2.tex|
as well as |childdoc.pdf|.

%%%%%%%%%%%%%%%%%%%%%%%%%%%%%%%%%%%%%%%%%%%%%%%%%%%%%%%%%%%%%%%%%%%%%%%%%%%%%%%%
\subsection{Files and Installation}

The package consists of the files:
%
\begin{center}
\begin{tabular}{ll}
    |README.txt|   & readme file \\
    |childdoc.ins| & installation file \\
    |childdoc.dtx| & source file \\
    |childdoc.def| & definition file \\
    |cdocsamp.tex| & sample main file \\
    |cdocsch1.tex| & sample include file \\
    |cdocsch2.tex| & sample include file \\
    |cdocspt3.tex| & sample part file \\
    |cdocspt4.tex| & sample part file \\
    |cdocsdrf.tex| & sample redirection file \\
    |cdocsfn1.tex| & sample redirection file \\
    |cdocsfn2.tex| & sample redirection file \\
    |childdoc.pdf| & manual
\end{tabular}
\end{center}
%
The distribution consists of the files
|README.txt|, |childdoc.ins| and |childdoc.dtx|.
%
\begin{itemize}
\item
Run (pdf)\LaTeX{} on |childdoc.dtx|
to compile the manual |childdoc.pdf| (this file).
\item
Run \LaTeX{} on |childdoc.ins| to create the definitions file |childdoc.def|
and the sample |cdocsamp.tex| with include files
|cdocsch1.tex|, |cdocsch2.tex|, |cdocspt3.tex|, |cdocspt4.tex|,
|cdocsdrf.tex|, |cdocsfn1.tex|, |cdocsfn2.tex|.
Then copy the file |childdoc.def| to an appropriate directory of your \LaTeX{}
distribution, e.g.\ \textit{texmf-root}|/tex/latex/childdoc|.
\end{itemize}

%%%%%%%%%%%%%%%%%%%%%%%%%%%%%%%%%%%%%%%%%%%%%%%%%%%%%%%%%%%%%%%%%%%%%%%%%%%%%%%%
\subsection{Related CTAN Packages}

There are several other packages which offer a similar functionality:
%
\begin{itemize}
\item
The packages
\href{http://ctan.org/pkg/docmute}{\textsf{docmute}},
\href{http://ctan.org/pkg/includex}{\textsf{includex}} and
\href{http://ctan.org/pkg/standalone}{\textsf{standalone}}
provide commands to include only the document body of
a child file thus allowing both files to be compiled individually.
\item
The packages \href{http://ctan.org/pkg/subdocs}{\textsf{subdocs}}
and \href{http://ctan.org/pkg/subfiles}{\textsf{subfiles}}
provide structures in which the main and child documents can be
encapsulated and allowing them to be compiled individually.
The inclusion mechanism is different from the conventional |\include|.
\item
The package \href{http://ctan.org/pkg/combine}{\textsf{combine}}
is an elaborate solution to combine several documents into one.
\end{itemize}
%
See also the CTAN topic \href{http://ctan.org/topic/subdocs}{\textsf{subdocs}}
for further related packages.
The present package differs from the above solutions in that
a document structure constructed with the conventional |\include| mechanism
just needs two extra commands at the top of every file
such that all constituent files can be compiled individually.

%%%%%%%%%%%%%%%%%%%%%%%%%%%%%%%%%%%%%%%%%%%%%%%%%%%%%%%%%%%%%%%%%%%%%%%%%%%%%%%%
%\subsection{Feature Suggestions}
%
%The following is a list of features which may be useful for future
%versions of this package:
%%
%\begin{itemize}
%\item
%\ldots
%\end{itemize}

%%%%%%%%%%%%%%%%%%%%%%%%%%%%%%%%%%%%%%%%%%%%%%%%%%%%%%%%%%%%%%%%%%%%%%%%%%%%%%%%
\subsection{Revision History}

%%%%%%%%%%%%%%%%%%%%%%%%%%%%%%%%%%%%%%%%
\paragraph{v2.0:} 2018/12/30

\begin{itemize}
\item
immediate forward processing
\item
added |\childdocby| mechanism
\item
manual restructured
\end{itemize}

%%%%%%%%%%%%%%%%%%%%%%%%%%%%%%%%%%%%%%%%
\paragraph{v1.6:} 2018/01/17

\begin{itemize}
\item
application for development of include files
\item
corrections to manual
\end{itemize}

%%%%%%%%%%%%%%%%%%%%%%%%%%%%%%%%%%%%%%%%
\paragraph{v1.5:} 2017/05/21

\begin{itemize}
\item
more complete structuring introduced
\item
|\childdocof| introduced
\item
|\childdoc| renamed to |\childdocmain|
\item
|\childredirect| renamed to |\childdocforward| and |\childdocforwardprefix|
and functionality expanded
\end{itemize}

%%%%%%%%%%%%%%%%%%%%%%%%%%%%%%%%%%%%%%%%
\paragraph{v1.0:} 2017/04/27

\begin{itemize}
\item
manual and install package
\item
first version published on CTAN
\end{itemize}

%%%%%%%%%%%%%%%%%%%%%%%%%%%%%%%%%%%%%%%%
\paragraph{v0.6:} 2017/04/26

\begin{itemize}
\item
redirection mechanism added
\end{itemize}

%%%%%%%%%%%%%%%%%%%%%%%%%%%%%%%%%%%%%%%%
\paragraph{v0.5:} 2017/04/26

\begin{itemize}
\item
functionality in definition file
\end{itemize}


%%%%%%%%%%%%%%%%%%%%%%%%%%%%%%%%%%%%%%%%%%%%%%%%%%%%%%%%%%%%%%%%%%%%%%%%%%%%%%%%
%%%%%%%%%%%%%%%%%%%%%%%%%%%%%%%%%%%%%%%%%%%%%%%%%%%%%%%%%%%%%%%%%%%%%%%%%%%%%%%%
%%%%%%%%%%%%%%%%%%%%%%%%%%%%%%%%%%%%%%%%%%%%%%%%%%%%%%%%%%%%%%%%%%%%%%%%%%%%%%%%
\appendix

\settowidth\MacroIndent{\rmfamily\scriptsize 000\ }

 \DocInput{childdoc.dtx}

\end{document}
%</driver>
% \fi
%
% %%%%%%%%%%%%%%%%%%%%%%%%%%%%%%%%%%%%%%%%%%%%%%%%%%%%%%%%%%%%%%%%%%%%%%%%%%%%%%
% %%%%%%%%%%%%%%%%%%%%%%%%%%%%%%%%%%%%%%%%%%%%%%%%%%%%%%%%%%%%%%%%%%%%%%%%%%%%%%
% \section{Sample}
%\iffalse
%<*samplemain>
%\fi
%
% The following presents a sample document
% with two chapters, two parts, a title page,
% a compile flag as well as three forwarding files to set the flag.
% It consists of eight |.tex| files:
% \begin{center}
% \begin{tabular}{ll}
% |cdocsamp.tex|&main file\\
% |cdocsch1.tex|&include file for chapter 1\\
% |cdocsch2.tex|&include file for chapter 2\\
% |cdocspt3.tex|&include file for part 3\\
% |cdocspt4.tex|&include file for part 4\\
% |cdocsdrf.tex|&forwarding file for main file in draft mode\\
% |cdocsfi1.tex|&forwarding file for final version of chapter 1\\
% |cdocsfi2.tex|&forwarding file for final version of chapter 2\\
% \end{tabular}
% \end{center}
% Each of the eight files can be compiled directly by the \LaTeX{} compiler.
%
% %%%%%%%%%%%%%%%%%%%%%%%%%%%%%%%%%%%%%%
% \paragraph{Main File.}
%
% The main file is called |cdocsamp.tex|.
%
% Load the \textsf{childdoc} definitions and
% declare the filename for the main document:
%    \begin{macrocode}
\input{childdoc.def}
\childdocmain{}
%    \end{macrocode}

% Optional override for |\version| flag:
%    \begin{macrocode}
%%\ifchilddoc\else\providecommand{\version}{draft}\fi
%    \end{macrocode}

% Define the default values for the |\version| flag
% (|final| for the main file and |draft| for childs):
%    \begin{macrocode}
\ifchilddoc
\providecommand{\version}{draft}
\else
\providecommand{\version}{final}
\fi
%    \end{macrocode}

% Load the standard document class:
%    \begin{macrocode}
\documentclass[12pt]{article}
%    \end{macrocode}

% Start the document body:
%    \begin{macrocode}
\begin{document}
%    \end{macrocode}

% Declare a title page.
% Print title, part of document being processed and version flag:
%    \begin{macrocode}
\addtocounter{page}{-1}
\begin{center}
{\LARGE\bfseries{}childdoc example\par}
\vspace{1cm}
\ifchilddoc
\ifchilddocmanual part\else chapter\fi:
`\childdocname' of `\childdocjob'\par
\else
main document: `\childdocjob'\par
\fi
version: \version\par
\end{center}
\newpage
%    \end{macrocode}

% Manually include selected file,
% otherwise process as usual:
%    \begin{macrocode}
\ifchilddocmanual
\section*{part `\childdocname'}
\input{\childdocname}
\else
%    \end{macrocode}

% Include the two chapters:
%    \begin{macrocode}
\include{cdocsch1}
\include{cdocsch2}
%    \end{macrocode}

% Include the two parts unless only chapters should be displayed:
%    \begin{macrocode}
\ifchilddoc\else
\section{part three}
\input{cdocspt3}
\section{part four}
\input{cdocspt4}
\fi
%    \end{macrocode}

% Process as usual until here:
%    \begin{macrocode}
\fi
%    \end{macrocode}

% End of document body:
%    \begin{macrocode}
\end{document}
%    \end{macrocode}
%\iffalse
%</samplemain>
%\fi
%
% %%%%%%%%%%%%%%%%%%%%%%%%%%%%%%%%%%%%%%
% \paragraph{Chapter Include Files.}
%
% The include files are called |cdocsch1.tex| and |cdocsch2.tex|.
%
%\iffalse
%<*samplechap1|samplechap2>
%\fi

% Optional override for |\version| flag:
%    \begin{macrocode}
%%\providecommand{\version}{final}
%    \end{macrocode}

% Include the main document:
%    \begin{macrocode}
\input{childdoc.def}
\childdocof{cdocsamp}
%    \end{macrocode}

%\iffalse
%</samplechap1|samplechap2>
%\fi
%
%\iffalse
%<*samplechap1>
%\fi
% Some text for chapter 1:
%    \begin{macrocode}
\section{one}
some text in chapter one
%    \end{macrocode}

%\iffalse
%</samplechap1>
%\fi
% Some text for chapter 2:
%\iffalse
%<*samplechap2>
%\fi
%    \begin{macrocode}
\section{two}
more text in chapter two
%    \end{macrocode}

%\iffalse
%</samplechap2>
%\fi
%
% %%%%%%%%%%%%%%%%%%%%%%%%%%%%%%%%%%%%%%
% \paragraph{Part Include Files.}
%
% The include files are called |cdocspt3.tex| and |cdocspt4.tex|.
%
%\iffalse
%<*samplepart3|samplepart4>
%\fi

% Optional override for |\version| flag:
%    \begin{macrocode}
%%\providecommand{\version}{final}
%    \end{macrocode}

% Include the main document:
%    \begin{macrocode}
\input{childdoc.def}
\childdocby{cdocsamp}
%    \end{macrocode}

%\iffalse
%</samplepart3|samplepart4>
%\fi
%
%\iffalse
%<*samplepart3>
%\fi
% Some text for part 3:
%    \begin{macrocode}
some text in part three
%    \end{macrocode}

%\iffalse
%</samplepart3>
%\fi
% Some text for part 4:
%\iffalse
%<*samplepart4>
%\fi
%    \begin{macrocode}
more text in part four
%    \end{macrocode}

%\iffalse
%</samplepart4>
%\fi
%
% %%%%%%%%%%%%%%%%%%%%%%%%%%%%%%%%%%%%%%
% \paragraph{Forwarding for a Complete Draft.}
%
% The following forwarding file |cdocsdrf.tex|
% compiles the main document in draft mode:
%\iffalse
%<*sampledraft>
%\fi
%    \begin{macrocode}
\def\version{draft}
\input{childdoc.def}
\childdocforward{cdocsamp}
%    \end{macrocode}

%\iffalse
%</sampledraft>
%\fi
%
% %%%%%%%%%%%%%%%%%%%%%%%%%%%%%%%%%%%%%%
% \paragraph{Forwarding for Final Version of the Chapters.}
%
% The following forwarding files |cdocsfn1.tex| and |cdocsfn2.tex|
% (with identical content)
% compile the final versions of the child documents
% |cdocsch1.tex| and |cdocsch2.tex|, respectively:
%\iffalse
%<*samplefinal>
%\fi
%    \begin{macrocode}
\def\version{final}
\input{childdoc.def}
\childdocforwardprefix[cdocsamp]{cdocsfn}{cdocsch}
%    \end{macrocode}

%\iffalse
%</samplefinal>
%\fi
%
% %%%%%%%%%%%%%%%%%%%%%%%%%%%%%%%%%%%%%%
% \paragraph{Command Line Processing.}
%
% The following three command lines generate the output files
% |cdocscld|, |cdocscl1| and |cdocscl2|
% which should be identical to
% |cdocsdrf|, |cdocsch1| and |cdocsfn2|, respectively:
% \begin{center}
% \begin{tabular}{l}
% |latex -jobname cdocscld \|\\
% |  "\def\version{draft}\input{childdoc.def}\childdocforward{cdocsamp}"|\\
% |latex -jobname cdocscl1 \|\\
% |  "\input{childdoc.def}\childdocforward[cdocsamp]{cdocsch1}"|\\
% |latex -jobname cdocscl2 \|\\
% |  "\def\version{final}\input{childdoc.def}\childdocforward{cdocsch2}"|
% \end{tabular}
% \end{center}
% Note that the trailing backslash on each first line
% merely continues the input to the second line
% (for convenient cut ant paste).
% Furthermore, the command |latex| can be replaced by any
% of its alternative versions such as |pdflatex|.
%
% %%%%%%%%%%%%%%%%%%%%%%%%%%%%%%%%%%%%%%%%%%%%%%%%%%%%%%%%%%%%%%%%%%%%%%%%%%%%%%
% %%%%%%%%%%%%%%%%%%%%%%%%%%%%%%%%%%%%%%%%%%%%%%%%%%%%%%%%%%%%%%%%%%%%%%%%%%%%%%
% \section{Implementation}
%\iffalse
%<*package>
%\fi
%
% This section describes the definitions file |childdoc.def|.

% The definitions cannot be loaded using |\usepackage| or |\RequirePackage|
% which has a mechanism to prevent loading a style file more than once.
% When loading the definitions by means of |\input|
% multiple instances have to be prevented manually:
%\iffalse
%This code needs to be before the `\ProvidesFile' directive
%which is defined at the beginning of this file.
%Therefore it is also placed there and commented out here.
%</package>
%<*discard>
%\fi
%    \begin{macrocode}
\ifdefined\childdocmain\endinput\fi
%    \end{macrocode}
%\iffalse
%</discard>
%<*package>
%\fi
%
% \macro{\ifchilddoc}
% \macro{\ifchilddocmanual}
% The conditional |\ifchilddoc| tells whether a
% child (true) or main (false) document is being compiled.
% The conditional |\ifchilddocmanual| tells whether
% the |\includeonly| mechanism is used (false) or
% the selection of child files must be performed manually (true).
% The definitions initialise to false:
%    \begin{macrocode}
\newif\ifchilddoc
\newif\ifchilddocmanual
%    \end{macrocode}

% \macro{\childdocname}
% \macro{\childdocjob}
% The macro |\childdocname| stores the name of the main document
% to be compiled. The macro |\childdocjob| stores the name of
% the document on which the \LaTeX{} compiler was originally invoked.
% The content of |\jobname| cannot be compared
% to filenames specified in the source due to different catcodes.
% The following code rescans |\jobname|, stores the result
% in |\childdocname| and saves a copy in |\childdocjob|:
%    \begin{macrocode}
\edef\childdocname{\scantokens\expandafter{\jobname\noexpand}}
\let\childdocjob\childdocname
%    \end{macrocode}

% \macro{\childdocdisable}
% The macro |\childdocdisable| prevents the main file
% from being processed more than once.
% At this stage, the main document command |\childdocmain|
% is assumed to be called once again where it should do nothing.
% Any subsequent call to it should prevent
% a secondary processing of the main document
% It overwrites the forwarding commands
% |\childdocof| and |\childdocforward|
% with empty macros to prevent further inclusions of the main document:
%    \begin{macrocode}
\newcommand{\childdocdisable}
{
  \renewcommand{\childdocmain}[1]{\renewcommand{\childdocmain}[1]{\endinput}}
  \renewcommand{\childdocof}[1]{}
  \renewcommand{\childdocby}[2][]{}
  \renewcommand{\childdocforward}[2][]{}
  \renewcommand{\childdocdisable}{}
}
%    \end{macrocode}

% \macro{\childdocmain}
% The macro |\childdocmain| is to be called at the top of the main file
% with nothing or the main filename (without extension) as argument.
% First, it breaks loops.
% If the argument is not empty and does not match |\childdocname|
% (which is set by the first inclusion of |childdoc.def|),
% |\ifchilddoc| is set to true, |\includeonly| is applied to the child file
% and |\jobname| is set to the main file
% (for proper handling of |.aux| files):
%    \begin{macrocode}
\newcommand{\childdocmain}[1]
{
  \childdocdisable\childdocmain{}
  \if?#1?\else
    \begingroup
      \def\childdoctmp{#1}
      \ifx\childdoctmp\childdocname
        \def\childdoctmp{}
      \else
        \def\childdoctmp
        {
          \childdoctrue
          \includeonly{\childdocname}
          \def\childdocjob{#1}
          \def\jobname{#1}
        }
      \fi
      \expandafter
    \endgroup
    \childdoctmp
  \fi
}
%    \end{macrocode}

% \macro{\childdocof}
% The command |\childdocof| redirects
% compilation to the main file |#1|.
%    \begin{macrocode}
\newcommand{\childdocof}[1]
{
  \childdocdisable
  \childdoctrue
  \includeonly{\childdocname}
  \def\jobname{#1}
  \def\childdocjob{#1}
  \input{#1}
}
%    \end{macrocode}

% \macro{\childdocby}
% The command |\childdocby| ....
%    \begin{macrocode}
\newcommand{\childdocby}[2][]
{
  \childdocdisable
  \childdoctrue
  \childdocmanualtrue
  \if?#1?\else
    \def\jobname{#2}
  \fi
  \def\childdocjob{#2}
  \input{#2}
  \endinput
}
%    \end{macrocode}

% \macro{\childdocforward}
% The command |\childdocforward| redirects
% compilation to the main file or
% (if the optional argument is given) a child file.
% Parameters are set as if the main file
% or a child file starting with |\childdocof| was compiled.
% Then compilation is handed over to the main file:
%    \begin{macrocode}
\newcommand{\childdocforward}[2][]
{
  \begingroup
    \if?#1?
      \def\childdoctmp
      {
        \def\childdocname{#2}
        \def\childdocjob{#2}
        \def\jobname{#2}
        \input{#2}
        \endinput
      }
    \else
      \def\childdoctmp
      {
        \childdocdisable
        \def\childdocname{#2}
        \childdoctrue
        \includeonly{#2}
        \def\childdocjob{#1}
        \def\jobname{#1}
        \input{#1}
        \endinput
      }
    \fi
    \expandafter
  \endgroup
  \childdoctmp
}
%    \end{macrocode}

% \macro{\childdocforwardprefix}
% The command |\childdocforwardprefix| redirects
% compilation to the main or a child file by means of a pattern.
% The prefix |#1| in the current filename is replaced by |#2|
% and the suffix of the current filename is kept
% (it is assumed that the filename does not contain the substring `|~~~|'
% which is used as a delimiter).
% Compilation is handed over to the new file by |\childdocforward|:
%    \begin{macrocode}
\newcommand{\childdocforwardprefix}[3][]
{
  \begingroup
    \def\childdocextract #2##1~~~{\def\childdoctmp{\childdocforward[#1]{#3##1}}}
    \expandafter\childdocextract\childdocname~~~
    \expandafter
  \endgroup
  \childdoctmp
}
%    \end{macrocode}

% \macro{\childdoc}
% The deprecated macro |\childdoc| is a legacy version of |\childdocmain|:
%    \begin{macrocode}
\newcommand{\childdoc}{\childdocmain}
%    \end{macrocode}

% \macro{\childdocredirect}
% The deprecated macro |\childdocredirect| is a legacy version
% of |\childdocforward| and |\childdocforwardprefix|:
%    \begin{macrocode}
\newcommand{\childdocredirect}[2][]
{
  \begingroup
    \if?#1?
      \def\childdoctmp{\childdocforward{#2}}
    \else
      \def\childdoctmp{\childdocforwardprefix{#1}{#2}}
    \fi
    \expandafter
  \endgroup
  \childdoctmp
}
%    \end{macrocode}

%\iffalse
%</package>
%\fi
%
\endinput
|\\
|\childdocof{|\textit{main}|}|\\
\end{tabular}
\end{center}
at the top of every child file \textit{child}
which is included by |\include{|\textit{child}|}|
from within the main file
(or at least for those files to be compiled individually).
The argument \textit{main} must be the filename of the main file.

There are a couple of
considerations in setting up the main and child documents:

%%%%%%%%%%%%%%%%%%%%%%%%%%%%%%%%%%%%%%%%
\paragraph{Restrictions.}

Please note the following restrictions:
\begin{itemize}
\item
|\childdocmain| must be called with one argument \textit{main}
to ensure compatibility with earlier version of the package.
It must either be empty (|\childdocmain{}|)
or precisely match the filename of the main file in which it is specified.
See \secref{sec:detection} for further information.
\item
The filename \textit{main} must be specified without the |.tex| extension.
\item
The filename \textit{main} is case sensitive
(even in case-insensitive file systems)
due to internal string comparison.
\item
The argument \textit{main} should be fully expanded, it cannot be a macro.
\item
Subdirectories and special characters should be avoided in filenames.
\item
The command |\childdocmain{|\textit{main}|}| must be followed by a whitespace.
It should not be followed immediately by another command
or by a comment mark `|%|'.
This is because the \TeX{} parser reads the token immediately following
the argument of |\childdocmain| and puts it
at the beginning of every child section;
however, a white\-space is ignored.
\end{itemize}

%%%%%%%%%%%%%%%%%%%%%%%%%%%%%%%%%%%%%%%%
\paragraph{Content of Main File.}

It is advisable to place all content in the child files included by |\include|.
Any output contained in the main file will appear in all child documents
unless suppressed manually;
it cannot be suppressed automatically by the |\includeonly| directive
and thus should normally be avoided.
A method to include some content in the main file
by means of conditional processing is described in \secref{sec:conditional}.

%%%%%%%%%%%%%%%%%%%%%%%%%%%%%%%%%%%%%%%%
\paragraph{Page Numbering.}

When only a part of the document is compiled,
the appropriate numbering of pages
(as well as other status parameters)
is determined from the |.aux| files.
The latter contain information from previous passes.
However this information needs to propagate through
all intermediate child documents.
Therefore the page numbering in child documents may well
be inconsistent until the complete document is compiled at least once.

A useful (if unconventional) way to always ensure a consistent
page numbering is to restart the numbering in each child document
and denote the pages by `\textit{child}|.|\textit{page}'
where \textit{child} represents the chapter/section number of the child file.
This can be achieved by the command
|\numberwithin{page}{|\textit{child}|}|
of the \textsf{amsmath} package
where \textit{child} can be |chapter| or |section|
depending on the chosen structuring.
Alternatively, one can modify the macro |\thepage| appropriately
and reset the counter |page| at the start of each child file.

%%%%%%%%%%%%%%%%%%%%%%%%%%%%%%%%%%%%%%%%%%%%%%%%%%%%%%%%%%%%%%%%%%%%%%%%%%%%%%%%
\subsection{Conditional Processing}
\label{sec:conditional}

The package provides a mechanism to compile different versions
of a document. To customise the versions further some conditional processing
can come in handy to distinguish which version is being compiled.
The package provides two macros to describe the compilation context:

%%%%%%%%%%%%%%%%%%%%%%%%%%%%%%%%%%%%%%%%
\DescribeMacro{\ifchilddoc}
The conditional |\ifchilddoc| distinguishes between the compilation of
child documents and the main document:
%
\begin{center}
|\ifchilddoc |\textit{child-code}| |[|\||else |\textit{main-code}]| \||fi|
\end{center}

%%%%%%%%%%%%%%%%%%%%%%%%%%%%%%%%%%%%%%%%
\DescribeMacro{\childdocname}
\DescribeMacro{\childdocjob}
The macro |\childdocname| contains the filename (without extension)
of the main or child file being processed.
Note that |\childdocjob| will always contain the name of the main file.

%%%%%%%%%%%%%%%%%%%%%%%%%%%%%%%%%%%%%%%%
\paragraph{Title Page.}

Conditional processing can be used to include a title or banner page
in the main document when proper precautions are taken.
Importantly, the code in the main file should ensure that the page counter
(as well as other status parameters which are stored in the |.aux| files)
takes the same value after the conditional processing.
Otherwise the page numbers may take divergent values
depending on which part is compiled.

For example, a title page could be declared by:
%
\begin{center}
\begin{tabular}{l}
|\ifchilddoc\||else|\\
|\addtocounter{page}{-1}|\\
\textit{code for title page}\\
|\newpage|\\
|\||fi|
\end{tabular}
\end{center}
%
A banner page for the child documents can be generated by:
%
\begin{center}
\begin{tabular}{l}
|\ifchilddoc|\\
|\addtocounter{page}{-1}|\\
\textit{code for banner page}\\
|\newpage|\\
|\||fi|
\end{tabular}
\end{center}
%
Here one could write a message such as:
\begin{center}
|This is the part \childdocname{} of \childdocjob{}.|
\end{center}

%%%%%%%%%%%%%%%%%%%%%%%%%%%%%%%%%%%%%%%%%%%%%%%%%%%%%%%%%%%%%%%%%%%%%%%%%%%%%%%%
\subsection{Flags}
\label{sec:flags}

The package makes it easy to generate different versions
of the main or child documents.
To this end compilation flags can be defined
and assigned different default values.
They will be particularly useful in conjunction
with the forwarding mechanism described in \secref{sec:forward}.

For example, it may be useful to have a flag |\version|
which can be set to |draft| or |final|.
The document source will contain some conditional code
depending on the value of |\version|.
Suppose further, the flag should default to |final| for the main file
and to |draft| for child files
which is a natural assignment for editing the document.
This is achieved by placing the following code
in the preamble of the main document
(below the |\childdocmain| directive):
%
\begin{center}
\begin{tabular}{l}
|\ifchilddoc|\\
|\providecommand{\version}{draft}|\\
|\||else|\\
|\providecommand{\version}{final}|\\
|\||fi|
\end{tabular}
\end{center}
%
The definition by |\providecommand| makes sure
that previous definitions are not overwritten.
Further statements |\providecommand{\version}{...}|
can thus be added before the above code to override it.

For the main file, one might add a line
(between |\childdocmain| and the above block)
%
\begin{center}
|%\ifchilddoc\||else\providecommand{\version}{draft}\||fi|
\end{center}
%
which can be uncommented to produce a draft version.
Likewise one can add a line to the very top of a child file
(above the |\childdocof{|\textit{main}|}| directive)
%
\begin{center}
|%\providecommand{\version}{final}|
\end{center}
%
which can be uncommented to produce the final version of this child document.

%%%%%%%%%%%%%%%%%%%%%%%%%%%%%%%%%%%%%%%%%%%%%%%%%%%%%%%%%%%%%%%%%%%%%%%%%%%%%%%%
\subsection{Forwarding}
\label{sec:forward}

Different versions of the main or child documents
using compilation flags as described in \secref{sec:flags}
can be (permanently) stored in different files
for convenient compilation, viewing and distribution.
To this end, the package defines a command
to pass on compilation to a different file:

%%%%%%%%%%%%%%%%%%%%%%%%%%%%%%%%%%%%%%%%
\DescribeMacro{\childdocforward}
The command |\childdocforward| redirects processing to
another source file:
%
\begin{center}
\begin{tabular}{l}
|% \iffalse
%
% childdoc.dtx Copyright (C) 2017-2018 Niklas Beisert
%
% This work may be distributed and/or modified under the
% conditions of the LaTeX Project Public License, either version 1.3
% of this license or (at your option) any later version.
% The latest version of this license is in
%   http://www.latex-project.org/lppl.txt
% and version 1.3 or later is part of all distributions of LaTeX
% version 2005/12/01 or later.
%
% This work has the LPPL maintenance status `maintained'.
%
% The Current Maintainer of this work is Niklas Beisert.
%
% This work consists of the files childdoc.dtx and childdoc.ins
% and the derived files childdoc.def and cdocsamp.tex with
% cdocsch1.tex, cdocsch2.tex, cdocsdrf.tex, cdocsfn1.tex, cdocsfn2.tex.
%
%<package>\ifdefined\childdocmain\endinput\fi
%<package>\ProvidesFile{childdoc.def}[2018/12/30 v2.0 child document driver]
%<samplemain>\ProvidesFile{cdocsamp.tex}[2018/12/30 v2.0 sample for childdoc]
%<*driver>
%\ProvidesFile{childdoc.drv}[2018/12/30 v2.0 childdoc reference manual file]
\PassOptionsToClass{10pt,a4paper}{article}
\documentclass{ltxdoc}

\usepackage[margin=35mm]{geometry}
\usepackage{hyperref}
\usepackage{hyperxmp}
\usepackage[usenames]{color}

\hypersetup{colorlinks=true}
\hypersetup{pdfstartview=FitH}
\hypersetup{pdfpagemode=UseNone}
\hypersetup{pdfsource={}}
\hypersetup{pdflang={en-UK}}
\hypersetup{pdfcopyright={Copyright 2017-2018 Niklas Beisert.
  This work may be distributed and/or modified under the
  conditions of the LaTeX Project Public License, either version 1.3
  of this license or (at your option) any later version.}}
\hypersetup{pdflicenseurl={http://www.latex-project.org/lppl.txt}}
\hypersetup{pdfcontactaddress={ETH Zurich, ITP, HIT K,
  Wolfgang-Pauli-Strasse 27}}
\hypersetup{pdfcontactpostcode={8093}}
\hypersetup{pdfcontactcity={Zurich}}
\hypersetup{pdfcontactcountry={Switzerland}}
\hypersetup{pdfcontactemail={nbeisert@itp.phys.ethz.ch}}
\hypersetup{pdfcontacturl={http://people.phys.ethz.ch/\xmptilde nbeisert/}}

\newcommand{\secref}[1]{\hyperref[#1]{section \ref*{#1}}}

\parskip1ex
\parindent0pt
\let\olditemize\itemize
\def\itemize{\olditemize\parskip0pt}

\begin{document}

\title{The \textsf{childdoc} Package}
\hypersetup{pdftitle={The childdoc Package}}
\author{Niklas Beisert\\[2ex]
  Institut f\"ur Theoretische Physik\\
  Eidgen\"ossische Technische Hochschule Z\"urich\\
  Wolfgang-Pauli-Strasse 27, 8093 Z\"urich, Switzerland\\[1ex]
  \href{mailto:nbeisert@itp.phys.ethz.ch}
  {\texttt{nbeisert@itp.phys.ethz.ch}}}
\hypersetup{pdfauthor={Niklas Beisert}}
\hypersetup{pdfsubject={Manual for the LaTeX2e Package childdoc}}
\date{30 December 2018, \textsf{v2.0}}
\maketitle

\begin{abstract}\noindent
\textsf{childdoc} is a \LaTeXe{} package
that enables the direct compilation
of document sections included by |\include|
to individual files.
\end{abstract}

\begingroup
\parskip0ex
\tableofcontents
\endgroup

%%%%%%%%%%%%%%%%%%%%%%%%%%%%%%%%%%%%%%%%%%%%%%%%%%%%%%%%%%%%%%%%%%%%%%%%%%%%%%%%
%%%%%%%%%%%%%%%%%%%%%%%%%%%%%%%%%%%%%%%%%%%%%%%%%%%%%%%%%%%%%%%%%%%%%%%%%%%%%%%%
\section{Introduction}

\LaTeX{} provides a mechanism to structure a large document (such as a book)
into a main file and several child files (containing the chapters)
using the |\include| command.
This mechanism is beneficial for documents
which span hundreds of pages in order to
make the source file(s) more manageable.
Moreover, compilation can be restricted to
selected child files by means of the |\includeonly| command.
The latter feature can be used to reduce the compilation time while editing
(this was significantly more useful in the earlier days of \LaTeX{})
or to generate a smaller document which is easier to navigate.
Another application of |\includeonly| is to generate
documents consisting of selected parts of the complete document.

However, there are a few drawbacks of the plain |\include| mechanism:
\begin{itemize}
\item
The child files cannot be compiled on their own,
they can only be compiled via the main file.
A naive editing environment
(such as a text editor with an option
to have the current file processed by \LaTeX)
may require one to switch to the main file before compiling;
attempting to compile the child file produces errors.
\item
The main file must be modified (each time)
to adjust the |\includeonly| command
to the present needs. This easily leaves the main file in a messy state.
\item
The generated document will always carry the filename
of the main document. This is inconvenient if
several child files are to be compiled and
to be kept for distribution.
\end{itemize}

The present package provides a simple interface
to make child files individually compilable by \LaTeX{}.
Compiling a child file then has the same effect as compiling
the main file with an |\includeonly| command
to select the appropriate child.
Moreover the generated document will carry the name of the child
rather than the main file.
This resolves all three above issues.

This feature is meant to make the editing of books,
thesis documents and lecture notes somewhat more convenient.
However, the package can also be used efficiently for
composing a series of documents (such as exercise sheets)
which are typically distributed individually.
It then assists the author in generating the individual documents
(potentially in different versions)
as well as a document containing the collected series.
Another application is in developing style files
or other kinds of included material
where compilation of the style file could redirect
to a sample or test file.

%%%%%%%%%%%%%%%%%%%%%%%%%%%%%%%%%%%%%%%%%%%%%%%%%%%%%%%%%%%%%%%%%%%%%%%%%%%%%%%%
%%%%%%%%%%%%%%%%%%%%%%%%%%%%%%%%%%%%%%%%%%%%%%%%%%%%%%%%%%%%%%%%%%%%%%%%%%%%%%%%
\section{Usage}

First of all, the package \textsf{childdoc} is \emph{not} a standard
\LaTeXe{} |.sty| style file! Therefore it needs to be invoked in
a non-standard way.

%%%%%%%%%%%%%%%%%%%%%%%%%%%%%%%%%%%%%%%%%%%%%%%%%%%%%%%%%%%%%%%%%%%%%%%%%%%%%%%%
\subsection{Included Files}
\label{sec:include}

%%%%%%%%%%%%%%%%%%%%%%%%%%%%%%%%%%%%%%%%
\DescribeMacro{\childdocmain}
To use the package, add the commands
\begin{center}
\begin{tabular}{l}
|\input{childdoc.def}|\\
|\childdocmain{}|\\
\end{tabular}
\end{center}
at the very top of the main \LaTeX{} file,
in particular \emph{before} the |\documentclass| statement!
The argument of |\childdocmain| should be left empty
(but it must be present).

%%%%%%%%%%%%%%%%%%%%%%%%%%%%%%%%%%%%%%%%
\DescribeMacro{\childdocof}
Furthermore, add the commands
\begin{center}
\begin{tabular}{l}
|\input{childdoc.def}|\\
|\childdocof{|\textit{main}|}|\\
\end{tabular}
\end{center}
at the top of every child file \textit{child}
which is included by |\include{|\textit{child}|}|
from within the main file
(or at least for those files to be compiled individually).
The argument \textit{main} must be the filename of the main file.

There are a couple of
considerations in setting up the main and child documents:

%%%%%%%%%%%%%%%%%%%%%%%%%%%%%%%%%%%%%%%%
\paragraph{Restrictions.}

Please note the following restrictions:
\begin{itemize}
\item
|\childdocmain| must be called with one argument \textit{main}
to ensure compatibility with earlier version of the package.
It must either be empty (|\childdocmain{}|)
or precisely match the filename of the main file in which it is specified.
See \secref{sec:detection} for further information.
\item
The filename \textit{main} must be specified without the |.tex| extension.
\item
The filename \textit{main} is case sensitive
(even in case-insensitive file systems)
due to internal string comparison.
\item
The argument \textit{main} should be fully expanded, it cannot be a macro.
\item
Subdirectories and special characters should be avoided in filenames.
\item
The command |\childdocmain{|\textit{main}|}| must be followed by a whitespace.
It should not be followed immediately by another command
or by a comment mark `|%|'.
This is because the \TeX{} parser reads the token immediately following
the argument of |\childdocmain| and puts it
at the beginning of every child section;
however, a white\-space is ignored.
\end{itemize}

%%%%%%%%%%%%%%%%%%%%%%%%%%%%%%%%%%%%%%%%
\paragraph{Content of Main File.}

It is advisable to place all content in the child files included by |\include|.
Any output contained in the main file will appear in all child documents
unless suppressed manually;
it cannot be suppressed automatically by the |\includeonly| directive
and thus should normally be avoided.
A method to include some content in the main file
by means of conditional processing is described in \secref{sec:conditional}.

%%%%%%%%%%%%%%%%%%%%%%%%%%%%%%%%%%%%%%%%
\paragraph{Page Numbering.}

When only a part of the document is compiled,
the appropriate numbering of pages
(as well as other status parameters)
is determined from the |.aux| files.
The latter contain information from previous passes.
However this information needs to propagate through
all intermediate child documents.
Therefore the page numbering in child documents may well
be inconsistent until the complete document is compiled at least once.

A useful (if unconventional) way to always ensure a consistent
page numbering is to restart the numbering in each child document
and denote the pages by `\textit{child}|.|\textit{page}'
where \textit{child} represents the chapter/section number of the child file.
This can be achieved by the command
|\numberwithin{page}{|\textit{child}|}|
of the \textsf{amsmath} package
where \textit{child} can be |chapter| or |section|
depending on the chosen structuring.
Alternatively, one can modify the macro |\thepage| appropriately
and reset the counter |page| at the start of each child file.

%%%%%%%%%%%%%%%%%%%%%%%%%%%%%%%%%%%%%%%%%%%%%%%%%%%%%%%%%%%%%%%%%%%%%%%%%%%%%%%%
\subsection{Conditional Processing}
\label{sec:conditional}

The package provides a mechanism to compile different versions
of a document. To customise the versions further some conditional processing
can come in handy to distinguish which version is being compiled.
The package provides two macros to describe the compilation context:

%%%%%%%%%%%%%%%%%%%%%%%%%%%%%%%%%%%%%%%%
\DescribeMacro{\ifchilddoc}
The conditional |\ifchilddoc| distinguishes between the compilation of
child documents and the main document:
%
\begin{center}
|\ifchilddoc |\textit{child-code}| |[|\||else |\textit{main-code}]| \||fi|
\end{center}

%%%%%%%%%%%%%%%%%%%%%%%%%%%%%%%%%%%%%%%%
\DescribeMacro{\childdocname}
\DescribeMacro{\childdocjob}
The macro |\childdocname| contains the filename (without extension)
of the main or child file being processed.
Note that |\childdocjob| will always contain the name of the main file.

%%%%%%%%%%%%%%%%%%%%%%%%%%%%%%%%%%%%%%%%
\paragraph{Title Page.}

Conditional processing can be used to include a title or banner page
in the main document when proper precautions are taken.
Importantly, the code in the main file should ensure that the page counter
(as well as other status parameters which are stored in the |.aux| files)
takes the same value after the conditional processing.
Otherwise the page numbers may take divergent values
depending on which part is compiled.

For example, a title page could be declared by:
%
\begin{center}
\begin{tabular}{l}
|\ifchilddoc\||else|\\
|\addtocounter{page}{-1}|\\
\textit{code for title page}\\
|\newpage|\\
|\||fi|
\end{tabular}
\end{center}
%
A banner page for the child documents can be generated by:
%
\begin{center}
\begin{tabular}{l}
|\ifchilddoc|\\
|\addtocounter{page}{-1}|\\
\textit{code for banner page}\\
|\newpage|\\
|\||fi|
\end{tabular}
\end{center}
%
Here one could write a message such as:
\begin{center}
|This is the part \childdocname{} of \childdocjob{}.|
\end{center}

%%%%%%%%%%%%%%%%%%%%%%%%%%%%%%%%%%%%%%%%%%%%%%%%%%%%%%%%%%%%%%%%%%%%%%%%%%%%%%%%
\subsection{Flags}
\label{sec:flags}

The package makes it easy to generate different versions
of the main or child documents.
To this end compilation flags can be defined
and assigned different default values.
They will be particularly useful in conjunction
with the forwarding mechanism described in \secref{sec:forward}.

For example, it may be useful to have a flag |\version|
which can be set to |draft| or |final|.
The document source will contain some conditional code
depending on the value of |\version|.
Suppose further, the flag should default to |final| for the main file
and to |draft| for child files
which is a natural assignment for editing the document.
This is achieved by placing the following code
in the preamble of the main document
(below the |\childdocmain| directive):
%
\begin{center}
\begin{tabular}{l}
|\ifchilddoc|\\
|\providecommand{\version}{draft}|\\
|\||else|\\
|\providecommand{\version}{final}|\\
|\||fi|
\end{tabular}
\end{center}
%
The definition by |\providecommand| makes sure
that previous definitions are not overwritten.
Further statements |\providecommand{\version}{...}|
can thus be added before the above code to override it.

For the main file, one might add a line
(between |\childdocmain| and the above block)
%
\begin{center}
|%\ifchilddoc\||else\providecommand{\version}{draft}\||fi|
\end{center}
%
which can be uncommented to produce a draft version.
Likewise one can add a line to the very top of a child file
(above the |\childdocof{|\textit{main}|}| directive)
%
\begin{center}
|%\providecommand{\version}{final}|
\end{center}
%
which can be uncommented to produce the final version of this child document.

%%%%%%%%%%%%%%%%%%%%%%%%%%%%%%%%%%%%%%%%%%%%%%%%%%%%%%%%%%%%%%%%%%%%%%%%%%%%%%%%
\subsection{Forwarding}
\label{sec:forward}

Different versions of the main or child documents
using compilation flags as described in \secref{sec:flags}
can be (permanently) stored in different files
for convenient compilation, viewing and distribution.
To this end, the package defines a command
to pass on compilation to a different file:

%%%%%%%%%%%%%%%%%%%%%%%%%%%%%%%%%%%%%%%%
\DescribeMacro{\childdocforward}
The command |\childdocforward| redirects processing to
another source file:
%
\begin{center}
\begin{tabular}{l}
|\input{childdoc.def}|\\
|\childdocforward[|\textit{main}|]{|\textit{dest}|}|\\
\end{tabular}
\end{center}
%
The argument \textit{dest} is the destination file
(without extension).
It should be the main file or one of the child files.
Note that further \textsf{childdoc} directives
such as |\childdocof| and |\childdocforward|
in the indicated file will be processed in this form.
The optional argument \textit{main}
passes on directly to the main file \textit{main}
while pretending to compile the child \textit{dest}.
This form behaves as if \textit{dest}
issues |\childdocof{|\textit{main}|}| right away,
and no further \textsf{childdoc} directives will be processed.

%%%%%%%%%%%%%%%%%%%%%%%%%%%%%%%%%%%%%%%%
\DescribeMacro{\...prefix}
In the alternative form |\childdocforwardprefix|,
%
\begin{center}
\begin{tabular}{l}
|\input{childdoc.def}|\\
|\childdocforwardprefix[|\textit{main}|]{|\textit{prefix}|}{|\textit{dest}|}|
\end{tabular}
\end{center}
%
the destination file is determined by a pattern
depending on the current file:
To make this work, the current file must be called
`{\textit{prefix}\hspace{0.2em}\textit{suffix}}'
with \textit{prefix} matching precisely the argument.
Processing is then passed on to the file
`{\textit{dest}\hspace{0.2em}\textit{suffix}}'.
Surely, the same effect is achieved by
directly specifying the
argument `{\textit{dest}\hspace{0.2em}\textit{suffix}}'
in the first form.
However, that requires to set up a different file
for each child. With the alternative form of the command
all these files can have exactly the same content
which simplifies setting them up and maintaining them.

For example, the following file |draft.tex|
with a compilation flag |\version| as described in \secref{sec:flags}
compiles the main document as a draft:
%
\begin{center}
\begin{tabular}{l}
|\def\version{draft}|\\
|\input{childdoc.def}|\\
|\childdocforward{|\textit{main}|}|
\end{tabular}
\end{center}
%
Likewise, the following files |final|\textit{nn}|.tex|
compile the final version of the child document
|child|\textit{nn}|.tex|:
%
\begin{center}
\begin{tabular}{l}
|\def\version{final}|\\
|\input{childdoc.def}|\\
|\childdocforwardprefix{final}{child}|
\end{tabular}
\end{center}
%

Note that when several versions of a main file and/or of each child file
are to be generated, it may be convenient to set up a |Makefile| or
shell script to automatise the process.

%%%%%%%%%%%%%%%%%%%%%%%%%%%%%%%%%%%%%%%%%%%%%%%%%%%%%%%%%%%%%%%%%%%%%%%%%%%%%%%%
\subsection{Command Line Processing}
\label{sec:commandline}

The effect of redirection files can also be achieved by invoking
the \LaTeX{} compiler with a more elaborate command line.
Most conveniently this should be done as part
of a shell script or a |Makefile|.

When using \textsf{childdoc} in the main file, the following
command lines effectively perform a redirection
(note that depending on the shell being used,
backslashes may have to be doubled: `|\|' $\to$ `|\\|'):
%
\begin{center}
|... -jobname "|\textit{target}|" |\\|"|[\textit{flags}]%
|\input{childdoc.def}\childdocforward[|\textit{main}|]{|\textit{dest}|}"|
\end{center}
%
Here \textit{target} is the name of the output file,
\textit{main} is the name of the main file
and \textit{dest} is the name of the main or child file to be processed
(all filenames without extensions).
The optional argument \textit{main} can be omitted
if \textit{main} matches \textit{dest}.
Optionally, compilation \textit{flags} can be defined via |\def| commands.
This command line makes the \TeX{} engine believe
it is compiling the file \textit{target}
whose content is specified as the latter parameter.
The provided code then forwards the processing to
\textit{main} or \textit{dest} as described in \secref{sec:forward}.

%%%%%%%%%%%%%%%%%%%%%%%%%%%%%%%%%%%%%%%%%%%%%%%%%%%%%%%%%%%%%%%%%%%%%%%%%%%%%%%%
\subsection{Include by Input}
\label{sec:input}

Including child documents by |\include| has some restrictions by design.
Most notably, the content of a child document always occupies
its own set of pages; pages cannot be shared between child documents.
Usually, this behaviour makes perfect sense
because each child document contain an essential part of the document.
However, in some situations it may be desirable to compose
a document from a collection of parts
without having mandatory page breaks between then.
For this case, the package
provides a mechanism to include parts
by |\input| which can also be processed individually.
However, by construction this mechanism
requires manual handling of the content to be output.

%%%%%%%%%%%%%%%%%%%%%%%%%%%%%%%%%%%%%%%%
\DescribeMacro{\ifchilddocmanual}
The main file should be prepared as usual, see \secref{sec:include}.
However, the document body must make a distinction
between processing of an individual part and of the main document, e.g.:
%
\begin{center}
\begin{tabular}{l}
|\ifchilddocmanual|\\
|\input{\childdocname}|\\
|\||else|\\
\textit{document body with }|\input{|\textit{part}|}|\\
|\||fi|
\end{tabular}
\end{center}
%
The conditional |\ifchilddocmanual| is true whenever
a part to be included by |\input| is being compiled,
and the name of the part is stored in |\childdocname|.

%%%%%%%%%%%%%%%%%%%%%%%%%%%%%%%%%%%%%%%%
\DescribeMacro{\childdocby}
Each part to be included by |\input| should start with:
%
\begin{center}
\begin{tabular}{l}
|\input{childdoc.def}|\\
|\childdocby{|\textit{main}|}|\\
\end{tabular}
\end{center}
%
The directive |\childdocby| is similar to |\childdocof|
described in \secref{sec:include},
but the subsequent selection of content must be done manually.
To that end, both |\ifchilddoc| and |\ifchilddocmanual|
will be true upon processing of a part,
and the name of the part is stored in |\childdocname|.
Note that |\jobname| will be set to the filename of the current part
so that each part receives an individual |.aux| file
that does not interfere with the |.aux| file(s) of the main document.
This behaviour can be altered by the alternative form
|\childdocby[*]{|\textit{main}|}| (with a non-empty optional argument)
which uses the |.aux| file of the main document
by setting |\jobname| to \textit{main}.

%%%%%%%%%%%%%%%%%%%%%%%%%%%%%%%%%%%%%%%%%%%%%%%%%%%%%%%%%%%%%%%%%%%%%%%%%%%%%%%%
\subsection{Driver Development}
\label{sec:driver}

The \textsf{childdoc} mechanism can also be use for the development
of definition files such as \LaTeX{} styles or classes.
This case differs from the above setup with multiple parts
included by |\include| in that no |\includeonly| should be invoked.
This can be achieved by starting the include file
(before |\ProvidesPackage|) with:
%
\begin{center}
\begin{tabular}{l}
|\input{childdoc.def}|\\
|\childdocforward{|\textit{main}|}|\\
\end{tabular}
\end{center}
%
or alternatively with:
%
\begin{center}
\begin{tabular}{l}
|\input{childdoc.def}|\\
|\childdocby{|\textit{main}|}|\\
\end{tabular}
\end{center}
%
Both forms have slightly different effects as described above.
The main file is prepared as usual, see \secref{sec:include}.

%%%%%%%%%%%%%%%%%%%%%%%%%%%%%%%%%%%%%%%%%%%%%%%%%%%%%%%%%%%%%%%%%%%%%%%%%%%%%%%%
\subsection{Legacy Detection}
\label{sec:detection}

The directive |\childdocmain| in the main file can detect
whether the complete document or merely a child is to be compiled
even without using the directive |\childdocof|.
This method is deprecated because it is less robust
and there is no compelling reason to use it;
it is merely provided for backward compatibility
and it may be removed in future versions.

If the detection mechanism is to be used,
it is mandatory to correctly specify
the filename of the main file as the argument of |\childdocmain|:
%
\begin{center}
\begin{tabular}{l}
|\input{childdoc.def}|\\
|\childdocmain{|\textit{main}|}|\\
\end{tabular}
\end{center}
%
If |\jobname| does not match the argument \textit{main} of |\childdocmain|,
it is assumed that |\jobname| points to the child file to be compiled.
When using |\childdocmain| with the main file specified as argument,
it suffices to start a child file
with just |\input{|\textit{main}|}|
without loading of the package and using |\childdocof|.
If instead all processing is done
with the appropriate \textsf{childdoc} directives,
the argument of \textit{main} of |\childdocmain| can be empty.

An alternative version of the command line processing described
in \secref{sec:commandline} using the detection mechanism reads:
%
\begin{center}
|... -jobname "|\textit{target}|" "|[\textit{flags}]%
[|\def\jobname{|\textit{dest}|}|]|\input{|\textit{main}|}"|
\end{center}

%%%%%%%%%%%%%%%%%%%%%%%%%%%%%%%%%%%%%%%%%%%%%%%%%%%%%%%%%%%%%%%%%%%%%%%%%%%%%%%%
\subsection{Manual Code}
\label{sec:manual}

In case one cannot be certain whether the definitions file |childdoc.def|
is installed on the target \TeX{} distribution
and one prefers not to ship it,
it is conceivable to paste a few relevant commands into the sources.

To that end, drop all statements |\input{childdoc.def}|
and perform the replacements as outlined below.
Instead of |\childdocmain{|\textit{main}|}| add the following code
to the top of the main file:
%
\begin{center}
\begin{tabular}{l}
|\||ifdefined\childdocname\endinput\||fi\newif\ifchilddoc|\\
|\edef\childdocname{\scantokens\expandafter{\jobname\noexpand}}|\\
|\def\childdocmain{|\textit{main}|}\||ifx\childdocmain\childdocname\||else|\\
|\childdoctrue\includeonly{\childdocname}\let\jobname\childdocmain\||fi|\\
\end{tabular}
\end{center}
%
Instead of |\childdocof{|\textit{main}|}| just include the main file
at the top of each child file:
%
\begin{center}
|\input{|\textit{main}|}|
\end{center}
%
A simple redirection |\childdocforward{|\textit{dest}|}| is achieved by:
%
\begin{center}
|\def\jobname{|\textit{dest}|}\input{\jobname}|
\end{center}
%
The redirection with prefix
|\childdocforwardprefix[|\textit{prefix}|]{|\textit{dest}|}|
is accomplished by:
%
\begin{center}
\begin{tabular}{l}
|{\edef\jobname{\scantokens\expandafter{\jobname\noexpand}}|\\
|\def\redirectjob |\textit{prefix}|#1~~~{\gdef\jobname{|\textit{dest}|#1}}|\\
|\expandafter\redirectjob\jobname~~~}\input{\jobname}|
\end{tabular}
\end{center}

In an alternative approach,
child documents can be compiled by a specific command line
without additional code or specific definitions:
%
\begin{center}
|... -jobname "|\textit{target}|" "|[\textit{flags}]%
|\includeonly{|\textit{dest}|}\input{|\textit{main}|}"|
\end{center}
%

%%%%%%%%%%%%%%%%%%%%%%%%%%%%%%%%%%%%%%%%%%%%%%%%%%%%%%%%%%%%%%%%%%%%%%%%%%%%%%%%
%%%%%%%%%%%%%%%%%%%%%%%%%%%%%%%%%%%%%%%%%%%%%%%%%%%%%%%%%%%%%%%%%%%%%%%%%%%%%%%%
\section{Information}

%%%%%%%%%%%%%%%%%%%%%%%%%%%%%%%%%%%%%%%%%%%%%%%%%%%%%%%%%%%%%%%%%%%%%%%%%%%%%%%%
\subsection{Copyright}

Copyright \copyright{} 2017--2018 Niklas Beisert

This work may be distributed and/or modified under the
conditions of the \LaTeX{} Project Public License, either version 1.3
of this license or (at your option) any later version.
The latest version of this license is in
  \url{http://www.latex-project.org/lppl.txt}
and version 1.3 or later is part of all distributions of \LaTeX{}
version 2005/12/01 or later.

This work has the LPPL maintenance status `maintained'.

The Current Maintainer of this work is Niklas Beisert.

This work consists of the files |README.txt|, |childdoc.ins| and |childdoc.dtx|
as well as the derived files |childdoc.def|, |cdocsamp.tex|
with |cdocsch1.tex|, |cdocsch2.tex|, |cdocspt3.tex|, |cdocspt4.tex|,
|cdocsdrf.tex|, |cdocsfn1.tex|, |cdocsfn2.tex|
as well as |childdoc.pdf|.

%%%%%%%%%%%%%%%%%%%%%%%%%%%%%%%%%%%%%%%%%%%%%%%%%%%%%%%%%%%%%%%%%%%%%%%%%%%%%%%%
\subsection{Files and Installation}

The package consists of the files:
%
\begin{center}
\begin{tabular}{ll}
    |README.txt|   & readme file \\
    |childdoc.ins| & installation file \\
    |childdoc.dtx| & source file \\
    |childdoc.def| & definition file \\
    |cdocsamp.tex| & sample main file \\
    |cdocsch1.tex| & sample include file \\
    |cdocsch2.tex| & sample include file \\
    |cdocspt3.tex| & sample part file \\
    |cdocspt4.tex| & sample part file \\
    |cdocsdrf.tex| & sample redirection file \\
    |cdocsfn1.tex| & sample redirection file \\
    |cdocsfn2.tex| & sample redirection file \\
    |childdoc.pdf| & manual
\end{tabular}
\end{center}
%
The distribution consists of the files
|README.txt|, |childdoc.ins| and |childdoc.dtx|.
%
\begin{itemize}
\item
Run (pdf)\LaTeX{} on |childdoc.dtx|
to compile the manual |childdoc.pdf| (this file).
\item
Run \LaTeX{} on |childdoc.ins| to create the definitions file |childdoc.def|
and the sample |cdocsamp.tex| with include files
|cdocsch1.tex|, |cdocsch2.tex|, |cdocspt3.tex|, |cdocspt4.tex|,
|cdocsdrf.tex|, |cdocsfn1.tex|, |cdocsfn2.tex|.
Then copy the file |childdoc.def| to an appropriate directory of your \LaTeX{}
distribution, e.g.\ \textit{texmf-root}|/tex/latex/childdoc|.
\end{itemize}

%%%%%%%%%%%%%%%%%%%%%%%%%%%%%%%%%%%%%%%%%%%%%%%%%%%%%%%%%%%%%%%%%%%%%%%%%%%%%%%%
\subsection{Related CTAN Packages}

There are several other packages which offer a similar functionality:
%
\begin{itemize}
\item
The packages
\href{http://ctan.org/pkg/docmute}{\textsf{docmute}},
\href{http://ctan.org/pkg/includex}{\textsf{includex}} and
\href{http://ctan.org/pkg/standalone}{\textsf{standalone}}
provide commands to include only the document body of
a child file thus allowing both files to be compiled individually.
\item
The packages \href{http://ctan.org/pkg/subdocs}{\textsf{subdocs}}
and \href{http://ctan.org/pkg/subfiles}{\textsf{subfiles}}
provide structures in which the main and child documents can be
encapsulated and allowing them to be compiled individually.
The inclusion mechanism is different from the conventional |\include|.
\item
The package \href{http://ctan.org/pkg/combine}{\textsf{combine}}
is an elaborate solution to combine several documents into one.
\end{itemize}
%
See also the CTAN topic \href{http://ctan.org/topic/subdocs}{\textsf{subdocs}}
for further related packages.
The present package differs from the above solutions in that
a document structure constructed with the conventional |\include| mechanism
just needs two extra commands at the top of every file
such that all constituent files can be compiled individually.

%%%%%%%%%%%%%%%%%%%%%%%%%%%%%%%%%%%%%%%%%%%%%%%%%%%%%%%%%%%%%%%%%%%%%%%%%%%%%%%%
%\subsection{Feature Suggestions}
%
%The following is a list of features which may be useful for future
%versions of this package:
%%
%\begin{itemize}
%\item
%\ldots
%\end{itemize}

%%%%%%%%%%%%%%%%%%%%%%%%%%%%%%%%%%%%%%%%%%%%%%%%%%%%%%%%%%%%%%%%%%%%%%%%%%%%%%%%
\subsection{Revision History}

%%%%%%%%%%%%%%%%%%%%%%%%%%%%%%%%%%%%%%%%
\paragraph{v2.0:} 2018/12/30

\begin{itemize}
\item
immediate forward processing
\item
added |\childdocby| mechanism
\item
manual restructured
\end{itemize}

%%%%%%%%%%%%%%%%%%%%%%%%%%%%%%%%%%%%%%%%
\paragraph{v1.6:} 2018/01/17

\begin{itemize}
\item
application for development of include files
\item
corrections to manual
\end{itemize}

%%%%%%%%%%%%%%%%%%%%%%%%%%%%%%%%%%%%%%%%
\paragraph{v1.5:} 2017/05/21

\begin{itemize}
\item
more complete structuring introduced
\item
|\childdocof| introduced
\item
|\childdoc| renamed to |\childdocmain|
\item
|\childredirect| renamed to |\childdocforward| and |\childdocforwardprefix|
and functionality expanded
\end{itemize}

%%%%%%%%%%%%%%%%%%%%%%%%%%%%%%%%%%%%%%%%
\paragraph{v1.0:} 2017/04/27

\begin{itemize}
\item
manual and install package
\item
first version published on CTAN
\end{itemize}

%%%%%%%%%%%%%%%%%%%%%%%%%%%%%%%%%%%%%%%%
\paragraph{v0.6:} 2017/04/26

\begin{itemize}
\item
redirection mechanism added
\end{itemize}

%%%%%%%%%%%%%%%%%%%%%%%%%%%%%%%%%%%%%%%%
\paragraph{v0.5:} 2017/04/26

\begin{itemize}
\item
functionality in definition file
\end{itemize}


%%%%%%%%%%%%%%%%%%%%%%%%%%%%%%%%%%%%%%%%%%%%%%%%%%%%%%%%%%%%%%%%%%%%%%%%%%%%%%%%
%%%%%%%%%%%%%%%%%%%%%%%%%%%%%%%%%%%%%%%%%%%%%%%%%%%%%%%%%%%%%%%%%%%%%%%%%%%%%%%%
%%%%%%%%%%%%%%%%%%%%%%%%%%%%%%%%%%%%%%%%%%%%%%%%%%%%%%%%%%%%%%%%%%%%%%%%%%%%%%%%
\appendix

\settowidth\MacroIndent{\rmfamily\scriptsize 000\ }

 \DocInput{childdoc.dtx}

\end{document}
%</driver>
% \fi
%
% %%%%%%%%%%%%%%%%%%%%%%%%%%%%%%%%%%%%%%%%%%%%%%%%%%%%%%%%%%%%%%%%%%%%%%%%%%%%%%
% %%%%%%%%%%%%%%%%%%%%%%%%%%%%%%%%%%%%%%%%%%%%%%%%%%%%%%%%%%%%%%%%%%%%%%%%%%%%%%
% \section{Sample}
%\iffalse
%<*samplemain>
%\fi
%
% The following presents a sample document
% with two chapters, two parts, a title page,
% a compile flag as well as three forwarding files to set the flag.
% It consists of eight |.tex| files:
% \begin{center}
% \begin{tabular}{ll}
% |cdocsamp.tex|&main file\\
% |cdocsch1.tex|&include file for chapter 1\\
% |cdocsch2.tex|&include file for chapter 2\\
% |cdocspt3.tex|&include file for part 3\\
% |cdocspt4.tex|&include file for part 4\\
% |cdocsdrf.tex|&forwarding file for main file in draft mode\\
% |cdocsfi1.tex|&forwarding file for final version of chapter 1\\
% |cdocsfi2.tex|&forwarding file for final version of chapter 2\\
% \end{tabular}
% \end{center}
% Each of the eight files can be compiled directly by the \LaTeX{} compiler.
%
% %%%%%%%%%%%%%%%%%%%%%%%%%%%%%%%%%%%%%%
% \paragraph{Main File.}
%
% The main file is called |cdocsamp.tex|.
%
% Load the \textsf{childdoc} definitions and
% declare the filename for the main document:
%    \begin{macrocode}
\input{childdoc.def}
\childdocmain{}
%    \end{macrocode}

% Optional override for |\version| flag:
%    \begin{macrocode}
%%\ifchilddoc\else\providecommand{\version}{draft}\fi
%    \end{macrocode}

% Define the default values for the |\version| flag
% (|final| for the main file and |draft| for childs):
%    \begin{macrocode}
\ifchilddoc
\providecommand{\version}{draft}
\else
\providecommand{\version}{final}
\fi
%    \end{macrocode}

% Load the standard document class:
%    \begin{macrocode}
\documentclass[12pt]{article}
%    \end{macrocode}

% Start the document body:
%    \begin{macrocode}
\begin{document}
%    \end{macrocode}

% Declare a title page.
% Print title, part of document being processed and version flag:
%    \begin{macrocode}
\addtocounter{page}{-1}
\begin{center}
{\LARGE\bfseries{}childdoc example\par}
\vspace{1cm}
\ifchilddoc
\ifchilddocmanual part\else chapter\fi:
`\childdocname' of `\childdocjob'\par
\else
main document: `\childdocjob'\par
\fi
version: \version\par
\end{center}
\newpage
%    \end{macrocode}

% Manually include selected file,
% otherwise process as usual:
%    \begin{macrocode}
\ifchilddocmanual
\section*{part `\childdocname'}
\input{\childdocname}
\else
%    \end{macrocode}

% Include the two chapters:
%    \begin{macrocode}
\include{cdocsch1}
\include{cdocsch2}
%    \end{macrocode}

% Include the two parts unless only chapters should be displayed:
%    \begin{macrocode}
\ifchilddoc\else
\section{part three}
\input{cdocspt3}
\section{part four}
\input{cdocspt4}
\fi
%    \end{macrocode}

% Process as usual until here:
%    \begin{macrocode}
\fi
%    \end{macrocode}

% End of document body:
%    \begin{macrocode}
\end{document}
%    \end{macrocode}
%\iffalse
%</samplemain>
%\fi
%
% %%%%%%%%%%%%%%%%%%%%%%%%%%%%%%%%%%%%%%
% \paragraph{Chapter Include Files.}
%
% The include files are called |cdocsch1.tex| and |cdocsch2.tex|.
%
%\iffalse
%<*samplechap1|samplechap2>
%\fi

% Optional override for |\version| flag:
%    \begin{macrocode}
%%\providecommand{\version}{final}
%    \end{macrocode}

% Include the main document:
%    \begin{macrocode}
\input{childdoc.def}
\childdocof{cdocsamp}
%    \end{macrocode}

%\iffalse
%</samplechap1|samplechap2>
%\fi
%
%\iffalse
%<*samplechap1>
%\fi
% Some text for chapter 1:
%    \begin{macrocode}
\section{one}
some text in chapter one
%    \end{macrocode}

%\iffalse
%</samplechap1>
%\fi
% Some text for chapter 2:
%\iffalse
%<*samplechap2>
%\fi
%    \begin{macrocode}
\section{two}
more text in chapter two
%    \end{macrocode}

%\iffalse
%</samplechap2>
%\fi
%
% %%%%%%%%%%%%%%%%%%%%%%%%%%%%%%%%%%%%%%
% \paragraph{Part Include Files.}
%
% The include files are called |cdocspt3.tex| and |cdocspt4.tex|.
%
%\iffalse
%<*samplepart3|samplepart4>
%\fi

% Optional override for |\version| flag:
%    \begin{macrocode}
%%\providecommand{\version}{final}
%    \end{macrocode}

% Include the main document:
%    \begin{macrocode}
\input{childdoc.def}
\childdocby{cdocsamp}
%    \end{macrocode}

%\iffalse
%</samplepart3|samplepart4>
%\fi
%
%\iffalse
%<*samplepart3>
%\fi
% Some text for part 3:
%    \begin{macrocode}
some text in part three
%    \end{macrocode}

%\iffalse
%</samplepart3>
%\fi
% Some text for part 4:
%\iffalse
%<*samplepart4>
%\fi
%    \begin{macrocode}
more text in part four
%    \end{macrocode}

%\iffalse
%</samplepart4>
%\fi
%
% %%%%%%%%%%%%%%%%%%%%%%%%%%%%%%%%%%%%%%
% \paragraph{Forwarding for a Complete Draft.}
%
% The following forwarding file |cdocsdrf.tex|
% compiles the main document in draft mode:
%\iffalse
%<*sampledraft>
%\fi
%    \begin{macrocode}
\def\version{draft}
\input{childdoc.def}
\childdocforward{cdocsamp}
%    \end{macrocode}

%\iffalse
%</sampledraft>
%\fi
%
% %%%%%%%%%%%%%%%%%%%%%%%%%%%%%%%%%%%%%%
% \paragraph{Forwarding for Final Version of the Chapters.}
%
% The following forwarding files |cdocsfn1.tex| and |cdocsfn2.tex|
% (with identical content)
% compile the final versions of the child documents
% |cdocsch1.tex| and |cdocsch2.tex|, respectively:
%\iffalse
%<*samplefinal>
%\fi
%    \begin{macrocode}
\def\version{final}
\input{childdoc.def}
\childdocforwardprefix[cdocsamp]{cdocsfn}{cdocsch}
%    \end{macrocode}

%\iffalse
%</samplefinal>
%\fi
%
% %%%%%%%%%%%%%%%%%%%%%%%%%%%%%%%%%%%%%%
% \paragraph{Command Line Processing.}
%
% The following three command lines generate the output files
% |cdocscld|, |cdocscl1| and |cdocscl2|
% which should be identical to
% |cdocsdrf|, |cdocsch1| and |cdocsfn2|, respectively:
% \begin{center}
% \begin{tabular}{l}
% |latex -jobname cdocscld \|\\
% |  "\def\version{draft}\input{childdoc.def}\childdocforward{cdocsamp}"|\\
% |latex -jobname cdocscl1 \|\\
% |  "\input{childdoc.def}\childdocforward[cdocsamp]{cdocsch1}"|\\
% |latex -jobname cdocscl2 \|\\
% |  "\def\version{final}\input{childdoc.def}\childdocforward{cdocsch2}"|
% \end{tabular}
% \end{center}
% Note that the trailing backslash on each first line
% merely continues the input to the second line
% (for convenient cut ant paste).
% Furthermore, the command |latex| can be replaced by any
% of its alternative versions such as |pdflatex|.
%
% %%%%%%%%%%%%%%%%%%%%%%%%%%%%%%%%%%%%%%%%%%%%%%%%%%%%%%%%%%%%%%%%%%%%%%%%%%%%%%
% %%%%%%%%%%%%%%%%%%%%%%%%%%%%%%%%%%%%%%%%%%%%%%%%%%%%%%%%%%%%%%%%%%%%%%%%%%%%%%
% \section{Implementation}
%\iffalse
%<*package>
%\fi
%
% This section describes the definitions file |childdoc.def|.

% The definitions cannot be loaded using |\usepackage| or |\RequirePackage|
% which has a mechanism to prevent loading a style file more than once.
% When loading the definitions by means of |\input|
% multiple instances have to be prevented manually:
%\iffalse
%This code needs to be before the `\ProvidesFile' directive
%which is defined at the beginning of this file.
%Therefore it is also placed there and commented out here.
%</package>
%<*discard>
%\fi
%    \begin{macrocode}
\ifdefined\childdocmain\endinput\fi
%    \end{macrocode}
%\iffalse
%</discard>
%<*package>
%\fi
%
% \macro{\ifchilddoc}
% \macro{\ifchilddocmanual}
% The conditional |\ifchilddoc| tells whether a
% child (true) or main (false) document is being compiled.
% The conditional |\ifchilddocmanual| tells whether
% the |\includeonly| mechanism is used (false) or
% the selection of child files must be performed manually (true).
% The definitions initialise to false:
%    \begin{macrocode}
\newif\ifchilddoc
\newif\ifchilddocmanual
%    \end{macrocode}

% \macro{\childdocname}
% \macro{\childdocjob}
% The macro |\childdocname| stores the name of the main document
% to be compiled. The macro |\childdocjob| stores the name of
% the document on which the \LaTeX{} compiler was originally invoked.
% The content of |\jobname| cannot be compared
% to filenames specified in the source due to different catcodes.
% The following code rescans |\jobname|, stores the result
% in |\childdocname| and saves a copy in |\childdocjob|:
%    \begin{macrocode}
\edef\childdocname{\scantokens\expandafter{\jobname\noexpand}}
\let\childdocjob\childdocname
%    \end{macrocode}

% \macro{\childdocdisable}
% The macro |\childdocdisable| prevents the main file
% from being processed more than once.
% At this stage, the main document command |\childdocmain|
% is assumed to be called once again where it should do nothing.
% Any subsequent call to it should prevent
% a secondary processing of the main document
% It overwrites the forwarding commands
% |\childdocof| and |\childdocforward|
% with empty macros to prevent further inclusions of the main document:
%    \begin{macrocode}
\newcommand{\childdocdisable}
{
  \renewcommand{\childdocmain}[1]{\renewcommand{\childdocmain}[1]{\endinput}}
  \renewcommand{\childdocof}[1]{}
  \renewcommand{\childdocby}[2][]{}
  \renewcommand{\childdocforward}[2][]{}
  \renewcommand{\childdocdisable}{}
}
%    \end{macrocode}

% \macro{\childdocmain}
% The macro |\childdocmain| is to be called at the top of the main file
% with nothing or the main filename (without extension) as argument.
% First, it breaks loops.
% If the argument is not empty and does not match |\childdocname|
% (which is set by the first inclusion of |childdoc.def|),
% |\ifchilddoc| is set to true, |\includeonly| is applied to the child file
% and |\jobname| is set to the main file
% (for proper handling of |.aux| files):
%    \begin{macrocode}
\newcommand{\childdocmain}[1]
{
  \childdocdisable\childdocmain{}
  \if?#1?\else
    \begingroup
      \def\childdoctmp{#1}
      \ifx\childdoctmp\childdocname
        \def\childdoctmp{}
      \else
        \def\childdoctmp
        {
          \childdoctrue
          \includeonly{\childdocname}
          \def\childdocjob{#1}
          \def\jobname{#1}
        }
      \fi
      \expandafter
    \endgroup
    \childdoctmp
  \fi
}
%    \end{macrocode}

% \macro{\childdocof}
% The command |\childdocof| redirects
% compilation to the main file |#1|.
%    \begin{macrocode}
\newcommand{\childdocof}[1]
{
  \childdocdisable
  \childdoctrue
  \includeonly{\childdocname}
  \def\jobname{#1}
  \def\childdocjob{#1}
  \input{#1}
}
%    \end{macrocode}

% \macro{\childdocby}
% The command |\childdocby| ....
%    \begin{macrocode}
\newcommand{\childdocby}[2][]
{
  \childdocdisable
  \childdoctrue
  \childdocmanualtrue
  \if?#1?\else
    \def\jobname{#2}
  \fi
  \def\childdocjob{#2}
  \input{#2}
  \endinput
}
%    \end{macrocode}

% \macro{\childdocforward}
% The command |\childdocforward| redirects
% compilation to the main file or
% (if the optional argument is given) a child file.
% Parameters are set as if the main file
% or a child file starting with |\childdocof| was compiled.
% Then compilation is handed over to the main file:
%    \begin{macrocode}
\newcommand{\childdocforward}[2][]
{
  \begingroup
    \if?#1?
      \def\childdoctmp
      {
        \def\childdocname{#2}
        \def\childdocjob{#2}
        \def\jobname{#2}
        \input{#2}
        \endinput
      }
    \else
      \def\childdoctmp
      {
        \childdocdisable
        \def\childdocname{#2}
        \childdoctrue
        \includeonly{#2}
        \def\childdocjob{#1}
        \def\jobname{#1}
        \input{#1}
        \endinput
      }
    \fi
    \expandafter
  \endgroup
  \childdoctmp
}
%    \end{macrocode}

% \macro{\childdocforwardprefix}
% The command |\childdocforwardprefix| redirects
% compilation to the main or a child file by means of a pattern.
% The prefix |#1| in the current filename is replaced by |#2|
% and the suffix of the current filename is kept
% (it is assumed that the filename does not contain the substring `|~~~|'
% which is used as a delimiter).
% Compilation is handed over to the new file by |\childdocforward|:
%    \begin{macrocode}
\newcommand{\childdocforwardprefix}[3][]
{
  \begingroup
    \def\childdocextract #2##1~~~{\def\childdoctmp{\childdocforward[#1]{#3##1}}}
    \expandafter\childdocextract\childdocname~~~
    \expandafter
  \endgroup
  \childdoctmp
}
%    \end{macrocode}

% \macro{\childdoc}
% The deprecated macro |\childdoc| is a legacy version of |\childdocmain|:
%    \begin{macrocode}
\newcommand{\childdoc}{\childdocmain}
%    \end{macrocode}

% \macro{\childdocredirect}
% The deprecated macro |\childdocredirect| is a legacy version
% of |\childdocforward| and |\childdocforwardprefix|:
%    \begin{macrocode}
\newcommand{\childdocredirect}[2][]
{
  \begingroup
    \if?#1?
      \def\childdoctmp{\childdocforward{#2}}
    \else
      \def\childdoctmp{\childdocforwardprefix{#1}{#2}}
    \fi
    \expandafter
  \endgroup
  \childdoctmp
}
%    \end{macrocode}

%\iffalse
%</package>
%\fi
%
\endinput
|\\
|\childdocforward[|\textit{main}|]{|\textit{dest}|}|\\
\end{tabular}
\end{center}
%
The argument \textit{dest} is the destination file
(without extension).
It should be the main file or one of the child files.
Note that further \textsf{childdoc} directives
such as |\childdocof| and |\childdocforward|
in the indicated file will be processed in this form.
The optional argument \textit{main}
passes on directly to the main file \textit{main}
while pretending to compile the child \textit{dest}.
This form behaves as if \textit{dest}
issues |\childdocof{|\textit{main}|}| right away,
and no further \textsf{childdoc} directives will be processed.

%%%%%%%%%%%%%%%%%%%%%%%%%%%%%%%%%%%%%%%%
\DescribeMacro{\...prefix}
In the alternative form |\childdocforwardprefix|,
%
\begin{center}
\begin{tabular}{l}
|% \iffalse
%
% childdoc.dtx Copyright (C) 2017-2018 Niklas Beisert
%
% This work may be distributed and/or modified under the
% conditions of the LaTeX Project Public License, either version 1.3
% of this license or (at your option) any later version.
% The latest version of this license is in
%   http://www.latex-project.org/lppl.txt
% and version 1.3 or later is part of all distributions of LaTeX
% version 2005/12/01 or later.
%
% This work has the LPPL maintenance status `maintained'.
%
% The Current Maintainer of this work is Niklas Beisert.
%
% This work consists of the files childdoc.dtx and childdoc.ins
% and the derived files childdoc.def and cdocsamp.tex with
% cdocsch1.tex, cdocsch2.tex, cdocsdrf.tex, cdocsfn1.tex, cdocsfn2.tex.
%
%<package>\ifdefined\childdocmain\endinput\fi
%<package>\ProvidesFile{childdoc.def}[2018/12/30 v2.0 child document driver]
%<samplemain>\ProvidesFile{cdocsamp.tex}[2018/12/30 v2.0 sample for childdoc]
%<*driver>
%\ProvidesFile{childdoc.drv}[2018/12/30 v2.0 childdoc reference manual file]
\PassOptionsToClass{10pt,a4paper}{article}
\documentclass{ltxdoc}

\usepackage[margin=35mm]{geometry}
\usepackage{hyperref}
\usepackage{hyperxmp}
\usepackage[usenames]{color}

\hypersetup{colorlinks=true}
\hypersetup{pdfstartview=FitH}
\hypersetup{pdfpagemode=UseNone}
\hypersetup{pdfsource={}}
\hypersetup{pdflang={en-UK}}
\hypersetup{pdfcopyright={Copyright 2017-2018 Niklas Beisert.
  This work may be distributed and/or modified under the
  conditions of the LaTeX Project Public License, either version 1.3
  of this license or (at your option) any later version.}}
\hypersetup{pdflicenseurl={http://www.latex-project.org/lppl.txt}}
\hypersetup{pdfcontactaddress={ETH Zurich, ITP, HIT K,
  Wolfgang-Pauli-Strasse 27}}
\hypersetup{pdfcontactpostcode={8093}}
\hypersetup{pdfcontactcity={Zurich}}
\hypersetup{pdfcontactcountry={Switzerland}}
\hypersetup{pdfcontactemail={nbeisert@itp.phys.ethz.ch}}
\hypersetup{pdfcontacturl={http://people.phys.ethz.ch/\xmptilde nbeisert/}}

\newcommand{\secref}[1]{\hyperref[#1]{section \ref*{#1}}}

\parskip1ex
\parindent0pt
\let\olditemize\itemize
\def\itemize{\olditemize\parskip0pt}

\begin{document}

\title{The \textsf{childdoc} Package}
\hypersetup{pdftitle={The childdoc Package}}
\author{Niklas Beisert\\[2ex]
  Institut f\"ur Theoretische Physik\\
  Eidgen\"ossische Technische Hochschule Z\"urich\\
  Wolfgang-Pauli-Strasse 27, 8093 Z\"urich, Switzerland\\[1ex]
  \href{mailto:nbeisert@itp.phys.ethz.ch}
  {\texttt{nbeisert@itp.phys.ethz.ch}}}
\hypersetup{pdfauthor={Niklas Beisert}}
\hypersetup{pdfsubject={Manual for the LaTeX2e Package childdoc}}
\date{30 December 2018, \textsf{v2.0}}
\maketitle

\begin{abstract}\noindent
\textsf{childdoc} is a \LaTeXe{} package
that enables the direct compilation
of document sections included by |\include|
to individual files.
\end{abstract}

\begingroup
\parskip0ex
\tableofcontents
\endgroup

%%%%%%%%%%%%%%%%%%%%%%%%%%%%%%%%%%%%%%%%%%%%%%%%%%%%%%%%%%%%%%%%%%%%%%%%%%%%%%%%
%%%%%%%%%%%%%%%%%%%%%%%%%%%%%%%%%%%%%%%%%%%%%%%%%%%%%%%%%%%%%%%%%%%%%%%%%%%%%%%%
\section{Introduction}

\LaTeX{} provides a mechanism to structure a large document (such as a book)
into a main file and several child files (containing the chapters)
using the |\include| command.
This mechanism is beneficial for documents
which span hundreds of pages in order to
make the source file(s) more manageable.
Moreover, compilation can be restricted to
selected child files by means of the |\includeonly| command.
The latter feature can be used to reduce the compilation time while editing
(this was significantly more useful in the earlier days of \LaTeX{})
or to generate a smaller document which is easier to navigate.
Another application of |\includeonly| is to generate
documents consisting of selected parts of the complete document.

However, there are a few drawbacks of the plain |\include| mechanism:
\begin{itemize}
\item
The child files cannot be compiled on their own,
they can only be compiled via the main file.
A naive editing environment
(such as a text editor with an option
to have the current file processed by \LaTeX)
may require one to switch to the main file before compiling;
attempting to compile the child file produces errors.
\item
The main file must be modified (each time)
to adjust the |\includeonly| command
to the present needs. This easily leaves the main file in a messy state.
\item
The generated document will always carry the filename
of the main document. This is inconvenient if
several child files are to be compiled and
to be kept for distribution.
\end{itemize}

The present package provides a simple interface
to make child files individually compilable by \LaTeX{}.
Compiling a child file then has the same effect as compiling
the main file with an |\includeonly| command
to select the appropriate child.
Moreover the generated document will carry the name of the child
rather than the main file.
This resolves all three above issues.

This feature is meant to make the editing of books,
thesis documents and lecture notes somewhat more convenient.
However, the package can also be used efficiently for
composing a series of documents (such as exercise sheets)
which are typically distributed individually.
It then assists the author in generating the individual documents
(potentially in different versions)
as well as a document containing the collected series.
Another application is in developing style files
or other kinds of included material
where compilation of the style file could redirect
to a sample or test file.

%%%%%%%%%%%%%%%%%%%%%%%%%%%%%%%%%%%%%%%%%%%%%%%%%%%%%%%%%%%%%%%%%%%%%%%%%%%%%%%%
%%%%%%%%%%%%%%%%%%%%%%%%%%%%%%%%%%%%%%%%%%%%%%%%%%%%%%%%%%%%%%%%%%%%%%%%%%%%%%%%
\section{Usage}

First of all, the package \textsf{childdoc} is \emph{not} a standard
\LaTeXe{} |.sty| style file! Therefore it needs to be invoked in
a non-standard way.

%%%%%%%%%%%%%%%%%%%%%%%%%%%%%%%%%%%%%%%%%%%%%%%%%%%%%%%%%%%%%%%%%%%%%%%%%%%%%%%%
\subsection{Included Files}
\label{sec:include}

%%%%%%%%%%%%%%%%%%%%%%%%%%%%%%%%%%%%%%%%
\DescribeMacro{\childdocmain}
To use the package, add the commands
\begin{center}
\begin{tabular}{l}
|\input{childdoc.def}|\\
|\childdocmain{}|\\
\end{tabular}
\end{center}
at the very top of the main \LaTeX{} file,
in particular \emph{before} the |\documentclass| statement!
The argument of |\childdocmain| should be left empty
(but it must be present).

%%%%%%%%%%%%%%%%%%%%%%%%%%%%%%%%%%%%%%%%
\DescribeMacro{\childdocof}
Furthermore, add the commands
\begin{center}
\begin{tabular}{l}
|\input{childdoc.def}|\\
|\childdocof{|\textit{main}|}|\\
\end{tabular}
\end{center}
at the top of every child file \textit{child}
which is included by |\include{|\textit{child}|}|
from within the main file
(or at least for those files to be compiled individually).
The argument \textit{main} must be the filename of the main file.

There are a couple of
considerations in setting up the main and child documents:

%%%%%%%%%%%%%%%%%%%%%%%%%%%%%%%%%%%%%%%%
\paragraph{Restrictions.}

Please note the following restrictions:
\begin{itemize}
\item
|\childdocmain| must be called with one argument \textit{main}
to ensure compatibility with earlier version of the package.
It must either be empty (|\childdocmain{}|)
or precisely match the filename of the main file in which it is specified.
See \secref{sec:detection} for further information.
\item
The filename \textit{main} must be specified without the |.tex| extension.
\item
The filename \textit{main} is case sensitive
(even in case-insensitive file systems)
due to internal string comparison.
\item
The argument \textit{main} should be fully expanded, it cannot be a macro.
\item
Subdirectories and special characters should be avoided in filenames.
\item
The command |\childdocmain{|\textit{main}|}| must be followed by a whitespace.
It should not be followed immediately by another command
or by a comment mark `|%|'.
This is because the \TeX{} parser reads the token immediately following
the argument of |\childdocmain| and puts it
at the beginning of every child section;
however, a white\-space is ignored.
\end{itemize}

%%%%%%%%%%%%%%%%%%%%%%%%%%%%%%%%%%%%%%%%
\paragraph{Content of Main File.}

It is advisable to place all content in the child files included by |\include|.
Any output contained in the main file will appear in all child documents
unless suppressed manually;
it cannot be suppressed automatically by the |\includeonly| directive
and thus should normally be avoided.
A method to include some content in the main file
by means of conditional processing is described in \secref{sec:conditional}.

%%%%%%%%%%%%%%%%%%%%%%%%%%%%%%%%%%%%%%%%
\paragraph{Page Numbering.}

When only a part of the document is compiled,
the appropriate numbering of pages
(as well as other status parameters)
is determined from the |.aux| files.
The latter contain information from previous passes.
However this information needs to propagate through
all intermediate child documents.
Therefore the page numbering in child documents may well
be inconsistent until the complete document is compiled at least once.

A useful (if unconventional) way to always ensure a consistent
page numbering is to restart the numbering in each child document
and denote the pages by `\textit{child}|.|\textit{page}'
where \textit{child} represents the chapter/section number of the child file.
This can be achieved by the command
|\numberwithin{page}{|\textit{child}|}|
of the \textsf{amsmath} package
where \textit{child} can be |chapter| or |section|
depending on the chosen structuring.
Alternatively, one can modify the macro |\thepage| appropriately
and reset the counter |page| at the start of each child file.

%%%%%%%%%%%%%%%%%%%%%%%%%%%%%%%%%%%%%%%%%%%%%%%%%%%%%%%%%%%%%%%%%%%%%%%%%%%%%%%%
\subsection{Conditional Processing}
\label{sec:conditional}

The package provides a mechanism to compile different versions
of a document. To customise the versions further some conditional processing
can come in handy to distinguish which version is being compiled.
The package provides two macros to describe the compilation context:

%%%%%%%%%%%%%%%%%%%%%%%%%%%%%%%%%%%%%%%%
\DescribeMacro{\ifchilddoc}
The conditional |\ifchilddoc| distinguishes between the compilation of
child documents and the main document:
%
\begin{center}
|\ifchilddoc |\textit{child-code}| |[|\||else |\textit{main-code}]| \||fi|
\end{center}

%%%%%%%%%%%%%%%%%%%%%%%%%%%%%%%%%%%%%%%%
\DescribeMacro{\childdocname}
\DescribeMacro{\childdocjob}
The macro |\childdocname| contains the filename (without extension)
of the main or child file being processed.
Note that |\childdocjob| will always contain the name of the main file.

%%%%%%%%%%%%%%%%%%%%%%%%%%%%%%%%%%%%%%%%
\paragraph{Title Page.}

Conditional processing can be used to include a title or banner page
in the main document when proper precautions are taken.
Importantly, the code in the main file should ensure that the page counter
(as well as other status parameters which are stored in the |.aux| files)
takes the same value after the conditional processing.
Otherwise the page numbers may take divergent values
depending on which part is compiled.

For example, a title page could be declared by:
%
\begin{center}
\begin{tabular}{l}
|\ifchilddoc\||else|\\
|\addtocounter{page}{-1}|\\
\textit{code for title page}\\
|\newpage|\\
|\||fi|
\end{tabular}
\end{center}
%
A banner page for the child documents can be generated by:
%
\begin{center}
\begin{tabular}{l}
|\ifchilddoc|\\
|\addtocounter{page}{-1}|\\
\textit{code for banner page}\\
|\newpage|\\
|\||fi|
\end{tabular}
\end{center}
%
Here one could write a message such as:
\begin{center}
|This is the part \childdocname{} of \childdocjob{}.|
\end{center}

%%%%%%%%%%%%%%%%%%%%%%%%%%%%%%%%%%%%%%%%%%%%%%%%%%%%%%%%%%%%%%%%%%%%%%%%%%%%%%%%
\subsection{Flags}
\label{sec:flags}

The package makes it easy to generate different versions
of the main or child documents.
To this end compilation flags can be defined
and assigned different default values.
They will be particularly useful in conjunction
with the forwarding mechanism described in \secref{sec:forward}.

For example, it may be useful to have a flag |\version|
which can be set to |draft| or |final|.
The document source will contain some conditional code
depending on the value of |\version|.
Suppose further, the flag should default to |final| for the main file
and to |draft| for child files
which is a natural assignment for editing the document.
This is achieved by placing the following code
in the preamble of the main document
(below the |\childdocmain| directive):
%
\begin{center}
\begin{tabular}{l}
|\ifchilddoc|\\
|\providecommand{\version}{draft}|\\
|\||else|\\
|\providecommand{\version}{final}|\\
|\||fi|
\end{tabular}
\end{center}
%
The definition by |\providecommand| makes sure
that previous definitions are not overwritten.
Further statements |\providecommand{\version}{...}|
can thus be added before the above code to override it.

For the main file, one might add a line
(between |\childdocmain| and the above block)
%
\begin{center}
|%\ifchilddoc\||else\providecommand{\version}{draft}\||fi|
\end{center}
%
which can be uncommented to produce a draft version.
Likewise one can add a line to the very top of a child file
(above the |\childdocof{|\textit{main}|}| directive)
%
\begin{center}
|%\providecommand{\version}{final}|
\end{center}
%
which can be uncommented to produce the final version of this child document.

%%%%%%%%%%%%%%%%%%%%%%%%%%%%%%%%%%%%%%%%%%%%%%%%%%%%%%%%%%%%%%%%%%%%%%%%%%%%%%%%
\subsection{Forwarding}
\label{sec:forward}

Different versions of the main or child documents
using compilation flags as described in \secref{sec:flags}
can be (permanently) stored in different files
for convenient compilation, viewing and distribution.
To this end, the package defines a command
to pass on compilation to a different file:

%%%%%%%%%%%%%%%%%%%%%%%%%%%%%%%%%%%%%%%%
\DescribeMacro{\childdocforward}
The command |\childdocforward| redirects processing to
another source file:
%
\begin{center}
\begin{tabular}{l}
|\input{childdoc.def}|\\
|\childdocforward[|\textit{main}|]{|\textit{dest}|}|\\
\end{tabular}
\end{center}
%
The argument \textit{dest} is the destination file
(without extension).
It should be the main file or one of the child files.
Note that further \textsf{childdoc} directives
such as |\childdocof| and |\childdocforward|
in the indicated file will be processed in this form.
The optional argument \textit{main}
passes on directly to the main file \textit{main}
while pretending to compile the child \textit{dest}.
This form behaves as if \textit{dest}
issues |\childdocof{|\textit{main}|}| right away,
and no further \textsf{childdoc} directives will be processed.

%%%%%%%%%%%%%%%%%%%%%%%%%%%%%%%%%%%%%%%%
\DescribeMacro{\...prefix}
In the alternative form |\childdocforwardprefix|,
%
\begin{center}
\begin{tabular}{l}
|\input{childdoc.def}|\\
|\childdocforwardprefix[|\textit{main}|]{|\textit{prefix}|}{|\textit{dest}|}|
\end{tabular}
\end{center}
%
the destination file is determined by a pattern
depending on the current file:
To make this work, the current file must be called
`{\textit{prefix}\hspace{0.2em}\textit{suffix}}'
with \textit{prefix} matching precisely the argument.
Processing is then passed on to the file
`{\textit{dest}\hspace{0.2em}\textit{suffix}}'.
Surely, the same effect is achieved by
directly specifying the
argument `{\textit{dest}\hspace{0.2em}\textit{suffix}}'
in the first form.
However, that requires to set up a different file
for each child. With the alternative form of the command
all these files can have exactly the same content
which simplifies setting them up and maintaining them.

For example, the following file |draft.tex|
with a compilation flag |\version| as described in \secref{sec:flags}
compiles the main document as a draft:
%
\begin{center}
\begin{tabular}{l}
|\def\version{draft}|\\
|\input{childdoc.def}|\\
|\childdocforward{|\textit{main}|}|
\end{tabular}
\end{center}
%
Likewise, the following files |final|\textit{nn}|.tex|
compile the final version of the child document
|child|\textit{nn}|.tex|:
%
\begin{center}
\begin{tabular}{l}
|\def\version{final}|\\
|\input{childdoc.def}|\\
|\childdocforwardprefix{final}{child}|
\end{tabular}
\end{center}
%

Note that when several versions of a main file and/or of each child file
are to be generated, it may be convenient to set up a |Makefile| or
shell script to automatise the process.

%%%%%%%%%%%%%%%%%%%%%%%%%%%%%%%%%%%%%%%%%%%%%%%%%%%%%%%%%%%%%%%%%%%%%%%%%%%%%%%%
\subsection{Command Line Processing}
\label{sec:commandline}

The effect of redirection files can also be achieved by invoking
the \LaTeX{} compiler with a more elaborate command line.
Most conveniently this should be done as part
of a shell script or a |Makefile|.

When using \textsf{childdoc} in the main file, the following
command lines effectively perform a redirection
(note that depending on the shell being used,
backslashes may have to be doubled: `|\|' $\to$ `|\\|'):
%
\begin{center}
|... -jobname "|\textit{target}|" |\\|"|[\textit{flags}]%
|\input{childdoc.def}\childdocforward[|\textit{main}|]{|\textit{dest}|}"|
\end{center}
%
Here \textit{target} is the name of the output file,
\textit{main} is the name of the main file
and \textit{dest} is the name of the main or child file to be processed
(all filenames without extensions).
The optional argument \textit{main} can be omitted
if \textit{main} matches \textit{dest}.
Optionally, compilation \textit{flags} can be defined via |\def| commands.
This command line makes the \TeX{} engine believe
it is compiling the file \textit{target}
whose content is specified as the latter parameter.
The provided code then forwards the processing to
\textit{main} or \textit{dest} as described in \secref{sec:forward}.

%%%%%%%%%%%%%%%%%%%%%%%%%%%%%%%%%%%%%%%%%%%%%%%%%%%%%%%%%%%%%%%%%%%%%%%%%%%%%%%%
\subsection{Include by Input}
\label{sec:input}

Including child documents by |\include| has some restrictions by design.
Most notably, the content of a child document always occupies
its own set of pages; pages cannot be shared between child documents.
Usually, this behaviour makes perfect sense
because each child document contain an essential part of the document.
However, in some situations it may be desirable to compose
a document from a collection of parts
without having mandatory page breaks between then.
For this case, the package
provides a mechanism to include parts
by |\input| which can also be processed individually.
However, by construction this mechanism
requires manual handling of the content to be output.

%%%%%%%%%%%%%%%%%%%%%%%%%%%%%%%%%%%%%%%%
\DescribeMacro{\ifchilddocmanual}
The main file should be prepared as usual, see \secref{sec:include}.
However, the document body must make a distinction
between processing of an individual part and of the main document, e.g.:
%
\begin{center}
\begin{tabular}{l}
|\ifchilddocmanual|\\
|\input{\childdocname}|\\
|\||else|\\
\textit{document body with }|\input{|\textit{part}|}|\\
|\||fi|
\end{tabular}
\end{center}
%
The conditional |\ifchilddocmanual| is true whenever
a part to be included by |\input| is being compiled,
and the name of the part is stored in |\childdocname|.

%%%%%%%%%%%%%%%%%%%%%%%%%%%%%%%%%%%%%%%%
\DescribeMacro{\childdocby}
Each part to be included by |\input| should start with:
%
\begin{center}
\begin{tabular}{l}
|\input{childdoc.def}|\\
|\childdocby{|\textit{main}|}|\\
\end{tabular}
\end{center}
%
The directive |\childdocby| is similar to |\childdocof|
described in \secref{sec:include},
but the subsequent selection of content must be done manually.
To that end, both |\ifchilddoc| and |\ifchilddocmanual|
will be true upon processing of a part,
and the name of the part is stored in |\childdocname|.
Note that |\jobname| will be set to the filename of the current part
so that each part receives an individual |.aux| file
that does not interfere with the |.aux| file(s) of the main document.
This behaviour can be altered by the alternative form
|\childdocby[*]{|\textit{main}|}| (with a non-empty optional argument)
which uses the |.aux| file of the main document
by setting |\jobname| to \textit{main}.

%%%%%%%%%%%%%%%%%%%%%%%%%%%%%%%%%%%%%%%%%%%%%%%%%%%%%%%%%%%%%%%%%%%%%%%%%%%%%%%%
\subsection{Driver Development}
\label{sec:driver}

The \textsf{childdoc} mechanism can also be use for the development
of definition files such as \LaTeX{} styles or classes.
This case differs from the above setup with multiple parts
included by |\include| in that no |\includeonly| should be invoked.
This can be achieved by starting the include file
(before |\ProvidesPackage|) with:
%
\begin{center}
\begin{tabular}{l}
|\input{childdoc.def}|\\
|\childdocforward{|\textit{main}|}|\\
\end{tabular}
\end{center}
%
or alternatively with:
%
\begin{center}
\begin{tabular}{l}
|\input{childdoc.def}|\\
|\childdocby{|\textit{main}|}|\\
\end{tabular}
\end{center}
%
Both forms have slightly different effects as described above.
The main file is prepared as usual, see \secref{sec:include}.

%%%%%%%%%%%%%%%%%%%%%%%%%%%%%%%%%%%%%%%%%%%%%%%%%%%%%%%%%%%%%%%%%%%%%%%%%%%%%%%%
\subsection{Legacy Detection}
\label{sec:detection}

The directive |\childdocmain| in the main file can detect
whether the complete document or merely a child is to be compiled
even without using the directive |\childdocof|.
This method is deprecated because it is less robust
and there is no compelling reason to use it;
it is merely provided for backward compatibility
and it may be removed in future versions.

If the detection mechanism is to be used,
it is mandatory to correctly specify
the filename of the main file as the argument of |\childdocmain|:
%
\begin{center}
\begin{tabular}{l}
|\input{childdoc.def}|\\
|\childdocmain{|\textit{main}|}|\\
\end{tabular}
\end{center}
%
If |\jobname| does not match the argument \textit{main} of |\childdocmain|,
it is assumed that |\jobname| points to the child file to be compiled.
When using |\childdocmain| with the main file specified as argument,
it suffices to start a child file
with just |\input{|\textit{main}|}|
without loading of the package and using |\childdocof|.
If instead all processing is done
with the appropriate \textsf{childdoc} directives,
the argument of \textit{main} of |\childdocmain| can be empty.

An alternative version of the command line processing described
in \secref{sec:commandline} using the detection mechanism reads:
%
\begin{center}
|... -jobname "|\textit{target}|" "|[\textit{flags}]%
[|\def\jobname{|\textit{dest}|}|]|\input{|\textit{main}|}"|
\end{center}

%%%%%%%%%%%%%%%%%%%%%%%%%%%%%%%%%%%%%%%%%%%%%%%%%%%%%%%%%%%%%%%%%%%%%%%%%%%%%%%%
\subsection{Manual Code}
\label{sec:manual}

In case one cannot be certain whether the definitions file |childdoc.def|
is installed on the target \TeX{} distribution
and one prefers not to ship it,
it is conceivable to paste a few relevant commands into the sources.

To that end, drop all statements |\input{childdoc.def}|
and perform the replacements as outlined below.
Instead of |\childdocmain{|\textit{main}|}| add the following code
to the top of the main file:
%
\begin{center}
\begin{tabular}{l}
|\||ifdefined\childdocname\endinput\||fi\newif\ifchilddoc|\\
|\edef\childdocname{\scantokens\expandafter{\jobname\noexpand}}|\\
|\def\childdocmain{|\textit{main}|}\||ifx\childdocmain\childdocname\||else|\\
|\childdoctrue\includeonly{\childdocname}\let\jobname\childdocmain\||fi|\\
\end{tabular}
\end{center}
%
Instead of |\childdocof{|\textit{main}|}| just include the main file
at the top of each child file:
%
\begin{center}
|\input{|\textit{main}|}|
\end{center}
%
A simple redirection |\childdocforward{|\textit{dest}|}| is achieved by:
%
\begin{center}
|\def\jobname{|\textit{dest}|}\input{\jobname}|
\end{center}
%
The redirection with prefix
|\childdocforwardprefix[|\textit{prefix}|]{|\textit{dest}|}|
is accomplished by:
%
\begin{center}
\begin{tabular}{l}
|{\edef\jobname{\scantokens\expandafter{\jobname\noexpand}}|\\
|\def\redirectjob |\textit{prefix}|#1~~~{\gdef\jobname{|\textit{dest}|#1}}|\\
|\expandafter\redirectjob\jobname~~~}\input{\jobname}|
\end{tabular}
\end{center}

In an alternative approach,
child documents can be compiled by a specific command line
without additional code or specific definitions:
%
\begin{center}
|... -jobname "|\textit{target}|" "|[\textit{flags}]%
|\includeonly{|\textit{dest}|}\input{|\textit{main}|}"|
\end{center}
%

%%%%%%%%%%%%%%%%%%%%%%%%%%%%%%%%%%%%%%%%%%%%%%%%%%%%%%%%%%%%%%%%%%%%%%%%%%%%%%%%
%%%%%%%%%%%%%%%%%%%%%%%%%%%%%%%%%%%%%%%%%%%%%%%%%%%%%%%%%%%%%%%%%%%%%%%%%%%%%%%%
\section{Information}

%%%%%%%%%%%%%%%%%%%%%%%%%%%%%%%%%%%%%%%%%%%%%%%%%%%%%%%%%%%%%%%%%%%%%%%%%%%%%%%%
\subsection{Copyright}

Copyright \copyright{} 2017--2018 Niklas Beisert

This work may be distributed and/or modified under the
conditions of the \LaTeX{} Project Public License, either version 1.3
of this license or (at your option) any later version.
The latest version of this license is in
  \url{http://www.latex-project.org/lppl.txt}
and version 1.3 or later is part of all distributions of \LaTeX{}
version 2005/12/01 or later.

This work has the LPPL maintenance status `maintained'.

The Current Maintainer of this work is Niklas Beisert.

This work consists of the files |README.txt|, |childdoc.ins| and |childdoc.dtx|
as well as the derived files |childdoc.def|, |cdocsamp.tex|
with |cdocsch1.tex|, |cdocsch2.tex|, |cdocspt3.tex|, |cdocspt4.tex|,
|cdocsdrf.tex|, |cdocsfn1.tex|, |cdocsfn2.tex|
as well as |childdoc.pdf|.

%%%%%%%%%%%%%%%%%%%%%%%%%%%%%%%%%%%%%%%%%%%%%%%%%%%%%%%%%%%%%%%%%%%%%%%%%%%%%%%%
\subsection{Files and Installation}

The package consists of the files:
%
\begin{center}
\begin{tabular}{ll}
    |README.txt|   & readme file \\
    |childdoc.ins| & installation file \\
    |childdoc.dtx| & source file \\
    |childdoc.def| & definition file \\
    |cdocsamp.tex| & sample main file \\
    |cdocsch1.tex| & sample include file \\
    |cdocsch2.tex| & sample include file \\
    |cdocspt3.tex| & sample part file \\
    |cdocspt4.tex| & sample part file \\
    |cdocsdrf.tex| & sample redirection file \\
    |cdocsfn1.tex| & sample redirection file \\
    |cdocsfn2.tex| & sample redirection file \\
    |childdoc.pdf| & manual
\end{tabular}
\end{center}
%
The distribution consists of the files
|README.txt|, |childdoc.ins| and |childdoc.dtx|.
%
\begin{itemize}
\item
Run (pdf)\LaTeX{} on |childdoc.dtx|
to compile the manual |childdoc.pdf| (this file).
\item
Run \LaTeX{} on |childdoc.ins| to create the definitions file |childdoc.def|
and the sample |cdocsamp.tex| with include files
|cdocsch1.tex|, |cdocsch2.tex|, |cdocspt3.tex|, |cdocspt4.tex|,
|cdocsdrf.tex|, |cdocsfn1.tex|, |cdocsfn2.tex|.
Then copy the file |childdoc.def| to an appropriate directory of your \LaTeX{}
distribution, e.g.\ \textit{texmf-root}|/tex/latex/childdoc|.
\end{itemize}

%%%%%%%%%%%%%%%%%%%%%%%%%%%%%%%%%%%%%%%%%%%%%%%%%%%%%%%%%%%%%%%%%%%%%%%%%%%%%%%%
\subsection{Related CTAN Packages}

There are several other packages which offer a similar functionality:
%
\begin{itemize}
\item
The packages
\href{http://ctan.org/pkg/docmute}{\textsf{docmute}},
\href{http://ctan.org/pkg/includex}{\textsf{includex}} and
\href{http://ctan.org/pkg/standalone}{\textsf{standalone}}
provide commands to include only the document body of
a child file thus allowing both files to be compiled individually.
\item
The packages \href{http://ctan.org/pkg/subdocs}{\textsf{subdocs}}
and \href{http://ctan.org/pkg/subfiles}{\textsf{subfiles}}
provide structures in which the main and child documents can be
encapsulated and allowing them to be compiled individually.
The inclusion mechanism is different from the conventional |\include|.
\item
The package \href{http://ctan.org/pkg/combine}{\textsf{combine}}
is an elaborate solution to combine several documents into one.
\end{itemize}
%
See also the CTAN topic \href{http://ctan.org/topic/subdocs}{\textsf{subdocs}}
for further related packages.
The present package differs from the above solutions in that
a document structure constructed with the conventional |\include| mechanism
just needs two extra commands at the top of every file
such that all constituent files can be compiled individually.

%%%%%%%%%%%%%%%%%%%%%%%%%%%%%%%%%%%%%%%%%%%%%%%%%%%%%%%%%%%%%%%%%%%%%%%%%%%%%%%%
%\subsection{Feature Suggestions}
%
%The following is a list of features which may be useful for future
%versions of this package:
%%
%\begin{itemize}
%\item
%\ldots
%\end{itemize}

%%%%%%%%%%%%%%%%%%%%%%%%%%%%%%%%%%%%%%%%%%%%%%%%%%%%%%%%%%%%%%%%%%%%%%%%%%%%%%%%
\subsection{Revision History}

%%%%%%%%%%%%%%%%%%%%%%%%%%%%%%%%%%%%%%%%
\paragraph{v2.0:} 2018/12/30

\begin{itemize}
\item
immediate forward processing
\item
added |\childdocby| mechanism
\item
manual restructured
\end{itemize}

%%%%%%%%%%%%%%%%%%%%%%%%%%%%%%%%%%%%%%%%
\paragraph{v1.6:} 2018/01/17

\begin{itemize}
\item
application for development of include files
\item
corrections to manual
\end{itemize}

%%%%%%%%%%%%%%%%%%%%%%%%%%%%%%%%%%%%%%%%
\paragraph{v1.5:} 2017/05/21

\begin{itemize}
\item
more complete structuring introduced
\item
|\childdocof| introduced
\item
|\childdoc| renamed to |\childdocmain|
\item
|\childredirect| renamed to |\childdocforward| and |\childdocforwardprefix|
and functionality expanded
\end{itemize}

%%%%%%%%%%%%%%%%%%%%%%%%%%%%%%%%%%%%%%%%
\paragraph{v1.0:} 2017/04/27

\begin{itemize}
\item
manual and install package
\item
first version published on CTAN
\end{itemize}

%%%%%%%%%%%%%%%%%%%%%%%%%%%%%%%%%%%%%%%%
\paragraph{v0.6:} 2017/04/26

\begin{itemize}
\item
redirection mechanism added
\end{itemize}

%%%%%%%%%%%%%%%%%%%%%%%%%%%%%%%%%%%%%%%%
\paragraph{v0.5:} 2017/04/26

\begin{itemize}
\item
functionality in definition file
\end{itemize}


%%%%%%%%%%%%%%%%%%%%%%%%%%%%%%%%%%%%%%%%%%%%%%%%%%%%%%%%%%%%%%%%%%%%%%%%%%%%%%%%
%%%%%%%%%%%%%%%%%%%%%%%%%%%%%%%%%%%%%%%%%%%%%%%%%%%%%%%%%%%%%%%%%%%%%%%%%%%%%%%%
%%%%%%%%%%%%%%%%%%%%%%%%%%%%%%%%%%%%%%%%%%%%%%%%%%%%%%%%%%%%%%%%%%%%%%%%%%%%%%%%
\appendix

\settowidth\MacroIndent{\rmfamily\scriptsize 000\ }

 \DocInput{childdoc.dtx}

\end{document}
%</driver>
% \fi
%
% %%%%%%%%%%%%%%%%%%%%%%%%%%%%%%%%%%%%%%%%%%%%%%%%%%%%%%%%%%%%%%%%%%%%%%%%%%%%%%
% %%%%%%%%%%%%%%%%%%%%%%%%%%%%%%%%%%%%%%%%%%%%%%%%%%%%%%%%%%%%%%%%%%%%%%%%%%%%%%
% \section{Sample}
%\iffalse
%<*samplemain>
%\fi
%
% The following presents a sample document
% with two chapters, two parts, a title page,
% a compile flag as well as three forwarding files to set the flag.
% It consists of eight |.tex| files:
% \begin{center}
% \begin{tabular}{ll}
% |cdocsamp.tex|&main file\\
% |cdocsch1.tex|&include file for chapter 1\\
% |cdocsch2.tex|&include file for chapter 2\\
% |cdocspt3.tex|&include file for part 3\\
% |cdocspt4.tex|&include file for part 4\\
% |cdocsdrf.tex|&forwarding file for main file in draft mode\\
% |cdocsfi1.tex|&forwarding file for final version of chapter 1\\
% |cdocsfi2.tex|&forwarding file for final version of chapter 2\\
% \end{tabular}
% \end{center}
% Each of the eight files can be compiled directly by the \LaTeX{} compiler.
%
% %%%%%%%%%%%%%%%%%%%%%%%%%%%%%%%%%%%%%%
% \paragraph{Main File.}
%
% The main file is called |cdocsamp.tex|.
%
% Load the \textsf{childdoc} definitions and
% declare the filename for the main document:
%    \begin{macrocode}
\input{childdoc.def}
\childdocmain{}
%    \end{macrocode}

% Optional override for |\version| flag:
%    \begin{macrocode}
%%\ifchilddoc\else\providecommand{\version}{draft}\fi
%    \end{macrocode}

% Define the default values for the |\version| flag
% (|final| for the main file and |draft| for childs):
%    \begin{macrocode}
\ifchilddoc
\providecommand{\version}{draft}
\else
\providecommand{\version}{final}
\fi
%    \end{macrocode}

% Load the standard document class:
%    \begin{macrocode}
\documentclass[12pt]{article}
%    \end{macrocode}

% Start the document body:
%    \begin{macrocode}
\begin{document}
%    \end{macrocode}

% Declare a title page.
% Print title, part of document being processed and version flag:
%    \begin{macrocode}
\addtocounter{page}{-1}
\begin{center}
{\LARGE\bfseries{}childdoc example\par}
\vspace{1cm}
\ifchilddoc
\ifchilddocmanual part\else chapter\fi:
`\childdocname' of `\childdocjob'\par
\else
main document: `\childdocjob'\par
\fi
version: \version\par
\end{center}
\newpage
%    \end{macrocode}

% Manually include selected file,
% otherwise process as usual:
%    \begin{macrocode}
\ifchilddocmanual
\section*{part `\childdocname'}
\input{\childdocname}
\else
%    \end{macrocode}

% Include the two chapters:
%    \begin{macrocode}
\include{cdocsch1}
\include{cdocsch2}
%    \end{macrocode}

% Include the two parts unless only chapters should be displayed:
%    \begin{macrocode}
\ifchilddoc\else
\section{part three}
\input{cdocspt3}
\section{part four}
\input{cdocspt4}
\fi
%    \end{macrocode}

% Process as usual until here:
%    \begin{macrocode}
\fi
%    \end{macrocode}

% End of document body:
%    \begin{macrocode}
\end{document}
%    \end{macrocode}
%\iffalse
%</samplemain>
%\fi
%
% %%%%%%%%%%%%%%%%%%%%%%%%%%%%%%%%%%%%%%
% \paragraph{Chapter Include Files.}
%
% The include files are called |cdocsch1.tex| and |cdocsch2.tex|.
%
%\iffalse
%<*samplechap1|samplechap2>
%\fi

% Optional override for |\version| flag:
%    \begin{macrocode}
%%\providecommand{\version}{final}
%    \end{macrocode}

% Include the main document:
%    \begin{macrocode}
\input{childdoc.def}
\childdocof{cdocsamp}
%    \end{macrocode}

%\iffalse
%</samplechap1|samplechap2>
%\fi
%
%\iffalse
%<*samplechap1>
%\fi
% Some text for chapter 1:
%    \begin{macrocode}
\section{one}
some text in chapter one
%    \end{macrocode}

%\iffalse
%</samplechap1>
%\fi
% Some text for chapter 2:
%\iffalse
%<*samplechap2>
%\fi
%    \begin{macrocode}
\section{two}
more text in chapter two
%    \end{macrocode}

%\iffalse
%</samplechap2>
%\fi
%
% %%%%%%%%%%%%%%%%%%%%%%%%%%%%%%%%%%%%%%
% \paragraph{Part Include Files.}
%
% The include files are called |cdocspt3.tex| and |cdocspt4.tex|.
%
%\iffalse
%<*samplepart3|samplepart4>
%\fi

% Optional override for |\version| flag:
%    \begin{macrocode}
%%\providecommand{\version}{final}
%    \end{macrocode}

% Include the main document:
%    \begin{macrocode}
\input{childdoc.def}
\childdocby{cdocsamp}
%    \end{macrocode}

%\iffalse
%</samplepart3|samplepart4>
%\fi
%
%\iffalse
%<*samplepart3>
%\fi
% Some text for part 3:
%    \begin{macrocode}
some text in part three
%    \end{macrocode}

%\iffalse
%</samplepart3>
%\fi
% Some text for part 4:
%\iffalse
%<*samplepart4>
%\fi
%    \begin{macrocode}
more text in part four
%    \end{macrocode}

%\iffalse
%</samplepart4>
%\fi
%
% %%%%%%%%%%%%%%%%%%%%%%%%%%%%%%%%%%%%%%
% \paragraph{Forwarding for a Complete Draft.}
%
% The following forwarding file |cdocsdrf.tex|
% compiles the main document in draft mode:
%\iffalse
%<*sampledraft>
%\fi
%    \begin{macrocode}
\def\version{draft}
\input{childdoc.def}
\childdocforward{cdocsamp}
%    \end{macrocode}

%\iffalse
%</sampledraft>
%\fi
%
% %%%%%%%%%%%%%%%%%%%%%%%%%%%%%%%%%%%%%%
% \paragraph{Forwarding for Final Version of the Chapters.}
%
% The following forwarding files |cdocsfn1.tex| and |cdocsfn2.tex|
% (with identical content)
% compile the final versions of the child documents
% |cdocsch1.tex| and |cdocsch2.tex|, respectively:
%\iffalse
%<*samplefinal>
%\fi
%    \begin{macrocode}
\def\version{final}
\input{childdoc.def}
\childdocforwardprefix[cdocsamp]{cdocsfn}{cdocsch}
%    \end{macrocode}

%\iffalse
%</samplefinal>
%\fi
%
% %%%%%%%%%%%%%%%%%%%%%%%%%%%%%%%%%%%%%%
% \paragraph{Command Line Processing.}
%
% The following three command lines generate the output files
% |cdocscld|, |cdocscl1| and |cdocscl2|
% which should be identical to
% |cdocsdrf|, |cdocsch1| and |cdocsfn2|, respectively:
% \begin{center}
% \begin{tabular}{l}
% |latex -jobname cdocscld \|\\
% |  "\def\version{draft}\input{childdoc.def}\childdocforward{cdocsamp}"|\\
% |latex -jobname cdocscl1 \|\\
% |  "\input{childdoc.def}\childdocforward[cdocsamp]{cdocsch1}"|\\
% |latex -jobname cdocscl2 \|\\
% |  "\def\version{final}\input{childdoc.def}\childdocforward{cdocsch2}"|
% \end{tabular}
% \end{center}
% Note that the trailing backslash on each first line
% merely continues the input to the second line
% (for convenient cut ant paste).
% Furthermore, the command |latex| can be replaced by any
% of its alternative versions such as |pdflatex|.
%
% %%%%%%%%%%%%%%%%%%%%%%%%%%%%%%%%%%%%%%%%%%%%%%%%%%%%%%%%%%%%%%%%%%%%%%%%%%%%%%
% %%%%%%%%%%%%%%%%%%%%%%%%%%%%%%%%%%%%%%%%%%%%%%%%%%%%%%%%%%%%%%%%%%%%%%%%%%%%%%
% \section{Implementation}
%\iffalse
%<*package>
%\fi
%
% This section describes the definitions file |childdoc.def|.

% The definitions cannot be loaded using |\usepackage| or |\RequirePackage|
% which has a mechanism to prevent loading a style file more than once.
% When loading the definitions by means of |\input|
% multiple instances have to be prevented manually:
%\iffalse
%This code needs to be before the `\ProvidesFile' directive
%which is defined at the beginning of this file.
%Therefore it is also placed there and commented out here.
%</package>
%<*discard>
%\fi
%    \begin{macrocode}
\ifdefined\childdocmain\endinput\fi
%    \end{macrocode}
%\iffalse
%</discard>
%<*package>
%\fi
%
% \macro{\ifchilddoc}
% \macro{\ifchilddocmanual}
% The conditional |\ifchilddoc| tells whether a
% child (true) or main (false) document is being compiled.
% The conditional |\ifchilddocmanual| tells whether
% the |\includeonly| mechanism is used (false) or
% the selection of child files must be performed manually (true).
% The definitions initialise to false:
%    \begin{macrocode}
\newif\ifchilddoc
\newif\ifchilddocmanual
%    \end{macrocode}

% \macro{\childdocname}
% \macro{\childdocjob}
% The macro |\childdocname| stores the name of the main document
% to be compiled. The macro |\childdocjob| stores the name of
% the document on which the \LaTeX{} compiler was originally invoked.
% The content of |\jobname| cannot be compared
% to filenames specified in the source due to different catcodes.
% The following code rescans |\jobname|, stores the result
% in |\childdocname| and saves a copy in |\childdocjob|:
%    \begin{macrocode}
\edef\childdocname{\scantokens\expandafter{\jobname\noexpand}}
\let\childdocjob\childdocname
%    \end{macrocode}

% \macro{\childdocdisable}
% The macro |\childdocdisable| prevents the main file
% from being processed more than once.
% At this stage, the main document command |\childdocmain|
% is assumed to be called once again where it should do nothing.
% Any subsequent call to it should prevent
% a secondary processing of the main document
% It overwrites the forwarding commands
% |\childdocof| and |\childdocforward|
% with empty macros to prevent further inclusions of the main document:
%    \begin{macrocode}
\newcommand{\childdocdisable}
{
  \renewcommand{\childdocmain}[1]{\renewcommand{\childdocmain}[1]{\endinput}}
  \renewcommand{\childdocof}[1]{}
  \renewcommand{\childdocby}[2][]{}
  \renewcommand{\childdocforward}[2][]{}
  \renewcommand{\childdocdisable}{}
}
%    \end{macrocode}

% \macro{\childdocmain}
% The macro |\childdocmain| is to be called at the top of the main file
% with nothing or the main filename (without extension) as argument.
% First, it breaks loops.
% If the argument is not empty and does not match |\childdocname|
% (which is set by the first inclusion of |childdoc.def|),
% |\ifchilddoc| is set to true, |\includeonly| is applied to the child file
% and |\jobname| is set to the main file
% (for proper handling of |.aux| files):
%    \begin{macrocode}
\newcommand{\childdocmain}[1]
{
  \childdocdisable\childdocmain{}
  \if?#1?\else
    \begingroup
      \def\childdoctmp{#1}
      \ifx\childdoctmp\childdocname
        \def\childdoctmp{}
      \else
        \def\childdoctmp
        {
          \childdoctrue
          \includeonly{\childdocname}
          \def\childdocjob{#1}
          \def\jobname{#1}
        }
      \fi
      \expandafter
    \endgroup
    \childdoctmp
  \fi
}
%    \end{macrocode}

% \macro{\childdocof}
% The command |\childdocof| redirects
% compilation to the main file |#1|.
%    \begin{macrocode}
\newcommand{\childdocof}[1]
{
  \childdocdisable
  \childdoctrue
  \includeonly{\childdocname}
  \def\jobname{#1}
  \def\childdocjob{#1}
  \input{#1}
}
%    \end{macrocode}

% \macro{\childdocby}
% The command |\childdocby| ....
%    \begin{macrocode}
\newcommand{\childdocby}[2][]
{
  \childdocdisable
  \childdoctrue
  \childdocmanualtrue
  \if?#1?\else
    \def\jobname{#2}
  \fi
  \def\childdocjob{#2}
  \input{#2}
  \endinput
}
%    \end{macrocode}

% \macro{\childdocforward}
% The command |\childdocforward| redirects
% compilation to the main file or
% (if the optional argument is given) a child file.
% Parameters are set as if the main file
% or a child file starting with |\childdocof| was compiled.
% Then compilation is handed over to the main file:
%    \begin{macrocode}
\newcommand{\childdocforward}[2][]
{
  \begingroup
    \if?#1?
      \def\childdoctmp
      {
        \def\childdocname{#2}
        \def\childdocjob{#2}
        \def\jobname{#2}
        \input{#2}
        \endinput
      }
    \else
      \def\childdoctmp
      {
        \childdocdisable
        \def\childdocname{#2}
        \childdoctrue
        \includeonly{#2}
        \def\childdocjob{#1}
        \def\jobname{#1}
        \input{#1}
        \endinput
      }
    \fi
    \expandafter
  \endgroup
  \childdoctmp
}
%    \end{macrocode}

% \macro{\childdocforwardprefix}
% The command |\childdocforwardprefix| redirects
% compilation to the main or a child file by means of a pattern.
% The prefix |#1| in the current filename is replaced by |#2|
% and the suffix of the current filename is kept
% (it is assumed that the filename does not contain the substring `|~~~|'
% which is used as a delimiter).
% Compilation is handed over to the new file by |\childdocforward|:
%    \begin{macrocode}
\newcommand{\childdocforwardprefix}[3][]
{
  \begingroup
    \def\childdocextract #2##1~~~{\def\childdoctmp{\childdocforward[#1]{#3##1}}}
    \expandafter\childdocextract\childdocname~~~
    \expandafter
  \endgroup
  \childdoctmp
}
%    \end{macrocode}

% \macro{\childdoc}
% The deprecated macro |\childdoc| is a legacy version of |\childdocmain|:
%    \begin{macrocode}
\newcommand{\childdoc}{\childdocmain}
%    \end{macrocode}

% \macro{\childdocredirect}
% The deprecated macro |\childdocredirect| is a legacy version
% of |\childdocforward| and |\childdocforwardprefix|:
%    \begin{macrocode}
\newcommand{\childdocredirect}[2][]
{
  \begingroup
    \if?#1?
      \def\childdoctmp{\childdocforward{#2}}
    \else
      \def\childdoctmp{\childdocforwardprefix{#1}{#2}}
    \fi
    \expandafter
  \endgroup
  \childdoctmp
}
%    \end{macrocode}

%\iffalse
%</package>
%\fi
%
\endinput
|\\
|\childdocforwardprefix[|\textit{main}|]{|\textit{prefix}|}{|\textit{dest}|}|
\end{tabular}
\end{center}
%
the destination file is determined by a pattern
depending on the current file:
To make this work, the current file must be called
`{\textit{prefix}\hspace{0.2em}\textit{suffix}}'
with \textit{prefix} matching precisely the argument.
Processing is then passed on to the file
`{\textit{dest}\hspace{0.2em}\textit{suffix}}'.
Surely, the same effect is achieved by
directly specifying the
argument `{\textit{dest}\hspace{0.2em}\textit{suffix}}'
in the first form.
However, that requires to set up a different file
for each child. With the alternative form of the command
all these files can have exactly the same content
which simplifies setting them up and maintaining them.

For example, the following file |draft.tex|
with a compilation flag |\version| as described in \secref{sec:flags}
compiles the main document as a draft:
%
\begin{center}
\begin{tabular}{l}
|\def\version{draft}|\\
|% \iffalse
%
% childdoc.dtx Copyright (C) 2017-2018 Niklas Beisert
%
% This work may be distributed and/or modified under the
% conditions of the LaTeX Project Public License, either version 1.3
% of this license or (at your option) any later version.
% The latest version of this license is in
%   http://www.latex-project.org/lppl.txt
% and version 1.3 or later is part of all distributions of LaTeX
% version 2005/12/01 or later.
%
% This work has the LPPL maintenance status `maintained'.
%
% The Current Maintainer of this work is Niklas Beisert.
%
% This work consists of the files childdoc.dtx and childdoc.ins
% and the derived files childdoc.def and cdocsamp.tex with
% cdocsch1.tex, cdocsch2.tex, cdocsdrf.tex, cdocsfn1.tex, cdocsfn2.tex.
%
%<package>\ifdefined\childdocmain\endinput\fi
%<package>\ProvidesFile{childdoc.def}[2018/12/30 v2.0 child document driver]
%<samplemain>\ProvidesFile{cdocsamp.tex}[2018/12/30 v2.0 sample for childdoc]
%<*driver>
%\ProvidesFile{childdoc.drv}[2018/12/30 v2.0 childdoc reference manual file]
\PassOptionsToClass{10pt,a4paper}{article}
\documentclass{ltxdoc}

\usepackage[margin=35mm]{geometry}
\usepackage{hyperref}
\usepackage{hyperxmp}
\usepackage[usenames]{color}

\hypersetup{colorlinks=true}
\hypersetup{pdfstartview=FitH}
\hypersetup{pdfpagemode=UseNone}
\hypersetup{pdfsource={}}
\hypersetup{pdflang={en-UK}}
\hypersetup{pdfcopyright={Copyright 2017-2018 Niklas Beisert.
  This work may be distributed and/or modified under the
  conditions of the LaTeX Project Public License, either version 1.3
  of this license or (at your option) any later version.}}
\hypersetup{pdflicenseurl={http://www.latex-project.org/lppl.txt}}
\hypersetup{pdfcontactaddress={ETH Zurich, ITP, HIT K,
  Wolfgang-Pauli-Strasse 27}}
\hypersetup{pdfcontactpostcode={8093}}
\hypersetup{pdfcontactcity={Zurich}}
\hypersetup{pdfcontactcountry={Switzerland}}
\hypersetup{pdfcontactemail={nbeisert@itp.phys.ethz.ch}}
\hypersetup{pdfcontacturl={http://people.phys.ethz.ch/\xmptilde nbeisert/}}

\newcommand{\secref}[1]{\hyperref[#1]{section \ref*{#1}}}

\parskip1ex
\parindent0pt
\let\olditemize\itemize
\def\itemize{\olditemize\parskip0pt}

\begin{document}

\title{The \textsf{childdoc} Package}
\hypersetup{pdftitle={The childdoc Package}}
\author{Niklas Beisert\\[2ex]
  Institut f\"ur Theoretische Physik\\
  Eidgen\"ossische Technische Hochschule Z\"urich\\
  Wolfgang-Pauli-Strasse 27, 8093 Z\"urich, Switzerland\\[1ex]
  \href{mailto:nbeisert@itp.phys.ethz.ch}
  {\texttt{nbeisert@itp.phys.ethz.ch}}}
\hypersetup{pdfauthor={Niklas Beisert}}
\hypersetup{pdfsubject={Manual for the LaTeX2e Package childdoc}}
\date{30 December 2018, \textsf{v2.0}}
\maketitle

\begin{abstract}\noindent
\textsf{childdoc} is a \LaTeXe{} package
that enables the direct compilation
of document sections included by |\include|
to individual files.
\end{abstract}

\begingroup
\parskip0ex
\tableofcontents
\endgroup

%%%%%%%%%%%%%%%%%%%%%%%%%%%%%%%%%%%%%%%%%%%%%%%%%%%%%%%%%%%%%%%%%%%%%%%%%%%%%%%%
%%%%%%%%%%%%%%%%%%%%%%%%%%%%%%%%%%%%%%%%%%%%%%%%%%%%%%%%%%%%%%%%%%%%%%%%%%%%%%%%
\section{Introduction}

\LaTeX{} provides a mechanism to structure a large document (such as a book)
into a main file and several child files (containing the chapters)
using the |\include| command.
This mechanism is beneficial for documents
which span hundreds of pages in order to
make the source file(s) more manageable.
Moreover, compilation can be restricted to
selected child files by means of the |\includeonly| command.
The latter feature can be used to reduce the compilation time while editing
(this was significantly more useful in the earlier days of \LaTeX{})
or to generate a smaller document which is easier to navigate.
Another application of |\includeonly| is to generate
documents consisting of selected parts of the complete document.

However, there are a few drawbacks of the plain |\include| mechanism:
\begin{itemize}
\item
The child files cannot be compiled on their own,
they can only be compiled via the main file.
A naive editing environment
(such as a text editor with an option
to have the current file processed by \LaTeX)
may require one to switch to the main file before compiling;
attempting to compile the child file produces errors.
\item
The main file must be modified (each time)
to adjust the |\includeonly| command
to the present needs. This easily leaves the main file in a messy state.
\item
The generated document will always carry the filename
of the main document. This is inconvenient if
several child files are to be compiled and
to be kept for distribution.
\end{itemize}

The present package provides a simple interface
to make child files individually compilable by \LaTeX{}.
Compiling a child file then has the same effect as compiling
the main file with an |\includeonly| command
to select the appropriate child.
Moreover the generated document will carry the name of the child
rather than the main file.
This resolves all three above issues.

This feature is meant to make the editing of books,
thesis documents and lecture notes somewhat more convenient.
However, the package can also be used efficiently for
composing a series of documents (such as exercise sheets)
which are typically distributed individually.
It then assists the author in generating the individual documents
(potentially in different versions)
as well as a document containing the collected series.
Another application is in developing style files
or other kinds of included material
where compilation of the style file could redirect
to a sample or test file.

%%%%%%%%%%%%%%%%%%%%%%%%%%%%%%%%%%%%%%%%%%%%%%%%%%%%%%%%%%%%%%%%%%%%%%%%%%%%%%%%
%%%%%%%%%%%%%%%%%%%%%%%%%%%%%%%%%%%%%%%%%%%%%%%%%%%%%%%%%%%%%%%%%%%%%%%%%%%%%%%%
\section{Usage}

First of all, the package \textsf{childdoc} is \emph{not} a standard
\LaTeXe{} |.sty| style file! Therefore it needs to be invoked in
a non-standard way.

%%%%%%%%%%%%%%%%%%%%%%%%%%%%%%%%%%%%%%%%%%%%%%%%%%%%%%%%%%%%%%%%%%%%%%%%%%%%%%%%
\subsection{Included Files}
\label{sec:include}

%%%%%%%%%%%%%%%%%%%%%%%%%%%%%%%%%%%%%%%%
\DescribeMacro{\childdocmain}
To use the package, add the commands
\begin{center}
\begin{tabular}{l}
|\input{childdoc.def}|\\
|\childdocmain{}|\\
\end{tabular}
\end{center}
at the very top of the main \LaTeX{} file,
in particular \emph{before} the |\documentclass| statement!
The argument of |\childdocmain| should be left empty
(but it must be present).

%%%%%%%%%%%%%%%%%%%%%%%%%%%%%%%%%%%%%%%%
\DescribeMacro{\childdocof}
Furthermore, add the commands
\begin{center}
\begin{tabular}{l}
|\input{childdoc.def}|\\
|\childdocof{|\textit{main}|}|\\
\end{tabular}
\end{center}
at the top of every child file \textit{child}
which is included by |\include{|\textit{child}|}|
from within the main file
(or at least for those files to be compiled individually).
The argument \textit{main} must be the filename of the main file.

There are a couple of
considerations in setting up the main and child documents:

%%%%%%%%%%%%%%%%%%%%%%%%%%%%%%%%%%%%%%%%
\paragraph{Restrictions.}

Please note the following restrictions:
\begin{itemize}
\item
|\childdocmain| must be called with one argument \textit{main}
to ensure compatibility with earlier version of the package.
It must either be empty (|\childdocmain{}|)
or precisely match the filename of the main file in which it is specified.
See \secref{sec:detection} for further information.
\item
The filename \textit{main} must be specified without the |.tex| extension.
\item
The filename \textit{main} is case sensitive
(even in case-insensitive file systems)
due to internal string comparison.
\item
The argument \textit{main} should be fully expanded, it cannot be a macro.
\item
Subdirectories and special characters should be avoided in filenames.
\item
The command |\childdocmain{|\textit{main}|}| must be followed by a whitespace.
It should not be followed immediately by another command
or by a comment mark `|%|'.
This is because the \TeX{} parser reads the token immediately following
the argument of |\childdocmain| and puts it
at the beginning of every child section;
however, a white\-space is ignored.
\end{itemize}

%%%%%%%%%%%%%%%%%%%%%%%%%%%%%%%%%%%%%%%%
\paragraph{Content of Main File.}

It is advisable to place all content in the child files included by |\include|.
Any output contained in the main file will appear in all child documents
unless suppressed manually;
it cannot be suppressed automatically by the |\includeonly| directive
and thus should normally be avoided.
A method to include some content in the main file
by means of conditional processing is described in \secref{sec:conditional}.

%%%%%%%%%%%%%%%%%%%%%%%%%%%%%%%%%%%%%%%%
\paragraph{Page Numbering.}

When only a part of the document is compiled,
the appropriate numbering of pages
(as well as other status parameters)
is determined from the |.aux| files.
The latter contain information from previous passes.
However this information needs to propagate through
all intermediate child documents.
Therefore the page numbering in child documents may well
be inconsistent until the complete document is compiled at least once.

A useful (if unconventional) way to always ensure a consistent
page numbering is to restart the numbering in each child document
and denote the pages by `\textit{child}|.|\textit{page}'
where \textit{child} represents the chapter/section number of the child file.
This can be achieved by the command
|\numberwithin{page}{|\textit{child}|}|
of the \textsf{amsmath} package
where \textit{child} can be |chapter| or |section|
depending on the chosen structuring.
Alternatively, one can modify the macro |\thepage| appropriately
and reset the counter |page| at the start of each child file.

%%%%%%%%%%%%%%%%%%%%%%%%%%%%%%%%%%%%%%%%%%%%%%%%%%%%%%%%%%%%%%%%%%%%%%%%%%%%%%%%
\subsection{Conditional Processing}
\label{sec:conditional}

The package provides a mechanism to compile different versions
of a document. To customise the versions further some conditional processing
can come in handy to distinguish which version is being compiled.
The package provides two macros to describe the compilation context:

%%%%%%%%%%%%%%%%%%%%%%%%%%%%%%%%%%%%%%%%
\DescribeMacro{\ifchilddoc}
The conditional |\ifchilddoc| distinguishes between the compilation of
child documents and the main document:
%
\begin{center}
|\ifchilddoc |\textit{child-code}| |[|\||else |\textit{main-code}]| \||fi|
\end{center}

%%%%%%%%%%%%%%%%%%%%%%%%%%%%%%%%%%%%%%%%
\DescribeMacro{\childdocname}
\DescribeMacro{\childdocjob}
The macro |\childdocname| contains the filename (without extension)
of the main or child file being processed.
Note that |\childdocjob| will always contain the name of the main file.

%%%%%%%%%%%%%%%%%%%%%%%%%%%%%%%%%%%%%%%%
\paragraph{Title Page.}

Conditional processing can be used to include a title or banner page
in the main document when proper precautions are taken.
Importantly, the code in the main file should ensure that the page counter
(as well as other status parameters which are stored in the |.aux| files)
takes the same value after the conditional processing.
Otherwise the page numbers may take divergent values
depending on which part is compiled.

For example, a title page could be declared by:
%
\begin{center}
\begin{tabular}{l}
|\ifchilddoc\||else|\\
|\addtocounter{page}{-1}|\\
\textit{code for title page}\\
|\newpage|\\
|\||fi|
\end{tabular}
\end{center}
%
A banner page for the child documents can be generated by:
%
\begin{center}
\begin{tabular}{l}
|\ifchilddoc|\\
|\addtocounter{page}{-1}|\\
\textit{code for banner page}\\
|\newpage|\\
|\||fi|
\end{tabular}
\end{center}
%
Here one could write a message such as:
\begin{center}
|This is the part \childdocname{} of \childdocjob{}.|
\end{center}

%%%%%%%%%%%%%%%%%%%%%%%%%%%%%%%%%%%%%%%%%%%%%%%%%%%%%%%%%%%%%%%%%%%%%%%%%%%%%%%%
\subsection{Flags}
\label{sec:flags}

The package makes it easy to generate different versions
of the main or child documents.
To this end compilation flags can be defined
and assigned different default values.
They will be particularly useful in conjunction
with the forwarding mechanism described in \secref{sec:forward}.

For example, it may be useful to have a flag |\version|
which can be set to |draft| or |final|.
The document source will contain some conditional code
depending on the value of |\version|.
Suppose further, the flag should default to |final| for the main file
and to |draft| for child files
which is a natural assignment for editing the document.
This is achieved by placing the following code
in the preamble of the main document
(below the |\childdocmain| directive):
%
\begin{center}
\begin{tabular}{l}
|\ifchilddoc|\\
|\providecommand{\version}{draft}|\\
|\||else|\\
|\providecommand{\version}{final}|\\
|\||fi|
\end{tabular}
\end{center}
%
The definition by |\providecommand| makes sure
that previous definitions are not overwritten.
Further statements |\providecommand{\version}{...}|
can thus be added before the above code to override it.

For the main file, one might add a line
(between |\childdocmain| and the above block)
%
\begin{center}
|%\ifchilddoc\||else\providecommand{\version}{draft}\||fi|
\end{center}
%
which can be uncommented to produce a draft version.
Likewise one can add a line to the very top of a child file
(above the |\childdocof{|\textit{main}|}| directive)
%
\begin{center}
|%\providecommand{\version}{final}|
\end{center}
%
which can be uncommented to produce the final version of this child document.

%%%%%%%%%%%%%%%%%%%%%%%%%%%%%%%%%%%%%%%%%%%%%%%%%%%%%%%%%%%%%%%%%%%%%%%%%%%%%%%%
\subsection{Forwarding}
\label{sec:forward}

Different versions of the main or child documents
using compilation flags as described in \secref{sec:flags}
can be (permanently) stored in different files
for convenient compilation, viewing and distribution.
To this end, the package defines a command
to pass on compilation to a different file:

%%%%%%%%%%%%%%%%%%%%%%%%%%%%%%%%%%%%%%%%
\DescribeMacro{\childdocforward}
The command |\childdocforward| redirects processing to
another source file:
%
\begin{center}
\begin{tabular}{l}
|\input{childdoc.def}|\\
|\childdocforward[|\textit{main}|]{|\textit{dest}|}|\\
\end{tabular}
\end{center}
%
The argument \textit{dest} is the destination file
(without extension).
It should be the main file or one of the child files.
Note that further \textsf{childdoc} directives
such as |\childdocof| and |\childdocforward|
in the indicated file will be processed in this form.
The optional argument \textit{main}
passes on directly to the main file \textit{main}
while pretending to compile the child \textit{dest}.
This form behaves as if \textit{dest}
issues |\childdocof{|\textit{main}|}| right away,
and no further \textsf{childdoc} directives will be processed.

%%%%%%%%%%%%%%%%%%%%%%%%%%%%%%%%%%%%%%%%
\DescribeMacro{\...prefix}
In the alternative form |\childdocforwardprefix|,
%
\begin{center}
\begin{tabular}{l}
|\input{childdoc.def}|\\
|\childdocforwardprefix[|\textit{main}|]{|\textit{prefix}|}{|\textit{dest}|}|
\end{tabular}
\end{center}
%
the destination file is determined by a pattern
depending on the current file:
To make this work, the current file must be called
`{\textit{prefix}\hspace{0.2em}\textit{suffix}}'
with \textit{prefix} matching precisely the argument.
Processing is then passed on to the file
`{\textit{dest}\hspace{0.2em}\textit{suffix}}'.
Surely, the same effect is achieved by
directly specifying the
argument `{\textit{dest}\hspace{0.2em}\textit{suffix}}'
in the first form.
However, that requires to set up a different file
for each child. With the alternative form of the command
all these files can have exactly the same content
which simplifies setting them up and maintaining them.

For example, the following file |draft.tex|
with a compilation flag |\version| as described in \secref{sec:flags}
compiles the main document as a draft:
%
\begin{center}
\begin{tabular}{l}
|\def\version{draft}|\\
|\input{childdoc.def}|\\
|\childdocforward{|\textit{main}|}|
\end{tabular}
\end{center}
%
Likewise, the following files |final|\textit{nn}|.tex|
compile the final version of the child document
|child|\textit{nn}|.tex|:
%
\begin{center}
\begin{tabular}{l}
|\def\version{final}|\\
|\input{childdoc.def}|\\
|\childdocforwardprefix{final}{child}|
\end{tabular}
\end{center}
%

Note that when several versions of a main file and/or of each child file
are to be generated, it may be convenient to set up a |Makefile| or
shell script to automatise the process.

%%%%%%%%%%%%%%%%%%%%%%%%%%%%%%%%%%%%%%%%%%%%%%%%%%%%%%%%%%%%%%%%%%%%%%%%%%%%%%%%
\subsection{Command Line Processing}
\label{sec:commandline}

The effect of redirection files can also be achieved by invoking
the \LaTeX{} compiler with a more elaborate command line.
Most conveniently this should be done as part
of a shell script or a |Makefile|.

When using \textsf{childdoc} in the main file, the following
command lines effectively perform a redirection
(note that depending on the shell being used,
backslashes may have to be doubled: `|\|' $\to$ `|\\|'):
%
\begin{center}
|... -jobname "|\textit{target}|" |\\|"|[\textit{flags}]%
|\input{childdoc.def}\childdocforward[|\textit{main}|]{|\textit{dest}|}"|
\end{center}
%
Here \textit{target} is the name of the output file,
\textit{main} is the name of the main file
and \textit{dest} is the name of the main or child file to be processed
(all filenames without extensions).
The optional argument \textit{main} can be omitted
if \textit{main} matches \textit{dest}.
Optionally, compilation \textit{flags} can be defined via |\def| commands.
This command line makes the \TeX{} engine believe
it is compiling the file \textit{target}
whose content is specified as the latter parameter.
The provided code then forwards the processing to
\textit{main} or \textit{dest} as described in \secref{sec:forward}.

%%%%%%%%%%%%%%%%%%%%%%%%%%%%%%%%%%%%%%%%%%%%%%%%%%%%%%%%%%%%%%%%%%%%%%%%%%%%%%%%
\subsection{Include by Input}
\label{sec:input}

Including child documents by |\include| has some restrictions by design.
Most notably, the content of a child document always occupies
its own set of pages; pages cannot be shared between child documents.
Usually, this behaviour makes perfect sense
because each child document contain an essential part of the document.
However, in some situations it may be desirable to compose
a document from a collection of parts
without having mandatory page breaks between then.
For this case, the package
provides a mechanism to include parts
by |\input| which can also be processed individually.
However, by construction this mechanism
requires manual handling of the content to be output.

%%%%%%%%%%%%%%%%%%%%%%%%%%%%%%%%%%%%%%%%
\DescribeMacro{\ifchilddocmanual}
The main file should be prepared as usual, see \secref{sec:include}.
However, the document body must make a distinction
between processing of an individual part and of the main document, e.g.:
%
\begin{center}
\begin{tabular}{l}
|\ifchilddocmanual|\\
|\input{\childdocname}|\\
|\||else|\\
\textit{document body with }|\input{|\textit{part}|}|\\
|\||fi|
\end{tabular}
\end{center}
%
The conditional |\ifchilddocmanual| is true whenever
a part to be included by |\input| is being compiled,
and the name of the part is stored in |\childdocname|.

%%%%%%%%%%%%%%%%%%%%%%%%%%%%%%%%%%%%%%%%
\DescribeMacro{\childdocby}
Each part to be included by |\input| should start with:
%
\begin{center}
\begin{tabular}{l}
|\input{childdoc.def}|\\
|\childdocby{|\textit{main}|}|\\
\end{tabular}
\end{center}
%
The directive |\childdocby| is similar to |\childdocof|
described in \secref{sec:include},
but the subsequent selection of content must be done manually.
To that end, both |\ifchilddoc| and |\ifchilddocmanual|
will be true upon processing of a part,
and the name of the part is stored in |\childdocname|.
Note that |\jobname| will be set to the filename of the current part
so that each part receives an individual |.aux| file
that does not interfere with the |.aux| file(s) of the main document.
This behaviour can be altered by the alternative form
|\childdocby[*]{|\textit{main}|}| (with a non-empty optional argument)
which uses the |.aux| file of the main document
by setting |\jobname| to \textit{main}.

%%%%%%%%%%%%%%%%%%%%%%%%%%%%%%%%%%%%%%%%%%%%%%%%%%%%%%%%%%%%%%%%%%%%%%%%%%%%%%%%
\subsection{Driver Development}
\label{sec:driver}

The \textsf{childdoc} mechanism can also be use for the development
of definition files such as \LaTeX{} styles or classes.
This case differs from the above setup with multiple parts
included by |\include| in that no |\includeonly| should be invoked.
This can be achieved by starting the include file
(before |\ProvidesPackage|) with:
%
\begin{center}
\begin{tabular}{l}
|\input{childdoc.def}|\\
|\childdocforward{|\textit{main}|}|\\
\end{tabular}
\end{center}
%
or alternatively with:
%
\begin{center}
\begin{tabular}{l}
|\input{childdoc.def}|\\
|\childdocby{|\textit{main}|}|\\
\end{tabular}
\end{center}
%
Both forms have slightly different effects as described above.
The main file is prepared as usual, see \secref{sec:include}.

%%%%%%%%%%%%%%%%%%%%%%%%%%%%%%%%%%%%%%%%%%%%%%%%%%%%%%%%%%%%%%%%%%%%%%%%%%%%%%%%
\subsection{Legacy Detection}
\label{sec:detection}

The directive |\childdocmain| in the main file can detect
whether the complete document or merely a child is to be compiled
even without using the directive |\childdocof|.
This method is deprecated because it is less robust
and there is no compelling reason to use it;
it is merely provided for backward compatibility
and it may be removed in future versions.

If the detection mechanism is to be used,
it is mandatory to correctly specify
the filename of the main file as the argument of |\childdocmain|:
%
\begin{center}
\begin{tabular}{l}
|\input{childdoc.def}|\\
|\childdocmain{|\textit{main}|}|\\
\end{tabular}
\end{center}
%
If |\jobname| does not match the argument \textit{main} of |\childdocmain|,
it is assumed that |\jobname| points to the child file to be compiled.
When using |\childdocmain| with the main file specified as argument,
it suffices to start a child file
with just |\input{|\textit{main}|}|
without loading of the package and using |\childdocof|.
If instead all processing is done
with the appropriate \textsf{childdoc} directives,
the argument of \textit{main} of |\childdocmain| can be empty.

An alternative version of the command line processing described
in \secref{sec:commandline} using the detection mechanism reads:
%
\begin{center}
|... -jobname "|\textit{target}|" "|[\textit{flags}]%
[|\def\jobname{|\textit{dest}|}|]|\input{|\textit{main}|}"|
\end{center}

%%%%%%%%%%%%%%%%%%%%%%%%%%%%%%%%%%%%%%%%%%%%%%%%%%%%%%%%%%%%%%%%%%%%%%%%%%%%%%%%
\subsection{Manual Code}
\label{sec:manual}

In case one cannot be certain whether the definitions file |childdoc.def|
is installed on the target \TeX{} distribution
and one prefers not to ship it,
it is conceivable to paste a few relevant commands into the sources.

To that end, drop all statements |\input{childdoc.def}|
and perform the replacements as outlined below.
Instead of |\childdocmain{|\textit{main}|}| add the following code
to the top of the main file:
%
\begin{center}
\begin{tabular}{l}
|\||ifdefined\childdocname\endinput\||fi\newif\ifchilddoc|\\
|\edef\childdocname{\scantokens\expandafter{\jobname\noexpand}}|\\
|\def\childdocmain{|\textit{main}|}\||ifx\childdocmain\childdocname\||else|\\
|\childdoctrue\includeonly{\childdocname}\let\jobname\childdocmain\||fi|\\
\end{tabular}
\end{center}
%
Instead of |\childdocof{|\textit{main}|}| just include the main file
at the top of each child file:
%
\begin{center}
|\input{|\textit{main}|}|
\end{center}
%
A simple redirection |\childdocforward{|\textit{dest}|}| is achieved by:
%
\begin{center}
|\def\jobname{|\textit{dest}|}\input{\jobname}|
\end{center}
%
The redirection with prefix
|\childdocforwardprefix[|\textit{prefix}|]{|\textit{dest}|}|
is accomplished by:
%
\begin{center}
\begin{tabular}{l}
|{\edef\jobname{\scantokens\expandafter{\jobname\noexpand}}|\\
|\def\redirectjob |\textit{prefix}|#1~~~{\gdef\jobname{|\textit{dest}|#1}}|\\
|\expandafter\redirectjob\jobname~~~}\input{\jobname}|
\end{tabular}
\end{center}

In an alternative approach,
child documents can be compiled by a specific command line
without additional code or specific definitions:
%
\begin{center}
|... -jobname "|\textit{target}|" "|[\textit{flags}]%
|\includeonly{|\textit{dest}|}\input{|\textit{main}|}"|
\end{center}
%

%%%%%%%%%%%%%%%%%%%%%%%%%%%%%%%%%%%%%%%%%%%%%%%%%%%%%%%%%%%%%%%%%%%%%%%%%%%%%%%%
%%%%%%%%%%%%%%%%%%%%%%%%%%%%%%%%%%%%%%%%%%%%%%%%%%%%%%%%%%%%%%%%%%%%%%%%%%%%%%%%
\section{Information}

%%%%%%%%%%%%%%%%%%%%%%%%%%%%%%%%%%%%%%%%%%%%%%%%%%%%%%%%%%%%%%%%%%%%%%%%%%%%%%%%
\subsection{Copyright}

Copyright \copyright{} 2017--2018 Niklas Beisert

This work may be distributed and/or modified under the
conditions of the \LaTeX{} Project Public License, either version 1.3
of this license or (at your option) any later version.
The latest version of this license is in
  \url{http://www.latex-project.org/lppl.txt}
and version 1.3 or later is part of all distributions of \LaTeX{}
version 2005/12/01 or later.

This work has the LPPL maintenance status `maintained'.

The Current Maintainer of this work is Niklas Beisert.

This work consists of the files |README.txt|, |childdoc.ins| and |childdoc.dtx|
as well as the derived files |childdoc.def|, |cdocsamp.tex|
with |cdocsch1.tex|, |cdocsch2.tex|, |cdocspt3.tex|, |cdocspt4.tex|,
|cdocsdrf.tex|, |cdocsfn1.tex|, |cdocsfn2.tex|
as well as |childdoc.pdf|.

%%%%%%%%%%%%%%%%%%%%%%%%%%%%%%%%%%%%%%%%%%%%%%%%%%%%%%%%%%%%%%%%%%%%%%%%%%%%%%%%
\subsection{Files and Installation}

The package consists of the files:
%
\begin{center}
\begin{tabular}{ll}
    |README.txt|   & readme file \\
    |childdoc.ins| & installation file \\
    |childdoc.dtx| & source file \\
    |childdoc.def| & definition file \\
    |cdocsamp.tex| & sample main file \\
    |cdocsch1.tex| & sample include file \\
    |cdocsch2.tex| & sample include file \\
    |cdocspt3.tex| & sample part file \\
    |cdocspt4.tex| & sample part file \\
    |cdocsdrf.tex| & sample redirection file \\
    |cdocsfn1.tex| & sample redirection file \\
    |cdocsfn2.tex| & sample redirection file \\
    |childdoc.pdf| & manual
\end{tabular}
\end{center}
%
The distribution consists of the files
|README.txt|, |childdoc.ins| and |childdoc.dtx|.
%
\begin{itemize}
\item
Run (pdf)\LaTeX{} on |childdoc.dtx|
to compile the manual |childdoc.pdf| (this file).
\item
Run \LaTeX{} on |childdoc.ins| to create the definitions file |childdoc.def|
and the sample |cdocsamp.tex| with include files
|cdocsch1.tex|, |cdocsch2.tex|, |cdocspt3.tex|, |cdocspt4.tex|,
|cdocsdrf.tex|, |cdocsfn1.tex|, |cdocsfn2.tex|.
Then copy the file |childdoc.def| to an appropriate directory of your \LaTeX{}
distribution, e.g.\ \textit{texmf-root}|/tex/latex/childdoc|.
\end{itemize}

%%%%%%%%%%%%%%%%%%%%%%%%%%%%%%%%%%%%%%%%%%%%%%%%%%%%%%%%%%%%%%%%%%%%%%%%%%%%%%%%
\subsection{Related CTAN Packages}

There are several other packages which offer a similar functionality:
%
\begin{itemize}
\item
The packages
\href{http://ctan.org/pkg/docmute}{\textsf{docmute}},
\href{http://ctan.org/pkg/includex}{\textsf{includex}} and
\href{http://ctan.org/pkg/standalone}{\textsf{standalone}}
provide commands to include only the document body of
a child file thus allowing both files to be compiled individually.
\item
The packages \href{http://ctan.org/pkg/subdocs}{\textsf{subdocs}}
and \href{http://ctan.org/pkg/subfiles}{\textsf{subfiles}}
provide structures in which the main and child documents can be
encapsulated and allowing them to be compiled individually.
The inclusion mechanism is different from the conventional |\include|.
\item
The package \href{http://ctan.org/pkg/combine}{\textsf{combine}}
is an elaborate solution to combine several documents into one.
\end{itemize}
%
See also the CTAN topic \href{http://ctan.org/topic/subdocs}{\textsf{subdocs}}
for further related packages.
The present package differs from the above solutions in that
a document structure constructed with the conventional |\include| mechanism
just needs two extra commands at the top of every file
such that all constituent files can be compiled individually.

%%%%%%%%%%%%%%%%%%%%%%%%%%%%%%%%%%%%%%%%%%%%%%%%%%%%%%%%%%%%%%%%%%%%%%%%%%%%%%%%
%\subsection{Feature Suggestions}
%
%The following is a list of features which may be useful for future
%versions of this package:
%%
%\begin{itemize}
%\item
%\ldots
%\end{itemize}

%%%%%%%%%%%%%%%%%%%%%%%%%%%%%%%%%%%%%%%%%%%%%%%%%%%%%%%%%%%%%%%%%%%%%%%%%%%%%%%%
\subsection{Revision History}

%%%%%%%%%%%%%%%%%%%%%%%%%%%%%%%%%%%%%%%%
\paragraph{v2.0:} 2018/12/30

\begin{itemize}
\item
immediate forward processing
\item
added |\childdocby| mechanism
\item
manual restructured
\end{itemize}

%%%%%%%%%%%%%%%%%%%%%%%%%%%%%%%%%%%%%%%%
\paragraph{v1.6:} 2018/01/17

\begin{itemize}
\item
application for development of include files
\item
corrections to manual
\end{itemize}

%%%%%%%%%%%%%%%%%%%%%%%%%%%%%%%%%%%%%%%%
\paragraph{v1.5:} 2017/05/21

\begin{itemize}
\item
more complete structuring introduced
\item
|\childdocof| introduced
\item
|\childdoc| renamed to |\childdocmain|
\item
|\childredirect| renamed to |\childdocforward| and |\childdocforwardprefix|
and functionality expanded
\end{itemize}

%%%%%%%%%%%%%%%%%%%%%%%%%%%%%%%%%%%%%%%%
\paragraph{v1.0:} 2017/04/27

\begin{itemize}
\item
manual and install package
\item
first version published on CTAN
\end{itemize}

%%%%%%%%%%%%%%%%%%%%%%%%%%%%%%%%%%%%%%%%
\paragraph{v0.6:} 2017/04/26

\begin{itemize}
\item
redirection mechanism added
\end{itemize}

%%%%%%%%%%%%%%%%%%%%%%%%%%%%%%%%%%%%%%%%
\paragraph{v0.5:} 2017/04/26

\begin{itemize}
\item
functionality in definition file
\end{itemize}


%%%%%%%%%%%%%%%%%%%%%%%%%%%%%%%%%%%%%%%%%%%%%%%%%%%%%%%%%%%%%%%%%%%%%%%%%%%%%%%%
%%%%%%%%%%%%%%%%%%%%%%%%%%%%%%%%%%%%%%%%%%%%%%%%%%%%%%%%%%%%%%%%%%%%%%%%%%%%%%%%
%%%%%%%%%%%%%%%%%%%%%%%%%%%%%%%%%%%%%%%%%%%%%%%%%%%%%%%%%%%%%%%%%%%%%%%%%%%%%%%%
\appendix

\settowidth\MacroIndent{\rmfamily\scriptsize 000\ }

 \DocInput{childdoc.dtx}

\end{document}
%</driver>
% \fi
%
% %%%%%%%%%%%%%%%%%%%%%%%%%%%%%%%%%%%%%%%%%%%%%%%%%%%%%%%%%%%%%%%%%%%%%%%%%%%%%%
% %%%%%%%%%%%%%%%%%%%%%%%%%%%%%%%%%%%%%%%%%%%%%%%%%%%%%%%%%%%%%%%%%%%%%%%%%%%%%%
% \section{Sample}
%\iffalse
%<*samplemain>
%\fi
%
% The following presents a sample document
% with two chapters, two parts, a title page,
% a compile flag as well as three forwarding files to set the flag.
% It consists of eight |.tex| files:
% \begin{center}
% \begin{tabular}{ll}
% |cdocsamp.tex|&main file\\
% |cdocsch1.tex|&include file for chapter 1\\
% |cdocsch2.tex|&include file for chapter 2\\
% |cdocspt3.tex|&include file for part 3\\
% |cdocspt4.tex|&include file for part 4\\
% |cdocsdrf.tex|&forwarding file for main file in draft mode\\
% |cdocsfi1.tex|&forwarding file for final version of chapter 1\\
% |cdocsfi2.tex|&forwarding file for final version of chapter 2\\
% \end{tabular}
% \end{center}
% Each of the eight files can be compiled directly by the \LaTeX{} compiler.
%
% %%%%%%%%%%%%%%%%%%%%%%%%%%%%%%%%%%%%%%
% \paragraph{Main File.}
%
% The main file is called |cdocsamp.tex|.
%
% Load the \textsf{childdoc} definitions and
% declare the filename for the main document:
%    \begin{macrocode}
\input{childdoc.def}
\childdocmain{}
%    \end{macrocode}

% Optional override for |\version| flag:
%    \begin{macrocode}
%%\ifchilddoc\else\providecommand{\version}{draft}\fi
%    \end{macrocode}

% Define the default values for the |\version| flag
% (|final| for the main file and |draft| for childs):
%    \begin{macrocode}
\ifchilddoc
\providecommand{\version}{draft}
\else
\providecommand{\version}{final}
\fi
%    \end{macrocode}

% Load the standard document class:
%    \begin{macrocode}
\documentclass[12pt]{article}
%    \end{macrocode}

% Start the document body:
%    \begin{macrocode}
\begin{document}
%    \end{macrocode}

% Declare a title page.
% Print title, part of document being processed and version flag:
%    \begin{macrocode}
\addtocounter{page}{-1}
\begin{center}
{\LARGE\bfseries{}childdoc example\par}
\vspace{1cm}
\ifchilddoc
\ifchilddocmanual part\else chapter\fi:
`\childdocname' of `\childdocjob'\par
\else
main document: `\childdocjob'\par
\fi
version: \version\par
\end{center}
\newpage
%    \end{macrocode}

% Manually include selected file,
% otherwise process as usual:
%    \begin{macrocode}
\ifchilddocmanual
\section*{part `\childdocname'}
\input{\childdocname}
\else
%    \end{macrocode}

% Include the two chapters:
%    \begin{macrocode}
\include{cdocsch1}
\include{cdocsch2}
%    \end{macrocode}

% Include the two parts unless only chapters should be displayed:
%    \begin{macrocode}
\ifchilddoc\else
\section{part three}
\input{cdocspt3}
\section{part four}
\input{cdocspt4}
\fi
%    \end{macrocode}

% Process as usual until here:
%    \begin{macrocode}
\fi
%    \end{macrocode}

% End of document body:
%    \begin{macrocode}
\end{document}
%    \end{macrocode}
%\iffalse
%</samplemain>
%\fi
%
% %%%%%%%%%%%%%%%%%%%%%%%%%%%%%%%%%%%%%%
% \paragraph{Chapter Include Files.}
%
% The include files are called |cdocsch1.tex| and |cdocsch2.tex|.
%
%\iffalse
%<*samplechap1|samplechap2>
%\fi

% Optional override for |\version| flag:
%    \begin{macrocode}
%%\providecommand{\version}{final}
%    \end{macrocode}

% Include the main document:
%    \begin{macrocode}
\input{childdoc.def}
\childdocof{cdocsamp}
%    \end{macrocode}

%\iffalse
%</samplechap1|samplechap2>
%\fi
%
%\iffalse
%<*samplechap1>
%\fi
% Some text for chapter 1:
%    \begin{macrocode}
\section{one}
some text in chapter one
%    \end{macrocode}

%\iffalse
%</samplechap1>
%\fi
% Some text for chapter 2:
%\iffalse
%<*samplechap2>
%\fi
%    \begin{macrocode}
\section{two}
more text in chapter two
%    \end{macrocode}

%\iffalse
%</samplechap2>
%\fi
%
% %%%%%%%%%%%%%%%%%%%%%%%%%%%%%%%%%%%%%%
% \paragraph{Part Include Files.}
%
% The include files are called |cdocspt3.tex| and |cdocspt4.tex|.
%
%\iffalse
%<*samplepart3|samplepart4>
%\fi

% Optional override for |\version| flag:
%    \begin{macrocode}
%%\providecommand{\version}{final}
%    \end{macrocode}

% Include the main document:
%    \begin{macrocode}
\input{childdoc.def}
\childdocby{cdocsamp}
%    \end{macrocode}

%\iffalse
%</samplepart3|samplepart4>
%\fi
%
%\iffalse
%<*samplepart3>
%\fi
% Some text for part 3:
%    \begin{macrocode}
some text in part three
%    \end{macrocode}

%\iffalse
%</samplepart3>
%\fi
% Some text for part 4:
%\iffalse
%<*samplepart4>
%\fi
%    \begin{macrocode}
more text in part four
%    \end{macrocode}

%\iffalse
%</samplepart4>
%\fi
%
% %%%%%%%%%%%%%%%%%%%%%%%%%%%%%%%%%%%%%%
% \paragraph{Forwarding for a Complete Draft.}
%
% The following forwarding file |cdocsdrf.tex|
% compiles the main document in draft mode:
%\iffalse
%<*sampledraft>
%\fi
%    \begin{macrocode}
\def\version{draft}
\input{childdoc.def}
\childdocforward{cdocsamp}
%    \end{macrocode}

%\iffalse
%</sampledraft>
%\fi
%
% %%%%%%%%%%%%%%%%%%%%%%%%%%%%%%%%%%%%%%
% \paragraph{Forwarding for Final Version of the Chapters.}
%
% The following forwarding files |cdocsfn1.tex| and |cdocsfn2.tex|
% (with identical content)
% compile the final versions of the child documents
% |cdocsch1.tex| and |cdocsch2.tex|, respectively:
%\iffalse
%<*samplefinal>
%\fi
%    \begin{macrocode}
\def\version{final}
\input{childdoc.def}
\childdocforwardprefix[cdocsamp]{cdocsfn}{cdocsch}
%    \end{macrocode}

%\iffalse
%</samplefinal>
%\fi
%
% %%%%%%%%%%%%%%%%%%%%%%%%%%%%%%%%%%%%%%
% \paragraph{Command Line Processing.}
%
% The following three command lines generate the output files
% |cdocscld|, |cdocscl1| and |cdocscl2|
% which should be identical to
% |cdocsdrf|, |cdocsch1| and |cdocsfn2|, respectively:
% \begin{center}
% \begin{tabular}{l}
% |latex -jobname cdocscld \|\\
% |  "\def\version{draft}\input{childdoc.def}\childdocforward{cdocsamp}"|\\
% |latex -jobname cdocscl1 \|\\
% |  "\input{childdoc.def}\childdocforward[cdocsamp]{cdocsch1}"|\\
% |latex -jobname cdocscl2 \|\\
% |  "\def\version{final}\input{childdoc.def}\childdocforward{cdocsch2}"|
% \end{tabular}
% \end{center}
% Note that the trailing backslash on each first line
% merely continues the input to the second line
% (for convenient cut ant paste).
% Furthermore, the command |latex| can be replaced by any
% of its alternative versions such as |pdflatex|.
%
% %%%%%%%%%%%%%%%%%%%%%%%%%%%%%%%%%%%%%%%%%%%%%%%%%%%%%%%%%%%%%%%%%%%%%%%%%%%%%%
% %%%%%%%%%%%%%%%%%%%%%%%%%%%%%%%%%%%%%%%%%%%%%%%%%%%%%%%%%%%%%%%%%%%%%%%%%%%%%%
% \section{Implementation}
%\iffalse
%<*package>
%\fi
%
% This section describes the definitions file |childdoc.def|.

% The definitions cannot be loaded using |\usepackage| or |\RequirePackage|
% which has a mechanism to prevent loading a style file more than once.
% When loading the definitions by means of |\input|
% multiple instances have to be prevented manually:
%\iffalse
%This code needs to be before the `\ProvidesFile' directive
%which is defined at the beginning of this file.
%Therefore it is also placed there and commented out here.
%</package>
%<*discard>
%\fi
%    \begin{macrocode}
\ifdefined\childdocmain\endinput\fi
%    \end{macrocode}
%\iffalse
%</discard>
%<*package>
%\fi
%
% \macro{\ifchilddoc}
% \macro{\ifchilddocmanual}
% The conditional |\ifchilddoc| tells whether a
% child (true) or main (false) document is being compiled.
% The conditional |\ifchilddocmanual| tells whether
% the |\includeonly| mechanism is used (false) or
% the selection of child files must be performed manually (true).
% The definitions initialise to false:
%    \begin{macrocode}
\newif\ifchilddoc
\newif\ifchilddocmanual
%    \end{macrocode}

% \macro{\childdocname}
% \macro{\childdocjob}
% The macro |\childdocname| stores the name of the main document
% to be compiled. The macro |\childdocjob| stores the name of
% the document on which the \LaTeX{} compiler was originally invoked.
% The content of |\jobname| cannot be compared
% to filenames specified in the source due to different catcodes.
% The following code rescans |\jobname|, stores the result
% in |\childdocname| and saves a copy in |\childdocjob|:
%    \begin{macrocode}
\edef\childdocname{\scantokens\expandafter{\jobname\noexpand}}
\let\childdocjob\childdocname
%    \end{macrocode}

% \macro{\childdocdisable}
% The macro |\childdocdisable| prevents the main file
% from being processed more than once.
% At this stage, the main document command |\childdocmain|
% is assumed to be called once again where it should do nothing.
% Any subsequent call to it should prevent
% a secondary processing of the main document
% It overwrites the forwarding commands
% |\childdocof| and |\childdocforward|
% with empty macros to prevent further inclusions of the main document:
%    \begin{macrocode}
\newcommand{\childdocdisable}
{
  \renewcommand{\childdocmain}[1]{\renewcommand{\childdocmain}[1]{\endinput}}
  \renewcommand{\childdocof}[1]{}
  \renewcommand{\childdocby}[2][]{}
  \renewcommand{\childdocforward}[2][]{}
  \renewcommand{\childdocdisable}{}
}
%    \end{macrocode}

% \macro{\childdocmain}
% The macro |\childdocmain| is to be called at the top of the main file
% with nothing or the main filename (without extension) as argument.
% First, it breaks loops.
% If the argument is not empty and does not match |\childdocname|
% (which is set by the first inclusion of |childdoc.def|),
% |\ifchilddoc| is set to true, |\includeonly| is applied to the child file
% and |\jobname| is set to the main file
% (for proper handling of |.aux| files):
%    \begin{macrocode}
\newcommand{\childdocmain}[1]
{
  \childdocdisable\childdocmain{}
  \if?#1?\else
    \begingroup
      \def\childdoctmp{#1}
      \ifx\childdoctmp\childdocname
        \def\childdoctmp{}
      \else
        \def\childdoctmp
        {
          \childdoctrue
          \includeonly{\childdocname}
          \def\childdocjob{#1}
          \def\jobname{#1}
        }
      \fi
      \expandafter
    \endgroup
    \childdoctmp
  \fi
}
%    \end{macrocode}

% \macro{\childdocof}
% The command |\childdocof| redirects
% compilation to the main file |#1|.
%    \begin{macrocode}
\newcommand{\childdocof}[1]
{
  \childdocdisable
  \childdoctrue
  \includeonly{\childdocname}
  \def\jobname{#1}
  \def\childdocjob{#1}
  \input{#1}
}
%    \end{macrocode}

% \macro{\childdocby}
% The command |\childdocby| ....
%    \begin{macrocode}
\newcommand{\childdocby}[2][]
{
  \childdocdisable
  \childdoctrue
  \childdocmanualtrue
  \if?#1?\else
    \def\jobname{#2}
  \fi
  \def\childdocjob{#2}
  \input{#2}
  \endinput
}
%    \end{macrocode}

% \macro{\childdocforward}
% The command |\childdocforward| redirects
% compilation to the main file or
% (if the optional argument is given) a child file.
% Parameters are set as if the main file
% or a child file starting with |\childdocof| was compiled.
% Then compilation is handed over to the main file:
%    \begin{macrocode}
\newcommand{\childdocforward}[2][]
{
  \begingroup
    \if?#1?
      \def\childdoctmp
      {
        \def\childdocname{#2}
        \def\childdocjob{#2}
        \def\jobname{#2}
        \input{#2}
        \endinput
      }
    \else
      \def\childdoctmp
      {
        \childdocdisable
        \def\childdocname{#2}
        \childdoctrue
        \includeonly{#2}
        \def\childdocjob{#1}
        \def\jobname{#1}
        \input{#1}
        \endinput
      }
    \fi
    \expandafter
  \endgroup
  \childdoctmp
}
%    \end{macrocode}

% \macro{\childdocforwardprefix}
% The command |\childdocforwardprefix| redirects
% compilation to the main or a child file by means of a pattern.
% The prefix |#1| in the current filename is replaced by |#2|
% and the suffix of the current filename is kept
% (it is assumed that the filename does not contain the substring `|~~~|'
% which is used as a delimiter).
% Compilation is handed over to the new file by |\childdocforward|:
%    \begin{macrocode}
\newcommand{\childdocforwardprefix}[3][]
{
  \begingroup
    \def\childdocextract #2##1~~~{\def\childdoctmp{\childdocforward[#1]{#3##1}}}
    \expandafter\childdocextract\childdocname~~~
    \expandafter
  \endgroup
  \childdoctmp
}
%    \end{macrocode}

% \macro{\childdoc}
% The deprecated macro |\childdoc| is a legacy version of |\childdocmain|:
%    \begin{macrocode}
\newcommand{\childdoc}{\childdocmain}
%    \end{macrocode}

% \macro{\childdocredirect}
% The deprecated macro |\childdocredirect| is a legacy version
% of |\childdocforward| and |\childdocforwardprefix|:
%    \begin{macrocode}
\newcommand{\childdocredirect}[2][]
{
  \begingroup
    \if?#1?
      \def\childdoctmp{\childdocforward{#2}}
    \else
      \def\childdoctmp{\childdocforwardprefix{#1}{#2}}
    \fi
    \expandafter
  \endgroup
  \childdoctmp
}
%    \end{macrocode}

%\iffalse
%</package>
%\fi
%
\endinput
|\\
|\childdocforward{|\textit{main}|}|
\end{tabular}
\end{center}
%
Likewise, the following files |final|\textit{nn}|.tex|
compile the final version of the child document
|child|\textit{nn}|.tex|:
%
\begin{center}
\begin{tabular}{l}
|\def\version{final}|\\
|% \iffalse
%
% childdoc.dtx Copyright (C) 2017-2018 Niklas Beisert
%
% This work may be distributed and/or modified under the
% conditions of the LaTeX Project Public License, either version 1.3
% of this license or (at your option) any later version.
% The latest version of this license is in
%   http://www.latex-project.org/lppl.txt
% and version 1.3 or later is part of all distributions of LaTeX
% version 2005/12/01 or later.
%
% This work has the LPPL maintenance status `maintained'.
%
% The Current Maintainer of this work is Niklas Beisert.
%
% This work consists of the files childdoc.dtx and childdoc.ins
% and the derived files childdoc.def and cdocsamp.tex with
% cdocsch1.tex, cdocsch2.tex, cdocsdrf.tex, cdocsfn1.tex, cdocsfn2.tex.
%
%<package>\ifdefined\childdocmain\endinput\fi
%<package>\ProvidesFile{childdoc.def}[2018/12/30 v2.0 child document driver]
%<samplemain>\ProvidesFile{cdocsamp.tex}[2018/12/30 v2.0 sample for childdoc]
%<*driver>
%\ProvidesFile{childdoc.drv}[2018/12/30 v2.0 childdoc reference manual file]
\PassOptionsToClass{10pt,a4paper}{article}
\documentclass{ltxdoc}

\usepackage[margin=35mm]{geometry}
\usepackage{hyperref}
\usepackage{hyperxmp}
\usepackage[usenames]{color}

\hypersetup{colorlinks=true}
\hypersetup{pdfstartview=FitH}
\hypersetup{pdfpagemode=UseNone}
\hypersetup{pdfsource={}}
\hypersetup{pdflang={en-UK}}
\hypersetup{pdfcopyright={Copyright 2017-2018 Niklas Beisert.
  This work may be distributed and/or modified under the
  conditions of the LaTeX Project Public License, either version 1.3
  of this license or (at your option) any later version.}}
\hypersetup{pdflicenseurl={http://www.latex-project.org/lppl.txt}}
\hypersetup{pdfcontactaddress={ETH Zurich, ITP, HIT K,
  Wolfgang-Pauli-Strasse 27}}
\hypersetup{pdfcontactpostcode={8093}}
\hypersetup{pdfcontactcity={Zurich}}
\hypersetup{pdfcontactcountry={Switzerland}}
\hypersetup{pdfcontactemail={nbeisert@itp.phys.ethz.ch}}
\hypersetup{pdfcontacturl={http://people.phys.ethz.ch/\xmptilde nbeisert/}}

\newcommand{\secref}[1]{\hyperref[#1]{section \ref*{#1}}}

\parskip1ex
\parindent0pt
\let\olditemize\itemize
\def\itemize{\olditemize\parskip0pt}

\begin{document}

\title{The \textsf{childdoc} Package}
\hypersetup{pdftitle={The childdoc Package}}
\author{Niklas Beisert\\[2ex]
  Institut f\"ur Theoretische Physik\\
  Eidgen\"ossische Technische Hochschule Z\"urich\\
  Wolfgang-Pauli-Strasse 27, 8093 Z\"urich, Switzerland\\[1ex]
  \href{mailto:nbeisert@itp.phys.ethz.ch}
  {\texttt{nbeisert@itp.phys.ethz.ch}}}
\hypersetup{pdfauthor={Niklas Beisert}}
\hypersetup{pdfsubject={Manual for the LaTeX2e Package childdoc}}
\date{30 December 2018, \textsf{v2.0}}
\maketitle

\begin{abstract}\noindent
\textsf{childdoc} is a \LaTeXe{} package
that enables the direct compilation
of document sections included by |\include|
to individual files.
\end{abstract}

\begingroup
\parskip0ex
\tableofcontents
\endgroup

%%%%%%%%%%%%%%%%%%%%%%%%%%%%%%%%%%%%%%%%%%%%%%%%%%%%%%%%%%%%%%%%%%%%%%%%%%%%%%%%
%%%%%%%%%%%%%%%%%%%%%%%%%%%%%%%%%%%%%%%%%%%%%%%%%%%%%%%%%%%%%%%%%%%%%%%%%%%%%%%%
\section{Introduction}

\LaTeX{} provides a mechanism to structure a large document (such as a book)
into a main file and several child files (containing the chapters)
using the |\include| command.
This mechanism is beneficial for documents
which span hundreds of pages in order to
make the source file(s) more manageable.
Moreover, compilation can be restricted to
selected child files by means of the |\includeonly| command.
The latter feature can be used to reduce the compilation time while editing
(this was significantly more useful in the earlier days of \LaTeX{})
or to generate a smaller document which is easier to navigate.
Another application of |\includeonly| is to generate
documents consisting of selected parts of the complete document.

However, there are a few drawbacks of the plain |\include| mechanism:
\begin{itemize}
\item
The child files cannot be compiled on their own,
they can only be compiled via the main file.
A naive editing environment
(such as a text editor with an option
to have the current file processed by \LaTeX)
may require one to switch to the main file before compiling;
attempting to compile the child file produces errors.
\item
The main file must be modified (each time)
to adjust the |\includeonly| command
to the present needs. This easily leaves the main file in a messy state.
\item
The generated document will always carry the filename
of the main document. This is inconvenient if
several child files are to be compiled and
to be kept for distribution.
\end{itemize}

The present package provides a simple interface
to make child files individually compilable by \LaTeX{}.
Compiling a child file then has the same effect as compiling
the main file with an |\includeonly| command
to select the appropriate child.
Moreover the generated document will carry the name of the child
rather than the main file.
This resolves all three above issues.

This feature is meant to make the editing of books,
thesis documents and lecture notes somewhat more convenient.
However, the package can also be used efficiently for
composing a series of documents (such as exercise sheets)
which are typically distributed individually.
It then assists the author in generating the individual documents
(potentially in different versions)
as well as a document containing the collected series.
Another application is in developing style files
or other kinds of included material
where compilation of the style file could redirect
to a sample or test file.

%%%%%%%%%%%%%%%%%%%%%%%%%%%%%%%%%%%%%%%%%%%%%%%%%%%%%%%%%%%%%%%%%%%%%%%%%%%%%%%%
%%%%%%%%%%%%%%%%%%%%%%%%%%%%%%%%%%%%%%%%%%%%%%%%%%%%%%%%%%%%%%%%%%%%%%%%%%%%%%%%
\section{Usage}

First of all, the package \textsf{childdoc} is \emph{not} a standard
\LaTeXe{} |.sty| style file! Therefore it needs to be invoked in
a non-standard way.

%%%%%%%%%%%%%%%%%%%%%%%%%%%%%%%%%%%%%%%%%%%%%%%%%%%%%%%%%%%%%%%%%%%%%%%%%%%%%%%%
\subsection{Included Files}
\label{sec:include}

%%%%%%%%%%%%%%%%%%%%%%%%%%%%%%%%%%%%%%%%
\DescribeMacro{\childdocmain}
To use the package, add the commands
\begin{center}
\begin{tabular}{l}
|\input{childdoc.def}|\\
|\childdocmain{}|\\
\end{tabular}
\end{center}
at the very top of the main \LaTeX{} file,
in particular \emph{before} the |\documentclass| statement!
The argument of |\childdocmain| should be left empty
(but it must be present).

%%%%%%%%%%%%%%%%%%%%%%%%%%%%%%%%%%%%%%%%
\DescribeMacro{\childdocof}
Furthermore, add the commands
\begin{center}
\begin{tabular}{l}
|\input{childdoc.def}|\\
|\childdocof{|\textit{main}|}|\\
\end{tabular}
\end{center}
at the top of every child file \textit{child}
which is included by |\include{|\textit{child}|}|
from within the main file
(or at least for those files to be compiled individually).
The argument \textit{main} must be the filename of the main file.

There are a couple of
considerations in setting up the main and child documents:

%%%%%%%%%%%%%%%%%%%%%%%%%%%%%%%%%%%%%%%%
\paragraph{Restrictions.}

Please note the following restrictions:
\begin{itemize}
\item
|\childdocmain| must be called with one argument \textit{main}
to ensure compatibility with earlier version of the package.
It must either be empty (|\childdocmain{}|)
or precisely match the filename of the main file in which it is specified.
See \secref{sec:detection} for further information.
\item
The filename \textit{main} must be specified without the |.tex| extension.
\item
The filename \textit{main} is case sensitive
(even in case-insensitive file systems)
due to internal string comparison.
\item
The argument \textit{main} should be fully expanded, it cannot be a macro.
\item
Subdirectories and special characters should be avoided in filenames.
\item
The command |\childdocmain{|\textit{main}|}| must be followed by a whitespace.
It should not be followed immediately by another command
or by a comment mark `|%|'.
This is because the \TeX{} parser reads the token immediately following
the argument of |\childdocmain| and puts it
at the beginning of every child section;
however, a white\-space is ignored.
\end{itemize}

%%%%%%%%%%%%%%%%%%%%%%%%%%%%%%%%%%%%%%%%
\paragraph{Content of Main File.}

It is advisable to place all content in the child files included by |\include|.
Any output contained in the main file will appear in all child documents
unless suppressed manually;
it cannot be suppressed automatically by the |\includeonly| directive
and thus should normally be avoided.
A method to include some content in the main file
by means of conditional processing is described in \secref{sec:conditional}.

%%%%%%%%%%%%%%%%%%%%%%%%%%%%%%%%%%%%%%%%
\paragraph{Page Numbering.}

When only a part of the document is compiled,
the appropriate numbering of pages
(as well as other status parameters)
is determined from the |.aux| files.
The latter contain information from previous passes.
However this information needs to propagate through
all intermediate child documents.
Therefore the page numbering in child documents may well
be inconsistent until the complete document is compiled at least once.

A useful (if unconventional) way to always ensure a consistent
page numbering is to restart the numbering in each child document
and denote the pages by `\textit{child}|.|\textit{page}'
where \textit{child} represents the chapter/section number of the child file.
This can be achieved by the command
|\numberwithin{page}{|\textit{child}|}|
of the \textsf{amsmath} package
where \textit{child} can be |chapter| or |section|
depending on the chosen structuring.
Alternatively, one can modify the macro |\thepage| appropriately
and reset the counter |page| at the start of each child file.

%%%%%%%%%%%%%%%%%%%%%%%%%%%%%%%%%%%%%%%%%%%%%%%%%%%%%%%%%%%%%%%%%%%%%%%%%%%%%%%%
\subsection{Conditional Processing}
\label{sec:conditional}

The package provides a mechanism to compile different versions
of a document. To customise the versions further some conditional processing
can come in handy to distinguish which version is being compiled.
The package provides two macros to describe the compilation context:

%%%%%%%%%%%%%%%%%%%%%%%%%%%%%%%%%%%%%%%%
\DescribeMacro{\ifchilddoc}
The conditional |\ifchilddoc| distinguishes between the compilation of
child documents and the main document:
%
\begin{center}
|\ifchilddoc |\textit{child-code}| |[|\||else |\textit{main-code}]| \||fi|
\end{center}

%%%%%%%%%%%%%%%%%%%%%%%%%%%%%%%%%%%%%%%%
\DescribeMacro{\childdocname}
\DescribeMacro{\childdocjob}
The macro |\childdocname| contains the filename (without extension)
of the main or child file being processed.
Note that |\childdocjob| will always contain the name of the main file.

%%%%%%%%%%%%%%%%%%%%%%%%%%%%%%%%%%%%%%%%
\paragraph{Title Page.}

Conditional processing can be used to include a title or banner page
in the main document when proper precautions are taken.
Importantly, the code in the main file should ensure that the page counter
(as well as other status parameters which are stored in the |.aux| files)
takes the same value after the conditional processing.
Otherwise the page numbers may take divergent values
depending on which part is compiled.

For example, a title page could be declared by:
%
\begin{center}
\begin{tabular}{l}
|\ifchilddoc\||else|\\
|\addtocounter{page}{-1}|\\
\textit{code for title page}\\
|\newpage|\\
|\||fi|
\end{tabular}
\end{center}
%
A banner page for the child documents can be generated by:
%
\begin{center}
\begin{tabular}{l}
|\ifchilddoc|\\
|\addtocounter{page}{-1}|\\
\textit{code for banner page}\\
|\newpage|\\
|\||fi|
\end{tabular}
\end{center}
%
Here one could write a message such as:
\begin{center}
|This is the part \childdocname{} of \childdocjob{}.|
\end{center}

%%%%%%%%%%%%%%%%%%%%%%%%%%%%%%%%%%%%%%%%%%%%%%%%%%%%%%%%%%%%%%%%%%%%%%%%%%%%%%%%
\subsection{Flags}
\label{sec:flags}

The package makes it easy to generate different versions
of the main or child documents.
To this end compilation flags can be defined
and assigned different default values.
They will be particularly useful in conjunction
with the forwarding mechanism described in \secref{sec:forward}.

For example, it may be useful to have a flag |\version|
which can be set to |draft| or |final|.
The document source will contain some conditional code
depending on the value of |\version|.
Suppose further, the flag should default to |final| for the main file
and to |draft| for child files
which is a natural assignment for editing the document.
This is achieved by placing the following code
in the preamble of the main document
(below the |\childdocmain| directive):
%
\begin{center}
\begin{tabular}{l}
|\ifchilddoc|\\
|\providecommand{\version}{draft}|\\
|\||else|\\
|\providecommand{\version}{final}|\\
|\||fi|
\end{tabular}
\end{center}
%
The definition by |\providecommand| makes sure
that previous definitions are not overwritten.
Further statements |\providecommand{\version}{...}|
can thus be added before the above code to override it.

For the main file, one might add a line
(between |\childdocmain| and the above block)
%
\begin{center}
|%\ifchilddoc\||else\providecommand{\version}{draft}\||fi|
\end{center}
%
which can be uncommented to produce a draft version.
Likewise one can add a line to the very top of a child file
(above the |\childdocof{|\textit{main}|}| directive)
%
\begin{center}
|%\providecommand{\version}{final}|
\end{center}
%
which can be uncommented to produce the final version of this child document.

%%%%%%%%%%%%%%%%%%%%%%%%%%%%%%%%%%%%%%%%%%%%%%%%%%%%%%%%%%%%%%%%%%%%%%%%%%%%%%%%
\subsection{Forwarding}
\label{sec:forward}

Different versions of the main or child documents
using compilation flags as described in \secref{sec:flags}
can be (permanently) stored in different files
for convenient compilation, viewing and distribution.
To this end, the package defines a command
to pass on compilation to a different file:

%%%%%%%%%%%%%%%%%%%%%%%%%%%%%%%%%%%%%%%%
\DescribeMacro{\childdocforward}
The command |\childdocforward| redirects processing to
another source file:
%
\begin{center}
\begin{tabular}{l}
|\input{childdoc.def}|\\
|\childdocforward[|\textit{main}|]{|\textit{dest}|}|\\
\end{tabular}
\end{center}
%
The argument \textit{dest} is the destination file
(without extension).
It should be the main file or one of the child files.
Note that further \textsf{childdoc} directives
such as |\childdocof| and |\childdocforward|
in the indicated file will be processed in this form.
The optional argument \textit{main}
passes on directly to the main file \textit{main}
while pretending to compile the child \textit{dest}.
This form behaves as if \textit{dest}
issues |\childdocof{|\textit{main}|}| right away,
and no further \textsf{childdoc} directives will be processed.

%%%%%%%%%%%%%%%%%%%%%%%%%%%%%%%%%%%%%%%%
\DescribeMacro{\...prefix}
In the alternative form |\childdocforwardprefix|,
%
\begin{center}
\begin{tabular}{l}
|\input{childdoc.def}|\\
|\childdocforwardprefix[|\textit{main}|]{|\textit{prefix}|}{|\textit{dest}|}|
\end{tabular}
\end{center}
%
the destination file is determined by a pattern
depending on the current file:
To make this work, the current file must be called
`{\textit{prefix}\hspace{0.2em}\textit{suffix}}'
with \textit{prefix} matching precisely the argument.
Processing is then passed on to the file
`{\textit{dest}\hspace{0.2em}\textit{suffix}}'.
Surely, the same effect is achieved by
directly specifying the
argument `{\textit{dest}\hspace{0.2em}\textit{suffix}}'
in the first form.
However, that requires to set up a different file
for each child. With the alternative form of the command
all these files can have exactly the same content
which simplifies setting them up and maintaining them.

For example, the following file |draft.tex|
with a compilation flag |\version| as described in \secref{sec:flags}
compiles the main document as a draft:
%
\begin{center}
\begin{tabular}{l}
|\def\version{draft}|\\
|\input{childdoc.def}|\\
|\childdocforward{|\textit{main}|}|
\end{tabular}
\end{center}
%
Likewise, the following files |final|\textit{nn}|.tex|
compile the final version of the child document
|child|\textit{nn}|.tex|:
%
\begin{center}
\begin{tabular}{l}
|\def\version{final}|\\
|\input{childdoc.def}|\\
|\childdocforwardprefix{final}{child}|
\end{tabular}
\end{center}
%

Note that when several versions of a main file and/or of each child file
are to be generated, it may be convenient to set up a |Makefile| or
shell script to automatise the process.

%%%%%%%%%%%%%%%%%%%%%%%%%%%%%%%%%%%%%%%%%%%%%%%%%%%%%%%%%%%%%%%%%%%%%%%%%%%%%%%%
\subsection{Command Line Processing}
\label{sec:commandline}

The effect of redirection files can also be achieved by invoking
the \LaTeX{} compiler with a more elaborate command line.
Most conveniently this should be done as part
of a shell script or a |Makefile|.

When using \textsf{childdoc} in the main file, the following
command lines effectively perform a redirection
(note that depending on the shell being used,
backslashes may have to be doubled: `|\|' $\to$ `|\\|'):
%
\begin{center}
|... -jobname "|\textit{target}|" |\\|"|[\textit{flags}]%
|\input{childdoc.def}\childdocforward[|\textit{main}|]{|\textit{dest}|}"|
\end{center}
%
Here \textit{target} is the name of the output file,
\textit{main} is the name of the main file
and \textit{dest} is the name of the main or child file to be processed
(all filenames without extensions).
The optional argument \textit{main} can be omitted
if \textit{main} matches \textit{dest}.
Optionally, compilation \textit{flags} can be defined via |\def| commands.
This command line makes the \TeX{} engine believe
it is compiling the file \textit{target}
whose content is specified as the latter parameter.
The provided code then forwards the processing to
\textit{main} or \textit{dest} as described in \secref{sec:forward}.

%%%%%%%%%%%%%%%%%%%%%%%%%%%%%%%%%%%%%%%%%%%%%%%%%%%%%%%%%%%%%%%%%%%%%%%%%%%%%%%%
\subsection{Include by Input}
\label{sec:input}

Including child documents by |\include| has some restrictions by design.
Most notably, the content of a child document always occupies
its own set of pages; pages cannot be shared between child documents.
Usually, this behaviour makes perfect sense
because each child document contain an essential part of the document.
However, in some situations it may be desirable to compose
a document from a collection of parts
without having mandatory page breaks between then.
For this case, the package
provides a mechanism to include parts
by |\input| which can also be processed individually.
However, by construction this mechanism
requires manual handling of the content to be output.

%%%%%%%%%%%%%%%%%%%%%%%%%%%%%%%%%%%%%%%%
\DescribeMacro{\ifchilddocmanual}
The main file should be prepared as usual, see \secref{sec:include}.
However, the document body must make a distinction
between processing of an individual part and of the main document, e.g.:
%
\begin{center}
\begin{tabular}{l}
|\ifchilddocmanual|\\
|\input{\childdocname}|\\
|\||else|\\
\textit{document body with }|\input{|\textit{part}|}|\\
|\||fi|
\end{tabular}
\end{center}
%
The conditional |\ifchilddocmanual| is true whenever
a part to be included by |\input| is being compiled,
and the name of the part is stored in |\childdocname|.

%%%%%%%%%%%%%%%%%%%%%%%%%%%%%%%%%%%%%%%%
\DescribeMacro{\childdocby}
Each part to be included by |\input| should start with:
%
\begin{center}
\begin{tabular}{l}
|\input{childdoc.def}|\\
|\childdocby{|\textit{main}|}|\\
\end{tabular}
\end{center}
%
The directive |\childdocby| is similar to |\childdocof|
described in \secref{sec:include},
but the subsequent selection of content must be done manually.
To that end, both |\ifchilddoc| and |\ifchilddocmanual|
will be true upon processing of a part,
and the name of the part is stored in |\childdocname|.
Note that |\jobname| will be set to the filename of the current part
so that each part receives an individual |.aux| file
that does not interfere with the |.aux| file(s) of the main document.
This behaviour can be altered by the alternative form
|\childdocby[*]{|\textit{main}|}| (with a non-empty optional argument)
which uses the |.aux| file of the main document
by setting |\jobname| to \textit{main}.

%%%%%%%%%%%%%%%%%%%%%%%%%%%%%%%%%%%%%%%%%%%%%%%%%%%%%%%%%%%%%%%%%%%%%%%%%%%%%%%%
\subsection{Driver Development}
\label{sec:driver}

The \textsf{childdoc} mechanism can also be use for the development
of definition files such as \LaTeX{} styles or classes.
This case differs from the above setup with multiple parts
included by |\include| in that no |\includeonly| should be invoked.
This can be achieved by starting the include file
(before |\ProvidesPackage|) with:
%
\begin{center}
\begin{tabular}{l}
|\input{childdoc.def}|\\
|\childdocforward{|\textit{main}|}|\\
\end{tabular}
\end{center}
%
or alternatively with:
%
\begin{center}
\begin{tabular}{l}
|\input{childdoc.def}|\\
|\childdocby{|\textit{main}|}|\\
\end{tabular}
\end{center}
%
Both forms have slightly different effects as described above.
The main file is prepared as usual, see \secref{sec:include}.

%%%%%%%%%%%%%%%%%%%%%%%%%%%%%%%%%%%%%%%%%%%%%%%%%%%%%%%%%%%%%%%%%%%%%%%%%%%%%%%%
\subsection{Legacy Detection}
\label{sec:detection}

The directive |\childdocmain| in the main file can detect
whether the complete document or merely a child is to be compiled
even without using the directive |\childdocof|.
This method is deprecated because it is less robust
and there is no compelling reason to use it;
it is merely provided for backward compatibility
and it may be removed in future versions.

If the detection mechanism is to be used,
it is mandatory to correctly specify
the filename of the main file as the argument of |\childdocmain|:
%
\begin{center}
\begin{tabular}{l}
|\input{childdoc.def}|\\
|\childdocmain{|\textit{main}|}|\\
\end{tabular}
\end{center}
%
If |\jobname| does not match the argument \textit{main} of |\childdocmain|,
it is assumed that |\jobname| points to the child file to be compiled.
When using |\childdocmain| with the main file specified as argument,
it suffices to start a child file
with just |\input{|\textit{main}|}|
without loading of the package and using |\childdocof|.
If instead all processing is done
with the appropriate \textsf{childdoc} directives,
the argument of \textit{main} of |\childdocmain| can be empty.

An alternative version of the command line processing described
in \secref{sec:commandline} using the detection mechanism reads:
%
\begin{center}
|... -jobname "|\textit{target}|" "|[\textit{flags}]%
[|\def\jobname{|\textit{dest}|}|]|\input{|\textit{main}|}"|
\end{center}

%%%%%%%%%%%%%%%%%%%%%%%%%%%%%%%%%%%%%%%%%%%%%%%%%%%%%%%%%%%%%%%%%%%%%%%%%%%%%%%%
\subsection{Manual Code}
\label{sec:manual}

In case one cannot be certain whether the definitions file |childdoc.def|
is installed on the target \TeX{} distribution
and one prefers not to ship it,
it is conceivable to paste a few relevant commands into the sources.

To that end, drop all statements |\input{childdoc.def}|
and perform the replacements as outlined below.
Instead of |\childdocmain{|\textit{main}|}| add the following code
to the top of the main file:
%
\begin{center}
\begin{tabular}{l}
|\||ifdefined\childdocname\endinput\||fi\newif\ifchilddoc|\\
|\edef\childdocname{\scantokens\expandafter{\jobname\noexpand}}|\\
|\def\childdocmain{|\textit{main}|}\||ifx\childdocmain\childdocname\||else|\\
|\childdoctrue\includeonly{\childdocname}\let\jobname\childdocmain\||fi|\\
\end{tabular}
\end{center}
%
Instead of |\childdocof{|\textit{main}|}| just include the main file
at the top of each child file:
%
\begin{center}
|\input{|\textit{main}|}|
\end{center}
%
A simple redirection |\childdocforward{|\textit{dest}|}| is achieved by:
%
\begin{center}
|\def\jobname{|\textit{dest}|}\input{\jobname}|
\end{center}
%
The redirection with prefix
|\childdocforwardprefix[|\textit{prefix}|]{|\textit{dest}|}|
is accomplished by:
%
\begin{center}
\begin{tabular}{l}
|{\edef\jobname{\scantokens\expandafter{\jobname\noexpand}}|\\
|\def\redirectjob |\textit{prefix}|#1~~~{\gdef\jobname{|\textit{dest}|#1}}|\\
|\expandafter\redirectjob\jobname~~~}\input{\jobname}|
\end{tabular}
\end{center}

In an alternative approach,
child documents can be compiled by a specific command line
without additional code or specific definitions:
%
\begin{center}
|... -jobname "|\textit{target}|" "|[\textit{flags}]%
|\includeonly{|\textit{dest}|}\input{|\textit{main}|}"|
\end{center}
%

%%%%%%%%%%%%%%%%%%%%%%%%%%%%%%%%%%%%%%%%%%%%%%%%%%%%%%%%%%%%%%%%%%%%%%%%%%%%%%%%
%%%%%%%%%%%%%%%%%%%%%%%%%%%%%%%%%%%%%%%%%%%%%%%%%%%%%%%%%%%%%%%%%%%%%%%%%%%%%%%%
\section{Information}

%%%%%%%%%%%%%%%%%%%%%%%%%%%%%%%%%%%%%%%%%%%%%%%%%%%%%%%%%%%%%%%%%%%%%%%%%%%%%%%%
\subsection{Copyright}

Copyright \copyright{} 2017--2018 Niklas Beisert

This work may be distributed and/or modified under the
conditions of the \LaTeX{} Project Public License, either version 1.3
of this license or (at your option) any later version.
The latest version of this license is in
  \url{http://www.latex-project.org/lppl.txt}
and version 1.3 or later is part of all distributions of \LaTeX{}
version 2005/12/01 or later.

This work has the LPPL maintenance status `maintained'.

The Current Maintainer of this work is Niklas Beisert.

This work consists of the files |README.txt|, |childdoc.ins| and |childdoc.dtx|
as well as the derived files |childdoc.def|, |cdocsamp.tex|
with |cdocsch1.tex|, |cdocsch2.tex|, |cdocspt3.tex|, |cdocspt4.tex|,
|cdocsdrf.tex|, |cdocsfn1.tex|, |cdocsfn2.tex|
as well as |childdoc.pdf|.

%%%%%%%%%%%%%%%%%%%%%%%%%%%%%%%%%%%%%%%%%%%%%%%%%%%%%%%%%%%%%%%%%%%%%%%%%%%%%%%%
\subsection{Files and Installation}

The package consists of the files:
%
\begin{center}
\begin{tabular}{ll}
    |README.txt|   & readme file \\
    |childdoc.ins| & installation file \\
    |childdoc.dtx| & source file \\
    |childdoc.def| & definition file \\
    |cdocsamp.tex| & sample main file \\
    |cdocsch1.tex| & sample include file \\
    |cdocsch2.tex| & sample include file \\
    |cdocspt3.tex| & sample part file \\
    |cdocspt4.tex| & sample part file \\
    |cdocsdrf.tex| & sample redirection file \\
    |cdocsfn1.tex| & sample redirection file \\
    |cdocsfn2.tex| & sample redirection file \\
    |childdoc.pdf| & manual
\end{tabular}
\end{center}
%
The distribution consists of the files
|README.txt|, |childdoc.ins| and |childdoc.dtx|.
%
\begin{itemize}
\item
Run (pdf)\LaTeX{} on |childdoc.dtx|
to compile the manual |childdoc.pdf| (this file).
\item
Run \LaTeX{} on |childdoc.ins| to create the definitions file |childdoc.def|
and the sample |cdocsamp.tex| with include files
|cdocsch1.tex|, |cdocsch2.tex|, |cdocspt3.tex|, |cdocspt4.tex|,
|cdocsdrf.tex|, |cdocsfn1.tex|, |cdocsfn2.tex|.
Then copy the file |childdoc.def| to an appropriate directory of your \LaTeX{}
distribution, e.g.\ \textit{texmf-root}|/tex/latex/childdoc|.
\end{itemize}

%%%%%%%%%%%%%%%%%%%%%%%%%%%%%%%%%%%%%%%%%%%%%%%%%%%%%%%%%%%%%%%%%%%%%%%%%%%%%%%%
\subsection{Related CTAN Packages}

There are several other packages which offer a similar functionality:
%
\begin{itemize}
\item
The packages
\href{http://ctan.org/pkg/docmute}{\textsf{docmute}},
\href{http://ctan.org/pkg/includex}{\textsf{includex}} and
\href{http://ctan.org/pkg/standalone}{\textsf{standalone}}
provide commands to include only the document body of
a child file thus allowing both files to be compiled individually.
\item
The packages \href{http://ctan.org/pkg/subdocs}{\textsf{subdocs}}
and \href{http://ctan.org/pkg/subfiles}{\textsf{subfiles}}
provide structures in which the main and child documents can be
encapsulated and allowing them to be compiled individually.
The inclusion mechanism is different from the conventional |\include|.
\item
The package \href{http://ctan.org/pkg/combine}{\textsf{combine}}
is an elaborate solution to combine several documents into one.
\end{itemize}
%
See also the CTAN topic \href{http://ctan.org/topic/subdocs}{\textsf{subdocs}}
for further related packages.
The present package differs from the above solutions in that
a document structure constructed with the conventional |\include| mechanism
just needs two extra commands at the top of every file
such that all constituent files can be compiled individually.

%%%%%%%%%%%%%%%%%%%%%%%%%%%%%%%%%%%%%%%%%%%%%%%%%%%%%%%%%%%%%%%%%%%%%%%%%%%%%%%%
%\subsection{Feature Suggestions}
%
%The following is a list of features which may be useful for future
%versions of this package:
%%
%\begin{itemize}
%\item
%\ldots
%\end{itemize}

%%%%%%%%%%%%%%%%%%%%%%%%%%%%%%%%%%%%%%%%%%%%%%%%%%%%%%%%%%%%%%%%%%%%%%%%%%%%%%%%
\subsection{Revision History}

%%%%%%%%%%%%%%%%%%%%%%%%%%%%%%%%%%%%%%%%
\paragraph{v2.0:} 2018/12/30

\begin{itemize}
\item
immediate forward processing
\item
added |\childdocby| mechanism
\item
manual restructured
\end{itemize}

%%%%%%%%%%%%%%%%%%%%%%%%%%%%%%%%%%%%%%%%
\paragraph{v1.6:} 2018/01/17

\begin{itemize}
\item
application for development of include files
\item
corrections to manual
\end{itemize}

%%%%%%%%%%%%%%%%%%%%%%%%%%%%%%%%%%%%%%%%
\paragraph{v1.5:} 2017/05/21

\begin{itemize}
\item
more complete structuring introduced
\item
|\childdocof| introduced
\item
|\childdoc| renamed to |\childdocmain|
\item
|\childredirect| renamed to |\childdocforward| and |\childdocforwardprefix|
and functionality expanded
\end{itemize}

%%%%%%%%%%%%%%%%%%%%%%%%%%%%%%%%%%%%%%%%
\paragraph{v1.0:} 2017/04/27

\begin{itemize}
\item
manual and install package
\item
first version published on CTAN
\end{itemize}

%%%%%%%%%%%%%%%%%%%%%%%%%%%%%%%%%%%%%%%%
\paragraph{v0.6:} 2017/04/26

\begin{itemize}
\item
redirection mechanism added
\end{itemize}

%%%%%%%%%%%%%%%%%%%%%%%%%%%%%%%%%%%%%%%%
\paragraph{v0.5:} 2017/04/26

\begin{itemize}
\item
functionality in definition file
\end{itemize}


%%%%%%%%%%%%%%%%%%%%%%%%%%%%%%%%%%%%%%%%%%%%%%%%%%%%%%%%%%%%%%%%%%%%%%%%%%%%%%%%
%%%%%%%%%%%%%%%%%%%%%%%%%%%%%%%%%%%%%%%%%%%%%%%%%%%%%%%%%%%%%%%%%%%%%%%%%%%%%%%%
%%%%%%%%%%%%%%%%%%%%%%%%%%%%%%%%%%%%%%%%%%%%%%%%%%%%%%%%%%%%%%%%%%%%%%%%%%%%%%%%
\appendix

\settowidth\MacroIndent{\rmfamily\scriptsize 000\ }

 \DocInput{childdoc.dtx}

\end{document}
%</driver>
% \fi
%
% %%%%%%%%%%%%%%%%%%%%%%%%%%%%%%%%%%%%%%%%%%%%%%%%%%%%%%%%%%%%%%%%%%%%%%%%%%%%%%
% %%%%%%%%%%%%%%%%%%%%%%%%%%%%%%%%%%%%%%%%%%%%%%%%%%%%%%%%%%%%%%%%%%%%%%%%%%%%%%
% \section{Sample}
%\iffalse
%<*samplemain>
%\fi
%
% The following presents a sample document
% with two chapters, two parts, a title page,
% a compile flag as well as three forwarding files to set the flag.
% It consists of eight |.tex| files:
% \begin{center}
% \begin{tabular}{ll}
% |cdocsamp.tex|&main file\\
% |cdocsch1.tex|&include file for chapter 1\\
% |cdocsch2.tex|&include file for chapter 2\\
% |cdocspt3.tex|&include file for part 3\\
% |cdocspt4.tex|&include file for part 4\\
% |cdocsdrf.tex|&forwarding file for main file in draft mode\\
% |cdocsfi1.tex|&forwarding file for final version of chapter 1\\
% |cdocsfi2.tex|&forwarding file for final version of chapter 2\\
% \end{tabular}
% \end{center}
% Each of the eight files can be compiled directly by the \LaTeX{} compiler.
%
% %%%%%%%%%%%%%%%%%%%%%%%%%%%%%%%%%%%%%%
% \paragraph{Main File.}
%
% The main file is called |cdocsamp.tex|.
%
% Load the \textsf{childdoc} definitions and
% declare the filename for the main document:
%    \begin{macrocode}
\input{childdoc.def}
\childdocmain{}
%    \end{macrocode}

% Optional override for |\version| flag:
%    \begin{macrocode}
%%\ifchilddoc\else\providecommand{\version}{draft}\fi
%    \end{macrocode}

% Define the default values for the |\version| flag
% (|final| for the main file and |draft| for childs):
%    \begin{macrocode}
\ifchilddoc
\providecommand{\version}{draft}
\else
\providecommand{\version}{final}
\fi
%    \end{macrocode}

% Load the standard document class:
%    \begin{macrocode}
\documentclass[12pt]{article}
%    \end{macrocode}

% Start the document body:
%    \begin{macrocode}
\begin{document}
%    \end{macrocode}

% Declare a title page.
% Print title, part of document being processed and version flag:
%    \begin{macrocode}
\addtocounter{page}{-1}
\begin{center}
{\LARGE\bfseries{}childdoc example\par}
\vspace{1cm}
\ifchilddoc
\ifchilddocmanual part\else chapter\fi:
`\childdocname' of `\childdocjob'\par
\else
main document: `\childdocjob'\par
\fi
version: \version\par
\end{center}
\newpage
%    \end{macrocode}

% Manually include selected file,
% otherwise process as usual:
%    \begin{macrocode}
\ifchilddocmanual
\section*{part `\childdocname'}
\input{\childdocname}
\else
%    \end{macrocode}

% Include the two chapters:
%    \begin{macrocode}
\include{cdocsch1}
\include{cdocsch2}
%    \end{macrocode}

% Include the two parts unless only chapters should be displayed:
%    \begin{macrocode}
\ifchilddoc\else
\section{part three}
\input{cdocspt3}
\section{part four}
\input{cdocspt4}
\fi
%    \end{macrocode}

% Process as usual until here:
%    \begin{macrocode}
\fi
%    \end{macrocode}

% End of document body:
%    \begin{macrocode}
\end{document}
%    \end{macrocode}
%\iffalse
%</samplemain>
%\fi
%
% %%%%%%%%%%%%%%%%%%%%%%%%%%%%%%%%%%%%%%
% \paragraph{Chapter Include Files.}
%
% The include files are called |cdocsch1.tex| and |cdocsch2.tex|.
%
%\iffalse
%<*samplechap1|samplechap2>
%\fi

% Optional override for |\version| flag:
%    \begin{macrocode}
%%\providecommand{\version}{final}
%    \end{macrocode}

% Include the main document:
%    \begin{macrocode}
\input{childdoc.def}
\childdocof{cdocsamp}
%    \end{macrocode}

%\iffalse
%</samplechap1|samplechap2>
%\fi
%
%\iffalse
%<*samplechap1>
%\fi
% Some text for chapter 1:
%    \begin{macrocode}
\section{one}
some text in chapter one
%    \end{macrocode}

%\iffalse
%</samplechap1>
%\fi
% Some text for chapter 2:
%\iffalse
%<*samplechap2>
%\fi
%    \begin{macrocode}
\section{two}
more text in chapter two
%    \end{macrocode}

%\iffalse
%</samplechap2>
%\fi
%
% %%%%%%%%%%%%%%%%%%%%%%%%%%%%%%%%%%%%%%
% \paragraph{Part Include Files.}
%
% The include files are called |cdocspt3.tex| and |cdocspt4.tex|.
%
%\iffalse
%<*samplepart3|samplepart4>
%\fi

% Optional override for |\version| flag:
%    \begin{macrocode}
%%\providecommand{\version}{final}
%    \end{macrocode}

% Include the main document:
%    \begin{macrocode}
\input{childdoc.def}
\childdocby{cdocsamp}
%    \end{macrocode}

%\iffalse
%</samplepart3|samplepart4>
%\fi
%
%\iffalse
%<*samplepart3>
%\fi
% Some text for part 3:
%    \begin{macrocode}
some text in part three
%    \end{macrocode}

%\iffalse
%</samplepart3>
%\fi
% Some text for part 4:
%\iffalse
%<*samplepart4>
%\fi
%    \begin{macrocode}
more text in part four
%    \end{macrocode}

%\iffalse
%</samplepart4>
%\fi
%
% %%%%%%%%%%%%%%%%%%%%%%%%%%%%%%%%%%%%%%
% \paragraph{Forwarding for a Complete Draft.}
%
% The following forwarding file |cdocsdrf.tex|
% compiles the main document in draft mode:
%\iffalse
%<*sampledraft>
%\fi
%    \begin{macrocode}
\def\version{draft}
\input{childdoc.def}
\childdocforward{cdocsamp}
%    \end{macrocode}

%\iffalse
%</sampledraft>
%\fi
%
% %%%%%%%%%%%%%%%%%%%%%%%%%%%%%%%%%%%%%%
% \paragraph{Forwarding for Final Version of the Chapters.}
%
% The following forwarding files |cdocsfn1.tex| and |cdocsfn2.tex|
% (with identical content)
% compile the final versions of the child documents
% |cdocsch1.tex| and |cdocsch2.tex|, respectively:
%\iffalse
%<*samplefinal>
%\fi
%    \begin{macrocode}
\def\version{final}
\input{childdoc.def}
\childdocforwardprefix[cdocsamp]{cdocsfn}{cdocsch}
%    \end{macrocode}

%\iffalse
%</samplefinal>
%\fi
%
% %%%%%%%%%%%%%%%%%%%%%%%%%%%%%%%%%%%%%%
% \paragraph{Command Line Processing.}
%
% The following three command lines generate the output files
% |cdocscld|, |cdocscl1| and |cdocscl2|
% which should be identical to
% |cdocsdrf|, |cdocsch1| and |cdocsfn2|, respectively:
% \begin{center}
% \begin{tabular}{l}
% |latex -jobname cdocscld \|\\
% |  "\def\version{draft}\input{childdoc.def}\childdocforward{cdocsamp}"|\\
% |latex -jobname cdocscl1 \|\\
% |  "\input{childdoc.def}\childdocforward[cdocsamp]{cdocsch1}"|\\
% |latex -jobname cdocscl2 \|\\
% |  "\def\version{final}\input{childdoc.def}\childdocforward{cdocsch2}"|
% \end{tabular}
% \end{center}
% Note that the trailing backslash on each first line
% merely continues the input to the second line
% (for convenient cut ant paste).
% Furthermore, the command |latex| can be replaced by any
% of its alternative versions such as |pdflatex|.
%
% %%%%%%%%%%%%%%%%%%%%%%%%%%%%%%%%%%%%%%%%%%%%%%%%%%%%%%%%%%%%%%%%%%%%%%%%%%%%%%
% %%%%%%%%%%%%%%%%%%%%%%%%%%%%%%%%%%%%%%%%%%%%%%%%%%%%%%%%%%%%%%%%%%%%%%%%%%%%%%
% \section{Implementation}
%\iffalse
%<*package>
%\fi
%
% This section describes the definitions file |childdoc.def|.

% The definitions cannot be loaded using |\usepackage| or |\RequirePackage|
% which has a mechanism to prevent loading a style file more than once.
% When loading the definitions by means of |\input|
% multiple instances have to be prevented manually:
%\iffalse
%This code needs to be before the `\ProvidesFile' directive
%which is defined at the beginning of this file.
%Therefore it is also placed there and commented out here.
%</package>
%<*discard>
%\fi
%    \begin{macrocode}
\ifdefined\childdocmain\endinput\fi
%    \end{macrocode}
%\iffalse
%</discard>
%<*package>
%\fi
%
% \macro{\ifchilddoc}
% \macro{\ifchilddocmanual}
% The conditional |\ifchilddoc| tells whether a
% child (true) or main (false) document is being compiled.
% The conditional |\ifchilddocmanual| tells whether
% the |\includeonly| mechanism is used (false) or
% the selection of child files must be performed manually (true).
% The definitions initialise to false:
%    \begin{macrocode}
\newif\ifchilddoc
\newif\ifchilddocmanual
%    \end{macrocode}

% \macro{\childdocname}
% \macro{\childdocjob}
% The macro |\childdocname| stores the name of the main document
% to be compiled. The macro |\childdocjob| stores the name of
% the document on which the \LaTeX{} compiler was originally invoked.
% The content of |\jobname| cannot be compared
% to filenames specified in the source due to different catcodes.
% The following code rescans |\jobname|, stores the result
% in |\childdocname| and saves a copy in |\childdocjob|:
%    \begin{macrocode}
\edef\childdocname{\scantokens\expandafter{\jobname\noexpand}}
\let\childdocjob\childdocname
%    \end{macrocode}

% \macro{\childdocdisable}
% The macro |\childdocdisable| prevents the main file
% from being processed more than once.
% At this stage, the main document command |\childdocmain|
% is assumed to be called once again where it should do nothing.
% Any subsequent call to it should prevent
% a secondary processing of the main document
% It overwrites the forwarding commands
% |\childdocof| and |\childdocforward|
% with empty macros to prevent further inclusions of the main document:
%    \begin{macrocode}
\newcommand{\childdocdisable}
{
  \renewcommand{\childdocmain}[1]{\renewcommand{\childdocmain}[1]{\endinput}}
  \renewcommand{\childdocof}[1]{}
  \renewcommand{\childdocby}[2][]{}
  \renewcommand{\childdocforward}[2][]{}
  \renewcommand{\childdocdisable}{}
}
%    \end{macrocode}

% \macro{\childdocmain}
% The macro |\childdocmain| is to be called at the top of the main file
% with nothing or the main filename (without extension) as argument.
% First, it breaks loops.
% If the argument is not empty and does not match |\childdocname|
% (which is set by the first inclusion of |childdoc.def|),
% |\ifchilddoc| is set to true, |\includeonly| is applied to the child file
% and |\jobname| is set to the main file
% (for proper handling of |.aux| files):
%    \begin{macrocode}
\newcommand{\childdocmain}[1]
{
  \childdocdisable\childdocmain{}
  \if?#1?\else
    \begingroup
      \def\childdoctmp{#1}
      \ifx\childdoctmp\childdocname
        \def\childdoctmp{}
      \else
        \def\childdoctmp
        {
          \childdoctrue
          \includeonly{\childdocname}
          \def\childdocjob{#1}
          \def\jobname{#1}
        }
      \fi
      \expandafter
    \endgroup
    \childdoctmp
  \fi
}
%    \end{macrocode}

% \macro{\childdocof}
% The command |\childdocof| redirects
% compilation to the main file |#1|.
%    \begin{macrocode}
\newcommand{\childdocof}[1]
{
  \childdocdisable
  \childdoctrue
  \includeonly{\childdocname}
  \def\jobname{#1}
  \def\childdocjob{#1}
  \input{#1}
}
%    \end{macrocode}

% \macro{\childdocby}
% The command |\childdocby| ....
%    \begin{macrocode}
\newcommand{\childdocby}[2][]
{
  \childdocdisable
  \childdoctrue
  \childdocmanualtrue
  \if?#1?\else
    \def\jobname{#2}
  \fi
  \def\childdocjob{#2}
  \input{#2}
  \endinput
}
%    \end{macrocode}

% \macro{\childdocforward}
% The command |\childdocforward| redirects
% compilation to the main file or
% (if the optional argument is given) a child file.
% Parameters are set as if the main file
% or a child file starting with |\childdocof| was compiled.
% Then compilation is handed over to the main file:
%    \begin{macrocode}
\newcommand{\childdocforward}[2][]
{
  \begingroup
    \if?#1?
      \def\childdoctmp
      {
        \def\childdocname{#2}
        \def\childdocjob{#2}
        \def\jobname{#2}
        \input{#2}
        \endinput
      }
    \else
      \def\childdoctmp
      {
        \childdocdisable
        \def\childdocname{#2}
        \childdoctrue
        \includeonly{#2}
        \def\childdocjob{#1}
        \def\jobname{#1}
        \input{#1}
        \endinput
      }
    \fi
    \expandafter
  \endgroup
  \childdoctmp
}
%    \end{macrocode}

% \macro{\childdocforwardprefix}
% The command |\childdocforwardprefix| redirects
% compilation to the main or a child file by means of a pattern.
% The prefix |#1| in the current filename is replaced by |#2|
% and the suffix of the current filename is kept
% (it is assumed that the filename does not contain the substring `|~~~|'
% which is used as a delimiter).
% Compilation is handed over to the new file by |\childdocforward|:
%    \begin{macrocode}
\newcommand{\childdocforwardprefix}[3][]
{
  \begingroup
    \def\childdocextract #2##1~~~{\def\childdoctmp{\childdocforward[#1]{#3##1}}}
    \expandafter\childdocextract\childdocname~~~
    \expandafter
  \endgroup
  \childdoctmp
}
%    \end{macrocode}

% \macro{\childdoc}
% The deprecated macro |\childdoc| is a legacy version of |\childdocmain|:
%    \begin{macrocode}
\newcommand{\childdoc}{\childdocmain}
%    \end{macrocode}

% \macro{\childdocredirect}
% The deprecated macro |\childdocredirect| is a legacy version
% of |\childdocforward| and |\childdocforwardprefix|:
%    \begin{macrocode}
\newcommand{\childdocredirect}[2][]
{
  \begingroup
    \if?#1?
      \def\childdoctmp{\childdocforward{#2}}
    \else
      \def\childdoctmp{\childdocforwardprefix{#1}{#2}}
    \fi
    \expandafter
  \endgroup
  \childdoctmp
}
%    \end{macrocode}

%\iffalse
%</package>
%\fi
%
\endinput
|\\
|\childdocforwardprefix{final}{child}|
\end{tabular}
\end{center}
%

Note that when several versions of a main file and/or of each child file
are to be generated, it may be convenient to set up a |Makefile| or
shell script to automatise the process.

%%%%%%%%%%%%%%%%%%%%%%%%%%%%%%%%%%%%%%%%%%%%%%%%%%%%%%%%%%%%%%%%%%%%%%%%%%%%%%%%
\subsection{Command Line Processing}
\label{sec:commandline}

The effect of redirection files can also be achieved by invoking
the \LaTeX{} compiler with a more elaborate command line.
Most conveniently this should be done as part
of a shell script or a |Makefile|.

When using \textsf{childdoc} in the main file, the following
command lines effectively perform a redirection
(note that depending on the shell being used,
backslashes may have to be doubled: `|\|' $\to$ `|\\|'):
%
\begin{center}
|... -jobname "|\textit{target}|" |\\|"|[\textit{flags}]%
|% \iffalse
%
% childdoc.dtx Copyright (C) 2017-2018 Niklas Beisert
%
% This work may be distributed and/or modified under the
% conditions of the LaTeX Project Public License, either version 1.3
% of this license or (at your option) any later version.
% The latest version of this license is in
%   http://www.latex-project.org/lppl.txt
% and version 1.3 or later is part of all distributions of LaTeX
% version 2005/12/01 or later.
%
% This work has the LPPL maintenance status `maintained'.
%
% The Current Maintainer of this work is Niklas Beisert.
%
% This work consists of the files childdoc.dtx and childdoc.ins
% and the derived files childdoc.def and cdocsamp.tex with
% cdocsch1.tex, cdocsch2.tex, cdocsdrf.tex, cdocsfn1.tex, cdocsfn2.tex.
%
%<package>\ifdefined\childdocmain\endinput\fi
%<package>\ProvidesFile{childdoc.def}[2018/12/30 v2.0 child document driver]
%<samplemain>\ProvidesFile{cdocsamp.tex}[2018/12/30 v2.0 sample for childdoc]
%<*driver>
%\ProvidesFile{childdoc.drv}[2018/12/30 v2.0 childdoc reference manual file]
\PassOptionsToClass{10pt,a4paper}{article}
\documentclass{ltxdoc}

\usepackage[margin=35mm]{geometry}
\usepackage{hyperref}
\usepackage{hyperxmp}
\usepackage[usenames]{color}

\hypersetup{colorlinks=true}
\hypersetup{pdfstartview=FitH}
\hypersetup{pdfpagemode=UseNone}
\hypersetup{pdfsource={}}
\hypersetup{pdflang={en-UK}}
\hypersetup{pdfcopyright={Copyright 2017-2018 Niklas Beisert.
  This work may be distributed and/or modified under the
  conditions of the LaTeX Project Public License, either version 1.3
  of this license or (at your option) any later version.}}
\hypersetup{pdflicenseurl={http://www.latex-project.org/lppl.txt}}
\hypersetup{pdfcontactaddress={ETH Zurich, ITP, HIT K,
  Wolfgang-Pauli-Strasse 27}}
\hypersetup{pdfcontactpostcode={8093}}
\hypersetup{pdfcontactcity={Zurich}}
\hypersetup{pdfcontactcountry={Switzerland}}
\hypersetup{pdfcontactemail={nbeisert@itp.phys.ethz.ch}}
\hypersetup{pdfcontacturl={http://people.phys.ethz.ch/\xmptilde nbeisert/}}

\newcommand{\secref}[1]{\hyperref[#1]{section \ref*{#1}}}

\parskip1ex
\parindent0pt
\let\olditemize\itemize
\def\itemize{\olditemize\parskip0pt}

\begin{document}

\title{The \textsf{childdoc} Package}
\hypersetup{pdftitle={The childdoc Package}}
\author{Niklas Beisert\\[2ex]
  Institut f\"ur Theoretische Physik\\
  Eidgen\"ossische Technische Hochschule Z\"urich\\
  Wolfgang-Pauli-Strasse 27, 8093 Z\"urich, Switzerland\\[1ex]
  \href{mailto:nbeisert@itp.phys.ethz.ch}
  {\texttt{nbeisert@itp.phys.ethz.ch}}}
\hypersetup{pdfauthor={Niklas Beisert}}
\hypersetup{pdfsubject={Manual for the LaTeX2e Package childdoc}}
\date{30 December 2018, \textsf{v2.0}}
\maketitle

\begin{abstract}\noindent
\textsf{childdoc} is a \LaTeXe{} package
that enables the direct compilation
of document sections included by |\include|
to individual files.
\end{abstract}

\begingroup
\parskip0ex
\tableofcontents
\endgroup

%%%%%%%%%%%%%%%%%%%%%%%%%%%%%%%%%%%%%%%%%%%%%%%%%%%%%%%%%%%%%%%%%%%%%%%%%%%%%%%%
%%%%%%%%%%%%%%%%%%%%%%%%%%%%%%%%%%%%%%%%%%%%%%%%%%%%%%%%%%%%%%%%%%%%%%%%%%%%%%%%
\section{Introduction}

\LaTeX{} provides a mechanism to structure a large document (such as a book)
into a main file and several child files (containing the chapters)
using the |\include| command.
This mechanism is beneficial for documents
which span hundreds of pages in order to
make the source file(s) more manageable.
Moreover, compilation can be restricted to
selected child files by means of the |\includeonly| command.
The latter feature can be used to reduce the compilation time while editing
(this was significantly more useful in the earlier days of \LaTeX{})
or to generate a smaller document which is easier to navigate.
Another application of |\includeonly| is to generate
documents consisting of selected parts of the complete document.

However, there are a few drawbacks of the plain |\include| mechanism:
\begin{itemize}
\item
The child files cannot be compiled on their own,
they can only be compiled via the main file.
A naive editing environment
(such as a text editor with an option
to have the current file processed by \LaTeX)
may require one to switch to the main file before compiling;
attempting to compile the child file produces errors.
\item
The main file must be modified (each time)
to adjust the |\includeonly| command
to the present needs. This easily leaves the main file in a messy state.
\item
The generated document will always carry the filename
of the main document. This is inconvenient if
several child files are to be compiled and
to be kept for distribution.
\end{itemize}

The present package provides a simple interface
to make child files individually compilable by \LaTeX{}.
Compiling a child file then has the same effect as compiling
the main file with an |\includeonly| command
to select the appropriate child.
Moreover the generated document will carry the name of the child
rather than the main file.
This resolves all three above issues.

This feature is meant to make the editing of books,
thesis documents and lecture notes somewhat more convenient.
However, the package can also be used efficiently for
composing a series of documents (such as exercise sheets)
which are typically distributed individually.
It then assists the author in generating the individual documents
(potentially in different versions)
as well as a document containing the collected series.
Another application is in developing style files
or other kinds of included material
where compilation of the style file could redirect
to a sample or test file.

%%%%%%%%%%%%%%%%%%%%%%%%%%%%%%%%%%%%%%%%%%%%%%%%%%%%%%%%%%%%%%%%%%%%%%%%%%%%%%%%
%%%%%%%%%%%%%%%%%%%%%%%%%%%%%%%%%%%%%%%%%%%%%%%%%%%%%%%%%%%%%%%%%%%%%%%%%%%%%%%%
\section{Usage}

First of all, the package \textsf{childdoc} is \emph{not} a standard
\LaTeXe{} |.sty| style file! Therefore it needs to be invoked in
a non-standard way.

%%%%%%%%%%%%%%%%%%%%%%%%%%%%%%%%%%%%%%%%%%%%%%%%%%%%%%%%%%%%%%%%%%%%%%%%%%%%%%%%
\subsection{Included Files}
\label{sec:include}

%%%%%%%%%%%%%%%%%%%%%%%%%%%%%%%%%%%%%%%%
\DescribeMacro{\childdocmain}
To use the package, add the commands
\begin{center}
\begin{tabular}{l}
|\input{childdoc.def}|\\
|\childdocmain{}|\\
\end{tabular}
\end{center}
at the very top of the main \LaTeX{} file,
in particular \emph{before} the |\documentclass| statement!
The argument of |\childdocmain| should be left empty
(but it must be present).

%%%%%%%%%%%%%%%%%%%%%%%%%%%%%%%%%%%%%%%%
\DescribeMacro{\childdocof}
Furthermore, add the commands
\begin{center}
\begin{tabular}{l}
|\input{childdoc.def}|\\
|\childdocof{|\textit{main}|}|\\
\end{tabular}
\end{center}
at the top of every child file \textit{child}
which is included by |\include{|\textit{child}|}|
from within the main file
(or at least for those files to be compiled individually).
The argument \textit{main} must be the filename of the main file.

There are a couple of
considerations in setting up the main and child documents:

%%%%%%%%%%%%%%%%%%%%%%%%%%%%%%%%%%%%%%%%
\paragraph{Restrictions.}

Please note the following restrictions:
\begin{itemize}
\item
|\childdocmain| must be called with one argument \textit{main}
to ensure compatibility with earlier version of the package.
It must either be empty (|\childdocmain{}|)
or precisely match the filename of the main file in which it is specified.
See \secref{sec:detection} for further information.
\item
The filename \textit{main} must be specified without the |.tex| extension.
\item
The filename \textit{main} is case sensitive
(even in case-insensitive file systems)
due to internal string comparison.
\item
The argument \textit{main} should be fully expanded, it cannot be a macro.
\item
Subdirectories and special characters should be avoided in filenames.
\item
The command |\childdocmain{|\textit{main}|}| must be followed by a whitespace.
It should not be followed immediately by another command
or by a comment mark `|%|'.
This is because the \TeX{} parser reads the token immediately following
the argument of |\childdocmain| and puts it
at the beginning of every child section;
however, a white\-space is ignored.
\end{itemize}

%%%%%%%%%%%%%%%%%%%%%%%%%%%%%%%%%%%%%%%%
\paragraph{Content of Main File.}

It is advisable to place all content in the child files included by |\include|.
Any output contained in the main file will appear in all child documents
unless suppressed manually;
it cannot be suppressed automatically by the |\includeonly| directive
and thus should normally be avoided.
A method to include some content in the main file
by means of conditional processing is described in \secref{sec:conditional}.

%%%%%%%%%%%%%%%%%%%%%%%%%%%%%%%%%%%%%%%%
\paragraph{Page Numbering.}

When only a part of the document is compiled,
the appropriate numbering of pages
(as well as other status parameters)
is determined from the |.aux| files.
The latter contain information from previous passes.
However this information needs to propagate through
all intermediate child documents.
Therefore the page numbering in child documents may well
be inconsistent until the complete document is compiled at least once.

A useful (if unconventional) way to always ensure a consistent
page numbering is to restart the numbering in each child document
and denote the pages by `\textit{child}|.|\textit{page}'
where \textit{child} represents the chapter/section number of the child file.
This can be achieved by the command
|\numberwithin{page}{|\textit{child}|}|
of the \textsf{amsmath} package
where \textit{child} can be |chapter| or |section|
depending on the chosen structuring.
Alternatively, one can modify the macro |\thepage| appropriately
and reset the counter |page| at the start of each child file.

%%%%%%%%%%%%%%%%%%%%%%%%%%%%%%%%%%%%%%%%%%%%%%%%%%%%%%%%%%%%%%%%%%%%%%%%%%%%%%%%
\subsection{Conditional Processing}
\label{sec:conditional}

The package provides a mechanism to compile different versions
of a document. To customise the versions further some conditional processing
can come in handy to distinguish which version is being compiled.
The package provides two macros to describe the compilation context:

%%%%%%%%%%%%%%%%%%%%%%%%%%%%%%%%%%%%%%%%
\DescribeMacro{\ifchilddoc}
The conditional |\ifchilddoc| distinguishes between the compilation of
child documents and the main document:
%
\begin{center}
|\ifchilddoc |\textit{child-code}| |[|\||else |\textit{main-code}]| \||fi|
\end{center}

%%%%%%%%%%%%%%%%%%%%%%%%%%%%%%%%%%%%%%%%
\DescribeMacro{\childdocname}
\DescribeMacro{\childdocjob}
The macro |\childdocname| contains the filename (without extension)
of the main or child file being processed.
Note that |\childdocjob| will always contain the name of the main file.

%%%%%%%%%%%%%%%%%%%%%%%%%%%%%%%%%%%%%%%%
\paragraph{Title Page.}

Conditional processing can be used to include a title or banner page
in the main document when proper precautions are taken.
Importantly, the code in the main file should ensure that the page counter
(as well as other status parameters which are stored in the |.aux| files)
takes the same value after the conditional processing.
Otherwise the page numbers may take divergent values
depending on which part is compiled.

For example, a title page could be declared by:
%
\begin{center}
\begin{tabular}{l}
|\ifchilddoc\||else|\\
|\addtocounter{page}{-1}|\\
\textit{code for title page}\\
|\newpage|\\
|\||fi|
\end{tabular}
\end{center}
%
A banner page for the child documents can be generated by:
%
\begin{center}
\begin{tabular}{l}
|\ifchilddoc|\\
|\addtocounter{page}{-1}|\\
\textit{code for banner page}\\
|\newpage|\\
|\||fi|
\end{tabular}
\end{center}
%
Here one could write a message such as:
\begin{center}
|This is the part \childdocname{} of \childdocjob{}.|
\end{center}

%%%%%%%%%%%%%%%%%%%%%%%%%%%%%%%%%%%%%%%%%%%%%%%%%%%%%%%%%%%%%%%%%%%%%%%%%%%%%%%%
\subsection{Flags}
\label{sec:flags}

The package makes it easy to generate different versions
of the main or child documents.
To this end compilation flags can be defined
and assigned different default values.
They will be particularly useful in conjunction
with the forwarding mechanism described in \secref{sec:forward}.

For example, it may be useful to have a flag |\version|
which can be set to |draft| or |final|.
The document source will contain some conditional code
depending on the value of |\version|.
Suppose further, the flag should default to |final| for the main file
and to |draft| for child files
which is a natural assignment for editing the document.
This is achieved by placing the following code
in the preamble of the main document
(below the |\childdocmain| directive):
%
\begin{center}
\begin{tabular}{l}
|\ifchilddoc|\\
|\providecommand{\version}{draft}|\\
|\||else|\\
|\providecommand{\version}{final}|\\
|\||fi|
\end{tabular}
\end{center}
%
The definition by |\providecommand| makes sure
that previous definitions are not overwritten.
Further statements |\providecommand{\version}{...}|
can thus be added before the above code to override it.

For the main file, one might add a line
(between |\childdocmain| and the above block)
%
\begin{center}
|%\ifchilddoc\||else\providecommand{\version}{draft}\||fi|
\end{center}
%
which can be uncommented to produce a draft version.
Likewise one can add a line to the very top of a child file
(above the |\childdocof{|\textit{main}|}| directive)
%
\begin{center}
|%\providecommand{\version}{final}|
\end{center}
%
which can be uncommented to produce the final version of this child document.

%%%%%%%%%%%%%%%%%%%%%%%%%%%%%%%%%%%%%%%%%%%%%%%%%%%%%%%%%%%%%%%%%%%%%%%%%%%%%%%%
\subsection{Forwarding}
\label{sec:forward}

Different versions of the main or child documents
using compilation flags as described in \secref{sec:flags}
can be (permanently) stored in different files
for convenient compilation, viewing and distribution.
To this end, the package defines a command
to pass on compilation to a different file:

%%%%%%%%%%%%%%%%%%%%%%%%%%%%%%%%%%%%%%%%
\DescribeMacro{\childdocforward}
The command |\childdocforward| redirects processing to
another source file:
%
\begin{center}
\begin{tabular}{l}
|\input{childdoc.def}|\\
|\childdocforward[|\textit{main}|]{|\textit{dest}|}|\\
\end{tabular}
\end{center}
%
The argument \textit{dest} is the destination file
(without extension).
It should be the main file or one of the child files.
Note that further \textsf{childdoc} directives
such as |\childdocof| and |\childdocforward|
in the indicated file will be processed in this form.
The optional argument \textit{main}
passes on directly to the main file \textit{main}
while pretending to compile the child \textit{dest}.
This form behaves as if \textit{dest}
issues |\childdocof{|\textit{main}|}| right away,
and no further \textsf{childdoc} directives will be processed.

%%%%%%%%%%%%%%%%%%%%%%%%%%%%%%%%%%%%%%%%
\DescribeMacro{\...prefix}
In the alternative form |\childdocforwardprefix|,
%
\begin{center}
\begin{tabular}{l}
|\input{childdoc.def}|\\
|\childdocforwardprefix[|\textit{main}|]{|\textit{prefix}|}{|\textit{dest}|}|
\end{tabular}
\end{center}
%
the destination file is determined by a pattern
depending on the current file:
To make this work, the current file must be called
`{\textit{prefix}\hspace{0.2em}\textit{suffix}}'
with \textit{prefix} matching precisely the argument.
Processing is then passed on to the file
`{\textit{dest}\hspace{0.2em}\textit{suffix}}'.
Surely, the same effect is achieved by
directly specifying the
argument `{\textit{dest}\hspace{0.2em}\textit{suffix}}'
in the first form.
However, that requires to set up a different file
for each child. With the alternative form of the command
all these files can have exactly the same content
which simplifies setting them up and maintaining them.

For example, the following file |draft.tex|
with a compilation flag |\version| as described in \secref{sec:flags}
compiles the main document as a draft:
%
\begin{center}
\begin{tabular}{l}
|\def\version{draft}|\\
|\input{childdoc.def}|\\
|\childdocforward{|\textit{main}|}|
\end{tabular}
\end{center}
%
Likewise, the following files |final|\textit{nn}|.tex|
compile the final version of the child document
|child|\textit{nn}|.tex|:
%
\begin{center}
\begin{tabular}{l}
|\def\version{final}|\\
|\input{childdoc.def}|\\
|\childdocforwardprefix{final}{child}|
\end{tabular}
\end{center}
%

Note that when several versions of a main file and/or of each child file
are to be generated, it may be convenient to set up a |Makefile| or
shell script to automatise the process.

%%%%%%%%%%%%%%%%%%%%%%%%%%%%%%%%%%%%%%%%%%%%%%%%%%%%%%%%%%%%%%%%%%%%%%%%%%%%%%%%
\subsection{Command Line Processing}
\label{sec:commandline}

The effect of redirection files can also be achieved by invoking
the \LaTeX{} compiler with a more elaborate command line.
Most conveniently this should be done as part
of a shell script or a |Makefile|.

When using \textsf{childdoc} in the main file, the following
command lines effectively perform a redirection
(note that depending on the shell being used,
backslashes may have to be doubled: `|\|' $\to$ `|\\|'):
%
\begin{center}
|... -jobname "|\textit{target}|" |\\|"|[\textit{flags}]%
|\input{childdoc.def}\childdocforward[|\textit{main}|]{|\textit{dest}|}"|
\end{center}
%
Here \textit{target} is the name of the output file,
\textit{main} is the name of the main file
and \textit{dest} is the name of the main or child file to be processed
(all filenames without extensions).
The optional argument \textit{main} can be omitted
if \textit{main} matches \textit{dest}.
Optionally, compilation \textit{flags} can be defined via |\def| commands.
This command line makes the \TeX{} engine believe
it is compiling the file \textit{target}
whose content is specified as the latter parameter.
The provided code then forwards the processing to
\textit{main} or \textit{dest} as described in \secref{sec:forward}.

%%%%%%%%%%%%%%%%%%%%%%%%%%%%%%%%%%%%%%%%%%%%%%%%%%%%%%%%%%%%%%%%%%%%%%%%%%%%%%%%
\subsection{Include by Input}
\label{sec:input}

Including child documents by |\include| has some restrictions by design.
Most notably, the content of a child document always occupies
its own set of pages; pages cannot be shared between child documents.
Usually, this behaviour makes perfect sense
because each child document contain an essential part of the document.
However, in some situations it may be desirable to compose
a document from a collection of parts
without having mandatory page breaks between then.
For this case, the package
provides a mechanism to include parts
by |\input| which can also be processed individually.
However, by construction this mechanism
requires manual handling of the content to be output.

%%%%%%%%%%%%%%%%%%%%%%%%%%%%%%%%%%%%%%%%
\DescribeMacro{\ifchilddocmanual}
The main file should be prepared as usual, see \secref{sec:include}.
However, the document body must make a distinction
between processing of an individual part and of the main document, e.g.:
%
\begin{center}
\begin{tabular}{l}
|\ifchilddocmanual|\\
|\input{\childdocname}|\\
|\||else|\\
\textit{document body with }|\input{|\textit{part}|}|\\
|\||fi|
\end{tabular}
\end{center}
%
The conditional |\ifchilddocmanual| is true whenever
a part to be included by |\input| is being compiled,
and the name of the part is stored in |\childdocname|.

%%%%%%%%%%%%%%%%%%%%%%%%%%%%%%%%%%%%%%%%
\DescribeMacro{\childdocby}
Each part to be included by |\input| should start with:
%
\begin{center}
\begin{tabular}{l}
|\input{childdoc.def}|\\
|\childdocby{|\textit{main}|}|\\
\end{tabular}
\end{center}
%
The directive |\childdocby| is similar to |\childdocof|
described in \secref{sec:include},
but the subsequent selection of content must be done manually.
To that end, both |\ifchilddoc| and |\ifchilddocmanual|
will be true upon processing of a part,
and the name of the part is stored in |\childdocname|.
Note that |\jobname| will be set to the filename of the current part
so that each part receives an individual |.aux| file
that does not interfere with the |.aux| file(s) of the main document.
This behaviour can be altered by the alternative form
|\childdocby[*]{|\textit{main}|}| (with a non-empty optional argument)
which uses the |.aux| file of the main document
by setting |\jobname| to \textit{main}.

%%%%%%%%%%%%%%%%%%%%%%%%%%%%%%%%%%%%%%%%%%%%%%%%%%%%%%%%%%%%%%%%%%%%%%%%%%%%%%%%
\subsection{Driver Development}
\label{sec:driver}

The \textsf{childdoc} mechanism can also be use for the development
of definition files such as \LaTeX{} styles or classes.
This case differs from the above setup with multiple parts
included by |\include| in that no |\includeonly| should be invoked.
This can be achieved by starting the include file
(before |\ProvidesPackage|) with:
%
\begin{center}
\begin{tabular}{l}
|\input{childdoc.def}|\\
|\childdocforward{|\textit{main}|}|\\
\end{tabular}
\end{center}
%
or alternatively with:
%
\begin{center}
\begin{tabular}{l}
|\input{childdoc.def}|\\
|\childdocby{|\textit{main}|}|\\
\end{tabular}
\end{center}
%
Both forms have slightly different effects as described above.
The main file is prepared as usual, see \secref{sec:include}.

%%%%%%%%%%%%%%%%%%%%%%%%%%%%%%%%%%%%%%%%%%%%%%%%%%%%%%%%%%%%%%%%%%%%%%%%%%%%%%%%
\subsection{Legacy Detection}
\label{sec:detection}

The directive |\childdocmain| in the main file can detect
whether the complete document or merely a child is to be compiled
even without using the directive |\childdocof|.
This method is deprecated because it is less robust
and there is no compelling reason to use it;
it is merely provided for backward compatibility
and it may be removed in future versions.

If the detection mechanism is to be used,
it is mandatory to correctly specify
the filename of the main file as the argument of |\childdocmain|:
%
\begin{center}
\begin{tabular}{l}
|\input{childdoc.def}|\\
|\childdocmain{|\textit{main}|}|\\
\end{tabular}
\end{center}
%
If |\jobname| does not match the argument \textit{main} of |\childdocmain|,
it is assumed that |\jobname| points to the child file to be compiled.
When using |\childdocmain| with the main file specified as argument,
it suffices to start a child file
with just |\input{|\textit{main}|}|
without loading of the package and using |\childdocof|.
If instead all processing is done
with the appropriate \textsf{childdoc} directives,
the argument of \textit{main} of |\childdocmain| can be empty.

An alternative version of the command line processing described
in \secref{sec:commandline} using the detection mechanism reads:
%
\begin{center}
|... -jobname "|\textit{target}|" "|[\textit{flags}]%
[|\def\jobname{|\textit{dest}|}|]|\input{|\textit{main}|}"|
\end{center}

%%%%%%%%%%%%%%%%%%%%%%%%%%%%%%%%%%%%%%%%%%%%%%%%%%%%%%%%%%%%%%%%%%%%%%%%%%%%%%%%
\subsection{Manual Code}
\label{sec:manual}

In case one cannot be certain whether the definitions file |childdoc.def|
is installed on the target \TeX{} distribution
and one prefers not to ship it,
it is conceivable to paste a few relevant commands into the sources.

To that end, drop all statements |\input{childdoc.def}|
and perform the replacements as outlined below.
Instead of |\childdocmain{|\textit{main}|}| add the following code
to the top of the main file:
%
\begin{center}
\begin{tabular}{l}
|\||ifdefined\childdocname\endinput\||fi\newif\ifchilddoc|\\
|\edef\childdocname{\scantokens\expandafter{\jobname\noexpand}}|\\
|\def\childdocmain{|\textit{main}|}\||ifx\childdocmain\childdocname\||else|\\
|\childdoctrue\includeonly{\childdocname}\let\jobname\childdocmain\||fi|\\
\end{tabular}
\end{center}
%
Instead of |\childdocof{|\textit{main}|}| just include the main file
at the top of each child file:
%
\begin{center}
|\input{|\textit{main}|}|
\end{center}
%
A simple redirection |\childdocforward{|\textit{dest}|}| is achieved by:
%
\begin{center}
|\def\jobname{|\textit{dest}|}\input{\jobname}|
\end{center}
%
The redirection with prefix
|\childdocforwardprefix[|\textit{prefix}|]{|\textit{dest}|}|
is accomplished by:
%
\begin{center}
\begin{tabular}{l}
|{\edef\jobname{\scantokens\expandafter{\jobname\noexpand}}|\\
|\def\redirectjob |\textit{prefix}|#1~~~{\gdef\jobname{|\textit{dest}|#1}}|\\
|\expandafter\redirectjob\jobname~~~}\input{\jobname}|
\end{tabular}
\end{center}

In an alternative approach,
child documents can be compiled by a specific command line
without additional code or specific definitions:
%
\begin{center}
|... -jobname "|\textit{target}|" "|[\textit{flags}]%
|\includeonly{|\textit{dest}|}\input{|\textit{main}|}"|
\end{center}
%

%%%%%%%%%%%%%%%%%%%%%%%%%%%%%%%%%%%%%%%%%%%%%%%%%%%%%%%%%%%%%%%%%%%%%%%%%%%%%%%%
%%%%%%%%%%%%%%%%%%%%%%%%%%%%%%%%%%%%%%%%%%%%%%%%%%%%%%%%%%%%%%%%%%%%%%%%%%%%%%%%
\section{Information}

%%%%%%%%%%%%%%%%%%%%%%%%%%%%%%%%%%%%%%%%%%%%%%%%%%%%%%%%%%%%%%%%%%%%%%%%%%%%%%%%
\subsection{Copyright}

Copyright \copyright{} 2017--2018 Niklas Beisert

This work may be distributed and/or modified under the
conditions of the \LaTeX{} Project Public License, either version 1.3
of this license or (at your option) any later version.
The latest version of this license is in
  \url{http://www.latex-project.org/lppl.txt}
and version 1.3 or later is part of all distributions of \LaTeX{}
version 2005/12/01 or later.

This work has the LPPL maintenance status `maintained'.

The Current Maintainer of this work is Niklas Beisert.

This work consists of the files |README.txt|, |childdoc.ins| and |childdoc.dtx|
as well as the derived files |childdoc.def|, |cdocsamp.tex|
with |cdocsch1.tex|, |cdocsch2.tex|, |cdocspt3.tex|, |cdocspt4.tex|,
|cdocsdrf.tex|, |cdocsfn1.tex|, |cdocsfn2.tex|
as well as |childdoc.pdf|.

%%%%%%%%%%%%%%%%%%%%%%%%%%%%%%%%%%%%%%%%%%%%%%%%%%%%%%%%%%%%%%%%%%%%%%%%%%%%%%%%
\subsection{Files and Installation}

The package consists of the files:
%
\begin{center}
\begin{tabular}{ll}
    |README.txt|   & readme file \\
    |childdoc.ins| & installation file \\
    |childdoc.dtx| & source file \\
    |childdoc.def| & definition file \\
    |cdocsamp.tex| & sample main file \\
    |cdocsch1.tex| & sample include file \\
    |cdocsch2.tex| & sample include file \\
    |cdocspt3.tex| & sample part file \\
    |cdocspt4.tex| & sample part file \\
    |cdocsdrf.tex| & sample redirection file \\
    |cdocsfn1.tex| & sample redirection file \\
    |cdocsfn2.tex| & sample redirection file \\
    |childdoc.pdf| & manual
\end{tabular}
\end{center}
%
The distribution consists of the files
|README.txt|, |childdoc.ins| and |childdoc.dtx|.
%
\begin{itemize}
\item
Run (pdf)\LaTeX{} on |childdoc.dtx|
to compile the manual |childdoc.pdf| (this file).
\item
Run \LaTeX{} on |childdoc.ins| to create the definitions file |childdoc.def|
and the sample |cdocsamp.tex| with include files
|cdocsch1.tex|, |cdocsch2.tex|, |cdocspt3.tex|, |cdocspt4.tex|,
|cdocsdrf.tex|, |cdocsfn1.tex|, |cdocsfn2.tex|.
Then copy the file |childdoc.def| to an appropriate directory of your \LaTeX{}
distribution, e.g.\ \textit{texmf-root}|/tex/latex/childdoc|.
\end{itemize}

%%%%%%%%%%%%%%%%%%%%%%%%%%%%%%%%%%%%%%%%%%%%%%%%%%%%%%%%%%%%%%%%%%%%%%%%%%%%%%%%
\subsection{Related CTAN Packages}

There are several other packages which offer a similar functionality:
%
\begin{itemize}
\item
The packages
\href{http://ctan.org/pkg/docmute}{\textsf{docmute}},
\href{http://ctan.org/pkg/includex}{\textsf{includex}} and
\href{http://ctan.org/pkg/standalone}{\textsf{standalone}}
provide commands to include only the document body of
a child file thus allowing both files to be compiled individually.
\item
The packages \href{http://ctan.org/pkg/subdocs}{\textsf{subdocs}}
and \href{http://ctan.org/pkg/subfiles}{\textsf{subfiles}}
provide structures in which the main and child documents can be
encapsulated and allowing them to be compiled individually.
The inclusion mechanism is different from the conventional |\include|.
\item
The package \href{http://ctan.org/pkg/combine}{\textsf{combine}}
is an elaborate solution to combine several documents into one.
\end{itemize}
%
See also the CTAN topic \href{http://ctan.org/topic/subdocs}{\textsf{subdocs}}
for further related packages.
The present package differs from the above solutions in that
a document structure constructed with the conventional |\include| mechanism
just needs two extra commands at the top of every file
such that all constituent files can be compiled individually.

%%%%%%%%%%%%%%%%%%%%%%%%%%%%%%%%%%%%%%%%%%%%%%%%%%%%%%%%%%%%%%%%%%%%%%%%%%%%%%%%
%\subsection{Feature Suggestions}
%
%The following is a list of features which may be useful for future
%versions of this package:
%%
%\begin{itemize}
%\item
%\ldots
%\end{itemize}

%%%%%%%%%%%%%%%%%%%%%%%%%%%%%%%%%%%%%%%%%%%%%%%%%%%%%%%%%%%%%%%%%%%%%%%%%%%%%%%%
\subsection{Revision History}

%%%%%%%%%%%%%%%%%%%%%%%%%%%%%%%%%%%%%%%%
\paragraph{v2.0:} 2018/12/30

\begin{itemize}
\item
immediate forward processing
\item
added |\childdocby| mechanism
\item
manual restructured
\end{itemize}

%%%%%%%%%%%%%%%%%%%%%%%%%%%%%%%%%%%%%%%%
\paragraph{v1.6:} 2018/01/17

\begin{itemize}
\item
application for development of include files
\item
corrections to manual
\end{itemize}

%%%%%%%%%%%%%%%%%%%%%%%%%%%%%%%%%%%%%%%%
\paragraph{v1.5:} 2017/05/21

\begin{itemize}
\item
more complete structuring introduced
\item
|\childdocof| introduced
\item
|\childdoc| renamed to |\childdocmain|
\item
|\childredirect| renamed to |\childdocforward| and |\childdocforwardprefix|
and functionality expanded
\end{itemize}

%%%%%%%%%%%%%%%%%%%%%%%%%%%%%%%%%%%%%%%%
\paragraph{v1.0:} 2017/04/27

\begin{itemize}
\item
manual and install package
\item
first version published on CTAN
\end{itemize}

%%%%%%%%%%%%%%%%%%%%%%%%%%%%%%%%%%%%%%%%
\paragraph{v0.6:} 2017/04/26

\begin{itemize}
\item
redirection mechanism added
\end{itemize}

%%%%%%%%%%%%%%%%%%%%%%%%%%%%%%%%%%%%%%%%
\paragraph{v0.5:} 2017/04/26

\begin{itemize}
\item
functionality in definition file
\end{itemize}


%%%%%%%%%%%%%%%%%%%%%%%%%%%%%%%%%%%%%%%%%%%%%%%%%%%%%%%%%%%%%%%%%%%%%%%%%%%%%%%%
%%%%%%%%%%%%%%%%%%%%%%%%%%%%%%%%%%%%%%%%%%%%%%%%%%%%%%%%%%%%%%%%%%%%%%%%%%%%%%%%
%%%%%%%%%%%%%%%%%%%%%%%%%%%%%%%%%%%%%%%%%%%%%%%%%%%%%%%%%%%%%%%%%%%%%%%%%%%%%%%%
\appendix

\settowidth\MacroIndent{\rmfamily\scriptsize 000\ }

 \DocInput{childdoc.dtx}

\end{document}
%</driver>
% \fi
%
% %%%%%%%%%%%%%%%%%%%%%%%%%%%%%%%%%%%%%%%%%%%%%%%%%%%%%%%%%%%%%%%%%%%%%%%%%%%%%%
% %%%%%%%%%%%%%%%%%%%%%%%%%%%%%%%%%%%%%%%%%%%%%%%%%%%%%%%%%%%%%%%%%%%%%%%%%%%%%%
% \section{Sample}
%\iffalse
%<*samplemain>
%\fi
%
% The following presents a sample document
% with two chapters, two parts, a title page,
% a compile flag as well as three forwarding files to set the flag.
% It consists of eight |.tex| files:
% \begin{center}
% \begin{tabular}{ll}
% |cdocsamp.tex|&main file\\
% |cdocsch1.tex|&include file for chapter 1\\
% |cdocsch2.tex|&include file for chapter 2\\
% |cdocspt3.tex|&include file for part 3\\
% |cdocspt4.tex|&include file for part 4\\
% |cdocsdrf.tex|&forwarding file for main file in draft mode\\
% |cdocsfi1.tex|&forwarding file for final version of chapter 1\\
% |cdocsfi2.tex|&forwarding file for final version of chapter 2\\
% \end{tabular}
% \end{center}
% Each of the eight files can be compiled directly by the \LaTeX{} compiler.
%
% %%%%%%%%%%%%%%%%%%%%%%%%%%%%%%%%%%%%%%
% \paragraph{Main File.}
%
% The main file is called |cdocsamp.tex|.
%
% Load the \textsf{childdoc} definitions and
% declare the filename for the main document:
%    \begin{macrocode}
\input{childdoc.def}
\childdocmain{}
%    \end{macrocode}

% Optional override for |\version| flag:
%    \begin{macrocode}
%%\ifchilddoc\else\providecommand{\version}{draft}\fi
%    \end{macrocode}

% Define the default values for the |\version| flag
% (|final| for the main file and |draft| for childs):
%    \begin{macrocode}
\ifchilddoc
\providecommand{\version}{draft}
\else
\providecommand{\version}{final}
\fi
%    \end{macrocode}

% Load the standard document class:
%    \begin{macrocode}
\documentclass[12pt]{article}
%    \end{macrocode}

% Start the document body:
%    \begin{macrocode}
\begin{document}
%    \end{macrocode}

% Declare a title page.
% Print title, part of document being processed and version flag:
%    \begin{macrocode}
\addtocounter{page}{-1}
\begin{center}
{\LARGE\bfseries{}childdoc example\par}
\vspace{1cm}
\ifchilddoc
\ifchilddocmanual part\else chapter\fi:
`\childdocname' of `\childdocjob'\par
\else
main document: `\childdocjob'\par
\fi
version: \version\par
\end{center}
\newpage
%    \end{macrocode}

% Manually include selected file,
% otherwise process as usual:
%    \begin{macrocode}
\ifchilddocmanual
\section*{part `\childdocname'}
\input{\childdocname}
\else
%    \end{macrocode}

% Include the two chapters:
%    \begin{macrocode}
\include{cdocsch1}
\include{cdocsch2}
%    \end{macrocode}

% Include the two parts unless only chapters should be displayed:
%    \begin{macrocode}
\ifchilddoc\else
\section{part three}
\input{cdocspt3}
\section{part four}
\input{cdocspt4}
\fi
%    \end{macrocode}

% Process as usual until here:
%    \begin{macrocode}
\fi
%    \end{macrocode}

% End of document body:
%    \begin{macrocode}
\end{document}
%    \end{macrocode}
%\iffalse
%</samplemain>
%\fi
%
% %%%%%%%%%%%%%%%%%%%%%%%%%%%%%%%%%%%%%%
% \paragraph{Chapter Include Files.}
%
% The include files are called |cdocsch1.tex| and |cdocsch2.tex|.
%
%\iffalse
%<*samplechap1|samplechap2>
%\fi

% Optional override for |\version| flag:
%    \begin{macrocode}
%%\providecommand{\version}{final}
%    \end{macrocode}

% Include the main document:
%    \begin{macrocode}
\input{childdoc.def}
\childdocof{cdocsamp}
%    \end{macrocode}

%\iffalse
%</samplechap1|samplechap2>
%\fi
%
%\iffalse
%<*samplechap1>
%\fi
% Some text for chapter 1:
%    \begin{macrocode}
\section{one}
some text in chapter one
%    \end{macrocode}

%\iffalse
%</samplechap1>
%\fi
% Some text for chapter 2:
%\iffalse
%<*samplechap2>
%\fi
%    \begin{macrocode}
\section{two}
more text in chapter two
%    \end{macrocode}

%\iffalse
%</samplechap2>
%\fi
%
% %%%%%%%%%%%%%%%%%%%%%%%%%%%%%%%%%%%%%%
% \paragraph{Part Include Files.}
%
% The include files are called |cdocspt3.tex| and |cdocspt4.tex|.
%
%\iffalse
%<*samplepart3|samplepart4>
%\fi

% Optional override for |\version| flag:
%    \begin{macrocode}
%%\providecommand{\version}{final}
%    \end{macrocode}

% Include the main document:
%    \begin{macrocode}
\input{childdoc.def}
\childdocby{cdocsamp}
%    \end{macrocode}

%\iffalse
%</samplepart3|samplepart4>
%\fi
%
%\iffalse
%<*samplepart3>
%\fi
% Some text for part 3:
%    \begin{macrocode}
some text in part three
%    \end{macrocode}

%\iffalse
%</samplepart3>
%\fi
% Some text for part 4:
%\iffalse
%<*samplepart4>
%\fi
%    \begin{macrocode}
more text in part four
%    \end{macrocode}

%\iffalse
%</samplepart4>
%\fi
%
% %%%%%%%%%%%%%%%%%%%%%%%%%%%%%%%%%%%%%%
% \paragraph{Forwarding for a Complete Draft.}
%
% The following forwarding file |cdocsdrf.tex|
% compiles the main document in draft mode:
%\iffalse
%<*sampledraft>
%\fi
%    \begin{macrocode}
\def\version{draft}
\input{childdoc.def}
\childdocforward{cdocsamp}
%    \end{macrocode}

%\iffalse
%</sampledraft>
%\fi
%
% %%%%%%%%%%%%%%%%%%%%%%%%%%%%%%%%%%%%%%
% \paragraph{Forwarding for Final Version of the Chapters.}
%
% The following forwarding files |cdocsfn1.tex| and |cdocsfn2.tex|
% (with identical content)
% compile the final versions of the child documents
% |cdocsch1.tex| and |cdocsch2.tex|, respectively:
%\iffalse
%<*samplefinal>
%\fi
%    \begin{macrocode}
\def\version{final}
\input{childdoc.def}
\childdocforwardprefix[cdocsamp]{cdocsfn}{cdocsch}
%    \end{macrocode}

%\iffalse
%</samplefinal>
%\fi
%
% %%%%%%%%%%%%%%%%%%%%%%%%%%%%%%%%%%%%%%
% \paragraph{Command Line Processing.}
%
% The following three command lines generate the output files
% |cdocscld|, |cdocscl1| and |cdocscl2|
% which should be identical to
% |cdocsdrf|, |cdocsch1| and |cdocsfn2|, respectively:
% \begin{center}
% \begin{tabular}{l}
% |latex -jobname cdocscld \|\\
% |  "\def\version{draft}\input{childdoc.def}\childdocforward{cdocsamp}"|\\
% |latex -jobname cdocscl1 \|\\
% |  "\input{childdoc.def}\childdocforward[cdocsamp]{cdocsch1}"|\\
% |latex -jobname cdocscl2 \|\\
% |  "\def\version{final}\input{childdoc.def}\childdocforward{cdocsch2}"|
% \end{tabular}
% \end{center}
% Note that the trailing backslash on each first line
% merely continues the input to the second line
% (for convenient cut ant paste).
% Furthermore, the command |latex| can be replaced by any
% of its alternative versions such as |pdflatex|.
%
% %%%%%%%%%%%%%%%%%%%%%%%%%%%%%%%%%%%%%%%%%%%%%%%%%%%%%%%%%%%%%%%%%%%%%%%%%%%%%%
% %%%%%%%%%%%%%%%%%%%%%%%%%%%%%%%%%%%%%%%%%%%%%%%%%%%%%%%%%%%%%%%%%%%%%%%%%%%%%%
% \section{Implementation}
%\iffalse
%<*package>
%\fi
%
% This section describes the definitions file |childdoc.def|.

% The definitions cannot be loaded using |\usepackage| or |\RequirePackage|
% which has a mechanism to prevent loading a style file more than once.
% When loading the definitions by means of |\input|
% multiple instances have to be prevented manually:
%\iffalse
%This code needs to be before the `\ProvidesFile' directive
%which is defined at the beginning of this file.
%Therefore it is also placed there and commented out here.
%</package>
%<*discard>
%\fi
%    \begin{macrocode}
\ifdefined\childdocmain\endinput\fi
%    \end{macrocode}
%\iffalse
%</discard>
%<*package>
%\fi
%
% \macro{\ifchilddoc}
% \macro{\ifchilddocmanual}
% The conditional |\ifchilddoc| tells whether a
% child (true) or main (false) document is being compiled.
% The conditional |\ifchilddocmanual| tells whether
% the |\includeonly| mechanism is used (false) or
% the selection of child files must be performed manually (true).
% The definitions initialise to false:
%    \begin{macrocode}
\newif\ifchilddoc
\newif\ifchilddocmanual
%    \end{macrocode}

% \macro{\childdocname}
% \macro{\childdocjob}
% The macro |\childdocname| stores the name of the main document
% to be compiled. The macro |\childdocjob| stores the name of
% the document on which the \LaTeX{} compiler was originally invoked.
% The content of |\jobname| cannot be compared
% to filenames specified in the source due to different catcodes.
% The following code rescans |\jobname|, stores the result
% in |\childdocname| and saves a copy in |\childdocjob|:
%    \begin{macrocode}
\edef\childdocname{\scantokens\expandafter{\jobname\noexpand}}
\let\childdocjob\childdocname
%    \end{macrocode}

% \macro{\childdocdisable}
% The macro |\childdocdisable| prevents the main file
% from being processed more than once.
% At this stage, the main document command |\childdocmain|
% is assumed to be called once again where it should do nothing.
% Any subsequent call to it should prevent
% a secondary processing of the main document
% It overwrites the forwarding commands
% |\childdocof| and |\childdocforward|
% with empty macros to prevent further inclusions of the main document:
%    \begin{macrocode}
\newcommand{\childdocdisable}
{
  \renewcommand{\childdocmain}[1]{\renewcommand{\childdocmain}[1]{\endinput}}
  \renewcommand{\childdocof}[1]{}
  \renewcommand{\childdocby}[2][]{}
  \renewcommand{\childdocforward}[2][]{}
  \renewcommand{\childdocdisable}{}
}
%    \end{macrocode}

% \macro{\childdocmain}
% The macro |\childdocmain| is to be called at the top of the main file
% with nothing or the main filename (without extension) as argument.
% First, it breaks loops.
% If the argument is not empty and does not match |\childdocname|
% (which is set by the first inclusion of |childdoc.def|),
% |\ifchilddoc| is set to true, |\includeonly| is applied to the child file
% and |\jobname| is set to the main file
% (for proper handling of |.aux| files):
%    \begin{macrocode}
\newcommand{\childdocmain}[1]
{
  \childdocdisable\childdocmain{}
  \if?#1?\else
    \begingroup
      \def\childdoctmp{#1}
      \ifx\childdoctmp\childdocname
        \def\childdoctmp{}
      \else
        \def\childdoctmp
        {
          \childdoctrue
          \includeonly{\childdocname}
          \def\childdocjob{#1}
          \def\jobname{#1}
        }
      \fi
      \expandafter
    \endgroup
    \childdoctmp
  \fi
}
%    \end{macrocode}

% \macro{\childdocof}
% The command |\childdocof| redirects
% compilation to the main file |#1|.
%    \begin{macrocode}
\newcommand{\childdocof}[1]
{
  \childdocdisable
  \childdoctrue
  \includeonly{\childdocname}
  \def\jobname{#1}
  \def\childdocjob{#1}
  \input{#1}
}
%    \end{macrocode}

% \macro{\childdocby}
% The command |\childdocby| ....
%    \begin{macrocode}
\newcommand{\childdocby}[2][]
{
  \childdocdisable
  \childdoctrue
  \childdocmanualtrue
  \if?#1?\else
    \def\jobname{#2}
  \fi
  \def\childdocjob{#2}
  \input{#2}
  \endinput
}
%    \end{macrocode}

% \macro{\childdocforward}
% The command |\childdocforward| redirects
% compilation to the main file or
% (if the optional argument is given) a child file.
% Parameters are set as if the main file
% or a child file starting with |\childdocof| was compiled.
% Then compilation is handed over to the main file:
%    \begin{macrocode}
\newcommand{\childdocforward}[2][]
{
  \begingroup
    \if?#1?
      \def\childdoctmp
      {
        \def\childdocname{#2}
        \def\childdocjob{#2}
        \def\jobname{#2}
        \input{#2}
        \endinput
      }
    \else
      \def\childdoctmp
      {
        \childdocdisable
        \def\childdocname{#2}
        \childdoctrue
        \includeonly{#2}
        \def\childdocjob{#1}
        \def\jobname{#1}
        \input{#1}
        \endinput
      }
    \fi
    \expandafter
  \endgroup
  \childdoctmp
}
%    \end{macrocode}

% \macro{\childdocforwardprefix}
% The command |\childdocforwardprefix| redirects
% compilation to the main or a child file by means of a pattern.
% The prefix |#1| in the current filename is replaced by |#2|
% and the suffix of the current filename is kept
% (it is assumed that the filename does not contain the substring `|~~~|'
% which is used as a delimiter).
% Compilation is handed over to the new file by |\childdocforward|:
%    \begin{macrocode}
\newcommand{\childdocforwardprefix}[3][]
{
  \begingroup
    \def\childdocextract #2##1~~~{\def\childdoctmp{\childdocforward[#1]{#3##1}}}
    \expandafter\childdocextract\childdocname~~~
    \expandafter
  \endgroup
  \childdoctmp
}
%    \end{macrocode}

% \macro{\childdoc}
% The deprecated macro |\childdoc| is a legacy version of |\childdocmain|:
%    \begin{macrocode}
\newcommand{\childdoc}{\childdocmain}
%    \end{macrocode}

% \macro{\childdocredirect}
% The deprecated macro |\childdocredirect| is a legacy version
% of |\childdocforward| and |\childdocforwardprefix|:
%    \begin{macrocode}
\newcommand{\childdocredirect}[2][]
{
  \begingroup
    \if?#1?
      \def\childdoctmp{\childdocforward{#2}}
    \else
      \def\childdoctmp{\childdocforwardprefix{#1}{#2}}
    \fi
    \expandafter
  \endgroup
  \childdoctmp
}
%    \end{macrocode}

%\iffalse
%</package>
%\fi
%
\endinput
\childdocforward[|\textit{main}|]{|\textit{dest}|}"|
\end{center}
%
Here \textit{target} is the name of the output file,
\textit{main} is the name of the main file
and \textit{dest} is the name of the main or child file to be processed
(all filenames without extensions).
The optional argument \textit{main} can be omitted
if \textit{main} matches \textit{dest}.
Optionally, compilation \textit{flags} can be defined via |\def| commands.
This command line makes the \TeX{} engine believe
it is compiling the file \textit{target}
whose content is specified as the latter parameter.
The provided code then forwards the processing to
\textit{main} or \textit{dest} as described in \secref{sec:forward}.

%%%%%%%%%%%%%%%%%%%%%%%%%%%%%%%%%%%%%%%%%%%%%%%%%%%%%%%%%%%%%%%%%%%%%%%%%%%%%%%%
\subsection{Include by Input}
\label{sec:input}

Including child documents by |\include| has some restrictions by design.
Most notably, the content of a child document always occupies
its own set of pages; pages cannot be shared between child documents.
Usually, this behaviour makes perfect sense
because each child document contain an essential part of the document.
However, in some situations it may be desirable to compose
a document from a collection of parts
without having mandatory page breaks between then.
For this case, the package
provides a mechanism to include parts
by |\input| which can also be processed individually.
However, by construction this mechanism
requires manual handling of the content to be output.

%%%%%%%%%%%%%%%%%%%%%%%%%%%%%%%%%%%%%%%%
\DescribeMacro{\ifchilddocmanual}
The main file should be prepared as usual, see \secref{sec:include}.
However, the document body must make a distinction
between processing of an individual part and of the main document, e.g.:
%
\begin{center}
\begin{tabular}{l}
|\ifchilddocmanual|\\
|\input{\childdocname}|\\
|\||else|\\
\textit{document body with }|\input{|\textit{part}|}|\\
|\||fi|
\end{tabular}
\end{center}
%
The conditional |\ifchilddocmanual| is true whenever
a part to be included by |\input| is being compiled,
and the name of the part is stored in |\childdocname|.

%%%%%%%%%%%%%%%%%%%%%%%%%%%%%%%%%%%%%%%%
\DescribeMacro{\childdocby}
Each part to be included by |\input| should start with:
%
\begin{center}
\begin{tabular}{l}
|% \iffalse
%
% childdoc.dtx Copyright (C) 2017-2018 Niklas Beisert
%
% This work may be distributed and/or modified under the
% conditions of the LaTeX Project Public License, either version 1.3
% of this license or (at your option) any later version.
% The latest version of this license is in
%   http://www.latex-project.org/lppl.txt
% and version 1.3 or later is part of all distributions of LaTeX
% version 2005/12/01 or later.
%
% This work has the LPPL maintenance status `maintained'.
%
% The Current Maintainer of this work is Niklas Beisert.
%
% This work consists of the files childdoc.dtx and childdoc.ins
% and the derived files childdoc.def and cdocsamp.tex with
% cdocsch1.tex, cdocsch2.tex, cdocsdrf.tex, cdocsfn1.tex, cdocsfn2.tex.
%
%<package>\ifdefined\childdocmain\endinput\fi
%<package>\ProvidesFile{childdoc.def}[2018/12/30 v2.0 child document driver]
%<samplemain>\ProvidesFile{cdocsamp.tex}[2018/12/30 v2.0 sample for childdoc]
%<*driver>
%\ProvidesFile{childdoc.drv}[2018/12/30 v2.0 childdoc reference manual file]
\PassOptionsToClass{10pt,a4paper}{article}
\documentclass{ltxdoc}

\usepackage[margin=35mm]{geometry}
\usepackage{hyperref}
\usepackage{hyperxmp}
\usepackage[usenames]{color}

\hypersetup{colorlinks=true}
\hypersetup{pdfstartview=FitH}
\hypersetup{pdfpagemode=UseNone}
\hypersetup{pdfsource={}}
\hypersetup{pdflang={en-UK}}
\hypersetup{pdfcopyright={Copyright 2017-2018 Niklas Beisert.
  This work may be distributed and/or modified under the
  conditions of the LaTeX Project Public License, either version 1.3
  of this license or (at your option) any later version.}}
\hypersetup{pdflicenseurl={http://www.latex-project.org/lppl.txt}}
\hypersetup{pdfcontactaddress={ETH Zurich, ITP, HIT K,
  Wolfgang-Pauli-Strasse 27}}
\hypersetup{pdfcontactpostcode={8093}}
\hypersetup{pdfcontactcity={Zurich}}
\hypersetup{pdfcontactcountry={Switzerland}}
\hypersetup{pdfcontactemail={nbeisert@itp.phys.ethz.ch}}
\hypersetup{pdfcontacturl={http://people.phys.ethz.ch/\xmptilde nbeisert/}}

\newcommand{\secref}[1]{\hyperref[#1]{section \ref*{#1}}}

\parskip1ex
\parindent0pt
\let\olditemize\itemize
\def\itemize{\olditemize\parskip0pt}

\begin{document}

\title{The \textsf{childdoc} Package}
\hypersetup{pdftitle={The childdoc Package}}
\author{Niklas Beisert\\[2ex]
  Institut f\"ur Theoretische Physik\\
  Eidgen\"ossische Technische Hochschule Z\"urich\\
  Wolfgang-Pauli-Strasse 27, 8093 Z\"urich, Switzerland\\[1ex]
  \href{mailto:nbeisert@itp.phys.ethz.ch}
  {\texttt{nbeisert@itp.phys.ethz.ch}}}
\hypersetup{pdfauthor={Niklas Beisert}}
\hypersetup{pdfsubject={Manual for the LaTeX2e Package childdoc}}
\date{30 December 2018, \textsf{v2.0}}
\maketitle

\begin{abstract}\noindent
\textsf{childdoc} is a \LaTeXe{} package
that enables the direct compilation
of document sections included by |\include|
to individual files.
\end{abstract}

\begingroup
\parskip0ex
\tableofcontents
\endgroup

%%%%%%%%%%%%%%%%%%%%%%%%%%%%%%%%%%%%%%%%%%%%%%%%%%%%%%%%%%%%%%%%%%%%%%%%%%%%%%%%
%%%%%%%%%%%%%%%%%%%%%%%%%%%%%%%%%%%%%%%%%%%%%%%%%%%%%%%%%%%%%%%%%%%%%%%%%%%%%%%%
\section{Introduction}

\LaTeX{} provides a mechanism to structure a large document (such as a book)
into a main file and several child files (containing the chapters)
using the |\include| command.
This mechanism is beneficial for documents
which span hundreds of pages in order to
make the source file(s) more manageable.
Moreover, compilation can be restricted to
selected child files by means of the |\includeonly| command.
The latter feature can be used to reduce the compilation time while editing
(this was significantly more useful in the earlier days of \LaTeX{})
or to generate a smaller document which is easier to navigate.
Another application of |\includeonly| is to generate
documents consisting of selected parts of the complete document.

However, there are a few drawbacks of the plain |\include| mechanism:
\begin{itemize}
\item
The child files cannot be compiled on their own,
they can only be compiled via the main file.
A naive editing environment
(such as a text editor with an option
to have the current file processed by \LaTeX)
may require one to switch to the main file before compiling;
attempting to compile the child file produces errors.
\item
The main file must be modified (each time)
to adjust the |\includeonly| command
to the present needs. This easily leaves the main file in a messy state.
\item
The generated document will always carry the filename
of the main document. This is inconvenient if
several child files are to be compiled and
to be kept for distribution.
\end{itemize}

The present package provides a simple interface
to make child files individually compilable by \LaTeX{}.
Compiling a child file then has the same effect as compiling
the main file with an |\includeonly| command
to select the appropriate child.
Moreover the generated document will carry the name of the child
rather than the main file.
This resolves all three above issues.

This feature is meant to make the editing of books,
thesis documents and lecture notes somewhat more convenient.
However, the package can also be used efficiently for
composing a series of documents (such as exercise sheets)
which are typically distributed individually.
It then assists the author in generating the individual documents
(potentially in different versions)
as well as a document containing the collected series.
Another application is in developing style files
or other kinds of included material
where compilation of the style file could redirect
to a sample or test file.

%%%%%%%%%%%%%%%%%%%%%%%%%%%%%%%%%%%%%%%%%%%%%%%%%%%%%%%%%%%%%%%%%%%%%%%%%%%%%%%%
%%%%%%%%%%%%%%%%%%%%%%%%%%%%%%%%%%%%%%%%%%%%%%%%%%%%%%%%%%%%%%%%%%%%%%%%%%%%%%%%
\section{Usage}

First of all, the package \textsf{childdoc} is \emph{not} a standard
\LaTeXe{} |.sty| style file! Therefore it needs to be invoked in
a non-standard way.

%%%%%%%%%%%%%%%%%%%%%%%%%%%%%%%%%%%%%%%%%%%%%%%%%%%%%%%%%%%%%%%%%%%%%%%%%%%%%%%%
\subsection{Included Files}
\label{sec:include}

%%%%%%%%%%%%%%%%%%%%%%%%%%%%%%%%%%%%%%%%
\DescribeMacro{\childdocmain}
To use the package, add the commands
\begin{center}
\begin{tabular}{l}
|\input{childdoc.def}|\\
|\childdocmain{}|\\
\end{tabular}
\end{center}
at the very top of the main \LaTeX{} file,
in particular \emph{before} the |\documentclass| statement!
The argument of |\childdocmain| should be left empty
(but it must be present).

%%%%%%%%%%%%%%%%%%%%%%%%%%%%%%%%%%%%%%%%
\DescribeMacro{\childdocof}
Furthermore, add the commands
\begin{center}
\begin{tabular}{l}
|\input{childdoc.def}|\\
|\childdocof{|\textit{main}|}|\\
\end{tabular}
\end{center}
at the top of every child file \textit{child}
which is included by |\include{|\textit{child}|}|
from within the main file
(or at least for those files to be compiled individually).
The argument \textit{main} must be the filename of the main file.

There are a couple of
considerations in setting up the main and child documents:

%%%%%%%%%%%%%%%%%%%%%%%%%%%%%%%%%%%%%%%%
\paragraph{Restrictions.}

Please note the following restrictions:
\begin{itemize}
\item
|\childdocmain| must be called with one argument \textit{main}
to ensure compatibility with earlier version of the package.
It must either be empty (|\childdocmain{}|)
or precisely match the filename of the main file in which it is specified.
See \secref{sec:detection} for further information.
\item
The filename \textit{main} must be specified without the |.tex| extension.
\item
The filename \textit{main} is case sensitive
(even in case-insensitive file systems)
due to internal string comparison.
\item
The argument \textit{main} should be fully expanded, it cannot be a macro.
\item
Subdirectories and special characters should be avoided in filenames.
\item
The command |\childdocmain{|\textit{main}|}| must be followed by a whitespace.
It should not be followed immediately by another command
or by a comment mark `|%|'.
This is because the \TeX{} parser reads the token immediately following
the argument of |\childdocmain| and puts it
at the beginning of every child section;
however, a white\-space is ignored.
\end{itemize}

%%%%%%%%%%%%%%%%%%%%%%%%%%%%%%%%%%%%%%%%
\paragraph{Content of Main File.}

It is advisable to place all content in the child files included by |\include|.
Any output contained in the main file will appear in all child documents
unless suppressed manually;
it cannot be suppressed automatically by the |\includeonly| directive
and thus should normally be avoided.
A method to include some content in the main file
by means of conditional processing is described in \secref{sec:conditional}.

%%%%%%%%%%%%%%%%%%%%%%%%%%%%%%%%%%%%%%%%
\paragraph{Page Numbering.}

When only a part of the document is compiled,
the appropriate numbering of pages
(as well as other status parameters)
is determined from the |.aux| files.
The latter contain information from previous passes.
However this information needs to propagate through
all intermediate child documents.
Therefore the page numbering in child documents may well
be inconsistent until the complete document is compiled at least once.

A useful (if unconventional) way to always ensure a consistent
page numbering is to restart the numbering in each child document
and denote the pages by `\textit{child}|.|\textit{page}'
where \textit{child} represents the chapter/section number of the child file.
This can be achieved by the command
|\numberwithin{page}{|\textit{child}|}|
of the \textsf{amsmath} package
where \textit{child} can be |chapter| or |section|
depending on the chosen structuring.
Alternatively, one can modify the macro |\thepage| appropriately
and reset the counter |page| at the start of each child file.

%%%%%%%%%%%%%%%%%%%%%%%%%%%%%%%%%%%%%%%%%%%%%%%%%%%%%%%%%%%%%%%%%%%%%%%%%%%%%%%%
\subsection{Conditional Processing}
\label{sec:conditional}

The package provides a mechanism to compile different versions
of a document. To customise the versions further some conditional processing
can come in handy to distinguish which version is being compiled.
The package provides two macros to describe the compilation context:

%%%%%%%%%%%%%%%%%%%%%%%%%%%%%%%%%%%%%%%%
\DescribeMacro{\ifchilddoc}
The conditional |\ifchilddoc| distinguishes between the compilation of
child documents and the main document:
%
\begin{center}
|\ifchilddoc |\textit{child-code}| |[|\||else |\textit{main-code}]| \||fi|
\end{center}

%%%%%%%%%%%%%%%%%%%%%%%%%%%%%%%%%%%%%%%%
\DescribeMacro{\childdocname}
\DescribeMacro{\childdocjob}
The macro |\childdocname| contains the filename (without extension)
of the main or child file being processed.
Note that |\childdocjob| will always contain the name of the main file.

%%%%%%%%%%%%%%%%%%%%%%%%%%%%%%%%%%%%%%%%
\paragraph{Title Page.}

Conditional processing can be used to include a title or banner page
in the main document when proper precautions are taken.
Importantly, the code in the main file should ensure that the page counter
(as well as other status parameters which are stored in the |.aux| files)
takes the same value after the conditional processing.
Otherwise the page numbers may take divergent values
depending on which part is compiled.

For example, a title page could be declared by:
%
\begin{center}
\begin{tabular}{l}
|\ifchilddoc\||else|\\
|\addtocounter{page}{-1}|\\
\textit{code for title page}\\
|\newpage|\\
|\||fi|
\end{tabular}
\end{center}
%
A banner page for the child documents can be generated by:
%
\begin{center}
\begin{tabular}{l}
|\ifchilddoc|\\
|\addtocounter{page}{-1}|\\
\textit{code for banner page}\\
|\newpage|\\
|\||fi|
\end{tabular}
\end{center}
%
Here one could write a message such as:
\begin{center}
|This is the part \childdocname{} of \childdocjob{}.|
\end{center}

%%%%%%%%%%%%%%%%%%%%%%%%%%%%%%%%%%%%%%%%%%%%%%%%%%%%%%%%%%%%%%%%%%%%%%%%%%%%%%%%
\subsection{Flags}
\label{sec:flags}

The package makes it easy to generate different versions
of the main or child documents.
To this end compilation flags can be defined
and assigned different default values.
They will be particularly useful in conjunction
with the forwarding mechanism described in \secref{sec:forward}.

For example, it may be useful to have a flag |\version|
which can be set to |draft| or |final|.
The document source will contain some conditional code
depending on the value of |\version|.
Suppose further, the flag should default to |final| for the main file
and to |draft| for child files
which is a natural assignment for editing the document.
This is achieved by placing the following code
in the preamble of the main document
(below the |\childdocmain| directive):
%
\begin{center}
\begin{tabular}{l}
|\ifchilddoc|\\
|\providecommand{\version}{draft}|\\
|\||else|\\
|\providecommand{\version}{final}|\\
|\||fi|
\end{tabular}
\end{center}
%
The definition by |\providecommand| makes sure
that previous definitions are not overwritten.
Further statements |\providecommand{\version}{...}|
can thus be added before the above code to override it.

For the main file, one might add a line
(between |\childdocmain| and the above block)
%
\begin{center}
|%\ifchilddoc\||else\providecommand{\version}{draft}\||fi|
\end{center}
%
which can be uncommented to produce a draft version.
Likewise one can add a line to the very top of a child file
(above the |\childdocof{|\textit{main}|}| directive)
%
\begin{center}
|%\providecommand{\version}{final}|
\end{center}
%
which can be uncommented to produce the final version of this child document.

%%%%%%%%%%%%%%%%%%%%%%%%%%%%%%%%%%%%%%%%%%%%%%%%%%%%%%%%%%%%%%%%%%%%%%%%%%%%%%%%
\subsection{Forwarding}
\label{sec:forward}

Different versions of the main or child documents
using compilation flags as described in \secref{sec:flags}
can be (permanently) stored in different files
for convenient compilation, viewing and distribution.
To this end, the package defines a command
to pass on compilation to a different file:

%%%%%%%%%%%%%%%%%%%%%%%%%%%%%%%%%%%%%%%%
\DescribeMacro{\childdocforward}
The command |\childdocforward| redirects processing to
another source file:
%
\begin{center}
\begin{tabular}{l}
|\input{childdoc.def}|\\
|\childdocforward[|\textit{main}|]{|\textit{dest}|}|\\
\end{tabular}
\end{center}
%
The argument \textit{dest} is the destination file
(without extension).
It should be the main file or one of the child files.
Note that further \textsf{childdoc} directives
such as |\childdocof| and |\childdocforward|
in the indicated file will be processed in this form.
The optional argument \textit{main}
passes on directly to the main file \textit{main}
while pretending to compile the child \textit{dest}.
This form behaves as if \textit{dest}
issues |\childdocof{|\textit{main}|}| right away,
and no further \textsf{childdoc} directives will be processed.

%%%%%%%%%%%%%%%%%%%%%%%%%%%%%%%%%%%%%%%%
\DescribeMacro{\...prefix}
In the alternative form |\childdocforwardprefix|,
%
\begin{center}
\begin{tabular}{l}
|\input{childdoc.def}|\\
|\childdocforwardprefix[|\textit{main}|]{|\textit{prefix}|}{|\textit{dest}|}|
\end{tabular}
\end{center}
%
the destination file is determined by a pattern
depending on the current file:
To make this work, the current file must be called
`{\textit{prefix}\hspace{0.2em}\textit{suffix}}'
with \textit{prefix} matching precisely the argument.
Processing is then passed on to the file
`{\textit{dest}\hspace{0.2em}\textit{suffix}}'.
Surely, the same effect is achieved by
directly specifying the
argument `{\textit{dest}\hspace{0.2em}\textit{suffix}}'
in the first form.
However, that requires to set up a different file
for each child. With the alternative form of the command
all these files can have exactly the same content
which simplifies setting them up and maintaining them.

For example, the following file |draft.tex|
with a compilation flag |\version| as described in \secref{sec:flags}
compiles the main document as a draft:
%
\begin{center}
\begin{tabular}{l}
|\def\version{draft}|\\
|\input{childdoc.def}|\\
|\childdocforward{|\textit{main}|}|
\end{tabular}
\end{center}
%
Likewise, the following files |final|\textit{nn}|.tex|
compile the final version of the child document
|child|\textit{nn}|.tex|:
%
\begin{center}
\begin{tabular}{l}
|\def\version{final}|\\
|\input{childdoc.def}|\\
|\childdocforwardprefix{final}{child}|
\end{tabular}
\end{center}
%

Note that when several versions of a main file and/or of each child file
are to be generated, it may be convenient to set up a |Makefile| or
shell script to automatise the process.

%%%%%%%%%%%%%%%%%%%%%%%%%%%%%%%%%%%%%%%%%%%%%%%%%%%%%%%%%%%%%%%%%%%%%%%%%%%%%%%%
\subsection{Command Line Processing}
\label{sec:commandline}

The effect of redirection files can also be achieved by invoking
the \LaTeX{} compiler with a more elaborate command line.
Most conveniently this should be done as part
of a shell script or a |Makefile|.

When using \textsf{childdoc} in the main file, the following
command lines effectively perform a redirection
(note that depending on the shell being used,
backslashes may have to be doubled: `|\|' $\to$ `|\\|'):
%
\begin{center}
|... -jobname "|\textit{target}|" |\\|"|[\textit{flags}]%
|\input{childdoc.def}\childdocforward[|\textit{main}|]{|\textit{dest}|}"|
\end{center}
%
Here \textit{target} is the name of the output file,
\textit{main} is the name of the main file
and \textit{dest} is the name of the main or child file to be processed
(all filenames without extensions).
The optional argument \textit{main} can be omitted
if \textit{main} matches \textit{dest}.
Optionally, compilation \textit{flags} can be defined via |\def| commands.
This command line makes the \TeX{} engine believe
it is compiling the file \textit{target}
whose content is specified as the latter parameter.
The provided code then forwards the processing to
\textit{main} or \textit{dest} as described in \secref{sec:forward}.

%%%%%%%%%%%%%%%%%%%%%%%%%%%%%%%%%%%%%%%%%%%%%%%%%%%%%%%%%%%%%%%%%%%%%%%%%%%%%%%%
\subsection{Include by Input}
\label{sec:input}

Including child documents by |\include| has some restrictions by design.
Most notably, the content of a child document always occupies
its own set of pages; pages cannot be shared between child documents.
Usually, this behaviour makes perfect sense
because each child document contain an essential part of the document.
However, in some situations it may be desirable to compose
a document from a collection of parts
without having mandatory page breaks between then.
For this case, the package
provides a mechanism to include parts
by |\input| which can also be processed individually.
However, by construction this mechanism
requires manual handling of the content to be output.

%%%%%%%%%%%%%%%%%%%%%%%%%%%%%%%%%%%%%%%%
\DescribeMacro{\ifchilddocmanual}
The main file should be prepared as usual, see \secref{sec:include}.
However, the document body must make a distinction
between processing of an individual part and of the main document, e.g.:
%
\begin{center}
\begin{tabular}{l}
|\ifchilddocmanual|\\
|\input{\childdocname}|\\
|\||else|\\
\textit{document body with }|\input{|\textit{part}|}|\\
|\||fi|
\end{tabular}
\end{center}
%
The conditional |\ifchilddocmanual| is true whenever
a part to be included by |\input| is being compiled,
and the name of the part is stored in |\childdocname|.

%%%%%%%%%%%%%%%%%%%%%%%%%%%%%%%%%%%%%%%%
\DescribeMacro{\childdocby}
Each part to be included by |\input| should start with:
%
\begin{center}
\begin{tabular}{l}
|\input{childdoc.def}|\\
|\childdocby{|\textit{main}|}|\\
\end{tabular}
\end{center}
%
The directive |\childdocby| is similar to |\childdocof|
described in \secref{sec:include},
but the subsequent selection of content must be done manually.
To that end, both |\ifchilddoc| and |\ifchilddocmanual|
will be true upon processing of a part,
and the name of the part is stored in |\childdocname|.
Note that |\jobname| will be set to the filename of the current part
so that each part receives an individual |.aux| file
that does not interfere with the |.aux| file(s) of the main document.
This behaviour can be altered by the alternative form
|\childdocby[*]{|\textit{main}|}| (with a non-empty optional argument)
which uses the |.aux| file of the main document
by setting |\jobname| to \textit{main}.

%%%%%%%%%%%%%%%%%%%%%%%%%%%%%%%%%%%%%%%%%%%%%%%%%%%%%%%%%%%%%%%%%%%%%%%%%%%%%%%%
\subsection{Driver Development}
\label{sec:driver}

The \textsf{childdoc} mechanism can also be use for the development
of definition files such as \LaTeX{} styles or classes.
This case differs from the above setup with multiple parts
included by |\include| in that no |\includeonly| should be invoked.
This can be achieved by starting the include file
(before |\ProvidesPackage|) with:
%
\begin{center}
\begin{tabular}{l}
|\input{childdoc.def}|\\
|\childdocforward{|\textit{main}|}|\\
\end{tabular}
\end{center}
%
or alternatively with:
%
\begin{center}
\begin{tabular}{l}
|\input{childdoc.def}|\\
|\childdocby{|\textit{main}|}|\\
\end{tabular}
\end{center}
%
Both forms have slightly different effects as described above.
The main file is prepared as usual, see \secref{sec:include}.

%%%%%%%%%%%%%%%%%%%%%%%%%%%%%%%%%%%%%%%%%%%%%%%%%%%%%%%%%%%%%%%%%%%%%%%%%%%%%%%%
\subsection{Legacy Detection}
\label{sec:detection}

The directive |\childdocmain| in the main file can detect
whether the complete document or merely a child is to be compiled
even without using the directive |\childdocof|.
This method is deprecated because it is less robust
and there is no compelling reason to use it;
it is merely provided for backward compatibility
and it may be removed in future versions.

If the detection mechanism is to be used,
it is mandatory to correctly specify
the filename of the main file as the argument of |\childdocmain|:
%
\begin{center}
\begin{tabular}{l}
|\input{childdoc.def}|\\
|\childdocmain{|\textit{main}|}|\\
\end{tabular}
\end{center}
%
If |\jobname| does not match the argument \textit{main} of |\childdocmain|,
it is assumed that |\jobname| points to the child file to be compiled.
When using |\childdocmain| with the main file specified as argument,
it suffices to start a child file
with just |\input{|\textit{main}|}|
without loading of the package and using |\childdocof|.
If instead all processing is done
with the appropriate \textsf{childdoc} directives,
the argument of \textit{main} of |\childdocmain| can be empty.

An alternative version of the command line processing described
in \secref{sec:commandline} using the detection mechanism reads:
%
\begin{center}
|... -jobname "|\textit{target}|" "|[\textit{flags}]%
[|\def\jobname{|\textit{dest}|}|]|\input{|\textit{main}|}"|
\end{center}

%%%%%%%%%%%%%%%%%%%%%%%%%%%%%%%%%%%%%%%%%%%%%%%%%%%%%%%%%%%%%%%%%%%%%%%%%%%%%%%%
\subsection{Manual Code}
\label{sec:manual}

In case one cannot be certain whether the definitions file |childdoc.def|
is installed on the target \TeX{} distribution
and one prefers not to ship it,
it is conceivable to paste a few relevant commands into the sources.

To that end, drop all statements |\input{childdoc.def}|
and perform the replacements as outlined below.
Instead of |\childdocmain{|\textit{main}|}| add the following code
to the top of the main file:
%
\begin{center}
\begin{tabular}{l}
|\||ifdefined\childdocname\endinput\||fi\newif\ifchilddoc|\\
|\edef\childdocname{\scantokens\expandafter{\jobname\noexpand}}|\\
|\def\childdocmain{|\textit{main}|}\||ifx\childdocmain\childdocname\||else|\\
|\childdoctrue\includeonly{\childdocname}\let\jobname\childdocmain\||fi|\\
\end{tabular}
\end{center}
%
Instead of |\childdocof{|\textit{main}|}| just include the main file
at the top of each child file:
%
\begin{center}
|\input{|\textit{main}|}|
\end{center}
%
A simple redirection |\childdocforward{|\textit{dest}|}| is achieved by:
%
\begin{center}
|\def\jobname{|\textit{dest}|}\input{\jobname}|
\end{center}
%
The redirection with prefix
|\childdocforwardprefix[|\textit{prefix}|]{|\textit{dest}|}|
is accomplished by:
%
\begin{center}
\begin{tabular}{l}
|{\edef\jobname{\scantokens\expandafter{\jobname\noexpand}}|\\
|\def\redirectjob |\textit{prefix}|#1~~~{\gdef\jobname{|\textit{dest}|#1}}|\\
|\expandafter\redirectjob\jobname~~~}\input{\jobname}|
\end{tabular}
\end{center}

In an alternative approach,
child documents can be compiled by a specific command line
without additional code or specific definitions:
%
\begin{center}
|... -jobname "|\textit{target}|" "|[\textit{flags}]%
|\includeonly{|\textit{dest}|}\input{|\textit{main}|}"|
\end{center}
%

%%%%%%%%%%%%%%%%%%%%%%%%%%%%%%%%%%%%%%%%%%%%%%%%%%%%%%%%%%%%%%%%%%%%%%%%%%%%%%%%
%%%%%%%%%%%%%%%%%%%%%%%%%%%%%%%%%%%%%%%%%%%%%%%%%%%%%%%%%%%%%%%%%%%%%%%%%%%%%%%%
\section{Information}

%%%%%%%%%%%%%%%%%%%%%%%%%%%%%%%%%%%%%%%%%%%%%%%%%%%%%%%%%%%%%%%%%%%%%%%%%%%%%%%%
\subsection{Copyright}

Copyright \copyright{} 2017--2018 Niklas Beisert

This work may be distributed and/or modified under the
conditions of the \LaTeX{} Project Public License, either version 1.3
of this license or (at your option) any later version.
The latest version of this license is in
  \url{http://www.latex-project.org/lppl.txt}
and version 1.3 or later is part of all distributions of \LaTeX{}
version 2005/12/01 or later.

This work has the LPPL maintenance status `maintained'.

The Current Maintainer of this work is Niklas Beisert.

This work consists of the files |README.txt|, |childdoc.ins| and |childdoc.dtx|
as well as the derived files |childdoc.def|, |cdocsamp.tex|
with |cdocsch1.tex|, |cdocsch2.tex|, |cdocspt3.tex|, |cdocspt4.tex|,
|cdocsdrf.tex|, |cdocsfn1.tex|, |cdocsfn2.tex|
as well as |childdoc.pdf|.

%%%%%%%%%%%%%%%%%%%%%%%%%%%%%%%%%%%%%%%%%%%%%%%%%%%%%%%%%%%%%%%%%%%%%%%%%%%%%%%%
\subsection{Files and Installation}

The package consists of the files:
%
\begin{center}
\begin{tabular}{ll}
    |README.txt|   & readme file \\
    |childdoc.ins| & installation file \\
    |childdoc.dtx| & source file \\
    |childdoc.def| & definition file \\
    |cdocsamp.tex| & sample main file \\
    |cdocsch1.tex| & sample include file \\
    |cdocsch2.tex| & sample include file \\
    |cdocspt3.tex| & sample part file \\
    |cdocspt4.tex| & sample part file \\
    |cdocsdrf.tex| & sample redirection file \\
    |cdocsfn1.tex| & sample redirection file \\
    |cdocsfn2.tex| & sample redirection file \\
    |childdoc.pdf| & manual
\end{tabular}
\end{center}
%
The distribution consists of the files
|README.txt|, |childdoc.ins| and |childdoc.dtx|.
%
\begin{itemize}
\item
Run (pdf)\LaTeX{} on |childdoc.dtx|
to compile the manual |childdoc.pdf| (this file).
\item
Run \LaTeX{} on |childdoc.ins| to create the definitions file |childdoc.def|
and the sample |cdocsamp.tex| with include files
|cdocsch1.tex|, |cdocsch2.tex|, |cdocspt3.tex|, |cdocspt4.tex|,
|cdocsdrf.tex|, |cdocsfn1.tex|, |cdocsfn2.tex|.
Then copy the file |childdoc.def| to an appropriate directory of your \LaTeX{}
distribution, e.g.\ \textit{texmf-root}|/tex/latex/childdoc|.
\end{itemize}

%%%%%%%%%%%%%%%%%%%%%%%%%%%%%%%%%%%%%%%%%%%%%%%%%%%%%%%%%%%%%%%%%%%%%%%%%%%%%%%%
\subsection{Related CTAN Packages}

There are several other packages which offer a similar functionality:
%
\begin{itemize}
\item
The packages
\href{http://ctan.org/pkg/docmute}{\textsf{docmute}},
\href{http://ctan.org/pkg/includex}{\textsf{includex}} and
\href{http://ctan.org/pkg/standalone}{\textsf{standalone}}
provide commands to include only the document body of
a child file thus allowing both files to be compiled individually.
\item
The packages \href{http://ctan.org/pkg/subdocs}{\textsf{subdocs}}
and \href{http://ctan.org/pkg/subfiles}{\textsf{subfiles}}
provide structures in which the main and child documents can be
encapsulated and allowing them to be compiled individually.
The inclusion mechanism is different from the conventional |\include|.
\item
The package \href{http://ctan.org/pkg/combine}{\textsf{combine}}
is an elaborate solution to combine several documents into one.
\end{itemize}
%
See also the CTAN topic \href{http://ctan.org/topic/subdocs}{\textsf{subdocs}}
for further related packages.
The present package differs from the above solutions in that
a document structure constructed with the conventional |\include| mechanism
just needs two extra commands at the top of every file
such that all constituent files can be compiled individually.

%%%%%%%%%%%%%%%%%%%%%%%%%%%%%%%%%%%%%%%%%%%%%%%%%%%%%%%%%%%%%%%%%%%%%%%%%%%%%%%%
%\subsection{Feature Suggestions}
%
%The following is a list of features which may be useful for future
%versions of this package:
%%
%\begin{itemize}
%\item
%\ldots
%\end{itemize}

%%%%%%%%%%%%%%%%%%%%%%%%%%%%%%%%%%%%%%%%%%%%%%%%%%%%%%%%%%%%%%%%%%%%%%%%%%%%%%%%
\subsection{Revision History}

%%%%%%%%%%%%%%%%%%%%%%%%%%%%%%%%%%%%%%%%
\paragraph{v2.0:} 2018/12/30

\begin{itemize}
\item
immediate forward processing
\item
added |\childdocby| mechanism
\item
manual restructured
\end{itemize}

%%%%%%%%%%%%%%%%%%%%%%%%%%%%%%%%%%%%%%%%
\paragraph{v1.6:} 2018/01/17

\begin{itemize}
\item
application for development of include files
\item
corrections to manual
\end{itemize}

%%%%%%%%%%%%%%%%%%%%%%%%%%%%%%%%%%%%%%%%
\paragraph{v1.5:} 2017/05/21

\begin{itemize}
\item
more complete structuring introduced
\item
|\childdocof| introduced
\item
|\childdoc| renamed to |\childdocmain|
\item
|\childredirect| renamed to |\childdocforward| and |\childdocforwardprefix|
and functionality expanded
\end{itemize}

%%%%%%%%%%%%%%%%%%%%%%%%%%%%%%%%%%%%%%%%
\paragraph{v1.0:} 2017/04/27

\begin{itemize}
\item
manual and install package
\item
first version published on CTAN
\end{itemize}

%%%%%%%%%%%%%%%%%%%%%%%%%%%%%%%%%%%%%%%%
\paragraph{v0.6:} 2017/04/26

\begin{itemize}
\item
redirection mechanism added
\end{itemize}

%%%%%%%%%%%%%%%%%%%%%%%%%%%%%%%%%%%%%%%%
\paragraph{v0.5:} 2017/04/26

\begin{itemize}
\item
functionality in definition file
\end{itemize}


%%%%%%%%%%%%%%%%%%%%%%%%%%%%%%%%%%%%%%%%%%%%%%%%%%%%%%%%%%%%%%%%%%%%%%%%%%%%%%%%
%%%%%%%%%%%%%%%%%%%%%%%%%%%%%%%%%%%%%%%%%%%%%%%%%%%%%%%%%%%%%%%%%%%%%%%%%%%%%%%%
%%%%%%%%%%%%%%%%%%%%%%%%%%%%%%%%%%%%%%%%%%%%%%%%%%%%%%%%%%%%%%%%%%%%%%%%%%%%%%%%
\appendix

\settowidth\MacroIndent{\rmfamily\scriptsize 000\ }

 \DocInput{childdoc.dtx}

\end{document}
%</driver>
% \fi
%
% %%%%%%%%%%%%%%%%%%%%%%%%%%%%%%%%%%%%%%%%%%%%%%%%%%%%%%%%%%%%%%%%%%%%%%%%%%%%%%
% %%%%%%%%%%%%%%%%%%%%%%%%%%%%%%%%%%%%%%%%%%%%%%%%%%%%%%%%%%%%%%%%%%%%%%%%%%%%%%
% \section{Sample}
%\iffalse
%<*samplemain>
%\fi
%
% The following presents a sample document
% with two chapters, two parts, a title page,
% a compile flag as well as three forwarding files to set the flag.
% It consists of eight |.tex| files:
% \begin{center}
% \begin{tabular}{ll}
% |cdocsamp.tex|&main file\\
% |cdocsch1.tex|&include file for chapter 1\\
% |cdocsch2.tex|&include file for chapter 2\\
% |cdocspt3.tex|&include file for part 3\\
% |cdocspt4.tex|&include file for part 4\\
% |cdocsdrf.tex|&forwarding file for main file in draft mode\\
% |cdocsfi1.tex|&forwarding file for final version of chapter 1\\
% |cdocsfi2.tex|&forwarding file for final version of chapter 2\\
% \end{tabular}
% \end{center}
% Each of the eight files can be compiled directly by the \LaTeX{} compiler.
%
% %%%%%%%%%%%%%%%%%%%%%%%%%%%%%%%%%%%%%%
% \paragraph{Main File.}
%
% The main file is called |cdocsamp.tex|.
%
% Load the \textsf{childdoc} definitions and
% declare the filename for the main document:
%    \begin{macrocode}
\input{childdoc.def}
\childdocmain{}
%    \end{macrocode}

% Optional override for |\version| flag:
%    \begin{macrocode}
%%\ifchilddoc\else\providecommand{\version}{draft}\fi
%    \end{macrocode}

% Define the default values for the |\version| flag
% (|final| for the main file and |draft| for childs):
%    \begin{macrocode}
\ifchilddoc
\providecommand{\version}{draft}
\else
\providecommand{\version}{final}
\fi
%    \end{macrocode}

% Load the standard document class:
%    \begin{macrocode}
\documentclass[12pt]{article}
%    \end{macrocode}

% Start the document body:
%    \begin{macrocode}
\begin{document}
%    \end{macrocode}

% Declare a title page.
% Print title, part of document being processed and version flag:
%    \begin{macrocode}
\addtocounter{page}{-1}
\begin{center}
{\LARGE\bfseries{}childdoc example\par}
\vspace{1cm}
\ifchilddoc
\ifchilddocmanual part\else chapter\fi:
`\childdocname' of `\childdocjob'\par
\else
main document: `\childdocjob'\par
\fi
version: \version\par
\end{center}
\newpage
%    \end{macrocode}

% Manually include selected file,
% otherwise process as usual:
%    \begin{macrocode}
\ifchilddocmanual
\section*{part `\childdocname'}
\input{\childdocname}
\else
%    \end{macrocode}

% Include the two chapters:
%    \begin{macrocode}
\include{cdocsch1}
\include{cdocsch2}
%    \end{macrocode}

% Include the two parts unless only chapters should be displayed:
%    \begin{macrocode}
\ifchilddoc\else
\section{part three}
\input{cdocspt3}
\section{part four}
\input{cdocspt4}
\fi
%    \end{macrocode}

% Process as usual until here:
%    \begin{macrocode}
\fi
%    \end{macrocode}

% End of document body:
%    \begin{macrocode}
\end{document}
%    \end{macrocode}
%\iffalse
%</samplemain>
%\fi
%
% %%%%%%%%%%%%%%%%%%%%%%%%%%%%%%%%%%%%%%
% \paragraph{Chapter Include Files.}
%
% The include files are called |cdocsch1.tex| and |cdocsch2.tex|.
%
%\iffalse
%<*samplechap1|samplechap2>
%\fi

% Optional override for |\version| flag:
%    \begin{macrocode}
%%\providecommand{\version}{final}
%    \end{macrocode}

% Include the main document:
%    \begin{macrocode}
\input{childdoc.def}
\childdocof{cdocsamp}
%    \end{macrocode}

%\iffalse
%</samplechap1|samplechap2>
%\fi
%
%\iffalse
%<*samplechap1>
%\fi
% Some text for chapter 1:
%    \begin{macrocode}
\section{one}
some text in chapter one
%    \end{macrocode}

%\iffalse
%</samplechap1>
%\fi
% Some text for chapter 2:
%\iffalse
%<*samplechap2>
%\fi
%    \begin{macrocode}
\section{two}
more text in chapter two
%    \end{macrocode}

%\iffalse
%</samplechap2>
%\fi
%
% %%%%%%%%%%%%%%%%%%%%%%%%%%%%%%%%%%%%%%
% \paragraph{Part Include Files.}
%
% The include files are called |cdocspt3.tex| and |cdocspt4.tex|.
%
%\iffalse
%<*samplepart3|samplepart4>
%\fi

% Optional override for |\version| flag:
%    \begin{macrocode}
%%\providecommand{\version}{final}
%    \end{macrocode}

% Include the main document:
%    \begin{macrocode}
\input{childdoc.def}
\childdocby{cdocsamp}
%    \end{macrocode}

%\iffalse
%</samplepart3|samplepart4>
%\fi
%
%\iffalse
%<*samplepart3>
%\fi
% Some text for part 3:
%    \begin{macrocode}
some text in part three
%    \end{macrocode}

%\iffalse
%</samplepart3>
%\fi
% Some text for part 4:
%\iffalse
%<*samplepart4>
%\fi
%    \begin{macrocode}
more text in part four
%    \end{macrocode}

%\iffalse
%</samplepart4>
%\fi
%
% %%%%%%%%%%%%%%%%%%%%%%%%%%%%%%%%%%%%%%
% \paragraph{Forwarding for a Complete Draft.}
%
% The following forwarding file |cdocsdrf.tex|
% compiles the main document in draft mode:
%\iffalse
%<*sampledraft>
%\fi
%    \begin{macrocode}
\def\version{draft}
\input{childdoc.def}
\childdocforward{cdocsamp}
%    \end{macrocode}

%\iffalse
%</sampledraft>
%\fi
%
% %%%%%%%%%%%%%%%%%%%%%%%%%%%%%%%%%%%%%%
% \paragraph{Forwarding for Final Version of the Chapters.}
%
% The following forwarding files |cdocsfn1.tex| and |cdocsfn2.tex|
% (with identical content)
% compile the final versions of the child documents
% |cdocsch1.tex| and |cdocsch2.tex|, respectively:
%\iffalse
%<*samplefinal>
%\fi
%    \begin{macrocode}
\def\version{final}
\input{childdoc.def}
\childdocforwardprefix[cdocsamp]{cdocsfn}{cdocsch}
%    \end{macrocode}

%\iffalse
%</samplefinal>
%\fi
%
% %%%%%%%%%%%%%%%%%%%%%%%%%%%%%%%%%%%%%%
% \paragraph{Command Line Processing.}
%
% The following three command lines generate the output files
% |cdocscld|, |cdocscl1| and |cdocscl2|
% which should be identical to
% |cdocsdrf|, |cdocsch1| and |cdocsfn2|, respectively:
% \begin{center}
% \begin{tabular}{l}
% |latex -jobname cdocscld \|\\
% |  "\def\version{draft}\input{childdoc.def}\childdocforward{cdocsamp}"|\\
% |latex -jobname cdocscl1 \|\\
% |  "\input{childdoc.def}\childdocforward[cdocsamp]{cdocsch1}"|\\
% |latex -jobname cdocscl2 \|\\
% |  "\def\version{final}\input{childdoc.def}\childdocforward{cdocsch2}"|
% \end{tabular}
% \end{center}
% Note that the trailing backslash on each first line
% merely continues the input to the second line
% (for convenient cut ant paste).
% Furthermore, the command |latex| can be replaced by any
% of its alternative versions such as |pdflatex|.
%
% %%%%%%%%%%%%%%%%%%%%%%%%%%%%%%%%%%%%%%%%%%%%%%%%%%%%%%%%%%%%%%%%%%%%%%%%%%%%%%
% %%%%%%%%%%%%%%%%%%%%%%%%%%%%%%%%%%%%%%%%%%%%%%%%%%%%%%%%%%%%%%%%%%%%%%%%%%%%%%
% \section{Implementation}
%\iffalse
%<*package>
%\fi
%
% This section describes the definitions file |childdoc.def|.

% The definitions cannot be loaded using |\usepackage| or |\RequirePackage|
% which has a mechanism to prevent loading a style file more than once.
% When loading the definitions by means of |\input|
% multiple instances have to be prevented manually:
%\iffalse
%This code needs to be before the `\ProvidesFile' directive
%which is defined at the beginning of this file.
%Therefore it is also placed there and commented out here.
%</package>
%<*discard>
%\fi
%    \begin{macrocode}
\ifdefined\childdocmain\endinput\fi
%    \end{macrocode}
%\iffalse
%</discard>
%<*package>
%\fi
%
% \macro{\ifchilddoc}
% \macro{\ifchilddocmanual}
% The conditional |\ifchilddoc| tells whether a
% child (true) or main (false) document is being compiled.
% The conditional |\ifchilddocmanual| tells whether
% the |\includeonly| mechanism is used (false) or
% the selection of child files must be performed manually (true).
% The definitions initialise to false:
%    \begin{macrocode}
\newif\ifchilddoc
\newif\ifchilddocmanual
%    \end{macrocode}

% \macro{\childdocname}
% \macro{\childdocjob}
% The macro |\childdocname| stores the name of the main document
% to be compiled. The macro |\childdocjob| stores the name of
% the document on which the \LaTeX{} compiler was originally invoked.
% The content of |\jobname| cannot be compared
% to filenames specified in the source due to different catcodes.
% The following code rescans |\jobname|, stores the result
% in |\childdocname| and saves a copy in |\childdocjob|:
%    \begin{macrocode}
\edef\childdocname{\scantokens\expandafter{\jobname\noexpand}}
\let\childdocjob\childdocname
%    \end{macrocode}

% \macro{\childdocdisable}
% The macro |\childdocdisable| prevents the main file
% from being processed more than once.
% At this stage, the main document command |\childdocmain|
% is assumed to be called once again where it should do nothing.
% Any subsequent call to it should prevent
% a secondary processing of the main document
% It overwrites the forwarding commands
% |\childdocof| and |\childdocforward|
% with empty macros to prevent further inclusions of the main document:
%    \begin{macrocode}
\newcommand{\childdocdisable}
{
  \renewcommand{\childdocmain}[1]{\renewcommand{\childdocmain}[1]{\endinput}}
  \renewcommand{\childdocof}[1]{}
  \renewcommand{\childdocby}[2][]{}
  \renewcommand{\childdocforward}[2][]{}
  \renewcommand{\childdocdisable}{}
}
%    \end{macrocode}

% \macro{\childdocmain}
% The macro |\childdocmain| is to be called at the top of the main file
% with nothing or the main filename (without extension) as argument.
% First, it breaks loops.
% If the argument is not empty and does not match |\childdocname|
% (which is set by the first inclusion of |childdoc.def|),
% |\ifchilddoc| is set to true, |\includeonly| is applied to the child file
% and |\jobname| is set to the main file
% (for proper handling of |.aux| files):
%    \begin{macrocode}
\newcommand{\childdocmain}[1]
{
  \childdocdisable\childdocmain{}
  \if?#1?\else
    \begingroup
      \def\childdoctmp{#1}
      \ifx\childdoctmp\childdocname
        \def\childdoctmp{}
      \else
        \def\childdoctmp
        {
          \childdoctrue
          \includeonly{\childdocname}
          \def\childdocjob{#1}
          \def\jobname{#1}
        }
      \fi
      \expandafter
    \endgroup
    \childdoctmp
  \fi
}
%    \end{macrocode}

% \macro{\childdocof}
% The command |\childdocof| redirects
% compilation to the main file |#1|.
%    \begin{macrocode}
\newcommand{\childdocof}[1]
{
  \childdocdisable
  \childdoctrue
  \includeonly{\childdocname}
  \def\jobname{#1}
  \def\childdocjob{#1}
  \input{#1}
}
%    \end{macrocode}

% \macro{\childdocby}
% The command |\childdocby| ....
%    \begin{macrocode}
\newcommand{\childdocby}[2][]
{
  \childdocdisable
  \childdoctrue
  \childdocmanualtrue
  \if?#1?\else
    \def\jobname{#2}
  \fi
  \def\childdocjob{#2}
  \input{#2}
  \endinput
}
%    \end{macrocode}

% \macro{\childdocforward}
% The command |\childdocforward| redirects
% compilation to the main file or
% (if the optional argument is given) a child file.
% Parameters are set as if the main file
% or a child file starting with |\childdocof| was compiled.
% Then compilation is handed over to the main file:
%    \begin{macrocode}
\newcommand{\childdocforward}[2][]
{
  \begingroup
    \if?#1?
      \def\childdoctmp
      {
        \def\childdocname{#2}
        \def\childdocjob{#2}
        \def\jobname{#2}
        \input{#2}
        \endinput
      }
    \else
      \def\childdoctmp
      {
        \childdocdisable
        \def\childdocname{#2}
        \childdoctrue
        \includeonly{#2}
        \def\childdocjob{#1}
        \def\jobname{#1}
        \input{#1}
        \endinput
      }
    \fi
    \expandafter
  \endgroup
  \childdoctmp
}
%    \end{macrocode}

% \macro{\childdocforwardprefix}
% The command |\childdocforwardprefix| redirects
% compilation to the main or a child file by means of a pattern.
% The prefix |#1| in the current filename is replaced by |#2|
% and the suffix of the current filename is kept
% (it is assumed that the filename does not contain the substring `|~~~|'
% which is used as a delimiter).
% Compilation is handed over to the new file by |\childdocforward|:
%    \begin{macrocode}
\newcommand{\childdocforwardprefix}[3][]
{
  \begingroup
    \def\childdocextract #2##1~~~{\def\childdoctmp{\childdocforward[#1]{#3##1}}}
    \expandafter\childdocextract\childdocname~~~
    \expandafter
  \endgroup
  \childdoctmp
}
%    \end{macrocode}

% \macro{\childdoc}
% The deprecated macro |\childdoc| is a legacy version of |\childdocmain|:
%    \begin{macrocode}
\newcommand{\childdoc}{\childdocmain}
%    \end{macrocode}

% \macro{\childdocredirect}
% The deprecated macro |\childdocredirect| is a legacy version
% of |\childdocforward| and |\childdocforwardprefix|:
%    \begin{macrocode}
\newcommand{\childdocredirect}[2][]
{
  \begingroup
    \if?#1?
      \def\childdoctmp{\childdocforward{#2}}
    \else
      \def\childdoctmp{\childdocforwardprefix{#1}{#2}}
    \fi
    \expandafter
  \endgroup
  \childdoctmp
}
%    \end{macrocode}

%\iffalse
%</package>
%\fi
%
\endinput
|\\
|\childdocby{|\textit{main}|}|\\
\end{tabular}
\end{center}
%
The directive |\childdocby| is similar to |\childdocof|
described in \secref{sec:include},
but the subsequent selection of content must be done manually.
To that end, both |\ifchilddoc| and |\ifchilddocmanual|
will be true upon processing of a part,
and the name of the part is stored in |\childdocname|.
Note that |\jobname| will be set to the filename of the current part
so that each part receives an individual |.aux| file
that does not interfere with the |.aux| file(s) of the main document.
This behaviour can be altered by the alternative form
|\childdocby[*]{|\textit{main}|}| (with a non-empty optional argument)
which uses the |.aux| file of the main document
by setting |\jobname| to \textit{main}.

%%%%%%%%%%%%%%%%%%%%%%%%%%%%%%%%%%%%%%%%%%%%%%%%%%%%%%%%%%%%%%%%%%%%%%%%%%%%%%%%
\subsection{Driver Development}
\label{sec:driver}

The \textsf{childdoc} mechanism can also be use for the development
of definition files such as \LaTeX{} styles or classes.
This case differs from the above setup with multiple parts
included by |\include| in that no |\includeonly| should be invoked.
This can be achieved by starting the include file
(before |\ProvidesPackage|) with:
%
\begin{center}
\begin{tabular}{l}
|% \iffalse
%
% childdoc.dtx Copyright (C) 2017-2018 Niklas Beisert
%
% This work may be distributed and/or modified under the
% conditions of the LaTeX Project Public License, either version 1.3
% of this license or (at your option) any later version.
% The latest version of this license is in
%   http://www.latex-project.org/lppl.txt
% and version 1.3 or later is part of all distributions of LaTeX
% version 2005/12/01 or later.
%
% This work has the LPPL maintenance status `maintained'.
%
% The Current Maintainer of this work is Niklas Beisert.
%
% This work consists of the files childdoc.dtx and childdoc.ins
% and the derived files childdoc.def and cdocsamp.tex with
% cdocsch1.tex, cdocsch2.tex, cdocsdrf.tex, cdocsfn1.tex, cdocsfn2.tex.
%
%<package>\ifdefined\childdocmain\endinput\fi
%<package>\ProvidesFile{childdoc.def}[2018/12/30 v2.0 child document driver]
%<samplemain>\ProvidesFile{cdocsamp.tex}[2018/12/30 v2.0 sample for childdoc]
%<*driver>
%\ProvidesFile{childdoc.drv}[2018/12/30 v2.0 childdoc reference manual file]
\PassOptionsToClass{10pt,a4paper}{article}
\documentclass{ltxdoc}

\usepackage[margin=35mm]{geometry}
\usepackage{hyperref}
\usepackage{hyperxmp}
\usepackage[usenames]{color}

\hypersetup{colorlinks=true}
\hypersetup{pdfstartview=FitH}
\hypersetup{pdfpagemode=UseNone}
\hypersetup{pdfsource={}}
\hypersetup{pdflang={en-UK}}
\hypersetup{pdfcopyright={Copyright 2017-2018 Niklas Beisert.
  This work may be distributed and/or modified under the
  conditions of the LaTeX Project Public License, either version 1.3
  of this license or (at your option) any later version.}}
\hypersetup{pdflicenseurl={http://www.latex-project.org/lppl.txt}}
\hypersetup{pdfcontactaddress={ETH Zurich, ITP, HIT K,
  Wolfgang-Pauli-Strasse 27}}
\hypersetup{pdfcontactpostcode={8093}}
\hypersetup{pdfcontactcity={Zurich}}
\hypersetup{pdfcontactcountry={Switzerland}}
\hypersetup{pdfcontactemail={nbeisert@itp.phys.ethz.ch}}
\hypersetup{pdfcontacturl={http://people.phys.ethz.ch/\xmptilde nbeisert/}}

\newcommand{\secref}[1]{\hyperref[#1]{section \ref*{#1}}}

\parskip1ex
\parindent0pt
\let\olditemize\itemize
\def\itemize{\olditemize\parskip0pt}

\begin{document}

\title{The \textsf{childdoc} Package}
\hypersetup{pdftitle={The childdoc Package}}
\author{Niklas Beisert\\[2ex]
  Institut f\"ur Theoretische Physik\\
  Eidgen\"ossische Technische Hochschule Z\"urich\\
  Wolfgang-Pauli-Strasse 27, 8093 Z\"urich, Switzerland\\[1ex]
  \href{mailto:nbeisert@itp.phys.ethz.ch}
  {\texttt{nbeisert@itp.phys.ethz.ch}}}
\hypersetup{pdfauthor={Niklas Beisert}}
\hypersetup{pdfsubject={Manual for the LaTeX2e Package childdoc}}
\date{30 December 2018, \textsf{v2.0}}
\maketitle

\begin{abstract}\noindent
\textsf{childdoc} is a \LaTeXe{} package
that enables the direct compilation
of document sections included by |\include|
to individual files.
\end{abstract}

\begingroup
\parskip0ex
\tableofcontents
\endgroup

%%%%%%%%%%%%%%%%%%%%%%%%%%%%%%%%%%%%%%%%%%%%%%%%%%%%%%%%%%%%%%%%%%%%%%%%%%%%%%%%
%%%%%%%%%%%%%%%%%%%%%%%%%%%%%%%%%%%%%%%%%%%%%%%%%%%%%%%%%%%%%%%%%%%%%%%%%%%%%%%%
\section{Introduction}

\LaTeX{} provides a mechanism to structure a large document (such as a book)
into a main file and several child files (containing the chapters)
using the |\include| command.
This mechanism is beneficial for documents
which span hundreds of pages in order to
make the source file(s) more manageable.
Moreover, compilation can be restricted to
selected child files by means of the |\includeonly| command.
The latter feature can be used to reduce the compilation time while editing
(this was significantly more useful in the earlier days of \LaTeX{})
or to generate a smaller document which is easier to navigate.
Another application of |\includeonly| is to generate
documents consisting of selected parts of the complete document.

However, there are a few drawbacks of the plain |\include| mechanism:
\begin{itemize}
\item
The child files cannot be compiled on their own,
they can only be compiled via the main file.
A naive editing environment
(such as a text editor with an option
to have the current file processed by \LaTeX)
may require one to switch to the main file before compiling;
attempting to compile the child file produces errors.
\item
The main file must be modified (each time)
to adjust the |\includeonly| command
to the present needs. This easily leaves the main file in a messy state.
\item
The generated document will always carry the filename
of the main document. This is inconvenient if
several child files are to be compiled and
to be kept for distribution.
\end{itemize}

The present package provides a simple interface
to make child files individually compilable by \LaTeX{}.
Compiling a child file then has the same effect as compiling
the main file with an |\includeonly| command
to select the appropriate child.
Moreover the generated document will carry the name of the child
rather than the main file.
This resolves all three above issues.

This feature is meant to make the editing of books,
thesis documents and lecture notes somewhat more convenient.
However, the package can also be used efficiently for
composing a series of documents (such as exercise sheets)
which are typically distributed individually.
It then assists the author in generating the individual documents
(potentially in different versions)
as well as a document containing the collected series.
Another application is in developing style files
or other kinds of included material
where compilation of the style file could redirect
to a sample or test file.

%%%%%%%%%%%%%%%%%%%%%%%%%%%%%%%%%%%%%%%%%%%%%%%%%%%%%%%%%%%%%%%%%%%%%%%%%%%%%%%%
%%%%%%%%%%%%%%%%%%%%%%%%%%%%%%%%%%%%%%%%%%%%%%%%%%%%%%%%%%%%%%%%%%%%%%%%%%%%%%%%
\section{Usage}

First of all, the package \textsf{childdoc} is \emph{not} a standard
\LaTeXe{} |.sty| style file! Therefore it needs to be invoked in
a non-standard way.

%%%%%%%%%%%%%%%%%%%%%%%%%%%%%%%%%%%%%%%%%%%%%%%%%%%%%%%%%%%%%%%%%%%%%%%%%%%%%%%%
\subsection{Included Files}
\label{sec:include}

%%%%%%%%%%%%%%%%%%%%%%%%%%%%%%%%%%%%%%%%
\DescribeMacro{\childdocmain}
To use the package, add the commands
\begin{center}
\begin{tabular}{l}
|\input{childdoc.def}|\\
|\childdocmain{}|\\
\end{tabular}
\end{center}
at the very top of the main \LaTeX{} file,
in particular \emph{before} the |\documentclass| statement!
The argument of |\childdocmain| should be left empty
(but it must be present).

%%%%%%%%%%%%%%%%%%%%%%%%%%%%%%%%%%%%%%%%
\DescribeMacro{\childdocof}
Furthermore, add the commands
\begin{center}
\begin{tabular}{l}
|\input{childdoc.def}|\\
|\childdocof{|\textit{main}|}|\\
\end{tabular}
\end{center}
at the top of every child file \textit{child}
which is included by |\include{|\textit{child}|}|
from within the main file
(or at least for those files to be compiled individually).
The argument \textit{main} must be the filename of the main file.

There are a couple of
considerations in setting up the main and child documents:

%%%%%%%%%%%%%%%%%%%%%%%%%%%%%%%%%%%%%%%%
\paragraph{Restrictions.}

Please note the following restrictions:
\begin{itemize}
\item
|\childdocmain| must be called with one argument \textit{main}
to ensure compatibility with earlier version of the package.
It must either be empty (|\childdocmain{}|)
or precisely match the filename of the main file in which it is specified.
See \secref{sec:detection} for further information.
\item
The filename \textit{main} must be specified without the |.tex| extension.
\item
The filename \textit{main} is case sensitive
(even in case-insensitive file systems)
due to internal string comparison.
\item
The argument \textit{main} should be fully expanded, it cannot be a macro.
\item
Subdirectories and special characters should be avoided in filenames.
\item
The command |\childdocmain{|\textit{main}|}| must be followed by a whitespace.
It should not be followed immediately by another command
or by a comment mark `|%|'.
This is because the \TeX{} parser reads the token immediately following
the argument of |\childdocmain| and puts it
at the beginning of every child section;
however, a white\-space is ignored.
\end{itemize}

%%%%%%%%%%%%%%%%%%%%%%%%%%%%%%%%%%%%%%%%
\paragraph{Content of Main File.}

It is advisable to place all content in the child files included by |\include|.
Any output contained in the main file will appear in all child documents
unless suppressed manually;
it cannot be suppressed automatically by the |\includeonly| directive
and thus should normally be avoided.
A method to include some content in the main file
by means of conditional processing is described in \secref{sec:conditional}.

%%%%%%%%%%%%%%%%%%%%%%%%%%%%%%%%%%%%%%%%
\paragraph{Page Numbering.}

When only a part of the document is compiled,
the appropriate numbering of pages
(as well as other status parameters)
is determined from the |.aux| files.
The latter contain information from previous passes.
However this information needs to propagate through
all intermediate child documents.
Therefore the page numbering in child documents may well
be inconsistent until the complete document is compiled at least once.

A useful (if unconventional) way to always ensure a consistent
page numbering is to restart the numbering in each child document
and denote the pages by `\textit{child}|.|\textit{page}'
where \textit{child} represents the chapter/section number of the child file.
This can be achieved by the command
|\numberwithin{page}{|\textit{child}|}|
of the \textsf{amsmath} package
where \textit{child} can be |chapter| or |section|
depending on the chosen structuring.
Alternatively, one can modify the macro |\thepage| appropriately
and reset the counter |page| at the start of each child file.

%%%%%%%%%%%%%%%%%%%%%%%%%%%%%%%%%%%%%%%%%%%%%%%%%%%%%%%%%%%%%%%%%%%%%%%%%%%%%%%%
\subsection{Conditional Processing}
\label{sec:conditional}

The package provides a mechanism to compile different versions
of a document. To customise the versions further some conditional processing
can come in handy to distinguish which version is being compiled.
The package provides two macros to describe the compilation context:

%%%%%%%%%%%%%%%%%%%%%%%%%%%%%%%%%%%%%%%%
\DescribeMacro{\ifchilddoc}
The conditional |\ifchilddoc| distinguishes between the compilation of
child documents and the main document:
%
\begin{center}
|\ifchilddoc |\textit{child-code}| |[|\||else |\textit{main-code}]| \||fi|
\end{center}

%%%%%%%%%%%%%%%%%%%%%%%%%%%%%%%%%%%%%%%%
\DescribeMacro{\childdocname}
\DescribeMacro{\childdocjob}
The macro |\childdocname| contains the filename (without extension)
of the main or child file being processed.
Note that |\childdocjob| will always contain the name of the main file.

%%%%%%%%%%%%%%%%%%%%%%%%%%%%%%%%%%%%%%%%
\paragraph{Title Page.}

Conditional processing can be used to include a title or banner page
in the main document when proper precautions are taken.
Importantly, the code in the main file should ensure that the page counter
(as well as other status parameters which are stored in the |.aux| files)
takes the same value after the conditional processing.
Otherwise the page numbers may take divergent values
depending on which part is compiled.

For example, a title page could be declared by:
%
\begin{center}
\begin{tabular}{l}
|\ifchilddoc\||else|\\
|\addtocounter{page}{-1}|\\
\textit{code for title page}\\
|\newpage|\\
|\||fi|
\end{tabular}
\end{center}
%
A banner page for the child documents can be generated by:
%
\begin{center}
\begin{tabular}{l}
|\ifchilddoc|\\
|\addtocounter{page}{-1}|\\
\textit{code for banner page}\\
|\newpage|\\
|\||fi|
\end{tabular}
\end{center}
%
Here one could write a message such as:
\begin{center}
|This is the part \childdocname{} of \childdocjob{}.|
\end{center}

%%%%%%%%%%%%%%%%%%%%%%%%%%%%%%%%%%%%%%%%%%%%%%%%%%%%%%%%%%%%%%%%%%%%%%%%%%%%%%%%
\subsection{Flags}
\label{sec:flags}

The package makes it easy to generate different versions
of the main or child documents.
To this end compilation flags can be defined
and assigned different default values.
They will be particularly useful in conjunction
with the forwarding mechanism described in \secref{sec:forward}.

For example, it may be useful to have a flag |\version|
which can be set to |draft| or |final|.
The document source will contain some conditional code
depending on the value of |\version|.
Suppose further, the flag should default to |final| for the main file
and to |draft| for child files
which is a natural assignment for editing the document.
This is achieved by placing the following code
in the preamble of the main document
(below the |\childdocmain| directive):
%
\begin{center}
\begin{tabular}{l}
|\ifchilddoc|\\
|\providecommand{\version}{draft}|\\
|\||else|\\
|\providecommand{\version}{final}|\\
|\||fi|
\end{tabular}
\end{center}
%
The definition by |\providecommand| makes sure
that previous definitions are not overwritten.
Further statements |\providecommand{\version}{...}|
can thus be added before the above code to override it.

For the main file, one might add a line
(between |\childdocmain| and the above block)
%
\begin{center}
|%\ifchilddoc\||else\providecommand{\version}{draft}\||fi|
\end{center}
%
which can be uncommented to produce a draft version.
Likewise one can add a line to the very top of a child file
(above the |\childdocof{|\textit{main}|}| directive)
%
\begin{center}
|%\providecommand{\version}{final}|
\end{center}
%
which can be uncommented to produce the final version of this child document.

%%%%%%%%%%%%%%%%%%%%%%%%%%%%%%%%%%%%%%%%%%%%%%%%%%%%%%%%%%%%%%%%%%%%%%%%%%%%%%%%
\subsection{Forwarding}
\label{sec:forward}

Different versions of the main or child documents
using compilation flags as described in \secref{sec:flags}
can be (permanently) stored in different files
for convenient compilation, viewing and distribution.
To this end, the package defines a command
to pass on compilation to a different file:

%%%%%%%%%%%%%%%%%%%%%%%%%%%%%%%%%%%%%%%%
\DescribeMacro{\childdocforward}
The command |\childdocforward| redirects processing to
another source file:
%
\begin{center}
\begin{tabular}{l}
|\input{childdoc.def}|\\
|\childdocforward[|\textit{main}|]{|\textit{dest}|}|\\
\end{tabular}
\end{center}
%
The argument \textit{dest} is the destination file
(without extension).
It should be the main file or one of the child files.
Note that further \textsf{childdoc} directives
such as |\childdocof| and |\childdocforward|
in the indicated file will be processed in this form.
The optional argument \textit{main}
passes on directly to the main file \textit{main}
while pretending to compile the child \textit{dest}.
This form behaves as if \textit{dest}
issues |\childdocof{|\textit{main}|}| right away,
and no further \textsf{childdoc} directives will be processed.

%%%%%%%%%%%%%%%%%%%%%%%%%%%%%%%%%%%%%%%%
\DescribeMacro{\...prefix}
In the alternative form |\childdocforwardprefix|,
%
\begin{center}
\begin{tabular}{l}
|\input{childdoc.def}|\\
|\childdocforwardprefix[|\textit{main}|]{|\textit{prefix}|}{|\textit{dest}|}|
\end{tabular}
\end{center}
%
the destination file is determined by a pattern
depending on the current file:
To make this work, the current file must be called
`{\textit{prefix}\hspace{0.2em}\textit{suffix}}'
with \textit{prefix} matching precisely the argument.
Processing is then passed on to the file
`{\textit{dest}\hspace{0.2em}\textit{suffix}}'.
Surely, the same effect is achieved by
directly specifying the
argument `{\textit{dest}\hspace{0.2em}\textit{suffix}}'
in the first form.
However, that requires to set up a different file
for each child. With the alternative form of the command
all these files can have exactly the same content
which simplifies setting them up and maintaining them.

For example, the following file |draft.tex|
with a compilation flag |\version| as described in \secref{sec:flags}
compiles the main document as a draft:
%
\begin{center}
\begin{tabular}{l}
|\def\version{draft}|\\
|\input{childdoc.def}|\\
|\childdocforward{|\textit{main}|}|
\end{tabular}
\end{center}
%
Likewise, the following files |final|\textit{nn}|.tex|
compile the final version of the child document
|child|\textit{nn}|.tex|:
%
\begin{center}
\begin{tabular}{l}
|\def\version{final}|\\
|\input{childdoc.def}|\\
|\childdocforwardprefix{final}{child}|
\end{tabular}
\end{center}
%

Note that when several versions of a main file and/or of each child file
are to be generated, it may be convenient to set up a |Makefile| or
shell script to automatise the process.

%%%%%%%%%%%%%%%%%%%%%%%%%%%%%%%%%%%%%%%%%%%%%%%%%%%%%%%%%%%%%%%%%%%%%%%%%%%%%%%%
\subsection{Command Line Processing}
\label{sec:commandline}

The effect of redirection files can also be achieved by invoking
the \LaTeX{} compiler with a more elaborate command line.
Most conveniently this should be done as part
of a shell script or a |Makefile|.

When using \textsf{childdoc} in the main file, the following
command lines effectively perform a redirection
(note that depending on the shell being used,
backslashes may have to be doubled: `|\|' $\to$ `|\\|'):
%
\begin{center}
|... -jobname "|\textit{target}|" |\\|"|[\textit{flags}]%
|\input{childdoc.def}\childdocforward[|\textit{main}|]{|\textit{dest}|}"|
\end{center}
%
Here \textit{target} is the name of the output file,
\textit{main} is the name of the main file
and \textit{dest} is the name of the main or child file to be processed
(all filenames without extensions).
The optional argument \textit{main} can be omitted
if \textit{main} matches \textit{dest}.
Optionally, compilation \textit{flags} can be defined via |\def| commands.
This command line makes the \TeX{} engine believe
it is compiling the file \textit{target}
whose content is specified as the latter parameter.
The provided code then forwards the processing to
\textit{main} or \textit{dest} as described in \secref{sec:forward}.

%%%%%%%%%%%%%%%%%%%%%%%%%%%%%%%%%%%%%%%%%%%%%%%%%%%%%%%%%%%%%%%%%%%%%%%%%%%%%%%%
\subsection{Include by Input}
\label{sec:input}

Including child documents by |\include| has some restrictions by design.
Most notably, the content of a child document always occupies
its own set of pages; pages cannot be shared between child documents.
Usually, this behaviour makes perfect sense
because each child document contain an essential part of the document.
However, in some situations it may be desirable to compose
a document from a collection of parts
without having mandatory page breaks between then.
For this case, the package
provides a mechanism to include parts
by |\input| which can also be processed individually.
However, by construction this mechanism
requires manual handling of the content to be output.

%%%%%%%%%%%%%%%%%%%%%%%%%%%%%%%%%%%%%%%%
\DescribeMacro{\ifchilddocmanual}
The main file should be prepared as usual, see \secref{sec:include}.
However, the document body must make a distinction
between processing of an individual part and of the main document, e.g.:
%
\begin{center}
\begin{tabular}{l}
|\ifchilddocmanual|\\
|\input{\childdocname}|\\
|\||else|\\
\textit{document body with }|\input{|\textit{part}|}|\\
|\||fi|
\end{tabular}
\end{center}
%
The conditional |\ifchilddocmanual| is true whenever
a part to be included by |\input| is being compiled,
and the name of the part is stored in |\childdocname|.

%%%%%%%%%%%%%%%%%%%%%%%%%%%%%%%%%%%%%%%%
\DescribeMacro{\childdocby}
Each part to be included by |\input| should start with:
%
\begin{center}
\begin{tabular}{l}
|\input{childdoc.def}|\\
|\childdocby{|\textit{main}|}|\\
\end{tabular}
\end{center}
%
The directive |\childdocby| is similar to |\childdocof|
described in \secref{sec:include},
but the subsequent selection of content must be done manually.
To that end, both |\ifchilddoc| and |\ifchilddocmanual|
will be true upon processing of a part,
and the name of the part is stored in |\childdocname|.
Note that |\jobname| will be set to the filename of the current part
so that each part receives an individual |.aux| file
that does not interfere with the |.aux| file(s) of the main document.
This behaviour can be altered by the alternative form
|\childdocby[*]{|\textit{main}|}| (with a non-empty optional argument)
which uses the |.aux| file of the main document
by setting |\jobname| to \textit{main}.

%%%%%%%%%%%%%%%%%%%%%%%%%%%%%%%%%%%%%%%%%%%%%%%%%%%%%%%%%%%%%%%%%%%%%%%%%%%%%%%%
\subsection{Driver Development}
\label{sec:driver}

The \textsf{childdoc} mechanism can also be use for the development
of definition files such as \LaTeX{} styles or classes.
This case differs from the above setup with multiple parts
included by |\include| in that no |\includeonly| should be invoked.
This can be achieved by starting the include file
(before |\ProvidesPackage|) with:
%
\begin{center}
\begin{tabular}{l}
|\input{childdoc.def}|\\
|\childdocforward{|\textit{main}|}|\\
\end{tabular}
\end{center}
%
or alternatively with:
%
\begin{center}
\begin{tabular}{l}
|\input{childdoc.def}|\\
|\childdocby{|\textit{main}|}|\\
\end{tabular}
\end{center}
%
Both forms have slightly different effects as described above.
The main file is prepared as usual, see \secref{sec:include}.

%%%%%%%%%%%%%%%%%%%%%%%%%%%%%%%%%%%%%%%%%%%%%%%%%%%%%%%%%%%%%%%%%%%%%%%%%%%%%%%%
\subsection{Legacy Detection}
\label{sec:detection}

The directive |\childdocmain| in the main file can detect
whether the complete document or merely a child is to be compiled
even without using the directive |\childdocof|.
This method is deprecated because it is less robust
and there is no compelling reason to use it;
it is merely provided for backward compatibility
and it may be removed in future versions.

If the detection mechanism is to be used,
it is mandatory to correctly specify
the filename of the main file as the argument of |\childdocmain|:
%
\begin{center}
\begin{tabular}{l}
|\input{childdoc.def}|\\
|\childdocmain{|\textit{main}|}|\\
\end{tabular}
\end{center}
%
If |\jobname| does not match the argument \textit{main} of |\childdocmain|,
it is assumed that |\jobname| points to the child file to be compiled.
When using |\childdocmain| with the main file specified as argument,
it suffices to start a child file
with just |\input{|\textit{main}|}|
without loading of the package and using |\childdocof|.
If instead all processing is done
with the appropriate \textsf{childdoc} directives,
the argument of \textit{main} of |\childdocmain| can be empty.

An alternative version of the command line processing described
in \secref{sec:commandline} using the detection mechanism reads:
%
\begin{center}
|... -jobname "|\textit{target}|" "|[\textit{flags}]%
[|\def\jobname{|\textit{dest}|}|]|\input{|\textit{main}|}"|
\end{center}

%%%%%%%%%%%%%%%%%%%%%%%%%%%%%%%%%%%%%%%%%%%%%%%%%%%%%%%%%%%%%%%%%%%%%%%%%%%%%%%%
\subsection{Manual Code}
\label{sec:manual}

In case one cannot be certain whether the definitions file |childdoc.def|
is installed on the target \TeX{} distribution
and one prefers not to ship it,
it is conceivable to paste a few relevant commands into the sources.

To that end, drop all statements |\input{childdoc.def}|
and perform the replacements as outlined below.
Instead of |\childdocmain{|\textit{main}|}| add the following code
to the top of the main file:
%
\begin{center}
\begin{tabular}{l}
|\||ifdefined\childdocname\endinput\||fi\newif\ifchilddoc|\\
|\edef\childdocname{\scantokens\expandafter{\jobname\noexpand}}|\\
|\def\childdocmain{|\textit{main}|}\||ifx\childdocmain\childdocname\||else|\\
|\childdoctrue\includeonly{\childdocname}\let\jobname\childdocmain\||fi|\\
\end{tabular}
\end{center}
%
Instead of |\childdocof{|\textit{main}|}| just include the main file
at the top of each child file:
%
\begin{center}
|\input{|\textit{main}|}|
\end{center}
%
A simple redirection |\childdocforward{|\textit{dest}|}| is achieved by:
%
\begin{center}
|\def\jobname{|\textit{dest}|}\input{\jobname}|
\end{center}
%
The redirection with prefix
|\childdocforwardprefix[|\textit{prefix}|]{|\textit{dest}|}|
is accomplished by:
%
\begin{center}
\begin{tabular}{l}
|{\edef\jobname{\scantokens\expandafter{\jobname\noexpand}}|\\
|\def\redirectjob |\textit{prefix}|#1~~~{\gdef\jobname{|\textit{dest}|#1}}|\\
|\expandafter\redirectjob\jobname~~~}\input{\jobname}|
\end{tabular}
\end{center}

In an alternative approach,
child documents can be compiled by a specific command line
without additional code or specific definitions:
%
\begin{center}
|... -jobname "|\textit{target}|" "|[\textit{flags}]%
|\includeonly{|\textit{dest}|}\input{|\textit{main}|}"|
\end{center}
%

%%%%%%%%%%%%%%%%%%%%%%%%%%%%%%%%%%%%%%%%%%%%%%%%%%%%%%%%%%%%%%%%%%%%%%%%%%%%%%%%
%%%%%%%%%%%%%%%%%%%%%%%%%%%%%%%%%%%%%%%%%%%%%%%%%%%%%%%%%%%%%%%%%%%%%%%%%%%%%%%%
\section{Information}

%%%%%%%%%%%%%%%%%%%%%%%%%%%%%%%%%%%%%%%%%%%%%%%%%%%%%%%%%%%%%%%%%%%%%%%%%%%%%%%%
\subsection{Copyright}

Copyright \copyright{} 2017--2018 Niklas Beisert

This work may be distributed and/or modified under the
conditions of the \LaTeX{} Project Public License, either version 1.3
of this license or (at your option) any later version.
The latest version of this license is in
  \url{http://www.latex-project.org/lppl.txt}
and version 1.3 or later is part of all distributions of \LaTeX{}
version 2005/12/01 or later.

This work has the LPPL maintenance status `maintained'.

The Current Maintainer of this work is Niklas Beisert.

This work consists of the files |README.txt|, |childdoc.ins| and |childdoc.dtx|
as well as the derived files |childdoc.def|, |cdocsamp.tex|
with |cdocsch1.tex|, |cdocsch2.tex|, |cdocspt3.tex|, |cdocspt4.tex|,
|cdocsdrf.tex|, |cdocsfn1.tex|, |cdocsfn2.tex|
as well as |childdoc.pdf|.

%%%%%%%%%%%%%%%%%%%%%%%%%%%%%%%%%%%%%%%%%%%%%%%%%%%%%%%%%%%%%%%%%%%%%%%%%%%%%%%%
\subsection{Files and Installation}

The package consists of the files:
%
\begin{center}
\begin{tabular}{ll}
    |README.txt|   & readme file \\
    |childdoc.ins| & installation file \\
    |childdoc.dtx| & source file \\
    |childdoc.def| & definition file \\
    |cdocsamp.tex| & sample main file \\
    |cdocsch1.tex| & sample include file \\
    |cdocsch2.tex| & sample include file \\
    |cdocspt3.tex| & sample part file \\
    |cdocspt4.tex| & sample part file \\
    |cdocsdrf.tex| & sample redirection file \\
    |cdocsfn1.tex| & sample redirection file \\
    |cdocsfn2.tex| & sample redirection file \\
    |childdoc.pdf| & manual
\end{tabular}
\end{center}
%
The distribution consists of the files
|README.txt|, |childdoc.ins| and |childdoc.dtx|.
%
\begin{itemize}
\item
Run (pdf)\LaTeX{} on |childdoc.dtx|
to compile the manual |childdoc.pdf| (this file).
\item
Run \LaTeX{} on |childdoc.ins| to create the definitions file |childdoc.def|
and the sample |cdocsamp.tex| with include files
|cdocsch1.tex|, |cdocsch2.tex|, |cdocspt3.tex|, |cdocspt4.tex|,
|cdocsdrf.tex|, |cdocsfn1.tex|, |cdocsfn2.tex|.
Then copy the file |childdoc.def| to an appropriate directory of your \LaTeX{}
distribution, e.g.\ \textit{texmf-root}|/tex/latex/childdoc|.
\end{itemize}

%%%%%%%%%%%%%%%%%%%%%%%%%%%%%%%%%%%%%%%%%%%%%%%%%%%%%%%%%%%%%%%%%%%%%%%%%%%%%%%%
\subsection{Related CTAN Packages}

There are several other packages which offer a similar functionality:
%
\begin{itemize}
\item
The packages
\href{http://ctan.org/pkg/docmute}{\textsf{docmute}},
\href{http://ctan.org/pkg/includex}{\textsf{includex}} and
\href{http://ctan.org/pkg/standalone}{\textsf{standalone}}
provide commands to include only the document body of
a child file thus allowing both files to be compiled individually.
\item
The packages \href{http://ctan.org/pkg/subdocs}{\textsf{subdocs}}
and \href{http://ctan.org/pkg/subfiles}{\textsf{subfiles}}
provide structures in which the main and child documents can be
encapsulated and allowing them to be compiled individually.
The inclusion mechanism is different from the conventional |\include|.
\item
The package \href{http://ctan.org/pkg/combine}{\textsf{combine}}
is an elaborate solution to combine several documents into one.
\end{itemize}
%
See also the CTAN topic \href{http://ctan.org/topic/subdocs}{\textsf{subdocs}}
for further related packages.
The present package differs from the above solutions in that
a document structure constructed with the conventional |\include| mechanism
just needs two extra commands at the top of every file
such that all constituent files can be compiled individually.

%%%%%%%%%%%%%%%%%%%%%%%%%%%%%%%%%%%%%%%%%%%%%%%%%%%%%%%%%%%%%%%%%%%%%%%%%%%%%%%%
%\subsection{Feature Suggestions}
%
%The following is a list of features which may be useful for future
%versions of this package:
%%
%\begin{itemize}
%\item
%\ldots
%\end{itemize}

%%%%%%%%%%%%%%%%%%%%%%%%%%%%%%%%%%%%%%%%%%%%%%%%%%%%%%%%%%%%%%%%%%%%%%%%%%%%%%%%
\subsection{Revision History}

%%%%%%%%%%%%%%%%%%%%%%%%%%%%%%%%%%%%%%%%
\paragraph{v2.0:} 2018/12/30

\begin{itemize}
\item
immediate forward processing
\item
added |\childdocby| mechanism
\item
manual restructured
\end{itemize}

%%%%%%%%%%%%%%%%%%%%%%%%%%%%%%%%%%%%%%%%
\paragraph{v1.6:} 2018/01/17

\begin{itemize}
\item
application for development of include files
\item
corrections to manual
\end{itemize}

%%%%%%%%%%%%%%%%%%%%%%%%%%%%%%%%%%%%%%%%
\paragraph{v1.5:} 2017/05/21

\begin{itemize}
\item
more complete structuring introduced
\item
|\childdocof| introduced
\item
|\childdoc| renamed to |\childdocmain|
\item
|\childredirect| renamed to |\childdocforward| and |\childdocforwardprefix|
and functionality expanded
\end{itemize}

%%%%%%%%%%%%%%%%%%%%%%%%%%%%%%%%%%%%%%%%
\paragraph{v1.0:} 2017/04/27

\begin{itemize}
\item
manual and install package
\item
first version published on CTAN
\end{itemize}

%%%%%%%%%%%%%%%%%%%%%%%%%%%%%%%%%%%%%%%%
\paragraph{v0.6:} 2017/04/26

\begin{itemize}
\item
redirection mechanism added
\end{itemize}

%%%%%%%%%%%%%%%%%%%%%%%%%%%%%%%%%%%%%%%%
\paragraph{v0.5:} 2017/04/26

\begin{itemize}
\item
functionality in definition file
\end{itemize}


%%%%%%%%%%%%%%%%%%%%%%%%%%%%%%%%%%%%%%%%%%%%%%%%%%%%%%%%%%%%%%%%%%%%%%%%%%%%%%%%
%%%%%%%%%%%%%%%%%%%%%%%%%%%%%%%%%%%%%%%%%%%%%%%%%%%%%%%%%%%%%%%%%%%%%%%%%%%%%%%%
%%%%%%%%%%%%%%%%%%%%%%%%%%%%%%%%%%%%%%%%%%%%%%%%%%%%%%%%%%%%%%%%%%%%%%%%%%%%%%%%
\appendix

\settowidth\MacroIndent{\rmfamily\scriptsize 000\ }

 \DocInput{childdoc.dtx}

\end{document}
%</driver>
% \fi
%
% %%%%%%%%%%%%%%%%%%%%%%%%%%%%%%%%%%%%%%%%%%%%%%%%%%%%%%%%%%%%%%%%%%%%%%%%%%%%%%
% %%%%%%%%%%%%%%%%%%%%%%%%%%%%%%%%%%%%%%%%%%%%%%%%%%%%%%%%%%%%%%%%%%%%%%%%%%%%%%
% \section{Sample}
%\iffalse
%<*samplemain>
%\fi
%
% The following presents a sample document
% with two chapters, two parts, a title page,
% a compile flag as well as three forwarding files to set the flag.
% It consists of eight |.tex| files:
% \begin{center}
% \begin{tabular}{ll}
% |cdocsamp.tex|&main file\\
% |cdocsch1.tex|&include file for chapter 1\\
% |cdocsch2.tex|&include file for chapter 2\\
% |cdocspt3.tex|&include file for part 3\\
% |cdocspt4.tex|&include file for part 4\\
% |cdocsdrf.tex|&forwarding file for main file in draft mode\\
% |cdocsfi1.tex|&forwarding file for final version of chapter 1\\
% |cdocsfi2.tex|&forwarding file for final version of chapter 2\\
% \end{tabular}
% \end{center}
% Each of the eight files can be compiled directly by the \LaTeX{} compiler.
%
% %%%%%%%%%%%%%%%%%%%%%%%%%%%%%%%%%%%%%%
% \paragraph{Main File.}
%
% The main file is called |cdocsamp.tex|.
%
% Load the \textsf{childdoc} definitions and
% declare the filename for the main document:
%    \begin{macrocode}
\input{childdoc.def}
\childdocmain{}
%    \end{macrocode}

% Optional override for |\version| flag:
%    \begin{macrocode}
%%\ifchilddoc\else\providecommand{\version}{draft}\fi
%    \end{macrocode}

% Define the default values for the |\version| flag
% (|final| for the main file and |draft| for childs):
%    \begin{macrocode}
\ifchilddoc
\providecommand{\version}{draft}
\else
\providecommand{\version}{final}
\fi
%    \end{macrocode}

% Load the standard document class:
%    \begin{macrocode}
\documentclass[12pt]{article}
%    \end{macrocode}

% Start the document body:
%    \begin{macrocode}
\begin{document}
%    \end{macrocode}

% Declare a title page.
% Print title, part of document being processed and version flag:
%    \begin{macrocode}
\addtocounter{page}{-1}
\begin{center}
{\LARGE\bfseries{}childdoc example\par}
\vspace{1cm}
\ifchilddoc
\ifchilddocmanual part\else chapter\fi:
`\childdocname' of `\childdocjob'\par
\else
main document: `\childdocjob'\par
\fi
version: \version\par
\end{center}
\newpage
%    \end{macrocode}

% Manually include selected file,
% otherwise process as usual:
%    \begin{macrocode}
\ifchilddocmanual
\section*{part `\childdocname'}
\input{\childdocname}
\else
%    \end{macrocode}

% Include the two chapters:
%    \begin{macrocode}
\include{cdocsch1}
\include{cdocsch2}
%    \end{macrocode}

% Include the two parts unless only chapters should be displayed:
%    \begin{macrocode}
\ifchilddoc\else
\section{part three}
\input{cdocspt3}
\section{part four}
\input{cdocspt4}
\fi
%    \end{macrocode}

% Process as usual until here:
%    \begin{macrocode}
\fi
%    \end{macrocode}

% End of document body:
%    \begin{macrocode}
\end{document}
%    \end{macrocode}
%\iffalse
%</samplemain>
%\fi
%
% %%%%%%%%%%%%%%%%%%%%%%%%%%%%%%%%%%%%%%
% \paragraph{Chapter Include Files.}
%
% The include files are called |cdocsch1.tex| and |cdocsch2.tex|.
%
%\iffalse
%<*samplechap1|samplechap2>
%\fi

% Optional override for |\version| flag:
%    \begin{macrocode}
%%\providecommand{\version}{final}
%    \end{macrocode}

% Include the main document:
%    \begin{macrocode}
\input{childdoc.def}
\childdocof{cdocsamp}
%    \end{macrocode}

%\iffalse
%</samplechap1|samplechap2>
%\fi
%
%\iffalse
%<*samplechap1>
%\fi
% Some text for chapter 1:
%    \begin{macrocode}
\section{one}
some text in chapter one
%    \end{macrocode}

%\iffalse
%</samplechap1>
%\fi
% Some text for chapter 2:
%\iffalse
%<*samplechap2>
%\fi
%    \begin{macrocode}
\section{two}
more text in chapter two
%    \end{macrocode}

%\iffalse
%</samplechap2>
%\fi
%
% %%%%%%%%%%%%%%%%%%%%%%%%%%%%%%%%%%%%%%
% \paragraph{Part Include Files.}
%
% The include files are called |cdocspt3.tex| and |cdocspt4.tex|.
%
%\iffalse
%<*samplepart3|samplepart4>
%\fi

% Optional override for |\version| flag:
%    \begin{macrocode}
%%\providecommand{\version}{final}
%    \end{macrocode}

% Include the main document:
%    \begin{macrocode}
\input{childdoc.def}
\childdocby{cdocsamp}
%    \end{macrocode}

%\iffalse
%</samplepart3|samplepart4>
%\fi
%
%\iffalse
%<*samplepart3>
%\fi
% Some text for part 3:
%    \begin{macrocode}
some text in part three
%    \end{macrocode}

%\iffalse
%</samplepart3>
%\fi
% Some text for part 4:
%\iffalse
%<*samplepart4>
%\fi
%    \begin{macrocode}
more text in part four
%    \end{macrocode}

%\iffalse
%</samplepart4>
%\fi
%
% %%%%%%%%%%%%%%%%%%%%%%%%%%%%%%%%%%%%%%
% \paragraph{Forwarding for a Complete Draft.}
%
% The following forwarding file |cdocsdrf.tex|
% compiles the main document in draft mode:
%\iffalse
%<*sampledraft>
%\fi
%    \begin{macrocode}
\def\version{draft}
\input{childdoc.def}
\childdocforward{cdocsamp}
%    \end{macrocode}

%\iffalse
%</sampledraft>
%\fi
%
% %%%%%%%%%%%%%%%%%%%%%%%%%%%%%%%%%%%%%%
% \paragraph{Forwarding for Final Version of the Chapters.}
%
% The following forwarding files |cdocsfn1.tex| and |cdocsfn2.tex|
% (with identical content)
% compile the final versions of the child documents
% |cdocsch1.tex| and |cdocsch2.tex|, respectively:
%\iffalse
%<*samplefinal>
%\fi
%    \begin{macrocode}
\def\version{final}
\input{childdoc.def}
\childdocforwardprefix[cdocsamp]{cdocsfn}{cdocsch}
%    \end{macrocode}

%\iffalse
%</samplefinal>
%\fi
%
% %%%%%%%%%%%%%%%%%%%%%%%%%%%%%%%%%%%%%%
% \paragraph{Command Line Processing.}
%
% The following three command lines generate the output files
% |cdocscld|, |cdocscl1| and |cdocscl2|
% which should be identical to
% |cdocsdrf|, |cdocsch1| and |cdocsfn2|, respectively:
% \begin{center}
% \begin{tabular}{l}
% |latex -jobname cdocscld \|\\
% |  "\def\version{draft}\input{childdoc.def}\childdocforward{cdocsamp}"|\\
% |latex -jobname cdocscl1 \|\\
% |  "\input{childdoc.def}\childdocforward[cdocsamp]{cdocsch1}"|\\
% |latex -jobname cdocscl2 \|\\
% |  "\def\version{final}\input{childdoc.def}\childdocforward{cdocsch2}"|
% \end{tabular}
% \end{center}
% Note that the trailing backslash on each first line
% merely continues the input to the second line
% (for convenient cut ant paste).
% Furthermore, the command |latex| can be replaced by any
% of its alternative versions such as |pdflatex|.
%
% %%%%%%%%%%%%%%%%%%%%%%%%%%%%%%%%%%%%%%%%%%%%%%%%%%%%%%%%%%%%%%%%%%%%%%%%%%%%%%
% %%%%%%%%%%%%%%%%%%%%%%%%%%%%%%%%%%%%%%%%%%%%%%%%%%%%%%%%%%%%%%%%%%%%%%%%%%%%%%
% \section{Implementation}
%\iffalse
%<*package>
%\fi
%
% This section describes the definitions file |childdoc.def|.

% The definitions cannot be loaded using |\usepackage| or |\RequirePackage|
% which has a mechanism to prevent loading a style file more than once.
% When loading the definitions by means of |\input|
% multiple instances have to be prevented manually:
%\iffalse
%This code needs to be before the `\ProvidesFile' directive
%which is defined at the beginning of this file.
%Therefore it is also placed there and commented out here.
%</package>
%<*discard>
%\fi
%    \begin{macrocode}
\ifdefined\childdocmain\endinput\fi
%    \end{macrocode}
%\iffalse
%</discard>
%<*package>
%\fi
%
% \macro{\ifchilddoc}
% \macro{\ifchilddocmanual}
% The conditional |\ifchilddoc| tells whether a
% child (true) or main (false) document is being compiled.
% The conditional |\ifchilddocmanual| tells whether
% the |\includeonly| mechanism is used (false) or
% the selection of child files must be performed manually (true).
% The definitions initialise to false:
%    \begin{macrocode}
\newif\ifchilddoc
\newif\ifchilddocmanual
%    \end{macrocode}

% \macro{\childdocname}
% \macro{\childdocjob}
% The macro |\childdocname| stores the name of the main document
% to be compiled. The macro |\childdocjob| stores the name of
% the document on which the \LaTeX{} compiler was originally invoked.
% The content of |\jobname| cannot be compared
% to filenames specified in the source due to different catcodes.
% The following code rescans |\jobname|, stores the result
% in |\childdocname| and saves a copy in |\childdocjob|:
%    \begin{macrocode}
\edef\childdocname{\scantokens\expandafter{\jobname\noexpand}}
\let\childdocjob\childdocname
%    \end{macrocode}

% \macro{\childdocdisable}
% The macro |\childdocdisable| prevents the main file
% from being processed more than once.
% At this stage, the main document command |\childdocmain|
% is assumed to be called once again where it should do nothing.
% Any subsequent call to it should prevent
% a secondary processing of the main document
% It overwrites the forwarding commands
% |\childdocof| and |\childdocforward|
% with empty macros to prevent further inclusions of the main document:
%    \begin{macrocode}
\newcommand{\childdocdisable}
{
  \renewcommand{\childdocmain}[1]{\renewcommand{\childdocmain}[1]{\endinput}}
  \renewcommand{\childdocof}[1]{}
  \renewcommand{\childdocby}[2][]{}
  \renewcommand{\childdocforward}[2][]{}
  \renewcommand{\childdocdisable}{}
}
%    \end{macrocode}

% \macro{\childdocmain}
% The macro |\childdocmain| is to be called at the top of the main file
% with nothing or the main filename (without extension) as argument.
% First, it breaks loops.
% If the argument is not empty and does not match |\childdocname|
% (which is set by the first inclusion of |childdoc.def|),
% |\ifchilddoc| is set to true, |\includeonly| is applied to the child file
% and |\jobname| is set to the main file
% (for proper handling of |.aux| files):
%    \begin{macrocode}
\newcommand{\childdocmain}[1]
{
  \childdocdisable\childdocmain{}
  \if?#1?\else
    \begingroup
      \def\childdoctmp{#1}
      \ifx\childdoctmp\childdocname
        \def\childdoctmp{}
      \else
        \def\childdoctmp
        {
          \childdoctrue
          \includeonly{\childdocname}
          \def\childdocjob{#1}
          \def\jobname{#1}
        }
      \fi
      \expandafter
    \endgroup
    \childdoctmp
  \fi
}
%    \end{macrocode}

% \macro{\childdocof}
% The command |\childdocof| redirects
% compilation to the main file |#1|.
%    \begin{macrocode}
\newcommand{\childdocof}[1]
{
  \childdocdisable
  \childdoctrue
  \includeonly{\childdocname}
  \def\jobname{#1}
  \def\childdocjob{#1}
  \input{#1}
}
%    \end{macrocode}

% \macro{\childdocby}
% The command |\childdocby| ....
%    \begin{macrocode}
\newcommand{\childdocby}[2][]
{
  \childdocdisable
  \childdoctrue
  \childdocmanualtrue
  \if?#1?\else
    \def\jobname{#2}
  \fi
  \def\childdocjob{#2}
  \input{#2}
  \endinput
}
%    \end{macrocode}

% \macro{\childdocforward}
% The command |\childdocforward| redirects
% compilation to the main file or
% (if the optional argument is given) a child file.
% Parameters are set as if the main file
% or a child file starting with |\childdocof| was compiled.
% Then compilation is handed over to the main file:
%    \begin{macrocode}
\newcommand{\childdocforward}[2][]
{
  \begingroup
    \if?#1?
      \def\childdoctmp
      {
        \def\childdocname{#2}
        \def\childdocjob{#2}
        \def\jobname{#2}
        \input{#2}
        \endinput
      }
    \else
      \def\childdoctmp
      {
        \childdocdisable
        \def\childdocname{#2}
        \childdoctrue
        \includeonly{#2}
        \def\childdocjob{#1}
        \def\jobname{#1}
        \input{#1}
        \endinput
      }
    \fi
    \expandafter
  \endgroup
  \childdoctmp
}
%    \end{macrocode}

% \macro{\childdocforwardprefix}
% The command |\childdocforwardprefix| redirects
% compilation to the main or a child file by means of a pattern.
% The prefix |#1| in the current filename is replaced by |#2|
% and the suffix of the current filename is kept
% (it is assumed that the filename does not contain the substring `|~~~|'
% which is used as a delimiter).
% Compilation is handed over to the new file by |\childdocforward|:
%    \begin{macrocode}
\newcommand{\childdocforwardprefix}[3][]
{
  \begingroup
    \def\childdocextract #2##1~~~{\def\childdoctmp{\childdocforward[#1]{#3##1}}}
    \expandafter\childdocextract\childdocname~~~
    \expandafter
  \endgroup
  \childdoctmp
}
%    \end{macrocode}

% \macro{\childdoc}
% The deprecated macro |\childdoc| is a legacy version of |\childdocmain|:
%    \begin{macrocode}
\newcommand{\childdoc}{\childdocmain}
%    \end{macrocode}

% \macro{\childdocredirect}
% The deprecated macro |\childdocredirect| is a legacy version
% of |\childdocforward| and |\childdocforwardprefix|:
%    \begin{macrocode}
\newcommand{\childdocredirect}[2][]
{
  \begingroup
    \if?#1?
      \def\childdoctmp{\childdocforward{#2}}
    \else
      \def\childdoctmp{\childdocforwardprefix{#1}{#2}}
    \fi
    \expandafter
  \endgroup
  \childdoctmp
}
%    \end{macrocode}

%\iffalse
%</package>
%\fi
%
\endinput
|\\
|\childdocforward{|\textit{main}|}|\\
\end{tabular}
\end{center}
%
or alternatively with:
%
\begin{center}
\begin{tabular}{l}
|% \iffalse
%
% childdoc.dtx Copyright (C) 2017-2018 Niklas Beisert
%
% This work may be distributed and/or modified under the
% conditions of the LaTeX Project Public License, either version 1.3
% of this license or (at your option) any later version.
% The latest version of this license is in
%   http://www.latex-project.org/lppl.txt
% and version 1.3 or later is part of all distributions of LaTeX
% version 2005/12/01 or later.
%
% This work has the LPPL maintenance status `maintained'.
%
% The Current Maintainer of this work is Niklas Beisert.
%
% This work consists of the files childdoc.dtx and childdoc.ins
% and the derived files childdoc.def and cdocsamp.tex with
% cdocsch1.tex, cdocsch2.tex, cdocsdrf.tex, cdocsfn1.tex, cdocsfn2.tex.
%
%<package>\ifdefined\childdocmain\endinput\fi
%<package>\ProvidesFile{childdoc.def}[2018/12/30 v2.0 child document driver]
%<samplemain>\ProvidesFile{cdocsamp.tex}[2018/12/30 v2.0 sample for childdoc]
%<*driver>
%\ProvidesFile{childdoc.drv}[2018/12/30 v2.0 childdoc reference manual file]
\PassOptionsToClass{10pt,a4paper}{article}
\documentclass{ltxdoc}

\usepackage[margin=35mm]{geometry}
\usepackage{hyperref}
\usepackage{hyperxmp}
\usepackage[usenames]{color}

\hypersetup{colorlinks=true}
\hypersetup{pdfstartview=FitH}
\hypersetup{pdfpagemode=UseNone}
\hypersetup{pdfsource={}}
\hypersetup{pdflang={en-UK}}
\hypersetup{pdfcopyright={Copyright 2017-2018 Niklas Beisert.
  This work may be distributed and/or modified under the
  conditions of the LaTeX Project Public License, either version 1.3
  of this license or (at your option) any later version.}}
\hypersetup{pdflicenseurl={http://www.latex-project.org/lppl.txt}}
\hypersetup{pdfcontactaddress={ETH Zurich, ITP, HIT K,
  Wolfgang-Pauli-Strasse 27}}
\hypersetup{pdfcontactpostcode={8093}}
\hypersetup{pdfcontactcity={Zurich}}
\hypersetup{pdfcontactcountry={Switzerland}}
\hypersetup{pdfcontactemail={nbeisert@itp.phys.ethz.ch}}
\hypersetup{pdfcontacturl={http://people.phys.ethz.ch/\xmptilde nbeisert/}}

\newcommand{\secref}[1]{\hyperref[#1]{section \ref*{#1}}}

\parskip1ex
\parindent0pt
\let\olditemize\itemize
\def\itemize{\olditemize\parskip0pt}

\begin{document}

\title{The \textsf{childdoc} Package}
\hypersetup{pdftitle={The childdoc Package}}
\author{Niklas Beisert\\[2ex]
  Institut f\"ur Theoretische Physik\\
  Eidgen\"ossische Technische Hochschule Z\"urich\\
  Wolfgang-Pauli-Strasse 27, 8093 Z\"urich, Switzerland\\[1ex]
  \href{mailto:nbeisert@itp.phys.ethz.ch}
  {\texttt{nbeisert@itp.phys.ethz.ch}}}
\hypersetup{pdfauthor={Niklas Beisert}}
\hypersetup{pdfsubject={Manual for the LaTeX2e Package childdoc}}
\date{30 December 2018, \textsf{v2.0}}
\maketitle

\begin{abstract}\noindent
\textsf{childdoc} is a \LaTeXe{} package
that enables the direct compilation
of document sections included by |\include|
to individual files.
\end{abstract}

\begingroup
\parskip0ex
\tableofcontents
\endgroup

%%%%%%%%%%%%%%%%%%%%%%%%%%%%%%%%%%%%%%%%%%%%%%%%%%%%%%%%%%%%%%%%%%%%%%%%%%%%%%%%
%%%%%%%%%%%%%%%%%%%%%%%%%%%%%%%%%%%%%%%%%%%%%%%%%%%%%%%%%%%%%%%%%%%%%%%%%%%%%%%%
\section{Introduction}

\LaTeX{} provides a mechanism to structure a large document (such as a book)
into a main file and several child files (containing the chapters)
using the |\include| command.
This mechanism is beneficial for documents
which span hundreds of pages in order to
make the source file(s) more manageable.
Moreover, compilation can be restricted to
selected child files by means of the |\includeonly| command.
The latter feature can be used to reduce the compilation time while editing
(this was significantly more useful in the earlier days of \LaTeX{})
or to generate a smaller document which is easier to navigate.
Another application of |\includeonly| is to generate
documents consisting of selected parts of the complete document.

However, there are a few drawbacks of the plain |\include| mechanism:
\begin{itemize}
\item
The child files cannot be compiled on their own,
they can only be compiled via the main file.
A naive editing environment
(such as a text editor with an option
to have the current file processed by \LaTeX)
may require one to switch to the main file before compiling;
attempting to compile the child file produces errors.
\item
The main file must be modified (each time)
to adjust the |\includeonly| command
to the present needs. This easily leaves the main file in a messy state.
\item
The generated document will always carry the filename
of the main document. This is inconvenient if
several child files are to be compiled and
to be kept for distribution.
\end{itemize}

The present package provides a simple interface
to make child files individually compilable by \LaTeX{}.
Compiling a child file then has the same effect as compiling
the main file with an |\includeonly| command
to select the appropriate child.
Moreover the generated document will carry the name of the child
rather than the main file.
This resolves all three above issues.

This feature is meant to make the editing of books,
thesis documents and lecture notes somewhat more convenient.
However, the package can also be used efficiently for
composing a series of documents (such as exercise sheets)
which are typically distributed individually.
It then assists the author in generating the individual documents
(potentially in different versions)
as well as a document containing the collected series.
Another application is in developing style files
or other kinds of included material
where compilation of the style file could redirect
to a sample or test file.

%%%%%%%%%%%%%%%%%%%%%%%%%%%%%%%%%%%%%%%%%%%%%%%%%%%%%%%%%%%%%%%%%%%%%%%%%%%%%%%%
%%%%%%%%%%%%%%%%%%%%%%%%%%%%%%%%%%%%%%%%%%%%%%%%%%%%%%%%%%%%%%%%%%%%%%%%%%%%%%%%
\section{Usage}

First of all, the package \textsf{childdoc} is \emph{not} a standard
\LaTeXe{} |.sty| style file! Therefore it needs to be invoked in
a non-standard way.

%%%%%%%%%%%%%%%%%%%%%%%%%%%%%%%%%%%%%%%%%%%%%%%%%%%%%%%%%%%%%%%%%%%%%%%%%%%%%%%%
\subsection{Included Files}
\label{sec:include}

%%%%%%%%%%%%%%%%%%%%%%%%%%%%%%%%%%%%%%%%
\DescribeMacro{\childdocmain}
To use the package, add the commands
\begin{center}
\begin{tabular}{l}
|\input{childdoc.def}|\\
|\childdocmain{}|\\
\end{tabular}
\end{center}
at the very top of the main \LaTeX{} file,
in particular \emph{before} the |\documentclass| statement!
The argument of |\childdocmain| should be left empty
(but it must be present).

%%%%%%%%%%%%%%%%%%%%%%%%%%%%%%%%%%%%%%%%
\DescribeMacro{\childdocof}
Furthermore, add the commands
\begin{center}
\begin{tabular}{l}
|\input{childdoc.def}|\\
|\childdocof{|\textit{main}|}|\\
\end{tabular}
\end{center}
at the top of every child file \textit{child}
which is included by |\include{|\textit{child}|}|
from within the main file
(or at least for those files to be compiled individually).
The argument \textit{main} must be the filename of the main file.

There are a couple of
considerations in setting up the main and child documents:

%%%%%%%%%%%%%%%%%%%%%%%%%%%%%%%%%%%%%%%%
\paragraph{Restrictions.}

Please note the following restrictions:
\begin{itemize}
\item
|\childdocmain| must be called with one argument \textit{main}
to ensure compatibility with earlier version of the package.
It must either be empty (|\childdocmain{}|)
or precisely match the filename of the main file in which it is specified.
See \secref{sec:detection} for further information.
\item
The filename \textit{main} must be specified without the |.tex| extension.
\item
The filename \textit{main} is case sensitive
(even in case-insensitive file systems)
due to internal string comparison.
\item
The argument \textit{main} should be fully expanded, it cannot be a macro.
\item
Subdirectories and special characters should be avoided in filenames.
\item
The command |\childdocmain{|\textit{main}|}| must be followed by a whitespace.
It should not be followed immediately by another command
or by a comment mark `|%|'.
This is because the \TeX{} parser reads the token immediately following
the argument of |\childdocmain| and puts it
at the beginning of every child section;
however, a white\-space is ignored.
\end{itemize}

%%%%%%%%%%%%%%%%%%%%%%%%%%%%%%%%%%%%%%%%
\paragraph{Content of Main File.}

It is advisable to place all content in the child files included by |\include|.
Any output contained in the main file will appear in all child documents
unless suppressed manually;
it cannot be suppressed automatically by the |\includeonly| directive
and thus should normally be avoided.
A method to include some content in the main file
by means of conditional processing is described in \secref{sec:conditional}.

%%%%%%%%%%%%%%%%%%%%%%%%%%%%%%%%%%%%%%%%
\paragraph{Page Numbering.}

When only a part of the document is compiled,
the appropriate numbering of pages
(as well as other status parameters)
is determined from the |.aux| files.
The latter contain information from previous passes.
However this information needs to propagate through
all intermediate child documents.
Therefore the page numbering in child documents may well
be inconsistent until the complete document is compiled at least once.

A useful (if unconventional) way to always ensure a consistent
page numbering is to restart the numbering in each child document
and denote the pages by `\textit{child}|.|\textit{page}'
where \textit{child} represents the chapter/section number of the child file.
This can be achieved by the command
|\numberwithin{page}{|\textit{child}|}|
of the \textsf{amsmath} package
where \textit{child} can be |chapter| or |section|
depending on the chosen structuring.
Alternatively, one can modify the macro |\thepage| appropriately
and reset the counter |page| at the start of each child file.

%%%%%%%%%%%%%%%%%%%%%%%%%%%%%%%%%%%%%%%%%%%%%%%%%%%%%%%%%%%%%%%%%%%%%%%%%%%%%%%%
\subsection{Conditional Processing}
\label{sec:conditional}

The package provides a mechanism to compile different versions
of a document. To customise the versions further some conditional processing
can come in handy to distinguish which version is being compiled.
The package provides two macros to describe the compilation context:

%%%%%%%%%%%%%%%%%%%%%%%%%%%%%%%%%%%%%%%%
\DescribeMacro{\ifchilddoc}
The conditional |\ifchilddoc| distinguishes between the compilation of
child documents and the main document:
%
\begin{center}
|\ifchilddoc |\textit{child-code}| |[|\||else |\textit{main-code}]| \||fi|
\end{center}

%%%%%%%%%%%%%%%%%%%%%%%%%%%%%%%%%%%%%%%%
\DescribeMacro{\childdocname}
\DescribeMacro{\childdocjob}
The macro |\childdocname| contains the filename (without extension)
of the main or child file being processed.
Note that |\childdocjob| will always contain the name of the main file.

%%%%%%%%%%%%%%%%%%%%%%%%%%%%%%%%%%%%%%%%
\paragraph{Title Page.}

Conditional processing can be used to include a title or banner page
in the main document when proper precautions are taken.
Importantly, the code in the main file should ensure that the page counter
(as well as other status parameters which are stored in the |.aux| files)
takes the same value after the conditional processing.
Otherwise the page numbers may take divergent values
depending on which part is compiled.

For example, a title page could be declared by:
%
\begin{center}
\begin{tabular}{l}
|\ifchilddoc\||else|\\
|\addtocounter{page}{-1}|\\
\textit{code for title page}\\
|\newpage|\\
|\||fi|
\end{tabular}
\end{center}
%
A banner page for the child documents can be generated by:
%
\begin{center}
\begin{tabular}{l}
|\ifchilddoc|\\
|\addtocounter{page}{-1}|\\
\textit{code for banner page}\\
|\newpage|\\
|\||fi|
\end{tabular}
\end{center}
%
Here one could write a message such as:
\begin{center}
|This is the part \childdocname{} of \childdocjob{}.|
\end{center}

%%%%%%%%%%%%%%%%%%%%%%%%%%%%%%%%%%%%%%%%%%%%%%%%%%%%%%%%%%%%%%%%%%%%%%%%%%%%%%%%
\subsection{Flags}
\label{sec:flags}

The package makes it easy to generate different versions
of the main or child documents.
To this end compilation flags can be defined
and assigned different default values.
They will be particularly useful in conjunction
with the forwarding mechanism described in \secref{sec:forward}.

For example, it may be useful to have a flag |\version|
which can be set to |draft| or |final|.
The document source will contain some conditional code
depending on the value of |\version|.
Suppose further, the flag should default to |final| for the main file
and to |draft| for child files
which is a natural assignment for editing the document.
This is achieved by placing the following code
in the preamble of the main document
(below the |\childdocmain| directive):
%
\begin{center}
\begin{tabular}{l}
|\ifchilddoc|\\
|\providecommand{\version}{draft}|\\
|\||else|\\
|\providecommand{\version}{final}|\\
|\||fi|
\end{tabular}
\end{center}
%
The definition by |\providecommand| makes sure
that previous definitions are not overwritten.
Further statements |\providecommand{\version}{...}|
can thus be added before the above code to override it.

For the main file, one might add a line
(between |\childdocmain| and the above block)
%
\begin{center}
|%\ifchilddoc\||else\providecommand{\version}{draft}\||fi|
\end{center}
%
which can be uncommented to produce a draft version.
Likewise one can add a line to the very top of a child file
(above the |\childdocof{|\textit{main}|}| directive)
%
\begin{center}
|%\providecommand{\version}{final}|
\end{center}
%
which can be uncommented to produce the final version of this child document.

%%%%%%%%%%%%%%%%%%%%%%%%%%%%%%%%%%%%%%%%%%%%%%%%%%%%%%%%%%%%%%%%%%%%%%%%%%%%%%%%
\subsection{Forwarding}
\label{sec:forward}

Different versions of the main or child documents
using compilation flags as described in \secref{sec:flags}
can be (permanently) stored in different files
for convenient compilation, viewing and distribution.
To this end, the package defines a command
to pass on compilation to a different file:

%%%%%%%%%%%%%%%%%%%%%%%%%%%%%%%%%%%%%%%%
\DescribeMacro{\childdocforward}
The command |\childdocforward| redirects processing to
another source file:
%
\begin{center}
\begin{tabular}{l}
|\input{childdoc.def}|\\
|\childdocforward[|\textit{main}|]{|\textit{dest}|}|\\
\end{tabular}
\end{center}
%
The argument \textit{dest} is the destination file
(without extension).
It should be the main file or one of the child files.
Note that further \textsf{childdoc} directives
such as |\childdocof| and |\childdocforward|
in the indicated file will be processed in this form.
The optional argument \textit{main}
passes on directly to the main file \textit{main}
while pretending to compile the child \textit{dest}.
This form behaves as if \textit{dest}
issues |\childdocof{|\textit{main}|}| right away,
and no further \textsf{childdoc} directives will be processed.

%%%%%%%%%%%%%%%%%%%%%%%%%%%%%%%%%%%%%%%%
\DescribeMacro{\...prefix}
In the alternative form |\childdocforwardprefix|,
%
\begin{center}
\begin{tabular}{l}
|\input{childdoc.def}|\\
|\childdocforwardprefix[|\textit{main}|]{|\textit{prefix}|}{|\textit{dest}|}|
\end{tabular}
\end{center}
%
the destination file is determined by a pattern
depending on the current file:
To make this work, the current file must be called
`{\textit{prefix}\hspace{0.2em}\textit{suffix}}'
with \textit{prefix} matching precisely the argument.
Processing is then passed on to the file
`{\textit{dest}\hspace{0.2em}\textit{suffix}}'.
Surely, the same effect is achieved by
directly specifying the
argument `{\textit{dest}\hspace{0.2em}\textit{suffix}}'
in the first form.
However, that requires to set up a different file
for each child. With the alternative form of the command
all these files can have exactly the same content
which simplifies setting them up and maintaining them.

For example, the following file |draft.tex|
with a compilation flag |\version| as described in \secref{sec:flags}
compiles the main document as a draft:
%
\begin{center}
\begin{tabular}{l}
|\def\version{draft}|\\
|\input{childdoc.def}|\\
|\childdocforward{|\textit{main}|}|
\end{tabular}
\end{center}
%
Likewise, the following files |final|\textit{nn}|.tex|
compile the final version of the child document
|child|\textit{nn}|.tex|:
%
\begin{center}
\begin{tabular}{l}
|\def\version{final}|\\
|\input{childdoc.def}|\\
|\childdocforwardprefix{final}{child}|
\end{tabular}
\end{center}
%

Note that when several versions of a main file and/or of each child file
are to be generated, it may be convenient to set up a |Makefile| or
shell script to automatise the process.

%%%%%%%%%%%%%%%%%%%%%%%%%%%%%%%%%%%%%%%%%%%%%%%%%%%%%%%%%%%%%%%%%%%%%%%%%%%%%%%%
\subsection{Command Line Processing}
\label{sec:commandline}

The effect of redirection files can also be achieved by invoking
the \LaTeX{} compiler with a more elaborate command line.
Most conveniently this should be done as part
of a shell script or a |Makefile|.

When using \textsf{childdoc} in the main file, the following
command lines effectively perform a redirection
(note that depending on the shell being used,
backslashes may have to be doubled: `|\|' $\to$ `|\\|'):
%
\begin{center}
|... -jobname "|\textit{target}|" |\\|"|[\textit{flags}]%
|\input{childdoc.def}\childdocforward[|\textit{main}|]{|\textit{dest}|}"|
\end{center}
%
Here \textit{target} is the name of the output file,
\textit{main} is the name of the main file
and \textit{dest} is the name of the main or child file to be processed
(all filenames without extensions).
The optional argument \textit{main} can be omitted
if \textit{main} matches \textit{dest}.
Optionally, compilation \textit{flags} can be defined via |\def| commands.
This command line makes the \TeX{} engine believe
it is compiling the file \textit{target}
whose content is specified as the latter parameter.
The provided code then forwards the processing to
\textit{main} or \textit{dest} as described in \secref{sec:forward}.

%%%%%%%%%%%%%%%%%%%%%%%%%%%%%%%%%%%%%%%%%%%%%%%%%%%%%%%%%%%%%%%%%%%%%%%%%%%%%%%%
\subsection{Include by Input}
\label{sec:input}

Including child documents by |\include| has some restrictions by design.
Most notably, the content of a child document always occupies
its own set of pages; pages cannot be shared between child documents.
Usually, this behaviour makes perfect sense
because each child document contain an essential part of the document.
However, in some situations it may be desirable to compose
a document from a collection of parts
without having mandatory page breaks between then.
For this case, the package
provides a mechanism to include parts
by |\input| which can also be processed individually.
However, by construction this mechanism
requires manual handling of the content to be output.

%%%%%%%%%%%%%%%%%%%%%%%%%%%%%%%%%%%%%%%%
\DescribeMacro{\ifchilddocmanual}
The main file should be prepared as usual, see \secref{sec:include}.
However, the document body must make a distinction
between processing of an individual part and of the main document, e.g.:
%
\begin{center}
\begin{tabular}{l}
|\ifchilddocmanual|\\
|\input{\childdocname}|\\
|\||else|\\
\textit{document body with }|\input{|\textit{part}|}|\\
|\||fi|
\end{tabular}
\end{center}
%
The conditional |\ifchilddocmanual| is true whenever
a part to be included by |\input| is being compiled,
and the name of the part is stored in |\childdocname|.

%%%%%%%%%%%%%%%%%%%%%%%%%%%%%%%%%%%%%%%%
\DescribeMacro{\childdocby}
Each part to be included by |\input| should start with:
%
\begin{center}
\begin{tabular}{l}
|\input{childdoc.def}|\\
|\childdocby{|\textit{main}|}|\\
\end{tabular}
\end{center}
%
The directive |\childdocby| is similar to |\childdocof|
described in \secref{sec:include},
but the subsequent selection of content must be done manually.
To that end, both |\ifchilddoc| and |\ifchilddocmanual|
will be true upon processing of a part,
and the name of the part is stored in |\childdocname|.
Note that |\jobname| will be set to the filename of the current part
so that each part receives an individual |.aux| file
that does not interfere with the |.aux| file(s) of the main document.
This behaviour can be altered by the alternative form
|\childdocby[*]{|\textit{main}|}| (with a non-empty optional argument)
which uses the |.aux| file of the main document
by setting |\jobname| to \textit{main}.

%%%%%%%%%%%%%%%%%%%%%%%%%%%%%%%%%%%%%%%%%%%%%%%%%%%%%%%%%%%%%%%%%%%%%%%%%%%%%%%%
\subsection{Driver Development}
\label{sec:driver}

The \textsf{childdoc} mechanism can also be use for the development
of definition files such as \LaTeX{} styles or classes.
This case differs from the above setup with multiple parts
included by |\include| in that no |\includeonly| should be invoked.
This can be achieved by starting the include file
(before |\ProvidesPackage|) with:
%
\begin{center}
\begin{tabular}{l}
|\input{childdoc.def}|\\
|\childdocforward{|\textit{main}|}|\\
\end{tabular}
\end{center}
%
or alternatively with:
%
\begin{center}
\begin{tabular}{l}
|\input{childdoc.def}|\\
|\childdocby{|\textit{main}|}|\\
\end{tabular}
\end{center}
%
Both forms have slightly different effects as described above.
The main file is prepared as usual, see \secref{sec:include}.

%%%%%%%%%%%%%%%%%%%%%%%%%%%%%%%%%%%%%%%%%%%%%%%%%%%%%%%%%%%%%%%%%%%%%%%%%%%%%%%%
\subsection{Legacy Detection}
\label{sec:detection}

The directive |\childdocmain| in the main file can detect
whether the complete document or merely a child is to be compiled
even without using the directive |\childdocof|.
This method is deprecated because it is less robust
and there is no compelling reason to use it;
it is merely provided for backward compatibility
and it may be removed in future versions.

If the detection mechanism is to be used,
it is mandatory to correctly specify
the filename of the main file as the argument of |\childdocmain|:
%
\begin{center}
\begin{tabular}{l}
|\input{childdoc.def}|\\
|\childdocmain{|\textit{main}|}|\\
\end{tabular}
\end{center}
%
If |\jobname| does not match the argument \textit{main} of |\childdocmain|,
it is assumed that |\jobname| points to the child file to be compiled.
When using |\childdocmain| with the main file specified as argument,
it suffices to start a child file
with just |\input{|\textit{main}|}|
without loading of the package and using |\childdocof|.
If instead all processing is done
with the appropriate \textsf{childdoc} directives,
the argument of \textit{main} of |\childdocmain| can be empty.

An alternative version of the command line processing described
in \secref{sec:commandline} using the detection mechanism reads:
%
\begin{center}
|... -jobname "|\textit{target}|" "|[\textit{flags}]%
[|\def\jobname{|\textit{dest}|}|]|\input{|\textit{main}|}"|
\end{center}

%%%%%%%%%%%%%%%%%%%%%%%%%%%%%%%%%%%%%%%%%%%%%%%%%%%%%%%%%%%%%%%%%%%%%%%%%%%%%%%%
\subsection{Manual Code}
\label{sec:manual}

In case one cannot be certain whether the definitions file |childdoc.def|
is installed on the target \TeX{} distribution
and one prefers not to ship it,
it is conceivable to paste a few relevant commands into the sources.

To that end, drop all statements |\input{childdoc.def}|
and perform the replacements as outlined below.
Instead of |\childdocmain{|\textit{main}|}| add the following code
to the top of the main file:
%
\begin{center}
\begin{tabular}{l}
|\||ifdefined\childdocname\endinput\||fi\newif\ifchilddoc|\\
|\edef\childdocname{\scantokens\expandafter{\jobname\noexpand}}|\\
|\def\childdocmain{|\textit{main}|}\||ifx\childdocmain\childdocname\||else|\\
|\childdoctrue\includeonly{\childdocname}\let\jobname\childdocmain\||fi|\\
\end{tabular}
\end{center}
%
Instead of |\childdocof{|\textit{main}|}| just include the main file
at the top of each child file:
%
\begin{center}
|\input{|\textit{main}|}|
\end{center}
%
A simple redirection |\childdocforward{|\textit{dest}|}| is achieved by:
%
\begin{center}
|\def\jobname{|\textit{dest}|}\input{\jobname}|
\end{center}
%
The redirection with prefix
|\childdocforwardprefix[|\textit{prefix}|]{|\textit{dest}|}|
is accomplished by:
%
\begin{center}
\begin{tabular}{l}
|{\edef\jobname{\scantokens\expandafter{\jobname\noexpand}}|\\
|\def\redirectjob |\textit{prefix}|#1~~~{\gdef\jobname{|\textit{dest}|#1}}|\\
|\expandafter\redirectjob\jobname~~~}\input{\jobname}|
\end{tabular}
\end{center}

In an alternative approach,
child documents can be compiled by a specific command line
without additional code or specific definitions:
%
\begin{center}
|... -jobname "|\textit{target}|" "|[\textit{flags}]%
|\includeonly{|\textit{dest}|}\input{|\textit{main}|}"|
\end{center}
%

%%%%%%%%%%%%%%%%%%%%%%%%%%%%%%%%%%%%%%%%%%%%%%%%%%%%%%%%%%%%%%%%%%%%%%%%%%%%%%%%
%%%%%%%%%%%%%%%%%%%%%%%%%%%%%%%%%%%%%%%%%%%%%%%%%%%%%%%%%%%%%%%%%%%%%%%%%%%%%%%%
\section{Information}

%%%%%%%%%%%%%%%%%%%%%%%%%%%%%%%%%%%%%%%%%%%%%%%%%%%%%%%%%%%%%%%%%%%%%%%%%%%%%%%%
\subsection{Copyright}

Copyright \copyright{} 2017--2018 Niklas Beisert

This work may be distributed and/or modified under the
conditions of the \LaTeX{} Project Public License, either version 1.3
of this license or (at your option) any later version.
The latest version of this license is in
  \url{http://www.latex-project.org/lppl.txt}
and version 1.3 or later is part of all distributions of \LaTeX{}
version 2005/12/01 or later.

This work has the LPPL maintenance status `maintained'.

The Current Maintainer of this work is Niklas Beisert.

This work consists of the files |README.txt|, |childdoc.ins| and |childdoc.dtx|
as well as the derived files |childdoc.def|, |cdocsamp.tex|
with |cdocsch1.tex|, |cdocsch2.tex|, |cdocspt3.tex|, |cdocspt4.tex|,
|cdocsdrf.tex|, |cdocsfn1.tex|, |cdocsfn2.tex|
as well as |childdoc.pdf|.

%%%%%%%%%%%%%%%%%%%%%%%%%%%%%%%%%%%%%%%%%%%%%%%%%%%%%%%%%%%%%%%%%%%%%%%%%%%%%%%%
\subsection{Files and Installation}

The package consists of the files:
%
\begin{center}
\begin{tabular}{ll}
    |README.txt|   & readme file \\
    |childdoc.ins| & installation file \\
    |childdoc.dtx| & source file \\
    |childdoc.def| & definition file \\
    |cdocsamp.tex| & sample main file \\
    |cdocsch1.tex| & sample include file \\
    |cdocsch2.tex| & sample include file \\
    |cdocspt3.tex| & sample part file \\
    |cdocspt4.tex| & sample part file \\
    |cdocsdrf.tex| & sample redirection file \\
    |cdocsfn1.tex| & sample redirection file \\
    |cdocsfn2.tex| & sample redirection file \\
    |childdoc.pdf| & manual
\end{tabular}
\end{center}
%
The distribution consists of the files
|README.txt|, |childdoc.ins| and |childdoc.dtx|.
%
\begin{itemize}
\item
Run (pdf)\LaTeX{} on |childdoc.dtx|
to compile the manual |childdoc.pdf| (this file).
\item
Run \LaTeX{} on |childdoc.ins| to create the definitions file |childdoc.def|
and the sample |cdocsamp.tex| with include files
|cdocsch1.tex|, |cdocsch2.tex|, |cdocspt3.tex|, |cdocspt4.tex|,
|cdocsdrf.tex|, |cdocsfn1.tex|, |cdocsfn2.tex|.
Then copy the file |childdoc.def| to an appropriate directory of your \LaTeX{}
distribution, e.g.\ \textit{texmf-root}|/tex/latex/childdoc|.
\end{itemize}

%%%%%%%%%%%%%%%%%%%%%%%%%%%%%%%%%%%%%%%%%%%%%%%%%%%%%%%%%%%%%%%%%%%%%%%%%%%%%%%%
\subsection{Related CTAN Packages}

There are several other packages which offer a similar functionality:
%
\begin{itemize}
\item
The packages
\href{http://ctan.org/pkg/docmute}{\textsf{docmute}},
\href{http://ctan.org/pkg/includex}{\textsf{includex}} and
\href{http://ctan.org/pkg/standalone}{\textsf{standalone}}
provide commands to include only the document body of
a child file thus allowing both files to be compiled individually.
\item
The packages \href{http://ctan.org/pkg/subdocs}{\textsf{subdocs}}
and \href{http://ctan.org/pkg/subfiles}{\textsf{subfiles}}
provide structures in which the main and child documents can be
encapsulated and allowing them to be compiled individually.
The inclusion mechanism is different from the conventional |\include|.
\item
The package \href{http://ctan.org/pkg/combine}{\textsf{combine}}
is an elaborate solution to combine several documents into one.
\end{itemize}
%
See also the CTAN topic \href{http://ctan.org/topic/subdocs}{\textsf{subdocs}}
for further related packages.
The present package differs from the above solutions in that
a document structure constructed with the conventional |\include| mechanism
just needs two extra commands at the top of every file
such that all constituent files can be compiled individually.

%%%%%%%%%%%%%%%%%%%%%%%%%%%%%%%%%%%%%%%%%%%%%%%%%%%%%%%%%%%%%%%%%%%%%%%%%%%%%%%%
%\subsection{Feature Suggestions}
%
%The following is a list of features which may be useful for future
%versions of this package:
%%
%\begin{itemize}
%\item
%\ldots
%\end{itemize}

%%%%%%%%%%%%%%%%%%%%%%%%%%%%%%%%%%%%%%%%%%%%%%%%%%%%%%%%%%%%%%%%%%%%%%%%%%%%%%%%
\subsection{Revision History}

%%%%%%%%%%%%%%%%%%%%%%%%%%%%%%%%%%%%%%%%
\paragraph{v2.0:} 2018/12/30

\begin{itemize}
\item
immediate forward processing
\item
added |\childdocby| mechanism
\item
manual restructured
\end{itemize}

%%%%%%%%%%%%%%%%%%%%%%%%%%%%%%%%%%%%%%%%
\paragraph{v1.6:} 2018/01/17

\begin{itemize}
\item
application for development of include files
\item
corrections to manual
\end{itemize}

%%%%%%%%%%%%%%%%%%%%%%%%%%%%%%%%%%%%%%%%
\paragraph{v1.5:} 2017/05/21

\begin{itemize}
\item
more complete structuring introduced
\item
|\childdocof| introduced
\item
|\childdoc| renamed to |\childdocmain|
\item
|\childredirect| renamed to |\childdocforward| and |\childdocforwardprefix|
and functionality expanded
\end{itemize}

%%%%%%%%%%%%%%%%%%%%%%%%%%%%%%%%%%%%%%%%
\paragraph{v1.0:} 2017/04/27

\begin{itemize}
\item
manual and install package
\item
first version published on CTAN
\end{itemize}

%%%%%%%%%%%%%%%%%%%%%%%%%%%%%%%%%%%%%%%%
\paragraph{v0.6:} 2017/04/26

\begin{itemize}
\item
redirection mechanism added
\end{itemize}

%%%%%%%%%%%%%%%%%%%%%%%%%%%%%%%%%%%%%%%%
\paragraph{v0.5:} 2017/04/26

\begin{itemize}
\item
functionality in definition file
\end{itemize}


%%%%%%%%%%%%%%%%%%%%%%%%%%%%%%%%%%%%%%%%%%%%%%%%%%%%%%%%%%%%%%%%%%%%%%%%%%%%%%%%
%%%%%%%%%%%%%%%%%%%%%%%%%%%%%%%%%%%%%%%%%%%%%%%%%%%%%%%%%%%%%%%%%%%%%%%%%%%%%%%%
%%%%%%%%%%%%%%%%%%%%%%%%%%%%%%%%%%%%%%%%%%%%%%%%%%%%%%%%%%%%%%%%%%%%%%%%%%%%%%%%
\appendix

\settowidth\MacroIndent{\rmfamily\scriptsize 000\ }

 \DocInput{childdoc.dtx}

\end{document}
%</driver>
% \fi
%
% %%%%%%%%%%%%%%%%%%%%%%%%%%%%%%%%%%%%%%%%%%%%%%%%%%%%%%%%%%%%%%%%%%%%%%%%%%%%%%
% %%%%%%%%%%%%%%%%%%%%%%%%%%%%%%%%%%%%%%%%%%%%%%%%%%%%%%%%%%%%%%%%%%%%%%%%%%%%%%
% \section{Sample}
%\iffalse
%<*samplemain>
%\fi
%
% The following presents a sample document
% with two chapters, two parts, a title page,
% a compile flag as well as three forwarding files to set the flag.
% It consists of eight |.tex| files:
% \begin{center}
% \begin{tabular}{ll}
% |cdocsamp.tex|&main file\\
% |cdocsch1.tex|&include file for chapter 1\\
% |cdocsch2.tex|&include file for chapter 2\\
% |cdocspt3.tex|&include file for part 3\\
% |cdocspt4.tex|&include file for part 4\\
% |cdocsdrf.tex|&forwarding file for main file in draft mode\\
% |cdocsfi1.tex|&forwarding file for final version of chapter 1\\
% |cdocsfi2.tex|&forwarding file for final version of chapter 2\\
% \end{tabular}
% \end{center}
% Each of the eight files can be compiled directly by the \LaTeX{} compiler.
%
% %%%%%%%%%%%%%%%%%%%%%%%%%%%%%%%%%%%%%%
% \paragraph{Main File.}
%
% The main file is called |cdocsamp.tex|.
%
% Load the \textsf{childdoc} definitions and
% declare the filename for the main document:
%    \begin{macrocode}
\input{childdoc.def}
\childdocmain{}
%    \end{macrocode}

% Optional override for |\version| flag:
%    \begin{macrocode}
%%\ifchilddoc\else\providecommand{\version}{draft}\fi
%    \end{macrocode}

% Define the default values for the |\version| flag
% (|final| for the main file and |draft| for childs):
%    \begin{macrocode}
\ifchilddoc
\providecommand{\version}{draft}
\else
\providecommand{\version}{final}
\fi
%    \end{macrocode}

% Load the standard document class:
%    \begin{macrocode}
\documentclass[12pt]{article}
%    \end{macrocode}

% Start the document body:
%    \begin{macrocode}
\begin{document}
%    \end{macrocode}

% Declare a title page.
% Print title, part of document being processed and version flag:
%    \begin{macrocode}
\addtocounter{page}{-1}
\begin{center}
{\LARGE\bfseries{}childdoc example\par}
\vspace{1cm}
\ifchilddoc
\ifchilddocmanual part\else chapter\fi:
`\childdocname' of `\childdocjob'\par
\else
main document: `\childdocjob'\par
\fi
version: \version\par
\end{center}
\newpage
%    \end{macrocode}

% Manually include selected file,
% otherwise process as usual:
%    \begin{macrocode}
\ifchilddocmanual
\section*{part `\childdocname'}
\input{\childdocname}
\else
%    \end{macrocode}

% Include the two chapters:
%    \begin{macrocode}
\include{cdocsch1}
\include{cdocsch2}
%    \end{macrocode}

% Include the two parts unless only chapters should be displayed:
%    \begin{macrocode}
\ifchilddoc\else
\section{part three}
\input{cdocspt3}
\section{part four}
\input{cdocspt4}
\fi
%    \end{macrocode}

% Process as usual until here:
%    \begin{macrocode}
\fi
%    \end{macrocode}

% End of document body:
%    \begin{macrocode}
\end{document}
%    \end{macrocode}
%\iffalse
%</samplemain>
%\fi
%
% %%%%%%%%%%%%%%%%%%%%%%%%%%%%%%%%%%%%%%
% \paragraph{Chapter Include Files.}
%
% The include files are called |cdocsch1.tex| and |cdocsch2.tex|.
%
%\iffalse
%<*samplechap1|samplechap2>
%\fi

% Optional override for |\version| flag:
%    \begin{macrocode}
%%\providecommand{\version}{final}
%    \end{macrocode}

% Include the main document:
%    \begin{macrocode}
\input{childdoc.def}
\childdocof{cdocsamp}
%    \end{macrocode}

%\iffalse
%</samplechap1|samplechap2>
%\fi
%
%\iffalse
%<*samplechap1>
%\fi
% Some text for chapter 1:
%    \begin{macrocode}
\section{one}
some text in chapter one
%    \end{macrocode}

%\iffalse
%</samplechap1>
%\fi
% Some text for chapter 2:
%\iffalse
%<*samplechap2>
%\fi
%    \begin{macrocode}
\section{two}
more text in chapter two
%    \end{macrocode}

%\iffalse
%</samplechap2>
%\fi
%
% %%%%%%%%%%%%%%%%%%%%%%%%%%%%%%%%%%%%%%
% \paragraph{Part Include Files.}
%
% The include files are called |cdocspt3.tex| and |cdocspt4.tex|.
%
%\iffalse
%<*samplepart3|samplepart4>
%\fi

% Optional override for |\version| flag:
%    \begin{macrocode}
%%\providecommand{\version}{final}
%    \end{macrocode}

% Include the main document:
%    \begin{macrocode}
\input{childdoc.def}
\childdocby{cdocsamp}
%    \end{macrocode}

%\iffalse
%</samplepart3|samplepart4>
%\fi
%
%\iffalse
%<*samplepart3>
%\fi
% Some text for part 3:
%    \begin{macrocode}
some text in part three
%    \end{macrocode}

%\iffalse
%</samplepart3>
%\fi
% Some text for part 4:
%\iffalse
%<*samplepart4>
%\fi
%    \begin{macrocode}
more text in part four
%    \end{macrocode}

%\iffalse
%</samplepart4>
%\fi
%
% %%%%%%%%%%%%%%%%%%%%%%%%%%%%%%%%%%%%%%
% \paragraph{Forwarding for a Complete Draft.}
%
% The following forwarding file |cdocsdrf.tex|
% compiles the main document in draft mode:
%\iffalse
%<*sampledraft>
%\fi
%    \begin{macrocode}
\def\version{draft}
\input{childdoc.def}
\childdocforward{cdocsamp}
%    \end{macrocode}

%\iffalse
%</sampledraft>
%\fi
%
% %%%%%%%%%%%%%%%%%%%%%%%%%%%%%%%%%%%%%%
% \paragraph{Forwarding for Final Version of the Chapters.}
%
% The following forwarding files |cdocsfn1.tex| and |cdocsfn2.tex|
% (with identical content)
% compile the final versions of the child documents
% |cdocsch1.tex| and |cdocsch2.tex|, respectively:
%\iffalse
%<*samplefinal>
%\fi
%    \begin{macrocode}
\def\version{final}
\input{childdoc.def}
\childdocforwardprefix[cdocsamp]{cdocsfn}{cdocsch}
%    \end{macrocode}

%\iffalse
%</samplefinal>
%\fi
%
% %%%%%%%%%%%%%%%%%%%%%%%%%%%%%%%%%%%%%%
% \paragraph{Command Line Processing.}
%
% The following three command lines generate the output files
% |cdocscld|, |cdocscl1| and |cdocscl2|
% which should be identical to
% |cdocsdrf|, |cdocsch1| and |cdocsfn2|, respectively:
% \begin{center}
% \begin{tabular}{l}
% |latex -jobname cdocscld \|\\
% |  "\def\version{draft}\input{childdoc.def}\childdocforward{cdocsamp}"|\\
% |latex -jobname cdocscl1 \|\\
% |  "\input{childdoc.def}\childdocforward[cdocsamp]{cdocsch1}"|\\
% |latex -jobname cdocscl2 \|\\
% |  "\def\version{final}\input{childdoc.def}\childdocforward{cdocsch2}"|
% \end{tabular}
% \end{center}
% Note that the trailing backslash on each first line
% merely continues the input to the second line
% (for convenient cut ant paste).
% Furthermore, the command |latex| can be replaced by any
% of its alternative versions such as |pdflatex|.
%
% %%%%%%%%%%%%%%%%%%%%%%%%%%%%%%%%%%%%%%%%%%%%%%%%%%%%%%%%%%%%%%%%%%%%%%%%%%%%%%
% %%%%%%%%%%%%%%%%%%%%%%%%%%%%%%%%%%%%%%%%%%%%%%%%%%%%%%%%%%%%%%%%%%%%%%%%%%%%%%
% \section{Implementation}
%\iffalse
%<*package>
%\fi
%
% This section describes the definitions file |childdoc.def|.

% The definitions cannot be loaded using |\usepackage| or |\RequirePackage|
% which has a mechanism to prevent loading a style file more than once.
% When loading the definitions by means of |\input|
% multiple instances have to be prevented manually:
%\iffalse
%This code needs to be before the `\ProvidesFile' directive
%which is defined at the beginning of this file.
%Therefore it is also placed there and commented out here.
%</package>
%<*discard>
%\fi
%    \begin{macrocode}
\ifdefined\childdocmain\endinput\fi
%    \end{macrocode}
%\iffalse
%</discard>
%<*package>
%\fi
%
% \macro{\ifchilddoc}
% \macro{\ifchilddocmanual}
% The conditional |\ifchilddoc| tells whether a
% child (true) or main (false) document is being compiled.
% The conditional |\ifchilddocmanual| tells whether
% the |\includeonly| mechanism is used (false) or
% the selection of child files must be performed manually (true).
% The definitions initialise to false:
%    \begin{macrocode}
\newif\ifchilddoc
\newif\ifchilddocmanual
%    \end{macrocode}

% \macro{\childdocname}
% \macro{\childdocjob}
% The macro |\childdocname| stores the name of the main document
% to be compiled. The macro |\childdocjob| stores the name of
% the document on which the \LaTeX{} compiler was originally invoked.
% The content of |\jobname| cannot be compared
% to filenames specified in the source due to different catcodes.
% The following code rescans |\jobname|, stores the result
% in |\childdocname| and saves a copy in |\childdocjob|:
%    \begin{macrocode}
\edef\childdocname{\scantokens\expandafter{\jobname\noexpand}}
\let\childdocjob\childdocname
%    \end{macrocode}

% \macro{\childdocdisable}
% The macro |\childdocdisable| prevents the main file
% from being processed more than once.
% At this stage, the main document command |\childdocmain|
% is assumed to be called once again where it should do nothing.
% Any subsequent call to it should prevent
% a secondary processing of the main document
% It overwrites the forwarding commands
% |\childdocof| and |\childdocforward|
% with empty macros to prevent further inclusions of the main document:
%    \begin{macrocode}
\newcommand{\childdocdisable}
{
  \renewcommand{\childdocmain}[1]{\renewcommand{\childdocmain}[1]{\endinput}}
  \renewcommand{\childdocof}[1]{}
  \renewcommand{\childdocby}[2][]{}
  \renewcommand{\childdocforward}[2][]{}
  \renewcommand{\childdocdisable}{}
}
%    \end{macrocode}

% \macro{\childdocmain}
% The macro |\childdocmain| is to be called at the top of the main file
% with nothing or the main filename (without extension) as argument.
% First, it breaks loops.
% If the argument is not empty and does not match |\childdocname|
% (which is set by the first inclusion of |childdoc.def|),
% |\ifchilddoc| is set to true, |\includeonly| is applied to the child file
% and |\jobname| is set to the main file
% (for proper handling of |.aux| files):
%    \begin{macrocode}
\newcommand{\childdocmain}[1]
{
  \childdocdisable\childdocmain{}
  \if?#1?\else
    \begingroup
      \def\childdoctmp{#1}
      \ifx\childdoctmp\childdocname
        \def\childdoctmp{}
      \else
        \def\childdoctmp
        {
          \childdoctrue
          \includeonly{\childdocname}
          \def\childdocjob{#1}
          \def\jobname{#1}
        }
      \fi
      \expandafter
    \endgroup
    \childdoctmp
  \fi
}
%    \end{macrocode}

% \macro{\childdocof}
% The command |\childdocof| redirects
% compilation to the main file |#1|.
%    \begin{macrocode}
\newcommand{\childdocof}[1]
{
  \childdocdisable
  \childdoctrue
  \includeonly{\childdocname}
  \def\jobname{#1}
  \def\childdocjob{#1}
  \input{#1}
}
%    \end{macrocode}

% \macro{\childdocby}
% The command |\childdocby| ....
%    \begin{macrocode}
\newcommand{\childdocby}[2][]
{
  \childdocdisable
  \childdoctrue
  \childdocmanualtrue
  \if?#1?\else
    \def\jobname{#2}
  \fi
  \def\childdocjob{#2}
  \input{#2}
  \endinput
}
%    \end{macrocode}

% \macro{\childdocforward}
% The command |\childdocforward| redirects
% compilation to the main file or
% (if the optional argument is given) a child file.
% Parameters are set as if the main file
% or a child file starting with |\childdocof| was compiled.
% Then compilation is handed over to the main file:
%    \begin{macrocode}
\newcommand{\childdocforward}[2][]
{
  \begingroup
    \if?#1?
      \def\childdoctmp
      {
        \def\childdocname{#2}
        \def\childdocjob{#2}
        \def\jobname{#2}
        \input{#2}
        \endinput
      }
    \else
      \def\childdoctmp
      {
        \childdocdisable
        \def\childdocname{#2}
        \childdoctrue
        \includeonly{#2}
        \def\childdocjob{#1}
        \def\jobname{#1}
        \input{#1}
        \endinput
      }
    \fi
    \expandafter
  \endgroup
  \childdoctmp
}
%    \end{macrocode}

% \macro{\childdocforwardprefix}
% The command |\childdocforwardprefix| redirects
% compilation to the main or a child file by means of a pattern.
% The prefix |#1| in the current filename is replaced by |#2|
% and the suffix of the current filename is kept
% (it is assumed that the filename does not contain the substring `|~~~|'
% which is used as a delimiter).
% Compilation is handed over to the new file by |\childdocforward|:
%    \begin{macrocode}
\newcommand{\childdocforwardprefix}[3][]
{
  \begingroup
    \def\childdocextract #2##1~~~{\def\childdoctmp{\childdocforward[#1]{#3##1}}}
    \expandafter\childdocextract\childdocname~~~
    \expandafter
  \endgroup
  \childdoctmp
}
%    \end{macrocode}

% \macro{\childdoc}
% The deprecated macro |\childdoc| is a legacy version of |\childdocmain|:
%    \begin{macrocode}
\newcommand{\childdoc}{\childdocmain}
%    \end{macrocode}

% \macro{\childdocredirect}
% The deprecated macro |\childdocredirect| is a legacy version
% of |\childdocforward| and |\childdocforwardprefix|:
%    \begin{macrocode}
\newcommand{\childdocredirect}[2][]
{
  \begingroup
    \if?#1?
      \def\childdoctmp{\childdocforward{#2}}
    \else
      \def\childdoctmp{\childdocforwardprefix{#1}{#2}}
    \fi
    \expandafter
  \endgroup
  \childdoctmp
}
%    \end{macrocode}

%\iffalse
%</package>
%\fi
%
\endinput
|\\
|\childdocby{|\textit{main}|}|\\
\end{tabular}
\end{center}
%
Both forms have slightly different effects as described above.
The main file is prepared as usual, see \secref{sec:include}.

%%%%%%%%%%%%%%%%%%%%%%%%%%%%%%%%%%%%%%%%%%%%%%%%%%%%%%%%%%%%%%%%%%%%%%%%%%%%%%%%
\subsection{Legacy Detection}
\label{sec:detection}

The directive |\childdocmain| in the main file can detect
whether the complete document or merely a child is to be compiled
even without using the directive |\childdocof|.
This method is deprecated because it is less robust
and there is no compelling reason to use it;
it is merely provided for backward compatibility
and it may be removed in future versions.

If the detection mechanism is to be used,
it is mandatory to correctly specify
the filename of the main file as the argument of |\childdocmain|:
%
\begin{center}
\begin{tabular}{l}
|% \iffalse
%
% childdoc.dtx Copyright (C) 2017-2018 Niklas Beisert
%
% This work may be distributed and/or modified under the
% conditions of the LaTeX Project Public License, either version 1.3
% of this license or (at your option) any later version.
% The latest version of this license is in
%   http://www.latex-project.org/lppl.txt
% and version 1.3 or later is part of all distributions of LaTeX
% version 2005/12/01 or later.
%
% This work has the LPPL maintenance status `maintained'.
%
% The Current Maintainer of this work is Niklas Beisert.
%
% This work consists of the files childdoc.dtx and childdoc.ins
% and the derived files childdoc.def and cdocsamp.tex with
% cdocsch1.tex, cdocsch2.tex, cdocsdrf.tex, cdocsfn1.tex, cdocsfn2.tex.
%
%<package>\ifdefined\childdocmain\endinput\fi
%<package>\ProvidesFile{childdoc.def}[2018/12/30 v2.0 child document driver]
%<samplemain>\ProvidesFile{cdocsamp.tex}[2018/12/30 v2.0 sample for childdoc]
%<*driver>
%\ProvidesFile{childdoc.drv}[2018/12/30 v2.0 childdoc reference manual file]
\PassOptionsToClass{10pt,a4paper}{article}
\documentclass{ltxdoc}

\usepackage[margin=35mm]{geometry}
\usepackage{hyperref}
\usepackage{hyperxmp}
\usepackage[usenames]{color}

\hypersetup{colorlinks=true}
\hypersetup{pdfstartview=FitH}
\hypersetup{pdfpagemode=UseNone}
\hypersetup{pdfsource={}}
\hypersetup{pdflang={en-UK}}
\hypersetup{pdfcopyright={Copyright 2017-2018 Niklas Beisert.
  This work may be distributed and/or modified under the
  conditions of the LaTeX Project Public License, either version 1.3
  of this license or (at your option) any later version.}}
\hypersetup{pdflicenseurl={http://www.latex-project.org/lppl.txt}}
\hypersetup{pdfcontactaddress={ETH Zurich, ITP, HIT K,
  Wolfgang-Pauli-Strasse 27}}
\hypersetup{pdfcontactpostcode={8093}}
\hypersetup{pdfcontactcity={Zurich}}
\hypersetup{pdfcontactcountry={Switzerland}}
\hypersetup{pdfcontactemail={nbeisert@itp.phys.ethz.ch}}
\hypersetup{pdfcontacturl={http://people.phys.ethz.ch/\xmptilde nbeisert/}}

\newcommand{\secref}[1]{\hyperref[#1]{section \ref*{#1}}}

\parskip1ex
\parindent0pt
\let\olditemize\itemize
\def\itemize{\olditemize\parskip0pt}

\begin{document}

\title{The \textsf{childdoc} Package}
\hypersetup{pdftitle={The childdoc Package}}
\author{Niklas Beisert\\[2ex]
  Institut f\"ur Theoretische Physik\\
  Eidgen\"ossische Technische Hochschule Z\"urich\\
  Wolfgang-Pauli-Strasse 27, 8093 Z\"urich, Switzerland\\[1ex]
  \href{mailto:nbeisert@itp.phys.ethz.ch}
  {\texttt{nbeisert@itp.phys.ethz.ch}}}
\hypersetup{pdfauthor={Niklas Beisert}}
\hypersetup{pdfsubject={Manual for the LaTeX2e Package childdoc}}
\date{30 December 2018, \textsf{v2.0}}
\maketitle

\begin{abstract}\noindent
\textsf{childdoc} is a \LaTeXe{} package
that enables the direct compilation
of document sections included by |\include|
to individual files.
\end{abstract}

\begingroup
\parskip0ex
\tableofcontents
\endgroup

%%%%%%%%%%%%%%%%%%%%%%%%%%%%%%%%%%%%%%%%%%%%%%%%%%%%%%%%%%%%%%%%%%%%%%%%%%%%%%%%
%%%%%%%%%%%%%%%%%%%%%%%%%%%%%%%%%%%%%%%%%%%%%%%%%%%%%%%%%%%%%%%%%%%%%%%%%%%%%%%%
\section{Introduction}

\LaTeX{} provides a mechanism to structure a large document (such as a book)
into a main file and several child files (containing the chapters)
using the |\include| command.
This mechanism is beneficial for documents
which span hundreds of pages in order to
make the source file(s) more manageable.
Moreover, compilation can be restricted to
selected child files by means of the |\includeonly| command.
The latter feature can be used to reduce the compilation time while editing
(this was significantly more useful in the earlier days of \LaTeX{})
or to generate a smaller document which is easier to navigate.
Another application of |\includeonly| is to generate
documents consisting of selected parts of the complete document.

However, there are a few drawbacks of the plain |\include| mechanism:
\begin{itemize}
\item
The child files cannot be compiled on their own,
they can only be compiled via the main file.
A naive editing environment
(such as a text editor with an option
to have the current file processed by \LaTeX)
may require one to switch to the main file before compiling;
attempting to compile the child file produces errors.
\item
The main file must be modified (each time)
to adjust the |\includeonly| command
to the present needs. This easily leaves the main file in a messy state.
\item
The generated document will always carry the filename
of the main document. This is inconvenient if
several child files are to be compiled and
to be kept for distribution.
\end{itemize}

The present package provides a simple interface
to make child files individually compilable by \LaTeX{}.
Compiling a child file then has the same effect as compiling
the main file with an |\includeonly| command
to select the appropriate child.
Moreover the generated document will carry the name of the child
rather than the main file.
This resolves all three above issues.

This feature is meant to make the editing of books,
thesis documents and lecture notes somewhat more convenient.
However, the package can also be used efficiently for
composing a series of documents (such as exercise sheets)
which are typically distributed individually.
It then assists the author in generating the individual documents
(potentially in different versions)
as well as a document containing the collected series.
Another application is in developing style files
or other kinds of included material
where compilation of the style file could redirect
to a sample or test file.

%%%%%%%%%%%%%%%%%%%%%%%%%%%%%%%%%%%%%%%%%%%%%%%%%%%%%%%%%%%%%%%%%%%%%%%%%%%%%%%%
%%%%%%%%%%%%%%%%%%%%%%%%%%%%%%%%%%%%%%%%%%%%%%%%%%%%%%%%%%%%%%%%%%%%%%%%%%%%%%%%
\section{Usage}

First of all, the package \textsf{childdoc} is \emph{not} a standard
\LaTeXe{} |.sty| style file! Therefore it needs to be invoked in
a non-standard way.

%%%%%%%%%%%%%%%%%%%%%%%%%%%%%%%%%%%%%%%%%%%%%%%%%%%%%%%%%%%%%%%%%%%%%%%%%%%%%%%%
\subsection{Included Files}
\label{sec:include}

%%%%%%%%%%%%%%%%%%%%%%%%%%%%%%%%%%%%%%%%
\DescribeMacro{\childdocmain}
To use the package, add the commands
\begin{center}
\begin{tabular}{l}
|\input{childdoc.def}|\\
|\childdocmain{}|\\
\end{tabular}
\end{center}
at the very top of the main \LaTeX{} file,
in particular \emph{before} the |\documentclass| statement!
The argument of |\childdocmain| should be left empty
(but it must be present).

%%%%%%%%%%%%%%%%%%%%%%%%%%%%%%%%%%%%%%%%
\DescribeMacro{\childdocof}
Furthermore, add the commands
\begin{center}
\begin{tabular}{l}
|\input{childdoc.def}|\\
|\childdocof{|\textit{main}|}|\\
\end{tabular}
\end{center}
at the top of every child file \textit{child}
which is included by |\include{|\textit{child}|}|
from within the main file
(or at least for those files to be compiled individually).
The argument \textit{main} must be the filename of the main file.

There are a couple of
considerations in setting up the main and child documents:

%%%%%%%%%%%%%%%%%%%%%%%%%%%%%%%%%%%%%%%%
\paragraph{Restrictions.}

Please note the following restrictions:
\begin{itemize}
\item
|\childdocmain| must be called with one argument \textit{main}
to ensure compatibility with earlier version of the package.
It must either be empty (|\childdocmain{}|)
or precisely match the filename of the main file in which it is specified.
See \secref{sec:detection} for further information.
\item
The filename \textit{main} must be specified without the |.tex| extension.
\item
The filename \textit{main} is case sensitive
(even in case-insensitive file systems)
due to internal string comparison.
\item
The argument \textit{main} should be fully expanded, it cannot be a macro.
\item
Subdirectories and special characters should be avoided in filenames.
\item
The command |\childdocmain{|\textit{main}|}| must be followed by a whitespace.
It should not be followed immediately by another command
or by a comment mark `|%|'.
This is because the \TeX{} parser reads the token immediately following
the argument of |\childdocmain| and puts it
at the beginning of every child section;
however, a white\-space is ignored.
\end{itemize}

%%%%%%%%%%%%%%%%%%%%%%%%%%%%%%%%%%%%%%%%
\paragraph{Content of Main File.}

It is advisable to place all content in the child files included by |\include|.
Any output contained in the main file will appear in all child documents
unless suppressed manually;
it cannot be suppressed automatically by the |\includeonly| directive
and thus should normally be avoided.
A method to include some content in the main file
by means of conditional processing is described in \secref{sec:conditional}.

%%%%%%%%%%%%%%%%%%%%%%%%%%%%%%%%%%%%%%%%
\paragraph{Page Numbering.}

When only a part of the document is compiled,
the appropriate numbering of pages
(as well as other status parameters)
is determined from the |.aux| files.
The latter contain information from previous passes.
However this information needs to propagate through
all intermediate child documents.
Therefore the page numbering in child documents may well
be inconsistent until the complete document is compiled at least once.

A useful (if unconventional) way to always ensure a consistent
page numbering is to restart the numbering in each child document
and denote the pages by `\textit{child}|.|\textit{page}'
where \textit{child} represents the chapter/section number of the child file.
This can be achieved by the command
|\numberwithin{page}{|\textit{child}|}|
of the \textsf{amsmath} package
where \textit{child} can be |chapter| or |section|
depending on the chosen structuring.
Alternatively, one can modify the macro |\thepage| appropriately
and reset the counter |page| at the start of each child file.

%%%%%%%%%%%%%%%%%%%%%%%%%%%%%%%%%%%%%%%%%%%%%%%%%%%%%%%%%%%%%%%%%%%%%%%%%%%%%%%%
\subsection{Conditional Processing}
\label{sec:conditional}

The package provides a mechanism to compile different versions
of a document. To customise the versions further some conditional processing
can come in handy to distinguish which version is being compiled.
The package provides two macros to describe the compilation context:

%%%%%%%%%%%%%%%%%%%%%%%%%%%%%%%%%%%%%%%%
\DescribeMacro{\ifchilddoc}
The conditional |\ifchilddoc| distinguishes between the compilation of
child documents and the main document:
%
\begin{center}
|\ifchilddoc |\textit{child-code}| |[|\||else |\textit{main-code}]| \||fi|
\end{center}

%%%%%%%%%%%%%%%%%%%%%%%%%%%%%%%%%%%%%%%%
\DescribeMacro{\childdocname}
\DescribeMacro{\childdocjob}
The macro |\childdocname| contains the filename (without extension)
of the main or child file being processed.
Note that |\childdocjob| will always contain the name of the main file.

%%%%%%%%%%%%%%%%%%%%%%%%%%%%%%%%%%%%%%%%
\paragraph{Title Page.}

Conditional processing can be used to include a title or banner page
in the main document when proper precautions are taken.
Importantly, the code in the main file should ensure that the page counter
(as well as other status parameters which are stored in the |.aux| files)
takes the same value after the conditional processing.
Otherwise the page numbers may take divergent values
depending on which part is compiled.

For example, a title page could be declared by:
%
\begin{center}
\begin{tabular}{l}
|\ifchilddoc\||else|\\
|\addtocounter{page}{-1}|\\
\textit{code for title page}\\
|\newpage|\\
|\||fi|
\end{tabular}
\end{center}
%
A banner page for the child documents can be generated by:
%
\begin{center}
\begin{tabular}{l}
|\ifchilddoc|\\
|\addtocounter{page}{-1}|\\
\textit{code for banner page}\\
|\newpage|\\
|\||fi|
\end{tabular}
\end{center}
%
Here one could write a message such as:
\begin{center}
|This is the part \childdocname{} of \childdocjob{}.|
\end{center}

%%%%%%%%%%%%%%%%%%%%%%%%%%%%%%%%%%%%%%%%%%%%%%%%%%%%%%%%%%%%%%%%%%%%%%%%%%%%%%%%
\subsection{Flags}
\label{sec:flags}

The package makes it easy to generate different versions
of the main or child documents.
To this end compilation flags can be defined
and assigned different default values.
They will be particularly useful in conjunction
with the forwarding mechanism described in \secref{sec:forward}.

For example, it may be useful to have a flag |\version|
which can be set to |draft| or |final|.
The document source will contain some conditional code
depending on the value of |\version|.
Suppose further, the flag should default to |final| for the main file
and to |draft| for child files
which is a natural assignment for editing the document.
This is achieved by placing the following code
in the preamble of the main document
(below the |\childdocmain| directive):
%
\begin{center}
\begin{tabular}{l}
|\ifchilddoc|\\
|\providecommand{\version}{draft}|\\
|\||else|\\
|\providecommand{\version}{final}|\\
|\||fi|
\end{tabular}
\end{center}
%
The definition by |\providecommand| makes sure
that previous definitions are not overwritten.
Further statements |\providecommand{\version}{...}|
can thus be added before the above code to override it.

For the main file, one might add a line
(between |\childdocmain| and the above block)
%
\begin{center}
|%\ifchilddoc\||else\providecommand{\version}{draft}\||fi|
\end{center}
%
which can be uncommented to produce a draft version.
Likewise one can add a line to the very top of a child file
(above the |\childdocof{|\textit{main}|}| directive)
%
\begin{center}
|%\providecommand{\version}{final}|
\end{center}
%
which can be uncommented to produce the final version of this child document.

%%%%%%%%%%%%%%%%%%%%%%%%%%%%%%%%%%%%%%%%%%%%%%%%%%%%%%%%%%%%%%%%%%%%%%%%%%%%%%%%
\subsection{Forwarding}
\label{sec:forward}

Different versions of the main or child documents
using compilation flags as described in \secref{sec:flags}
can be (permanently) stored in different files
for convenient compilation, viewing and distribution.
To this end, the package defines a command
to pass on compilation to a different file:

%%%%%%%%%%%%%%%%%%%%%%%%%%%%%%%%%%%%%%%%
\DescribeMacro{\childdocforward}
The command |\childdocforward| redirects processing to
another source file:
%
\begin{center}
\begin{tabular}{l}
|\input{childdoc.def}|\\
|\childdocforward[|\textit{main}|]{|\textit{dest}|}|\\
\end{tabular}
\end{center}
%
The argument \textit{dest} is the destination file
(without extension).
It should be the main file or one of the child files.
Note that further \textsf{childdoc} directives
such as |\childdocof| and |\childdocforward|
in the indicated file will be processed in this form.
The optional argument \textit{main}
passes on directly to the main file \textit{main}
while pretending to compile the child \textit{dest}.
This form behaves as if \textit{dest}
issues |\childdocof{|\textit{main}|}| right away,
and no further \textsf{childdoc} directives will be processed.

%%%%%%%%%%%%%%%%%%%%%%%%%%%%%%%%%%%%%%%%
\DescribeMacro{\...prefix}
In the alternative form |\childdocforwardprefix|,
%
\begin{center}
\begin{tabular}{l}
|\input{childdoc.def}|\\
|\childdocforwardprefix[|\textit{main}|]{|\textit{prefix}|}{|\textit{dest}|}|
\end{tabular}
\end{center}
%
the destination file is determined by a pattern
depending on the current file:
To make this work, the current file must be called
`{\textit{prefix}\hspace{0.2em}\textit{suffix}}'
with \textit{prefix} matching precisely the argument.
Processing is then passed on to the file
`{\textit{dest}\hspace{0.2em}\textit{suffix}}'.
Surely, the same effect is achieved by
directly specifying the
argument `{\textit{dest}\hspace{0.2em}\textit{suffix}}'
in the first form.
However, that requires to set up a different file
for each child. With the alternative form of the command
all these files can have exactly the same content
which simplifies setting them up and maintaining them.

For example, the following file |draft.tex|
with a compilation flag |\version| as described in \secref{sec:flags}
compiles the main document as a draft:
%
\begin{center}
\begin{tabular}{l}
|\def\version{draft}|\\
|\input{childdoc.def}|\\
|\childdocforward{|\textit{main}|}|
\end{tabular}
\end{center}
%
Likewise, the following files |final|\textit{nn}|.tex|
compile the final version of the child document
|child|\textit{nn}|.tex|:
%
\begin{center}
\begin{tabular}{l}
|\def\version{final}|\\
|\input{childdoc.def}|\\
|\childdocforwardprefix{final}{child}|
\end{tabular}
\end{center}
%

Note that when several versions of a main file and/or of each child file
are to be generated, it may be convenient to set up a |Makefile| or
shell script to automatise the process.

%%%%%%%%%%%%%%%%%%%%%%%%%%%%%%%%%%%%%%%%%%%%%%%%%%%%%%%%%%%%%%%%%%%%%%%%%%%%%%%%
\subsection{Command Line Processing}
\label{sec:commandline}

The effect of redirection files can also be achieved by invoking
the \LaTeX{} compiler with a more elaborate command line.
Most conveniently this should be done as part
of a shell script or a |Makefile|.

When using \textsf{childdoc} in the main file, the following
command lines effectively perform a redirection
(note that depending on the shell being used,
backslashes may have to be doubled: `|\|' $\to$ `|\\|'):
%
\begin{center}
|... -jobname "|\textit{target}|" |\\|"|[\textit{flags}]%
|\input{childdoc.def}\childdocforward[|\textit{main}|]{|\textit{dest}|}"|
\end{center}
%
Here \textit{target} is the name of the output file,
\textit{main} is the name of the main file
and \textit{dest} is the name of the main or child file to be processed
(all filenames without extensions).
The optional argument \textit{main} can be omitted
if \textit{main} matches \textit{dest}.
Optionally, compilation \textit{flags} can be defined via |\def| commands.
This command line makes the \TeX{} engine believe
it is compiling the file \textit{target}
whose content is specified as the latter parameter.
The provided code then forwards the processing to
\textit{main} or \textit{dest} as described in \secref{sec:forward}.

%%%%%%%%%%%%%%%%%%%%%%%%%%%%%%%%%%%%%%%%%%%%%%%%%%%%%%%%%%%%%%%%%%%%%%%%%%%%%%%%
\subsection{Include by Input}
\label{sec:input}

Including child documents by |\include| has some restrictions by design.
Most notably, the content of a child document always occupies
its own set of pages; pages cannot be shared between child documents.
Usually, this behaviour makes perfect sense
because each child document contain an essential part of the document.
However, in some situations it may be desirable to compose
a document from a collection of parts
without having mandatory page breaks between then.
For this case, the package
provides a mechanism to include parts
by |\input| which can also be processed individually.
However, by construction this mechanism
requires manual handling of the content to be output.

%%%%%%%%%%%%%%%%%%%%%%%%%%%%%%%%%%%%%%%%
\DescribeMacro{\ifchilddocmanual}
The main file should be prepared as usual, see \secref{sec:include}.
However, the document body must make a distinction
between processing of an individual part and of the main document, e.g.:
%
\begin{center}
\begin{tabular}{l}
|\ifchilddocmanual|\\
|\input{\childdocname}|\\
|\||else|\\
\textit{document body with }|\input{|\textit{part}|}|\\
|\||fi|
\end{tabular}
\end{center}
%
The conditional |\ifchilddocmanual| is true whenever
a part to be included by |\input| is being compiled,
and the name of the part is stored in |\childdocname|.

%%%%%%%%%%%%%%%%%%%%%%%%%%%%%%%%%%%%%%%%
\DescribeMacro{\childdocby}
Each part to be included by |\input| should start with:
%
\begin{center}
\begin{tabular}{l}
|\input{childdoc.def}|\\
|\childdocby{|\textit{main}|}|\\
\end{tabular}
\end{center}
%
The directive |\childdocby| is similar to |\childdocof|
described in \secref{sec:include},
but the subsequent selection of content must be done manually.
To that end, both |\ifchilddoc| and |\ifchilddocmanual|
will be true upon processing of a part,
and the name of the part is stored in |\childdocname|.
Note that |\jobname| will be set to the filename of the current part
so that each part receives an individual |.aux| file
that does not interfere with the |.aux| file(s) of the main document.
This behaviour can be altered by the alternative form
|\childdocby[*]{|\textit{main}|}| (with a non-empty optional argument)
which uses the |.aux| file of the main document
by setting |\jobname| to \textit{main}.

%%%%%%%%%%%%%%%%%%%%%%%%%%%%%%%%%%%%%%%%%%%%%%%%%%%%%%%%%%%%%%%%%%%%%%%%%%%%%%%%
\subsection{Driver Development}
\label{sec:driver}

The \textsf{childdoc} mechanism can also be use for the development
of definition files such as \LaTeX{} styles or classes.
This case differs from the above setup with multiple parts
included by |\include| in that no |\includeonly| should be invoked.
This can be achieved by starting the include file
(before |\ProvidesPackage|) with:
%
\begin{center}
\begin{tabular}{l}
|\input{childdoc.def}|\\
|\childdocforward{|\textit{main}|}|\\
\end{tabular}
\end{center}
%
or alternatively with:
%
\begin{center}
\begin{tabular}{l}
|\input{childdoc.def}|\\
|\childdocby{|\textit{main}|}|\\
\end{tabular}
\end{center}
%
Both forms have slightly different effects as described above.
The main file is prepared as usual, see \secref{sec:include}.

%%%%%%%%%%%%%%%%%%%%%%%%%%%%%%%%%%%%%%%%%%%%%%%%%%%%%%%%%%%%%%%%%%%%%%%%%%%%%%%%
\subsection{Legacy Detection}
\label{sec:detection}

The directive |\childdocmain| in the main file can detect
whether the complete document or merely a child is to be compiled
even without using the directive |\childdocof|.
This method is deprecated because it is less robust
and there is no compelling reason to use it;
it is merely provided for backward compatibility
and it may be removed in future versions.

If the detection mechanism is to be used,
it is mandatory to correctly specify
the filename of the main file as the argument of |\childdocmain|:
%
\begin{center}
\begin{tabular}{l}
|\input{childdoc.def}|\\
|\childdocmain{|\textit{main}|}|\\
\end{tabular}
\end{center}
%
If |\jobname| does not match the argument \textit{main} of |\childdocmain|,
it is assumed that |\jobname| points to the child file to be compiled.
When using |\childdocmain| with the main file specified as argument,
it suffices to start a child file
with just |\input{|\textit{main}|}|
without loading of the package and using |\childdocof|.
If instead all processing is done
with the appropriate \textsf{childdoc} directives,
the argument of \textit{main} of |\childdocmain| can be empty.

An alternative version of the command line processing described
in \secref{sec:commandline} using the detection mechanism reads:
%
\begin{center}
|... -jobname "|\textit{target}|" "|[\textit{flags}]%
[|\def\jobname{|\textit{dest}|}|]|\input{|\textit{main}|}"|
\end{center}

%%%%%%%%%%%%%%%%%%%%%%%%%%%%%%%%%%%%%%%%%%%%%%%%%%%%%%%%%%%%%%%%%%%%%%%%%%%%%%%%
\subsection{Manual Code}
\label{sec:manual}

In case one cannot be certain whether the definitions file |childdoc.def|
is installed on the target \TeX{} distribution
and one prefers not to ship it,
it is conceivable to paste a few relevant commands into the sources.

To that end, drop all statements |\input{childdoc.def}|
and perform the replacements as outlined below.
Instead of |\childdocmain{|\textit{main}|}| add the following code
to the top of the main file:
%
\begin{center}
\begin{tabular}{l}
|\||ifdefined\childdocname\endinput\||fi\newif\ifchilddoc|\\
|\edef\childdocname{\scantokens\expandafter{\jobname\noexpand}}|\\
|\def\childdocmain{|\textit{main}|}\||ifx\childdocmain\childdocname\||else|\\
|\childdoctrue\includeonly{\childdocname}\let\jobname\childdocmain\||fi|\\
\end{tabular}
\end{center}
%
Instead of |\childdocof{|\textit{main}|}| just include the main file
at the top of each child file:
%
\begin{center}
|\input{|\textit{main}|}|
\end{center}
%
A simple redirection |\childdocforward{|\textit{dest}|}| is achieved by:
%
\begin{center}
|\def\jobname{|\textit{dest}|}\input{\jobname}|
\end{center}
%
The redirection with prefix
|\childdocforwardprefix[|\textit{prefix}|]{|\textit{dest}|}|
is accomplished by:
%
\begin{center}
\begin{tabular}{l}
|{\edef\jobname{\scantokens\expandafter{\jobname\noexpand}}|\\
|\def\redirectjob |\textit{prefix}|#1~~~{\gdef\jobname{|\textit{dest}|#1}}|\\
|\expandafter\redirectjob\jobname~~~}\input{\jobname}|
\end{tabular}
\end{center}

In an alternative approach,
child documents can be compiled by a specific command line
without additional code or specific definitions:
%
\begin{center}
|... -jobname "|\textit{target}|" "|[\textit{flags}]%
|\includeonly{|\textit{dest}|}\input{|\textit{main}|}"|
\end{center}
%

%%%%%%%%%%%%%%%%%%%%%%%%%%%%%%%%%%%%%%%%%%%%%%%%%%%%%%%%%%%%%%%%%%%%%%%%%%%%%%%%
%%%%%%%%%%%%%%%%%%%%%%%%%%%%%%%%%%%%%%%%%%%%%%%%%%%%%%%%%%%%%%%%%%%%%%%%%%%%%%%%
\section{Information}

%%%%%%%%%%%%%%%%%%%%%%%%%%%%%%%%%%%%%%%%%%%%%%%%%%%%%%%%%%%%%%%%%%%%%%%%%%%%%%%%
\subsection{Copyright}

Copyright \copyright{} 2017--2018 Niklas Beisert

This work may be distributed and/or modified under the
conditions of the \LaTeX{} Project Public License, either version 1.3
of this license or (at your option) any later version.
The latest version of this license is in
  \url{http://www.latex-project.org/lppl.txt}
and version 1.3 or later is part of all distributions of \LaTeX{}
version 2005/12/01 or later.

This work has the LPPL maintenance status `maintained'.

The Current Maintainer of this work is Niklas Beisert.

This work consists of the files |README.txt|, |childdoc.ins| and |childdoc.dtx|
as well as the derived files |childdoc.def|, |cdocsamp.tex|
with |cdocsch1.tex|, |cdocsch2.tex|, |cdocspt3.tex|, |cdocspt4.tex|,
|cdocsdrf.tex|, |cdocsfn1.tex|, |cdocsfn2.tex|
as well as |childdoc.pdf|.

%%%%%%%%%%%%%%%%%%%%%%%%%%%%%%%%%%%%%%%%%%%%%%%%%%%%%%%%%%%%%%%%%%%%%%%%%%%%%%%%
\subsection{Files and Installation}

The package consists of the files:
%
\begin{center}
\begin{tabular}{ll}
    |README.txt|   & readme file \\
    |childdoc.ins| & installation file \\
    |childdoc.dtx| & source file \\
    |childdoc.def| & definition file \\
    |cdocsamp.tex| & sample main file \\
    |cdocsch1.tex| & sample include file \\
    |cdocsch2.tex| & sample include file \\
    |cdocspt3.tex| & sample part file \\
    |cdocspt4.tex| & sample part file \\
    |cdocsdrf.tex| & sample redirection file \\
    |cdocsfn1.tex| & sample redirection file \\
    |cdocsfn2.tex| & sample redirection file \\
    |childdoc.pdf| & manual
\end{tabular}
\end{center}
%
The distribution consists of the files
|README.txt|, |childdoc.ins| and |childdoc.dtx|.
%
\begin{itemize}
\item
Run (pdf)\LaTeX{} on |childdoc.dtx|
to compile the manual |childdoc.pdf| (this file).
\item
Run \LaTeX{} on |childdoc.ins| to create the definitions file |childdoc.def|
and the sample |cdocsamp.tex| with include files
|cdocsch1.tex|, |cdocsch2.tex|, |cdocspt3.tex|, |cdocspt4.tex|,
|cdocsdrf.tex|, |cdocsfn1.tex|, |cdocsfn2.tex|.
Then copy the file |childdoc.def| to an appropriate directory of your \LaTeX{}
distribution, e.g.\ \textit{texmf-root}|/tex/latex/childdoc|.
\end{itemize}

%%%%%%%%%%%%%%%%%%%%%%%%%%%%%%%%%%%%%%%%%%%%%%%%%%%%%%%%%%%%%%%%%%%%%%%%%%%%%%%%
\subsection{Related CTAN Packages}

There are several other packages which offer a similar functionality:
%
\begin{itemize}
\item
The packages
\href{http://ctan.org/pkg/docmute}{\textsf{docmute}},
\href{http://ctan.org/pkg/includex}{\textsf{includex}} and
\href{http://ctan.org/pkg/standalone}{\textsf{standalone}}
provide commands to include only the document body of
a child file thus allowing both files to be compiled individually.
\item
The packages \href{http://ctan.org/pkg/subdocs}{\textsf{subdocs}}
and \href{http://ctan.org/pkg/subfiles}{\textsf{subfiles}}
provide structures in which the main and child documents can be
encapsulated and allowing them to be compiled individually.
The inclusion mechanism is different from the conventional |\include|.
\item
The package \href{http://ctan.org/pkg/combine}{\textsf{combine}}
is an elaborate solution to combine several documents into one.
\end{itemize}
%
See also the CTAN topic \href{http://ctan.org/topic/subdocs}{\textsf{subdocs}}
for further related packages.
The present package differs from the above solutions in that
a document structure constructed with the conventional |\include| mechanism
just needs two extra commands at the top of every file
such that all constituent files can be compiled individually.

%%%%%%%%%%%%%%%%%%%%%%%%%%%%%%%%%%%%%%%%%%%%%%%%%%%%%%%%%%%%%%%%%%%%%%%%%%%%%%%%
%\subsection{Feature Suggestions}
%
%The following is a list of features which may be useful for future
%versions of this package:
%%
%\begin{itemize}
%\item
%\ldots
%\end{itemize}

%%%%%%%%%%%%%%%%%%%%%%%%%%%%%%%%%%%%%%%%%%%%%%%%%%%%%%%%%%%%%%%%%%%%%%%%%%%%%%%%
\subsection{Revision History}

%%%%%%%%%%%%%%%%%%%%%%%%%%%%%%%%%%%%%%%%
\paragraph{v2.0:} 2018/12/30

\begin{itemize}
\item
immediate forward processing
\item
added |\childdocby| mechanism
\item
manual restructured
\end{itemize}

%%%%%%%%%%%%%%%%%%%%%%%%%%%%%%%%%%%%%%%%
\paragraph{v1.6:} 2018/01/17

\begin{itemize}
\item
application for development of include files
\item
corrections to manual
\end{itemize}

%%%%%%%%%%%%%%%%%%%%%%%%%%%%%%%%%%%%%%%%
\paragraph{v1.5:} 2017/05/21

\begin{itemize}
\item
more complete structuring introduced
\item
|\childdocof| introduced
\item
|\childdoc| renamed to |\childdocmain|
\item
|\childredirect| renamed to |\childdocforward| and |\childdocforwardprefix|
and functionality expanded
\end{itemize}

%%%%%%%%%%%%%%%%%%%%%%%%%%%%%%%%%%%%%%%%
\paragraph{v1.0:} 2017/04/27

\begin{itemize}
\item
manual and install package
\item
first version published on CTAN
\end{itemize}

%%%%%%%%%%%%%%%%%%%%%%%%%%%%%%%%%%%%%%%%
\paragraph{v0.6:} 2017/04/26

\begin{itemize}
\item
redirection mechanism added
\end{itemize}

%%%%%%%%%%%%%%%%%%%%%%%%%%%%%%%%%%%%%%%%
\paragraph{v0.5:} 2017/04/26

\begin{itemize}
\item
functionality in definition file
\end{itemize}


%%%%%%%%%%%%%%%%%%%%%%%%%%%%%%%%%%%%%%%%%%%%%%%%%%%%%%%%%%%%%%%%%%%%%%%%%%%%%%%%
%%%%%%%%%%%%%%%%%%%%%%%%%%%%%%%%%%%%%%%%%%%%%%%%%%%%%%%%%%%%%%%%%%%%%%%%%%%%%%%%
%%%%%%%%%%%%%%%%%%%%%%%%%%%%%%%%%%%%%%%%%%%%%%%%%%%%%%%%%%%%%%%%%%%%%%%%%%%%%%%%
\appendix

\settowidth\MacroIndent{\rmfamily\scriptsize 000\ }

 \DocInput{childdoc.dtx}

\end{document}
%</driver>
% \fi
%
% %%%%%%%%%%%%%%%%%%%%%%%%%%%%%%%%%%%%%%%%%%%%%%%%%%%%%%%%%%%%%%%%%%%%%%%%%%%%%%
% %%%%%%%%%%%%%%%%%%%%%%%%%%%%%%%%%%%%%%%%%%%%%%%%%%%%%%%%%%%%%%%%%%%%%%%%%%%%%%
% \section{Sample}
%\iffalse
%<*samplemain>
%\fi
%
% The following presents a sample document
% with two chapters, two parts, a title page,
% a compile flag as well as three forwarding files to set the flag.
% It consists of eight |.tex| files:
% \begin{center}
% \begin{tabular}{ll}
% |cdocsamp.tex|&main file\\
% |cdocsch1.tex|&include file for chapter 1\\
% |cdocsch2.tex|&include file for chapter 2\\
% |cdocspt3.tex|&include file for part 3\\
% |cdocspt4.tex|&include file for part 4\\
% |cdocsdrf.tex|&forwarding file for main file in draft mode\\
% |cdocsfi1.tex|&forwarding file for final version of chapter 1\\
% |cdocsfi2.tex|&forwarding file for final version of chapter 2\\
% \end{tabular}
% \end{center}
% Each of the eight files can be compiled directly by the \LaTeX{} compiler.
%
% %%%%%%%%%%%%%%%%%%%%%%%%%%%%%%%%%%%%%%
% \paragraph{Main File.}
%
% The main file is called |cdocsamp.tex|.
%
% Load the \textsf{childdoc} definitions and
% declare the filename for the main document:
%    \begin{macrocode}
\input{childdoc.def}
\childdocmain{}
%    \end{macrocode}

% Optional override for |\version| flag:
%    \begin{macrocode}
%%\ifchilddoc\else\providecommand{\version}{draft}\fi
%    \end{macrocode}

% Define the default values for the |\version| flag
% (|final| for the main file and |draft| for childs):
%    \begin{macrocode}
\ifchilddoc
\providecommand{\version}{draft}
\else
\providecommand{\version}{final}
\fi
%    \end{macrocode}

% Load the standard document class:
%    \begin{macrocode}
\documentclass[12pt]{article}
%    \end{macrocode}

% Start the document body:
%    \begin{macrocode}
\begin{document}
%    \end{macrocode}

% Declare a title page.
% Print title, part of document being processed and version flag:
%    \begin{macrocode}
\addtocounter{page}{-1}
\begin{center}
{\LARGE\bfseries{}childdoc example\par}
\vspace{1cm}
\ifchilddoc
\ifchilddocmanual part\else chapter\fi:
`\childdocname' of `\childdocjob'\par
\else
main document: `\childdocjob'\par
\fi
version: \version\par
\end{center}
\newpage
%    \end{macrocode}

% Manually include selected file,
% otherwise process as usual:
%    \begin{macrocode}
\ifchilddocmanual
\section*{part `\childdocname'}
\input{\childdocname}
\else
%    \end{macrocode}

% Include the two chapters:
%    \begin{macrocode}
\include{cdocsch1}
\include{cdocsch2}
%    \end{macrocode}

% Include the two parts unless only chapters should be displayed:
%    \begin{macrocode}
\ifchilddoc\else
\section{part three}
\input{cdocspt3}
\section{part four}
\input{cdocspt4}
\fi
%    \end{macrocode}

% Process as usual until here:
%    \begin{macrocode}
\fi
%    \end{macrocode}

% End of document body:
%    \begin{macrocode}
\end{document}
%    \end{macrocode}
%\iffalse
%</samplemain>
%\fi
%
% %%%%%%%%%%%%%%%%%%%%%%%%%%%%%%%%%%%%%%
% \paragraph{Chapter Include Files.}
%
% The include files are called |cdocsch1.tex| and |cdocsch2.tex|.
%
%\iffalse
%<*samplechap1|samplechap2>
%\fi

% Optional override for |\version| flag:
%    \begin{macrocode}
%%\providecommand{\version}{final}
%    \end{macrocode}

% Include the main document:
%    \begin{macrocode}
\input{childdoc.def}
\childdocof{cdocsamp}
%    \end{macrocode}

%\iffalse
%</samplechap1|samplechap2>
%\fi
%
%\iffalse
%<*samplechap1>
%\fi
% Some text for chapter 1:
%    \begin{macrocode}
\section{one}
some text in chapter one
%    \end{macrocode}

%\iffalse
%</samplechap1>
%\fi
% Some text for chapter 2:
%\iffalse
%<*samplechap2>
%\fi
%    \begin{macrocode}
\section{two}
more text in chapter two
%    \end{macrocode}

%\iffalse
%</samplechap2>
%\fi
%
% %%%%%%%%%%%%%%%%%%%%%%%%%%%%%%%%%%%%%%
% \paragraph{Part Include Files.}
%
% The include files are called |cdocspt3.tex| and |cdocspt4.tex|.
%
%\iffalse
%<*samplepart3|samplepart4>
%\fi

% Optional override for |\version| flag:
%    \begin{macrocode}
%%\providecommand{\version}{final}
%    \end{macrocode}

% Include the main document:
%    \begin{macrocode}
\input{childdoc.def}
\childdocby{cdocsamp}
%    \end{macrocode}

%\iffalse
%</samplepart3|samplepart4>
%\fi
%
%\iffalse
%<*samplepart3>
%\fi
% Some text for part 3:
%    \begin{macrocode}
some text in part three
%    \end{macrocode}

%\iffalse
%</samplepart3>
%\fi
% Some text for part 4:
%\iffalse
%<*samplepart4>
%\fi
%    \begin{macrocode}
more text in part four
%    \end{macrocode}

%\iffalse
%</samplepart4>
%\fi
%
% %%%%%%%%%%%%%%%%%%%%%%%%%%%%%%%%%%%%%%
% \paragraph{Forwarding for a Complete Draft.}
%
% The following forwarding file |cdocsdrf.tex|
% compiles the main document in draft mode:
%\iffalse
%<*sampledraft>
%\fi
%    \begin{macrocode}
\def\version{draft}
\input{childdoc.def}
\childdocforward{cdocsamp}
%    \end{macrocode}

%\iffalse
%</sampledraft>
%\fi
%
% %%%%%%%%%%%%%%%%%%%%%%%%%%%%%%%%%%%%%%
% \paragraph{Forwarding for Final Version of the Chapters.}
%
% The following forwarding files |cdocsfn1.tex| and |cdocsfn2.tex|
% (with identical content)
% compile the final versions of the child documents
% |cdocsch1.tex| and |cdocsch2.tex|, respectively:
%\iffalse
%<*samplefinal>
%\fi
%    \begin{macrocode}
\def\version{final}
\input{childdoc.def}
\childdocforwardprefix[cdocsamp]{cdocsfn}{cdocsch}
%    \end{macrocode}

%\iffalse
%</samplefinal>
%\fi
%
% %%%%%%%%%%%%%%%%%%%%%%%%%%%%%%%%%%%%%%
% \paragraph{Command Line Processing.}
%
% The following three command lines generate the output files
% |cdocscld|, |cdocscl1| and |cdocscl2|
% which should be identical to
% |cdocsdrf|, |cdocsch1| and |cdocsfn2|, respectively:
% \begin{center}
% \begin{tabular}{l}
% |latex -jobname cdocscld \|\\
% |  "\def\version{draft}\input{childdoc.def}\childdocforward{cdocsamp}"|\\
% |latex -jobname cdocscl1 \|\\
% |  "\input{childdoc.def}\childdocforward[cdocsamp]{cdocsch1}"|\\
% |latex -jobname cdocscl2 \|\\
% |  "\def\version{final}\input{childdoc.def}\childdocforward{cdocsch2}"|
% \end{tabular}
% \end{center}
% Note that the trailing backslash on each first line
% merely continues the input to the second line
% (for convenient cut ant paste).
% Furthermore, the command |latex| can be replaced by any
% of its alternative versions such as |pdflatex|.
%
% %%%%%%%%%%%%%%%%%%%%%%%%%%%%%%%%%%%%%%%%%%%%%%%%%%%%%%%%%%%%%%%%%%%%%%%%%%%%%%
% %%%%%%%%%%%%%%%%%%%%%%%%%%%%%%%%%%%%%%%%%%%%%%%%%%%%%%%%%%%%%%%%%%%%%%%%%%%%%%
% \section{Implementation}
%\iffalse
%<*package>
%\fi
%
% This section describes the definitions file |childdoc.def|.

% The definitions cannot be loaded using |\usepackage| or |\RequirePackage|
% which has a mechanism to prevent loading a style file more than once.
% When loading the definitions by means of |\input|
% multiple instances have to be prevented manually:
%\iffalse
%This code needs to be before the `\ProvidesFile' directive
%which is defined at the beginning of this file.
%Therefore it is also placed there and commented out here.
%</package>
%<*discard>
%\fi
%    \begin{macrocode}
\ifdefined\childdocmain\endinput\fi
%    \end{macrocode}
%\iffalse
%</discard>
%<*package>
%\fi
%
% \macro{\ifchilddoc}
% \macro{\ifchilddocmanual}
% The conditional |\ifchilddoc| tells whether a
% child (true) or main (false) document is being compiled.
% The conditional |\ifchilddocmanual| tells whether
% the |\includeonly| mechanism is used (false) or
% the selection of child files must be performed manually (true).
% The definitions initialise to false:
%    \begin{macrocode}
\newif\ifchilddoc
\newif\ifchilddocmanual
%    \end{macrocode}

% \macro{\childdocname}
% \macro{\childdocjob}
% The macro |\childdocname| stores the name of the main document
% to be compiled. The macro |\childdocjob| stores the name of
% the document on which the \LaTeX{} compiler was originally invoked.
% The content of |\jobname| cannot be compared
% to filenames specified in the source due to different catcodes.
% The following code rescans |\jobname|, stores the result
% in |\childdocname| and saves a copy in |\childdocjob|:
%    \begin{macrocode}
\edef\childdocname{\scantokens\expandafter{\jobname\noexpand}}
\let\childdocjob\childdocname
%    \end{macrocode}

% \macro{\childdocdisable}
% The macro |\childdocdisable| prevents the main file
% from being processed more than once.
% At this stage, the main document command |\childdocmain|
% is assumed to be called once again where it should do nothing.
% Any subsequent call to it should prevent
% a secondary processing of the main document
% It overwrites the forwarding commands
% |\childdocof| and |\childdocforward|
% with empty macros to prevent further inclusions of the main document:
%    \begin{macrocode}
\newcommand{\childdocdisable}
{
  \renewcommand{\childdocmain}[1]{\renewcommand{\childdocmain}[1]{\endinput}}
  \renewcommand{\childdocof}[1]{}
  \renewcommand{\childdocby}[2][]{}
  \renewcommand{\childdocforward}[2][]{}
  \renewcommand{\childdocdisable}{}
}
%    \end{macrocode}

% \macro{\childdocmain}
% The macro |\childdocmain| is to be called at the top of the main file
% with nothing or the main filename (without extension) as argument.
% First, it breaks loops.
% If the argument is not empty and does not match |\childdocname|
% (which is set by the first inclusion of |childdoc.def|),
% |\ifchilddoc| is set to true, |\includeonly| is applied to the child file
% and |\jobname| is set to the main file
% (for proper handling of |.aux| files):
%    \begin{macrocode}
\newcommand{\childdocmain}[1]
{
  \childdocdisable\childdocmain{}
  \if?#1?\else
    \begingroup
      \def\childdoctmp{#1}
      \ifx\childdoctmp\childdocname
        \def\childdoctmp{}
      \else
        \def\childdoctmp
        {
          \childdoctrue
          \includeonly{\childdocname}
          \def\childdocjob{#1}
          \def\jobname{#1}
        }
      \fi
      \expandafter
    \endgroup
    \childdoctmp
  \fi
}
%    \end{macrocode}

% \macro{\childdocof}
% The command |\childdocof| redirects
% compilation to the main file |#1|.
%    \begin{macrocode}
\newcommand{\childdocof}[1]
{
  \childdocdisable
  \childdoctrue
  \includeonly{\childdocname}
  \def\jobname{#1}
  \def\childdocjob{#1}
  \input{#1}
}
%    \end{macrocode}

% \macro{\childdocby}
% The command |\childdocby| ....
%    \begin{macrocode}
\newcommand{\childdocby}[2][]
{
  \childdocdisable
  \childdoctrue
  \childdocmanualtrue
  \if?#1?\else
    \def\jobname{#2}
  \fi
  \def\childdocjob{#2}
  \input{#2}
  \endinput
}
%    \end{macrocode}

% \macro{\childdocforward}
% The command |\childdocforward| redirects
% compilation to the main file or
% (if the optional argument is given) a child file.
% Parameters are set as if the main file
% or a child file starting with |\childdocof| was compiled.
% Then compilation is handed over to the main file:
%    \begin{macrocode}
\newcommand{\childdocforward}[2][]
{
  \begingroup
    \if?#1?
      \def\childdoctmp
      {
        \def\childdocname{#2}
        \def\childdocjob{#2}
        \def\jobname{#2}
        \input{#2}
        \endinput
      }
    \else
      \def\childdoctmp
      {
        \childdocdisable
        \def\childdocname{#2}
        \childdoctrue
        \includeonly{#2}
        \def\childdocjob{#1}
        \def\jobname{#1}
        \input{#1}
        \endinput
      }
    \fi
    \expandafter
  \endgroup
  \childdoctmp
}
%    \end{macrocode}

% \macro{\childdocforwardprefix}
% The command |\childdocforwardprefix| redirects
% compilation to the main or a child file by means of a pattern.
% The prefix |#1| in the current filename is replaced by |#2|
% and the suffix of the current filename is kept
% (it is assumed that the filename does not contain the substring `|~~~|'
% which is used as a delimiter).
% Compilation is handed over to the new file by |\childdocforward|:
%    \begin{macrocode}
\newcommand{\childdocforwardprefix}[3][]
{
  \begingroup
    \def\childdocextract #2##1~~~{\def\childdoctmp{\childdocforward[#1]{#3##1}}}
    \expandafter\childdocextract\childdocname~~~
    \expandafter
  \endgroup
  \childdoctmp
}
%    \end{macrocode}

% \macro{\childdoc}
% The deprecated macro |\childdoc| is a legacy version of |\childdocmain|:
%    \begin{macrocode}
\newcommand{\childdoc}{\childdocmain}
%    \end{macrocode}

% \macro{\childdocredirect}
% The deprecated macro |\childdocredirect| is a legacy version
% of |\childdocforward| and |\childdocforwardprefix|:
%    \begin{macrocode}
\newcommand{\childdocredirect}[2][]
{
  \begingroup
    \if?#1?
      \def\childdoctmp{\childdocforward{#2}}
    \else
      \def\childdoctmp{\childdocforwardprefix{#1}{#2}}
    \fi
    \expandafter
  \endgroup
  \childdoctmp
}
%    \end{macrocode}

%\iffalse
%</package>
%\fi
%
\endinput
|\\
|\childdocmain{|\textit{main}|}|\\
\end{tabular}
\end{center}
%
If |\jobname| does not match the argument \textit{main} of |\childdocmain|,
it is assumed that |\jobname| points to the child file to be compiled.
When using |\childdocmain| with the main file specified as argument,
it suffices to start a child file
with just |\input{|\textit{main}|}|
without loading of the package and using |\childdocof|.
If instead all processing is done
with the appropriate \textsf{childdoc} directives,
the argument of \textit{main} of |\childdocmain| can be empty.

An alternative version of the command line processing described
in \secref{sec:commandline} using the detection mechanism reads:
%
\begin{center}
|... -jobname "|\textit{target}|" "|[\textit{flags}]%
[|\def\jobname{|\textit{dest}|}|]|\input{|\textit{main}|}"|
\end{center}

%%%%%%%%%%%%%%%%%%%%%%%%%%%%%%%%%%%%%%%%%%%%%%%%%%%%%%%%%%%%%%%%%%%%%%%%%%%%%%%%
\subsection{Manual Code}
\label{sec:manual}

In case one cannot be certain whether the definitions file |childdoc.def|
is installed on the target \TeX{} distribution
and one prefers not to ship it,
it is conceivable to paste a few relevant commands into the sources.

To that end, drop all statements |% \iffalse
%
% childdoc.dtx Copyright (C) 2017-2018 Niklas Beisert
%
% This work may be distributed and/or modified under the
% conditions of the LaTeX Project Public License, either version 1.3
% of this license or (at your option) any later version.
% The latest version of this license is in
%   http://www.latex-project.org/lppl.txt
% and version 1.3 or later is part of all distributions of LaTeX
% version 2005/12/01 or later.
%
% This work has the LPPL maintenance status `maintained'.
%
% The Current Maintainer of this work is Niklas Beisert.
%
% This work consists of the files childdoc.dtx and childdoc.ins
% and the derived files childdoc.def and cdocsamp.tex with
% cdocsch1.tex, cdocsch2.tex, cdocsdrf.tex, cdocsfn1.tex, cdocsfn2.tex.
%
%<package>\ifdefined\childdocmain\endinput\fi
%<package>\ProvidesFile{childdoc.def}[2018/12/30 v2.0 child document driver]
%<samplemain>\ProvidesFile{cdocsamp.tex}[2018/12/30 v2.0 sample for childdoc]
%<*driver>
%\ProvidesFile{childdoc.drv}[2018/12/30 v2.0 childdoc reference manual file]
\PassOptionsToClass{10pt,a4paper}{article}
\documentclass{ltxdoc}

\usepackage[margin=35mm]{geometry}
\usepackage{hyperref}
\usepackage{hyperxmp}
\usepackage[usenames]{color}

\hypersetup{colorlinks=true}
\hypersetup{pdfstartview=FitH}
\hypersetup{pdfpagemode=UseNone}
\hypersetup{pdfsource={}}
\hypersetup{pdflang={en-UK}}
\hypersetup{pdfcopyright={Copyright 2017-2018 Niklas Beisert.
  This work may be distributed and/or modified under the
  conditions of the LaTeX Project Public License, either version 1.3
  of this license or (at your option) any later version.}}
\hypersetup{pdflicenseurl={http://www.latex-project.org/lppl.txt}}
\hypersetup{pdfcontactaddress={ETH Zurich, ITP, HIT K,
  Wolfgang-Pauli-Strasse 27}}
\hypersetup{pdfcontactpostcode={8093}}
\hypersetup{pdfcontactcity={Zurich}}
\hypersetup{pdfcontactcountry={Switzerland}}
\hypersetup{pdfcontactemail={nbeisert@itp.phys.ethz.ch}}
\hypersetup{pdfcontacturl={http://people.phys.ethz.ch/\xmptilde nbeisert/}}

\newcommand{\secref}[1]{\hyperref[#1]{section \ref*{#1}}}

\parskip1ex
\parindent0pt
\let\olditemize\itemize
\def\itemize{\olditemize\parskip0pt}

\begin{document}

\title{The \textsf{childdoc} Package}
\hypersetup{pdftitle={The childdoc Package}}
\author{Niklas Beisert\\[2ex]
  Institut f\"ur Theoretische Physik\\
  Eidgen\"ossische Technische Hochschule Z\"urich\\
  Wolfgang-Pauli-Strasse 27, 8093 Z\"urich, Switzerland\\[1ex]
  \href{mailto:nbeisert@itp.phys.ethz.ch}
  {\texttt{nbeisert@itp.phys.ethz.ch}}}
\hypersetup{pdfauthor={Niklas Beisert}}
\hypersetup{pdfsubject={Manual for the LaTeX2e Package childdoc}}
\date{30 December 2018, \textsf{v2.0}}
\maketitle

\begin{abstract}\noindent
\textsf{childdoc} is a \LaTeXe{} package
that enables the direct compilation
of document sections included by |\include|
to individual files.
\end{abstract}

\begingroup
\parskip0ex
\tableofcontents
\endgroup

%%%%%%%%%%%%%%%%%%%%%%%%%%%%%%%%%%%%%%%%%%%%%%%%%%%%%%%%%%%%%%%%%%%%%%%%%%%%%%%%
%%%%%%%%%%%%%%%%%%%%%%%%%%%%%%%%%%%%%%%%%%%%%%%%%%%%%%%%%%%%%%%%%%%%%%%%%%%%%%%%
\section{Introduction}

\LaTeX{} provides a mechanism to structure a large document (such as a book)
into a main file and several child files (containing the chapters)
using the |\include| command.
This mechanism is beneficial for documents
which span hundreds of pages in order to
make the source file(s) more manageable.
Moreover, compilation can be restricted to
selected child files by means of the |\includeonly| command.
The latter feature can be used to reduce the compilation time while editing
(this was significantly more useful in the earlier days of \LaTeX{})
or to generate a smaller document which is easier to navigate.
Another application of |\includeonly| is to generate
documents consisting of selected parts of the complete document.

However, there are a few drawbacks of the plain |\include| mechanism:
\begin{itemize}
\item
The child files cannot be compiled on their own,
they can only be compiled via the main file.
A naive editing environment
(such as a text editor with an option
to have the current file processed by \LaTeX)
may require one to switch to the main file before compiling;
attempting to compile the child file produces errors.
\item
The main file must be modified (each time)
to adjust the |\includeonly| command
to the present needs. This easily leaves the main file in a messy state.
\item
The generated document will always carry the filename
of the main document. This is inconvenient if
several child files are to be compiled and
to be kept for distribution.
\end{itemize}

The present package provides a simple interface
to make child files individually compilable by \LaTeX{}.
Compiling a child file then has the same effect as compiling
the main file with an |\includeonly| command
to select the appropriate child.
Moreover the generated document will carry the name of the child
rather than the main file.
This resolves all three above issues.

This feature is meant to make the editing of books,
thesis documents and lecture notes somewhat more convenient.
However, the package can also be used efficiently for
composing a series of documents (such as exercise sheets)
which are typically distributed individually.
It then assists the author in generating the individual documents
(potentially in different versions)
as well as a document containing the collected series.
Another application is in developing style files
or other kinds of included material
where compilation of the style file could redirect
to a sample or test file.

%%%%%%%%%%%%%%%%%%%%%%%%%%%%%%%%%%%%%%%%%%%%%%%%%%%%%%%%%%%%%%%%%%%%%%%%%%%%%%%%
%%%%%%%%%%%%%%%%%%%%%%%%%%%%%%%%%%%%%%%%%%%%%%%%%%%%%%%%%%%%%%%%%%%%%%%%%%%%%%%%
\section{Usage}

First of all, the package \textsf{childdoc} is \emph{not} a standard
\LaTeXe{} |.sty| style file! Therefore it needs to be invoked in
a non-standard way.

%%%%%%%%%%%%%%%%%%%%%%%%%%%%%%%%%%%%%%%%%%%%%%%%%%%%%%%%%%%%%%%%%%%%%%%%%%%%%%%%
\subsection{Included Files}
\label{sec:include}

%%%%%%%%%%%%%%%%%%%%%%%%%%%%%%%%%%%%%%%%
\DescribeMacro{\childdocmain}
To use the package, add the commands
\begin{center}
\begin{tabular}{l}
|\input{childdoc.def}|\\
|\childdocmain{}|\\
\end{tabular}
\end{center}
at the very top of the main \LaTeX{} file,
in particular \emph{before} the |\documentclass| statement!
The argument of |\childdocmain| should be left empty
(but it must be present).

%%%%%%%%%%%%%%%%%%%%%%%%%%%%%%%%%%%%%%%%
\DescribeMacro{\childdocof}
Furthermore, add the commands
\begin{center}
\begin{tabular}{l}
|\input{childdoc.def}|\\
|\childdocof{|\textit{main}|}|\\
\end{tabular}
\end{center}
at the top of every child file \textit{child}
which is included by |\include{|\textit{child}|}|
from within the main file
(or at least for those files to be compiled individually).
The argument \textit{main} must be the filename of the main file.

There are a couple of
considerations in setting up the main and child documents:

%%%%%%%%%%%%%%%%%%%%%%%%%%%%%%%%%%%%%%%%
\paragraph{Restrictions.}

Please note the following restrictions:
\begin{itemize}
\item
|\childdocmain| must be called with one argument \textit{main}
to ensure compatibility with earlier version of the package.
It must either be empty (|\childdocmain{}|)
or precisely match the filename of the main file in which it is specified.
See \secref{sec:detection} for further information.
\item
The filename \textit{main} must be specified without the |.tex| extension.
\item
The filename \textit{main} is case sensitive
(even in case-insensitive file systems)
due to internal string comparison.
\item
The argument \textit{main} should be fully expanded, it cannot be a macro.
\item
Subdirectories and special characters should be avoided in filenames.
\item
The command |\childdocmain{|\textit{main}|}| must be followed by a whitespace.
It should not be followed immediately by another command
or by a comment mark `|%|'.
This is because the \TeX{} parser reads the token immediately following
the argument of |\childdocmain| and puts it
at the beginning of every child section;
however, a white\-space is ignored.
\end{itemize}

%%%%%%%%%%%%%%%%%%%%%%%%%%%%%%%%%%%%%%%%
\paragraph{Content of Main File.}

It is advisable to place all content in the child files included by |\include|.
Any output contained in the main file will appear in all child documents
unless suppressed manually;
it cannot be suppressed automatically by the |\includeonly| directive
and thus should normally be avoided.
A method to include some content in the main file
by means of conditional processing is described in \secref{sec:conditional}.

%%%%%%%%%%%%%%%%%%%%%%%%%%%%%%%%%%%%%%%%
\paragraph{Page Numbering.}

When only a part of the document is compiled,
the appropriate numbering of pages
(as well as other status parameters)
is determined from the |.aux| files.
The latter contain information from previous passes.
However this information needs to propagate through
all intermediate child documents.
Therefore the page numbering in child documents may well
be inconsistent until the complete document is compiled at least once.

A useful (if unconventional) way to always ensure a consistent
page numbering is to restart the numbering in each child document
and denote the pages by `\textit{child}|.|\textit{page}'
where \textit{child} represents the chapter/section number of the child file.
This can be achieved by the command
|\numberwithin{page}{|\textit{child}|}|
of the \textsf{amsmath} package
where \textit{child} can be |chapter| or |section|
depending on the chosen structuring.
Alternatively, one can modify the macro |\thepage| appropriately
and reset the counter |page| at the start of each child file.

%%%%%%%%%%%%%%%%%%%%%%%%%%%%%%%%%%%%%%%%%%%%%%%%%%%%%%%%%%%%%%%%%%%%%%%%%%%%%%%%
\subsection{Conditional Processing}
\label{sec:conditional}

The package provides a mechanism to compile different versions
of a document. To customise the versions further some conditional processing
can come in handy to distinguish which version is being compiled.
The package provides two macros to describe the compilation context:

%%%%%%%%%%%%%%%%%%%%%%%%%%%%%%%%%%%%%%%%
\DescribeMacro{\ifchilddoc}
The conditional |\ifchilddoc| distinguishes between the compilation of
child documents and the main document:
%
\begin{center}
|\ifchilddoc |\textit{child-code}| |[|\||else |\textit{main-code}]| \||fi|
\end{center}

%%%%%%%%%%%%%%%%%%%%%%%%%%%%%%%%%%%%%%%%
\DescribeMacro{\childdocname}
\DescribeMacro{\childdocjob}
The macro |\childdocname| contains the filename (without extension)
of the main or child file being processed.
Note that |\childdocjob| will always contain the name of the main file.

%%%%%%%%%%%%%%%%%%%%%%%%%%%%%%%%%%%%%%%%
\paragraph{Title Page.}

Conditional processing can be used to include a title or banner page
in the main document when proper precautions are taken.
Importantly, the code in the main file should ensure that the page counter
(as well as other status parameters which are stored in the |.aux| files)
takes the same value after the conditional processing.
Otherwise the page numbers may take divergent values
depending on which part is compiled.

For example, a title page could be declared by:
%
\begin{center}
\begin{tabular}{l}
|\ifchilddoc\||else|\\
|\addtocounter{page}{-1}|\\
\textit{code for title page}\\
|\newpage|\\
|\||fi|
\end{tabular}
\end{center}
%
A banner page for the child documents can be generated by:
%
\begin{center}
\begin{tabular}{l}
|\ifchilddoc|\\
|\addtocounter{page}{-1}|\\
\textit{code for banner page}\\
|\newpage|\\
|\||fi|
\end{tabular}
\end{center}
%
Here one could write a message such as:
\begin{center}
|This is the part \childdocname{} of \childdocjob{}.|
\end{center}

%%%%%%%%%%%%%%%%%%%%%%%%%%%%%%%%%%%%%%%%%%%%%%%%%%%%%%%%%%%%%%%%%%%%%%%%%%%%%%%%
\subsection{Flags}
\label{sec:flags}

The package makes it easy to generate different versions
of the main or child documents.
To this end compilation flags can be defined
and assigned different default values.
They will be particularly useful in conjunction
with the forwarding mechanism described in \secref{sec:forward}.

For example, it may be useful to have a flag |\version|
which can be set to |draft| or |final|.
The document source will contain some conditional code
depending on the value of |\version|.
Suppose further, the flag should default to |final| for the main file
and to |draft| for child files
which is a natural assignment for editing the document.
This is achieved by placing the following code
in the preamble of the main document
(below the |\childdocmain| directive):
%
\begin{center}
\begin{tabular}{l}
|\ifchilddoc|\\
|\providecommand{\version}{draft}|\\
|\||else|\\
|\providecommand{\version}{final}|\\
|\||fi|
\end{tabular}
\end{center}
%
The definition by |\providecommand| makes sure
that previous definitions are not overwritten.
Further statements |\providecommand{\version}{...}|
can thus be added before the above code to override it.

For the main file, one might add a line
(between |\childdocmain| and the above block)
%
\begin{center}
|%\ifchilddoc\||else\providecommand{\version}{draft}\||fi|
\end{center}
%
which can be uncommented to produce a draft version.
Likewise one can add a line to the very top of a child file
(above the |\childdocof{|\textit{main}|}| directive)
%
\begin{center}
|%\providecommand{\version}{final}|
\end{center}
%
which can be uncommented to produce the final version of this child document.

%%%%%%%%%%%%%%%%%%%%%%%%%%%%%%%%%%%%%%%%%%%%%%%%%%%%%%%%%%%%%%%%%%%%%%%%%%%%%%%%
\subsection{Forwarding}
\label{sec:forward}

Different versions of the main or child documents
using compilation flags as described in \secref{sec:flags}
can be (permanently) stored in different files
for convenient compilation, viewing and distribution.
To this end, the package defines a command
to pass on compilation to a different file:

%%%%%%%%%%%%%%%%%%%%%%%%%%%%%%%%%%%%%%%%
\DescribeMacro{\childdocforward}
The command |\childdocforward| redirects processing to
another source file:
%
\begin{center}
\begin{tabular}{l}
|\input{childdoc.def}|\\
|\childdocforward[|\textit{main}|]{|\textit{dest}|}|\\
\end{tabular}
\end{center}
%
The argument \textit{dest} is the destination file
(without extension).
It should be the main file or one of the child files.
Note that further \textsf{childdoc} directives
such as |\childdocof| and |\childdocforward|
in the indicated file will be processed in this form.
The optional argument \textit{main}
passes on directly to the main file \textit{main}
while pretending to compile the child \textit{dest}.
This form behaves as if \textit{dest}
issues |\childdocof{|\textit{main}|}| right away,
and no further \textsf{childdoc} directives will be processed.

%%%%%%%%%%%%%%%%%%%%%%%%%%%%%%%%%%%%%%%%
\DescribeMacro{\...prefix}
In the alternative form |\childdocforwardprefix|,
%
\begin{center}
\begin{tabular}{l}
|\input{childdoc.def}|\\
|\childdocforwardprefix[|\textit{main}|]{|\textit{prefix}|}{|\textit{dest}|}|
\end{tabular}
\end{center}
%
the destination file is determined by a pattern
depending on the current file:
To make this work, the current file must be called
`{\textit{prefix}\hspace{0.2em}\textit{suffix}}'
with \textit{prefix} matching precisely the argument.
Processing is then passed on to the file
`{\textit{dest}\hspace{0.2em}\textit{suffix}}'.
Surely, the same effect is achieved by
directly specifying the
argument `{\textit{dest}\hspace{0.2em}\textit{suffix}}'
in the first form.
However, that requires to set up a different file
for each child. With the alternative form of the command
all these files can have exactly the same content
which simplifies setting them up and maintaining them.

For example, the following file |draft.tex|
with a compilation flag |\version| as described in \secref{sec:flags}
compiles the main document as a draft:
%
\begin{center}
\begin{tabular}{l}
|\def\version{draft}|\\
|\input{childdoc.def}|\\
|\childdocforward{|\textit{main}|}|
\end{tabular}
\end{center}
%
Likewise, the following files |final|\textit{nn}|.tex|
compile the final version of the child document
|child|\textit{nn}|.tex|:
%
\begin{center}
\begin{tabular}{l}
|\def\version{final}|\\
|\input{childdoc.def}|\\
|\childdocforwardprefix{final}{child}|
\end{tabular}
\end{center}
%

Note that when several versions of a main file and/or of each child file
are to be generated, it may be convenient to set up a |Makefile| or
shell script to automatise the process.

%%%%%%%%%%%%%%%%%%%%%%%%%%%%%%%%%%%%%%%%%%%%%%%%%%%%%%%%%%%%%%%%%%%%%%%%%%%%%%%%
\subsection{Command Line Processing}
\label{sec:commandline}

The effect of redirection files can also be achieved by invoking
the \LaTeX{} compiler with a more elaborate command line.
Most conveniently this should be done as part
of a shell script or a |Makefile|.

When using \textsf{childdoc} in the main file, the following
command lines effectively perform a redirection
(note that depending on the shell being used,
backslashes may have to be doubled: `|\|' $\to$ `|\\|'):
%
\begin{center}
|... -jobname "|\textit{target}|" |\\|"|[\textit{flags}]%
|\input{childdoc.def}\childdocforward[|\textit{main}|]{|\textit{dest}|}"|
\end{center}
%
Here \textit{target} is the name of the output file,
\textit{main} is the name of the main file
and \textit{dest} is the name of the main or child file to be processed
(all filenames without extensions).
The optional argument \textit{main} can be omitted
if \textit{main} matches \textit{dest}.
Optionally, compilation \textit{flags} can be defined via |\def| commands.
This command line makes the \TeX{} engine believe
it is compiling the file \textit{target}
whose content is specified as the latter parameter.
The provided code then forwards the processing to
\textit{main} or \textit{dest} as described in \secref{sec:forward}.

%%%%%%%%%%%%%%%%%%%%%%%%%%%%%%%%%%%%%%%%%%%%%%%%%%%%%%%%%%%%%%%%%%%%%%%%%%%%%%%%
\subsection{Include by Input}
\label{sec:input}

Including child documents by |\include| has some restrictions by design.
Most notably, the content of a child document always occupies
its own set of pages; pages cannot be shared between child documents.
Usually, this behaviour makes perfect sense
because each child document contain an essential part of the document.
However, in some situations it may be desirable to compose
a document from a collection of parts
without having mandatory page breaks between then.
For this case, the package
provides a mechanism to include parts
by |\input| which can also be processed individually.
However, by construction this mechanism
requires manual handling of the content to be output.

%%%%%%%%%%%%%%%%%%%%%%%%%%%%%%%%%%%%%%%%
\DescribeMacro{\ifchilddocmanual}
The main file should be prepared as usual, see \secref{sec:include}.
However, the document body must make a distinction
between processing of an individual part and of the main document, e.g.:
%
\begin{center}
\begin{tabular}{l}
|\ifchilddocmanual|\\
|\input{\childdocname}|\\
|\||else|\\
\textit{document body with }|\input{|\textit{part}|}|\\
|\||fi|
\end{tabular}
\end{center}
%
The conditional |\ifchilddocmanual| is true whenever
a part to be included by |\input| is being compiled,
and the name of the part is stored in |\childdocname|.

%%%%%%%%%%%%%%%%%%%%%%%%%%%%%%%%%%%%%%%%
\DescribeMacro{\childdocby}
Each part to be included by |\input| should start with:
%
\begin{center}
\begin{tabular}{l}
|\input{childdoc.def}|\\
|\childdocby{|\textit{main}|}|\\
\end{tabular}
\end{center}
%
The directive |\childdocby| is similar to |\childdocof|
described in \secref{sec:include},
but the subsequent selection of content must be done manually.
To that end, both |\ifchilddoc| and |\ifchilddocmanual|
will be true upon processing of a part,
and the name of the part is stored in |\childdocname|.
Note that |\jobname| will be set to the filename of the current part
so that each part receives an individual |.aux| file
that does not interfere with the |.aux| file(s) of the main document.
This behaviour can be altered by the alternative form
|\childdocby[*]{|\textit{main}|}| (with a non-empty optional argument)
which uses the |.aux| file of the main document
by setting |\jobname| to \textit{main}.

%%%%%%%%%%%%%%%%%%%%%%%%%%%%%%%%%%%%%%%%%%%%%%%%%%%%%%%%%%%%%%%%%%%%%%%%%%%%%%%%
\subsection{Driver Development}
\label{sec:driver}

The \textsf{childdoc} mechanism can also be use for the development
of definition files such as \LaTeX{} styles or classes.
This case differs from the above setup with multiple parts
included by |\include| in that no |\includeonly| should be invoked.
This can be achieved by starting the include file
(before |\ProvidesPackage|) with:
%
\begin{center}
\begin{tabular}{l}
|\input{childdoc.def}|\\
|\childdocforward{|\textit{main}|}|\\
\end{tabular}
\end{center}
%
or alternatively with:
%
\begin{center}
\begin{tabular}{l}
|\input{childdoc.def}|\\
|\childdocby{|\textit{main}|}|\\
\end{tabular}
\end{center}
%
Both forms have slightly different effects as described above.
The main file is prepared as usual, see \secref{sec:include}.

%%%%%%%%%%%%%%%%%%%%%%%%%%%%%%%%%%%%%%%%%%%%%%%%%%%%%%%%%%%%%%%%%%%%%%%%%%%%%%%%
\subsection{Legacy Detection}
\label{sec:detection}

The directive |\childdocmain| in the main file can detect
whether the complete document or merely a child is to be compiled
even without using the directive |\childdocof|.
This method is deprecated because it is less robust
and there is no compelling reason to use it;
it is merely provided for backward compatibility
and it may be removed in future versions.

If the detection mechanism is to be used,
it is mandatory to correctly specify
the filename of the main file as the argument of |\childdocmain|:
%
\begin{center}
\begin{tabular}{l}
|\input{childdoc.def}|\\
|\childdocmain{|\textit{main}|}|\\
\end{tabular}
\end{center}
%
If |\jobname| does not match the argument \textit{main} of |\childdocmain|,
it is assumed that |\jobname| points to the child file to be compiled.
When using |\childdocmain| with the main file specified as argument,
it suffices to start a child file
with just |\input{|\textit{main}|}|
without loading of the package and using |\childdocof|.
If instead all processing is done
with the appropriate \textsf{childdoc} directives,
the argument of \textit{main} of |\childdocmain| can be empty.

An alternative version of the command line processing described
in \secref{sec:commandline} using the detection mechanism reads:
%
\begin{center}
|... -jobname "|\textit{target}|" "|[\textit{flags}]%
[|\def\jobname{|\textit{dest}|}|]|\input{|\textit{main}|}"|
\end{center}

%%%%%%%%%%%%%%%%%%%%%%%%%%%%%%%%%%%%%%%%%%%%%%%%%%%%%%%%%%%%%%%%%%%%%%%%%%%%%%%%
\subsection{Manual Code}
\label{sec:manual}

In case one cannot be certain whether the definitions file |childdoc.def|
is installed on the target \TeX{} distribution
and one prefers not to ship it,
it is conceivable to paste a few relevant commands into the sources.

To that end, drop all statements |\input{childdoc.def}|
and perform the replacements as outlined below.
Instead of |\childdocmain{|\textit{main}|}| add the following code
to the top of the main file:
%
\begin{center}
\begin{tabular}{l}
|\||ifdefined\childdocname\endinput\||fi\newif\ifchilddoc|\\
|\edef\childdocname{\scantokens\expandafter{\jobname\noexpand}}|\\
|\def\childdocmain{|\textit{main}|}\||ifx\childdocmain\childdocname\||else|\\
|\childdoctrue\includeonly{\childdocname}\let\jobname\childdocmain\||fi|\\
\end{tabular}
\end{center}
%
Instead of |\childdocof{|\textit{main}|}| just include the main file
at the top of each child file:
%
\begin{center}
|\input{|\textit{main}|}|
\end{center}
%
A simple redirection |\childdocforward{|\textit{dest}|}| is achieved by:
%
\begin{center}
|\def\jobname{|\textit{dest}|}\input{\jobname}|
\end{center}
%
The redirection with prefix
|\childdocforwardprefix[|\textit{prefix}|]{|\textit{dest}|}|
is accomplished by:
%
\begin{center}
\begin{tabular}{l}
|{\edef\jobname{\scantokens\expandafter{\jobname\noexpand}}|\\
|\def\redirectjob |\textit{prefix}|#1~~~{\gdef\jobname{|\textit{dest}|#1}}|\\
|\expandafter\redirectjob\jobname~~~}\input{\jobname}|
\end{tabular}
\end{center}

In an alternative approach,
child documents can be compiled by a specific command line
without additional code or specific definitions:
%
\begin{center}
|... -jobname "|\textit{target}|" "|[\textit{flags}]%
|\includeonly{|\textit{dest}|}\input{|\textit{main}|}"|
\end{center}
%

%%%%%%%%%%%%%%%%%%%%%%%%%%%%%%%%%%%%%%%%%%%%%%%%%%%%%%%%%%%%%%%%%%%%%%%%%%%%%%%%
%%%%%%%%%%%%%%%%%%%%%%%%%%%%%%%%%%%%%%%%%%%%%%%%%%%%%%%%%%%%%%%%%%%%%%%%%%%%%%%%
\section{Information}

%%%%%%%%%%%%%%%%%%%%%%%%%%%%%%%%%%%%%%%%%%%%%%%%%%%%%%%%%%%%%%%%%%%%%%%%%%%%%%%%
\subsection{Copyright}

Copyright \copyright{} 2017--2018 Niklas Beisert

This work may be distributed and/or modified under the
conditions of the \LaTeX{} Project Public License, either version 1.3
of this license or (at your option) any later version.
The latest version of this license is in
  \url{http://www.latex-project.org/lppl.txt}
and version 1.3 or later is part of all distributions of \LaTeX{}
version 2005/12/01 or later.

This work has the LPPL maintenance status `maintained'.

The Current Maintainer of this work is Niklas Beisert.

This work consists of the files |README.txt|, |childdoc.ins| and |childdoc.dtx|
as well as the derived files |childdoc.def|, |cdocsamp.tex|
with |cdocsch1.tex|, |cdocsch2.tex|, |cdocspt3.tex|, |cdocspt4.tex|,
|cdocsdrf.tex|, |cdocsfn1.tex|, |cdocsfn2.tex|
as well as |childdoc.pdf|.

%%%%%%%%%%%%%%%%%%%%%%%%%%%%%%%%%%%%%%%%%%%%%%%%%%%%%%%%%%%%%%%%%%%%%%%%%%%%%%%%
\subsection{Files and Installation}

The package consists of the files:
%
\begin{center}
\begin{tabular}{ll}
    |README.txt|   & readme file \\
    |childdoc.ins| & installation file \\
    |childdoc.dtx| & source file \\
    |childdoc.def| & definition file \\
    |cdocsamp.tex| & sample main file \\
    |cdocsch1.tex| & sample include file \\
    |cdocsch2.tex| & sample include file \\
    |cdocspt3.tex| & sample part file \\
    |cdocspt4.tex| & sample part file \\
    |cdocsdrf.tex| & sample redirection file \\
    |cdocsfn1.tex| & sample redirection file \\
    |cdocsfn2.tex| & sample redirection file \\
    |childdoc.pdf| & manual
\end{tabular}
\end{center}
%
The distribution consists of the files
|README.txt|, |childdoc.ins| and |childdoc.dtx|.
%
\begin{itemize}
\item
Run (pdf)\LaTeX{} on |childdoc.dtx|
to compile the manual |childdoc.pdf| (this file).
\item
Run \LaTeX{} on |childdoc.ins| to create the definitions file |childdoc.def|
and the sample |cdocsamp.tex| with include files
|cdocsch1.tex|, |cdocsch2.tex|, |cdocspt3.tex|, |cdocspt4.tex|,
|cdocsdrf.tex|, |cdocsfn1.tex|, |cdocsfn2.tex|.
Then copy the file |childdoc.def| to an appropriate directory of your \LaTeX{}
distribution, e.g.\ \textit{texmf-root}|/tex/latex/childdoc|.
\end{itemize}

%%%%%%%%%%%%%%%%%%%%%%%%%%%%%%%%%%%%%%%%%%%%%%%%%%%%%%%%%%%%%%%%%%%%%%%%%%%%%%%%
\subsection{Related CTAN Packages}

There are several other packages which offer a similar functionality:
%
\begin{itemize}
\item
The packages
\href{http://ctan.org/pkg/docmute}{\textsf{docmute}},
\href{http://ctan.org/pkg/includex}{\textsf{includex}} and
\href{http://ctan.org/pkg/standalone}{\textsf{standalone}}
provide commands to include only the document body of
a child file thus allowing both files to be compiled individually.
\item
The packages \href{http://ctan.org/pkg/subdocs}{\textsf{subdocs}}
and \href{http://ctan.org/pkg/subfiles}{\textsf{subfiles}}
provide structures in which the main and child documents can be
encapsulated and allowing them to be compiled individually.
The inclusion mechanism is different from the conventional |\include|.
\item
The package \href{http://ctan.org/pkg/combine}{\textsf{combine}}
is an elaborate solution to combine several documents into one.
\end{itemize}
%
See also the CTAN topic \href{http://ctan.org/topic/subdocs}{\textsf{subdocs}}
for further related packages.
The present package differs from the above solutions in that
a document structure constructed with the conventional |\include| mechanism
just needs two extra commands at the top of every file
such that all constituent files can be compiled individually.

%%%%%%%%%%%%%%%%%%%%%%%%%%%%%%%%%%%%%%%%%%%%%%%%%%%%%%%%%%%%%%%%%%%%%%%%%%%%%%%%
%\subsection{Feature Suggestions}
%
%The following is a list of features which may be useful for future
%versions of this package:
%%
%\begin{itemize}
%\item
%\ldots
%\end{itemize}

%%%%%%%%%%%%%%%%%%%%%%%%%%%%%%%%%%%%%%%%%%%%%%%%%%%%%%%%%%%%%%%%%%%%%%%%%%%%%%%%
\subsection{Revision History}

%%%%%%%%%%%%%%%%%%%%%%%%%%%%%%%%%%%%%%%%
\paragraph{v2.0:} 2018/12/30

\begin{itemize}
\item
immediate forward processing
\item
added |\childdocby| mechanism
\item
manual restructured
\end{itemize}

%%%%%%%%%%%%%%%%%%%%%%%%%%%%%%%%%%%%%%%%
\paragraph{v1.6:} 2018/01/17

\begin{itemize}
\item
application for development of include files
\item
corrections to manual
\end{itemize}

%%%%%%%%%%%%%%%%%%%%%%%%%%%%%%%%%%%%%%%%
\paragraph{v1.5:} 2017/05/21

\begin{itemize}
\item
more complete structuring introduced
\item
|\childdocof| introduced
\item
|\childdoc| renamed to |\childdocmain|
\item
|\childredirect| renamed to |\childdocforward| and |\childdocforwardprefix|
and functionality expanded
\end{itemize}

%%%%%%%%%%%%%%%%%%%%%%%%%%%%%%%%%%%%%%%%
\paragraph{v1.0:} 2017/04/27

\begin{itemize}
\item
manual and install package
\item
first version published on CTAN
\end{itemize}

%%%%%%%%%%%%%%%%%%%%%%%%%%%%%%%%%%%%%%%%
\paragraph{v0.6:} 2017/04/26

\begin{itemize}
\item
redirection mechanism added
\end{itemize}

%%%%%%%%%%%%%%%%%%%%%%%%%%%%%%%%%%%%%%%%
\paragraph{v0.5:} 2017/04/26

\begin{itemize}
\item
functionality in definition file
\end{itemize}


%%%%%%%%%%%%%%%%%%%%%%%%%%%%%%%%%%%%%%%%%%%%%%%%%%%%%%%%%%%%%%%%%%%%%%%%%%%%%%%%
%%%%%%%%%%%%%%%%%%%%%%%%%%%%%%%%%%%%%%%%%%%%%%%%%%%%%%%%%%%%%%%%%%%%%%%%%%%%%%%%
%%%%%%%%%%%%%%%%%%%%%%%%%%%%%%%%%%%%%%%%%%%%%%%%%%%%%%%%%%%%%%%%%%%%%%%%%%%%%%%%
\appendix

\settowidth\MacroIndent{\rmfamily\scriptsize 000\ }

 \DocInput{childdoc.dtx}

\end{document}
%</driver>
% \fi
%
% %%%%%%%%%%%%%%%%%%%%%%%%%%%%%%%%%%%%%%%%%%%%%%%%%%%%%%%%%%%%%%%%%%%%%%%%%%%%%%
% %%%%%%%%%%%%%%%%%%%%%%%%%%%%%%%%%%%%%%%%%%%%%%%%%%%%%%%%%%%%%%%%%%%%%%%%%%%%%%
% \section{Sample}
%\iffalse
%<*samplemain>
%\fi
%
% The following presents a sample document
% with two chapters, two parts, a title page,
% a compile flag as well as three forwarding files to set the flag.
% It consists of eight |.tex| files:
% \begin{center}
% \begin{tabular}{ll}
% |cdocsamp.tex|&main file\\
% |cdocsch1.tex|&include file for chapter 1\\
% |cdocsch2.tex|&include file for chapter 2\\
% |cdocspt3.tex|&include file for part 3\\
% |cdocspt4.tex|&include file for part 4\\
% |cdocsdrf.tex|&forwarding file for main file in draft mode\\
% |cdocsfi1.tex|&forwarding file for final version of chapter 1\\
% |cdocsfi2.tex|&forwarding file for final version of chapter 2\\
% \end{tabular}
% \end{center}
% Each of the eight files can be compiled directly by the \LaTeX{} compiler.
%
% %%%%%%%%%%%%%%%%%%%%%%%%%%%%%%%%%%%%%%
% \paragraph{Main File.}
%
% The main file is called |cdocsamp.tex|.
%
% Load the \textsf{childdoc} definitions and
% declare the filename for the main document:
%    \begin{macrocode}
\input{childdoc.def}
\childdocmain{}
%    \end{macrocode}

% Optional override for |\version| flag:
%    \begin{macrocode}
%%\ifchilddoc\else\providecommand{\version}{draft}\fi
%    \end{macrocode}

% Define the default values for the |\version| flag
% (|final| for the main file and |draft| for childs):
%    \begin{macrocode}
\ifchilddoc
\providecommand{\version}{draft}
\else
\providecommand{\version}{final}
\fi
%    \end{macrocode}

% Load the standard document class:
%    \begin{macrocode}
\documentclass[12pt]{article}
%    \end{macrocode}

% Start the document body:
%    \begin{macrocode}
\begin{document}
%    \end{macrocode}

% Declare a title page.
% Print title, part of document being processed and version flag:
%    \begin{macrocode}
\addtocounter{page}{-1}
\begin{center}
{\LARGE\bfseries{}childdoc example\par}
\vspace{1cm}
\ifchilddoc
\ifchilddocmanual part\else chapter\fi:
`\childdocname' of `\childdocjob'\par
\else
main document: `\childdocjob'\par
\fi
version: \version\par
\end{center}
\newpage
%    \end{macrocode}

% Manually include selected file,
% otherwise process as usual:
%    \begin{macrocode}
\ifchilddocmanual
\section*{part `\childdocname'}
\input{\childdocname}
\else
%    \end{macrocode}

% Include the two chapters:
%    \begin{macrocode}
\include{cdocsch1}
\include{cdocsch2}
%    \end{macrocode}

% Include the two parts unless only chapters should be displayed:
%    \begin{macrocode}
\ifchilddoc\else
\section{part three}
\input{cdocspt3}
\section{part four}
\input{cdocspt4}
\fi
%    \end{macrocode}

% Process as usual until here:
%    \begin{macrocode}
\fi
%    \end{macrocode}

% End of document body:
%    \begin{macrocode}
\end{document}
%    \end{macrocode}
%\iffalse
%</samplemain>
%\fi
%
% %%%%%%%%%%%%%%%%%%%%%%%%%%%%%%%%%%%%%%
% \paragraph{Chapter Include Files.}
%
% The include files are called |cdocsch1.tex| and |cdocsch2.tex|.
%
%\iffalse
%<*samplechap1|samplechap2>
%\fi

% Optional override for |\version| flag:
%    \begin{macrocode}
%%\providecommand{\version}{final}
%    \end{macrocode}

% Include the main document:
%    \begin{macrocode}
\input{childdoc.def}
\childdocof{cdocsamp}
%    \end{macrocode}

%\iffalse
%</samplechap1|samplechap2>
%\fi
%
%\iffalse
%<*samplechap1>
%\fi
% Some text for chapter 1:
%    \begin{macrocode}
\section{one}
some text in chapter one
%    \end{macrocode}

%\iffalse
%</samplechap1>
%\fi
% Some text for chapter 2:
%\iffalse
%<*samplechap2>
%\fi
%    \begin{macrocode}
\section{two}
more text in chapter two
%    \end{macrocode}

%\iffalse
%</samplechap2>
%\fi
%
% %%%%%%%%%%%%%%%%%%%%%%%%%%%%%%%%%%%%%%
% \paragraph{Part Include Files.}
%
% The include files are called |cdocspt3.tex| and |cdocspt4.tex|.
%
%\iffalse
%<*samplepart3|samplepart4>
%\fi

% Optional override for |\version| flag:
%    \begin{macrocode}
%%\providecommand{\version}{final}
%    \end{macrocode}

% Include the main document:
%    \begin{macrocode}
\input{childdoc.def}
\childdocby{cdocsamp}
%    \end{macrocode}

%\iffalse
%</samplepart3|samplepart4>
%\fi
%
%\iffalse
%<*samplepart3>
%\fi
% Some text for part 3:
%    \begin{macrocode}
some text in part three
%    \end{macrocode}

%\iffalse
%</samplepart3>
%\fi
% Some text for part 4:
%\iffalse
%<*samplepart4>
%\fi
%    \begin{macrocode}
more text in part four
%    \end{macrocode}

%\iffalse
%</samplepart4>
%\fi
%
% %%%%%%%%%%%%%%%%%%%%%%%%%%%%%%%%%%%%%%
% \paragraph{Forwarding for a Complete Draft.}
%
% The following forwarding file |cdocsdrf.tex|
% compiles the main document in draft mode:
%\iffalse
%<*sampledraft>
%\fi
%    \begin{macrocode}
\def\version{draft}
\input{childdoc.def}
\childdocforward{cdocsamp}
%    \end{macrocode}

%\iffalse
%</sampledraft>
%\fi
%
% %%%%%%%%%%%%%%%%%%%%%%%%%%%%%%%%%%%%%%
% \paragraph{Forwarding for Final Version of the Chapters.}
%
% The following forwarding files |cdocsfn1.tex| and |cdocsfn2.tex|
% (with identical content)
% compile the final versions of the child documents
% |cdocsch1.tex| and |cdocsch2.tex|, respectively:
%\iffalse
%<*samplefinal>
%\fi
%    \begin{macrocode}
\def\version{final}
\input{childdoc.def}
\childdocforwardprefix[cdocsamp]{cdocsfn}{cdocsch}
%    \end{macrocode}

%\iffalse
%</samplefinal>
%\fi
%
% %%%%%%%%%%%%%%%%%%%%%%%%%%%%%%%%%%%%%%
% \paragraph{Command Line Processing.}
%
% The following three command lines generate the output files
% |cdocscld|, |cdocscl1| and |cdocscl2|
% which should be identical to
% |cdocsdrf|, |cdocsch1| and |cdocsfn2|, respectively:
% \begin{center}
% \begin{tabular}{l}
% |latex -jobname cdocscld \|\\
% |  "\def\version{draft}\input{childdoc.def}\childdocforward{cdocsamp}"|\\
% |latex -jobname cdocscl1 \|\\
% |  "\input{childdoc.def}\childdocforward[cdocsamp]{cdocsch1}"|\\
% |latex -jobname cdocscl2 \|\\
% |  "\def\version{final}\input{childdoc.def}\childdocforward{cdocsch2}"|
% \end{tabular}
% \end{center}
% Note that the trailing backslash on each first line
% merely continues the input to the second line
% (for convenient cut ant paste).
% Furthermore, the command |latex| can be replaced by any
% of its alternative versions such as |pdflatex|.
%
% %%%%%%%%%%%%%%%%%%%%%%%%%%%%%%%%%%%%%%%%%%%%%%%%%%%%%%%%%%%%%%%%%%%%%%%%%%%%%%
% %%%%%%%%%%%%%%%%%%%%%%%%%%%%%%%%%%%%%%%%%%%%%%%%%%%%%%%%%%%%%%%%%%%%%%%%%%%%%%
% \section{Implementation}
%\iffalse
%<*package>
%\fi
%
% This section describes the definitions file |childdoc.def|.

% The definitions cannot be loaded using |\usepackage| or |\RequirePackage|
% which has a mechanism to prevent loading a style file more than once.
% When loading the definitions by means of |\input|
% multiple instances have to be prevented manually:
%\iffalse
%This code needs to be before the `\ProvidesFile' directive
%which is defined at the beginning of this file.
%Therefore it is also placed there and commented out here.
%</package>
%<*discard>
%\fi
%    \begin{macrocode}
\ifdefined\childdocmain\endinput\fi
%    \end{macrocode}
%\iffalse
%</discard>
%<*package>
%\fi
%
% \macro{\ifchilddoc}
% \macro{\ifchilddocmanual}
% The conditional |\ifchilddoc| tells whether a
% child (true) or main (false) document is being compiled.
% The conditional |\ifchilddocmanual| tells whether
% the |\includeonly| mechanism is used (false) or
% the selection of child files must be performed manually (true).
% The definitions initialise to false:
%    \begin{macrocode}
\newif\ifchilddoc
\newif\ifchilddocmanual
%    \end{macrocode}

% \macro{\childdocname}
% \macro{\childdocjob}
% The macro |\childdocname| stores the name of the main document
% to be compiled. The macro |\childdocjob| stores the name of
% the document on which the \LaTeX{} compiler was originally invoked.
% The content of |\jobname| cannot be compared
% to filenames specified in the source due to different catcodes.
% The following code rescans |\jobname|, stores the result
% in |\childdocname| and saves a copy in |\childdocjob|:
%    \begin{macrocode}
\edef\childdocname{\scantokens\expandafter{\jobname\noexpand}}
\let\childdocjob\childdocname
%    \end{macrocode}

% \macro{\childdocdisable}
% The macro |\childdocdisable| prevents the main file
% from being processed more than once.
% At this stage, the main document command |\childdocmain|
% is assumed to be called once again where it should do nothing.
% Any subsequent call to it should prevent
% a secondary processing of the main document
% It overwrites the forwarding commands
% |\childdocof| and |\childdocforward|
% with empty macros to prevent further inclusions of the main document:
%    \begin{macrocode}
\newcommand{\childdocdisable}
{
  \renewcommand{\childdocmain}[1]{\renewcommand{\childdocmain}[1]{\endinput}}
  \renewcommand{\childdocof}[1]{}
  \renewcommand{\childdocby}[2][]{}
  \renewcommand{\childdocforward}[2][]{}
  \renewcommand{\childdocdisable}{}
}
%    \end{macrocode}

% \macro{\childdocmain}
% The macro |\childdocmain| is to be called at the top of the main file
% with nothing or the main filename (without extension) as argument.
% First, it breaks loops.
% If the argument is not empty and does not match |\childdocname|
% (which is set by the first inclusion of |childdoc.def|),
% |\ifchilddoc| is set to true, |\includeonly| is applied to the child file
% and |\jobname| is set to the main file
% (for proper handling of |.aux| files):
%    \begin{macrocode}
\newcommand{\childdocmain}[1]
{
  \childdocdisable\childdocmain{}
  \if?#1?\else
    \begingroup
      \def\childdoctmp{#1}
      \ifx\childdoctmp\childdocname
        \def\childdoctmp{}
      \else
        \def\childdoctmp
        {
          \childdoctrue
          \includeonly{\childdocname}
          \def\childdocjob{#1}
          \def\jobname{#1}
        }
      \fi
      \expandafter
    \endgroup
    \childdoctmp
  \fi
}
%    \end{macrocode}

% \macro{\childdocof}
% The command |\childdocof| redirects
% compilation to the main file |#1|.
%    \begin{macrocode}
\newcommand{\childdocof}[1]
{
  \childdocdisable
  \childdoctrue
  \includeonly{\childdocname}
  \def\jobname{#1}
  \def\childdocjob{#1}
  \input{#1}
}
%    \end{macrocode}

% \macro{\childdocby}
% The command |\childdocby| ....
%    \begin{macrocode}
\newcommand{\childdocby}[2][]
{
  \childdocdisable
  \childdoctrue
  \childdocmanualtrue
  \if?#1?\else
    \def\jobname{#2}
  \fi
  \def\childdocjob{#2}
  \input{#2}
  \endinput
}
%    \end{macrocode}

% \macro{\childdocforward}
% The command |\childdocforward| redirects
% compilation to the main file or
% (if the optional argument is given) a child file.
% Parameters are set as if the main file
% or a child file starting with |\childdocof| was compiled.
% Then compilation is handed over to the main file:
%    \begin{macrocode}
\newcommand{\childdocforward}[2][]
{
  \begingroup
    \if?#1?
      \def\childdoctmp
      {
        \def\childdocname{#2}
        \def\childdocjob{#2}
        \def\jobname{#2}
        \input{#2}
        \endinput
      }
    \else
      \def\childdoctmp
      {
        \childdocdisable
        \def\childdocname{#2}
        \childdoctrue
        \includeonly{#2}
        \def\childdocjob{#1}
        \def\jobname{#1}
        \input{#1}
        \endinput
      }
    \fi
    \expandafter
  \endgroup
  \childdoctmp
}
%    \end{macrocode}

% \macro{\childdocforwardprefix}
% The command |\childdocforwardprefix| redirects
% compilation to the main or a child file by means of a pattern.
% The prefix |#1| in the current filename is replaced by |#2|
% and the suffix of the current filename is kept
% (it is assumed that the filename does not contain the substring `|~~~|'
% which is used as a delimiter).
% Compilation is handed over to the new file by |\childdocforward|:
%    \begin{macrocode}
\newcommand{\childdocforwardprefix}[3][]
{
  \begingroup
    \def\childdocextract #2##1~~~{\def\childdoctmp{\childdocforward[#1]{#3##1}}}
    \expandafter\childdocextract\childdocname~~~
    \expandafter
  \endgroup
  \childdoctmp
}
%    \end{macrocode}

% \macro{\childdoc}
% The deprecated macro |\childdoc| is a legacy version of |\childdocmain|:
%    \begin{macrocode}
\newcommand{\childdoc}{\childdocmain}
%    \end{macrocode}

% \macro{\childdocredirect}
% The deprecated macro |\childdocredirect| is a legacy version
% of |\childdocforward| and |\childdocforwardprefix|:
%    \begin{macrocode}
\newcommand{\childdocredirect}[2][]
{
  \begingroup
    \if?#1?
      \def\childdoctmp{\childdocforward{#2}}
    \else
      \def\childdoctmp{\childdocforwardprefix{#1}{#2}}
    \fi
    \expandafter
  \endgroup
  \childdoctmp
}
%    \end{macrocode}

%\iffalse
%</package>
%\fi
%
\endinput
|
and perform the replacements as outlined below.
Instead of |\childdocmain{|\textit{main}|}| add the following code
to the top of the main file:
%
\begin{center}
\begin{tabular}{l}
|\||ifdefined\childdocname\endinput\||fi\newif\ifchilddoc|\\
|\edef\childdocname{\scantokens\expandafter{\jobname\noexpand}}|\\
|\def\childdocmain{|\textit{main}|}\||ifx\childdocmain\childdocname\||else|\\
|\childdoctrue\includeonly{\childdocname}\let\jobname\childdocmain\||fi|\\
\end{tabular}
\end{center}
%
Instead of |\childdocof{|\textit{main}|}| just include the main file
at the top of each child file:
%
\begin{center}
|\input{|\textit{main}|}|
\end{center}
%
A simple redirection |\childdocforward{|\textit{dest}|}| is achieved by:
%
\begin{center}
|\def\jobname{|\textit{dest}|}\input{\jobname}|
\end{center}
%
The redirection with prefix
|\childdocforwardprefix[|\textit{prefix}|]{|\textit{dest}|}|
is accomplished by:
%
\begin{center}
\begin{tabular}{l}
|{\edef\jobname{\scantokens\expandafter{\jobname\noexpand}}|\\
|\def\redirectjob |\textit{prefix}|#1~~~{\gdef\jobname{|\textit{dest}|#1}}|\\
|\expandafter\redirectjob\jobname~~~}\input{\jobname}|
\end{tabular}
\end{center}

In an alternative approach,
child documents can be compiled by a specific command line
without additional code or specific definitions:
%
\begin{center}
|... -jobname "|\textit{target}|" "|[\textit{flags}]%
|\includeonly{|\textit{dest}|}\input{|\textit{main}|}"|
\end{center}
%

%%%%%%%%%%%%%%%%%%%%%%%%%%%%%%%%%%%%%%%%%%%%%%%%%%%%%%%%%%%%%%%%%%%%%%%%%%%%%%%%
%%%%%%%%%%%%%%%%%%%%%%%%%%%%%%%%%%%%%%%%%%%%%%%%%%%%%%%%%%%%%%%%%%%%%%%%%%%%%%%%
\section{Information}

%%%%%%%%%%%%%%%%%%%%%%%%%%%%%%%%%%%%%%%%%%%%%%%%%%%%%%%%%%%%%%%%%%%%%%%%%%%%%%%%
\subsection{Copyright}

Copyright \copyright{} 2017--2018 Niklas Beisert

This work may be distributed and/or modified under the
conditions of the \LaTeX{} Project Public License, either version 1.3
of this license or (at your option) any later version.
The latest version of this license is in
  \url{http://www.latex-project.org/lppl.txt}
and version 1.3 or later is part of all distributions of \LaTeX{}
version 2005/12/01 or later.

This work has the LPPL maintenance status `maintained'.

The Current Maintainer of this work is Niklas Beisert.

This work consists of the files |README.txt|, |childdoc.ins| and |childdoc.dtx|
as well as the derived files |childdoc.def|, |cdocsamp.tex|
with |cdocsch1.tex|, |cdocsch2.tex|, |cdocspt3.tex|, |cdocspt4.tex|,
|cdocsdrf.tex|, |cdocsfn1.tex|, |cdocsfn2.tex|
as well as |childdoc.pdf|.

%%%%%%%%%%%%%%%%%%%%%%%%%%%%%%%%%%%%%%%%%%%%%%%%%%%%%%%%%%%%%%%%%%%%%%%%%%%%%%%%
\subsection{Files and Installation}

The package consists of the files:
%
\begin{center}
\begin{tabular}{ll}
    |README.txt|   & readme file \\
    |childdoc.ins| & installation file \\
    |childdoc.dtx| & source file \\
    |childdoc.def| & definition file \\
    |cdocsamp.tex| & sample main file \\
    |cdocsch1.tex| & sample include file \\
    |cdocsch2.tex| & sample include file \\
    |cdocspt3.tex| & sample part file \\
    |cdocspt4.tex| & sample part file \\
    |cdocsdrf.tex| & sample redirection file \\
    |cdocsfn1.tex| & sample redirection file \\
    |cdocsfn2.tex| & sample redirection file \\
    |childdoc.pdf| & manual
\end{tabular}
\end{center}
%
The distribution consists of the files
|README.txt|, |childdoc.ins| and |childdoc.dtx|.
%
\begin{itemize}
\item
Run (pdf)\LaTeX{} on |childdoc.dtx|
to compile the manual |childdoc.pdf| (this file).
\item
Run \LaTeX{} on |childdoc.ins| to create the definitions file |childdoc.def|
and the sample |cdocsamp.tex| with include files
|cdocsch1.tex|, |cdocsch2.tex|, |cdocspt3.tex|, |cdocspt4.tex|,
|cdocsdrf.tex|, |cdocsfn1.tex|, |cdocsfn2.tex|.
Then copy the file |childdoc.def| to an appropriate directory of your \LaTeX{}
distribution, e.g.\ \textit{texmf-root}|/tex/latex/childdoc|.
\end{itemize}

%%%%%%%%%%%%%%%%%%%%%%%%%%%%%%%%%%%%%%%%%%%%%%%%%%%%%%%%%%%%%%%%%%%%%%%%%%%%%%%%
\subsection{Related CTAN Packages}

There are several other packages which offer a similar functionality:
%
\begin{itemize}
\item
The packages
\href{http://ctan.org/pkg/docmute}{\textsf{docmute}},
\href{http://ctan.org/pkg/includex}{\textsf{includex}} and
\href{http://ctan.org/pkg/standalone}{\textsf{standalone}}
provide commands to include only the document body of
a child file thus allowing both files to be compiled individually.
\item
The packages \href{http://ctan.org/pkg/subdocs}{\textsf{subdocs}}
and \href{http://ctan.org/pkg/subfiles}{\textsf{subfiles}}
provide structures in which the main and child documents can be
encapsulated and allowing them to be compiled individually.
The inclusion mechanism is different from the conventional |\include|.
\item
The package \href{http://ctan.org/pkg/combine}{\textsf{combine}}
is an elaborate solution to combine several documents into one.
\end{itemize}
%
See also the CTAN topic \href{http://ctan.org/topic/subdocs}{\textsf{subdocs}}
for further related packages.
The present package differs from the above solutions in that
a document structure constructed with the conventional |\include| mechanism
just needs two extra commands at the top of every file
such that all constituent files can be compiled individually.

%%%%%%%%%%%%%%%%%%%%%%%%%%%%%%%%%%%%%%%%%%%%%%%%%%%%%%%%%%%%%%%%%%%%%%%%%%%%%%%%
%\subsection{Feature Suggestions}
%
%The following is a list of features which may be useful for future
%versions of this package:
%%
%\begin{itemize}
%\item
%\ldots
%\end{itemize}

%%%%%%%%%%%%%%%%%%%%%%%%%%%%%%%%%%%%%%%%%%%%%%%%%%%%%%%%%%%%%%%%%%%%%%%%%%%%%%%%
\subsection{Revision History}

%%%%%%%%%%%%%%%%%%%%%%%%%%%%%%%%%%%%%%%%
\paragraph{v2.0:} 2018/12/30

\begin{itemize}
\item
immediate forward processing
\item
added |\childdocby| mechanism
\item
manual restructured
\end{itemize}

%%%%%%%%%%%%%%%%%%%%%%%%%%%%%%%%%%%%%%%%
\paragraph{v1.6:} 2018/01/17

\begin{itemize}
\item
application for development of include files
\item
corrections to manual
\end{itemize}

%%%%%%%%%%%%%%%%%%%%%%%%%%%%%%%%%%%%%%%%
\paragraph{v1.5:} 2017/05/21

\begin{itemize}
\item
more complete structuring introduced
\item
|\childdocof| introduced
\item
|\childdoc| renamed to |\childdocmain|
\item
|\childredirect| renamed to |\childdocforward| and |\childdocforwardprefix|
and functionality expanded
\end{itemize}

%%%%%%%%%%%%%%%%%%%%%%%%%%%%%%%%%%%%%%%%
\paragraph{v1.0:} 2017/04/27

\begin{itemize}
\item
manual and install package
\item
first version published on CTAN
\end{itemize}

%%%%%%%%%%%%%%%%%%%%%%%%%%%%%%%%%%%%%%%%
\paragraph{v0.6:} 2017/04/26

\begin{itemize}
\item
redirection mechanism added
\end{itemize}

%%%%%%%%%%%%%%%%%%%%%%%%%%%%%%%%%%%%%%%%
\paragraph{v0.5:} 2017/04/26

\begin{itemize}
\item
functionality in definition file
\end{itemize}


%%%%%%%%%%%%%%%%%%%%%%%%%%%%%%%%%%%%%%%%%%%%%%%%%%%%%%%%%%%%%%%%%%%%%%%%%%%%%%%%
%%%%%%%%%%%%%%%%%%%%%%%%%%%%%%%%%%%%%%%%%%%%%%%%%%%%%%%%%%%%%%%%%%%%%%%%%%%%%%%%
%%%%%%%%%%%%%%%%%%%%%%%%%%%%%%%%%%%%%%%%%%%%%%%%%%%%%%%%%%%%%%%%%%%%%%%%%%%%%%%%
\appendix

\settowidth\MacroIndent{\rmfamily\scriptsize 000\ }

 \DocInput{childdoc.dtx}

\end{document}
%</driver>
% \fi
%
% %%%%%%%%%%%%%%%%%%%%%%%%%%%%%%%%%%%%%%%%%%%%%%%%%%%%%%%%%%%%%%%%%%%%%%%%%%%%%%
% %%%%%%%%%%%%%%%%%%%%%%%%%%%%%%%%%%%%%%%%%%%%%%%%%%%%%%%%%%%%%%%%%%%%%%%%%%%%%%
% \section{Sample}
%\iffalse
%<*samplemain>
%\fi
%
% The following presents a sample document
% with two chapters, two parts, a title page,
% a compile flag as well as three forwarding files to set the flag.
% It consists of eight |.tex| files:
% \begin{center}
% \begin{tabular}{ll}
% |cdocsamp.tex|&main file\\
% |cdocsch1.tex|&include file for chapter 1\\
% |cdocsch2.tex|&include file for chapter 2\\
% |cdocspt3.tex|&include file for part 3\\
% |cdocspt4.tex|&include file for part 4\\
% |cdocsdrf.tex|&forwarding file for main file in draft mode\\
% |cdocsfi1.tex|&forwarding file for final version of chapter 1\\
% |cdocsfi2.tex|&forwarding file for final version of chapter 2\\
% \end{tabular}
% \end{center}
% Each of the eight files can be compiled directly by the \LaTeX{} compiler.
%
% %%%%%%%%%%%%%%%%%%%%%%%%%%%%%%%%%%%%%%
% \paragraph{Main File.}
%
% The main file is called |cdocsamp.tex|.
%
% Load the \textsf{childdoc} definitions and
% declare the filename for the main document:
%    \begin{macrocode}
% \iffalse
%
% childdoc.dtx Copyright (C) 2017-2018 Niklas Beisert
%
% This work may be distributed and/or modified under the
% conditions of the LaTeX Project Public License, either version 1.3
% of this license or (at your option) any later version.
% The latest version of this license is in
%   http://www.latex-project.org/lppl.txt
% and version 1.3 or later is part of all distributions of LaTeX
% version 2005/12/01 or later.
%
% This work has the LPPL maintenance status `maintained'.
%
% The Current Maintainer of this work is Niklas Beisert.
%
% This work consists of the files childdoc.dtx and childdoc.ins
% and the derived files childdoc.def and cdocsamp.tex with
% cdocsch1.tex, cdocsch2.tex, cdocsdrf.tex, cdocsfn1.tex, cdocsfn2.tex.
%
%<package>\ifdefined\childdocmain\endinput\fi
%<package>\ProvidesFile{childdoc.def}[2018/12/30 v2.0 child document driver]
%<samplemain>\ProvidesFile{cdocsamp.tex}[2018/12/30 v2.0 sample for childdoc]
%<*driver>
%\ProvidesFile{childdoc.drv}[2018/12/30 v2.0 childdoc reference manual file]
\PassOptionsToClass{10pt,a4paper}{article}
\documentclass{ltxdoc}

\usepackage[margin=35mm]{geometry}
\usepackage{hyperref}
\usepackage{hyperxmp}
\usepackage[usenames]{color}

\hypersetup{colorlinks=true}
\hypersetup{pdfstartview=FitH}
\hypersetup{pdfpagemode=UseNone}
\hypersetup{pdfsource={}}
\hypersetup{pdflang={en-UK}}
\hypersetup{pdfcopyright={Copyright 2017-2018 Niklas Beisert.
  This work may be distributed and/or modified under the
  conditions of the LaTeX Project Public License, either version 1.3
  of this license or (at your option) any later version.}}
\hypersetup{pdflicenseurl={http://www.latex-project.org/lppl.txt}}
\hypersetup{pdfcontactaddress={ETH Zurich, ITP, HIT K,
  Wolfgang-Pauli-Strasse 27}}
\hypersetup{pdfcontactpostcode={8093}}
\hypersetup{pdfcontactcity={Zurich}}
\hypersetup{pdfcontactcountry={Switzerland}}
\hypersetup{pdfcontactemail={nbeisert@itp.phys.ethz.ch}}
\hypersetup{pdfcontacturl={http://people.phys.ethz.ch/\xmptilde nbeisert/}}

\newcommand{\secref}[1]{\hyperref[#1]{section \ref*{#1}}}

\parskip1ex
\parindent0pt
\let\olditemize\itemize
\def\itemize{\olditemize\parskip0pt}

\begin{document}

\title{The \textsf{childdoc} Package}
\hypersetup{pdftitle={The childdoc Package}}
\author{Niklas Beisert\\[2ex]
  Institut f\"ur Theoretische Physik\\
  Eidgen\"ossische Technische Hochschule Z\"urich\\
  Wolfgang-Pauli-Strasse 27, 8093 Z\"urich, Switzerland\\[1ex]
  \href{mailto:nbeisert@itp.phys.ethz.ch}
  {\texttt{nbeisert@itp.phys.ethz.ch}}}
\hypersetup{pdfauthor={Niklas Beisert}}
\hypersetup{pdfsubject={Manual for the LaTeX2e Package childdoc}}
\date{30 December 2018, \textsf{v2.0}}
\maketitle

\begin{abstract}\noindent
\textsf{childdoc} is a \LaTeXe{} package
that enables the direct compilation
of document sections included by |\include|
to individual files.
\end{abstract}

\begingroup
\parskip0ex
\tableofcontents
\endgroup

%%%%%%%%%%%%%%%%%%%%%%%%%%%%%%%%%%%%%%%%%%%%%%%%%%%%%%%%%%%%%%%%%%%%%%%%%%%%%%%%
%%%%%%%%%%%%%%%%%%%%%%%%%%%%%%%%%%%%%%%%%%%%%%%%%%%%%%%%%%%%%%%%%%%%%%%%%%%%%%%%
\section{Introduction}

\LaTeX{} provides a mechanism to structure a large document (such as a book)
into a main file and several child files (containing the chapters)
using the |\include| command.
This mechanism is beneficial for documents
which span hundreds of pages in order to
make the source file(s) more manageable.
Moreover, compilation can be restricted to
selected child files by means of the |\includeonly| command.
The latter feature can be used to reduce the compilation time while editing
(this was significantly more useful in the earlier days of \LaTeX{})
or to generate a smaller document which is easier to navigate.
Another application of |\includeonly| is to generate
documents consisting of selected parts of the complete document.

However, there are a few drawbacks of the plain |\include| mechanism:
\begin{itemize}
\item
The child files cannot be compiled on their own,
they can only be compiled via the main file.
A naive editing environment
(such as a text editor with an option
to have the current file processed by \LaTeX)
may require one to switch to the main file before compiling;
attempting to compile the child file produces errors.
\item
The main file must be modified (each time)
to adjust the |\includeonly| command
to the present needs. This easily leaves the main file in a messy state.
\item
The generated document will always carry the filename
of the main document. This is inconvenient if
several child files are to be compiled and
to be kept for distribution.
\end{itemize}

The present package provides a simple interface
to make child files individually compilable by \LaTeX{}.
Compiling a child file then has the same effect as compiling
the main file with an |\includeonly| command
to select the appropriate child.
Moreover the generated document will carry the name of the child
rather than the main file.
This resolves all three above issues.

This feature is meant to make the editing of books,
thesis documents and lecture notes somewhat more convenient.
However, the package can also be used efficiently for
composing a series of documents (such as exercise sheets)
which are typically distributed individually.
It then assists the author in generating the individual documents
(potentially in different versions)
as well as a document containing the collected series.
Another application is in developing style files
or other kinds of included material
where compilation of the style file could redirect
to a sample or test file.

%%%%%%%%%%%%%%%%%%%%%%%%%%%%%%%%%%%%%%%%%%%%%%%%%%%%%%%%%%%%%%%%%%%%%%%%%%%%%%%%
%%%%%%%%%%%%%%%%%%%%%%%%%%%%%%%%%%%%%%%%%%%%%%%%%%%%%%%%%%%%%%%%%%%%%%%%%%%%%%%%
\section{Usage}

First of all, the package \textsf{childdoc} is \emph{not} a standard
\LaTeXe{} |.sty| style file! Therefore it needs to be invoked in
a non-standard way.

%%%%%%%%%%%%%%%%%%%%%%%%%%%%%%%%%%%%%%%%%%%%%%%%%%%%%%%%%%%%%%%%%%%%%%%%%%%%%%%%
\subsection{Included Files}
\label{sec:include}

%%%%%%%%%%%%%%%%%%%%%%%%%%%%%%%%%%%%%%%%
\DescribeMacro{\childdocmain}
To use the package, add the commands
\begin{center}
\begin{tabular}{l}
|\input{childdoc.def}|\\
|\childdocmain{}|\\
\end{tabular}
\end{center}
at the very top of the main \LaTeX{} file,
in particular \emph{before} the |\documentclass| statement!
The argument of |\childdocmain| should be left empty
(but it must be present).

%%%%%%%%%%%%%%%%%%%%%%%%%%%%%%%%%%%%%%%%
\DescribeMacro{\childdocof}
Furthermore, add the commands
\begin{center}
\begin{tabular}{l}
|\input{childdoc.def}|\\
|\childdocof{|\textit{main}|}|\\
\end{tabular}
\end{center}
at the top of every child file \textit{child}
which is included by |\include{|\textit{child}|}|
from within the main file
(or at least for those files to be compiled individually).
The argument \textit{main} must be the filename of the main file.

There are a couple of
considerations in setting up the main and child documents:

%%%%%%%%%%%%%%%%%%%%%%%%%%%%%%%%%%%%%%%%
\paragraph{Restrictions.}

Please note the following restrictions:
\begin{itemize}
\item
|\childdocmain| must be called with one argument \textit{main}
to ensure compatibility with earlier version of the package.
It must either be empty (|\childdocmain{}|)
or precisely match the filename of the main file in which it is specified.
See \secref{sec:detection} for further information.
\item
The filename \textit{main} must be specified without the |.tex| extension.
\item
The filename \textit{main} is case sensitive
(even in case-insensitive file systems)
due to internal string comparison.
\item
The argument \textit{main} should be fully expanded, it cannot be a macro.
\item
Subdirectories and special characters should be avoided in filenames.
\item
The command |\childdocmain{|\textit{main}|}| must be followed by a whitespace.
It should not be followed immediately by another command
or by a comment mark `|%|'.
This is because the \TeX{} parser reads the token immediately following
the argument of |\childdocmain| and puts it
at the beginning of every child section;
however, a white\-space is ignored.
\end{itemize}

%%%%%%%%%%%%%%%%%%%%%%%%%%%%%%%%%%%%%%%%
\paragraph{Content of Main File.}

It is advisable to place all content in the child files included by |\include|.
Any output contained in the main file will appear in all child documents
unless suppressed manually;
it cannot be suppressed automatically by the |\includeonly| directive
and thus should normally be avoided.
A method to include some content in the main file
by means of conditional processing is described in \secref{sec:conditional}.

%%%%%%%%%%%%%%%%%%%%%%%%%%%%%%%%%%%%%%%%
\paragraph{Page Numbering.}

When only a part of the document is compiled,
the appropriate numbering of pages
(as well as other status parameters)
is determined from the |.aux| files.
The latter contain information from previous passes.
However this information needs to propagate through
all intermediate child documents.
Therefore the page numbering in child documents may well
be inconsistent until the complete document is compiled at least once.

A useful (if unconventional) way to always ensure a consistent
page numbering is to restart the numbering in each child document
and denote the pages by `\textit{child}|.|\textit{page}'
where \textit{child} represents the chapter/section number of the child file.
This can be achieved by the command
|\numberwithin{page}{|\textit{child}|}|
of the \textsf{amsmath} package
where \textit{child} can be |chapter| or |section|
depending on the chosen structuring.
Alternatively, one can modify the macro |\thepage| appropriately
and reset the counter |page| at the start of each child file.

%%%%%%%%%%%%%%%%%%%%%%%%%%%%%%%%%%%%%%%%%%%%%%%%%%%%%%%%%%%%%%%%%%%%%%%%%%%%%%%%
\subsection{Conditional Processing}
\label{sec:conditional}

The package provides a mechanism to compile different versions
of a document. To customise the versions further some conditional processing
can come in handy to distinguish which version is being compiled.
The package provides two macros to describe the compilation context:

%%%%%%%%%%%%%%%%%%%%%%%%%%%%%%%%%%%%%%%%
\DescribeMacro{\ifchilddoc}
The conditional |\ifchilddoc| distinguishes between the compilation of
child documents and the main document:
%
\begin{center}
|\ifchilddoc |\textit{child-code}| |[|\||else |\textit{main-code}]| \||fi|
\end{center}

%%%%%%%%%%%%%%%%%%%%%%%%%%%%%%%%%%%%%%%%
\DescribeMacro{\childdocname}
\DescribeMacro{\childdocjob}
The macro |\childdocname| contains the filename (without extension)
of the main or child file being processed.
Note that |\childdocjob| will always contain the name of the main file.

%%%%%%%%%%%%%%%%%%%%%%%%%%%%%%%%%%%%%%%%
\paragraph{Title Page.}

Conditional processing can be used to include a title or banner page
in the main document when proper precautions are taken.
Importantly, the code in the main file should ensure that the page counter
(as well as other status parameters which are stored in the |.aux| files)
takes the same value after the conditional processing.
Otherwise the page numbers may take divergent values
depending on which part is compiled.

For example, a title page could be declared by:
%
\begin{center}
\begin{tabular}{l}
|\ifchilddoc\||else|\\
|\addtocounter{page}{-1}|\\
\textit{code for title page}\\
|\newpage|\\
|\||fi|
\end{tabular}
\end{center}
%
A banner page for the child documents can be generated by:
%
\begin{center}
\begin{tabular}{l}
|\ifchilddoc|\\
|\addtocounter{page}{-1}|\\
\textit{code for banner page}\\
|\newpage|\\
|\||fi|
\end{tabular}
\end{center}
%
Here one could write a message such as:
\begin{center}
|This is the part \childdocname{} of \childdocjob{}.|
\end{center}

%%%%%%%%%%%%%%%%%%%%%%%%%%%%%%%%%%%%%%%%%%%%%%%%%%%%%%%%%%%%%%%%%%%%%%%%%%%%%%%%
\subsection{Flags}
\label{sec:flags}

The package makes it easy to generate different versions
of the main or child documents.
To this end compilation flags can be defined
and assigned different default values.
They will be particularly useful in conjunction
with the forwarding mechanism described in \secref{sec:forward}.

For example, it may be useful to have a flag |\version|
which can be set to |draft| or |final|.
The document source will contain some conditional code
depending on the value of |\version|.
Suppose further, the flag should default to |final| for the main file
and to |draft| for child files
which is a natural assignment for editing the document.
This is achieved by placing the following code
in the preamble of the main document
(below the |\childdocmain| directive):
%
\begin{center}
\begin{tabular}{l}
|\ifchilddoc|\\
|\providecommand{\version}{draft}|\\
|\||else|\\
|\providecommand{\version}{final}|\\
|\||fi|
\end{tabular}
\end{center}
%
The definition by |\providecommand| makes sure
that previous definitions are not overwritten.
Further statements |\providecommand{\version}{...}|
can thus be added before the above code to override it.

For the main file, one might add a line
(between |\childdocmain| and the above block)
%
\begin{center}
|%\ifchilddoc\||else\providecommand{\version}{draft}\||fi|
\end{center}
%
which can be uncommented to produce a draft version.
Likewise one can add a line to the very top of a child file
(above the |\childdocof{|\textit{main}|}| directive)
%
\begin{center}
|%\providecommand{\version}{final}|
\end{center}
%
which can be uncommented to produce the final version of this child document.

%%%%%%%%%%%%%%%%%%%%%%%%%%%%%%%%%%%%%%%%%%%%%%%%%%%%%%%%%%%%%%%%%%%%%%%%%%%%%%%%
\subsection{Forwarding}
\label{sec:forward}

Different versions of the main or child documents
using compilation flags as described in \secref{sec:flags}
can be (permanently) stored in different files
for convenient compilation, viewing and distribution.
To this end, the package defines a command
to pass on compilation to a different file:

%%%%%%%%%%%%%%%%%%%%%%%%%%%%%%%%%%%%%%%%
\DescribeMacro{\childdocforward}
The command |\childdocforward| redirects processing to
another source file:
%
\begin{center}
\begin{tabular}{l}
|\input{childdoc.def}|\\
|\childdocforward[|\textit{main}|]{|\textit{dest}|}|\\
\end{tabular}
\end{center}
%
The argument \textit{dest} is the destination file
(without extension).
It should be the main file or one of the child files.
Note that further \textsf{childdoc} directives
such as |\childdocof| and |\childdocforward|
in the indicated file will be processed in this form.
The optional argument \textit{main}
passes on directly to the main file \textit{main}
while pretending to compile the child \textit{dest}.
This form behaves as if \textit{dest}
issues |\childdocof{|\textit{main}|}| right away,
and no further \textsf{childdoc} directives will be processed.

%%%%%%%%%%%%%%%%%%%%%%%%%%%%%%%%%%%%%%%%
\DescribeMacro{\...prefix}
In the alternative form |\childdocforwardprefix|,
%
\begin{center}
\begin{tabular}{l}
|\input{childdoc.def}|\\
|\childdocforwardprefix[|\textit{main}|]{|\textit{prefix}|}{|\textit{dest}|}|
\end{tabular}
\end{center}
%
the destination file is determined by a pattern
depending on the current file:
To make this work, the current file must be called
`{\textit{prefix}\hspace{0.2em}\textit{suffix}}'
with \textit{prefix} matching precisely the argument.
Processing is then passed on to the file
`{\textit{dest}\hspace{0.2em}\textit{suffix}}'.
Surely, the same effect is achieved by
directly specifying the
argument `{\textit{dest}\hspace{0.2em}\textit{suffix}}'
in the first form.
However, that requires to set up a different file
for each child. With the alternative form of the command
all these files can have exactly the same content
which simplifies setting them up and maintaining them.

For example, the following file |draft.tex|
with a compilation flag |\version| as described in \secref{sec:flags}
compiles the main document as a draft:
%
\begin{center}
\begin{tabular}{l}
|\def\version{draft}|\\
|\input{childdoc.def}|\\
|\childdocforward{|\textit{main}|}|
\end{tabular}
\end{center}
%
Likewise, the following files |final|\textit{nn}|.tex|
compile the final version of the child document
|child|\textit{nn}|.tex|:
%
\begin{center}
\begin{tabular}{l}
|\def\version{final}|\\
|\input{childdoc.def}|\\
|\childdocforwardprefix{final}{child}|
\end{tabular}
\end{center}
%

Note that when several versions of a main file and/or of each child file
are to be generated, it may be convenient to set up a |Makefile| or
shell script to automatise the process.

%%%%%%%%%%%%%%%%%%%%%%%%%%%%%%%%%%%%%%%%%%%%%%%%%%%%%%%%%%%%%%%%%%%%%%%%%%%%%%%%
\subsection{Command Line Processing}
\label{sec:commandline}

The effect of redirection files can also be achieved by invoking
the \LaTeX{} compiler with a more elaborate command line.
Most conveniently this should be done as part
of a shell script or a |Makefile|.

When using \textsf{childdoc} in the main file, the following
command lines effectively perform a redirection
(note that depending on the shell being used,
backslashes may have to be doubled: `|\|' $\to$ `|\\|'):
%
\begin{center}
|... -jobname "|\textit{target}|" |\\|"|[\textit{flags}]%
|\input{childdoc.def}\childdocforward[|\textit{main}|]{|\textit{dest}|}"|
\end{center}
%
Here \textit{target} is the name of the output file,
\textit{main} is the name of the main file
and \textit{dest} is the name of the main or child file to be processed
(all filenames without extensions).
The optional argument \textit{main} can be omitted
if \textit{main} matches \textit{dest}.
Optionally, compilation \textit{flags} can be defined via |\def| commands.
This command line makes the \TeX{} engine believe
it is compiling the file \textit{target}
whose content is specified as the latter parameter.
The provided code then forwards the processing to
\textit{main} or \textit{dest} as described in \secref{sec:forward}.

%%%%%%%%%%%%%%%%%%%%%%%%%%%%%%%%%%%%%%%%%%%%%%%%%%%%%%%%%%%%%%%%%%%%%%%%%%%%%%%%
\subsection{Include by Input}
\label{sec:input}

Including child documents by |\include| has some restrictions by design.
Most notably, the content of a child document always occupies
its own set of pages; pages cannot be shared between child documents.
Usually, this behaviour makes perfect sense
because each child document contain an essential part of the document.
However, in some situations it may be desirable to compose
a document from a collection of parts
without having mandatory page breaks between then.
For this case, the package
provides a mechanism to include parts
by |\input| which can also be processed individually.
However, by construction this mechanism
requires manual handling of the content to be output.

%%%%%%%%%%%%%%%%%%%%%%%%%%%%%%%%%%%%%%%%
\DescribeMacro{\ifchilddocmanual}
The main file should be prepared as usual, see \secref{sec:include}.
However, the document body must make a distinction
between processing of an individual part and of the main document, e.g.:
%
\begin{center}
\begin{tabular}{l}
|\ifchilddocmanual|\\
|\input{\childdocname}|\\
|\||else|\\
\textit{document body with }|\input{|\textit{part}|}|\\
|\||fi|
\end{tabular}
\end{center}
%
The conditional |\ifchilddocmanual| is true whenever
a part to be included by |\input| is being compiled,
and the name of the part is stored in |\childdocname|.

%%%%%%%%%%%%%%%%%%%%%%%%%%%%%%%%%%%%%%%%
\DescribeMacro{\childdocby}
Each part to be included by |\input| should start with:
%
\begin{center}
\begin{tabular}{l}
|\input{childdoc.def}|\\
|\childdocby{|\textit{main}|}|\\
\end{tabular}
\end{center}
%
The directive |\childdocby| is similar to |\childdocof|
described in \secref{sec:include},
but the subsequent selection of content must be done manually.
To that end, both |\ifchilddoc| and |\ifchilddocmanual|
will be true upon processing of a part,
and the name of the part is stored in |\childdocname|.
Note that |\jobname| will be set to the filename of the current part
so that each part receives an individual |.aux| file
that does not interfere with the |.aux| file(s) of the main document.
This behaviour can be altered by the alternative form
|\childdocby[*]{|\textit{main}|}| (with a non-empty optional argument)
which uses the |.aux| file of the main document
by setting |\jobname| to \textit{main}.

%%%%%%%%%%%%%%%%%%%%%%%%%%%%%%%%%%%%%%%%%%%%%%%%%%%%%%%%%%%%%%%%%%%%%%%%%%%%%%%%
\subsection{Driver Development}
\label{sec:driver}

The \textsf{childdoc} mechanism can also be use for the development
of definition files such as \LaTeX{} styles or classes.
This case differs from the above setup with multiple parts
included by |\include| in that no |\includeonly| should be invoked.
This can be achieved by starting the include file
(before |\ProvidesPackage|) with:
%
\begin{center}
\begin{tabular}{l}
|\input{childdoc.def}|\\
|\childdocforward{|\textit{main}|}|\\
\end{tabular}
\end{center}
%
or alternatively with:
%
\begin{center}
\begin{tabular}{l}
|\input{childdoc.def}|\\
|\childdocby{|\textit{main}|}|\\
\end{tabular}
\end{center}
%
Both forms have slightly different effects as described above.
The main file is prepared as usual, see \secref{sec:include}.

%%%%%%%%%%%%%%%%%%%%%%%%%%%%%%%%%%%%%%%%%%%%%%%%%%%%%%%%%%%%%%%%%%%%%%%%%%%%%%%%
\subsection{Legacy Detection}
\label{sec:detection}

The directive |\childdocmain| in the main file can detect
whether the complete document or merely a child is to be compiled
even without using the directive |\childdocof|.
This method is deprecated because it is less robust
and there is no compelling reason to use it;
it is merely provided for backward compatibility
and it may be removed in future versions.

If the detection mechanism is to be used,
it is mandatory to correctly specify
the filename of the main file as the argument of |\childdocmain|:
%
\begin{center}
\begin{tabular}{l}
|\input{childdoc.def}|\\
|\childdocmain{|\textit{main}|}|\\
\end{tabular}
\end{center}
%
If |\jobname| does not match the argument \textit{main} of |\childdocmain|,
it is assumed that |\jobname| points to the child file to be compiled.
When using |\childdocmain| with the main file specified as argument,
it suffices to start a child file
with just |\input{|\textit{main}|}|
without loading of the package and using |\childdocof|.
If instead all processing is done
with the appropriate \textsf{childdoc} directives,
the argument of \textit{main} of |\childdocmain| can be empty.

An alternative version of the command line processing described
in \secref{sec:commandline} using the detection mechanism reads:
%
\begin{center}
|... -jobname "|\textit{target}|" "|[\textit{flags}]%
[|\def\jobname{|\textit{dest}|}|]|\input{|\textit{main}|}"|
\end{center}

%%%%%%%%%%%%%%%%%%%%%%%%%%%%%%%%%%%%%%%%%%%%%%%%%%%%%%%%%%%%%%%%%%%%%%%%%%%%%%%%
\subsection{Manual Code}
\label{sec:manual}

In case one cannot be certain whether the definitions file |childdoc.def|
is installed on the target \TeX{} distribution
and one prefers not to ship it,
it is conceivable to paste a few relevant commands into the sources.

To that end, drop all statements |\input{childdoc.def}|
and perform the replacements as outlined below.
Instead of |\childdocmain{|\textit{main}|}| add the following code
to the top of the main file:
%
\begin{center}
\begin{tabular}{l}
|\||ifdefined\childdocname\endinput\||fi\newif\ifchilddoc|\\
|\edef\childdocname{\scantokens\expandafter{\jobname\noexpand}}|\\
|\def\childdocmain{|\textit{main}|}\||ifx\childdocmain\childdocname\||else|\\
|\childdoctrue\includeonly{\childdocname}\let\jobname\childdocmain\||fi|\\
\end{tabular}
\end{center}
%
Instead of |\childdocof{|\textit{main}|}| just include the main file
at the top of each child file:
%
\begin{center}
|\input{|\textit{main}|}|
\end{center}
%
A simple redirection |\childdocforward{|\textit{dest}|}| is achieved by:
%
\begin{center}
|\def\jobname{|\textit{dest}|}\input{\jobname}|
\end{center}
%
The redirection with prefix
|\childdocforwardprefix[|\textit{prefix}|]{|\textit{dest}|}|
is accomplished by:
%
\begin{center}
\begin{tabular}{l}
|{\edef\jobname{\scantokens\expandafter{\jobname\noexpand}}|\\
|\def\redirectjob |\textit{prefix}|#1~~~{\gdef\jobname{|\textit{dest}|#1}}|\\
|\expandafter\redirectjob\jobname~~~}\input{\jobname}|
\end{tabular}
\end{center}

In an alternative approach,
child documents can be compiled by a specific command line
without additional code or specific definitions:
%
\begin{center}
|... -jobname "|\textit{target}|" "|[\textit{flags}]%
|\includeonly{|\textit{dest}|}\input{|\textit{main}|}"|
\end{center}
%

%%%%%%%%%%%%%%%%%%%%%%%%%%%%%%%%%%%%%%%%%%%%%%%%%%%%%%%%%%%%%%%%%%%%%%%%%%%%%%%%
%%%%%%%%%%%%%%%%%%%%%%%%%%%%%%%%%%%%%%%%%%%%%%%%%%%%%%%%%%%%%%%%%%%%%%%%%%%%%%%%
\section{Information}

%%%%%%%%%%%%%%%%%%%%%%%%%%%%%%%%%%%%%%%%%%%%%%%%%%%%%%%%%%%%%%%%%%%%%%%%%%%%%%%%
\subsection{Copyright}

Copyright \copyright{} 2017--2018 Niklas Beisert

This work may be distributed and/or modified under the
conditions of the \LaTeX{} Project Public License, either version 1.3
of this license or (at your option) any later version.
The latest version of this license is in
  \url{http://www.latex-project.org/lppl.txt}
and version 1.3 or later is part of all distributions of \LaTeX{}
version 2005/12/01 or later.

This work has the LPPL maintenance status `maintained'.

The Current Maintainer of this work is Niklas Beisert.

This work consists of the files |README.txt|, |childdoc.ins| and |childdoc.dtx|
as well as the derived files |childdoc.def|, |cdocsamp.tex|
with |cdocsch1.tex|, |cdocsch2.tex|, |cdocspt3.tex|, |cdocspt4.tex|,
|cdocsdrf.tex|, |cdocsfn1.tex|, |cdocsfn2.tex|
as well as |childdoc.pdf|.

%%%%%%%%%%%%%%%%%%%%%%%%%%%%%%%%%%%%%%%%%%%%%%%%%%%%%%%%%%%%%%%%%%%%%%%%%%%%%%%%
\subsection{Files and Installation}

The package consists of the files:
%
\begin{center}
\begin{tabular}{ll}
    |README.txt|   & readme file \\
    |childdoc.ins| & installation file \\
    |childdoc.dtx| & source file \\
    |childdoc.def| & definition file \\
    |cdocsamp.tex| & sample main file \\
    |cdocsch1.tex| & sample include file \\
    |cdocsch2.tex| & sample include file \\
    |cdocspt3.tex| & sample part file \\
    |cdocspt4.tex| & sample part file \\
    |cdocsdrf.tex| & sample redirection file \\
    |cdocsfn1.tex| & sample redirection file \\
    |cdocsfn2.tex| & sample redirection file \\
    |childdoc.pdf| & manual
\end{tabular}
\end{center}
%
The distribution consists of the files
|README.txt|, |childdoc.ins| and |childdoc.dtx|.
%
\begin{itemize}
\item
Run (pdf)\LaTeX{} on |childdoc.dtx|
to compile the manual |childdoc.pdf| (this file).
\item
Run \LaTeX{} on |childdoc.ins| to create the definitions file |childdoc.def|
and the sample |cdocsamp.tex| with include files
|cdocsch1.tex|, |cdocsch2.tex|, |cdocspt3.tex|, |cdocspt4.tex|,
|cdocsdrf.tex|, |cdocsfn1.tex|, |cdocsfn2.tex|.
Then copy the file |childdoc.def| to an appropriate directory of your \LaTeX{}
distribution, e.g.\ \textit{texmf-root}|/tex/latex/childdoc|.
\end{itemize}

%%%%%%%%%%%%%%%%%%%%%%%%%%%%%%%%%%%%%%%%%%%%%%%%%%%%%%%%%%%%%%%%%%%%%%%%%%%%%%%%
\subsection{Related CTAN Packages}

There are several other packages which offer a similar functionality:
%
\begin{itemize}
\item
The packages
\href{http://ctan.org/pkg/docmute}{\textsf{docmute}},
\href{http://ctan.org/pkg/includex}{\textsf{includex}} and
\href{http://ctan.org/pkg/standalone}{\textsf{standalone}}
provide commands to include only the document body of
a child file thus allowing both files to be compiled individually.
\item
The packages \href{http://ctan.org/pkg/subdocs}{\textsf{subdocs}}
and \href{http://ctan.org/pkg/subfiles}{\textsf{subfiles}}
provide structures in which the main and child documents can be
encapsulated and allowing them to be compiled individually.
The inclusion mechanism is different from the conventional |\include|.
\item
The package \href{http://ctan.org/pkg/combine}{\textsf{combine}}
is an elaborate solution to combine several documents into one.
\end{itemize}
%
See also the CTAN topic \href{http://ctan.org/topic/subdocs}{\textsf{subdocs}}
for further related packages.
The present package differs from the above solutions in that
a document structure constructed with the conventional |\include| mechanism
just needs two extra commands at the top of every file
such that all constituent files can be compiled individually.

%%%%%%%%%%%%%%%%%%%%%%%%%%%%%%%%%%%%%%%%%%%%%%%%%%%%%%%%%%%%%%%%%%%%%%%%%%%%%%%%
%\subsection{Feature Suggestions}
%
%The following is a list of features which may be useful for future
%versions of this package:
%%
%\begin{itemize}
%\item
%\ldots
%\end{itemize}

%%%%%%%%%%%%%%%%%%%%%%%%%%%%%%%%%%%%%%%%%%%%%%%%%%%%%%%%%%%%%%%%%%%%%%%%%%%%%%%%
\subsection{Revision History}

%%%%%%%%%%%%%%%%%%%%%%%%%%%%%%%%%%%%%%%%
\paragraph{v2.0:} 2018/12/30

\begin{itemize}
\item
immediate forward processing
\item
added |\childdocby| mechanism
\item
manual restructured
\end{itemize}

%%%%%%%%%%%%%%%%%%%%%%%%%%%%%%%%%%%%%%%%
\paragraph{v1.6:} 2018/01/17

\begin{itemize}
\item
application for development of include files
\item
corrections to manual
\end{itemize}

%%%%%%%%%%%%%%%%%%%%%%%%%%%%%%%%%%%%%%%%
\paragraph{v1.5:} 2017/05/21

\begin{itemize}
\item
more complete structuring introduced
\item
|\childdocof| introduced
\item
|\childdoc| renamed to |\childdocmain|
\item
|\childredirect| renamed to |\childdocforward| and |\childdocforwardprefix|
and functionality expanded
\end{itemize}

%%%%%%%%%%%%%%%%%%%%%%%%%%%%%%%%%%%%%%%%
\paragraph{v1.0:} 2017/04/27

\begin{itemize}
\item
manual and install package
\item
first version published on CTAN
\end{itemize}

%%%%%%%%%%%%%%%%%%%%%%%%%%%%%%%%%%%%%%%%
\paragraph{v0.6:} 2017/04/26

\begin{itemize}
\item
redirection mechanism added
\end{itemize}

%%%%%%%%%%%%%%%%%%%%%%%%%%%%%%%%%%%%%%%%
\paragraph{v0.5:} 2017/04/26

\begin{itemize}
\item
functionality in definition file
\end{itemize}


%%%%%%%%%%%%%%%%%%%%%%%%%%%%%%%%%%%%%%%%%%%%%%%%%%%%%%%%%%%%%%%%%%%%%%%%%%%%%%%%
%%%%%%%%%%%%%%%%%%%%%%%%%%%%%%%%%%%%%%%%%%%%%%%%%%%%%%%%%%%%%%%%%%%%%%%%%%%%%%%%
%%%%%%%%%%%%%%%%%%%%%%%%%%%%%%%%%%%%%%%%%%%%%%%%%%%%%%%%%%%%%%%%%%%%%%%%%%%%%%%%
\appendix

\settowidth\MacroIndent{\rmfamily\scriptsize 000\ }

 \DocInput{childdoc.dtx}

\end{document}
%</driver>
% \fi
%
% %%%%%%%%%%%%%%%%%%%%%%%%%%%%%%%%%%%%%%%%%%%%%%%%%%%%%%%%%%%%%%%%%%%%%%%%%%%%%%
% %%%%%%%%%%%%%%%%%%%%%%%%%%%%%%%%%%%%%%%%%%%%%%%%%%%%%%%%%%%%%%%%%%%%%%%%%%%%%%
% \section{Sample}
%\iffalse
%<*samplemain>
%\fi
%
% The following presents a sample document
% with two chapters, two parts, a title page,
% a compile flag as well as three forwarding files to set the flag.
% It consists of eight |.tex| files:
% \begin{center}
% \begin{tabular}{ll}
% |cdocsamp.tex|&main file\\
% |cdocsch1.tex|&include file for chapter 1\\
% |cdocsch2.tex|&include file for chapter 2\\
% |cdocspt3.tex|&include file for part 3\\
% |cdocspt4.tex|&include file for part 4\\
% |cdocsdrf.tex|&forwarding file for main file in draft mode\\
% |cdocsfi1.tex|&forwarding file for final version of chapter 1\\
% |cdocsfi2.tex|&forwarding file for final version of chapter 2\\
% \end{tabular}
% \end{center}
% Each of the eight files can be compiled directly by the \LaTeX{} compiler.
%
% %%%%%%%%%%%%%%%%%%%%%%%%%%%%%%%%%%%%%%
% \paragraph{Main File.}
%
% The main file is called |cdocsamp.tex|.
%
% Load the \textsf{childdoc} definitions and
% declare the filename for the main document:
%    \begin{macrocode}
\input{childdoc.def}
\childdocmain{}
%    \end{macrocode}

% Optional override for |\version| flag:
%    \begin{macrocode}
%%\ifchilddoc\else\providecommand{\version}{draft}\fi
%    \end{macrocode}

% Define the default values for the |\version| flag
% (|final| for the main file and |draft| for childs):
%    \begin{macrocode}
\ifchilddoc
\providecommand{\version}{draft}
\else
\providecommand{\version}{final}
\fi
%    \end{macrocode}

% Load the standard document class:
%    \begin{macrocode}
\documentclass[12pt]{article}
%    \end{macrocode}

% Start the document body:
%    \begin{macrocode}
\begin{document}
%    \end{macrocode}

% Declare a title page.
% Print title, part of document being processed and version flag:
%    \begin{macrocode}
\addtocounter{page}{-1}
\begin{center}
{\LARGE\bfseries{}childdoc example\par}
\vspace{1cm}
\ifchilddoc
\ifchilddocmanual part\else chapter\fi:
`\childdocname' of `\childdocjob'\par
\else
main document: `\childdocjob'\par
\fi
version: \version\par
\end{center}
\newpage
%    \end{macrocode}

% Manually include selected file,
% otherwise process as usual:
%    \begin{macrocode}
\ifchilddocmanual
\section*{part `\childdocname'}
\input{\childdocname}
\else
%    \end{macrocode}

% Include the two chapters:
%    \begin{macrocode}
\include{cdocsch1}
\include{cdocsch2}
%    \end{macrocode}

% Include the two parts unless only chapters should be displayed:
%    \begin{macrocode}
\ifchilddoc\else
\section{part three}
\input{cdocspt3}
\section{part four}
\input{cdocspt4}
\fi
%    \end{macrocode}

% Process as usual until here:
%    \begin{macrocode}
\fi
%    \end{macrocode}

% End of document body:
%    \begin{macrocode}
\end{document}
%    \end{macrocode}
%\iffalse
%</samplemain>
%\fi
%
% %%%%%%%%%%%%%%%%%%%%%%%%%%%%%%%%%%%%%%
% \paragraph{Chapter Include Files.}
%
% The include files are called |cdocsch1.tex| and |cdocsch2.tex|.
%
%\iffalse
%<*samplechap1|samplechap2>
%\fi

% Optional override for |\version| flag:
%    \begin{macrocode}
%%\providecommand{\version}{final}
%    \end{macrocode}

% Include the main document:
%    \begin{macrocode}
\input{childdoc.def}
\childdocof{cdocsamp}
%    \end{macrocode}

%\iffalse
%</samplechap1|samplechap2>
%\fi
%
%\iffalse
%<*samplechap1>
%\fi
% Some text for chapter 1:
%    \begin{macrocode}
\section{one}
some text in chapter one
%    \end{macrocode}

%\iffalse
%</samplechap1>
%\fi
% Some text for chapter 2:
%\iffalse
%<*samplechap2>
%\fi
%    \begin{macrocode}
\section{two}
more text in chapter two
%    \end{macrocode}

%\iffalse
%</samplechap2>
%\fi
%
% %%%%%%%%%%%%%%%%%%%%%%%%%%%%%%%%%%%%%%
% \paragraph{Part Include Files.}
%
% The include files are called |cdocspt3.tex| and |cdocspt4.tex|.
%
%\iffalse
%<*samplepart3|samplepart4>
%\fi

% Optional override for |\version| flag:
%    \begin{macrocode}
%%\providecommand{\version}{final}
%    \end{macrocode}

% Include the main document:
%    \begin{macrocode}
\input{childdoc.def}
\childdocby{cdocsamp}
%    \end{macrocode}

%\iffalse
%</samplepart3|samplepart4>
%\fi
%
%\iffalse
%<*samplepart3>
%\fi
% Some text for part 3:
%    \begin{macrocode}
some text in part three
%    \end{macrocode}

%\iffalse
%</samplepart3>
%\fi
% Some text for part 4:
%\iffalse
%<*samplepart4>
%\fi
%    \begin{macrocode}
more text in part four
%    \end{macrocode}

%\iffalse
%</samplepart4>
%\fi
%
% %%%%%%%%%%%%%%%%%%%%%%%%%%%%%%%%%%%%%%
% \paragraph{Forwarding for a Complete Draft.}
%
% The following forwarding file |cdocsdrf.tex|
% compiles the main document in draft mode:
%\iffalse
%<*sampledraft>
%\fi
%    \begin{macrocode}
\def\version{draft}
\input{childdoc.def}
\childdocforward{cdocsamp}
%    \end{macrocode}

%\iffalse
%</sampledraft>
%\fi
%
% %%%%%%%%%%%%%%%%%%%%%%%%%%%%%%%%%%%%%%
% \paragraph{Forwarding for Final Version of the Chapters.}
%
% The following forwarding files |cdocsfn1.tex| and |cdocsfn2.tex|
% (with identical content)
% compile the final versions of the child documents
% |cdocsch1.tex| and |cdocsch2.tex|, respectively:
%\iffalse
%<*samplefinal>
%\fi
%    \begin{macrocode}
\def\version{final}
\input{childdoc.def}
\childdocforwardprefix[cdocsamp]{cdocsfn}{cdocsch}
%    \end{macrocode}

%\iffalse
%</samplefinal>
%\fi
%
% %%%%%%%%%%%%%%%%%%%%%%%%%%%%%%%%%%%%%%
% \paragraph{Command Line Processing.}
%
% The following three command lines generate the output files
% |cdocscld|, |cdocscl1| and |cdocscl2|
% which should be identical to
% |cdocsdrf|, |cdocsch1| and |cdocsfn2|, respectively:
% \begin{center}
% \begin{tabular}{l}
% |latex -jobname cdocscld \|\\
% |  "\def\version{draft}\input{childdoc.def}\childdocforward{cdocsamp}"|\\
% |latex -jobname cdocscl1 \|\\
% |  "\input{childdoc.def}\childdocforward[cdocsamp]{cdocsch1}"|\\
% |latex -jobname cdocscl2 \|\\
% |  "\def\version{final}\input{childdoc.def}\childdocforward{cdocsch2}"|
% \end{tabular}
% \end{center}
% Note that the trailing backslash on each first line
% merely continues the input to the second line
% (for convenient cut ant paste).
% Furthermore, the command |latex| can be replaced by any
% of its alternative versions such as |pdflatex|.
%
% %%%%%%%%%%%%%%%%%%%%%%%%%%%%%%%%%%%%%%%%%%%%%%%%%%%%%%%%%%%%%%%%%%%%%%%%%%%%%%
% %%%%%%%%%%%%%%%%%%%%%%%%%%%%%%%%%%%%%%%%%%%%%%%%%%%%%%%%%%%%%%%%%%%%%%%%%%%%%%
% \section{Implementation}
%\iffalse
%<*package>
%\fi
%
% This section describes the definitions file |childdoc.def|.

% The definitions cannot be loaded using |\usepackage| or |\RequirePackage|
% which has a mechanism to prevent loading a style file more than once.
% When loading the definitions by means of |\input|
% multiple instances have to be prevented manually:
%\iffalse
%This code needs to be before the `\ProvidesFile' directive
%which is defined at the beginning of this file.
%Therefore it is also placed there and commented out here.
%</package>
%<*discard>
%\fi
%    \begin{macrocode}
\ifdefined\childdocmain\endinput\fi
%    \end{macrocode}
%\iffalse
%</discard>
%<*package>
%\fi
%
% \macro{\ifchilddoc}
% \macro{\ifchilddocmanual}
% The conditional |\ifchilddoc| tells whether a
% child (true) or main (false) document is being compiled.
% The conditional |\ifchilddocmanual| tells whether
% the |\includeonly| mechanism is used (false) or
% the selection of child files must be performed manually (true).
% The definitions initialise to false:
%    \begin{macrocode}
\newif\ifchilddoc
\newif\ifchilddocmanual
%    \end{macrocode}

% \macro{\childdocname}
% \macro{\childdocjob}
% The macro |\childdocname| stores the name of the main document
% to be compiled. The macro |\childdocjob| stores the name of
% the document on which the \LaTeX{} compiler was originally invoked.
% The content of |\jobname| cannot be compared
% to filenames specified in the source due to different catcodes.
% The following code rescans |\jobname|, stores the result
% in |\childdocname| and saves a copy in |\childdocjob|:
%    \begin{macrocode}
\edef\childdocname{\scantokens\expandafter{\jobname\noexpand}}
\let\childdocjob\childdocname
%    \end{macrocode}

% \macro{\childdocdisable}
% The macro |\childdocdisable| prevents the main file
% from being processed more than once.
% At this stage, the main document command |\childdocmain|
% is assumed to be called once again where it should do nothing.
% Any subsequent call to it should prevent
% a secondary processing of the main document
% It overwrites the forwarding commands
% |\childdocof| and |\childdocforward|
% with empty macros to prevent further inclusions of the main document:
%    \begin{macrocode}
\newcommand{\childdocdisable}
{
  \renewcommand{\childdocmain}[1]{\renewcommand{\childdocmain}[1]{\endinput}}
  \renewcommand{\childdocof}[1]{}
  \renewcommand{\childdocby}[2][]{}
  \renewcommand{\childdocforward}[2][]{}
  \renewcommand{\childdocdisable}{}
}
%    \end{macrocode}

% \macro{\childdocmain}
% The macro |\childdocmain| is to be called at the top of the main file
% with nothing or the main filename (without extension) as argument.
% First, it breaks loops.
% If the argument is not empty and does not match |\childdocname|
% (which is set by the first inclusion of |childdoc.def|),
% |\ifchilddoc| is set to true, |\includeonly| is applied to the child file
% and |\jobname| is set to the main file
% (for proper handling of |.aux| files):
%    \begin{macrocode}
\newcommand{\childdocmain}[1]
{
  \childdocdisable\childdocmain{}
  \if?#1?\else
    \begingroup
      \def\childdoctmp{#1}
      \ifx\childdoctmp\childdocname
        \def\childdoctmp{}
      \else
        \def\childdoctmp
        {
          \childdoctrue
          \includeonly{\childdocname}
          \def\childdocjob{#1}
          \def\jobname{#1}
        }
      \fi
      \expandafter
    \endgroup
    \childdoctmp
  \fi
}
%    \end{macrocode}

% \macro{\childdocof}
% The command |\childdocof| redirects
% compilation to the main file |#1|.
%    \begin{macrocode}
\newcommand{\childdocof}[1]
{
  \childdocdisable
  \childdoctrue
  \includeonly{\childdocname}
  \def\jobname{#1}
  \def\childdocjob{#1}
  \input{#1}
}
%    \end{macrocode}

% \macro{\childdocby}
% The command |\childdocby| ....
%    \begin{macrocode}
\newcommand{\childdocby}[2][]
{
  \childdocdisable
  \childdoctrue
  \childdocmanualtrue
  \if?#1?\else
    \def\jobname{#2}
  \fi
  \def\childdocjob{#2}
  \input{#2}
  \endinput
}
%    \end{macrocode}

% \macro{\childdocforward}
% The command |\childdocforward| redirects
% compilation to the main file or
% (if the optional argument is given) a child file.
% Parameters are set as if the main file
% or a child file starting with |\childdocof| was compiled.
% Then compilation is handed over to the main file:
%    \begin{macrocode}
\newcommand{\childdocforward}[2][]
{
  \begingroup
    \if?#1?
      \def\childdoctmp
      {
        \def\childdocname{#2}
        \def\childdocjob{#2}
        \def\jobname{#2}
        \input{#2}
        \endinput
      }
    \else
      \def\childdoctmp
      {
        \childdocdisable
        \def\childdocname{#2}
        \childdoctrue
        \includeonly{#2}
        \def\childdocjob{#1}
        \def\jobname{#1}
        \input{#1}
        \endinput
      }
    \fi
    \expandafter
  \endgroup
  \childdoctmp
}
%    \end{macrocode}

% \macro{\childdocforwardprefix}
% The command |\childdocforwardprefix| redirects
% compilation to the main or a child file by means of a pattern.
% The prefix |#1| in the current filename is replaced by |#2|
% and the suffix of the current filename is kept
% (it is assumed that the filename does not contain the substring `|~~~|'
% which is used as a delimiter).
% Compilation is handed over to the new file by |\childdocforward|:
%    \begin{macrocode}
\newcommand{\childdocforwardprefix}[3][]
{
  \begingroup
    \def\childdocextract #2##1~~~{\def\childdoctmp{\childdocforward[#1]{#3##1}}}
    \expandafter\childdocextract\childdocname~~~
    \expandafter
  \endgroup
  \childdoctmp
}
%    \end{macrocode}

% \macro{\childdoc}
% The deprecated macro |\childdoc| is a legacy version of |\childdocmain|:
%    \begin{macrocode}
\newcommand{\childdoc}{\childdocmain}
%    \end{macrocode}

% \macro{\childdocredirect}
% The deprecated macro |\childdocredirect| is a legacy version
% of |\childdocforward| and |\childdocforwardprefix|:
%    \begin{macrocode}
\newcommand{\childdocredirect}[2][]
{
  \begingroup
    \if?#1?
      \def\childdoctmp{\childdocforward{#2}}
    \else
      \def\childdoctmp{\childdocforwardprefix{#1}{#2}}
    \fi
    \expandafter
  \endgroup
  \childdoctmp
}
%    \end{macrocode}

%\iffalse
%</package>
%\fi
%
\endinput

\childdocmain{}
%    \end{macrocode}

% Optional override for |\version| flag:
%    \begin{macrocode}
%%\ifchilddoc\else\providecommand{\version}{draft}\fi
%    \end{macrocode}

% Define the default values for the |\version| flag
% (|final| for the main file and |draft| for childs):
%    \begin{macrocode}
\ifchilddoc
\providecommand{\version}{draft}
\else
\providecommand{\version}{final}
\fi
%    \end{macrocode}

% Load the standard document class:
%    \begin{macrocode}
\documentclass[12pt]{article}
%    \end{macrocode}

% Start the document body:
%    \begin{macrocode}
\begin{document}
%    \end{macrocode}

% Declare a title page.
% Print title, part of document being processed and version flag:
%    \begin{macrocode}
\addtocounter{page}{-1}
\begin{center}
{\LARGE\bfseries{}childdoc example\par}
\vspace{1cm}
\ifchilddoc
\ifchilddocmanual part\else chapter\fi:
`\childdocname' of `\childdocjob'\par
\else
main document: `\childdocjob'\par
\fi
version: \version\par
\end{center}
\newpage
%    \end{macrocode}

% Manually include selected file,
% otherwise process as usual:
%    \begin{macrocode}
\ifchilddocmanual
\section*{part `\childdocname'}
\input{\childdocname}
\else
%    \end{macrocode}

% Include the two chapters:
%    \begin{macrocode}
\include{cdocsch1}
\include{cdocsch2}
%    \end{macrocode}

% Include the two parts unless only chapters should be displayed:
%    \begin{macrocode}
\ifchilddoc\else
\section{part three}
\input{cdocspt3}
\section{part four}
\input{cdocspt4}
\fi
%    \end{macrocode}

% Process as usual until here:
%    \begin{macrocode}
\fi
%    \end{macrocode}

% End of document body:
%    \begin{macrocode}
\end{document}
%    \end{macrocode}
%\iffalse
%</samplemain>
%\fi
%
% %%%%%%%%%%%%%%%%%%%%%%%%%%%%%%%%%%%%%%
% \paragraph{Chapter Include Files.}
%
% The include files are called |cdocsch1.tex| and |cdocsch2.tex|.
%
%\iffalse
%<*samplechap1|samplechap2>
%\fi

% Optional override for |\version| flag:
%    \begin{macrocode}
%%\providecommand{\version}{final}
%    \end{macrocode}

% Include the main document:
%    \begin{macrocode}
% \iffalse
%
% childdoc.dtx Copyright (C) 2017-2018 Niklas Beisert
%
% This work may be distributed and/or modified under the
% conditions of the LaTeX Project Public License, either version 1.3
% of this license or (at your option) any later version.
% The latest version of this license is in
%   http://www.latex-project.org/lppl.txt
% and version 1.3 or later is part of all distributions of LaTeX
% version 2005/12/01 or later.
%
% This work has the LPPL maintenance status `maintained'.
%
% The Current Maintainer of this work is Niklas Beisert.
%
% This work consists of the files childdoc.dtx and childdoc.ins
% and the derived files childdoc.def and cdocsamp.tex with
% cdocsch1.tex, cdocsch2.tex, cdocsdrf.tex, cdocsfn1.tex, cdocsfn2.tex.
%
%<package>\ifdefined\childdocmain\endinput\fi
%<package>\ProvidesFile{childdoc.def}[2018/12/30 v2.0 child document driver]
%<samplemain>\ProvidesFile{cdocsamp.tex}[2018/12/30 v2.0 sample for childdoc]
%<*driver>
%\ProvidesFile{childdoc.drv}[2018/12/30 v2.0 childdoc reference manual file]
\PassOptionsToClass{10pt,a4paper}{article}
\documentclass{ltxdoc}

\usepackage[margin=35mm]{geometry}
\usepackage{hyperref}
\usepackage{hyperxmp}
\usepackage[usenames]{color}

\hypersetup{colorlinks=true}
\hypersetup{pdfstartview=FitH}
\hypersetup{pdfpagemode=UseNone}
\hypersetup{pdfsource={}}
\hypersetup{pdflang={en-UK}}
\hypersetup{pdfcopyright={Copyright 2017-2018 Niklas Beisert.
  This work may be distributed and/or modified under the
  conditions of the LaTeX Project Public License, either version 1.3
  of this license or (at your option) any later version.}}
\hypersetup{pdflicenseurl={http://www.latex-project.org/lppl.txt}}
\hypersetup{pdfcontactaddress={ETH Zurich, ITP, HIT K,
  Wolfgang-Pauli-Strasse 27}}
\hypersetup{pdfcontactpostcode={8093}}
\hypersetup{pdfcontactcity={Zurich}}
\hypersetup{pdfcontactcountry={Switzerland}}
\hypersetup{pdfcontactemail={nbeisert@itp.phys.ethz.ch}}
\hypersetup{pdfcontacturl={http://people.phys.ethz.ch/\xmptilde nbeisert/}}

\newcommand{\secref}[1]{\hyperref[#1]{section \ref*{#1}}}

\parskip1ex
\parindent0pt
\let\olditemize\itemize
\def\itemize{\olditemize\parskip0pt}

\begin{document}

\title{The \textsf{childdoc} Package}
\hypersetup{pdftitle={The childdoc Package}}
\author{Niklas Beisert\\[2ex]
  Institut f\"ur Theoretische Physik\\
  Eidgen\"ossische Technische Hochschule Z\"urich\\
  Wolfgang-Pauli-Strasse 27, 8093 Z\"urich, Switzerland\\[1ex]
  \href{mailto:nbeisert@itp.phys.ethz.ch}
  {\texttt{nbeisert@itp.phys.ethz.ch}}}
\hypersetup{pdfauthor={Niklas Beisert}}
\hypersetup{pdfsubject={Manual for the LaTeX2e Package childdoc}}
\date{30 December 2018, \textsf{v2.0}}
\maketitle

\begin{abstract}\noindent
\textsf{childdoc} is a \LaTeXe{} package
that enables the direct compilation
of document sections included by |\include|
to individual files.
\end{abstract}

\begingroup
\parskip0ex
\tableofcontents
\endgroup

%%%%%%%%%%%%%%%%%%%%%%%%%%%%%%%%%%%%%%%%%%%%%%%%%%%%%%%%%%%%%%%%%%%%%%%%%%%%%%%%
%%%%%%%%%%%%%%%%%%%%%%%%%%%%%%%%%%%%%%%%%%%%%%%%%%%%%%%%%%%%%%%%%%%%%%%%%%%%%%%%
\section{Introduction}

\LaTeX{} provides a mechanism to structure a large document (such as a book)
into a main file and several child files (containing the chapters)
using the |\include| command.
This mechanism is beneficial for documents
which span hundreds of pages in order to
make the source file(s) more manageable.
Moreover, compilation can be restricted to
selected child files by means of the |\includeonly| command.
The latter feature can be used to reduce the compilation time while editing
(this was significantly more useful in the earlier days of \LaTeX{})
or to generate a smaller document which is easier to navigate.
Another application of |\includeonly| is to generate
documents consisting of selected parts of the complete document.

However, there are a few drawbacks of the plain |\include| mechanism:
\begin{itemize}
\item
The child files cannot be compiled on their own,
they can only be compiled via the main file.
A naive editing environment
(such as a text editor with an option
to have the current file processed by \LaTeX)
may require one to switch to the main file before compiling;
attempting to compile the child file produces errors.
\item
The main file must be modified (each time)
to adjust the |\includeonly| command
to the present needs. This easily leaves the main file in a messy state.
\item
The generated document will always carry the filename
of the main document. This is inconvenient if
several child files are to be compiled and
to be kept for distribution.
\end{itemize}

The present package provides a simple interface
to make child files individually compilable by \LaTeX{}.
Compiling a child file then has the same effect as compiling
the main file with an |\includeonly| command
to select the appropriate child.
Moreover the generated document will carry the name of the child
rather than the main file.
This resolves all three above issues.

This feature is meant to make the editing of books,
thesis documents and lecture notes somewhat more convenient.
However, the package can also be used efficiently for
composing a series of documents (such as exercise sheets)
which are typically distributed individually.
It then assists the author in generating the individual documents
(potentially in different versions)
as well as a document containing the collected series.
Another application is in developing style files
or other kinds of included material
where compilation of the style file could redirect
to a sample or test file.

%%%%%%%%%%%%%%%%%%%%%%%%%%%%%%%%%%%%%%%%%%%%%%%%%%%%%%%%%%%%%%%%%%%%%%%%%%%%%%%%
%%%%%%%%%%%%%%%%%%%%%%%%%%%%%%%%%%%%%%%%%%%%%%%%%%%%%%%%%%%%%%%%%%%%%%%%%%%%%%%%
\section{Usage}

First of all, the package \textsf{childdoc} is \emph{not} a standard
\LaTeXe{} |.sty| style file! Therefore it needs to be invoked in
a non-standard way.

%%%%%%%%%%%%%%%%%%%%%%%%%%%%%%%%%%%%%%%%%%%%%%%%%%%%%%%%%%%%%%%%%%%%%%%%%%%%%%%%
\subsection{Included Files}
\label{sec:include}

%%%%%%%%%%%%%%%%%%%%%%%%%%%%%%%%%%%%%%%%
\DescribeMacro{\childdocmain}
To use the package, add the commands
\begin{center}
\begin{tabular}{l}
|\input{childdoc.def}|\\
|\childdocmain{}|\\
\end{tabular}
\end{center}
at the very top of the main \LaTeX{} file,
in particular \emph{before} the |\documentclass| statement!
The argument of |\childdocmain| should be left empty
(but it must be present).

%%%%%%%%%%%%%%%%%%%%%%%%%%%%%%%%%%%%%%%%
\DescribeMacro{\childdocof}
Furthermore, add the commands
\begin{center}
\begin{tabular}{l}
|\input{childdoc.def}|\\
|\childdocof{|\textit{main}|}|\\
\end{tabular}
\end{center}
at the top of every child file \textit{child}
which is included by |\include{|\textit{child}|}|
from within the main file
(or at least for those files to be compiled individually).
The argument \textit{main} must be the filename of the main file.

There are a couple of
considerations in setting up the main and child documents:

%%%%%%%%%%%%%%%%%%%%%%%%%%%%%%%%%%%%%%%%
\paragraph{Restrictions.}

Please note the following restrictions:
\begin{itemize}
\item
|\childdocmain| must be called with one argument \textit{main}
to ensure compatibility with earlier version of the package.
It must either be empty (|\childdocmain{}|)
or precisely match the filename of the main file in which it is specified.
See \secref{sec:detection} for further information.
\item
The filename \textit{main} must be specified without the |.tex| extension.
\item
The filename \textit{main} is case sensitive
(even in case-insensitive file systems)
due to internal string comparison.
\item
The argument \textit{main} should be fully expanded, it cannot be a macro.
\item
Subdirectories and special characters should be avoided in filenames.
\item
The command |\childdocmain{|\textit{main}|}| must be followed by a whitespace.
It should not be followed immediately by another command
or by a comment mark `|%|'.
This is because the \TeX{} parser reads the token immediately following
the argument of |\childdocmain| and puts it
at the beginning of every child section;
however, a white\-space is ignored.
\end{itemize}

%%%%%%%%%%%%%%%%%%%%%%%%%%%%%%%%%%%%%%%%
\paragraph{Content of Main File.}

It is advisable to place all content in the child files included by |\include|.
Any output contained in the main file will appear in all child documents
unless suppressed manually;
it cannot be suppressed automatically by the |\includeonly| directive
and thus should normally be avoided.
A method to include some content in the main file
by means of conditional processing is described in \secref{sec:conditional}.

%%%%%%%%%%%%%%%%%%%%%%%%%%%%%%%%%%%%%%%%
\paragraph{Page Numbering.}

When only a part of the document is compiled,
the appropriate numbering of pages
(as well as other status parameters)
is determined from the |.aux| files.
The latter contain information from previous passes.
However this information needs to propagate through
all intermediate child documents.
Therefore the page numbering in child documents may well
be inconsistent until the complete document is compiled at least once.

A useful (if unconventional) way to always ensure a consistent
page numbering is to restart the numbering in each child document
and denote the pages by `\textit{child}|.|\textit{page}'
where \textit{child} represents the chapter/section number of the child file.
This can be achieved by the command
|\numberwithin{page}{|\textit{child}|}|
of the \textsf{amsmath} package
where \textit{child} can be |chapter| or |section|
depending on the chosen structuring.
Alternatively, one can modify the macro |\thepage| appropriately
and reset the counter |page| at the start of each child file.

%%%%%%%%%%%%%%%%%%%%%%%%%%%%%%%%%%%%%%%%%%%%%%%%%%%%%%%%%%%%%%%%%%%%%%%%%%%%%%%%
\subsection{Conditional Processing}
\label{sec:conditional}

The package provides a mechanism to compile different versions
of a document. To customise the versions further some conditional processing
can come in handy to distinguish which version is being compiled.
The package provides two macros to describe the compilation context:

%%%%%%%%%%%%%%%%%%%%%%%%%%%%%%%%%%%%%%%%
\DescribeMacro{\ifchilddoc}
The conditional |\ifchilddoc| distinguishes between the compilation of
child documents and the main document:
%
\begin{center}
|\ifchilddoc |\textit{child-code}| |[|\||else |\textit{main-code}]| \||fi|
\end{center}

%%%%%%%%%%%%%%%%%%%%%%%%%%%%%%%%%%%%%%%%
\DescribeMacro{\childdocname}
\DescribeMacro{\childdocjob}
The macro |\childdocname| contains the filename (without extension)
of the main or child file being processed.
Note that |\childdocjob| will always contain the name of the main file.

%%%%%%%%%%%%%%%%%%%%%%%%%%%%%%%%%%%%%%%%
\paragraph{Title Page.}

Conditional processing can be used to include a title or banner page
in the main document when proper precautions are taken.
Importantly, the code in the main file should ensure that the page counter
(as well as other status parameters which are stored in the |.aux| files)
takes the same value after the conditional processing.
Otherwise the page numbers may take divergent values
depending on which part is compiled.

For example, a title page could be declared by:
%
\begin{center}
\begin{tabular}{l}
|\ifchilddoc\||else|\\
|\addtocounter{page}{-1}|\\
\textit{code for title page}\\
|\newpage|\\
|\||fi|
\end{tabular}
\end{center}
%
A banner page for the child documents can be generated by:
%
\begin{center}
\begin{tabular}{l}
|\ifchilddoc|\\
|\addtocounter{page}{-1}|\\
\textit{code for banner page}\\
|\newpage|\\
|\||fi|
\end{tabular}
\end{center}
%
Here one could write a message such as:
\begin{center}
|This is the part \childdocname{} of \childdocjob{}.|
\end{center}

%%%%%%%%%%%%%%%%%%%%%%%%%%%%%%%%%%%%%%%%%%%%%%%%%%%%%%%%%%%%%%%%%%%%%%%%%%%%%%%%
\subsection{Flags}
\label{sec:flags}

The package makes it easy to generate different versions
of the main or child documents.
To this end compilation flags can be defined
and assigned different default values.
They will be particularly useful in conjunction
with the forwarding mechanism described in \secref{sec:forward}.

For example, it may be useful to have a flag |\version|
which can be set to |draft| or |final|.
The document source will contain some conditional code
depending on the value of |\version|.
Suppose further, the flag should default to |final| for the main file
and to |draft| for child files
which is a natural assignment for editing the document.
This is achieved by placing the following code
in the preamble of the main document
(below the |\childdocmain| directive):
%
\begin{center}
\begin{tabular}{l}
|\ifchilddoc|\\
|\providecommand{\version}{draft}|\\
|\||else|\\
|\providecommand{\version}{final}|\\
|\||fi|
\end{tabular}
\end{center}
%
The definition by |\providecommand| makes sure
that previous definitions are not overwritten.
Further statements |\providecommand{\version}{...}|
can thus be added before the above code to override it.

For the main file, one might add a line
(between |\childdocmain| and the above block)
%
\begin{center}
|%\ifchilddoc\||else\providecommand{\version}{draft}\||fi|
\end{center}
%
which can be uncommented to produce a draft version.
Likewise one can add a line to the very top of a child file
(above the |\childdocof{|\textit{main}|}| directive)
%
\begin{center}
|%\providecommand{\version}{final}|
\end{center}
%
which can be uncommented to produce the final version of this child document.

%%%%%%%%%%%%%%%%%%%%%%%%%%%%%%%%%%%%%%%%%%%%%%%%%%%%%%%%%%%%%%%%%%%%%%%%%%%%%%%%
\subsection{Forwarding}
\label{sec:forward}

Different versions of the main or child documents
using compilation flags as described in \secref{sec:flags}
can be (permanently) stored in different files
for convenient compilation, viewing and distribution.
To this end, the package defines a command
to pass on compilation to a different file:

%%%%%%%%%%%%%%%%%%%%%%%%%%%%%%%%%%%%%%%%
\DescribeMacro{\childdocforward}
The command |\childdocforward| redirects processing to
another source file:
%
\begin{center}
\begin{tabular}{l}
|\input{childdoc.def}|\\
|\childdocforward[|\textit{main}|]{|\textit{dest}|}|\\
\end{tabular}
\end{center}
%
The argument \textit{dest} is the destination file
(without extension).
It should be the main file or one of the child files.
Note that further \textsf{childdoc} directives
such as |\childdocof| and |\childdocforward|
in the indicated file will be processed in this form.
The optional argument \textit{main}
passes on directly to the main file \textit{main}
while pretending to compile the child \textit{dest}.
This form behaves as if \textit{dest}
issues |\childdocof{|\textit{main}|}| right away,
and no further \textsf{childdoc} directives will be processed.

%%%%%%%%%%%%%%%%%%%%%%%%%%%%%%%%%%%%%%%%
\DescribeMacro{\...prefix}
In the alternative form |\childdocforwardprefix|,
%
\begin{center}
\begin{tabular}{l}
|\input{childdoc.def}|\\
|\childdocforwardprefix[|\textit{main}|]{|\textit{prefix}|}{|\textit{dest}|}|
\end{tabular}
\end{center}
%
the destination file is determined by a pattern
depending on the current file:
To make this work, the current file must be called
`{\textit{prefix}\hspace{0.2em}\textit{suffix}}'
with \textit{prefix} matching precisely the argument.
Processing is then passed on to the file
`{\textit{dest}\hspace{0.2em}\textit{suffix}}'.
Surely, the same effect is achieved by
directly specifying the
argument `{\textit{dest}\hspace{0.2em}\textit{suffix}}'
in the first form.
However, that requires to set up a different file
for each child. With the alternative form of the command
all these files can have exactly the same content
which simplifies setting them up and maintaining them.

For example, the following file |draft.tex|
with a compilation flag |\version| as described in \secref{sec:flags}
compiles the main document as a draft:
%
\begin{center}
\begin{tabular}{l}
|\def\version{draft}|\\
|\input{childdoc.def}|\\
|\childdocforward{|\textit{main}|}|
\end{tabular}
\end{center}
%
Likewise, the following files |final|\textit{nn}|.tex|
compile the final version of the child document
|child|\textit{nn}|.tex|:
%
\begin{center}
\begin{tabular}{l}
|\def\version{final}|\\
|\input{childdoc.def}|\\
|\childdocforwardprefix{final}{child}|
\end{tabular}
\end{center}
%

Note that when several versions of a main file and/or of each child file
are to be generated, it may be convenient to set up a |Makefile| or
shell script to automatise the process.

%%%%%%%%%%%%%%%%%%%%%%%%%%%%%%%%%%%%%%%%%%%%%%%%%%%%%%%%%%%%%%%%%%%%%%%%%%%%%%%%
\subsection{Command Line Processing}
\label{sec:commandline}

The effect of redirection files can also be achieved by invoking
the \LaTeX{} compiler with a more elaborate command line.
Most conveniently this should be done as part
of a shell script or a |Makefile|.

When using \textsf{childdoc} in the main file, the following
command lines effectively perform a redirection
(note that depending on the shell being used,
backslashes may have to be doubled: `|\|' $\to$ `|\\|'):
%
\begin{center}
|... -jobname "|\textit{target}|" |\\|"|[\textit{flags}]%
|\input{childdoc.def}\childdocforward[|\textit{main}|]{|\textit{dest}|}"|
\end{center}
%
Here \textit{target} is the name of the output file,
\textit{main} is the name of the main file
and \textit{dest} is the name of the main or child file to be processed
(all filenames without extensions).
The optional argument \textit{main} can be omitted
if \textit{main} matches \textit{dest}.
Optionally, compilation \textit{flags} can be defined via |\def| commands.
This command line makes the \TeX{} engine believe
it is compiling the file \textit{target}
whose content is specified as the latter parameter.
The provided code then forwards the processing to
\textit{main} or \textit{dest} as described in \secref{sec:forward}.

%%%%%%%%%%%%%%%%%%%%%%%%%%%%%%%%%%%%%%%%%%%%%%%%%%%%%%%%%%%%%%%%%%%%%%%%%%%%%%%%
\subsection{Include by Input}
\label{sec:input}

Including child documents by |\include| has some restrictions by design.
Most notably, the content of a child document always occupies
its own set of pages; pages cannot be shared between child documents.
Usually, this behaviour makes perfect sense
because each child document contain an essential part of the document.
However, in some situations it may be desirable to compose
a document from a collection of parts
without having mandatory page breaks between then.
For this case, the package
provides a mechanism to include parts
by |\input| which can also be processed individually.
However, by construction this mechanism
requires manual handling of the content to be output.

%%%%%%%%%%%%%%%%%%%%%%%%%%%%%%%%%%%%%%%%
\DescribeMacro{\ifchilddocmanual}
The main file should be prepared as usual, see \secref{sec:include}.
However, the document body must make a distinction
between processing of an individual part and of the main document, e.g.:
%
\begin{center}
\begin{tabular}{l}
|\ifchilddocmanual|\\
|\input{\childdocname}|\\
|\||else|\\
\textit{document body with }|\input{|\textit{part}|}|\\
|\||fi|
\end{tabular}
\end{center}
%
The conditional |\ifchilddocmanual| is true whenever
a part to be included by |\input| is being compiled,
and the name of the part is stored in |\childdocname|.

%%%%%%%%%%%%%%%%%%%%%%%%%%%%%%%%%%%%%%%%
\DescribeMacro{\childdocby}
Each part to be included by |\input| should start with:
%
\begin{center}
\begin{tabular}{l}
|\input{childdoc.def}|\\
|\childdocby{|\textit{main}|}|\\
\end{tabular}
\end{center}
%
The directive |\childdocby| is similar to |\childdocof|
described in \secref{sec:include},
but the subsequent selection of content must be done manually.
To that end, both |\ifchilddoc| and |\ifchilddocmanual|
will be true upon processing of a part,
and the name of the part is stored in |\childdocname|.
Note that |\jobname| will be set to the filename of the current part
so that each part receives an individual |.aux| file
that does not interfere with the |.aux| file(s) of the main document.
This behaviour can be altered by the alternative form
|\childdocby[*]{|\textit{main}|}| (with a non-empty optional argument)
which uses the |.aux| file of the main document
by setting |\jobname| to \textit{main}.

%%%%%%%%%%%%%%%%%%%%%%%%%%%%%%%%%%%%%%%%%%%%%%%%%%%%%%%%%%%%%%%%%%%%%%%%%%%%%%%%
\subsection{Driver Development}
\label{sec:driver}

The \textsf{childdoc} mechanism can also be use for the development
of definition files such as \LaTeX{} styles or classes.
This case differs from the above setup with multiple parts
included by |\include| in that no |\includeonly| should be invoked.
This can be achieved by starting the include file
(before |\ProvidesPackage|) with:
%
\begin{center}
\begin{tabular}{l}
|\input{childdoc.def}|\\
|\childdocforward{|\textit{main}|}|\\
\end{tabular}
\end{center}
%
or alternatively with:
%
\begin{center}
\begin{tabular}{l}
|\input{childdoc.def}|\\
|\childdocby{|\textit{main}|}|\\
\end{tabular}
\end{center}
%
Both forms have slightly different effects as described above.
The main file is prepared as usual, see \secref{sec:include}.

%%%%%%%%%%%%%%%%%%%%%%%%%%%%%%%%%%%%%%%%%%%%%%%%%%%%%%%%%%%%%%%%%%%%%%%%%%%%%%%%
\subsection{Legacy Detection}
\label{sec:detection}

The directive |\childdocmain| in the main file can detect
whether the complete document or merely a child is to be compiled
even without using the directive |\childdocof|.
This method is deprecated because it is less robust
and there is no compelling reason to use it;
it is merely provided for backward compatibility
and it may be removed in future versions.

If the detection mechanism is to be used,
it is mandatory to correctly specify
the filename of the main file as the argument of |\childdocmain|:
%
\begin{center}
\begin{tabular}{l}
|\input{childdoc.def}|\\
|\childdocmain{|\textit{main}|}|\\
\end{tabular}
\end{center}
%
If |\jobname| does not match the argument \textit{main} of |\childdocmain|,
it is assumed that |\jobname| points to the child file to be compiled.
When using |\childdocmain| with the main file specified as argument,
it suffices to start a child file
with just |\input{|\textit{main}|}|
without loading of the package and using |\childdocof|.
If instead all processing is done
with the appropriate \textsf{childdoc} directives,
the argument of \textit{main} of |\childdocmain| can be empty.

An alternative version of the command line processing described
in \secref{sec:commandline} using the detection mechanism reads:
%
\begin{center}
|... -jobname "|\textit{target}|" "|[\textit{flags}]%
[|\def\jobname{|\textit{dest}|}|]|\input{|\textit{main}|}"|
\end{center}

%%%%%%%%%%%%%%%%%%%%%%%%%%%%%%%%%%%%%%%%%%%%%%%%%%%%%%%%%%%%%%%%%%%%%%%%%%%%%%%%
\subsection{Manual Code}
\label{sec:manual}

In case one cannot be certain whether the definitions file |childdoc.def|
is installed on the target \TeX{} distribution
and one prefers not to ship it,
it is conceivable to paste a few relevant commands into the sources.

To that end, drop all statements |\input{childdoc.def}|
and perform the replacements as outlined below.
Instead of |\childdocmain{|\textit{main}|}| add the following code
to the top of the main file:
%
\begin{center}
\begin{tabular}{l}
|\||ifdefined\childdocname\endinput\||fi\newif\ifchilddoc|\\
|\edef\childdocname{\scantokens\expandafter{\jobname\noexpand}}|\\
|\def\childdocmain{|\textit{main}|}\||ifx\childdocmain\childdocname\||else|\\
|\childdoctrue\includeonly{\childdocname}\let\jobname\childdocmain\||fi|\\
\end{tabular}
\end{center}
%
Instead of |\childdocof{|\textit{main}|}| just include the main file
at the top of each child file:
%
\begin{center}
|\input{|\textit{main}|}|
\end{center}
%
A simple redirection |\childdocforward{|\textit{dest}|}| is achieved by:
%
\begin{center}
|\def\jobname{|\textit{dest}|}\input{\jobname}|
\end{center}
%
The redirection with prefix
|\childdocforwardprefix[|\textit{prefix}|]{|\textit{dest}|}|
is accomplished by:
%
\begin{center}
\begin{tabular}{l}
|{\edef\jobname{\scantokens\expandafter{\jobname\noexpand}}|\\
|\def\redirectjob |\textit{prefix}|#1~~~{\gdef\jobname{|\textit{dest}|#1}}|\\
|\expandafter\redirectjob\jobname~~~}\input{\jobname}|
\end{tabular}
\end{center}

In an alternative approach,
child documents can be compiled by a specific command line
without additional code or specific definitions:
%
\begin{center}
|... -jobname "|\textit{target}|" "|[\textit{flags}]%
|\includeonly{|\textit{dest}|}\input{|\textit{main}|}"|
\end{center}
%

%%%%%%%%%%%%%%%%%%%%%%%%%%%%%%%%%%%%%%%%%%%%%%%%%%%%%%%%%%%%%%%%%%%%%%%%%%%%%%%%
%%%%%%%%%%%%%%%%%%%%%%%%%%%%%%%%%%%%%%%%%%%%%%%%%%%%%%%%%%%%%%%%%%%%%%%%%%%%%%%%
\section{Information}

%%%%%%%%%%%%%%%%%%%%%%%%%%%%%%%%%%%%%%%%%%%%%%%%%%%%%%%%%%%%%%%%%%%%%%%%%%%%%%%%
\subsection{Copyright}

Copyright \copyright{} 2017--2018 Niklas Beisert

This work may be distributed and/or modified under the
conditions of the \LaTeX{} Project Public License, either version 1.3
of this license or (at your option) any later version.
The latest version of this license is in
  \url{http://www.latex-project.org/lppl.txt}
and version 1.3 or later is part of all distributions of \LaTeX{}
version 2005/12/01 or later.

This work has the LPPL maintenance status `maintained'.

The Current Maintainer of this work is Niklas Beisert.

This work consists of the files |README.txt|, |childdoc.ins| and |childdoc.dtx|
as well as the derived files |childdoc.def|, |cdocsamp.tex|
with |cdocsch1.tex|, |cdocsch2.tex|, |cdocspt3.tex|, |cdocspt4.tex|,
|cdocsdrf.tex|, |cdocsfn1.tex|, |cdocsfn2.tex|
as well as |childdoc.pdf|.

%%%%%%%%%%%%%%%%%%%%%%%%%%%%%%%%%%%%%%%%%%%%%%%%%%%%%%%%%%%%%%%%%%%%%%%%%%%%%%%%
\subsection{Files and Installation}

The package consists of the files:
%
\begin{center}
\begin{tabular}{ll}
    |README.txt|   & readme file \\
    |childdoc.ins| & installation file \\
    |childdoc.dtx| & source file \\
    |childdoc.def| & definition file \\
    |cdocsamp.tex| & sample main file \\
    |cdocsch1.tex| & sample include file \\
    |cdocsch2.tex| & sample include file \\
    |cdocspt3.tex| & sample part file \\
    |cdocspt4.tex| & sample part file \\
    |cdocsdrf.tex| & sample redirection file \\
    |cdocsfn1.tex| & sample redirection file \\
    |cdocsfn2.tex| & sample redirection file \\
    |childdoc.pdf| & manual
\end{tabular}
\end{center}
%
The distribution consists of the files
|README.txt|, |childdoc.ins| and |childdoc.dtx|.
%
\begin{itemize}
\item
Run (pdf)\LaTeX{} on |childdoc.dtx|
to compile the manual |childdoc.pdf| (this file).
\item
Run \LaTeX{} on |childdoc.ins| to create the definitions file |childdoc.def|
and the sample |cdocsamp.tex| with include files
|cdocsch1.tex|, |cdocsch2.tex|, |cdocspt3.tex|, |cdocspt4.tex|,
|cdocsdrf.tex|, |cdocsfn1.tex|, |cdocsfn2.tex|.
Then copy the file |childdoc.def| to an appropriate directory of your \LaTeX{}
distribution, e.g.\ \textit{texmf-root}|/tex/latex/childdoc|.
\end{itemize}

%%%%%%%%%%%%%%%%%%%%%%%%%%%%%%%%%%%%%%%%%%%%%%%%%%%%%%%%%%%%%%%%%%%%%%%%%%%%%%%%
\subsection{Related CTAN Packages}

There are several other packages which offer a similar functionality:
%
\begin{itemize}
\item
The packages
\href{http://ctan.org/pkg/docmute}{\textsf{docmute}},
\href{http://ctan.org/pkg/includex}{\textsf{includex}} and
\href{http://ctan.org/pkg/standalone}{\textsf{standalone}}
provide commands to include only the document body of
a child file thus allowing both files to be compiled individually.
\item
The packages \href{http://ctan.org/pkg/subdocs}{\textsf{subdocs}}
and \href{http://ctan.org/pkg/subfiles}{\textsf{subfiles}}
provide structures in which the main and child documents can be
encapsulated and allowing them to be compiled individually.
The inclusion mechanism is different from the conventional |\include|.
\item
The package \href{http://ctan.org/pkg/combine}{\textsf{combine}}
is an elaborate solution to combine several documents into one.
\end{itemize}
%
See also the CTAN topic \href{http://ctan.org/topic/subdocs}{\textsf{subdocs}}
for further related packages.
The present package differs from the above solutions in that
a document structure constructed with the conventional |\include| mechanism
just needs two extra commands at the top of every file
such that all constituent files can be compiled individually.

%%%%%%%%%%%%%%%%%%%%%%%%%%%%%%%%%%%%%%%%%%%%%%%%%%%%%%%%%%%%%%%%%%%%%%%%%%%%%%%%
%\subsection{Feature Suggestions}
%
%The following is a list of features which may be useful for future
%versions of this package:
%%
%\begin{itemize}
%\item
%\ldots
%\end{itemize}

%%%%%%%%%%%%%%%%%%%%%%%%%%%%%%%%%%%%%%%%%%%%%%%%%%%%%%%%%%%%%%%%%%%%%%%%%%%%%%%%
\subsection{Revision History}

%%%%%%%%%%%%%%%%%%%%%%%%%%%%%%%%%%%%%%%%
\paragraph{v2.0:} 2018/12/30

\begin{itemize}
\item
immediate forward processing
\item
added |\childdocby| mechanism
\item
manual restructured
\end{itemize}

%%%%%%%%%%%%%%%%%%%%%%%%%%%%%%%%%%%%%%%%
\paragraph{v1.6:} 2018/01/17

\begin{itemize}
\item
application for development of include files
\item
corrections to manual
\end{itemize}

%%%%%%%%%%%%%%%%%%%%%%%%%%%%%%%%%%%%%%%%
\paragraph{v1.5:} 2017/05/21

\begin{itemize}
\item
more complete structuring introduced
\item
|\childdocof| introduced
\item
|\childdoc| renamed to |\childdocmain|
\item
|\childredirect| renamed to |\childdocforward| and |\childdocforwardprefix|
and functionality expanded
\end{itemize}

%%%%%%%%%%%%%%%%%%%%%%%%%%%%%%%%%%%%%%%%
\paragraph{v1.0:} 2017/04/27

\begin{itemize}
\item
manual and install package
\item
first version published on CTAN
\end{itemize}

%%%%%%%%%%%%%%%%%%%%%%%%%%%%%%%%%%%%%%%%
\paragraph{v0.6:} 2017/04/26

\begin{itemize}
\item
redirection mechanism added
\end{itemize}

%%%%%%%%%%%%%%%%%%%%%%%%%%%%%%%%%%%%%%%%
\paragraph{v0.5:} 2017/04/26

\begin{itemize}
\item
functionality in definition file
\end{itemize}


%%%%%%%%%%%%%%%%%%%%%%%%%%%%%%%%%%%%%%%%%%%%%%%%%%%%%%%%%%%%%%%%%%%%%%%%%%%%%%%%
%%%%%%%%%%%%%%%%%%%%%%%%%%%%%%%%%%%%%%%%%%%%%%%%%%%%%%%%%%%%%%%%%%%%%%%%%%%%%%%%
%%%%%%%%%%%%%%%%%%%%%%%%%%%%%%%%%%%%%%%%%%%%%%%%%%%%%%%%%%%%%%%%%%%%%%%%%%%%%%%%
\appendix

\settowidth\MacroIndent{\rmfamily\scriptsize 000\ }

 \DocInput{childdoc.dtx}

\end{document}
%</driver>
% \fi
%
% %%%%%%%%%%%%%%%%%%%%%%%%%%%%%%%%%%%%%%%%%%%%%%%%%%%%%%%%%%%%%%%%%%%%%%%%%%%%%%
% %%%%%%%%%%%%%%%%%%%%%%%%%%%%%%%%%%%%%%%%%%%%%%%%%%%%%%%%%%%%%%%%%%%%%%%%%%%%%%
% \section{Sample}
%\iffalse
%<*samplemain>
%\fi
%
% The following presents a sample document
% with two chapters, two parts, a title page,
% a compile flag as well as three forwarding files to set the flag.
% It consists of eight |.tex| files:
% \begin{center}
% \begin{tabular}{ll}
% |cdocsamp.tex|&main file\\
% |cdocsch1.tex|&include file for chapter 1\\
% |cdocsch2.tex|&include file for chapter 2\\
% |cdocspt3.tex|&include file for part 3\\
% |cdocspt4.tex|&include file for part 4\\
% |cdocsdrf.tex|&forwarding file for main file in draft mode\\
% |cdocsfi1.tex|&forwarding file for final version of chapter 1\\
% |cdocsfi2.tex|&forwarding file for final version of chapter 2\\
% \end{tabular}
% \end{center}
% Each of the eight files can be compiled directly by the \LaTeX{} compiler.
%
% %%%%%%%%%%%%%%%%%%%%%%%%%%%%%%%%%%%%%%
% \paragraph{Main File.}
%
% The main file is called |cdocsamp.tex|.
%
% Load the \textsf{childdoc} definitions and
% declare the filename for the main document:
%    \begin{macrocode}
\input{childdoc.def}
\childdocmain{}
%    \end{macrocode}

% Optional override for |\version| flag:
%    \begin{macrocode}
%%\ifchilddoc\else\providecommand{\version}{draft}\fi
%    \end{macrocode}

% Define the default values for the |\version| flag
% (|final| for the main file and |draft| for childs):
%    \begin{macrocode}
\ifchilddoc
\providecommand{\version}{draft}
\else
\providecommand{\version}{final}
\fi
%    \end{macrocode}

% Load the standard document class:
%    \begin{macrocode}
\documentclass[12pt]{article}
%    \end{macrocode}

% Start the document body:
%    \begin{macrocode}
\begin{document}
%    \end{macrocode}

% Declare a title page.
% Print title, part of document being processed and version flag:
%    \begin{macrocode}
\addtocounter{page}{-1}
\begin{center}
{\LARGE\bfseries{}childdoc example\par}
\vspace{1cm}
\ifchilddoc
\ifchilddocmanual part\else chapter\fi:
`\childdocname' of `\childdocjob'\par
\else
main document: `\childdocjob'\par
\fi
version: \version\par
\end{center}
\newpage
%    \end{macrocode}

% Manually include selected file,
% otherwise process as usual:
%    \begin{macrocode}
\ifchilddocmanual
\section*{part `\childdocname'}
\input{\childdocname}
\else
%    \end{macrocode}

% Include the two chapters:
%    \begin{macrocode}
\include{cdocsch1}
\include{cdocsch2}
%    \end{macrocode}

% Include the two parts unless only chapters should be displayed:
%    \begin{macrocode}
\ifchilddoc\else
\section{part three}
\input{cdocspt3}
\section{part four}
\input{cdocspt4}
\fi
%    \end{macrocode}

% Process as usual until here:
%    \begin{macrocode}
\fi
%    \end{macrocode}

% End of document body:
%    \begin{macrocode}
\end{document}
%    \end{macrocode}
%\iffalse
%</samplemain>
%\fi
%
% %%%%%%%%%%%%%%%%%%%%%%%%%%%%%%%%%%%%%%
% \paragraph{Chapter Include Files.}
%
% The include files are called |cdocsch1.tex| and |cdocsch2.tex|.
%
%\iffalse
%<*samplechap1|samplechap2>
%\fi

% Optional override for |\version| flag:
%    \begin{macrocode}
%%\providecommand{\version}{final}
%    \end{macrocode}

% Include the main document:
%    \begin{macrocode}
\input{childdoc.def}
\childdocof{cdocsamp}
%    \end{macrocode}

%\iffalse
%</samplechap1|samplechap2>
%\fi
%
%\iffalse
%<*samplechap1>
%\fi
% Some text for chapter 1:
%    \begin{macrocode}
\section{one}
some text in chapter one
%    \end{macrocode}

%\iffalse
%</samplechap1>
%\fi
% Some text for chapter 2:
%\iffalse
%<*samplechap2>
%\fi
%    \begin{macrocode}
\section{two}
more text in chapter two
%    \end{macrocode}

%\iffalse
%</samplechap2>
%\fi
%
% %%%%%%%%%%%%%%%%%%%%%%%%%%%%%%%%%%%%%%
% \paragraph{Part Include Files.}
%
% The include files are called |cdocspt3.tex| and |cdocspt4.tex|.
%
%\iffalse
%<*samplepart3|samplepart4>
%\fi

% Optional override for |\version| flag:
%    \begin{macrocode}
%%\providecommand{\version}{final}
%    \end{macrocode}

% Include the main document:
%    \begin{macrocode}
\input{childdoc.def}
\childdocby{cdocsamp}
%    \end{macrocode}

%\iffalse
%</samplepart3|samplepart4>
%\fi
%
%\iffalse
%<*samplepart3>
%\fi
% Some text for part 3:
%    \begin{macrocode}
some text in part three
%    \end{macrocode}

%\iffalse
%</samplepart3>
%\fi
% Some text for part 4:
%\iffalse
%<*samplepart4>
%\fi
%    \begin{macrocode}
more text in part four
%    \end{macrocode}

%\iffalse
%</samplepart4>
%\fi
%
% %%%%%%%%%%%%%%%%%%%%%%%%%%%%%%%%%%%%%%
% \paragraph{Forwarding for a Complete Draft.}
%
% The following forwarding file |cdocsdrf.tex|
% compiles the main document in draft mode:
%\iffalse
%<*sampledraft>
%\fi
%    \begin{macrocode}
\def\version{draft}
\input{childdoc.def}
\childdocforward{cdocsamp}
%    \end{macrocode}

%\iffalse
%</sampledraft>
%\fi
%
% %%%%%%%%%%%%%%%%%%%%%%%%%%%%%%%%%%%%%%
% \paragraph{Forwarding for Final Version of the Chapters.}
%
% The following forwarding files |cdocsfn1.tex| and |cdocsfn2.tex|
% (with identical content)
% compile the final versions of the child documents
% |cdocsch1.tex| and |cdocsch2.tex|, respectively:
%\iffalse
%<*samplefinal>
%\fi
%    \begin{macrocode}
\def\version{final}
\input{childdoc.def}
\childdocforwardprefix[cdocsamp]{cdocsfn}{cdocsch}
%    \end{macrocode}

%\iffalse
%</samplefinal>
%\fi
%
% %%%%%%%%%%%%%%%%%%%%%%%%%%%%%%%%%%%%%%
% \paragraph{Command Line Processing.}
%
% The following three command lines generate the output files
% |cdocscld|, |cdocscl1| and |cdocscl2|
% which should be identical to
% |cdocsdrf|, |cdocsch1| and |cdocsfn2|, respectively:
% \begin{center}
% \begin{tabular}{l}
% |latex -jobname cdocscld \|\\
% |  "\def\version{draft}\input{childdoc.def}\childdocforward{cdocsamp}"|\\
% |latex -jobname cdocscl1 \|\\
% |  "\input{childdoc.def}\childdocforward[cdocsamp]{cdocsch1}"|\\
% |latex -jobname cdocscl2 \|\\
% |  "\def\version{final}\input{childdoc.def}\childdocforward{cdocsch2}"|
% \end{tabular}
% \end{center}
% Note that the trailing backslash on each first line
% merely continues the input to the second line
% (for convenient cut ant paste).
% Furthermore, the command |latex| can be replaced by any
% of its alternative versions such as |pdflatex|.
%
% %%%%%%%%%%%%%%%%%%%%%%%%%%%%%%%%%%%%%%%%%%%%%%%%%%%%%%%%%%%%%%%%%%%%%%%%%%%%%%
% %%%%%%%%%%%%%%%%%%%%%%%%%%%%%%%%%%%%%%%%%%%%%%%%%%%%%%%%%%%%%%%%%%%%%%%%%%%%%%
% \section{Implementation}
%\iffalse
%<*package>
%\fi
%
% This section describes the definitions file |childdoc.def|.

% The definitions cannot be loaded using |\usepackage| or |\RequirePackage|
% which has a mechanism to prevent loading a style file more than once.
% When loading the definitions by means of |\input|
% multiple instances have to be prevented manually:
%\iffalse
%This code needs to be before the `\ProvidesFile' directive
%which is defined at the beginning of this file.
%Therefore it is also placed there and commented out here.
%</package>
%<*discard>
%\fi
%    \begin{macrocode}
\ifdefined\childdocmain\endinput\fi
%    \end{macrocode}
%\iffalse
%</discard>
%<*package>
%\fi
%
% \macro{\ifchilddoc}
% \macro{\ifchilddocmanual}
% The conditional |\ifchilddoc| tells whether a
% child (true) or main (false) document is being compiled.
% The conditional |\ifchilddocmanual| tells whether
% the |\includeonly| mechanism is used (false) or
% the selection of child files must be performed manually (true).
% The definitions initialise to false:
%    \begin{macrocode}
\newif\ifchilddoc
\newif\ifchilddocmanual
%    \end{macrocode}

% \macro{\childdocname}
% \macro{\childdocjob}
% The macro |\childdocname| stores the name of the main document
% to be compiled. The macro |\childdocjob| stores the name of
% the document on which the \LaTeX{} compiler was originally invoked.
% The content of |\jobname| cannot be compared
% to filenames specified in the source due to different catcodes.
% The following code rescans |\jobname|, stores the result
% in |\childdocname| and saves a copy in |\childdocjob|:
%    \begin{macrocode}
\edef\childdocname{\scantokens\expandafter{\jobname\noexpand}}
\let\childdocjob\childdocname
%    \end{macrocode}

% \macro{\childdocdisable}
% The macro |\childdocdisable| prevents the main file
% from being processed more than once.
% At this stage, the main document command |\childdocmain|
% is assumed to be called once again where it should do nothing.
% Any subsequent call to it should prevent
% a secondary processing of the main document
% It overwrites the forwarding commands
% |\childdocof| and |\childdocforward|
% with empty macros to prevent further inclusions of the main document:
%    \begin{macrocode}
\newcommand{\childdocdisable}
{
  \renewcommand{\childdocmain}[1]{\renewcommand{\childdocmain}[1]{\endinput}}
  \renewcommand{\childdocof}[1]{}
  \renewcommand{\childdocby}[2][]{}
  \renewcommand{\childdocforward}[2][]{}
  \renewcommand{\childdocdisable}{}
}
%    \end{macrocode}

% \macro{\childdocmain}
% The macro |\childdocmain| is to be called at the top of the main file
% with nothing or the main filename (without extension) as argument.
% First, it breaks loops.
% If the argument is not empty and does not match |\childdocname|
% (which is set by the first inclusion of |childdoc.def|),
% |\ifchilddoc| is set to true, |\includeonly| is applied to the child file
% and |\jobname| is set to the main file
% (for proper handling of |.aux| files):
%    \begin{macrocode}
\newcommand{\childdocmain}[1]
{
  \childdocdisable\childdocmain{}
  \if?#1?\else
    \begingroup
      \def\childdoctmp{#1}
      \ifx\childdoctmp\childdocname
        \def\childdoctmp{}
      \else
        \def\childdoctmp
        {
          \childdoctrue
          \includeonly{\childdocname}
          \def\childdocjob{#1}
          \def\jobname{#1}
        }
      \fi
      \expandafter
    \endgroup
    \childdoctmp
  \fi
}
%    \end{macrocode}

% \macro{\childdocof}
% The command |\childdocof| redirects
% compilation to the main file |#1|.
%    \begin{macrocode}
\newcommand{\childdocof}[1]
{
  \childdocdisable
  \childdoctrue
  \includeonly{\childdocname}
  \def\jobname{#1}
  \def\childdocjob{#1}
  \input{#1}
}
%    \end{macrocode}

% \macro{\childdocby}
% The command |\childdocby| ....
%    \begin{macrocode}
\newcommand{\childdocby}[2][]
{
  \childdocdisable
  \childdoctrue
  \childdocmanualtrue
  \if?#1?\else
    \def\jobname{#2}
  \fi
  \def\childdocjob{#2}
  \input{#2}
  \endinput
}
%    \end{macrocode}

% \macro{\childdocforward}
% The command |\childdocforward| redirects
% compilation to the main file or
% (if the optional argument is given) a child file.
% Parameters are set as if the main file
% or a child file starting with |\childdocof| was compiled.
% Then compilation is handed over to the main file:
%    \begin{macrocode}
\newcommand{\childdocforward}[2][]
{
  \begingroup
    \if?#1?
      \def\childdoctmp
      {
        \def\childdocname{#2}
        \def\childdocjob{#2}
        \def\jobname{#2}
        \input{#2}
        \endinput
      }
    \else
      \def\childdoctmp
      {
        \childdocdisable
        \def\childdocname{#2}
        \childdoctrue
        \includeonly{#2}
        \def\childdocjob{#1}
        \def\jobname{#1}
        \input{#1}
        \endinput
      }
    \fi
    \expandafter
  \endgroup
  \childdoctmp
}
%    \end{macrocode}

% \macro{\childdocforwardprefix}
% The command |\childdocforwardprefix| redirects
% compilation to the main or a child file by means of a pattern.
% The prefix |#1| in the current filename is replaced by |#2|
% and the suffix of the current filename is kept
% (it is assumed that the filename does not contain the substring `|~~~|'
% which is used as a delimiter).
% Compilation is handed over to the new file by |\childdocforward|:
%    \begin{macrocode}
\newcommand{\childdocforwardprefix}[3][]
{
  \begingroup
    \def\childdocextract #2##1~~~{\def\childdoctmp{\childdocforward[#1]{#3##1}}}
    \expandafter\childdocextract\childdocname~~~
    \expandafter
  \endgroup
  \childdoctmp
}
%    \end{macrocode}

% \macro{\childdoc}
% The deprecated macro |\childdoc| is a legacy version of |\childdocmain|:
%    \begin{macrocode}
\newcommand{\childdoc}{\childdocmain}
%    \end{macrocode}

% \macro{\childdocredirect}
% The deprecated macro |\childdocredirect| is a legacy version
% of |\childdocforward| and |\childdocforwardprefix|:
%    \begin{macrocode}
\newcommand{\childdocredirect}[2][]
{
  \begingroup
    \if?#1?
      \def\childdoctmp{\childdocforward{#2}}
    \else
      \def\childdoctmp{\childdocforwardprefix{#1}{#2}}
    \fi
    \expandafter
  \endgroup
  \childdoctmp
}
%    \end{macrocode}

%\iffalse
%</package>
%\fi
%
\endinput

\childdocof{cdocsamp}
%    \end{macrocode}

%\iffalse
%</samplechap1|samplechap2>
%\fi
%
%\iffalse
%<*samplechap1>
%\fi
% Some text for chapter 1:
%    \begin{macrocode}
\section{one}
some text in chapter one
%    \end{macrocode}

%\iffalse
%</samplechap1>
%\fi
% Some text for chapter 2:
%\iffalse
%<*samplechap2>
%\fi
%    \begin{macrocode}
\section{two}
more text in chapter two
%    \end{macrocode}

%\iffalse
%</samplechap2>
%\fi
%
% %%%%%%%%%%%%%%%%%%%%%%%%%%%%%%%%%%%%%%
% \paragraph{Part Include Files.}
%
% The include files are called |cdocspt3.tex| and |cdocspt4.tex|.
%
%\iffalse
%<*samplepart3|samplepart4>
%\fi

% Optional override for |\version| flag:
%    \begin{macrocode}
%%\providecommand{\version}{final}
%    \end{macrocode}

% Include the main document:
%    \begin{macrocode}
% \iffalse
%
% childdoc.dtx Copyright (C) 2017-2018 Niklas Beisert
%
% This work may be distributed and/or modified under the
% conditions of the LaTeX Project Public License, either version 1.3
% of this license or (at your option) any later version.
% The latest version of this license is in
%   http://www.latex-project.org/lppl.txt
% and version 1.3 or later is part of all distributions of LaTeX
% version 2005/12/01 or later.
%
% This work has the LPPL maintenance status `maintained'.
%
% The Current Maintainer of this work is Niklas Beisert.
%
% This work consists of the files childdoc.dtx and childdoc.ins
% and the derived files childdoc.def and cdocsamp.tex with
% cdocsch1.tex, cdocsch2.tex, cdocsdrf.tex, cdocsfn1.tex, cdocsfn2.tex.
%
%<package>\ifdefined\childdocmain\endinput\fi
%<package>\ProvidesFile{childdoc.def}[2018/12/30 v2.0 child document driver]
%<samplemain>\ProvidesFile{cdocsamp.tex}[2018/12/30 v2.0 sample for childdoc]
%<*driver>
%\ProvidesFile{childdoc.drv}[2018/12/30 v2.0 childdoc reference manual file]
\PassOptionsToClass{10pt,a4paper}{article}
\documentclass{ltxdoc}

\usepackage[margin=35mm]{geometry}
\usepackage{hyperref}
\usepackage{hyperxmp}
\usepackage[usenames]{color}

\hypersetup{colorlinks=true}
\hypersetup{pdfstartview=FitH}
\hypersetup{pdfpagemode=UseNone}
\hypersetup{pdfsource={}}
\hypersetup{pdflang={en-UK}}
\hypersetup{pdfcopyright={Copyright 2017-2018 Niklas Beisert.
  This work may be distributed and/or modified under the
  conditions of the LaTeX Project Public License, either version 1.3
  of this license or (at your option) any later version.}}
\hypersetup{pdflicenseurl={http://www.latex-project.org/lppl.txt}}
\hypersetup{pdfcontactaddress={ETH Zurich, ITP, HIT K,
  Wolfgang-Pauli-Strasse 27}}
\hypersetup{pdfcontactpostcode={8093}}
\hypersetup{pdfcontactcity={Zurich}}
\hypersetup{pdfcontactcountry={Switzerland}}
\hypersetup{pdfcontactemail={nbeisert@itp.phys.ethz.ch}}
\hypersetup{pdfcontacturl={http://people.phys.ethz.ch/\xmptilde nbeisert/}}

\newcommand{\secref}[1]{\hyperref[#1]{section \ref*{#1}}}

\parskip1ex
\parindent0pt
\let\olditemize\itemize
\def\itemize{\olditemize\parskip0pt}

\begin{document}

\title{The \textsf{childdoc} Package}
\hypersetup{pdftitle={The childdoc Package}}
\author{Niklas Beisert\\[2ex]
  Institut f\"ur Theoretische Physik\\
  Eidgen\"ossische Technische Hochschule Z\"urich\\
  Wolfgang-Pauli-Strasse 27, 8093 Z\"urich, Switzerland\\[1ex]
  \href{mailto:nbeisert@itp.phys.ethz.ch}
  {\texttt{nbeisert@itp.phys.ethz.ch}}}
\hypersetup{pdfauthor={Niklas Beisert}}
\hypersetup{pdfsubject={Manual for the LaTeX2e Package childdoc}}
\date{30 December 2018, \textsf{v2.0}}
\maketitle

\begin{abstract}\noindent
\textsf{childdoc} is a \LaTeXe{} package
that enables the direct compilation
of document sections included by |\include|
to individual files.
\end{abstract}

\begingroup
\parskip0ex
\tableofcontents
\endgroup

%%%%%%%%%%%%%%%%%%%%%%%%%%%%%%%%%%%%%%%%%%%%%%%%%%%%%%%%%%%%%%%%%%%%%%%%%%%%%%%%
%%%%%%%%%%%%%%%%%%%%%%%%%%%%%%%%%%%%%%%%%%%%%%%%%%%%%%%%%%%%%%%%%%%%%%%%%%%%%%%%
\section{Introduction}

\LaTeX{} provides a mechanism to structure a large document (such as a book)
into a main file and several child files (containing the chapters)
using the |\include| command.
This mechanism is beneficial for documents
which span hundreds of pages in order to
make the source file(s) more manageable.
Moreover, compilation can be restricted to
selected child files by means of the |\includeonly| command.
The latter feature can be used to reduce the compilation time while editing
(this was significantly more useful in the earlier days of \LaTeX{})
or to generate a smaller document which is easier to navigate.
Another application of |\includeonly| is to generate
documents consisting of selected parts of the complete document.

However, there are a few drawbacks of the plain |\include| mechanism:
\begin{itemize}
\item
The child files cannot be compiled on their own,
they can only be compiled via the main file.
A naive editing environment
(such as a text editor with an option
to have the current file processed by \LaTeX)
may require one to switch to the main file before compiling;
attempting to compile the child file produces errors.
\item
The main file must be modified (each time)
to adjust the |\includeonly| command
to the present needs. This easily leaves the main file in a messy state.
\item
The generated document will always carry the filename
of the main document. This is inconvenient if
several child files are to be compiled and
to be kept for distribution.
\end{itemize}

The present package provides a simple interface
to make child files individually compilable by \LaTeX{}.
Compiling a child file then has the same effect as compiling
the main file with an |\includeonly| command
to select the appropriate child.
Moreover the generated document will carry the name of the child
rather than the main file.
This resolves all three above issues.

This feature is meant to make the editing of books,
thesis documents and lecture notes somewhat more convenient.
However, the package can also be used efficiently for
composing a series of documents (such as exercise sheets)
which are typically distributed individually.
It then assists the author in generating the individual documents
(potentially in different versions)
as well as a document containing the collected series.
Another application is in developing style files
or other kinds of included material
where compilation of the style file could redirect
to a sample or test file.

%%%%%%%%%%%%%%%%%%%%%%%%%%%%%%%%%%%%%%%%%%%%%%%%%%%%%%%%%%%%%%%%%%%%%%%%%%%%%%%%
%%%%%%%%%%%%%%%%%%%%%%%%%%%%%%%%%%%%%%%%%%%%%%%%%%%%%%%%%%%%%%%%%%%%%%%%%%%%%%%%
\section{Usage}

First of all, the package \textsf{childdoc} is \emph{not} a standard
\LaTeXe{} |.sty| style file! Therefore it needs to be invoked in
a non-standard way.

%%%%%%%%%%%%%%%%%%%%%%%%%%%%%%%%%%%%%%%%%%%%%%%%%%%%%%%%%%%%%%%%%%%%%%%%%%%%%%%%
\subsection{Included Files}
\label{sec:include}

%%%%%%%%%%%%%%%%%%%%%%%%%%%%%%%%%%%%%%%%
\DescribeMacro{\childdocmain}
To use the package, add the commands
\begin{center}
\begin{tabular}{l}
|\input{childdoc.def}|\\
|\childdocmain{}|\\
\end{tabular}
\end{center}
at the very top of the main \LaTeX{} file,
in particular \emph{before} the |\documentclass| statement!
The argument of |\childdocmain| should be left empty
(but it must be present).

%%%%%%%%%%%%%%%%%%%%%%%%%%%%%%%%%%%%%%%%
\DescribeMacro{\childdocof}
Furthermore, add the commands
\begin{center}
\begin{tabular}{l}
|\input{childdoc.def}|\\
|\childdocof{|\textit{main}|}|\\
\end{tabular}
\end{center}
at the top of every child file \textit{child}
which is included by |\include{|\textit{child}|}|
from within the main file
(or at least for those files to be compiled individually).
The argument \textit{main} must be the filename of the main file.

There are a couple of
considerations in setting up the main and child documents:

%%%%%%%%%%%%%%%%%%%%%%%%%%%%%%%%%%%%%%%%
\paragraph{Restrictions.}

Please note the following restrictions:
\begin{itemize}
\item
|\childdocmain| must be called with one argument \textit{main}
to ensure compatibility with earlier version of the package.
It must either be empty (|\childdocmain{}|)
or precisely match the filename of the main file in which it is specified.
See \secref{sec:detection} for further information.
\item
The filename \textit{main} must be specified without the |.tex| extension.
\item
The filename \textit{main} is case sensitive
(even in case-insensitive file systems)
due to internal string comparison.
\item
The argument \textit{main} should be fully expanded, it cannot be a macro.
\item
Subdirectories and special characters should be avoided in filenames.
\item
The command |\childdocmain{|\textit{main}|}| must be followed by a whitespace.
It should not be followed immediately by another command
or by a comment mark `|%|'.
This is because the \TeX{} parser reads the token immediately following
the argument of |\childdocmain| and puts it
at the beginning of every child section;
however, a white\-space is ignored.
\end{itemize}

%%%%%%%%%%%%%%%%%%%%%%%%%%%%%%%%%%%%%%%%
\paragraph{Content of Main File.}

It is advisable to place all content in the child files included by |\include|.
Any output contained in the main file will appear in all child documents
unless suppressed manually;
it cannot be suppressed automatically by the |\includeonly| directive
and thus should normally be avoided.
A method to include some content in the main file
by means of conditional processing is described in \secref{sec:conditional}.

%%%%%%%%%%%%%%%%%%%%%%%%%%%%%%%%%%%%%%%%
\paragraph{Page Numbering.}

When only a part of the document is compiled,
the appropriate numbering of pages
(as well as other status parameters)
is determined from the |.aux| files.
The latter contain information from previous passes.
However this information needs to propagate through
all intermediate child documents.
Therefore the page numbering in child documents may well
be inconsistent until the complete document is compiled at least once.

A useful (if unconventional) way to always ensure a consistent
page numbering is to restart the numbering in each child document
and denote the pages by `\textit{child}|.|\textit{page}'
where \textit{child} represents the chapter/section number of the child file.
This can be achieved by the command
|\numberwithin{page}{|\textit{child}|}|
of the \textsf{amsmath} package
where \textit{child} can be |chapter| or |section|
depending on the chosen structuring.
Alternatively, one can modify the macro |\thepage| appropriately
and reset the counter |page| at the start of each child file.

%%%%%%%%%%%%%%%%%%%%%%%%%%%%%%%%%%%%%%%%%%%%%%%%%%%%%%%%%%%%%%%%%%%%%%%%%%%%%%%%
\subsection{Conditional Processing}
\label{sec:conditional}

The package provides a mechanism to compile different versions
of a document. To customise the versions further some conditional processing
can come in handy to distinguish which version is being compiled.
The package provides two macros to describe the compilation context:

%%%%%%%%%%%%%%%%%%%%%%%%%%%%%%%%%%%%%%%%
\DescribeMacro{\ifchilddoc}
The conditional |\ifchilddoc| distinguishes between the compilation of
child documents and the main document:
%
\begin{center}
|\ifchilddoc |\textit{child-code}| |[|\||else |\textit{main-code}]| \||fi|
\end{center}

%%%%%%%%%%%%%%%%%%%%%%%%%%%%%%%%%%%%%%%%
\DescribeMacro{\childdocname}
\DescribeMacro{\childdocjob}
The macro |\childdocname| contains the filename (without extension)
of the main or child file being processed.
Note that |\childdocjob| will always contain the name of the main file.

%%%%%%%%%%%%%%%%%%%%%%%%%%%%%%%%%%%%%%%%
\paragraph{Title Page.}

Conditional processing can be used to include a title or banner page
in the main document when proper precautions are taken.
Importantly, the code in the main file should ensure that the page counter
(as well as other status parameters which are stored in the |.aux| files)
takes the same value after the conditional processing.
Otherwise the page numbers may take divergent values
depending on which part is compiled.

For example, a title page could be declared by:
%
\begin{center}
\begin{tabular}{l}
|\ifchilddoc\||else|\\
|\addtocounter{page}{-1}|\\
\textit{code for title page}\\
|\newpage|\\
|\||fi|
\end{tabular}
\end{center}
%
A banner page for the child documents can be generated by:
%
\begin{center}
\begin{tabular}{l}
|\ifchilddoc|\\
|\addtocounter{page}{-1}|\\
\textit{code for banner page}\\
|\newpage|\\
|\||fi|
\end{tabular}
\end{center}
%
Here one could write a message such as:
\begin{center}
|This is the part \childdocname{} of \childdocjob{}.|
\end{center}

%%%%%%%%%%%%%%%%%%%%%%%%%%%%%%%%%%%%%%%%%%%%%%%%%%%%%%%%%%%%%%%%%%%%%%%%%%%%%%%%
\subsection{Flags}
\label{sec:flags}

The package makes it easy to generate different versions
of the main or child documents.
To this end compilation flags can be defined
and assigned different default values.
They will be particularly useful in conjunction
with the forwarding mechanism described in \secref{sec:forward}.

For example, it may be useful to have a flag |\version|
which can be set to |draft| or |final|.
The document source will contain some conditional code
depending on the value of |\version|.
Suppose further, the flag should default to |final| for the main file
and to |draft| for child files
which is a natural assignment for editing the document.
This is achieved by placing the following code
in the preamble of the main document
(below the |\childdocmain| directive):
%
\begin{center}
\begin{tabular}{l}
|\ifchilddoc|\\
|\providecommand{\version}{draft}|\\
|\||else|\\
|\providecommand{\version}{final}|\\
|\||fi|
\end{tabular}
\end{center}
%
The definition by |\providecommand| makes sure
that previous definitions are not overwritten.
Further statements |\providecommand{\version}{...}|
can thus be added before the above code to override it.

For the main file, one might add a line
(between |\childdocmain| and the above block)
%
\begin{center}
|%\ifchilddoc\||else\providecommand{\version}{draft}\||fi|
\end{center}
%
which can be uncommented to produce a draft version.
Likewise one can add a line to the very top of a child file
(above the |\childdocof{|\textit{main}|}| directive)
%
\begin{center}
|%\providecommand{\version}{final}|
\end{center}
%
which can be uncommented to produce the final version of this child document.

%%%%%%%%%%%%%%%%%%%%%%%%%%%%%%%%%%%%%%%%%%%%%%%%%%%%%%%%%%%%%%%%%%%%%%%%%%%%%%%%
\subsection{Forwarding}
\label{sec:forward}

Different versions of the main or child documents
using compilation flags as described in \secref{sec:flags}
can be (permanently) stored in different files
for convenient compilation, viewing and distribution.
To this end, the package defines a command
to pass on compilation to a different file:

%%%%%%%%%%%%%%%%%%%%%%%%%%%%%%%%%%%%%%%%
\DescribeMacro{\childdocforward}
The command |\childdocforward| redirects processing to
another source file:
%
\begin{center}
\begin{tabular}{l}
|\input{childdoc.def}|\\
|\childdocforward[|\textit{main}|]{|\textit{dest}|}|\\
\end{tabular}
\end{center}
%
The argument \textit{dest} is the destination file
(without extension).
It should be the main file or one of the child files.
Note that further \textsf{childdoc} directives
such as |\childdocof| and |\childdocforward|
in the indicated file will be processed in this form.
The optional argument \textit{main}
passes on directly to the main file \textit{main}
while pretending to compile the child \textit{dest}.
This form behaves as if \textit{dest}
issues |\childdocof{|\textit{main}|}| right away,
and no further \textsf{childdoc} directives will be processed.

%%%%%%%%%%%%%%%%%%%%%%%%%%%%%%%%%%%%%%%%
\DescribeMacro{\...prefix}
In the alternative form |\childdocforwardprefix|,
%
\begin{center}
\begin{tabular}{l}
|\input{childdoc.def}|\\
|\childdocforwardprefix[|\textit{main}|]{|\textit{prefix}|}{|\textit{dest}|}|
\end{tabular}
\end{center}
%
the destination file is determined by a pattern
depending on the current file:
To make this work, the current file must be called
`{\textit{prefix}\hspace{0.2em}\textit{suffix}}'
with \textit{prefix} matching precisely the argument.
Processing is then passed on to the file
`{\textit{dest}\hspace{0.2em}\textit{suffix}}'.
Surely, the same effect is achieved by
directly specifying the
argument `{\textit{dest}\hspace{0.2em}\textit{suffix}}'
in the first form.
However, that requires to set up a different file
for each child. With the alternative form of the command
all these files can have exactly the same content
which simplifies setting them up and maintaining them.

For example, the following file |draft.tex|
with a compilation flag |\version| as described in \secref{sec:flags}
compiles the main document as a draft:
%
\begin{center}
\begin{tabular}{l}
|\def\version{draft}|\\
|\input{childdoc.def}|\\
|\childdocforward{|\textit{main}|}|
\end{tabular}
\end{center}
%
Likewise, the following files |final|\textit{nn}|.tex|
compile the final version of the child document
|child|\textit{nn}|.tex|:
%
\begin{center}
\begin{tabular}{l}
|\def\version{final}|\\
|\input{childdoc.def}|\\
|\childdocforwardprefix{final}{child}|
\end{tabular}
\end{center}
%

Note that when several versions of a main file and/or of each child file
are to be generated, it may be convenient to set up a |Makefile| or
shell script to automatise the process.

%%%%%%%%%%%%%%%%%%%%%%%%%%%%%%%%%%%%%%%%%%%%%%%%%%%%%%%%%%%%%%%%%%%%%%%%%%%%%%%%
\subsection{Command Line Processing}
\label{sec:commandline}

The effect of redirection files can also be achieved by invoking
the \LaTeX{} compiler with a more elaborate command line.
Most conveniently this should be done as part
of a shell script or a |Makefile|.

When using \textsf{childdoc} in the main file, the following
command lines effectively perform a redirection
(note that depending on the shell being used,
backslashes may have to be doubled: `|\|' $\to$ `|\\|'):
%
\begin{center}
|... -jobname "|\textit{target}|" |\\|"|[\textit{flags}]%
|\input{childdoc.def}\childdocforward[|\textit{main}|]{|\textit{dest}|}"|
\end{center}
%
Here \textit{target} is the name of the output file,
\textit{main} is the name of the main file
and \textit{dest} is the name of the main or child file to be processed
(all filenames without extensions).
The optional argument \textit{main} can be omitted
if \textit{main} matches \textit{dest}.
Optionally, compilation \textit{flags} can be defined via |\def| commands.
This command line makes the \TeX{} engine believe
it is compiling the file \textit{target}
whose content is specified as the latter parameter.
The provided code then forwards the processing to
\textit{main} or \textit{dest} as described in \secref{sec:forward}.

%%%%%%%%%%%%%%%%%%%%%%%%%%%%%%%%%%%%%%%%%%%%%%%%%%%%%%%%%%%%%%%%%%%%%%%%%%%%%%%%
\subsection{Include by Input}
\label{sec:input}

Including child documents by |\include| has some restrictions by design.
Most notably, the content of a child document always occupies
its own set of pages; pages cannot be shared between child documents.
Usually, this behaviour makes perfect sense
because each child document contain an essential part of the document.
However, in some situations it may be desirable to compose
a document from a collection of parts
without having mandatory page breaks between then.
For this case, the package
provides a mechanism to include parts
by |\input| which can also be processed individually.
However, by construction this mechanism
requires manual handling of the content to be output.

%%%%%%%%%%%%%%%%%%%%%%%%%%%%%%%%%%%%%%%%
\DescribeMacro{\ifchilddocmanual}
The main file should be prepared as usual, see \secref{sec:include}.
However, the document body must make a distinction
between processing of an individual part and of the main document, e.g.:
%
\begin{center}
\begin{tabular}{l}
|\ifchilddocmanual|\\
|\input{\childdocname}|\\
|\||else|\\
\textit{document body with }|\input{|\textit{part}|}|\\
|\||fi|
\end{tabular}
\end{center}
%
The conditional |\ifchilddocmanual| is true whenever
a part to be included by |\input| is being compiled,
and the name of the part is stored in |\childdocname|.

%%%%%%%%%%%%%%%%%%%%%%%%%%%%%%%%%%%%%%%%
\DescribeMacro{\childdocby}
Each part to be included by |\input| should start with:
%
\begin{center}
\begin{tabular}{l}
|\input{childdoc.def}|\\
|\childdocby{|\textit{main}|}|\\
\end{tabular}
\end{center}
%
The directive |\childdocby| is similar to |\childdocof|
described in \secref{sec:include},
but the subsequent selection of content must be done manually.
To that end, both |\ifchilddoc| and |\ifchilddocmanual|
will be true upon processing of a part,
and the name of the part is stored in |\childdocname|.
Note that |\jobname| will be set to the filename of the current part
so that each part receives an individual |.aux| file
that does not interfere with the |.aux| file(s) of the main document.
This behaviour can be altered by the alternative form
|\childdocby[*]{|\textit{main}|}| (with a non-empty optional argument)
which uses the |.aux| file of the main document
by setting |\jobname| to \textit{main}.

%%%%%%%%%%%%%%%%%%%%%%%%%%%%%%%%%%%%%%%%%%%%%%%%%%%%%%%%%%%%%%%%%%%%%%%%%%%%%%%%
\subsection{Driver Development}
\label{sec:driver}

The \textsf{childdoc} mechanism can also be use for the development
of definition files such as \LaTeX{} styles or classes.
This case differs from the above setup with multiple parts
included by |\include| in that no |\includeonly| should be invoked.
This can be achieved by starting the include file
(before |\ProvidesPackage|) with:
%
\begin{center}
\begin{tabular}{l}
|\input{childdoc.def}|\\
|\childdocforward{|\textit{main}|}|\\
\end{tabular}
\end{center}
%
or alternatively with:
%
\begin{center}
\begin{tabular}{l}
|\input{childdoc.def}|\\
|\childdocby{|\textit{main}|}|\\
\end{tabular}
\end{center}
%
Both forms have slightly different effects as described above.
The main file is prepared as usual, see \secref{sec:include}.

%%%%%%%%%%%%%%%%%%%%%%%%%%%%%%%%%%%%%%%%%%%%%%%%%%%%%%%%%%%%%%%%%%%%%%%%%%%%%%%%
\subsection{Legacy Detection}
\label{sec:detection}

The directive |\childdocmain| in the main file can detect
whether the complete document or merely a child is to be compiled
even without using the directive |\childdocof|.
This method is deprecated because it is less robust
and there is no compelling reason to use it;
it is merely provided for backward compatibility
and it may be removed in future versions.

If the detection mechanism is to be used,
it is mandatory to correctly specify
the filename of the main file as the argument of |\childdocmain|:
%
\begin{center}
\begin{tabular}{l}
|\input{childdoc.def}|\\
|\childdocmain{|\textit{main}|}|\\
\end{tabular}
\end{center}
%
If |\jobname| does not match the argument \textit{main} of |\childdocmain|,
it is assumed that |\jobname| points to the child file to be compiled.
When using |\childdocmain| with the main file specified as argument,
it suffices to start a child file
with just |\input{|\textit{main}|}|
without loading of the package and using |\childdocof|.
If instead all processing is done
with the appropriate \textsf{childdoc} directives,
the argument of \textit{main} of |\childdocmain| can be empty.

An alternative version of the command line processing described
in \secref{sec:commandline} using the detection mechanism reads:
%
\begin{center}
|... -jobname "|\textit{target}|" "|[\textit{flags}]%
[|\def\jobname{|\textit{dest}|}|]|\input{|\textit{main}|}"|
\end{center}

%%%%%%%%%%%%%%%%%%%%%%%%%%%%%%%%%%%%%%%%%%%%%%%%%%%%%%%%%%%%%%%%%%%%%%%%%%%%%%%%
\subsection{Manual Code}
\label{sec:manual}

In case one cannot be certain whether the definitions file |childdoc.def|
is installed on the target \TeX{} distribution
and one prefers not to ship it,
it is conceivable to paste a few relevant commands into the sources.

To that end, drop all statements |\input{childdoc.def}|
and perform the replacements as outlined below.
Instead of |\childdocmain{|\textit{main}|}| add the following code
to the top of the main file:
%
\begin{center}
\begin{tabular}{l}
|\||ifdefined\childdocname\endinput\||fi\newif\ifchilddoc|\\
|\edef\childdocname{\scantokens\expandafter{\jobname\noexpand}}|\\
|\def\childdocmain{|\textit{main}|}\||ifx\childdocmain\childdocname\||else|\\
|\childdoctrue\includeonly{\childdocname}\let\jobname\childdocmain\||fi|\\
\end{tabular}
\end{center}
%
Instead of |\childdocof{|\textit{main}|}| just include the main file
at the top of each child file:
%
\begin{center}
|\input{|\textit{main}|}|
\end{center}
%
A simple redirection |\childdocforward{|\textit{dest}|}| is achieved by:
%
\begin{center}
|\def\jobname{|\textit{dest}|}\input{\jobname}|
\end{center}
%
The redirection with prefix
|\childdocforwardprefix[|\textit{prefix}|]{|\textit{dest}|}|
is accomplished by:
%
\begin{center}
\begin{tabular}{l}
|{\edef\jobname{\scantokens\expandafter{\jobname\noexpand}}|\\
|\def\redirectjob |\textit{prefix}|#1~~~{\gdef\jobname{|\textit{dest}|#1}}|\\
|\expandafter\redirectjob\jobname~~~}\input{\jobname}|
\end{tabular}
\end{center}

In an alternative approach,
child documents can be compiled by a specific command line
without additional code or specific definitions:
%
\begin{center}
|... -jobname "|\textit{target}|" "|[\textit{flags}]%
|\includeonly{|\textit{dest}|}\input{|\textit{main}|}"|
\end{center}
%

%%%%%%%%%%%%%%%%%%%%%%%%%%%%%%%%%%%%%%%%%%%%%%%%%%%%%%%%%%%%%%%%%%%%%%%%%%%%%%%%
%%%%%%%%%%%%%%%%%%%%%%%%%%%%%%%%%%%%%%%%%%%%%%%%%%%%%%%%%%%%%%%%%%%%%%%%%%%%%%%%
\section{Information}

%%%%%%%%%%%%%%%%%%%%%%%%%%%%%%%%%%%%%%%%%%%%%%%%%%%%%%%%%%%%%%%%%%%%%%%%%%%%%%%%
\subsection{Copyright}

Copyright \copyright{} 2017--2018 Niklas Beisert

This work may be distributed and/or modified under the
conditions of the \LaTeX{} Project Public License, either version 1.3
of this license or (at your option) any later version.
The latest version of this license is in
  \url{http://www.latex-project.org/lppl.txt}
and version 1.3 or later is part of all distributions of \LaTeX{}
version 2005/12/01 or later.

This work has the LPPL maintenance status `maintained'.

The Current Maintainer of this work is Niklas Beisert.

This work consists of the files |README.txt|, |childdoc.ins| and |childdoc.dtx|
as well as the derived files |childdoc.def|, |cdocsamp.tex|
with |cdocsch1.tex|, |cdocsch2.tex|, |cdocspt3.tex|, |cdocspt4.tex|,
|cdocsdrf.tex|, |cdocsfn1.tex|, |cdocsfn2.tex|
as well as |childdoc.pdf|.

%%%%%%%%%%%%%%%%%%%%%%%%%%%%%%%%%%%%%%%%%%%%%%%%%%%%%%%%%%%%%%%%%%%%%%%%%%%%%%%%
\subsection{Files and Installation}

The package consists of the files:
%
\begin{center}
\begin{tabular}{ll}
    |README.txt|   & readme file \\
    |childdoc.ins| & installation file \\
    |childdoc.dtx| & source file \\
    |childdoc.def| & definition file \\
    |cdocsamp.tex| & sample main file \\
    |cdocsch1.tex| & sample include file \\
    |cdocsch2.tex| & sample include file \\
    |cdocspt3.tex| & sample part file \\
    |cdocspt4.tex| & sample part file \\
    |cdocsdrf.tex| & sample redirection file \\
    |cdocsfn1.tex| & sample redirection file \\
    |cdocsfn2.tex| & sample redirection file \\
    |childdoc.pdf| & manual
\end{tabular}
\end{center}
%
The distribution consists of the files
|README.txt|, |childdoc.ins| and |childdoc.dtx|.
%
\begin{itemize}
\item
Run (pdf)\LaTeX{} on |childdoc.dtx|
to compile the manual |childdoc.pdf| (this file).
\item
Run \LaTeX{} on |childdoc.ins| to create the definitions file |childdoc.def|
and the sample |cdocsamp.tex| with include files
|cdocsch1.tex|, |cdocsch2.tex|, |cdocspt3.tex|, |cdocspt4.tex|,
|cdocsdrf.tex|, |cdocsfn1.tex|, |cdocsfn2.tex|.
Then copy the file |childdoc.def| to an appropriate directory of your \LaTeX{}
distribution, e.g.\ \textit{texmf-root}|/tex/latex/childdoc|.
\end{itemize}

%%%%%%%%%%%%%%%%%%%%%%%%%%%%%%%%%%%%%%%%%%%%%%%%%%%%%%%%%%%%%%%%%%%%%%%%%%%%%%%%
\subsection{Related CTAN Packages}

There are several other packages which offer a similar functionality:
%
\begin{itemize}
\item
The packages
\href{http://ctan.org/pkg/docmute}{\textsf{docmute}},
\href{http://ctan.org/pkg/includex}{\textsf{includex}} and
\href{http://ctan.org/pkg/standalone}{\textsf{standalone}}
provide commands to include only the document body of
a child file thus allowing both files to be compiled individually.
\item
The packages \href{http://ctan.org/pkg/subdocs}{\textsf{subdocs}}
and \href{http://ctan.org/pkg/subfiles}{\textsf{subfiles}}
provide structures in which the main and child documents can be
encapsulated and allowing them to be compiled individually.
The inclusion mechanism is different from the conventional |\include|.
\item
The package \href{http://ctan.org/pkg/combine}{\textsf{combine}}
is an elaborate solution to combine several documents into one.
\end{itemize}
%
See also the CTAN topic \href{http://ctan.org/topic/subdocs}{\textsf{subdocs}}
for further related packages.
The present package differs from the above solutions in that
a document structure constructed with the conventional |\include| mechanism
just needs two extra commands at the top of every file
such that all constituent files can be compiled individually.

%%%%%%%%%%%%%%%%%%%%%%%%%%%%%%%%%%%%%%%%%%%%%%%%%%%%%%%%%%%%%%%%%%%%%%%%%%%%%%%%
%\subsection{Feature Suggestions}
%
%The following is a list of features which may be useful for future
%versions of this package:
%%
%\begin{itemize}
%\item
%\ldots
%\end{itemize}

%%%%%%%%%%%%%%%%%%%%%%%%%%%%%%%%%%%%%%%%%%%%%%%%%%%%%%%%%%%%%%%%%%%%%%%%%%%%%%%%
\subsection{Revision History}

%%%%%%%%%%%%%%%%%%%%%%%%%%%%%%%%%%%%%%%%
\paragraph{v2.0:} 2018/12/30

\begin{itemize}
\item
immediate forward processing
\item
added |\childdocby| mechanism
\item
manual restructured
\end{itemize}

%%%%%%%%%%%%%%%%%%%%%%%%%%%%%%%%%%%%%%%%
\paragraph{v1.6:} 2018/01/17

\begin{itemize}
\item
application for development of include files
\item
corrections to manual
\end{itemize}

%%%%%%%%%%%%%%%%%%%%%%%%%%%%%%%%%%%%%%%%
\paragraph{v1.5:} 2017/05/21

\begin{itemize}
\item
more complete structuring introduced
\item
|\childdocof| introduced
\item
|\childdoc| renamed to |\childdocmain|
\item
|\childredirect| renamed to |\childdocforward| and |\childdocforwardprefix|
and functionality expanded
\end{itemize}

%%%%%%%%%%%%%%%%%%%%%%%%%%%%%%%%%%%%%%%%
\paragraph{v1.0:} 2017/04/27

\begin{itemize}
\item
manual and install package
\item
first version published on CTAN
\end{itemize}

%%%%%%%%%%%%%%%%%%%%%%%%%%%%%%%%%%%%%%%%
\paragraph{v0.6:} 2017/04/26

\begin{itemize}
\item
redirection mechanism added
\end{itemize}

%%%%%%%%%%%%%%%%%%%%%%%%%%%%%%%%%%%%%%%%
\paragraph{v0.5:} 2017/04/26

\begin{itemize}
\item
functionality in definition file
\end{itemize}


%%%%%%%%%%%%%%%%%%%%%%%%%%%%%%%%%%%%%%%%%%%%%%%%%%%%%%%%%%%%%%%%%%%%%%%%%%%%%%%%
%%%%%%%%%%%%%%%%%%%%%%%%%%%%%%%%%%%%%%%%%%%%%%%%%%%%%%%%%%%%%%%%%%%%%%%%%%%%%%%%
%%%%%%%%%%%%%%%%%%%%%%%%%%%%%%%%%%%%%%%%%%%%%%%%%%%%%%%%%%%%%%%%%%%%%%%%%%%%%%%%
\appendix

\settowidth\MacroIndent{\rmfamily\scriptsize 000\ }

 \DocInput{childdoc.dtx}

\end{document}
%</driver>
% \fi
%
% %%%%%%%%%%%%%%%%%%%%%%%%%%%%%%%%%%%%%%%%%%%%%%%%%%%%%%%%%%%%%%%%%%%%%%%%%%%%%%
% %%%%%%%%%%%%%%%%%%%%%%%%%%%%%%%%%%%%%%%%%%%%%%%%%%%%%%%%%%%%%%%%%%%%%%%%%%%%%%
% \section{Sample}
%\iffalse
%<*samplemain>
%\fi
%
% The following presents a sample document
% with two chapters, two parts, a title page,
% a compile flag as well as three forwarding files to set the flag.
% It consists of eight |.tex| files:
% \begin{center}
% \begin{tabular}{ll}
% |cdocsamp.tex|&main file\\
% |cdocsch1.tex|&include file for chapter 1\\
% |cdocsch2.tex|&include file for chapter 2\\
% |cdocspt3.tex|&include file for part 3\\
% |cdocspt4.tex|&include file for part 4\\
% |cdocsdrf.tex|&forwarding file for main file in draft mode\\
% |cdocsfi1.tex|&forwarding file for final version of chapter 1\\
% |cdocsfi2.tex|&forwarding file for final version of chapter 2\\
% \end{tabular}
% \end{center}
% Each of the eight files can be compiled directly by the \LaTeX{} compiler.
%
% %%%%%%%%%%%%%%%%%%%%%%%%%%%%%%%%%%%%%%
% \paragraph{Main File.}
%
% The main file is called |cdocsamp.tex|.
%
% Load the \textsf{childdoc} definitions and
% declare the filename for the main document:
%    \begin{macrocode}
\input{childdoc.def}
\childdocmain{}
%    \end{macrocode}

% Optional override for |\version| flag:
%    \begin{macrocode}
%%\ifchilddoc\else\providecommand{\version}{draft}\fi
%    \end{macrocode}

% Define the default values for the |\version| flag
% (|final| for the main file and |draft| for childs):
%    \begin{macrocode}
\ifchilddoc
\providecommand{\version}{draft}
\else
\providecommand{\version}{final}
\fi
%    \end{macrocode}

% Load the standard document class:
%    \begin{macrocode}
\documentclass[12pt]{article}
%    \end{macrocode}

% Start the document body:
%    \begin{macrocode}
\begin{document}
%    \end{macrocode}

% Declare a title page.
% Print title, part of document being processed and version flag:
%    \begin{macrocode}
\addtocounter{page}{-1}
\begin{center}
{\LARGE\bfseries{}childdoc example\par}
\vspace{1cm}
\ifchilddoc
\ifchilddocmanual part\else chapter\fi:
`\childdocname' of `\childdocjob'\par
\else
main document: `\childdocjob'\par
\fi
version: \version\par
\end{center}
\newpage
%    \end{macrocode}

% Manually include selected file,
% otherwise process as usual:
%    \begin{macrocode}
\ifchilddocmanual
\section*{part `\childdocname'}
\input{\childdocname}
\else
%    \end{macrocode}

% Include the two chapters:
%    \begin{macrocode}
\include{cdocsch1}
\include{cdocsch2}
%    \end{macrocode}

% Include the two parts unless only chapters should be displayed:
%    \begin{macrocode}
\ifchilddoc\else
\section{part three}
\input{cdocspt3}
\section{part four}
\input{cdocspt4}
\fi
%    \end{macrocode}

% Process as usual until here:
%    \begin{macrocode}
\fi
%    \end{macrocode}

% End of document body:
%    \begin{macrocode}
\end{document}
%    \end{macrocode}
%\iffalse
%</samplemain>
%\fi
%
% %%%%%%%%%%%%%%%%%%%%%%%%%%%%%%%%%%%%%%
% \paragraph{Chapter Include Files.}
%
% The include files are called |cdocsch1.tex| and |cdocsch2.tex|.
%
%\iffalse
%<*samplechap1|samplechap2>
%\fi

% Optional override for |\version| flag:
%    \begin{macrocode}
%%\providecommand{\version}{final}
%    \end{macrocode}

% Include the main document:
%    \begin{macrocode}
\input{childdoc.def}
\childdocof{cdocsamp}
%    \end{macrocode}

%\iffalse
%</samplechap1|samplechap2>
%\fi
%
%\iffalse
%<*samplechap1>
%\fi
% Some text for chapter 1:
%    \begin{macrocode}
\section{one}
some text in chapter one
%    \end{macrocode}

%\iffalse
%</samplechap1>
%\fi
% Some text for chapter 2:
%\iffalse
%<*samplechap2>
%\fi
%    \begin{macrocode}
\section{two}
more text in chapter two
%    \end{macrocode}

%\iffalse
%</samplechap2>
%\fi
%
% %%%%%%%%%%%%%%%%%%%%%%%%%%%%%%%%%%%%%%
% \paragraph{Part Include Files.}
%
% The include files are called |cdocspt3.tex| and |cdocspt4.tex|.
%
%\iffalse
%<*samplepart3|samplepart4>
%\fi

% Optional override for |\version| flag:
%    \begin{macrocode}
%%\providecommand{\version}{final}
%    \end{macrocode}

% Include the main document:
%    \begin{macrocode}
\input{childdoc.def}
\childdocby{cdocsamp}
%    \end{macrocode}

%\iffalse
%</samplepart3|samplepart4>
%\fi
%
%\iffalse
%<*samplepart3>
%\fi
% Some text for part 3:
%    \begin{macrocode}
some text in part three
%    \end{macrocode}

%\iffalse
%</samplepart3>
%\fi
% Some text for part 4:
%\iffalse
%<*samplepart4>
%\fi
%    \begin{macrocode}
more text in part four
%    \end{macrocode}

%\iffalse
%</samplepart4>
%\fi
%
% %%%%%%%%%%%%%%%%%%%%%%%%%%%%%%%%%%%%%%
% \paragraph{Forwarding for a Complete Draft.}
%
% The following forwarding file |cdocsdrf.tex|
% compiles the main document in draft mode:
%\iffalse
%<*sampledraft>
%\fi
%    \begin{macrocode}
\def\version{draft}
\input{childdoc.def}
\childdocforward{cdocsamp}
%    \end{macrocode}

%\iffalse
%</sampledraft>
%\fi
%
% %%%%%%%%%%%%%%%%%%%%%%%%%%%%%%%%%%%%%%
% \paragraph{Forwarding for Final Version of the Chapters.}
%
% The following forwarding files |cdocsfn1.tex| and |cdocsfn2.tex|
% (with identical content)
% compile the final versions of the child documents
% |cdocsch1.tex| and |cdocsch2.tex|, respectively:
%\iffalse
%<*samplefinal>
%\fi
%    \begin{macrocode}
\def\version{final}
\input{childdoc.def}
\childdocforwardprefix[cdocsamp]{cdocsfn}{cdocsch}
%    \end{macrocode}

%\iffalse
%</samplefinal>
%\fi
%
% %%%%%%%%%%%%%%%%%%%%%%%%%%%%%%%%%%%%%%
% \paragraph{Command Line Processing.}
%
% The following three command lines generate the output files
% |cdocscld|, |cdocscl1| and |cdocscl2|
% which should be identical to
% |cdocsdrf|, |cdocsch1| and |cdocsfn2|, respectively:
% \begin{center}
% \begin{tabular}{l}
% |latex -jobname cdocscld \|\\
% |  "\def\version{draft}\input{childdoc.def}\childdocforward{cdocsamp}"|\\
% |latex -jobname cdocscl1 \|\\
% |  "\input{childdoc.def}\childdocforward[cdocsamp]{cdocsch1}"|\\
% |latex -jobname cdocscl2 \|\\
% |  "\def\version{final}\input{childdoc.def}\childdocforward{cdocsch2}"|
% \end{tabular}
% \end{center}
% Note that the trailing backslash on each first line
% merely continues the input to the second line
% (for convenient cut ant paste).
% Furthermore, the command |latex| can be replaced by any
% of its alternative versions such as |pdflatex|.
%
% %%%%%%%%%%%%%%%%%%%%%%%%%%%%%%%%%%%%%%%%%%%%%%%%%%%%%%%%%%%%%%%%%%%%%%%%%%%%%%
% %%%%%%%%%%%%%%%%%%%%%%%%%%%%%%%%%%%%%%%%%%%%%%%%%%%%%%%%%%%%%%%%%%%%%%%%%%%%%%
% \section{Implementation}
%\iffalse
%<*package>
%\fi
%
% This section describes the definitions file |childdoc.def|.

% The definitions cannot be loaded using |\usepackage| or |\RequirePackage|
% which has a mechanism to prevent loading a style file more than once.
% When loading the definitions by means of |\input|
% multiple instances have to be prevented manually:
%\iffalse
%This code needs to be before the `\ProvidesFile' directive
%which is defined at the beginning of this file.
%Therefore it is also placed there and commented out here.
%</package>
%<*discard>
%\fi
%    \begin{macrocode}
\ifdefined\childdocmain\endinput\fi
%    \end{macrocode}
%\iffalse
%</discard>
%<*package>
%\fi
%
% \macro{\ifchilddoc}
% \macro{\ifchilddocmanual}
% The conditional |\ifchilddoc| tells whether a
% child (true) or main (false) document is being compiled.
% The conditional |\ifchilddocmanual| tells whether
% the |\includeonly| mechanism is used (false) or
% the selection of child files must be performed manually (true).
% The definitions initialise to false:
%    \begin{macrocode}
\newif\ifchilddoc
\newif\ifchilddocmanual
%    \end{macrocode}

% \macro{\childdocname}
% \macro{\childdocjob}
% The macro |\childdocname| stores the name of the main document
% to be compiled. The macro |\childdocjob| stores the name of
% the document on which the \LaTeX{} compiler was originally invoked.
% The content of |\jobname| cannot be compared
% to filenames specified in the source due to different catcodes.
% The following code rescans |\jobname|, stores the result
% in |\childdocname| and saves a copy in |\childdocjob|:
%    \begin{macrocode}
\edef\childdocname{\scantokens\expandafter{\jobname\noexpand}}
\let\childdocjob\childdocname
%    \end{macrocode}

% \macro{\childdocdisable}
% The macro |\childdocdisable| prevents the main file
% from being processed more than once.
% At this stage, the main document command |\childdocmain|
% is assumed to be called once again where it should do nothing.
% Any subsequent call to it should prevent
% a secondary processing of the main document
% It overwrites the forwarding commands
% |\childdocof| and |\childdocforward|
% with empty macros to prevent further inclusions of the main document:
%    \begin{macrocode}
\newcommand{\childdocdisable}
{
  \renewcommand{\childdocmain}[1]{\renewcommand{\childdocmain}[1]{\endinput}}
  \renewcommand{\childdocof}[1]{}
  \renewcommand{\childdocby}[2][]{}
  \renewcommand{\childdocforward}[2][]{}
  \renewcommand{\childdocdisable}{}
}
%    \end{macrocode}

% \macro{\childdocmain}
% The macro |\childdocmain| is to be called at the top of the main file
% with nothing or the main filename (without extension) as argument.
% First, it breaks loops.
% If the argument is not empty and does not match |\childdocname|
% (which is set by the first inclusion of |childdoc.def|),
% |\ifchilddoc| is set to true, |\includeonly| is applied to the child file
% and |\jobname| is set to the main file
% (for proper handling of |.aux| files):
%    \begin{macrocode}
\newcommand{\childdocmain}[1]
{
  \childdocdisable\childdocmain{}
  \if?#1?\else
    \begingroup
      \def\childdoctmp{#1}
      \ifx\childdoctmp\childdocname
        \def\childdoctmp{}
      \else
        \def\childdoctmp
        {
          \childdoctrue
          \includeonly{\childdocname}
          \def\childdocjob{#1}
          \def\jobname{#1}
        }
      \fi
      \expandafter
    \endgroup
    \childdoctmp
  \fi
}
%    \end{macrocode}

% \macro{\childdocof}
% The command |\childdocof| redirects
% compilation to the main file |#1|.
%    \begin{macrocode}
\newcommand{\childdocof}[1]
{
  \childdocdisable
  \childdoctrue
  \includeonly{\childdocname}
  \def\jobname{#1}
  \def\childdocjob{#1}
  \input{#1}
}
%    \end{macrocode}

% \macro{\childdocby}
% The command |\childdocby| ....
%    \begin{macrocode}
\newcommand{\childdocby}[2][]
{
  \childdocdisable
  \childdoctrue
  \childdocmanualtrue
  \if?#1?\else
    \def\jobname{#2}
  \fi
  \def\childdocjob{#2}
  \input{#2}
  \endinput
}
%    \end{macrocode}

% \macro{\childdocforward}
% The command |\childdocforward| redirects
% compilation to the main file or
% (if the optional argument is given) a child file.
% Parameters are set as if the main file
% or a child file starting with |\childdocof| was compiled.
% Then compilation is handed over to the main file:
%    \begin{macrocode}
\newcommand{\childdocforward}[2][]
{
  \begingroup
    \if?#1?
      \def\childdoctmp
      {
        \def\childdocname{#2}
        \def\childdocjob{#2}
        \def\jobname{#2}
        \input{#2}
        \endinput
      }
    \else
      \def\childdoctmp
      {
        \childdocdisable
        \def\childdocname{#2}
        \childdoctrue
        \includeonly{#2}
        \def\childdocjob{#1}
        \def\jobname{#1}
        \input{#1}
        \endinput
      }
    \fi
    \expandafter
  \endgroup
  \childdoctmp
}
%    \end{macrocode}

% \macro{\childdocforwardprefix}
% The command |\childdocforwardprefix| redirects
% compilation to the main or a child file by means of a pattern.
% The prefix |#1| in the current filename is replaced by |#2|
% and the suffix of the current filename is kept
% (it is assumed that the filename does not contain the substring `|~~~|'
% which is used as a delimiter).
% Compilation is handed over to the new file by |\childdocforward|:
%    \begin{macrocode}
\newcommand{\childdocforwardprefix}[3][]
{
  \begingroup
    \def\childdocextract #2##1~~~{\def\childdoctmp{\childdocforward[#1]{#3##1}}}
    \expandafter\childdocextract\childdocname~~~
    \expandafter
  \endgroup
  \childdoctmp
}
%    \end{macrocode}

% \macro{\childdoc}
% The deprecated macro |\childdoc| is a legacy version of |\childdocmain|:
%    \begin{macrocode}
\newcommand{\childdoc}{\childdocmain}
%    \end{macrocode}

% \macro{\childdocredirect}
% The deprecated macro |\childdocredirect| is a legacy version
% of |\childdocforward| and |\childdocforwardprefix|:
%    \begin{macrocode}
\newcommand{\childdocredirect}[2][]
{
  \begingroup
    \if?#1?
      \def\childdoctmp{\childdocforward{#2}}
    \else
      \def\childdoctmp{\childdocforwardprefix{#1}{#2}}
    \fi
    \expandafter
  \endgroup
  \childdoctmp
}
%    \end{macrocode}

%\iffalse
%</package>
%\fi
%
\endinput

\childdocby{cdocsamp}
%    \end{macrocode}

%\iffalse
%</samplepart3|samplepart4>
%\fi
%
%\iffalse
%<*samplepart3>
%\fi
% Some text for part 3:
%    \begin{macrocode}
some text in part three
%    \end{macrocode}

%\iffalse
%</samplepart3>
%\fi
% Some text for part 4:
%\iffalse
%<*samplepart4>
%\fi
%    \begin{macrocode}
more text in part four
%    \end{macrocode}

%\iffalse
%</samplepart4>
%\fi
%
% %%%%%%%%%%%%%%%%%%%%%%%%%%%%%%%%%%%%%%
% \paragraph{Forwarding for a Complete Draft.}
%
% The following forwarding file |cdocsdrf.tex|
% compiles the main document in draft mode:
%\iffalse
%<*sampledraft>
%\fi
%    \begin{macrocode}
\def\version{draft}
% \iffalse
%
% childdoc.dtx Copyright (C) 2017-2018 Niklas Beisert
%
% This work may be distributed and/or modified under the
% conditions of the LaTeX Project Public License, either version 1.3
% of this license or (at your option) any later version.
% The latest version of this license is in
%   http://www.latex-project.org/lppl.txt
% and version 1.3 or later is part of all distributions of LaTeX
% version 2005/12/01 or later.
%
% This work has the LPPL maintenance status `maintained'.
%
% The Current Maintainer of this work is Niklas Beisert.
%
% This work consists of the files childdoc.dtx and childdoc.ins
% and the derived files childdoc.def and cdocsamp.tex with
% cdocsch1.tex, cdocsch2.tex, cdocsdrf.tex, cdocsfn1.tex, cdocsfn2.tex.
%
%<package>\ifdefined\childdocmain\endinput\fi
%<package>\ProvidesFile{childdoc.def}[2018/12/30 v2.0 child document driver]
%<samplemain>\ProvidesFile{cdocsamp.tex}[2018/12/30 v2.0 sample for childdoc]
%<*driver>
%\ProvidesFile{childdoc.drv}[2018/12/30 v2.0 childdoc reference manual file]
\PassOptionsToClass{10pt,a4paper}{article}
\documentclass{ltxdoc}

\usepackage[margin=35mm]{geometry}
\usepackage{hyperref}
\usepackage{hyperxmp}
\usepackage[usenames]{color}

\hypersetup{colorlinks=true}
\hypersetup{pdfstartview=FitH}
\hypersetup{pdfpagemode=UseNone}
\hypersetup{pdfsource={}}
\hypersetup{pdflang={en-UK}}
\hypersetup{pdfcopyright={Copyright 2017-2018 Niklas Beisert.
  This work may be distributed and/or modified under the
  conditions of the LaTeX Project Public License, either version 1.3
  of this license or (at your option) any later version.}}
\hypersetup{pdflicenseurl={http://www.latex-project.org/lppl.txt}}
\hypersetup{pdfcontactaddress={ETH Zurich, ITP, HIT K,
  Wolfgang-Pauli-Strasse 27}}
\hypersetup{pdfcontactpostcode={8093}}
\hypersetup{pdfcontactcity={Zurich}}
\hypersetup{pdfcontactcountry={Switzerland}}
\hypersetup{pdfcontactemail={nbeisert@itp.phys.ethz.ch}}
\hypersetup{pdfcontacturl={http://people.phys.ethz.ch/\xmptilde nbeisert/}}

\newcommand{\secref}[1]{\hyperref[#1]{section \ref*{#1}}}

\parskip1ex
\parindent0pt
\let\olditemize\itemize
\def\itemize{\olditemize\parskip0pt}

\begin{document}

\title{The \textsf{childdoc} Package}
\hypersetup{pdftitle={The childdoc Package}}
\author{Niklas Beisert\\[2ex]
  Institut f\"ur Theoretische Physik\\
  Eidgen\"ossische Technische Hochschule Z\"urich\\
  Wolfgang-Pauli-Strasse 27, 8093 Z\"urich, Switzerland\\[1ex]
  \href{mailto:nbeisert@itp.phys.ethz.ch}
  {\texttt{nbeisert@itp.phys.ethz.ch}}}
\hypersetup{pdfauthor={Niklas Beisert}}
\hypersetup{pdfsubject={Manual for the LaTeX2e Package childdoc}}
\date{30 December 2018, \textsf{v2.0}}
\maketitle

\begin{abstract}\noindent
\textsf{childdoc} is a \LaTeXe{} package
that enables the direct compilation
of document sections included by |\include|
to individual files.
\end{abstract}

\begingroup
\parskip0ex
\tableofcontents
\endgroup

%%%%%%%%%%%%%%%%%%%%%%%%%%%%%%%%%%%%%%%%%%%%%%%%%%%%%%%%%%%%%%%%%%%%%%%%%%%%%%%%
%%%%%%%%%%%%%%%%%%%%%%%%%%%%%%%%%%%%%%%%%%%%%%%%%%%%%%%%%%%%%%%%%%%%%%%%%%%%%%%%
\section{Introduction}

\LaTeX{} provides a mechanism to structure a large document (such as a book)
into a main file and several child files (containing the chapters)
using the |\include| command.
This mechanism is beneficial for documents
which span hundreds of pages in order to
make the source file(s) more manageable.
Moreover, compilation can be restricted to
selected child files by means of the |\includeonly| command.
The latter feature can be used to reduce the compilation time while editing
(this was significantly more useful in the earlier days of \LaTeX{})
or to generate a smaller document which is easier to navigate.
Another application of |\includeonly| is to generate
documents consisting of selected parts of the complete document.

However, there are a few drawbacks of the plain |\include| mechanism:
\begin{itemize}
\item
The child files cannot be compiled on their own,
they can only be compiled via the main file.
A naive editing environment
(such as a text editor with an option
to have the current file processed by \LaTeX)
may require one to switch to the main file before compiling;
attempting to compile the child file produces errors.
\item
The main file must be modified (each time)
to adjust the |\includeonly| command
to the present needs. This easily leaves the main file in a messy state.
\item
The generated document will always carry the filename
of the main document. This is inconvenient if
several child files are to be compiled and
to be kept for distribution.
\end{itemize}

The present package provides a simple interface
to make child files individually compilable by \LaTeX{}.
Compiling a child file then has the same effect as compiling
the main file with an |\includeonly| command
to select the appropriate child.
Moreover the generated document will carry the name of the child
rather than the main file.
This resolves all three above issues.

This feature is meant to make the editing of books,
thesis documents and lecture notes somewhat more convenient.
However, the package can also be used efficiently for
composing a series of documents (such as exercise sheets)
which are typically distributed individually.
It then assists the author in generating the individual documents
(potentially in different versions)
as well as a document containing the collected series.
Another application is in developing style files
or other kinds of included material
where compilation of the style file could redirect
to a sample or test file.

%%%%%%%%%%%%%%%%%%%%%%%%%%%%%%%%%%%%%%%%%%%%%%%%%%%%%%%%%%%%%%%%%%%%%%%%%%%%%%%%
%%%%%%%%%%%%%%%%%%%%%%%%%%%%%%%%%%%%%%%%%%%%%%%%%%%%%%%%%%%%%%%%%%%%%%%%%%%%%%%%
\section{Usage}

First of all, the package \textsf{childdoc} is \emph{not} a standard
\LaTeXe{} |.sty| style file! Therefore it needs to be invoked in
a non-standard way.

%%%%%%%%%%%%%%%%%%%%%%%%%%%%%%%%%%%%%%%%%%%%%%%%%%%%%%%%%%%%%%%%%%%%%%%%%%%%%%%%
\subsection{Included Files}
\label{sec:include}

%%%%%%%%%%%%%%%%%%%%%%%%%%%%%%%%%%%%%%%%
\DescribeMacro{\childdocmain}
To use the package, add the commands
\begin{center}
\begin{tabular}{l}
|\input{childdoc.def}|\\
|\childdocmain{}|\\
\end{tabular}
\end{center}
at the very top of the main \LaTeX{} file,
in particular \emph{before} the |\documentclass| statement!
The argument of |\childdocmain| should be left empty
(but it must be present).

%%%%%%%%%%%%%%%%%%%%%%%%%%%%%%%%%%%%%%%%
\DescribeMacro{\childdocof}
Furthermore, add the commands
\begin{center}
\begin{tabular}{l}
|\input{childdoc.def}|\\
|\childdocof{|\textit{main}|}|\\
\end{tabular}
\end{center}
at the top of every child file \textit{child}
which is included by |\include{|\textit{child}|}|
from within the main file
(or at least for those files to be compiled individually).
The argument \textit{main} must be the filename of the main file.

There are a couple of
considerations in setting up the main and child documents:

%%%%%%%%%%%%%%%%%%%%%%%%%%%%%%%%%%%%%%%%
\paragraph{Restrictions.}

Please note the following restrictions:
\begin{itemize}
\item
|\childdocmain| must be called with one argument \textit{main}
to ensure compatibility with earlier version of the package.
It must either be empty (|\childdocmain{}|)
or precisely match the filename of the main file in which it is specified.
See \secref{sec:detection} for further information.
\item
The filename \textit{main} must be specified without the |.tex| extension.
\item
The filename \textit{main} is case sensitive
(even in case-insensitive file systems)
due to internal string comparison.
\item
The argument \textit{main} should be fully expanded, it cannot be a macro.
\item
Subdirectories and special characters should be avoided in filenames.
\item
The command |\childdocmain{|\textit{main}|}| must be followed by a whitespace.
It should not be followed immediately by another command
or by a comment mark `|%|'.
This is because the \TeX{} parser reads the token immediately following
the argument of |\childdocmain| and puts it
at the beginning of every child section;
however, a white\-space is ignored.
\end{itemize}

%%%%%%%%%%%%%%%%%%%%%%%%%%%%%%%%%%%%%%%%
\paragraph{Content of Main File.}

It is advisable to place all content in the child files included by |\include|.
Any output contained in the main file will appear in all child documents
unless suppressed manually;
it cannot be suppressed automatically by the |\includeonly| directive
and thus should normally be avoided.
A method to include some content in the main file
by means of conditional processing is described in \secref{sec:conditional}.

%%%%%%%%%%%%%%%%%%%%%%%%%%%%%%%%%%%%%%%%
\paragraph{Page Numbering.}

When only a part of the document is compiled,
the appropriate numbering of pages
(as well as other status parameters)
is determined from the |.aux| files.
The latter contain information from previous passes.
However this information needs to propagate through
all intermediate child documents.
Therefore the page numbering in child documents may well
be inconsistent until the complete document is compiled at least once.

A useful (if unconventional) way to always ensure a consistent
page numbering is to restart the numbering in each child document
and denote the pages by `\textit{child}|.|\textit{page}'
where \textit{child} represents the chapter/section number of the child file.
This can be achieved by the command
|\numberwithin{page}{|\textit{child}|}|
of the \textsf{amsmath} package
where \textit{child} can be |chapter| or |section|
depending on the chosen structuring.
Alternatively, one can modify the macro |\thepage| appropriately
and reset the counter |page| at the start of each child file.

%%%%%%%%%%%%%%%%%%%%%%%%%%%%%%%%%%%%%%%%%%%%%%%%%%%%%%%%%%%%%%%%%%%%%%%%%%%%%%%%
\subsection{Conditional Processing}
\label{sec:conditional}

The package provides a mechanism to compile different versions
of a document. To customise the versions further some conditional processing
can come in handy to distinguish which version is being compiled.
The package provides two macros to describe the compilation context:

%%%%%%%%%%%%%%%%%%%%%%%%%%%%%%%%%%%%%%%%
\DescribeMacro{\ifchilddoc}
The conditional |\ifchilddoc| distinguishes between the compilation of
child documents and the main document:
%
\begin{center}
|\ifchilddoc |\textit{child-code}| |[|\||else |\textit{main-code}]| \||fi|
\end{center}

%%%%%%%%%%%%%%%%%%%%%%%%%%%%%%%%%%%%%%%%
\DescribeMacro{\childdocname}
\DescribeMacro{\childdocjob}
The macro |\childdocname| contains the filename (without extension)
of the main or child file being processed.
Note that |\childdocjob| will always contain the name of the main file.

%%%%%%%%%%%%%%%%%%%%%%%%%%%%%%%%%%%%%%%%
\paragraph{Title Page.}

Conditional processing can be used to include a title or banner page
in the main document when proper precautions are taken.
Importantly, the code in the main file should ensure that the page counter
(as well as other status parameters which are stored in the |.aux| files)
takes the same value after the conditional processing.
Otherwise the page numbers may take divergent values
depending on which part is compiled.

For example, a title page could be declared by:
%
\begin{center}
\begin{tabular}{l}
|\ifchilddoc\||else|\\
|\addtocounter{page}{-1}|\\
\textit{code for title page}\\
|\newpage|\\
|\||fi|
\end{tabular}
\end{center}
%
A banner page for the child documents can be generated by:
%
\begin{center}
\begin{tabular}{l}
|\ifchilddoc|\\
|\addtocounter{page}{-1}|\\
\textit{code for banner page}\\
|\newpage|\\
|\||fi|
\end{tabular}
\end{center}
%
Here one could write a message such as:
\begin{center}
|This is the part \childdocname{} of \childdocjob{}.|
\end{center}

%%%%%%%%%%%%%%%%%%%%%%%%%%%%%%%%%%%%%%%%%%%%%%%%%%%%%%%%%%%%%%%%%%%%%%%%%%%%%%%%
\subsection{Flags}
\label{sec:flags}

The package makes it easy to generate different versions
of the main or child documents.
To this end compilation flags can be defined
and assigned different default values.
They will be particularly useful in conjunction
with the forwarding mechanism described in \secref{sec:forward}.

For example, it may be useful to have a flag |\version|
which can be set to |draft| or |final|.
The document source will contain some conditional code
depending on the value of |\version|.
Suppose further, the flag should default to |final| for the main file
and to |draft| for child files
which is a natural assignment for editing the document.
This is achieved by placing the following code
in the preamble of the main document
(below the |\childdocmain| directive):
%
\begin{center}
\begin{tabular}{l}
|\ifchilddoc|\\
|\providecommand{\version}{draft}|\\
|\||else|\\
|\providecommand{\version}{final}|\\
|\||fi|
\end{tabular}
\end{center}
%
The definition by |\providecommand| makes sure
that previous definitions are not overwritten.
Further statements |\providecommand{\version}{...}|
can thus be added before the above code to override it.

For the main file, one might add a line
(between |\childdocmain| and the above block)
%
\begin{center}
|%\ifchilddoc\||else\providecommand{\version}{draft}\||fi|
\end{center}
%
which can be uncommented to produce a draft version.
Likewise one can add a line to the very top of a child file
(above the |\childdocof{|\textit{main}|}| directive)
%
\begin{center}
|%\providecommand{\version}{final}|
\end{center}
%
which can be uncommented to produce the final version of this child document.

%%%%%%%%%%%%%%%%%%%%%%%%%%%%%%%%%%%%%%%%%%%%%%%%%%%%%%%%%%%%%%%%%%%%%%%%%%%%%%%%
\subsection{Forwarding}
\label{sec:forward}

Different versions of the main or child documents
using compilation flags as described in \secref{sec:flags}
can be (permanently) stored in different files
for convenient compilation, viewing and distribution.
To this end, the package defines a command
to pass on compilation to a different file:

%%%%%%%%%%%%%%%%%%%%%%%%%%%%%%%%%%%%%%%%
\DescribeMacro{\childdocforward}
The command |\childdocforward| redirects processing to
another source file:
%
\begin{center}
\begin{tabular}{l}
|\input{childdoc.def}|\\
|\childdocforward[|\textit{main}|]{|\textit{dest}|}|\\
\end{tabular}
\end{center}
%
The argument \textit{dest} is the destination file
(without extension).
It should be the main file or one of the child files.
Note that further \textsf{childdoc} directives
such as |\childdocof| and |\childdocforward|
in the indicated file will be processed in this form.
The optional argument \textit{main}
passes on directly to the main file \textit{main}
while pretending to compile the child \textit{dest}.
This form behaves as if \textit{dest}
issues |\childdocof{|\textit{main}|}| right away,
and no further \textsf{childdoc} directives will be processed.

%%%%%%%%%%%%%%%%%%%%%%%%%%%%%%%%%%%%%%%%
\DescribeMacro{\...prefix}
In the alternative form |\childdocforwardprefix|,
%
\begin{center}
\begin{tabular}{l}
|\input{childdoc.def}|\\
|\childdocforwardprefix[|\textit{main}|]{|\textit{prefix}|}{|\textit{dest}|}|
\end{tabular}
\end{center}
%
the destination file is determined by a pattern
depending on the current file:
To make this work, the current file must be called
`{\textit{prefix}\hspace{0.2em}\textit{suffix}}'
with \textit{prefix} matching precisely the argument.
Processing is then passed on to the file
`{\textit{dest}\hspace{0.2em}\textit{suffix}}'.
Surely, the same effect is achieved by
directly specifying the
argument `{\textit{dest}\hspace{0.2em}\textit{suffix}}'
in the first form.
However, that requires to set up a different file
for each child. With the alternative form of the command
all these files can have exactly the same content
which simplifies setting them up and maintaining them.

For example, the following file |draft.tex|
with a compilation flag |\version| as described in \secref{sec:flags}
compiles the main document as a draft:
%
\begin{center}
\begin{tabular}{l}
|\def\version{draft}|\\
|\input{childdoc.def}|\\
|\childdocforward{|\textit{main}|}|
\end{tabular}
\end{center}
%
Likewise, the following files |final|\textit{nn}|.tex|
compile the final version of the child document
|child|\textit{nn}|.tex|:
%
\begin{center}
\begin{tabular}{l}
|\def\version{final}|\\
|\input{childdoc.def}|\\
|\childdocforwardprefix{final}{child}|
\end{tabular}
\end{center}
%

Note that when several versions of a main file and/or of each child file
are to be generated, it may be convenient to set up a |Makefile| or
shell script to automatise the process.

%%%%%%%%%%%%%%%%%%%%%%%%%%%%%%%%%%%%%%%%%%%%%%%%%%%%%%%%%%%%%%%%%%%%%%%%%%%%%%%%
\subsection{Command Line Processing}
\label{sec:commandline}

The effect of redirection files can also be achieved by invoking
the \LaTeX{} compiler with a more elaborate command line.
Most conveniently this should be done as part
of a shell script or a |Makefile|.

When using \textsf{childdoc} in the main file, the following
command lines effectively perform a redirection
(note that depending on the shell being used,
backslashes may have to be doubled: `|\|' $\to$ `|\\|'):
%
\begin{center}
|... -jobname "|\textit{target}|" |\\|"|[\textit{flags}]%
|\input{childdoc.def}\childdocforward[|\textit{main}|]{|\textit{dest}|}"|
\end{center}
%
Here \textit{target} is the name of the output file,
\textit{main} is the name of the main file
and \textit{dest} is the name of the main or child file to be processed
(all filenames without extensions).
The optional argument \textit{main} can be omitted
if \textit{main} matches \textit{dest}.
Optionally, compilation \textit{flags} can be defined via |\def| commands.
This command line makes the \TeX{} engine believe
it is compiling the file \textit{target}
whose content is specified as the latter parameter.
The provided code then forwards the processing to
\textit{main} or \textit{dest} as described in \secref{sec:forward}.

%%%%%%%%%%%%%%%%%%%%%%%%%%%%%%%%%%%%%%%%%%%%%%%%%%%%%%%%%%%%%%%%%%%%%%%%%%%%%%%%
\subsection{Include by Input}
\label{sec:input}

Including child documents by |\include| has some restrictions by design.
Most notably, the content of a child document always occupies
its own set of pages; pages cannot be shared between child documents.
Usually, this behaviour makes perfect sense
because each child document contain an essential part of the document.
However, in some situations it may be desirable to compose
a document from a collection of parts
without having mandatory page breaks between then.
For this case, the package
provides a mechanism to include parts
by |\input| which can also be processed individually.
However, by construction this mechanism
requires manual handling of the content to be output.

%%%%%%%%%%%%%%%%%%%%%%%%%%%%%%%%%%%%%%%%
\DescribeMacro{\ifchilddocmanual}
The main file should be prepared as usual, see \secref{sec:include}.
However, the document body must make a distinction
between processing of an individual part and of the main document, e.g.:
%
\begin{center}
\begin{tabular}{l}
|\ifchilddocmanual|\\
|\input{\childdocname}|\\
|\||else|\\
\textit{document body with }|\input{|\textit{part}|}|\\
|\||fi|
\end{tabular}
\end{center}
%
The conditional |\ifchilddocmanual| is true whenever
a part to be included by |\input| is being compiled,
and the name of the part is stored in |\childdocname|.

%%%%%%%%%%%%%%%%%%%%%%%%%%%%%%%%%%%%%%%%
\DescribeMacro{\childdocby}
Each part to be included by |\input| should start with:
%
\begin{center}
\begin{tabular}{l}
|\input{childdoc.def}|\\
|\childdocby{|\textit{main}|}|\\
\end{tabular}
\end{center}
%
The directive |\childdocby| is similar to |\childdocof|
described in \secref{sec:include},
but the subsequent selection of content must be done manually.
To that end, both |\ifchilddoc| and |\ifchilddocmanual|
will be true upon processing of a part,
and the name of the part is stored in |\childdocname|.
Note that |\jobname| will be set to the filename of the current part
so that each part receives an individual |.aux| file
that does not interfere with the |.aux| file(s) of the main document.
This behaviour can be altered by the alternative form
|\childdocby[*]{|\textit{main}|}| (with a non-empty optional argument)
which uses the |.aux| file of the main document
by setting |\jobname| to \textit{main}.

%%%%%%%%%%%%%%%%%%%%%%%%%%%%%%%%%%%%%%%%%%%%%%%%%%%%%%%%%%%%%%%%%%%%%%%%%%%%%%%%
\subsection{Driver Development}
\label{sec:driver}

The \textsf{childdoc} mechanism can also be use for the development
of definition files such as \LaTeX{} styles or classes.
This case differs from the above setup with multiple parts
included by |\include| in that no |\includeonly| should be invoked.
This can be achieved by starting the include file
(before |\ProvidesPackage|) with:
%
\begin{center}
\begin{tabular}{l}
|\input{childdoc.def}|\\
|\childdocforward{|\textit{main}|}|\\
\end{tabular}
\end{center}
%
or alternatively with:
%
\begin{center}
\begin{tabular}{l}
|\input{childdoc.def}|\\
|\childdocby{|\textit{main}|}|\\
\end{tabular}
\end{center}
%
Both forms have slightly different effects as described above.
The main file is prepared as usual, see \secref{sec:include}.

%%%%%%%%%%%%%%%%%%%%%%%%%%%%%%%%%%%%%%%%%%%%%%%%%%%%%%%%%%%%%%%%%%%%%%%%%%%%%%%%
\subsection{Legacy Detection}
\label{sec:detection}

The directive |\childdocmain| in the main file can detect
whether the complete document or merely a child is to be compiled
even without using the directive |\childdocof|.
This method is deprecated because it is less robust
and there is no compelling reason to use it;
it is merely provided for backward compatibility
and it may be removed in future versions.

If the detection mechanism is to be used,
it is mandatory to correctly specify
the filename of the main file as the argument of |\childdocmain|:
%
\begin{center}
\begin{tabular}{l}
|\input{childdoc.def}|\\
|\childdocmain{|\textit{main}|}|\\
\end{tabular}
\end{center}
%
If |\jobname| does not match the argument \textit{main} of |\childdocmain|,
it is assumed that |\jobname| points to the child file to be compiled.
When using |\childdocmain| with the main file specified as argument,
it suffices to start a child file
with just |\input{|\textit{main}|}|
without loading of the package and using |\childdocof|.
If instead all processing is done
with the appropriate \textsf{childdoc} directives,
the argument of \textit{main} of |\childdocmain| can be empty.

An alternative version of the command line processing described
in \secref{sec:commandline} using the detection mechanism reads:
%
\begin{center}
|... -jobname "|\textit{target}|" "|[\textit{flags}]%
[|\def\jobname{|\textit{dest}|}|]|\input{|\textit{main}|}"|
\end{center}

%%%%%%%%%%%%%%%%%%%%%%%%%%%%%%%%%%%%%%%%%%%%%%%%%%%%%%%%%%%%%%%%%%%%%%%%%%%%%%%%
\subsection{Manual Code}
\label{sec:manual}

In case one cannot be certain whether the definitions file |childdoc.def|
is installed on the target \TeX{} distribution
and one prefers not to ship it,
it is conceivable to paste a few relevant commands into the sources.

To that end, drop all statements |\input{childdoc.def}|
and perform the replacements as outlined below.
Instead of |\childdocmain{|\textit{main}|}| add the following code
to the top of the main file:
%
\begin{center}
\begin{tabular}{l}
|\||ifdefined\childdocname\endinput\||fi\newif\ifchilddoc|\\
|\edef\childdocname{\scantokens\expandafter{\jobname\noexpand}}|\\
|\def\childdocmain{|\textit{main}|}\||ifx\childdocmain\childdocname\||else|\\
|\childdoctrue\includeonly{\childdocname}\let\jobname\childdocmain\||fi|\\
\end{tabular}
\end{center}
%
Instead of |\childdocof{|\textit{main}|}| just include the main file
at the top of each child file:
%
\begin{center}
|\input{|\textit{main}|}|
\end{center}
%
A simple redirection |\childdocforward{|\textit{dest}|}| is achieved by:
%
\begin{center}
|\def\jobname{|\textit{dest}|}\input{\jobname}|
\end{center}
%
The redirection with prefix
|\childdocforwardprefix[|\textit{prefix}|]{|\textit{dest}|}|
is accomplished by:
%
\begin{center}
\begin{tabular}{l}
|{\edef\jobname{\scantokens\expandafter{\jobname\noexpand}}|\\
|\def\redirectjob |\textit{prefix}|#1~~~{\gdef\jobname{|\textit{dest}|#1}}|\\
|\expandafter\redirectjob\jobname~~~}\input{\jobname}|
\end{tabular}
\end{center}

In an alternative approach,
child documents can be compiled by a specific command line
without additional code or specific definitions:
%
\begin{center}
|... -jobname "|\textit{target}|" "|[\textit{flags}]%
|\includeonly{|\textit{dest}|}\input{|\textit{main}|}"|
\end{center}
%

%%%%%%%%%%%%%%%%%%%%%%%%%%%%%%%%%%%%%%%%%%%%%%%%%%%%%%%%%%%%%%%%%%%%%%%%%%%%%%%%
%%%%%%%%%%%%%%%%%%%%%%%%%%%%%%%%%%%%%%%%%%%%%%%%%%%%%%%%%%%%%%%%%%%%%%%%%%%%%%%%
\section{Information}

%%%%%%%%%%%%%%%%%%%%%%%%%%%%%%%%%%%%%%%%%%%%%%%%%%%%%%%%%%%%%%%%%%%%%%%%%%%%%%%%
\subsection{Copyright}

Copyright \copyright{} 2017--2018 Niklas Beisert

This work may be distributed and/or modified under the
conditions of the \LaTeX{} Project Public License, either version 1.3
of this license or (at your option) any later version.
The latest version of this license is in
  \url{http://www.latex-project.org/lppl.txt}
and version 1.3 or later is part of all distributions of \LaTeX{}
version 2005/12/01 or later.

This work has the LPPL maintenance status `maintained'.

The Current Maintainer of this work is Niklas Beisert.

This work consists of the files |README.txt|, |childdoc.ins| and |childdoc.dtx|
as well as the derived files |childdoc.def|, |cdocsamp.tex|
with |cdocsch1.tex|, |cdocsch2.tex|, |cdocspt3.tex|, |cdocspt4.tex|,
|cdocsdrf.tex|, |cdocsfn1.tex|, |cdocsfn2.tex|
as well as |childdoc.pdf|.

%%%%%%%%%%%%%%%%%%%%%%%%%%%%%%%%%%%%%%%%%%%%%%%%%%%%%%%%%%%%%%%%%%%%%%%%%%%%%%%%
\subsection{Files and Installation}

The package consists of the files:
%
\begin{center}
\begin{tabular}{ll}
    |README.txt|   & readme file \\
    |childdoc.ins| & installation file \\
    |childdoc.dtx| & source file \\
    |childdoc.def| & definition file \\
    |cdocsamp.tex| & sample main file \\
    |cdocsch1.tex| & sample include file \\
    |cdocsch2.tex| & sample include file \\
    |cdocspt3.tex| & sample part file \\
    |cdocspt4.tex| & sample part file \\
    |cdocsdrf.tex| & sample redirection file \\
    |cdocsfn1.tex| & sample redirection file \\
    |cdocsfn2.tex| & sample redirection file \\
    |childdoc.pdf| & manual
\end{tabular}
\end{center}
%
The distribution consists of the files
|README.txt|, |childdoc.ins| and |childdoc.dtx|.
%
\begin{itemize}
\item
Run (pdf)\LaTeX{} on |childdoc.dtx|
to compile the manual |childdoc.pdf| (this file).
\item
Run \LaTeX{} on |childdoc.ins| to create the definitions file |childdoc.def|
and the sample |cdocsamp.tex| with include files
|cdocsch1.tex|, |cdocsch2.tex|, |cdocspt3.tex|, |cdocspt4.tex|,
|cdocsdrf.tex|, |cdocsfn1.tex|, |cdocsfn2.tex|.
Then copy the file |childdoc.def| to an appropriate directory of your \LaTeX{}
distribution, e.g.\ \textit{texmf-root}|/tex/latex/childdoc|.
\end{itemize}

%%%%%%%%%%%%%%%%%%%%%%%%%%%%%%%%%%%%%%%%%%%%%%%%%%%%%%%%%%%%%%%%%%%%%%%%%%%%%%%%
\subsection{Related CTAN Packages}

There are several other packages which offer a similar functionality:
%
\begin{itemize}
\item
The packages
\href{http://ctan.org/pkg/docmute}{\textsf{docmute}},
\href{http://ctan.org/pkg/includex}{\textsf{includex}} and
\href{http://ctan.org/pkg/standalone}{\textsf{standalone}}
provide commands to include only the document body of
a child file thus allowing both files to be compiled individually.
\item
The packages \href{http://ctan.org/pkg/subdocs}{\textsf{subdocs}}
and \href{http://ctan.org/pkg/subfiles}{\textsf{subfiles}}
provide structures in which the main and child documents can be
encapsulated and allowing them to be compiled individually.
The inclusion mechanism is different from the conventional |\include|.
\item
The package \href{http://ctan.org/pkg/combine}{\textsf{combine}}
is an elaborate solution to combine several documents into one.
\end{itemize}
%
See also the CTAN topic \href{http://ctan.org/topic/subdocs}{\textsf{subdocs}}
for further related packages.
The present package differs from the above solutions in that
a document structure constructed with the conventional |\include| mechanism
just needs two extra commands at the top of every file
such that all constituent files can be compiled individually.

%%%%%%%%%%%%%%%%%%%%%%%%%%%%%%%%%%%%%%%%%%%%%%%%%%%%%%%%%%%%%%%%%%%%%%%%%%%%%%%%
%\subsection{Feature Suggestions}
%
%The following is a list of features which may be useful for future
%versions of this package:
%%
%\begin{itemize}
%\item
%\ldots
%\end{itemize}

%%%%%%%%%%%%%%%%%%%%%%%%%%%%%%%%%%%%%%%%%%%%%%%%%%%%%%%%%%%%%%%%%%%%%%%%%%%%%%%%
\subsection{Revision History}

%%%%%%%%%%%%%%%%%%%%%%%%%%%%%%%%%%%%%%%%
\paragraph{v2.0:} 2018/12/30

\begin{itemize}
\item
immediate forward processing
\item
added |\childdocby| mechanism
\item
manual restructured
\end{itemize}

%%%%%%%%%%%%%%%%%%%%%%%%%%%%%%%%%%%%%%%%
\paragraph{v1.6:} 2018/01/17

\begin{itemize}
\item
application for development of include files
\item
corrections to manual
\end{itemize}

%%%%%%%%%%%%%%%%%%%%%%%%%%%%%%%%%%%%%%%%
\paragraph{v1.5:} 2017/05/21

\begin{itemize}
\item
more complete structuring introduced
\item
|\childdocof| introduced
\item
|\childdoc| renamed to |\childdocmain|
\item
|\childredirect| renamed to |\childdocforward| and |\childdocforwardprefix|
and functionality expanded
\end{itemize}

%%%%%%%%%%%%%%%%%%%%%%%%%%%%%%%%%%%%%%%%
\paragraph{v1.0:} 2017/04/27

\begin{itemize}
\item
manual and install package
\item
first version published on CTAN
\end{itemize}

%%%%%%%%%%%%%%%%%%%%%%%%%%%%%%%%%%%%%%%%
\paragraph{v0.6:} 2017/04/26

\begin{itemize}
\item
redirection mechanism added
\end{itemize}

%%%%%%%%%%%%%%%%%%%%%%%%%%%%%%%%%%%%%%%%
\paragraph{v0.5:} 2017/04/26

\begin{itemize}
\item
functionality in definition file
\end{itemize}


%%%%%%%%%%%%%%%%%%%%%%%%%%%%%%%%%%%%%%%%%%%%%%%%%%%%%%%%%%%%%%%%%%%%%%%%%%%%%%%%
%%%%%%%%%%%%%%%%%%%%%%%%%%%%%%%%%%%%%%%%%%%%%%%%%%%%%%%%%%%%%%%%%%%%%%%%%%%%%%%%
%%%%%%%%%%%%%%%%%%%%%%%%%%%%%%%%%%%%%%%%%%%%%%%%%%%%%%%%%%%%%%%%%%%%%%%%%%%%%%%%
\appendix

\settowidth\MacroIndent{\rmfamily\scriptsize 000\ }

 \DocInput{childdoc.dtx}

\end{document}
%</driver>
% \fi
%
% %%%%%%%%%%%%%%%%%%%%%%%%%%%%%%%%%%%%%%%%%%%%%%%%%%%%%%%%%%%%%%%%%%%%%%%%%%%%%%
% %%%%%%%%%%%%%%%%%%%%%%%%%%%%%%%%%%%%%%%%%%%%%%%%%%%%%%%%%%%%%%%%%%%%%%%%%%%%%%
% \section{Sample}
%\iffalse
%<*samplemain>
%\fi
%
% The following presents a sample document
% with two chapters, two parts, a title page,
% a compile flag as well as three forwarding files to set the flag.
% It consists of eight |.tex| files:
% \begin{center}
% \begin{tabular}{ll}
% |cdocsamp.tex|&main file\\
% |cdocsch1.tex|&include file for chapter 1\\
% |cdocsch2.tex|&include file for chapter 2\\
% |cdocspt3.tex|&include file for part 3\\
% |cdocspt4.tex|&include file for part 4\\
% |cdocsdrf.tex|&forwarding file for main file in draft mode\\
% |cdocsfi1.tex|&forwarding file for final version of chapter 1\\
% |cdocsfi2.tex|&forwarding file for final version of chapter 2\\
% \end{tabular}
% \end{center}
% Each of the eight files can be compiled directly by the \LaTeX{} compiler.
%
% %%%%%%%%%%%%%%%%%%%%%%%%%%%%%%%%%%%%%%
% \paragraph{Main File.}
%
% The main file is called |cdocsamp.tex|.
%
% Load the \textsf{childdoc} definitions and
% declare the filename for the main document:
%    \begin{macrocode}
\input{childdoc.def}
\childdocmain{}
%    \end{macrocode}

% Optional override for |\version| flag:
%    \begin{macrocode}
%%\ifchilddoc\else\providecommand{\version}{draft}\fi
%    \end{macrocode}

% Define the default values for the |\version| flag
% (|final| for the main file and |draft| for childs):
%    \begin{macrocode}
\ifchilddoc
\providecommand{\version}{draft}
\else
\providecommand{\version}{final}
\fi
%    \end{macrocode}

% Load the standard document class:
%    \begin{macrocode}
\documentclass[12pt]{article}
%    \end{macrocode}

% Start the document body:
%    \begin{macrocode}
\begin{document}
%    \end{macrocode}

% Declare a title page.
% Print title, part of document being processed and version flag:
%    \begin{macrocode}
\addtocounter{page}{-1}
\begin{center}
{\LARGE\bfseries{}childdoc example\par}
\vspace{1cm}
\ifchilddoc
\ifchilddocmanual part\else chapter\fi:
`\childdocname' of `\childdocjob'\par
\else
main document: `\childdocjob'\par
\fi
version: \version\par
\end{center}
\newpage
%    \end{macrocode}

% Manually include selected file,
% otherwise process as usual:
%    \begin{macrocode}
\ifchilddocmanual
\section*{part `\childdocname'}
\input{\childdocname}
\else
%    \end{macrocode}

% Include the two chapters:
%    \begin{macrocode}
\include{cdocsch1}
\include{cdocsch2}
%    \end{macrocode}

% Include the two parts unless only chapters should be displayed:
%    \begin{macrocode}
\ifchilddoc\else
\section{part three}
\input{cdocspt3}
\section{part four}
\input{cdocspt4}
\fi
%    \end{macrocode}

% Process as usual until here:
%    \begin{macrocode}
\fi
%    \end{macrocode}

% End of document body:
%    \begin{macrocode}
\end{document}
%    \end{macrocode}
%\iffalse
%</samplemain>
%\fi
%
% %%%%%%%%%%%%%%%%%%%%%%%%%%%%%%%%%%%%%%
% \paragraph{Chapter Include Files.}
%
% The include files are called |cdocsch1.tex| and |cdocsch2.tex|.
%
%\iffalse
%<*samplechap1|samplechap2>
%\fi

% Optional override for |\version| flag:
%    \begin{macrocode}
%%\providecommand{\version}{final}
%    \end{macrocode}

% Include the main document:
%    \begin{macrocode}
\input{childdoc.def}
\childdocof{cdocsamp}
%    \end{macrocode}

%\iffalse
%</samplechap1|samplechap2>
%\fi
%
%\iffalse
%<*samplechap1>
%\fi
% Some text for chapter 1:
%    \begin{macrocode}
\section{one}
some text in chapter one
%    \end{macrocode}

%\iffalse
%</samplechap1>
%\fi
% Some text for chapter 2:
%\iffalse
%<*samplechap2>
%\fi
%    \begin{macrocode}
\section{two}
more text in chapter two
%    \end{macrocode}

%\iffalse
%</samplechap2>
%\fi
%
% %%%%%%%%%%%%%%%%%%%%%%%%%%%%%%%%%%%%%%
% \paragraph{Part Include Files.}
%
% The include files are called |cdocspt3.tex| and |cdocspt4.tex|.
%
%\iffalse
%<*samplepart3|samplepart4>
%\fi

% Optional override for |\version| flag:
%    \begin{macrocode}
%%\providecommand{\version}{final}
%    \end{macrocode}

% Include the main document:
%    \begin{macrocode}
\input{childdoc.def}
\childdocby{cdocsamp}
%    \end{macrocode}

%\iffalse
%</samplepart3|samplepart4>
%\fi
%
%\iffalse
%<*samplepart3>
%\fi
% Some text for part 3:
%    \begin{macrocode}
some text in part three
%    \end{macrocode}

%\iffalse
%</samplepart3>
%\fi
% Some text for part 4:
%\iffalse
%<*samplepart4>
%\fi
%    \begin{macrocode}
more text in part four
%    \end{macrocode}

%\iffalse
%</samplepart4>
%\fi
%
% %%%%%%%%%%%%%%%%%%%%%%%%%%%%%%%%%%%%%%
% \paragraph{Forwarding for a Complete Draft.}
%
% The following forwarding file |cdocsdrf.tex|
% compiles the main document in draft mode:
%\iffalse
%<*sampledraft>
%\fi
%    \begin{macrocode}
\def\version{draft}
\input{childdoc.def}
\childdocforward{cdocsamp}
%    \end{macrocode}

%\iffalse
%</sampledraft>
%\fi
%
% %%%%%%%%%%%%%%%%%%%%%%%%%%%%%%%%%%%%%%
% \paragraph{Forwarding for Final Version of the Chapters.}
%
% The following forwarding files |cdocsfn1.tex| and |cdocsfn2.tex|
% (with identical content)
% compile the final versions of the child documents
% |cdocsch1.tex| and |cdocsch2.tex|, respectively:
%\iffalse
%<*samplefinal>
%\fi
%    \begin{macrocode}
\def\version{final}
\input{childdoc.def}
\childdocforwardprefix[cdocsamp]{cdocsfn}{cdocsch}
%    \end{macrocode}

%\iffalse
%</samplefinal>
%\fi
%
% %%%%%%%%%%%%%%%%%%%%%%%%%%%%%%%%%%%%%%
% \paragraph{Command Line Processing.}
%
% The following three command lines generate the output files
% |cdocscld|, |cdocscl1| and |cdocscl2|
% which should be identical to
% |cdocsdrf|, |cdocsch1| and |cdocsfn2|, respectively:
% \begin{center}
% \begin{tabular}{l}
% |latex -jobname cdocscld \|\\
% |  "\def\version{draft}\input{childdoc.def}\childdocforward{cdocsamp}"|\\
% |latex -jobname cdocscl1 \|\\
% |  "\input{childdoc.def}\childdocforward[cdocsamp]{cdocsch1}"|\\
% |latex -jobname cdocscl2 \|\\
% |  "\def\version{final}\input{childdoc.def}\childdocforward{cdocsch2}"|
% \end{tabular}
% \end{center}
% Note that the trailing backslash on each first line
% merely continues the input to the second line
% (for convenient cut ant paste).
% Furthermore, the command |latex| can be replaced by any
% of its alternative versions such as |pdflatex|.
%
% %%%%%%%%%%%%%%%%%%%%%%%%%%%%%%%%%%%%%%%%%%%%%%%%%%%%%%%%%%%%%%%%%%%%%%%%%%%%%%
% %%%%%%%%%%%%%%%%%%%%%%%%%%%%%%%%%%%%%%%%%%%%%%%%%%%%%%%%%%%%%%%%%%%%%%%%%%%%%%
% \section{Implementation}
%\iffalse
%<*package>
%\fi
%
% This section describes the definitions file |childdoc.def|.

% The definitions cannot be loaded using |\usepackage| or |\RequirePackage|
% which has a mechanism to prevent loading a style file more than once.
% When loading the definitions by means of |\input|
% multiple instances have to be prevented manually:
%\iffalse
%This code needs to be before the `\ProvidesFile' directive
%which is defined at the beginning of this file.
%Therefore it is also placed there and commented out here.
%</package>
%<*discard>
%\fi
%    \begin{macrocode}
\ifdefined\childdocmain\endinput\fi
%    \end{macrocode}
%\iffalse
%</discard>
%<*package>
%\fi
%
% \macro{\ifchilddoc}
% \macro{\ifchilddocmanual}
% The conditional |\ifchilddoc| tells whether a
% child (true) or main (false) document is being compiled.
% The conditional |\ifchilddocmanual| tells whether
% the |\includeonly| mechanism is used (false) or
% the selection of child files must be performed manually (true).
% The definitions initialise to false:
%    \begin{macrocode}
\newif\ifchilddoc
\newif\ifchilddocmanual
%    \end{macrocode}

% \macro{\childdocname}
% \macro{\childdocjob}
% The macro |\childdocname| stores the name of the main document
% to be compiled. The macro |\childdocjob| stores the name of
% the document on which the \LaTeX{} compiler was originally invoked.
% The content of |\jobname| cannot be compared
% to filenames specified in the source due to different catcodes.
% The following code rescans |\jobname|, stores the result
% in |\childdocname| and saves a copy in |\childdocjob|:
%    \begin{macrocode}
\edef\childdocname{\scantokens\expandafter{\jobname\noexpand}}
\let\childdocjob\childdocname
%    \end{macrocode}

% \macro{\childdocdisable}
% The macro |\childdocdisable| prevents the main file
% from being processed more than once.
% At this stage, the main document command |\childdocmain|
% is assumed to be called once again where it should do nothing.
% Any subsequent call to it should prevent
% a secondary processing of the main document
% It overwrites the forwarding commands
% |\childdocof| and |\childdocforward|
% with empty macros to prevent further inclusions of the main document:
%    \begin{macrocode}
\newcommand{\childdocdisable}
{
  \renewcommand{\childdocmain}[1]{\renewcommand{\childdocmain}[1]{\endinput}}
  \renewcommand{\childdocof}[1]{}
  \renewcommand{\childdocby}[2][]{}
  \renewcommand{\childdocforward}[2][]{}
  \renewcommand{\childdocdisable}{}
}
%    \end{macrocode}

% \macro{\childdocmain}
% The macro |\childdocmain| is to be called at the top of the main file
% with nothing or the main filename (without extension) as argument.
% First, it breaks loops.
% If the argument is not empty and does not match |\childdocname|
% (which is set by the first inclusion of |childdoc.def|),
% |\ifchilddoc| is set to true, |\includeonly| is applied to the child file
% and |\jobname| is set to the main file
% (for proper handling of |.aux| files):
%    \begin{macrocode}
\newcommand{\childdocmain}[1]
{
  \childdocdisable\childdocmain{}
  \if?#1?\else
    \begingroup
      \def\childdoctmp{#1}
      \ifx\childdoctmp\childdocname
        \def\childdoctmp{}
      \else
        \def\childdoctmp
        {
          \childdoctrue
          \includeonly{\childdocname}
          \def\childdocjob{#1}
          \def\jobname{#1}
        }
      \fi
      \expandafter
    \endgroup
    \childdoctmp
  \fi
}
%    \end{macrocode}

% \macro{\childdocof}
% The command |\childdocof| redirects
% compilation to the main file |#1|.
%    \begin{macrocode}
\newcommand{\childdocof}[1]
{
  \childdocdisable
  \childdoctrue
  \includeonly{\childdocname}
  \def\jobname{#1}
  \def\childdocjob{#1}
  \input{#1}
}
%    \end{macrocode}

% \macro{\childdocby}
% The command |\childdocby| ....
%    \begin{macrocode}
\newcommand{\childdocby}[2][]
{
  \childdocdisable
  \childdoctrue
  \childdocmanualtrue
  \if?#1?\else
    \def\jobname{#2}
  \fi
  \def\childdocjob{#2}
  \input{#2}
  \endinput
}
%    \end{macrocode}

% \macro{\childdocforward}
% The command |\childdocforward| redirects
% compilation to the main file or
% (if the optional argument is given) a child file.
% Parameters are set as if the main file
% or a child file starting with |\childdocof| was compiled.
% Then compilation is handed over to the main file:
%    \begin{macrocode}
\newcommand{\childdocforward}[2][]
{
  \begingroup
    \if?#1?
      \def\childdoctmp
      {
        \def\childdocname{#2}
        \def\childdocjob{#2}
        \def\jobname{#2}
        \input{#2}
        \endinput
      }
    \else
      \def\childdoctmp
      {
        \childdocdisable
        \def\childdocname{#2}
        \childdoctrue
        \includeonly{#2}
        \def\childdocjob{#1}
        \def\jobname{#1}
        \input{#1}
        \endinput
      }
    \fi
    \expandafter
  \endgroup
  \childdoctmp
}
%    \end{macrocode}

% \macro{\childdocforwardprefix}
% The command |\childdocforwardprefix| redirects
% compilation to the main or a child file by means of a pattern.
% The prefix |#1| in the current filename is replaced by |#2|
% and the suffix of the current filename is kept
% (it is assumed that the filename does not contain the substring `|~~~|'
% which is used as a delimiter).
% Compilation is handed over to the new file by |\childdocforward|:
%    \begin{macrocode}
\newcommand{\childdocforwardprefix}[3][]
{
  \begingroup
    \def\childdocextract #2##1~~~{\def\childdoctmp{\childdocforward[#1]{#3##1}}}
    \expandafter\childdocextract\childdocname~~~
    \expandafter
  \endgroup
  \childdoctmp
}
%    \end{macrocode}

% \macro{\childdoc}
% The deprecated macro |\childdoc| is a legacy version of |\childdocmain|:
%    \begin{macrocode}
\newcommand{\childdoc}{\childdocmain}
%    \end{macrocode}

% \macro{\childdocredirect}
% The deprecated macro |\childdocredirect| is a legacy version
% of |\childdocforward| and |\childdocforwardprefix|:
%    \begin{macrocode}
\newcommand{\childdocredirect}[2][]
{
  \begingroup
    \if?#1?
      \def\childdoctmp{\childdocforward{#2}}
    \else
      \def\childdoctmp{\childdocforwardprefix{#1}{#2}}
    \fi
    \expandafter
  \endgroup
  \childdoctmp
}
%    \end{macrocode}

%\iffalse
%</package>
%\fi
%
\endinput

\childdocforward{cdocsamp}
%    \end{macrocode}

%\iffalse
%</sampledraft>
%\fi
%
% %%%%%%%%%%%%%%%%%%%%%%%%%%%%%%%%%%%%%%
% \paragraph{Forwarding for Final Version of the Chapters.}
%
% The following forwarding files |cdocsfn1.tex| and |cdocsfn2.tex|
% (with identical content)
% compile the final versions of the child documents
% |cdocsch1.tex| and |cdocsch2.tex|, respectively:
%\iffalse
%<*samplefinal>
%\fi
%    \begin{macrocode}
\def\version{final}
% \iffalse
%
% childdoc.dtx Copyright (C) 2017-2018 Niklas Beisert
%
% This work may be distributed and/or modified under the
% conditions of the LaTeX Project Public License, either version 1.3
% of this license or (at your option) any later version.
% The latest version of this license is in
%   http://www.latex-project.org/lppl.txt
% and version 1.3 or later is part of all distributions of LaTeX
% version 2005/12/01 or later.
%
% This work has the LPPL maintenance status `maintained'.
%
% The Current Maintainer of this work is Niklas Beisert.
%
% This work consists of the files childdoc.dtx and childdoc.ins
% and the derived files childdoc.def and cdocsamp.tex with
% cdocsch1.tex, cdocsch2.tex, cdocsdrf.tex, cdocsfn1.tex, cdocsfn2.tex.
%
%<package>\ifdefined\childdocmain\endinput\fi
%<package>\ProvidesFile{childdoc.def}[2018/12/30 v2.0 child document driver]
%<samplemain>\ProvidesFile{cdocsamp.tex}[2018/12/30 v2.0 sample for childdoc]
%<*driver>
%\ProvidesFile{childdoc.drv}[2018/12/30 v2.0 childdoc reference manual file]
\PassOptionsToClass{10pt,a4paper}{article}
\documentclass{ltxdoc}

\usepackage[margin=35mm]{geometry}
\usepackage{hyperref}
\usepackage{hyperxmp}
\usepackage[usenames]{color}

\hypersetup{colorlinks=true}
\hypersetup{pdfstartview=FitH}
\hypersetup{pdfpagemode=UseNone}
\hypersetup{pdfsource={}}
\hypersetup{pdflang={en-UK}}
\hypersetup{pdfcopyright={Copyright 2017-2018 Niklas Beisert.
  This work may be distributed and/or modified under the
  conditions of the LaTeX Project Public License, either version 1.3
  of this license or (at your option) any later version.}}
\hypersetup{pdflicenseurl={http://www.latex-project.org/lppl.txt}}
\hypersetup{pdfcontactaddress={ETH Zurich, ITP, HIT K,
  Wolfgang-Pauli-Strasse 27}}
\hypersetup{pdfcontactpostcode={8093}}
\hypersetup{pdfcontactcity={Zurich}}
\hypersetup{pdfcontactcountry={Switzerland}}
\hypersetup{pdfcontactemail={nbeisert@itp.phys.ethz.ch}}
\hypersetup{pdfcontacturl={http://people.phys.ethz.ch/\xmptilde nbeisert/}}

\newcommand{\secref}[1]{\hyperref[#1]{section \ref*{#1}}}

\parskip1ex
\parindent0pt
\let\olditemize\itemize
\def\itemize{\olditemize\parskip0pt}

\begin{document}

\title{The \textsf{childdoc} Package}
\hypersetup{pdftitle={The childdoc Package}}
\author{Niklas Beisert\\[2ex]
  Institut f\"ur Theoretische Physik\\
  Eidgen\"ossische Technische Hochschule Z\"urich\\
  Wolfgang-Pauli-Strasse 27, 8093 Z\"urich, Switzerland\\[1ex]
  \href{mailto:nbeisert@itp.phys.ethz.ch}
  {\texttt{nbeisert@itp.phys.ethz.ch}}}
\hypersetup{pdfauthor={Niklas Beisert}}
\hypersetup{pdfsubject={Manual for the LaTeX2e Package childdoc}}
\date{30 December 2018, \textsf{v2.0}}
\maketitle

\begin{abstract}\noindent
\textsf{childdoc} is a \LaTeXe{} package
that enables the direct compilation
of document sections included by |\include|
to individual files.
\end{abstract}

\begingroup
\parskip0ex
\tableofcontents
\endgroup

%%%%%%%%%%%%%%%%%%%%%%%%%%%%%%%%%%%%%%%%%%%%%%%%%%%%%%%%%%%%%%%%%%%%%%%%%%%%%%%%
%%%%%%%%%%%%%%%%%%%%%%%%%%%%%%%%%%%%%%%%%%%%%%%%%%%%%%%%%%%%%%%%%%%%%%%%%%%%%%%%
\section{Introduction}

\LaTeX{} provides a mechanism to structure a large document (such as a book)
into a main file and several child files (containing the chapters)
using the |\include| command.
This mechanism is beneficial for documents
which span hundreds of pages in order to
make the source file(s) more manageable.
Moreover, compilation can be restricted to
selected child files by means of the |\includeonly| command.
The latter feature can be used to reduce the compilation time while editing
(this was significantly more useful in the earlier days of \LaTeX{})
or to generate a smaller document which is easier to navigate.
Another application of |\includeonly| is to generate
documents consisting of selected parts of the complete document.

However, there are a few drawbacks of the plain |\include| mechanism:
\begin{itemize}
\item
The child files cannot be compiled on their own,
they can only be compiled via the main file.
A naive editing environment
(such as a text editor with an option
to have the current file processed by \LaTeX)
may require one to switch to the main file before compiling;
attempting to compile the child file produces errors.
\item
The main file must be modified (each time)
to adjust the |\includeonly| command
to the present needs. This easily leaves the main file in a messy state.
\item
The generated document will always carry the filename
of the main document. This is inconvenient if
several child files are to be compiled and
to be kept for distribution.
\end{itemize}

The present package provides a simple interface
to make child files individually compilable by \LaTeX{}.
Compiling a child file then has the same effect as compiling
the main file with an |\includeonly| command
to select the appropriate child.
Moreover the generated document will carry the name of the child
rather than the main file.
This resolves all three above issues.

This feature is meant to make the editing of books,
thesis documents and lecture notes somewhat more convenient.
However, the package can also be used efficiently for
composing a series of documents (such as exercise sheets)
which are typically distributed individually.
It then assists the author in generating the individual documents
(potentially in different versions)
as well as a document containing the collected series.
Another application is in developing style files
or other kinds of included material
where compilation of the style file could redirect
to a sample or test file.

%%%%%%%%%%%%%%%%%%%%%%%%%%%%%%%%%%%%%%%%%%%%%%%%%%%%%%%%%%%%%%%%%%%%%%%%%%%%%%%%
%%%%%%%%%%%%%%%%%%%%%%%%%%%%%%%%%%%%%%%%%%%%%%%%%%%%%%%%%%%%%%%%%%%%%%%%%%%%%%%%
\section{Usage}

First of all, the package \textsf{childdoc} is \emph{not} a standard
\LaTeXe{} |.sty| style file! Therefore it needs to be invoked in
a non-standard way.

%%%%%%%%%%%%%%%%%%%%%%%%%%%%%%%%%%%%%%%%%%%%%%%%%%%%%%%%%%%%%%%%%%%%%%%%%%%%%%%%
\subsection{Included Files}
\label{sec:include}

%%%%%%%%%%%%%%%%%%%%%%%%%%%%%%%%%%%%%%%%
\DescribeMacro{\childdocmain}
To use the package, add the commands
\begin{center}
\begin{tabular}{l}
|\input{childdoc.def}|\\
|\childdocmain{}|\\
\end{tabular}
\end{center}
at the very top of the main \LaTeX{} file,
in particular \emph{before} the |\documentclass| statement!
The argument of |\childdocmain| should be left empty
(but it must be present).

%%%%%%%%%%%%%%%%%%%%%%%%%%%%%%%%%%%%%%%%
\DescribeMacro{\childdocof}
Furthermore, add the commands
\begin{center}
\begin{tabular}{l}
|\input{childdoc.def}|\\
|\childdocof{|\textit{main}|}|\\
\end{tabular}
\end{center}
at the top of every child file \textit{child}
which is included by |\include{|\textit{child}|}|
from within the main file
(or at least for those files to be compiled individually).
The argument \textit{main} must be the filename of the main file.

There are a couple of
considerations in setting up the main and child documents:

%%%%%%%%%%%%%%%%%%%%%%%%%%%%%%%%%%%%%%%%
\paragraph{Restrictions.}

Please note the following restrictions:
\begin{itemize}
\item
|\childdocmain| must be called with one argument \textit{main}
to ensure compatibility with earlier version of the package.
It must either be empty (|\childdocmain{}|)
or precisely match the filename of the main file in which it is specified.
See \secref{sec:detection} for further information.
\item
The filename \textit{main} must be specified without the |.tex| extension.
\item
The filename \textit{main} is case sensitive
(even in case-insensitive file systems)
due to internal string comparison.
\item
The argument \textit{main} should be fully expanded, it cannot be a macro.
\item
Subdirectories and special characters should be avoided in filenames.
\item
The command |\childdocmain{|\textit{main}|}| must be followed by a whitespace.
It should not be followed immediately by another command
or by a comment mark `|%|'.
This is because the \TeX{} parser reads the token immediately following
the argument of |\childdocmain| and puts it
at the beginning of every child section;
however, a white\-space is ignored.
\end{itemize}

%%%%%%%%%%%%%%%%%%%%%%%%%%%%%%%%%%%%%%%%
\paragraph{Content of Main File.}

It is advisable to place all content in the child files included by |\include|.
Any output contained in the main file will appear in all child documents
unless suppressed manually;
it cannot be suppressed automatically by the |\includeonly| directive
and thus should normally be avoided.
A method to include some content in the main file
by means of conditional processing is described in \secref{sec:conditional}.

%%%%%%%%%%%%%%%%%%%%%%%%%%%%%%%%%%%%%%%%
\paragraph{Page Numbering.}

When only a part of the document is compiled,
the appropriate numbering of pages
(as well as other status parameters)
is determined from the |.aux| files.
The latter contain information from previous passes.
However this information needs to propagate through
all intermediate child documents.
Therefore the page numbering in child documents may well
be inconsistent until the complete document is compiled at least once.

A useful (if unconventional) way to always ensure a consistent
page numbering is to restart the numbering in each child document
and denote the pages by `\textit{child}|.|\textit{page}'
where \textit{child} represents the chapter/section number of the child file.
This can be achieved by the command
|\numberwithin{page}{|\textit{child}|}|
of the \textsf{amsmath} package
where \textit{child} can be |chapter| or |section|
depending on the chosen structuring.
Alternatively, one can modify the macro |\thepage| appropriately
and reset the counter |page| at the start of each child file.

%%%%%%%%%%%%%%%%%%%%%%%%%%%%%%%%%%%%%%%%%%%%%%%%%%%%%%%%%%%%%%%%%%%%%%%%%%%%%%%%
\subsection{Conditional Processing}
\label{sec:conditional}

The package provides a mechanism to compile different versions
of a document. To customise the versions further some conditional processing
can come in handy to distinguish which version is being compiled.
The package provides two macros to describe the compilation context:

%%%%%%%%%%%%%%%%%%%%%%%%%%%%%%%%%%%%%%%%
\DescribeMacro{\ifchilddoc}
The conditional |\ifchilddoc| distinguishes between the compilation of
child documents and the main document:
%
\begin{center}
|\ifchilddoc |\textit{child-code}| |[|\||else |\textit{main-code}]| \||fi|
\end{center}

%%%%%%%%%%%%%%%%%%%%%%%%%%%%%%%%%%%%%%%%
\DescribeMacro{\childdocname}
\DescribeMacro{\childdocjob}
The macro |\childdocname| contains the filename (without extension)
of the main or child file being processed.
Note that |\childdocjob| will always contain the name of the main file.

%%%%%%%%%%%%%%%%%%%%%%%%%%%%%%%%%%%%%%%%
\paragraph{Title Page.}

Conditional processing can be used to include a title or banner page
in the main document when proper precautions are taken.
Importantly, the code in the main file should ensure that the page counter
(as well as other status parameters which are stored in the |.aux| files)
takes the same value after the conditional processing.
Otherwise the page numbers may take divergent values
depending on which part is compiled.

For example, a title page could be declared by:
%
\begin{center}
\begin{tabular}{l}
|\ifchilddoc\||else|\\
|\addtocounter{page}{-1}|\\
\textit{code for title page}\\
|\newpage|\\
|\||fi|
\end{tabular}
\end{center}
%
A banner page for the child documents can be generated by:
%
\begin{center}
\begin{tabular}{l}
|\ifchilddoc|\\
|\addtocounter{page}{-1}|\\
\textit{code for banner page}\\
|\newpage|\\
|\||fi|
\end{tabular}
\end{center}
%
Here one could write a message such as:
\begin{center}
|This is the part \childdocname{} of \childdocjob{}.|
\end{center}

%%%%%%%%%%%%%%%%%%%%%%%%%%%%%%%%%%%%%%%%%%%%%%%%%%%%%%%%%%%%%%%%%%%%%%%%%%%%%%%%
\subsection{Flags}
\label{sec:flags}

The package makes it easy to generate different versions
of the main or child documents.
To this end compilation flags can be defined
and assigned different default values.
They will be particularly useful in conjunction
with the forwarding mechanism described in \secref{sec:forward}.

For example, it may be useful to have a flag |\version|
which can be set to |draft| or |final|.
The document source will contain some conditional code
depending on the value of |\version|.
Suppose further, the flag should default to |final| for the main file
and to |draft| for child files
which is a natural assignment for editing the document.
This is achieved by placing the following code
in the preamble of the main document
(below the |\childdocmain| directive):
%
\begin{center}
\begin{tabular}{l}
|\ifchilddoc|\\
|\providecommand{\version}{draft}|\\
|\||else|\\
|\providecommand{\version}{final}|\\
|\||fi|
\end{tabular}
\end{center}
%
The definition by |\providecommand| makes sure
that previous definitions are not overwritten.
Further statements |\providecommand{\version}{...}|
can thus be added before the above code to override it.

For the main file, one might add a line
(between |\childdocmain| and the above block)
%
\begin{center}
|%\ifchilddoc\||else\providecommand{\version}{draft}\||fi|
\end{center}
%
which can be uncommented to produce a draft version.
Likewise one can add a line to the very top of a child file
(above the |\childdocof{|\textit{main}|}| directive)
%
\begin{center}
|%\providecommand{\version}{final}|
\end{center}
%
which can be uncommented to produce the final version of this child document.

%%%%%%%%%%%%%%%%%%%%%%%%%%%%%%%%%%%%%%%%%%%%%%%%%%%%%%%%%%%%%%%%%%%%%%%%%%%%%%%%
\subsection{Forwarding}
\label{sec:forward}

Different versions of the main or child documents
using compilation flags as described in \secref{sec:flags}
can be (permanently) stored in different files
for convenient compilation, viewing and distribution.
To this end, the package defines a command
to pass on compilation to a different file:

%%%%%%%%%%%%%%%%%%%%%%%%%%%%%%%%%%%%%%%%
\DescribeMacro{\childdocforward}
The command |\childdocforward| redirects processing to
another source file:
%
\begin{center}
\begin{tabular}{l}
|\input{childdoc.def}|\\
|\childdocforward[|\textit{main}|]{|\textit{dest}|}|\\
\end{tabular}
\end{center}
%
The argument \textit{dest} is the destination file
(without extension).
It should be the main file or one of the child files.
Note that further \textsf{childdoc} directives
such as |\childdocof| and |\childdocforward|
in the indicated file will be processed in this form.
The optional argument \textit{main}
passes on directly to the main file \textit{main}
while pretending to compile the child \textit{dest}.
This form behaves as if \textit{dest}
issues |\childdocof{|\textit{main}|}| right away,
and no further \textsf{childdoc} directives will be processed.

%%%%%%%%%%%%%%%%%%%%%%%%%%%%%%%%%%%%%%%%
\DescribeMacro{\...prefix}
In the alternative form |\childdocforwardprefix|,
%
\begin{center}
\begin{tabular}{l}
|\input{childdoc.def}|\\
|\childdocforwardprefix[|\textit{main}|]{|\textit{prefix}|}{|\textit{dest}|}|
\end{tabular}
\end{center}
%
the destination file is determined by a pattern
depending on the current file:
To make this work, the current file must be called
`{\textit{prefix}\hspace{0.2em}\textit{suffix}}'
with \textit{prefix} matching precisely the argument.
Processing is then passed on to the file
`{\textit{dest}\hspace{0.2em}\textit{suffix}}'.
Surely, the same effect is achieved by
directly specifying the
argument `{\textit{dest}\hspace{0.2em}\textit{suffix}}'
in the first form.
However, that requires to set up a different file
for each child. With the alternative form of the command
all these files can have exactly the same content
which simplifies setting them up and maintaining them.

For example, the following file |draft.tex|
with a compilation flag |\version| as described in \secref{sec:flags}
compiles the main document as a draft:
%
\begin{center}
\begin{tabular}{l}
|\def\version{draft}|\\
|\input{childdoc.def}|\\
|\childdocforward{|\textit{main}|}|
\end{tabular}
\end{center}
%
Likewise, the following files |final|\textit{nn}|.tex|
compile the final version of the child document
|child|\textit{nn}|.tex|:
%
\begin{center}
\begin{tabular}{l}
|\def\version{final}|\\
|\input{childdoc.def}|\\
|\childdocforwardprefix{final}{child}|
\end{tabular}
\end{center}
%

Note that when several versions of a main file and/or of each child file
are to be generated, it may be convenient to set up a |Makefile| or
shell script to automatise the process.

%%%%%%%%%%%%%%%%%%%%%%%%%%%%%%%%%%%%%%%%%%%%%%%%%%%%%%%%%%%%%%%%%%%%%%%%%%%%%%%%
\subsection{Command Line Processing}
\label{sec:commandline}

The effect of redirection files can also be achieved by invoking
the \LaTeX{} compiler with a more elaborate command line.
Most conveniently this should be done as part
of a shell script or a |Makefile|.

When using \textsf{childdoc} in the main file, the following
command lines effectively perform a redirection
(note that depending on the shell being used,
backslashes may have to be doubled: `|\|' $\to$ `|\\|'):
%
\begin{center}
|... -jobname "|\textit{target}|" |\\|"|[\textit{flags}]%
|\input{childdoc.def}\childdocforward[|\textit{main}|]{|\textit{dest}|}"|
\end{center}
%
Here \textit{target} is the name of the output file,
\textit{main} is the name of the main file
and \textit{dest} is the name of the main or child file to be processed
(all filenames without extensions).
The optional argument \textit{main} can be omitted
if \textit{main} matches \textit{dest}.
Optionally, compilation \textit{flags} can be defined via |\def| commands.
This command line makes the \TeX{} engine believe
it is compiling the file \textit{target}
whose content is specified as the latter parameter.
The provided code then forwards the processing to
\textit{main} or \textit{dest} as described in \secref{sec:forward}.

%%%%%%%%%%%%%%%%%%%%%%%%%%%%%%%%%%%%%%%%%%%%%%%%%%%%%%%%%%%%%%%%%%%%%%%%%%%%%%%%
\subsection{Include by Input}
\label{sec:input}

Including child documents by |\include| has some restrictions by design.
Most notably, the content of a child document always occupies
its own set of pages; pages cannot be shared between child documents.
Usually, this behaviour makes perfect sense
because each child document contain an essential part of the document.
However, in some situations it may be desirable to compose
a document from a collection of parts
without having mandatory page breaks between then.
For this case, the package
provides a mechanism to include parts
by |\input| which can also be processed individually.
However, by construction this mechanism
requires manual handling of the content to be output.

%%%%%%%%%%%%%%%%%%%%%%%%%%%%%%%%%%%%%%%%
\DescribeMacro{\ifchilddocmanual}
The main file should be prepared as usual, see \secref{sec:include}.
However, the document body must make a distinction
between processing of an individual part and of the main document, e.g.:
%
\begin{center}
\begin{tabular}{l}
|\ifchilddocmanual|\\
|\input{\childdocname}|\\
|\||else|\\
\textit{document body with }|\input{|\textit{part}|}|\\
|\||fi|
\end{tabular}
\end{center}
%
The conditional |\ifchilddocmanual| is true whenever
a part to be included by |\input| is being compiled,
and the name of the part is stored in |\childdocname|.

%%%%%%%%%%%%%%%%%%%%%%%%%%%%%%%%%%%%%%%%
\DescribeMacro{\childdocby}
Each part to be included by |\input| should start with:
%
\begin{center}
\begin{tabular}{l}
|\input{childdoc.def}|\\
|\childdocby{|\textit{main}|}|\\
\end{tabular}
\end{center}
%
The directive |\childdocby| is similar to |\childdocof|
described in \secref{sec:include},
but the subsequent selection of content must be done manually.
To that end, both |\ifchilddoc| and |\ifchilddocmanual|
will be true upon processing of a part,
and the name of the part is stored in |\childdocname|.
Note that |\jobname| will be set to the filename of the current part
so that each part receives an individual |.aux| file
that does not interfere with the |.aux| file(s) of the main document.
This behaviour can be altered by the alternative form
|\childdocby[*]{|\textit{main}|}| (with a non-empty optional argument)
which uses the |.aux| file of the main document
by setting |\jobname| to \textit{main}.

%%%%%%%%%%%%%%%%%%%%%%%%%%%%%%%%%%%%%%%%%%%%%%%%%%%%%%%%%%%%%%%%%%%%%%%%%%%%%%%%
\subsection{Driver Development}
\label{sec:driver}

The \textsf{childdoc} mechanism can also be use for the development
of definition files such as \LaTeX{} styles or classes.
This case differs from the above setup with multiple parts
included by |\include| in that no |\includeonly| should be invoked.
This can be achieved by starting the include file
(before |\ProvidesPackage|) with:
%
\begin{center}
\begin{tabular}{l}
|\input{childdoc.def}|\\
|\childdocforward{|\textit{main}|}|\\
\end{tabular}
\end{center}
%
or alternatively with:
%
\begin{center}
\begin{tabular}{l}
|\input{childdoc.def}|\\
|\childdocby{|\textit{main}|}|\\
\end{tabular}
\end{center}
%
Both forms have slightly different effects as described above.
The main file is prepared as usual, see \secref{sec:include}.

%%%%%%%%%%%%%%%%%%%%%%%%%%%%%%%%%%%%%%%%%%%%%%%%%%%%%%%%%%%%%%%%%%%%%%%%%%%%%%%%
\subsection{Legacy Detection}
\label{sec:detection}

The directive |\childdocmain| in the main file can detect
whether the complete document or merely a child is to be compiled
even without using the directive |\childdocof|.
This method is deprecated because it is less robust
and there is no compelling reason to use it;
it is merely provided for backward compatibility
and it may be removed in future versions.

If the detection mechanism is to be used,
it is mandatory to correctly specify
the filename of the main file as the argument of |\childdocmain|:
%
\begin{center}
\begin{tabular}{l}
|\input{childdoc.def}|\\
|\childdocmain{|\textit{main}|}|\\
\end{tabular}
\end{center}
%
If |\jobname| does not match the argument \textit{main} of |\childdocmain|,
it is assumed that |\jobname| points to the child file to be compiled.
When using |\childdocmain| with the main file specified as argument,
it suffices to start a child file
with just |\input{|\textit{main}|}|
without loading of the package and using |\childdocof|.
If instead all processing is done
with the appropriate \textsf{childdoc} directives,
the argument of \textit{main} of |\childdocmain| can be empty.

An alternative version of the command line processing described
in \secref{sec:commandline} using the detection mechanism reads:
%
\begin{center}
|... -jobname "|\textit{target}|" "|[\textit{flags}]%
[|\def\jobname{|\textit{dest}|}|]|\input{|\textit{main}|}"|
\end{center}

%%%%%%%%%%%%%%%%%%%%%%%%%%%%%%%%%%%%%%%%%%%%%%%%%%%%%%%%%%%%%%%%%%%%%%%%%%%%%%%%
\subsection{Manual Code}
\label{sec:manual}

In case one cannot be certain whether the definitions file |childdoc.def|
is installed on the target \TeX{} distribution
and one prefers not to ship it,
it is conceivable to paste a few relevant commands into the sources.

To that end, drop all statements |\input{childdoc.def}|
and perform the replacements as outlined below.
Instead of |\childdocmain{|\textit{main}|}| add the following code
to the top of the main file:
%
\begin{center}
\begin{tabular}{l}
|\||ifdefined\childdocname\endinput\||fi\newif\ifchilddoc|\\
|\edef\childdocname{\scantokens\expandafter{\jobname\noexpand}}|\\
|\def\childdocmain{|\textit{main}|}\||ifx\childdocmain\childdocname\||else|\\
|\childdoctrue\includeonly{\childdocname}\let\jobname\childdocmain\||fi|\\
\end{tabular}
\end{center}
%
Instead of |\childdocof{|\textit{main}|}| just include the main file
at the top of each child file:
%
\begin{center}
|\input{|\textit{main}|}|
\end{center}
%
A simple redirection |\childdocforward{|\textit{dest}|}| is achieved by:
%
\begin{center}
|\def\jobname{|\textit{dest}|}\input{\jobname}|
\end{center}
%
The redirection with prefix
|\childdocforwardprefix[|\textit{prefix}|]{|\textit{dest}|}|
is accomplished by:
%
\begin{center}
\begin{tabular}{l}
|{\edef\jobname{\scantokens\expandafter{\jobname\noexpand}}|\\
|\def\redirectjob |\textit{prefix}|#1~~~{\gdef\jobname{|\textit{dest}|#1}}|\\
|\expandafter\redirectjob\jobname~~~}\input{\jobname}|
\end{tabular}
\end{center}

In an alternative approach,
child documents can be compiled by a specific command line
without additional code or specific definitions:
%
\begin{center}
|... -jobname "|\textit{target}|" "|[\textit{flags}]%
|\includeonly{|\textit{dest}|}\input{|\textit{main}|}"|
\end{center}
%

%%%%%%%%%%%%%%%%%%%%%%%%%%%%%%%%%%%%%%%%%%%%%%%%%%%%%%%%%%%%%%%%%%%%%%%%%%%%%%%%
%%%%%%%%%%%%%%%%%%%%%%%%%%%%%%%%%%%%%%%%%%%%%%%%%%%%%%%%%%%%%%%%%%%%%%%%%%%%%%%%
\section{Information}

%%%%%%%%%%%%%%%%%%%%%%%%%%%%%%%%%%%%%%%%%%%%%%%%%%%%%%%%%%%%%%%%%%%%%%%%%%%%%%%%
\subsection{Copyright}

Copyright \copyright{} 2017--2018 Niklas Beisert

This work may be distributed and/or modified under the
conditions of the \LaTeX{} Project Public License, either version 1.3
of this license or (at your option) any later version.
The latest version of this license is in
  \url{http://www.latex-project.org/lppl.txt}
and version 1.3 or later is part of all distributions of \LaTeX{}
version 2005/12/01 or later.

This work has the LPPL maintenance status `maintained'.

The Current Maintainer of this work is Niklas Beisert.

This work consists of the files |README.txt|, |childdoc.ins| and |childdoc.dtx|
as well as the derived files |childdoc.def|, |cdocsamp.tex|
with |cdocsch1.tex|, |cdocsch2.tex|, |cdocspt3.tex|, |cdocspt4.tex|,
|cdocsdrf.tex|, |cdocsfn1.tex|, |cdocsfn2.tex|
as well as |childdoc.pdf|.

%%%%%%%%%%%%%%%%%%%%%%%%%%%%%%%%%%%%%%%%%%%%%%%%%%%%%%%%%%%%%%%%%%%%%%%%%%%%%%%%
\subsection{Files and Installation}

The package consists of the files:
%
\begin{center}
\begin{tabular}{ll}
    |README.txt|   & readme file \\
    |childdoc.ins| & installation file \\
    |childdoc.dtx| & source file \\
    |childdoc.def| & definition file \\
    |cdocsamp.tex| & sample main file \\
    |cdocsch1.tex| & sample include file \\
    |cdocsch2.tex| & sample include file \\
    |cdocspt3.tex| & sample part file \\
    |cdocspt4.tex| & sample part file \\
    |cdocsdrf.tex| & sample redirection file \\
    |cdocsfn1.tex| & sample redirection file \\
    |cdocsfn2.tex| & sample redirection file \\
    |childdoc.pdf| & manual
\end{tabular}
\end{center}
%
The distribution consists of the files
|README.txt|, |childdoc.ins| and |childdoc.dtx|.
%
\begin{itemize}
\item
Run (pdf)\LaTeX{} on |childdoc.dtx|
to compile the manual |childdoc.pdf| (this file).
\item
Run \LaTeX{} on |childdoc.ins| to create the definitions file |childdoc.def|
and the sample |cdocsamp.tex| with include files
|cdocsch1.tex|, |cdocsch2.tex|, |cdocspt3.tex|, |cdocspt4.tex|,
|cdocsdrf.tex|, |cdocsfn1.tex|, |cdocsfn2.tex|.
Then copy the file |childdoc.def| to an appropriate directory of your \LaTeX{}
distribution, e.g.\ \textit{texmf-root}|/tex/latex/childdoc|.
\end{itemize}

%%%%%%%%%%%%%%%%%%%%%%%%%%%%%%%%%%%%%%%%%%%%%%%%%%%%%%%%%%%%%%%%%%%%%%%%%%%%%%%%
\subsection{Related CTAN Packages}

There are several other packages which offer a similar functionality:
%
\begin{itemize}
\item
The packages
\href{http://ctan.org/pkg/docmute}{\textsf{docmute}},
\href{http://ctan.org/pkg/includex}{\textsf{includex}} and
\href{http://ctan.org/pkg/standalone}{\textsf{standalone}}
provide commands to include only the document body of
a child file thus allowing both files to be compiled individually.
\item
The packages \href{http://ctan.org/pkg/subdocs}{\textsf{subdocs}}
and \href{http://ctan.org/pkg/subfiles}{\textsf{subfiles}}
provide structures in which the main and child documents can be
encapsulated and allowing them to be compiled individually.
The inclusion mechanism is different from the conventional |\include|.
\item
The package \href{http://ctan.org/pkg/combine}{\textsf{combine}}
is an elaborate solution to combine several documents into one.
\end{itemize}
%
See also the CTAN topic \href{http://ctan.org/topic/subdocs}{\textsf{subdocs}}
for further related packages.
The present package differs from the above solutions in that
a document structure constructed with the conventional |\include| mechanism
just needs two extra commands at the top of every file
such that all constituent files can be compiled individually.

%%%%%%%%%%%%%%%%%%%%%%%%%%%%%%%%%%%%%%%%%%%%%%%%%%%%%%%%%%%%%%%%%%%%%%%%%%%%%%%%
%\subsection{Feature Suggestions}
%
%The following is a list of features which may be useful for future
%versions of this package:
%%
%\begin{itemize}
%\item
%\ldots
%\end{itemize}

%%%%%%%%%%%%%%%%%%%%%%%%%%%%%%%%%%%%%%%%%%%%%%%%%%%%%%%%%%%%%%%%%%%%%%%%%%%%%%%%
\subsection{Revision History}

%%%%%%%%%%%%%%%%%%%%%%%%%%%%%%%%%%%%%%%%
\paragraph{v2.0:} 2018/12/30

\begin{itemize}
\item
immediate forward processing
\item
added |\childdocby| mechanism
\item
manual restructured
\end{itemize}

%%%%%%%%%%%%%%%%%%%%%%%%%%%%%%%%%%%%%%%%
\paragraph{v1.6:} 2018/01/17

\begin{itemize}
\item
application for development of include files
\item
corrections to manual
\end{itemize}

%%%%%%%%%%%%%%%%%%%%%%%%%%%%%%%%%%%%%%%%
\paragraph{v1.5:} 2017/05/21

\begin{itemize}
\item
more complete structuring introduced
\item
|\childdocof| introduced
\item
|\childdoc| renamed to |\childdocmain|
\item
|\childredirect| renamed to |\childdocforward| and |\childdocforwardprefix|
and functionality expanded
\end{itemize}

%%%%%%%%%%%%%%%%%%%%%%%%%%%%%%%%%%%%%%%%
\paragraph{v1.0:} 2017/04/27

\begin{itemize}
\item
manual and install package
\item
first version published on CTAN
\end{itemize}

%%%%%%%%%%%%%%%%%%%%%%%%%%%%%%%%%%%%%%%%
\paragraph{v0.6:} 2017/04/26

\begin{itemize}
\item
redirection mechanism added
\end{itemize}

%%%%%%%%%%%%%%%%%%%%%%%%%%%%%%%%%%%%%%%%
\paragraph{v0.5:} 2017/04/26

\begin{itemize}
\item
functionality in definition file
\end{itemize}


%%%%%%%%%%%%%%%%%%%%%%%%%%%%%%%%%%%%%%%%%%%%%%%%%%%%%%%%%%%%%%%%%%%%%%%%%%%%%%%%
%%%%%%%%%%%%%%%%%%%%%%%%%%%%%%%%%%%%%%%%%%%%%%%%%%%%%%%%%%%%%%%%%%%%%%%%%%%%%%%%
%%%%%%%%%%%%%%%%%%%%%%%%%%%%%%%%%%%%%%%%%%%%%%%%%%%%%%%%%%%%%%%%%%%%%%%%%%%%%%%%
\appendix

\settowidth\MacroIndent{\rmfamily\scriptsize 000\ }

 \DocInput{childdoc.dtx}

\end{document}
%</driver>
% \fi
%
% %%%%%%%%%%%%%%%%%%%%%%%%%%%%%%%%%%%%%%%%%%%%%%%%%%%%%%%%%%%%%%%%%%%%%%%%%%%%%%
% %%%%%%%%%%%%%%%%%%%%%%%%%%%%%%%%%%%%%%%%%%%%%%%%%%%%%%%%%%%%%%%%%%%%%%%%%%%%%%
% \section{Sample}
%\iffalse
%<*samplemain>
%\fi
%
% The following presents a sample document
% with two chapters, two parts, a title page,
% a compile flag as well as three forwarding files to set the flag.
% It consists of eight |.tex| files:
% \begin{center}
% \begin{tabular}{ll}
% |cdocsamp.tex|&main file\\
% |cdocsch1.tex|&include file for chapter 1\\
% |cdocsch2.tex|&include file for chapter 2\\
% |cdocspt3.tex|&include file for part 3\\
% |cdocspt4.tex|&include file for part 4\\
% |cdocsdrf.tex|&forwarding file for main file in draft mode\\
% |cdocsfi1.tex|&forwarding file for final version of chapter 1\\
% |cdocsfi2.tex|&forwarding file for final version of chapter 2\\
% \end{tabular}
% \end{center}
% Each of the eight files can be compiled directly by the \LaTeX{} compiler.
%
% %%%%%%%%%%%%%%%%%%%%%%%%%%%%%%%%%%%%%%
% \paragraph{Main File.}
%
% The main file is called |cdocsamp.tex|.
%
% Load the \textsf{childdoc} definitions and
% declare the filename for the main document:
%    \begin{macrocode}
\input{childdoc.def}
\childdocmain{}
%    \end{macrocode}

% Optional override for |\version| flag:
%    \begin{macrocode}
%%\ifchilddoc\else\providecommand{\version}{draft}\fi
%    \end{macrocode}

% Define the default values for the |\version| flag
% (|final| for the main file and |draft| for childs):
%    \begin{macrocode}
\ifchilddoc
\providecommand{\version}{draft}
\else
\providecommand{\version}{final}
\fi
%    \end{macrocode}

% Load the standard document class:
%    \begin{macrocode}
\documentclass[12pt]{article}
%    \end{macrocode}

% Start the document body:
%    \begin{macrocode}
\begin{document}
%    \end{macrocode}

% Declare a title page.
% Print title, part of document being processed and version flag:
%    \begin{macrocode}
\addtocounter{page}{-1}
\begin{center}
{\LARGE\bfseries{}childdoc example\par}
\vspace{1cm}
\ifchilddoc
\ifchilddocmanual part\else chapter\fi:
`\childdocname' of `\childdocjob'\par
\else
main document: `\childdocjob'\par
\fi
version: \version\par
\end{center}
\newpage
%    \end{macrocode}

% Manually include selected file,
% otherwise process as usual:
%    \begin{macrocode}
\ifchilddocmanual
\section*{part `\childdocname'}
\input{\childdocname}
\else
%    \end{macrocode}

% Include the two chapters:
%    \begin{macrocode}
\include{cdocsch1}
\include{cdocsch2}
%    \end{macrocode}

% Include the two parts unless only chapters should be displayed:
%    \begin{macrocode}
\ifchilddoc\else
\section{part three}
\input{cdocspt3}
\section{part four}
\input{cdocspt4}
\fi
%    \end{macrocode}

% Process as usual until here:
%    \begin{macrocode}
\fi
%    \end{macrocode}

% End of document body:
%    \begin{macrocode}
\end{document}
%    \end{macrocode}
%\iffalse
%</samplemain>
%\fi
%
% %%%%%%%%%%%%%%%%%%%%%%%%%%%%%%%%%%%%%%
% \paragraph{Chapter Include Files.}
%
% The include files are called |cdocsch1.tex| and |cdocsch2.tex|.
%
%\iffalse
%<*samplechap1|samplechap2>
%\fi

% Optional override for |\version| flag:
%    \begin{macrocode}
%%\providecommand{\version}{final}
%    \end{macrocode}

% Include the main document:
%    \begin{macrocode}
\input{childdoc.def}
\childdocof{cdocsamp}
%    \end{macrocode}

%\iffalse
%</samplechap1|samplechap2>
%\fi
%
%\iffalse
%<*samplechap1>
%\fi
% Some text for chapter 1:
%    \begin{macrocode}
\section{one}
some text in chapter one
%    \end{macrocode}

%\iffalse
%</samplechap1>
%\fi
% Some text for chapter 2:
%\iffalse
%<*samplechap2>
%\fi
%    \begin{macrocode}
\section{two}
more text in chapter two
%    \end{macrocode}

%\iffalse
%</samplechap2>
%\fi
%
% %%%%%%%%%%%%%%%%%%%%%%%%%%%%%%%%%%%%%%
% \paragraph{Part Include Files.}
%
% The include files are called |cdocspt3.tex| and |cdocspt4.tex|.
%
%\iffalse
%<*samplepart3|samplepart4>
%\fi

% Optional override for |\version| flag:
%    \begin{macrocode}
%%\providecommand{\version}{final}
%    \end{macrocode}

% Include the main document:
%    \begin{macrocode}
\input{childdoc.def}
\childdocby{cdocsamp}
%    \end{macrocode}

%\iffalse
%</samplepart3|samplepart4>
%\fi
%
%\iffalse
%<*samplepart3>
%\fi
% Some text for part 3:
%    \begin{macrocode}
some text in part three
%    \end{macrocode}

%\iffalse
%</samplepart3>
%\fi
% Some text for part 4:
%\iffalse
%<*samplepart4>
%\fi
%    \begin{macrocode}
more text in part four
%    \end{macrocode}

%\iffalse
%</samplepart4>
%\fi
%
% %%%%%%%%%%%%%%%%%%%%%%%%%%%%%%%%%%%%%%
% \paragraph{Forwarding for a Complete Draft.}
%
% The following forwarding file |cdocsdrf.tex|
% compiles the main document in draft mode:
%\iffalse
%<*sampledraft>
%\fi
%    \begin{macrocode}
\def\version{draft}
\input{childdoc.def}
\childdocforward{cdocsamp}
%    \end{macrocode}

%\iffalse
%</sampledraft>
%\fi
%
% %%%%%%%%%%%%%%%%%%%%%%%%%%%%%%%%%%%%%%
% \paragraph{Forwarding for Final Version of the Chapters.}
%
% The following forwarding files |cdocsfn1.tex| and |cdocsfn2.tex|
% (with identical content)
% compile the final versions of the child documents
% |cdocsch1.tex| and |cdocsch2.tex|, respectively:
%\iffalse
%<*samplefinal>
%\fi
%    \begin{macrocode}
\def\version{final}
\input{childdoc.def}
\childdocforwardprefix[cdocsamp]{cdocsfn}{cdocsch}
%    \end{macrocode}

%\iffalse
%</samplefinal>
%\fi
%
% %%%%%%%%%%%%%%%%%%%%%%%%%%%%%%%%%%%%%%
% \paragraph{Command Line Processing.}
%
% The following three command lines generate the output files
% |cdocscld|, |cdocscl1| and |cdocscl2|
% which should be identical to
% |cdocsdrf|, |cdocsch1| and |cdocsfn2|, respectively:
% \begin{center}
% \begin{tabular}{l}
% |latex -jobname cdocscld \|\\
% |  "\def\version{draft}\input{childdoc.def}\childdocforward{cdocsamp}"|\\
% |latex -jobname cdocscl1 \|\\
% |  "\input{childdoc.def}\childdocforward[cdocsamp]{cdocsch1}"|\\
% |latex -jobname cdocscl2 \|\\
% |  "\def\version{final}\input{childdoc.def}\childdocforward{cdocsch2}"|
% \end{tabular}
% \end{center}
% Note that the trailing backslash on each first line
% merely continues the input to the second line
% (for convenient cut ant paste).
% Furthermore, the command |latex| can be replaced by any
% of its alternative versions such as |pdflatex|.
%
% %%%%%%%%%%%%%%%%%%%%%%%%%%%%%%%%%%%%%%%%%%%%%%%%%%%%%%%%%%%%%%%%%%%%%%%%%%%%%%
% %%%%%%%%%%%%%%%%%%%%%%%%%%%%%%%%%%%%%%%%%%%%%%%%%%%%%%%%%%%%%%%%%%%%%%%%%%%%%%
% \section{Implementation}
%\iffalse
%<*package>
%\fi
%
% This section describes the definitions file |childdoc.def|.

% The definitions cannot be loaded using |\usepackage| or |\RequirePackage|
% which has a mechanism to prevent loading a style file more than once.
% When loading the definitions by means of |\input|
% multiple instances have to be prevented manually:
%\iffalse
%This code needs to be before the `\ProvidesFile' directive
%which is defined at the beginning of this file.
%Therefore it is also placed there and commented out here.
%</package>
%<*discard>
%\fi
%    \begin{macrocode}
\ifdefined\childdocmain\endinput\fi
%    \end{macrocode}
%\iffalse
%</discard>
%<*package>
%\fi
%
% \macro{\ifchilddoc}
% \macro{\ifchilddocmanual}
% The conditional |\ifchilddoc| tells whether a
% child (true) or main (false) document is being compiled.
% The conditional |\ifchilddocmanual| tells whether
% the |\includeonly| mechanism is used (false) or
% the selection of child files must be performed manually (true).
% The definitions initialise to false:
%    \begin{macrocode}
\newif\ifchilddoc
\newif\ifchilddocmanual
%    \end{macrocode}

% \macro{\childdocname}
% \macro{\childdocjob}
% The macro |\childdocname| stores the name of the main document
% to be compiled. The macro |\childdocjob| stores the name of
% the document on which the \LaTeX{} compiler was originally invoked.
% The content of |\jobname| cannot be compared
% to filenames specified in the source due to different catcodes.
% The following code rescans |\jobname|, stores the result
% in |\childdocname| and saves a copy in |\childdocjob|:
%    \begin{macrocode}
\edef\childdocname{\scantokens\expandafter{\jobname\noexpand}}
\let\childdocjob\childdocname
%    \end{macrocode}

% \macro{\childdocdisable}
% The macro |\childdocdisable| prevents the main file
% from being processed more than once.
% At this stage, the main document command |\childdocmain|
% is assumed to be called once again where it should do nothing.
% Any subsequent call to it should prevent
% a secondary processing of the main document
% It overwrites the forwarding commands
% |\childdocof| and |\childdocforward|
% with empty macros to prevent further inclusions of the main document:
%    \begin{macrocode}
\newcommand{\childdocdisable}
{
  \renewcommand{\childdocmain}[1]{\renewcommand{\childdocmain}[1]{\endinput}}
  \renewcommand{\childdocof}[1]{}
  \renewcommand{\childdocby}[2][]{}
  \renewcommand{\childdocforward}[2][]{}
  \renewcommand{\childdocdisable}{}
}
%    \end{macrocode}

% \macro{\childdocmain}
% The macro |\childdocmain| is to be called at the top of the main file
% with nothing or the main filename (without extension) as argument.
% First, it breaks loops.
% If the argument is not empty and does not match |\childdocname|
% (which is set by the first inclusion of |childdoc.def|),
% |\ifchilddoc| is set to true, |\includeonly| is applied to the child file
% and |\jobname| is set to the main file
% (for proper handling of |.aux| files):
%    \begin{macrocode}
\newcommand{\childdocmain}[1]
{
  \childdocdisable\childdocmain{}
  \if?#1?\else
    \begingroup
      \def\childdoctmp{#1}
      \ifx\childdoctmp\childdocname
        \def\childdoctmp{}
      \else
        \def\childdoctmp
        {
          \childdoctrue
          \includeonly{\childdocname}
          \def\childdocjob{#1}
          \def\jobname{#1}
        }
      \fi
      \expandafter
    \endgroup
    \childdoctmp
  \fi
}
%    \end{macrocode}

% \macro{\childdocof}
% The command |\childdocof| redirects
% compilation to the main file |#1|.
%    \begin{macrocode}
\newcommand{\childdocof}[1]
{
  \childdocdisable
  \childdoctrue
  \includeonly{\childdocname}
  \def\jobname{#1}
  \def\childdocjob{#1}
  \input{#1}
}
%    \end{macrocode}

% \macro{\childdocby}
% The command |\childdocby| ....
%    \begin{macrocode}
\newcommand{\childdocby}[2][]
{
  \childdocdisable
  \childdoctrue
  \childdocmanualtrue
  \if?#1?\else
    \def\jobname{#2}
  \fi
  \def\childdocjob{#2}
  \input{#2}
  \endinput
}
%    \end{macrocode}

% \macro{\childdocforward}
% The command |\childdocforward| redirects
% compilation to the main file or
% (if the optional argument is given) a child file.
% Parameters are set as if the main file
% or a child file starting with |\childdocof| was compiled.
% Then compilation is handed over to the main file:
%    \begin{macrocode}
\newcommand{\childdocforward}[2][]
{
  \begingroup
    \if?#1?
      \def\childdoctmp
      {
        \def\childdocname{#2}
        \def\childdocjob{#2}
        \def\jobname{#2}
        \input{#2}
        \endinput
      }
    \else
      \def\childdoctmp
      {
        \childdocdisable
        \def\childdocname{#2}
        \childdoctrue
        \includeonly{#2}
        \def\childdocjob{#1}
        \def\jobname{#1}
        \input{#1}
        \endinput
      }
    \fi
    \expandafter
  \endgroup
  \childdoctmp
}
%    \end{macrocode}

% \macro{\childdocforwardprefix}
% The command |\childdocforwardprefix| redirects
% compilation to the main or a child file by means of a pattern.
% The prefix |#1| in the current filename is replaced by |#2|
% and the suffix of the current filename is kept
% (it is assumed that the filename does not contain the substring `|~~~|'
% which is used as a delimiter).
% Compilation is handed over to the new file by |\childdocforward|:
%    \begin{macrocode}
\newcommand{\childdocforwardprefix}[3][]
{
  \begingroup
    \def\childdocextract #2##1~~~{\def\childdoctmp{\childdocforward[#1]{#3##1}}}
    \expandafter\childdocextract\childdocname~~~
    \expandafter
  \endgroup
  \childdoctmp
}
%    \end{macrocode}

% \macro{\childdoc}
% The deprecated macro |\childdoc| is a legacy version of |\childdocmain|:
%    \begin{macrocode}
\newcommand{\childdoc}{\childdocmain}
%    \end{macrocode}

% \macro{\childdocredirect}
% The deprecated macro |\childdocredirect| is a legacy version
% of |\childdocforward| and |\childdocforwardprefix|:
%    \begin{macrocode}
\newcommand{\childdocredirect}[2][]
{
  \begingroup
    \if?#1?
      \def\childdoctmp{\childdocforward{#2}}
    \else
      \def\childdoctmp{\childdocforwardprefix{#1}{#2}}
    \fi
    \expandafter
  \endgroup
  \childdoctmp
}
%    \end{macrocode}

%\iffalse
%</package>
%\fi
%
\endinput

\childdocforwardprefix[cdocsamp]{cdocsfn}{cdocsch}
%    \end{macrocode}

%\iffalse
%</samplefinal>
%\fi
%
% %%%%%%%%%%%%%%%%%%%%%%%%%%%%%%%%%%%%%%
% \paragraph{Command Line Processing.}
%
% The following three command lines generate the output files
% |cdocscld|, |cdocscl1| and |cdocscl2|
% which should be identical to
% |cdocsdrf|, |cdocsch1| and |cdocsfn2|, respectively:
% \begin{center}
% \begin{tabular}{l}
% |latex -jobname cdocscld \|\\
% |  "\def\version{draft}% \iffalse
%
% childdoc.dtx Copyright (C) 2017-2018 Niklas Beisert
%
% This work may be distributed and/or modified under the
% conditions of the LaTeX Project Public License, either version 1.3
% of this license or (at your option) any later version.
% The latest version of this license is in
%   http://www.latex-project.org/lppl.txt
% and version 1.3 or later is part of all distributions of LaTeX
% version 2005/12/01 or later.
%
% This work has the LPPL maintenance status `maintained'.
%
% The Current Maintainer of this work is Niklas Beisert.
%
% This work consists of the files childdoc.dtx and childdoc.ins
% and the derived files childdoc.def and cdocsamp.tex with
% cdocsch1.tex, cdocsch2.tex, cdocsdrf.tex, cdocsfn1.tex, cdocsfn2.tex.
%
%<package>\ifdefined\childdocmain\endinput\fi
%<package>\ProvidesFile{childdoc.def}[2018/12/30 v2.0 child document driver]
%<samplemain>\ProvidesFile{cdocsamp.tex}[2018/12/30 v2.0 sample for childdoc]
%<*driver>
%\ProvidesFile{childdoc.drv}[2018/12/30 v2.0 childdoc reference manual file]
\PassOptionsToClass{10pt,a4paper}{article}
\documentclass{ltxdoc}

\usepackage[margin=35mm]{geometry}
\usepackage{hyperref}
\usepackage{hyperxmp}
\usepackage[usenames]{color}

\hypersetup{colorlinks=true}
\hypersetup{pdfstartview=FitH}
\hypersetup{pdfpagemode=UseNone}
\hypersetup{pdfsource={}}
\hypersetup{pdflang={en-UK}}
\hypersetup{pdfcopyright={Copyright 2017-2018 Niklas Beisert.
  This work may be distributed and/or modified under the
  conditions of the LaTeX Project Public License, either version 1.3
  of this license or (at your option) any later version.}}
\hypersetup{pdflicenseurl={http://www.latex-project.org/lppl.txt}}
\hypersetup{pdfcontactaddress={ETH Zurich, ITP, HIT K,
  Wolfgang-Pauli-Strasse 27}}
\hypersetup{pdfcontactpostcode={8093}}
\hypersetup{pdfcontactcity={Zurich}}
\hypersetup{pdfcontactcountry={Switzerland}}
\hypersetup{pdfcontactemail={nbeisert@itp.phys.ethz.ch}}
\hypersetup{pdfcontacturl={http://people.phys.ethz.ch/\xmptilde nbeisert/}}

\newcommand{\secref}[1]{\hyperref[#1]{section \ref*{#1}}}

\parskip1ex
\parindent0pt
\let\olditemize\itemize
\def\itemize{\olditemize\parskip0pt}

\begin{document}

\title{The \textsf{childdoc} Package}
\hypersetup{pdftitle={The childdoc Package}}
\author{Niklas Beisert\\[2ex]
  Institut f\"ur Theoretische Physik\\
  Eidgen\"ossische Technische Hochschule Z\"urich\\
  Wolfgang-Pauli-Strasse 27, 8093 Z\"urich, Switzerland\\[1ex]
  \href{mailto:nbeisert@itp.phys.ethz.ch}
  {\texttt{nbeisert@itp.phys.ethz.ch}}}
\hypersetup{pdfauthor={Niklas Beisert}}
\hypersetup{pdfsubject={Manual for the LaTeX2e Package childdoc}}
\date{30 December 2018, \textsf{v2.0}}
\maketitle

\begin{abstract}\noindent
\textsf{childdoc} is a \LaTeXe{} package
that enables the direct compilation
of document sections included by |\include|
to individual files.
\end{abstract}

\begingroup
\parskip0ex
\tableofcontents
\endgroup

%%%%%%%%%%%%%%%%%%%%%%%%%%%%%%%%%%%%%%%%%%%%%%%%%%%%%%%%%%%%%%%%%%%%%%%%%%%%%%%%
%%%%%%%%%%%%%%%%%%%%%%%%%%%%%%%%%%%%%%%%%%%%%%%%%%%%%%%%%%%%%%%%%%%%%%%%%%%%%%%%
\section{Introduction}

\LaTeX{} provides a mechanism to structure a large document (such as a book)
into a main file and several child files (containing the chapters)
using the |\include| command.
This mechanism is beneficial for documents
which span hundreds of pages in order to
make the source file(s) more manageable.
Moreover, compilation can be restricted to
selected child files by means of the |\includeonly| command.
The latter feature can be used to reduce the compilation time while editing
(this was significantly more useful in the earlier days of \LaTeX{})
or to generate a smaller document which is easier to navigate.
Another application of |\includeonly| is to generate
documents consisting of selected parts of the complete document.

However, there are a few drawbacks of the plain |\include| mechanism:
\begin{itemize}
\item
The child files cannot be compiled on their own,
they can only be compiled via the main file.
A naive editing environment
(such as a text editor with an option
to have the current file processed by \LaTeX)
may require one to switch to the main file before compiling;
attempting to compile the child file produces errors.
\item
The main file must be modified (each time)
to adjust the |\includeonly| command
to the present needs. This easily leaves the main file in a messy state.
\item
The generated document will always carry the filename
of the main document. This is inconvenient if
several child files are to be compiled and
to be kept for distribution.
\end{itemize}

The present package provides a simple interface
to make child files individually compilable by \LaTeX{}.
Compiling a child file then has the same effect as compiling
the main file with an |\includeonly| command
to select the appropriate child.
Moreover the generated document will carry the name of the child
rather than the main file.
This resolves all three above issues.

This feature is meant to make the editing of books,
thesis documents and lecture notes somewhat more convenient.
However, the package can also be used efficiently for
composing a series of documents (such as exercise sheets)
which are typically distributed individually.
It then assists the author in generating the individual documents
(potentially in different versions)
as well as a document containing the collected series.
Another application is in developing style files
or other kinds of included material
where compilation of the style file could redirect
to a sample or test file.

%%%%%%%%%%%%%%%%%%%%%%%%%%%%%%%%%%%%%%%%%%%%%%%%%%%%%%%%%%%%%%%%%%%%%%%%%%%%%%%%
%%%%%%%%%%%%%%%%%%%%%%%%%%%%%%%%%%%%%%%%%%%%%%%%%%%%%%%%%%%%%%%%%%%%%%%%%%%%%%%%
\section{Usage}

First of all, the package \textsf{childdoc} is \emph{not} a standard
\LaTeXe{} |.sty| style file! Therefore it needs to be invoked in
a non-standard way.

%%%%%%%%%%%%%%%%%%%%%%%%%%%%%%%%%%%%%%%%%%%%%%%%%%%%%%%%%%%%%%%%%%%%%%%%%%%%%%%%
\subsection{Included Files}
\label{sec:include}

%%%%%%%%%%%%%%%%%%%%%%%%%%%%%%%%%%%%%%%%
\DescribeMacro{\childdocmain}
To use the package, add the commands
\begin{center}
\begin{tabular}{l}
|\input{childdoc.def}|\\
|\childdocmain{}|\\
\end{tabular}
\end{center}
at the very top of the main \LaTeX{} file,
in particular \emph{before} the |\documentclass| statement!
The argument of |\childdocmain| should be left empty
(but it must be present).

%%%%%%%%%%%%%%%%%%%%%%%%%%%%%%%%%%%%%%%%
\DescribeMacro{\childdocof}
Furthermore, add the commands
\begin{center}
\begin{tabular}{l}
|\input{childdoc.def}|\\
|\childdocof{|\textit{main}|}|\\
\end{tabular}
\end{center}
at the top of every child file \textit{child}
which is included by |\include{|\textit{child}|}|
from within the main file
(or at least for those files to be compiled individually).
The argument \textit{main} must be the filename of the main file.

There are a couple of
considerations in setting up the main and child documents:

%%%%%%%%%%%%%%%%%%%%%%%%%%%%%%%%%%%%%%%%
\paragraph{Restrictions.}

Please note the following restrictions:
\begin{itemize}
\item
|\childdocmain| must be called with one argument \textit{main}
to ensure compatibility with earlier version of the package.
It must either be empty (|\childdocmain{}|)
or precisely match the filename of the main file in which it is specified.
See \secref{sec:detection} for further information.
\item
The filename \textit{main} must be specified without the |.tex| extension.
\item
The filename \textit{main} is case sensitive
(even in case-insensitive file systems)
due to internal string comparison.
\item
The argument \textit{main} should be fully expanded, it cannot be a macro.
\item
Subdirectories and special characters should be avoided in filenames.
\item
The command |\childdocmain{|\textit{main}|}| must be followed by a whitespace.
It should not be followed immediately by another command
or by a comment mark `|%|'.
This is because the \TeX{} parser reads the token immediately following
the argument of |\childdocmain| and puts it
at the beginning of every child section;
however, a white\-space is ignored.
\end{itemize}

%%%%%%%%%%%%%%%%%%%%%%%%%%%%%%%%%%%%%%%%
\paragraph{Content of Main File.}

It is advisable to place all content in the child files included by |\include|.
Any output contained in the main file will appear in all child documents
unless suppressed manually;
it cannot be suppressed automatically by the |\includeonly| directive
and thus should normally be avoided.
A method to include some content in the main file
by means of conditional processing is described in \secref{sec:conditional}.

%%%%%%%%%%%%%%%%%%%%%%%%%%%%%%%%%%%%%%%%
\paragraph{Page Numbering.}

When only a part of the document is compiled,
the appropriate numbering of pages
(as well as other status parameters)
is determined from the |.aux| files.
The latter contain information from previous passes.
However this information needs to propagate through
all intermediate child documents.
Therefore the page numbering in child documents may well
be inconsistent until the complete document is compiled at least once.

A useful (if unconventional) way to always ensure a consistent
page numbering is to restart the numbering in each child document
and denote the pages by `\textit{child}|.|\textit{page}'
where \textit{child} represents the chapter/section number of the child file.
This can be achieved by the command
|\numberwithin{page}{|\textit{child}|}|
of the \textsf{amsmath} package
where \textit{child} can be |chapter| or |section|
depending on the chosen structuring.
Alternatively, one can modify the macro |\thepage| appropriately
and reset the counter |page| at the start of each child file.

%%%%%%%%%%%%%%%%%%%%%%%%%%%%%%%%%%%%%%%%%%%%%%%%%%%%%%%%%%%%%%%%%%%%%%%%%%%%%%%%
\subsection{Conditional Processing}
\label{sec:conditional}

The package provides a mechanism to compile different versions
of a document. To customise the versions further some conditional processing
can come in handy to distinguish which version is being compiled.
The package provides two macros to describe the compilation context:

%%%%%%%%%%%%%%%%%%%%%%%%%%%%%%%%%%%%%%%%
\DescribeMacro{\ifchilddoc}
The conditional |\ifchilddoc| distinguishes between the compilation of
child documents and the main document:
%
\begin{center}
|\ifchilddoc |\textit{child-code}| |[|\||else |\textit{main-code}]| \||fi|
\end{center}

%%%%%%%%%%%%%%%%%%%%%%%%%%%%%%%%%%%%%%%%
\DescribeMacro{\childdocname}
\DescribeMacro{\childdocjob}
The macro |\childdocname| contains the filename (without extension)
of the main or child file being processed.
Note that |\childdocjob| will always contain the name of the main file.

%%%%%%%%%%%%%%%%%%%%%%%%%%%%%%%%%%%%%%%%
\paragraph{Title Page.}

Conditional processing can be used to include a title or banner page
in the main document when proper precautions are taken.
Importantly, the code in the main file should ensure that the page counter
(as well as other status parameters which are stored in the |.aux| files)
takes the same value after the conditional processing.
Otherwise the page numbers may take divergent values
depending on which part is compiled.

For example, a title page could be declared by:
%
\begin{center}
\begin{tabular}{l}
|\ifchilddoc\||else|\\
|\addtocounter{page}{-1}|\\
\textit{code for title page}\\
|\newpage|\\
|\||fi|
\end{tabular}
\end{center}
%
A banner page for the child documents can be generated by:
%
\begin{center}
\begin{tabular}{l}
|\ifchilddoc|\\
|\addtocounter{page}{-1}|\\
\textit{code for banner page}\\
|\newpage|\\
|\||fi|
\end{tabular}
\end{center}
%
Here one could write a message such as:
\begin{center}
|This is the part \childdocname{} of \childdocjob{}.|
\end{center}

%%%%%%%%%%%%%%%%%%%%%%%%%%%%%%%%%%%%%%%%%%%%%%%%%%%%%%%%%%%%%%%%%%%%%%%%%%%%%%%%
\subsection{Flags}
\label{sec:flags}

The package makes it easy to generate different versions
of the main or child documents.
To this end compilation flags can be defined
and assigned different default values.
They will be particularly useful in conjunction
with the forwarding mechanism described in \secref{sec:forward}.

For example, it may be useful to have a flag |\version|
which can be set to |draft| or |final|.
The document source will contain some conditional code
depending on the value of |\version|.
Suppose further, the flag should default to |final| for the main file
and to |draft| for child files
which is a natural assignment for editing the document.
This is achieved by placing the following code
in the preamble of the main document
(below the |\childdocmain| directive):
%
\begin{center}
\begin{tabular}{l}
|\ifchilddoc|\\
|\providecommand{\version}{draft}|\\
|\||else|\\
|\providecommand{\version}{final}|\\
|\||fi|
\end{tabular}
\end{center}
%
The definition by |\providecommand| makes sure
that previous definitions are not overwritten.
Further statements |\providecommand{\version}{...}|
can thus be added before the above code to override it.

For the main file, one might add a line
(between |\childdocmain| and the above block)
%
\begin{center}
|%\ifchilddoc\||else\providecommand{\version}{draft}\||fi|
\end{center}
%
which can be uncommented to produce a draft version.
Likewise one can add a line to the very top of a child file
(above the |\childdocof{|\textit{main}|}| directive)
%
\begin{center}
|%\providecommand{\version}{final}|
\end{center}
%
which can be uncommented to produce the final version of this child document.

%%%%%%%%%%%%%%%%%%%%%%%%%%%%%%%%%%%%%%%%%%%%%%%%%%%%%%%%%%%%%%%%%%%%%%%%%%%%%%%%
\subsection{Forwarding}
\label{sec:forward}

Different versions of the main or child documents
using compilation flags as described in \secref{sec:flags}
can be (permanently) stored in different files
for convenient compilation, viewing and distribution.
To this end, the package defines a command
to pass on compilation to a different file:

%%%%%%%%%%%%%%%%%%%%%%%%%%%%%%%%%%%%%%%%
\DescribeMacro{\childdocforward}
The command |\childdocforward| redirects processing to
another source file:
%
\begin{center}
\begin{tabular}{l}
|\input{childdoc.def}|\\
|\childdocforward[|\textit{main}|]{|\textit{dest}|}|\\
\end{tabular}
\end{center}
%
The argument \textit{dest} is the destination file
(without extension).
It should be the main file or one of the child files.
Note that further \textsf{childdoc} directives
such as |\childdocof| and |\childdocforward|
in the indicated file will be processed in this form.
The optional argument \textit{main}
passes on directly to the main file \textit{main}
while pretending to compile the child \textit{dest}.
This form behaves as if \textit{dest}
issues |\childdocof{|\textit{main}|}| right away,
and no further \textsf{childdoc} directives will be processed.

%%%%%%%%%%%%%%%%%%%%%%%%%%%%%%%%%%%%%%%%
\DescribeMacro{\...prefix}
In the alternative form |\childdocforwardprefix|,
%
\begin{center}
\begin{tabular}{l}
|\input{childdoc.def}|\\
|\childdocforwardprefix[|\textit{main}|]{|\textit{prefix}|}{|\textit{dest}|}|
\end{tabular}
\end{center}
%
the destination file is determined by a pattern
depending on the current file:
To make this work, the current file must be called
`{\textit{prefix}\hspace{0.2em}\textit{suffix}}'
with \textit{prefix} matching precisely the argument.
Processing is then passed on to the file
`{\textit{dest}\hspace{0.2em}\textit{suffix}}'.
Surely, the same effect is achieved by
directly specifying the
argument `{\textit{dest}\hspace{0.2em}\textit{suffix}}'
in the first form.
However, that requires to set up a different file
for each child. With the alternative form of the command
all these files can have exactly the same content
which simplifies setting them up and maintaining them.

For example, the following file |draft.tex|
with a compilation flag |\version| as described in \secref{sec:flags}
compiles the main document as a draft:
%
\begin{center}
\begin{tabular}{l}
|\def\version{draft}|\\
|\input{childdoc.def}|\\
|\childdocforward{|\textit{main}|}|
\end{tabular}
\end{center}
%
Likewise, the following files |final|\textit{nn}|.tex|
compile the final version of the child document
|child|\textit{nn}|.tex|:
%
\begin{center}
\begin{tabular}{l}
|\def\version{final}|\\
|\input{childdoc.def}|\\
|\childdocforwardprefix{final}{child}|
\end{tabular}
\end{center}
%

Note that when several versions of a main file and/or of each child file
are to be generated, it may be convenient to set up a |Makefile| or
shell script to automatise the process.

%%%%%%%%%%%%%%%%%%%%%%%%%%%%%%%%%%%%%%%%%%%%%%%%%%%%%%%%%%%%%%%%%%%%%%%%%%%%%%%%
\subsection{Command Line Processing}
\label{sec:commandline}

The effect of redirection files can also be achieved by invoking
the \LaTeX{} compiler with a more elaborate command line.
Most conveniently this should be done as part
of a shell script or a |Makefile|.

When using \textsf{childdoc} in the main file, the following
command lines effectively perform a redirection
(note that depending on the shell being used,
backslashes may have to be doubled: `|\|' $\to$ `|\\|'):
%
\begin{center}
|... -jobname "|\textit{target}|" |\\|"|[\textit{flags}]%
|\input{childdoc.def}\childdocforward[|\textit{main}|]{|\textit{dest}|}"|
\end{center}
%
Here \textit{target} is the name of the output file,
\textit{main} is the name of the main file
and \textit{dest} is the name of the main or child file to be processed
(all filenames without extensions).
The optional argument \textit{main} can be omitted
if \textit{main} matches \textit{dest}.
Optionally, compilation \textit{flags} can be defined via |\def| commands.
This command line makes the \TeX{} engine believe
it is compiling the file \textit{target}
whose content is specified as the latter parameter.
The provided code then forwards the processing to
\textit{main} or \textit{dest} as described in \secref{sec:forward}.

%%%%%%%%%%%%%%%%%%%%%%%%%%%%%%%%%%%%%%%%%%%%%%%%%%%%%%%%%%%%%%%%%%%%%%%%%%%%%%%%
\subsection{Include by Input}
\label{sec:input}

Including child documents by |\include| has some restrictions by design.
Most notably, the content of a child document always occupies
its own set of pages; pages cannot be shared between child documents.
Usually, this behaviour makes perfect sense
because each child document contain an essential part of the document.
However, in some situations it may be desirable to compose
a document from a collection of parts
without having mandatory page breaks between then.
For this case, the package
provides a mechanism to include parts
by |\input| which can also be processed individually.
However, by construction this mechanism
requires manual handling of the content to be output.

%%%%%%%%%%%%%%%%%%%%%%%%%%%%%%%%%%%%%%%%
\DescribeMacro{\ifchilddocmanual}
The main file should be prepared as usual, see \secref{sec:include}.
However, the document body must make a distinction
between processing of an individual part and of the main document, e.g.:
%
\begin{center}
\begin{tabular}{l}
|\ifchilddocmanual|\\
|\input{\childdocname}|\\
|\||else|\\
\textit{document body with }|\input{|\textit{part}|}|\\
|\||fi|
\end{tabular}
\end{center}
%
The conditional |\ifchilddocmanual| is true whenever
a part to be included by |\input| is being compiled,
and the name of the part is stored in |\childdocname|.

%%%%%%%%%%%%%%%%%%%%%%%%%%%%%%%%%%%%%%%%
\DescribeMacro{\childdocby}
Each part to be included by |\input| should start with:
%
\begin{center}
\begin{tabular}{l}
|\input{childdoc.def}|\\
|\childdocby{|\textit{main}|}|\\
\end{tabular}
\end{center}
%
The directive |\childdocby| is similar to |\childdocof|
described in \secref{sec:include},
but the subsequent selection of content must be done manually.
To that end, both |\ifchilddoc| and |\ifchilddocmanual|
will be true upon processing of a part,
and the name of the part is stored in |\childdocname|.
Note that |\jobname| will be set to the filename of the current part
so that each part receives an individual |.aux| file
that does not interfere with the |.aux| file(s) of the main document.
This behaviour can be altered by the alternative form
|\childdocby[*]{|\textit{main}|}| (with a non-empty optional argument)
which uses the |.aux| file of the main document
by setting |\jobname| to \textit{main}.

%%%%%%%%%%%%%%%%%%%%%%%%%%%%%%%%%%%%%%%%%%%%%%%%%%%%%%%%%%%%%%%%%%%%%%%%%%%%%%%%
\subsection{Driver Development}
\label{sec:driver}

The \textsf{childdoc} mechanism can also be use for the development
of definition files such as \LaTeX{} styles or classes.
This case differs from the above setup with multiple parts
included by |\include| in that no |\includeonly| should be invoked.
This can be achieved by starting the include file
(before |\ProvidesPackage|) with:
%
\begin{center}
\begin{tabular}{l}
|\input{childdoc.def}|\\
|\childdocforward{|\textit{main}|}|\\
\end{tabular}
\end{center}
%
or alternatively with:
%
\begin{center}
\begin{tabular}{l}
|\input{childdoc.def}|\\
|\childdocby{|\textit{main}|}|\\
\end{tabular}
\end{center}
%
Both forms have slightly different effects as described above.
The main file is prepared as usual, see \secref{sec:include}.

%%%%%%%%%%%%%%%%%%%%%%%%%%%%%%%%%%%%%%%%%%%%%%%%%%%%%%%%%%%%%%%%%%%%%%%%%%%%%%%%
\subsection{Legacy Detection}
\label{sec:detection}

The directive |\childdocmain| in the main file can detect
whether the complete document or merely a child is to be compiled
even without using the directive |\childdocof|.
This method is deprecated because it is less robust
and there is no compelling reason to use it;
it is merely provided for backward compatibility
and it may be removed in future versions.

If the detection mechanism is to be used,
it is mandatory to correctly specify
the filename of the main file as the argument of |\childdocmain|:
%
\begin{center}
\begin{tabular}{l}
|\input{childdoc.def}|\\
|\childdocmain{|\textit{main}|}|\\
\end{tabular}
\end{center}
%
If |\jobname| does not match the argument \textit{main} of |\childdocmain|,
it is assumed that |\jobname| points to the child file to be compiled.
When using |\childdocmain| with the main file specified as argument,
it suffices to start a child file
with just |\input{|\textit{main}|}|
without loading of the package and using |\childdocof|.
If instead all processing is done
with the appropriate \textsf{childdoc} directives,
the argument of \textit{main} of |\childdocmain| can be empty.

An alternative version of the command line processing described
in \secref{sec:commandline} using the detection mechanism reads:
%
\begin{center}
|... -jobname "|\textit{target}|" "|[\textit{flags}]%
[|\def\jobname{|\textit{dest}|}|]|\input{|\textit{main}|}"|
\end{center}

%%%%%%%%%%%%%%%%%%%%%%%%%%%%%%%%%%%%%%%%%%%%%%%%%%%%%%%%%%%%%%%%%%%%%%%%%%%%%%%%
\subsection{Manual Code}
\label{sec:manual}

In case one cannot be certain whether the definitions file |childdoc.def|
is installed on the target \TeX{} distribution
and one prefers not to ship it,
it is conceivable to paste a few relevant commands into the sources.

To that end, drop all statements |\input{childdoc.def}|
and perform the replacements as outlined below.
Instead of |\childdocmain{|\textit{main}|}| add the following code
to the top of the main file:
%
\begin{center}
\begin{tabular}{l}
|\||ifdefined\childdocname\endinput\||fi\newif\ifchilddoc|\\
|\edef\childdocname{\scantokens\expandafter{\jobname\noexpand}}|\\
|\def\childdocmain{|\textit{main}|}\||ifx\childdocmain\childdocname\||else|\\
|\childdoctrue\includeonly{\childdocname}\let\jobname\childdocmain\||fi|\\
\end{tabular}
\end{center}
%
Instead of |\childdocof{|\textit{main}|}| just include the main file
at the top of each child file:
%
\begin{center}
|\input{|\textit{main}|}|
\end{center}
%
A simple redirection |\childdocforward{|\textit{dest}|}| is achieved by:
%
\begin{center}
|\def\jobname{|\textit{dest}|}\input{\jobname}|
\end{center}
%
The redirection with prefix
|\childdocforwardprefix[|\textit{prefix}|]{|\textit{dest}|}|
is accomplished by:
%
\begin{center}
\begin{tabular}{l}
|{\edef\jobname{\scantokens\expandafter{\jobname\noexpand}}|\\
|\def\redirectjob |\textit{prefix}|#1~~~{\gdef\jobname{|\textit{dest}|#1}}|\\
|\expandafter\redirectjob\jobname~~~}\input{\jobname}|
\end{tabular}
\end{center}

In an alternative approach,
child documents can be compiled by a specific command line
without additional code or specific definitions:
%
\begin{center}
|... -jobname "|\textit{target}|" "|[\textit{flags}]%
|\includeonly{|\textit{dest}|}\input{|\textit{main}|}"|
\end{center}
%

%%%%%%%%%%%%%%%%%%%%%%%%%%%%%%%%%%%%%%%%%%%%%%%%%%%%%%%%%%%%%%%%%%%%%%%%%%%%%%%%
%%%%%%%%%%%%%%%%%%%%%%%%%%%%%%%%%%%%%%%%%%%%%%%%%%%%%%%%%%%%%%%%%%%%%%%%%%%%%%%%
\section{Information}

%%%%%%%%%%%%%%%%%%%%%%%%%%%%%%%%%%%%%%%%%%%%%%%%%%%%%%%%%%%%%%%%%%%%%%%%%%%%%%%%
\subsection{Copyright}

Copyright \copyright{} 2017--2018 Niklas Beisert

This work may be distributed and/or modified under the
conditions of the \LaTeX{} Project Public License, either version 1.3
of this license or (at your option) any later version.
The latest version of this license is in
  \url{http://www.latex-project.org/lppl.txt}
and version 1.3 or later is part of all distributions of \LaTeX{}
version 2005/12/01 or later.

This work has the LPPL maintenance status `maintained'.

The Current Maintainer of this work is Niklas Beisert.

This work consists of the files |README.txt|, |childdoc.ins| and |childdoc.dtx|
as well as the derived files |childdoc.def|, |cdocsamp.tex|
with |cdocsch1.tex|, |cdocsch2.tex|, |cdocspt3.tex|, |cdocspt4.tex|,
|cdocsdrf.tex|, |cdocsfn1.tex|, |cdocsfn2.tex|
as well as |childdoc.pdf|.

%%%%%%%%%%%%%%%%%%%%%%%%%%%%%%%%%%%%%%%%%%%%%%%%%%%%%%%%%%%%%%%%%%%%%%%%%%%%%%%%
\subsection{Files and Installation}

The package consists of the files:
%
\begin{center}
\begin{tabular}{ll}
    |README.txt|   & readme file \\
    |childdoc.ins| & installation file \\
    |childdoc.dtx| & source file \\
    |childdoc.def| & definition file \\
    |cdocsamp.tex| & sample main file \\
    |cdocsch1.tex| & sample include file \\
    |cdocsch2.tex| & sample include file \\
    |cdocspt3.tex| & sample part file \\
    |cdocspt4.tex| & sample part file \\
    |cdocsdrf.tex| & sample redirection file \\
    |cdocsfn1.tex| & sample redirection file \\
    |cdocsfn2.tex| & sample redirection file \\
    |childdoc.pdf| & manual
\end{tabular}
\end{center}
%
The distribution consists of the files
|README.txt|, |childdoc.ins| and |childdoc.dtx|.
%
\begin{itemize}
\item
Run (pdf)\LaTeX{} on |childdoc.dtx|
to compile the manual |childdoc.pdf| (this file).
\item
Run \LaTeX{} on |childdoc.ins| to create the definitions file |childdoc.def|
and the sample |cdocsamp.tex| with include files
|cdocsch1.tex|, |cdocsch2.tex|, |cdocspt3.tex|, |cdocspt4.tex|,
|cdocsdrf.tex|, |cdocsfn1.tex|, |cdocsfn2.tex|.
Then copy the file |childdoc.def| to an appropriate directory of your \LaTeX{}
distribution, e.g.\ \textit{texmf-root}|/tex/latex/childdoc|.
\end{itemize}

%%%%%%%%%%%%%%%%%%%%%%%%%%%%%%%%%%%%%%%%%%%%%%%%%%%%%%%%%%%%%%%%%%%%%%%%%%%%%%%%
\subsection{Related CTAN Packages}

There are several other packages which offer a similar functionality:
%
\begin{itemize}
\item
The packages
\href{http://ctan.org/pkg/docmute}{\textsf{docmute}},
\href{http://ctan.org/pkg/includex}{\textsf{includex}} and
\href{http://ctan.org/pkg/standalone}{\textsf{standalone}}
provide commands to include only the document body of
a child file thus allowing both files to be compiled individually.
\item
The packages \href{http://ctan.org/pkg/subdocs}{\textsf{subdocs}}
and \href{http://ctan.org/pkg/subfiles}{\textsf{subfiles}}
provide structures in which the main and child documents can be
encapsulated and allowing them to be compiled individually.
The inclusion mechanism is different from the conventional |\include|.
\item
The package \href{http://ctan.org/pkg/combine}{\textsf{combine}}
is an elaborate solution to combine several documents into one.
\end{itemize}
%
See also the CTAN topic \href{http://ctan.org/topic/subdocs}{\textsf{subdocs}}
for further related packages.
The present package differs from the above solutions in that
a document structure constructed with the conventional |\include| mechanism
just needs two extra commands at the top of every file
such that all constituent files can be compiled individually.

%%%%%%%%%%%%%%%%%%%%%%%%%%%%%%%%%%%%%%%%%%%%%%%%%%%%%%%%%%%%%%%%%%%%%%%%%%%%%%%%
%\subsection{Feature Suggestions}
%
%The following is a list of features which may be useful for future
%versions of this package:
%%
%\begin{itemize}
%\item
%\ldots
%\end{itemize}

%%%%%%%%%%%%%%%%%%%%%%%%%%%%%%%%%%%%%%%%%%%%%%%%%%%%%%%%%%%%%%%%%%%%%%%%%%%%%%%%
\subsection{Revision History}

%%%%%%%%%%%%%%%%%%%%%%%%%%%%%%%%%%%%%%%%
\paragraph{v2.0:} 2018/12/30

\begin{itemize}
\item
immediate forward processing
\item
added |\childdocby| mechanism
\item
manual restructured
\end{itemize}

%%%%%%%%%%%%%%%%%%%%%%%%%%%%%%%%%%%%%%%%
\paragraph{v1.6:} 2018/01/17

\begin{itemize}
\item
application for development of include files
\item
corrections to manual
\end{itemize}

%%%%%%%%%%%%%%%%%%%%%%%%%%%%%%%%%%%%%%%%
\paragraph{v1.5:} 2017/05/21

\begin{itemize}
\item
more complete structuring introduced
\item
|\childdocof| introduced
\item
|\childdoc| renamed to |\childdocmain|
\item
|\childredirect| renamed to |\childdocforward| and |\childdocforwardprefix|
and functionality expanded
\end{itemize}

%%%%%%%%%%%%%%%%%%%%%%%%%%%%%%%%%%%%%%%%
\paragraph{v1.0:} 2017/04/27

\begin{itemize}
\item
manual and install package
\item
first version published on CTAN
\end{itemize}

%%%%%%%%%%%%%%%%%%%%%%%%%%%%%%%%%%%%%%%%
\paragraph{v0.6:} 2017/04/26

\begin{itemize}
\item
redirection mechanism added
\end{itemize}

%%%%%%%%%%%%%%%%%%%%%%%%%%%%%%%%%%%%%%%%
\paragraph{v0.5:} 2017/04/26

\begin{itemize}
\item
functionality in definition file
\end{itemize}


%%%%%%%%%%%%%%%%%%%%%%%%%%%%%%%%%%%%%%%%%%%%%%%%%%%%%%%%%%%%%%%%%%%%%%%%%%%%%%%%
%%%%%%%%%%%%%%%%%%%%%%%%%%%%%%%%%%%%%%%%%%%%%%%%%%%%%%%%%%%%%%%%%%%%%%%%%%%%%%%%
%%%%%%%%%%%%%%%%%%%%%%%%%%%%%%%%%%%%%%%%%%%%%%%%%%%%%%%%%%%%%%%%%%%%%%%%%%%%%%%%
\appendix

\settowidth\MacroIndent{\rmfamily\scriptsize 000\ }

 \DocInput{childdoc.dtx}

\end{document}
%</driver>
% \fi
%
% %%%%%%%%%%%%%%%%%%%%%%%%%%%%%%%%%%%%%%%%%%%%%%%%%%%%%%%%%%%%%%%%%%%%%%%%%%%%%%
% %%%%%%%%%%%%%%%%%%%%%%%%%%%%%%%%%%%%%%%%%%%%%%%%%%%%%%%%%%%%%%%%%%%%%%%%%%%%%%
% \section{Sample}
%\iffalse
%<*samplemain>
%\fi
%
% The following presents a sample document
% with two chapters, two parts, a title page,
% a compile flag as well as three forwarding files to set the flag.
% It consists of eight |.tex| files:
% \begin{center}
% \begin{tabular}{ll}
% |cdocsamp.tex|&main file\\
% |cdocsch1.tex|&include file for chapter 1\\
% |cdocsch2.tex|&include file for chapter 2\\
% |cdocspt3.tex|&include file for part 3\\
% |cdocspt4.tex|&include file for part 4\\
% |cdocsdrf.tex|&forwarding file for main file in draft mode\\
% |cdocsfi1.tex|&forwarding file for final version of chapter 1\\
% |cdocsfi2.tex|&forwarding file for final version of chapter 2\\
% \end{tabular}
% \end{center}
% Each of the eight files can be compiled directly by the \LaTeX{} compiler.
%
% %%%%%%%%%%%%%%%%%%%%%%%%%%%%%%%%%%%%%%
% \paragraph{Main File.}
%
% The main file is called |cdocsamp.tex|.
%
% Load the \textsf{childdoc} definitions and
% declare the filename for the main document:
%    \begin{macrocode}
\input{childdoc.def}
\childdocmain{}
%    \end{macrocode}

% Optional override for |\version| flag:
%    \begin{macrocode}
%%\ifchilddoc\else\providecommand{\version}{draft}\fi
%    \end{macrocode}

% Define the default values for the |\version| flag
% (|final| for the main file and |draft| for childs):
%    \begin{macrocode}
\ifchilddoc
\providecommand{\version}{draft}
\else
\providecommand{\version}{final}
\fi
%    \end{macrocode}

% Load the standard document class:
%    \begin{macrocode}
\documentclass[12pt]{article}
%    \end{macrocode}

% Start the document body:
%    \begin{macrocode}
\begin{document}
%    \end{macrocode}

% Declare a title page.
% Print title, part of document being processed and version flag:
%    \begin{macrocode}
\addtocounter{page}{-1}
\begin{center}
{\LARGE\bfseries{}childdoc example\par}
\vspace{1cm}
\ifchilddoc
\ifchilddocmanual part\else chapter\fi:
`\childdocname' of `\childdocjob'\par
\else
main document: `\childdocjob'\par
\fi
version: \version\par
\end{center}
\newpage
%    \end{macrocode}

% Manually include selected file,
% otherwise process as usual:
%    \begin{macrocode}
\ifchilddocmanual
\section*{part `\childdocname'}
\input{\childdocname}
\else
%    \end{macrocode}

% Include the two chapters:
%    \begin{macrocode}
\include{cdocsch1}
\include{cdocsch2}
%    \end{macrocode}

% Include the two parts unless only chapters should be displayed:
%    \begin{macrocode}
\ifchilddoc\else
\section{part three}
\input{cdocspt3}
\section{part four}
\input{cdocspt4}
\fi
%    \end{macrocode}

% Process as usual until here:
%    \begin{macrocode}
\fi
%    \end{macrocode}

% End of document body:
%    \begin{macrocode}
\end{document}
%    \end{macrocode}
%\iffalse
%</samplemain>
%\fi
%
% %%%%%%%%%%%%%%%%%%%%%%%%%%%%%%%%%%%%%%
% \paragraph{Chapter Include Files.}
%
% The include files are called |cdocsch1.tex| and |cdocsch2.tex|.
%
%\iffalse
%<*samplechap1|samplechap2>
%\fi

% Optional override for |\version| flag:
%    \begin{macrocode}
%%\providecommand{\version}{final}
%    \end{macrocode}

% Include the main document:
%    \begin{macrocode}
\input{childdoc.def}
\childdocof{cdocsamp}
%    \end{macrocode}

%\iffalse
%</samplechap1|samplechap2>
%\fi
%
%\iffalse
%<*samplechap1>
%\fi
% Some text for chapter 1:
%    \begin{macrocode}
\section{one}
some text in chapter one
%    \end{macrocode}

%\iffalse
%</samplechap1>
%\fi
% Some text for chapter 2:
%\iffalse
%<*samplechap2>
%\fi
%    \begin{macrocode}
\section{two}
more text in chapter two
%    \end{macrocode}

%\iffalse
%</samplechap2>
%\fi
%
% %%%%%%%%%%%%%%%%%%%%%%%%%%%%%%%%%%%%%%
% \paragraph{Part Include Files.}
%
% The include files are called |cdocspt3.tex| and |cdocspt4.tex|.
%
%\iffalse
%<*samplepart3|samplepart4>
%\fi

% Optional override for |\version| flag:
%    \begin{macrocode}
%%\providecommand{\version}{final}
%    \end{macrocode}

% Include the main document:
%    \begin{macrocode}
\input{childdoc.def}
\childdocby{cdocsamp}
%    \end{macrocode}

%\iffalse
%</samplepart3|samplepart4>
%\fi
%
%\iffalse
%<*samplepart3>
%\fi
% Some text for part 3:
%    \begin{macrocode}
some text in part three
%    \end{macrocode}

%\iffalse
%</samplepart3>
%\fi
% Some text for part 4:
%\iffalse
%<*samplepart4>
%\fi
%    \begin{macrocode}
more text in part four
%    \end{macrocode}

%\iffalse
%</samplepart4>
%\fi
%
% %%%%%%%%%%%%%%%%%%%%%%%%%%%%%%%%%%%%%%
% \paragraph{Forwarding for a Complete Draft.}
%
% The following forwarding file |cdocsdrf.tex|
% compiles the main document in draft mode:
%\iffalse
%<*sampledraft>
%\fi
%    \begin{macrocode}
\def\version{draft}
\input{childdoc.def}
\childdocforward{cdocsamp}
%    \end{macrocode}

%\iffalse
%</sampledraft>
%\fi
%
% %%%%%%%%%%%%%%%%%%%%%%%%%%%%%%%%%%%%%%
% \paragraph{Forwarding for Final Version of the Chapters.}
%
% The following forwarding files |cdocsfn1.tex| and |cdocsfn2.tex|
% (with identical content)
% compile the final versions of the child documents
% |cdocsch1.tex| and |cdocsch2.tex|, respectively:
%\iffalse
%<*samplefinal>
%\fi
%    \begin{macrocode}
\def\version{final}
\input{childdoc.def}
\childdocforwardprefix[cdocsamp]{cdocsfn}{cdocsch}
%    \end{macrocode}

%\iffalse
%</samplefinal>
%\fi
%
% %%%%%%%%%%%%%%%%%%%%%%%%%%%%%%%%%%%%%%
% \paragraph{Command Line Processing.}
%
% The following three command lines generate the output files
% |cdocscld|, |cdocscl1| and |cdocscl2|
% which should be identical to
% |cdocsdrf|, |cdocsch1| and |cdocsfn2|, respectively:
% \begin{center}
% \begin{tabular}{l}
% |latex -jobname cdocscld \|\\
% |  "\def\version{draft}\input{childdoc.def}\childdocforward{cdocsamp}"|\\
% |latex -jobname cdocscl1 \|\\
% |  "\input{childdoc.def}\childdocforward[cdocsamp]{cdocsch1}"|\\
% |latex -jobname cdocscl2 \|\\
% |  "\def\version{final}\input{childdoc.def}\childdocforward{cdocsch2}"|
% \end{tabular}
% \end{center}
% Note that the trailing backslash on each first line
% merely continues the input to the second line
% (for convenient cut ant paste).
% Furthermore, the command |latex| can be replaced by any
% of its alternative versions such as |pdflatex|.
%
% %%%%%%%%%%%%%%%%%%%%%%%%%%%%%%%%%%%%%%%%%%%%%%%%%%%%%%%%%%%%%%%%%%%%%%%%%%%%%%
% %%%%%%%%%%%%%%%%%%%%%%%%%%%%%%%%%%%%%%%%%%%%%%%%%%%%%%%%%%%%%%%%%%%%%%%%%%%%%%
% \section{Implementation}
%\iffalse
%<*package>
%\fi
%
% This section describes the definitions file |childdoc.def|.

% The definitions cannot be loaded using |\usepackage| or |\RequirePackage|
% which has a mechanism to prevent loading a style file more than once.
% When loading the definitions by means of |\input|
% multiple instances have to be prevented manually:
%\iffalse
%This code needs to be before the `\ProvidesFile' directive
%which is defined at the beginning of this file.
%Therefore it is also placed there and commented out here.
%</package>
%<*discard>
%\fi
%    \begin{macrocode}
\ifdefined\childdocmain\endinput\fi
%    \end{macrocode}
%\iffalse
%</discard>
%<*package>
%\fi
%
% \macro{\ifchilddoc}
% \macro{\ifchilddocmanual}
% The conditional |\ifchilddoc| tells whether a
% child (true) or main (false) document is being compiled.
% The conditional |\ifchilddocmanual| tells whether
% the |\includeonly| mechanism is used (false) or
% the selection of child files must be performed manually (true).
% The definitions initialise to false:
%    \begin{macrocode}
\newif\ifchilddoc
\newif\ifchilddocmanual
%    \end{macrocode}

% \macro{\childdocname}
% \macro{\childdocjob}
% The macro |\childdocname| stores the name of the main document
% to be compiled. The macro |\childdocjob| stores the name of
% the document on which the \LaTeX{} compiler was originally invoked.
% The content of |\jobname| cannot be compared
% to filenames specified in the source due to different catcodes.
% The following code rescans |\jobname|, stores the result
% in |\childdocname| and saves a copy in |\childdocjob|:
%    \begin{macrocode}
\edef\childdocname{\scantokens\expandafter{\jobname\noexpand}}
\let\childdocjob\childdocname
%    \end{macrocode}

% \macro{\childdocdisable}
% The macro |\childdocdisable| prevents the main file
% from being processed more than once.
% At this stage, the main document command |\childdocmain|
% is assumed to be called once again where it should do nothing.
% Any subsequent call to it should prevent
% a secondary processing of the main document
% It overwrites the forwarding commands
% |\childdocof| and |\childdocforward|
% with empty macros to prevent further inclusions of the main document:
%    \begin{macrocode}
\newcommand{\childdocdisable}
{
  \renewcommand{\childdocmain}[1]{\renewcommand{\childdocmain}[1]{\endinput}}
  \renewcommand{\childdocof}[1]{}
  \renewcommand{\childdocby}[2][]{}
  \renewcommand{\childdocforward}[2][]{}
  \renewcommand{\childdocdisable}{}
}
%    \end{macrocode}

% \macro{\childdocmain}
% The macro |\childdocmain| is to be called at the top of the main file
% with nothing or the main filename (without extension) as argument.
% First, it breaks loops.
% If the argument is not empty and does not match |\childdocname|
% (which is set by the first inclusion of |childdoc.def|),
% |\ifchilddoc| is set to true, |\includeonly| is applied to the child file
% and |\jobname| is set to the main file
% (for proper handling of |.aux| files):
%    \begin{macrocode}
\newcommand{\childdocmain}[1]
{
  \childdocdisable\childdocmain{}
  \if?#1?\else
    \begingroup
      \def\childdoctmp{#1}
      \ifx\childdoctmp\childdocname
        \def\childdoctmp{}
      \else
        \def\childdoctmp
        {
          \childdoctrue
          \includeonly{\childdocname}
          \def\childdocjob{#1}
          \def\jobname{#1}
        }
      \fi
      \expandafter
    \endgroup
    \childdoctmp
  \fi
}
%    \end{macrocode}

% \macro{\childdocof}
% The command |\childdocof| redirects
% compilation to the main file |#1|.
%    \begin{macrocode}
\newcommand{\childdocof}[1]
{
  \childdocdisable
  \childdoctrue
  \includeonly{\childdocname}
  \def\jobname{#1}
  \def\childdocjob{#1}
  \input{#1}
}
%    \end{macrocode}

% \macro{\childdocby}
% The command |\childdocby| ....
%    \begin{macrocode}
\newcommand{\childdocby}[2][]
{
  \childdocdisable
  \childdoctrue
  \childdocmanualtrue
  \if?#1?\else
    \def\jobname{#2}
  \fi
  \def\childdocjob{#2}
  \input{#2}
  \endinput
}
%    \end{macrocode}

% \macro{\childdocforward}
% The command |\childdocforward| redirects
% compilation to the main file or
% (if the optional argument is given) a child file.
% Parameters are set as if the main file
% or a child file starting with |\childdocof| was compiled.
% Then compilation is handed over to the main file:
%    \begin{macrocode}
\newcommand{\childdocforward}[2][]
{
  \begingroup
    \if?#1?
      \def\childdoctmp
      {
        \def\childdocname{#2}
        \def\childdocjob{#2}
        \def\jobname{#2}
        \input{#2}
        \endinput
      }
    \else
      \def\childdoctmp
      {
        \childdocdisable
        \def\childdocname{#2}
        \childdoctrue
        \includeonly{#2}
        \def\childdocjob{#1}
        \def\jobname{#1}
        \input{#1}
        \endinput
      }
    \fi
    \expandafter
  \endgroup
  \childdoctmp
}
%    \end{macrocode}

% \macro{\childdocforwardprefix}
% The command |\childdocforwardprefix| redirects
% compilation to the main or a child file by means of a pattern.
% The prefix |#1| in the current filename is replaced by |#2|
% and the suffix of the current filename is kept
% (it is assumed that the filename does not contain the substring `|~~~|'
% which is used as a delimiter).
% Compilation is handed over to the new file by |\childdocforward|:
%    \begin{macrocode}
\newcommand{\childdocforwardprefix}[3][]
{
  \begingroup
    \def\childdocextract #2##1~~~{\def\childdoctmp{\childdocforward[#1]{#3##1}}}
    \expandafter\childdocextract\childdocname~~~
    \expandafter
  \endgroup
  \childdoctmp
}
%    \end{macrocode}

% \macro{\childdoc}
% The deprecated macro |\childdoc| is a legacy version of |\childdocmain|:
%    \begin{macrocode}
\newcommand{\childdoc}{\childdocmain}
%    \end{macrocode}

% \macro{\childdocredirect}
% The deprecated macro |\childdocredirect| is a legacy version
% of |\childdocforward| and |\childdocforwardprefix|:
%    \begin{macrocode}
\newcommand{\childdocredirect}[2][]
{
  \begingroup
    \if?#1?
      \def\childdoctmp{\childdocforward{#2}}
    \else
      \def\childdoctmp{\childdocforwardprefix{#1}{#2}}
    \fi
    \expandafter
  \endgroup
  \childdoctmp
}
%    \end{macrocode}

%\iffalse
%</package>
%\fi
%
\endinput
\childdocforward{cdocsamp}"|\\
% |latex -jobname cdocscl1 \|\\
% |  "% \iffalse
%
% childdoc.dtx Copyright (C) 2017-2018 Niklas Beisert
%
% This work may be distributed and/or modified under the
% conditions of the LaTeX Project Public License, either version 1.3
% of this license or (at your option) any later version.
% The latest version of this license is in
%   http://www.latex-project.org/lppl.txt
% and version 1.3 or later is part of all distributions of LaTeX
% version 2005/12/01 or later.
%
% This work has the LPPL maintenance status `maintained'.
%
% The Current Maintainer of this work is Niklas Beisert.
%
% This work consists of the files childdoc.dtx and childdoc.ins
% and the derived files childdoc.def and cdocsamp.tex with
% cdocsch1.tex, cdocsch2.tex, cdocsdrf.tex, cdocsfn1.tex, cdocsfn2.tex.
%
%<package>\ifdefined\childdocmain\endinput\fi
%<package>\ProvidesFile{childdoc.def}[2018/12/30 v2.0 child document driver]
%<samplemain>\ProvidesFile{cdocsamp.tex}[2018/12/30 v2.0 sample for childdoc]
%<*driver>
%\ProvidesFile{childdoc.drv}[2018/12/30 v2.0 childdoc reference manual file]
\PassOptionsToClass{10pt,a4paper}{article}
\documentclass{ltxdoc}

\usepackage[margin=35mm]{geometry}
\usepackage{hyperref}
\usepackage{hyperxmp}
\usepackage[usenames]{color}

\hypersetup{colorlinks=true}
\hypersetup{pdfstartview=FitH}
\hypersetup{pdfpagemode=UseNone}
\hypersetup{pdfsource={}}
\hypersetup{pdflang={en-UK}}
\hypersetup{pdfcopyright={Copyright 2017-2018 Niklas Beisert.
  This work may be distributed and/or modified under the
  conditions of the LaTeX Project Public License, either version 1.3
  of this license or (at your option) any later version.}}
\hypersetup{pdflicenseurl={http://www.latex-project.org/lppl.txt}}
\hypersetup{pdfcontactaddress={ETH Zurich, ITP, HIT K,
  Wolfgang-Pauli-Strasse 27}}
\hypersetup{pdfcontactpostcode={8093}}
\hypersetup{pdfcontactcity={Zurich}}
\hypersetup{pdfcontactcountry={Switzerland}}
\hypersetup{pdfcontactemail={nbeisert@itp.phys.ethz.ch}}
\hypersetup{pdfcontacturl={http://people.phys.ethz.ch/\xmptilde nbeisert/}}

\newcommand{\secref}[1]{\hyperref[#1]{section \ref*{#1}}}

\parskip1ex
\parindent0pt
\let\olditemize\itemize
\def\itemize{\olditemize\parskip0pt}

\begin{document}

\title{The \textsf{childdoc} Package}
\hypersetup{pdftitle={The childdoc Package}}
\author{Niklas Beisert\\[2ex]
  Institut f\"ur Theoretische Physik\\
  Eidgen\"ossische Technische Hochschule Z\"urich\\
  Wolfgang-Pauli-Strasse 27, 8093 Z\"urich, Switzerland\\[1ex]
  \href{mailto:nbeisert@itp.phys.ethz.ch}
  {\texttt{nbeisert@itp.phys.ethz.ch}}}
\hypersetup{pdfauthor={Niklas Beisert}}
\hypersetup{pdfsubject={Manual for the LaTeX2e Package childdoc}}
\date{30 December 2018, \textsf{v2.0}}
\maketitle

\begin{abstract}\noindent
\textsf{childdoc} is a \LaTeXe{} package
that enables the direct compilation
of document sections included by |\include|
to individual files.
\end{abstract}

\begingroup
\parskip0ex
\tableofcontents
\endgroup

%%%%%%%%%%%%%%%%%%%%%%%%%%%%%%%%%%%%%%%%%%%%%%%%%%%%%%%%%%%%%%%%%%%%%%%%%%%%%%%%
%%%%%%%%%%%%%%%%%%%%%%%%%%%%%%%%%%%%%%%%%%%%%%%%%%%%%%%%%%%%%%%%%%%%%%%%%%%%%%%%
\section{Introduction}

\LaTeX{} provides a mechanism to structure a large document (such as a book)
into a main file and several child files (containing the chapters)
using the |\include| command.
This mechanism is beneficial for documents
which span hundreds of pages in order to
make the source file(s) more manageable.
Moreover, compilation can be restricted to
selected child files by means of the |\includeonly| command.
The latter feature can be used to reduce the compilation time while editing
(this was significantly more useful in the earlier days of \LaTeX{})
or to generate a smaller document which is easier to navigate.
Another application of |\includeonly| is to generate
documents consisting of selected parts of the complete document.

However, there are a few drawbacks of the plain |\include| mechanism:
\begin{itemize}
\item
The child files cannot be compiled on their own,
they can only be compiled via the main file.
A naive editing environment
(such as a text editor with an option
to have the current file processed by \LaTeX)
may require one to switch to the main file before compiling;
attempting to compile the child file produces errors.
\item
The main file must be modified (each time)
to adjust the |\includeonly| command
to the present needs. This easily leaves the main file in a messy state.
\item
The generated document will always carry the filename
of the main document. This is inconvenient if
several child files are to be compiled and
to be kept for distribution.
\end{itemize}

The present package provides a simple interface
to make child files individually compilable by \LaTeX{}.
Compiling a child file then has the same effect as compiling
the main file with an |\includeonly| command
to select the appropriate child.
Moreover the generated document will carry the name of the child
rather than the main file.
This resolves all three above issues.

This feature is meant to make the editing of books,
thesis documents and lecture notes somewhat more convenient.
However, the package can also be used efficiently for
composing a series of documents (such as exercise sheets)
which are typically distributed individually.
It then assists the author in generating the individual documents
(potentially in different versions)
as well as a document containing the collected series.
Another application is in developing style files
or other kinds of included material
where compilation of the style file could redirect
to a sample or test file.

%%%%%%%%%%%%%%%%%%%%%%%%%%%%%%%%%%%%%%%%%%%%%%%%%%%%%%%%%%%%%%%%%%%%%%%%%%%%%%%%
%%%%%%%%%%%%%%%%%%%%%%%%%%%%%%%%%%%%%%%%%%%%%%%%%%%%%%%%%%%%%%%%%%%%%%%%%%%%%%%%
\section{Usage}

First of all, the package \textsf{childdoc} is \emph{not} a standard
\LaTeXe{} |.sty| style file! Therefore it needs to be invoked in
a non-standard way.

%%%%%%%%%%%%%%%%%%%%%%%%%%%%%%%%%%%%%%%%%%%%%%%%%%%%%%%%%%%%%%%%%%%%%%%%%%%%%%%%
\subsection{Included Files}
\label{sec:include}

%%%%%%%%%%%%%%%%%%%%%%%%%%%%%%%%%%%%%%%%
\DescribeMacro{\childdocmain}
To use the package, add the commands
\begin{center}
\begin{tabular}{l}
|\input{childdoc.def}|\\
|\childdocmain{}|\\
\end{tabular}
\end{center}
at the very top of the main \LaTeX{} file,
in particular \emph{before} the |\documentclass| statement!
The argument of |\childdocmain| should be left empty
(but it must be present).

%%%%%%%%%%%%%%%%%%%%%%%%%%%%%%%%%%%%%%%%
\DescribeMacro{\childdocof}
Furthermore, add the commands
\begin{center}
\begin{tabular}{l}
|\input{childdoc.def}|\\
|\childdocof{|\textit{main}|}|\\
\end{tabular}
\end{center}
at the top of every child file \textit{child}
which is included by |\include{|\textit{child}|}|
from within the main file
(or at least for those files to be compiled individually).
The argument \textit{main} must be the filename of the main file.

There are a couple of
considerations in setting up the main and child documents:

%%%%%%%%%%%%%%%%%%%%%%%%%%%%%%%%%%%%%%%%
\paragraph{Restrictions.}

Please note the following restrictions:
\begin{itemize}
\item
|\childdocmain| must be called with one argument \textit{main}
to ensure compatibility with earlier version of the package.
It must either be empty (|\childdocmain{}|)
or precisely match the filename of the main file in which it is specified.
See \secref{sec:detection} for further information.
\item
The filename \textit{main} must be specified without the |.tex| extension.
\item
The filename \textit{main} is case sensitive
(even in case-insensitive file systems)
due to internal string comparison.
\item
The argument \textit{main} should be fully expanded, it cannot be a macro.
\item
Subdirectories and special characters should be avoided in filenames.
\item
The command |\childdocmain{|\textit{main}|}| must be followed by a whitespace.
It should not be followed immediately by another command
or by a comment mark `|%|'.
This is because the \TeX{} parser reads the token immediately following
the argument of |\childdocmain| and puts it
at the beginning of every child section;
however, a white\-space is ignored.
\end{itemize}

%%%%%%%%%%%%%%%%%%%%%%%%%%%%%%%%%%%%%%%%
\paragraph{Content of Main File.}

It is advisable to place all content in the child files included by |\include|.
Any output contained in the main file will appear in all child documents
unless suppressed manually;
it cannot be suppressed automatically by the |\includeonly| directive
and thus should normally be avoided.
A method to include some content in the main file
by means of conditional processing is described in \secref{sec:conditional}.

%%%%%%%%%%%%%%%%%%%%%%%%%%%%%%%%%%%%%%%%
\paragraph{Page Numbering.}

When only a part of the document is compiled,
the appropriate numbering of pages
(as well as other status parameters)
is determined from the |.aux| files.
The latter contain information from previous passes.
However this information needs to propagate through
all intermediate child documents.
Therefore the page numbering in child documents may well
be inconsistent until the complete document is compiled at least once.

A useful (if unconventional) way to always ensure a consistent
page numbering is to restart the numbering in each child document
and denote the pages by `\textit{child}|.|\textit{page}'
where \textit{child} represents the chapter/section number of the child file.
This can be achieved by the command
|\numberwithin{page}{|\textit{child}|}|
of the \textsf{amsmath} package
where \textit{child} can be |chapter| or |section|
depending on the chosen structuring.
Alternatively, one can modify the macro |\thepage| appropriately
and reset the counter |page| at the start of each child file.

%%%%%%%%%%%%%%%%%%%%%%%%%%%%%%%%%%%%%%%%%%%%%%%%%%%%%%%%%%%%%%%%%%%%%%%%%%%%%%%%
\subsection{Conditional Processing}
\label{sec:conditional}

The package provides a mechanism to compile different versions
of a document. To customise the versions further some conditional processing
can come in handy to distinguish which version is being compiled.
The package provides two macros to describe the compilation context:

%%%%%%%%%%%%%%%%%%%%%%%%%%%%%%%%%%%%%%%%
\DescribeMacro{\ifchilddoc}
The conditional |\ifchilddoc| distinguishes between the compilation of
child documents and the main document:
%
\begin{center}
|\ifchilddoc |\textit{child-code}| |[|\||else |\textit{main-code}]| \||fi|
\end{center}

%%%%%%%%%%%%%%%%%%%%%%%%%%%%%%%%%%%%%%%%
\DescribeMacro{\childdocname}
\DescribeMacro{\childdocjob}
The macro |\childdocname| contains the filename (without extension)
of the main or child file being processed.
Note that |\childdocjob| will always contain the name of the main file.

%%%%%%%%%%%%%%%%%%%%%%%%%%%%%%%%%%%%%%%%
\paragraph{Title Page.}

Conditional processing can be used to include a title or banner page
in the main document when proper precautions are taken.
Importantly, the code in the main file should ensure that the page counter
(as well as other status parameters which are stored in the |.aux| files)
takes the same value after the conditional processing.
Otherwise the page numbers may take divergent values
depending on which part is compiled.

For example, a title page could be declared by:
%
\begin{center}
\begin{tabular}{l}
|\ifchilddoc\||else|\\
|\addtocounter{page}{-1}|\\
\textit{code for title page}\\
|\newpage|\\
|\||fi|
\end{tabular}
\end{center}
%
A banner page for the child documents can be generated by:
%
\begin{center}
\begin{tabular}{l}
|\ifchilddoc|\\
|\addtocounter{page}{-1}|\\
\textit{code for banner page}\\
|\newpage|\\
|\||fi|
\end{tabular}
\end{center}
%
Here one could write a message such as:
\begin{center}
|This is the part \childdocname{} of \childdocjob{}.|
\end{center}

%%%%%%%%%%%%%%%%%%%%%%%%%%%%%%%%%%%%%%%%%%%%%%%%%%%%%%%%%%%%%%%%%%%%%%%%%%%%%%%%
\subsection{Flags}
\label{sec:flags}

The package makes it easy to generate different versions
of the main or child documents.
To this end compilation flags can be defined
and assigned different default values.
They will be particularly useful in conjunction
with the forwarding mechanism described in \secref{sec:forward}.

For example, it may be useful to have a flag |\version|
which can be set to |draft| or |final|.
The document source will contain some conditional code
depending on the value of |\version|.
Suppose further, the flag should default to |final| for the main file
and to |draft| for child files
which is a natural assignment for editing the document.
This is achieved by placing the following code
in the preamble of the main document
(below the |\childdocmain| directive):
%
\begin{center}
\begin{tabular}{l}
|\ifchilddoc|\\
|\providecommand{\version}{draft}|\\
|\||else|\\
|\providecommand{\version}{final}|\\
|\||fi|
\end{tabular}
\end{center}
%
The definition by |\providecommand| makes sure
that previous definitions are not overwritten.
Further statements |\providecommand{\version}{...}|
can thus be added before the above code to override it.

For the main file, one might add a line
(between |\childdocmain| and the above block)
%
\begin{center}
|%\ifchilddoc\||else\providecommand{\version}{draft}\||fi|
\end{center}
%
which can be uncommented to produce a draft version.
Likewise one can add a line to the very top of a child file
(above the |\childdocof{|\textit{main}|}| directive)
%
\begin{center}
|%\providecommand{\version}{final}|
\end{center}
%
which can be uncommented to produce the final version of this child document.

%%%%%%%%%%%%%%%%%%%%%%%%%%%%%%%%%%%%%%%%%%%%%%%%%%%%%%%%%%%%%%%%%%%%%%%%%%%%%%%%
\subsection{Forwarding}
\label{sec:forward}

Different versions of the main or child documents
using compilation flags as described in \secref{sec:flags}
can be (permanently) stored in different files
for convenient compilation, viewing and distribution.
To this end, the package defines a command
to pass on compilation to a different file:

%%%%%%%%%%%%%%%%%%%%%%%%%%%%%%%%%%%%%%%%
\DescribeMacro{\childdocforward}
The command |\childdocforward| redirects processing to
another source file:
%
\begin{center}
\begin{tabular}{l}
|\input{childdoc.def}|\\
|\childdocforward[|\textit{main}|]{|\textit{dest}|}|\\
\end{tabular}
\end{center}
%
The argument \textit{dest} is the destination file
(without extension).
It should be the main file or one of the child files.
Note that further \textsf{childdoc} directives
such as |\childdocof| and |\childdocforward|
in the indicated file will be processed in this form.
The optional argument \textit{main}
passes on directly to the main file \textit{main}
while pretending to compile the child \textit{dest}.
This form behaves as if \textit{dest}
issues |\childdocof{|\textit{main}|}| right away,
and no further \textsf{childdoc} directives will be processed.

%%%%%%%%%%%%%%%%%%%%%%%%%%%%%%%%%%%%%%%%
\DescribeMacro{\...prefix}
In the alternative form |\childdocforwardprefix|,
%
\begin{center}
\begin{tabular}{l}
|\input{childdoc.def}|\\
|\childdocforwardprefix[|\textit{main}|]{|\textit{prefix}|}{|\textit{dest}|}|
\end{tabular}
\end{center}
%
the destination file is determined by a pattern
depending on the current file:
To make this work, the current file must be called
`{\textit{prefix}\hspace{0.2em}\textit{suffix}}'
with \textit{prefix} matching precisely the argument.
Processing is then passed on to the file
`{\textit{dest}\hspace{0.2em}\textit{suffix}}'.
Surely, the same effect is achieved by
directly specifying the
argument `{\textit{dest}\hspace{0.2em}\textit{suffix}}'
in the first form.
However, that requires to set up a different file
for each child. With the alternative form of the command
all these files can have exactly the same content
which simplifies setting them up and maintaining them.

For example, the following file |draft.tex|
with a compilation flag |\version| as described in \secref{sec:flags}
compiles the main document as a draft:
%
\begin{center}
\begin{tabular}{l}
|\def\version{draft}|\\
|\input{childdoc.def}|\\
|\childdocforward{|\textit{main}|}|
\end{tabular}
\end{center}
%
Likewise, the following files |final|\textit{nn}|.tex|
compile the final version of the child document
|child|\textit{nn}|.tex|:
%
\begin{center}
\begin{tabular}{l}
|\def\version{final}|\\
|\input{childdoc.def}|\\
|\childdocforwardprefix{final}{child}|
\end{tabular}
\end{center}
%

Note that when several versions of a main file and/or of each child file
are to be generated, it may be convenient to set up a |Makefile| or
shell script to automatise the process.

%%%%%%%%%%%%%%%%%%%%%%%%%%%%%%%%%%%%%%%%%%%%%%%%%%%%%%%%%%%%%%%%%%%%%%%%%%%%%%%%
\subsection{Command Line Processing}
\label{sec:commandline}

The effect of redirection files can also be achieved by invoking
the \LaTeX{} compiler with a more elaborate command line.
Most conveniently this should be done as part
of a shell script or a |Makefile|.

When using \textsf{childdoc} in the main file, the following
command lines effectively perform a redirection
(note that depending on the shell being used,
backslashes may have to be doubled: `|\|' $\to$ `|\\|'):
%
\begin{center}
|... -jobname "|\textit{target}|" |\\|"|[\textit{flags}]%
|\input{childdoc.def}\childdocforward[|\textit{main}|]{|\textit{dest}|}"|
\end{center}
%
Here \textit{target} is the name of the output file,
\textit{main} is the name of the main file
and \textit{dest} is the name of the main or child file to be processed
(all filenames without extensions).
The optional argument \textit{main} can be omitted
if \textit{main} matches \textit{dest}.
Optionally, compilation \textit{flags} can be defined via |\def| commands.
This command line makes the \TeX{} engine believe
it is compiling the file \textit{target}
whose content is specified as the latter parameter.
The provided code then forwards the processing to
\textit{main} or \textit{dest} as described in \secref{sec:forward}.

%%%%%%%%%%%%%%%%%%%%%%%%%%%%%%%%%%%%%%%%%%%%%%%%%%%%%%%%%%%%%%%%%%%%%%%%%%%%%%%%
\subsection{Include by Input}
\label{sec:input}

Including child documents by |\include| has some restrictions by design.
Most notably, the content of a child document always occupies
its own set of pages; pages cannot be shared between child documents.
Usually, this behaviour makes perfect sense
because each child document contain an essential part of the document.
However, in some situations it may be desirable to compose
a document from a collection of parts
without having mandatory page breaks between then.
For this case, the package
provides a mechanism to include parts
by |\input| which can also be processed individually.
However, by construction this mechanism
requires manual handling of the content to be output.

%%%%%%%%%%%%%%%%%%%%%%%%%%%%%%%%%%%%%%%%
\DescribeMacro{\ifchilddocmanual}
The main file should be prepared as usual, see \secref{sec:include}.
However, the document body must make a distinction
between processing of an individual part and of the main document, e.g.:
%
\begin{center}
\begin{tabular}{l}
|\ifchilddocmanual|\\
|\input{\childdocname}|\\
|\||else|\\
\textit{document body with }|\input{|\textit{part}|}|\\
|\||fi|
\end{tabular}
\end{center}
%
The conditional |\ifchilddocmanual| is true whenever
a part to be included by |\input| is being compiled,
and the name of the part is stored in |\childdocname|.

%%%%%%%%%%%%%%%%%%%%%%%%%%%%%%%%%%%%%%%%
\DescribeMacro{\childdocby}
Each part to be included by |\input| should start with:
%
\begin{center}
\begin{tabular}{l}
|\input{childdoc.def}|\\
|\childdocby{|\textit{main}|}|\\
\end{tabular}
\end{center}
%
The directive |\childdocby| is similar to |\childdocof|
described in \secref{sec:include},
but the subsequent selection of content must be done manually.
To that end, both |\ifchilddoc| and |\ifchilddocmanual|
will be true upon processing of a part,
and the name of the part is stored in |\childdocname|.
Note that |\jobname| will be set to the filename of the current part
so that each part receives an individual |.aux| file
that does not interfere with the |.aux| file(s) of the main document.
This behaviour can be altered by the alternative form
|\childdocby[*]{|\textit{main}|}| (with a non-empty optional argument)
which uses the |.aux| file of the main document
by setting |\jobname| to \textit{main}.

%%%%%%%%%%%%%%%%%%%%%%%%%%%%%%%%%%%%%%%%%%%%%%%%%%%%%%%%%%%%%%%%%%%%%%%%%%%%%%%%
\subsection{Driver Development}
\label{sec:driver}

The \textsf{childdoc} mechanism can also be use for the development
of definition files such as \LaTeX{} styles or classes.
This case differs from the above setup with multiple parts
included by |\include| in that no |\includeonly| should be invoked.
This can be achieved by starting the include file
(before |\ProvidesPackage|) with:
%
\begin{center}
\begin{tabular}{l}
|\input{childdoc.def}|\\
|\childdocforward{|\textit{main}|}|\\
\end{tabular}
\end{center}
%
or alternatively with:
%
\begin{center}
\begin{tabular}{l}
|\input{childdoc.def}|\\
|\childdocby{|\textit{main}|}|\\
\end{tabular}
\end{center}
%
Both forms have slightly different effects as described above.
The main file is prepared as usual, see \secref{sec:include}.

%%%%%%%%%%%%%%%%%%%%%%%%%%%%%%%%%%%%%%%%%%%%%%%%%%%%%%%%%%%%%%%%%%%%%%%%%%%%%%%%
\subsection{Legacy Detection}
\label{sec:detection}

The directive |\childdocmain| in the main file can detect
whether the complete document or merely a child is to be compiled
even without using the directive |\childdocof|.
This method is deprecated because it is less robust
and there is no compelling reason to use it;
it is merely provided for backward compatibility
and it may be removed in future versions.

If the detection mechanism is to be used,
it is mandatory to correctly specify
the filename of the main file as the argument of |\childdocmain|:
%
\begin{center}
\begin{tabular}{l}
|\input{childdoc.def}|\\
|\childdocmain{|\textit{main}|}|\\
\end{tabular}
\end{center}
%
If |\jobname| does not match the argument \textit{main} of |\childdocmain|,
it is assumed that |\jobname| points to the child file to be compiled.
When using |\childdocmain| with the main file specified as argument,
it suffices to start a child file
with just |\input{|\textit{main}|}|
without loading of the package and using |\childdocof|.
If instead all processing is done
with the appropriate \textsf{childdoc} directives,
the argument of \textit{main} of |\childdocmain| can be empty.

An alternative version of the command line processing described
in \secref{sec:commandline} using the detection mechanism reads:
%
\begin{center}
|... -jobname "|\textit{target}|" "|[\textit{flags}]%
[|\def\jobname{|\textit{dest}|}|]|\input{|\textit{main}|}"|
\end{center}

%%%%%%%%%%%%%%%%%%%%%%%%%%%%%%%%%%%%%%%%%%%%%%%%%%%%%%%%%%%%%%%%%%%%%%%%%%%%%%%%
\subsection{Manual Code}
\label{sec:manual}

In case one cannot be certain whether the definitions file |childdoc.def|
is installed on the target \TeX{} distribution
and one prefers not to ship it,
it is conceivable to paste a few relevant commands into the sources.

To that end, drop all statements |\input{childdoc.def}|
and perform the replacements as outlined below.
Instead of |\childdocmain{|\textit{main}|}| add the following code
to the top of the main file:
%
\begin{center}
\begin{tabular}{l}
|\||ifdefined\childdocname\endinput\||fi\newif\ifchilddoc|\\
|\edef\childdocname{\scantokens\expandafter{\jobname\noexpand}}|\\
|\def\childdocmain{|\textit{main}|}\||ifx\childdocmain\childdocname\||else|\\
|\childdoctrue\includeonly{\childdocname}\let\jobname\childdocmain\||fi|\\
\end{tabular}
\end{center}
%
Instead of |\childdocof{|\textit{main}|}| just include the main file
at the top of each child file:
%
\begin{center}
|\input{|\textit{main}|}|
\end{center}
%
A simple redirection |\childdocforward{|\textit{dest}|}| is achieved by:
%
\begin{center}
|\def\jobname{|\textit{dest}|}\input{\jobname}|
\end{center}
%
The redirection with prefix
|\childdocforwardprefix[|\textit{prefix}|]{|\textit{dest}|}|
is accomplished by:
%
\begin{center}
\begin{tabular}{l}
|{\edef\jobname{\scantokens\expandafter{\jobname\noexpand}}|\\
|\def\redirectjob |\textit{prefix}|#1~~~{\gdef\jobname{|\textit{dest}|#1}}|\\
|\expandafter\redirectjob\jobname~~~}\input{\jobname}|
\end{tabular}
\end{center}

In an alternative approach,
child documents can be compiled by a specific command line
without additional code or specific definitions:
%
\begin{center}
|... -jobname "|\textit{target}|" "|[\textit{flags}]%
|\includeonly{|\textit{dest}|}\input{|\textit{main}|}"|
\end{center}
%

%%%%%%%%%%%%%%%%%%%%%%%%%%%%%%%%%%%%%%%%%%%%%%%%%%%%%%%%%%%%%%%%%%%%%%%%%%%%%%%%
%%%%%%%%%%%%%%%%%%%%%%%%%%%%%%%%%%%%%%%%%%%%%%%%%%%%%%%%%%%%%%%%%%%%%%%%%%%%%%%%
\section{Information}

%%%%%%%%%%%%%%%%%%%%%%%%%%%%%%%%%%%%%%%%%%%%%%%%%%%%%%%%%%%%%%%%%%%%%%%%%%%%%%%%
\subsection{Copyright}

Copyright \copyright{} 2017--2018 Niklas Beisert

This work may be distributed and/or modified under the
conditions of the \LaTeX{} Project Public License, either version 1.3
of this license or (at your option) any later version.
The latest version of this license is in
  \url{http://www.latex-project.org/lppl.txt}
and version 1.3 or later is part of all distributions of \LaTeX{}
version 2005/12/01 or later.

This work has the LPPL maintenance status `maintained'.

The Current Maintainer of this work is Niklas Beisert.

This work consists of the files |README.txt|, |childdoc.ins| and |childdoc.dtx|
as well as the derived files |childdoc.def|, |cdocsamp.tex|
with |cdocsch1.tex|, |cdocsch2.tex|, |cdocspt3.tex|, |cdocspt4.tex|,
|cdocsdrf.tex|, |cdocsfn1.tex|, |cdocsfn2.tex|
as well as |childdoc.pdf|.

%%%%%%%%%%%%%%%%%%%%%%%%%%%%%%%%%%%%%%%%%%%%%%%%%%%%%%%%%%%%%%%%%%%%%%%%%%%%%%%%
\subsection{Files and Installation}

The package consists of the files:
%
\begin{center}
\begin{tabular}{ll}
    |README.txt|   & readme file \\
    |childdoc.ins| & installation file \\
    |childdoc.dtx| & source file \\
    |childdoc.def| & definition file \\
    |cdocsamp.tex| & sample main file \\
    |cdocsch1.tex| & sample include file \\
    |cdocsch2.tex| & sample include file \\
    |cdocspt3.tex| & sample part file \\
    |cdocspt4.tex| & sample part file \\
    |cdocsdrf.tex| & sample redirection file \\
    |cdocsfn1.tex| & sample redirection file \\
    |cdocsfn2.tex| & sample redirection file \\
    |childdoc.pdf| & manual
\end{tabular}
\end{center}
%
The distribution consists of the files
|README.txt|, |childdoc.ins| and |childdoc.dtx|.
%
\begin{itemize}
\item
Run (pdf)\LaTeX{} on |childdoc.dtx|
to compile the manual |childdoc.pdf| (this file).
\item
Run \LaTeX{} on |childdoc.ins| to create the definitions file |childdoc.def|
and the sample |cdocsamp.tex| with include files
|cdocsch1.tex|, |cdocsch2.tex|, |cdocspt3.tex|, |cdocspt4.tex|,
|cdocsdrf.tex|, |cdocsfn1.tex|, |cdocsfn2.tex|.
Then copy the file |childdoc.def| to an appropriate directory of your \LaTeX{}
distribution, e.g.\ \textit{texmf-root}|/tex/latex/childdoc|.
\end{itemize}

%%%%%%%%%%%%%%%%%%%%%%%%%%%%%%%%%%%%%%%%%%%%%%%%%%%%%%%%%%%%%%%%%%%%%%%%%%%%%%%%
\subsection{Related CTAN Packages}

There are several other packages which offer a similar functionality:
%
\begin{itemize}
\item
The packages
\href{http://ctan.org/pkg/docmute}{\textsf{docmute}},
\href{http://ctan.org/pkg/includex}{\textsf{includex}} and
\href{http://ctan.org/pkg/standalone}{\textsf{standalone}}
provide commands to include only the document body of
a child file thus allowing both files to be compiled individually.
\item
The packages \href{http://ctan.org/pkg/subdocs}{\textsf{subdocs}}
and \href{http://ctan.org/pkg/subfiles}{\textsf{subfiles}}
provide structures in which the main and child documents can be
encapsulated and allowing them to be compiled individually.
The inclusion mechanism is different from the conventional |\include|.
\item
The package \href{http://ctan.org/pkg/combine}{\textsf{combine}}
is an elaborate solution to combine several documents into one.
\end{itemize}
%
See also the CTAN topic \href{http://ctan.org/topic/subdocs}{\textsf{subdocs}}
for further related packages.
The present package differs from the above solutions in that
a document structure constructed with the conventional |\include| mechanism
just needs two extra commands at the top of every file
such that all constituent files can be compiled individually.

%%%%%%%%%%%%%%%%%%%%%%%%%%%%%%%%%%%%%%%%%%%%%%%%%%%%%%%%%%%%%%%%%%%%%%%%%%%%%%%%
%\subsection{Feature Suggestions}
%
%The following is a list of features which may be useful for future
%versions of this package:
%%
%\begin{itemize}
%\item
%\ldots
%\end{itemize}

%%%%%%%%%%%%%%%%%%%%%%%%%%%%%%%%%%%%%%%%%%%%%%%%%%%%%%%%%%%%%%%%%%%%%%%%%%%%%%%%
\subsection{Revision History}

%%%%%%%%%%%%%%%%%%%%%%%%%%%%%%%%%%%%%%%%
\paragraph{v2.0:} 2018/12/30

\begin{itemize}
\item
immediate forward processing
\item
added |\childdocby| mechanism
\item
manual restructured
\end{itemize}

%%%%%%%%%%%%%%%%%%%%%%%%%%%%%%%%%%%%%%%%
\paragraph{v1.6:} 2018/01/17

\begin{itemize}
\item
application for development of include files
\item
corrections to manual
\end{itemize}

%%%%%%%%%%%%%%%%%%%%%%%%%%%%%%%%%%%%%%%%
\paragraph{v1.5:} 2017/05/21

\begin{itemize}
\item
more complete structuring introduced
\item
|\childdocof| introduced
\item
|\childdoc| renamed to |\childdocmain|
\item
|\childredirect| renamed to |\childdocforward| and |\childdocforwardprefix|
and functionality expanded
\end{itemize}

%%%%%%%%%%%%%%%%%%%%%%%%%%%%%%%%%%%%%%%%
\paragraph{v1.0:} 2017/04/27

\begin{itemize}
\item
manual and install package
\item
first version published on CTAN
\end{itemize}

%%%%%%%%%%%%%%%%%%%%%%%%%%%%%%%%%%%%%%%%
\paragraph{v0.6:} 2017/04/26

\begin{itemize}
\item
redirection mechanism added
\end{itemize}

%%%%%%%%%%%%%%%%%%%%%%%%%%%%%%%%%%%%%%%%
\paragraph{v0.5:} 2017/04/26

\begin{itemize}
\item
functionality in definition file
\end{itemize}


%%%%%%%%%%%%%%%%%%%%%%%%%%%%%%%%%%%%%%%%%%%%%%%%%%%%%%%%%%%%%%%%%%%%%%%%%%%%%%%%
%%%%%%%%%%%%%%%%%%%%%%%%%%%%%%%%%%%%%%%%%%%%%%%%%%%%%%%%%%%%%%%%%%%%%%%%%%%%%%%%
%%%%%%%%%%%%%%%%%%%%%%%%%%%%%%%%%%%%%%%%%%%%%%%%%%%%%%%%%%%%%%%%%%%%%%%%%%%%%%%%
\appendix

\settowidth\MacroIndent{\rmfamily\scriptsize 000\ }

 \DocInput{childdoc.dtx}

\end{document}
%</driver>
% \fi
%
% %%%%%%%%%%%%%%%%%%%%%%%%%%%%%%%%%%%%%%%%%%%%%%%%%%%%%%%%%%%%%%%%%%%%%%%%%%%%%%
% %%%%%%%%%%%%%%%%%%%%%%%%%%%%%%%%%%%%%%%%%%%%%%%%%%%%%%%%%%%%%%%%%%%%%%%%%%%%%%
% \section{Sample}
%\iffalse
%<*samplemain>
%\fi
%
% The following presents a sample document
% with two chapters, two parts, a title page,
% a compile flag as well as three forwarding files to set the flag.
% It consists of eight |.tex| files:
% \begin{center}
% \begin{tabular}{ll}
% |cdocsamp.tex|&main file\\
% |cdocsch1.tex|&include file for chapter 1\\
% |cdocsch2.tex|&include file for chapter 2\\
% |cdocspt3.tex|&include file for part 3\\
% |cdocspt4.tex|&include file for part 4\\
% |cdocsdrf.tex|&forwarding file for main file in draft mode\\
% |cdocsfi1.tex|&forwarding file for final version of chapter 1\\
% |cdocsfi2.tex|&forwarding file for final version of chapter 2\\
% \end{tabular}
% \end{center}
% Each of the eight files can be compiled directly by the \LaTeX{} compiler.
%
% %%%%%%%%%%%%%%%%%%%%%%%%%%%%%%%%%%%%%%
% \paragraph{Main File.}
%
% The main file is called |cdocsamp.tex|.
%
% Load the \textsf{childdoc} definitions and
% declare the filename for the main document:
%    \begin{macrocode}
\input{childdoc.def}
\childdocmain{}
%    \end{macrocode}

% Optional override for |\version| flag:
%    \begin{macrocode}
%%\ifchilddoc\else\providecommand{\version}{draft}\fi
%    \end{macrocode}

% Define the default values for the |\version| flag
% (|final| for the main file and |draft| for childs):
%    \begin{macrocode}
\ifchilddoc
\providecommand{\version}{draft}
\else
\providecommand{\version}{final}
\fi
%    \end{macrocode}

% Load the standard document class:
%    \begin{macrocode}
\documentclass[12pt]{article}
%    \end{macrocode}

% Start the document body:
%    \begin{macrocode}
\begin{document}
%    \end{macrocode}

% Declare a title page.
% Print title, part of document being processed and version flag:
%    \begin{macrocode}
\addtocounter{page}{-1}
\begin{center}
{\LARGE\bfseries{}childdoc example\par}
\vspace{1cm}
\ifchilddoc
\ifchilddocmanual part\else chapter\fi:
`\childdocname' of `\childdocjob'\par
\else
main document: `\childdocjob'\par
\fi
version: \version\par
\end{center}
\newpage
%    \end{macrocode}

% Manually include selected file,
% otherwise process as usual:
%    \begin{macrocode}
\ifchilddocmanual
\section*{part `\childdocname'}
\input{\childdocname}
\else
%    \end{macrocode}

% Include the two chapters:
%    \begin{macrocode}
\include{cdocsch1}
\include{cdocsch2}
%    \end{macrocode}

% Include the two parts unless only chapters should be displayed:
%    \begin{macrocode}
\ifchilddoc\else
\section{part three}
\input{cdocspt3}
\section{part four}
\input{cdocspt4}
\fi
%    \end{macrocode}

% Process as usual until here:
%    \begin{macrocode}
\fi
%    \end{macrocode}

% End of document body:
%    \begin{macrocode}
\end{document}
%    \end{macrocode}
%\iffalse
%</samplemain>
%\fi
%
% %%%%%%%%%%%%%%%%%%%%%%%%%%%%%%%%%%%%%%
% \paragraph{Chapter Include Files.}
%
% The include files are called |cdocsch1.tex| and |cdocsch2.tex|.
%
%\iffalse
%<*samplechap1|samplechap2>
%\fi

% Optional override for |\version| flag:
%    \begin{macrocode}
%%\providecommand{\version}{final}
%    \end{macrocode}

% Include the main document:
%    \begin{macrocode}
\input{childdoc.def}
\childdocof{cdocsamp}
%    \end{macrocode}

%\iffalse
%</samplechap1|samplechap2>
%\fi
%
%\iffalse
%<*samplechap1>
%\fi
% Some text for chapter 1:
%    \begin{macrocode}
\section{one}
some text in chapter one
%    \end{macrocode}

%\iffalse
%</samplechap1>
%\fi
% Some text for chapter 2:
%\iffalse
%<*samplechap2>
%\fi
%    \begin{macrocode}
\section{two}
more text in chapter two
%    \end{macrocode}

%\iffalse
%</samplechap2>
%\fi
%
% %%%%%%%%%%%%%%%%%%%%%%%%%%%%%%%%%%%%%%
% \paragraph{Part Include Files.}
%
% The include files are called |cdocspt3.tex| and |cdocspt4.tex|.
%
%\iffalse
%<*samplepart3|samplepart4>
%\fi

% Optional override for |\version| flag:
%    \begin{macrocode}
%%\providecommand{\version}{final}
%    \end{macrocode}

% Include the main document:
%    \begin{macrocode}
\input{childdoc.def}
\childdocby{cdocsamp}
%    \end{macrocode}

%\iffalse
%</samplepart3|samplepart4>
%\fi
%
%\iffalse
%<*samplepart3>
%\fi
% Some text for part 3:
%    \begin{macrocode}
some text in part three
%    \end{macrocode}

%\iffalse
%</samplepart3>
%\fi
% Some text for part 4:
%\iffalse
%<*samplepart4>
%\fi
%    \begin{macrocode}
more text in part four
%    \end{macrocode}

%\iffalse
%</samplepart4>
%\fi
%
% %%%%%%%%%%%%%%%%%%%%%%%%%%%%%%%%%%%%%%
% \paragraph{Forwarding for a Complete Draft.}
%
% The following forwarding file |cdocsdrf.tex|
% compiles the main document in draft mode:
%\iffalse
%<*sampledraft>
%\fi
%    \begin{macrocode}
\def\version{draft}
\input{childdoc.def}
\childdocforward{cdocsamp}
%    \end{macrocode}

%\iffalse
%</sampledraft>
%\fi
%
% %%%%%%%%%%%%%%%%%%%%%%%%%%%%%%%%%%%%%%
% \paragraph{Forwarding for Final Version of the Chapters.}
%
% The following forwarding files |cdocsfn1.tex| and |cdocsfn2.tex|
% (with identical content)
% compile the final versions of the child documents
% |cdocsch1.tex| and |cdocsch2.tex|, respectively:
%\iffalse
%<*samplefinal>
%\fi
%    \begin{macrocode}
\def\version{final}
\input{childdoc.def}
\childdocforwardprefix[cdocsamp]{cdocsfn}{cdocsch}
%    \end{macrocode}

%\iffalse
%</samplefinal>
%\fi
%
% %%%%%%%%%%%%%%%%%%%%%%%%%%%%%%%%%%%%%%
% \paragraph{Command Line Processing.}
%
% The following three command lines generate the output files
% |cdocscld|, |cdocscl1| and |cdocscl2|
% which should be identical to
% |cdocsdrf|, |cdocsch1| and |cdocsfn2|, respectively:
% \begin{center}
% \begin{tabular}{l}
% |latex -jobname cdocscld \|\\
% |  "\def\version{draft}\input{childdoc.def}\childdocforward{cdocsamp}"|\\
% |latex -jobname cdocscl1 \|\\
% |  "\input{childdoc.def}\childdocforward[cdocsamp]{cdocsch1}"|\\
% |latex -jobname cdocscl2 \|\\
% |  "\def\version{final}\input{childdoc.def}\childdocforward{cdocsch2}"|
% \end{tabular}
% \end{center}
% Note that the trailing backslash on each first line
% merely continues the input to the second line
% (for convenient cut ant paste).
% Furthermore, the command |latex| can be replaced by any
% of its alternative versions such as |pdflatex|.
%
% %%%%%%%%%%%%%%%%%%%%%%%%%%%%%%%%%%%%%%%%%%%%%%%%%%%%%%%%%%%%%%%%%%%%%%%%%%%%%%
% %%%%%%%%%%%%%%%%%%%%%%%%%%%%%%%%%%%%%%%%%%%%%%%%%%%%%%%%%%%%%%%%%%%%%%%%%%%%%%
% \section{Implementation}
%\iffalse
%<*package>
%\fi
%
% This section describes the definitions file |childdoc.def|.

% The definitions cannot be loaded using |\usepackage| or |\RequirePackage|
% which has a mechanism to prevent loading a style file more than once.
% When loading the definitions by means of |\input|
% multiple instances have to be prevented manually:
%\iffalse
%This code needs to be before the `\ProvidesFile' directive
%which is defined at the beginning of this file.
%Therefore it is also placed there and commented out here.
%</package>
%<*discard>
%\fi
%    \begin{macrocode}
\ifdefined\childdocmain\endinput\fi
%    \end{macrocode}
%\iffalse
%</discard>
%<*package>
%\fi
%
% \macro{\ifchilddoc}
% \macro{\ifchilddocmanual}
% The conditional |\ifchilddoc| tells whether a
% child (true) or main (false) document is being compiled.
% The conditional |\ifchilddocmanual| tells whether
% the |\includeonly| mechanism is used (false) or
% the selection of child files must be performed manually (true).
% The definitions initialise to false:
%    \begin{macrocode}
\newif\ifchilddoc
\newif\ifchilddocmanual
%    \end{macrocode}

% \macro{\childdocname}
% \macro{\childdocjob}
% The macro |\childdocname| stores the name of the main document
% to be compiled. The macro |\childdocjob| stores the name of
% the document on which the \LaTeX{} compiler was originally invoked.
% The content of |\jobname| cannot be compared
% to filenames specified in the source due to different catcodes.
% The following code rescans |\jobname|, stores the result
% in |\childdocname| and saves a copy in |\childdocjob|:
%    \begin{macrocode}
\edef\childdocname{\scantokens\expandafter{\jobname\noexpand}}
\let\childdocjob\childdocname
%    \end{macrocode}

% \macro{\childdocdisable}
% The macro |\childdocdisable| prevents the main file
% from being processed more than once.
% At this stage, the main document command |\childdocmain|
% is assumed to be called once again where it should do nothing.
% Any subsequent call to it should prevent
% a secondary processing of the main document
% It overwrites the forwarding commands
% |\childdocof| and |\childdocforward|
% with empty macros to prevent further inclusions of the main document:
%    \begin{macrocode}
\newcommand{\childdocdisable}
{
  \renewcommand{\childdocmain}[1]{\renewcommand{\childdocmain}[1]{\endinput}}
  \renewcommand{\childdocof}[1]{}
  \renewcommand{\childdocby}[2][]{}
  \renewcommand{\childdocforward}[2][]{}
  \renewcommand{\childdocdisable}{}
}
%    \end{macrocode}

% \macro{\childdocmain}
% The macro |\childdocmain| is to be called at the top of the main file
% with nothing or the main filename (without extension) as argument.
% First, it breaks loops.
% If the argument is not empty and does not match |\childdocname|
% (which is set by the first inclusion of |childdoc.def|),
% |\ifchilddoc| is set to true, |\includeonly| is applied to the child file
% and |\jobname| is set to the main file
% (for proper handling of |.aux| files):
%    \begin{macrocode}
\newcommand{\childdocmain}[1]
{
  \childdocdisable\childdocmain{}
  \if?#1?\else
    \begingroup
      \def\childdoctmp{#1}
      \ifx\childdoctmp\childdocname
        \def\childdoctmp{}
      \else
        \def\childdoctmp
        {
          \childdoctrue
          \includeonly{\childdocname}
          \def\childdocjob{#1}
          \def\jobname{#1}
        }
      \fi
      \expandafter
    \endgroup
    \childdoctmp
  \fi
}
%    \end{macrocode}

% \macro{\childdocof}
% The command |\childdocof| redirects
% compilation to the main file |#1|.
%    \begin{macrocode}
\newcommand{\childdocof}[1]
{
  \childdocdisable
  \childdoctrue
  \includeonly{\childdocname}
  \def\jobname{#1}
  \def\childdocjob{#1}
  \input{#1}
}
%    \end{macrocode}

% \macro{\childdocby}
% The command |\childdocby| ....
%    \begin{macrocode}
\newcommand{\childdocby}[2][]
{
  \childdocdisable
  \childdoctrue
  \childdocmanualtrue
  \if?#1?\else
    \def\jobname{#2}
  \fi
  \def\childdocjob{#2}
  \input{#2}
  \endinput
}
%    \end{macrocode}

% \macro{\childdocforward}
% The command |\childdocforward| redirects
% compilation to the main file or
% (if the optional argument is given) a child file.
% Parameters are set as if the main file
% or a child file starting with |\childdocof| was compiled.
% Then compilation is handed over to the main file:
%    \begin{macrocode}
\newcommand{\childdocforward}[2][]
{
  \begingroup
    \if?#1?
      \def\childdoctmp
      {
        \def\childdocname{#2}
        \def\childdocjob{#2}
        \def\jobname{#2}
        \input{#2}
        \endinput
      }
    \else
      \def\childdoctmp
      {
        \childdocdisable
        \def\childdocname{#2}
        \childdoctrue
        \includeonly{#2}
        \def\childdocjob{#1}
        \def\jobname{#1}
        \input{#1}
        \endinput
      }
    \fi
    \expandafter
  \endgroup
  \childdoctmp
}
%    \end{macrocode}

% \macro{\childdocforwardprefix}
% The command |\childdocforwardprefix| redirects
% compilation to the main or a child file by means of a pattern.
% The prefix |#1| in the current filename is replaced by |#2|
% and the suffix of the current filename is kept
% (it is assumed that the filename does not contain the substring `|~~~|'
% which is used as a delimiter).
% Compilation is handed over to the new file by |\childdocforward|:
%    \begin{macrocode}
\newcommand{\childdocforwardprefix}[3][]
{
  \begingroup
    \def\childdocextract #2##1~~~{\def\childdoctmp{\childdocforward[#1]{#3##1}}}
    \expandafter\childdocextract\childdocname~~~
    \expandafter
  \endgroup
  \childdoctmp
}
%    \end{macrocode}

% \macro{\childdoc}
% The deprecated macro |\childdoc| is a legacy version of |\childdocmain|:
%    \begin{macrocode}
\newcommand{\childdoc}{\childdocmain}
%    \end{macrocode}

% \macro{\childdocredirect}
% The deprecated macro |\childdocredirect| is a legacy version
% of |\childdocforward| and |\childdocforwardprefix|:
%    \begin{macrocode}
\newcommand{\childdocredirect}[2][]
{
  \begingroup
    \if?#1?
      \def\childdoctmp{\childdocforward{#2}}
    \else
      \def\childdoctmp{\childdocforwardprefix{#1}{#2}}
    \fi
    \expandafter
  \endgroup
  \childdoctmp
}
%    \end{macrocode}

%\iffalse
%</package>
%\fi
%
\endinput
\childdocforward[cdocsamp]{cdocsch1}"|\\
% |latex -jobname cdocscl2 \|\\
% |  "\def\version{final}% \iffalse
%
% childdoc.dtx Copyright (C) 2017-2018 Niklas Beisert
%
% This work may be distributed and/or modified under the
% conditions of the LaTeX Project Public License, either version 1.3
% of this license or (at your option) any later version.
% The latest version of this license is in
%   http://www.latex-project.org/lppl.txt
% and version 1.3 or later is part of all distributions of LaTeX
% version 2005/12/01 or later.
%
% This work has the LPPL maintenance status `maintained'.
%
% The Current Maintainer of this work is Niklas Beisert.
%
% This work consists of the files childdoc.dtx and childdoc.ins
% and the derived files childdoc.def and cdocsamp.tex with
% cdocsch1.tex, cdocsch2.tex, cdocsdrf.tex, cdocsfn1.tex, cdocsfn2.tex.
%
%<package>\ifdefined\childdocmain\endinput\fi
%<package>\ProvidesFile{childdoc.def}[2018/12/30 v2.0 child document driver]
%<samplemain>\ProvidesFile{cdocsamp.tex}[2018/12/30 v2.0 sample for childdoc]
%<*driver>
%\ProvidesFile{childdoc.drv}[2018/12/30 v2.0 childdoc reference manual file]
\PassOptionsToClass{10pt,a4paper}{article}
\documentclass{ltxdoc}

\usepackage[margin=35mm]{geometry}
\usepackage{hyperref}
\usepackage{hyperxmp}
\usepackage[usenames]{color}

\hypersetup{colorlinks=true}
\hypersetup{pdfstartview=FitH}
\hypersetup{pdfpagemode=UseNone}
\hypersetup{pdfsource={}}
\hypersetup{pdflang={en-UK}}
\hypersetup{pdfcopyright={Copyright 2017-2018 Niklas Beisert.
  This work may be distributed and/or modified under the
  conditions of the LaTeX Project Public License, either version 1.3
  of this license or (at your option) any later version.}}
\hypersetup{pdflicenseurl={http://www.latex-project.org/lppl.txt}}
\hypersetup{pdfcontactaddress={ETH Zurich, ITP, HIT K,
  Wolfgang-Pauli-Strasse 27}}
\hypersetup{pdfcontactpostcode={8093}}
\hypersetup{pdfcontactcity={Zurich}}
\hypersetup{pdfcontactcountry={Switzerland}}
\hypersetup{pdfcontactemail={nbeisert@itp.phys.ethz.ch}}
\hypersetup{pdfcontacturl={http://people.phys.ethz.ch/\xmptilde nbeisert/}}

\newcommand{\secref}[1]{\hyperref[#1]{section \ref*{#1}}}

\parskip1ex
\parindent0pt
\let\olditemize\itemize
\def\itemize{\olditemize\parskip0pt}

\begin{document}

\title{The \textsf{childdoc} Package}
\hypersetup{pdftitle={The childdoc Package}}
\author{Niklas Beisert\\[2ex]
  Institut f\"ur Theoretische Physik\\
  Eidgen\"ossische Technische Hochschule Z\"urich\\
  Wolfgang-Pauli-Strasse 27, 8093 Z\"urich, Switzerland\\[1ex]
  \href{mailto:nbeisert@itp.phys.ethz.ch}
  {\texttt{nbeisert@itp.phys.ethz.ch}}}
\hypersetup{pdfauthor={Niklas Beisert}}
\hypersetup{pdfsubject={Manual for the LaTeX2e Package childdoc}}
\date{30 December 2018, \textsf{v2.0}}
\maketitle

\begin{abstract}\noindent
\textsf{childdoc} is a \LaTeXe{} package
that enables the direct compilation
of document sections included by |\include|
to individual files.
\end{abstract}

\begingroup
\parskip0ex
\tableofcontents
\endgroup

%%%%%%%%%%%%%%%%%%%%%%%%%%%%%%%%%%%%%%%%%%%%%%%%%%%%%%%%%%%%%%%%%%%%%%%%%%%%%%%%
%%%%%%%%%%%%%%%%%%%%%%%%%%%%%%%%%%%%%%%%%%%%%%%%%%%%%%%%%%%%%%%%%%%%%%%%%%%%%%%%
\section{Introduction}

\LaTeX{} provides a mechanism to structure a large document (such as a book)
into a main file and several child files (containing the chapters)
using the |\include| command.
This mechanism is beneficial for documents
which span hundreds of pages in order to
make the source file(s) more manageable.
Moreover, compilation can be restricted to
selected child files by means of the |\includeonly| command.
The latter feature can be used to reduce the compilation time while editing
(this was significantly more useful in the earlier days of \LaTeX{})
or to generate a smaller document which is easier to navigate.
Another application of |\includeonly| is to generate
documents consisting of selected parts of the complete document.

However, there are a few drawbacks of the plain |\include| mechanism:
\begin{itemize}
\item
The child files cannot be compiled on their own,
they can only be compiled via the main file.
A naive editing environment
(such as a text editor with an option
to have the current file processed by \LaTeX)
may require one to switch to the main file before compiling;
attempting to compile the child file produces errors.
\item
The main file must be modified (each time)
to adjust the |\includeonly| command
to the present needs. This easily leaves the main file in a messy state.
\item
The generated document will always carry the filename
of the main document. This is inconvenient if
several child files are to be compiled and
to be kept for distribution.
\end{itemize}

The present package provides a simple interface
to make child files individually compilable by \LaTeX{}.
Compiling a child file then has the same effect as compiling
the main file with an |\includeonly| command
to select the appropriate child.
Moreover the generated document will carry the name of the child
rather than the main file.
This resolves all three above issues.

This feature is meant to make the editing of books,
thesis documents and lecture notes somewhat more convenient.
However, the package can also be used efficiently for
composing a series of documents (such as exercise sheets)
which are typically distributed individually.
It then assists the author in generating the individual documents
(potentially in different versions)
as well as a document containing the collected series.
Another application is in developing style files
or other kinds of included material
where compilation of the style file could redirect
to a sample or test file.

%%%%%%%%%%%%%%%%%%%%%%%%%%%%%%%%%%%%%%%%%%%%%%%%%%%%%%%%%%%%%%%%%%%%%%%%%%%%%%%%
%%%%%%%%%%%%%%%%%%%%%%%%%%%%%%%%%%%%%%%%%%%%%%%%%%%%%%%%%%%%%%%%%%%%%%%%%%%%%%%%
\section{Usage}

First of all, the package \textsf{childdoc} is \emph{not} a standard
\LaTeXe{} |.sty| style file! Therefore it needs to be invoked in
a non-standard way.

%%%%%%%%%%%%%%%%%%%%%%%%%%%%%%%%%%%%%%%%%%%%%%%%%%%%%%%%%%%%%%%%%%%%%%%%%%%%%%%%
\subsection{Included Files}
\label{sec:include}

%%%%%%%%%%%%%%%%%%%%%%%%%%%%%%%%%%%%%%%%
\DescribeMacro{\childdocmain}
To use the package, add the commands
\begin{center}
\begin{tabular}{l}
|\input{childdoc.def}|\\
|\childdocmain{}|\\
\end{tabular}
\end{center}
at the very top of the main \LaTeX{} file,
in particular \emph{before} the |\documentclass| statement!
The argument of |\childdocmain| should be left empty
(but it must be present).

%%%%%%%%%%%%%%%%%%%%%%%%%%%%%%%%%%%%%%%%
\DescribeMacro{\childdocof}
Furthermore, add the commands
\begin{center}
\begin{tabular}{l}
|\input{childdoc.def}|\\
|\childdocof{|\textit{main}|}|\\
\end{tabular}
\end{center}
at the top of every child file \textit{child}
which is included by |\include{|\textit{child}|}|
from within the main file
(or at least for those files to be compiled individually).
The argument \textit{main} must be the filename of the main file.

There are a couple of
considerations in setting up the main and child documents:

%%%%%%%%%%%%%%%%%%%%%%%%%%%%%%%%%%%%%%%%
\paragraph{Restrictions.}

Please note the following restrictions:
\begin{itemize}
\item
|\childdocmain| must be called with one argument \textit{main}
to ensure compatibility with earlier version of the package.
It must either be empty (|\childdocmain{}|)
or precisely match the filename of the main file in which it is specified.
See \secref{sec:detection} for further information.
\item
The filename \textit{main} must be specified without the |.tex| extension.
\item
The filename \textit{main} is case sensitive
(even in case-insensitive file systems)
due to internal string comparison.
\item
The argument \textit{main} should be fully expanded, it cannot be a macro.
\item
Subdirectories and special characters should be avoided in filenames.
\item
The command |\childdocmain{|\textit{main}|}| must be followed by a whitespace.
It should not be followed immediately by another command
or by a comment mark `|%|'.
This is because the \TeX{} parser reads the token immediately following
the argument of |\childdocmain| and puts it
at the beginning of every child section;
however, a white\-space is ignored.
\end{itemize}

%%%%%%%%%%%%%%%%%%%%%%%%%%%%%%%%%%%%%%%%
\paragraph{Content of Main File.}

It is advisable to place all content in the child files included by |\include|.
Any output contained in the main file will appear in all child documents
unless suppressed manually;
it cannot be suppressed automatically by the |\includeonly| directive
and thus should normally be avoided.
A method to include some content in the main file
by means of conditional processing is described in \secref{sec:conditional}.

%%%%%%%%%%%%%%%%%%%%%%%%%%%%%%%%%%%%%%%%
\paragraph{Page Numbering.}

When only a part of the document is compiled,
the appropriate numbering of pages
(as well as other status parameters)
is determined from the |.aux| files.
The latter contain information from previous passes.
However this information needs to propagate through
all intermediate child documents.
Therefore the page numbering in child documents may well
be inconsistent until the complete document is compiled at least once.

A useful (if unconventional) way to always ensure a consistent
page numbering is to restart the numbering in each child document
and denote the pages by `\textit{child}|.|\textit{page}'
where \textit{child} represents the chapter/section number of the child file.
This can be achieved by the command
|\numberwithin{page}{|\textit{child}|}|
of the \textsf{amsmath} package
where \textit{child} can be |chapter| or |section|
depending on the chosen structuring.
Alternatively, one can modify the macro |\thepage| appropriately
and reset the counter |page| at the start of each child file.

%%%%%%%%%%%%%%%%%%%%%%%%%%%%%%%%%%%%%%%%%%%%%%%%%%%%%%%%%%%%%%%%%%%%%%%%%%%%%%%%
\subsection{Conditional Processing}
\label{sec:conditional}

The package provides a mechanism to compile different versions
of a document. To customise the versions further some conditional processing
can come in handy to distinguish which version is being compiled.
The package provides two macros to describe the compilation context:

%%%%%%%%%%%%%%%%%%%%%%%%%%%%%%%%%%%%%%%%
\DescribeMacro{\ifchilddoc}
The conditional |\ifchilddoc| distinguishes between the compilation of
child documents and the main document:
%
\begin{center}
|\ifchilddoc |\textit{child-code}| |[|\||else |\textit{main-code}]| \||fi|
\end{center}

%%%%%%%%%%%%%%%%%%%%%%%%%%%%%%%%%%%%%%%%
\DescribeMacro{\childdocname}
\DescribeMacro{\childdocjob}
The macro |\childdocname| contains the filename (without extension)
of the main or child file being processed.
Note that |\childdocjob| will always contain the name of the main file.

%%%%%%%%%%%%%%%%%%%%%%%%%%%%%%%%%%%%%%%%
\paragraph{Title Page.}

Conditional processing can be used to include a title or banner page
in the main document when proper precautions are taken.
Importantly, the code in the main file should ensure that the page counter
(as well as other status parameters which are stored in the |.aux| files)
takes the same value after the conditional processing.
Otherwise the page numbers may take divergent values
depending on which part is compiled.

For example, a title page could be declared by:
%
\begin{center}
\begin{tabular}{l}
|\ifchilddoc\||else|\\
|\addtocounter{page}{-1}|\\
\textit{code for title page}\\
|\newpage|\\
|\||fi|
\end{tabular}
\end{center}
%
A banner page for the child documents can be generated by:
%
\begin{center}
\begin{tabular}{l}
|\ifchilddoc|\\
|\addtocounter{page}{-1}|\\
\textit{code for banner page}\\
|\newpage|\\
|\||fi|
\end{tabular}
\end{center}
%
Here one could write a message such as:
\begin{center}
|This is the part \childdocname{} of \childdocjob{}.|
\end{center}

%%%%%%%%%%%%%%%%%%%%%%%%%%%%%%%%%%%%%%%%%%%%%%%%%%%%%%%%%%%%%%%%%%%%%%%%%%%%%%%%
\subsection{Flags}
\label{sec:flags}

The package makes it easy to generate different versions
of the main or child documents.
To this end compilation flags can be defined
and assigned different default values.
They will be particularly useful in conjunction
with the forwarding mechanism described in \secref{sec:forward}.

For example, it may be useful to have a flag |\version|
which can be set to |draft| or |final|.
The document source will contain some conditional code
depending on the value of |\version|.
Suppose further, the flag should default to |final| for the main file
and to |draft| for child files
which is a natural assignment for editing the document.
This is achieved by placing the following code
in the preamble of the main document
(below the |\childdocmain| directive):
%
\begin{center}
\begin{tabular}{l}
|\ifchilddoc|\\
|\providecommand{\version}{draft}|\\
|\||else|\\
|\providecommand{\version}{final}|\\
|\||fi|
\end{tabular}
\end{center}
%
The definition by |\providecommand| makes sure
that previous definitions are not overwritten.
Further statements |\providecommand{\version}{...}|
can thus be added before the above code to override it.

For the main file, one might add a line
(between |\childdocmain| and the above block)
%
\begin{center}
|%\ifchilddoc\||else\providecommand{\version}{draft}\||fi|
\end{center}
%
which can be uncommented to produce a draft version.
Likewise one can add a line to the very top of a child file
(above the |\childdocof{|\textit{main}|}| directive)
%
\begin{center}
|%\providecommand{\version}{final}|
\end{center}
%
which can be uncommented to produce the final version of this child document.

%%%%%%%%%%%%%%%%%%%%%%%%%%%%%%%%%%%%%%%%%%%%%%%%%%%%%%%%%%%%%%%%%%%%%%%%%%%%%%%%
\subsection{Forwarding}
\label{sec:forward}

Different versions of the main or child documents
using compilation flags as described in \secref{sec:flags}
can be (permanently) stored in different files
for convenient compilation, viewing and distribution.
To this end, the package defines a command
to pass on compilation to a different file:

%%%%%%%%%%%%%%%%%%%%%%%%%%%%%%%%%%%%%%%%
\DescribeMacro{\childdocforward}
The command |\childdocforward| redirects processing to
another source file:
%
\begin{center}
\begin{tabular}{l}
|\input{childdoc.def}|\\
|\childdocforward[|\textit{main}|]{|\textit{dest}|}|\\
\end{tabular}
\end{center}
%
The argument \textit{dest} is the destination file
(without extension).
It should be the main file or one of the child files.
Note that further \textsf{childdoc} directives
such as |\childdocof| and |\childdocforward|
in the indicated file will be processed in this form.
The optional argument \textit{main}
passes on directly to the main file \textit{main}
while pretending to compile the child \textit{dest}.
This form behaves as if \textit{dest}
issues |\childdocof{|\textit{main}|}| right away,
and no further \textsf{childdoc} directives will be processed.

%%%%%%%%%%%%%%%%%%%%%%%%%%%%%%%%%%%%%%%%
\DescribeMacro{\...prefix}
In the alternative form |\childdocforwardprefix|,
%
\begin{center}
\begin{tabular}{l}
|\input{childdoc.def}|\\
|\childdocforwardprefix[|\textit{main}|]{|\textit{prefix}|}{|\textit{dest}|}|
\end{tabular}
\end{center}
%
the destination file is determined by a pattern
depending on the current file:
To make this work, the current file must be called
`{\textit{prefix}\hspace{0.2em}\textit{suffix}}'
with \textit{prefix} matching precisely the argument.
Processing is then passed on to the file
`{\textit{dest}\hspace{0.2em}\textit{suffix}}'.
Surely, the same effect is achieved by
directly specifying the
argument `{\textit{dest}\hspace{0.2em}\textit{suffix}}'
in the first form.
However, that requires to set up a different file
for each child. With the alternative form of the command
all these files can have exactly the same content
which simplifies setting them up and maintaining them.

For example, the following file |draft.tex|
with a compilation flag |\version| as described in \secref{sec:flags}
compiles the main document as a draft:
%
\begin{center}
\begin{tabular}{l}
|\def\version{draft}|\\
|\input{childdoc.def}|\\
|\childdocforward{|\textit{main}|}|
\end{tabular}
\end{center}
%
Likewise, the following files |final|\textit{nn}|.tex|
compile the final version of the child document
|child|\textit{nn}|.tex|:
%
\begin{center}
\begin{tabular}{l}
|\def\version{final}|\\
|\input{childdoc.def}|\\
|\childdocforwardprefix{final}{child}|
\end{tabular}
\end{center}
%

Note that when several versions of a main file and/or of each child file
are to be generated, it may be convenient to set up a |Makefile| or
shell script to automatise the process.

%%%%%%%%%%%%%%%%%%%%%%%%%%%%%%%%%%%%%%%%%%%%%%%%%%%%%%%%%%%%%%%%%%%%%%%%%%%%%%%%
\subsection{Command Line Processing}
\label{sec:commandline}

The effect of redirection files can also be achieved by invoking
the \LaTeX{} compiler with a more elaborate command line.
Most conveniently this should be done as part
of a shell script or a |Makefile|.

When using \textsf{childdoc} in the main file, the following
command lines effectively perform a redirection
(note that depending on the shell being used,
backslashes may have to be doubled: `|\|' $\to$ `|\\|'):
%
\begin{center}
|... -jobname "|\textit{target}|" |\\|"|[\textit{flags}]%
|\input{childdoc.def}\childdocforward[|\textit{main}|]{|\textit{dest}|}"|
\end{center}
%
Here \textit{target} is the name of the output file,
\textit{main} is the name of the main file
and \textit{dest} is the name of the main or child file to be processed
(all filenames without extensions).
The optional argument \textit{main} can be omitted
if \textit{main} matches \textit{dest}.
Optionally, compilation \textit{flags} can be defined via |\def| commands.
This command line makes the \TeX{} engine believe
it is compiling the file \textit{target}
whose content is specified as the latter parameter.
The provided code then forwards the processing to
\textit{main} or \textit{dest} as described in \secref{sec:forward}.

%%%%%%%%%%%%%%%%%%%%%%%%%%%%%%%%%%%%%%%%%%%%%%%%%%%%%%%%%%%%%%%%%%%%%%%%%%%%%%%%
\subsection{Include by Input}
\label{sec:input}

Including child documents by |\include| has some restrictions by design.
Most notably, the content of a child document always occupies
its own set of pages; pages cannot be shared between child documents.
Usually, this behaviour makes perfect sense
because each child document contain an essential part of the document.
However, in some situations it may be desirable to compose
a document from a collection of parts
without having mandatory page breaks between then.
For this case, the package
provides a mechanism to include parts
by |\input| which can also be processed individually.
However, by construction this mechanism
requires manual handling of the content to be output.

%%%%%%%%%%%%%%%%%%%%%%%%%%%%%%%%%%%%%%%%
\DescribeMacro{\ifchilddocmanual}
The main file should be prepared as usual, see \secref{sec:include}.
However, the document body must make a distinction
between processing of an individual part and of the main document, e.g.:
%
\begin{center}
\begin{tabular}{l}
|\ifchilddocmanual|\\
|\input{\childdocname}|\\
|\||else|\\
\textit{document body with }|\input{|\textit{part}|}|\\
|\||fi|
\end{tabular}
\end{center}
%
The conditional |\ifchilddocmanual| is true whenever
a part to be included by |\input| is being compiled,
and the name of the part is stored in |\childdocname|.

%%%%%%%%%%%%%%%%%%%%%%%%%%%%%%%%%%%%%%%%
\DescribeMacro{\childdocby}
Each part to be included by |\input| should start with:
%
\begin{center}
\begin{tabular}{l}
|\input{childdoc.def}|\\
|\childdocby{|\textit{main}|}|\\
\end{tabular}
\end{center}
%
The directive |\childdocby| is similar to |\childdocof|
described in \secref{sec:include},
but the subsequent selection of content must be done manually.
To that end, both |\ifchilddoc| and |\ifchilddocmanual|
will be true upon processing of a part,
and the name of the part is stored in |\childdocname|.
Note that |\jobname| will be set to the filename of the current part
so that each part receives an individual |.aux| file
that does not interfere with the |.aux| file(s) of the main document.
This behaviour can be altered by the alternative form
|\childdocby[*]{|\textit{main}|}| (with a non-empty optional argument)
which uses the |.aux| file of the main document
by setting |\jobname| to \textit{main}.

%%%%%%%%%%%%%%%%%%%%%%%%%%%%%%%%%%%%%%%%%%%%%%%%%%%%%%%%%%%%%%%%%%%%%%%%%%%%%%%%
\subsection{Driver Development}
\label{sec:driver}

The \textsf{childdoc} mechanism can also be use for the development
of definition files such as \LaTeX{} styles or classes.
This case differs from the above setup with multiple parts
included by |\include| in that no |\includeonly| should be invoked.
This can be achieved by starting the include file
(before |\ProvidesPackage|) with:
%
\begin{center}
\begin{tabular}{l}
|\input{childdoc.def}|\\
|\childdocforward{|\textit{main}|}|\\
\end{tabular}
\end{center}
%
or alternatively with:
%
\begin{center}
\begin{tabular}{l}
|\input{childdoc.def}|\\
|\childdocby{|\textit{main}|}|\\
\end{tabular}
\end{center}
%
Both forms have slightly different effects as described above.
The main file is prepared as usual, see \secref{sec:include}.

%%%%%%%%%%%%%%%%%%%%%%%%%%%%%%%%%%%%%%%%%%%%%%%%%%%%%%%%%%%%%%%%%%%%%%%%%%%%%%%%
\subsection{Legacy Detection}
\label{sec:detection}

The directive |\childdocmain| in the main file can detect
whether the complete document or merely a child is to be compiled
even without using the directive |\childdocof|.
This method is deprecated because it is less robust
and there is no compelling reason to use it;
it is merely provided for backward compatibility
and it may be removed in future versions.

If the detection mechanism is to be used,
it is mandatory to correctly specify
the filename of the main file as the argument of |\childdocmain|:
%
\begin{center}
\begin{tabular}{l}
|\input{childdoc.def}|\\
|\childdocmain{|\textit{main}|}|\\
\end{tabular}
\end{center}
%
If |\jobname| does not match the argument \textit{main} of |\childdocmain|,
it is assumed that |\jobname| points to the child file to be compiled.
When using |\childdocmain| with the main file specified as argument,
it suffices to start a child file
with just |\input{|\textit{main}|}|
without loading of the package and using |\childdocof|.
If instead all processing is done
with the appropriate \textsf{childdoc} directives,
the argument of \textit{main} of |\childdocmain| can be empty.

An alternative version of the command line processing described
in \secref{sec:commandline} using the detection mechanism reads:
%
\begin{center}
|... -jobname "|\textit{target}|" "|[\textit{flags}]%
[|\def\jobname{|\textit{dest}|}|]|\input{|\textit{main}|}"|
\end{center}

%%%%%%%%%%%%%%%%%%%%%%%%%%%%%%%%%%%%%%%%%%%%%%%%%%%%%%%%%%%%%%%%%%%%%%%%%%%%%%%%
\subsection{Manual Code}
\label{sec:manual}

In case one cannot be certain whether the definitions file |childdoc.def|
is installed on the target \TeX{} distribution
and one prefers not to ship it,
it is conceivable to paste a few relevant commands into the sources.

To that end, drop all statements |\input{childdoc.def}|
and perform the replacements as outlined below.
Instead of |\childdocmain{|\textit{main}|}| add the following code
to the top of the main file:
%
\begin{center}
\begin{tabular}{l}
|\||ifdefined\childdocname\endinput\||fi\newif\ifchilddoc|\\
|\edef\childdocname{\scantokens\expandafter{\jobname\noexpand}}|\\
|\def\childdocmain{|\textit{main}|}\||ifx\childdocmain\childdocname\||else|\\
|\childdoctrue\includeonly{\childdocname}\let\jobname\childdocmain\||fi|\\
\end{tabular}
\end{center}
%
Instead of |\childdocof{|\textit{main}|}| just include the main file
at the top of each child file:
%
\begin{center}
|\input{|\textit{main}|}|
\end{center}
%
A simple redirection |\childdocforward{|\textit{dest}|}| is achieved by:
%
\begin{center}
|\def\jobname{|\textit{dest}|}\input{\jobname}|
\end{center}
%
The redirection with prefix
|\childdocforwardprefix[|\textit{prefix}|]{|\textit{dest}|}|
is accomplished by:
%
\begin{center}
\begin{tabular}{l}
|{\edef\jobname{\scantokens\expandafter{\jobname\noexpand}}|\\
|\def\redirectjob |\textit{prefix}|#1~~~{\gdef\jobname{|\textit{dest}|#1}}|\\
|\expandafter\redirectjob\jobname~~~}\input{\jobname}|
\end{tabular}
\end{center}

In an alternative approach,
child documents can be compiled by a specific command line
without additional code or specific definitions:
%
\begin{center}
|... -jobname "|\textit{target}|" "|[\textit{flags}]%
|\includeonly{|\textit{dest}|}\input{|\textit{main}|}"|
\end{center}
%

%%%%%%%%%%%%%%%%%%%%%%%%%%%%%%%%%%%%%%%%%%%%%%%%%%%%%%%%%%%%%%%%%%%%%%%%%%%%%%%%
%%%%%%%%%%%%%%%%%%%%%%%%%%%%%%%%%%%%%%%%%%%%%%%%%%%%%%%%%%%%%%%%%%%%%%%%%%%%%%%%
\section{Information}

%%%%%%%%%%%%%%%%%%%%%%%%%%%%%%%%%%%%%%%%%%%%%%%%%%%%%%%%%%%%%%%%%%%%%%%%%%%%%%%%
\subsection{Copyright}

Copyright \copyright{} 2017--2018 Niklas Beisert

This work may be distributed and/or modified under the
conditions of the \LaTeX{} Project Public License, either version 1.3
of this license or (at your option) any later version.
The latest version of this license is in
  \url{http://www.latex-project.org/lppl.txt}
and version 1.3 or later is part of all distributions of \LaTeX{}
version 2005/12/01 or later.

This work has the LPPL maintenance status `maintained'.

The Current Maintainer of this work is Niklas Beisert.

This work consists of the files |README.txt|, |childdoc.ins| and |childdoc.dtx|
as well as the derived files |childdoc.def|, |cdocsamp.tex|
with |cdocsch1.tex|, |cdocsch2.tex|, |cdocspt3.tex|, |cdocspt4.tex|,
|cdocsdrf.tex|, |cdocsfn1.tex|, |cdocsfn2.tex|
as well as |childdoc.pdf|.

%%%%%%%%%%%%%%%%%%%%%%%%%%%%%%%%%%%%%%%%%%%%%%%%%%%%%%%%%%%%%%%%%%%%%%%%%%%%%%%%
\subsection{Files and Installation}

The package consists of the files:
%
\begin{center}
\begin{tabular}{ll}
    |README.txt|   & readme file \\
    |childdoc.ins| & installation file \\
    |childdoc.dtx| & source file \\
    |childdoc.def| & definition file \\
    |cdocsamp.tex| & sample main file \\
    |cdocsch1.tex| & sample include file \\
    |cdocsch2.tex| & sample include file \\
    |cdocspt3.tex| & sample part file \\
    |cdocspt4.tex| & sample part file \\
    |cdocsdrf.tex| & sample redirection file \\
    |cdocsfn1.tex| & sample redirection file \\
    |cdocsfn2.tex| & sample redirection file \\
    |childdoc.pdf| & manual
\end{tabular}
\end{center}
%
The distribution consists of the files
|README.txt|, |childdoc.ins| and |childdoc.dtx|.
%
\begin{itemize}
\item
Run (pdf)\LaTeX{} on |childdoc.dtx|
to compile the manual |childdoc.pdf| (this file).
\item
Run \LaTeX{} on |childdoc.ins| to create the definitions file |childdoc.def|
and the sample |cdocsamp.tex| with include files
|cdocsch1.tex|, |cdocsch2.tex|, |cdocspt3.tex|, |cdocspt4.tex|,
|cdocsdrf.tex|, |cdocsfn1.tex|, |cdocsfn2.tex|.
Then copy the file |childdoc.def| to an appropriate directory of your \LaTeX{}
distribution, e.g.\ \textit{texmf-root}|/tex/latex/childdoc|.
\end{itemize}

%%%%%%%%%%%%%%%%%%%%%%%%%%%%%%%%%%%%%%%%%%%%%%%%%%%%%%%%%%%%%%%%%%%%%%%%%%%%%%%%
\subsection{Related CTAN Packages}

There are several other packages which offer a similar functionality:
%
\begin{itemize}
\item
The packages
\href{http://ctan.org/pkg/docmute}{\textsf{docmute}},
\href{http://ctan.org/pkg/includex}{\textsf{includex}} and
\href{http://ctan.org/pkg/standalone}{\textsf{standalone}}
provide commands to include only the document body of
a child file thus allowing both files to be compiled individually.
\item
The packages \href{http://ctan.org/pkg/subdocs}{\textsf{subdocs}}
and \href{http://ctan.org/pkg/subfiles}{\textsf{subfiles}}
provide structures in which the main and child documents can be
encapsulated and allowing them to be compiled individually.
The inclusion mechanism is different from the conventional |\include|.
\item
The package \href{http://ctan.org/pkg/combine}{\textsf{combine}}
is an elaborate solution to combine several documents into one.
\end{itemize}
%
See also the CTAN topic \href{http://ctan.org/topic/subdocs}{\textsf{subdocs}}
for further related packages.
The present package differs from the above solutions in that
a document structure constructed with the conventional |\include| mechanism
just needs two extra commands at the top of every file
such that all constituent files can be compiled individually.

%%%%%%%%%%%%%%%%%%%%%%%%%%%%%%%%%%%%%%%%%%%%%%%%%%%%%%%%%%%%%%%%%%%%%%%%%%%%%%%%
%\subsection{Feature Suggestions}
%
%The following is a list of features which may be useful for future
%versions of this package:
%%
%\begin{itemize}
%\item
%\ldots
%\end{itemize}

%%%%%%%%%%%%%%%%%%%%%%%%%%%%%%%%%%%%%%%%%%%%%%%%%%%%%%%%%%%%%%%%%%%%%%%%%%%%%%%%
\subsection{Revision History}

%%%%%%%%%%%%%%%%%%%%%%%%%%%%%%%%%%%%%%%%
\paragraph{v2.0:} 2018/12/30

\begin{itemize}
\item
immediate forward processing
\item
added |\childdocby| mechanism
\item
manual restructured
\end{itemize}

%%%%%%%%%%%%%%%%%%%%%%%%%%%%%%%%%%%%%%%%
\paragraph{v1.6:} 2018/01/17

\begin{itemize}
\item
application for development of include files
\item
corrections to manual
\end{itemize}

%%%%%%%%%%%%%%%%%%%%%%%%%%%%%%%%%%%%%%%%
\paragraph{v1.5:} 2017/05/21

\begin{itemize}
\item
more complete structuring introduced
\item
|\childdocof| introduced
\item
|\childdoc| renamed to |\childdocmain|
\item
|\childredirect| renamed to |\childdocforward| and |\childdocforwardprefix|
and functionality expanded
\end{itemize}

%%%%%%%%%%%%%%%%%%%%%%%%%%%%%%%%%%%%%%%%
\paragraph{v1.0:} 2017/04/27

\begin{itemize}
\item
manual and install package
\item
first version published on CTAN
\end{itemize}

%%%%%%%%%%%%%%%%%%%%%%%%%%%%%%%%%%%%%%%%
\paragraph{v0.6:} 2017/04/26

\begin{itemize}
\item
redirection mechanism added
\end{itemize}

%%%%%%%%%%%%%%%%%%%%%%%%%%%%%%%%%%%%%%%%
\paragraph{v0.5:} 2017/04/26

\begin{itemize}
\item
functionality in definition file
\end{itemize}


%%%%%%%%%%%%%%%%%%%%%%%%%%%%%%%%%%%%%%%%%%%%%%%%%%%%%%%%%%%%%%%%%%%%%%%%%%%%%%%%
%%%%%%%%%%%%%%%%%%%%%%%%%%%%%%%%%%%%%%%%%%%%%%%%%%%%%%%%%%%%%%%%%%%%%%%%%%%%%%%%
%%%%%%%%%%%%%%%%%%%%%%%%%%%%%%%%%%%%%%%%%%%%%%%%%%%%%%%%%%%%%%%%%%%%%%%%%%%%%%%%
\appendix

\settowidth\MacroIndent{\rmfamily\scriptsize 000\ }

 \DocInput{childdoc.dtx}

\end{document}
%</driver>
% \fi
%
% %%%%%%%%%%%%%%%%%%%%%%%%%%%%%%%%%%%%%%%%%%%%%%%%%%%%%%%%%%%%%%%%%%%%%%%%%%%%%%
% %%%%%%%%%%%%%%%%%%%%%%%%%%%%%%%%%%%%%%%%%%%%%%%%%%%%%%%%%%%%%%%%%%%%%%%%%%%%%%
% \section{Sample}
%\iffalse
%<*samplemain>
%\fi
%
% The following presents a sample document
% with two chapters, two parts, a title page,
% a compile flag as well as three forwarding files to set the flag.
% It consists of eight |.tex| files:
% \begin{center}
% \begin{tabular}{ll}
% |cdocsamp.tex|&main file\\
% |cdocsch1.tex|&include file for chapter 1\\
% |cdocsch2.tex|&include file for chapter 2\\
% |cdocspt3.tex|&include file for part 3\\
% |cdocspt4.tex|&include file for part 4\\
% |cdocsdrf.tex|&forwarding file for main file in draft mode\\
% |cdocsfi1.tex|&forwarding file for final version of chapter 1\\
% |cdocsfi2.tex|&forwarding file for final version of chapter 2\\
% \end{tabular}
% \end{center}
% Each of the eight files can be compiled directly by the \LaTeX{} compiler.
%
% %%%%%%%%%%%%%%%%%%%%%%%%%%%%%%%%%%%%%%
% \paragraph{Main File.}
%
% The main file is called |cdocsamp.tex|.
%
% Load the \textsf{childdoc} definitions and
% declare the filename for the main document:
%    \begin{macrocode}
\input{childdoc.def}
\childdocmain{}
%    \end{macrocode}

% Optional override for |\version| flag:
%    \begin{macrocode}
%%\ifchilddoc\else\providecommand{\version}{draft}\fi
%    \end{macrocode}

% Define the default values for the |\version| flag
% (|final| for the main file and |draft| for childs):
%    \begin{macrocode}
\ifchilddoc
\providecommand{\version}{draft}
\else
\providecommand{\version}{final}
\fi
%    \end{macrocode}

% Load the standard document class:
%    \begin{macrocode}
\documentclass[12pt]{article}
%    \end{macrocode}

% Start the document body:
%    \begin{macrocode}
\begin{document}
%    \end{macrocode}

% Declare a title page.
% Print title, part of document being processed and version flag:
%    \begin{macrocode}
\addtocounter{page}{-1}
\begin{center}
{\LARGE\bfseries{}childdoc example\par}
\vspace{1cm}
\ifchilddoc
\ifchilddocmanual part\else chapter\fi:
`\childdocname' of `\childdocjob'\par
\else
main document: `\childdocjob'\par
\fi
version: \version\par
\end{center}
\newpage
%    \end{macrocode}

% Manually include selected file,
% otherwise process as usual:
%    \begin{macrocode}
\ifchilddocmanual
\section*{part `\childdocname'}
\input{\childdocname}
\else
%    \end{macrocode}

% Include the two chapters:
%    \begin{macrocode}
\include{cdocsch1}
\include{cdocsch2}
%    \end{macrocode}

% Include the two parts unless only chapters should be displayed:
%    \begin{macrocode}
\ifchilddoc\else
\section{part three}
\input{cdocspt3}
\section{part four}
\input{cdocspt4}
\fi
%    \end{macrocode}

% Process as usual until here:
%    \begin{macrocode}
\fi
%    \end{macrocode}

% End of document body:
%    \begin{macrocode}
\end{document}
%    \end{macrocode}
%\iffalse
%</samplemain>
%\fi
%
% %%%%%%%%%%%%%%%%%%%%%%%%%%%%%%%%%%%%%%
% \paragraph{Chapter Include Files.}
%
% The include files are called |cdocsch1.tex| and |cdocsch2.tex|.
%
%\iffalse
%<*samplechap1|samplechap2>
%\fi

% Optional override for |\version| flag:
%    \begin{macrocode}
%%\providecommand{\version}{final}
%    \end{macrocode}

% Include the main document:
%    \begin{macrocode}
\input{childdoc.def}
\childdocof{cdocsamp}
%    \end{macrocode}

%\iffalse
%</samplechap1|samplechap2>
%\fi
%
%\iffalse
%<*samplechap1>
%\fi
% Some text for chapter 1:
%    \begin{macrocode}
\section{one}
some text in chapter one
%    \end{macrocode}

%\iffalse
%</samplechap1>
%\fi
% Some text for chapter 2:
%\iffalse
%<*samplechap2>
%\fi
%    \begin{macrocode}
\section{two}
more text in chapter two
%    \end{macrocode}

%\iffalse
%</samplechap2>
%\fi
%
% %%%%%%%%%%%%%%%%%%%%%%%%%%%%%%%%%%%%%%
% \paragraph{Part Include Files.}
%
% The include files are called |cdocspt3.tex| and |cdocspt4.tex|.
%
%\iffalse
%<*samplepart3|samplepart4>
%\fi

% Optional override for |\version| flag:
%    \begin{macrocode}
%%\providecommand{\version}{final}
%    \end{macrocode}

% Include the main document:
%    \begin{macrocode}
\input{childdoc.def}
\childdocby{cdocsamp}
%    \end{macrocode}

%\iffalse
%</samplepart3|samplepart4>
%\fi
%
%\iffalse
%<*samplepart3>
%\fi
% Some text for part 3:
%    \begin{macrocode}
some text in part three
%    \end{macrocode}

%\iffalse
%</samplepart3>
%\fi
% Some text for part 4:
%\iffalse
%<*samplepart4>
%\fi
%    \begin{macrocode}
more text in part four
%    \end{macrocode}

%\iffalse
%</samplepart4>
%\fi
%
% %%%%%%%%%%%%%%%%%%%%%%%%%%%%%%%%%%%%%%
% \paragraph{Forwarding for a Complete Draft.}
%
% The following forwarding file |cdocsdrf.tex|
% compiles the main document in draft mode:
%\iffalse
%<*sampledraft>
%\fi
%    \begin{macrocode}
\def\version{draft}
\input{childdoc.def}
\childdocforward{cdocsamp}
%    \end{macrocode}

%\iffalse
%</sampledraft>
%\fi
%
% %%%%%%%%%%%%%%%%%%%%%%%%%%%%%%%%%%%%%%
% \paragraph{Forwarding for Final Version of the Chapters.}
%
% The following forwarding files |cdocsfn1.tex| and |cdocsfn2.tex|
% (with identical content)
% compile the final versions of the child documents
% |cdocsch1.tex| and |cdocsch2.tex|, respectively:
%\iffalse
%<*samplefinal>
%\fi
%    \begin{macrocode}
\def\version{final}
\input{childdoc.def}
\childdocforwardprefix[cdocsamp]{cdocsfn}{cdocsch}
%    \end{macrocode}

%\iffalse
%</samplefinal>
%\fi
%
% %%%%%%%%%%%%%%%%%%%%%%%%%%%%%%%%%%%%%%
% \paragraph{Command Line Processing.}
%
% The following three command lines generate the output files
% |cdocscld|, |cdocscl1| and |cdocscl2|
% which should be identical to
% |cdocsdrf|, |cdocsch1| and |cdocsfn2|, respectively:
% \begin{center}
% \begin{tabular}{l}
% |latex -jobname cdocscld \|\\
% |  "\def\version{draft}\input{childdoc.def}\childdocforward{cdocsamp}"|\\
% |latex -jobname cdocscl1 \|\\
% |  "\input{childdoc.def}\childdocforward[cdocsamp]{cdocsch1}"|\\
% |latex -jobname cdocscl2 \|\\
% |  "\def\version{final}\input{childdoc.def}\childdocforward{cdocsch2}"|
% \end{tabular}
% \end{center}
% Note that the trailing backslash on each first line
% merely continues the input to the second line
% (for convenient cut ant paste).
% Furthermore, the command |latex| can be replaced by any
% of its alternative versions such as |pdflatex|.
%
% %%%%%%%%%%%%%%%%%%%%%%%%%%%%%%%%%%%%%%%%%%%%%%%%%%%%%%%%%%%%%%%%%%%%%%%%%%%%%%
% %%%%%%%%%%%%%%%%%%%%%%%%%%%%%%%%%%%%%%%%%%%%%%%%%%%%%%%%%%%%%%%%%%%%%%%%%%%%%%
% \section{Implementation}
%\iffalse
%<*package>
%\fi
%
% This section describes the definitions file |childdoc.def|.

% The definitions cannot be loaded using |\usepackage| or |\RequirePackage|
% which has a mechanism to prevent loading a style file more than once.
% When loading the definitions by means of |\input|
% multiple instances have to be prevented manually:
%\iffalse
%This code needs to be before the `\ProvidesFile' directive
%which is defined at the beginning of this file.
%Therefore it is also placed there and commented out here.
%</package>
%<*discard>
%\fi
%    \begin{macrocode}
\ifdefined\childdocmain\endinput\fi
%    \end{macrocode}
%\iffalse
%</discard>
%<*package>
%\fi
%
% \macro{\ifchilddoc}
% \macro{\ifchilddocmanual}
% The conditional |\ifchilddoc| tells whether a
% child (true) or main (false) document is being compiled.
% The conditional |\ifchilddocmanual| tells whether
% the |\includeonly| mechanism is used (false) or
% the selection of child files must be performed manually (true).
% The definitions initialise to false:
%    \begin{macrocode}
\newif\ifchilddoc
\newif\ifchilddocmanual
%    \end{macrocode}

% \macro{\childdocname}
% \macro{\childdocjob}
% The macro |\childdocname| stores the name of the main document
% to be compiled. The macro |\childdocjob| stores the name of
% the document on which the \LaTeX{} compiler was originally invoked.
% The content of |\jobname| cannot be compared
% to filenames specified in the source due to different catcodes.
% The following code rescans |\jobname|, stores the result
% in |\childdocname| and saves a copy in |\childdocjob|:
%    \begin{macrocode}
\edef\childdocname{\scantokens\expandafter{\jobname\noexpand}}
\let\childdocjob\childdocname
%    \end{macrocode}

% \macro{\childdocdisable}
% The macro |\childdocdisable| prevents the main file
% from being processed more than once.
% At this stage, the main document command |\childdocmain|
% is assumed to be called once again where it should do nothing.
% Any subsequent call to it should prevent
% a secondary processing of the main document
% It overwrites the forwarding commands
% |\childdocof| and |\childdocforward|
% with empty macros to prevent further inclusions of the main document:
%    \begin{macrocode}
\newcommand{\childdocdisable}
{
  \renewcommand{\childdocmain}[1]{\renewcommand{\childdocmain}[1]{\endinput}}
  \renewcommand{\childdocof}[1]{}
  \renewcommand{\childdocby}[2][]{}
  \renewcommand{\childdocforward}[2][]{}
  \renewcommand{\childdocdisable}{}
}
%    \end{macrocode}

% \macro{\childdocmain}
% The macro |\childdocmain| is to be called at the top of the main file
% with nothing or the main filename (without extension) as argument.
% First, it breaks loops.
% If the argument is not empty and does not match |\childdocname|
% (which is set by the first inclusion of |childdoc.def|),
% |\ifchilddoc| is set to true, |\includeonly| is applied to the child file
% and |\jobname| is set to the main file
% (for proper handling of |.aux| files):
%    \begin{macrocode}
\newcommand{\childdocmain}[1]
{
  \childdocdisable\childdocmain{}
  \if?#1?\else
    \begingroup
      \def\childdoctmp{#1}
      \ifx\childdoctmp\childdocname
        \def\childdoctmp{}
      \else
        \def\childdoctmp
        {
          \childdoctrue
          \includeonly{\childdocname}
          \def\childdocjob{#1}
          \def\jobname{#1}
        }
      \fi
      \expandafter
    \endgroup
    \childdoctmp
  \fi
}
%    \end{macrocode}

% \macro{\childdocof}
% The command |\childdocof| redirects
% compilation to the main file |#1|.
%    \begin{macrocode}
\newcommand{\childdocof}[1]
{
  \childdocdisable
  \childdoctrue
  \includeonly{\childdocname}
  \def\jobname{#1}
  \def\childdocjob{#1}
  \input{#1}
}
%    \end{macrocode}

% \macro{\childdocby}
% The command |\childdocby| ....
%    \begin{macrocode}
\newcommand{\childdocby}[2][]
{
  \childdocdisable
  \childdoctrue
  \childdocmanualtrue
  \if?#1?\else
    \def\jobname{#2}
  \fi
  \def\childdocjob{#2}
  \input{#2}
  \endinput
}
%    \end{macrocode}

% \macro{\childdocforward}
% The command |\childdocforward| redirects
% compilation to the main file or
% (if the optional argument is given) a child file.
% Parameters are set as if the main file
% or a child file starting with |\childdocof| was compiled.
% Then compilation is handed over to the main file:
%    \begin{macrocode}
\newcommand{\childdocforward}[2][]
{
  \begingroup
    \if?#1?
      \def\childdoctmp
      {
        \def\childdocname{#2}
        \def\childdocjob{#2}
        \def\jobname{#2}
        \input{#2}
        \endinput
      }
    \else
      \def\childdoctmp
      {
        \childdocdisable
        \def\childdocname{#2}
        \childdoctrue
        \includeonly{#2}
        \def\childdocjob{#1}
        \def\jobname{#1}
        \input{#1}
        \endinput
      }
    \fi
    \expandafter
  \endgroup
  \childdoctmp
}
%    \end{macrocode}

% \macro{\childdocforwardprefix}
% The command |\childdocforwardprefix| redirects
% compilation to the main or a child file by means of a pattern.
% The prefix |#1| in the current filename is replaced by |#2|
% and the suffix of the current filename is kept
% (it is assumed that the filename does not contain the substring `|~~~|'
% which is used as a delimiter).
% Compilation is handed over to the new file by |\childdocforward|:
%    \begin{macrocode}
\newcommand{\childdocforwardprefix}[3][]
{
  \begingroup
    \def\childdocextract #2##1~~~{\def\childdoctmp{\childdocforward[#1]{#3##1}}}
    \expandafter\childdocextract\childdocname~~~
    \expandafter
  \endgroup
  \childdoctmp
}
%    \end{macrocode}

% \macro{\childdoc}
% The deprecated macro |\childdoc| is a legacy version of |\childdocmain|:
%    \begin{macrocode}
\newcommand{\childdoc}{\childdocmain}
%    \end{macrocode}

% \macro{\childdocredirect}
% The deprecated macro |\childdocredirect| is a legacy version
% of |\childdocforward| and |\childdocforwardprefix|:
%    \begin{macrocode}
\newcommand{\childdocredirect}[2][]
{
  \begingroup
    \if?#1?
      \def\childdoctmp{\childdocforward{#2}}
    \else
      \def\childdoctmp{\childdocforwardprefix{#1}{#2}}
    \fi
    \expandafter
  \endgroup
  \childdoctmp
}
%    \end{macrocode}

%\iffalse
%</package>
%\fi
%
\endinput
\childdocforward{cdocsch2}"|
% \end{tabular}
% \end{center}
% Note that the trailing backslash on each first line
% merely continues the input to the second line
% (for convenient cut ant paste).
% Furthermore, the command |latex| can be replaced by any
% of its alternative versions such as |pdflatex|.
%
% %%%%%%%%%%%%%%%%%%%%%%%%%%%%%%%%%%%%%%%%%%%%%%%%%%%%%%%%%%%%%%%%%%%%%%%%%%%%%%
% %%%%%%%%%%%%%%%%%%%%%%%%%%%%%%%%%%%%%%%%%%%%%%%%%%%%%%%%%%%%%%%%%%%%%%%%%%%%%%
% \section{Implementation}
%\iffalse
%<*package>
%\fi
%
% This section describes the definitions file |childdoc.def|.

% The definitions cannot be loaded using |\usepackage| or |\RequirePackage|
% which has a mechanism to prevent loading a style file more than once.
% When loading the definitions by means of |\input|
% multiple instances have to be prevented manually:
%\iffalse
%This code needs to be before the `\ProvidesFile' directive
%which is defined at the beginning of this file.
%Therefore it is also placed there and commented out here.
%</package>
%<*discard>
%\fi
%    \begin{macrocode}
\ifdefined\childdocmain\endinput\fi
%    \end{macrocode}
%\iffalse
%</discard>
%<*package>
%\fi
%
% \macro{\ifchilddoc}
% \macro{\ifchilddocmanual}
% The conditional |\ifchilddoc| tells whether a
% child (true) or main (false) document is being compiled.
% The conditional |\ifchilddocmanual| tells whether
% the |\includeonly| mechanism is used (false) or
% the selection of child files must be performed manually (true).
% The definitions initialise to false:
%    \begin{macrocode}
\newif\ifchilddoc
\newif\ifchilddocmanual
%    \end{macrocode}

% \macro{\childdocname}
% \macro{\childdocjob}
% The macro |\childdocname| stores the name of the main document
% to be compiled. The macro |\childdocjob| stores the name of
% the document on which the \LaTeX{} compiler was originally invoked.
% The content of |\jobname| cannot be compared
% to filenames specified in the source due to different catcodes.
% The following code rescans |\jobname|, stores the result
% in |\childdocname| and saves a copy in |\childdocjob|:
%    \begin{macrocode}
\edef\childdocname{\scantokens\expandafter{\jobname\noexpand}}
\let\childdocjob\childdocname
%    \end{macrocode}

% \macro{\childdocdisable}
% The macro |\childdocdisable| prevents the main file
% from being processed more than once.
% At this stage, the main document command |\childdocmain|
% is assumed to be called once again where it should do nothing.
% Any subsequent call to it should prevent
% a secondary processing of the main document
% It overwrites the forwarding commands
% |\childdocof| and |\childdocforward|
% with empty macros to prevent further inclusions of the main document:
%    \begin{macrocode}
\newcommand{\childdocdisable}
{
  \renewcommand{\childdocmain}[1]{\renewcommand{\childdocmain}[1]{\endinput}}
  \renewcommand{\childdocof}[1]{}
  \renewcommand{\childdocby}[2][]{}
  \renewcommand{\childdocforward}[2][]{}
  \renewcommand{\childdocdisable}{}
}
%    \end{macrocode}

% \macro{\childdocmain}
% The macro |\childdocmain| is to be called at the top of the main file
% with nothing or the main filename (without extension) as argument.
% First, it breaks loops.
% If the argument is not empty and does not match |\childdocname|
% (which is set by the first inclusion of |childdoc.def|),
% |\ifchilddoc| is set to true, |\includeonly| is applied to the child file
% and |\jobname| is set to the main file
% (for proper handling of |.aux| files):
%    \begin{macrocode}
\newcommand{\childdocmain}[1]
{
  \childdocdisable\childdocmain{}
  \if?#1?\else
    \begingroup
      \def\childdoctmp{#1}
      \ifx\childdoctmp\childdocname
        \def\childdoctmp{}
      \else
        \def\childdoctmp
        {
          \childdoctrue
          \includeonly{\childdocname}
          \def\childdocjob{#1}
          \def\jobname{#1}
        }
      \fi
      \expandafter
    \endgroup
    \childdoctmp
  \fi
}
%    \end{macrocode}

% \macro{\childdocof}
% The command |\childdocof| redirects
% compilation to the main file |#1|.
%    \begin{macrocode}
\newcommand{\childdocof}[1]
{
  \childdocdisable
  \childdoctrue
  \includeonly{\childdocname}
  \def\jobname{#1}
  \def\childdocjob{#1}
  \input{#1}
}
%    \end{macrocode}

% \macro{\childdocby}
% The command |\childdocby| ....
%    \begin{macrocode}
\newcommand{\childdocby}[2][]
{
  \childdocdisable
  \childdoctrue
  \childdocmanualtrue
  \if?#1?\else
    \def\jobname{#2}
  \fi
  \def\childdocjob{#2}
  \input{#2}
  \endinput
}
%    \end{macrocode}

% \macro{\childdocforward}
% The command |\childdocforward| redirects
% compilation to the main file or
% (if the optional argument is given) a child file.
% Parameters are set as if the main file
% or a child file starting with |\childdocof| was compiled.
% Then compilation is handed over to the main file:
%    \begin{macrocode}
\newcommand{\childdocforward}[2][]
{
  \begingroup
    \if?#1?
      \def\childdoctmp
      {
        \def\childdocname{#2}
        \def\childdocjob{#2}
        \def\jobname{#2}
        \input{#2}
        \endinput
      }
    \else
      \def\childdoctmp
      {
        \childdocdisable
        \def\childdocname{#2}
        \childdoctrue
        \includeonly{#2}
        \def\childdocjob{#1}
        \def\jobname{#1}
        \input{#1}
        \endinput
      }
    \fi
    \expandafter
  \endgroup
  \childdoctmp
}
%    \end{macrocode}

% \macro{\childdocforwardprefix}
% The command |\childdocforwardprefix| redirects
% compilation to the main or a child file by means of a pattern.
% The prefix |#1| in the current filename is replaced by |#2|
% and the suffix of the current filename is kept
% (it is assumed that the filename does not contain the substring `|~~~|'
% which is used as a delimiter).
% Compilation is handed over to the new file by |\childdocforward|:
%    \begin{macrocode}
\newcommand{\childdocforwardprefix}[3][]
{
  \begingroup
    \def\childdocextract #2##1~~~{\def\childdoctmp{\childdocforward[#1]{#3##1}}}
    \expandafter\childdocextract\childdocname~~~
    \expandafter
  \endgroup
  \childdoctmp
}
%    \end{macrocode}

% \macro{\childdoc}
% The deprecated macro |\childdoc| is a legacy version of |\childdocmain|:
%    \begin{macrocode}
\newcommand{\childdoc}{\childdocmain}
%    \end{macrocode}

% \macro{\childdocredirect}
% The deprecated macro |\childdocredirect| is a legacy version
% of |\childdocforward| and |\childdocforwardprefix|:
%    \begin{macrocode}
\newcommand{\childdocredirect}[2][]
{
  \begingroup
    \if?#1?
      \def\childdoctmp{\childdocforward{#2}}
    \else
      \def\childdoctmp{\childdocforwardprefix{#1}{#2}}
    \fi
    \expandafter
  \endgroup
  \childdoctmp
}
%    \end{macrocode}

%\iffalse
%</package>
%\fi
%
\endinput
\childdocforward{cdocsamp}"|\\
% |latex -jobname cdocscl1 \|\\
% |  "% \iffalse
%
% childdoc.dtx Copyright (C) 2017-2018 Niklas Beisert
%
% This work may be distributed and/or modified under the
% conditions of the LaTeX Project Public License, either version 1.3
% of this license or (at your option) any later version.
% The latest version of this license is in
%   http://www.latex-project.org/lppl.txt
% and version 1.3 or later is part of all distributions of LaTeX
% version 2005/12/01 or later.
%
% This work has the LPPL maintenance status `maintained'.
%
% The Current Maintainer of this work is Niklas Beisert.
%
% This work consists of the files childdoc.dtx and childdoc.ins
% and the derived files childdoc.def and cdocsamp.tex with
% cdocsch1.tex, cdocsch2.tex, cdocsdrf.tex, cdocsfn1.tex, cdocsfn2.tex.
%
%<package>\ifdefined\childdocmain\endinput\fi
%<package>\ProvidesFile{childdoc.def}[2018/12/30 v2.0 child document driver]
%<samplemain>\ProvidesFile{cdocsamp.tex}[2018/12/30 v2.0 sample for childdoc]
%<*driver>
%\ProvidesFile{childdoc.drv}[2018/12/30 v2.0 childdoc reference manual file]
\PassOptionsToClass{10pt,a4paper}{article}
\documentclass{ltxdoc}

\usepackage[margin=35mm]{geometry}
\usepackage{hyperref}
\usepackage{hyperxmp}
\usepackage[usenames]{color}

\hypersetup{colorlinks=true}
\hypersetup{pdfstartview=FitH}
\hypersetup{pdfpagemode=UseNone}
\hypersetup{pdfsource={}}
\hypersetup{pdflang={en-UK}}
\hypersetup{pdfcopyright={Copyright 2017-2018 Niklas Beisert.
  This work may be distributed and/or modified under the
  conditions of the LaTeX Project Public License, either version 1.3
  of this license or (at your option) any later version.}}
\hypersetup{pdflicenseurl={http://www.latex-project.org/lppl.txt}}
\hypersetup{pdfcontactaddress={ETH Zurich, ITP, HIT K,
  Wolfgang-Pauli-Strasse 27}}
\hypersetup{pdfcontactpostcode={8093}}
\hypersetup{pdfcontactcity={Zurich}}
\hypersetup{pdfcontactcountry={Switzerland}}
\hypersetup{pdfcontactemail={nbeisert@itp.phys.ethz.ch}}
\hypersetup{pdfcontacturl={http://people.phys.ethz.ch/\xmptilde nbeisert/}}

\newcommand{\secref}[1]{\hyperref[#1]{section \ref*{#1}}}

\parskip1ex
\parindent0pt
\let\olditemize\itemize
\def\itemize{\olditemize\parskip0pt}

\begin{document}

\title{The \textsf{childdoc} Package}
\hypersetup{pdftitle={The childdoc Package}}
\author{Niklas Beisert\\[2ex]
  Institut f\"ur Theoretische Physik\\
  Eidgen\"ossische Technische Hochschule Z\"urich\\
  Wolfgang-Pauli-Strasse 27, 8093 Z\"urich, Switzerland\\[1ex]
  \href{mailto:nbeisert@itp.phys.ethz.ch}
  {\texttt{nbeisert@itp.phys.ethz.ch}}}
\hypersetup{pdfauthor={Niklas Beisert}}
\hypersetup{pdfsubject={Manual for the LaTeX2e Package childdoc}}
\date{30 December 2018, \textsf{v2.0}}
\maketitle

\begin{abstract}\noindent
\textsf{childdoc} is a \LaTeXe{} package
that enables the direct compilation
of document sections included by |\include|
to individual files.
\end{abstract}

\begingroup
\parskip0ex
\tableofcontents
\endgroup

%%%%%%%%%%%%%%%%%%%%%%%%%%%%%%%%%%%%%%%%%%%%%%%%%%%%%%%%%%%%%%%%%%%%%%%%%%%%%%%%
%%%%%%%%%%%%%%%%%%%%%%%%%%%%%%%%%%%%%%%%%%%%%%%%%%%%%%%%%%%%%%%%%%%%%%%%%%%%%%%%
\section{Introduction}

\LaTeX{} provides a mechanism to structure a large document (such as a book)
into a main file and several child files (containing the chapters)
using the |\include| command.
This mechanism is beneficial for documents
which span hundreds of pages in order to
make the source file(s) more manageable.
Moreover, compilation can be restricted to
selected child files by means of the |\includeonly| command.
The latter feature can be used to reduce the compilation time while editing
(this was significantly more useful in the earlier days of \LaTeX{})
or to generate a smaller document which is easier to navigate.
Another application of |\includeonly| is to generate
documents consisting of selected parts of the complete document.

However, there are a few drawbacks of the plain |\include| mechanism:
\begin{itemize}
\item
The child files cannot be compiled on their own,
they can only be compiled via the main file.
A naive editing environment
(such as a text editor with an option
to have the current file processed by \LaTeX)
may require one to switch to the main file before compiling;
attempting to compile the child file produces errors.
\item
The main file must be modified (each time)
to adjust the |\includeonly| command
to the present needs. This easily leaves the main file in a messy state.
\item
The generated document will always carry the filename
of the main document. This is inconvenient if
several child files are to be compiled and
to be kept for distribution.
\end{itemize}

The present package provides a simple interface
to make child files individually compilable by \LaTeX{}.
Compiling a child file then has the same effect as compiling
the main file with an |\includeonly| command
to select the appropriate child.
Moreover the generated document will carry the name of the child
rather than the main file.
This resolves all three above issues.

This feature is meant to make the editing of books,
thesis documents and lecture notes somewhat more convenient.
However, the package can also be used efficiently for
composing a series of documents (such as exercise sheets)
which are typically distributed individually.
It then assists the author in generating the individual documents
(potentially in different versions)
as well as a document containing the collected series.
Another application is in developing style files
or other kinds of included material
where compilation of the style file could redirect
to a sample or test file.

%%%%%%%%%%%%%%%%%%%%%%%%%%%%%%%%%%%%%%%%%%%%%%%%%%%%%%%%%%%%%%%%%%%%%%%%%%%%%%%%
%%%%%%%%%%%%%%%%%%%%%%%%%%%%%%%%%%%%%%%%%%%%%%%%%%%%%%%%%%%%%%%%%%%%%%%%%%%%%%%%
\section{Usage}

First of all, the package \textsf{childdoc} is \emph{not} a standard
\LaTeXe{} |.sty| style file! Therefore it needs to be invoked in
a non-standard way.

%%%%%%%%%%%%%%%%%%%%%%%%%%%%%%%%%%%%%%%%%%%%%%%%%%%%%%%%%%%%%%%%%%%%%%%%%%%%%%%%
\subsection{Included Files}
\label{sec:include}

%%%%%%%%%%%%%%%%%%%%%%%%%%%%%%%%%%%%%%%%
\DescribeMacro{\childdocmain}
To use the package, add the commands
\begin{center}
\begin{tabular}{l}
|% \iffalse
%
% childdoc.dtx Copyright (C) 2017-2018 Niklas Beisert
%
% This work may be distributed and/or modified under the
% conditions of the LaTeX Project Public License, either version 1.3
% of this license or (at your option) any later version.
% The latest version of this license is in
%   http://www.latex-project.org/lppl.txt
% and version 1.3 or later is part of all distributions of LaTeX
% version 2005/12/01 or later.
%
% This work has the LPPL maintenance status `maintained'.
%
% The Current Maintainer of this work is Niklas Beisert.
%
% This work consists of the files childdoc.dtx and childdoc.ins
% and the derived files childdoc.def and cdocsamp.tex with
% cdocsch1.tex, cdocsch2.tex, cdocsdrf.tex, cdocsfn1.tex, cdocsfn2.tex.
%
%<package>\ifdefined\childdocmain\endinput\fi
%<package>\ProvidesFile{childdoc.def}[2018/12/30 v2.0 child document driver]
%<samplemain>\ProvidesFile{cdocsamp.tex}[2018/12/30 v2.0 sample for childdoc]
%<*driver>
%\ProvidesFile{childdoc.drv}[2018/12/30 v2.0 childdoc reference manual file]
\PassOptionsToClass{10pt,a4paper}{article}
\documentclass{ltxdoc}

\usepackage[margin=35mm]{geometry}
\usepackage{hyperref}
\usepackage{hyperxmp}
\usepackage[usenames]{color}

\hypersetup{colorlinks=true}
\hypersetup{pdfstartview=FitH}
\hypersetup{pdfpagemode=UseNone}
\hypersetup{pdfsource={}}
\hypersetup{pdflang={en-UK}}
\hypersetup{pdfcopyright={Copyright 2017-2018 Niklas Beisert.
  This work may be distributed and/or modified under the
  conditions of the LaTeX Project Public License, either version 1.3
  of this license or (at your option) any later version.}}
\hypersetup{pdflicenseurl={http://www.latex-project.org/lppl.txt}}
\hypersetup{pdfcontactaddress={ETH Zurich, ITP, HIT K,
  Wolfgang-Pauli-Strasse 27}}
\hypersetup{pdfcontactpostcode={8093}}
\hypersetup{pdfcontactcity={Zurich}}
\hypersetup{pdfcontactcountry={Switzerland}}
\hypersetup{pdfcontactemail={nbeisert@itp.phys.ethz.ch}}
\hypersetup{pdfcontacturl={http://people.phys.ethz.ch/\xmptilde nbeisert/}}

\newcommand{\secref}[1]{\hyperref[#1]{section \ref*{#1}}}

\parskip1ex
\parindent0pt
\let\olditemize\itemize
\def\itemize{\olditemize\parskip0pt}

\begin{document}

\title{The \textsf{childdoc} Package}
\hypersetup{pdftitle={The childdoc Package}}
\author{Niklas Beisert\\[2ex]
  Institut f\"ur Theoretische Physik\\
  Eidgen\"ossische Technische Hochschule Z\"urich\\
  Wolfgang-Pauli-Strasse 27, 8093 Z\"urich, Switzerland\\[1ex]
  \href{mailto:nbeisert@itp.phys.ethz.ch}
  {\texttt{nbeisert@itp.phys.ethz.ch}}}
\hypersetup{pdfauthor={Niklas Beisert}}
\hypersetup{pdfsubject={Manual for the LaTeX2e Package childdoc}}
\date{30 December 2018, \textsf{v2.0}}
\maketitle

\begin{abstract}\noindent
\textsf{childdoc} is a \LaTeXe{} package
that enables the direct compilation
of document sections included by |\include|
to individual files.
\end{abstract}

\begingroup
\parskip0ex
\tableofcontents
\endgroup

%%%%%%%%%%%%%%%%%%%%%%%%%%%%%%%%%%%%%%%%%%%%%%%%%%%%%%%%%%%%%%%%%%%%%%%%%%%%%%%%
%%%%%%%%%%%%%%%%%%%%%%%%%%%%%%%%%%%%%%%%%%%%%%%%%%%%%%%%%%%%%%%%%%%%%%%%%%%%%%%%
\section{Introduction}

\LaTeX{} provides a mechanism to structure a large document (such as a book)
into a main file and several child files (containing the chapters)
using the |\include| command.
This mechanism is beneficial for documents
which span hundreds of pages in order to
make the source file(s) more manageable.
Moreover, compilation can be restricted to
selected child files by means of the |\includeonly| command.
The latter feature can be used to reduce the compilation time while editing
(this was significantly more useful in the earlier days of \LaTeX{})
or to generate a smaller document which is easier to navigate.
Another application of |\includeonly| is to generate
documents consisting of selected parts of the complete document.

However, there are a few drawbacks of the plain |\include| mechanism:
\begin{itemize}
\item
The child files cannot be compiled on their own,
they can only be compiled via the main file.
A naive editing environment
(such as a text editor with an option
to have the current file processed by \LaTeX)
may require one to switch to the main file before compiling;
attempting to compile the child file produces errors.
\item
The main file must be modified (each time)
to adjust the |\includeonly| command
to the present needs. This easily leaves the main file in a messy state.
\item
The generated document will always carry the filename
of the main document. This is inconvenient if
several child files are to be compiled and
to be kept for distribution.
\end{itemize}

The present package provides a simple interface
to make child files individually compilable by \LaTeX{}.
Compiling a child file then has the same effect as compiling
the main file with an |\includeonly| command
to select the appropriate child.
Moreover the generated document will carry the name of the child
rather than the main file.
This resolves all three above issues.

This feature is meant to make the editing of books,
thesis documents and lecture notes somewhat more convenient.
However, the package can also be used efficiently for
composing a series of documents (such as exercise sheets)
which are typically distributed individually.
It then assists the author in generating the individual documents
(potentially in different versions)
as well as a document containing the collected series.
Another application is in developing style files
or other kinds of included material
where compilation of the style file could redirect
to a sample or test file.

%%%%%%%%%%%%%%%%%%%%%%%%%%%%%%%%%%%%%%%%%%%%%%%%%%%%%%%%%%%%%%%%%%%%%%%%%%%%%%%%
%%%%%%%%%%%%%%%%%%%%%%%%%%%%%%%%%%%%%%%%%%%%%%%%%%%%%%%%%%%%%%%%%%%%%%%%%%%%%%%%
\section{Usage}

First of all, the package \textsf{childdoc} is \emph{not} a standard
\LaTeXe{} |.sty| style file! Therefore it needs to be invoked in
a non-standard way.

%%%%%%%%%%%%%%%%%%%%%%%%%%%%%%%%%%%%%%%%%%%%%%%%%%%%%%%%%%%%%%%%%%%%%%%%%%%%%%%%
\subsection{Included Files}
\label{sec:include}

%%%%%%%%%%%%%%%%%%%%%%%%%%%%%%%%%%%%%%%%
\DescribeMacro{\childdocmain}
To use the package, add the commands
\begin{center}
\begin{tabular}{l}
|\input{childdoc.def}|\\
|\childdocmain{}|\\
\end{tabular}
\end{center}
at the very top of the main \LaTeX{} file,
in particular \emph{before} the |\documentclass| statement!
The argument of |\childdocmain| should be left empty
(but it must be present).

%%%%%%%%%%%%%%%%%%%%%%%%%%%%%%%%%%%%%%%%
\DescribeMacro{\childdocof}
Furthermore, add the commands
\begin{center}
\begin{tabular}{l}
|\input{childdoc.def}|\\
|\childdocof{|\textit{main}|}|\\
\end{tabular}
\end{center}
at the top of every child file \textit{child}
which is included by |\include{|\textit{child}|}|
from within the main file
(or at least for those files to be compiled individually).
The argument \textit{main} must be the filename of the main file.

There are a couple of
considerations in setting up the main and child documents:

%%%%%%%%%%%%%%%%%%%%%%%%%%%%%%%%%%%%%%%%
\paragraph{Restrictions.}

Please note the following restrictions:
\begin{itemize}
\item
|\childdocmain| must be called with one argument \textit{main}
to ensure compatibility with earlier version of the package.
It must either be empty (|\childdocmain{}|)
or precisely match the filename of the main file in which it is specified.
See \secref{sec:detection} for further information.
\item
The filename \textit{main} must be specified without the |.tex| extension.
\item
The filename \textit{main} is case sensitive
(even in case-insensitive file systems)
due to internal string comparison.
\item
The argument \textit{main} should be fully expanded, it cannot be a macro.
\item
Subdirectories and special characters should be avoided in filenames.
\item
The command |\childdocmain{|\textit{main}|}| must be followed by a whitespace.
It should not be followed immediately by another command
or by a comment mark `|%|'.
This is because the \TeX{} parser reads the token immediately following
the argument of |\childdocmain| and puts it
at the beginning of every child section;
however, a white\-space is ignored.
\end{itemize}

%%%%%%%%%%%%%%%%%%%%%%%%%%%%%%%%%%%%%%%%
\paragraph{Content of Main File.}

It is advisable to place all content in the child files included by |\include|.
Any output contained in the main file will appear in all child documents
unless suppressed manually;
it cannot be suppressed automatically by the |\includeonly| directive
and thus should normally be avoided.
A method to include some content in the main file
by means of conditional processing is described in \secref{sec:conditional}.

%%%%%%%%%%%%%%%%%%%%%%%%%%%%%%%%%%%%%%%%
\paragraph{Page Numbering.}

When only a part of the document is compiled,
the appropriate numbering of pages
(as well as other status parameters)
is determined from the |.aux| files.
The latter contain information from previous passes.
However this information needs to propagate through
all intermediate child documents.
Therefore the page numbering in child documents may well
be inconsistent until the complete document is compiled at least once.

A useful (if unconventional) way to always ensure a consistent
page numbering is to restart the numbering in each child document
and denote the pages by `\textit{child}|.|\textit{page}'
where \textit{child} represents the chapter/section number of the child file.
This can be achieved by the command
|\numberwithin{page}{|\textit{child}|}|
of the \textsf{amsmath} package
where \textit{child} can be |chapter| or |section|
depending on the chosen structuring.
Alternatively, one can modify the macro |\thepage| appropriately
and reset the counter |page| at the start of each child file.

%%%%%%%%%%%%%%%%%%%%%%%%%%%%%%%%%%%%%%%%%%%%%%%%%%%%%%%%%%%%%%%%%%%%%%%%%%%%%%%%
\subsection{Conditional Processing}
\label{sec:conditional}

The package provides a mechanism to compile different versions
of a document. To customise the versions further some conditional processing
can come in handy to distinguish which version is being compiled.
The package provides two macros to describe the compilation context:

%%%%%%%%%%%%%%%%%%%%%%%%%%%%%%%%%%%%%%%%
\DescribeMacro{\ifchilddoc}
The conditional |\ifchilddoc| distinguishes between the compilation of
child documents and the main document:
%
\begin{center}
|\ifchilddoc |\textit{child-code}| |[|\||else |\textit{main-code}]| \||fi|
\end{center}

%%%%%%%%%%%%%%%%%%%%%%%%%%%%%%%%%%%%%%%%
\DescribeMacro{\childdocname}
\DescribeMacro{\childdocjob}
The macro |\childdocname| contains the filename (without extension)
of the main or child file being processed.
Note that |\childdocjob| will always contain the name of the main file.

%%%%%%%%%%%%%%%%%%%%%%%%%%%%%%%%%%%%%%%%
\paragraph{Title Page.}

Conditional processing can be used to include a title or banner page
in the main document when proper precautions are taken.
Importantly, the code in the main file should ensure that the page counter
(as well as other status parameters which are stored in the |.aux| files)
takes the same value after the conditional processing.
Otherwise the page numbers may take divergent values
depending on which part is compiled.

For example, a title page could be declared by:
%
\begin{center}
\begin{tabular}{l}
|\ifchilddoc\||else|\\
|\addtocounter{page}{-1}|\\
\textit{code for title page}\\
|\newpage|\\
|\||fi|
\end{tabular}
\end{center}
%
A banner page for the child documents can be generated by:
%
\begin{center}
\begin{tabular}{l}
|\ifchilddoc|\\
|\addtocounter{page}{-1}|\\
\textit{code for banner page}\\
|\newpage|\\
|\||fi|
\end{tabular}
\end{center}
%
Here one could write a message such as:
\begin{center}
|This is the part \childdocname{} of \childdocjob{}.|
\end{center}

%%%%%%%%%%%%%%%%%%%%%%%%%%%%%%%%%%%%%%%%%%%%%%%%%%%%%%%%%%%%%%%%%%%%%%%%%%%%%%%%
\subsection{Flags}
\label{sec:flags}

The package makes it easy to generate different versions
of the main or child documents.
To this end compilation flags can be defined
and assigned different default values.
They will be particularly useful in conjunction
with the forwarding mechanism described in \secref{sec:forward}.

For example, it may be useful to have a flag |\version|
which can be set to |draft| or |final|.
The document source will contain some conditional code
depending on the value of |\version|.
Suppose further, the flag should default to |final| for the main file
and to |draft| for child files
which is a natural assignment for editing the document.
This is achieved by placing the following code
in the preamble of the main document
(below the |\childdocmain| directive):
%
\begin{center}
\begin{tabular}{l}
|\ifchilddoc|\\
|\providecommand{\version}{draft}|\\
|\||else|\\
|\providecommand{\version}{final}|\\
|\||fi|
\end{tabular}
\end{center}
%
The definition by |\providecommand| makes sure
that previous definitions are not overwritten.
Further statements |\providecommand{\version}{...}|
can thus be added before the above code to override it.

For the main file, one might add a line
(between |\childdocmain| and the above block)
%
\begin{center}
|%\ifchilddoc\||else\providecommand{\version}{draft}\||fi|
\end{center}
%
which can be uncommented to produce a draft version.
Likewise one can add a line to the very top of a child file
(above the |\childdocof{|\textit{main}|}| directive)
%
\begin{center}
|%\providecommand{\version}{final}|
\end{center}
%
which can be uncommented to produce the final version of this child document.

%%%%%%%%%%%%%%%%%%%%%%%%%%%%%%%%%%%%%%%%%%%%%%%%%%%%%%%%%%%%%%%%%%%%%%%%%%%%%%%%
\subsection{Forwarding}
\label{sec:forward}

Different versions of the main or child documents
using compilation flags as described in \secref{sec:flags}
can be (permanently) stored in different files
for convenient compilation, viewing and distribution.
To this end, the package defines a command
to pass on compilation to a different file:

%%%%%%%%%%%%%%%%%%%%%%%%%%%%%%%%%%%%%%%%
\DescribeMacro{\childdocforward}
The command |\childdocforward| redirects processing to
another source file:
%
\begin{center}
\begin{tabular}{l}
|\input{childdoc.def}|\\
|\childdocforward[|\textit{main}|]{|\textit{dest}|}|\\
\end{tabular}
\end{center}
%
The argument \textit{dest} is the destination file
(without extension).
It should be the main file or one of the child files.
Note that further \textsf{childdoc} directives
such as |\childdocof| and |\childdocforward|
in the indicated file will be processed in this form.
The optional argument \textit{main}
passes on directly to the main file \textit{main}
while pretending to compile the child \textit{dest}.
This form behaves as if \textit{dest}
issues |\childdocof{|\textit{main}|}| right away,
and no further \textsf{childdoc} directives will be processed.

%%%%%%%%%%%%%%%%%%%%%%%%%%%%%%%%%%%%%%%%
\DescribeMacro{\...prefix}
In the alternative form |\childdocforwardprefix|,
%
\begin{center}
\begin{tabular}{l}
|\input{childdoc.def}|\\
|\childdocforwardprefix[|\textit{main}|]{|\textit{prefix}|}{|\textit{dest}|}|
\end{tabular}
\end{center}
%
the destination file is determined by a pattern
depending on the current file:
To make this work, the current file must be called
`{\textit{prefix}\hspace{0.2em}\textit{suffix}}'
with \textit{prefix} matching precisely the argument.
Processing is then passed on to the file
`{\textit{dest}\hspace{0.2em}\textit{suffix}}'.
Surely, the same effect is achieved by
directly specifying the
argument `{\textit{dest}\hspace{0.2em}\textit{suffix}}'
in the first form.
However, that requires to set up a different file
for each child. With the alternative form of the command
all these files can have exactly the same content
which simplifies setting them up and maintaining them.

For example, the following file |draft.tex|
with a compilation flag |\version| as described in \secref{sec:flags}
compiles the main document as a draft:
%
\begin{center}
\begin{tabular}{l}
|\def\version{draft}|\\
|\input{childdoc.def}|\\
|\childdocforward{|\textit{main}|}|
\end{tabular}
\end{center}
%
Likewise, the following files |final|\textit{nn}|.tex|
compile the final version of the child document
|child|\textit{nn}|.tex|:
%
\begin{center}
\begin{tabular}{l}
|\def\version{final}|\\
|\input{childdoc.def}|\\
|\childdocforwardprefix{final}{child}|
\end{tabular}
\end{center}
%

Note that when several versions of a main file and/or of each child file
are to be generated, it may be convenient to set up a |Makefile| or
shell script to automatise the process.

%%%%%%%%%%%%%%%%%%%%%%%%%%%%%%%%%%%%%%%%%%%%%%%%%%%%%%%%%%%%%%%%%%%%%%%%%%%%%%%%
\subsection{Command Line Processing}
\label{sec:commandline}

The effect of redirection files can also be achieved by invoking
the \LaTeX{} compiler with a more elaborate command line.
Most conveniently this should be done as part
of a shell script or a |Makefile|.

When using \textsf{childdoc} in the main file, the following
command lines effectively perform a redirection
(note that depending on the shell being used,
backslashes may have to be doubled: `|\|' $\to$ `|\\|'):
%
\begin{center}
|... -jobname "|\textit{target}|" |\\|"|[\textit{flags}]%
|\input{childdoc.def}\childdocforward[|\textit{main}|]{|\textit{dest}|}"|
\end{center}
%
Here \textit{target} is the name of the output file,
\textit{main} is the name of the main file
and \textit{dest} is the name of the main or child file to be processed
(all filenames without extensions).
The optional argument \textit{main} can be omitted
if \textit{main} matches \textit{dest}.
Optionally, compilation \textit{flags} can be defined via |\def| commands.
This command line makes the \TeX{} engine believe
it is compiling the file \textit{target}
whose content is specified as the latter parameter.
The provided code then forwards the processing to
\textit{main} or \textit{dest} as described in \secref{sec:forward}.

%%%%%%%%%%%%%%%%%%%%%%%%%%%%%%%%%%%%%%%%%%%%%%%%%%%%%%%%%%%%%%%%%%%%%%%%%%%%%%%%
\subsection{Include by Input}
\label{sec:input}

Including child documents by |\include| has some restrictions by design.
Most notably, the content of a child document always occupies
its own set of pages; pages cannot be shared between child documents.
Usually, this behaviour makes perfect sense
because each child document contain an essential part of the document.
However, in some situations it may be desirable to compose
a document from a collection of parts
without having mandatory page breaks between then.
For this case, the package
provides a mechanism to include parts
by |\input| which can also be processed individually.
However, by construction this mechanism
requires manual handling of the content to be output.

%%%%%%%%%%%%%%%%%%%%%%%%%%%%%%%%%%%%%%%%
\DescribeMacro{\ifchilddocmanual}
The main file should be prepared as usual, see \secref{sec:include}.
However, the document body must make a distinction
between processing of an individual part and of the main document, e.g.:
%
\begin{center}
\begin{tabular}{l}
|\ifchilddocmanual|\\
|\input{\childdocname}|\\
|\||else|\\
\textit{document body with }|\input{|\textit{part}|}|\\
|\||fi|
\end{tabular}
\end{center}
%
The conditional |\ifchilddocmanual| is true whenever
a part to be included by |\input| is being compiled,
and the name of the part is stored in |\childdocname|.

%%%%%%%%%%%%%%%%%%%%%%%%%%%%%%%%%%%%%%%%
\DescribeMacro{\childdocby}
Each part to be included by |\input| should start with:
%
\begin{center}
\begin{tabular}{l}
|\input{childdoc.def}|\\
|\childdocby{|\textit{main}|}|\\
\end{tabular}
\end{center}
%
The directive |\childdocby| is similar to |\childdocof|
described in \secref{sec:include},
but the subsequent selection of content must be done manually.
To that end, both |\ifchilddoc| and |\ifchilddocmanual|
will be true upon processing of a part,
and the name of the part is stored in |\childdocname|.
Note that |\jobname| will be set to the filename of the current part
so that each part receives an individual |.aux| file
that does not interfere with the |.aux| file(s) of the main document.
This behaviour can be altered by the alternative form
|\childdocby[*]{|\textit{main}|}| (with a non-empty optional argument)
which uses the |.aux| file of the main document
by setting |\jobname| to \textit{main}.

%%%%%%%%%%%%%%%%%%%%%%%%%%%%%%%%%%%%%%%%%%%%%%%%%%%%%%%%%%%%%%%%%%%%%%%%%%%%%%%%
\subsection{Driver Development}
\label{sec:driver}

The \textsf{childdoc} mechanism can also be use for the development
of definition files such as \LaTeX{} styles or classes.
This case differs from the above setup with multiple parts
included by |\include| in that no |\includeonly| should be invoked.
This can be achieved by starting the include file
(before |\ProvidesPackage|) with:
%
\begin{center}
\begin{tabular}{l}
|\input{childdoc.def}|\\
|\childdocforward{|\textit{main}|}|\\
\end{tabular}
\end{center}
%
or alternatively with:
%
\begin{center}
\begin{tabular}{l}
|\input{childdoc.def}|\\
|\childdocby{|\textit{main}|}|\\
\end{tabular}
\end{center}
%
Both forms have slightly different effects as described above.
The main file is prepared as usual, see \secref{sec:include}.

%%%%%%%%%%%%%%%%%%%%%%%%%%%%%%%%%%%%%%%%%%%%%%%%%%%%%%%%%%%%%%%%%%%%%%%%%%%%%%%%
\subsection{Legacy Detection}
\label{sec:detection}

The directive |\childdocmain| in the main file can detect
whether the complete document or merely a child is to be compiled
even without using the directive |\childdocof|.
This method is deprecated because it is less robust
and there is no compelling reason to use it;
it is merely provided for backward compatibility
and it may be removed in future versions.

If the detection mechanism is to be used,
it is mandatory to correctly specify
the filename of the main file as the argument of |\childdocmain|:
%
\begin{center}
\begin{tabular}{l}
|\input{childdoc.def}|\\
|\childdocmain{|\textit{main}|}|\\
\end{tabular}
\end{center}
%
If |\jobname| does not match the argument \textit{main} of |\childdocmain|,
it is assumed that |\jobname| points to the child file to be compiled.
When using |\childdocmain| with the main file specified as argument,
it suffices to start a child file
with just |\input{|\textit{main}|}|
without loading of the package and using |\childdocof|.
If instead all processing is done
with the appropriate \textsf{childdoc} directives,
the argument of \textit{main} of |\childdocmain| can be empty.

An alternative version of the command line processing described
in \secref{sec:commandline} using the detection mechanism reads:
%
\begin{center}
|... -jobname "|\textit{target}|" "|[\textit{flags}]%
[|\def\jobname{|\textit{dest}|}|]|\input{|\textit{main}|}"|
\end{center}

%%%%%%%%%%%%%%%%%%%%%%%%%%%%%%%%%%%%%%%%%%%%%%%%%%%%%%%%%%%%%%%%%%%%%%%%%%%%%%%%
\subsection{Manual Code}
\label{sec:manual}

In case one cannot be certain whether the definitions file |childdoc.def|
is installed on the target \TeX{} distribution
and one prefers not to ship it,
it is conceivable to paste a few relevant commands into the sources.

To that end, drop all statements |\input{childdoc.def}|
and perform the replacements as outlined below.
Instead of |\childdocmain{|\textit{main}|}| add the following code
to the top of the main file:
%
\begin{center}
\begin{tabular}{l}
|\||ifdefined\childdocname\endinput\||fi\newif\ifchilddoc|\\
|\edef\childdocname{\scantokens\expandafter{\jobname\noexpand}}|\\
|\def\childdocmain{|\textit{main}|}\||ifx\childdocmain\childdocname\||else|\\
|\childdoctrue\includeonly{\childdocname}\let\jobname\childdocmain\||fi|\\
\end{tabular}
\end{center}
%
Instead of |\childdocof{|\textit{main}|}| just include the main file
at the top of each child file:
%
\begin{center}
|\input{|\textit{main}|}|
\end{center}
%
A simple redirection |\childdocforward{|\textit{dest}|}| is achieved by:
%
\begin{center}
|\def\jobname{|\textit{dest}|}\input{\jobname}|
\end{center}
%
The redirection with prefix
|\childdocforwardprefix[|\textit{prefix}|]{|\textit{dest}|}|
is accomplished by:
%
\begin{center}
\begin{tabular}{l}
|{\edef\jobname{\scantokens\expandafter{\jobname\noexpand}}|\\
|\def\redirectjob |\textit{prefix}|#1~~~{\gdef\jobname{|\textit{dest}|#1}}|\\
|\expandafter\redirectjob\jobname~~~}\input{\jobname}|
\end{tabular}
\end{center}

In an alternative approach,
child documents can be compiled by a specific command line
without additional code or specific definitions:
%
\begin{center}
|... -jobname "|\textit{target}|" "|[\textit{flags}]%
|\includeonly{|\textit{dest}|}\input{|\textit{main}|}"|
\end{center}
%

%%%%%%%%%%%%%%%%%%%%%%%%%%%%%%%%%%%%%%%%%%%%%%%%%%%%%%%%%%%%%%%%%%%%%%%%%%%%%%%%
%%%%%%%%%%%%%%%%%%%%%%%%%%%%%%%%%%%%%%%%%%%%%%%%%%%%%%%%%%%%%%%%%%%%%%%%%%%%%%%%
\section{Information}

%%%%%%%%%%%%%%%%%%%%%%%%%%%%%%%%%%%%%%%%%%%%%%%%%%%%%%%%%%%%%%%%%%%%%%%%%%%%%%%%
\subsection{Copyright}

Copyright \copyright{} 2017--2018 Niklas Beisert

This work may be distributed and/or modified under the
conditions of the \LaTeX{} Project Public License, either version 1.3
of this license or (at your option) any later version.
The latest version of this license is in
  \url{http://www.latex-project.org/lppl.txt}
and version 1.3 or later is part of all distributions of \LaTeX{}
version 2005/12/01 or later.

This work has the LPPL maintenance status `maintained'.

The Current Maintainer of this work is Niklas Beisert.

This work consists of the files |README.txt|, |childdoc.ins| and |childdoc.dtx|
as well as the derived files |childdoc.def|, |cdocsamp.tex|
with |cdocsch1.tex|, |cdocsch2.tex|, |cdocspt3.tex|, |cdocspt4.tex|,
|cdocsdrf.tex|, |cdocsfn1.tex|, |cdocsfn2.tex|
as well as |childdoc.pdf|.

%%%%%%%%%%%%%%%%%%%%%%%%%%%%%%%%%%%%%%%%%%%%%%%%%%%%%%%%%%%%%%%%%%%%%%%%%%%%%%%%
\subsection{Files and Installation}

The package consists of the files:
%
\begin{center}
\begin{tabular}{ll}
    |README.txt|   & readme file \\
    |childdoc.ins| & installation file \\
    |childdoc.dtx| & source file \\
    |childdoc.def| & definition file \\
    |cdocsamp.tex| & sample main file \\
    |cdocsch1.tex| & sample include file \\
    |cdocsch2.tex| & sample include file \\
    |cdocspt3.tex| & sample part file \\
    |cdocspt4.tex| & sample part file \\
    |cdocsdrf.tex| & sample redirection file \\
    |cdocsfn1.tex| & sample redirection file \\
    |cdocsfn2.tex| & sample redirection file \\
    |childdoc.pdf| & manual
\end{tabular}
\end{center}
%
The distribution consists of the files
|README.txt|, |childdoc.ins| and |childdoc.dtx|.
%
\begin{itemize}
\item
Run (pdf)\LaTeX{} on |childdoc.dtx|
to compile the manual |childdoc.pdf| (this file).
\item
Run \LaTeX{} on |childdoc.ins| to create the definitions file |childdoc.def|
and the sample |cdocsamp.tex| with include files
|cdocsch1.tex|, |cdocsch2.tex|, |cdocspt3.tex|, |cdocspt4.tex|,
|cdocsdrf.tex|, |cdocsfn1.tex|, |cdocsfn2.tex|.
Then copy the file |childdoc.def| to an appropriate directory of your \LaTeX{}
distribution, e.g.\ \textit{texmf-root}|/tex/latex/childdoc|.
\end{itemize}

%%%%%%%%%%%%%%%%%%%%%%%%%%%%%%%%%%%%%%%%%%%%%%%%%%%%%%%%%%%%%%%%%%%%%%%%%%%%%%%%
\subsection{Related CTAN Packages}

There are several other packages which offer a similar functionality:
%
\begin{itemize}
\item
The packages
\href{http://ctan.org/pkg/docmute}{\textsf{docmute}},
\href{http://ctan.org/pkg/includex}{\textsf{includex}} and
\href{http://ctan.org/pkg/standalone}{\textsf{standalone}}
provide commands to include only the document body of
a child file thus allowing both files to be compiled individually.
\item
The packages \href{http://ctan.org/pkg/subdocs}{\textsf{subdocs}}
and \href{http://ctan.org/pkg/subfiles}{\textsf{subfiles}}
provide structures in which the main and child documents can be
encapsulated and allowing them to be compiled individually.
The inclusion mechanism is different from the conventional |\include|.
\item
The package \href{http://ctan.org/pkg/combine}{\textsf{combine}}
is an elaborate solution to combine several documents into one.
\end{itemize}
%
See also the CTAN topic \href{http://ctan.org/topic/subdocs}{\textsf{subdocs}}
for further related packages.
The present package differs from the above solutions in that
a document structure constructed with the conventional |\include| mechanism
just needs two extra commands at the top of every file
such that all constituent files can be compiled individually.

%%%%%%%%%%%%%%%%%%%%%%%%%%%%%%%%%%%%%%%%%%%%%%%%%%%%%%%%%%%%%%%%%%%%%%%%%%%%%%%%
%\subsection{Feature Suggestions}
%
%The following is a list of features which may be useful for future
%versions of this package:
%%
%\begin{itemize}
%\item
%\ldots
%\end{itemize}

%%%%%%%%%%%%%%%%%%%%%%%%%%%%%%%%%%%%%%%%%%%%%%%%%%%%%%%%%%%%%%%%%%%%%%%%%%%%%%%%
\subsection{Revision History}

%%%%%%%%%%%%%%%%%%%%%%%%%%%%%%%%%%%%%%%%
\paragraph{v2.0:} 2018/12/30

\begin{itemize}
\item
immediate forward processing
\item
added |\childdocby| mechanism
\item
manual restructured
\end{itemize}

%%%%%%%%%%%%%%%%%%%%%%%%%%%%%%%%%%%%%%%%
\paragraph{v1.6:} 2018/01/17

\begin{itemize}
\item
application for development of include files
\item
corrections to manual
\end{itemize}

%%%%%%%%%%%%%%%%%%%%%%%%%%%%%%%%%%%%%%%%
\paragraph{v1.5:} 2017/05/21

\begin{itemize}
\item
more complete structuring introduced
\item
|\childdocof| introduced
\item
|\childdoc| renamed to |\childdocmain|
\item
|\childredirect| renamed to |\childdocforward| and |\childdocforwardprefix|
and functionality expanded
\end{itemize}

%%%%%%%%%%%%%%%%%%%%%%%%%%%%%%%%%%%%%%%%
\paragraph{v1.0:} 2017/04/27

\begin{itemize}
\item
manual and install package
\item
first version published on CTAN
\end{itemize}

%%%%%%%%%%%%%%%%%%%%%%%%%%%%%%%%%%%%%%%%
\paragraph{v0.6:} 2017/04/26

\begin{itemize}
\item
redirection mechanism added
\end{itemize}

%%%%%%%%%%%%%%%%%%%%%%%%%%%%%%%%%%%%%%%%
\paragraph{v0.5:} 2017/04/26

\begin{itemize}
\item
functionality in definition file
\end{itemize}


%%%%%%%%%%%%%%%%%%%%%%%%%%%%%%%%%%%%%%%%%%%%%%%%%%%%%%%%%%%%%%%%%%%%%%%%%%%%%%%%
%%%%%%%%%%%%%%%%%%%%%%%%%%%%%%%%%%%%%%%%%%%%%%%%%%%%%%%%%%%%%%%%%%%%%%%%%%%%%%%%
%%%%%%%%%%%%%%%%%%%%%%%%%%%%%%%%%%%%%%%%%%%%%%%%%%%%%%%%%%%%%%%%%%%%%%%%%%%%%%%%
\appendix

\settowidth\MacroIndent{\rmfamily\scriptsize 000\ }

 \DocInput{childdoc.dtx}

\end{document}
%</driver>
% \fi
%
% %%%%%%%%%%%%%%%%%%%%%%%%%%%%%%%%%%%%%%%%%%%%%%%%%%%%%%%%%%%%%%%%%%%%%%%%%%%%%%
% %%%%%%%%%%%%%%%%%%%%%%%%%%%%%%%%%%%%%%%%%%%%%%%%%%%%%%%%%%%%%%%%%%%%%%%%%%%%%%
% \section{Sample}
%\iffalse
%<*samplemain>
%\fi
%
% The following presents a sample document
% with two chapters, two parts, a title page,
% a compile flag as well as three forwarding files to set the flag.
% It consists of eight |.tex| files:
% \begin{center}
% \begin{tabular}{ll}
% |cdocsamp.tex|&main file\\
% |cdocsch1.tex|&include file for chapter 1\\
% |cdocsch2.tex|&include file for chapter 2\\
% |cdocspt3.tex|&include file for part 3\\
% |cdocspt4.tex|&include file for part 4\\
% |cdocsdrf.tex|&forwarding file for main file in draft mode\\
% |cdocsfi1.tex|&forwarding file for final version of chapter 1\\
% |cdocsfi2.tex|&forwarding file for final version of chapter 2\\
% \end{tabular}
% \end{center}
% Each of the eight files can be compiled directly by the \LaTeX{} compiler.
%
% %%%%%%%%%%%%%%%%%%%%%%%%%%%%%%%%%%%%%%
% \paragraph{Main File.}
%
% The main file is called |cdocsamp.tex|.
%
% Load the \textsf{childdoc} definitions and
% declare the filename for the main document:
%    \begin{macrocode}
\input{childdoc.def}
\childdocmain{}
%    \end{macrocode}

% Optional override for |\version| flag:
%    \begin{macrocode}
%%\ifchilddoc\else\providecommand{\version}{draft}\fi
%    \end{macrocode}

% Define the default values for the |\version| flag
% (|final| for the main file and |draft| for childs):
%    \begin{macrocode}
\ifchilddoc
\providecommand{\version}{draft}
\else
\providecommand{\version}{final}
\fi
%    \end{macrocode}

% Load the standard document class:
%    \begin{macrocode}
\documentclass[12pt]{article}
%    \end{macrocode}

% Start the document body:
%    \begin{macrocode}
\begin{document}
%    \end{macrocode}

% Declare a title page.
% Print title, part of document being processed and version flag:
%    \begin{macrocode}
\addtocounter{page}{-1}
\begin{center}
{\LARGE\bfseries{}childdoc example\par}
\vspace{1cm}
\ifchilddoc
\ifchilddocmanual part\else chapter\fi:
`\childdocname' of `\childdocjob'\par
\else
main document: `\childdocjob'\par
\fi
version: \version\par
\end{center}
\newpage
%    \end{macrocode}

% Manually include selected file,
% otherwise process as usual:
%    \begin{macrocode}
\ifchilddocmanual
\section*{part `\childdocname'}
\input{\childdocname}
\else
%    \end{macrocode}

% Include the two chapters:
%    \begin{macrocode}
\include{cdocsch1}
\include{cdocsch2}
%    \end{macrocode}

% Include the two parts unless only chapters should be displayed:
%    \begin{macrocode}
\ifchilddoc\else
\section{part three}
\input{cdocspt3}
\section{part four}
\input{cdocspt4}
\fi
%    \end{macrocode}

% Process as usual until here:
%    \begin{macrocode}
\fi
%    \end{macrocode}

% End of document body:
%    \begin{macrocode}
\end{document}
%    \end{macrocode}
%\iffalse
%</samplemain>
%\fi
%
% %%%%%%%%%%%%%%%%%%%%%%%%%%%%%%%%%%%%%%
% \paragraph{Chapter Include Files.}
%
% The include files are called |cdocsch1.tex| and |cdocsch2.tex|.
%
%\iffalse
%<*samplechap1|samplechap2>
%\fi

% Optional override for |\version| flag:
%    \begin{macrocode}
%%\providecommand{\version}{final}
%    \end{macrocode}

% Include the main document:
%    \begin{macrocode}
\input{childdoc.def}
\childdocof{cdocsamp}
%    \end{macrocode}

%\iffalse
%</samplechap1|samplechap2>
%\fi
%
%\iffalse
%<*samplechap1>
%\fi
% Some text for chapter 1:
%    \begin{macrocode}
\section{one}
some text in chapter one
%    \end{macrocode}

%\iffalse
%</samplechap1>
%\fi
% Some text for chapter 2:
%\iffalse
%<*samplechap2>
%\fi
%    \begin{macrocode}
\section{two}
more text in chapter two
%    \end{macrocode}

%\iffalse
%</samplechap2>
%\fi
%
% %%%%%%%%%%%%%%%%%%%%%%%%%%%%%%%%%%%%%%
% \paragraph{Part Include Files.}
%
% The include files are called |cdocspt3.tex| and |cdocspt4.tex|.
%
%\iffalse
%<*samplepart3|samplepart4>
%\fi

% Optional override for |\version| flag:
%    \begin{macrocode}
%%\providecommand{\version}{final}
%    \end{macrocode}

% Include the main document:
%    \begin{macrocode}
\input{childdoc.def}
\childdocby{cdocsamp}
%    \end{macrocode}

%\iffalse
%</samplepart3|samplepart4>
%\fi
%
%\iffalse
%<*samplepart3>
%\fi
% Some text for part 3:
%    \begin{macrocode}
some text in part three
%    \end{macrocode}

%\iffalse
%</samplepart3>
%\fi
% Some text for part 4:
%\iffalse
%<*samplepart4>
%\fi
%    \begin{macrocode}
more text in part four
%    \end{macrocode}

%\iffalse
%</samplepart4>
%\fi
%
% %%%%%%%%%%%%%%%%%%%%%%%%%%%%%%%%%%%%%%
% \paragraph{Forwarding for a Complete Draft.}
%
% The following forwarding file |cdocsdrf.tex|
% compiles the main document in draft mode:
%\iffalse
%<*sampledraft>
%\fi
%    \begin{macrocode}
\def\version{draft}
\input{childdoc.def}
\childdocforward{cdocsamp}
%    \end{macrocode}

%\iffalse
%</sampledraft>
%\fi
%
% %%%%%%%%%%%%%%%%%%%%%%%%%%%%%%%%%%%%%%
% \paragraph{Forwarding for Final Version of the Chapters.}
%
% The following forwarding files |cdocsfn1.tex| and |cdocsfn2.tex|
% (with identical content)
% compile the final versions of the child documents
% |cdocsch1.tex| and |cdocsch2.tex|, respectively:
%\iffalse
%<*samplefinal>
%\fi
%    \begin{macrocode}
\def\version{final}
\input{childdoc.def}
\childdocforwardprefix[cdocsamp]{cdocsfn}{cdocsch}
%    \end{macrocode}

%\iffalse
%</samplefinal>
%\fi
%
% %%%%%%%%%%%%%%%%%%%%%%%%%%%%%%%%%%%%%%
% \paragraph{Command Line Processing.}
%
% The following three command lines generate the output files
% |cdocscld|, |cdocscl1| and |cdocscl2|
% which should be identical to
% |cdocsdrf|, |cdocsch1| and |cdocsfn2|, respectively:
% \begin{center}
% \begin{tabular}{l}
% |latex -jobname cdocscld \|\\
% |  "\def\version{draft}\input{childdoc.def}\childdocforward{cdocsamp}"|\\
% |latex -jobname cdocscl1 \|\\
% |  "\input{childdoc.def}\childdocforward[cdocsamp]{cdocsch1}"|\\
% |latex -jobname cdocscl2 \|\\
% |  "\def\version{final}\input{childdoc.def}\childdocforward{cdocsch2}"|
% \end{tabular}
% \end{center}
% Note that the trailing backslash on each first line
% merely continues the input to the second line
% (for convenient cut ant paste).
% Furthermore, the command |latex| can be replaced by any
% of its alternative versions such as |pdflatex|.
%
% %%%%%%%%%%%%%%%%%%%%%%%%%%%%%%%%%%%%%%%%%%%%%%%%%%%%%%%%%%%%%%%%%%%%%%%%%%%%%%
% %%%%%%%%%%%%%%%%%%%%%%%%%%%%%%%%%%%%%%%%%%%%%%%%%%%%%%%%%%%%%%%%%%%%%%%%%%%%%%
% \section{Implementation}
%\iffalse
%<*package>
%\fi
%
% This section describes the definitions file |childdoc.def|.

% The definitions cannot be loaded using |\usepackage| or |\RequirePackage|
% which has a mechanism to prevent loading a style file more than once.
% When loading the definitions by means of |\input|
% multiple instances have to be prevented manually:
%\iffalse
%This code needs to be before the `\ProvidesFile' directive
%which is defined at the beginning of this file.
%Therefore it is also placed there and commented out here.
%</package>
%<*discard>
%\fi
%    \begin{macrocode}
\ifdefined\childdocmain\endinput\fi
%    \end{macrocode}
%\iffalse
%</discard>
%<*package>
%\fi
%
% \macro{\ifchilddoc}
% \macro{\ifchilddocmanual}
% The conditional |\ifchilddoc| tells whether a
% child (true) or main (false) document is being compiled.
% The conditional |\ifchilddocmanual| tells whether
% the |\includeonly| mechanism is used (false) or
% the selection of child files must be performed manually (true).
% The definitions initialise to false:
%    \begin{macrocode}
\newif\ifchilddoc
\newif\ifchilddocmanual
%    \end{macrocode}

% \macro{\childdocname}
% \macro{\childdocjob}
% The macro |\childdocname| stores the name of the main document
% to be compiled. The macro |\childdocjob| stores the name of
% the document on which the \LaTeX{} compiler was originally invoked.
% The content of |\jobname| cannot be compared
% to filenames specified in the source due to different catcodes.
% The following code rescans |\jobname|, stores the result
% in |\childdocname| and saves a copy in |\childdocjob|:
%    \begin{macrocode}
\edef\childdocname{\scantokens\expandafter{\jobname\noexpand}}
\let\childdocjob\childdocname
%    \end{macrocode}

% \macro{\childdocdisable}
% The macro |\childdocdisable| prevents the main file
% from being processed more than once.
% At this stage, the main document command |\childdocmain|
% is assumed to be called once again where it should do nothing.
% Any subsequent call to it should prevent
% a secondary processing of the main document
% It overwrites the forwarding commands
% |\childdocof| and |\childdocforward|
% with empty macros to prevent further inclusions of the main document:
%    \begin{macrocode}
\newcommand{\childdocdisable}
{
  \renewcommand{\childdocmain}[1]{\renewcommand{\childdocmain}[1]{\endinput}}
  \renewcommand{\childdocof}[1]{}
  \renewcommand{\childdocby}[2][]{}
  \renewcommand{\childdocforward}[2][]{}
  \renewcommand{\childdocdisable}{}
}
%    \end{macrocode}

% \macro{\childdocmain}
% The macro |\childdocmain| is to be called at the top of the main file
% with nothing or the main filename (without extension) as argument.
% First, it breaks loops.
% If the argument is not empty and does not match |\childdocname|
% (which is set by the first inclusion of |childdoc.def|),
% |\ifchilddoc| is set to true, |\includeonly| is applied to the child file
% and |\jobname| is set to the main file
% (for proper handling of |.aux| files):
%    \begin{macrocode}
\newcommand{\childdocmain}[1]
{
  \childdocdisable\childdocmain{}
  \if?#1?\else
    \begingroup
      \def\childdoctmp{#1}
      \ifx\childdoctmp\childdocname
        \def\childdoctmp{}
      \else
        \def\childdoctmp
        {
          \childdoctrue
          \includeonly{\childdocname}
          \def\childdocjob{#1}
          \def\jobname{#1}
        }
      \fi
      \expandafter
    \endgroup
    \childdoctmp
  \fi
}
%    \end{macrocode}

% \macro{\childdocof}
% The command |\childdocof| redirects
% compilation to the main file |#1|.
%    \begin{macrocode}
\newcommand{\childdocof}[1]
{
  \childdocdisable
  \childdoctrue
  \includeonly{\childdocname}
  \def\jobname{#1}
  \def\childdocjob{#1}
  \input{#1}
}
%    \end{macrocode}

% \macro{\childdocby}
% The command |\childdocby| ....
%    \begin{macrocode}
\newcommand{\childdocby}[2][]
{
  \childdocdisable
  \childdoctrue
  \childdocmanualtrue
  \if?#1?\else
    \def\jobname{#2}
  \fi
  \def\childdocjob{#2}
  \input{#2}
  \endinput
}
%    \end{macrocode}

% \macro{\childdocforward}
% The command |\childdocforward| redirects
% compilation to the main file or
% (if the optional argument is given) a child file.
% Parameters are set as if the main file
% or a child file starting with |\childdocof| was compiled.
% Then compilation is handed over to the main file:
%    \begin{macrocode}
\newcommand{\childdocforward}[2][]
{
  \begingroup
    \if?#1?
      \def\childdoctmp
      {
        \def\childdocname{#2}
        \def\childdocjob{#2}
        \def\jobname{#2}
        \input{#2}
        \endinput
      }
    \else
      \def\childdoctmp
      {
        \childdocdisable
        \def\childdocname{#2}
        \childdoctrue
        \includeonly{#2}
        \def\childdocjob{#1}
        \def\jobname{#1}
        \input{#1}
        \endinput
      }
    \fi
    \expandafter
  \endgroup
  \childdoctmp
}
%    \end{macrocode}

% \macro{\childdocforwardprefix}
% The command |\childdocforwardprefix| redirects
% compilation to the main or a child file by means of a pattern.
% The prefix |#1| in the current filename is replaced by |#2|
% and the suffix of the current filename is kept
% (it is assumed that the filename does not contain the substring `|~~~|'
% which is used as a delimiter).
% Compilation is handed over to the new file by |\childdocforward|:
%    \begin{macrocode}
\newcommand{\childdocforwardprefix}[3][]
{
  \begingroup
    \def\childdocextract #2##1~~~{\def\childdoctmp{\childdocforward[#1]{#3##1}}}
    \expandafter\childdocextract\childdocname~~~
    \expandafter
  \endgroup
  \childdoctmp
}
%    \end{macrocode}

% \macro{\childdoc}
% The deprecated macro |\childdoc| is a legacy version of |\childdocmain|:
%    \begin{macrocode}
\newcommand{\childdoc}{\childdocmain}
%    \end{macrocode}

% \macro{\childdocredirect}
% The deprecated macro |\childdocredirect| is a legacy version
% of |\childdocforward| and |\childdocforwardprefix|:
%    \begin{macrocode}
\newcommand{\childdocredirect}[2][]
{
  \begingroup
    \if?#1?
      \def\childdoctmp{\childdocforward{#2}}
    \else
      \def\childdoctmp{\childdocforwardprefix{#1}{#2}}
    \fi
    \expandafter
  \endgroup
  \childdoctmp
}
%    \end{macrocode}

%\iffalse
%</package>
%\fi
%
\endinput
|\\
|\childdocmain{}|\\
\end{tabular}
\end{center}
at the very top of the main \LaTeX{} file,
in particular \emph{before} the |\documentclass| statement!
The argument of |\childdocmain| should be left empty
(but it must be present).

%%%%%%%%%%%%%%%%%%%%%%%%%%%%%%%%%%%%%%%%
\DescribeMacro{\childdocof}
Furthermore, add the commands
\begin{center}
\begin{tabular}{l}
|% \iffalse
%
% childdoc.dtx Copyright (C) 2017-2018 Niklas Beisert
%
% This work may be distributed and/or modified under the
% conditions of the LaTeX Project Public License, either version 1.3
% of this license or (at your option) any later version.
% The latest version of this license is in
%   http://www.latex-project.org/lppl.txt
% and version 1.3 or later is part of all distributions of LaTeX
% version 2005/12/01 or later.
%
% This work has the LPPL maintenance status `maintained'.
%
% The Current Maintainer of this work is Niklas Beisert.
%
% This work consists of the files childdoc.dtx and childdoc.ins
% and the derived files childdoc.def and cdocsamp.tex with
% cdocsch1.tex, cdocsch2.tex, cdocsdrf.tex, cdocsfn1.tex, cdocsfn2.tex.
%
%<package>\ifdefined\childdocmain\endinput\fi
%<package>\ProvidesFile{childdoc.def}[2018/12/30 v2.0 child document driver]
%<samplemain>\ProvidesFile{cdocsamp.tex}[2018/12/30 v2.0 sample for childdoc]
%<*driver>
%\ProvidesFile{childdoc.drv}[2018/12/30 v2.0 childdoc reference manual file]
\PassOptionsToClass{10pt,a4paper}{article}
\documentclass{ltxdoc}

\usepackage[margin=35mm]{geometry}
\usepackage{hyperref}
\usepackage{hyperxmp}
\usepackage[usenames]{color}

\hypersetup{colorlinks=true}
\hypersetup{pdfstartview=FitH}
\hypersetup{pdfpagemode=UseNone}
\hypersetup{pdfsource={}}
\hypersetup{pdflang={en-UK}}
\hypersetup{pdfcopyright={Copyright 2017-2018 Niklas Beisert.
  This work may be distributed and/or modified under the
  conditions of the LaTeX Project Public License, either version 1.3
  of this license or (at your option) any later version.}}
\hypersetup{pdflicenseurl={http://www.latex-project.org/lppl.txt}}
\hypersetup{pdfcontactaddress={ETH Zurich, ITP, HIT K,
  Wolfgang-Pauli-Strasse 27}}
\hypersetup{pdfcontactpostcode={8093}}
\hypersetup{pdfcontactcity={Zurich}}
\hypersetup{pdfcontactcountry={Switzerland}}
\hypersetup{pdfcontactemail={nbeisert@itp.phys.ethz.ch}}
\hypersetup{pdfcontacturl={http://people.phys.ethz.ch/\xmptilde nbeisert/}}

\newcommand{\secref}[1]{\hyperref[#1]{section \ref*{#1}}}

\parskip1ex
\parindent0pt
\let\olditemize\itemize
\def\itemize{\olditemize\parskip0pt}

\begin{document}

\title{The \textsf{childdoc} Package}
\hypersetup{pdftitle={The childdoc Package}}
\author{Niklas Beisert\\[2ex]
  Institut f\"ur Theoretische Physik\\
  Eidgen\"ossische Technische Hochschule Z\"urich\\
  Wolfgang-Pauli-Strasse 27, 8093 Z\"urich, Switzerland\\[1ex]
  \href{mailto:nbeisert@itp.phys.ethz.ch}
  {\texttt{nbeisert@itp.phys.ethz.ch}}}
\hypersetup{pdfauthor={Niklas Beisert}}
\hypersetup{pdfsubject={Manual for the LaTeX2e Package childdoc}}
\date{30 December 2018, \textsf{v2.0}}
\maketitle

\begin{abstract}\noindent
\textsf{childdoc} is a \LaTeXe{} package
that enables the direct compilation
of document sections included by |\include|
to individual files.
\end{abstract}

\begingroup
\parskip0ex
\tableofcontents
\endgroup

%%%%%%%%%%%%%%%%%%%%%%%%%%%%%%%%%%%%%%%%%%%%%%%%%%%%%%%%%%%%%%%%%%%%%%%%%%%%%%%%
%%%%%%%%%%%%%%%%%%%%%%%%%%%%%%%%%%%%%%%%%%%%%%%%%%%%%%%%%%%%%%%%%%%%%%%%%%%%%%%%
\section{Introduction}

\LaTeX{} provides a mechanism to structure a large document (such as a book)
into a main file and several child files (containing the chapters)
using the |\include| command.
This mechanism is beneficial for documents
which span hundreds of pages in order to
make the source file(s) more manageable.
Moreover, compilation can be restricted to
selected child files by means of the |\includeonly| command.
The latter feature can be used to reduce the compilation time while editing
(this was significantly more useful in the earlier days of \LaTeX{})
or to generate a smaller document which is easier to navigate.
Another application of |\includeonly| is to generate
documents consisting of selected parts of the complete document.

However, there are a few drawbacks of the plain |\include| mechanism:
\begin{itemize}
\item
The child files cannot be compiled on their own,
they can only be compiled via the main file.
A naive editing environment
(such as a text editor with an option
to have the current file processed by \LaTeX)
may require one to switch to the main file before compiling;
attempting to compile the child file produces errors.
\item
The main file must be modified (each time)
to adjust the |\includeonly| command
to the present needs. This easily leaves the main file in a messy state.
\item
The generated document will always carry the filename
of the main document. This is inconvenient if
several child files are to be compiled and
to be kept for distribution.
\end{itemize}

The present package provides a simple interface
to make child files individually compilable by \LaTeX{}.
Compiling a child file then has the same effect as compiling
the main file with an |\includeonly| command
to select the appropriate child.
Moreover the generated document will carry the name of the child
rather than the main file.
This resolves all three above issues.

This feature is meant to make the editing of books,
thesis documents and lecture notes somewhat more convenient.
However, the package can also be used efficiently for
composing a series of documents (such as exercise sheets)
which are typically distributed individually.
It then assists the author in generating the individual documents
(potentially in different versions)
as well as a document containing the collected series.
Another application is in developing style files
or other kinds of included material
where compilation of the style file could redirect
to a sample or test file.

%%%%%%%%%%%%%%%%%%%%%%%%%%%%%%%%%%%%%%%%%%%%%%%%%%%%%%%%%%%%%%%%%%%%%%%%%%%%%%%%
%%%%%%%%%%%%%%%%%%%%%%%%%%%%%%%%%%%%%%%%%%%%%%%%%%%%%%%%%%%%%%%%%%%%%%%%%%%%%%%%
\section{Usage}

First of all, the package \textsf{childdoc} is \emph{not} a standard
\LaTeXe{} |.sty| style file! Therefore it needs to be invoked in
a non-standard way.

%%%%%%%%%%%%%%%%%%%%%%%%%%%%%%%%%%%%%%%%%%%%%%%%%%%%%%%%%%%%%%%%%%%%%%%%%%%%%%%%
\subsection{Included Files}
\label{sec:include}

%%%%%%%%%%%%%%%%%%%%%%%%%%%%%%%%%%%%%%%%
\DescribeMacro{\childdocmain}
To use the package, add the commands
\begin{center}
\begin{tabular}{l}
|\input{childdoc.def}|\\
|\childdocmain{}|\\
\end{tabular}
\end{center}
at the very top of the main \LaTeX{} file,
in particular \emph{before} the |\documentclass| statement!
The argument of |\childdocmain| should be left empty
(but it must be present).

%%%%%%%%%%%%%%%%%%%%%%%%%%%%%%%%%%%%%%%%
\DescribeMacro{\childdocof}
Furthermore, add the commands
\begin{center}
\begin{tabular}{l}
|\input{childdoc.def}|\\
|\childdocof{|\textit{main}|}|\\
\end{tabular}
\end{center}
at the top of every child file \textit{child}
which is included by |\include{|\textit{child}|}|
from within the main file
(or at least for those files to be compiled individually).
The argument \textit{main} must be the filename of the main file.

There are a couple of
considerations in setting up the main and child documents:

%%%%%%%%%%%%%%%%%%%%%%%%%%%%%%%%%%%%%%%%
\paragraph{Restrictions.}

Please note the following restrictions:
\begin{itemize}
\item
|\childdocmain| must be called with one argument \textit{main}
to ensure compatibility with earlier version of the package.
It must either be empty (|\childdocmain{}|)
or precisely match the filename of the main file in which it is specified.
See \secref{sec:detection} for further information.
\item
The filename \textit{main} must be specified without the |.tex| extension.
\item
The filename \textit{main} is case sensitive
(even in case-insensitive file systems)
due to internal string comparison.
\item
The argument \textit{main} should be fully expanded, it cannot be a macro.
\item
Subdirectories and special characters should be avoided in filenames.
\item
The command |\childdocmain{|\textit{main}|}| must be followed by a whitespace.
It should not be followed immediately by another command
or by a comment mark `|%|'.
This is because the \TeX{} parser reads the token immediately following
the argument of |\childdocmain| and puts it
at the beginning of every child section;
however, a white\-space is ignored.
\end{itemize}

%%%%%%%%%%%%%%%%%%%%%%%%%%%%%%%%%%%%%%%%
\paragraph{Content of Main File.}

It is advisable to place all content in the child files included by |\include|.
Any output contained in the main file will appear in all child documents
unless suppressed manually;
it cannot be suppressed automatically by the |\includeonly| directive
and thus should normally be avoided.
A method to include some content in the main file
by means of conditional processing is described in \secref{sec:conditional}.

%%%%%%%%%%%%%%%%%%%%%%%%%%%%%%%%%%%%%%%%
\paragraph{Page Numbering.}

When only a part of the document is compiled,
the appropriate numbering of pages
(as well as other status parameters)
is determined from the |.aux| files.
The latter contain information from previous passes.
However this information needs to propagate through
all intermediate child documents.
Therefore the page numbering in child documents may well
be inconsistent until the complete document is compiled at least once.

A useful (if unconventional) way to always ensure a consistent
page numbering is to restart the numbering in each child document
and denote the pages by `\textit{child}|.|\textit{page}'
where \textit{child} represents the chapter/section number of the child file.
This can be achieved by the command
|\numberwithin{page}{|\textit{child}|}|
of the \textsf{amsmath} package
where \textit{child} can be |chapter| or |section|
depending on the chosen structuring.
Alternatively, one can modify the macro |\thepage| appropriately
and reset the counter |page| at the start of each child file.

%%%%%%%%%%%%%%%%%%%%%%%%%%%%%%%%%%%%%%%%%%%%%%%%%%%%%%%%%%%%%%%%%%%%%%%%%%%%%%%%
\subsection{Conditional Processing}
\label{sec:conditional}

The package provides a mechanism to compile different versions
of a document. To customise the versions further some conditional processing
can come in handy to distinguish which version is being compiled.
The package provides two macros to describe the compilation context:

%%%%%%%%%%%%%%%%%%%%%%%%%%%%%%%%%%%%%%%%
\DescribeMacro{\ifchilddoc}
The conditional |\ifchilddoc| distinguishes between the compilation of
child documents and the main document:
%
\begin{center}
|\ifchilddoc |\textit{child-code}| |[|\||else |\textit{main-code}]| \||fi|
\end{center}

%%%%%%%%%%%%%%%%%%%%%%%%%%%%%%%%%%%%%%%%
\DescribeMacro{\childdocname}
\DescribeMacro{\childdocjob}
The macro |\childdocname| contains the filename (without extension)
of the main or child file being processed.
Note that |\childdocjob| will always contain the name of the main file.

%%%%%%%%%%%%%%%%%%%%%%%%%%%%%%%%%%%%%%%%
\paragraph{Title Page.}

Conditional processing can be used to include a title or banner page
in the main document when proper precautions are taken.
Importantly, the code in the main file should ensure that the page counter
(as well as other status parameters which are stored in the |.aux| files)
takes the same value after the conditional processing.
Otherwise the page numbers may take divergent values
depending on which part is compiled.

For example, a title page could be declared by:
%
\begin{center}
\begin{tabular}{l}
|\ifchilddoc\||else|\\
|\addtocounter{page}{-1}|\\
\textit{code for title page}\\
|\newpage|\\
|\||fi|
\end{tabular}
\end{center}
%
A banner page for the child documents can be generated by:
%
\begin{center}
\begin{tabular}{l}
|\ifchilddoc|\\
|\addtocounter{page}{-1}|\\
\textit{code for banner page}\\
|\newpage|\\
|\||fi|
\end{tabular}
\end{center}
%
Here one could write a message such as:
\begin{center}
|This is the part \childdocname{} of \childdocjob{}.|
\end{center}

%%%%%%%%%%%%%%%%%%%%%%%%%%%%%%%%%%%%%%%%%%%%%%%%%%%%%%%%%%%%%%%%%%%%%%%%%%%%%%%%
\subsection{Flags}
\label{sec:flags}

The package makes it easy to generate different versions
of the main or child documents.
To this end compilation flags can be defined
and assigned different default values.
They will be particularly useful in conjunction
with the forwarding mechanism described in \secref{sec:forward}.

For example, it may be useful to have a flag |\version|
which can be set to |draft| or |final|.
The document source will contain some conditional code
depending on the value of |\version|.
Suppose further, the flag should default to |final| for the main file
and to |draft| for child files
which is a natural assignment for editing the document.
This is achieved by placing the following code
in the preamble of the main document
(below the |\childdocmain| directive):
%
\begin{center}
\begin{tabular}{l}
|\ifchilddoc|\\
|\providecommand{\version}{draft}|\\
|\||else|\\
|\providecommand{\version}{final}|\\
|\||fi|
\end{tabular}
\end{center}
%
The definition by |\providecommand| makes sure
that previous definitions are not overwritten.
Further statements |\providecommand{\version}{...}|
can thus be added before the above code to override it.

For the main file, one might add a line
(between |\childdocmain| and the above block)
%
\begin{center}
|%\ifchilddoc\||else\providecommand{\version}{draft}\||fi|
\end{center}
%
which can be uncommented to produce a draft version.
Likewise one can add a line to the very top of a child file
(above the |\childdocof{|\textit{main}|}| directive)
%
\begin{center}
|%\providecommand{\version}{final}|
\end{center}
%
which can be uncommented to produce the final version of this child document.

%%%%%%%%%%%%%%%%%%%%%%%%%%%%%%%%%%%%%%%%%%%%%%%%%%%%%%%%%%%%%%%%%%%%%%%%%%%%%%%%
\subsection{Forwarding}
\label{sec:forward}

Different versions of the main or child documents
using compilation flags as described in \secref{sec:flags}
can be (permanently) stored in different files
for convenient compilation, viewing and distribution.
To this end, the package defines a command
to pass on compilation to a different file:

%%%%%%%%%%%%%%%%%%%%%%%%%%%%%%%%%%%%%%%%
\DescribeMacro{\childdocforward}
The command |\childdocforward| redirects processing to
another source file:
%
\begin{center}
\begin{tabular}{l}
|\input{childdoc.def}|\\
|\childdocforward[|\textit{main}|]{|\textit{dest}|}|\\
\end{tabular}
\end{center}
%
The argument \textit{dest} is the destination file
(without extension).
It should be the main file or one of the child files.
Note that further \textsf{childdoc} directives
such as |\childdocof| and |\childdocforward|
in the indicated file will be processed in this form.
The optional argument \textit{main}
passes on directly to the main file \textit{main}
while pretending to compile the child \textit{dest}.
This form behaves as if \textit{dest}
issues |\childdocof{|\textit{main}|}| right away,
and no further \textsf{childdoc} directives will be processed.

%%%%%%%%%%%%%%%%%%%%%%%%%%%%%%%%%%%%%%%%
\DescribeMacro{\...prefix}
In the alternative form |\childdocforwardprefix|,
%
\begin{center}
\begin{tabular}{l}
|\input{childdoc.def}|\\
|\childdocforwardprefix[|\textit{main}|]{|\textit{prefix}|}{|\textit{dest}|}|
\end{tabular}
\end{center}
%
the destination file is determined by a pattern
depending on the current file:
To make this work, the current file must be called
`{\textit{prefix}\hspace{0.2em}\textit{suffix}}'
with \textit{prefix} matching precisely the argument.
Processing is then passed on to the file
`{\textit{dest}\hspace{0.2em}\textit{suffix}}'.
Surely, the same effect is achieved by
directly specifying the
argument `{\textit{dest}\hspace{0.2em}\textit{suffix}}'
in the first form.
However, that requires to set up a different file
for each child. With the alternative form of the command
all these files can have exactly the same content
which simplifies setting them up and maintaining them.

For example, the following file |draft.tex|
with a compilation flag |\version| as described in \secref{sec:flags}
compiles the main document as a draft:
%
\begin{center}
\begin{tabular}{l}
|\def\version{draft}|\\
|\input{childdoc.def}|\\
|\childdocforward{|\textit{main}|}|
\end{tabular}
\end{center}
%
Likewise, the following files |final|\textit{nn}|.tex|
compile the final version of the child document
|child|\textit{nn}|.tex|:
%
\begin{center}
\begin{tabular}{l}
|\def\version{final}|\\
|\input{childdoc.def}|\\
|\childdocforwardprefix{final}{child}|
\end{tabular}
\end{center}
%

Note that when several versions of a main file and/or of each child file
are to be generated, it may be convenient to set up a |Makefile| or
shell script to automatise the process.

%%%%%%%%%%%%%%%%%%%%%%%%%%%%%%%%%%%%%%%%%%%%%%%%%%%%%%%%%%%%%%%%%%%%%%%%%%%%%%%%
\subsection{Command Line Processing}
\label{sec:commandline}

The effect of redirection files can also be achieved by invoking
the \LaTeX{} compiler with a more elaborate command line.
Most conveniently this should be done as part
of a shell script or a |Makefile|.

When using \textsf{childdoc} in the main file, the following
command lines effectively perform a redirection
(note that depending on the shell being used,
backslashes may have to be doubled: `|\|' $\to$ `|\\|'):
%
\begin{center}
|... -jobname "|\textit{target}|" |\\|"|[\textit{flags}]%
|\input{childdoc.def}\childdocforward[|\textit{main}|]{|\textit{dest}|}"|
\end{center}
%
Here \textit{target} is the name of the output file,
\textit{main} is the name of the main file
and \textit{dest} is the name of the main or child file to be processed
(all filenames without extensions).
The optional argument \textit{main} can be omitted
if \textit{main} matches \textit{dest}.
Optionally, compilation \textit{flags} can be defined via |\def| commands.
This command line makes the \TeX{} engine believe
it is compiling the file \textit{target}
whose content is specified as the latter parameter.
The provided code then forwards the processing to
\textit{main} or \textit{dest} as described in \secref{sec:forward}.

%%%%%%%%%%%%%%%%%%%%%%%%%%%%%%%%%%%%%%%%%%%%%%%%%%%%%%%%%%%%%%%%%%%%%%%%%%%%%%%%
\subsection{Include by Input}
\label{sec:input}

Including child documents by |\include| has some restrictions by design.
Most notably, the content of a child document always occupies
its own set of pages; pages cannot be shared between child documents.
Usually, this behaviour makes perfect sense
because each child document contain an essential part of the document.
However, in some situations it may be desirable to compose
a document from a collection of parts
without having mandatory page breaks between then.
For this case, the package
provides a mechanism to include parts
by |\input| which can also be processed individually.
However, by construction this mechanism
requires manual handling of the content to be output.

%%%%%%%%%%%%%%%%%%%%%%%%%%%%%%%%%%%%%%%%
\DescribeMacro{\ifchilddocmanual}
The main file should be prepared as usual, see \secref{sec:include}.
However, the document body must make a distinction
between processing of an individual part and of the main document, e.g.:
%
\begin{center}
\begin{tabular}{l}
|\ifchilddocmanual|\\
|\input{\childdocname}|\\
|\||else|\\
\textit{document body with }|\input{|\textit{part}|}|\\
|\||fi|
\end{tabular}
\end{center}
%
The conditional |\ifchilddocmanual| is true whenever
a part to be included by |\input| is being compiled,
and the name of the part is stored in |\childdocname|.

%%%%%%%%%%%%%%%%%%%%%%%%%%%%%%%%%%%%%%%%
\DescribeMacro{\childdocby}
Each part to be included by |\input| should start with:
%
\begin{center}
\begin{tabular}{l}
|\input{childdoc.def}|\\
|\childdocby{|\textit{main}|}|\\
\end{tabular}
\end{center}
%
The directive |\childdocby| is similar to |\childdocof|
described in \secref{sec:include},
but the subsequent selection of content must be done manually.
To that end, both |\ifchilddoc| and |\ifchilddocmanual|
will be true upon processing of a part,
and the name of the part is stored in |\childdocname|.
Note that |\jobname| will be set to the filename of the current part
so that each part receives an individual |.aux| file
that does not interfere with the |.aux| file(s) of the main document.
This behaviour can be altered by the alternative form
|\childdocby[*]{|\textit{main}|}| (with a non-empty optional argument)
which uses the |.aux| file of the main document
by setting |\jobname| to \textit{main}.

%%%%%%%%%%%%%%%%%%%%%%%%%%%%%%%%%%%%%%%%%%%%%%%%%%%%%%%%%%%%%%%%%%%%%%%%%%%%%%%%
\subsection{Driver Development}
\label{sec:driver}

The \textsf{childdoc} mechanism can also be use for the development
of definition files such as \LaTeX{} styles or classes.
This case differs from the above setup with multiple parts
included by |\include| in that no |\includeonly| should be invoked.
This can be achieved by starting the include file
(before |\ProvidesPackage|) with:
%
\begin{center}
\begin{tabular}{l}
|\input{childdoc.def}|\\
|\childdocforward{|\textit{main}|}|\\
\end{tabular}
\end{center}
%
or alternatively with:
%
\begin{center}
\begin{tabular}{l}
|\input{childdoc.def}|\\
|\childdocby{|\textit{main}|}|\\
\end{tabular}
\end{center}
%
Both forms have slightly different effects as described above.
The main file is prepared as usual, see \secref{sec:include}.

%%%%%%%%%%%%%%%%%%%%%%%%%%%%%%%%%%%%%%%%%%%%%%%%%%%%%%%%%%%%%%%%%%%%%%%%%%%%%%%%
\subsection{Legacy Detection}
\label{sec:detection}

The directive |\childdocmain| in the main file can detect
whether the complete document or merely a child is to be compiled
even without using the directive |\childdocof|.
This method is deprecated because it is less robust
and there is no compelling reason to use it;
it is merely provided for backward compatibility
and it may be removed in future versions.

If the detection mechanism is to be used,
it is mandatory to correctly specify
the filename of the main file as the argument of |\childdocmain|:
%
\begin{center}
\begin{tabular}{l}
|\input{childdoc.def}|\\
|\childdocmain{|\textit{main}|}|\\
\end{tabular}
\end{center}
%
If |\jobname| does not match the argument \textit{main} of |\childdocmain|,
it is assumed that |\jobname| points to the child file to be compiled.
When using |\childdocmain| with the main file specified as argument,
it suffices to start a child file
with just |\input{|\textit{main}|}|
without loading of the package and using |\childdocof|.
If instead all processing is done
with the appropriate \textsf{childdoc} directives,
the argument of \textit{main} of |\childdocmain| can be empty.

An alternative version of the command line processing described
in \secref{sec:commandline} using the detection mechanism reads:
%
\begin{center}
|... -jobname "|\textit{target}|" "|[\textit{flags}]%
[|\def\jobname{|\textit{dest}|}|]|\input{|\textit{main}|}"|
\end{center}

%%%%%%%%%%%%%%%%%%%%%%%%%%%%%%%%%%%%%%%%%%%%%%%%%%%%%%%%%%%%%%%%%%%%%%%%%%%%%%%%
\subsection{Manual Code}
\label{sec:manual}

In case one cannot be certain whether the definitions file |childdoc.def|
is installed on the target \TeX{} distribution
and one prefers not to ship it,
it is conceivable to paste a few relevant commands into the sources.

To that end, drop all statements |\input{childdoc.def}|
and perform the replacements as outlined below.
Instead of |\childdocmain{|\textit{main}|}| add the following code
to the top of the main file:
%
\begin{center}
\begin{tabular}{l}
|\||ifdefined\childdocname\endinput\||fi\newif\ifchilddoc|\\
|\edef\childdocname{\scantokens\expandafter{\jobname\noexpand}}|\\
|\def\childdocmain{|\textit{main}|}\||ifx\childdocmain\childdocname\||else|\\
|\childdoctrue\includeonly{\childdocname}\let\jobname\childdocmain\||fi|\\
\end{tabular}
\end{center}
%
Instead of |\childdocof{|\textit{main}|}| just include the main file
at the top of each child file:
%
\begin{center}
|\input{|\textit{main}|}|
\end{center}
%
A simple redirection |\childdocforward{|\textit{dest}|}| is achieved by:
%
\begin{center}
|\def\jobname{|\textit{dest}|}\input{\jobname}|
\end{center}
%
The redirection with prefix
|\childdocforwardprefix[|\textit{prefix}|]{|\textit{dest}|}|
is accomplished by:
%
\begin{center}
\begin{tabular}{l}
|{\edef\jobname{\scantokens\expandafter{\jobname\noexpand}}|\\
|\def\redirectjob |\textit{prefix}|#1~~~{\gdef\jobname{|\textit{dest}|#1}}|\\
|\expandafter\redirectjob\jobname~~~}\input{\jobname}|
\end{tabular}
\end{center}

In an alternative approach,
child documents can be compiled by a specific command line
without additional code or specific definitions:
%
\begin{center}
|... -jobname "|\textit{target}|" "|[\textit{flags}]%
|\includeonly{|\textit{dest}|}\input{|\textit{main}|}"|
\end{center}
%

%%%%%%%%%%%%%%%%%%%%%%%%%%%%%%%%%%%%%%%%%%%%%%%%%%%%%%%%%%%%%%%%%%%%%%%%%%%%%%%%
%%%%%%%%%%%%%%%%%%%%%%%%%%%%%%%%%%%%%%%%%%%%%%%%%%%%%%%%%%%%%%%%%%%%%%%%%%%%%%%%
\section{Information}

%%%%%%%%%%%%%%%%%%%%%%%%%%%%%%%%%%%%%%%%%%%%%%%%%%%%%%%%%%%%%%%%%%%%%%%%%%%%%%%%
\subsection{Copyright}

Copyright \copyright{} 2017--2018 Niklas Beisert

This work may be distributed and/or modified under the
conditions of the \LaTeX{} Project Public License, either version 1.3
of this license or (at your option) any later version.
The latest version of this license is in
  \url{http://www.latex-project.org/lppl.txt}
and version 1.3 or later is part of all distributions of \LaTeX{}
version 2005/12/01 or later.

This work has the LPPL maintenance status `maintained'.

The Current Maintainer of this work is Niklas Beisert.

This work consists of the files |README.txt|, |childdoc.ins| and |childdoc.dtx|
as well as the derived files |childdoc.def|, |cdocsamp.tex|
with |cdocsch1.tex|, |cdocsch2.tex|, |cdocspt3.tex|, |cdocspt4.tex|,
|cdocsdrf.tex|, |cdocsfn1.tex|, |cdocsfn2.tex|
as well as |childdoc.pdf|.

%%%%%%%%%%%%%%%%%%%%%%%%%%%%%%%%%%%%%%%%%%%%%%%%%%%%%%%%%%%%%%%%%%%%%%%%%%%%%%%%
\subsection{Files and Installation}

The package consists of the files:
%
\begin{center}
\begin{tabular}{ll}
    |README.txt|   & readme file \\
    |childdoc.ins| & installation file \\
    |childdoc.dtx| & source file \\
    |childdoc.def| & definition file \\
    |cdocsamp.tex| & sample main file \\
    |cdocsch1.tex| & sample include file \\
    |cdocsch2.tex| & sample include file \\
    |cdocspt3.tex| & sample part file \\
    |cdocspt4.tex| & sample part file \\
    |cdocsdrf.tex| & sample redirection file \\
    |cdocsfn1.tex| & sample redirection file \\
    |cdocsfn2.tex| & sample redirection file \\
    |childdoc.pdf| & manual
\end{tabular}
\end{center}
%
The distribution consists of the files
|README.txt|, |childdoc.ins| and |childdoc.dtx|.
%
\begin{itemize}
\item
Run (pdf)\LaTeX{} on |childdoc.dtx|
to compile the manual |childdoc.pdf| (this file).
\item
Run \LaTeX{} on |childdoc.ins| to create the definitions file |childdoc.def|
and the sample |cdocsamp.tex| with include files
|cdocsch1.tex|, |cdocsch2.tex|, |cdocspt3.tex|, |cdocspt4.tex|,
|cdocsdrf.tex|, |cdocsfn1.tex|, |cdocsfn2.tex|.
Then copy the file |childdoc.def| to an appropriate directory of your \LaTeX{}
distribution, e.g.\ \textit{texmf-root}|/tex/latex/childdoc|.
\end{itemize}

%%%%%%%%%%%%%%%%%%%%%%%%%%%%%%%%%%%%%%%%%%%%%%%%%%%%%%%%%%%%%%%%%%%%%%%%%%%%%%%%
\subsection{Related CTAN Packages}

There are several other packages which offer a similar functionality:
%
\begin{itemize}
\item
The packages
\href{http://ctan.org/pkg/docmute}{\textsf{docmute}},
\href{http://ctan.org/pkg/includex}{\textsf{includex}} and
\href{http://ctan.org/pkg/standalone}{\textsf{standalone}}
provide commands to include only the document body of
a child file thus allowing both files to be compiled individually.
\item
The packages \href{http://ctan.org/pkg/subdocs}{\textsf{subdocs}}
and \href{http://ctan.org/pkg/subfiles}{\textsf{subfiles}}
provide structures in which the main and child documents can be
encapsulated and allowing them to be compiled individually.
The inclusion mechanism is different from the conventional |\include|.
\item
The package \href{http://ctan.org/pkg/combine}{\textsf{combine}}
is an elaborate solution to combine several documents into one.
\end{itemize}
%
See also the CTAN topic \href{http://ctan.org/topic/subdocs}{\textsf{subdocs}}
for further related packages.
The present package differs from the above solutions in that
a document structure constructed with the conventional |\include| mechanism
just needs two extra commands at the top of every file
such that all constituent files can be compiled individually.

%%%%%%%%%%%%%%%%%%%%%%%%%%%%%%%%%%%%%%%%%%%%%%%%%%%%%%%%%%%%%%%%%%%%%%%%%%%%%%%%
%\subsection{Feature Suggestions}
%
%The following is a list of features which may be useful for future
%versions of this package:
%%
%\begin{itemize}
%\item
%\ldots
%\end{itemize}

%%%%%%%%%%%%%%%%%%%%%%%%%%%%%%%%%%%%%%%%%%%%%%%%%%%%%%%%%%%%%%%%%%%%%%%%%%%%%%%%
\subsection{Revision History}

%%%%%%%%%%%%%%%%%%%%%%%%%%%%%%%%%%%%%%%%
\paragraph{v2.0:} 2018/12/30

\begin{itemize}
\item
immediate forward processing
\item
added |\childdocby| mechanism
\item
manual restructured
\end{itemize}

%%%%%%%%%%%%%%%%%%%%%%%%%%%%%%%%%%%%%%%%
\paragraph{v1.6:} 2018/01/17

\begin{itemize}
\item
application for development of include files
\item
corrections to manual
\end{itemize}

%%%%%%%%%%%%%%%%%%%%%%%%%%%%%%%%%%%%%%%%
\paragraph{v1.5:} 2017/05/21

\begin{itemize}
\item
more complete structuring introduced
\item
|\childdocof| introduced
\item
|\childdoc| renamed to |\childdocmain|
\item
|\childredirect| renamed to |\childdocforward| and |\childdocforwardprefix|
and functionality expanded
\end{itemize}

%%%%%%%%%%%%%%%%%%%%%%%%%%%%%%%%%%%%%%%%
\paragraph{v1.0:} 2017/04/27

\begin{itemize}
\item
manual and install package
\item
first version published on CTAN
\end{itemize}

%%%%%%%%%%%%%%%%%%%%%%%%%%%%%%%%%%%%%%%%
\paragraph{v0.6:} 2017/04/26

\begin{itemize}
\item
redirection mechanism added
\end{itemize}

%%%%%%%%%%%%%%%%%%%%%%%%%%%%%%%%%%%%%%%%
\paragraph{v0.5:} 2017/04/26

\begin{itemize}
\item
functionality in definition file
\end{itemize}


%%%%%%%%%%%%%%%%%%%%%%%%%%%%%%%%%%%%%%%%%%%%%%%%%%%%%%%%%%%%%%%%%%%%%%%%%%%%%%%%
%%%%%%%%%%%%%%%%%%%%%%%%%%%%%%%%%%%%%%%%%%%%%%%%%%%%%%%%%%%%%%%%%%%%%%%%%%%%%%%%
%%%%%%%%%%%%%%%%%%%%%%%%%%%%%%%%%%%%%%%%%%%%%%%%%%%%%%%%%%%%%%%%%%%%%%%%%%%%%%%%
\appendix

\settowidth\MacroIndent{\rmfamily\scriptsize 000\ }

 \DocInput{childdoc.dtx}

\end{document}
%</driver>
% \fi
%
% %%%%%%%%%%%%%%%%%%%%%%%%%%%%%%%%%%%%%%%%%%%%%%%%%%%%%%%%%%%%%%%%%%%%%%%%%%%%%%
% %%%%%%%%%%%%%%%%%%%%%%%%%%%%%%%%%%%%%%%%%%%%%%%%%%%%%%%%%%%%%%%%%%%%%%%%%%%%%%
% \section{Sample}
%\iffalse
%<*samplemain>
%\fi
%
% The following presents a sample document
% with two chapters, two parts, a title page,
% a compile flag as well as three forwarding files to set the flag.
% It consists of eight |.tex| files:
% \begin{center}
% \begin{tabular}{ll}
% |cdocsamp.tex|&main file\\
% |cdocsch1.tex|&include file for chapter 1\\
% |cdocsch2.tex|&include file for chapter 2\\
% |cdocspt3.tex|&include file for part 3\\
% |cdocspt4.tex|&include file for part 4\\
% |cdocsdrf.tex|&forwarding file for main file in draft mode\\
% |cdocsfi1.tex|&forwarding file for final version of chapter 1\\
% |cdocsfi2.tex|&forwarding file for final version of chapter 2\\
% \end{tabular}
% \end{center}
% Each of the eight files can be compiled directly by the \LaTeX{} compiler.
%
% %%%%%%%%%%%%%%%%%%%%%%%%%%%%%%%%%%%%%%
% \paragraph{Main File.}
%
% The main file is called |cdocsamp.tex|.
%
% Load the \textsf{childdoc} definitions and
% declare the filename for the main document:
%    \begin{macrocode}
\input{childdoc.def}
\childdocmain{}
%    \end{macrocode}

% Optional override for |\version| flag:
%    \begin{macrocode}
%%\ifchilddoc\else\providecommand{\version}{draft}\fi
%    \end{macrocode}

% Define the default values for the |\version| flag
% (|final| for the main file and |draft| for childs):
%    \begin{macrocode}
\ifchilddoc
\providecommand{\version}{draft}
\else
\providecommand{\version}{final}
\fi
%    \end{macrocode}

% Load the standard document class:
%    \begin{macrocode}
\documentclass[12pt]{article}
%    \end{macrocode}

% Start the document body:
%    \begin{macrocode}
\begin{document}
%    \end{macrocode}

% Declare a title page.
% Print title, part of document being processed and version flag:
%    \begin{macrocode}
\addtocounter{page}{-1}
\begin{center}
{\LARGE\bfseries{}childdoc example\par}
\vspace{1cm}
\ifchilddoc
\ifchilddocmanual part\else chapter\fi:
`\childdocname' of `\childdocjob'\par
\else
main document: `\childdocjob'\par
\fi
version: \version\par
\end{center}
\newpage
%    \end{macrocode}

% Manually include selected file,
% otherwise process as usual:
%    \begin{macrocode}
\ifchilddocmanual
\section*{part `\childdocname'}
\input{\childdocname}
\else
%    \end{macrocode}

% Include the two chapters:
%    \begin{macrocode}
\include{cdocsch1}
\include{cdocsch2}
%    \end{macrocode}

% Include the two parts unless only chapters should be displayed:
%    \begin{macrocode}
\ifchilddoc\else
\section{part three}
\input{cdocspt3}
\section{part four}
\input{cdocspt4}
\fi
%    \end{macrocode}

% Process as usual until here:
%    \begin{macrocode}
\fi
%    \end{macrocode}

% End of document body:
%    \begin{macrocode}
\end{document}
%    \end{macrocode}
%\iffalse
%</samplemain>
%\fi
%
% %%%%%%%%%%%%%%%%%%%%%%%%%%%%%%%%%%%%%%
% \paragraph{Chapter Include Files.}
%
% The include files are called |cdocsch1.tex| and |cdocsch2.tex|.
%
%\iffalse
%<*samplechap1|samplechap2>
%\fi

% Optional override for |\version| flag:
%    \begin{macrocode}
%%\providecommand{\version}{final}
%    \end{macrocode}

% Include the main document:
%    \begin{macrocode}
\input{childdoc.def}
\childdocof{cdocsamp}
%    \end{macrocode}

%\iffalse
%</samplechap1|samplechap2>
%\fi
%
%\iffalse
%<*samplechap1>
%\fi
% Some text for chapter 1:
%    \begin{macrocode}
\section{one}
some text in chapter one
%    \end{macrocode}

%\iffalse
%</samplechap1>
%\fi
% Some text for chapter 2:
%\iffalse
%<*samplechap2>
%\fi
%    \begin{macrocode}
\section{two}
more text in chapter two
%    \end{macrocode}

%\iffalse
%</samplechap2>
%\fi
%
% %%%%%%%%%%%%%%%%%%%%%%%%%%%%%%%%%%%%%%
% \paragraph{Part Include Files.}
%
% The include files are called |cdocspt3.tex| and |cdocspt4.tex|.
%
%\iffalse
%<*samplepart3|samplepart4>
%\fi

% Optional override for |\version| flag:
%    \begin{macrocode}
%%\providecommand{\version}{final}
%    \end{macrocode}

% Include the main document:
%    \begin{macrocode}
\input{childdoc.def}
\childdocby{cdocsamp}
%    \end{macrocode}

%\iffalse
%</samplepart3|samplepart4>
%\fi
%
%\iffalse
%<*samplepart3>
%\fi
% Some text for part 3:
%    \begin{macrocode}
some text in part three
%    \end{macrocode}

%\iffalse
%</samplepart3>
%\fi
% Some text for part 4:
%\iffalse
%<*samplepart4>
%\fi
%    \begin{macrocode}
more text in part four
%    \end{macrocode}

%\iffalse
%</samplepart4>
%\fi
%
% %%%%%%%%%%%%%%%%%%%%%%%%%%%%%%%%%%%%%%
% \paragraph{Forwarding for a Complete Draft.}
%
% The following forwarding file |cdocsdrf.tex|
% compiles the main document in draft mode:
%\iffalse
%<*sampledraft>
%\fi
%    \begin{macrocode}
\def\version{draft}
\input{childdoc.def}
\childdocforward{cdocsamp}
%    \end{macrocode}

%\iffalse
%</sampledraft>
%\fi
%
% %%%%%%%%%%%%%%%%%%%%%%%%%%%%%%%%%%%%%%
% \paragraph{Forwarding for Final Version of the Chapters.}
%
% The following forwarding files |cdocsfn1.tex| and |cdocsfn2.tex|
% (with identical content)
% compile the final versions of the child documents
% |cdocsch1.tex| and |cdocsch2.tex|, respectively:
%\iffalse
%<*samplefinal>
%\fi
%    \begin{macrocode}
\def\version{final}
\input{childdoc.def}
\childdocforwardprefix[cdocsamp]{cdocsfn}{cdocsch}
%    \end{macrocode}

%\iffalse
%</samplefinal>
%\fi
%
% %%%%%%%%%%%%%%%%%%%%%%%%%%%%%%%%%%%%%%
% \paragraph{Command Line Processing.}
%
% The following three command lines generate the output files
% |cdocscld|, |cdocscl1| and |cdocscl2|
% which should be identical to
% |cdocsdrf|, |cdocsch1| and |cdocsfn2|, respectively:
% \begin{center}
% \begin{tabular}{l}
% |latex -jobname cdocscld \|\\
% |  "\def\version{draft}\input{childdoc.def}\childdocforward{cdocsamp}"|\\
% |latex -jobname cdocscl1 \|\\
% |  "\input{childdoc.def}\childdocforward[cdocsamp]{cdocsch1}"|\\
% |latex -jobname cdocscl2 \|\\
% |  "\def\version{final}\input{childdoc.def}\childdocforward{cdocsch2}"|
% \end{tabular}
% \end{center}
% Note that the trailing backslash on each first line
% merely continues the input to the second line
% (for convenient cut ant paste).
% Furthermore, the command |latex| can be replaced by any
% of its alternative versions such as |pdflatex|.
%
% %%%%%%%%%%%%%%%%%%%%%%%%%%%%%%%%%%%%%%%%%%%%%%%%%%%%%%%%%%%%%%%%%%%%%%%%%%%%%%
% %%%%%%%%%%%%%%%%%%%%%%%%%%%%%%%%%%%%%%%%%%%%%%%%%%%%%%%%%%%%%%%%%%%%%%%%%%%%%%
% \section{Implementation}
%\iffalse
%<*package>
%\fi
%
% This section describes the definitions file |childdoc.def|.

% The definitions cannot be loaded using |\usepackage| or |\RequirePackage|
% which has a mechanism to prevent loading a style file more than once.
% When loading the definitions by means of |\input|
% multiple instances have to be prevented manually:
%\iffalse
%This code needs to be before the `\ProvidesFile' directive
%which is defined at the beginning of this file.
%Therefore it is also placed there and commented out here.
%</package>
%<*discard>
%\fi
%    \begin{macrocode}
\ifdefined\childdocmain\endinput\fi
%    \end{macrocode}
%\iffalse
%</discard>
%<*package>
%\fi
%
% \macro{\ifchilddoc}
% \macro{\ifchilddocmanual}
% The conditional |\ifchilddoc| tells whether a
% child (true) or main (false) document is being compiled.
% The conditional |\ifchilddocmanual| tells whether
% the |\includeonly| mechanism is used (false) or
% the selection of child files must be performed manually (true).
% The definitions initialise to false:
%    \begin{macrocode}
\newif\ifchilddoc
\newif\ifchilddocmanual
%    \end{macrocode}

% \macro{\childdocname}
% \macro{\childdocjob}
% The macro |\childdocname| stores the name of the main document
% to be compiled. The macro |\childdocjob| stores the name of
% the document on which the \LaTeX{} compiler was originally invoked.
% The content of |\jobname| cannot be compared
% to filenames specified in the source due to different catcodes.
% The following code rescans |\jobname|, stores the result
% in |\childdocname| and saves a copy in |\childdocjob|:
%    \begin{macrocode}
\edef\childdocname{\scantokens\expandafter{\jobname\noexpand}}
\let\childdocjob\childdocname
%    \end{macrocode}

% \macro{\childdocdisable}
% The macro |\childdocdisable| prevents the main file
% from being processed more than once.
% At this stage, the main document command |\childdocmain|
% is assumed to be called once again where it should do nothing.
% Any subsequent call to it should prevent
% a secondary processing of the main document
% It overwrites the forwarding commands
% |\childdocof| and |\childdocforward|
% with empty macros to prevent further inclusions of the main document:
%    \begin{macrocode}
\newcommand{\childdocdisable}
{
  \renewcommand{\childdocmain}[1]{\renewcommand{\childdocmain}[1]{\endinput}}
  \renewcommand{\childdocof}[1]{}
  \renewcommand{\childdocby}[2][]{}
  \renewcommand{\childdocforward}[2][]{}
  \renewcommand{\childdocdisable}{}
}
%    \end{macrocode}

% \macro{\childdocmain}
% The macro |\childdocmain| is to be called at the top of the main file
% with nothing or the main filename (without extension) as argument.
% First, it breaks loops.
% If the argument is not empty and does not match |\childdocname|
% (which is set by the first inclusion of |childdoc.def|),
% |\ifchilddoc| is set to true, |\includeonly| is applied to the child file
% and |\jobname| is set to the main file
% (for proper handling of |.aux| files):
%    \begin{macrocode}
\newcommand{\childdocmain}[1]
{
  \childdocdisable\childdocmain{}
  \if?#1?\else
    \begingroup
      \def\childdoctmp{#1}
      \ifx\childdoctmp\childdocname
        \def\childdoctmp{}
      \else
        \def\childdoctmp
        {
          \childdoctrue
          \includeonly{\childdocname}
          \def\childdocjob{#1}
          \def\jobname{#1}
        }
      \fi
      \expandafter
    \endgroup
    \childdoctmp
  \fi
}
%    \end{macrocode}

% \macro{\childdocof}
% The command |\childdocof| redirects
% compilation to the main file |#1|.
%    \begin{macrocode}
\newcommand{\childdocof}[1]
{
  \childdocdisable
  \childdoctrue
  \includeonly{\childdocname}
  \def\jobname{#1}
  \def\childdocjob{#1}
  \input{#1}
}
%    \end{macrocode}

% \macro{\childdocby}
% The command |\childdocby| ....
%    \begin{macrocode}
\newcommand{\childdocby}[2][]
{
  \childdocdisable
  \childdoctrue
  \childdocmanualtrue
  \if?#1?\else
    \def\jobname{#2}
  \fi
  \def\childdocjob{#2}
  \input{#2}
  \endinput
}
%    \end{macrocode}

% \macro{\childdocforward}
% The command |\childdocforward| redirects
% compilation to the main file or
% (if the optional argument is given) a child file.
% Parameters are set as if the main file
% or a child file starting with |\childdocof| was compiled.
% Then compilation is handed over to the main file:
%    \begin{macrocode}
\newcommand{\childdocforward}[2][]
{
  \begingroup
    \if?#1?
      \def\childdoctmp
      {
        \def\childdocname{#2}
        \def\childdocjob{#2}
        \def\jobname{#2}
        \input{#2}
        \endinput
      }
    \else
      \def\childdoctmp
      {
        \childdocdisable
        \def\childdocname{#2}
        \childdoctrue
        \includeonly{#2}
        \def\childdocjob{#1}
        \def\jobname{#1}
        \input{#1}
        \endinput
      }
    \fi
    \expandafter
  \endgroup
  \childdoctmp
}
%    \end{macrocode}

% \macro{\childdocforwardprefix}
% The command |\childdocforwardprefix| redirects
% compilation to the main or a child file by means of a pattern.
% The prefix |#1| in the current filename is replaced by |#2|
% and the suffix of the current filename is kept
% (it is assumed that the filename does not contain the substring `|~~~|'
% which is used as a delimiter).
% Compilation is handed over to the new file by |\childdocforward|:
%    \begin{macrocode}
\newcommand{\childdocforwardprefix}[3][]
{
  \begingroup
    \def\childdocextract #2##1~~~{\def\childdoctmp{\childdocforward[#1]{#3##1}}}
    \expandafter\childdocextract\childdocname~~~
    \expandafter
  \endgroup
  \childdoctmp
}
%    \end{macrocode}

% \macro{\childdoc}
% The deprecated macro |\childdoc| is a legacy version of |\childdocmain|:
%    \begin{macrocode}
\newcommand{\childdoc}{\childdocmain}
%    \end{macrocode}

% \macro{\childdocredirect}
% The deprecated macro |\childdocredirect| is a legacy version
% of |\childdocforward| and |\childdocforwardprefix|:
%    \begin{macrocode}
\newcommand{\childdocredirect}[2][]
{
  \begingroup
    \if?#1?
      \def\childdoctmp{\childdocforward{#2}}
    \else
      \def\childdoctmp{\childdocforwardprefix{#1}{#2}}
    \fi
    \expandafter
  \endgroup
  \childdoctmp
}
%    \end{macrocode}

%\iffalse
%</package>
%\fi
%
\endinput
|\\
|\childdocof{|\textit{main}|}|\\
\end{tabular}
\end{center}
at the top of every child file \textit{child}
which is included by |\include{|\textit{child}|}|
from within the main file
(or at least for those files to be compiled individually).
The argument \textit{main} must be the filename of the main file.

There are a couple of
considerations in setting up the main and child documents:

%%%%%%%%%%%%%%%%%%%%%%%%%%%%%%%%%%%%%%%%
\paragraph{Restrictions.}

Please note the following restrictions:
\begin{itemize}
\item
|\childdocmain| must be called with one argument \textit{main}
to ensure compatibility with earlier version of the package.
It must either be empty (|\childdocmain{}|)
or precisely match the filename of the main file in which it is specified.
See \secref{sec:detection} for further information.
\item
The filename \textit{main} must be specified without the |.tex| extension.
\item
The filename \textit{main} is case sensitive
(even in case-insensitive file systems)
due to internal string comparison.
\item
The argument \textit{main} should be fully expanded, it cannot be a macro.
\item
Subdirectories and special characters should be avoided in filenames.
\item
The command |\childdocmain{|\textit{main}|}| must be followed by a whitespace.
It should not be followed immediately by another command
or by a comment mark `|%|'.
This is because the \TeX{} parser reads the token immediately following
the argument of |\childdocmain| and puts it
at the beginning of every child section;
however, a white\-space is ignored.
\end{itemize}

%%%%%%%%%%%%%%%%%%%%%%%%%%%%%%%%%%%%%%%%
\paragraph{Content of Main File.}

It is advisable to place all content in the child files included by |\include|.
Any output contained in the main file will appear in all child documents
unless suppressed manually;
it cannot be suppressed automatically by the |\includeonly| directive
and thus should normally be avoided.
A method to include some content in the main file
by means of conditional processing is described in \secref{sec:conditional}.

%%%%%%%%%%%%%%%%%%%%%%%%%%%%%%%%%%%%%%%%
\paragraph{Page Numbering.}

When only a part of the document is compiled,
the appropriate numbering of pages
(as well as other status parameters)
is determined from the |.aux| files.
The latter contain information from previous passes.
However this information needs to propagate through
all intermediate child documents.
Therefore the page numbering in child documents may well
be inconsistent until the complete document is compiled at least once.

A useful (if unconventional) way to always ensure a consistent
page numbering is to restart the numbering in each child document
and denote the pages by `\textit{child}|.|\textit{page}'
where \textit{child} represents the chapter/section number of the child file.
This can be achieved by the command
|\numberwithin{page}{|\textit{child}|}|
of the \textsf{amsmath} package
where \textit{child} can be |chapter| or |section|
depending on the chosen structuring.
Alternatively, one can modify the macro |\thepage| appropriately
and reset the counter |page| at the start of each child file.

%%%%%%%%%%%%%%%%%%%%%%%%%%%%%%%%%%%%%%%%%%%%%%%%%%%%%%%%%%%%%%%%%%%%%%%%%%%%%%%%
\subsection{Conditional Processing}
\label{sec:conditional}

The package provides a mechanism to compile different versions
of a document. To customise the versions further some conditional processing
can come in handy to distinguish which version is being compiled.
The package provides two macros to describe the compilation context:

%%%%%%%%%%%%%%%%%%%%%%%%%%%%%%%%%%%%%%%%
\DescribeMacro{\ifchilddoc}
The conditional |\ifchilddoc| distinguishes between the compilation of
child documents and the main document:
%
\begin{center}
|\ifchilddoc |\textit{child-code}| |[|\||else |\textit{main-code}]| \||fi|
\end{center}

%%%%%%%%%%%%%%%%%%%%%%%%%%%%%%%%%%%%%%%%
\DescribeMacro{\childdocname}
\DescribeMacro{\childdocjob}
The macro |\childdocname| contains the filename (without extension)
of the main or child file being processed.
Note that |\childdocjob| will always contain the name of the main file.

%%%%%%%%%%%%%%%%%%%%%%%%%%%%%%%%%%%%%%%%
\paragraph{Title Page.}

Conditional processing can be used to include a title or banner page
in the main document when proper precautions are taken.
Importantly, the code in the main file should ensure that the page counter
(as well as other status parameters which are stored in the |.aux| files)
takes the same value after the conditional processing.
Otherwise the page numbers may take divergent values
depending on which part is compiled.

For example, a title page could be declared by:
%
\begin{center}
\begin{tabular}{l}
|\ifchilddoc\||else|\\
|\addtocounter{page}{-1}|\\
\textit{code for title page}\\
|\newpage|\\
|\||fi|
\end{tabular}
\end{center}
%
A banner page for the child documents can be generated by:
%
\begin{center}
\begin{tabular}{l}
|\ifchilddoc|\\
|\addtocounter{page}{-1}|\\
\textit{code for banner page}\\
|\newpage|\\
|\||fi|
\end{tabular}
\end{center}
%
Here one could write a message such as:
\begin{center}
|This is the part \childdocname{} of \childdocjob{}.|
\end{center}

%%%%%%%%%%%%%%%%%%%%%%%%%%%%%%%%%%%%%%%%%%%%%%%%%%%%%%%%%%%%%%%%%%%%%%%%%%%%%%%%
\subsection{Flags}
\label{sec:flags}

The package makes it easy to generate different versions
of the main or child documents.
To this end compilation flags can be defined
and assigned different default values.
They will be particularly useful in conjunction
with the forwarding mechanism described in \secref{sec:forward}.

For example, it may be useful to have a flag |\version|
which can be set to |draft| or |final|.
The document source will contain some conditional code
depending on the value of |\version|.
Suppose further, the flag should default to |final| for the main file
and to |draft| for child files
which is a natural assignment for editing the document.
This is achieved by placing the following code
in the preamble of the main document
(below the |\childdocmain| directive):
%
\begin{center}
\begin{tabular}{l}
|\ifchilddoc|\\
|\providecommand{\version}{draft}|\\
|\||else|\\
|\providecommand{\version}{final}|\\
|\||fi|
\end{tabular}
\end{center}
%
The definition by |\providecommand| makes sure
that previous definitions are not overwritten.
Further statements |\providecommand{\version}{...}|
can thus be added before the above code to override it.

For the main file, one might add a line
(between |\childdocmain| and the above block)
%
\begin{center}
|%\ifchilddoc\||else\providecommand{\version}{draft}\||fi|
\end{center}
%
which can be uncommented to produce a draft version.
Likewise one can add a line to the very top of a child file
(above the |\childdocof{|\textit{main}|}| directive)
%
\begin{center}
|%\providecommand{\version}{final}|
\end{center}
%
which can be uncommented to produce the final version of this child document.

%%%%%%%%%%%%%%%%%%%%%%%%%%%%%%%%%%%%%%%%%%%%%%%%%%%%%%%%%%%%%%%%%%%%%%%%%%%%%%%%
\subsection{Forwarding}
\label{sec:forward}

Different versions of the main or child documents
using compilation flags as described in \secref{sec:flags}
can be (permanently) stored in different files
for convenient compilation, viewing and distribution.
To this end, the package defines a command
to pass on compilation to a different file:

%%%%%%%%%%%%%%%%%%%%%%%%%%%%%%%%%%%%%%%%
\DescribeMacro{\childdocforward}
The command |\childdocforward| redirects processing to
another source file:
%
\begin{center}
\begin{tabular}{l}
|% \iffalse
%
% childdoc.dtx Copyright (C) 2017-2018 Niklas Beisert
%
% This work may be distributed and/or modified under the
% conditions of the LaTeX Project Public License, either version 1.3
% of this license or (at your option) any later version.
% The latest version of this license is in
%   http://www.latex-project.org/lppl.txt
% and version 1.3 or later is part of all distributions of LaTeX
% version 2005/12/01 or later.
%
% This work has the LPPL maintenance status `maintained'.
%
% The Current Maintainer of this work is Niklas Beisert.
%
% This work consists of the files childdoc.dtx and childdoc.ins
% and the derived files childdoc.def and cdocsamp.tex with
% cdocsch1.tex, cdocsch2.tex, cdocsdrf.tex, cdocsfn1.tex, cdocsfn2.tex.
%
%<package>\ifdefined\childdocmain\endinput\fi
%<package>\ProvidesFile{childdoc.def}[2018/12/30 v2.0 child document driver]
%<samplemain>\ProvidesFile{cdocsamp.tex}[2018/12/30 v2.0 sample for childdoc]
%<*driver>
%\ProvidesFile{childdoc.drv}[2018/12/30 v2.0 childdoc reference manual file]
\PassOptionsToClass{10pt,a4paper}{article}
\documentclass{ltxdoc}

\usepackage[margin=35mm]{geometry}
\usepackage{hyperref}
\usepackage{hyperxmp}
\usepackage[usenames]{color}

\hypersetup{colorlinks=true}
\hypersetup{pdfstartview=FitH}
\hypersetup{pdfpagemode=UseNone}
\hypersetup{pdfsource={}}
\hypersetup{pdflang={en-UK}}
\hypersetup{pdfcopyright={Copyright 2017-2018 Niklas Beisert.
  This work may be distributed and/or modified under the
  conditions of the LaTeX Project Public License, either version 1.3
  of this license or (at your option) any later version.}}
\hypersetup{pdflicenseurl={http://www.latex-project.org/lppl.txt}}
\hypersetup{pdfcontactaddress={ETH Zurich, ITP, HIT K,
  Wolfgang-Pauli-Strasse 27}}
\hypersetup{pdfcontactpostcode={8093}}
\hypersetup{pdfcontactcity={Zurich}}
\hypersetup{pdfcontactcountry={Switzerland}}
\hypersetup{pdfcontactemail={nbeisert@itp.phys.ethz.ch}}
\hypersetup{pdfcontacturl={http://people.phys.ethz.ch/\xmptilde nbeisert/}}

\newcommand{\secref}[1]{\hyperref[#1]{section \ref*{#1}}}

\parskip1ex
\parindent0pt
\let\olditemize\itemize
\def\itemize{\olditemize\parskip0pt}

\begin{document}

\title{The \textsf{childdoc} Package}
\hypersetup{pdftitle={The childdoc Package}}
\author{Niklas Beisert\\[2ex]
  Institut f\"ur Theoretische Physik\\
  Eidgen\"ossische Technische Hochschule Z\"urich\\
  Wolfgang-Pauli-Strasse 27, 8093 Z\"urich, Switzerland\\[1ex]
  \href{mailto:nbeisert@itp.phys.ethz.ch}
  {\texttt{nbeisert@itp.phys.ethz.ch}}}
\hypersetup{pdfauthor={Niklas Beisert}}
\hypersetup{pdfsubject={Manual for the LaTeX2e Package childdoc}}
\date{30 December 2018, \textsf{v2.0}}
\maketitle

\begin{abstract}\noindent
\textsf{childdoc} is a \LaTeXe{} package
that enables the direct compilation
of document sections included by |\include|
to individual files.
\end{abstract}

\begingroup
\parskip0ex
\tableofcontents
\endgroup

%%%%%%%%%%%%%%%%%%%%%%%%%%%%%%%%%%%%%%%%%%%%%%%%%%%%%%%%%%%%%%%%%%%%%%%%%%%%%%%%
%%%%%%%%%%%%%%%%%%%%%%%%%%%%%%%%%%%%%%%%%%%%%%%%%%%%%%%%%%%%%%%%%%%%%%%%%%%%%%%%
\section{Introduction}

\LaTeX{} provides a mechanism to structure a large document (such as a book)
into a main file and several child files (containing the chapters)
using the |\include| command.
This mechanism is beneficial for documents
which span hundreds of pages in order to
make the source file(s) more manageable.
Moreover, compilation can be restricted to
selected child files by means of the |\includeonly| command.
The latter feature can be used to reduce the compilation time while editing
(this was significantly more useful in the earlier days of \LaTeX{})
or to generate a smaller document which is easier to navigate.
Another application of |\includeonly| is to generate
documents consisting of selected parts of the complete document.

However, there are a few drawbacks of the plain |\include| mechanism:
\begin{itemize}
\item
The child files cannot be compiled on their own,
they can only be compiled via the main file.
A naive editing environment
(such as a text editor with an option
to have the current file processed by \LaTeX)
may require one to switch to the main file before compiling;
attempting to compile the child file produces errors.
\item
The main file must be modified (each time)
to adjust the |\includeonly| command
to the present needs. This easily leaves the main file in a messy state.
\item
The generated document will always carry the filename
of the main document. This is inconvenient if
several child files are to be compiled and
to be kept for distribution.
\end{itemize}

The present package provides a simple interface
to make child files individually compilable by \LaTeX{}.
Compiling a child file then has the same effect as compiling
the main file with an |\includeonly| command
to select the appropriate child.
Moreover the generated document will carry the name of the child
rather than the main file.
This resolves all three above issues.

This feature is meant to make the editing of books,
thesis documents and lecture notes somewhat more convenient.
However, the package can also be used efficiently for
composing a series of documents (such as exercise sheets)
which are typically distributed individually.
It then assists the author in generating the individual documents
(potentially in different versions)
as well as a document containing the collected series.
Another application is in developing style files
or other kinds of included material
where compilation of the style file could redirect
to a sample or test file.

%%%%%%%%%%%%%%%%%%%%%%%%%%%%%%%%%%%%%%%%%%%%%%%%%%%%%%%%%%%%%%%%%%%%%%%%%%%%%%%%
%%%%%%%%%%%%%%%%%%%%%%%%%%%%%%%%%%%%%%%%%%%%%%%%%%%%%%%%%%%%%%%%%%%%%%%%%%%%%%%%
\section{Usage}

First of all, the package \textsf{childdoc} is \emph{not} a standard
\LaTeXe{} |.sty| style file! Therefore it needs to be invoked in
a non-standard way.

%%%%%%%%%%%%%%%%%%%%%%%%%%%%%%%%%%%%%%%%%%%%%%%%%%%%%%%%%%%%%%%%%%%%%%%%%%%%%%%%
\subsection{Included Files}
\label{sec:include}

%%%%%%%%%%%%%%%%%%%%%%%%%%%%%%%%%%%%%%%%
\DescribeMacro{\childdocmain}
To use the package, add the commands
\begin{center}
\begin{tabular}{l}
|\input{childdoc.def}|\\
|\childdocmain{}|\\
\end{tabular}
\end{center}
at the very top of the main \LaTeX{} file,
in particular \emph{before} the |\documentclass| statement!
The argument of |\childdocmain| should be left empty
(but it must be present).

%%%%%%%%%%%%%%%%%%%%%%%%%%%%%%%%%%%%%%%%
\DescribeMacro{\childdocof}
Furthermore, add the commands
\begin{center}
\begin{tabular}{l}
|\input{childdoc.def}|\\
|\childdocof{|\textit{main}|}|\\
\end{tabular}
\end{center}
at the top of every child file \textit{child}
which is included by |\include{|\textit{child}|}|
from within the main file
(or at least for those files to be compiled individually).
The argument \textit{main} must be the filename of the main file.

There are a couple of
considerations in setting up the main and child documents:

%%%%%%%%%%%%%%%%%%%%%%%%%%%%%%%%%%%%%%%%
\paragraph{Restrictions.}

Please note the following restrictions:
\begin{itemize}
\item
|\childdocmain| must be called with one argument \textit{main}
to ensure compatibility with earlier version of the package.
It must either be empty (|\childdocmain{}|)
or precisely match the filename of the main file in which it is specified.
See \secref{sec:detection} for further information.
\item
The filename \textit{main} must be specified without the |.tex| extension.
\item
The filename \textit{main} is case sensitive
(even in case-insensitive file systems)
due to internal string comparison.
\item
The argument \textit{main} should be fully expanded, it cannot be a macro.
\item
Subdirectories and special characters should be avoided in filenames.
\item
The command |\childdocmain{|\textit{main}|}| must be followed by a whitespace.
It should not be followed immediately by another command
or by a comment mark `|%|'.
This is because the \TeX{} parser reads the token immediately following
the argument of |\childdocmain| and puts it
at the beginning of every child section;
however, a white\-space is ignored.
\end{itemize}

%%%%%%%%%%%%%%%%%%%%%%%%%%%%%%%%%%%%%%%%
\paragraph{Content of Main File.}

It is advisable to place all content in the child files included by |\include|.
Any output contained in the main file will appear in all child documents
unless suppressed manually;
it cannot be suppressed automatically by the |\includeonly| directive
and thus should normally be avoided.
A method to include some content in the main file
by means of conditional processing is described in \secref{sec:conditional}.

%%%%%%%%%%%%%%%%%%%%%%%%%%%%%%%%%%%%%%%%
\paragraph{Page Numbering.}

When only a part of the document is compiled,
the appropriate numbering of pages
(as well as other status parameters)
is determined from the |.aux| files.
The latter contain information from previous passes.
However this information needs to propagate through
all intermediate child documents.
Therefore the page numbering in child documents may well
be inconsistent until the complete document is compiled at least once.

A useful (if unconventional) way to always ensure a consistent
page numbering is to restart the numbering in each child document
and denote the pages by `\textit{child}|.|\textit{page}'
where \textit{child} represents the chapter/section number of the child file.
This can be achieved by the command
|\numberwithin{page}{|\textit{child}|}|
of the \textsf{amsmath} package
where \textit{child} can be |chapter| or |section|
depending on the chosen structuring.
Alternatively, one can modify the macro |\thepage| appropriately
and reset the counter |page| at the start of each child file.

%%%%%%%%%%%%%%%%%%%%%%%%%%%%%%%%%%%%%%%%%%%%%%%%%%%%%%%%%%%%%%%%%%%%%%%%%%%%%%%%
\subsection{Conditional Processing}
\label{sec:conditional}

The package provides a mechanism to compile different versions
of a document. To customise the versions further some conditional processing
can come in handy to distinguish which version is being compiled.
The package provides two macros to describe the compilation context:

%%%%%%%%%%%%%%%%%%%%%%%%%%%%%%%%%%%%%%%%
\DescribeMacro{\ifchilddoc}
The conditional |\ifchilddoc| distinguishes between the compilation of
child documents and the main document:
%
\begin{center}
|\ifchilddoc |\textit{child-code}| |[|\||else |\textit{main-code}]| \||fi|
\end{center}

%%%%%%%%%%%%%%%%%%%%%%%%%%%%%%%%%%%%%%%%
\DescribeMacro{\childdocname}
\DescribeMacro{\childdocjob}
The macro |\childdocname| contains the filename (without extension)
of the main or child file being processed.
Note that |\childdocjob| will always contain the name of the main file.

%%%%%%%%%%%%%%%%%%%%%%%%%%%%%%%%%%%%%%%%
\paragraph{Title Page.}

Conditional processing can be used to include a title or banner page
in the main document when proper precautions are taken.
Importantly, the code in the main file should ensure that the page counter
(as well as other status parameters which are stored in the |.aux| files)
takes the same value after the conditional processing.
Otherwise the page numbers may take divergent values
depending on which part is compiled.

For example, a title page could be declared by:
%
\begin{center}
\begin{tabular}{l}
|\ifchilddoc\||else|\\
|\addtocounter{page}{-1}|\\
\textit{code for title page}\\
|\newpage|\\
|\||fi|
\end{tabular}
\end{center}
%
A banner page for the child documents can be generated by:
%
\begin{center}
\begin{tabular}{l}
|\ifchilddoc|\\
|\addtocounter{page}{-1}|\\
\textit{code for banner page}\\
|\newpage|\\
|\||fi|
\end{tabular}
\end{center}
%
Here one could write a message such as:
\begin{center}
|This is the part \childdocname{} of \childdocjob{}.|
\end{center}

%%%%%%%%%%%%%%%%%%%%%%%%%%%%%%%%%%%%%%%%%%%%%%%%%%%%%%%%%%%%%%%%%%%%%%%%%%%%%%%%
\subsection{Flags}
\label{sec:flags}

The package makes it easy to generate different versions
of the main or child documents.
To this end compilation flags can be defined
and assigned different default values.
They will be particularly useful in conjunction
with the forwarding mechanism described in \secref{sec:forward}.

For example, it may be useful to have a flag |\version|
which can be set to |draft| or |final|.
The document source will contain some conditional code
depending on the value of |\version|.
Suppose further, the flag should default to |final| for the main file
and to |draft| for child files
which is a natural assignment for editing the document.
This is achieved by placing the following code
in the preamble of the main document
(below the |\childdocmain| directive):
%
\begin{center}
\begin{tabular}{l}
|\ifchilddoc|\\
|\providecommand{\version}{draft}|\\
|\||else|\\
|\providecommand{\version}{final}|\\
|\||fi|
\end{tabular}
\end{center}
%
The definition by |\providecommand| makes sure
that previous definitions are not overwritten.
Further statements |\providecommand{\version}{...}|
can thus be added before the above code to override it.

For the main file, one might add a line
(between |\childdocmain| and the above block)
%
\begin{center}
|%\ifchilddoc\||else\providecommand{\version}{draft}\||fi|
\end{center}
%
which can be uncommented to produce a draft version.
Likewise one can add a line to the very top of a child file
(above the |\childdocof{|\textit{main}|}| directive)
%
\begin{center}
|%\providecommand{\version}{final}|
\end{center}
%
which can be uncommented to produce the final version of this child document.

%%%%%%%%%%%%%%%%%%%%%%%%%%%%%%%%%%%%%%%%%%%%%%%%%%%%%%%%%%%%%%%%%%%%%%%%%%%%%%%%
\subsection{Forwarding}
\label{sec:forward}

Different versions of the main or child documents
using compilation flags as described in \secref{sec:flags}
can be (permanently) stored in different files
for convenient compilation, viewing and distribution.
To this end, the package defines a command
to pass on compilation to a different file:

%%%%%%%%%%%%%%%%%%%%%%%%%%%%%%%%%%%%%%%%
\DescribeMacro{\childdocforward}
The command |\childdocforward| redirects processing to
another source file:
%
\begin{center}
\begin{tabular}{l}
|\input{childdoc.def}|\\
|\childdocforward[|\textit{main}|]{|\textit{dest}|}|\\
\end{tabular}
\end{center}
%
The argument \textit{dest} is the destination file
(without extension).
It should be the main file or one of the child files.
Note that further \textsf{childdoc} directives
such as |\childdocof| and |\childdocforward|
in the indicated file will be processed in this form.
The optional argument \textit{main}
passes on directly to the main file \textit{main}
while pretending to compile the child \textit{dest}.
This form behaves as if \textit{dest}
issues |\childdocof{|\textit{main}|}| right away,
and no further \textsf{childdoc} directives will be processed.

%%%%%%%%%%%%%%%%%%%%%%%%%%%%%%%%%%%%%%%%
\DescribeMacro{\...prefix}
In the alternative form |\childdocforwardprefix|,
%
\begin{center}
\begin{tabular}{l}
|\input{childdoc.def}|\\
|\childdocforwardprefix[|\textit{main}|]{|\textit{prefix}|}{|\textit{dest}|}|
\end{tabular}
\end{center}
%
the destination file is determined by a pattern
depending on the current file:
To make this work, the current file must be called
`{\textit{prefix}\hspace{0.2em}\textit{suffix}}'
with \textit{prefix} matching precisely the argument.
Processing is then passed on to the file
`{\textit{dest}\hspace{0.2em}\textit{suffix}}'.
Surely, the same effect is achieved by
directly specifying the
argument `{\textit{dest}\hspace{0.2em}\textit{suffix}}'
in the first form.
However, that requires to set up a different file
for each child. With the alternative form of the command
all these files can have exactly the same content
which simplifies setting them up and maintaining them.

For example, the following file |draft.tex|
with a compilation flag |\version| as described in \secref{sec:flags}
compiles the main document as a draft:
%
\begin{center}
\begin{tabular}{l}
|\def\version{draft}|\\
|\input{childdoc.def}|\\
|\childdocforward{|\textit{main}|}|
\end{tabular}
\end{center}
%
Likewise, the following files |final|\textit{nn}|.tex|
compile the final version of the child document
|child|\textit{nn}|.tex|:
%
\begin{center}
\begin{tabular}{l}
|\def\version{final}|\\
|\input{childdoc.def}|\\
|\childdocforwardprefix{final}{child}|
\end{tabular}
\end{center}
%

Note that when several versions of a main file and/or of each child file
are to be generated, it may be convenient to set up a |Makefile| or
shell script to automatise the process.

%%%%%%%%%%%%%%%%%%%%%%%%%%%%%%%%%%%%%%%%%%%%%%%%%%%%%%%%%%%%%%%%%%%%%%%%%%%%%%%%
\subsection{Command Line Processing}
\label{sec:commandline}

The effect of redirection files can also be achieved by invoking
the \LaTeX{} compiler with a more elaborate command line.
Most conveniently this should be done as part
of a shell script or a |Makefile|.

When using \textsf{childdoc} in the main file, the following
command lines effectively perform a redirection
(note that depending on the shell being used,
backslashes may have to be doubled: `|\|' $\to$ `|\\|'):
%
\begin{center}
|... -jobname "|\textit{target}|" |\\|"|[\textit{flags}]%
|\input{childdoc.def}\childdocforward[|\textit{main}|]{|\textit{dest}|}"|
\end{center}
%
Here \textit{target} is the name of the output file,
\textit{main} is the name of the main file
and \textit{dest} is the name of the main or child file to be processed
(all filenames without extensions).
The optional argument \textit{main} can be omitted
if \textit{main} matches \textit{dest}.
Optionally, compilation \textit{flags} can be defined via |\def| commands.
This command line makes the \TeX{} engine believe
it is compiling the file \textit{target}
whose content is specified as the latter parameter.
The provided code then forwards the processing to
\textit{main} or \textit{dest} as described in \secref{sec:forward}.

%%%%%%%%%%%%%%%%%%%%%%%%%%%%%%%%%%%%%%%%%%%%%%%%%%%%%%%%%%%%%%%%%%%%%%%%%%%%%%%%
\subsection{Include by Input}
\label{sec:input}

Including child documents by |\include| has some restrictions by design.
Most notably, the content of a child document always occupies
its own set of pages; pages cannot be shared between child documents.
Usually, this behaviour makes perfect sense
because each child document contain an essential part of the document.
However, in some situations it may be desirable to compose
a document from a collection of parts
without having mandatory page breaks between then.
For this case, the package
provides a mechanism to include parts
by |\input| which can also be processed individually.
However, by construction this mechanism
requires manual handling of the content to be output.

%%%%%%%%%%%%%%%%%%%%%%%%%%%%%%%%%%%%%%%%
\DescribeMacro{\ifchilddocmanual}
The main file should be prepared as usual, see \secref{sec:include}.
However, the document body must make a distinction
between processing of an individual part and of the main document, e.g.:
%
\begin{center}
\begin{tabular}{l}
|\ifchilddocmanual|\\
|\input{\childdocname}|\\
|\||else|\\
\textit{document body with }|\input{|\textit{part}|}|\\
|\||fi|
\end{tabular}
\end{center}
%
The conditional |\ifchilddocmanual| is true whenever
a part to be included by |\input| is being compiled,
and the name of the part is stored in |\childdocname|.

%%%%%%%%%%%%%%%%%%%%%%%%%%%%%%%%%%%%%%%%
\DescribeMacro{\childdocby}
Each part to be included by |\input| should start with:
%
\begin{center}
\begin{tabular}{l}
|\input{childdoc.def}|\\
|\childdocby{|\textit{main}|}|\\
\end{tabular}
\end{center}
%
The directive |\childdocby| is similar to |\childdocof|
described in \secref{sec:include},
but the subsequent selection of content must be done manually.
To that end, both |\ifchilddoc| and |\ifchilddocmanual|
will be true upon processing of a part,
and the name of the part is stored in |\childdocname|.
Note that |\jobname| will be set to the filename of the current part
so that each part receives an individual |.aux| file
that does not interfere with the |.aux| file(s) of the main document.
This behaviour can be altered by the alternative form
|\childdocby[*]{|\textit{main}|}| (with a non-empty optional argument)
which uses the |.aux| file of the main document
by setting |\jobname| to \textit{main}.

%%%%%%%%%%%%%%%%%%%%%%%%%%%%%%%%%%%%%%%%%%%%%%%%%%%%%%%%%%%%%%%%%%%%%%%%%%%%%%%%
\subsection{Driver Development}
\label{sec:driver}

The \textsf{childdoc} mechanism can also be use for the development
of definition files such as \LaTeX{} styles or classes.
This case differs from the above setup with multiple parts
included by |\include| in that no |\includeonly| should be invoked.
This can be achieved by starting the include file
(before |\ProvidesPackage|) with:
%
\begin{center}
\begin{tabular}{l}
|\input{childdoc.def}|\\
|\childdocforward{|\textit{main}|}|\\
\end{tabular}
\end{center}
%
or alternatively with:
%
\begin{center}
\begin{tabular}{l}
|\input{childdoc.def}|\\
|\childdocby{|\textit{main}|}|\\
\end{tabular}
\end{center}
%
Both forms have slightly different effects as described above.
The main file is prepared as usual, see \secref{sec:include}.

%%%%%%%%%%%%%%%%%%%%%%%%%%%%%%%%%%%%%%%%%%%%%%%%%%%%%%%%%%%%%%%%%%%%%%%%%%%%%%%%
\subsection{Legacy Detection}
\label{sec:detection}

The directive |\childdocmain| in the main file can detect
whether the complete document or merely a child is to be compiled
even without using the directive |\childdocof|.
This method is deprecated because it is less robust
and there is no compelling reason to use it;
it is merely provided for backward compatibility
and it may be removed in future versions.

If the detection mechanism is to be used,
it is mandatory to correctly specify
the filename of the main file as the argument of |\childdocmain|:
%
\begin{center}
\begin{tabular}{l}
|\input{childdoc.def}|\\
|\childdocmain{|\textit{main}|}|\\
\end{tabular}
\end{center}
%
If |\jobname| does not match the argument \textit{main} of |\childdocmain|,
it is assumed that |\jobname| points to the child file to be compiled.
When using |\childdocmain| with the main file specified as argument,
it suffices to start a child file
with just |\input{|\textit{main}|}|
without loading of the package and using |\childdocof|.
If instead all processing is done
with the appropriate \textsf{childdoc} directives,
the argument of \textit{main} of |\childdocmain| can be empty.

An alternative version of the command line processing described
in \secref{sec:commandline} using the detection mechanism reads:
%
\begin{center}
|... -jobname "|\textit{target}|" "|[\textit{flags}]%
[|\def\jobname{|\textit{dest}|}|]|\input{|\textit{main}|}"|
\end{center}

%%%%%%%%%%%%%%%%%%%%%%%%%%%%%%%%%%%%%%%%%%%%%%%%%%%%%%%%%%%%%%%%%%%%%%%%%%%%%%%%
\subsection{Manual Code}
\label{sec:manual}

In case one cannot be certain whether the definitions file |childdoc.def|
is installed on the target \TeX{} distribution
and one prefers not to ship it,
it is conceivable to paste a few relevant commands into the sources.

To that end, drop all statements |\input{childdoc.def}|
and perform the replacements as outlined below.
Instead of |\childdocmain{|\textit{main}|}| add the following code
to the top of the main file:
%
\begin{center}
\begin{tabular}{l}
|\||ifdefined\childdocname\endinput\||fi\newif\ifchilddoc|\\
|\edef\childdocname{\scantokens\expandafter{\jobname\noexpand}}|\\
|\def\childdocmain{|\textit{main}|}\||ifx\childdocmain\childdocname\||else|\\
|\childdoctrue\includeonly{\childdocname}\let\jobname\childdocmain\||fi|\\
\end{tabular}
\end{center}
%
Instead of |\childdocof{|\textit{main}|}| just include the main file
at the top of each child file:
%
\begin{center}
|\input{|\textit{main}|}|
\end{center}
%
A simple redirection |\childdocforward{|\textit{dest}|}| is achieved by:
%
\begin{center}
|\def\jobname{|\textit{dest}|}\input{\jobname}|
\end{center}
%
The redirection with prefix
|\childdocforwardprefix[|\textit{prefix}|]{|\textit{dest}|}|
is accomplished by:
%
\begin{center}
\begin{tabular}{l}
|{\edef\jobname{\scantokens\expandafter{\jobname\noexpand}}|\\
|\def\redirectjob |\textit{prefix}|#1~~~{\gdef\jobname{|\textit{dest}|#1}}|\\
|\expandafter\redirectjob\jobname~~~}\input{\jobname}|
\end{tabular}
\end{center}

In an alternative approach,
child documents can be compiled by a specific command line
without additional code or specific definitions:
%
\begin{center}
|... -jobname "|\textit{target}|" "|[\textit{flags}]%
|\includeonly{|\textit{dest}|}\input{|\textit{main}|}"|
\end{center}
%

%%%%%%%%%%%%%%%%%%%%%%%%%%%%%%%%%%%%%%%%%%%%%%%%%%%%%%%%%%%%%%%%%%%%%%%%%%%%%%%%
%%%%%%%%%%%%%%%%%%%%%%%%%%%%%%%%%%%%%%%%%%%%%%%%%%%%%%%%%%%%%%%%%%%%%%%%%%%%%%%%
\section{Information}

%%%%%%%%%%%%%%%%%%%%%%%%%%%%%%%%%%%%%%%%%%%%%%%%%%%%%%%%%%%%%%%%%%%%%%%%%%%%%%%%
\subsection{Copyright}

Copyright \copyright{} 2017--2018 Niklas Beisert

This work may be distributed and/or modified under the
conditions of the \LaTeX{} Project Public License, either version 1.3
of this license or (at your option) any later version.
The latest version of this license is in
  \url{http://www.latex-project.org/lppl.txt}
and version 1.3 or later is part of all distributions of \LaTeX{}
version 2005/12/01 or later.

This work has the LPPL maintenance status `maintained'.

The Current Maintainer of this work is Niklas Beisert.

This work consists of the files |README.txt|, |childdoc.ins| and |childdoc.dtx|
as well as the derived files |childdoc.def|, |cdocsamp.tex|
with |cdocsch1.tex|, |cdocsch2.tex|, |cdocspt3.tex|, |cdocspt4.tex|,
|cdocsdrf.tex|, |cdocsfn1.tex|, |cdocsfn2.tex|
as well as |childdoc.pdf|.

%%%%%%%%%%%%%%%%%%%%%%%%%%%%%%%%%%%%%%%%%%%%%%%%%%%%%%%%%%%%%%%%%%%%%%%%%%%%%%%%
\subsection{Files and Installation}

The package consists of the files:
%
\begin{center}
\begin{tabular}{ll}
    |README.txt|   & readme file \\
    |childdoc.ins| & installation file \\
    |childdoc.dtx| & source file \\
    |childdoc.def| & definition file \\
    |cdocsamp.tex| & sample main file \\
    |cdocsch1.tex| & sample include file \\
    |cdocsch2.tex| & sample include file \\
    |cdocspt3.tex| & sample part file \\
    |cdocspt4.tex| & sample part file \\
    |cdocsdrf.tex| & sample redirection file \\
    |cdocsfn1.tex| & sample redirection file \\
    |cdocsfn2.tex| & sample redirection file \\
    |childdoc.pdf| & manual
\end{tabular}
\end{center}
%
The distribution consists of the files
|README.txt|, |childdoc.ins| and |childdoc.dtx|.
%
\begin{itemize}
\item
Run (pdf)\LaTeX{} on |childdoc.dtx|
to compile the manual |childdoc.pdf| (this file).
\item
Run \LaTeX{} on |childdoc.ins| to create the definitions file |childdoc.def|
and the sample |cdocsamp.tex| with include files
|cdocsch1.tex|, |cdocsch2.tex|, |cdocspt3.tex|, |cdocspt4.tex|,
|cdocsdrf.tex|, |cdocsfn1.tex|, |cdocsfn2.tex|.
Then copy the file |childdoc.def| to an appropriate directory of your \LaTeX{}
distribution, e.g.\ \textit{texmf-root}|/tex/latex/childdoc|.
\end{itemize}

%%%%%%%%%%%%%%%%%%%%%%%%%%%%%%%%%%%%%%%%%%%%%%%%%%%%%%%%%%%%%%%%%%%%%%%%%%%%%%%%
\subsection{Related CTAN Packages}

There are several other packages which offer a similar functionality:
%
\begin{itemize}
\item
The packages
\href{http://ctan.org/pkg/docmute}{\textsf{docmute}},
\href{http://ctan.org/pkg/includex}{\textsf{includex}} and
\href{http://ctan.org/pkg/standalone}{\textsf{standalone}}
provide commands to include only the document body of
a child file thus allowing both files to be compiled individually.
\item
The packages \href{http://ctan.org/pkg/subdocs}{\textsf{subdocs}}
and \href{http://ctan.org/pkg/subfiles}{\textsf{subfiles}}
provide structures in which the main and child documents can be
encapsulated and allowing them to be compiled individually.
The inclusion mechanism is different from the conventional |\include|.
\item
The package \href{http://ctan.org/pkg/combine}{\textsf{combine}}
is an elaborate solution to combine several documents into one.
\end{itemize}
%
See also the CTAN topic \href{http://ctan.org/topic/subdocs}{\textsf{subdocs}}
for further related packages.
The present package differs from the above solutions in that
a document structure constructed with the conventional |\include| mechanism
just needs two extra commands at the top of every file
such that all constituent files can be compiled individually.

%%%%%%%%%%%%%%%%%%%%%%%%%%%%%%%%%%%%%%%%%%%%%%%%%%%%%%%%%%%%%%%%%%%%%%%%%%%%%%%%
%\subsection{Feature Suggestions}
%
%The following is a list of features which may be useful for future
%versions of this package:
%%
%\begin{itemize}
%\item
%\ldots
%\end{itemize}

%%%%%%%%%%%%%%%%%%%%%%%%%%%%%%%%%%%%%%%%%%%%%%%%%%%%%%%%%%%%%%%%%%%%%%%%%%%%%%%%
\subsection{Revision History}

%%%%%%%%%%%%%%%%%%%%%%%%%%%%%%%%%%%%%%%%
\paragraph{v2.0:} 2018/12/30

\begin{itemize}
\item
immediate forward processing
\item
added |\childdocby| mechanism
\item
manual restructured
\end{itemize}

%%%%%%%%%%%%%%%%%%%%%%%%%%%%%%%%%%%%%%%%
\paragraph{v1.6:} 2018/01/17

\begin{itemize}
\item
application for development of include files
\item
corrections to manual
\end{itemize}

%%%%%%%%%%%%%%%%%%%%%%%%%%%%%%%%%%%%%%%%
\paragraph{v1.5:} 2017/05/21

\begin{itemize}
\item
more complete structuring introduced
\item
|\childdocof| introduced
\item
|\childdoc| renamed to |\childdocmain|
\item
|\childredirect| renamed to |\childdocforward| and |\childdocforwardprefix|
and functionality expanded
\end{itemize}

%%%%%%%%%%%%%%%%%%%%%%%%%%%%%%%%%%%%%%%%
\paragraph{v1.0:} 2017/04/27

\begin{itemize}
\item
manual and install package
\item
first version published on CTAN
\end{itemize}

%%%%%%%%%%%%%%%%%%%%%%%%%%%%%%%%%%%%%%%%
\paragraph{v0.6:} 2017/04/26

\begin{itemize}
\item
redirection mechanism added
\end{itemize}

%%%%%%%%%%%%%%%%%%%%%%%%%%%%%%%%%%%%%%%%
\paragraph{v0.5:} 2017/04/26

\begin{itemize}
\item
functionality in definition file
\end{itemize}


%%%%%%%%%%%%%%%%%%%%%%%%%%%%%%%%%%%%%%%%%%%%%%%%%%%%%%%%%%%%%%%%%%%%%%%%%%%%%%%%
%%%%%%%%%%%%%%%%%%%%%%%%%%%%%%%%%%%%%%%%%%%%%%%%%%%%%%%%%%%%%%%%%%%%%%%%%%%%%%%%
%%%%%%%%%%%%%%%%%%%%%%%%%%%%%%%%%%%%%%%%%%%%%%%%%%%%%%%%%%%%%%%%%%%%%%%%%%%%%%%%
\appendix

\settowidth\MacroIndent{\rmfamily\scriptsize 000\ }

 \DocInput{childdoc.dtx}

\end{document}
%</driver>
% \fi
%
% %%%%%%%%%%%%%%%%%%%%%%%%%%%%%%%%%%%%%%%%%%%%%%%%%%%%%%%%%%%%%%%%%%%%%%%%%%%%%%
% %%%%%%%%%%%%%%%%%%%%%%%%%%%%%%%%%%%%%%%%%%%%%%%%%%%%%%%%%%%%%%%%%%%%%%%%%%%%%%
% \section{Sample}
%\iffalse
%<*samplemain>
%\fi
%
% The following presents a sample document
% with two chapters, two parts, a title page,
% a compile flag as well as three forwarding files to set the flag.
% It consists of eight |.tex| files:
% \begin{center}
% \begin{tabular}{ll}
% |cdocsamp.tex|&main file\\
% |cdocsch1.tex|&include file for chapter 1\\
% |cdocsch2.tex|&include file for chapter 2\\
% |cdocspt3.tex|&include file for part 3\\
% |cdocspt4.tex|&include file for part 4\\
% |cdocsdrf.tex|&forwarding file for main file in draft mode\\
% |cdocsfi1.tex|&forwarding file for final version of chapter 1\\
% |cdocsfi2.tex|&forwarding file for final version of chapter 2\\
% \end{tabular}
% \end{center}
% Each of the eight files can be compiled directly by the \LaTeX{} compiler.
%
% %%%%%%%%%%%%%%%%%%%%%%%%%%%%%%%%%%%%%%
% \paragraph{Main File.}
%
% The main file is called |cdocsamp.tex|.
%
% Load the \textsf{childdoc} definitions and
% declare the filename for the main document:
%    \begin{macrocode}
\input{childdoc.def}
\childdocmain{}
%    \end{macrocode}

% Optional override for |\version| flag:
%    \begin{macrocode}
%%\ifchilddoc\else\providecommand{\version}{draft}\fi
%    \end{macrocode}

% Define the default values for the |\version| flag
% (|final| for the main file and |draft| for childs):
%    \begin{macrocode}
\ifchilddoc
\providecommand{\version}{draft}
\else
\providecommand{\version}{final}
\fi
%    \end{macrocode}

% Load the standard document class:
%    \begin{macrocode}
\documentclass[12pt]{article}
%    \end{macrocode}

% Start the document body:
%    \begin{macrocode}
\begin{document}
%    \end{macrocode}

% Declare a title page.
% Print title, part of document being processed and version flag:
%    \begin{macrocode}
\addtocounter{page}{-1}
\begin{center}
{\LARGE\bfseries{}childdoc example\par}
\vspace{1cm}
\ifchilddoc
\ifchilddocmanual part\else chapter\fi:
`\childdocname' of `\childdocjob'\par
\else
main document: `\childdocjob'\par
\fi
version: \version\par
\end{center}
\newpage
%    \end{macrocode}

% Manually include selected file,
% otherwise process as usual:
%    \begin{macrocode}
\ifchilddocmanual
\section*{part `\childdocname'}
\input{\childdocname}
\else
%    \end{macrocode}

% Include the two chapters:
%    \begin{macrocode}
\include{cdocsch1}
\include{cdocsch2}
%    \end{macrocode}

% Include the two parts unless only chapters should be displayed:
%    \begin{macrocode}
\ifchilddoc\else
\section{part three}
\input{cdocspt3}
\section{part four}
\input{cdocspt4}
\fi
%    \end{macrocode}

% Process as usual until here:
%    \begin{macrocode}
\fi
%    \end{macrocode}

% End of document body:
%    \begin{macrocode}
\end{document}
%    \end{macrocode}
%\iffalse
%</samplemain>
%\fi
%
% %%%%%%%%%%%%%%%%%%%%%%%%%%%%%%%%%%%%%%
% \paragraph{Chapter Include Files.}
%
% The include files are called |cdocsch1.tex| and |cdocsch2.tex|.
%
%\iffalse
%<*samplechap1|samplechap2>
%\fi

% Optional override for |\version| flag:
%    \begin{macrocode}
%%\providecommand{\version}{final}
%    \end{macrocode}

% Include the main document:
%    \begin{macrocode}
\input{childdoc.def}
\childdocof{cdocsamp}
%    \end{macrocode}

%\iffalse
%</samplechap1|samplechap2>
%\fi
%
%\iffalse
%<*samplechap1>
%\fi
% Some text for chapter 1:
%    \begin{macrocode}
\section{one}
some text in chapter one
%    \end{macrocode}

%\iffalse
%</samplechap1>
%\fi
% Some text for chapter 2:
%\iffalse
%<*samplechap2>
%\fi
%    \begin{macrocode}
\section{two}
more text in chapter two
%    \end{macrocode}

%\iffalse
%</samplechap2>
%\fi
%
% %%%%%%%%%%%%%%%%%%%%%%%%%%%%%%%%%%%%%%
% \paragraph{Part Include Files.}
%
% The include files are called |cdocspt3.tex| and |cdocspt4.tex|.
%
%\iffalse
%<*samplepart3|samplepart4>
%\fi

% Optional override for |\version| flag:
%    \begin{macrocode}
%%\providecommand{\version}{final}
%    \end{macrocode}

% Include the main document:
%    \begin{macrocode}
\input{childdoc.def}
\childdocby{cdocsamp}
%    \end{macrocode}

%\iffalse
%</samplepart3|samplepart4>
%\fi
%
%\iffalse
%<*samplepart3>
%\fi
% Some text for part 3:
%    \begin{macrocode}
some text in part three
%    \end{macrocode}

%\iffalse
%</samplepart3>
%\fi
% Some text for part 4:
%\iffalse
%<*samplepart4>
%\fi
%    \begin{macrocode}
more text in part four
%    \end{macrocode}

%\iffalse
%</samplepart4>
%\fi
%
% %%%%%%%%%%%%%%%%%%%%%%%%%%%%%%%%%%%%%%
% \paragraph{Forwarding for a Complete Draft.}
%
% The following forwarding file |cdocsdrf.tex|
% compiles the main document in draft mode:
%\iffalse
%<*sampledraft>
%\fi
%    \begin{macrocode}
\def\version{draft}
\input{childdoc.def}
\childdocforward{cdocsamp}
%    \end{macrocode}

%\iffalse
%</sampledraft>
%\fi
%
% %%%%%%%%%%%%%%%%%%%%%%%%%%%%%%%%%%%%%%
% \paragraph{Forwarding for Final Version of the Chapters.}
%
% The following forwarding files |cdocsfn1.tex| and |cdocsfn2.tex|
% (with identical content)
% compile the final versions of the child documents
% |cdocsch1.tex| and |cdocsch2.tex|, respectively:
%\iffalse
%<*samplefinal>
%\fi
%    \begin{macrocode}
\def\version{final}
\input{childdoc.def}
\childdocforwardprefix[cdocsamp]{cdocsfn}{cdocsch}
%    \end{macrocode}

%\iffalse
%</samplefinal>
%\fi
%
% %%%%%%%%%%%%%%%%%%%%%%%%%%%%%%%%%%%%%%
% \paragraph{Command Line Processing.}
%
% The following three command lines generate the output files
% |cdocscld|, |cdocscl1| and |cdocscl2|
% which should be identical to
% |cdocsdrf|, |cdocsch1| and |cdocsfn2|, respectively:
% \begin{center}
% \begin{tabular}{l}
% |latex -jobname cdocscld \|\\
% |  "\def\version{draft}\input{childdoc.def}\childdocforward{cdocsamp}"|\\
% |latex -jobname cdocscl1 \|\\
% |  "\input{childdoc.def}\childdocforward[cdocsamp]{cdocsch1}"|\\
% |latex -jobname cdocscl2 \|\\
% |  "\def\version{final}\input{childdoc.def}\childdocforward{cdocsch2}"|
% \end{tabular}
% \end{center}
% Note that the trailing backslash on each first line
% merely continues the input to the second line
% (for convenient cut ant paste).
% Furthermore, the command |latex| can be replaced by any
% of its alternative versions such as |pdflatex|.
%
% %%%%%%%%%%%%%%%%%%%%%%%%%%%%%%%%%%%%%%%%%%%%%%%%%%%%%%%%%%%%%%%%%%%%%%%%%%%%%%
% %%%%%%%%%%%%%%%%%%%%%%%%%%%%%%%%%%%%%%%%%%%%%%%%%%%%%%%%%%%%%%%%%%%%%%%%%%%%%%
% \section{Implementation}
%\iffalse
%<*package>
%\fi
%
% This section describes the definitions file |childdoc.def|.

% The definitions cannot be loaded using |\usepackage| or |\RequirePackage|
% which has a mechanism to prevent loading a style file more than once.
% When loading the definitions by means of |\input|
% multiple instances have to be prevented manually:
%\iffalse
%This code needs to be before the `\ProvidesFile' directive
%which is defined at the beginning of this file.
%Therefore it is also placed there and commented out here.
%</package>
%<*discard>
%\fi
%    \begin{macrocode}
\ifdefined\childdocmain\endinput\fi
%    \end{macrocode}
%\iffalse
%</discard>
%<*package>
%\fi
%
% \macro{\ifchilddoc}
% \macro{\ifchilddocmanual}
% The conditional |\ifchilddoc| tells whether a
% child (true) or main (false) document is being compiled.
% The conditional |\ifchilddocmanual| tells whether
% the |\includeonly| mechanism is used (false) or
% the selection of child files must be performed manually (true).
% The definitions initialise to false:
%    \begin{macrocode}
\newif\ifchilddoc
\newif\ifchilddocmanual
%    \end{macrocode}

% \macro{\childdocname}
% \macro{\childdocjob}
% The macro |\childdocname| stores the name of the main document
% to be compiled. The macro |\childdocjob| stores the name of
% the document on which the \LaTeX{} compiler was originally invoked.
% The content of |\jobname| cannot be compared
% to filenames specified in the source due to different catcodes.
% The following code rescans |\jobname|, stores the result
% in |\childdocname| and saves a copy in |\childdocjob|:
%    \begin{macrocode}
\edef\childdocname{\scantokens\expandafter{\jobname\noexpand}}
\let\childdocjob\childdocname
%    \end{macrocode}

% \macro{\childdocdisable}
% The macro |\childdocdisable| prevents the main file
% from being processed more than once.
% At this stage, the main document command |\childdocmain|
% is assumed to be called once again where it should do nothing.
% Any subsequent call to it should prevent
% a secondary processing of the main document
% It overwrites the forwarding commands
% |\childdocof| and |\childdocforward|
% with empty macros to prevent further inclusions of the main document:
%    \begin{macrocode}
\newcommand{\childdocdisable}
{
  \renewcommand{\childdocmain}[1]{\renewcommand{\childdocmain}[1]{\endinput}}
  \renewcommand{\childdocof}[1]{}
  \renewcommand{\childdocby}[2][]{}
  \renewcommand{\childdocforward}[2][]{}
  \renewcommand{\childdocdisable}{}
}
%    \end{macrocode}

% \macro{\childdocmain}
% The macro |\childdocmain| is to be called at the top of the main file
% with nothing or the main filename (without extension) as argument.
% First, it breaks loops.
% If the argument is not empty and does not match |\childdocname|
% (which is set by the first inclusion of |childdoc.def|),
% |\ifchilddoc| is set to true, |\includeonly| is applied to the child file
% and |\jobname| is set to the main file
% (for proper handling of |.aux| files):
%    \begin{macrocode}
\newcommand{\childdocmain}[1]
{
  \childdocdisable\childdocmain{}
  \if?#1?\else
    \begingroup
      \def\childdoctmp{#1}
      \ifx\childdoctmp\childdocname
        \def\childdoctmp{}
      \else
        \def\childdoctmp
        {
          \childdoctrue
          \includeonly{\childdocname}
          \def\childdocjob{#1}
          \def\jobname{#1}
        }
      \fi
      \expandafter
    \endgroup
    \childdoctmp
  \fi
}
%    \end{macrocode}

% \macro{\childdocof}
% The command |\childdocof| redirects
% compilation to the main file |#1|.
%    \begin{macrocode}
\newcommand{\childdocof}[1]
{
  \childdocdisable
  \childdoctrue
  \includeonly{\childdocname}
  \def\jobname{#1}
  \def\childdocjob{#1}
  \input{#1}
}
%    \end{macrocode}

% \macro{\childdocby}
% The command |\childdocby| ....
%    \begin{macrocode}
\newcommand{\childdocby}[2][]
{
  \childdocdisable
  \childdoctrue
  \childdocmanualtrue
  \if?#1?\else
    \def\jobname{#2}
  \fi
  \def\childdocjob{#2}
  \input{#2}
  \endinput
}
%    \end{macrocode}

% \macro{\childdocforward}
% The command |\childdocforward| redirects
% compilation to the main file or
% (if the optional argument is given) a child file.
% Parameters are set as if the main file
% or a child file starting with |\childdocof| was compiled.
% Then compilation is handed over to the main file:
%    \begin{macrocode}
\newcommand{\childdocforward}[2][]
{
  \begingroup
    \if?#1?
      \def\childdoctmp
      {
        \def\childdocname{#2}
        \def\childdocjob{#2}
        \def\jobname{#2}
        \input{#2}
        \endinput
      }
    \else
      \def\childdoctmp
      {
        \childdocdisable
        \def\childdocname{#2}
        \childdoctrue
        \includeonly{#2}
        \def\childdocjob{#1}
        \def\jobname{#1}
        \input{#1}
        \endinput
      }
    \fi
    \expandafter
  \endgroup
  \childdoctmp
}
%    \end{macrocode}

% \macro{\childdocforwardprefix}
% The command |\childdocforwardprefix| redirects
% compilation to the main or a child file by means of a pattern.
% The prefix |#1| in the current filename is replaced by |#2|
% and the suffix of the current filename is kept
% (it is assumed that the filename does not contain the substring `|~~~|'
% which is used as a delimiter).
% Compilation is handed over to the new file by |\childdocforward|:
%    \begin{macrocode}
\newcommand{\childdocforwardprefix}[3][]
{
  \begingroup
    \def\childdocextract #2##1~~~{\def\childdoctmp{\childdocforward[#1]{#3##1}}}
    \expandafter\childdocextract\childdocname~~~
    \expandafter
  \endgroup
  \childdoctmp
}
%    \end{macrocode}

% \macro{\childdoc}
% The deprecated macro |\childdoc| is a legacy version of |\childdocmain|:
%    \begin{macrocode}
\newcommand{\childdoc}{\childdocmain}
%    \end{macrocode}

% \macro{\childdocredirect}
% The deprecated macro |\childdocredirect| is a legacy version
% of |\childdocforward| and |\childdocforwardprefix|:
%    \begin{macrocode}
\newcommand{\childdocredirect}[2][]
{
  \begingroup
    \if?#1?
      \def\childdoctmp{\childdocforward{#2}}
    \else
      \def\childdoctmp{\childdocforwardprefix{#1}{#2}}
    \fi
    \expandafter
  \endgroup
  \childdoctmp
}
%    \end{macrocode}

%\iffalse
%</package>
%\fi
%
\endinput
|\\
|\childdocforward[|\textit{main}|]{|\textit{dest}|}|\\
\end{tabular}
\end{center}
%
The argument \textit{dest} is the destination file
(without extension).
It should be the main file or one of the child files.
Note that further \textsf{childdoc} directives
such as |\childdocof| and |\childdocforward|
in the indicated file will be processed in this form.
The optional argument \textit{main}
passes on directly to the main file \textit{main}
while pretending to compile the child \textit{dest}.
This form behaves as if \textit{dest}
issues |\childdocof{|\textit{main}|}| right away,
and no further \textsf{childdoc} directives will be processed.

%%%%%%%%%%%%%%%%%%%%%%%%%%%%%%%%%%%%%%%%
\DescribeMacro{\...prefix}
In the alternative form |\childdocforwardprefix|,
%
\begin{center}
\begin{tabular}{l}
|% \iffalse
%
% childdoc.dtx Copyright (C) 2017-2018 Niklas Beisert
%
% This work may be distributed and/or modified under the
% conditions of the LaTeX Project Public License, either version 1.3
% of this license or (at your option) any later version.
% The latest version of this license is in
%   http://www.latex-project.org/lppl.txt
% and version 1.3 or later is part of all distributions of LaTeX
% version 2005/12/01 or later.
%
% This work has the LPPL maintenance status `maintained'.
%
% The Current Maintainer of this work is Niklas Beisert.
%
% This work consists of the files childdoc.dtx and childdoc.ins
% and the derived files childdoc.def and cdocsamp.tex with
% cdocsch1.tex, cdocsch2.tex, cdocsdrf.tex, cdocsfn1.tex, cdocsfn2.tex.
%
%<package>\ifdefined\childdocmain\endinput\fi
%<package>\ProvidesFile{childdoc.def}[2018/12/30 v2.0 child document driver]
%<samplemain>\ProvidesFile{cdocsamp.tex}[2018/12/30 v2.0 sample for childdoc]
%<*driver>
%\ProvidesFile{childdoc.drv}[2018/12/30 v2.0 childdoc reference manual file]
\PassOptionsToClass{10pt,a4paper}{article}
\documentclass{ltxdoc}

\usepackage[margin=35mm]{geometry}
\usepackage{hyperref}
\usepackage{hyperxmp}
\usepackage[usenames]{color}

\hypersetup{colorlinks=true}
\hypersetup{pdfstartview=FitH}
\hypersetup{pdfpagemode=UseNone}
\hypersetup{pdfsource={}}
\hypersetup{pdflang={en-UK}}
\hypersetup{pdfcopyright={Copyright 2017-2018 Niklas Beisert.
  This work may be distributed and/or modified under the
  conditions of the LaTeX Project Public License, either version 1.3
  of this license or (at your option) any later version.}}
\hypersetup{pdflicenseurl={http://www.latex-project.org/lppl.txt}}
\hypersetup{pdfcontactaddress={ETH Zurich, ITP, HIT K,
  Wolfgang-Pauli-Strasse 27}}
\hypersetup{pdfcontactpostcode={8093}}
\hypersetup{pdfcontactcity={Zurich}}
\hypersetup{pdfcontactcountry={Switzerland}}
\hypersetup{pdfcontactemail={nbeisert@itp.phys.ethz.ch}}
\hypersetup{pdfcontacturl={http://people.phys.ethz.ch/\xmptilde nbeisert/}}

\newcommand{\secref}[1]{\hyperref[#1]{section \ref*{#1}}}

\parskip1ex
\parindent0pt
\let\olditemize\itemize
\def\itemize{\olditemize\parskip0pt}

\begin{document}

\title{The \textsf{childdoc} Package}
\hypersetup{pdftitle={The childdoc Package}}
\author{Niklas Beisert\\[2ex]
  Institut f\"ur Theoretische Physik\\
  Eidgen\"ossische Technische Hochschule Z\"urich\\
  Wolfgang-Pauli-Strasse 27, 8093 Z\"urich, Switzerland\\[1ex]
  \href{mailto:nbeisert@itp.phys.ethz.ch}
  {\texttt{nbeisert@itp.phys.ethz.ch}}}
\hypersetup{pdfauthor={Niklas Beisert}}
\hypersetup{pdfsubject={Manual for the LaTeX2e Package childdoc}}
\date{30 December 2018, \textsf{v2.0}}
\maketitle

\begin{abstract}\noindent
\textsf{childdoc} is a \LaTeXe{} package
that enables the direct compilation
of document sections included by |\include|
to individual files.
\end{abstract}

\begingroup
\parskip0ex
\tableofcontents
\endgroup

%%%%%%%%%%%%%%%%%%%%%%%%%%%%%%%%%%%%%%%%%%%%%%%%%%%%%%%%%%%%%%%%%%%%%%%%%%%%%%%%
%%%%%%%%%%%%%%%%%%%%%%%%%%%%%%%%%%%%%%%%%%%%%%%%%%%%%%%%%%%%%%%%%%%%%%%%%%%%%%%%
\section{Introduction}

\LaTeX{} provides a mechanism to structure a large document (such as a book)
into a main file and several child files (containing the chapters)
using the |\include| command.
This mechanism is beneficial for documents
which span hundreds of pages in order to
make the source file(s) more manageable.
Moreover, compilation can be restricted to
selected child files by means of the |\includeonly| command.
The latter feature can be used to reduce the compilation time while editing
(this was significantly more useful in the earlier days of \LaTeX{})
or to generate a smaller document which is easier to navigate.
Another application of |\includeonly| is to generate
documents consisting of selected parts of the complete document.

However, there are a few drawbacks of the plain |\include| mechanism:
\begin{itemize}
\item
The child files cannot be compiled on their own,
they can only be compiled via the main file.
A naive editing environment
(such as a text editor with an option
to have the current file processed by \LaTeX)
may require one to switch to the main file before compiling;
attempting to compile the child file produces errors.
\item
The main file must be modified (each time)
to adjust the |\includeonly| command
to the present needs. This easily leaves the main file in a messy state.
\item
The generated document will always carry the filename
of the main document. This is inconvenient if
several child files are to be compiled and
to be kept for distribution.
\end{itemize}

The present package provides a simple interface
to make child files individually compilable by \LaTeX{}.
Compiling a child file then has the same effect as compiling
the main file with an |\includeonly| command
to select the appropriate child.
Moreover the generated document will carry the name of the child
rather than the main file.
This resolves all three above issues.

This feature is meant to make the editing of books,
thesis documents and lecture notes somewhat more convenient.
However, the package can also be used efficiently for
composing a series of documents (such as exercise sheets)
which are typically distributed individually.
It then assists the author in generating the individual documents
(potentially in different versions)
as well as a document containing the collected series.
Another application is in developing style files
or other kinds of included material
where compilation of the style file could redirect
to a sample or test file.

%%%%%%%%%%%%%%%%%%%%%%%%%%%%%%%%%%%%%%%%%%%%%%%%%%%%%%%%%%%%%%%%%%%%%%%%%%%%%%%%
%%%%%%%%%%%%%%%%%%%%%%%%%%%%%%%%%%%%%%%%%%%%%%%%%%%%%%%%%%%%%%%%%%%%%%%%%%%%%%%%
\section{Usage}

First of all, the package \textsf{childdoc} is \emph{not} a standard
\LaTeXe{} |.sty| style file! Therefore it needs to be invoked in
a non-standard way.

%%%%%%%%%%%%%%%%%%%%%%%%%%%%%%%%%%%%%%%%%%%%%%%%%%%%%%%%%%%%%%%%%%%%%%%%%%%%%%%%
\subsection{Included Files}
\label{sec:include}

%%%%%%%%%%%%%%%%%%%%%%%%%%%%%%%%%%%%%%%%
\DescribeMacro{\childdocmain}
To use the package, add the commands
\begin{center}
\begin{tabular}{l}
|\input{childdoc.def}|\\
|\childdocmain{}|\\
\end{tabular}
\end{center}
at the very top of the main \LaTeX{} file,
in particular \emph{before} the |\documentclass| statement!
The argument of |\childdocmain| should be left empty
(but it must be present).

%%%%%%%%%%%%%%%%%%%%%%%%%%%%%%%%%%%%%%%%
\DescribeMacro{\childdocof}
Furthermore, add the commands
\begin{center}
\begin{tabular}{l}
|\input{childdoc.def}|\\
|\childdocof{|\textit{main}|}|\\
\end{tabular}
\end{center}
at the top of every child file \textit{child}
which is included by |\include{|\textit{child}|}|
from within the main file
(or at least for those files to be compiled individually).
The argument \textit{main} must be the filename of the main file.

There are a couple of
considerations in setting up the main and child documents:

%%%%%%%%%%%%%%%%%%%%%%%%%%%%%%%%%%%%%%%%
\paragraph{Restrictions.}

Please note the following restrictions:
\begin{itemize}
\item
|\childdocmain| must be called with one argument \textit{main}
to ensure compatibility with earlier version of the package.
It must either be empty (|\childdocmain{}|)
or precisely match the filename of the main file in which it is specified.
See \secref{sec:detection} for further information.
\item
The filename \textit{main} must be specified without the |.tex| extension.
\item
The filename \textit{main} is case sensitive
(even in case-insensitive file systems)
due to internal string comparison.
\item
The argument \textit{main} should be fully expanded, it cannot be a macro.
\item
Subdirectories and special characters should be avoided in filenames.
\item
The command |\childdocmain{|\textit{main}|}| must be followed by a whitespace.
It should not be followed immediately by another command
or by a comment mark `|%|'.
This is because the \TeX{} parser reads the token immediately following
the argument of |\childdocmain| and puts it
at the beginning of every child section;
however, a white\-space is ignored.
\end{itemize}

%%%%%%%%%%%%%%%%%%%%%%%%%%%%%%%%%%%%%%%%
\paragraph{Content of Main File.}

It is advisable to place all content in the child files included by |\include|.
Any output contained in the main file will appear in all child documents
unless suppressed manually;
it cannot be suppressed automatically by the |\includeonly| directive
and thus should normally be avoided.
A method to include some content in the main file
by means of conditional processing is described in \secref{sec:conditional}.

%%%%%%%%%%%%%%%%%%%%%%%%%%%%%%%%%%%%%%%%
\paragraph{Page Numbering.}

When only a part of the document is compiled,
the appropriate numbering of pages
(as well as other status parameters)
is determined from the |.aux| files.
The latter contain information from previous passes.
However this information needs to propagate through
all intermediate child documents.
Therefore the page numbering in child documents may well
be inconsistent until the complete document is compiled at least once.

A useful (if unconventional) way to always ensure a consistent
page numbering is to restart the numbering in each child document
and denote the pages by `\textit{child}|.|\textit{page}'
where \textit{child} represents the chapter/section number of the child file.
This can be achieved by the command
|\numberwithin{page}{|\textit{child}|}|
of the \textsf{amsmath} package
where \textit{child} can be |chapter| or |section|
depending on the chosen structuring.
Alternatively, one can modify the macro |\thepage| appropriately
and reset the counter |page| at the start of each child file.

%%%%%%%%%%%%%%%%%%%%%%%%%%%%%%%%%%%%%%%%%%%%%%%%%%%%%%%%%%%%%%%%%%%%%%%%%%%%%%%%
\subsection{Conditional Processing}
\label{sec:conditional}

The package provides a mechanism to compile different versions
of a document. To customise the versions further some conditional processing
can come in handy to distinguish which version is being compiled.
The package provides two macros to describe the compilation context:

%%%%%%%%%%%%%%%%%%%%%%%%%%%%%%%%%%%%%%%%
\DescribeMacro{\ifchilddoc}
The conditional |\ifchilddoc| distinguishes between the compilation of
child documents and the main document:
%
\begin{center}
|\ifchilddoc |\textit{child-code}| |[|\||else |\textit{main-code}]| \||fi|
\end{center}

%%%%%%%%%%%%%%%%%%%%%%%%%%%%%%%%%%%%%%%%
\DescribeMacro{\childdocname}
\DescribeMacro{\childdocjob}
The macro |\childdocname| contains the filename (without extension)
of the main or child file being processed.
Note that |\childdocjob| will always contain the name of the main file.

%%%%%%%%%%%%%%%%%%%%%%%%%%%%%%%%%%%%%%%%
\paragraph{Title Page.}

Conditional processing can be used to include a title or banner page
in the main document when proper precautions are taken.
Importantly, the code in the main file should ensure that the page counter
(as well as other status parameters which are stored in the |.aux| files)
takes the same value after the conditional processing.
Otherwise the page numbers may take divergent values
depending on which part is compiled.

For example, a title page could be declared by:
%
\begin{center}
\begin{tabular}{l}
|\ifchilddoc\||else|\\
|\addtocounter{page}{-1}|\\
\textit{code for title page}\\
|\newpage|\\
|\||fi|
\end{tabular}
\end{center}
%
A banner page for the child documents can be generated by:
%
\begin{center}
\begin{tabular}{l}
|\ifchilddoc|\\
|\addtocounter{page}{-1}|\\
\textit{code for banner page}\\
|\newpage|\\
|\||fi|
\end{tabular}
\end{center}
%
Here one could write a message such as:
\begin{center}
|This is the part \childdocname{} of \childdocjob{}.|
\end{center}

%%%%%%%%%%%%%%%%%%%%%%%%%%%%%%%%%%%%%%%%%%%%%%%%%%%%%%%%%%%%%%%%%%%%%%%%%%%%%%%%
\subsection{Flags}
\label{sec:flags}

The package makes it easy to generate different versions
of the main or child documents.
To this end compilation flags can be defined
and assigned different default values.
They will be particularly useful in conjunction
with the forwarding mechanism described in \secref{sec:forward}.

For example, it may be useful to have a flag |\version|
which can be set to |draft| or |final|.
The document source will contain some conditional code
depending on the value of |\version|.
Suppose further, the flag should default to |final| for the main file
and to |draft| for child files
which is a natural assignment for editing the document.
This is achieved by placing the following code
in the preamble of the main document
(below the |\childdocmain| directive):
%
\begin{center}
\begin{tabular}{l}
|\ifchilddoc|\\
|\providecommand{\version}{draft}|\\
|\||else|\\
|\providecommand{\version}{final}|\\
|\||fi|
\end{tabular}
\end{center}
%
The definition by |\providecommand| makes sure
that previous definitions are not overwritten.
Further statements |\providecommand{\version}{...}|
can thus be added before the above code to override it.

For the main file, one might add a line
(between |\childdocmain| and the above block)
%
\begin{center}
|%\ifchilddoc\||else\providecommand{\version}{draft}\||fi|
\end{center}
%
which can be uncommented to produce a draft version.
Likewise one can add a line to the very top of a child file
(above the |\childdocof{|\textit{main}|}| directive)
%
\begin{center}
|%\providecommand{\version}{final}|
\end{center}
%
which can be uncommented to produce the final version of this child document.

%%%%%%%%%%%%%%%%%%%%%%%%%%%%%%%%%%%%%%%%%%%%%%%%%%%%%%%%%%%%%%%%%%%%%%%%%%%%%%%%
\subsection{Forwarding}
\label{sec:forward}

Different versions of the main or child documents
using compilation flags as described in \secref{sec:flags}
can be (permanently) stored in different files
for convenient compilation, viewing and distribution.
To this end, the package defines a command
to pass on compilation to a different file:

%%%%%%%%%%%%%%%%%%%%%%%%%%%%%%%%%%%%%%%%
\DescribeMacro{\childdocforward}
The command |\childdocforward| redirects processing to
another source file:
%
\begin{center}
\begin{tabular}{l}
|\input{childdoc.def}|\\
|\childdocforward[|\textit{main}|]{|\textit{dest}|}|\\
\end{tabular}
\end{center}
%
The argument \textit{dest} is the destination file
(without extension).
It should be the main file or one of the child files.
Note that further \textsf{childdoc} directives
such as |\childdocof| and |\childdocforward|
in the indicated file will be processed in this form.
The optional argument \textit{main}
passes on directly to the main file \textit{main}
while pretending to compile the child \textit{dest}.
This form behaves as if \textit{dest}
issues |\childdocof{|\textit{main}|}| right away,
and no further \textsf{childdoc} directives will be processed.

%%%%%%%%%%%%%%%%%%%%%%%%%%%%%%%%%%%%%%%%
\DescribeMacro{\...prefix}
In the alternative form |\childdocforwardprefix|,
%
\begin{center}
\begin{tabular}{l}
|\input{childdoc.def}|\\
|\childdocforwardprefix[|\textit{main}|]{|\textit{prefix}|}{|\textit{dest}|}|
\end{tabular}
\end{center}
%
the destination file is determined by a pattern
depending on the current file:
To make this work, the current file must be called
`{\textit{prefix}\hspace{0.2em}\textit{suffix}}'
with \textit{prefix} matching precisely the argument.
Processing is then passed on to the file
`{\textit{dest}\hspace{0.2em}\textit{suffix}}'.
Surely, the same effect is achieved by
directly specifying the
argument `{\textit{dest}\hspace{0.2em}\textit{suffix}}'
in the first form.
However, that requires to set up a different file
for each child. With the alternative form of the command
all these files can have exactly the same content
which simplifies setting them up and maintaining them.

For example, the following file |draft.tex|
with a compilation flag |\version| as described in \secref{sec:flags}
compiles the main document as a draft:
%
\begin{center}
\begin{tabular}{l}
|\def\version{draft}|\\
|\input{childdoc.def}|\\
|\childdocforward{|\textit{main}|}|
\end{tabular}
\end{center}
%
Likewise, the following files |final|\textit{nn}|.tex|
compile the final version of the child document
|child|\textit{nn}|.tex|:
%
\begin{center}
\begin{tabular}{l}
|\def\version{final}|\\
|\input{childdoc.def}|\\
|\childdocforwardprefix{final}{child}|
\end{tabular}
\end{center}
%

Note that when several versions of a main file and/or of each child file
are to be generated, it may be convenient to set up a |Makefile| or
shell script to automatise the process.

%%%%%%%%%%%%%%%%%%%%%%%%%%%%%%%%%%%%%%%%%%%%%%%%%%%%%%%%%%%%%%%%%%%%%%%%%%%%%%%%
\subsection{Command Line Processing}
\label{sec:commandline}

The effect of redirection files can also be achieved by invoking
the \LaTeX{} compiler with a more elaborate command line.
Most conveniently this should be done as part
of a shell script or a |Makefile|.

When using \textsf{childdoc} in the main file, the following
command lines effectively perform a redirection
(note that depending on the shell being used,
backslashes may have to be doubled: `|\|' $\to$ `|\\|'):
%
\begin{center}
|... -jobname "|\textit{target}|" |\\|"|[\textit{flags}]%
|\input{childdoc.def}\childdocforward[|\textit{main}|]{|\textit{dest}|}"|
\end{center}
%
Here \textit{target} is the name of the output file,
\textit{main} is the name of the main file
and \textit{dest} is the name of the main or child file to be processed
(all filenames without extensions).
The optional argument \textit{main} can be omitted
if \textit{main} matches \textit{dest}.
Optionally, compilation \textit{flags} can be defined via |\def| commands.
This command line makes the \TeX{} engine believe
it is compiling the file \textit{target}
whose content is specified as the latter parameter.
The provided code then forwards the processing to
\textit{main} or \textit{dest} as described in \secref{sec:forward}.

%%%%%%%%%%%%%%%%%%%%%%%%%%%%%%%%%%%%%%%%%%%%%%%%%%%%%%%%%%%%%%%%%%%%%%%%%%%%%%%%
\subsection{Include by Input}
\label{sec:input}

Including child documents by |\include| has some restrictions by design.
Most notably, the content of a child document always occupies
its own set of pages; pages cannot be shared between child documents.
Usually, this behaviour makes perfect sense
because each child document contain an essential part of the document.
However, in some situations it may be desirable to compose
a document from a collection of parts
without having mandatory page breaks between then.
For this case, the package
provides a mechanism to include parts
by |\input| which can also be processed individually.
However, by construction this mechanism
requires manual handling of the content to be output.

%%%%%%%%%%%%%%%%%%%%%%%%%%%%%%%%%%%%%%%%
\DescribeMacro{\ifchilddocmanual}
The main file should be prepared as usual, see \secref{sec:include}.
However, the document body must make a distinction
between processing of an individual part and of the main document, e.g.:
%
\begin{center}
\begin{tabular}{l}
|\ifchilddocmanual|\\
|\input{\childdocname}|\\
|\||else|\\
\textit{document body with }|\input{|\textit{part}|}|\\
|\||fi|
\end{tabular}
\end{center}
%
The conditional |\ifchilddocmanual| is true whenever
a part to be included by |\input| is being compiled,
and the name of the part is stored in |\childdocname|.

%%%%%%%%%%%%%%%%%%%%%%%%%%%%%%%%%%%%%%%%
\DescribeMacro{\childdocby}
Each part to be included by |\input| should start with:
%
\begin{center}
\begin{tabular}{l}
|\input{childdoc.def}|\\
|\childdocby{|\textit{main}|}|\\
\end{tabular}
\end{center}
%
The directive |\childdocby| is similar to |\childdocof|
described in \secref{sec:include},
but the subsequent selection of content must be done manually.
To that end, both |\ifchilddoc| and |\ifchilddocmanual|
will be true upon processing of a part,
and the name of the part is stored in |\childdocname|.
Note that |\jobname| will be set to the filename of the current part
so that each part receives an individual |.aux| file
that does not interfere with the |.aux| file(s) of the main document.
This behaviour can be altered by the alternative form
|\childdocby[*]{|\textit{main}|}| (with a non-empty optional argument)
which uses the |.aux| file of the main document
by setting |\jobname| to \textit{main}.

%%%%%%%%%%%%%%%%%%%%%%%%%%%%%%%%%%%%%%%%%%%%%%%%%%%%%%%%%%%%%%%%%%%%%%%%%%%%%%%%
\subsection{Driver Development}
\label{sec:driver}

The \textsf{childdoc} mechanism can also be use for the development
of definition files such as \LaTeX{} styles or classes.
This case differs from the above setup with multiple parts
included by |\include| in that no |\includeonly| should be invoked.
This can be achieved by starting the include file
(before |\ProvidesPackage|) with:
%
\begin{center}
\begin{tabular}{l}
|\input{childdoc.def}|\\
|\childdocforward{|\textit{main}|}|\\
\end{tabular}
\end{center}
%
or alternatively with:
%
\begin{center}
\begin{tabular}{l}
|\input{childdoc.def}|\\
|\childdocby{|\textit{main}|}|\\
\end{tabular}
\end{center}
%
Both forms have slightly different effects as described above.
The main file is prepared as usual, see \secref{sec:include}.

%%%%%%%%%%%%%%%%%%%%%%%%%%%%%%%%%%%%%%%%%%%%%%%%%%%%%%%%%%%%%%%%%%%%%%%%%%%%%%%%
\subsection{Legacy Detection}
\label{sec:detection}

The directive |\childdocmain| in the main file can detect
whether the complete document or merely a child is to be compiled
even without using the directive |\childdocof|.
This method is deprecated because it is less robust
and there is no compelling reason to use it;
it is merely provided for backward compatibility
and it may be removed in future versions.

If the detection mechanism is to be used,
it is mandatory to correctly specify
the filename of the main file as the argument of |\childdocmain|:
%
\begin{center}
\begin{tabular}{l}
|\input{childdoc.def}|\\
|\childdocmain{|\textit{main}|}|\\
\end{tabular}
\end{center}
%
If |\jobname| does not match the argument \textit{main} of |\childdocmain|,
it is assumed that |\jobname| points to the child file to be compiled.
When using |\childdocmain| with the main file specified as argument,
it suffices to start a child file
with just |\input{|\textit{main}|}|
without loading of the package and using |\childdocof|.
If instead all processing is done
with the appropriate \textsf{childdoc} directives,
the argument of \textit{main} of |\childdocmain| can be empty.

An alternative version of the command line processing described
in \secref{sec:commandline} using the detection mechanism reads:
%
\begin{center}
|... -jobname "|\textit{target}|" "|[\textit{flags}]%
[|\def\jobname{|\textit{dest}|}|]|\input{|\textit{main}|}"|
\end{center}

%%%%%%%%%%%%%%%%%%%%%%%%%%%%%%%%%%%%%%%%%%%%%%%%%%%%%%%%%%%%%%%%%%%%%%%%%%%%%%%%
\subsection{Manual Code}
\label{sec:manual}

In case one cannot be certain whether the definitions file |childdoc.def|
is installed on the target \TeX{} distribution
and one prefers not to ship it,
it is conceivable to paste a few relevant commands into the sources.

To that end, drop all statements |\input{childdoc.def}|
and perform the replacements as outlined below.
Instead of |\childdocmain{|\textit{main}|}| add the following code
to the top of the main file:
%
\begin{center}
\begin{tabular}{l}
|\||ifdefined\childdocname\endinput\||fi\newif\ifchilddoc|\\
|\edef\childdocname{\scantokens\expandafter{\jobname\noexpand}}|\\
|\def\childdocmain{|\textit{main}|}\||ifx\childdocmain\childdocname\||else|\\
|\childdoctrue\includeonly{\childdocname}\let\jobname\childdocmain\||fi|\\
\end{tabular}
\end{center}
%
Instead of |\childdocof{|\textit{main}|}| just include the main file
at the top of each child file:
%
\begin{center}
|\input{|\textit{main}|}|
\end{center}
%
A simple redirection |\childdocforward{|\textit{dest}|}| is achieved by:
%
\begin{center}
|\def\jobname{|\textit{dest}|}\input{\jobname}|
\end{center}
%
The redirection with prefix
|\childdocforwardprefix[|\textit{prefix}|]{|\textit{dest}|}|
is accomplished by:
%
\begin{center}
\begin{tabular}{l}
|{\edef\jobname{\scantokens\expandafter{\jobname\noexpand}}|\\
|\def\redirectjob |\textit{prefix}|#1~~~{\gdef\jobname{|\textit{dest}|#1}}|\\
|\expandafter\redirectjob\jobname~~~}\input{\jobname}|
\end{tabular}
\end{center}

In an alternative approach,
child documents can be compiled by a specific command line
without additional code or specific definitions:
%
\begin{center}
|... -jobname "|\textit{target}|" "|[\textit{flags}]%
|\includeonly{|\textit{dest}|}\input{|\textit{main}|}"|
\end{center}
%

%%%%%%%%%%%%%%%%%%%%%%%%%%%%%%%%%%%%%%%%%%%%%%%%%%%%%%%%%%%%%%%%%%%%%%%%%%%%%%%%
%%%%%%%%%%%%%%%%%%%%%%%%%%%%%%%%%%%%%%%%%%%%%%%%%%%%%%%%%%%%%%%%%%%%%%%%%%%%%%%%
\section{Information}

%%%%%%%%%%%%%%%%%%%%%%%%%%%%%%%%%%%%%%%%%%%%%%%%%%%%%%%%%%%%%%%%%%%%%%%%%%%%%%%%
\subsection{Copyright}

Copyright \copyright{} 2017--2018 Niklas Beisert

This work may be distributed and/or modified under the
conditions of the \LaTeX{} Project Public License, either version 1.3
of this license or (at your option) any later version.
The latest version of this license is in
  \url{http://www.latex-project.org/lppl.txt}
and version 1.3 or later is part of all distributions of \LaTeX{}
version 2005/12/01 or later.

This work has the LPPL maintenance status `maintained'.

The Current Maintainer of this work is Niklas Beisert.

This work consists of the files |README.txt|, |childdoc.ins| and |childdoc.dtx|
as well as the derived files |childdoc.def|, |cdocsamp.tex|
with |cdocsch1.tex|, |cdocsch2.tex|, |cdocspt3.tex|, |cdocspt4.tex|,
|cdocsdrf.tex|, |cdocsfn1.tex|, |cdocsfn2.tex|
as well as |childdoc.pdf|.

%%%%%%%%%%%%%%%%%%%%%%%%%%%%%%%%%%%%%%%%%%%%%%%%%%%%%%%%%%%%%%%%%%%%%%%%%%%%%%%%
\subsection{Files and Installation}

The package consists of the files:
%
\begin{center}
\begin{tabular}{ll}
    |README.txt|   & readme file \\
    |childdoc.ins| & installation file \\
    |childdoc.dtx| & source file \\
    |childdoc.def| & definition file \\
    |cdocsamp.tex| & sample main file \\
    |cdocsch1.tex| & sample include file \\
    |cdocsch2.tex| & sample include file \\
    |cdocspt3.tex| & sample part file \\
    |cdocspt4.tex| & sample part file \\
    |cdocsdrf.tex| & sample redirection file \\
    |cdocsfn1.tex| & sample redirection file \\
    |cdocsfn2.tex| & sample redirection file \\
    |childdoc.pdf| & manual
\end{tabular}
\end{center}
%
The distribution consists of the files
|README.txt|, |childdoc.ins| and |childdoc.dtx|.
%
\begin{itemize}
\item
Run (pdf)\LaTeX{} on |childdoc.dtx|
to compile the manual |childdoc.pdf| (this file).
\item
Run \LaTeX{} on |childdoc.ins| to create the definitions file |childdoc.def|
and the sample |cdocsamp.tex| with include files
|cdocsch1.tex|, |cdocsch2.tex|, |cdocspt3.tex|, |cdocspt4.tex|,
|cdocsdrf.tex|, |cdocsfn1.tex|, |cdocsfn2.tex|.
Then copy the file |childdoc.def| to an appropriate directory of your \LaTeX{}
distribution, e.g.\ \textit{texmf-root}|/tex/latex/childdoc|.
\end{itemize}

%%%%%%%%%%%%%%%%%%%%%%%%%%%%%%%%%%%%%%%%%%%%%%%%%%%%%%%%%%%%%%%%%%%%%%%%%%%%%%%%
\subsection{Related CTAN Packages}

There are several other packages which offer a similar functionality:
%
\begin{itemize}
\item
The packages
\href{http://ctan.org/pkg/docmute}{\textsf{docmute}},
\href{http://ctan.org/pkg/includex}{\textsf{includex}} and
\href{http://ctan.org/pkg/standalone}{\textsf{standalone}}
provide commands to include only the document body of
a child file thus allowing both files to be compiled individually.
\item
The packages \href{http://ctan.org/pkg/subdocs}{\textsf{subdocs}}
and \href{http://ctan.org/pkg/subfiles}{\textsf{subfiles}}
provide structures in which the main and child documents can be
encapsulated and allowing them to be compiled individually.
The inclusion mechanism is different from the conventional |\include|.
\item
The package \href{http://ctan.org/pkg/combine}{\textsf{combine}}
is an elaborate solution to combine several documents into one.
\end{itemize}
%
See also the CTAN topic \href{http://ctan.org/topic/subdocs}{\textsf{subdocs}}
for further related packages.
The present package differs from the above solutions in that
a document structure constructed with the conventional |\include| mechanism
just needs two extra commands at the top of every file
such that all constituent files can be compiled individually.

%%%%%%%%%%%%%%%%%%%%%%%%%%%%%%%%%%%%%%%%%%%%%%%%%%%%%%%%%%%%%%%%%%%%%%%%%%%%%%%%
%\subsection{Feature Suggestions}
%
%The following is a list of features which may be useful for future
%versions of this package:
%%
%\begin{itemize}
%\item
%\ldots
%\end{itemize}

%%%%%%%%%%%%%%%%%%%%%%%%%%%%%%%%%%%%%%%%%%%%%%%%%%%%%%%%%%%%%%%%%%%%%%%%%%%%%%%%
\subsection{Revision History}

%%%%%%%%%%%%%%%%%%%%%%%%%%%%%%%%%%%%%%%%
\paragraph{v2.0:} 2018/12/30

\begin{itemize}
\item
immediate forward processing
\item
added |\childdocby| mechanism
\item
manual restructured
\end{itemize}

%%%%%%%%%%%%%%%%%%%%%%%%%%%%%%%%%%%%%%%%
\paragraph{v1.6:} 2018/01/17

\begin{itemize}
\item
application for development of include files
\item
corrections to manual
\end{itemize}

%%%%%%%%%%%%%%%%%%%%%%%%%%%%%%%%%%%%%%%%
\paragraph{v1.5:} 2017/05/21

\begin{itemize}
\item
more complete structuring introduced
\item
|\childdocof| introduced
\item
|\childdoc| renamed to |\childdocmain|
\item
|\childredirect| renamed to |\childdocforward| and |\childdocforwardprefix|
and functionality expanded
\end{itemize}

%%%%%%%%%%%%%%%%%%%%%%%%%%%%%%%%%%%%%%%%
\paragraph{v1.0:} 2017/04/27

\begin{itemize}
\item
manual and install package
\item
first version published on CTAN
\end{itemize}

%%%%%%%%%%%%%%%%%%%%%%%%%%%%%%%%%%%%%%%%
\paragraph{v0.6:} 2017/04/26

\begin{itemize}
\item
redirection mechanism added
\end{itemize}

%%%%%%%%%%%%%%%%%%%%%%%%%%%%%%%%%%%%%%%%
\paragraph{v0.5:} 2017/04/26

\begin{itemize}
\item
functionality in definition file
\end{itemize}


%%%%%%%%%%%%%%%%%%%%%%%%%%%%%%%%%%%%%%%%%%%%%%%%%%%%%%%%%%%%%%%%%%%%%%%%%%%%%%%%
%%%%%%%%%%%%%%%%%%%%%%%%%%%%%%%%%%%%%%%%%%%%%%%%%%%%%%%%%%%%%%%%%%%%%%%%%%%%%%%%
%%%%%%%%%%%%%%%%%%%%%%%%%%%%%%%%%%%%%%%%%%%%%%%%%%%%%%%%%%%%%%%%%%%%%%%%%%%%%%%%
\appendix

\settowidth\MacroIndent{\rmfamily\scriptsize 000\ }

 \DocInput{childdoc.dtx}

\end{document}
%</driver>
% \fi
%
% %%%%%%%%%%%%%%%%%%%%%%%%%%%%%%%%%%%%%%%%%%%%%%%%%%%%%%%%%%%%%%%%%%%%%%%%%%%%%%
% %%%%%%%%%%%%%%%%%%%%%%%%%%%%%%%%%%%%%%%%%%%%%%%%%%%%%%%%%%%%%%%%%%%%%%%%%%%%%%
% \section{Sample}
%\iffalse
%<*samplemain>
%\fi
%
% The following presents a sample document
% with two chapters, two parts, a title page,
% a compile flag as well as three forwarding files to set the flag.
% It consists of eight |.tex| files:
% \begin{center}
% \begin{tabular}{ll}
% |cdocsamp.tex|&main file\\
% |cdocsch1.tex|&include file for chapter 1\\
% |cdocsch2.tex|&include file for chapter 2\\
% |cdocspt3.tex|&include file for part 3\\
% |cdocspt4.tex|&include file for part 4\\
% |cdocsdrf.tex|&forwarding file for main file in draft mode\\
% |cdocsfi1.tex|&forwarding file for final version of chapter 1\\
% |cdocsfi2.tex|&forwarding file for final version of chapter 2\\
% \end{tabular}
% \end{center}
% Each of the eight files can be compiled directly by the \LaTeX{} compiler.
%
% %%%%%%%%%%%%%%%%%%%%%%%%%%%%%%%%%%%%%%
% \paragraph{Main File.}
%
% The main file is called |cdocsamp.tex|.
%
% Load the \textsf{childdoc} definitions and
% declare the filename for the main document:
%    \begin{macrocode}
\input{childdoc.def}
\childdocmain{}
%    \end{macrocode}

% Optional override for |\version| flag:
%    \begin{macrocode}
%%\ifchilddoc\else\providecommand{\version}{draft}\fi
%    \end{macrocode}

% Define the default values for the |\version| flag
% (|final| for the main file and |draft| for childs):
%    \begin{macrocode}
\ifchilddoc
\providecommand{\version}{draft}
\else
\providecommand{\version}{final}
\fi
%    \end{macrocode}

% Load the standard document class:
%    \begin{macrocode}
\documentclass[12pt]{article}
%    \end{macrocode}

% Start the document body:
%    \begin{macrocode}
\begin{document}
%    \end{macrocode}

% Declare a title page.
% Print title, part of document being processed and version flag:
%    \begin{macrocode}
\addtocounter{page}{-1}
\begin{center}
{\LARGE\bfseries{}childdoc example\par}
\vspace{1cm}
\ifchilddoc
\ifchilddocmanual part\else chapter\fi:
`\childdocname' of `\childdocjob'\par
\else
main document: `\childdocjob'\par
\fi
version: \version\par
\end{center}
\newpage
%    \end{macrocode}

% Manually include selected file,
% otherwise process as usual:
%    \begin{macrocode}
\ifchilddocmanual
\section*{part `\childdocname'}
\input{\childdocname}
\else
%    \end{macrocode}

% Include the two chapters:
%    \begin{macrocode}
\include{cdocsch1}
\include{cdocsch2}
%    \end{macrocode}

% Include the two parts unless only chapters should be displayed:
%    \begin{macrocode}
\ifchilddoc\else
\section{part three}
\input{cdocspt3}
\section{part four}
\input{cdocspt4}
\fi
%    \end{macrocode}

% Process as usual until here:
%    \begin{macrocode}
\fi
%    \end{macrocode}

% End of document body:
%    \begin{macrocode}
\end{document}
%    \end{macrocode}
%\iffalse
%</samplemain>
%\fi
%
% %%%%%%%%%%%%%%%%%%%%%%%%%%%%%%%%%%%%%%
% \paragraph{Chapter Include Files.}
%
% The include files are called |cdocsch1.tex| and |cdocsch2.tex|.
%
%\iffalse
%<*samplechap1|samplechap2>
%\fi

% Optional override for |\version| flag:
%    \begin{macrocode}
%%\providecommand{\version}{final}
%    \end{macrocode}

% Include the main document:
%    \begin{macrocode}
\input{childdoc.def}
\childdocof{cdocsamp}
%    \end{macrocode}

%\iffalse
%</samplechap1|samplechap2>
%\fi
%
%\iffalse
%<*samplechap1>
%\fi
% Some text for chapter 1:
%    \begin{macrocode}
\section{one}
some text in chapter one
%    \end{macrocode}

%\iffalse
%</samplechap1>
%\fi
% Some text for chapter 2:
%\iffalse
%<*samplechap2>
%\fi
%    \begin{macrocode}
\section{two}
more text in chapter two
%    \end{macrocode}

%\iffalse
%</samplechap2>
%\fi
%
% %%%%%%%%%%%%%%%%%%%%%%%%%%%%%%%%%%%%%%
% \paragraph{Part Include Files.}
%
% The include files are called |cdocspt3.tex| and |cdocspt4.tex|.
%
%\iffalse
%<*samplepart3|samplepart4>
%\fi

% Optional override for |\version| flag:
%    \begin{macrocode}
%%\providecommand{\version}{final}
%    \end{macrocode}

% Include the main document:
%    \begin{macrocode}
\input{childdoc.def}
\childdocby{cdocsamp}
%    \end{macrocode}

%\iffalse
%</samplepart3|samplepart4>
%\fi
%
%\iffalse
%<*samplepart3>
%\fi
% Some text for part 3:
%    \begin{macrocode}
some text in part three
%    \end{macrocode}

%\iffalse
%</samplepart3>
%\fi
% Some text for part 4:
%\iffalse
%<*samplepart4>
%\fi
%    \begin{macrocode}
more text in part four
%    \end{macrocode}

%\iffalse
%</samplepart4>
%\fi
%
% %%%%%%%%%%%%%%%%%%%%%%%%%%%%%%%%%%%%%%
% \paragraph{Forwarding for a Complete Draft.}
%
% The following forwarding file |cdocsdrf.tex|
% compiles the main document in draft mode:
%\iffalse
%<*sampledraft>
%\fi
%    \begin{macrocode}
\def\version{draft}
\input{childdoc.def}
\childdocforward{cdocsamp}
%    \end{macrocode}

%\iffalse
%</sampledraft>
%\fi
%
% %%%%%%%%%%%%%%%%%%%%%%%%%%%%%%%%%%%%%%
% \paragraph{Forwarding for Final Version of the Chapters.}
%
% The following forwarding files |cdocsfn1.tex| and |cdocsfn2.tex|
% (with identical content)
% compile the final versions of the child documents
% |cdocsch1.tex| and |cdocsch2.tex|, respectively:
%\iffalse
%<*samplefinal>
%\fi
%    \begin{macrocode}
\def\version{final}
\input{childdoc.def}
\childdocforwardprefix[cdocsamp]{cdocsfn}{cdocsch}
%    \end{macrocode}

%\iffalse
%</samplefinal>
%\fi
%
% %%%%%%%%%%%%%%%%%%%%%%%%%%%%%%%%%%%%%%
% \paragraph{Command Line Processing.}
%
% The following three command lines generate the output files
% |cdocscld|, |cdocscl1| and |cdocscl2|
% which should be identical to
% |cdocsdrf|, |cdocsch1| and |cdocsfn2|, respectively:
% \begin{center}
% \begin{tabular}{l}
% |latex -jobname cdocscld \|\\
% |  "\def\version{draft}\input{childdoc.def}\childdocforward{cdocsamp}"|\\
% |latex -jobname cdocscl1 \|\\
% |  "\input{childdoc.def}\childdocforward[cdocsamp]{cdocsch1}"|\\
% |latex -jobname cdocscl2 \|\\
% |  "\def\version{final}\input{childdoc.def}\childdocforward{cdocsch2}"|
% \end{tabular}
% \end{center}
% Note that the trailing backslash on each first line
% merely continues the input to the second line
% (for convenient cut ant paste).
% Furthermore, the command |latex| can be replaced by any
% of its alternative versions such as |pdflatex|.
%
% %%%%%%%%%%%%%%%%%%%%%%%%%%%%%%%%%%%%%%%%%%%%%%%%%%%%%%%%%%%%%%%%%%%%%%%%%%%%%%
% %%%%%%%%%%%%%%%%%%%%%%%%%%%%%%%%%%%%%%%%%%%%%%%%%%%%%%%%%%%%%%%%%%%%%%%%%%%%%%
% \section{Implementation}
%\iffalse
%<*package>
%\fi
%
% This section describes the definitions file |childdoc.def|.

% The definitions cannot be loaded using |\usepackage| or |\RequirePackage|
% which has a mechanism to prevent loading a style file more than once.
% When loading the definitions by means of |\input|
% multiple instances have to be prevented manually:
%\iffalse
%This code needs to be before the `\ProvidesFile' directive
%which is defined at the beginning of this file.
%Therefore it is also placed there and commented out here.
%</package>
%<*discard>
%\fi
%    \begin{macrocode}
\ifdefined\childdocmain\endinput\fi
%    \end{macrocode}
%\iffalse
%</discard>
%<*package>
%\fi
%
% \macro{\ifchilddoc}
% \macro{\ifchilddocmanual}
% The conditional |\ifchilddoc| tells whether a
% child (true) or main (false) document is being compiled.
% The conditional |\ifchilddocmanual| tells whether
% the |\includeonly| mechanism is used (false) or
% the selection of child files must be performed manually (true).
% The definitions initialise to false:
%    \begin{macrocode}
\newif\ifchilddoc
\newif\ifchilddocmanual
%    \end{macrocode}

% \macro{\childdocname}
% \macro{\childdocjob}
% The macro |\childdocname| stores the name of the main document
% to be compiled. The macro |\childdocjob| stores the name of
% the document on which the \LaTeX{} compiler was originally invoked.
% The content of |\jobname| cannot be compared
% to filenames specified in the source due to different catcodes.
% The following code rescans |\jobname|, stores the result
% in |\childdocname| and saves a copy in |\childdocjob|:
%    \begin{macrocode}
\edef\childdocname{\scantokens\expandafter{\jobname\noexpand}}
\let\childdocjob\childdocname
%    \end{macrocode}

% \macro{\childdocdisable}
% The macro |\childdocdisable| prevents the main file
% from being processed more than once.
% At this stage, the main document command |\childdocmain|
% is assumed to be called once again where it should do nothing.
% Any subsequent call to it should prevent
% a secondary processing of the main document
% It overwrites the forwarding commands
% |\childdocof| and |\childdocforward|
% with empty macros to prevent further inclusions of the main document:
%    \begin{macrocode}
\newcommand{\childdocdisable}
{
  \renewcommand{\childdocmain}[1]{\renewcommand{\childdocmain}[1]{\endinput}}
  \renewcommand{\childdocof}[1]{}
  \renewcommand{\childdocby}[2][]{}
  \renewcommand{\childdocforward}[2][]{}
  \renewcommand{\childdocdisable}{}
}
%    \end{macrocode}

% \macro{\childdocmain}
% The macro |\childdocmain| is to be called at the top of the main file
% with nothing or the main filename (without extension) as argument.
% First, it breaks loops.
% If the argument is not empty and does not match |\childdocname|
% (which is set by the first inclusion of |childdoc.def|),
% |\ifchilddoc| is set to true, |\includeonly| is applied to the child file
% and |\jobname| is set to the main file
% (for proper handling of |.aux| files):
%    \begin{macrocode}
\newcommand{\childdocmain}[1]
{
  \childdocdisable\childdocmain{}
  \if?#1?\else
    \begingroup
      \def\childdoctmp{#1}
      \ifx\childdoctmp\childdocname
        \def\childdoctmp{}
      \else
        \def\childdoctmp
        {
          \childdoctrue
          \includeonly{\childdocname}
          \def\childdocjob{#1}
          \def\jobname{#1}
        }
      \fi
      \expandafter
    \endgroup
    \childdoctmp
  \fi
}
%    \end{macrocode}

% \macro{\childdocof}
% The command |\childdocof| redirects
% compilation to the main file |#1|.
%    \begin{macrocode}
\newcommand{\childdocof}[1]
{
  \childdocdisable
  \childdoctrue
  \includeonly{\childdocname}
  \def\jobname{#1}
  \def\childdocjob{#1}
  \input{#1}
}
%    \end{macrocode}

% \macro{\childdocby}
% The command |\childdocby| ....
%    \begin{macrocode}
\newcommand{\childdocby}[2][]
{
  \childdocdisable
  \childdoctrue
  \childdocmanualtrue
  \if?#1?\else
    \def\jobname{#2}
  \fi
  \def\childdocjob{#2}
  \input{#2}
  \endinput
}
%    \end{macrocode}

% \macro{\childdocforward}
% The command |\childdocforward| redirects
% compilation to the main file or
% (if the optional argument is given) a child file.
% Parameters are set as if the main file
% or a child file starting with |\childdocof| was compiled.
% Then compilation is handed over to the main file:
%    \begin{macrocode}
\newcommand{\childdocforward}[2][]
{
  \begingroup
    \if?#1?
      \def\childdoctmp
      {
        \def\childdocname{#2}
        \def\childdocjob{#2}
        \def\jobname{#2}
        \input{#2}
        \endinput
      }
    \else
      \def\childdoctmp
      {
        \childdocdisable
        \def\childdocname{#2}
        \childdoctrue
        \includeonly{#2}
        \def\childdocjob{#1}
        \def\jobname{#1}
        \input{#1}
        \endinput
      }
    \fi
    \expandafter
  \endgroup
  \childdoctmp
}
%    \end{macrocode}

% \macro{\childdocforwardprefix}
% The command |\childdocforwardprefix| redirects
% compilation to the main or a child file by means of a pattern.
% The prefix |#1| in the current filename is replaced by |#2|
% and the suffix of the current filename is kept
% (it is assumed that the filename does not contain the substring `|~~~|'
% which is used as a delimiter).
% Compilation is handed over to the new file by |\childdocforward|:
%    \begin{macrocode}
\newcommand{\childdocforwardprefix}[3][]
{
  \begingroup
    \def\childdocextract #2##1~~~{\def\childdoctmp{\childdocforward[#1]{#3##1}}}
    \expandafter\childdocextract\childdocname~~~
    \expandafter
  \endgroup
  \childdoctmp
}
%    \end{macrocode}

% \macro{\childdoc}
% The deprecated macro |\childdoc| is a legacy version of |\childdocmain|:
%    \begin{macrocode}
\newcommand{\childdoc}{\childdocmain}
%    \end{macrocode}

% \macro{\childdocredirect}
% The deprecated macro |\childdocredirect| is a legacy version
% of |\childdocforward| and |\childdocforwardprefix|:
%    \begin{macrocode}
\newcommand{\childdocredirect}[2][]
{
  \begingroup
    \if?#1?
      \def\childdoctmp{\childdocforward{#2}}
    \else
      \def\childdoctmp{\childdocforwardprefix{#1}{#2}}
    \fi
    \expandafter
  \endgroup
  \childdoctmp
}
%    \end{macrocode}

%\iffalse
%</package>
%\fi
%
\endinput
|\\
|\childdocforwardprefix[|\textit{main}|]{|\textit{prefix}|}{|\textit{dest}|}|
\end{tabular}
\end{center}
%
the destination file is determined by a pattern
depending on the current file:
To make this work, the current file must be called
`{\textit{prefix}\hspace{0.2em}\textit{suffix}}'
with \textit{prefix} matching precisely the argument.
Processing is then passed on to the file
`{\textit{dest}\hspace{0.2em}\textit{suffix}}'.
Surely, the same effect is achieved by
directly specifying the
argument `{\textit{dest}\hspace{0.2em}\textit{suffix}}'
in the first form.
However, that requires to set up a different file
for each child. With the alternative form of the command
all these files can have exactly the same content
which simplifies setting them up and maintaining them.

For example, the following file |draft.tex|
with a compilation flag |\version| as described in \secref{sec:flags}
compiles the main document as a draft:
%
\begin{center}
\begin{tabular}{l}
|\def\version{draft}|\\
|% \iffalse
%
% childdoc.dtx Copyright (C) 2017-2018 Niklas Beisert
%
% This work may be distributed and/or modified under the
% conditions of the LaTeX Project Public License, either version 1.3
% of this license or (at your option) any later version.
% The latest version of this license is in
%   http://www.latex-project.org/lppl.txt
% and version 1.3 or later is part of all distributions of LaTeX
% version 2005/12/01 or later.
%
% This work has the LPPL maintenance status `maintained'.
%
% The Current Maintainer of this work is Niklas Beisert.
%
% This work consists of the files childdoc.dtx and childdoc.ins
% and the derived files childdoc.def and cdocsamp.tex with
% cdocsch1.tex, cdocsch2.tex, cdocsdrf.tex, cdocsfn1.tex, cdocsfn2.tex.
%
%<package>\ifdefined\childdocmain\endinput\fi
%<package>\ProvidesFile{childdoc.def}[2018/12/30 v2.0 child document driver]
%<samplemain>\ProvidesFile{cdocsamp.tex}[2018/12/30 v2.0 sample for childdoc]
%<*driver>
%\ProvidesFile{childdoc.drv}[2018/12/30 v2.0 childdoc reference manual file]
\PassOptionsToClass{10pt,a4paper}{article}
\documentclass{ltxdoc}

\usepackage[margin=35mm]{geometry}
\usepackage{hyperref}
\usepackage{hyperxmp}
\usepackage[usenames]{color}

\hypersetup{colorlinks=true}
\hypersetup{pdfstartview=FitH}
\hypersetup{pdfpagemode=UseNone}
\hypersetup{pdfsource={}}
\hypersetup{pdflang={en-UK}}
\hypersetup{pdfcopyright={Copyright 2017-2018 Niklas Beisert.
  This work may be distributed and/or modified under the
  conditions of the LaTeX Project Public License, either version 1.3
  of this license or (at your option) any later version.}}
\hypersetup{pdflicenseurl={http://www.latex-project.org/lppl.txt}}
\hypersetup{pdfcontactaddress={ETH Zurich, ITP, HIT K,
  Wolfgang-Pauli-Strasse 27}}
\hypersetup{pdfcontactpostcode={8093}}
\hypersetup{pdfcontactcity={Zurich}}
\hypersetup{pdfcontactcountry={Switzerland}}
\hypersetup{pdfcontactemail={nbeisert@itp.phys.ethz.ch}}
\hypersetup{pdfcontacturl={http://people.phys.ethz.ch/\xmptilde nbeisert/}}

\newcommand{\secref}[1]{\hyperref[#1]{section \ref*{#1}}}

\parskip1ex
\parindent0pt
\let\olditemize\itemize
\def\itemize{\olditemize\parskip0pt}

\begin{document}

\title{The \textsf{childdoc} Package}
\hypersetup{pdftitle={The childdoc Package}}
\author{Niklas Beisert\\[2ex]
  Institut f\"ur Theoretische Physik\\
  Eidgen\"ossische Technische Hochschule Z\"urich\\
  Wolfgang-Pauli-Strasse 27, 8093 Z\"urich, Switzerland\\[1ex]
  \href{mailto:nbeisert@itp.phys.ethz.ch}
  {\texttt{nbeisert@itp.phys.ethz.ch}}}
\hypersetup{pdfauthor={Niklas Beisert}}
\hypersetup{pdfsubject={Manual for the LaTeX2e Package childdoc}}
\date{30 December 2018, \textsf{v2.0}}
\maketitle

\begin{abstract}\noindent
\textsf{childdoc} is a \LaTeXe{} package
that enables the direct compilation
of document sections included by |\include|
to individual files.
\end{abstract}

\begingroup
\parskip0ex
\tableofcontents
\endgroup

%%%%%%%%%%%%%%%%%%%%%%%%%%%%%%%%%%%%%%%%%%%%%%%%%%%%%%%%%%%%%%%%%%%%%%%%%%%%%%%%
%%%%%%%%%%%%%%%%%%%%%%%%%%%%%%%%%%%%%%%%%%%%%%%%%%%%%%%%%%%%%%%%%%%%%%%%%%%%%%%%
\section{Introduction}

\LaTeX{} provides a mechanism to structure a large document (such as a book)
into a main file and several child files (containing the chapters)
using the |\include| command.
This mechanism is beneficial for documents
which span hundreds of pages in order to
make the source file(s) more manageable.
Moreover, compilation can be restricted to
selected child files by means of the |\includeonly| command.
The latter feature can be used to reduce the compilation time while editing
(this was significantly more useful in the earlier days of \LaTeX{})
or to generate a smaller document which is easier to navigate.
Another application of |\includeonly| is to generate
documents consisting of selected parts of the complete document.

However, there are a few drawbacks of the plain |\include| mechanism:
\begin{itemize}
\item
The child files cannot be compiled on their own,
they can only be compiled via the main file.
A naive editing environment
(such as a text editor with an option
to have the current file processed by \LaTeX)
may require one to switch to the main file before compiling;
attempting to compile the child file produces errors.
\item
The main file must be modified (each time)
to adjust the |\includeonly| command
to the present needs. This easily leaves the main file in a messy state.
\item
The generated document will always carry the filename
of the main document. This is inconvenient if
several child files are to be compiled and
to be kept for distribution.
\end{itemize}

The present package provides a simple interface
to make child files individually compilable by \LaTeX{}.
Compiling a child file then has the same effect as compiling
the main file with an |\includeonly| command
to select the appropriate child.
Moreover the generated document will carry the name of the child
rather than the main file.
This resolves all three above issues.

This feature is meant to make the editing of books,
thesis documents and lecture notes somewhat more convenient.
However, the package can also be used efficiently for
composing a series of documents (such as exercise sheets)
which are typically distributed individually.
It then assists the author in generating the individual documents
(potentially in different versions)
as well as a document containing the collected series.
Another application is in developing style files
or other kinds of included material
where compilation of the style file could redirect
to a sample or test file.

%%%%%%%%%%%%%%%%%%%%%%%%%%%%%%%%%%%%%%%%%%%%%%%%%%%%%%%%%%%%%%%%%%%%%%%%%%%%%%%%
%%%%%%%%%%%%%%%%%%%%%%%%%%%%%%%%%%%%%%%%%%%%%%%%%%%%%%%%%%%%%%%%%%%%%%%%%%%%%%%%
\section{Usage}

First of all, the package \textsf{childdoc} is \emph{not} a standard
\LaTeXe{} |.sty| style file! Therefore it needs to be invoked in
a non-standard way.

%%%%%%%%%%%%%%%%%%%%%%%%%%%%%%%%%%%%%%%%%%%%%%%%%%%%%%%%%%%%%%%%%%%%%%%%%%%%%%%%
\subsection{Included Files}
\label{sec:include}

%%%%%%%%%%%%%%%%%%%%%%%%%%%%%%%%%%%%%%%%
\DescribeMacro{\childdocmain}
To use the package, add the commands
\begin{center}
\begin{tabular}{l}
|\input{childdoc.def}|\\
|\childdocmain{}|\\
\end{tabular}
\end{center}
at the very top of the main \LaTeX{} file,
in particular \emph{before} the |\documentclass| statement!
The argument of |\childdocmain| should be left empty
(but it must be present).

%%%%%%%%%%%%%%%%%%%%%%%%%%%%%%%%%%%%%%%%
\DescribeMacro{\childdocof}
Furthermore, add the commands
\begin{center}
\begin{tabular}{l}
|\input{childdoc.def}|\\
|\childdocof{|\textit{main}|}|\\
\end{tabular}
\end{center}
at the top of every child file \textit{child}
which is included by |\include{|\textit{child}|}|
from within the main file
(or at least for those files to be compiled individually).
The argument \textit{main} must be the filename of the main file.

There are a couple of
considerations in setting up the main and child documents:

%%%%%%%%%%%%%%%%%%%%%%%%%%%%%%%%%%%%%%%%
\paragraph{Restrictions.}

Please note the following restrictions:
\begin{itemize}
\item
|\childdocmain| must be called with one argument \textit{main}
to ensure compatibility with earlier version of the package.
It must either be empty (|\childdocmain{}|)
or precisely match the filename of the main file in which it is specified.
See \secref{sec:detection} for further information.
\item
The filename \textit{main} must be specified without the |.tex| extension.
\item
The filename \textit{main} is case sensitive
(even in case-insensitive file systems)
due to internal string comparison.
\item
The argument \textit{main} should be fully expanded, it cannot be a macro.
\item
Subdirectories and special characters should be avoided in filenames.
\item
The command |\childdocmain{|\textit{main}|}| must be followed by a whitespace.
It should not be followed immediately by another command
or by a comment mark `|%|'.
This is because the \TeX{} parser reads the token immediately following
the argument of |\childdocmain| and puts it
at the beginning of every child section;
however, a white\-space is ignored.
\end{itemize}

%%%%%%%%%%%%%%%%%%%%%%%%%%%%%%%%%%%%%%%%
\paragraph{Content of Main File.}

It is advisable to place all content in the child files included by |\include|.
Any output contained in the main file will appear in all child documents
unless suppressed manually;
it cannot be suppressed automatically by the |\includeonly| directive
and thus should normally be avoided.
A method to include some content in the main file
by means of conditional processing is described in \secref{sec:conditional}.

%%%%%%%%%%%%%%%%%%%%%%%%%%%%%%%%%%%%%%%%
\paragraph{Page Numbering.}

When only a part of the document is compiled,
the appropriate numbering of pages
(as well as other status parameters)
is determined from the |.aux| files.
The latter contain information from previous passes.
However this information needs to propagate through
all intermediate child documents.
Therefore the page numbering in child documents may well
be inconsistent until the complete document is compiled at least once.

A useful (if unconventional) way to always ensure a consistent
page numbering is to restart the numbering in each child document
and denote the pages by `\textit{child}|.|\textit{page}'
where \textit{child} represents the chapter/section number of the child file.
This can be achieved by the command
|\numberwithin{page}{|\textit{child}|}|
of the \textsf{amsmath} package
where \textit{child} can be |chapter| or |section|
depending on the chosen structuring.
Alternatively, one can modify the macro |\thepage| appropriately
and reset the counter |page| at the start of each child file.

%%%%%%%%%%%%%%%%%%%%%%%%%%%%%%%%%%%%%%%%%%%%%%%%%%%%%%%%%%%%%%%%%%%%%%%%%%%%%%%%
\subsection{Conditional Processing}
\label{sec:conditional}

The package provides a mechanism to compile different versions
of a document. To customise the versions further some conditional processing
can come in handy to distinguish which version is being compiled.
The package provides two macros to describe the compilation context:

%%%%%%%%%%%%%%%%%%%%%%%%%%%%%%%%%%%%%%%%
\DescribeMacro{\ifchilddoc}
The conditional |\ifchilddoc| distinguishes between the compilation of
child documents and the main document:
%
\begin{center}
|\ifchilddoc |\textit{child-code}| |[|\||else |\textit{main-code}]| \||fi|
\end{center}

%%%%%%%%%%%%%%%%%%%%%%%%%%%%%%%%%%%%%%%%
\DescribeMacro{\childdocname}
\DescribeMacro{\childdocjob}
The macro |\childdocname| contains the filename (without extension)
of the main or child file being processed.
Note that |\childdocjob| will always contain the name of the main file.

%%%%%%%%%%%%%%%%%%%%%%%%%%%%%%%%%%%%%%%%
\paragraph{Title Page.}

Conditional processing can be used to include a title or banner page
in the main document when proper precautions are taken.
Importantly, the code in the main file should ensure that the page counter
(as well as other status parameters which are stored in the |.aux| files)
takes the same value after the conditional processing.
Otherwise the page numbers may take divergent values
depending on which part is compiled.

For example, a title page could be declared by:
%
\begin{center}
\begin{tabular}{l}
|\ifchilddoc\||else|\\
|\addtocounter{page}{-1}|\\
\textit{code for title page}\\
|\newpage|\\
|\||fi|
\end{tabular}
\end{center}
%
A banner page for the child documents can be generated by:
%
\begin{center}
\begin{tabular}{l}
|\ifchilddoc|\\
|\addtocounter{page}{-1}|\\
\textit{code for banner page}\\
|\newpage|\\
|\||fi|
\end{tabular}
\end{center}
%
Here one could write a message such as:
\begin{center}
|This is the part \childdocname{} of \childdocjob{}.|
\end{center}

%%%%%%%%%%%%%%%%%%%%%%%%%%%%%%%%%%%%%%%%%%%%%%%%%%%%%%%%%%%%%%%%%%%%%%%%%%%%%%%%
\subsection{Flags}
\label{sec:flags}

The package makes it easy to generate different versions
of the main or child documents.
To this end compilation flags can be defined
and assigned different default values.
They will be particularly useful in conjunction
with the forwarding mechanism described in \secref{sec:forward}.

For example, it may be useful to have a flag |\version|
which can be set to |draft| or |final|.
The document source will contain some conditional code
depending on the value of |\version|.
Suppose further, the flag should default to |final| for the main file
and to |draft| for child files
which is a natural assignment for editing the document.
This is achieved by placing the following code
in the preamble of the main document
(below the |\childdocmain| directive):
%
\begin{center}
\begin{tabular}{l}
|\ifchilddoc|\\
|\providecommand{\version}{draft}|\\
|\||else|\\
|\providecommand{\version}{final}|\\
|\||fi|
\end{tabular}
\end{center}
%
The definition by |\providecommand| makes sure
that previous definitions are not overwritten.
Further statements |\providecommand{\version}{...}|
can thus be added before the above code to override it.

For the main file, one might add a line
(between |\childdocmain| and the above block)
%
\begin{center}
|%\ifchilddoc\||else\providecommand{\version}{draft}\||fi|
\end{center}
%
which can be uncommented to produce a draft version.
Likewise one can add a line to the very top of a child file
(above the |\childdocof{|\textit{main}|}| directive)
%
\begin{center}
|%\providecommand{\version}{final}|
\end{center}
%
which can be uncommented to produce the final version of this child document.

%%%%%%%%%%%%%%%%%%%%%%%%%%%%%%%%%%%%%%%%%%%%%%%%%%%%%%%%%%%%%%%%%%%%%%%%%%%%%%%%
\subsection{Forwarding}
\label{sec:forward}

Different versions of the main or child documents
using compilation flags as described in \secref{sec:flags}
can be (permanently) stored in different files
for convenient compilation, viewing and distribution.
To this end, the package defines a command
to pass on compilation to a different file:

%%%%%%%%%%%%%%%%%%%%%%%%%%%%%%%%%%%%%%%%
\DescribeMacro{\childdocforward}
The command |\childdocforward| redirects processing to
another source file:
%
\begin{center}
\begin{tabular}{l}
|\input{childdoc.def}|\\
|\childdocforward[|\textit{main}|]{|\textit{dest}|}|\\
\end{tabular}
\end{center}
%
The argument \textit{dest} is the destination file
(without extension).
It should be the main file or one of the child files.
Note that further \textsf{childdoc} directives
such as |\childdocof| and |\childdocforward|
in the indicated file will be processed in this form.
The optional argument \textit{main}
passes on directly to the main file \textit{main}
while pretending to compile the child \textit{dest}.
This form behaves as if \textit{dest}
issues |\childdocof{|\textit{main}|}| right away,
and no further \textsf{childdoc} directives will be processed.

%%%%%%%%%%%%%%%%%%%%%%%%%%%%%%%%%%%%%%%%
\DescribeMacro{\...prefix}
In the alternative form |\childdocforwardprefix|,
%
\begin{center}
\begin{tabular}{l}
|\input{childdoc.def}|\\
|\childdocforwardprefix[|\textit{main}|]{|\textit{prefix}|}{|\textit{dest}|}|
\end{tabular}
\end{center}
%
the destination file is determined by a pattern
depending on the current file:
To make this work, the current file must be called
`{\textit{prefix}\hspace{0.2em}\textit{suffix}}'
with \textit{prefix} matching precisely the argument.
Processing is then passed on to the file
`{\textit{dest}\hspace{0.2em}\textit{suffix}}'.
Surely, the same effect is achieved by
directly specifying the
argument `{\textit{dest}\hspace{0.2em}\textit{suffix}}'
in the first form.
However, that requires to set up a different file
for each child. With the alternative form of the command
all these files can have exactly the same content
which simplifies setting them up and maintaining them.

For example, the following file |draft.tex|
with a compilation flag |\version| as described in \secref{sec:flags}
compiles the main document as a draft:
%
\begin{center}
\begin{tabular}{l}
|\def\version{draft}|\\
|\input{childdoc.def}|\\
|\childdocforward{|\textit{main}|}|
\end{tabular}
\end{center}
%
Likewise, the following files |final|\textit{nn}|.tex|
compile the final version of the child document
|child|\textit{nn}|.tex|:
%
\begin{center}
\begin{tabular}{l}
|\def\version{final}|\\
|\input{childdoc.def}|\\
|\childdocforwardprefix{final}{child}|
\end{tabular}
\end{center}
%

Note that when several versions of a main file and/or of each child file
are to be generated, it may be convenient to set up a |Makefile| or
shell script to automatise the process.

%%%%%%%%%%%%%%%%%%%%%%%%%%%%%%%%%%%%%%%%%%%%%%%%%%%%%%%%%%%%%%%%%%%%%%%%%%%%%%%%
\subsection{Command Line Processing}
\label{sec:commandline}

The effect of redirection files can also be achieved by invoking
the \LaTeX{} compiler with a more elaborate command line.
Most conveniently this should be done as part
of a shell script or a |Makefile|.

When using \textsf{childdoc} in the main file, the following
command lines effectively perform a redirection
(note that depending on the shell being used,
backslashes may have to be doubled: `|\|' $\to$ `|\\|'):
%
\begin{center}
|... -jobname "|\textit{target}|" |\\|"|[\textit{flags}]%
|\input{childdoc.def}\childdocforward[|\textit{main}|]{|\textit{dest}|}"|
\end{center}
%
Here \textit{target} is the name of the output file,
\textit{main} is the name of the main file
and \textit{dest} is the name of the main or child file to be processed
(all filenames without extensions).
The optional argument \textit{main} can be omitted
if \textit{main} matches \textit{dest}.
Optionally, compilation \textit{flags} can be defined via |\def| commands.
This command line makes the \TeX{} engine believe
it is compiling the file \textit{target}
whose content is specified as the latter parameter.
The provided code then forwards the processing to
\textit{main} or \textit{dest} as described in \secref{sec:forward}.

%%%%%%%%%%%%%%%%%%%%%%%%%%%%%%%%%%%%%%%%%%%%%%%%%%%%%%%%%%%%%%%%%%%%%%%%%%%%%%%%
\subsection{Include by Input}
\label{sec:input}

Including child documents by |\include| has some restrictions by design.
Most notably, the content of a child document always occupies
its own set of pages; pages cannot be shared between child documents.
Usually, this behaviour makes perfect sense
because each child document contain an essential part of the document.
However, in some situations it may be desirable to compose
a document from a collection of parts
without having mandatory page breaks between then.
For this case, the package
provides a mechanism to include parts
by |\input| which can also be processed individually.
However, by construction this mechanism
requires manual handling of the content to be output.

%%%%%%%%%%%%%%%%%%%%%%%%%%%%%%%%%%%%%%%%
\DescribeMacro{\ifchilddocmanual}
The main file should be prepared as usual, see \secref{sec:include}.
However, the document body must make a distinction
between processing of an individual part and of the main document, e.g.:
%
\begin{center}
\begin{tabular}{l}
|\ifchilddocmanual|\\
|\input{\childdocname}|\\
|\||else|\\
\textit{document body with }|\input{|\textit{part}|}|\\
|\||fi|
\end{tabular}
\end{center}
%
The conditional |\ifchilddocmanual| is true whenever
a part to be included by |\input| is being compiled,
and the name of the part is stored in |\childdocname|.

%%%%%%%%%%%%%%%%%%%%%%%%%%%%%%%%%%%%%%%%
\DescribeMacro{\childdocby}
Each part to be included by |\input| should start with:
%
\begin{center}
\begin{tabular}{l}
|\input{childdoc.def}|\\
|\childdocby{|\textit{main}|}|\\
\end{tabular}
\end{center}
%
The directive |\childdocby| is similar to |\childdocof|
described in \secref{sec:include},
but the subsequent selection of content must be done manually.
To that end, both |\ifchilddoc| and |\ifchilddocmanual|
will be true upon processing of a part,
and the name of the part is stored in |\childdocname|.
Note that |\jobname| will be set to the filename of the current part
so that each part receives an individual |.aux| file
that does not interfere with the |.aux| file(s) of the main document.
This behaviour can be altered by the alternative form
|\childdocby[*]{|\textit{main}|}| (with a non-empty optional argument)
which uses the |.aux| file of the main document
by setting |\jobname| to \textit{main}.

%%%%%%%%%%%%%%%%%%%%%%%%%%%%%%%%%%%%%%%%%%%%%%%%%%%%%%%%%%%%%%%%%%%%%%%%%%%%%%%%
\subsection{Driver Development}
\label{sec:driver}

The \textsf{childdoc} mechanism can also be use for the development
of definition files such as \LaTeX{} styles or classes.
This case differs from the above setup with multiple parts
included by |\include| in that no |\includeonly| should be invoked.
This can be achieved by starting the include file
(before |\ProvidesPackage|) with:
%
\begin{center}
\begin{tabular}{l}
|\input{childdoc.def}|\\
|\childdocforward{|\textit{main}|}|\\
\end{tabular}
\end{center}
%
or alternatively with:
%
\begin{center}
\begin{tabular}{l}
|\input{childdoc.def}|\\
|\childdocby{|\textit{main}|}|\\
\end{tabular}
\end{center}
%
Both forms have slightly different effects as described above.
The main file is prepared as usual, see \secref{sec:include}.

%%%%%%%%%%%%%%%%%%%%%%%%%%%%%%%%%%%%%%%%%%%%%%%%%%%%%%%%%%%%%%%%%%%%%%%%%%%%%%%%
\subsection{Legacy Detection}
\label{sec:detection}

The directive |\childdocmain| in the main file can detect
whether the complete document or merely a child is to be compiled
even without using the directive |\childdocof|.
This method is deprecated because it is less robust
and there is no compelling reason to use it;
it is merely provided for backward compatibility
and it may be removed in future versions.

If the detection mechanism is to be used,
it is mandatory to correctly specify
the filename of the main file as the argument of |\childdocmain|:
%
\begin{center}
\begin{tabular}{l}
|\input{childdoc.def}|\\
|\childdocmain{|\textit{main}|}|\\
\end{tabular}
\end{center}
%
If |\jobname| does not match the argument \textit{main} of |\childdocmain|,
it is assumed that |\jobname| points to the child file to be compiled.
When using |\childdocmain| with the main file specified as argument,
it suffices to start a child file
with just |\input{|\textit{main}|}|
without loading of the package and using |\childdocof|.
If instead all processing is done
with the appropriate \textsf{childdoc} directives,
the argument of \textit{main} of |\childdocmain| can be empty.

An alternative version of the command line processing described
in \secref{sec:commandline} using the detection mechanism reads:
%
\begin{center}
|... -jobname "|\textit{target}|" "|[\textit{flags}]%
[|\def\jobname{|\textit{dest}|}|]|\input{|\textit{main}|}"|
\end{center}

%%%%%%%%%%%%%%%%%%%%%%%%%%%%%%%%%%%%%%%%%%%%%%%%%%%%%%%%%%%%%%%%%%%%%%%%%%%%%%%%
\subsection{Manual Code}
\label{sec:manual}

In case one cannot be certain whether the definitions file |childdoc.def|
is installed on the target \TeX{} distribution
and one prefers not to ship it,
it is conceivable to paste a few relevant commands into the sources.

To that end, drop all statements |\input{childdoc.def}|
and perform the replacements as outlined below.
Instead of |\childdocmain{|\textit{main}|}| add the following code
to the top of the main file:
%
\begin{center}
\begin{tabular}{l}
|\||ifdefined\childdocname\endinput\||fi\newif\ifchilddoc|\\
|\edef\childdocname{\scantokens\expandafter{\jobname\noexpand}}|\\
|\def\childdocmain{|\textit{main}|}\||ifx\childdocmain\childdocname\||else|\\
|\childdoctrue\includeonly{\childdocname}\let\jobname\childdocmain\||fi|\\
\end{tabular}
\end{center}
%
Instead of |\childdocof{|\textit{main}|}| just include the main file
at the top of each child file:
%
\begin{center}
|\input{|\textit{main}|}|
\end{center}
%
A simple redirection |\childdocforward{|\textit{dest}|}| is achieved by:
%
\begin{center}
|\def\jobname{|\textit{dest}|}\input{\jobname}|
\end{center}
%
The redirection with prefix
|\childdocforwardprefix[|\textit{prefix}|]{|\textit{dest}|}|
is accomplished by:
%
\begin{center}
\begin{tabular}{l}
|{\edef\jobname{\scantokens\expandafter{\jobname\noexpand}}|\\
|\def\redirectjob |\textit{prefix}|#1~~~{\gdef\jobname{|\textit{dest}|#1}}|\\
|\expandafter\redirectjob\jobname~~~}\input{\jobname}|
\end{tabular}
\end{center}

In an alternative approach,
child documents can be compiled by a specific command line
without additional code or specific definitions:
%
\begin{center}
|... -jobname "|\textit{target}|" "|[\textit{flags}]%
|\includeonly{|\textit{dest}|}\input{|\textit{main}|}"|
\end{center}
%

%%%%%%%%%%%%%%%%%%%%%%%%%%%%%%%%%%%%%%%%%%%%%%%%%%%%%%%%%%%%%%%%%%%%%%%%%%%%%%%%
%%%%%%%%%%%%%%%%%%%%%%%%%%%%%%%%%%%%%%%%%%%%%%%%%%%%%%%%%%%%%%%%%%%%%%%%%%%%%%%%
\section{Information}

%%%%%%%%%%%%%%%%%%%%%%%%%%%%%%%%%%%%%%%%%%%%%%%%%%%%%%%%%%%%%%%%%%%%%%%%%%%%%%%%
\subsection{Copyright}

Copyright \copyright{} 2017--2018 Niklas Beisert

This work may be distributed and/or modified under the
conditions of the \LaTeX{} Project Public License, either version 1.3
of this license or (at your option) any later version.
The latest version of this license is in
  \url{http://www.latex-project.org/lppl.txt}
and version 1.3 or later is part of all distributions of \LaTeX{}
version 2005/12/01 or later.

This work has the LPPL maintenance status `maintained'.

The Current Maintainer of this work is Niklas Beisert.

This work consists of the files |README.txt|, |childdoc.ins| and |childdoc.dtx|
as well as the derived files |childdoc.def|, |cdocsamp.tex|
with |cdocsch1.tex|, |cdocsch2.tex|, |cdocspt3.tex|, |cdocspt4.tex|,
|cdocsdrf.tex|, |cdocsfn1.tex|, |cdocsfn2.tex|
as well as |childdoc.pdf|.

%%%%%%%%%%%%%%%%%%%%%%%%%%%%%%%%%%%%%%%%%%%%%%%%%%%%%%%%%%%%%%%%%%%%%%%%%%%%%%%%
\subsection{Files and Installation}

The package consists of the files:
%
\begin{center}
\begin{tabular}{ll}
    |README.txt|   & readme file \\
    |childdoc.ins| & installation file \\
    |childdoc.dtx| & source file \\
    |childdoc.def| & definition file \\
    |cdocsamp.tex| & sample main file \\
    |cdocsch1.tex| & sample include file \\
    |cdocsch2.tex| & sample include file \\
    |cdocspt3.tex| & sample part file \\
    |cdocspt4.tex| & sample part file \\
    |cdocsdrf.tex| & sample redirection file \\
    |cdocsfn1.tex| & sample redirection file \\
    |cdocsfn2.tex| & sample redirection file \\
    |childdoc.pdf| & manual
\end{tabular}
\end{center}
%
The distribution consists of the files
|README.txt|, |childdoc.ins| and |childdoc.dtx|.
%
\begin{itemize}
\item
Run (pdf)\LaTeX{} on |childdoc.dtx|
to compile the manual |childdoc.pdf| (this file).
\item
Run \LaTeX{} on |childdoc.ins| to create the definitions file |childdoc.def|
and the sample |cdocsamp.tex| with include files
|cdocsch1.tex|, |cdocsch2.tex|, |cdocspt3.tex|, |cdocspt4.tex|,
|cdocsdrf.tex|, |cdocsfn1.tex|, |cdocsfn2.tex|.
Then copy the file |childdoc.def| to an appropriate directory of your \LaTeX{}
distribution, e.g.\ \textit{texmf-root}|/tex/latex/childdoc|.
\end{itemize}

%%%%%%%%%%%%%%%%%%%%%%%%%%%%%%%%%%%%%%%%%%%%%%%%%%%%%%%%%%%%%%%%%%%%%%%%%%%%%%%%
\subsection{Related CTAN Packages}

There are several other packages which offer a similar functionality:
%
\begin{itemize}
\item
The packages
\href{http://ctan.org/pkg/docmute}{\textsf{docmute}},
\href{http://ctan.org/pkg/includex}{\textsf{includex}} and
\href{http://ctan.org/pkg/standalone}{\textsf{standalone}}
provide commands to include only the document body of
a child file thus allowing both files to be compiled individually.
\item
The packages \href{http://ctan.org/pkg/subdocs}{\textsf{subdocs}}
and \href{http://ctan.org/pkg/subfiles}{\textsf{subfiles}}
provide structures in which the main and child documents can be
encapsulated and allowing them to be compiled individually.
The inclusion mechanism is different from the conventional |\include|.
\item
The package \href{http://ctan.org/pkg/combine}{\textsf{combine}}
is an elaborate solution to combine several documents into one.
\end{itemize}
%
See also the CTAN topic \href{http://ctan.org/topic/subdocs}{\textsf{subdocs}}
for further related packages.
The present package differs from the above solutions in that
a document structure constructed with the conventional |\include| mechanism
just needs two extra commands at the top of every file
such that all constituent files can be compiled individually.

%%%%%%%%%%%%%%%%%%%%%%%%%%%%%%%%%%%%%%%%%%%%%%%%%%%%%%%%%%%%%%%%%%%%%%%%%%%%%%%%
%\subsection{Feature Suggestions}
%
%The following is a list of features which may be useful for future
%versions of this package:
%%
%\begin{itemize}
%\item
%\ldots
%\end{itemize}

%%%%%%%%%%%%%%%%%%%%%%%%%%%%%%%%%%%%%%%%%%%%%%%%%%%%%%%%%%%%%%%%%%%%%%%%%%%%%%%%
\subsection{Revision History}

%%%%%%%%%%%%%%%%%%%%%%%%%%%%%%%%%%%%%%%%
\paragraph{v2.0:} 2018/12/30

\begin{itemize}
\item
immediate forward processing
\item
added |\childdocby| mechanism
\item
manual restructured
\end{itemize}

%%%%%%%%%%%%%%%%%%%%%%%%%%%%%%%%%%%%%%%%
\paragraph{v1.6:} 2018/01/17

\begin{itemize}
\item
application for development of include files
\item
corrections to manual
\end{itemize}

%%%%%%%%%%%%%%%%%%%%%%%%%%%%%%%%%%%%%%%%
\paragraph{v1.5:} 2017/05/21

\begin{itemize}
\item
more complete structuring introduced
\item
|\childdocof| introduced
\item
|\childdoc| renamed to |\childdocmain|
\item
|\childredirect| renamed to |\childdocforward| and |\childdocforwardprefix|
and functionality expanded
\end{itemize}

%%%%%%%%%%%%%%%%%%%%%%%%%%%%%%%%%%%%%%%%
\paragraph{v1.0:} 2017/04/27

\begin{itemize}
\item
manual and install package
\item
first version published on CTAN
\end{itemize}

%%%%%%%%%%%%%%%%%%%%%%%%%%%%%%%%%%%%%%%%
\paragraph{v0.6:} 2017/04/26

\begin{itemize}
\item
redirection mechanism added
\end{itemize}

%%%%%%%%%%%%%%%%%%%%%%%%%%%%%%%%%%%%%%%%
\paragraph{v0.5:} 2017/04/26

\begin{itemize}
\item
functionality in definition file
\end{itemize}


%%%%%%%%%%%%%%%%%%%%%%%%%%%%%%%%%%%%%%%%%%%%%%%%%%%%%%%%%%%%%%%%%%%%%%%%%%%%%%%%
%%%%%%%%%%%%%%%%%%%%%%%%%%%%%%%%%%%%%%%%%%%%%%%%%%%%%%%%%%%%%%%%%%%%%%%%%%%%%%%%
%%%%%%%%%%%%%%%%%%%%%%%%%%%%%%%%%%%%%%%%%%%%%%%%%%%%%%%%%%%%%%%%%%%%%%%%%%%%%%%%
\appendix

\settowidth\MacroIndent{\rmfamily\scriptsize 000\ }

 \DocInput{childdoc.dtx}

\end{document}
%</driver>
% \fi
%
% %%%%%%%%%%%%%%%%%%%%%%%%%%%%%%%%%%%%%%%%%%%%%%%%%%%%%%%%%%%%%%%%%%%%%%%%%%%%%%
% %%%%%%%%%%%%%%%%%%%%%%%%%%%%%%%%%%%%%%%%%%%%%%%%%%%%%%%%%%%%%%%%%%%%%%%%%%%%%%
% \section{Sample}
%\iffalse
%<*samplemain>
%\fi
%
% The following presents a sample document
% with two chapters, two parts, a title page,
% a compile flag as well as three forwarding files to set the flag.
% It consists of eight |.tex| files:
% \begin{center}
% \begin{tabular}{ll}
% |cdocsamp.tex|&main file\\
% |cdocsch1.tex|&include file for chapter 1\\
% |cdocsch2.tex|&include file for chapter 2\\
% |cdocspt3.tex|&include file for part 3\\
% |cdocspt4.tex|&include file for part 4\\
% |cdocsdrf.tex|&forwarding file for main file in draft mode\\
% |cdocsfi1.tex|&forwarding file for final version of chapter 1\\
% |cdocsfi2.tex|&forwarding file for final version of chapter 2\\
% \end{tabular}
% \end{center}
% Each of the eight files can be compiled directly by the \LaTeX{} compiler.
%
% %%%%%%%%%%%%%%%%%%%%%%%%%%%%%%%%%%%%%%
% \paragraph{Main File.}
%
% The main file is called |cdocsamp.tex|.
%
% Load the \textsf{childdoc} definitions and
% declare the filename for the main document:
%    \begin{macrocode}
\input{childdoc.def}
\childdocmain{}
%    \end{macrocode}

% Optional override for |\version| flag:
%    \begin{macrocode}
%%\ifchilddoc\else\providecommand{\version}{draft}\fi
%    \end{macrocode}

% Define the default values for the |\version| flag
% (|final| for the main file and |draft| for childs):
%    \begin{macrocode}
\ifchilddoc
\providecommand{\version}{draft}
\else
\providecommand{\version}{final}
\fi
%    \end{macrocode}

% Load the standard document class:
%    \begin{macrocode}
\documentclass[12pt]{article}
%    \end{macrocode}

% Start the document body:
%    \begin{macrocode}
\begin{document}
%    \end{macrocode}

% Declare a title page.
% Print title, part of document being processed and version flag:
%    \begin{macrocode}
\addtocounter{page}{-1}
\begin{center}
{\LARGE\bfseries{}childdoc example\par}
\vspace{1cm}
\ifchilddoc
\ifchilddocmanual part\else chapter\fi:
`\childdocname' of `\childdocjob'\par
\else
main document: `\childdocjob'\par
\fi
version: \version\par
\end{center}
\newpage
%    \end{macrocode}

% Manually include selected file,
% otherwise process as usual:
%    \begin{macrocode}
\ifchilddocmanual
\section*{part `\childdocname'}
\input{\childdocname}
\else
%    \end{macrocode}

% Include the two chapters:
%    \begin{macrocode}
\include{cdocsch1}
\include{cdocsch2}
%    \end{macrocode}

% Include the two parts unless only chapters should be displayed:
%    \begin{macrocode}
\ifchilddoc\else
\section{part three}
\input{cdocspt3}
\section{part four}
\input{cdocspt4}
\fi
%    \end{macrocode}

% Process as usual until here:
%    \begin{macrocode}
\fi
%    \end{macrocode}

% End of document body:
%    \begin{macrocode}
\end{document}
%    \end{macrocode}
%\iffalse
%</samplemain>
%\fi
%
% %%%%%%%%%%%%%%%%%%%%%%%%%%%%%%%%%%%%%%
% \paragraph{Chapter Include Files.}
%
% The include files are called |cdocsch1.tex| and |cdocsch2.tex|.
%
%\iffalse
%<*samplechap1|samplechap2>
%\fi

% Optional override for |\version| flag:
%    \begin{macrocode}
%%\providecommand{\version}{final}
%    \end{macrocode}

% Include the main document:
%    \begin{macrocode}
\input{childdoc.def}
\childdocof{cdocsamp}
%    \end{macrocode}

%\iffalse
%</samplechap1|samplechap2>
%\fi
%
%\iffalse
%<*samplechap1>
%\fi
% Some text for chapter 1:
%    \begin{macrocode}
\section{one}
some text in chapter one
%    \end{macrocode}

%\iffalse
%</samplechap1>
%\fi
% Some text for chapter 2:
%\iffalse
%<*samplechap2>
%\fi
%    \begin{macrocode}
\section{two}
more text in chapter two
%    \end{macrocode}

%\iffalse
%</samplechap2>
%\fi
%
% %%%%%%%%%%%%%%%%%%%%%%%%%%%%%%%%%%%%%%
% \paragraph{Part Include Files.}
%
% The include files are called |cdocspt3.tex| and |cdocspt4.tex|.
%
%\iffalse
%<*samplepart3|samplepart4>
%\fi

% Optional override for |\version| flag:
%    \begin{macrocode}
%%\providecommand{\version}{final}
%    \end{macrocode}

% Include the main document:
%    \begin{macrocode}
\input{childdoc.def}
\childdocby{cdocsamp}
%    \end{macrocode}

%\iffalse
%</samplepart3|samplepart4>
%\fi
%
%\iffalse
%<*samplepart3>
%\fi
% Some text for part 3:
%    \begin{macrocode}
some text in part three
%    \end{macrocode}

%\iffalse
%</samplepart3>
%\fi
% Some text for part 4:
%\iffalse
%<*samplepart4>
%\fi
%    \begin{macrocode}
more text in part four
%    \end{macrocode}

%\iffalse
%</samplepart4>
%\fi
%
% %%%%%%%%%%%%%%%%%%%%%%%%%%%%%%%%%%%%%%
% \paragraph{Forwarding for a Complete Draft.}
%
% The following forwarding file |cdocsdrf.tex|
% compiles the main document in draft mode:
%\iffalse
%<*sampledraft>
%\fi
%    \begin{macrocode}
\def\version{draft}
\input{childdoc.def}
\childdocforward{cdocsamp}
%    \end{macrocode}

%\iffalse
%</sampledraft>
%\fi
%
% %%%%%%%%%%%%%%%%%%%%%%%%%%%%%%%%%%%%%%
% \paragraph{Forwarding for Final Version of the Chapters.}
%
% The following forwarding files |cdocsfn1.tex| and |cdocsfn2.tex|
% (with identical content)
% compile the final versions of the child documents
% |cdocsch1.tex| and |cdocsch2.tex|, respectively:
%\iffalse
%<*samplefinal>
%\fi
%    \begin{macrocode}
\def\version{final}
\input{childdoc.def}
\childdocforwardprefix[cdocsamp]{cdocsfn}{cdocsch}
%    \end{macrocode}

%\iffalse
%</samplefinal>
%\fi
%
% %%%%%%%%%%%%%%%%%%%%%%%%%%%%%%%%%%%%%%
% \paragraph{Command Line Processing.}
%
% The following three command lines generate the output files
% |cdocscld|, |cdocscl1| and |cdocscl2|
% which should be identical to
% |cdocsdrf|, |cdocsch1| and |cdocsfn2|, respectively:
% \begin{center}
% \begin{tabular}{l}
% |latex -jobname cdocscld \|\\
% |  "\def\version{draft}\input{childdoc.def}\childdocforward{cdocsamp}"|\\
% |latex -jobname cdocscl1 \|\\
% |  "\input{childdoc.def}\childdocforward[cdocsamp]{cdocsch1}"|\\
% |latex -jobname cdocscl2 \|\\
% |  "\def\version{final}\input{childdoc.def}\childdocforward{cdocsch2}"|
% \end{tabular}
% \end{center}
% Note that the trailing backslash on each first line
% merely continues the input to the second line
% (for convenient cut ant paste).
% Furthermore, the command |latex| can be replaced by any
% of its alternative versions such as |pdflatex|.
%
% %%%%%%%%%%%%%%%%%%%%%%%%%%%%%%%%%%%%%%%%%%%%%%%%%%%%%%%%%%%%%%%%%%%%%%%%%%%%%%
% %%%%%%%%%%%%%%%%%%%%%%%%%%%%%%%%%%%%%%%%%%%%%%%%%%%%%%%%%%%%%%%%%%%%%%%%%%%%%%
% \section{Implementation}
%\iffalse
%<*package>
%\fi
%
% This section describes the definitions file |childdoc.def|.

% The definitions cannot be loaded using |\usepackage| or |\RequirePackage|
% which has a mechanism to prevent loading a style file more than once.
% When loading the definitions by means of |\input|
% multiple instances have to be prevented manually:
%\iffalse
%This code needs to be before the `\ProvidesFile' directive
%which is defined at the beginning of this file.
%Therefore it is also placed there and commented out here.
%</package>
%<*discard>
%\fi
%    \begin{macrocode}
\ifdefined\childdocmain\endinput\fi
%    \end{macrocode}
%\iffalse
%</discard>
%<*package>
%\fi
%
% \macro{\ifchilddoc}
% \macro{\ifchilddocmanual}
% The conditional |\ifchilddoc| tells whether a
% child (true) or main (false) document is being compiled.
% The conditional |\ifchilddocmanual| tells whether
% the |\includeonly| mechanism is used (false) or
% the selection of child files must be performed manually (true).
% The definitions initialise to false:
%    \begin{macrocode}
\newif\ifchilddoc
\newif\ifchilddocmanual
%    \end{macrocode}

% \macro{\childdocname}
% \macro{\childdocjob}
% The macro |\childdocname| stores the name of the main document
% to be compiled. The macro |\childdocjob| stores the name of
% the document on which the \LaTeX{} compiler was originally invoked.
% The content of |\jobname| cannot be compared
% to filenames specified in the source due to different catcodes.
% The following code rescans |\jobname|, stores the result
% in |\childdocname| and saves a copy in |\childdocjob|:
%    \begin{macrocode}
\edef\childdocname{\scantokens\expandafter{\jobname\noexpand}}
\let\childdocjob\childdocname
%    \end{macrocode}

% \macro{\childdocdisable}
% The macro |\childdocdisable| prevents the main file
% from being processed more than once.
% At this stage, the main document command |\childdocmain|
% is assumed to be called once again where it should do nothing.
% Any subsequent call to it should prevent
% a secondary processing of the main document
% It overwrites the forwarding commands
% |\childdocof| and |\childdocforward|
% with empty macros to prevent further inclusions of the main document:
%    \begin{macrocode}
\newcommand{\childdocdisable}
{
  \renewcommand{\childdocmain}[1]{\renewcommand{\childdocmain}[1]{\endinput}}
  \renewcommand{\childdocof}[1]{}
  \renewcommand{\childdocby}[2][]{}
  \renewcommand{\childdocforward}[2][]{}
  \renewcommand{\childdocdisable}{}
}
%    \end{macrocode}

% \macro{\childdocmain}
% The macro |\childdocmain| is to be called at the top of the main file
% with nothing or the main filename (without extension) as argument.
% First, it breaks loops.
% If the argument is not empty and does not match |\childdocname|
% (which is set by the first inclusion of |childdoc.def|),
% |\ifchilddoc| is set to true, |\includeonly| is applied to the child file
% and |\jobname| is set to the main file
% (for proper handling of |.aux| files):
%    \begin{macrocode}
\newcommand{\childdocmain}[1]
{
  \childdocdisable\childdocmain{}
  \if?#1?\else
    \begingroup
      \def\childdoctmp{#1}
      \ifx\childdoctmp\childdocname
        \def\childdoctmp{}
      \else
        \def\childdoctmp
        {
          \childdoctrue
          \includeonly{\childdocname}
          \def\childdocjob{#1}
          \def\jobname{#1}
        }
      \fi
      \expandafter
    \endgroup
    \childdoctmp
  \fi
}
%    \end{macrocode}

% \macro{\childdocof}
% The command |\childdocof| redirects
% compilation to the main file |#1|.
%    \begin{macrocode}
\newcommand{\childdocof}[1]
{
  \childdocdisable
  \childdoctrue
  \includeonly{\childdocname}
  \def\jobname{#1}
  \def\childdocjob{#1}
  \input{#1}
}
%    \end{macrocode}

% \macro{\childdocby}
% The command |\childdocby| ....
%    \begin{macrocode}
\newcommand{\childdocby}[2][]
{
  \childdocdisable
  \childdoctrue
  \childdocmanualtrue
  \if?#1?\else
    \def\jobname{#2}
  \fi
  \def\childdocjob{#2}
  \input{#2}
  \endinput
}
%    \end{macrocode}

% \macro{\childdocforward}
% The command |\childdocforward| redirects
% compilation to the main file or
% (if the optional argument is given) a child file.
% Parameters are set as if the main file
% or a child file starting with |\childdocof| was compiled.
% Then compilation is handed over to the main file:
%    \begin{macrocode}
\newcommand{\childdocforward}[2][]
{
  \begingroup
    \if?#1?
      \def\childdoctmp
      {
        \def\childdocname{#2}
        \def\childdocjob{#2}
        \def\jobname{#2}
        \input{#2}
        \endinput
      }
    \else
      \def\childdoctmp
      {
        \childdocdisable
        \def\childdocname{#2}
        \childdoctrue
        \includeonly{#2}
        \def\childdocjob{#1}
        \def\jobname{#1}
        \input{#1}
        \endinput
      }
    \fi
    \expandafter
  \endgroup
  \childdoctmp
}
%    \end{macrocode}

% \macro{\childdocforwardprefix}
% The command |\childdocforwardprefix| redirects
% compilation to the main or a child file by means of a pattern.
% The prefix |#1| in the current filename is replaced by |#2|
% and the suffix of the current filename is kept
% (it is assumed that the filename does not contain the substring `|~~~|'
% which is used as a delimiter).
% Compilation is handed over to the new file by |\childdocforward|:
%    \begin{macrocode}
\newcommand{\childdocforwardprefix}[3][]
{
  \begingroup
    \def\childdocextract #2##1~~~{\def\childdoctmp{\childdocforward[#1]{#3##1}}}
    \expandafter\childdocextract\childdocname~~~
    \expandafter
  \endgroup
  \childdoctmp
}
%    \end{macrocode}

% \macro{\childdoc}
% The deprecated macro |\childdoc| is a legacy version of |\childdocmain|:
%    \begin{macrocode}
\newcommand{\childdoc}{\childdocmain}
%    \end{macrocode}

% \macro{\childdocredirect}
% The deprecated macro |\childdocredirect| is a legacy version
% of |\childdocforward| and |\childdocforwardprefix|:
%    \begin{macrocode}
\newcommand{\childdocredirect}[2][]
{
  \begingroup
    \if?#1?
      \def\childdoctmp{\childdocforward{#2}}
    \else
      \def\childdoctmp{\childdocforwardprefix{#1}{#2}}
    \fi
    \expandafter
  \endgroup
  \childdoctmp
}
%    \end{macrocode}

%\iffalse
%</package>
%\fi
%
\endinput
|\\
|\childdocforward{|\textit{main}|}|
\end{tabular}
\end{center}
%
Likewise, the following files |final|\textit{nn}|.tex|
compile the final version of the child document
|child|\textit{nn}|.tex|:
%
\begin{center}
\begin{tabular}{l}
|\def\version{final}|\\
|% \iffalse
%
% childdoc.dtx Copyright (C) 2017-2018 Niklas Beisert
%
% This work may be distributed and/or modified under the
% conditions of the LaTeX Project Public License, either version 1.3
% of this license or (at your option) any later version.
% The latest version of this license is in
%   http://www.latex-project.org/lppl.txt
% and version 1.3 or later is part of all distributions of LaTeX
% version 2005/12/01 or later.
%
% This work has the LPPL maintenance status `maintained'.
%
% The Current Maintainer of this work is Niklas Beisert.
%
% This work consists of the files childdoc.dtx and childdoc.ins
% and the derived files childdoc.def and cdocsamp.tex with
% cdocsch1.tex, cdocsch2.tex, cdocsdrf.tex, cdocsfn1.tex, cdocsfn2.tex.
%
%<package>\ifdefined\childdocmain\endinput\fi
%<package>\ProvidesFile{childdoc.def}[2018/12/30 v2.0 child document driver]
%<samplemain>\ProvidesFile{cdocsamp.tex}[2018/12/30 v2.0 sample for childdoc]
%<*driver>
%\ProvidesFile{childdoc.drv}[2018/12/30 v2.0 childdoc reference manual file]
\PassOptionsToClass{10pt,a4paper}{article}
\documentclass{ltxdoc}

\usepackage[margin=35mm]{geometry}
\usepackage{hyperref}
\usepackage{hyperxmp}
\usepackage[usenames]{color}

\hypersetup{colorlinks=true}
\hypersetup{pdfstartview=FitH}
\hypersetup{pdfpagemode=UseNone}
\hypersetup{pdfsource={}}
\hypersetup{pdflang={en-UK}}
\hypersetup{pdfcopyright={Copyright 2017-2018 Niklas Beisert.
  This work may be distributed and/or modified under the
  conditions of the LaTeX Project Public License, either version 1.3
  of this license or (at your option) any later version.}}
\hypersetup{pdflicenseurl={http://www.latex-project.org/lppl.txt}}
\hypersetup{pdfcontactaddress={ETH Zurich, ITP, HIT K,
  Wolfgang-Pauli-Strasse 27}}
\hypersetup{pdfcontactpostcode={8093}}
\hypersetup{pdfcontactcity={Zurich}}
\hypersetup{pdfcontactcountry={Switzerland}}
\hypersetup{pdfcontactemail={nbeisert@itp.phys.ethz.ch}}
\hypersetup{pdfcontacturl={http://people.phys.ethz.ch/\xmptilde nbeisert/}}

\newcommand{\secref}[1]{\hyperref[#1]{section \ref*{#1}}}

\parskip1ex
\parindent0pt
\let\olditemize\itemize
\def\itemize{\olditemize\parskip0pt}

\begin{document}

\title{The \textsf{childdoc} Package}
\hypersetup{pdftitle={The childdoc Package}}
\author{Niklas Beisert\\[2ex]
  Institut f\"ur Theoretische Physik\\
  Eidgen\"ossische Technische Hochschule Z\"urich\\
  Wolfgang-Pauli-Strasse 27, 8093 Z\"urich, Switzerland\\[1ex]
  \href{mailto:nbeisert@itp.phys.ethz.ch}
  {\texttt{nbeisert@itp.phys.ethz.ch}}}
\hypersetup{pdfauthor={Niklas Beisert}}
\hypersetup{pdfsubject={Manual for the LaTeX2e Package childdoc}}
\date{30 December 2018, \textsf{v2.0}}
\maketitle

\begin{abstract}\noindent
\textsf{childdoc} is a \LaTeXe{} package
that enables the direct compilation
of document sections included by |\include|
to individual files.
\end{abstract}

\begingroup
\parskip0ex
\tableofcontents
\endgroup

%%%%%%%%%%%%%%%%%%%%%%%%%%%%%%%%%%%%%%%%%%%%%%%%%%%%%%%%%%%%%%%%%%%%%%%%%%%%%%%%
%%%%%%%%%%%%%%%%%%%%%%%%%%%%%%%%%%%%%%%%%%%%%%%%%%%%%%%%%%%%%%%%%%%%%%%%%%%%%%%%
\section{Introduction}

\LaTeX{} provides a mechanism to structure a large document (such as a book)
into a main file and several child files (containing the chapters)
using the |\include| command.
This mechanism is beneficial for documents
which span hundreds of pages in order to
make the source file(s) more manageable.
Moreover, compilation can be restricted to
selected child files by means of the |\includeonly| command.
The latter feature can be used to reduce the compilation time while editing
(this was significantly more useful in the earlier days of \LaTeX{})
or to generate a smaller document which is easier to navigate.
Another application of |\includeonly| is to generate
documents consisting of selected parts of the complete document.

However, there are a few drawbacks of the plain |\include| mechanism:
\begin{itemize}
\item
The child files cannot be compiled on their own,
they can only be compiled via the main file.
A naive editing environment
(such as a text editor with an option
to have the current file processed by \LaTeX)
may require one to switch to the main file before compiling;
attempting to compile the child file produces errors.
\item
The main file must be modified (each time)
to adjust the |\includeonly| command
to the present needs. This easily leaves the main file in a messy state.
\item
The generated document will always carry the filename
of the main document. This is inconvenient if
several child files are to be compiled and
to be kept for distribution.
\end{itemize}

The present package provides a simple interface
to make child files individually compilable by \LaTeX{}.
Compiling a child file then has the same effect as compiling
the main file with an |\includeonly| command
to select the appropriate child.
Moreover the generated document will carry the name of the child
rather than the main file.
This resolves all three above issues.

This feature is meant to make the editing of books,
thesis documents and lecture notes somewhat more convenient.
However, the package can also be used efficiently for
composing a series of documents (such as exercise sheets)
which are typically distributed individually.
It then assists the author in generating the individual documents
(potentially in different versions)
as well as a document containing the collected series.
Another application is in developing style files
or other kinds of included material
where compilation of the style file could redirect
to a sample or test file.

%%%%%%%%%%%%%%%%%%%%%%%%%%%%%%%%%%%%%%%%%%%%%%%%%%%%%%%%%%%%%%%%%%%%%%%%%%%%%%%%
%%%%%%%%%%%%%%%%%%%%%%%%%%%%%%%%%%%%%%%%%%%%%%%%%%%%%%%%%%%%%%%%%%%%%%%%%%%%%%%%
\section{Usage}

First of all, the package \textsf{childdoc} is \emph{not} a standard
\LaTeXe{} |.sty| style file! Therefore it needs to be invoked in
a non-standard way.

%%%%%%%%%%%%%%%%%%%%%%%%%%%%%%%%%%%%%%%%%%%%%%%%%%%%%%%%%%%%%%%%%%%%%%%%%%%%%%%%
\subsection{Included Files}
\label{sec:include}

%%%%%%%%%%%%%%%%%%%%%%%%%%%%%%%%%%%%%%%%
\DescribeMacro{\childdocmain}
To use the package, add the commands
\begin{center}
\begin{tabular}{l}
|\input{childdoc.def}|\\
|\childdocmain{}|\\
\end{tabular}
\end{center}
at the very top of the main \LaTeX{} file,
in particular \emph{before} the |\documentclass| statement!
The argument of |\childdocmain| should be left empty
(but it must be present).

%%%%%%%%%%%%%%%%%%%%%%%%%%%%%%%%%%%%%%%%
\DescribeMacro{\childdocof}
Furthermore, add the commands
\begin{center}
\begin{tabular}{l}
|\input{childdoc.def}|\\
|\childdocof{|\textit{main}|}|\\
\end{tabular}
\end{center}
at the top of every child file \textit{child}
which is included by |\include{|\textit{child}|}|
from within the main file
(or at least for those files to be compiled individually).
The argument \textit{main} must be the filename of the main file.

There are a couple of
considerations in setting up the main and child documents:

%%%%%%%%%%%%%%%%%%%%%%%%%%%%%%%%%%%%%%%%
\paragraph{Restrictions.}

Please note the following restrictions:
\begin{itemize}
\item
|\childdocmain| must be called with one argument \textit{main}
to ensure compatibility with earlier version of the package.
It must either be empty (|\childdocmain{}|)
or precisely match the filename of the main file in which it is specified.
See \secref{sec:detection} for further information.
\item
The filename \textit{main} must be specified without the |.tex| extension.
\item
The filename \textit{main} is case sensitive
(even in case-insensitive file systems)
due to internal string comparison.
\item
The argument \textit{main} should be fully expanded, it cannot be a macro.
\item
Subdirectories and special characters should be avoided in filenames.
\item
The command |\childdocmain{|\textit{main}|}| must be followed by a whitespace.
It should not be followed immediately by another command
or by a comment mark `|%|'.
This is because the \TeX{} parser reads the token immediately following
the argument of |\childdocmain| and puts it
at the beginning of every child section;
however, a white\-space is ignored.
\end{itemize}

%%%%%%%%%%%%%%%%%%%%%%%%%%%%%%%%%%%%%%%%
\paragraph{Content of Main File.}

It is advisable to place all content in the child files included by |\include|.
Any output contained in the main file will appear in all child documents
unless suppressed manually;
it cannot be suppressed automatically by the |\includeonly| directive
and thus should normally be avoided.
A method to include some content in the main file
by means of conditional processing is described in \secref{sec:conditional}.

%%%%%%%%%%%%%%%%%%%%%%%%%%%%%%%%%%%%%%%%
\paragraph{Page Numbering.}

When only a part of the document is compiled,
the appropriate numbering of pages
(as well as other status parameters)
is determined from the |.aux| files.
The latter contain information from previous passes.
However this information needs to propagate through
all intermediate child documents.
Therefore the page numbering in child documents may well
be inconsistent until the complete document is compiled at least once.

A useful (if unconventional) way to always ensure a consistent
page numbering is to restart the numbering in each child document
and denote the pages by `\textit{child}|.|\textit{page}'
where \textit{child} represents the chapter/section number of the child file.
This can be achieved by the command
|\numberwithin{page}{|\textit{child}|}|
of the \textsf{amsmath} package
where \textit{child} can be |chapter| or |section|
depending on the chosen structuring.
Alternatively, one can modify the macro |\thepage| appropriately
and reset the counter |page| at the start of each child file.

%%%%%%%%%%%%%%%%%%%%%%%%%%%%%%%%%%%%%%%%%%%%%%%%%%%%%%%%%%%%%%%%%%%%%%%%%%%%%%%%
\subsection{Conditional Processing}
\label{sec:conditional}

The package provides a mechanism to compile different versions
of a document. To customise the versions further some conditional processing
can come in handy to distinguish which version is being compiled.
The package provides two macros to describe the compilation context:

%%%%%%%%%%%%%%%%%%%%%%%%%%%%%%%%%%%%%%%%
\DescribeMacro{\ifchilddoc}
The conditional |\ifchilddoc| distinguishes between the compilation of
child documents and the main document:
%
\begin{center}
|\ifchilddoc |\textit{child-code}| |[|\||else |\textit{main-code}]| \||fi|
\end{center}

%%%%%%%%%%%%%%%%%%%%%%%%%%%%%%%%%%%%%%%%
\DescribeMacro{\childdocname}
\DescribeMacro{\childdocjob}
The macro |\childdocname| contains the filename (without extension)
of the main or child file being processed.
Note that |\childdocjob| will always contain the name of the main file.

%%%%%%%%%%%%%%%%%%%%%%%%%%%%%%%%%%%%%%%%
\paragraph{Title Page.}

Conditional processing can be used to include a title or banner page
in the main document when proper precautions are taken.
Importantly, the code in the main file should ensure that the page counter
(as well as other status parameters which are stored in the |.aux| files)
takes the same value after the conditional processing.
Otherwise the page numbers may take divergent values
depending on which part is compiled.

For example, a title page could be declared by:
%
\begin{center}
\begin{tabular}{l}
|\ifchilddoc\||else|\\
|\addtocounter{page}{-1}|\\
\textit{code for title page}\\
|\newpage|\\
|\||fi|
\end{tabular}
\end{center}
%
A banner page for the child documents can be generated by:
%
\begin{center}
\begin{tabular}{l}
|\ifchilddoc|\\
|\addtocounter{page}{-1}|\\
\textit{code for banner page}\\
|\newpage|\\
|\||fi|
\end{tabular}
\end{center}
%
Here one could write a message such as:
\begin{center}
|This is the part \childdocname{} of \childdocjob{}.|
\end{center}

%%%%%%%%%%%%%%%%%%%%%%%%%%%%%%%%%%%%%%%%%%%%%%%%%%%%%%%%%%%%%%%%%%%%%%%%%%%%%%%%
\subsection{Flags}
\label{sec:flags}

The package makes it easy to generate different versions
of the main or child documents.
To this end compilation flags can be defined
and assigned different default values.
They will be particularly useful in conjunction
with the forwarding mechanism described in \secref{sec:forward}.

For example, it may be useful to have a flag |\version|
which can be set to |draft| or |final|.
The document source will contain some conditional code
depending on the value of |\version|.
Suppose further, the flag should default to |final| for the main file
and to |draft| for child files
which is a natural assignment for editing the document.
This is achieved by placing the following code
in the preamble of the main document
(below the |\childdocmain| directive):
%
\begin{center}
\begin{tabular}{l}
|\ifchilddoc|\\
|\providecommand{\version}{draft}|\\
|\||else|\\
|\providecommand{\version}{final}|\\
|\||fi|
\end{tabular}
\end{center}
%
The definition by |\providecommand| makes sure
that previous definitions are not overwritten.
Further statements |\providecommand{\version}{...}|
can thus be added before the above code to override it.

For the main file, one might add a line
(between |\childdocmain| and the above block)
%
\begin{center}
|%\ifchilddoc\||else\providecommand{\version}{draft}\||fi|
\end{center}
%
which can be uncommented to produce a draft version.
Likewise one can add a line to the very top of a child file
(above the |\childdocof{|\textit{main}|}| directive)
%
\begin{center}
|%\providecommand{\version}{final}|
\end{center}
%
which can be uncommented to produce the final version of this child document.

%%%%%%%%%%%%%%%%%%%%%%%%%%%%%%%%%%%%%%%%%%%%%%%%%%%%%%%%%%%%%%%%%%%%%%%%%%%%%%%%
\subsection{Forwarding}
\label{sec:forward}

Different versions of the main or child documents
using compilation flags as described in \secref{sec:flags}
can be (permanently) stored in different files
for convenient compilation, viewing and distribution.
To this end, the package defines a command
to pass on compilation to a different file:

%%%%%%%%%%%%%%%%%%%%%%%%%%%%%%%%%%%%%%%%
\DescribeMacro{\childdocforward}
The command |\childdocforward| redirects processing to
another source file:
%
\begin{center}
\begin{tabular}{l}
|\input{childdoc.def}|\\
|\childdocforward[|\textit{main}|]{|\textit{dest}|}|\\
\end{tabular}
\end{center}
%
The argument \textit{dest} is the destination file
(without extension).
It should be the main file or one of the child files.
Note that further \textsf{childdoc} directives
such as |\childdocof| and |\childdocforward|
in the indicated file will be processed in this form.
The optional argument \textit{main}
passes on directly to the main file \textit{main}
while pretending to compile the child \textit{dest}.
This form behaves as if \textit{dest}
issues |\childdocof{|\textit{main}|}| right away,
and no further \textsf{childdoc} directives will be processed.

%%%%%%%%%%%%%%%%%%%%%%%%%%%%%%%%%%%%%%%%
\DescribeMacro{\...prefix}
In the alternative form |\childdocforwardprefix|,
%
\begin{center}
\begin{tabular}{l}
|\input{childdoc.def}|\\
|\childdocforwardprefix[|\textit{main}|]{|\textit{prefix}|}{|\textit{dest}|}|
\end{tabular}
\end{center}
%
the destination file is determined by a pattern
depending on the current file:
To make this work, the current file must be called
`{\textit{prefix}\hspace{0.2em}\textit{suffix}}'
with \textit{prefix} matching precisely the argument.
Processing is then passed on to the file
`{\textit{dest}\hspace{0.2em}\textit{suffix}}'.
Surely, the same effect is achieved by
directly specifying the
argument `{\textit{dest}\hspace{0.2em}\textit{suffix}}'
in the first form.
However, that requires to set up a different file
for each child. With the alternative form of the command
all these files can have exactly the same content
which simplifies setting them up and maintaining them.

For example, the following file |draft.tex|
with a compilation flag |\version| as described in \secref{sec:flags}
compiles the main document as a draft:
%
\begin{center}
\begin{tabular}{l}
|\def\version{draft}|\\
|\input{childdoc.def}|\\
|\childdocforward{|\textit{main}|}|
\end{tabular}
\end{center}
%
Likewise, the following files |final|\textit{nn}|.tex|
compile the final version of the child document
|child|\textit{nn}|.tex|:
%
\begin{center}
\begin{tabular}{l}
|\def\version{final}|\\
|\input{childdoc.def}|\\
|\childdocforwardprefix{final}{child}|
\end{tabular}
\end{center}
%

Note that when several versions of a main file and/or of each child file
are to be generated, it may be convenient to set up a |Makefile| or
shell script to automatise the process.

%%%%%%%%%%%%%%%%%%%%%%%%%%%%%%%%%%%%%%%%%%%%%%%%%%%%%%%%%%%%%%%%%%%%%%%%%%%%%%%%
\subsection{Command Line Processing}
\label{sec:commandline}

The effect of redirection files can also be achieved by invoking
the \LaTeX{} compiler with a more elaborate command line.
Most conveniently this should be done as part
of a shell script or a |Makefile|.

When using \textsf{childdoc} in the main file, the following
command lines effectively perform a redirection
(note that depending on the shell being used,
backslashes may have to be doubled: `|\|' $\to$ `|\\|'):
%
\begin{center}
|... -jobname "|\textit{target}|" |\\|"|[\textit{flags}]%
|\input{childdoc.def}\childdocforward[|\textit{main}|]{|\textit{dest}|}"|
\end{center}
%
Here \textit{target} is the name of the output file,
\textit{main} is the name of the main file
and \textit{dest} is the name of the main or child file to be processed
(all filenames without extensions).
The optional argument \textit{main} can be omitted
if \textit{main} matches \textit{dest}.
Optionally, compilation \textit{flags} can be defined via |\def| commands.
This command line makes the \TeX{} engine believe
it is compiling the file \textit{target}
whose content is specified as the latter parameter.
The provided code then forwards the processing to
\textit{main} or \textit{dest} as described in \secref{sec:forward}.

%%%%%%%%%%%%%%%%%%%%%%%%%%%%%%%%%%%%%%%%%%%%%%%%%%%%%%%%%%%%%%%%%%%%%%%%%%%%%%%%
\subsection{Include by Input}
\label{sec:input}

Including child documents by |\include| has some restrictions by design.
Most notably, the content of a child document always occupies
its own set of pages; pages cannot be shared between child documents.
Usually, this behaviour makes perfect sense
because each child document contain an essential part of the document.
However, in some situations it may be desirable to compose
a document from a collection of parts
without having mandatory page breaks between then.
For this case, the package
provides a mechanism to include parts
by |\input| which can also be processed individually.
However, by construction this mechanism
requires manual handling of the content to be output.

%%%%%%%%%%%%%%%%%%%%%%%%%%%%%%%%%%%%%%%%
\DescribeMacro{\ifchilddocmanual}
The main file should be prepared as usual, see \secref{sec:include}.
However, the document body must make a distinction
between processing of an individual part and of the main document, e.g.:
%
\begin{center}
\begin{tabular}{l}
|\ifchilddocmanual|\\
|\input{\childdocname}|\\
|\||else|\\
\textit{document body with }|\input{|\textit{part}|}|\\
|\||fi|
\end{tabular}
\end{center}
%
The conditional |\ifchilddocmanual| is true whenever
a part to be included by |\input| is being compiled,
and the name of the part is stored in |\childdocname|.

%%%%%%%%%%%%%%%%%%%%%%%%%%%%%%%%%%%%%%%%
\DescribeMacro{\childdocby}
Each part to be included by |\input| should start with:
%
\begin{center}
\begin{tabular}{l}
|\input{childdoc.def}|\\
|\childdocby{|\textit{main}|}|\\
\end{tabular}
\end{center}
%
The directive |\childdocby| is similar to |\childdocof|
described in \secref{sec:include},
but the subsequent selection of content must be done manually.
To that end, both |\ifchilddoc| and |\ifchilddocmanual|
will be true upon processing of a part,
and the name of the part is stored in |\childdocname|.
Note that |\jobname| will be set to the filename of the current part
so that each part receives an individual |.aux| file
that does not interfere with the |.aux| file(s) of the main document.
This behaviour can be altered by the alternative form
|\childdocby[*]{|\textit{main}|}| (with a non-empty optional argument)
which uses the |.aux| file of the main document
by setting |\jobname| to \textit{main}.

%%%%%%%%%%%%%%%%%%%%%%%%%%%%%%%%%%%%%%%%%%%%%%%%%%%%%%%%%%%%%%%%%%%%%%%%%%%%%%%%
\subsection{Driver Development}
\label{sec:driver}

The \textsf{childdoc} mechanism can also be use for the development
of definition files such as \LaTeX{} styles or classes.
This case differs from the above setup with multiple parts
included by |\include| in that no |\includeonly| should be invoked.
This can be achieved by starting the include file
(before |\ProvidesPackage|) with:
%
\begin{center}
\begin{tabular}{l}
|\input{childdoc.def}|\\
|\childdocforward{|\textit{main}|}|\\
\end{tabular}
\end{center}
%
or alternatively with:
%
\begin{center}
\begin{tabular}{l}
|\input{childdoc.def}|\\
|\childdocby{|\textit{main}|}|\\
\end{tabular}
\end{center}
%
Both forms have slightly different effects as described above.
The main file is prepared as usual, see \secref{sec:include}.

%%%%%%%%%%%%%%%%%%%%%%%%%%%%%%%%%%%%%%%%%%%%%%%%%%%%%%%%%%%%%%%%%%%%%%%%%%%%%%%%
\subsection{Legacy Detection}
\label{sec:detection}

The directive |\childdocmain| in the main file can detect
whether the complete document or merely a child is to be compiled
even without using the directive |\childdocof|.
This method is deprecated because it is less robust
and there is no compelling reason to use it;
it is merely provided for backward compatibility
and it may be removed in future versions.

If the detection mechanism is to be used,
it is mandatory to correctly specify
the filename of the main file as the argument of |\childdocmain|:
%
\begin{center}
\begin{tabular}{l}
|\input{childdoc.def}|\\
|\childdocmain{|\textit{main}|}|\\
\end{tabular}
\end{center}
%
If |\jobname| does not match the argument \textit{main} of |\childdocmain|,
it is assumed that |\jobname| points to the child file to be compiled.
When using |\childdocmain| with the main file specified as argument,
it suffices to start a child file
with just |\input{|\textit{main}|}|
without loading of the package and using |\childdocof|.
If instead all processing is done
with the appropriate \textsf{childdoc} directives,
the argument of \textit{main} of |\childdocmain| can be empty.

An alternative version of the command line processing described
in \secref{sec:commandline} using the detection mechanism reads:
%
\begin{center}
|... -jobname "|\textit{target}|" "|[\textit{flags}]%
[|\def\jobname{|\textit{dest}|}|]|\input{|\textit{main}|}"|
\end{center}

%%%%%%%%%%%%%%%%%%%%%%%%%%%%%%%%%%%%%%%%%%%%%%%%%%%%%%%%%%%%%%%%%%%%%%%%%%%%%%%%
\subsection{Manual Code}
\label{sec:manual}

In case one cannot be certain whether the definitions file |childdoc.def|
is installed on the target \TeX{} distribution
and one prefers not to ship it,
it is conceivable to paste a few relevant commands into the sources.

To that end, drop all statements |\input{childdoc.def}|
and perform the replacements as outlined below.
Instead of |\childdocmain{|\textit{main}|}| add the following code
to the top of the main file:
%
\begin{center}
\begin{tabular}{l}
|\||ifdefined\childdocname\endinput\||fi\newif\ifchilddoc|\\
|\edef\childdocname{\scantokens\expandafter{\jobname\noexpand}}|\\
|\def\childdocmain{|\textit{main}|}\||ifx\childdocmain\childdocname\||else|\\
|\childdoctrue\includeonly{\childdocname}\let\jobname\childdocmain\||fi|\\
\end{tabular}
\end{center}
%
Instead of |\childdocof{|\textit{main}|}| just include the main file
at the top of each child file:
%
\begin{center}
|\input{|\textit{main}|}|
\end{center}
%
A simple redirection |\childdocforward{|\textit{dest}|}| is achieved by:
%
\begin{center}
|\def\jobname{|\textit{dest}|}\input{\jobname}|
\end{center}
%
The redirection with prefix
|\childdocforwardprefix[|\textit{prefix}|]{|\textit{dest}|}|
is accomplished by:
%
\begin{center}
\begin{tabular}{l}
|{\edef\jobname{\scantokens\expandafter{\jobname\noexpand}}|\\
|\def\redirectjob |\textit{prefix}|#1~~~{\gdef\jobname{|\textit{dest}|#1}}|\\
|\expandafter\redirectjob\jobname~~~}\input{\jobname}|
\end{tabular}
\end{center}

In an alternative approach,
child documents can be compiled by a specific command line
without additional code or specific definitions:
%
\begin{center}
|... -jobname "|\textit{target}|" "|[\textit{flags}]%
|\includeonly{|\textit{dest}|}\input{|\textit{main}|}"|
\end{center}
%

%%%%%%%%%%%%%%%%%%%%%%%%%%%%%%%%%%%%%%%%%%%%%%%%%%%%%%%%%%%%%%%%%%%%%%%%%%%%%%%%
%%%%%%%%%%%%%%%%%%%%%%%%%%%%%%%%%%%%%%%%%%%%%%%%%%%%%%%%%%%%%%%%%%%%%%%%%%%%%%%%
\section{Information}

%%%%%%%%%%%%%%%%%%%%%%%%%%%%%%%%%%%%%%%%%%%%%%%%%%%%%%%%%%%%%%%%%%%%%%%%%%%%%%%%
\subsection{Copyright}

Copyright \copyright{} 2017--2018 Niklas Beisert

This work may be distributed and/or modified under the
conditions of the \LaTeX{} Project Public License, either version 1.3
of this license or (at your option) any later version.
The latest version of this license is in
  \url{http://www.latex-project.org/lppl.txt}
and version 1.3 or later is part of all distributions of \LaTeX{}
version 2005/12/01 or later.

This work has the LPPL maintenance status `maintained'.

The Current Maintainer of this work is Niklas Beisert.

This work consists of the files |README.txt|, |childdoc.ins| and |childdoc.dtx|
as well as the derived files |childdoc.def|, |cdocsamp.tex|
with |cdocsch1.tex|, |cdocsch2.tex|, |cdocspt3.tex|, |cdocspt4.tex|,
|cdocsdrf.tex|, |cdocsfn1.tex|, |cdocsfn2.tex|
as well as |childdoc.pdf|.

%%%%%%%%%%%%%%%%%%%%%%%%%%%%%%%%%%%%%%%%%%%%%%%%%%%%%%%%%%%%%%%%%%%%%%%%%%%%%%%%
\subsection{Files and Installation}

The package consists of the files:
%
\begin{center}
\begin{tabular}{ll}
    |README.txt|   & readme file \\
    |childdoc.ins| & installation file \\
    |childdoc.dtx| & source file \\
    |childdoc.def| & definition file \\
    |cdocsamp.tex| & sample main file \\
    |cdocsch1.tex| & sample include file \\
    |cdocsch2.tex| & sample include file \\
    |cdocspt3.tex| & sample part file \\
    |cdocspt4.tex| & sample part file \\
    |cdocsdrf.tex| & sample redirection file \\
    |cdocsfn1.tex| & sample redirection file \\
    |cdocsfn2.tex| & sample redirection file \\
    |childdoc.pdf| & manual
\end{tabular}
\end{center}
%
The distribution consists of the files
|README.txt|, |childdoc.ins| and |childdoc.dtx|.
%
\begin{itemize}
\item
Run (pdf)\LaTeX{} on |childdoc.dtx|
to compile the manual |childdoc.pdf| (this file).
\item
Run \LaTeX{} on |childdoc.ins| to create the definitions file |childdoc.def|
and the sample |cdocsamp.tex| with include files
|cdocsch1.tex|, |cdocsch2.tex|, |cdocspt3.tex|, |cdocspt4.tex|,
|cdocsdrf.tex|, |cdocsfn1.tex|, |cdocsfn2.tex|.
Then copy the file |childdoc.def| to an appropriate directory of your \LaTeX{}
distribution, e.g.\ \textit{texmf-root}|/tex/latex/childdoc|.
\end{itemize}

%%%%%%%%%%%%%%%%%%%%%%%%%%%%%%%%%%%%%%%%%%%%%%%%%%%%%%%%%%%%%%%%%%%%%%%%%%%%%%%%
\subsection{Related CTAN Packages}

There are several other packages which offer a similar functionality:
%
\begin{itemize}
\item
The packages
\href{http://ctan.org/pkg/docmute}{\textsf{docmute}},
\href{http://ctan.org/pkg/includex}{\textsf{includex}} and
\href{http://ctan.org/pkg/standalone}{\textsf{standalone}}
provide commands to include only the document body of
a child file thus allowing both files to be compiled individually.
\item
The packages \href{http://ctan.org/pkg/subdocs}{\textsf{subdocs}}
and \href{http://ctan.org/pkg/subfiles}{\textsf{subfiles}}
provide structures in which the main and child documents can be
encapsulated and allowing them to be compiled individually.
The inclusion mechanism is different from the conventional |\include|.
\item
The package \href{http://ctan.org/pkg/combine}{\textsf{combine}}
is an elaborate solution to combine several documents into one.
\end{itemize}
%
See also the CTAN topic \href{http://ctan.org/topic/subdocs}{\textsf{subdocs}}
for further related packages.
The present package differs from the above solutions in that
a document structure constructed with the conventional |\include| mechanism
just needs two extra commands at the top of every file
such that all constituent files can be compiled individually.

%%%%%%%%%%%%%%%%%%%%%%%%%%%%%%%%%%%%%%%%%%%%%%%%%%%%%%%%%%%%%%%%%%%%%%%%%%%%%%%%
%\subsection{Feature Suggestions}
%
%The following is a list of features which may be useful for future
%versions of this package:
%%
%\begin{itemize}
%\item
%\ldots
%\end{itemize}

%%%%%%%%%%%%%%%%%%%%%%%%%%%%%%%%%%%%%%%%%%%%%%%%%%%%%%%%%%%%%%%%%%%%%%%%%%%%%%%%
\subsection{Revision History}

%%%%%%%%%%%%%%%%%%%%%%%%%%%%%%%%%%%%%%%%
\paragraph{v2.0:} 2018/12/30

\begin{itemize}
\item
immediate forward processing
\item
added |\childdocby| mechanism
\item
manual restructured
\end{itemize}

%%%%%%%%%%%%%%%%%%%%%%%%%%%%%%%%%%%%%%%%
\paragraph{v1.6:} 2018/01/17

\begin{itemize}
\item
application for development of include files
\item
corrections to manual
\end{itemize}

%%%%%%%%%%%%%%%%%%%%%%%%%%%%%%%%%%%%%%%%
\paragraph{v1.5:} 2017/05/21

\begin{itemize}
\item
more complete structuring introduced
\item
|\childdocof| introduced
\item
|\childdoc| renamed to |\childdocmain|
\item
|\childredirect| renamed to |\childdocforward| and |\childdocforwardprefix|
and functionality expanded
\end{itemize}

%%%%%%%%%%%%%%%%%%%%%%%%%%%%%%%%%%%%%%%%
\paragraph{v1.0:} 2017/04/27

\begin{itemize}
\item
manual and install package
\item
first version published on CTAN
\end{itemize}

%%%%%%%%%%%%%%%%%%%%%%%%%%%%%%%%%%%%%%%%
\paragraph{v0.6:} 2017/04/26

\begin{itemize}
\item
redirection mechanism added
\end{itemize}

%%%%%%%%%%%%%%%%%%%%%%%%%%%%%%%%%%%%%%%%
\paragraph{v0.5:} 2017/04/26

\begin{itemize}
\item
functionality in definition file
\end{itemize}


%%%%%%%%%%%%%%%%%%%%%%%%%%%%%%%%%%%%%%%%%%%%%%%%%%%%%%%%%%%%%%%%%%%%%%%%%%%%%%%%
%%%%%%%%%%%%%%%%%%%%%%%%%%%%%%%%%%%%%%%%%%%%%%%%%%%%%%%%%%%%%%%%%%%%%%%%%%%%%%%%
%%%%%%%%%%%%%%%%%%%%%%%%%%%%%%%%%%%%%%%%%%%%%%%%%%%%%%%%%%%%%%%%%%%%%%%%%%%%%%%%
\appendix

\settowidth\MacroIndent{\rmfamily\scriptsize 000\ }

 \DocInput{childdoc.dtx}

\end{document}
%</driver>
% \fi
%
% %%%%%%%%%%%%%%%%%%%%%%%%%%%%%%%%%%%%%%%%%%%%%%%%%%%%%%%%%%%%%%%%%%%%%%%%%%%%%%
% %%%%%%%%%%%%%%%%%%%%%%%%%%%%%%%%%%%%%%%%%%%%%%%%%%%%%%%%%%%%%%%%%%%%%%%%%%%%%%
% \section{Sample}
%\iffalse
%<*samplemain>
%\fi
%
% The following presents a sample document
% with two chapters, two parts, a title page,
% a compile flag as well as three forwarding files to set the flag.
% It consists of eight |.tex| files:
% \begin{center}
% \begin{tabular}{ll}
% |cdocsamp.tex|&main file\\
% |cdocsch1.tex|&include file for chapter 1\\
% |cdocsch2.tex|&include file for chapter 2\\
% |cdocspt3.tex|&include file for part 3\\
% |cdocspt4.tex|&include file for part 4\\
% |cdocsdrf.tex|&forwarding file for main file in draft mode\\
% |cdocsfi1.tex|&forwarding file for final version of chapter 1\\
% |cdocsfi2.tex|&forwarding file for final version of chapter 2\\
% \end{tabular}
% \end{center}
% Each of the eight files can be compiled directly by the \LaTeX{} compiler.
%
% %%%%%%%%%%%%%%%%%%%%%%%%%%%%%%%%%%%%%%
% \paragraph{Main File.}
%
% The main file is called |cdocsamp.tex|.
%
% Load the \textsf{childdoc} definitions and
% declare the filename for the main document:
%    \begin{macrocode}
\input{childdoc.def}
\childdocmain{}
%    \end{macrocode}

% Optional override for |\version| flag:
%    \begin{macrocode}
%%\ifchilddoc\else\providecommand{\version}{draft}\fi
%    \end{macrocode}

% Define the default values for the |\version| flag
% (|final| for the main file and |draft| for childs):
%    \begin{macrocode}
\ifchilddoc
\providecommand{\version}{draft}
\else
\providecommand{\version}{final}
\fi
%    \end{macrocode}

% Load the standard document class:
%    \begin{macrocode}
\documentclass[12pt]{article}
%    \end{macrocode}

% Start the document body:
%    \begin{macrocode}
\begin{document}
%    \end{macrocode}

% Declare a title page.
% Print title, part of document being processed and version flag:
%    \begin{macrocode}
\addtocounter{page}{-1}
\begin{center}
{\LARGE\bfseries{}childdoc example\par}
\vspace{1cm}
\ifchilddoc
\ifchilddocmanual part\else chapter\fi:
`\childdocname' of `\childdocjob'\par
\else
main document: `\childdocjob'\par
\fi
version: \version\par
\end{center}
\newpage
%    \end{macrocode}

% Manually include selected file,
% otherwise process as usual:
%    \begin{macrocode}
\ifchilddocmanual
\section*{part `\childdocname'}
\input{\childdocname}
\else
%    \end{macrocode}

% Include the two chapters:
%    \begin{macrocode}
\include{cdocsch1}
\include{cdocsch2}
%    \end{macrocode}

% Include the two parts unless only chapters should be displayed:
%    \begin{macrocode}
\ifchilddoc\else
\section{part three}
\input{cdocspt3}
\section{part four}
\input{cdocspt4}
\fi
%    \end{macrocode}

% Process as usual until here:
%    \begin{macrocode}
\fi
%    \end{macrocode}

% End of document body:
%    \begin{macrocode}
\end{document}
%    \end{macrocode}
%\iffalse
%</samplemain>
%\fi
%
% %%%%%%%%%%%%%%%%%%%%%%%%%%%%%%%%%%%%%%
% \paragraph{Chapter Include Files.}
%
% The include files are called |cdocsch1.tex| and |cdocsch2.tex|.
%
%\iffalse
%<*samplechap1|samplechap2>
%\fi

% Optional override for |\version| flag:
%    \begin{macrocode}
%%\providecommand{\version}{final}
%    \end{macrocode}

% Include the main document:
%    \begin{macrocode}
\input{childdoc.def}
\childdocof{cdocsamp}
%    \end{macrocode}

%\iffalse
%</samplechap1|samplechap2>
%\fi
%
%\iffalse
%<*samplechap1>
%\fi
% Some text for chapter 1:
%    \begin{macrocode}
\section{one}
some text in chapter one
%    \end{macrocode}

%\iffalse
%</samplechap1>
%\fi
% Some text for chapter 2:
%\iffalse
%<*samplechap2>
%\fi
%    \begin{macrocode}
\section{two}
more text in chapter two
%    \end{macrocode}

%\iffalse
%</samplechap2>
%\fi
%
% %%%%%%%%%%%%%%%%%%%%%%%%%%%%%%%%%%%%%%
% \paragraph{Part Include Files.}
%
% The include files are called |cdocspt3.tex| and |cdocspt4.tex|.
%
%\iffalse
%<*samplepart3|samplepart4>
%\fi

% Optional override for |\version| flag:
%    \begin{macrocode}
%%\providecommand{\version}{final}
%    \end{macrocode}

% Include the main document:
%    \begin{macrocode}
\input{childdoc.def}
\childdocby{cdocsamp}
%    \end{macrocode}

%\iffalse
%</samplepart3|samplepart4>
%\fi
%
%\iffalse
%<*samplepart3>
%\fi
% Some text for part 3:
%    \begin{macrocode}
some text in part three
%    \end{macrocode}

%\iffalse
%</samplepart3>
%\fi
% Some text for part 4:
%\iffalse
%<*samplepart4>
%\fi
%    \begin{macrocode}
more text in part four
%    \end{macrocode}

%\iffalse
%</samplepart4>
%\fi
%
% %%%%%%%%%%%%%%%%%%%%%%%%%%%%%%%%%%%%%%
% \paragraph{Forwarding for a Complete Draft.}
%
% The following forwarding file |cdocsdrf.tex|
% compiles the main document in draft mode:
%\iffalse
%<*sampledraft>
%\fi
%    \begin{macrocode}
\def\version{draft}
\input{childdoc.def}
\childdocforward{cdocsamp}
%    \end{macrocode}

%\iffalse
%</sampledraft>
%\fi
%
% %%%%%%%%%%%%%%%%%%%%%%%%%%%%%%%%%%%%%%
% \paragraph{Forwarding for Final Version of the Chapters.}
%
% The following forwarding files |cdocsfn1.tex| and |cdocsfn2.tex|
% (with identical content)
% compile the final versions of the child documents
% |cdocsch1.tex| and |cdocsch2.tex|, respectively:
%\iffalse
%<*samplefinal>
%\fi
%    \begin{macrocode}
\def\version{final}
\input{childdoc.def}
\childdocforwardprefix[cdocsamp]{cdocsfn}{cdocsch}
%    \end{macrocode}

%\iffalse
%</samplefinal>
%\fi
%
% %%%%%%%%%%%%%%%%%%%%%%%%%%%%%%%%%%%%%%
% \paragraph{Command Line Processing.}
%
% The following three command lines generate the output files
% |cdocscld|, |cdocscl1| and |cdocscl2|
% which should be identical to
% |cdocsdrf|, |cdocsch1| and |cdocsfn2|, respectively:
% \begin{center}
% \begin{tabular}{l}
% |latex -jobname cdocscld \|\\
% |  "\def\version{draft}\input{childdoc.def}\childdocforward{cdocsamp}"|\\
% |latex -jobname cdocscl1 \|\\
% |  "\input{childdoc.def}\childdocforward[cdocsamp]{cdocsch1}"|\\
% |latex -jobname cdocscl2 \|\\
% |  "\def\version{final}\input{childdoc.def}\childdocforward{cdocsch2}"|
% \end{tabular}
% \end{center}
% Note that the trailing backslash on each first line
% merely continues the input to the second line
% (for convenient cut ant paste).
% Furthermore, the command |latex| can be replaced by any
% of its alternative versions such as |pdflatex|.
%
% %%%%%%%%%%%%%%%%%%%%%%%%%%%%%%%%%%%%%%%%%%%%%%%%%%%%%%%%%%%%%%%%%%%%%%%%%%%%%%
% %%%%%%%%%%%%%%%%%%%%%%%%%%%%%%%%%%%%%%%%%%%%%%%%%%%%%%%%%%%%%%%%%%%%%%%%%%%%%%
% \section{Implementation}
%\iffalse
%<*package>
%\fi
%
% This section describes the definitions file |childdoc.def|.

% The definitions cannot be loaded using |\usepackage| or |\RequirePackage|
% which has a mechanism to prevent loading a style file more than once.
% When loading the definitions by means of |\input|
% multiple instances have to be prevented manually:
%\iffalse
%This code needs to be before the `\ProvidesFile' directive
%which is defined at the beginning of this file.
%Therefore it is also placed there and commented out here.
%</package>
%<*discard>
%\fi
%    \begin{macrocode}
\ifdefined\childdocmain\endinput\fi
%    \end{macrocode}
%\iffalse
%</discard>
%<*package>
%\fi
%
% \macro{\ifchilddoc}
% \macro{\ifchilddocmanual}
% The conditional |\ifchilddoc| tells whether a
% child (true) or main (false) document is being compiled.
% The conditional |\ifchilddocmanual| tells whether
% the |\includeonly| mechanism is used (false) or
% the selection of child files must be performed manually (true).
% The definitions initialise to false:
%    \begin{macrocode}
\newif\ifchilddoc
\newif\ifchilddocmanual
%    \end{macrocode}

% \macro{\childdocname}
% \macro{\childdocjob}
% The macro |\childdocname| stores the name of the main document
% to be compiled. The macro |\childdocjob| stores the name of
% the document on which the \LaTeX{} compiler was originally invoked.
% The content of |\jobname| cannot be compared
% to filenames specified in the source due to different catcodes.
% The following code rescans |\jobname|, stores the result
% in |\childdocname| and saves a copy in |\childdocjob|:
%    \begin{macrocode}
\edef\childdocname{\scantokens\expandafter{\jobname\noexpand}}
\let\childdocjob\childdocname
%    \end{macrocode}

% \macro{\childdocdisable}
% The macro |\childdocdisable| prevents the main file
% from being processed more than once.
% At this stage, the main document command |\childdocmain|
% is assumed to be called once again where it should do nothing.
% Any subsequent call to it should prevent
% a secondary processing of the main document
% It overwrites the forwarding commands
% |\childdocof| and |\childdocforward|
% with empty macros to prevent further inclusions of the main document:
%    \begin{macrocode}
\newcommand{\childdocdisable}
{
  \renewcommand{\childdocmain}[1]{\renewcommand{\childdocmain}[1]{\endinput}}
  \renewcommand{\childdocof}[1]{}
  \renewcommand{\childdocby}[2][]{}
  \renewcommand{\childdocforward}[2][]{}
  \renewcommand{\childdocdisable}{}
}
%    \end{macrocode}

% \macro{\childdocmain}
% The macro |\childdocmain| is to be called at the top of the main file
% with nothing or the main filename (without extension) as argument.
% First, it breaks loops.
% If the argument is not empty and does not match |\childdocname|
% (which is set by the first inclusion of |childdoc.def|),
% |\ifchilddoc| is set to true, |\includeonly| is applied to the child file
% and |\jobname| is set to the main file
% (for proper handling of |.aux| files):
%    \begin{macrocode}
\newcommand{\childdocmain}[1]
{
  \childdocdisable\childdocmain{}
  \if?#1?\else
    \begingroup
      \def\childdoctmp{#1}
      \ifx\childdoctmp\childdocname
        \def\childdoctmp{}
      \else
        \def\childdoctmp
        {
          \childdoctrue
          \includeonly{\childdocname}
          \def\childdocjob{#1}
          \def\jobname{#1}
        }
      \fi
      \expandafter
    \endgroup
    \childdoctmp
  \fi
}
%    \end{macrocode}

% \macro{\childdocof}
% The command |\childdocof| redirects
% compilation to the main file |#1|.
%    \begin{macrocode}
\newcommand{\childdocof}[1]
{
  \childdocdisable
  \childdoctrue
  \includeonly{\childdocname}
  \def\jobname{#1}
  \def\childdocjob{#1}
  \input{#1}
}
%    \end{macrocode}

% \macro{\childdocby}
% The command |\childdocby| ....
%    \begin{macrocode}
\newcommand{\childdocby}[2][]
{
  \childdocdisable
  \childdoctrue
  \childdocmanualtrue
  \if?#1?\else
    \def\jobname{#2}
  \fi
  \def\childdocjob{#2}
  \input{#2}
  \endinput
}
%    \end{macrocode}

% \macro{\childdocforward}
% The command |\childdocforward| redirects
% compilation to the main file or
% (if the optional argument is given) a child file.
% Parameters are set as if the main file
% or a child file starting with |\childdocof| was compiled.
% Then compilation is handed over to the main file:
%    \begin{macrocode}
\newcommand{\childdocforward}[2][]
{
  \begingroup
    \if?#1?
      \def\childdoctmp
      {
        \def\childdocname{#2}
        \def\childdocjob{#2}
        \def\jobname{#2}
        \input{#2}
        \endinput
      }
    \else
      \def\childdoctmp
      {
        \childdocdisable
        \def\childdocname{#2}
        \childdoctrue
        \includeonly{#2}
        \def\childdocjob{#1}
        \def\jobname{#1}
        \input{#1}
        \endinput
      }
    \fi
    \expandafter
  \endgroup
  \childdoctmp
}
%    \end{macrocode}

% \macro{\childdocforwardprefix}
% The command |\childdocforwardprefix| redirects
% compilation to the main or a child file by means of a pattern.
% The prefix |#1| in the current filename is replaced by |#2|
% and the suffix of the current filename is kept
% (it is assumed that the filename does not contain the substring `|~~~|'
% which is used as a delimiter).
% Compilation is handed over to the new file by |\childdocforward|:
%    \begin{macrocode}
\newcommand{\childdocforwardprefix}[3][]
{
  \begingroup
    \def\childdocextract #2##1~~~{\def\childdoctmp{\childdocforward[#1]{#3##1}}}
    \expandafter\childdocextract\childdocname~~~
    \expandafter
  \endgroup
  \childdoctmp
}
%    \end{macrocode}

% \macro{\childdoc}
% The deprecated macro |\childdoc| is a legacy version of |\childdocmain|:
%    \begin{macrocode}
\newcommand{\childdoc}{\childdocmain}
%    \end{macrocode}

% \macro{\childdocredirect}
% The deprecated macro |\childdocredirect| is a legacy version
% of |\childdocforward| and |\childdocforwardprefix|:
%    \begin{macrocode}
\newcommand{\childdocredirect}[2][]
{
  \begingroup
    \if?#1?
      \def\childdoctmp{\childdocforward{#2}}
    \else
      \def\childdoctmp{\childdocforwardprefix{#1}{#2}}
    \fi
    \expandafter
  \endgroup
  \childdoctmp
}
%    \end{macrocode}

%\iffalse
%</package>
%\fi
%
\endinput
|\\
|\childdocforwardprefix{final}{child}|
\end{tabular}
\end{center}
%

Note that when several versions of a main file and/or of each child file
are to be generated, it may be convenient to set up a |Makefile| or
shell script to automatise the process.

%%%%%%%%%%%%%%%%%%%%%%%%%%%%%%%%%%%%%%%%%%%%%%%%%%%%%%%%%%%%%%%%%%%%%%%%%%%%%%%%
\subsection{Command Line Processing}
\label{sec:commandline}

The effect of redirection files can also be achieved by invoking
the \LaTeX{} compiler with a more elaborate command line.
Most conveniently this should be done as part
of a shell script or a |Makefile|.

When using \textsf{childdoc} in the main file, the following
command lines effectively perform a redirection
(note that depending on the shell being used,
backslashes may have to be doubled: `|\|' $\to$ `|\\|'):
%
\begin{center}
|... -jobname "|\textit{target}|" |\\|"|[\textit{flags}]%
|% \iffalse
%
% childdoc.dtx Copyright (C) 2017-2018 Niklas Beisert
%
% This work may be distributed and/or modified under the
% conditions of the LaTeX Project Public License, either version 1.3
% of this license or (at your option) any later version.
% The latest version of this license is in
%   http://www.latex-project.org/lppl.txt
% and version 1.3 or later is part of all distributions of LaTeX
% version 2005/12/01 or later.
%
% This work has the LPPL maintenance status `maintained'.
%
% The Current Maintainer of this work is Niklas Beisert.
%
% This work consists of the files childdoc.dtx and childdoc.ins
% and the derived files childdoc.def and cdocsamp.tex with
% cdocsch1.tex, cdocsch2.tex, cdocsdrf.tex, cdocsfn1.tex, cdocsfn2.tex.
%
%<package>\ifdefined\childdocmain\endinput\fi
%<package>\ProvidesFile{childdoc.def}[2018/12/30 v2.0 child document driver]
%<samplemain>\ProvidesFile{cdocsamp.tex}[2018/12/30 v2.0 sample for childdoc]
%<*driver>
%\ProvidesFile{childdoc.drv}[2018/12/30 v2.0 childdoc reference manual file]
\PassOptionsToClass{10pt,a4paper}{article}
\documentclass{ltxdoc}

\usepackage[margin=35mm]{geometry}
\usepackage{hyperref}
\usepackage{hyperxmp}
\usepackage[usenames]{color}

\hypersetup{colorlinks=true}
\hypersetup{pdfstartview=FitH}
\hypersetup{pdfpagemode=UseNone}
\hypersetup{pdfsource={}}
\hypersetup{pdflang={en-UK}}
\hypersetup{pdfcopyright={Copyright 2017-2018 Niklas Beisert.
  This work may be distributed and/or modified under the
  conditions of the LaTeX Project Public License, either version 1.3
  of this license or (at your option) any later version.}}
\hypersetup{pdflicenseurl={http://www.latex-project.org/lppl.txt}}
\hypersetup{pdfcontactaddress={ETH Zurich, ITP, HIT K,
  Wolfgang-Pauli-Strasse 27}}
\hypersetup{pdfcontactpostcode={8093}}
\hypersetup{pdfcontactcity={Zurich}}
\hypersetup{pdfcontactcountry={Switzerland}}
\hypersetup{pdfcontactemail={nbeisert@itp.phys.ethz.ch}}
\hypersetup{pdfcontacturl={http://people.phys.ethz.ch/\xmptilde nbeisert/}}

\newcommand{\secref}[1]{\hyperref[#1]{section \ref*{#1}}}

\parskip1ex
\parindent0pt
\let\olditemize\itemize
\def\itemize{\olditemize\parskip0pt}

\begin{document}

\title{The \textsf{childdoc} Package}
\hypersetup{pdftitle={The childdoc Package}}
\author{Niklas Beisert\\[2ex]
  Institut f\"ur Theoretische Physik\\
  Eidgen\"ossische Technische Hochschule Z\"urich\\
  Wolfgang-Pauli-Strasse 27, 8093 Z\"urich, Switzerland\\[1ex]
  \href{mailto:nbeisert@itp.phys.ethz.ch}
  {\texttt{nbeisert@itp.phys.ethz.ch}}}
\hypersetup{pdfauthor={Niklas Beisert}}
\hypersetup{pdfsubject={Manual for the LaTeX2e Package childdoc}}
\date{30 December 2018, \textsf{v2.0}}
\maketitle

\begin{abstract}\noindent
\textsf{childdoc} is a \LaTeXe{} package
that enables the direct compilation
of document sections included by |\include|
to individual files.
\end{abstract}

\begingroup
\parskip0ex
\tableofcontents
\endgroup

%%%%%%%%%%%%%%%%%%%%%%%%%%%%%%%%%%%%%%%%%%%%%%%%%%%%%%%%%%%%%%%%%%%%%%%%%%%%%%%%
%%%%%%%%%%%%%%%%%%%%%%%%%%%%%%%%%%%%%%%%%%%%%%%%%%%%%%%%%%%%%%%%%%%%%%%%%%%%%%%%
\section{Introduction}

\LaTeX{} provides a mechanism to structure a large document (such as a book)
into a main file and several child files (containing the chapters)
using the |\include| command.
This mechanism is beneficial for documents
which span hundreds of pages in order to
make the source file(s) more manageable.
Moreover, compilation can be restricted to
selected child files by means of the |\includeonly| command.
The latter feature can be used to reduce the compilation time while editing
(this was significantly more useful in the earlier days of \LaTeX{})
or to generate a smaller document which is easier to navigate.
Another application of |\includeonly| is to generate
documents consisting of selected parts of the complete document.

However, there are a few drawbacks of the plain |\include| mechanism:
\begin{itemize}
\item
The child files cannot be compiled on their own,
they can only be compiled via the main file.
A naive editing environment
(such as a text editor with an option
to have the current file processed by \LaTeX)
may require one to switch to the main file before compiling;
attempting to compile the child file produces errors.
\item
The main file must be modified (each time)
to adjust the |\includeonly| command
to the present needs. This easily leaves the main file in a messy state.
\item
The generated document will always carry the filename
of the main document. This is inconvenient if
several child files are to be compiled and
to be kept for distribution.
\end{itemize}

The present package provides a simple interface
to make child files individually compilable by \LaTeX{}.
Compiling a child file then has the same effect as compiling
the main file with an |\includeonly| command
to select the appropriate child.
Moreover the generated document will carry the name of the child
rather than the main file.
This resolves all three above issues.

This feature is meant to make the editing of books,
thesis documents and lecture notes somewhat more convenient.
However, the package can also be used efficiently for
composing a series of documents (such as exercise sheets)
which are typically distributed individually.
It then assists the author in generating the individual documents
(potentially in different versions)
as well as a document containing the collected series.
Another application is in developing style files
or other kinds of included material
where compilation of the style file could redirect
to a sample or test file.

%%%%%%%%%%%%%%%%%%%%%%%%%%%%%%%%%%%%%%%%%%%%%%%%%%%%%%%%%%%%%%%%%%%%%%%%%%%%%%%%
%%%%%%%%%%%%%%%%%%%%%%%%%%%%%%%%%%%%%%%%%%%%%%%%%%%%%%%%%%%%%%%%%%%%%%%%%%%%%%%%
\section{Usage}

First of all, the package \textsf{childdoc} is \emph{not} a standard
\LaTeXe{} |.sty| style file! Therefore it needs to be invoked in
a non-standard way.

%%%%%%%%%%%%%%%%%%%%%%%%%%%%%%%%%%%%%%%%%%%%%%%%%%%%%%%%%%%%%%%%%%%%%%%%%%%%%%%%
\subsection{Included Files}
\label{sec:include}

%%%%%%%%%%%%%%%%%%%%%%%%%%%%%%%%%%%%%%%%
\DescribeMacro{\childdocmain}
To use the package, add the commands
\begin{center}
\begin{tabular}{l}
|\input{childdoc.def}|\\
|\childdocmain{}|\\
\end{tabular}
\end{center}
at the very top of the main \LaTeX{} file,
in particular \emph{before} the |\documentclass| statement!
The argument of |\childdocmain| should be left empty
(but it must be present).

%%%%%%%%%%%%%%%%%%%%%%%%%%%%%%%%%%%%%%%%
\DescribeMacro{\childdocof}
Furthermore, add the commands
\begin{center}
\begin{tabular}{l}
|\input{childdoc.def}|\\
|\childdocof{|\textit{main}|}|\\
\end{tabular}
\end{center}
at the top of every child file \textit{child}
which is included by |\include{|\textit{child}|}|
from within the main file
(or at least for those files to be compiled individually).
The argument \textit{main} must be the filename of the main file.

There are a couple of
considerations in setting up the main and child documents:

%%%%%%%%%%%%%%%%%%%%%%%%%%%%%%%%%%%%%%%%
\paragraph{Restrictions.}

Please note the following restrictions:
\begin{itemize}
\item
|\childdocmain| must be called with one argument \textit{main}
to ensure compatibility with earlier version of the package.
It must either be empty (|\childdocmain{}|)
or precisely match the filename of the main file in which it is specified.
See \secref{sec:detection} for further information.
\item
The filename \textit{main} must be specified without the |.tex| extension.
\item
The filename \textit{main} is case sensitive
(even in case-insensitive file systems)
due to internal string comparison.
\item
The argument \textit{main} should be fully expanded, it cannot be a macro.
\item
Subdirectories and special characters should be avoided in filenames.
\item
The command |\childdocmain{|\textit{main}|}| must be followed by a whitespace.
It should not be followed immediately by another command
or by a comment mark `|%|'.
This is because the \TeX{} parser reads the token immediately following
the argument of |\childdocmain| and puts it
at the beginning of every child section;
however, a white\-space is ignored.
\end{itemize}

%%%%%%%%%%%%%%%%%%%%%%%%%%%%%%%%%%%%%%%%
\paragraph{Content of Main File.}

It is advisable to place all content in the child files included by |\include|.
Any output contained in the main file will appear in all child documents
unless suppressed manually;
it cannot be suppressed automatically by the |\includeonly| directive
and thus should normally be avoided.
A method to include some content in the main file
by means of conditional processing is described in \secref{sec:conditional}.

%%%%%%%%%%%%%%%%%%%%%%%%%%%%%%%%%%%%%%%%
\paragraph{Page Numbering.}

When only a part of the document is compiled,
the appropriate numbering of pages
(as well as other status parameters)
is determined from the |.aux| files.
The latter contain information from previous passes.
However this information needs to propagate through
all intermediate child documents.
Therefore the page numbering in child documents may well
be inconsistent until the complete document is compiled at least once.

A useful (if unconventional) way to always ensure a consistent
page numbering is to restart the numbering in each child document
and denote the pages by `\textit{child}|.|\textit{page}'
where \textit{child} represents the chapter/section number of the child file.
This can be achieved by the command
|\numberwithin{page}{|\textit{child}|}|
of the \textsf{amsmath} package
where \textit{child} can be |chapter| or |section|
depending on the chosen structuring.
Alternatively, one can modify the macro |\thepage| appropriately
and reset the counter |page| at the start of each child file.

%%%%%%%%%%%%%%%%%%%%%%%%%%%%%%%%%%%%%%%%%%%%%%%%%%%%%%%%%%%%%%%%%%%%%%%%%%%%%%%%
\subsection{Conditional Processing}
\label{sec:conditional}

The package provides a mechanism to compile different versions
of a document. To customise the versions further some conditional processing
can come in handy to distinguish which version is being compiled.
The package provides two macros to describe the compilation context:

%%%%%%%%%%%%%%%%%%%%%%%%%%%%%%%%%%%%%%%%
\DescribeMacro{\ifchilddoc}
The conditional |\ifchilddoc| distinguishes between the compilation of
child documents and the main document:
%
\begin{center}
|\ifchilddoc |\textit{child-code}| |[|\||else |\textit{main-code}]| \||fi|
\end{center}

%%%%%%%%%%%%%%%%%%%%%%%%%%%%%%%%%%%%%%%%
\DescribeMacro{\childdocname}
\DescribeMacro{\childdocjob}
The macro |\childdocname| contains the filename (without extension)
of the main or child file being processed.
Note that |\childdocjob| will always contain the name of the main file.

%%%%%%%%%%%%%%%%%%%%%%%%%%%%%%%%%%%%%%%%
\paragraph{Title Page.}

Conditional processing can be used to include a title or banner page
in the main document when proper precautions are taken.
Importantly, the code in the main file should ensure that the page counter
(as well as other status parameters which are stored in the |.aux| files)
takes the same value after the conditional processing.
Otherwise the page numbers may take divergent values
depending on which part is compiled.

For example, a title page could be declared by:
%
\begin{center}
\begin{tabular}{l}
|\ifchilddoc\||else|\\
|\addtocounter{page}{-1}|\\
\textit{code for title page}\\
|\newpage|\\
|\||fi|
\end{tabular}
\end{center}
%
A banner page for the child documents can be generated by:
%
\begin{center}
\begin{tabular}{l}
|\ifchilddoc|\\
|\addtocounter{page}{-1}|\\
\textit{code for banner page}\\
|\newpage|\\
|\||fi|
\end{tabular}
\end{center}
%
Here one could write a message such as:
\begin{center}
|This is the part \childdocname{} of \childdocjob{}.|
\end{center}

%%%%%%%%%%%%%%%%%%%%%%%%%%%%%%%%%%%%%%%%%%%%%%%%%%%%%%%%%%%%%%%%%%%%%%%%%%%%%%%%
\subsection{Flags}
\label{sec:flags}

The package makes it easy to generate different versions
of the main or child documents.
To this end compilation flags can be defined
and assigned different default values.
They will be particularly useful in conjunction
with the forwarding mechanism described in \secref{sec:forward}.

For example, it may be useful to have a flag |\version|
which can be set to |draft| or |final|.
The document source will contain some conditional code
depending on the value of |\version|.
Suppose further, the flag should default to |final| for the main file
and to |draft| for child files
which is a natural assignment for editing the document.
This is achieved by placing the following code
in the preamble of the main document
(below the |\childdocmain| directive):
%
\begin{center}
\begin{tabular}{l}
|\ifchilddoc|\\
|\providecommand{\version}{draft}|\\
|\||else|\\
|\providecommand{\version}{final}|\\
|\||fi|
\end{tabular}
\end{center}
%
The definition by |\providecommand| makes sure
that previous definitions are not overwritten.
Further statements |\providecommand{\version}{...}|
can thus be added before the above code to override it.

For the main file, one might add a line
(between |\childdocmain| and the above block)
%
\begin{center}
|%\ifchilddoc\||else\providecommand{\version}{draft}\||fi|
\end{center}
%
which can be uncommented to produce a draft version.
Likewise one can add a line to the very top of a child file
(above the |\childdocof{|\textit{main}|}| directive)
%
\begin{center}
|%\providecommand{\version}{final}|
\end{center}
%
which can be uncommented to produce the final version of this child document.

%%%%%%%%%%%%%%%%%%%%%%%%%%%%%%%%%%%%%%%%%%%%%%%%%%%%%%%%%%%%%%%%%%%%%%%%%%%%%%%%
\subsection{Forwarding}
\label{sec:forward}

Different versions of the main or child documents
using compilation flags as described in \secref{sec:flags}
can be (permanently) stored in different files
for convenient compilation, viewing and distribution.
To this end, the package defines a command
to pass on compilation to a different file:

%%%%%%%%%%%%%%%%%%%%%%%%%%%%%%%%%%%%%%%%
\DescribeMacro{\childdocforward}
The command |\childdocforward| redirects processing to
another source file:
%
\begin{center}
\begin{tabular}{l}
|\input{childdoc.def}|\\
|\childdocforward[|\textit{main}|]{|\textit{dest}|}|\\
\end{tabular}
\end{center}
%
The argument \textit{dest} is the destination file
(without extension).
It should be the main file or one of the child files.
Note that further \textsf{childdoc} directives
such as |\childdocof| and |\childdocforward|
in the indicated file will be processed in this form.
The optional argument \textit{main}
passes on directly to the main file \textit{main}
while pretending to compile the child \textit{dest}.
This form behaves as if \textit{dest}
issues |\childdocof{|\textit{main}|}| right away,
and no further \textsf{childdoc} directives will be processed.

%%%%%%%%%%%%%%%%%%%%%%%%%%%%%%%%%%%%%%%%
\DescribeMacro{\...prefix}
In the alternative form |\childdocforwardprefix|,
%
\begin{center}
\begin{tabular}{l}
|\input{childdoc.def}|\\
|\childdocforwardprefix[|\textit{main}|]{|\textit{prefix}|}{|\textit{dest}|}|
\end{tabular}
\end{center}
%
the destination file is determined by a pattern
depending on the current file:
To make this work, the current file must be called
`{\textit{prefix}\hspace{0.2em}\textit{suffix}}'
with \textit{prefix} matching precisely the argument.
Processing is then passed on to the file
`{\textit{dest}\hspace{0.2em}\textit{suffix}}'.
Surely, the same effect is achieved by
directly specifying the
argument `{\textit{dest}\hspace{0.2em}\textit{suffix}}'
in the first form.
However, that requires to set up a different file
for each child. With the alternative form of the command
all these files can have exactly the same content
which simplifies setting them up and maintaining them.

For example, the following file |draft.tex|
with a compilation flag |\version| as described in \secref{sec:flags}
compiles the main document as a draft:
%
\begin{center}
\begin{tabular}{l}
|\def\version{draft}|\\
|\input{childdoc.def}|\\
|\childdocforward{|\textit{main}|}|
\end{tabular}
\end{center}
%
Likewise, the following files |final|\textit{nn}|.tex|
compile the final version of the child document
|child|\textit{nn}|.tex|:
%
\begin{center}
\begin{tabular}{l}
|\def\version{final}|\\
|\input{childdoc.def}|\\
|\childdocforwardprefix{final}{child}|
\end{tabular}
\end{center}
%

Note that when several versions of a main file and/or of each child file
are to be generated, it may be convenient to set up a |Makefile| or
shell script to automatise the process.

%%%%%%%%%%%%%%%%%%%%%%%%%%%%%%%%%%%%%%%%%%%%%%%%%%%%%%%%%%%%%%%%%%%%%%%%%%%%%%%%
\subsection{Command Line Processing}
\label{sec:commandline}

The effect of redirection files can also be achieved by invoking
the \LaTeX{} compiler with a more elaborate command line.
Most conveniently this should be done as part
of a shell script or a |Makefile|.

When using \textsf{childdoc} in the main file, the following
command lines effectively perform a redirection
(note that depending on the shell being used,
backslashes may have to be doubled: `|\|' $\to$ `|\\|'):
%
\begin{center}
|... -jobname "|\textit{target}|" |\\|"|[\textit{flags}]%
|\input{childdoc.def}\childdocforward[|\textit{main}|]{|\textit{dest}|}"|
\end{center}
%
Here \textit{target} is the name of the output file,
\textit{main} is the name of the main file
and \textit{dest} is the name of the main or child file to be processed
(all filenames without extensions).
The optional argument \textit{main} can be omitted
if \textit{main} matches \textit{dest}.
Optionally, compilation \textit{flags} can be defined via |\def| commands.
This command line makes the \TeX{} engine believe
it is compiling the file \textit{target}
whose content is specified as the latter parameter.
The provided code then forwards the processing to
\textit{main} or \textit{dest} as described in \secref{sec:forward}.

%%%%%%%%%%%%%%%%%%%%%%%%%%%%%%%%%%%%%%%%%%%%%%%%%%%%%%%%%%%%%%%%%%%%%%%%%%%%%%%%
\subsection{Include by Input}
\label{sec:input}

Including child documents by |\include| has some restrictions by design.
Most notably, the content of a child document always occupies
its own set of pages; pages cannot be shared between child documents.
Usually, this behaviour makes perfect sense
because each child document contain an essential part of the document.
However, in some situations it may be desirable to compose
a document from a collection of parts
without having mandatory page breaks between then.
For this case, the package
provides a mechanism to include parts
by |\input| which can also be processed individually.
However, by construction this mechanism
requires manual handling of the content to be output.

%%%%%%%%%%%%%%%%%%%%%%%%%%%%%%%%%%%%%%%%
\DescribeMacro{\ifchilddocmanual}
The main file should be prepared as usual, see \secref{sec:include}.
However, the document body must make a distinction
between processing of an individual part and of the main document, e.g.:
%
\begin{center}
\begin{tabular}{l}
|\ifchilddocmanual|\\
|\input{\childdocname}|\\
|\||else|\\
\textit{document body with }|\input{|\textit{part}|}|\\
|\||fi|
\end{tabular}
\end{center}
%
The conditional |\ifchilddocmanual| is true whenever
a part to be included by |\input| is being compiled,
and the name of the part is stored in |\childdocname|.

%%%%%%%%%%%%%%%%%%%%%%%%%%%%%%%%%%%%%%%%
\DescribeMacro{\childdocby}
Each part to be included by |\input| should start with:
%
\begin{center}
\begin{tabular}{l}
|\input{childdoc.def}|\\
|\childdocby{|\textit{main}|}|\\
\end{tabular}
\end{center}
%
The directive |\childdocby| is similar to |\childdocof|
described in \secref{sec:include},
but the subsequent selection of content must be done manually.
To that end, both |\ifchilddoc| and |\ifchilddocmanual|
will be true upon processing of a part,
and the name of the part is stored in |\childdocname|.
Note that |\jobname| will be set to the filename of the current part
so that each part receives an individual |.aux| file
that does not interfere with the |.aux| file(s) of the main document.
This behaviour can be altered by the alternative form
|\childdocby[*]{|\textit{main}|}| (with a non-empty optional argument)
which uses the |.aux| file of the main document
by setting |\jobname| to \textit{main}.

%%%%%%%%%%%%%%%%%%%%%%%%%%%%%%%%%%%%%%%%%%%%%%%%%%%%%%%%%%%%%%%%%%%%%%%%%%%%%%%%
\subsection{Driver Development}
\label{sec:driver}

The \textsf{childdoc} mechanism can also be use for the development
of definition files such as \LaTeX{} styles or classes.
This case differs from the above setup with multiple parts
included by |\include| in that no |\includeonly| should be invoked.
This can be achieved by starting the include file
(before |\ProvidesPackage|) with:
%
\begin{center}
\begin{tabular}{l}
|\input{childdoc.def}|\\
|\childdocforward{|\textit{main}|}|\\
\end{tabular}
\end{center}
%
or alternatively with:
%
\begin{center}
\begin{tabular}{l}
|\input{childdoc.def}|\\
|\childdocby{|\textit{main}|}|\\
\end{tabular}
\end{center}
%
Both forms have slightly different effects as described above.
The main file is prepared as usual, see \secref{sec:include}.

%%%%%%%%%%%%%%%%%%%%%%%%%%%%%%%%%%%%%%%%%%%%%%%%%%%%%%%%%%%%%%%%%%%%%%%%%%%%%%%%
\subsection{Legacy Detection}
\label{sec:detection}

The directive |\childdocmain| in the main file can detect
whether the complete document or merely a child is to be compiled
even without using the directive |\childdocof|.
This method is deprecated because it is less robust
and there is no compelling reason to use it;
it is merely provided for backward compatibility
and it may be removed in future versions.

If the detection mechanism is to be used,
it is mandatory to correctly specify
the filename of the main file as the argument of |\childdocmain|:
%
\begin{center}
\begin{tabular}{l}
|\input{childdoc.def}|\\
|\childdocmain{|\textit{main}|}|\\
\end{tabular}
\end{center}
%
If |\jobname| does not match the argument \textit{main} of |\childdocmain|,
it is assumed that |\jobname| points to the child file to be compiled.
When using |\childdocmain| with the main file specified as argument,
it suffices to start a child file
with just |\input{|\textit{main}|}|
without loading of the package and using |\childdocof|.
If instead all processing is done
with the appropriate \textsf{childdoc} directives,
the argument of \textit{main} of |\childdocmain| can be empty.

An alternative version of the command line processing described
in \secref{sec:commandline} using the detection mechanism reads:
%
\begin{center}
|... -jobname "|\textit{target}|" "|[\textit{flags}]%
[|\def\jobname{|\textit{dest}|}|]|\input{|\textit{main}|}"|
\end{center}

%%%%%%%%%%%%%%%%%%%%%%%%%%%%%%%%%%%%%%%%%%%%%%%%%%%%%%%%%%%%%%%%%%%%%%%%%%%%%%%%
\subsection{Manual Code}
\label{sec:manual}

In case one cannot be certain whether the definitions file |childdoc.def|
is installed on the target \TeX{} distribution
and one prefers not to ship it,
it is conceivable to paste a few relevant commands into the sources.

To that end, drop all statements |\input{childdoc.def}|
and perform the replacements as outlined below.
Instead of |\childdocmain{|\textit{main}|}| add the following code
to the top of the main file:
%
\begin{center}
\begin{tabular}{l}
|\||ifdefined\childdocname\endinput\||fi\newif\ifchilddoc|\\
|\edef\childdocname{\scantokens\expandafter{\jobname\noexpand}}|\\
|\def\childdocmain{|\textit{main}|}\||ifx\childdocmain\childdocname\||else|\\
|\childdoctrue\includeonly{\childdocname}\let\jobname\childdocmain\||fi|\\
\end{tabular}
\end{center}
%
Instead of |\childdocof{|\textit{main}|}| just include the main file
at the top of each child file:
%
\begin{center}
|\input{|\textit{main}|}|
\end{center}
%
A simple redirection |\childdocforward{|\textit{dest}|}| is achieved by:
%
\begin{center}
|\def\jobname{|\textit{dest}|}\input{\jobname}|
\end{center}
%
The redirection with prefix
|\childdocforwardprefix[|\textit{prefix}|]{|\textit{dest}|}|
is accomplished by:
%
\begin{center}
\begin{tabular}{l}
|{\edef\jobname{\scantokens\expandafter{\jobname\noexpand}}|\\
|\def\redirectjob |\textit{prefix}|#1~~~{\gdef\jobname{|\textit{dest}|#1}}|\\
|\expandafter\redirectjob\jobname~~~}\input{\jobname}|
\end{tabular}
\end{center}

In an alternative approach,
child documents can be compiled by a specific command line
without additional code or specific definitions:
%
\begin{center}
|... -jobname "|\textit{target}|" "|[\textit{flags}]%
|\includeonly{|\textit{dest}|}\input{|\textit{main}|}"|
\end{center}
%

%%%%%%%%%%%%%%%%%%%%%%%%%%%%%%%%%%%%%%%%%%%%%%%%%%%%%%%%%%%%%%%%%%%%%%%%%%%%%%%%
%%%%%%%%%%%%%%%%%%%%%%%%%%%%%%%%%%%%%%%%%%%%%%%%%%%%%%%%%%%%%%%%%%%%%%%%%%%%%%%%
\section{Information}

%%%%%%%%%%%%%%%%%%%%%%%%%%%%%%%%%%%%%%%%%%%%%%%%%%%%%%%%%%%%%%%%%%%%%%%%%%%%%%%%
\subsection{Copyright}

Copyright \copyright{} 2017--2018 Niklas Beisert

This work may be distributed and/or modified under the
conditions of the \LaTeX{} Project Public License, either version 1.3
of this license or (at your option) any later version.
The latest version of this license is in
  \url{http://www.latex-project.org/lppl.txt}
and version 1.3 or later is part of all distributions of \LaTeX{}
version 2005/12/01 or later.

This work has the LPPL maintenance status `maintained'.

The Current Maintainer of this work is Niklas Beisert.

This work consists of the files |README.txt|, |childdoc.ins| and |childdoc.dtx|
as well as the derived files |childdoc.def|, |cdocsamp.tex|
with |cdocsch1.tex|, |cdocsch2.tex|, |cdocspt3.tex|, |cdocspt4.tex|,
|cdocsdrf.tex|, |cdocsfn1.tex|, |cdocsfn2.tex|
as well as |childdoc.pdf|.

%%%%%%%%%%%%%%%%%%%%%%%%%%%%%%%%%%%%%%%%%%%%%%%%%%%%%%%%%%%%%%%%%%%%%%%%%%%%%%%%
\subsection{Files and Installation}

The package consists of the files:
%
\begin{center}
\begin{tabular}{ll}
    |README.txt|   & readme file \\
    |childdoc.ins| & installation file \\
    |childdoc.dtx| & source file \\
    |childdoc.def| & definition file \\
    |cdocsamp.tex| & sample main file \\
    |cdocsch1.tex| & sample include file \\
    |cdocsch2.tex| & sample include file \\
    |cdocspt3.tex| & sample part file \\
    |cdocspt4.tex| & sample part file \\
    |cdocsdrf.tex| & sample redirection file \\
    |cdocsfn1.tex| & sample redirection file \\
    |cdocsfn2.tex| & sample redirection file \\
    |childdoc.pdf| & manual
\end{tabular}
\end{center}
%
The distribution consists of the files
|README.txt|, |childdoc.ins| and |childdoc.dtx|.
%
\begin{itemize}
\item
Run (pdf)\LaTeX{} on |childdoc.dtx|
to compile the manual |childdoc.pdf| (this file).
\item
Run \LaTeX{} on |childdoc.ins| to create the definitions file |childdoc.def|
and the sample |cdocsamp.tex| with include files
|cdocsch1.tex|, |cdocsch2.tex|, |cdocspt3.tex|, |cdocspt4.tex|,
|cdocsdrf.tex|, |cdocsfn1.tex|, |cdocsfn2.tex|.
Then copy the file |childdoc.def| to an appropriate directory of your \LaTeX{}
distribution, e.g.\ \textit{texmf-root}|/tex/latex/childdoc|.
\end{itemize}

%%%%%%%%%%%%%%%%%%%%%%%%%%%%%%%%%%%%%%%%%%%%%%%%%%%%%%%%%%%%%%%%%%%%%%%%%%%%%%%%
\subsection{Related CTAN Packages}

There are several other packages which offer a similar functionality:
%
\begin{itemize}
\item
The packages
\href{http://ctan.org/pkg/docmute}{\textsf{docmute}},
\href{http://ctan.org/pkg/includex}{\textsf{includex}} and
\href{http://ctan.org/pkg/standalone}{\textsf{standalone}}
provide commands to include only the document body of
a child file thus allowing both files to be compiled individually.
\item
The packages \href{http://ctan.org/pkg/subdocs}{\textsf{subdocs}}
and \href{http://ctan.org/pkg/subfiles}{\textsf{subfiles}}
provide structures in which the main and child documents can be
encapsulated and allowing them to be compiled individually.
The inclusion mechanism is different from the conventional |\include|.
\item
The package \href{http://ctan.org/pkg/combine}{\textsf{combine}}
is an elaborate solution to combine several documents into one.
\end{itemize}
%
See also the CTAN topic \href{http://ctan.org/topic/subdocs}{\textsf{subdocs}}
for further related packages.
The present package differs from the above solutions in that
a document structure constructed with the conventional |\include| mechanism
just needs two extra commands at the top of every file
such that all constituent files can be compiled individually.

%%%%%%%%%%%%%%%%%%%%%%%%%%%%%%%%%%%%%%%%%%%%%%%%%%%%%%%%%%%%%%%%%%%%%%%%%%%%%%%%
%\subsection{Feature Suggestions}
%
%The following is a list of features which may be useful for future
%versions of this package:
%%
%\begin{itemize}
%\item
%\ldots
%\end{itemize}

%%%%%%%%%%%%%%%%%%%%%%%%%%%%%%%%%%%%%%%%%%%%%%%%%%%%%%%%%%%%%%%%%%%%%%%%%%%%%%%%
\subsection{Revision History}

%%%%%%%%%%%%%%%%%%%%%%%%%%%%%%%%%%%%%%%%
\paragraph{v2.0:} 2018/12/30

\begin{itemize}
\item
immediate forward processing
\item
added |\childdocby| mechanism
\item
manual restructured
\end{itemize}

%%%%%%%%%%%%%%%%%%%%%%%%%%%%%%%%%%%%%%%%
\paragraph{v1.6:} 2018/01/17

\begin{itemize}
\item
application for development of include files
\item
corrections to manual
\end{itemize}

%%%%%%%%%%%%%%%%%%%%%%%%%%%%%%%%%%%%%%%%
\paragraph{v1.5:} 2017/05/21

\begin{itemize}
\item
more complete structuring introduced
\item
|\childdocof| introduced
\item
|\childdoc| renamed to |\childdocmain|
\item
|\childredirect| renamed to |\childdocforward| and |\childdocforwardprefix|
and functionality expanded
\end{itemize}

%%%%%%%%%%%%%%%%%%%%%%%%%%%%%%%%%%%%%%%%
\paragraph{v1.0:} 2017/04/27

\begin{itemize}
\item
manual and install package
\item
first version published on CTAN
\end{itemize}

%%%%%%%%%%%%%%%%%%%%%%%%%%%%%%%%%%%%%%%%
\paragraph{v0.6:} 2017/04/26

\begin{itemize}
\item
redirection mechanism added
\end{itemize}

%%%%%%%%%%%%%%%%%%%%%%%%%%%%%%%%%%%%%%%%
\paragraph{v0.5:} 2017/04/26

\begin{itemize}
\item
functionality in definition file
\end{itemize}


%%%%%%%%%%%%%%%%%%%%%%%%%%%%%%%%%%%%%%%%%%%%%%%%%%%%%%%%%%%%%%%%%%%%%%%%%%%%%%%%
%%%%%%%%%%%%%%%%%%%%%%%%%%%%%%%%%%%%%%%%%%%%%%%%%%%%%%%%%%%%%%%%%%%%%%%%%%%%%%%%
%%%%%%%%%%%%%%%%%%%%%%%%%%%%%%%%%%%%%%%%%%%%%%%%%%%%%%%%%%%%%%%%%%%%%%%%%%%%%%%%
\appendix

\settowidth\MacroIndent{\rmfamily\scriptsize 000\ }

 \DocInput{childdoc.dtx}

\end{document}
%</driver>
% \fi
%
% %%%%%%%%%%%%%%%%%%%%%%%%%%%%%%%%%%%%%%%%%%%%%%%%%%%%%%%%%%%%%%%%%%%%%%%%%%%%%%
% %%%%%%%%%%%%%%%%%%%%%%%%%%%%%%%%%%%%%%%%%%%%%%%%%%%%%%%%%%%%%%%%%%%%%%%%%%%%%%
% \section{Sample}
%\iffalse
%<*samplemain>
%\fi
%
% The following presents a sample document
% with two chapters, two parts, a title page,
% a compile flag as well as three forwarding files to set the flag.
% It consists of eight |.tex| files:
% \begin{center}
% \begin{tabular}{ll}
% |cdocsamp.tex|&main file\\
% |cdocsch1.tex|&include file for chapter 1\\
% |cdocsch2.tex|&include file for chapter 2\\
% |cdocspt3.tex|&include file for part 3\\
% |cdocspt4.tex|&include file for part 4\\
% |cdocsdrf.tex|&forwarding file for main file in draft mode\\
% |cdocsfi1.tex|&forwarding file for final version of chapter 1\\
% |cdocsfi2.tex|&forwarding file for final version of chapter 2\\
% \end{tabular}
% \end{center}
% Each of the eight files can be compiled directly by the \LaTeX{} compiler.
%
% %%%%%%%%%%%%%%%%%%%%%%%%%%%%%%%%%%%%%%
% \paragraph{Main File.}
%
% The main file is called |cdocsamp.tex|.
%
% Load the \textsf{childdoc} definitions and
% declare the filename for the main document:
%    \begin{macrocode}
\input{childdoc.def}
\childdocmain{}
%    \end{macrocode}

% Optional override for |\version| flag:
%    \begin{macrocode}
%%\ifchilddoc\else\providecommand{\version}{draft}\fi
%    \end{macrocode}

% Define the default values for the |\version| flag
% (|final| for the main file and |draft| for childs):
%    \begin{macrocode}
\ifchilddoc
\providecommand{\version}{draft}
\else
\providecommand{\version}{final}
\fi
%    \end{macrocode}

% Load the standard document class:
%    \begin{macrocode}
\documentclass[12pt]{article}
%    \end{macrocode}

% Start the document body:
%    \begin{macrocode}
\begin{document}
%    \end{macrocode}

% Declare a title page.
% Print title, part of document being processed and version flag:
%    \begin{macrocode}
\addtocounter{page}{-1}
\begin{center}
{\LARGE\bfseries{}childdoc example\par}
\vspace{1cm}
\ifchilddoc
\ifchilddocmanual part\else chapter\fi:
`\childdocname' of `\childdocjob'\par
\else
main document: `\childdocjob'\par
\fi
version: \version\par
\end{center}
\newpage
%    \end{macrocode}

% Manually include selected file,
% otherwise process as usual:
%    \begin{macrocode}
\ifchilddocmanual
\section*{part `\childdocname'}
\input{\childdocname}
\else
%    \end{macrocode}

% Include the two chapters:
%    \begin{macrocode}
\include{cdocsch1}
\include{cdocsch2}
%    \end{macrocode}

% Include the two parts unless only chapters should be displayed:
%    \begin{macrocode}
\ifchilddoc\else
\section{part three}
\input{cdocspt3}
\section{part four}
\input{cdocspt4}
\fi
%    \end{macrocode}

% Process as usual until here:
%    \begin{macrocode}
\fi
%    \end{macrocode}

% End of document body:
%    \begin{macrocode}
\end{document}
%    \end{macrocode}
%\iffalse
%</samplemain>
%\fi
%
% %%%%%%%%%%%%%%%%%%%%%%%%%%%%%%%%%%%%%%
% \paragraph{Chapter Include Files.}
%
% The include files are called |cdocsch1.tex| and |cdocsch2.tex|.
%
%\iffalse
%<*samplechap1|samplechap2>
%\fi

% Optional override for |\version| flag:
%    \begin{macrocode}
%%\providecommand{\version}{final}
%    \end{macrocode}

% Include the main document:
%    \begin{macrocode}
\input{childdoc.def}
\childdocof{cdocsamp}
%    \end{macrocode}

%\iffalse
%</samplechap1|samplechap2>
%\fi
%
%\iffalse
%<*samplechap1>
%\fi
% Some text for chapter 1:
%    \begin{macrocode}
\section{one}
some text in chapter one
%    \end{macrocode}

%\iffalse
%</samplechap1>
%\fi
% Some text for chapter 2:
%\iffalse
%<*samplechap2>
%\fi
%    \begin{macrocode}
\section{two}
more text in chapter two
%    \end{macrocode}

%\iffalse
%</samplechap2>
%\fi
%
% %%%%%%%%%%%%%%%%%%%%%%%%%%%%%%%%%%%%%%
% \paragraph{Part Include Files.}
%
% The include files are called |cdocspt3.tex| and |cdocspt4.tex|.
%
%\iffalse
%<*samplepart3|samplepart4>
%\fi

% Optional override for |\version| flag:
%    \begin{macrocode}
%%\providecommand{\version}{final}
%    \end{macrocode}

% Include the main document:
%    \begin{macrocode}
\input{childdoc.def}
\childdocby{cdocsamp}
%    \end{macrocode}

%\iffalse
%</samplepart3|samplepart4>
%\fi
%
%\iffalse
%<*samplepart3>
%\fi
% Some text for part 3:
%    \begin{macrocode}
some text in part three
%    \end{macrocode}

%\iffalse
%</samplepart3>
%\fi
% Some text for part 4:
%\iffalse
%<*samplepart4>
%\fi
%    \begin{macrocode}
more text in part four
%    \end{macrocode}

%\iffalse
%</samplepart4>
%\fi
%
% %%%%%%%%%%%%%%%%%%%%%%%%%%%%%%%%%%%%%%
% \paragraph{Forwarding for a Complete Draft.}
%
% The following forwarding file |cdocsdrf.tex|
% compiles the main document in draft mode:
%\iffalse
%<*sampledraft>
%\fi
%    \begin{macrocode}
\def\version{draft}
\input{childdoc.def}
\childdocforward{cdocsamp}
%    \end{macrocode}

%\iffalse
%</sampledraft>
%\fi
%
% %%%%%%%%%%%%%%%%%%%%%%%%%%%%%%%%%%%%%%
% \paragraph{Forwarding for Final Version of the Chapters.}
%
% The following forwarding files |cdocsfn1.tex| and |cdocsfn2.tex|
% (with identical content)
% compile the final versions of the child documents
% |cdocsch1.tex| and |cdocsch2.tex|, respectively:
%\iffalse
%<*samplefinal>
%\fi
%    \begin{macrocode}
\def\version{final}
\input{childdoc.def}
\childdocforwardprefix[cdocsamp]{cdocsfn}{cdocsch}
%    \end{macrocode}

%\iffalse
%</samplefinal>
%\fi
%
% %%%%%%%%%%%%%%%%%%%%%%%%%%%%%%%%%%%%%%
% \paragraph{Command Line Processing.}
%
% The following three command lines generate the output files
% |cdocscld|, |cdocscl1| and |cdocscl2|
% which should be identical to
% |cdocsdrf|, |cdocsch1| and |cdocsfn2|, respectively:
% \begin{center}
% \begin{tabular}{l}
% |latex -jobname cdocscld \|\\
% |  "\def\version{draft}\input{childdoc.def}\childdocforward{cdocsamp}"|\\
% |latex -jobname cdocscl1 \|\\
% |  "\input{childdoc.def}\childdocforward[cdocsamp]{cdocsch1}"|\\
% |latex -jobname cdocscl2 \|\\
% |  "\def\version{final}\input{childdoc.def}\childdocforward{cdocsch2}"|
% \end{tabular}
% \end{center}
% Note that the trailing backslash on each first line
% merely continues the input to the second line
% (for convenient cut ant paste).
% Furthermore, the command |latex| can be replaced by any
% of its alternative versions such as |pdflatex|.
%
% %%%%%%%%%%%%%%%%%%%%%%%%%%%%%%%%%%%%%%%%%%%%%%%%%%%%%%%%%%%%%%%%%%%%%%%%%%%%%%
% %%%%%%%%%%%%%%%%%%%%%%%%%%%%%%%%%%%%%%%%%%%%%%%%%%%%%%%%%%%%%%%%%%%%%%%%%%%%%%
% \section{Implementation}
%\iffalse
%<*package>
%\fi
%
% This section describes the definitions file |childdoc.def|.

% The definitions cannot be loaded using |\usepackage| or |\RequirePackage|
% which has a mechanism to prevent loading a style file more than once.
% When loading the definitions by means of |\input|
% multiple instances have to be prevented manually:
%\iffalse
%This code needs to be before the `\ProvidesFile' directive
%which is defined at the beginning of this file.
%Therefore it is also placed there and commented out here.
%</package>
%<*discard>
%\fi
%    \begin{macrocode}
\ifdefined\childdocmain\endinput\fi
%    \end{macrocode}
%\iffalse
%</discard>
%<*package>
%\fi
%
% \macro{\ifchilddoc}
% \macro{\ifchilddocmanual}
% The conditional |\ifchilddoc| tells whether a
% child (true) or main (false) document is being compiled.
% The conditional |\ifchilddocmanual| tells whether
% the |\includeonly| mechanism is used (false) or
% the selection of child files must be performed manually (true).
% The definitions initialise to false:
%    \begin{macrocode}
\newif\ifchilddoc
\newif\ifchilddocmanual
%    \end{macrocode}

% \macro{\childdocname}
% \macro{\childdocjob}
% The macro |\childdocname| stores the name of the main document
% to be compiled. The macro |\childdocjob| stores the name of
% the document on which the \LaTeX{} compiler was originally invoked.
% The content of |\jobname| cannot be compared
% to filenames specified in the source due to different catcodes.
% The following code rescans |\jobname|, stores the result
% in |\childdocname| and saves a copy in |\childdocjob|:
%    \begin{macrocode}
\edef\childdocname{\scantokens\expandafter{\jobname\noexpand}}
\let\childdocjob\childdocname
%    \end{macrocode}

% \macro{\childdocdisable}
% The macro |\childdocdisable| prevents the main file
% from being processed more than once.
% At this stage, the main document command |\childdocmain|
% is assumed to be called once again where it should do nothing.
% Any subsequent call to it should prevent
% a secondary processing of the main document
% It overwrites the forwarding commands
% |\childdocof| and |\childdocforward|
% with empty macros to prevent further inclusions of the main document:
%    \begin{macrocode}
\newcommand{\childdocdisable}
{
  \renewcommand{\childdocmain}[1]{\renewcommand{\childdocmain}[1]{\endinput}}
  \renewcommand{\childdocof}[1]{}
  \renewcommand{\childdocby}[2][]{}
  \renewcommand{\childdocforward}[2][]{}
  \renewcommand{\childdocdisable}{}
}
%    \end{macrocode}

% \macro{\childdocmain}
% The macro |\childdocmain| is to be called at the top of the main file
% with nothing or the main filename (without extension) as argument.
% First, it breaks loops.
% If the argument is not empty and does not match |\childdocname|
% (which is set by the first inclusion of |childdoc.def|),
% |\ifchilddoc| is set to true, |\includeonly| is applied to the child file
% and |\jobname| is set to the main file
% (for proper handling of |.aux| files):
%    \begin{macrocode}
\newcommand{\childdocmain}[1]
{
  \childdocdisable\childdocmain{}
  \if?#1?\else
    \begingroup
      \def\childdoctmp{#1}
      \ifx\childdoctmp\childdocname
        \def\childdoctmp{}
      \else
        \def\childdoctmp
        {
          \childdoctrue
          \includeonly{\childdocname}
          \def\childdocjob{#1}
          \def\jobname{#1}
        }
      \fi
      \expandafter
    \endgroup
    \childdoctmp
  \fi
}
%    \end{macrocode}

% \macro{\childdocof}
% The command |\childdocof| redirects
% compilation to the main file |#1|.
%    \begin{macrocode}
\newcommand{\childdocof}[1]
{
  \childdocdisable
  \childdoctrue
  \includeonly{\childdocname}
  \def\jobname{#1}
  \def\childdocjob{#1}
  \input{#1}
}
%    \end{macrocode}

% \macro{\childdocby}
% The command |\childdocby| ....
%    \begin{macrocode}
\newcommand{\childdocby}[2][]
{
  \childdocdisable
  \childdoctrue
  \childdocmanualtrue
  \if?#1?\else
    \def\jobname{#2}
  \fi
  \def\childdocjob{#2}
  \input{#2}
  \endinput
}
%    \end{macrocode}

% \macro{\childdocforward}
% The command |\childdocforward| redirects
% compilation to the main file or
% (if the optional argument is given) a child file.
% Parameters are set as if the main file
% or a child file starting with |\childdocof| was compiled.
% Then compilation is handed over to the main file:
%    \begin{macrocode}
\newcommand{\childdocforward}[2][]
{
  \begingroup
    \if?#1?
      \def\childdoctmp
      {
        \def\childdocname{#2}
        \def\childdocjob{#2}
        \def\jobname{#2}
        \input{#2}
        \endinput
      }
    \else
      \def\childdoctmp
      {
        \childdocdisable
        \def\childdocname{#2}
        \childdoctrue
        \includeonly{#2}
        \def\childdocjob{#1}
        \def\jobname{#1}
        \input{#1}
        \endinput
      }
    \fi
    \expandafter
  \endgroup
  \childdoctmp
}
%    \end{macrocode}

% \macro{\childdocforwardprefix}
% The command |\childdocforwardprefix| redirects
% compilation to the main or a child file by means of a pattern.
% The prefix |#1| in the current filename is replaced by |#2|
% and the suffix of the current filename is kept
% (it is assumed that the filename does not contain the substring `|~~~|'
% which is used as a delimiter).
% Compilation is handed over to the new file by |\childdocforward|:
%    \begin{macrocode}
\newcommand{\childdocforwardprefix}[3][]
{
  \begingroup
    \def\childdocextract #2##1~~~{\def\childdoctmp{\childdocforward[#1]{#3##1}}}
    \expandafter\childdocextract\childdocname~~~
    \expandafter
  \endgroup
  \childdoctmp
}
%    \end{macrocode}

% \macro{\childdoc}
% The deprecated macro |\childdoc| is a legacy version of |\childdocmain|:
%    \begin{macrocode}
\newcommand{\childdoc}{\childdocmain}
%    \end{macrocode}

% \macro{\childdocredirect}
% The deprecated macro |\childdocredirect| is a legacy version
% of |\childdocforward| and |\childdocforwardprefix|:
%    \begin{macrocode}
\newcommand{\childdocredirect}[2][]
{
  \begingroup
    \if?#1?
      \def\childdoctmp{\childdocforward{#2}}
    \else
      \def\childdoctmp{\childdocforwardprefix{#1}{#2}}
    \fi
    \expandafter
  \endgroup
  \childdoctmp
}
%    \end{macrocode}

%\iffalse
%</package>
%\fi
%
\endinput
\childdocforward[|\textit{main}|]{|\textit{dest}|}"|
\end{center}
%
Here \textit{target} is the name of the output file,
\textit{main} is the name of the main file
and \textit{dest} is the name of the main or child file to be processed
(all filenames without extensions).
The optional argument \textit{main} can be omitted
if \textit{main} matches \textit{dest}.
Optionally, compilation \textit{flags} can be defined via |\def| commands.
This command line makes the \TeX{} engine believe
it is compiling the file \textit{target}
whose content is specified as the latter parameter.
The provided code then forwards the processing to
\textit{main} or \textit{dest} as described in \secref{sec:forward}.

%%%%%%%%%%%%%%%%%%%%%%%%%%%%%%%%%%%%%%%%%%%%%%%%%%%%%%%%%%%%%%%%%%%%%%%%%%%%%%%%
\subsection{Include by Input}
\label{sec:input}

Including child documents by |\include| has some restrictions by design.
Most notably, the content of a child document always occupies
its own set of pages; pages cannot be shared between child documents.
Usually, this behaviour makes perfect sense
because each child document contain an essential part of the document.
However, in some situations it may be desirable to compose
a document from a collection of parts
without having mandatory page breaks between then.
For this case, the package
provides a mechanism to include parts
by |\input| which can also be processed individually.
However, by construction this mechanism
requires manual handling of the content to be output.

%%%%%%%%%%%%%%%%%%%%%%%%%%%%%%%%%%%%%%%%
\DescribeMacro{\ifchilddocmanual}
The main file should be prepared as usual, see \secref{sec:include}.
However, the document body must make a distinction
between processing of an individual part and of the main document, e.g.:
%
\begin{center}
\begin{tabular}{l}
|\ifchilddocmanual|\\
|\input{\childdocname}|\\
|\||else|\\
\textit{document body with }|\input{|\textit{part}|}|\\
|\||fi|
\end{tabular}
\end{center}
%
The conditional |\ifchilddocmanual| is true whenever
a part to be included by |\input| is being compiled,
and the name of the part is stored in |\childdocname|.

%%%%%%%%%%%%%%%%%%%%%%%%%%%%%%%%%%%%%%%%
\DescribeMacro{\childdocby}
Each part to be included by |\input| should start with:
%
\begin{center}
\begin{tabular}{l}
|% \iffalse
%
% childdoc.dtx Copyright (C) 2017-2018 Niklas Beisert
%
% This work may be distributed and/or modified under the
% conditions of the LaTeX Project Public License, either version 1.3
% of this license or (at your option) any later version.
% The latest version of this license is in
%   http://www.latex-project.org/lppl.txt
% and version 1.3 or later is part of all distributions of LaTeX
% version 2005/12/01 or later.
%
% This work has the LPPL maintenance status `maintained'.
%
% The Current Maintainer of this work is Niklas Beisert.
%
% This work consists of the files childdoc.dtx and childdoc.ins
% and the derived files childdoc.def and cdocsamp.tex with
% cdocsch1.tex, cdocsch2.tex, cdocsdrf.tex, cdocsfn1.tex, cdocsfn2.tex.
%
%<package>\ifdefined\childdocmain\endinput\fi
%<package>\ProvidesFile{childdoc.def}[2018/12/30 v2.0 child document driver]
%<samplemain>\ProvidesFile{cdocsamp.tex}[2018/12/30 v2.0 sample for childdoc]
%<*driver>
%\ProvidesFile{childdoc.drv}[2018/12/30 v2.0 childdoc reference manual file]
\PassOptionsToClass{10pt,a4paper}{article}
\documentclass{ltxdoc}

\usepackage[margin=35mm]{geometry}
\usepackage{hyperref}
\usepackage{hyperxmp}
\usepackage[usenames]{color}

\hypersetup{colorlinks=true}
\hypersetup{pdfstartview=FitH}
\hypersetup{pdfpagemode=UseNone}
\hypersetup{pdfsource={}}
\hypersetup{pdflang={en-UK}}
\hypersetup{pdfcopyright={Copyright 2017-2018 Niklas Beisert.
  This work may be distributed and/or modified under the
  conditions of the LaTeX Project Public License, either version 1.3
  of this license or (at your option) any later version.}}
\hypersetup{pdflicenseurl={http://www.latex-project.org/lppl.txt}}
\hypersetup{pdfcontactaddress={ETH Zurich, ITP, HIT K,
  Wolfgang-Pauli-Strasse 27}}
\hypersetup{pdfcontactpostcode={8093}}
\hypersetup{pdfcontactcity={Zurich}}
\hypersetup{pdfcontactcountry={Switzerland}}
\hypersetup{pdfcontactemail={nbeisert@itp.phys.ethz.ch}}
\hypersetup{pdfcontacturl={http://people.phys.ethz.ch/\xmptilde nbeisert/}}

\newcommand{\secref}[1]{\hyperref[#1]{section \ref*{#1}}}

\parskip1ex
\parindent0pt
\let\olditemize\itemize
\def\itemize{\olditemize\parskip0pt}

\begin{document}

\title{The \textsf{childdoc} Package}
\hypersetup{pdftitle={The childdoc Package}}
\author{Niklas Beisert\\[2ex]
  Institut f\"ur Theoretische Physik\\
  Eidgen\"ossische Technische Hochschule Z\"urich\\
  Wolfgang-Pauli-Strasse 27, 8093 Z\"urich, Switzerland\\[1ex]
  \href{mailto:nbeisert@itp.phys.ethz.ch}
  {\texttt{nbeisert@itp.phys.ethz.ch}}}
\hypersetup{pdfauthor={Niklas Beisert}}
\hypersetup{pdfsubject={Manual for the LaTeX2e Package childdoc}}
\date{30 December 2018, \textsf{v2.0}}
\maketitle

\begin{abstract}\noindent
\textsf{childdoc} is a \LaTeXe{} package
that enables the direct compilation
of document sections included by |\include|
to individual files.
\end{abstract}

\begingroup
\parskip0ex
\tableofcontents
\endgroup

%%%%%%%%%%%%%%%%%%%%%%%%%%%%%%%%%%%%%%%%%%%%%%%%%%%%%%%%%%%%%%%%%%%%%%%%%%%%%%%%
%%%%%%%%%%%%%%%%%%%%%%%%%%%%%%%%%%%%%%%%%%%%%%%%%%%%%%%%%%%%%%%%%%%%%%%%%%%%%%%%
\section{Introduction}

\LaTeX{} provides a mechanism to structure a large document (such as a book)
into a main file and several child files (containing the chapters)
using the |\include| command.
This mechanism is beneficial for documents
which span hundreds of pages in order to
make the source file(s) more manageable.
Moreover, compilation can be restricted to
selected child files by means of the |\includeonly| command.
The latter feature can be used to reduce the compilation time while editing
(this was significantly more useful in the earlier days of \LaTeX{})
or to generate a smaller document which is easier to navigate.
Another application of |\includeonly| is to generate
documents consisting of selected parts of the complete document.

However, there are a few drawbacks of the plain |\include| mechanism:
\begin{itemize}
\item
The child files cannot be compiled on their own,
they can only be compiled via the main file.
A naive editing environment
(such as a text editor with an option
to have the current file processed by \LaTeX)
may require one to switch to the main file before compiling;
attempting to compile the child file produces errors.
\item
The main file must be modified (each time)
to adjust the |\includeonly| command
to the present needs. This easily leaves the main file in a messy state.
\item
The generated document will always carry the filename
of the main document. This is inconvenient if
several child files are to be compiled and
to be kept for distribution.
\end{itemize}

The present package provides a simple interface
to make child files individually compilable by \LaTeX{}.
Compiling a child file then has the same effect as compiling
the main file with an |\includeonly| command
to select the appropriate child.
Moreover the generated document will carry the name of the child
rather than the main file.
This resolves all three above issues.

This feature is meant to make the editing of books,
thesis documents and lecture notes somewhat more convenient.
However, the package can also be used efficiently for
composing a series of documents (such as exercise sheets)
which are typically distributed individually.
It then assists the author in generating the individual documents
(potentially in different versions)
as well as a document containing the collected series.
Another application is in developing style files
or other kinds of included material
where compilation of the style file could redirect
to a sample or test file.

%%%%%%%%%%%%%%%%%%%%%%%%%%%%%%%%%%%%%%%%%%%%%%%%%%%%%%%%%%%%%%%%%%%%%%%%%%%%%%%%
%%%%%%%%%%%%%%%%%%%%%%%%%%%%%%%%%%%%%%%%%%%%%%%%%%%%%%%%%%%%%%%%%%%%%%%%%%%%%%%%
\section{Usage}

First of all, the package \textsf{childdoc} is \emph{not} a standard
\LaTeXe{} |.sty| style file! Therefore it needs to be invoked in
a non-standard way.

%%%%%%%%%%%%%%%%%%%%%%%%%%%%%%%%%%%%%%%%%%%%%%%%%%%%%%%%%%%%%%%%%%%%%%%%%%%%%%%%
\subsection{Included Files}
\label{sec:include}

%%%%%%%%%%%%%%%%%%%%%%%%%%%%%%%%%%%%%%%%
\DescribeMacro{\childdocmain}
To use the package, add the commands
\begin{center}
\begin{tabular}{l}
|\input{childdoc.def}|\\
|\childdocmain{}|\\
\end{tabular}
\end{center}
at the very top of the main \LaTeX{} file,
in particular \emph{before} the |\documentclass| statement!
The argument of |\childdocmain| should be left empty
(but it must be present).

%%%%%%%%%%%%%%%%%%%%%%%%%%%%%%%%%%%%%%%%
\DescribeMacro{\childdocof}
Furthermore, add the commands
\begin{center}
\begin{tabular}{l}
|\input{childdoc.def}|\\
|\childdocof{|\textit{main}|}|\\
\end{tabular}
\end{center}
at the top of every child file \textit{child}
which is included by |\include{|\textit{child}|}|
from within the main file
(or at least for those files to be compiled individually).
The argument \textit{main} must be the filename of the main file.

There are a couple of
considerations in setting up the main and child documents:

%%%%%%%%%%%%%%%%%%%%%%%%%%%%%%%%%%%%%%%%
\paragraph{Restrictions.}

Please note the following restrictions:
\begin{itemize}
\item
|\childdocmain| must be called with one argument \textit{main}
to ensure compatibility with earlier version of the package.
It must either be empty (|\childdocmain{}|)
or precisely match the filename of the main file in which it is specified.
See \secref{sec:detection} for further information.
\item
The filename \textit{main} must be specified without the |.tex| extension.
\item
The filename \textit{main} is case sensitive
(even in case-insensitive file systems)
due to internal string comparison.
\item
The argument \textit{main} should be fully expanded, it cannot be a macro.
\item
Subdirectories and special characters should be avoided in filenames.
\item
The command |\childdocmain{|\textit{main}|}| must be followed by a whitespace.
It should not be followed immediately by another command
or by a comment mark `|%|'.
This is because the \TeX{} parser reads the token immediately following
the argument of |\childdocmain| and puts it
at the beginning of every child section;
however, a white\-space is ignored.
\end{itemize}

%%%%%%%%%%%%%%%%%%%%%%%%%%%%%%%%%%%%%%%%
\paragraph{Content of Main File.}

It is advisable to place all content in the child files included by |\include|.
Any output contained in the main file will appear in all child documents
unless suppressed manually;
it cannot be suppressed automatically by the |\includeonly| directive
and thus should normally be avoided.
A method to include some content in the main file
by means of conditional processing is described in \secref{sec:conditional}.

%%%%%%%%%%%%%%%%%%%%%%%%%%%%%%%%%%%%%%%%
\paragraph{Page Numbering.}

When only a part of the document is compiled,
the appropriate numbering of pages
(as well as other status parameters)
is determined from the |.aux| files.
The latter contain information from previous passes.
However this information needs to propagate through
all intermediate child documents.
Therefore the page numbering in child documents may well
be inconsistent until the complete document is compiled at least once.

A useful (if unconventional) way to always ensure a consistent
page numbering is to restart the numbering in each child document
and denote the pages by `\textit{child}|.|\textit{page}'
where \textit{child} represents the chapter/section number of the child file.
This can be achieved by the command
|\numberwithin{page}{|\textit{child}|}|
of the \textsf{amsmath} package
where \textit{child} can be |chapter| or |section|
depending on the chosen structuring.
Alternatively, one can modify the macro |\thepage| appropriately
and reset the counter |page| at the start of each child file.

%%%%%%%%%%%%%%%%%%%%%%%%%%%%%%%%%%%%%%%%%%%%%%%%%%%%%%%%%%%%%%%%%%%%%%%%%%%%%%%%
\subsection{Conditional Processing}
\label{sec:conditional}

The package provides a mechanism to compile different versions
of a document. To customise the versions further some conditional processing
can come in handy to distinguish which version is being compiled.
The package provides two macros to describe the compilation context:

%%%%%%%%%%%%%%%%%%%%%%%%%%%%%%%%%%%%%%%%
\DescribeMacro{\ifchilddoc}
The conditional |\ifchilddoc| distinguishes between the compilation of
child documents and the main document:
%
\begin{center}
|\ifchilddoc |\textit{child-code}| |[|\||else |\textit{main-code}]| \||fi|
\end{center}

%%%%%%%%%%%%%%%%%%%%%%%%%%%%%%%%%%%%%%%%
\DescribeMacro{\childdocname}
\DescribeMacro{\childdocjob}
The macro |\childdocname| contains the filename (without extension)
of the main or child file being processed.
Note that |\childdocjob| will always contain the name of the main file.

%%%%%%%%%%%%%%%%%%%%%%%%%%%%%%%%%%%%%%%%
\paragraph{Title Page.}

Conditional processing can be used to include a title or banner page
in the main document when proper precautions are taken.
Importantly, the code in the main file should ensure that the page counter
(as well as other status parameters which are stored in the |.aux| files)
takes the same value after the conditional processing.
Otherwise the page numbers may take divergent values
depending on which part is compiled.

For example, a title page could be declared by:
%
\begin{center}
\begin{tabular}{l}
|\ifchilddoc\||else|\\
|\addtocounter{page}{-1}|\\
\textit{code for title page}\\
|\newpage|\\
|\||fi|
\end{tabular}
\end{center}
%
A banner page for the child documents can be generated by:
%
\begin{center}
\begin{tabular}{l}
|\ifchilddoc|\\
|\addtocounter{page}{-1}|\\
\textit{code for banner page}\\
|\newpage|\\
|\||fi|
\end{tabular}
\end{center}
%
Here one could write a message such as:
\begin{center}
|This is the part \childdocname{} of \childdocjob{}.|
\end{center}

%%%%%%%%%%%%%%%%%%%%%%%%%%%%%%%%%%%%%%%%%%%%%%%%%%%%%%%%%%%%%%%%%%%%%%%%%%%%%%%%
\subsection{Flags}
\label{sec:flags}

The package makes it easy to generate different versions
of the main or child documents.
To this end compilation flags can be defined
and assigned different default values.
They will be particularly useful in conjunction
with the forwarding mechanism described in \secref{sec:forward}.

For example, it may be useful to have a flag |\version|
which can be set to |draft| or |final|.
The document source will contain some conditional code
depending on the value of |\version|.
Suppose further, the flag should default to |final| for the main file
and to |draft| for child files
which is a natural assignment for editing the document.
This is achieved by placing the following code
in the preamble of the main document
(below the |\childdocmain| directive):
%
\begin{center}
\begin{tabular}{l}
|\ifchilddoc|\\
|\providecommand{\version}{draft}|\\
|\||else|\\
|\providecommand{\version}{final}|\\
|\||fi|
\end{tabular}
\end{center}
%
The definition by |\providecommand| makes sure
that previous definitions are not overwritten.
Further statements |\providecommand{\version}{...}|
can thus be added before the above code to override it.

For the main file, one might add a line
(between |\childdocmain| and the above block)
%
\begin{center}
|%\ifchilddoc\||else\providecommand{\version}{draft}\||fi|
\end{center}
%
which can be uncommented to produce a draft version.
Likewise one can add a line to the very top of a child file
(above the |\childdocof{|\textit{main}|}| directive)
%
\begin{center}
|%\providecommand{\version}{final}|
\end{center}
%
which can be uncommented to produce the final version of this child document.

%%%%%%%%%%%%%%%%%%%%%%%%%%%%%%%%%%%%%%%%%%%%%%%%%%%%%%%%%%%%%%%%%%%%%%%%%%%%%%%%
\subsection{Forwarding}
\label{sec:forward}

Different versions of the main or child documents
using compilation flags as described in \secref{sec:flags}
can be (permanently) stored in different files
for convenient compilation, viewing and distribution.
To this end, the package defines a command
to pass on compilation to a different file:

%%%%%%%%%%%%%%%%%%%%%%%%%%%%%%%%%%%%%%%%
\DescribeMacro{\childdocforward}
The command |\childdocforward| redirects processing to
another source file:
%
\begin{center}
\begin{tabular}{l}
|\input{childdoc.def}|\\
|\childdocforward[|\textit{main}|]{|\textit{dest}|}|\\
\end{tabular}
\end{center}
%
The argument \textit{dest} is the destination file
(without extension).
It should be the main file or one of the child files.
Note that further \textsf{childdoc} directives
such as |\childdocof| and |\childdocforward|
in the indicated file will be processed in this form.
The optional argument \textit{main}
passes on directly to the main file \textit{main}
while pretending to compile the child \textit{dest}.
This form behaves as if \textit{dest}
issues |\childdocof{|\textit{main}|}| right away,
and no further \textsf{childdoc} directives will be processed.

%%%%%%%%%%%%%%%%%%%%%%%%%%%%%%%%%%%%%%%%
\DescribeMacro{\...prefix}
In the alternative form |\childdocforwardprefix|,
%
\begin{center}
\begin{tabular}{l}
|\input{childdoc.def}|\\
|\childdocforwardprefix[|\textit{main}|]{|\textit{prefix}|}{|\textit{dest}|}|
\end{tabular}
\end{center}
%
the destination file is determined by a pattern
depending on the current file:
To make this work, the current file must be called
`{\textit{prefix}\hspace{0.2em}\textit{suffix}}'
with \textit{prefix} matching precisely the argument.
Processing is then passed on to the file
`{\textit{dest}\hspace{0.2em}\textit{suffix}}'.
Surely, the same effect is achieved by
directly specifying the
argument `{\textit{dest}\hspace{0.2em}\textit{suffix}}'
in the first form.
However, that requires to set up a different file
for each child. With the alternative form of the command
all these files can have exactly the same content
which simplifies setting them up and maintaining them.

For example, the following file |draft.tex|
with a compilation flag |\version| as described in \secref{sec:flags}
compiles the main document as a draft:
%
\begin{center}
\begin{tabular}{l}
|\def\version{draft}|\\
|\input{childdoc.def}|\\
|\childdocforward{|\textit{main}|}|
\end{tabular}
\end{center}
%
Likewise, the following files |final|\textit{nn}|.tex|
compile the final version of the child document
|child|\textit{nn}|.tex|:
%
\begin{center}
\begin{tabular}{l}
|\def\version{final}|\\
|\input{childdoc.def}|\\
|\childdocforwardprefix{final}{child}|
\end{tabular}
\end{center}
%

Note that when several versions of a main file and/or of each child file
are to be generated, it may be convenient to set up a |Makefile| or
shell script to automatise the process.

%%%%%%%%%%%%%%%%%%%%%%%%%%%%%%%%%%%%%%%%%%%%%%%%%%%%%%%%%%%%%%%%%%%%%%%%%%%%%%%%
\subsection{Command Line Processing}
\label{sec:commandline}

The effect of redirection files can also be achieved by invoking
the \LaTeX{} compiler with a more elaborate command line.
Most conveniently this should be done as part
of a shell script or a |Makefile|.

When using \textsf{childdoc} in the main file, the following
command lines effectively perform a redirection
(note that depending on the shell being used,
backslashes may have to be doubled: `|\|' $\to$ `|\\|'):
%
\begin{center}
|... -jobname "|\textit{target}|" |\\|"|[\textit{flags}]%
|\input{childdoc.def}\childdocforward[|\textit{main}|]{|\textit{dest}|}"|
\end{center}
%
Here \textit{target} is the name of the output file,
\textit{main} is the name of the main file
and \textit{dest} is the name of the main or child file to be processed
(all filenames without extensions).
The optional argument \textit{main} can be omitted
if \textit{main} matches \textit{dest}.
Optionally, compilation \textit{flags} can be defined via |\def| commands.
This command line makes the \TeX{} engine believe
it is compiling the file \textit{target}
whose content is specified as the latter parameter.
The provided code then forwards the processing to
\textit{main} or \textit{dest} as described in \secref{sec:forward}.

%%%%%%%%%%%%%%%%%%%%%%%%%%%%%%%%%%%%%%%%%%%%%%%%%%%%%%%%%%%%%%%%%%%%%%%%%%%%%%%%
\subsection{Include by Input}
\label{sec:input}

Including child documents by |\include| has some restrictions by design.
Most notably, the content of a child document always occupies
its own set of pages; pages cannot be shared between child documents.
Usually, this behaviour makes perfect sense
because each child document contain an essential part of the document.
However, in some situations it may be desirable to compose
a document from a collection of parts
without having mandatory page breaks between then.
For this case, the package
provides a mechanism to include parts
by |\input| which can also be processed individually.
However, by construction this mechanism
requires manual handling of the content to be output.

%%%%%%%%%%%%%%%%%%%%%%%%%%%%%%%%%%%%%%%%
\DescribeMacro{\ifchilddocmanual}
The main file should be prepared as usual, see \secref{sec:include}.
However, the document body must make a distinction
between processing of an individual part and of the main document, e.g.:
%
\begin{center}
\begin{tabular}{l}
|\ifchilddocmanual|\\
|\input{\childdocname}|\\
|\||else|\\
\textit{document body with }|\input{|\textit{part}|}|\\
|\||fi|
\end{tabular}
\end{center}
%
The conditional |\ifchilddocmanual| is true whenever
a part to be included by |\input| is being compiled,
and the name of the part is stored in |\childdocname|.

%%%%%%%%%%%%%%%%%%%%%%%%%%%%%%%%%%%%%%%%
\DescribeMacro{\childdocby}
Each part to be included by |\input| should start with:
%
\begin{center}
\begin{tabular}{l}
|\input{childdoc.def}|\\
|\childdocby{|\textit{main}|}|\\
\end{tabular}
\end{center}
%
The directive |\childdocby| is similar to |\childdocof|
described in \secref{sec:include},
but the subsequent selection of content must be done manually.
To that end, both |\ifchilddoc| and |\ifchilddocmanual|
will be true upon processing of a part,
and the name of the part is stored in |\childdocname|.
Note that |\jobname| will be set to the filename of the current part
so that each part receives an individual |.aux| file
that does not interfere with the |.aux| file(s) of the main document.
This behaviour can be altered by the alternative form
|\childdocby[*]{|\textit{main}|}| (with a non-empty optional argument)
which uses the |.aux| file of the main document
by setting |\jobname| to \textit{main}.

%%%%%%%%%%%%%%%%%%%%%%%%%%%%%%%%%%%%%%%%%%%%%%%%%%%%%%%%%%%%%%%%%%%%%%%%%%%%%%%%
\subsection{Driver Development}
\label{sec:driver}

The \textsf{childdoc} mechanism can also be use for the development
of definition files such as \LaTeX{} styles or classes.
This case differs from the above setup with multiple parts
included by |\include| in that no |\includeonly| should be invoked.
This can be achieved by starting the include file
(before |\ProvidesPackage|) with:
%
\begin{center}
\begin{tabular}{l}
|\input{childdoc.def}|\\
|\childdocforward{|\textit{main}|}|\\
\end{tabular}
\end{center}
%
or alternatively with:
%
\begin{center}
\begin{tabular}{l}
|\input{childdoc.def}|\\
|\childdocby{|\textit{main}|}|\\
\end{tabular}
\end{center}
%
Both forms have slightly different effects as described above.
The main file is prepared as usual, see \secref{sec:include}.

%%%%%%%%%%%%%%%%%%%%%%%%%%%%%%%%%%%%%%%%%%%%%%%%%%%%%%%%%%%%%%%%%%%%%%%%%%%%%%%%
\subsection{Legacy Detection}
\label{sec:detection}

The directive |\childdocmain| in the main file can detect
whether the complete document or merely a child is to be compiled
even without using the directive |\childdocof|.
This method is deprecated because it is less robust
and there is no compelling reason to use it;
it is merely provided for backward compatibility
and it may be removed in future versions.

If the detection mechanism is to be used,
it is mandatory to correctly specify
the filename of the main file as the argument of |\childdocmain|:
%
\begin{center}
\begin{tabular}{l}
|\input{childdoc.def}|\\
|\childdocmain{|\textit{main}|}|\\
\end{tabular}
\end{center}
%
If |\jobname| does not match the argument \textit{main} of |\childdocmain|,
it is assumed that |\jobname| points to the child file to be compiled.
When using |\childdocmain| with the main file specified as argument,
it suffices to start a child file
with just |\input{|\textit{main}|}|
without loading of the package and using |\childdocof|.
If instead all processing is done
with the appropriate \textsf{childdoc} directives,
the argument of \textit{main} of |\childdocmain| can be empty.

An alternative version of the command line processing described
in \secref{sec:commandline} using the detection mechanism reads:
%
\begin{center}
|... -jobname "|\textit{target}|" "|[\textit{flags}]%
[|\def\jobname{|\textit{dest}|}|]|\input{|\textit{main}|}"|
\end{center}

%%%%%%%%%%%%%%%%%%%%%%%%%%%%%%%%%%%%%%%%%%%%%%%%%%%%%%%%%%%%%%%%%%%%%%%%%%%%%%%%
\subsection{Manual Code}
\label{sec:manual}

In case one cannot be certain whether the definitions file |childdoc.def|
is installed on the target \TeX{} distribution
and one prefers not to ship it,
it is conceivable to paste a few relevant commands into the sources.

To that end, drop all statements |\input{childdoc.def}|
and perform the replacements as outlined below.
Instead of |\childdocmain{|\textit{main}|}| add the following code
to the top of the main file:
%
\begin{center}
\begin{tabular}{l}
|\||ifdefined\childdocname\endinput\||fi\newif\ifchilddoc|\\
|\edef\childdocname{\scantokens\expandafter{\jobname\noexpand}}|\\
|\def\childdocmain{|\textit{main}|}\||ifx\childdocmain\childdocname\||else|\\
|\childdoctrue\includeonly{\childdocname}\let\jobname\childdocmain\||fi|\\
\end{tabular}
\end{center}
%
Instead of |\childdocof{|\textit{main}|}| just include the main file
at the top of each child file:
%
\begin{center}
|\input{|\textit{main}|}|
\end{center}
%
A simple redirection |\childdocforward{|\textit{dest}|}| is achieved by:
%
\begin{center}
|\def\jobname{|\textit{dest}|}\input{\jobname}|
\end{center}
%
The redirection with prefix
|\childdocforwardprefix[|\textit{prefix}|]{|\textit{dest}|}|
is accomplished by:
%
\begin{center}
\begin{tabular}{l}
|{\edef\jobname{\scantokens\expandafter{\jobname\noexpand}}|\\
|\def\redirectjob |\textit{prefix}|#1~~~{\gdef\jobname{|\textit{dest}|#1}}|\\
|\expandafter\redirectjob\jobname~~~}\input{\jobname}|
\end{tabular}
\end{center}

In an alternative approach,
child documents can be compiled by a specific command line
without additional code or specific definitions:
%
\begin{center}
|... -jobname "|\textit{target}|" "|[\textit{flags}]%
|\includeonly{|\textit{dest}|}\input{|\textit{main}|}"|
\end{center}
%

%%%%%%%%%%%%%%%%%%%%%%%%%%%%%%%%%%%%%%%%%%%%%%%%%%%%%%%%%%%%%%%%%%%%%%%%%%%%%%%%
%%%%%%%%%%%%%%%%%%%%%%%%%%%%%%%%%%%%%%%%%%%%%%%%%%%%%%%%%%%%%%%%%%%%%%%%%%%%%%%%
\section{Information}

%%%%%%%%%%%%%%%%%%%%%%%%%%%%%%%%%%%%%%%%%%%%%%%%%%%%%%%%%%%%%%%%%%%%%%%%%%%%%%%%
\subsection{Copyright}

Copyright \copyright{} 2017--2018 Niklas Beisert

This work may be distributed and/or modified under the
conditions of the \LaTeX{} Project Public License, either version 1.3
of this license or (at your option) any later version.
The latest version of this license is in
  \url{http://www.latex-project.org/lppl.txt}
and version 1.3 or later is part of all distributions of \LaTeX{}
version 2005/12/01 or later.

This work has the LPPL maintenance status `maintained'.

The Current Maintainer of this work is Niklas Beisert.

This work consists of the files |README.txt|, |childdoc.ins| and |childdoc.dtx|
as well as the derived files |childdoc.def|, |cdocsamp.tex|
with |cdocsch1.tex|, |cdocsch2.tex|, |cdocspt3.tex|, |cdocspt4.tex|,
|cdocsdrf.tex|, |cdocsfn1.tex|, |cdocsfn2.tex|
as well as |childdoc.pdf|.

%%%%%%%%%%%%%%%%%%%%%%%%%%%%%%%%%%%%%%%%%%%%%%%%%%%%%%%%%%%%%%%%%%%%%%%%%%%%%%%%
\subsection{Files and Installation}

The package consists of the files:
%
\begin{center}
\begin{tabular}{ll}
    |README.txt|   & readme file \\
    |childdoc.ins| & installation file \\
    |childdoc.dtx| & source file \\
    |childdoc.def| & definition file \\
    |cdocsamp.tex| & sample main file \\
    |cdocsch1.tex| & sample include file \\
    |cdocsch2.tex| & sample include file \\
    |cdocspt3.tex| & sample part file \\
    |cdocspt4.tex| & sample part file \\
    |cdocsdrf.tex| & sample redirection file \\
    |cdocsfn1.tex| & sample redirection file \\
    |cdocsfn2.tex| & sample redirection file \\
    |childdoc.pdf| & manual
\end{tabular}
\end{center}
%
The distribution consists of the files
|README.txt|, |childdoc.ins| and |childdoc.dtx|.
%
\begin{itemize}
\item
Run (pdf)\LaTeX{} on |childdoc.dtx|
to compile the manual |childdoc.pdf| (this file).
\item
Run \LaTeX{} on |childdoc.ins| to create the definitions file |childdoc.def|
and the sample |cdocsamp.tex| with include files
|cdocsch1.tex|, |cdocsch2.tex|, |cdocspt3.tex|, |cdocspt4.tex|,
|cdocsdrf.tex|, |cdocsfn1.tex|, |cdocsfn2.tex|.
Then copy the file |childdoc.def| to an appropriate directory of your \LaTeX{}
distribution, e.g.\ \textit{texmf-root}|/tex/latex/childdoc|.
\end{itemize}

%%%%%%%%%%%%%%%%%%%%%%%%%%%%%%%%%%%%%%%%%%%%%%%%%%%%%%%%%%%%%%%%%%%%%%%%%%%%%%%%
\subsection{Related CTAN Packages}

There are several other packages which offer a similar functionality:
%
\begin{itemize}
\item
The packages
\href{http://ctan.org/pkg/docmute}{\textsf{docmute}},
\href{http://ctan.org/pkg/includex}{\textsf{includex}} and
\href{http://ctan.org/pkg/standalone}{\textsf{standalone}}
provide commands to include only the document body of
a child file thus allowing both files to be compiled individually.
\item
The packages \href{http://ctan.org/pkg/subdocs}{\textsf{subdocs}}
and \href{http://ctan.org/pkg/subfiles}{\textsf{subfiles}}
provide structures in which the main and child documents can be
encapsulated and allowing them to be compiled individually.
The inclusion mechanism is different from the conventional |\include|.
\item
The package \href{http://ctan.org/pkg/combine}{\textsf{combine}}
is an elaborate solution to combine several documents into one.
\end{itemize}
%
See also the CTAN topic \href{http://ctan.org/topic/subdocs}{\textsf{subdocs}}
for further related packages.
The present package differs from the above solutions in that
a document structure constructed with the conventional |\include| mechanism
just needs two extra commands at the top of every file
such that all constituent files can be compiled individually.

%%%%%%%%%%%%%%%%%%%%%%%%%%%%%%%%%%%%%%%%%%%%%%%%%%%%%%%%%%%%%%%%%%%%%%%%%%%%%%%%
%\subsection{Feature Suggestions}
%
%The following is a list of features which may be useful for future
%versions of this package:
%%
%\begin{itemize}
%\item
%\ldots
%\end{itemize}

%%%%%%%%%%%%%%%%%%%%%%%%%%%%%%%%%%%%%%%%%%%%%%%%%%%%%%%%%%%%%%%%%%%%%%%%%%%%%%%%
\subsection{Revision History}

%%%%%%%%%%%%%%%%%%%%%%%%%%%%%%%%%%%%%%%%
\paragraph{v2.0:} 2018/12/30

\begin{itemize}
\item
immediate forward processing
\item
added |\childdocby| mechanism
\item
manual restructured
\end{itemize}

%%%%%%%%%%%%%%%%%%%%%%%%%%%%%%%%%%%%%%%%
\paragraph{v1.6:} 2018/01/17

\begin{itemize}
\item
application for development of include files
\item
corrections to manual
\end{itemize}

%%%%%%%%%%%%%%%%%%%%%%%%%%%%%%%%%%%%%%%%
\paragraph{v1.5:} 2017/05/21

\begin{itemize}
\item
more complete structuring introduced
\item
|\childdocof| introduced
\item
|\childdoc| renamed to |\childdocmain|
\item
|\childredirect| renamed to |\childdocforward| and |\childdocforwardprefix|
and functionality expanded
\end{itemize}

%%%%%%%%%%%%%%%%%%%%%%%%%%%%%%%%%%%%%%%%
\paragraph{v1.0:} 2017/04/27

\begin{itemize}
\item
manual and install package
\item
first version published on CTAN
\end{itemize}

%%%%%%%%%%%%%%%%%%%%%%%%%%%%%%%%%%%%%%%%
\paragraph{v0.6:} 2017/04/26

\begin{itemize}
\item
redirection mechanism added
\end{itemize}

%%%%%%%%%%%%%%%%%%%%%%%%%%%%%%%%%%%%%%%%
\paragraph{v0.5:} 2017/04/26

\begin{itemize}
\item
functionality in definition file
\end{itemize}


%%%%%%%%%%%%%%%%%%%%%%%%%%%%%%%%%%%%%%%%%%%%%%%%%%%%%%%%%%%%%%%%%%%%%%%%%%%%%%%%
%%%%%%%%%%%%%%%%%%%%%%%%%%%%%%%%%%%%%%%%%%%%%%%%%%%%%%%%%%%%%%%%%%%%%%%%%%%%%%%%
%%%%%%%%%%%%%%%%%%%%%%%%%%%%%%%%%%%%%%%%%%%%%%%%%%%%%%%%%%%%%%%%%%%%%%%%%%%%%%%%
\appendix

\settowidth\MacroIndent{\rmfamily\scriptsize 000\ }

 \DocInput{childdoc.dtx}

\end{document}
%</driver>
% \fi
%
% %%%%%%%%%%%%%%%%%%%%%%%%%%%%%%%%%%%%%%%%%%%%%%%%%%%%%%%%%%%%%%%%%%%%%%%%%%%%%%
% %%%%%%%%%%%%%%%%%%%%%%%%%%%%%%%%%%%%%%%%%%%%%%%%%%%%%%%%%%%%%%%%%%%%%%%%%%%%%%
% \section{Sample}
%\iffalse
%<*samplemain>
%\fi
%
% The following presents a sample document
% with two chapters, two parts, a title page,
% a compile flag as well as three forwarding files to set the flag.
% It consists of eight |.tex| files:
% \begin{center}
% \begin{tabular}{ll}
% |cdocsamp.tex|&main file\\
% |cdocsch1.tex|&include file for chapter 1\\
% |cdocsch2.tex|&include file for chapter 2\\
% |cdocspt3.tex|&include file for part 3\\
% |cdocspt4.tex|&include file for part 4\\
% |cdocsdrf.tex|&forwarding file for main file in draft mode\\
% |cdocsfi1.tex|&forwarding file for final version of chapter 1\\
% |cdocsfi2.tex|&forwarding file for final version of chapter 2\\
% \end{tabular}
% \end{center}
% Each of the eight files can be compiled directly by the \LaTeX{} compiler.
%
% %%%%%%%%%%%%%%%%%%%%%%%%%%%%%%%%%%%%%%
% \paragraph{Main File.}
%
% The main file is called |cdocsamp.tex|.
%
% Load the \textsf{childdoc} definitions and
% declare the filename for the main document:
%    \begin{macrocode}
\input{childdoc.def}
\childdocmain{}
%    \end{macrocode}

% Optional override for |\version| flag:
%    \begin{macrocode}
%%\ifchilddoc\else\providecommand{\version}{draft}\fi
%    \end{macrocode}

% Define the default values for the |\version| flag
% (|final| for the main file and |draft| for childs):
%    \begin{macrocode}
\ifchilddoc
\providecommand{\version}{draft}
\else
\providecommand{\version}{final}
\fi
%    \end{macrocode}

% Load the standard document class:
%    \begin{macrocode}
\documentclass[12pt]{article}
%    \end{macrocode}

% Start the document body:
%    \begin{macrocode}
\begin{document}
%    \end{macrocode}

% Declare a title page.
% Print title, part of document being processed and version flag:
%    \begin{macrocode}
\addtocounter{page}{-1}
\begin{center}
{\LARGE\bfseries{}childdoc example\par}
\vspace{1cm}
\ifchilddoc
\ifchilddocmanual part\else chapter\fi:
`\childdocname' of `\childdocjob'\par
\else
main document: `\childdocjob'\par
\fi
version: \version\par
\end{center}
\newpage
%    \end{macrocode}

% Manually include selected file,
% otherwise process as usual:
%    \begin{macrocode}
\ifchilddocmanual
\section*{part `\childdocname'}
\input{\childdocname}
\else
%    \end{macrocode}

% Include the two chapters:
%    \begin{macrocode}
\include{cdocsch1}
\include{cdocsch2}
%    \end{macrocode}

% Include the two parts unless only chapters should be displayed:
%    \begin{macrocode}
\ifchilddoc\else
\section{part three}
\input{cdocspt3}
\section{part four}
\input{cdocspt4}
\fi
%    \end{macrocode}

% Process as usual until here:
%    \begin{macrocode}
\fi
%    \end{macrocode}

% End of document body:
%    \begin{macrocode}
\end{document}
%    \end{macrocode}
%\iffalse
%</samplemain>
%\fi
%
% %%%%%%%%%%%%%%%%%%%%%%%%%%%%%%%%%%%%%%
% \paragraph{Chapter Include Files.}
%
% The include files are called |cdocsch1.tex| and |cdocsch2.tex|.
%
%\iffalse
%<*samplechap1|samplechap2>
%\fi

% Optional override for |\version| flag:
%    \begin{macrocode}
%%\providecommand{\version}{final}
%    \end{macrocode}

% Include the main document:
%    \begin{macrocode}
\input{childdoc.def}
\childdocof{cdocsamp}
%    \end{macrocode}

%\iffalse
%</samplechap1|samplechap2>
%\fi
%
%\iffalse
%<*samplechap1>
%\fi
% Some text for chapter 1:
%    \begin{macrocode}
\section{one}
some text in chapter one
%    \end{macrocode}

%\iffalse
%</samplechap1>
%\fi
% Some text for chapter 2:
%\iffalse
%<*samplechap2>
%\fi
%    \begin{macrocode}
\section{two}
more text in chapter two
%    \end{macrocode}

%\iffalse
%</samplechap2>
%\fi
%
% %%%%%%%%%%%%%%%%%%%%%%%%%%%%%%%%%%%%%%
% \paragraph{Part Include Files.}
%
% The include files are called |cdocspt3.tex| and |cdocspt4.tex|.
%
%\iffalse
%<*samplepart3|samplepart4>
%\fi

% Optional override for |\version| flag:
%    \begin{macrocode}
%%\providecommand{\version}{final}
%    \end{macrocode}

% Include the main document:
%    \begin{macrocode}
\input{childdoc.def}
\childdocby{cdocsamp}
%    \end{macrocode}

%\iffalse
%</samplepart3|samplepart4>
%\fi
%
%\iffalse
%<*samplepart3>
%\fi
% Some text for part 3:
%    \begin{macrocode}
some text in part three
%    \end{macrocode}

%\iffalse
%</samplepart3>
%\fi
% Some text for part 4:
%\iffalse
%<*samplepart4>
%\fi
%    \begin{macrocode}
more text in part four
%    \end{macrocode}

%\iffalse
%</samplepart4>
%\fi
%
% %%%%%%%%%%%%%%%%%%%%%%%%%%%%%%%%%%%%%%
% \paragraph{Forwarding for a Complete Draft.}
%
% The following forwarding file |cdocsdrf.tex|
% compiles the main document in draft mode:
%\iffalse
%<*sampledraft>
%\fi
%    \begin{macrocode}
\def\version{draft}
\input{childdoc.def}
\childdocforward{cdocsamp}
%    \end{macrocode}

%\iffalse
%</sampledraft>
%\fi
%
% %%%%%%%%%%%%%%%%%%%%%%%%%%%%%%%%%%%%%%
% \paragraph{Forwarding for Final Version of the Chapters.}
%
% The following forwarding files |cdocsfn1.tex| and |cdocsfn2.tex|
% (with identical content)
% compile the final versions of the child documents
% |cdocsch1.tex| and |cdocsch2.tex|, respectively:
%\iffalse
%<*samplefinal>
%\fi
%    \begin{macrocode}
\def\version{final}
\input{childdoc.def}
\childdocforwardprefix[cdocsamp]{cdocsfn}{cdocsch}
%    \end{macrocode}

%\iffalse
%</samplefinal>
%\fi
%
% %%%%%%%%%%%%%%%%%%%%%%%%%%%%%%%%%%%%%%
% \paragraph{Command Line Processing.}
%
% The following three command lines generate the output files
% |cdocscld|, |cdocscl1| and |cdocscl2|
% which should be identical to
% |cdocsdrf|, |cdocsch1| and |cdocsfn2|, respectively:
% \begin{center}
% \begin{tabular}{l}
% |latex -jobname cdocscld \|\\
% |  "\def\version{draft}\input{childdoc.def}\childdocforward{cdocsamp}"|\\
% |latex -jobname cdocscl1 \|\\
% |  "\input{childdoc.def}\childdocforward[cdocsamp]{cdocsch1}"|\\
% |latex -jobname cdocscl2 \|\\
% |  "\def\version{final}\input{childdoc.def}\childdocforward{cdocsch2}"|
% \end{tabular}
% \end{center}
% Note that the trailing backslash on each first line
% merely continues the input to the second line
% (for convenient cut ant paste).
% Furthermore, the command |latex| can be replaced by any
% of its alternative versions such as |pdflatex|.
%
% %%%%%%%%%%%%%%%%%%%%%%%%%%%%%%%%%%%%%%%%%%%%%%%%%%%%%%%%%%%%%%%%%%%%%%%%%%%%%%
% %%%%%%%%%%%%%%%%%%%%%%%%%%%%%%%%%%%%%%%%%%%%%%%%%%%%%%%%%%%%%%%%%%%%%%%%%%%%%%
% \section{Implementation}
%\iffalse
%<*package>
%\fi
%
% This section describes the definitions file |childdoc.def|.

% The definitions cannot be loaded using |\usepackage| or |\RequirePackage|
% which has a mechanism to prevent loading a style file more than once.
% When loading the definitions by means of |\input|
% multiple instances have to be prevented manually:
%\iffalse
%This code needs to be before the `\ProvidesFile' directive
%which is defined at the beginning of this file.
%Therefore it is also placed there and commented out here.
%</package>
%<*discard>
%\fi
%    \begin{macrocode}
\ifdefined\childdocmain\endinput\fi
%    \end{macrocode}
%\iffalse
%</discard>
%<*package>
%\fi
%
% \macro{\ifchilddoc}
% \macro{\ifchilddocmanual}
% The conditional |\ifchilddoc| tells whether a
% child (true) or main (false) document is being compiled.
% The conditional |\ifchilddocmanual| tells whether
% the |\includeonly| mechanism is used (false) or
% the selection of child files must be performed manually (true).
% The definitions initialise to false:
%    \begin{macrocode}
\newif\ifchilddoc
\newif\ifchilddocmanual
%    \end{macrocode}

% \macro{\childdocname}
% \macro{\childdocjob}
% The macro |\childdocname| stores the name of the main document
% to be compiled. The macro |\childdocjob| stores the name of
% the document on which the \LaTeX{} compiler was originally invoked.
% The content of |\jobname| cannot be compared
% to filenames specified in the source due to different catcodes.
% The following code rescans |\jobname|, stores the result
% in |\childdocname| and saves a copy in |\childdocjob|:
%    \begin{macrocode}
\edef\childdocname{\scantokens\expandafter{\jobname\noexpand}}
\let\childdocjob\childdocname
%    \end{macrocode}

% \macro{\childdocdisable}
% The macro |\childdocdisable| prevents the main file
% from being processed more than once.
% At this stage, the main document command |\childdocmain|
% is assumed to be called once again where it should do nothing.
% Any subsequent call to it should prevent
% a secondary processing of the main document
% It overwrites the forwarding commands
% |\childdocof| and |\childdocforward|
% with empty macros to prevent further inclusions of the main document:
%    \begin{macrocode}
\newcommand{\childdocdisable}
{
  \renewcommand{\childdocmain}[1]{\renewcommand{\childdocmain}[1]{\endinput}}
  \renewcommand{\childdocof}[1]{}
  \renewcommand{\childdocby}[2][]{}
  \renewcommand{\childdocforward}[2][]{}
  \renewcommand{\childdocdisable}{}
}
%    \end{macrocode}

% \macro{\childdocmain}
% The macro |\childdocmain| is to be called at the top of the main file
% with nothing or the main filename (without extension) as argument.
% First, it breaks loops.
% If the argument is not empty and does not match |\childdocname|
% (which is set by the first inclusion of |childdoc.def|),
% |\ifchilddoc| is set to true, |\includeonly| is applied to the child file
% and |\jobname| is set to the main file
% (for proper handling of |.aux| files):
%    \begin{macrocode}
\newcommand{\childdocmain}[1]
{
  \childdocdisable\childdocmain{}
  \if?#1?\else
    \begingroup
      \def\childdoctmp{#1}
      \ifx\childdoctmp\childdocname
        \def\childdoctmp{}
      \else
        \def\childdoctmp
        {
          \childdoctrue
          \includeonly{\childdocname}
          \def\childdocjob{#1}
          \def\jobname{#1}
        }
      \fi
      \expandafter
    \endgroup
    \childdoctmp
  \fi
}
%    \end{macrocode}

% \macro{\childdocof}
% The command |\childdocof| redirects
% compilation to the main file |#1|.
%    \begin{macrocode}
\newcommand{\childdocof}[1]
{
  \childdocdisable
  \childdoctrue
  \includeonly{\childdocname}
  \def\jobname{#1}
  \def\childdocjob{#1}
  \input{#1}
}
%    \end{macrocode}

% \macro{\childdocby}
% The command |\childdocby| ....
%    \begin{macrocode}
\newcommand{\childdocby}[2][]
{
  \childdocdisable
  \childdoctrue
  \childdocmanualtrue
  \if?#1?\else
    \def\jobname{#2}
  \fi
  \def\childdocjob{#2}
  \input{#2}
  \endinput
}
%    \end{macrocode}

% \macro{\childdocforward}
% The command |\childdocforward| redirects
% compilation to the main file or
% (if the optional argument is given) a child file.
% Parameters are set as if the main file
% or a child file starting with |\childdocof| was compiled.
% Then compilation is handed over to the main file:
%    \begin{macrocode}
\newcommand{\childdocforward}[2][]
{
  \begingroup
    \if?#1?
      \def\childdoctmp
      {
        \def\childdocname{#2}
        \def\childdocjob{#2}
        \def\jobname{#2}
        \input{#2}
        \endinput
      }
    \else
      \def\childdoctmp
      {
        \childdocdisable
        \def\childdocname{#2}
        \childdoctrue
        \includeonly{#2}
        \def\childdocjob{#1}
        \def\jobname{#1}
        \input{#1}
        \endinput
      }
    \fi
    \expandafter
  \endgroup
  \childdoctmp
}
%    \end{macrocode}

% \macro{\childdocforwardprefix}
% The command |\childdocforwardprefix| redirects
% compilation to the main or a child file by means of a pattern.
% The prefix |#1| in the current filename is replaced by |#2|
% and the suffix of the current filename is kept
% (it is assumed that the filename does not contain the substring `|~~~|'
% which is used as a delimiter).
% Compilation is handed over to the new file by |\childdocforward|:
%    \begin{macrocode}
\newcommand{\childdocforwardprefix}[3][]
{
  \begingroup
    \def\childdocextract #2##1~~~{\def\childdoctmp{\childdocforward[#1]{#3##1}}}
    \expandafter\childdocextract\childdocname~~~
    \expandafter
  \endgroup
  \childdoctmp
}
%    \end{macrocode}

% \macro{\childdoc}
% The deprecated macro |\childdoc| is a legacy version of |\childdocmain|:
%    \begin{macrocode}
\newcommand{\childdoc}{\childdocmain}
%    \end{macrocode}

% \macro{\childdocredirect}
% The deprecated macro |\childdocredirect| is a legacy version
% of |\childdocforward| and |\childdocforwardprefix|:
%    \begin{macrocode}
\newcommand{\childdocredirect}[2][]
{
  \begingroup
    \if?#1?
      \def\childdoctmp{\childdocforward{#2}}
    \else
      \def\childdoctmp{\childdocforwardprefix{#1}{#2}}
    \fi
    \expandafter
  \endgroup
  \childdoctmp
}
%    \end{macrocode}

%\iffalse
%</package>
%\fi
%
\endinput
|\\
|\childdocby{|\textit{main}|}|\\
\end{tabular}
\end{center}
%
The directive |\childdocby| is similar to |\childdocof|
described in \secref{sec:include},
but the subsequent selection of content must be done manually.
To that end, both |\ifchilddoc| and |\ifchilddocmanual|
will be true upon processing of a part,
and the name of the part is stored in |\childdocname|.
Note that |\jobname| will be set to the filename of the current part
so that each part receives an individual |.aux| file
that does not interfere with the |.aux| file(s) of the main document.
This behaviour can be altered by the alternative form
|\childdocby[*]{|\textit{main}|}| (with a non-empty optional argument)
which uses the |.aux| file of the main document
by setting |\jobname| to \textit{main}.

%%%%%%%%%%%%%%%%%%%%%%%%%%%%%%%%%%%%%%%%%%%%%%%%%%%%%%%%%%%%%%%%%%%%%%%%%%%%%%%%
\subsection{Driver Development}
\label{sec:driver}

The \textsf{childdoc} mechanism can also be use for the development
of definition files such as \LaTeX{} styles or classes.
This case differs from the above setup with multiple parts
included by |\include| in that no |\includeonly| should be invoked.
This can be achieved by starting the include file
(before |\ProvidesPackage|) with:
%
\begin{center}
\begin{tabular}{l}
|% \iffalse
%
% childdoc.dtx Copyright (C) 2017-2018 Niklas Beisert
%
% This work may be distributed and/or modified under the
% conditions of the LaTeX Project Public License, either version 1.3
% of this license or (at your option) any later version.
% The latest version of this license is in
%   http://www.latex-project.org/lppl.txt
% and version 1.3 or later is part of all distributions of LaTeX
% version 2005/12/01 or later.
%
% This work has the LPPL maintenance status `maintained'.
%
% The Current Maintainer of this work is Niklas Beisert.
%
% This work consists of the files childdoc.dtx and childdoc.ins
% and the derived files childdoc.def and cdocsamp.tex with
% cdocsch1.tex, cdocsch2.tex, cdocsdrf.tex, cdocsfn1.tex, cdocsfn2.tex.
%
%<package>\ifdefined\childdocmain\endinput\fi
%<package>\ProvidesFile{childdoc.def}[2018/12/30 v2.0 child document driver]
%<samplemain>\ProvidesFile{cdocsamp.tex}[2018/12/30 v2.0 sample for childdoc]
%<*driver>
%\ProvidesFile{childdoc.drv}[2018/12/30 v2.0 childdoc reference manual file]
\PassOptionsToClass{10pt,a4paper}{article}
\documentclass{ltxdoc}

\usepackage[margin=35mm]{geometry}
\usepackage{hyperref}
\usepackage{hyperxmp}
\usepackage[usenames]{color}

\hypersetup{colorlinks=true}
\hypersetup{pdfstartview=FitH}
\hypersetup{pdfpagemode=UseNone}
\hypersetup{pdfsource={}}
\hypersetup{pdflang={en-UK}}
\hypersetup{pdfcopyright={Copyright 2017-2018 Niklas Beisert.
  This work may be distributed and/or modified under the
  conditions of the LaTeX Project Public License, either version 1.3
  of this license or (at your option) any later version.}}
\hypersetup{pdflicenseurl={http://www.latex-project.org/lppl.txt}}
\hypersetup{pdfcontactaddress={ETH Zurich, ITP, HIT K,
  Wolfgang-Pauli-Strasse 27}}
\hypersetup{pdfcontactpostcode={8093}}
\hypersetup{pdfcontactcity={Zurich}}
\hypersetup{pdfcontactcountry={Switzerland}}
\hypersetup{pdfcontactemail={nbeisert@itp.phys.ethz.ch}}
\hypersetup{pdfcontacturl={http://people.phys.ethz.ch/\xmptilde nbeisert/}}

\newcommand{\secref}[1]{\hyperref[#1]{section \ref*{#1}}}

\parskip1ex
\parindent0pt
\let\olditemize\itemize
\def\itemize{\olditemize\parskip0pt}

\begin{document}

\title{The \textsf{childdoc} Package}
\hypersetup{pdftitle={The childdoc Package}}
\author{Niklas Beisert\\[2ex]
  Institut f\"ur Theoretische Physik\\
  Eidgen\"ossische Technische Hochschule Z\"urich\\
  Wolfgang-Pauli-Strasse 27, 8093 Z\"urich, Switzerland\\[1ex]
  \href{mailto:nbeisert@itp.phys.ethz.ch}
  {\texttt{nbeisert@itp.phys.ethz.ch}}}
\hypersetup{pdfauthor={Niklas Beisert}}
\hypersetup{pdfsubject={Manual for the LaTeX2e Package childdoc}}
\date{30 December 2018, \textsf{v2.0}}
\maketitle

\begin{abstract}\noindent
\textsf{childdoc} is a \LaTeXe{} package
that enables the direct compilation
of document sections included by |\include|
to individual files.
\end{abstract}

\begingroup
\parskip0ex
\tableofcontents
\endgroup

%%%%%%%%%%%%%%%%%%%%%%%%%%%%%%%%%%%%%%%%%%%%%%%%%%%%%%%%%%%%%%%%%%%%%%%%%%%%%%%%
%%%%%%%%%%%%%%%%%%%%%%%%%%%%%%%%%%%%%%%%%%%%%%%%%%%%%%%%%%%%%%%%%%%%%%%%%%%%%%%%
\section{Introduction}

\LaTeX{} provides a mechanism to structure a large document (such as a book)
into a main file and several child files (containing the chapters)
using the |\include| command.
This mechanism is beneficial for documents
which span hundreds of pages in order to
make the source file(s) more manageable.
Moreover, compilation can be restricted to
selected child files by means of the |\includeonly| command.
The latter feature can be used to reduce the compilation time while editing
(this was significantly more useful in the earlier days of \LaTeX{})
or to generate a smaller document which is easier to navigate.
Another application of |\includeonly| is to generate
documents consisting of selected parts of the complete document.

However, there are a few drawbacks of the plain |\include| mechanism:
\begin{itemize}
\item
The child files cannot be compiled on their own,
they can only be compiled via the main file.
A naive editing environment
(such as a text editor with an option
to have the current file processed by \LaTeX)
may require one to switch to the main file before compiling;
attempting to compile the child file produces errors.
\item
The main file must be modified (each time)
to adjust the |\includeonly| command
to the present needs. This easily leaves the main file in a messy state.
\item
The generated document will always carry the filename
of the main document. This is inconvenient if
several child files are to be compiled and
to be kept for distribution.
\end{itemize}

The present package provides a simple interface
to make child files individually compilable by \LaTeX{}.
Compiling a child file then has the same effect as compiling
the main file with an |\includeonly| command
to select the appropriate child.
Moreover the generated document will carry the name of the child
rather than the main file.
This resolves all three above issues.

This feature is meant to make the editing of books,
thesis documents and lecture notes somewhat more convenient.
However, the package can also be used efficiently for
composing a series of documents (such as exercise sheets)
which are typically distributed individually.
It then assists the author in generating the individual documents
(potentially in different versions)
as well as a document containing the collected series.
Another application is in developing style files
or other kinds of included material
where compilation of the style file could redirect
to a sample or test file.

%%%%%%%%%%%%%%%%%%%%%%%%%%%%%%%%%%%%%%%%%%%%%%%%%%%%%%%%%%%%%%%%%%%%%%%%%%%%%%%%
%%%%%%%%%%%%%%%%%%%%%%%%%%%%%%%%%%%%%%%%%%%%%%%%%%%%%%%%%%%%%%%%%%%%%%%%%%%%%%%%
\section{Usage}

First of all, the package \textsf{childdoc} is \emph{not} a standard
\LaTeXe{} |.sty| style file! Therefore it needs to be invoked in
a non-standard way.

%%%%%%%%%%%%%%%%%%%%%%%%%%%%%%%%%%%%%%%%%%%%%%%%%%%%%%%%%%%%%%%%%%%%%%%%%%%%%%%%
\subsection{Included Files}
\label{sec:include}

%%%%%%%%%%%%%%%%%%%%%%%%%%%%%%%%%%%%%%%%
\DescribeMacro{\childdocmain}
To use the package, add the commands
\begin{center}
\begin{tabular}{l}
|\input{childdoc.def}|\\
|\childdocmain{}|\\
\end{tabular}
\end{center}
at the very top of the main \LaTeX{} file,
in particular \emph{before} the |\documentclass| statement!
The argument of |\childdocmain| should be left empty
(but it must be present).

%%%%%%%%%%%%%%%%%%%%%%%%%%%%%%%%%%%%%%%%
\DescribeMacro{\childdocof}
Furthermore, add the commands
\begin{center}
\begin{tabular}{l}
|\input{childdoc.def}|\\
|\childdocof{|\textit{main}|}|\\
\end{tabular}
\end{center}
at the top of every child file \textit{child}
which is included by |\include{|\textit{child}|}|
from within the main file
(or at least for those files to be compiled individually).
The argument \textit{main} must be the filename of the main file.

There are a couple of
considerations in setting up the main and child documents:

%%%%%%%%%%%%%%%%%%%%%%%%%%%%%%%%%%%%%%%%
\paragraph{Restrictions.}

Please note the following restrictions:
\begin{itemize}
\item
|\childdocmain| must be called with one argument \textit{main}
to ensure compatibility with earlier version of the package.
It must either be empty (|\childdocmain{}|)
or precisely match the filename of the main file in which it is specified.
See \secref{sec:detection} for further information.
\item
The filename \textit{main} must be specified without the |.tex| extension.
\item
The filename \textit{main} is case sensitive
(even in case-insensitive file systems)
due to internal string comparison.
\item
The argument \textit{main} should be fully expanded, it cannot be a macro.
\item
Subdirectories and special characters should be avoided in filenames.
\item
The command |\childdocmain{|\textit{main}|}| must be followed by a whitespace.
It should not be followed immediately by another command
or by a comment mark `|%|'.
This is because the \TeX{} parser reads the token immediately following
the argument of |\childdocmain| and puts it
at the beginning of every child section;
however, a white\-space is ignored.
\end{itemize}

%%%%%%%%%%%%%%%%%%%%%%%%%%%%%%%%%%%%%%%%
\paragraph{Content of Main File.}

It is advisable to place all content in the child files included by |\include|.
Any output contained in the main file will appear in all child documents
unless suppressed manually;
it cannot be suppressed automatically by the |\includeonly| directive
and thus should normally be avoided.
A method to include some content in the main file
by means of conditional processing is described in \secref{sec:conditional}.

%%%%%%%%%%%%%%%%%%%%%%%%%%%%%%%%%%%%%%%%
\paragraph{Page Numbering.}

When only a part of the document is compiled,
the appropriate numbering of pages
(as well as other status parameters)
is determined from the |.aux| files.
The latter contain information from previous passes.
However this information needs to propagate through
all intermediate child documents.
Therefore the page numbering in child documents may well
be inconsistent until the complete document is compiled at least once.

A useful (if unconventional) way to always ensure a consistent
page numbering is to restart the numbering in each child document
and denote the pages by `\textit{child}|.|\textit{page}'
where \textit{child} represents the chapter/section number of the child file.
This can be achieved by the command
|\numberwithin{page}{|\textit{child}|}|
of the \textsf{amsmath} package
where \textit{child} can be |chapter| or |section|
depending on the chosen structuring.
Alternatively, one can modify the macro |\thepage| appropriately
and reset the counter |page| at the start of each child file.

%%%%%%%%%%%%%%%%%%%%%%%%%%%%%%%%%%%%%%%%%%%%%%%%%%%%%%%%%%%%%%%%%%%%%%%%%%%%%%%%
\subsection{Conditional Processing}
\label{sec:conditional}

The package provides a mechanism to compile different versions
of a document. To customise the versions further some conditional processing
can come in handy to distinguish which version is being compiled.
The package provides two macros to describe the compilation context:

%%%%%%%%%%%%%%%%%%%%%%%%%%%%%%%%%%%%%%%%
\DescribeMacro{\ifchilddoc}
The conditional |\ifchilddoc| distinguishes between the compilation of
child documents and the main document:
%
\begin{center}
|\ifchilddoc |\textit{child-code}| |[|\||else |\textit{main-code}]| \||fi|
\end{center}

%%%%%%%%%%%%%%%%%%%%%%%%%%%%%%%%%%%%%%%%
\DescribeMacro{\childdocname}
\DescribeMacro{\childdocjob}
The macro |\childdocname| contains the filename (without extension)
of the main or child file being processed.
Note that |\childdocjob| will always contain the name of the main file.

%%%%%%%%%%%%%%%%%%%%%%%%%%%%%%%%%%%%%%%%
\paragraph{Title Page.}

Conditional processing can be used to include a title or banner page
in the main document when proper precautions are taken.
Importantly, the code in the main file should ensure that the page counter
(as well as other status parameters which are stored in the |.aux| files)
takes the same value after the conditional processing.
Otherwise the page numbers may take divergent values
depending on which part is compiled.

For example, a title page could be declared by:
%
\begin{center}
\begin{tabular}{l}
|\ifchilddoc\||else|\\
|\addtocounter{page}{-1}|\\
\textit{code for title page}\\
|\newpage|\\
|\||fi|
\end{tabular}
\end{center}
%
A banner page for the child documents can be generated by:
%
\begin{center}
\begin{tabular}{l}
|\ifchilddoc|\\
|\addtocounter{page}{-1}|\\
\textit{code for banner page}\\
|\newpage|\\
|\||fi|
\end{tabular}
\end{center}
%
Here one could write a message such as:
\begin{center}
|This is the part \childdocname{} of \childdocjob{}.|
\end{center}

%%%%%%%%%%%%%%%%%%%%%%%%%%%%%%%%%%%%%%%%%%%%%%%%%%%%%%%%%%%%%%%%%%%%%%%%%%%%%%%%
\subsection{Flags}
\label{sec:flags}

The package makes it easy to generate different versions
of the main or child documents.
To this end compilation flags can be defined
and assigned different default values.
They will be particularly useful in conjunction
with the forwarding mechanism described in \secref{sec:forward}.

For example, it may be useful to have a flag |\version|
which can be set to |draft| or |final|.
The document source will contain some conditional code
depending on the value of |\version|.
Suppose further, the flag should default to |final| for the main file
and to |draft| for child files
which is a natural assignment for editing the document.
This is achieved by placing the following code
in the preamble of the main document
(below the |\childdocmain| directive):
%
\begin{center}
\begin{tabular}{l}
|\ifchilddoc|\\
|\providecommand{\version}{draft}|\\
|\||else|\\
|\providecommand{\version}{final}|\\
|\||fi|
\end{tabular}
\end{center}
%
The definition by |\providecommand| makes sure
that previous definitions are not overwritten.
Further statements |\providecommand{\version}{...}|
can thus be added before the above code to override it.

For the main file, one might add a line
(between |\childdocmain| and the above block)
%
\begin{center}
|%\ifchilddoc\||else\providecommand{\version}{draft}\||fi|
\end{center}
%
which can be uncommented to produce a draft version.
Likewise one can add a line to the very top of a child file
(above the |\childdocof{|\textit{main}|}| directive)
%
\begin{center}
|%\providecommand{\version}{final}|
\end{center}
%
which can be uncommented to produce the final version of this child document.

%%%%%%%%%%%%%%%%%%%%%%%%%%%%%%%%%%%%%%%%%%%%%%%%%%%%%%%%%%%%%%%%%%%%%%%%%%%%%%%%
\subsection{Forwarding}
\label{sec:forward}

Different versions of the main or child documents
using compilation flags as described in \secref{sec:flags}
can be (permanently) stored in different files
for convenient compilation, viewing and distribution.
To this end, the package defines a command
to pass on compilation to a different file:

%%%%%%%%%%%%%%%%%%%%%%%%%%%%%%%%%%%%%%%%
\DescribeMacro{\childdocforward}
The command |\childdocforward| redirects processing to
another source file:
%
\begin{center}
\begin{tabular}{l}
|\input{childdoc.def}|\\
|\childdocforward[|\textit{main}|]{|\textit{dest}|}|\\
\end{tabular}
\end{center}
%
The argument \textit{dest} is the destination file
(without extension).
It should be the main file or one of the child files.
Note that further \textsf{childdoc} directives
such as |\childdocof| and |\childdocforward|
in the indicated file will be processed in this form.
The optional argument \textit{main}
passes on directly to the main file \textit{main}
while pretending to compile the child \textit{dest}.
This form behaves as if \textit{dest}
issues |\childdocof{|\textit{main}|}| right away,
and no further \textsf{childdoc} directives will be processed.

%%%%%%%%%%%%%%%%%%%%%%%%%%%%%%%%%%%%%%%%
\DescribeMacro{\...prefix}
In the alternative form |\childdocforwardprefix|,
%
\begin{center}
\begin{tabular}{l}
|\input{childdoc.def}|\\
|\childdocforwardprefix[|\textit{main}|]{|\textit{prefix}|}{|\textit{dest}|}|
\end{tabular}
\end{center}
%
the destination file is determined by a pattern
depending on the current file:
To make this work, the current file must be called
`{\textit{prefix}\hspace{0.2em}\textit{suffix}}'
with \textit{prefix} matching precisely the argument.
Processing is then passed on to the file
`{\textit{dest}\hspace{0.2em}\textit{suffix}}'.
Surely, the same effect is achieved by
directly specifying the
argument `{\textit{dest}\hspace{0.2em}\textit{suffix}}'
in the first form.
However, that requires to set up a different file
for each child. With the alternative form of the command
all these files can have exactly the same content
which simplifies setting them up and maintaining them.

For example, the following file |draft.tex|
with a compilation flag |\version| as described in \secref{sec:flags}
compiles the main document as a draft:
%
\begin{center}
\begin{tabular}{l}
|\def\version{draft}|\\
|\input{childdoc.def}|\\
|\childdocforward{|\textit{main}|}|
\end{tabular}
\end{center}
%
Likewise, the following files |final|\textit{nn}|.tex|
compile the final version of the child document
|child|\textit{nn}|.tex|:
%
\begin{center}
\begin{tabular}{l}
|\def\version{final}|\\
|\input{childdoc.def}|\\
|\childdocforwardprefix{final}{child}|
\end{tabular}
\end{center}
%

Note that when several versions of a main file and/or of each child file
are to be generated, it may be convenient to set up a |Makefile| or
shell script to automatise the process.

%%%%%%%%%%%%%%%%%%%%%%%%%%%%%%%%%%%%%%%%%%%%%%%%%%%%%%%%%%%%%%%%%%%%%%%%%%%%%%%%
\subsection{Command Line Processing}
\label{sec:commandline}

The effect of redirection files can also be achieved by invoking
the \LaTeX{} compiler with a more elaborate command line.
Most conveniently this should be done as part
of a shell script or a |Makefile|.

When using \textsf{childdoc} in the main file, the following
command lines effectively perform a redirection
(note that depending on the shell being used,
backslashes may have to be doubled: `|\|' $\to$ `|\\|'):
%
\begin{center}
|... -jobname "|\textit{target}|" |\\|"|[\textit{flags}]%
|\input{childdoc.def}\childdocforward[|\textit{main}|]{|\textit{dest}|}"|
\end{center}
%
Here \textit{target} is the name of the output file,
\textit{main} is the name of the main file
and \textit{dest} is the name of the main or child file to be processed
(all filenames without extensions).
The optional argument \textit{main} can be omitted
if \textit{main} matches \textit{dest}.
Optionally, compilation \textit{flags} can be defined via |\def| commands.
This command line makes the \TeX{} engine believe
it is compiling the file \textit{target}
whose content is specified as the latter parameter.
The provided code then forwards the processing to
\textit{main} or \textit{dest} as described in \secref{sec:forward}.

%%%%%%%%%%%%%%%%%%%%%%%%%%%%%%%%%%%%%%%%%%%%%%%%%%%%%%%%%%%%%%%%%%%%%%%%%%%%%%%%
\subsection{Include by Input}
\label{sec:input}

Including child documents by |\include| has some restrictions by design.
Most notably, the content of a child document always occupies
its own set of pages; pages cannot be shared between child documents.
Usually, this behaviour makes perfect sense
because each child document contain an essential part of the document.
However, in some situations it may be desirable to compose
a document from a collection of parts
without having mandatory page breaks between then.
For this case, the package
provides a mechanism to include parts
by |\input| which can also be processed individually.
However, by construction this mechanism
requires manual handling of the content to be output.

%%%%%%%%%%%%%%%%%%%%%%%%%%%%%%%%%%%%%%%%
\DescribeMacro{\ifchilddocmanual}
The main file should be prepared as usual, see \secref{sec:include}.
However, the document body must make a distinction
between processing of an individual part and of the main document, e.g.:
%
\begin{center}
\begin{tabular}{l}
|\ifchilddocmanual|\\
|\input{\childdocname}|\\
|\||else|\\
\textit{document body with }|\input{|\textit{part}|}|\\
|\||fi|
\end{tabular}
\end{center}
%
The conditional |\ifchilddocmanual| is true whenever
a part to be included by |\input| is being compiled,
and the name of the part is stored in |\childdocname|.

%%%%%%%%%%%%%%%%%%%%%%%%%%%%%%%%%%%%%%%%
\DescribeMacro{\childdocby}
Each part to be included by |\input| should start with:
%
\begin{center}
\begin{tabular}{l}
|\input{childdoc.def}|\\
|\childdocby{|\textit{main}|}|\\
\end{tabular}
\end{center}
%
The directive |\childdocby| is similar to |\childdocof|
described in \secref{sec:include},
but the subsequent selection of content must be done manually.
To that end, both |\ifchilddoc| and |\ifchilddocmanual|
will be true upon processing of a part,
and the name of the part is stored in |\childdocname|.
Note that |\jobname| will be set to the filename of the current part
so that each part receives an individual |.aux| file
that does not interfere with the |.aux| file(s) of the main document.
This behaviour can be altered by the alternative form
|\childdocby[*]{|\textit{main}|}| (with a non-empty optional argument)
which uses the |.aux| file of the main document
by setting |\jobname| to \textit{main}.

%%%%%%%%%%%%%%%%%%%%%%%%%%%%%%%%%%%%%%%%%%%%%%%%%%%%%%%%%%%%%%%%%%%%%%%%%%%%%%%%
\subsection{Driver Development}
\label{sec:driver}

The \textsf{childdoc} mechanism can also be use for the development
of definition files such as \LaTeX{} styles or classes.
This case differs from the above setup with multiple parts
included by |\include| in that no |\includeonly| should be invoked.
This can be achieved by starting the include file
(before |\ProvidesPackage|) with:
%
\begin{center}
\begin{tabular}{l}
|\input{childdoc.def}|\\
|\childdocforward{|\textit{main}|}|\\
\end{tabular}
\end{center}
%
or alternatively with:
%
\begin{center}
\begin{tabular}{l}
|\input{childdoc.def}|\\
|\childdocby{|\textit{main}|}|\\
\end{tabular}
\end{center}
%
Both forms have slightly different effects as described above.
The main file is prepared as usual, see \secref{sec:include}.

%%%%%%%%%%%%%%%%%%%%%%%%%%%%%%%%%%%%%%%%%%%%%%%%%%%%%%%%%%%%%%%%%%%%%%%%%%%%%%%%
\subsection{Legacy Detection}
\label{sec:detection}

The directive |\childdocmain| in the main file can detect
whether the complete document or merely a child is to be compiled
even without using the directive |\childdocof|.
This method is deprecated because it is less robust
and there is no compelling reason to use it;
it is merely provided for backward compatibility
and it may be removed in future versions.

If the detection mechanism is to be used,
it is mandatory to correctly specify
the filename of the main file as the argument of |\childdocmain|:
%
\begin{center}
\begin{tabular}{l}
|\input{childdoc.def}|\\
|\childdocmain{|\textit{main}|}|\\
\end{tabular}
\end{center}
%
If |\jobname| does not match the argument \textit{main} of |\childdocmain|,
it is assumed that |\jobname| points to the child file to be compiled.
When using |\childdocmain| with the main file specified as argument,
it suffices to start a child file
with just |\input{|\textit{main}|}|
without loading of the package and using |\childdocof|.
If instead all processing is done
with the appropriate \textsf{childdoc} directives,
the argument of \textit{main} of |\childdocmain| can be empty.

An alternative version of the command line processing described
in \secref{sec:commandline} using the detection mechanism reads:
%
\begin{center}
|... -jobname "|\textit{target}|" "|[\textit{flags}]%
[|\def\jobname{|\textit{dest}|}|]|\input{|\textit{main}|}"|
\end{center}

%%%%%%%%%%%%%%%%%%%%%%%%%%%%%%%%%%%%%%%%%%%%%%%%%%%%%%%%%%%%%%%%%%%%%%%%%%%%%%%%
\subsection{Manual Code}
\label{sec:manual}

In case one cannot be certain whether the definitions file |childdoc.def|
is installed on the target \TeX{} distribution
and one prefers not to ship it,
it is conceivable to paste a few relevant commands into the sources.

To that end, drop all statements |\input{childdoc.def}|
and perform the replacements as outlined below.
Instead of |\childdocmain{|\textit{main}|}| add the following code
to the top of the main file:
%
\begin{center}
\begin{tabular}{l}
|\||ifdefined\childdocname\endinput\||fi\newif\ifchilddoc|\\
|\edef\childdocname{\scantokens\expandafter{\jobname\noexpand}}|\\
|\def\childdocmain{|\textit{main}|}\||ifx\childdocmain\childdocname\||else|\\
|\childdoctrue\includeonly{\childdocname}\let\jobname\childdocmain\||fi|\\
\end{tabular}
\end{center}
%
Instead of |\childdocof{|\textit{main}|}| just include the main file
at the top of each child file:
%
\begin{center}
|\input{|\textit{main}|}|
\end{center}
%
A simple redirection |\childdocforward{|\textit{dest}|}| is achieved by:
%
\begin{center}
|\def\jobname{|\textit{dest}|}\input{\jobname}|
\end{center}
%
The redirection with prefix
|\childdocforwardprefix[|\textit{prefix}|]{|\textit{dest}|}|
is accomplished by:
%
\begin{center}
\begin{tabular}{l}
|{\edef\jobname{\scantokens\expandafter{\jobname\noexpand}}|\\
|\def\redirectjob |\textit{prefix}|#1~~~{\gdef\jobname{|\textit{dest}|#1}}|\\
|\expandafter\redirectjob\jobname~~~}\input{\jobname}|
\end{tabular}
\end{center}

In an alternative approach,
child documents can be compiled by a specific command line
without additional code or specific definitions:
%
\begin{center}
|... -jobname "|\textit{target}|" "|[\textit{flags}]%
|\includeonly{|\textit{dest}|}\input{|\textit{main}|}"|
\end{center}
%

%%%%%%%%%%%%%%%%%%%%%%%%%%%%%%%%%%%%%%%%%%%%%%%%%%%%%%%%%%%%%%%%%%%%%%%%%%%%%%%%
%%%%%%%%%%%%%%%%%%%%%%%%%%%%%%%%%%%%%%%%%%%%%%%%%%%%%%%%%%%%%%%%%%%%%%%%%%%%%%%%
\section{Information}

%%%%%%%%%%%%%%%%%%%%%%%%%%%%%%%%%%%%%%%%%%%%%%%%%%%%%%%%%%%%%%%%%%%%%%%%%%%%%%%%
\subsection{Copyright}

Copyright \copyright{} 2017--2018 Niklas Beisert

This work may be distributed and/or modified under the
conditions of the \LaTeX{} Project Public License, either version 1.3
of this license or (at your option) any later version.
The latest version of this license is in
  \url{http://www.latex-project.org/lppl.txt}
and version 1.3 or later is part of all distributions of \LaTeX{}
version 2005/12/01 or later.

This work has the LPPL maintenance status `maintained'.

The Current Maintainer of this work is Niklas Beisert.

This work consists of the files |README.txt|, |childdoc.ins| and |childdoc.dtx|
as well as the derived files |childdoc.def|, |cdocsamp.tex|
with |cdocsch1.tex|, |cdocsch2.tex|, |cdocspt3.tex|, |cdocspt4.tex|,
|cdocsdrf.tex|, |cdocsfn1.tex|, |cdocsfn2.tex|
as well as |childdoc.pdf|.

%%%%%%%%%%%%%%%%%%%%%%%%%%%%%%%%%%%%%%%%%%%%%%%%%%%%%%%%%%%%%%%%%%%%%%%%%%%%%%%%
\subsection{Files and Installation}

The package consists of the files:
%
\begin{center}
\begin{tabular}{ll}
    |README.txt|   & readme file \\
    |childdoc.ins| & installation file \\
    |childdoc.dtx| & source file \\
    |childdoc.def| & definition file \\
    |cdocsamp.tex| & sample main file \\
    |cdocsch1.tex| & sample include file \\
    |cdocsch2.tex| & sample include file \\
    |cdocspt3.tex| & sample part file \\
    |cdocspt4.tex| & sample part file \\
    |cdocsdrf.tex| & sample redirection file \\
    |cdocsfn1.tex| & sample redirection file \\
    |cdocsfn2.tex| & sample redirection file \\
    |childdoc.pdf| & manual
\end{tabular}
\end{center}
%
The distribution consists of the files
|README.txt|, |childdoc.ins| and |childdoc.dtx|.
%
\begin{itemize}
\item
Run (pdf)\LaTeX{} on |childdoc.dtx|
to compile the manual |childdoc.pdf| (this file).
\item
Run \LaTeX{} on |childdoc.ins| to create the definitions file |childdoc.def|
and the sample |cdocsamp.tex| with include files
|cdocsch1.tex|, |cdocsch2.tex|, |cdocspt3.tex|, |cdocspt4.tex|,
|cdocsdrf.tex|, |cdocsfn1.tex|, |cdocsfn2.tex|.
Then copy the file |childdoc.def| to an appropriate directory of your \LaTeX{}
distribution, e.g.\ \textit{texmf-root}|/tex/latex/childdoc|.
\end{itemize}

%%%%%%%%%%%%%%%%%%%%%%%%%%%%%%%%%%%%%%%%%%%%%%%%%%%%%%%%%%%%%%%%%%%%%%%%%%%%%%%%
\subsection{Related CTAN Packages}

There are several other packages which offer a similar functionality:
%
\begin{itemize}
\item
The packages
\href{http://ctan.org/pkg/docmute}{\textsf{docmute}},
\href{http://ctan.org/pkg/includex}{\textsf{includex}} and
\href{http://ctan.org/pkg/standalone}{\textsf{standalone}}
provide commands to include only the document body of
a child file thus allowing both files to be compiled individually.
\item
The packages \href{http://ctan.org/pkg/subdocs}{\textsf{subdocs}}
and \href{http://ctan.org/pkg/subfiles}{\textsf{subfiles}}
provide structures in which the main and child documents can be
encapsulated and allowing them to be compiled individually.
The inclusion mechanism is different from the conventional |\include|.
\item
The package \href{http://ctan.org/pkg/combine}{\textsf{combine}}
is an elaborate solution to combine several documents into one.
\end{itemize}
%
See also the CTAN topic \href{http://ctan.org/topic/subdocs}{\textsf{subdocs}}
for further related packages.
The present package differs from the above solutions in that
a document structure constructed with the conventional |\include| mechanism
just needs two extra commands at the top of every file
such that all constituent files can be compiled individually.

%%%%%%%%%%%%%%%%%%%%%%%%%%%%%%%%%%%%%%%%%%%%%%%%%%%%%%%%%%%%%%%%%%%%%%%%%%%%%%%%
%\subsection{Feature Suggestions}
%
%The following is a list of features which may be useful for future
%versions of this package:
%%
%\begin{itemize}
%\item
%\ldots
%\end{itemize}

%%%%%%%%%%%%%%%%%%%%%%%%%%%%%%%%%%%%%%%%%%%%%%%%%%%%%%%%%%%%%%%%%%%%%%%%%%%%%%%%
\subsection{Revision History}

%%%%%%%%%%%%%%%%%%%%%%%%%%%%%%%%%%%%%%%%
\paragraph{v2.0:} 2018/12/30

\begin{itemize}
\item
immediate forward processing
\item
added |\childdocby| mechanism
\item
manual restructured
\end{itemize}

%%%%%%%%%%%%%%%%%%%%%%%%%%%%%%%%%%%%%%%%
\paragraph{v1.6:} 2018/01/17

\begin{itemize}
\item
application for development of include files
\item
corrections to manual
\end{itemize}

%%%%%%%%%%%%%%%%%%%%%%%%%%%%%%%%%%%%%%%%
\paragraph{v1.5:} 2017/05/21

\begin{itemize}
\item
more complete structuring introduced
\item
|\childdocof| introduced
\item
|\childdoc| renamed to |\childdocmain|
\item
|\childredirect| renamed to |\childdocforward| and |\childdocforwardprefix|
and functionality expanded
\end{itemize}

%%%%%%%%%%%%%%%%%%%%%%%%%%%%%%%%%%%%%%%%
\paragraph{v1.0:} 2017/04/27

\begin{itemize}
\item
manual and install package
\item
first version published on CTAN
\end{itemize}

%%%%%%%%%%%%%%%%%%%%%%%%%%%%%%%%%%%%%%%%
\paragraph{v0.6:} 2017/04/26

\begin{itemize}
\item
redirection mechanism added
\end{itemize}

%%%%%%%%%%%%%%%%%%%%%%%%%%%%%%%%%%%%%%%%
\paragraph{v0.5:} 2017/04/26

\begin{itemize}
\item
functionality in definition file
\end{itemize}


%%%%%%%%%%%%%%%%%%%%%%%%%%%%%%%%%%%%%%%%%%%%%%%%%%%%%%%%%%%%%%%%%%%%%%%%%%%%%%%%
%%%%%%%%%%%%%%%%%%%%%%%%%%%%%%%%%%%%%%%%%%%%%%%%%%%%%%%%%%%%%%%%%%%%%%%%%%%%%%%%
%%%%%%%%%%%%%%%%%%%%%%%%%%%%%%%%%%%%%%%%%%%%%%%%%%%%%%%%%%%%%%%%%%%%%%%%%%%%%%%%
\appendix

\settowidth\MacroIndent{\rmfamily\scriptsize 000\ }

 \DocInput{childdoc.dtx}

\end{document}
%</driver>
% \fi
%
% %%%%%%%%%%%%%%%%%%%%%%%%%%%%%%%%%%%%%%%%%%%%%%%%%%%%%%%%%%%%%%%%%%%%%%%%%%%%%%
% %%%%%%%%%%%%%%%%%%%%%%%%%%%%%%%%%%%%%%%%%%%%%%%%%%%%%%%%%%%%%%%%%%%%%%%%%%%%%%
% \section{Sample}
%\iffalse
%<*samplemain>
%\fi
%
% The following presents a sample document
% with two chapters, two parts, a title page,
% a compile flag as well as three forwarding files to set the flag.
% It consists of eight |.tex| files:
% \begin{center}
% \begin{tabular}{ll}
% |cdocsamp.tex|&main file\\
% |cdocsch1.tex|&include file for chapter 1\\
% |cdocsch2.tex|&include file for chapter 2\\
% |cdocspt3.tex|&include file for part 3\\
% |cdocspt4.tex|&include file for part 4\\
% |cdocsdrf.tex|&forwarding file for main file in draft mode\\
% |cdocsfi1.tex|&forwarding file for final version of chapter 1\\
% |cdocsfi2.tex|&forwarding file for final version of chapter 2\\
% \end{tabular}
% \end{center}
% Each of the eight files can be compiled directly by the \LaTeX{} compiler.
%
% %%%%%%%%%%%%%%%%%%%%%%%%%%%%%%%%%%%%%%
% \paragraph{Main File.}
%
% The main file is called |cdocsamp.tex|.
%
% Load the \textsf{childdoc} definitions and
% declare the filename for the main document:
%    \begin{macrocode}
\input{childdoc.def}
\childdocmain{}
%    \end{macrocode}

% Optional override for |\version| flag:
%    \begin{macrocode}
%%\ifchilddoc\else\providecommand{\version}{draft}\fi
%    \end{macrocode}

% Define the default values for the |\version| flag
% (|final| for the main file and |draft| for childs):
%    \begin{macrocode}
\ifchilddoc
\providecommand{\version}{draft}
\else
\providecommand{\version}{final}
\fi
%    \end{macrocode}

% Load the standard document class:
%    \begin{macrocode}
\documentclass[12pt]{article}
%    \end{macrocode}

% Start the document body:
%    \begin{macrocode}
\begin{document}
%    \end{macrocode}

% Declare a title page.
% Print title, part of document being processed and version flag:
%    \begin{macrocode}
\addtocounter{page}{-1}
\begin{center}
{\LARGE\bfseries{}childdoc example\par}
\vspace{1cm}
\ifchilddoc
\ifchilddocmanual part\else chapter\fi:
`\childdocname' of `\childdocjob'\par
\else
main document: `\childdocjob'\par
\fi
version: \version\par
\end{center}
\newpage
%    \end{macrocode}

% Manually include selected file,
% otherwise process as usual:
%    \begin{macrocode}
\ifchilddocmanual
\section*{part `\childdocname'}
\input{\childdocname}
\else
%    \end{macrocode}

% Include the two chapters:
%    \begin{macrocode}
\include{cdocsch1}
\include{cdocsch2}
%    \end{macrocode}

% Include the two parts unless only chapters should be displayed:
%    \begin{macrocode}
\ifchilddoc\else
\section{part three}
\input{cdocspt3}
\section{part four}
\input{cdocspt4}
\fi
%    \end{macrocode}

% Process as usual until here:
%    \begin{macrocode}
\fi
%    \end{macrocode}

% End of document body:
%    \begin{macrocode}
\end{document}
%    \end{macrocode}
%\iffalse
%</samplemain>
%\fi
%
% %%%%%%%%%%%%%%%%%%%%%%%%%%%%%%%%%%%%%%
% \paragraph{Chapter Include Files.}
%
% The include files are called |cdocsch1.tex| and |cdocsch2.tex|.
%
%\iffalse
%<*samplechap1|samplechap2>
%\fi

% Optional override for |\version| flag:
%    \begin{macrocode}
%%\providecommand{\version}{final}
%    \end{macrocode}

% Include the main document:
%    \begin{macrocode}
\input{childdoc.def}
\childdocof{cdocsamp}
%    \end{macrocode}

%\iffalse
%</samplechap1|samplechap2>
%\fi
%
%\iffalse
%<*samplechap1>
%\fi
% Some text for chapter 1:
%    \begin{macrocode}
\section{one}
some text in chapter one
%    \end{macrocode}

%\iffalse
%</samplechap1>
%\fi
% Some text for chapter 2:
%\iffalse
%<*samplechap2>
%\fi
%    \begin{macrocode}
\section{two}
more text in chapter two
%    \end{macrocode}

%\iffalse
%</samplechap2>
%\fi
%
% %%%%%%%%%%%%%%%%%%%%%%%%%%%%%%%%%%%%%%
% \paragraph{Part Include Files.}
%
% The include files are called |cdocspt3.tex| and |cdocspt4.tex|.
%
%\iffalse
%<*samplepart3|samplepart4>
%\fi

% Optional override for |\version| flag:
%    \begin{macrocode}
%%\providecommand{\version}{final}
%    \end{macrocode}

% Include the main document:
%    \begin{macrocode}
\input{childdoc.def}
\childdocby{cdocsamp}
%    \end{macrocode}

%\iffalse
%</samplepart3|samplepart4>
%\fi
%
%\iffalse
%<*samplepart3>
%\fi
% Some text for part 3:
%    \begin{macrocode}
some text in part three
%    \end{macrocode}

%\iffalse
%</samplepart3>
%\fi
% Some text for part 4:
%\iffalse
%<*samplepart4>
%\fi
%    \begin{macrocode}
more text in part four
%    \end{macrocode}

%\iffalse
%</samplepart4>
%\fi
%
% %%%%%%%%%%%%%%%%%%%%%%%%%%%%%%%%%%%%%%
% \paragraph{Forwarding for a Complete Draft.}
%
% The following forwarding file |cdocsdrf.tex|
% compiles the main document in draft mode:
%\iffalse
%<*sampledraft>
%\fi
%    \begin{macrocode}
\def\version{draft}
\input{childdoc.def}
\childdocforward{cdocsamp}
%    \end{macrocode}

%\iffalse
%</sampledraft>
%\fi
%
% %%%%%%%%%%%%%%%%%%%%%%%%%%%%%%%%%%%%%%
% \paragraph{Forwarding for Final Version of the Chapters.}
%
% The following forwarding files |cdocsfn1.tex| and |cdocsfn2.tex|
% (with identical content)
% compile the final versions of the child documents
% |cdocsch1.tex| and |cdocsch2.tex|, respectively:
%\iffalse
%<*samplefinal>
%\fi
%    \begin{macrocode}
\def\version{final}
\input{childdoc.def}
\childdocforwardprefix[cdocsamp]{cdocsfn}{cdocsch}
%    \end{macrocode}

%\iffalse
%</samplefinal>
%\fi
%
% %%%%%%%%%%%%%%%%%%%%%%%%%%%%%%%%%%%%%%
% \paragraph{Command Line Processing.}
%
% The following three command lines generate the output files
% |cdocscld|, |cdocscl1| and |cdocscl2|
% which should be identical to
% |cdocsdrf|, |cdocsch1| and |cdocsfn2|, respectively:
% \begin{center}
% \begin{tabular}{l}
% |latex -jobname cdocscld \|\\
% |  "\def\version{draft}\input{childdoc.def}\childdocforward{cdocsamp}"|\\
% |latex -jobname cdocscl1 \|\\
% |  "\input{childdoc.def}\childdocforward[cdocsamp]{cdocsch1}"|\\
% |latex -jobname cdocscl2 \|\\
% |  "\def\version{final}\input{childdoc.def}\childdocforward{cdocsch2}"|
% \end{tabular}
% \end{center}
% Note that the trailing backslash on each first line
% merely continues the input to the second line
% (for convenient cut ant paste).
% Furthermore, the command |latex| can be replaced by any
% of its alternative versions such as |pdflatex|.
%
% %%%%%%%%%%%%%%%%%%%%%%%%%%%%%%%%%%%%%%%%%%%%%%%%%%%%%%%%%%%%%%%%%%%%%%%%%%%%%%
% %%%%%%%%%%%%%%%%%%%%%%%%%%%%%%%%%%%%%%%%%%%%%%%%%%%%%%%%%%%%%%%%%%%%%%%%%%%%%%
% \section{Implementation}
%\iffalse
%<*package>
%\fi
%
% This section describes the definitions file |childdoc.def|.

% The definitions cannot be loaded using |\usepackage| or |\RequirePackage|
% which has a mechanism to prevent loading a style file more than once.
% When loading the definitions by means of |\input|
% multiple instances have to be prevented manually:
%\iffalse
%This code needs to be before the `\ProvidesFile' directive
%which is defined at the beginning of this file.
%Therefore it is also placed there and commented out here.
%</package>
%<*discard>
%\fi
%    \begin{macrocode}
\ifdefined\childdocmain\endinput\fi
%    \end{macrocode}
%\iffalse
%</discard>
%<*package>
%\fi
%
% \macro{\ifchilddoc}
% \macro{\ifchilddocmanual}
% The conditional |\ifchilddoc| tells whether a
% child (true) or main (false) document is being compiled.
% The conditional |\ifchilddocmanual| tells whether
% the |\includeonly| mechanism is used (false) or
% the selection of child files must be performed manually (true).
% The definitions initialise to false:
%    \begin{macrocode}
\newif\ifchilddoc
\newif\ifchilddocmanual
%    \end{macrocode}

% \macro{\childdocname}
% \macro{\childdocjob}
% The macro |\childdocname| stores the name of the main document
% to be compiled. The macro |\childdocjob| stores the name of
% the document on which the \LaTeX{} compiler was originally invoked.
% The content of |\jobname| cannot be compared
% to filenames specified in the source due to different catcodes.
% The following code rescans |\jobname|, stores the result
% in |\childdocname| and saves a copy in |\childdocjob|:
%    \begin{macrocode}
\edef\childdocname{\scantokens\expandafter{\jobname\noexpand}}
\let\childdocjob\childdocname
%    \end{macrocode}

% \macro{\childdocdisable}
% The macro |\childdocdisable| prevents the main file
% from being processed more than once.
% At this stage, the main document command |\childdocmain|
% is assumed to be called once again where it should do nothing.
% Any subsequent call to it should prevent
% a secondary processing of the main document
% It overwrites the forwarding commands
% |\childdocof| and |\childdocforward|
% with empty macros to prevent further inclusions of the main document:
%    \begin{macrocode}
\newcommand{\childdocdisable}
{
  \renewcommand{\childdocmain}[1]{\renewcommand{\childdocmain}[1]{\endinput}}
  \renewcommand{\childdocof}[1]{}
  \renewcommand{\childdocby}[2][]{}
  \renewcommand{\childdocforward}[2][]{}
  \renewcommand{\childdocdisable}{}
}
%    \end{macrocode}

% \macro{\childdocmain}
% The macro |\childdocmain| is to be called at the top of the main file
% with nothing or the main filename (without extension) as argument.
% First, it breaks loops.
% If the argument is not empty and does not match |\childdocname|
% (which is set by the first inclusion of |childdoc.def|),
% |\ifchilddoc| is set to true, |\includeonly| is applied to the child file
% and |\jobname| is set to the main file
% (for proper handling of |.aux| files):
%    \begin{macrocode}
\newcommand{\childdocmain}[1]
{
  \childdocdisable\childdocmain{}
  \if?#1?\else
    \begingroup
      \def\childdoctmp{#1}
      \ifx\childdoctmp\childdocname
        \def\childdoctmp{}
      \else
        \def\childdoctmp
        {
          \childdoctrue
          \includeonly{\childdocname}
          \def\childdocjob{#1}
          \def\jobname{#1}
        }
      \fi
      \expandafter
    \endgroup
    \childdoctmp
  \fi
}
%    \end{macrocode}

% \macro{\childdocof}
% The command |\childdocof| redirects
% compilation to the main file |#1|.
%    \begin{macrocode}
\newcommand{\childdocof}[1]
{
  \childdocdisable
  \childdoctrue
  \includeonly{\childdocname}
  \def\jobname{#1}
  \def\childdocjob{#1}
  \input{#1}
}
%    \end{macrocode}

% \macro{\childdocby}
% The command |\childdocby| ....
%    \begin{macrocode}
\newcommand{\childdocby}[2][]
{
  \childdocdisable
  \childdoctrue
  \childdocmanualtrue
  \if?#1?\else
    \def\jobname{#2}
  \fi
  \def\childdocjob{#2}
  \input{#2}
  \endinput
}
%    \end{macrocode}

% \macro{\childdocforward}
% The command |\childdocforward| redirects
% compilation to the main file or
% (if the optional argument is given) a child file.
% Parameters are set as if the main file
% or a child file starting with |\childdocof| was compiled.
% Then compilation is handed over to the main file:
%    \begin{macrocode}
\newcommand{\childdocforward}[2][]
{
  \begingroup
    \if?#1?
      \def\childdoctmp
      {
        \def\childdocname{#2}
        \def\childdocjob{#2}
        \def\jobname{#2}
        \input{#2}
        \endinput
      }
    \else
      \def\childdoctmp
      {
        \childdocdisable
        \def\childdocname{#2}
        \childdoctrue
        \includeonly{#2}
        \def\childdocjob{#1}
        \def\jobname{#1}
        \input{#1}
        \endinput
      }
    \fi
    \expandafter
  \endgroup
  \childdoctmp
}
%    \end{macrocode}

% \macro{\childdocforwardprefix}
% The command |\childdocforwardprefix| redirects
% compilation to the main or a child file by means of a pattern.
% The prefix |#1| in the current filename is replaced by |#2|
% and the suffix of the current filename is kept
% (it is assumed that the filename does not contain the substring `|~~~|'
% which is used as a delimiter).
% Compilation is handed over to the new file by |\childdocforward|:
%    \begin{macrocode}
\newcommand{\childdocforwardprefix}[3][]
{
  \begingroup
    \def\childdocextract #2##1~~~{\def\childdoctmp{\childdocforward[#1]{#3##1}}}
    \expandafter\childdocextract\childdocname~~~
    \expandafter
  \endgroup
  \childdoctmp
}
%    \end{macrocode}

% \macro{\childdoc}
% The deprecated macro |\childdoc| is a legacy version of |\childdocmain|:
%    \begin{macrocode}
\newcommand{\childdoc}{\childdocmain}
%    \end{macrocode}

% \macro{\childdocredirect}
% The deprecated macro |\childdocredirect| is a legacy version
% of |\childdocforward| and |\childdocforwardprefix|:
%    \begin{macrocode}
\newcommand{\childdocredirect}[2][]
{
  \begingroup
    \if?#1?
      \def\childdoctmp{\childdocforward{#2}}
    \else
      \def\childdoctmp{\childdocforwardprefix{#1}{#2}}
    \fi
    \expandafter
  \endgroup
  \childdoctmp
}
%    \end{macrocode}

%\iffalse
%</package>
%\fi
%
\endinput
|\\
|\childdocforward{|\textit{main}|}|\\
\end{tabular}
\end{center}
%
or alternatively with:
%
\begin{center}
\begin{tabular}{l}
|% \iffalse
%
% childdoc.dtx Copyright (C) 2017-2018 Niklas Beisert
%
% This work may be distributed and/or modified under the
% conditions of the LaTeX Project Public License, either version 1.3
% of this license or (at your option) any later version.
% The latest version of this license is in
%   http://www.latex-project.org/lppl.txt
% and version 1.3 or later is part of all distributions of LaTeX
% version 2005/12/01 or later.
%
% This work has the LPPL maintenance status `maintained'.
%
% The Current Maintainer of this work is Niklas Beisert.
%
% This work consists of the files childdoc.dtx and childdoc.ins
% and the derived files childdoc.def and cdocsamp.tex with
% cdocsch1.tex, cdocsch2.tex, cdocsdrf.tex, cdocsfn1.tex, cdocsfn2.tex.
%
%<package>\ifdefined\childdocmain\endinput\fi
%<package>\ProvidesFile{childdoc.def}[2018/12/30 v2.0 child document driver]
%<samplemain>\ProvidesFile{cdocsamp.tex}[2018/12/30 v2.0 sample for childdoc]
%<*driver>
%\ProvidesFile{childdoc.drv}[2018/12/30 v2.0 childdoc reference manual file]
\PassOptionsToClass{10pt,a4paper}{article}
\documentclass{ltxdoc}

\usepackage[margin=35mm]{geometry}
\usepackage{hyperref}
\usepackage{hyperxmp}
\usepackage[usenames]{color}

\hypersetup{colorlinks=true}
\hypersetup{pdfstartview=FitH}
\hypersetup{pdfpagemode=UseNone}
\hypersetup{pdfsource={}}
\hypersetup{pdflang={en-UK}}
\hypersetup{pdfcopyright={Copyright 2017-2018 Niklas Beisert.
  This work may be distributed and/or modified under the
  conditions of the LaTeX Project Public License, either version 1.3
  of this license or (at your option) any later version.}}
\hypersetup{pdflicenseurl={http://www.latex-project.org/lppl.txt}}
\hypersetup{pdfcontactaddress={ETH Zurich, ITP, HIT K,
  Wolfgang-Pauli-Strasse 27}}
\hypersetup{pdfcontactpostcode={8093}}
\hypersetup{pdfcontactcity={Zurich}}
\hypersetup{pdfcontactcountry={Switzerland}}
\hypersetup{pdfcontactemail={nbeisert@itp.phys.ethz.ch}}
\hypersetup{pdfcontacturl={http://people.phys.ethz.ch/\xmptilde nbeisert/}}

\newcommand{\secref}[1]{\hyperref[#1]{section \ref*{#1}}}

\parskip1ex
\parindent0pt
\let\olditemize\itemize
\def\itemize{\olditemize\parskip0pt}

\begin{document}

\title{The \textsf{childdoc} Package}
\hypersetup{pdftitle={The childdoc Package}}
\author{Niklas Beisert\\[2ex]
  Institut f\"ur Theoretische Physik\\
  Eidgen\"ossische Technische Hochschule Z\"urich\\
  Wolfgang-Pauli-Strasse 27, 8093 Z\"urich, Switzerland\\[1ex]
  \href{mailto:nbeisert@itp.phys.ethz.ch}
  {\texttt{nbeisert@itp.phys.ethz.ch}}}
\hypersetup{pdfauthor={Niklas Beisert}}
\hypersetup{pdfsubject={Manual for the LaTeX2e Package childdoc}}
\date{30 December 2018, \textsf{v2.0}}
\maketitle

\begin{abstract}\noindent
\textsf{childdoc} is a \LaTeXe{} package
that enables the direct compilation
of document sections included by |\include|
to individual files.
\end{abstract}

\begingroup
\parskip0ex
\tableofcontents
\endgroup

%%%%%%%%%%%%%%%%%%%%%%%%%%%%%%%%%%%%%%%%%%%%%%%%%%%%%%%%%%%%%%%%%%%%%%%%%%%%%%%%
%%%%%%%%%%%%%%%%%%%%%%%%%%%%%%%%%%%%%%%%%%%%%%%%%%%%%%%%%%%%%%%%%%%%%%%%%%%%%%%%
\section{Introduction}

\LaTeX{} provides a mechanism to structure a large document (such as a book)
into a main file and several child files (containing the chapters)
using the |\include| command.
This mechanism is beneficial for documents
which span hundreds of pages in order to
make the source file(s) more manageable.
Moreover, compilation can be restricted to
selected child files by means of the |\includeonly| command.
The latter feature can be used to reduce the compilation time while editing
(this was significantly more useful in the earlier days of \LaTeX{})
or to generate a smaller document which is easier to navigate.
Another application of |\includeonly| is to generate
documents consisting of selected parts of the complete document.

However, there are a few drawbacks of the plain |\include| mechanism:
\begin{itemize}
\item
The child files cannot be compiled on their own,
they can only be compiled via the main file.
A naive editing environment
(such as a text editor with an option
to have the current file processed by \LaTeX)
may require one to switch to the main file before compiling;
attempting to compile the child file produces errors.
\item
The main file must be modified (each time)
to adjust the |\includeonly| command
to the present needs. This easily leaves the main file in a messy state.
\item
The generated document will always carry the filename
of the main document. This is inconvenient if
several child files are to be compiled and
to be kept for distribution.
\end{itemize}

The present package provides a simple interface
to make child files individually compilable by \LaTeX{}.
Compiling a child file then has the same effect as compiling
the main file with an |\includeonly| command
to select the appropriate child.
Moreover the generated document will carry the name of the child
rather than the main file.
This resolves all three above issues.

This feature is meant to make the editing of books,
thesis documents and lecture notes somewhat more convenient.
However, the package can also be used efficiently for
composing a series of documents (such as exercise sheets)
which are typically distributed individually.
It then assists the author in generating the individual documents
(potentially in different versions)
as well as a document containing the collected series.
Another application is in developing style files
or other kinds of included material
where compilation of the style file could redirect
to a sample or test file.

%%%%%%%%%%%%%%%%%%%%%%%%%%%%%%%%%%%%%%%%%%%%%%%%%%%%%%%%%%%%%%%%%%%%%%%%%%%%%%%%
%%%%%%%%%%%%%%%%%%%%%%%%%%%%%%%%%%%%%%%%%%%%%%%%%%%%%%%%%%%%%%%%%%%%%%%%%%%%%%%%
\section{Usage}

First of all, the package \textsf{childdoc} is \emph{not} a standard
\LaTeXe{} |.sty| style file! Therefore it needs to be invoked in
a non-standard way.

%%%%%%%%%%%%%%%%%%%%%%%%%%%%%%%%%%%%%%%%%%%%%%%%%%%%%%%%%%%%%%%%%%%%%%%%%%%%%%%%
\subsection{Included Files}
\label{sec:include}

%%%%%%%%%%%%%%%%%%%%%%%%%%%%%%%%%%%%%%%%
\DescribeMacro{\childdocmain}
To use the package, add the commands
\begin{center}
\begin{tabular}{l}
|\input{childdoc.def}|\\
|\childdocmain{}|\\
\end{tabular}
\end{center}
at the very top of the main \LaTeX{} file,
in particular \emph{before} the |\documentclass| statement!
The argument of |\childdocmain| should be left empty
(but it must be present).

%%%%%%%%%%%%%%%%%%%%%%%%%%%%%%%%%%%%%%%%
\DescribeMacro{\childdocof}
Furthermore, add the commands
\begin{center}
\begin{tabular}{l}
|\input{childdoc.def}|\\
|\childdocof{|\textit{main}|}|\\
\end{tabular}
\end{center}
at the top of every child file \textit{child}
which is included by |\include{|\textit{child}|}|
from within the main file
(or at least for those files to be compiled individually).
The argument \textit{main} must be the filename of the main file.

There are a couple of
considerations in setting up the main and child documents:

%%%%%%%%%%%%%%%%%%%%%%%%%%%%%%%%%%%%%%%%
\paragraph{Restrictions.}

Please note the following restrictions:
\begin{itemize}
\item
|\childdocmain| must be called with one argument \textit{main}
to ensure compatibility with earlier version of the package.
It must either be empty (|\childdocmain{}|)
or precisely match the filename of the main file in which it is specified.
See \secref{sec:detection} for further information.
\item
The filename \textit{main} must be specified without the |.tex| extension.
\item
The filename \textit{main} is case sensitive
(even in case-insensitive file systems)
due to internal string comparison.
\item
The argument \textit{main} should be fully expanded, it cannot be a macro.
\item
Subdirectories and special characters should be avoided in filenames.
\item
The command |\childdocmain{|\textit{main}|}| must be followed by a whitespace.
It should not be followed immediately by another command
or by a comment mark `|%|'.
This is because the \TeX{} parser reads the token immediately following
the argument of |\childdocmain| and puts it
at the beginning of every child section;
however, a white\-space is ignored.
\end{itemize}

%%%%%%%%%%%%%%%%%%%%%%%%%%%%%%%%%%%%%%%%
\paragraph{Content of Main File.}

It is advisable to place all content in the child files included by |\include|.
Any output contained in the main file will appear in all child documents
unless suppressed manually;
it cannot be suppressed automatically by the |\includeonly| directive
and thus should normally be avoided.
A method to include some content in the main file
by means of conditional processing is described in \secref{sec:conditional}.

%%%%%%%%%%%%%%%%%%%%%%%%%%%%%%%%%%%%%%%%
\paragraph{Page Numbering.}

When only a part of the document is compiled,
the appropriate numbering of pages
(as well as other status parameters)
is determined from the |.aux| files.
The latter contain information from previous passes.
However this information needs to propagate through
all intermediate child documents.
Therefore the page numbering in child documents may well
be inconsistent until the complete document is compiled at least once.

A useful (if unconventional) way to always ensure a consistent
page numbering is to restart the numbering in each child document
and denote the pages by `\textit{child}|.|\textit{page}'
where \textit{child} represents the chapter/section number of the child file.
This can be achieved by the command
|\numberwithin{page}{|\textit{child}|}|
of the \textsf{amsmath} package
where \textit{child} can be |chapter| or |section|
depending on the chosen structuring.
Alternatively, one can modify the macro |\thepage| appropriately
and reset the counter |page| at the start of each child file.

%%%%%%%%%%%%%%%%%%%%%%%%%%%%%%%%%%%%%%%%%%%%%%%%%%%%%%%%%%%%%%%%%%%%%%%%%%%%%%%%
\subsection{Conditional Processing}
\label{sec:conditional}

The package provides a mechanism to compile different versions
of a document. To customise the versions further some conditional processing
can come in handy to distinguish which version is being compiled.
The package provides two macros to describe the compilation context:

%%%%%%%%%%%%%%%%%%%%%%%%%%%%%%%%%%%%%%%%
\DescribeMacro{\ifchilddoc}
The conditional |\ifchilddoc| distinguishes between the compilation of
child documents and the main document:
%
\begin{center}
|\ifchilddoc |\textit{child-code}| |[|\||else |\textit{main-code}]| \||fi|
\end{center}

%%%%%%%%%%%%%%%%%%%%%%%%%%%%%%%%%%%%%%%%
\DescribeMacro{\childdocname}
\DescribeMacro{\childdocjob}
The macro |\childdocname| contains the filename (without extension)
of the main or child file being processed.
Note that |\childdocjob| will always contain the name of the main file.

%%%%%%%%%%%%%%%%%%%%%%%%%%%%%%%%%%%%%%%%
\paragraph{Title Page.}

Conditional processing can be used to include a title or banner page
in the main document when proper precautions are taken.
Importantly, the code in the main file should ensure that the page counter
(as well as other status parameters which are stored in the |.aux| files)
takes the same value after the conditional processing.
Otherwise the page numbers may take divergent values
depending on which part is compiled.

For example, a title page could be declared by:
%
\begin{center}
\begin{tabular}{l}
|\ifchilddoc\||else|\\
|\addtocounter{page}{-1}|\\
\textit{code for title page}\\
|\newpage|\\
|\||fi|
\end{tabular}
\end{center}
%
A banner page for the child documents can be generated by:
%
\begin{center}
\begin{tabular}{l}
|\ifchilddoc|\\
|\addtocounter{page}{-1}|\\
\textit{code for banner page}\\
|\newpage|\\
|\||fi|
\end{tabular}
\end{center}
%
Here one could write a message such as:
\begin{center}
|This is the part \childdocname{} of \childdocjob{}.|
\end{center}

%%%%%%%%%%%%%%%%%%%%%%%%%%%%%%%%%%%%%%%%%%%%%%%%%%%%%%%%%%%%%%%%%%%%%%%%%%%%%%%%
\subsection{Flags}
\label{sec:flags}

The package makes it easy to generate different versions
of the main or child documents.
To this end compilation flags can be defined
and assigned different default values.
They will be particularly useful in conjunction
with the forwarding mechanism described in \secref{sec:forward}.

For example, it may be useful to have a flag |\version|
which can be set to |draft| or |final|.
The document source will contain some conditional code
depending on the value of |\version|.
Suppose further, the flag should default to |final| for the main file
and to |draft| for child files
which is a natural assignment for editing the document.
This is achieved by placing the following code
in the preamble of the main document
(below the |\childdocmain| directive):
%
\begin{center}
\begin{tabular}{l}
|\ifchilddoc|\\
|\providecommand{\version}{draft}|\\
|\||else|\\
|\providecommand{\version}{final}|\\
|\||fi|
\end{tabular}
\end{center}
%
The definition by |\providecommand| makes sure
that previous definitions are not overwritten.
Further statements |\providecommand{\version}{...}|
can thus be added before the above code to override it.

For the main file, one might add a line
(between |\childdocmain| and the above block)
%
\begin{center}
|%\ifchilddoc\||else\providecommand{\version}{draft}\||fi|
\end{center}
%
which can be uncommented to produce a draft version.
Likewise one can add a line to the very top of a child file
(above the |\childdocof{|\textit{main}|}| directive)
%
\begin{center}
|%\providecommand{\version}{final}|
\end{center}
%
which can be uncommented to produce the final version of this child document.

%%%%%%%%%%%%%%%%%%%%%%%%%%%%%%%%%%%%%%%%%%%%%%%%%%%%%%%%%%%%%%%%%%%%%%%%%%%%%%%%
\subsection{Forwarding}
\label{sec:forward}

Different versions of the main or child documents
using compilation flags as described in \secref{sec:flags}
can be (permanently) stored in different files
for convenient compilation, viewing and distribution.
To this end, the package defines a command
to pass on compilation to a different file:

%%%%%%%%%%%%%%%%%%%%%%%%%%%%%%%%%%%%%%%%
\DescribeMacro{\childdocforward}
The command |\childdocforward| redirects processing to
another source file:
%
\begin{center}
\begin{tabular}{l}
|\input{childdoc.def}|\\
|\childdocforward[|\textit{main}|]{|\textit{dest}|}|\\
\end{tabular}
\end{center}
%
The argument \textit{dest} is the destination file
(without extension).
It should be the main file or one of the child files.
Note that further \textsf{childdoc} directives
such as |\childdocof| and |\childdocforward|
in the indicated file will be processed in this form.
The optional argument \textit{main}
passes on directly to the main file \textit{main}
while pretending to compile the child \textit{dest}.
This form behaves as if \textit{dest}
issues |\childdocof{|\textit{main}|}| right away,
and no further \textsf{childdoc} directives will be processed.

%%%%%%%%%%%%%%%%%%%%%%%%%%%%%%%%%%%%%%%%
\DescribeMacro{\...prefix}
In the alternative form |\childdocforwardprefix|,
%
\begin{center}
\begin{tabular}{l}
|\input{childdoc.def}|\\
|\childdocforwardprefix[|\textit{main}|]{|\textit{prefix}|}{|\textit{dest}|}|
\end{tabular}
\end{center}
%
the destination file is determined by a pattern
depending on the current file:
To make this work, the current file must be called
`{\textit{prefix}\hspace{0.2em}\textit{suffix}}'
with \textit{prefix} matching precisely the argument.
Processing is then passed on to the file
`{\textit{dest}\hspace{0.2em}\textit{suffix}}'.
Surely, the same effect is achieved by
directly specifying the
argument `{\textit{dest}\hspace{0.2em}\textit{suffix}}'
in the first form.
However, that requires to set up a different file
for each child. With the alternative form of the command
all these files can have exactly the same content
which simplifies setting them up and maintaining them.

For example, the following file |draft.tex|
with a compilation flag |\version| as described in \secref{sec:flags}
compiles the main document as a draft:
%
\begin{center}
\begin{tabular}{l}
|\def\version{draft}|\\
|\input{childdoc.def}|\\
|\childdocforward{|\textit{main}|}|
\end{tabular}
\end{center}
%
Likewise, the following files |final|\textit{nn}|.tex|
compile the final version of the child document
|child|\textit{nn}|.tex|:
%
\begin{center}
\begin{tabular}{l}
|\def\version{final}|\\
|\input{childdoc.def}|\\
|\childdocforwardprefix{final}{child}|
\end{tabular}
\end{center}
%

Note that when several versions of a main file and/or of each child file
are to be generated, it may be convenient to set up a |Makefile| or
shell script to automatise the process.

%%%%%%%%%%%%%%%%%%%%%%%%%%%%%%%%%%%%%%%%%%%%%%%%%%%%%%%%%%%%%%%%%%%%%%%%%%%%%%%%
\subsection{Command Line Processing}
\label{sec:commandline}

The effect of redirection files can also be achieved by invoking
the \LaTeX{} compiler with a more elaborate command line.
Most conveniently this should be done as part
of a shell script or a |Makefile|.

When using \textsf{childdoc} in the main file, the following
command lines effectively perform a redirection
(note that depending on the shell being used,
backslashes may have to be doubled: `|\|' $\to$ `|\\|'):
%
\begin{center}
|... -jobname "|\textit{target}|" |\\|"|[\textit{flags}]%
|\input{childdoc.def}\childdocforward[|\textit{main}|]{|\textit{dest}|}"|
\end{center}
%
Here \textit{target} is the name of the output file,
\textit{main} is the name of the main file
and \textit{dest} is the name of the main or child file to be processed
(all filenames without extensions).
The optional argument \textit{main} can be omitted
if \textit{main} matches \textit{dest}.
Optionally, compilation \textit{flags} can be defined via |\def| commands.
This command line makes the \TeX{} engine believe
it is compiling the file \textit{target}
whose content is specified as the latter parameter.
The provided code then forwards the processing to
\textit{main} or \textit{dest} as described in \secref{sec:forward}.

%%%%%%%%%%%%%%%%%%%%%%%%%%%%%%%%%%%%%%%%%%%%%%%%%%%%%%%%%%%%%%%%%%%%%%%%%%%%%%%%
\subsection{Include by Input}
\label{sec:input}

Including child documents by |\include| has some restrictions by design.
Most notably, the content of a child document always occupies
its own set of pages; pages cannot be shared between child documents.
Usually, this behaviour makes perfect sense
because each child document contain an essential part of the document.
However, in some situations it may be desirable to compose
a document from a collection of parts
without having mandatory page breaks between then.
For this case, the package
provides a mechanism to include parts
by |\input| which can also be processed individually.
However, by construction this mechanism
requires manual handling of the content to be output.

%%%%%%%%%%%%%%%%%%%%%%%%%%%%%%%%%%%%%%%%
\DescribeMacro{\ifchilddocmanual}
The main file should be prepared as usual, see \secref{sec:include}.
However, the document body must make a distinction
between processing of an individual part and of the main document, e.g.:
%
\begin{center}
\begin{tabular}{l}
|\ifchilddocmanual|\\
|\input{\childdocname}|\\
|\||else|\\
\textit{document body with }|\input{|\textit{part}|}|\\
|\||fi|
\end{tabular}
\end{center}
%
The conditional |\ifchilddocmanual| is true whenever
a part to be included by |\input| is being compiled,
and the name of the part is stored in |\childdocname|.

%%%%%%%%%%%%%%%%%%%%%%%%%%%%%%%%%%%%%%%%
\DescribeMacro{\childdocby}
Each part to be included by |\input| should start with:
%
\begin{center}
\begin{tabular}{l}
|\input{childdoc.def}|\\
|\childdocby{|\textit{main}|}|\\
\end{tabular}
\end{center}
%
The directive |\childdocby| is similar to |\childdocof|
described in \secref{sec:include},
but the subsequent selection of content must be done manually.
To that end, both |\ifchilddoc| and |\ifchilddocmanual|
will be true upon processing of a part,
and the name of the part is stored in |\childdocname|.
Note that |\jobname| will be set to the filename of the current part
so that each part receives an individual |.aux| file
that does not interfere with the |.aux| file(s) of the main document.
This behaviour can be altered by the alternative form
|\childdocby[*]{|\textit{main}|}| (with a non-empty optional argument)
which uses the |.aux| file of the main document
by setting |\jobname| to \textit{main}.

%%%%%%%%%%%%%%%%%%%%%%%%%%%%%%%%%%%%%%%%%%%%%%%%%%%%%%%%%%%%%%%%%%%%%%%%%%%%%%%%
\subsection{Driver Development}
\label{sec:driver}

The \textsf{childdoc} mechanism can also be use for the development
of definition files such as \LaTeX{} styles or classes.
This case differs from the above setup with multiple parts
included by |\include| in that no |\includeonly| should be invoked.
This can be achieved by starting the include file
(before |\ProvidesPackage|) with:
%
\begin{center}
\begin{tabular}{l}
|\input{childdoc.def}|\\
|\childdocforward{|\textit{main}|}|\\
\end{tabular}
\end{center}
%
or alternatively with:
%
\begin{center}
\begin{tabular}{l}
|\input{childdoc.def}|\\
|\childdocby{|\textit{main}|}|\\
\end{tabular}
\end{center}
%
Both forms have slightly different effects as described above.
The main file is prepared as usual, see \secref{sec:include}.

%%%%%%%%%%%%%%%%%%%%%%%%%%%%%%%%%%%%%%%%%%%%%%%%%%%%%%%%%%%%%%%%%%%%%%%%%%%%%%%%
\subsection{Legacy Detection}
\label{sec:detection}

The directive |\childdocmain| in the main file can detect
whether the complete document or merely a child is to be compiled
even without using the directive |\childdocof|.
This method is deprecated because it is less robust
and there is no compelling reason to use it;
it is merely provided for backward compatibility
and it may be removed in future versions.

If the detection mechanism is to be used,
it is mandatory to correctly specify
the filename of the main file as the argument of |\childdocmain|:
%
\begin{center}
\begin{tabular}{l}
|\input{childdoc.def}|\\
|\childdocmain{|\textit{main}|}|\\
\end{tabular}
\end{center}
%
If |\jobname| does not match the argument \textit{main} of |\childdocmain|,
it is assumed that |\jobname| points to the child file to be compiled.
When using |\childdocmain| with the main file specified as argument,
it suffices to start a child file
with just |\input{|\textit{main}|}|
without loading of the package and using |\childdocof|.
If instead all processing is done
with the appropriate \textsf{childdoc} directives,
the argument of \textit{main} of |\childdocmain| can be empty.

An alternative version of the command line processing described
in \secref{sec:commandline} using the detection mechanism reads:
%
\begin{center}
|... -jobname "|\textit{target}|" "|[\textit{flags}]%
[|\def\jobname{|\textit{dest}|}|]|\input{|\textit{main}|}"|
\end{center}

%%%%%%%%%%%%%%%%%%%%%%%%%%%%%%%%%%%%%%%%%%%%%%%%%%%%%%%%%%%%%%%%%%%%%%%%%%%%%%%%
\subsection{Manual Code}
\label{sec:manual}

In case one cannot be certain whether the definitions file |childdoc.def|
is installed on the target \TeX{} distribution
and one prefers not to ship it,
it is conceivable to paste a few relevant commands into the sources.

To that end, drop all statements |\input{childdoc.def}|
and perform the replacements as outlined below.
Instead of |\childdocmain{|\textit{main}|}| add the following code
to the top of the main file:
%
\begin{center}
\begin{tabular}{l}
|\||ifdefined\childdocname\endinput\||fi\newif\ifchilddoc|\\
|\edef\childdocname{\scantokens\expandafter{\jobname\noexpand}}|\\
|\def\childdocmain{|\textit{main}|}\||ifx\childdocmain\childdocname\||else|\\
|\childdoctrue\includeonly{\childdocname}\let\jobname\childdocmain\||fi|\\
\end{tabular}
\end{center}
%
Instead of |\childdocof{|\textit{main}|}| just include the main file
at the top of each child file:
%
\begin{center}
|\input{|\textit{main}|}|
\end{center}
%
A simple redirection |\childdocforward{|\textit{dest}|}| is achieved by:
%
\begin{center}
|\def\jobname{|\textit{dest}|}\input{\jobname}|
\end{center}
%
The redirection with prefix
|\childdocforwardprefix[|\textit{prefix}|]{|\textit{dest}|}|
is accomplished by:
%
\begin{center}
\begin{tabular}{l}
|{\edef\jobname{\scantokens\expandafter{\jobname\noexpand}}|\\
|\def\redirectjob |\textit{prefix}|#1~~~{\gdef\jobname{|\textit{dest}|#1}}|\\
|\expandafter\redirectjob\jobname~~~}\input{\jobname}|
\end{tabular}
\end{center}

In an alternative approach,
child documents can be compiled by a specific command line
without additional code or specific definitions:
%
\begin{center}
|... -jobname "|\textit{target}|" "|[\textit{flags}]%
|\includeonly{|\textit{dest}|}\input{|\textit{main}|}"|
\end{center}
%

%%%%%%%%%%%%%%%%%%%%%%%%%%%%%%%%%%%%%%%%%%%%%%%%%%%%%%%%%%%%%%%%%%%%%%%%%%%%%%%%
%%%%%%%%%%%%%%%%%%%%%%%%%%%%%%%%%%%%%%%%%%%%%%%%%%%%%%%%%%%%%%%%%%%%%%%%%%%%%%%%
\section{Information}

%%%%%%%%%%%%%%%%%%%%%%%%%%%%%%%%%%%%%%%%%%%%%%%%%%%%%%%%%%%%%%%%%%%%%%%%%%%%%%%%
\subsection{Copyright}

Copyright \copyright{} 2017--2018 Niklas Beisert

This work may be distributed and/or modified under the
conditions of the \LaTeX{} Project Public License, either version 1.3
of this license or (at your option) any later version.
The latest version of this license is in
  \url{http://www.latex-project.org/lppl.txt}
and version 1.3 or later is part of all distributions of \LaTeX{}
version 2005/12/01 or later.

This work has the LPPL maintenance status `maintained'.

The Current Maintainer of this work is Niklas Beisert.

This work consists of the files |README.txt|, |childdoc.ins| and |childdoc.dtx|
as well as the derived files |childdoc.def|, |cdocsamp.tex|
with |cdocsch1.tex|, |cdocsch2.tex|, |cdocspt3.tex|, |cdocspt4.tex|,
|cdocsdrf.tex|, |cdocsfn1.tex|, |cdocsfn2.tex|
as well as |childdoc.pdf|.

%%%%%%%%%%%%%%%%%%%%%%%%%%%%%%%%%%%%%%%%%%%%%%%%%%%%%%%%%%%%%%%%%%%%%%%%%%%%%%%%
\subsection{Files and Installation}

The package consists of the files:
%
\begin{center}
\begin{tabular}{ll}
    |README.txt|   & readme file \\
    |childdoc.ins| & installation file \\
    |childdoc.dtx| & source file \\
    |childdoc.def| & definition file \\
    |cdocsamp.tex| & sample main file \\
    |cdocsch1.tex| & sample include file \\
    |cdocsch2.tex| & sample include file \\
    |cdocspt3.tex| & sample part file \\
    |cdocspt4.tex| & sample part file \\
    |cdocsdrf.tex| & sample redirection file \\
    |cdocsfn1.tex| & sample redirection file \\
    |cdocsfn2.tex| & sample redirection file \\
    |childdoc.pdf| & manual
\end{tabular}
\end{center}
%
The distribution consists of the files
|README.txt|, |childdoc.ins| and |childdoc.dtx|.
%
\begin{itemize}
\item
Run (pdf)\LaTeX{} on |childdoc.dtx|
to compile the manual |childdoc.pdf| (this file).
\item
Run \LaTeX{} on |childdoc.ins| to create the definitions file |childdoc.def|
and the sample |cdocsamp.tex| with include files
|cdocsch1.tex|, |cdocsch2.tex|, |cdocspt3.tex|, |cdocspt4.tex|,
|cdocsdrf.tex|, |cdocsfn1.tex|, |cdocsfn2.tex|.
Then copy the file |childdoc.def| to an appropriate directory of your \LaTeX{}
distribution, e.g.\ \textit{texmf-root}|/tex/latex/childdoc|.
\end{itemize}

%%%%%%%%%%%%%%%%%%%%%%%%%%%%%%%%%%%%%%%%%%%%%%%%%%%%%%%%%%%%%%%%%%%%%%%%%%%%%%%%
\subsection{Related CTAN Packages}

There are several other packages which offer a similar functionality:
%
\begin{itemize}
\item
The packages
\href{http://ctan.org/pkg/docmute}{\textsf{docmute}},
\href{http://ctan.org/pkg/includex}{\textsf{includex}} and
\href{http://ctan.org/pkg/standalone}{\textsf{standalone}}
provide commands to include only the document body of
a child file thus allowing both files to be compiled individually.
\item
The packages \href{http://ctan.org/pkg/subdocs}{\textsf{subdocs}}
and \href{http://ctan.org/pkg/subfiles}{\textsf{subfiles}}
provide structures in which the main and child documents can be
encapsulated and allowing them to be compiled individually.
The inclusion mechanism is different from the conventional |\include|.
\item
The package \href{http://ctan.org/pkg/combine}{\textsf{combine}}
is an elaborate solution to combine several documents into one.
\end{itemize}
%
See also the CTAN topic \href{http://ctan.org/topic/subdocs}{\textsf{subdocs}}
for further related packages.
The present package differs from the above solutions in that
a document structure constructed with the conventional |\include| mechanism
just needs two extra commands at the top of every file
such that all constituent files can be compiled individually.

%%%%%%%%%%%%%%%%%%%%%%%%%%%%%%%%%%%%%%%%%%%%%%%%%%%%%%%%%%%%%%%%%%%%%%%%%%%%%%%%
%\subsection{Feature Suggestions}
%
%The following is a list of features which may be useful for future
%versions of this package:
%%
%\begin{itemize}
%\item
%\ldots
%\end{itemize}

%%%%%%%%%%%%%%%%%%%%%%%%%%%%%%%%%%%%%%%%%%%%%%%%%%%%%%%%%%%%%%%%%%%%%%%%%%%%%%%%
\subsection{Revision History}

%%%%%%%%%%%%%%%%%%%%%%%%%%%%%%%%%%%%%%%%
\paragraph{v2.0:} 2018/12/30

\begin{itemize}
\item
immediate forward processing
\item
added |\childdocby| mechanism
\item
manual restructured
\end{itemize}

%%%%%%%%%%%%%%%%%%%%%%%%%%%%%%%%%%%%%%%%
\paragraph{v1.6:} 2018/01/17

\begin{itemize}
\item
application for development of include files
\item
corrections to manual
\end{itemize}

%%%%%%%%%%%%%%%%%%%%%%%%%%%%%%%%%%%%%%%%
\paragraph{v1.5:} 2017/05/21

\begin{itemize}
\item
more complete structuring introduced
\item
|\childdocof| introduced
\item
|\childdoc| renamed to |\childdocmain|
\item
|\childredirect| renamed to |\childdocforward| and |\childdocforwardprefix|
and functionality expanded
\end{itemize}

%%%%%%%%%%%%%%%%%%%%%%%%%%%%%%%%%%%%%%%%
\paragraph{v1.0:} 2017/04/27

\begin{itemize}
\item
manual and install package
\item
first version published on CTAN
\end{itemize}

%%%%%%%%%%%%%%%%%%%%%%%%%%%%%%%%%%%%%%%%
\paragraph{v0.6:} 2017/04/26

\begin{itemize}
\item
redirection mechanism added
\end{itemize}

%%%%%%%%%%%%%%%%%%%%%%%%%%%%%%%%%%%%%%%%
\paragraph{v0.5:} 2017/04/26

\begin{itemize}
\item
functionality in definition file
\end{itemize}


%%%%%%%%%%%%%%%%%%%%%%%%%%%%%%%%%%%%%%%%%%%%%%%%%%%%%%%%%%%%%%%%%%%%%%%%%%%%%%%%
%%%%%%%%%%%%%%%%%%%%%%%%%%%%%%%%%%%%%%%%%%%%%%%%%%%%%%%%%%%%%%%%%%%%%%%%%%%%%%%%
%%%%%%%%%%%%%%%%%%%%%%%%%%%%%%%%%%%%%%%%%%%%%%%%%%%%%%%%%%%%%%%%%%%%%%%%%%%%%%%%
\appendix

\settowidth\MacroIndent{\rmfamily\scriptsize 000\ }

 \DocInput{childdoc.dtx}

\end{document}
%</driver>
% \fi
%
% %%%%%%%%%%%%%%%%%%%%%%%%%%%%%%%%%%%%%%%%%%%%%%%%%%%%%%%%%%%%%%%%%%%%%%%%%%%%%%
% %%%%%%%%%%%%%%%%%%%%%%%%%%%%%%%%%%%%%%%%%%%%%%%%%%%%%%%%%%%%%%%%%%%%%%%%%%%%%%
% \section{Sample}
%\iffalse
%<*samplemain>
%\fi
%
% The following presents a sample document
% with two chapters, two parts, a title page,
% a compile flag as well as three forwarding files to set the flag.
% It consists of eight |.tex| files:
% \begin{center}
% \begin{tabular}{ll}
% |cdocsamp.tex|&main file\\
% |cdocsch1.tex|&include file for chapter 1\\
% |cdocsch2.tex|&include file for chapter 2\\
% |cdocspt3.tex|&include file for part 3\\
% |cdocspt4.tex|&include file for part 4\\
% |cdocsdrf.tex|&forwarding file for main file in draft mode\\
% |cdocsfi1.tex|&forwarding file for final version of chapter 1\\
% |cdocsfi2.tex|&forwarding file for final version of chapter 2\\
% \end{tabular}
% \end{center}
% Each of the eight files can be compiled directly by the \LaTeX{} compiler.
%
% %%%%%%%%%%%%%%%%%%%%%%%%%%%%%%%%%%%%%%
% \paragraph{Main File.}
%
% The main file is called |cdocsamp.tex|.
%
% Load the \textsf{childdoc} definitions and
% declare the filename for the main document:
%    \begin{macrocode}
\input{childdoc.def}
\childdocmain{}
%    \end{macrocode}

% Optional override for |\version| flag:
%    \begin{macrocode}
%%\ifchilddoc\else\providecommand{\version}{draft}\fi
%    \end{macrocode}

% Define the default values for the |\version| flag
% (|final| for the main file and |draft| for childs):
%    \begin{macrocode}
\ifchilddoc
\providecommand{\version}{draft}
\else
\providecommand{\version}{final}
\fi
%    \end{macrocode}

% Load the standard document class:
%    \begin{macrocode}
\documentclass[12pt]{article}
%    \end{macrocode}

% Start the document body:
%    \begin{macrocode}
\begin{document}
%    \end{macrocode}

% Declare a title page.
% Print title, part of document being processed and version flag:
%    \begin{macrocode}
\addtocounter{page}{-1}
\begin{center}
{\LARGE\bfseries{}childdoc example\par}
\vspace{1cm}
\ifchilddoc
\ifchilddocmanual part\else chapter\fi:
`\childdocname' of `\childdocjob'\par
\else
main document: `\childdocjob'\par
\fi
version: \version\par
\end{center}
\newpage
%    \end{macrocode}

% Manually include selected file,
% otherwise process as usual:
%    \begin{macrocode}
\ifchilddocmanual
\section*{part `\childdocname'}
\input{\childdocname}
\else
%    \end{macrocode}

% Include the two chapters:
%    \begin{macrocode}
\include{cdocsch1}
\include{cdocsch2}
%    \end{macrocode}

% Include the two parts unless only chapters should be displayed:
%    \begin{macrocode}
\ifchilddoc\else
\section{part three}
\input{cdocspt3}
\section{part four}
\input{cdocspt4}
\fi
%    \end{macrocode}

% Process as usual until here:
%    \begin{macrocode}
\fi
%    \end{macrocode}

% End of document body:
%    \begin{macrocode}
\end{document}
%    \end{macrocode}
%\iffalse
%</samplemain>
%\fi
%
% %%%%%%%%%%%%%%%%%%%%%%%%%%%%%%%%%%%%%%
% \paragraph{Chapter Include Files.}
%
% The include files are called |cdocsch1.tex| and |cdocsch2.tex|.
%
%\iffalse
%<*samplechap1|samplechap2>
%\fi

% Optional override for |\version| flag:
%    \begin{macrocode}
%%\providecommand{\version}{final}
%    \end{macrocode}

% Include the main document:
%    \begin{macrocode}
\input{childdoc.def}
\childdocof{cdocsamp}
%    \end{macrocode}

%\iffalse
%</samplechap1|samplechap2>
%\fi
%
%\iffalse
%<*samplechap1>
%\fi
% Some text for chapter 1:
%    \begin{macrocode}
\section{one}
some text in chapter one
%    \end{macrocode}

%\iffalse
%</samplechap1>
%\fi
% Some text for chapter 2:
%\iffalse
%<*samplechap2>
%\fi
%    \begin{macrocode}
\section{two}
more text in chapter two
%    \end{macrocode}

%\iffalse
%</samplechap2>
%\fi
%
% %%%%%%%%%%%%%%%%%%%%%%%%%%%%%%%%%%%%%%
% \paragraph{Part Include Files.}
%
% The include files are called |cdocspt3.tex| and |cdocspt4.tex|.
%
%\iffalse
%<*samplepart3|samplepart4>
%\fi

% Optional override for |\version| flag:
%    \begin{macrocode}
%%\providecommand{\version}{final}
%    \end{macrocode}

% Include the main document:
%    \begin{macrocode}
\input{childdoc.def}
\childdocby{cdocsamp}
%    \end{macrocode}

%\iffalse
%</samplepart3|samplepart4>
%\fi
%
%\iffalse
%<*samplepart3>
%\fi
% Some text for part 3:
%    \begin{macrocode}
some text in part three
%    \end{macrocode}

%\iffalse
%</samplepart3>
%\fi
% Some text for part 4:
%\iffalse
%<*samplepart4>
%\fi
%    \begin{macrocode}
more text in part four
%    \end{macrocode}

%\iffalse
%</samplepart4>
%\fi
%
% %%%%%%%%%%%%%%%%%%%%%%%%%%%%%%%%%%%%%%
% \paragraph{Forwarding for a Complete Draft.}
%
% The following forwarding file |cdocsdrf.tex|
% compiles the main document in draft mode:
%\iffalse
%<*sampledraft>
%\fi
%    \begin{macrocode}
\def\version{draft}
\input{childdoc.def}
\childdocforward{cdocsamp}
%    \end{macrocode}

%\iffalse
%</sampledraft>
%\fi
%
% %%%%%%%%%%%%%%%%%%%%%%%%%%%%%%%%%%%%%%
% \paragraph{Forwarding for Final Version of the Chapters.}
%
% The following forwarding files |cdocsfn1.tex| and |cdocsfn2.tex|
% (with identical content)
% compile the final versions of the child documents
% |cdocsch1.tex| and |cdocsch2.tex|, respectively:
%\iffalse
%<*samplefinal>
%\fi
%    \begin{macrocode}
\def\version{final}
\input{childdoc.def}
\childdocforwardprefix[cdocsamp]{cdocsfn}{cdocsch}
%    \end{macrocode}

%\iffalse
%</samplefinal>
%\fi
%
% %%%%%%%%%%%%%%%%%%%%%%%%%%%%%%%%%%%%%%
% \paragraph{Command Line Processing.}
%
% The following three command lines generate the output files
% |cdocscld|, |cdocscl1| and |cdocscl2|
% which should be identical to
% |cdocsdrf|, |cdocsch1| and |cdocsfn2|, respectively:
% \begin{center}
% \begin{tabular}{l}
% |latex -jobname cdocscld \|\\
% |  "\def\version{draft}\input{childdoc.def}\childdocforward{cdocsamp}"|\\
% |latex -jobname cdocscl1 \|\\
% |  "\input{childdoc.def}\childdocforward[cdocsamp]{cdocsch1}"|\\
% |latex -jobname cdocscl2 \|\\
% |  "\def\version{final}\input{childdoc.def}\childdocforward{cdocsch2}"|
% \end{tabular}
% \end{center}
% Note that the trailing backslash on each first line
% merely continues the input to the second line
% (for convenient cut ant paste).
% Furthermore, the command |latex| can be replaced by any
% of its alternative versions such as |pdflatex|.
%
% %%%%%%%%%%%%%%%%%%%%%%%%%%%%%%%%%%%%%%%%%%%%%%%%%%%%%%%%%%%%%%%%%%%%%%%%%%%%%%
% %%%%%%%%%%%%%%%%%%%%%%%%%%%%%%%%%%%%%%%%%%%%%%%%%%%%%%%%%%%%%%%%%%%%%%%%%%%%%%
% \section{Implementation}
%\iffalse
%<*package>
%\fi
%
% This section describes the definitions file |childdoc.def|.

% The definitions cannot be loaded using |\usepackage| or |\RequirePackage|
% which has a mechanism to prevent loading a style file more than once.
% When loading the definitions by means of |\input|
% multiple instances have to be prevented manually:
%\iffalse
%This code needs to be before the `\ProvidesFile' directive
%which is defined at the beginning of this file.
%Therefore it is also placed there and commented out here.
%</package>
%<*discard>
%\fi
%    \begin{macrocode}
\ifdefined\childdocmain\endinput\fi
%    \end{macrocode}
%\iffalse
%</discard>
%<*package>
%\fi
%
% \macro{\ifchilddoc}
% \macro{\ifchilddocmanual}
% The conditional |\ifchilddoc| tells whether a
% child (true) or main (false) document is being compiled.
% The conditional |\ifchilddocmanual| tells whether
% the |\includeonly| mechanism is used (false) or
% the selection of child files must be performed manually (true).
% The definitions initialise to false:
%    \begin{macrocode}
\newif\ifchilddoc
\newif\ifchilddocmanual
%    \end{macrocode}

% \macro{\childdocname}
% \macro{\childdocjob}
% The macro |\childdocname| stores the name of the main document
% to be compiled. The macro |\childdocjob| stores the name of
% the document on which the \LaTeX{} compiler was originally invoked.
% The content of |\jobname| cannot be compared
% to filenames specified in the source due to different catcodes.
% The following code rescans |\jobname|, stores the result
% in |\childdocname| and saves a copy in |\childdocjob|:
%    \begin{macrocode}
\edef\childdocname{\scantokens\expandafter{\jobname\noexpand}}
\let\childdocjob\childdocname
%    \end{macrocode}

% \macro{\childdocdisable}
% The macro |\childdocdisable| prevents the main file
% from being processed more than once.
% At this stage, the main document command |\childdocmain|
% is assumed to be called once again where it should do nothing.
% Any subsequent call to it should prevent
% a secondary processing of the main document
% It overwrites the forwarding commands
% |\childdocof| and |\childdocforward|
% with empty macros to prevent further inclusions of the main document:
%    \begin{macrocode}
\newcommand{\childdocdisable}
{
  \renewcommand{\childdocmain}[1]{\renewcommand{\childdocmain}[1]{\endinput}}
  \renewcommand{\childdocof}[1]{}
  \renewcommand{\childdocby}[2][]{}
  \renewcommand{\childdocforward}[2][]{}
  \renewcommand{\childdocdisable}{}
}
%    \end{macrocode}

% \macro{\childdocmain}
% The macro |\childdocmain| is to be called at the top of the main file
% with nothing or the main filename (without extension) as argument.
% First, it breaks loops.
% If the argument is not empty and does not match |\childdocname|
% (which is set by the first inclusion of |childdoc.def|),
% |\ifchilddoc| is set to true, |\includeonly| is applied to the child file
% and |\jobname| is set to the main file
% (for proper handling of |.aux| files):
%    \begin{macrocode}
\newcommand{\childdocmain}[1]
{
  \childdocdisable\childdocmain{}
  \if?#1?\else
    \begingroup
      \def\childdoctmp{#1}
      \ifx\childdoctmp\childdocname
        \def\childdoctmp{}
      \else
        \def\childdoctmp
        {
          \childdoctrue
          \includeonly{\childdocname}
          \def\childdocjob{#1}
          \def\jobname{#1}
        }
      \fi
      \expandafter
    \endgroup
    \childdoctmp
  \fi
}
%    \end{macrocode}

% \macro{\childdocof}
% The command |\childdocof| redirects
% compilation to the main file |#1|.
%    \begin{macrocode}
\newcommand{\childdocof}[1]
{
  \childdocdisable
  \childdoctrue
  \includeonly{\childdocname}
  \def\jobname{#1}
  \def\childdocjob{#1}
  \input{#1}
}
%    \end{macrocode}

% \macro{\childdocby}
% The command |\childdocby| ....
%    \begin{macrocode}
\newcommand{\childdocby}[2][]
{
  \childdocdisable
  \childdoctrue
  \childdocmanualtrue
  \if?#1?\else
    \def\jobname{#2}
  \fi
  \def\childdocjob{#2}
  \input{#2}
  \endinput
}
%    \end{macrocode}

% \macro{\childdocforward}
% The command |\childdocforward| redirects
% compilation to the main file or
% (if the optional argument is given) a child file.
% Parameters are set as if the main file
% or a child file starting with |\childdocof| was compiled.
% Then compilation is handed over to the main file:
%    \begin{macrocode}
\newcommand{\childdocforward}[2][]
{
  \begingroup
    \if?#1?
      \def\childdoctmp
      {
        \def\childdocname{#2}
        \def\childdocjob{#2}
        \def\jobname{#2}
        \input{#2}
        \endinput
      }
    \else
      \def\childdoctmp
      {
        \childdocdisable
        \def\childdocname{#2}
        \childdoctrue
        \includeonly{#2}
        \def\childdocjob{#1}
        \def\jobname{#1}
        \input{#1}
        \endinput
      }
    \fi
    \expandafter
  \endgroup
  \childdoctmp
}
%    \end{macrocode}

% \macro{\childdocforwardprefix}
% The command |\childdocforwardprefix| redirects
% compilation to the main or a child file by means of a pattern.
% The prefix |#1| in the current filename is replaced by |#2|
% and the suffix of the current filename is kept
% (it is assumed that the filename does not contain the substring `|~~~|'
% which is used as a delimiter).
% Compilation is handed over to the new file by |\childdocforward|:
%    \begin{macrocode}
\newcommand{\childdocforwardprefix}[3][]
{
  \begingroup
    \def\childdocextract #2##1~~~{\def\childdoctmp{\childdocforward[#1]{#3##1}}}
    \expandafter\childdocextract\childdocname~~~
    \expandafter
  \endgroup
  \childdoctmp
}
%    \end{macrocode}

% \macro{\childdoc}
% The deprecated macro |\childdoc| is a legacy version of |\childdocmain|:
%    \begin{macrocode}
\newcommand{\childdoc}{\childdocmain}
%    \end{macrocode}

% \macro{\childdocredirect}
% The deprecated macro |\childdocredirect| is a legacy version
% of |\childdocforward| and |\childdocforwardprefix|:
%    \begin{macrocode}
\newcommand{\childdocredirect}[2][]
{
  \begingroup
    \if?#1?
      \def\childdoctmp{\childdocforward{#2}}
    \else
      \def\childdoctmp{\childdocforwardprefix{#1}{#2}}
    \fi
    \expandafter
  \endgroup
  \childdoctmp
}
%    \end{macrocode}

%\iffalse
%</package>
%\fi
%
\endinput
|\\
|\childdocby{|\textit{main}|}|\\
\end{tabular}
\end{center}
%
Both forms have slightly different effects as described above.
The main file is prepared as usual, see \secref{sec:include}.

%%%%%%%%%%%%%%%%%%%%%%%%%%%%%%%%%%%%%%%%%%%%%%%%%%%%%%%%%%%%%%%%%%%%%%%%%%%%%%%%
\subsection{Legacy Detection}
\label{sec:detection}

The directive |\childdocmain| in the main file can detect
whether the complete document or merely a child is to be compiled
even without using the directive |\childdocof|.
This method is deprecated because it is less robust
and there is no compelling reason to use it;
it is merely provided for backward compatibility
and it may be removed in future versions.

If the detection mechanism is to be used,
it is mandatory to correctly specify
the filename of the main file as the argument of |\childdocmain|:
%
\begin{center}
\begin{tabular}{l}
|% \iffalse
%
% childdoc.dtx Copyright (C) 2017-2018 Niklas Beisert
%
% This work may be distributed and/or modified under the
% conditions of the LaTeX Project Public License, either version 1.3
% of this license or (at your option) any later version.
% The latest version of this license is in
%   http://www.latex-project.org/lppl.txt
% and version 1.3 or later is part of all distributions of LaTeX
% version 2005/12/01 or later.
%
% This work has the LPPL maintenance status `maintained'.
%
% The Current Maintainer of this work is Niklas Beisert.
%
% This work consists of the files childdoc.dtx and childdoc.ins
% and the derived files childdoc.def and cdocsamp.tex with
% cdocsch1.tex, cdocsch2.tex, cdocsdrf.tex, cdocsfn1.tex, cdocsfn2.tex.
%
%<package>\ifdefined\childdocmain\endinput\fi
%<package>\ProvidesFile{childdoc.def}[2018/12/30 v2.0 child document driver]
%<samplemain>\ProvidesFile{cdocsamp.tex}[2018/12/30 v2.0 sample for childdoc]
%<*driver>
%\ProvidesFile{childdoc.drv}[2018/12/30 v2.0 childdoc reference manual file]
\PassOptionsToClass{10pt,a4paper}{article}
\documentclass{ltxdoc}

\usepackage[margin=35mm]{geometry}
\usepackage{hyperref}
\usepackage{hyperxmp}
\usepackage[usenames]{color}

\hypersetup{colorlinks=true}
\hypersetup{pdfstartview=FitH}
\hypersetup{pdfpagemode=UseNone}
\hypersetup{pdfsource={}}
\hypersetup{pdflang={en-UK}}
\hypersetup{pdfcopyright={Copyright 2017-2018 Niklas Beisert.
  This work may be distributed and/or modified under the
  conditions of the LaTeX Project Public License, either version 1.3
  of this license or (at your option) any later version.}}
\hypersetup{pdflicenseurl={http://www.latex-project.org/lppl.txt}}
\hypersetup{pdfcontactaddress={ETH Zurich, ITP, HIT K,
  Wolfgang-Pauli-Strasse 27}}
\hypersetup{pdfcontactpostcode={8093}}
\hypersetup{pdfcontactcity={Zurich}}
\hypersetup{pdfcontactcountry={Switzerland}}
\hypersetup{pdfcontactemail={nbeisert@itp.phys.ethz.ch}}
\hypersetup{pdfcontacturl={http://people.phys.ethz.ch/\xmptilde nbeisert/}}

\newcommand{\secref}[1]{\hyperref[#1]{section \ref*{#1}}}

\parskip1ex
\parindent0pt
\let\olditemize\itemize
\def\itemize{\olditemize\parskip0pt}

\begin{document}

\title{The \textsf{childdoc} Package}
\hypersetup{pdftitle={The childdoc Package}}
\author{Niklas Beisert\\[2ex]
  Institut f\"ur Theoretische Physik\\
  Eidgen\"ossische Technische Hochschule Z\"urich\\
  Wolfgang-Pauli-Strasse 27, 8093 Z\"urich, Switzerland\\[1ex]
  \href{mailto:nbeisert@itp.phys.ethz.ch}
  {\texttt{nbeisert@itp.phys.ethz.ch}}}
\hypersetup{pdfauthor={Niklas Beisert}}
\hypersetup{pdfsubject={Manual for the LaTeX2e Package childdoc}}
\date{30 December 2018, \textsf{v2.0}}
\maketitle

\begin{abstract}\noindent
\textsf{childdoc} is a \LaTeXe{} package
that enables the direct compilation
of document sections included by |\include|
to individual files.
\end{abstract}

\begingroup
\parskip0ex
\tableofcontents
\endgroup

%%%%%%%%%%%%%%%%%%%%%%%%%%%%%%%%%%%%%%%%%%%%%%%%%%%%%%%%%%%%%%%%%%%%%%%%%%%%%%%%
%%%%%%%%%%%%%%%%%%%%%%%%%%%%%%%%%%%%%%%%%%%%%%%%%%%%%%%%%%%%%%%%%%%%%%%%%%%%%%%%
\section{Introduction}

\LaTeX{} provides a mechanism to structure a large document (such as a book)
into a main file and several child files (containing the chapters)
using the |\include| command.
This mechanism is beneficial for documents
which span hundreds of pages in order to
make the source file(s) more manageable.
Moreover, compilation can be restricted to
selected child files by means of the |\includeonly| command.
The latter feature can be used to reduce the compilation time while editing
(this was significantly more useful in the earlier days of \LaTeX{})
or to generate a smaller document which is easier to navigate.
Another application of |\includeonly| is to generate
documents consisting of selected parts of the complete document.

However, there are a few drawbacks of the plain |\include| mechanism:
\begin{itemize}
\item
The child files cannot be compiled on their own,
they can only be compiled via the main file.
A naive editing environment
(such as a text editor with an option
to have the current file processed by \LaTeX)
may require one to switch to the main file before compiling;
attempting to compile the child file produces errors.
\item
The main file must be modified (each time)
to adjust the |\includeonly| command
to the present needs. This easily leaves the main file in a messy state.
\item
The generated document will always carry the filename
of the main document. This is inconvenient if
several child files are to be compiled and
to be kept for distribution.
\end{itemize}

The present package provides a simple interface
to make child files individually compilable by \LaTeX{}.
Compiling a child file then has the same effect as compiling
the main file with an |\includeonly| command
to select the appropriate child.
Moreover the generated document will carry the name of the child
rather than the main file.
This resolves all three above issues.

This feature is meant to make the editing of books,
thesis documents and lecture notes somewhat more convenient.
However, the package can also be used efficiently for
composing a series of documents (such as exercise sheets)
which are typically distributed individually.
It then assists the author in generating the individual documents
(potentially in different versions)
as well as a document containing the collected series.
Another application is in developing style files
or other kinds of included material
where compilation of the style file could redirect
to a sample or test file.

%%%%%%%%%%%%%%%%%%%%%%%%%%%%%%%%%%%%%%%%%%%%%%%%%%%%%%%%%%%%%%%%%%%%%%%%%%%%%%%%
%%%%%%%%%%%%%%%%%%%%%%%%%%%%%%%%%%%%%%%%%%%%%%%%%%%%%%%%%%%%%%%%%%%%%%%%%%%%%%%%
\section{Usage}

First of all, the package \textsf{childdoc} is \emph{not} a standard
\LaTeXe{} |.sty| style file! Therefore it needs to be invoked in
a non-standard way.

%%%%%%%%%%%%%%%%%%%%%%%%%%%%%%%%%%%%%%%%%%%%%%%%%%%%%%%%%%%%%%%%%%%%%%%%%%%%%%%%
\subsection{Included Files}
\label{sec:include}

%%%%%%%%%%%%%%%%%%%%%%%%%%%%%%%%%%%%%%%%
\DescribeMacro{\childdocmain}
To use the package, add the commands
\begin{center}
\begin{tabular}{l}
|\input{childdoc.def}|\\
|\childdocmain{}|\\
\end{tabular}
\end{center}
at the very top of the main \LaTeX{} file,
in particular \emph{before} the |\documentclass| statement!
The argument of |\childdocmain| should be left empty
(but it must be present).

%%%%%%%%%%%%%%%%%%%%%%%%%%%%%%%%%%%%%%%%
\DescribeMacro{\childdocof}
Furthermore, add the commands
\begin{center}
\begin{tabular}{l}
|\input{childdoc.def}|\\
|\childdocof{|\textit{main}|}|\\
\end{tabular}
\end{center}
at the top of every child file \textit{child}
which is included by |\include{|\textit{child}|}|
from within the main file
(or at least for those files to be compiled individually).
The argument \textit{main} must be the filename of the main file.

There are a couple of
considerations in setting up the main and child documents:

%%%%%%%%%%%%%%%%%%%%%%%%%%%%%%%%%%%%%%%%
\paragraph{Restrictions.}

Please note the following restrictions:
\begin{itemize}
\item
|\childdocmain| must be called with one argument \textit{main}
to ensure compatibility with earlier version of the package.
It must either be empty (|\childdocmain{}|)
or precisely match the filename of the main file in which it is specified.
See \secref{sec:detection} for further information.
\item
The filename \textit{main} must be specified without the |.tex| extension.
\item
The filename \textit{main} is case sensitive
(even in case-insensitive file systems)
due to internal string comparison.
\item
The argument \textit{main} should be fully expanded, it cannot be a macro.
\item
Subdirectories and special characters should be avoided in filenames.
\item
The command |\childdocmain{|\textit{main}|}| must be followed by a whitespace.
It should not be followed immediately by another command
or by a comment mark `|%|'.
This is because the \TeX{} parser reads the token immediately following
the argument of |\childdocmain| and puts it
at the beginning of every child section;
however, a white\-space is ignored.
\end{itemize}

%%%%%%%%%%%%%%%%%%%%%%%%%%%%%%%%%%%%%%%%
\paragraph{Content of Main File.}

It is advisable to place all content in the child files included by |\include|.
Any output contained in the main file will appear in all child documents
unless suppressed manually;
it cannot be suppressed automatically by the |\includeonly| directive
and thus should normally be avoided.
A method to include some content in the main file
by means of conditional processing is described in \secref{sec:conditional}.

%%%%%%%%%%%%%%%%%%%%%%%%%%%%%%%%%%%%%%%%
\paragraph{Page Numbering.}

When only a part of the document is compiled,
the appropriate numbering of pages
(as well as other status parameters)
is determined from the |.aux| files.
The latter contain information from previous passes.
However this information needs to propagate through
all intermediate child documents.
Therefore the page numbering in child documents may well
be inconsistent until the complete document is compiled at least once.

A useful (if unconventional) way to always ensure a consistent
page numbering is to restart the numbering in each child document
and denote the pages by `\textit{child}|.|\textit{page}'
where \textit{child} represents the chapter/section number of the child file.
This can be achieved by the command
|\numberwithin{page}{|\textit{child}|}|
of the \textsf{amsmath} package
where \textit{child} can be |chapter| or |section|
depending on the chosen structuring.
Alternatively, one can modify the macro |\thepage| appropriately
and reset the counter |page| at the start of each child file.

%%%%%%%%%%%%%%%%%%%%%%%%%%%%%%%%%%%%%%%%%%%%%%%%%%%%%%%%%%%%%%%%%%%%%%%%%%%%%%%%
\subsection{Conditional Processing}
\label{sec:conditional}

The package provides a mechanism to compile different versions
of a document. To customise the versions further some conditional processing
can come in handy to distinguish which version is being compiled.
The package provides two macros to describe the compilation context:

%%%%%%%%%%%%%%%%%%%%%%%%%%%%%%%%%%%%%%%%
\DescribeMacro{\ifchilddoc}
The conditional |\ifchilddoc| distinguishes between the compilation of
child documents and the main document:
%
\begin{center}
|\ifchilddoc |\textit{child-code}| |[|\||else |\textit{main-code}]| \||fi|
\end{center}

%%%%%%%%%%%%%%%%%%%%%%%%%%%%%%%%%%%%%%%%
\DescribeMacro{\childdocname}
\DescribeMacro{\childdocjob}
The macro |\childdocname| contains the filename (without extension)
of the main or child file being processed.
Note that |\childdocjob| will always contain the name of the main file.

%%%%%%%%%%%%%%%%%%%%%%%%%%%%%%%%%%%%%%%%
\paragraph{Title Page.}

Conditional processing can be used to include a title or banner page
in the main document when proper precautions are taken.
Importantly, the code in the main file should ensure that the page counter
(as well as other status parameters which are stored in the |.aux| files)
takes the same value after the conditional processing.
Otherwise the page numbers may take divergent values
depending on which part is compiled.

For example, a title page could be declared by:
%
\begin{center}
\begin{tabular}{l}
|\ifchilddoc\||else|\\
|\addtocounter{page}{-1}|\\
\textit{code for title page}\\
|\newpage|\\
|\||fi|
\end{tabular}
\end{center}
%
A banner page for the child documents can be generated by:
%
\begin{center}
\begin{tabular}{l}
|\ifchilddoc|\\
|\addtocounter{page}{-1}|\\
\textit{code for banner page}\\
|\newpage|\\
|\||fi|
\end{tabular}
\end{center}
%
Here one could write a message such as:
\begin{center}
|This is the part \childdocname{} of \childdocjob{}.|
\end{center}

%%%%%%%%%%%%%%%%%%%%%%%%%%%%%%%%%%%%%%%%%%%%%%%%%%%%%%%%%%%%%%%%%%%%%%%%%%%%%%%%
\subsection{Flags}
\label{sec:flags}

The package makes it easy to generate different versions
of the main or child documents.
To this end compilation flags can be defined
and assigned different default values.
They will be particularly useful in conjunction
with the forwarding mechanism described in \secref{sec:forward}.

For example, it may be useful to have a flag |\version|
which can be set to |draft| or |final|.
The document source will contain some conditional code
depending on the value of |\version|.
Suppose further, the flag should default to |final| for the main file
and to |draft| for child files
which is a natural assignment for editing the document.
This is achieved by placing the following code
in the preamble of the main document
(below the |\childdocmain| directive):
%
\begin{center}
\begin{tabular}{l}
|\ifchilddoc|\\
|\providecommand{\version}{draft}|\\
|\||else|\\
|\providecommand{\version}{final}|\\
|\||fi|
\end{tabular}
\end{center}
%
The definition by |\providecommand| makes sure
that previous definitions are not overwritten.
Further statements |\providecommand{\version}{...}|
can thus be added before the above code to override it.

For the main file, one might add a line
(between |\childdocmain| and the above block)
%
\begin{center}
|%\ifchilddoc\||else\providecommand{\version}{draft}\||fi|
\end{center}
%
which can be uncommented to produce a draft version.
Likewise one can add a line to the very top of a child file
(above the |\childdocof{|\textit{main}|}| directive)
%
\begin{center}
|%\providecommand{\version}{final}|
\end{center}
%
which can be uncommented to produce the final version of this child document.

%%%%%%%%%%%%%%%%%%%%%%%%%%%%%%%%%%%%%%%%%%%%%%%%%%%%%%%%%%%%%%%%%%%%%%%%%%%%%%%%
\subsection{Forwarding}
\label{sec:forward}

Different versions of the main or child documents
using compilation flags as described in \secref{sec:flags}
can be (permanently) stored in different files
for convenient compilation, viewing and distribution.
To this end, the package defines a command
to pass on compilation to a different file:

%%%%%%%%%%%%%%%%%%%%%%%%%%%%%%%%%%%%%%%%
\DescribeMacro{\childdocforward}
The command |\childdocforward| redirects processing to
another source file:
%
\begin{center}
\begin{tabular}{l}
|\input{childdoc.def}|\\
|\childdocforward[|\textit{main}|]{|\textit{dest}|}|\\
\end{tabular}
\end{center}
%
The argument \textit{dest} is the destination file
(without extension).
It should be the main file or one of the child files.
Note that further \textsf{childdoc} directives
such as |\childdocof| and |\childdocforward|
in the indicated file will be processed in this form.
The optional argument \textit{main}
passes on directly to the main file \textit{main}
while pretending to compile the child \textit{dest}.
This form behaves as if \textit{dest}
issues |\childdocof{|\textit{main}|}| right away,
and no further \textsf{childdoc} directives will be processed.

%%%%%%%%%%%%%%%%%%%%%%%%%%%%%%%%%%%%%%%%
\DescribeMacro{\...prefix}
In the alternative form |\childdocforwardprefix|,
%
\begin{center}
\begin{tabular}{l}
|\input{childdoc.def}|\\
|\childdocforwardprefix[|\textit{main}|]{|\textit{prefix}|}{|\textit{dest}|}|
\end{tabular}
\end{center}
%
the destination file is determined by a pattern
depending on the current file:
To make this work, the current file must be called
`{\textit{prefix}\hspace{0.2em}\textit{suffix}}'
with \textit{prefix} matching precisely the argument.
Processing is then passed on to the file
`{\textit{dest}\hspace{0.2em}\textit{suffix}}'.
Surely, the same effect is achieved by
directly specifying the
argument `{\textit{dest}\hspace{0.2em}\textit{suffix}}'
in the first form.
However, that requires to set up a different file
for each child. With the alternative form of the command
all these files can have exactly the same content
which simplifies setting them up and maintaining them.

For example, the following file |draft.tex|
with a compilation flag |\version| as described in \secref{sec:flags}
compiles the main document as a draft:
%
\begin{center}
\begin{tabular}{l}
|\def\version{draft}|\\
|\input{childdoc.def}|\\
|\childdocforward{|\textit{main}|}|
\end{tabular}
\end{center}
%
Likewise, the following files |final|\textit{nn}|.tex|
compile the final version of the child document
|child|\textit{nn}|.tex|:
%
\begin{center}
\begin{tabular}{l}
|\def\version{final}|\\
|\input{childdoc.def}|\\
|\childdocforwardprefix{final}{child}|
\end{tabular}
\end{center}
%

Note that when several versions of a main file and/or of each child file
are to be generated, it may be convenient to set up a |Makefile| or
shell script to automatise the process.

%%%%%%%%%%%%%%%%%%%%%%%%%%%%%%%%%%%%%%%%%%%%%%%%%%%%%%%%%%%%%%%%%%%%%%%%%%%%%%%%
\subsection{Command Line Processing}
\label{sec:commandline}

The effect of redirection files can also be achieved by invoking
the \LaTeX{} compiler with a more elaborate command line.
Most conveniently this should be done as part
of a shell script or a |Makefile|.

When using \textsf{childdoc} in the main file, the following
command lines effectively perform a redirection
(note that depending on the shell being used,
backslashes may have to be doubled: `|\|' $\to$ `|\\|'):
%
\begin{center}
|... -jobname "|\textit{target}|" |\\|"|[\textit{flags}]%
|\input{childdoc.def}\childdocforward[|\textit{main}|]{|\textit{dest}|}"|
\end{center}
%
Here \textit{target} is the name of the output file,
\textit{main} is the name of the main file
and \textit{dest} is the name of the main or child file to be processed
(all filenames without extensions).
The optional argument \textit{main} can be omitted
if \textit{main} matches \textit{dest}.
Optionally, compilation \textit{flags} can be defined via |\def| commands.
This command line makes the \TeX{} engine believe
it is compiling the file \textit{target}
whose content is specified as the latter parameter.
The provided code then forwards the processing to
\textit{main} or \textit{dest} as described in \secref{sec:forward}.

%%%%%%%%%%%%%%%%%%%%%%%%%%%%%%%%%%%%%%%%%%%%%%%%%%%%%%%%%%%%%%%%%%%%%%%%%%%%%%%%
\subsection{Include by Input}
\label{sec:input}

Including child documents by |\include| has some restrictions by design.
Most notably, the content of a child document always occupies
its own set of pages; pages cannot be shared between child documents.
Usually, this behaviour makes perfect sense
because each child document contain an essential part of the document.
However, in some situations it may be desirable to compose
a document from a collection of parts
without having mandatory page breaks between then.
For this case, the package
provides a mechanism to include parts
by |\input| which can also be processed individually.
However, by construction this mechanism
requires manual handling of the content to be output.

%%%%%%%%%%%%%%%%%%%%%%%%%%%%%%%%%%%%%%%%
\DescribeMacro{\ifchilddocmanual}
The main file should be prepared as usual, see \secref{sec:include}.
However, the document body must make a distinction
between processing of an individual part and of the main document, e.g.:
%
\begin{center}
\begin{tabular}{l}
|\ifchilddocmanual|\\
|\input{\childdocname}|\\
|\||else|\\
\textit{document body with }|\input{|\textit{part}|}|\\
|\||fi|
\end{tabular}
\end{center}
%
The conditional |\ifchilddocmanual| is true whenever
a part to be included by |\input| is being compiled,
and the name of the part is stored in |\childdocname|.

%%%%%%%%%%%%%%%%%%%%%%%%%%%%%%%%%%%%%%%%
\DescribeMacro{\childdocby}
Each part to be included by |\input| should start with:
%
\begin{center}
\begin{tabular}{l}
|\input{childdoc.def}|\\
|\childdocby{|\textit{main}|}|\\
\end{tabular}
\end{center}
%
The directive |\childdocby| is similar to |\childdocof|
described in \secref{sec:include},
but the subsequent selection of content must be done manually.
To that end, both |\ifchilddoc| and |\ifchilddocmanual|
will be true upon processing of a part,
and the name of the part is stored in |\childdocname|.
Note that |\jobname| will be set to the filename of the current part
so that each part receives an individual |.aux| file
that does not interfere with the |.aux| file(s) of the main document.
This behaviour can be altered by the alternative form
|\childdocby[*]{|\textit{main}|}| (with a non-empty optional argument)
which uses the |.aux| file of the main document
by setting |\jobname| to \textit{main}.

%%%%%%%%%%%%%%%%%%%%%%%%%%%%%%%%%%%%%%%%%%%%%%%%%%%%%%%%%%%%%%%%%%%%%%%%%%%%%%%%
\subsection{Driver Development}
\label{sec:driver}

The \textsf{childdoc} mechanism can also be use for the development
of definition files such as \LaTeX{} styles or classes.
This case differs from the above setup with multiple parts
included by |\include| in that no |\includeonly| should be invoked.
This can be achieved by starting the include file
(before |\ProvidesPackage|) with:
%
\begin{center}
\begin{tabular}{l}
|\input{childdoc.def}|\\
|\childdocforward{|\textit{main}|}|\\
\end{tabular}
\end{center}
%
or alternatively with:
%
\begin{center}
\begin{tabular}{l}
|\input{childdoc.def}|\\
|\childdocby{|\textit{main}|}|\\
\end{tabular}
\end{center}
%
Both forms have slightly different effects as described above.
The main file is prepared as usual, see \secref{sec:include}.

%%%%%%%%%%%%%%%%%%%%%%%%%%%%%%%%%%%%%%%%%%%%%%%%%%%%%%%%%%%%%%%%%%%%%%%%%%%%%%%%
\subsection{Legacy Detection}
\label{sec:detection}

The directive |\childdocmain| in the main file can detect
whether the complete document or merely a child is to be compiled
even without using the directive |\childdocof|.
This method is deprecated because it is less robust
and there is no compelling reason to use it;
it is merely provided for backward compatibility
and it may be removed in future versions.

If the detection mechanism is to be used,
it is mandatory to correctly specify
the filename of the main file as the argument of |\childdocmain|:
%
\begin{center}
\begin{tabular}{l}
|\input{childdoc.def}|\\
|\childdocmain{|\textit{main}|}|\\
\end{tabular}
\end{center}
%
If |\jobname| does not match the argument \textit{main} of |\childdocmain|,
it is assumed that |\jobname| points to the child file to be compiled.
When using |\childdocmain| with the main file specified as argument,
it suffices to start a child file
with just |\input{|\textit{main}|}|
without loading of the package and using |\childdocof|.
If instead all processing is done
with the appropriate \textsf{childdoc} directives,
the argument of \textit{main} of |\childdocmain| can be empty.

An alternative version of the command line processing described
in \secref{sec:commandline} using the detection mechanism reads:
%
\begin{center}
|... -jobname "|\textit{target}|" "|[\textit{flags}]%
[|\def\jobname{|\textit{dest}|}|]|\input{|\textit{main}|}"|
\end{center}

%%%%%%%%%%%%%%%%%%%%%%%%%%%%%%%%%%%%%%%%%%%%%%%%%%%%%%%%%%%%%%%%%%%%%%%%%%%%%%%%
\subsection{Manual Code}
\label{sec:manual}

In case one cannot be certain whether the definitions file |childdoc.def|
is installed on the target \TeX{} distribution
and one prefers not to ship it,
it is conceivable to paste a few relevant commands into the sources.

To that end, drop all statements |\input{childdoc.def}|
and perform the replacements as outlined below.
Instead of |\childdocmain{|\textit{main}|}| add the following code
to the top of the main file:
%
\begin{center}
\begin{tabular}{l}
|\||ifdefined\childdocname\endinput\||fi\newif\ifchilddoc|\\
|\edef\childdocname{\scantokens\expandafter{\jobname\noexpand}}|\\
|\def\childdocmain{|\textit{main}|}\||ifx\childdocmain\childdocname\||else|\\
|\childdoctrue\includeonly{\childdocname}\let\jobname\childdocmain\||fi|\\
\end{tabular}
\end{center}
%
Instead of |\childdocof{|\textit{main}|}| just include the main file
at the top of each child file:
%
\begin{center}
|\input{|\textit{main}|}|
\end{center}
%
A simple redirection |\childdocforward{|\textit{dest}|}| is achieved by:
%
\begin{center}
|\def\jobname{|\textit{dest}|}\input{\jobname}|
\end{center}
%
The redirection with prefix
|\childdocforwardprefix[|\textit{prefix}|]{|\textit{dest}|}|
is accomplished by:
%
\begin{center}
\begin{tabular}{l}
|{\edef\jobname{\scantokens\expandafter{\jobname\noexpand}}|\\
|\def\redirectjob |\textit{prefix}|#1~~~{\gdef\jobname{|\textit{dest}|#1}}|\\
|\expandafter\redirectjob\jobname~~~}\input{\jobname}|
\end{tabular}
\end{center}

In an alternative approach,
child documents can be compiled by a specific command line
without additional code or specific definitions:
%
\begin{center}
|... -jobname "|\textit{target}|" "|[\textit{flags}]%
|\includeonly{|\textit{dest}|}\input{|\textit{main}|}"|
\end{center}
%

%%%%%%%%%%%%%%%%%%%%%%%%%%%%%%%%%%%%%%%%%%%%%%%%%%%%%%%%%%%%%%%%%%%%%%%%%%%%%%%%
%%%%%%%%%%%%%%%%%%%%%%%%%%%%%%%%%%%%%%%%%%%%%%%%%%%%%%%%%%%%%%%%%%%%%%%%%%%%%%%%
\section{Information}

%%%%%%%%%%%%%%%%%%%%%%%%%%%%%%%%%%%%%%%%%%%%%%%%%%%%%%%%%%%%%%%%%%%%%%%%%%%%%%%%
\subsection{Copyright}

Copyright \copyright{} 2017--2018 Niklas Beisert

This work may be distributed and/or modified under the
conditions of the \LaTeX{} Project Public License, either version 1.3
of this license or (at your option) any later version.
The latest version of this license is in
  \url{http://www.latex-project.org/lppl.txt}
and version 1.3 or later is part of all distributions of \LaTeX{}
version 2005/12/01 or later.

This work has the LPPL maintenance status `maintained'.

The Current Maintainer of this work is Niklas Beisert.

This work consists of the files |README.txt|, |childdoc.ins| and |childdoc.dtx|
as well as the derived files |childdoc.def|, |cdocsamp.tex|
with |cdocsch1.tex|, |cdocsch2.tex|, |cdocspt3.tex|, |cdocspt4.tex|,
|cdocsdrf.tex|, |cdocsfn1.tex|, |cdocsfn2.tex|
as well as |childdoc.pdf|.

%%%%%%%%%%%%%%%%%%%%%%%%%%%%%%%%%%%%%%%%%%%%%%%%%%%%%%%%%%%%%%%%%%%%%%%%%%%%%%%%
\subsection{Files and Installation}

The package consists of the files:
%
\begin{center}
\begin{tabular}{ll}
    |README.txt|   & readme file \\
    |childdoc.ins| & installation file \\
    |childdoc.dtx| & source file \\
    |childdoc.def| & definition file \\
    |cdocsamp.tex| & sample main file \\
    |cdocsch1.tex| & sample include file \\
    |cdocsch2.tex| & sample include file \\
    |cdocspt3.tex| & sample part file \\
    |cdocspt4.tex| & sample part file \\
    |cdocsdrf.tex| & sample redirection file \\
    |cdocsfn1.tex| & sample redirection file \\
    |cdocsfn2.tex| & sample redirection file \\
    |childdoc.pdf| & manual
\end{tabular}
\end{center}
%
The distribution consists of the files
|README.txt|, |childdoc.ins| and |childdoc.dtx|.
%
\begin{itemize}
\item
Run (pdf)\LaTeX{} on |childdoc.dtx|
to compile the manual |childdoc.pdf| (this file).
\item
Run \LaTeX{} on |childdoc.ins| to create the definitions file |childdoc.def|
and the sample |cdocsamp.tex| with include files
|cdocsch1.tex|, |cdocsch2.tex|, |cdocspt3.tex|, |cdocspt4.tex|,
|cdocsdrf.tex|, |cdocsfn1.tex|, |cdocsfn2.tex|.
Then copy the file |childdoc.def| to an appropriate directory of your \LaTeX{}
distribution, e.g.\ \textit{texmf-root}|/tex/latex/childdoc|.
\end{itemize}

%%%%%%%%%%%%%%%%%%%%%%%%%%%%%%%%%%%%%%%%%%%%%%%%%%%%%%%%%%%%%%%%%%%%%%%%%%%%%%%%
\subsection{Related CTAN Packages}

There are several other packages which offer a similar functionality:
%
\begin{itemize}
\item
The packages
\href{http://ctan.org/pkg/docmute}{\textsf{docmute}},
\href{http://ctan.org/pkg/includex}{\textsf{includex}} and
\href{http://ctan.org/pkg/standalone}{\textsf{standalone}}
provide commands to include only the document body of
a child file thus allowing both files to be compiled individually.
\item
The packages \href{http://ctan.org/pkg/subdocs}{\textsf{subdocs}}
and \href{http://ctan.org/pkg/subfiles}{\textsf{subfiles}}
provide structures in which the main and child documents can be
encapsulated and allowing them to be compiled individually.
The inclusion mechanism is different from the conventional |\include|.
\item
The package \href{http://ctan.org/pkg/combine}{\textsf{combine}}
is an elaborate solution to combine several documents into one.
\end{itemize}
%
See also the CTAN topic \href{http://ctan.org/topic/subdocs}{\textsf{subdocs}}
for further related packages.
The present package differs from the above solutions in that
a document structure constructed with the conventional |\include| mechanism
just needs two extra commands at the top of every file
such that all constituent files can be compiled individually.

%%%%%%%%%%%%%%%%%%%%%%%%%%%%%%%%%%%%%%%%%%%%%%%%%%%%%%%%%%%%%%%%%%%%%%%%%%%%%%%%
%\subsection{Feature Suggestions}
%
%The following is a list of features which may be useful for future
%versions of this package:
%%
%\begin{itemize}
%\item
%\ldots
%\end{itemize}

%%%%%%%%%%%%%%%%%%%%%%%%%%%%%%%%%%%%%%%%%%%%%%%%%%%%%%%%%%%%%%%%%%%%%%%%%%%%%%%%
\subsection{Revision History}

%%%%%%%%%%%%%%%%%%%%%%%%%%%%%%%%%%%%%%%%
\paragraph{v2.0:} 2018/12/30

\begin{itemize}
\item
immediate forward processing
\item
added |\childdocby| mechanism
\item
manual restructured
\end{itemize}

%%%%%%%%%%%%%%%%%%%%%%%%%%%%%%%%%%%%%%%%
\paragraph{v1.6:} 2018/01/17

\begin{itemize}
\item
application for development of include files
\item
corrections to manual
\end{itemize}

%%%%%%%%%%%%%%%%%%%%%%%%%%%%%%%%%%%%%%%%
\paragraph{v1.5:} 2017/05/21

\begin{itemize}
\item
more complete structuring introduced
\item
|\childdocof| introduced
\item
|\childdoc| renamed to |\childdocmain|
\item
|\childredirect| renamed to |\childdocforward| and |\childdocforwardprefix|
and functionality expanded
\end{itemize}

%%%%%%%%%%%%%%%%%%%%%%%%%%%%%%%%%%%%%%%%
\paragraph{v1.0:} 2017/04/27

\begin{itemize}
\item
manual and install package
\item
first version published on CTAN
\end{itemize}

%%%%%%%%%%%%%%%%%%%%%%%%%%%%%%%%%%%%%%%%
\paragraph{v0.6:} 2017/04/26

\begin{itemize}
\item
redirection mechanism added
\end{itemize}

%%%%%%%%%%%%%%%%%%%%%%%%%%%%%%%%%%%%%%%%
\paragraph{v0.5:} 2017/04/26

\begin{itemize}
\item
functionality in definition file
\end{itemize}


%%%%%%%%%%%%%%%%%%%%%%%%%%%%%%%%%%%%%%%%%%%%%%%%%%%%%%%%%%%%%%%%%%%%%%%%%%%%%%%%
%%%%%%%%%%%%%%%%%%%%%%%%%%%%%%%%%%%%%%%%%%%%%%%%%%%%%%%%%%%%%%%%%%%%%%%%%%%%%%%%
%%%%%%%%%%%%%%%%%%%%%%%%%%%%%%%%%%%%%%%%%%%%%%%%%%%%%%%%%%%%%%%%%%%%%%%%%%%%%%%%
\appendix

\settowidth\MacroIndent{\rmfamily\scriptsize 000\ }

 \DocInput{childdoc.dtx}

\end{document}
%</driver>
% \fi
%
% %%%%%%%%%%%%%%%%%%%%%%%%%%%%%%%%%%%%%%%%%%%%%%%%%%%%%%%%%%%%%%%%%%%%%%%%%%%%%%
% %%%%%%%%%%%%%%%%%%%%%%%%%%%%%%%%%%%%%%%%%%%%%%%%%%%%%%%%%%%%%%%%%%%%%%%%%%%%%%
% \section{Sample}
%\iffalse
%<*samplemain>
%\fi
%
% The following presents a sample document
% with two chapters, two parts, a title page,
% a compile flag as well as three forwarding files to set the flag.
% It consists of eight |.tex| files:
% \begin{center}
% \begin{tabular}{ll}
% |cdocsamp.tex|&main file\\
% |cdocsch1.tex|&include file for chapter 1\\
% |cdocsch2.tex|&include file for chapter 2\\
% |cdocspt3.tex|&include file for part 3\\
% |cdocspt4.tex|&include file for part 4\\
% |cdocsdrf.tex|&forwarding file for main file in draft mode\\
% |cdocsfi1.tex|&forwarding file for final version of chapter 1\\
% |cdocsfi2.tex|&forwarding file for final version of chapter 2\\
% \end{tabular}
% \end{center}
% Each of the eight files can be compiled directly by the \LaTeX{} compiler.
%
% %%%%%%%%%%%%%%%%%%%%%%%%%%%%%%%%%%%%%%
% \paragraph{Main File.}
%
% The main file is called |cdocsamp.tex|.
%
% Load the \textsf{childdoc} definitions and
% declare the filename for the main document:
%    \begin{macrocode}
\input{childdoc.def}
\childdocmain{}
%    \end{macrocode}

% Optional override for |\version| flag:
%    \begin{macrocode}
%%\ifchilddoc\else\providecommand{\version}{draft}\fi
%    \end{macrocode}

% Define the default values for the |\version| flag
% (|final| for the main file and |draft| for childs):
%    \begin{macrocode}
\ifchilddoc
\providecommand{\version}{draft}
\else
\providecommand{\version}{final}
\fi
%    \end{macrocode}

% Load the standard document class:
%    \begin{macrocode}
\documentclass[12pt]{article}
%    \end{macrocode}

% Start the document body:
%    \begin{macrocode}
\begin{document}
%    \end{macrocode}

% Declare a title page.
% Print title, part of document being processed and version flag:
%    \begin{macrocode}
\addtocounter{page}{-1}
\begin{center}
{\LARGE\bfseries{}childdoc example\par}
\vspace{1cm}
\ifchilddoc
\ifchilddocmanual part\else chapter\fi:
`\childdocname' of `\childdocjob'\par
\else
main document: `\childdocjob'\par
\fi
version: \version\par
\end{center}
\newpage
%    \end{macrocode}

% Manually include selected file,
% otherwise process as usual:
%    \begin{macrocode}
\ifchilddocmanual
\section*{part `\childdocname'}
\input{\childdocname}
\else
%    \end{macrocode}

% Include the two chapters:
%    \begin{macrocode}
\include{cdocsch1}
\include{cdocsch2}
%    \end{macrocode}

% Include the two parts unless only chapters should be displayed:
%    \begin{macrocode}
\ifchilddoc\else
\section{part three}
\input{cdocspt3}
\section{part four}
\input{cdocspt4}
\fi
%    \end{macrocode}

% Process as usual until here:
%    \begin{macrocode}
\fi
%    \end{macrocode}

% End of document body:
%    \begin{macrocode}
\end{document}
%    \end{macrocode}
%\iffalse
%</samplemain>
%\fi
%
% %%%%%%%%%%%%%%%%%%%%%%%%%%%%%%%%%%%%%%
% \paragraph{Chapter Include Files.}
%
% The include files are called |cdocsch1.tex| and |cdocsch2.tex|.
%
%\iffalse
%<*samplechap1|samplechap2>
%\fi

% Optional override for |\version| flag:
%    \begin{macrocode}
%%\providecommand{\version}{final}
%    \end{macrocode}

% Include the main document:
%    \begin{macrocode}
\input{childdoc.def}
\childdocof{cdocsamp}
%    \end{macrocode}

%\iffalse
%</samplechap1|samplechap2>
%\fi
%
%\iffalse
%<*samplechap1>
%\fi
% Some text for chapter 1:
%    \begin{macrocode}
\section{one}
some text in chapter one
%    \end{macrocode}

%\iffalse
%</samplechap1>
%\fi
% Some text for chapter 2:
%\iffalse
%<*samplechap2>
%\fi
%    \begin{macrocode}
\section{two}
more text in chapter two
%    \end{macrocode}

%\iffalse
%</samplechap2>
%\fi
%
% %%%%%%%%%%%%%%%%%%%%%%%%%%%%%%%%%%%%%%
% \paragraph{Part Include Files.}
%
% The include files are called |cdocspt3.tex| and |cdocspt4.tex|.
%
%\iffalse
%<*samplepart3|samplepart4>
%\fi

% Optional override for |\version| flag:
%    \begin{macrocode}
%%\providecommand{\version}{final}
%    \end{macrocode}

% Include the main document:
%    \begin{macrocode}
\input{childdoc.def}
\childdocby{cdocsamp}
%    \end{macrocode}

%\iffalse
%</samplepart3|samplepart4>
%\fi
%
%\iffalse
%<*samplepart3>
%\fi
% Some text for part 3:
%    \begin{macrocode}
some text in part three
%    \end{macrocode}

%\iffalse
%</samplepart3>
%\fi
% Some text for part 4:
%\iffalse
%<*samplepart4>
%\fi
%    \begin{macrocode}
more text in part four
%    \end{macrocode}

%\iffalse
%</samplepart4>
%\fi
%
% %%%%%%%%%%%%%%%%%%%%%%%%%%%%%%%%%%%%%%
% \paragraph{Forwarding for a Complete Draft.}
%
% The following forwarding file |cdocsdrf.tex|
% compiles the main document in draft mode:
%\iffalse
%<*sampledraft>
%\fi
%    \begin{macrocode}
\def\version{draft}
\input{childdoc.def}
\childdocforward{cdocsamp}
%    \end{macrocode}

%\iffalse
%</sampledraft>
%\fi
%
% %%%%%%%%%%%%%%%%%%%%%%%%%%%%%%%%%%%%%%
% \paragraph{Forwarding for Final Version of the Chapters.}
%
% The following forwarding files |cdocsfn1.tex| and |cdocsfn2.tex|
% (with identical content)
% compile the final versions of the child documents
% |cdocsch1.tex| and |cdocsch2.tex|, respectively:
%\iffalse
%<*samplefinal>
%\fi
%    \begin{macrocode}
\def\version{final}
\input{childdoc.def}
\childdocforwardprefix[cdocsamp]{cdocsfn}{cdocsch}
%    \end{macrocode}

%\iffalse
%</samplefinal>
%\fi
%
% %%%%%%%%%%%%%%%%%%%%%%%%%%%%%%%%%%%%%%
% \paragraph{Command Line Processing.}
%
% The following three command lines generate the output files
% |cdocscld|, |cdocscl1| and |cdocscl2|
% which should be identical to
% |cdocsdrf|, |cdocsch1| and |cdocsfn2|, respectively:
% \begin{center}
% \begin{tabular}{l}
% |latex -jobname cdocscld \|\\
% |  "\def\version{draft}\input{childdoc.def}\childdocforward{cdocsamp}"|\\
% |latex -jobname cdocscl1 \|\\
% |  "\input{childdoc.def}\childdocforward[cdocsamp]{cdocsch1}"|\\
% |latex -jobname cdocscl2 \|\\
% |  "\def\version{final}\input{childdoc.def}\childdocforward{cdocsch2}"|
% \end{tabular}
% \end{center}
% Note that the trailing backslash on each first line
% merely continues the input to the second line
% (for convenient cut ant paste).
% Furthermore, the command |latex| can be replaced by any
% of its alternative versions such as |pdflatex|.
%
% %%%%%%%%%%%%%%%%%%%%%%%%%%%%%%%%%%%%%%%%%%%%%%%%%%%%%%%%%%%%%%%%%%%%%%%%%%%%%%
% %%%%%%%%%%%%%%%%%%%%%%%%%%%%%%%%%%%%%%%%%%%%%%%%%%%%%%%%%%%%%%%%%%%%%%%%%%%%%%
% \section{Implementation}
%\iffalse
%<*package>
%\fi
%
% This section describes the definitions file |childdoc.def|.

% The definitions cannot be loaded using |\usepackage| or |\RequirePackage|
% which has a mechanism to prevent loading a style file more than once.
% When loading the definitions by means of |\input|
% multiple instances have to be prevented manually:
%\iffalse
%This code needs to be before the `\ProvidesFile' directive
%which is defined at the beginning of this file.
%Therefore it is also placed there and commented out here.
%</package>
%<*discard>
%\fi
%    \begin{macrocode}
\ifdefined\childdocmain\endinput\fi
%    \end{macrocode}
%\iffalse
%</discard>
%<*package>
%\fi
%
% \macro{\ifchilddoc}
% \macro{\ifchilddocmanual}
% The conditional |\ifchilddoc| tells whether a
% child (true) or main (false) document is being compiled.
% The conditional |\ifchilddocmanual| tells whether
% the |\includeonly| mechanism is used (false) or
% the selection of child files must be performed manually (true).
% The definitions initialise to false:
%    \begin{macrocode}
\newif\ifchilddoc
\newif\ifchilddocmanual
%    \end{macrocode}

% \macro{\childdocname}
% \macro{\childdocjob}
% The macro |\childdocname| stores the name of the main document
% to be compiled. The macro |\childdocjob| stores the name of
% the document on which the \LaTeX{} compiler was originally invoked.
% The content of |\jobname| cannot be compared
% to filenames specified in the source due to different catcodes.
% The following code rescans |\jobname|, stores the result
% in |\childdocname| and saves a copy in |\childdocjob|:
%    \begin{macrocode}
\edef\childdocname{\scantokens\expandafter{\jobname\noexpand}}
\let\childdocjob\childdocname
%    \end{macrocode}

% \macro{\childdocdisable}
% The macro |\childdocdisable| prevents the main file
% from being processed more than once.
% At this stage, the main document command |\childdocmain|
% is assumed to be called once again where it should do nothing.
% Any subsequent call to it should prevent
% a secondary processing of the main document
% It overwrites the forwarding commands
% |\childdocof| and |\childdocforward|
% with empty macros to prevent further inclusions of the main document:
%    \begin{macrocode}
\newcommand{\childdocdisable}
{
  \renewcommand{\childdocmain}[1]{\renewcommand{\childdocmain}[1]{\endinput}}
  \renewcommand{\childdocof}[1]{}
  \renewcommand{\childdocby}[2][]{}
  \renewcommand{\childdocforward}[2][]{}
  \renewcommand{\childdocdisable}{}
}
%    \end{macrocode}

% \macro{\childdocmain}
% The macro |\childdocmain| is to be called at the top of the main file
% with nothing or the main filename (without extension) as argument.
% First, it breaks loops.
% If the argument is not empty and does not match |\childdocname|
% (which is set by the first inclusion of |childdoc.def|),
% |\ifchilddoc| is set to true, |\includeonly| is applied to the child file
% and |\jobname| is set to the main file
% (for proper handling of |.aux| files):
%    \begin{macrocode}
\newcommand{\childdocmain}[1]
{
  \childdocdisable\childdocmain{}
  \if?#1?\else
    \begingroup
      \def\childdoctmp{#1}
      \ifx\childdoctmp\childdocname
        \def\childdoctmp{}
      \else
        \def\childdoctmp
        {
          \childdoctrue
          \includeonly{\childdocname}
          \def\childdocjob{#1}
          \def\jobname{#1}
        }
      \fi
      \expandafter
    \endgroup
    \childdoctmp
  \fi
}
%    \end{macrocode}

% \macro{\childdocof}
% The command |\childdocof| redirects
% compilation to the main file |#1|.
%    \begin{macrocode}
\newcommand{\childdocof}[1]
{
  \childdocdisable
  \childdoctrue
  \includeonly{\childdocname}
  \def\jobname{#1}
  \def\childdocjob{#1}
  \input{#1}
}
%    \end{macrocode}

% \macro{\childdocby}
% The command |\childdocby| ....
%    \begin{macrocode}
\newcommand{\childdocby}[2][]
{
  \childdocdisable
  \childdoctrue
  \childdocmanualtrue
  \if?#1?\else
    \def\jobname{#2}
  \fi
  \def\childdocjob{#2}
  \input{#2}
  \endinput
}
%    \end{macrocode}

% \macro{\childdocforward}
% The command |\childdocforward| redirects
% compilation to the main file or
% (if the optional argument is given) a child file.
% Parameters are set as if the main file
% or a child file starting with |\childdocof| was compiled.
% Then compilation is handed over to the main file:
%    \begin{macrocode}
\newcommand{\childdocforward}[2][]
{
  \begingroup
    \if?#1?
      \def\childdoctmp
      {
        \def\childdocname{#2}
        \def\childdocjob{#2}
        \def\jobname{#2}
        \input{#2}
        \endinput
      }
    \else
      \def\childdoctmp
      {
        \childdocdisable
        \def\childdocname{#2}
        \childdoctrue
        \includeonly{#2}
        \def\childdocjob{#1}
        \def\jobname{#1}
        \input{#1}
        \endinput
      }
    \fi
    \expandafter
  \endgroup
  \childdoctmp
}
%    \end{macrocode}

% \macro{\childdocforwardprefix}
% The command |\childdocforwardprefix| redirects
% compilation to the main or a child file by means of a pattern.
% The prefix |#1| in the current filename is replaced by |#2|
% and the suffix of the current filename is kept
% (it is assumed that the filename does not contain the substring `|~~~|'
% which is used as a delimiter).
% Compilation is handed over to the new file by |\childdocforward|:
%    \begin{macrocode}
\newcommand{\childdocforwardprefix}[3][]
{
  \begingroup
    \def\childdocextract #2##1~~~{\def\childdoctmp{\childdocforward[#1]{#3##1}}}
    \expandafter\childdocextract\childdocname~~~
    \expandafter
  \endgroup
  \childdoctmp
}
%    \end{macrocode}

% \macro{\childdoc}
% The deprecated macro |\childdoc| is a legacy version of |\childdocmain|:
%    \begin{macrocode}
\newcommand{\childdoc}{\childdocmain}
%    \end{macrocode}

% \macro{\childdocredirect}
% The deprecated macro |\childdocredirect| is a legacy version
% of |\childdocforward| and |\childdocforwardprefix|:
%    \begin{macrocode}
\newcommand{\childdocredirect}[2][]
{
  \begingroup
    \if?#1?
      \def\childdoctmp{\childdocforward{#2}}
    \else
      \def\childdoctmp{\childdocforwardprefix{#1}{#2}}
    \fi
    \expandafter
  \endgroup
  \childdoctmp
}
%    \end{macrocode}

%\iffalse
%</package>
%\fi
%
\endinput
|\\
|\childdocmain{|\textit{main}|}|\\
\end{tabular}
\end{center}
%
If |\jobname| does not match the argument \textit{main} of |\childdocmain|,
it is assumed that |\jobname| points to the child file to be compiled.
When using |\childdocmain| with the main file specified as argument,
it suffices to start a child file
with just |\input{|\textit{main}|}|
without loading of the package and using |\childdocof|.
If instead all processing is done
with the appropriate \textsf{childdoc} directives,
the argument of \textit{main} of |\childdocmain| can be empty.

An alternative version of the command line processing described
in \secref{sec:commandline} using the detection mechanism reads:
%
\begin{center}
|... -jobname "|\textit{target}|" "|[\textit{flags}]%
[|\def\jobname{|\textit{dest}|}|]|\input{|\textit{main}|}"|
\end{center}

%%%%%%%%%%%%%%%%%%%%%%%%%%%%%%%%%%%%%%%%%%%%%%%%%%%%%%%%%%%%%%%%%%%%%%%%%%%%%%%%
\subsection{Manual Code}
\label{sec:manual}

In case one cannot be certain whether the definitions file |childdoc.def|
is installed on the target \TeX{} distribution
and one prefers not to ship it,
it is conceivable to paste a few relevant commands into the sources.

To that end, drop all statements |% \iffalse
%
% childdoc.dtx Copyright (C) 2017-2018 Niklas Beisert
%
% This work may be distributed and/or modified under the
% conditions of the LaTeX Project Public License, either version 1.3
% of this license or (at your option) any later version.
% The latest version of this license is in
%   http://www.latex-project.org/lppl.txt
% and version 1.3 or later is part of all distributions of LaTeX
% version 2005/12/01 or later.
%
% This work has the LPPL maintenance status `maintained'.
%
% The Current Maintainer of this work is Niklas Beisert.
%
% This work consists of the files childdoc.dtx and childdoc.ins
% and the derived files childdoc.def and cdocsamp.tex with
% cdocsch1.tex, cdocsch2.tex, cdocsdrf.tex, cdocsfn1.tex, cdocsfn2.tex.
%
%<package>\ifdefined\childdocmain\endinput\fi
%<package>\ProvidesFile{childdoc.def}[2018/12/30 v2.0 child document driver]
%<samplemain>\ProvidesFile{cdocsamp.tex}[2018/12/30 v2.0 sample for childdoc]
%<*driver>
%\ProvidesFile{childdoc.drv}[2018/12/30 v2.0 childdoc reference manual file]
\PassOptionsToClass{10pt,a4paper}{article}
\documentclass{ltxdoc}

\usepackage[margin=35mm]{geometry}
\usepackage{hyperref}
\usepackage{hyperxmp}
\usepackage[usenames]{color}

\hypersetup{colorlinks=true}
\hypersetup{pdfstartview=FitH}
\hypersetup{pdfpagemode=UseNone}
\hypersetup{pdfsource={}}
\hypersetup{pdflang={en-UK}}
\hypersetup{pdfcopyright={Copyright 2017-2018 Niklas Beisert.
  This work may be distributed and/or modified under the
  conditions of the LaTeX Project Public License, either version 1.3
  of this license or (at your option) any later version.}}
\hypersetup{pdflicenseurl={http://www.latex-project.org/lppl.txt}}
\hypersetup{pdfcontactaddress={ETH Zurich, ITP, HIT K,
  Wolfgang-Pauli-Strasse 27}}
\hypersetup{pdfcontactpostcode={8093}}
\hypersetup{pdfcontactcity={Zurich}}
\hypersetup{pdfcontactcountry={Switzerland}}
\hypersetup{pdfcontactemail={nbeisert@itp.phys.ethz.ch}}
\hypersetup{pdfcontacturl={http://people.phys.ethz.ch/\xmptilde nbeisert/}}

\newcommand{\secref}[1]{\hyperref[#1]{section \ref*{#1}}}

\parskip1ex
\parindent0pt
\let\olditemize\itemize
\def\itemize{\olditemize\parskip0pt}

\begin{document}

\title{The \textsf{childdoc} Package}
\hypersetup{pdftitle={The childdoc Package}}
\author{Niklas Beisert\\[2ex]
  Institut f\"ur Theoretische Physik\\
  Eidgen\"ossische Technische Hochschule Z\"urich\\
  Wolfgang-Pauli-Strasse 27, 8093 Z\"urich, Switzerland\\[1ex]
  \href{mailto:nbeisert@itp.phys.ethz.ch}
  {\texttt{nbeisert@itp.phys.ethz.ch}}}
\hypersetup{pdfauthor={Niklas Beisert}}
\hypersetup{pdfsubject={Manual for the LaTeX2e Package childdoc}}
\date{30 December 2018, \textsf{v2.0}}
\maketitle

\begin{abstract}\noindent
\textsf{childdoc} is a \LaTeXe{} package
that enables the direct compilation
of document sections included by |\include|
to individual files.
\end{abstract}

\begingroup
\parskip0ex
\tableofcontents
\endgroup

%%%%%%%%%%%%%%%%%%%%%%%%%%%%%%%%%%%%%%%%%%%%%%%%%%%%%%%%%%%%%%%%%%%%%%%%%%%%%%%%
%%%%%%%%%%%%%%%%%%%%%%%%%%%%%%%%%%%%%%%%%%%%%%%%%%%%%%%%%%%%%%%%%%%%%%%%%%%%%%%%
\section{Introduction}

\LaTeX{} provides a mechanism to structure a large document (such as a book)
into a main file and several child files (containing the chapters)
using the |\include| command.
This mechanism is beneficial for documents
which span hundreds of pages in order to
make the source file(s) more manageable.
Moreover, compilation can be restricted to
selected child files by means of the |\includeonly| command.
The latter feature can be used to reduce the compilation time while editing
(this was significantly more useful in the earlier days of \LaTeX{})
or to generate a smaller document which is easier to navigate.
Another application of |\includeonly| is to generate
documents consisting of selected parts of the complete document.

However, there are a few drawbacks of the plain |\include| mechanism:
\begin{itemize}
\item
The child files cannot be compiled on their own,
they can only be compiled via the main file.
A naive editing environment
(such as a text editor with an option
to have the current file processed by \LaTeX)
may require one to switch to the main file before compiling;
attempting to compile the child file produces errors.
\item
The main file must be modified (each time)
to adjust the |\includeonly| command
to the present needs. This easily leaves the main file in a messy state.
\item
The generated document will always carry the filename
of the main document. This is inconvenient if
several child files are to be compiled and
to be kept for distribution.
\end{itemize}

The present package provides a simple interface
to make child files individually compilable by \LaTeX{}.
Compiling a child file then has the same effect as compiling
the main file with an |\includeonly| command
to select the appropriate child.
Moreover the generated document will carry the name of the child
rather than the main file.
This resolves all three above issues.

This feature is meant to make the editing of books,
thesis documents and lecture notes somewhat more convenient.
However, the package can also be used efficiently for
composing a series of documents (such as exercise sheets)
which are typically distributed individually.
It then assists the author in generating the individual documents
(potentially in different versions)
as well as a document containing the collected series.
Another application is in developing style files
or other kinds of included material
where compilation of the style file could redirect
to a sample or test file.

%%%%%%%%%%%%%%%%%%%%%%%%%%%%%%%%%%%%%%%%%%%%%%%%%%%%%%%%%%%%%%%%%%%%%%%%%%%%%%%%
%%%%%%%%%%%%%%%%%%%%%%%%%%%%%%%%%%%%%%%%%%%%%%%%%%%%%%%%%%%%%%%%%%%%%%%%%%%%%%%%
\section{Usage}

First of all, the package \textsf{childdoc} is \emph{not} a standard
\LaTeXe{} |.sty| style file! Therefore it needs to be invoked in
a non-standard way.

%%%%%%%%%%%%%%%%%%%%%%%%%%%%%%%%%%%%%%%%%%%%%%%%%%%%%%%%%%%%%%%%%%%%%%%%%%%%%%%%
\subsection{Included Files}
\label{sec:include}

%%%%%%%%%%%%%%%%%%%%%%%%%%%%%%%%%%%%%%%%
\DescribeMacro{\childdocmain}
To use the package, add the commands
\begin{center}
\begin{tabular}{l}
|\input{childdoc.def}|\\
|\childdocmain{}|\\
\end{tabular}
\end{center}
at the very top of the main \LaTeX{} file,
in particular \emph{before} the |\documentclass| statement!
The argument of |\childdocmain| should be left empty
(but it must be present).

%%%%%%%%%%%%%%%%%%%%%%%%%%%%%%%%%%%%%%%%
\DescribeMacro{\childdocof}
Furthermore, add the commands
\begin{center}
\begin{tabular}{l}
|\input{childdoc.def}|\\
|\childdocof{|\textit{main}|}|\\
\end{tabular}
\end{center}
at the top of every child file \textit{child}
which is included by |\include{|\textit{child}|}|
from within the main file
(or at least for those files to be compiled individually).
The argument \textit{main} must be the filename of the main file.

There are a couple of
considerations in setting up the main and child documents:

%%%%%%%%%%%%%%%%%%%%%%%%%%%%%%%%%%%%%%%%
\paragraph{Restrictions.}

Please note the following restrictions:
\begin{itemize}
\item
|\childdocmain| must be called with one argument \textit{main}
to ensure compatibility with earlier version of the package.
It must either be empty (|\childdocmain{}|)
or precisely match the filename of the main file in which it is specified.
See \secref{sec:detection} for further information.
\item
The filename \textit{main} must be specified without the |.tex| extension.
\item
The filename \textit{main} is case sensitive
(even in case-insensitive file systems)
due to internal string comparison.
\item
The argument \textit{main} should be fully expanded, it cannot be a macro.
\item
Subdirectories and special characters should be avoided in filenames.
\item
The command |\childdocmain{|\textit{main}|}| must be followed by a whitespace.
It should not be followed immediately by another command
or by a comment mark `|%|'.
This is because the \TeX{} parser reads the token immediately following
the argument of |\childdocmain| and puts it
at the beginning of every child section;
however, a white\-space is ignored.
\end{itemize}

%%%%%%%%%%%%%%%%%%%%%%%%%%%%%%%%%%%%%%%%
\paragraph{Content of Main File.}

It is advisable to place all content in the child files included by |\include|.
Any output contained in the main file will appear in all child documents
unless suppressed manually;
it cannot be suppressed automatically by the |\includeonly| directive
and thus should normally be avoided.
A method to include some content in the main file
by means of conditional processing is described in \secref{sec:conditional}.

%%%%%%%%%%%%%%%%%%%%%%%%%%%%%%%%%%%%%%%%
\paragraph{Page Numbering.}

When only a part of the document is compiled,
the appropriate numbering of pages
(as well as other status parameters)
is determined from the |.aux| files.
The latter contain information from previous passes.
However this information needs to propagate through
all intermediate child documents.
Therefore the page numbering in child documents may well
be inconsistent until the complete document is compiled at least once.

A useful (if unconventional) way to always ensure a consistent
page numbering is to restart the numbering in each child document
and denote the pages by `\textit{child}|.|\textit{page}'
where \textit{child} represents the chapter/section number of the child file.
This can be achieved by the command
|\numberwithin{page}{|\textit{child}|}|
of the \textsf{amsmath} package
where \textit{child} can be |chapter| or |section|
depending on the chosen structuring.
Alternatively, one can modify the macro |\thepage| appropriately
and reset the counter |page| at the start of each child file.

%%%%%%%%%%%%%%%%%%%%%%%%%%%%%%%%%%%%%%%%%%%%%%%%%%%%%%%%%%%%%%%%%%%%%%%%%%%%%%%%
\subsection{Conditional Processing}
\label{sec:conditional}

The package provides a mechanism to compile different versions
of a document. To customise the versions further some conditional processing
can come in handy to distinguish which version is being compiled.
The package provides two macros to describe the compilation context:

%%%%%%%%%%%%%%%%%%%%%%%%%%%%%%%%%%%%%%%%
\DescribeMacro{\ifchilddoc}
The conditional |\ifchilddoc| distinguishes between the compilation of
child documents and the main document:
%
\begin{center}
|\ifchilddoc |\textit{child-code}| |[|\||else |\textit{main-code}]| \||fi|
\end{center}

%%%%%%%%%%%%%%%%%%%%%%%%%%%%%%%%%%%%%%%%
\DescribeMacro{\childdocname}
\DescribeMacro{\childdocjob}
The macro |\childdocname| contains the filename (without extension)
of the main or child file being processed.
Note that |\childdocjob| will always contain the name of the main file.

%%%%%%%%%%%%%%%%%%%%%%%%%%%%%%%%%%%%%%%%
\paragraph{Title Page.}

Conditional processing can be used to include a title or banner page
in the main document when proper precautions are taken.
Importantly, the code in the main file should ensure that the page counter
(as well as other status parameters which are stored in the |.aux| files)
takes the same value after the conditional processing.
Otherwise the page numbers may take divergent values
depending on which part is compiled.

For example, a title page could be declared by:
%
\begin{center}
\begin{tabular}{l}
|\ifchilddoc\||else|\\
|\addtocounter{page}{-1}|\\
\textit{code for title page}\\
|\newpage|\\
|\||fi|
\end{tabular}
\end{center}
%
A banner page for the child documents can be generated by:
%
\begin{center}
\begin{tabular}{l}
|\ifchilddoc|\\
|\addtocounter{page}{-1}|\\
\textit{code for banner page}\\
|\newpage|\\
|\||fi|
\end{tabular}
\end{center}
%
Here one could write a message such as:
\begin{center}
|This is the part \childdocname{} of \childdocjob{}.|
\end{center}

%%%%%%%%%%%%%%%%%%%%%%%%%%%%%%%%%%%%%%%%%%%%%%%%%%%%%%%%%%%%%%%%%%%%%%%%%%%%%%%%
\subsection{Flags}
\label{sec:flags}

The package makes it easy to generate different versions
of the main or child documents.
To this end compilation flags can be defined
and assigned different default values.
They will be particularly useful in conjunction
with the forwarding mechanism described in \secref{sec:forward}.

For example, it may be useful to have a flag |\version|
which can be set to |draft| or |final|.
The document source will contain some conditional code
depending on the value of |\version|.
Suppose further, the flag should default to |final| for the main file
and to |draft| for child files
which is a natural assignment for editing the document.
This is achieved by placing the following code
in the preamble of the main document
(below the |\childdocmain| directive):
%
\begin{center}
\begin{tabular}{l}
|\ifchilddoc|\\
|\providecommand{\version}{draft}|\\
|\||else|\\
|\providecommand{\version}{final}|\\
|\||fi|
\end{tabular}
\end{center}
%
The definition by |\providecommand| makes sure
that previous definitions are not overwritten.
Further statements |\providecommand{\version}{...}|
can thus be added before the above code to override it.

For the main file, one might add a line
(between |\childdocmain| and the above block)
%
\begin{center}
|%\ifchilddoc\||else\providecommand{\version}{draft}\||fi|
\end{center}
%
which can be uncommented to produce a draft version.
Likewise one can add a line to the very top of a child file
(above the |\childdocof{|\textit{main}|}| directive)
%
\begin{center}
|%\providecommand{\version}{final}|
\end{center}
%
which can be uncommented to produce the final version of this child document.

%%%%%%%%%%%%%%%%%%%%%%%%%%%%%%%%%%%%%%%%%%%%%%%%%%%%%%%%%%%%%%%%%%%%%%%%%%%%%%%%
\subsection{Forwarding}
\label{sec:forward}

Different versions of the main or child documents
using compilation flags as described in \secref{sec:flags}
can be (permanently) stored in different files
for convenient compilation, viewing and distribution.
To this end, the package defines a command
to pass on compilation to a different file:

%%%%%%%%%%%%%%%%%%%%%%%%%%%%%%%%%%%%%%%%
\DescribeMacro{\childdocforward}
The command |\childdocforward| redirects processing to
another source file:
%
\begin{center}
\begin{tabular}{l}
|\input{childdoc.def}|\\
|\childdocforward[|\textit{main}|]{|\textit{dest}|}|\\
\end{tabular}
\end{center}
%
The argument \textit{dest} is the destination file
(without extension).
It should be the main file or one of the child files.
Note that further \textsf{childdoc} directives
such as |\childdocof| and |\childdocforward|
in the indicated file will be processed in this form.
The optional argument \textit{main}
passes on directly to the main file \textit{main}
while pretending to compile the child \textit{dest}.
This form behaves as if \textit{dest}
issues |\childdocof{|\textit{main}|}| right away,
and no further \textsf{childdoc} directives will be processed.

%%%%%%%%%%%%%%%%%%%%%%%%%%%%%%%%%%%%%%%%
\DescribeMacro{\...prefix}
In the alternative form |\childdocforwardprefix|,
%
\begin{center}
\begin{tabular}{l}
|\input{childdoc.def}|\\
|\childdocforwardprefix[|\textit{main}|]{|\textit{prefix}|}{|\textit{dest}|}|
\end{tabular}
\end{center}
%
the destination file is determined by a pattern
depending on the current file:
To make this work, the current file must be called
`{\textit{prefix}\hspace{0.2em}\textit{suffix}}'
with \textit{prefix} matching precisely the argument.
Processing is then passed on to the file
`{\textit{dest}\hspace{0.2em}\textit{suffix}}'.
Surely, the same effect is achieved by
directly specifying the
argument `{\textit{dest}\hspace{0.2em}\textit{suffix}}'
in the first form.
However, that requires to set up a different file
for each child. With the alternative form of the command
all these files can have exactly the same content
which simplifies setting them up and maintaining them.

For example, the following file |draft.tex|
with a compilation flag |\version| as described in \secref{sec:flags}
compiles the main document as a draft:
%
\begin{center}
\begin{tabular}{l}
|\def\version{draft}|\\
|\input{childdoc.def}|\\
|\childdocforward{|\textit{main}|}|
\end{tabular}
\end{center}
%
Likewise, the following files |final|\textit{nn}|.tex|
compile the final version of the child document
|child|\textit{nn}|.tex|:
%
\begin{center}
\begin{tabular}{l}
|\def\version{final}|\\
|\input{childdoc.def}|\\
|\childdocforwardprefix{final}{child}|
\end{tabular}
\end{center}
%

Note that when several versions of a main file and/or of each child file
are to be generated, it may be convenient to set up a |Makefile| or
shell script to automatise the process.

%%%%%%%%%%%%%%%%%%%%%%%%%%%%%%%%%%%%%%%%%%%%%%%%%%%%%%%%%%%%%%%%%%%%%%%%%%%%%%%%
\subsection{Command Line Processing}
\label{sec:commandline}

The effect of redirection files can also be achieved by invoking
the \LaTeX{} compiler with a more elaborate command line.
Most conveniently this should be done as part
of a shell script or a |Makefile|.

When using \textsf{childdoc} in the main file, the following
command lines effectively perform a redirection
(note that depending on the shell being used,
backslashes may have to be doubled: `|\|' $\to$ `|\\|'):
%
\begin{center}
|... -jobname "|\textit{target}|" |\\|"|[\textit{flags}]%
|\input{childdoc.def}\childdocforward[|\textit{main}|]{|\textit{dest}|}"|
\end{center}
%
Here \textit{target} is the name of the output file,
\textit{main} is the name of the main file
and \textit{dest} is the name of the main or child file to be processed
(all filenames without extensions).
The optional argument \textit{main} can be omitted
if \textit{main} matches \textit{dest}.
Optionally, compilation \textit{flags} can be defined via |\def| commands.
This command line makes the \TeX{} engine believe
it is compiling the file \textit{target}
whose content is specified as the latter parameter.
The provided code then forwards the processing to
\textit{main} or \textit{dest} as described in \secref{sec:forward}.

%%%%%%%%%%%%%%%%%%%%%%%%%%%%%%%%%%%%%%%%%%%%%%%%%%%%%%%%%%%%%%%%%%%%%%%%%%%%%%%%
\subsection{Include by Input}
\label{sec:input}

Including child documents by |\include| has some restrictions by design.
Most notably, the content of a child document always occupies
its own set of pages; pages cannot be shared between child documents.
Usually, this behaviour makes perfect sense
because each child document contain an essential part of the document.
However, in some situations it may be desirable to compose
a document from a collection of parts
without having mandatory page breaks between then.
For this case, the package
provides a mechanism to include parts
by |\input| which can also be processed individually.
However, by construction this mechanism
requires manual handling of the content to be output.

%%%%%%%%%%%%%%%%%%%%%%%%%%%%%%%%%%%%%%%%
\DescribeMacro{\ifchilddocmanual}
The main file should be prepared as usual, see \secref{sec:include}.
However, the document body must make a distinction
between processing of an individual part and of the main document, e.g.:
%
\begin{center}
\begin{tabular}{l}
|\ifchilddocmanual|\\
|\input{\childdocname}|\\
|\||else|\\
\textit{document body with }|\input{|\textit{part}|}|\\
|\||fi|
\end{tabular}
\end{center}
%
The conditional |\ifchilddocmanual| is true whenever
a part to be included by |\input| is being compiled,
and the name of the part is stored in |\childdocname|.

%%%%%%%%%%%%%%%%%%%%%%%%%%%%%%%%%%%%%%%%
\DescribeMacro{\childdocby}
Each part to be included by |\input| should start with:
%
\begin{center}
\begin{tabular}{l}
|\input{childdoc.def}|\\
|\childdocby{|\textit{main}|}|\\
\end{tabular}
\end{center}
%
The directive |\childdocby| is similar to |\childdocof|
described in \secref{sec:include},
but the subsequent selection of content must be done manually.
To that end, both |\ifchilddoc| and |\ifchilddocmanual|
will be true upon processing of a part,
and the name of the part is stored in |\childdocname|.
Note that |\jobname| will be set to the filename of the current part
so that each part receives an individual |.aux| file
that does not interfere with the |.aux| file(s) of the main document.
This behaviour can be altered by the alternative form
|\childdocby[*]{|\textit{main}|}| (with a non-empty optional argument)
which uses the |.aux| file of the main document
by setting |\jobname| to \textit{main}.

%%%%%%%%%%%%%%%%%%%%%%%%%%%%%%%%%%%%%%%%%%%%%%%%%%%%%%%%%%%%%%%%%%%%%%%%%%%%%%%%
\subsection{Driver Development}
\label{sec:driver}

The \textsf{childdoc} mechanism can also be use for the development
of definition files such as \LaTeX{} styles or classes.
This case differs from the above setup with multiple parts
included by |\include| in that no |\includeonly| should be invoked.
This can be achieved by starting the include file
(before |\ProvidesPackage|) with:
%
\begin{center}
\begin{tabular}{l}
|\input{childdoc.def}|\\
|\childdocforward{|\textit{main}|}|\\
\end{tabular}
\end{center}
%
or alternatively with:
%
\begin{center}
\begin{tabular}{l}
|\input{childdoc.def}|\\
|\childdocby{|\textit{main}|}|\\
\end{tabular}
\end{center}
%
Both forms have slightly different effects as described above.
The main file is prepared as usual, see \secref{sec:include}.

%%%%%%%%%%%%%%%%%%%%%%%%%%%%%%%%%%%%%%%%%%%%%%%%%%%%%%%%%%%%%%%%%%%%%%%%%%%%%%%%
\subsection{Legacy Detection}
\label{sec:detection}

The directive |\childdocmain| in the main file can detect
whether the complete document or merely a child is to be compiled
even without using the directive |\childdocof|.
This method is deprecated because it is less robust
and there is no compelling reason to use it;
it is merely provided for backward compatibility
and it may be removed in future versions.

If the detection mechanism is to be used,
it is mandatory to correctly specify
the filename of the main file as the argument of |\childdocmain|:
%
\begin{center}
\begin{tabular}{l}
|\input{childdoc.def}|\\
|\childdocmain{|\textit{main}|}|\\
\end{tabular}
\end{center}
%
If |\jobname| does not match the argument \textit{main} of |\childdocmain|,
it is assumed that |\jobname| points to the child file to be compiled.
When using |\childdocmain| with the main file specified as argument,
it suffices to start a child file
with just |\input{|\textit{main}|}|
without loading of the package and using |\childdocof|.
If instead all processing is done
with the appropriate \textsf{childdoc} directives,
the argument of \textit{main} of |\childdocmain| can be empty.

An alternative version of the command line processing described
in \secref{sec:commandline} using the detection mechanism reads:
%
\begin{center}
|... -jobname "|\textit{target}|" "|[\textit{flags}]%
[|\def\jobname{|\textit{dest}|}|]|\input{|\textit{main}|}"|
\end{center}

%%%%%%%%%%%%%%%%%%%%%%%%%%%%%%%%%%%%%%%%%%%%%%%%%%%%%%%%%%%%%%%%%%%%%%%%%%%%%%%%
\subsection{Manual Code}
\label{sec:manual}

In case one cannot be certain whether the definitions file |childdoc.def|
is installed on the target \TeX{} distribution
and one prefers not to ship it,
it is conceivable to paste a few relevant commands into the sources.

To that end, drop all statements |\input{childdoc.def}|
and perform the replacements as outlined below.
Instead of |\childdocmain{|\textit{main}|}| add the following code
to the top of the main file:
%
\begin{center}
\begin{tabular}{l}
|\||ifdefined\childdocname\endinput\||fi\newif\ifchilddoc|\\
|\edef\childdocname{\scantokens\expandafter{\jobname\noexpand}}|\\
|\def\childdocmain{|\textit{main}|}\||ifx\childdocmain\childdocname\||else|\\
|\childdoctrue\includeonly{\childdocname}\let\jobname\childdocmain\||fi|\\
\end{tabular}
\end{center}
%
Instead of |\childdocof{|\textit{main}|}| just include the main file
at the top of each child file:
%
\begin{center}
|\input{|\textit{main}|}|
\end{center}
%
A simple redirection |\childdocforward{|\textit{dest}|}| is achieved by:
%
\begin{center}
|\def\jobname{|\textit{dest}|}\input{\jobname}|
\end{center}
%
The redirection with prefix
|\childdocforwardprefix[|\textit{prefix}|]{|\textit{dest}|}|
is accomplished by:
%
\begin{center}
\begin{tabular}{l}
|{\edef\jobname{\scantokens\expandafter{\jobname\noexpand}}|\\
|\def\redirectjob |\textit{prefix}|#1~~~{\gdef\jobname{|\textit{dest}|#1}}|\\
|\expandafter\redirectjob\jobname~~~}\input{\jobname}|
\end{tabular}
\end{center}

In an alternative approach,
child documents can be compiled by a specific command line
without additional code or specific definitions:
%
\begin{center}
|... -jobname "|\textit{target}|" "|[\textit{flags}]%
|\includeonly{|\textit{dest}|}\input{|\textit{main}|}"|
\end{center}
%

%%%%%%%%%%%%%%%%%%%%%%%%%%%%%%%%%%%%%%%%%%%%%%%%%%%%%%%%%%%%%%%%%%%%%%%%%%%%%%%%
%%%%%%%%%%%%%%%%%%%%%%%%%%%%%%%%%%%%%%%%%%%%%%%%%%%%%%%%%%%%%%%%%%%%%%%%%%%%%%%%
\section{Information}

%%%%%%%%%%%%%%%%%%%%%%%%%%%%%%%%%%%%%%%%%%%%%%%%%%%%%%%%%%%%%%%%%%%%%%%%%%%%%%%%
\subsection{Copyright}

Copyright \copyright{} 2017--2018 Niklas Beisert

This work may be distributed and/or modified under the
conditions of the \LaTeX{} Project Public License, either version 1.3
of this license or (at your option) any later version.
The latest version of this license is in
  \url{http://www.latex-project.org/lppl.txt}
and version 1.3 or later is part of all distributions of \LaTeX{}
version 2005/12/01 or later.

This work has the LPPL maintenance status `maintained'.

The Current Maintainer of this work is Niklas Beisert.

This work consists of the files |README.txt|, |childdoc.ins| and |childdoc.dtx|
as well as the derived files |childdoc.def|, |cdocsamp.tex|
with |cdocsch1.tex|, |cdocsch2.tex|, |cdocspt3.tex|, |cdocspt4.tex|,
|cdocsdrf.tex|, |cdocsfn1.tex|, |cdocsfn2.tex|
as well as |childdoc.pdf|.

%%%%%%%%%%%%%%%%%%%%%%%%%%%%%%%%%%%%%%%%%%%%%%%%%%%%%%%%%%%%%%%%%%%%%%%%%%%%%%%%
\subsection{Files and Installation}

The package consists of the files:
%
\begin{center}
\begin{tabular}{ll}
    |README.txt|   & readme file \\
    |childdoc.ins| & installation file \\
    |childdoc.dtx| & source file \\
    |childdoc.def| & definition file \\
    |cdocsamp.tex| & sample main file \\
    |cdocsch1.tex| & sample include file \\
    |cdocsch2.tex| & sample include file \\
    |cdocspt3.tex| & sample part file \\
    |cdocspt4.tex| & sample part file \\
    |cdocsdrf.tex| & sample redirection file \\
    |cdocsfn1.tex| & sample redirection file \\
    |cdocsfn2.tex| & sample redirection file \\
    |childdoc.pdf| & manual
\end{tabular}
\end{center}
%
The distribution consists of the files
|README.txt|, |childdoc.ins| and |childdoc.dtx|.
%
\begin{itemize}
\item
Run (pdf)\LaTeX{} on |childdoc.dtx|
to compile the manual |childdoc.pdf| (this file).
\item
Run \LaTeX{} on |childdoc.ins| to create the definitions file |childdoc.def|
and the sample |cdocsamp.tex| with include files
|cdocsch1.tex|, |cdocsch2.tex|, |cdocspt3.tex|, |cdocspt4.tex|,
|cdocsdrf.tex|, |cdocsfn1.tex|, |cdocsfn2.tex|.
Then copy the file |childdoc.def| to an appropriate directory of your \LaTeX{}
distribution, e.g.\ \textit{texmf-root}|/tex/latex/childdoc|.
\end{itemize}

%%%%%%%%%%%%%%%%%%%%%%%%%%%%%%%%%%%%%%%%%%%%%%%%%%%%%%%%%%%%%%%%%%%%%%%%%%%%%%%%
\subsection{Related CTAN Packages}

There are several other packages which offer a similar functionality:
%
\begin{itemize}
\item
The packages
\href{http://ctan.org/pkg/docmute}{\textsf{docmute}},
\href{http://ctan.org/pkg/includex}{\textsf{includex}} and
\href{http://ctan.org/pkg/standalone}{\textsf{standalone}}
provide commands to include only the document body of
a child file thus allowing both files to be compiled individually.
\item
The packages \href{http://ctan.org/pkg/subdocs}{\textsf{subdocs}}
and \href{http://ctan.org/pkg/subfiles}{\textsf{subfiles}}
provide structures in which the main and child documents can be
encapsulated and allowing them to be compiled individually.
The inclusion mechanism is different from the conventional |\include|.
\item
The package \href{http://ctan.org/pkg/combine}{\textsf{combine}}
is an elaborate solution to combine several documents into one.
\end{itemize}
%
See also the CTAN topic \href{http://ctan.org/topic/subdocs}{\textsf{subdocs}}
for further related packages.
The present package differs from the above solutions in that
a document structure constructed with the conventional |\include| mechanism
just needs two extra commands at the top of every file
such that all constituent files can be compiled individually.

%%%%%%%%%%%%%%%%%%%%%%%%%%%%%%%%%%%%%%%%%%%%%%%%%%%%%%%%%%%%%%%%%%%%%%%%%%%%%%%%
%\subsection{Feature Suggestions}
%
%The following is a list of features which may be useful for future
%versions of this package:
%%
%\begin{itemize}
%\item
%\ldots
%\end{itemize}

%%%%%%%%%%%%%%%%%%%%%%%%%%%%%%%%%%%%%%%%%%%%%%%%%%%%%%%%%%%%%%%%%%%%%%%%%%%%%%%%
\subsection{Revision History}

%%%%%%%%%%%%%%%%%%%%%%%%%%%%%%%%%%%%%%%%
\paragraph{v2.0:} 2018/12/30

\begin{itemize}
\item
immediate forward processing
\item
added |\childdocby| mechanism
\item
manual restructured
\end{itemize}

%%%%%%%%%%%%%%%%%%%%%%%%%%%%%%%%%%%%%%%%
\paragraph{v1.6:} 2018/01/17

\begin{itemize}
\item
application for development of include files
\item
corrections to manual
\end{itemize}

%%%%%%%%%%%%%%%%%%%%%%%%%%%%%%%%%%%%%%%%
\paragraph{v1.5:} 2017/05/21

\begin{itemize}
\item
more complete structuring introduced
\item
|\childdocof| introduced
\item
|\childdoc| renamed to |\childdocmain|
\item
|\childredirect| renamed to |\childdocforward| and |\childdocforwardprefix|
and functionality expanded
\end{itemize}

%%%%%%%%%%%%%%%%%%%%%%%%%%%%%%%%%%%%%%%%
\paragraph{v1.0:} 2017/04/27

\begin{itemize}
\item
manual and install package
\item
first version published on CTAN
\end{itemize}

%%%%%%%%%%%%%%%%%%%%%%%%%%%%%%%%%%%%%%%%
\paragraph{v0.6:} 2017/04/26

\begin{itemize}
\item
redirection mechanism added
\end{itemize}

%%%%%%%%%%%%%%%%%%%%%%%%%%%%%%%%%%%%%%%%
\paragraph{v0.5:} 2017/04/26

\begin{itemize}
\item
functionality in definition file
\end{itemize}


%%%%%%%%%%%%%%%%%%%%%%%%%%%%%%%%%%%%%%%%%%%%%%%%%%%%%%%%%%%%%%%%%%%%%%%%%%%%%%%%
%%%%%%%%%%%%%%%%%%%%%%%%%%%%%%%%%%%%%%%%%%%%%%%%%%%%%%%%%%%%%%%%%%%%%%%%%%%%%%%%
%%%%%%%%%%%%%%%%%%%%%%%%%%%%%%%%%%%%%%%%%%%%%%%%%%%%%%%%%%%%%%%%%%%%%%%%%%%%%%%%
\appendix

\settowidth\MacroIndent{\rmfamily\scriptsize 000\ }

 \DocInput{childdoc.dtx}

\end{document}
%</driver>
% \fi
%
% %%%%%%%%%%%%%%%%%%%%%%%%%%%%%%%%%%%%%%%%%%%%%%%%%%%%%%%%%%%%%%%%%%%%%%%%%%%%%%
% %%%%%%%%%%%%%%%%%%%%%%%%%%%%%%%%%%%%%%%%%%%%%%%%%%%%%%%%%%%%%%%%%%%%%%%%%%%%%%
% \section{Sample}
%\iffalse
%<*samplemain>
%\fi
%
% The following presents a sample document
% with two chapters, two parts, a title page,
% a compile flag as well as three forwarding files to set the flag.
% It consists of eight |.tex| files:
% \begin{center}
% \begin{tabular}{ll}
% |cdocsamp.tex|&main file\\
% |cdocsch1.tex|&include file for chapter 1\\
% |cdocsch2.tex|&include file for chapter 2\\
% |cdocspt3.tex|&include file for part 3\\
% |cdocspt4.tex|&include file for part 4\\
% |cdocsdrf.tex|&forwarding file for main file in draft mode\\
% |cdocsfi1.tex|&forwarding file for final version of chapter 1\\
% |cdocsfi2.tex|&forwarding file for final version of chapter 2\\
% \end{tabular}
% \end{center}
% Each of the eight files can be compiled directly by the \LaTeX{} compiler.
%
% %%%%%%%%%%%%%%%%%%%%%%%%%%%%%%%%%%%%%%
% \paragraph{Main File.}
%
% The main file is called |cdocsamp.tex|.
%
% Load the \textsf{childdoc} definitions and
% declare the filename for the main document:
%    \begin{macrocode}
\input{childdoc.def}
\childdocmain{}
%    \end{macrocode}

% Optional override for |\version| flag:
%    \begin{macrocode}
%%\ifchilddoc\else\providecommand{\version}{draft}\fi
%    \end{macrocode}

% Define the default values for the |\version| flag
% (|final| for the main file and |draft| for childs):
%    \begin{macrocode}
\ifchilddoc
\providecommand{\version}{draft}
\else
\providecommand{\version}{final}
\fi
%    \end{macrocode}

% Load the standard document class:
%    \begin{macrocode}
\documentclass[12pt]{article}
%    \end{macrocode}

% Start the document body:
%    \begin{macrocode}
\begin{document}
%    \end{macrocode}

% Declare a title page.
% Print title, part of document being processed and version flag:
%    \begin{macrocode}
\addtocounter{page}{-1}
\begin{center}
{\LARGE\bfseries{}childdoc example\par}
\vspace{1cm}
\ifchilddoc
\ifchilddocmanual part\else chapter\fi:
`\childdocname' of `\childdocjob'\par
\else
main document: `\childdocjob'\par
\fi
version: \version\par
\end{center}
\newpage
%    \end{macrocode}

% Manually include selected file,
% otherwise process as usual:
%    \begin{macrocode}
\ifchilddocmanual
\section*{part `\childdocname'}
\input{\childdocname}
\else
%    \end{macrocode}

% Include the two chapters:
%    \begin{macrocode}
\include{cdocsch1}
\include{cdocsch2}
%    \end{macrocode}

% Include the two parts unless only chapters should be displayed:
%    \begin{macrocode}
\ifchilddoc\else
\section{part three}
\input{cdocspt3}
\section{part four}
\input{cdocspt4}
\fi
%    \end{macrocode}

% Process as usual until here:
%    \begin{macrocode}
\fi
%    \end{macrocode}

% End of document body:
%    \begin{macrocode}
\end{document}
%    \end{macrocode}
%\iffalse
%</samplemain>
%\fi
%
% %%%%%%%%%%%%%%%%%%%%%%%%%%%%%%%%%%%%%%
% \paragraph{Chapter Include Files.}
%
% The include files are called |cdocsch1.tex| and |cdocsch2.tex|.
%
%\iffalse
%<*samplechap1|samplechap2>
%\fi

% Optional override for |\version| flag:
%    \begin{macrocode}
%%\providecommand{\version}{final}
%    \end{macrocode}

% Include the main document:
%    \begin{macrocode}
\input{childdoc.def}
\childdocof{cdocsamp}
%    \end{macrocode}

%\iffalse
%</samplechap1|samplechap2>
%\fi
%
%\iffalse
%<*samplechap1>
%\fi
% Some text for chapter 1:
%    \begin{macrocode}
\section{one}
some text in chapter one
%    \end{macrocode}

%\iffalse
%</samplechap1>
%\fi
% Some text for chapter 2:
%\iffalse
%<*samplechap2>
%\fi
%    \begin{macrocode}
\section{two}
more text in chapter two
%    \end{macrocode}

%\iffalse
%</samplechap2>
%\fi
%
% %%%%%%%%%%%%%%%%%%%%%%%%%%%%%%%%%%%%%%
% \paragraph{Part Include Files.}
%
% The include files are called |cdocspt3.tex| and |cdocspt4.tex|.
%
%\iffalse
%<*samplepart3|samplepart4>
%\fi

% Optional override for |\version| flag:
%    \begin{macrocode}
%%\providecommand{\version}{final}
%    \end{macrocode}

% Include the main document:
%    \begin{macrocode}
\input{childdoc.def}
\childdocby{cdocsamp}
%    \end{macrocode}

%\iffalse
%</samplepart3|samplepart4>
%\fi
%
%\iffalse
%<*samplepart3>
%\fi
% Some text for part 3:
%    \begin{macrocode}
some text in part three
%    \end{macrocode}

%\iffalse
%</samplepart3>
%\fi
% Some text for part 4:
%\iffalse
%<*samplepart4>
%\fi
%    \begin{macrocode}
more text in part four
%    \end{macrocode}

%\iffalse
%</samplepart4>
%\fi
%
% %%%%%%%%%%%%%%%%%%%%%%%%%%%%%%%%%%%%%%
% \paragraph{Forwarding for a Complete Draft.}
%
% The following forwarding file |cdocsdrf.tex|
% compiles the main document in draft mode:
%\iffalse
%<*sampledraft>
%\fi
%    \begin{macrocode}
\def\version{draft}
\input{childdoc.def}
\childdocforward{cdocsamp}
%    \end{macrocode}

%\iffalse
%</sampledraft>
%\fi
%
% %%%%%%%%%%%%%%%%%%%%%%%%%%%%%%%%%%%%%%
% \paragraph{Forwarding for Final Version of the Chapters.}
%
% The following forwarding files |cdocsfn1.tex| and |cdocsfn2.tex|
% (with identical content)
% compile the final versions of the child documents
% |cdocsch1.tex| and |cdocsch2.tex|, respectively:
%\iffalse
%<*samplefinal>
%\fi
%    \begin{macrocode}
\def\version{final}
\input{childdoc.def}
\childdocforwardprefix[cdocsamp]{cdocsfn}{cdocsch}
%    \end{macrocode}

%\iffalse
%</samplefinal>
%\fi
%
% %%%%%%%%%%%%%%%%%%%%%%%%%%%%%%%%%%%%%%
% \paragraph{Command Line Processing.}
%
% The following three command lines generate the output files
% |cdocscld|, |cdocscl1| and |cdocscl2|
% which should be identical to
% |cdocsdrf|, |cdocsch1| and |cdocsfn2|, respectively:
% \begin{center}
% \begin{tabular}{l}
% |latex -jobname cdocscld \|\\
% |  "\def\version{draft}\input{childdoc.def}\childdocforward{cdocsamp}"|\\
% |latex -jobname cdocscl1 \|\\
% |  "\input{childdoc.def}\childdocforward[cdocsamp]{cdocsch1}"|\\
% |latex -jobname cdocscl2 \|\\
% |  "\def\version{final}\input{childdoc.def}\childdocforward{cdocsch2}"|
% \end{tabular}
% \end{center}
% Note that the trailing backslash on each first line
% merely continues the input to the second line
% (for convenient cut ant paste).
% Furthermore, the command |latex| can be replaced by any
% of its alternative versions such as |pdflatex|.
%
% %%%%%%%%%%%%%%%%%%%%%%%%%%%%%%%%%%%%%%%%%%%%%%%%%%%%%%%%%%%%%%%%%%%%%%%%%%%%%%
% %%%%%%%%%%%%%%%%%%%%%%%%%%%%%%%%%%%%%%%%%%%%%%%%%%%%%%%%%%%%%%%%%%%%%%%%%%%%%%
% \section{Implementation}
%\iffalse
%<*package>
%\fi
%
% This section describes the definitions file |childdoc.def|.

% The definitions cannot be loaded using |\usepackage| or |\RequirePackage|
% which has a mechanism to prevent loading a style file more than once.
% When loading the definitions by means of |\input|
% multiple instances have to be prevented manually:
%\iffalse
%This code needs to be before the `\ProvidesFile' directive
%which is defined at the beginning of this file.
%Therefore it is also placed there and commented out here.
%</package>
%<*discard>
%\fi
%    \begin{macrocode}
\ifdefined\childdocmain\endinput\fi
%    \end{macrocode}
%\iffalse
%</discard>
%<*package>
%\fi
%
% \macro{\ifchilddoc}
% \macro{\ifchilddocmanual}
% The conditional |\ifchilddoc| tells whether a
% child (true) or main (false) document is being compiled.
% The conditional |\ifchilddocmanual| tells whether
% the |\includeonly| mechanism is used (false) or
% the selection of child files must be performed manually (true).
% The definitions initialise to false:
%    \begin{macrocode}
\newif\ifchilddoc
\newif\ifchilddocmanual
%    \end{macrocode}

% \macro{\childdocname}
% \macro{\childdocjob}
% The macro |\childdocname| stores the name of the main document
% to be compiled. The macro |\childdocjob| stores the name of
% the document on which the \LaTeX{} compiler was originally invoked.
% The content of |\jobname| cannot be compared
% to filenames specified in the source due to different catcodes.
% The following code rescans |\jobname|, stores the result
% in |\childdocname| and saves a copy in |\childdocjob|:
%    \begin{macrocode}
\edef\childdocname{\scantokens\expandafter{\jobname\noexpand}}
\let\childdocjob\childdocname
%    \end{macrocode}

% \macro{\childdocdisable}
% The macro |\childdocdisable| prevents the main file
% from being processed more than once.
% At this stage, the main document command |\childdocmain|
% is assumed to be called once again where it should do nothing.
% Any subsequent call to it should prevent
% a secondary processing of the main document
% It overwrites the forwarding commands
% |\childdocof| and |\childdocforward|
% with empty macros to prevent further inclusions of the main document:
%    \begin{macrocode}
\newcommand{\childdocdisable}
{
  \renewcommand{\childdocmain}[1]{\renewcommand{\childdocmain}[1]{\endinput}}
  \renewcommand{\childdocof}[1]{}
  \renewcommand{\childdocby}[2][]{}
  \renewcommand{\childdocforward}[2][]{}
  \renewcommand{\childdocdisable}{}
}
%    \end{macrocode}

% \macro{\childdocmain}
% The macro |\childdocmain| is to be called at the top of the main file
% with nothing or the main filename (without extension) as argument.
% First, it breaks loops.
% If the argument is not empty and does not match |\childdocname|
% (which is set by the first inclusion of |childdoc.def|),
% |\ifchilddoc| is set to true, |\includeonly| is applied to the child file
% and |\jobname| is set to the main file
% (for proper handling of |.aux| files):
%    \begin{macrocode}
\newcommand{\childdocmain}[1]
{
  \childdocdisable\childdocmain{}
  \if?#1?\else
    \begingroup
      \def\childdoctmp{#1}
      \ifx\childdoctmp\childdocname
        \def\childdoctmp{}
      \else
        \def\childdoctmp
        {
          \childdoctrue
          \includeonly{\childdocname}
          \def\childdocjob{#1}
          \def\jobname{#1}
        }
      \fi
      \expandafter
    \endgroup
    \childdoctmp
  \fi
}
%    \end{macrocode}

% \macro{\childdocof}
% The command |\childdocof| redirects
% compilation to the main file |#1|.
%    \begin{macrocode}
\newcommand{\childdocof}[1]
{
  \childdocdisable
  \childdoctrue
  \includeonly{\childdocname}
  \def\jobname{#1}
  \def\childdocjob{#1}
  \input{#1}
}
%    \end{macrocode}

% \macro{\childdocby}
% The command |\childdocby| ....
%    \begin{macrocode}
\newcommand{\childdocby}[2][]
{
  \childdocdisable
  \childdoctrue
  \childdocmanualtrue
  \if?#1?\else
    \def\jobname{#2}
  \fi
  \def\childdocjob{#2}
  \input{#2}
  \endinput
}
%    \end{macrocode}

% \macro{\childdocforward}
% The command |\childdocforward| redirects
% compilation to the main file or
% (if the optional argument is given) a child file.
% Parameters are set as if the main file
% or a child file starting with |\childdocof| was compiled.
% Then compilation is handed over to the main file:
%    \begin{macrocode}
\newcommand{\childdocforward}[2][]
{
  \begingroup
    \if?#1?
      \def\childdoctmp
      {
        \def\childdocname{#2}
        \def\childdocjob{#2}
        \def\jobname{#2}
        \input{#2}
        \endinput
      }
    \else
      \def\childdoctmp
      {
        \childdocdisable
        \def\childdocname{#2}
        \childdoctrue
        \includeonly{#2}
        \def\childdocjob{#1}
        \def\jobname{#1}
        \input{#1}
        \endinput
      }
    \fi
    \expandafter
  \endgroup
  \childdoctmp
}
%    \end{macrocode}

% \macro{\childdocforwardprefix}
% The command |\childdocforwardprefix| redirects
% compilation to the main or a child file by means of a pattern.
% The prefix |#1| in the current filename is replaced by |#2|
% and the suffix of the current filename is kept
% (it is assumed that the filename does not contain the substring `|~~~|'
% which is used as a delimiter).
% Compilation is handed over to the new file by |\childdocforward|:
%    \begin{macrocode}
\newcommand{\childdocforwardprefix}[3][]
{
  \begingroup
    \def\childdocextract #2##1~~~{\def\childdoctmp{\childdocforward[#1]{#3##1}}}
    \expandafter\childdocextract\childdocname~~~
    \expandafter
  \endgroup
  \childdoctmp
}
%    \end{macrocode}

% \macro{\childdoc}
% The deprecated macro |\childdoc| is a legacy version of |\childdocmain|:
%    \begin{macrocode}
\newcommand{\childdoc}{\childdocmain}
%    \end{macrocode}

% \macro{\childdocredirect}
% The deprecated macro |\childdocredirect| is a legacy version
% of |\childdocforward| and |\childdocforwardprefix|:
%    \begin{macrocode}
\newcommand{\childdocredirect}[2][]
{
  \begingroup
    \if?#1?
      \def\childdoctmp{\childdocforward{#2}}
    \else
      \def\childdoctmp{\childdocforwardprefix{#1}{#2}}
    \fi
    \expandafter
  \endgroup
  \childdoctmp
}
%    \end{macrocode}

%\iffalse
%</package>
%\fi
%
\endinput
|
and perform the replacements as outlined below.
Instead of |\childdocmain{|\textit{main}|}| add the following code
to the top of the main file:
%
\begin{center}
\begin{tabular}{l}
|\||ifdefined\childdocname\endinput\||fi\newif\ifchilddoc|\\
|\edef\childdocname{\scantokens\expandafter{\jobname\noexpand}}|\\
|\def\childdocmain{|\textit{main}|}\||ifx\childdocmain\childdocname\||else|\\
|\childdoctrue\includeonly{\childdocname}\let\jobname\childdocmain\||fi|\\
\end{tabular}
\end{center}
%
Instead of |\childdocof{|\textit{main}|}| just include the main file
at the top of each child file:
%
\begin{center}
|\input{|\textit{main}|}|
\end{center}
%
A simple redirection |\childdocforward{|\textit{dest}|}| is achieved by:
%
\begin{center}
|\def\jobname{|\textit{dest}|}\input{\jobname}|
\end{center}
%
The redirection with prefix
|\childdocforwardprefix[|\textit{prefix}|]{|\textit{dest}|}|
is accomplished by:
%
\begin{center}
\begin{tabular}{l}
|{\edef\jobname{\scantokens\expandafter{\jobname\noexpand}}|\\
|\def\redirectjob |\textit{prefix}|#1~~~{\gdef\jobname{|\textit{dest}|#1}}|\\
|\expandafter\redirectjob\jobname~~~}\input{\jobname}|
\end{tabular}
\end{center}

In an alternative approach,
child documents can be compiled by a specific command line
without additional code or specific definitions:
%
\begin{center}
|... -jobname "|\textit{target}|" "|[\textit{flags}]%
|\includeonly{|\textit{dest}|}\input{|\textit{main}|}"|
\end{center}
%

%%%%%%%%%%%%%%%%%%%%%%%%%%%%%%%%%%%%%%%%%%%%%%%%%%%%%%%%%%%%%%%%%%%%%%%%%%%%%%%%
%%%%%%%%%%%%%%%%%%%%%%%%%%%%%%%%%%%%%%%%%%%%%%%%%%%%%%%%%%%%%%%%%%%%%%%%%%%%%%%%
\section{Information}

%%%%%%%%%%%%%%%%%%%%%%%%%%%%%%%%%%%%%%%%%%%%%%%%%%%%%%%%%%%%%%%%%%%%%%%%%%%%%%%%
\subsection{Copyright}

Copyright \copyright{} 2017--2018 Niklas Beisert

This work may be distributed and/or modified under the
conditions of the \LaTeX{} Project Public License, either version 1.3
of this license or (at your option) any later version.
The latest version of this license is in
  \url{http://www.latex-project.org/lppl.txt}
and version 1.3 or later is part of all distributions of \LaTeX{}
version 2005/12/01 or later.

This work has the LPPL maintenance status `maintained'.

The Current Maintainer of this work is Niklas Beisert.

This work consists of the files |README.txt|, |childdoc.ins| and |childdoc.dtx|
as well as the derived files |childdoc.def|, |cdocsamp.tex|
with |cdocsch1.tex|, |cdocsch2.tex|, |cdocspt3.tex|, |cdocspt4.tex|,
|cdocsdrf.tex|, |cdocsfn1.tex|, |cdocsfn2.tex|
as well as |childdoc.pdf|.

%%%%%%%%%%%%%%%%%%%%%%%%%%%%%%%%%%%%%%%%%%%%%%%%%%%%%%%%%%%%%%%%%%%%%%%%%%%%%%%%
\subsection{Files and Installation}

The package consists of the files:
%
\begin{center}
\begin{tabular}{ll}
    |README.txt|   & readme file \\
    |childdoc.ins| & installation file \\
    |childdoc.dtx| & source file \\
    |childdoc.def| & definition file \\
    |cdocsamp.tex| & sample main file \\
    |cdocsch1.tex| & sample include file \\
    |cdocsch2.tex| & sample include file \\
    |cdocspt3.tex| & sample part file \\
    |cdocspt4.tex| & sample part file \\
    |cdocsdrf.tex| & sample redirection file \\
    |cdocsfn1.tex| & sample redirection file \\
    |cdocsfn2.tex| & sample redirection file \\
    |childdoc.pdf| & manual
\end{tabular}
\end{center}
%
The distribution consists of the files
|README.txt|, |childdoc.ins| and |childdoc.dtx|.
%
\begin{itemize}
\item
Run (pdf)\LaTeX{} on |childdoc.dtx|
to compile the manual |childdoc.pdf| (this file).
\item
Run \LaTeX{} on |childdoc.ins| to create the definitions file |childdoc.def|
and the sample |cdocsamp.tex| with include files
|cdocsch1.tex|, |cdocsch2.tex|, |cdocspt3.tex|, |cdocspt4.tex|,
|cdocsdrf.tex|, |cdocsfn1.tex|, |cdocsfn2.tex|.
Then copy the file |childdoc.def| to an appropriate directory of your \LaTeX{}
distribution, e.g.\ \textit{texmf-root}|/tex/latex/childdoc|.
\end{itemize}

%%%%%%%%%%%%%%%%%%%%%%%%%%%%%%%%%%%%%%%%%%%%%%%%%%%%%%%%%%%%%%%%%%%%%%%%%%%%%%%%
\subsection{Related CTAN Packages}

There are several other packages which offer a similar functionality:
%
\begin{itemize}
\item
The packages
\href{http://ctan.org/pkg/docmute}{\textsf{docmute}},
\href{http://ctan.org/pkg/includex}{\textsf{includex}} and
\href{http://ctan.org/pkg/standalone}{\textsf{standalone}}
provide commands to include only the document body of
a child file thus allowing both files to be compiled individually.
\item
The packages \href{http://ctan.org/pkg/subdocs}{\textsf{subdocs}}
and \href{http://ctan.org/pkg/subfiles}{\textsf{subfiles}}
provide structures in which the main and child documents can be
encapsulated and allowing them to be compiled individually.
The inclusion mechanism is different from the conventional |\include|.
\item
The package \href{http://ctan.org/pkg/combine}{\textsf{combine}}
is an elaborate solution to combine several documents into one.
\end{itemize}
%
See also the CTAN topic \href{http://ctan.org/topic/subdocs}{\textsf{subdocs}}
for further related packages.
The present package differs from the above solutions in that
a document structure constructed with the conventional |\include| mechanism
just needs two extra commands at the top of every file
such that all constituent files can be compiled individually.

%%%%%%%%%%%%%%%%%%%%%%%%%%%%%%%%%%%%%%%%%%%%%%%%%%%%%%%%%%%%%%%%%%%%%%%%%%%%%%%%
%\subsection{Feature Suggestions}
%
%The following is a list of features which may be useful for future
%versions of this package:
%%
%\begin{itemize}
%\item
%\ldots
%\end{itemize}

%%%%%%%%%%%%%%%%%%%%%%%%%%%%%%%%%%%%%%%%%%%%%%%%%%%%%%%%%%%%%%%%%%%%%%%%%%%%%%%%
\subsection{Revision History}

%%%%%%%%%%%%%%%%%%%%%%%%%%%%%%%%%%%%%%%%
\paragraph{v2.0:} 2018/12/30

\begin{itemize}
\item
immediate forward processing
\item
added |\childdocby| mechanism
\item
manual restructured
\end{itemize}

%%%%%%%%%%%%%%%%%%%%%%%%%%%%%%%%%%%%%%%%
\paragraph{v1.6:} 2018/01/17

\begin{itemize}
\item
application for development of include files
\item
corrections to manual
\end{itemize}

%%%%%%%%%%%%%%%%%%%%%%%%%%%%%%%%%%%%%%%%
\paragraph{v1.5:} 2017/05/21

\begin{itemize}
\item
more complete structuring introduced
\item
|\childdocof| introduced
\item
|\childdoc| renamed to |\childdocmain|
\item
|\childredirect| renamed to |\childdocforward| and |\childdocforwardprefix|
and functionality expanded
\end{itemize}

%%%%%%%%%%%%%%%%%%%%%%%%%%%%%%%%%%%%%%%%
\paragraph{v1.0:} 2017/04/27

\begin{itemize}
\item
manual and install package
\item
first version published on CTAN
\end{itemize}

%%%%%%%%%%%%%%%%%%%%%%%%%%%%%%%%%%%%%%%%
\paragraph{v0.6:} 2017/04/26

\begin{itemize}
\item
redirection mechanism added
\end{itemize}

%%%%%%%%%%%%%%%%%%%%%%%%%%%%%%%%%%%%%%%%
\paragraph{v0.5:} 2017/04/26

\begin{itemize}
\item
functionality in definition file
\end{itemize}


%%%%%%%%%%%%%%%%%%%%%%%%%%%%%%%%%%%%%%%%%%%%%%%%%%%%%%%%%%%%%%%%%%%%%%%%%%%%%%%%
%%%%%%%%%%%%%%%%%%%%%%%%%%%%%%%%%%%%%%%%%%%%%%%%%%%%%%%%%%%%%%%%%%%%%%%%%%%%%%%%
%%%%%%%%%%%%%%%%%%%%%%%%%%%%%%%%%%%%%%%%%%%%%%%%%%%%%%%%%%%%%%%%%%%%%%%%%%%%%%%%
\appendix

\settowidth\MacroIndent{\rmfamily\scriptsize 000\ }

 \DocInput{childdoc.dtx}

\end{document}
%</driver>
% \fi
%
% %%%%%%%%%%%%%%%%%%%%%%%%%%%%%%%%%%%%%%%%%%%%%%%%%%%%%%%%%%%%%%%%%%%%%%%%%%%%%%
% %%%%%%%%%%%%%%%%%%%%%%%%%%%%%%%%%%%%%%%%%%%%%%%%%%%%%%%%%%%%%%%%%%%%%%%%%%%%%%
% \section{Sample}
%\iffalse
%<*samplemain>
%\fi
%
% The following presents a sample document
% with two chapters, two parts, a title page,
% a compile flag as well as three forwarding files to set the flag.
% It consists of eight |.tex| files:
% \begin{center}
% \begin{tabular}{ll}
% |cdocsamp.tex|&main file\\
% |cdocsch1.tex|&include file for chapter 1\\
% |cdocsch2.tex|&include file for chapter 2\\
% |cdocspt3.tex|&include file for part 3\\
% |cdocspt4.tex|&include file for part 4\\
% |cdocsdrf.tex|&forwarding file for main file in draft mode\\
% |cdocsfi1.tex|&forwarding file for final version of chapter 1\\
% |cdocsfi2.tex|&forwarding file for final version of chapter 2\\
% \end{tabular}
% \end{center}
% Each of the eight files can be compiled directly by the \LaTeX{} compiler.
%
% %%%%%%%%%%%%%%%%%%%%%%%%%%%%%%%%%%%%%%
% \paragraph{Main File.}
%
% The main file is called |cdocsamp.tex|.
%
% Load the \textsf{childdoc} definitions and
% declare the filename for the main document:
%    \begin{macrocode}
% \iffalse
%
% childdoc.dtx Copyright (C) 2017-2018 Niklas Beisert
%
% This work may be distributed and/or modified under the
% conditions of the LaTeX Project Public License, either version 1.3
% of this license or (at your option) any later version.
% The latest version of this license is in
%   http://www.latex-project.org/lppl.txt
% and version 1.3 or later is part of all distributions of LaTeX
% version 2005/12/01 or later.
%
% This work has the LPPL maintenance status `maintained'.
%
% The Current Maintainer of this work is Niklas Beisert.
%
% This work consists of the files childdoc.dtx and childdoc.ins
% and the derived files childdoc.def and cdocsamp.tex with
% cdocsch1.tex, cdocsch2.tex, cdocsdrf.tex, cdocsfn1.tex, cdocsfn2.tex.
%
%<package>\ifdefined\childdocmain\endinput\fi
%<package>\ProvidesFile{childdoc.def}[2018/12/30 v2.0 child document driver]
%<samplemain>\ProvidesFile{cdocsamp.tex}[2018/12/30 v2.0 sample for childdoc]
%<*driver>
%\ProvidesFile{childdoc.drv}[2018/12/30 v2.0 childdoc reference manual file]
\PassOptionsToClass{10pt,a4paper}{article}
\documentclass{ltxdoc}

\usepackage[margin=35mm]{geometry}
\usepackage{hyperref}
\usepackage{hyperxmp}
\usepackage[usenames]{color}

\hypersetup{colorlinks=true}
\hypersetup{pdfstartview=FitH}
\hypersetup{pdfpagemode=UseNone}
\hypersetup{pdfsource={}}
\hypersetup{pdflang={en-UK}}
\hypersetup{pdfcopyright={Copyright 2017-2018 Niklas Beisert.
  This work may be distributed and/or modified under the
  conditions of the LaTeX Project Public License, either version 1.3
  of this license or (at your option) any later version.}}
\hypersetup{pdflicenseurl={http://www.latex-project.org/lppl.txt}}
\hypersetup{pdfcontactaddress={ETH Zurich, ITP, HIT K,
  Wolfgang-Pauli-Strasse 27}}
\hypersetup{pdfcontactpostcode={8093}}
\hypersetup{pdfcontactcity={Zurich}}
\hypersetup{pdfcontactcountry={Switzerland}}
\hypersetup{pdfcontactemail={nbeisert@itp.phys.ethz.ch}}
\hypersetup{pdfcontacturl={http://people.phys.ethz.ch/\xmptilde nbeisert/}}

\newcommand{\secref}[1]{\hyperref[#1]{section \ref*{#1}}}

\parskip1ex
\parindent0pt
\let\olditemize\itemize
\def\itemize{\olditemize\parskip0pt}

\begin{document}

\title{The \textsf{childdoc} Package}
\hypersetup{pdftitle={The childdoc Package}}
\author{Niklas Beisert\\[2ex]
  Institut f\"ur Theoretische Physik\\
  Eidgen\"ossische Technische Hochschule Z\"urich\\
  Wolfgang-Pauli-Strasse 27, 8093 Z\"urich, Switzerland\\[1ex]
  \href{mailto:nbeisert@itp.phys.ethz.ch}
  {\texttt{nbeisert@itp.phys.ethz.ch}}}
\hypersetup{pdfauthor={Niklas Beisert}}
\hypersetup{pdfsubject={Manual for the LaTeX2e Package childdoc}}
\date{30 December 2018, \textsf{v2.0}}
\maketitle

\begin{abstract}\noindent
\textsf{childdoc} is a \LaTeXe{} package
that enables the direct compilation
of document sections included by |\include|
to individual files.
\end{abstract}

\begingroup
\parskip0ex
\tableofcontents
\endgroup

%%%%%%%%%%%%%%%%%%%%%%%%%%%%%%%%%%%%%%%%%%%%%%%%%%%%%%%%%%%%%%%%%%%%%%%%%%%%%%%%
%%%%%%%%%%%%%%%%%%%%%%%%%%%%%%%%%%%%%%%%%%%%%%%%%%%%%%%%%%%%%%%%%%%%%%%%%%%%%%%%
\section{Introduction}

\LaTeX{} provides a mechanism to structure a large document (such as a book)
into a main file and several child files (containing the chapters)
using the |\include| command.
This mechanism is beneficial for documents
which span hundreds of pages in order to
make the source file(s) more manageable.
Moreover, compilation can be restricted to
selected child files by means of the |\includeonly| command.
The latter feature can be used to reduce the compilation time while editing
(this was significantly more useful in the earlier days of \LaTeX{})
or to generate a smaller document which is easier to navigate.
Another application of |\includeonly| is to generate
documents consisting of selected parts of the complete document.

However, there are a few drawbacks of the plain |\include| mechanism:
\begin{itemize}
\item
The child files cannot be compiled on their own,
they can only be compiled via the main file.
A naive editing environment
(such as a text editor with an option
to have the current file processed by \LaTeX)
may require one to switch to the main file before compiling;
attempting to compile the child file produces errors.
\item
The main file must be modified (each time)
to adjust the |\includeonly| command
to the present needs. This easily leaves the main file in a messy state.
\item
The generated document will always carry the filename
of the main document. This is inconvenient if
several child files are to be compiled and
to be kept for distribution.
\end{itemize}

The present package provides a simple interface
to make child files individually compilable by \LaTeX{}.
Compiling a child file then has the same effect as compiling
the main file with an |\includeonly| command
to select the appropriate child.
Moreover the generated document will carry the name of the child
rather than the main file.
This resolves all three above issues.

This feature is meant to make the editing of books,
thesis documents and lecture notes somewhat more convenient.
However, the package can also be used efficiently for
composing a series of documents (such as exercise sheets)
which are typically distributed individually.
It then assists the author in generating the individual documents
(potentially in different versions)
as well as a document containing the collected series.
Another application is in developing style files
or other kinds of included material
where compilation of the style file could redirect
to a sample or test file.

%%%%%%%%%%%%%%%%%%%%%%%%%%%%%%%%%%%%%%%%%%%%%%%%%%%%%%%%%%%%%%%%%%%%%%%%%%%%%%%%
%%%%%%%%%%%%%%%%%%%%%%%%%%%%%%%%%%%%%%%%%%%%%%%%%%%%%%%%%%%%%%%%%%%%%%%%%%%%%%%%
\section{Usage}

First of all, the package \textsf{childdoc} is \emph{not} a standard
\LaTeXe{} |.sty| style file! Therefore it needs to be invoked in
a non-standard way.

%%%%%%%%%%%%%%%%%%%%%%%%%%%%%%%%%%%%%%%%%%%%%%%%%%%%%%%%%%%%%%%%%%%%%%%%%%%%%%%%
\subsection{Included Files}
\label{sec:include}

%%%%%%%%%%%%%%%%%%%%%%%%%%%%%%%%%%%%%%%%
\DescribeMacro{\childdocmain}
To use the package, add the commands
\begin{center}
\begin{tabular}{l}
|\input{childdoc.def}|\\
|\childdocmain{}|\\
\end{tabular}
\end{center}
at the very top of the main \LaTeX{} file,
in particular \emph{before} the |\documentclass| statement!
The argument of |\childdocmain| should be left empty
(but it must be present).

%%%%%%%%%%%%%%%%%%%%%%%%%%%%%%%%%%%%%%%%
\DescribeMacro{\childdocof}
Furthermore, add the commands
\begin{center}
\begin{tabular}{l}
|\input{childdoc.def}|\\
|\childdocof{|\textit{main}|}|\\
\end{tabular}
\end{center}
at the top of every child file \textit{child}
which is included by |\include{|\textit{child}|}|
from within the main file
(or at least for those files to be compiled individually).
The argument \textit{main} must be the filename of the main file.

There are a couple of
considerations in setting up the main and child documents:

%%%%%%%%%%%%%%%%%%%%%%%%%%%%%%%%%%%%%%%%
\paragraph{Restrictions.}

Please note the following restrictions:
\begin{itemize}
\item
|\childdocmain| must be called with one argument \textit{main}
to ensure compatibility with earlier version of the package.
It must either be empty (|\childdocmain{}|)
or precisely match the filename of the main file in which it is specified.
See \secref{sec:detection} for further information.
\item
The filename \textit{main} must be specified without the |.tex| extension.
\item
The filename \textit{main} is case sensitive
(even in case-insensitive file systems)
due to internal string comparison.
\item
The argument \textit{main} should be fully expanded, it cannot be a macro.
\item
Subdirectories and special characters should be avoided in filenames.
\item
The command |\childdocmain{|\textit{main}|}| must be followed by a whitespace.
It should not be followed immediately by another command
or by a comment mark `|%|'.
This is because the \TeX{} parser reads the token immediately following
the argument of |\childdocmain| and puts it
at the beginning of every child section;
however, a white\-space is ignored.
\end{itemize}

%%%%%%%%%%%%%%%%%%%%%%%%%%%%%%%%%%%%%%%%
\paragraph{Content of Main File.}

It is advisable to place all content in the child files included by |\include|.
Any output contained in the main file will appear in all child documents
unless suppressed manually;
it cannot be suppressed automatically by the |\includeonly| directive
and thus should normally be avoided.
A method to include some content in the main file
by means of conditional processing is described in \secref{sec:conditional}.

%%%%%%%%%%%%%%%%%%%%%%%%%%%%%%%%%%%%%%%%
\paragraph{Page Numbering.}

When only a part of the document is compiled,
the appropriate numbering of pages
(as well as other status parameters)
is determined from the |.aux| files.
The latter contain information from previous passes.
However this information needs to propagate through
all intermediate child documents.
Therefore the page numbering in child documents may well
be inconsistent until the complete document is compiled at least once.

A useful (if unconventional) way to always ensure a consistent
page numbering is to restart the numbering in each child document
and denote the pages by `\textit{child}|.|\textit{page}'
where \textit{child} represents the chapter/section number of the child file.
This can be achieved by the command
|\numberwithin{page}{|\textit{child}|}|
of the \textsf{amsmath} package
where \textit{child} can be |chapter| or |section|
depending on the chosen structuring.
Alternatively, one can modify the macro |\thepage| appropriately
and reset the counter |page| at the start of each child file.

%%%%%%%%%%%%%%%%%%%%%%%%%%%%%%%%%%%%%%%%%%%%%%%%%%%%%%%%%%%%%%%%%%%%%%%%%%%%%%%%
\subsection{Conditional Processing}
\label{sec:conditional}

The package provides a mechanism to compile different versions
of a document. To customise the versions further some conditional processing
can come in handy to distinguish which version is being compiled.
The package provides two macros to describe the compilation context:

%%%%%%%%%%%%%%%%%%%%%%%%%%%%%%%%%%%%%%%%
\DescribeMacro{\ifchilddoc}
The conditional |\ifchilddoc| distinguishes between the compilation of
child documents and the main document:
%
\begin{center}
|\ifchilddoc |\textit{child-code}| |[|\||else |\textit{main-code}]| \||fi|
\end{center}

%%%%%%%%%%%%%%%%%%%%%%%%%%%%%%%%%%%%%%%%
\DescribeMacro{\childdocname}
\DescribeMacro{\childdocjob}
The macro |\childdocname| contains the filename (without extension)
of the main or child file being processed.
Note that |\childdocjob| will always contain the name of the main file.

%%%%%%%%%%%%%%%%%%%%%%%%%%%%%%%%%%%%%%%%
\paragraph{Title Page.}

Conditional processing can be used to include a title or banner page
in the main document when proper precautions are taken.
Importantly, the code in the main file should ensure that the page counter
(as well as other status parameters which are stored in the |.aux| files)
takes the same value after the conditional processing.
Otherwise the page numbers may take divergent values
depending on which part is compiled.

For example, a title page could be declared by:
%
\begin{center}
\begin{tabular}{l}
|\ifchilddoc\||else|\\
|\addtocounter{page}{-1}|\\
\textit{code for title page}\\
|\newpage|\\
|\||fi|
\end{tabular}
\end{center}
%
A banner page for the child documents can be generated by:
%
\begin{center}
\begin{tabular}{l}
|\ifchilddoc|\\
|\addtocounter{page}{-1}|\\
\textit{code for banner page}\\
|\newpage|\\
|\||fi|
\end{tabular}
\end{center}
%
Here one could write a message such as:
\begin{center}
|This is the part \childdocname{} of \childdocjob{}.|
\end{center}

%%%%%%%%%%%%%%%%%%%%%%%%%%%%%%%%%%%%%%%%%%%%%%%%%%%%%%%%%%%%%%%%%%%%%%%%%%%%%%%%
\subsection{Flags}
\label{sec:flags}

The package makes it easy to generate different versions
of the main or child documents.
To this end compilation flags can be defined
and assigned different default values.
They will be particularly useful in conjunction
with the forwarding mechanism described in \secref{sec:forward}.

For example, it may be useful to have a flag |\version|
which can be set to |draft| or |final|.
The document source will contain some conditional code
depending on the value of |\version|.
Suppose further, the flag should default to |final| for the main file
and to |draft| for child files
which is a natural assignment for editing the document.
This is achieved by placing the following code
in the preamble of the main document
(below the |\childdocmain| directive):
%
\begin{center}
\begin{tabular}{l}
|\ifchilddoc|\\
|\providecommand{\version}{draft}|\\
|\||else|\\
|\providecommand{\version}{final}|\\
|\||fi|
\end{tabular}
\end{center}
%
The definition by |\providecommand| makes sure
that previous definitions are not overwritten.
Further statements |\providecommand{\version}{...}|
can thus be added before the above code to override it.

For the main file, one might add a line
(between |\childdocmain| and the above block)
%
\begin{center}
|%\ifchilddoc\||else\providecommand{\version}{draft}\||fi|
\end{center}
%
which can be uncommented to produce a draft version.
Likewise one can add a line to the very top of a child file
(above the |\childdocof{|\textit{main}|}| directive)
%
\begin{center}
|%\providecommand{\version}{final}|
\end{center}
%
which can be uncommented to produce the final version of this child document.

%%%%%%%%%%%%%%%%%%%%%%%%%%%%%%%%%%%%%%%%%%%%%%%%%%%%%%%%%%%%%%%%%%%%%%%%%%%%%%%%
\subsection{Forwarding}
\label{sec:forward}

Different versions of the main or child documents
using compilation flags as described in \secref{sec:flags}
can be (permanently) stored in different files
for convenient compilation, viewing and distribution.
To this end, the package defines a command
to pass on compilation to a different file:

%%%%%%%%%%%%%%%%%%%%%%%%%%%%%%%%%%%%%%%%
\DescribeMacro{\childdocforward}
The command |\childdocforward| redirects processing to
another source file:
%
\begin{center}
\begin{tabular}{l}
|\input{childdoc.def}|\\
|\childdocforward[|\textit{main}|]{|\textit{dest}|}|\\
\end{tabular}
\end{center}
%
The argument \textit{dest} is the destination file
(without extension).
It should be the main file or one of the child files.
Note that further \textsf{childdoc} directives
such as |\childdocof| and |\childdocforward|
in the indicated file will be processed in this form.
The optional argument \textit{main}
passes on directly to the main file \textit{main}
while pretending to compile the child \textit{dest}.
This form behaves as if \textit{dest}
issues |\childdocof{|\textit{main}|}| right away,
and no further \textsf{childdoc} directives will be processed.

%%%%%%%%%%%%%%%%%%%%%%%%%%%%%%%%%%%%%%%%
\DescribeMacro{\...prefix}
In the alternative form |\childdocforwardprefix|,
%
\begin{center}
\begin{tabular}{l}
|\input{childdoc.def}|\\
|\childdocforwardprefix[|\textit{main}|]{|\textit{prefix}|}{|\textit{dest}|}|
\end{tabular}
\end{center}
%
the destination file is determined by a pattern
depending on the current file:
To make this work, the current file must be called
`{\textit{prefix}\hspace{0.2em}\textit{suffix}}'
with \textit{prefix} matching precisely the argument.
Processing is then passed on to the file
`{\textit{dest}\hspace{0.2em}\textit{suffix}}'.
Surely, the same effect is achieved by
directly specifying the
argument `{\textit{dest}\hspace{0.2em}\textit{suffix}}'
in the first form.
However, that requires to set up a different file
for each child. With the alternative form of the command
all these files can have exactly the same content
which simplifies setting them up and maintaining them.

For example, the following file |draft.tex|
with a compilation flag |\version| as described in \secref{sec:flags}
compiles the main document as a draft:
%
\begin{center}
\begin{tabular}{l}
|\def\version{draft}|\\
|\input{childdoc.def}|\\
|\childdocforward{|\textit{main}|}|
\end{tabular}
\end{center}
%
Likewise, the following files |final|\textit{nn}|.tex|
compile the final version of the child document
|child|\textit{nn}|.tex|:
%
\begin{center}
\begin{tabular}{l}
|\def\version{final}|\\
|\input{childdoc.def}|\\
|\childdocforwardprefix{final}{child}|
\end{tabular}
\end{center}
%

Note that when several versions of a main file and/or of each child file
are to be generated, it may be convenient to set up a |Makefile| or
shell script to automatise the process.

%%%%%%%%%%%%%%%%%%%%%%%%%%%%%%%%%%%%%%%%%%%%%%%%%%%%%%%%%%%%%%%%%%%%%%%%%%%%%%%%
\subsection{Command Line Processing}
\label{sec:commandline}

The effect of redirection files can also be achieved by invoking
the \LaTeX{} compiler with a more elaborate command line.
Most conveniently this should be done as part
of a shell script or a |Makefile|.

When using \textsf{childdoc} in the main file, the following
command lines effectively perform a redirection
(note that depending on the shell being used,
backslashes may have to be doubled: `|\|' $\to$ `|\\|'):
%
\begin{center}
|... -jobname "|\textit{target}|" |\\|"|[\textit{flags}]%
|\input{childdoc.def}\childdocforward[|\textit{main}|]{|\textit{dest}|}"|
\end{center}
%
Here \textit{target} is the name of the output file,
\textit{main} is the name of the main file
and \textit{dest} is the name of the main or child file to be processed
(all filenames without extensions).
The optional argument \textit{main} can be omitted
if \textit{main} matches \textit{dest}.
Optionally, compilation \textit{flags} can be defined via |\def| commands.
This command line makes the \TeX{} engine believe
it is compiling the file \textit{target}
whose content is specified as the latter parameter.
The provided code then forwards the processing to
\textit{main} or \textit{dest} as described in \secref{sec:forward}.

%%%%%%%%%%%%%%%%%%%%%%%%%%%%%%%%%%%%%%%%%%%%%%%%%%%%%%%%%%%%%%%%%%%%%%%%%%%%%%%%
\subsection{Include by Input}
\label{sec:input}

Including child documents by |\include| has some restrictions by design.
Most notably, the content of a child document always occupies
its own set of pages; pages cannot be shared between child documents.
Usually, this behaviour makes perfect sense
because each child document contain an essential part of the document.
However, in some situations it may be desirable to compose
a document from a collection of parts
without having mandatory page breaks between then.
For this case, the package
provides a mechanism to include parts
by |\input| which can also be processed individually.
However, by construction this mechanism
requires manual handling of the content to be output.

%%%%%%%%%%%%%%%%%%%%%%%%%%%%%%%%%%%%%%%%
\DescribeMacro{\ifchilddocmanual}
The main file should be prepared as usual, see \secref{sec:include}.
However, the document body must make a distinction
between processing of an individual part and of the main document, e.g.:
%
\begin{center}
\begin{tabular}{l}
|\ifchilddocmanual|\\
|\input{\childdocname}|\\
|\||else|\\
\textit{document body with }|\input{|\textit{part}|}|\\
|\||fi|
\end{tabular}
\end{center}
%
The conditional |\ifchilddocmanual| is true whenever
a part to be included by |\input| is being compiled,
and the name of the part is stored in |\childdocname|.

%%%%%%%%%%%%%%%%%%%%%%%%%%%%%%%%%%%%%%%%
\DescribeMacro{\childdocby}
Each part to be included by |\input| should start with:
%
\begin{center}
\begin{tabular}{l}
|\input{childdoc.def}|\\
|\childdocby{|\textit{main}|}|\\
\end{tabular}
\end{center}
%
The directive |\childdocby| is similar to |\childdocof|
described in \secref{sec:include},
but the subsequent selection of content must be done manually.
To that end, both |\ifchilddoc| and |\ifchilddocmanual|
will be true upon processing of a part,
and the name of the part is stored in |\childdocname|.
Note that |\jobname| will be set to the filename of the current part
so that each part receives an individual |.aux| file
that does not interfere with the |.aux| file(s) of the main document.
This behaviour can be altered by the alternative form
|\childdocby[*]{|\textit{main}|}| (with a non-empty optional argument)
which uses the |.aux| file of the main document
by setting |\jobname| to \textit{main}.

%%%%%%%%%%%%%%%%%%%%%%%%%%%%%%%%%%%%%%%%%%%%%%%%%%%%%%%%%%%%%%%%%%%%%%%%%%%%%%%%
\subsection{Driver Development}
\label{sec:driver}

The \textsf{childdoc} mechanism can also be use for the development
of definition files such as \LaTeX{} styles or classes.
This case differs from the above setup with multiple parts
included by |\include| in that no |\includeonly| should be invoked.
This can be achieved by starting the include file
(before |\ProvidesPackage|) with:
%
\begin{center}
\begin{tabular}{l}
|\input{childdoc.def}|\\
|\childdocforward{|\textit{main}|}|\\
\end{tabular}
\end{center}
%
or alternatively with:
%
\begin{center}
\begin{tabular}{l}
|\input{childdoc.def}|\\
|\childdocby{|\textit{main}|}|\\
\end{tabular}
\end{center}
%
Both forms have slightly different effects as described above.
The main file is prepared as usual, see \secref{sec:include}.

%%%%%%%%%%%%%%%%%%%%%%%%%%%%%%%%%%%%%%%%%%%%%%%%%%%%%%%%%%%%%%%%%%%%%%%%%%%%%%%%
\subsection{Legacy Detection}
\label{sec:detection}

The directive |\childdocmain| in the main file can detect
whether the complete document or merely a child is to be compiled
even without using the directive |\childdocof|.
This method is deprecated because it is less robust
and there is no compelling reason to use it;
it is merely provided for backward compatibility
and it may be removed in future versions.

If the detection mechanism is to be used,
it is mandatory to correctly specify
the filename of the main file as the argument of |\childdocmain|:
%
\begin{center}
\begin{tabular}{l}
|\input{childdoc.def}|\\
|\childdocmain{|\textit{main}|}|\\
\end{tabular}
\end{center}
%
If |\jobname| does not match the argument \textit{main} of |\childdocmain|,
it is assumed that |\jobname| points to the child file to be compiled.
When using |\childdocmain| with the main file specified as argument,
it suffices to start a child file
with just |\input{|\textit{main}|}|
without loading of the package and using |\childdocof|.
If instead all processing is done
with the appropriate \textsf{childdoc} directives,
the argument of \textit{main} of |\childdocmain| can be empty.

An alternative version of the command line processing described
in \secref{sec:commandline} using the detection mechanism reads:
%
\begin{center}
|... -jobname "|\textit{target}|" "|[\textit{flags}]%
[|\def\jobname{|\textit{dest}|}|]|\input{|\textit{main}|}"|
\end{center}

%%%%%%%%%%%%%%%%%%%%%%%%%%%%%%%%%%%%%%%%%%%%%%%%%%%%%%%%%%%%%%%%%%%%%%%%%%%%%%%%
\subsection{Manual Code}
\label{sec:manual}

In case one cannot be certain whether the definitions file |childdoc.def|
is installed on the target \TeX{} distribution
and one prefers not to ship it,
it is conceivable to paste a few relevant commands into the sources.

To that end, drop all statements |\input{childdoc.def}|
and perform the replacements as outlined below.
Instead of |\childdocmain{|\textit{main}|}| add the following code
to the top of the main file:
%
\begin{center}
\begin{tabular}{l}
|\||ifdefined\childdocname\endinput\||fi\newif\ifchilddoc|\\
|\edef\childdocname{\scantokens\expandafter{\jobname\noexpand}}|\\
|\def\childdocmain{|\textit{main}|}\||ifx\childdocmain\childdocname\||else|\\
|\childdoctrue\includeonly{\childdocname}\let\jobname\childdocmain\||fi|\\
\end{tabular}
\end{center}
%
Instead of |\childdocof{|\textit{main}|}| just include the main file
at the top of each child file:
%
\begin{center}
|\input{|\textit{main}|}|
\end{center}
%
A simple redirection |\childdocforward{|\textit{dest}|}| is achieved by:
%
\begin{center}
|\def\jobname{|\textit{dest}|}\input{\jobname}|
\end{center}
%
The redirection with prefix
|\childdocforwardprefix[|\textit{prefix}|]{|\textit{dest}|}|
is accomplished by:
%
\begin{center}
\begin{tabular}{l}
|{\edef\jobname{\scantokens\expandafter{\jobname\noexpand}}|\\
|\def\redirectjob |\textit{prefix}|#1~~~{\gdef\jobname{|\textit{dest}|#1}}|\\
|\expandafter\redirectjob\jobname~~~}\input{\jobname}|
\end{tabular}
\end{center}

In an alternative approach,
child documents can be compiled by a specific command line
without additional code or specific definitions:
%
\begin{center}
|... -jobname "|\textit{target}|" "|[\textit{flags}]%
|\includeonly{|\textit{dest}|}\input{|\textit{main}|}"|
\end{center}
%

%%%%%%%%%%%%%%%%%%%%%%%%%%%%%%%%%%%%%%%%%%%%%%%%%%%%%%%%%%%%%%%%%%%%%%%%%%%%%%%%
%%%%%%%%%%%%%%%%%%%%%%%%%%%%%%%%%%%%%%%%%%%%%%%%%%%%%%%%%%%%%%%%%%%%%%%%%%%%%%%%
\section{Information}

%%%%%%%%%%%%%%%%%%%%%%%%%%%%%%%%%%%%%%%%%%%%%%%%%%%%%%%%%%%%%%%%%%%%%%%%%%%%%%%%
\subsection{Copyright}

Copyright \copyright{} 2017--2018 Niklas Beisert

This work may be distributed and/or modified under the
conditions of the \LaTeX{} Project Public License, either version 1.3
of this license or (at your option) any later version.
The latest version of this license is in
  \url{http://www.latex-project.org/lppl.txt}
and version 1.3 or later is part of all distributions of \LaTeX{}
version 2005/12/01 or later.

This work has the LPPL maintenance status `maintained'.

The Current Maintainer of this work is Niklas Beisert.

This work consists of the files |README.txt|, |childdoc.ins| and |childdoc.dtx|
as well as the derived files |childdoc.def|, |cdocsamp.tex|
with |cdocsch1.tex|, |cdocsch2.tex|, |cdocspt3.tex|, |cdocspt4.tex|,
|cdocsdrf.tex|, |cdocsfn1.tex|, |cdocsfn2.tex|
as well as |childdoc.pdf|.

%%%%%%%%%%%%%%%%%%%%%%%%%%%%%%%%%%%%%%%%%%%%%%%%%%%%%%%%%%%%%%%%%%%%%%%%%%%%%%%%
\subsection{Files and Installation}

The package consists of the files:
%
\begin{center}
\begin{tabular}{ll}
    |README.txt|   & readme file \\
    |childdoc.ins| & installation file \\
    |childdoc.dtx| & source file \\
    |childdoc.def| & definition file \\
    |cdocsamp.tex| & sample main file \\
    |cdocsch1.tex| & sample include file \\
    |cdocsch2.tex| & sample include file \\
    |cdocspt3.tex| & sample part file \\
    |cdocspt4.tex| & sample part file \\
    |cdocsdrf.tex| & sample redirection file \\
    |cdocsfn1.tex| & sample redirection file \\
    |cdocsfn2.tex| & sample redirection file \\
    |childdoc.pdf| & manual
\end{tabular}
\end{center}
%
The distribution consists of the files
|README.txt|, |childdoc.ins| and |childdoc.dtx|.
%
\begin{itemize}
\item
Run (pdf)\LaTeX{} on |childdoc.dtx|
to compile the manual |childdoc.pdf| (this file).
\item
Run \LaTeX{} on |childdoc.ins| to create the definitions file |childdoc.def|
and the sample |cdocsamp.tex| with include files
|cdocsch1.tex|, |cdocsch2.tex|, |cdocspt3.tex|, |cdocspt4.tex|,
|cdocsdrf.tex|, |cdocsfn1.tex|, |cdocsfn2.tex|.
Then copy the file |childdoc.def| to an appropriate directory of your \LaTeX{}
distribution, e.g.\ \textit{texmf-root}|/tex/latex/childdoc|.
\end{itemize}

%%%%%%%%%%%%%%%%%%%%%%%%%%%%%%%%%%%%%%%%%%%%%%%%%%%%%%%%%%%%%%%%%%%%%%%%%%%%%%%%
\subsection{Related CTAN Packages}

There are several other packages which offer a similar functionality:
%
\begin{itemize}
\item
The packages
\href{http://ctan.org/pkg/docmute}{\textsf{docmute}},
\href{http://ctan.org/pkg/includex}{\textsf{includex}} and
\href{http://ctan.org/pkg/standalone}{\textsf{standalone}}
provide commands to include only the document body of
a child file thus allowing both files to be compiled individually.
\item
The packages \href{http://ctan.org/pkg/subdocs}{\textsf{subdocs}}
and \href{http://ctan.org/pkg/subfiles}{\textsf{subfiles}}
provide structures in which the main and child documents can be
encapsulated and allowing them to be compiled individually.
The inclusion mechanism is different from the conventional |\include|.
\item
The package \href{http://ctan.org/pkg/combine}{\textsf{combine}}
is an elaborate solution to combine several documents into one.
\end{itemize}
%
See also the CTAN topic \href{http://ctan.org/topic/subdocs}{\textsf{subdocs}}
for further related packages.
The present package differs from the above solutions in that
a document structure constructed with the conventional |\include| mechanism
just needs two extra commands at the top of every file
such that all constituent files can be compiled individually.

%%%%%%%%%%%%%%%%%%%%%%%%%%%%%%%%%%%%%%%%%%%%%%%%%%%%%%%%%%%%%%%%%%%%%%%%%%%%%%%%
%\subsection{Feature Suggestions}
%
%The following is a list of features which may be useful for future
%versions of this package:
%%
%\begin{itemize}
%\item
%\ldots
%\end{itemize}

%%%%%%%%%%%%%%%%%%%%%%%%%%%%%%%%%%%%%%%%%%%%%%%%%%%%%%%%%%%%%%%%%%%%%%%%%%%%%%%%
\subsection{Revision History}

%%%%%%%%%%%%%%%%%%%%%%%%%%%%%%%%%%%%%%%%
\paragraph{v2.0:} 2018/12/30

\begin{itemize}
\item
immediate forward processing
\item
added |\childdocby| mechanism
\item
manual restructured
\end{itemize}

%%%%%%%%%%%%%%%%%%%%%%%%%%%%%%%%%%%%%%%%
\paragraph{v1.6:} 2018/01/17

\begin{itemize}
\item
application for development of include files
\item
corrections to manual
\end{itemize}

%%%%%%%%%%%%%%%%%%%%%%%%%%%%%%%%%%%%%%%%
\paragraph{v1.5:} 2017/05/21

\begin{itemize}
\item
more complete structuring introduced
\item
|\childdocof| introduced
\item
|\childdoc| renamed to |\childdocmain|
\item
|\childredirect| renamed to |\childdocforward| and |\childdocforwardprefix|
and functionality expanded
\end{itemize}

%%%%%%%%%%%%%%%%%%%%%%%%%%%%%%%%%%%%%%%%
\paragraph{v1.0:} 2017/04/27

\begin{itemize}
\item
manual and install package
\item
first version published on CTAN
\end{itemize}

%%%%%%%%%%%%%%%%%%%%%%%%%%%%%%%%%%%%%%%%
\paragraph{v0.6:} 2017/04/26

\begin{itemize}
\item
redirection mechanism added
\end{itemize}

%%%%%%%%%%%%%%%%%%%%%%%%%%%%%%%%%%%%%%%%
\paragraph{v0.5:} 2017/04/26

\begin{itemize}
\item
functionality in definition file
\end{itemize}


%%%%%%%%%%%%%%%%%%%%%%%%%%%%%%%%%%%%%%%%%%%%%%%%%%%%%%%%%%%%%%%%%%%%%%%%%%%%%%%%
%%%%%%%%%%%%%%%%%%%%%%%%%%%%%%%%%%%%%%%%%%%%%%%%%%%%%%%%%%%%%%%%%%%%%%%%%%%%%%%%
%%%%%%%%%%%%%%%%%%%%%%%%%%%%%%%%%%%%%%%%%%%%%%%%%%%%%%%%%%%%%%%%%%%%%%%%%%%%%%%%
\appendix

\settowidth\MacroIndent{\rmfamily\scriptsize 000\ }

 \DocInput{childdoc.dtx}

\end{document}
%</driver>
% \fi
%
% %%%%%%%%%%%%%%%%%%%%%%%%%%%%%%%%%%%%%%%%%%%%%%%%%%%%%%%%%%%%%%%%%%%%%%%%%%%%%%
% %%%%%%%%%%%%%%%%%%%%%%%%%%%%%%%%%%%%%%%%%%%%%%%%%%%%%%%%%%%%%%%%%%%%%%%%%%%%%%
% \section{Sample}
%\iffalse
%<*samplemain>
%\fi
%
% The following presents a sample document
% with two chapters, two parts, a title page,
% a compile flag as well as three forwarding files to set the flag.
% It consists of eight |.tex| files:
% \begin{center}
% \begin{tabular}{ll}
% |cdocsamp.tex|&main file\\
% |cdocsch1.tex|&include file for chapter 1\\
% |cdocsch2.tex|&include file for chapter 2\\
% |cdocspt3.tex|&include file for part 3\\
% |cdocspt4.tex|&include file for part 4\\
% |cdocsdrf.tex|&forwarding file for main file in draft mode\\
% |cdocsfi1.tex|&forwarding file for final version of chapter 1\\
% |cdocsfi2.tex|&forwarding file for final version of chapter 2\\
% \end{tabular}
% \end{center}
% Each of the eight files can be compiled directly by the \LaTeX{} compiler.
%
% %%%%%%%%%%%%%%%%%%%%%%%%%%%%%%%%%%%%%%
% \paragraph{Main File.}
%
% The main file is called |cdocsamp.tex|.
%
% Load the \textsf{childdoc} definitions and
% declare the filename for the main document:
%    \begin{macrocode}
\input{childdoc.def}
\childdocmain{}
%    \end{macrocode}

% Optional override for |\version| flag:
%    \begin{macrocode}
%%\ifchilddoc\else\providecommand{\version}{draft}\fi
%    \end{macrocode}

% Define the default values for the |\version| flag
% (|final| for the main file and |draft| for childs):
%    \begin{macrocode}
\ifchilddoc
\providecommand{\version}{draft}
\else
\providecommand{\version}{final}
\fi
%    \end{macrocode}

% Load the standard document class:
%    \begin{macrocode}
\documentclass[12pt]{article}
%    \end{macrocode}

% Start the document body:
%    \begin{macrocode}
\begin{document}
%    \end{macrocode}

% Declare a title page.
% Print title, part of document being processed and version flag:
%    \begin{macrocode}
\addtocounter{page}{-1}
\begin{center}
{\LARGE\bfseries{}childdoc example\par}
\vspace{1cm}
\ifchilddoc
\ifchilddocmanual part\else chapter\fi:
`\childdocname' of `\childdocjob'\par
\else
main document: `\childdocjob'\par
\fi
version: \version\par
\end{center}
\newpage
%    \end{macrocode}

% Manually include selected file,
% otherwise process as usual:
%    \begin{macrocode}
\ifchilddocmanual
\section*{part `\childdocname'}
\input{\childdocname}
\else
%    \end{macrocode}

% Include the two chapters:
%    \begin{macrocode}
\include{cdocsch1}
\include{cdocsch2}
%    \end{macrocode}

% Include the two parts unless only chapters should be displayed:
%    \begin{macrocode}
\ifchilddoc\else
\section{part three}
\input{cdocspt3}
\section{part four}
\input{cdocspt4}
\fi
%    \end{macrocode}

% Process as usual until here:
%    \begin{macrocode}
\fi
%    \end{macrocode}

% End of document body:
%    \begin{macrocode}
\end{document}
%    \end{macrocode}
%\iffalse
%</samplemain>
%\fi
%
% %%%%%%%%%%%%%%%%%%%%%%%%%%%%%%%%%%%%%%
% \paragraph{Chapter Include Files.}
%
% The include files are called |cdocsch1.tex| and |cdocsch2.tex|.
%
%\iffalse
%<*samplechap1|samplechap2>
%\fi

% Optional override for |\version| flag:
%    \begin{macrocode}
%%\providecommand{\version}{final}
%    \end{macrocode}

% Include the main document:
%    \begin{macrocode}
\input{childdoc.def}
\childdocof{cdocsamp}
%    \end{macrocode}

%\iffalse
%</samplechap1|samplechap2>
%\fi
%
%\iffalse
%<*samplechap1>
%\fi
% Some text for chapter 1:
%    \begin{macrocode}
\section{one}
some text in chapter one
%    \end{macrocode}

%\iffalse
%</samplechap1>
%\fi
% Some text for chapter 2:
%\iffalse
%<*samplechap2>
%\fi
%    \begin{macrocode}
\section{two}
more text in chapter two
%    \end{macrocode}

%\iffalse
%</samplechap2>
%\fi
%
% %%%%%%%%%%%%%%%%%%%%%%%%%%%%%%%%%%%%%%
% \paragraph{Part Include Files.}
%
% The include files are called |cdocspt3.tex| and |cdocspt4.tex|.
%
%\iffalse
%<*samplepart3|samplepart4>
%\fi

% Optional override for |\version| flag:
%    \begin{macrocode}
%%\providecommand{\version}{final}
%    \end{macrocode}

% Include the main document:
%    \begin{macrocode}
\input{childdoc.def}
\childdocby{cdocsamp}
%    \end{macrocode}

%\iffalse
%</samplepart3|samplepart4>
%\fi
%
%\iffalse
%<*samplepart3>
%\fi
% Some text for part 3:
%    \begin{macrocode}
some text in part three
%    \end{macrocode}

%\iffalse
%</samplepart3>
%\fi
% Some text for part 4:
%\iffalse
%<*samplepart4>
%\fi
%    \begin{macrocode}
more text in part four
%    \end{macrocode}

%\iffalse
%</samplepart4>
%\fi
%
% %%%%%%%%%%%%%%%%%%%%%%%%%%%%%%%%%%%%%%
% \paragraph{Forwarding for a Complete Draft.}
%
% The following forwarding file |cdocsdrf.tex|
% compiles the main document in draft mode:
%\iffalse
%<*sampledraft>
%\fi
%    \begin{macrocode}
\def\version{draft}
\input{childdoc.def}
\childdocforward{cdocsamp}
%    \end{macrocode}

%\iffalse
%</sampledraft>
%\fi
%
% %%%%%%%%%%%%%%%%%%%%%%%%%%%%%%%%%%%%%%
% \paragraph{Forwarding for Final Version of the Chapters.}
%
% The following forwarding files |cdocsfn1.tex| and |cdocsfn2.tex|
% (with identical content)
% compile the final versions of the child documents
% |cdocsch1.tex| and |cdocsch2.tex|, respectively:
%\iffalse
%<*samplefinal>
%\fi
%    \begin{macrocode}
\def\version{final}
\input{childdoc.def}
\childdocforwardprefix[cdocsamp]{cdocsfn}{cdocsch}
%    \end{macrocode}

%\iffalse
%</samplefinal>
%\fi
%
% %%%%%%%%%%%%%%%%%%%%%%%%%%%%%%%%%%%%%%
% \paragraph{Command Line Processing.}
%
% The following three command lines generate the output files
% |cdocscld|, |cdocscl1| and |cdocscl2|
% which should be identical to
% |cdocsdrf|, |cdocsch1| and |cdocsfn2|, respectively:
% \begin{center}
% \begin{tabular}{l}
% |latex -jobname cdocscld \|\\
% |  "\def\version{draft}\input{childdoc.def}\childdocforward{cdocsamp}"|\\
% |latex -jobname cdocscl1 \|\\
% |  "\input{childdoc.def}\childdocforward[cdocsamp]{cdocsch1}"|\\
% |latex -jobname cdocscl2 \|\\
% |  "\def\version{final}\input{childdoc.def}\childdocforward{cdocsch2}"|
% \end{tabular}
% \end{center}
% Note that the trailing backslash on each first line
% merely continues the input to the second line
% (for convenient cut ant paste).
% Furthermore, the command |latex| can be replaced by any
% of its alternative versions such as |pdflatex|.
%
% %%%%%%%%%%%%%%%%%%%%%%%%%%%%%%%%%%%%%%%%%%%%%%%%%%%%%%%%%%%%%%%%%%%%%%%%%%%%%%
% %%%%%%%%%%%%%%%%%%%%%%%%%%%%%%%%%%%%%%%%%%%%%%%%%%%%%%%%%%%%%%%%%%%%%%%%%%%%%%
% \section{Implementation}
%\iffalse
%<*package>
%\fi
%
% This section describes the definitions file |childdoc.def|.

% The definitions cannot be loaded using |\usepackage| or |\RequirePackage|
% which has a mechanism to prevent loading a style file more than once.
% When loading the definitions by means of |\input|
% multiple instances have to be prevented manually:
%\iffalse
%This code needs to be before the `\ProvidesFile' directive
%which is defined at the beginning of this file.
%Therefore it is also placed there and commented out here.
%</package>
%<*discard>
%\fi
%    \begin{macrocode}
\ifdefined\childdocmain\endinput\fi
%    \end{macrocode}
%\iffalse
%</discard>
%<*package>
%\fi
%
% \macro{\ifchilddoc}
% \macro{\ifchilddocmanual}
% The conditional |\ifchilddoc| tells whether a
% child (true) or main (false) document is being compiled.
% The conditional |\ifchilddocmanual| tells whether
% the |\includeonly| mechanism is used (false) or
% the selection of child files must be performed manually (true).
% The definitions initialise to false:
%    \begin{macrocode}
\newif\ifchilddoc
\newif\ifchilddocmanual
%    \end{macrocode}

% \macro{\childdocname}
% \macro{\childdocjob}
% The macro |\childdocname| stores the name of the main document
% to be compiled. The macro |\childdocjob| stores the name of
% the document on which the \LaTeX{} compiler was originally invoked.
% The content of |\jobname| cannot be compared
% to filenames specified in the source due to different catcodes.
% The following code rescans |\jobname|, stores the result
% in |\childdocname| and saves a copy in |\childdocjob|:
%    \begin{macrocode}
\edef\childdocname{\scantokens\expandafter{\jobname\noexpand}}
\let\childdocjob\childdocname
%    \end{macrocode}

% \macro{\childdocdisable}
% The macro |\childdocdisable| prevents the main file
% from being processed more than once.
% At this stage, the main document command |\childdocmain|
% is assumed to be called once again where it should do nothing.
% Any subsequent call to it should prevent
% a secondary processing of the main document
% It overwrites the forwarding commands
% |\childdocof| and |\childdocforward|
% with empty macros to prevent further inclusions of the main document:
%    \begin{macrocode}
\newcommand{\childdocdisable}
{
  \renewcommand{\childdocmain}[1]{\renewcommand{\childdocmain}[1]{\endinput}}
  \renewcommand{\childdocof}[1]{}
  \renewcommand{\childdocby}[2][]{}
  \renewcommand{\childdocforward}[2][]{}
  \renewcommand{\childdocdisable}{}
}
%    \end{macrocode}

% \macro{\childdocmain}
% The macro |\childdocmain| is to be called at the top of the main file
% with nothing or the main filename (without extension) as argument.
% First, it breaks loops.
% If the argument is not empty and does not match |\childdocname|
% (which is set by the first inclusion of |childdoc.def|),
% |\ifchilddoc| is set to true, |\includeonly| is applied to the child file
% and |\jobname| is set to the main file
% (for proper handling of |.aux| files):
%    \begin{macrocode}
\newcommand{\childdocmain}[1]
{
  \childdocdisable\childdocmain{}
  \if?#1?\else
    \begingroup
      \def\childdoctmp{#1}
      \ifx\childdoctmp\childdocname
        \def\childdoctmp{}
      \else
        \def\childdoctmp
        {
          \childdoctrue
          \includeonly{\childdocname}
          \def\childdocjob{#1}
          \def\jobname{#1}
        }
      \fi
      \expandafter
    \endgroup
    \childdoctmp
  \fi
}
%    \end{macrocode}

% \macro{\childdocof}
% The command |\childdocof| redirects
% compilation to the main file |#1|.
%    \begin{macrocode}
\newcommand{\childdocof}[1]
{
  \childdocdisable
  \childdoctrue
  \includeonly{\childdocname}
  \def\jobname{#1}
  \def\childdocjob{#1}
  \input{#1}
}
%    \end{macrocode}

% \macro{\childdocby}
% The command |\childdocby| ....
%    \begin{macrocode}
\newcommand{\childdocby}[2][]
{
  \childdocdisable
  \childdoctrue
  \childdocmanualtrue
  \if?#1?\else
    \def\jobname{#2}
  \fi
  \def\childdocjob{#2}
  \input{#2}
  \endinput
}
%    \end{macrocode}

% \macro{\childdocforward}
% The command |\childdocforward| redirects
% compilation to the main file or
% (if the optional argument is given) a child file.
% Parameters are set as if the main file
% or a child file starting with |\childdocof| was compiled.
% Then compilation is handed over to the main file:
%    \begin{macrocode}
\newcommand{\childdocforward}[2][]
{
  \begingroup
    \if?#1?
      \def\childdoctmp
      {
        \def\childdocname{#2}
        \def\childdocjob{#2}
        \def\jobname{#2}
        \input{#2}
        \endinput
      }
    \else
      \def\childdoctmp
      {
        \childdocdisable
        \def\childdocname{#2}
        \childdoctrue
        \includeonly{#2}
        \def\childdocjob{#1}
        \def\jobname{#1}
        \input{#1}
        \endinput
      }
    \fi
    \expandafter
  \endgroup
  \childdoctmp
}
%    \end{macrocode}

% \macro{\childdocforwardprefix}
% The command |\childdocforwardprefix| redirects
% compilation to the main or a child file by means of a pattern.
% The prefix |#1| in the current filename is replaced by |#2|
% and the suffix of the current filename is kept
% (it is assumed that the filename does not contain the substring `|~~~|'
% which is used as a delimiter).
% Compilation is handed over to the new file by |\childdocforward|:
%    \begin{macrocode}
\newcommand{\childdocforwardprefix}[3][]
{
  \begingroup
    \def\childdocextract #2##1~~~{\def\childdoctmp{\childdocforward[#1]{#3##1}}}
    \expandafter\childdocextract\childdocname~~~
    \expandafter
  \endgroup
  \childdoctmp
}
%    \end{macrocode}

% \macro{\childdoc}
% The deprecated macro |\childdoc| is a legacy version of |\childdocmain|:
%    \begin{macrocode}
\newcommand{\childdoc}{\childdocmain}
%    \end{macrocode}

% \macro{\childdocredirect}
% The deprecated macro |\childdocredirect| is a legacy version
% of |\childdocforward| and |\childdocforwardprefix|:
%    \begin{macrocode}
\newcommand{\childdocredirect}[2][]
{
  \begingroup
    \if?#1?
      \def\childdoctmp{\childdocforward{#2}}
    \else
      \def\childdoctmp{\childdocforwardprefix{#1}{#2}}
    \fi
    \expandafter
  \endgroup
  \childdoctmp
}
%    \end{macrocode}

%\iffalse
%</package>
%\fi
%
\endinput

\childdocmain{}
%    \end{macrocode}

% Optional override for |\version| flag:
%    \begin{macrocode}
%%\ifchilddoc\else\providecommand{\version}{draft}\fi
%    \end{macrocode}

% Define the default values for the |\version| flag
% (|final| for the main file and |draft| for childs):
%    \begin{macrocode}
\ifchilddoc
\providecommand{\version}{draft}
\else
\providecommand{\version}{final}
\fi
%    \end{macrocode}

% Load the standard document class:
%    \begin{macrocode}
\documentclass[12pt]{article}
%    \end{macrocode}

% Start the document body:
%    \begin{macrocode}
\begin{document}
%    \end{macrocode}

% Declare a title page.
% Print title, part of document being processed and version flag:
%    \begin{macrocode}
\addtocounter{page}{-1}
\begin{center}
{\LARGE\bfseries{}childdoc example\par}
\vspace{1cm}
\ifchilddoc
\ifchilddocmanual part\else chapter\fi:
`\childdocname' of `\childdocjob'\par
\else
main document: `\childdocjob'\par
\fi
version: \version\par
\end{center}
\newpage
%    \end{macrocode}

% Manually include selected file,
% otherwise process as usual:
%    \begin{macrocode}
\ifchilddocmanual
\section*{part `\childdocname'}
\input{\childdocname}
\else
%    \end{macrocode}

% Include the two chapters:
%    \begin{macrocode}
\include{cdocsch1}
\include{cdocsch2}
%    \end{macrocode}

% Include the two parts unless only chapters should be displayed:
%    \begin{macrocode}
\ifchilddoc\else
\section{part three}
\input{cdocspt3}
\section{part four}
\input{cdocspt4}
\fi
%    \end{macrocode}

% Process as usual until here:
%    \begin{macrocode}
\fi
%    \end{macrocode}

% End of document body:
%    \begin{macrocode}
\end{document}
%    \end{macrocode}
%\iffalse
%</samplemain>
%\fi
%
% %%%%%%%%%%%%%%%%%%%%%%%%%%%%%%%%%%%%%%
% \paragraph{Chapter Include Files.}
%
% The include files are called |cdocsch1.tex| and |cdocsch2.tex|.
%
%\iffalse
%<*samplechap1|samplechap2>
%\fi

% Optional override for |\version| flag:
%    \begin{macrocode}
%%\providecommand{\version}{final}
%    \end{macrocode}

% Include the main document:
%    \begin{macrocode}
% \iffalse
%
% childdoc.dtx Copyright (C) 2017-2018 Niklas Beisert
%
% This work may be distributed and/or modified under the
% conditions of the LaTeX Project Public License, either version 1.3
% of this license or (at your option) any later version.
% The latest version of this license is in
%   http://www.latex-project.org/lppl.txt
% and version 1.3 or later is part of all distributions of LaTeX
% version 2005/12/01 or later.
%
% This work has the LPPL maintenance status `maintained'.
%
% The Current Maintainer of this work is Niklas Beisert.
%
% This work consists of the files childdoc.dtx and childdoc.ins
% and the derived files childdoc.def and cdocsamp.tex with
% cdocsch1.tex, cdocsch2.tex, cdocsdrf.tex, cdocsfn1.tex, cdocsfn2.tex.
%
%<package>\ifdefined\childdocmain\endinput\fi
%<package>\ProvidesFile{childdoc.def}[2018/12/30 v2.0 child document driver]
%<samplemain>\ProvidesFile{cdocsamp.tex}[2018/12/30 v2.0 sample for childdoc]
%<*driver>
%\ProvidesFile{childdoc.drv}[2018/12/30 v2.0 childdoc reference manual file]
\PassOptionsToClass{10pt,a4paper}{article}
\documentclass{ltxdoc}

\usepackage[margin=35mm]{geometry}
\usepackage{hyperref}
\usepackage{hyperxmp}
\usepackage[usenames]{color}

\hypersetup{colorlinks=true}
\hypersetup{pdfstartview=FitH}
\hypersetup{pdfpagemode=UseNone}
\hypersetup{pdfsource={}}
\hypersetup{pdflang={en-UK}}
\hypersetup{pdfcopyright={Copyright 2017-2018 Niklas Beisert.
  This work may be distributed and/or modified under the
  conditions of the LaTeX Project Public License, either version 1.3
  of this license or (at your option) any later version.}}
\hypersetup{pdflicenseurl={http://www.latex-project.org/lppl.txt}}
\hypersetup{pdfcontactaddress={ETH Zurich, ITP, HIT K,
  Wolfgang-Pauli-Strasse 27}}
\hypersetup{pdfcontactpostcode={8093}}
\hypersetup{pdfcontactcity={Zurich}}
\hypersetup{pdfcontactcountry={Switzerland}}
\hypersetup{pdfcontactemail={nbeisert@itp.phys.ethz.ch}}
\hypersetup{pdfcontacturl={http://people.phys.ethz.ch/\xmptilde nbeisert/}}

\newcommand{\secref}[1]{\hyperref[#1]{section \ref*{#1}}}

\parskip1ex
\parindent0pt
\let\olditemize\itemize
\def\itemize{\olditemize\parskip0pt}

\begin{document}

\title{The \textsf{childdoc} Package}
\hypersetup{pdftitle={The childdoc Package}}
\author{Niklas Beisert\\[2ex]
  Institut f\"ur Theoretische Physik\\
  Eidgen\"ossische Technische Hochschule Z\"urich\\
  Wolfgang-Pauli-Strasse 27, 8093 Z\"urich, Switzerland\\[1ex]
  \href{mailto:nbeisert@itp.phys.ethz.ch}
  {\texttt{nbeisert@itp.phys.ethz.ch}}}
\hypersetup{pdfauthor={Niklas Beisert}}
\hypersetup{pdfsubject={Manual for the LaTeX2e Package childdoc}}
\date{30 December 2018, \textsf{v2.0}}
\maketitle

\begin{abstract}\noindent
\textsf{childdoc} is a \LaTeXe{} package
that enables the direct compilation
of document sections included by |\include|
to individual files.
\end{abstract}

\begingroup
\parskip0ex
\tableofcontents
\endgroup

%%%%%%%%%%%%%%%%%%%%%%%%%%%%%%%%%%%%%%%%%%%%%%%%%%%%%%%%%%%%%%%%%%%%%%%%%%%%%%%%
%%%%%%%%%%%%%%%%%%%%%%%%%%%%%%%%%%%%%%%%%%%%%%%%%%%%%%%%%%%%%%%%%%%%%%%%%%%%%%%%
\section{Introduction}

\LaTeX{} provides a mechanism to structure a large document (such as a book)
into a main file and several child files (containing the chapters)
using the |\include| command.
This mechanism is beneficial for documents
which span hundreds of pages in order to
make the source file(s) more manageable.
Moreover, compilation can be restricted to
selected child files by means of the |\includeonly| command.
The latter feature can be used to reduce the compilation time while editing
(this was significantly more useful in the earlier days of \LaTeX{})
or to generate a smaller document which is easier to navigate.
Another application of |\includeonly| is to generate
documents consisting of selected parts of the complete document.

However, there are a few drawbacks of the plain |\include| mechanism:
\begin{itemize}
\item
The child files cannot be compiled on their own,
they can only be compiled via the main file.
A naive editing environment
(such as a text editor with an option
to have the current file processed by \LaTeX)
may require one to switch to the main file before compiling;
attempting to compile the child file produces errors.
\item
The main file must be modified (each time)
to adjust the |\includeonly| command
to the present needs. This easily leaves the main file in a messy state.
\item
The generated document will always carry the filename
of the main document. This is inconvenient if
several child files are to be compiled and
to be kept for distribution.
\end{itemize}

The present package provides a simple interface
to make child files individually compilable by \LaTeX{}.
Compiling a child file then has the same effect as compiling
the main file with an |\includeonly| command
to select the appropriate child.
Moreover the generated document will carry the name of the child
rather than the main file.
This resolves all three above issues.

This feature is meant to make the editing of books,
thesis documents and lecture notes somewhat more convenient.
However, the package can also be used efficiently for
composing a series of documents (such as exercise sheets)
which are typically distributed individually.
It then assists the author in generating the individual documents
(potentially in different versions)
as well as a document containing the collected series.
Another application is in developing style files
or other kinds of included material
where compilation of the style file could redirect
to a sample or test file.

%%%%%%%%%%%%%%%%%%%%%%%%%%%%%%%%%%%%%%%%%%%%%%%%%%%%%%%%%%%%%%%%%%%%%%%%%%%%%%%%
%%%%%%%%%%%%%%%%%%%%%%%%%%%%%%%%%%%%%%%%%%%%%%%%%%%%%%%%%%%%%%%%%%%%%%%%%%%%%%%%
\section{Usage}

First of all, the package \textsf{childdoc} is \emph{not} a standard
\LaTeXe{} |.sty| style file! Therefore it needs to be invoked in
a non-standard way.

%%%%%%%%%%%%%%%%%%%%%%%%%%%%%%%%%%%%%%%%%%%%%%%%%%%%%%%%%%%%%%%%%%%%%%%%%%%%%%%%
\subsection{Included Files}
\label{sec:include}

%%%%%%%%%%%%%%%%%%%%%%%%%%%%%%%%%%%%%%%%
\DescribeMacro{\childdocmain}
To use the package, add the commands
\begin{center}
\begin{tabular}{l}
|\input{childdoc.def}|\\
|\childdocmain{}|\\
\end{tabular}
\end{center}
at the very top of the main \LaTeX{} file,
in particular \emph{before} the |\documentclass| statement!
The argument of |\childdocmain| should be left empty
(but it must be present).

%%%%%%%%%%%%%%%%%%%%%%%%%%%%%%%%%%%%%%%%
\DescribeMacro{\childdocof}
Furthermore, add the commands
\begin{center}
\begin{tabular}{l}
|\input{childdoc.def}|\\
|\childdocof{|\textit{main}|}|\\
\end{tabular}
\end{center}
at the top of every child file \textit{child}
which is included by |\include{|\textit{child}|}|
from within the main file
(or at least for those files to be compiled individually).
The argument \textit{main} must be the filename of the main file.

There are a couple of
considerations in setting up the main and child documents:

%%%%%%%%%%%%%%%%%%%%%%%%%%%%%%%%%%%%%%%%
\paragraph{Restrictions.}

Please note the following restrictions:
\begin{itemize}
\item
|\childdocmain| must be called with one argument \textit{main}
to ensure compatibility with earlier version of the package.
It must either be empty (|\childdocmain{}|)
or precisely match the filename of the main file in which it is specified.
See \secref{sec:detection} for further information.
\item
The filename \textit{main} must be specified without the |.tex| extension.
\item
The filename \textit{main} is case sensitive
(even in case-insensitive file systems)
due to internal string comparison.
\item
The argument \textit{main} should be fully expanded, it cannot be a macro.
\item
Subdirectories and special characters should be avoided in filenames.
\item
The command |\childdocmain{|\textit{main}|}| must be followed by a whitespace.
It should not be followed immediately by another command
or by a comment mark `|%|'.
This is because the \TeX{} parser reads the token immediately following
the argument of |\childdocmain| and puts it
at the beginning of every child section;
however, a white\-space is ignored.
\end{itemize}

%%%%%%%%%%%%%%%%%%%%%%%%%%%%%%%%%%%%%%%%
\paragraph{Content of Main File.}

It is advisable to place all content in the child files included by |\include|.
Any output contained in the main file will appear in all child documents
unless suppressed manually;
it cannot be suppressed automatically by the |\includeonly| directive
and thus should normally be avoided.
A method to include some content in the main file
by means of conditional processing is described in \secref{sec:conditional}.

%%%%%%%%%%%%%%%%%%%%%%%%%%%%%%%%%%%%%%%%
\paragraph{Page Numbering.}

When only a part of the document is compiled,
the appropriate numbering of pages
(as well as other status parameters)
is determined from the |.aux| files.
The latter contain information from previous passes.
However this information needs to propagate through
all intermediate child documents.
Therefore the page numbering in child documents may well
be inconsistent until the complete document is compiled at least once.

A useful (if unconventional) way to always ensure a consistent
page numbering is to restart the numbering in each child document
and denote the pages by `\textit{child}|.|\textit{page}'
where \textit{child} represents the chapter/section number of the child file.
This can be achieved by the command
|\numberwithin{page}{|\textit{child}|}|
of the \textsf{amsmath} package
where \textit{child} can be |chapter| or |section|
depending on the chosen structuring.
Alternatively, one can modify the macro |\thepage| appropriately
and reset the counter |page| at the start of each child file.

%%%%%%%%%%%%%%%%%%%%%%%%%%%%%%%%%%%%%%%%%%%%%%%%%%%%%%%%%%%%%%%%%%%%%%%%%%%%%%%%
\subsection{Conditional Processing}
\label{sec:conditional}

The package provides a mechanism to compile different versions
of a document. To customise the versions further some conditional processing
can come in handy to distinguish which version is being compiled.
The package provides two macros to describe the compilation context:

%%%%%%%%%%%%%%%%%%%%%%%%%%%%%%%%%%%%%%%%
\DescribeMacro{\ifchilddoc}
The conditional |\ifchilddoc| distinguishes between the compilation of
child documents and the main document:
%
\begin{center}
|\ifchilddoc |\textit{child-code}| |[|\||else |\textit{main-code}]| \||fi|
\end{center}

%%%%%%%%%%%%%%%%%%%%%%%%%%%%%%%%%%%%%%%%
\DescribeMacro{\childdocname}
\DescribeMacro{\childdocjob}
The macro |\childdocname| contains the filename (without extension)
of the main or child file being processed.
Note that |\childdocjob| will always contain the name of the main file.

%%%%%%%%%%%%%%%%%%%%%%%%%%%%%%%%%%%%%%%%
\paragraph{Title Page.}

Conditional processing can be used to include a title or banner page
in the main document when proper precautions are taken.
Importantly, the code in the main file should ensure that the page counter
(as well as other status parameters which are stored in the |.aux| files)
takes the same value after the conditional processing.
Otherwise the page numbers may take divergent values
depending on which part is compiled.

For example, a title page could be declared by:
%
\begin{center}
\begin{tabular}{l}
|\ifchilddoc\||else|\\
|\addtocounter{page}{-1}|\\
\textit{code for title page}\\
|\newpage|\\
|\||fi|
\end{tabular}
\end{center}
%
A banner page for the child documents can be generated by:
%
\begin{center}
\begin{tabular}{l}
|\ifchilddoc|\\
|\addtocounter{page}{-1}|\\
\textit{code for banner page}\\
|\newpage|\\
|\||fi|
\end{tabular}
\end{center}
%
Here one could write a message such as:
\begin{center}
|This is the part \childdocname{} of \childdocjob{}.|
\end{center}

%%%%%%%%%%%%%%%%%%%%%%%%%%%%%%%%%%%%%%%%%%%%%%%%%%%%%%%%%%%%%%%%%%%%%%%%%%%%%%%%
\subsection{Flags}
\label{sec:flags}

The package makes it easy to generate different versions
of the main or child documents.
To this end compilation flags can be defined
and assigned different default values.
They will be particularly useful in conjunction
with the forwarding mechanism described in \secref{sec:forward}.

For example, it may be useful to have a flag |\version|
which can be set to |draft| or |final|.
The document source will contain some conditional code
depending on the value of |\version|.
Suppose further, the flag should default to |final| for the main file
and to |draft| for child files
which is a natural assignment for editing the document.
This is achieved by placing the following code
in the preamble of the main document
(below the |\childdocmain| directive):
%
\begin{center}
\begin{tabular}{l}
|\ifchilddoc|\\
|\providecommand{\version}{draft}|\\
|\||else|\\
|\providecommand{\version}{final}|\\
|\||fi|
\end{tabular}
\end{center}
%
The definition by |\providecommand| makes sure
that previous definitions are not overwritten.
Further statements |\providecommand{\version}{...}|
can thus be added before the above code to override it.

For the main file, one might add a line
(between |\childdocmain| and the above block)
%
\begin{center}
|%\ifchilddoc\||else\providecommand{\version}{draft}\||fi|
\end{center}
%
which can be uncommented to produce a draft version.
Likewise one can add a line to the very top of a child file
(above the |\childdocof{|\textit{main}|}| directive)
%
\begin{center}
|%\providecommand{\version}{final}|
\end{center}
%
which can be uncommented to produce the final version of this child document.

%%%%%%%%%%%%%%%%%%%%%%%%%%%%%%%%%%%%%%%%%%%%%%%%%%%%%%%%%%%%%%%%%%%%%%%%%%%%%%%%
\subsection{Forwarding}
\label{sec:forward}

Different versions of the main or child documents
using compilation flags as described in \secref{sec:flags}
can be (permanently) stored in different files
for convenient compilation, viewing and distribution.
To this end, the package defines a command
to pass on compilation to a different file:

%%%%%%%%%%%%%%%%%%%%%%%%%%%%%%%%%%%%%%%%
\DescribeMacro{\childdocforward}
The command |\childdocforward| redirects processing to
another source file:
%
\begin{center}
\begin{tabular}{l}
|\input{childdoc.def}|\\
|\childdocforward[|\textit{main}|]{|\textit{dest}|}|\\
\end{tabular}
\end{center}
%
The argument \textit{dest} is the destination file
(without extension).
It should be the main file or one of the child files.
Note that further \textsf{childdoc} directives
such as |\childdocof| and |\childdocforward|
in the indicated file will be processed in this form.
The optional argument \textit{main}
passes on directly to the main file \textit{main}
while pretending to compile the child \textit{dest}.
This form behaves as if \textit{dest}
issues |\childdocof{|\textit{main}|}| right away,
and no further \textsf{childdoc} directives will be processed.

%%%%%%%%%%%%%%%%%%%%%%%%%%%%%%%%%%%%%%%%
\DescribeMacro{\...prefix}
In the alternative form |\childdocforwardprefix|,
%
\begin{center}
\begin{tabular}{l}
|\input{childdoc.def}|\\
|\childdocforwardprefix[|\textit{main}|]{|\textit{prefix}|}{|\textit{dest}|}|
\end{tabular}
\end{center}
%
the destination file is determined by a pattern
depending on the current file:
To make this work, the current file must be called
`{\textit{prefix}\hspace{0.2em}\textit{suffix}}'
with \textit{prefix} matching precisely the argument.
Processing is then passed on to the file
`{\textit{dest}\hspace{0.2em}\textit{suffix}}'.
Surely, the same effect is achieved by
directly specifying the
argument `{\textit{dest}\hspace{0.2em}\textit{suffix}}'
in the first form.
However, that requires to set up a different file
for each child. With the alternative form of the command
all these files can have exactly the same content
which simplifies setting them up and maintaining them.

For example, the following file |draft.tex|
with a compilation flag |\version| as described in \secref{sec:flags}
compiles the main document as a draft:
%
\begin{center}
\begin{tabular}{l}
|\def\version{draft}|\\
|\input{childdoc.def}|\\
|\childdocforward{|\textit{main}|}|
\end{tabular}
\end{center}
%
Likewise, the following files |final|\textit{nn}|.tex|
compile the final version of the child document
|child|\textit{nn}|.tex|:
%
\begin{center}
\begin{tabular}{l}
|\def\version{final}|\\
|\input{childdoc.def}|\\
|\childdocforwardprefix{final}{child}|
\end{tabular}
\end{center}
%

Note that when several versions of a main file and/or of each child file
are to be generated, it may be convenient to set up a |Makefile| or
shell script to automatise the process.

%%%%%%%%%%%%%%%%%%%%%%%%%%%%%%%%%%%%%%%%%%%%%%%%%%%%%%%%%%%%%%%%%%%%%%%%%%%%%%%%
\subsection{Command Line Processing}
\label{sec:commandline}

The effect of redirection files can also be achieved by invoking
the \LaTeX{} compiler with a more elaborate command line.
Most conveniently this should be done as part
of a shell script or a |Makefile|.

When using \textsf{childdoc} in the main file, the following
command lines effectively perform a redirection
(note that depending on the shell being used,
backslashes may have to be doubled: `|\|' $\to$ `|\\|'):
%
\begin{center}
|... -jobname "|\textit{target}|" |\\|"|[\textit{flags}]%
|\input{childdoc.def}\childdocforward[|\textit{main}|]{|\textit{dest}|}"|
\end{center}
%
Here \textit{target} is the name of the output file,
\textit{main} is the name of the main file
and \textit{dest} is the name of the main or child file to be processed
(all filenames without extensions).
The optional argument \textit{main} can be omitted
if \textit{main} matches \textit{dest}.
Optionally, compilation \textit{flags} can be defined via |\def| commands.
This command line makes the \TeX{} engine believe
it is compiling the file \textit{target}
whose content is specified as the latter parameter.
The provided code then forwards the processing to
\textit{main} or \textit{dest} as described in \secref{sec:forward}.

%%%%%%%%%%%%%%%%%%%%%%%%%%%%%%%%%%%%%%%%%%%%%%%%%%%%%%%%%%%%%%%%%%%%%%%%%%%%%%%%
\subsection{Include by Input}
\label{sec:input}

Including child documents by |\include| has some restrictions by design.
Most notably, the content of a child document always occupies
its own set of pages; pages cannot be shared between child documents.
Usually, this behaviour makes perfect sense
because each child document contain an essential part of the document.
However, in some situations it may be desirable to compose
a document from a collection of parts
without having mandatory page breaks between then.
For this case, the package
provides a mechanism to include parts
by |\input| which can also be processed individually.
However, by construction this mechanism
requires manual handling of the content to be output.

%%%%%%%%%%%%%%%%%%%%%%%%%%%%%%%%%%%%%%%%
\DescribeMacro{\ifchilddocmanual}
The main file should be prepared as usual, see \secref{sec:include}.
However, the document body must make a distinction
between processing of an individual part and of the main document, e.g.:
%
\begin{center}
\begin{tabular}{l}
|\ifchilddocmanual|\\
|\input{\childdocname}|\\
|\||else|\\
\textit{document body with }|\input{|\textit{part}|}|\\
|\||fi|
\end{tabular}
\end{center}
%
The conditional |\ifchilddocmanual| is true whenever
a part to be included by |\input| is being compiled,
and the name of the part is stored in |\childdocname|.

%%%%%%%%%%%%%%%%%%%%%%%%%%%%%%%%%%%%%%%%
\DescribeMacro{\childdocby}
Each part to be included by |\input| should start with:
%
\begin{center}
\begin{tabular}{l}
|\input{childdoc.def}|\\
|\childdocby{|\textit{main}|}|\\
\end{tabular}
\end{center}
%
The directive |\childdocby| is similar to |\childdocof|
described in \secref{sec:include},
but the subsequent selection of content must be done manually.
To that end, both |\ifchilddoc| and |\ifchilddocmanual|
will be true upon processing of a part,
and the name of the part is stored in |\childdocname|.
Note that |\jobname| will be set to the filename of the current part
so that each part receives an individual |.aux| file
that does not interfere with the |.aux| file(s) of the main document.
This behaviour can be altered by the alternative form
|\childdocby[*]{|\textit{main}|}| (with a non-empty optional argument)
which uses the |.aux| file of the main document
by setting |\jobname| to \textit{main}.

%%%%%%%%%%%%%%%%%%%%%%%%%%%%%%%%%%%%%%%%%%%%%%%%%%%%%%%%%%%%%%%%%%%%%%%%%%%%%%%%
\subsection{Driver Development}
\label{sec:driver}

The \textsf{childdoc} mechanism can also be use for the development
of definition files such as \LaTeX{} styles or classes.
This case differs from the above setup with multiple parts
included by |\include| in that no |\includeonly| should be invoked.
This can be achieved by starting the include file
(before |\ProvidesPackage|) with:
%
\begin{center}
\begin{tabular}{l}
|\input{childdoc.def}|\\
|\childdocforward{|\textit{main}|}|\\
\end{tabular}
\end{center}
%
or alternatively with:
%
\begin{center}
\begin{tabular}{l}
|\input{childdoc.def}|\\
|\childdocby{|\textit{main}|}|\\
\end{tabular}
\end{center}
%
Both forms have slightly different effects as described above.
The main file is prepared as usual, see \secref{sec:include}.

%%%%%%%%%%%%%%%%%%%%%%%%%%%%%%%%%%%%%%%%%%%%%%%%%%%%%%%%%%%%%%%%%%%%%%%%%%%%%%%%
\subsection{Legacy Detection}
\label{sec:detection}

The directive |\childdocmain| in the main file can detect
whether the complete document or merely a child is to be compiled
even without using the directive |\childdocof|.
This method is deprecated because it is less robust
and there is no compelling reason to use it;
it is merely provided for backward compatibility
and it may be removed in future versions.

If the detection mechanism is to be used,
it is mandatory to correctly specify
the filename of the main file as the argument of |\childdocmain|:
%
\begin{center}
\begin{tabular}{l}
|\input{childdoc.def}|\\
|\childdocmain{|\textit{main}|}|\\
\end{tabular}
\end{center}
%
If |\jobname| does not match the argument \textit{main} of |\childdocmain|,
it is assumed that |\jobname| points to the child file to be compiled.
When using |\childdocmain| with the main file specified as argument,
it suffices to start a child file
with just |\input{|\textit{main}|}|
without loading of the package and using |\childdocof|.
If instead all processing is done
with the appropriate \textsf{childdoc} directives,
the argument of \textit{main} of |\childdocmain| can be empty.

An alternative version of the command line processing described
in \secref{sec:commandline} using the detection mechanism reads:
%
\begin{center}
|... -jobname "|\textit{target}|" "|[\textit{flags}]%
[|\def\jobname{|\textit{dest}|}|]|\input{|\textit{main}|}"|
\end{center}

%%%%%%%%%%%%%%%%%%%%%%%%%%%%%%%%%%%%%%%%%%%%%%%%%%%%%%%%%%%%%%%%%%%%%%%%%%%%%%%%
\subsection{Manual Code}
\label{sec:manual}

In case one cannot be certain whether the definitions file |childdoc.def|
is installed on the target \TeX{} distribution
and one prefers not to ship it,
it is conceivable to paste a few relevant commands into the sources.

To that end, drop all statements |\input{childdoc.def}|
and perform the replacements as outlined below.
Instead of |\childdocmain{|\textit{main}|}| add the following code
to the top of the main file:
%
\begin{center}
\begin{tabular}{l}
|\||ifdefined\childdocname\endinput\||fi\newif\ifchilddoc|\\
|\edef\childdocname{\scantokens\expandafter{\jobname\noexpand}}|\\
|\def\childdocmain{|\textit{main}|}\||ifx\childdocmain\childdocname\||else|\\
|\childdoctrue\includeonly{\childdocname}\let\jobname\childdocmain\||fi|\\
\end{tabular}
\end{center}
%
Instead of |\childdocof{|\textit{main}|}| just include the main file
at the top of each child file:
%
\begin{center}
|\input{|\textit{main}|}|
\end{center}
%
A simple redirection |\childdocforward{|\textit{dest}|}| is achieved by:
%
\begin{center}
|\def\jobname{|\textit{dest}|}\input{\jobname}|
\end{center}
%
The redirection with prefix
|\childdocforwardprefix[|\textit{prefix}|]{|\textit{dest}|}|
is accomplished by:
%
\begin{center}
\begin{tabular}{l}
|{\edef\jobname{\scantokens\expandafter{\jobname\noexpand}}|\\
|\def\redirectjob |\textit{prefix}|#1~~~{\gdef\jobname{|\textit{dest}|#1}}|\\
|\expandafter\redirectjob\jobname~~~}\input{\jobname}|
\end{tabular}
\end{center}

In an alternative approach,
child documents can be compiled by a specific command line
without additional code or specific definitions:
%
\begin{center}
|... -jobname "|\textit{target}|" "|[\textit{flags}]%
|\includeonly{|\textit{dest}|}\input{|\textit{main}|}"|
\end{center}
%

%%%%%%%%%%%%%%%%%%%%%%%%%%%%%%%%%%%%%%%%%%%%%%%%%%%%%%%%%%%%%%%%%%%%%%%%%%%%%%%%
%%%%%%%%%%%%%%%%%%%%%%%%%%%%%%%%%%%%%%%%%%%%%%%%%%%%%%%%%%%%%%%%%%%%%%%%%%%%%%%%
\section{Information}

%%%%%%%%%%%%%%%%%%%%%%%%%%%%%%%%%%%%%%%%%%%%%%%%%%%%%%%%%%%%%%%%%%%%%%%%%%%%%%%%
\subsection{Copyright}

Copyright \copyright{} 2017--2018 Niklas Beisert

This work may be distributed and/or modified under the
conditions of the \LaTeX{} Project Public License, either version 1.3
of this license or (at your option) any later version.
The latest version of this license is in
  \url{http://www.latex-project.org/lppl.txt}
and version 1.3 or later is part of all distributions of \LaTeX{}
version 2005/12/01 or later.

This work has the LPPL maintenance status `maintained'.

The Current Maintainer of this work is Niklas Beisert.

This work consists of the files |README.txt|, |childdoc.ins| and |childdoc.dtx|
as well as the derived files |childdoc.def|, |cdocsamp.tex|
with |cdocsch1.tex|, |cdocsch2.tex|, |cdocspt3.tex|, |cdocspt4.tex|,
|cdocsdrf.tex|, |cdocsfn1.tex|, |cdocsfn2.tex|
as well as |childdoc.pdf|.

%%%%%%%%%%%%%%%%%%%%%%%%%%%%%%%%%%%%%%%%%%%%%%%%%%%%%%%%%%%%%%%%%%%%%%%%%%%%%%%%
\subsection{Files and Installation}

The package consists of the files:
%
\begin{center}
\begin{tabular}{ll}
    |README.txt|   & readme file \\
    |childdoc.ins| & installation file \\
    |childdoc.dtx| & source file \\
    |childdoc.def| & definition file \\
    |cdocsamp.tex| & sample main file \\
    |cdocsch1.tex| & sample include file \\
    |cdocsch2.tex| & sample include file \\
    |cdocspt3.tex| & sample part file \\
    |cdocspt4.tex| & sample part file \\
    |cdocsdrf.tex| & sample redirection file \\
    |cdocsfn1.tex| & sample redirection file \\
    |cdocsfn2.tex| & sample redirection file \\
    |childdoc.pdf| & manual
\end{tabular}
\end{center}
%
The distribution consists of the files
|README.txt|, |childdoc.ins| and |childdoc.dtx|.
%
\begin{itemize}
\item
Run (pdf)\LaTeX{} on |childdoc.dtx|
to compile the manual |childdoc.pdf| (this file).
\item
Run \LaTeX{} on |childdoc.ins| to create the definitions file |childdoc.def|
and the sample |cdocsamp.tex| with include files
|cdocsch1.tex|, |cdocsch2.tex|, |cdocspt3.tex|, |cdocspt4.tex|,
|cdocsdrf.tex|, |cdocsfn1.tex|, |cdocsfn2.tex|.
Then copy the file |childdoc.def| to an appropriate directory of your \LaTeX{}
distribution, e.g.\ \textit{texmf-root}|/tex/latex/childdoc|.
\end{itemize}

%%%%%%%%%%%%%%%%%%%%%%%%%%%%%%%%%%%%%%%%%%%%%%%%%%%%%%%%%%%%%%%%%%%%%%%%%%%%%%%%
\subsection{Related CTAN Packages}

There are several other packages which offer a similar functionality:
%
\begin{itemize}
\item
The packages
\href{http://ctan.org/pkg/docmute}{\textsf{docmute}},
\href{http://ctan.org/pkg/includex}{\textsf{includex}} and
\href{http://ctan.org/pkg/standalone}{\textsf{standalone}}
provide commands to include only the document body of
a child file thus allowing both files to be compiled individually.
\item
The packages \href{http://ctan.org/pkg/subdocs}{\textsf{subdocs}}
and \href{http://ctan.org/pkg/subfiles}{\textsf{subfiles}}
provide structures in which the main and child documents can be
encapsulated and allowing them to be compiled individually.
The inclusion mechanism is different from the conventional |\include|.
\item
The package \href{http://ctan.org/pkg/combine}{\textsf{combine}}
is an elaborate solution to combine several documents into one.
\end{itemize}
%
See also the CTAN topic \href{http://ctan.org/topic/subdocs}{\textsf{subdocs}}
for further related packages.
The present package differs from the above solutions in that
a document structure constructed with the conventional |\include| mechanism
just needs two extra commands at the top of every file
such that all constituent files can be compiled individually.

%%%%%%%%%%%%%%%%%%%%%%%%%%%%%%%%%%%%%%%%%%%%%%%%%%%%%%%%%%%%%%%%%%%%%%%%%%%%%%%%
%\subsection{Feature Suggestions}
%
%The following is a list of features which may be useful for future
%versions of this package:
%%
%\begin{itemize}
%\item
%\ldots
%\end{itemize}

%%%%%%%%%%%%%%%%%%%%%%%%%%%%%%%%%%%%%%%%%%%%%%%%%%%%%%%%%%%%%%%%%%%%%%%%%%%%%%%%
\subsection{Revision History}

%%%%%%%%%%%%%%%%%%%%%%%%%%%%%%%%%%%%%%%%
\paragraph{v2.0:} 2018/12/30

\begin{itemize}
\item
immediate forward processing
\item
added |\childdocby| mechanism
\item
manual restructured
\end{itemize}

%%%%%%%%%%%%%%%%%%%%%%%%%%%%%%%%%%%%%%%%
\paragraph{v1.6:} 2018/01/17

\begin{itemize}
\item
application for development of include files
\item
corrections to manual
\end{itemize}

%%%%%%%%%%%%%%%%%%%%%%%%%%%%%%%%%%%%%%%%
\paragraph{v1.5:} 2017/05/21

\begin{itemize}
\item
more complete structuring introduced
\item
|\childdocof| introduced
\item
|\childdoc| renamed to |\childdocmain|
\item
|\childredirect| renamed to |\childdocforward| and |\childdocforwardprefix|
and functionality expanded
\end{itemize}

%%%%%%%%%%%%%%%%%%%%%%%%%%%%%%%%%%%%%%%%
\paragraph{v1.0:} 2017/04/27

\begin{itemize}
\item
manual and install package
\item
first version published on CTAN
\end{itemize}

%%%%%%%%%%%%%%%%%%%%%%%%%%%%%%%%%%%%%%%%
\paragraph{v0.6:} 2017/04/26

\begin{itemize}
\item
redirection mechanism added
\end{itemize}

%%%%%%%%%%%%%%%%%%%%%%%%%%%%%%%%%%%%%%%%
\paragraph{v0.5:} 2017/04/26

\begin{itemize}
\item
functionality in definition file
\end{itemize}


%%%%%%%%%%%%%%%%%%%%%%%%%%%%%%%%%%%%%%%%%%%%%%%%%%%%%%%%%%%%%%%%%%%%%%%%%%%%%%%%
%%%%%%%%%%%%%%%%%%%%%%%%%%%%%%%%%%%%%%%%%%%%%%%%%%%%%%%%%%%%%%%%%%%%%%%%%%%%%%%%
%%%%%%%%%%%%%%%%%%%%%%%%%%%%%%%%%%%%%%%%%%%%%%%%%%%%%%%%%%%%%%%%%%%%%%%%%%%%%%%%
\appendix

\settowidth\MacroIndent{\rmfamily\scriptsize 000\ }

 \DocInput{childdoc.dtx}

\end{document}
%</driver>
% \fi
%
% %%%%%%%%%%%%%%%%%%%%%%%%%%%%%%%%%%%%%%%%%%%%%%%%%%%%%%%%%%%%%%%%%%%%%%%%%%%%%%
% %%%%%%%%%%%%%%%%%%%%%%%%%%%%%%%%%%%%%%%%%%%%%%%%%%%%%%%%%%%%%%%%%%%%%%%%%%%%%%
% \section{Sample}
%\iffalse
%<*samplemain>
%\fi
%
% The following presents a sample document
% with two chapters, two parts, a title page,
% a compile flag as well as three forwarding files to set the flag.
% It consists of eight |.tex| files:
% \begin{center}
% \begin{tabular}{ll}
% |cdocsamp.tex|&main file\\
% |cdocsch1.tex|&include file for chapter 1\\
% |cdocsch2.tex|&include file for chapter 2\\
% |cdocspt3.tex|&include file for part 3\\
% |cdocspt4.tex|&include file for part 4\\
% |cdocsdrf.tex|&forwarding file for main file in draft mode\\
% |cdocsfi1.tex|&forwarding file for final version of chapter 1\\
% |cdocsfi2.tex|&forwarding file for final version of chapter 2\\
% \end{tabular}
% \end{center}
% Each of the eight files can be compiled directly by the \LaTeX{} compiler.
%
% %%%%%%%%%%%%%%%%%%%%%%%%%%%%%%%%%%%%%%
% \paragraph{Main File.}
%
% The main file is called |cdocsamp.tex|.
%
% Load the \textsf{childdoc} definitions and
% declare the filename for the main document:
%    \begin{macrocode}
\input{childdoc.def}
\childdocmain{}
%    \end{macrocode}

% Optional override for |\version| flag:
%    \begin{macrocode}
%%\ifchilddoc\else\providecommand{\version}{draft}\fi
%    \end{macrocode}

% Define the default values for the |\version| flag
% (|final| for the main file and |draft| for childs):
%    \begin{macrocode}
\ifchilddoc
\providecommand{\version}{draft}
\else
\providecommand{\version}{final}
\fi
%    \end{macrocode}

% Load the standard document class:
%    \begin{macrocode}
\documentclass[12pt]{article}
%    \end{macrocode}

% Start the document body:
%    \begin{macrocode}
\begin{document}
%    \end{macrocode}

% Declare a title page.
% Print title, part of document being processed and version flag:
%    \begin{macrocode}
\addtocounter{page}{-1}
\begin{center}
{\LARGE\bfseries{}childdoc example\par}
\vspace{1cm}
\ifchilddoc
\ifchilddocmanual part\else chapter\fi:
`\childdocname' of `\childdocjob'\par
\else
main document: `\childdocjob'\par
\fi
version: \version\par
\end{center}
\newpage
%    \end{macrocode}

% Manually include selected file,
% otherwise process as usual:
%    \begin{macrocode}
\ifchilddocmanual
\section*{part `\childdocname'}
\input{\childdocname}
\else
%    \end{macrocode}

% Include the two chapters:
%    \begin{macrocode}
\include{cdocsch1}
\include{cdocsch2}
%    \end{macrocode}

% Include the two parts unless only chapters should be displayed:
%    \begin{macrocode}
\ifchilddoc\else
\section{part three}
\input{cdocspt3}
\section{part four}
\input{cdocspt4}
\fi
%    \end{macrocode}

% Process as usual until here:
%    \begin{macrocode}
\fi
%    \end{macrocode}

% End of document body:
%    \begin{macrocode}
\end{document}
%    \end{macrocode}
%\iffalse
%</samplemain>
%\fi
%
% %%%%%%%%%%%%%%%%%%%%%%%%%%%%%%%%%%%%%%
% \paragraph{Chapter Include Files.}
%
% The include files are called |cdocsch1.tex| and |cdocsch2.tex|.
%
%\iffalse
%<*samplechap1|samplechap2>
%\fi

% Optional override for |\version| flag:
%    \begin{macrocode}
%%\providecommand{\version}{final}
%    \end{macrocode}

% Include the main document:
%    \begin{macrocode}
\input{childdoc.def}
\childdocof{cdocsamp}
%    \end{macrocode}

%\iffalse
%</samplechap1|samplechap2>
%\fi
%
%\iffalse
%<*samplechap1>
%\fi
% Some text for chapter 1:
%    \begin{macrocode}
\section{one}
some text in chapter one
%    \end{macrocode}

%\iffalse
%</samplechap1>
%\fi
% Some text for chapter 2:
%\iffalse
%<*samplechap2>
%\fi
%    \begin{macrocode}
\section{two}
more text in chapter two
%    \end{macrocode}

%\iffalse
%</samplechap2>
%\fi
%
% %%%%%%%%%%%%%%%%%%%%%%%%%%%%%%%%%%%%%%
% \paragraph{Part Include Files.}
%
% The include files are called |cdocspt3.tex| and |cdocspt4.tex|.
%
%\iffalse
%<*samplepart3|samplepart4>
%\fi

% Optional override for |\version| flag:
%    \begin{macrocode}
%%\providecommand{\version}{final}
%    \end{macrocode}

% Include the main document:
%    \begin{macrocode}
\input{childdoc.def}
\childdocby{cdocsamp}
%    \end{macrocode}

%\iffalse
%</samplepart3|samplepart4>
%\fi
%
%\iffalse
%<*samplepart3>
%\fi
% Some text for part 3:
%    \begin{macrocode}
some text in part three
%    \end{macrocode}

%\iffalse
%</samplepart3>
%\fi
% Some text for part 4:
%\iffalse
%<*samplepart4>
%\fi
%    \begin{macrocode}
more text in part four
%    \end{macrocode}

%\iffalse
%</samplepart4>
%\fi
%
% %%%%%%%%%%%%%%%%%%%%%%%%%%%%%%%%%%%%%%
% \paragraph{Forwarding for a Complete Draft.}
%
% The following forwarding file |cdocsdrf.tex|
% compiles the main document in draft mode:
%\iffalse
%<*sampledraft>
%\fi
%    \begin{macrocode}
\def\version{draft}
\input{childdoc.def}
\childdocforward{cdocsamp}
%    \end{macrocode}

%\iffalse
%</sampledraft>
%\fi
%
% %%%%%%%%%%%%%%%%%%%%%%%%%%%%%%%%%%%%%%
% \paragraph{Forwarding for Final Version of the Chapters.}
%
% The following forwarding files |cdocsfn1.tex| and |cdocsfn2.tex|
% (with identical content)
% compile the final versions of the child documents
% |cdocsch1.tex| and |cdocsch2.tex|, respectively:
%\iffalse
%<*samplefinal>
%\fi
%    \begin{macrocode}
\def\version{final}
\input{childdoc.def}
\childdocforwardprefix[cdocsamp]{cdocsfn}{cdocsch}
%    \end{macrocode}

%\iffalse
%</samplefinal>
%\fi
%
% %%%%%%%%%%%%%%%%%%%%%%%%%%%%%%%%%%%%%%
% \paragraph{Command Line Processing.}
%
% The following three command lines generate the output files
% |cdocscld|, |cdocscl1| and |cdocscl2|
% which should be identical to
% |cdocsdrf|, |cdocsch1| and |cdocsfn2|, respectively:
% \begin{center}
% \begin{tabular}{l}
% |latex -jobname cdocscld \|\\
% |  "\def\version{draft}\input{childdoc.def}\childdocforward{cdocsamp}"|\\
% |latex -jobname cdocscl1 \|\\
% |  "\input{childdoc.def}\childdocforward[cdocsamp]{cdocsch1}"|\\
% |latex -jobname cdocscl2 \|\\
% |  "\def\version{final}\input{childdoc.def}\childdocforward{cdocsch2}"|
% \end{tabular}
% \end{center}
% Note that the trailing backslash on each first line
% merely continues the input to the second line
% (for convenient cut ant paste).
% Furthermore, the command |latex| can be replaced by any
% of its alternative versions such as |pdflatex|.
%
% %%%%%%%%%%%%%%%%%%%%%%%%%%%%%%%%%%%%%%%%%%%%%%%%%%%%%%%%%%%%%%%%%%%%%%%%%%%%%%
% %%%%%%%%%%%%%%%%%%%%%%%%%%%%%%%%%%%%%%%%%%%%%%%%%%%%%%%%%%%%%%%%%%%%%%%%%%%%%%
% \section{Implementation}
%\iffalse
%<*package>
%\fi
%
% This section describes the definitions file |childdoc.def|.

% The definitions cannot be loaded using |\usepackage| or |\RequirePackage|
% which has a mechanism to prevent loading a style file more than once.
% When loading the definitions by means of |\input|
% multiple instances have to be prevented manually:
%\iffalse
%This code needs to be before the `\ProvidesFile' directive
%which is defined at the beginning of this file.
%Therefore it is also placed there and commented out here.
%</package>
%<*discard>
%\fi
%    \begin{macrocode}
\ifdefined\childdocmain\endinput\fi
%    \end{macrocode}
%\iffalse
%</discard>
%<*package>
%\fi
%
% \macro{\ifchilddoc}
% \macro{\ifchilddocmanual}
% The conditional |\ifchilddoc| tells whether a
% child (true) or main (false) document is being compiled.
% The conditional |\ifchilddocmanual| tells whether
% the |\includeonly| mechanism is used (false) or
% the selection of child files must be performed manually (true).
% The definitions initialise to false:
%    \begin{macrocode}
\newif\ifchilddoc
\newif\ifchilddocmanual
%    \end{macrocode}

% \macro{\childdocname}
% \macro{\childdocjob}
% The macro |\childdocname| stores the name of the main document
% to be compiled. The macro |\childdocjob| stores the name of
% the document on which the \LaTeX{} compiler was originally invoked.
% The content of |\jobname| cannot be compared
% to filenames specified in the source due to different catcodes.
% The following code rescans |\jobname|, stores the result
% in |\childdocname| and saves a copy in |\childdocjob|:
%    \begin{macrocode}
\edef\childdocname{\scantokens\expandafter{\jobname\noexpand}}
\let\childdocjob\childdocname
%    \end{macrocode}

% \macro{\childdocdisable}
% The macro |\childdocdisable| prevents the main file
% from being processed more than once.
% At this stage, the main document command |\childdocmain|
% is assumed to be called once again where it should do nothing.
% Any subsequent call to it should prevent
% a secondary processing of the main document
% It overwrites the forwarding commands
% |\childdocof| and |\childdocforward|
% with empty macros to prevent further inclusions of the main document:
%    \begin{macrocode}
\newcommand{\childdocdisable}
{
  \renewcommand{\childdocmain}[1]{\renewcommand{\childdocmain}[1]{\endinput}}
  \renewcommand{\childdocof}[1]{}
  \renewcommand{\childdocby}[2][]{}
  \renewcommand{\childdocforward}[2][]{}
  \renewcommand{\childdocdisable}{}
}
%    \end{macrocode}

% \macro{\childdocmain}
% The macro |\childdocmain| is to be called at the top of the main file
% with nothing or the main filename (without extension) as argument.
% First, it breaks loops.
% If the argument is not empty and does not match |\childdocname|
% (which is set by the first inclusion of |childdoc.def|),
% |\ifchilddoc| is set to true, |\includeonly| is applied to the child file
% and |\jobname| is set to the main file
% (for proper handling of |.aux| files):
%    \begin{macrocode}
\newcommand{\childdocmain}[1]
{
  \childdocdisable\childdocmain{}
  \if?#1?\else
    \begingroup
      \def\childdoctmp{#1}
      \ifx\childdoctmp\childdocname
        \def\childdoctmp{}
      \else
        \def\childdoctmp
        {
          \childdoctrue
          \includeonly{\childdocname}
          \def\childdocjob{#1}
          \def\jobname{#1}
        }
      \fi
      \expandafter
    \endgroup
    \childdoctmp
  \fi
}
%    \end{macrocode}

% \macro{\childdocof}
% The command |\childdocof| redirects
% compilation to the main file |#1|.
%    \begin{macrocode}
\newcommand{\childdocof}[1]
{
  \childdocdisable
  \childdoctrue
  \includeonly{\childdocname}
  \def\jobname{#1}
  \def\childdocjob{#1}
  \input{#1}
}
%    \end{macrocode}

% \macro{\childdocby}
% The command |\childdocby| ....
%    \begin{macrocode}
\newcommand{\childdocby}[2][]
{
  \childdocdisable
  \childdoctrue
  \childdocmanualtrue
  \if?#1?\else
    \def\jobname{#2}
  \fi
  \def\childdocjob{#2}
  \input{#2}
  \endinput
}
%    \end{macrocode}

% \macro{\childdocforward}
% The command |\childdocforward| redirects
% compilation to the main file or
% (if the optional argument is given) a child file.
% Parameters are set as if the main file
% or a child file starting with |\childdocof| was compiled.
% Then compilation is handed over to the main file:
%    \begin{macrocode}
\newcommand{\childdocforward}[2][]
{
  \begingroup
    \if?#1?
      \def\childdoctmp
      {
        \def\childdocname{#2}
        \def\childdocjob{#2}
        \def\jobname{#2}
        \input{#2}
        \endinput
      }
    \else
      \def\childdoctmp
      {
        \childdocdisable
        \def\childdocname{#2}
        \childdoctrue
        \includeonly{#2}
        \def\childdocjob{#1}
        \def\jobname{#1}
        \input{#1}
        \endinput
      }
    \fi
    \expandafter
  \endgroup
  \childdoctmp
}
%    \end{macrocode}

% \macro{\childdocforwardprefix}
% The command |\childdocforwardprefix| redirects
% compilation to the main or a child file by means of a pattern.
% The prefix |#1| in the current filename is replaced by |#2|
% and the suffix of the current filename is kept
% (it is assumed that the filename does not contain the substring `|~~~|'
% which is used as a delimiter).
% Compilation is handed over to the new file by |\childdocforward|:
%    \begin{macrocode}
\newcommand{\childdocforwardprefix}[3][]
{
  \begingroup
    \def\childdocextract #2##1~~~{\def\childdoctmp{\childdocforward[#1]{#3##1}}}
    \expandafter\childdocextract\childdocname~~~
    \expandafter
  \endgroup
  \childdoctmp
}
%    \end{macrocode}

% \macro{\childdoc}
% The deprecated macro |\childdoc| is a legacy version of |\childdocmain|:
%    \begin{macrocode}
\newcommand{\childdoc}{\childdocmain}
%    \end{macrocode}

% \macro{\childdocredirect}
% The deprecated macro |\childdocredirect| is a legacy version
% of |\childdocforward| and |\childdocforwardprefix|:
%    \begin{macrocode}
\newcommand{\childdocredirect}[2][]
{
  \begingroup
    \if?#1?
      \def\childdoctmp{\childdocforward{#2}}
    \else
      \def\childdoctmp{\childdocforwardprefix{#1}{#2}}
    \fi
    \expandafter
  \endgroup
  \childdoctmp
}
%    \end{macrocode}

%\iffalse
%</package>
%\fi
%
\endinput

\childdocof{cdocsamp}
%    \end{macrocode}

%\iffalse
%</samplechap1|samplechap2>
%\fi
%
%\iffalse
%<*samplechap1>
%\fi
% Some text for chapter 1:
%    \begin{macrocode}
\section{one}
some text in chapter one
%    \end{macrocode}

%\iffalse
%</samplechap1>
%\fi
% Some text for chapter 2:
%\iffalse
%<*samplechap2>
%\fi
%    \begin{macrocode}
\section{two}
more text in chapter two
%    \end{macrocode}

%\iffalse
%</samplechap2>
%\fi
%
% %%%%%%%%%%%%%%%%%%%%%%%%%%%%%%%%%%%%%%
% \paragraph{Part Include Files.}
%
% The include files are called |cdocspt3.tex| and |cdocspt4.tex|.
%
%\iffalse
%<*samplepart3|samplepart4>
%\fi

% Optional override for |\version| flag:
%    \begin{macrocode}
%%\providecommand{\version}{final}
%    \end{macrocode}

% Include the main document:
%    \begin{macrocode}
% \iffalse
%
% childdoc.dtx Copyright (C) 2017-2018 Niklas Beisert
%
% This work may be distributed and/or modified under the
% conditions of the LaTeX Project Public License, either version 1.3
% of this license or (at your option) any later version.
% The latest version of this license is in
%   http://www.latex-project.org/lppl.txt
% and version 1.3 or later is part of all distributions of LaTeX
% version 2005/12/01 or later.
%
% This work has the LPPL maintenance status `maintained'.
%
% The Current Maintainer of this work is Niklas Beisert.
%
% This work consists of the files childdoc.dtx and childdoc.ins
% and the derived files childdoc.def and cdocsamp.tex with
% cdocsch1.tex, cdocsch2.tex, cdocsdrf.tex, cdocsfn1.tex, cdocsfn2.tex.
%
%<package>\ifdefined\childdocmain\endinput\fi
%<package>\ProvidesFile{childdoc.def}[2018/12/30 v2.0 child document driver]
%<samplemain>\ProvidesFile{cdocsamp.tex}[2018/12/30 v2.0 sample for childdoc]
%<*driver>
%\ProvidesFile{childdoc.drv}[2018/12/30 v2.0 childdoc reference manual file]
\PassOptionsToClass{10pt,a4paper}{article}
\documentclass{ltxdoc}

\usepackage[margin=35mm]{geometry}
\usepackage{hyperref}
\usepackage{hyperxmp}
\usepackage[usenames]{color}

\hypersetup{colorlinks=true}
\hypersetup{pdfstartview=FitH}
\hypersetup{pdfpagemode=UseNone}
\hypersetup{pdfsource={}}
\hypersetup{pdflang={en-UK}}
\hypersetup{pdfcopyright={Copyright 2017-2018 Niklas Beisert.
  This work may be distributed and/or modified under the
  conditions of the LaTeX Project Public License, either version 1.3
  of this license or (at your option) any later version.}}
\hypersetup{pdflicenseurl={http://www.latex-project.org/lppl.txt}}
\hypersetup{pdfcontactaddress={ETH Zurich, ITP, HIT K,
  Wolfgang-Pauli-Strasse 27}}
\hypersetup{pdfcontactpostcode={8093}}
\hypersetup{pdfcontactcity={Zurich}}
\hypersetup{pdfcontactcountry={Switzerland}}
\hypersetup{pdfcontactemail={nbeisert@itp.phys.ethz.ch}}
\hypersetup{pdfcontacturl={http://people.phys.ethz.ch/\xmptilde nbeisert/}}

\newcommand{\secref}[1]{\hyperref[#1]{section \ref*{#1}}}

\parskip1ex
\parindent0pt
\let\olditemize\itemize
\def\itemize{\olditemize\parskip0pt}

\begin{document}

\title{The \textsf{childdoc} Package}
\hypersetup{pdftitle={The childdoc Package}}
\author{Niklas Beisert\\[2ex]
  Institut f\"ur Theoretische Physik\\
  Eidgen\"ossische Technische Hochschule Z\"urich\\
  Wolfgang-Pauli-Strasse 27, 8093 Z\"urich, Switzerland\\[1ex]
  \href{mailto:nbeisert@itp.phys.ethz.ch}
  {\texttt{nbeisert@itp.phys.ethz.ch}}}
\hypersetup{pdfauthor={Niklas Beisert}}
\hypersetup{pdfsubject={Manual for the LaTeX2e Package childdoc}}
\date{30 December 2018, \textsf{v2.0}}
\maketitle

\begin{abstract}\noindent
\textsf{childdoc} is a \LaTeXe{} package
that enables the direct compilation
of document sections included by |\include|
to individual files.
\end{abstract}

\begingroup
\parskip0ex
\tableofcontents
\endgroup

%%%%%%%%%%%%%%%%%%%%%%%%%%%%%%%%%%%%%%%%%%%%%%%%%%%%%%%%%%%%%%%%%%%%%%%%%%%%%%%%
%%%%%%%%%%%%%%%%%%%%%%%%%%%%%%%%%%%%%%%%%%%%%%%%%%%%%%%%%%%%%%%%%%%%%%%%%%%%%%%%
\section{Introduction}

\LaTeX{} provides a mechanism to structure a large document (such as a book)
into a main file and several child files (containing the chapters)
using the |\include| command.
This mechanism is beneficial for documents
which span hundreds of pages in order to
make the source file(s) more manageable.
Moreover, compilation can be restricted to
selected child files by means of the |\includeonly| command.
The latter feature can be used to reduce the compilation time while editing
(this was significantly more useful in the earlier days of \LaTeX{})
or to generate a smaller document which is easier to navigate.
Another application of |\includeonly| is to generate
documents consisting of selected parts of the complete document.

However, there are a few drawbacks of the plain |\include| mechanism:
\begin{itemize}
\item
The child files cannot be compiled on their own,
they can only be compiled via the main file.
A naive editing environment
(such as a text editor with an option
to have the current file processed by \LaTeX)
may require one to switch to the main file before compiling;
attempting to compile the child file produces errors.
\item
The main file must be modified (each time)
to adjust the |\includeonly| command
to the present needs. This easily leaves the main file in a messy state.
\item
The generated document will always carry the filename
of the main document. This is inconvenient if
several child files are to be compiled and
to be kept for distribution.
\end{itemize}

The present package provides a simple interface
to make child files individually compilable by \LaTeX{}.
Compiling a child file then has the same effect as compiling
the main file with an |\includeonly| command
to select the appropriate child.
Moreover the generated document will carry the name of the child
rather than the main file.
This resolves all three above issues.

This feature is meant to make the editing of books,
thesis documents and lecture notes somewhat more convenient.
However, the package can also be used efficiently for
composing a series of documents (such as exercise sheets)
which are typically distributed individually.
It then assists the author in generating the individual documents
(potentially in different versions)
as well as a document containing the collected series.
Another application is in developing style files
or other kinds of included material
where compilation of the style file could redirect
to a sample or test file.

%%%%%%%%%%%%%%%%%%%%%%%%%%%%%%%%%%%%%%%%%%%%%%%%%%%%%%%%%%%%%%%%%%%%%%%%%%%%%%%%
%%%%%%%%%%%%%%%%%%%%%%%%%%%%%%%%%%%%%%%%%%%%%%%%%%%%%%%%%%%%%%%%%%%%%%%%%%%%%%%%
\section{Usage}

First of all, the package \textsf{childdoc} is \emph{not} a standard
\LaTeXe{} |.sty| style file! Therefore it needs to be invoked in
a non-standard way.

%%%%%%%%%%%%%%%%%%%%%%%%%%%%%%%%%%%%%%%%%%%%%%%%%%%%%%%%%%%%%%%%%%%%%%%%%%%%%%%%
\subsection{Included Files}
\label{sec:include}

%%%%%%%%%%%%%%%%%%%%%%%%%%%%%%%%%%%%%%%%
\DescribeMacro{\childdocmain}
To use the package, add the commands
\begin{center}
\begin{tabular}{l}
|\input{childdoc.def}|\\
|\childdocmain{}|\\
\end{tabular}
\end{center}
at the very top of the main \LaTeX{} file,
in particular \emph{before} the |\documentclass| statement!
The argument of |\childdocmain| should be left empty
(but it must be present).

%%%%%%%%%%%%%%%%%%%%%%%%%%%%%%%%%%%%%%%%
\DescribeMacro{\childdocof}
Furthermore, add the commands
\begin{center}
\begin{tabular}{l}
|\input{childdoc.def}|\\
|\childdocof{|\textit{main}|}|\\
\end{tabular}
\end{center}
at the top of every child file \textit{child}
which is included by |\include{|\textit{child}|}|
from within the main file
(or at least for those files to be compiled individually).
The argument \textit{main} must be the filename of the main file.

There are a couple of
considerations in setting up the main and child documents:

%%%%%%%%%%%%%%%%%%%%%%%%%%%%%%%%%%%%%%%%
\paragraph{Restrictions.}

Please note the following restrictions:
\begin{itemize}
\item
|\childdocmain| must be called with one argument \textit{main}
to ensure compatibility with earlier version of the package.
It must either be empty (|\childdocmain{}|)
or precisely match the filename of the main file in which it is specified.
See \secref{sec:detection} for further information.
\item
The filename \textit{main} must be specified without the |.tex| extension.
\item
The filename \textit{main} is case sensitive
(even in case-insensitive file systems)
due to internal string comparison.
\item
The argument \textit{main} should be fully expanded, it cannot be a macro.
\item
Subdirectories and special characters should be avoided in filenames.
\item
The command |\childdocmain{|\textit{main}|}| must be followed by a whitespace.
It should not be followed immediately by another command
or by a comment mark `|%|'.
This is because the \TeX{} parser reads the token immediately following
the argument of |\childdocmain| and puts it
at the beginning of every child section;
however, a white\-space is ignored.
\end{itemize}

%%%%%%%%%%%%%%%%%%%%%%%%%%%%%%%%%%%%%%%%
\paragraph{Content of Main File.}

It is advisable to place all content in the child files included by |\include|.
Any output contained in the main file will appear in all child documents
unless suppressed manually;
it cannot be suppressed automatically by the |\includeonly| directive
and thus should normally be avoided.
A method to include some content in the main file
by means of conditional processing is described in \secref{sec:conditional}.

%%%%%%%%%%%%%%%%%%%%%%%%%%%%%%%%%%%%%%%%
\paragraph{Page Numbering.}

When only a part of the document is compiled,
the appropriate numbering of pages
(as well as other status parameters)
is determined from the |.aux| files.
The latter contain information from previous passes.
However this information needs to propagate through
all intermediate child documents.
Therefore the page numbering in child documents may well
be inconsistent until the complete document is compiled at least once.

A useful (if unconventional) way to always ensure a consistent
page numbering is to restart the numbering in each child document
and denote the pages by `\textit{child}|.|\textit{page}'
where \textit{child} represents the chapter/section number of the child file.
This can be achieved by the command
|\numberwithin{page}{|\textit{child}|}|
of the \textsf{amsmath} package
where \textit{child} can be |chapter| or |section|
depending on the chosen structuring.
Alternatively, one can modify the macro |\thepage| appropriately
and reset the counter |page| at the start of each child file.

%%%%%%%%%%%%%%%%%%%%%%%%%%%%%%%%%%%%%%%%%%%%%%%%%%%%%%%%%%%%%%%%%%%%%%%%%%%%%%%%
\subsection{Conditional Processing}
\label{sec:conditional}

The package provides a mechanism to compile different versions
of a document. To customise the versions further some conditional processing
can come in handy to distinguish which version is being compiled.
The package provides two macros to describe the compilation context:

%%%%%%%%%%%%%%%%%%%%%%%%%%%%%%%%%%%%%%%%
\DescribeMacro{\ifchilddoc}
The conditional |\ifchilddoc| distinguishes between the compilation of
child documents and the main document:
%
\begin{center}
|\ifchilddoc |\textit{child-code}| |[|\||else |\textit{main-code}]| \||fi|
\end{center}

%%%%%%%%%%%%%%%%%%%%%%%%%%%%%%%%%%%%%%%%
\DescribeMacro{\childdocname}
\DescribeMacro{\childdocjob}
The macro |\childdocname| contains the filename (without extension)
of the main or child file being processed.
Note that |\childdocjob| will always contain the name of the main file.

%%%%%%%%%%%%%%%%%%%%%%%%%%%%%%%%%%%%%%%%
\paragraph{Title Page.}

Conditional processing can be used to include a title or banner page
in the main document when proper precautions are taken.
Importantly, the code in the main file should ensure that the page counter
(as well as other status parameters which are stored in the |.aux| files)
takes the same value after the conditional processing.
Otherwise the page numbers may take divergent values
depending on which part is compiled.

For example, a title page could be declared by:
%
\begin{center}
\begin{tabular}{l}
|\ifchilddoc\||else|\\
|\addtocounter{page}{-1}|\\
\textit{code for title page}\\
|\newpage|\\
|\||fi|
\end{tabular}
\end{center}
%
A banner page for the child documents can be generated by:
%
\begin{center}
\begin{tabular}{l}
|\ifchilddoc|\\
|\addtocounter{page}{-1}|\\
\textit{code for banner page}\\
|\newpage|\\
|\||fi|
\end{tabular}
\end{center}
%
Here one could write a message such as:
\begin{center}
|This is the part \childdocname{} of \childdocjob{}.|
\end{center}

%%%%%%%%%%%%%%%%%%%%%%%%%%%%%%%%%%%%%%%%%%%%%%%%%%%%%%%%%%%%%%%%%%%%%%%%%%%%%%%%
\subsection{Flags}
\label{sec:flags}

The package makes it easy to generate different versions
of the main or child documents.
To this end compilation flags can be defined
and assigned different default values.
They will be particularly useful in conjunction
with the forwarding mechanism described in \secref{sec:forward}.

For example, it may be useful to have a flag |\version|
which can be set to |draft| or |final|.
The document source will contain some conditional code
depending on the value of |\version|.
Suppose further, the flag should default to |final| for the main file
and to |draft| for child files
which is a natural assignment for editing the document.
This is achieved by placing the following code
in the preamble of the main document
(below the |\childdocmain| directive):
%
\begin{center}
\begin{tabular}{l}
|\ifchilddoc|\\
|\providecommand{\version}{draft}|\\
|\||else|\\
|\providecommand{\version}{final}|\\
|\||fi|
\end{tabular}
\end{center}
%
The definition by |\providecommand| makes sure
that previous definitions are not overwritten.
Further statements |\providecommand{\version}{...}|
can thus be added before the above code to override it.

For the main file, one might add a line
(between |\childdocmain| and the above block)
%
\begin{center}
|%\ifchilddoc\||else\providecommand{\version}{draft}\||fi|
\end{center}
%
which can be uncommented to produce a draft version.
Likewise one can add a line to the very top of a child file
(above the |\childdocof{|\textit{main}|}| directive)
%
\begin{center}
|%\providecommand{\version}{final}|
\end{center}
%
which can be uncommented to produce the final version of this child document.

%%%%%%%%%%%%%%%%%%%%%%%%%%%%%%%%%%%%%%%%%%%%%%%%%%%%%%%%%%%%%%%%%%%%%%%%%%%%%%%%
\subsection{Forwarding}
\label{sec:forward}

Different versions of the main or child documents
using compilation flags as described in \secref{sec:flags}
can be (permanently) stored in different files
for convenient compilation, viewing and distribution.
To this end, the package defines a command
to pass on compilation to a different file:

%%%%%%%%%%%%%%%%%%%%%%%%%%%%%%%%%%%%%%%%
\DescribeMacro{\childdocforward}
The command |\childdocforward| redirects processing to
another source file:
%
\begin{center}
\begin{tabular}{l}
|\input{childdoc.def}|\\
|\childdocforward[|\textit{main}|]{|\textit{dest}|}|\\
\end{tabular}
\end{center}
%
The argument \textit{dest} is the destination file
(without extension).
It should be the main file or one of the child files.
Note that further \textsf{childdoc} directives
such as |\childdocof| and |\childdocforward|
in the indicated file will be processed in this form.
The optional argument \textit{main}
passes on directly to the main file \textit{main}
while pretending to compile the child \textit{dest}.
This form behaves as if \textit{dest}
issues |\childdocof{|\textit{main}|}| right away,
and no further \textsf{childdoc} directives will be processed.

%%%%%%%%%%%%%%%%%%%%%%%%%%%%%%%%%%%%%%%%
\DescribeMacro{\...prefix}
In the alternative form |\childdocforwardprefix|,
%
\begin{center}
\begin{tabular}{l}
|\input{childdoc.def}|\\
|\childdocforwardprefix[|\textit{main}|]{|\textit{prefix}|}{|\textit{dest}|}|
\end{tabular}
\end{center}
%
the destination file is determined by a pattern
depending on the current file:
To make this work, the current file must be called
`{\textit{prefix}\hspace{0.2em}\textit{suffix}}'
with \textit{prefix} matching precisely the argument.
Processing is then passed on to the file
`{\textit{dest}\hspace{0.2em}\textit{suffix}}'.
Surely, the same effect is achieved by
directly specifying the
argument `{\textit{dest}\hspace{0.2em}\textit{suffix}}'
in the first form.
However, that requires to set up a different file
for each child. With the alternative form of the command
all these files can have exactly the same content
which simplifies setting them up and maintaining them.

For example, the following file |draft.tex|
with a compilation flag |\version| as described in \secref{sec:flags}
compiles the main document as a draft:
%
\begin{center}
\begin{tabular}{l}
|\def\version{draft}|\\
|\input{childdoc.def}|\\
|\childdocforward{|\textit{main}|}|
\end{tabular}
\end{center}
%
Likewise, the following files |final|\textit{nn}|.tex|
compile the final version of the child document
|child|\textit{nn}|.tex|:
%
\begin{center}
\begin{tabular}{l}
|\def\version{final}|\\
|\input{childdoc.def}|\\
|\childdocforwardprefix{final}{child}|
\end{tabular}
\end{center}
%

Note that when several versions of a main file and/or of each child file
are to be generated, it may be convenient to set up a |Makefile| or
shell script to automatise the process.

%%%%%%%%%%%%%%%%%%%%%%%%%%%%%%%%%%%%%%%%%%%%%%%%%%%%%%%%%%%%%%%%%%%%%%%%%%%%%%%%
\subsection{Command Line Processing}
\label{sec:commandline}

The effect of redirection files can also be achieved by invoking
the \LaTeX{} compiler with a more elaborate command line.
Most conveniently this should be done as part
of a shell script or a |Makefile|.

When using \textsf{childdoc} in the main file, the following
command lines effectively perform a redirection
(note that depending on the shell being used,
backslashes may have to be doubled: `|\|' $\to$ `|\\|'):
%
\begin{center}
|... -jobname "|\textit{target}|" |\\|"|[\textit{flags}]%
|\input{childdoc.def}\childdocforward[|\textit{main}|]{|\textit{dest}|}"|
\end{center}
%
Here \textit{target} is the name of the output file,
\textit{main} is the name of the main file
and \textit{dest} is the name of the main or child file to be processed
(all filenames without extensions).
The optional argument \textit{main} can be omitted
if \textit{main} matches \textit{dest}.
Optionally, compilation \textit{flags} can be defined via |\def| commands.
This command line makes the \TeX{} engine believe
it is compiling the file \textit{target}
whose content is specified as the latter parameter.
The provided code then forwards the processing to
\textit{main} or \textit{dest} as described in \secref{sec:forward}.

%%%%%%%%%%%%%%%%%%%%%%%%%%%%%%%%%%%%%%%%%%%%%%%%%%%%%%%%%%%%%%%%%%%%%%%%%%%%%%%%
\subsection{Include by Input}
\label{sec:input}

Including child documents by |\include| has some restrictions by design.
Most notably, the content of a child document always occupies
its own set of pages; pages cannot be shared between child documents.
Usually, this behaviour makes perfect sense
because each child document contain an essential part of the document.
However, in some situations it may be desirable to compose
a document from a collection of parts
without having mandatory page breaks between then.
For this case, the package
provides a mechanism to include parts
by |\input| which can also be processed individually.
However, by construction this mechanism
requires manual handling of the content to be output.

%%%%%%%%%%%%%%%%%%%%%%%%%%%%%%%%%%%%%%%%
\DescribeMacro{\ifchilddocmanual}
The main file should be prepared as usual, see \secref{sec:include}.
However, the document body must make a distinction
between processing of an individual part and of the main document, e.g.:
%
\begin{center}
\begin{tabular}{l}
|\ifchilddocmanual|\\
|\input{\childdocname}|\\
|\||else|\\
\textit{document body with }|\input{|\textit{part}|}|\\
|\||fi|
\end{tabular}
\end{center}
%
The conditional |\ifchilddocmanual| is true whenever
a part to be included by |\input| is being compiled,
and the name of the part is stored in |\childdocname|.

%%%%%%%%%%%%%%%%%%%%%%%%%%%%%%%%%%%%%%%%
\DescribeMacro{\childdocby}
Each part to be included by |\input| should start with:
%
\begin{center}
\begin{tabular}{l}
|\input{childdoc.def}|\\
|\childdocby{|\textit{main}|}|\\
\end{tabular}
\end{center}
%
The directive |\childdocby| is similar to |\childdocof|
described in \secref{sec:include},
but the subsequent selection of content must be done manually.
To that end, both |\ifchilddoc| and |\ifchilddocmanual|
will be true upon processing of a part,
and the name of the part is stored in |\childdocname|.
Note that |\jobname| will be set to the filename of the current part
so that each part receives an individual |.aux| file
that does not interfere with the |.aux| file(s) of the main document.
This behaviour can be altered by the alternative form
|\childdocby[*]{|\textit{main}|}| (with a non-empty optional argument)
which uses the |.aux| file of the main document
by setting |\jobname| to \textit{main}.

%%%%%%%%%%%%%%%%%%%%%%%%%%%%%%%%%%%%%%%%%%%%%%%%%%%%%%%%%%%%%%%%%%%%%%%%%%%%%%%%
\subsection{Driver Development}
\label{sec:driver}

The \textsf{childdoc} mechanism can also be use for the development
of definition files such as \LaTeX{} styles or classes.
This case differs from the above setup with multiple parts
included by |\include| in that no |\includeonly| should be invoked.
This can be achieved by starting the include file
(before |\ProvidesPackage|) with:
%
\begin{center}
\begin{tabular}{l}
|\input{childdoc.def}|\\
|\childdocforward{|\textit{main}|}|\\
\end{tabular}
\end{center}
%
or alternatively with:
%
\begin{center}
\begin{tabular}{l}
|\input{childdoc.def}|\\
|\childdocby{|\textit{main}|}|\\
\end{tabular}
\end{center}
%
Both forms have slightly different effects as described above.
The main file is prepared as usual, see \secref{sec:include}.

%%%%%%%%%%%%%%%%%%%%%%%%%%%%%%%%%%%%%%%%%%%%%%%%%%%%%%%%%%%%%%%%%%%%%%%%%%%%%%%%
\subsection{Legacy Detection}
\label{sec:detection}

The directive |\childdocmain| in the main file can detect
whether the complete document or merely a child is to be compiled
even without using the directive |\childdocof|.
This method is deprecated because it is less robust
and there is no compelling reason to use it;
it is merely provided for backward compatibility
and it may be removed in future versions.

If the detection mechanism is to be used,
it is mandatory to correctly specify
the filename of the main file as the argument of |\childdocmain|:
%
\begin{center}
\begin{tabular}{l}
|\input{childdoc.def}|\\
|\childdocmain{|\textit{main}|}|\\
\end{tabular}
\end{center}
%
If |\jobname| does not match the argument \textit{main} of |\childdocmain|,
it is assumed that |\jobname| points to the child file to be compiled.
When using |\childdocmain| with the main file specified as argument,
it suffices to start a child file
with just |\input{|\textit{main}|}|
without loading of the package and using |\childdocof|.
If instead all processing is done
with the appropriate \textsf{childdoc} directives,
the argument of \textit{main} of |\childdocmain| can be empty.

An alternative version of the command line processing described
in \secref{sec:commandline} using the detection mechanism reads:
%
\begin{center}
|... -jobname "|\textit{target}|" "|[\textit{flags}]%
[|\def\jobname{|\textit{dest}|}|]|\input{|\textit{main}|}"|
\end{center}

%%%%%%%%%%%%%%%%%%%%%%%%%%%%%%%%%%%%%%%%%%%%%%%%%%%%%%%%%%%%%%%%%%%%%%%%%%%%%%%%
\subsection{Manual Code}
\label{sec:manual}

In case one cannot be certain whether the definitions file |childdoc.def|
is installed on the target \TeX{} distribution
and one prefers not to ship it,
it is conceivable to paste a few relevant commands into the sources.

To that end, drop all statements |\input{childdoc.def}|
and perform the replacements as outlined below.
Instead of |\childdocmain{|\textit{main}|}| add the following code
to the top of the main file:
%
\begin{center}
\begin{tabular}{l}
|\||ifdefined\childdocname\endinput\||fi\newif\ifchilddoc|\\
|\edef\childdocname{\scantokens\expandafter{\jobname\noexpand}}|\\
|\def\childdocmain{|\textit{main}|}\||ifx\childdocmain\childdocname\||else|\\
|\childdoctrue\includeonly{\childdocname}\let\jobname\childdocmain\||fi|\\
\end{tabular}
\end{center}
%
Instead of |\childdocof{|\textit{main}|}| just include the main file
at the top of each child file:
%
\begin{center}
|\input{|\textit{main}|}|
\end{center}
%
A simple redirection |\childdocforward{|\textit{dest}|}| is achieved by:
%
\begin{center}
|\def\jobname{|\textit{dest}|}\input{\jobname}|
\end{center}
%
The redirection with prefix
|\childdocforwardprefix[|\textit{prefix}|]{|\textit{dest}|}|
is accomplished by:
%
\begin{center}
\begin{tabular}{l}
|{\edef\jobname{\scantokens\expandafter{\jobname\noexpand}}|\\
|\def\redirectjob |\textit{prefix}|#1~~~{\gdef\jobname{|\textit{dest}|#1}}|\\
|\expandafter\redirectjob\jobname~~~}\input{\jobname}|
\end{tabular}
\end{center}

In an alternative approach,
child documents can be compiled by a specific command line
without additional code or specific definitions:
%
\begin{center}
|... -jobname "|\textit{target}|" "|[\textit{flags}]%
|\includeonly{|\textit{dest}|}\input{|\textit{main}|}"|
\end{center}
%

%%%%%%%%%%%%%%%%%%%%%%%%%%%%%%%%%%%%%%%%%%%%%%%%%%%%%%%%%%%%%%%%%%%%%%%%%%%%%%%%
%%%%%%%%%%%%%%%%%%%%%%%%%%%%%%%%%%%%%%%%%%%%%%%%%%%%%%%%%%%%%%%%%%%%%%%%%%%%%%%%
\section{Information}

%%%%%%%%%%%%%%%%%%%%%%%%%%%%%%%%%%%%%%%%%%%%%%%%%%%%%%%%%%%%%%%%%%%%%%%%%%%%%%%%
\subsection{Copyright}

Copyright \copyright{} 2017--2018 Niklas Beisert

This work may be distributed and/or modified under the
conditions of the \LaTeX{} Project Public License, either version 1.3
of this license or (at your option) any later version.
The latest version of this license is in
  \url{http://www.latex-project.org/lppl.txt}
and version 1.3 or later is part of all distributions of \LaTeX{}
version 2005/12/01 or later.

This work has the LPPL maintenance status `maintained'.

The Current Maintainer of this work is Niklas Beisert.

This work consists of the files |README.txt|, |childdoc.ins| and |childdoc.dtx|
as well as the derived files |childdoc.def|, |cdocsamp.tex|
with |cdocsch1.tex|, |cdocsch2.tex|, |cdocspt3.tex|, |cdocspt4.tex|,
|cdocsdrf.tex|, |cdocsfn1.tex|, |cdocsfn2.tex|
as well as |childdoc.pdf|.

%%%%%%%%%%%%%%%%%%%%%%%%%%%%%%%%%%%%%%%%%%%%%%%%%%%%%%%%%%%%%%%%%%%%%%%%%%%%%%%%
\subsection{Files and Installation}

The package consists of the files:
%
\begin{center}
\begin{tabular}{ll}
    |README.txt|   & readme file \\
    |childdoc.ins| & installation file \\
    |childdoc.dtx| & source file \\
    |childdoc.def| & definition file \\
    |cdocsamp.tex| & sample main file \\
    |cdocsch1.tex| & sample include file \\
    |cdocsch2.tex| & sample include file \\
    |cdocspt3.tex| & sample part file \\
    |cdocspt4.tex| & sample part file \\
    |cdocsdrf.tex| & sample redirection file \\
    |cdocsfn1.tex| & sample redirection file \\
    |cdocsfn2.tex| & sample redirection file \\
    |childdoc.pdf| & manual
\end{tabular}
\end{center}
%
The distribution consists of the files
|README.txt|, |childdoc.ins| and |childdoc.dtx|.
%
\begin{itemize}
\item
Run (pdf)\LaTeX{} on |childdoc.dtx|
to compile the manual |childdoc.pdf| (this file).
\item
Run \LaTeX{} on |childdoc.ins| to create the definitions file |childdoc.def|
and the sample |cdocsamp.tex| with include files
|cdocsch1.tex|, |cdocsch2.tex|, |cdocspt3.tex|, |cdocspt4.tex|,
|cdocsdrf.tex|, |cdocsfn1.tex|, |cdocsfn2.tex|.
Then copy the file |childdoc.def| to an appropriate directory of your \LaTeX{}
distribution, e.g.\ \textit{texmf-root}|/tex/latex/childdoc|.
\end{itemize}

%%%%%%%%%%%%%%%%%%%%%%%%%%%%%%%%%%%%%%%%%%%%%%%%%%%%%%%%%%%%%%%%%%%%%%%%%%%%%%%%
\subsection{Related CTAN Packages}

There are several other packages which offer a similar functionality:
%
\begin{itemize}
\item
The packages
\href{http://ctan.org/pkg/docmute}{\textsf{docmute}},
\href{http://ctan.org/pkg/includex}{\textsf{includex}} and
\href{http://ctan.org/pkg/standalone}{\textsf{standalone}}
provide commands to include only the document body of
a child file thus allowing both files to be compiled individually.
\item
The packages \href{http://ctan.org/pkg/subdocs}{\textsf{subdocs}}
and \href{http://ctan.org/pkg/subfiles}{\textsf{subfiles}}
provide structures in which the main and child documents can be
encapsulated and allowing them to be compiled individually.
The inclusion mechanism is different from the conventional |\include|.
\item
The package \href{http://ctan.org/pkg/combine}{\textsf{combine}}
is an elaborate solution to combine several documents into one.
\end{itemize}
%
See also the CTAN topic \href{http://ctan.org/topic/subdocs}{\textsf{subdocs}}
for further related packages.
The present package differs from the above solutions in that
a document structure constructed with the conventional |\include| mechanism
just needs two extra commands at the top of every file
such that all constituent files can be compiled individually.

%%%%%%%%%%%%%%%%%%%%%%%%%%%%%%%%%%%%%%%%%%%%%%%%%%%%%%%%%%%%%%%%%%%%%%%%%%%%%%%%
%\subsection{Feature Suggestions}
%
%The following is a list of features which may be useful for future
%versions of this package:
%%
%\begin{itemize}
%\item
%\ldots
%\end{itemize}

%%%%%%%%%%%%%%%%%%%%%%%%%%%%%%%%%%%%%%%%%%%%%%%%%%%%%%%%%%%%%%%%%%%%%%%%%%%%%%%%
\subsection{Revision History}

%%%%%%%%%%%%%%%%%%%%%%%%%%%%%%%%%%%%%%%%
\paragraph{v2.0:} 2018/12/30

\begin{itemize}
\item
immediate forward processing
\item
added |\childdocby| mechanism
\item
manual restructured
\end{itemize}

%%%%%%%%%%%%%%%%%%%%%%%%%%%%%%%%%%%%%%%%
\paragraph{v1.6:} 2018/01/17

\begin{itemize}
\item
application for development of include files
\item
corrections to manual
\end{itemize}

%%%%%%%%%%%%%%%%%%%%%%%%%%%%%%%%%%%%%%%%
\paragraph{v1.5:} 2017/05/21

\begin{itemize}
\item
more complete structuring introduced
\item
|\childdocof| introduced
\item
|\childdoc| renamed to |\childdocmain|
\item
|\childredirect| renamed to |\childdocforward| and |\childdocforwardprefix|
and functionality expanded
\end{itemize}

%%%%%%%%%%%%%%%%%%%%%%%%%%%%%%%%%%%%%%%%
\paragraph{v1.0:} 2017/04/27

\begin{itemize}
\item
manual and install package
\item
first version published on CTAN
\end{itemize}

%%%%%%%%%%%%%%%%%%%%%%%%%%%%%%%%%%%%%%%%
\paragraph{v0.6:} 2017/04/26

\begin{itemize}
\item
redirection mechanism added
\end{itemize}

%%%%%%%%%%%%%%%%%%%%%%%%%%%%%%%%%%%%%%%%
\paragraph{v0.5:} 2017/04/26

\begin{itemize}
\item
functionality in definition file
\end{itemize}


%%%%%%%%%%%%%%%%%%%%%%%%%%%%%%%%%%%%%%%%%%%%%%%%%%%%%%%%%%%%%%%%%%%%%%%%%%%%%%%%
%%%%%%%%%%%%%%%%%%%%%%%%%%%%%%%%%%%%%%%%%%%%%%%%%%%%%%%%%%%%%%%%%%%%%%%%%%%%%%%%
%%%%%%%%%%%%%%%%%%%%%%%%%%%%%%%%%%%%%%%%%%%%%%%%%%%%%%%%%%%%%%%%%%%%%%%%%%%%%%%%
\appendix

\settowidth\MacroIndent{\rmfamily\scriptsize 000\ }

 \DocInput{childdoc.dtx}

\end{document}
%</driver>
% \fi
%
% %%%%%%%%%%%%%%%%%%%%%%%%%%%%%%%%%%%%%%%%%%%%%%%%%%%%%%%%%%%%%%%%%%%%%%%%%%%%%%
% %%%%%%%%%%%%%%%%%%%%%%%%%%%%%%%%%%%%%%%%%%%%%%%%%%%%%%%%%%%%%%%%%%%%%%%%%%%%%%
% \section{Sample}
%\iffalse
%<*samplemain>
%\fi
%
% The following presents a sample document
% with two chapters, two parts, a title page,
% a compile flag as well as three forwarding files to set the flag.
% It consists of eight |.tex| files:
% \begin{center}
% \begin{tabular}{ll}
% |cdocsamp.tex|&main file\\
% |cdocsch1.tex|&include file for chapter 1\\
% |cdocsch2.tex|&include file for chapter 2\\
% |cdocspt3.tex|&include file for part 3\\
% |cdocspt4.tex|&include file for part 4\\
% |cdocsdrf.tex|&forwarding file for main file in draft mode\\
% |cdocsfi1.tex|&forwarding file for final version of chapter 1\\
% |cdocsfi2.tex|&forwarding file for final version of chapter 2\\
% \end{tabular}
% \end{center}
% Each of the eight files can be compiled directly by the \LaTeX{} compiler.
%
% %%%%%%%%%%%%%%%%%%%%%%%%%%%%%%%%%%%%%%
% \paragraph{Main File.}
%
% The main file is called |cdocsamp.tex|.
%
% Load the \textsf{childdoc} definitions and
% declare the filename for the main document:
%    \begin{macrocode}
\input{childdoc.def}
\childdocmain{}
%    \end{macrocode}

% Optional override for |\version| flag:
%    \begin{macrocode}
%%\ifchilddoc\else\providecommand{\version}{draft}\fi
%    \end{macrocode}

% Define the default values for the |\version| flag
% (|final| for the main file and |draft| for childs):
%    \begin{macrocode}
\ifchilddoc
\providecommand{\version}{draft}
\else
\providecommand{\version}{final}
\fi
%    \end{macrocode}

% Load the standard document class:
%    \begin{macrocode}
\documentclass[12pt]{article}
%    \end{macrocode}

% Start the document body:
%    \begin{macrocode}
\begin{document}
%    \end{macrocode}

% Declare a title page.
% Print title, part of document being processed and version flag:
%    \begin{macrocode}
\addtocounter{page}{-1}
\begin{center}
{\LARGE\bfseries{}childdoc example\par}
\vspace{1cm}
\ifchilddoc
\ifchilddocmanual part\else chapter\fi:
`\childdocname' of `\childdocjob'\par
\else
main document: `\childdocjob'\par
\fi
version: \version\par
\end{center}
\newpage
%    \end{macrocode}

% Manually include selected file,
% otherwise process as usual:
%    \begin{macrocode}
\ifchilddocmanual
\section*{part `\childdocname'}
\input{\childdocname}
\else
%    \end{macrocode}

% Include the two chapters:
%    \begin{macrocode}
\include{cdocsch1}
\include{cdocsch2}
%    \end{macrocode}

% Include the two parts unless only chapters should be displayed:
%    \begin{macrocode}
\ifchilddoc\else
\section{part three}
\input{cdocspt3}
\section{part four}
\input{cdocspt4}
\fi
%    \end{macrocode}

% Process as usual until here:
%    \begin{macrocode}
\fi
%    \end{macrocode}

% End of document body:
%    \begin{macrocode}
\end{document}
%    \end{macrocode}
%\iffalse
%</samplemain>
%\fi
%
% %%%%%%%%%%%%%%%%%%%%%%%%%%%%%%%%%%%%%%
% \paragraph{Chapter Include Files.}
%
% The include files are called |cdocsch1.tex| and |cdocsch2.tex|.
%
%\iffalse
%<*samplechap1|samplechap2>
%\fi

% Optional override for |\version| flag:
%    \begin{macrocode}
%%\providecommand{\version}{final}
%    \end{macrocode}

% Include the main document:
%    \begin{macrocode}
\input{childdoc.def}
\childdocof{cdocsamp}
%    \end{macrocode}

%\iffalse
%</samplechap1|samplechap2>
%\fi
%
%\iffalse
%<*samplechap1>
%\fi
% Some text for chapter 1:
%    \begin{macrocode}
\section{one}
some text in chapter one
%    \end{macrocode}

%\iffalse
%</samplechap1>
%\fi
% Some text for chapter 2:
%\iffalse
%<*samplechap2>
%\fi
%    \begin{macrocode}
\section{two}
more text in chapter two
%    \end{macrocode}

%\iffalse
%</samplechap2>
%\fi
%
% %%%%%%%%%%%%%%%%%%%%%%%%%%%%%%%%%%%%%%
% \paragraph{Part Include Files.}
%
% The include files are called |cdocspt3.tex| and |cdocspt4.tex|.
%
%\iffalse
%<*samplepart3|samplepart4>
%\fi

% Optional override for |\version| flag:
%    \begin{macrocode}
%%\providecommand{\version}{final}
%    \end{macrocode}

% Include the main document:
%    \begin{macrocode}
\input{childdoc.def}
\childdocby{cdocsamp}
%    \end{macrocode}

%\iffalse
%</samplepart3|samplepart4>
%\fi
%
%\iffalse
%<*samplepart3>
%\fi
% Some text for part 3:
%    \begin{macrocode}
some text in part three
%    \end{macrocode}

%\iffalse
%</samplepart3>
%\fi
% Some text for part 4:
%\iffalse
%<*samplepart4>
%\fi
%    \begin{macrocode}
more text in part four
%    \end{macrocode}

%\iffalse
%</samplepart4>
%\fi
%
% %%%%%%%%%%%%%%%%%%%%%%%%%%%%%%%%%%%%%%
% \paragraph{Forwarding for a Complete Draft.}
%
% The following forwarding file |cdocsdrf.tex|
% compiles the main document in draft mode:
%\iffalse
%<*sampledraft>
%\fi
%    \begin{macrocode}
\def\version{draft}
\input{childdoc.def}
\childdocforward{cdocsamp}
%    \end{macrocode}

%\iffalse
%</sampledraft>
%\fi
%
% %%%%%%%%%%%%%%%%%%%%%%%%%%%%%%%%%%%%%%
% \paragraph{Forwarding for Final Version of the Chapters.}
%
% The following forwarding files |cdocsfn1.tex| and |cdocsfn2.tex|
% (with identical content)
% compile the final versions of the child documents
% |cdocsch1.tex| and |cdocsch2.tex|, respectively:
%\iffalse
%<*samplefinal>
%\fi
%    \begin{macrocode}
\def\version{final}
\input{childdoc.def}
\childdocforwardprefix[cdocsamp]{cdocsfn}{cdocsch}
%    \end{macrocode}

%\iffalse
%</samplefinal>
%\fi
%
% %%%%%%%%%%%%%%%%%%%%%%%%%%%%%%%%%%%%%%
% \paragraph{Command Line Processing.}
%
% The following three command lines generate the output files
% |cdocscld|, |cdocscl1| and |cdocscl2|
% which should be identical to
% |cdocsdrf|, |cdocsch1| and |cdocsfn2|, respectively:
% \begin{center}
% \begin{tabular}{l}
% |latex -jobname cdocscld \|\\
% |  "\def\version{draft}\input{childdoc.def}\childdocforward{cdocsamp}"|\\
% |latex -jobname cdocscl1 \|\\
% |  "\input{childdoc.def}\childdocforward[cdocsamp]{cdocsch1}"|\\
% |latex -jobname cdocscl2 \|\\
% |  "\def\version{final}\input{childdoc.def}\childdocforward{cdocsch2}"|
% \end{tabular}
% \end{center}
% Note that the trailing backslash on each first line
% merely continues the input to the second line
% (for convenient cut ant paste).
% Furthermore, the command |latex| can be replaced by any
% of its alternative versions such as |pdflatex|.
%
% %%%%%%%%%%%%%%%%%%%%%%%%%%%%%%%%%%%%%%%%%%%%%%%%%%%%%%%%%%%%%%%%%%%%%%%%%%%%%%
% %%%%%%%%%%%%%%%%%%%%%%%%%%%%%%%%%%%%%%%%%%%%%%%%%%%%%%%%%%%%%%%%%%%%%%%%%%%%%%
% \section{Implementation}
%\iffalse
%<*package>
%\fi
%
% This section describes the definitions file |childdoc.def|.

% The definitions cannot be loaded using |\usepackage| or |\RequirePackage|
% which has a mechanism to prevent loading a style file more than once.
% When loading the definitions by means of |\input|
% multiple instances have to be prevented manually:
%\iffalse
%This code needs to be before the `\ProvidesFile' directive
%which is defined at the beginning of this file.
%Therefore it is also placed there and commented out here.
%</package>
%<*discard>
%\fi
%    \begin{macrocode}
\ifdefined\childdocmain\endinput\fi
%    \end{macrocode}
%\iffalse
%</discard>
%<*package>
%\fi
%
% \macro{\ifchilddoc}
% \macro{\ifchilddocmanual}
% The conditional |\ifchilddoc| tells whether a
% child (true) or main (false) document is being compiled.
% The conditional |\ifchilddocmanual| tells whether
% the |\includeonly| mechanism is used (false) or
% the selection of child files must be performed manually (true).
% The definitions initialise to false:
%    \begin{macrocode}
\newif\ifchilddoc
\newif\ifchilddocmanual
%    \end{macrocode}

% \macro{\childdocname}
% \macro{\childdocjob}
% The macro |\childdocname| stores the name of the main document
% to be compiled. The macro |\childdocjob| stores the name of
% the document on which the \LaTeX{} compiler was originally invoked.
% The content of |\jobname| cannot be compared
% to filenames specified in the source due to different catcodes.
% The following code rescans |\jobname|, stores the result
% in |\childdocname| and saves a copy in |\childdocjob|:
%    \begin{macrocode}
\edef\childdocname{\scantokens\expandafter{\jobname\noexpand}}
\let\childdocjob\childdocname
%    \end{macrocode}

% \macro{\childdocdisable}
% The macro |\childdocdisable| prevents the main file
% from being processed more than once.
% At this stage, the main document command |\childdocmain|
% is assumed to be called once again where it should do nothing.
% Any subsequent call to it should prevent
% a secondary processing of the main document
% It overwrites the forwarding commands
% |\childdocof| and |\childdocforward|
% with empty macros to prevent further inclusions of the main document:
%    \begin{macrocode}
\newcommand{\childdocdisable}
{
  \renewcommand{\childdocmain}[1]{\renewcommand{\childdocmain}[1]{\endinput}}
  \renewcommand{\childdocof}[1]{}
  \renewcommand{\childdocby}[2][]{}
  \renewcommand{\childdocforward}[2][]{}
  \renewcommand{\childdocdisable}{}
}
%    \end{macrocode}

% \macro{\childdocmain}
% The macro |\childdocmain| is to be called at the top of the main file
% with nothing or the main filename (without extension) as argument.
% First, it breaks loops.
% If the argument is not empty and does not match |\childdocname|
% (which is set by the first inclusion of |childdoc.def|),
% |\ifchilddoc| is set to true, |\includeonly| is applied to the child file
% and |\jobname| is set to the main file
% (for proper handling of |.aux| files):
%    \begin{macrocode}
\newcommand{\childdocmain}[1]
{
  \childdocdisable\childdocmain{}
  \if?#1?\else
    \begingroup
      \def\childdoctmp{#1}
      \ifx\childdoctmp\childdocname
        \def\childdoctmp{}
      \else
        \def\childdoctmp
        {
          \childdoctrue
          \includeonly{\childdocname}
          \def\childdocjob{#1}
          \def\jobname{#1}
        }
      \fi
      \expandafter
    \endgroup
    \childdoctmp
  \fi
}
%    \end{macrocode}

% \macro{\childdocof}
% The command |\childdocof| redirects
% compilation to the main file |#1|.
%    \begin{macrocode}
\newcommand{\childdocof}[1]
{
  \childdocdisable
  \childdoctrue
  \includeonly{\childdocname}
  \def\jobname{#1}
  \def\childdocjob{#1}
  \input{#1}
}
%    \end{macrocode}

% \macro{\childdocby}
% The command |\childdocby| ....
%    \begin{macrocode}
\newcommand{\childdocby}[2][]
{
  \childdocdisable
  \childdoctrue
  \childdocmanualtrue
  \if?#1?\else
    \def\jobname{#2}
  \fi
  \def\childdocjob{#2}
  \input{#2}
  \endinput
}
%    \end{macrocode}

% \macro{\childdocforward}
% The command |\childdocforward| redirects
% compilation to the main file or
% (if the optional argument is given) a child file.
% Parameters are set as if the main file
% or a child file starting with |\childdocof| was compiled.
% Then compilation is handed over to the main file:
%    \begin{macrocode}
\newcommand{\childdocforward}[2][]
{
  \begingroup
    \if?#1?
      \def\childdoctmp
      {
        \def\childdocname{#2}
        \def\childdocjob{#2}
        \def\jobname{#2}
        \input{#2}
        \endinput
      }
    \else
      \def\childdoctmp
      {
        \childdocdisable
        \def\childdocname{#2}
        \childdoctrue
        \includeonly{#2}
        \def\childdocjob{#1}
        \def\jobname{#1}
        \input{#1}
        \endinput
      }
    \fi
    \expandafter
  \endgroup
  \childdoctmp
}
%    \end{macrocode}

% \macro{\childdocforwardprefix}
% The command |\childdocforwardprefix| redirects
% compilation to the main or a child file by means of a pattern.
% The prefix |#1| in the current filename is replaced by |#2|
% and the suffix of the current filename is kept
% (it is assumed that the filename does not contain the substring `|~~~|'
% which is used as a delimiter).
% Compilation is handed over to the new file by |\childdocforward|:
%    \begin{macrocode}
\newcommand{\childdocforwardprefix}[3][]
{
  \begingroup
    \def\childdocextract #2##1~~~{\def\childdoctmp{\childdocforward[#1]{#3##1}}}
    \expandafter\childdocextract\childdocname~~~
    \expandafter
  \endgroup
  \childdoctmp
}
%    \end{macrocode}

% \macro{\childdoc}
% The deprecated macro |\childdoc| is a legacy version of |\childdocmain|:
%    \begin{macrocode}
\newcommand{\childdoc}{\childdocmain}
%    \end{macrocode}

% \macro{\childdocredirect}
% The deprecated macro |\childdocredirect| is a legacy version
% of |\childdocforward| and |\childdocforwardprefix|:
%    \begin{macrocode}
\newcommand{\childdocredirect}[2][]
{
  \begingroup
    \if?#1?
      \def\childdoctmp{\childdocforward{#2}}
    \else
      \def\childdoctmp{\childdocforwardprefix{#1}{#2}}
    \fi
    \expandafter
  \endgroup
  \childdoctmp
}
%    \end{macrocode}

%\iffalse
%</package>
%\fi
%
\endinput

\childdocby{cdocsamp}
%    \end{macrocode}

%\iffalse
%</samplepart3|samplepart4>
%\fi
%
%\iffalse
%<*samplepart3>
%\fi
% Some text for part 3:
%    \begin{macrocode}
some text in part three
%    \end{macrocode}

%\iffalse
%</samplepart3>
%\fi
% Some text for part 4:
%\iffalse
%<*samplepart4>
%\fi
%    \begin{macrocode}
more text in part four
%    \end{macrocode}

%\iffalse
%</samplepart4>
%\fi
%
% %%%%%%%%%%%%%%%%%%%%%%%%%%%%%%%%%%%%%%
% \paragraph{Forwarding for a Complete Draft.}
%
% The following forwarding file |cdocsdrf.tex|
% compiles the main document in draft mode:
%\iffalse
%<*sampledraft>
%\fi
%    \begin{macrocode}
\def\version{draft}
% \iffalse
%
% childdoc.dtx Copyright (C) 2017-2018 Niklas Beisert
%
% This work may be distributed and/or modified under the
% conditions of the LaTeX Project Public License, either version 1.3
% of this license or (at your option) any later version.
% The latest version of this license is in
%   http://www.latex-project.org/lppl.txt
% and version 1.3 or later is part of all distributions of LaTeX
% version 2005/12/01 or later.
%
% This work has the LPPL maintenance status `maintained'.
%
% The Current Maintainer of this work is Niklas Beisert.
%
% This work consists of the files childdoc.dtx and childdoc.ins
% and the derived files childdoc.def and cdocsamp.tex with
% cdocsch1.tex, cdocsch2.tex, cdocsdrf.tex, cdocsfn1.tex, cdocsfn2.tex.
%
%<package>\ifdefined\childdocmain\endinput\fi
%<package>\ProvidesFile{childdoc.def}[2018/12/30 v2.0 child document driver]
%<samplemain>\ProvidesFile{cdocsamp.tex}[2018/12/30 v2.0 sample for childdoc]
%<*driver>
%\ProvidesFile{childdoc.drv}[2018/12/30 v2.0 childdoc reference manual file]
\PassOptionsToClass{10pt,a4paper}{article}
\documentclass{ltxdoc}

\usepackage[margin=35mm]{geometry}
\usepackage{hyperref}
\usepackage{hyperxmp}
\usepackage[usenames]{color}

\hypersetup{colorlinks=true}
\hypersetup{pdfstartview=FitH}
\hypersetup{pdfpagemode=UseNone}
\hypersetup{pdfsource={}}
\hypersetup{pdflang={en-UK}}
\hypersetup{pdfcopyright={Copyright 2017-2018 Niklas Beisert.
  This work may be distributed and/or modified under the
  conditions of the LaTeX Project Public License, either version 1.3
  of this license or (at your option) any later version.}}
\hypersetup{pdflicenseurl={http://www.latex-project.org/lppl.txt}}
\hypersetup{pdfcontactaddress={ETH Zurich, ITP, HIT K,
  Wolfgang-Pauli-Strasse 27}}
\hypersetup{pdfcontactpostcode={8093}}
\hypersetup{pdfcontactcity={Zurich}}
\hypersetup{pdfcontactcountry={Switzerland}}
\hypersetup{pdfcontactemail={nbeisert@itp.phys.ethz.ch}}
\hypersetup{pdfcontacturl={http://people.phys.ethz.ch/\xmptilde nbeisert/}}

\newcommand{\secref}[1]{\hyperref[#1]{section \ref*{#1}}}

\parskip1ex
\parindent0pt
\let\olditemize\itemize
\def\itemize{\olditemize\parskip0pt}

\begin{document}

\title{The \textsf{childdoc} Package}
\hypersetup{pdftitle={The childdoc Package}}
\author{Niklas Beisert\\[2ex]
  Institut f\"ur Theoretische Physik\\
  Eidgen\"ossische Technische Hochschule Z\"urich\\
  Wolfgang-Pauli-Strasse 27, 8093 Z\"urich, Switzerland\\[1ex]
  \href{mailto:nbeisert@itp.phys.ethz.ch}
  {\texttt{nbeisert@itp.phys.ethz.ch}}}
\hypersetup{pdfauthor={Niklas Beisert}}
\hypersetup{pdfsubject={Manual for the LaTeX2e Package childdoc}}
\date{30 December 2018, \textsf{v2.0}}
\maketitle

\begin{abstract}\noindent
\textsf{childdoc} is a \LaTeXe{} package
that enables the direct compilation
of document sections included by |\include|
to individual files.
\end{abstract}

\begingroup
\parskip0ex
\tableofcontents
\endgroup

%%%%%%%%%%%%%%%%%%%%%%%%%%%%%%%%%%%%%%%%%%%%%%%%%%%%%%%%%%%%%%%%%%%%%%%%%%%%%%%%
%%%%%%%%%%%%%%%%%%%%%%%%%%%%%%%%%%%%%%%%%%%%%%%%%%%%%%%%%%%%%%%%%%%%%%%%%%%%%%%%
\section{Introduction}

\LaTeX{} provides a mechanism to structure a large document (such as a book)
into a main file and several child files (containing the chapters)
using the |\include| command.
This mechanism is beneficial for documents
which span hundreds of pages in order to
make the source file(s) more manageable.
Moreover, compilation can be restricted to
selected child files by means of the |\includeonly| command.
The latter feature can be used to reduce the compilation time while editing
(this was significantly more useful in the earlier days of \LaTeX{})
or to generate a smaller document which is easier to navigate.
Another application of |\includeonly| is to generate
documents consisting of selected parts of the complete document.

However, there are a few drawbacks of the plain |\include| mechanism:
\begin{itemize}
\item
The child files cannot be compiled on their own,
they can only be compiled via the main file.
A naive editing environment
(such as a text editor with an option
to have the current file processed by \LaTeX)
may require one to switch to the main file before compiling;
attempting to compile the child file produces errors.
\item
The main file must be modified (each time)
to adjust the |\includeonly| command
to the present needs. This easily leaves the main file in a messy state.
\item
The generated document will always carry the filename
of the main document. This is inconvenient if
several child files are to be compiled and
to be kept for distribution.
\end{itemize}

The present package provides a simple interface
to make child files individually compilable by \LaTeX{}.
Compiling a child file then has the same effect as compiling
the main file with an |\includeonly| command
to select the appropriate child.
Moreover the generated document will carry the name of the child
rather than the main file.
This resolves all three above issues.

This feature is meant to make the editing of books,
thesis documents and lecture notes somewhat more convenient.
However, the package can also be used efficiently for
composing a series of documents (such as exercise sheets)
which are typically distributed individually.
It then assists the author in generating the individual documents
(potentially in different versions)
as well as a document containing the collected series.
Another application is in developing style files
or other kinds of included material
where compilation of the style file could redirect
to a sample or test file.

%%%%%%%%%%%%%%%%%%%%%%%%%%%%%%%%%%%%%%%%%%%%%%%%%%%%%%%%%%%%%%%%%%%%%%%%%%%%%%%%
%%%%%%%%%%%%%%%%%%%%%%%%%%%%%%%%%%%%%%%%%%%%%%%%%%%%%%%%%%%%%%%%%%%%%%%%%%%%%%%%
\section{Usage}

First of all, the package \textsf{childdoc} is \emph{not} a standard
\LaTeXe{} |.sty| style file! Therefore it needs to be invoked in
a non-standard way.

%%%%%%%%%%%%%%%%%%%%%%%%%%%%%%%%%%%%%%%%%%%%%%%%%%%%%%%%%%%%%%%%%%%%%%%%%%%%%%%%
\subsection{Included Files}
\label{sec:include}

%%%%%%%%%%%%%%%%%%%%%%%%%%%%%%%%%%%%%%%%
\DescribeMacro{\childdocmain}
To use the package, add the commands
\begin{center}
\begin{tabular}{l}
|\input{childdoc.def}|\\
|\childdocmain{}|\\
\end{tabular}
\end{center}
at the very top of the main \LaTeX{} file,
in particular \emph{before} the |\documentclass| statement!
The argument of |\childdocmain| should be left empty
(but it must be present).

%%%%%%%%%%%%%%%%%%%%%%%%%%%%%%%%%%%%%%%%
\DescribeMacro{\childdocof}
Furthermore, add the commands
\begin{center}
\begin{tabular}{l}
|\input{childdoc.def}|\\
|\childdocof{|\textit{main}|}|\\
\end{tabular}
\end{center}
at the top of every child file \textit{child}
which is included by |\include{|\textit{child}|}|
from within the main file
(or at least for those files to be compiled individually).
The argument \textit{main} must be the filename of the main file.

There are a couple of
considerations in setting up the main and child documents:

%%%%%%%%%%%%%%%%%%%%%%%%%%%%%%%%%%%%%%%%
\paragraph{Restrictions.}

Please note the following restrictions:
\begin{itemize}
\item
|\childdocmain| must be called with one argument \textit{main}
to ensure compatibility with earlier version of the package.
It must either be empty (|\childdocmain{}|)
or precisely match the filename of the main file in which it is specified.
See \secref{sec:detection} for further information.
\item
The filename \textit{main} must be specified without the |.tex| extension.
\item
The filename \textit{main} is case sensitive
(even in case-insensitive file systems)
due to internal string comparison.
\item
The argument \textit{main} should be fully expanded, it cannot be a macro.
\item
Subdirectories and special characters should be avoided in filenames.
\item
The command |\childdocmain{|\textit{main}|}| must be followed by a whitespace.
It should not be followed immediately by another command
or by a comment mark `|%|'.
This is because the \TeX{} parser reads the token immediately following
the argument of |\childdocmain| and puts it
at the beginning of every child section;
however, a white\-space is ignored.
\end{itemize}

%%%%%%%%%%%%%%%%%%%%%%%%%%%%%%%%%%%%%%%%
\paragraph{Content of Main File.}

It is advisable to place all content in the child files included by |\include|.
Any output contained in the main file will appear in all child documents
unless suppressed manually;
it cannot be suppressed automatically by the |\includeonly| directive
and thus should normally be avoided.
A method to include some content in the main file
by means of conditional processing is described in \secref{sec:conditional}.

%%%%%%%%%%%%%%%%%%%%%%%%%%%%%%%%%%%%%%%%
\paragraph{Page Numbering.}

When only a part of the document is compiled,
the appropriate numbering of pages
(as well as other status parameters)
is determined from the |.aux| files.
The latter contain information from previous passes.
However this information needs to propagate through
all intermediate child documents.
Therefore the page numbering in child documents may well
be inconsistent until the complete document is compiled at least once.

A useful (if unconventional) way to always ensure a consistent
page numbering is to restart the numbering in each child document
and denote the pages by `\textit{child}|.|\textit{page}'
where \textit{child} represents the chapter/section number of the child file.
This can be achieved by the command
|\numberwithin{page}{|\textit{child}|}|
of the \textsf{amsmath} package
where \textit{child} can be |chapter| or |section|
depending on the chosen structuring.
Alternatively, one can modify the macro |\thepage| appropriately
and reset the counter |page| at the start of each child file.

%%%%%%%%%%%%%%%%%%%%%%%%%%%%%%%%%%%%%%%%%%%%%%%%%%%%%%%%%%%%%%%%%%%%%%%%%%%%%%%%
\subsection{Conditional Processing}
\label{sec:conditional}

The package provides a mechanism to compile different versions
of a document. To customise the versions further some conditional processing
can come in handy to distinguish which version is being compiled.
The package provides two macros to describe the compilation context:

%%%%%%%%%%%%%%%%%%%%%%%%%%%%%%%%%%%%%%%%
\DescribeMacro{\ifchilddoc}
The conditional |\ifchilddoc| distinguishes between the compilation of
child documents and the main document:
%
\begin{center}
|\ifchilddoc |\textit{child-code}| |[|\||else |\textit{main-code}]| \||fi|
\end{center}

%%%%%%%%%%%%%%%%%%%%%%%%%%%%%%%%%%%%%%%%
\DescribeMacro{\childdocname}
\DescribeMacro{\childdocjob}
The macro |\childdocname| contains the filename (without extension)
of the main or child file being processed.
Note that |\childdocjob| will always contain the name of the main file.

%%%%%%%%%%%%%%%%%%%%%%%%%%%%%%%%%%%%%%%%
\paragraph{Title Page.}

Conditional processing can be used to include a title or banner page
in the main document when proper precautions are taken.
Importantly, the code in the main file should ensure that the page counter
(as well as other status parameters which are stored in the |.aux| files)
takes the same value after the conditional processing.
Otherwise the page numbers may take divergent values
depending on which part is compiled.

For example, a title page could be declared by:
%
\begin{center}
\begin{tabular}{l}
|\ifchilddoc\||else|\\
|\addtocounter{page}{-1}|\\
\textit{code for title page}\\
|\newpage|\\
|\||fi|
\end{tabular}
\end{center}
%
A banner page for the child documents can be generated by:
%
\begin{center}
\begin{tabular}{l}
|\ifchilddoc|\\
|\addtocounter{page}{-1}|\\
\textit{code for banner page}\\
|\newpage|\\
|\||fi|
\end{tabular}
\end{center}
%
Here one could write a message such as:
\begin{center}
|This is the part \childdocname{} of \childdocjob{}.|
\end{center}

%%%%%%%%%%%%%%%%%%%%%%%%%%%%%%%%%%%%%%%%%%%%%%%%%%%%%%%%%%%%%%%%%%%%%%%%%%%%%%%%
\subsection{Flags}
\label{sec:flags}

The package makes it easy to generate different versions
of the main or child documents.
To this end compilation flags can be defined
and assigned different default values.
They will be particularly useful in conjunction
with the forwarding mechanism described in \secref{sec:forward}.

For example, it may be useful to have a flag |\version|
which can be set to |draft| or |final|.
The document source will contain some conditional code
depending on the value of |\version|.
Suppose further, the flag should default to |final| for the main file
and to |draft| for child files
which is a natural assignment for editing the document.
This is achieved by placing the following code
in the preamble of the main document
(below the |\childdocmain| directive):
%
\begin{center}
\begin{tabular}{l}
|\ifchilddoc|\\
|\providecommand{\version}{draft}|\\
|\||else|\\
|\providecommand{\version}{final}|\\
|\||fi|
\end{tabular}
\end{center}
%
The definition by |\providecommand| makes sure
that previous definitions are not overwritten.
Further statements |\providecommand{\version}{...}|
can thus be added before the above code to override it.

For the main file, one might add a line
(between |\childdocmain| and the above block)
%
\begin{center}
|%\ifchilddoc\||else\providecommand{\version}{draft}\||fi|
\end{center}
%
which can be uncommented to produce a draft version.
Likewise one can add a line to the very top of a child file
(above the |\childdocof{|\textit{main}|}| directive)
%
\begin{center}
|%\providecommand{\version}{final}|
\end{center}
%
which can be uncommented to produce the final version of this child document.

%%%%%%%%%%%%%%%%%%%%%%%%%%%%%%%%%%%%%%%%%%%%%%%%%%%%%%%%%%%%%%%%%%%%%%%%%%%%%%%%
\subsection{Forwarding}
\label{sec:forward}

Different versions of the main or child documents
using compilation flags as described in \secref{sec:flags}
can be (permanently) stored in different files
for convenient compilation, viewing and distribution.
To this end, the package defines a command
to pass on compilation to a different file:

%%%%%%%%%%%%%%%%%%%%%%%%%%%%%%%%%%%%%%%%
\DescribeMacro{\childdocforward}
The command |\childdocforward| redirects processing to
another source file:
%
\begin{center}
\begin{tabular}{l}
|\input{childdoc.def}|\\
|\childdocforward[|\textit{main}|]{|\textit{dest}|}|\\
\end{tabular}
\end{center}
%
The argument \textit{dest} is the destination file
(without extension).
It should be the main file or one of the child files.
Note that further \textsf{childdoc} directives
such as |\childdocof| and |\childdocforward|
in the indicated file will be processed in this form.
The optional argument \textit{main}
passes on directly to the main file \textit{main}
while pretending to compile the child \textit{dest}.
This form behaves as if \textit{dest}
issues |\childdocof{|\textit{main}|}| right away,
and no further \textsf{childdoc} directives will be processed.

%%%%%%%%%%%%%%%%%%%%%%%%%%%%%%%%%%%%%%%%
\DescribeMacro{\...prefix}
In the alternative form |\childdocforwardprefix|,
%
\begin{center}
\begin{tabular}{l}
|\input{childdoc.def}|\\
|\childdocforwardprefix[|\textit{main}|]{|\textit{prefix}|}{|\textit{dest}|}|
\end{tabular}
\end{center}
%
the destination file is determined by a pattern
depending on the current file:
To make this work, the current file must be called
`{\textit{prefix}\hspace{0.2em}\textit{suffix}}'
with \textit{prefix} matching precisely the argument.
Processing is then passed on to the file
`{\textit{dest}\hspace{0.2em}\textit{suffix}}'.
Surely, the same effect is achieved by
directly specifying the
argument `{\textit{dest}\hspace{0.2em}\textit{suffix}}'
in the first form.
However, that requires to set up a different file
for each child. With the alternative form of the command
all these files can have exactly the same content
which simplifies setting them up and maintaining them.

For example, the following file |draft.tex|
with a compilation flag |\version| as described in \secref{sec:flags}
compiles the main document as a draft:
%
\begin{center}
\begin{tabular}{l}
|\def\version{draft}|\\
|\input{childdoc.def}|\\
|\childdocforward{|\textit{main}|}|
\end{tabular}
\end{center}
%
Likewise, the following files |final|\textit{nn}|.tex|
compile the final version of the child document
|child|\textit{nn}|.tex|:
%
\begin{center}
\begin{tabular}{l}
|\def\version{final}|\\
|\input{childdoc.def}|\\
|\childdocforwardprefix{final}{child}|
\end{tabular}
\end{center}
%

Note that when several versions of a main file and/or of each child file
are to be generated, it may be convenient to set up a |Makefile| or
shell script to automatise the process.

%%%%%%%%%%%%%%%%%%%%%%%%%%%%%%%%%%%%%%%%%%%%%%%%%%%%%%%%%%%%%%%%%%%%%%%%%%%%%%%%
\subsection{Command Line Processing}
\label{sec:commandline}

The effect of redirection files can also be achieved by invoking
the \LaTeX{} compiler with a more elaborate command line.
Most conveniently this should be done as part
of a shell script or a |Makefile|.

When using \textsf{childdoc} in the main file, the following
command lines effectively perform a redirection
(note that depending on the shell being used,
backslashes may have to be doubled: `|\|' $\to$ `|\\|'):
%
\begin{center}
|... -jobname "|\textit{target}|" |\\|"|[\textit{flags}]%
|\input{childdoc.def}\childdocforward[|\textit{main}|]{|\textit{dest}|}"|
\end{center}
%
Here \textit{target} is the name of the output file,
\textit{main} is the name of the main file
and \textit{dest} is the name of the main or child file to be processed
(all filenames without extensions).
The optional argument \textit{main} can be omitted
if \textit{main} matches \textit{dest}.
Optionally, compilation \textit{flags} can be defined via |\def| commands.
This command line makes the \TeX{} engine believe
it is compiling the file \textit{target}
whose content is specified as the latter parameter.
The provided code then forwards the processing to
\textit{main} or \textit{dest} as described in \secref{sec:forward}.

%%%%%%%%%%%%%%%%%%%%%%%%%%%%%%%%%%%%%%%%%%%%%%%%%%%%%%%%%%%%%%%%%%%%%%%%%%%%%%%%
\subsection{Include by Input}
\label{sec:input}

Including child documents by |\include| has some restrictions by design.
Most notably, the content of a child document always occupies
its own set of pages; pages cannot be shared between child documents.
Usually, this behaviour makes perfect sense
because each child document contain an essential part of the document.
However, in some situations it may be desirable to compose
a document from a collection of parts
without having mandatory page breaks between then.
For this case, the package
provides a mechanism to include parts
by |\input| which can also be processed individually.
However, by construction this mechanism
requires manual handling of the content to be output.

%%%%%%%%%%%%%%%%%%%%%%%%%%%%%%%%%%%%%%%%
\DescribeMacro{\ifchilddocmanual}
The main file should be prepared as usual, see \secref{sec:include}.
However, the document body must make a distinction
between processing of an individual part and of the main document, e.g.:
%
\begin{center}
\begin{tabular}{l}
|\ifchilddocmanual|\\
|\input{\childdocname}|\\
|\||else|\\
\textit{document body with }|\input{|\textit{part}|}|\\
|\||fi|
\end{tabular}
\end{center}
%
The conditional |\ifchilddocmanual| is true whenever
a part to be included by |\input| is being compiled,
and the name of the part is stored in |\childdocname|.

%%%%%%%%%%%%%%%%%%%%%%%%%%%%%%%%%%%%%%%%
\DescribeMacro{\childdocby}
Each part to be included by |\input| should start with:
%
\begin{center}
\begin{tabular}{l}
|\input{childdoc.def}|\\
|\childdocby{|\textit{main}|}|\\
\end{tabular}
\end{center}
%
The directive |\childdocby| is similar to |\childdocof|
described in \secref{sec:include},
but the subsequent selection of content must be done manually.
To that end, both |\ifchilddoc| and |\ifchilddocmanual|
will be true upon processing of a part,
and the name of the part is stored in |\childdocname|.
Note that |\jobname| will be set to the filename of the current part
so that each part receives an individual |.aux| file
that does not interfere with the |.aux| file(s) of the main document.
This behaviour can be altered by the alternative form
|\childdocby[*]{|\textit{main}|}| (with a non-empty optional argument)
which uses the |.aux| file of the main document
by setting |\jobname| to \textit{main}.

%%%%%%%%%%%%%%%%%%%%%%%%%%%%%%%%%%%%%%%%%%%%%%%%%%%%%%%%%%%%%%%%%%%%%%%%%%%%%%%%
\subsection{Driver Development}
\label{sec:driver}

The \textsf{childdoc} mechanism can also be use for the development
of definition files such as \LaTeX{} styles or classes.
This case differs from the above setup with multiple parts
included by |\include| in that no |\includeonly| should be invoked.
This can be achieved by starting the include file
(before |\ProvidesPackage|) with:
%
\begin{center}
\begin{tabular}{l}
|\input{childdoc.def}|\\
|\childdocforward{|\textit{main}|}|\\
\end{tabular}
\end{center}
%
or alternatively with:
%
\begin{center}
\begin{tabular}{l}
|\input{childdoc.def}|\\
|\childdocby{|\textit{main}|}|\\
\end{tabular}
\end{center}
%
Both forms have slightly different effects as described above.
The main file is prepared as usual, see \secref{sec:include}.

%%%%%%%%%%%%%%%%%%%%%%%%%%%%%%%%%%%%%%%%%%%%%%%%%%%%%%%%%%%%%%%%%%%%%%%%%%%%%%%%
\subsection{Legacy Detection}
\label{sec:detection}

The directive |\childdocmain| in the main file can detect
whether the complete document or merely a child is to be compiled
even without using the directive |\childdocof|.
This method is deprecated because it is less robust
and there is no compelling reason to use it;
it is merely provided for backward compatibility
and it may be removed in future versions.

If the detection mechanism is to be used,
it is mandatory to correctly specify
the filename of the main file as the argument of |\childdocmain|:
%
\begin{center}
\begin{tabular}{l}
|\input{childdoc.def}|\\
|\childdocmain{|\textit{main}|}|\\
\end{tabular}
\end{center}
%
If |\jobname| does not match the argument \textit{main} of |\childdocmain|,
it is assumed that |\jobname| points to the child file to be compiled.
When using |\childdocmain| with the main file specified as argument,
it suffices to start a child file
with just |\input{|\textit{main}|}|
without loading of the package and using |\childdocof|.
If instead all processing is done
with the appropriate \textsf{childdoc} directives,
the argument of \textit{main} of |\childdocmain| can be empty.

An alternative version of the command line processing described
in \secref{sec:commandline} using the detection mechanism reads:
%
\begin{center}
|... -jobname "|\textit{target}|" "|[\textit{flags}]%
[|\def\jobname{|\textit{dest}|}|]|\input{|\textit{main}|}"|
\end{center}

%%%%%%%%%%%%%%%%%%%%%%%%%%%%%%%%%%%%%%%%%%%%%%%%%%%%%%%%%%%%%%%%%%%%%%%%%%%%%%%%
\subsection{Manual Code}
\label{sec:manual}

In case one cannot be certain whether the definitions file |childdoc.def|
is installed on the target \TeX{} distribution
and one prefers not to ship it,
it is conceivable to paste a few relevant commands into the sources.

To that end, drop all statements |\input{childdoc.def}|
and perform the replacements as outlined below.
Instead of |\childdocmain{|\textit{main}|}| add the following code
to the top of the main file:
%
\begin{center}
\begin{tabular}{l}
|\||ifdefined\childdocname\endinput\||fi\newif\ifchilddoc|\\
|\edef\childdocname{\scantokens\expandafter{\jobname\noexpand}}|\\
|\def\childdocmain{|\textit{main}|}\||ifx\childdocmain\childdocname\||else|\\
|\childdoctrue\includeonly{\childdocname}\let\jobname\childdocmain\||fi|\\
\end{tabular}
\end{center}
%
Instead of |\childdocof{|\textit{main}|}| just include the main file
at the top of each child file:
%
\begin{center}
|\input{|\textit{main}|}|
\end{center}
%
A simple redirection |\childdocforward{|\textit{dest}|}| is achieved by:
%
\begin{center}
|\def\jobname{|\textit{dest}|}\input{\jobname}|
\end{center}
%
The redirection with prefix
|\childdocforwardprefix[|\textit{prefix}|]{|\textit{dest}|}|
is accomplished by:
%
\begin{center}
\begin{tabular}{l}
|{\edef\jobname{\scantokens\expandafter{\jobname\noexpand}}|\\
|\def\redirectjob |\textit{prefix}|#1~~~{\gdef\jobname{|\textit{dest}|#1}}|\\
|\expandafter\redirectjob\jobname~~~}\input{\jobname}|
\end{tabular}
\end{center}

In an alternative approach,
child documents can be compiled by a specific command line
without additional code or specific definitions:
%
\begin{center}
|... -jobname "|\textit{target}|" "|[\textit{flags}]%
|\includeonly{|\textit{dest}|}\input{|\textit{main}|}"|
\end{center}
%

%%%%%%%%%%%%%%%%%%%%%%%%%%%%%%%%%%%%%%%%%%%%%%%%%%%%%%%%%%%%%%%%%%%%%%%%%%%%%%%%
%%%%%%%%%%%%%%%%%%%%%%%%%%%%%%%%%%%%%%%%%%%%%%%%%%%%%%%%%%%%%%%%%%%%%%%%%%%%%%%%
\section{Information}

%%%%%%%%%%%%%%%%%%%%%%%%%%%%%%%%%%%%%%%%%%%%%%%%%%%%%%%%%%%%%%%%%%%%%%%%%%%%%%%%
\subsection{Copyright}

Copyright \copyright{} 2017--2018 Niklas Beisert

This work may be distributed and/or modified under the
conditions of the \LaTeX{} Project Public License, either version 1.3
of this license or (at your option) any later version.
The latest version of this license is in
  \url{http://www.latex-project.org/lppl.txt}
and version 1.3 or later is part of all distributions of \LaTeX{}
version 2005/12/01 or later.

This work has the LPPL maintenance status `maintained'.

The Current Maintainer of this work is Niklas Beisert.

This work consists of the files |README.txt|, |childdoc.ins| and |childdoc.dtx|
as well as the derived files |childdoc.def|, |cdocsamp.tex|
with |cdocsch1.tex|, |cdocsch2.tex|, |cdocspt3.tex|, |cdocspt4.tex|,
|cdocsdrf.tex|, |cdocsfn1.tex|, |cdocsfn2.tex|
as well as |childdoc.pdf|.

%%%%%%%%%%%%%%%%%%%%%%%%%%%%%%%%%%%%%%%%%%%%%%%%%%%%%%%%%%%%%%%%%%%%%%%%%%%%%%%%
\subsection{Files and Installation}

The package consists of the files:
%
\begin{center}
\begin{tabular}{ll}
    |README.txt|   & readme file \\
    |childdoc.ins| & installation file \\
    |childdoc.dtx| & source file \\
    |childdoc.def| & definition file \\
    |cdocsamp.tex| & sample main file \\
    |cdocsch1.tex| & sample include file \\
    |cdocsch2.tex| & sample include file \\
    |cdocspt3.tex| & sample part file \\
    |cdocspt4.tex| & sample part file \\
    |cdocsdrf.tex| & sample redirection file \\
    |cdocsfn1.tex| & sample redirection file \\
    |cdocsfn2.tex| & sample redirection file \\
    |childdoc.pdf| & manual
\end{tabular}
\end{center}
%
The distribution consists of the files
|README.txt|, |childdoc.ins| and |childdoc.dtx|.
%
\begin{itemize}
\item
Run (pdf)\LaTeX{} on |childdoc.dtx|
to compile the manual |childdoc.pdf| (this file).
\item
Run \LaTeX{} on |childdoc.ins| to create the definitions file |childdoc.def|
and the sample |cdocsamp.tex| with include files
|cdocsch1.tex|, |cdocsch2.tex|, |cdocspt3.tex|, |cdocspt4.tex|,
|cdocsdrf.tex|, |cdocsfn1.tex|, |cdocsfn2.tex|.
Then copy the file |childdoc.def| to an appropriate directory of your \LaTeX{}
distribution, e.g.\ \textit{texmf-root}|/tex/latex/childdoc|.
\end{itemize}

%%%%%%%%%%%%%%%%%%%%%%%%%%%%%%%%%%%%%%%%%%%%%%%%%%%%%%%%%%%%%%%%%%%%%%%%%%%%%%%%
\subsection{Related CTAN Packages}

There are several other packages which offer a similar functionality:
%
\begin{itemize}
\item
The packages
\href{http://ctan.org/pkg/docmute}{\textsf{docmute}},
\href{http://ctan.org/pkg/includex}{\textsf{includex}} and
\href{http://ctan.org/pkg/standalone}{\textsf{standalone}}
provide commands to include only the document body of
a child file thus allowing both files to be compiled individually.
\item
The packages \href{http://ctan.org/pkg/subdocs}{\textsf{subdocs}}
and \href{http://ctan.org/pkg/subfiles}{\textsf{subfiles}}
provide structures in which the main and child documents can be
encapsulated and allowing them to be compiled individually.
The inclusion mechanism is different from the conventional |\include|.
\item
The package \href{http://ctan.org/pkg/combine}{\textsf{combine}}
is an elaborate solution to combine several documents into one.
\end{itemize}
%
See also the CTAN topic \href{http://ctan.org/topic/subdocs}{\textsf{subdocs}}
for further related packages.
The present package differs from the above solutions in that
a document structure constructed with the conventional |\include| mechanism
just needs two extra commands at the top of every file
such that all constituent files can be compiled individually.

%%%%%%%%%%%%%%%%%%%%%%%%%%%%%%%%%%%%%%%%%%%%%%%%%%%%%%%%%%%%%%%%%%%%%%%%%%%%%%%%
%\subsection{Feature Suggestions}
%
%The following is a list of features which may be useful for future
%versions of this package:
%%
%\begin{itemize}
%\item
%\ldots
%\end{itemize}

%%%%%%%%%%%%%%%%%%%%%%%%%%%%%%%%%%%%%%%%%%%%%%%%%%%%%%%%%%%%%%%%%%%%%%%%%%%%%%%%
\subsection{Revision History}

%%%%%%%%%%%%%%%%%%%%%%%%%%%%%%%%%%%%%%%%
\paragraph{v2.0:} 2018/12/30

\begin{itemize}
\item
immediate forward processing
\item
added |\childdocby| mechanism
\item
manual restructured
\end{itemize}

%%%%%%%%%%%%%%%%%%%%%%%%%%%%%%%%%%%%%%%%
\paragraph{v1.6:} 2018/01/17

\begin{itemize}
\item
application for development of include files
\item
corrections to manual
\end{itemize}

%%%%%%%%%%%%%%%%%%%%%%%%%%%%%%%%%%%%%%%%
\paragraph{v1.5:} 2017/05/21

\begin{itemize}
\item
more complete structuring introduced
\item
|\childdocof| introduced
\item
|\childdoc| renamed to |\childdocmain|
\item
|\childredirect| renamed to |\childdocforward| and |\childdocforwardprefix|
and functionality expanded
\end{itemize}

%%%%%%%%%%%%%%%%%%%%%%%%%%%%%%%%%%%%%%%%
\paragraph{v1.0:} 2017/04/27

\begin{itemize}
\item
manual and install package
\item
first version published on CTAN
\end{itemize}

%%%%%%%%%%%%%%%%%%%%%%%%%%%%%%%%%%%%%%%%
\paragraph{v0.6:} 2017/04/26

\begin{itemize}
\item
redirection mechanism added
\end{itemize}

%%%%%%%%%%%%%%%%%%%%%%%%%%%%%%%%%%%%%%%%
\paragraph{v0.5:} 2017/04/26

\begin{itemize}
\item
functionality in definition file
\end{itemize}


%%%%%%%%%%%%%%%%%%%%%%%%%%%%%%%%%%%%%%%%%%%%%%%%%%%%%%%%%%%%%%%%%%%%%%%%%%%%%%%%
%%%%%%%%%%%%%%%%%%%%%%%%%%%%%%%%%%%%%%%%%%%%%%%%%%%%%%%%%%%%%%%%%%%%%%%%%%%%%%%%
%%%%%%%%%%%%%%%%%%%%%%%%%%%%%%%%%%%%%%%%%%%%%%%%%%%%%%%%%%%%%%%%%%%%%%%%%%%%%%%%
\appendix

\settowidth\MacroIndent{\rmfamily\scriptsize 000\ }

 \DocInput{childdoc.dtx}

\end{document}
%</driver>
% \fi
%
% %%%%%%%%%%%%%%%%%%%%%%%%%%%%%%%%%%%%%%%%%%%%%%%%%%%%%%%%%%%%%%%%%%%%%%%%%%%%%%
% %%%%%%%%%%%%%%%%%%%%%%%%%%%%%%%%%%%%%%%%%%%%%%%%%%%%%%%%%%%%%%%%%%%%%%%%%%%%%%
% \section{Sample}
%\iffalse
%<*samplemain>
%\fi
%
% The following presents a sample document
% with two chapters, two parts, a title page,
% a compile flag as well as three forwarding files to set the flag.
% It consists of eight |.tex| files:
% \begin{center}
% \begin{tabular}{ll}
% |cdocsamp.tex|&main file\\
% |cdocsch1.tex|&include file for chapter 1\\
% |cdocsch2.tex|&include file for chapter 2\\
% |cdocspt3.tex|&include file for part 3\\
% |cdocspt4.tex|&include file for part 4\\
% |cdocsdrf.tex|&forwarding file for main file in draft mode\\
% |cdocsfi1.tex|&forwarding file for final version of chapter 1\\
% |cdocsfi2.tex|&forwarding file for final version of chapter 2\\
% \end{tabular}
% \end{center}
% Each of the eight files can be compiled directly by the \LaTeX{} compiler.
%
% %%%%%%%%%%%%%%%%%%%%%%%%%%%%%%%%%%%%%%
% \paragraph{Main File.}
%
% The main file is called |cdocsamp.tex|.
%
% Load the \textsf{childdoc} definitions and
% declare the filename for the main document:
%    \begin{macrocode}
\input{childdoc.def}
\childdocmain{}
%    \end{macrocode}

% Optional override for |\version| flag:
%    \begin{macrocode}
%%\ifchilddoc\else\providecommand{\version}{draft}\fi
%    \end{macrocode}

% Define the default values for the |\version| flag
% (|final| for the main file and |draft| for childs):
%    \begin{macrocode}
\ifchilddoc
\providecommand{\version}{draft}
\else
\providecommand{\version}{final}
\fi
%    \end{macrocode}

% Load the standard document class:
%    \begin{macrocode}
\documentclass[12pt]{article}
%    \end{macrocode}

% Start the document body:
%    \begin{macrocode}
\begin{document}
%    \end{macrocode}

% Declare a title page.
% Print title, part of document being processed and version flag:
%    \begin{macrocode}
\addtocounter{page}{-1}
\begin{center}
{\LARGE\bfseries{}childdoc example\par}
\vspace{1cm}
\ifchilddoc
\ifchilddocmanual part\else chapter\fi:
`\childdocname' of `\childdocjob'\par
\else
main document: `\childdocjob'\par
\fi
version: \version\par
\end{center}
\newpage
%    \end{macrocode}

% Manually include selected file,
% otherwise process as usual:
%    \begin{macrocode}
\ifchilddocmanual
\section*{part `\childdocname'}
\input{\childdocname}
\else
%    \end{macrocode}

% Include the two chapters:
%    \begin{macrocode}
\include{cdocsch1}
\include{cdocsch2}
%    \end{macrocode}

% Include the two parts unless only chapters should be displayed:
%    \begin{macrocode}
\ifchilddoc\else
\section{part three}
\input{cdocspt3}
\section{part four}
\input{cdocspt4}
\fi
%    \end{macrocode}

% Process as usual until here:
%    \begin{macrocode}
\fi
%    \end{macrocode}

% End of document body:
%    \begin{macrocode}
\end{document}
%    \end{macrocode}
%\iffalse
%</samplemain>
%\fi
%
% %%%%%%%%%%%%%%%%%%%%%%%%%%%%%%%%%%%%%%
% \paragraph{Chapter Include Files.}
%
% The include files are called |cdocsch1.tex| and |cdocsch2.tex|.
%
%\iffalse
%<*samplechap1|samplechap2>
%\fi

% Optional override for |\version| flag:
%    \begin{macrocode}
%%\providecommand{\version}{final}
%    \end{macrocode}

% Include the main document:
%    \begin{macrocode}
\input{childdoc.def}
\childdocof{cdocsamp}
%    \end{macrocode}

%\iffalse
%</samplechap1|samplechap2>
%\fi
%
%\iffalse
%<*samplechap1>
%\fi
% Some text for chapter 1:
%    \begin{macrocode}
\section{one}
some text in chapter one
%    \end{macrocode}

%\iffalse
%</samplechap1>
%\fi
% Some text for chapter 2:
%\iffalse
%<*samplechap2>
%\fi
%    \begin{macrocode}
\section{two}
more text in chapter two
%    \end{macrocode}

%\iffalse
%</samplechap2>
%\fi
%
% %%%%%%%%%%%%%%%%%%%%%%%%%%%%%%%%%%%%%%
% \paragraph{Part Include Files.}
%
% The include files are called |cdocspt3.tex| and |cdocspt4.tex|.
%
%\iffalse
%<*samplepart3|samplepart4>
%\fi

% Optional override for |\version| flag:
%    \begin{macrocode}
%%\providecommand{\version}{final}
%    \end{macrocode}

% Include the main document:
%    \begin{macrocode}
\input{childdoc.def}
\childdocby{cdocsamp}
%    \end{macrocode}

%\iffalse
%</samplepart3|samplepart4>
%\fi
%
%\iffalse
%<*samplepart3>
%\fi
% Some text for part 3:
%    \begin{macrocode}
some text in part three
%    \end{macrocode}

%\iffalse
%</samplepart3>
%\fi
% Some text for part 4:
%\iffalse
%<*samplepart4>
%\fi
%    \begin{macrocode}
more text in part four
%    \end{macrocode}

%\iffalse
%</samplepart4>
%\fi
%
% %%%%%%%%%%%%%%%%%%%%%%%%%%%%%%%%%%%%%%
% \paragraph{Forwarding for a Complete Draft.}
%
% The following forwarding file |cdocsdrf.tex|
% compiles the main document in draft mode:
%\iffalse
%<*sampledraft>
%\fi
%    \begin{macrocode}
\def\version{draft}
\input{childdoc.def}
\childdocforward{cdocsamp}
%    \end{macrocode}

%\iffalse
%</sampledraft>
%\fi
%
% %%%%%%%%%%%%%%%%%%%%%%%%%%%%%%%%%%%%%%
% \paragraph{Forwarding for Final Version of the Chapters.}
%
% The following forwarding files |cdocsfn1.tex| and |cdocsfn2.tex|
% (with identical content)
% compile the final versions of the child documents
% |cdocsch1.tex| and |cdocsch2.tex|, respectively:
%\iffalse
%<*samplefinal>
%\fi
%    \begin{macrocode}
\def\version{final}
\input{childdoc.def}
\childdocforwardprefix[cdocsamp]{cdocsfn}{cdocsch}
%    \end{macrocode}

%\iffalse
%</samplefinal>
%\fi
%
% %%%%%%%%%%%%%%%%%%%%%%%%%%%%%%%%%%%%%%
% \paragraph{Command Line Processing.}
%
% The following three command lines generate the output files
% |cdocscld|, |cdocscl1| and |cdocscl2|
% which should be identical to
% |cdocsdrf|, |cdocsch1| and |cdocsfn2|, respectively:
% \begin{center}
% \begin{tabular}{l}
% |latex -jobname cdocscld \|\\
% |  "\def\version{draft}\input{childdoc.def}\childdocforward{cdocsamp}"|\\
% |latex -jobname cdocscl1 \|\\
% |  "\input{childdoc.def}\childdocforward[cdocsamp]{cdocsch1}"|\\
% |latex -jobname cdocscl2 \|\\
% |  "\def\version{final}\input{childdoc.def}\childdocforward{cdocsch2}"|
% \end{tabular}
% \end{center}
% Note that the trailing backslash on each first line
% merely continues the input to the second line
% (for convenient cut ant paste).
% Furthermore, the command |latex| can be replaced by any
% of its alternative versions such as |pdflatex|.
%
% %%%%%%%%%%%%%%%%%%%%%%%%%%%%%%%%%%%%%%%%%%%%%%%%%%%%%%%%%%%%%%%%%%%%%%%%%%%%%%
% %%%%%%%%%%%%%%%%%%%%%%%%%%%%%%%%%%%%%%%%%%%%%%%%%%%%%%%%%%%%%%%%%%%%%%%%%%%%%%
% \section{Implementation}
%\iffalse
%<*package>
%\fi
%
% This section describes the definitions file |childdoc.def|.

% The definitions cannot be loaded using |\usepackage| or |\RequirePackage|
% which has a mechanism to prevent loading a style file more than once.
% When loading the definitions by means of |\input|
% multiple instances have to be prevented manually:
%\iffalse
%This code needs to be before the `\ProvidesFile' directive
%which is defined at the beginning of this file.
%Therefore it is also placed there and commented out here.
%</package>
%<*discard>
%\fi
%    \begin{macrocode}
\ifdefined\childdocmain\endinput\fi
%    \end{macrocode}
%\iffalse
%</discard>
%<*package>
%\fi
%
% \macro{\ifchilddoc}
% \macro{\ifchilddocmanual}
% The conditional |\ifchilddoc| tells whether a
% child (true) or main (false) document is being compiled.
% The conditional |\ifchilddocmanual| tells whether
% the |\includeonly| mechanism is used (false) or
% the selection of child files must be performed manually (true).
% The definitions initialise to false:
%    \begin{macrocode}
\newif\ifchilddoc
\newif\ifchilddocmanual
%    \end{macrocode}

% \macro{\childdocname}
% \macro{\childdocjob}
% The macro |\childdocname| stores the name of the main document
% to be compiled. The macro |\childdocjob| stores the name of
% the document on which the \LaTeX{} compiler was originally invoked.
% The content of |\jobname| cannot be compared
% to filenames specified in the source due to different catcodes.
% The following code rescans |\jobname|, stores the result
% in |\childdocname| and saves a copy in |\childdocjob|:
%    \begin{macrocode}
\edef\childdocname{\scantokens\expandafter{\jobname\noexpand}}
\let\childdocjob\childdocname
%    \end{macrocode}

% \macro{\childdocdisable}
% The macro |\childdocdisable| prevents the main file
% from being processed more than once.
% At this stage, the main document command |\childdocmain|
% is assumed to be called once again where it should do nothing.
% Any subsequent call to it should prevent
% a secondary processing of the main document
% It overwrites the forwarding commands
% |\childdocof| and |\childdocforward|
% with empty macros to prevent further inclusions of the main document:
%    \begin{macrocode}
\newcommand{\childdocdisable}
{
  \renewcommand{\childdocmain}[1]{\renewcommand{\childdocmain}[1]{\endinput}}
  \renewcommand{\childdocof}[1]{}
  \renewcommand{\childdocby}[2][]{}
  \renewcommand{\childdocforward}[2][]{}
  \renewcommand{\childdocdisable}{}
}
%    \end{macrocode}

% \macro{\childdocmain}
% The macro |\childdocmain| is to be called at the top of the main file
% with nothing or the main filename (without extension) as argument.
% First, it breaks loops.
% If the argument is not empty and does not match |\childdocname|
% (which is set by the first inclusion of |childdoc.def|),
% |\ifchilddoc| is set to true, |\includeonly| is applied to the child file
% and |\jobname| is set to the main file
% (for proper handling of |.aux| files):
%    \begin{macrocode}
\newcommand{\childdocmain}[1]
{
  \childdocdisable\childdocmain{}
  \if?#1?\else
    \begingroup
      \def\childdoctmp{#1}
      \ifx\childdoctmp\childdocname
        \def\childdoctmp{}
      \else
        \def\childdoctmp
        {
          \childdoctrue
          \includeonly{\childdocname}
          \def\childdocjob{#1}
          \def\jobname{#1}
        }
      \fi
      \expandafter
    \endgroup
    \childdoctmp
  \fi
}
%    \end{macrocode}

% \macro{\childdocof}
% The command |\childdocof| redirects
% compilation to the main file |#1|.
%    \begin{macrocode}
\newcommand{\childdocof}[1]
{
  \childdocdisable
  \childdoctrue
  \includeonly{\childdocname}
  \def\jobname{#1}
  \def\childdocjob{#1}
  \input{#1}
}
%    \end{macrocode}

% \macro{\childdocby}
% The command |\childdocby| ....
%    \begin{macrocode}
\newcommand{\childdocby}[2][]
{
  \childdocdisable
  \childdoctrue
  \childdocmanualtrue
  \if?#1?\else
    \def\jobname{#2}
  \fi
  \def\childdocjob{#2}
  \input{#2}
  \endinput
}
%    \end{macrocode}

% \macro{\childdocforward}
% The command |\childdocforward| redirects
% compilation to the main file or
% (if the optional argument is given) a child file.
% Parameters are set as if the main file
% or a child file starting with |\childdocof| was compiled.
% Then compilation is handed over to the main file:
%    \begin{macrocode}
\newcommand{\childdocforward}[2][]
{
  \begingroup
    \if?#1?
      \def\childdoctmp
      {
        \def\childdocname{#2}
        \def\childdocjob{#2}
        \def\jobname{#2}
        \input{#2}
        \endinput
      }
    \else
      \def\childdoctmp
      {
        \childdocdisable
        \def\childdocname{#2}
        \childdoctrue
        \includeonly{#2}
        \def\childdocjob{#1}
        \def\jobname{#1}
        \input{#1}
        \endinput
      }
    \fi
    \expandafter
  \endgroup
  \childdoctmp
}
%    \end{macrocode}

% \macro{\childdocforwardprefix}
% The command |\childdocforwardprefix| redirects
% compilation to the main or a child file by means of a pattern.
% The prefix |#1| in the current filename is replaced by |#2|
% and the suffix of the current filename is kept
% (it is assumed that the filename does not contain the substring `|~~~|'
% which is used as a delimiter).
% Compilation is handed over to the new file by |\childdocforward|:
%    \begin{macrocode}
\newcommand{\childdocforwardprefix}[3][]
{
  \begingroup
    \def\childdocextract #2##1~~~{\def\childdoctmp{\childdocforward[#1]{#3##1}}}
    \expandafter\childdocextract\childdocname~~~
    \expandafter
  \endgroup
  \childdoctmp
}
%    \end{macrocode}

% \macro{\childdoc}
% The deprecated macro |\childdoc| is a legacy version of |\childdocmain|:
%    \begin{macrocode}
\newcommand{\childdoc}{\childdocmain}
%    \end{macrocode}

% \macro{\childdocredirect}
% The deprecated macro |\childdocredirect| is a legacy version
% of |\childdocforward| and |\childdocforwardprefix|:
%    \begin{macrocode}
\newcommand{\childdocredirect}[2][]
{
  \begingroup
    \if?#1?
      \def\childdoctmp{\childdocforward{#2}}
    \else
      \def\childdoctmp{\childdocforwardprefix{#1}{#2}}
    \fi
    \expandafter
  \endgroup
  \childdoctmp
}
%    \end{macrocode}

%\iffalse
%</package>
%\fi
%
\endinput

\childdocforward{cdocsamp}
%    \end{macrocode}

%\iffalse
%</sampledraft>
%\fi
%
% %%%%%%%%%%%%%%%%%%%%%%%%%%%%%%%%%%%%%%
% \paragraph{Forwarding for Final Version of the Chapters.}
%
% The following forwarding files |cdocsfn1.tex| and |cdocsfn2.tex|
% (with identical content)
% compile the final versions of the child documents
% |cdocsch1.tex| and |cdocsch2.tex|, respectively:
%\iffalse
%<*samplefinal>
%\fi
%    \begin{macrocode}
\def\version{final}
% \iffalse
%
% childdoc.dtx Copyright (C) 2017-2018 Niklas Beisert
%
% This work may be distributed and/or modified under the
% conditions of the LaTeX Project Public License, either version 1.3
% of this license or (at your option) any later version.
% The latest version of this license is in
%   http://www.latex-project.org/lppl.txt
% and version 1.3 or later is part of all distributions of LaTeX
% version 2005/12/01 or later.
%
% This work has the LPPL maintenance status `maintained'.
%
% The Current Maintainer of this work is Niklas Beisert.
%
% This work consists of the files childdoc.dtx and childdoc.ins
% and the derived files childdoc.def and cdocsamp.tex with
% cdocsch1.tex, cdocsch2.tex, cdocsdrf.tex, cdocsfn1.tex, cdocsfn2.tex.
%
%<package>\ifdefined\childdocmain\endinput\fi
%<package>\ProvidesFile{childdoc.def}[2018/12/30 v2.0 child document driver]
%<samplemain>\ProvidesFile{cdocsamp.tex}[2018/12/30 v2.0 sample for childdoc]
%<*driver>
%\ProvidesFile{childdoc.drv}[2018/12/30 v2.0 childdoc reference manual file]
\PassOptionsToClass{10pt,a4paper}{article}
\documentclass{ltxdoc}

\usepackage[margin=35mm]{geometry}
\usepackage{hyperref}
\usepackage{hyperxmp}
\usepackage[usenames]{color}

\hypersetup{colorlinks=true}
\hypersetup{pdfstartview=FitH}
\hypersetup{pdfpagemode=UseNone}
\hypersetup{pdfsource={}}
\hypersetup{pdflang={en-UK}}
\hypersetup{pdfcopyright={Copyright 2017-2018 Niklas Beisert.
  This work may be distributed and/or modified under the
  conditions of the LaTeX Project Public License, either version 1.3
  of this license or (at your option) any later version.}}
\hypersetup{pdflicenseurl={http://www.latex-project.org/lppl.txt}}
\hypersetup{pdfcontactaddress={ETH Zurich, ITP, HIT K,
  Wolfgang-Pauli-Strasse 27}}
\hypersetup{pdfcontactpostcode={8093}}
\hypersetup{pdfcontactcity={Zurich}}
\hypersetup{pdfcontactcountry={Switzerland}}
\hypersetup{pdfcontactemail={nbeisert@itp.phys.ethz.ch}}
\hypersetup{pdfcontacturl={http://people.phys.ethz.ch/\xmptilde nbeisert/}}

\newcommand{\secref}[1]{\hyperref[#1]{section \ref*{#1}}}

\parskip1ex
\parindent0pt
\let\olditemize\itemize
\def\itemize{\olditemize\parskip0pt}

\begin{document}

\title{The \textsf{childdoc} Package}
\hypersetup{pdftitle={The childdoc Package}}
\author{Niklas Beisert\\[2ex]
  Institut f\"ur Theoretische Physik\\
  Eidgen\"ossische Technische Hochschule Z\"urich\\
  Wolfgang-Pauli-Strasse 27, 8093 Z\"urich, Switzerland\\[1ex]
  \href{mailto:nbeisert@itp.phys.ethz.ch}
  {\texttt{nbeisert@itp.phys.ethz.ch}}}
\hypersetup{pdfauthor={Niklas Beisert}}
\hypersetup{pdfsubject={Manual for the LaTeX2e Package childdoc}}
\date{30 December 2018, \textsf{v2.0}}
\maketitle

\begin{abstract}\noindent
\textsf{childdoc} is a \LaTeXe{} package
that enables the direct compilation
of document sections included by |\include|
to individual files.
\end{abstract}

\begingroup
\parskip0ex
\tableofcontents
\endgroup

%%%%%%%%%%%%%%%%%%%%%%%%%%%%%%%%%%%%%%%%%%%%%%%%%%%%%%%%%%%%%%%%%%%%%%%%%%%%%%%%
%%%%%%%%%%%%%%%%%%%%%%%%%%%%%%%%%%%%%%%%%%%%%%%%%%%%%%%%%%%%%%%%%%%%%%%%%%%%%%%%
\section{Introduction}

\LaTeX{} provides a mechanism to structure a large document (such as a book)
into a main file and several child files (containing the chapters)
using the |\include| command.
This mechanism is beneficial for documents
which span hundreds of pages in order to
make the source file(s) more manageable.
Moreover, compilation can be restricted to
selected child files by means of the |\includeonly| command.
The latter feature can be used to reduce the compilation time while editing
(this was significantly more useful in the earlier days of \LaTeX{})
or to generate a smaller document which is easier to navigate.
Another application of |\includeonly| is to generate
documents consisting of selected parts of the complete document.

However, there are a few drawbacks of the plain |\include| mechanism:
\begin{itemize}
\item
The child files cannot be compiled on their own,
they can only be compiled via the main file.
A naive editing environment
(such as a text editor with an option
to have the current file processed by \LaTeX)
may require one to switch to the main file before compiling;
attempting to compile the child file produces errors.
\item
The main file must be modified (each time)
to adjust the |\includeonly| command
to the present needs. This easily leaves the main file in a messy state.
\item
The generated document will always carry the filename
of the main document. This is inconvenient if
several child files are to be compiled and
to be kept for distribution.
\end{itemize}

The present package provides a simple interface
to make child files individually compilable by \LaTeX{}.
Compiling a child file then has the same effect as compiling
the main file with an |\includeonly| command
to select the appropriate child.
Moreover the generated document will carry the name of the child
rather than the main file.
This resolves all three above issues.

This feature is meant to make the editing of books,
thesis documents and lecture notes somewhat more convenient.
However, the package can also be used efficiently for
composing a series of documents (such as exercise sheets)
which are typically distributed individually.
It then assists the author in generating the individual documents
(potentially in different versions)
as well as a document containing the collected series.
Another application is in developing style files
or other kinds of included material
where compilation of the style file could redirect
to a sample or test file.

%%%%%%%%%%%%%%%%%%%%%%%%%%%%%%%%%%%%%%%%%%%%%%%%%%%%%%%%%%%%%%%%%%%%%%%%%%%%%%%%
%%%%%%%%%%%%%%%%%%%%%%%%%%%%%%%%%%%%%%%%%%%%%%%%%%%%%%%%%%%%%%%%%%%%%%%%%%%%%%%%
\section{Usage}

First of all, the package \textsf{childdoc} is \emph{not} a standard
\LaTeXe{} |.sty| style file! Therefore it needs to be invoked in
a non-standard way.

%%%%%%%%%%%%%%%%%%%%%%%%%%%%%%%%%%%%%%%%%%%%%%%%%%%%%%%%%%%%%%%%%%%%%%%%%%%%%%%%
\subsection{Included Files}
\label{sec:include}

%%%%%%%%%%%%%%%%%%%%%%%%%%%%%%%%%%%%%%%%
\DescribeMacro{\childdocmain}
To use the package, add the commands
\begin{center}
\begin{tabular}{l}
|\input{childdoc.def}|\\
|\childdocmain{}|\\
\end{tabular}
\end{center}
at the very top of the main \LaTeX{} file,
in particular \emph{before} the |\documentclass| statement!
The argument of |\childdocmain| should be left empty
(but it must be present).

%%%%%%%%%%%%%%%%%%%%%%%%%%%%%%%%%%%%%%%%
\DescribeMacro{\childdocof}
Furthermore, add the commands
\begin{center}
\begin{tabular}{l}
|\input{childdoc.def}|\\
|\childdocof{|\textit{main}|}|\\
\end{tabular}
\end{center}
at the top of every child file \textit{child}
which is included by |\include{|\textit{child}|}|
from within the main file
(or at least for those files to be compiled individually).
The argument \textit{main} must be the filename of the main file.

There are a couple of
considerations in setting up the main and child documents:

%%%%%%%%%%%%%%%%%%%%%%%%%%%%%%%%%%%%%%%%
\paragraph{Restrictions.}

Please note the following restrictions:
\begin{itemize}
\item
|\childdocmain| must be called with one argument \textit{main}
to ensure compatibility with earlier version of the package.
It must either be empty (|\childdocmain{}|)
or precisely match the filename of the main file in which it is specified.
See \secref{sec:detection} for further information.
\item
The filename \textit{main} must be specified without the |.tex| extension.
\item
The filename \textit{main} is case sensitive
(even in case-insensitive file systems)
due to internal string comparison.
\item
The argument \textit{main} should be fully expanded, it cannot be a macro.
\item
Subdirectories and special characters should be avoided in filenames.
\item
The command |\childdocmain{|\textit{main}|}| must be followed by a whitespace.
It should not be followed immediately by another command
or by a comment mark `|%|'.
This is because the \TeX{} parser reads the token immediately following
the argument of |\childdocmain| and puts it
at the beginning of every child section;
however, a white\-space is ignored.
\end{itemize}

%%%%%%%%%%%%%%%%%%%%%%%%%%%%%%%%%%%%%%%%
\paragraph{Content of Main File.}

It is advisable to place all content in the child files included by |\include|.
Any output contained in the main file will appear in all child documents
unless suppressed manually;
it cannot be suppressed automatically by the |\includeonly| directive
and thus should normally be avoided.
A method to include some content in the main file
by means of conditional processing is described in \secref{sec:conditional}.

%%%%%%%%%%%%%%%%%%%%%%%%%%%%%%%%%%%%%%%%
\paragraph{Page Numbering.}

When only a part of the document is compiled,
the appropriate numbering of pages
(as well as other status parameters)
is determined from the |.aux| files.
The latter contain information from previous passes.
However this information needs to propagate through
all intermediate child documents.
Therefore the page numbering in child documents may well
be inconsistent until the complete document is compiled at least once.

A useful (if unconventional) way to always ensure a consistent
page numbering is to restart the numbering in each child document
and denote the pages by `\textit{child}|.|\textit{page}'
where \textit{child} represents the chapter/section number of the child file.
This can be achieved by the command
|\numberwithin{page}{|\textit{child}|}|
of the \textsf{amsmath} package
where \textit{child} can be |chapter| or |section|
depending on the chosen structuring.
Alternatively, one can modify the macro |\thepage| appropriately
and reset the counter |page| at the start of each child file.

%%%%%%%%%%%%%%%%%%%%%%%%%%%%%%%%%%%%%%%%%%%%%%%%%%%%%%%%%%%%%%%%%%%%%%%%%%%%%%%%
\subsection{Conditional Processing}
\label{sec:conditional}

The package provides a mechanism to compile different versions
of a document. To customise the versions further some conditional processing
can come in handy to distinguish which version is being compiled.
The package provides two macros to describe the compilation context:

%%%%%%%%%%%%%%%%%%%%%%%%%%%%%%%%%%%%%%%%
\DescribeMacro{\ifchilddoc}
The conditional |\ifchilddoc| distinguishes between the compilation of
child documents and the main document:
%
\begin{center}
|\ifchilddoc |\textit{child-code}| |[|\||else |\textit{main-code}]| \||fi|
\end{center}

%%%%%%%%%%%%%%%%%%%%%%%%%%%%%%%%%%%%%%%%
\DescribeMacro{\childdocname}
\DescribeMacro{\childdocjob}
The macro |\childdocname| contains the filename (without extension)
of the main or child file being processed.
Note that |\childdocjob| will always contain the name of the main file.

%%%%%%%%%%%%%%%%%%%%%%%%%%%%%%%%%%%%%%%%
\paragraph{Title Page.}

Conditional processing can be used to include a title or banner page
in the main document when proper precautions are taken.
Importantly, the code in the main file should ensure that the page counter
(as well as other status parameters which are stored in the |.aux| files)
takes the same value after the conditional processing.
Otherwise the page numbers may take divergent values
depending on which part is compiled.

For example, a title page could be declared by:
%
\begin{center}
\begin{tabular}{l}
|\ifchilddoc\||else|\\
|\addtocounter{page}{-1}|\\
\textit{code for title page}\\
|\newpage|\\
|\||fi|
\end{tabular}
\end{center}
%
A banner page for the child documents can be generated by:
%
\begin{center}
\begin{tabular}{l}
|\ifchilddoc|\\
|\addtocounter{page}{-1}|\\
\textit{code for banner page}\\
|\newpage|\\
|\||fi|
\end{tabular}
\end{center}
%
Here one could write a message such as:
\begin{center}
|This is the part \childdocname{} of \childdocjob{}.|
\end{center}

%%%%%%%%%%%%%%%%%%%%%%%%%%%%%%%%%%%%%%%%%%%%%%%%%%%%%%%%%%%%%%%%%%%%%%%%%%%%%%%%
\subsection{Flags}
\label{sec:flags}

The package makes it easy to generate different versions
of the main or child documents.
To this end compilation flags can be defined
and assigned different default values.
They will be particularly useful in conjunction
with the forwarding mechanism described in \secref{sec:forward}.

For example, it may be useful to have a flag |\version|
which can be set to |draft| or |final|.
The document source will contain some conditional code
depending on the value of |\version|.
Suppose further, the flag should default to |final| for the main file
and to |draft| for child files
which is a natural assignment for editing the document.
This is achieved by placing the following code
in the preamble of the main document
(below the |\childdocmain| directive):
%
\begin{center}
\begin{tabular}{l}
|\ifchilddoc|\\
|\providecommand{\version}{draft}|\\
|\||else|\\
|\providecommand{\version}{final}|\\
|\||fi|
\end{tabular}
\end{center}
%
The definition by |\providecommand| makes sure
that previous definitions are not overwritten.
Further statements |\providecommand{\version}{...}|
can thus be added before the above code to override it.

For the main file, one might add a line
(between |\childdocmain| and the above block)
%
\begin{center}
|%\ifchilddoc\||else\providecommand{\version}{draft}\||fi|
\end{center}
%
which can be uncommented to produce a draft version.
Likewise one can add a line to the very top of a child file
(above the |\childdocof{|\textit{main}|}| directive)
%
\begin{center}
|%\providecommand{\version}{final}|
\end{center}
%
which can be uncommented to produce the final version of this child document.

%%%%%%%%%%%%%%%%%%%%%%%%%%%%%%%%%%%%%%%%%%%%%%%%%%%%%%%%%%%%%%%%%%%%%%%%%%%%%%%%
\subsection{Forwarding}
\label{sec:forward}

Different versions of the main or child documents
using compilation flags as described in \secref{sec:flags}
can be (permanently) stored in different files
for convenient compilation, viewing and distribution.
To this end, the package defines a command
to pass on compilation to a different file:

%%%%%%%%%%%%%%%%%%%%%%%%%%%%%%%%%%%%%%%%
\DescribeMacro{\childdocforward}
The command |\childdocforward| redirects processing to
another source file:
%
\begin{center}
\begin{tabular}{l}
|\input{childdoc.def}|\\
|\childdocforward[|\textit{main}|]{|\textit{dest}|}|\\
\end{tabular}
\end{center}
%
The argument \textit{dest} is the destination file
(without extension).
It should be the main file or one of the child files.
Note that further \textsf{childdoc} directives
such as |\childdocof| and |\childdocforward|
in the indicated file will be processed in this form.
The optional argument \textit{main}
passes on directly to the main file \textit{main}
while pretending to compile the child \textit{dest}.
This form behaves as if \textit{dest}
issues |\childdocof{|\textit{main}|}| right away,
and no further \textsf{childdoc} directives will be processed.

%%%%%%%%%%%%%%%%%%%%%%%%%%%%%%%%%%%%%%%%
\DescribeMacro{\...prefix}
In the alternative form |\childdocforwardprefix|,
%
\begin{center}
\begin{tabular}{l}
|\input{childdoc.def}|\\
|\childdocforwardprefix[|\textit{main}|]{|\textit{prefix}|}{|\textit{dest}|}|
\end{tabular}
\end{center}
%
the destination file is determined by a pattern
depending on the current file:
To make this work, the current file must be called
`{\textit{prefix}\hspace{0.2em}\textit{suffix}}'
with \textit{prefix} matching precisely the argument.
Processing is then passed on to the file
`{\textit{dest}\hspace{0.2em}\textit{suffix}}'.
Surely, the same effect is achieved by
directly specifying the
argument `{\textit{dest}\hspace{0.2em}\textit{suffix}}'
in the first form.
However, that requires to set up a different file
for each child. With the alternative form of the command
all these files can have exactly the same content
which simplifies setting them up and maintaining them.

For example, the following file |draft.tex|
with a compilation flag |\version| as described in \secref{sec:flags}
compiles the main document as a draft:
%
\begin{center}
\begin{tabular}{l}
|\def\version{draft}|\\
|\input{childdoc.def}|\\
|\childdocforward{|\textit{main}|}|
\end{tabular}
\end{center}
%
Likewise, the following files |final|\textit{nn}|.tex|
compile the final version of the child document
|child|\textit{nn}|.tex|:
%
\begin{center}
\begin{tabular}{l}
|\def\version{final}|\\
|\input{childdoc.def}|\\
|\childdocforwardprefix{final}{child}|
\end{tabular}
\end{center}
%

Note that when several versions of a main file and/or of each child file
are to be generated, it may be convenient to set up a |Makefile| or
shell script to automatise the process.

%%%%%%%%%%%%%%%%%%%%%%%%%%%%%%%%%%%%%%%%%%%%%%%%%%%%%%%%%%%%%%%%%%%%%%%%%%%%%%%%
\subsection{Command Line Processing}
\label{sec:commandline}

The effect of redirection files can also be achieved by invoking
the \LaTeX{} compiler with a more elaborate command line.
Most conveniently this should be done as part
of a shell script or a |Makefile|.

When using \textsf{childdoc} in the main file, the following
command lines effectively perform a redirection
(note that depending on the shell being used,
backslashes may have to be doubled: `|\|' $\to$ `|\\|'):
%
\begin{center}
|... -jobname "|\textit{target}|" |\\|"|[\textit{flags}]%
|\input{childdoc.def}\childdocforward[|\textit{main}|]{|\textit{dest}|}"|
\end{center}
%
Here \textit{target} is the name of the output file,
\textit{main} is the name of the main file
and \textit{dest} is the name of the main or child file to be processed
(all filenames without extensions).
The optional argument \textit{main} can be omitted
if \textit{main} matches \textit{dest}.
Optionally, compilation \textit{flags} can be defined via |\def| commands.
This command line makes the \TeX{} engine believe
it is compiling the file \textit{target}
whose content is specified as the latter parameter.
The provided code then forwards the processing to
\textit{main} or \textit{dest} as described in \secref{sec:forward}.

%%%%%%%%%%%%%%%%%%%%%%%%%%%%%%%%%%%%%%%%%%%%%%%%%%%%%%%%%%%%%%%%%%%%%%%%%%%%%%%%
\subsection{Include by Input}
\label{sec:input}

Including child documents by |\include| has some restrictions by design.
Most notably, the content of a child document always occupies
its own set of pages; pages cannot be shared between child documents.
Usually, this behaviour makes perfect sense
because each child document contain an essential part of the document.
However, in some situations it may be desirable to compose
a document from a collection of parts
without having mandatory page breaks between then.
For this case, the package
provides a mechanism to include parts
by |\input| which can also be processed individually.
However, by construction this mechanism
requires manual handling of the content to be output.

%%%%%%%%%%%%%%%%%%%%%%%%%%%%%%%%%%%%%%%%
\DescribeMacro{\ifchilddocmanual}
The main file should be prepared as usual, see \secref{sec:include}.
However, the document body must make a distinction
between processing of an individual part and of the main document, e.g.:
%
\begin{center}
\begin{tabular}{l}
|\ifchilddocmanual|\\
|\input{\childdocname}|\\
|\||else|\\
\textit{document body with }|\input{|\textit{part}|}|\\
|\||fi|
\end{tabular}
\end{center}
%
The conditional |\ifchilddocmanual| is true whenever
a part to be included by |\input| is being compiled,
and the name of the part is stored in |\childdocname|.

%%%%%%%%%%%%%%%%%%%%%%%%%%%%%%%%%%%%%%%%
\DescribeMacro{\childdocby}
Each part to be included by |\input| should start with:
%
\begin{center}
\begin{tabular}{l}
|\input{childdoc.def}|\\
|\childdocby{|\textit{main}|}|\\
\end{tabular}
\end{center}
%
The directive |\childdocby| is similar to |\childdocof|
described in \secref{sec:include},
but the subsequent selection of content must be done manually.
To that end, both |\ifchilddoc| and |\ifchilddocmanual|
will be true upon processing of a part,
and the name of the part is stored in |\childdocname|.
Note that |\jobname| will be set to the filename of the current part
so that each part receives an individual |.aux| file
that does not interfere with the |.aux| file(s) of the main document.
This behaviour can be altered by the alternative form
|\childdocby[*]{|\textit{main}|}| (with a non-empty optional argument)
which uses the |.aux| file of the main document
by setting |\jobname| to \textit{main}.

%%%%%%%%%%%%%%%%%%%%%%%%%%%%%%%%%%%%%%%%%%%%%%%%%%%%%%%%%%%%%%%%%%%%%%%%%%%%%%%%
\subsection{Driver Development}
\label{sec:driver}

The \textsf{childdoc} mechanism can also be use for the development
of definition files such as \LaTeX{} styles or classes.
This case differs from the above setup with multiple parts
included by |\include| in that no |\includeonly| should be invoked.
This can be achieved by starting the include file
(before |\ProvidesPackage|) with:
%
\begin{center}
\begin{tabular}{l}
|\input{childdoc.def}|\\
|\childdocforward{|\textit{main}|}|\\
\end{tabular}
\end{center}
%
or alternatively with:
%
\begin{center}
\begin{tabular}{l}
|\input{childdoc.def}|\\
|\childdocby{|\textit{main}|}|\\
\end{tabular}
\end{center}
%
Both forms have slightly different effects as described above.
The main file is prepared as usual, see \secref{sec:include}.

%%%%%%%%%%%%%%%%%%%%%%%%%%%%%%%%%%%%%%%%%%%%%%%%%%%%%%%%%%%%%%%%%%%%%%%%%%%%%%%%
\subsection{Legacy Detection}
\label{sec:detection}

The directive |\childdocmain| in the main file can detect
whether the complete document or merely a child is to be compiled
even without using the directive |\childdocof|.
This method is deprecated because it is less robust
and there is no compelling reason to use it;
it is merely provided for backward compatibility
and it may be removed in future versions.

If the detection mechanism is to be used,
it is mandatory to correctly specify
the filename of the main file as the argument of |\childdocmain|:
%
\begin{center}
\begin{tabular}{l}
|\input{childdoc.def}|\\
|\childdocmain{|\textit{main}|}|\\
\end{tabular}
\end{center}
%
If |\jobname| does not match the argument \textit{main} of |\childdocmain|,
it is assumed that |\jobname| points to the child file to be compiled.
When using |\childdocmain| with the main file specified as argument,
it suffices to start a child file
with just |\input{|\textit{main}|}|
without loading of the package and using |\childdocof|.
If instead all processing is done
with the appropriate \textsf{childdoc} directives,
the argument of \textit{main} of |\childdocmain| can be empty.

An alternative version of the command line processing described
in \secref{sec:commandline} using the detection mechanism reads:
%
\begin{center}
|... -jobname "|\textit{target}|" "|[\textit{flags}]%
[|\def\jobname{|\textit{dest}|}|]|\input{|\textit{main}|}"|
\end{center}

%%%%%%%%%%%%%%%%%%%%%%%%%%%%%%%%%%%%%%%%%%%%%%%%%%%%%%%%%%%%%%%%%%%%%%%%%%%%%%%%
\subsection{Manual Code}
\label{sec:manual}

In case one cannot be certain whether the definitions file |childdoc.def|
is installed on the target \TeX{} distribution
and one prefers not to ship it,
it is conceivable to paste a few relevant commands into the sources.

To that end, drop all statements |\input{childdoc.def}|
and perform the replacements as outlined below.
Instead of |\childdocmain{|\textit{main}|}| add the following code
to the top of the main file:
%
\begin{center}
\begin{tabular}{l}
|\||ifdefined\childdocname\endinput\||fi\newif\ifchilddoc|\\
|\edef\childdocname{\scantokens\expandafter{\jobname\noexpand}}|\\
|\def\childdocmain{|\textit{main}|}\||ifx\childdocmain\childdocname\||else|\\
|\childdoctrue\includeonly{\childdocname}\let\jobname\childdocmain\||fi|\\
\end{tabular}
\end{center}
%
Instead of |\childdocof{|\textit{main}|}| just include the main file
at the top of each child file:
%
\begin{center}
|\input{|\textit{main}|}|
\end{center}
%
A simple redirection |\childdocforward{|\textit{dest}|}| is achieved by:
%
\begin{center}
|\def\jobname{|\textit{dest}|}\input{\jobname}|
\end{center}
%
The redirection with prefix
|\childdocforwardprefix[|\textit{prefix}|]{|\textit{dest}|}|
is accomplished by:
%
\begin{center}
\begin{tabular}{l}
|{\edef\jobname{\scantokens\expandafter{\jobname\noexpand}}|\\
|\def\redirectjob |\textit{prefix}|#1~~~{\gdef\jobname{|\textit{dest}|#1}}|\\
|\expandafter\redirectjob\jobname~~~}\input{\jobname}|
\end{tabular}
\end{center}

In an alternative approach,
child documents can be compiled by a specific command line
without additional code or specific definitions:
%
\begin{center}
|... -jobname "|\textit{target}|" "|[\textit{flags}]%
|\includeonly{|\textit{dest}|}\input{|\textit{main}|}"|
\end{center}
%

%%%%%%%%%%%%%%%%%%%%%%%%%%%%%%%%%%%%%%%%%%%%%%%%%%%%%%%%%%%%%%%%%%%%%%%%%%%%%%%%
%%%%%%%%%%%%%%%%%%%%%%%%%%%%%%%%%%%%%%%%%%%%%%%%%%%%%%%%%%%%%%%%%%%%%%%%%%%%%%%%
\section{Information}

%%%%%%%%%%%%%%%%%%%%%%%%%%%%%%%%%%%%%%%%%%%%%%%%%%%%%%%%%%%%%%%%%%%%%%%%%%%%%%%%
\subsection{Copyright}

Copyright \copyright{} 2017--2018 Niklas Beisert

This work may be distributed and/or modified under the
conditions of the \LaTeX{} Project Public License, either version 1.3
of this license or (at your option) any later version.
The latest version of this license is in
  \url{http://www.latex-project.org/lppl.txt}
and version 1.3 or later is part of all distributions of \LaTeX{}
version 2005/12/01 or later.

This work has the LPPL maintenance status `maintained'.

The Current Maintainer of this work is Niklas Beisert.

This work consists of the files |README.txt|, |childdoc.ins| and |childdoc.dtx|
as well as the derived files |childdoc.def|, |cdocsamp.tex|
with |cdocsch1.tex|, |cdocsch2.tex|, |cdocspt3.tex|, |cdocspt4.tex|,
|cdocsdrf.tex|, |cdocsfn1.tex|, |cdocsfn2.tex|
as well as |childdoc.pdf|.

%%%%%%%%%%%%%%%%%%%%%%%%%%%%%%%%%%%%%%%%%%%%%%%%%%%%%%%%%%%%%%%%%%%%%%%%%%%%%%%%
\subsection{Files and Installation}

The package consists of the files:
%
\begin{center}
\begin{tabular}{ll}
    |README.txt|   & readme file \\
    |childdoc.ins| & installation file \\
    |childdoc.dtx| & source file \\
    |childdoc.def| & definition file \\
    |cdocsamp.tex| & sample main file \\
    |cdocsch1.tex| & sample include file \\
    |cdocsch2.tex| & sample include file \\
    |cdocspt3.tex| & sample part file \\
    |cdocspt4.tex| & sample part file \\
    |cdocsdrf.tex| & sample redirection file \\
    |cdocsfn1.tex| & sample redirection file \\
    |cdocsfn2.tex| & sample redirection file \\
    |childdoc.pdf| & manual
\end{tabular}
\end{center}
%
The distribution consists of the files
|README.txt|, |childdoc.ins| and |childdoc.dtx|.
%
\begin{itemize}
\item
Run (pdf)\LaTeX{} on |childdoc.dtx|
to compile the manual |childdoc.pdf| (this file).
\item
Run \LaTeX{} on |childdoc.ins| to create the definitions file |childdoc.def|
and the sample |cdocsamp.tex| with include files
|cdocsch1.tex|, |cdocsch2.tex|, |cdocspt3.tex|, |cdocspt4.tex|,
|cdocsdrf.tex|, |cdocsfn1.tex|, |cdocsfn2.tex|.
Then copy the file |childdoc.def| to an appropriate directory of your \LaTeX{}
distribution, e.g.\ \textit{texmf-root}|/tex/latex/childdoc|.
\end{itemize}

%%%%%%%%%%%%%%%%%%%%%%%%%%%%%%%%%%%%%%%%%%%%%%%%%%%%%%%%%%%%%%%%%%%%%%%%%%%%%%%%
\subsection{Related CTAN Packages}

There are several other packages which offer a similar functionality:
%
\begin{itemize}
\item
The packages
\href{http://ctan.org/pkg/docmute}{\textsf{docmute}},
\href{http://ctan.org/pkg/includex}{\textsf{includex}} and
\href{http://ctan.org/pkg/standalone}{\textsf{standalone}}
provide commands to include only the document body of
a child file thus allowing both files to be compiled individually.
\item
The packages \href{http://ctan.org/pkg/subdocs}{\textsf{subdocs}}
and \href{http://ctan.org/pkg/subfiles}{\textsf{subfiles}}
provide structures in which the main and child documents can be
encapsulated and allowing them to be compiled individually.
The inclusion mechanism is different from the conventional |\include|.
\item
The package \href{http://ctan.org/pkg/combine}{\textsf{combine}}
is an elaborate solution to combine several documents into one.
\end{itemize}
%
See also the CTAN topic \href{http://ctan.org/topic/subdocs}{\textsf{subdocs}}
for further related packages.
The present package differs from the above solutions in that
a document structure constructed with the conventional |\include| mechanism
just needs two extra commands at the top of every file
such that all constituent files can be compiled individually.

%%%%%%%%%%%%%%%%%%%%%%%%%%%%%%%%%%%%%%%%%%%%%%%%%%%%%%%%%%%%%%%%%%%%%%%%%%%%%%%%
%\subsection{Feature Suggestions}
%
%The following is a list of features which may be useful for future
%versions of this package:
%%
%\begin{itemize}
%\item
%\ldots
%\end{itemize}

%%%%%%%%%%%%%%%%%%%%%%%%%%%%%%%%%%%%%%%%%%%%%%%%%%%%%%%%%%%%%%%%%%%%%%%%%%%%%%%%
\subsection{Revision History}

%%%%%%%%%%%%%%%%%%%%%%%%%%%%%%%%%%%%%%%%
\paragraph{v2.0:} 2018/12/30

\begin{itemize}
\item
immediate forward processing
\item
added |\childdocby| mechanism
\item
manual restructured
\end{itemize}

%%%%%%%%%%%%%%%%%%%%%%%%%%%%%%%%%%%%%%%%
\paragraph{v1.6:} 2018/01/17

\begin{itemize}
\item
application for development of include files
\item
corrections to manual
\end{itemize}

%%%%%%%%%%%%%%%%%%%%%%%%%%%%%%%%%%%%%%%%
\paragraph{v1.5:} 2017/05/21

\begin{itemize}
\item
more complete structuring introduced
\item
|\childdocof| introduced
\item
|\childdoc| renamed to |\childdocmain|
\item
|\childredirect| renamed to |\childdocforward| and |\childdocforwardprefix|
and functionality expanded
\end{itemize}

%%%%%%%%%%%%%%%%%%%%%%%%%%%%%%%%%%%%%%%%
\paragraph{v1.0:} 2017/04/27

\begin{itemize}
\item
manual and install package
\item
first version published on CTAN
\end{itemize}

%%%%%%%%%%%%%%%%%%%%%%%%%%%%%%%%%%%%%%%%
\paragraph{v0.6:} 2017/04/26

\begin{itemize}
\item
redirection mechanism added
\end{itemize}

%%%%%%%%%%%%%%%%%%%%%%%%%%%%%%%%%%%%%%%%
\paragraph{v0.5:} 2017/04/26

\begin{itemize}
\item
functionality in definition file
\end{itemize}


%%%%%%%%%%%%%%%%%%%%%%%%%%%%%%%%%%%%%%%%%%%%%%%%%%%%%%%%%%%%%%%%%%%%%%%%%%%%%%%%
%%%%%%%%%%%%%%%%%%%%%%%%%%%%%%%%%%%%%%%%%%%%%%%%%%%%%%%%%%%%%%%%%%%%%%%%%%%%%%%%
%%%%%%%%%%%%%%%%%%%%%%%%%%%%%%%%%%%%%%%%%%%%%%%%%%%%%%%%%%%%%%%%%%%%%%%%%%%%%%%%
\appendix

\settowidth\MacroIndent{\rmfamily\scriptsize 000\ }

 \DocInput{childdoc.dtx}

\end{document}
%</driver>
% \fi
%
% %%%%%%%%%%%%%%%%%%%%%%%%%%%%%%%%%%%%%%%%%%%%%%%%%%%%%%%%%%%%%%%%%%%%%%%%%%%%%%
% %%%%%%%%%%%%%%%%%%%%%%%%%%%%%%%%%%%%%%%%%%%%%%%%%%%%%%%%%%%%%%%%%%%%%%%%%%%%%%
% \section{Sample}
%\iffalse
%<*samplemain>
%\fi
%
% The following presents a sample document
% with two chapters, two parts, a title page,
% a compile flag as well as three forwarding files to set the flag.
% It consists of eight |.tex| files:
% \begin{center}
% \begin{tabular}{ll}
% |cdocsamp.tex|&main file\\
% |cdocsch1.tex|&include file for chapter 1\\
% |cdocsch2.tex|&include file for chapter 2\\
% |cdocspt3.tex|&include file for part 3\\
% |cdocspt4.tex|&include file for part 4\\
% |cdocsdrf.tex|&forwarding file for main file in draft mode\\
% |cdocsfi1.tex|&forwarding file for final version of chapter 1\\
% |cdocsfi2.tex|&forwarding file for final version of chapter 2\\
% \end{tabular}
% \end{center}
% Each of the eight files can be compiled directly by the \LaTeX{} compiler.
%
% %%%%%%%%%%%%%%%%%%%%%%%%%%%%%%%%%%%%%%
% \paragraph{Main File.}
%
% The main file is called |cdocsamp.tex|.
%
% Load the \textsf{childdoc} definitions and
% declare the filename for the main document:
%    \begin{macrocode}
\input{childdoc.def}
\childdocmain{}
%    \end{macrocode}

% Optional override for |\version| flag:
%    \begin{macrocode}
%%\ifchilddoc\else\providecommand{\version}{draft}\fi
%    \end{macrocode}

% Define the default values for the |\version| flag
% (|final| for the main file and |draft| for childs):
%    \begin{macrocode}
\ifchilddoc
\providecommand{\version}{draft}
\else
\providecommand{\version}{final}
\fi
%    \end{macrocode}

% Load the standard document class:
%    \begin{macrocode}
\documentclass[12pt]{article}
%    \end{macrocode}

% Start the document body:
%    \begin{macrocode}
\begin{document}
%    \end{macrocode}

% Declare a title page.
% Print title, part of document being processed and version flag:
%    \begin{macrocode}
\addtocounter{page}{-1}
\begin{center}
{\LARGE\bfseries{}childdoc example\par}
\vspace{1cm}
\ifchilddoc
\ifchilddocmanual part\else chapter\fi:
`\childdocname' of `\childdocjob'\par
\else
main document: `\childdocjob'\par
\fi
version: \version\par
\end{center}
\newpage
%    \end{macrocode}

% Manually include selected file,
% otherwise process as usual:
%    \begin{macrocode}
\ifchilddocmanual
\section*{part `\childdocname'}
\input{\childdocname}
\else
%    \end{macrocode}

% Include the two chapters:
%    \begin{macrocode}
\include{cdocsch1}
\include{cdocsch2}
%    \end{macrocode}

% Include the two parts unless only chapters should be displayed:
%    \begin{macrocode}
\ifchilddoc\else
\section{part three}
\input{cdocspt3}
\section{part four}
\input{cdocspt4}
\fi
%    \end{macrocode}

% Process as usual until here:
%    \begin{macrocode}
\fi
%    \end{macrocode}

% End of document body:
%    \begin{macrocode}
\end{document}
%    \end{macrocode}
%\iffalse
%</samplemain>
%\fi
%
% %%%%%%%%%%%%%%%%%%%%%%%%%%%%%%%%%%%%%%
% \paragraph{Chapter Include Files.}
%
% The include files are called |cdocsch1.tex| and |cdocsch2.tex|.
%
%\iffalse
%<*samplechap1|samplechap2>
%\fi

% Optional override for |\version| flag:
%    \begin{macrocode}
%%\providecommand{\version}{final}
%    \end{macrocode}

% Include the main document:
%    \begin{macrocode}
\input{childdoc.def}
\childdocof{cdocsamp}
%    \end{macrocode}

%\iffalse
%</samplechap1|samplechap2>
%\fi
%
%\iffalse
%<*samplechap1>
%\fi
% Some text for chapter 1:
%    \begin{macrocode}
\section{one}
some text in chapter one
%    \end{macrocode}

%\iffalse
%</samplechap1>
%\fi
% Some text for chapter 2:
%\iffalse
%<*samplechap2>
%\fi
%    \begin{macrocode}
\section{two}
more text in chapter two
%    \end{macrocode}

%\iffalse
%</samplechap2>
%\fi
%
% %%%%%%%%%%%%%%%%%%%%%%%%%%%%%%%%%%%%%%
% \paragraph{Part Include Files.}
%
% The include files are called |cdocspt3.tex| and |cdocspt4.tex|.
%
%\iffalse
%<*samplepart3|samplepart4>
%\fi

% Optional override for |\version| flag:
%    \begin{macrocode}
%%\providecommand{\version}{final}
%    \end{macrocode}

% Include the main document:
%    \begin{macrocode}
\input{childdoc.def}
\childdocby{cdocsamp}
%    \end{macrocode}

%\iffalse
%</samplepart3|samplepart4>
%\fi
%
%\iffalse
%<*samplepart3>
%\fi
% Some text for part 3:
%    \begin{macrocode}
some text in part three
%    \end{macrocode}

%\iffalse
%</samplepart3>
%\fi
% Some text for part 4:
%\iffalse
%<*samplepart4>
%\fi
%    \begin{macrocode}
more text in part four
%    \end{macrocode}

%\iffalse
%</samplepart4>
%\fi
%
% %%%%%%%%%%%%%%%%%%%%%%%%%%%%%%%%%%%%%%
% \paragraph{Forwarding for a Complete Draft.}
%
% The following forwarding file |cdocsdrf.tex|
% compiles the main document in draft mode:
%\iffalse
%<*sampledraft>
%\fi
%    \begin{macrocode}
\def\version{draft}
\input{childdoc.def}
\childdocforward{cdocsamp}
%    \end{macrocode}

%\iffalse
%</sampledraft>
%\fi
%
% %%%%%%%%%%%%%%%%%%%%%%%%%%%%%%%%%%%%%%
% \paragraph{Forwarding for Final Version of the Chapters.}
%
% The following forwarding files |cdocsfn1.tex| and |cdocsfn2.tex|
% (with identical content)
% compile the final versions of the child documents
% |cdocsch1.tex| and |cdocsch2.tex|, respectively:
%\iffalse
%<*samplefinal>
%\fi
%    \begin{macrocode}
\def\version{final}
\input{childdoc.def}
\childdocforwardprefix[cdocsamp]{cdocsfn}{cdocsch}
%    \end{macrocode}

%\iffalse
%</samplefinal>
%\fi
%
% %%%%%%%%%%%%%%%%%%%%%%%%%%%%%%%%%%%%%%
% \paragraph{Command Line Processing.}
%
% The following three command lines generate the output files
% |cdocscld|, |cdocscl1| and |cdocscl2|
% which should be identical to
% |cdocsdrf|, |cdocsch1| and |cdocsfn2|, respectively:
% \begin{center}
% \begin{tabular}{l}
% |latex -jobname cdocscld \|\\
% |  "\def\version{draft}\input{childdoc.def}\childdocforward{cdocsamp}"|\\
% |latex -jobname cdocscl1 \|\\
% |  "\input{childdoc.def}\childdocforward[cdocsamp]{cdocsch1}"|\\
% |latex -jobname cdocscl2 \|\\
% |  "\def\version{final}\input{childdoc.def}\childdocforward{cdocsch2}"|
% \end{tabular}
% \end{center}
% Note that the trailing backslash on each first line
% merely continues the input to the second line
% (for convenient cut ant paste).
% Furthermore, the command |latex| can be replaced by any
% of its alternative versions such as |pdflatex|.
%
% %%%%%%%%%%%%%%%%%%%%%%%%%%%%%%%%%%%%%%%%%%%%%%%%%%%%%%%%%%%%%%%%%%%%%%%%%%%%%%
% %%%%%%%%%%%%%%%%%%%%%%%%%%%%%%%%%%%%%%%%%%%%%%%%%%%%%%%%%%%%%%%%%%%%%%%%%%%%%%
% \section{Implementation}
%\iffalse
%<*package>
%\fi
%
% This section describes the definitions file |childdoc.def|.

% The definitions cannot be loaded using |\usepackage| or |\RequirePackage|
% which has a mechanism to prevent loading a style file more than once.
% When loading the definitions by means of |\input|
% multiple instances have to be prevented manually:
%\iffalse
%This code needs to be before the `\ProvidesFile' directive
%which is defined at the beginning of this file.
%Therefore it is also placed there and commented out here.
%</package>
%<*discard>
%\fi
%    \begin{macrocode}
\ifdefined\childdocmain\endinput\fi
%    \end{macrocode}
%\iffalse
%</discard>
%<*package>
%\fi
%
% \macro{\ifchilddoc}
% \macro{\ifchilddocmanual}
% The conditional |\ifchilddoc| tells whether a
% child (true) or main (false) document is being compiled.
% The conditional |\ifchilddocmanual| tells whether
% the |\includeonly| mechanism is used (false) or
% the selection of child files must be performed manually (true).
% The definitions initialise to false:
%    \begin{macrocode}
\newif\ifchilddoc
\newif\ifchilddocmanual
%    \end{macrocode}

% \macro{\childdocname}
% \macro{\childdocjob}
% The macro |\childdocname| stores the name of the main document
% to be compiled. The macro |\childdocjob| stores the name of
% the document on which the \LaTeX{} compiler was originally invoked.
% The content of |\jobname| cannot be compared
% to filenames specified in the source due to different catcodes.
% The following code rescans |\jobname|, stores the result
% in |\childdocname| and saves a copy in |\childdocjob|:
%    \begin{macrocode}
\edef\childdocname{\scantokens\expandafter{\jobname\noexpand}}
\let\childdocjob\childdocname
%    \end{macrocode}

% \macro{\childdocdisable}
% The macro |\childdocdisable| prevents the main file
% from being processed more than once.
% At this stage, the main document command |\childdocmain|
% is assumed to be called once again where it should do nothing.
% Any subsequent call to it should prevent
% a secondary processing of the main document
% It overwrites the forwarding commands
% |\childdocof| and |\childdocforward|
% with empty macros to prevent further inclusions of the main document:
%    \begin{macrocode}
\newcommand{\childdocdisable}
{
  \renewcommand{\childdocmain}[1]{\renewcommand{\childdocmain}[1]{\endinput}}
  \renewcommand{\childdocof}[1]{}
  \renewcommand{\childdocby}[2][]{}
  \renewcommand{\childdocforward}[2][]{}
  \renewcommand{\childdocdisable}{}
}
%    \end{macrocode}

% \macro{\childdocmain}
% The macro |\childdocmain| is to be called at the top of the main file
% with nothing or the main filename (without extension) as argument.
% First, it breaks loops.
% If the argument is not empty and does not match |\childdocname|
% (which is set by the first inclusion of |childdoc.def|),
% |\ifchilddoc| is set to true, |\includeonly| is applied to the child file
% and |\jobname| is set to the main file
% (for proper handling of |.aux| files):
%    \begin{macrocode}
\newcommand{\childdocmain}[1]
{
  \childdocdisable\childdocmain{}
  \if?#1?\else
    \begingroup
      \def\childdoctmp{#1}
      \ifx\childdoctmp\childdocname
        \def\childdoctmp{}
      \else
        \def\childdoctmp
        {
          \childdoctrue
          \includeonly{\childdocname}
          \def\childdocjob{#1}
          \def\jobname{#1}
        }
      \fi
      \expandafter
    \endgroup
    \childdoctmp
  \fi
}
%    \end{macrocode}

% \macro{\childdocof}
% The command |\childdocof| redirects
% compilation to the main file |#1|.
%    \begin{macrocode}
\newcommand{\childdocof}[1]
{
  \childdocdisable
  \childdoctrue
  \includeonly{\childdocname}
  \def\jobname{#1}
  \def\childdocjob{#1}
  \input{#1}
}
%    \end{macrocode}

% \macro{\childdocby}
% The command |\childdocby| ....
%    \begin{macrocode}
\newcommand{\childdocby}[2][]
{
  \childdocdisable
  \childdoctrue
  \childdocmanualtrue
  \if?#1?\else
    \def\jobname{#2}
  \fi
  \def\childdocjob{#2}
  \input{#2}
  \endinput
}
%    \end{macrocode}

% \macro{\childdocforward}
% The command |\childdocforward| redirects
% compilation to the main file or
% (if the optional argument is given) a child file.
% Parameters are set as if the main file
% or a child file starting with |\childdocof| was compiled.
% Then compilation is handed over to the main file:
%    \begin{macrocode}
\newcommand{\childdocforward}[2][]
{
  \begingroup
    \if?#1?
      \def\childdoctmp
      {
        \def\childdocname{#2}
        \def\childdocjob{#2}
        \def\jobname{#2}
        \input{#2}
        \endinput
      }
    \else
      \def\childdoctmp
      {
        \childdocdisable
        \def\childdocname{#2}
        \childdoctrue
        \includeonly{#2}
        \def\childdocjob{#1}
        \def\jobname{#1}
        \input{#1}
        \endinput
      }
    \fi
    \expandafter
  \endgroup
  \childdoctmp
}
%    \end{macrocode}

% \macro{\childdocforwardprefix}
% The command |\childdocforwardprefix| redirects
% compilation to the main or a child file by means of a pattern.
% The prefix |#1| in the current filename is replaced by |#2|
% and the suffix of the current filename is kept
% (it is assumed that the filename does not contain the substring `|~~~|'
% which is used as a delimiter).
% Compilation is handed over to the new file by |\childdocforward|:
%    \begin{macrocode}
\newcommand{\childdocforwardprefix}[3][]
{
  \begingroup
    \def\childdocextract #2##1~~~{\def\childdoctmp{\childdocforward[#1]{#3##1}}}
    \expandafter\childdocextract\childdocname~~~
    \expandafter
  \endgroup
  \childdoctmp
}
%    \end{macrocode}

% \macro{\childdoc}
% The deprecated macro |\childdoc| is a legacy version of |\childdocmain|:
%    \begin{macrocode}
\newcommand{\childdoc}{\childdocmain}
%    \end{macrocode}

% \macro{\childdocredirect}
% The deprecated macro |\childdocredirect| is a legacy version
% of |\childdocforward| and |\childdocforwardprefix|:
%    \begin{macrocode}
\newcommand{\childdocredirect}[2][]
{
  \begingroup
    \if?#1?
      \def\childdoctmp{\childdocforward{#2}}
    \else
      \def\childdoctmp{\childdocforwardprefix{#1}{#2}}
    \fi
    \expandafter
  \endgroup
  \childdoctmp
}
%    \end{macrocode}

%\iffalse
%</package>
%\fi
%
\endinput

\childdocforwardprefix[cdocsamp]{cdocsfn}{cdocsch}
%    \end{macrocode}

%\iffalse
%</samplefinal>
%\fi
%
% %%%%%%%%%%%%%%%%%%%%%%%%%%%%%%%%%%%%%%
% \paragraph{Command Line Processing.}
%
% The following three command lines generate the output files
% |cdocscld|, |cdocscl1| and |cdocscl2|
% which should be identical to
% |cdocsdrf|, |cdocsch1| and |cdocsfn2|, respectively:
% \begin{center}
% \begin{tabular}{l}
% |latex -jobname cdocscld \|\\
% |  "\def\version{draft}% \iffalse
%
% childdoc.dtx Copyright (C) 2017-2018 Niklas Beisert
%
% This work may be distributed and/or modified under the
% conditions of the LaTeX Project Public License, either version 1.3
% of this license or (at your option) any later version.
% The latest version of this license is in
%   http://www.latex-project.org/lppl.txt
% and version 1.3 or later is part of all distributions of LaTeX
% version 2005/12/01 or later.
%
% This work has the LPPL maintenance status `maintained'.
%
% The Current Maintainer of this work is Niklas Beisert.
%
% This work consists of the files childdoc.dtx and childdoc.ins
% and the derived files childdoc.def and cdocsamp.tex with
% cdocsch1.tex, cdocsch2.tex, cdocsdrf.tex, cdocsfn1.tex, cdocsfn2.tex.
%
%<package>\ifdefined\childdocmain\endinput\fi
%<package>\ProvidesFile{childdoc.def}[2018/12/30 v2.0 child document driver]
%<samplemain>\ProvidesFile{cdocsamp.tex}[2018/12/30 v2.0 sample for childdoc]
%<*driver>
%\ProvidesFile{childdoc.drv}[2018/12/30 v2.0 childdoc reference manual file]
\PassOptionsToClass{10pt,a4paper}{article}
\documentclass{ltxdoc}

\usepackage[margin=35mm]{geometry}
\usepackage{hyperref}
\usepackage{hyperxmp}
\usepackage[usenames]{color}

\hypersetup{colorlinks=true}
\hypersetup{pdfstartview=FitH}
\hypersetup{pdfpagemode=UseNone}
\hypersetup{pdfsource={}}
\hypersetup{pdflang={en-UK}}
\hypersetup{pdfcopyright={Copyright 2017-2018 Niklas Beisert.
  This work may be distributed and/or modified under the
  conditions of the LaTeX Project Public License, either version 1.3
  of this license or (at your option) any later version.}}
\hypersetup{pdflicenseurl={http://www.latex-project.org/lppl.txt}}
\hypersetup{pdfcontactaddress={ETH Zurich, ITP, HIT K,
  Wolfgang-Pauli-Strasse 27}}
\hypersetup{pdfcontactpostcode={8093}}
\hypersetup{pdfcontactcity={Zurich}}
\hypersetup{pdfcontactcountry={Switzerland}}
\hypersetup{pdfcontactemail={nbeisert@itp.phys.ethz.ch}}
\hypersetup{pdfcontacturl={http://people.phys.ethz.ch/\xmptilde nbeisert/}}

\newcommand{\secref}[1]{\hyperref[#1]{section \ref*{#1}}}

\parskip1ex
\parindent0pt
\let\olditemize\itemize
\def\itemize{\olditemize\parskip0pt}

\begin{document}

\title{The \textsf{childdoc} Package}
\hypersetup{pdftitle={The childdoc Package}}
\author{Niklas Beisert\\[2ex]
  Institut f\"ur Theoretische Physik\\
  Eidgen\"ossische Technische Hochschule Z\"urich\\
  Wolfgang-Pauli-Strasse 27, 8093 Z\"urich, Switzerland\\[1ex]
  \href{mailto:nbeisert@itp.phys.ethz.ch}
  {\texttt{nbeisert@itp.phys.ethz.ch}}}
\hypersetup{pdfauthor={Niklas Beisert}}
\hypersetup{pdfsubject={Manual for the LaTeX2e Package childdoc}}
\date{30 December 2018, \textsf{v2.0}}
\maketitle

\begin{abstract}\noindent
\textsf{childdoc} is a \LaTeXe{} package
that enables the direct compilation
of document sections included by |\include|
to individual files.
\end{abstract}

\begingroup
\parskip0ex
\tableofcontents
\endgroup

%%%%%%%%%%%%%%%%%%%%%%%%%%%%%%%%%%%%%%%%%%%%%%%%%%%%%%%%%%%%%%%%%%%%%%%%%%%%%%%%
%%%%%%%%%%%%%%%%%%%%%%%%%%%%%%%%%%%%%%%%%%%%%%%%%%%%%%%%%%%%%%%%%%%%%%%%%%%%%%%%
\section{Introduction}

\LaTeX{} provides a mechanism to structure a large document (such as a book)
into a main file and several child files (containing the chapters)
using the |\include| command.
This mechanism is beneficial for documents
which span hundreds of pages in order to
make the source file(s) more manageable.
Moreover, compilation can be restricted to
selected child files by means of the |\includeonly| command.
The latter feature can be used to reduce the compilation time while editing
(this was significantly more useful in the earlier days of \LaTeX{})
or to generate a smaller document which is easier to navigate.
Another application of |\includeonly| is to generate
documents consisting of selected parts of the complete document.

However, there are a few drawbacks of the plain |\include| mechanism:
\begin{itemize}
\item
The child files cannot be compiled on their own,
they can only be compiled via the main file.
A naive editing environment
(such as a text editor with an option
to have the current file processed by \LaTeX)
may require one to switch to the main file before compiling;
attempting to compile the child file produces errors.
\item
The main file must be modified (each time)
to adjust the |\includeonly| command
to the present needs. This easily leaves the main file in a messy state.
\item
The generated document will always carry the filename
of the main document. This is inconvenient if
several child files are to be compiled and
to be kept for distribution.
\end{itemize}

The present package provides a simple interface
to make child files individually compilable by \LaTeX{}.
Compiling a child file then has the same effect as compiling
the main file with an |\includeonly| command
to select the appropriate child.
Moreover the generated document will carry the name of the child
rather than the main file.
This resolves all three above issues.

This feature is meant to make the editing of books,
thesis documents and lecture notes somewhat more convenient.
However, the package can also be used efficiently for
composing a series of documents (such as exercise sheets)
which are typically distributed individually.
It then assists the author in generating the individual documents
(potentially in different versions)
as well as a document containing the collected series.
Another application is in developing style files
or other kinds of included material
where compilation of the style file could redirect
to a sample or test file.

%%%%%%%%%%%%%%%%%%%%%%%%%%%%%%%%%%%%%%%%%%%%%%%%%%%%%%%%%%%%%%%%%%%%%%%%%%%%%%%%
%%%%%%%%%%%%%%%%%%%%%%%%%%%%%%%%%%%%%%%%%%%%%%%%%%%%%%%%%%%%%%%%%%%%%%%%%%%%%%%%
\section{Usage}

First of all, the package \textsf{childdoc} is \emph{not} a standard
\LaTeXe{} |.sty| style file! Therefore it needs to be invoked in
a non-standard way.

%%%%%%%%%%%%%%%%%%%%%%%%%%%%%%%%%%%%%%%%%%%%%%%%%%%%%%%%%%%%%%%%%%%%%%%%%%%%%%%%
\subsection{Included Files}
\label{sec:include}

%%%%%%%%%%%%%%%%%%%%%%%%%%%%%%%%%%%%%%%%
\DescribeMacro{\childdocmain}
To use the package, add the commands
\begin{center}
\begin{tabular}{l}
|\input{childdoc.def}|\\
|\childdocmain{}|\\
\end{tabular}
\end{center}
at the very top of the main \LaTeX{} file,
in particular \emph{before} the |\documentclass| statement!
The argument of |\childdocmain| should be left empty
(but it must be present).

%%%%%%%%%%%%%%%%%%%%%%%%%%%%%%%%%%%%%%%%
\DescribeMacro{\childdocof}
Furthermore, add the commands
\begin{center}
\begin{tabular}{l}
|\input{childdoc.def}|\\
|\childdocof{|\textit{main}|}|\\
\end{tabular}
\end{center}
at the top of every child file \textit{child}
which is included by |\include{|\textit{child}|}|
from within the main file
(or at least for those files to be compiled individually).
The argument \textit{main} must be the filename of the main file.

There are a couple of
considerations in setting up the main and child documents:

%%%%%%%%%%%%%%%%%%%%%%%%%%%%%%%%%%%%%%%%
\paragraph{Restrictions.}

Please note the following restrictions:
\begin{itemize}
\item
|\childdocmain| must be called with one argument \textit{main}
to ensure compatibility with earlier version of the package.
It must either be empty (|\childdocmain{}|)
or precisely match the filename of the main file in which it is specified.
See \secref{sec:detection} for further information.
\item
The filename \textit{main} must be specified without the |.tex| extension.
\item
The filename \textit{main} is case sensitive
(even in case-insensitive file systems)
due to internal string comparison.
\item
The argument \textit{main} should be fully expanded, it cannot be a macro.
\item
Subdirectories and special characters should be avoided in filenames.
\item
The command |\childdocmain{|\textit{main}|}| must be followed by a whitespace.
It should not be followed immediately by another command
or by a comment mark `|%|'.
This is because the \TeX{} parser reads the token immediately following
the argument of |\childdocmain| and puts it
at the beginning of every child section;
however, a white\-space is ignored.
\end{itemize}

%%%%%%%%%%%%%%%%%%%%%%%%%%%%%%%%%%%%%%%%
\paragraph{Content of Main File.}

It is advisable to place all content in the child files included by |\include|.
Any output contained in the main file will appear in all child documents
unless suppressed manually;
it cannot be suppressed automatically by the |\includeonly| directive
and thus should normally be avoided.
A method to include some content in the main file
by means of conditional processing is described in \secref{sec:conditional}.

%%%%%%%%%%%%%%%%%%%%%%%%%%%%%%%%%%%%%%%%
\paragraph{Page Numbering.}

When only a part of the document is compiled,
the appropriate numbering of pages
(as well as other status parameters)
is determined from the |.aux| files.
The latter contain information from previous passes.
However this information needs to propagate through
all intermediate child documents.
Therefore the page numbering in child documents may well
be inconsistent until the complete document is compiled at least once.

A useful (if unconventional) way to always ensure a consistent
page numbering is to restart the numbering in each child document
and denote the pages by `\textit{child}|.|\textit{page}'
where \textit{child} represents the chapter/section number of the child file.
This can be achieved by the command
|\numberwithin{page}{|\textit{child}|}|
of the \textsf{amsmath} package
where \textit{child} can be |chapter| or |section|
depending on the chosen structuring.
Alternatively, one can modify the macro |\thepage| appropriately
and reset the counter |page| at the start of each child file.

%%%%%%%%%%%%%%%%%%%%%%%%%%%%%%%%%%%%%%%%%%%%%%%%%%%%%%%%%%%%%%%%%%%%%%%%%%%%%%%%
\subsection{Conditional Processing}
\label{sec:conditional}

The package provides a mechanism to compile different versions
of a document. To customise the versions further some conditional processing
can come in handy to distinguish which version is being compiled.
The package provides two macros to describe the compilation context:

%%%%%%%%%%%%%%%%%%%%%%%%%%%%%%%%%%%%%%%%
\DescribeMacro{\ifchilddoc}
The conditional |\ifchilddoc| distinguishes between the compilation of
child documents and the main document:
%
\begin{center}
|\ifchilddoc |\textit{child-code}| |[|\||else |\textit{main-code}]| \||fi|
\end{center}

%%%%%%%%%%%%%%%%%%%%%%%%%%%%%%%%%%%%%%%%
\DescribeMacro{\childdocname}
\DescribeMacro{\childdocjob}
The macro |\childdocname| contains the filename (without extension)
of the main or child file being processed.
Note that |\childdocjob| will always contain the name of the main file.

%%%%%%%%%%%%%%%%%%%%%%%%%%%%%%%%%%%%%%%%
\paragraph{Title Page.}

Conditional processing can be used to include a title or banner page
in the main document when proper precautions are taken.
Importantly, the code in the main file should ensure that the page counter
(as well as other status parameters which are stored in the |.aux| files)
takes the same value after the conditional processing.
Otherwise the page numbers may take divergent values
depending on which part is compiled.

For example, a title page could be declared by:
%
\begin{center}
\begin{tabular}{l}
|\ifchilddoc\||else|\\
|\addtocounter{page}{-1}|\\
\textit{code for title page}\\
|\newpage|\\
|\||fi|
\end{tabular}
\end{center}
%
A banner page for the child documents can be generated by:
%
\begin{center}
\begin{tabular}{l}
|\ifchilddoc|\\
|\addtocounter{page}{-1}|\\
\textit{code for banner page}\\
|\newpage|\\
|\||fi|
\end{tabular}
\end{center}
%
Here one could write a message such as:
\begin{center}
|This is the part \childdocname{} of \childdocjob{}.|
\end{center}

%%%%%%%%%%%%%%%%%%%%%%%%%%%%%%%%%%%%%%%%%%%%%%%%%%%%%%%%%%%%%%%%%%%%%%%%%%%%%%%%
\subsection{Flags}
\label{sec:flags}

The package makes it easy to generate different versions
of the main or child documents.
To this end compilation flags can be defined
and assigned different default values.
They will be particularly useful in conjunction
with the forwarding mechanism described in \secref{sec:forward}.

For example, it may be useful to have a flag |\version|
which can be set to |draft| or |final|.
The document source will contain some conditional code
depending on the value of |\version|.
Suppose further, the flag should default to |final| for the main file
and to |draft| for child files
which is a natural assignment for editing the document.
This is achieved by placing the following code
in the preamble of the main document
(below the |\childdocmain| directive):
%
\begin{center}
\begin{tabular}{l}
|\ifchilddoc|\\
|\providecommand{\version}{draft}|\\
|\||else|\\
|\providecommand{\version}{final}|\\
|\||fi|
\end{tabular}
\end{center}
%
The definition by |\providecommand| makes sure
that previous definitions are not overwritten.
Further statements |\providecommand{\version}{...}|
can thus be added before the above code to override it.

For the main file, one might add a line
(between |\childdocmain| and the above block)
%
\begin{center}
|%\ifchilddoc\||else\providecommand{\version}{draft}\||fi|
\end{center}
%
which can be uncommented to produce a draft version.
Likewise one can add a line to the very top of a child file
(above the |\childdocof{|\textit{main}|}| directive)
%
\begin{center}
|%\providecommand{\version}{final}|
\end{center}
%
which can be uncommented to produce the final version of this child document.

%%%%%%%%%%%%%%%%%%%%%%%%%%%%%%%%%%%%%%%%%%%%%%%%%%%%%%%%%%%%%%%%%%%%%%%%%%%%%%%%
\subsection{Forwarding}
\label{sec:forward}

Different versions of the main or child documents
using compilation flags as described in \secref{sec:flags}
can be (permanently) stored in different files
for convenient compilation, viewing and distribution.
To this end, the package defines a command
to pass on compilation to a different file:

%%%%%%%%%%%%%%%%%%%%%%%%%%%%%%%%%%%%%%%%
\DescribeMacro{\childdocforward}
The command |\childdocforward| redirects processing to
another source file:
%
\begin{center}
\begin{tabular}{l}
|\input{childdoc.def}|\\
|\childdocforward[|\textit{main}|]{|\textit{dest}|}|\\
\end{tabular}
\end{center}
%
The argument \textit{dest} is the destination file
(without extension).
It should be the main file or one of the child files.
Note that further \textsf{childdoc} directives
such as |\childdocof| and |\childdocforward|
in the indicated file will be processed in this form.
The optional argument \textit{main}
passes on directly to the main file \textit{main}
while pretending to compile the child \textit{dest}.
This form behaves as if \textit{dest}
issues |\childdocof{|\textit{main}|}| right away,
and no further \textsf{childdoc} directives will be processed.

%%%%%%%%%%%%%%%%%%%%%%%%%%%%%%%%%%%%%%%%
\DescribeMacro{\...prefix}
In the alternative form |\childdocforwardprefix|,
%
\begin{center}
\begin{tabular}{l}
|\input{childdoc.def}|\\
|\childdocforwardprefix[|\textit{main}|]{|\textit{prefix}|}{|\textit{dest}|}|
\end{tabular}
\end{center}
%
the destination file is determined by a pattern
depending on the current file:
To make this work, the current file must be called
`{\textit{prefix}\hspace{0.2em}\textit{suffix}}'
with \textit{prefix} matching precisely the argument.
Processing is then passed on to the file
`{\textit{dest}\hspace{0.2em}\textit{suffix}}'.
Surely, the same effect is achieved by
directly specifying the
argument `{\textit{dest}\hspace{0.2em}\textit{suffix}}'
in the first form.
However, that requires to set up a different file
for each child. With the alternative form of the command
all these files can have exactly the same content
which simplifies setting them up and maintaining them.

For example, the following file |draft.tex|
with a compilation flag |\version| as described in \secref{sec:flags}
compiles the main document as a draft:
%
\begin{center}
\begin{tabular}{l}
|\def\version{draft}|\\
|\input{childdoc.def}|\\
|\childdocforward{|\textit{main}|}|
\end{tabular}
\end{center}
%
Likewise, the following files |final|\textit{nn}|.tex|
compile the final version of the child document
|child|\textit{nn}|.tex|:
%
\begin{center}
\begin{tabular}{l}
|\def\version{final}|\\
|\input{childdoc.def}|\\
|\childdocforwardprefix{final}{child}|
\end{tabular}
\end{center}
%

Note that when several versions of a main file and/or of each child file
are to be generated, it may be convenient to set up a |Makefile| or
shell script to automatise the process.

%%%%%%%%%%%%%%%%%%%%%%%%%%%%%%%%%%%%%%%%%%%%%%%%%%%%%%%%%%%%%%%%%%%%%%%%%%%%%%%%
\subsection{Command Line Processing}
\label{sec:commandline}

The effect of redirection files can also be achieved by invoking
the \LaTeX{} compiler with a more elaborate command line.
Most conveniently this should be done as part
of a shell script or a |Makefile|.

When using \textsf{childdoc} in the main file, the following
command lines effectively perform a redirection
(note that depending on the shell being used,
backslashes may have to be doubled: `|\|' $\to$ `|\\|'):
%
\begin{center}
|... -jobname "|\textit{target}|" |\\|"|[\textit{flags}]%
|\input{childdoc.def}\childdocforward[|\textit{main}|]{|\textit{dest}|}"|
\end{center}
%
Here \textit{target} is the name of the output file,
\textit{main} is the name of the main file
and \textit{dest} is the name of the main or child file to be processed
(all filenames without extensions).
The optional argument \textit{main} can be omitted
if \textit{main} matches \textit{dest}.
Optionally, compilation \textit{flags} can be defined via |\def| commands.
This command line makes the \TeX{} engine believe
it is compiling the file \textit{target}
whose content is specified as the latter parameter.
The provided code then forwards the processing to
\textit{main} or \textit{dest} as described in \secref{sec:forward}.

%%%%%%%%%%%%%%%%%%%%%%%%%%%%%%%%%%%%%%%%%%%%%%%%%%%%%%%%%%%%%%%%%%%%%%%%%%%%%%%%
\subsection{Include by Input}
\label{sec:input}

Including child documents by |\include| has some restrictions by design.
Most notably, the content of a child document always occupies
its own set of pages; pages cannot be shared between child documents.
Usually, this behaviour makes perfect sense
because each child document contain an essential part of the document.
However, in some situations it may be desirable to compose
a document from a collection of parts
without having mandatory page breaks between then.
For this case, the package
provides a mechanism to include parts
by |\input| which can also be processed individually.
However, by construction this mechanism
requires manual handling of the content to be output.

%%%%%%%%%%%%%%%%%%%%%%%%%%%%%%%%%%%%%%%%
\DescribeMacro{\ifchilddocmanual}
The main file should be prepared as usual, see \secref{sec:include}.
However, the document body must make a distinction
between processing of an individual part and of the main document, e.g.:
%
\begin{center}
\begin{tabular}{l}
|\ifchilddocmanual|\\
|\input{\childdocname}|\\
|\||else|\\
\textit{document body with }|\input{|\textit{part}|}|\\
|\||fi|
\end{tabular}
\end{center}
%
The conditional |\ifchilddocmanual| is true whenever
a part to be included by |\input| is being compiled,
and the name of the part is stored in |\childdocname|.

%%%%%%%%%%%%%%%%%%%%%%%%%%%%%%%%%%%%%%%%
\DescribeMacro{\childdocby}
Each part to be included by |\input| should start with:
%
\begin{center}
\begin{tabular}{l}
|\input{childdoc.def}|\\
|\childdocby{|\textit{main}|}|\\
\end{tabular}
\end{center}
%
The directive |\childdocby| is similar to |\childdocof|
described in \secref{sec:include},
but the subsequent selection of content must be done manually.
To that end, both |\ifchilddoc| and |\ifchilddocmanual|
will be true upon processing of a part,
and the name of the part is stored in |\childdocname|.
Note that |\jobname| will be set to the filename of the current part
so that each part receives an individual |.aux| file
that does not interfere with the |.aux| file(s) of the main document.
This behaviour can be altered by the alternative form
|\childdocby[*]{|\textit{main}|}| (with a non-empty optional argument)
which uses the |.aux| file of the main document
by setting |\jobname| to \textit{main}.

%%%%%%%%%%%%%%%%%%%%%%%%%%%%%%%%%%%%%%%%%%%%%%%%%%%%%%%%%%%%%%%%%%%%%%%%%%%%%%%%
\subsection{Driver Development}
\label{sec:driver}

The \textsf{childdoc} mechanism can also be use for the development
of definition files such as \LaTeX{} styles or classes.
This case differs from the above setup with multiple parts
included by |\include| in that no |\includeonly| should be invoked.
This can be achieved by starting the include file
(before |\ProvidesPackage|) with:
%
\begin{center}
\begin{tabular}{l}
|\input{childdoc.def}|\\
|\childdocforward{|\textit{main}|}|\\
\end{tabular}
\end{center}
%
or alternatively with:
%
\begin{center}
\begin{tabular}{l}
|\input{childdoc.def}|\\
|\childdocby{|\textit{main}|}|\\
\end{tabular}
\end{center}
%
Both forms have slightly different effects as described above.
The main file is prepared as usual, see \secref{sec:include}.

%%%%%%%%%%%%%%%%%%%%%%%%%%%%%%%%%%%%%%%%%%%%%%%%%%%%%%%%%%%%%%%%%%%%%%%%%%%%%%%%
\subsection{Legacy Detection}
\label{sec:detection}

The directive |\childdocmain| in the main file can detect
whether the complete document or merely a child is to be compiled
even without using the directive |\childdocof|.
This method is deprecated because it is less robust
and there is no compelling reason to use it;
it is merely provided for backward compatibility
and it may be removed in future versions.

If the detection mechanism is to be used,
it is mandatory to correctly specify
the filename of the main file as the argument of |\childdocmain|:
%
\begin{center}
\begin{tabular}{l}
|\input{childdoc.def}|\\
|\childdocmain{|\textit{main}|}|\\
\end{tabular}
\end{center}
%
If |\jobname| does not match the argument \textit{main} of |\childdocmain|,
it is assumed that |\jobname| points to the child file to be compiled.
When using |\childdocmain| with the main file specified as argument,
it suffices to start a child file
with just |\input{|\textit{main}|}|
without loading of the package and using |\childdocof|.
If instead all processing is done
with the appropriate \textsf{childdoc} directives,
the argument of \textit{main} of |\childdocmain| can be empty.

An alternative version of the command line processing described
in \secref{sec:commandline} using the detection mechanism reads:
%
\begin{center}
|... -jobname "|\textit{target}|" "|[\textit{flags}]%
[|\def\jobname{|\textit{dest}|}|]|\input{|\textit{main}|}"|
\end{center}

%%%%%%%%%%%%%%%%%%%%%%%%%%%%%%%%%%%%%%%%%%%%%%%%%%%%%%%%%%%%%%%%%%%%%%%%%%%%%%%%
\subsection{Manual Code}
\label{sec:manual}

In case one cannot be certain whether the definitions file |childdoc.def|
is installed on the target \TeX{} distribution
and one prefers not to ship it,
it is conceivable to paste a few relevant commands into the sources.

To that end, drop all statements |\input{childdoc.def}|
and perform the replacements as outlined below.
Instead of |\childdocmain{|\textit{main}|}| add the following code
to the top of the main file:
%
\begin{center}
\begin{tabular}{l}
|\||ifdefined\childdocname\endinput\||fi\newif\ifchilddoc|\\
|\edef\childdocname{\scantokens\expandafter{\jobname\noexpand}}|\\
|\def\childdocmain{|\textit{main}|}\||ifx\childdocmain\childdocname\||else|\\
|\childdoctrue\includeonly{\childdocname}\let\jobname\childdocmain\||fi|\\
\end{tabular}
\end{center}
%
Instead of |\childdocof{|\textit{main}|}| just include the main file
at the top of each child file:
%
\begin{center}
|\input{|\textit{main}|}|
\end{center}
%
A simple redirection |\childdocforward{|\textit{dest}|}| is achieved by:
%
\begin{center}
|\def\jobname{|\textit{dest}|}\input{\jobname}|
\end{center}
%
The redirection with prefix
|\childdocforwardprefix[|\textit{prefix}|]{|\textit{dest}|}|
is accomplished by:
%
\begin{center}
\begin{tabular}{l}
|{\edef\jobname{\scantokens\expandafter{\jobname\noexpand}}|\\
|\def\redirectjob |\textit{prefix}|#1~~~{\gdef\jobname{|\textit{dest}|#1}}|\\
|\expandafter\redirectjob\jobname~~~}\input{\jobname}|
\end{tabular}
\end{center}

In an alternative approach,
child documents can be compiled by a specific command line
without additional code or specific definitions:
%
\begin{center}
|... -jobname "|\textit{target}|" "|[\textit{flags}]%
|\includeonly{|\textit{dest}|}\input{|\textit{main}|}"|
\end{center}
%

%%%%%%%%%%%%%%%%%%%%%%%%%%%%%%%%%%%%%%%%%%%%%%%%%%%%%%%%%%%%%%%%%%%%%%%%%%%%%%%%
%%%%%%%%%%%%%%%%%%%%%%%%%%%%%%%%%%%%%%%%%%%%%%%%%%%%%%%%%%%%%%%%%%%%%%%%%%%%%%%%
\section{Information}

%%%%%%%%%%%%%%%%%%%%%%%%%%%%%%%%%%%%%%%%%%%%%%%%%%%%%%%%%%%%%%%%%%%%%%%%%%%%%%%%
\subsection{Copyright}

Copyright \copyright{} 2017--2018 Niklas Beisert

This work may be distributed and/or modified under the
conditions of the \LaTeX{} Project Public License, either version 1.3
of this license or (at your option) any later version.
The latest version of this license is in
  \url{http://www.latex-project.org/lppl.txt}
and version 1.3 or later is part of all distributions of \LaTeX{}
version 2005/12/01 or later.

This work has the LPPL maintenance status `maintained'.

The Current Maintainer of this work is Niklas Beisert.

This work consists of the files |README.txt|, |childdoc.ins| and |childdoc.dtx|
as well as the derived files |childdoc.def|, |cdocsamp.tex|
with |cdocsch1.tex|, |cdocsch2.tex|, |cdocspt3.tex|, |cdocspt4.tex|,
|cdocsdrf.tex|, |cdocsfn1.tex|, |cdocsfn2.tex|
as well as |childdoc.pdf|.

%%%%%%%%%%%%%%%%%%%%%%%%%%%%%%%%%%%%%%%%%%%%%%%%%%%%%%%%%%%%%%%%%%%%%%%%%%%%%%%%
\subsection{Files and Installation}

The package consists of the files:
%
\begin{center}
\begin{tabular}{ll}
    |README.txt|   & readme file \\
    |childdoc.ins| & installation file \\
    |childdoc.dtx| & source file \\
    |childdoc.def| & definition file \\
    |cdocsamp.tex| & sample main file \\
    |cdocsch1.tex| & sample include file \\
    |cdocsch2.tex| & sample include file \\
    |cdocspt3.tex| & sample part file \\
    |cdocspt4.tex| & sample part file \\
    |cdocsdrf.tex| & sample redirection file \\
    |cdocsfn1.tex| & sample redirection file \\
    |cdocsfn2.tex| & sample redirection file \\
    |childdoc.pdf| & manual
\end{tabular}
\end{center}
%
The distribution consists of the files
|README.txt|, |childdoc.ins| and |childdoc.dtx|.
%
\begin{itemize}
\item
Run (pdf)\LaTeX{} on |childdoc.dtx|
to compile the manual |childdoc.pdf| (this file).
\item
Run \LaTeX{} on |childdoc.ins| to create the definitions file |childdoc.def|
and the sample |cdocsamp.tex| with include files
|cdocsch1.tex|, |cdocsch2.tex|, |cdocspt3.tex|, |cdocspt4.tex|,
|cdocsdrf.tex|, |cdocsfn1.tex|, |cdocsfn2.tex|.
Then copy the file |childdoc.def| to an appropriate directory of your \LaTeX{}
distribution, e.g.\ \textit{texmf-root}|/tex/latex/childdoc|.
\end{itemize}

%%%%%%%%%%%%%%%%%%%%%%%%%%%%%%%%%%%%%%%%%%%%%%%%%%%%%%%%%%%%%%%%%%%%%%%%%%%%%%%%
\subsection{Related CTAN Packages}

There are several other packages which offer a similar functionality:
%
\begin{itemize}
\item
The packages
\href{http://ctan.org/pkg/docmute}{\textsf{docmute}},
\href{http://ctan.org/pkg/includex}{\textsf{includex}} and
\href{http://ctan.org/pkg/standalone}{\textsf{standalone}}
provide commands to include only the document body of
a child file thus allowing both files to be compiled individually.
\item
The packages \href{http://ctan.org/pkg/subdocs}{\textsf{subdocs}}
and \href{http://ctan.org/pkg/subfiles}{\textsf{subfiles}}
provide structures in which the main and child documents can be
encapsulated and allowing them to be compiled individually.
The inclusion mechanism is different from the conventional |\include|.
\item
The package \href{http://ctan.org/pkg/combine}{\textsf{combine}}
is an elaborate solution to combine several documents into one.
\end{itemize}
%
See also the CTAN topic \href{http://ctan.org/topic/subdocs}{\textsf{subdocs}}
for further related packages.
The present package differs from the above solutions in that
a document structure constructed with the conventional |\include| mechanism
just needs two extra commands at the top of every file
such that all constituent files can be compiled individually.

%%%%%%%%%%%%%%%%%%%%%%%%%%%%%%%%%%%%%%%%%%%%%%%%%%%%%%%%%%%%%%%%%%%%%%%%%%%%%%%%
%\subsection{Feature Suggestions}
%
%The following is a list of features which may be useful for future
%versions of this package:
%%
%\begin{itemize}
%\item
%\ldots
%\end{itemize}

%%%%%%%%%%%%%%%%%%%%%%%%%%%%%%%%%%%%%%%%%%%%%%%%%%%%%%%%%%%%%%%%%%%%%%%%%%%%%%%%
\subsection{Revision History}

%%%%%%%%%%%%%%%%%%%%%%%%%%%%%%%%%%%%%%%%
\paragraph{v2.0:} 2018/12/30

\begin{itemize}
\item
immediate forward processing
\item
added |\childdocby| mechanism
\item
manual restructured
\end{itemize}

%%%%%%%%%%%%%%%%%%%%%%%%%%%%%%%%%%%%%%%%
\paragraph{v1.6:} 2018/01/17

\begin{itemize}
\item
application for development of include files
\item
corrections to manual
\end{itemize}

%%%%%%%%%%%%%%%%%%%%%%%%%%%%%%%%%%%%%%%%
\paragraph{v1.5:} 2017/05/21

\begin{itemize}
\item
more complete structuring introduced
\item
|\childdocof| introduced
\item
|\childdoc| renamed to |\childdocmain|
\item
|\childredirect| renamed to |\childdocforward| and |\childdocforwardprefix|
and functionality expanded
\end{itemize}

%%%%%%%%%%%%%%%%%%%%%%%%%%%%%%%%%%%%%%%%
\paragraph{v1.0:} 2017/04/27

\begin{itemize}
\item
manual and install package
\item
first version published on CTAN
\end{itemize}

%%%%%%%%%%%%%%%%%%%%%%%%%%%%%%%%%%%%%%%%
\paragraph{v0.6:} 2017/04/26

\begin{itemize}
\item
redirection mechanism added
\end{itemize}

%%%%%%%%%%%%%%%%%%%%%%%%%%%%%%%%%%%%%%%%
\paragraph{v0.5:} 2017/04/26

\begin{itemize}
\item
functionality in definition file
\end{itemize}


%%%%%%%%%%%%%%%%%%%%%%%%%%%%%%%%%%%%%%%%%%%%%%%%%%%%%%%%%%%%%%%%%%%%%%%%%%%%%%%%
%%%%%%%%%%%%%%%%%%%%%%%%%%%%%%%%%%%%%%%%%%%%%%%%%%%%%%%%%%%%%%%%%%%%%%%%%%%%%%%%
%%%%%%%%%%%%%%%%%%%%%%%%%%%%%%%%%%%%%%%%%%%%%%%%%%%%%%%%%%%%%%%%%%%%%%%%%%%%%%%%
\appendix

\settowidth\MacroIndent{\rmfamily\scriptsize 000\ }

 \DocInput{childdoc.dtx}

\end{document}
%</driver>
% \fi
%
% %%%%%%%%%%%%%%%%%%%%%%%%%%%%%%%%%%%%%%%%%%%%%%%%%%%%%%%%%%%%%%%%%%%%%%%%%%%%%%
% %%%%%%%%%%%%%%%%%%%%%%%%%%%%%%%%%%%%%%%%%%%%%%%%%%%%%%%%%%%%%%%%%%%%%%%%%%%%%%
% \section{Sample}
%\iffalse
%<*samplemain>
%\fi
%
% The following presents a sample document
% with two chapters, two parts, a title page,
% a compile flag as well as three forwarding files to set the flag.
% It consists of eight |.tex| files:
% \begin{center}
% \begin{tabular}{ll}
% |cdocsamp.tex|&main file\\
% |cdocsch1.tex|&include file for chapter 1\\
% |cdocsch2.tex|&include file for chapter 2\\
% |cdocspt3.tex|&include file for part 3\\
% |cdocspt4.tex|&include file for part 4\\
% |cdocsdrf.tex|&forwarding file for main file in draft mode\\
% |cdocsfi1.tex|&forwarding file for final version of chapter 1\\
% |cdocsfi2.tex|&forwarding file for final version of chapter 2\\
% \end{tabular}
% \end{center}
% Each of the eight files can be compiled directly by the \LaTeX{} compiler.
%
% %%%%%%%%%%%%%%%%%%%%%%%%%%%%%%%%%%%%%%
% \paragraph{Main File.}
%
% The main file is called |cdocsamp.tex|.
%
% Load the \textsf{childdoc} definitions and
% declare the filename for the main document:
%    \begin{macrocode}
\input{childdoc.def}
\childdocmain{}
%    \end{macrocode}

% Optional override for |\version| flag:
%    \begin{macrocode}
%%\ifchilddoc\else\providecommand{\version}{draft}\fi
%    \end{macrocode}

% Define the default values for the |\version| flag
% (|final| for the main file and |draft| for childs):
%    \begin{macrocode}
\ifchilddoc
\providecommand{\version}{draft}
\else
\providecommand{\version}{final}
\fi
%    \end{macrocode}

% Load the standard document class:
%    \begin{macrocode}
\documentclass[12pt]{article}
%    \end{macrocode}

% Start the document body:
%    \begin{macrocode}
\begin{document}
%    \end{macrocode}

% Declare a title page.
% Print title, part of document being processed and version flag:
%    \begin{macrocode}
\addtocounter{page}{-1}
\begin{center}
{\LARGE\bfseries{}childdoc example\par}
\vspace{1cm}
\ifchilddoc
\ifchilddocmanual part\else chapter\fi:
`\childdocname' of `\childdocjob'\par
\else
main document: `\childdocjob'\par
\fi
version: \version\par
\end{center}
\newpage
%    \end{macrocode}

% Manually include selected file,
% otherwise process as usual:
%    \begin{macrocode}
\ifchilddocmanual
\section*{part `\childdocname'}
\input{\childdocname}
\else
%    \end{macrocode}

% Include the two chapters:
%    \begin{macrocode}
\include{cdocsch1}
\include{cdocsch2}
%    \end{macrocode}

% Include the two parts unless only chapters should be displayed:
%    \begin{macrocode}
\ifchilddoc\else
\section{part three}
\input{cdocspt3}
\section{part four}
\input{cdocspt4}
\fi
%    \end{macrocode}

% Process as usual until here:
%    \begin{macrocode}
\fi
%    \end{macrocode}

% End of document body:
%    \begin{macrocode}
\end{document}
%    \end{macrocode}
%\iffalse
%</samplemain>
%\fi
%
% %%%%%%%%%%%%%%%%%%%%%%%%%%%%%%%%%%%%%%
% \paragraph{Chapter Include Files.}
%
% The include files are called |cdocsch1.tex| and |cdocsch2.tex|.
%
%\iffalse
%<*samplechap1|samplechap2>
%\fi

% Optional override for |\version| flag:
%    \begin{macrocode}
%%\providecommand{\version}{final}
%    \end{macrocode}

% Include the main document:
%    \begin{macrocode}
\input{childdoc.def}
\childdocof{cdocsamp}
%    \end{macrocode}

%\iffalse
%</samplechap1|samplechap2>
%\fi
%
%\iffalse
%<*samplechap1>
%\fi
% Some text for chapter 1:
%    \begin{macrocode}
\section{one}
some text in chapter one
%    \end{macrocode}

%\iffalse
%</samplechap1>
%\fi
% Some text for chapter 2:
%\iffalse
%<*samplechap2>
%\fi
%    \begin{macrocode}
\section{two}
more text in chapter two
%    \end{macrocode}

%\iffalse
%</samplechap2>
%\fi
%
% %%%%%%%%%%%%%%%%%%%%%%%%%%%%%%%%%%%%%%
% \paragraph{Part Include Files.}
%
% The include files are called |cdocspt3.tex| and |cdocspt4.tex|.
%
%\iffalse
%<*samplepart3|samplepart4>
%\fi

% Optional override for |\version| flag:
%    \begin{macrocode}
%%\providecommand{\version}{final}
%    \end{macrocode}

% Include the main document:
%    \begin{macrocode}
\input{childdoc.def}
\childdocby{cdocsamp}
%    \end{macrocode}

%\iffalse
%</samplepart3|samplepart4>
%\fi
%
%\iffalse
%<*samplepart3>
%\fi
% Some text for part 3:
%    \begin{macrocode}
some text in part three
%    \end{macrocode}

%\iffalse
%</samplepart3>
%\fi
% Some text for part 4:
%\iffalse
%<*samplepart4>
%\fi
%    \begin{macrocode}
more text in part four
%    \end{macrocode}

%\iffalse
%</samplepart4>
%\fi
%
% %%%%%%%%%%%%%%%%%%%%%%%%%%%%%%%%%%%%%%
% \paragraph{Forwarding for a Complete Draft.}
%
% The following forwarding file |cdocsdrf.tex|
% compiles the main document in draft mode:
%\iffalse
%<*sampledraft>
%\fi
%    \begin{macrocode}
\def\version{draft}
\input{childdoc.def}
\childdocforward{cdocsamp}
%    \end{macrocode}

%\iffalse
%</sampledraft>
%\fi
%
% %%%%%%%%%%%%%%%%%%%%%%%%%%%%%%%%%%%%%%
% \paragraph{Forwarding for Final Version of the Chapters.}
%
% The following forwarding files |cdocsfn1.tex| and |cdocsfn2.tex|
% (with identical content)
% compile the final versions of the child documents
% |cdocsch1.tex| and |cdocsch2.tex|, respectively:
%\iffalse
%<*samplefinal>
%\fi
%    \begin{macrocode}
\def\version{final}
\input{childdoc.def}
\childdocforwardprefix[cdocsamp]{cdocsfn}{cdocsch}
%    \end{macrocode}

%\iffalse
%</samplefinal>
%\fi
%
% %%%%%%%%%%%%%%%%%%%%%%%%%%%%%%%%%%%%%%
% \paragraph{Command Line Processing.}
%
% The following three command lines generate the output files
% |cdocscld|, |cdocscl1| and |cdocscl2|
% which should be identical to
% |cdocsdrf|, |cdocsch1| and |cdocsfn2|, respectively:
% \begin{center}
% \begin{tabular}{l}
% |latex -jobname cdocscld \|\\
% |  "\def\version{draft}\input{childdoc.def}\childdocforward{cdocsamp}"|\\
% |latex -jobname cdocscl1 \|\\
% |  "\input{childdoc.def}\childdocforward[cdocsamp]{cdocsch1}"|\\
% |latex -jobname cdocscl2 \|\\
% |  "\def\version{final}\input{childdoc.def}\childdocforward{cdocsch2}"|
% \end{tabular}
% \end{center}
% Note that the trailing backslash on each first line
% merely continues the input to the second line
% (for convenient cut ant paste).
% Furthermore, the command |latex| can be replaced by any
% of its alternative versions such as |pdflatex|.
%
% %%%%%%%%%%%%%%%%%%%%%%%%%%%%%%%%%%%%%%%%%%%%%%%%%%%%%%%%%%%%%%%%%%%%%%%%%%%%%%
% %%%%%%%%%%%%%%%%%%%%%%%%%%%%%%%%%%%%%%%%%%%%%%%%%%%%%%%%%%%%%%%%%%%%%%%%%%%%%%
% \section{Implementation}
%\iffalse
%<*package>
%\fi
%
% This section describes the definitions file |childdoc.def|.

% The definitions cannot be loaded using |\usepackage| or |\RequirePackage|
% which has a mechanism to prevent loading a style file more than once.
% When loading the definitions by means of |\input|
% multiple instances have to be prevented manually:
%\iffalse
%This code needs to be before the `\ProvidesFile' directive
%which is defined at the beginning of this file.
%Therefore it is also placed there and commented out here.
%</package>
%<*discard>
%\fi
%    \begin{macrocode}
\ifdefined\childdocmain\endinput\fi
%    \end{macrocode}
%\iffalse
%</discard>
%<*package>
%\fi
%
% \macro{\ifchilddoc}
% \macro{\ifchilddocmanual}
% The conditional |\ifchilddoc| tells whether a
% child (true) or main (false) document is being compiled.
% The conditional |\ifchilddocmanual| tells whether
% the |\includeonly| mechanism is used (false) or
% the selection of child files must be performed manually (true).
% The definitions initialise to false:
%    \begin{macrocode}
\newif\ifchilddoc
\newif\ifchilddocmanual
%    \end{macrocode}

% \macro{\childdocname}
% \macro{\childdocjob}
% The macro |\childdocname| stores the name of the main document
% to be compiled. The macro |\childdocjob| stores the name of
% the document on which the \LaTeX{} compiler was originally invoked.
% The content of |\jobname| cannot be compared
% to filenames specified in the source due to different catcodes.
% The following code rescans |\jobname|, stores the result
% in |\childdocname| and saves a copy in |\childdocjob|:
%    \begin{macrocode}
\edef\childdocname{\scantokens\expandafter{\jobname\noexpand}}
\let\childdocjob\childdocname
%    \end{macrocode}

% \macro{\childdocdisable}
% The macro |\childdocdisable| prevents the main file
% from being processed more than once.
% At this stage, the main document command |\childdocmain|
% is assumed to be called once again where it should do nothing.
% Any subsequent call to it should prevent
% a secondary processing of the main document
% It overwrites the forwarding commands
% |\childdocof| and |\childdocforward|
% with empty macros to prevent further inclusions of the main document:
%    \begin{macrocode}
\newcommand{\childdocdisable}
{
  \renewcommand{\childdocmain}[1]{\renewcommand{\childdocmain}[1]{\endinput}}
  \renewcommand{\childdocof}[1]{}
  \renewcommand{\childdocby}[2][]{}
  \renewcommand{\childdocforward}[2][]{}
  \renewcommand{\childdocdisable}{}
}
%    \end{macrocode}

% \macro{\childdocmain}
% The macro |\childdocmain| is to be called at the top of the main file
% with nothing or the main filename (without extension) as argument.
% First, it breaks loops.
% If the argument is not empty and does not match |\childdocname|
% (which is set by the first inclusion of |childdoc.def|),
% |\ifchilddoc| is set to true, |\includeonly| is applied to the child file
% and |\jobname| is set to the main file
% (for proper handling of |.aux| files):
%    \begin{macrocode}
\newcommand{\childdocmain}[1]
{
  \childdocdisable\childdocmain{}
  \if?#1?\else
    \begingroup
      \def\childdoctmp{#1}
      \ifx\childdoctmp\childdocname
        \def\childdoctmp{}
      \else
        \def\childdoctmp
        {
          \childdoctrue
          \includeonly{\childdocname}
          \def\childdocjob{#1}
          \def\jobname{#1}
        }
      \fi
      \expandafter
    \endgroup
    \childdoctmp
  \fi
}
%    \end{macrocode}

% \macro{\childdocof}
% The command |\childdocof| redirects
% compilation to the main file |#1|.
%    \begin{macrocode}
\newcommand{\childdocof}[1]
{
  \childdocdisable
  \childdoctrue
  \includeonly{\childdocname}
  \def\jobname{#1}
  \def\childdocjob{#1}
  \input{#1}
}
%    \end{macrocode}

% \macro{\childdocby}
% The command |\childdocby| ....
%    \begin{macrocode}
\newcommand{\childdocby}[2][]
{
  \childdocdisable
  \childdoctrue
  \childdocmanualtrue
  \if?#1?\else
    \def\jobname{#2}
  \fi
  \def\childdocjob{#2}
  \input{#2}
  \endinput
}
%    \end{macrocode}

% \macro{\childdocforward}
% The command |\childdocforward| redirects
% compilation to the main file or
% (if the optional argument is given) a child file.
% Parameters are set as if the main file
% or a child file starting with |\childdocof| was compiled.
% Then compilation is handed over to the main file:
%    \begin{macrocode}
\newcommand{\childdocforward}[2][]
{
  \begingroup
    \if?#1?
      \def\childdoctmp
      {
        \def\childdocname{#2}
        \def\childdocjob{#2}
        \def\jobname{#2}
        \input{#2}
        \endinput
      }
    \else
      \def\childdoctmp
      {
        \childdocdisable
        \def\childdocname{#2}
        \childdoctrue
        \includeonly{#2}
        \def\childdocjob{#1}
        \def\jobname{#1}
        \input{#1}
        \endinput
      }
    \fi
    \expandafter
  \endgroup
  \childdoctmp
}
%    \end{macrocode}

% \macro{\childdocforwardprefix}
% The command |\childdocforwardprefix| redirects
% compilation to the main or a child file by means of a pattern.
% The prefix |#1| in the current filename is replaced by |#2|
% and the suffix of the current filename is kept
% (it is assumed that the filename does not contain the substring `|~~~|'
% which is used as a delimiter).
% Compilation is handed over to the new file by |\childdocforward|:
%    \begin{macrocode}
\newcommand{\childdocforwardprefix}[3][]
{
  \begingroup
    \def\childdocextract #2##1~~~{\def\childdoctmp{\childdocforward[#1]{#3##1}}}
    \expandafter\childdocextract\childdocname~~~
    \expandafter
  \endgroup
  \childdoctmp
}
%    \end{macrocode}

% \macro{\childdoc}
% The deprecated macro |\childdoc| is a legacy version of |\childdocmain|:
%    \begin{macrocode}
\newcommand{\childdoc}{\childdocmain}
%    \end{macrocode}

% \macro{\childdocredirect}
% The deprecated macro |\childdocredirect| is a legacy version
% of |\childdocforward| and |\childdocforwardprefix|:
%    \begin{macrocode}
\newcommand{\childdocredirect}[2][]
{
  \begingroup
    \if?#1?
      \def\childdoctmp{\childdocforward{#2}}
    \else
      \def\childdoctmp{\childdocforwardprefix{#1}{#2}}
    \fi
    \expandafter
  \endgroup
  \childdoctmp
}
%    \end{macrocode}

%\iffalse
%</package>
%\fi
%
\endinput
\childdocforward{cdocsamp}"|\\
% |latex -jobname cdocscl1 \|\\
% |  "% \iffalse
%
% childdoc.dtx Copyright (C) 2017-2018 Niklas Beisert
%
% This work may be distributed and/or modified under the
% conditions of the LaTeX Project Public License, either version 1.3
% of this license or (at your option) any later version.
% The latest version of this license is in
%   http://www.latex-project.org/lppl.txt
% and version 1.3 or later is part of all distributions of LaTeX
% version 2005/12/01 or later.
%
% This work has the LPPL maintenance status `maintained'.
%
% The Current Maintainer of this work is Niklas Beisert.
%
% This work consists of the files childdoc.dtx and childdoc.ins
% and the derived files childdoc.def and cdocsamp.tex with
% cdocsch1.tex, cdocsch2.tex, cdocsdrf.tex, cdocsfn1.tex, cdocsfn2.tex.
%
%<package>\ifdefined\childdocmain\endinput\fi
%<package>\ProvidesFile{childdoc.def}[2018/12/30 v2.0 child document driver]
%<samplemain>\ProvidesFile{cdocsamp.tex}[2018/12/30 v2.0 sample for childdoc]
%<*driver>
%\ProvidesFile{childdoc.drv}[2018/12/30 v2.0 childdoc reference manual file]
\PassOptionsToClass{10pt,a4paper}{article}
\documentclass{ltxdoc}

\usepackage[margin=35mm]{geometry}
\usepackage{hyperref}
\usepackage{hyperxmp}
\usepackage[usenames]{color}

\hypersetup{colorlinks=true}
\hypersetup{pdfstartview=FitH}
\hypersetup{pdfpagemode=UseNone}
\hypersetup{pdfsource={}}
\hypersetup{pdflang={en-UK}}
\hypersetup{pdfcopyright={Copyright 2017-2018 Niklas Beisert.
  This work may be distributed and/or modified under the
  conditions of the LaTeX Project Public License, either version 1.3
  of this license or (at your option) any later version.}}
\hypersetup{pdflicenseurl={http://www.latex-project.org/lppl.txt}}
\hypersetup{pdfcontactaddress={ETH Zurich, ITP, HIT K,
  Wolfgang-Pauli-Strasse 27}}
\hypersetup{pdfcontactpostcode={8093}}
\hypersetup{pdfcontactcity={Zurich}}
\hypersetup{pdfcontactcountry={Switzerland}}
\hypersetup{pdfcontactemail={nbeisert@itp.phys.ethz.ch}}
\hypersetup{pdfcontacturl={http://people.phys.ethz.ch/\xmptilde nbeisert/}}

\newcommand{\secref}[1]{\hyperref[#1]{section \ref*{#1}}}

\parskip1ex
\parindent0pt
\let\olditemize\itemize
\def\itemize{\olditemize\parskip0pt}

\begin{document}

\title{The \textsf{childdoc} Package}
\hypersetup{pdftitle={The childdoc Package}}
\author{Niklas Beisert\\[2ex]
  Institut f\"ur Theoretische Physik\\
  Eidgen\"ossische Technische Hochschule Z\"urich\\
  Wolfgang-Pauli-Strasse 27, 8093 Z\"urich, Switzerland\\[1ex]
  \href{mailto:nbeisert@itp.phys.ethz.ch}
  {\texttt{nbeisert@itp.phys.ethz.ch}}}
\hypersetup{pdfauthor={Niklas Beisert}}
\hypersetup{pdfsubject={Manual for the LaTeX2e Package childdoc}}
\date{30 December 2018, \textsf{v2.0}}
\maketitle

\begin{abstract}\noindent
\textsf{childdoc} is a \LaTeXe{} package
that enables the direct compilation
of document sections included by |\include|
to individual files.
\end{abstract}

\begingroup
\parskip0ex
\tableofcontents
\endgroup

%%%%%%%%%%%%%%%%%%%%%%%%%%%%%%%%%%%%%%%%%%%%%%%%%%%%%%%%%%%%%%%%%%%%%%%%%%%%%%%%
%%%%%%%%%%%%%%%%%%%%%%%%%%%%%%%%%%%%%%%%%%%%%%%%%%%%%%%%%%%%%%%%%%%%%%%%%%%%%%%%
\section{Introduction}

\LaTeX{} provides a mechanism to structure a large document (such as a book)
into a main file and several child files (containing the chapters)
using the |\include| command.
This mechanism is beneficial for documents
which span hundreds of pages in order to
make the source file(s) more manageable.
Moreover, compilation can be restricted to
selected child files by means of the |\includeonly| command.
The latter feature can be used to reduce the compilation time while editing
(this was significantly more useful in the earlier days of \LaTeX{})
or to generate a smaller document which is easier to navigate.
Another application of |\includeonly| is to generate
documents consisting of selected parts of the complete document.

However, there are a few drawbacks of the plain |\include| mechanism:
\begin{itemize}
\item
The child files cannot be compiled on their own,
they can only be compiled via the main file.
A naive editing environment
(such as a text editor with an option
to have the current file processed by \LaTeX)
may require one to switch to the main file before compiling;
attempting to compile the child file produces errors.
\item
The main file must be modified (each time)
to adjust the |\includeonly| command
to the present needs. This easily leaves the main file in a messy state.
\item
The generated document will always carry the filename
of the main document. This is inconvenient if
several child files are to be compiled and
to be kept for distribution.
\end{itemize}

The present package provides a simple interface
to make child files individually compilable by \LaTeX{}.
Compiling a child file then has the same effect as compiling
the main file with an |\includeonly| command
to select the appropriate child.
Moreover the generated document will carry the name of the child
rather than the main file.
This resolves all three above issues.

This feature is meant to make the editing of books,
thesis documents and lecture notes somewhat more convenient.
However, the package can also be used efficiently for
composing a series of documents (such as exercise sheets)
which are typically distributed individually.
It then assists the author in generating the individual documents
(potentially in different versions)
as well as a document containing the collected series.
Another application is in developing style files
or other kinds of included material
where compilation of the style file could redirect
to a sample or test file.

%%%%%%%%%%%%%%%%%%%%%%%%%%%%%%%%%%%%%%%%%%%%%%%%%%%%%%%%%%%%%%%%%%%%%%%%%%%%%%%%
%%%%%%%%%%%%%%%%%%%%%%%%%%%%%%%%%%%%%%%%%%%%%%%%%%%%%%%%%%%%%%%%%%%%%%%%%%%%%%%%
\section{Usage}

First of all, the package \textsf{childdoc} is \emph{not} a standard
\LaTeXe{} |.sty| style file! Therefore it needs to be invoked in
a non-standard way.

%%%%%%%%%%%%%%%%%%%%%%%%%%%%%%%%%%%%%%%%%%%%%%%%%%%%%%%%%%%%%%%%%%%%%%%%%%%%%%%%
\subsection{Included Files}
\label{sec:include}

%%%%%%%%%%%%%%%%%%%%%%%%%%%%%%%%%%%%%%%%
\DescribeMacro{\childdocmain}
To use the package, add the commands
\begin{center}
\begin{tabular}{l}
|\input{childdoc.def}|\\
|\childdocmain{}|\\
\end{tabular}
\end{center}
at the very top of the main \LaTeX{} file,
in particular \emph{before} the |\documentclass| statement!
The argument of |\childdocmain| should be left empty
(but it must be present).

%%%%%%%%%%%%%%%%%%%%%%%%%%%%%%%%%%%%%%%%
\DescribeMacro{\childdocof}
Furthermore, add the commands
\begin{center}
\begin{tabular}{l}
|\input{childdoc.def}|\\
|\childdocof{|\textit{main}|}|\\
\end{tabular}
\end{center}
at the top of every child file \textit{child}
which is included by |\include{|\textit{child}|}|
from within the main file
(or at least for those files to be compiled individually).
The argument \textit{main} must be the filename of the main file.

There are a couple of
considerations in setting up the main and child documents:

%%%%%%%%%%%%%%%%%%%%%%%%%%%%%%%%%%%%%%%%
\paragraph{Restrictions.}

Please note the following restrictions:
\begin{itemize}
\item
|\childdocmain| must be called with one argument \textit{main}
to ensure compatibility with earlier version of the package.
It must either be empty (|\childdocmain{}|)
or precisely match the filename of the main file in which it is specified.
See \secref{sec:detection} for further information.
\item
The filename \textit{main} must be specified without the |.tex| extension.
\item
The filename \textit{main} is case sensitive
(even in case-insensitive file systems)
due to internal string comparison.
\item
The argument \textit{main} should be fully expanded, it cannot be a macro.
\item
Subdirectories and special characters should be avoided in filenames.
\item
The command |\childdocmain{|\textit{main}|}| must be followed by a whitespace.
It should not be followed immediately by another command
or by a comment mark `|%|'.
This is because the \TeX{} parser reads the token immediately following
the argument of |\childdocmain| and puts it
at the beginning of every child section;
however, a white\-space is ignored.
\end{itemize}

%%%%%%%%%%%%%%%%%%%%%%%%%%%%%%%%%%%%%%%%
\paragraph{Content of Main File.}

It is advisable to place all content in the child files included by |\include|.
Any output contained in the main file will appear in all child documents
unless suppressed manually;
it cannot be suppressed automatically by the |\includeonly| directive
and thus should normally be avoided.
A method to include some content in the main file
by means of conditional processing is described in \secref{sec:conditional}.

%%%%%%%%%%%%%%%%%%%%%%%%%%%%%%%%%%%%%%%%
\paragraph{Page Numbering.}

When only a part of the document is compiled,
the appropriate numbering of pages
(as well as other status parameters)
is determined from the |.aux| files.
The latter contain information from previous passes.
However this information needs to propagate through
all intermediate child documents.
Therefore the page numbering in child documents may well
be inconsistent until the complete document is compiled at least once.

A useful (if unconventional) way to always ensure a consistent
page numbering is to restart the numbering in each child document
and denote the pages by `\textit{child}|.|\textit{page}'
where \textit{child} represents the chapter/section number of the child file.
This can be achieved by the command
|\numberwithin{page}{|\textit{child}|}|
of the \textsf{amsmath} package
where \textit{child} can be |chapter| or |section|
depending on the chosen structuring.
Alternatively, one can modify the macro |\thepage| appropriately
and reset the counter |page| at the start of each child file.

%%%%%%%%%%%%%%%%%%%%%%%%%%%%%%%%%%%%%%%%%%%%%%%%%%%%%%%%%%%%%%%%%%%%%%%%%%%%%%%%
\subsection{Conditional Processing}
\label{sec:conditional}

The package provides a mechanism to compile different versions
of a document. To customise the versions further some conditional processing
can come in handy to distinguish which version is being compiled.
The package provides two macros to describe the compilation context:

%%%%%%%%%%%%%%%%%%%%%%%%%%%%%%%%%%%%%%%%
\DescribeMacro{\ifchilddoc}
The conditional |\ifchilddoc| distinguishes between the compilation of
child documents and the main document:
%
\begin{center}
|\ifchilddoc |\textit{child-code}| |[|\||else |\textit{main-code}]| \||fi|
\end{center}

%%%%%%%%%%%%%%%%%%%%%%%%%%%%%%%%%%%%%%%%
\DescribeMacro{\childdocname}
\DescribeMacro{\childdocjob}
The macro |\childdocname| contains the filename (without extension)
of the main or child file being processed.
Note that |\childdocjob| will always contain the name of the main file.

%%%%%%%%%%%%%%%%%%%%%%%%%%%%%%%%%%%%%%%%
\paragraph{Title Page.}

Conditional processing can be used to include a title or banner page
in the main document when proper precautions are taken.
Importantly, the code in the main file should ensure that the page counter
(as well as other status parameters which are stored in the |.aux| files)
takes the same value after the conditional processing.
Otherwise the page numbers may take divergent values
depending on which part is compiled.

For example, a title page could be declared by:
%
\begin{center}
\begin{tabular}{l}
|\ifchilddoc\||else|\\
|\addtocounter{page}{-1}|\\
\textit{code for title page}\\
|\newpage|\\
|\||fi|
\end{tabular}
\end{center}
%
A banner page for the child documents can be generated by:
%
\begin{center}
\begin{tabular}{l}
|\ifchilddoc|\\
|\addtocounter{page}{-1}|\\
\textit{code for banner page}\\
|\newpage|\\
|\||fi|
\end{tabular}
\end{center}
%
Here one could write a message such as:
\begin{center}
|This is the part \childdocname{} of \childdocjob{}.|
\end{center}

%%%%%%%%%%%%%%%%%%%%%%%%%%%%%%%%%%%%%%%%%%%%%%%%%%%%%%%%%%%%%%%%%%%%%%%%%%%%%%%%
\subsection{Flags}
\label{sec:flags}

The package makes it easy to generate different versions
of the main or child documents.
To this end compilation flags can be defined
and assigned different default values.
They will be particularly useful in conjunction
with the forwarding mechanism described in \secref{sec:forward}.

For example, it may be useful to have a flag |\version|
which can be set to |draft| or |final|.
The document source will contain some conditional code
depending on the value of |\version|.
Suppose further, the flag should default to |final| for the main file
and to |draft| for child files
which is a natural assignment for editing the document.
This is achieved by placing the following code
in the preamble of the main document
(below the |\childdocmain| directive):
%
\begin{center}
\begin{tabular}{l}
|\ifchilddoc|\\
|\providecommand{\version}{draft}|\\
|\||else|\\
|\providecommand{\version}{final}|\\
|\||fi|
\end{tabular}
\end{center}
%
The definition by |\providecommand| makes sure
that previous definitions are not overwritten.
Further statements |\providecommand{\version}{...}|
can thus be added before the above code to override it.

For the main file, one might add a line
(between |\childdocmain| and the above block)
%
\begin{center}
|%\ifchilddoc\||else\providecommand{\version}{draft}\||fi|
\end{center}
%
which can be uncommented to produce a draft version.
Likewise one can add a line to the very top of a child file
(above the |\childdocof{|\textit{main}|}| directive)
%
\begin{center}
|%\providecommand{\version}{final}|
\end{center}
%
which can be uncommented to produce the final version of this child document.

%%%%%%%%%%%%%%%%%%%%%%%%%%%%%%%%%%%%%%%%%%%%%%%%%%%%%%%%%%%%%%%%%%%%%%%%%%%%%%%%
\subsection{Forwarding}
\label{sec:forward}

Different versions of the main or child documents
using compilation flags as described in \secref{sec:flags}
can be (permanently) stored in different files
for convenient compilation, viewing and distribution.
To this end, the package defines a command
to pass on compilation to a different file:

%%%%%%%%%%%%%%%%%%%%%%%%%%%%%%%%%%%%%%%%
\DescribeMacro{\childdocforward}
The command |\childdocforward| redirects processing to
another source file:
%
\begin{center}
\begin{tabular}{l}
|\input{childdoc.def}|\\
|\childdocforward[|\textit{main}|]{|\textit{dest}|}|\\
\end{tabular}
\end{center}
%
The argument \textit{dest} is the destination file
(without extension).
It should be the main file or one of the child files.
Note that further \textsf{childdoc} directives
such as |\childdocof| and |\childdocforward|
in the indicated file will be processed in this form.
The optional argument \textit{main}
passes on directly to the main file \textit{main}
while pretending to compile the child \textit{dest}.
This form behaves as if \textit{dest}
issues |\childdocof{|\textit{main}|}| right away,
and no further \textsf{childdoc} directives will be processed.

%%%%%%%%%%%%%%%%%%%%%%%%%%%%%%%%%%%%%%%%
\DescribeMacro{\...prefix}
In the alternative form |\childdocforwardprefix|,
%
\begin{center}
\begin{tabular}{l}
|\input{childdoc.def}|\\
|\childdocforwardprefix[|\textit{main}|]{|\textit{prefix}|}{|\textit{dest}|}|
\end{tabular}
\end{center}
%
the destination file is determined by a pattern
depending on the current file:
To make this work, the current file must be called
`{\textit{prefix}\hspace{0.2em}\textit{suffix}}'
with \textit{prefix} matching precisely the argument.
Processing is then passed on to the file
`{\textit{dest}\hspace{0.2em}\textit{suffix}}'.
Surely, the same effect is achieved by
directly specifying the
argument `{\textit{dest}\hspace{0.2em}\textit{suffix}}'
in the first form.
However, that requires to set up a different file
for each child. With the alternative form of the command
all these files can have exactly the same content
which simplifies setting them up and maintaining them.

For example, the following file |draft.tex|
with a compilation flag |\version| as described in \secref{sec:flags}
compiles the main document as a draft:
%
\begin{center}
\begin{tabular}{l}
|\def\version{draft}|\\
|\input{childdoc.def}|\\
|\childdocforward{|\textit{main}|}|
\end{tabular}
\end{center}
%
Likewise, the following files |final|\textit{nn}|.tex|
compile the final version of the child document
|child|\textit{nn}|.tex|:
%
\begin{center}
\begin{tabular}{l}
|\def\version{final}|\\
|\input{childdoc.def}|\\
|\childdocforwardprefix{final}{child}|
\end{tabular}
\end{center}
%

Note that when several versions of a main file and/or of each child file
are to be generated, it may be convenient to set up a |Makefile| or
shell script to automatise the process.

%%%%%%%%%%%%%%%%%%%%%%%%%%%%%%%%%%%%%%%%%%%%%%%%%%%%%%%%%%%%%%%%%%%%%%%%%%%%%%%%
\subsection{Command Line Processing}
\label{sec:commandline}

The effect of redirection files can also be achieved by invoking
the \LaTeX{} compiler with a more elaborate command line.
Most conveniently this should be done as part
of a shell script or a |Makefile|.

When using \textsf{childdoc} in the main file, the following
command lines effectively perform a redirection
(note that depending on the shell being used,
backslashes may have to be doubled: `|\|' $\to$ `|\\|'):
%
\begin{center}
|... -jobname "|\textit{target}|" |\\|"|[\textit{flags}]%
|\input{childdoc.def}\childdocforward[|\textit{main}|]{|\textit{dest}|}"|
\end{center}
%
Here \textit{target} is the name of the output file,
\textit{main} is the name of the main file
and \textit{dest} is the name of the main or child file to be processed
(all filenames without extensions).
The optional argument \textit{main} can be omitted
if \textit{main} matches \textit{dest}.
Optionally, compilation \textit{flags} can be defined via |\def| commands.
This command line makes the \TeX{} engine believe
it is compiling the file \textit{target}
whose content is specified as the latter parameter.
The provided code then forwards the processing to
\textit{main} or \textit{dest} as described in \secref{sec:forward}.

%%%%%%%%%%%%%%%%%%%%%%%%%%%%%%%%%%%%%%%%%%%%%%%%%%%%%%%%%%%%%%%%%%%%%%%%%%%%%%%%
\subsection{Include by Input}
\label{sec:input}

Including child documents by |\include| has some restrictions by design.
Most notably, the content of a child document always occupies
its own set of pages; pages cannot be shared between child documents.
Usually, this behaviour makes perfect sense
because each child document contain an essential part of the document.
However, in some situations it may be desirable to compose
a document from a collection of parts
without having mandatory page breaks between then.
For this case, the package
provides a mechanism to include parts
by |\input| which can also be processed individually.
However, by construction this mechanism
requires manual handling of the content to be output.

%%%%%%%%%%%%%%%%%%%%%%%%%%%%%%%%%%%%%%%%
\DescribeMacro{\ifchilddocmanual}
The main file should be prepared as usual, see \secref{sec:include}.
However, the document body must make a distinction
between processing of an individual part and of the main document, e.g.:
%
\begin{center}
\begin{tabular}{l}
|\ifchilddocmanual|\\
|\input{\childdocname}|\\
|\||else|\\
\textit{document body with }|\input{|\textit{part}|}|\\
|\||fi|
\end{tabular}
\end{center}
%
The conditional |\ifchilddocmanual| is true whenever
a part to be included by |\input| is being compiled,
and the name of the part is stored in |\childdocname|.

%%%%%%%%%%%%%%%%%%%%%%%%%%%%%%%%%%%%%%%%
\DescribeMacro{\childdocby}
Each part to be included by |\input| should start with:
%
\begin{center}
\begin{tabular}{l}
|\input{childdoc.def}|\\
|\childdocby{|\textit{main}|}|\\
\end{tabular}
\end{center}
%
The directive |\childdocby| is similar to |\childdocof|
described in \secref{sec:include},
but the subsequent selection of content must be done manually.
To that end, both |\ifchilddoc| and |\ifchilddocmanual|
will be true upon processing of a part,
and the name of the part is stored in |\childdocname|.
Note that |\jobname| will be set to the filename of the current part
so that each part receives an individual |.aux| file
that does not interfere with the |.aux| file(s) of the main document.
This behaviour can be altered by the alternative form
|\childdocby[*]{|\textit{main}|}| (with a non-empty optional argument)
which uses the |.aux| file of the main document
by setting |\jobname| to \textit{main}.

%%%%%%%%%%%%%%%%%%%%%%%%%%%%%%%%%%%%%%%%%%%%%%%%%%%%%%%%%%%%%%%%%%%%%%%%%%%%%%%%
\subsection{Driver Development}
\label{sec:driver}

The \textsf{childdoc} mechanism can also be use for the development
of definition files such as \LaTeX{} styles or classes.
This case differs from the above setup with multiple parts
included by |\include| in that no |\includeonly| should be invoked.
This can be achieved by starting the include file
(before |\ProvidesPackage|) with:
%
\begin{center}
\begin{tabular}{l}
|\input{childdoc.def}|\\
|\childdocforward{|\textit{main}|}|\\
\end{tabular}
\end{center}
%
or alternatively with:
%
\begin{center}
\begin{tabular}{l}
|\input{childdoc.def}|\\
|\childdocby{|\textit{main}|}|\\
\end{tabular}
\end{center}
%
Both forms have slightly different effects as described above.
The main file is prepared as usual, see \secref{sec:include}.

%%%%%%%%%%%%%%%%%%%%%%%%%%%%%%%%%%%%%%%%%%%%%%%%%%%%%%%%%%%%%%%%%%%%%%%%%%%%%%%%
\subsection{Legacy Detection}
\label{sec:detection}

The directive |\childdocmain| in the main file can detect
whether the complete document or merely a child is to be compiled
even without using the directive |\childdocof|.
This method is deprecated because it is less robust
and there is no compelling reason to use it;
it is merely provided for backward compatibility
and it may be removed in future versions.

If the detection mechanism is to be used,
it is mandatory to correctly specify
the filename of the main file as the argument of |\childdocmain|:
%
\begin{center}
\begin{tabular}{l}
|\input{childdoc.def}|\\
|\childdocmain{|\textit{main}|}|\\
\end{tabular}
\end{center}
%
If |\jobname| does not match the argument \textit{main} of |\childdocmain|,
it is assumed that |\jobname| points to the child file to be compiled.
When using |\childdocmain| with the main file specified as argument,
it suffices to start a child file
with just |\input{|\textit{main}|}|
without loading of the package and using |\childdocof|.
If instead all processing is done
with the appropriate \textsf{childdoc} directives,
the argument of \textit{main} of |\childdocmain| can be empty.

An alternative version of the command line processing described
in \secref{sec:commandline} using the detection mechanism reads:
%
\begin{center}
|... -jobname "|\textit{target}|" "|[\textit{flags}]%
[|\def\jobname{|\textit{dest}|}|]|\input{|\textit{main}|}"|
\end{center}

%%%%%%%%%%%%%%%%%%%%%%%%%%%%%%%%%%%%%%%%%%%%%%%%%%%%%%%%%%%%%%%%%%%%%%%%%%%%%%%%
\subsection{Manual Code}
\label{sec:manual}

In case one cannot be certain whether the definitions file |childdoc.def|
is installed on the target \TeX{} distribution
and one prefers not to ship it,
it is conceivable to paste a few relevant commands into the sources.

To that end, drop all statements |\input{childdoc.def}|
and perform the replacements as outlined below.
Instead of |\childdocmain{|\textit{main}|}| add the following code
to the top of the main file:
%
\begin{center}
\begin{tabular}{l}
|\||ifdefined\childdocname\endinput\||fi\newif\ifchilddoc|\\
|\edef\childdocname{\scantokens\expandafter{\jobname\noexpand}}|\\
|\def\childdocmain{|\textit{main}|}\||ifx\childdocmain\childdocname\||else|\\
|\childdoctrue\includeonly{\childdocname}\let\jobname\childdocmain\||fi|\\
\end{tabular}
\end{center}
%
Instead of |\childdocof{|\textit{main}|}| just include the main file
at the top of each child file:
%
\begin{center}
|\input{|\textit{main}|}|
\end{center}
%
A simple redirection |\childdocforward{|\textit{dest}|}| is achieved by:
%
\begin{center}
|\def\jobname{|\textit{dest}|}\input{\jobname}|
\end{center}
%
The redirection with prefix
|\childdocforwardprefix[|\textit{prefix}|]{|\textit{dest}|}|
is accomplished by:
%
\begin{center}
\begin{tabular}{l}
|{\edef\jobname{\scantokens\expandafter{\jobname\noexpand}}|\\
|\def\redirectjob |\textit{prefix}|#1~~~{\gdef\jobname{|\textit{dest}|#1}}|\\
|\expandafter\redirectjob\jobname~~~}\input{\jobname}|
\end{tabular}
\end{center}

In an alternative approach,
child documents can be compiled by a specific command line
without additional code or specific definitions:
%
\begin{center}
|... -jobname "|\textit{target}|" "|[\textit{flags}]%
|\includeonly{|\textit{dest}|}\input{|\textit{main}|}"|
\end{center}
%

%%%%%%%%%%%%%%%%%%%%%%%%%%%%%%%%%%%%%%%%%%%%%%%%%%%%%%%%%%%%%%%%%%%%%%%%%%%%%%%%
%%%%%%%%%%%%%%%%%%%%%%%%%%%%%%%%%%%%%%%%%%%%%%%%%%%%%%%%%%%%%%%%%%%%%%%%%%%%%%%%
\section{Information}

%%%%%%%%%%%%%%%%%%%%%%%%%%%%%%%%%%%%%%%%%%%%%%%%%%%%%%%%%%%%%%%%%%%%%%%%%%%%%%%%
\subsection{Copyright}

Copyright \copyright{} 2017--2018 Niklas Beisert

This work may be distributed and/or modified under the
conditions of the \LaTeX{} Project Public License, either version 1.3
of this license or (at your option) any later version.
The latest version of this license is in
  \url{http://www.latex-project.org/lppl.txt}
and version 1.3 or later is part of all distributions of \LaTeX{}
version 2005/12/01 or later.

This work has the LPPL maintenance status `maintained'.

The Current Maintainer of this work is Niklas Beisert.

This work consists of the files |README.txt|, |childdoc.ins| and |childdoc.dtx|
as well as the derived files |childdoc.def|, |cdocsamp.tex|
with |cdocsch1.tex|, |cdocsch2.tex|, |cdocspt3.tex|, |cdocspt4.tex|,
|cdocsdrf.tex|, |cdocsfn1.tex|, |cdocsfn2.tex|
as well as |childdoc.pdf|.

%%%%%%%%%%%%%%%%%%%%%%%%%%%%%%%%%%%%%%%%%%%%%%%%%%%%%%%%%%%%%%%%%%%%%%%%%%%%%%%%
\subsection{Files and Installation}

The package consists of the files:
%
\begin{center}
\begin{tabular}{ll}
    |README.txt|   & readme file \\
    |childdoc.ins| & installation file \\
    |childdoc.dtx| & source file \\
    |childdoc.def| & definition file \\
    |cdocsamp.tex| & sample main file \\
    |cdocsch1.tex| & sample include file \\
    |cdocsch2.tex| & sample include file \\
    |cdocspt3.tex| & sample part file \\
    |cdocspt4.tex| & sample part file \\
    |cdocsdrf.tex| & sample redirection file \\
    |cdocsfn1.tex| & sample redirection file \\
    |cdocsfn2.tex| & sample redirection file \\
    |childdoc.pdf| & manual
\end{tabular}
\end{center}
%
The distribution consists of the files
|README.txt|, |childdoc.ins| and |childdoc.dtx|.
%
\begin{itemize}
\item
Run (pdf)\LaTeX{} on |childdoc.dtx|
to compile the manual |childdoc.pdf| (this file).
\item
Run \LaTeX{} on |childdoc.ins| to create the definitions file |childdoc.def|
and the sample |cdocsamp.tex| with include files
|cdocsch1.tex|, |cdocsch2.tex|, |cdocspt3.tex|, |cdocspt4.tex|,
|cdocsdrf.tex|, |cdocsfn1.tex|, |cdocsfn2.tex|.
Then copy the file |childdoc.def| to an appropriate directory of your \LaTeX{}
distribution, e.g.\ \textit{texmf-root}|/tex/latex/childdoc|.
\end{itemize}

%%%%%%%%%%%%%%%%%%%%%%%%%%%%%%%%%%%%%%%%%%%%%%%%%%%%%%%%%%%%%%%%%%%%%%%%%%%%%%%%
\subsection{Related CTAN Packages}

There are several other packages which offer a similar functionality:
%
\begin{itemize}
\item
The packages
\href{http://ctan.org/pkg/docmute}{\textsf{docmute}},
\href{http://ctan.org/pkg/includex}{\textsf{includex}} and
\href{http://ctan.org/pkg/standalone}{\textsf{standalone}}
provide commands to include only the document body of
a child file thus allowing both files to be compiled individually.
\item
The packages \href{http://ctan.org/pkg/subdocs}{\textsf{subdocs}}
and \href{http://ctan.org/pkg/subfiles}{\textsf{subfiles}}
provide structures in which the main and child documents can be
encapsulated and allowing them to be compiled individually.
The inclusion mechanism is different from the conventional |\include|.
\item
The package \href{http://ctan.org/pkg/combine}{\textsf{combine}}
is an elaborate solution to combine several documents into one.
\end{itemize}
%
See also the CTAN topic \href{http://ctan.org/topic/subdocs}{\textsf{subdocs}}
for further related packages.
The present package differs from the above solutions in that
a document structure constructed with the conventional |\include| mechanism
just needs two extra commands at the top of every file
such that all constituent files can be compiled individually.

%%%%%%%%%%%%%%%%%%%%%%%%%%%%%%%%%%%%%%%%%%%%%%%%%%%%%%%%%%%%%%%%%%%%%%%%%%%%%%%%
%\subsection{Feature Suggestions}
%
%The following is a list of features which may be useful for future
%versions of this package:
%%
%\begin{itemize}
%\item
%\ldots
%\end{itemize}

%%%%%%%%%%%%%%%%%%%%%%%%%%%%%%%%%%%%%%%%%%%%%%%%%%%%%%%%%%%%%%%%%%%%%%%%%%%%%%%%
\subsection{Revision History}

%%%%%%%%%%%%%%%%%%%%%%%%%%%%%%%%%%%%%%%%
\paragraph{v2.0:} 2018/12/30

\begin{itemize}
\item
immediate forward processing
\item
added |\childdocby| mechanism
\item
manual restructured
\end{itemize}

%%%%%%%%%%%%%%%%%%%%%%%%%%%%%%%%%%%%%%%%
\paragraph{v1.6:} 2018/01/17

\begin{itemize}
\item
application for development of include files
\item
corrections to manual
\end{itemize}

%%%%%%%%%%%%%%%%%%%%%%%%%%%%%%%%%%%%%%%%
\paragraph{v1.5:} 2017/05/21

\begin{itemize}
\item
more complete structuring introduced
\item
|\childdocof| introduced
\item
|\childdoc| renamed to |\childdocmain|
\item
|\childredirect| renamed to |\childdocforward| and |\childdocforwardprefix|
and functionality expanded
\end{itemize}

%%%%%%%%%%%%%%%%%%%%%%%%%%%%%%%%%%%%%%%%
\paragraph{v1.0:} 2017/04/27

\begin{itemize}
\item
manual and install package
\item
first version published on CTAN
\end{itemize}

%%%%%%%%%%%%%%%%%%%%%%%%%%%%%%%%%%%%%%%%
\paragraph{v0.6:} 2017/04/26

\begin{itemize}
\item
redirection mechanism added
\end{itemize}

%%%%%%%%%%%%%%%%%%%%%%%%%%%%%%%%%%%%%%%%
\paragraph{v0.5:} 2017/04/26

\begin{itemize}
\item
functionality in definition file
\end{itemize}


%%%%%%%%%%%%%%%%%%%%%%%%%%%%%%%%%%%%%%%%%%%%%%%%%%%%%%%%%%%%%%%%%%%%%%%%%%%%%%%%
%%%%%%%%%%%%%%%%%%%%%%%%%%%%%%%%%%%%%%%%%%%%%%%%%%%%%%%%%%%%%%%%%%%%%%%%%%%%%%%%
%%%%%%%%%%%%%%%%%%%%%%%%%%%%%%%%%%%%%%%%%%%%%%%%%%%%%%%%%%%%%%%%%%%%%%%%%%%%%%%%
\appendix

\settowidth\MacroIndent{\rmfamily\scriptsize 000\ }

 \DocInput{childdoc.dtx}

\end{document}
%</driver>
% \fi
%
% %%%%%%%%%%%%%%%%%%%%%%%%%%%%%%%%%%%%%%%%%%%%%%%%%%%%%%%%%%%%%%%%%%%%%%%%%%%%%%
% %%%%%%%%%%%%%%%%%%%%%%%%%%%%%%%%%%%%%%%%%%%%%%%%%%%%%%%%%%%%%%%%%%%%%%%%%%%%%%
% \section{Sample}
%\iffalse
%<*samplemain>
%\fi
%
% The following presents a sample document
% with two chapters, two parts, a title page,
% a compile flag as well as three forwarding files to set the flag.
% It consists of eight |.tex| files:
% \begin{center}
% \begin{tabular}{ll}
% |cdocsamp.tex|&main file\\
% |cdocsch1.tex|&include file for chapter 1\\
% |cdocsch2.tex|&include file for chapter 2\\
% |cdocspt3.tex|&include file for part 3\\
% |cdocspt4.tex|&include file for part 4\\
% |cdocsdrf.tex|&forwarding file for main file in draft mode\\
% |cdocsfi1.tex|&forwarding file for final version of chapter 1\\
% |cdocsfi2.tex|&forwarding file for final version of chapter 2\\
% \end{tabular}
% \end{center}
% Each of the eight files can be compiled directly by the \LaTeX{} compiler.
%
% %%%%%%%%%%%%%%%%%%%%%%%%%%%%%%%%%%%%%%
% \paragraph{Main File.}
%
% The main file is called |cdocsamp.tex|.
%
% Load the \textsf{childdoc} definitions and
% declare the filename for the main document:
%    \begin{macrocode}
\input{childdoc.def}
\childdocmain{}
%    \end{macrocode}

% Optional override for |\version| flag:
%    \begin{macrocode}
%%\ifchilddoc\else\providecommand{\version}{draft}\fi
%    \end{macrocode}

% Define the default values for the |\version| flag
% (|final| for the main file and |draft| for childs):
%    \begin{macrocode}
\ifchilddoc
\providecommand{\version}{draft}
\else
\providecommand{\version}{final}
\fi
%    \end{macrocode}

% Load the standard document class:
%    \begin{macrocode}
\documentclass[12pt]{article}
%    \end{macrocode}

% Start the document body:
%    \begin{macrocode}
\begin{document}
%    \end{macrocode}

% Declare a title page.
% Print title, part of document being processed and version flag:
%    \begin{macrocode}
\addtocounter{page}{-1}
\begin{center}
{\LARGE\bfseries{}childdoc example\par}
\vspace{1cm}
\ifchilddoc
\ifchilddocmanual part\else chapter\fi:
`\childdocname' of `\childdocjob'\par
\else
main document: `\childdocjob'\par
\fi
version: \version\par
\end{center}
\newpage
%    \end{macrocode}

% Manually include selected file,
% otherwise process as usual:
%    \begin{macrocode}
\ifchilddocmanual
\section*{part `\childdocname'}
\input{\childdocname}
\else
%    \end{macrocode}

% Include the two chapters:
%    \begin{macrocode}
\include{cdocsch1}
\include{cdocsch2}
%    \end{macrocode}

% Include the two parts unless only chapters should be displayed:
%    \begin{macrocode}
\ifchilddoc\else
\section{part three}
\input{cdocspt3}
\section{part four}
\input{cdocspt4}
\fi
%    \end{macrocode}

% Process as usual until here:
%    \begin{macrocode}
\fi
%    \end{macrocode}

% End of document body:
%    \begin{macrocode}
\end{document}
%    \end{macrocode}
%\iffalse
%</samplemain>
%\fi
%
% %%%%%%%%%%%%%%%%%%%%%%%%%%%%%%%%%%%%%%
% \paragraph{Chapter Include Files.}
%
% The include files are called |cdocsch1.tex| and |cdocsch2.tex|.
%
%\iffalse
%<*samplechap1|samplechap2>
%\fi

% Optional override for |\version| flag:
%    \begin{macrocode}
%%\providecommand{\version}{final}
%    \end{macrocode}

% Include the main document:
%    \begin{macrocode}
\input{childdoc.def}
\childdocof{cdocsamp}
%    \end{macrocode}

%\iffalse
%</samplechap1|samplechap2>
%\fi
%
%\iffalse
%<*samplechap1>
%\fi
% Some text for chapter 1:
%    \begin{macrocode}
\section{one}
some text in chapter one
%    \end{macrocode}

%\iffalse
%</samplechap1>
%\fi
% Some text for chapter 2:
%\iffalse
%<*samplechap2>
%\fi
%    \begin{macrocode}
\section{two}
more text in chapter two
%    \end{macrocode}

%\iffalse
%</samplechap2>
%\fi
%
% %%%%%%%%%%%%%%%%%%%%%%%%%%%%%%%%%%%%%%
% \paragraph{Part Include Files.}
%
% The include files are called |cdocspt3.tex| and |cdocspt4.tex|.
%
%\iffalse
%<*samplepart3|samplepart4>
%\fi

% Optional override for |\version| flag:
%    \begin{macrocode}
%%\providecommand{\version}{final}
%    \end{macrocode}

% Include the main document:
%    \begin{macrocode}
\input{childdoc.def}
\childdocby{cdocsamp}
%    \end{macrocode}

%\iffalse
%</samplepart3|samplepart4>
%\fi
%
%\iffalse
%<*samplepart3>
%\fi
% Some text for part 3:
%    \begin{macrocode}
some text in part three
%    \end{macrocode}

%\iffalse
%</samplepart3>
%\fi
% Some text for part 4:
%\iffalse
%<*samplepart4>
%\fi
%    \begin{macrocode}
more text in part four
%    \end{macrocode}

%\iffalse
%</samplepart4>
%\fi
%
% %%%%%%%%%%%%%%%%%%%%%%%%%%%%%%%%%%%%%%
% \paragraph{Forwarding for a Complete Draft.}
%
% The following forwarding file |cdocsdrf.tex|
% compiles the main document in draft mode:
%\iffalse
%<*sampledraft>
%\fi
%    \begin{macrocode}
\def\version{draft}
\input{childdoc.def}
\childdocforward{cdocsamp}
%    \end{macrocode}

%\iffalse
%</sampledraft>
%\fi
%
% %%%%%%%%%%%%%%%%%%%%%%%%%%%%%%%%%%%%%%
% \paragraph{Forwarding for Final Version of the Chapters.}
%
% The following forwarding files |cdocsfn1.tex| and |cdocsfn2.tex|
% (with identical content)
% compile the final versions of the child documents
% |cdocsch1.tex| and |cdocsch2.tex|, respectively:
%\iffalse
%<*samplefinal>
%\fi
%    \begin{macrocode}
\def\version{final}
\input{childdoc.def}
\childdocforwardprefix[cdocsamp]{cdocsfn}{cdocsch}
%    \end{macrocode}

%\iffalse
%</samplefinal>
%\fi
%
% %%%%%%%%%%%%%%%%%%%%%%%%%%%%%%%%%%%%%%
% \paragraph{Command Line Processing.}
%
% The following three command lines generate the output files
% |cdocscld|, |cdocscl1| and |cdocscl2|
% which should be identical to
% |cdocsdrf|, |cdocsch1| and |cdocsfn2|, respectively:
% \begin{center}
% \begin{tabular}{l}
% |latex -jobname cdocscld \|\\
% |  "\def\version{draft}\input{childdoc.def}\childdocforward{cdocsamp}"|\\
% |latex -jobname cdocscl1 \|\\
% |  "\input{childdoc.def}\childdocforward[cdocsamp]{cdocsch1}"|\\
% |latex -jobname cdocscl2 \|\\
% |  "\def\version{final}\input{childdoc.def}\childdocforward{cdocsch2}"|
% \end{tabular}
% \end{center}
% Note that the trailing backslash on each first line
% merely continues the input to the second line
% (for convenient cut ant paste).
% Furthermore, the command |latex| can be replaced by any
% of its alternative versions such as |pdflatex|.
%
% %%%%%%%%%%%%%%%%%%%%%%%%%%%%%%%%%%%%%%%%%%%%%%%%%%%%%%%%%%%%%%%%%%%%%%%%%%%%%%
% %%%%%%%%%%%%%%%%%%%%%%%%%%%%%%%%%%%%%%%%%%%%%%%%%%%%%%%%%%%%%%%%%%%%%%%%%%%%%%
% \section{Implementation}
%\iffalse
%<*package>
%\fi
%
% This section describes the definitions file |childdoc.def|.

% The definitions cannot be loaded using |\usepackage| or |\RequirePackage|
% which has a mechanism to prevent loading a style file more than once.
% When loading the definitions by means of |\input|
% multiple instances have to be prevented manually:
%\iffalse
%This code needs to be before the `\ProvidesFile' directive
%which is defined at the beginning of this file.
%Therefore it is also placed there and commented out here.
%</package>
%<*discard>
%\fi
%    \begin{macrocode}
\ifdefined\childdocmain\endinput\fi
%    \end{macrocode}
%\iffalse
%</discard>
%<*package>
%\fi
%
% \macro{\ifchilddoc}
% \macro{\ifchilddocmanual}
% The conditional |\ifchilddoc| tells whether a
% child (true) or main (false) document is being compiled.
% The conditional |\ifchilddocmanual| tells whether
% the |\includeonly| mechanism is used (false) or
% the selection of child files must be performed manually (true).
% The definitions initialise to false:
%    \begin{macrocode}
\newif\ifchilddoc
\newif\ifchilddocmanual
%    \end{macrocode}

% \macro{\childdocname}
% \macro{\childdocjob}
% The macro |\childdocname| stores the name of the main document
% to be compiled. The macro |\childdocjob| stores the name of
% the document on which the \LaTeX{} compiler was originally invoked.
% The content of |\jobname| cannot be compared
% to filenames specified in the source due to different catcodes.
% The following code rescans |\jobname|, stores the result
% in |\childdocname| and saves a copy in |\childdocjob|:
%    \begin{macrocode}
\edef\childdocname{\scantokens\expandafter{\jobname\noexpand}}
\let\childdocjob\childdocname
%    \end{macrocode}

% \macro{\childdocdisable}
% The macro |\childdocdisable| prevents the main file
% from being processed more than once.
% At this stage, the main document command |\childdocmain|
% is assumed to be called once again where it should do nothing.
% Any subsequent call to it should prevent
% a secondary processing of the main document
% It overwrites the forwarding commands
% |\childdocof| and |\childdocforward|
% with empty macros to prevent further inclusions of the main document:
%    \begin{macrocode}
\newcommand{\childdocdisable}
{
  \renewcommand{\childdocmain}[1]{\renewcommand{\childdocmain}[1]{\endinput}}
  \renewcommand{\childdocof}[1]{}
  \renewcommand{\childdocby}[2][]{}
  \renewcommand{\childdocforward}[2][]{}
  \renewcommand{\childdocdisable}{}
}
%    \end{macrocode}

% \macro{\childdocmain}
% The macro |\childdocmain| is to be called at the top of the main file
% with nothing or the main filename (without extension) as argument.
% First, it breaks loops.
% If the argument is not empty and does not match |\childdocname|
% (which is set by the first inclusion of |childdoc.def|),
% |\ifchilddoc| is set to true, |\includeonly| is applied to the child file
% and |\jobname| is set to the main file
% (for proper handling of |.aux| files):
%    \begin{macrocode}
\newcommand{\childdocmain}[1]
{
  \childdocdisable\childdocmain{}
  \if?#1?\else
    \begingroup
      \def\childdoctmp{#1}
      \ifx\childdoctmp\childdocname
        \def\childdoctmp{}
      \else
        \def\childdoctmp
        {
          \childdoctrue
          \includeonly{\childdocname}
          \def\childdocjob{#1}
          \def\jobname{#1}
        }
      \fi
      \expandafter
    \endgroup
    \childdoctmp
  \fi
}
%    \end{macrocode}

% \macro{\childdocof}
% The command |\childdocof| redirects
% compilation to the main file |#1|.
%    \begin{macrocode}
\newcommand{\childdocof}[1]
{
  \childdocdisable
  \childdoctrue
  \includeonly{\childdocname}
  \def\jobname{#1}
  \def\childdocjob{#1}
  \input{#1}
}
%    \end{macrocode}

% \macro{\childdocby}
% The command |\childdocby| ....
%    \begin{macrocode}
\newcommand{\childdocby}[2][]
{
  \childdocdisable
  \childdoctrue
  \childdocmanualtrue
  \if?#1?\else
    \def\jobname{#2}
  \fi
  \def\childdocjob{#2}
  \input{#2}
  \endinput
}
%    \end{macrocode}

% \macro{\childdocforward}
% The command |\childdocforward| redirects
% compilation to the main file or
% (if the optional argument is given) a child file.
% Parameters are set as if the main file
% or a child file starting with |\childdocof| was compiled.
% Then compilation is handed over to the main file:
%    \begin{macrocode}
\newcommand{\childdocforward}[2][]
{
  \begingroup
    \if?#1?
      \def\childdoctmp
      {
        \def\childdocname{#2}
        \def\childdocjob{#2}
        \def\jobname{#2}
        \input{#2}
        \endinput
      }
    \else
      \def\childdoctmp
      {
        \childdocdisable
        \def\childdocname{#2}
        \childdoctrue
        \includeonly{#2}
        \def\childdocjob{#1}
        \def\jobname{#1}
        \input{#1}
        \endinput
      }
    \fi
    \expandafter
  \endgroup
  \childdoctmp
}
%    \end{macrocode}

% \macro{\childdocforwardprefix}
% The command |\childdocforwardprefix| redirects
% compilation to the main or a child file by means of a pattern.
% The prefix |#1| in the current filename is replaced by |#2|
% and the suffix of the current filename is kept
% (it is assumed that the filename does not contain the substring `|~~~|'
% which is used as a delimiter).
% Compilation is handed over to the new file by |\childdocforward|:
%    \begin{macrocode}
\newcommand{\childdocforwardprefix}[3][]
{
  \begingroup
    \def\childdocextract #2##1~~~{\def\childdoctmp{\childdocforward[#1]{#3##1}}}
    \expandafter\childdocextract\childdocname~~~
    \expandafter
  \endgroup
  \childdoctmp
}
%    \end{macrocode}

% \macro{\childdoc}
% The deprecated macro |\childdoc| is a legacy version of |\childdocmain|:
%    \begin{macrocode}
\newcommand{\childdoc}{\childdocmain}
%    \end{macrocode}

% \macro{\childdocredirect}
% The deprecated macro |\childdocredirect| is a legacy version
% of |\childdocforward| and |\childdocforwardprefix|:
%    \begin{macrocode}
\newcommand{\childdocredirect}[2][]
{
  \begingroup
    \if?#1?
      \def\childdoctmp{\childdocforward{#2}}
    \else
      \def\childdoctmp{\childdocforwardprefix{#1}{#2}}
    \fi
    \expandafter
  \endgroup
  \childdoctmp
}
%    \end{macrocode}

%\iffalse
%</package>
%\fi
%
\endinput
\childdocforward[cdocsamp]{cdocsch1}"|\\
% |latex -jobname cdocscl2 \|\\
% |  "\def\version{final}% \iffalse
%
% childdoc.dtx Copyright (C) 2017-2018 Niklas Beisert
%
% This work may be distributed and/or modified under the
% conditions of the LaTeX Project Public License, either version 1.3
% of this license or (at your option) any later version.
% The latest version of this license is in
%   http://www.latex-project.org/lppl.txt
% and version 1.3 or later is part of all distributions of LaTeX
% version 2005/12/01 or later.
%
% This work has the LPPL maintenance status `maintained'.
%
% The Current Maintainer of this work is Niklas Beisert.
%
% This work consists of the files childdoc.dtx and childdoc.ins
% and the derived files childdoc.def and cdocsamp.tex with
% cdocsch1.tex, cdocsch2.tex, cdocsdrf.tex, cdocsfn1.tex, cdocsfn2.tex.
%
%<package>\ifdefined\childdocmain\endinput\fi
%<package>\ProvidesFile{childdoc.def}[2018/12/30 v2.0 child document driver]
%<samplemain>\ProvidesFile{cdocsamp.tex}[2018/12/30 v2.0 sample for childdoc]
%<*driver>
%\ProvidesFile{childdoc.drv}[2018/12/30 v2.0 childdoc reference manual file]
\PassOptionsToClass{10pt,a4paper}{article}
\documentclass{ltxdoc}

\usepackage[margin=35mm]{geometry}
\usepackage{hyperref}
\usepackage{hyperxmp}
\usepackage[usenames]{color}

\hypersetup{colorlinks=true}
\hypersetup{pdfstartview=FitH}
\hypersetup{pdfpagemode=UseNone}
\hypersetup{pdfsource={}}
\hypersetup{pdflang={en-UK}}
\hypersetup{pdfcopyright={Copyright 2017-2018 Niklas Beisert.
  This work may be distributed and/or modified under the
  conditions of the LaTeX Project Public License, either version 1.3
  of this license or (at your option) any later version.}}
\hypersetup{pdflicenseurl={http://www.latex-project.org/lppl.txt}}
\hypersetup{pdfcontactaddress={ETH Zurich, ITP, HIT K,
  Wolfgang-Pauli-Strasse 27}}
\hypersetup{pdfcontactpostcode={8093}}
\hypersetup{pdfcontactcity={Zurich}}
\hypersetup{pdfcontactcountry={Switzerland}}
\hypersetup{pdfcontactemail={nbeisert@itp.phys.ethz.ch}}
\hypersetup{pdfcontacturl={http://people.phys.ethz.ch/\xmptilde nbeisert/}}

\newcommand{\secref}[1]{\hyperref[#1]{section \ref*{#1}}}

\parskip1ex
\parindent0pt
\let\olditemize\itemize
\def\itemize{\olditemize\parskip0pt}

\begin{document}

\title{The \textsf{childdoc} Package}
\hypersetup{pdftitle={The childdoc Package}}
\author{Niklas Beisert\\[2ex]
  Institut f\"ur Theoretische Physik\\
  Eidgen\"ossische Technische Hochschule Z\"urich\\
  Wolfgang-Pauli-Strasse 27, 8093 Z\"urich, Switzerland\\[1ex]
  \href{mailto:nbeisert@itp.phys.ethz.ch}
  {\texttt{nbeisert@itp.phys.ethz.ch}}}
\hypersetup{pdfauthor={Niklas Beisert}}
\hypersetup{pdfsubject={Manual for the LaTeX2e Package childdoc}}
\date{30 December 2018, \textsf{v2.0}}
\maketitle

\begin{abstract}\noindent
\textsf{childdoc} is a \LaTeXe{} package
that enables the direct compilation
of document sections included by |\include|
to individual files.
\end{abstract}

\begingroup
\parskip0ex
\tableofcontents
\endgroup

%%%%%%%%%%%%%%%%%%%%%%%%%%%%%%%%%%%%%%%%%%%%%%%%%%%%%%%%%%%%%%%%%%%%%%%%%%%%%%%%
%%%%%%%%%%%%%%%%%%%%%%%%%%%%%%%%%%%%%%%%%%%%%%%%%%%%%%%%%%%%%%%%%%%%%%%%%%%%%%%%
\section{Introduction}

\LaTeX{} provides a mechanism to structure a large document (such as a book)
into a main file and several child files (containing the chapters)
using the |\include| command.
This mechanism is beneficial for documents
which span hundreds of pages in order to
make the source file(s) more manageable.
Moreover, compilation can be restricted to
selected child files by means of the |\includeonly| command.
The latter feature can be used to reduce the compilation time while editing
(this was significantly more useful in the earlier days of \LaTeX{})
or to generate a smaller document which is easier to navigate.
Another application of |\includeonly| is to generate
documents consisting of selected parts of the complete document.

However, there are a few drawbacks of the plain |\include| mechanism:
\begin{itemize}
\item
The child files cannot be compiled on their own,
they can only be compiled via the main file.
A naive editing environment
(such as a text editor with an option
to have the current file processed by \LaTeX)
may require one to switch to the main file before compiling;
attempting to compile the child file produces errors.
\item
The main file must be modified (each time)
to adjust the |\includeonly| command
to the present needs. This easily leaves the main file in a messy state.
\item
The generated document will always carry the filename
of the main document. This is inconvenient if
several child files are to be compiled and
to be kept for distribution.
\end{itemize}

The present package provides a simple interface
to make child files individually compilable by \LaTeX{}.
Compiling a child file then has the same effect as compiling
the main file with an |\includeonly| command
to select the appropriate child.
Moreover the generated document will carry the name of the child
rather than the main file.
This resolves all three above issues.

This feature is meant to make the editing of books,
thesis documents and lecture notes somewhat more convenient.
However, the package can also be used efficiently for
composing a series of documents (such as exercise sheets)
which are typically distributed individually.
It then assists the author in generating the individual documents
(potentially in different versions)
as well as a document containing the collected series.
Another application is in developing style files
or other kinds of included material
where compilation of the style file could redirect
to a sample or test file.

%%%%%%%%%%%%%%%%%%%%%%%%%%%%%%%%%%%%%%%%%%%%%%%%%%%%%%%%%%%%%%%%%%%%%%%%%%%%%%%%
%%%%%%%%%%%%%%%%%%%%%%%%%%%%%%%%%%%%%%%%%%%%%%%%%%%%%%%%%%%%%%%%%%%%%%%%%%%%%%%%
\section{Usage}

First of all, the package \textsf{childdoc} is \emph{not} a standard
\LaTeXe{} |.sty| style file! Therefore it needs to be invoked in
a non-standard way.

%%%%%%%%%%%%%%%%%%%%%%%%%%%%%%%%%%%%%%%%%%%%%%%%%%%%%%%%%%%%%%%%%%%%%%%%%%%%%%%%
\subsection{Included Files}
\label{sec:include}

%%%%%%%%%%%%%%%%%%%%%%%%%%%%%%%%%%%%%%%%
\DescribeMacro{\childdocmain}
To use the package, add the commands
\begin{center}
\begin{tabular}{l}
|\input{childdoc.def}|\\
|\childdocmain{}|\\
\end{tabular}
\end{center}
at the very top of the main \LaTeX{} file,
in particular \emph{before} the |\documentclass| statement!
The argument of |\childdocmain| should be left empty
(but it must be present).

%%%%%%%%%%%%%%%%%%%%%%%%%%%%%%%%%%%%%%%%
\DescribeMacro{\childdocof}
Furthermore, add the commands
\begin{center}
\begin{tabular}{l}
|\input{childdoc.def}|\\
|\childdocof{|\textit{main}|}|\\
\end{tabular}
\end{center}
at the top of every child file \textit{child}
which is included by |\include{|\textit{child}|}|
from within the main file
(or at least for those files to be compiled individually).
The argument \textit{main} must be the filename of the main file.

There are a couple of
considerations in setting up the main and child documents:

%%%%%%%%%%%%%%%%%%%%%%%%%%%%%%%%%%%%%%%%
\paragraph{Restrictions.}

Please note the following restrictions:
\begin{itemize}
\item
|\childdocmain| must be called with one argument \textit{main}
to ensure compatibility with earlier version of the package.
It must either be empty (|\childdocmain{}|)
or precisely match the filename of the main file in which it is specified.
See \secref{sec:detection} for further information.
\item
The filename \textit{main} must be specified without the |.tex| extension.
\item
The filename \textit{main} is case sensitive
(even in case-insensitive file systems)
due to internal string comparison.
\item
The argument \textit{main} should be fully expanded, it cannot be a macro.
\item
Subdirectories and special characters should be avoided in filenames.
\item
The command |\childdocmain{|\textit{main}|}| must be followed by a whitespace.
It should not be followed immediately by another command
or by a comment mark `|%|'.
This is because the \TeX{} parser reads the token immediately following
the argument of |\childdocmain| and puts it
at the beginning of every child section;
however, a white\-space is ignored.
\end{itemize}

%%%%%%%%%%%%%%%%%%%%%%%%%%%%%%%%%%%%%%%%
\paragraph{Content of Main File.}

It is advisable to place all content in the child files included by |\include|.
Any output contained in the main file will appear in all child documents
unless suppressed manually;
it cannot be suppressed automatically by the |\includeonly| directive
and thus should normally be avoided.
A method to include some content in the main file
by means of conditional processing is described in \secref{sec:conditional}.

%%%%%%%%%%%%%%%%%%%%%%%%%%%%%%%%%%%%%%%%
\paragraph{Page Numbering.}

When only a part of the document is compiled,
the appropriate numbering of pages
(as well as other status parameters)
is determined from the |.aux| files.
The latter contain information from previous passes.
However this information needs to propagate through
all intermediate child documents.
Therefore the page numbering in child documents may well
be inconsistent until the complete document is compiled at least once.

A useful (if unconventional) way to always ensure a consistent
page numbering is to restart the numbering in each child document
and denote the pages by `\textit{child}|.|\textit{page}'
where \textit{child} represents the chapter/section number of the child file.
This can be achieved by the command
|\numberwithin{page}{|\textit{child}|}|
of the \textsf{amsmath} package
where \textit{child} can be |chapter| or |section|
depending on the chosen structuring.
Alternatively, one can modify the macro |\thepage| appropriately
and reset the counter |page| at the start of each child file.

%%%%%%%%%%%%%%%%%%%%%%%%%%%%%%%%%%%%%%%%%%%%%%%%%%%%%%%%%%%%%%%%%%%%%%%%%%%%%%%%
\subsection{Conditional Processing}
\label{sec:conditional}

The package provides a mechanism to compile different versions
of a document. To customise the versions further some conditional processing
can come in handy to distinguish which version is being compiled.
The package provides two macros to describe the compilation context:

%%%%%%%%%%%%%%%%%%%%%%%%%%%%%%%%%%%%%%%%
\DescribeMacro{\ifchilddoc}
The conditional |\ifchilddoc| distinguishes between the compilation of
child documents and the main document:
%
\begin{center}
|\ifchilddoc |\textit{child-code}| |[|\||else |\textit{main-code}]| \||fi|
\end{center}

%%%%%%%%%%%%%%%%%%%%%%%%%%%%%%%%%%%%%%%%
\DescribeMacro{\childdocname}
\DescribeMacro{\childdocjob}
The macro |\childdocname| contains the filename (without extension)
of the main or child file being processed.
Note that |\childdocjob| will always contain the name of the main file.

%%%%%%%%%%%%%%%%%%%%%%%%%%%%%%%%%%%%%%%%
\paragraph{Title Page.}

Conditional processing can be used to include a title or banner page
in the main document when proper precautions are taken.
Importantly, the code in the main file should ensure that the page counter
(as well as other status parameters which are stored in the |.aux| files)
takes the same value after the conditional processing.
Otherwise the page numbers may take divergent values
depending on which part is compiled.

For example, a title page could be declared by:
%
\begin{center}
\begin{tabular}{l}
|\ifchilddoc\||else|\\
|\addtocounter{page}{-1}|\\
\textit{code for title page}\\
|\newpage|\\
|\||fi|
\end{tabular}
\end{center}
%
A banner page for the child documents can be generated by:
%
\begin{center}
\begin{tabular}{l}
|\ifchilddoc|\\
|\addtocounter{page}{-1}|\\
\textit{code for banner page}\\
|\newpage|\\
|\||fi|
\end{tabular}
\end{center}
%
Here one could write a message such as:
\begin{center}
|This is the part \childdocname{} of \childdocjob{}.|
\end{center}

%%%%%%%%%%%%%%%%%%%%%%%%%%%%%%%%%%%%%%%%%%%%%%%%%%%%%%%%%%%%%%%%%%%%%%%%%%%%%%%%
\subsection{Flags}
\label{sec:flags}

The package makes it easy to generate different versions
of the main or child documents.
To this end compilation flags can be defined
and assigned different default values.
They will be particularly useful in conjunction
with the forwarding mechanism described in \secref{sec:forward}.

For example, it may be useful to have a flag |\version|
which can be set to |draft| or |final|.
The document source will contain some conditional code
depending on the value of |\version|.
Suppose further, the flag should default to |final| for the main file
and to |draft| for child files
which is a natural assignment for editing the document.
This is achieved by placing the following code
in the preamble of the main document
(below the |\childdocmain| directive):
%
\begin{center}
\begin{tabular}{l}
|\ifchilddoc|\\
|\providecommand{\version}{draft}|\\
|\||else|\\
|\providecommand{\version}{final}|\\
|\||fi|
\end{tabular}
\end{center}
%
The definition by |\providecommand| makes sure
that previous definitions are not overwritten.
Further statements |\providecommand{\version}{...}|
can thus be added before the above code to override it.

For the main file, one might add a line
(between |\childdocmain| and the above block)
%
\begin{center}
|%\ifchilddoc\||else\providecommand{\version}{draft}\||fi|
\end{center}
%
which can be uncommented to produce a draft version.
Likewise one can add a line to the very top of a child file
(above the |\childdocof{|\textit{main}|}| directive)
%
\begin{center}
|%\providecommand{\version}{final}|
\end{center}
%
which can be uncommented to produce the final version of this child document.

%%%%%%%%%%%%%%%%%%%%%%%%%%%%%%%%%%%%%%%%%%%%%%%%%%%%%%%%%%%%%%%%%%%%%%%%%%%%%%%%
\subsection{Forwarding}
\label{sec:forward}

Different versions of the main or child documents
using compilation flags as described in \secref{sec:flags}
can be (permanently) stored in different files
for convenient compilation, viewing and distribution.
To this end, the package defines a command
to pass on compilation to a different file:

%%%%%%%%%%%%%%%%%%%%%%%%%%%%%%%%%%%%%%%%
\DescribeMacro{\childdocforward}
The command |\childdocforward| redirects processing to
another source file:
%
\begin{center}
\begin{tabular}{l}
|\input{childdoc.def}|\\
|\childdocforward[|\textit{main}|]{|\textit{dest}|}|\\
\end{tabular}
\end{center}
%
The argument \textit{dest} is the destination file
(without extension).
It should be the main file or one of the child files.
Note that further \textsf{childdoc} directives
such as |\childdocof| and |\childdocforward|
in the indicated file will be processed in this form.
The optional argument \textit{main}
passes on directly to the main file \textit{main}
while pretending to compile the child \textit{dest}.
This form behaves as if \textit{dest}
issues |\childdocof{|\textit{main}|}| right away,
and no further \textsf{childdoc} directives will be processed.

%%%%%%%%%%%%%%%%%%%%%%%%%%%%%%%%%%%%%%%%
\DescribeMacro{\...prefix}
In the alternative form |\childdocforwardprefix|,
%
\begin{center}
\begin{tabular}{l}
|\input{childdoc.def}|\\
|\childdocforwardprefix[|\textit{main}|]{|\textit{prefix}|}{|\textit{dest}|}|
\end{tabular}
\end{center}
%
the destination file is determined by a pattern
depending on the current file:
To make this work, the current file must be called
`{\textit{prefix}\hspace{0.2em}\textit{suffix}}'
with \textit{prefix} matching precisely the argument.
Processing is then passed on to the file
`{\textit{dest}\hspace{0.2em}\textit{suffix}}'.
Surely, the same effect is achieved by
directly specifying the
argument `{\textit{dest}\hspace{0.2em}\textit{suffix}}'
in the first form.
However, that requires to set up a different file
for each child. With the alternative form of the command
all these files can have exactly the same content
which simplifies setting them up and maintaining them.

For example, the following file |draft.tex|
with a compilation flag |\version| as described in \secref{sec:flags}
compiles the main document as a draft:
%
\begin{center}
\begin{tabular}{l}
|\def\version{draft}|\\
|\input{childdoc.def}|\\
|\childdocforward{|\textit{main}|}|
\end{tabular}
\end{center}
%
Likewise, the following files |final|\textit{nn}|.tex|
compile the final version of the child document
|child|\textit{nn}|.tex|:
%
\begin{center}
\begin{tabular}{l}
|\def\version{final}|\\
|\input{childdoc.def}|\\
|\childdocforwardprefix{final}{child}|
\end{tabular}
\end{center}
%

Note that when several versions of a main file and/or of each child file
are to be generated, it may be convenient to set up a |Makefile| or
shell script to automatise the process.

%%%%%%%%%%%%%%%%%%%%%%%%%%%%%%%%%%%%%%%%%%%%%%%%%%%%%%%%%%%%%%%%%%%%%%%%%%%%%%%%
\subsection{Command Line Processing}
\label{sec:commandline}

The effect of redirection files can also be achieved by invoking
the \LaTeX{} compiler with a more elaborate command line.
Most conveniently this should be done as part
of a shell script or a |Makefile|.

When using \textsf{childdoc} in the main file, the following
command lines effectively perform a redirection
(note that depending on the shell being used,
backslashes may have to be doubled: `|\|' $\to$ `|\\|'):
%
\begin{center}
|... -jobname "|\textit{target}|" |\\|"|[\textit{flags}]%
|\input{childdoc.def}\childdocforward[|\textit{main}|]{|\textit{dest}|}"|
\end{center}
%
Here \textit{target} is the name of the output file,
\textit{main} is the name of the main file
and \textit{dest} is the name of the main or child file to be processed
(all filenames without extensions).
The optional argument \textit{main} can be omitted
if \textit{main} matches \textit{dest}.
Optionally, compilation \textit{flags} can be defined via |\def| commands.
This command line makes the \TeX{} engine believe
it is compiling the file \textit{target}
whose content is specified as the latter parameter.
The provided code then forwards the processing to
\textit{main} or \textit{dest} as described in \secref{sec:forward}.

%%%%%%%%%%%%%%%%%%%%%%%%%%%%%%%%%%%%%%%%%%%%%%%%%%%%%%%%%%%%%%%%%%%%%%%%%%%%%%%%
\subsection{Include by Input}
\label{sec:input}

Including child documents by |\include| has some restrictions by design.
Most notably, the content of a child document always occupies
its own set of pages; pages cannot be shared between child documents.
Usually, this behaviour makes perfect sense
because each child document contain an essential part of the document.
However, in some situations it may be desirable to compose
a document from a collection of parts
without having mandatory page breaks between then.
For this case, the package
provides a mechanism to include parts
by |\input| which can also be processed individually.
However, by construction this mechanism
requires manual handling of the content to be output.

%%%%%%%%%%%%%%%%%%%%%%%%%%%%%%%%%%%%%%%%
\DescribeMacro{\ifchilddocmanual}
The main file should be prepared as usual, see \secref{sec:include}.
However, the document body must make a distinction
between processing of an individual part and of the main document, e.g.:
%
\begin{center}
\begin{tabular}{l}
|\ifchilddocmanual|\\
|\input{\childdocname}|\\
|\||else|\\
\textit{document body with }|\input{|\textit{part}|}|\\
|\||fi|
\end{tabular}
\end{center}
%
The conditional |\ifchilddocmanual| is true whenever
a part to be included by |\input| is being compiled,
and the name of the part is stored in |\childdocname|.

%%%%%%%%%%%%%%%%%%%%%%%%%%%%%%%%%%%%%%%%
\DescribeMacro{\childdocby}
Each part to be included by |\input| should start with:
%
\begin{center}
\begin{tabular}{l}
|\input{childdoc.def}|\\
|\childdocby{|\textit{main}|}|\\
\end{tabular}
\end{center}
%
The directive |\childdocby| is similar to |\childdocof|
described in \secref{sec:include},
but the subsequent selection of content must be done manually.
To that end, both |\ifchilddoc| and |\ifchilddocmanual|
will be true upon processing of a part,
and the name of the part is stored in |\childdocname|.
Note that |\jobname| will be set to the filename of the current part
so that each part receives an individual |.aux| file
that does not interfere with the |.aux| file(s) of the main document.
This behaviour can be altered by the alternative form
|\childdocby[*]{|\textit{main}|}| (with a non-empty optional argument)
which uses the |.aux| file of the main document
by setting |\jobname| to \textit{main}.

%%%%%%%%%%%%%%%%%%%%%%%%%%%%%%%%%%%%%%%%%%%%%%%%%%%%%%%%%%%%%%%%%%%%%%%%%%%%%%%%
\subsection{Driver Development}
\label{sec:driver}

The \textsf{childdoc} mechanism can also be use for the development
of definition files such as \LaTeX{} styles or classes.
This case differs from the above setup with multiple parts
included by |\include| in that no |\includeonly| should be invoked.
This can be achieved by starting the include file
(before |\ProvidesPackage|) with:
%
\begin{center}
\begin{tabular}{l}
|\input{childdoc.def}|\\
|\childdocforward{|\textit{main}|}|\\
\end{tabular}
\end{center}
%
or alternatively with:
%
\begin{center}
\begin{tabular}{l}
|\input{childdoc.def}|\\
|\childdocby{|\textit{main}|}|\\
\end{tabular}
\end{center}
%
Both forms have slightly different effects as described above.
The main file is prepared as usual, see \secref{sec:include}.

%%%%%%%%%%%%%%%%%%%%%%%%%%%%%%%%%%%%%%%%%%%%%%%%%%%%%%%%%%%%%%%%%%%%%%%%%%%%%%%%
\subsection{Legacy Detection}
\label{sec:detection}

The directive |\childdocmain| in the main file can detect
whether the complete document or merely a child is to be compiled
even without using the directive |\childdocof|.
This method is deprecated because it is less robust
and there is no compelling reason to use it;
it is merely provided for backward compatibility
and it may be removed in future versions.

If the detection mechanism is to be used,
it is mandatory to correctly specify
the filename of the main file as the argument of |\childdocmain|:
%
\begin{center}
\begin{tabular}{l}
|\input{childdoc.def}|\\
|\childdocmain{|\textit{main}|}|\\
\end{tabular}
\end{center}
%
If |\jobname| does not match the argument \textit{main} of |\childdocmain|,
it is assumed that |\jobname| points to the child file to be compiled.
When using |\childdocmain| with the main file specified as argument,
it suffices to start a child file
with just |\input{|\textit{main}|}|
without loading of the package and using |\childdocof|.
If instead all processing is done
with the appropriate \textsf{childdoc} directives,
the argument of \textit{main} of |\childdocmain| can be empty.

An alternative version of the command line processing described
in \secref{sec:commandline} using the detection mechanism reads:
%
\begin{center}
|... -jobname "|\textit{target}|" "|[\textit{flags}]%
[|\def\jobname{|\textit{dest}|}|]|\input{|\textit{main}|}"|
\end{center}

%%%%%%%%%%%%%%%%%%%%%%%%%%%%%%%%%%%%%%%%%%%%%%%%%%%%%%%%%%%%%%%%%%%%%%%%%%%%%%%%
\subsection{Manual Code}
\label{sec:manual}

In case one cannot be certain whether the definitions file |childdoc.def|
is installed on the target \TeX{} distribution
and one prefers not to ship it,
it is conceivable to paste a few relevant commands into the sources.

To that end, drop all statements |\input{childdoc.def}|
and perform the replacements as outlined below.
Instead of |\childdocmain{|\textit{main}|}| add the following code
to the top of the main file:
%
\begin{center}
\begin{tabular}{l}
|\||ifdefined\childdocname\endinput\||fi\newif\ifchilddoc|\\
|\edef\childdocname{\scantokens\expandafter{\jobname\noexpand}}|\\
|\def\childdocmain{|\textit{main}|}\||ifx\childdocmain\childdocname\||else|\\
|\childdoctrue\includeonly{\childdocname}\let\jobname\childdocmain\||fi|\\
\end{tabular}
\end{center}
%
Instead of |\childdocof{|\textit{main}|}| just include the main file
at the top of each child file:
%
\begin{center}
|\input{|\textit{main}|}|
\end{center}
%
A simple redirection |\childdocforward{|\textit{dest}|}| is achieved by:
%
\begin{center}
|\def\jobname{|\textit{dest}|}\input{\jobname}|
\end{center}
%
The redirection with prefix
|\childdocforwardprefix[|\textit{prefix}|]{|\textit{dest}|}|
is accomplished by:
%
\begin{center}
\begin{tabular}{l}
|{\edef\jobname{\scantokens\expandafter{\jobname\noexpand}}|\\
|\def\redirectjob |\textit{prefix}|#1~~~{\gdef\jobname{|\textit{dest}|#1}}|\\
|\expandafter\redirectjob\jobname~~~}\input{\jobname}|
\end{tabular}
\end{center}

In an alternative approach,
child documents can be compiled by a specific command line
without additional code or specific definitions:
%
\begin{center}
|... -jobname "|\textit{target}|" "|[\textit{flags}]%
|\includeonly{|\textit{dest}|}\input{|\textit{main}|}"|
\end{center}
%

%%%%%%%%%%%%%%%%%%%%%%%%%%%%%%%%%%%%%%%%%%%%%%%%%%%%%%%%%%%%%%%%%%%%%%%%%%%%%%%%
%%%%%%%%%%%%%%%%%%%%%%%%%%%%%%%%%%%%%%%%%%%%%%%%%%%%%%%%%%%%%%%%%%%%%%%%%%%%%%%%
\section{Information}

%%%%%%%%%%%%%%%%%%%%%%%%%%%%%%%%%%%%%%%%%%%%%%%%%%%%%%%%%%%%%%%%%%%%%%%%%%%%%%%%
\subsection{Copyright}

Copyright \copyright{} 2017--2018 Niklas Beisert

This work may be distributed and/or modified under the
conditions of the \LaTeX{} Project Public License, either version 1.3
of this license or (at your option) any later version.
The latest version of this license is in
  \url{http://www.latex-project.org/lppl.txt}
and version 1.3 or later is part of all distributions of \LaTeX{}
version 2005/12/01 or later.

This work has the LPPL maintenance status `maintained'.

The Current Maintainer of this work is Niklas Beisert.

This work consists of the files |README.txt|, |childdoc.ins| and |childdoc.dtx|
as well as the derived files |childdoc.def|, |cdocsamp.tex|
with |cdocsch1.tex|, |cdocsch2.tex|, |cdocspt3.tex|, |cdocspt4.tex|,
|cdocsdrf.tex|, |cdocsfn1.tex|, |cdocsfn2.tex|
as well as |childdoc.pdf|.

%%%%%%%%%%%%%%%%%%%%%%%%%%%%%%%%%%%%%%%%%%%%%%%%%%%%%%%%%%%%%%%%%%%%%%%%%%%%%%%%
\subsection{Files and Installation}

The package consists of the files:
%
\begin{center}
\begin{tabular}{ll}
    |README.txt|   & readme file \\
    |childdoc.ins| & installation file \\
    |childdoc.dtx| & source file \\
    |childdoc.def| & definition file \\
    |cdocsamp.tex| & sample main file \\
    |cdocsch1.tex| & sample include file \\
    |cdocsch2.tex| & sample include file \\
    |cdocspt3.tex| & sample part file \\
    |cdocspt4.tex| & sample part file \\
    |cdocsdrf.tex| & sample redirection file \\
    |cdocsfn1.tex| & sample redirection file \\
    |cdocsfn2.tex| & sample redirection file \\
    |childdoc.pdf| & manual
\end{tabular}
\end{center}
%
The distribution consists of the files
|README.txt|, |childdoc.ins| and |childdoc.dtx|.
%
\begin{itemize}
\item
Run (pdf)\LaTeX{} on |childdoc.dtx|
to compile the manual |childdoc.pdf| (this file).
\item
Run \LaTeX{} on |childdoc.ins| to create the definitions file |childdoc.def|
and the sample |cdocsamp.tex| with include files
|cdocsch1.tex|, |cdocsch2.tex|, |cdocspt3.tex|, |cdocspt4.tex|,
|cdocsdrf.tex|, |cdocsfn1.tex|, |cdocsfn2.tex|.
Then copy the file |childdoc.def| to an appropriate directory of your \LaTeX{}
distribution, e.g.\ \textit{texmf-root}|/tex/latex/childdoc|.
\end{itemize}

%%%%%%%%%%%%%%%%%%%%%%%%%%%%%%%%%%%%%%%%%%%%%%%%%%%%%%%%%%%%%%%%%%%%%%%%%%%%%%%%
\subsection{Related CTAN Packages}

There are several other packages which offer a similar functionality:
%
\begin{itemize}
\item
The packages
\href{http://ctan.org/pkg/docmute}{\textsf{docmute}},
\href{http://ctan.org/pkg/includex}{\textsf{includex}} and
\href{http://ctan.org/pkg/standalone}{\textsf{standalone}}
provide commands to include only the document body of
a child file thus allowing both files to be compiled individually.
\item
The packages \href{http://ctan.org/pkg/subdocs}{\textsf{subdocs}}
and \href{http://ctan.org/pkg/subfiles}{\textsf{subfiles}}
provide structures in which the main and child documents can be
encapsulated and allowing them to be compiled individually.
The inclusion mechanism is different from the conventional |\include|.
\item
The package \href{http://ctan.org/pkg/combine}{\textsf{combine}}
is an elaborate solution to combine several documents into one.
\end{itemize}
%
See also the CTAN topic \href{http://ctan.org/topic/subdocs}{\textsf{subdocs}}
for further related packages.
The present package differs from the above solutions in that
a document structure constructed with the conventional |\include| mechanism
just needs two extra commands at the top of every file
such that all constituent files can be compiled individually.

%%%%%%%%%%%%%%%%%%%%%%%%%%%%%%%%%%%%%%%%%%%%%%%%%%%%%%%%%%%%%%%%%%%%%%%%%%%%%%%%
%\subsection{Feature Suggestions}
%
%The following is a list of features which may be useful for future
%versions of this package:
%%
%\begin{itemize}
%\item
%\ldots
%\end{itemize}

%%%%%%%%%%%%%%%%%%%%%%%%%%%%%%%%%%%%%%%%%%%%%%%%%%%%%%%%%%%%%%%%%%%%%%%%%%%%%%%%
\subsection{Revision History}

%%%%%%%%%%%%%%%%%%%%%%%%%%%%%%%%%%%%%%%%
\paragraph{v2.0:} 2018/12/30

\begin{itemize}
\item
immediate forward processing
\item
added |\childdocby| mechanism
\item
manual restructured
\end{itemize}

%%%%%%%%%%%%%%%%%%%%%%%%%%%%%%%%%%%%%%%%
\paragraph{v1.6:} 2018/01/17

\begin{itemize}
\item
application for development of include files
\item
corrections to manual
\end{itemize}

%%%%%%%%%%%%%%%%%%%%%%%%%%%%%%%%%%%%%%%%
\paragraph{v1.5:} 2017/05/21

\begin{itemize}
\item
more complete structuring introduced
\item
|\childdocof| introduced
\item
|\childdoc| renamed to |\childdocmain|
\item
|\childredirect| renamed to |\childdocforward| and |\childdocforwardprefix|
and functionality expanded
\end{itemize}

%%%%%%%%%%%%%%%%%%%%%%%%%%%%%%%%%%%%%%%%
\paragraph{v1.0:} 2017/04/27

\begin{itemize}
\item
manual and install package
\item
first version published on CTAN
\end{itemize}

%%%%%%%%%%%%%%%%%%%%%%%%%%%%%%%%%%%%%%%%
\paragraph{v0.6:} 2017/04/26

\begin{itemize}
\item
redirection mechanism added
\end{itemize}

%%%%%%%%%%%%%%%%%%%%%%%%%%%%%%%%%%%%%%%%
\paragraph{v0.5:} 2017/04/26

\begin{itemize}
\item
functionality in definition file
\end{itemize}


%%%%%%%%%%%%%%%%%%%%%%%%%%%%%%%%%%%%%%%%%%%%%%%%%%%%%%%%%%%%%%%%%%%%%%%%%%%%%%%%
%%%%%%%%%%%%%%%%%%%%%%%%%%%%%%%%%%%%%%%%%%%%%%%%%%%%%%%%%%%%%%%%%%%%%%%%%%%%%%%%
%%%%%%%%%%%%%%%%%%%%%%%%%%%%%%%%%%%%%%%%%%%%%%%%%%%%%%%%%%%%%%%%%%%%%%%%%%%%%%%%
\appendix

\settowidth\MacroIndent{\rmfamily\scriptsize 000\ }

 \DocInput{childdoc.dtx}

\end{document}
%</driver>
% \fi
%
% %%%%%%%%%%%%%%%%%%%%%%%%%%%%%%%%%%%%%%%%%%%%%%%%%%%%%%%%%%%%%%%%%%%%%%%%%%%%%%
% %%%%%%%%%%%%%%%%%%%%%%%%%%%%%%%%%%%%%%%%%%%%%%%%%%%%%%%%%%%%%%%%%%%%%%%%%%%%%%
% \section{Sample}
%\iffalse
%<*samplemain>
%\fi
%
% The following presents a sample document
% with two chapters, two parts, a title page,
% a compile flag as well as three forwarding files to set the flag.
% It consists of eight |.tex| files:
% \begin{center}
% \begin{tabular}{ll}
% |cdocsamp.tex|&main file\\
% |cdocsch1.tex|&include file for chapter 1\\
% |cdocsch2.tex|&include file for chapter 2\\
% |cdocspt3.tex|&include file for part 3\\
% |cdocspt4.tex|&include file for part 4\\
% |cdocsdrf.tex|&forwarding file for main file in draft mode\\
% |cdocsfi1.tex|&forwarding file for final version of chapter 1\\
% |cdocsfi2.tex|&forwarding file for final version of chapter 2\\
% \end{tabular}
% \end{center}
% Each of the eight files can be compiled directly by the \LaTeX{} compiler.
%
% %%%%%%%%%%%%%%%%%%%%%%%%%%%%%%%%%%%%%%
% \paragraph{Main File.}
%
% The main file is called |cdocsamp.tex|.
%
% Load the \textsf{childdoc} definitions and
% declare the filename for the main document:
%    \begin{macrocode}
\input{childdoc.def}
\childdocmain{}
%    \end{macrocode}

% Optional override for |\version| flag:
%    \begin{macrocode}
%%\ifchilddoc\else\providecommand{\version}{draft}\fi
%    \end{macrocode}

% Define the default values for the |\version| flag
% (|final| for the main file and |draft| for childs):
%    \begin{macrocode}
\ifchilddoc
\providecommand{\version}{draft}
\else
\providecommand{\version}{final}
\fi
%    \end{macrocode}

% Load the standard document class:
%    \begin{macrocode}
\documentclass[12pt]{article}
%    \end{macrocode}

% Start the document body:
%    \begin{macrocode}
\begin{document}
%    \end{macrocode}

% Declare a title page.
% Print title, part of document being processed and version flag:
%    \begin{macrocode}
\addtocounter{page}{-1}
\begin{center}
{\LARGE\bfseries{}childdoc example\par}
\vspace{1cm}
\ifchilddoc
\ifchilddocmanual part\else chapter\fi:
`\childdocname' of `\childdocjob'\par
\else
main document: `\childdocjob'\par
\fi
version: \version\par
\end{center}
\newpage
%    \end{macrocode}

% Manually include selected file,
% otherwise process as usual:
%    \begin{macrocode}
\ifchilddocmanual
\section*{part `\childdocname'}
\input{\childdocname}
\else
%    \end{macrocode}

% Include the two chapters:
%    \begin{macrocode}
\include{cdocsch1}
\include{cdocsch2}
%    \end{macrocode}

% Include the two parts unless only chapters should be displayed:
%    \begin{macrocode}
\ifchilddoc\else
\section{part three}
\input{cdocspt3}
\section{part four}
\input{cdocspt4}
\fi
%    \end{macrocode}

% Process as usual until here:
%    \begin{macrocode}
\fi
%    \end{macrocode}

% End of document body:
%    \begin{macrocode}
\end{document}
%    \end{macrocode}
%\iffalse
%</samplemain>
%\fi
%
% %%%%%%%%%%%%%%%%%%%%%%%%%%%%%%%%%%%%%%
% \paragraph{Chapter Include Files.}
%
% The include files are called |cdocsch1.tex| and |cdocsch2.tex|.
%
%\iffalse
%<*samplechap1|samplechap2>
%\fi

% Optional override for |\version| flag:
%    \begin{macrocode}
%%\providecommand{\version}{final}
%    \end{macrocode}

% Include the main document:
%    \begin{macrocode}
\input{childdoc.def}
\childdocof{cdocsamp}
%    \end{macrocode}

%\iffalse
%</samplechap1|samplechap2>
%\fi
%
%\iffalse
%<*samplechap1>
%\fi
% Some text for chapter 1:
%    \begin{macrocode}
\section{one}
some text in chapter one
%    \end{macrocode}

%\iffalse
%</samplechap1>
%\fi
% Some text for chapter 2:
%\iffalse
%<*samplechap2>
%\fi
%    \begin{macrocode}
\section{two}
more text in chapter two
%    \end{macrocode}

%\iffalse
%</samplechap2>
%\fi
%
% %%%%%%%%%%%%%%%%%%%%%%%%%%%%%%%%%%%%%%
% \paragraph{Part Include Files.}
%
% The include files are called |cdocspt3.tex| and |cdocspt4.tex|.
%
%\iffalse
%<*samplepart3|samplepart4>
%\fi

% Optional override for |\version| flag:
%    \begin{macrocode}
%%\providecommand{\version}{final}
%    \end{macrocode}

% Include the main document:
%    \begin{macrocode}
\input{childdoc.def}
\childdocby{cdocsamp}
%    \end{macrocode}

%\iffalse
%</samplepart3|samplepart4>
%\fi
%
%\iffalse
%<*samplepart3>
%\fi
% Some text for part 3:
%    \begin{macrocode}
some text in part three
%    \end{macrocode}

%\iffalse
%</samplepart3>
%\fi
% Some text for part 4:
%\iffalse
%<*samplepart4>
%\fi
%    \begin{macrocode}
more text in part four
%    \end{macrocode}

%\iffalse
%</samplepart4>
%\fi
%
% %%%%%%%%%%%%%%%%%%%%%%%%%%%%%%%%%%%%%%
% \paragraph{Forwarding for a Complete Draft.}
%
% The following forwarding file |cdocsdrf.tex|
% compiles the main document in draft mode:
%\iffalse
%<*sampledraft>
%\fi
%    \begin{macrocode}
\def\version{draft}
\input{childdoc.def}
\childdocforward{cdocsamp}
%    \end{macrocode}

%\iffalse
%</sampledraft>
%\fi
%
% %%%%%%%%%%%%%%%%%%%%%%%%%%%%%%%%%%%%%%
% \paragraph{Forwarding for Final Version of the Chapters.}
%
% The following forwarding files |cdocsfn1.tex| and |cdocsfn2.tex|
% (with identical content)
% compile the final versions of the child documents
% |cdocsch1.tex| and |cdocsch2.tex|, respectively:
%\iffalse
%<*samplefinal>
%\fi
%    \begin{macrocode}
\def\version{final}
\input{childdoc.def}
\childdocforwardprefix[cdocsamp]{cdocsfn}{cdocsch}
%    \end{macrocode}

%\iffalse
%</samplefinal>
%\fi
%
% %%%%%%%%%%%%%%%%%%%%%%%%%%%%%%%%%%%%%%
% \paragraph{Command Line Processing.}
%
% The following three command lines generate the output files
% |cdocscld|, |cdocscl1| and |cdocscl2|
% which should be identical to
% |cdocsdrf|, |cdocsch1| and |cdocsfn2|, respectively:
% \begin{center}
% \begin{tabular}{l}
% |latex -jobname cdocscld \|\\
% |  "\def\version{draft}\input{childdoc.def}\childdocforward{cdocsamp}"|\\
% |latex -jobname cdocscl1 \|\\
% |  "\input{childdoc.def}\childdocforward[cdocsamp]{cdocsch1}"|\\
% |latex -jobname cdocscl2 \|\\
% |  "\def\version{final}\input{childdoc.def}\childdocforward{cdocsch2}"|
% \end{tabular}
% \end{center}
% Note that the trailing backslash on each first line
% merely continues the input to the second line
% (for convenient cut ant paste).
% Furthermore, the command |latex| can be replaced by any
% of its alternative versions such as |pdflatex|.
%
% %%%%%%%%%%%%%%%%%%%%%%%%%%%%%%%%%%%%%%%%%%%%%%%%%%%%%%%%%%%%%%%%%%%%%%%%%%%%%%
% %%%%%%%%%%%%%%%%%%%%%%%%%%%%%%%%%%%%%%%%%%%%%%%%%%%%%%%%%%%%%%%%%%%%%%%%%%%%%%
% \section{Implementation}
%\iffalse
%<*package>
%\fi
%
% This section describes the definitions file |childdoc.def|.

% The definitions cannot be loaded using |\usepackage| or |\RequirePackage|
% which has a mechanism to prevent loading a style file more than once.
% When loading the definitions by means of |\input|
% multiple instances have to be prevented manually:
%\iffalse
%This code needs to be before the `\ProvidesFile' directive
%which is defined at the beginning of this file.
%Therefore it is also placed there and commented out here.
%</package>
%<*discard>
%\fi
%    \begin{macrocode}
\ifdefined\childdocmain\endinput\fi
%    \end{macrocode}
%\iffalse
%</discard>
%<*package>
%\fi
%
% \macro{\ifchilddoc}
% \macro{\ifchilddocmanual}
% The conditional |\ifchilddoc| tells whether a
% child (true) or main (false) document is being compiled.
% The conditional |\ifchilddocmanual| tells whether
% the |\includeonly| mechanism is used (false) or
% the selection of child files must be performed manually (true).
% The definitions initialise to false:
%    \begin{macrocode}
\newif\ifchilddoc
\newif\ifchilddocmanual
%    \end{macrocode}

% \macro{\childdocname}
% \macro{\childdocjob}
% The macro |\childdocname| stores the name of the main document
% to be compiled. The macro |\childdocjob| stores the name of
% the document on which the \LaTeX{} compiler was originally invoked.
% The content of |\jobname| cannot be compared
% to filenames specified in the source due to different catcodes.
% The following code rescans |\jobname|, stores the result
% in |\childdocname| and saves a copy in |\childdocjob|:
%    \begin{macrocode}
\edef\childdocname{\scantokens\expandafter{\jobname\noexpand}}
\let\childdocjob\childdocname
%    \end{macrocode}

% \macro{\childdocdisable}
% The macro |\childdocdisable| prevents the main file
% from being processed more than once.
% At this stage, the main document command |\childdocmain|
% is assumed to be called once again where it should do nothing.
% Any subsequent call to it should prevent
% a secondary processing of the main document
% It overwrites the forwarding commands
% |\childdocof| and |\childdocforward|
% with empty macros to prevent further inclusions of the main document:
%    \begin{macrocode}
\newcommand{\childdocdisable}
{
  \renewcommand{\childdocmain}[1]{\renewcommand{\childdocmain}[1]{\endinput}}
  \renewcommand{\childdocof}[1]{}
  \renewcommand{\childdocby}[2][]{}
  \renewcommand{\childdocforward}[2][]{}
  \renewcommand{\childdocdisable}{}
}
%    \end{macrocode}

% \macro{\childdocmain}
% The macro |\childdocmain| is to be called at the top of the main file
% with nothing or the main filename (without extension) as argument.
% First, it breaks loops.
% If the argument is not empty and does not match |\childdocname|
% (which is set by the first inclusion of |childdoc.def|),
% |\ifchilddoc| is set to true, |\includeonly| is applied to the child file
% and |\jobname| is set to the main file
% (for proper handling of |.aux| files):
%    \begin{macrocode}
\newcommand{\childdocmain}[1]
{
  \childdocdisable\childdocmain{}
  \if?#1?\else
    \begingroup
      \def\childdoctmp{#1}
      \ifx\childdoctmp\childdocname
        \def\childdoctmp{}
      \else
        \def\childdoctmp
        {
          \childdoctrue
          \includeonly{\childdocname}
          \def\childdocjob{#1}
          \def\jobname{#1}
        }
      \fi
      \expandafter
    \endgroup
    \childdoctmp
  \fi
}
%    \end{macrocode}

% \macro{\childdocof}
% The command |\childdocof| redirects
% compilation to the main file |#1|.
%    \begin{macrocode}
\newcommand{\childdocof}[1]
{
  \childdocdisable
  \childdoctrue
  \includeonly{\childdocname}
  \def\jobname{#1}
  \def\childdocjob{#1}
  \input{#1}
}
%    \end{macrocode}

% \macro{\childdocby}
% The command |\childdocby| ....
%    \begin{macrocode}
\newcommand{\childdocby}[2][]
{
  \childdocdisable
  \childdoctrue
  \childdocmanualtrue
  \if?#1?\else
    \def\jobname{#2}
  \fi
  \def\childdocjob{#2}
  \input{#2}
  \endinput
}
%    \end{macrocode}

% \macro{\childdocforward}
% The command |\childdocforward| redirects
% compilation to the main file or
% (if the optional argument is given) a child file.
% Parameters are set as if the main file
% or a child file starting with |\childdocof| was compiled.
% Then compilation is handed over to the main file:
%    \begin{macrocode}
\newcommand{\childdocforward}[2][]
{
  \begingroup
    \if?#1?
      \def\childdoctmp
      {
        \def\childdocname{#2}
        \def\childdocjob{#2}
        \def\jobname{#2}
        \input{#2}
        \endinput
      }
    \else
      \def\childdoctmp
      {
        \childdocdisable
        \def\childdocname{#2}
        \childdoctrue
        \includeonly{#2}
        \def\childdocjob{#1}
        \def\jobname{#1}
        \input{#1}
        \endinput
      }
    \fi
    \expandafter
  \endgroup
  \childdoctmp
}
%    \end{macrocode}

% \macro{\childdocforwardprefix}
% The command |\childdocforwardprefix| redirects
% compilation to the main or a child file by means of a pattern.
% The prefix |#1| in the current filename is replaced by |#2|
% and the suffix of the current filename is kept
% (it is assumed that the filename does not contain the substring `|~~~|'
% which is used as a delimiter).
% Compilation is handed over to the new file by |\childdocforward|:
%    \begin{macrocode}
\newcommand{\childdocforwardprefix}[3][]
{
  \begingroup
    \def\childdocextract #2##1~~~{\def\childdoctmp{\childdocforward[#1]{#3##1}}}
    \expandafter\childdocextract\childdocname~~~
    \expandafter
  \endgroup
  \childdoctmp
}
%    \end{macrocode}

% \macro{\childdoc}
% The deprecated macro |\childdoc| is a legacy version of |\childdocmain|:
%    \begin{macrocode}
\newcommand{\childdoc}{\childdocmain}
%    \end{macrocode}

% \macro{\childdocredirect}
% The deprecated macro |\childdocredirect| is a legacy version
% of |\childdocforward| and |\childdocforwardprefix|:
%    \begin{macrocode}
\newcommand{\childdocredirect}[2][]
{
  \begingroup
    \if?#1?
      \def\childdoctmp{\childdocforward{#2}}
    \else
      \def\childdoctmp{\childdocforwardprefix{#1}{#2}}
    \fi
    \expandafter
  \endgroup
  \childdoctmp
}
%    \end{macrocode}

%\iffalse
%</package>
%\fi
%
\endinput
\childdocforward{cdocsch2}"|
% \end{tabular}
% \end{center}
% Note that the trailing backslash on each first line
% merely continues the input to the second line
% (for convenient cut ant paste).
% Furthermore, the command |latex| can be replaced by any
% of its alternative versions such as |pdflatex|.
%
% %%%%%%%%%%%%%%%%%%%%%%%%%%%%%%%%%%%%%%%%%%%%%%%%%%%%%%%%%%%%%%%%%%%%%%%%%%%%%%
% %%%%%%%%%%%%%%%%%%%%%%%%%%%%%%%%%%%%%%%%%%%%%%%%%%%%%%%%%%%%%%%%%%%%%%%%%%%%%%
% \section{Implementation}
%\iffalse
%<*package>
%\fi
%
% This section describes the definitions file |childdoc.def|.

% The definitions cannot be loaded using |\usepackage| or |\RequirePackage|
% which has a mechanism to prevent loading a style file more than once.
% When loading the definitions by means of |\input|
% multiple instances have to be prevented manually:
%\iffalse
%This code needs to be before the `\ProvidesFile' directive
%which is defined at the beginning of this file.
%Therefore it is also placed there and commented out here.
%</package>
%<*discard>
%\fi
%    \begin{macrocode}
\ifdefined\childdocmain\endinput\fi
%    \end{macrocode}
%\iffalse
%</discard>
%<*package>
%\fi
%
% \macro{\ifchilddoc}
% \macro{\ifchilddocmanual}
% The conditional |\ifchilddoc| tells whether a
% child (true) or main (false) document is being compiled.
% The conditional |\ifchilddocmanual| tells whether
% the |\includeonly| mechanism is used (false) or
% the selection of child files must be performed manually (true).
% The definitions initialise to false:
%    \begin{macrocode}
\newif\ifchilddoc
\newif\ifchilddocmanual
%    \end{macrocode}

% \macro{\childdocname}
% \macro{\childdocjob}
% The macro |\childdocname| stores the name of the main document
% to be compiled. The macro |\childdocjob| stores the name of
% the document on which the \LaTeX{} compiler was originally invoked.
% The content of |\jobname| cannot be compared
% to filenames specified in the source due to different catcodes.
% The following code rescans |\jobname|, stores the result
% in |\childdocname| and saves a copy in |\childdocjob|:
%    \begin{macrocode}
\edef\childdocname{\scantokens\expandafter{\jobname\noexpand}}
\let\childdocjob\childdocname
%    \end{macrocode}

% \macro{\childdocdisable}
% The macro |\childdocdisable| prevents the main file
% from being processed more than once.
% At this stage, the main document command |\childdocmain|
% is assumed to be called once again where it should do nothing.
% Any subsequent call to it should prevent
% a secondary processing of the main document
% It overwrites the forwarding commands
% |\childdocof| and |\childdocforward|
% with empty macros to prevent further inclusions of the main document:
%    \begin{macrocode}
\newcommand{\childdocdisable}
{
  \renewcommand{\childdocmain}[1]{\renewcommand{\childdocmain}[1]{\endinput}}
  \renewcommand{\childdocof}[1]{}
  \renewcommand{\childdocby}[2][]{}
  \renewcommand{\childdocforward}[2][]{}
  \renewcommand{\childdocdisable}{}
}
%    \end{macrocode}

% \macro{\childdocmain}
% The macro |\childdocmain| is to be called at the top of the main file
% with nothing or the main filename (without extension) as argument.
% First, it breaks loops.
% If the argument is not empty and does not match |\childdocname|
% (which is set by the first inclusion of |childdoc.def|),
% |\ifchilddoc| is set to true, |\includeonly| is applied to the child file
% and |\jobname| is set to the main file
% (for proper handling of |.aux| files):
%    \begin{macrocode}
\newcommand{\childdocmain}[1]
{
  \childdocdisable\childdocmain{}
  \if?#1?\else
    \begingroup
      \def\childdoctmp{#1}
      \ifx\childdoctmp\childdocname
        \def\childdoctmp{}
      \else
        \def\childdoctmp
        {
          \childdoctrue
          \includeonly{\childdocname}
          \def\childdocjob{#1}
          \def\jobname{#1}
        }
      \fi
      \expandafter
    \endgroup
    \childdoctmp
  \fi
}
%    \end{macrocode}

% \macro{\childdocof}
% The command |\childdocof| redirects
% compilation to the main file |#1|.
%    \begin{macrocode}
\newcommand{\childdocof}[1]
{
  \childdocdisable
  \childdoctrue
  \includeonly{\childdocname}
  \def\jobname{#1}
  \def\childdocjob{#1}
  \input{#1}
}
%    \end{macrocode}

% \macro{\childdocby}
% The command |\childdocby| ....
%    \begin{macrocode}
\newcommand{\childdocby}[2][]
{
  \childdocdisable
  \childdoctrue
  \childdocmanualtrue
  \if?#1?\else
    \def\jobname{#2}
  \fi
  \def\childdocjob{#2}
  \input{#2}
  \endinput
}
%    \end{macrocode}

% \macro{\childdocforward}
% The command |\childdocforward| redirects
% compilation to the main file or
% (if the optional argument is given) a child file.
% Parameters are set as if the main file
% or a child file starting with |\childdocof| was compiled.
% Then compilation is handed over to the main file:
%    \begin{macrocode}
\newcommand{\childdocforward}[2][]
{
  \begingroup
    \if?#1?
      \def\childdoctmp
      {
        \def\childdocname{#2}
        \def\childdocjob{#2}
        \def\jobname{#2}
        \input{#2}
        \endinput
      }
    \else
      \def\childdoctmp
      {
        \childdocdisable
        \def\childdocname{#2}
        \childdoctrue
        \includeonly{#2}
        \def\childdocjob{#1}
        \def\jobname{#1}
        \input{#1}
        \endinput
      }
    \fi
    \expandafter
  \endgroup
  \childdoctmp
}
%    \end{macrocode}

% \macro{\childdocforwardprefix}
% The command |\childdocforwardprefix| redirects
% compilation to the main or a child file by means of a pattern.
% The prefix |#1| in the current filename is replaced by |#2|
% and the suffix of the current filename is kept
% (it is assumed that the filename does not contain the substring `|~~~|'
% which is used as a delimiter).
% Compilation is handed over to the new file by |\childdocforward|:
%    \begin{macrocode}
\newcommand{\childdocforwardprefix}[3][]
{
  \begingroup
    \def\childdocextract #2##1~~~{\def\childdoctmp{\childdocforward[#1]{#3##1}}}
    \expandafter\childdocextract\childdocname~~~
    \expandafter
  \endgroup
  \childdoctmp
}
%    \end{macrocode}

% \macro{\childdoc}
% The deprecated macro |\childdoc| is a legacy version of |\childdocmain|:
%    \begin{macrocode}
\newcommand{\childdoc}{\childdocmain}
%    \end{macrocode}

% \macro{\childdocredirect}
% The deprecated macro |\childdocredirect| is a legacy version
% of |\childdocforward| and |\childdocforwardprefix|:
%    \begin{macrocode}
\newcommand{\childdocredirect}[2][]
{
  \begingroup
    \if?#1?
      \def\childdoctmp{\childdocforward{#2}}
    \else
      \def\childdoctmp{\childdocforwardprefix{#1}{#2}}
    \fi
    \expandafter
  \endgroup
  \childdoctmp
}
%    \end{macrocode}

%\iffalse
%</package>
%\fi
%
\endinput
\childdocforward[cdocsamp]{cdocsch1}"|\\
% |latex -jobname cdocscl2 \|\\
% |  "\def\version{final}% \iffalse
%
% childdoc.dtx Copyright (C) 2017-2018 Niklas Beisert
%
% This work may be distributed and/or modified under the
% conditions of the LaTeX Project Public License, either version 1.3
% of this license or (at your option) any later version.
% The latest version of this license is in
%   http://www.latex-project.org/lppl.txt
% and version 1.3 or later is part of all distributions of LaTeX
% version 2005/12/01 or later.
%
% This work has the LPPL maintenance status `maintained'.
%
% The Current Maintainer of this work is Niklas Beisert.
%
% This work consists of the files childdoc.dtx and childdoc.ins
% and the derived files childdoc.def and cdocsamp.tex with
% cdocsch1.tex, cdocsch2.tex, cdocsdrf.tex, cdocsfn1.tex, cdocsfn2.tex.
%
%<package>\ifdefined\childdocmain\endinput\fi
%<package>\ProvidesFile{childdoc.def}[2018/12/30 v2.0 child document driver]
%<samplemain>\ProvidesFile{cdocsamp.tex}[2018/12/30 v2.0 sample for childdoc]
%<*driver>
%\ProvidesFile{childdoc.drv}[2018/12/30 v2.0 childdoc reference manual file]
\PassOptionsToClass{10pt,a4paper}{article}
\documentclass{ltxdoc}

\usepackage[margin=35mm]{geometry}
\usepackage{hyperref}
\usepackage{hyperxmp}
\usepackage[usenames]{color}

\hypersetup{colorlinks=true}
\hypersetup{pdfstartview=FitH}
\hypersetup{pdfpagemode=UseNone}
\hypersetup{pdfsource={}}
\hypersetup{pdflang={en-UK}}
\hypersetup{pdfcopyright={Copyright 2017-2018 Niklas Beisert.
  This work may be distributed and/or modified under the
  conditions of the LaTeX Project Public License, either version 1.3
  of this license or (at your option) any later version.}}
\hypersetup{pdflicenseurl={http://www.latex-project.org/lppl.txt}}
\hypersetup{pdfcontactaddress={ETH Zurich, ITP, HIT K,
  Wolfgang-Pauli-Strasse 27}}
\hypersetup{pdfcontactpostcode={8093}}
\hypersetup{pdfcontactcity={Zurich}}
\hypersetup{pdfcontactcountry={Switzerland}}
\hypersetup{pdfcontactemail={nbeisert@itp.phys.ethz.ch}}
\hypersetup{pdfcontacturl={http://people.phys.ethz.ch/\xmptilde nbeisert/}}

\newcommand{\secref}[1]{\hyperref[#1]{section \ref*{#1}}}

\parskip1ex
\parindent0pt
\let\olditemize\itemize
\def\itemize{\olditemize\parskip0pt}

\begin{document}

\title{The \textsf{childdoc} Package}
\hypersetup{pdftitle={The childdoc Package}}
\author{Niklas Beisert\\[2ex]
  Institut f\"ur Theoretische Physik\\
  Eidgen\"ossische Technische Hochschule Z\"urich\\
  Wolfgang-Pauli-Strasse 27, 8093 Z\"urich, Switzerland\\[1ex]
  \href{mailto:nbeisert@itp.phys.ethz.ch}
  {\texttt{nbeisert@itp.phys.ethz.ch}}}
\hypersetup{pdfauthor={Niklas Beisert}}
\hypersetup{pdfsubject={Manual for the LaTeX2e Package childdoc}}
\date{30 December 2018, \textsf{v2.0}}
\maketitle

\begin{abstract}\noindent
\textsf{childdoc} is a \LaTeXe{} package
that enables the direct compilation
of document sections included by |\include|
to individual files.
\end{abstract}

\begingroup
\parskip0ex
\tableofcontents
\endgroup

%%%%%%%%%%%%%%%%%%%%%%%%%%%%%%%%%%%%%%%%%%%%%%%%%%%%%%%%%%%%%%%%%%%%%%%%%%%%%%%%
%%%%%%%%%%%%%%%%%%%%%%%%%%%%%%%%%%%%%%%%%%%%%%%%%%%%%%%%%%%%%%%%%%%%%%%%%%%%%%%%
\section{Introduction}

\LaTeX{} provides a mechanism to structure a large document (such as a book)
into a main file and several child files (containing the chapters)
using the |\include| command.
This mechanism is beneficial for documents
which span hundreds of pages in order to
make the source file(s) more manageable.
Moreover, compilation can be restricted to
selected child files by means of the |\includeonly| command.
The latter feature can be used to reduce the compilation time while editing
(this was significantly more useful in the earlier days of \LaTeX{})
or to generate a smaller document which is easier to navigate.
Another application of |\includeonly| is to generate
documents consisting of selected parts of the complete document.

However, there are a few drawbacks of the plain |\include| mechanism:
\begin{itemize}
\item
The child files cannot be compiled on their own,
they can only be compiled via the main file.
A naive editing environment
(such as a text editor with an option
to have the current file processed by \LaTeX)
may require one to switch to the main file before compiling;
attempting to compile the child file produces errors.
\item
The main file must be modified (each time)
to adjust the |\includeonly| command
to the present needs. This easily leaves the main file in a messy state.
\item
The generated document will always carry the filename
of the main document. This is inconvenient if
several child files are to be compiled and
to be kept for distribution.
\end{itemize}

The present package provides a simple interface
to make child files individually compilable by \LaTeX{}.
Compiling a child file then has the same effect as compiling
the main file with an |\includeonly| command
to select the appropriate child.
Moreover the generated document will carry the name of the child
rather than the main file.
This resolves all three above issues.

This feature is meant to make the editing of books,
thesis documents and lecture notes somewhat more convenient.
However, the package can also be used efficiently for
composing a series of documents (such as exercise sheets)
which are typically distributed individually.
It then assists the author in generating the individual documents
(potentially in different versions)
as well as a document containing the collected series.
Another application is in developing style files
or other kinds of included material
where compilation of the style file could redirect
to a sample or test file.

%%%%%%%%%%%%%%%%%%%%%%%%%%%%%%%%%%%%%%%%%%%%%%%%%%%%%%%%%%%%%%%%%%%%%%%%%%%%%%%%
%%%%%%%%%%%%%%%%%%%%%%%%%%%%%%%%%%%%%%%%%%%%%%%%%%%%%%%%%%%%%%%%%%%%%%%%%%%%%%%%
\section{Usage}

First of all, the package \textsf{childdoc} is \emph{not} a standard
\LaTeXe{} |.sty| style file! Therefore it needs to be invoked in
a non-standard way.

%%%%%%%%%%%%%%%%%%%%%%%%%%%%%%%%%%%%%%%%%%%%%%%%%%%%%%%%%%%%%%%%%%%%%%%%%%%%%%%%
\subsection{Included Files}
\label{sec:include}

%%%%%%%%%%%%%%%%%%%%%%%%%%%%%%%%%%%%%%%%
\DescribeMacro{\childdocmain}
To use the package, add the commands
\begin{center}
\begin{tabular}{l}
|% \iffalse
%
% childdoc.dtx Copyright (C) 2017-2018 Niklas Beisert
%
% This work may be distributed and/or modified under the
% conditions of the LaTeX Project Public License, either version 1.3
% of this license or (at your option) any later version.
% The latest version of this license is in
%   http://www.latex-project.org/lppl.txt
% and version 1.3 or later is part of all distributions of LaTeX
% version 2005/12/01 or later.
%
% This work has the LPPL maintenance status `maintained'.
%
% The Current Maintainer of this work is Niklas Beisert.
%
% This work consists of the files childdoc.dtx and childdoc.ins
% and the derived files childdoc.def and cdocsamp.tex with
% cdocsch1.tex, cdocsch2.tex, cdocsdrf.tex, cdocsfn1.tex, cdocsfn2.tex.
%
%<package>\ifdefined\childdocmain\endinput\fi
%<package>\ProvidesFile{childdoc.def}[2018/12/30 v2.0 child document driver]
%<samplemain>\ProvidesFile{cdocsamp.tex}[2018/12/30 v2.0 sample for childdoc]
%<*driver>
%\ProvidesFile{childdoc.drv}[2018/12/30 v2.0 childdoc reference manual file]
\PassOptionsToClass{10pt,a4paper}{article}
\documentclass{ltxdoc}

\usepackage[margin=35mm]{geometry}
\usepackage{hyperref}
\usepackage{hyperxmp}
\usepackage[usenames]{color}

\hypersetup{colorlinks=true}
\hypersetup{pdfstartview=FitH}
\hypersetup{pdfpagemode=UseNone}
\hypersetup{pdfsource={}}
\hypersetup{pdflang={en-UK}}
\hypersetup{pdfcopyright={Copyright 2017-2018 Niklas Beisert.
  This work may be distributed and/or modified under the
  conditions of the LaTeX Project Public License, either version 1.3
  of this license or (at your option) any later version.}}
\hypersetup{pdflicenseurl={http://www.latex-project.org/lppl.txt}}
\hypersetup{pdfcontactaddress={ETH Zurich, ITP, HIT K,
  Wolfgang-Pauli-Strasse 27}}
\hypersetup{pdfcontactpostcode={8093}}
\hypersetup{pdfcontactcity={Zurich}}
\hypersetup{pdfcontactcountry={Switzerland}}
\hypersetup{pdfcontactemail={nbeisert@itp.phys.ethz.ch}}
\hypersetup{pdfcontacturl={http://people.phys.ethz.ch/\xmptilde nbeisert/}}

\newcommand{\secref}[1]{\hyperref[#1]{section \ref*{#1}}}

\parskip1ex
\parindent0pt
\let\olditemize\itemize
\def\itemize{\olditemize\parskip0pt}

\begin{document}

\title{The \textsf{childdoc} Package}
\hypersetup{pdftitle={The childdoc Package}}
\author{Niklas Beisert\\[2ex]
  Institut f\"ur Theoretische Physik\\
  Eidgen\"ossische Technische Hochschule Z\"urich\\
  Wolfgang-Pauli-Strasse 27, 8093 Z\"urich, Switzerland\\[1ex]
  \href{mailto:nbeisert@itp.phys.ethz.ch}
  {\texttt{nbeisert@itp.phys.ethz.ch}}}
\hypersetup{pdfauthor={Niklas Beisert}}
\hypersetup{pdfsubject={Manual for the LaTeX2e Package childdoc}}
\date{30 December 2018, \textsf{v2.0}}
\maketitle

\begin{abstract}\noindent
\textsf{childdoc} is a \LaTeXe{} package
that enables the direct compilation
of document sections included by |\include|
to individual files.
\end{abstract}

\begingroup
\parskip0ex
\tableofcontents
\endgroup

%%%%%%%%%%%%%%%%%%%%%%%%%%%%%%%%%%%%%%%%%%%%%%%%%%%%%%%%%%%%%%%%%%%%%%%%%%%%%%%%
%%%%%%%%%%%%%%%%%%%%%%%%%%%%%%%%%%%%%%%%%%%%%%%%%%%%%%%%%%%%%%%%%%%%%%%%%%%%%%%%
\section{Introduction}

\LaTeX{} provides a mechanism to structure a large document (such as a book)
into a main file and several child files (containing the chapters)
using the |\include| command.
This mechanism is beneficial for documents
which span hundreds of pages in order to
make the source file(s) more manageable.
Moreover, compilation can be restricted to
selected child files by means of the |\includeonly| command.
The latter feature can be used to reduce the compilation time while editing
(this was significantly more useful in the earlier days of \LaTeX{})
or to generate a smaller document which is easier to navigate.
Another application of |\includeonly| is to generate
documents consisting of selected parts of the complete document.

However, there are a few drawbacks of the plain |\include| mechanism:
\begin{itemize}
\item
The child files cannot be compiled on their own,
they can only be compiled via the main file.
A naive editing environment
(such as a text editor with an option
to have the current file processed by \LaTeX)
may require one to switch to the main file before compiling;
attempting to compile the child file produces errors.
\item
The main file must be modified (each time)
to adjust the |\includeonly| command
to the present needs. This easily leaves the main file in a messy state.
\item
The generated document will always carry the filename
of the main document. This is inconvenient if
several child files are to be compiled and
to be kept for distribution.
\end{itemize}

The present package provides a simple interface
to make child files individually compilable by \LaTeX{}.
Compiling a child file then has the same effect as compiling
the main file with an |\includeonly| command
to select the appropriate child.
Moreover the generated document will carry the name of the child
rather than the main file.
This resolves all three above issues.

This feature is meant to make the editing of books,
thesis documents and lecture notes somewhat more convenient.
However, the package can also be used efficiently for
composing a series of documents (such as exercise sheets)
which are typically distributed individually.
It then assists the author in generating the individual documents
(potentially in different versions)
as well as a document containing the collected series.
Another application is in developing style files
or other kinds of included material
where compilation of the style file could redirect
to a sample or test file.

%%%%%%%%%%%%%%%%%%%%%%%%%%%%%%%%%%%%%%%%%%%%%%%%%%%%%%%%%%%%%%%%%%%%%%%%%%%%%%%%
%%%%%%%%%%%%%%%%%%%%%%%%%%%%%%%%%%%%%%%%%%%%%%%%%%%%%%%%%%%%%%%%%%%%%%%%%%%%%%%%
\section{Usage}

First of all, the package \textsf{childdoc} is \emph{not} a standard
\LaTeXe{} |.sty| style file! Therefore it needs to be invoked in
a non-standard way.

%%%%%%%%%%%%%%%%%%%%%%%%%%%%%%%%%%%%%%%%%%%%%%%%%%%%%%%%%%%%%%%%%%%%%%%%%%%%%%%%
\subsection{Included Files}
\label{sec:include}

%%%%%%%%%%%%%%%%%%%%%%%%%%%%%%%%%%%%%%%%
\DescribeMacro{\childdocmain}
To use the package, add the commands
\begin{center}
\begin{tabular}{l}
|\input{childdoc.def}|\\
|\childdocmain{}|\\
\end{tabular}
\end{center}
at the very top of the main \LaTeX{} file,
in particular \emph{before} the |\documentclass| statement!
The argument of |\childdocmain| should be left empty
(but it must be present).

%%%%%%%%%%%%%%%%%%%%%%%%%%%%%%%%%%%%%%%%
\DescribeMacro{\childdocof}
Furthermore, add the commands
\begin{center}
\begin{tabular}{l}
|\input{childdoc.def}|\\
|\childdocof{|\textit{main}|}|\\
\end{tabular}
\end{center}
at the top of every child file \textit{child}
which is included by |\include{|\textit{child}|}|
from within the main file
(or at least for those files to be compiled individually).
The argument \textit{main} must be the filename of the main file.

There are a couple of
considerations in setting up the main and child documents:

%%%%%%%%%%%%%%%%%%%%%%%%%%%%%%%%%%%%%%%%
\paragraph{Restrictions.}

Please note the following restrictions:
\begin{itemize}
\item
|\childdocmain| must be called with one argument \textit{main}
to ensure compatibility with earlier version of the package.
It must either be empty (|\childdocmain{}|)
or precisely match the filename of the main file in which it is specified.
See \secref{sec:detection} for further information.
\item
The filename \textit{main} must be specified without the |.tex| extension.
\item
The filename \textit{main} is case sensitive
(even in case-insensitive file systems)
due to internal string comparison.
\item
The argument \textit{main} should be fully expanded, it cannot be a macro.
\item
Subdirectories and special characters should be avoided in filenames.
\item
The command |\childdocmain{|\textit{main}|}| must be followed by a whitespace.
It should not be followed immediately by another command
or by a comment mark `|%|'.
This is because the \TeX{} parser reads the token immediately following
the argument of |\childdocmain| and puts it
at the beginning of every child section;
however, a white\-space is ignored.
\end{itemize}

%%%%%%%%%%%%%%%%%%%%%%%%%%%%%%%%%%%%%%%%
\paragraph{Content of Main File.}

It is advisable to place all content in the child files included by |\include|.
Any output contained in the main file will appear in all child documents
unless suppressed manually;
it cannot be suppressed automatically by the |\includeonly| directive
and thus should normally be avoided.
A method to include some content in the main file
by means of conditional processing is described in \secref{sec:conditional}.

%%%%%%%%%%%%%%%%%%%%%%%%%%%%%%%%%%%%%%%%
\paragraph{Page Numbering.}

When only a part of the document is compiled,
the appropriate numbering of pages
(as well as other status parameters)
is determined from the |.aux| files.
The latter contain information from previous passes.
However this information needs to propagate through
all intermediate child documents.
Therefore the page numbering in child documents may well
be inconsistent until the complete document is compiled at least once.

A useful (if unconventional) way to always ensure a consistent
page numbering is to restart the numbering in each child document
and denote the pages by `\textit{child}|.|\textit{page}'
where \textit{child} represents the chapter/section number of the child file.
This can be achieved by the command
|\numberwithin{page}{|\textit{child}|}|
of the \textsf{amsmath} package
where \textit{child} can be |chapter| or |section|
depending on the chosen structuring.
Alternatively, one can modify the macro |\thepage| appropriately
and reset the counter |page| at the start of each child file.

%%%%%%%%%%%%%%%%%%%%%%%%%%%%%%%%%%%%%%%%%%%%%%%%%%%%%%%%%%%%%%%%%%%%%%%%%%%%%%%%
\subsection{Conditional Processing}
\label{sec:conditional}

The package provides a mechanism to compile different versions
of a document. To customise the versions further some conditional processing
can come in handy to distinguish which version is being compiled.
The package provides two macros to describe the compilation context:

%%%%%%%%%%%%%%%%%%%%%%%%%%%%%%%%%%%%%%%%
\DescribeMacro{\ifchilddoc}
The conditional |\ifchilddoc| distinguishes between the compilation of
child documents and the main document:
%
\begin{center}
|\ifchilddoc |\textit{child-code}| |[|\||else |\textit{main-code}]| \||fi|
\end{center}

%%%%%%%%%%%%%%%%%%%%%%%%%%%%%%%%%%%%%%%%
\DescribeMacro{\childdocname}
\DescribeMacro{\childdocjob}
The macro |\childdocname| contains the filename (without extension)
of the main or child file being processed.
Note that |\childdocjob| will always contain the name of the main file.

%%%%%%%%%%%%%%%%%%%%%%%%%%%%%%%%%%%%%%%%
\paragraph{Title Page.}

Conditional processing can be used to include a title or banner page
in the main document when proper precautions are taken.
Importantly, the code in the main file should ensure that the page counter
(as well as other status parameters which are stored in the |.aux| files)
takes the same value after the conditional processing.
Otherwise the page numbers may take divergent values
depending on which part is compiled.

For example, a title page could be declared by:
%
\begin{center}
\begin{tabular}{l}
|\ifchilddoc\||else|\\
|\addtocounter{page}{-1}|\\
\textit{code for title page}\\
|\newpage|\\
|\||fi|
\end{tabular}
\end{center}
%
A banner page for the child documents can be generated by:
%
\begin{center}
\begin{tabular}{l}
|\ifchilddoc|\\
|\addtocounter{page}{-1}|\\
\textit{code for banner page}\\
|\newpage|\\
|\||fi|
\end{tabular}
\end{center}
%
Here one could write a message such as:
\begin{center}
|This is the part \childdocname{} of \childdocjob{}.|
\end{center}

%%%%%%%%%%%%%%%%%%%%%%%%%%%%%%%%%%%%%%%%%%%%%%%%%%%%%%%%%%%%%%%%%%%%%%%%%%%%%%%%
\subsection{Flags}
\label{sec:flags}

The package makes it easy to generate different versions
of the main or child documents.
To this end compilation flags can be defined
and assigned different default values.
They will be particularly useful in conjunction
with the forwarding mechanism described in \secref{sec:forward}.

For example, it may be useful to have a flag |\version|
which can be set to |draft| or |final|.
The document source will contain some conditional code
depending on the value of |\version|.
Suppose further, the flag should default to |final| for the main file
and to |draft| for child files
which is a natural assignment for editing the document.
This is achieved by placing the following code
in the preamble of the main document
(below the |\childdocmain| directive):
%
\begin{center}
\begin{tabular}{l}
|\ifchilddoc|\\
|\providecommand{\version}{draft}|\\
|\||else|\\
|\providecommand{\version}{final}|\\
|\||fi|
\end{tabular}
\end{center}
%
The definition by |\providecommand| makes sure
that previous definitions are not overwritten.
Further statements |\providecommand{\version}{...}|
can thus be added before the above code to override it.

For the main file, one might add a line
(between |\childdocmain| and the above block)
%
\begin{center}
|%\ifchilddoc\||else\providecommand{\version}{draft}\||fi|
\end{center}
%
which can be uncommented to produce a draft version.
Likewise one can add a line to the very top of a child file
(above the |\childdocof{|\textit{main}|}| directive)
%
\begin{center}
|%\providecommand{\version}{final}|
\end{center}
%
which can be uncommented to produce the final version of this child document.

%%%%%%%%%%%%%%%%%%%%%%%%%%%%%%%%%%%%%%%%%%%%%%%%%%%%%%%%%%%%%%%%%%%%%%%%%%%%%%%%
\subsection{Forwarding}
\label{sec:forward}

Different versions of the main or child documents
using compilation flags as described in \secref{sec:flags}
can be (permanently) stored in different files
for convenient compilation, viewing and distribution.
To this end, the package defines a command
to pass on compilation to a different file:

%%%%%%%%%%%%%%%%%%%%%%%%%%%%%%%%%%%%%%%%
\DescribeMacro{\childdocforward}
The command |\childdocforward| redirects processing to
another source file:
%
\begin{center}
\begin{tabular}{l}
|\input{childdoc.def}|\\
|\childdocforward[|\textit{main}|]{|\textit{dest}|}|\\
\end{tabular}
\end{center}
%
The argument \textit{dest} is the destination file
(without extension).
It should be the main file or one of the child files.
Note that further \textsf{childdoc} directives
such as |\childdocof| and |\childdocforward|
in the indicated file will be processed in this form.
The optional argument \textit{main}
passes on directly to the main file \textit{main}
while pretending to compile the child \textit{dest}.
This form behaves as if \textit{dest}
issues |\childdocof{|\textit{main}|}| right away,
and no further \textsf{childdoc} directives will be processed.

%%%%%%%%%%%%%%%%%%%%%%%%%%%%%%%%%%%%%%%%
\DescribeMacro{\...prefix}
In the alternative form |\childdocforwardprefix|,
%
\begin{center}
\begin{tabular}{l}
|\input{childdoc.def}|\\
|\childdocforwardprefix[|\textit{main}|]{|\textit{prefix}|}{|\textit{dest}|}|
\end{tabular}
\end{center}
%
the destination file is determined by a pattern
depending on the current file:
To make this work, the current file must be called
`{\textit{prefix}\hspace{0.2em}\textit{suffix}}'
with \textit{prefix} matching precisely the argument.
Processing is then passed on to the file
`{\textit{dest}\hspace{0.2em}\textit{suffix}}'.
Surely, the same effect is achieved by
directly specifying the
argument `{\textit{dest}\hspace{0.2em}\textit{suffix}}'
in the first form.
However, that requires to set up a different file
for each child. With the alternative form of the command
all these files can have exactly the same content
which simplifies setting them up and maintaining them.

For example, the following file |draft.tex|
with a compilation flag |\version| as described in \secref{sec:flags}
compiles the main document as a draft:
%
\begin{center}
\begin{tabular}{l}
|\def\version{draft}|\\
|\input{childdoc.def}|\\
|\childdocforward{|\textit{main}|}|
\end{tabular}
\end{center}
%
Likewise, the following files |final|\textit{nn}|.tex|
compile the final version of the child document
|child|\textit{nn}|.tex|:
%
\begin{center}
\begin{tabular}{l}
|\def\version{final}|\\
|\input{childdoc.def}|\\
|\childdocforwardprefix{final}{child}|
\end{tabular}
\end{center}
%

Note that when several versions of a main file and/or of each child file
are to be generated, it may be convenient to set up a |Makefile| or
shell script to automatise the process.

%%%%%%%%%%%%%%%%%%%%%%%%%%%%%%%%%%%%%%%%%%%%%%%%%%%%%%%%%%%%%%%%%%%%%%%%%%%%%%%%
\subsection{Command Line Processing}
\label{sec:commandline}

The effect of redirection files can also be achieved by invoking
the \LaTeX{} compiler with a more elaborate command line.
Most conveniently this should be done as part
of a shell script or a |Makefile|.

When using \textsf{childdoc} in the main file, the following
command lines effectively perform a redirection
(note that depending on the shell being used,
backslashes may have to be doubled: `|\|' $\to$ `|\\|'):
%
\begin{center}
|... -jobname "|\textit{target}|" |\\|"|[\textit{flags}]%
|\input{childdoc.def}\childdocforward[|\textit{main}|]{|\textit{dest}|}"|
\end{center}
%
Here \textit{target} is the name of the output file,
\textit{main} is the name of the main file
and \textit{dest} is the name of the main or child file to be processed
(all filenames without extensions).
The optional argument \textit{main} can be omitted
if \textit{main} matches \textit{dest}.
Optionally, compilation \textit{flags} can be defined via |\def| commands.
This command line makes the \TeX{} engine believe
it is compiling the file \textit{target}
whose content is specified as the latter parameter.
The provided code then forwards the processing to
\textit{main} or \textit{dest} as described in \secref{sec:forward}.

%%%%%%%%%%%%%%%%%%%%%%%%%%%%%%%%%%%%%%%%%%%%%%%%%%%%%%%%%%%%%%%%%%%%%%%%%%%%%%%%
\subsection{Include by Input}
\label{sec:input}

Including child documents by |\include| has some restrictions by design.
Most notably, the content of a child document always occupies
its own set of pages; pages cannot be shared between child documents.
Usually, this behaviour makes perfect sense
because each child document contain an essential part of the document.
However, in some situations it may be desirable to compose
a document from a collection of parts
without having mandatory page breaks between then.
For this case, the package
provides a mechanism to include parts
by |\input| which can also be processed individually.
However, by construction this mechanism
requires manual handling of the content to be output.

%%%%%%%%%%%%%%%%%%%%%%%%%%%%%%%%%%%%%%%%
\DescribeMacro{\ifchilddocmanual}
The main file should be prepared as usual, see \secref{sec:include}.
However, the document body must make a distinction
between processing of an individual part and of the main document, e.g.:
%
\begin{center}
\begin{tabular}{l}
|\ifchilddocmanual|\\
|\input{\childdocname}|\\
|\||else|\\
\textit{document body with }|\input{|\textit{part}|}|\\
|\||fi|
\end{tabular}
\end{center}
%
The conditional |\ifchilddocmanual| is true whenever
a part to be included by |\input| is being compiled,
and the name of the part is stored in |\childdocname|.

%%%%%%%%%%%%%%%%%%%%%%%%%%%%%%%%%%%%%%%%
\DescribeMacro{\childdocby}
Each part to be included by |\input| should start with:
%
\begin{center}
\begin{tabular}{l}
|\input{childdoc.def}|\\
|\childdocby{|\textit{main}|}|\\
\end{tabular}
\end{center}
%
The directive |\childdocby| is similar to |\childdocof|
described in \secref{sec:include},
but the subsequent selection of content must be done manually.
To that end, both |\ifchilddoc| and |\ifchilddocmanual|
will be true upon processing of a part,
and the name of the part is stored in |\childdocname|.
Note that |\jobname| will be set to the filename of the current part
so that each part receives an individual |.aux| file
that does not interfere with the |.aux| file(s) of the main document.
This behaviour can be altered by the alternative form
|\childdocby[*]{|\textit{main}|}| (with a non-empty optional argument)
which uses the |.aux| file of the main document
by setting |\jobname| to \textit{main}.

%%%%%%%%%%%%%%%%%%%%%%%%%%%%%%%%%%%%%%%%%%%%%%%%%%%%%%%%%%%%%%%%%%%%%%%%%%%%%%%%
\subsection{Driver Development}
\label{sec:driver}

The \textsf{childdoc} mechanism can also be use for the development
of definition files such as \LaTeX{} styles or classes.
This case differs from the above setup with multiple parts
included by |\include| in that no |\includeonly| should be invoked.
This can be achieved by starting the include file
(before |\ProvidesPackage|) with:
%
\begin{center}
\begin{tabular}{l}
|\input{childdoc.def}|\\
|\childdocforward{|\textit{main}|}|\\
\end{tabular}
\end{center}
%
or alternatively with:
%
\begin{center}
\begin{tabular}{l}
|\input{childdoc.def}|\\
|\childdocby{|\textit{main}|}|\\
\end{tabular}
\end{center}
%
Both forms have slightly different effects as described above.
The main file is prepared as usual, see \secref{sec:include}.

%%%%%%%%%%%%%%%%%%%%%%%%%%%%%%%%%%%%%%%%%%%%%%%%%%%%%%%%%%%%%%%%%%%%%%%%%%%%%%%%
\subsection{Legacy Detection}
\label{sec:detection}

The directive |\childdocmain| in the main file can detect
whether the complete document or merely a child is to be compiled
even without using the directive |\childdocof|.
This method is deprecated because it is less robust
and there is no compelling reason to use it;
it is merely provided for backward compatibility
and it may be removed in future versions.

If the detection mechanism is to be used,
it is mandatory to correctly specify
the filename of the main file as the argument of |\childdocmain|:
%
\begin{center}
\begin{tabular}{l}
|\input{childdoc.def}|\\
|\childdocmain{|\textit{main}|}|\\
\end{tabular}
\end{center}
%
If |\jobname| does not match the argument \textit{main} of |\childdocmain|,
it is assumed that |\jobname| points to the child file to be compiled.
When using |\childdocmain| with the main file specified as argument,
it suffices to start a child file
with just |\input{|\textit{main}|}|
without loading of the package and using |\childdocof|.
If instead all processing is done
with the appropriate \textsf{childdoc} directives,
the argument of \textit{main} of |\childdocmain| can be empty.

An alternative version of the command line processing described
in \secref{sec:commandline} using the detection mechanism reads:
%
\begin{center}
|... -jobname "|\textit{target}|" "|[\textit{flags}]%
[|\def\jobname{|\textit{dest}|}|]|\input{|\textit{main}|}"|
\end{center}

%%%%%%%%%%%%%%%%%%%%%%%%%%%%%%%%%%%%%%%%%%%%%%%%%%%%%%%%%%%%%%%%%%%%%%%%%%%%%%%%
\subsection{Manual Code}
\label{sec:manual}

In case one cannot be certain whether the definitions file |childdoc.def|
is installed on the target \TeX{} distribution
and one prefers not to ship it,
it is conceivable to paste a few relevant commands into the sources.

To that end, drop all statements |\input{childdoc.def}|
and perform the replacements as outlined below.
Instead of |\childdocmain{|\textit{main}|}| add the following code
to the top of the main file:
%
\begin{center}
\begin{tabular}{l}
|\||ifdefined\childdocname\endinput\||fi\newif\ifchilddoc|\\
|\edef\childdocname{\scantokens\expandafter{\jobname\noexpand}}|\\
|\def\childdocmain{|\textit{main}|}\||ifx\childdocmain\childdocname\||else|\\
|\childdoctrue\includeonly{\childdocname}\let\jobname\childdocmain\||fi|\\
\end{tabular}
\end{center}
%
Instead of |\childdocof{|\textit{main}|}| just include the main file
at the top of each child file:
%
\begin{center}
|\input{|\textit{main}|}|
\end{center}
%
A simple redirection |\childdocforward{|\textit{dest}|}| is achieved by:
%
\begin{center}
|\def\jobname{|\textit{dest}|}\input{\jobname}|
\end{center}
%
The redirection with prefix
|\childdocforwardprefix[|\textit{prefix}|]{|\textit{dest}|}|
is accomplished by:
%
\begin{center}
\begin{tabular}{l}
|{\edef\jobname{\scantokens\expandafter{\jobname\noexpand}}|\\
|\def\redirectjob |\textit{prefix}|#1~~~{\gdef\jobname{|\textit{dest}|#1}}|\\
|\expandafter\redirectjob\jobname~~~}\input{\jobname}|
\end{tabular}
\end{center}

In an alternative approach,
child documents can be compiled by a specific command line
without additional code or specific definitions:
%
\begin{center}
|... -jobname "|\textit{target}|" "|[\textit{flags}]%
|\includeonly{|\textit{dest}|}\input{|\textit{main}|}"|
\end{center}
%

%%%%%%%%%%%%%%%%%%%%%%%%%%%%%%%%%%%%%%%%%%%%%%%%%%%%%%%%%%%%%%%%%%%%%%%%%%%%%%%%
%%%%%%%%%%%%%%%%%%%%%%%%%%%%%%%%%%%%%%%%%%%%%%%%%%%%%%%%%%%%%%%%%%%%%%%%%%%%%%%%
\section{Information}

%%%%%%%%%%%%%%%%%%%%%%%%%%%%%%%%%%%%%%%%%%%%%%%%%%%%%%%%%%%%%%%%%%%%%%%%%%%%%%%%
\subsection{Copyright}

Copyright \copyright{} 2017--2018 Niklas Beisert

This work may be distributed and/or modified under the
conditions of the \LaTeX{} Project Public License, either version 1.3
of this license or (at your option) any later version.
The latest version of this license is in
  \url{http://www.latex-project.org/lppl.txt}
and version 1.3 or later is part of all distributions of \LaTeX{}
version 2005/12/01 or later.

This work has the LPPL maintenance status `maintained'.

The Current Maintainer of this work is Niklas Beisert.

This work consists of the files |README.txt|, |childdoc.ins| and |childdoc.dtx|
as well as the derived files |childdoc.def|, |cdocsamp.tex|
with |cdocsch1.tex|, |cdocsch2.tex|, |cdocspt3.tex|, |cdocspt4.tex|,
|cdocsdrf.tex|, |cdocsfn1.tex|, |cdocsfn2.tex|
as well as |childdoc.pdf|.

%%%%%%%%%%%%%%%%%%%%%%%%%%%%%%%%%%%%%%%%%%%%%%%%%%%%%%%%%%%%%%%%%%%%%%%%%%%%%%%%
\subsection{Files and Installation}

The package consists of the files:
%
\begin{center}
\begin{tabular}{ll}
    |README.txt|   & readme file \\
    |childdoc.ins| & installation file \\
    |childdoc.dtx| & source file \\
    |childdoc.def| & definition file \\
    |cdocsamp.tex| & sample main file \\
    |cdocsch1.tex| & sample include file \\
    |cdocsch2.tex| & sample include file \\
    |cdocspt3.tex| & sample part file \\
    |cdocspt4.tex| & sample part file \\
    |cdocsdrf.tex| & sample redirection file \\
    |cdocsfn1.tex| & sample redirection file \\
    |cdocsfn2.tex| & sample redirection file \\
    |childdoc.pdf| & manual
\end{tabular}
\end{center}
%
The distribution consists of the files
|README.txt|, |childdoc.ins| and |childdoc.dtx|.
%
\begin{itemize}
\item
Run (pdf)\LaTeX{} on |childdoc.dtx|
to compile the manual |childdoc.pdf| (this file).
\item
Run \LaTeX{} on |childdoc.ins| to create the definitions file |childdoc.def|
and the sample |cdocsamp.tex| with include files
|cdocsch1.tex|, |cdocsch2.tex|, |cdocspt3.tex|, |cdocspt4.tex|,
|cdocsdrf.tex|, |cdocsfn1.tex|, |cdocsfn2.tex|.
Then copy the file |childdoc.def| to an appropriate directory of your \LaTeX{}
distribution, e.g.\ \textit{texmf-root}|/tex/latex/childdoc|.
\end{itemize}

%%%%%%%%%%%%%%%%%%%%%%%%%%%%%%%%%%%%%%%%%%%%%%%%%%%%%%%%%%%%%%%%%%%%%%%%%%%%%%%%
\subsection{Related CTAN Packages}

There are several other packages which offer a similar functionality:
%
\begin{itemize}
\item
The packages
\href{http://ctan.org/pkg/docmute}{\textsf{docmute}},
\href{http://ctan.org/pkg/includex}{\textsf{includex}} and
\href{http://ctan.org/pkg/standalone}{\textsf{standalone}}
provide commands to include only the document body of
a child file thus allowing both files to be compiled individually.
\item
The packages \href{http://ctan.org/pkg/subdocs}{\textsf{subdocs}}
and \href{http://ctan.org/pkg/subfiles}{\textsf{subfiles}}
provide structures in which the main and child documents can be
encapsulated and allowing them to be compiled individually.
The inclusion mechanism is different from the conventional |\include|.
\item
The package \href{http://ctan.org/pkg/combine}{\textsf{combine}}
is an elaborate solution to combine several documents into one.
\end{itemize}
%
See also the CTAN topic \href{http://ctan.org/topic/subdocs}{\textsf{subdocs}}
for further related packages.
The present package differs from the above solutions in that
a document structure constructed with the conventional |\include| mechanism
just needs two extra commands at the top of every file
such that all constituent files can be compiled individually.

%%%%%%%%%%%%%%%%%%%%%%%%%%%%%%%%%%%%%%%%%%%%%%%%%%%%%%%%%%%%%%%%%%%%%%%%%%%%%%%%
%\subsection{Feature Suggestions}
%
%The following is a list of features which may be useful for future
%versions of this package:
%%
%\begin{itemize}
%\item
%\ldots
%\end{itemize}

%%%%%%%%%%%%%%%%%%%%%%%%%%%%%%%%%%%%%%%%%%%%%%%%%%%%%%%%%%%%%%%%%%%%%%%%%%%%%%%%
\subsection{Revision History}

%%%%%%%%%%%%%%%%%%%%%%%%%%%%%%%%%%%%%%%%
\paragraph{v2.0:} 2018/12/30

\begin{itemize}
\item
immediate forward processing
\item
added |\childdocby| mechanism
\item
manual restructured
\end{itemize}

%%%%%%%%%%%%%%%%%%%%%%%%%%%%%%%%%%%%%%%%
\paragraph{v1.6:} 2018/01/17

\begin{itemize}
\item
application for development of include files
\item
corrections to manual
\end{itemize}

%%%%%%%%%%%%%%%%%%%%%%%%%%%%%%%%%%%%%%%%
\paragraph{v1.5:} 2017/05/21

\begin{itemize}
\item
more complete structuring introduced
\item
|\childdocof| introduced
\item
|\childdoc| renamed to |\childdocmain|
\item
|\childredirect| renamed to |\childdocforward| and |\childdocforwardprefix|
and functionality expanded
\end{itemize}

%%%%%%%%%%%%%%%%%%%%%%%%%%%%%%%%%%%%%%%%
\paragraph{v1.0:} 2017/04/27

\begin{itemize}
\item
manual and install package
\item
first version published on CTAN
\end{itemize}

%%%%%%%%%%%%%%%%%%%%%%%%%%%%%%%%%%%%%%%%
\paragraph{v0.6:} 2017/04/26

\begin{itemize}
\item
redirection mechanism added
\end{itemize}

%%%%%%%%%%%%%%%%%%%%%%%%%%%%%%%%%%%%%%%%
\paragraph{v0.5:} 2017/04/26

\begin{itemize}
\item
functionality in definition file
\end{itemize}


%%%%%%%%%%%%%%%%%%%%%%%%%%%%%%%%%%%%%%%%%%%%%%%%%%%%%%%%%%%%%%%%%%%%%%%%%%%%%%%%
%%%%%%%%%%%%%%%%%%%%%%%%%%%%%%%%%%%%%%%%%%%%%%%%%%%%%%%%%%%%%%%%%%%%%%%%%%%%%%%%
%%%%%%%%%%%%%%%%%%%%%%%%%%%%%%%%%%%%%%%%%%%%%%%%%%%%%%%%%%%%%%%%%%%%%%%%%%%%%%%%
\appendix

\settowidth\MacroIndent{\rmfamily\scriptsize 000\ }

 \DocInput{childdoc.dtx}

\end{document}
%</driver>
% \fi
%
% %%%%%%%%%%%%%%%%%%%%%%%%%%%%%%%%%%%%%%%%%%%%%%%%%%%%%%%%%%%%%%%%%%%%%%%%%%%%%%
% %%%%%%%%%%%%%%%%%%%%%%%%%%%%%%%%%%%%%%%%%%%%%%%%%%%%%%%%%%%%%%%%%%%%%%%%%%%%%%
% \section{Sample}
%\iffalse
%<*samplemain>
%\fi
%
% The following presents a sample document
% with two chapters, two parts, a title page,
% a compile flag as well as three forwarding files to set the flag.
% It consists of eight |.tex| files:
% \begin{center}
% \begin{tabular}{ll}
% |cdocsamp.tex|&main file\\
% |cdocsch1.tex|&include file for chapter 1\\
% |cdocsch2.tex|&include file for chapter 2\\
% |cdocspt3.tex|&include file for part 3\\
% |cdocspt4.tex|&include file for part 4\\
% |cdocsdrf.tex|&forwarding file for main file in draft mode\\
% |cdocsfi1.tex|&forwarding file for final version of chapter 1\\
% |cdocsfi2.tex|&forwarding file for final version of chapter 2\\
% \end{tabular}
% \end{center}
% Each of the eight files can be compiled directly by the \LaTeX{} compiler.
%
% %%%%%%%%%%%%%%%%%%%%%%%%%%%%%%%%%%%%%%
% \paragraph{Main File.}
%
% The main file is called |cdocsamp.tex|.
%
% Load the \textsf{childdoc} definitions and
% declare the filename for the main document:
%    \begin{macrocode}
\input{childdoc.def}
\childdocmain{}
%    \end{macrocode}

% Optional override for |\version| flag:
%    \begin{macrocode}
%%\ifchilddoc\else\providecommand{\version}{draft}\fi
%    \end{macrocode}

% Define the default values for the |\version| flag
% (|final| for the main file and |draft| for childs):
%    \begin{macrocode}
\ifchilddoc
\providecommand{\version}{draft}
\else
\providecommand{\version}{final}
\fi
%    \end{macrocode}

% Load the standard document class:
%    \begin{macrocode}
\documentclass[12pt]{article}
%    \end{macrocode}

% Start the document body:
%    \begin{macrocode}
\begin{document}
%    \end{macrocode}

% Declare a title page.
% Print title, part of document being processed and version flag:
%    \begin{macrocode}
\addtocounter{page}{-1}
\begin{center}
{\LARGE\bfseries{}childdoc example\par}
\vspace{1cm}
\ifchilddoc
\ifchilddocmanual part\else chapter\fi:
`\childdocname' of `\childdocjob'\par
\else
main document: `\childdocjob'\par
\fi
version: \version\par
\end{center}
\newpage
%    \end{macrocode}

% Manually include selected file,
% otherwise process as usual:
%    \begin{macrocode}
\ifchilddocmanual
\section*{part `\childdocname'}
\input{\childdocname}
\else
%    \end{macrocode}

% Include the two chapters:
%    \begin{macrocode}
\include{cdocsch1}
\include{cdocsch2}
%    \end{macrocode}

% Include the two parts unless only chapters should be displayed:
%    \begin{macrocode}
\ifchilddoc\else
\section{part three}
\input{cdocspt3}
\section{part four}
\input{cdocspt4}
\fi
%    \end{macrocode}

% Process as usual until here:
%    \begin{macrocode}
\fi
%    \end{macrocode}

% End of document body:
%    \begin{macrocode}
\end{document}
%    \end{macrocode}
%\iffalse
%</samplemain>
%\fi
%
% %%%%%%%%%%%%%%%%%%%%%%%%%%%%%%%%%%%%%%
% \paragraph{Chapter Include Files.}
%
% The include files are called |cdocsch1.tex| and |cdocsch2.tex|.
%
%\iffalse
%<*samplechap1|samplechap2>
%\fi

% Optional override for |\version| flag:
%    \begin{macrocode}
%%\providecommand{\version}{final}
%    \end{macrocode}

% Include the main document:
%    \begin{macrocode}
\input{childdoc.def}
\childdocof{cdocsamp}
%    \end{macrocode}

%\iffalse
%</samplechap1|samplechap2>
%\fi
%
%\iffalse
%<*samplechap1>
%\fi
% Some text for chapter 1:
%    \begin{macrocode}
\section{one}
some text in chapter one
%    \end{macrocode}

%\iffalse
%</samplechap1>
%\fi
% Some text for chapter 2:
%\iffalse
%<*samplechap2>
%\fi
%    \begin{macrocode}
\section{two}
more text in chapter two
%    \end{macrocode}

%\iffalse
%</samplechap2>
%\fi
%
% %%%%%%%%%%%%%%%%%%%%%%%%%%%%%%%%%%%%%%
% \paragraph{Part Include Files.}
%
% The include files are called |cdocspt3.tex| and |cdocspt4.tex|.
%
%\iffalse
%<*samplepart3|samplepart4>
%\fi

% Optional override for |\version| flag:
%    \begin{macrocode}
%%\providecommand{\version}{final}
%    \end{macrocode}

% Include the main document:
%    \begin{macrocode}
\input{childdoc.def}
\childdocby{cdocsamp}
%    \end{macrocode}

%\iffalse
%</samplepart3|samplepart4>
%\fi
%
%\iffalse
%<*samplepart3>
%\fi
% Some text for part 3:
%    \begin{macrocode}
some text in part three
%    \end{macrocode}

%\iffalse
%</samplepart3>
%\fi
% Some text for part 4:
%\iffalse
%<*samplepart4>
%\fi
%    \begin{macrocode}
more text in part four
%    \end{macrocode}

%\iffalse
%</samplepart4>
%\fi
%
% %%%%%%%%%%%%%%%%%%%%%%%%%%%%%%%%%%%%%%
% \paragraph{Forwarding for a Complete Draft.}
%
% The following forwarding file |cdocsdrf.tex|
% compiles the main document in draft mode:
%\iffalse
%<*sampledraft>
%\fi
%    \begin{macrocode}
\def\version{draft}
\input{childdoc.def}
\childdocforward{cdocsamp}
%    \end{macrocode}

%\iffalse
%</sampledraft>
%\fi
%
% %%%%%%%%%%%%%%%%%%%%%%%%%%%%%%%%%%%%%%
% \paragraph{Forwarding for Final Version of the Chapters.}
%
% The following forwarding files |cdocsfn1.tex| and |cdocsfn2.tex|
% (with identical content)
% compile the final versions of the child documents
% |cdocsch1.tex| and |cdocsch2.tex|, respectively:
%\iffalse
%<*samplefinal>
%\fi
%    \begin{macrocode}
\def\version{final}
\input{childdoc.def}
\childdocforwardprefix[cdocsamp]{cdocsfn}{cdocsch}
%    \end{macrocode}

%\iffalse
%</samplefinal>
%\fi
%
% %%%%%%%%%%%%%%%%%%%%%%%%%%%%%%%%%%%%%%
% \paragraph{Command Line Processing.}
%
% The following three command lines generate the output files
% |cdocscld|, |cdocscl1| and |cdocscl2|
% which should be identical to
% |cdocsdrf|, |cdocsch1| and |cdocsfn2|, respectively:
% \begin{center}
% \begin{tabular}{l}
% |latex -jobname cdocscld \|\\
% |  "\def\version{draft}\input{childdoc.def}\childdocforward{cdocsamp}"|\\
% |latex -jobname cdocscl1 \|\\
% |  "\input{childdoc.def}\childdocforward[cdocsamp]{cdocsch1}"|\\
% |latex -jobname cdocscl2 \|\\
% |  "\def\version{final}\input{childdoc.def}\childdocforward{cdocsch2}"|
% \end{tabular}
% \end{center}
% Note that the trailing backslash on each first line
% merely continues the input to the second line
% (for convenient cut ant paste).
% Furthermore, the command |latex| can be replaced by any
% of its alternative versions such as |pdflatex|.
%
% %%%%%%%%%%%%%%%%%%%%%%%%%%%%%%%%%%%%%%%%%%%%%%%%%%%%%%%%%%%%%%%%%%%%%%%%%%%%%%
% %%%%%%%%%%%%%%%%%%%%%%%%%%%%%%%%%%%%%%%%%%%%%%%%%%%%%%%%%%%%%%%%%%%%%%%%%%%%%%
% \section{Implementation}
%\iffalse
%<*package>
%\fi
%
% This section describes the definitions file |childdoc.def|.

% The definitions cannot be loaded using |\usepackage| or |\RequirePackage|
% which has a mechanism to prevent loading a style file more than once.
% When loading the definitions by means of |\input|
% multiple instances have to be prevented manually:
%\iffalse
%This code needs to be before the `\ProvidesFile' directive
%which is defined at the beginning of this file.
%Therefore it is also placed there and commented out here.
%</package>
%<*discard>
%\fi
%    \begin{macrocode}
\ifdefined\childdocmain\endinput\fi
%    \end{macrocode}
%\iffalse
%</discard>
%<*package>
%\fi
%
% \macro{\ifchilddoc}
% \macro{\ifchilddocmanual}
% The conditional |\ifchilddoc| tells whether a
% child (true) or main (false) document is being compiled.
% The conditional |\ifchilddocmanual| tells whether
% the |\includeonly| mechanism is used (false) or
% the selection of child files must be performed manually (true).
% The definitions initialise to false:
%    \begin{macrocode}
\newif\ifchilddoc
\newif\ifchilddocmanual
%    \end{macrocode}

% \macro{\childdocname}
% \macro{\childdocjob}
% The macro |\childdocname| stores the name of the main document
% to be compiled. The macro |\childdocjob| stores the name of
% the document on which the \LaTeX{} compiler was originally invoked.
% The content of |\jobname| cannot be compared
% to filenames specified in the source due to different catcodes.
% The following code rescans |\jobname|, stores the result
% in |\childdocname| and saves a copy in |\childdocjob|:
%    \begin{macrocode}
\edef\childdocname{\scantokens\expandafter{\jobname\noexpand}}
\let\childdocjob\childdocname
%    \end{macrocode}

% \macro{\childdocdisable}
% The macro |\childdocdisable| prevents the main file
% from being processed more than once.
% At this stage, the main document command |\childdocmain|
% is assumed to be called once again where it should do nothing.
% Any subsequent call to it should prevent
% a secondary processing of the main document
% It overwrites the forwarding commands
% |\childdocof| and |\childdocforward|
% with empty macros to prevent further inclusions of the main document:
%    \begin{macrocode}
\newcommand{\childdocdisable}
{
  \renewcommand{\childdocmain}[1]{\renewcommand{\childdocmain}[1]{\endinput}}
  \renewcommand{\childdocof}[1]{}
  \renewcommand{\childdocby}[2][]{}
  \renewcommand{\childdocforward}[2][]{}
  \renewcommand{\childdocdisable}{}
}
%    \end{macrocode}

% \macro{\childdocmain}
% The macro |\childdocmain| is to be called at the top of the main file
% with nothing or the main filename (without extension) as argument.
% First, it breaks loops.
% If the argument is not empty and does not match |\childdocname|
% (which is set by the first inclusion of |childdoc.def|),
% |\ifchilddoc| is set to true, |\includeonly| is applied to the child file
% and |\jobname| is set to the main file
% (for proper handling of |.aux| files):
%    \begin{macrocode}
\newcommand{\childdocmain}[1]
{
  \childdocdisable\childdocmain{}
  \if?#1?\else
    \begingroup
      \def\childdoctmp{#1}
      \ifx\childdoctmp\childdocname
        \def\childdoctmp{}
      \else
        \def\childdoctmp
        {
          \childdoctrue
          \includeonly{\childdocname}
          \def\childdocjob{#1}
          \def\jobname{#1}
        }
      \fi
      \expandafter
    \endgroup
    \childdoctmp
  \fi
}
%    \end{macrocode}

% \macro{\childdocof}
% The command |\childdocof| redirects
% compilation to the main file |#1|.
%    \begin{macrocode}
\newcommand{\childdocof}[1]
{
  \childdocdisable
  \childdoctrue
  \includeonly{\childdocname}
  \def\jobname{#1}
  \def\childdocjob{#1}
  \input{#1}
}
%    \end{macrocode}

% \macro{\childdocby}
% The command |\childdocby| ....
%    \begin{macrocode}
\newcommand{\childdocby}[2][]
{
  \childdocdisable
  \childdoctrue
  \childdocmanualtrue
  \if?#1?\else
    \def\jobname{#2}
  \fi
  \def\childdocjob{#2}
  \input{#2}
  \endinput
}
%    \end{macrocode}

% \macro{\childdocforward}
% The command |\childdocforward| redirects
% compilation to the main file or
% (if the optional argument is given) a child file.
% Parameters are set as if the main file
% or a child file starting with |\childdocof| was compiled.
% Then compilation is handed over to the main file:
%    \begin{macrocode}
\newcommand{\childdocforward}[2][]
{
  \begingroup
    \if?#1?
      \def\childdoctmp
      {
        \def\childdocname{#2}
        \def\childdocjob{#2}
        \def\jobname{#2}
        \input{#2}
        \endinput
      }
    \else
      \def\childdoctmp
      {
        \childdocdisable
        \def\childdocname{#2}
        \childdoctrue
        \includeonly{#2}
        \def\childdocjob{#1}
        \def\jobname{#1}
        \input{#1}
        \endinput
      }
    \fi
    \expandafter
  \endgroup
  \childdoctmp
}
%    \end{macrocode}

% \macro{\childdocforwardprefix}
% The command |\childdocforwardprefix| redirects
% compilation to the main or a child file by means of a pattern.
% The prefix |#1| in the current filename is replaced by |#2|
% and the suffix of the current filename is kept
% (it is assumed that the filename does not contain the substring `|~~~|'
% which is used as a delimiter).
% Compilation is handed over to the new file by |\childdocforward|:
%    \begin{macrocode}
\newcommand{\childdocforwardprefix}[3][]
{
  \begingroup
    \def\childdocextract #2##1~~~{\def\childdoctmp{\childdocforward[#1]{#3##1}}}
    \expandafter\childdocextract\childdocname~~~
    \expandafter
  \endgroup
  \childdoctmp
}
%    \end{macrocode}

% \macro{\childdoc}
% The deprecated macro |\childdoc| is a legacy version of |\childdocmain|:
%    \begin{macrocode}
\newcommand{\childdoc}{\childdocmain}
%    \end{macrocode}

% \macro{\childdocredirect}
% The deprecated macro |\childdocredirect| is a legacy version
% of |\childdocforward| and |\childdocforwardprefix|:
%    \begin{macrocode}
\newcommand{\childdocredirect}[2][]
{
  \begingroup
    \if?#1?
      \def\childdoctmp{\childdocforward{#2}}
    \else
      \def\childdoctmp{\childdocforwardprefix{#1}{#2}}
    \fi
    \expandafter
  \endgroup
  \childdoctmp
}
%    \end{macrocode}

%\iffalse
%</package>
%\fi
%
\endinput
|\\
|\childdocmain{}|\\
\end{tabular}
\end{center}
at the very top of the main \LaTeX{} file,
in particular \emph{before} the |\documentclass| statement!
The argument of |\childdocmain| should be left empty
(but it must be present).

%%%%%%%%%%%%%%%%%%%%%%%%%%%%%%%%%%%%%%%%
\DescribeMacro{\childdocof}
Furthermore, add the commands
\begin{center}
\begin{tabular}{l}
|% \iffalse
%
% childdoc.dtx Copyright (C) 2017-2018 Niklas Beisert
%
% This work may be distributed and/or modified under the
% conditions of the LaTeX Project Public License, either version 1.3
% of this license or (at your option) any later version.
% The latest version of this license is in
%   http://www.latex-project.org/lppl.txt
% and version 1.3 or later is part of all distributions of LaTeX
% version 2005/12/01 or later.
%
% This work has the LPPL maintenance status `maintained'.
%
% The Current Maintainer of this work is Niklas Beisert.
%
% This work consists of the files childdoc.dtx and childdoc.ins
% and the derived files childdoc.def and cdocsamp.tex with
% cdocsch1.tex, cdocsch2.tex, cdocsdrf.tex, cdocsfn1.tex, cdocsfn2.tex.
%
%<package>\ifdefined\childdocmain\endinput\fi
%<package>\ProvidesFile{childdoc.def}[2018/12/30 v2.0 child document driver]
%<samplemain>\ProvidesFile{cdocsamp.tex}[2018/12/30 v2.0 sample for childdoc]
%<*driver>
%\ProvidesFile{childdoc.drv}[2018/12/30 v2.0 childdoc reference manual file]
\PassOptionsToClass{10pt,a4paper}{article}
\documentclass{ltxdoc}

\usepackage[margin=35mm]{geometry}
\usepackage{hyperref}
\usepackage{hyperxmp}
\usepackage[usenames]{color}

\hypersetup{colorlinks=true}
\hypersetup{pdfstartview=FitH}
\hypersetup{pdfpagemode=UseNone}
\hypersetup{pdfsource={}}
\hypersetup{pdflang={en-UK}}
\hypersetup{pdfcopyright={Copyright 2017-2018 Niklas Beisert.
  This work may be distributed and/or modified under the
  conditions of the LaTeX Project Public License, either version 1.3
  of this license or (at your option) any later version.}}
\hypersetup{pdflicenseurl={http://www.latex-project.org/lppl.txt}}
\hypersetup{pdfcontactaddress={ETH Zurich, ITP, HIT K,
  Wolfgang-Pauli-Strasse 27}}
\hypersetup{pdfcontactpostcode={8093}}
\hypersetup{pdfcontactcity={Zurich}}
\hypersetup{pdfcontactcountry={Switzerland}}
\hypersetup{pdfcontactemail={nbeisert@itp.phys.ethz.ch}}
\hypersetup{pdfcontacturl={http://people.phys.ethz.ch/\xmptilde nbeisert/}}

\newcommand{\secref}[1]{\hyperref[#1]{section \ref*{#1}}}

\parskip1ex
\parindent0pt
\let\olditemize\itemize
\def\itemize{\olditemize\parskip0pt}

\begin{document}

\title{The \textsf{childdoc} Package}
\hypersetup{pdftitle={The childdoc Package}}
\author{Niklas Beisert\\[2ex]
  Institut f\"ur Theoretische Physik\\
  Eidgen\"ossische Technische Hochschule Z\"urich\\
  Wolfgang-Pauli-Strasse 27, 8093 Z\"urich, Switzerland\\[1ex]
  \href{mailto:nbeisert@itp.phys.ethz.ch}
  {\texttt{nbeisert@itp.phys.ethz.ch}}}
\hypersetup{pdfauthor={Niklas Beisert}}
\hypersetup{pdfsubject={Manual for the LaTeX2e Package childdoc}}
\date{30 December 2018, \textsf{v2.0}}
\maketitle

\begin{abstract}\noindent
\textsf{childdoc} is a \LaTeXe{} package
that enables the direct compilation
of document sections included by |\include|
to individual files.
\end{abstract}

\begingroup
\parskip0ex
\tableofcontents
\endgroup

%%%%%%%%%%%%%%%%%%%%%%%%%%%%%%%%%%%%%%%%%%%%%%%%%%%%%%%%%%%%%%%%%%%%%%%%%%%%%%%%
%%%%%%%%%%%%%%%%%%%%%%%%%%%%%%%%%%%%%%%%%%%%%%%%%%%%%%%%%%%%%%%%%%%%%%%%%%%%%%%%
\section{Introduction}

\LaTeX{} provides a mechanism to structure a large document (such as a book)
into a main file and several child files (containing the chapters)
using the |\include| command.
This mechanism is beneficial for documents
which span hundreds of pages in order to
make the source file(s) more manageable.
Moreover, compilation can be restricted to
selected child files by means of the |\includeonly| command.
The latter feature can be used to reduce the compilation time while editing
(this was significantly more useful in the earlier days of \LaTeX{})
or to generate a smaller document which is easier to navigate.
Another application of |\includeonly| is to generate
documents consisting of selected parts of the complete document.

However, there are a few drawbacks of the plain |\include| mechanism:
\begin{itemize}
\item
The child files cannot be compiled on their own,
they can only be compiled via the main file.
A naive editing environment
(such as a text editor with an option
to have the current file processed by \LaTeX)
may require one to switch to the main file before compiling;
attempting to compile the child file produces errors.
\item
The main file must be modified (each time)
to adjust the |\includeonly| command
to the present needs. This easily leaves the main file in a messy state.
\item
The generated document will always carry the filename
of the main document. This is inconvenient if
several child files are to be compiled and
to be kept for distribution.
\end{itemize}

The present package provides a simple interface
to make child files individually compilable by \LaTeX{}.
Compiling a child file then has the same effect as compiling
the main file with an |\includeonly| command
to select the appropriate child.
Moreover the generated document will carry the name of the child
rather than the main file.
This resolves all three above issues.

This feature is meant to make the editing of books,
thesis documents and lecture notes somewhat more convenient.
However, the package can also be used efficiently for
composing a series of documents (such as exercise sheets)
which are typically distributed individually.
It then assists the author in generating the individual documents
(potentially in different versions)
as well as a document containing the collected series.
Another application is in developing style files
or other kinds of included material
where compilation of the style file could redirect
to a sample or test file.

%%%%%%%%%%%%%%%%%%%%%%%%%%%%%%%%%%%%%%%%%%%%%%%%%%%%%%%%%%%%%%%%%%%%%%%%%%%%%%%%
%%%%%%%%%%%%%%%%%%%%%%%%%%%%%%%%%%%%%%%%%%%%%%%%%%%%%%%%%%%%%%%%%%%%%%%%%%%%%%%%
\section{Usage}

First of all, the package \textsf{childdoc} is \emph{not} a standard
\LaTeXe{} |.sty| style file! Therefore it needs to be invoked in
a non-standard way.

%%%%%%%%%%%%%%%%%%%%%%%%%%%%%%%%%%%%%%%%%%%%%%%%%%%%%%%%%%%%%%%%%%%%%%%%%%%%%%%%
\subsection{Included Files}
\label{sec:include}

%%%%%%%%%%%%%%%%%%%%%%%%%%%%%%%%%%%%%%%%
\DescribeMacro{\childdocmain}
To use the package, add the commands
\begin{center}
\begin{tabular}{l}
|\input{childdoc.def}|\\
|\childdocmain{}|\\
\end{tabular}
\end{center}
at the very top of the main \LaTeX{} file,
in particular \emph{before} the |\documentclass| statement!
The argument of |\childdocmain| should be left empty
(but it must be present).

%%%%%%%%%%%%%%%%%%%%%%%%%%%%%%%%%%%%%%%%
\DescribeMacro{\childdocof}
Furthermore, add the commands
\begin{center}
\begin{tabular}{l}
|\input{childdoc.def}|\\
|\childdocof{|\textit{main}|}|\\
\end{tabular}
\end{center}
at the top of every child file \textit{child}
which is included by |\include{|\textit{child}|}|
from within the main file
(or at least for those files to be compiled individually).
The argument \textit{main} must be the filename of the main file.

There are a couple of
considerations in setting up the main and child documents:

%%%%%%%%%%%%%%%%%%%%%%%%%%%%%%%%%%%%%%%%
\paragraph{Restrictions.}

Please note the following restrictions:
\begin{itemize}
\item
|\childdocmain| must be called with one argument \textit{main}
to ensure compatibility with earlier version of the package.
It must either be empty (|\childdocmain{}|)
or precisely match the filename of the main file in which it is specified.
See \secref{sec:detection} for further information.
\item
The filename \textit{main} must be specified without the |.tex| extension.
\item
The filename \textit{main} is case sensitive
(even in case-insensitive file systems)
due to internal string comparison.
\item
The argument \textit{main} should be fully expanded, it cannot be a macro.
\item
Subdirectories and special characters should be avoided in filenames.
\item
The command |\childdocmain{|\textit{main}|}| must be followed by a whitespace.
It should not be followed immediately by another command
or by a comment mark `|%|'.
This is because the \TeX{} parser reads the token immediately following
the argument of |\childdocmain| and puts it
at the beginning of every child section;
however, a white\-space is ignored.
\end{itemize}

%%%%%%%%%%%%%%%%%%%%%%%%%%%%%%%%%%%%%%%%
\paragraph{Content of Main File.}

It is advisable to place all content in the child files included by |\include|.
Any output contained in the main file will appear in all child documents
unless suppressed manually;
it cannot be suppressed automatically by the |\includeonly| directive
and thus should normally be avoided.
A method to include some content in the main file
by means of conditional processing is described in \secref{sec:conditional}.

%%%%%%%%%%%%%%%%%%%%%%%%%%%%%%%%%%%%%%%%
\paragraph{Page Numbering.}

When only a part of the document is compiled,
the appropriate numbering of pages
(as well as other status parameters)
is determined from the |.aux| files.
The latter contain information from previous passes.
However this information needs to propagate through
all intermediate child documents.
Therefore the page numbering in child documents may well
be inconsistent until the complete document is compiled at least once.

A useful (if unconventional) way to always ensure a consistent
page numbering is to restart the numbering in each child document
and denote the pages by `\textit{child}|.|\textit{page}'
where \textit{child} represents the chapter/section number of the child file.
This can be achieved by the command
|\numberwithin{page}{|\textit{child}|}|
of the \textsf{amsmath} package
where \textit{child} can be |chapter| or |section|
depending on the chosen structuring.
Alternatively, one can modify the macro |\thepage| appropriately
and reset the counter |page| at the start of each child file.

%%%%%%%%%%%%%%%%%%%%%%%%%%%%%%%%%%%%%%%%%%%%%%%%%%%%%%%%%%%%%%%%%%%%%%%%%%%%%%%%
\subsection{Conditional Processing}
\label{sec:conditional}

The package provides a mechanism to compile different versions
of a document. To customise the versions further some conditional processing
can come in handy to distinguish which version is being compiled.
The package provides two macros to describe the compilation context:

%%%%%%%%%%%%%%%%%%%%%%%%%%%%%%%%%%%%%%%%
\DescribeMacro{\ifchilddoc}
The conditional |\ifchilddoc| distinguishes between the compilation of
child documents and the main document:
%
\begin{center}
|\ifchilddoc |\textit{child-code}| |[|\||else |\textit{main-code}]| \||fi|
\end{center}

%%%%%%%%%%%%%%%%%%%%%%%%%%%%%%%%%%%%%%%%
\DescribeMacro{\childdocname}
\DescribeMacro{\childdocjob}
The macro |\childdocname| contains the filename (without extension)
of the main or child file being processed.
Note that |\childdocjob| will always contain the name of the main file.

%%%%%%%%%%%%%%%%%%%%%%%%%%%%%%%%%%%%%%%%
\paragraph{Title Page.}

Conditional processing can be used to include a title or banner page
in the main document when proper precautions are taken.
Importantly, the code in the main file should ensure that the page counter
(as well as other status parameters which are stored in the |.aux| files)
takes the same value after the conditional processing.
Otherwise the page numbers may take divergent values
depending on which part is compiled.

For example, a title page could be declared by:
%
\begin{center}
\begin{tabular}{l}
|\ifchilddoc\||else|\\
|\addtocounter{page}{-1}|\\
\textit{code for title page}\\
|\newpage|\\
|\||fi|
\end{tabular}
\end{center}
%
A banner page for the child documents can be generated by:
%
\begin{center}
\begin{tabular}{l}
|\ifchilddoc|\\
|\addtocounter{page}{-1}|\\
\textit{code for banner page}\\
|\newpage|\\
|\||fi|
\end{tabular}
\end{center}
%
Here one could write a message such as:
\begin{center}
|This is the part \childdocname{} of \childdocjob{}.|
\end{center}

%%%%%%%%%%%%%%%%%%%%%%%%%%%%%%%%%%%%%%%%%%%%%%%%%%%%%%%%%%%%%%%%%%%%%%%%%%%%%%%%
\subsection{Flags}
\label{sec:flags}

The package makes it easy to generate different versions
of the main or child documents.
To this end compilation flags can be defined
and assigned different default values.
They will be particularly useful in conjunction
with the forwarding mechanism described in \secref{sec:forward}.

For example, it may be useful to have a flag |\version|
which can be set to |draft| or |final|.
The document source will contain some conditional code
depending on the value of |\version|.
Suppose further, the flag should default to |final| for the main file
and to |draft| for child files
which is a natural assignment for editing the document.
This is achieved by placing the following code
in the preamble of the main document
(below the |\childdocmain| directive):
%
\begin{center}
\begin{tabular}{l}
|\ifchilddoc|\\
|\providecommand{\version}{draft}|\\
|\||else|\\
|\providecommand{\version}{final}|\\
|\||fi|
\end{tabular}
\end{center}
%
The definition by |\providecommand| makes sure
that previous definitions are not overwritten.
Further statements |\providecommand{\version}{...}|
can thus be added before the above code to override it.

For the main file, one might add a line
(between |\childdocmain| and the above block)
%
\begin{center}
|%\ifchilddoc\||else\providecommand{\version}{draft}\||fi|
\end{center}
%
which can be uncommented to produce a draft version.
Likewise one can add a line to the very top of a child file
(above the |\childdocof{|\textit{main}|}| directive)
%
\begin{center}
|%\providecommand{\version}{final}|
\end{center}
%
which can be uncommented to produce the final version of this child document.

%%%%%%%%%%%%%%%%%%%%%%%%%%%%%%%%%%%%%%%%%%%%%%%%%%%%%%%%%%%%%%%%%%%%%%%%%%%%%%%%
\subsection{Forwarding}
\label{sec:forward}

Different versions of the main or child documents
using compilation flags as described in \secref{sec:flags}
can be (permanently) stored in different files
for convenient compilation, viewing and distribution.
To this end, the package defines a command
to pass on compilation to a different file:

%%%%%%%%%%%%%%%%%%%%%%%%%%%%%%%%%%%%%%%%
\DescribeMacro{\childdocforward}
The command |\childdocforward| redirects processing to
another source file:
%
\begin{center}
\begin{tabular}{l}
|\input{childdoc.def}|\\
|\childdocforward[|\textit{main}|]{|\textit{dest}|}|\\
\end{tabular}
\end{center}
%
The argument \textit{dest} is the destination file
(without extension).
It should be the main file or one of the child files.
Note that further \textsf{childdoc} directives
such as |\childdocof| and |\childdocforward|
in the indicated file will be processed in this form.
The optional argument \textit{main}
passes on directly to the main file \textit{main}
while pretending to compile the child \textit{dest}.
This form behaves as if \textit{dest}
issues |\childdocof{|\textit{main}|}| right away,
and no further \textsf{childdoc} directives will be processed.

%%%%%%%%%%%%%%%%%%%%%%%%%%%%%%%%%%%%%%%%
\DescribeMacro{\...prefix}
In the alternative form |\childdocforwardprefix|,
%
\begin{center}
\begin{tabular}{l}
|\input{childdoc.def}|\\
|\childdocforwardprefix[|\textit{main}|]{|\textit{prefix}|}{|\textit{dest}|}|
\end{tabular}
\end{center}
%
the destination file is determined by a pattern
depending on the current file:
To make this work, the current file must be called
`{\textit{prefix}\hspace{0.2em}\textit{suffix}}'
with \textit{prefix} matching precisely the argument.
Processing is then passed on to the file
`{\textit{dest}\hspace{0.2em}\textit{suffix}}'.
Surely, the same effect is achieved by
directly specifying the
argument `{\textit{dest}\hspace{0.2em}\textit{suffix}}'
in the first form.
However, that requires to set up a different file
for each child. With the alternative form of the command
all these files can have exactly the same content
which simplifies setting them up and maintaining them.

For example, the following file |draft.tex|
with a compilation flag |\version| as described in \secref{sec:flags}
compiles the main document as a draft:
%
\begin{center}
\begin{tabular}{l}
|\def\version{draft}|\\
|\input{childdoc.def}|\\
|\childdocforward{|\textit{main}|}|
\end{tabular}
\end{center}
%
Likewise, the following files |final|\textit{nn}|.tex|
compile the final version of the child document
|child|\textit{nn}|.tex|:
%
\begin{center}
\begin{tabular}{l}
|\def\version{final}|\\
|\input{childdoc.def}|\\
|\childdocforwardprefix{final}{child}|
\end{tabular}
\end{center}
%

Note that when several versions of a main file and/or of each child file
are to be generated, it may be convenient to set up a |Makefile| or
shell script to automatise the process.

%%%%%%%%%%%%%%%%%%%%%%%%%%%%%%%%%%%%%%%%%%%%%%%%%%%%%%%%%%%%%%%%%%%%%%%%%%%%%%%%
\subsection{Command Line Processing}
\label{sec:commandline}

The effect of redirection files can also be achieved by invoking
the \LaTeX{} compiler with a more elaborate command line.
Most conveniently this should be done as part
of a shell script or a |Makefile|.

When using \textsf{childdoc} in the main file, the following
command lines effectively perform a redirection
(note that depending on the shell being used,
backslashes may have to be doubled: `|\|' $\to$ `|\\|'):
%
\begin{center}
|... -jobname "|\textit{target}|" |\\|"|[\textit{flags}]%
|\input{childdoc.def}\childdocforward[|\textit{main}|]{|\textit{dest}|}"|
\end{center}
%
Here \textit{target} is the name of the output file,
\textit{main} is the name of the main file
and \textit{dest} is the name of the main or child file to be processed
(all filenames without extensions).
The optional argument \textit{main} can be omitted
if \textit{main} matches \textit{dest}.
Optionally, compilation \textit{flags} can be defined via |\def| commands.
This command line makes the \TeX{} engine believe
it is compiling the file \textit{target}
whose content is specified as the latter parameter.
The provided code then forwards the processing to
\textit{main} or \textit{dest} as described in \secref{sec:forward}.

%%%%%%%%%%%%%%%%%%%%%%%%%%%%%%%%%%%%%%%%%%%%%%%%%%%%%%%%%%%%%%%%%%%%%%%%%%%%%%%%
\subsection{Include by Input}
\label{sec:input}

Including child documents by |\include| has some restrictions by design.
Most notably, the content of a child document always occupies
its own set of pages; pages cannot be shared between child documents.
Usually, this behaviour makes perfect sense
because each child document contain an essential part of the document.
However, in some situations it may be desirable to compose
a document from a collection of parts
without having mandatory page breaks between then.
For this case, the package
provides a mechanism to include parts
by |\input| which can also be processed individually.
However, by construction this mechanism
requires manual handling of the content to be output.

%%%%%%%%%%%%%%%%%%%%%%%%%%%%%%%%%%%%%%%%
\DescribeMacro{\ifchilddocmanual}
The main file should be prepared as usual, see \secref{sec:include}.
However, the document body must make a distinction
between processing of an individual part and of the main document, e.g.:
%
\begin{center}
\begin{tabular}{l}
|\ifchilddocmanual|\\
|\input{\childdocname}|\\
|\||else|\\
\textit{document body with }|\input{|\textit{part}|}|\\
|\||fi|
\end{tabular}
\end{center}
%
The conditional |\ifchilddocmanual| is true whenever
a part to be included by |\input| is being compiled,
and the name of the part is stored in |\childdocname|.

%%%%%%%%%%%%%%%%%%%%%%%%%%%%%%%%%%%%%%%%
\DescribeMacro{\childdocby}
Each part to be included by |\input| should start with:
%
\begin{center}
\begin{tabular}{l}
|\input{childdoc.def}|\\
|\childdocby{|\textit{main}|}|\\
\end{tabular}
\end{center}
%
The directive |\childdocby| is similar to |\childdocof|
described in \secref{sec:include},
but the subsequent selection of content must be done manually.
To that end, both |\ifchilddoc| and |\ifchilddocmanual|
will be true upon processing of a part,
and the name of the part is stored in |\childdocname|.
Note that |\jobname| will be set to the filename of the current part
so that each part receives an individual |.aux| file
that does not interfere with the |.aux| file(s) of the main document.
This behaviour can be altered by the alternative form
|\childdocby[*]{|\textit{main}|}| (with a non-empty optional argument)
which uses the |.aux| file of the main document
by setting |\jobname| to \textit{main}.

%%%%%%%%%%%%%%%%%%%%%%%%%%%%%%%%%%%%%%%%%%%%%%%%%%%%%%%%%%%%%%%%%%%%%%%%%%%%%%%%
\subsection{Driver Development}
\label{sec:driver}

The \textsf{childdoc} mechanism can also be use for the development
of definition files such as \LaTeX{} styles or classes.
This case differs from the above setup with multiple parts
included by |\include| in that no |\includeonly| should be invoked.
This can be achieved by starting the include file
(before |\ProvidesPackage|) with:
%
\begin{center}
\begin{tabular}{l}
|\input{childdoc.def}|\\
|\childdocforward{|\textit{main}|}|\\
\end{tabular}
\end{center}
%
or alternatively with:
%
\begin{center}
\begin{tabular}{l}
|\input{childdoc.def}|\\
|\childdocby{|\textit{main}|}|\\
\end{tabular}
\end{center}
%
Both forms have slightly different effects as described above.
The main file is prepared as usual, see \secref{sec:include}.

%%%%%%%%%%%%%%%%%%%%%%%%%%%%%%%%%%%%%%%%%%%%%%%%%%%%%%%%%%%%%%%%%%%%%%%%%%%%%%%%
\subsection{Legacy Detection}
\label{sec:detection}

The directive |\childdocmain| in the main file can detect
whether the complete document or merely a child is to be compiled
even without using the directive |\childdocof|.
This method is deprecated because it is less robust
and there is no compelling reason to use it;
it is merely provided for backward compatibility
and it may be removed in future versions.

If the detection mechanism is to be used,
it is mandatory to correctly specify
the filename of the main file as the argument of |\childdocmain|:
%
\begin{center}
\begin{tabular}{l}
|\input{childdoc.def}|\\
|\childdocmain{|\textit{main}|}|\\
\end{tabular}
\end{center}
%
If |\jobname| does not match the argument \textit{main} of |\childdocmain|,
it is assumed that |\jobname| points to the child file to be compiled.
When using |\childdocmain| with the main file specified as argument,
it suffices to start a child file
with just |\input{|\textit{main}|}|
without loading of the package and using |\childdocof|.
If instead all processing is done
with the appropriate \textsf{childdoc} directives,
the argument of \textit{main} of |\childdocmain| can be empty.

An alternative version of the command line processing described
in \secref{sec:commandline} using the detection mechanism reads:
%
\begin{center}
|... -jobname "|\textit{target}|" "|[\textit{flags}]%
[|\def\jobname{|\textit{dest}|}|]|\input{|\textit{main}|}"|
\end{center}

%%%%%%%%%%%%%%%%%%%%%%%%%%%%%%%%%%%%%%%%%%%%%%%%%%%%%%%%%%%%%%%%%%%%%%%%%%%%%%%%
\subsection{Manual Code}
\label{sec:manual}

In case one cannot be certain whether the definitions file |childdoc.def|
is installed on the target \TeX{} distribution
and one prefers not to ship it,
it is conceivable to paste a few relevant commands into the sources.

To that end, drop all statements |\input{childdoc.def}|
and perform the replacements as outlined below.
Instead of |\childdocmain{|\textit{main}|}| add the following code
to the top of the main file:
%
\begin{center}
\begin{tabular}{l}
|\||ifdefined\childdocname\endinput\||fi\newif\ifchilddoc|\\
|\edef\childdocname{\scantokens\expandafter{\jobname\noexpand}}|\\
|\def\childdocmain{|\textit{main}|}\||ifx\childdocmain\childdocname\||else|\\
|\childdoctrue\includeonly{\childdocname}\let\jobname\childdocmain\||fi|\\
\end{tabular}
\end{center}
%
Instead of |\childdocof{|\textit{main}|}| just include the main file
at the top of each child file:
%
\begin{center}
|\input{|\textit{main}|}|
\end{center}
%
A simple redirection |\childdocforward{|\textit{dest}|}| is achieved by:
%
\begin{center}
|\def\jobname{|\textit{dest}|}\input{\jobname}|
\end{center}
%
The redirection with prefix
|\childdocforwardprefix[|\textit{prefix}|]{|\textit{dest}|}|
is accomplished by:
%
\begin{center}
\begin{tabular}{l}
|{\edef\jobname{\scantokens\expandafter{\jobname\noexpand}}|\\
|\def\redirectjob |\textit{prefix}|#1~~~{\gdef\jobname{|\textit{dest}|#1}}|\\
|\expandafter\redirectjob\jobname~~~}\input{\jobname}|
\end{tabular}
\end{center}

In an alternative approach,
child documents can be compiled by a specific command line
without additional code or specific definitions:
%
\begin{center}
|... -jobname "|\textit{target}|" "|[\textit{flags}]%
|\includeonly{|\textit{dest}|}\input{|\textit{main}|}"|
\end{center}
%

%%%%%%%%%%%%%%%%%%%%%%%%%%%%%%%%%%%%%%%%%%%%%%%%%%%%%%%%%%%%%%%%%%%%%%%%%%%%%%%%
%%%%%%%%%%%%%%%%%%%%%%%%%%%%%%%%%%%%%%%%%%%%%%%%%%%%%%%%%%%%%%%%%%%%%%%%%%%%%%%%
\section{Information}

%%%%%%%%%%%%%%%%%%%%%%%%%%%%%%%%%%%%%%%%%%%%%%%%%%%%%%%%%%%%%%%%%%%%%%%%%%%%%%%%
\subsection{Copyright}

Copyright \copyright{} 2017--2018 Niklas Beisert

This work may be distributed and/or modified under the
conditions of the \LaTeX{} Project Public License, either version 1.3
of this license or (at your option) any later version.
The latest version of this license is in
  \url{http://www.latex-project.org/lppl.txt}
and version 1.3 or later is part of all distributions of \LaTeX{}
version 2005/12/01 or later.

This work has the LPPL maintenance status `maintained'.

The Current Maintainer of this work is Niklas Beisert.

This work consists of the files |README.txt|, |childdoc.ins| and |childdoc.dtx|
as well as the derived files |childdoc.def|, |cdocsamp.tex|
with |cdocsch1.tex|, |cdocsch2.tex|, |cdocspt3.tex|, |cdocspt4.tex|,
|cdocsdrf.tex|, |cdocsfn1.tex|, |cdocsfn2.tex|
as well as |childdoc.pdf|.

%%%%%%%%%%%%%%%%%%%%%%%%%%%%%%%%%%%%%%%%%%%%%%%%%%%%%%%%%%%%%%%%%%%%%%%%%%%%%%%%
\subsection{Files and Installation}

The package consists of the files:
%
\begin{center}
\begin{tabular}{ll}
    |README.txt|   & readme file \\
    |childdoc.ins| & installation file \\
    |childdoc.dtx| & source file \\
    |childdoc.def| & definition file \\
    |cdocsamp.tex| & sample main file \\
    |cdocsch1.tex| & sample include file \\
    |cdocsch2.tex| & sample include file \\
    |cdocspt3.tex| & sample part file \\
    |cdocspt4.tex| & sample part file \\
    |cdocsdrf.tex| & sample redirection file \\
    |cdocsfn1.tex| & sample redirection file \\
    |cdocsfn2.tex| & sample redirection file \\
    |childdoc.pdf| & manual
\end{tabular}
\end{center}
%
The distribution consists of the files
|README.txt|, |childdoc.ins| and |childdoc.dtx|.
%
\begin{itemize}
\item
Run (pdf)\LaTeX{} on |childdoc.dtx|
to compile the manual |childdoc.pdf| (this file).
\item
Run \LaTeX{} on |childdoc.ins| to create the definitions file |childdoc.def|
and the sample |cdocsamp.tex| with include files
|cdocsch1.tex|, |cdocsch2.tex|, |cdocspt3.tex|, |cdocspt4.tex|,
|cdocsdrf.tex|, |cdocsfn1.tex|, |cdocsfn2.tex|.
Then copy the file |childdoc.def| to an appropriate directory of your \LaTeX{}
distribution, e.g.\ \textit{texmf-root}|/tex/latex/childdoc|.
\end{itemize}

%%%%%%%%%%%%%%%%%%%%%%%%%%%%%%%%%%%%%%%%%%%%%%%%%%%%%%%%%%%%%%%%%%%%%%%%%%%%%%%%
\subsection{Related CTAN Packages}

There are several other packages which offer a similar functionality:
%
\begin{itemize}
\item
The packages
\href{http://ctan.org/pkg/docmute}{\textsf{docmute}},
\href{http://ctan.org/pkg/includex}{\textsf{includex}} and
\href{http://ctan.org/pkg/standalone}{\textsf{standalone}}
provide commands to include only the document body of
a child file thus allowing both files to be compiled individually.
\item
The packages \href{http://ctan.org/pkg/subdocs}{\textsf{subdocs}}
and \href{http://ctan.org/pkg/subfiles}{\textsf{subfiles}}
provide structures in which the main and child documents can be
encapsulated and allowing them to be compiled individually.
The inclusion mechanism is different from the conventional |\include|.
\item
The package \href{http://ctan.org/pkg/combine}{\textsf{combine}}
is an elaborate solution to combine several documents into one.
\end{itemize}
%
See also the CTAN topic \href{http://ctan.org/topic/subdocs}{\textsf{subdocs}}
for further related packages.
The present package differs from the above solutions in that
a document structure constructed with the conventional |\include| mechanism
just needs two extra commands at the top of every file
such that all constituent files can be compiled individually.

%%%%%%%%%%%%%%%%%%%%%%%%%%%%%%%%%%%%%%%%%%%%%%%%%%%%%%%%%%%%%%%%%%%%%%%%%%%%%%%%
%\subsection{Feature Suggestions}
%
%The following is a list of features which may be useful for future
%versions of this package:
%%
%\begin{itemize}
%\item
%\ldots
%\end{itemize}

%%%%%%%%%%%%%%%%%%%%%%%%%%%%%%%%%%%%%%%%%%%%%%%%%%%%%%%%%%%%%%%%%%%%%%%%%%%%%%%%
\subsection{Revision History}

%%%%%%%%%%%%%%%%%%%%%%%%%%%%%%%%%%%%%%%%
\paragraph{v2.0:} 2018/12/30

\begin{itemize}
\item
immediate forward processing
\item
added |\childdocby| mechanism
\item
manual restructured
\end{itemize}

%%%%%%%%%%%%%%%%%%%%%%%%%%%%%%%%%%%%%%%%
\paragraph{v1.6:} 2018/01/17

\begin{itemize}
\item
application for development of include files
\item
corrections to manual
\end{itemize}

%%%%%%%%%%%%%%%%%%%%%%%%%%%%%%%%%%%%%%%%
\paragraph{v1.5:} 2017/05/21

\begin{itemize}
\item
more complete structuring introduced
\item
|\childdocof| introduced
\item
|\childdoc| renamed to |\childdocmain|
\item
|\childredirect| renamed to |\childdocforward| and |\childdocforwardprefix|
and functionality expanded
\end{itemize}

%%%%%%%%%%%%%%%%%%%%%%%%%%%%%%%%%%%%%%%%
\paragraph{v1.0:} 2017/04/27

\begin{itemize}
\item
manual and install package
\item
first version published on CTAN
\end{itemize}

%%%%%%%%%%%%%%%%%%%%%%%%%%%%%%%%%%%%%%%%
\paragraph{v0.6:} 2017/04/26

\begin{itemize}
\item
redirection mechanism added
\end{itemize}

%%%%%%%%%%%%%%%%%%%%%%%%%%%%%%%%%%%%%%%%
\paragraph{v0.5:} 2017/04/26

\begin{itemize}
\item
functionality in definition file
\end{itemize}


%%%%%%%%%%%%%%%%%%%%%%%%%%%%%%%%%%%%%%%%%%%%%%%%%%%%%%%%%%%%%%%%%%%%%%%%%%%%%%%%
%%%%%%%%%%%%%%%%%%%%%%%%%%%%%%%%%%%%%%%%%%%%%%%%%%%%%%%%%%%%%%%%%%%%%%%%%%%%%%%%
%%%%%%%%%%%%%%%%%%%%%%%%%%%%%%%%%%%%%%%%%%%%%%%%%%%%%%%%%%%%%%%%%%%%%%%%%%%%%%%%
\appendix

\settowidth\MacroIndent{\rmfamily\scriptsize 000\ }

 \DocInput{childdoc.dtx}

\end{document}
%</driver>
% \fi
%
% %%%%%%%%%%%%%%%%%%%%%%%%%%%%%%%%%%%%%%%%%%%%%%%%%%%%%%%%%%%%%%%%%%%%%%%%%%%%%%
% %%%%%%%%%%%%%%%%%%%%%%%%%%%%%%%%%%%%%%%%%%%%%%%%%%%%%%%%%%%%%%%%%%%%%%%%%%%%%%
% \section{Sample}
%\iffalse
%<*samplemain>
%\fi
%
% The following presents a sample document
% with two chapters, two parts, a title page,
% a compile flag as well as three forwarding files to set the flag.
% It consists of eight |.tex| files:
% \begin{center}
% \begin{tabular}{ll}
% |cdocsamp.tex|&main file\\
% |cdocsch1.tex|&include file for chapter 1\\
% |cdocsch2.tex|&include file for chapter 2\\
% |cdocspt3.tex|&include file for part 3\\
% |cdocspt4.tex|&include file for part 4\\
% |cdocsdrf.tex|&forwarding file for main file in draft mode\\
% |cdocsfi1.tex|&forwarding file for final version of chapter 1\\
% |cdocsfi2.tex|&forwarding file for final version of chapter 2\\
% \end{tabular}
% \end{center}
% Each of the eight files can be compiled directly by the \LaTeX{} compiler.
%
% %%%%%%%%%%%%%%%%%%%%%%%%%%%%%%%%%%%%%%
% \paragraph{Main File.}
%
% The main file is called |cdocsamp.tex|.
%
% Load the \textsf{childdoc} definitions and
% declare the filename for the main document:
%    \begin{macrocode}
\input{childdoc.def}
\childdocmain{}
%    \end{macrocode}

% Optional override for |\version| flag:
%    \begin{macrocode}
%%\ifchilddoc\else\providecommand{\version}{draft}\fi
%    \end{macrocode}

% Define the default values for the |\version| flag
% (|final| for the main file and |draft| for childs):
%    \begin{macrocode}
\ifchilddoc
\providecommand{\version}{draft}
\else
\providecommand{\version}{final}
\fi
%    \end{macrocode}

% Load the standard document class:
%    \begin{macrocode}
\documentclass[12pt]{article}
%    \end{macrocode}

% Start the document body:
%    \begin{macrocode}
\begin{document}
%    \end{macrocode}

% Declare a title page.
% Print title, part of document being processed and version flag:
%    \begin{macrocode}
\addtocounter{page}{-1}
\begin{center}
{\LARGE\bfseries{}childdoc example\par}
\vspace{1cm}
\ifchilddoc
\ifchilddocmanual part\else chapter\fi:
`\childdocname' of `\childdocjob'\par
\else
main document: `\childdocjob'\par
\fi
version: \version\par
\end{center}
\newpage
%    \end{macrocode}

% Manually include selected file,
% otherwise process as usual:
%    \begin{macrocode}
\ifchilddocmanual
\section*{part `\childdocname'}
\input{\childdocname}
\else
%    \end{macrocode}

% Include the two chapters:
%    \begin{macrocode}
\include{cdocsch1}
\include{cdocsch2}
%    \end{macrocode}

% Include the two parts unless only chapters should be displayed:
%    \begin{macrocode}
\ifchilddoc\else
\section{part three}
\input{cdocspt3}
\section{part four}
\input{cdocspt4}
\fi
%    \end{macrocode}

% Process as usual until here:
%    \begin{macrocode}
\fi
%    \end{macrocode}

% End of document body:
%    \begin{macrocode}
\end{document}
%    \end{macrocode}
%\iffalse
%</samplemain>
%\fi
%
% %%%%%%%%%%%%%%%%%%%%%%%%%%%%%%%%%%%%%%
% \paragraph{Chapter Include Files.}
%
% The include files are called |cdocsch1.tex| and |cdocsch2.tex|.
%
%\iffalse
%<*samplechap1|samplechap2>
%\fi

% Optional override for |\version| flag:
%    \begin{macrocode}
%%\providecommand{\version}{final}
%    \end{macrocode}

% Include the main document:
%    \begin{macrocode}
\input{childdoc.def}
\childdocof{cdocsamp}
%    \end{macrocode}

%\iffalse
%</samplechap1|samplechap2>
%\fi
%
%\iffalse
%<*samplechap1>
%\fi
% Some text for chapter 1:
%    \begin{macrocode}
\section{one}
some text in chapter one
%    \end{macrocode}

%\iffalse
%</samplechap1>
%\fi
% Some text for chapter 2:
%\iffalse
%<*samplechap2>
%\fi
%    \begin{macrocode}
\section{two}
more text in chapter two
%    \end{macrocode}

%\iffalse
%</samplechap2>
%\fi
%
% %%%%%%%%%%%%%%%%%%%%%%%%%%%%%%%%%%%%%%
% \paragraph{Part Include Files.}
%
% The include files are called |cdocspt3.tex| and |cdocspt4.tex|.
%
%\iffalse
%<*samplepart3|samplepart4>
%\fi

% Optional override for |\version| flag:
%    \begin{macrocode}
%%\providecommand{\version}{final}
%    \end{macrocode}

% Include the main document:
%    \begin{macrocode}
\input{childdoc.def}
\childdocby{cdocsamp}
%    \end{macrocode}

%\iffalse
%</samplepart3|samplepart4>
%\fi
%
%\iffalse
%<*samplepart3>
%\fi
% Some text for part 3:
%    \begin{macrocode}
some text in part three
%    \end{macrocode}

%\iffalse
%</samplepart3>
%\fi
% Some text for part 4:
%\iffalse
%<*samplepart4>
%\fi
%    \begin{macrocode}
more text in part four
%    \end{macrocode}

%\iffalse
%</samplepart4>
%\fi
%
% %%%%%%%%%%%%%%%%%%%%%%%%%%%%%%%%%%%%%%
% \paragraph{Forwarding for a Complete Draft.}
%
% The following forwarding file |cdocsdrf.tex|
% compiles the main document in draft mode:
%\iffalse
%<*sampledraft>
%\fi
%    \begin{macrocode}
\def\version{draft}
\input{childdoc.def}
\childdocforward{cdocsamp}
%    \end{macrocode}

%\iffalse
%</sampledraft>
%\fi
%
% %%%%%%%%%%%%%%%%%%%%%%%%%%%%%%%%%%%%%%
% \paragraph{Forwarding for Final Version of the Chapters.}
%
% The following forwarding files |cdocsfn1.tex| and |cdocsfn2.tex|
% (with identical content)
% compile the final versions of the child documents
% |cdocsch1.tex| and |cdocsch2.tex|, respectively:
%\iffalse
%<*samplefinal>
%\fi
%    \begin{macrocode}
\def\version{final}
\input{childdoc.def}
\childdocforwardprefix[cdocsamp]{cdocsfn}{cdocsch}
%    \end{macrocode}

%\iffalse
%</samplefinal>
%\fi
%
% %%%%%%%%%%%%%%%%%%%%%%%%%%%%%%%%%%%%%%
% \paragraph{Command Line Processing.}
%
% The following three command lines generate the output files
% |cdocscld|, |cdocscl1| and |cdocscl2|
% which should be identical to
% |cdocsdrf|, |cdocsch1| and |cdocsfn2|, respectively:
% \begin{center}
% \begin{tabular}{l}
% |latex -jobname cdocscld \|\\
% |  "\def\version{draft}\input{childdoc.def}\childdocforward{cdocsamp}"|\\
% |latex -jobname cdocscl1 \|\\
% |  "\input{childdoc.def}\childdocforward[cdocsamp]{cdocsch1}"|\\
% |latex -jobname cdocscl2 \|\\
% |  "\def\version{final}\input{childdoc.def}\childdocforward{cdocsch2}"|
% \end{tabular}
% \end{center}
% Note that the trailing backslash on each first line
% merely continues the input to the second line
% (for convenient cut ant paste).
% Furthermore, the command |latex| can be replaced by any
% of its alternative versions such as |pdflatex|.
%
% %%%%%%%%%%%%%%%%%%%%%%%%%%%%%%%%%%%%%%%%%%%%%%%%%%%%%%%%%%%%%%%%%%%%%%%%%%%%%%
% %%%%%%%%%%%%%%%%%%%%%%%%%%%%%%%%%%%%%%%%%%%%%%%%%%%%%%%%%%%%%%%%%%%%%%%%%%%%%%
% \section{Implementation}
%\iffalse
%<*package>
%\fi
%
% This section describes the definitions file |childdoc.def|.

% The definitions cannot be loaded using |\usepackage| or |\RequirePackage|
% which has a mechanism to prevent loading a style file more than once.
% When loading the definitions by means of |\input|
% multiple instances have to be prevented manually:
%\iffalse
%This code needs to be before the `\ProvidesFile' directive
%which is defined at the beginning of this file.
%Therefore it is also placed there and commented out here.
%</package>
%<*discard>
%\fi
%    \begin{macrocode}
\ifdefined\childdocmain\endinput\fi
%    \end{macrocode}
%\iffalse
%</discard>
%<*package>
%\fi
%
% \macro{\ifchilddoc}
% \macro{\ifchilddocmanual}
% The conditional |\ifchilddoc| tells whether a
% child (true) or main (false) document is being compiled.
% The conditional |\ifchilddocmanual| tells whether
% the |\includeonly| mechanism is used (false) or
% the selection of child files must be performed manually (true).
% The definitions initialise to false:
%    \begin{macrocode}
\newif\ifchilddoc
\newif\ifchilddocmanual
%    \end{macrocode}

% \macro{\childdocname}
% \macro{\childdocjob}
% The macro |\childdocname| stores the name of the main document
% to be compiled. The macro |\childdocjob| stores the name of
% the document on which the \LaTeX{} compiler was originally invoked.
% The content of |\jobname| cannot be compared
% to filenames specified in the source due to different catcodes.
% The following code rescans |\jobname|, stores the result
% in |\childdocname| and saves a copy in |\childdocjob|:
%    \begin{macrocode}
\edef\childdocname{\scantokens\expandafter{\jobname\noexpand}}
\let\childdocjob\childdocname
%    \end{macrocode}

% \macro{\childdocdisable}
% The macro |\childdocdisable| prevents the main file
% from being processed more than once.
% At this stage, the main document command |\childdocmain|
% is assumed to be called once again where it should do nothing.
% Any subsequent call to it should prevent
% a secondary processing of the main document
% It overwrites the forwarding commands
% |\childdocof| and |\childdocforward|
% with empty macros to prevent further inclusions of the main document:
%    \begin{macrocode}
\newcommand{\childdocdisable}
{
  \renewcommand{\childdocmain}[1]{\renewcommand{\childdocmain}[1]{\endinput}}
  \renewcommand{\childdocof}[1]{}
  \renewcommand{\childdocby}[2][]{}
  \renewcommand{\childdocforward}[2][]{}
  \renewcommand{\childdocdisable}{}
}
%    \end{macrocode}

% \macro{\childdocmain}
% The macro |\childdocmain| is to be called at the top of the main file
% with nothing or the main filename (without extension) as argument.
% First, it breaks loops.
% If the argument is not empty and does not match |\childdocname|
% (which is set by the first inclusion of |childdoc.def|),
% |\ifchilddoc| is set to true, |\includeonly| is applied to the child file
% and |\jobname| is set to the main file
% (for proper handling of |.aux| files):
%    \begin{macrocode}
\newcommand{\childdocmain}[1]
{
  \childdocdisable\childdocmain{}
  \if?#1?\else
    \begingroup
      \def\childdoctmp{#1}
      \ifx\childdoctmp\childdocname
        \def\childdoctmp{}
      \else
        \def\childdoctmp
        {
          \childdoctrue
          \includeonly{\childdocname}
          \def\childdocjob{#1}
          \def\jobname{#1}
        }
      \fi
      \expandafter
    \endgroup
    \childdoctmp
  \fi
}
%    \end{macrocode}

% \macro{\childdocof}
% The command |\childdocof| redirects
% compilation to the main file |#1|.
%    \begin{macrocode}
\newcommand{\childdocof}[1]
{
  \childdocdisable
  \childdoctrue
  \includeonly{\childdocname}
  \def\jobname{#1}
  \def\childdocjob{#1}
  \input{#1}
}
%    \end{macrocode}

% \macro{\childdocby}
% The command |\childdocby| ....
%    \begin{macrocode}
\newcommand{\childdocby}[2][]
{
  \childdocdisable
  \childdoctrue
  \childdocmanualtrue
  \if?#1?\else
    \def\jobname{#2}
  \fi
  \def\childdocjob{#2}
  \input{#2}
  \endinput
}
%    \end{macrocode}

% \macro{\childdocforward}
% The command |\childdocforward| redirects
% compilation to the main file or
% (if the optional argument is given) a child file.
% Parameters are set as if the main file
% or a child file starting with |\childdocof| was compiled.
% Then compilation is handed over to the main file:
%    \begin{macrocode}
\newcommand{\childdocforward}[2][]
{
  \begingroup
    \if?#1?
      \def\childdoctmp
      {
        \def\childdocname{#2}
        \def\childdocjob{#2}
        \def\jobname{#2}
        \input{#2}
        \endinput
      }
    \else
      \def\childdoctmp
      {
        \childdocdisable
        \def\childdocname{#2}
        \childdoctrue
        \includeonly{#2}
        \def\childdocjob{#1}
        \def\jobname{#1}
        \input{#1}
        \endinput
      }
    \fi
    \expandafter
  \endgroup
  \childdoctmp
}
%    \end{macrocode}

% \macro{\childdocforwardprefix}
% The command |\childdocforwardprefix| redirects
% compilation to the main or a child file by means of a pattern.
% The prefix |#1| in the current filename is replaced by |#2|
% and the suffix of the current filename is kept
% (it is assumed that the filename does not contain the substring `|~~~|'
% which is used as a delimiter).
% Compilation is handed over to the new file by |\childdocforward|:
%    \begin{macrocode}
\newcommand{\childdocforwardprefix}[3][]
{
  \begingroup
    \def\childdocextract #2##1~~~{\def\childdoctmp{\childdocforward[#1]{#3##1}}}
    \expandafter\childdocextract\childdocname~~~
    \expandafter
  \endgroup
  \childdoctmp
}
%    \end{macrocode}

% \macro{\childdoc}
% The deprecated macro |\childdoc| is a legacy version of |\childdocmain|:
%    \begin{macrocode}
\newcommand{\childdoc}{\childdocmain}
%    \end{macrocode}

% \macro{\childdocredirect}
% The deprecated macro |\childdocredirect| is a legacy version
% of |\childdocforward| and |\childdocforwardprefix|:
%    \begin{macrocode}
\newcommand{\childdocredirect}[2][]
{
  \begingroup
    \if?#1?
      \def\childdoctmp{\childdocforward{#2}}
    \else
      \def\childdoctmp{\childdocforwardprefix{#1}{#2}}
    \fi
    \expandafter
  \endgroup
  \childdoctmp
}
%    \end{macrocode}

%\iffalse
%</package>
%\fi
%
\endinput
|\\
|\childdocof{|\textit{main}|}|\\
\end{tabular}
\end{center}
at the top of every child file \textit{child}
which is included by |\include{|\textit{child}|}|
from within the main file
(or at least for those files to be compiled individually).
The argument \textit{main} must be the filename of the main file.

There are a couple of
considerations in setting up the main and child documents:

%%%%%%%%%%%%%%%%%%%%%%%%%%%%%%%%%%%%%%%%
\paragraph{Restrictions.}

Please note the following restrictions:
\begin{itemize}
\item
|\childdocmain| must be called with one argument \textit{main}
to ensure compatibility with earlier version of the package.
It must either be empty (|\childdocmain{}|)
or precisely match the filename of the main file in which it is specified.
See \secref{sec:detection} for further information.
\item
The filename \textit{main} must be specified without the |.tex| extension.
\item
The filename \textit{main} is case sensitive
(even in case-insensitive file systems)
due to internal string comparison.
\item
The argument \textit{main} should be fully expanded, it cannot be a macro.
\item
Subdirectories and special characters should be avoided in filenames.
\item
The command |\childdocmain{|\textit{main}|}| must be followed by a whitespace.
It should not be followed immediately by another command
or by a comment mark `|%|'.
This is because the \TeX{} parser reads the token immediately following
the argument of |\childdocmain| and puts it
at the beginning of every child section;
however, a white\-space is ignored.
\end{itemize}

%%%%%%%%%%%%%%%%%%%%%%%%%%%%%%%%%%%%%%%%
\paragraph{Content of Main File.}

It is advisable to place all content in the child files included by |\include|.
Any output contained in the main file will appear in all child documents
unless suppressed manually;
it cannot be suppressed automatically by the |\includeonly| directive
and thus should normally be avoided.
A method to include some content in the main file
by means of conditional processing is described in \secref{sec:conditional}.

%%%%%%%%%%%%%%%%%%%%%%%%%%%%%%%%%%%%%%%%
\paragraph{Page Numbering.}

When only a part of the document is compiled,
the appropriate numbering of pages
(as well as other status parameters)
is determined from the |.aux| files.
The latter contain information from previous passes.
However this information needs to propagate through
all intermediate child documents.
Therefore the page numbering in child documents may well
be inconsistent until the complete document is compiled at least once.

A useful (if unconventional) way to always ensure a consistent
page numbering is to restart the numbering in each child document
and denote the pages by `\textit{child}|.|\textit{page}'
where \textit{child} represents the chapter/section number of the child file.
This can be achieved by the command
|\numberwithin{page}{|\textit{child}|}|
of the \textsf{amsmath} package
where \textit{child} can be |chapter| or |section|
depending on the chosen structuring.
Alternatively, one can modify the macro |\thepage| appropriately
and reset the counter |page| at the start of each child file.

%%%%%%%%%%%%%%%%%%%%%%%%%%%%%%%%%%%%%%%%%%%%%%%%%%%%%%%%%%%%%%%%%%%%%%%%%%%%%%%%
\subsection{Conditional Processing}
\label{sec:conditional}

The package provides a mechanism to compile different versions
of a document. To customise the versions further some conditional processing
can come in handy to distinguish which version is being compiled.
The package provides two macros to describe the compilation context:

%%%%%%%%%%%%%%%%%%%%%%%%%%%%%%%%%%%%%%%%
\DescribeMacro{\ifchilddoc}
The conditional |\ifchilddoc| distinguishes between the compilation of
child documents and the main document:
%
\begin{center}
|\ifchilddoc |\textit{child-code}| |[|\||else |\textit{main-code}]| \||fi|
\end{center}

%%%%%%%%%%%%%%%%%%%%%%%%%%%%%%%%%%%%%%%%
\DescribeMacro{\childdocname}
\DescribeMacro{\childdocjob}
The macro |\childdocname| contains the filename (without extension)
of the main or child file being processed.
Note that |\childdocjob| will always contain the name of the main file.

%%%%%%%%%%%%%%%%%%%%%%%%%%%%%%%%%%%%%%%%
\paragraph{Title Page.}

Conditional processing can be used to include a title or banner page
in the main document when proper precautions are taken.
Importantly, the code in the main file should ensure that the page counter
(as well as other status parameters which are stored in the |.aux| files)
takes the same value after the conditional processing.
Otherwise the page numbers may take divergent values
depending on which part is compiled.

For example, a title page could be declared by:
%
\begin{center}
\begin{tabular}{l}
|\ifchilddoc\||else|\\
|\addtocounter{page}{-1}|\\
\textit{code for title page}\\
|\newpage|\\
|\||fi|
\end{tabular}
\end{center}
%
A banner page for the child documents can be generated by:
%
\begin{center}
\begin{tabular}{l}
|\ifchilddoc|\\
|\addtocounter{page}{-1}|\\
\textit{code for banner page}\\
|\newpage|\\
|\||fi|
\end{tabular}
\end{center}
%
Here one could write a message such as:
\begin{center}
|This is the part \childdocname{} of \childdocjob{}.|
\end{center}

%%%%%%%%%%%%%%%%%%%%%%%%%%%%%%%%%%%%%%%%%%%%%%%%%%%%%%%%%%%%%%%%%%%%%%%%%%%%%%%%
\subsection{Flags}
\label{sec:flags}

The package makes it easy to generate different versions
of the main or child documents.
To this end compilation flags can be defined
and assigned different default values.
They will be particularly useful in conjunction
with the forwarding mechanism described in \secref{sec:forward}.

For example, it may be useful to have a flag |\version|
which can be set to |draft| or |final|.
The document source will contain some conditional code
depending on the value of |\version|.
Suppose further, the flag should default to |final| for the main file
and to |draft| for child files
which is a natural assignment for editing the document.
This is achieved by placing the following code
in the preamble of the main document
(below the |\childdocmain| directive):
%
\begin{center}
\begin{tabular}{l}
|\ifchilddoc|\\
|\providecommand{\version}{draft}|\\
|\||else|\\
|\providecommand{\version}{final}|\\
|\||fi|
\end{tabular}
\end{center}
%
The definition by |\providecommand| makes sure
that previous definitions are not overwritten.
Further statements |\providecommand{\version}{...}|
can thus be added before the above code to override it.

For the main file, one might add a line
(between |\childdocmain| and the above block)
%
\begin{center}
|%\ifchilddoc\||else\providecommand{\version}{draft}\||fi|
\end{center}
%
which can be uncommented to produce a draft version.
Likewise one can add a line to the very top of a child file
(above the |\childdocof{|\textit{main}|}| directive)
%
\begin{center}
|%\providecommand{\version}{final}|
\end{center}
%
which can be uncommented to produce the final version of this child document.

%%%%%%%%%%%%%%%%%%%%%%%%%%%%%%%%%%%%%%%%%%%%%%%%%%%%%%%%%%%%%%%%%%%%%%%%%%%%%%%%
\subsection{Forwarding}
\label{sec:forward}

Different versions of the main or child documents
using compilation flags as described in \secref{sec:flags}
can be (permanently) stored in different files
for convenient compilation, viewing and distribution.
To this end, the package defines a command
to pass on compilation to a different file:

%%%%%%%%%%%%%%%%%%%%%%%%%%%%%%%%%%%%%%%%
\DescribeMacro{\childdocforward}
The command |\childdocforward| redirects processing to
another source file:
%
\begin{center}
\begin{tabular}{l}
|% \iffalse
%
% childdoc.dtx Copyright (C) 2017-2018 Niklas Beisert
%
% This work may be distributed and/or modified under the
% conditions of the LaTeX Project Public License, either version 1.3
% of this license or (at your option) any later version.
% The latest version of this license is in
%   http://www.latex-project.org/lppl.txt
% and version 1.3 or later is part of all distributions of LaTeX
% version 2005/12/01 or later.
%
% This work has the LPPL maintenance status `maintained'.
%
% The Current Maintainer of this work is Niklas Beisert.
%
% This work consists of the files childdoc.dtx and childdoc.ins
% and the derived files childdoc.def and cdocsamp.tex with
% cdocsch1.tex, cdocsch2.tex, cdocsdrf.tex, cdocsfn1.tex, cdocsfn2.tex.
%
%<package>\ifdefined\childdocmain\endinput\fi
%<package>\ProvidesFile{childdoc.def}[2018/12/30 v2.0 child document driver]
%<samplemain>\ProvidesFile{cdocsamp.tex}[2018/12/30 v2.0 sample for childdoc]
%<*driver>
%\ProvidesFile{childdoc.drv}[2018/12/30 v2.0 childdoc reference manual file]
\PassOptionsToClass{10pt,a4paper}{article}
\documentclass{ltxdoc}

\usepackage[margin=35mm]{geometry}
\usepackage{hyperref}
\usepackage{hyperxmp}
\usepackage[usenames]{color}

\hypersetup{colorlinks=true}
\hypersetup{pdfstartview=FitH}
\hypersetup{pdfpagemode=UseNone}
\hypersetup{pdfsource={}}
\hypersetup{pdflang={en-UK}}
\hypersetup{pdfcopyright={Copyright 2017-2018 Niklas Beisert.
  This work may be distributed and/or modified under the
  conditions of the LaTeX Project Public License, either version 1.3
  of this license or (at your option) any later version.}}
\hypersetup{pdflicenseurl={http://www.latex-project.org/lppl.txt}}
\hypersetup{pdfcontactaddress={ETH Zurich, ITP, HIT K,
  Wolfgang-Pauli-Strasse 27}}
\hypersetup{pdfcontactpostcode={8093}}
\hypersetup{pdfcontactcity={Zurich}}
\hypersetup{pdfcontactcountry={Switzerland}}
\hypersetup{pdfcontactemail={nbeisert@itp.phys.ethz.ch}}
\hypersetup{pdfcontacturl={http://people.phys.ethz.ch/\xmptilde nbeisert/}}

\newcommand{\secref}[1]{\hyperref[#1]{section \ref*{#1}}}

\parskip1ex
\parindent0pt
\let\olditemize\itemize
\def\itemize{\olditemize\parskip0pt}

\begin{document}

\title{The \textsf{childdoc} Package}
\hypersetup{pdftitle={The childdoc Package}}
\author{Niklas Beisert\\[2ex]
  Institut f\"ur Theoretische Physik\\
  Eidgen\"ossische Technische Hochschule Z\"urich\\
  Wolfgang-Pauli-Strasse 27, 8093 Z\"urich, Switzerland\\[1ex]
  \href{mailto:nbeisert@itp.phys.ethz.ch}
  {\texttt{nbeisert@itp.phys.ethz.ch}}}
\hypersetup{pdfauthor={Niklas Beisert}}
\hypersetup{pdfsubject={Manual for the LaTeX2e Package childdoc}}
\date{30 December 2018, \textsf{v2.0}}
\maketitle

\begin{abstract}\noindent
\textsf{childdoc} is a \LaTeXe{} package
that enables the direct compilation
of document sections included by |\include|
to individual files.
\end{abstract}

\begingroup
\parskip0ex
\tableofcontents
\endgroup

%%%%%%%%%%%%%%%%%%%%%%%%%%%%%%%%%%%%%%%%%%%%%%%%%%%%%%%%%%%%%%%%%%%%%%%%%%%%%%%%
%%%%%%%%%%%%%%%%%%%%%%%%%%%%%%%%%%%%%%%%%%%%%%%%%%%%%%%%%%%%%%%%%%%%%%%%%%%%%%%%
\section{Introduction}

\LaTeX{} provides a mechanism to structure a large document (such as a book)
into a main file and several child files (containing the chapters)
using the |\include| command.
This mechanism is beneficial for documents
which span hundreds of pages in order to
make the source file(s) more manageable.
Moreover, compilation can be restricted to
selected child files by means of the |\includeonly| command.
The latter feature can be used to reduce the compilation time while editing
(this was significantly more useful in the earlier days of \LaTeX{})
or to generate a smaller document which is easier to navigate.
Another application of |\includeonly| is to generate
documents consisting of selected parts of the complete document.

However, there are a few drawbacks of the plain |\include| mechanism:
\begin{itemize}
\item
The child files cannot be compiled on their own,
they can only be compiled via the main file.
A naive editing environment
(such as a text editor with an option
to have the current file processed by \LaTeX)
may require one to switch to the main file before compiling;
attempting to compile the child file produces errors.
\item
The main file must be modified (each time)
to adjust the |\includeonly| command
to the present needs. This easily leaves the main file in a messy state.
\item
The generated document will always carry the filename
of the main document. This is inconvenient if
several child files are to be compiled and
to be kept for distribution.
\end{itemize}

The present package provides a simple interface
to make child files individually compilable by \LaTeX{}.
Compiling a child file then has the same effect as compiling
the main file with an |\includeonly| command
to select the appropriate child.
Moreover the generated document will carry the name of the child
rather than the main file.
This resolves all three above issues.

This feature is meant to make the editing of books,
thesis documents and lecture notes somewhat more convenient.
However, the package can also be used efficiently for
composing a series of documents (such as exercise sheets)
which are typically distributed individually.
It then assists the author in generating the individual documents
(potentially in different versions)
as well as a document containing the collected series.
Another application is in developing style files
or other kinds of included material
where compilation of the style file could redirect
to a sample or test file.

%%%%%%%%%%%%%%%%%%%%%%%%%%%%%%%%%%%%%%%%%%%%%%%%%%%%%%%%%%%%%%%%%%%%%%%%%%%%%%%%
%%%%%%%%%%%%%%%%%%%%%%%%%%%%%%%%%%%%%%%%%%%%%%%%%%%%%%%%%%%%%%%%%%%%%%%%%%%%%%%%
\section{Usage}

First of all, the package \textsf{childdoc} is \emph{not} a standard
\LaTeXe{} |.sty| style file! Therefore it needs to be invoked in
a non-standard way.

%%%%%%%%%%%%%%%%%%%%%%%%%%%%%%%%%%%%%%%%%%%%%%%%%%%%%%%%%%%%%%%%%%%%%%%%%%%%%%%%
\subsection{Included Files}
\label{sec:include}

%%%%%%%%%%%%%%%%%%%%%%%%%%%%%%%%%%%%%%%%
\DescribeMacro{\childdocmain}
To use the package, add the commands
\begin{center}
\begin{tabular}{l}
|\input{childdoc.def}|\\
|\childdocmain{}|\\
\end{tabular}
\end{center}
at the very top of the main \LaTeX{} file,
in particular \emph{before} the |\documentclass| statement!
The argument of |\childdocmain| should be left empty
(but it must be present).

%%%%%%%%%%%%%%%%%%%%%%%%%%%%%%%%%%%%%%%%
\DescribeMacro{\childdocof}
Furthermore, add the commands
\begin{center}
\begin{tabular}{l}
|\input{childdoc.def}|\\
|\childdocof{|\textit{main}|}|\\
\end{tabular}
\end{center}
at the top of every child file \textit{child}
which is included by |\include{|\textit{child}|}|
from within the main file
(or at least for those files to be compiled individually).
The argument \textit{main} must be the filename of the main file.

There are a couple of
considerations in setting up the main and child documents:

%%%%%%%%%%%%%%%%%%%%%%%%%%%%%%%%%%%%%%%%
\paragraph{Restrictions.}

Please note the following restrictions:
\begin{itemize}
\item
|\childdocmain| must be called with one argument \textit{main}
to ensure compatibility with earlier version of the package.
It must either be empty (|\childdocmain{}|)
or precisely match the filename of the main file in which it is specified.
See \secref{sec:detection} for further information.
\item
The filename \textit{main} must be specified without the |.tex| extension.
\item
The filename \textit{main} is case sensitive
(even in case-insensitive file systems)
due to internal string comparison.
\item
The argument \textit{main} should be fully expanded, it cannot be a macro.
\item
Subdirectories and special characters should be avoided in filenames.
\item
The command |\childdocmain{|\textit{main}|}| must be followed by a whitespace.
It should not be followed immediately by another command
or by a comment mark `|%|'.
This is because the \TeX{} parser reads the token immediately following
the argument of |\childdocmain| and puts it
at the beginning of every child section;
however, a white\-space is ignored.
\end{itemize}

%%%%%%%%%%%%%%%%%%%%%%%%%%%%%%%%%%%%%%%%
\paragraph{Content of Main File.}

It is advisable to place all content in the child files included by |\include|.
Any output contained in the main file will appear in all child documents
unless suppressed manually;
it cannot be suppressed automatically by the |\includeonly| directive
and thus should normally be avoided.
A method to include some content in the main file
by means of conditional processing is described in \secref{sec:conditional}.

%%%%%%%%%%%%%%%%%%%%%%%%%%%%%%%%%%%%%%%%
\paragraph{Page Numbering.}

When only a part of the document is compiled,
the appropriate numbering of pages
(as well as other status parameters)
is determined from the |.aux| files.
The latter contain information from previous passes.
However this information needs to propagate through
all intermediate child documents.
Therefore the page numbering in child documents may well
be inconsistent until the complete document is compiled at least once.

A useful (if unconventional) way to always ensure a consistent
page numbering is to restart the numbering in each child document
and denote the pages by `\textit{child}|.|\textit{page}'
where \textit{child} represents the chapter/section number of the child file.
This can be achieved by the command
|\numberwithin{page}{|\textit{child}|}|
of the \textsf{amsmath} package
where \textit{child} can be |chapter| or |section|
depending on the chosen structuring.
Alternatively, one can modify the macro |\thepage| appropriately
and reset the counter |page| at the start of each child file.

%%%%%%%%%%%%%%%%%%%%%%%%%%%%%%%%%%%%%%%%%%%%%%%%%%%%%%%%%%%%%%%%%%%%%%%%%%%%%%%%
\subsection{Conditional Processing}
\label{sec:conditional}

The package provides a mechanism to compile different versions
of a document. To customise the versions further some conditional processing
can come in handy to distinguish which version is being compiled.
The package provides two macros to describe the compilation context:

%%%%%%%%%%%%%%%%%%%%%%%%%%%%%%%%%%%%%%%%
\DescribeMacro{\ifchilddoc}
The conditional |\ifchilddoc| distinguishes between the compilation of
child documents and the main document:
%
\begin{center}
|\ifchilddoc |\textit{child-code}| |[|\||else |\textit{main-code}]| \||fi|
\end{center}

%%%%%%%%%%%%%%%%%%%%%%%%%%%%%%%%%%%%%%%%
\DescribeMacro{\childdocname}
\DescribeMacro{\childdocjob}
The macro |\childdocname| contains the filename (without extension)
of the main or child file being processed.
Note that |\childdocjob| will always contain the name of the main file.

%%%%%%%%%%%%%%%%%%%%%%%%%%%%%%%%%%%%%%%%
\paragraph{Title Page.}

Conditional processing can be used to include a title or banner page
in the main document when proper precautions are taken.
Importantly, the code in the main file should ensure that the page counter
(as well as other status parameters which are stored in the |.aux| files)
takes the same value after the conditional processing.
Otherwise the page numbers may take divergent values
depending on which part is compiled.

For example, a title page could be declared by:
%
\begin{center}
\begin{tabular}{l}
|\ifchilddoc\||else|\\
|\addtocounter{page}{-1}|\\
\textit{code for title page}\\
|\newpage|\\
|\||fi|
\end{tabular}
\end{center}
%
A banner page for the child documents can be generated by:
%
\begin{center}
\begin{tabular}{l}
|\ifchilddoc|\\
|\addtocounter{page}{-1}|\\
\textit{code for banner page}\\
|\newpage|\\
|\||fi|
\end{tabular}
\end{center}
%
Here one could write a message such as:
\begin{center}
|This is the part \childdocname{} of \childdocjob{}.|
\end{center}

%%%%%%%%%%%%%%%%%%%%%%%%%%%%%%%%%%%%%%%%%%%%%%%%%%%%%%%%%%%%%%%%%%%%%%%%%%%%%%%%
\subsection{Flags}
\label{sec:flags}

The package makes it easy to generate different versions
of the main or child documents.
To this end compilation flags can be defined
and assigned different default values.
They will be particularly useful in conjunction
with the forwarding mechanism described in \secref{sec:forward}.

For example, it may be useful to have a flag |\version|
which can be set to |draft| or |final|.
The document source will contain some conditional code
depending on the value of |\version|.
Suppose further, the flag should default to |final| for the main file
and to |draft| for child files
which is a natural assignment for editing the document.
This is achieved by placing the following code
in the preamble of the main document
(below the |\childdocmain| directive):
%
\begin{center}
\begin{tabular}{l}
|\ifchilddoc|\\
|\providecommand{\version}{draft}|\\
|\||else|\\
|\providecommand{\version}{final}|\\
|\||fi|
\end{tabular}
\end{center}
%
The definition by |\providecommand| makes sure
that previous definitions are not overwritten.
Further statements |\providecommand{\version}{...}|
can thus be added before the above code to override it.

For the main file, one might add a line
(between |\childdocmain| and the above block)
%
\begin{center}
|%\ifchilddoc\||else\providecommand{\version}{draft}\||fi|
\end{center}
%
which can be uncommented to produce a draft version.
Likewise one can add a line to the very top of a child file
(above the |\childdocof{|\textit{main}|}| directive)
%
\begin{center}
|%\providecommand{\version}{final}|
\end{center}
%
which can be uncommented to produce the final version of this child document.

%%%%%%%%%%%%%%%%%%%%%%%%%%%%%%%%%%%%%%%%%%%%%%%%%%%%%%%%%%%%%%%%%%%%%%%%%%%%%%%%
\subsection{Forwarding}
\label{sec:forward}

Different versions of the main or child documents
using compilation flags as described in \secref{sec:flags}
can be (permanently) stored in different files
for convenient compilation, viewing and distribution.
To this end, the package defines a command
to pass on compilation to a different file:

%%%%%%%%%%%%%%%%%%%%%%%%%%%%%%%%%%%%%%%%
\DescribeMacro{\childdocforward}
The command |\childdocforward| redirects processing to
another source file:
%
\begin{center}
\begin{tabular}{l}
|\input{childdoc.def}|\\
|\childdocforward[|\textit{main}|]{|\textit{dest}|}|\\
\end{tabular}
\end{center}
%
The argument \textit{dest} is the destination file
(without extension).
It should be the main file or one of the child files.
Note that further \textsf{childdoc} directives
such as |\childdocof| and |\childdocforward|
in the indicated file will be processed in this form.
The optional argument \textit{main}
passes on directly to the main file \textit{main}
while pretending to compile the child \textit{dest}.
This form behaves as if \textit{dest}
issues |\childdocof{|\textit{main}|}| right away,
and no further \textsf{childdoc} directives will be processed.

%%%%%%%%%%%%%%%%%%%%%%%%%%%%%%%%%%%%%%%%
\DescribeMacro{\...prefix}
In the alternative form |\childdocforwardprefix|,
%
\begin{center}
\begin{tabular}{l}
|\input{childdoc.def}|\\
|\childdocforwardprefix[|\textit{main}|]{|\textit{prefix}|}{|\textit{dest}|}|
\end{tabular}
\end{center}
%
the destination file is determined by a pattern
depending on the current file:
To make this work, the current file must be called
`{\textit{prefix}\hspace{0.2em}\textit{suffix}}'
with \textit{prefix} matching precisely the argument.
Processing is then passed on to the file
`{\textit{dest}\hspace{0.2em}\textit{suffix}}'.
Surely, the same effect is achieved by
directly specifying the
argument `{\textit{dest}\hspace{0.2em}\textit{suffix}}'
in the first form.
However, that requires to set up a different file
for each child. With the alternative form of the command
all these files can have exactly the same content
which simplifies setting them up and maintaining them.

For example, the following file |draft.tex|
with a compilation flag |\version| as described in \secref{sec:flags}
compiles the main document as a draft:
%
\begin{center}
\begin{tabular}{l}
|\def\version{draft}|\\
|\input{childdoc.def}|\\
|\childdocforward{|\textit{main}|}|
\end{tabular}
\end{center}
%
Likewise, the following files |final|\textit{nn}|.tex|
compile the final version of the child document
|child|\textit{nn}|.tex|:
%
\begin{center}
\begin{tabular}{l}
|\def\version{final}|\\
|\input{childdoc.def}|\\
|\childdocforwardprefix{final}{child}|
\end{tabular}
\end{center}
%

Note that when several versions of a main file and/or of each child file
are to be generated, it may be convenient to set up a |Makefile| or
shell script to automatise the process.

%%%%%%%%%%%%%%%%%%%%%%%%%%%%%%%%%%%%%%%%%%%%%%%%%%%%%%%%%%%%%%%%%%%%%%%%%%%%%%%%
\subsection{Command Line Processing}
\label{sec:commandline}

The effect of redirection files can also be achieved by invoking
the \LaTeX{} compiler with a more elaborate command line.
Most conveniently this should be done as part
of a shell script or a |Makefile|.

When using \textsf{childdoc} in the main file, the following
command lines effectively perform a redirection
(note that depending on the shell being used,
backslashes may have to be doubled: `|\|' $\to$ `|\\|'):
%
\begin{center}
|... -jobname "|\textit{target}|" |\\|"|[\textit{flags}]%
|\input{childdoc.def}\childdocforward[|\textit{main}|]{|\textit{dest}|}"|
\end{center}
%
Here \textit{target} is the name of the output file,
\textit{main} is the name of the main file
and \textit{dest} is the name of the main or child file to be processed
(all filenames without extensions).
The optional argument \textit{main} can be omitted
if \textit{main} matches \textit{dest}.
Optionally, compilation \textit{flags} can be defined via |\def| commands.
This command line makes the \TeX{} engine believe
it is compiling the file \textit{target}
whose content is specified as the latter parameter.
The provided code then forwards the processing to
\textit{main} or \textit{dest} as described in \secref{sec:forward}.

%%%%%%%%%%%%%%%%%%%%%%%%%%%%%%%%%%%%%%%%%%%%%%%%%%%%%%%%%%%%%%%%%%%%%%%%%%%%%%%%
\subsection{Include by Input}
\label{sec:input}

Including child documents by |\include| has some restrictions by design.
Most notably, the content of a child document always occupies
its own set of pages; pages cannot be shared between child documents.
Usually, this behaviour makes perfect sense
because each child document contain an essential part of the document.
However, in some situations it may be desirable to compose
a document from a collection of parts
without having mandatory page breaks between then.
For this case, the package
provides a mechanism to include parts
by |\input| which can also be processed individually.
However, by construction this mechanism
requires manual handling of the content to be output.

%%%%%%%%%%%%%%%%%%%%%%%%%%%%%%%%%%%%%%%%
\DescribeMacro{\ifchilddocmanual}
The main file should be prepared as usual, see \secref{sec:include}.
However, the document body must make a distinction
between processing of an individual part and of the main document, e.g.:
%
\begin{center}
\begin{tabular}{l}
|\ifchilddocmanual|\\
|\input{\childdocname}|\\
|\||else|\\
\textit{document body with }|\input{|\textit{part}|}|\\
|\||fi|
\end{tabular}
\end{center}
%
The conditional |\ifchilddocmanual| is true whenever
a part to be included by |\input| is being compiled,
and the name of the part is stored in |\childdocname|.

%%%%%%%%%%%%%%%%%%%%%%%%%%%%%%%%%%%%%%%%
\DescribeMacro{\childdocby}
Each part to be included by |\input| should start with:
%
\begin{center}
\begin{tabular}{l}
|\input{childdoc.def}|\\
|\childdocby{|\textit{main}|}|\\
\end{tabular}
\end{center}
%
The directive |\childdocby| is similar to |\childdocof|
described in \secref{sec:include},
but the subsequent selection of content must be done manually.
To that end, both |\ifchilddoc| and |\ifchilddocmanual|
will be true upon processing of a part,
and the name of the part is stored in |\childdocname|.
Note that |\jobname| will be set to the filename of the current part
so that each part receives an individual |.aux| file
that does not interfere with the |.aux| file(s) of the main document.
This behaviour can be altered by the alternative form
|\childdocby[*]{|\textit{main}|}| (with a non-empty optional argument)
which uses the |.aux| file of the main document
by setting |\jobname| to \textit{main}.

%%%%%%%%%%%%%%%%%%%%%%%%%%%%%%%%%%%%%%%%%%%%%%%%%%%%%%%%%%%%%%%%%%%%%%%%%%%%%%%%
\subsection{Driver Development}
\label{sec:driver}

The \textsf{childdoc} mechanism can also be use for the development
of definition files such as \LaTeX{} styles or classes.
This case differs from the above setup with multiple parts
included by |\include| in that no |\includeonly| should be invoked.
This can be achieved by starting the include file
(before |\ProvidesPackage|) with:
%
\begin{center}
\begin{tabular}{l}
|\input{childdoc.def}|\\
|\childdocforward{|\textit{main}|}|\\
\end{tabular}
\end{center}
%
or alternatively with:
%
\begin{center}
\begin{tabular}{l}
|\input{childdoc.def}|\\
|\childdocby{|\textit{main}|}|\\
\end{tabular}
\end{center}
%
Both forms have slightly different effects as described above.
The main file is prepared as usual, see \secref{sec:include}.

%%%%%%%%%%%%%%%%%%%%%%%%%%%%%%%%%%%%%%%%%%%%%%%%%%%%%%%%%%%%%%%%%%%%%%%%%%%%%%%%
\subsection{Legacy Detection}
\label{sec:detection}

The directive |\childdocmain| in the main file can detect
whether the complete document or merely a child is to be compiled
even without using the directive |\childdocof|.
This method is deprecated because it is less robust
and there is no compelling reason to use it;
it is merely provided for backward compatibility
and it may be removed in future versions.

If the detection mechanism is to be used,
it is mandatory to correctly specify
the filename of the main file as the argument of |\childdocmain|:
%
\begin{center}
\begin{tabular}{l}
|\input{childdoc.def}|\\
|\childdocmain{|\textit{main}|}|\\
\end{tabular}
\end{center}
%
If |\jobname| does not match the argument \textit{main} of |\childdocmain|,
it is assumed that |\jobname| points to the child file to be compiled.
When using |\childdocmain| with the main file specified as argument,
it suffices to start a child file
with just |\input{|\textit{main}|}|
without loading of the package and using |\childdocof|.
If instead all processing is done
with the appropriate \textsf{childdoc} directives,
the argument of \textit{main} of |\childdocmain| can be empty.

An alternative version of the command line processing described
in \secref{sec:commandline} using the detection mechanism reads:
%
\begin{center}
|... -jobname "|\textit{target}|" "|[\textit{flags}]%
[|\def\jobname{|\textit{dest}|}|]|\input{|\textit{main}|}"|
\end{center}

%%%%%%%%%%%%%%%%%%%%%%%%%%%%%%%%%%%%%%%%%%%%%%%%%%%%%%%%%%%%%%%%%%%%%%%%%%%%%%%%
\subsection{Manual Code}
\label{sec:manual}

In case one cannot be certain whether the definitions file |childdoc.def|
is installed on the target \TeX{} distribution
and one prefers not to ship it,
it is conceivable to paste a few relevant commands into the sources.

To that end, drop all statements |\input{childdoc.def}|
and perform the replacements as outlined below.
Instead of |\childdocmain{|\textit{main}|}| add the following code
to the top of the main file:
%
\begin{center}
\begin{tabular}{l}
|\||ifdefined\childdocname\endinput\||fi\newif\ifchilddoc|\\
|\edef\childdocname{\scantokens\expandafter{\jobname\noexpand}}|\\
|\def\childdocmain{|\textit{main}|}\||ifx\childdocmain\childdocname\||else|\\
|\childdoctrue\includeonly{\childdocname}\let\jobname\childdocmain\||fi|\\
\end{tabular}
\end{center}
%
Instead of |\childdocof{|\textit{main}|}| just include the main file
at the top of each child file:
%
\begin{center}
|\input{|\textit{main}|}|
\end{center}
%
A simple redirection |\childdocforward{|\textit{dest}|}| is achieved by:
%
\begin{center}
|\def\jobname{|\textit{dest}|}\input{\jobname}|
\end{center}
%
The redirection with prefix
|\childdocforwardprefix[|\textit{prefix}|]{|\textit{dest}|}|
is accomplished by:
%
\begin{center}
\begin{tabular}{l}
|{\edef\jobname{\scantokens\expandafter{\jobname\noexpand}}|\\
|\def\redirectjob |\textit{prefix}|#1~~~{\gdef\jobname{|\textit{dest}|#1}}|\\
|\expandafter\redirectjob\jobname~~~}\input{\jobname}|
\end{tabular}
\end{center}

In an alternative approach,
child documents can be compiled by a specific command line
without additional code or specific definitions:
%
\begin{center}
|... -jobname "|\textit{target}|" "|[\textit{flags}]%
|\includeonly{|\textit{dest}|}\input{|\textit{main}|}"|
\end{center}
%

%%%%%%%%%%%%%%%%%%%%%%%%%%%%%%%%%%%%%%%%%%%%%%%%%%%%%%%%%%%%%%%%%%%%%%%%%%%%%%%%
%%%%%%%%%%%%%%%%%%%%%%%%%%%%%%%%%%%%%%%%%%%%%%%%%%%%%%%%%%%%%%%%%%%%%%%%%%%%%%%%
\section{Information}

%%%%%%%%%%%%%%%%%%%%%%%%%%%%%%%%%%%%%%%%%%%%%%%%%%%%%%%%%%%%%%%%%%%%%%%%%%%%%%%%
\subsection{Copyright}

Copyright \copyright{} 2017--2018 Niklas Beisert

This work may be distributed and/or modified under the
conditions of the \LaTeX{} Project Public License, either version 1.3
of this license or (at your option) any later version.
The latest version of this license is in
  \url{http://www.latex-project.org/lppl.txt}
and version 1.3 or later is part of all distributions of \LaTeX{}
version 2005/12/01 or later.

This work has the LPPL maintenance status `maintained'.

The Current Maintainer of this work is Niklas Beisert.

This work consists of the files |README.txt|, |childdoc.ins| and |childdoc.dtx|
as well as the derived files |childdoc.def|, |cdocsamp.tex|
with |cdocsch1.tex|, |cdocsch2.tex|, |cdocspt3.tex|, |cdocspt4.tex|,
|cdocsdrf.tex|, |cdocsfn1.tex|, |cdocsfn2.tex|
as well as |childdoc.pdf|.

%%%%%%%%%%%%%%%%%%%%%%%%%%%%%%%%%%%%%%%%%%%%%%%%%%%%%%%%%%%%%%%%%%%%%%%%%%%%%%%%
\subsection{Files and Installation}

The package consists of the files:
%
\begin{center}
\begin{tabular}{ll}
    |README.txt|   & readme file \\
    |childdoc.ins| & installation file \\
    |childdoc.dtx| & source file \\
    |childdoc.def| & definition file \\
    |cdocsamp.tex| & sample main file \\
    |cdocsch1.tex| & sample include file \\
    |cdocsch2.tex| & sample include file \\
    |cdocspt3.tex| & sample part file \\
    |cdocspt4.tex| & sample part file \\
    |cdocsdrf.tex| & sample redirection file \\
    |cdocsfn1.tex| & sample redirection file \\
    |cdocsfn2.tex| & sample redirection file \\
    |childdoc.pdf| & manual
\end{tabular}
\end{center}
%
The distribution consists of the files
|README.txt|, |childdoc.ins| and |childdoc.dtx|.
%
\begin{itemize}
\item
Run (pdf)\LaTeX{} on |childdoc.dtx|
to compile the manual |childdoc.pdf| (this file).
\item
Run \LaTeX{} on |childdoc.ins| to create the definitions file |childdoc.def|
and the sample |cdocsamp.tex| with include files
|cdocsch1.tex|, |cdocsch2.tex|, |cdocspt3.tex|, |cdocspt4.tex|,
|cdocsdrf.tex|, |cdocsfn1.tex|, |cdocsfn2.tex|.
Then copy the file |childdoc.def| to an appropriate directory of your \LaTeX{}
distribution, e.g.\ \textit{texmf-root}|/tex/latex/childdoc|.
\end{itemize}

%%%%%%%%%%%%%%%%%%%%%%%%%%%%%%%%%%%%%%%%%%%%%%%%%%%%%%%%%%%%%%%%%%%%%%%%%%%%%%%%
\subsection{Related CTAN Packages}

There are several other packages which offer a similar functionality:
%
\begin{itemize}
\item
The packages
\href{http://ctan.org/pkg/docmute}{\textsf{docmute}},
\href{http://ctan.org/pkg/includex}{\textsf{includex}} and
\href{http://ctan.org/pkg/standalone}{\textsf{standalone}}
provide commands to include only the document body of
a child file thus allowing both files to be compiled individually.
\item
The packages \href{http://ctan.org/pkg/subdocs}{\textsf{subdocs}}
and \href{http://ctan.org/pkg/subfiles}{\textsf{subfiles}}
provide structures in which the main and child documents can be
encapsulated and allowing them to be compiled individually.
The inclusion mechanism is different from the conventional |\include|.
\item
The package \href{http://ctan.org/pkg/combine}{\textsf{combine}}
is an elaborate solution to combine several documents into one.
\end{itemize}
%
See also the CTAN topic \href{http://ctan.org/topic/subdocs}{\textsf{subdocs}}
for further related packages.
The present package differs from the above solutions in that
a document structure constructed with the conventional |\include| mechanism
just needs two extra commands at the top of every file
such that all constituent files can be compiled individually.

%%%%%%%%%%%%%%%%%%%%%%%%%%%%%%%%%%%%%%%%%%%%%%%%%%%%%%%%%%%%%%%%%%%%%%%%%%%%%%%%
%\subsection{Feature Suggestions}
%
%The following is a list of features which may be useful for future
%versions of this package:
%%
%\begin{itemize}
%\item
%\ldots
%\end{itemize}

%%%%%%%%%%%%%%%%%%%%%%%%%%%%%%%%%%%%%%%%%%%%%%%%%%%%%%%%%%%%%%%%%%%%%%%%%%%%%%%%
\subsection{Revision History}

%%%%%%%%%%%%%%%%%%%%%%%%%%%%%%%%%%%%%%%%
\paragraph{v2.0:} 2018/12/30

\begin{itemize}
\item
immediate forward processing
\item
added |\childdocby| mechanism
\item
manual restructured
\end{itemize}

%%%%%%%%%%%%%%%%%%%%%%%%%%%%%%%%%%%%%%%%
\paragraph{v1.6:} 2018/01/17

\begin{itemize}
\item
application for development of include files
\item
corrections to manual
\end{itemize}

%%%%%%%%%%%%%%%%%%%%%%%%%%%%%%%%%%%%%%%%
\paragraph{v1.5:} 2017/05/21

\begin{itemize}
\item
more complete structuring introduced
\item
|\childdocof| introduced
\item
|\childdoc| renamed to |\childdocmain|
\item
|\childredirect| renamed to |\childdocforward| and |\childdocforwardprefix|
and functionality expanded
\end{itemize}

%%%%%%%%%%%%%%%%%%%%%%%%%%%%%%%%%%%%%%%%
\paragraph{v1.0:} 2017/04/27

\begin{itemize}
\item
manual and install package
\item
first version published on CTAN
\end{itemize}

%%%%%%%%%%%%%%%%%%%%%%%%%%%%%%%%%%%%%%%%
\paragraph{v0.6:} 2017/04/26

\begin{itemize}
\item
redirection mechanism added
\end{itemize}

%%%%%%%%%%%%%%%%%%%%%%%%%%%%%%%%%%%%%%%%
\paragraph{v0.5:} 2017/04/26

\begin{itemize}
\item
functionality in definition file
\end{itemize}


%%%%%%%%%%%%%%%%%%%%%%%%%%%%%%%%%%%%%%%%%%%%%%%%%%%%%%%%%%%%%%%%%%%%%%%%%%%%%%%%
%%%%%%%%%%%%%%%%%%%%%%%%%%%%%%%%%%%%%%%%%%%%%%%%%%%%%%%%%%%%%%%%%%%%%%%%%%%%%%%%
%%%%%%%%%%%%%%%%%%%%%%%%%%%%%%%%%%%%%%%%%%%%%%%%%%%%%%%%%%%%%%%%%%%%%%%%%%%%%%%%
\appendix

\settowidth\MacroIndent{\rmfamily\scriptsize 000\ }

 \DocInput{childdoc.dtx}

\end{document}
%</driver>
% \fi
%
% %%%%%%%%%%%%%%%%%%%%%%%%%%%%%%%%%%%%%%%%%%%%%%%%%%%%%%%%%%%%%%%%%%%%%%%%%%%%%%
% %%%%%%%%%%%%%%%%%%%%%%%%%%%%%%%%%%%%%%%%%%%%%%%%%%%%%%%%%%%%%%%%%%%%%%%%%%%%%%
% \section{Sample}
%\iffalse
%<*samplemain>
%\fi
%
% The following presents a sample document
% with two chapters, two parts, a title page,
% a compile flag as well as three forwarding files to set the flag.
% It consists of eight |.tex| files:
% \begin{center}
% \begin{tabular}{ll}
% |cdocsamp.tex|&main file\\
% |cdocsch1.tex|&include file for chapter 1\\
% |cdocsch2.tex|&include file for chapter 2\\
% |cdocspt3.tex|&include file for part 3\\
% |cdocspt4.tex|&include file for part 4\\
% |cdocsdrf.tex|&forwarding file for main file in draft mode\\
% |cdocsfi1.tex|&forwarding file for final version of chapter 1\\
% |cdocsfi2.tex|&forwarding file for final version of chapter 2\\
% \end{tabular}
% \end{center}
% Each of the eight files can be compiled directly by the \LaTeX{} compiler.
%
% %%%%%%%%%%%%%%%%%%%%%%%%%%%%%%%%%%%%%%
% \paragraph{Main File.}
%
% The main file is called |cdocsamp.tex|.
%
% Load the \textsf{childdoc} definitions and
% declare the filename for the main document:
%    \begin{macrocode}
\input{childdoc.def}
\childdocmain{}
%    \end{macrocode}

% Optional override for |\version| flag:
%    \begin{macrocode}
%%\ifchilddoc\else\providecommand{\version}{draft}\fi
%    \end{macrocode}

% Define the default values for the |\version| flag
% (|final| for the main file and |draft| for childs):
%    \begin{macrocode}
\ifchilddoc
\providecommand{\version}{draft}
\else
\providecommand{\version}{final}
\fi
%    \end{macrocode}

% Load the standard document class:
%    \begin{macrocode}
\documentclass[12pt]{article}
%    \end{macrocode}

% Start the document body:
%    \begin{macrocode}
\begin{document}
%    \end{macrocode}

% Declare a title page.
% Print title, part of document being processed and version flag:
%    \begin{macrocode}
\addtocounter{page}{-1}
\begin{center}
{\LARGE\bfseries{}childdoc example\par}
\vspace{1cm}
\ifchilddoc
\ifchilddocmanual part\else chapter\fi:
`\childdocname' of `\childdocjob'\par
\else
main document: `\childdocjob'\par
\fi
version: \version\par
\end{center}
\newpage
%    \end{macrocode}

% Manually include selected file,
% otherwise process as usual:
%    \begin{macrocode}
\ifchilddocmanual
\section*{part `\childdocname'}
\input{\childdocname}
\else
%    \end{macrocode}

% Include the two chapters:
%    \begin{macrocode}
\include{cdocsch1}
\include{cdocsch2}
%    \end{macrocode}

% Include the two parts unless only chapters should be displayed:
%    \begin{macrocode}
\ifchilddoc\else
\section{part three}
\input{cdocspt3}
\section{part four}
\input{cdocspt4}
\fi
%    \end{macrocode}

% Process as usual until here:
%    \begin{macrocode}
\fi
%    \end{macrocode}

% End of document body:
%    \begin{macrocode}
\end{document}
%    \end{macrocode}
%\iffalse
%</samplemain>
%\fi
%
% %%%%%%%%%%%%%%%%%%%%%%%%%%%%%%%%%%%%%%
% \paragraph{Chapter Include Files.}
%
% The include files are called |cdocsch1.tex| and |cdocsch2.tex|.
%
%\iffalse
%<*samplechap1|samplechap2>
%\fi

% Optional override for |\version| flag:
%    \begin{macrocode}
%%\providecommand{\version}{final}
%    \end{macrocode}

% Include the main document:
%    \begin{macrocode}
\input{childdoc.def}
\childdocof{cdocsamp}
%    \end{macrocode}

%\iffalse
%</samplechap1|samplechap2>
%\fi
%
%\iffalse
%<*samplechap1>
%\fi
% Some text for chapter 1:
%    \begin{macrocode}
\section{one}
some text in chapter one
%    \end{macrocode}

%\iffalse
%</samplechap1>
%\fi
% Some text for chapter 2:
%\iffalse
%<*samplechap2>
%\fi
%    \begin{macrocode}
\section{two}
more text in chapter two
%    \end{macrocode}

%\iffalse
%</samplechap2>
%\fi
%
% %%%%%%%%%%%%%%%%%%%%%%%%%%%%%%%%%%%%%%
% \paragraph{Part Include Files.}
%
% The include files are called |cdocspt3.tex| and |cdocspt4.tex|.
%
%\iffalse
%<*samplepart3|samplepart4>
%\fi

% Optional override for |\version| flag:
%    \begin{macrocode}
%%\providecommand{\version}{final}
%    \end{macrocode}

% Include the main document:
%    \begin{macrocode}
\input{childdoc.def}
\childdocby{cdocsamp}
%    \end{macrocode}

%\iffalse
%</samplepart3|samplepart4>
%\fi
%
%\iffalse
%<*samplepart3>
%\fi
% Some text for part 3:
%    \begin{macrocode}
some text in part three
%    \end{macrocode}

%\iffalse
%</samplepart3>
%\fi
% Some text for part 4:
%\iffalse
%<*samplepart4>
%\fi
%    \begin{macrocode}
more text in part four
%    \end{macrocode}

%\iffalse
%</samplepart4>
%\fi
%
% %%%%%%%%%%%%%%%%%%%%%%%%%%%%%%%%%%%%%%
% \paragraph{Forwarding for a Complete Draft.}
%
% The following forwarding file |cdocsdrf.tex|
% compiles the main document in draft mode:
%\iffalse
%<*sampledraft>
%\fi
%    \begin{macrocode}
\def\version{draft}
\input{childdoc.def}
\childdocforward{cdocsamp}
%    \end{macrocode}

%\iffalse
%</sampledraft>
%\fi
%
% %%%%%%%%%%%%%%%%%%%%%%%%%%%%%%%%%%%%%%
% \paragraph{Forwarding for Final Version of the Chapters.}
%
% The following forwarding files |cdocsfn1.tex| and |cdocsfn2.tex|
% (with identical content)
% compile the final versions of the child documents
% |cdocsch1.tex| and |cdocsch2.tex|, respectively:
%\iffalse
%<*samplefinal>
%\fi
%    \begin{macrocode}
\def\version{final}
\input{childdoc.def}
\childdocforwardprefix[cdocsamp]{cdocsfn}{cdocsch}
%    \end{macrocode}

%\iffalse
%</samplefinal>
%\fi
%
% %%%%%%%%%%%%%%%%%%%%%%%%%%%%%%%%%%%%%%
% \paragraph{Command Line Processing.}
%
% The following three command lines generate the output files
% |cdocscld|, |cdocscl1| and |cdocscl2|
% which should be identical to
% |cdocsdrf|, |cdocsch1| and |cdocsfn2|, respectively:
% \begin{center}
% \begin{tabular}{l}
% |latex -jobname cdocscld \|\\
% |  "\def\version{draft}\input{childdoc.def}\childdocforward{cdocsamp}"|\\
% |latex -jobname cdocscl1 \|\\
% |  "\input{childdoc.def}\childdocforward[cdocsamp]{cdocsch1}"|\\
% |latex -jobname cdocscl2 \|\\
% |  "\def\version{final}\input{childdoc.def}\childdocforward{cdocsch2}"|
% \end{tabular}
% \end{center}
% Note that the trailing backslash on each first line
% merely continues the input to the second line
% (for convenient cut ant paste).
% Furthermore, the command |latex| can be replaced by any
% of its alternative versions such as |pdflatex|.
%
% %%%%%%%%%%%%%%%%%%%%%%%%%%%%%%%%%%%%%%%%%%%%%%%%%%%%%%%%%%%%%%%%%%%%%%%%%%%%%%
% %%%%%%%%%%%%%%%%%%%%%%%%%%%%%%%%%%%%%%%%%%%%%%%%%%%%%%%%%%%%%%%%%%%%%%%%%%%%%%
% \section{Implementation}
%\iffalse
%<*package>
%\fi
%
% This section describes the definitions file |childdoc.def|.

% The definitions cannot be loaded using |\usepackage| or |\RequirePackage|
% which has a mechanism to prevent loading a style file more than once.
% When loading the definitions by means of |\input|
% multiple instances have to be prevented manually:
%\iffalse
%This code needs to be before the `\ProvidesFile' directive
%which is defined at the beginning of this file.
%Therefore it is also placed there and commented out here.
%</package>
%<*discard>
%\fi
%    \begin{macrocode}
\ifdefined\childdocmain\endinput\fi
%    \end{macrocode}
%\iffalse
%</discard>
%<*package>
%\fi
%
% \macro{\ifchilddoc}
% \macro{\ifchilddocmanual}
% The conditional |\ifchilddoc| tells whether a
% child (true) or main (false) document is being compiled.
% The conditional |\ifchilddocmanual| tells whether
% the |\includeonly| mechanism is used (false) or
% the selection of child files must be performed manually (true).
% The definitions initialise to false:
%    \begin{macrocode}
\newif\ifchilddoc
\newif\ifchilddocmanual
%    \end{macrocode}

% \macro{\childdocname}
% \macro{\childdocjob}
% The macro |\childdocname| stores the name of the main document
% to be compiled. The macro |\childdocjob| stores the name of
% the document on which the \LaTeX{} compiler was originally invoked.
% The content of |\jobname| cannot be compared
% to filenames specified in the source due to different catcodes.
% The following code rescans |\jobname|, stores the result
% in |\childdocname| and saves a copy in |\childdocjob|:
%    \begin{macrocode}
\edef\childdocname{\scantokens\expandafter{\jobname\noexpand}}
\let\childdocjob\childdocname
%    \end{macrocode}

% \macro{\childdocdisable}
% The macro |\childdocdisable| prevents the main file
% from being processed more than once.
% At this stage, the main document command |\childdocmain|
% is assumed to be called once again where it should do nothing.
% Any subsequent call to it should prevent
% a secondary processing of the main document
% It overwrites the forwarding commands
% |\childdocof| and |\childdocforward|
% with empty macros to prevent further inclusions of the main document:
%    \begin{macrocode}
\newcommand{\childdocdisable}
{
  \renewcommand{\childdocmain}[1]{\renewcommand{\childdocmain}[1]{\endinput}}
  \renewcommand{\childdocof}[1]{}
  \renewcommand{\childdocby}[2][]{}
  \renewcommand{\childdocforward}[2][]{}
  \renewcommand{\childdocdisable}{}
}
%    \end{macrocode}

% \macro{\childdocmain}
% The macro |\childdocmain| is to be called at the top of the main file
% with nothing or the main filename (without extension) as argument.
% First, it breaks loops.
% If the argument is not empty and does not match |\childdocname|
% (which is set by the first inclusion of |childdoc.def|),
% |\ifchilddoc| is set to true, |\includeonly| is applied to the child file
% and |\jobname| is set to the main file
% (for proper handling of |.aux| files):
%    \begin{macrocode}
\newcommand{\childdocmain}[1]
{
  \childdocdisable\childdocmain{}
  \if?#1?\else
    \begingroup
      \def\childdoctmp{#1}
      \ifx\childdoctmp\childdocname
        \def\childdoctmp{}
      \else
        \def\childdoctmp
        {
          \childdoctrue
          \includeonly{\childdocname}
          \def\childdocjob{#1}
          \def\jobname{#1}
        }
      \fi
      \expandafter
    \endgroup
    \childdoctmp
  \fi
}
%    \end{macrocode}

% \macro{\childdocof}
% The command |\childdocof| redirects
% compilation to the main file |#1|.
%    \begin{macrocode}
\newcommand{\childdocof}[1]
{
  \childdocdisable
  \childdoctrue
  \includeonly{\childdocname}
  \def\jobname{#1}
  \def\childdocjob{#1}
  \input{#1}
}
%    \end{macrocode}

% \macro{\childdocby}
% The command |\childdocby| ....
%    \begin{macrocode}
\newcommand{\childdocby}[2][]
{
  \childdocdisable
  \childdoctrue
  \childdocmanualtrue
  \if?#1?\else
    \def\jobname{#2}
  \fi
  \def\childdocjob{#2}
  \input{#2}
  \endinput
}
%    \end{macrocode}

% \macro{\childdocforward}
% The command |\childdocforward| redirects
% compilation to the main file or
% (if the optional argument is given) a child file.
% Parameters are set as if the main file
% or a child file starting with |\childdocof| was compiled.
% Then compilation is handed over to the main file:
%    \begin{macrocode}
\newcommand{\childdocforward}[2][]
{
  \begingroup
    \if?#1?
      \def\childdoctmp
      {
        \def\childdocname{#2}
        \def\childdocjob{#2}
        \def\jobname{#2}
        \input{#2}
        \endinput
      }
    \else
      \def\childdoctmp
      {
        \childdocdisable
        \def\childdocname{#2}
        \childdoctrue
        \includeonly{#2}
        \def\childdocjob{#1}
        \def\jobname{#1}
        \input{#1}
        \endinput
      }
    \fi
    \expandafter
  \endgroup
  \childdoctmp
}
%    \end{macrocode}

% \macro{\childdocforwardprefix}
% The command |\childdocforwardprefix| redirects
% compilation to the main or a child file by means of a pattern.
% The prefix |#1| in the current filename is replaced by |#2|
% and the suffix of the current filename is kept
% (it is assumed that the filename does not contain the substring `|~~~|'
% which is used as a delimiter).
% Compilation is handed over to the new file by |\childdocforward|:
%    \begin{macrocode}
\newcommand{\childdocforwardprefix}[3][]
{
  \begingroup
    \def\childdocextract #2##1~~~{\def\childdoctmp{\childdocforward[#1]{#3##1}}}
    \expandafter\childdocextract\childdocname~~~
    \expandafter
  \endgroup
  \childdoctmp
}
%    \end{macrocode}

% \macro{\childdoc}
% The deprecated macro |\childdoc| is a legacy version of |\childdocmain|:
%    \begin{macrocode}
\newcommand{\childdoc}{\childdocmain}
%    \end{macrocode}

% \macro{\childdocredirect}
% The deprecated macro |\childdocredirect| is a legacy version
% of |\childdocforward| and |\childdocforwardprefix|:
%    \begin{macrocode}
\newcommand{\childdocredirect}[2][]
{
  \begingroup
    \if?#1?
      \def\childdoctmp{\childdocforward{#2}}
    \else
      \def\childdoctmp{\childdocforwardprefix{#1}{#2}}
    \fi
    \expandafter
  \endgroup
  \childdoctmp
}
%    \end{macrocode}

%\iffalse
%</package>
%\fi
%
\endinput
|\\
|\childdocforward[|\textit{main}|]{|\textit{dest}|}|\\
\end{tabular}
\end{center}
%
The argument \textit{dest} is the destination file
(without extension).
It should be the main file or one of the child files.
Note that further \textsf{childdoc} directives
such as |\childdocof| and |\childdocforward|
in the indicated file will be processed in this form.
The optional argument \textit{main}
passes on directly to the main file \textit{main}
while pretending to compile the child \textit{dest}.
This form behaves as if \textit{dest}
issues |\childdocof{|\textit{main}|}| right away,
and no further \textsf{childdoc} directives will be processed.

%%%%%%%%%%%%%%%%%%%%%%%%%%%%%%%%%%%%%%%%
\DescribeMacro{\...prefix}
In the alternative form |\childdocforwardprefix|,
%
\begin{center}
\begin{tabular}{l}
|% \iffalse
%
% childdoc.dtx Copyright (C) 2017-2018 Niklas Beisert
%
% This work may be distributed and/or modified under the
% conditions of the LaTeX Project Public License, either version 1.3
% of this license or (at your option) any later version.
% The latest version of this license is in
%   http://www.latex-project.org/lppl.txt
% and version 1.3 or later is part of all distributions of LaTeX
% version 2005/12/01 or later.
%
% This work has the LPPL maintenance status `maintained'.
%
% The Current Maintainer of this work is Niklas Beisert.
%
% This work consists of the files childdoc.dtx and childdoc.ins
% and the derived files childdoc.def and cdocsamp.tex with
% cdocsch1.tex, cdocsch2.tex, cdocsdrf.tex, cdocsfn1.tex, cdocsfn2.tex.
%
%<package>\ifdefined\childdocmain\endinput\fi
%<package>\ProvidesFile{childdoc.def}[2018/12/30 v2.0 child document driver]
%<samplemain>\ProvidesFile{cdocsamp.tex}[2018/12/30 v2.0 sample for childdoc]
%<*driver>
%\ProvidesFile{childdoc.drv}[2018/12/30 v2.0 childdoc reference manual file]
\PassOptionsToClass{10pt,a4paper}{article}
\documentclass{ltxdoc}

\usepackage[margin=35mm]{geometry}
\usepackage{hyperref}
\usepackage{hyperxmp}
\usepackage[usenames]{color}

\hypersetup{colorlinks=true}
\hypersetup{pdfstartview=FitH}
\hypersetup{pdfpagemode=UseNone}
\hypersetup{pdfsource={}}
\hypersetup{pdflang={en-UK}}
\hypersetup{pdfcopyright={Copyright 2017-2018 Niklas Beisert.
  This work may be distributed and/or modified under the
  conditions of the LaTeX Project Public License, either version 1.3
  of this license or (at your option) any later version.}}
\hypersetup{pdflicenseurl={http://www.latex-project.org/lppl.txt}}
\hypersetup{pdfcontactaddress={ETH Zurich, ITP, HIT K,
  Wolfgang-Pauli-Strasse 27}}
\hypersetup{pdfcontactpostcode={8093}}
\hypersetup{pdfcontactcity={Zurich}}
\hypersetup{pdfcontactcountry={Switzerland}}
\hypersetup{pdfcontactemail={nbeisert@itp.phys.ethz.ch}}
\hypersetup{pdfcontacturl={http://people.phys.ethz.ch/\xmptilde nbeisert/}}

\newcommand{\secref}[1]{\hyperref[#1]{section \ref*{#1}}}

\parskip1ex
\parindent0pt
\let\olditemize\itemize
\def\itemize{\olditemize\parskip0pt}

\begin{document}

\title{The \textsf{childdoc} Package}
\hypersetup{pdftitle={The childdoc Package}}
\author{Niklas Beisert\\[2ex]
  Institut f\"ur Theoretische Physik\\
  Eidgen\"ossische Technische Hochschule Z\"urich\\
  Wolfgang-Pauli-Strasse 27, 8093 Z\"urich, Switzerland\\[1ex]
  \href{mailto:nbeisert@itp.phys.ethz.ch}
  {\texttt{nbeisert@itp.phys.ethz.ch}}}
\hypersetup{pdfauthor={Niklas Beisert}}
\hypersetup{pdfsubject={Manual for the LaTeX2e Package childdoc}}
\date{30 December 2018, \textsf{v2.0}}
\maketitle

\begin{abstract}\noindent
\textsf{childdoc} is a \LaTeXe{} package
that enables the direct compilation
of document sections included by |\include|
to individual files.
\end{abstract}

\begingroup
\parskip0ex
\tableofcontents
\endgroup

%%%%%%%%%%%%%%%%%%%%%%%%%%%%%%%%%%%%%%%%%%%%%%%%%%%%%%%%%%%%%%%%%%%%%%%%%%%%%%%%
%%%%%%%%%%%%%%%%%%%%%%%%%%%%%%%%%%%%%%%%%%%%%%%%%%%%%%%%%%%%%%%%%%%%%%%%%%%%%%%%
\section{Introduction}

\LaTeX{} provides a mechanism to structure a large document (such as a book)
into a main file and several child files (containing the chapters)
using the |\include| command.
This mechanism is beneficial for documents
which span hundreds of pages in order to
make the source file(s) more manageable.
Moreover, compilation can be restricted to
selected child files by means of the |\includeonly| command.
The latter feature can be used to reduce the compilation time while editing
(this was significantly more useful in the earlier days of \LaTeX{})
or to generate a smaller document which is easier to navigate.
Another application of |\includeonly| is to generate
documents consisting of selected parts of the complete document.

However, there are a few drawbacks of the plain |\include| mechanism:
\begin{itemize}
\item
The child files cannot be compiled on their own,
they can only be compiled via the main file.
A naive editing environment
(such as a text editor with an option
to have the current file processed by \LaTeX)
may require one to switch to the main file before compiling;
attempting to compile the child file produces errors.
\item
The main file must be modified (each time)
to adjust the |\includeonly| command
to the present needs. This easily leaves the main file in a messy state.
\item
The generated document will always carry the filename
of the main document. This is inconvenient if
several child files are to be compiled and
to be kept for distribution.
\end{itemize}

The present package provides a simple interface
to make child files individually compilable by \LaTeX{}.
Compiling a child file then has the same effect as compiling
the main file with an |\includeonly| command
to select the appropriate child.
Moreover the generated document will carry the name of the child
rather than the main file.
This resolves all three above issues.

This feature is meant to make the editing of books,
thesis documents and lecture notes somewhat more convenient.
However, the package can also be used efficiently for
composing a series of documents (such as exercise sheets)
which are typically distributed individually.
It then assists the author in generating the individual documents
(potentially in different versions)
as well as a document containing the collected series.
Another application is in developing style files
or other kinds of included material
where compilation of the style file could redirect
to a sample or test file.

%%%%%%%%%%%%%%%%%%%%%%%%%%%%%%%%%%%%%%%%%%%%%%%%%%%%%%%%%%%%%%%%%%%%%%%%%%%%%%%%
%%%%%%%%%%%%%%%%%%%%%%%%%%%%%%%%%%%%%%%%%%%%%%%%%%%%%%%%%%%%%%%%%%%%%%%%%%%%%%%%
\section{Usage}

First of all, the package \textsf{childdoc} is \emph{not} a standard
\LaTeXe{} |.sty| style file! Therefore it needs to be invoked in
a non-standard way.

%%%%%%%%%%%%%%%%%%%%%%%%%%%%%%%%%%%%%%%%%%%%%%%%%%%%%%%%%%%%%%%%%%%%%%%%%%%%%%%%
\subsection{Included Files}
\label{sec:include}

%%%%%%%%%%%%%%%%%%%%%%%%%%%%%%%%%%%%%%%%
\DescribeMacro{\childdocmain}
To use the package, add the commands
\begin{center}
\begin{tabular}{l}
|\input{childdoc.def}|\\
|\childdocmain{}|\\
\end{tabular}
\end{center}
at the very top of the main \LaTeX{} file,
in particular \emph{before} the |\documentclass| statement!
The argument of |\childdocmain| should be left empty
(but it must be present).

%%%%%%%%%%%%%%%%%%%%%%%%%%%%%%%%%%%%%%%%
\DescribeMacro{\childdocof}
Furthermore, add the commands
\begin{center}
\begin{tabular}{l}
|\input{childdoc.def}|\\
|\childdocof{|\textit{main}|}|\\
\end{tabular}
\end{center}
at the top of every child file \textit{child}
which is included by |\include{|\textit{child}|}|
from within the main file
(or at least for those files to be compiled individually).
The argument \textit{main} must be the filename of the main file.

There are a couple of
considerations in setting up the main and child documents:

%%%%%%%%%%%%%%%%%%%%%%%%%%%%%%%%%%%%%%%%
\paragraph{Restrictions.}

Please note the following restrictions:
\begin{itemize}
\item
|\childdocmain| must be called with one argument \textit{main}
to ensure compatibility with earlier version of the package.
It must either be empty (|\childdocmain{}|)
or precisely match the filename of the main file in which it is specified.
See \secref{sec:detection} for further information.
\item
The filename \textit{main} must be specified without the |.tex| extension.
\item
The filename \textit{main} is case sensitive
(even in case-insensitive file systems)
due to internal string comparison.
\item
The argument \textit{main} should be fully expanded, it cannot be a macro.
\item
Subdirectories and special characters should be avoided in filenames.
\item
The command |\childdocmain{|\textit{main}|}| must be followed by a whitespace.
It should not be followed immediately by another command
or by a comment mark `|%|'.
This is because the \TeX{} parser reads the token immediately following
the argument of |\childdocmain| and puts it
at the beginning of every child section;
however, a white\-space is ignored.
\end{itemize}

%%%%%%%%%%%%%%%%%%%%%%%%%%%%%%%%%%%%%%%%
\paragraph{Content of Main File.}

It is advisable to place all content in the child files included by |\include|.
Any output contained in the main file will appear in all child documents
unless suppressed manually;
it cannot be suppressed automatically by the |\includeonly| directive
and thus should normally be avoided.
A method to include some content in the main file
by means of conditional processing is described in \secref{sec:conditional}.

%%%%%%%%%%%%%%%%%%%%%%%%%%%%%%%%%%%%%%%%
\paragraph{Page Numbering.}

When only a part of the document is compiled,
the appropriate numbering of pages
(as well as other status parameters)
is determined from the |.aux| files.
The latter contain information from previous passes.
However this information needs to propagate through
all intermediate child documents.
Therefore the page numbering in child documents may well
be inconsistent until the complete document is compiled at least once.

A useful (if unconventional) way to always ensure a consistent
page numbering is to restart the numbering in each child document
and denote the pages by `\textit{child}|.|\textit{page}'
where \textit{child} represents the chapter/section number of the child file.
This can be achieved by the command
|\numberwithin{page}{|\textit{child}|}|
of the \textsf{amsmath} package
where \textit{child} can be |chapter| or |section|
depending on the chosen structuring.
Alternatively, one can modify the macro |\thepage| appropriately
and reset the counter |page| at the start of each child file.

%%%%%%%%%%%%%%%%%%%%%%%%%%%%%%%%%%%%%%%%%%%%%%%%%%%%%%%%%%%%%%%%%%%%%%%%%%%%%%%%
\subsection{Conditional Processing}
\label{sec:conditional}

The package provides a mechanism to compile different versions
of a document. To customise the versions further some conditional processing
can come in handy to distinguish which version is being compiled.
The package provides two macros to describe the compilation context:

%%%%%%%%%%%%%%%%%%%%%%%%%%%%%%%%%%%%%%%%
\DescribeMacro{\ifchilddoc}
The conditional |\ifchilddoc| distinguishes between the compilation of
child documents and the main document:
%
\begin{center}
|\ifchilddoc |\textit{child-code}| |[|\||else |\textit{main-code}]| \||fi|
\end{center}

%%%%%%%%%%%%%%%%%%%%%%%%%%%%%%%%%%%%%%%%
\DescribeMacro{\childdocname}
\DescribeMacro{\childdocjob}
The macro |\childdocname| contains the filename (without extension)
of the main or child file being processed.
Note that |\childdocjob| will always contain the name of the main file.

%%%%%%%%%%%%%%%%%%%%%%%%%%%%%%%%%%%%%%%%
\paragraph{Title Page.}

Conditional processing can be used to include a title or banner page
in the main document when proper precautions are taken.
Importantly, the code in the main file should ensure that the page counter
(as well as other status parameters which are stored in the |.aux| files)
takes the same value after the conditional processing.
Otherwise the page numbers may take divergent values
depending on which part is compiled.

For example, a title page could be declared by:
%
\begin{center}
\begin{tabular}{l}
|\ifchilddoc\||else|\\
|\addtocounter{page}{-1}|\\
\textit{code for title page}\\
|\newpage|\\
|\||fi|
\end{tabular}
\end{center}
%
A banner page for the child documents can be generated by:
%
\begin{center}
\begin{tabular}{l}
|\ifchilddoc|\\
|\addtocounter{page}{-1}|\\
\textit{code for banner page}\\
|\newpage|\\
|\||fi|
\end{tabular}
\end{center}
%
Here one could write a message such as:
\begin{center}
|This is the part \childdocname{} of \childdocjob{}.|
\end{center}

%%%%%%%%%%%%%%%%%%%%%%%%%%%%%%%%%%%%%%%%%%%%%%%%%%%%%%%%%%%%%%%%%%%%%%%%%%%%%%%%
\subsection{Flags}
\label{sec:flags}

The package makes it easy to generate different versions
of the main or child documents.
To this end compilation flags can be defined
and assigned different default values.
They will be particularly useful in conjunction
with the forwarding mechanism described in \secref{sec:forward}.

For example, it may be useful to have a flag |\version|
which can be set to |draft| or |final|.
The document source will contain some conditional code
depending on the value of |\version|.
Suppose further, the flag should default to |final| for the main file
and to |draft| for child files
which is a natural assignment for editing the document.
This is achieved by placing the following code
in the preamble of the main document
(below the |\childdocmain| directive):
%
\begin{center}
\begin{tabular}{l}
|\ifchilddoc|\\
|\providecommand{\version}{draft}|\\
|\||else|\\
|\providecommand{\version}{final}|\\
|\||fi|
\end{tabular}
\end{center}
%
The definition by |\providecommand| makes sure
that previous definitions are not overwritten.
Further statements |\providecommand{\version}{...}|
can thus be added before the above code to override it.

For the main file, one might add a line
(between |\childdocmain| and the above block)
%
\begin{center}
|%\ifchilddoc\||else\providecommand{\version}{draft}\||fi|
\end{center}
%
which can be uncommented to produce a draft version.
Likewise one can add a line to the very top of a child file
(above the |\childdocof{|\textit{main}|}| directive)
%
\begin{center}
|%\providecommand{\version}{final}|
\end{center}
%
which can be uncommented to produce the final version of this child document.

%%%%%%%%%%%%%%%%%%%%%%%%%%%%%%%%%%%%%%%%%%%%%%%%%%%%%%%%%%%%%%%%%%%%%%%%%%%%%%%%
\subsection{Forwarding}
\label{sec:forward}

Different versions of the main or child documents
using compilation flags as described in \secref{sec:flags}
can be (permanently) stored in different files
for convenient compilation, viewing and distribution.
To this end, the package defines a command
to pass on compilation to a different file:

%%%%%%%%%%%%%%%%%%%%%%%%%%%%%%%%%%%%%%%%
\DescribeMacro{\childdocforward}
The command |\childdocforward| redirects processing to
another source file:
%
\begin{center}
\begin{tabular}{l}
|\input{childdoc.def}|\\
|\childdocforward[|\textit{main}|]{|\textit{dest}|}|\\
\end{tabular}
\end{center}
%
The argument \textit{dest} is the destination file
(without extension).
It should be the main file or one of the child files.
Note that further \textsf{childdoc} directives
such as |\childdocof| and |\childdocforward|
in the indicated file will be processed in this form.
The optional argument \textit{main}
passes on directly to the main file \textit{main}
while pretending to compile the child \textit{dest}.
This form behaves as if \textit{dest}
issues |\childdocof{|\textit{main}|}| right away,
and no further \textsf{childdoc} directives will be processed.

%%%%%%%%%%%%%%%%%%%%%%%%%%%%%%%%%%%%%%%%
\DescribeMacro{\...prefix}
In the alternative form |\childdocforwardprefix|,
%
\begin{center}
\begin{tabular}{l}
|\input{childdoc.def}|\\
|\childdocforwardprefix[|\textit{main}|]{|\textit{prefix}|}{|\textit{dest}|}|
\end{tabular}
\end{center}
%
the destination file is determined by a pattern
depending on the current file:
To make this work, the current file must be called
`{\textit{prefix}\hspace{0.2em}\textit{suffix}}'
with \textit{prefix} matching precisely the argument.
Processing is then passed on to the file
`{\textit{dest}\hspace{0.2em}\textit{suffix}}'.
Surely, the same effect is achieved by
directly specifying the
argument `{\textit{dest}\hspace{0.2em}\textit{suffix}}'
in the first form.
However, that requires to set up a different file
for each child. With the alternative form of the command
all these files can have exactly the same content
which simplifies setting them up and maintaining them.

For example, the following file |draft.tex|
with a compilation flag |\version| as described in \secref{sec:flags}
compiles the main document as a draft:
%
\begin{center}
\begin{tabular}{l}
|\def\version{draft}|\\
|\input{childdoc.def}|\\
|\childdocforward{|\textit{main}|}|
\end{tabular}
\end{center}
%
Likewise, the following files |final|\textit{nn}|.tex|
compile the final version of the child document
|child|\textit{nn}|.tex|:
%
\begin{center}
\begin{tabular}{l}
|\def\version{final}|\\
|\input{childdoc.def}|\\
|\childdocforwardprefix{final}{child}|
\end{tabular}
\end{center}
%

Note that when several versions of a main file and/or of each child file
are to be generated, it may be convenient to set up a |Makefile| or
shell script to automatise the process.

%%%%%%%%%%%%%%%%%%%%%%%%%%%%%%%%%%%%%%%%%%%%%%%%%%%%%%%%%%%%%%%%%%%%%%%%%%%%%%%%
\subsection{Command Line Processing}
\label{sec:commandline}

The effect of redirection files can also be achieved by invoking
the \LaTeX{} compiler with a more elaborate command line.
Most conveniently this should be done as part
of a shell script or a |Makefile|.

When using \textsf{childdoc} in the main file, the following
command lines effectively perform a redirection
(note that depending on the shell being used,
backslashes may have to be doubled: `|\|' $\to$ `|\\|'):
%
\begin{center}
|... -jobname "|\textit{target}|" |\\|"|[\textit{flags}]%
|\input{childdoc.def}\childdocforward[|\textit{main}|]{|\textit{dest}|}"|
\end{center}
%
Here \textit{target} is the name of the output file,
\textit{main} is the name of the main file
and \textit{dest} is the name of the main or child file to be processed
(all filenames without extensions).
The optional argument \textit{main} can be omitted
if \textit{main} matches \textit{dest}.
Optionally, compilation \textit{flags} can be defined via |\def| commands.
This command line makes the \TeX{} engine believe
it is compiling the file \textit{target}
whose content is specified as the latter parameter.
The provided code then forwards the processing to
\textit{main} or \textit{dest} as described in \secref{sec:forward}.

%%%%%%%%%%%%%%%%%%%%%%%%%%%%%%%%%%%%%%%%%%%%%%%%%%%%%%%%%%%%%%%%%%%%%%%%%%%%%%%%
\subsection{Include by Input}
\label{sec:input}

Including child documents by |\include| has some restrictions by design.
Most notably, the content of a child document always occupies
its own set of pages; pages cannot be shared between child documents.
Usually, this behaviour makes perfect sense
because each child document contain an essential part of the document.
However, in some situations it may be desirable to compose
a document from a collection of parts
without having mandatory page breaks between then.
For this case, the package
provides a mechanism to include parts
by |\input| which can also be processed individually.
However, by construction this mechanism
requires manual handling of the content to be output.

%%%%%%%%%%%%%%%%%%%%%%%%%%%%%%%%%%%%%%%%
\DescribeMacro{\ifchilddocmanual}
The main file should be prepared as usual, see \secref{sec:include}.
However, the document body must make a distinction
between processing of an individual part and of the main document, e.g.:
%
\begin{center}
\begin{tabular}{l}
|\ifchilddocmanual|\\
|\input{\childdocname}|\\
|\||else|\\
\textit{document body with }|\input{|\textit{part}|}|\\
|\||fi|
\end{tabular}
\end{center}
%
The conditional |\ifchilddocmanual| is true whenever
a part to be included by |\input| is being compiled,
and the name of the part is stored in |\childdocname|.

%%%%%%%%%%%%%%%%%%%%%%%%%%%%%%%%%%%%%%%%
\DescribeMacro{\childdocby}
Each part to be included by |\input| should start with:
%
\begin{center}
\begin{tabular}{l}
|\input{childdoc.def}|\\
|\childdocby{|\textit{main}|}|\\
\end{tabular}
\end{center}
%
The directive |\childdocby| is similar to |\childdocof|
described in \secref{sec:include},
but the subsequent selection of content must be done manually.
To that end, both |\ifchilddoc| and |\ifchilddocmanual|
will be true upon processing of a part,
and the name of the part is stored in |\childdocname|.
Note that |\jobname| will be set to the filename of the current part
so that each part receives an individual |.aux| file
that does not interfere with the |.aux| file(s) of the main document.
This behaviour can be altered by the alternative form
|\childdocby[*]{|\textit{main}|}| (with a non-empty optional argument)
which uses the |.aux| file of the main document
by setting |\jobname| to \textit{main}.

%%%%%%%%%%%%%%%%%%%%%%%%%%%%%%%%%%%%%%%%%%%%%%%%%%%%%%%%%%%%%%%%%%%%%%%%%%%%%%%%
\subsection{Driver Development}
\label{sec:driver}

The \textsf{childdoc} mechanism can also be use for the development
of definition files such as \LaTeX{} styles or classes.
This case differs from the above setup with multiple parts
included by |\include| in that no |\includeonly| should be invoked.
This can be achieved by starting the include file
(before |\ProvidesPackage|) with:
%
\begin{center}
\begin{tabular}{l}
|\input{childdoc.def}|\\
|\childdocforward{|\textit{main}|}|\\
\end{tabular}
\end{center}
%
or alternatively with:
%
\begin{center}
\begin{tabular}{l}
|\input{childdoc.def}|\\
|\childdocby{|\textit{main}|}|\\
\end{tabular}
\end{center}
%
Both forms have slightly different effects as described above.
The main file is prepared as usual, see \secref{sec:include}.

%%%%%%%%%%%%%%%%%%%%%%%%%%%%%%%%%%%%%%%%%%%%%%%%%%%%%%%%%%%%%%%%%%%%%%%%%%%%%%%%
\subsection{Legacy Detection}
\label{sec:detection}

The directive |\childdocmain| in the main file can detect
whether the complete document or merely a child is to be compiled
even without using the directive |\childdocof|.
This method is deprecated because it is less robust
and there is no compelling reason to use it;
it is merely provided for backward compatibility
and it may be removed in future versions.

If the detection mechanism is to be used,
it is mandatory to correctly specify
the filename of the main file as the argument of |\childdocmain|:
%
\begin{center}
\begin{tabular}{l}
|\input{childdoc.def}|\\
|\childdocmain{|\textit{main}|}|\\
\end{tabular}
\end{center}
%
If |\jobname| does not match the argument \textit{main} of |\childdocmain|,
it is assumed that |\jobname| points to the child file to be compiled.
When using |\childdocmain| with the main file specified as argument,
it suffices to start a child file
with just |\input{|\textit{main}|}|
without loading of the package and using |\childdocof|.
If instead all processing is done
with the appropriate \textsf{childdoc} directives,
the argument of \textit{main} of |\childdocmain| can be empty.

An alternative version of the command line processing described
in \secref{sec:commandline} using the detection mechanism reads:
%
\begin{center}
|... -jobname "|\textit{target}|" "|[\textit{flags}]%
[|\def\jobname{|\textit{dest}|}|]|\input{|\textit{main}|}"|
\end{center}

%%%%%%%%%%%%%%%%%%%%%%%%%%%%%%%%%%%%%%%%%%%%%%%%%%%%%%%%%%%%%%%%%%%%%%%%%%%%%%%%
\subsection{Manual Code}
\label{sec:manual}

In case one cannot be certain whether the definitions file |childdoc.def|
is installed on the target \TeX{} distribution
and one prefers not to ship it,
it is conceivable to paste a few relevant commands into the sources.

To that end, drop all statements |\input{childdoc.def}|
and perform the replacements as outlined below.
Instead of |\childdocmain{|\textit{main}|}| add the following code
to the top of the main file:
%
\begin{center}
\begin{tabular}{l}
|\||ifdefined\childdocname\endinput\||fi\newif\ifchilddoc|\\
|\edef\childdocname{\scantokens\expandafter{\jobname\noexpand}}|\\
|\def\childdocmain{|\textit{main}|}\||ifx\childdocmain\childdocname\||else|\\
|\childdoctrue\includeonly{\childdocname}\let\jobname\childdocmain\||fi|\\
\end{tabular}
\end{center}
%
Instead of |\childdocof{|\textit{main}|}| just include the main file
at the top of each child file:
%
\begin{center}
|\input{|\textit{main}|}|
\end{center}
%
A simple redirection |\childdocforward{|\textit{dest}|}| is achieved by:
%
\begin{center}
|\def\jobname{|\textit{dest}|}\input{\jobname}|
\end{center}
%
The redirection with prefix
|\childdocforwardprefix[|\textit{prefix}|]{|\textit{dest}|}|
is accomplished by:
%
\begin{center}
\begin{tabular}{l}
|{\edef\jobname{\scantokens\expandafter{\jobname\noexpand}}|\\
|\def\redirectjob |\textit{prefix}|#1~~~{\gdef\jobname{|\textit{dest}|#1}}|\\
|\expandafter\redirectjob\jobname~~~}\input{\jobname}|
\end{tabular}
\end{center}

In an alternative approach,
child documents can be compiled by a specific command line
without additional code or specific definitions:
%
\begin{center}
|... -jobname "|\textit{target}|" "|[\textit{flags}]%
|\includeonly{|\textit{dest}|}\input{|\textit{main}|}"|
\end{center}
%

%%%%%%%%%%%%%%%%%%%%%%%%%%%%%%%%%%%%%%%%%%%%%%%%%%%%%%%%%%%%%%%%%%%%%%%%%%%%%%%%
%%%%%%%%%%%%%%%%%%%%%%%%%%%%%%%%%%%%%%%%%%%%%%%%%%%%%%%%%%%%%%%%%%%%%%%%%%%%%%%%
\section{Information}

%%%%%%%%%%%%%%%%%%%%%%%%%%%%%%%%%%%%%%%%%%%%%%%%%%%%%%%%%%%%%%%%%%%%%%%%%%%%%%%%
\subsection{Copyright}

Copyright \copyright{} 2017--2018 Niklas Beisert

This work may be distributed and/or modified under the
conditions of the \LaTeX{} Project Public License, either version 1.3
of this license or (at your option) any later version.
The latest version of this license is in
  \url{http://www.latex-project.org/lppl.txt}
and version 1.3 or later is part of all distributions of \LaTeX{}
version 2005/12/01 or later.

This work has the LPPL maintenance status `maintained'.

The Current Maintainer of this work is Niklas Beisert.

This work consists of the files |README.txt|, |childdoc.ins| and |childdoc.dtx|
as well as the derived files |childdoc.def|, |cdocsamp.tex|
with |cdocsch1.tex|, |cdocsch2.tex|, |cdocspt3.tex|, |cdocspt4.tex|,
|cdocsdrf.tex|, |cdocsfn1.tex|, |cdocsfn2.tex|
as well as |childdoc.pdf|.

%%%%%%%%%%%%%%%%%%%%%%%%%%%%%%%%%%%%%%%%%%%%%%%%%%%%%%%%%%%%%%%%%%%%%%%%%%%%%%%%
\subsection{Files and Installation}

The package consists of the files:
%
\begin{center}
\begin{tabular}{ll}
    |README.txt|   & readme file \\
    |childdoc.ins| & installation file \\
    |childdoc.dtx| & source file \\
    |childdoc.def| & definition file \\
    |cdocsamp.tex| & sample main file \\
    |cdocsch1.tex| & sample include file \\
    |cdocsch2.tex| & sample include file \\
    |cdocspt3.tex| & sample part file \\
    |cdocspt4.tex| & sample part file \\
    |cdocsdrf.tex| & sample redirection file \\
    |cdocsfn1.tex| & sample redirection file \\
    |cdocsfn2.tex| & sample redirection file \\
    |childdoc.pdf| & manual
\end{tabular}
\end{center}
%
The distribution consists of the files
|README.txt|, |childdoc.ins| and |childdoc.dtx|.
%
\begin{itemize}
\item
Run (pdf)\LaTeX{} on |childdoc.dtx|
to compile the manual |childdoc.pdf| (this file).
\item
Run \LaTeX{} on |childdoc.ins| to create the definitions file |childdoc.def|
and the sample |cdocsamp.tex| with include files
|cdocsch1.tex|, |cdocsch2.tex|, |cdocspt3.tex|, |cdocspt4.tex|,
|cdocsdrf.tex|, |cdocsfn1.tex|, |cdocsfn2.tex|.
Then copy the file |childdoc.def| to an appropriate directory of your \LaTeX{}
distribution, e.g.\ \textit{texmf-root}|/tex/latex/childdoc|.
\end{itemize}

%%%%%%%%%%%%%%%%%%%%%%%%%%%%%%%%%%%%%%%%%%%%%%%%%%%%%%%%%%%%%%%%%%%%%%%%%%%%%%%%
\subsection{Related CTAN Packages}

There are several other packages which offer a similar functionality:
%
\begin{itemize}
\item
The packages
\href{http://ctan.org/pkg/docmute}{\textsf{docmute}},
\href{http://ctan.org/pkg/includex}{\textsf{includex}} and
\href{http://ctan.org/pkg/standalone}{\textsf{standalone}}
provide commands to include only the document body of
a child file thus allowing both files to be compiled individually.
\item
The packages \href{http://ctan.org/pkg/subdocs}{\textsf{subdocs}}
and \href{http://ctan.org/pkg/subfiles}{\textsf{subfiles}}
provide structures in which the main and child documents can be
encapsulated and allowing them to be compiled individually.
The inclusion mechanism is different from the conventional |\include|.
\item
The package \href{http://ctan.org/pkg/combine}{\textsf{combine}}
is an elaborate solution to combine several documents into one.
\end{itemize}
%
See also the CTAN topic \href{http://ctan.org/topic/subdocs}{\textsf{subdocs}}
for further related packages.
The present package differs from the above solutions in that
a document structure constructed with the conventional |\include| mechanism
just needs two extra commands at the top of every file
such that all constituent files can be compiled individually.

%%%%%%%%%%%%%%%%%%%%%%%%%%%%%%%%%%%%%%%%%%%%%%%%%%%%%%%%%%%%%%%%%%%%%%%%%%%%%%%%
%\subsection{Feature Suggestions}
%
%The following is a list of features which may be useful for future
%versions of this package:
%%
%\begin{itemize}
%\item
%\ldots
%\end{itemize}

%%%%%%%%%%%%%%%%%%%%%%%%%%%%%%%%%%%%%%%%%%%%%%%%%%%%%%%%%%%%%%%%%%%%%%%%%%%%%%%%
\subsection{Revision History}

%%%%%%%%%%%%%%%%%%%%%%%%%%%%%%%%%%%%%%%%
\paragraph{v2.0:} 2018/12/30

\begin{itemize}
\item
immediate forward processing
\item
added |\childdocby| mechanism
\item
manual restructured
\end{itemize}

%%%%%%%%%%%%%%%%%%%%%%%%%%%%%%%%%%%%%%%%
\paragraph{v1.6:} 2018/01/17

\begin{itemize}
\item
application for development of include files
\item
corrections to manual
\end{itemize}

%%%%%%%%%%%%%%%%%%%%%%%%%%%%%%%%%%%%%%%%
\paragraph{v1.5:} 2017/05/21

\begin{itemize}
\item
more complete structuring introduced
\item
|\childdocof| introduced
\item
|\childdoc| renamed to |\childdocmain|
\item
|\childredirect| renamed to |\childdocforward| and |\childdocforwardprefix|
and functionality expanded
\end{itemize}

%%%%%%%%%%%%%%%%%%%%%%%%%%%%%%%%%%%%%%%%
\paragraph{v1.0:} 2017/04/27

\begin{itemize}
\item
manual and install package
\item
first version published on CTAN
\end{itemize}

%%%%%%%%%%%%%%%%%%%%%%%%%%%%%%%%%%%%%%%%
\paragraph{v0.6:} 2017/04/26

\begin{itemize}
\item
redirection mechanism added
\end{itemize}

%%%%%%%%%%%%%%%%%%%%%%%%%%%%%%%%%%%%%%%%
\paragraph{v0.5:} 2017/04/26

\begin{itemize}
\item
functionality in definition file
\end{itemize}


%%%%%%%%%%%%%%%%%%%%%%%%%%%%%%%%%%%%%%%%%%%%%%%%%%%%%%%%%%%%%%%%%%%%%%%%%%%%%%%%
%%%%%%%%%%%%%%%%%%%%%%%%%%%%%%%%%%%%%%%%%%%%%%%%%%%%%%%%%%%%%%%%%%%%%%%%%%%%%%%%
%%%%%%%%%%%%%%%%%%%%%%%%%%%%%%%%%%%%%%%%%%%%%%%%%%%%%%%%%%%%%%%%%%%%%%%%%%%%%%%%
\appendix

\settowidth\MacroIndent{\rmfamily\scriptsize 000\ }

 \DocInput{childdoc.dtx}

\end{document}
%</driver>
% \fi
%
% %%%%%%%%%%%%%%%%%%%%%%%%%%%%%%%%%%%%%%%%%%%%%%%%%%%%%%%%%%%%%%%%%%%%%%%%%%%%%%
% %%%%%%%%%%%%%%%%%%%%%%%%%%%%%%%%%%%%%%%%%%%%%%%%%%%%%%%%%%%%%%%%%%%%%%%%%%%%%%
% \section{Sample}
%\iffalse
%<*samplemain>
%\fi
%
% The following presents a sample document
% with two chapters, two parts, a title page,
% a compile flag as well as three forwarding files to set the flag.
% It consists of eight |.tex| files:
% \begin{center}
% \begin{tabular}{ll}
% |cdocsamp.tex|&main file\\
% |cdocsch1.tex|&include file for chapter 1\\
% |cdocsch2.tex|&include file for chapter 2\\
% |cdocspt3.tex|&include file for part 3\\
% |cdocspt4.tex|&include file for part 4\\
% |cdocsdrf.tex|&forwarding file for main file in draft mode\\
% |cdocsfi1.tex|&forwarding file for final version of chapter 1\\
% |cdocsfi2.tex|&forwarding file for final version of chapter 2\\
% \end{tabular}
% \end{center}
% Each of the eight files can be compiled directly by the \LaTeX{} compiler.
%
% %%%%%%%%%%%%%%%%%%%%%%%%%%%%%%%%%%%%%%
% \paragraph{Main File.}
%
% The main file is called |cdocsamp.tex|.
%
% Load the \textsf{childdoc} definitions and
% declare the filename for the main document:
%    \begin{macrocode}
\input{childdoc.def}
\childdocmain{}
%    \end{macrocode}

% Optional override for |\version| flag:
%    \begin{macrocode}
%%\ifchilddoc\else\providecommand{\version}{draft}\fi
%    \end{macrocode}

% Define the default values for the |\version| flag
% (|final| for the main file and |draft| for childs):
%    \begin{macrocode}
\ifchilddoc
\providecommand{\version}{draft}
\else
\providecommand{\version}{final}
\fi
%    \end{macrocode}

% Load the standard document class:
%    \begin{macrocode}
\documentclass[12pt]{article}
%    \end{macrocode}

% Start the document body:
%    \begin{macrocode}
\begin{document}
%    \end{macrocode}

% Declare a title page.
% Print title, part of document being processed and version flag:
%    \begin{macrocode}
\addtocounter{page}{-1}
\begin{center}
{\LARGE\bfseries{}childdoc example\par}
\vspace{1cm}
\ifchilddoc
\ifchilddocmanual part\else chapter\fi:
`\childdocname' of `\childdocjob'\par
\else
main document: `\childdocjob'\par
\fi
version: \version\par
\end{center}
\newpage
%    \end{macrocode}

% Manually include selected file,
% otherwise process as usual:
%    \begin{macrocode}
\ifchilddocmanual
\section*{part `\childdocname'}
\input{\childdocname}
\else
%    \end{macrocode}

% Include the two chapters:
%    \begin{macrocode}
\include{cdocsch1}
\include{cdocsch2}
%    \end{macrocode}

% Include the two parts unless only chapters should be displayed:
%    \begin{macrocode}
\ifchilddoc\else
\section{part three}
\input{cdocspt3}
\section{part four}
\input{cdocspt4}
\fi
%    \end{macrocode}

% Process as usual until here:
%    \begin{macrocode}
\fi
%    \end{macrocode}

% End of document body:
%    \begin{macrocode}
\end{document}
%    \end{macrocode}
%\iffalse
%</samplemain>
%\fi
%
% %%%%%%%%%%%%%%%%%%%%%%%%%%%%%%%%%%%%%%
% \paragraph{Chapter Include Files.}
%
% The include files are called |cdocsch1.tex| and |cdocsch2.tex|.
%
%\iffalse
%<*samplechap1|samplechap2>
%\fi

% Optional override for |\version| flag:
%    \begin{macrocode}
%%\providecommand{\version}{final}
%    \end{macrocode}

% Include the main document:
%    \begin{macrocode}
\input{childdoc.def}
\childdocof{cdocsamp}
%    \end{macrocode}

%\iffalse
%</samplechap1|samplechap2>
%\fi
%
%\iffalse
%<*samplechap1>
%\fi
% Some text for chapter 1:
%    \begin{macrocode}
\section{one}
some text in chapter one
%    \end{macrocode}

%\iffalse
%</samplechap1>
%\fi
% Some text for chapter 2:
%\iffalse
%<*samplechap2>
%\fi
%    \begin{macrocode}
\section{two}
more text in chapter two
%    \end{macrocode}

%\iffalse
%</samplechap2>
%\fi
%
% %%%%%%%%%%%%%%%%%%%%%%%%%%%%%%%%%%%%%%
% \paragraph{Part Include Files.}
%
% The include files are called |cdocspt3.tex| and |cdocspt4.tex|.
%
%\iffalse
%<*samplepart3|samplepart4>
%\fi

% Optional override for |\version| flag:
%    \begin{macrocode}
%%\providecommand{\version}{final}
%    \end{macrocode}

% Include the main document:
%    \begin{macrocode}
\input{childdoc.def}
\childdocby{cdocsamp}
%    \end{macrocode}

%\iffalse
%</samplepart3|samplepart4>
%\fi
%
%\iffalse
%<*samplepart3>
%\fi
% Some text for part 3:
%    \begin{macrocode}
some text in part three
%    \end{macrocode}

%\iffalse
%</samplepart3>
%\fi
% Some text for part 4:
%\iffalse
%<*samplepart4>
%\fi
%    \begin{macrocode}
more text in part four
%    \end{macrocode}

%\iffalse
%</samplepart4>
%\fi
%
% %%%%%%%%%%%%%%%%%%%%%%%%%%%%%%%%%%%%%%
% \paragraph{Forwarding for a Complete Draft.}
%
% The following forwarding file |cdocsdrf.tex|
% compiles the main document in draft mode:
%\iffalse
%<*sampledraft>
%\fi
%    \begin{macrocode}
\def\version{draft}
\input{childdoc.def}
\childdocforward{cdocsamp}
%    \end{macrocode}

%\iffalse
%</sampledraft>
%\fi
%
% %%%%%%%%%%%%%%%%%%%%%%%%%%%%%%%%%%%%%%
% \paragraph{Forwarding for Final Version of the Chapters.}
%
% The following forwarding files |cdocsfn1.tex| and |cdocsfn2.tex|
% (with identical content)
% compile the final versions of the child documents
% |cdocsch1.tex| and |cdocsch2.tex|, respectively:
%\iffalse
%<*samplefinal>
%\fi
%    \begin{macrocode}
\def\version{final}
\input{childdoc.def}
\childdocforwardprefix[cdocsamp]{cdocsfn}{cdocsch}
%    \end{macrocode}

%\iffalse
%</samplefinal>
%\fi
%
% %%%%%%%%%%%%%%%%%%%%%%%%%%%%%%%%%%%%%%
% \paragraph{Command Line Processing.}
%
% The following three command lines generate the output files
% |cdocscld|, |cdocscl1| and |cdocscl2|
% which should be identical to
% |cdocsdrf|, |cdocsch1| and |cdocsfn2|, respectively:
% \begin{center}
% \begin{tabular}{l}
% |latex -jobname cdocscld \|\\
% |  "\def\version{draft}\input{childdoc.def}\childdocforward{cdocsamp}"|\\
% |latex -jobname cdocscl1 \|\\
% |  "\input{childdoc.def}\childdocforward[cdocsamp]{cdocsch1}"|\\
% |latex -jobname cdocscl2 \|\\
% |  "\def\version{final}\input{childdoc.def}\childdocforward{cdocsch2}"|
% \end{tabular}
% \end{center}
% Note that the trailing backslash on each first line
% merely continues the input to the second line
% (for convenient cut ant paste).
% Furthermore, the command |latex| can be replaced by any
% of its alternative versions such as |pdflatex|.
%
% %%%%%%%%%%%%%%%%%%%%%%%%%%%%%%%%%%%%%%%%%%%%%%%%%%%%%%%%%%%%%%%%%%%%%%%%%%%%%%
% %%%%%%%%%%%%%%%%%%%%%%%%%%%%%%%%%%%%%%%%%%%%%%%%%%%%%%%%%%%%%%%%%%%%%%%%%%%%%%
% \section{Implementation}
%\iffalse
%<*package>
%\fi
%
% This section describes the definitions file |childdoc.def|.

% The definitions cannot be loaded using |\usepackage| or |\RequirePackage|
% which has a mechanism to prevent loading a style file more than once.
% When loading the definitions by means of |\input|
% multiple instances have to be prevented manually:
%\iffalse
%This code needs to be before the `\ProvidesFile' directive
%which is defined at the beginning of this file.
%Therefore it is also placed there and commented out here.
%</package>
%<*discard>
%\fi
%    \begin{macrocode}
\ifdefined\childdocmain\endinput\fi
%    \end{macrocode}
%\iffalse
%</discard>
%<*package>
%\fi
%
% \macro{\ifchilddoc}
% \macro{\ifchilddocmanual}
% The conditional |\ifchilddoc| tells whether a
% child (true) or main (false) document is being compiled.
% The conditional |\ifchilddocmanual| tells whether
% the |\includeonly| mechanism is used (false) or
% the selection of child files must be performed manually (true).
% The definitions initialise to false:
%    \begin{macrocode}
\newif\ifchilddoc
\newif\ifchilddocmanual
%    \end{macrocode}

% \macro{\childdocname}
% \macro{\childdocjob}
% The macro |\childdocname| stores the name of the main document
% to be compiled. The macro |\childdocjob| stores the name of
% the document on which the \LaTeX{} compiler was originally invoked.
% The content of |\jobname| cannot be compared
% to filenames specified in the source due to different catcodes.
% The following code rescans |\jobname|, stores the result
% in |\childdocname| and saves a copy in |\childdocjob|:
%    \begin{macrocode}
\edef\childdocname{\scantokens\expandafter{\jobname\noexpand}}
\let\childdocjob\childdocname
%    \end{macrocode}

% \macro{\childdocdisable}
% The macro |\childdocdisable| prevents the main file
% from being processed more than once.
% At this stage, the main document command |\childdocmain|
% is assumed to be called once again where it should do nothing.
% Any subsequent call to it should prevent
% a secondary processing of the main document
% It overwrites the forwarding commands
% |\childdocof| and |\childdocforward|
% with empty macros to prevent further inclusions of the main document:
%    \begin{macrocode}
\newcommand{\childdocdisable}
{
  \renewcommand{\childdocmain}[1]{\renewcommand{\childdocmain}[1]{\endinput}}
  \renewcommand{\childdocof}[1]{}
  \renewcommand{\childdocby}[2][]{}
  \renewcommand{\childdocforward}[2][]{}
  \renewcommand{\childdocdisable}{}
}
%    \end{macrocode}

% \macro{\childdocmain}
% The macro |\childdocmain| is to be called at the top of the main file
% with nothing or the main filename (without extension) as argument.
% First, it breaks loops.
% If the argument is not empty and does not match |\childdocname|
% (which is set by the first inclusion of |childdoc.def|),
% |\ifchilddoc| is set to true, |\includeonly| is applied to the child file
% and |\jobname| is set to the main file
% (for proper handling of |.aux| files):
%    \begin{macrocode}
\newcommand{\childdocmain}[1]
{
  \childdocdisable\childdocmain{}
  \if?#1?\else
    \begingroup
      \def\childdoctmp{#1}
      \ifx\childdoctmp\childdocname
        \def\childdoctmp{}
      \else
        \def\childdoctmp
        {
          \childdoctrue
          \includeonly{\childdocname}
          \def\childdocjob{#1}
          \def\jobname{#1}
        }
      \fi
      \expandafter
    \endgroup
    \childdoctmp
  \fi
}
%    \end{macrocode}

% \macro{\childdocof}
% The command |\childdocof| redirects
% compilation to the main file |#1|.
%    \begin{macrocode}
\newcommand{\childdocof}[1]
{
  \childdocdisable
  \childdoctrue
  \includeonly{\childdocname}
  \def\jobname{#1}
  \def\childdocjob{#1}
  \input{#1}
}
%    \end{macrocode}

% \macro{\childdocby}
% The command |\childdocby| ....
%    \begin{macrocode}
\newcommand{\childdocby}[2][]
{
  \childdocdisable
  \childdoctrue
  \childdocmanualtrue
  \if?#1?\else
    \def\jobname{#2}
  \fi
  \def\childdocjob{#2}
  \input{#2}
  \endinput
}
%    \end{macrocode}

% \macro{\childdocforward}
% The command |\childdocforward| redirects
% compilation to the main file or
% (if the optional argument is given) a child file.
% Parameters are set as if the main file
% or a child file starting with |\childdocof| was compiled.
% Then compilation is handed over to the main file:
%    \begin{macrocode}
\newcommand{\childdocforward}[2][]
{
  \begingroup
    \if?#1?
      \def\childdoctmp
      {
        \def\childdocname{#2}
        \def\childdocjob{#2}
        \def\jobname{#2}
        \input{#2}
        \endinput
      }
    \else
      \def\childdoctmp
      {
        \childdocdisable
        \def\childdocname{#2}
        \childdoctrue
        \includeonly{#2}
        \def\childdocjob{#1}
        \def\jobname{#1}
        \input{#1}
        \endinput
      }
    \fi
    \expandafter
  \endgroup
  \childdoctmp
}
%    \end{macrocode}

% \macro{\childdocforwardprefix}
% The command |\childdocforwardprefix| redirects
% compilation to the main or a child file by means of a pattern.
% The prefix |#1| in the current filename is replaced by |#2|
% and the suffix of the current filename is kept
% (it is assumed that the filename does not contain the substring `|~~~|'
% which is used as a delimiter).
% Compilation is handed over to the new file by |\childdocforward|:
%    \begin{macrocode}
\newcommand{\childdocforwardprefix}[3][]
{
  \begingroup
    \def\childdocextract #2##1~~~{\def\childdoctmp{\childdocforward[#1]{#3##1}}}
    \expandafter\childdocextract\childdocname~~~
    \expandafter
  \endgroup
  \childdoctmp
}
%    \end{macrocode}

% \macro{\childdoc}
% The deprecated macro |\childdoc| is a legacy version of |\childdocmain|:
%    \begin{macrocode}
\newcommand{\childdoc}{\childdocmain}
%    \end{macrocode}

% \macro{\childdocredirect}
% The deprecated macro |\childdocredirect| is a legacy version
% of |\childdocforward| and |\childdocforwardprefix|:
%    \begin{macrocode}
\newcommand{\childdocredirect}[2][]
{
  \begingroup
    \if?#1?
      \def\childdoctmp{\childdocforward{#2}}
    \else
      \def\childdoctmp{\childdocforwardprefix{#1}{#2}}
    \fi
    \expandafter
  \endgroup
  \childdoctmp
}
%    \end{macrocode}

%\iffalse
%</package>
%\fi
%
\endinput
|\\
|\childdocforwardprefix[|\textit{main}|]{|\textit{prefix}|}{|\textit{dest}|}|
\end{tabular}
\end{center}
%
the destination file is determined by a pattern
depending on the current file:
To make this work, the current file must be called
`{\textit{prefix}\hspace{0.2em}\textit{suffix}}'
with \textit{prefix} matching precisely the argument.
Processing is then passed on to the file
`{\textit{dest}\hspace{0.2em}\textit{suffix}}'.
Surely, the same effect is achieved by
directly specifying the
argument `{\textit{dest}\hspace{0.2em}\textit{suffix}}'
in the first form.
However, that requires to set up a different file
for each child. With the alternative form of the command
all these files can have exactly the same content
which simplifies setting them up and maintaining them.

For example, the following file |draft.tex|
with a compilation flag |\version| as described in \secref{sec:flags}
compiles the main document as a draft:
%
\begin{center}
\begin{tabular}{l}
|\def\version{draft}|\\
|% \iffalse
%
% childdoc.dtx Copyright (C) 2017-2018 Niklas Beisert
%
% This work may be distributed and/or modified under the
% conditions of the LaTeX Project Public License, either version 1.3
% of this license or (at your option) any later version.
% The latest version of this license is in
%   http://www.latex-project.org/lppl.txt
% and version 1.3 or later is part of all distributions of LaTeX
% version 2005/12/01 or later.
%
% This work has the LPPL maintenance status `maintained'.
%
% The Current Maintainer of this work is Niklas Beisert.
%
% This work consists of the files childdoc.dtx and childdoc.ins
% and the derived files childdoc.def and cdocsamp.tex with
% cdocsch1.tex, cdocsch2.tex, cdocsdrf.tex, cdocsfn1.tex, cdocsfn2.tex.
%
%<package>\ifdefined\childdocmain\endinput\fi
%<package>\ProvidesFile{childdoc.def}[2018/12/30 v2.0 child document driver]
%<samplemain>\ProvidesFile{cdocsamp.tex}[2018/12/30 v2.0 sample for childdoc]
%<*driver>
%\ProvidesFile{childdoc.drv}[2018/12/30 v2.0 childdoc reference manual file]
\PassOptionsToClass{10pt,a4paper}{article}
\documentclass{ltxdoc}

\usepackage[margin=35mm]{geometry}
\usepackage{hyperref}
\usepackage{hyperxmp}
\usepackage[usenames]{color}

\hypersetup{colorlinks=true}
\hypersetup{pdfstartview=FitH}
\hypersetup{pdfpagemode=UseNone}
\hypersetup{pdfsource={}}
\hypersetup{pdflang={en-UK}}
\hypersetup{pdfcopyright={Copyright 2017-2018 Niklas Beisert.
  This work may be distributed and/or modified under the
  conditions of the LaTeX Project Public License, either version 1.3
  of this license or (at your option) any later version.}}
\hypersetup{pdflicenseurl={http://www.latex-project.org/lppl.txt}}
\hypersetup{pdfcontactaddress={ETH Zurich, ITP, HIT K,
  Wolfgang-Pauli-Strasse 27}}
\hypersetup{pdfcontactpostcode={8093}}
\hypersetup{pdfcontactcity={Zurich}}
\hypersetup{pdfcontactcountry={Switzerland}}
\hypersetup{pdfcontactemail={nbeisert@itp.phys.ethz.ch}}
\hypersetup{pdfcontacturl={http://people.phys.ethz.ch/\xmptilde nbeisert/}}

\newcommand{\secref}[1]{\hyperref[#1]{section \ref*{#1}}}

\parskip1ex
\parindent0pt
\let\olditemize\itemize
\def\itemize{\olditemize\parskip0pt}

\begin{document}

\title{The \textsf{childdoc} Package}
\hypersetup{pdftitle={The childdoc Package}}
\author{Niklas Beisert\\[2ex]
  Institut f\"ur Theoretische Physik\\
  Eidgen\"ossische Technische Hochschule Z\"urich\\
  Wolfgang-Pauli-Strasse 27, 8093 Z\"urich, Switzerland\\[1ex]
  \href{mailto:nbeisert@itp.phys.ethz.ch}
  {\texttt{nbeisert@itp.phys.ethz.ch}}}
\hypersetup{pdfauthor={Niklas Beisert}}
\hypersetup{pdfsubject={Manual for the LaTeX2e Package childdoc}}
\date{30 December 2018, \textsf{v2.0}}
\maketitle

\begin{abstract}\noindent
\textsf{childdoc} is a \LaTeXe{} package
that enables the direct compilation
of document sections included by |\include|
to individual files.
\end{abstract}

\begingroup
\parskip0ex
\tableofcontents
\endgroup

%%%%%%%%%%%%%%%%%%%%%%%%%%%%%%%%%%%%%%%%%%%%%%%%%%%%%%%%%%%%%%%%%%%%%%%%%%%%%%%%
%%%%%%%%%%%%%%%%%%%%%%%%%%%%%%%%%%%%%%%%%%%%%%%%%%%%%%%%%%%%%%%%%%%%%%%%%%%%%%%%
\section{Introduction}

\LaTeX{} provides a mechanism to structure a large document (such as a book)
into a main file and several child files (containing the chapters)
using the |\include| command.
This mechanism is beneficial for documents
which span hundreds of pages in order to
make the source file(s) more manageable.
Moreover, compilation can be restricted to
selected child files by means of the |\includeonly| command.
The latter feature can be used to reduce the compilation time while editing
(this was significantly more useful in the earlier days of \LaTeX{})
or to generate a smaller document which is easier to navigate.
Another application of |\includeonly| is to generate
documents consisting of selected parts of the complete document.

However, there are a few drawbacks of the plain |\include| mechanism:
\begin{itemize}
\item
The child files cannot be compiled on their own,
they can only be compiled via the main file.
A naive editing environment
(such as a text editor with an option
to have the current file processed by \LaTeX)
may require one to switch to the main file before compiling;
attempting to compile the child file produces errors.
\item
The main file must be modified (each time)
to adjust the |\includeonly| command
to the present needs. This easily leaves the main file in a messy state.
\item
The generated document will always carry the filename
of the main document. This is inconvenient if
several child files are to be compiled and
to be kept for distribution.
\end{itemize}

The present package provides a simple interface
to make child files individually compilable by \LaTeX{}.
Compiling a child file then has the same effect as compiling
the main file with an |\includeonly| command
to select the appropriate child.
Moreover the generated document will carry the name of the child
rather than the main file.
This resolves all three above issues.

This feature is meant to make the editing of books,
thesis documents and lecture notes somewhat more convenient.
However, the package can also be used efficiently for
composing a series of documents (such as exercise sheets)
which are typically distributed individually.
It then assists the author in generating the individual documents
(potentially in different versions)
as well as a document containing the collected series.
Another application is in developing style files
or other kinds of included material
where compilation of the style file could redirect
to a sample or test file.

%%%%%%%%%%%%%%%%%%%%%%%%%%%%%%%%%%%%%%%%%%%%%%%%%%%%%%%%%%%%%%%%%%%%%%%%%%%%%%%%
%%%%%%%%%%%%%%%%%%%%%%%%%%%%%%%%%%%%%%%%%%%%%%%%%%%%%%%%%%%%%%%%%%%%%%%%%%%%%%%%
\section{Usage}

First of all, the package \textsf{childdoc} is \emph{not} a standard
\LaTeXe{} |.sty| style file! Therefore it needs to be invoked in
a non-standard way.

%%%%%%%%%%%%%%%%%%%%%%%%%%%%%%%%%%%%%%%%%%%%%%%%%%%%%%%%%%%%%%%%%%%%%%%%%%%%%%%%
\subsection{Included Files}
\label{sec:include}

%%%%%%%%%%%%%%%%%%%%%%%%%%%%%%%%%%%%%%%%
\DescribeMacro{\childdocmain}
To use the package, add the commands
\begin{center}
\begin{tabular}{l}
|\input{childdoc.def}|\\
|\childdocmain{}|\\
\end{tabular}
\end{center}
at the very top of the main \LaTeX{} file,
in particular \emph{before} the |\documentclass| statement!
The argument of |\childdocmain| should be left empty
(but it must be present).

%%%%%%%%%%%%%%%%%%%%%%%%%%%%%%%%%%%%%%%%
\DescribeMacro{\childdocof}
Furthermore, add the commands
\begin{center}
\begin{tabular}{l}
|\input{childdoc.def}|\\
|\childdocof{|\textit{main}|}|\\
\end{tabular}
\end{center}
at the top of every child file \textit{child}
which is included by |\include{|\textit{child}|}|
from within the main file
(or at least for those files to be compiled individually).
The argument \textit{main} must be the filename of the main file.

There are a couple of
considerations in setting up the main and child documents:

%%%%%%%%%%%%%%%%%%%%%%%%%%%%%%%%%%%%%%%%
\paragraph{Restrictions.}

Please note the following restrictions:
\begin{itemize}
\item
|\childdocmain| must be called with one argument \textit{main}
to ensure compatibility with earlier version of the package.
It must either be empty (|\childdocmain{}|)
or precisely match the filename of the main file in which it is specified.
See \secref{sec:detection} for further information.
\item
The filename \textit{main} must be specified without the |.tex| extension.
\item
The filename \textit{main} is case sensitive
(even in case-insensitive file systems)
due to internal string comparison.
\item
The argument \textit{main} should be fully expanded, it cannot be a macro.
\item
Subdirectories and special characters should be avoided in filenames.
\item
The command |\childdocmain{|\textit{main}|}| must be followed by a whitespace.
It should not be followed immediately by another command
or by a comment mark `|%|'.
This is because the \TeX{} parser reads the token immediately following
the argument of |\childdocmain| and puts it
at the beginning of every child section;
however, a white\-space is ignored.
\end{itemize}

%%%%%%%%%%%%%%%%%%%%%%%%%%%%%%%%%%%%%%%%
\paragraph{Content of Main File.}

It is advisable to place all content in the child files included by |\include|.
Any output contained in the main file will appear in all child documents
unless suppressed manually;
it cannot be suppressed automatically by the |\includeonly| directive
and thus should normally be avoided.
A method to include some content in the main file
by means of conditional processing is described in \secref{sec:conditional}.

%%%%%%%%%%%%%%%%%%%%%%%%%%%%%%%%%%%%%%%%
\paragraph{Page Numbering.}

When only a part of the document is compiled,
the appropriate numbering of pages
(as well as other status parameters)
is determined from the |.aux| files.
The latter contain information from previous passes.
However this information needs to propagate through
all intermediate child documents.
Therefore the page numbering in child documents may well
be inconsistent until the complete document is compiled at least once.

A useful (if unconventional) way to always ensure a consistent
page numbering is to restart the numbering in each child document
and denote the pages by `\textit{child}|.|\textit{page}'
where \textit{child} represents the chapter/section number of the child file.
This can be achieved by the command
|\numberwithin{page}{|\textit{child}|}|
of the \textsf{amsmath} package
where \textit{child} can be |chapter| or |section|
depending on the chosen structuring.
Alternatively, one can modify the macro |\thepage| appropriately
and reset the counter |page| at the start of each child file.

%%%%%%%%%%%%%%%%%%%%%%%%%%%%%%%%%%%%%%%%%%%%%%%%%%%%%%%%%%%%%%%%%%%%%%%%%%%%%%%%
\subsection{Conditional Processing}
\label{sec:conditional}

The package provides a mechanism to compile different versions
of a document. To customise the versions further some conditional processing
can come in handy to distinguish which version is being compiled.
The package provides two macros to describe the compilation context:

%%%%%%%%%%%%%%%%%%%%%%%%%%%%%%%%%%%%%%%%
\DescribeMacro{\ifchilddoc}
The conditional |\ifchilddoc| distinguishes between the compilation of
child documents and the main document:
%
\begin{center}
|\ifchilddoc |\textit{child-code}| |[|\||else |\textit{main-code}]| \||fi|
\end{center}

%%%%%%%%%%%%%%%%%%%%%%%%%%%%%%%%%%%%%%%%
\DescribeMacro{\childdocname}
\DescribeMacro{\childdocjob}
The macro |\childdocname| contains the filename (without extension)
of the main or child file being processed.
Note that |\childdocjob| will always contain the name of the main file.

%%%%%%%%%%%%%%%%%%%%%%%%%%%%%%%%%%%%%%%%
\paragraph{Title Page.}

Conditional processing can be used to include a title or banner page
in the main document when proper precautions are taken.
Importantly, the code in the main file should ensure that the page counter
(as well as other status parameters which are stored in the |.aux| files)
takes the same value after the conditional processing.
Otherwise the page numbers may take divergent values
depending on which part is compiled.

For example, a title page could be declared by:
%
\begin{center}
\begin{tabular}{l}
|\ifchilddoc\||else|\\
|\addtocounter{page}{-1}|\\
\textit{code for title page}\\
|\newpage|\\
|\||fi|
\end{tabular}
\end{center}
%
A banner page for the child documents can be generated by:
%
\begin{center}
\begin{tabular}{l}
|\ifchilddoc|\\
|\addtocounter{page}{-1}|\\
\textit{code for banner page}\\
|\newpage|\\
|\||fi|
\end{tabular}
\end{center}
%
Here one could write a message such as:
\begin{center}
|This is the part \childdocname{} of \childdocjob{}.|
\end{center}

%%%%%%%%%%%%%%%%%%%%%%%%%%%%%%%%%%%%%%%%%%%%%%%%%%%%%%%%%%%%%%%%%%%%%%%%%%%%%%%%
\subsection{Flags}
\label{sec:flags}

The package makes it easy to generate different versions
of the main or child documents.
To this end compilation flags can be defined
and assigned different default values.
They will be particularly useful in conjunction
with the forwarding mechanism described in \secref{sec:forward}.

For example, it may be useful to have a flag |\version|
which can be set to |draft| or |final|.
The document source will contain some conditional code
depending on the value of |\version|.
Suppose further, the flag should default to |final| for the main file
and to |draft| for child files
which is a natural assignment for editing the document.
This is achieved by placing the following code
in the preamble of the main document
(below the |\childdocmain| directive):
%
\begin{center}
\begin{tabular}{l}
|\ifchilddoc|\\
|\providecommand{\version}{draft}|\\
|\||else|\\
|\providecommand{\version}{final}|\\
|\||fi|
\end{tabular}
\end{center}
%
The definition by |\providecommand| makes sure
that previous definitions are not overwritten.
Further statements |\providecommand{\version}{...}|
can thus be added before the above code to override it.

For the main file, one might add a line
(between |\childdocmain| and the above block)
%
\begin{center}
|%\ifchilddoc\||else\providecommand{\version}{draft}\||fi|
\end{center}
%
which can be uncommented to produce a draft version.
Likewise one can add a line to the very top of a child file
(above the |\childdocof{|\textit{main}|}| directive)
%
\begin{center}
|%\providecommand{\version}{final}|
\end{center}
%
which can be uncommented to produce the final version of this child document.

%%%%%%%%%%%%%%%%%%%%%%%%%%%%%%%%%%%%%%%%%%%%%%%%%%%%%%%%%%%%%%%%%%%%%%%%%%%%%%%%
\subsection{Forwarding}
\label{sec:forward}

Different versions of the main or child documents
using compilation flags as described in \secref{sec:flags}
can be (permanently) stored in different files
for convenient compilation, viewing and distribution.
To this end, the package defines a command
to pass on compilation to a different file:

%%%%%%%%%%%%%%%%%%%%%%%%%%%%%%%%%%%%%%%%
\DescribeMacro{\childdocforward}
The command |\childdocforward| redirects processing to
another source file:
%
\begin{center}
\begin{tabular}{l}
|\input{childdoc.def}|\\
|\childdocforward[|\textit{main}|]{|\textit{dest}|}|\\
\end{tabular}
\end{center}
%
The argument \textit{dest} is the destination file
(without extension).
It should be the main file or one of the child files.
Note that further \textsf{childdoc} directives
such as |\childdocof| and |\childdocforward|
in the indicated file will be processed in this form.
The optional argument \textit{main}
passes on directly to the main file \textit{main}
while pretending to compile the child \textit{dest}.
This form behaves as if \textit{dest}
issues |\childdocof{|\textit{main}|}| right away,
and no further \textsf{childdoc} directives will be processed.

%%%%%%%%%%%%%%%%%%%%%%%%%%%%%%%%%%%%%%%%
\DescribeMacro{\...prefix}
In the alternative form |\childdocforwardprefix|,
%
\begin{center}
\begin{tabular}{l}
|\input{childdoc.def}|\\
|\childdocforwardprefix[|\textit{main}|]{|\textit{prefix}|}{|\textit{dest}|}|
\end{tabular}
\end{center}
%
the destination file is determined by a pattern
depending on the current file:
To make this work, the current file must be called
`{\textit{prefix}\hspace{0.2em}\textit{suffix}}'
with \textit{prefix} matching precisely the argument.
Processing is then passed on to the file
`{\textit{dest}\hspace{0.2em}\textit{suffix}}'.
Surely, the same effect is achieved by
directly specifying the
argument `{\textit{dest}\hspace{0.2em}\textit{suffix}}'
in the first form.
However, that requires to set up a different file
for each child. With the alternative form of the command
all these files can have exactly the same content
which simplifies setting them up and maintaining them.

For example, the following file |draft.tex|
with a compilation flag |\version| as described in \secref{sec:flags}
compiles the main document as a draft:
%
\begin{center}
\begin{tabular}{l}
|\def\version{draft}|\\
|\input{childdoc.def}|\\
|\childdocforward{|\textit{main}|}|
\end{tabular}
\end{center}
%
Likewise, the following files |final|\textit{nn}|.tex|
compile the final version of the child document
|child|\textit{nn}|.tex|:
%
\begin{center}
\begin{tabular}{l}
|\def\version{final}|\\
|\input{childdoc.def}|\\
|\childdocforwardprefix{final}{child}|
\end{tabular}
\end{center}
%

Note that when several versions of a main file and/or of each child file
are to be generated, it may be convenient to set up a |Makefile| or
shell script to automatise the process.

%%%%%%%%%%%%%%%%%%%%%%%%%%%%%%%%%%%%%%%%%%%%%%%%%%%%%%%%%%%%%%%%%%%%%%%%%%%%%%%%
\subsection{Command Line Processing}
\label{sec:commandline}

The effect of redirection files can also be achieved by invoking
the \LaTeX{} compiler with a more elaborate command line.
Most conveniently this should be done as part
of a shell script or a |Makefile|.

When using \textsf{childdoc} in the main file, the following
command lines effectively perform a redirection
(note that depending on the shell being used,
backslashes may have to be doubled: `|\|' $\to$ `|\\|'):
%
\begin{center}
|... -jobname "|\textit{target}|" |\\|"|[\textit{flags}]%
|\input{childdoc.def}\childdocforward[|\textit{main}|]{|\textit{dest}|}"|
\end{center}
%
Here \textit{target} is the name of the output file,
\textit{main} is the name of the main file
and \textit{dest} is the name of the main or child file to be processed
(all filenames without extensions).
The optional argument \textit{main} can be omitted
if \textit{main} matches \textit{dest}.
Optionally, compilation \textit{flags} can be defined via |\def| commands.
This command line makes the \TeX{} engine believe
it is compiling the file \textit{target}
whose content is specified as the latter parameter.
The provided code then forwards the processing to
\textit{main} or \textit{dest} as described in \secref{sec:forward}.

%%%%%%%%%%%%%%%%%%%%%%%%%%%%%%%%%%%%%%%%%%%%%%%%%%%%%%%%%%%%%%%%%%%%%%%%%%%%%%%%
\subsection{Include by Input}
\label{sec:input}

Including child documents by |\include| has some restrictions by design.
Most notably, the content of a child document always occupies
its own set of pages; pages cannot be shared between child documents.
Usually, this behaviour makes perfect sense
because each child document contain an essential part of the document.
However, in some situations it may be desirable to compose
a document from a collection of parts
without having mandatory page breaks between then.
For this case, the package
provides a mechanism to include parts
by |\input| which can also be processed individually.
However, by construction this mechanism
requires manual handling of the content to be output.

%%%%%%%%%%%%%%%%%%%%%%%%%%%%%%%%%%%%%%%%
\DescribeMacro{\ifchilddocmanual}
The main file should be prepared as usual, see \secref{sec:include}.
However, the document body must make a distinction
between processing of an individual part and of the main document, e.g.:
%
\begin{center}
\begin{tabular}{l}
|\ifchilddocmanual|\\
|\input{\childdocname}|\\
|\||else|\\
\textit{document body with }|\input{|\textit{part}|}|\\
|\||fi|
\end{tabular}
\end{center}
%
The conditional |\ifchilddocmanual| is true whenever
a part to be included by |\input| is being compiled,
and the name of the part is stored in |\childdocname|.

%%%%%%%%%%%%%%%%%%%%%%%%%%%%%%%%%%%%%%%%
\DescribeMacro{\childdocby}
Each part to be included by |\input| should start with:
%
\begin{center}
\begin{tabular}{l}
|\input{childdoc.def}|\\
|\childdocby{|\textit{main}|}|\\
\end{tabular}
\end{center}
%
The directive |\childdocby| is similar to |\childdocof|
described in \secref{sec:include},
but the subsequent selection of content must be done manually.
To that end, both |\ifchilddoc| and |\ifchilddocmanual|
will be true upon processing of a part,
and the name of the part is stored in |\childdocname|.
Note that |\jobname| will be set to the filename of the current part
so that each part receives an individual |.aux| file
that does not interfere with the |.aux| file(s) of the main document.
This behaviour can be altered by the alternative form
|\childdocby[*]{|\textit{main}|}| (with a non-empty optional argument)
which uses the |.aux| file of the main document
by setting |\jobname| to \textit{main}.

%%%%%%%%%%%%%%%%%%%%%%%%%%%%%%%%%%%%%%%%%%%%%%%%%%%%%%%%%%%%%%%%%%%%%%%%%%%%%%%%
\subsection{Driver Development}
\label{sec:driver}

The \textsf{childdoc} mechanism can also be use for the development
of definition files such as \LaTeX{} styles or classes.
This case differs from the above setup with multiple parts
included by |\include| in that no |\includeonly| should be invoked.
This can be achieved by starting the include file
(before |\ProvidesPackage|) with:
%
\begin{center}
\begin{tabular}{l}
|\input{childdoc.def}|\\
|\childdocforward{|\textit{main}|}|\\
\end{tabular}
\end{center}
%
or alternatively with:
%
\begin{center}
\begin{tabular}{l}
|\input{childdoc.def}|\\
|\childdocby{|\textit{main}|}|\\
\end{tabular}
\end{center}
%
Both forms have slightly different effects as described above.
The main file is prepared as usual, see \secref{sec:include}.

%%%%%%%%%%%%%%%%%%%%%%%%%%%%%%%%%%%%%%%%%%%%%%%%%%%%%%%%%%%%%%%%%%%%%%%%%%%%%%%%
\subsection{Legacy Detection}
\label{sec:detection}

The directive |\childdocmain| in the main file can detect
whether the complete document or merely a child is to be compiled
even without using the directive |\childdocof|.
This method is deprecated because it is less robust
and there is no compelling reason to use it;
it is merely provided for backward compatibility
and it may be removed in future versions.

If the detection mechanism is to be used,
it is mandatory to correctly specify
the filename of the main file as the argument of |\childdocmain|:
%
\begin{center}
\begin{tabular}{l}
|\input{childdoc.def}|\\
|\childdocmain{|\textit{main}|}|\\
\end{tabular}
\end{center}
%
If |\jobname| does not match the argument \textit{main} of |\childdocmain|,
it is assumed that |\jobname| points to the child file to be compiled.
When using |\childdocmain| with the main file specified as argument,
it suffices to start a child file
with just |\input{|\textit{main}|}|
without loading of the package and using |\childdocof|.
If instead all processing is done
with the appropriate \textsf{childdoc} directives,
the argument of \textit{main} of |\childdocmain| can be empty.

An alternative version of the command line processing described
in \secref{sec:commandline} using the detection mechanism reads:
%
\begin{center}
|... -jobname "|\textit{target}|" "|[\textit{flags}]%
[|\def\jobname{|\textit{dest}|}|]|\input{|\textit{main}|}"|
\end{center}

%%%%%%%%%%%%%%%%%%%%%%%%%%%%%%%%%%%%%%%%%%%%%%%%%%%%%%%%%%%%%%%%%%%%%%%%%%%%%%%%
\subsection{Manual Code}
\label{sec:manual}

In case one cannot be certain whether the definitions file |childdoc.def|
is installed on the target \TeX{} distribution
and one prefers not to ship it,
it is conceivable to paste a few relevant commands into the sources.

To that end, drop all statements |\input{childdoc.def}|
and perform the replacements as outlined below.
Instead of |\childdocmain{|\textit{main}|}| add the following code
to the top of the main file:
%
\begin{center}
\begin{tabular}{l}
|\||ifdefined\childdocname\endinput\||fi\newif\ifchilddoc|\\
|\edef\childdocname{\scantokens\expandafter{\jobname\noexpand}}|\\
|\def\childdocmain{|\textit{main}|}\||ifx\childdocmain\childdocname\||else|\\
|\childdoctrue\includeonly{\childdocname}\let\jobname\childdocmain\||fi|\\
\end{tabular}
\end{center}
%
Instead of |\childdocof{|\textit{main}|}| just include the main file
at the top of each child file:
%
\begin{center}
|\input{|\textit{main}|}|
\end{center}
%
A simple redirection |\childdocforward{|\textit{dest}|}| is achieved by:
%
\begin{center}
|\def\jobname{|\textit{dest}|}\input{\jobname}|
\end{center}
%
The redirection with prefix
|\childdocforwardprefix[|\textit{prefix}|]{|\textit{dest}|}|
is accomplished by:
%
\begin{center}
\begin{tabular}{l}
|{\edef\jobname{\scantokens\expandafter{\jobname\noexpand}}|\\
|\def\redirectjob |\textit{prefix}|#1~~~{\gdef\jobname{|\textit{dest}|#1}}|\\
|\expandafter\redirectjob\jobname~~~}\input{\jobname}|
\end{tabular}
\end{center}

In an alternative approach,
child documents can be compiled by a specific command line
without additional code or specific definitions:
%
\begin{center}
|... -jobname "|\textit{target}|" "|[\textit{flags}]%
|\includeonly{|\textit{dest}|}\input{|\textit{main}|}"|
\end{center}
%

%%%%%%%%%%%%%%%%%%%%%%%%%%%%%%%%%%%%%%%%%%%%%%%%%%%%%%%%%%%%%%%%%%%%%%%%%%%%%%%%
%%%%%%%%%%%%%%%%%%%%%%%%%%%%%%%%%%%%%%%%%%%%%%%%%%%%%%%%%%%%%%%%%%%%%%%%%%%%%%%%
\section{Information}

%%%%%%%%%%%%%%%%%%%%%%%%%%%%%%%%%%%%%%%%%%%%%%%%%%%%%%%%%%%%%%%%%%%%%%%%%%%%%%%%
\subsection{Copyright}

Copyright \copyright{} 2017--2018 Niklas Beisert

This work may be distributed and/or modified under the
conditions of the \LaTeX{} Project Public License, either version 1.3
of this license or (at your option) any later version.
The latest version of this license is in
  \url{http://www.latex-project.org/lppl.txt}
and version 1.3 or later is part of all distributions of \LaTeX{}
version 2005/12/01 or later.

This work has the LPPL maintenance status `maintained'.

The Current Maintainer of this work is Niklas Beisert.

This work consists of the files |README.txt|, |childdoc.ins| and |childdoc.dtx|
as well as the derived files |childdoc.def|, |cdocsamp.tex|
with |cdocsch1.tex|, |cdocsch2.tex|, |cdocspt3.tex|, |cdocspt4.tex|,
|cdocsdrf.tex|, |cdocsfn1.tex|, |cdocsfn2.tex|
as well as |childdoc.pdf|.

%%%%%%%%%%%%%%%%%%%%%%%%%%%%%%%%%%%%%%%%%%%%%%%%%%%%%%%%%%%%%%%%%%%%%%%%%%%%%%%%
\subsection{Files and Installation}

The package consists of the files:
%
\begin{center}
\begin{tabular}{ll}
    |README.txt|   & readme file \\
    |childdoc.ins| & installation file \\
    |childdoc.dtx| & source file \\
    |childdoc.def| & definition file \\
    |cdocsamp.tex| & sample main file \\
    |cdocsch1.tex| & sample include file \\
    |cdocsch2.tex| & sample include file \\
    |cdocspt3.tex| & sample part file \\
    |cdocspt4.tex| & sample part file \\
    |cdocsdrf.tex| & sample redirection file \\
    |cdocsfn1.tex| & sample redirection file \\
    |cdocsfn2.tex| & sample redirection file \\
    |childdoc.pdf| & manual
\end{tabular}
\end{center}
%
The distribution consists of the files
|README.txt|, |childdoc.ins| and |childdoc.dtx|.
%
\begin{itemize}
\item
Run (pdf)\LaTeX{} on |childdoc.dtx|
to compile the manual |childdoc.pdf| (this file).
\item
Run \LaTeX{} on |childdoc.ins| to create the definitions file |childdoc.def|
and the sample |cdocsamp.tex| with include files
|cdocsch1.tex|, |cdocsch2.tex|, |cdocspt3.tex|, |cdocspt4.tex|,
|cdocsdrf.tex|, |cdocsfn1.tex|, |cdocsfn2.tex|.
Then copy the file |childdoc.def| to an appropriate directory of your \LaTeX{}
distribution, e.g.\ \textit{texmf-root}|/tex/latex/childdoc|.
\end{itemize}

%%%%%%%%%%%%%%%%%%%%%%%%%%%%%%%%%%%%%%%%%%%%%%%%%%%%%%%%%%%%%%%%%%%%%%%%%%%%%%%%
\subsection{Related CTAN Packages}

There are several other packages which offer a similar functionality:
%
\begin{itemize}
\item
The packages
\href{http://ctan.org/pkg/docmute}{\textsf{docmute}},
\href{http://ctan.org/pkg/includex}{\textsf{includex}} and
\href{http://ctan.org/pkg/standalone}{\textsf{standalone}}
provide commands to include only the document body of
a child file thus allowing both files to be compiled individually.
\item
The packages \href{http://ctan.org/pkg/subdocs}{\textsf{subdocs}}
and \href{http://ctan.org/pkg/subfiles}{\textsf{subfiles}}
provide structures in which the main and child documents can be
encapsulated and allowing them to be compiled individually.
The inclusion mechanism is different from the conventional |\include|.
\item
The package \href{http://ctan.org/pkg/combine}{\textsf{combine}}
is an elaborate solution to combine several documents into one.
\end{itemize}
%
See also the CTAN topic \href{http://ctan.org/topic/subdocs}{\textsf{subdocs}}
for further related packages.
The present package differs from the above solutions in that
a document structure constructed with the conventional |\include| mechanism
just needs two extra commands at the top of every file
such that all constituent files can be compiled individually.

%%%%%%%%%%%%%%%%%%%%%%%%%%%%%%%%%%%%%%%%%%%%%%%%%%%%%%%%%%%%%%%%%%%%%%%%%%%%%%%%
%\subsection{Feature Suggestions}
%
%The following is a list of features which may be useful for future
%versions of this package:
%%
%\begin{itemize}
%\item
%\ldots
%\end{itemize}

%%%%%%%%%%%%%%%%%%%%%%%%%%%%%%%%%%%%%%%%%%%%%%%%%%%%%%%%%%%%%%%%%%%%%%%%%%%%%%%%
\subsection{Revision History}

%%%%%%%%%%%%%%%%%%%%%%%%%%%%%%%%%%%%%%%%
\paragraph{v2.0:} 2018/12/30

\begin{itemize}
\item
immediate forward processing
\item
added |\childdocby| mechanism
\item
manual restructured
\end{itemize}

%%%%%%%%%%%%%%%%%%%%%%%%%%%%%%%%%%%%%%%%
\paragraph{v1.6:} 2018/01/17

\begin{itemize}
\item
application for development of include files
\item
corrections to manual
\end{itemize}

%%%%%%%%%%%%%%%%%%%%%%%%%%%%%%%%%%%%%%%%
\paragraph{v1.5:} 2017/05/21

\begin{itemize}
\item
more complete structuring introduced
\item
|\childdocof| introduced
\item
|\childdoc| renamed to |\childdocmain|
\item
|\childredirect| renamed to |\childdocforward| and |\childdocforwardprefix|
and functionality expanded
\end{itemize}

%%%%%%%%%%%%%%%%%%%%%%%%%%%%%%%%%%%%%%%%
\paragraph{v1.0:} 2017/04/27

\begin{itemize}
\item
manual and install package
\item
first version published on CTAN
\end{itemize}

%%%%%%%%%%%%%%%%%%%%%%%%%%%%%%%%%%%%%%%%
\paragraph{v0.6:} 2017/04/26

\begin{itemize}
\item
redirection mechanism added
\end{itemize}

%%%%%%%%%%%%%%%%%%%%%%%%%%%%%%%%%%%%%%%%
\paragraph{v0.5:} 2017/04/26

\begin{itemize}
\item
functionality in definition file
\end{itemize}


%%%%%%%%%%%%%%%%%%%%%%%%%%%%%%%%%%%%%%%%%%%%%%%%%%%%%%%%%%%%%%%%%%%%%%%%%%%%%%%%
%%%%%%%%%%%%%%%%%%%%%%%%%%%%%%%%%%%%%%%%%%%%%%%%%%%%%%%%%%%%%%%%%%%%%%%%%%%%%%%%
%%%%%%%%%%%%%%%%%%%%%%%%%%%%%%%%%%%%%%%%%%%%%%%%%%%%%%%%%%%%%%%%%%%%%%%%%%%%%%%%
\appendix

\settowidth\MacroIndent{\rmfamily\scriptsize 000\ }

 \DocInput{childdoc.dtx}

\end{document}
%</driver>
% \fi
%
% %%%%%%%%%%%%%%%%%%%%%%%%%%%%%%%%%%%%%%%%%%%%%%%%%%%%%%%%%%%%%%%%%%%%%%%%%%%%%%
% %%%%%%%%%%%%%%%%%%%%%%%%%%%%%%%%%%%%%%%%%%%%%%%%%%%%%%%%%%%%%%%%%%%%%%%%%%%%%%
% \section{Sample}
%\iffalse
%<*samplemain>
%\fi
%
% The following presents a sample document
% with two chapters, two parts, a title page,
% a compile flag as well as three forwarding files to set the flag.
% It consists of eight |.tex| files:
% \begin{center}
% \begin{tabular}{ll}
% |cdocsamp.tex|&main file\\
% |cdocsch1.tex|&include file for chapter 1\\
% |cdocsch2.tex|&include file for chapter 2\\
% |cdocspt3.tex|&include file for part 3\\
% |cdocspt4.tex|&include file for part 4\\
% |cdocsdrf.tex|&forwarding file for main file in draft mode\\
% |cdocsfi1.tex|&forwarding file for final version of chapter 1\\
% |cdocsfi2.tex|&forwarding file for final version of chapter 2\\
% \end{tabular}
% \end{center}
% Each of the eight files can be compiled directly by the \LaTeX{} compiler.
%
% %%%%%%%%%%%%%%%%%%%%%%%%%%%%%%%%%%%%%%
% \paragraph{Main File.}
%
% The main file is called |cdocsamp.tex|.
%
% Load the \textsf{childdoc} definitions and
% declare the filename for the main document:
%    \begin{macrocode}
\input{childdoc.def}
\childdocmain{}
%    \end{macrocode}

% Optional override for |\version| flag:
%    \begin{macrocode}
%%\ifchilddoc\else\providecommand{\version}{draft}\fi
%    \end{macrocode}

% Define the default values for the |\version| flag
% (|final| for the main file and |draft| for childs):
%    \begin{macrocode}
\ifchilddoc
\providecommand{\version}{draft}
\else
\providecommand{\version}{final}
\fi
%    \end{macrocode}

% Load the standard document class:
%    \begin{macrocode}
\documentclass[12pt]{article}
%    \end{macrocode}

% Start the document body:
%    \begin{macrocode}
\begin{document}
%    \end{macrocode}

% Declare a title page.
% Print title, part of document being processed and version flag:
%    \begin{macrocode}
\addtocounter{page}{-1}
\begin{center}
{\LARGE\bfseries{}childdoc example\par}
\vspace{1cm}
\ifchilddoc
\ifchilddocmanual part\else chapter\fi:
`\childdocname' of `\childdocjob'\par
\else
main document: `\childdocjob'\par
\fi
version: \version\par
\end{center}
\newpage
%    \end{macrocode}

% Manually include selected file,
% otherwise process as usual:
%    \begin{macrocode}
\ifchilddocmanual
\section*{part `\childdocname'}
\input{\childdocname}
\else
%    \end{macrocode}

% Include the two chapters:
%    \begin{macrocode}
\include{cdocsch1}
\include{cdocsch2}
%    \end{macrocode}

% Include the two parts unless only chapters should be displayed:
%    \begin{macrocode}
\ifchilddoc\else
\section{part three}
\input{cdocspt3}
\section{part four}
\input{cdocspt4}
\fi
%    \end{macrocode}

% Process as usual until here:
%    \begin{macrocode}
\fi
%    \end{macrocode}

% End of document body:
%    \begin{macrocode}
\end{document}
%    \end{macrocode}
%\iffalse
%</samplemain>
%\fi
%
% %%%%%%%%%%%%%%%%%%%%%%%%%%%%%%%%%%%%%%
% \paragraph{Chapter Include Files.}
%
% The include files are called |cdocsch1.tex| and |cdocsch2.tex|.
%
%\iffalse
%<*samplechap1|samplechap2>
%\fi

% Optional override for |\version| flag:
%    \begin{macrocode}
%%\providecommand{\version}{final}
%    \end{macrocode}

% Include the main document:
%    \begin{macrocode}
\input{childdoc.def}
\childdocof{cdocsamp}
%    \end{macrocode}

%\iffalse
%</samplechap1|samplechap2>
%\fi
%
%\iffalse
%<*samplechap1>
%\fi
% Some text for chapter 1:
%    \begin{macrocode}
\section{one}
some text in chapter one
%    \end{macrocode}

%\iffalse
%</samplechap1>
%\fi
% Some text for chapter 2:
%\iffalse
%<*samplechap2>
%\fi
%    \begin{macrocode}
\section{two}
more text in chapter two
%    \end{macrocode}

%\iffalse
%</samplechap2>
%\fi
%
% %%%%%%%%%%%%%%%%%%%%%%%%%%%%%%%%%%%%%%
% \paragraph{Part Include Files.}
%
% The include files are called |cdocspt3.tex| and |cdocspt4.tex|.
%
%\iffalse
%<*samplepart3|samplepart4>
%\fi

% Optional override for |\version| flag:
%    \begin{macrocode}
%%\providecommand{\version}{final}
%    \end{macrocode}

% Include the main document:
%    \begin{macrocode}
\input{childdoc.def}
\childdocby{cdocsamp}
%    \end{macrocode}

%\iffalse
%</samplepart3|samplepart4>
%\fi
%
%\iffalse
%<*samplepart3>
%\fi
% Some text for part 3:
%    \begin{macrocode}
some text in part three
%    \end{macrocode}

%\iffalse
%</samplepart3>
%\fi
% Some text for part 4:
%\iffalse
%<*samplepart4>
%\fi
%    \begin{macrocode}
more text in part four
%    \end{macrocode}

%\iffalse
%</samplepart4>
%\fi
%
% %%%%%%%%%%%%%%%%%%%%%%%%%%%%%%%%%%%%%%
% \paragraph{Forwarding for a Complete Draft.}
%
% The following forwarding file |cdocsdrf.tex|
% compiles the main document in draft mode:
%\iffalse
%<*sampledraft>
%\fi
%    \begin{macrocode}
\def\version{draft}
\input{childdoc.def}
\childdocforward{cdocsamp}
%    \end{macrocode}

%\iffalse
%</sampledraft>
%\fi
%
% %%%%%%%%%%%%%%%%%%%%%%%%%%%%%%%%%%%%%%
% \paragraph{Forwarding for Final Version of the Chapters.}
%
% The following forwarding files |cdocsfn1.tex| and |cdocsfn2.tex|
% (with identical content)
% compile the final versions of the child documents
% |cdocsch1.tex| and |cdocsch2.tex|, respectively:
%\iffalse
%<*samplefinal>
%\fi
%    \begin{macrocode}
\def\version{final}
\input{childdoc.def}
\childdocforwardprefix[cdocsamp]{cdocsfn}{cdocsch}
%    \end{macrocode}

%\iffalse
%</samplefinal>
%\fi
%
% %%%%%%%%%%%%%%%%%%%%%%%%%%%%%%%%%%%%%%
% \paragraph{Command Line Processing.}
%
% The following three command lines generate the output files
% |cdocscld|, |cdocscl1| and |cdocscl2|
% which should be identical to
% |cdocsdrf|, |cdocsch1| and |cdocsfn2|, respectively:
% \begin{center}
% \begin{tabular}{l}
% |latex -jobname cdocscld \|\\
% |  "\def\version{draft}\input{childdoc.def}\childdocforward{cdocsamp}"|\\
% |latex -jobname cdocscl1 \|\\
% |  "\input{childdoc.def}\childdocforward[cdocsamp]{cdocsch1}"|\\
% |latex -jobname cdocscl2 \|\\
% |  "\def\version{final}\input{childdoc.def}\childdocforward{cdocsch2}"|
% \end{tabular}
% \end{center}
% Note that the trailing backslash on each first line
% merely continues the input to the second line
% (for convenient cut ant paste).
% Furthermore, the command |latex| can be replaced by any
% of its alternative versions such as |pdflatex|.
%
% %%%%%%%%%%%%%%%%%%%%%%%%%%%%%%%%%%%%%%%%%%%%%%%%%%%%%%%%%%%%%%%%%%%%%%%%%%%%%%
% %%%%%%%%%%%%%%%%%%%%%%%%%%%%%%%%%%%%%%%%%%%%%%%%%%%%%%%%%%%%%%%%%%%%%%%%%%%%%%
% \section{Implementation}
%\iffalse
%<*package>
%\fi
%
% This section describes the definitions file |childdoc.def|.

% The definitions cannot be loaded using |\usepackage| or |\RequirePackage|
% which has a mechanism to prevent loading a style file more than once.
% When loading the definitions by means of |\input|
% multiple instances have to be prevented manually:
%\iffalse
%This code needs to be before the `\ProvidesFile' directive
%which is defined at the beginning of this file.
%Therefore it is also placed there and commented out here.
%</package>
%<*discard>
%\fi
%    \begin{macrocode}
\ifdefined\childdocmain\endinput\fi
%    \end{macrocode}
%\iffalse
%</discard>
%<*package>
%\fi
%
% \macro{\ifchilddoc}
% \macro{\ifchilddocmanual}
% The conditional |\ifchilddoc| tells whether a
% child (true) or main (false) document is being compiled.
% The conditional |\ifchilddocmanual| tells whether
% the |\includeonly| mechanism is used (false) or
% the selection of child files must be performed manually (true).
% The definitions initialise to false:
%    \begin{macrocode}
\newif\ifchilddoc
\newif\ifchilddocmanual
%    \end{macrocode}

% \macro{\childdocname}
% \macro{\childdocjob}
% The macro |\childdocname| stores the name of the main document
% to be compiled. The macro |\childdocjob| stores the name of
% the document on which the \LaTeX{} compiler was originally invoked.
% The content of |\jobname| cannot be compared
% to filenames specified in the source due to different catcodes.
% The following code rescans |\jobname|, stores the result
% in |\childdocname| and saves a copy in |\childdocjob|:
%    \begin{macrocode}
\edef\childdocname{\scantokens\expandafter{\jobname\noexpand}}
\let\childdocjob\childdocname
%    \end{macrocode}

% \macro{\childdocdisable}
% The macro |\childdocdisable| prevents the main file
% from being processed more than once.
% At this stage, the main document command |\childdocmain|
% is assumed to be called once again where it should do nothing.
% Any subsequent call to it should prevent
% a secondary processing of the main document
% It overwrites the forwarding commands
% |\childdocof| and |\childdocforward|
% with empty macros to prevent further inclusions of the main document:
%    \begin{macrocode}
\newcommand{\childdocdisable}
{
  \renewcommand{\childdocmain}[1]{\renewcommand{\childdocmain}[1]{\endinput}}
  \renewcommand{\childdocof}[1]{}
  \renewcommand{\childdocby}[2][]{}
  \renewcommand{\childdocforward}[2][]{}
  \renewcommand{\childdocdisable}{}
}
%    \end{macrocode}

% \macro{\childdocmain}
% The macro |\childdocmain| is to be called at the top of the main file
% with nothing or the main filename (without extension) as argument.
% First, it breaks loops.
% If the argument is not empty and does not match |\childdocname|
% (which is set by the first inclusion of |childdoc.def|),
% |\ifchilddoc| is set to true, |\includeonly| is applied to the child file
% and |\jobname| is set to the main file
% (for proper handling of |.aux| files):
%    \begin{macrocode}
\newcommand{\childdocmain}[1]
{
  \childdocdisable\childdocmain{}
  \if?#1?\else
    \begingroup
      \def\childdoctmp{#1}
      \ifx\childdoctmp\childdocname
        \def\childdoctmp{}
      \else
        \def\childdoctmp
        {
          \childdoctrue
          \includeonly{\childdocname}
          \def\childdocjob{#1}
          \def\jobname{#1}
        }
      \fi
      \expandafter
    \endgroup
    \childdoctmp
  \fi
}
%    \end{macrocode}

% \macro{\childdocof}
% The command |\childdocof| redirects
% compilation to the main file |#1|.
%    \begin{macrocode}
\newcommand{\childdocof}[1]
{
  \childdocdisable
  \childdoctrue
  \includeonly{\childdocname}
  \def\jobname{#1}
  \def\childdocjob{#1}
  \input{#1}
}
%    \end{macrocode}

% \macro{\childdocby}
% The command |\childdocby| ....
%    \begin{macrocode}
\newcommand{\childdocby}[2][]
{
  \childdocdisable
  \childdoctrue
  \childdocmanualtrue
  \if?#1?\else
    \def\jobname{#2}
  \fi
  \def\childdocjob{#2}
  \input{#2}
  \endinput
}
%    \end{macrocode}

% \macro{\childdocforward}
% The command |\childdocforward| redirects
% compilation to the main file or
% (if the optional argument is given) a child file.
% Parameters are set as if the main file
% or a child file starting with |\childdocof| was compiled.
% Then compilation is handed over to the main file:
%    \begin{macrocode}
\newcommand{\childdocforward}[2][]
{
  \begingroup
    \if?#1?
      \def\childdoctmp
      {
        \def\childdocname{#2}
        \def\childdocjob{#2}
        \def\jobname{#2}
        \input{#2}
        \endinput
      }
    \else
      \def\childdoctmp
      {
        \childdocdisable
        \def\childdocname{#2}
        \childdoctrue
        \includeonly{#2}
        \def\childdocjob{#1}
        \def\jobname{#1}
        \input{#1}
        \endinput
      }
    \fi
    \expandafter
  \endgroup
  \childdoctmp
}
%    \end{macrocode}

% \macro{\childdocforwardprefix}
% The command |\childdocforwardprefix| redirects
% compilation to the main or a child file by means of a pattern.
% The prefix |#1| in the current filename is replaced by |#2|
% and the suffix of the current filename is kept
% (it is assumed that the filename does not contain the substring `|~~~|'
% which is used as a delimiter).
% Compilation is handed over to the new file by |\childdocforward|:
%    \begin{macrocode}
\newcommand{\childdocforwardprefix}[3][]
{
  \begingroup
    \def\childdocextract #2##1~~~{\def\childdoctmp{\childdocforward[#1]{#3##1}}}
    \expandafter\childdocextract\childdocname~~~
    \expandafter
  \endgroup
  \childdoctmp
}
%    \end{macrocode}

% \macro{\childdoc}
% The deprecated macro |\childdoc| is a legacy version of |\childdocmain|:
%    \begin{macrocode}
\newcommand{\childdoc}{\childdocmain}
%    \end{macrocode}

% \macro{\childdocredirect}
% The deprecated macro |\childdocredirect| is a legacy version
% of |\childdocforward| and |\childdocforwardprefix|:
%    \begin{macrocode}
\newcommand{\childdocredirect}[2][]
{
  \begingroup
    \if?#1?
      \def\childdoctmp{\childdocforward{#2}}
    \else
      \def\childdoctmp{\childdocforwardprefix{#1}{#2}}
    \fi
    \expandafter
  \endgroup
  \childdoctmp
}
%    \end{macrocode}

%\iffalse
%</package>
%\fi
%
\endinput
|\\
|\childdocforward{|\textit{main}|}|
\end{tabular}
\end{center}
%
Likewise, the following files |final|\textit{nn}|.tex|
compile the final version of the child document
|child|\textit{nn}|.tex|:
%
\begin{center}
\begin{tabular}{l}
|\def\version{final}|\\
|% \iffalse
%
% childdoc.dtx Copyright (C) 2017-2018 Niklas Beisert
%
% This work may be distributed and/or modified under the
% conditions of the LaTeX Project Public License, either version 1.3
% of this license or (at your option) any later version.
% The latest version of this license is in
%   http://www.latex-project.org/lppl.txt
% and version 1.3 or later is part of all distributions of LaTeX
% version 2005/12/01 or later.
%
% This work has the LPPL maintenance status `maintained'.
%
% The Current Maintainer of this work is Niklas Beisert.
%
% This work consists of the files childdoc.dtx and childdoc.ins
% and the derived files childdoc.def and cdocsamp.tex with
% cdocsch1.tex, cdocsch2.tex, cdocsdrf.tex, cdocsfn1.tex, cdocsfn2.tex.
%
%<package>\ifdefined\childdocmain\endinput\fi
%<package>\ProvidesFile{childdoc.def}[2018/12/30 v2.0 child document driver]
%<samplemain>\ProvidesFile{cdocsamp.tex}[2018/12/30 v2.0 sample for childdoc]
%<*driver>
%\ProvidesFile{childdoc.drv}[2018/12/30 v2.0 childdoc reference manual file]
\PassOptionsToClass{10pt,a4paper}{article}
\documentclass{ltxdoc}

\usepackage[margin=35mm]{geometry}
\usepackage{hyperref}
\usepackage{hyperxmp}
\usepackage[usenames]{color}

\hypersetup{colorlinks=true}
\hypersetup{pdfstartview=FitH}
\hypersetup{pdfpagemode=UseNone}
\hypersetup{pdfsource={}}
\hypersetup{pdflang={en-UK}}
\hypersetup{pdfcopyright={Copyright 2017-2018 Niklas Beisert.
  This work may be distributed and/or modified under the
  conditions of the LaTeX Project Public License, either version 1.3
  of this license or (at your option) any later version.}}
\hypersetup{pdflicenseurl={http://www.latex-project.org/lppl.txt}}
\hypersetup{pdfcontactaddress={ETH Zurich, ITP, HIT K,
  Wolfgang-Pauli-Strasse 27}}
\hypersetup{pdfcontactpostcode={8093}}
\hypersetup{pdfcontactcity={Zurich}}
\hypersetup{pdfcontactcountry={Switzerland}}
\hypersetup{pdfcontactemail={nbeisert@itp.phys.ethz.ch}}
\hypersetup{pdfcontacturl={http://people.phys.ethz.ch/\xmptilde nbeisert/}}

\newcommand{\secref}[1]{\hyperref[#1]{section \ref*{#1}}}

\parskip1ex
\parindent0pt
\let\olditemize\itemize
\def\itemize{\olditemize\parskip0pt}

\begin{document}

\title{The \textsf{childdoc} Package}
\hypersetup{pdftitle={The childdoc Package}}
\author{Niklas Beisert\\[2ex]
  Institut f\"ur Theoretische Physik\\
  Eidgen\"ossische Technische Hochschule Z\"urich\\
  Wolfgang-Pauli-Strasse 27, 8093 Z\"urich, Switzerland\\[1ex]
  \href{mailto:nbeisert@itp.phys.ethz.ch}
  {\texttt{nbeisert@itp.phys.ethz.ch}}}
\hypersetup{pdfauthor={Niklas Beisert}}
\hypersetup{pdfsubject={Manual for the LaTeX2e Package childdoc}}
\date{30 December 2018, \textsf{v2.0}}
\maketitle

\begin{abstract}\noindent
\textsf{childdoc} is a \LaTeXe{} package
that enables the direct compilation
of document sections included by |\include|
to individual files.
\end{abstract}

\begingroup
\parskip0ex
\tableofcontents
\endgroup

%%%%%%%%%%%%%%%%%%%%%%%%%%%%%%%%%%%%%%%%%%%%%%%%%%%%%%%%%%%%%%%%%%%%%%%%%%%%%%%%
%%%%%%%%%%%%%%%%%%%%%%%%%%%%%%%%%%%%%%%%%%%%%%%%%%%%%%%%%%%%%%%%%%%%%%%%%%%%%%%%
\section{Introduction}

\LaTeX{} provides a mechanism to structure a large document (such as a book)
into a main file and several child files (containing the chapters)
using the |\include| command.
This mechanism is beneficial for documents
which span hundreds of pages in order to
make the source file(s) more manageable.
Moreover, compilation can be restricted to
selected child files by means of the |\includeonly| command.
The latter feature can be used to reduce the compilation time while editing
(this was significantly more useful in the earlier days of \LaTeX{})
or to generate a smaller document which is easier to navigate.
Another application of |\includeonly| is to generate
documents consisting of selected parts of the complete document.

However, there are a few drawbacks of the plain |\include| mechanism:
\begin{itemize}
\item
The child files cannot be compiled on their own,
they can only be compiled via the main file.
A naive editing environment
(such as a text editor with an option
to have the current file processed by \LaTeX)
may require one to switch to the main file before compiling;
attempting to compile the child file produces errors.
\item
The main file must be modified (each time)
to adjust the |\includeonly| command
to the present needs. This easily leaves the main file in a messy state.
\item
The generated document will always carry the filename
of the main document. This is inconvenient if
several child files are to be compiled and
to be kept for distribution.
\end{itemize}

The present package provides a simple interface
to make child files individually compilable by \LaTeX{}.
Compiling a child file then has the same effect as compiling
the main file with an |\includeonly| command
to select the appropriate child.
Moreover the generated document will carry the name of the child
rather than the main file.
This resolves all three above issues.

This feature is meant to make the editing of books,
thesis documents and lecture notes somewhat more convenient.
However, the package can also be used efficiently for
composing a series of documents (such as exercise sheets)
which are typically distributed individually.
It then assists the author in generating the individual documents
(potentially in different versions)
as well as a document containing the collected series.
Another application is in developing style files
or other kinds of included material
where compilation of the style file could redirect
to a sample or test file.

%%%%%%%%%%%%%%%%%%%%%%%%%%%%%%%%%%%%%%%%%%%%%%%%%%%%%%%%%%%%%%%%%%%%%%%%%%%%%%%%
%%%%%%%%%%%%%%%%%%%%%%%%%%%%%%%%%%%%%%%%%%%%%%%%%%%%%%%%%%%%%%%%%%%%%%%%%%%%%%%%
\section{Usage}

First of all, the package \textsf{childdoc} is \emph{not} a standard
\LaTeXe{} |.sty| style file! Therefore it needs to be invoked in
a non-standard way.

%%%%%%%%%%%%%%%%%%%%%%%%%%%%%%%%%%%%%%%%%%%%%%%%%%%%%%%%%%%%%%%%%%%%%%%%%%%%%%%%
\subsection{Included Files}
\label{sec:include}

%%%%%%%%%%%%%%%%%%%%%%%%%%%%%%%%%%%%%%%%
\DescribeMacro{\childdocmain}
To use the package, add the commands
\begin{center}
\begin{tabular}{l}
|\input{childdoc.def}|\\
|\childdocmain{}|\\
\end{tabular}
\end{center}
at the very top of the main \LaTeX{} file,
in particular \emph{before} the |\documentclass| statement!
The argument of |\childdocmain| should be left empty
(but it must be present).

%%%%%%%%%%%%%%%%%%%%%%%%%%%%%%%%%%%%%%%%
\DescribeMacro{\childdocof}
Furthermore, add the commands
\begin{center}
\begin{tabular}{l}
|\input{childdoc.def}|\\
|\childdocof{|\textit{main}|}|\\
\end{tabular}
\end{center}
at the top of every child file \textit{child}
which is included by |\include{|\textit{child}|}|
from within the main file
(or at least for those files to be compiled individually).
The argument \textit{main} must be the filename of the main file.

There are a couple of
considerations in setting up the main and child documents:

%%%%%%%%%%%%%%%%%%%%%%%%%%%%%%%%%%%%%%%%
\paragraph{Restrictions.}

Please note the following restrictions:
\begin{itemize}
\item
|\childdocmain| must be called with one argument \textit{main}
to ensure compatibility with earlier version of the package.
It must either be empty (|\childdocmain{}|)
or precisely match the filename of the main file in which it is specified.
See \secref{sec:detection} for further information.
\item
The filename \textit{main} must be specified without the |.tex| extension.
\item
The filename \textit{main} is case sensitive
(even in case-insensitive file systems)
due to internal string comparison.
\item
The argument \textit{main} should be fully expanded, it cannot be a macro.
\item
Subdirectories and special characters should be avoided in filenames.
\item
The command |\childdocmain{|\textit{main}|}| must be followed by a whitespace.
It should not be followed immediately by another command
or by a comment mark `|%|'.
This is because the \TeX{} parser reads the token immediately following
the argument of |\childdocmain| and puts it
at the beginning of every child section;
however, a white\-space is ignored.
\end{itemize}

%%%%%%%%%%%%%%%%%%%%%%%%%%%%%%%%%%%%%%%%
\paragraph{Content of Main File.}

It is advisable to place all content in the child files included by |\include|.
Any output contained in the main file will appear in all child documents
unless suppressed manually;
it cannot be suppressed automatically by the |\includeonly| directive
and thus should normally be avoided.
A method to include some content in the main file
by means of conditional processing is described in \secref{sec:conditional}.

%%%%%%%%%%%%%%%%%%%%%%%%%%%%%%%%%%%%%%%%
\paragraph{Page Numbering.}

When only a part of the document is compiled,
the appropriate numbering of pages
(as well as other status parameters)
is determined from the |.aux| files.
The latter contain information from previous passes.
However this information needs to propagate through
all intermediate child documents.
Therefore the page numbering in child documents may well
be inconsistent until the complete document is compiled at least once.

A useful (if unconventional) way to always ensure a consistent
page numbering is to restart the numbering in each child document
and denote the pages by `\textit{child}|.|\textit{page}'
where \textit{child} represents the chapter/section number of the child file.
This can be achieved by the command
|\numberwithin{page}{|\textit{child}|}|
of the \textsf{amsmath} package
where \textit{child} can be |chapter| or |section|
depending on the chosen structuring.
Alternatively, one can modify the macro |\thepage| appropriately
and reset the counter |page| at the start of each child file.

%%%%%%%%%%%%%%%%%%%%%%%%%%%%%%%%%%%%%%%%%%%%%%%%%%%%%%%%%%%%%%%%%%%%%%%%%%%%%%%%
\subsection{Conditional Processing}
\label{sec:conditional}

The package provides a mechanism to compile different versions
of a document. To customise the versions further some conditional processing
can come in handy to distinguish which version is being compiled.
The package provides two macros to describe the compilation context:

%%%%%%%%%%%%%%%%%%%%%%%%%%%%%%%%%%%%%%%%
\DescribeMacro{\ifchilddoc}
The conditional |\ifchilddoc| distinguishes between the compilation of
child documents and the main document:
%
\begin{center}
|\ifchilddoc |\textit{child-code}| |[|\||else |\textit{main-code}]| \||fi|
\end{center}

%%%%%%%%%%%%%%%%%%%%%%%%%%%%%%%%%%%%%%%%
\DescribeMacro{\childdocname}
\DescribeMacro{\childdocjob}
The macro |\childdocname| contains the filename (without extension)
of the main or child file being processed.
Note that |\childdocjob| will always contain the name of the main file.

%%%%%%%%%%%%%%%%%%%%%%%%%%%%%%%%%%%%%%%%
\paragraph{Title Page.}

Conditional processing can be used to include a title or banner page
in the main document when proper precautions are taken.
Importantly, the code in the main file should ensure that the page counter
(as well as other status parameters which are stored in the |.aux| files)
takes the same value after the conditional processing.
Otherwise the page numbers may take divergent values
depending on which part is compiled.

For example, a title page could be declared by:
%
\begin{center}
\begin{tabular}{l}
|\ifchilddoc\||else|\\
|\addtocounter{page}{-1}|\\
\textit{code for title page}\\
|\newpage|\\
|\||fi|
\end{tabular}
\end{center}
%
A banner page for the child documents can be generated by:
%
\begin{center}
\begin{tabular}{l}
|\ifchilddoc|\\
|\addtocounter{page}{-1}|\\
\textit{code for banner page}\\
|\newpage|\\
|\||fi|
\end{tabular}
\end{center}
%
Here one could write a message such as:
\begin{center}
|This is the part \childdocname{} of \childdocjob{}.|
\end{center}

%%%%%%%%%%%%%%%%%%%%%%%%%%%%%%%%%%%%%%%%%%%%%%%%%%%%%%%%%%%%%%%%%%%%%%%%%%%%%%%%
\subsection{Flags}
\label{sec:flags}

The package makes it easy to generate different versions
of the main or child documents.
To this end compilation flags can be defined
and assigned different default values.
They will be particularly useful in conjunction
with the forwarding mechanism described in \secref{sec:forward}.

For example, it may be useful to have a flag |\version|
which can be set to |draft| or |final|.
The document source will contain some conditional code
depending on the value of |\version|.
Suppose further, the flag should default to |final| for the main file
and to |draft| for child files
which is a natural assignment for editing the document.
This is achieved by placing the following code
in the preamble of the main document
(below the |\childdocmain| directive):
%
\begin{center}
\begin{tabular}{l}
|\ifchilddoc|\\
|\providecommand{\version}{draft}|\\
|\||else|\\
|\providecommand{\version}{final}|\\
|\||fi|
\end{tabular}
\end{center}
%
The definition by |\providecommand| makes sure
that previous definitions are not overwritten.
Further statements |\providecommand{\version}{...}|
can thus be added before the above code to override it.

For the main file, one might add a line
(between |\childdocmain| and the above block)
%
\begin{center}
|%\ifchilddoc\||else\providecommand{\version}{draft}\||fi|
\end{center}
%
which can be uncommented to produce a draft version.
Likewise one can add a line to the very top of a child file
(above the |\childdocof{|\textit{main}|}| directive)
%
\begin{center}
|%\providecommand{\version}{final}|
\end{center}
%
which can be uncommented to produce the final version of this child document.

%%%%%%%%%%%%%%%%%%%%%%%%%%%%%%%%%%%%%%%%%%%%%%%%%%%%%%%%%%%%%%%%%%%%%%%%%%%%%%%%
\subsection{Forwarding}
\label{sec:forward}

Different versions of the main or child documents
using compilation flags as described in \secref{sec:flags}
can be (permanently) stored in different files
for convenient compilation, viewing and distribution.
To this end, the package defines a command
to pass on compilation to a different file:

%%%%%%%%%%%%%%%%%%%%%%%%%%%%%%%%%%%%%%%%
\DescribeMacro{\childdocforward}
The command |\childdocforward| redirects processing to
another source file:
%
\begin{center}
\begin{tabular}{l}
|\input{childdoc.def}|\\
|\childdocforward[|\textit{main}|]{|\textit{dest}|}|\\
\end{tabular}
\end{center}
%
The argument \textit{dest} is the destination file
(without extension).
It should be the main file or one of the child files.
Note that further \textsf{childdoc} directives
such as |\childdocof| and |\childdocforward|
in the indicated file will be processed in this form.
The optional argument \textit{main}
passes on directly to the main file \textit{main}
while pretending to compile the child \textit{dest}.
This form behaves as if \textit{dest}
issues |\childdocof{|\textit{main}|}| right away,
and no further \textsf{childdoc} directives will be processed.

%%%%%%%%%%%%%%%%%%%%%%%%%%%%%%%%%%%%%%%%
\DescribeMacro{\...prefix}
In the alternative form |\childdocforwardprefix|,
%
\begin{center}
\begin{tabular}{l}
|\input{childdoc.def}|\\
|\childdocforwardprefix[|\textit{main}|]{|\textit{prefix}|}{|\textit{dest}|}|
\end{tabular}
\end{center}
%
the destination file is determined by a pattern
depending on the current file:
To make this work, the current file must be called
`{\textit{prefix}\hspace{0.2em}\textit{suffix}}'
with \textit{prefix} matching precisely the argument.
Processing is then passed on to the file
`{\textit{dest}\hspace{0.2em}\textit{suffix}}'.
Surely, the same effect is achieved by
directly specifying the
argument `{\textit{dest}\hspace{0.2em}\textit{suffix}}'
in the first form.
However, that requires to set up a different file
for each child. With the alternative form of the command
all these files can have exactly the same content
which simplifies setting them up and maintaining them.

For example, the following file |draft.tex|
with a compilation flag |\version| as described in \secref{sec:flags}
compiles the main document as a draft:
%
\begin{center}
\begin{tabular}{l}
|\def\version{draft}|\\
|\input{childdoc.def}|\\
|\childdocforward{|\textit{main}|}|
\end{tabular}
\end{center}
%
Likewise, the following files |final|\textit{nn}|.tex|
compile the final version of the child document
|child|\textit{nn}|.tex|:
%
\begin{center}
\begin{tabular}{l}
|\def\version{final}|\\
|\input{childdoc.def}|\\
|\childdocforwardprefix{final}{child}|
\end{tabular}
\end{center}
%

Note that when several versions of a main file and/or of each child file
are to be generated, it may be convenient to set up a |Makefile| or
shell script to automatise the process.

%%%%%%%%%%%%%%%%%%%%%%%%%%%%%%%%%%%%%%%%%%%%%%%%%%%%%%%%%%%%%%%%%%%%%%%%%%%%%%%%
\subsection{Command Line Processing}
\label{sec:commandline}

The effect of redirection files can also be achieved by invoking
the \LaTeX{} compiler with a more elaborate command line.
Most conveniently this should be done as part
of a shell script or a |Makefile|.

When using \textsf{childdoc} in the main file, the following
command lines effectively perform a redirection
(note that depending on the shell being used,
backslashes may have to be doubled: `|\|' $\to$ `|\\|'):
%
\begin{center}
|... -jobname "|\textit{target}|" |\\|"|[\textit{flags}]%
|\input{childdoc.def}\childdocforward[|\textit{main}|]{|\textit{dest}|}"|
\end{center}
%
Here \textit{target} is the name of the output file,
\textit{main} is the name of the main file
and \textit{dest} is the name of the main or child file to be processed
(all filenames without extensions).
The optional argument \textit{main} can be omitted
if \textit{main} matches \textit{dest}.
Optionally, compilation \textit{flags} can be defined via |\def| commands.
This command line makes the \TeX{} engine believe
it is compiling the file \textit{target}
whose content is specified as the latter parameter.
The provided code then forwards the processing to
\textit{main} or \textit{dest} as described in \secref{sec:forward}.

%%%%%%%%%%%%%%%%%%%%%%%%%%%%%%%%%%%%%%%%%%%%%%%%%%%%%%%%%%%%%%%%%%%%%%%%%%%%%%%%
\subsection{Include by Input}
\label{sec:input}

Including child documents by |\include| has some restrictions by design.
Most notably, the content of a child document always occupies
its own set of pages; pages cannot be shared between child documents.
Usually, this behaviour makes perfect sense
because each child document contain an essential part of the document.
However, in some situations it may be desirable to compose
a document from a collection of parts
without having mandatory page breaks between then.
For this case, the package
provides a mechanism to include parts
by |\input| which can also be processed individually.
However, by construction this mechanism
requires manual handling of the content to be output.

%%%%%%%%%%%%%%%%%%%%%%%%%%%%%%%%%%%%%%%%
\DescribeMacro{\ifchilddocmanual}
The main file should be prepared as usual, see \secref{sec:include}.
However, the document body must make a distinction
between processing of an individual part and of the main document, e.g.:
%
\begin{center}
\begin{tabular}{l}
|\ifchilddocmanual|\\
|\input{\childdocname}|\\
|\||else|\\
\textit{document body with }|\input{|\textit{part}|}|\\
|\||fi|
\end{tabular}
\end{center}
%
The conditional |\ifchilddocmanual| is true whenever
a part to be included by |\input| is being compiled,
and the name of the part is stored in |\childdocname|.

%%%%%%%%%%%%%%%%%%%%%%%%%%%%%%%%%%%%%%%%
\DescribeMacro{\childdocby}
Each part to be included by |\input| should start with:
%
\begin{center}
\begin{tabular}{l}
|\input{childdoc.def}|\\
|\childdocby{|\textit{main}|}|\\
\end{tabular}
\end{center}
%
The directive |\childdocby| is similar to |\childdocof|
described in \secref{sec:include},
but the subsequent selection of content must be done manually.
To that end, both |\ifchilddoc| and |\ifchilddocmanual|
will be true upon processing of a part,
and the name of the part is stored in |\childdocname|.
Note that |\jobname| will be set to the filename of the current part
so that each part receives an individual |.aux| file
that does not interfere with the |.aux| file(s) of the main document.
This behaviour can be altered by the alternative form
|\childdocby[*]{|\textit{main}|}| (with a non-empty optional argument)
which uses the |.aux| file of the main document
by setting |\jobname| to \textit{main}.

%%%%%%%%%%%%%%%%%%%%%%%%%%%%%%%%%%%%%%%%%%%%%%%%%%%%%%%%%%%%%%%%%%%%%%%%%%%%%%%%
\subsection{Driver Development}
\label{sec:driver}

The \textsf{childdoc} mechanism can also be use for the development
of definition files such as \LaTeX{} styles or classes.
This case differs from the above setup with multiple parts
included by |\include| in that no |\includeonly| should be invoked.
This can be achieved by starting the include file
(before |\ProvidesPackage|) with:
%
\begin{center}
\begin{tabular}{l}
|\input{childdoc.def}|\\
|\childdocforward{|\textit{main}|}|\\
\end{tabular}
\end{center}
%
or alternatively with:
%
\begin{center}
\begin{tabular}{l}
|\input{childdoc.def}|\\
|\childdocby{|\textit{main}|}|\\
\end{tabular}
\end{center}
%
Both forms have slightly different effects as described above.
The main file is prepared as usual, see \secref{sec:include}.

%%%%%%%%%%%%%%%%%%%%%%%%%%%%%%%%%%%%%%%%%%%%%%%%%%%%%%%%%%%%%%%%%%%%%%%%%%%%%%%%
\subsection{Legacy Detection}
\label{sec:detection}

The directive |\childdocmain| in the main file can detect
whether the complete document or merely a child is to be compiled
even without using the directive |\childdocof|.
This method is deprecated because it is less robust
and there is no compelling reason to use it;
it is merely provided for backward compatibility
and it may be removed in future versions.

If the detection mechanism is to be used,
it is mandatory to correctly specify
the filename of the main file as the argument of |\childdocmain|:
%
\begin{center}
\begin{tabular}{l}
|\input{childdoc.def}|\\
|\childdocmain{|\textit{main}|}|\\
\end{tabular}
\end{center}
%
If |\jobname| does not match the argument \textit{main} of |\childdocmain|,
it is assumed that |\jobname| points to the child file to be compiled.
When using |\childdocmain| with the main file specified as argument,
it suffices to start a child file
with just |\input{|\textit{main}|}|
without loading of the package and using |\childdocof|.
If instead all processing is done
with the appropriate \textsf{childdoc} directives,
the argument of \textit{main} of |\childdocmain| can be empty.

An alternative version of the command line processing described
in \secref{sec:commandline} using the detection mechanism reads:
%
\begin{center}
|... -jobname "|\textit{target}|" "|[\textit{flags}]%
[|\def\jobname{|\textit{dest}|}|]|\input{|\textit{main}|}"|
\end{center}

%%%%%%%%%%%%%%%%%%%%%%%%%%%%%%%%%%%%%%%%%%%%%%%%%%%%%%%%%%%%%%%%%%%%%%%%%%%%%%%%
\subsection{Manual Code}
\label{sec:manual}

In case one cannot be certain whether the definitions file |childdoc.def|
is installed on the target \TeX{} distribution
and one prefers not to ship it,
it is conceivable to paste a few relevant commands into the sources.

To that end, drop all statements |\input{childdoc.def}|
and perform the replacements as outlined below.
Instead of |\childdocmain{|\textit{main}|}| add the following code
to the top of the main file:
%
\begin{center}
\begin{tabular}{l}
|\||ifdefined\childdocname\endinput\||fi\newif\ifchilddoc|\\
|\edef\childdocname{\scantokens\expandafter{\jobname\noexpand}}|\\
|\def\childdocmain{|\textit{main}|}\||ifx\childdocmain\childdocname\||else|\\
|\childdoctrue\includeonly{\childdocname}\let\jobname\childdocmain\||fi|\\
\end{tabular}
\end{center}
%
Instead of |\childdocof{|\textit{main}|}| just include the main file
at the top of each child file:
%
\begin{center}
|\input{|\textit{main}|}|
\end{center}
%
A simple redirection |\childdocforward{|\textit{dest}|}| is achieved by:
%
\begin{center}
|\def\jobname{|\textit{dest}|}\input{\jobname}|
\end{center}
%
The redirection with prefix
|\childdocforwardprefix[|\textit{prefix}|]{|\textit{dest}|}|
is accomplished by:
%
\begin{center}
\begin{tabular}{l}
|{\edef\jobname{\scantokens\expandafter{\jobname\noexpand}}|\\
|\def\redirectjob |\textit{prefix}|#1~~~{\gdef\jobname{|\textit{dest}|#1}}|\\
|\expandafter\redirectjob\jobname~~~}\input{\jobname}|
\end{tabular}
\end{center}

In an alternative approach,
child documents can be compiled by a specific command line
without additional code or specific definitions:
%
\begin{center}
|... -jobname "|\textit{target}|" "|[\textit{flags}]%
|\includeonly{|\textit{dest}|}\input{|\textit{main}|}"|
\end{center}
%

%%%%%%%%%%%%%%%%%%%%%%%%%%%%%%%%%%%%%%%%%%%%%%%%%%%%%%%%%%%%%%%%%%%%%%%%%%%%%%%%
%%%%%%%%%%%%%%%%%%%%%%%%%%%%%%%%%%%%%%%%%%%%%%%%%%%%%%%%%%%%%%%%%%%%%%%%%%%%%%%%
\section{Information}

%%%%%%%%%%%%%%%%%%%%%%%%%%%%%%%%%%%%%%%%%%%%%%%%%%%%%%%%%%%%%%%%%%%%%%%%%%%%%%%%
\subsection{Copyright}

Copyright \copyright{} 2017--2018 Niklas Beisert

This work may be distributed and/or modified under the
conditions of the \LaTeX{} Project Public License, either version 1.3
of this license or (at your option) any later version.
The latest version of this license is in
  \url{http://www.latex-project.org/lppl.txt}
and version 1.3 or later is part of all distributions of \LaTeX{}
version 2005/12/01 or later.

This work has the LPPL maintenance status `maintained'.

The Current Maintainer of this work is Niklas Beisert.

This work consists of the files |README.txt|, |childdoc.ins| and |childdoc.dtx|
as well as the derived files |childdoc.def|, |cdocsamp.tex|
with |cdocsch1.tex|, |cdocsch2.tex|, |cdocspt3.tex|, |cdocspt4.tex|,
|cdocsdrf.tex|, |cdocsfn1.tex|, |cdocsfn2.tex|
as well as |childdoc.pdf|.

%%%%%%%%%%%%%%%%%%%%%%%%%%%%%%%%%%%%%%%%%%%%%%%%%%%%%%%%%%%%%%%%%%%%%%%%%%%%%%%%
\subsection{Files and Installation}

The package consists of the files:
%
\begin{center}
\begin{tabular}{ll}
    |README.txt|   & readme file \\
    |childdoc.ins| & installation file \\
    |childdoc.dtx| & source file \\
    |childdoc.def| & definition file \\
    |cdocsamp.tex| & sample main file \\
    |cdocsch1.tex| & sample include file \\
    |cdocsch2.tex| & sample include file \\
    |cdocspt3.tex| & sample part file \\
    |cdocspt4.tex| & sample part file \\
    |cdocsdrf.tex| & sample redirection file \\
    |cdocsfn1.tex| & sample redirection file \\
    |cdocsfn2.tex| & sample redirection file \\
    |childdoc.pdf| & manual
\end{tabular}
\end{center}
%
The distribution consists of the files
|README.txt|, |childdoc.ins| and |childdoc.dtx|.
%
\begin{itemize}
\item
Run (pdf)\LaTeX{} on |childdoc.dtx|
to compile the manual |childdoc.pdf| (this file).
\item
Run \LaTeX{} on |childdoc.ins| to create the definitions file |childdoc.def|
and the sample |cdocsamp.tex| with include files
|cdocsch1.tex|, |cdocsch2.tex|, |cdocspt3.tex|, |cdocspt4.tex|,
|cdocsdrf.tex|, |cdocsfn1.tex|, |cdocsfn2.tex|.
Then copy the file |childdoc.def| to an appropriate directory of your \LaTeX{}
distribution, e.g.\ \textit{texmf-root}|/tex/latex/childdoc|.
\end{itemize}

%%%%%%%%%%%%%%%%%%%%%%%%%%%%%%%%%%%%%%%%%%%%%%%%%%%%%%%%%%%%%%%%%%%%%%%%%%%%%%%%
\subsection{Related CTAN Packages}

There are several other packages which offer a similar functionality:
%
\begin{itemize}
\item
The packages
\href{http://ctan.org/pkg/docmute}{\textsf{docmute}},
\href{http://ctan.org/pkg/includex}{\textsf{includex}} and
\href{http://ctan.org/pkg/standalone}{\textsf{standalone}}
provide commands to include only the document body of
a child file thus allowing both files to be compiled individually.
\item
The packages \href{http://ctan.org/pkg/subdocs}{\textsf{subdocs}}
and \href{http://ctan.org/pkg/subfiles}{\textsf{subfiles}}
provide structures in which the main and child documents can be
encapsulated and allowing them to be compiled individually.
The inclusion mechanism is different from the conventional |\include|.
\item
The package \href{http://ctan.org/pkg/combine}{\textsf{combine}}
is an elaborate solution to combine several documents into one.
\end{itemize}
%
See also the CTAN topic \href{http://ctan.org/topic/subdocs}{\textsf{subdocs}}
for further related packages.
The present package differs from the above solutions in that
a document structure constructed with the conventional |\include| mechanism
just needs two extra commands at the top of every file
such that all constituent files can be compiled individually.

%%%%%%%%%%%%%%%%%%%%%%%%%%%%%%%%%%%%%%%%%%%%%%%%%%%%%%%%%%%%%%%%%%%%%%%%%%%%%%%%
%\subsection{Feature Suggestions}
%
%The following is a list of features which may be useful for future
%versions of this package:
%%
%\begin{itemize}
%\item
%\ldots
%\end{itemize}

%%%%%%%%%%%%%%%%%%%%%%%%%%%%%%%%%%%%%%%%%%%%%%%%%%%%%%%%%%%%%%%%%%%%%%%%%%%%%%%%
\subsection{Revision History}

%%%%%%%%%%%%%%%%%%%%%%%%%%%%%%%%%%%%%%%%
\paragraph{v2.0:} 2018/12/30

\begin{itemize}
\item
immediate forward processing
\item
added |\childdocby| mechanism
\item
manual restructured
\end{itemize}

%%%%%%%%%%%%%%%%%%%%%%%%%%%%%%%%%%%%%%%%
\paragraph{v1.6:} 2018/01/17

\begin{itemize}
\item
application for development of include files
\item
corrections to manual
\end{itemize}

%%%%%%%%%%%%%%%%%%%%%%%%%%%%%%%%%%%%%%%%
\paragraph{v1.5:} 2017/05/21

\begin{itemize}
\item
more complete structuring introduced
\item
|\childdocof| introduced
\item
|\childdoc| renamed to |\childdocmain|
\item
|\childredirect| renamed to |\childdocforward| and |\childdocforwardprefix|
and functionality expanded
\end{itemize}

%%%%%%%%%%%%%%%%%%%%%%%%%%%%%%%%%%%%%%%%
\paragraph{v1.0:} 2017/04/27

\begin{itemize}
\item
manual and install package
\item
first version published on CTAN
\end{itemize}

%%%%%%%%%%%%%%%%%%%%%%%%%%%%%%%%%%%%%%%%
\paragraph{v0.6:} 2017/04/26

\begin{itemize}
\item
redirection mechanism added
\end{itemize}

%%%%%%%%%%%%%%%%%%%%%%%%%%%%%%%%%%%%%%%%
\paragraph{v0.5:} 2017/04/26

\begin{itemize}
\item
functionality in definition file
\end{itemize}


%%%%%%%%%%%%%%%%%%%%%%%%%%%%%%%%%%%%%%%%%%%%%%%%%%%%%%%%%%%%%%%%%%%%%%%%%%%%%%%%
%%%%%%%%%%%%%%%%%%%%%%%%%%%%%%%%%%%%%%%%%%%%%%%%%%%%%%%%%%%%%%%%%%%%%%%%%%%%%%%%
%%%%%%%%%%%%%%%%%%%%%%%%%%%%%%%%%%%%%%%%%%%%%%%%%%%%%%%%%%%%%%%%%%%%%%%%%%%%%%%%
\appendix

\settowidth\MacroIndent{\rmfamily\scriptsize 000\ }

 \DocInput{childdoc.dtx}

\end{document}
%</driver>
% \fi
%
% %%%%%%%%%%%%%%%%%%%%%%%%%%%%%%%%%%%%%%%%%%%%%%%%%%%%%%%%%%%%%%%%%%%%%%%%%%%%%%
% %%%%%%%%%%%%%%%%%%%%%%%%%%%%%%%%%%%%%%%%%%%%%%%%%%%%%%%%%%%%%%%%%%%%%%%%%%%%%%
% \section{Sample}
%\iffalse
%<*samplemain>
%\fi
%
% The following presents a sample document
% with two chapters, two parts, a title page,
% a compile flag as well as three forwarding files to set the flag.
% It consists of eight |.tex| files:
% \begin{center}
% \begin{tabular}{ll}
% |cdocsamp.tex|&main file\\
% |cdocsch1.tex|&include file for chapter 1\\
% |cdocsch2.tex|&include file for chapter 2\\
% |cdocspt3.tex|&include file for part 3\\
% |cdocspt4.tex|&include file for part 4\\
% |cdocsdrf.tex|&forwarding file for main file in draft mode\\
% |cdocsfi1.tex|&forwarding file for final version of chapter 1\\
% |cdocsfi2.tex|&forwarding file for final version of chapter 2\\
% \end{tabular}
% \end{center}
% Each of the eight files can be compiled directly by the \LaTeX{} compiler.
%
% %%%%%%%%%%%%%%%%%%%%%%%%%%%%%%%%%%%%%%
% \paragraph{Main File.}
%
% The main file is called |cdocsamp.tex|.
%
% Load the \textsf{childdoc} definitions and
% declare the filename for the main document:
%    \begin{macrocode}
\input{childdoc.def}
\childdocmain{}
%    \end{macrocode}

% Optional override for |\version| flag:
%    \begin{macrocode}
%%\ifchilddoc\else\providecommand{\version}{draft}\fi
%    \end{macrocode}

% Define the default values for the |\version| flag
% (|final| for the main file and |draft| for childs):
%    \begin{macrocode}
\ifchilddoc
\providecommand{\version}{draft}
\else
\providecommand{\version}{final}
\fi
%    \end{macrocode}

% Load the standard document class:
%    \begin{macrocode}
\documentclass[12pt]{article}
%    \end{macrocode}

% Start the document body:
%    \begin{macrocode}
\begin{document}
%    \end{macrocode}

% Declare a title page.
% Print title, part of document being processed and version flag:
%    \begin{macrocode}
\addtocounter{page}{-1}
\begin{center}
{\LARGE\bfseries{}childdoc example\par}
\vspace{1cm}
\ifchilddoc
\ifchilddocmanual part\else chapter\fi:
`\childdocname' of `\childdocjob'\par
\else
main document: `\childdocjob'\par
\fi
version: \version\par
\end{center}
\newpage
%    \end{macrocode}

% Manually include selected file,
% otherwise process as usual:
%    \begin{macrocode}
\ifchilddocmanual
\section*{part `\childdocname'}
\input{\childdocname}
\else
%    \end{macrocode}

% Include the two chapters:
%    \begin{macrocode}
\include{cdocsch1}
\include{cdocsch2}
%    \end{macrocode}

% Include the two parts unless only chapters should be displayed:
%    \begin{macrocode}
\ifchilddoc\else
\section{part three}
\input{cdocspt3}
\section{part four}
\input{cdocspt4}
\fi
%    \end{macrocode}

% Process as usual until here:
%    \begin{macrocode}
\fi
%    \end{macrocode}

% End of document body:
%    \begin{macrocode}
\end{document}
%    \end{macrocode}
%\iffalse
%</samplemain>
%\fi
%
% %%%%%%%%%%%%%%%%%%%%%%%%%%%%%%%%%%%%%%
% \paragraph{Chapter Include Files.}
%
% The include files are called |cdocsch1.tex| and |cdocsch2.tex|.
%
%\iffalse
%<*samplechap1|samplechap2>
%\fi

% Optional override for |\version| flag:
%    \begin{macrocode}
%%\providecommand{\version}{final}
%    \end{macrocode}

% Include the main document:
%    \begin{macrocode}
\input{childdoc.def}
\childdocof{cdocsamp}
%    \end{macrocode}

%\iffalse
%</samplechap1|samplechap2>
%\fi
%
%\iffalse
%<*samplechap1>
%\fi
% Some text for chapter 1:
%    \begin{macrocode}
\section{one}
some text in chapter one
%    \end{macrocode}

%\iffalse
%</samplechap1>
%\fi
% Some text for chapter 2:
%\iffalse
%<*samplechap2>
%\fi
%    \begin{macrocode}
\section{two}
more text in chapter two
%    \end{macrocode}

%\iffalse
%</samplechap2>
%\fi
%
% %%%%%%%%%%%%%%%%%%%%%%%%%%%%%%%%%%%%%%
% \paragraph{Part Include Files.}
%
% The include files are called |cdocspt3.tex| and |cdocspt4.tex|.
%
%\iffalse
%<*samplepart3|samplepart4>
%\fi

% Optional override for |\version| flag:
%    \begin{macrocode}
%%\providecommand{\version}{final}
%    \end{macrocode}

% Include the main document:
%    \begin{macrocode}
\input{childdoc.def}
\childdocby{cdocsamp}
%    \end{macrocode}

%\iffalse
%</samplepart3|samplepart4>
%\fi
%
%\iffalse
%<*samplepart3>
%\fi
% Some text for part 3:
%    \begin{macrocode}
some text in part three
%    \end{macrocode}

%\iffalse
%</samplepart3>
%\fi
% Some text for part 4:
%\iffalse
%<*samplepart4>
%\fi
%    \begin{macrocode}
more text in part four
%    \end{macrocode}

%\iffalse
%</samplepart4>
%\fi
%
% %%%%%%%%%%%%%%%%%%%%%%%%%%%%%%%%%%%%%%
% \paragraph{Forwarding for a Complete Draft.}
%
% The following forwarding file |cdocsdrf.tex|
% compiles the main document in draft mode:
%\iffalse
%<*sampledraft>
%\fi
%    \begin{macrocode}
\def\version{draft}
\input{childdoc.def}
\childdocforward{cdocsamp}
%    \end{macrocode}

%\iffalse
%</sampledraft>
%\fi
%
% %%%%%%%%%%%%%%%%%%%%%%%%%%%%%%%%%%%%%%
% \paragraph{Forwarding for Final Version of the Chapters.}
%
% The following forwarding files |cdocsfn1.tex| and |cdocsfn2.tex|
% (with identical content)
% compile the final versions of the child documents
% |cdocsch1.tex| and |cdocsch2.tex|, respectively:
%\iffalse
%<*samplefinal>
%\fi
%    \begin{macrocode}
\def\version{final}
\input{childdoc.def}
\childdocforwardprefix[cdocsamp]{cdocsfn}{cdocsch}
%    \end{macrocode}

%\iffalse
%</samplefinal>
%\fi
%
% %%%%%%%%%%%%%%%%%%%%%%%%%%%%%%%%%%%%%%
% \paragraph{Command Line Processing.}
%
% The following three command lines generate the output files
% |cdocscld|, |cdocscl1| and |cdocscl2|
% which should be identical to
% |cdocsdrf|, |cdocsch1| and |cdocsfn2|, respectively:
% \begin{center}
% \begin{tabular}{l}
% |latex -jobname cdocscld \|\\
% |  "\def\version{draft}\input{childdoc.def}\childdocforward{cdocsamp}"|\\
% |latex -jobname cdocscl1 \|\\
% |  "\input{childdoc.def}\childdocforward[cdocsamp]{cdocsch1}"|\\
% |latex -jobname cdocscl2 \|\\
% |  "\def\version{final}\input{childdoc.def}\childdocforward{cdocsch2}"|
% \end{tabular}
% \end{center}
% Note that the trailing backslash on each first line
% merely continues the input to the second line
% (for convenient cut ant paste).
% Furthermore, the command |latex| can be replaced by any
% of its alternative versions such as |pdflatex|.
%
% %%%%%%%%%%%%%%%%%%%%%%%%%%%%%%%%%%%%%%%%%%%%%%%%%%%%%%%%%%%%%%%%%%%%%%%%%%%%%%
% %%%%%%%%%%%%%%%%%%%%%%%%%%%%%%%%%%%%%%%%%%%%%%%%%%%%%%%%%%%%%%%%%%%%%%%%%%%%%%
% \section{Implementation}
%\iffalse
%<*package>
%\fi
%
% This section describes the definitions file |childdoc.def|.

% The definitions cannot be loaded using |\usepackage| or |\RequirePackage|
% which has a mechanism to prevent loading a style file more than once.
% When loading the definitions by means of |\input|
% multiple instances have to be prevented manually:
%\iffalse
%This code needs to be before the `\ProvidesFile' directive
%which is defined at the beginning of this file.
%Therefore it is also placed there and commented out here.
%</package>
%<*discard>
%\fi
%    \begin{macrocode}
\ifdefined\childdocmain\endinput\fi
%    \end{macrocode}
%\iffalse
%</discard>
%<*package>
%\fi
%
% \macro{\ifchilddoc}
% \macro{\ifchilddocmanual}
% The conditional |\ifchilddoc| tells whether a
% child (true) or main (false) document is being compiled.
% The conditional |\ifchilddocmanual| tells whether
% the |\includeonly| mechanism is used (false) or
% the selection of child files must be performed manually (true).
% The definitions initialise to false:
%    \begin{macrocode}
\newif\ifchilddoc
\newif\ifchilddocmanual
%    \end{macrocode}

% \macro{\childdocname}
% \macro{\childdocjob}
% The macro |\childdocname| stores the name of the main document
% to be compiled. The macro |\childdocjob| stores the name of
% the document on which the \LaTeX{} compiler was originally invoked.
% The content of |\jobname| cannot be compared
% to filenames specified in the source due to different catcodes.
% The following code rescans |\jobname|, stores the result
% in |\childdocname| and saves a copy in |\childdocjob|:
%    \begin{macrocode}
\edef\childdocname{\scantokens\expandafter{\jobname\noexpand}}
\let\childdocjob\childdocname
%    \end{macrocode}

% \macro{\childdocdisable}
% The macro |\childdocdisable| prevents the main file
% from being processed more than once.
% At this stage, the main document command |\childdocmain|
% is assumed to be called once again where it should do nothing.
% Any subsequent call to it should prevent
% a secondary processing of the main document
% It overwrites the forwarding commands
% |\childdocof| and |\childdocforward|
% with empty macros to prevent further inclusions of the main document:
%    \begin{macrocode}
\newcommand{\childdocdisable}
{
  \renewcommand{\childdocmain}[1]{\renewcommand{\childdocmain}[1]{\endinput}}
  \renewcommand{\childdocof}[1]{}
  \renewcommand{\childdocby}[2][]{}
  \renewcommand{\childdocforward}[2][]{}
  \renewcommand{\childdocdisable}{}
}
%    \end{macrocode}

% \macro{\childdocmain}
% The macro |\childdocmain| is to be called at the top of the main file
% with nothing or the main filename (without extension) as argument.
% First, it breaks loops.
% If the argument is not empty and does not match |\childdocname|
% (which is set by the first inclusion of |childdoc.def|),
% |\ifchilddoc| is set to true, |\includeonly| is applied to the child file
% and |\jobname| is set to the main file
% (for proper handling of |.aux| files):
%    \begin{macrocode}
\newcommand{\childdocmain}[1]
{
  \childdocdisable\childdocmain{}
  \if?#1?\else
    \begingroup
      \def\childdoctmp{#1}
      \ifx\childdoctmp\childdocname
        \def\childdoctmp{}
      \else
        \def\childdoctmp
        {
          \childdoctrue
          \includeonly{\childdocname}
          \def\childdocjob{#1}
          \def\jobname{#1}
        }
      \fi
      \expandafter
    \endgroup
    \childdoctmp
  \fi
}
%    \end{macrocode}

% \macro{\childdocof}
% The command |\childdocof| redirects
% compilation to the main file |#1|.
%    \begin{macrocode}
\newcommand{\childdocof}[1]
{
  \childdocdisable
  \childdoctrue
  \includeonly{\childdocname}
  \def\jobname{#1}
  \def\childdocjob{#1}
  \input{#1}
}
%    \end{macrocode}

% \macro{\childdocby}
% The command |\childdocby| ....
%    \begin{macrocode}
\newcommand{\childdocby}[2][]
{
  \childdocdisable
  \childdoctrue
  \childdocmanualtrue
  \if?#1?\else
    \def\jobname{#2}
  \fi
  \def\childdocjob{#2}
  \input{#2}
  \endinput
}
%    \end{macrocode}

% \macro{\childdocforward}
% The command |\childdocforward| redirects
% compilation to the main file or
% (if the optional argument is given) a child file.
% Parameters are set as if the main file
% or a child file starting with |\childdocof| was compiled.
% Then compilation is handed over to the main file:
%    \begin{macrocode}
\newcommand{\childdocforward}[2][]
{
  \begingroup
    \if?#1?
      \def\childdoctmp
      {
        \def\childdocname{#2}
        \def\childdocjob{#2}
        \def\jobname{#2}
        \input{#2}
        \endinput
      }
    \else
      \def\childdoctmp
      {
        \childdocdisable
        \def\childdocname{#2}
        \childdoctrue
        \includeonly{#2}
        \def\childdocjob{#1}
        \def\jobname{#1}
        \input{#1}
        \endinput
      }
    \fi
    \expandafter
  \endgroup
  \childdoctmp
}
%    \end{macrocode}

% \macro{\childdocforwardprefix}
% The command |\childdocforwardprefix| redirects
% compilation to the main or a child file by means of a pattern.
% The prefix |#1| in the current filename is replaced by |#2|
% and the suffix of the current filename is kept
% (it is assumed that the filename does not contain the substring `|~~~|'
% which is used as a delimiter).
% Compilation is handed over to the new file by |\childdocforward|:
%    \begin{macrocode}
\newcommand{\childdocforwardprefix}[3][]
{
  \begingroup
    \def\childdocextract #2##1~~~{\def\childdoctmp{\childdocforward[#1]{#3##1}}}
    \expandafter\childdocextract\childdocname~~~
    \expandafter
  \endgroup
  \childdoctmp
}
%    \end{macrocode}

% \macro{\childdoc}
% The deprecated macro |\childdoc| is a legacy version of |\childdocmain|:
%    \begin{macrocode}
\newcommand{\childdoc}{\childdocmain}
%    \end{macrocode}

% \macro{\childdocredirect}
% The deprecated macro |\childdocredirect| is a legacy version
% of |\childdocforward| and |\childdocforwardprefix|:
%    \begin{macrocode}
\newcommand{\childdocredirect}[2][]
{
  \begingroup
    \if?#1?
      \def\childdoctmp{\childdocforward{#2}}
    \else
      \def\childdoctmp{\childdocforwardprefix{#1}{#2}}
    \fi
    \expandafter
  \endgroup
  \childdoctmp
}
%    \end{macrocode}

%\iffalse
%</package>
%\fi
%
\endinput
|\\
|\childdocforwardprefix{final}{child}|
\end{tabular}
\end{center}
%

Note that when several versions of a main file and/or of each child file
are to be generated, it may be convenient to set up a |Makefile| or
shell script to automatise the process.

%%%%%%%%%%%%%%%%%%%%%%%%%%%%%%%%%%%%%%%%%%%%%%%%%%%%%%%%%%%%%%%%%%%%%%%%%%%%%%%%
\subsection{Command Line Processing}
\label{sec:commandline}

The effect of redirection files can also be achieved by invoking
the \LaTeX{} compiler with a more elaborate command line.
Most conveniently this should be done as part
of a shell script or a |Makefile|.

When using \textsf{childdoc} in the main file, the following
command lines effectively perform a redirection
(note that depending on the shell being used,
backslashes may have to be doubled: `|\|' $\to$ `|\\|'):
%
\begin{center}
|... -jobname "|\textit{target}|" |\\|"|[\textit{flags}]%
|% \iffalse
%
% childdoc.dtx Copyright (C) 2017-2018 Niklas Beisert
%
% This work may be distributed and/or modified under the
% conditions of the LaTeX Project Public License, either version 1.3
% of this license or (at your option) any later version.
% The latest version of this license is in
%   http://www.latex-project.org/lppl.txt
% and version 1.3 or later is part of all distributions of LaTeX
% version 2005/12/01 or later.
%
% This work has the LPPL maintenance status `maintained'.
%
% The Current Maintainer of this work is Niklas Beisert.
%
% This work consists of the files childdoc.dtx and childdoc.ins
% and the derived files childdoc.def and cdocsamp.tex with
% cdocsch1.tex, cdocsch2.tex, cdocsdrf.tex, cdocsfn1.tex, cdocsfn2.tex.
%
%<package>\ifdefined\childdocmain\endinput\fi
%<package>\ProvidesFile{childdoc.def}[2018/12/30 v2.0 child document driver]
%<samplemain>\ProvidesFile{cdocsamp.tex}[2018/12/30 v2.0 sample for childdoc]
%<*driver>
%\ProvidesFile{childdoc.drv}[2018/12/30 v2.0 childdoc reference manual file]
\PassOptionsToClass{10pt,a4paper}{article}
\documentclass{ltxdoc}

\usepackage[margin=35mm]{geometry}
\usepackage{hyperref}
\usepackage{hyperxmp}
\usepackage[usenames]{color}

\hypersetup{colorlinks=true}
\hypersetup{pdfstartview=FitH}
\hypersetup{pdfpagemode=UseNone}
\hypersetup{pdfsource={}}
\hypersetup{pdflang={en-UK}}
\hypersetup{pdfcopyright={Copyright 2017-2018 Niklas Beisert.
  This work may be distributed and/or modified under the
  conditions of the LaTeX Project Public License, either version 1.3
  of this license or (at your option) any later version.}}
\hypersetup{pdflicenseurl={http://www.latex-project.org/lppl.txt}}
\hypersetup{pdfcontactaddress={ETH Zurich, ITP, HIT K,
  Wolfgang-Pauli-Strasse 27}}
\hypersetup{pdfcontactpostcode={8093}}
\hypersetup{pdfcontactcity={Zurich}}
\hypersetup{pdfcontactcountry={Switzerland}}
\hypersetup{pdfcontactemail={nbeisert@itp.phys.ethz.ch}}
\hypersetup{pdfcontacturl={http://people.phys.ethz.ch/\xmptilde nbeisert/}}

\newcommand{\secref}[1]{\hyperref[#1]{section \ref*{#1}}}

\parskip1ex
\parindent0pt
\let\olditemize\itemize
\def\itemize{\olditemize\parskip0pt}

\begin{document}

\title{The \textsf{childdoc} Package}
\hypersetup{pdftitle={The childdoc Package}}
\author{Niklas Beisert\\[2ex]
  Institut f\"ur Theoretische Physik\\
  Eidgen\"ossische Technische Hochschule Z\"urich\\
  Wolfgang-Pauli-Strasse 27, 8093 Z\"urich, Switzerland\\[1ex]
  \href{mailto:nbeisert@itp.phys.ethz.ch}
  {\texttt{nbeisert@itp.phys.ethz.ch}}}
\hypersetup{pdfauthor={Niklas Beisert}}
\hypersetup{pdfsubject={Manual for the LaTeX2e Package childdoc}}
\date{30 December 2018, \textsf{v2.0}}
\maketitle

\begin{abstract}\noindent
\textsf{childdoc} is a \LaTeXe{} package
that enables the direct compilation
of document sections included by |\include|
to individual files.
\end{abstract}

\begingroup
\parskip0ex
\tableofcontents
\endgroup

%%%%%%%%%%%%%%%%%%%%%%%%%%%%%%%%%%%%%%%%%%%%%%%%%%%%%%%%%%%%%%%%%%%%%%%%%%%%%%%%
%%%%%%%%%%%%%%%%%%%%%%%%%%%%%%%%%%%%%%%%%%%%%%%%%%%%%%%%%%%%%%%%%%%%%%%%%%%%%%%%
\section{Introduction}

\LaTeX{} provides a mechanism to structure a large document (such as a book)
into a main file and several child files (containing the chapters)
using the |\include| command.
This mechanism is beneficial for documents
which span hundreds of pages in order to
make the source file(s) more manageable.
Moreover, compilation can be restricted to
selected child files by means of the |\includeonly| command.
The latter feature can be used to reduce the compilation time while editing
(this was significantly more useful in the earlier days of \LaTeX{})
or to generate a smaller document which is easier to navigate.
Another application of |\includeonly| is to generate
documents consisting of selected parts of the complete document.

However, there are a few drawbacks of the plain |\include| mechanism:
\begin{itemize}
\item
The child files cannot be compiled on their own,
they can only be compiled via the main file.
A naive editing environment
(such as a text editor with an option
to have the current file processed by \LaTeX)
may require one to switch to the main file before compiling;
attempting to compile the child file produces errors.
\item
The main file must be modified (each time)
to adjust the |\includeonly| command
to the present needs. This easily leaves the main file in a messy state.
\item
The generated document will always carry the filename
of the main document. This is inconvenient if
several child files are to be compiled and
to be kept for distribution.
\end{itemize}

The present package provides a simple interface
to make child files individually compilable by \LaTeX{}.
Compiling a child file then has the same effect as compiling
the main file with an |\includeonly| command
to select the appropriate child.
Moreover the generated document will carry the name of the child
rather than the main file.
This resolves all three above issues.

This feature is meant to make the editing of books,
thesis documents and lecture notes somewhat more convenient.
However, the package can also be used efficiently for
composing a series of documents (such as exercise sheets)
which are typically distributed individually.
It then assists the author in generating the individual documents
(potentially in different versions)
as well as a document containing the collected series.
Another application is in developing style files
or other kinds of included material
where compilation of the style file could redirect
to a sample or test file.

%%%%%%%%%%%%%%%%%%%%%%%%%%%%%%%%%%%%%%%%%%%%%%%%%%%%%%%%%%%%%%%%%%%%%%%%%%%%%%%%
%%%%%%%%%%%%%%%%%%%%%%%%%%%%%%%%%%%%%%%%%%%%%%%%%%%%%%%%%%%%%%%%%%%%%%%%%%%%%%%%
\section{Usage}

First of all, the package \textsf{childdoc} is \emph{not} a standard
\LaTeXe{} |.sty| style file! Therefore it needs to be invoked in
a non-standard way.

%%%%%%%%%%%%%%%%%%%%%%%%%%%%%%%%%%%%%%%%%%%%%%%%%%%%%%%%%%%%%%%%%%%%%%%%%%%%%%%%
\subsection{Included Files}
\label{sec:include}

%%%%%%%%%%%%%%%%%%%%%%%%%%%%%%%%%%%%%%%%
\DescribeMacro{\childdocmain}
To use the package, add the commands
\begin{center}
\begin{tabular}{l}
|\input{childdoc.def}|\\
|\childdocmain{}|\\
\end{tabular}
\end{center}
at the very top of the main \LaTeX{} file,
in particular \emph{before} the |\documentclass| statement!
The argument of |\childdocmain| should be left empty
(but it must be present).

%%%%%%%%%%%%%%%%%%%%%%%%%%%%%%%%%%%%%%%%
\DescribeMacro{\childdocof}
Furthermore, add the commands
\begin{center}
\begin{tabular}{l}
|\input{childdoc.def}|\\
|\childdocof{|\textit{main}|}|\\
\end{tabular}
\end{center}
at the top of every child file \textit{child}
which is included by |\include{|\textit{child}|}|
from within the main file
(or at least for those files to be compiled individually).
The argument \textit{main} must be the filename of the main file.

There are a couple of
considerations in setting up the main and child documents:

%%%%%%%%%%%%%%%%%%%%%%%%%%%%%%%%%%%%%%%%
\paragraph{Restrictions.}

Please note the following restrictions:
\begin{itemize}
\item
|\childdocmain| must be called with one argument \textit{main}
to ensure compatibility with earlier version of the package.
It must either be empty (|\childdocmain{}|)
or precisely match the filename of the main file in which it is specified.
See \secref{sec:detection} for further information.
\item
The filename \textit{main} must be specified without the |.tex| extension.
\item
The filename \textit{main} is case sensitive
(even in case-insensitive file systems)
due to internal string comparison.
\item
The argument \textit{main} should be fully expanded, it cannot be a macro.
\item
Subdirectories and special characters should be avoided in filenames.
\item
The command |\childdocmain{|\textit{main}|}| must be followed by a whitespace.
It should not be followed immediately by another command
or by a comment mark `|%|'.
This is because the \TeX{} parser reads the token immediately following
the argument of |\childdocmain| and puts it
at the beginning of every child section;
however, a white\-space is ignored.
\end{itemize}

%%%%%%%%%%%%%%%%%%%%%%%%%%%%%%%%%%%%%%%%
\paragraph{Content of Main File.}

It is advisable to place all content in the child files included by |\include|.
Any output contained in the main file will appear in all child documents
unless suppressed manually;
it cannot be suppressed automatically by the |\includeonly| directive
and thus should normally be avoided.
A method to include some content in the main file
by means of conditional processing is described in \secref{sec:conditional}.

%%%%%%%%%%%%%%%%%%%%%%%%%%%%%%%%%%%%%%%%
\paragraph{Page Numbering.}

When only a part of the document is compiled,
the appropriate numbering of pages
(as well as other status parameters)
is determined from the |.aux| files.
The latter contain information from previous passes.
However this information needs to propagate through
all intermediate child documents.
Therefore the page numbering in child documents may well
be inconsistent until the complete document is compiled at least once.

A useful (if unconventional) way to always ensure a consistent
page numbering is to restart the numbering in each child document
and denote the pages by `\textit{child}|.|\textit{page}'
where \textit{child} represents the chapter/section number of the child file.
This can be achieved by the command
|\numberwithin{page}{|\textit{child}|}|
of the \textsf{amsmath} package
where \textit{child} can be |chapter| or |section|
depending on the chosen structuring.
Alternatively, one can modify the macro |\thepage| appropriately
and reset the counter |page| at the start of each child file.

%%%%%%%%%%%%%%%%%%%%%%%%%%%%%%%%%%%%%%%%%%%%%%%%%%%%%%%%%%%%%%%%%%%%%%%%%%%%%%%%
\subsection{Conditional Processing}
\label{sec:conditional}

The package provides a mechanism to compile different versions
of a document. To customise the versions further some conditional processing
can come in handy to distinguish which version is being compiled.
The package provides two macros to describe the compilation context:

%%%%%%%%%%%%%%%%%%%%%%%%%%%%%%%%%%%%%%%%
\DescribeMacro{\ifchilddoc}
The conditional |\ifchilddoc| distinguishes between the compilation of
child documents and the main document:
%
\begin{center}
|\ifchilddoc |\textit{child-code}| |[|\||else |\textit{main-code}]| \||fi|
\end{center}

%%%%%%%%%%%%%%%%%%%%%%%%%%%%%%%%%%%%%%%%
\DescribeMacro{\childdocname}
\DescribeMacro{\childdocjob}
The macro |\childdocname| contains the filename (without extension)
of the main or child file being processed.
Note that |\childdocjob| will always contain the name of the main file.

%%%%%%%%%%%%%%%%%%%%%%%%%%%%%%%%%%%%%%%%
\paragraph{Title Page.}

Conditional processing can be used to include a title or banner page
in the main document when proper precautions are taken.
Importantly, the code in the main file should ensure that the page counter
(as well as other status parameters which are stored in the |.aux| files)
takes the same value after the conditional processing.
Otherwise the page numbers may take divergent values
depending on which part is compiled.

For example, a title page could be declared by:
%
\begin{center}
\begin{tabular}{l}
|\ifchilddoc\||else|\\
|\addtocounter{page}{-1}|\\
\textit{code for title page}\\
|\newpage|\\
|\||fi|
\end{tabular}
\end{center}
%
A banner page for the child documents can be generated by:
%
\begin{center}
\begin{tabular}{l}
|\ifchilddoc|\\
|\addtocounter{page}{-1}|\\
\textit{code for banner page}\\
|\newpage|\\
|\||fi|
\end{tabular}
\end{center}
%
Here one could write a message such as:
\begin{center}
|This is the part \childdocname{} of \childdocjob{}.|
\end{center}

%%%%%%%%%%%%%%%%%%%%%%%%%%%%%%%%%%%%%%%%%%%%%%%%%%%%%%%%%%%%%%%%%%%%%%%%%%%%%%%%
\subsection{Flags}
\label{sec:flags}

The package makes it easy to generate different versions
of the main or child documents.
To this end compilation flags can be defined
and assigned different default values.
They will be particularly useful in conjunction
with the forwarding mechanism described in \secref{sec:forward}.

For example, it may be useful to have a flag |\version|
which can be set to |draft| or |final|.
The document source will contain some conditional code
depending on the value of |\version|.
Suppose further, the flag should default to |final| for the main file
and to |draft| for child files
which is a natural assignment for editing the document.
This is achieved by placing the following code
in the preamble of the main document
(below the |\childdocmain| directive):
%
\begin{center}
\begin{tabular}{l}
|\ifchilddoc|\\
|\providecommand{\version}{draft}|\\
|\||else|\\
|\providecommand{\version}{final}|\\
|\||fi|
\end{tabular}
\end{center}
%
The definition by |\providecommand| makes sure
that previous definitions are not overwritten.
Further statements |\providecommand{\version}{...}|
can thus be added before the above code to override it.

For the main file, one might add a line
(between |\childdocmain| and the above block)
%
\begin{center}
|%\ifchilddoc\||else\providecommand{\version}{draft}\||fi|
\end{center}
%
which can be uncommented to produce a draft version.
Likewise one can add a line to the very top of a child file
(above the |\childdocof{|\textit{main}|}| directive)
%
\begin{center}
|%\providecommand{\version}{final}|
\end{center}
%
which can be uncommented to produce the final version of this child document.

%%%%%%%%%%%%%%%%%%%%%%%%%%%%%%%%%%%%%%%%%%%%%%%%%%%%%%%%%%%%%%%%%%%%%%%%%%%%%%%%
\subsection{Forwarding}
\label{sec:forward}

Different versions of the main or child documents
using compilation flags as described in \secref{sec:flags}
can be (permanently) stored in different files
for convenient compilation, viewing and distribution.
To this end, the package defines a command
to pass on compilation to a different file:

%%%%%%%%%%%%%%%%%%%%%%%%%%%%%%%%%%%%%%%%
\DescribeMacro{\childdocforward}
The command |\childdocforward| redirects processing to
another source file:
%
\begin{center}
\begin{tabular}{l}
|\input{childdoc.def}|\\
|\childdocforward[|\textit{main}|]{|\textit{dest}|}|\\
\end{tabular}
\end{center}
%
The argument \textit{dest} is the destination file
(without extension).
It should be the main file or one of the child files.
Note that further \textsf{childdoc} directives
such as |\childdocof| and |\childdocforward|
in the indicated file will be processed in this form.
The optional argument \textit{main}
passes on directly to the main file \textit{main}
while pretending to compile the child \textit{dest}.
This form behaves as if \textit{dest}
issues |\childdocof{|\textit{main}|}| right away,
and no further \textsf{childdoc} directives will be processed.

%%%%%%%%%%%%%%%%%%%%%%%%%%%%%%%%%%%%%%%%
\DescribeMacro{\...prefix}
In the alternative form |\childdocforwardprefix|,
%
\begin{center}
\begin{tabular}{l}
|\input{childdoc.def}|\\
|\childdocforwardprefix[|\textit{main}|]{|\textit{prefix}|}{|\textit{dest}|}|
\end{tabular}
\end{center}
%
the destination file is determined by a pattern
depending on the current file:
To make this work, the current file must be called
`{\textit{prefix}\hspace{0.2em}\textit{suffix}}'
with \textit{prefix} matching precisely the argument.
Processing is then passed on to the file
`{\textit{dest}\hspace{0.2em}\textit{suffix}}'.
Surely, the same effect is achieved by
directly specifying the
argument `{\textit{dest}\hspace{0.2em}\textit{suffix}}'
in the first form.
However, that requires to set up a different file
for each child. With the alternative form of the command
all these files can have exactly the same content
which simplifies setting them up and maintaining them.

For example, the following file |draft.tex|
with a compilation flag |\version| as described in \secref{sec:flags}
compiles the main document as a draft:
%
\begin{center}
\begin{tabular}{l}
|\def\version{draft}|\\
|\input{childdoc.def}|\\
|\childdocforward{|\textit{main}|}|
\end{tabular}
\end{center}
%
Likewise, the following files |final|\textit{nn}|.tex|
compile the final version of the child document
|child|\textit{nn}|.tex|:
%
\begin{center}
\begin{tabular}{l}
|\def\version{final}|\\
|\input{childdoc.def}|\\
|\childdocforwardprefix{final}{child}|
\end{tabular}
\end{center}
%

Note that when several versions of a main file and/or of each child file
are to be generated, it may be convenient to set up a |Makefile| or
shell script to automatise the process.

%%%%%%%%%%%%%%%%%%%%%%%%%%%%%%%%%%%%%%%%%%%%%%%%%%%%%%%%%%%%%%%%%%%%%%%%%%%%%%%%
\subsection{Command Line Processing}
\label{sec:commandline}

The effect of redirection files can also be achieved by invoking
the \LaTeX{} compiler with a more elaborate command line.
Most conveniently this should be done as part
of a shell script or a |Makefile|.

When using \textsf{childdoc} in the main file, the following
command lines effectively perform a redirection
(note that depending on the shell being used,
backslashes may have to be doubled: `|\|' $\to$ `|\\|'):
%
\begin{center}
|... -jobname "|\textit{target}|" |\\|"|[\textit{flags}]%
|\input{childdoc.def}\childdocforward[|\textit{main}|]{|\textit{dest}|}"|
\end{center}
%
Here \textit{target} is the name of the output file,
\textit{main} is the name of the main file
and \textit{dest} is the name of the main or child file to be processed
(all filenames without extensions).
The optional argument \textit{main} can be omitted
if \textit{main} matches \textit{dest}.
Optionally, compilation \textit{flags} can be defined via |\def| commands.
This command line makes the \TeX{} engine believe
it is compiling the file \textit{target}
whose content is specified as the latter parameter.
The provided code then forwards the processing to
\textit{main} or \textit{dest} as described in \secref{sec:forward}.

%%%%%%%%%%%%%%%%%%%%%%%%%%%%%%%%%%%%%%%%%%%%%%%%%%%%%%%%%%%%%%%%%%%%%%%%%%%%%%%%
\subsection{Include by Input}
\label{sec:input}

Including child documents by |\include| has some restrictions by design.
Most notably, the content of a child document always occupies
its own set of pages; pages cannot be shared between child documents.
Usually, this behaviour makes perfect sense
because each child document contain an essential part of the document.
However, in some situations it may be desirable to compose
a document from a collection of parts
without having mandatory page breaks between then.
For this case, the package
provides a mechanism to include parts
by |\input| which can also be processed individually.
However, by construction this mechanism
requires manual handling of the content to be output.

%%%%%%%%%%%%%%%%%%%%%%%%%%%%%%%%%%%%%%%%
\DescribeMacro{\ifchilddocmanual}
The main file should be prepared as usual, see \secref{sec:include}.
However, the document body must make a distinction
between processing of an individual part and of the main document, e.g.:
%
\begin{center}
\begin{tabular}{l}
|\ifchilddocmanual|\\
|\input{\childdocname}|\\
|\||else|\\
\textit{document body with }|\input{|\textit{part}|}|\\
|\||fi|
\end{tabular}
\end{center}
%
The conditional |\ifchilddocmanual| is true whenever
a part to be included by |\input| is being compiled,
and the name of the part is stored in |\childdocname|.

%%%%%%%%%%%%%%%%%%%%%%%%%%%%%%%%%%%%%%%%
\DescribeMacro{\childdocby}
Each part to be included by |\input| should start with:
%
\begin{center}
\begin{tabular}{l}
|\input{childdoc.def}|\\
|\childdocby{|\textit{main}|}|\\
\end{tabular}
\end{center}
%
The directive |\childdocby| is similar to |\childdocof|
described in \secref{sec:include},
but the subsequent selection of content must be done manually.
To that end, both |\ifchilddoc| and |\ifchilddocmanual|
will be true upon processing of a part,
and the name of the part is stored in |\childdocname|.
Note that |\jobname| will be set to the filename of the current part
so that each part receives an individual |.aux| file
that does not interfere with the |.aux| file(s) of the main document.
This behaviour can be altered by the alternative form
|\childdocby[*]{|\textit{main}|}| (with a non-empty optional argument)
which uses the |.aux| file of the main document
by setting |\jobname| to \textit{main}.

%%%%%%%%%%%%%%%%%%%%%%%%%%%%%%%%%%%%%%%%%%%%%%%%%%%%%%%%%%%%%%%%%%%%%%%%%%%%%%%%
\subsection{Driver Development}
\label{sec:driver}

The \textsf{childdoc} mechanism can also be use for the development
of definition files such as \LaTeX{} styles or classes.
This case differs from the above setup with multiple parts
included by |\include| in that no |\includeonly| should be invoked.
This can be achieved by starting the include file
(before |\ProvidesPackage|) with:
%
\begin{center}
\begin{tabular}{l}
|\input{childdoc.def}|\\
|\childdocforward{|\textit{main}|}|\\
\end{tabular}
\end{center}
%
or alternatively with:
%
\begin{center}
\begin{tabular}{l}
|\input{childdoc.def}|\\
|\childdocby{|\textit{main}|}|\\
\end{tabular}
\end{center}
%
Both forms have slightly different effects as described above.
The main file is prepared as usual, see \secref{sec:include}.

%%%%%%%%%%%%%%%%%%%%%%%%%%%%%%%%%%%%%%%%%%%%%%%%%%%%%%%%%%%%%%%%%%%%%%%%%%%%%%%%
\subsection{Legacy Detection}
\label{sec:detection}

The directive |\childdocmain| in the main file can detect
whether the complete document or merely a child is to be compiled
even without using the directive |\childdocof|.
This method is deprecated because it is less robust
and there is no compelling reason to use it;
it is merely provided for backward compatibility
and it may be removed in future versions.

If the detection mechanism is to be used,
it is mandatory to correctly specify
the filename of the main file as the argument of |\childdocmain|:
%
\begin{center}
\begin{tabular}{l}
|\input{childdoc.def}|\\
|\childdocmain{|\textit{main}|}|\\
\end{tabular}
\end{center}
%
If |\jobname| does not match the argument \textit{main} of |\childdocmain|,
it is assumed that |\jobname| points to the child file to be compiled.
When using |\childdocmain| with the main file specified as argument,
it suffices to start a child file
with just |\input{|\textit{main}|}|
without loading of the package and using |\childdocof|.
If instead all processing is done
with the appropriate \textsf{childdoc} directives,
the argument of \textit{main} of |\childdocmain| can be empty.

An alternative version of the command line processing described
in \secref{sec:commandline} using the detection mechanism reads:
%
\begin{center}
|... -jobname "|\textit{target}|" "|[\textit{flags}]%
[|\def\jobname{|\textit{dest}|}|]|\input{|\textit{main}|}"|
\end{center}

%%%%%%%%%%%%%%%%%%%%%%%%%%%%%%%%%%%%%%%%%%%%%%%%%%%%%%%%%%%%%%%%%%%%%%%%%%%%%%%%
\subsection{Manual Code}
\label{sec:manual}

In case one cannot be certain whether the definitions file |childdoc.def|
is installed on the target \TeX{} distribution
and one prefers not to ship it,
it is conceivable to paste a few relevant commands into the sources.

To that end, drop all statements |\input{childdoc.def}|
and perform the replacements as outlined below.
Instead of |\childdocmain{|\textit{main}|}| add the following code
to the top of the main file:
%
\begin{center}
\begin{tabular}{l}
|\||ifdefined\childdocname\endinput\||fi\newif\ifchilddoc|\\
|\edef\childdocname{\scantokens\expandafter{\jobname\noexpand}}|\\
|\def\childdocmain{|\textit{main}|}\||ifx\childdocmain\childdocname\||else|\\
|\childdoctrue\includeonly{\childdocname}\let\jobname\childdocmain\||fi|\\
\end{tabular}
\end{center}
%
Instead of |\childdocof{|\textit{main}|}| just include the main file
at the top of each child file:
%
\begin{center}
|\input{|\textit{main}|}|
\end{center}
%
A simple redirection |\childdocforward{|\textit{dest}|}| is achieved by:
%
\begin{center}
|\def\jobname{|\textit{dest}|}\input{\jobname}|
\end{center}
%
The redirection with prefix
|\childdocforwardprefix[|\textit{prefix}|]{|\textit{dest}|}|
is accomplished by:
%
\begin{center}
\begin{tabular}{l}
|{\edef\jobname{\scantokens\expandafter{\jobname\noexpand}}|\\
|\def\redirectjob |\textit{prefix}|#1~~~{\gdef\jobname{|\textit{dest}|#1}}|\\
|\expandafter\redirectjob\jobname~~~}\input{\jobname}|
\end{tabular}
\end{center}

In an alternative approach,
child documents can be compiled by a specific command line
without additional code or specific definitions:
%
\begin{center}
|... -jobname "|\textit{target}|" "|[\textit{flags}]%
|\includeonly{|\textit{dest}|}\input{|\textit{main}|}"|
\end{center}
%

%%%%%%%%%%%%%%%%%%%%%%%%%%%%%%%%%%%%%%%%%%%%%%%%%%%%%%%%%%%%%%%%%%%%%%%%%%%%%%%%
%%%%%%%%%%%%%%%%%%%%%%%%%%%%%%%%%%%%%%%%%%%%%%%%%%%%%%%%%%%%%%%%%%%%%%%%%%%%%%%%
\section{Information}

%%%%%%%%%%%%%%%%%%%%%%%%%%%%%%%%%%%%%%%%%%%%%%%%%%%%%%%%%%%%%%%%%%%%%%%%%%%%%%%%
\subsection{Copyright}

Copyright \copyright{} 2017--2018 Niklas Beisert

This work may be distributed and/or modified under the
conditions of the \LaTeX{} Project Public License, either version 1.3
of this license or (at your option) any later version.
The latest version of this license is in
  \url{http://www.latex-project.org/lppl.txt}
and version 1.3 or later is part of all distributions of \LaTeX{}
version 2005/12/01 or later.

This work has the LPPL maintenance status `maintained'.

The Current Maintainer of this work is Niklas Beisert.

This work consists of the files |README.txt|, |childdoc.ins| and |childdoc.dtx|
as well as the derived files |childdoc.def|, |cdocsamp.tex|
with |cdocsch1.tex|, |cdocsch2.tex|, |cdocspt3.tex|, |cdocspt4.tex|,
|cdocsdrf.tex|, |cdocsfn1.tex|, |cdocsfn2.tex|
as well as |childdoc.pdf|.

%%%%%%%%%%%%%%%%%%%%%%%%%%%%%%%%%%%%%%%%%%%%%%%%%%%%%%%%%%%%%%%%%%%%%%%%%%%%%%%%
\subsection{Files and Installation}

The package consists of the files:
%
\begin{center}
\begin{tabular}{ll}
    |README.txt|   & readme file \\
    |childdoc.ins| & installation file \\
    |childdoc.dtx| & source file \\
    |childdoc.def| & definition file \\
    |cdocsamp.tex| & sample main file \\
    |cdocsch1.tex| & sample include file \\
    |cdocsch2.tex| & sample include file \\
    |cdocspt3.tex| & sample part file \\
    |cdocspt4.tex| & sample part file \\
    |cdocsdrf.tex| & sample redirection file \\
    |cdocsfn1.tex| & sample redirection file \\
    |cdocsfn2.tex| & sample redirection file \\
    |childdoc.pdf| & manual
\end{tabular}
\end{center}
%
The distribution consists of the files
|README.txt|, |childdoc.ins| and |childdoc.dtx|.
%
\begin{itemize}
\item
Run (pdf)\LaTeX{} on |childdoc.dtx|
to compile the manual |childdoc.pdf| (this file).
\item
Run \LaTeX{} on |childdoc.ins| to create the definitions file |childdoc.def|
and the sample |cdocsamp.tex| with include files
|cdocsch1.tex|, |cdocsch2.tex|, |cdocspt3.tex|, |cdocspt4.tex|,
|cdocsdrf.tex|, |cdocsfn1.tex|, |cdocsfn2.tex|.
Then copy the file |childdoc.def| to an appropriate directory of your \LaTeX{}
distribution, e.g.\ \textit{texmf-root}|/tex/latex/childdoc|.
\end{itemize}

%%%%%%%%%%%%%%%%%%%%%%%%%%%%%%%%%%%%%%%%%%%%%%%%%%%%%%%%%%%%%%%%%%%%%%%%%%%%%%%%
\subsection{Related CTAN Packages}

There are several other packages which offer a similar functionality:
%
\begin{itemize}
\item
The packages
\href{http://ctan.org/pkg/docmute}{\textsf{docmute}},
\href{http://ctan.org/pkg/includex}{\textsf{includex}} and
\href{http://ctan.org/pkg/standalone}{\textsf{standalone}}
provide commands to include only the document body of
a child file thus allowing both files to be compiled individually.
\item
The packages \href{http://ctan.org/pkg/subdocs}{\textsf{subdocs}}
and \href{http://ctan.org/pkg/subfiles}{\textsf{subfiles}}
provide structures in which the main and child documents can be
encapsulated and allowing them to be compiled individually.
The inclusion mechanism is different from the conventional |\include|.
\item
The package \href{http://ctan.org/pkg/combine}{\textsf{combine}}
is an elaborate solution to combine several documents into one.
\end{itemize}
%
See also the CTAN topic \href{http://ctan.org/topic/subdocs}{\textsf{subdocs}}
for further related packages.
The present package differs from the above solutions in that
a document structure constructed with the conventional |\include| mechanism
just needs two extra commands at the top of every file
such that all constituent files can be compiled individually.

%%%%%%%%%%%%%%%%%%%%%%%%%%%%%%%%%%%%%%%%%%%%%%%%%%%%%%%%%%%%%%%%%%%%%%%%%%%%%%%%
%\subsection{Feature Suggestions}
%
%The following is a list of features which may be useful for future
%versions of this package:
%%
%\begin{itemize}
%\item
%\ldots
%\end{itemize}

%%%%%%%%%%%%%%%%%%%%%%%%%%%%%%%%%%%%%%%%%%%%%%%%%%%%%%%%%%%%%%%%%%%%%%%%%%%%%%%%
\subsection{Revision History}

%%%%%%%%%%%%%%%%%%%%%%%%%%%%%%%%%%%%%%%%
\paragraph{v2.0:} 2018/12/30

\begin{itemize}
\item
immediate forward processing
\item
added |\childdocby| mechanism
\item
manual restructured
\end{itemize}

%%%%%%%%%%%%%%%%%%%%%%%%%%%%%%%%%%%%%%%%
\paragraph{v1.6:} 2018/01/17

\begin{itemize}
\item
application for development of include files
\item
corrections to manual
\end{itemize}

%%%%%%%%%%%%%%%%%%%%%%%%%%%%%%%%%%%%%%%%
\paragraph{v1.5:} 2017/05/21

\begin{itemize}
\item
more complete structuring introduced
\item
|\childdocof| introduced
\item
|\childdoc| renamed to |\childdocmain|
\item
|\childredirect| renamed to |\childdocforward| and |\childdocforwardprefix|
and functionality expanded
\end{itemize}

%%%%%%%%%%%%%%%%%%%%%%%%%%%%%%%%%%%%%%%%
\paragraph{v1.0:} 2017/04/27

\begin{itemize}
\item
manual and install package
\item
first version published on CTAN
\end{itemize}

%%%%%%%%%%%%%%%%%%%%%%%%%%%%%%%%%%%%%%%%
\paragraph{v0.6:} 2017/04/26

\begin{itemize}
\item
redirection mechanism added
\end{itemize}

%%%%%%%%%%%%%%%%%%%%%%%%%%%%%%%%%%%%%%%%
\paragraph{v0.5:} 2017/04/26

\begin{itemize}
\item
functionality in definition file
\end{itemize}


%%%%%%%%%%%%%%%%%%%%%%%%%%%%%%%%%%%%%%%%%%%%%%%%%%%%%%%%%%%%%%%%%%%%%%%%%%%%%%%%
%%%%%%%%%%%%%%%%%%%%%%%%%%%%%%%%%%%%%%%%%%%%%%%%%%%%%%%%%%%%%%%%%%%%%%%%%%%%%%%%
%%%%%%%%%%%%%%%%%%%%%%%%%%%%%%%%%%%%%%%%%%%%%%%%%%%%%%%%%%%%%%%%%%%%%%%%%%%%%%%%
\appendix

\settowidth\MacroIndent{\rmfamily\scriptsize 000\ }

 \DocInput{childdoc.dtx}

\end{document}
%</driver>
% \fi
%
% %%%%%%%%%%%%%%%%%%%%%%%%%%%%%%%%%%%%%%%%%%%%%%%%%%%%%%%%%%%%%%%%%%%%%%%%%%%%%%
% %%%%%%%%%%%%%%%%%%%%%%%%%%%%%%%%%%%%%%%%%%%%%%%%%%%%%%%%%%%%%%%%%%%%%%%%%%%%%%
% \section{Sample}
%\iffalse
%<*samplemain>
%\fi
%
% The following presents a sample document
% with two chapters, two parts, a title page,
% a compile flag as well as three forwarding files to set the flag.
% It consists of eight |.tex| files:
% \begin{center}
% \begin{tabular}{ll}
% |cdocsamp.tex|&main file\\
% |cdocsch1.tex|&include file for chapter 1\\
% |cdocsch2.tex|&include file for chapter 2\\
% |cdocspt3.tex|&include file for part 3\\
% |cdocspt4.tex|&include file for part 4\\
% |cdocsdrf.tex|&forwarding file for main file in draft mode\\
% |cdocsfi1.tex|&forwarding file for final version of chapter 1\\
% |cdocsfi2.tex|&forwarding file for final version of chapter 2\\
% \end{tabular}
% \end{center}
% Each of the eight files can be compiled directly by the \LaTeX{} compiler.
%
% %%%%%%%%%%%%%%%%%%%%%%%%%%%%%%%%%%%%%%
% \paragraph{Main File.}
%
% The main file is called |cdocsamp.tex|.
%
% Load the \textsf{childdoc} definitions and
% declare the filename for the main document:
%    \begin{macrocode}
\input{childdoc.def}
\childdocmain{}
%    \end{macrocode}

% Optional override for |\version| flag:
%    \begin{macrocode}
%%\ifchilddoc\else\providecommand{\version}{draft}\fi
%    \end{macrocode}

% Define the default values for the |\version| flag
% (|final| for the main file and |draft| for childs):
%    \begin{macrocode}
\ifchilddoc
\providecommand{\version}{draft}
\else
\providecommand{\version}{final}
\fi
%    \end{macrocode}

% Load the standard document class:
%    \begin{macrocode}
\documentclass[12pt]{article}
%    \end{macrocode}

% Start the document body:
%    \begin{macrocode}
\begin{document}
%    \end{macrocode}

% Declare a title page.
% Print title, part of document being processed and version flag:
%    \begin{macrocode}
\addtocounter{page}{-1}
\begin{center}
{\LARGE\bfseries{}childdoc example\par}
\vspace{1cm}
\ifchilddoc
\ifchilddocmanual part\else chapter\fi:
`\childdocname' of `\childdocjob'\par
\else
main document: `\childdocjob'\par
\fi
version: \version\par
\end{center}
\newpage
%    \end{macrocode}

% Manually include selected file,
% otherwise process as usual:
%    \begin{macrocode}
\ifchilddocmanual
\section*{part `\childdocname'}
\input{\childdocname}
\else
%    \end{macrocode}

% Include the two chapters:
%    \begin{macrocode}
\include{cdocsch1}
\include{cdocsch2}
%    \end{macrocode}

% Include the two parts unless only chapters should be displayed:
%    \begin{macrocode}
\ifchilddoc\else
\section{part three}
\input{cdocspt3}
\section{part four}
\input{cdocspt4}
\fi
%    \end{macrocode}

% Process as usual until here:
%    \begin{macrocode}
\fi
%    \end{macrocode}

% End of document body:
%    \begin{macrocode}
\end{document}
%    \end{macrocode}
%\iffalse
%</samplemain>
%\fi
%
% %%%%%%%%%%%%%%%%%%%%%%%%%%%%%%%%%%%%%%
% \paragraph{Chapter Include Files.}
%
% The include files are called |cdocsch1.tex| and |cdocsch2.tex|.
%
%\iffalse
%<*samplechap1|samplechap2>
%\fi

% Optional override for |\version| flag:
%    \begin{macrocode}
%%\providecommand{\version}{final}
%    \end{macrocode}

% Include the main document:
%    \begin{macrocode}
\input{childdoc.def}
\childdocof{cdocsamp}
%    \end{macrocode}

%\iffalse
%</samplechap1|samplechap2>
%\fi
%
%\iffalse
%<*samplechap1>
%\fi
% Some text for chapter 1:
%    \begin{macrocode}
\section{one}
some text in chapter one
%    \end{macrocode}

%\iffalse
%</samplechap1>
%\fi
% Some text for chapter 2:
%\iffalse
%<*samplechap2>
%\fi
%    \begin{macrocode}
\section{two}
more text in chapter two
%    \end{macrocode}

%\iffalse
%</samplechap2>
%\fi
%
% %%%%%%%%%%%%%%%%%%%%%%%%%%%%%%%%%%%%%%
% \paragraph{Part Include Files.}
%
% The include files are called |cdocspt3.tex| and |cdocspt4.tex|.
%
%\iffalse
%<*samplepart3|samplepart4>
%\fi

% Optional override for |\version| flag:
%    \begin{macrocode}
%%\providecommand{\version}{final}
%    \end{macrocode}

% Include the main document:
%    \begin{macrocode}
\input{childdoc.def}
\childdocby{cdocsamp}
%    \end{macrocode}

%\iffalse
%</samplepart3|samplepart4>
%\fi
%
%\iffalse
%<*samplepart3>
%\fi
% Some text for part 3:
%    \begin{macrocode}
some text in part three
%    \end{macrocode}

%\iffalse
%</samplepart3>
%\fi
% Some text for part 4:
%\iffalse
%<*samplepart4>
%\fi
%    \begin{macrocode}
more text in part four
%    \end{macrocode}

%\iffalse
%</samplepart4>
%\fi
%
% %%%%%%%%%%%%%%%%%%%%%%%%%%%%%%%%%%%%%%
% \paragraph{Forwarding for a Complete Draft.}
%
% The following forwarding file |cdocsdrf.tex|
% compiles the main document in draft mode:
%\iffalse
%<*sampledraft>
%\fi
%    \begin{macrocode}
\def\version{draft}
\input{childdoc.def}
\childdocforward{cdocsamp}
%    \end{macrocode}

%\iffalse
%</sampledraft>
%\fi
%
% %%%%%%%%%%%%%%%%%%%%%%%%%%%%%%%%%%%%%%
% \paragraph{Forwarding for Final Version of the Chapters.}
%
% The following forwarding files |cdocsfn1.tex| and |cdocsfn2.tex|
% (with identical content)
% compile the final versions of the child documents
% |cdocsch1.tex| and |cdocsch2.tex|, respectively:
%\iffalse
%<*samplefinal>
%\fi
%    \begin{macrocode}
\def\version{final}
\input{childdoc.def}
\childdocforwardprefix[cdocsamp]{cdocsfn}{cdocsch}
%    \end{macrocode}

%\iffalse
%</samplefinal>
%\fi
%
% %%%%%%%%%%%%%%%%%%%%%%%%%%%%%%%%%%%%%%
% \paragraph{Command Line Processing.}
%
% The following three command lines generate the output files
% |cdocscld|, |cdocscl1| and |cdocscl2|
% which should be identical to
% |cdocsdrf|, |cdocsch1| and |cdocsfn2|, respectively:
% \begin{center}
% \begin{tabular}{l}
% |latex -jobname cdocscld \|\\
% |  "\def\version{draft}\input{childdoc.def}\childdocforward{cdocsamp}"|\\
% |latex -jobname cdocscl1 \|\\
% |  "\input{childdoc.def}\childdocforward[cdocsamp]{cdocsch1}"|\\
% |latex -jobname cdocscl2 \|\\
% |  "\def\version{final}\input{childdoc.def}\childdocforward{cdocsch2}"|
% \end{tabular}
% \end{center}
% Note that the trailing backslash on each first line
% merely continues the input to the second line
% (for convenient cut ant paste).
% Furthermore, the command |latex| can be replaced by any
% of its alternative versions such as |pdflatex|.
%
% %%%%%%%%%%%%%%%%%%%%%%%%%%%%%%%%%%%%%%%%%%%%%%%%%%%%%%%%%%%%%%%%%%%%%%%%%%%%%%
% %%%%%%%%%%%%%%%%%%%%%%%%%%%%%%%%%%%%%%%%%%%%%%%%%%%%%%%%%%%%%%%%%%%%%%%%%%%%%%
% \section{Implementation}
%\iffalse
%<*package>
%\fi
%
% This section describes the definitions file |childdoc.def|.

% The definitions cannot be loaded using |\usepackage| or |\RequirePackage|
% which has a mechanism to prevent loading a style file more than once.
% When loading the definitions by means of |\input|
% multiple instances have to be prevented manually:
%\iffalse
%This code needs to be before the `\ProvidesFile' directive
%which is defined at the beginning of this file.
%Therefore it is also placed there and commented out here.
%</package>
%<*discard>
%\fi
%    \begin{macrocode}
\ifdefined\childdocmain\endinput\fi
%    \end{macrocode}
%\iffalse
%</discard>
%<*package>
%\fi
%
% \macro{\ifchilddoc}
% \macro{\ifchilddocmanual}
% The conditional |\ifchilddoc| tells whether a
% child (true) or main (false) document is being compiled.
% The conditional |\ifchilddocmanual| tells whether
% the |\includeonly| mechanism is used (false) or
% the selection of child files must be performed manually (true).
% The definitions initialise to false:
%    \begin{macrocode}
\newif\ifchilddoc
\newif\ifchilddocmanual
%    \end{macrocode}

% \macro{\childdocname}
% \macro{\childdocjob}
% The macro |\childdocname| stores the name of the main document
% to be compiled. The macro |\childdocjob| stores the name of
% the document on which the \LaTeX{} compiler was originally invoked.
% The content of |\jobname| cannot be compared
% to filenames specified in the source due to different catcodes.
% The following code rescans |\jobname|, stores the result
% in |\childdocname| and saves a copy in |\childdocjob|:
%    \begin{macrocode}
\edef\childdocname{\scantokens\expandafter{\jobname\noexpand}}
\let\childdocjob\childdocname
%    \end{macrocode}

% \macro{\childdocdisable}
% The macro |\childdocdisable| prevents the main file
% from being processed more than once.
% At this stage, the main document command |\childdocmain|
% is assumed to be called once again where it should do nothing.
% Any subsequent call to it should prevent
% a secondary processing of the main document
% It overwrites the forwarding commands
% |\childdocof| and |\childdocforward|
% with empty macros to prevent further inclusions of the main document:
%    \begin{macrocode}
\newcommand{\childdocdisable}
{
  \renewcommand{\childdocmain}[1]{\renewcommand{\childdocmain}[1]{\endinput}}
  \renewcommand{\childdocof}[1]{}
  \renewcommand{\childdocby}[2][]{}
  \renewcommand{\childdocforward}[2][]{}
  \renewcommand{\childdocdisable}{}
}
%    \end{macrocode}

% \macro{\childdocmain}
% The macro |\childdocmain| is to be called at the top of the main file
% with nothing or the main filename (without extension) as argument.
% First, it breaks loops.
% If the argument is not empty and does not match |\childdocname|
% (which is set by the first inclusion of |childdoc.def|),
% |\ifchilddoc| is set to true, |\includeonly| is applied to the child file
% and |\jobname| is set to the main file
% (for proper handling of |.aux| files):
%    \begin{macrocode}
\newcommand{\childdocmain}[1]
{
  \childdocdisable\childdocmain{}
  \if?#1?\else
    \begingroup
      \def\childdoctmp{#1}
      \ifx\childdoctmp\childdocname
        \def\childdoctmp{}
      \else
        \def\childdoctmp
        {
          \childdoctrue
          \includeonly{\childdocname}
          \def\childdocjob{#1}
          \def\jobname{#1}
        }
      \fi
      \expandafter
    \endgroup
    \childdoctmp
  \fi
}
%    \end{macrocode}

% \macro{\childdocof}
% The command |\childdocof| redirects
% compilation to the main file |#1|.
%    \begin{macrocode}
\newcommand{\childdocof}[1]
{
  \childdocdisable
  \childdoctrue
  \includeonly{\childdocname}
  \def\jobname{#1}
  \def\childdocjob{#1}
  \input{#1}
}
%    \end{macrocode}

% \macro{\childdocby}
% The command |\childdocby| ....
%    \begin{macrocode}
\newcommand{\childdocby}[2][]
{
  \childdocdisable
  \childdoctrue
  \childdocmanualtrue
  \if?#1?\else
    \def\jobname{#2}
  \fi
  \def\childdocjob{#2}
  \input{#2}
  \endinput
}
%    \end{macrocode}

% \macro{\childdocforward}
% The command |\childdocforward| redirects
% compilation to the main file or
% (if the optional argument is given) a child file.
% Parameters are set as if the main file
% or a child file starting with |\childdocof| was compiled.
% Then compilation is handed over to the main file:
%    \begin{macrocode}
\newcommand{\childdocforward}[2][]
{
  \begingroup
    \if?#1?
      \def\childdoctmp
      {
        \def\childdocname{#2}
        \def\childdocjob{#2}
        \def\jobname{#2}
        \input{#2}
        \endinput
      }
    \else
      \def\childdoctmp
      {
        \childdocdisable
        \def\childdocname{#2}
        \childdoctrue
        \includeonly{#2}
        \def\childdocjob{#1}
        \def\jobname{#1}
        \input{#1}
        \endinput
      }
    \fi
    \expandafter
  \endgroup
  \childdoctmp
}
%    \end{macrocode}

% \macro{\childdocforwardprefix}
% The command |\childdocforwardprefix| redirects
% compilation to the main or a child file by means of a pattern.
% The prefix |#1| in the current filename is replaced by |#2|
% and the suffix of the current filename is kept
% (it is assumed that the filename does not contain the substring `|~~~|'
% which is used as a delimiter).
% Compilation is handed over to the new file by |\childdocforward|:
%    \begin{macrocode}
\newcommand{\childdocforwardprefix}[3][]
{
  \begingroup
    \def\childdocextract #2##1~~~{\def\childdoctmp{\childdocforward[#1]{#3##1}}}
    \expandafter\childdocextract\childdocname~~~
    \expandafter
  \endgroup
  \childdoctmp
}
%    \end{macrocode}

% \macro{\childdoc}
% The deprecated macro |\childdoc| is a legacy version of |\childdocmain|:
%    \begin{macrocode}
\newcommand{\childdoc}{\childdocmain}
%    \end{macrocode}

% \macro{\childdocredirect}
% The deprecated macro |\childdocredirect| is a legacy version
% of |\childdocforward| and |\childdocforwardprefix|:
%    \begin{macrocode}
\newcommand{\childdocredirect}[2][]
{
  \begingroup
    \if?#1?
      \def\childdoctmp{\childdocforward{#2}}
    \else
      \def\childdoctmp{\childdocforwardprefix{#1}{#2}}
    \fi
    \expandafter
  \endgroup
  \childdoctmp
}
%    \end{macrocode}

%\iffalse
%</package>
%\fi
%
\endinput
\childdocforward[|\textit{main}|]{|\textit{dest}|}"|
\end{center}
%
Here \textit{target} is the name of the output file,
\textit{main} is the name of the main file
and \textit{dest} is the name of the main or child file to be processed
(all filenames without extensions).
The optional argument \textit{main} can be omitted
if \textit{main} matches \textit{dest}.
Optionally, compilation \textit{flags} can be defined via |\def| commands.
This command line makes the \TeX{} engine believe
it is compiling the file \textit{target}
whose content is specified as the latter parameter.
The provided code then forwards the processing to
\textit{main} or \textit{dest} as described in \secref{sec:forward}.

%%%%%%%%%%%%%%%%%%%%%%%%%%%%%%%%%%%%%%%%%%%%%%%%%%%%%%%%%%%%%%%%%%%%%%%%%%%%%%%%
\subsection{Include by Input}
\label{sec:input}

Including child documents by |\include| has some restrictions by design.
Most notably, the content of a child document always occupies
its own set of pages; pages cannot be shared between child documents.
Usually, this behaviour makes perfect sense
because each child document contain an essential part of the document.
However, in some situations it may be desirable to compose
a document from a collection of parts
without having mandatory page breaks between then.
For this case, the package
provides a mechanism to include parts
by |\input| which can also be processed individually.
However, by construction this mechanism
requires manual handling of the content to be output.

%%%%%%%%%%%%%%%%%%%%%%%%%%%%%%%%%%%%%%%%
\DescribeMacro{\ifchilddocmanual}
The main file should be prepared as usual, see \secref{sec:include}.
However, the document body must make a distinction
between processing of an individual part and of the main document, e.g.:
%
\begin{center}
\begin{tabular}{l}
|\ifchilddocmanual|\\
|\input{\childdocname}|\\
|\||else|\\
\textit{document body with }|\input{|\textit{part}|}|\\
|\||fi|
\end{tabular}
\end{center}
%
The conditional |\ifchilddocmanual| is true whenever
a part to be included by |\input| is being compiled,
and the name of the part is stored in |\childdocname|.

%%%%%%%%%%%%%%%%%%%%%%%%%%%%%%%%%%%%%%%%
\DescribeMacro{\childdocby}
Each part to be included by |\input| should start with:
%
\begin{center}
\begin{tabular}{l}
|% \iffalse
%
% childdoc.dtx Copyright (C) 2017-2018 Niklas Beisert
%
% This work may be distributed and/or modified under the
% conditions of the LaTeX Project Public License, either version 1.3
% of this license or (at your option) any later version.
% The latest version of this license is in
%   http://www.latex-project.org/lppl.txt
% and version 1.3 or later is part of all distributions of LaTeX
% version 2005/12/01 or later.
%
% This work has the LPPL maintenance status `maintained'.
%
% The Current Maintainer of this work is Niklas Beisert.
%
% This work consists of the files childdoc.dtx and childdoc.ins
% and the derived files childdoc.def and cdocsamp.tex with
% cdocsch1.tex, cdocsch2.tex, cdocsdrf.tex, cdocsfn1.tex, cdocsfn2.tex.
%
%<package>\ifdefined\childdocmain\endinput\fi
%<package>\ProvidesFile{childdoc.def}[2018/12/30 v2.0 child document driver]
%<samplemain>\ProvidesFile{cdocsamp.tex}[2018/12/30 v2.0 sample for childdoc]
%<*driver>
%\ProvidesFile{childdoc.drv}[2018/12/30 v2.0 childdoc reference manual file]
\PassOptionsToClass{10pt,a4paper}{article}
\documentclass{ltxdoc}

\usepackage[margin=35mm]{geometry}
\usepackage{hyperref}
\usepackage{hyperxmp}
\usepackage[usenames]{color}

\hypersetup{colorlinks=true}
\hypersetup{pdfstartview=FitH}
\hypersetup{pdfpagemode=UseNone}
\hypersetup{pdfsource={}}
\hypersetup{pdflang={en-UK}}
\hypersetup{pdfcopyright={Copyright 2017-2018 Niklas Beisert.
  This work may be distributed and/or modified under the
  conditions of the LaTeX Project Public License, either version 1.3
  of this license or (at your option) any later version.}}
\hypersetup{pdflicenseurl={http://www.latex-project.org/lppl.txt}}
\hypersetup{pdfcontactaddress={ETH Zurich, ITP, HIT K,
  Wolfgang-Pauli-Strasse 27}}
\hypersetup{pdfcontactpostcode={8093}}
\hypersetup{pdfcontactcity={Zurich}}
\hypersetup{pdfcontactcountry={Switzerland}}
\hypersetup{pdfcontactemail={nbeisert@itp.phys.ethz.ch}}
\hypersetup{pdfcontacturl={http://people.phys.ethz.ch/\xmptilde nbeisert/}}

\newcommand{\secref}[1]{\hyperref[#1]{section \ref*{#1}}}

\parskip1ex
\parindent0pt
\let\olditemize\itemize
\def\itemize{\olditemize\parskip0pt}

\begin{document}

\title{The \textsf{childdoc} Package}
\hypersetup{pdftitle={The childdoc Package}}
\author{Niklas Beisert\\[2ex]
  Institut f\"ur Theoretische Physik\\
  Eidgen\"ossische Technische Hochschule Z\"urich\\
  Wolfgang-Pauli-Strasse 27, 8093 Z\"urich, Switzerland\\[1ex]
  \href{mailto:nbeisert@itp.phys.ethz.ch}
  {\texttt{nbeisert@itp.phys.ethz.ch}}}
\hypersetup{pdfauthor={Niklas Beisert}}
\hypersetup{pdfsubject={Manual for the LaTeX2e Package childdoc}}
\date{30 December 2018, \textsf{v2.0}}
\maketitle

\begin{abstract}\noindent
\textsf{childdoc} is a \LaTeXe{} package
that enables the direct compilation
of document sections included by |\include|
to individual files.
\end{abstract}

\begingroup
\parskip0ex
\tableofcontents
\endgroup

%%%%%%%%%%%%%%%%%%%%%%%%%%%%%%%%%%%%%%%%%%%%%%%%%%%%%%%%%%%%%%%%%%%%%%%%%%%%%%%%
%%%%%%%%%%%%%%%%%%%%%%%%%%%%%%%%%%%%%%%%%%%%%%%%%%%%%%%%%%%%%%%%%%%%%%%%%%%%%%%%
\section{Introduction}

\LaTeX{} provides a mechanism to structure a large document (such as a book)
into a main file and several child files (containing the chapters)
using the |\include| command.
This mechanism is beneficial for documents
which span hundreds of pages in order to
make the source file(s) more manageable.
Moreover, compilation can be restricted to
selected child files by means of the |\includeonly| command.
The latter feature can be used to reduce the compilation time while editing
(this was significantly more useful in the earlier days of \LaTeX{})
or to generate a smaller document which is easier to navigate.
Another application of |\includeonly| is to generate
documents consisting of selected parts of the complete document.

However, there are a few drawbacks of the plain |\include| mechanism:
\begin{itemize}
\item
The child files cannot be compiled on their own,
they can only be compiled via the main file.
A naive editing environment
(such as a text editor with an option
to have the current file processed by \LaTeX)
may require one to switch to the main file before compiling;
attempting to compile the child file produces errors.
\item
The main file must be modified (each time)
to adjust the |\includeonly| command
to the present needs. This easily leaves the main file in a messy state.
\item
The generated document will always carry the filename
of the main document. This is inconvenient if
several child files are to be compiled and
to be kept for distribution.
\end{itemize}

The present package provides a simple interface
to make child files individually compilable by \LaTeX{}.
Compiling a child file then has the same effect as compiling
the main file with an |\includeonly| command
to select the appropriate child.
Moreover the generated document will carry the name of the child
rather than the main file.
This resolves all three above issues.

This feature is meant to make the editing of books,
thesis documents and lecture notes somewhat more convenient.
However, the package can also be used efficiently for
composing a series of documents (such as exercise sheets)
which are typically distributed individually.
It then assists the author in generating the individual documents
(potentially in different versions)
as well as a document containing the collected series.
Another application is in developing style files
or other kinds of included material
where compilation of the style file could redirect
to a sample or test file.

%%%%%%%%%%%%%%%%%%%%%%%%%%%%%%%%%%%%%%%%%%%%%%%%%%%%%%%%%%%%%%%%%%%%%%%%%%%%%%%%
%%%%%%%%%%%%%%%%%%%%%%%%%%%%%%%%%%%%%%%%%%%%%%%%%%%%%%%%%%%%%%%%%%%%%%%%%%%%%%%%
\section{Usage}

First of all, the package \textsf{childdoc} is \emph{not} a standard
\LaTeXe{} |.sty| style file! Therefore it needs to be invoked in
a non-standard way.

%%%%%%%%%%%%%%%%%%%%%%%%%%%%%%%%%%%%%%%%%%%%%%%%%%%%%%%%%%%%%%%%%%%%%%%%%%%%%%%%
\subsection{Included Files}
\label{sec:include}

%%%%%%%%%%%%%%%%%%%%%%%%%%%%%%%%%%%%%%%%
\DescribeMacro{\childdocmain}
To use the package, add the commands
\begin{center}
\begin{tabular}{l}
|\input{childdoc.def}|\\
|\childdocmain{}|\\
\end{tabular}
\end{center}
at the very top of the main \LaTeX{} file,
in particular \emph{before} the |\documentclass| statement!
The argument of |\childdocmain| should be left empty
(but it must be present).

%%%%%%%%%%%%%%%%%%%%%%%%%%%%%%%%%%%%%%%%
\DescribeMacro{\childdocof}
Furthermore, add the commands
\begin{center}
\begin{tabular}{l}
|\input{childdoc.def}|\\
|\childdocof{|\textit{main}|}|\\
\end{tabular}
\end{center}
at the top of every child file \textit{child}
which is included by |\include{|\textit{child}|}|
from within the main file
(or at least for those files to be compiled individually).
The argument \textit{main} must be the filename of the main file.

There are a couple of
considerations in setting up the main and child documents:

%%%%%%%%%%%%%%%%%%%%%%%%%%%%%%%%%%%%%%%%
\paragraph{Restrictions.}

Please note the following restrictions:
\begin{itemize}
\item
|\childdocmain| must be called with one argument \textit{main}
to ensure compatibility with earlier version of the package.
It must either be empty (|\childdocmain{}|)
or precisely match the filename of the main file in which it is specified.
See \secref{sec:detection} for further information.
\item
The filename \textit{main} must be specified without the |.tex| extension.
\item
The filename \textit{main} is case sensitive
(even in case-insensitive file systems)
due to internal string comparison.
\item
The argument \textit{main} should be fully expanded, it cannot be a macro.
\item
Subdirectories and special characters should be avoided in filenames.
\item
The command |\childdocmain{|\textit{main}|}| must be followed by a whitespace.
It should not be followed immediately by another command
or by a comment mark `|%|'.
This is because the \TeX{} parser reads the token immediately following
the argument of |\childdocmain| and puts it
at the beginning of every child section;
however, a white\-space is ignored.
\end{itemize}

%%%%%%%%%%%%%%%%%%%%%%%%%%%%%%%%%%%%%%%%
\paragraph{Content of Main File.}

It is advisable to place all content in the child files included by |\include|.
Any output contained in the main file will appear in all child documents
unless suppressed manually;
it cannot be suppressed automatically by the |\includeonly| directive
and thus should normally be avoided.
A method to include some content in the main file
by means of conditional processing is described in \secref{sec:conditional}.

%%%%%%%%%%%%%%%%%%%%%%%%%%%%%%%%%%%%%%%%
\paragraph{Page Numbering.}

When only a part of the document is compiled,
the appropriate numbering of pages
(as well as other status parameters)
is determined from the |.aux| files.
The latter contain information from previous passes.
However this information needs to propagate through
all intermediate child documents.
Therefore the page numbering in child documents may well
be inconsistent until the complete document is compiled at least once.

A useful (if unconventional) way to always ensure a consistent
page numbering is to restart the numbering in each child document
and denote the pages by `\textit{child}|.|\textit{page}'
where \textit{child} represents the chapter/section number of the child file.
This can be achieved by the command
|\numberwithin{page}{|\textit{child}|}|
of the \textsf{amsmath} package
where \textit{child} can be |chapter| or |section|
depending on the chosen structuring.
Alternatively, one can modify the macro |\thepage| appropriately
and reset the counter |page| at the start of each child file.

%%%%%%%%%%%%%%%%%%%%%%%%%%%%%%%%%%%%%%%%%%%%%%%%%%%%%%%%%%%%%%%%%%%%%%%%%%%%%%%%
\subsection{Conditional Processing}
\label{sec:conditional}

The package provides a mechanism to compile different versions
of a document. To customise the versions further some conditional processing
can come in handy to distinguish which version is being compiled.
The package provides two macros to describe the compilation context:

%%%%%%%%%%%%%%%%%%%%%%%%%%%%%%%%%%%%%%%%
\DescribeMacro{\ifchilddoc}
The conditional |\ifchilddoc| distinguishes between the compilation of
child documents and the main document:
%
\begin{center}
|\ifchilddoc |\textit{child-code}| |[|\||else |\textit{main-code}]| \||fi|
\end{center}

%%%%%%%%%%%%%%%%%%%%%%%%%%%%%%%%%%%%%%%%
\DescribeMacro{\childdocname}
\DescribeMacro{\childdocjob}
The macro |\childdocname| contains the filename (without extension)
of the main or child file being processed.
Note that |\childdocjob| will always contain the name of the main file.

%%%%%%%%%%%%%%%%%%%%%%%%%%%%%%%%%%%%%%%%
\paragraph{Title Page.}

Conditional processing can be used to include a title or banner page
in the main document when proper precautions are taken.
Importantly, the code in the main file should ensure that the page counter
(as well as other status parameters which are stored in the |.aux| files)
takes the same value after the conditional processing.
Otherwise the page numbers may take divergent values
depending on which part is compiled.

For example, a title page could be declared by:
%
\begin{center}
\begin{tabular}{l}
|\ifchilddoc\||else|\\
|\addtocounter{page}{-1}|\\
\textit{code for title page}\\
|\newpage|\\
|\||fi|
\end{tabular}
\end{center}
%
A banner page for the child documents can be generated by:
%
\begin{center}
\begin{tabular}{l}
|\ifchilddoc|\\
|\addtocounter{page}{-1}|\\
\textit{code for banner page}\\
|\newpage|\\
|\||fi|
\end{tabular}
\end{center}
%
Here one could write a message such as:
\begin{center}
|This is the part \childdocname{} of \childdocjob{}.|
\end{center}

%%%%%%%%%%%%%%%%%%%%%%%%%%%%%%%%%%%%%%%%%%%%%%%%%%%%%%%%%%%%%%%%%%%%%%%%%%%%%%%%
\subsection{Flags}
\label{sec:flags}

The package makes it easy to generate different versions
of the main or child documents.
To this end compilation flags can be defined
and assigned different default values.
They will be particularly useful in conjunction
with the forwarding mechanism described in \secref{sec:forward}.

For example, it may be useful to have a flag |\version|
which can be set to |draft| or |final|.
The document source will contain some conditional code
depending on the value of |\version|.
Suppose further, the flag should default to |final| for the main file
and to |draft| for child files
which is a natural assignment for editing the document.
This is achieved by placing the following code
in the preamble of the main document
(below the |\childdocmain| directive):
%
\begin{center}
\begin{tabular}{l}
|\ifchilddoc|\\
|\providecommand{\version}{draft}|\\
|\||else|\\
|\providecommand{\version}{final}|\\
|\||fi|
\end{tabular}
\end{center}
%
The definition by |\providecommand| makes sure
that previous definitions are not overwritten.
Further statements |\providecommand{\version}{...}|
can thus be added before the above code to override it.

For the main file, one might add a line
(between |\childdocmain| and the above block)
%
\begin{center}
|%\ifchilddoc\||else\providecommand{\version}{draft}\||fi|
\end{center}
%
which can be uncommented to produce a draft version.
Likewise one can add a line to the very top of a child file
(above the |\childdocof{|\textit{main}|}| directive)
%
\begin{center}
|%\providecommand{\version}{final}|
\end{center}
%
which can be uncommented to produce the final version of this child document.

%%%%%%%%%%%%%%%%%%%%%%%%%%%%%%%%%%%%%%%%%%%%%%%%%%%%%%%%%%%%%%%%%%%%%%%%%%%%%%%%
\subsection{Forwarding}
\label{sec:forward}

Different versions of the main or child documents
using compilation flags as described in \secref{sec:flags}
can be (permanently) stored in different files
for convenient compilation, viewing and distribution.
To this end, the package defines a command
to pass on compilation to a different file:

%%%%%%%%%%%%%%%%%%%%%%%%%%%%%%%%%%%%%%%%
\DescribeMacro{\childdocforward}
The command |\childdocforward| redirects processing to
another source file:
%
\begin{center}
\begin{tabular}{l}
|\input{childdoc.def}|\\
|\childdocforward[|\textit{main}|]{|\textit{dest}|}|\\
\end{tabular}
\end{center}
%
The argument \textit{dest} is the destination file
(without extension).
It should be the main file or one of the child files.
Note that further \textsf{childdoc} directives
such as |\childdocof| and |\childdocforward|
in the indicated file will be processed in this form.
The optional argument \textit{main}
passes on directly to the main file \textit{main}
while pretending to compile the child \textit{dest}.
This form behaves as if \textit{dest}
issues |\childdocof{|\textit{main}|}| right away,
and no further \textsf{childdoc} directives will be processed.

%%%%%%%%%%%%%%%%%%%%%%%%%%%%%%%%%%%%%%%%
\DescribeMacro{\...prefix}
In the alternative form |\childdocforwardprefix|,
%
\begin{center}
\begin{tabular}{l}
|\input{childdoc.def}|\\
|\childdocforwardprefix[|\textit{main}|]{|\textit{prefix}|}{|\textit{dest}|}|
\end{tabular}
\end{center}
%
the destination file is determined by a pattern
depending on the current file:
To make this work, the current file must be called
`{\textit{prefix}\hspace{0.2em}\textit{suffix}}'
with \textit{prefix} matching precisely the argument.
Processing is then passed on to the file
`{\textit{dest}\hspace{0.2em}\textit{suffix}}'.
Surely, the same effect is achieved by
directly specifying the
argument `{\textit{dest}\hspace{0.2em}\textit{suffix}}'
in the first form.
However, that requires to set up a different file
for each child. With the alternative form of the command
all these files can have exactly the same content
which simplifies setting them up and maintaining them.

For example, the following file |draft.tex|
with a compilation flag |\version| as described in \secref{sec:flags}
compiles the main document as a draft:
%
\begin{center}
\begin{tabular}{l}
|\def\version{draft}|\\
|\input{childdoc.def}|\\
|\childdocforward{|\textit{main}|}|
\end{tabular}
\end{center}
%
Likewise, the following files |final|\textit{nn}|.tex|
compile the final version of the child document
|child|\textit{nn}|.tex|:
%
\begin{center}
\begin{tabular}{l}
|\def\version{final}|\\
|\input{childdoc.def}|\\
|\childdocforwardprefix{final}{child}|
\end{tabular}
\end{center}
%

Note that when several versions of a main file and/or of each child file
are to be generated, it may be convenient to set up a |Makefile| or
shell script to automatise the process.

%%%%%%%%%%%%%%%%%%%%%%%%%%%%%%%%%%%%%%%%%%%%%%%%%%%%%%%%%%%%%%%%%%%%%%%%%%%%%%%%
\subsection{Command Line Processing}
\label{sec:commandline}

The effect of redirection files can also be achieved by invoking
the \LaTeX{} compiler with a more elaborate command line.
Most conveniently this should be done as part
of a shell script or a |Makefile|.

When using \textsf{childdoc} in the main file, the following
command lines effectively perform a redirection
(note that depending on the shell being used,
backslashes may have to be doubled: `|\|' $\to$ `|\\|'):
%
\begin{center}
|... -jobname "|\textit{target}|" |\\|"|[\textit{flags}]%
|\input{childdoc.def}\childdocforward[|\textit{main}|]{|\textit{dest}|}"|
\end{center}
%
Here \textit{target} is the name of the output file,
\textit{main} is the name of the main file
and \textit{dest} is the name of the main or child file to be processed
(all filenames without extensions).
The optional argument \textit{main} can be omitted
if \textit{main} matches \textit{dest}.
Optionally, compilation \textit{flags} can be defined via |\def| commands.
This command line makes the \TeX{} engine believe
it is compiling the file \textit{target}
whose content is specified as the latter parameter.
The provided code then forwards the processing to
\textit{main} or \textit{dest} as described in \secref{sec:forward}.

%%%%%%%%%%%%%%%%%%%%%%%%%%%%%%%%%%%%%%%%%%%%%%%%%%%%%%%%%%%%%%%%%%%%%%%%%%%%%%%%
\subsection{Include by Input}
\label{sec:input}

Including child documents by |\include| has some restrictions by design.
Most notably, the content of a child document always occupies
its own set of pages; pages cannot be shared between child documents.
Usually, this behaviour makes perfect sense
because each child document contain an essential part of the document.
However, in some situations it may be desirable to compose
a document from a collection of parts
without having mandatory page breaks between then.
For this case, the package
provides a mechanism to include parts
by |\input| which can also be processed individually.
However, by construction this mechanism
requires manual handling of the content to be output.

%%%%%%%%%%%%%%%%%%%%%%%%%%%%%%%%%%%%%%%%
\DescribeMacro{\ifchilddocmanual}
The main file should be prepared as usual, see \secref{sec:include}.
However, the document body must make a distinction
between processing of an individual part and of the main document, e.g.:
%
\begin{center}
\begin{tabular}{l}
|\ifchilddocmanual|\\
|\input{\childdocname}|\\
|\||else|\\
\textit{document body with }|\input{|\textit{part}|}|\\
|\||fi|
\end{tabular}
\end{center}
%
The conditional |\ifchilddocmanual| is true whenever
a part to be included by |\input| is being compiled,
and the name of the part is stored in |\childdocname|.

%%%%%%%%%%%%%%%%%%%%%%%%%%%%%%%%%%%%%%%%
\DescribeMacro{\childdocby}
Each part to be included by |\input| should start with:
%
\begin{center}
\begin{tabular}{l}
|\input{childdoc.def}|\\
|\childdocby{|\textit{main}|}|\\
\end{tabular}
\end{center}
%
The directive |\childdocby| is similar to |\childdocof|
described in \secref{sec:include},
but the subsequent selection of content must be done manually.
To that end, both |\ifchilddoc| and |\ifchilddocmanual|
will be true upon processing of a part,
and the name of the part is stored in |\childdocname|.
Note that |\jobname| will be set to the filename of the current part
so that each part receives an individual |.aux| file
that does not interfere with the |.aux| file(s) of the main document.
This behaviour can be altered by the alternative form
|\childdocby[*]{|\textit{main}|}| (with a non-empty optional argument)
which uses the |.aux| file of the main document
by setting |\jobname| to \textit{main}.

%%%%%%%%%%%%%%%%%%%%%%%%%%%%%%%%%%%%%%%%%%%%%%%%%%%%%%%%%%%%%%%%%%%%%%%%%%%%%%%%
\subsection{Driver Development}
\label{sec:driver}

The \textsf{childdoc} mechanism can also be use for the development
of definition files such as \LaTeX{} styles or classes.
This case differs from the above setup with multiple parts
included by |\include| in that no |\includeonly| should be invoked.
This can be achieved by starting the include file
(before |\ProvidesPackage|) with:
%
\begin{center}
\begin{tabular}{l}
|\input{childdoc.def}|\\
|\childdocforward{|\textit{main}|}|\\
\end{tabular}
\end{center}
%
or alternatively with:
%
\begin{center}
\begin{tabular}{l}
|\input{childdoc.def}|\\
|\childdocby{|\textit{main}|}|\\
\end{tabular}
\end{center}
%
Both forms have slightly different effects as described above.
The main file is prepared as usual, see \secref{sec:include}.

%%%%%%%%%%%%%%%%%%%%%%%%%%%%%%%%%%%%%%%%%%%%%%%%%%%%%%%%%%%%%%%%%%%%%%%%%%%%%%%%
\subsection{Legacy Detection}
\label{sec:detection}

The directive |\childdocmain| in the main file can detect
whether the complete document or merely a child is to be compiled
even without using the directive |\childdocof|.
This method is deprecated because it is less robust
and there is no compelling reason to use it;
it is merely provided for backward compatibility
and it may be removed in future versions.

If the detection mechanism is to be used,
it is mandatory to correctly specify
the filename of the main file as the argument of |\childdocmain|:
%
\begin{center}
\begin{tabular}{l}
|\input{childdoc.def}|\\
|\childdocmain{|\textit{main}|}|\\
\end{tabular}
\end{center}
%
If |\jobname| does not match the argument \textit{main} of |\childdocmain|,
it is assumed that |\jobname| points to the child file to be compiled.
When using |\childdocmain| with the main file specified as argument,
it suffices to start a child file
with just |\input{|\textit{main}|}|
without loading of the package and using |\childdocof|.
If instead all processing is done
with the appropriate \textsf{childdoc} directives,
the argument of \textit{main} of |\childdocmain| can be empty.

An alternative version of the command line processing described
in \secref{sec:commandline} using the detection mechanism reads:
%
\begin{center}
|... -jobname "|\textit{target}|" "|[\textit{flags}]%
[|\def\jobname{|\textit{dest}|}|]|\input{|\textit{main}|}"|
\end{center}

%%%%%%%%%%%%%%%%%%%%%%%%%%%%%%%%%%%%%%%%%%%%%%%%%%%%%%%%%%%%%%%%%%%%%%%%%%%%%%%%
\subsection{Manual Code}
\label{sec:manual}

In case one cannot be certain whether the definitions file |childdoc.def|
is installed on the target \TeX{} distribution
and one prefers not to ship it,
it is conceivable to paste a few relevant commands into the sources.

To that end, drop all statements |\input{childdoc.def}|
and perform the replacements as outlined below.
Instead of |\childdocmain{|\textit{main}|}| add the following code
to the top of the main file:
%
\begin{center}
\begin{tabular}{l}
|\||ifdefined\childdocname\endinput\||fi\newif\ifchilddoc|\\
|\edef\childdocname{\scantokens\expandafter{\jobname\noexpand}}|\\
|\def\childdocmain{|\textit{main}|}\||ifx\childdocmain\childdocname\||else|\\
|\childdoctrue\includeonly{\childdocname}\let\jobname\childdocmain\||fi|\\
\end{tabular}
\end{center}
%
Instead of |\childdocof{|\textit{main}|}| just include the main file
at the top of each child file:
%
\begin{center}
|\input{|\textit{main}|}|
\end{center}
%
A simple redirection |\childdocforward{|\textit{dest}|}| is achieved by:
%
\begin{center}
|\def\jobname{|\textit{dest}|}\input{\jobname}|
\end{center}
%
The redirection with prefix
|\childdocforwardprefix[|\textit{prefix}|]{|\textit{dest}|}|
is accomplished by:
%
\begin{center}
\begin{tabular}{l}
|{\edef\jobname{\scantokens\expandafter{\jobname\noexpand}}|\\
|\def\redirectjob |\textit{prefix}|#1~~~{\gdef\jobname{|\textit{dest}|#1}}|\\
|\expandafter\redirectjob\jobname~~~}\input{\jobname}|
\end{tabular}
\end{center}

In an alternative approach,
child documents can be compiled by a specific command line
without additional code or specific definitions:
%
\begin{center}
|... -jobname "|\textit{target}|" "|[\textit{flags}]%
|\includeonly{|\textit{dest}|}\input{|\textit{main}|}"|
\end{center}
%

%%%%%%%%%%%%%%%%%%%%%%%%%%%%%%%%%%%%%%%%%%%%%%%%%%%%%%%%%%%%%%%%%%%%%%%%%%%%%%%%
%%%%%%%%%%%%%%%%%%%%%%%%%%%%%%%%%%%%%%%%%%%%%%%%%%%%%%%%%%%%%%%%%%%%%%%%%%%%%%%%
\section{Information}

%%%%%%%%%%%%%%%%%%%%%%%%%%%%%%%%%%%%%%%%%%%%%%%%%%%%%%%%%%%%%%%%%%%%%%%%%%%%%%%%
\subsection{Copyright}

Copyright \copyright{} 2017--2018 Niklas Beisert

This work may be distributed and/or modified under the
conditions of the \LaTeX{} Project Public License, either version 1.3
of this license or (at your option) any later version.
The latest version of this license is in
  \url{http://www.latex-project.org/lppl.txt}
and version 1.3 or later is part of all distributions of \LaTeX{}
version 2005/12/01 or later.

This work has the LPPL maintenance status `maintained'.

The Current Maintainer of this work is Niklas Beisert.

This work consists of the files |README.txt|, |childdoc.ins| and |childdoc.dtx|
as well as the derived files |childdoc.def|, |cdocsamp.tex|
with |cdocsch1.tex|, |cdocsch2.tex|, |cdocspt3.tex|, |cdocspt4.tex|,
|cdocsdrf.tex|, |cdocsfn1.tex|, |cdocsfn2.tex|
as well as |childdoc.pdf|.

%%%%%%%%%%%%%%%%%%%%%%%%%%%%%%%%%%%%%%%%%%%%%%%%%%%%%%%%%%%%%%%%%%%%%%%%%%%%%%%%
\subsection{Files and Installation}

The package consists of the files:
%
\begin{center}
\begin{tabular}{ll}
    |README.txt|   & readme file \\
    |childdoc.ins| & installation file \\
    |childdoc.dtx| & source file \\
    |childdoc.def| & definition file \\
    |cdocsamp.tex| & sample main file \\
    |cdocsch1.tex| & sample include file \\
    |cdocsch2.tex| & sample include file \\
    |cdocspt3.tex| & sample part file \\
    |cdocspt4.tex| & sample part file \\
    |cdocsdrf.tex| & sample redirection file \\
    |cdocsfn1.tex| & sample redirection file \\
    |cdocsfn2.tex| & sample redirection file \\
    |childdoc.pdf| & manual
\end{tabular}
\end{center}
%
The distribution consists of the files
|README.txt|, |childdoc.ins| and |childdoc.dtx|.
%
\begin{itemize}
\item
Run (pdf)\LaTeX{} on |childdoc.dtx|
to compile the manual |childdoc.pdf| (this file).
\item
Run \LaTeX{} on |childdoc.ins| to create the definitions file |childdoc.def|
and the sample |cdocsamp.tex| with include files
|cdocsch1.tex|, |cdocsch2.tex|, |cdocspt3.tex|, |cdocspt4.tex|,
|cdocsdrf.tex|, |cdocsfn1.tex|, |cdocsfn2.tex|.
Then copy the file |childdoc.def| to an appropriate directory of your \LaTeX{}
distribution, e.g.\ \textit{texmf-root}|/tex/latex/childdoc|.
\end{itemize}

%%%%%%%%%%%%%%%%%%%%%%%%%%%%%%%%%%%%%%%%%%%%%%%%%%%%%%%%%%%%%%%%%%%%%%%%%%%%%%%%
\subsection{Related CTAN Packages}

There are several other packages which offer a similar functionality:
%
\begin{itemize}
\item
The packages
\href{http://ctan.org/pkg/docmute}{\textsf{docmute}},
\href{http://ctan.org/pkg/includex}{\textsf{includex}} and
\href{http://ctan.org/pkg/standalone}{\textsf{standalone}}
provide commands to include only the document body of
a child file thus allowing both files to be compiled individually.
\item
The packages \href{http://ctan.org/pkg/subdocs}{\textsf{subdocs}}
and \href{http://ctan.org/pkg/subfiles}{\textsf{subfiles}}
provide structures in which the main and child documents can be
encapsulated and allowing them to be compiled individually.
The inclusion mechanism is different from the conventional |\include|.
\item
The package \href{http://ctan.org/pkg/combine}{\textsf{combine}}
is an elaborate solution to combine several documents into one.
\end{itemize}
%
See also the CTAN topic \href{http://ctan.org/topic/subdocs}{\textsf{subdocs}}
for further related packages.
The present package differs from the above solutions in that
a document structure constructed with the conventional |\include| mechanism
just needs two extra commands at the top of every file
such that all constituent files can be compiled individually.

%%%%%%%%%%%%%%%%%%%%%%%%%%%%%%%%%%%%%%%%%%%%%%%%%%%%%%%%%%%%%%%%%%%%%%%%%%%%%%%%
%\subsection{Feature Suggestions}
%
%The following is a list of features which may be useful for future
%versions of this package:
%%
%\begin{itemize}
%\item
%\ldots
%\end{itemize}

%%%%%%%%%%%%%%%%%%%%%%%%%%%%%%%%%%%%%%%%%%%%%%%%%%%%%%%%%%%%%%%%%%%%%%%%%%%%%%%%
\subsection{Revision History}

%%%%%%%%%%%%%%%%%%%%%%%%%%%%%%%%%%%%%%%%
\paragraph{v2.0:} 2018/12/30

\begin{itemize}
\item
immediate forward processing
\item
added |\childdocby| mechanism
\item
manual restructured
\end{itemize}

%%%%%%%%%%%%%%%%%%%%%%%%%%%%%%%%%%%%%%%%
\paragraph{v1.6:} 2018/01/17

\begin{itemize}
\item
application for development of include files
\item
corrections to manual
\end{itemize}

%%%%%%%%%%%%%%%%%%%%%%%%%%%%%%%%%%%%%%%%
\paragraph{v1.5:} 2017/05/21

\begin{itemize}
\item
more complete structuring introduced
\item
|\childdocof| introduced
\item
|\childdoc| renamed to |\childdocmain|
\item
|\childredirect| renamed to |\childdocforward| and |\childdocforwardprefix|
and functionality expanded
\end{itemize}

%%%%%%%%%%%%%%%%%%%%%%%%%%%%%%%%%%%%%%%%
\paragraph{v1.0:} 2017/04/27

\begin{itemize}
\item
manual and install package
\item
first version published on CTAN
\end{itemize}

%%%%%%%%%%%%%%%%%%%%%%%%%%%%%%%%%%%%%%%%
\paragraph{v0.6:} 2017/04/26

\begin{itemize}
\item
redirection mechanism added
\end{itemize}

%%%%%%%%%%%%%%%%%%%%%%%%%%%%%%%%%%%%%%%%
\paragraph{v0.5:} 2017/04/26

\begin{itemize}
\item
functionality in definition file
\end{itemize}


%%%%%%%%%%%%%%%%%%%%%%%%%%%%%%%%%%%%%%%%%%%%%%%%%%%%%%%%%%%%%%%%%%%%%%%%%%%%%%%%
%%%%%%%%%%%%%%%%%%%%%%%%%%%%%%%%%%%%%%%%%%%%%%%%%%%%%%%%%%%%%%%%%%%%%%%%%%%%%%%%
%%%%%%%%%%%%%%%%%%%%%%%%%%%%%%%%%%%%%%%%%%%%%%%%%%%%%%%%%%%%%%%%%%%%%%%%%%%%%%%%
\appendix

\settowidth\MacroIndent{\rmfamily\scriptsize 000\ }

 \DocInput{childdoc.dtx}

\end{document}
%</driver>
% \fi
%
% %%%%%%%%%%%%%%%%%%%%%%%%%%%%%%%%%%%%%%%%%%%%%%%%%%%%%%%%%%%%%%%%%%%%%%%%%%%%%%
% %%%%%%%%%%%%%%%%%%%%%%%%%%%%%%%%%%%%%%%%%%%%%%%%%%%%%%%%%%%%%%%%%%%%%%%%%%%%%%
% \section{Sample}
%\iffalse
%<*samplemain>
%\fi
%
% The following presents a sample document
% with two chapters, two parts, a title page,
% a compile flag as well as three forwarding files to set the flag.
% It consists of eight |.tex| files:
% \begin{center}
% \begin{tabular}{ll}
% |cdocsamp.tex|&main file\\
% |cdocsch1.tex|&include file for chapter 1\\
% |cdocsch2.tex|&include file for chapter 2\\
% |cdocspt3.tex|&include file for part 3\\
% |cdocspt4.tex|&include file for part 4\\
% |cdocsdrf.tex|&forwarding file for main file in draft mode\\
% |cdocsfi1.tex|&forwarding file for final version of chapter 1\\
% |cdocsfi2.tex|&forwarding file for final version of chapter 2\\
% \end{tabular}
% \end{center}
% Each of the eight files can be compiled directly by the \LaTeX{} compiler.
%
% %%%%%%%%%%%%%%%%%%%%%%%%%%%%%%%%%%%%%%
% \paragraph{Main File.}
%
% The main file is called |cdocsamp.tex|.
%
% Load the \textsf{childdoc} definitions and
% declare the filename for the main document:
%    \begin{macrocode}
\input{childdoc.def}
\childdocmain{}
%    \end{macrocode}

% Optional override for |\version| flag:
%    \begin{macrocode}
%%\ifchilddoc\else\providecommand{\version}{draft}\fi
%    \end{macrocode}

% Define the default values for the |\version| flag
% (|final| for the main file and |draft| for childs):
%    \begin{macrocode}
\ifchilddoc
\providecommand{\version}{draft}
\else
\providecommand{\version}{final}
\fi
%    \end{macrocode}

% Load the standard document class:
%    \begin{macrocode}
\documentclass[12pt]{article}
%    \end{macrocode}

% Start the document body:
%    \begin{macrocode}
\begin{document}
%    \end{macrocode}

% Declare a title page.
% Print title, part of document being processed and version flag:
%    \begin{macrocode}
\addtocounter{page}{-1}
\begin{center}
{\LARGE\bfseries{}childdoc example\par}
\vspace{1cm}
\ifchilddoc
\ifchilddocmanual part\else chapter\fi:
`\childdocname' of `\childdocjob'\par
\else
main document: `\childdocjob'\par
\fi
version: \version\par
\end{center}
\newpage
%    \end{macrocode}

% Manually include selected file,
% otherwise process as usual:
%    \begin{macrocode}
\ifchilddocmanual
\section*{part `\childdocname'}
\input{\childdocname}
\else
%    \end{macrocode}

% Include the two chapters:
%    \begin{macrocode}
\include{cdocsch1}
\include{cdocsch2}
%    \end{macrocode}

% Include the two parts unless only chapters should be displayed:
%    \begin{macrocode}
\ifchilddoc\else
\section{part three}
\input{cdocspt3}
\section{part four}
\input{cdocspt4}
\fi
%    \end{macrocode}

% Process as usual until here:
%    \begin{macrocode}
\fi
%    \end{macrocode}

% End of document body:
%    \begin{macrocode}
\end{document}
%    \end{macrocode}
%\iffalse
%</samplemain>
%\fi
%
% %%%%%%%%%%%%%%%%%%%%%%%%%%%%%%%%%%%%%%
% \paragraph{Chapter Include Files.}
%
% The include files are called |cdocsch1.tex| and |cdocsch2.tex|.
%
%\iffalse
%<*samplechap1|samplechap2>
%\fi

% Optional override for |\version| flag:
%    \begin{macrocode}
%%\providecommand{\version}{final}
%    \end{macrocode}

% Include the main document:
%    \begin{macrocode}
\input{childdoc.def}
\childdocof{cdocsamp}
%    \end{macrocode}

%\iffalse
%</samplechap1|samplechap2>
%\fi
%
%\iffalse
%<*samplechap1>
%\fi
% Some text for chapter 1:
%    \begin{macrocode}
\section{one}
some text in chapter one
%    \end{macrocode}

%\iffalse
%</samplechap1>
%\fi
% Some text for chapter 2:
%\iffalse
%<*samplechap2>
%\fi
%    \begin{macrocode}
\section{two}
more text in chapter two
%    \end{macrocode}

%\iffalse
%</samplechap2>
%\fi
%
% %%%%%%%%%%%%%%%%%%%%%%%%%%%%%%%%%%%%%%
% \paragraph{Part Include Files.}
%
% The include files are called |cdocspt3.tex| and |cdocspt4.tex|.
%
%\iffalse
%<*samplepart3|samplepart4>
%\fi

% Optional override for |\version| flag:
%    \begin{macrocode}
%%\providecommand{\version}{final}
%    \end{macrocode}

% Include the main document:
%    \begin{macrocode}
\input{childdoc.def}
\childdocby{cdocsamp}
%    \end{macrocode}

%\iffalse
%</samplepart3|samplepart4>
%\fi
%
%\iffalse
%<*samplepart3>
%\fi
% Some text for part 3:
%    \begin{macrocode}
some text in part three
%    \end{macrocode}

%\iffalse
%</samplepart3>
%\fi
% Some text for part 4:
%\iffalse
%<*samplepart4>
%\fi
%    \begin{macrocode}
more text in part four
%    \end{macrocode}

%\iffalse
%</samplepart4>
%\fi
%
% %%%%%%%%%%%%%%%%%%%%%%%%%%%%%%%%%%%%%%
% \paragraph{Forwarding for a Complete Draft.}
%
% The following forwarding file |cdocsdrf.tex|
% compiles the main document in draft mode:
%\iffalse
%<*sampledraft>
%\fi
%    \begin{macrocode}
\def\version{draft}
\input{childdoc.def}
\childdocforward{cdocsamp}
%    \end{macrocode}

%\iffalse
%</sampledraft>
%\fi
%
% %%%%%%%%%%%%%%%%%%%%%%%%%%%%%%%%%%%%%%
% \paragraph{Forwarding for Final Version of the Chapters.}
%
% The following forwarding files |cdocsfn1.tex| and |cdocsfn2.tex|
% (with identical content)
% compile the final versions of the child documents
% |cdocsch1.tex| and |cdocsch2.tex|, respectively:
%\iffalse
%<*samplefinal>
%\fi
%    \begin{macrocode}
\def\version{final}
\input{childdoc.def}
\childdocforwardprefix[cdocsamp]{cdocsfn}{cdocsch}
%    \end{macrocode}

%\iffalse
%</samplefinal>
%\fi
%
% %%%%%%%%%%%%%%%%%%%%%%%%%%%%%%%%%%%%%%
% \paragraph{Command Line Processing.}
%
% The following three command lines generate the output files
% |cdocscld|, |cdocscl1| and |cdocscl2|
% which should be identical to
% |cdocsdrf|, |cdocsch1| and |cdocsfn2|, respectively:
% \begin{center}
% \begin{tabular}{l}
% |latex -jobname cdocscld \|\\
% |  "\def\version{draft}\input{childdoc.def}\childdocforward{cdocsamp}"|\\
% |latex -jobname cdocscl1 \|\\
% |  "\input{childdoc.def}\childdocforward[cdocsamp]{cdocsch1}"|\\
% |latex -jobname cdocscl2 \|\\
% |  "\def\version{final}\input{childdoc.def}\childdocforward{cdocsch2}"|
% \end{tabular}
% \end{center}
% Note that the trailing backslash on each first line
% merely continues the input to the second line
% (for convenient cut ant paste).
% Furthermore, the command |latex| can be replaced by any
% of its alternative versions such as |pdflatex|.
%
% %%%%%%%%%%%%%%%%%%%%%%%%%%%%%%%%%%%%%%%%%%%%%%%%%%%%%%%%%%%%%%%%%%%%%%%%%%%%%%
% %%%%%%%%%%%%%%%%%%%%%%%%%%%%%%%%%%%%%%%%%%%%%%%%%%%%%%%%%%%%%%%%%%%%%%%%%%%%%%
% \section{Implementation}
%\iffalse
%<*package>
%\fi
%
% This section describes the definitions file |childdoc.def|.

% The definitions cannot be loaded using |\usepackage| or |\RequirePackage|
% which has a mechanism to prevent loading a style file more than once.
% When loading the definitions by means of |\input|
% multiple instances have to be prevented manually:
%\iffalse
%This code needs to be before the `\ProvidesFile' directive
%which is defined at the beginning of this file.
%Therefore it is also placed there and commented out here.
%</package>
%<*discard>
%\fi
%    \begin{macrocode}
\ifdefined\childdocmain\endinput\fi
%    \end{macrocode}
%\iffalse
%</discard>
%<*package>
%\fi
%
% \macro{\ifchilddoc}
% \macro{\ifchilddocmanual}
% The conditional |\ifchilddoc| tells whether a
% child (true) or main (false) document is being compiled.
% The conditional |\ifchilddocmanual| tells whether
% the |\includeonly| mechanism is used (false) or
% the selection of child files must be performed manually (true).
% The definitions initialise to false:
%    \begin{macrocode}
\newif\ifchilddoc
\newif\ifchilddocmanual
%    \end{macrocode}

% \macro{\childdocname}
% \macro{\childdocjob}
% The macro |\childdocname| stores the name of the main document
% to be compiled. The macro |\childdocjob| stores the name of
% the document on which the \LaTeX{} compiler was originally invoked.
% The content of |\jobname| cannot be compared
% to filenames specified in the source due to different catcodes.
% The following code rescans |\jobname|, stores the result
% in |\childdocname| and saves a copy in |\childdocjob|:
%    \begin{macrocode}
\edef\childdocname{\scantokens\expandafter{\jobname\noexpand}}
\let\childdocjob\childdocname
%    \end{macrocode}

% \macro{\childdocdisable}
% The macro |\childdocdisable| prevents the main file
% from being processed more than once.
% At this stage, the main document command |\childdocmain|
% is assumed to be called once again where it should do nothing.
% Any subsequent call to it should prevent
% a secondary processing of the main document
% It overwrites the forwarding commands
% |\childdocof| and |\childdocforward|
% with empty macros to prevent further inclusions of the main document:
%    \begin{macrocode}
\newcommand{\childdocdisable}
{
  \renewcommand{\childdocmain}[1]{\renewcommand{\childdocmain}[1]{\endinput}}
  \renewcommand{\childdocof}[1]{}
  \renewcommand{\childdocby}[2][]{}
  \renewcommand{\childdocforward}[2][]{}
  \renewcommand{\childdocdisable}{}
}
%    \end{macrocode}

% \macro{\childdocmain}
% The macro |\childdocmain| is to be called at the top of the main file
% with nothing or the main filename (without extension) as argument.
% First, it breaks loops.
% If the argument is not empty and does not match |\childdocname|
% (which is set by the first inclusion of |childdoc.def|),
% |\ifchilddoc| is set to true, |\includeonly| is applied to the child file
% and |\jobname| is set to the main file
% (for proper handling of |.aux| files):
%    \begin{macrocode}
\newcommand{\childdocmain}[1]
{
  \childdocdisable\childdocmain{}
  \if?#1?\else
    \begingroup
      \def\childdoctmp{#1}
      \ifx\childdoctmp\childdocname
        \def\childdoctmp{}
      \else
        \def\childdoctmp
        {
          \childdoctrue
          \includeonly{\childdocname}
          \def\childdocjob{#1}
          \def\jobname{#1}
        }
      \fi
      \expandafter
    \endgroup
    \childdoctmp
  \fi
}
%    \end{macrocode}

% \macro{\childdocof}
% The command |\childdocof| redirects
% compilation to the main file |#1|.
%    \begin{macrocode}
\newcommand{\childdocof}[1]
{
  \childdocdisable
  \childdoctrue
  \includeonly{\childdocname}
  \def\jobname{#1}
  \def\childdocjob{#1}
  \input{#1}
}
%    \end{macrocode}

% \macro{\childdocby}
% The command |\childdocby| ....
%    \begin{macrocode}
\newcommand{\childdocby}[2][]
{
  \childdocdisable
  \childdoctrue
  \childdocmanualtrue
  \if?#1?\else
    \def\jobname{#2}
  \fi
  \def\childdocjob{#2}
  \input{#2}
  \endinput
}
%    \end{macrocode}

% \macro{\childdocforward}
% The command |\childdocforward| redirects
% compilation to the main file or
% (if the optional argument is given) a child file.
% Parameters are set as if the main file
% or a child file starting with |\childdocof| was compiled.
% Then compilation is handed over to the main file:
%    \begin{macrocode}
\newcommand{\childdocforward}[2][]
{
  \begingroup
    \if?#1?
      \def\childdoctmp
      {
        \def\childdocname{#2}
        \def\childdocjob{#2}
        \def\jobname{#2}
        \input{#2}
        \endinput
      }
    \else
      \def\childdoctmp
      {
        \childdocdisable
        \def\childdocname{#2}
        \childdoctrue
        \includeonly{#2}
        \def\childdocjob{#1}
        \def\jobname{#1}
        \input{#1}
        \endinput
      }
    \fi
    \expandafter
  \endgroup
  \childdoctmp
}
%    \end{macrocode}

% \macro{\childdocforwardprefix}
% The command |\childdocforwardprefix| redirects
% compilation to the main or a child file by means of a pattern.
% The prefix |#1| in the current filename is replaced by |#2|
% and the suffix of the current filename is kept
% (it is assumed that the filename does not contain the substring `|~~~|'
% which is used as a delimiter).
% Compilation is handed over to the new file by |\childdocforward|:
%    \begin{macrocode}
\newcommand{\childdocforwardprefix}[3][]
{
  \begingroup
    \def\childdocextract #2##1~~~{\def\childdoctmp{\childdocforward[#1]{#3##1}}}
    \expandafter\childdocextract\childdocname~~~
    \expandafter
  \endgroup
  \childdoctmp
}
%    \end{macrocode}

% \macro{\childdoc}
% The deprecated macro |\childdoc| is a legacy version of |\childdocmain|:
%    \begin{macrocode}
\newcommand{\childdoc}{\childdocmain}
%    \end{macrocode}

% \macro{\childdocredirect}
% The deprecated macro |\childdocredirect| is a legacy version
% of |\childdocforward| and |\childdocforwardprefix|:
%    \begin{macrocode}
\newcommand{\childdocredirect}[2][]
{
  \begingroup
    \if?#1?
      \def\childdoctmp{\childdocforward{#2}}
    \else
      \def\childdoctmp{\childdocforwardprefix{#1}{#2}}
    \fi
    \expandafter
  \endgroup
  \childdoctmp
}
%    \end{macrocode}

%\iffalse
%</package>
%\fi
%
\endinput
|\\
|\childdocby{|\textit{main}|}|\\
\end{tabular}
\end{center}
%
The directive |\childdocby| is similar to |\childdocof|
described in \secref{sec:include},
but the subsequent selection of content must be done manually.
To that end, both |\ifchilddoc| and |\ifchilddocmanual|
will be true upon processing of a part,
and the name of the part is stored in |\childdocname|.
Note that |\jobname| will be set to the filename of the current part
so that each part receives an individual |.aux| file
that does not interfere with the |.aux| file(s) of the main document.
This behaviour can be altered by the alternative form
|\childdocby[*]{|\textit{main}|}| (with a non-empty optional argument)
which uses the |.aux| file of the main document
by setting |\jobname| to \textit{main}.

%%%%%%%%%%%%%%%%%%%%%%%%%%%%%%%%%%%%%%%%%%%%%%%%%%%%%%%%%%%%%%%%%%%%%%%%%%%%%%%%
\subsection{Driver Development}
\label{sec:driver}

The \textsf{childdoc} mechanism can also be use for the development
of definition files such as \LaTeX{} styles or classes.
This case differs from the above setup with multiple parts
included by |\include| in that no |\includeonly| should be invoked.
This can be achieved by starting the include file
(before |\ProvidesPackage|) with:
%
\begin{center}
\begin{tabular}{l}
|% \iffalse
%
% childdoc.dtx Copyright (C) 2017-2018 Niklas Beisert
%
% This work may be distributed and/or modified under the
% conditions of the LaTeX Project Public License, either version 1.3
% of this license or (at your option) any later version.
% The latest version of this license is in
%   http://www.latex-project.org/lppl.txt
% and version 1.3 or later is part of all distributions of LaTeX
% version 2005/12/01 or later.
%
% This work has the LPPL maintenance status `maintained'.
%
% The Current Maintainer of this work is Niklas Beisert.
%
% This work consists of the files childdoc.dtx and childdoc.ins
% and the derived files childdoc.def and cdocsamp.tex with
% cdocsch1.tex, cdocsch2.tex, cdocsdrf.tex, cdocsfn1.tex, cdocsfn2.tex.
%
%<package>\ifdefined\childdocmain\endinput\fi
%<package>\ProvidesFile{childdoc.def}[2018/12/30 v2.0 child document driver]
%<samplemain>\ProvidesFile{cdocsamp.tex}[2018/12/30 v2.0 sample for childdoc]
%<*driver>
%\ProvidesFile{childdoc.drv}[2018/12/30 v2.0 childdoc reference manual file]
\PassOptionsToClass{10pt,a4paper}{article}
\documentclass{ltxdoc}

\usepackage[margin=35mm]{geometry}
\usepackage{hyperref}
\usepackage{hyperxmp}
\usepackage[usenames]{color}

\hypersetup{colorlinks=true}
\hypersetup{pdfstartview=FitH}
\hypersetup{pdfpagemode=UseNone}
\hypersetup{pdfsource={}}
\hypersetup{pdflang={en-UK}}
\hypersetup{pdfcopyright={Copyright 2017-2018 Niklas Beisert.
  This work may be distributed and/or modified under the
  conditions of the LaTeX Project Public License, either version 1.3
  of this license or (at your option) any later version.}}
\hypersetup{pdflicenseurl={http://www.latex-project.org/lppl.txt}}
\hypersetup{pdfcontactaddress={ETH Zurich, ITP, HIT K,
  Wolfgang-Pauli-Strasse 27}}
\hypersetup{pdfcontactpostcode={8093}}
\hypersetup{pdfcontactcity={Zurich}}
\hypersetup{pdfcontactcountry={Switzerland}}
\hypersetup{pdfcontactemail={nbeisert@itp.phys.ethz.ch}}
\hypersetup{pdfcontacturl={http://people.phys.ethz.ch/\xmptilde nbeisert/}}

\newcommand{\secref}[1]{\hyperref[#1]{section \ref*{#1}}}

\parskip1ex
\parindent0pt
\let\olditemize\itemize
\def\itemize{\olditemize\parskip0pt}

\begin{document}

\title{The \textsf{childdoc} Package}
\hypersetup{pdftitle={The childdoc Package}}
\author{Niklas Beisert\\[2ex]
  Institut f\"ur Theoretische Physik\\
  Eidgen\"ossische Technische Hochschule Z\"urich\\
  Wolfgang-Pauli-Strasse 27, 8093 Z\"urich, Switzerland\\[1ex]
  \href{mailto:nbeisert@itp.phys.ethz.ch}
  {\texttt{nbeisert@itp.phys.ethz.ch}}}
\hypersetup{pdfauthor={Niklas Beisert}}
\hypersetup{pdfsubject={Manual for the LaTeX2e Package childdoc}}
\date{30 December 2018, \textsf{v2.0}}
\maketitle

\begin{abstract}\noindent
\textsf{childdoc} is a \LaTeXe{} package
that enables the direct compilation
of document sections included by |\include|
to individual files.
\end{abstract}

\begingroup
\parskip0ex
\tableofcontents
\endgroup

%%%%%%%%%%%%%%%%%%%%%%%%%%%%%%%%%%%%%%%%%%%%%%%%%%%%%%%%%%%%%%%%%%%%%%%%%%%%%%%%
%%%%%%%%%%%%%%%%%%%%%%%%%%%%%%%%%%%%%%%%%%%%%%%%%%%%%%%%%%%%%%%%%%%%%%%%%%%%%%%%
\section{Introduction}

\LaTeX{} provides a mechanism to structure a large document (such as a book)
into a main file and several child files (containing the chapters)
using the |\include| command.
This mechanism is beneficial for documents
which span hundreds of pages in order to
make the source file(s) more manageable.
Moreover, compilation can be restricted to
selected child files by means of the |\includeonly| command.
The latter feature can be used to reduce the compilation time while editing
(this was significantly more useful in the earlier days of \LaTeX{})
or to generate a smaller document which is easier to navigate.
Another application of |\includeonly| is to generate
documents consisting of selected parts of the complete document.

However, there are a few drawbacks of the plain |\include| mechanism:
\begin{itemize}
\item
The child files cannot be compiled on their own,
they can only be compiled via the main file.
A naive editing environment
(such as a text editor with an option
to have the current file processed by \LaTeX)
may require one to switch to the main file before compiling;
attempting to compile the child file produces errors.
\item
The main file must be modified (each time)
to adjust the |\includeonly| command
to the present needs. This easily leaves the main file in a messy state.
\item
The generated document will always carry the filename
of the main document. This is inconvenient if
several child files are to be compiled and
to be kept for distribution.
\end{itemize}

The present package provides a simple interface
to make child files individually compilable by \LaTeX{}.
Compiling a child file then has the same effect as compiling
the main file with an |\includeonly| command
to select the appropriate child.
Moreover the generated document will carry the name of the child
rather than the main file.
This resolves all three above issues.

This feature is meant to make the editing of books,
thesis documents and lecture notes somewhat more convenient.
However, the package can also be used efficiently for
composing a series of documents (such as exercise sheets)
which are typically distributed individually.
It then assists the author in generating the individual documents
(potentially in different versions)
as well as a document containing the collected series.
Another application is in developing style files
or other kinds of included material
where compilation of the style file could redirect
to a sample or test file.

%%%%%%%%%%%%%%%%%%%%%%%%%%%%%%%%%%%%%%%%%%%%%%%%%%%%%%%%%%%%%%%%%%%%%%%%%%%%%%%%
%%%%%%%%%%%%%%%%%%%%%%%%%%%%%%%%%%%%%%%%%%%%%%%%%%%%%%%%%%%%%%%%%%%%%%%%%%%%%%%%
\section{Usage}

First of all, the package \textsf{childdoc} is \emph{not} a standard
\LaTeXe{} |.sty| style file! Therefore it needs to be invoked in
a non-standard way.

%%%%%%%%%%%%%%%%%%%%%%%%%%%%%%%%%%%%%%%%%%%%%%%%%%%%%%%%%%%%%%%%%%%%%%%%%%%%%%%%
\subsection{Included Files}
\label{sec:include}

%%%%%%%%%%%%%%%%%%%%%%%%%%%%%%%%%%%%%%%%
\DescribeMacro{\childdocmain}
To use the package, add the commands
\begin{center}
\begin{tabular}{l}
|\input{childdoc.def}|\\
|\childdocmain{}|\\
\end{tabular}
\end{center}
at the very top of the main \LaTeX{} file,
in particular \emph{before} the |\documentclass| statement!
The argument of |\childdocmain| should be left empty
(but it must be present).

%%%%%%%%%%%%%%%%%%%%%%%%%%%%%%%%%%%%%%%%
\DescribeMacro{\childdocof}
Furthermore, add the commands
\begin{center}
\begin{tabular}{l}
|\input{childdoc.def}|\\
|\childdocof{|\textit{main}|}|\\
\end{tabular}
\end{center}
at the top of every child file \textit{child}
which is included by |\include{|\textit{child}|}|
from within the main file
(or at least for those files to be compiled individually).
The argument \textit{main} must be the filename of the main file.

There are a couple of
considerations in setting up the main and child documents:

%%%%%%%%%%%%%%%%%%%%%%%%%%%%%%%%%%%%%%%%
\paragraph{Restrictions.}

Please note the following restrictions:
\begin{itemize}
\item
|\childdocmain| must be called with one argument \textit{main}
to ensure compatibility with earlier version of the package.
It must either be empty (|\childdocmain{}|)
or precisely match the filename of the main file in which it is specified.
See \secref{sec:detection} for further information.
\item
The filename \textit{main} must be specified without the |.tex| extension.
\item
The filename \textit{main} is case sensitive
(even in case-insensitive file systems)
due to internal string comparison.
\item
The argument \textit{main} should be fully expanded, it cannot be a macro.
\item
Subdirectories and special characters should be avoided in filenames.
\item
The command |\childdocmain{|\textit{main}|}| must be followed by a whitespace.
It should not be followed immediately by another command
or by a comment mark `|%|'.
This is because the \TeX{} parser reads the token immediately following
the argument of |\childdocmain| and puts it
at the beginning of every child section;
however, a white\-space is ignored.
\end{itemize}

%%%%%%%%%%%%%%%%%%%%%%%%%%%%%%%%%%%%%%%%
\paragraph{Content of Main File.}

It is advisable to place all content in the child files included by |\include|.
Any output contained in the main file will appear in all child documents
unless suppressed manually;
it cannot be suppressed automatically by the |\includeonly| directive
and thus should normally be avoided.
A method to include some content in the main file
by means of conditional processing is described in \secref{sec:conditional}.

%%%%%%%%%%%%%%%%%%%%%%%%%%%%%%%%%%%%%%%%
\paragraph{Page Numbering.}

When only a part of the document is compiled,
the appropriate numbering of pages
(as well as other status parameters)
is determined from the |.aux| files.
The latter contain information from previous passes.
However this information needs to propagate through
all intermediate child documents.
Therefore the page numbering in child documents may well
be inconsistent until the complete document is compiled at least once.

A useful (if unconventional) way to always ensure a consistent
page numbering is to restart the numbering in each child document
and denote the pages by `\textit{child}|.|\textit{page}'
where \textit{child} represents the chapter/section number of the child file.
This can be achieved by the command
|\numberwithin{page}{|\textit{child}|}|
of the \textsf{amsmath} package
where \textit{child} can be |chapter| or |section|
depending on the chosen structuring.
Alternatively, one can modify the macro |\thepage| appropriately
and reset the counter |page| at the start of each child file.

%%%%%%%%%%%%%%%%%%%%%%%%%%%%%%%%%%%%%%%%%%%%%%%%%%%%%%%%%%%%%%%%%%%%%%%%%%%%%%%%
\subsection{Conditional Processing}
\label{sec:conditional}

The package provides a mechanism to compile different versions
of a document. To customise the versions further some conditional processing
can come in handy to distinguish which version is being compiled.
The package provides two macros to describe the compilation context:

%%%%%%%%%%%%%%%%%%%%%%%%%%%%%%%%%%%%%%%%
\DescribeMacro{\ifchilddoc}
The conditional |\ifchilddoc| distinguishes between the compilation of
child documents and the main document:
%
\begin{center}
|\ifchilddoc |\textit{child-code}| |[|\||else |\textit{main-code}]| \||fi|
\end{center}

%%%%%%%%%%%%%%%%%%%%%%%%%%%%%%%%%%%%%%%%
\DescribeMacro{\childdocname}
\DescribeMacro{\childdocjob}
The macro |\childdocname| contains the filename (without extension)
of the main or child file being processed.
Note that |\childdocjob| will always contain the name of the main file.

%%%%%%%%%%%%%%%%%%%%%%%%%%%%%%%%%%%%%%%%
\paragraph{Title Page.}

Conditional processing can be used to include a title or banner page
in the main document when proper precautions are taken.
Importantly, the code in the main file should ensure that the page counter
(as well as other status parameters which are stored in the |.aux| files)
takes the same value after the conditional processing.
Otherwise the page numbers may take divergent values
depending on which part is compiled.

For example, a title page could be declared by:
%
\begin{center}
\begin{tabular}{l}
|\ifchilddoc\||else|\\
|\addtocounter{page}{-1}|\\
\textit{code for title page}\\
|\newpage|\\
|\||fi|
\end{tabular}
\end{center}
%
A banner page for the child documents can be generated by:
%
\begin{center}
\begin{tabular}{l}
|\ifchilddoc|\\
|\addtocounter{page}{-1}|\\
\textit{code for banner page}\\
|\newpage|\\
|\||fi|
\end{tabular}
\end{center}
%
Here one could write a message such as:
\begin{center}
|This is the part \childdocname{} of \childdocjob{}.|
\end{center}

%%%%%%%%%%%%%%%%%%%%%%%%%%%%%%%%%%%%%%%%%%%%%%%%%%%%%%%%%%%%%%%%%%%%%%%%%%%%%%%%
\subsection{Flags}
\label{sec:flags}

The package makes it easy to generate different versions
of the main or child documents.
To this end compilation flags can be defined
and assigned different default values.
They will be particularly useful in conjunction
with the forwarding mechanism described in \secref{sec:forward}.

For example, it may be useful to have a flag |\version|
which can be set to |draft| or |final|.
The document source will contain some conditional code
depending on the value of |\version|.
Suppose further, the flag should default to |final| for the main file
and to |draft| for child files
which is a natural assignment for editing the document.
This is achieved by placing the following code
in the preamble of the main document
(below the |\childdocmain| directive):
%
\begin{center}
\begin{tabular}{l}
|\ifchilddoc|\\
|\providecommand{\version}{draft}|\\
|\||else|\\
|\providecommand{\version}{final}|\\
|\||fi|
\end{tabular}
\end{center}
%
The definition by |\providecommand| makes sure
that previous definitions are not overwritten.
Further statements |\providecommand{\version}{...}|
can thus be added before the above code to override it.

For the main file, one might add a line
(between |\childdocmain| and the above block)
%
\begin{center}
|%\ifchilddoc\||else\providecommand{\version}{draft}\||fi|
\end{center}
%
which can be uncommented to produce a draft version.
Likewise one can add a line to the very top of a child file
(above the |\childdocof{|\textit{main}|}| directive)
%
\begin{center}
|%\providecommand{\version}{final}|
\end{center}
%
which can be uncommented to produce the final version of this child document.

%%%%%%%%%%%%%%%%%%%%%%%%%%%%%%%%%%%%%%%%%%%%%%%%%%%%%%%%%%%%%%%%%%%%%%%%%%%%%%%%
\subsection{Forwarding}
\label{sec:forward}

Different versions of the main or child documents
using compilation flags as described in \secref{sec:flags}
can be (permanently) stored in different files
for convenient compilation, viewing and distribution.
To this end, the package defines a command
to pass on compilation to a different file:

%%%%%%%%%%%%%%%%%%%%%%%%%%%%%%%%%%%%%%%%
\DescribeMacro{\childdocforward}
The command |\childdocforward| redirects processing to
another source file:
%
\begin{center}
\begin{tabular}{l}
|\input{childdoc.def}|\\
|\childdocforward[|\textit{main}|]{|\textit{dest}|}|\\
\end{tabular}
\end{center}
%
The argument \textit{dest} is the destination file
(without extension).
It should be the main file or one of the child files.
Note that further \textsf{childdoc} directives
such as |\childdocof| and |\childdocforward|
in the indicated file will be processed in this form.
The optional argument \textit{main}
passes on directly to the main file \textit{main}
while pretending to compile the child \textit{dest}.
This form behaves as if \textit{dest}
issues |\childdocof{|\textit{main}|}| right away,
and no further \textsf{childdoc} directives will be processed.

%%%%%%%%%%%%%%%%%%%%%%%%%%%%%%%%%%%%%%%%
\DescribeMacro{\...prefix}
In the alternative form |\childdocforwardprefix|,
%
\begin{center}
\begin{tabular}{l}
|\input{childdoc.def}|\\
|\childdocforwardprefix[|\textit{main}|]{|\textit{prefix}|}{|\textit{dest}|}|
\end{tabular}
\end{center}
%
the destination file is determined by a pattern
depending on the current file:
To make this work, the current file must be called
`{\textit{prefix}\hspace{0.2em}\textit{suffix}}'
with \textit{prefix} matching precisely the argument.
Processing is then passed on to the file
`{\textit{dest}\hspace{0.2em}\textit{suffix}}'.
Surely, the same effect is achieved by
directly specifying the
argument `{\textit{dest}\hspace{0.2em}\textit{suffix}}'
in the first form.
However, that requires to set up a different file
for each child. With the alternative form of the command
all these files can have exactly the same content
which simplifies setting them up and maintaining them.

For example, the following file |draft.tex|
with a compilation flag |\version| as described in \secref{sec:flags}
compiles the main document as a draft:
%
\begin{center}
\begin{tabular}{l}
|\def\version{draft}|\\
|\input{childdoc.def}|\\
|\childdocforward{|\textit{main}|}|
\end{tabular}
\end{center}
%
Likewise, the following files |final|\textit{nn}|.tex|
compile the final version of the child document
|child|\textit{nn}|.tex|:
%
\begin{center}
\begin{tabular}{l}
|\def\version{final}|\\
|\input{childdoc.def}|\\
|\childdocforwardprefix{final}{child}|
\end{tabular}
\end{center}
%

Note that when several versions of a main file and/or of each child file
are to be generated, it may be convenient to set up a |Makefile| or
shell script to automatise the process.

%%%%%%%%%%%%%%%%%%%%%%%%%%%%%%%%%%%%%%%%%%%%%%%%%%%%%%%%%%%%%%%%%%%%%%%%%%%%%%%%
\subsection{Command Line Processing}
\label{sec:commandline}

The effect of redirection files can also be achieved by invoking
the \LaTeX{} compiler with a more elaborate command line.
Most conveniently this should be done as part
of a shell script or a |Makefile|.

When using \textsf{childdoc} in the main file, the following
command lines effectively perform a redirection
(note that depending on the shell being used,
backslashes may have to be doubled: `|\|' $\to$ `|\\|'):
%
\begin{center}
|... -jobname "|\textit{target}|" |\\|"|[\textit{flags}]%
|\input{childdoc.def}\childdocforward[|\textit{main}|]{|\textit{dest}|}"|
\end{center}
%
Here \textit{target} is the name of the output file,
\textit{main} is the name of the main file
and \textit{dest} is the name of the main or child file to be processed
(all filenames without extensions).
The optional argument \textit{main} can be omitted
if \textit{main} matches \textit{dest}.
Optionally, compilation \textit{flags} can be defined via |\def| commands.
This command line makes the \TeX{} engine believe
it is compiling the file \textit{target}
whose content is specified as the latter parameter.
The provided code then forwards the processing to
\textit{main} or \textit{dest} as described in \secref{sec:forward}.

%%%%%%%%%%%%%%%%%%%%%%%%%%%%%%%%%%%%%%%%%%%%%%%%%%%%%%%%%%%%%%%%%%%%%%%%%%%%%%%%
\subsection{Include by Input}
\label{sec:input}

Including child documents by |\include| has some restrictions by design.
Most notably, the content of a child document always occupies
its own set of pages; pages cannot be shared between child documents.
Usually, this behaviour makes perfect sense
because each child document contain an essential part of the document.
However, in some situations it may be desirable to compose
a document from a collection of parts
without having mandatory page breaks between then.
For this case, the package
provides a mechanism to include parts
by |\input| which can also be processed individually.
However, by construction this mechanism
requires manual handling of the content to be output.

%%%%%%%%%%%%%%%%%%%%%%%%%%%%%%%%%%%%%%%%
\DescribeMacro{\ifchilddocmanual}
The main file should be prepared as usual, see \secref{sec:include}.
However, the document body must make a distinction
between processing of an individual part and of the main document, e.g.:
%
\begin{center}
\begin{tabular}{l}
|\ifchilddocmanual|\\
|\input{\childdocname}|\\
|\||else|\\
\textit{document body with }|\input{|\textit{part}|}|\\
|\||fi|
\end{tabular}
\end{center}
%
The conditional |\ifchilddocmanual| is true whenever
a part to be included by |\input| is being compiled,
and the name of the part is stored in |\childdocname|.

%%%%%%%%%%%%%%%%%%%%%%%%%%%%%%%%%%%%%%%%
\DescribeMacro{\childdocby}
Each part to be included by |\input| should start with:
%
\begin{center}
\begin{tabular}{l}
|\input{childdoc.def}|\\
|\childdocby{|\textit{main}|}|\\
\end{tabular}
\end{center}
%
The directive |\childdocby| is similar to |\childdocof|
described in \secref{sec:include},
but the subsequent selection of content must be done manually.
To that end, both |\ifchilddoc| and |\ifchilddocmanual|
will be true upon processing of a part,
and the name of the part is stored in |\childdocname|.
Note that |\jobname| will be set to the filename of the current part
so that each part receives an individual |.aux| file
that does not interfere with the |.aux| file(s) of the main document.
This behaviour can be altered by the alternative form
|\childdocby[*]{|\textit{main}|}| (with a non-empty optional argument)
which uses the |.aux| file of the main document
by setting |\jobname| to \textit{main}.

%%%%%%%%%%%%%%%%%%%%%%%%%%%%%%%%%%%%%%%%%%%%%%%%%%%%%%%%%%%%%%%%%%%%%%%%%%%%%%%%
\subsection{Driver Development}
\label{sec:driver}

The \textsf{childdoc} mechanism can also be use for the development
of definition files such as \LaTeX{} styles or classes.
This case differs from the above setup with multiple parts
included by |\include| in that no |\includeonly| should be invoked.
This can be achieved by starting the include file
(before |\ProvidesPackage|) with:
%
\begin{center}
\begin{tabular}{l}
|\input{childdoc.def}|\\
|\childdocforward{|\textit{main}|}|\\
\end{tabular}
\end{center}
%
or alternatively with:
%
\begin{center}
\begin{tabular}{l}
|\input{childdoc.def}|\\
|\childdocby{|\textit{main}|}|\\
\end{tabular}
\end{center}
%
Both forms have slightly different effects as described above.
The main file is prepared as usual, see \secref{sec:include}.

%%%%%%%%%%%%%%%%%%%%%%%%%%%%%%%%%%%%%%%%%%%%%%%%%%%%%%%%%%%%%%%%%%%%%%%%%%%%%%%%
\subsection{Legacy Detection}
\label{sec:detection}

The directive |\childdocmain| in the main file can detect
whether the complete document or merely a child is to be compiled
even without using the directive |\childdocof|.
This method is deprecated because it is less robust
and there is no compelling reason to use it;
it is merely provided for backward compatibility
and it may be removed in future versions.

If the detection mechanism is to be used,
it is mandatory to correctly specify
the filename of the main file as the argument of |\childdocmain|:
%
\begin{center}
\begin{tabular}{l}
|\input{childdoc.def}|\\
|\childdocmain{|\textit{main}|}|\\
\end{tabular}
\end{center}
%
If |\jobname| does not match the argument \textit{main} of |\childdocmain|,
it is assumed that |\jobname| points to the child file to be compiled.
When using |\childdocmain| with the main file specified as argument,
it suffices to start a child file
with just |\input{|\textit{main}|}|
without loading of the package and using |\childdocof|.
If instead all processing is done
with the appropriate \textsf{childdoc} directives,
the argument of \textit{main} of |\childdocmain| can be empty.

An alternative version of the command line processing described
in \secref{sec:commandline} using the detection mechanism reads:
%
\begin{center}
|... -jobname "|\textit{target}|" "|[\textit{flags}]%
[|\def\jobname{|\textit{dest}|}|]|\input{|\textit{main}|}"|
\end{center}

%%%%%%%%%%%%%%%%%%%%%%%%%%%%%%%%%%%%%%%%%%%%%%%%%%%%%%%%%%%%%%%%%%%%%%%%%%%%%%%%
\subsection{Manual Code}
\label{sec:manual}

In case one cannot be certain whether the definitions file |childdoc.def|
is installed on the target \TeX{} distribution
and one prefers not to ship it,
it is conceivable to paste a few relevant commands into the sources.

To that end, drop all statements |\input{childdoc.def}|
and perform the replacements as outlined below.
Instead of |\childdocmain{|\textit{main}|}| add the following code
to the top of the main file:
%
\begin{center}
\begin{tabular}{l}
|\||ifdefined\childdocname\endinput\||fi\newif\ifchilddoc|\\
|\edef\childdocname{\scantokens\expandafter{\jobname\noexpand}}|\\
|\def\childdocmain{|\textit{main}|}\||ifx\childdocmain\childdocname\||else|\\
|\childdoctrue\includeonly{\childdocname}\let\jobname\childdocmain\||fi|\\
\end{tabular}
\end{center}
%
Instead of |\childdocof{|\textit{main}|}| just include the main file
at the top of each child file:
%
\begin{center}
|\input{|\textit{main}|}|
\end{center}
%
A simple redirection |\childdocforward{|\textit{dest}|}| is achieved by:
%
\begin{center}
|\def\jobname{|\textit{dest}|}\input{\jobname}|
\end{center}
%
The redirection with prefix
|\childdocforwardprefix[|\textit{prefix}|]{|\textit{dest}|}|
is accomplished by:
%
\begin{center}
\begin{tabular}{l}
|{\edef\jobname{\scantokens\expandafter{\jobname\noexpand}}|\\
|\def\redirectjob |\textit{prefix}|#1~~~{\gdef\jobname{|\textit{dest}|#1}}|\\
|\expandafter\redirectjob\jobname~~~}\input{\jobname}|
\end{tabular}
\end{center}

In an alternative approach,
child documents can be compiled by a specific command line
without additional code or specific definitions:
%
\begin{center}
|... -jobname "|\textit{target}|" "|[\textit{flags}]%
|\includeonly{|\textit{dest}|}\input{|\textit{main}|}"|
\end{center}
%

%%%%%%%%%%%%%%%%%%%%%%%%%%%%%%%%%%%%%%%%%%%%%%%%%%%%%%%%%%%%%%%%%%%%%%%%%%%%%%%%
%%%%%%%%%%%%%%%%%%%%%%%%%%%%%%%%%%%%%%%%%%%%%%%%%%%%%%%%%%%%%%%%%%%%%%%%%%%%%%%%
\section{Information}

%%%%%%%%%%%%%%%%%%%%%%%%%%%%%%%%%%%%%%%%%%%%%%%%%%%%%%%%%%%%%%%%%%%%%%%%%%%%%%%%
\subsection{Copyright}

Copyright \copyright{} 2017--2018 Niklas Beisert

This work may be distributed and/or modified under the
conditions of the \LaTeX{} Project Public License, either version 1.3
of this license or (at your option) any later version.
The latest version of this license is in
  \url{http://www.latex-project.org/lppl.txt}
and version 1.3 or later is part of all distributions of \LaTeX{}
version 2005/12/01 or later.

This work has the LPPL maintenance status `maintained'.

The Current Maintainer of this work is Niklas Beisert.

This work consists of the files |README.txt|, |childdoc.ins| and |childdoc.dtx|
as well as the derived files |childdoc.def|, |cdocsamp.tex|
with |cdocsch1.tex|, |cdocsch2.tex|, |cdocspt3.tex|, |cdocspt4.tex|,
|cdocsdrf.tex|, |cdocsfn1.tex|, |cdocsfn2.tex|
as well as |childdoc.pdf|.

%%%%%%%%%%%%%%%%%%%%%%%%%%%%%%%%%%%%%%%%%%%%%%%%%%%%%%%%%%%%%%%%%%%%%%%%%%%%%%%%
\subsection{Files and Installation}

The package consists of the files:
%
\begin{center}
\begin{tabular}{ll}
    |README.txt|   & readme file \\
    |childdoc.ins| & installation file \\
    |childdoc.dtx| & source file \\
    |childdoc.def| & definition file \\
    |cdocsamp.tex| & sample main file \\
    |cdocsch1.tex| & sample include file \\
    |cdocsch2.tex| & sample include file \\
    |cdocspt3.tex| & sample part file \\
    |cdocspt4.tex| & sample part file \\
    |cdocsdrf.tex| & sample redirection file \\
    |cdocsfn1.tex| & sample redirection file \\
    |cdocsfn2.tex| & sample redirection file \\
    |childdoc.pdf| & manual
\end{tabular}
\end{center}
%
The distribution consists of the files
|README.txt|, |childdoc.ins| and |childdoc.dtx|.
%
\begin{itemize}
\item
Run (pdf)\LaTeX{} on |childdoc.dtx|
to compile the manual |childdoc.pdf| (this file).
\item
Run \LaTeX{} on |childdoc.ins| to create the definitions file |childdoc.def|
and the sample |cdocsamp.tex| with include files
|cdocsch1.tex|, |cdocsch2.tex|, |cdocspt3.tex|, |cdocspt4.tex|,
|cdocsdrf.tex|, |cdocsfn1.tex|, |cdocsfn2.tex|.
Then copy the file |childdoc.def| to an appropriate directory of your \LaTeX{}
distribution, e.g.\ \textit{texmf-root}|/tex/latex/childdoc|.
\end{itemize}

%%%%%%%%%%%%%%%%%%%%%%%%%%%%%%%%%%%%%%%%%%%%%%%%%%%%%%%%%%%%%%%%%%%%%%%%%%%%%%%%
\subsection{Related CTAN Packages}

There are several other packages which offer a similar functionality:
%
\begin{itemize}
\item
The packages
\href{http://ctan.org/pkg/docmute}{\textsf{docmute}},
\href{http://ctan.org/pkg/includex}{\textsf{includex}} and
\href{http://ctan.org/pkg/standalone}{\textsf{standalone}}
provide commands to include only the document body of
a child file thus allowing both files to be compiled individually.
\item
The packages \href{http://ctan.org/pkg/subdocs}{\textsf{subdocs}}
and \href{http://ctan.org/pkg/subfiles}{\textsf{subfiles}}
provide structures in which the main and child documents can be
encapsulated and allowing them to be compiled individually.
The inclusion mechanism is different from the conventional |\include|.
\item
The package \href{http://ctan.org/pkg/combine}{\textsf{combine}}
is an elaborate solution to combine several documents into one.
\end{itemize}
%
See also the CTAN topic \href{http://ctan.org/topic/subdocs}{\textsf{subdocs}}
for further related packages.
The present package differs from the above solutions in that
a document structure constructed with the conventional |\include| mechanism
just needs two extra commands at the top of every file
such that all constituent files can be compiled individually.

%%%%%%%%%%%%%%%%%%%%%%%%%%%%%%%%%%%%%%%%%%%%%%%%%%%%%%%%%%%%%%%%%%%%%%%%%%%%%%%%
%\subsection{Feature Suggestions}
%
%The following is a list of features which may be useful for future
%versions of this package:
%%
%\begin{itemize}
%\item
%\ldots
%\end{itemize}

%%%%%%%%%%%%%%%%%%%%%%%%%%%%%%%%%%%%%%%%%%%%%%%%%%%%%%%%%%%%%%%%%%%%%%%%%%%%%%%%
\subsection{Revision History}

%%%%%%%%%%%%%%%%%%%%%%%%%%%%%%%%%%%%%%%%
\paragraph{v2.0:} 2018/12/30

\begin{itemize}
\item
immediate forward processing
\item
added |\childdocby| mechanism
\item
manual restructured
\end{itemize}

%%%%%%%%%%%%%%%%%%%%%%%%%%%%%%%%%%%%%%%%
\paragraph{v1.6:} 2018/01/17

\begin{itemize}
\item
application for development of include files
\item
corrections to manual
\end{itemize}

%%%%%%%%%%%%%%%%%%%%%%%%%%%%%%%%%%%%%%%%
\paragraph{v1.5:} 2017/05/21

\begin{itemize}
\item
more complete structuring introduced
\item
|\childdocof| introduced
\item
|\childdoc| renamed to |\childdocmain|
\item
|\childredirect| renamed to |\childdocforward| and |\childdocforwardprefix|
and functionality expanded
\end{itemize}

%%%%%%%%%%%%%%%%%%%%%%%%%%%%%%%%%%%%%%%%
\paragraph{v1.0:} 2017/04/27

\begin{itemize}
\item
manual and install package
\item
first version published on CTAN
\end{itemize}

%%%%%%%%%%%%%%%%%%%%%%%%%%%%%%%%%%%%%%%%
\paragraph{v0.6:} 2017/04/26

\begin{itemize}
\item
redirection mechanism added
\end{itemize}

%%%%%%%%%%%%%%%%%%%%%%%%%%%%%%%%%%%%%%%%
\paragraph{v0.5:} 2017/04/26

\begin{itemize}
\item
functionality in definition file
\end{itemize}


%%%%%%%%%%%%%%%%%%%%%%%%%%%%%%%%%%%%%%%%%%%%%%%%%%%%%%%%%%%%%%%%%%%%%%%%%%%%%%%%
%%%%%%%%%%%%%%%%%%%%%%%%%%%%%%%%%%%%%%%%%%%%%%%%%%%%%%%%%%%%%%%%%%%%%%%%%%%%%%%%
%%%%%%%%%%%%%%%%%%%%%%%%%%%%%%%%%%%%%%%%%%%%%%%%%%%%%%%%%%%%%%%%%%%%%%%%%%%%%%%%
\appendix

\settowidth\MacroIndent{\rmfamily\scriptsize 000\ }

 \DocInput{childdoc.dtx}

\end{document}
%</driver>
% \fi
%
% %%%%%%%%%%%%%%%%%%%%%%%%%%%%%%%%%%%%%%%%%%%%%%%%%%%%%%%%%%%%%%%%%%%%%%%%%%%%%%
% %%%%%%%%%%%%%%%%%%%%%%%%%%%%%%%%%%%%%%%%%%%%%%%%%%%%%%%%%%%%%%%%%%%%%%%%%%%%%%
% \section{Sample}
%\iffalse
%<*samplemain>
%\fi
%
% The following presents a sample document
% with two chapters, two parts, a title page,
% a compile flag as well as three forwarding files to set the flag.
% It consists of eight |.tex| files:
% \begin{center}
% \begin{tabular}{ll}
% |cdocsamp.tex|&main file\\
% |cdocsch1.tex|&include file for chapter 1\\
% |cdocsch2.tex|&include file for chapter 2\\
% |cdocspt3.tex|&include file for part 3\\
% |cdocspt4.tex|&include file for part 4\\
% |cdocsdrf.tex|&forwarding file for main file in draft mode\\
% |cdocsfi1.tex|&forwarding file for final version of chapter 1\\
% |cdocsfi2.tex|&forwarding file for final version of chapter 2\\
% \end{tabular}
% \end{center}
% Each of the eight files can be compiled directly by the \LaTeX{} compiler.
%
% %%%%%%%%%%%%%%%%%%%%%%%%%%%%%%%%%%%%%%
% \paragraph{Main File.}
%
% The main file is called |cdocsamp.tex|.
%
% Load the \textsf{childdoc} definitions and
% declare the filename for the main document:
%    \begin{macrocode}
\input{childdoc.def}
\childdocmain{}
%    \end{macrocode}

% Optional override for |\version| flag:
%    \begin{macrocode}
%%\ifchilddoc\else\providecommand{\version}{draft}\fi
%    \end{macrocode}

% Define the default values for the |\version| flag
% (|final| for the main file and |draft| for childs):
%    \begin{macrocode}
\ifchilddoc
\providecommand{\version}{draft}
\else
\providecommand{\version}{final}
\fi
%    \end{macrocode}

% Load the standard document class:
%    \begin{macrocode}
\documentclass[12pt]{article}
%    \end{macrocode}

% Start the document body:
%    \begin{macrocode}
\begin{document}
%    \end{macrocode}

% Declare a title page.
% Print title, part of document being processed and version flag:
%    \begin{macrocode}
\addtocounter{page}{-1}
\begin{center}
{\LARGE\bfseries{}childdoc example\par}
\vspace{1cm}
\ifchilddoc
\ifchilddocmanual part\else chapter\fi:
`\childdocname' of `\childdocjob'\par
\else
main document: `\childdocjob'\par
\fi
version: \version\par
\end{center}
\newpage
%    \end{macrocode}

% Manually include selected file,
% otherwise process as usual:
%    \begin{macrocode}
\ifchilddocmanual
\section*{part `\childdocname'}
\input{\childdocname}
\else
%    \end{macrocode}

% Include the two chapters:
%    \begin{macrocode}
\include{cdocsch1}
\include{cdocsch2}
%    \end{macrocode}

% Include the two parts unless only chapters should be displayed:
%    \begin{macrocode}
\ifchilddoc\else
\section{part three}
\input{cdocspt3}
\section{part four}
\input{cdocspt4}
\fi
%    \end{macrocode}

% Process as usual until here:
%    \begin{macrocode}
\fi
%    \end{macrocode}

% End of document body:
%    \begin{macrocode}
\end{document}
%    \end{macrocode}
%\iffalse
%</samplemain>
%\fi
%
% %%%%%%%%%%%%%%%%%%%%%%%%%%%%%%%%%%%%%%
% \paragraph{Chapter Include Files.}
%
% The include files are called |cdocsch1.tex| and |cdocsch2.tex|.
%
%\iffalse
%<*samplechap1|samplechap2>
%\fi

% Optional override for |\version| flag:
%    \begin{macrocode}
%%\providecommand{\version}{final}
%    \end{macrocode}

% Include the main document:
%    \begin{macrocode}
\input{childdoc.def}
\childdocof{cdocsamp}
%    \end{macrocode}

%\iffalse
%</samplechap1|samplechap2>
%\fi
%
%\iffalse
%<*samplechap1>
%\fi
% Some text for chapter 1:
%    \begin{macrocode}
\section{one}
some text in chapter one
%    \end{macrocode}

%\iffalse
%</samplechap1>
%\fi
% Some text for chapter 2:
%\iffalse
%<*samplechap2>
%\fi
%    \begin{macrocode}
\section{two}
more text in chapter two
%    \end{macrocode}

%\iffalse
%</samplechap2>
%\fi
%
% %%%%%%%%%%%%%%%%%%%%%%%%%%%%%%%%%%%%%%
% \paragraph{Part Include Files.}
%
% The include files are called |cdocspt3.tex| and |cdocspt4.tex|.
%
%\iffalse
%<*samplepart3|samplepart4>
%\fi

% Optional override for |\version| flag:
%    \begin{macrocode}
%%\providecommand{\version}{final}
%    \end{macrocode}

% Include the main document:
%    \begin{macrocode}
\input{childdoc.def}
\childdocby{cdocsamp}
%    \end{macrocode}

%\iffalse
%</samplepart3|samplepart4>
%\fi
%
%\iffalse
%<*samplepart3>
%\fi
% Some text for part 3:
%    \begin{macrocode}
some text in part three
%    \end{macrocode}

%\iffalse
%</samplepart3>
%\fi
% Some text for part 4:
%\iffalse
%<*samplepart4>
%\fi
%    \begin{macrocode}
more text in part four
%    \end{macrocode}

%\iffalse
%</samplepart4>
%\fi
%
% %%%%%%%%%%%%%%%%%%%%%%%%%%%%%%%%%%%%%%
% \paragraph{Forwarding for a Complete Draft.}
%
% The following forwarding file |cdocsdrf.tex|
% compiles the main document in draft mode:
%\iffalse
%<*sampledraft>
%\fi
%    \begin{macrocode}
\def\version{draft}
\input{childdoc.def}
\childdocforward{cdocsamp}
%    \end{macrocode}

%\iffalse
%</sampledraft>
%\fi
%
% %%%%%%%%%%%%%%%%%%%%%%%%%%%%%%%%%%%%%%
% \paragraph{Forwarding for Final Version of the Chapters.}
%
% The following forwarding files |cdocsfn1.tex| and |cdocsfn2.tex|
% (with identical content)
% compile the final versions of the child documents
% |cdocsch1.tex| and |cdocsch2.tex|, respectively:
%\iffalse
%<*samplefinal>
%\fi
%    \begin{macrocode}
\def\version{final}
\input{childdoc.def}
\childdocforwardprefix[cdocsamp]{cdocsfn}{cdocsch}
%    \end{macrocode}

%\iffalse
%</samplefinal>
%\fi
%
% %%%%%%%%%%%%%%%%%%%%%%%%%%%%%%%%%%%%%%
% \paragraph{Command Line Processing.}
%
% The following three command lines generate the output files
% |cdocscld|, |cdocscl1| and |cdocscl2|
% which should be identical to
% |cdocsdrf|, |cdocsch1| and |cdocsfn2|, respectively:
% \begin{center}
% \begin{tabular}{l}
% |latex -jobname cdocscld \|\\
% |  "\def\version{draft}\input{childdoc.def}\childdocforward{cdocsamp}"|\\
% |latex -jobname cdocscl1 \|\\
% |  "\input{childdoc.def}\childdocforward[cdocsamp]{cdocsch1}"|\\
% |latex -jobname cdocscl2 \|\\
% |  "\def\version{final}\input{childdoc.def}\childdocforward{cdocsch2}"|
% \end{tabular}
% \end{center}
% Note that the trailing backslash on each first line
% merely continues the input to the second line
% (for convenient cut ant paste).
% Furthermore, the command |latex| can be replaced by any
% of its alternative versions such as |pdflatex|.
%
% %%%%%%%%%%%%%%%%%%%%%%%%%%%%%%%%%%%%%%%%%%%%%%%%%%%%%%%%%%%%%%%%%%%%%%%%%%%%%%
% %%%%%%%%%%%%%%%%%%%%%%%%%%%%%%%%%%%%%%%%%%%%%%%%%%%%%%%%%%%%%%%%%%%%%%%%%%%%%%
% \section{Implementation}
%\iffalse
%<*package>
%\fi
%
% This section describes the definitions file |childdoc.def|.

% The definitions cannot be loaded using |\usepackage| or |\RequirePackage|
% which has a mechanism to prevent loading a style file more than once.
% When loading the definitions by means of |\input|
% multiple instances have to be prevented manually:
%\iffalse
%This code needs to be before the `\ProvidesFile' directive
%which is defined at the beginning of this file.
%Therefore it is also placed there and commented out here.
%</package>
%<*discard>
%\fi
%    \begin{macrocode}
\ifdefined\childdocmain\endinput\fi
%    \end{macrocode}
%\iffalse
%</discard>
%<*package>
%\fi
%
% \macro{\ifchilddoc}
% \macro{\ifchilddocmanual}
% The conditional |\ifchilddoc| tells whether a
% child (true) or main (false) document is being compiled.
% The conditional |\ifchilddocmanual| tells whether
% the |\includeonly| mechanism is used (false) or
% the selection of child files must be performed manually (true).
% The definitions initialise to false:
%    \begin{macrocode}
\newif\ifchilddoc
\newif\ifchilddocmanual
%    \end{macrocode}

% \macro{\childdocname}
% \macro{\childdocjob}
% The macro |\childdocname| stores the name of the main document
% to be compiled. The macro |\childdocjob| stores the name of
% the document on which the \LaTeX{} compiler was originally invoked.
% The content of |\jobname| cannot be compared
% to filenames specified in the source due to different catcodes.
% The following code rescans |\jobname|, stores the result
% in |\childdocname| and saves a copy in |\childdocjob|:
%    \begin{macrocode}
\edef\childdocname{\scantokens\expandafter{\jobname\noexpand}}
\let\childdocjob\childdocname
%    \end{macrocode}

% \macro{\childdocdisable}
% The macro |\childdocdisable| prevents the main file
% from being processed more than once.
% At this stage, the main document command |\childdocmain|
% is assumed to be called once again where it should do nothing.
% Any subsequent call to it should prevent
% a secondary processing of the main document
% It overwrites the forwarding commands
% |\childdocof| and |\childdocforward|
% with empty macros to prevent further inclusions of the main document:
%    \begin{macrocode}
\newcommand{\childdocdisable}
{
  \renewcommand{\childdocmain}[1]{\renewcommand{\childdocmain}[1]{\endinput}}
  \renewcommand{\childdocof}[1]{}
  \renewcommand{\childdocby}[2][]{}
  \renewcommand{\childdocforward}[2][]{}
  \renewcommand{\childdocdisable}{}
}
%    \end{macrocode}

% \macro{\childdocmain}
% The macro |\childdocmain| is to be called at the top of the main file
% with nothing or the main filename (without extension) as argument.
% First, it breaks loops.
% If the argument is not empty and does not match |\childdocname|
% (which is set by the first inclusion of |childdoc.def|),
% |\ifchilddoc| is set to true, |\includeonly| is applied to the child file
% and |\jobname| is set to the main file
% (for proper handling of |.aux| files):
%    \begin{macrocode}
\newcommand{\childdocmain}[1]
{
  \childdocdisable\childdocmain{}
  \if?#1?\else
    \begingroup
      \def\childdoctmp{#1}
      \ifx\childdoctmp\childdocname
        \def\childdoctmp{}
      \else
        \def\childdoctmp
        {
          \childdoctrue
          \includeonly{\childdocname}
          \def\childdocjob{#1}
          \def\jobname{#1}
        }
      \fi
      \expandafter
    \endgroup
    \childdoctmp
  \fi
}
%    \end{macrocode}

% \macro{\childdocof}
% The command |\childdocof| redirects
% compilation to the main file |#1|.
%    \begin{macrocode}
\newcommand{\childdocof}[1]
{
  \childdocdisable
  \childdoctrue
  \includeonly{\childdocname}
  \def\jobname{#1}
  \def\childdocjob{#1}
  \input{#1}
}
%    \end{macrocode}

% \macro{\childdocby}
% The command |\childdocby| ....
%    \begin{macrocode}
\newcommand{\childdocby}[2][]
{
  \childdocdisable
  \childdoctrue
  \childdocmanualtrue
  \if?#1?\else
    \def\jobname{#2}
  \fi
  \def\childdocjob{#2}
  \input{#2}
  \endinput
}
%    \end{macrocode}

% \macro{\childdocforward}
% The command |\childdocforward| redirects
% compilation to the main file or
% (if the optional argument is given) a child file.
% Parameters are set as if the main file
% or a child file starting with |\childdocof| was compiled.
% Then compilation is handed over to the main file:
%    \begin{macrocode}
\newcommand{\childdocforward}[2][]
{
  \begingroup
    \if?#1?
      \def\childdoctmp
      {
        \def\childdocname{#2}
        \def\childdocjob{#2}
        \def\jobname{#2}
        \input{#2}
        \endinput
      }
    \else
      \def\childdoctmp
      {
        \childdocdisable
        \def\childdocname{#2}
        \childdoctrue
        \includeonly{#2}
        \def\childdocjob{#1}
        \def\jobname{#1}
        \input{#1}
        \endinput
      }
    \fi
    \expandafter
  \endgroup
  \childdoctmp
}
%    \end{macrocode}

% \macro{\childdocforwardprefix}
% The command |\childdocforwardprefix| redirects
% compilation to the main or a child file by means of a pattern.
% The prefix |#1| in the current filename is replaced by |#2|
% and the suffix of the current filename is kept
% (it is assumed that the filename does not contain the substring `|~~~|'
% which is used as a delimiter).
% Compilation is handed over to the new file by |\childdocforward|:
%    \begin{macrocode}
\newcommand{\childdocforwardprefix}[3][]
{
  \begingroup
    \def\childdocextract #2##1~~~{\def\childdoctmp{\childdocforward[#1]{#3##1}}}
    \expandafter\childdocextract\childdocname~~~
    \expandafter
  \endgroup
  \childdoctmp
}
%    \end{macrocode}

% \macro{\childdoc}
% The deprecated macro |\childdoc| is a legacy version of |\childdocmain|:
%    \begin{macrocode}
\newcommand{\childdoc}{\childdocmain}
%    \end{macrocode}

% \macro{\childdocredirect}
% The deprecated macro |\childdocredirect| is a legacy version
% of |\childdocforward| and |\childdocforwardprefix|:
%    \begin{macrocode}
\newcommand{\childdocredirect}[2][]
{
  \begingroup
    \if?#1?
      \def\childdoctmp{\childdocforward{#2}}
    \else
      \def\childdoctmp{\childdocforwardprefix{#1}{#2}}
    \fi
    \expandafter
  \endgroup
  \childdoctmp
}
%    \end{macrocode}

%\iffalse
%</package>
%\fi
%
\endinput
|\\
|\childdocforward{|\textit{main}|}|\\
\end{tabular}
\end{center}
%
or alternatively with:
%
\begin{center}
\begin{tabular}{l}
|% \iffalse
%
% childdoc.dtx Copyright (C) 2017-2018 Niklas Beisert
%
% This work may be distributed and/or modified under the
% conditions of the LaTeX Project Public License, either version 1.3
% of this license or (at your option) any later version.
% The latest version of this license is in
%   http://www.latex-project.org/lppl.txt
% and version 1.3 or later is part of all distributions of LaTeX
% version 2005/12/01 or later.
%
% This work has the LPPL maintenance status `maintained'.
%
% The Current Maintainer of this work is Niklas Beisert.
%
% This work consists of the files childdoc.dtx and childdoc.ins
% and the derived files childdoc.def and cdocsamp.tex with
% cdocsch1.tex, cdocsch2.tex, cdocsdrf.tex, cdocsfn1.tex, cdocsfn2.tex.
%
%<package>\ifdefined\childdocmain\endinput\fi
%<package>\ProvidesFile{childdoc.def}[2018/12/30 v2.0 child document driver]
%<samplemain>\ProvidesFile{cdocsamp.tex}[2018/12/30 v2.0 sample for childdoc]
%<*driver>
%\ProvidesFile{childdoc.drv}[2018/12/30 v2.0 childdoc reference manual file]
\PassOptionsToClass{10pt,a4paper}{article}
\documentclass{ltxdoc}

\usepackage[margin=35mm]{geometry}
\usepackage{hyperref}
\usepackage{hyperxmp}
\usepackage[usenames]{color}

\hypersetup{colorlinks=true}
\hypersetup{pdfstartview=FitH}
\hypersetup{pdfpagemode=UseNone}
\hypersetup{pdfsource={}}
\hypersetup{pdflang={en-UK}}
\hypersetup{pdfcopyright={Copyright 2017-2018 Niklas Beisert.
  This work may be distributed and/or modified under the
  conditions of the LaTeX Project Public License, either version 1.3
  of this license or (at your option) any later version.}}
\hypersetup{pdflicenseurl={http://www.latex-project.org/lppl.txt}}
\hypersetup{pdfcontactaddress={ETH Zurich, ITP, HIT K,
  Wolfgang-Pauli-Strasse 27}}
\hypersetup{pdfcontactpostcode={8093}}
\hypersetup{pdfcontactcity={Zurich}}
\hypersetup{pdfcontactcountry={Switzerland}}
\hypersetup{pdfcontactemail={nbeisert@itp.phys.ethz.ch}}
\hypersetup{pdfcontacturl={http://people.phys.ethz.ch/\xmptilde nbeisert/}}

\newcommand{\secref}[1]{\hyperref[#1]{section \ref*{#1}}}

\parskip1ex
\parindent0pt
\let\olditemize\itemize
\def\itemize{\olditemize\parskip0pt}

\begin{document}

\title{The \textsf{childdoc} Package}
\hypersetup{pdftitle={The childdoc Package}}
\author{Niklas Beisert\\[2ex]
  Institut f\"ur Theoretische Physik\\
  Eidgen\"ossische Technische Hochschule Z\"urich\\
  Wolfgang-Pauli-Strasse 27, 8093 Z\"urich, Switzerland\\[1ex]
  \href{mailto:nbeisert@itp.phys.ethz.ch}
  {\texttt{nbeisert@itp.phys.ethz.ch}}}
\hypersetup{pdfauthor={Niklas Beisert}}
\hypersetup{pdfsubject={Manual for the LaTeX2e Package childdoc}}
\date{30 December 2018, \textsf{v2.0}}
\maketitle

\begin{abstract}\noindent
\textsf{childdoc} is a \LaTeXe{} package
that enables the direct compilation
of document sections included by |\include|
to individual files.
\end{abstract}

\begingroup
\parskip0ex
\tableofcontents
\endgroup

%%%%%%%%%%%%%%%%%%%%%%%%%%%%%%%%%%%%%%%%%%%%%%%%%%%%%%%%%%%%%%%%%%%%%%%%%%%%%%%%
%%%%%%%%%%%%%%%%%%%%%%%%%%%%%%%%%%%%%%%%%%%%%%%%%%%%%%%%%%%%%%%%%%%%%%%%%%%%%%%%
\section{Introduction}

\LaTeX{} provides a mechanism to structure a large document (such as a book)
into a main file and several child files (containing the chapters)
using the |\include| command.
This mechanism is beneficial for documents
which span hundreds of pages in order to
make the source file(s) more manageable.
Moreover, compilation can be restricted to
selected child files by means of the |\includeonly| command.
The latter feature can be used to reduce the compilation time while editing
(this was significantly more useful in the earlier days of \LaTeX{})
or to generate a smaller document which is easier to navigate.
Another application of |\includeonly| is to generate
documents consisting of selected parts of the complete document.

However, there are a few drawbacks of the plain |\include| mechanism:
\begin{itemize}
\item
The child files cannot be compiled on their own,
they can only be compiled via the main file.
A naive editing environment
(such as a text editor with an option
to have the current file processed by \LaTeX)
may require one to switch to the main file before compiling;
attempting to compile the child file produces errors.
\item
The main file must be modified (each time)
to adjust the |\includeonly| command
to the present needs. This easily leaves the main file in a messy state.
\item
The generated document will always carry the filename
of the main document. This is inconvenient if
several child files are to be compiled and
to be kept for distribution.
\end{itemize}

The present package provides a simple interface
to make child files individually compilable by \LaTeX{}.
Compiling a child file then has the same effect as compiling
the main file with an |\includeonly| command
to select the appropriate child.
Moreover the generated document will carry the name of the child
rather than the main file.
This resolves all three above issues.

This feature is meant to make the editing of books,
thesis documents and lecture notes somewhat more convenient.
However, the package can also be used efficiently for
composing a series of documents (such as exercise sheets)
which are typically distributed individually.
It then assists the author in generating the individual documents
(potentially in different versions)
as well as a document containing the collected series.
Another application is in developing style files
or other kinds of included material
where compilation of the style file could redirect
to a sample or test file.

%%%%%%%%%%%%%%%%%%%%%%%%%%%%%%%%%%%%%%%%%%%%%%%%%%%%%%%%%%%%%%%%%%%%%%%%%%%%%%%%
%%%%%%%%%%%%%%%%%%%%%%%%%%%%%%%%%%%%%%%%%%%%%%%%%%%%%%%%%%%%%%%%%%%%%%%%%%%%%%%%
\section{Usage}

First of all, the package \textsf{childdoc} is \emph{not} a standard
\LaTeXe{} |.sty| style file! Therefore it needs to be invoked in
a non-standard way.

%%%%%%%%%%%%%%%%%%%%%%%%%%%%%%%%%%%%%%%%%%%%%%%%%%%%%%%%%%%%%%%%%%%%%%%%%%%%%%%%
\subsection{Included Files}
\label{sec:include}

%%%%%%%%%%%%%%%%%%%%%%%%%%%%%%%%%%%%%%%%
\DescribeMacro{\childdocmain}
To use the package, add the commands
\begin{center}
\begin{tabular}{l}
|\input{childdoc.def}|\\
|\childdocmain{}|\\
\end{tabular}
\end{center}
at the very top of the main \LaTeX{} file,
in particular \emph{before} the |\documentclass| statement!
The argument of |\childdocmain| should be left empty
(but it must be present).

%%%%%%%%%%%%%%%%%%%%%%%%%%%%%%%%%%%%%%%%
\DescribeMacro{\childdocof}
Furthermore, add the commands
\begin{center}
\begin{tabular}{l}
|\input{childdoc.def}|\\
|\childdocof{|\textit{main}|}|\\
\end{tabular}
\end{center}
at the top of every child file \textit{child}
which is included by |\include{|\textit{child}|}|
from within the main file
(or at least for those files to be compiled individually).
The argument \textit{main} must be the filename of the main file.

There are a couple of
considerations in setting up the main and child documents:

%%%%%%%%%%%%%%%%%%%%%%%%%%%%%%%%%%%%%%%%
\paragraph{Restrictions.}

Please note the following restrictions:
\begin{itemize}
\item
|\childdocmain| must be called with one argument \textit{main}
to ensure compatibility with earlier version of the package.
It must either be empty (|\childdocmain{}|)
or precisely match the filename of the main file in which it is specified.
See \secref{sec:detection} for further information.
\item
The filename \textit{main} must be specified without the |.tex| extension.
\item
The filename \textit{main} is case sensitive
(even in case-insensitive file systems)
due to internal string comparison.
\item
The argument \textit{main} should be fully expanded, it cannot be a macro.
\item
Subdirectories and special characters should be avoided in filenames.
\item
The command |\childdocmain{|\textit{main}|}| must be followed by a whitespace.
It should not be followed immediately by another command
or by a comment mark `|%|'.
This is because the \TeX{} parser reads the token immediately following
the argument of |\childdocmain| and puts it
at the beginning of every child section;
however, a white\-space is ignored.
\end{itemize}

%%%%%%%%%%%%%%%%%%%%%%%%%%%%%%%%%%%%%%%%
\paragraph{Content of Main File.}

It is advisable to place all content in the child files included by |\include|.
Any output contained in the main file will appear in all child documents
unless suppressed manually;
it cannot be suppressed automatically by the |\includeonly| directive
and thus should normally be avoided.
A method to include some content in the main file
by means of conditional processing is described in \secref{sec:conditional}.

%%%%%%%%%%%%%%%%%%%%%%%%%%%%%%%%%%%%%%%%
\paragraph{Page Numbering.}

When only a part of the document is compiled,
the appropriate numbering of pages
(as well as other status parameters)
is determined from the |.aux| files.
The latter contain information from previous passes.
However this information needs to propagate through
all intermediate child documents.
Therefore the page numbering in child documents may well
be inconsistent until the complete document is compiled at least once.

A useful (if unconventional) way to always ensure a consistent
page numbering is to restart the numbering in each child document
and denote the pages by `\textit{child}|.|\textit{page}'
where \textit{child} represents the chapter/section number of the child file.
This can be achieved by the command
|\numberwithin{page}{|\textit{child}|}|
of the \textsf{amsmath} package
where \textit{child} can be |chapter| or |section|
depending on the chosen structuring.
Alternatively, one can modify the macro |\thepage| appropriately
and reset the counter |page| at the start of each child file.

%%%%%%%%%%%%%%%%%%%%%%%%%%%%%%%%%%%%%%%%%%%%%%%%%%%%%%%%%%%%%%%%%%%%%%%%%%%%%%%%
\subsection{Conditional Processing}
\label{sec:conditional}

The package provides a mechanism to compile different versions
of a document. To customise the versions further some conditional processing
can come in handy to distinguish which version is being compiled.
The package provides two macros to describe the compilation context:

%%%%%%%%%%%%%%%%%%%%%%%%%%%%%%%%%%%%%%%%
\DescribeMacro{\ifchilddoc}
The conditional |\ifchilddoc| distinguishes between the compilation of
child documents and the main document:
%
\begin{center}
|\ifchilddoc |\textit{child-code}| |[|\||else |\textit{main-code}]| \||fi|
\end{center}

%%%%%%%%%%%%%%%%%%%%%%%%%%%%%%%%%%%%%%%%
\DescribeMacro{\childdocname}
\DescribeMacro{\childdocjob}
The macro |\childdocname| contains the filename (without extension)
of the main or child file being processed.
Note that |\childdocjob| will always contain the name of the main file.

%%%%%%%%%%%%%%%%%%%%%%%%%%%%%%%%%%%%%%%%
\paragraph{Title Page.}

Conditional processing can be used to include a title or banner page
in the main document when proper precautions are taken.
Importantly, the code in the main file should ensure that the page counter
(as well as other status parameters which are stored in the |.aux| files)
takes the same value after the conditional processing.
Otherwise the page numbers may take divergent values
depending on which part is compiled.

For example, a title page could be declared by:
%
\begin{center}
\begin{tabular}{l}
|\ifchilddoc\||else|\\
|\addtocounter{page}{-1}|\\
\textit{code for title page}\\
|\newpage|\\
|\||fi|
\end{tabular}
\end{center}
%
A banner page for the child documents can be generated by:
%
\begin{center}
\begin{tabular}{l}
|\ifchilddoc|\\
|\addtocounter{page}{-1}|\\
\textit{code for banner page}\\
|\newpage|\\
|\||fi|
\end{tabular}
\end{center}
%
Here one could write a message such as:
\begin{center}
|This is the part \childdocname{} of \childdocjob{}.|
\end{center}

%%%%%%%%%%%%%%%%%%%%%%%%%%%%%%%%%%%%%%%%%%%%%%%%%%%%%%%%%%%%%%%%%%%%%%%%%%%%%%%%
\subsection{Flags}
\label{sec:flags}

The package makes it easy to generate different versions
of the main or child documents.
To this end compilation flags can be defined
and assigned different default values.
They will be particularly useful in conjunction
with the forwarding mechanism described in \secref{sec:forward}.

For example, it may be useful to have a flag |\version|
which can be set to |draft| or |final|.
The document source will contain some conditional code
depending on the value of |\version|.
Suppose further, the flag should default to |final| for the main file
and to |draft| for child files
which is a natural assignment for editing the document.
This is achieved by placing the following code
in the preamble of the main document
(below the |\childdocmain| directive):
%
\begin{center}
\begin{tabular}{l}
|\ifchilddoc|\\
|\providecommand{\version}{draft}|\\
|\||else|\\
|\providecommand{\version}{final}|\\
|\||fi|
\end{tabular}
\end{center}
%
The definition by |\providecommand| makes sure
that previous definitions are not overwritten.
Further statements |\providecommand{\version}{...}|
can thus be added before the above code to override it.

For the main file, one might add a line
(between |\childdocmain| and the above block)
%
\begin{center}
|%\ifchilddoc\||else\providecommand{\version}{draft}\||fi|
\end{center}
%
which can be uncommented to produce a draft version.
Likewise one can add a line to the very top of a child file
(above the |\childdocof{|\textit{main}|}| directive)
%
\begin{center}
|%\providecommand{\version}{final}|
\end{center}
%
which can be uncommented to produce the final version of this child document.

%%%%%%%%%%%%%%%%%%%%%%%%%%%%%%%%%%%%%%%%%%%%%%%%%%%%%%%%%%%%%%%%%%%%%%%%%%%%%%%%
\subsection{Forwarding}
\label{sec:forward}

Different versions of the main or child documents
using compilation flags as described in \secref{sec:flags}
can be (permanently) stored in different files
for convenient compilation, viewing and distribution.
To this end, the package defines a command
to pass on compilation to a different file:

%%%%%%%%%%%%%%%%%%%%%%%%%%%%%%%%%%%%%%%%
\DescribeMacro{\childdocforward}
The command |\childdocforward| redirects processing to
another source file:
%
\begin{center}
\begin{tabular}{l}
|\input{childdoc.def}|\\
|\childdocforward[|\textit{main}|]{|\textit{dest}|}|\\
\end{tabular}
\end{center}
%
The argument \textit{dest} is the destination file
(without extension).
It should be the main file or one of the child files.
Note that further \textsf{childdoc} directives
such as |\childdocof| and |\childdocforward|
in the indicated file will be processed in this form.
The optional argument \textit{main}
passes on directly to the main file \textit{main}
while pretending to compile the child \textit{dest}.
This form behaves as if \textit{dest}
issues |\childdocof{|\textit{main}|}| right away,
and no further \textsf{childdoc} directives will be processed.

%%%%%%%%%%%%%%%%%%%%%%%%%%%%%%%%%%%%%%%%
\DescribeMacro{\...prefix}
In the alternative form |\childdocforwardprefix|,
%
\begin{center}
\begin{tabular}{l}
|\input{childdoc.def}|\\
|\childdocforwardprefix[|\textit{main}|]{|\textit{prefix}|}{|\textit{dest}|}|
\end{tabular}
\end{center}
%
the destination file is determined by a pattern
depending on the current file:
To make this work, the current file must be called
`{\textit{prefix}\hspace{0.2em}\textit{suffix}}'
with \textit{prefix} matching precisely the argument.
Processing is then passed on to the file
`{\textit{dest}\hspace{0.2em}\textit{suffix}}'.
Surely, the same effect is achieved by
directly specifying the
argument `{\textit{dest}\hspace{0.2em}\textit{suffix}}'
in the first form.
However, that requires to set up a different file
for each child. With the alternative form of the command
all these files can have exactly the same content
which simplifies setting them up and maintaining them.

For example, the following file |draft.tex|
with a compilation flag |\version| as described in \secref{sec:flags}
compiles the main document as a draft:
%
\begin{center}
\begin{tabular}{l}
|\def\version{draft}|\\
|\input{childdoc.def}|\\
|\childdocforward{|\textit{main}|}|
\end{tabular}
\end{center}
%
Likewise, the following files |final|\textit{nn}|.tex|
compile the final version of the child document
|child|\textit{nn}|.tex|:
%
\begin{center}
\begin{tabular}{l}
|\def\version{final}|\\
|\input{childdoc.def}|\\
|\childdocforwardprefix{final}{child}|
\end{tabular}
\end{center}
%

Note that when several versions of a main file and/or of each child file
are to be generated, it may be convenient to set up a |Makefile| or
shell script to automatise the process.

%%%%%%%%%%%%%%%%%%%%%%%%%%%%%%%%%%%%%%%%%%%%%%%%%%%%%%%%%%%%%%%%%%%%%%%%%%%%%%%%
\subsection{Command Line Processing}
\label{sec:commandline}

The effect of redirection files can also be achieved by invoking
the \LaTeX{} compiler with a more elaborate command line.
Most conveniently this should be done as part
of a shell script or a |Makefile|.

When using \textsf{childdoc} in the main file, the following
command lines effectively perform a redirection
(note that depending on the shell being used,
backslashes may have to be doubled: `|\|' $\to$ `|\\|'):
%
\begin{center}
|... -jobname "|\textit{target}|" |\\|"|[\textit{flags}]%
|\input{childdoc.def}\childdocforward[|\textit{main}|]{|\textit{dest}|}"|
\end{center}
%
Here \textit{target} is the name of the output file,
\textit{main} is the name of the main file
and \textit{dest} is the name of the main or child file to be processed
(all filenames without extensions).
The optional argument \textit{main} can be omitted
if \textit{main} matches \textit{dest}.
Optionally, compilation \textit{flags} can be defined via |\def| commands.
This command line makes the \TeX{} engine believe
it is compiling the file \textit{target}
whose content is specified as the latter parameter.
The provided code then forwards the processing to
\textit{main} or \textit{dest} as described in \secref{sec:forward}.

%%%%%%%%%%%%%%%%%%%%%%%%%%%%%%%%%%%%%%%%%%%%%%%%%%%%%%%%%%%%%%%%%%%%%%%%%%%%%%%%
\subsection{Include by Input}
\label{sec:input}

Including child documents by |\include| has some restrictions by design.
Most notably, the content of a child document always occupies
its own set of pages; pages cannot be shared between child documents.
Usually, this behaviour makes perfect sense
because each child document contain an essential part of the document.
However, in some situations it may be desirable to compose
a document from a collection of parts
without having mandatory page breaks between then.
For this case, the package
provides a mechanism to include parts
by |\input| which can also be processed individually.
However, by construction this mechanism
requires manual handling of the content to be output.

%%%%%%%%%%%%%%%%%%%%%%%%%%%%%%%%%%%%%%%%
\DescribeMacro{\ifchilddocmanual}
The main file should be prepared as usual, see \secref{sec:include}.
However, the document body must make a distinction
between processing of an individual part and of the main document, e.g.:
%
\begin{center}
\begin{tabular}{l}
|\ifchilddocmanual|\\
|\input{\childdocname}|\\
|\||else|\\
\textit{document body with }|\input{|\textit{part}|}|\\
|\||fi|
\end{tabular}
\end{center}
%
The conditional |\ifchilddocmanual| is true whenever
a part to be included by |\input| is being compiled,
and the name of the part is stored in |\childdocname|.

%%%%%%%%%%%%%%%%%%%%%%%%%%%%%%%%%%%%%%%%
\DescribeMacro{\childdocby}
Each part to be included by |\input| should start with:
%
\begin{center}
\begin{tabular}{l}
|\input{childdoc.def}|\\
|\childdocby{|\textit{main}|}|\\
\end{tabular}
\end{center}
%
The directive |\childdocby| is similar to |\childdocof|
described in \secref{sec:include},
but the subsequent selection of content must be done manually.
To that end, both |\ifchilddoc| and |\ifchilddocmanual|
will be true upon processing of a part,
and the name of the part is stored in |\childdocname|.
Note that |\jobname| will be set to the filename of the current part
so that each part receives an individual |.aux| file
that does not interfere with the |.aux| file(s) of the main document.
This behaviour can be altered by the alternative form
|\childdocby[*]{|\textit{main}|}| (with a non-empty optional argument)
which uses the |.aux| file of the main document
by setting |\jobname| to \textit{main}.

%%%%%%%%%%%%%%%%%%%%%%%%%%%%%%%%%%%%%%%%%%%%%%%%%%%%%%%%%%%%%%%%%%%%%%%%%%%%%%%%
\subsection{Driver Development}
\label{sec:driver}

The \textsf{childdoc} mechanism can also be use for the development
of definition files such as \LaTeX{} styles or classes.
This case differs from the above setup with multiple parts
included by |\include| in that no |\includeonly| should be invoked.
This can be achieved by starting the include file
(before |\ProvidesPackage|) with:
%
\begin{center}
\begin{tabular}{l}
|\input{childdoc.def}|\\
|\childdocforward{|\textit{main}|}|\\
\end{tabular}
\end{center}
%
or alternatively with:
%
\begin{center}
\begin{tabular}{l}
|\input{childdoc.def}|\\
|\childdocby{|\textit{main}|}|\\
\end{tabular}
\end{center}
%
Both forms have slightly different effects as described above.
The main file is prepared as usual, see \secref{sec:include}.

%%%%%%%%%%%%%%%%%%%%%%%%%%%%%%%%%%%%%%%%%%%%%%%%%%%%%%%%%%%%%%%%%%%%%%%%%%%%%%%%
\subsection{Legacy Detection}
\label{sec:detection}

The directive |\childdocmain| in the main file can detect
whether the complete document or merely a child is to be compiled
even without using the directive |\childdocof|.
This method is deprecated because it is less robust
and there is no compelling reason to use it;
it is merely provided for backward compatibility
and it may be removed in future versions.

If the detection mechanism is to be used,
it is mandatory to correctly specify
the filename of the main file as the argument of |\childdocmain|:
%
\begin{center}
\begin{tabular}{l}
|\input{childdoc.def}|\\
|\childdocmain{|\textit{main}|}|\\
\end{tabular}
\end{center}
%
If |\jobname| does not match the argument \textit{main} of |\childdocmain|,
it is assumed that |\jobname| points to the child file to be compiled.
When using |\childdocmain| with the main file specified as argument,
it suffices to start a child file
with just |\input{|\textit{main}|}|
without loading of the package and using |\childdocof|.
If instead all processing is done
with the appropriate \textsf{childdoc} directives,
the argument of \textit{main} of |\childdocmain| can be empty.

An alternative version of the command line processing described
in \secref{sec:commandline} using the detection mechanism reads:
%
\begin{center}
|... -jobname "|\textit{target}|" "|[\textit{flags}]%
[|\def\jobname{|\textit{dest}|}|]|\input{|\textit{main}|}"|
\end{center}

%%%%%%%%%%%%%%%%%%%%%%%%%%%%%%%%%%%%%%%%%%%%%%%%%%%%%%%%%%%%%%%%%%%%%%%%%%%%%%%%
\subsection{Manual Code}
\label{sec:manual}

In case one cannot be certain whether the definitions file |childdoc.def|
is installed on the target \TeX{} distribution
and one prefers not to ship it,
it is conceivable to paste a few relevant commands into the sources.

To that end, drop all statements |\input{childdoc.def}|
and perform the replacements as outlined below.
Instead of |\childdocmain{|\textit{main}|}| add the following code
to the top of the main file:
%
\begin{center}
\begin{tabular}{l}
|\||ifdefined\childdocname\endinput\||fi\newif\ifchilddoc|\\
|\edef\childdocname{\scantokens\expandafter{\jobname\noexpand}}|\\
|\def\childdocmain{|\textit{main}|}\||ifx\childdocmain\childdocname\||else|\\
|\childdoctrue\includeonly{\childdocname}\let\jobname\childdocmain\||fi|\\
\end{tabular}
\end{center}
%
Instead of |\childdocof{|\textit{main}|}| just include the main file
at the top of each child file:
%
\begin{center}
|\input{|\textit{main}|}|
\end{center}
%
A simple redirection |\childdocforward{|\textit{dest}|}| is achieved by:
%
\begin{center}
|\def\jobname{|\textit{dest}|}\input{\jobname}|
\end{center}
%
The redirection with prefix
|\childdocforwardprefix[|\textit{prefix}|]{|\textit{dest}|}|
is accomplished by:
%
\begin{center}
\begin{tabular}{l}
|{\edef\jobname{\scantokens\expandafter{\jobname\noexpand}}|\\
|\def\redirectjob |\textit{prefix}|#1~~~{\gdef\jobname{|\textit{dest}|#1}}|\\
|\expandafter\redirectjob\jobname~~~}\input{\jobname}|
\end{tabular}
\end{center}

In an alternative approach,
child documents can be compiled by a specific command line
without additional code or specific definitions:
%
\begin{center}
|... -jobname "|\textit{target}|" "|[\textit{flags}]%
|\includeonly{|\textit{dest}|}\input{|\textit{main}|}"|
\end{center}
%

%%%%%%%%%%%%%%%%%%%%%%%%%%%%%%%%%%%%%%%%%%%%%%%%%%%%%%%%%%%%%%%%%%%%%%%%%%%%%%%%
%%%%%%%%%%%%%%%%%%%%%%%%%%%%%%%%%%%%%%%%%%%%%%%%%%%%%%%%%%%%%%%%%%%%%%%%%%%%%%%%
\section{Information}

%%%%%%%%%%%%%%%%%%%%%%%%%%%%%%%%%%%%%%%%%%%%%%%%%%%%%%%%%%%%%%%%%%%%%%%%%%%%%%%%
\subsection{Copyright}

Copyright \copyright{} 2017--2018 Niklas Beisert

This work may be distributed and/or modified under the
conditions of the \LaTeX{} Project Public License, either version 1.3
of this license or (at your option) any later version.
The latest version of this license is in
  \url{http://www.latex-project.org/lppl.txt}
and version 1.3 or later is part of all distributions of \LaTeX{}
version 2005/12/01 or later.

This work has the LPPL maintenance status `maintained'.

The Current Maintainer of this work is Niklas Beisert.

This work consists of the files |README.txt|, |childdoc.ins| and |childdoc.dtx|
as well as the derived files |childdoc.def|, |cdocsamp.tex|
with |cdocsch1.tex|, |cdocsch2.tex|, |cdocspt3.tex|, |cdocspt4.tex|,
|cdocsdrf.tex|, |cdocsfn1.tex|, |cdocsfn2.tex|
as well as |childdoc.pdf|.

%%%%%%%%%%%%%%%%%%%%%%%%%%%%%%%%%%%%%%%%%%%%%%%%%%%%%%%%%%%%%%%%%%%%%%%%%%%%%%%%
\subsection{Files and Installation}

The package consists of the files:
%
\begin{center}
\begin{tabular}{ll}
    |README.txt|   & readme file \\
    |childdoc.ins| & installation file \\
    |childdoc.dtx| & source file \\
    |childdoc.def| & definition file \\
    |cdocsamp.tex| & sample main file \\
    |cdocsch1.tex| & sample include file \\
    |cdocsch2.tex| & sample include file \\
    |cdocspt3.tex| & sample part file \\
    |cdocspt4.tex| & sample part file \\
    |cdocsdrf.tex| & sample redirection file \\
    |cdocsfn1.tex| & sample redirection file \\
    |cdocsfn2.tex| & sample redirection file \\
    |childdoc.pdf| & manual
\end{tabular}
\end{center}
%
The distribution consists of the files
|README.txt|, |childdoc.ins| and |childdoc.dtx|.
%
\begin{itemize}
\item
Run (pdf)\LaTeX{} on |childdoc.dtx|
to compile the manual |childdoc.pdf| (this file).
\item
Run \LaTeX{} on |childdoc.ins| to create the definitions file |childdoc.def|
and the sample |cdocsamp.tex| with include files
|cdocsch1.tex|, |cdocsch2.tex|, |cdocspt3.tex|, |cdocspt4.tex|,
|cdocsdrf.tex|, |cdocsfn1.tex|, |cdocsfn2.tex|.
Then copy the file |childdoc.def| to an appropriate directory of your \LaTeX{}
distribution, e.g.\ \textit{texmf-root}|/tex/latex/childdoc|.
\end{itemize}

%%%%%%%%%%%%%%%%%%%%%%%%%%%%%%%%%%%%%%%%%%%%%%%%%%%%%%%%%%%%%%%%%%%%%%%%%%%%%%%%
\subsection{Related CTAN Packages}

There are several other packages which offer a similar functionality:
%
\begin{itemize}
\item
The packages
\href{http://ctan.org/pkg/docmute}{\textsf{docmute}},
\href{http://ctan.org/pkg/includex}{\textsf{includex}} and
\href{http://ctan.org/pkg/standalone}{\textsf{standalone}}
provide commands to include only the document body of
a child file thus allowing both files to be compiled individually.
\item
The packages \href{http://ctan.org/pkg/subdocs}{\textsf{subdocs}}
and \href{http://ctan.org/pkg/subfiles}{\textsf{subfiles}}
provide structures in which the main and child documents can be
encapsulated and allowing them to be compiled individually.
The inclusion mechanism is different from the conventional |\include|.
\item
The package \href{http://ctan.org/pkg/combine}{\textsf{combine}}
is an elaborate solution to combine several documents into one.
\end{itemize}
%
See also the CTAN topic \href{http://ctan.org/topic/subdocs}{\textsf{subdocs}}
for further related packages.
The present package differs from the above solutions in that
a document structure constructed with the conventional |\include| mechanism
just needs two extra commands at the top of every file
such that all constituent files can be compiled individually.

%%%%%%%%%%%%%%%%%%%%%%%%%%%%%%%%%%%%%%%%%%%%%%%%%%%%%%%%%%%%%%%%%%%%%%%%%%%%%%%%
%\subsection{Feature Suggestions}
%
%The following is a list of features which may be useful for future
%versions of this package:
%%
%\begin{itemize}
%\item
%\ldots
%\end{itemize}

%%%%%%%%%%%%%%%%%%%%%%%%%%%%%%%%%%%%%%%%%%%%%%%%%%%%%%%%%%%%%%%%%%%%%%%%%%%%%%%%
\subsection{Revision History}

%%%%%%%%%%%%%%%%%%%%%%%%%%%%%%%%%%%%%%%%
\paragraph{v2.0:} 2018/12/30

\begin{itemize}
\item
immediate forward processing
\item
added |\childdocby| mechanism
\item
manual restructured
\end{itemize}

%%%%%%%%%%%%%%%%%%%%%%%%%%%%%%%%%%%%%%%%
\paragraph{v1.6:} 2018/01/17

\begin{itemize}
\item
application for development of include files
\item
corrections to manual
\end{itemize}

%%%%%%%%%%%%%%%%%%%%%%%%%%%%%%%%%%%%%%%%
\paragraph{v1.5:} 2017/05/21

\begin{itemize}
\item
more complete structuring introduced
\item
|\childdocof| introduced
\item
|\childdoc| renamed to |\childdocmain|
\item
|\childredirect| renamed to |\childdocforward| and |\childdocforwardprefix|
and functionality expanded
\end{itemize}

%%%%%%%%%%%%%%%%%%%%%%%%%%%%%%%%%%%%%%%%
\paragraph{v1.0:} 2017/04/27

\begin{itemize}
\item
manual and install package
\item
first version published on CTAN
\end{itemize}

%%%%%%%%%%%%%%%%%%%%%%%%%%%%%%%%%%%%%%%%
\paragraph{v0.6:} 2017/04/26

\begin{itemize}
\item
redirection mechanism added
\end{itemize}

%%%%%%%%%%%%%%%%%%%%%%%%%%%%%%%%%%%%%%%%
\paragraph{v0.5:} 2017/04/26

\begin{itemize}
\item
functionality in definition file
\end{itemize}


%%%%%%%%%%%%%%%%%%%%%%%%%%%%%%%%%%%%%%%%%%%%%%%%%%%%%%%%%%%%%%%%%%%%%%%%%%%%%%%%
%%%%%%%%%%%%%%%%%%%%%%%%%%%%%%%%%%%%%%%%%%%%%%%%%%%%%%%%%%%%%%%%%%%%%%%%%%%%%%%%
%%%%%%%%%%%%%%%%%%%%%%%%%%%%%%%%%%%%%%%%%%%%%%%%%%%%%%%%%%%%%%%%%%%%%%%%%%%%%%%%
\appendix

\settowidth\MacroIndent{\rmfamily\scriptsize 000\ }

 \DocInput{childdoc.dtx}

\end{document}
%</driver>
% \fi
%
% %%%%%%%%%%%%%%%%%%%%%%%%%%%%%%%%%%%%%%%%%%%%%%%%%%%%%%%%%%%%%%%%%%%%%%%%%%%%%%
% %%%%%%%%%%%%%%%%%%%%%%%%%%%%%%%%%%%%%%%%%%%%%%%%%%%%%%%%%%%%%%%%%%%%%%%%%%%%%%
% \section{Sample}
%\iffalse
%<*samplemain>
%\fi
%
% The following presents a sample document
% with two chapters, two parts, a title page,
% a compile flag as well as three forwarding files to set the flag.
% It consists of eight |.tex| files:
% \begin{center}
% \begin{tabular}{ll}
% |cdocsamp.tex|&main file\\
% |cdocsch1.tex|&include file for chapter 1\\
% |cdocsch2.tex|&include file for chapter 2\\
% |cdocspt3.tex|&include file for part 3\\
% |cdocspt4.tex|&include file for part 4\\
% |cdocsdrf.tex|&forwarding file for main file in draft mode\\
% |cdocsfi1.tex|&forwarding file for final version of chapter 1\\
% |cdocsfi2.tex|&forwarding file for final version of chapter 2\\
% \end{tabular}
% \end{center}
% Each of the eight files can be compiled directly by the \LaTeX{} compiler.
%
% %%%%%%%%%%%%%%%%%%%%%%%%%%%%%%%%%%%%%%
% \paragraph{Main File.}
%
% The main file is called |cdocsamp.tex|.
%
% Load the \textsf{childdoc} definitions and
% declare the filename for the main document:
%    \begin{macrocode}
\input{childdoc.def}
\childdocmain{}
%    \end{macrocode}

% Optional override for |\version| flag:
%    \begin{macrocode}
%%\ifchilddoc\else\providecommand{\version}{draft}\fi
%    \end{macrocode}

% Define the default values for the |\version| flag
% (|final| for the main file and |draft| for childs):
%    \begin{macrocode}
\ifchilddoc
\providecommand{\version}{draft}
\else
\providecommand{\version}{final}
\fi
%    \end{macrocode}

% Load the standard document class:
%    \begin{macrocode}
\documentclass[12pt]{article}
%    \end{macrocode}

% Start the document body:
%    \begin{macrocode}
\begin{document}
%    \end{macrocode}

% Declare a title page.
% Print title, part of document being processed and version flag:
%    \begin{macrocode}
\addtocounter{page}{-1}
\begin{center}
{\LARGE\bfseries{}childdoc example\par}
\vspace{1cm}
\ifchilddoc
\ifchilddocmanual part\else chapter\fi:
`\childdocname' of `\childdocjob'\par
\else
main document: `\childdocjob'\par
\fi
version: \version\par
\end{center}
\newpage
%    \end{macrocode}

% Manually include selected file,
% otherwise process as usual:
%    \begin{macrocode}
\ifchilddocmanual
\section*{part `\childdocname'}
\input{\childdocname}
\else
%    \end{macrocode}

% Include the two chapters:
%    \begin{macrocode}
\include{cdocsch1}
\include{cdocsch2}
%    \end{macrocode}

% Include the two parts unless only chapters should be displayed:
%    \begin{macrocode}
\ifchilddoc\else
\section{part three}
\input{cdocspt3}
\section{part four}
\input{cdocspt4}
\fi
%    \end{macrocode}

% Process as usual until here:
%    \begin{macrocode}
\fi
%    \end{macrocode}

% End of document body:
%    \begin{macrocode}
\end{document}
%    \end{macrocode}
%\iffalse
%</samplemain>
%\fi
%
% %%%%%%%%%%%%%%%%%%%%%%%%%%%%%%%%%%%%%%
% \paragraph{Chapter Include Files.}
%
% The include files are called |cdocsch1.tex| and |cdocsch2.tex|.
%
%\iffalse
%<*samplechap1|samplechap2>
%\fi

% Optional override for |\version| flag:
%    \begin{macrocode}
%%\providecommand{\version}{final}
%    \end{macrocode}

% Include the main document:
%    \begin{macrocode}
\input{childdoc.def}
\childdocof{cdocsamp}
%    \end{macrocode}

%\iffalse
%</samplechap1|samplechap2>
%\fi
%
%\iffalse
%<*samplechap1>
%\fi
% Some text for chapter 1:
%    \begin{macrocode}
\section{one}
some text in chapter one
%    \end{macrocode}

%\iffalse
%</samplechap1>
%\fi
% Some text for chapter 2:
%\iffalse
%<*samplechap2>
%\fi
%    \begin{macrocode}
\section{two}
more text in chapter two
%    \end{macrocode}

%\iffalse
%</samplechap2>
%\fi
%
% %%%%%%%%%%%%%%%%%%%%%%%%%%%%%%%%%%%%%%
% \paragraph{Part Include Files.}
%
% The include files are called |cdocspt3.tex| and |cdocspt4.tex|.
%
%\iffalse
%<*samplepart3|samplepart4>
%\fi

% Optional override for |\version| flag:
%    \begin{macrocode}
%%\providecommand{\version}{final}
%    \end{macrocode}

% Include the main document:
%    \begin{macrocode}
\input{childdoc.def}
\childdocby{cdocsamp}
%    \end{macrocode}

%\iffalse
%</samplepart3|samplepart4>
%\fi
%
%\iffalse
%<*samplepart3>
%\fi
% Some text for part 3:
%    \begin{macrocode}
some text in part three
%    \end{macrocode}

%\iffalse
%</samplepart3>
%\fi
% Some text for part 4:
%\iffalse
%<*samplepart4>
%\fi
%    \begin{macrocode}
more text in part four
%    \end{macrocode}

%\iffalse
%</samplepart4>
%\fi
%
% %%%%%%%%%%%%%%%%%%%%%%%%%%%%%%%%%%%%%%
% \paragraph{Forwarding for a Complete Draft.}
%
% The following forwarding file |cdocsdrf.tex|
% compiles the main document in draft mode:
%\iffalse
%<*sampledraft>
%\fi
%    \begin{macrocode}
\def\version{draft}
\input{childdoc.def}
\childdocforward{cdocsamp}
%    \end{macrocode}

%\iffalse
%</sampledraft>
%\fi
%
% %%%%%%%%%%%%%%%%%%%%%%%%%%%%%%%%%%%%%%
% \paragraph{Forwarding for Final Version of the Chapters.}
%
% The following forwarding files |cdocsfn1.tex| and |cdocsfn2.tex|
% (with identical content)
% compile the final versions of the child documents
% |cdocsch1.tex| and |cdocsch2.tex|, respectively:
%\iffalse
%<*samplefinal>
%\fi
%    \begin{macrocode}
\def\version{final}
\input{childdoc.def}
\childdocforwardprefix[cdocsamp]{cdocsfn}{cdocsch}
%    \end{macrocode}

%\iffalse
%</samplefinal>
%\fi
%
% %%%%%%%%%%%%%%%%%%%%%%%%%%%%%%%%%%%%%%
% \paragraph{Command Line Processing.}
%
% The following three command lines generate the output files
% |cdocscld|, |cdocscl1| and |cdocscl2|
% which should be identical to
% |cdocsdrf|, |cdocsch1| and |cdocsfn2|, respectively:
% \begin{center}
% \begin{tabular}{l}
% |latex -jobname cdocscld \|\\
% |  "\def\version{draft}\input{childdoc.def}\childdocforward{cdocsamp}"|\\
% |latex -jobname cdocscl1 \|\\
% |  "\input{childdoc.def}\childdocforward[cdocsamp]{cdocsch1}"|\\
% |latex -jobname cdocscl2 \|\\
% |  "\def\version{final}\input{childdoc.def}\childdocforward{cdocsch2}"|
% \end{tabular}
% \end{center}
% Note that the trailing backslash on each first line
% merely continues the input to the second line
% (for convenient cut ant paste).
% Furthermore, the command |latex| can be replaced by any
% of its alternative versions such as |pdflatex|.
%
% %%%%%%%%%%%%%%%%%%%%%%%%%%%%%%%%%%%%%%%%%%%%%%%%%%%%%%%%%%%%%%%%%%%%%%%%%%%%%%
% %%%%%%%%%%%%%%%%%%%%%%%%%%%%%%%%%%%%%%%%%%%%%%%%%%%%%%%%%%%%%%%%%%%%%%%%%%%%%%
% \section{Implementation}
%\iffalse
%<*package>
%\fi
%
% This section describes the definitions file |childdoc.def|.

% The definitions cannot be loaded using |\usepackage| or |\RequirePackage|
% which has a mechanism to prevent loading a style file more than once.
% When loading the definitions by means of |\input|
% multiple instances have to be prevented manually:
%\iffalse
%This code needs to be before the `\ProvidesFile' directive
%which is defined at the beginning of this file.
%Therefore it is also placed there and commented out here.
%</package>
%<*discard>
%\fi
%    \begin{macrocode}
\ifdefined\childdocmain\endinput\fi
%    \end{macrocode}
%\iffalse
%</discard>
%<*package>
%\fi
%
% \macro{\ifchilddoc}
% \macro{\ifchilddocmanual}
% The conditional |\ifchilddoc| tells whether a
% child (true) or main (false) document is being compiled.
% The conditional |\ifchilddocmanual| tells whether
% the |\includeonly| mechanism is used (false) or
% the selection of child files must be performed manually (true).
% The definitions initialise to false:
%    \begin{macrocode}
\newif\ifchilddoc
\newif\ifchilddocmanual
%    \end{macrocode}

% \macro{\childdocname}
% \macro{\childdocjob}
% The macro |\childdocname| stores the name of the main document
% to be compiled. The macro |\childdocjob| stores the name of
% the document on which the \LaTeX{} compiler was originally invoked.
% The content of |\jobname| cannot be compared
% to filenames specified in the source due to different catcodes.
% The following code rescans |\jobname|, stores the result
% in |\childdocname| and saves a copy in |\childdocjob|:
%    \begin{macrocode}
\edef\childdocname{\scantokens\expandafter{\jobname\noexpand}}
\let\childdocjob\childdocname
%    \end{macrocode}

% \macro{\childdocdisable}
% The macro |\childdocdisable| prevents the main file
% from being processed more than once.
% At this stage, the main document command |\childdocmain|
% is assumed to be called once again where it should do nothing.
% Any subsequent call to it should prevent
% a secondary processing of the main document
% It overwrites the forwarding commands
% |\childdocof| and |\childdocforward|
% with empty macros to prevent further inclusions of the main document:
%    \begin{macrocode}
\newcommand{\childdocdisable}
{
  \renewcommand{\childdocmain}[1]{\renewcommand{\childdocmain}[1]{\endinput}}
  \renewcommand{\childdocof}[1]{}
  \renewcommand{\childdocby}[2][]{}
  \renewcommand{\childdocforward}[2][]{}
  \renewcommand{\childdocdisable}{}
}
%    \end{macrocode}

% \macro{\childdocmain}
% The macro |\childdocmain| is to be called at the top of the main file
% with nothing or the main filename (without extension) as argument.
% First, it breaks loops.
% If the argument is not empty and does not match |\childdocname|
% (which is set by the first inclusion of |childdoc.def|),
% |\ifchilddoc| is set to true, |\includeonly| is applied to the child file
% and |\jobname| is set to the main file
% (for proper handling of |.aux| files):
%    \begin{macrocode}
\newcommand{\childdocmain}[1]
{
  \childdocdisable\childdocmain{}
  \if?#1?\else
    \begingroup
      \def\childdoctmp{#1}
      \ifx\childdoctmp\childdocname
        \def\childdoctmp{}
      \else
        \def\childdoctmp
        {
          \childdoctrue
          \includeonly{\childdocname}
          \def\childdocjob{#1}
          \def\jobname{#1}
        }
      \fi
      \expandafter
    \endgroup
    \childdoctmp
  \fi
}
%    \end{macrocode}

% \macro{\childdocof}
% The command |\childdocof| redirects
% compilation to the main file |#1|.
%    \begin{macrocode}
\newcommand{\childdocof}[1]
{
  \childdocdisable
  \childdoctrue
  \includeonly{\childdocname}
  \def\jobname{#1}
  \def\childdocjob{#1}
  \input{#1}
}
%    \end{macrocode}

% \macro{\childdocby}
% The command |\childdocby| ....
%    \begin{macrocode}
\newcommand{\childdocby}[2][]
{
  \childdocdisable
  \childdoctrue
  \childdocmanualtrue
  \if?#1?\else
    \def\jobname{#2}
  \fi
  \def\childdocjob{#2}
  \input{#2}
  \endinput
}
%    \end{macrocode}

% \macro{\childdocforward}
% The command |\childdocforward| redirects
% compilation to the main file or
% (if the optional argument is given) a child file.
% Parameters are set as if the main file
% or a child file starting with |\childdocof| was compiled.
% Then compilation is handed over to the main file:
%    \begin{macrocode}
\newcommand{\childdocforward}[2][]
{
  \begingroup
    \if?#1?
      \def\childdoctmp
      {
        \def\childdocname{#2}
        \def\childdocjob{#2}
        \def\jobname{#2}
        \input{#2}
        \endinput
      }
    \else
      \def\childdoctmp
      {
        \childdocdisable
        \def\childdocname{#2}
        \childdoctrue
        \includeonly{#2}
        \def\childdocjob{#1}
        \def\jobname{#1}
        \input{#1}
        \endinput
      }
    \fi
    \expandafter
  \endgroup
  \childdoctmp
}
%    \end{macrocode}

% \macro{\childdocforwardprefix}
% The command |\childdocforwardprefix| redirects
% compilation to the main or a child file by means of a pattern.
% The prefix |#1| in the current filename is replaced by |#2|
% and the suffix of the current filename is kept
% (it is assumed that the filename does not contain the substring `|~~~|'
% which is used as a delimiter).
% Compilation is handed over to the new file by |\childdocforward|:
%    \begin{macrocode}
\newcommand{\childdocforwardprefix}[3][]
{
  \begingroup
    \def\childdocextract #2##1~~~{\def\childdoctmp{\childdocforward[#1]{#3##1}}}
    \expandafter\childdocextract\childdocname~~~
    \expandafter
  \endgroup
  \childdoctmp
}
%    \end{macrocode}

% \macro{\childdoc}
% The deprecated macro |\childdoc| is a legacy version of |\childdocmain|:
%    \begin{macrocode}
\newcommand{\childdoc}{\childdocmain}
%    \end{macrocode}

% \macro{\childdocredirect}
% The deprecated macro |\childdocredirect| is a legacy version
% of |\childdocforward| and |\childdocforwardprefix|:
%    \begin{macrocode}
\newcommand{\childdocredirect}[2][]
{
  \begingroup
    \if?#1?
      \def\childdoctmp{\childdocforward{#2}}
    \else
      \def\childdoctmp{\childdocforwardprefix{#1}{#2}}
    \fi
    \expandafter
  \endgroup
  \childdoctmp
}
%    \end{macrocode}

%\iffalse
%</package>
%\fi
%
\endinput
|\\
|\childdocby{|\textit{main}|}|\\
\end{tabular}
\end{center}
%
Both forms have slightly different effects as described above.
The main file is prepared as usual, see \secref{sec:include}.

%%%%%%%%%%%%%%%%%%%%%%%%%%%%%%%%%%%%%%%%%%%%%%%%%%%%%%%%%%%%%%%%%%%%%%%%%%%%%%%%
\subsection{Legacy Detection}
\label{sec:detection}

The directive |\childdocmain| in the main file can detect
whether the complete document or merely a child is to be compiled
even without using the directive |\childdocof|.
This method is deprecated because it is less robust
and there is no compelling reason to use it;
it is merely provided for backward compatibility
and it may be removed in future versions.

If the detection mechanism is to be used,
it is mandatory to correctly specify
the filename of the main file as the argument of |\childdocmain|:
%
\begin{center}
\begin{tabular}{l}
|% \iffalse
%
% childdoc.dtx Copyright (C) 2017-2018 Niklas Beisert
%
% This work may be distributed and/or modified under the
% conditions of the LaTeX Project Public License, either version 1.3
% of this license or (at your option) any later version.
% The latest version of this license is in
%   http://www.latex-project.org/lppl.txt
% and version 1.3 or later is part of all distributions of LaTeX
% version 2005/12/01 or later.
%
% This work has the LPPL maintenance status `maintained'.
%
% The Current Maintainer of this work is Niklas Beisert.
%
% This work consists of the files childdoc.dtx and childdoc.ins
% and the derived files childdoc.def and cdocsamp.tex with
% cdocsch1.tex, cdocsch2.tex, cdocsdrf.tex, cdocsfn1.tex, cdocsfn2.tex.
%
%<package>\ifdefined\childdocmain\endinput\fi
%<package>\ProvidesFile{childdoc.def}[2018/12/30 v2.0 child document driver]
%<samplemain>\ProvidesFile{cdocsamp.tex}[2018/12/30 v2.0 sample for childdoc]
%<*driver>
%\ProvidesFile{childdoc.drv}[2018/12/30 v2.0 childdoc reference manual file]
\PassOptionsToClass{10pt,a4paper}{article}
\documentclass{ltxdoc}

\usepackage[margin=35mm]{geometry}
\usepackage{hyperref}
\usepackage{hyperxmp}
\usepackage[usenames]{color}

\hypersetup{colorlinks=true}
\hypersetup{pdfstartview=FitH}
\hypersetup{pdfpagemode=UseNone}
\hypersetup{pdfsource={}}
\hypersetup{pdflang={en-UK}}
\hypersetup{pdfcopyright={Copyright 2017-2018 Niklas Beisert.
  This work may be distributed and/or modified under the
  conditions of the LaTeX Project Public License, either version 1.3
  of this license or (at your option) any later version.}}
\hypersetup{pdflicenseurl={http://www.latex-project.org/lppl.txt}}
\hypersetup{pdfcontactaddress={ETH Zurich, ITP, HIT K,
  Wolfgang-Pauli-Strasse 27}}
\hypersetup{pdfcontactpostcode={8093}}
\hypersetup{pdfcontactcity={Zurich}}
\hypersetup{pdfcontactcountry={Switzerland}}
\hypersetup{pdfcontactemail={nbeisert@itp.phys.ethz.ch}}
\hypersetup{pdfcontacturl={http://people.phys.ethz.ch/\xmptilde nbeisert/}}

\newcommand{\secref}[1]{\hyperref[#1]{section \ref*{#1}}}

\parskip1ex
\parindent0pt
\let\olditemize\itemize
\def\itemize{\olditemize\parskip0pt}

\begin{document}

\title{The \textsf{childdoc} Package}
\hypersetup{pdftitle={The childdoc Package}}
\author{Niklas Beisert\\[2ex]
  Institut f\"ur Theoretische Physik\\
  Eidgen\"ossische Technische Hochschule Z\"urich\\
  Wolfgang-Pauli-Strasse 27, 8093 Z\"urich, Switzerland\\[1ex]
  \href{mailto:nbeisert@itp.phys.ethz.ch}
  {\texttt{nbeisert@itp.phys.ethz.ch}}}
\hypersetup{pdfauthor={Niklas Beisert}}
\hypersetup{pdfsubject={Manual for the LaTeX2e Package childdoc}}
\date{30 December 2018, \textsf{v2.0}}
\maketitle

\begin{abstract}\noindent
\textsf{childdoc} is a \LaTeXe{} package
that enables the direct compilation
of document sections included by |\include|
to individual files.
\end{abstract}

\begingroup
\parskip0ex
\tableofcontents
\endgroup

%%%%%%%%%%%%%%%%%%%%%%%%%%%%%%%%%%%%%%%%%%%%%%%%%%%%%%%%%%%%%%%%%%%%%%%%%%%%%%%%
%%%%%%%%%%%%%%%%%%%%%%%%%%%%%%%%%%%%%%%%%%%%%%%%%%%%%%%%%%%%%%%%%%%%%%%%%%%%%%%%
\section{Introduction}

\LaTeX{} provides a mechanism to structure a large document (such as a book)
into a main file and several child files (containing the chapters)
using the |\include| command.
This mechanism is beneficial for documents
which span hundreds of pages in order to
make the source file(s) more manageable.
Moreover, compilation can be restricted to
selected child files by means of the |\includeonly| command.
The latter feature can be used to reduce the compilation time while editing
(this was significantly more useful in the earlier days of \LaTeX{})
or to generate a smaller document which is easier to navigate.
Another application of |\includeonly| is to generate
documents consisting of selected parts of the complete document.

However, there are a few drawbacks of the plain |\include| mechanism:
\begin{itemize}
\item
The child files cannot be compiled on their own,
they can only be compiled via the main file.
A naive editing environment
(such as a text editor with an option
to have the current file processed by \LaTeX)
may require one to switch to the main file before compiling;
attempting to compile the child file produces errors.
\item
The main file must be modified (each time)
to adjust the |\includeonly| command
to the present needs. This easily leaves the main file in a messy state.
\item
The generated document will always carry the filename
of the main document. This is inconvenient if
several child files are to be compiled and
to be kept for distribution.
\end{itemize}

The present package provides a simple interface
to make child files individually compilable by \LaTeX{}.
Compiling a child file then has the same effect as compiling
the main file with an |\includeonly| command
to select the appropriate child.
Moreover the generated document will carry the name of the child
rather than the main file.
This resolves all three above issues.

This feature is meant to make the editing of books,
thesis documents and lecture notes somewhat more convenient.
However, the package can also be used efficiently for
composing a series of documents (such as exercise sheets)
which are typically distributed individually.
It then assists the author in generating the individual documents
(potentially in different versions)
as well as a document containing the collected series.
Another application is in developing style files
or other kinds of included material
where compilation of the style file could redirect
to a sample or test file.

%%%%%%%%%%%%%%%%%%%%%%%%%%%%%%%%%%%%%%%%%%%%%%%%%%%%%%%%%%%%%%%%%%%%%%%%%%%%%%%%
%%%%%%%%%%%%%%%%%%%%%%%%%%%%%%%%%%%%%%%%%%%%%%%%%%%%%%%%%%%%%%%%%%%%%%%%%%%%%%%%
\section{Usage}

First of all, the package \textsf{childdoc} is \emph{not} a standard
\LaTeXe{} |.sty| style file! Therefore it needs to be invoked in
a non-standard way.

%%%%%%%%%%%%%%%%%%%%%%%%%%%%%%%%%%%%%%%%%%%%%%%%%%%%%%%%%%%%%%%%%%%%%%%%%%%%%%%%
\subsection{Included Files}
\label{sec:include}

%%%%%%%%%%%%%%%%%%%%%%%%%%%%%%%%%%%%%%%%
\DescribeMacro{\childdocmain}
To use the package, add the commands
\begin{center}
\begin{tabular}{l}
|\input{childdoc.def}|\\
|\childdocmain{}|\\
\end{tabular}
\end{center}
at the very top of the main \LaTeX{} file,
in particular \emph{before} the |\documentclass| statement!
The argument of |\childdocmain| should be left empty
(but it must be present).

%%%%%%%%%%%%%%%%%%%%%%%%%%%%%%%%%%%%%%%%
\DescribeMacro{\childdocof}
Furthermore, add the commands
\begin{center}
\begin{tabular}{l}
|\input{childdoc.def}|\\
|\childdocof{|\textit{main}|}|\\
\end{tabular}
\end{center}
at the top of every child file \textit{child}
which is included by |\include{|\textit{child}|}|
from within the main file
(or at least for those files to be compiled individually).
The argument \textit{main} must be the filename of the main file.

There are a couple of
considerations in setting up the main and child documents:

%%%%%%%%%%%%%%%%%%%%%%%%%%%%%%%%%%%%%%%%
\paragraph{Restrictions.}

Please note the following restrictions:
\begin{itemize}
\item
|\childdocmain| must be called with one argument \textit{main}
to ensure compatibility with earlier version of the package.
It must either be empty (|\childdocmain{}|)
or precisely match the filename of the main file in which it is specified.
See \secref{sec:detection} for further information.
\item
The filename \textit{main} must be specified without the |.tex| extension.
\item
The filename \textit{main} is case sensitive
(even in case-insensitive file systems)
due to internal string comparison.
\item
The argument \textit{main} should be fully expanded, it cannot be a macro.
\item
Subdirectories and special characters should be avoided in filenames.
\item
The command |\childdocmain{|\textit{main}|}| must be followed by a whitespace.
It should not be followed immediately by another command
or by a comment mark `|%|'.
This is because the \TeX{} parser reads the token immediately following
the argument of |\childdocmain| and puts it
at the beginning of every child section;
however, a white\-space is ignored.
\end{itemize}

%%%%%%%%%%%%%%%%%%%%%%%%%%%%%%%%%%%%%%%%
\paragraph{Content of Main File.}

It is advisable to place all content in the child files included by |\include|.
Any output contained in the main file will appear in all child documents
unless suppressed manually;
it cannot be suppressed automatically by the |\includeonly| directive
and thus should normally be avoided.
A method to include some content in the main file
by means of conditional processing is described in \secref{sec:conditional}.

%%%%%%%%%%%%%%%%%%%%%%%%%%%%%%%%%%%%%%%%
\paragraph{Page Numbering.}

When only a part of the document is compiled,
the appropriate numbering of pages
(as well as other status parameters)
is determined from the |.aux| files.
The latter contain information from previous passes.
However this information needs to propagate through
all intermediate child documents.
Therefore the page numbering in child documents may well
be inconsistent until the complete document is compiled at least once.

A useful (if unconventional) way to always ensure a consistent
page numbering is to restart the numbering in each child document
and denote the pages by `\textit{child}|.|\textit{page}'
where \textit{child} represents the chapter/section number of the child file.
This can be achieved by the command
|\numberwithin{page}{|\textit{child}|}|
of the \textsf{amsmath} package
where \textit{child} can be |chapter| or |section|
depending on the chosen structuring.
Alternatively, one can modify the macro |\thepage| appropriately
and reset the counter |page| at the start of each child file.

%%%%%%%%%%%%%%%%%%%%%%%%%%%%%%%%%%%%%%%%%%%%%%%%%%%%%%%%%%%%%%%%%%%%%%%%%%%%%%%%
\subsection{Conditional Processing}
\label{sec:conditional}

The package provides a mechanism to compile different versions
of a document. To customise the versions further some conditional processing
can come in handy to distinguish which version is being compiled.
The package provides two macros to describe the compilation context:

%%%%%%%%%%%%%%%%%%%%%%%%%%%%%%%%%%%%%%%%
\DescribeMacro{\ifchilddoc}
The conditional |\ifchilddoc| distinguishes between the compilation of
child documents and the main document:
%
\begin{center}
|\ifchilddoc |\textit{child-code}| |[|\||else |\textit{main-code}]| \||fi|
\end{center}

%%%%%%%%%%%%%%%%%%%%%%%%%%%%%%%%%%%%%%%%
\DescribeMacro{\childdocname}
\DescribeMacro{\childdocjob}
The macro |\childdocname| contains the filename (without extension)
of the main or child file being processed.
Note that |\childdocjob| will always contain the name of the main file.

%%%%%%%%%%%%%%%%%%%%%%%%%%%%%%%%%%%%%%%%
\paragraph{Title Page.}

Conditional processing can be used to include a title or banner page
in the main document when proper precautions are taken.
Importantly, the code in the main file should ensure that the page counter
(as well as other status parameters which are stored in the |.aux| files)
takes the same value after the conditional processing.
Otherwise the page numbers may take divergent values
depending on which part is compiled.

For example, a title page could be declared by:
%
\begin{center}
\begin{tabular}{l}
|\ifchilddoc\||else|\\
|\addtocounter{page}{-1}|\\
\textit{code for title page}\\
|\newpage|\\
|\||fi|
\end{tabular}
\end{center}
%
A banner page for the child documents can be generated by:
%
\begin{center}
\begin{tabular}{l}
|\ifchilddoc|\\
|\addtocounter{page}{-1}|\\
\textit{code for banner page}\\
|\newpage|\\
|\||fi|
\end{tabular}
\end{center}
%
Here one could write a message such as:
\begin{center}
|This is the part \childdocname{} of \childdocjob{}.|
\end{center}

%%%%%%%%%%%%%%%%%%%%%%%%%%%%%%%%%%%%%%%%%%%%%%%%%%%%%%%%%%%%%%%%%%%%%%%%%%%%%%%%
\subsection{Flags}
\label{sec:flags}

The package makes it easy to generate different versions
of the main or child documents.
To this end compilation flags can be defined
and assigned different default values.
They will be particularly useful in conjunction
with the forwarding mechanism described in \secref{sec:forward}.

For example, it may be useful to have a flag |\version|
which can be set to |draft| or |final|.
The document source will contain some conditional code
depending on the value of |\version|.
Suppose further, the flag should default to |final| for the main file
and to |draft| for child files
which is a natural assignment for editing the document.
This is achieved by placing the following code
in the preamble of the main document
(below the |\childdocmain| directive):
%
\begin{center}
\begin{tabular}{l}
|\ifchilddoc|\\
|\providecommand{\version}{draft}|\\
|\||else|\\
|\providecommand{\version}{final}|\\
|\||fi|
\end{tabular}
\end{center}
%
The definition by |\providecommand| makes sure
that previous definitions are not overwritten.
Further statements |\providecommand{\version}{...}|
can thus be added before the above code to override it.

For the main file, one might add a line
(between |\childdocmain| and the above block)
%
\begin{center}
|%\ifchilddoc\||else\providecommand{\version}{draft}\||fi|
\end{center}
%
which can be uncommented to produce a draft version.
Likewise one can add a line to the very top of a child file
(above the |\childdocof{|\textit{main}|}| directive)
%
\begin{center}
|%\providecommand{\version}{final}|
\end{center}
%
which can be uncommented to produce the final version of this child document.

%%%%%%%%%%%%%%%%%%%%%%%%%%%%%%%%%%%%%%%%%%%%%%%%%%%%%%%%%%%%%%%%%%%%%%%%%%%%%%%%
\subsection{Forwarding}
\label{sec:forward}

Different versions of the main or child documents
using compilation flags as described in \secref{sec:flags}
can be (permanently) stored in different files
for convenient compilation, viewing and distribution.
To this end, the package defines a command
to pass on compilation to a different file:

%%%%%%%%%%%%%%%%%%%%%%%%%%%%%%%%%%%%%%%%
\DescribeMacro{\childdocforward}
The command |\childdocforward| redirects processing to
another source file:
%
\begin{center}
\begin{tabular}{l}
|\input{childdoc.def}|\\
|\childdocforward[|\textit{main}|]{|\textit{dest}|}|\\
\end{tabular}
\end{center}
%
The argument \textit{dest} is the destination file
(without extension).
It should be the main file or one of the child files.
Note that further \textsf{childdoc} directives
such as |\childdocof| and |\childdocforward|
in the indicated file will be processed in this form.
The optional argument \textit{main}
passes on directly to the main file \textit{main}
while pretending to compile the child \textit{dest}.
This form behaves as if \textit{dest}
issues |\childdocof{|\textit{main}|}| right away,
and no further \textsf{childdoc} directives will be processed.

%%%%%%%%%%%%%%%%%%%%%%%%%%%%%%%%%%%%%%%%
\DescribeMacro{\...prefix}
In the alternative form |\childdocforwardprefix|,
%
\begin{center}
\begin{tabular}{l}
|\input{childdoc.def}|\\
|\childdocforwardprefix[|\textit{main}|]{|\textit{prefix}|}{|\textit{dest}|}|
\end{tabular}
\end{center}
%
the destination file is determined by a pattern
depending on the current file:
To make this work, the current file must be called
`{\textit{prefix}\hspace{0.2em}\textit{suffix}}'
with \textit{prefix} matching precisely the argument.
Processing is then passed on to the file
`{\textit{dest}\hspace{0.2em}\textit{suffix}}'.
Surely, the same effect is achieved by
directly specifying the
argument `{\textit{dest}\hspace{0.2em}\textit{suffix}}'
in the first form.
However, that requires to set up a different file
for each child. With the alternative form of the command
all these files can have exactly the same content
which simplifies setting them up and maintaining them.

For example, the following file |draft.tex|
with a compilation flag |\version| as described in \secref{sec:flags}
compiles the main document as a draft:
%
\begin{center}
\begin{tabular}{l}
|\def\version{draft}|\\
|\input{childdoc.def}|\\
|\childdocforward{|\textit{main}|}|
\end{tabular}
\end{center}
%
Likewise, the following files |final|\textit{nn}|.tex|
compile the final version of the child document
|child|\textit{nn}|.tex|:
%
\begin{center}
\begin{tabular}{l}
|\def\version{final}|\\
|\input{childdoc.def}|\\
|\childdocforwardprefix{final}{child}|
\end{tabular}
\end{center}
%

Note that when several versions of a main file and/or of each child file
are to be generated, it may be convenient to set up a |Makefile| or
shell script to automatise the process.

%%%%%%%%%%%%%%%%%%%%%%%%%%%%%%%%%%%%%%%%%%%%%%%%%%%%%%%%%%%%%%%%%%%%%%%%%%%%%%%%
\subsection{Command Line Processing}
\label{sec:commandline}

The effect of redirection files can also be achieved by invoking
the \LaTeX{} compiler with a more elaborate command line.
Most conveniently this should be done as part
of a shell script or a |Makefile|.

When using \textsf{childdoc} in the main file, the following
command lines effectively perform a redirection
(note that depending on the shell being used,
backslashes may have to be doubled: `|\|' $\to$ `|\\|'):
%
\begin{center}
|... -jobname "|\textit{target}|" |\\|"|[\textit{flags}]%
|\input{childdoc.def}\childdocforward[|\textit{main}|]{|\textit{dest}|}"|
\end{center}
%
Here \textit{target} is the name of the output file,
\textit{main} is the name of the main file
and \textit{dest} is the name of the main or child file to be processed
(all filenames without extensions).
The optional argument \textit{main} can be omitted
if \textit{main} matches \textit{dest}.
Optionally, compilation \textit{flags} can be defined via |\def| commands.
This command line makes the \TeX{} engine believe
it is compiling the file \textit{target}
whose content is specified as the latter parameter.
The provided code then forwards the processing to
\textit{main} or \textit{dest} as described in \secref{sec:forward}.

%%%%%%%%%%%%%%%%%%%%%%%%%%%%%%%%%%%%%%%%%%%%%%%%%%%%%%%%%%%%%%%%%%%%%%%%%%%%%%%%
\subsection{Include by Input}
\label{sec:input}

Including child documents by |\include| has some restrictions by design.
Most notably, the content of a child document always occupies
its own set of pages; pages cannot be shared between child documents.
Usually, this behaviour makes perfect sense
because each child document contain an essential part of the document.
However, in some situations it may be desirable to compose
a document from a collection of parts
without having mandatory page breaks between then.
For this case, the package
provides a mechanism to include parts
by |\input| which can also be processed individually.
However, by construction this mechanism
requires manual handling of the content to be output.

%%%%%%%%%%%%%%%%%%%%%%%%%%%%%%%%%%%%%%%%
\DescribeMacro{\ifchilddocmanual}
The main file should be prepared as usual, see \secref{sec:include}.
However, the document body must make a distinction
between processing of an individual part and of the main document, e.g.:
%
\begin{center}
\begin{tabular}{l}
|\ifchilddocmanual|\\
|\input{\childdocname}|\\
|\||else|\\
\textit{document body with }|\input{|\textit{part}|}|\\
|\||fi|
\end{tabular}
\end{center}
%
The conditional |\ifchilddocmanual| is true whenever
a part to be included by |\input| is being compiled,
and the name of the part is stored in |\childdocname|.

%%%%%%%%%%%%%%%%%%%%%%%%%%%%%%%%%%%%%%%%
\DescribeMacro{\childdocby}
Each part to be included by |\input| should start with:
%
\begin{center}
\begin{tabular}{l}
|\input{childdoc.def}|\\
|\childdocby{|\textit{main}|}|\\
\end{tabular}
\end{center}
%
The directive |\childdocby| is similar to |\childdocof|
described in \secref{sec:include},
but the subsequent selection of content must be done manually.
To that end, both |\ifchilddoc| and |\ifchilddocmanual|
will be true upon processing of a part,
and the name of the part is stored in |\childdocname|.
Note that |\jobname| will be set to the filename of the current part
so that each part receives an individual |.aux| file
that does not interfere with the |.aux| file(s) of the main document.
This behaviour can be altered by the alternative form
|\childdocby[*]{|\textit{main}|}| (with a non-empty optional argument)
which uses the |.aux| file of the main document
by setting |\jobname| to \textit{main}.

%%%%%%%%%%%%%%%%%%%%%%%%%%%%%%%%%%%%%%%%%%%%%%%%%%%%%%%%%%%%%%%%%%%%%%%%%%%%%%%%
\subsection{Driver Development}
\label{sec:driver}

The \textsf{childdoc} mechanism can also be use for the development
of definition files such as \LaTeX{} styles or classes.
This case differs from the above setup with multiple parts
included by |\include| in that no |\includeonly| should be invoked.
This can be achieved by starting the include file
(before |\ProvidesPackage|) with:
%
\begin{center}
\begin{tabular}{l}
|\input{childdoc.def}|\\
|\childdocforward{|\textit{main}|}|\\
\end{tabular}
\end{center}
%
or alternatively with:
%
\begin{center}
\begin{tabular}{l}
|\input{childdoc.def}|\\
|\childdocby{|\textit{main}|}|\\
\end{tabular}
\end{center}
%
Both forms have slightly different effects as described above.
The main file is prepared as usual, see \secref{sec:include}.

%%%%%%%%%%%%%%%%%%%%%%%%%%%%%%%%%%%%%%%%%%%%%%%%%%%%%%%%%%%%%%%%%%%%%%%%%%%%%%%%
\subsection{Legacy Detection}
\label{sec:detection}

The directive |\childdocmain| in the main file can detect
whether the complete document or merely a child is to be compiled
even without using the directive |\childdocof|.
This method is deprecated because it is less robust
and there is no compelling reason to use it;
it is merely provided for backward compatibility
and it may be removed in future versions.

If the detection mechanism is to be used,
it is mandatory to correctly specify
the filename of the main file as the argument of |\childdocmain|:
%
\begin{center}
\begin{tabular}{l}
|\input{childdoc.def}|\\
|\childdocmain{|\textit{main}|}|\\
\end{tabular}
\end{center}
%
If |\jobname| does not match the argument \textit{main} of |\childdocmain|,
it is assumed that |\jobname| points to the child file to be compiled.
When using |\childdocmain| with the main file specified as argument,
it suffices to start a child file
with just |\input{|\textit{main}|}|
without loading of the package and using |\childdocof|.
If instead all processing is done
with the appropriate \textsf{childdoc} directives,
the argument of \textit{main} of |\childdocmain| can be empty.

An alternative version of the command line processing described
in \secref{sec:commandline} using the detection mechanism reads:
%
\begin{center}
|... -jobname "|\textit{target}|" "|[\textit{flags}]%
[|\def\jobname{|\textit{dest}|}|]|\input{|\textit{main}|}"|
\end{center}

%%%%%%%%%%%%%%%%%%%%%%%%%%%%%%%%%%%%%%%%%%%%%%%%%%%%%%%%%%%%%%%%%%%%%%%%%%%%%%%%
\subsection{Manual Code}
\label{sec:manual}

In case one cannot be certain whether the definitions file |childdoc.def|
is installed on the target \TeX{} distribution
and one prefers not to ship it,
it is conceivable to paste a few relevant commands into the sources.

To that end, drop all statements |\input{childdoc.def}|
and perform the replacements as outlined below.
Instead of |\childdocmain{|\textit{main}|}| add the following code
to the top of the main file:
%
\begin{center}
\begin{tabular}{l}
|\||ifdefined\childdocname\endinput\||fi\newif\ifchilddoc|\\
|\edef\childdocname{\scantokens\expandafter{\jobname\noexpand}}|\\
|\def\childdocmain{|\textit{main}|}\||ifx\childdocmain\childdocname\||else|\\
|\childdoctrue\includeonly{\childdocname}\let\jobname\childdocmain\||fi|\\
\end{tabular}
\end{center}
%
Instead of |\childdocof{|\textit{main}|}| just include the main file
at the top of each child file:
%
\begin{center}
|\input{|\textit{main}|}|
\end{center}
%
A simple redirection |\childdocforward{|\textit{dest}|}| is achieved by:
%
\begin{center}
|\def\jobname{|\textit{dest}|}\input{\jobname}|
\end{center}
%
The redirection with prefix
|\childdocforwardprefix[|\textit{prefix}|]{|\textit{dest}|}|
is accomplished by:
%
\begin{center}
\begin{tabular}{l}
|{\edef\jobname{\scantokens\expandafter{\jobname\noexpand}}|\\
|\def\redirectjob |\textit{prefix}|#1~~~{\gdef\jobname{|\textit{dest}|#1}}|\\
|\expandafter\redirectjob\jobname~~~}\input{\jobname}|
\end{tabular}
\end{center}

In an alternative approach,
child documents can be compiled by a specific command line
without additional code or specific definitions:
%
\begin{center}
|... -jobname "|\textit{target}|" "|[\textit{flags}]%
|\includeonly{|\textit{dest}|}\input{|\textit{main}|}"|
\end{center}
%

%%%%%%%%%%%%%%%%%%%%%%%%%%%%%%%%%%%%%%%%%%%%%%%%%%%%%%%%%%%%%%%%%%%%%%%%%%%%%%%%
%%%%%%%%%%%%%%%%%%%%%%%%%%%%%%%%%%%%%%%%%%%%%%%%%%%%%%%%%%%%%%%%%%%%%%%%%%%%%%%%
\section{Information}

%%%%%%%%%%%%%%%%%%%%%%%%%%%%%%%%%%%%%%%%%%%%%%%%%%%%%%%%%%%%%%%%%%%%%%%%%%%%%%%%
\subsection{Copyright}

Copyright \copyright{} 2017--2018 Niklas Beisert

This work may be distributed and/or modified under the
conditions of the \LaTeX{} Project Public License, either version 1.3
of this license or (at your option) any later version.
The latest version of this license is in
  \url{http://www.latex-project.org/lppl.txt}
and version 1.3 or later is part of all distributions of \LaTeX{}
version 2005/12/01 or later.

This work has the LPPL maintenance status `maintained'.

The Current Maintainer of this work is Niklas Beisert.

This work consists of the files |README.txt|, |childdoc.ins| and |childdoc.dtx|
as well as the derived files |childdoc.def|, |cdocsamp.tex|
with |cdocsch1.tex|, |cdocsch2.tex|, |cdocspt3.tex|, |cdocspt4.tex|,
|cdocsdrf.tex|, |cdocsfn1.tex|, |cdocsfn2.tex|
as well as |childdoc.pdf|.

%%%%%%%%%%%%%%%%%%%%%%%%%%%%%%%%%%%%%%%%%%%%%%%%%%%%%%%%%%%%%%%%%%%%%%%%%%%%%%%%
\subsection{Files and Installation}

The package consists of the files:
%
\begin{center}
\begin{tabular}{ll}
    |README.txt|   & readme file \\
    |childdoc.ins| & installation file \\
    |childdoc.dtx| & source file \\
    |childdoc.def| & definition file \\
    |cdocsamp.tex| & sample main file \\
    |cdocsch1.tex| & sample include file \\
    |cdocsch2.tex| & sample include file \\
    |cdocspt3.tex| & sample part file \\
    |cdocspt4.tex| & sample part file \\
    |cdocsdrf.tex| & sample redirection file \\
    |cdocsfn1.tex| & sample redirection file \\
    |cdocsfn2.tex| & sample redirection file \\
    |childdoc.pdf| & manual
\end{tabular}
\end{center}
%
The distribution consists of the files
|README.txt|, |childdoc.ins| and |childdoc.dtx|.
%
\begin{itemize}
\item
Run (pdf)\LaTeX{} on |childdoc.dtx|
to compile the manual |childdoc.pdf| (this file).
\item
Run \LaTeX{} on |childdoc.ins| to create the definitions file |childdoc.def|
and the sample |cdocsamp.tex| with include files
|cdocsch1.tex|, |cdocsch2.tex|, |cdocspt3.tex|, |cdocspt4.tex|,
|cdocsdrf.tex|, |cdocsfn1.tex|, |cdocsfn2.tex|.
Then copy the file |childdoc.def| to an appropriate directory of your \LaTeX{}
distribution, e.g.\ \textit{texmf-root}|/tex/latex/childdoc|.
\end{itemize}

%%%%%%%%%%%%%%%%%%%%%%%%%%%%%%%%%%%%%%%%%%%%%%%%%%%%%%%%%%%%%%%%%%%%%%%%%%%%%%%%
\subsection{Related CTAN Packages}

There are several other packages which offer a similar functionality:
%
\begin{itemize}
\item
The packages
\href{http://ctan.org/pkg/docmute}{\textsf{docmute}},
\href{http://ctan.org/pkg/includex}{\textsf{includex}} and
\href{http://ctan.org/pkg/standalone}{\textsf{standalone}}
provide commands to include only the document body of
a child file thus allowing both files to be compiled individually.
\item
The packages \href{http://ctan.org/pkg/subdocs}{\textsf{subdocs}}
and \href{http://ctan.org/pkg/subfiles}{\textsf{subfiles}}
provide structures in which the main and child documents can be
encapsulated and allowing them to be compiled individually.
The inclusion mechanism is different from the conventional |\include|.
\item
The package \href{http://ctan.org/pkg/combine}{\textsf{combine}}
is an elaborate solution to combine several documents into one.
\end{itemize}
%
See also the CTAN topic \href{http://ctan.org/topic/subdocs}{\textsf{subdocs}}
for further related packages.
The present package differs from the above solutions in that
a document structure constructed with the conventional |\include| mechanism
just needs two extra commands at the top of every file
such that all constituent files can be compiled individually.

%%%%%%%%%%%%%%%%%%%%%%%%%%%%%%%%%%%%%%%%%%%%%%%%%%%%%%%%%%%%%%%%%%%%%%%%%%%%%%%%
%\subsection{Feature Suggestions}
%
%The following is a list of features which may be useful for future
%versions of this package:
%%
%\begin{itemize}
%\item
%\ldots
%\end{itemize}

%%%%%%%%%%%%%%%%%%%%%%%%%%%%%%%%%%%%%%%%%%%%%%%%%%%%%%%%%%%%%%%%%%%%%%%%%%%%%%%%
\subsection{Revision History}

%%%%%%%%%%%%%%%%%%%%%%%%%%%%%%%%%%%%%%%%
\paragraph{v2.0:} 2018/12/30

\begin{itemize}
\item
immediate forward processing
\item
added |\childdocby| mechanism
\item
manual restructured
\end{itemize}

%%%%%%%%%%%%%%%%%%%%%%%%%%%%%%%%%%%%%%%%
\paragraph{v1.6:} 2018/01/17

\begin{itemize}
\item
application for development of include files
\item
corrections to manual
\end{itemize}

%%%%%%%%%%%%%%%%%%%%%%%%%%%%%%%%%%%%%%%%
\paragraph{v1.5:} 2017/05/21

\begin{itemize}
\item
more complete structuring introduced
\item
|\childdocof| introduced
\item
|\childdoc| renamed to |\childdocmain|
\item
|\childredirect| renamed to |\childdocforward| and |\childdocforwardprefix|
and functionality expanded
\end{itemize}

%%%%%%%%%%%%%%%%%%%%%%%%%%%%%%%%%%%%%%%%
\paragraph{v1.0:} 2017/04/27

\begin{itemize}
\item
manual and install package
\item
first version published on CTAN
\end{itemize}

%%%%%%%%%%%%%%%%%%%%%%%%%%%%%%%%%%%%%%%%
\paragraph{v0.6:} 2017/04/26

\begin{itemize}
\item
redirection mechanism added
\end{itemize}

%%%%%%%%%%%%%%%%%%%%%%%%%%%%%%%%%%%%%%%%
\paragraph{v0.5:} 2017/04/26

\begin{itemize}
\item
functionality in definition file
\end{itemize}


%%%%%%%%%%%%%%%%%%%%%%%%%%%%%%%%%%%%%%%%%%%%%%%%%%%%%%%%%%%%%%%%%%%%%%%%%%%%%%%%
%%%%%%%%%%%%%%%%%%%%%%%%%%%%%%%%%%%%%%%%%%%%%%%%%%%%%%%%%%%%%%%%%%%%%%%%%%%%%%%%
%%%%%%%%%%%%%%%%%%%%%%%%%%%%%%%%%%%%%%%%%%%%%%%%%%%%%%%%%%%%%%%%%%%%%%%%%%%%%%%%
\appendix

\settowidth\MacroIndent{\rmfamily\scriptsize 000\ }

 \DocInput{childdoc.dtx}

\end{document}
%</driver>
% \fi
%
% %%%%%%%%%%%%%%%%%%%%%%%%%%%%%%%%%%%%%%%%%%%%%%%%%%%%%%%%%%%%%%%%%%%%%%%%%%%%%%
% %%%%%%%%%%%%%%%%%%%%%%%%%%%%%%%%%%%%%%%%%%%%%%%%%%%%%%%%%%%%%%%%%%%%%%%%%%%%%%
% \section{Sample}
%\iffalse
%<*samplemain>
%\fi
%
% The following presents a sample document
% with two chapters, two parts, a title page,
% a compile flag as well as three forwarding files to set the flag.
% It consists of eight |.tex| files:
% \begin{center}
% \begin{tabular}{ll}
% |cdocsamp.tex|&main file\\
% |cdocsch1.tex|&include file for chapter 1\\
% |cdocsch2.tex|&include file for chapter 2\\
% |cdocspt3.tex|&include file for part 3\\
% |cdocspt4.tex|&include file for part 4\\
% |cdocsdrf.tex|&forwarding file for main file in draft mode\\
% |cdocsfi1.tex|&forwarding file for final version of chapter 1\\
% |cdocsfi2.tex|&forwarding file for final version of chapter 2\\
% \end{tabular}
% \end{center}
% Each of the eight files can be compiled directly by the \LaTeX{} compiler.
%
% %%%%%%%%%%%%%%%%%%%%%%%%%%%%%%%%%%%%%%
% \paragraph{Main File.}
%
% The main file is called |cdocsamp.tex|.
%
% Load the \textsf{childdoc} definitions and
% declare the filename for the main document:
%    \begin{macrocode}
\input{childdoc.def}
\childdocmain{}
%    \end{macrocode}

% Optional override for |\version| flag:
%    \begin{macrocode}
%%\ifchilddoc\else\providecommand{\version}{draft}\fi
%    \end{macrocode}

% Define the default values for the |\version| flag
% (|final| for the main file and |draft| for childs):
%    \begin{macrocode}
\ifchilddoc
\providecommand{\version}{draft}
\else
\providecommand{\version}{final}
\fi
%    \end{macrocode}

% Load the standard document class:
%    \begin{macrocode}
\documentclass[12pt]{article}
%    \end{macrocode}

% Start the document body:
%    \begin{macrocode}
\begin{document}
%    \end{macrocode}

% Declare a title page.
% Print title, part of document being processed and version flag:
%    \begin{macrocode}
\addtocounter{page}{-1}
\begin{center}
{\LARGE\bfseries{}childdoc example\par}
\vspace{1cm}
\ifchilddoc
\ifchilddocmanual part\else chapter\fi:
`\childdocname' of `\childdocjob'\par
\else
main document: `\childdocjob'\par
\fi
version: \version\par
\end{center}
\newpage
%    \end{macrocode}

% Manually include selected file,
% otherwise process as usual:
%    \begin{macrocode}
\ifchilddocmanual
\section*{part `\childdocname'}
\input{\childdocname}
\else
%    \end{macrocode}

% Include the two chapters:
%    \begin{macrocode}
\include{cdocsch1}
\include{cdocsch2}
%    \end{macrocode}

% Include the two parts unless only chapters should be displayed:
%    \begin{macrocode}
\ifchilddoc\else
\section{part three}
\input{cdocspt3}
\section{part four}
\input{cdocspt4}
\fi
%    \end{macrocode}

% Process as usual until here:
%    \begin{macrocode}
\fi
%    \end{macrocode}

% End of document body:
%    \begin{macrocode}
\end{document}
%    \end{macrocode}
%\iffalse
%</samplemain>
%\fi
%
% %%%%%%%%%%%%%%%%%%%%%%%%%%%%%%%%%%%%%%
% \paragraph{Chapter Include Files.}
%
% The include files are called |cdocsch1.tex| and |cdocsch2.tex|.
%
%\iffalse
%<*samplechap1|samplechap2>
%\fi

% Optional override for |\version| flag:
%    \begin{macrocode}
%%\providecommand{\version}{final}
%    \end{macrocode}

% Include the main document:
%    \begin{macrocode}
\input{childdoc.def}
\childdocof{cdocsamp}
%    \end{macrocode}

%\iffalse
%</samplechap1|samplechap2>
%\fi
%
%\iffalse
%<*samplechap1>
%\fi
% Some text for chapter 1:
%    \begin{macrocode}
\section{one}
some text in chapter one
%    \end{macrocode}

%\iffalse
%</samplechap1>
%\fi
% Some text for chapter 2:
%\iffalse
%<*samplechap2>
%\fi
%    \begin{macrocode}
\section{two}
more text in chapter two
%    \end{macrocode}

%\iffalse
%</samplechap2>
%\fi
%
% %%%%%%%%%%%%%%%%%%%%%%%%%%%%%%%%%%%%%%
% \paragraph{Part Include Files.}
%
% The include files are called |cdocspt3.tex| and |cdocspt4.tex|.
%
%\iffalse
%<*samplepart3|samplepart4>
%\fi

% Optional override for |\version| flag:
%    \begin{macrocode}
%%\providecommand{\version}{final}
%    \end{macrocode}

% Include the main document:
%    \begin{macrocode}
\input{childdoc.def}
\childdocby{cdocsamp}
%    \end{macrocode}

%\iffalse
%</samplepart3|samplepart4>
%\fi
%
%\iffalse
%<*samplepart3>
%\fi
% Some text for part 3:
%    \begin{macrocode}
some text in part three
%    \end{macrocode}

%\iffalse
%</samplepart3>
%\fi
% Some text for part 4:
%\iffalse
%<*samplepart4>
%\fi
%    \begin{macrocode}
more text in part four
%    \end{macrocode}

%\iffalse
%</samplepart4>
%\fi
%
% %%%%%%%%%%%%%%%%%%%%%%%%%%%%%%%%%%%%%%
% \paragraph{Forwarding for a Complete Draft.}
%
% The following forwarding file |cdocsdrf.tex|
% compiles the main document in draft mode:
%\iffalse
%<*sampledraft>
%\fi
%    \begin{macrocode}
\def\version{draft}
\input{childdoc.def}
\childdocforward{cdocsamp}
%    \end{macrocode}

%\iffalse
%</sampledraft>
%\fi
%
% %%%%%%%%%%%%%%%%%%%%%%%%%%%%%%%%%%%%%%
% \paragraph{Forwarding for Final Version of the Chapters.}
%
% The following forwarding files |cdocsfn1.tex| and |cdocsfn2.tex|
% (with identical content)
% compile the final versions of the child documents
% |cdocsch1.tex| and |cdocsch2.tex|, respectively:
%\iffalse
%<*samplefinal>
%\fi
%    \begin{macrocode}
\def\version{final}
\input{childdoc.def}
\childdocforwardprefix[cdocsamp]{cdocsfn}{cdocsch}
%    \end{macrocode}

%\iffalse
%</samplefinal>
%\fi
%
% %%%%%%%%%%%%%%%%%%%%%%%%%%%%%%%%%%%%%%
% \paragraph{Command Line Processing.}
%
% The following three command lines generate the output files
% |cdocscld|, |cdocscl1| and |cdocscl2|
% which should be identical to
% |cdocsdrf|, |cdocsch1| and |cdocsfn2|, respectively:
% \begin{center}
% \begin{tabular}{l}
% |latex -jobname cdocscld \|\\
% |  "\def\version{draft}\input{childdoc.def}\childdocforward{cdocsamp}"|\\
% |latex -jobname cdocscl1 \|\\
% |  "\input{childdoc.def}\childdocforward[cdocsamp]{cdocsch1}"|\\
% |latex -jobname cdocscl2 \|\\
% |  "\def\version{final}\input{childdoc.def}\childdocforward{cdocsch2}"|
% \end{tabular}
% \end{center}
% Note that the trailing backslash on each first line
% merely continues the input to the second line
% (for convenient cut ant paste).
% Furthermore, the command |latex| can be replaced by any
% of its alternative versions such as |pdflatex|.
%
% %%%%%%%%%%%%%%%%%%%%%%%%%%%%%%%%%%%%%%%%%%%%%%%%%%%%%%%%%%%%%%%%%%%%%%%%%%%%%%
% %%%%%%%%%%%%%%%%%%%%%%%%%%%%%%%%%%%%%%%%%%%%%%%%%%%%%%%%%%%%%%%%%%%%%%%%%%%%%%
% \section{Implementation}
%\iffalse
%<*package>
%\fi
%
% This section describes the definitions file |childdoc.def|.

% The definitions cannot be loaded using |\usepackage| or |\RequirePackage|
% which has a mechanism to prevent loading a style file more than once.
% When loading the definitions by means of |\input|
% multiple instances have to be prevented manually:
%\iffalse
%This code needs to be before the `\ProvidesFile' directive
%which is defined at the beginning of this file.
%Therefore it is also placed there and commented out here.
%</package>
%<*discard>
%\fi
%    \begin{macrocode}
\ifdefined\childdocmain\endinput\fi
%    \end{macrocode}
%\iffalse
%</discard>
%<*package>
%\fi
%
% \macro{\ifchilddoc}
% \macro{\ifchilddocmanual}
% The conditional |\ifchilddoc| tells whether a
% child (true) or main (false) document is being compiled.
% The conditional |\ifchilddocmanual| tells whether
% the |\includeonly| mechanism is used (false) or
% the selection of child files must be performed manually (true).
% The definitions initialise to false:
%    \begin{macrocode}
\newif\ifchilddoc
\newif\ifchilddocmanual
%    \end{macrocode}

% \macro{\childdocname}
% \macro{\childdocjob}
% The macro |\childdocname| stores the name of the main document
% to be compiled. The macro |\childdocjob| stores the name of
% the document on which the \LaTeX{} compiler was originally invoked.
% The content of |\jobname| cannot be compared
% to filenames specified in the source due to different catcodes.
% The following code rescans |\jobname|, stores the result
% in |\childdocname| and saves a copy in |\childdocjob|:
%    \begin{macrocode}
\edef\childdocname{\scantokens\expandafter{\jobname\noexpand}}
\let\childdocjob\childdocname
%    \end{macrocode}

% \macro{\childdocdisable}
% The macro |\childdocdisable| prevents the main file
% from being processed more than once.
% At this stage, the main document command |\childdocmain|
% is assumed to be called once again where it should do nothing.
% Any subsequent call to it should prevent
% a secondary processing of the main document
% It overwrites the forwarding commands
% |\childdocof| and |\childdocforward|
% with empty macros to prevent further inclusions of the main document:
%    \begin{macrocode}
\newcommand{\childdocdisable}
{
  \renewcommand{\childdocmain}[1]{\renewcommand{\childdocmain}[1]{\endinput}}
  \renewcommand{\childdocof}[1]{}
  \renewcommand{\childdocby}[2][]{}
  \renewcommand{\childdocforward}[2][]{}
  \renewcommand{\childdocdisable}{}
}
%    \end{macrocode}

% \macro{\childdocmain}
% The macro |\childdocmain| is to be called at the top of the main file
% with nothing or the main filename (without extension) as argument.
% First, it breaks loops.
% If the argument is not empty and does not match |\childdocname|
% (which is set by the first inclusion of |childdoc.def|),
% |\ifchilddoc| is set to true, |\includeonly| is applied to the child file
% and |\jobname| is set to the main file
% (for proper handling of |.aux| files):
%    \begin{macrocode}
\newcommand{\childdocmain}[1]
{
  \childdocdisable\childdocmain{}
  \if?#1?\else
    \begingroup
      \def\childdoctmp{#1}
      \ifx\childdoctmp\childdocname
        \def\childdoctmp{}
      \else
        \def\childdoctmp
        {
          \childdoctrue
          \includeonly{\childdocname}
          \def\childdocjob{#1}
          \def\jobname{#1}
        }
      \fi
      \expandafter
    \endgroup
    \childdoctmp
  \fi
}
%    \end{macrocode}

% \macro{\childdocof}
% The command |\childdocof| redirects
% compilation to the main file |#1|.
%    \begin{macrocode}
\newcommand{\childdocof}[1]
{
  \childdocdisable
  \childdoctrue
  \includeonly{\childdocname}
  \def\jobname{#1}
  \def\childdocjob{#1}
  \input{#1}
}
%    \end{macrocode}

% \macro{\childdocby}
% The command |\childdocby| ....
%    \begin{macrocode}
\newcommand{\childdocby}[2][]
{
  \childdocdisable
  \childdoctrue
  \childdocmanualtrue
  \if?#1?\else
    \def\jobname{#2}
  \fi
  \def\childdocjob{#2}
  \input{#2}
  \endinput
}
%    \end{macrocode}

% \macro{\childdocforward}
% The command |\childdocforward| redirects
% compilation to the main file or
% (if the optional argument is given) a child file.
% Parameters are set as if the main file
% or a child file starting with |\childdocof| was compiled.
% Then compilation is handed over to the main file:
%    \begin{macrocode}
\newcommand{\childdocforward}[2][]
{
  \begingroup
    \if?#1?
      \def\childdoctmp
      {
        \def\childdocname{#2}
        \def\childdocjob{#2}
        \def\jobname{#2}
        \input{#2}
        \endinput
      }
    \else
      \def\childdoctmp
      {
        \childdocdisable
        \def\childdocname{#2}
        \childdoctrue
        \includeonly{#2}
        \def\childdocjob{#1}
        \def\jobname{#1}
        \input{#1}
        \endinput
      }
    \fi
    \expandafter
  \endgroup
  \childdoctmp
}
%    \end{macrocode}

% \macro{\childdocforwardprefix}
% The command |\childdocforwardprefix| redirects
% compilation to the main or a child file by means of a pattern.
% The prefix |#1| in the current filename is replaced by |#2|
% and the suffix of the current filename is kept
% (it is assumed that the filename does not contain the substring `|~~~|'
% which is used as a delimiter).
% Compilation is handed over to the new file by |\childdocforward|:
%    \begin{macrocode}
\newcommand{\childdocforwardprefix}[3][]
{
  \begingroup
    \def\childdocextract #2##1~~~{\def\childdoctmp{\childdocforward[#1]{#3##1}}}
    \expandafter\childdocextract\childdocname~~~
    \expandafter
  \endgroup
  \childdoctmp
}
%    \end{macrocode}

% \macro{\childdoc}
% The deprecated macro |\childdoc| is a legacy version of |\childdocmain|:
%    \begin{macrocode}
\newcommand{\childdoc}{\childdocmain}
%    \end{macrocode}

% \macro{\childdocredirect}
% The deprecated macro |\childdocredirect| is a legacy version
% of |\childdocforward| and |\childdocforwardprefix|:
%    \begin{macrocode}
\newcommand{\childdocredirect}[2][]
{
  \begingroup
    \if?#1?
      \def\childdoctmp{\childdocforward{#2}}
    \else
      \def\childdoctmp{\childdocforwardprefix{#1}{#2}}
    \fi
    \expandafter
  \endgroup
  \childdoctmp
}
%    \end{macrocode}

%\iffalse
%</package>
%\fi
%
\endinput
|\\
|\childdocmain{|\textit{main}|}|\\
\end{tabular}
\end{center}
%
If |\jobname| does not match the argument \textit{main} of |\childdocmain|,
it is assumed that |\jobname| points to the child file to be compiled.
When using |\childdocmain| with the main file specified as argument,
it suffices to start a child file
with just |\input{|\textit{main}|}|
without loading of the package and using |\childdocof|.
If instead all processing is done
with the appropriate \textsf{childdoc} directives,
the argument of \textit{main} of |\childdocmain| can be empty.

An alternative version of the command line processing described
in \secref{sec:commandline} using the detection mechanism reads:
%
\begin{center}
|... -jobname "|\textit{target}|" "|[\textit{flags}]%
[|\def\jobname{|\textit{dest}|}|]|\input{|\textit{main}|}"|
\end{center}

%%%%%%%%%%%%%%%%%%%%%%%%%%%%%%%%%%%%%%%%%%%%%%%%%%%%%%%%%%%%%%%%%%%%%%%%%%%%%%%%
\subsection{Manual Code}
\label{sec:manual}

In case one cannot be certain whether the definitions file |childdoc.def|
is installed on the target \TeX{} distribution
and one prefers not to ship it,
it is conceivable to paste a few relevant commands into the sources.

To that end, drop all statements |% \iffalse
%
% childdoc.dtx Copyright (C) 2017-2018 Niklas Beisert
%
% This work may be distributed and/or modified under the
% conditions of the LaTeX Project Public License, either version 1.3
% of this license or (at your option) any later version.
% The latest version of this license is in
%   http://www.latex-project.org/lppl.txt
% and version 1.3 or later is part of all distributions of LaTeX
% version 2005/12/01 or later.
%
% This work has the LPPL maintenance status `maintained'.
%
% The Current Maintainer of this work is Niklas Beisert.
%
% This work consists of the files childdoc.dtx and childdoc.ins
% and the derived files childdoc.def and cdocsamp.tex with
% cdocsch1.tex, cdocsch2.tex, cdocsdrf.tex, cdocsfn1.tex, cdocsfn2.tex.
%
%<package>\ifdefined\childdocmain\endinput\fi
%<package>\ProvidesFile{childdoc.def}[2018/12/30 v2.0 child document driver]
%<samplemain>\ProvidesFile{cdocsamp.tex}[2018/12/30 v2.0 sample for childdoc]
%<*driver>
%\ProvidesFile{childdoc.drv}[2018/12/30 v2.0 childdoc reference manual file]
\PassOptionsToClass{10pt,a4paper}{article}
\documentclass{ltxdoc}

\usepackage[margin=35mm]{geometry}
\usepackage{hyperref}
\usepackage{hyperxmp}
\usepackage[usenames]{color}

\hypersetup{colorlinks=true}
\hypersetup{pdfstartview=FitH}
\hypersetup{pdfpagemode=UseNone}
\hypersetup{pdfsource={}}
\hypersetup{pdflang={en-UK}}
\hypersetup{pdfcopyright={Copyright 2017-2018 Niklas Beisert.
  This work may be distributed and/or modified under the
  conditions of the LaTeX Project Public License, either version 1.3
  of this license or (at your option) any later version.}}
\hypersetup{pdflicenseurl={http://www.latex-project.org/lppl.txt}}
\hypersetup{pdfcontactaddress={ETH Zurich, ITP, HIT K,
  Wolfgang-Pauli-Strasse 27}}
\hypersetup{pdfcontactpostcode={8093}}
\hypersetup{pdfcontactcity={Zurich}}
\hypersetup{pdfcontactcountry={Switzerland}}
\hypersetup{pdfcontactemail={nbeisert@itp.phys.ethz.ch}}
\hypersetup{pdfcontacturl={http://people.phys.ethz.ch/\xmptilde nbeisert/}}

\newcommand{\secref}[1]{\hyperref[#1]{section \ref*{#1}}}

\parskip1ex
\parindent0pt
\let\olditemize\itemize
\def\itemize{\olditemize\parskip0pt}

\begin{document}

\title{The \textsf{childdoc} Package}
\hypersetup{pdftitle={The childdoc Package}}
\author{Niklas Beisert\\[2ex]
  Institut f\"ur Theoretische Physik\\
  Eidgen\"ossische Technische Hochschule Z\"urich\\
  Wolfgang-Pauli-Strasse 27, 8093 Z\"urich, Switzerland\\[1ex]
  \href{mailto:nbeisert@itp.phys.ethz.ch}
  {\texttt{nbeisert@itp.phys.ethz.ch}}}
\hypersetup{pdfauthor={Niklas Beisert}}
\hypersetup{pdfsubject={Manual for the LaTeX2e Package childdoc}}
\date{30 December 2018, \textsf{v2.0}}
\maketitle

\begin{abstract}\noindent
\textsf{childdoc} is a \LaTeXe{} package
that enables the direct compilation
of document sections included by |\include|
to individual files.
\end{abstract}

\begingroup
\parskip0ex
\tableofcontents
\endgroup

%%%%%%%%%%%%%%%%%%%%%%%%%%%%%%%%%%%%%%%%%%%%%%%%%%%%%%%%%%%%%%%%%%%%%%%%%%%%%%%%
%%%%%%%%%%%%%%%%%%%%%%%%%%%%%%%%%%%%%%%%%%%%%%%%%%%%%%%%%%%%%%%%%%%%%%%%%%%%%%%%
\section{Introduction}

\LaTeX{} provides a mechanism to structure a large document (such as a book)
into a main file and several child files (containing the chapters)
using the |\include| command.
This mechanism is beneficial for documents
which span hundreds of pages in order to
make the source file(s) more manageable.
Moreover, compilation can be restricted to
selected child files by means of the |\includeonly| command.
The latter feature can be used to reduce the compilation time while editing
(this was significantly more useful in the earlier days of \LaTeX{})
or to generate a smaller document which is easier to navigate.
Another application of |\includeonly| is to generate
documents consisting of selected parts of the complete document.

However, there are a few drawbacks of the plain |\include| mechanism:
\begin{itemize}
\item
The child files cannot be compiled on their own,
they can only be compiled via the main file.
A naive editing environment
(such as a text editor with an option
to have the current file processed by \LaTeX)
may require one to switch to the main file before compiling;
attempting to compile the child file produces errors.
\item
The main file must be modified (each time)
to adjust the |\includeonly| command
to the present needs. This easily leaves the main file in a messy state.
\item
The generated document will always carry the filename
of the main document. This is inconvenient if
several child files are to be compiled and
to be kept for distribution.
\end{itemize}

The present package provides a simple interface
to make child files individually compilable by \LaTeX{}.
Compiling a child file then has the same effect as compiling
the main file with an |\includeonly| command
to select the appropriate child.
Moreover the generated document will carry the name of the child
rather than the main file.
This resolves all three above issues.

This feature is meant to make the editing of books,
thesis documents and lecture notes somewhat more convenient.
However, the package can also be used efficiently for
composing a series of documents (such as exercise sheets)
which are typically distributed individually.
It then assists the author in generating the individual documents
(potentially in different versions)
as well as a document containing the collected series.
Another application is in developing style files
or other kinds of included material
where compilation of the style file could redirect
to a sample or test file.

%%%%%%%%%%%%%%%%%%%%%%%%%%%%%%%%%%%%%%%%%%%%%%%%%%%%%%%%%%%%%%%%%%%%%%%%%%%%%%%%
%%%%%%%%%%%%%%%%%%%%%%%%%%%%%%%%%%%%%%%%%%%%%%%%%%%%%%%%%%%%%%%%%%%%%%%%%%%%%%%%
\section{Usage}

First of all, the package \textsf{childdoc} is \emph{not} a standard
\LaTeXe{} |.sty| style file! Therefore it needs to be invoked in
a non-standard way.

%%%%%%%%%%%%%%%%%%%%%%%%%%%%%%%%%%%%%%%%%%%%%%%%%%%%%%%%%%%%%%%%%%%%%%%%%%%%%%%%
\subsection{Included Files}
\label{sec:include}

%%%%%%%%%%%%%%%%%%%%%%%%%%%%%%%%%%%%%%%%
\DescribeMacro{\childdocmain}
To use the package, add the commands
\begin{center}
\begin{tabular}{l}
|\input{childdoc.def}|\\
|\childdocmain{}|\\
\end{tabular}
\end{center}
at the very top of the main \LaTeX{} file,
in particular \emph{before} the |\documentclass| statement!
The argument of |\childdocmain| should be left empty
(but it must be present).

%%%%%%%%%%%%%%%%%%%%%%%%%%%%%%%%%%%%%%%%
\DescribeMacro{\childdocof}
Furthermore, add the commands
\begin{center}
\begin{tabular}{l}
|\input{childdoc.def}|\\
|\childdocof{|\textit{main}|}|\\
\end{tabular}
\end{center}
at the top of every child file \textit{child}
which is included by |\include{|\textit{child}|}|
from within the main file
(or at least for those files to be compiled individually).
The argument \textit{main} must be the filename of the main file.

There are a couple of
considerations in setting up the main and child documents:

%%%%%%%%%%%%%%%%%%%%%%%%%%%%%%%%%%%%%%%%
\paragraph{Restrictions.}

Please note the following restrictions:
\begin{itemize}
\item
|\childdocmain| must be called with one argument \textit{main}
to ensure compatibility with earlier version of the package.
It must either be empty (|\childdocmain{}|)
or precisely match the filename of the main file in which it is specified.
See \secref{sec:detection} for further information.
\item
The filename \textit{main} must be specified without the |.tex| extension.
\item
The filename \textit{main} is case sensitive
(even in case-insensitive file systems)
due to internal string comparison.
\item
The argument \textit{main} should be fully expanded, it cannot be a macro.
\item
Subdirectories and special characters should be avoided in filenames.
\item
The command |\childdocmain{|\textit{main}|}| must be followed by a whitespace.
It should not be followed immediately by another command
or by a comment mark `|%|'.
This is because the \TeX{} parser reads the token immediately following
the argument of |\childdocmain| and puts it
at the beginning of every child section;
however, a white\-space is ignored.
\end{itemize}

%%%%%%%%%%%%%%%%%%%%%%%%%%%%%%%%%%%%%%%%
\paragraph{Content of Main File.}

It is advisable to place all content in the child files included by |\include|.
Any output contained in the main file will appear in all child documents
unless suppressed manually;
it cannot be suppressed automatically by the |\includeonly| directive
and thus should normally be avoided.
A method to include some content in the main file
by means of conditional processing is described in \secref{sec:conditional}.

%%%%%%%%%%%%%%%%%%%%%%%%%%%%%%%%%%%%%%%%
\paragraph{Page Numbering.}

When only a part of the document is compiled,
the appropriate numbering of pages
(as well as other status parameters)
is determined from the |.aux| files.
The latter contain information from previous passes.
However this information needs to propagate through
all intermediate child documents.
Therefore the page numbering in child documents may well
be inconsistent until the complete document is compiled at least once.

A useful (if unconventional) way to always ensure a consistent
page numbering is to restart the numbering in each child document
and denote the pages by `\textit{child}|.|\textit{page}'
where \textit{child} represents the chapter/section number of the child file.
This can be achieved by the command
|\numberwithin{page}{|\textit{child}|}|
of the \textsf{amsmath} package
where \textit{child} can be |chapter| or |section|
depending on the chosen structuring.
Alternatively, one can modify the macro |\thepage| appropriately
and reset the counter |page| at the start of each child file.

%%%%%%%%%%%%%%%%%%%%%%%%%%%%%%%%%%%%%%%%%%%%%%%%%%%%%%%%%%%%%%%%%%%%%%%%%%%%%%%%
\subsection{Conditional Processing}
\label{sec:conditional}

The package provides a mechanism to compile different versions
of a document. To customise the versions further some conditional processing
can come in handy to distinguish which version is being compiled.
The package provides two macros to describe the compilation context:

%%%%%%%%%%%%%%%%%%%%%%%%%%%%%%%%%%%%%%%%
\DescribeMacro{\ifchilddoc}
The conditional |\ifchilddoc| distinguishes between the compilation of
child documents and the main document:
%
\begin{center}
|\ifchilddoc |\textit{child-code}| |[|\||else |\textit{main-code}]| \||fi|
\end{center}

%%%%%%%%%%%%%%%%%%%%%%%%%%%%%%%%%%%%%%%%
\DescribeMacro{\childdocname}
\DescribeMacro{\childdocjob}
The macro |\childdocname| contains the filename (without extension)
of the main or child file being processed.
Note that |\childdocjob| will always contain the name of the main file.

%%%%%%%%%%%%%%%%%%%%%%%%%%%%%%%%%%%%%%%%
\paragraph{Title Page.}

Conditional processing can be used to include a title or banner page
in the main document when proper precautions are taken.
Importantly, the code in the main file should ensure that the page counter
(as well as other status parameters which are stored in the |.aux| files)
takes the same value after the conditional processing.
Otherwise the page numbers may take divergent values
depending on which part is compiled.

For example, a title page could be declared by:
%
\begin{center}
\begin{tabular}{l}
|\ifchilddoc\||else|\\
|\addtocounter{page}{-1}|\\
\textit{code for title page}\\
|\newpage|\\
|\||fi|
\end{tabular}
\end{center}
%
A banner page for the child documents can be generated by:
%
\begin{center}
\begin{tabular}{l}
|\ifchilddoc|\\
|\addtocounter{page}{-1}|\\
\textit{code for banner page}\\
|\newpage|\\
|\||fi|
\end{tabular}
\end{center}
%
Here one could write a message such as:
\begin{center}
|This is the part \childdocname{} of \childdocjob{}.|
\end{center}

%%%%%%%%%%%%%%%%%%%%%%%%%%%%%%%%%%%%%%%%%%%%%%%%%%%%%%%%%%%%%%%%%%%%%%%%%%%%%%%%
\subsection{Flags}
\label{sec:flags}

The package makes it easy to generate different versions
of the main or child documents.
To this end compilation flags can be defined
and assigned different default values.
They will be particularly useful in conjunction
with the forwarding mechanism described in \secref{sec:forward}.

For example, it may be useful to have a flag |\version|
which can be set to |draft| or |final|.
The document source will contain some conditional code
depending on the value of |\version|.
Suppose further, the flag should default to |final| for the main file
and to |draft| for child files
which is a natural assignment for editing the document.
This is achieved by placing the following code
in the preamble of the main document
(below the |\childdocmain| directive):
%
\begin{center}
\begin{tabular}{l}
|\ifchilddoc|\\
|\providecommand{\version}{draft}|\\
|\||else|\\
|\providecommand{\version}{final}|\\
|\||fi|
\end{tabular}
\end{center}
%
The definition by |\providecommand| makes sure
that previous definitions are not overwritten.
Further statements |\providecommand{\version}{...}|
can thus be added before the above code to override it.

For the main file, one might add a line
(between |\childdocmain| and the above block)
%
\begin{center}
|%\ifchilddoc\||else\providecommand{\version}{draft}\||fi|
\end{center}
%
which can be uncommented to produce a draft version.
Likewise one can add a line to the very top of a child file
(above the |\childdocof{|\textit{main}|}| directive)
%
\begin{center}
|%\providecommand{\version}{final}|
\end{center}
%
which can be uncommented to produce the final version of this child document.

%%%%%%%%%%%%%%%%%%%%%%%%%%%%%%%%%%%%%%%%%%%%%%%%%%%%%%%%%%%%%%%%%%%%%%%%%%%%%%%%
\subsection{Forwarding}
\label{sec:forward}

Different versions of the main or child documents
using compilation flags as described in \secref{sec:flags}
can be (permanently) stored in different files
for convenient compilation, viewing and distribution.
To this end, the package defines a command
to pass on compilation to a different file:

%%%%%%%%%%%%%%%%%%%%%%%%%%%%%%%%%%%%%%%%
\DescribeMacro{\childdocforward}
The command |\childdocforward| redirects processing to
another source file:
%
\begin{center}
\begin{tabular}{l}
|\input{childdoc.def}|\\
|\childdocforward[|\textit{main}|]{|\textit{dest}|}|\\
\end{tabular}
\end{center}
%
The argument \textit{dest} is the destination file
(without extension).
It should be the main file or one of the child files.
Note that further \textsf{childdoc} directives
such as |\childdocof| and |\childdocforward|
in the indicated file will be processed in this form.
The optional argument \textit{main}
passes on directly to the main file \textit{main}
while pretending to compile the child \textit{dest}.
This form behaves as if \textit{dest}
issues |\childdocof{|\textit{main}|}| right away,
and no further \textsf{childdoc} directives will be processed.

%%%%%%%%%%%%%%%%%%%%%%%%%%%%%%%%%%%%%%%%
\DescribeMacro{\...prefix}
In the alternative form |\childdocforwardprefix|,
%
\begin{center}
\begin{tabular}{l}
|\input{childdoc.def}|\\
|\childdocforwardprefix[|\textit{main}|]{|\textit{prefix}|}{|\textit{dest}|}|
\end{tabular}
\end{center}
%
the destination file is determined by a pattern
depending on the current file:
To make this work, the current file must be called
`{\textit{prefix}\hspace{0.2em}\textit{suffix}}'
with \textit{prefix} matching precisely the argument.
Processing is then passed on to the file
`{\textit{dest}\hspace{0.2em}\textit{suffix}}'.
Surely, the same effect is achieved by
directly specifying the
argument `{\textit{dest}\hspace{0.2em}\textit{suffix}}'
in the first form.
However, that requires to set up a different file
for each child. With the alternative form of the command
all these files can have exactly the same content
which simplifies setting them up and maintaining them.

For example, the following file |draft.tex|
with a compilation flag |\version| as described in \secref{sec:flags}
compiles the main document as a draft:
%
\begin{center}
\begin{tabular}{l}
|\def\version{draft}|\\
|\input{childdoc.def}|\\
|\childdocforward{|\textit{main}|}|
\end{tabular}
\end{center}
%
Likewise, the following files |final|\textit{nn}|.tex|
compile the final version of the child document
|child|\textit{nn}|.tex|:
%
\begin{center}
\begin{tabular}{l}
|\def\version{final}|\\
|\input{childdoc.def}|\\
|\childdocforwardprefix{final}{child}|
\end{tabular}
\end{center}
%

Note that when several versions of a main file and/or of each child file
are to be generated, it may be convenient to set up a |Makefile| or
shell script to automatise the process.

%%%%%%%%%%%%%%%%%%%%%%%%%%%%%%%%%%%%%%%%%%%%%%%%%%%%%%%%%%%%%%%%%%%%%%%%%%%%%%%%
\subsection{Command Line Processing}
\label{sec:commandline}

The effect of redirection files can also be achieved by invoking
the \LaTeX{} compiler with a more elaborate command line.
Most conveniently this should be done as part
of a shell script or a |Makefile|.

When using \textsf{childdoc} in the main file, the following
command lines effectively perform a redirection
(note that depending on the shell being used,
backslashes may have to be doubled: `|\|' $\to$ `|\\|'):
%
\begin{center}
|... -jobname "|\textit{target}|" |\\|"|[\textit{flags}]%
|\input{childdoc.def}\childdocforward[|\textit{main}|]{|\textit{dest}|}"|
\end{center}
%
Here \textit{target} is the name of the output file,
\textit{main} is the name of the main file
and \textit{dest} is the name of the main or child file to be processed
(all filenames without extensions).
The optional argument \textit{main} can be omitted
if \textit{main} matches \textit{dest}.
Optionally, compilation \textit{flags} can be defined via |\def| commands.
This command line makes the \TeX{} engine believe
it is compiling the file \textit{target}
whose content is specified as the latter parameter.
The provided code then forwards the processing to
\textit{main} or \textit{dest} as described in \secref{sec:forward}.

%%%%%%%%%%%%%%%%%%%%%%%%%%%%%%%%%%%%%%%%%%%%%%%%%%%%%%%%%%%%%%%%%%%%%%%%%%%%%%%%
\subsection{Include by Input}
\label{sec:input}

Including child documents by |\include| has some restrictions by design.
Most notably, the content of a child document always occupies
its own set of pages; pages cannot be shared between child documents.
Usually, this behaviour makes perfect sense
because each child document contain an essential part of the document.
However, in some situations it may be desirable to compose
a document from a collection of parts
without having mandatory page breaks between then.
For this case, the package
provides a mechanism to include parts
by |\input| which can also be processed individually.
However, by construction this mechanism
requires manual handling of the content to be output.

%%%%%%%%%%%%%%%%%%%%%%%%%%%%%%%%%%%%%%%%
\DescribeMacro{\ifchilddocmanual}
The main file should be prepared as usual, see \secref{sec:include}.
However, the document body must make a distinction
between processing of an individual part and of the main document, e.g.:
%
\begin{center}
\begin{tabular}{l}
|\ifchilddocmanual|\\
|\input{\childdocname}|\\
|\||else|\\
\textit{document body with }|\input{|\textit{part}|}|\\
|\||fi|
\end{tabular}
\end{center}
%
The conditional |\ifchilddocmanual| is true whenever
a part to be included by |\input| is being compiled,
and the name of the part is stored in |\childdocname|.

%%%%%%%%%%%%%%%%%%%%%%%%%%%%%%%%%%%%%%%%
\DescribeMacro{\childdocby}
Each part to be included by |\input| should start with:
%
\begin{center}
\begin{tabular}{l}
|\input{childdoc.def}|\\
|\childdocby{|\textit{main}|}|\\
\end{tabular}
\end{center}
%
The directive |\childdocby| is similar to |\childdocof|
described in \secref{sec:include},
but the subsequent selection of content must be done manually.
To that end, both |\ifchilddoc| and |\ifchilddocmanual|
will be true upon processing of a part,
and the name of the part is stored in |\childdocname|.
Note that |\jobname| will be set to the filename of the current part
so that each part receives an individual |.aux| file
that does not interfere with the |.aux| file(s) of the main document.
This behaviour can be altered by the alternative form
|\childdocby[*]{|\textit{main}|}| (with a non-empty optional argument)
which uses the |.aux| file of the main document
by setting |\jobname| to \textit{main}.

%%%%%%%%%%%%%%%%%%%%%%%%%%%%%%%%%%%%%%%%%%%%%%%%%%%%%%%%%%%%%%%%%%%%%%%%%%%%%%%%
\subsection{Driver Development}
\label{sec:driver}

The \textsf{childdoc} mechanism can also be use for the development
of definition files such as \LaTeX{} styles or classes.
This case differs from the above setup with multiple parts
included by |\include| in that no |\includeonly| should be invoked.
This can be achieved by starting the include file
(before |\ProvidesPackage|) with:
%
\begin{center}
\begin{tabular}{l}
|\input{childdoc.def}|\\
|\childdocforward{|\textit{main}|}|\\
\end{tabular}
\end{center}
%
or alternatively with:
%
\begin{center}
\begin{tabular}{l}
|\input{childdoc.def}|\\
|\childdocby{|\textit{main}|}|\\
\end{tabular}
\end{center}
%
Both forms have slightly different effects as described above.
The main file is prepared as usual, see \secref{sec:include}.

%%%%%%%%%%%%%%%%%%%%%%%%%%%%%%%%%%%%%%%%%%%%%%%%%%%%%%%%%%%%%%%%%%%%%%%%%%%%%%%%
\subsection{Legacy Detection}
\label{sec:detection}

The directive |\childdocmain| in the main file can detect
whether the complete document or merely a child is to be compiled
even without using the directive |\childdocof|.
This method is deprecated because it is less robust
and there is no compelling reason to use it;
it is merely provided for backward compatibility
and it may be removed in future versions.

If the detection mechanism is to be used,
it is mandatory to correctly specify
the filename of the main file as the argument of |\childdocmain|:
%
\begin{center}
\begin{tabular}{l}
|\input{childdoc.def}|\\
|\childdocmain{|\textit{main}|}|\\
\end{tabular}
\end{center}
%
If |\jobname| does not match the argument \textit{main} of |\childdocmain|,
it is assumed that |\jobname| points to the child file to be compiled.
When using |\childdocmain| with the main file specified as argument,
it suffices to start a child file
with just |\input{|\textit{main}|}|
without loading of the package and using |\childdocof|.
If instead all processing is done
with the appropriate \textsf{childdoc} directives,
the argument of \textit{main} of |\childdocmain| can be empty.

An alternative version of the command line processing described
in \secref{sec:commandline} using the detection mechanism reads:
%
\begin{center}
|... -jobname "|\textit{target}|" "|[\textit{flags}]%
[|\def\jobname{|\textit{dest}|}|]|\input{|\textit{main}|}"|
\end{center}

%%%%%%%%%%%%%%%%%%%%%%%%%%%%%%%%%%%%%%%%%%%%%%%%%%%%%%%%%%%%%%%%%%%%%%%%%%%%%%%%
\subsection{Manual Code}
\label{sec:manual}

In case one cannot be certain whether the definitions file |childdoc.def|
is installed on the target \TeX{} distribution
and one prefers not to ship it,
it is conceivable to paste a few relevant commands into the sources.

To that end, drop all statements |\input{childdoc.def}|
and perform the replacements as outlined below.
Instead of |\childdocmain{|\textit{main}|}| add the following code
to the top of the main file:
%
\begin{center}
\begin{tabular}{l}
|\||ifdefined\childdocname\endinput\||fi\newif\ifchilddoc|\\
|\edef\childdocname{\scantokens\expandafter{\jobname\noexpand}}|\\
|\def\childdocmain{|\textit{main}|}\||ifx\childdocmain\childdocname\||else|\\
|\childdoctrue\includeonly{\childdocname}\let\jobname\childdocmain\||fi|\\
\end{tabular}
\end{center}
%
Instead of |\childdocof{|\textit{main}|}| just include the main file
at the top of each child file:
%
\begin{center}
|\input{|\textit{main}|}|
\end{center}
%
A simple redirection |\childdocforward{|\textit{dest}|}| is achieved by:
%
\begin{center}
|\def\jobname{|\textit{dest}|}\input{\jobname}|
\end{center}
%
The redirection with prefix
|\childdocforwardprefix[|\textit{prefix}|]{|\textit{dest}|}|
is accomplished by:
%
\begin{center}
\begin{tabular}{l}
|{\edef\jobname{\scantokens\expandafter{\jobname\noexpand}}|\\
|\def\redirectjob |\textit{prefix}|#1~~~{\gdef\jobname{|\textit{dest}|#1}}|\\
|\expandafter\redirectjob\jobname~~~}\input{\jobname}|
\end{tabular}
\end{center}

In an alternative approach,
child documents can be compiled by a specific command line
without additional code or specific definitions:
%
\begin{center}
|... -jobname "|\textit{target}|" "|[\textit{flags}]%
|\includeonly{|\textit{dest}|}\input{|\textit{main}|}"|
\end{center}
%

%%%%%%%%%%%%%%%%%%%%%%%%%%%%%%%%%%%%%%%%%%%%%%%%%%%%%%%%%%%%%%%%%%%%%%%%%%%%%%%%
%%%%%%%%%%%%%%%%%%%%%%%%%%%%%%%%%%%%%%%%%%%%%%%%%%%%%%%%%%%%%%%%%%%%%%%%%%%%%%%%
\section{Information}

%%%%%%%%%%%%%%%%%%%%%%%%%%%%%%%%%%%%%%%%%%%%%%%%%%%%%%%%%%%%%%%%%%%%%%%%%%%%%%%%
\subsection{Copyright}

Copyright \copyright{} 2017--2018 Niklas Beisert

This work may be distributed and/or modified under the
conditions of the \LaTeX{} Project Public License, either version 1.3
of this license or (at your option) any later version.
The latest version of this license is in
  \url{http://www.latex-project.org/lppl.txt}
and version 1.3 or later is part of all distributions of \LaTeX{}
version 2005/12/01 or later.

This work has the LPPL maintenance status `maintained'.

The Current Maintainer of this work is Niklas Beisert.

This work consists of the files |README.txt|, |childdoc.ins| and |childdoc.dtx|
as well as the derived files |childdoc.def|, |cdocsamp.tex|
with |cdocsch1.tex|, |cdocsch2.tex|, |cdocspt3.tex|, |cdocspt4.tex|,
|cdocsdrf.tex|, |cdocsfn1.tex|, |cdocsfn2.tex|
as well as |childdoc.pdf|.

%%%%%%%%%%%%%%%%%%%%%%%%%%%%%%%%%%%%%%%%%%%%%%%%%%%%%%%%%%%%%%%%%%%%%%%%%%%%%%%%
\subsection{Files and Installation}

The package consists of the files:
%
\begin{center}
\begin{tabular}{ll}
    |README.txt|   & readme file \\
    |childdoc.ins| & installation file \\
    |childdoc.dtx| & source file \\
    |childdoc.def| & definition file \\
    |cdocsamp.tex| & sample main file \\
    |cdocsch1.tex| & sample include file \\
    |cdocsch2.tex| & sample include file \\
    |cdocspt3.tex| & sample part file \\
    |cdocspt4.tex| & sample part file \\
    |cdocsdrf.tex| & sample redirection file \\
    |cdocsfn1.tex| & sample redirection file \\
    |cdocsfn2.tex| & sample redirection file \\
    |childdoc.pdf| & manual
\end{tabular}
\end{center}
%
The distribution consists of the files
|README.txt|, |childdoc.ins| and |childdoc.dtx|.
%
\begin{itemize}
\item
Run (pdf)\LaTeX{} on |childdoc.dtx|
to compile the manual |childdoc.pdf| (this file).
\item
Run \LaTeX{} on |childdoc.ins| to create the definitions file |childdoc.def|
and the sample |cdocsamp.tex| with include files
|cdocsch1.tex|, |cdocsch2.tex|, |cdocspt3.tex|, |cdocspt4.tex|,
|cdocsdrf.tex|, |cdocsfn1.tex|, |cdocsfn2.tex|.
Then copy the file |childdoc.def| to an appropriate directory of your \LaTeX{}
distribution, e.g.\ \textit{texmf-root}|/tex/latex/childdoc|.
\end{itemize}

%%%%%%%%%%%%%%%%%%%%%%%%%%%%%%%%%%%%%%%%%%%%%%%%%%%%%%%%%%%%%%%%%%%%%%%%%%%%%%%%
\subsection{Related CTAN Packages}

There are several other packages which offer a similar functionality:
%
\begin{itemize}
\item
The packages
\href{http://ctan.org/pkg/docmute}{\textsf{docmute}},
\href{http://ctan.org/pkg/includex}{\textsf{includex}} and
\href{http://ctan.org/pkg/standalone}{\textsf{standalone}}
provide commands to include only the document body of
a child file thus allowing both files to be compiled individually.
\item
The packages \href{http://ctan.org/pkg/subdocs}{\textsf{subdocs}}
and \href{http://ctan.org/pkg/subfiles}{\textsf{subfiles}}
provide structures in which the main and child documents can be
encapsulated and allowing them to be compiled individually.
The inclusion mechanism is different from the conventional |\include|.
\item
The package \href{http://ctan.org/pkg/combine}{\textsf{combine}}
is an elaborate solution to combine several documents into one.
\end{itemize}
%
See also the CTAN topic \href{http://ctan.org/topic/subdocs}{\textsf{subdocs}}
for further related packages.
The present package differs from the above solutions in that
a document structure constructed with the conventional |\include| mechanism
just needs two extra commands at the top of every file
such that all constituent files can be compiled individually.

%%%%%%%%%%%%%%%%%%%%%%%%%%%%%%%%%%%%%%%%%%%%%%%%%%%%%%%%%%%%%%%%%%%%%%%%%%%%%%%%
%\subsection{Feature Suggestions}
%
%The following is a list of features which may be useful for future
%versions of this package:
%%
%\begin{itemize}
%\item
%\ldots
%\end{itemize}

%%%%%%%%%%%%%%%%%%%%%%%%%%%%%%%%%%%%%%%%%%%%%%%%%%%%%%%%%%%%%%%%%%%%%%%%%%%%%%%%
\subsection{Revision History}

%%%%%%%%%%%%%%%%%%%%%%%%%%%%%%%%%%%%%%%%
\paragraph{v2.0:} 2018/12/30

\begin{itemize}
\item
immediate forward processing
\item
added |\childdocby| mechanism
\item
manual restructured
\end{itemize}

%%%%%%%%%%%%%%%%%%%%%%%%%%%%%%%%%%%%%%%%
\paragraph{v1.6:} 2018/01/17

\begin{itemize}
\item
application for development of include files
\item
corrections to manual
\end{itemize}

%%%%%%%%%%%%%%%%%%%%%%%%%%%%%%%%%%%%%%%%
\paragraph{v1.5:} 2017/05/21

\begin{itemize}
\item
more complete structuring introduced
\item
|\childdocof| introduced
\item
|\childdoc| renamed to |\childdocmain|
\item
|\childredirect| renamed to |\childdocforward| and |\childdocforwardprefix|
and functionality expanded
\end{itemize}

%%%%%%%%%%%%%%%%%%%%%%%%%%%%%%%%%%%%%%%%
\paragraph{v1.0:} 2017/04/27

\begin{itemize}
\item
manual and install package
\item
first version published on CTAN
\end{itemize}

%%%%%%%%%%%%%%%%%%%%%%%%%%%%%%%%%%%%%%%%
\paragraph{v0.6:} 2017/04/26

\begin{itemize}
\item
redirection mechanism added
\end{itemize}

%%%%%%%%%%%%%%%%%%%%%%%%%%%%%%%%%%%%%%%%
\paragraph{v0.5:} 2017/04/26

\begin{itemize}
\item
functionality in definition file
\end{itemize}


%%%%%%%%%%%%%%%%%%%%%%%%%%%%%%%%%%%%%%%%%%%%%%%%%%%%%%%%%%%%%%%%%%%%%%%%%%%%%%%%
%%%%%%%%%%%%%%%%%%%%%%%%%%%%%%%%%%%%%%%%%%%%%%%%%%%%%%%%%%%%%%%%%%%%%%%%%%%%%%%%
%%%%%%%%%%%%%%%%%%%%%%%%%%%%%%%%%%%%%%%%%%%%%%%%%%%%%%%%%%%%%%%%%%%%%%%%%%%%%%%%
\appendix

\settowidth\MacroIndent{\rmfamily\scriptsize 000\ }

 \DocInput{childdoc.dtx}

\end{document}
%</driver>
% \fi
%
% %%%%%%%%%%%%%%%%%%%%%%%%%%%%%%%%%%%%%%%%%%%%%%%%%%%%%%%%%%%%%%%%%%%%%%%%%%%%%%
% %%%%%%%%%%%%%%%%%%%%%%%%%%%%%%%%%%%%%%%%%%%%%%%%%%%%%%%%%%%%%%%%%%%%%%%%%%%%%%
% \section{Sample}
%\iffalse
%<*samplemain>
%\fi
%
% The following presents a sample document
% with two chapters, two parts, a title page,
% a compile flag as well as three forwarding files to set the flag.
% It consists of eight |.tex| files:
% \begin{center}
% \begin{tabular}{ll}
% |cdocsamp.tex|&main file\\
% |cdocsch1.tex|&include file for chapter 1\\
% |cdocsch2.tex|&include file for chapter 2\\
% |cdocspt3.tex|&include file for part 3\\
% |cdocspt4.tex|&include file for part 4\\
% |cdocsdrf.tex|&forwarding file for main file in draft mode\\
% |cdocsfi1.tex|&forwarding file for final version of chapter 1\\
% |cdocsfi2.tex|&forwarding file for final version of chapter 2\\
% \end{tabular}
% \end{center}
% Each of the eight files can be compiled directly by the \LaTeX{} compiler.
%
% %%%%%%%%%%%%%%%%%%%%%%%%%%%%%%%%%%%%%%
% \paragraph{Main File.}
%
% The main file is called |cdocsamp.tex|.
%
% Load the \textsf{childdoc} definitions and
% declare the filename for the main document:
%    \begin{macrocode}
\input{childdoc.def}
\childdocmain{}
%    \end{macrocode}

% Optional override for |\version| flag:
%    \begin{macrocode}
%%\ifchilddoc\else\providecommand{\version}{draft}\fi
%    \end{macrocode}

% Define the default values for the |\version| flag
% (|final| for the main file and |draft| for childs):
%    \begin{macrocode}
\ifchilddoc
\providecommand{\version}{draft}
\else
\providecommand{\version}{final}
\fi
%    \end{macrocode}

% Load the standard document class:
%    \begin{macrocode}
\documentclass[12pt]{article}
%    \end{macrocode}

% Start the document body:
%    \begin{macrocode}
\begin{document}
%    \end{macrocode}

% Declare a title page.
% Print title, part of document being processed and version flag:
%    \begin{macrocode}
\addtocounter{page}{-1}
\begin{center}
{\LARGE\bfseries{}childdoc example\par}
\vspace{1cm}
\ifchilddoc
\ifchilddocmanual part\else chapter\fi:
`\childdocname' of `\childdocjob'\par
\else
main document: `\childdocjob'\par
\fi
version: \version\par
\end{center}
\newpage
%    \end{macrocode}

% Manually include selected file,
% otherwise process as usual:
%    \begin{macrocode}
\ifchilddocmanual
\section*{part `\childdocname'}
\input{\childdocname}
\else
%    \end{macrocode}

% Include the two chapters:
%    \begin{macrocode}
\include{cdocsch1}
\include{cdocsch2}
%    \end{macrocode}

% Include the two parts unless only chapters should be displayed:
%    \begin{macrocode}
\ifchilddoc\else
\section{part three}
\input{cdocspt3}
\section{part four}
\input{cdocspt4}
\fi
%    \end{macrocode}

% Process as usual until here:
%    \begin{macrocode}
\fi
%    \end{macrocode}

% End of document body:
%    \begin{macrocode}
\end{document}
%    \end{macrocode}
%\iffalse
%</samplemain>
%\fi
%
% %%%%%%%%%%%%%%%%%%%%%%%%%%%%%%%%%%%%%%
% \paragraph{Chapter Include Files.}
%
% The include files are called |cdocsch1.tex| and |cdocsch2.tex|.
%
%\iffalse
%<*samplechap1|samplechap2>
%\fi

% Optional override for |\version| flag:
%    \begin{macrocode}
%%\providecommand{\version}{final}
%    \end{macrocode}

% Include the main document:
%    \begin{macrocode}
\input{childdoc.def}
\childdocof{cdocsamp}
%    \end{macrocode}

%\iffalse
%</samplechap1|samplechap2>
%\fi
%
%\iffalse
%<*samplechap1>
%\fi
% Some text for chapter 1:
%    \begin{macrocode}
\section{one}
some text in chapter one
%    \end{macrocode}

%\iffalse
%</samplechap1>
%\fi
% Some text for chapter 2:
%\iffalse
%<*samplechap2>
%\fi
%    \begin{macrocode}
\section{two}
more text in chapter two
%    \end{macrocode}

%\iffalse
%</samplechap2>
%\fi
%
% %%%%%%%%%%%%%%%%%%%%%%%%%%%%%%%%%%%%%%
% \paragraph{Part Include Files.}
%
% The include files are called |cdocspt3.tex| and |cdocspt4.tex|.
%
%\iffalse
%<*samplepart3|samplepart4>
%\fi

% Optional override for |\version| flag:
%    \begin{macrocode}
%%\providecommand{\version}{final}
%    \end{macrocode}

% Include the main document:
%    \begin{macrocode}
\input{childdoc.def}
\childdocby{cdocsamp}
%    \end{macrocode}

%\iffalse
%</samplepart3|samplepart4>
%\fi
%
%\iffalse
%<*samplepart3>
%\fi
% Some text for part 3:
%    \begin{macrocode}
some text in part three
%    \end{macrocode}

%\iffalse
%</samplepart3>
%\fi
% Some text for part 4:
%\iffalse
%<*samplepart4>
%\fi
%    \begin{macrocode}
more text in part four
%    \end{macrocode}

%\iffalse
%</samplepart4>
%\fi
%
% %%%%%%%%%%%%%%%%%%%%%%%%%%%%%%%%%%%%%%
% \paragraph{Forwarding for a Complete Draft.}
%
% The following forwarding file |cdocsdrf.tex|
% compiles the main document in draft mode:
%\iffalse
%<*sampledraft>
%\fi
%    \begin{macrocode}
\def\version{draft}
\input{childdoc.def}
\childdocforward{cdocsamp}
%    \end{macrocode}

%\iffalse
%</sampledraft>
%\fi
%
% %%%%%%%%%%%%%%%%%%%%%%%%%%%%%%%%%%%%%%
% \paragraph{Forwarding for Final Version of the Chapters.}
%
% The following forwarding files |cdocsfn1.tex| and |cdocsfn2.tex|
% (with identical content)
% compile the final versions of the child documents
% |cdocsch1.tex| and |cdocsch2.tex|, respectively:
%\iffalse
%<*samplefinal>
%\fi
%    \begin{macrocode}
\def\version{final}
\input{childdoc.def}
\childdocforwardprefix[cdocsamp]{cdocsfn}{cdocsch}
%    \end{macrocode}

%\iffalse
%</samplefinal>
%\fi
%
% %%%%%%%%%%%%%%%%%%%%%%%%%%%%%%%%%%%%%%
% \paragraph{Command Line Processing.}
%
% The following three command lines generate the output files
% |cdocscld|, |cdocscl1| and |cdocscl2|
% which should be identical to
% |cdocsdrf|, |cdocsch1| and |cdocsfn2|, respectively:
% \begin{center}
% \begin{tabular}{l}
% |latex -jobname cdocscld \|\\
% |  "\def\version{draft}\input{childdoc.def}\childdocforward{cdocsamp}"|\\
% |latex -jobname cdocscl1 \|\\
% |  "\input{childdoc.def}\childdocforward[cdocsamp]{cdocsch1}"|\\
% |latex -jobname cdocscl2 \|\\
% |  "\def\version{final}\input{childdoc.def}\childdocforward{cdocsch2}"|
% \end{tabular}
% \end{center}
% Note that the trailing backslash on each first line
% merely continues the input to the second line
% (for convenient cut ant paste).
% Furthermore, the command |latex| can be replaced by any
% of its alternative versions such as |pdflatex|.
%
% %%%%%%%%%%%%%%%%%%%%%%%%%%%%%%%%%%%%%%%%%%%%%%%%%%%%%%%%%%%%%%%%%%%%%%%%%%%%%%
% %%%%%%%%%%%%%%%%%%%%%%%%%%%%%%%%%%%%%%%%%%%%%%%%%%%%%%%%%%%%%%%%%%%%%%%%%%%%%%
% \section{Implementation}
%\iffalse
%<*package>
%\fi
%
% This section describes the definitions file |childdoc.def|.

% The definitions cannot be loaded using |\usepackage| or |\RequirePackage|
% which has a mechanism to prevent loading a style file more than once.
% When loading the definitions by means of |\input|
% multiple instances have to be prevented manually:
%\iffalse
%This code needs to be before the `\ProvidesFile' directive
%which is defined at the beginning of this file.
%Therefore it is also placed there and commented out here.
%</package>
%<*discard>
%\fi
%    \begin{macrocode}
\ifdefined\childdocmain\endinput\fi
%    \end{macrocode}
%\iffalse
%</discard>
%<*package>
%\fi
%
% \macro{\ifchilddoc}
% \macro{\ifchilddocmanual}
% The conditional |\ifchilddoc| tells whether a
% child (true) or main (false) document is being compiled.
% The conditional |\ifchilddocmanual| tells whether
% the |\includeonly| mechanism is used (false) or
% the selection of child files must be performed manually (true).
% The definitions initialise to false:
%    \begin{macrocode}
\newif\ifchilddoc
\newif\ifchilddocmanual
%    \end{macrocode}

% \macro{\childdocname}
% \macro{\childdocjob}
% The macro |\childdocname| stores the name of the main document
% to be compiled. The macro |\childdocjob| stores the name of
% the document on which the \LaTeX{} compiler was originally invoked.
% The content of |\jobname| cannot be compared
% to filenames specified in the source due to different catcodes.
% The following code rescans |\jobname|, stores the result
% in |\childdocname| and saves a copy in |\childdocjob|:
%    \begin{macrocode}
\edef\childdocname{\scantokens\expandafter{\jobname\noexpand}}
\let\childdocjob\childdocname
%    \end{macrocode}

% \macro{\childdocdisable}
% The macro |\childdocdisable| prevents the main file
% from being processed more than once.
% At this stage, the main document command |\childdocmain|
% is assumed to be called once again where it should do nothing.
% Any subsequent call to it should prevent
% a secondary processing of the main document
% It overwrites the forwarding commands
% |\childdocof| and |\childdocforward|
% with empty macros to prevent further inclusions of the main document:
%    \begin{macrocode}
\newcommand{\childdocdisable}
{
  \renewcommand{\childdocmain}[1]{\renewcommand{\childdocmain}[1]{\endinput}}
  \renewcommand{\childdocof}[1]{}
  \renewcommand{\childdocby}[2][]{}
  \renewcommand{\childdocforward}[2][]{}
  \renewcommand{\childdocdisable}{}
}
%    \end{macrocode}

% \macro{\childdocmain}
% The macro |\childdocmain| is to be called at the top of the main file
% with nothing or the main filename (without extension) as argument.
% First, it breaks loops.
% If the argument is not empty and does not match |\childdocname|
% (which is set by the first inclusion of |childdoc.def|),
% |\ifchilddoc| is set to true, |\includeonly| is applied to the child file
% and |\jobname| is set to the main file
% (for proper handling of |.aux| files):
%    \begin{macrocode}
\newcommand{\childdocmain}[1]
{
  \childdocdisable\childdocmain{}
  \if?#1?\else
    \begingroup
      \def\childdoctmp{#1}
      \ifx\childdoctmp\childdocname
        \def\childdoctmp{}
      \else
        \def\childdoctmp
        {
          \childdoctrue
          \includeonly{\childdocname}
          \def\childdocjob{#1}
          \def\jobname{#1}
        }
      \fi
      \expandafter
    \endgroup
    \childdoctmp
  \fi
}
%    \end{macrocode}

% \macro{\childdocof}
% The command |\childdocof| redirects
% compilation to the main file |#1|.
%    \begin{macrocode}
\newcommand{\childdocof}[1]
{
  \childdocdisable
  \childdoctrue
  \includeonly{\childdocname}
  \def\jobname{#1}
  \def\childdocjob{#1}
  \input{#1}
}
%    \end{macrocode}

% \macro{\childdocby}
% The command |\childdocby| ....
%    \begin{macrocode}
\newcommand{\childdocby}[2][]
{
  \childdocdisable
  \childdoctrue
  \childdocmanualtrue
  \if?#1?\else
    \def\jobname{#2}
  \fi
  \def\childdocjob{#2}
  \input{#2}
  \endinput
}
%    \end{macrocode}

% \macro{\childdocforward}
% The command |\childdocforward| redirects
% compilation to the main file or
% (if the optional argument is given) a child file.
% Parameters are set as if the main file
% or a child file starting with |\childdocof| was compiled.
% Then compilation is handed over to the main file:
%    \begin{macrocode}
\newcommand{\childdocforward}[2][]
{
  \begingroup
    \if?#1?
      \def\childdoctmp
      {
        \def\childdocname{#2}
        \def\childdocjob{#2}
        \def\jobname{#2}
        \input{#2}
        \endinput
      }
    \else
      \def\childdoctmp
      {
        \childdocdisable
        \def\childdocname{#2}
        \childdoctrue
        \includeonly{#2}
        \def\childdocjob{#1}
        \def\jobname{#1}
        \input{#1}
        \endinput
      }
    \fi
    \expandafter
  \endgroup
  \childdoctmp
}
%    \end{macrocode}

% \macro{\childdocforwardprefix}
% The command |\childdocforwardprefix| redirects
% compilation to the main or a child file by means of a pattern.
% The prefix |#1| in the current filename is replaced by |#2|
% and the suffix of the current filename is kept
% (it is assumed that the filename does not contain the substring `|~~~|'
% which is used as a delimiter).
% Compilation is handed over to the new file by |\childdocforward|:
%    \begin{macrocode}
\newcommand{\childdocforwardprefix}[3][]
{
  \begingroup
    \def\childdocextract #2##1~~~{\def\childdoctmp{\childdocforward[#1]{#3##1}}}
    \expandafter\childdocextract\childdocname~~~
    \expandafter
  \endgroup
  \childdoctmp
}
%    \end{macrocode}

% \macro{\childdoc}
% The deprecated macro |\childdoc| is a legacy version of |\childdocmain|:
%    \begin{macrocode}
\newcommand{\childdoc}{\childdocmain}
%    \end{macrocode}

% \macro{\childdocredirect}
% The deprecated macro |\childdocredirect| is a legacy version
% of |\childdocforward| and |\childdocforwardprefix|:
%    \begin{macrocode}
\newcommand{\childdocredirect}[2][]
{
  \begingroup
    \if?#1?
      \def\childdoctmp{\childdocforward{#2}}
    \else
      \def\childdoctmp{\childdocforwardprefix{#1}{#2}}
    \fi
    \expandafter
  \endgroup
  \childdoctmp
}
%    \end{macrocode}

%\iffalse
%</package>
%\fi
%
\endinput
|
and perform the replacements as outlined below.
Instead of |\childdocmain{|\textit{main}|}| add the following code
to the top of the main file:
%
\begin{center}
\begin{tabular}{l}
|\||ifdefined\childdocname\endinput\||fi\newif\ifchilddoc|\\
|\edef\childdocname{\scantokens\expandafter{\jobname\noexpand}}|\\
|\def\childdocmain{|\textit{main}|}\||ifx\childdocmain\childdocname\||else|\\
|\childdoctrue\includeonly{\childdocname}\let\jobname\childdocmain\||fi|\\
\end{tabular}
\end{center}
%
Instead of |\childdocof{|\textit{main}|}| just include the main file
at the top of each child file:
%
\begin{center}
|\input{|\textit{main}|}|
\end{center}
%
A simple redirection |\childdocforward{|\textit{dest}|}| is achieved by:
%
\begin{center}
|\def\jobname{|\textit{dest}|}\input{\jobname}|
\end{center}
%
The redirection with prefix
|\childdocforwardprefix[|\textit{prefix}|]{|\textit{dest}|}|
is accomplished by:
%
\begin{center}
\begin{tabular}{l}
|{\edef\jobname{\scantokens\expandafter{\jobname\noexpand}}|\\
|\def\redirectjob |\textit{prefix}|#1~~~{\gdef\jobname{|\textit{dest}|#1}}|\\
|\expandafter\redirectjob\jobname~~~}\input{\jobname}|
\end{tabular}
\end{center}

In an alternative approach,
child documents can be compiled by a specific command line
without additional code or specific definitions:
%
\begin{center}
|... -jobname "|\textit{target}|" "|[\textit{flags}]%
|\includeonly{|\textit{dest}|}\input{|\textit{main}|}"|
\end{center}
%

%%%%%%%%%%%%%%%%%%%%%%%%%%%%%%%%%%%%%%%%%%%%%%%%%%%%%%%%%%%%%%%%%%%%%%%%%%%%%%%%
%%%%%%%%%%%%%%%%%%%%%%%%%%%%%%%%%%%%%%%%%%%%%%%%%%%%%%%%%%%%%%%%%%%%%%%%%%%%%%%%
\section{Information}

%%%%%%%%%%%%%%%%%%%%%%%%%%%%%%%%%%%%%%%%%%%%%%%%%%%%%%%%%%%%%%%%%%%%%%%%%%%%%%%%
\subsection{Copyright}

Copyright \copyright{} 2017--2018 Niklas Beisert

This work may be distributed and/or modified under the
conditions of the \LaTeX{} Project Public License, either version 1.3
of this license or (at your option) any later version.
The latest version of this license is in
  \url{http://www.latex-project.org/lppl.txt}
and version 1.3 or later is part of all distributions of \LaTeX{}
version 2005/12/01 or later.

This work has the LPPL maintenance status `maintained'.

The Current Maintainer of this work is Niklas Beisert.

This work consists of the files |README.txt|, |childdoc.ins| and |childdoc.dtx|
as well as the derived files |childdoc.def|, |cdocsamp.tex|
with |cdocsch1.tex|, |cdocsch2.tex|, |cdocspt3.tex|, |cdocspt4.tex|,
|cdocsdrf.tex|, |cdocsfn1.tex|, |cdocsfn2.tex|
as well as |childdoc.pdf|.

%%%%%%%%%%%%%%%%%%%%%%%%%%%%%%%%%%%%%%%%%%%%%%%%%%%%%%%%%%%%%%%%%%%%%%%%%%%%%%%%
\subsection{Files and Installation}

The package consists of the files:
%
\begin{center}
\begin{tabular}{ll}
    |README.txt|   & readme file \\
    |childdoc.ins| & installation file \\
    |childdoc.dtx| & source file \\
    |childdoc.def| & definition file \\
    |cdocsamp.tex| & sample main file \\
    |cdocsch1.tex| & sample include file \\
    |cdocsch2.tex| & sample include file \\
    |cdocspt3.tex| & sample part file \\
    |cdocspt4.tex| & sample part file \\
    |cdocsdrf.tex| & sample redirection file \\
    |cdocsfn1.tex| & sample redirection file \\
    |cdocsfn2.tex| & sample redirection file \\
    |childdoc.pdf| & manual
\end{tabular}
\end{center}
%
The distribution consists of the files
|README.txt|, |childdoc.ins| and |childdoc.dtx|.
%
\begin{itemize}
\item
Run (pdf)\LaTeX{} on |childdoc.dtx|
to compile the manual |childdoc.pdf| (this file).
\item
Run \LaTeX{} on |childdoc.ins| to create the definitions file |childdoc.def|
and the sample |cdocsamp.tex| with include files
|cdocsch1.tex|, |cdocsch2.tex|, |cdocspt3.tex|, |cdocspt4.tex|,
|cdocsdrf.tex|, |cdocsfn1.tex|, |cdocsfn2.tex|.
Then copy the file |childdoc.def| to an appropriate directory of your \LaTeX{}
distribution, e.g.\ \textit{texmf-root}|/tex/latex/childdoc|.
\end{itemize}

%%%%%%%%%%%%%%%%%%%%%%%%%%%%%%%%%%%%%%%%%%%%%%%%%%%%%%%%%%%%%%%%%%%%%%%%%%%%%%%%
\subsection{Related CTAN Packages}

There are several other packages which offer a similar functionality:
%
\begin{itemize}
\item
The packages
\href{http://ctan.org/pkg/docmute}{\textsf{docmute}},
\href{http://ctan.org/pkg/includex}{\textsf{includex}} and
\href{http://ctan.org/pkg/standalone}{\textsf{standalone}}
provide commands to include only the document body of
a child file thus allowing both files to be compiled individually.
\item
The packages \href{http://ctan.org/pkg/subdocs}{\textsf{subdocs}}
and \href{http://ctan.org/pkg/subfiles}{\textsf{subfiles}}
provide structures in which the main and child documents can be
encapsulated and allowing them to be compiled individually.
The inclusion mechanism is different from the conventional |\include|.
\item
The package \href{http://ctan.org/pkg/combine}{\textsf{combine}}
is an elaborate solution to combine several documents into one.
\end{itemize}
%
See also the CTAN topic \href{http://ctan.org/topic/subdocs}{\textsf{subdocs}}
for further related packages.
The present package differs from the above solutions in that
a document structure constructed with the conventional |\include| mechanism
just needs two extra commands at the top of every file
such that all constituent files can be compiled individually.

%%%%%%%%%%%%%%%%%%%%%%%%%%%%%%%%%%%%%%%%%%%%%%%%%%%%%%%%%%%%%%%%%%%%%%%%%%%%%%%%
%\subsection{Feature Suggestions}
%
%The following is a list of features which may be useful for future
%versions of this package:
%%
%\begin{itemize}
%\item
%\ldots
%\end{itemize}

%%%%%%%%%%%%%%%%%%%%%%%%%%%%%%%%%%%%%%%%%%%%%%%%%%%%%%%%%%%%%%%%%%%%%%%%%%%%%%%%
\subsection{Revision History}

%%%%%%%%%%%%%%%%%%%%%%%%%%%%%%%%%%%%%%%%
\paragraph{v2.0:} 2018/12/30

\begin{itemize}
\item
immediate forward processing
\item
added |\childdocby| mechanism
\item
manual restructured
\end{itemize}

%%%%%%%%%%%%%%%%%%%%%%%%%%%%%%%%%%%%%%%%
\paragraph{v1.6:} 2018/01/17

\begin{itemize}
\item
application for development of include files
\item
corrections to manual
\end{itemize}

%%%%%%%%%%%%%%%%%%%%%%%%%%%%%%%%%%%%%%%%
\paragraph{v1.5:} 2017/05/21

\begin{itemize}
\item
more complete structuring introduced
\item
|\childdocof| introduced
\item
|\childdoc| renamed to |\childdocmain|
\item
|\childredirect| renamed to |\childdocforward| and |\childdocforwardprefix|
and functionality expanded
\end{itemize}

%%%%%%%%%%%%%%%%%%%%%%%%%%%%%%%%%%%%%%%%
\paragraph{v1.0:} 2017/04/27

\begin{itemize}
\item
manual and install package
\item
first version published on CTAN
\end{itemize}

%%%%%%%%%%%%%%%%%%%%%%%%%%%%%%%%%%%%%%%%
\paragraph{v0.6:} 2017/04/26

\begin{itemize}
\item
redirection mechanism added
\end{itemize}

%%%%%%%%%%%%%%%%%%%%%%%%%%%%%%%%%%%%%%%%
\paragraph{v0.5:} 2017/04/26

\begin{itemize}
\item
functionality in definition file
\end{itemize}


%%%%%%%%%%%%%%%%%%%%%%%%%%%%%%%%%%%%%%%%%%%%%%%%%%%%%%%%%%%%%%%%%%%%%%%%%%%%%%%%
%%%%%%%%%%%%%%%%%%%%%%%%%%%%%%%%%%%%%%%%%%%%%%%%%%%%%%%%%%%%%%%%%%%%%%%%%%%%%%%%
%%%%%%%%%%%%%%%%%%%%%%%%%%%%%%%%%%%%%%%%%%%%%%%%%%%%%%%%%%%%%%%%%%%%%%%%%%%%%%%%
\appendix

\settowidth\MacroIndent{\rmfamily\scriptsize 000\ }

 \DocInput{childdoc.dtx}

\end{document}
%</driver>
% \fi
%
% %%%%%%%%%%%%%%%%%%%%%%%%%%%%%%%%%%%%%%%%%%%%%%%%%%%%%%%%%%%%%%%%%%%%%%%%%%%%%%
% %%%%%%%%%%%%%%%%%%%%%%%%%%%%%%%%%%%%%%%%%%%%%%%%%%%%%%%%%%%%%%%%%%%%%%%%%%%%%%
% \section{Sample}
%\iffalse
%<*samplemain>
%\fi
%
% The following presents a sample document
% with two chapters, two parts, a title page,
% a compile flag as well as three forwarding files to set the flag.
% It consists of eight |.tex| files:
% \begin{center}
% \begin{tabular}{ll}
% |cdocsamp.tex|&main file\\
% |cdocsch1.tex|&include file for chapter 1\\
% |cdocsch2.tex|&include file for chapter 2\\
% |cdocspt3.tex|&include file for part 3\\
% |cdocspt4.tex|&include file for part 4\\
% |cdocsdrf.tex|&forwarding file for main file in draft mode\\
% |cdocsfi1.tex|&forwarding file for final version of chapter 1\\
% |cdocsfi2.tex|&forwarding file for final version of chapter 2\\
% \end{tabular}
% \end{center}
% Each of the eight files can be compiled directly by the \LaTeX{} compiler.
%
% %%%%%%%%%%%%%%%%%%%%%%%%%%%%%%%%%%%%%%
% \paragraph{Main File.}
%
% The main file is called |cdocsamp.tex|.
%
% Load the \textsf{childdoc} definitions and
% declare the filename for the main document:
%    \begin{macrocode}
% \iffalse
%
% childdoc.dtx Copyright (C) 2017-2018 Niklas Beisert
%
% This work may be distributed and/or modified under the
% conditions of the LaTeX Project Public License, either version 1.3
% of this license or (at your option) any later version.
% The latest version of this license is in
%   http://www.latex-project.org/lppl.txt
% and version 1.3 or later is part of all distributions of LaTeX
% version 2005/12/01 or later.
%
% This work has the LPPL maintenance status `maintained'.
%
% The Current Maintainer of this work is Niklas Beisert.
%
% This work consists of the files childdoc.dtx and childdoc.ins
% and the derived files childdoc.def and cdocsamp.tex with
% cdocsch1.tex, cdocsch2.tex, cdocsdrf.tex, cdocsfn1.tex, cdocsfn2.tex.
%
%<package>\ifdefined\childdocmain\endinput\fi
%<package>\ProvidesFile{childdoc.def}[2018/12/30 v2.0 child document driver]
%<samplemain>\ProvidesFile{cdocsamp.tex}[2018/12/30 v2.0 sample for childdoc]
%<*driver>
%\ProvidesFile{childdoc.drv}[2018/12/30 v2.0 childdoc reference manual file]
\PassOptionsToClass{10pt,a4paper}{article}
\documentclass{ltxdoc}

\usepackage[margin=35mm]{geometry}
\usepackage{hyperref}
\usepackage{hyperxmp}
\usepackage[usenames]{color}

\hypersetup{colorlinks=true}
\hypersetup{pdfstartview=FitH}
\hypersetup{pdfpagemode=UseNone}
\hypersetup{pdfsource={}}
\hypersetup{pdflang={en-UK}}
\hypersetup{pdfcopyright={Copyright 2017-2018 Niklas Beisert.
  This work may be distributed and/or modified under the
  conditions of the LaTeX Project Public License, either version 1.3
  of this license or (at your option) any later version.}}
\hypersetup{pdflicenseurl={http://www.latex-project.org/lppl.txt}}
\hypersetup{pdfcontactaddress={ETH Zurich, ITP, HIT K,
  Wolfgang-Pauli-Strasse 27}}
\hypersetup{pdfcontactpostcode={8093}}
\hypersetup{pdfcontactcity={Zurich}}
\hypersetup{pdfcontactcountry={Switzerland}}
\hypersetup{pdfcontactemail={nbeisert@itp.phys.ethz.ch}}
\hypersetup{pdfcontacturl={http://people.phys.ethz.ch/\xmptilde nbeisert/}}

\newcommand{\secref}[1]{\hyperref[#1]{section \ref*{#1}}}

\parskip1ex
\parindent0pt
\let\olditemize\itemize
\def\itemize{\olditemize\parskip0pt}

\begin{document}

\title{The \textsf{childdoc} Package}
\hypersetup{pdftitle={The childdoc Package}}
\author{Niklas Beisert\\[2ex]
  Institut f\"ur Theoretische Physik\\
  Eidgen\"ossische Technische Hochschule Z\"urich\\
  Wolfgang-Pauli-Strasse 27, 8093 Z\"urich, Switzerland\\[1ex]
  \href{mailto:nbeisert@itp.phys.ethz.ch}
  {\texttt{nbeisert@itp.phys.ethz.ch}}}
\hypersetup{pdfauthor={Niklas Beisert}}
\hypersetup{pdfsubject={Manual for the LaTeX2e Package childdoc}}
\date{30 December 2018, \textsf{v2.0}}
\maketitle

\begin{abstract}\noindent
\textsf{childdoc} is a \LaTeXe{} package
that enables the direct compilation
of document sections included by |\include|
to individual files.
\end{abstract}

\begingroup
\parskip0ex
\tableofcontents
\endgroup

%%%%%%%%%%%%%%%%%%%%%%%%%%%%%%%%%%%%%%%%%%%%%%%%%%%%%%%%%%%%%%%%%%%%%%%%%%%%%%%%
%%%%%%%%%%%%%%%%%%%%%%%%%%%%%%%%%%%%%%%%%%%%%%%%%%%%%%%%%%%%%%%%%%%%%%%%%%%%%%%%
\section{Introduction}

\LaTeX{} provides a mechanism to structure a large document (such as a book)
into a main file and several child files (containing the chapters)
using the |\include| command.
This mechanism is beneficial for documents
which span hundreds of pages in order to
make the source file(s) more manageable.
Moreover, compilation can be restricted to
selected child files by means of the |\includeonly| command.
The latter feature can be used to reduce the compilation time while editing
(this was significantly more useful in the earlier days of \LaTeX{})
or to generate a smaller document which is easier to navigate.
Another application of |\includeonly| is to generate
documents consisting of selected parts of the complete document.

However, there are a few drawbacks of the plain |\include| mechanism:
\begin{itemize}
\item
The child files cannot be compiled on their own,
they can only be compiled via the main file.
A naive editing environment
(such as a text editor with an option
to have the current file processed by \LaTeX)
may require one to switch to the main file before compiling;
attempting to compile the child file produces errors.
\item
The main file must be modified (each time)
to adjust the |\includeonly| command
to the present needs. This easily leaves the main file in a messy state.
\item
The generated document will always carry the filename
of the main document. This is inconvenient if
several child files are to be compiled and
to be kept for distribution.
\end{itemize}

The present package provides a simple interface
to make child files individually compilable by \LaTeX{}.
Compiling a child file then has the same effect as compiling
the main file with an |\includeonly| command
to select the appropriate child.
Moreover the generated document will carry the name of the child
rather than the main file.
This resolves all three above issues.

This feature is meant to make the editing of books,
thesis documents and lecture notes somewhat more convenient.
However, the package can also be used efficiently for
composing a series of documents (such as exercise sheets)
which are typically distributed individually.
It then assists the author in generating the individual documents
(potentially in different versions)
as well as a document containing the collected series.
Another application is in developing style files
or other kinds of included material
where compilation of the style file could redirect
to a sample or test file.

%%%%%%%%%%%%%%%%%%%%%%%%%%%%%%%%%%%%%%%%%%%%%%%%%%%%%%%%%%%%%%%%%%%%%%%%%%%%%%%%
%%%%%%%%%%%%%%%%%%%%%%%%%%%%%%%%%%%%%%%%%%%%%%%%%%%%%%%%%%%%%%%%%%%%%%%%%%%%%%%%
\section{Usage}

First of all, the package \textsf{childdoc} is \emph{not} a standard
\LaTeXe{} |.sty| style file! Therefore it needs to be invoked in
a non-standard way.

%%%%%%%%%%%%%%%%%%%%%%%%%%%%%%%%%%%%%%%%%%%%%%%%%%%%%%%%%%%%%%%%%%%%%%%%%%%%%%%%
\subsection{Included Files}
\label{sec:include}

%%%%%%%%%%%%%%%%%%%%%%%%%%%%%%%%%%%%%%%%
\DescribeMacro{\childdocmain}
To use the package, add the commands
\begin{center}
\begin{tabular}{l}
|\input{childdoc.def}|\\
|\childdocmain{}|\\
\end{tabular}
\end{center}
at the very top of the main \LaTeX{} file,
in particular \emph{before} the |\documentclass| statement!
The argument of |\childdocmain| should be left empty
(but it must be present).

%%%%%%%%%%%%%%%%%%%%%%%%%%%%%%%%%%%%%%%%
\DescribeMacro{\childdocof}
Furthermore, add the commands
\begin{center}
\begin{tabular}{l}
|\input{childdoc.def}|\\
|\childdocof{|\textit{main}|}|\\
\end{tabular}
\end{center}
at the top of every child file \textit{child}
which is included by |\include{|\textit{child}|}|
from within the main file
(or at least for those files to be compiled individually).
The argument \textit{main} must be the filename of the main file.

There are a couple of
considerations in setting up the main and child documents:

%%%%%%%%%%%%%%%%%%%%%%%%%%%%%%%%%%%%%%%%
\paragraph{Restrictions.}

Please note the following restrictions:
\begin{itemize}
\item
|\childdocmain| must be called with one argument \textit{main}
to ensure compatibility with earlier version of the package.
It must either be empty (|\childdocmain{}|)
or precisely match the filename of the main file in which it is specified.
See \secref{sec:detection} for further information.
\item
The filename \textit{main} must be specified without the |.tex| extension.
\item
The filename \textit{main} is case sensitive
(even in case-insensitive file systems)
due to internal string comparison.
\item
The argument \textit{main} should be fully expanded, it cannot be a macro.
\item
Subdirectories and special characters should be avoided in filenames.
\item
The command |\childdocmain{|\textit{main}|}| must be followed by a whitespace.
It should not be followed immediately by another command
or by a comment mark `|%|'.
This is because the \TeX{} parser reads the token immediately following
the argument of |\childdocmain| and puts it
at the beginning of every child section;
however, a white\-space is ignored.
\end{itemize}

%%%%%%%%%%%%%%%%%%%%%%%%%%%%%%%%%%%%%%%%
\paragraph{Content of Main File.}

It is advisable to place all content in the child files included by |\include|.
Any output contained in the main file will appear in all child documents
unless suppressed manually;
it cannot be suppressed automatically by the |\includeonly| directive
and thus should normally be avoided.
A method to include some content in the main file
by means of conditional processing is described in \secref{sec:conditional}.

%%%%%%%%%%%%%%%%%%%%%%%%%%%%%%%%%%%%%%%%
\paragraph{Page Numbering.}

When only a part of the document is compiled,
the appropriate numbering of pages
(as well as other status parameters)
is determined from the |.aux| files.
The latter contain information from previous passes.
However this information needs to propagate through
all intermediate child documents.
Therefore the page numbering in child documents may well
be inconsistent until the complete document is compiled at least once.

A useful (if unconventional) way to always ensure a consistent
page numbering is to restart the numbering in each child document
and denote the pages by `\textit{child}|.|\textit{page}'
where \textit{child} represents the chapter/section number of the child file.
This can be achieved by the command
|\numberwithin{page}{|\textit{child}|}|
of the \textsf{amsmath} package
where \textit{child} can be |chapter| or |section|
depending on the chosen structuring.
Alternatively, one can modify the macro |\thepage| appropriately
and reset the counter |page| at the start of each child file.

%%%%%%%%%%%%%%%%%%%%%%%%%%%%%%%%%%%%%%%%%%%%%%%%%%%%%%%%%%%%%%%%%%%%%%%%%%%%%%%%
\subsection{Conditional Processing}
\label{sec:conditional}

The package provides a mechanism to compile different versions
of a document. To customise the versions further some conditional processing
can come in handy to distinguish which version is being compiled.
The package provides two macros to describe the compilation context:

%%%%%%%%%%%%%%%%%%%%%%%%%%%%%%%%%%%%%%%%
\DescribeMacro{\ifchilddoc}
The conditional |\ifchilddoc| distinguishes between the compilation of
child documents and the main document:
%
\begin{center}
|\ifchilddoc |\textit{child-code}| |[|\||else |\textit{main-code}]| \||fi|
\end{center}

%%%%%%%%%%%%%%%%%%%%%%%%%%%%%%%%%%%%%%%%
\DescribeMacro{\childdocname}
\DescribeMacro{\childdocjob}
The macro |\childdocname| contains the filename (without extension)
of the main or child file being processed.
Note that |\childdocjob| will always contain the name of the main file.

%%%%%%%%%%%%%%%%%%%%%%%%%%%%%%%%%%%%%%%%
\paragraph{Title Page.}

Conditional processing can be used to include a title or banner page
in the main document when proper precautions are taken.
Importantly, the code in the main file should ensure that the page counter
(as well as other status parameters which are stored in the |.aux| files)
takes the same value after the conditional processing.
Otherwise the page numbers may take divergent values
depending on which part is compiled.

For example, a title page could be declared by:
%
\begin{center}
\begin{tabular}{l}
|\ifchilddoc\||else|\\
|\addtocounter{page}{-1}|\\
\textit{code for title page}\\
|\newpage|\\
|\||fi|
\end{tabular}
\end{center}
%
A banner page for the child documents can be generated by:
%
\begin{center}
\begin{tabular}{l}
|\ifchilddoc|\\
|\addtocounter{page}{-1}|\\
\textit{code for banner page}\\
|\newpage|\\
|\||fi|
\end{tabular}
\end{center}
%
Here one could write a message such as:
\begin{center}
|This is the part \childdocname{} of \childdocjob{}.|
\end{center}

%%%%%%%%%%%%%%%%%%%%%%%%%%%%%%%%%%%%%%%%%%%%%%%%%%%%%%%%%%%%%%%%%%%%%%%%%%%%%%%%
\subsection{Flags}
\label{sec:flags}

The package makes it easy to generate different versions
of the main or child documents.
To this end compilation flags can be defined
and assigned different default values.
They will be particularly useful in conjunction
with the forwarding mechanism described in \secref{sec:forward}.

For example, it may be useful to have a flag |\version|
which can be set to |draft| or |final|.
The document source will contain some conditional code
depending on the value of |\version|.
Suppose further, the flag should default to |final| for the main file
and to |draft| for child files
which is a natural assignment for editing the document.
This is achieved by placing the following code
in the preamble of the main document
(below the |\childdocmain| directive):
%
\begin{center}
\begin{tabular}{l}
|\ifchilddoc|\\
|\providecommand{\version}{draft}|\\
|\||else|\\
|\providecommand{\version}{final}|\\
|\||fi|
\end{tabular}
\end{center}
%
The definition by |\providecommand| makes sure
that previous definitions are not overwritten.
Further statements |\providecommand{\version}{...}|
can thus be added before the above code to override it.

For the main file, one might add a line
(between |\childdocmain| and the above block)
%
\begin{center}
|%\ifchilddoc\||else\providecommand{\version}{draft}\||fi|
\end{center}
%
which can be uncommented to produce a draft version.
Likewise one can add a line to the very top of a child file
(above the |\childdocof{|\textit{main}|}| directive)
%
\begin{center}
|%\providecommand{\version}{final}|
\end{center}
%
which can be uncommented to produce the final version of this child document.

%%%%%%%%%%%%%%%%%%%%%%%%%%%%%%%%%%%%%%%%%%%%%%%%%%%%%%%%%%%%%%%%%%%%%%%%%%%%%%%%
\subsection{Forwarding}
\label{sec:forward}

Different versions of the main or child documents
using compilation flags as described in \secref{sec:flags}
can be (permanently) stored in different files
for convenient compilation, viewing and distribution.
To this end, the package defines a command
to pass on compilation to a different file:

%%%%%%%%%%%%%%%%%%%%%%%%%%%%%%%%%%%%%%%%
\DescribeMacro{\childdocforward}
The command |\childdocforward| redirects processing to
another source file:
%
\begin{center}
\begin{tabular}{l}
|\input{childdoc.def}|\\
|\childdocforward[|\textit{main}|]{|\textit{dest}|}|\\
\end{tabular}
\end{center}
%
The argument \textit{dest} is the destination file
(without extension).
It should be the main file or one of the child files.
Note that further \textsf{childdoc} directives
such as |\childdocof| and |\childdocforward|
in the indicated file will be processed in this form.
The optional argument \textit{main}
passes on directly to the main file \textit{main}
while pretending to compile the child \textit{dest}.
This form behaves as if \textit{dest}
issues |\childdocof{|\textit{main}|}| right away,
and no further \textsf{childdoc} directives will be processed.

%%%%%%%%%%%%%%%%%%%%%%%%%%%%%%%%%%%%%%%%
\DescribeMacro{\...prefix}
In the alternative form |\childdocforwardprefix|,
%
\begin{center}
\begin{tabular}{l}
|\input{childdoc.def}|\\
|\childdocforwardprefix[|\textit{main}|]{|\textit{prefix}|}{|\textit{dest}|}|
\end{tabular}
\end{center}
%
the destination file is determined by a pattern
depending on the current file:
To make this work, the current file must be called
`{\textit{prefix}\hspace{0.2em}\textit{suffix}}'
with \textit{prefix} matching precisely the argument.
Processing is then passed on to the file
`{\textit{dest}\hspace{0.2em}\textit{suffix}}'.
Surely, the same effect is achieved by
directly specifying the
argument `{\textit{dest}\hspace{0.2em}\textit{suffix}}'
in the first form.
However, that requires to set up a different file
for each child. With the alternative form of the command
all these files can have exactly the same content
which simplifies setting them up and maintaining them.

For example, the following file |draft.tex|
with a compilation flag |\version| as described in \secref{sec:flags}
compiles the main document as a draft:
%
\begin{center}
\begin{tabular}{l}
|\def\version{draft}|\\
|\input{childdoc.def}|\\
|\childdocforward{|\textit{main}|}|
\end{tabular}
\end{center}
%
Likewise, the following files |final|\textit{nn}|.tex|
compile the final version of the child document
|child|\textit{nn}|.tex|:
%
\begin{center}
\begin{tabular}{l}
|\def\version{final}|\\
|\input{childdoc.def}|\\
|\childdocforwardprefix{final}{child}|
\end{tabular}
\end{center}
%

Note that when several versions of a main file and/or of each child file
are to be generated, it may be convenient to set up a |Makefile| or
shell script to automatise the process.

%%%%%%%%%%%%%%%%%%%%%%%%%%%%%%%%%%%%%%%%%%%%%%%%%%%%%%%%%%%%%%%%%%%%%%%%%%%%%%%%
\subsection{Command Line Processing}
\label{sec:commandline}

The effect of redirection files can also be achieved by invoking
the \LaTeX{} compiler with a more elaborate command line.
Most conveniently this should be done as part
of a shell script or a |Makefile|.

When using \textsf{childdoc} in the main file, the following
command lines effectively perform a redirection
(note that depending on the shell being used,
backslashes may have to be doubled: `|\|' $\to$ `|\\|'):
%
\begin{center}
|... -jobname "|\textit{target}|" |\\|"|[\textit{flags}]%
|\input{childdoc.def}\childdocforward[|\textit{main}|]{|\textit{dest}|}"|
\end{center}
%
Here \textit{target} is the name of the output file,
\textit{main} is the name of the main file
and \textit{dest} is the name of the main or child file to be processed
(all filenames without extensions).
The optional argument \textit{main} can be omitted
if \textit{main} matches \textit{dest}.
Optionally, compilation \textit{flags} can be defined via |\def| commands.
This command line makes the \TeX{} engine believe
it is compiling the file \textit{target}
whose content is specified as the latter parameter.
The provided code then forwards the processing to
\textit{main} or \textit{dest} as described in \secref{sec:forward}.

%%%%%%%%%%%%%%%%%%%%%%%%%%%%%%%%%%%%%%%%%%%%%%%%%%%%%%%%%%%%%%%%%%%%%%%%%%%%%%%%
\subsection{Include by Input}
\label{sec:input}

Including child documents by |\include| has some restrictions by design.
Most notably, the content of a child document always occupies
its own set of pages; pages cannot be shared between child documents.
Usually, this behaviour makes perfect sense
because each child document contain an essential part of the document.
However, in some situations it may be desirable to compose
a document from a collection of parts
without having mandatory page breaks between then.
For this case, the package
provides a mechanism to include parts
by |\input| which can also be processed individually.
However, by construction this mechanism
requires manual handling of the content to be output.

%%%%%%%%%%%%%%%%%%%%%%%%%%%%%%%%%%%%%%%%
\DescribeMacro{\ifchilddocmanual}
The main file should be prepared as usual, see \secref{sec:include}.
However, the document body must make a distinction
between processing of an individual part and of the main document, e.g.:
%
\begin{center}
\begin{tabular}{l}
|\ifchilddocmanual|\\
|\input{\childdocname}|\\
|\||else|\\
\textit{document body with }|\input{|\textit{part}|}|\\
|\||fi|
\end{tabular}
\end{center}
%
The conditional |\ifchilddocmanual| is true whenever
a part to be included by |\input| is being compiled,
and the name of the part is stored in |\childdocname|.

%%%%%%%%%%%%%%%%%%%%%%%%%%%%%%%%%%%%%%%%
\DescribeMacro{\childdocby}
Each part to be included by |\input| should start with:
%
\begin{center}
\begin{tabular}{l}
|\input{childdoc.def}|\\
|\childdocby{|\textit{main}|}|\\
\end{tabular}
\end{center}
%
The directive |\childdocby| is similar to |\childdocof|
described in \secref{sec:include},
but the subsequent selection of content must be done manually.
To that end, both |\ifchilddoc| and |\ifchilddocmanual|
will be true upon processing of a part,
and the name of the part is stored in |\childdocname|.
Note that |\jobname| will be set to the filename of the current part
so that each part receives an individual |.aux| file
that does not interfere with the |.aux| file(s) of the main document.
This behaviour can be altered by the alternative form
|\childdocby[*]{|\textit{main}|}| (with a non-empty optional argument)
which uses the |.aux| file of the main document
by setting |\jobname| to \textit{main}.

%%%%%%%%%%%%%%%%%%%%%%%%%%%%%%%%%%%%%%%%%%%%%%%%%%%%%%%%%%%%%%%%%%%%%%%%%%%%%%%%
\subsection{Driver Development}
\label{sec:driver}

The \textsf{childdoc} mechanism can also be use for the development
of definition files such as \LaTeX{} styles or classes.
This case differs from the above setup with multiple parts
included by |\include| in that no |\includeonly| should be invoked.
This can be achieved by starting the include file
(before |\ProvidesPackage|) with:
%
\begin{center}
\begin{tabular}{l}
|\input{childdoc.def}|\\
|\childdocforward{|\textit{main}|}|\\
\end{tabular}
\end{center}
%
or alternatively with:
%
\begin{center}
\begin{tabular}{l}
|\input{childdoc.def}|\\
|\childdocby{|\textit{main}|}|\\
\end{tabular}
\end{center}
%
Both forms have slightly different effects as described above.
The main file is prepared as usual, see \secref{sec:include}.

%%%%%%%%%%%%%%%%%%%%%%%%%%%%%%%%%%%%%%%%%%%%%%%%%%%%%%%%%%%%%%%%%%%%%%%%%%%%%%%%
\subsection{Legacy Detection}
\label{sec:detection}

The directive |\childdocmain| in the main file can detect
whether the complete document or merely a child is to be compiled
even without using the directive |\childdocof|.
This method is deprecated because it is less robust
and there is no compelling reason to use it;
it is merely provided for backward compatibility
and it may be removed in future versions.

If the detection mechanism is to be used,
it is mandatory to correctly specify
the filename of the main file as the argument of |\childdocmain|:
%
\begin{center}
\begin{tabular}{l}
|\input{childdoc.def}|\\
|\childdocmain{|\textit{main}|}|\\
\end{tabular}
\end{center}
%
If |\jobname| does not match the argument \textit{main} of |\childdocmain|,
it is assumed that |\jobname| points to the child file to be compiled.
When using |\childdocmain| with the main file specified as argument,
it suffices to start a child file
with just |\input{|\textit{main}|}|
without loading of the package and using |\childdocof|.
If instead all processing is done
with the appropriate \textsf{childdoc} directives,
the argument of \textit{main} of |\childdocmain| can be empty.

An alternative version of the command line processing described
in \secref{sec:commandline} using the detection mechanism reads:
%
\begin{center}
|... -jobname "|\textit{target}|" "|[\textit{flags}]%
[|\def\jobname{|\textit{dest}|}|]|\input{|\textit{main}|}"|
\end{center}

%%%%%%%%%%%%%%%%%%%%%%%%%%%%%%%%%%%%%%%%%%%%%%%%%%%%%%%%%%%%%%%%%%%%%%%%%%%%%%%%
\subsection{Manual Code}
\label{sec:manual}

In case one cannot be certain whether the definitions file |childdoc.def|
is installed on the target \TeX{} distribution
and one prefers not to ship it,
it is conceivable to paste a few relevant commands into the sources.

To that end, drop all statements |\input{childdoc.def}|
and perform the replacements as outlined below.
Instead of |\childdocmain{|\textit{main}|}| add the following code
to the top of the main file:
%
\begin{center}
\begin{tabular}{l}
|\||ifdefined\childdocname\endinput\||fi\newif\ifchilddoc|\\
|\edef\childdocname{\scantokens\expandafter{\jobname\noexpand}}|\\
|\def\childdocmain{|\textit{main}|}\||ifx\childdocmain\childdocname\||else|\\
|\childdoctrue\includeonly{\childdocname}\let\jobname\childdocmain\||fi|\\
\end{tabular}
\end{center}
%
Instead of |\childdocof{|\textit{main}|}| just include the main file
at the top of each child file:
%
\begin{center}
|\input{|\textit{main}|}|
\end{center}
%
A simple redirection |\childdocforward{|\textit{dest}|}| is achieved by:
%
\begin{center}
|\def\jobname{|\textit{dest}|}\input{\jobname}|
\end{center}
%
The redirection with prefix
|\childdocforwardprefix[|\textit{prefix}|]{|\textit{dest}|}|
is accomplished by:
%
\begin{center}
\begin{tabular}{l}
|{\edef\jobname{\scantokens\expandafter{\jobname\noexpand}}|\\
|\def\redirectjob |\textit{prefix}|#1~~~{\gdef\jobname{|\textit{dest}|#1}}|\\
|\expandafter\redirectjob\jobname~~~}\input{\jobname}|
\end{tabular}
\end{center}

In an alternative approach,
child documents can be compiled by a specific command line
without additional code or specific definitions:
%
\begin{center}
|... -jobname "|\textit{target}|" "|[\textit{flags}]%
|\includeonly{|\textit{dest}|}\input{|\textit{main}|}"|
\end{center}
%

%%%%%%%%%%%%%%%%%%%%%%%%%%%%%%%%%%%%%%%%%%%%%%%%%%%%%%%%%%%%%%%%%%%%%%%%%%%%%%%%
%%%%%%%%%%%%%%%%%%%%%%%%%%%%%%%%%%%%%%%%%%%%%%%%%%%%%%%%%%%%%%%%%%%%%%%%%%%%%%%%
\section{Information}

%%%%%%%%%%%%%%%%%%%%%%%%%%%%%%%%%%%%%%%%%%%%%%%%%%%%%%%%%%%%%%%%%%%%%%%%%%%%%%%%
\subsection{Copyright}

Copyright \copyright{} 2017--2018 Niklas Beisert

This work may be distributed and/or modified under the
conditions of the \LaTeX{} Project Public License, either version 1.3
of this license or (at your option) any later version.
The latest version of this license is in
  \url{http://www.latex-project.org/lppl.txt}
and version 1.3 or later is part of all distributions of \LaTeX{}
version 2005/12/01 or later.

This work has the LPPL maintenance status `maintained'.

The Current Maintainer of this work is Niklas Beisert.

This work consists of the files |README.txt|, |childdoc.ins| and |childdoc.dtx|
as well as the derived files |childdoc.def|, |cdocsamp.tex|
with |cdocsch1.tex|, |cdocsch2.tex|, |cdocspt3.tex|, |cdocspt4.tex|,
|cdocsdrf.tex|, |cdocsfn1.tex|, |cdocsfn2.tex|
as well as |childdoc.pdf|.

%%%%%%%%%%%%%%%%%%%%%%%%%%%%%%%%%%%%%%%%%%%%%%%%%%%%%%%%%%%%%%%%%%%%%%%%%%%%%%%%
\subsection{Files and Installation}

The package consists of the files:
%
\begin{center}
\begin{tabular}{ll}
    |README.txt|   & readme file \\
    |childdoc.ins| & installation file \\
    |childdoc.dtx| & source file \\
    |childdoc.def| & definition file \\
    |cdocsamp.tex| & sample main file \\
    |cdocsch1.tex| & sample include file \\
    |cdocsch2.tex| & sample include file \\
    |cdocspt3.tex| & sample part file \\
    |cdocspt4.tex| & sample part file \\
    |cdocsdrf.tex| & sample redirection file \\
    |cdocsfn1.tex| & sample redirection file \\
    |cdocsfn2.tex| & sample redirection file \\
    |childdoc.pdf| & manual
\end{tabular}
\end{center}
%
The distribution consists of the files
|README.txt|, |childdoc.ins| and |childdoc.dtx|.
%
\begin{itemize}
\item
Run (pdf)\LaTeX{} on |childdoc.dtx|
to compile the manual |childdoc.pdf| (this file).
\item
Run \LaTeX{} on |childdoc.ins| to create the definitions file |childdoc.def|
and the sample |cdocsamp.tex| with include files
|cdocsch1.tex|, |cdocsch2.tex|, |cdocspt3.tex|, |cdocspt4.tex|,
|cdocsdrf.tex|, |cdocsfn1.tex|, |cdocsfn2.tex|.
Then copy the file |childdoc.def| to an appropriate directory of your \LaTeX{}
distribution, e.g.\ \textit{texmf-root}|/tex/latex/childdoc|.
\end{itemize}

%%%%%%%%%%%%%%%%%%%%%%%%%%%%%%%%%%%%%%%%%%%%%%%%%%%%%%%%%%%%%%%%%%%%%%%%%%%%%%%%
\subsection{Related CTAN Packages}

There are several other packages which offer a similar functionality:
%
\begin{itemize}
\item
The packages
\href{http://ctan.org/pkg/docmute}{\textsf{docmute}},
\href{http://ctan.org/pkg/includex}{\textsf{includex}} and
\href{http://ctan.org/pkg/standalone}{\textsf{standalone}}
provide commands to include only the document body of
a child file thus allowing both files to be compiled individually.
\item
The packages \href{http://ctan.org/pkg/subdocs}{\textsf{subdocs}}
and \href{http://ctan.org/pkg/subfiles}{\textsf{subfiles}}
provide structures in which the main and child documents can be
encapsulated and allowing them to be compiled individually.
The inclusion mechanism is different from the conventional |\include|.
\item
The package \href{http://ctan.org/pkg/combine}{\textsf{combine}}
is an elaborate solution to combine several documents into one.
\end{itemize}
%
See also the CTAN topic \href{http://ctan.org/topic/subdocs}{\textsf{subdocs}}
for further related packages.
The present package differs from the above solutions in that
a document structure constructed with the conventional |\include| mechanism
just needs two extra commands at the top of every file
such that all constituent files can be compiled individually.

%%%%%%%%%%%%%%%%%%%%%%%%%%%%%%%%%%%%%%%%%%%%%%%%%%%%%%%%%%%%%%%%%%%%%%%%%%%%%%%%
%\subsection{Feature Suggestions}
%
%The following is a list of features which may be useful for future
%versions of this package:
%%
%\begin{itemize}
%\item
%\ldots
%\end{itemize}

%%%%%%%%%%%%%%%%%%%%%%%%%%%%%%%%%%%%%%%%%%%%%%%%%%%%%%%%%%%%%%%%%%%%%%%%%%%%%%%%
\subsection{Revision History}

%%%%%%%%%%%%%%%%%%%%%%%%%%%%%%%%%%%%%%%%
\paragraph{v2.0:} 2018/12/30

\begin{itemize}
\item
immediate forward processing
\item
added |\childdocby| mechanism
\item
manual restructured
\end{itemize}

%%%%%%%%%%%%%%%%%%%%%%%%%%%%%%%%%%%%%%%%
\paragraph{v1.6:} 2018/01/17

\begin{itemize}
\item
application for development of include files
\item
corrections to manual
\end{itemize}

%%%%%%%%%%%%%%%%%%%%%%%%%%%%%%%%%%%%%%%%
\paragraph{v1.5:} 2017/05/21

\begin{itemize}
\item
more complete structuring introduced
\item
|\childdocof| introduced
\item
|\childdoc| renamed to |\childdocmain|
\item
|\childredirect| renamed to |\childdocforward| and |\childdocforwardprefix|
and functionality expanded
\end{itemize}

%%%%%%%%%%%%%%%%%%%%%%%%%%%%%%%%%%%%%%%%
\paragraph{v1.0:} 2017/04/27

\begin{itemize}
\item
manual and install package
\item
first version published on CTAN
\end{itemize}

%%%%%%%%%%%%%%%%%%%%%%%%%%%%%%%%%%%%%%%%
\paragraph{v0.6:} 2017/04/26

\begin{itemize}
\item
redirection mechanism added
\end{itemize}

%%%%%%%%%%%%%%%%%%%%%%%%%%%%%%%%%%%%%%%%
\paragraph{v0.5:} 2017/04/26

\begin{itemize}
\item
functionality in definition file
\end{itemize}


%%%%%%%%%%%%%%%%%%%%%%%%%%%%%%%%%%%%%%%%%%%%%%%%%%%%%%%%%%%%%%%%%%%%%%%%%%%%%%%%
%%%%%%%%%%%%%%%%%%%%%%%%%%%%%%%%%%%%%%%%%%%%%%%%%%%%%%%%%%%%%%%%%%%%%%%%%%%%%%%%
%%%%%%%%%%%%%%%%%%%%%%%%%%%%%%%%%%%%%%%%%%%%%%%%%%%%%%%%%%%%%%%%%%%%%%%%%%%%%%%%
\appendix

\settowidth\MacroIndent{\rmfamily\scriptsize 000\ }

 \DocInput{childdoc.dtx}

\end{document}
%</driver>
% \fi
%
% %%%%%%%%%%%%%%%%%%%%%%%%%%%%%%%%%%%%%%%%%%%%%%%%%%%%%%%%%%%%%%%%%%%%%%%%%%%%%%
% %%%%%%%%%%%%%%%%%%%%%%%%%%%%%%%%%%%%%%%%%%%%%%%%%%%%%%%%%%%%%%%%%%%%%%%%%%%%%%
% \section{Sample}
%\iffalse
%<*samplemain>
%\fi
%
% The following presents a sample document
% with two chapters, two parts, a title page,
% a compile flag as well as three forwarding files to set the flag.
% It consists of eight |.tex| files:
% \begin{center}
% \begin{tabular}{ll}
% |cdocsamp.tex|&main file\\
% |cdocsch1.tex|&include file for chapter 1\\
% |cdocsch2.tex|&include file for chapter 2\\
% |cdocspt3.tex|&include file for part 3\\
% |cdocspt4.tex|&include file for part 4\\
% |cdocsdrf.tex|&forwarding file for main file in draft mode\\
% |cdocsfi1.tex|&forwarding file for final version of chapter 1\\
% |cdocsfi2.tex|&forwarding file for final version of chapter 2\\
% \end{tabular}
% \end{center}
% Each of the eight files can be compiled directly by the \LaTeX{} compiler.
%
% %%%%%%%%%%%%%%%%%%%%%%%%%%%%%%%%%%%%%%
% \paragraph{Main File.}
%
% The main file is called |cdocsamp.tex|.
%
% Load the \textsf{childdoc} definitions and
% declare the filename for the main document:
%    \begin{macrocode}
\input{childdoc.def}
\childdocmain{}
%    \end{macrocode}

% Optional override for |\version| flag:
%    \begin{macrocode}
%%\ifchilddoc\else\providecommand{\version}{draft}\fi
%    \end{macrocode}

% Define the default values for the |\version| flag
% (|final| for the main file and |draft| for childs):
%    \begin{macrocode}
\ifchilddoc
\providecommand{\version}{draft}
\else
\providecommand{\version}{final}
\fi
%    \end{macrocode}

% Load the standard document class:
%    \begin{macrocode}
\documentclass[12pt]{article}
%    \end{macrocode}

% Start the document body:
%    \begin{macrocode}
\begin{document}
%    \end{macrocode}

% Declare a title page.
% Print title, part of document being processed and version flag:
%    \begin{macrocode}
\addtocounter{page}{-1}
\begin{center}
{\LARGE\bfseries{}childdoc example\par}
\vspace{1cm}
\ifchilddoc
\ifchilddocmanual part\else chapter\fi:
`\childdocname' of `\childdocjob'\par
\else
main document: `\childdocjob'\par
\fi
version: \version\par
\end{center}
\newpage
%    \end{macrocode}

% Manually include selected file,
% otherwise process as usual:
%    \begin{macrocode}
\ifchilddocmanual
\section*{part `\childdocname'}
\input{\childdocname}
\else
%    \end{macrocode}

% Include the two chapters:
%    \begin{macrocode}
\include{cdocsch1}
\include{cdocsch2}
%    \end{macrocode}

% Include the two parts unless only chapters should be displayed:
%    \begin{macrocode}
\ifchilddoc\else
\section{part three}
\input{cdocspt3}
\section{part four}
\input{cdocspt4}
\fi
%    \end{macrocode}

% Process as usual until here:
%    \begin{macrocode}
\fi
%    \end{macrocode}

% End of document body:
%    \begin{macrocode}
\end{document}
%    \end{macrocode}
%\iffalse
%</samplemain>
%\fi
%
% %%%%%%%%%%%%%%%%%%%%%%%%%%%%%%%%%%%%%%
% \paragraph{Chapter Include Files.}
%
% The include files are called |cdocsch1.tex| and |cdocsch2.tex|.
%
%\iffalse
%<*samplechap1|samplechap2>
%\fi

% Optional override for |\version| flag:
%    \begin{macrocode}
%%\providecommand{\version}{final}
%    \end{macrocode}

% Include the main document:
%    \begin{macrocode}
\input{childdoc.def}
\childdocof{cdocsamp}
%    \end{macrocode}

%\iffalse
%</samplechap1|samplechap2>
%\fi
%
%\iffalse
%<*samplechap1>
%\fi
% Some text for chapter 1:
%    \begin{macrocode}
\section{one}
some text in chapter one
%    \end{macrocode}

%\iffalse
%</samplechap1>
%\fi
% Some text for chapter 2:
%\iffalse
%<*samplechap2>
%\fi
%    \begin{macrocode}
\section{two}
more text in chapter two
%    \end{macrocode}

%\iffalse
%</samplechap2>
%\fi
%
% %%%%%%%%%%%%%%%%%%%%%%%%%%%%%%%%%%%%%%
% \paragraph{Part Include Files.}
%
% The include files are called |cdocspt3.tex| and |cdocspt4.tex|.
%
%\iffalse
%<*samplepart3|samplepart4>
%\fi

% Optional override for |\version| flag:
%    \begin{macrocode}
%%\providecommand{\version}{final}
%    \end{macrocode}

% Include the main document:
%    \begin{macrocode}
\input{childdoc.def}
\childdocby{cdocsamp}
%    \end{macrocode}

%\iffalse
%</samplepart3|samplepart4>
%\fi
%
%\iffalse
%<*samplepart3>
%\fi
% Some text for part 3:
%    \begin{macrocode}
some text in part three
%    \end{macrocode}

%\iffalse
%</samplepart3>
%\fi
% Some text for part 4:
%\iffalse
%<*samplepart4>
%\fi
%    \begin{macrocode}
more text in part four
%    \end{macrocode}

%\iffalse
%</samplepart4>
%\fi
%
% %%%%%%%%%%%%%%%%%%%%%%%%%%%%%%%%%%%%%%
% \paragraph{Forwarding for a Complete Draft.}
%
% The following forwarding file |cdocsdrf.tex|
% compiles the main document in draft mode:
%\iffalse
%<*sampledraft>
%\fi
%    \begin{macrocode}
\def\version{draft}
\input{childdoc.def}
\childdocforward{cdocsamp}
%    \end{macrocode}

%\iffalse
%</sampledraft>
%\fi
%
% %%%%%%%%%%%%%%%%%%%%%%%%%%%%%%%%%%%%%%
% \paragraph{Forwarding for Final Version of the Chapters.}
%
% The following forwarding files |cdocsfn1.tex| and |cdocsfn2.tex|
% (with identical content)
% compile the final versions of the child documents
% |cdocsch1.tex| and |cdocsch2.tex|, respectively:
%\iffalse
%<*samplefinal>
%\fi
%    \begin{macrocode}
\def\version{final}
\input{childdoc.def}
\childdocforwardprefix[cdocsamp]{cdocsfn}{cdocsch}
%    \end{macrocode}

%\iffalse
%</samplefinal>
%\fi
%
% %%%%%%%%%%%%%%%%%%%%%%%%%%%%%%%%%%%%%%
% \paragraph{Command Line Processing.}
%
% The following three command lines generate the output files
% |cdocscld|, |cdocscl1| and |cdocscl2|
% which should be identical to
% |cdocsdrf|, |cdocsch1| and |cdocsfn2|, respectively:
% \begin{center}
% \begin{tabular}{l}
% |latex -jobname cdocscld \|\\
% |  "\def\version{draft}\input{childdoc.def}\childdocforward{cdocsamp}"|\\
% |latex -jobname cdocscl1 \|\\
% |  "\input{childdoc.def}\childdocforward[cdocsamp]{cdocsch1}"|\\
% |latex -jobname cdocscl2 \|\\
% |  "\def\version{final}\input{childdoc.def}\childdocforward{cdocsch2}"|
% \end{tabular}
% \end{center}
% Note that the trailing backslash on each first line
% merely continues the input to the second line
% (for convenient cut ant paste).
% Furthermore, the command |latex| can be replaced by any
% of its alternative versions such as |pdflatex|.
%
% %%%%%%%%%%%%%%%%%%%%%%%%%%%%%%%%%%%%%%%%%%%%%%%%%%%%%%%%%%%%%%%%%%%%%%%%%%%%%%
% %%%%%%%%%%%%%%%%%%%%%%%%%%%%%%%%%%%%%%%%%%%%%%%%%%%%%%%%%%%%%%%%%%%%%%%%%%%%%%
% \section{Implementation}
%\iffalse
%<*package>
%\fi
%
% This section describes the definitions file |childdoc.def|.

% The definitions cannot be loaded using |\usepackage| or |\RequirePackage|
% which has a mechanism to prevent loading a style file more than once.
% When loading the definitions by means of |\input|
% multiple instances have to be prevented manually:
%\iffalse
%This code needs to be before the `\ProvidesFile' directive
%which is defined at the beginning of this file.
%Therefore it is also placed there and commented out here.
%</package>
%<*discard>
%\fi
%    \begin{macrocode}
\ifdefined\childdocmain\endinput\fi
%    \end{macrocode}
%\iffalse
%</discard>
%<*package>
%\fi
%
% \macro{\ifchilddoc}
% \macro{\ifchilddocmanual}
% The conditional |\ifchilddoc| tells whether a
% child (true) or main (false) document is being compiled.
% The conditional |\ifchilddocmanual| tells whether
% the |\includeonly| mechanism is used (false) or
% the selection of child files must be performed manually (true).
% The definitions initialise to false:
%    \begin{macrocode}
\newif\ifchilddoc
\newif\ifchilddocmanual
%    \end{macrocode}

% \macro{\childdocname}
% \macro{\childdocjob}
% The macro |\childdocname| stores the name of the main document
% to be compiled. The macro |\childdocjob| stores the name of
% the document on which the \LaTeX{} compiler was originally invoked.
% The content of |\jobname| cannot be compared
% to filenames specified in the source due to different catcodes.
% The following code rescans |\jobname|, stores the result
% in |\childdocname| and saves a copy in |\childdocjob|:
%    \begin{macrocode}
\edef\childdocname{\scantokens\expandafter{\jobname\noexpand}}
\let\childdocjob\childdocname
%    \end{macrocode}

% \macro{\childdocdisable}
% The macro |\childdocdisable| prevents the main file
% from being processed more than once.
% At this stage, the main document command |\childdocmain|
% is assumed to be called once again where it should do nothing.
% Any subsequent call to it should prevent
% a secondary processing of the main document
% It overwrites the forwarding commands
% |\childdocof| and |\childdocforward|
% with empty macros to prevent further inclusions of the main document:
%    \begin{macrocode}
\newcommand{\childdocdisable}
{
  \renewcommand{\childdocmain}[1]{\renewcommand{\childdocmain}[1]{\endinput}}
  \renewcommand{\childdocof}[1]{}
  \renewcommand{\childdocby}[2][]{}
  \renewcommand{\childdocforward}[2][]{}
  \renewcommand{\childdocdisable}{}
}
%    \end{macrocode}

% \macro{\childdocmain}
% The macro |\childdocmain| is to be called at the top of the main file
% with nothing or the main filename (without extension) as argument.
% First, it breaks loops.
% If the argument is not empty and does not match |\childdocname|
% (which is set by the first inclusion of |childdoc.def|),
% |\ifchilddoc| is set to true, |\includeonly| is applied to the child file
% and |\jobname| is set to the main file
% (for proper handling of |.aux| files):
%    \begin{macrocode}
\newcommand{\childdocmain}[1]
{
  \childdocdisable\childdocmain{}
  \if?#1?\else
    \begingroup
      \def\childdoctmp{#1}
      \ifx\childdoctmp\childdocname
        \def\childdoctmp{}
      \else
        \def\childdoctmp
        {
          \childdoctrue
          \includeonly{\childdocname}
          \def\childdocjob{#1}
          \def\jobname{#1}
        }
      \fi
      \expandafter
    \endgroup
    \childdoctmp
  \fi
}
%    \end{macrocode}

% \macro{\childdocof}
% The command |\childdocof| redirects
% compilation to the main file |#1|.
%    \begin{macrocode}
\newcommand{\childdocof}[1]
{
  \childdocdisable
  \childdoctrue
  \includeonly{\childdocname}
  \def\jobname{#1}
  \def\childdocjob{#1}
  \input{#1}
}
%    \end{macrocode}

% \macro{\childdocby}
% The command |\childdocby| ....
%    \begin{macrocode}
\newcommand{\childdocby}[2][]
{
  \childdocdisable
  \childdoctrue
  \childdocmanualtrue
  \if?#1?\else
    \def\jobname{#2}
  \fi
  \def\childdocjob{#2}
  \input{#2}
  \endinput
}
%    \end{macrocode}

% \macro{\childdocforward}
% The command |\childdocforward| redirects
% compilation to the main file or
% (if the optional argument is given) a child file.
% Parameters are set as if the main file
% or a child file starting with |\childdocof| was compiled.
% Then compilation is handed over to the main file:
%    \begin{macrocode}
\newcommand{\childdocforward}[2][]
{
  \begingroup
    \if?#1?
      \def\childdoctmp
      {
        \def\childdocname{#2}
        \def\childdocjob{#2}
        \def\jobname{#2}
        \input{#2}
        \endinput
      }
    \else
      \def\childdoctmp
      {
        \childdocdisable
        \def\childdocname{#2}
        \childdoctrue
        \includeonly{#2}
        \def\childdocjob{#1}
        \def\jobname{#1}
        \input{#1}
        \endinput
      }
    \fi
    \expandafter
  \endgroup
  \childdoctmp
}
%    \end{macrocode}

% \macro{\childdocforwardprefix}
% The command |\childdocforwardprefix| redirects
% compilation to the main or a child file by means of a pattern.
% The prefix |#1| in the current filename is replaced by |#2|
% and the suffix of the current filename is kept
% (it is assumed that the filename does not contain the substring `|~~~|'
% which is used as a delimiter).
% Compilation is handed over to the new file by |\childdocforward|:
%    \begin{macrocode}
\newcommand{\childdocforwardprefix}[3][]
{
  \begingroup
    \def\childdocextract #2##1~~~{\def\childdoctmp{\childdocforward[#1]{#3##1}}}
    \expandafter\childdocextract\childdocname~~~
    \expandafter
  \endgroup
  \childdoctmp
}
%    \end{macrocode}

% \macro{\childdoc}
% The deprecated macro |\childdoc| is a legacy version of |\childdocmain|:
%    \begin{macrocode}
\newcommand{\childdoc}{\childdocmain}
%    \end{macrocode}

% \macro{\childdocredirect}
% The deprecated macro |\childdocredirect| is a legacy version
% of |\childdocforward| and |\childdocforwardprefix|:
%    \begin{macrocode}
\newcommand{\childdocredirect}[2][]
{
  \begingroup
    \if?#1?
      \def\childdoctmp{\childdocforward{#2}}
    \else
      \def\childdoctmp{\childdocforwardprefix{#1}{#2}}
    \fi
    \expandafter
  \endgroup
  \childdoctmp
}
%    \end{macrocode}

%\iffalse
%</package>
%\fi
%
\endinput

\childdocmain{}
%    \end{macrocode}

% Optional override for |\version| flag:
%    \begin{macrocode}
%%\ifchilddoc\else\providecommand{\version}{draft}\fi
%    \end{macrocode}

% Define the default values for the |\version| flag
% (|final| for the main file and |draft| for childs):
%    \begin{macrocode}
\ifchilddoc
\providecommand{\version}{draft}
\else
\providecommand{\version}{final}
\fi
%    \end{macrocode}

% Load the standard document class:
%    \begin{macrocode}
\documentclass[12pt]{article}
%    \end{macrocode}

% Start the document body:
%    \begin{macrocode}
\begin{document}
%    \end{macrocode}

% Declare a title page.
% Print title, part of document being processed and version flag:
%    \begin{macrocode}
\addtocounter{page}{-1}
\begin{center}
{\LARGE\bfseries{}childdoc example\par}
\vspace{1cm}
\ifchilddoc
\ifchilddocmanual part\else chapter\fi:
`\childdocname' of `\childdocjob'\par
\else
main document: `\childdocjob'\par
\fi
version: \version\par
\end{center}
\newpage
%    \end{macrocode}

% Manually include selected file,
% otherwise process as usual:
%    \begin{macrocode}
\ifchilddocmanual
\section*{part `\childdocname'}
\input{\childdocname}
\else
%    \end{macrocode}

% Include the two chapters:
%    \begin{macrocode}
\include{cdocsch1}
\include{cdocsch2}
%    \end{macrocode}

% Include the two parts unless only chapters should be displayed:
%    \begin{macrocode}
\ifchilddoc\else
\section{part three}
\input{cdocspt3}
\section{part four}
\input{cdocspt4}
\fi
%    \end{macrocode}

% Process as usual until here:
%    \begin{macrocode}
\fi
%    \end{macrocode}

% End of document body:
%    \begin{macrocode}
\end{document}
%    \end{macrocode}
%\iffalse
%</samplemain>
%\fi
%
% %%%%%%%%%%%%%%%%%%%%%%%%%%%%%%%%%%%%%%
% \paragraph{Chapter Include Files.}
%
% The include files are called |cdocsch1.tex| and |cdocsch2.tex|.
%
%\iffalse
%<*samplechap1|samplechap2>
%\fi

% Optional override for |\version| flag:
%    \begin{macrocode}
%%\providecommand{\version}{final}
%    \end{macrocode}

% Include the main document:
%    \begin{macrocode}
% \iffalse
%
% childdoc.dtx Copyright (C) 2017-2018 Niklas Beisert
%
% This work may be distributed and/or modified under the
% conditions of the LaTeX Project Public License, either version 1.3
% of this license or (at your option) any later version.
% The latest version of this license is in
%   http://www.latex-project.org/lppl.txt
% and version 1.3 or later is part of all distributions of LaTeX
% version 2005/12/01 or later.
%
% This work has the LPPL maintenance status `maintained'.
%
% The Current Maintainer of this work is Niklas Beisert.
%
% This work consists of the files childdoc.dtx and childdoc.ins
% and the derived files childdoc.def and cdocsamp.tex with
% cdocsch1.tex, cdocsch2.tex, cdocsdrf.tex, cdocsfn1.tex, cdocsfn2.tex.
%
%<package>\ifdefined\childdocmain\endinput\fi
%<package>\ProvidesFile{childdoc.def}[2018/12/30 v2.0 child document driver]
%<samplemain>\ProvidesFile{cdocsamp.tex}[2018/12/30 v2.0 sample for childdoc]
%<*driver>
%\ProvidesFile{childdoc.drv}[2018/12/30 v2.0 childdoc reference manual file]
\PassOptionsToClass{10pt,a4paper}{article}
\documentclass{ltxdoc}

\usepackage[margin=35mm]{geometry}
\usepackage{hyperref}
\usepackage{hyperxmp}
\usepackage[usenames]{color}

\hypersetup{colorlinks=true}
\hypersetup{pdfstartview=FitH}
\hypersetup{pdfpagemode=UseNone}
\hypersetup{pdfsource={}}
\hypersetup{pdflang={en-UK}}
\hypersetup{pdfcopyright={Copyright 2017-2018 Niklas Beisert.
  This work may be distributed and/or modified under the
  conditions of the LaTeX Project Public License, either version 1.3
  of this license or (at your option) any later version.}}
\hypersetup{pdflicenseurl={http://www.latex-project.org/lppl.txt}}
\hypersetup{pdfcontactaddress={ETH Zurich, ITP, HIT K,
  Wolfgang-Pauli-Strasse 27}}
\hypersetup{pdfcontactpostcode={8093}}
\hypersetup{pdfcontactcity={Zurich}}
\hypersetup{pdfcontactcountry={Switzerland}}
\hypersetup{pdfcontactemail={nbeisert@itp.phys.ethz.ch}}
\hypersetup{pdfcontacturl={http://people.phys.ethz.ch/\xmptilde nbeisert/}}

\newcommand{\secref}[1]{\hyperref[#1]{section \ref*{#1}}}

\parskip1ex
\parindent0pt
\let\olditemize\itemize
\def\itemize{\olditemize\parskip0pt}

\begin{document}

\title{The \textsf{childdoc} Package}
\hypersetup{pdftitle={The childdoc Package}}
\author{Niklas Beisert\\[2ex]
  Institut f\"ur Theoretische Physik\\
  Eidgen\"ossische Technische Hochschule Z\"urich\\
  Wolfgang-Pauli-Strasse 27, 8093 Z\"urich, Switzerland\\[1ex]
  \href{mailto:nbeisert@itp.phys.ethz.ch}
  {\texttt{nbeisert@itp.phys.ethz.ch}}}
\hypersetup{pdfauthor={Niklas Beisert}}
\hypersetup{pdfsubject={Manual for the LaTeX2e Package childdoc}}
\date{30 December 2018, \textsf{v2.0}}
\maketitle

\begin{abstract}\noindent
\textsf{childdoc} is a \LaTeXe{} package
that enables the direct compilation
of document sections included by |\include|
to individual files.
\end{abstract}

\begingroup
\parskip0ex
\tableofcontents
\endgroup

%%%%%%%%%%%%%%%%%%%%%%%%%%%%%%%%%%%%%%%%%%%%%%%%%%%%%%%%%%%%%%%%%%%%%%%%%%%%%%%%
%%%%%%%%%%%%%%%%%%%%%%%%%%%%%%%%%%%%%%%%%%%%%%%%%%%%%%%%%%%%%%%%%%%%%%%%%%%%%%%%
\section{Introduction}

\LaTeX{} provides a mechanism to structure a large document (such as a book)
into a main file and several child files (containing the chapters)
using the |\include| command.
This mechanism is beneficial for documents
which span hundreds of pages in order to
make the source file(s) more manageable.
Moreover, compilation can be restricted to
selected child files by means of the |\includeonly| command.
The latter feature can be used to reduce the compilation time while editing
(this was significantly more useful in the earlier days of \LaTeX{})
or to generate a smaller document which is easier to navigate.
Another application of |\includeonly| is to generate
documents consisting of selected parts of the complete document.

However, there are a few drawbacks of the plain |\include| mechanism:
\begin{itemize}
\item
The child files cannot be compiled on their own,
they can only be compiled via the main file.
A naive editing environment
(such as a text editor with an option
to have the current file processed by \LaTeX)
may require one to switch to the main file before compiling;
attempting to compile the child file produces errors.
\item
The main file must be modified (each time)
to adjust the |\includeonly| command
to the present needs. This easily leaves the main file in a messy state.
\item
The generated document will always carry the filename
of the main document. This is inconvenient if
several child files are to be compiled and
to be kept for distribution.
\end{itemize}

The present package provides a simple interface
to make child files individually compilable by \LaTeX{}.
Compiling a child file then has the same effect as compiling
the main file with an |\includeonly| command
to select the appropriate child.
Moreover the generated document will carry the name of the child
rather than the main file.
This resolves all three above issues.

This feature is meant to make the editing of books,
thesis documents and lecture notes somewhat more convenient.
However, the package can also be used efficiently for
composing a series of documents (such as exercise sheets)
which are typically distributed individually.
It then assists the author in generating the individual documents
(potentially in different versions)
as well as a document containing the collected series.
Another application is in developing style files
or other kinds of included material
where compilation of the style file could redirect
to a sample or test file.

%%%%%%%%%%%%%%%%%%%%%%%%%%%%%%%%%%%%%%%%%%%%%%%%%%%%%%%%%%%%%%%%%%%%%%%%%%%%%%%%
%%%%%%%%%%%%%%%%%%%%%%%%%%%%%%%%%%%%%%%%%%%%%%%%%%%%%%%%%%%%%%%%%%%%%%%%%%%%%%%%
\section{Usage}

First of all, the package \textsf{childdoc} is \emph{not} a standard
\LaTeXe{} |.sty| style file! Therefore it needs to be invoked in
a non-standard way.

%%%%%%%%%%%%%%%%%%%%%%%%%%%%%%%%%%%%%%%%%%%%%%%%%%%%%%%%%%%%%%%%%%%%%%%%%%%%%%%%
\subsection{Included Files}
\label{sec:include}

%%%%%%%%%%%%%%%%%%%%%%%%%%%%%%%%%%%%%%%%
\DescribeMacro{\childdocmain}
To use the package, add the commands
\begin{center}
\begin{tabular}{l}
|\input{childdoc.def}|\\
|\childdocmain{}|\\
\end{tabular}
\end{center}
at the very top of the main \LaTeX{} file,
in particular \emph{before} the |\documentclass| statement!
The argument of |\childdocmain| should be left empty
(but it must be present).

%%%%%%%%%%%%%%%%%%%%%%%%%%%%%%%%%%%%%%%%
\DescribeMacro{\childdocof}
Furthermore, add the commands
\begin{center}
\begin{tabular}{l}
|\input{childdoc.def}|\\
|\childdocof{|\textit{main}|}|\\
\end{tabular}
\end{center}
at the top of every child file \textit{child}
which is included by |\include{|\textit{child}|}|
from within the main file
(or at least for those files to be compiled individually).
The argument \textit{main} must be the filename of the main file.

There are a couple of
considerations in setting up the main and child documents:

%%%%%%%%%%%%%%%%%%%%%%%%%%%%%%%%%%%%%%%%
\paragraph{Restrictions.}

Please note the following restrictions:
\begin{itemize}
\item
|\childdocmain| must be called with one argument \textit{main}
to ensure compatibility with earlier version of the package.
It must either be empty (|\childdocmain{}|)
or precisely match the filename of the main file in which it is specified.
See \secref{sec:detection} for further information.
\item
The filename \textit{main} must be specified without the |.tex| extension.
\item
The filename \textit{main} is case sensitive
(even in case-insensitive file systems)
due to internal string comparison.
\item
The argument \textit{main} should be fully expanded, it cannot be a macro.
\item
Subdirectories and special characters should be avoided in filenames.
\item
The command |\childdocmain{|\textit{main}|}| must be followed by a whitespace.
It should not be followed immediately by another command
or by a comment mark `|%|'.
This is because the \TeX{} parser reads the token immediately following
the argument of |\childdocmain| and puts it
at the beginning of every child section;
however, a white\-space is ignored.
\end{itemize}

%%%%%%%%%%%%%%%%%%%%%%%%%%%%%%%%%%%%%%%%
\paragraph{Content of Main File.}

It is advisable to place all content in the child files included by |\include|.
Any output contained in the main file will appear in all child documents
unless suppressed manually;
it cannot be suppressed automatically by the |\includeonly| directive
and thus should normally be avoided.
A method to include some content in the main file
by means of conditional processing is described in \secref{sec:conditional}.

%%%%%%%%%%%%%%%%%%%%%%%%%%%%%%%%%%%%%%%%
\paragraph{Page Numbering.}

When only a part of the document is compiled,
the appropriate numbering of pages
(as well as other status parameters)
is determined from the |.aux| files.
The latter contain information from previous passes.
However this information needs to propagate through
all intermediate child documents.
Therefore the page numbering in child documents may well
be inconsistent until the complete document is compiled at least once.

A useful (if unconventional) way to always ensure a consistent
page numbering is to restart the numbering in each child document
and denote the pages by `\textit{child}|.|\textit{page}'
where \textit{child} represents the chapter/section number of the child file.
This can be achieved by the command
|\numberwithin{page}{|\textit{child}|}|
of the \textsf{amsmath} package
where \textit{child} can be |chapter| or |section|
depending on the chosen structuring.
Alternatively, one can modify the macro |\thepage| appropriately
and reset the counter |page| at the start of each child file.

%%%%%%%%%%%%%%%%%%%%%%%%%%%%%%%%%%%%%%%%%%%%%%%%%%%%%%%%%%%%%%%%%%%%%%%%%%%%%%%%
\subsection{Conditional Processing}
\label{sec:conditional}

The package provides a mechanism to compile different versions
of a document. To customise the versions further some conditional processing
can come in handy to distinguish which version is being compiled.
The package provides two macros to describe the compilation context:

%%%%%%%%%%%%%%%%%%%%%%%%%%%%%%%%%%%%%%%%
\DescribeMacro{\ifchilddoc}
The conditional |\ifchilddoc| distinguishes between the compilation of
child documents and the main document:
%
\begin{center}
|\ifchilddoc |\textit{child-code}| |[|\||else |\textit{main-code}]| \||fi|
\end{center}

%%%%%%%%%%%%%%%%%%%%%%%%%%%%%%%%%%%%%%%%
\DescribeMacro{\childdocname}
\DescribeMacro{\childdocjob}
The macro |\childdocname| contains the filename (without extension)
of the main or child file being processed.
Note that |\childdocjob| will always contain the name of the main file.

%%%%%%%%%%%%%%%%%%%%%%%%%%%%%%%%%%%%%%%%
\paragraph{Title Page.}

Conditional processing can be used to include a title or banner page
in the main document when proper precautions are taken.
Importantly, the code in the main file should ensure that the page counter
(as well as other status parameters which are stored in the |.aux| files)
takes the same value after the conditional processing.
Otherwise the page numbers may take divergent values
depending on which part is compiled.

For example, a title page could be declared by:
%
\begin{center}
\begin{tabular}{l}
|\ifchilddoc\||else|\\
|\addtocounter{page}{-1}|\\
\textit{code for title page}\\
|\newpage|\\
|\||fi|
\end{tabular}
\end{center}
%
A banner page for the child documents can be generated by:
%
\begin{center}
\begin{tabular}{l}
|\ifchilddoc|\\
|\addtocounter{page}{-1}|\\
\textit{code for banner page}\\
|\newpage|\\
|\||fi|
\end{tabular}
\end{center}
%
Here one could write a message such as:
\begin{center}
|This is the part \childdocname{} of \childdocjob{}.|
\end{center}

%%%%%%%%%%%%%%%%%%%%%%%%%%%%%%%%%%%%%%%%%%%%%%%%%%%%%%%%%%%%%%%%%%%%%%%%%%%%%%%%
\subsection{Flags}
\label{sec:flags}

The package makes it easy to generate different versions
of the main or child documents.
To this end compilation flags can be defined
and assigned different default values.
They will be particularly useful in conjunction
with the forwarding mechanism described in \secref{sec:forward}.

For example, it may be useful to have a flag |\version|
which can be set to |draft| or |final|.
The document source will contain some conditional code
depending on the value of |\version|.
Suppose further, the flag should default to |final| for the main file
and to |draft| for child files
which is a natural assignment for editing the document.
This is achieved by placing the following code
in the preamble of the main document
(below the |\childdocmain| directive):
%
\begin{center}
\begin{tabular}{l}
|\ifchilddoc|\\
|\providecommand{\version}{draft}|\\
|\||else|\\
|\providecommand{\version}{final}|\\
|\||fi|
\end{tabular}
\end{center}
%
The definition by |\providecommand| makes sure
that previous definitions are not overwritten.
Further statements |\providecommand{\version}{...}|
can thus be added before the above code to override it.

For the main file, one might add a line
(between |\childdocmain| and the above block)
%
\begin{center}
|%\ifchilddoc\||else\providecommand{\version}{draft}\||fi|
\end{center}
%
which can be uncommented to produce a draft version.
Likewise one can add a line to the very top of a child file
(above the |\childdocof{|\textit{main}|}| directive)
%
\begin{center}
|%\providecommand{\version}{final}|
\end{center}
%
which can be uncommented to produce the final version of this child document.

%%%%%%%%%%%%%%%%%%%%%%%%%%%%%%%%%%%%%%%%%%%%%%%%%%%%%%%%%%%%%%%%%%%%%%%%%%%%%%%%
\subsection{Forwarding}
\label{sec:forward}

Different versions of the main or child documents
using compilation flags as described in \secref{sec:flags}
can be (permanently) stored in different files
for convenient compilation, viewing and distribution.
To this end, the package defines a command
to pass on compilation to a different file:

%%%%%%%%%%%%%%%%%%%%%%%%%%%%%%%%%%%%%%%%
\DescribeMacro{\childdocforward}
The command |\childdocforward| redirects processing to
another source file:
%
\begin{center}
\begin{tabular}{l}
|\input{childdoc.def}|\\
|\childdocforward[|\textit{main}|]{|\textit{dest}|}|\\
\end{tabular}
\end{center}
%
The argument \textit{dest} is the destination file
(without extension).
It should be the main file or one of the child files.
Note that further \textsf{childdoc} directives
such as |\childdocof| and |\childdocforward|
in the indicated file will be processed in this form.
The optional argument \textit{main}
passes on directly to the main file \textit{main}
while pretending to compile the child \textit{dest}.
This form behaves as if \textit{dest}
issues |\childdocof{|\textit{main}|}| right away,
and no further \textsf{childdoc} directives will be processed.

%%%%%%%%%%%%%%%%%%%%%%%%%%%%%%%%%%%%%%%%
\DescribeMacro{\...prefix}
In the alternative form |\childdocforwardprefix|,
%
\begin{center}
\begin{tabular}{l}
|\input{childdoc.def}|\\
|\childdocforwardprefix[|\textit{main}|]{|\textit{prefix}|}{|\textit{dest}|}|
\end{tabular}
\end{center}
%
the destination file is determined by a pattern
depending on the current file:
To make this work, the current file must be called
`{\textit{prefix}\hspace{0.2em}\textit{suffix}}'
with \textit{prefix} matching precisely the argument.
Processing is then passed on to the file
`{\textit{dest}\hspace{0.2em}\textit{suffix}}'.
Surely, the same effect is achieved by
directly specifying the
argument `{\textit{dest}\hspace{0.2em}\textit{suffix}}'
in the first form.
However, that requires to set up a different file
for each child. With the alternative form of the command
all these files can have exactly the same content
which simplifies setting them up and maintaining them.

For example, the following file |draft.tex|
with a compilation flag |\version| as described in \secref{sec:flags}
compiles the main document as a draft:
%
\begin{center}
\begin{tabular}{l}
|\def\version{draft}|\\
|\input{childdoc.def}|\\
|\childdocforward{|\textit{main}|}|
\end{tabular}
\end{center}
%
Likewise, the following files |final|\textit{nn}|.tex|
compile the final version of the child document
|child|\textit{nn}|.tex|:
%
\begin{center}
\begin{tabular}{l}
|\def\version{final}|\\
|\input{childdoc.def}|\\
|\childdocforwardprefix{final}{child}|
\end{tabular}
\end{center}
%

Note that when several versions of a main file and/or of each child file
are to be generated, it may be convenient to set up a |Makefile| or
shell script to automatise the process.

%%%%%%%%%%%%%%%%%%%%%%%%%%%%%%%%%%%%%%%%%%%%%%%%%%%%%%%%%%%%%%%%%%%%%%%%%%%%%%%%
\subsection{Command Line Processing}
\label{sec:commandline}

The effect of redirection files can also be achieved by invoking
the \LaTeX{} compiler with a more elaborate command line.
Most conveniently this should be done as part
of a shell script or a |Makefile|.

When using \textsf{childdoc} in the main file, the following
command lines effectively perform a redirection
(note that depending on the shell being used,
backslashes may have to be doubled: `|\|' $\to$ `|\\|'):
%
\begin{center}
|... -jobname "|\textit{target}|" |\\|"|[\textit{flags}]%
|\input{childdoc.def}\childdocforward[|\textit{main}|]{|\textit{dest}|}"|
\end{center}
%
Here \textit{target} is the name of the output file,
\textit{main} is the name of the main file
and \textit{dest} is the name of the main or child file to be processed
(all filenames without extensions).
The optional argument \textit{main} can be omitted
if \textit{main} matches \textit{dest}.
Optionally, compilation \textit{flags} can be defined via |\def| commands.
This command line makes the \TeX{} engine believe
it is compiling the file \textit{target}
whose content is specified as the latter parameter.
The provided code then forwards the processing to
\textit{main} or \textit{dest} as described in \secref{sec:forward}.

%%%%%%%%%%%%%%%%%%%%%%%%%%%%%%%%%%%%%%%%%%%%%%%%%%%%%%%%%%%%%%%%%%%%%%%%%%%%%%%%
\subsection{Include by Input}
\label{sec:input}

Including child documents by |\include| has some restrictions by design.
Most notably, the content of a child document always occupies
its own set of pages; pages cannot be shared between child documents.
Usually, this behaviour makes perfect sense
because each child document contain an essential part of the document.
However, in some situations it may be desirable to compose
a document from a collection of parts
without having mandatory page breaks between then.
For this case, the package
provides a mechanism to include parts
by |\input| which can also be processed individually.
However, by construction this mechanism
requires manual handling of the content to be output.

%%%%%%%%%%%%%%%%%%%%%%%%%%%%%%%%%%%%%%%%
\DescribeMacro{\ifchilddocmanual}
The main file should be prepared as usual, see \secref{sec:include}.
However, the document body must make a distinction
between processing of an individual part and of the main document, e.g.:
%
\begin{center}
\begin{tabular}{l}
|\ifchilddocmanual|\\
|\input{\childdocname}|\\
|\||else|\\
\textit{document body with }|\input{|\textit{part}|}|\\
|\||fi|
\end{tabular}
\end{center}
%
The conditional |\ifchilddocmanual| is true whenever
a part to be included by |\input| is being compiled,
and the name of the part is stored in |\childdocname|.

%%%%%%%%%%%%%%%%%%%%%%%%%%%%%%%%%%%%%%%%
\DescribeMacro{\childdocby}
Each part to be included by |\input| should start with:
%
\begin{center}
\begin{tabular}{l}
|\input{childdoc.def}|\\
|\childdocby{|\textit{main}|}|\\
\end{tabular}
\end{center}
%
The directive |\childdocby| is similar to |\childdocof|
described in \secref{sec:include},
but the subsequent selection of content must be done manually.
To that end, both |\ifchilddoc| and |\ifchilddocmanual|
will be true upon processing of a part,
and the name of the part is stored in |\childdocname|.
Note that |\jobname| will be set to the filename of the current part
so that each part receives an individual |.aux| file
that does not interfere with the |.aux| file(s) of the main document.
This behaviour can be altered by the alternative form
|\childdocby[*]{|\textit{main}|}| (with a non-empty optional argument)
which uses the |.aux| file of the main document
by setting |\jobname| to \textit{main}.

%%%%%%%%%%%%%%%%%%%%%%%%%%%%%%%%%%%%%%%%%%%%%%%%%%%%%%%%%%%%%%%%%%%%%%%%%%%%%%%%
\subsection{Driver Development}
\label{sec:driver}

The \textsf{childdoc} mechanism can also be use for the development
of definition files such as \LaTeX{} styles or classes.
This case differs from the above setup with multiple parts
included by |\include| in that no |\includeonly| should be invoked.
This can be achieved by starting the include file
(before |\ProvidesPackage|) with:
%
\begin{center}
\begin{tabular}{l}
|\input{childdoc.def}|\\
|\childdocforward{|\textit{main}|}|\\
\end{tabular}
\end{center}
%
or alternatively with:
%
\begin{center}
\begin{tabular}{l}
|\input{childdoc.def}|\\
|\childdocby{|\textit{main}|}|\\
\end{tabular}
\end{center}
%
Both forms have slightly different effects as described above.
The main file is prepared as usual, see \secref{sec:include}.

%%%%%%%%%%%%%%%%%%%%%%%%%%%%%%%%%%%%%%%%%%%%%%%%%%%%%%%%%%%%%%%%%%%%%%%%%%%%%%%%
\subsection{Legacy Detection}
\label{sec:detection}

The directive |\childdocmain| in the main file can detect
whether the complete document or merely a child is to be compiled
even without using the directive |\childdocof|.
This method is deprecated because it is less robust
and there is no compelling reason to use it;
it is merely provided for backward compatibility
and it may be removed in future versions.

If the detection mechanism is to be used,
it is mandatory to correctly specify
the filename of the main file as the argument of |\childdocmain|:
%
\begin{center}
\begin{tabular}{l}
|\input{childdoc.def}|\\
|\childdocmain{|\textit{main}|}|\\
\end{tabular}
\end{center}
%
If |\jobname| does not match the argument \textit{main} of |\childdocmain|,
it is assumed that |\jobname| points to the child file to be compiled.
When using |\childdocmain| with the main file specified as argument,
it suffices to start a child file
with just |\input{|\textit{main}|}|
without loading of the package and using |\childdocof|.
If instead all processing is done
with the appropriate \textsf{childdoc} directives,
the argument of \textit{main} of |\childdocmain| can be empty.

An alternative version of the command line processing described
in \secref{sec:commandline} using the detection mechanism reads:
%
\begin{center}
|... -jobname "|\textit{target}|" "|[\textit{flags}]%
[|\def\jobname{|\textit{dest}|}|]|\input{|\textit{main}|}"|
\end{center}

%%%%%%%%%%%%%%%%%%%%%%%%%%%%%%%%%%%%%%%%%%%%%%%%%%%%%%%%%%%%%%%%%%%%%%%%%%%%%%%%
\subsection{Manual Code}
\label{sec:manual}

In case one cannot be certain whether the definitions file |childdoc.def|
is installed on the target \TeX{} distribution
and one prefers not to ship it,
it is conceivable to paste a few relevant commands into the sources.

To that end, drop all statements |\input{childdoc.def}|
and perform the replacements as outlined below.
Instead of |\childdocmain{|\textit{main}|}| add the following code
to the top of the main file:
%
\begin{center}
\begin{tabular}{l}
|\||ifdefined\childdocname\endinput\||fi\newif\ifchilddoc|\\
|\edef\childdocname{\scantokens\expandafter{\jobname\noexpand}}|\\
|\def\childdocmain{|\textit{main}|}\||ifx\childdocmain\childdocname\||else|\\
|\childdoctrue\includeonly{\childdocname}\let\jobname\childdocmain\||fi|\\
\end{tabular}
\end{center}
%
Instead of |\childdocof{|\textit{main}|}| just include the main file
at the top of each child file:
%
\begin{center}
|\input{|\textit{main}|}|
\end{center}
%
A simple redirection |\childdocforward{|\textit{dest}|}| is achieved by:
%
\begin{center}
|\def\jobname{|\textit{dest}|}\input{\jobname}|
\end{center}
%
The redirection with prefix
|\childdocforwardprefix[|\textit{prefix}|]{|\textit{dest}|}|
is accomplished by:
%
\begin{center}
\begin{tabular}{l}
|{\edef\jobname{\scantokens\expandafter{\jobname\noexpand}}|\\
|\def\redirectjob |\textit{prefix}|#1~~~{\gdef\jobname{|\textit{dest}|#1}}|\\
|\expandafter\redirectjob\jobname~~~}\input{\jobname}|
\end{tabular}
\end{center}

In an alternative approach,
child documents can be compiled by a specific command line
without additional code or specific definitions:
%
\begin{center}
|... -jobname "|\textit{target}|" "|[\textit{flags}]%
|\includeonly{|\textit{dest}|}\input{|\textit{main}|}"|
\end{center}
%

%%%%%%%%%%%%%%%%%%%%%%%%%%%%%%%%%%%%%%%%%%%%%%%%%%%%%%%%%%%%%%%%%%%%%%%%%%%%%%%%
%%%%%%%%%%%%%%%%%%%%%%%%%%%%%%%%%%%%%%%%%%%%%%%%%%%%%%%%%%%%%%%%%%%%%%%%%%%%%%%%
\section{Information}

%%%%%%%%%%%%%%%%%%%%%%%%%%%%%%%%%%%%%%%%%%%%%%%%%%%%%%%%%%%%%%%%%%%%%%%%%%%%%%%%
\subsection{Copyright}

Copyright \copyright{} 2017--2018 Niklas Beisert

This work may be distributed and/or modified under the
conditions of the \LaTeX{} Project Public License, either version 1.3
of this license or (at your option) any later version.
The latest version of this license is in
  \url{http://www.latex-project.org/lppl.txt}
and version 1.3 or later is part of all distributions of \LaTeX{}
version 2005/12/01 or later.

This work has the LPPL maintenance status `maintained'.

The Current Maintainer of this work is Niklas Beisert.

This work consists of the files |README.txt|, |childdoc.ins| and |childdoc.dtx|
as well as the derived files |childdoc.def|, |cdocsamp.tex|
with |cdocsch1.tex|, |cdocsch2.tex|, |cdocspt3.tex|, |cdocspt4.tex|,
|cdocsdrf.tex|, |cdocsfn1.tex|, |cdocsfn2.tex|
as well as |childdoc.pdf|.

%%%%%%%%%%%%%%%%%%%%%%%%%%%%%%%%%%%%%%%%%%%%%%%%%%%%%%%%%%%%%%%%%%%%%%%%%%%%%%%%
\subsection{Files and Installation}

The package consists of the files:
%
\begin{center}
\begin{tabular}{ll}
    |README.txt|   & readme file \\
    |childdoc.ins| & installation file \\
    |childdoc.dtx| & source file \\
    |childdoc.def| & definition file \\
    |cdocsamp.tex| & sample main file \\
    |cdocsch1.tex| & sample include file \\
    |cdocsch2.tex| & sample include file \\
    |cdocspt3.tex| & sample part file \\
    |cdocspt4.tex| & sample part file \\
    |cdocsdrf.tex| & sample redirection file \\
    |cdocsfn1.tex| & sample redirection file \\
    |cdocsfn2.tex| & sample redirection file \\
    |childdoc.pdf| & manual
\end{tabular}
\end{center}
%
The distribution consists of the files
|README.txt|, |childdoc.ins| and |childdoc.dtx|.
%
\begin{itemize}
\item
Run (pdf)\LaTeX{} on |childdoc.dtx|
to compile the manual |childdoc.pdf| (this file).
\item
Run \LaTeX{} on |childdoc.ins| to create the definitions file |childdoc.def|
and the sample |cdocsamp.tex| with include files
|cdocsch1.tex|, |cdocsch2.tex|, |cdocspt3.tex|, |cdocspt4.tex|,
|cdocsdrf.tex|, |cdocsfn1.tex|, |cdocsfn2.tex|.
Then copy the file |childdoc.def| to an appropriate directory of your \LaTeX{}
distribution, e.g.\ \textit{texmf-root}|/tex/latex/childdoc|.
\end{itemize}

%%%%%%%%%%%%%%%%%%%%%%%%%%%%%%%%%%%%%%%%%%%%%%%%%%%%%%%%%%%%%%%%%%%%%%%%%%%%%%%%
\subsection{Related CTAN Packages}

There are several other packages which offer a similar functionality:
%
\begin{itemize}
\item
The packages
\href{http://ctan.org/pkg/docmute}{\textsf{docmute}},
\href{http://ctan.org/pkg/includex}{\textsf{includex}} and
\href{http://ctan.org/pkg/standalone}{\textsf{standalone}}
provide commands to include only the document body of
a child file thus allowing both files to be compiled individually.
\item
The packages \href{http://ctan.org/pkg/subdocs}{\textsf{subdocs}}
and \href{http://ctan.org/pkg/subfiles}{\textsf{subfiles}}
provide structures in which the main and child documents can be
encapsulated and allowing them to be compiled individually.
The inclusion mechanism is different from the conventional |\include|.
\item
The package \href{http://ctan.org/pkg/combine}{\textsf{combine}}
is an elaborate solution to combine several documents into one.
\end{itemize}
%
See also the CTAN topic \href{http://ctan.org/topic/subdocs}{\textsf{subdocs}}
for further related packages.
The present package differs from the above solutions in that
a document structure constructed with the conventional |\include| mechanism
just needs two extra commands at the top of every file
such that all constituent files can be compiled individually.

%%%%%%%%%%%%%%%%%%%%%%%%%%%%%%%%%%%%%%%%%%%%%%%%%%%%%%%%%%%%%%%%%%%%%%%%%%%%%%%%
%\subsection{Feature Suggestions}
%
%The following is a list of features which may be useful for future
%versions of this package:
%%
%\begin{itemize}
%\item
%\ldots
%\end{itemize}

%%%%%%%%%%%%%%%%%%%%%%%%%%%%%%%%%%%%%%%%%%%%%%%%%%%%%%%%%%%%%%%%%%%%%%%%%%%%%%%%
\subsection{Revision History}

%%%%%%%%%%%%%%%%%%%%%%%%%%%%%%%%%%%%%%%%
\paragraph{v2.0:} 2018/12/30

\begin{itemize}
\item
immediate forward processing
\item
added |\childdocby| mechanism
\item
manual restructured
\end{itemize}

%%%%%%%%%%%%%%%%%%%%%%%%%%%%%%%%%%%%%%%%
\paragraph{v1.6:} 2018/01/17

\begin{itemize}
\item
application for development of include files
\item
corrections to manual
\end{itemize}

%%%%%%%%%%%%%%%%%%%%%%%%%%%%%%%%%%%%%%%%
\paragraph{v1.5:} 2017/05/21

\begin{itemize}
\item
more complete structuring introduced
\item
|\childdocof| introduced
\item
|\childdoc| renamed to |\childdocmain|
\item
|\childredirect| renamed to |\childdocforward| and |\childdocforwardprefix|
and functionality expanded
\end{itemize}

%%%%%%%%%%%%%%%%%%%%%%%%%%%%%%%%%%%%%%%%
\paragraph{v1.0:} 2017/04/27

\begin{itemize}
\item
manual and install package
\item
first version published on CTAN
\end{itemize}

%%%%%%%%%%%%%%%%%%%%%%%%%%%%%%%%%%%%%%%%
\paragraph{v0.6:} 2017/04/26

\begin{itemize}
\item
redirection mechanism added
\end{itemize}

%%%%%%%%%%%%%%%%%%%%%%%%%%%%%%%%%%%%%%%%
\paragraph{v0.5:} 2017/04/26

\begin{itemize}
\item
functionality in definition file
\end{itemize}


%%%%%%%%%%%%%%%%%%%%%%%%%%%%%%%%%%%%%%%%%%%%%%%%%%%%%%%%%%%%%%%%%%%%%%%%%%%%%%%%
%%%%%%%%%%%%%%%%%%%%%%%%%%%%%%%%%%%%%%%%%%%%%%%%%%%%%%%%%%%%%%%%%%%%%%%%%%%%%%%%
%%%%%%%%%%%%%%%%%%%%%%%%%%%%%%%%%%%%%%%%%%%%%%%%%%%%%%%%%%%%%%%%%%%%%%%%%%%%%%%%
\appendix

\settowidth\MacroIndent{\rmfamily\scriptsize 000\ }

 \DocInput{childdoc.dtx}

\end{document}
%</driver>
% \fi
%
% %%%%%%%%%%%%%%%%%%%%%%%%%%%%%%%%%%%%%%%%%%%%%%%%%%%%%%%%%%%%%%%%%%%%%%%%%%%%%%
% %%%%%%%%%%%%%%%%%%%%%%%%%%%%%%%%%%%%%%%%%%%%%%%%%%%%%%%%%%%%%%%%%%%%%%%%%%%%%%
% \section{Sample}
%\iffalse
%<*samplemain>
%\fi
%
% The following presents a sample document
% with two chapters, two parts, a title page,
% a compile flag as well as three forwarding files to set the flag.
% It consists of eight |.tex| files:
% \begin{center}
% \begin{tabular}{ll}
% |cdocsamp.tex|&main file\\
% |cdocsch1.tex|&include file for chapter 1\\
% |cdocsch2.tex|&include file for chapter 2\\
% |cdocspt3.tex|&include file for part 3\\
% |cdocspt4.tex|&include file for part 4\\
% |cdocsdrf.tex|&forwarding file for main file in draft mode\\
% |cdocsfi1.tex|&forwarding file for final version of chapter 1\\
% |cdocsfi2.tex|&forwarding file for final version of chapter 2\\
% \end{tabular}
% \end{center}
% Each of the eight files can be compiled directly by the \LaTeX{} compiler.
%
% %%%%%%%%%%%%%%%%%%%%%%%%%%%%%%%%%%%%%%
% \paragraph{Main File.}
%
% The main file is called |cdocsamp.tex|.
%
% Load the \textsf{childdoc} definitions and
% declare the filename for the main document:
%    \begin{macrocode}
\input{childdoc.def}
\childdocmain{}
%    \end{macrocode}

% Optional override for |\version| flag:
%    \begin{macrocode}
%%\ifchilddoc\else\providecommand{\version}{draft}\fi
%    \end{macrocode}

% Define the default values for the |\version| flag
% (|final| for the main file and |draft| for childs):
%    \begin{macrocode}
\ifchilddoc
\providecommand{\version}{draft}
\else
\providecommand{\version}{final}
\fi
%    \end{macrocode}

% Load the standard document class:
%    \begin{macrocode}
\documentclass[12pt]{article}
%    \end{macrocode}

% Start the document body:
%    \begin{macrocode}
\begin{document}
%    \end{macrocode}

% Declare a title page.
% Print title, part of document being processed and version flag:
%    \begin{macrocode}
\addtocounter{page}{-1}
\begin{center}
{\LARGE\bfseries{}childdoc example\par}
\vspace{1cm}
\ifchilddoc
\ifchilddocmanual part\else chapter\fi:
`\childdocname' of `\childdocjob'\par
\else
main document: `\childdocjob'\par
\fi
version: \version\par
\end{center}
\newpage
%    \end{macrocode}

% Manually include selected file,
% otherwise process as usual:
%    \begin{macrocode}
\ifchilddocmanual
\section*{part `\childdocname'}
\input{\childdocname}
\else
%    \end{macrocode}

% Include the two chapters:
%    \begin{macrocode}
\include{cdocsch1}
\include{cdocsch2}
%    \end{macrocode}

% Include the two parts unless only chapters should be displayed:
%    \begin{macrocode}
\ifchilddoc\else
\section{part three}
\input{cdocspt3}
\section{part four}
\input{cdocspt4}
\fi
%    \end{macrocode}

% Process as usual until here:
%    \begin{macrocode}
\fi
%    \end{macrocode}

% End of document body:
%    \begin{macrocode}
\end{document}
%    \end{macrocode}
%\iffalse
%</samplemain>
%\fi
%
% %%%%%%%%%%%%%%%%%%%%%%%%%%%%%%%%%%%%%%
% \paragraph{Chapter Include Files.}
%
% The include files are called |cdocsch1.tex| and |cdocsch2.tex|.
%
%\iffalse
%<*samplechap1|samplechap2>
%\fi

% Optional override for |\version| flag:
%    \begin{macrocode}
%%\providecommand{\version}{final}
%    \end{macrocode}

% Include the main document:
%    \begin{macrocode}
\input{childdoc.def}
\childdocof{cdocsamp}
%    \end{macrocode}

%\iffalse
%</samplechap1|samplechap2>
%\fi
%
%\iffalse
%<*samplechap1>
%\fi
% Some text for chapter 1:
%    \begin{macrocode}
\section{one}
some text in chapter one
%    \end{macrocode}

%\iffalse
%</samplechap1>
%\fi
% Some text for chapter 2:
%\iffalse
%<*samplechap2>
%\fi
%    \begin{macrocode}
\section{two}
more text in chapter two
%    \end{macrocode}

%\iffalse
%</samplechap2>
%\fi
%
% %%%%%%%%%%%%%%%%%%%%%%%%%%%%%%%%%%%%%%
% \paragraph{Part Include Files.}
%
% The include files are called |cdocspt3.tex| and |cdocspt4.tex|.
%
%\iffalse
%<*samplepart3|samplepart4>
%\fi

% Optional override for |\version| flag:
%    \begin{macrocode}
%%\providecommand{\version}{final}
%    \end{macrocode}

% Include the main document:
%    \begin{macrocode}
\input{childdoc.def}
\childdocby{cdocsamp}
%    \end{macrocode}

%\iffalse
%</samplepart3|samplepart4>
%\fi
%
%\iffalse
%<*samplepart3>
%\fi
% Some text for part 3:
%    \begin{macrocode}
some text in part three
%    \end{macrocode}

%\iffalse
%</samplepart3>
%\fi
% Some text for part 4:
%\iffalse
%<*samplepart4>
%\fi
%    \begin{macrocode}
more text in part four
%    \end{macrocode}

%\iffalse
%</samplepart4>
%\fi
%
% %%%%%%%%%%%%%%%%%%%%%%%%%%%%%%%%%%%%%%
% \paragraph{Forwarding for a Complete Draft.}
%
% The following forwarding file |cdocsdrf.tex|
% compiles the main document in draft mode:
%\iffalse
%<*sampledraft>
%\fi
%    \begin{macrocode}
\def\version{draft}
\input{childdoc.def}
\childdocforward{cdocsamp}
%    \end{macrocode}

%\iffalse
%</sampledraft>
%\fi
%
% %%%%%%%%%%%%%%%%%%%%%%%%%%%%%%%%%%%%%%
% \paragraph{Forwarding for Final Version of the Chapters.}
%
% The following forwarding files |cdocsfn1.tex| and |cdocsfn2.tex|
% (with identical content)
% compile the final versions of the child documents
% |cdocsch1.tex| and |cdocsch2.tex|, respectively:
%\iffalse
%<*samplefinal>
%\fi
%    \begin{macrocode}
\def\version{final}
\input{childdoc.def}
\childdocforwardprefix[cdocsamp]{cdocsfn}{cdocsch}
%    \end{macrocode}

%\iffalse
%</samplefinal>
%\fi
%
% %%%%%%%%%%%%%%%%%%%%%%%%%%%%%%%%%%%%%%
% \paragraph{Command Line Processing.}
%
% The following three command lines generate the output files
% |cdocscld|, |cdocscl1| and |cdocscl2|
% which should be identical to
% |cdocsdrf|, |cdocsch1| and |cdocsfn2|, respectively:
% \begin{center}
% \begin{tabular}{l}
% |latex -jobname cdocscld \|\\
% |  "\def\version{draft}\input{childdoc.def}\childdocforward{cdocsamp}"|\\
% |latex -jobname cdocscl1 \|\\
% |  "\input{childdoc.def}\childdocforward[cdocsamp]{cdocsch1}"|\\
% |latex -jobname cdocscl2 \|\\
% |  "\def\version{final}\input{childdoc.def}\childdocforward{cdocsch2}"|
% \end{tabular}
% \end{center}
% Note that the trailing backslash on each first line
% merely continues the input to the second line
% (for convenient cut ant paste).
% Furthermore, the command |latex| can be replaced by any
% of its alternative versions such as |pdflatex|.
%
% %%%%%%%%%%%%%%%%%%%%%%%%%%%%%%%%%%%%%%%%%%%%%%%%%%%%%%%%%%%%%%%%%%%%%%%%%%%%%%
% %%%%%%%%%%%%%%%%%%%%%%%%%%%%%%%%%%%%%%%%%%%%%%%%%%%%%%%%%%%%%%%%%%%%%%%%%%%%%%
% \section{Implementation}
%\iffalse
%<*package>
%\fi
%
% This section describes the definitions file |childdoc.def|.

% The definitions cannot be loaded using |\usepackage| or |\RequirePackage|
% which has a mechanism to prevent loading a style file more than once.
% When loading the definitions by means of |\input|
% multiple instances have to be prevented manually:
%\iffalse
%This code needs to be before the `\ProvidesFile' directive
%which is defined at the beginning of this file.
%Therefore it is also placed there and commented out here.
%</package>
%<*discard>
%\fi
%    \begin{macrocode}
\ifdefined\childdocmain\endinput\fi
%    \end{macrocode}
%\iffalse
%</discard>
%<*package>
%\fi
%
% \macro{\ifchilddoc}
% \macro{\ifchilddocmanual}
% The conditional |\ifchilddoc| tells whether a
% child (true) or main (false) document is being compiled.
% The conditional |\ifchilddocmanual| tells whether
% the |\includeonly| mechanism is used (false) or
% the selection of child files must be performed manually (true).
% The definitions initialise to false:
%    \begin{macrocode}
\newif\ifchilddoc
\newif\ifchilddocmanual
%    \end{macrocode}

% \macro{\childdocname}
% \macro{\childdocjob}
% The macro |\childdocname| stores the name of the main document
% to be compiled. The macro |\childdocjob| stores the name of
% the document on which the \LaTeX{} compiler was originally invoked.
% The content of |\jobname| cannot be compared
% to filenames specified in the source due to different catcodes.
% The following code rescans |\jobname|, stores the result
% in |\childdocname| and saves a copy in |\childdocjob|:
%    \begin{macrocode}
\edef\childdocname{\scantokens\expandafter{\jobname\noexpand}}
\let\childdocjob\childdocname
%    \end{macrocode}

% \macro{\childdocdisable}
% The macro |\childdocdisable| prevents the main file
% from being processed more than once.
% At this stage, the main document command |\childdocmain|
% is assumed to be called once again where it should do nothing.
% Any subsequent call to it should prevent
% a secondary processing of the main document
% It overwrites the forwarding commands
% |\childdocof| and |\childdocforward|
% with empty macros to prevent further inclusions of the main document:
%    \begin{macrocode}
\newcommand{\childdocdisable}
{
  \renewcommand{\childdocmain}[1]{\renewcommand{\childdocmain}[1]{\endinput}}
  \renewcommand{\childdocof}[1]{}
  \renewcommand{\childdocby}[2][]{}
  \renewcommand{\childdocforward}[2][]{}
  \renewcommand{\childdocdisable}{}
}
%    \end{macrocode}

% \macro{\childdocmain}
% The macro |\childdocmain| is to be called at the top of the main file
% with nothing or the main filename (without extension) as argument.
% First, it breaks loops.
% If the argument is not empty and does not match |\childdocname|
% (which is set by the first inclusion of |childdoc.def|),
% |\ifchilddoc| is set to true, |\includeonly| is applied to the child file
% and |\jobname| is set to the main file
% (for proper handling of |.aux| files):
%    \begin{macrocode}
\newcommand{\childdocmain}[1]
{
  \childdocdisable\childdocmain{}
  \if?#1?\else
    \begingroup
      \def\childdoctmp{#1}
      \ifx\childdoctmp\childdocname
        \def\childdoctmp{}
      \else
        \def\childdoctmp
        {
          \childdoctrue
          \includeonly{\childdocname}
          \def\childdocjob{#1}
          \def\jobname{#1}
        }
      \fi
      \expandafter
    \endgroup
    \childdoctmp
  \fi
}
%    \end{macrocode}

% \macro{\childdocof}
% The command |\childdocof| redirects
% compilation to the main file |#1|.
%    \begin{macrocode}
\newcommand{\childdocof}[1]
{
  \childdocdisable
  \childdoctrue
  \includeonly{\childdocname}
  \def\jobname{#1}
  \def\childdocjob{#1}
  \input{#1}
}
%    \end{macrocode}

% \macro{\childdocby}
% The command |\childdocby| ....
%    \begin{macrocode}
\newcommand{\childdocby}[2][]
{
  \childdocdisable
  \childdoctrue
  \childdocmanualtrue
  \if?#1?\else
    \def\jobname{#2}
  \fi
  \def\childdocjob{#2}
  \input{#2}
  \endinput
}
%    \end{macrocode}

% \macro{\childdocforward}
% The command |\childdocforward| redirects
% compilation to the main file or
% (if the optional argument is given) a child file.
% Parameters are set as if the main file
% or a child file starting with |\childdocof| was compiled.
% Then compilation is handed over to the main file:
%    \begin{macrocode}
\newcommand{\childdocforward}[2][]
{
  \begingroup
    \if?#1?
      \def\childdoctmp
      {
        \def\childdocname{#2}
        \def\childdocjob{#2}
        \def\jobname{#2}
        \input{#2}
        \endinput
      }
    \else
      \def\childdoctmp
      {
        \childdocdisable
        \def\childdocname{#2}
        \childdoctrue
        \includeonly{#2}
        \def\childdocjob{#1}
        \def\jobname{#1}
        \input{#1}
        \endinput
      }
    \fi
    \expandafter
  \endgroup
  \childdoctmp
}
%    \end{macrocode}

% \macro{\childdocforwardprefix}
% The command |\childdocforwardprefix| redirects
% compilation to the main or a child file by means of a pattern.
% The prefix |#1| in the current filename is replaced by |#2|
% and the suffix of the current filename is kept
% (it is assumed that the filename does not contain the substring `|~~~|'
% which is used as a delimiter).
% Compilation is handed over to the new file by |\childdocforward|:
%    \begin{macrocode}
\newcommand{\childdocforwardprefix}[3][]
{
  \begingroup
    \def\childdocextract #2##1~~~{\def\childdoctmp{\childdocforward[#1]{#3##1}}}
    \expandafter\childdocextract\childdocname~~~
    \expandafter
  \endgroup
  \childdoctmp
}
%    \end{macrocode}

% \macro{\childdoc}
% The deprecated macro |\childdoc| is a legacy version of |\childdocmain|:
%    \begin{macrocode}
\newcommand{\childdoc}{\childdocmain}
%    \end{macrocode}

% \macro{\childdocredirect}
% The deprecated macro |\childdocredirect| is a legacy version
% of |\childdocforward| and |\childdocforwardprefix|:
%    \begin{macrocode}
\newcommand{\childdocredirect}[2][]
{
  \begingroup
    \if?#1?
      \def\childdoctmp{\childdocforward{#2}}
    \else
      \def\childdoctmp{\childdocforwardprefix{#1}{#2}}
    \fi
    \expandafter
  \endgroup
  \childdoctmp
}
%    \end{macrocode}

%\iffalse
%</package>
%\fi
%
\endinput

\childdocof{cdocsamp}
%    \end{macrocode}

%\iffalse
%</samplechap1|samplechap2>
%\fi
%
%\iffalse
%<*samplechap1>
%\fi
% Some text for chapter 1:
%    \begin{macrocode}
\section{one}
some text in chapter one
%    \end{macrocode}

%\iffalse
%</samplechap1>
%\fi
% Some text for chapter 2:
%\iffalse
%<*samplechap2>
%\fi
%    \begin{macrocode}
\section{two}
more text in chapter two
%    \end{macrocode}

%\iffalse
%</samplechap2>
%\fi
%
% %%%%%%%%%%%%%%%%%%%%%%%%%%%%%%%%%%%%%%
% \paragraph{Part Include Files.}
%
% The include files are called |cdocspt3.tex| and |cdocspt4.tex|.
%
%\iffalse
%<*samplepart3|samplepart4>
%\fi

% Optional override for |\version| flag:
%    \begin{macrocode}
%%\providecommand{\version}{final}
%    \end{macrocode}

% Include the main document:
%    \begin{macrocode}
% \iffalse
%
% childdoc.dtx Copyright (C) 2017-2018 Niklas Beisert
%
% This work may be distributed and/or modified under the
% conditions of the LaTeX Project Public License, either version 1.3
% of this license or (at your option) any later version.
% The latest version of this license is in
%   http://www.latex-project.org/lppl.txt
% and version 1.3 or later is part of all distributions of LaTeX
% version 2005/12/01 or later.
%
% This work has the LPPL maintenance status `maintained'.
%
% The Current Maintainer of this work is Niklas Beisert.
%
% This work consists of the files childdoc.dtx and childdoc.ins
% and the derived files childdoc.def and cdocsamp.tex with
% cdocsch1.tex, cdocsch2.tex, cdocsdrf.tex, cdocsfn1.tex, cdocsfn2.tex.
%
%<package>\ifdefined\childdocmain\endinput\fi
%<package>\ProvidesFile{childdoc.def}[2018/12/30 v2.0 child document driver]
%<samplemain>\ProvidesFile{cdocsamp.tex}[2018/12/30 v2.0 sample for childdoc]
%<*driver>
%\ProvidesFile{childdoc.drv}[2018/12/30 v2.0 childdoc reference manual file]
\PassOptionsToClass{10pt,a4paper}{article}
\documentclass{ltxdoc}

\usepackage[margin=35mm]{geometry}
\usepackage{hyperref}
\usepackage{hyperxmp}
\usepackage[usenames]{color}

\hypersetup{colorlinks=true}
\hypersetup{pdfstartview=FitH}
\hypersetup{pdfpagemode=UseNone}
\hypersetup{pdfsource={}}
\hypersetup{pdflang={en-UK}}
\hypersetup{pdfcopyright={Copyright 2017-2018 Niklas Beisert.
  This work may be distributed and/or modified under the
  conditions of the LaTeX Project Public License, either version 1.3
  of this license or (at your option) any later version.}}
\hypersetup{pdflicenseurl={http://www.latex-project.org/lppl.txt}}
\hypersetup{pdfcontactaddress={ETH Zurich, ITP, HIT K,
  Wolfgang-Pauli-Strasse 27}}
\hypersetup{pdfcontactpostcode={8093}}
\hypersetup{pdfcontactcity={Zurich}}
\hypersetup{pdfcontactcountry={Switzerland}}
\hypersetup{pdfcontactemail={nbeisert@itp.phys.ethz.ch}}
\hypersetup{pdfcontacturl={http://people.phys.ethz.ch/\xmptilde nbeisert/}}

\newcommand{\secref}[1]{\hyperref[#1]{section \ref*{#1}}}

\parskip1ex
\parindent0pt
\let\olditemize\itemize
\def\itemize{\olditemize\parskip0pt}

\begin{document}

\title{The \textsf{childdoc} Package}
\hypersetup{pdftitle={The childdoc Package}}
\author{Niklas Beisert\\[2ex]
  Institut f\"ur Theoretische Physik\\
  Eidgen\"ossische Technische Hochschule Z\"urich\\
  Wolfgang-Pauli-Strasse 27, 8093 Z\"urich, Switzerland\\[1ex]
  \href{mailto:nbeisert@itp.phys.ethz.ch}
  {\texttt{nbeisert@itp.phys.ethz.ch}}}
\hypersetup{pdfauthor={Niklas Beisert}}
\hypersetup{pdfsubject={Manual for the LaTeX2e Package childdoc}}
\date{30 December 2018, \textsf{v2.0}}
\maketitle

\begin{abstract}\noindent
\textsf{childdoc} is a \LaTeXe{} package
that enables the direct compilation
of document sections included by |\include|
to individual files.
\end{abstract}

\begingroup
\parskip0ex
\tableofcontents
\endgroup

%%%%%%%%%%%%%%%%%%%%%%%%%%%%%%%%%%%%%%%%%%%%%%%%%%%%%%%%%%%%%%%%%%%%%%%%%%%%%%%%
%%%%%%%%%%%%%%%%%%%%%%%%%%%%%%%%%%%%%%%%%%%%%%%%%%%%%%%%%%%%%%%%%%%%%%%%%%%%%%%%
\section{Introduction}

\LaTeX{} provides a mechanism to structure a large document (such as a book)
into a main file and several child files (containing the chapters)
using the |\include| command.
This mechanism is beneficial for documents
which span hundreds of pages in order to
make the source file(s) more manageable.
Moreover, compilation can be restricted to
selected child files by means of the |\includeonly| command.
The latter feature can be used to reduce the compilation time while editing
(this was significantly more useful in the earlier days of \LaTeX{})
or to generate a smaller document which is easier to navigate.
Another application of |\includeonly| is to generate
documents consisting of selected parts of the complete document.

However, there are a few drawbacks of the plain |\include| mechanism:
\begin{itemize}
\item
The child files cannot be compiled on their own,
they can only be compiled via the main file.
A naive editing environment
(such as a text editor with an option
to have the current file processed by \LaTeX)
may require one to switch to the main file before compiling;
attempting to compile the child file produces errors.
\item
The main file must be modified (each time)
to adjust the |\includeonly| command
to the present needs. This easily leaves the main file in a messy state.
\item
The generated document will always carry the filename
of the main document. This is inconvenient if
several child files are to be compiled and
to be kept for distribution.
\end{itemize}

The present package provides a simple interface
to make child files individually compilable by \LaTeX{}.
Compiling a child file then has the same effect as compiling
the main file with an |\includeonly| command
to select the appropriate child.
Moreover the generated document will carry the name of the child
rather than the main file.
This resolves all three above issues.

This feature is meant to make the editing of books,
thesis documents and lecture notes somewhat more convenient.
However, the package can also be used efficiently for
composing a series of documents (such as exercise sheets)
which are typically distributed individually.
It then assists the author in generating the individual documents
(potentially in different versions)
as well as a document containing the collected series.
Another application is in developing style files
or other kinds of included material
where compilation of the style file could redirect
to a sample or test file.

%%%%%%%%%%%%%%%%%%%%%%%%%%%%%%%%%%%%%%%%%%%%%%%%%%%%%%%%%%%%%%%%%%%%%%%%%%%%%%%%
%%%%%%%%%%%%%%%%%%%%%%%%%%%%%%%%%%%%%%%%%%%%%%%%%%%%%%%%%%%%%%%%%%%%%%%%%%%%%%%%
\section{Usage}

First of all, the package \textsf{childdoc} is \emph{not} a standard
\LaTeXe{} |.sty| style file! Therefore it needs to be invoked in
a non-standard way.

%%%%%%%%%%%%%%%%%%%%%%%%%%%%%%%%%%%%%%%%%%%%%%%%%%%%%%%%%%%%%%%%%%%%%%%%%%%%%%%%
\subsection{Included Files}
\label{sec:include}

%%%%%%%%%%%%%%%%%%%%%%%%%%%%%%%%%%%%%%%%
\DescribeMacro{\childdocmain}
To use the package, add the commands
\begin{center}
\begin{tabular}{l}
|\input{childdoc.def}|\\
|\childdocmain{}|\\
\end{tabular}
\end{center}
at the very top of the main \LaTeX{} file,
in particular \emph{before} the |\documentclass| statement!
The argument of |\childdocmain| should be left empty
(but it must be present).

%%%%%%%%%%%%%%%%%%%%%%%%%%%%%%%%%%%%%%%%
\DescribeMacro{\childdocof}
Furthermore, add the commands
\begin{center}
\begin{tabular}{l}
|\input{childdoc.def}|\\
|\childdocof{|\textit{main}|}|\\
\end{tabular}
\end{center}
at the top of every child file \textit{child}
which is included by |\include{|\textit{child}|}|
from within the main file
(or at least for those files to be compiled individually).
The argument \textit{main} must be the filename of the main file.

There are a couple of
considerations in setting up the main and child documents:

%%%%%%%%%%%%%%%%%%%%%%%%%%%%%%%%%%%%%%%%
\paragraph{Restrictions.}

Please note the following restrictions:
\begin{itemize}
\item
|\childdocmain| must be called with one argument \textit{main}
to ensure compatibility with earlier version of the package.
It must either be empty (|\childdocmain{}|)
or precisely match the filename of the main file in which it is specified.
See \secref{sec:detection} for further information.
\item
The filename \textit{main} must be specified without the |.tex| extension.
\item
The filename \textit{main} is case sensitive
(even in case-insensitive file systems)
due to internal string comparison.
\item
The argument \textit{main} should be fully expanded, it cannot be a macro.
\item
Subdirectories and special characters should be avoided in filenames.
\item
The command |\childdocmain{|\textit{main}|}| must be followed by a whitespace.
It should not be followed immediately by another command
or by a comment mark `|%|'.
This is because the \TeX{} parser reads the token immediately following
the argument of |\childdocmain| and puts it
at the beginning of every child section;
however, a white\-space is ignored.
\end{itemize}

%%%%%%%%%%%%%%%%%%%%%%%%%%%%%%%%%%%%%%%%
\paragraph{Content of Main File.}

It is advisable to place all content in the child files included by |\include|.
Any output contained in the main file will appear in all child documents
unless suppressed manually;
it cannot be suppressed automatically by the |\includeonly| directive
and thus should normally be avoided.
A method to include some content in the main file
by means of conditional processing is described in \secref{sec:conditional}.

%%%%%%%%%%%%%%%%%%%%%%%%%%%%%%%%%%%%%%%%
\paragraph{Page Numbering.}

When only a part of the document is compiled,
the appropriate numbering of pages
(as well as other status parameters)
is determined from the |.aux| files.
The latter contain information from previous passes.
However this information needs to propagate through
all intermediate child documents.
Therefore the page numbering in child documents may well
be inconsistent until the complete document is compiled at least once.

A useful (if unconventional) way to always ensure a consistent
page numbering is to restart the numbering in each child document
and denote the pages by `\textit{child}|.|\textit{page}'
where \textit{child} represents the chapter/section number of the child file.
This can be achieved by the command
|\numberwithin{page}{|\textit{child}|}|
of the \textsf{amsmath} package
where \textit{child} can be |chapter| or |section|
depending on the chosen structuring.
Alternatively, one can modify the macro |\thepage| appropriately
and reset the counter |page| at the start of each child file.

%%%%%%%%%%%%%%%%%%%%%%%%%%%%%%%%%%%%%%%%%%%%%%%%%%%%%%%%%%%%%%%%%%%%%%%%%%%%%%%%
\subsection{Conditional Processing}
\label{sec:conditional}

The package provides a mechanism to compile different versions
of a document. To customise the versions further some conditional processing
can come in handy to distinguish which version is being compiled.
The package provides two macros to describe the compilation context:

%%%%%%%%%%%%%%%%%%%%%%%%%%%%%%%%%%%%%%%%
\DescribeMacro{\ifchilddoc}
The conditional |\ifchilddoc| distinguishes between the compilation of
child documents and the main document:
%
\begin{center}
|\ifchilddoc |\textit{child-code}| |[|\||else |\textit{main-code}]| \||fi|
\end{center}

%%%%%%%%%%%%%%%%%%%%%%%%%%%%%%%%%%%%%%%%
\DescribeMacro{\childdocname}
\DescribeMacro{\childdocjob}
The macro |\childdocname| contains the filename (without extension)
of the main or child file being processed.
Note that |\childdocjob| will always contain the name of the main file.

%%%%%%%%%%%%%%%%%%%%%%%%%%%%%%%%%%%%%%%%
\paragraph{Title Page.}

Conditional processing can be used to include a title or banner page
in the main document when proper precautions are taken.
Importantly, the code in the main file should ensure that the page counter
(as well as other status parameters which are stored in the |.aux| files)
takes the same value after the conditional processing.
Otherwise the page numbers may take divergent values
depending on which part is compiled.

For example, a title page could be declared by:
%
\begin{center}
\begin{tabular}{l}
|\ifchilddoc\||else|\\
|\addtocounter{page}{-1}|\\
\textit{code for title page}\\
|\newpage|\\
|\||fi|
\end{tabular}
\end{center}
%
A banner page for the child documents can be generated by:
%
\begin{center}
\begin{tabular}{l}
|\ifchilddoc|\\
|\addtocounter{page}{-1}|\\
\textit{code for banner page}\\
|\newpage|\\
|\||fi|
\end{tabular}
\end{center}
%
Here one could write a message such as:
\begin{center}
|This is the part \childdocname{} of \childdocjob{}.|
\end{center}

%%%%%%%%%%%%%%%%%%%%%%%%%%%%%%%%%%%%%%%%%%%%%%%%%%%%%%%%%%%%%%%%%%%%%%%%%%%%%%%%
\subsection{Flags}
\label{sec:flags}

The package makes it easy to generate different versions
of the main or child documents.
To this end compilation flags can be defined
and assigned different default values.
They will be particularly useful in conjunction
with the forwarding mechanism described in \secref{sec:forward}.

For example, it may be useful to have a flag |\version|
which can be set to |draft| or |final|.
The document source will contain some conditional code
depending on the value of |\version|.
Suppose further, the flag should default to |final| for the main file
and to |draft| for child files
which is a natural assignment for editing the document.
This is achieved by placing the following code
in the preamble of the main document
(below the |\childdocmain| directive):
%
\begin{center}
\begin{tabular}{l}
|\ifchilddoc|\\
|\providecommand{\version}{draft}|\\
|\||else|\\
|\providecommand{\version}{final}|\\
|\||fi|
\end{tabular}
\end{center}
%
The definition by |\providecommand| makes sure
that previous definitions are not overwritten.
Further statements |\providecommand{\version}{...}|
can thus be added before the above code to override it.

For the main file, one might add a line
(between |\childdocmain| and the above block)
%
\begin{center}
|%\ifchilddoc\||else\providecommand{\version}{draft}\||fi|
\end{center}
%
which can be uncommented to produce a draft version.
Likewise one can add a line to the very top of a child file
(above the |\childdocof{|\textit{main}|}| directive)
%
\begin{center}
|%\providecommand{\version}{final}|
\end{center}
%
which can be uncommented to produce the final version of this child document.

%%%%%%%%%%%%%%%%%%%%%%%%%%%%%%%%%%%%%%%%%%%%%%%%%%%%%%%%%%%%%%%%%%%%%%%%%%%%%%%%
\subsection{Forwarding}
\label{sec:forward}

Different versions of the main or child documents
using compilation flags as described in \secref{sec:flags}
can be (permanently) stored in different files
for convenient compilation, viewing and distribution.
To this end, the package defines a command
to pass on compilation to a different file:

%%%%%%%%%%%%%%%%%%%%%%%%%%%%%%%%%%%%%%%%
\DescribeMacro{\childdocforward}
The command |\childdocforward| redirects processing to
another source file:
%
\begin{center}
\begin{tabular}{l}
|\input{childdoc.def}|\\
|\childdocforward[|\textit{main}|]{|\textit{dest}|}|\\
\end{tabular}
\end{center}
%
The argument \textit{dest} is the destination file
(without extension).
It should be the main file or one of the child files.
Note that further \textsf{childdoc} directives
such as |\childdocof| and |\childdocforward|
in the indicated file will be processed in this form.
The optional argument \textit{main}
passes on directly to the main file \textit{main}
while pretending to compile the child \textit{dest}.
This form behaves as if \textit{dest}
issues |\childdocof{|\textit{main}|}| right away,
and no further \textsf{childdoc} directives will be processed.

%%%%%%%%%%%%%%%%%%%%%%%%%%%%%%%%%%%%%%%%
\DescribeMacro{\...prefix}
In the alternative form |\childdocforwardprefix|,
%
\begin{center}
\begin{tabular}{l}
|\input{childdoc.def}|\\
|\childdocforwardprefix[|\textit{main}|]{|\textit{prefix}|}{|\textit{dest}|}|
\end{tabular}
\end{center}
%
the destination file is determined by a pattern
depending on the current file:
To make this work, the current file must be called
`{\textit{prefix}\hspace{0.2em}\textit{suffix}}'
with \textit{prefix} matching precisely the argument.
Processing is then passed on to the file
`{\textit{dest}\hspace{0.2em}\textit{suffix}}'.
Surely, the same effect is achieved by
directly specifying the
argument `{\textit{dest}\hspace{0.2em}\textit{suffix}}'
in the first form.
However, that requires to set up a different file
for each child. With the alternative form of the command
all these files can have exactly the same content
which simplifies setting them up and maintaining them.

For example, the following file |draft.tex|
with a compilation flag |\version| as described in \secref{sec:flags}
compiles the main document as a draft:
%
\begin{center}
\begin{tabular}{l}
|\def\version{draft}|\\
|\input{childdoc.def}|\\
|\childdocforward{|\textit{main}|}|
\end{tabular}
\end{center}
%
Likewise, the following files |final|\textit{nn}|.tex|
compile the final version of the child document
|child|\textit{nn}|.tex|:
%
\begin{center}
\begin{tabular}{l}
|\def\version{final}|\\
|\input{childdoc.def}|\\
|\childdocforwardprefix{final}{child}|
\end{tabular}
\end{center}
%

Note that when several versions of a main file and/or of each child file
are to be generated, it may be convenient to set up a |Makefile| or
shell script to automatise the process.

%%%%%%%%%%%%%%%%%%%%%%%%%%%%%%%%%%%%%%%%%%%%%%%%%%%%%%%%%%%%%%%%%%%%%%%%%%%%%%%%
\subsection{Command Line Processing}
\label{sec:commandline}

The effect of redirection files can also be achieved by invoking
the \LaTeX{} compiler with a more elaborate command line.
Most conveniently this should be done as part
of a shell script or a |Makefile|.

When using \textsf{childdoc} in the main file, the following
command lines effectively perform a redirection
(note that depending on the shell being used,
backslashes may have to be doubled: `|\|' $\to$ `|\\|'):
%
\begin{center}
|... -jobname "|\textit{target}|" |\\|"|[\textit{flags}]%
|\input{childdoc.def}\childdocforward[|\textit{main}|]{|\textit{dest}|}"|
\end{center}
%
Here \textit{target} is the name of the output file,
\textit{main} is the name of the main file
and \textit{dest} is the name of the main or child file to be processed
(all filenames without extensions).
The optional argument \textit{main} can be omitted
if \textit{main} matches \textit{dest}.
Optionally, compilation \textit{flags} can be defined via |\def| commands.
This command line makes the \TeX{} engine believe
it is compiling the file \textit{target}
whose content is specified as the latter parameter.
The provided code then forwards the processing to
\textit{main} or \textit{dest} as described in \secref{sec:forward}.

%%%%%%%%%%%%%%%%%%%%%%%%%%%%%%%%%%%%%%%%%%%%%%%%%%%%%%%%%%%%%%%%%%%%%%%%%%%%%%%%
\subsection{Include by Input}
\label{sec:input}

Including child documents by |\include| has some restrictions by design.
Most notably, the content of a child document always occupies
its own set of pages; pages cannot be shared between child documents.
Usually, this behaviour makes perfect sense
because each child document contain an essential part of the document.
However, in some situations it may be desirable to compose
a document from a collection of parts
without having mandatory page breaks between then.
For this case, the package
provides a mechanism to include parts
by |\input| which can also be processed individually.
However, by construction this mechanism
requires manual handling of the content to be output.

%%%%%%%%%%%%%%%%%%%%%%%%%%%%%%%%%%%%%%%%
\DescribeMacro{\ifchilddocmanual}
The main file should be prepared as usual, see \secref{sec:include}.
However, the document body must make a distinction
between processing of an individual part and of the main document, e.g.:
%
\begin{center}
\begin{tabular}{l}
|\ifchilddocmanual|\\
|\input{\childdocname}|\\
|\||else|\\
\textit{document body with }|\input{|\textit{part}|}|\\
|\||fi|
\end{tabular}
\end{center}
%
The conditional |\ifchilddocmanual| is true whenever
a part to be included by |\input| is being compiled,
and the name of the part is stored in |\childdocname|.

%%%%%%%%%%%%%%%%%%%%%%%%%%%%%%%%%%%%%%%%
\DescribeMacro{\childdocby}
Each part to be included by |\input| should start with:
%
\begin{center}
\begin{tabular}{l}
|\input{childdoc.def}|\\
|\childdocby{|\textit{main}|}|\\
\end{tabular}
\end{center}
%
The directive |\childdocby| is similar to |\childdocof|
described in \secref{sec:include},
but the subsequent selection of content must be done manually.
To that end, both |\ifchilddoc| and |\ifchilddocmanual|
will be true upon processing of a part,
and the name of the part is stored in |\childdocname|.
Note that |\jobname| will be set to the filename of the current part
so that each part receives an individual |.aux| file
that does not interfere with the |.aux| file(s) of the main document.
This behaviour can be altered by the alternative form
|\childdocby[*]{|\textit{main}|}| (with a non-empty optional argument)
which uses the |.aux| file of the main document
by setting |\jobname| to \textit{main}.

%%%%%%%%%%%%%%%%%%%%%%%%%%%%%%%%%%%%%%%%%%%%%%%%%%%%%%%%%%%%%%%%%%%%%%%%%%%%%%%%
\subsection{Driver Development}
\label{sec:driver}

The \textsf{childdoc} mechanism can also be use for the development
of definition files such as \LaTeX{} styles or classes.
This case differs from the above setup with multiple parts
included by |\include| in that no |\includeonly| should be invoked.
This can be achieved by starting the include file
(before |\ProvidesPackage|) with:
%
\begin{center}
\begin{tabular}{l}
|\input{childdoc.def}|\\
|\childdocforward{|\textit{main}|}|\\
\end{tabular}
\end{center}
%
or alternatively with:
%
\begin{center}
\begin{tabular}{l}
|\input{childdoc.def}|\\
|\childdocby{|\textit{main}|}|\\
\end{tabular}
\end{center}
%
Both forms have slightly different effects as described above.
The main file is prepared as usual, see \secref{sec:include}.

%%%%%%%%%%%%%%%%%%%%%%%%%%%%%%%%%%%%%%%%%%%%%%%%%%%%%%%%%%%%%%%%%%%%%%%%%%%%%%%%
\subsection{Legacy Detection}
\label{sec:detection}

The directive |\childdocmain| in the main file can detect
whether the complete document or merely a child is to be compiled
even without using the directive |\childdocof|.
This method is deprecated because it is less robust
and there is no compelling reason to use it;
it is merely provided for backward compatibility
and it may be removed in future versions.

If the detection mechanism is to be used,
it is mandatory to correctly specify
the filename of the main file as the argument of |\childdocmain|:
%
\begin{center}
\begin{tabular}{l}
|\input{childdoc.def}|\\
|\childdocmain{|\textit{main}|}|\\
\end{tabular}
\end{center}
%
If |\jobname| does not match the argument \textit{main} of |\childdocmain|,
it is assumed that |\jobname| points to the child file to be compiled.
When using |\childdocmain| with the main file specified as argument,
it suffices to start a child file
with just |\input{|\textit{main}|}|
without loading of the package and using |\childdocof|.
If instead all processing is done
with the appropriate \textsf{childdoc} directives,
the argument of \textit{main} of |\childdocmain| can be empty.

An alternative version of the command line processing described
in \secref{sec:commandline} using the detection mechanism reads:
%
\begin{center}
|... -jobname "|\textit{target}|" "|[\textit{flags}]%
[|\def\jobname{|\textit{dest}|}|]|\input{|\textit{main}|}"|
\end{center}

%%%%%%%%%%%%%%%%%%%%%%%%%%%%%%%%%%%%%%%%%%%%%%%%%%%%%%%%%%%%%%%%%%%%%%%%%%%%%%%%
\subsection{Manual Code}
\label{sec:manual}

In case one cannot be certain whether the definitions file |childdoc.def|
is installed on the target \TeX{} distribution
and one prefers not to ship it,
it is conceivable to paste a few relevant commands into the sources.

To that end, drop all statements |\input{childdoc.def}|
and perform the replacements as outlined below.
Instead of |\childdocmain{|\textit{main}|}| add the following code
to the top of the main file:
%
\begin{center}
\begin{tabular}{l}
|\||ifdefined\childdocname\endinput\||fi\newif\ifchilddoc|\\
|\edef\childdocname{\scantokens\expandafter{\jobname\noexpand}}|\\
|\def\childdocmain{|\textit{main}|}\||ifx\childdocmain\childdocname\||else|\\
|\childdoctrue\includeonly{\childdocname}\let\jobname\childdocmain\||fi|\\
\end{tabular}
\end{center}
%
Instead of |\childdocof{|\textit{main}|}| just include the main file
at the top of each child file:
%
\begin{center}
|\input{|\textit{main}|}|
\end{center}
%
A simple redirection |\childdocforward{|\textit{dest}|}| is achieved by:
%
\begin{center}
|\def\jobname{|\textit{dest}|}\input{\jobname}|
\end{center}
%
The redirection with prefix
|\childdocforwardprefix[|\textit{prefix}|]{|\textit{dest}|}|
is accomplished by:
%
\begin{center}
\begin{tabular}{l}
|{\edef\jobname{\scantokens\expandafter{\jobname\noexpand}}|\\
|\def\redirectjob |\textit{prefix}|#1~~~{\gdef\jobname{|\textit{dest}|#1}}|\\
|\expandafter\redirectjob\jobname~~~}\input{\jobname}|
\end{tabular}
\end{center}

In an alternative approach,
child documents can be compiled by a specific command line
without additional code or specific definitions:
%
\begin{center}
|... -jobname "|\textit{target}|" "|[\textit{flags}]%
|\includeonly{|\textit{dest}|}\input{|\textit{main}|}"|
\end{center}
%

%%%%%%%%%%%%%%%%%%%%%%%%%%%%%%%%%%%%%%%%%%%%%%%%%%%%%%%%%%%%%%%%%%%%%%%%%%%%%%%%
%%%%%%%%%%%%%%%%%%%%%%%%%%%%%%%%%%%%%%%%%%%%%%%%%%%%%%%%%%%%%%%%%%%%%%%%%%%%%%%%
\section{Information}

%%%%%%%%%%%%%%%%%%%%%%%%%%%%%%%%%%%%%%%%%%%%%%%%%%%%%%%%%%%%%%%%%%%%%%%%%%%%%%%%
\subsection{Copyright}

Copyright \copyright{} 2017--2018 Niklas Beisert

This work may be distributed and/or modified under the
conditions of the \LaTeX{} Project Public License, either version 1.3
of this license or (at your option) any later version.
The latest version of this license is in
  \url{http://www.latex-project.org/lppl.txt}
and version 1.3 or later is part of all distributions of \LaTeX{}
version 2005/12/01 or later.

This work has the LPPL maintenance status `maintained'.

The Current Maintainer of this work is Niklas Beisert.

This work consists of the files |README.txt|, |childdoc.ins| and |childdoc.dtx|
as well as the derived files |childdoc.def|, |cdocsamp.tex|
with |cdocsch1.tex|, |cdocsch2.tex|, |cdocspt3.tex|, |cdocspt4.tex|,
|cdocsdrf.tex|, |cdocsfn1.tex|, |cdocsfn2.tex|
as well as |childdoc.pdf|.

%%%%%%%%%%%%%%%%%%%%%%%%%%%%%%%%%%%%%%%%%%%%%%%%%%%%%%%%%%%%%%%%%%%%%%%%%%%%%%%%
\subsection{Files and Installation}

The package consists of the files:
%
\begin{center}
\begin{tabular}{ll}
    |README.txt|   & readme file \\
    |childdoc.ins| & installation file \\
    |childdoc.dtx| & source file \\
    |childdoc.def| & definition file \\
    |cdocsamp.tex| & sample main file \\
    |cdocsch1.tex| & sample include file \\
    |cdocsch2.tex| & sample include file \\
    |cdocspt3.tex| & sample part file \\
    |cdocspt4.tex| & sample part file \\
    |cdocsdrf.tex| & sample redirection file \\
    |cdocsfn1.tex| & sample redirection file \\
    |cdocsfn2.tex| & sample redirection file \\
    |childdoc.pdf| & manual
\end{tabular}
\end{center}
%
The distribution consists of the files
|README.txt|, |childdoc.ins| and |childdoc.dtx|.
%
\begin{itemize}
\item
Run (pdf)\LaTeX{} on |childdoc.dtx|
to compile the manual |childdoc.pdf| (this file).
\item
Run \LaTeX{} on |childdoc.ins| to create the definitions file |childdoc.def|
and the sample |cdocsamp.tex| with include files
|cdocsch1.tex|, |cdocsch2.tex|, |cdocspt3.tex|, |cdocspt4.tex|,
|cdocsdrf.tex|, |cdocsfn1.tex|, |cdocsfn2.tex|.
Then copy the file |childdoc.def| to an appropriate directory of your \LaTeX{}
distribution, e.g.\ \textit{texmf-root}|/tex/latex/childdoc|.
\end{itemize}

%%%%%%%%%%%%%%%%%%%%%%%%%%%%%%%%%%%%%%%%%%%%%%%%%%%%%%%%%%%%%%%%%%%%%%%%%%%%%%%%
\subsection{Related CTAN Packages}

There are several other packages which offer a similar functionality:
%
\begin{itemize}
\item
The packages
\href{http://ctan.org/pkg/docmute}{\textsf{docmute}},
\href{http://ctan.org/pkg/includex}{\textsf{includex}} and
\href{http://ctan.org/pkg/standalone}{\textsf{standalone}}
provide commands to include only the document body of
a child file thus allowing both files to be compiled individually.
\item
The packages \href{http://ctan.org/pkg/subdocs}{\textsf{subdocs}}
and \href{http://ctan.org/pkg/subfiles}{\textsf{subfiles}}
provide structures in which the main and child documents can be
encapsulated and allowing them to be compiled individually.
The inclusion mechanism is different from the conventional |\include|.
\item
The package \href{http://ctan.org/pkg/combine}{\textsf{combine}}
is an elaborate solution to combine several documents into one.
\end{itemize}
%
See also the CTAN topic \href{http://ctan.org/topic/subdocs}{\textsf{subdocs}}
for further related packages.
The present package differs from the above solutions in that
a document structure constructed with the conventional |\include| mechanism
just needs two extra commands at the top of every file
such that all constituent files can be compiled individually.

%%%%%%%%%%%%%%%%%%%%%%%%%%%%%%%%%%%%%%%%%%%%%%%%%%%%%%%%%%%%%%%%%%%%%%%%%%%%%%%%
%\subsection{Feature Suggestions}
%
%The following is a list of features which may be useful for future
%versions of this package:
%%
%\begin{itemize}
%\item
%\ldots
%\end{itemize}

%%%%%%%%%%%%%%%%%%%%%%%%%%%%%%%%%%%%%%%%%%%%%%%%%%%%%%%%%%%%%%%%%%%%%%%%%%%%%%%%
\subsection{Revision History}

%%%%%%%%%%%%%%%%%%%%%%%%%%%%%%%%%%%%%%%%
\paragraph{v2.0:} 2018/12/30

\begin{itemize}
\item
immediate forward processing
\item
added |\childdocby| mechanism
\item
manual restructured
\end{itemize}

%%%%%%%%%%%%%%%%%%%%%%%%%%%%%%%%%%%%%%%%
\paragraph{v1.6:} 2018/01/17

\begin{itemize}
\item
application for development of include files
\item
corrections to manual
\end{itemize}

%%%%%%%%%%%%%%%%%%%%%%%%%%%%%%%%%%%%%%%%
\paragraph{v1.5:} 2017/05/21

\begin{itemize}
\item
more complete structuring introduced
\item
|\childdocof| introduced
\item
|\childdoc| renamed to |\childdocmain|
\item
|\childredirect| renamed to |\childdocforward| and |\childdocforwardprefix|
and functionality expanded
\end{itemize}

%%%%%%%%%%%%%%%%%%%%%%%%%%%%%%%%%%%%%%%%
\paragraph{v1.0:} 2017/04/27

\begin{itemize}
\item
manual and install package
\item
first version published on CTAN
\end{itemize}

%%%%%%%%%%%%%%%%%%%%%%%%%%%%%%%%%%%%%%%%
\paragraph{v0.6:} 2017/04/26

\begin{itemize}
\item
redirection mechanism added
\end{itemize}

%%%%%%%%%%%%%%%%%%%%%%%%%%%%%%%%%%%%%%%%
\paragraph{v0.5:} 2017/04/26

\begin{itemize}
\item
functionality in definition file
\end{itemize}


%%%%%%%%%%%%%%%%%%%%%%%%%%%%%%%%%%%%%%%%%%%%%%%%%%%%%%%%%%%%%%%%%%%%%%%%%%%%%%%%
%%%%%%%%%%%%%%%%%%%%%%%%%%%%%%%%%%%%%%%%%%%%%%%%%%%%%%%%%%%%%%%%%%%%%%%%%%%%%%%%
%%%%%%%%%%%%%%%%%%%%%%%%%%%%%%%%%%%%%%%%%%%%%%%%%%%%%%%%%%%%%%%%%%%%%%%%%%%%%%%%
\appendix

\settowidth\MacroIndent{\rmfamily\scriptsize 000\ }

 \DocInput{childdoc.dtx}

\end{document}
%</driver>
% \fi
%
% %%%%%%%%%%%%%%%%%%%%%%%%%%%%%%%%%%%%%%%%%%%%%%%%%%%%%%%%%%%%%%%%%%%%%%%%%%%%%%
% %%%%%%%%%%%%%%%%%%%%%%%%%%%%%%%%%%%%%%%%%%%%%%%%%%%%%%%%%%%%%%%%%%%%%%%%%%%%%%
% \section{Sample}
%\iffalse
%<*samplemain>
%\fi
%
% The following presents a sample document
% with two chapters, two parts, a title page,
% a compile flag as well as three forwarding files to set the flag.
% It consists of eight |.tex| files:
% \begin{center}
% \begin{tabular}{ll}
% |cdocsamp.tex|&main file\\
% |cdocsch1.tex|&include file for chapter 1\\
% |cdocsch2.tex|&include file for chapter 2\\
% |cdocspt3.tex|&include file for part 3\\
% |cdocspt4.tex|&include file for part 4\\
% |cdocsdrf.tex|&forwarding file for main file in draft mode\\
% |cdocsfi1.tex|&forwarding file for final version of chapter 1\\
% |cdocsfi2.tex|&forwarding file for final version of chapter 2\\
% \end{tabular}
% \end{center}
% Each of the eight files can be compiled directly by the \LaTeX{} compiler.
%
% %%%%%%%%%%%%%%%%%%%%%%%%%%%%%%%%%%%%%%
% \paragraph{Main File.}
%
% The main file is called |cdocsamp.tex|.
%
% Load the \textsf{childdoc} definitions and
% declare the filename for the main document:
%    \begin{macrocode}
\input{childdoc.def}
\childdocmain{}
%    \end{macrocode}

% Optional override for |\version| flag:
%    \begin{macrocode}
%%\ifchilddoc\else\providecommand{\version}{draft}\fi
%    \end{macrocode}

% Define the default values for the |\version| flag
% (|final| for the main file and |draft| for childs):
%    \begin{macrocode}
\ifchilddoc
\providecommand{\version}{draft}
\else
\providecommand{\version}{final}
\fi
%    \end{macrocode}

% Load the standard document class:
%    \begin{macrocode}
\documentclass[12pt]{article}
%    \end{macrocode}

% Start the document body:
%    \begin{macrocode}
\begin{document}
%    \end{macrocode}

% Declare a title page.
% Print title, part of document being processed and version flag:
%    \begin{macrocode}
\addtocounter{page}{-1}
\begin{center}
{\LARGE\bfseries{}childdoc example\par}
\vspace{1cm}
\ifchilddoc
\ifchilddocmanual part\else chapter\fi:
`\childdocname' of `\childdocjob'\par
\else
main document: `\childdocjob'\par
\fi
version: \version\par
\end{center}
\newpage
%    \end{macrocode}

% Manually include selected file,
% otherwise process as usual:
%    \begin{macrocode}
\ifchilddocmanual
\section*{part `\childdocname'}
\input{\childdocname}
\else
%    \end{macrocode}

% Include the two chapters:
%    \begin{macrocode}
\include{cdocsch1}
\include{cdocsch2}
%    \end{macrocode}

% Include the two parts unless only chapters should be displayed:
%    \begin{macrocode}
\ifchilddoc\else
\section{part three}
\input{cdocspt3}
\section{part four}
\input{cdocspt4}
\fi
%    \end{macrocode}

% Process as usual until here:
%    \begin{macrocode}
\fi
%    \end{macrocode}

% End of document body:
%    \begin{macrocode}
\end{document}
%    \end{macrocode}
%\iffalse
%</samplemain>
%\fi
%
% %%%%%%%%%%%%%%%%%%%%%%%%%%%%%%%%%%%%%%
% \paragraph{Chapter Include Files.}
%
% The include files are called |cdocsch1.tex| and |cdocsch2.tex|.
%
%\iffalse
%<*samplechap1|samplechap2>
%\fi

% Optional override for |\version| flag:
%    \begin{macrocode}
%%\providecommand{\version}{final}
%    \end{macrocode}

% Include the main document:
%    \begin{macrocode}
\input{childdoc.def}
\childdocof{cdocsamp}
%    \end{macrocode}

%\iffalse
%</samplechap1|samplechap2>
%\fi
%
%\iffalse
%<*samplechap1>
%\fi
% Some text for chapter 1:
%    \begin{macrocode}
\section{one}
some text in chapter one
%    \end{macrocode}

%\iffalse
%</samplechap1>
%\fi
% Some text for chapter 2:
%\iffalse
%<*samplechap2>
%\fi
%    \begin{macrocode}
\section{two}
more text in chapter two
%    \end{macrocode}

%\iffalse
%</samplechap2>
%\fi
%
% %%%%%%%%%%%%%%%%%%%%%%%%%%%%%%%%%%%%%%
% \paragraph{Part Include Files.}
%
% The include files are called |cdocspt3.tex| and |cdocspt4.tex|.
%
%\iffalse
%<*samplepart3|samplepart4>
%\fi

% Optional override for |\version| flag:
%    \begin{macrocode}
%%\providecommand{\version}{final}
%    \end{macrocode}

% Include the main document:
%    \begin{macrocode}
\input{childdoc.def}
\childdocby{cdocsamp}
%    \end{macrocode}

%\iffalse
%</samplepart3|samplepart4>
%\fi
%
%\iffalse
%<*samplepart3>
%\fi
% Some text for part 3:
%    \begin{macrocode}
some text in part three
%    \end{macrocode}

%\iffalse
%</samplepart3>
%\fi
% Some text for part 4:
%\iffalse
%<*samplepart4>
%\fi
%    \begin{macrocode}
more text in part four
%    \end{macrocode}

%\iffalse
%</samplepart4>
%\fi
%
% %%%%%%%%%%%%%%%%%%%%%%%%%%%%%%%%%%%%%%
% \paragraph{Forwarding for a Complete Draft.}
%
% The following forwarding file |cdocsdrf.tex|
% compiles the main document in draft mode:
%\iffalse
%<*sampledraft>
%\fi
%    \begin{macrocode}
\def\version{draft}
\input{childdoc.def}
\childdocforward{cdocsamp}
%    \end{macrocode}

%\iffalse
%</sampledraft>
%\fi
%
% %%%%%%%%%%%%%%%%%%%%%%%%%%%%%%%%%%%%%%
% \paragraph{Forwarding for Final Version of the Chapters.}
%
% The following forwarding files |cdocsfn1.tex| and |cdocsfn2.tex|
% (with identical content)
% compile the final versions of the child documents
% |cdocsch1.tex| and |cdocsch2.tex|, respectively:
%\iffalse
%<*samplefinal>
%\fi
%    \begin{macrocode}
\def\version{final}
\input{childdoc.def}
\childdocforwardprefix[cdocsamp]{cdocsfn}{cdocsch}
%    \end{macrocode}

%\iffalse
%</samplefinal>
%\fi
%
% %%%%%%%%%%%%%%%%%%%%%%%%%%%%%%%%%%%%%%
% \paragraph{Command Line Processing.}
%
% The following three command lines generate the output files
% |cdocscld|, |cdocscl1| and |cdocscl2|
% which should be identical to
% |cdocsdrf|, |cdocsch1| and |cdocsfn2|, respectively:
% \begin{center}
% \begin{tabular}{l}
% |latex -jobname cdocscld \|\\
% |  "\def\version{draft}\input{childdoc.def}\childdocforward{cdocsamp}"|\\
% |latex -jobname cdocscl1 \|\\
% |  "\input{childdoc.def}\childdocforward[cdocsamp]{cdocsch1}"|\\
% |latex -jobname cdocscl2 \|\\
% |  "\def\version{final}\input{childdoc.def}\childdocforward{cdocsch2}"|
% \end{tabular}
% \end{center}
% Note that the trailing backslash on each first line
% merely continues the input to the second line
% (for convenient cut ant paste).
% Furthermore, the command |latex| can be replaced by any
% of its alternative versions such as |pdflatex|.
%
% %%%%%%%%%%%%%%%%%%%%%%%%%%%%%%%%%%%%%%%%%%%%%%%%%%%%%%%%%%%%%%%%%%%%%%%%%%%%%%
% %%%%%%%%%%%%%%%%%%%%%%%%%%%%%%%%%%%%%%%%%%%%%%%%%%%%%%%%%%%%%%%%%%%%%%%%%%%%%%
% \section{Implementation}
%\iffalse
%<*package>
%\fi
%
% This section describes the definitions file |childdoc.def|.

% The definitions cannot be loaded using |\usepackage| or |\RequirePackage|
% which has a mechanism to prevent loading a style file more than once.
% When loading the definitions by means of |\input|
% multiple instances have to be prevented manually:
%\iffalse
%This code needs to be before the `\ProvidesFile' directive
%which is defined at the beginning of this file.
%Therefore it is also placed there and commented out here.
%</package>
%<*discard>
%\fi
%    \begin{macrocode}
\ifdefined\childdocmain\endinput\fi
%    \end{macrocode}
%\iffalse
%</discard>
%<*package>
%\fi
%
% \macro{\ifchilddoc}
% \macro{\ifchilddocmanual}
% The conditional |\ifchilddoc| tells whether a
% child (true) or main (false) document is being compiled.
% The conditional |\ifchilddocmanual| tells whether
% the |\includeonly| mechanism is used (false) or
% the selection of child files must be performed manually (true).
% The definitions initialise to false:
%    \begin{macrocode}
\newif\ifchilddoc
\newif\ifchilddocmanual
%    \end{macrocode}

% \macro{\childdocname}
% \macro{\childdocjob}
% The macro |\childdocname| stores the name of the main document
% to be compiled. The macro |\childdocjob| stores the name of
% the document on which the \LaTeX{} compiler was originally invoked.
% The content of |\jobname| cannot be compared
% to filenames specified in the source due to different catcodes.
% The following code rescans |\jobname|, stores the result
% in |\childdocname| and saves a copy in |\childdocjob|:
%    \begin{macrocode}
\edef\childdocname{\scantokens\expandafter{\jobname\noexpand}}
\let\childdocjob\childdocname
%    \end{macrocode}

% \macro{\childdocdisable}
% The macro |\childdocdisable| prevents the main file
% from being processed more than once.
% At this stage, the main document command |\childdocmain|
% is assumed to be called once again where it should do nothing.
% Any subsequent call to it should prevent
% a secondary processing of the main document
% It overwrites the forwarding commands
% |\childdocof| and |\childdocforward|
% with empty macros to prevent further inclusions of the main document:
%    \begin{macrocode}
\newcommand{\childdocdisable}
{
  \renewcommand{\childdocmain}[1]{\renewcommand{\childdocmain}[1]{\endinput}}
  \renewcommand{\childdocof}[1]{}
  \renewcommand{\childdocby}[2][]{}
  \renewcommand{\childdocforward}[2][]{}
  \renewcommand{\childdocdisable}{}
}
%    \end{macrocode}

% \macro{\childdocmain}
% The macro |\childdocmain| is to be called at the top of the main file
% with nothing or the main filename (without extension) as argument.
% First, it breaks loops.
% If the argument is not empty and does not match |\childdocname|
% (which is set by the first inclusion of |childdoc.def|),
% |\ifchilddoc| is set to true, |\includeonly| is applied to the child file
% and |\jobname| is set to the main file
% (for proper handling of |.aux| files):
%    \begin{macrocode}
\newcommand{\childdocmain}[1]
{
  \childdocdisable\childdocmain{}
  \if?#1?\else
    \begingroup
      \def\childdoctmp{#1}
      \ifx\childdoctmp\childdocname
        \def\childdoctmp{}
      \else
        \def\childdoctmp
        {
          \childdoctrue
          \includeonly{\childdocname}
          \def\childdocjob{#1}
          \def\jobname{#1}
        }
      \fi
      \expandafter
    \endgroup
    \childdoctmp
  \fi
}
%    \end{macrocode}

% \macro{\childdocof}
% The command |\childdocof| redirects
% compilation to the main file |#1|.
%    \begin{macrocode}
\newcommand{\childdocof}[1]
{
  \childdocdisable
  \childdoctrue
  \includeonly{\childdocname}
  \def\jobname{#1}
  \def\childdocjob{#1}
  \input{#1}
}
%    \end{macrocode}

% \macro{\childdocby}
% The command |\childdocby| ....
%    \begin{macrocode}
\newcommand{\childdocby}[2][]
{
  \childdocdisable
  \childdoctrue
  \childdocmanualtrue
  \if?#1?\else
    \def\jobname{#2}
  \fi
  \def\childdocjob{#2}
  \input{#2}
  \endinput
}
%    \end{macrocode}

% \macro{\childdocforward}
% The command |\childdocforward| redirects
% compilation to the main file or
% (if the optional argument is given) a child file.
% Parameters are set as if the main file
% or a child file starting with |\childdocof| was compiled.
% Then compilation is handed over to the main file:
%    \begin{macrocode}
\newcommand{\childdocforward}[2][]
{
  \begingroup
    \if?#1?
      \def\childdoctmp
      {
        \def\childdocname{#2}
        \def\childdocjob{#2}
        \def\jobname{#2}
        \input{#2}
        \endinput
      }
    \else
      \def\childdoctmp
      {
        \childdocdisable
        \def\childdocname{#2}
        \childdoctrue
        \includeonly{#2}
        \def\childdocjob{#1}
        \def\jobname{#1}
        \input{#1}
        \endinput
      }
    \fi
    \expandafter
  \endgroup
  \childdoctmp
}
%    \end{macrocode}

% \macro{\childdocforwardprefix}
% The command |\childdocforwardprefix| redirects
% compilation to the main or a child file by means of a pattern.
% The prefix |#1| in the current filename is replaced by |#2|
% and the suffix of the current filename is kept
% (it is assumed that the filename does not contain the substring `|~~~|'
% which is used as a delimiter).
% Compilation is handed over to the new file by |\childdocforward|:
%    \begin{macrocode}
\newcommand{\childdocforwardprefix}[3][]
{
  \begingroup
    \def\childdocextract #2##1~~~{\def\childdoctmp{\childdocforward[#1]{#3##1}}}
    \expandafter\childdocextract\childdocname~~~
    \expandafter
  \endgroup
  \childdoctmp
}
%    \end{macrocode}

% \macro{\childdoc}
% The deprecated macro |\childdoc| is a legacy version of |\childdocmain|:
%    \begin{macrocode}
\newcommand{\childdoc}{\childdocmain}
%    \end{macrocode}

% \macro{\childdocredirect}
% The deprecated macro |\childdocredirect| is a legacy version
% of |\childdocforward| and |\childdocforwardprefix|:
%    \begin{macrocode}
\newcommand{\childdocredirect}[2][]
{
  \begingroup
    \if?#1?
      \def\childdoctmp{\childdocforward{#2}}
    \else
      \def\childdoctmp{\childdocforwardprefix{#1}{#2}}
    \fi
    \expandafter
  \endgroup
  \childdoctmp
}
%    \end{macrocode}

%\iffalse
%</package>
%\fi
%
\endinput

\childdocby{cdocsamp}
%    \end{macrocode}

%\iffalse
%</samplepart3|samplepart4>
%\fi
%
%\iffalse
%<*samplepart3>
%\fi
% Some text for part 3:
%    \begin{macrocode}
some text in part three
%    \end{macrocode}

%\iffalse
%</samplepart3>
%\fi
% Some text for part 4:
%\iffalse
%<*samplepart4>
%\fi
%    \begin{macrocode}
more text in part four
%    \end{macrocode}

%\iffalse
%</samplepart4>
%\fi
%
% %%%%%%%%%%%%%%%%%%%%%%%%%%%%%%%%%%%%%%
% \paragraph{Forwarding for a Complete Draft.}
%
% The following forwarding file |cdocsdrf.tex|
% compiles the main document in draft mode:
%\iffalse
%<*sampledraft>
%\fi
%    \begin{macrocode}
\def\version{draft}
% \iffalse
%
% childdoc.dtx Copyright (C) 2017-2018 Niklas Beisert
%
% This work may be distributed and/or modified under the
% conditions of the LaTeX Project Public License, either version 1.3
% of this license or (at your option) any later version.
% The latest version of this license is in
%   http://www.latex-project.org/lppl.txt
% and version 1.3 or later is part of all distributions of LaTeX
% version 2005/12/01 or later.
%
% This work has the LPPL maintenance status `maintained'.
%
% The Current Maintainer of this work is Niklas Beisert.
%
% This work consists of the files childdoc.dtx and childdoc.ins
% and the derived files childdoc.def and cdocsamp.tex with
% cdocsch1.tex, cdocsch2.tex, cdocsdrf.tex, cdocsfn1.tex, cdocsfn2.tex.
%
%<package>\ifdefined\childdocmain\endinput\fi
%<package>\ProvidesFile{childdoc.def}[2018/12/30 v2.0 child document driver]
%<samplemain>\ProvidesFile{cdocsamp.tex}[2018/12/30 v2.0 sample for childdoc]
%<*driver>
%\ProvidesFile{childdoc.drv}[2018/12/30 v2.0 childdoc reference manual file]
\PassOptionsToClass{10pt,a4paper}{article}
\documentclass{ltxdoc}

\usepackage[margin=35mm]{geometry}
\usepackage{hyperref}
\usepackage{hyperxmp}
\usepackage[usenames]{color}

\hypersetup{colorlinks=true}
\hypersetup{pdfstartview=FitH}
\hypersetup{pdfpagemode=UseNone}
\hypersetup{pdfsource={}}
\hypersetup{pdflang={en-UK}}
\hypersetup{pdfcopyright={Copyright 2017-2018 Niklas Beisert.
  This work may be distributed and/or modified under the
  conditions of the LaTeX Project Public License, either version 1.3
  of this license or (at your option) any later version.}}
\hypersetup{pdflicenseurl={http://www.latex-project.org/lppl.txt}}
\hypersetup{pdfcontactaddress={ETH Zurich, ITP, HIT K,
  Wolfgang-Pauli-Strasse 27}}
\hypersetup{pdfcontactpostcode={8093}}
\hypersetup{pdfcontactcity={Zurich}}
\hypersetup{pdfcontactcountry={Switzerland}}
\hypersetup{pdfcontactemail={nbeisert@itp.phys.ethz.ch}}
\hypersetup{pdfcontacturl={http://people.phys.ethz.ch/\xmptilde nbeisert/}}

\newcommand{\secref}[1]{\hyperref[#1]{section \ref*{#1}}}

\parskip1ex
\parindent0pt
\let\olditemize\itemize
\def\itemize{\olditemize\parskip0pt}

\begin{document}

\title{The \textsf{childdoc} Package}
\hypersetup{pdftitle={The childdoc Package}}
\author{Niklas Beisert\\[2ex]
  Institut f\"ur Theoretische Physik\\
  Eidgen\"ossische Technische Hochschule Z\"urich\\
  Wolfgang-Pauli-Strasse 27, 8093 Z\"urich, Switzerland\\[1ex]
  \href{mailto:nbeisert@itp.phys.ethz.ch}
  {\texttt{nbeisert@itp.phys.ethz.ch}}}
\hypersetup{pdfauthor={Niklas Beisert}}
\hypersetup{pdfsubject={Manual for the LaTeX2e Package childdoc}}
\date{30 December 2018, \textsf{v2.0}}
\maketitle

\begin{abstract}\noindent
\textsf{childdoc} is a \LaTeXe{} package
that enables the direct compilation
of document sections included by |\include|
to individual files.
\end{abstract}

\begingroup
\parskip0ex
\tableofcontents
\endgroup

%%%%%%%%%%%%%%%%%%%%%%%%%%%%%%%%%%%%%%%%%%%%%%%%%%%%%%%%%%%%%%%%%%%%%%%%%%%%%%%%
%%%%%%%%%%%%%%%%%%%%%%%%%%%%%%%%%%%%%%%%%%%%%%%%%%%%%%%%%%%%%%%%%%%%%%%%%%%%%%%%
\section{Introduction}

\LaTeX{} provides a mechanism to structure a large document (such as a book)
into a main file and several child files (containing the chapters)
using the |\include| command.
This mechanism is beneficial for documents
which span hundreds of pages in order to
make the source file(s) more manageable.
Moreover, compilation can be restricted to
selected child files by means of the |\includeonly| command.
The latter feature can be used to reduce the compilation time while editing
(this was significantly more useful in the earlier days of \LaTeX{})
or to generate a smaller document which is easier to navigate.
Another application of |\includeonly| is to generate
documents consisting of selected parts of the complete document.

However, there are a few drawbacks of the plain |\include| mechanism:
\begin{itemize}
\item
The child files cannot be compiled on their own,
they can only be compiled via the main file.
A naive editing environment
(such as a text editor with an option
to have the current file processed by \LaTeX)
may require one to switch to the main file before compiling;
attempting to compile the child file produces errors.
\item
The main file must be modified (each time)
to adjust the |\includeonly| command
to the present needs. This easily leaves the main file in a messy state.
\item
The generated document will always carry the filename
of the main document. This is inconvenient if
several child files are to be compiled and
to be kept for distribution.
\end{itemize}

The present package provides a simple interface
to make child files individually compilable by \LaTeX{}.
Compiling a child file then has the same effect as compiling
the main file with an |\includeonly| command
to select the appropriate child.
Moreover the generated document will carry the name of the child
rather than the main file.
This resolves all three above issues.

This feature is meant to make the editing of books,
thesis documents and lecture notes somewhat more convenient.
However, the package can also be used efficiently for
composing a series of documents (such as exercise sheets)
which are typically distributed individually.
It then assists the author in generating the individual documents
(potentially in different versions)
as well as a document containing the collected series.
Another application is in developing style files
or other kinds of included material
where compilation of the style file could redirect
to a sample or test file.

%%%%%%%%%%%%%%%%%%%%%%%%%%%%%%%%%%%%%%%%%%%%%%%%%%%%%%%%%%%%%%%%%%%%%%%%%%%%%%%%
%%%%%%%%%%%%%%%%%%%%%%%%%%%%%%%%%%%%%%%%%%%%%%%%%%%%%%%%%%%%%%%%%%%%%%%%%%%%%%%%
\section{Usage}

First of all, the package \textsf{childdoc} is \emph{not} a standard
\LaTeXe{} |.sty| style file! Therefore it needs to be invoked in
a non-standard way.

%%%%%%%%%%%%%%%%%%%%%%%%%%%%%%%%%%%%%%%%%%%%%%%%%%%%%%%%%%%%%%%%%%%%%%%%%%%%%%%%
\subsection{Included Files}
\label{sec:include}

%%%%%%%%%%%%%%%%%%%%%%%%%%%%%%%%%%%%%%%%
\DescribeMacro{\childdocmain}
To use the package, add the commands
\begin{center}
\begin{tabular}{l}
|\input{childdoc.def}|\\
|\childdocmain{}|\\
\end{tabular}
\end{center}
at the very top of the main \LaTeX{} file,
in particular \emph{before} the |\documentclass| statement!
The argument of |\childdocmain| should be left empty
(but it must be present).

%%%%%%%%%%%%%%%%%%%%%%%%%%%%%%%%%%%%%%%%
\DescribeMacro{\childdocof}
Furthermore, add the commands
\begin{center}
\begin{tabular}{l}
|\input{childdoc.def}|\\
|\childdocof{|\textit{main}|}|\\
\end{tabular}
\end{center}
at the top of every child file \textit{child}
which is included by |\include{|\textit{child}|}|
from within the main file
(or at least for those files to be compiled individually).
The argument \textit{main} must be the filename of the main file.

There are a couple of
considerations in setting up the main and child documents:

%%%%%%%%%%%%%%%%%%%%%%%%%%%%%%%%%%%%%%%%
\paragraph{Restrictions.}

Please note the following restrictions:
\begin{itemize}
\item
|\childdocmain| must be called with one argument \textit{main}
to ensure compatibility with earlier version of the package.
It must either be empty (|\childdocmain{}|)
or precisely match the filename of the main file in which it is specified.
See \secref{sec:detection} for further information.
\item
The filename \textit{main} must be specified without the |.tex| extension.
\item
The filename \textit{main} is case sensitive
(even in case-insensitive file systems)
due to internal string comparison.
\item
The argument \textit{main} should be fully expanded, it cannot be a macro.
\item
Subdirectories and special characters should be avoided in filenames.
\item
The command |\childdocmain{|\textit{main}|}| must be followed by a whitespace.
It should not be followed immediately by another command
or by a comment mark `|%|'.
This is because the \TeX{} parser reads the token immediately following
the argument of |\childdocmain| and puts it
at the beginning of every child section;
however, a white\-space is ignored.
\end{itemize}

%%%%%%%%%%%%%%%%%%%%%%%%%%%%%%%%%%%%%%%%
\paragraph{Content of Main File.}

It is advisable to place all content in the child files included by |\include|.
Any output contained in the main file will appear in all child documents
unless suppressed manually;
it cannot be suppressed automatically by the |\includeonly| directive
and thus should normally be avoided.
A method to include some content in the main file
by means of conditional processing is described in \secref{sec:conditional}.

%%%%%%%%%%%%%%%%%%%%%%%%%%%%%%%%%%%%%%%%
\paragraph{Page Numbering.}

When only a part of the document is compiled,
the appropriate numbering of pages
(as well as other status parameters)
is determined from the |.aux| files.
The latter contain information from previous passes.
However this information needs to propagate through
all intermediate child documents.
Therefore the page numbering in child documents may well
be inconsistent until the complete document is compiled at least once.

A useful (if unconventional) way to always ensure a consistent
page numbering is to restart the numbering in each child document
and denote the pages by `\textit{child}|.|\textit{page}'
where \textit{child} represents the chapter/section number of the child file.
This can be achieved by the command
|\numberwithin{page}{|\textit{child}|}|
of the \textsf{amsmath} package
where \textit{child} can be |chapter| or |section|
depending on the chosen structuring.
Alternatively, one can modify the macro |\thepage| appropriately
and reset the counter |page| at the start of each child file.

%%%%%%%%%%%%%%%%%%%%%%%%%%%%%%%%%%%%%%%%%%%%%%%%%%%%%%%%%%%%%%%%%%%%%%%%%%%%%%%%
\subsection{Conditional Processing}
\label{sec:conditional}

The package provides a mechanism to compile different versions
of a document. To customise the versions further some conditional processing
can come in handy to distinguish which version is being compiled.
The package provides two macros to describe the compilation context:

%%%%%%%%%%%%%%%%%%%%%%%%%%%%%%%%%%%%%%%%
\DescribeMacro{\ifchilddoc}
The conditional |\ifchilddoc| distinguishes between the compilation of
child documents and the main document:
%
\begin{center}
|\ifchilddoc |\textit{child-code}| |[|\||else |\textit{main-code}]| \||fi|
\end{center}

%%%%%%%%%%%%%%%%%%%%%%%%%%%%%%%%%%%%%%%%
\DescribeMacro{\childdocname}
\DescribeMacro{\childdocjob}
The macro |\childdocname| contains the filename (without extension)
of the main or child file being processed.
Note that |\childdocjob| will always contain the name of the main file.

%%%%%%%%%%%%%%%%%%%%%%%%%%%%%%%%%%%%%%%%
\paragraph{Title Page.}

Conditional processing can be used to include a title or banner page
in the main document when proper precautions are taken.
Importantly, the code in the main file should ensure that the page counter
(as well as other status parameters which are stored in the |.aux| files)
takes the same value after the conditional processing.
Otherwise the page numbers may take divergent values
depending on which part is compiled.

For example, a title page could be declared by:
%
\begin{center}
\begin{tabular}{l}
|\ifchilddoc\||else|\\
|\addtocounter{page}{-1}|\\
\textit{code for title page}\\
|\newpage|\\
|\||fi|
\end{tabular}
\end{center}
%
A banner page for the child documents can be generated by:
%
\begin{center}
\begin{tabular}{l}
|\ifchilddoc|\\
|\addtocounter{page}{-1}|\\
\textit{code for banner page}\\
|\newpage|\\
|\||fi|
\end{tabular}
\end{center}
%
Here one could write a message such as:
\begin{center}
|This is the part \childdocname{} of \childdocjob{}.|
\end{center}

%%%%%%%%%%%%%%%%%%%%%%%%%%%%%%%%%%%%%%%%%%%%%%%%%%%%%%%%%%%%%%%%%%%%%%%%%%%%%%%%
\subsection{Flags}
\label{sec:flags}

The package makes it easy to generate different versions
of the main or child documents.
To this end compilation flags can be defined
and assigned different default values.
They will be particularly useful in conjunction
with the forwarding mechanism described in \secref{sec:forward}.

For example, it may be useful to have a flag |\version|
which can be set to |draft| or |final|.
The document source will contain some conditional code
depending on the value of |\version|.
Suppose further, the flag should default to |final| for the main file
and to |draft| for child files
which is a natural assignment for editing the document.
This is achieved by placing the following code
in the preamble of the main document
(below the |\childdocmain| directive):
%
\begin{center}
\begin{tabular}{l}
|\ifchilddoc|\\
|\providecommand{\version}{draft}|\\
|\||else|\\
|\providecommand{\version}{final}|\\
|\||fi|
\end{tabular}
\end{center}
%
The definition by |\providecommand| makes sure
that previous definitions are not overwritten.
Further statements |\providecommand{\version}{...}|
can thus be added before the above code to override it.

For the main file, one might add a line
(between |\childdocmain| and the above block)
%
\begin{center}
|%\ifchilddoc\||else\providecommand{\version}{draft}\||fi|
\end{center}
%
which can be uncommented to produce a draft version.
Likewise one can add a line to the very top of a child file
(above the |\childdocof{|\textit{main}|}| directive)
%
\begin{center}
|%\providecommand{\version}{final}|
\end{center}
%
which can be uncommented to produce the final version of this child document.

%%%%%%%%%%%%%%%%%%%%%%%%%%%%%%%%%%%%%%%%%%%%%%%%%%%%%%%%%%%%%%%%%%%%%%%%%%%%%%%%
\subsection{Forwarding}
\label{sec:forward}

Different versions of the main or child documents
using compilation flags as described in \secref{sec:flags}
can be (permanently) stored in different files
for convenient compilation, viewing and distribution.
To this end, the package defines a command
to pass on compilation to a different file:

%%%%%%%%%%%%%%%%%%%%%%%%%%%%%%%%%%%%%%%%
\DescribeMacro{\childdocforward}
The command |\childdocforward| redirects processing to
another source file:
%
\begin{center}
\begin{tabular}{l}
|\input{childdoc.def}|\\
|\childdocforward[|\textit{main}|]{|\textit{dest}|}|\\
\end{tabular}
\end{center}
%
The argument \textit{dest} is the destination file
(without extension).
It should be the main file or one of the child files.
Note that further \textsf{childdoc} directives
such as |\childdocof| and |\childdocforward|
in the indicated file will be processed in this form.
The optional argument \textit{main}
passes on directly to the main file \textit{main}
while pretending to compile the child \textit{dest}.
This form behaves as if \textit{dest}
issues |\childdocof{|\textit{main}|}| right away,
and no further \textsf{childdoc} directives will be processed.

%%%%%%%%%%%%%%%%%%%%%%%%%%%%%%%%%%%%%%%%
\DescribeMacro{\...prefix}
In the alternative form |\childdocforwardprefix|,
%
\begin{center}
\begin{tabular}{l}
|\input{childdoc.def}|\\
|\childdocforwardprefix[|\textit{main}|]{|\textit{prefix}|}{|\textit{dest}|}|
\end{tabular}
\end{center}
%
the destination file is determined by a pattern
depending on the current file:
To make this work, the current file must be called
`{\textit{prefix}\hspace{0.2em}\textit{suffix}}'
with \textit{prefix} matching precisely the argument.
Processing is then passed on to the file
`{\textit{dest}\hspace{0.2em}\textit{suffix}}'.
Surely, the same effect is achieved by
directly specifying the
argument `{\textit{dest}\hspace{0.2em}\textit{suffix}}'
in the first form.
However, that requires to set up a different file
for each child. With the alternative form of the command
all these files can have exactly the same content
which simplifies setting them up and maintaining them.

For example, the following file |draft.tex|
with a compilation flag |\version| as described in \secref{sec:flags}
compiles the main document as a draft:
%
\begin{center}
\begin{tabular}{l}
|\def\version{draft}|\\
|\input{childdoc.def}|\\
|\childdocforward{|\textit{main}|}|
\end{tabular}
\end{center}
%
Likewise, the following files |final|\textit{nn}|.tex|
compile the final version of the child document
|child|\textit{nn}|.tex|:
%
\begin{center}
\begin{tabular}{l}
|\def\version{final}|\\
|\input{childdoc.def}|\\
|\childdocforwardprefix{final}{child}|
\end{tabular}
\end{center}
%

Note that when several versions of a main file and/or of each child file
are to be generated, it may be convenient to set up a |Makefile| or
shell script to automatise the process.

%%%%%%%%%%%%%%%%%%%%%%%%%%%%%%%%%%%%%%%%%%%%%%%%%%%%%%%%%%%%%%%%%%%%%%%%%%%%%%%%
\subsection{Command Line Processing}
\label{sec:commandline}

The effect of redirection files can also be achieved by invoking
the \LaTeX{} compiler with a more elaborate command line.
Most conveniently this should be done as part
of a shell script or a |Makefile|.

When using \textsf{childdoc} in the main file, the following
command lines effectively perform a redirection
(note that depending on the shell being used,
backslashes may have to be doubled: `|\|' $\to$ `|\\|'):
%
\begin{center}
|... -jobname "|\textit{target}|" |\\|"|[\textit{flags}]%
|\input{childdoc.def}\childdocforward[|\textit{main}|]{|\textit{dest}|}"|
\end{center}
%
Here \textit{target} is the name of the output file,
\textit{main} is the name of the main file
and \textit{dest} is the name of the main or child file to be processed
(all filenames without extensions).
The optional argument \textit{main} can be omitted
if \textit{main} matches \textit{dest}.
Optionally, compilation \textit{flags} can be defined via |\def| commands.
This command line makes the \TeX{} engine believe
it is compiling the file \textit{target}
whose content is specified as the latter parameter.
The provided code then forwards the processing to
\textit{main} or \textit{dest} as described in \secref{sec:forward}.

%%%%%%%%%%%%%%%%%%%%%%%%%%%%%%%%%%%%%%%%%%%%%%%%%%%%%%%%%%%%%%%%%%%%%%%%%%%%%%%%
\subsection{Include by Input}
\label{sec:input}

Including child documents by |\include| has some restrictions by design.
Most notably, the content of a child document always occupies
its own set of pages; pages cannot be shared between child documents.
Usually, this behaviour makes perfect sense
because each child document contain an essential part of the document.
However, in some situations it may be desirable to compose
a document from a collection of parts
without having mandatory page breaks between then.
For this case, the package
provides a mechanism to include parts
by |\input| which can also be processed individually.
However, by construction this mechanism
requires manual handling of the content to be output.

%%%%%%%%%%%%%%%%%%%%%%%%%%%%%%%%%%%%%%%%
\DescribeMacro{\ifchilddocmanual}
The main file should be prepared as usual, see \secref{sec:include}.
However, the document body must make a distinction
between processing of an individual part and of the main document, e.g.:
%
\begin{center}
\begin{tabular}{l}
|\ifchilddocmanual|\\
|\input{\childdocname}|\\
|\||else|\\
\textit{document body with }|\input{|\textit{part}|}|\\
|\||fi|
\end{tabular}
\end{center}
%
The conditional |\ifchilddocmanual| is true whenever
a part to be included by |\input| is being compiled,
and the name of the part is stored in |\childdocname|.

%%%%%%%%%%%%%%%%%%%%%%%%%%%%%%%%%%%%%%%%
\DescribeMacro{\childdocby}
Each part to be included by |\input| should start with:
%
\begin{center}
\begin{tabular}{l}
|\input{childdoc.def}|\\
|\childdocby{|\textit{main}|}|\\
\end{tabular}
\end{center}
%
The directive |\childdocby| is similar to |\childdocof|
described in \secref{sec:include},
but the subsequent selection of content must be done manually.
To that end, both |\ifchilddoc| and |\ifchilddocmanual|
will be true upon processing of a part,
and the name of the part is stored in |\childdocname|.
Note that |\jobname| will be set to the filename of the current part
so that each part receives an individual |.aux| file
that does not interfere with the |.aux| file(s) of the main document.
This behaviour can be altered by the alternative form
|\childdocby[*]{|\textit{main}|}| (with a non-empty optional argument)
which uses the |.aux| file of the main document
by setting |\jobname| to \textit{main}.

%%%%%%%%%%%%%%%%%%%%%%%%%%%%%%%%%%%%%%%%%%%%%%%%%%%%%%%%%%%%%%%%%%%%%%%%%%%%%%%%
\subsection{Driver Development}
\label{sec:driver}

The \textsf{childdoc} mechanism can also be use for the development
of definition files such as \LaTeX{} styles or classes.
This case differs from the above setup with multiple parts
included by |\include| in that no |\includeonly| should be invoked.
This can be achieved by starting the include file
(before |\ProvidesPackage|) with:
%
\begin{center}
\begin{tabular}{l}
|\input{childdoc.def}|\\
|\childdocforward{|\textit{main}|}|\\
\end{tabular}
\end{center}
%
or alternatively with:
%
\begin{center}
\begin{tabular}{l}
|\input{childdoc.def}|\\
|\childdocby{|\textit{main}|}|\\
\end{tabular}
\end{center}
%
Both forms have slightly different effects as described above.
The main file is prepared as usual, see \secref{sec:include}.

%%%%%%%%%%%%%%%%%%%%%%%%%%%%%%%%%%%%%%%%%%%%%%%%%%%%%%%%%%%%%%%%%%%%%%%%%%%%%%%%
\subsection{Legacy Detection}
\label{sec:detection}

The directive |\childdocmain| in the main file can detect
whether the complete document or merely a child is to be compiled
even without using the directive |\childdocof|.
This method is deprecated because it is less robust
and there is no compelling reason to use it;
it is merely provided for backward compatibility
and it may be removed in future versions.

If the detection mechanism is to be used,
it is mandatory to correctly specify
the filename of the main file as the argument of |\childdocmain|:
%
\begin{center}
\begin{tabular}{l}
|\input{childdoc.def}|\\
|\childdocmain{|\textit{main}|}|\\
\end{tabular}
\end{center}
%
If |\jobname| does not match the argument \textit{main} of |\childdocmain|,
it is assumed that |\jobname| points to the child file to be compiled.
When using |\childdocmain| with the main file specified as argument,
it suffices to start a child file
with just |\input{|\textit{main}|}|
without loading of the package and using |\childdocof|.
If instead all processing is done
with the appropriate \textsf{childdoc} directives,
the argument of \textit{main} of |\childdocmain| can be empty.

An alternative version of the command line processing described
in \secref{sec:commandline} using the detection mechanism reads:
%
\begin{center}
|... -jobname "|\textit{target}|" "|[\textit{flags}]%
[|\def\jobname{|\textit{dest}|}|]|\input{|\textit{main}|}"|
\end{center}

%%%%%%%%%%%%%%%%%%%%%%%%%%%%%%%%%%%%%%%%%%%%%%%%%%%%%%%%%%%%%%%%%%%%%%%%%%%%%%%%
\subsection{Manual Code}
\label{sec:manual}

In case one cannot be certain whether the definitions file |childdoc.def|
is installed on the target \TeX{} distribution
and one prefers not to ship it,
it is conceivable to paste a few relevant commands into the sources.

To that end, drop all statements |\input{childdoc.def}|
and perform the replacements as outlined below.
Instead of |\childdocmain{|\textit{main}|}| add the following code
to the top of the main file:
%
\begin{center}
\begin{tabular}{l}
|\||ifdefined\childdocname\endinput\||fi\newif\ifchilddoc|\\
|\edef\childdocname{\scantokens\expandafter{\jobname\noexpand}}|\\
|\def\childdocmain{|\textit{main}|}\||ifx\childdocmain\childdocname\||else|\\
|\childdoctrue\includeonly{\childdocname}\let\jobname\childdocmain\||fi|\\
\end{tabular}
\end{center}
%
Instead of |\childdocof{|\textit{main}|}| just include the main file
at the top of each child file:
%
\begin{center}
|\input{|\textit{main}|}|
\end{center}
%
A simple redirection |\childdocforward{|\textit{dest}|}| is achieved by:
%
\begin{center}
|\def\jobname{|\textit{dest}|}\input{\jobname}|
\end{center}
%
The redirection with prefix
|\childdocforwardprefix[|\textit{prefix}|]{|\textit{dest}|}|
is accomplished by:
%
\begin{center}
\begin{tabular}{l}
|{\edef\jobname{\scantokens\expandafter{\jobname\noexpand}}|\\
|\def\redirectjob |\textit{prefix}|#1~~~{\gdef\jobname{|\textit{dest}|#1}}|\\
|\expandafter\redirectjob\jobname~~~}\input{\jobname}|
\end{tabular}
\end{center}

In an alternative approach,
child documents can be compiled by a specific command line
without additional code or specific definitions:
%
\begin{center}
|... -jobname "|\textit{target}|" "|[\textit{flags}]%
|\includeonly{|\textit{dest}|}\input{|\textit{main}|}"|
\end{center}
%

%%%%%%%%%%%%%%%%%%%%%%%%%%%%%%%%%%%%%%%%%%%%%%%%%%%%%%%%%%%%%%%%%%%%%%%%%%%%%%%%
%%%%%%%%%%%%%%%%%%%%%%%%%%%%%%%%%%%%%%%%%%%%%%%%%%%%%%%%%%%%%%%%%%%%%%%%%%%%%%%%
\section{Information}

%%%%%%%%%%%%%%%%%%%%%%%%%%%%%%%%%%%%%%%%%%%%%%%%%%%%%%%%%%%%%%%%%%%%%%%%%%%%%%%%
\subsection{Copyright}

Copyright \copyright{} 2017--2018 Niklas Beisert

This work may be distributed and/or modified under the
conditions of the \LaTeX{} Project Public License, either version 1.3
of this license or (at your option) any later version.
The latest version of this license is in
  \url{http://www.latex-project.org/lppl.txt}
and version 1.3 or later is part of all distributions of \LaTeX{}
version 2005/12/01 or later.

This work has the LPPL maintenance status `maintained'.

The Current Maintainer of this work is Niklas Beisert.

This work consists of the files |README.txt|, |childdoc.ins| and |childdoc.dtx|
as well as the derived files |childdoc.def|, |cdocsamp.tex|
with |cdocsch1.tex|, |cdocsch2.tex|, |cdocspt3.tex|, |cdocspt4.tex|,
|cdocsdrf.tex|, |cdocsfn1.tex|, |cdocsfn2.tex|
as well as |childdoc.pdf|.

%%%%%%%%%%%%%%%%%%%%%%%%%%%%%%%%%%%%%%%%%%%%%%%%%%%%%%%%%%%%%%%%%%%%%%%%%%%%%%%%
\subsection{Files and Installation}

The package consists of the files:
%
\begin{center}
\begin{tabular}{ll}
    |README.txt|   & readme file \\
    |childdoc.ins| & installation file \\
    |childdoc.dtx| & source file \\
    |childdoc.def| & definition file \\
    |cdocsamp.tex| & sample main file \\
    |cdocsch1.tex| & sample include file \\
    |cdocsch2.tex| & sample include file \\
    |cdocspt3.tex| & sample part file \\
    |cdocspt4.tex| & sample part file \\
    |cdocsdrf.tex| & sample redirection file \\
    |cdocsfn1.tex| & sample redirection file \\
    |cdocsfn2.tex| & sample redirection file \\
    |childdoc.pdf| & manual
\end{tabular}
\end{center}
%
The distribution consists of the files
|README.txt|, |childdoc.ins| and |childdoc.dtx|.
%
\begin{itemize}
\item
Run (pdf)\LaTeX{} on |childdoc.dtx|
to compile the manual |childdoc.pdf| (this file).
\item
Run \LaTeX{} on |childdoc.ins| to create the definitions file |childdoc.def|
and the sample |cdocsamp.tex| with include files
|cdocsch1.tex|, |cdocsch2.tex|, |cdocspt3.tex|, |cdocspt4.tex|,
|cdocsdrf.tex|, |cdocsfn1.tex|, |cdocsfn2.tex|.
Then copy the file |childdoc.def| to an appropriate directory of your \LaTeX{}
distribution, e.g.\ \textit{texmf-root}|/tex/latex/childdoc|.
\end{itemize}

%%%%%%%%%%%%%%%%%%%%%%%%%%%%%%%%%%%%%%%%%%%%%%%%%%%%%%%%%%%%%%%%%%%%%%%%%%%%%%%%
\subsection{Related CTAN Packages}

There are several other packages which offer a similar functionality:
%
\begin{itemize}
\item
The packages
\href{http://ctan.org/pkg/docmute}{\textsf{docmute}},
\href{http://ctan.org/pkg/includex}{\textsf{includex}} and
\href{http://ctan.org/pkg/standalone}{\textsf{standalone}}
provide commands to include only the document body of
a child file thus allowing both files to be compiled individually.
\item
The packages \href{http://ctan.org/pkg/subdocs}{\textsf{subdocs}}
and \href{http://ctan.org/pkg/subfiles}{\textsf{subfiles}}
provide structures in which the main and child documents can be
encapsulated and allowing them to be compiled individually.
The inclusion mechanism is different from the conventional |\include|.
\item
The package \href{http://ctan.org/pkg/combine}{\textsf{combine}}
is an elaborate solution to combine several documents into one.
\end{itemize}
%
See also the CTAN topic \href{http://ctan.org/topic/subdocs}{\textsf{subdocs}}
for further related packages.
The present package differs from the above solutions in that
a document structure constructed with the conventional |\include| mechanism
just needs two extra commands at the top of every file
such that all constituent files can be compiled individually.

%%%%%%%%%%%%%%%%%%%%%%%%%%%%%%%%%%%%%%%%%%%%%%%%%%%%%%%%%%%%%%%%%%%%%%%%%%%%%%%%
%\subsection{Feature Suggestions}
%
%The following is a list of features which may be useful for future
%versions of this package:
%%
%\begin{itemize}
%\item
%\ldots
%\end{itemize}

%%%%%%%%%%%%%%%%%%%%%%%%%%%%%%%%%%%%%%%%%%%%%%%%%%%%%%%%%%%%%%%%%%%%%%%%%%%%%%%%
\subsection{Revision History}

%%%%%%%%%%%%%%%%%%%%%%%%%%%%%%%%%%%%%%%%
\paragraph{v2.0:} 2018/12/30

\begin{itemize}
\item
immediate forward processing
\item
added |\childdocby| mechanism
\item
manual restructured
\end{itemize}

%%%%%%%%%%%%%%%%%%%%%%%%%%%%%%%%%%%%%%%%
\paragraph{v1.6:} 2018/01/17

\begin{itemize}
\item
application for development of include files
\item
corrections to manual
\end{itemize}

%%%%%%%%%%%%%%%%%%%%%%%%%%%%%%%%%%%%%%%%
\paragraph{v1.5:} 2017/05/21

\begin{itemize}
\item
more complete structuring introduced
\item
|\childdocof| introduced
\item
|\childdoc| renamed to |\childdocmain|
\item
|\childredirect| renamed to |\childdocforward| and |\childdocforwardprefix|
and functionality expanded
\end{itemize}

%%%%%%%%%%%%%%%%%%%%%%%%%%%%%%%%%%%%%%%%
\paragraph{v1.0:} 2017/04/27

\begin{itemize}
\item
manual and install package
\item
first version published on CTAN
\end{itemize}

%%%%%%%%%%%%%%%%%%%%%%%%%%%%%%%%%%%%%%%%
\paragraph{v0.6:} 2017/04/26

\begin{itemize}
\item
redirection mechanism added
\end{itemize}

%%%%%%%%%%%%%%%%%%%%%%%%%%%%%%%%%%%%%%%%
\paragraph{v0.5:} 2017/04/26

\begin{itemize}
\item
functionality in definition file
\end{itemize}


%%%%%%%%%%%%%%%%%%%%%%%%%%%%%%%%%%%%%%%%%%%%%%%%%%%%%%%%%%%%%%%%%%%%%%%%%%%%%%%%
%%%%%%%%%%%%%%%%%%%%%%%%%%%%%%%%%%%%%%%%%%%%%%%%%%%%%%%%%%%%%%%%%%%%%%%%%%%%%%%%
%%%%%%%%%%%%%%%%%%%%%%%%%%%%%%%%%%%%%%%%%%%%%%%%%%%%%%%%%%%%%%%%%%%%%%%%%%%%%%%%
\appendix

\settowidth\MacroIndent{\rmfamily\scriptsize 000\ }

 \DocInput{childdoc.dtx}

\end{document}
%</driver>
% \fi
%
% %%%%%%%%%%%%%%%%%%%%%%%%%%%%%%%%%%%%%%%%%%%%%%%%%%%%%%%%%%%%%%%%%%%%%%%%%%%%%%
% %%%%%%%%%%%%%%%%%%%%%%%%%%%%%%%%%%%%%%%%%%%%%%%%%%%%%%%%%%%%%%%%%%%%%%%%%%%%%%
% \section{Sample}
%\iffalse
%<*samplemain>
%\fi
%
% The following presents a sample document
% with two chapters, two parts, a title page,
% a compile flag as well as three forwarding files to set the flag.
% It consists of eight |.tex| files:
% \begin{center}
% \begin{tabular}{ll}
% |cdocsamp.tex|&main file\\
% |cdocsch1.tex|&include file for chapter 1\\
% |cdocsch2.tex|&include file for chapter 2\\
% |cdocspt3.tex|&include file for part 3\\
% |cdocspt4.tex|&include file for part 4\\
% |cdocsdrf.tex|&forwarding file for main file in draft mode\\
% |cdocsfi1.tex|&forwarding file for final version of chapter 1\\
% |cdocsfi2.tex|&forwarding file for final version of chapter 2\\
% \end{tabular}
% \end{center}
% Each of the eight files can be compiled directly by the \LaTeX{} compiler.
%
% %%%%%%%%%%%%%%%%%%%%%%%%%%%%%%%%%%%%%%
% \paragraph{Main File.}
%
% The main file is called |cdocsamp.tex|.
%
% Load the \textsf{childdoc} definitions and
% declare the filename for the main document:
%    \begin{macrocode}
\input{childdoc.def}
\childdocmain{}
%    \end{macrocode}

% Optional override for |\version| flag:
%    \begin{macrocode}
%%\ifchilddoc\else\providecommand{\version}{draft}\fi
%    \end{macrocode}

% Define the default values for the |\version| flag
% (|final| for the main file and |draft| for childs):
%    \begin{macrocode}
\ifchilddoc
\providecommand{\version}{draft}
\else
\providecommand{\version}{final}
\fi
%    \end{macrocode}

% Load the standard document class:
%    \begin{macrocode}
\documentclass[12pt]{article}
%    \end{macrocode}

% Start the document body:
%    \begin{macrocode}
\begin{document}
%    \end{macrocode}

% Declare a title page.
% Print title, part of document being processed and version flag:
%    \begin{macrocode}
\addtocounter{page}{-1}
\begin{center}
{\LARGE\bfseries{}childdoc example\par}
\vspace{1cm}
\ifchilddoc
\ifchilddocmanual part\else chapter\fi:
`\childdocname' of `\childdocjob'\par
\else
main document: `\childdocjob'\par
\fi
version: \version\par
\end{center}
\newpage
%    \end{macrocode}

% Manually include selected file,
% otherwise process as usual:
%    \begin{macrocode}
\ifchilddocmanual
\section*{part `\childdocname'}
\input{\childdocname}
\else
%    \end{macrocode}

% Include the two chapters:
%    \begin{macrocode}
\include{cdocsch1}
\include{cdocsch2}
%    \end{macrocode}

% Include the two parts unless only chapters should be displayed:
%    \begin{macrocode}
\ifchilddoc\else
\section{part three}
\input{cdocspt3}
\section{part four}
\input{cdocspt4}
\fi
%    \end{macrocode}

% Process as usual until here:
%    \begin{macrocode}
\fi
%    \end{macrocode}

% End of document body:
%    \begin{macrocode}
\end{document}
%    \end{macrocode}
%\iffalse
%</samplemain>
%\fi
%
% %%%%%%%%%%%%%%%%%%%%%%%%%%%%%%%%%%%%%%
% \paragraph{Chapter Include Files.}
%
% The include files are called |cdocsch1.tex| and |cdocsch2.tex|.
%
%\iffalse
%<*samplechap1|samplechap2>
%\fi

% Optional override for |\version| flag:
%    \begin{macrocode}
%%\providecommand{\version}{final}
%    \end{macrocode}

% Include the main document:
%    \begin{macrocode}
\input{childdoc.def}
\childdocof{cdocsamp}
%    \end{macrocode}

%\iffalse
%</samplechap1|samplechap2>
%\fi
%
%\iffalse
%<*samplechap1>
%\fi
% Some text for chapter 1:
%    \begin{macrocode}
\section{one}
some text in chapter one
%    \end{macrocode}

%\iffalse
%</samplechap1>
%\fi
% Some text for chapter 2:
%\iffalse
%<*samplechap2>
%\fi
%    \begin{macrocode}
\section{two}
more text in chapter two
%    \end{macrocode}

%\iffalse
%</samplechap2>
%\fi
%
% %%%%%%%%%%%%%%%%%%%%%%%%%%%%%%%%%%%%%%
% \paragraph{Part Include Files.}
%
% The include files are called |cdocspt3.tex| and |cdocspt4.tex|.
%
%\iffalse
%<*samplepart3|samplepart4>
%\fi

% Optional override for |\version| flag:
%    \begin{macrocode}
%%\providecommand{\version}{final}
%    \end{macrocode}

% Include the main document:
%    \begin{macrocode}
\input{childdoc.def}
\childdocby{cdocsamp}
%    \end{macrocode}

%\iffalse
%</samplepart3|samplepart4>
%\fi
%
%\iffalse
%<*samplepart3>
%\fi
% Some text for part 3:
%    \begin{macrocode}
some text in part three
%    \end{macrocode}

%\iffalse
%</samplepart3>
%\fi
% Some text for part 4:
%\iffalse
%<*samplepart4>
%\fi
%    \begin{macrocode}
more text in part four
%    \end{macrocode}

%\iffalse
%</samplepart4>
%\fi
%
% %%%%%%%%%%%%%%%%%%%%%%%%%%%%%%%%%%%%%%
% \paragraph{Forwarding for a Complete Draft.}
%
% The following forwarding file |cdocsdrf.tex|
% compiles the main document in draft mode:
%\iffalse
%<*sampledraft>
%\fi
%    \begin{macrocode}
\def\version{draft}
\input{childdoc.def}
\childdocforward{cdocsamp}
%    \end{macrocode}

%\iffalse
%</sampledraft>
%\fi
%
% %%%%%%%%%%%%%%%%%%%%%%%%%%%%%%%%%%%%%%
% \paragraph{Forwarding for Final Version of the Chapters.}
%
% The following forwarding files |cdocsfn1.tex| and |cdocsfn2.tex|
% (with identical content)
% compile the final versions of the child documents
% |cdocsch1.tex| and |cdocsch2.tex|, respectively:
%\iffalse
%<*samplefinal>
%\fi
%    \begin{macrocode}
\def\version{final}
\input{childdoc.def}
\childdocforwardprefix[cdocsamp]{cdocsfn}{cdocsch}
%    \end{macrocode}

%\iffalse
%</samplefinal>
%\fi
%
% %%%%%%%%%%%%%%%%%%%%%%%%%%%%%%%%%%%%%%
% \paragraph{Command Line Processing.}
%
% The following three command lines generate the output files
% |cdocscld|, |cdocscl1| and |cdocscl2|
% which should be identical to
% |cdocsdrf|, |cdocsch1| and |cdocsfn2|, respectively:
% \begin{center}
% \begin{tabular}{l}
% |latex -jobname cdocscld \|\\
% |  "\def\version{draft}\input{childdoc.def}\childdocforward{cdocsamp}"|\\
% |latex -jobname cdocscl1 \|\\
% |  "\input{childdoc.def}\childdocforward[cdocsamp]{cdocsch1}"|\\
% |latex -jobname cdocscl2 \|\\
% |  "\def\version{final}\input{childdoc.def}\childdocforward{cdocsch2}"|
% \end{tabular}
% \end{center}
% Note that the trailing backslash on each first line
% merely continues the input to the second line
% (for convenient cut ant paste).
% Furthermore, the command |latex| can be replaced by any
% of its alternative versions such as |pdflatex|.
%
% %%%%%%%%%%%%%%%%%%%%%%%%%%%%%%%%%%%%%%%%%%%%%%%%%%%%%%%%%%%%%%%%%%%%%%%%%%%%%%
% %%%%%%%%%%%%%%%%%%%%%%%%%%%%%%%%%%%%%%%%%%%%%%%%%%%%%%%%%%%%%%%%%%%%%%%%%%%%%%
% \section{Implementation}
%\iffalse
%<*package>
%\fi
%
% This section describes the definitions file |childdoc.def|.

% The definitions cannot be loaded using |\usepackage| or |\RequirePackage|
% which has a mechanism to prevent loading a style file more than once.
% When loading the definitions by means of |\input|
% multiple instances have to be prevented manually:
%\iffalse
%This code needs to be before the `\ProvidesFile' directive
%which is defined at the beginning of this file.
%Therefore it is also placed there and commented out here.
%</package>
%<*discard>
%\fi
%    \begin{macrocode}
\ifdefined\childdocmain\endinput\fi
%    \end{macrocode}
%\iffalse
%</discard>
%<*package>
%\fi
%
% \macro{\ifchilddoc}
% \macro{\ifchilddocmanual}
% The conditional |\ifchilddoc| tells whether a
% child (true) or main (false) document is being compiled.
% The conditional |\ifchilddocmanual| tells whether
% the |\includeonly| mechanism is used (false) or
% the selection of child files must be performed manually (true).
% The definitions initialise to false:
%    \begin{macrocode}
\newif\ifchilddoc
\newif\ifchilddocmanual
%    \end{macrocode}

% \macro{\childdocname}
% \macro{\childdocjob}
% The macro |\childdocname| stores the name of the main document
% to be compiled. The macro |\childdocjob| stores the name of
% the document on which the \LaTeX{} compiler was originally invoked.
% The content of |\jobname| cannot be compared
% to filenames specified in the source due to different catcodes.
% The following code rescans |\jobname|, stores the result
% in |\childdocname| and saves a copy in |\childdocjob|:
%    \begin{macrocode}
\edef\childdocname{\scantokens\expandafter{\jobname\noexpand}}
\let\childdocjob\childdocname
%    \end{macrocode}

% \macro{\childdocdisable}
% The macro |\childdocdisable| prevents the main file
% from being processed more than once.
% At this stage, the main document command |\childdocmain|
% is assumed to be called once again where it should do nothing.
% Any subsequent call to it should prevent
% a secondary processing of the main document
% It overwrites the forwarding commands
% |\childdocof| and |\childdocforward|
% with empty macros to prevent further inclusions of the main document:
%    \begin{macrocode}
\newcommand{\childdocdisable}
{
  \renewcommand{\childdocmain}[1]{\renewcommand{\childdocmain}[1]{\endinput}}
  \renewcommand{\childdocof}[1]{}
  \renewcommand{\childdocby}[2][]{}
  \renewcommand{\childdocforward}[2][]{}
  \renewcommand{\childdocdisable}{}
}
%    \end{macrocode}

% \macro{\childdocmain}
% The macro |\childdocmain| is to be called at the top of the main file
% with nothing or the main filename (without extension) as argument.
% First, it breaks loops.
% If the argument is not empty and does not match |\childdocname|
% (which is set by the first inclusion of |childdoc.def|),
% |\ifchilddoc| is set to true, |\includeonly| is applied to the child file
% and |\jobname| is set to the main file
% (for proper handling of |.aux| files):
%    \begin{macrocode}
\newcommand{\childdocmain}[1]
{
  \childdocdisable\childdocmain{}
  \if?#1?\else
    \begingroup
      \def\childdoctmp{#1}
      \ifx\childdoctmp\childdocname
        \def\childdoctmp{}
      \else
        \def\childdoctmp
        {
          \childdoctrue
          \includeonly{\childdocname}
          \def\childdocjob{#1}
          \def\jobname{#1}
        }
      \fi
      \expandafter
    \endgroup
    \childdoctmp
  \fi
}
%    \end{macrocode}

% \macro{\childdocof}
% The command |\childdocof| redirects
% compilation to the main file |#1|.
%    \begin{macrocode}
\newcommand{\childdocof}[1]
{
  \childdocdisable
  \childdoctrue
  \includeonly{\childdocname}
  \def\jobname{#1}
  \def\childdocjob{#1}
  \input{#1}
}
%    \end{macrocode}

% \macro{\childdocby}
% The command |\childdocby| ....
%    \begin{macrocode}
\newcommand{\childdocby}[2][]
{
  \childdocdisable
  \childdoctrue
  \childdocmanualtrue
  \if?#1?\else
    \def\jobname{#2}
  \fi
  \def\childdocjob{#2}
  \input{#2}
  \endinput
}
%    \end{macrocode}

% \macro{\childdocforward}
% The command |\childdocforward| redirects
% compilation to the main file or
% (if the optional argument is given) a child file.
% Parameters are set as if the main file
% or a child file starting with |\childdocof| was compiled.
% Then compilation is handed over to the main file:
%    \begin{macrocode}
\newcommand{\childdocforward}[2][]
{
  \begingroup
    \if?#1?
      \def\childdoctmp
      {
        \def\childdocname{#2}
        \def\childdocjob{#2}
        \def\jobname{#2}
        \input{#2}
        \endinput
      }
    \else
      \def\childdoctmp
      {
        \childdocdisable
        \def\childdocname{#2}
        \childdoctrue
        \includeonly{#2}
        \def\childdocjob{#1}
        \def\jobname{#1}
        \input{#1}
        \endinput
      }
    \fi
    \expandafter
  \endgroup
  \childdoctmp
}
%    \end{macrocode}

% \macro{\childdocforwardprefix}
% The command |\childdocforwardprefix| redirects
% compilation to the main or a child file by means of a pattern.
% The prefix |#1| in the current filename is replaced by |#2|
% and the suffix of the current filename is kept
% (it is assumed that the filename does not contain the substring `|~~~|'
% which is used as a delimiter).
% Compilation is handed over to the new file by |\childdocforward|:
%    \begin{macrocode}
\newcommand{\childdocforwardprefix}[3][]
{
  \begingroup
    \def\childdocextract #2##1~~~{\def\childdoctmp{\childdocforward[#1]{#3##1}}}
    \expandafter\childdocextract\childdocname~~~
    \expandafter
  \endgroup
  \childdoctmp
}
%    \end{macrocode}

% \macro{\childdoc}
% The deprecated macro |\childdoc| is a legacy version of |\childdocmain|:
%    \begin{macrocode}
\newcommand{\childdoc}{\childdocmain}
%    \end{macrocode}

% \macro{\childdocredirect}
% The deprecated macro |\childdocredirect| is a legacy version
% of |\childdocforward| and |\childdocforwardprefix|:
%    \begin{macrocode}
\newcommand{\childdocredirect}[2][]
{
  \begingroup
    \if?#1?
      \def\childdoctmp{\childdocforward{#2}}
    \else
      \def\childdoctmp{\childdocforwardprefix{#1}{#2}}
    \fi
    \expandafter
  \endgroup
  \childdoctmp
}
%    \end{macrocode}

%\iffalse
%</package>
%\fi
%
\endinput

\childdocforward{cdocsamp}
%    \end{macrocode}

%\iffalse
%</sampledraft>
%\fi
%
% %%%%%%%%%%%%%%%%%%%%%%%%%%%%%%%%%%%%%%
% \paragraph{Forwarding for Final Version of the Chapters.}
%
% The following forwarding files |cdocsfn1.tex| and |cdocsfn2.tex|
% (with identical content)
% compile the final versions of the child documents
% |cdocsch1.tex| and |cdocsch2.tex|, respectively:
%\iffalse
%<*samplefinal>
%\fi
%    \begin{macrocode}
\def\version{final}
% \iffalse
%
% childdoc.dtx Copyright (C) 2017-2018 Niklas Beisert
%
% This work may be distributed and/or modified under the
% conditions of the LaTeX Project Public License, either version 1.3
% of this license or (at your option) any later version.
% The latest version of this license is in
%   http://www.latex-project.org/lppl.txt
% and version 1.3 or later is part of all distributions of LaTeX
% version 2005/12/01 or later.
%
% This work has the LPPL maintenance status `maintained'.
%
% The Current Maintainer of this work is Niklas Beisert.
%
% This work consists of the files childdoc.dtx and childdoc.ins
% and the derived files childdoc.def and cdocsamp.tex with
% cdocsch1.tex, cdocsch2.tex, cdocsdrf.tex, cdocsfn1.tex, cdocsfn2.tex.
%
%<package>\ifdefined\childdocmain\endinput\fi
%<package>\ProvidesFile{childdoc.def}[2018/12/30 v2.0 child document driver]
%<samplemain>\ProvidesFile{cdocsamp.tex}[2018/12/30 v2.0 sample for childdoc]
%<*driver>
%\ProvidesFile{childdoc.drv}[2018/12/30 v2.0 childdoc reference manual file]
\PassOptionsToClass{10pt,a4paper}{article}
\documentclass{ltxdoc}

\usepackage[margin=35mm]{geometry}
\usepackage{hyperref}
\usepackage{hyperxmp}
\usepackage[usenames]{color}

\hypersetup{colorlinks=true}
\hypersetup{pdfstartview=FitH}
\hypersetup{pdfpagemode=UseNone}
\hypersetup{pdfsource={}}
\hypersetup{pdflang={en-UK}}
\hypersetup{pdfcopyright={Copyright 2017-2018 Niklas Beisert.
  This work may be distributed and/or modified under the
  conditions of the LaTeX Project Public License, either version 1.3
  of this license or (at your option) any later version.}}
\hypersetup{pdflicenseurl={http://www.latex-project.org/lppl.txt}}
\hypersetup{pdfcontactaddress={ETH Zurich, ITP, HIT K,
  Wolfgang-Pauli-Strasse 27}}
\hypersetup{pdfcontactpostcode={8093}}
\hypersetup{pdfcontactcity={Zurich}}
\hypersetup{pdfcontactcountry={Switzerland}}
\hypersetup{pdfcontactemail={nbeisert@itp.phys.ethz.ch}}
\hypersetup{pdfcontacturl={http://people.phys.ethz.ch/\xmptilde nbeisert/}}

\newcommand{\secref}[1]{\hyperref[#1]{section \ref*{#1}}}

\parskip1ex
\parindent0pt
\let\olditemize\itemize
\def\itemize{\olditemize\parskip0pt}

\begin{document}

\title{The \textsf{childdoc} Package}
\hypersetup{pdftitle={The childdoc Package}}
\author{Niklas Beisert\\[2ex]
  Institut f\"ur Theoretische Physik\\
  Eidgen\"ossische Technische Hochschule Z\"urich\\
  Wolfgang-Pauli-Strasse 27, 8093 Z\"urich, Switzerland\\[1ex]
  \href{mailto:nbeisert@itp.phys.ethz.ch}
  {\texttt{nbeisert@itp.phys.ethz.ch}}}
\hypersetup{pdfauthor={Niklas Beisert}}
\hypersetup{pdfsubject={Manual for the LaTeX2e Package childdoc}}
\date{30 December 2018, \textsf{v2.0}}
\maketitle

\begin{abstract}\noindent
\textsf{childdoc} is a \LaTeXe{} package
that enables the direct compilation
of document sections included by |\include|
to individual files.
\end{abstract}

\begingroup
\parskip0ex
\tableofcontents
\endgroup

%%%%%%%%%%%%%%%%%%%%%%%%%%%%%%%%%%%%%%%%%%%%%%%%%%%%%%%%%%%%%%%%%%%%%%%%%%%%%%%%
%%%%%%%%%%%%%%%%%%%%%%%%%%%%%%%%%%%%%%%%%%%%%%%%%%%%%%%%%%%%%%%%%%%%%%%%%%%%%%%%
\section{Introduction}

\LaTeX{} provides a mechanism to structure a large document (such as a book)
into a main file and several child files (containing the chapters)
using the |\include| command.
This mechanism is beneficial for documents
which span hundreds of pages in order to
make the source file(s) more manageable.
Moreover, compilation can be restricted to
selected child files by means of the |\includeonly| command.
The latter feature can be used to reduce the compilation time while editing
(this was significantly more useful in the earlier days of \LaTeX{})
or to generate a smaller document which is easier to navigate.
Another application of |\includeonly| is to generate
documents consisting of selected parts of the complete document.

However, there are a few drawbacks of the plain |\include| mechanism:
\begin{itemize}
\item
The child files cannot be compiled on their own,
they can only be compiled via the main file.
A naive editing environment
(such as a text editor with an option
to have the current file processed by \LaTeX)
may require one to switch to the main file before compiling;
attempting to compile the child file produces errors.
\item
The main file must be modified (each time)
to adjust the |\includeonly| command
to the present needs. This easily leaves the main file in a messy state.
\item
The generated document will always carry the filename
of the main document. This is inconvenient if
several child files are to be compiled and
to be kept for distribution.
\end{itemize}

The present package provides a simple interface
to make child files individually compilable by \LaTeX{}.
Compiling a child file then has the same effect as compiling
the main file with an |\includeonly| command
to select the appropriate child.
Moreover the generated document will carry the name of the child
rather than the main file.
This resolves all three above issues.

This feature is meant to make the editing of books,
thesis documents and lecture notes somewhat more convenient.
However, the package can also be used efficiently for
composing a series of documents (such as exercise sheets)
which are typically distributed individually.
It then assists the author in generating the individual documents
(potentially in different versions)
as well as a document containing the collected series.
Another application is in developing style files
or other kinds of included material
where compilation of the style file could redirect
to a sample or test file.

%%%%%%%%%%%%%%%%%%%%%%%%%%%%%%%%%%%%%%%%%%%%%%%%%%%%%%%%%%%%%%%%%%%%%%%%%%%%%%%%
%%%%%%%%%%%%%%%%%%%%%%%%%%%%%%%%%%%%%%%%%%%%%%%%%%%%%%%%%%%%%%%%%%%%%%%%%%%%%%%%
\section{Usage}

First of all, the package \textsf{childdoc} is \emph{not} a standard
\LaTeXe{} |.sty| style file! Therefore it needs to be invoked in
a non-standard way.

%%%%%%%%%%%%%%%%%%%%%%%%%%%%%%%%%%%%%%%%%%%%%%%%%%%%%%%%%%%%%%%%%%%%%%%%%%%%%%%%
\subsection{Included Files}
\label{sec:include}

%%%%%%%%%%%%%%%%%%%%%%%%%%%%%%%%%%%%%%%%
\DescribeMacro{\childdocmain}
To use the package, add the commands
\begin{center}
\begin{tabular}{l}
|\input{childdoc.def}|\\
|\childdocmain{}|\\
\end{tabular}
\end{center}
at the very top of the main \LaTeX{} file,
in particular \emph{before} the |\documentclass| statement!
The argument of |\childdocmain| should be left empty
(but it must be present).

%%%%%%%%%%%%%%%%%%%%%%%%%%%%%%%%%%%%%%%%
\DescribeMacro{\childdocof}
Furthermore, add the commands
\begin{center}
\begin{tabular}{l}
|\input{childdoc.def}|\\
|\childdocof{|\textit{main}|}|\\
\end{tabular}
\end{center}
at the top of every child file \textit{child}
which is included by |\include{|\textit{child}|}|
from within the main file
(or at least for those files to be compiled individually).
The argument \textit{main} must be the filename of the main file.

There are a couple of
considerations in setting up the main and child documents:

%%%%%%%%%%%%%%%%%%%%%%%%%%%%%%%%%%%%%%%%
\paragraph{Restrictions.}

Please note the following restrictions:
\begin{itemize}
\item
|\childdocmain| must be called with one argument \textit{main}
to ensure compatibility with earlier version of the package.
It must either be empty (|\childdocmain{}|)
or precisely match the filename of the main file in which it is specified.
See \secref{sec:detection} for further information.
\item
The filename \textit{main} must be specified without the |.tex| extension.
\item
The filename \textit{main} is case sensitive
(even in case-insensitive file systems)
due to internal string comparison.
\item
The argument \textit{main} should be fully expanded, it cannot be a macro.
\item
Subdirectories and special characters should be avoided in filenames.
\item
The command |\childdocmain{|\textit{main}|}| must be followed by a whitespace.
It should not be followed immediately by another command
or by a comment mark `|%|'.
This is because the \TeX{} parser reads the token immediately following
the argument of |\childdocmain| and puts it
at the beginning of every child section;
however, a white\-space is ignored.
\end{itemize}

%%%%%%%%%%%%%%%%%%%%%%%%%%%%%%%%%%%%%%%%
\paragraph{Content of Main File.}

It is advisable to place all content in the child files included by |\include|.
Any output contained in the main file will appear in all child documents
unless suppressed manually;
it cannot be suppressed automatically by the |\includeonly| directive
and thus should normally be avoided.
A method to include some content in the main file
by means of conditional processing is described in \secref{sec:conditional}.

%%%%%%%%%%%%%%%%%%%%%%%%%%%%%%%%%%%%%%%%
\paragraph{Page Numbering.}

When only a part of the document is compiled,
the appropriate numbering of pages
(as well as other status parameters)
is determined from the |.aux| files.
The latter contain information from previous passes.
However this information needs to propagate through
all intermediate child documents.
Therefore the page numbering in child documents may well
be inconsistent until the complete document is compiled at least once.

A useful (if unconventional) way to always ensure a consistent
page numbering is to restart the numbering in each child document
and denote the pages by `\textit{child}|.|\textit{page}'
where \textit{child} represents the chapter/section number of the child file.
This can be achieved by the command
|\numberwithin{page}{|\textit{child}|}|
of the \textsf{amsmath} package
where \textit{child} can be |chapter| or |section|
depending on the chosen structuring.
Alternatively, one can modify the macro |\thepage| appropriately
and reset the counter |page| at the start of each child file.

%%%%%%%%%%%%%%%%%%%%%%%%%%%%%%%%%%%%%%%%%%%%%%%%%%%%%%%%%%%%%%%%%%%%%%%%%%%%%%%%
\subsection{Conditional Processing}
\label{sec:conditional}

The package provides a mechanism to compile different versions
of a document. To customise the versions further some conditional processing
can come in handy to distinguish which version is being compiled.
The package provides two macros to describe the compilation context:

%%%%%%%%%%%%%%%%%%%%%%%%%%%%%%%%%%%%%%%%
\DescribeMacro{\ifchilddoc}
The conditional |\ifchilddoc| distinguishes between the compilation of
child documents and the main document:
%
\begin{center}
|\ifchilddoc |\textit{child-code}| |[|\||else |\textit{main-code}]| \||fi|
\end{center}

%%%%%%%%%%%%%%%%%%%%%%%%%%%%%%%%%%%%%%%%
\DescribeMacro{\childdocname}
\DescribeMacro{\childdocjob}
The macro |\childdocname| contains the filename (without extension)
of the main or child file being processed.
Note that |\childdocjob| will always contain the name of the main file.

%%%%%%%%%%%%%%%%%%%%%%%%%%%%%%%%%%%%%%%%
\paragraph{Title Page.}

Conditional processing can be used to include a title or banner page
in the main document when proper precautions are taken.
Importantly, the code in the main file should ensure that the page counter
(as well as other status parameters which are stored in the |.aux| files)
takes the same value after the conditional processing.
Otherwise the page numbers may take divergent values
depending on which part is compiled.

For example, a title page could be declared by:
%
\begin{center}
\begin{tabular}{l}
|\ifchilddoc\||else|\\
|\addtocounter{page}{-1}|\\
\textit{code for title page}\\
|\newpage|\\
|\||fi|
\end{tabular}
\end{center}
%
A banner page for the child documents can be generated by:
%
\begin{center}
\begin{tabular}{l}
|\ifchilddoc|\\
|\addtocounter{page}{-1}|\\
\textit{code for banner page}\\
|\newpage|\\
|\||fi|
\end{tabular}
\end{center}
%
Here one could write a message such as:
\begin{center}
|This is the part \childdocname{} of \childdocjob{}.|
\end{center}

%%%%%%%%%%%%%%%%%%%%%%%%%%%%%%%%%%%%%%%%%%%%%%%%%%%%%%%%%%%%%%%%%%%%%%%%%%%%%%%%
\subsection{Flags}
\label{sec:flags}

The package makes it easy to generate different versions
of the main or child documents.
To this end compilation flags can be defined
and assigned different default values.
They will be particularly useful in conjunction
with the forwarding mechanism described in \secref{sec:forward}.

For example, it may be useful to have a flag |\version|
which can be set to |draft| or |final|.
The document source will contain some conditional code
depending on the value of |\version|.
Suppose further, the flag should default to |final| for the main file
and to |draft| for child files
which is a natural assignment for editing the document.
This is achieved by placing the following code
in the preamble of the main document
(below the |\childdocmain| directive):
%
\begin{center}
\begin{tabular}{l}
|\ifchilddoc|\\
|\providecommand{\version}{draft}|\\
|\||else|\\
|\providecommand{\version}{final}|\\
|\||fi|
\end{tabular}
\end{center}
%
The definition by |\providecommand| makes sure
that previous definitions are not overwritten.
Further statements |\providecommand{\version}{...}|
can thus be added before the above code to override it.

For the main file, one might add a line
(between |\childdocmain| and the above block)
%
\begin{center}
|%\ifchilddoc\||else\providecommand{\version}{draft}\||fi|
\end{center}
%
which can be uncommented to produce a draft version.
Likewise one can add a line to the very top of a child file
(above the |\childdocof{|\textit{main}|}| directive)
%
\begin{center}
|%\providecommand{\version}{final}|
\end{center}
%
which can be uncommented to produce the final version of this child document.

%%%%%%%%%%%%%%%%%%%%%%%%%%%%%%%%%%%%%%%%%%%%%%%%%%%%%%%%%%%%%%%%%%%%%%%%%%%%%%%%
\subsection{Forwarding}
\label{sec:forward}

Different versions of the main or child documents
using compilation flags as described in \secref{sec:flags}
can be (permanently) stored in different files
for convenient compilation, viewing and distribution.
To this end, the package defines a command
to pass on compilation to a different file:

%%%%%%%%%%%%%%%%%%%%%%%%%%%%%%%%%%%%%%%%
\DescribeMacro{\childdocforward}
The command |\childdocforward| redirects processing to
another source file:
%
\begin{center}
\begin{tabular}{l}
|\input{childdoc.def}|\\
|\childdocforward[|\textit{main}|]{|\textit{dest}|}|\\
\end{tabular}
\end{center}
%
The argument \textit{dest} is the destination file
(without extension).
It should be the main file or one of the child files.
Note that further \textsf{childdoc} directives
such as |\childdocof| and |\childdocforward|
in the indicated file will be processed in this form.
The optional argument \textit{main}
passes on directly to the main file \textit{main}
while pretending to compile the child \textit{dest}.
This form behaves as if \textit{dest}
issues |\childdocof{|\textit{main}|}| right away,
and no further \textsf{childdoc} directives will be processed.

%%%%%%%%%%%%%%%%%%%%%%%%%%%%%%%%%%%%%%%%
\DescribeMacro{\...prefix}
In the alternative form |\childdocforwardprefix|,
%
\begin{center}
\begin{tabular}{l}
|\input{childdoc.def}|\\
|\childdocforwardprefix[|\textit{main}|]{|\textit{prefix}|}{|\textit{dest}|}|
\end{tabular}
\end{center}
%
the destination file is determined by a pattern
depending on the current file:
To make this work, the current file must be called
`{\textit{prefix}\hspace{0.2em}\textit{suffix}}'
with \textit{prefix} matching precisely the argument.
Processing is then passed on to the file
`{\textit{dest}\hspace{0.2em}\textit{suffix}}'.
Surely, the same effect is achieved by
directly specifying the
argument `{\textit{dest}\hspace{0.2em}\textit{suffix}}'
in the first form.
However, that requires to set up a different file
for each child. With the alternative form of the command
all these files can have exactly the same content
which simplifies setting them up and maintaining them.

For example, the following file |draft.tex|
with a compilation flag |\version| as described in \secref{sec:flags}
compiles the main document as a draft:
%
\begin{center}
\begin{tabular}{l}
|\def\version{draft}|\\
|\input{childdoc.def}|\\
|\childdocforward{|\textit{main}|}|
\end{tabular}
\end{center}
%
Likewise, the following files |final|\textit{nn}|.tex|
compile the final version of the child document
|child|\textit{nn}|.tex|:
%
\begin{center}
\begin{tabular}{l}
|\def\version{final}|\\
|\input{childdoc.def}|\\
|\childdocforwardprefix{final}{child}|
\end{tabular}
\end{center}
%

Note that when several versions of a main file and/or of each child file
are to be generated, it may be convenient to set up a |Makefile| or
shell script to automatise the process.

%%%%%%%%%%%%%%%%%%%%%%%%%%%%%%%%%%%%%%%%%%%%%%%%%%%%%%%%%%%%%%%%%%%%%%%%%%%%%%%%
\subsection{Command Line Processing}
\label{sec:commandline}

The effect of redirection files can also be achieved by invoking
the \LaTeX{} compiler with a more elaborate command line.
Most conveniently this should be done as part
of a shell script or a |Makefile|.

When using \textsf{childdoc} in the main file, the following
command lines effectively perform a redirection
(note that depending on the shell being used,
backslashes may have to be doubled: `|\|' $\to$ `|\\|'):
%
\begin{center}
|... -jobname "|\textit{target}|" |\\|"|[\textit{flags}]%
|\input{childdoc.def}\childdocforward[|\textit{main}|]{|\textit{dest}|}"|
\end{center}
%
Here \textit{target} is the name of the output file,
\textit{main} is the name of the main file
and \textit{dest} is the name of the main or child file to be processed
(all filenames without extensions).
The optional argument \textit{main} can be omitted
if \textit{main} matches \textit{dest}.
Optionally, compilation \textit{flags} can be defined via |\def| commands.
This command line makes the \TeX{} engine believe
it is compiling the file \textit{target}
whose content is specified as the latter parameter.
The provided code then forwards the processing to
\textit{main} or \textit{dest} as described in \secref{sec:forward}.

%%%%%%%%%%%%%%%%%%%%%%%%%%%%%%%%%%%%%%%%%%%%%%%%%%%%%%%%%%%%%%%%%%%%%%%%%%%%%%%%
\subsection{Include by Input}
\label{sec:input}

Including child documents by |\include| has some restrictions by design.
Most notably, the content of a child document always occupies
its own set of pages; pages cannot be shared between child documents.
Usually, this behaviour makes perfect sense
because each child document contain an essential part of the document.
However, in some situations it may be desirable to compose
a document from a collection of parts
without having mandatory page breaks between then.
For this case, the package
provides a mechanism to include parts
by |\input| which can also be processed individually.
However, by construction this mechanism
requires manual handling of the content to be output.

%%%%%%%%%%%%%%%%%%%%%%%%%%%%%%%%%%%%%%%%
\DescribeMacro{\ifchilddocmanual}
The main file should be prepared as usual, see \secref{sec:include}.
However, the document body must make a distinction
between processing of an individual part and of the main document, e.g.:
%
\begin{center}
\begin{tabular}{l}
|\ifchilddocmanual|\\
|\input{\childdocname}|\\
|\||else|\\
\textit{document body with }|\input{|\textit{part}|}|\\
|\||fi|
\end{tabular}
\end{center}
%
The conditional |\ifchilddocmanual| is true whenever
a part to be included by |\input| is being compiled,
and the name of the part is stored in |\childdocname|.

%%%%%%%%%%%%%%%%%%%%%%%%%%%%%%%%%%%%%%%%
\DescribeMacro{\childdocby}
Each part to be included by |\input| should start with:
%
\begin{center}
\begin{tabular}{l}
|\input{childdoc.def}|\\
|\childdocby{|\textit{main}|}|\\
\end{tabular}
\end{center}
%
The directive |\childdocby| is similar to |\childdocof|
described in \secref{sec:include},
but the subsequent selection of content must be done manually.
To that end, both |\ifchilddoc| and |\ifchilddocmanual|
will be true upon processing of a part,
and the name of the part is stored in |\childdocname|.
Note that |\jobname| will be set to the filename of the current part
so that each part receives an individual |.aux| file
that does not interfere with the |.aux| file(s) of the main document.
This behaviour can be altered by the alternative form
|\childdocby[*]{|\textit{main}|}| (with a non-empty optional argument)
which uses the |.aux| file of the main document
by setting |\jobname| to \textit{main}.

%%%%%%%%%%%%%%%%%%%%%%%%%%%%%%%%%%%%%%%%%%%%%%%%%%%%%%%%%%%%%%%%%%%%%%%%%%%%%%%%
\subsection{Driver Development}
\label{sec:driver}

The \textsf{childdoc} mechanism can also be use for the development
of definition files such as \LaTeX{} styles or classes.
This case differs from the above setup with multiple parts
included by |\include| in that no |\includeonly| should be invoked.
This can be achieved by starting the include file
(before |\ProvidesPackage|) with:
%
\begin{center}
\begin{tabular}{l}
|\input{childdoc.def}|\\
|\childdocforward{|\textit{main}|}|\\
\end{tabular}
\end{center}
%
or alternatively with:
%
\begin{center}
\begin{tabular}{l}
|\input{childdoc.def}|\\
|\childdocby{|\textit{main}|}|\\
\end{tabular}
\end{center}
%
Both forms have slightly different effects as described above.
The main file is prepared as usual, see \secref{sec:include}.

%%%%%%%%%%%%%%%%%%%%%%%%%%%%%%%%%%%%%%%%%%%%%%%%%%%%%%%%%%%%%%%%%%%%%%%%%%%%%%%%
\subsection{Legacy Detection}
\label{sec:detection}

The directive |\childdocmain| in the main file can detect
whether the complete document or merely a child is to be compiled
even without using the directive |\childdocof|.
This method is deprecated because it is less robust
and there is no compelling reason to use it;
it is merely provided for backward compatibility
and it may be removed in future versions.

If the detection mechanism is to be used,
it is mandatory to correctly specify
the filename of the main file as the argument of |\childdocmain|:
%
\begin{center}
\begin{tabular}{l}
|\input{childdoc.def}|\\
|\childdocmain{|\textit{main}|}|\\
\end{tabular}
\end{center}
%
If |\jobname| does not match the argument \textit{main} of |\childdocmain|,
it is assumed that |\jobname| points to the child file to be compiled.
When using |\childdocmain| with the main file specified as argument,
it suffices to start a child file
with just |\input{|\textit{main}|}|
without loading of the package and using |\childdocof|.
If instead all processing is done
with the appropriate \textsf{childdoc} directives,
the argument of \textit{main} of |\childdocmain| can be empty.

An alternative version of the command line processing described
in \secref{sec:commandline} using the detection mechanism reads:
%
\begin{center}
|... -jobname "|\textit{target}|" "|[\textit{flags}]%
[|\def\jobname{|\textit{dest}|}|]|\input{|\textit{main}|}"|
\end{center}

%%%%%%%%%%%%%%%%%%%%%%%%%%%%%%%%%%%%%%%%%%%%%%%%%%%%%%%%%%%%%%%%%%%%%%%%%%%%%%%%
\subsection{Manual Code}
\label{sec:manual}

In case one cannot be certain whether the definitions file |childdoc.def|
is installed on the target \TeX{} distribution
and one prefers not to ship it,
it is conceivable to paste a few relevant commands into the sources.

To that end, drop all statements |\input{childdoc.def}|
and perform the replacements as outlined below.
Instead of |\childdocmain{|\textit{main}|}| add the following code
to the top of the main file:
%
\begin{center}
\begin{tabular}{l}
|\||ifdefined\childdocname\endinput\||fi\newif\ifchilddoc|\\
|\edef\childdocname{\scantokens\expandafter{\jobname\noexpand}}|\\
|\def\childdocmain{|\textit{main}|}\||ifx\childdocmain\childdocname\||else|\\
|\childdoctrue\includeonly{\childdocname}\let\jobname\childdocmain\||fi|\\
\end{tabular}
\end{center}
%
Instead of |\childdocof{|\textit{main}|}| just include the main file
at the top of each child file:
%
\begin{center}
|\input{|\textit{main}|}|
\end{center}
%
A simple redirection |\childdocforward{|\textit{dest}|}| is achieved by:
%
\begin{center}
|\def\jobname{|\textit{dest}|}\input{\jobname}|
\end{center}
%
The redirection with prefix
|\childdocforwardprefix[|\textit{prefix}|]{|\textit{dest}|}|
is accomplished by:
%
\begin{center}
\begin{tabular}{l}
|{\edef\jobname{\scantokens\expandafter{\jobname\noexpand}}|\\
|\def\redirectjob |\textit{prefix}|#1~~~{\gdef\jobname{|\textit{dest}|#1}}|\\
|\expandafter\redirectjob\jobname~~~}\input{\jobname}|
\end{tabular}
\end{center}

In an alternative approach,
child documents can be compiled by a specific command line
without additional code or specific definitions:
%
\begin{center}
|... -jobname "|\textit{target}|" "|[\textit{flags}]%
|\includeonly{|\textit{dest}|}\input{|\textit{main}|}"|
\end{center}
%

%%%%%%%%%%%%%%%%%%%%%%%%%%%%%%%%%%%%%%%%%%%%%%%%%%%%%%%%%%%%%%%%%%%%%%%%%%%%%%%%
%%%%%%%%%%%%%%%%%%%%%%%%%%%%%%%%%%%%%%%%%%%%%%%%%%%%%%%%%%%%%%%%%%%%%%%%%%%%%%%%
\section{Information}

%%%%%%%%%%%%%%%%%%%%%%%%%%%%%%%%%%%%%%%%%%%%%%%%%%%%%%%%%%%%%%%%%%%%%%%%%%%%%%%%
\subsection{Copyright}

Copyright \copyright{} 2017--2018 Niklas Beisert

This work may be distributed and/or modified under the
conditions of the \LaTeX{} Project Public License, either version 1.3
of this license or (at your option) any later version.
The latest version of this license is in
  \url{http://www.latex-project.org/lppl.txt}
and version 1.3 or later is part of all distributions of \LaTeX{}
version 2005/12/01 or later.

This work has the LPPL maintenance status `maintained'.

The Current Maintainer of this work is Niklas Beisert.

This work consists of the files |README.txt|, |childdoc.ins| and |childdoc.dtx|
as well as the derived files |childdoc.def|, |cdocsamp.tex|
with |cdocsch1.tex|, |cdocsch2.tex|, |cdocspt3.tex|, |cdocspt4.tex|,
|cdocsdrf.tex|, |cdocsfn1.tex|, |cdocsfn2.tex|
as well as |childdoc.pdf|.

%%%%%%%%%%%%%%%%%%%%%%%%%%%%%%%%%%%%%%%%%%%%%%%%%%%%%%%%%%%%%%%%%%%%%%%%%%%%%%%%
\subsection{Files and Installation}

The package consists of the files:
%
\begin{center}
\begin{tabular}{ll}
    |README.txt|   & readme file \\
    |childdoc.ins| & installation file \\
    |childdoc.dtx| & source file \\
    |childdoc.def| & definition file \\
    |cdocsamp.tex| & sample main file \\
    |cdocsch1.tex| & sample include file \\
    |cdocsch2.tex| & sample include file \\
    |cdocspt3.tex| & sample part file \\
    |cdocspt4.tex| & sample part file \\
    |cdocsdrf.tex| & sample redirection file \\
    |cdocsfn1.tex| & sample redirection file \\
    |cdocsfn2.tex| & sample redirection file \\
    |childdoc.pdf| & manual
\end{tabular}
\end{center}
%
The distribution consists of the files
|README.txt|, |childdoc.ins| and |childdoc.dtx|.
%
\begin{itemize}
\item
Run (pdf)\LaTeX{} on |childdoc.dtx|
to compile the manual |childdoc.pdf| (this file).
\item
Run \LaTeX{} on |childdoc.ins| to create the definitions file |childdoc.def|
and the sample |cdocsamp.tex| with include files
|cdocsch1.tex|, |cdocsch2.tex|, |cdocspt3.tex|, |cdocspt4.tex|,
|cdocsdrf.tex|, |cdocsfn1.tex|, |cdocsfn2.tex|.
Then copy the file |childdoc.def| to an appropriate directory of your \LaTeX{}
distribution, e.g.\ \textit{texmf-root}|/tex/latex/childdoc|.
\end{itemize}

%%%%%%%%%%%%%%%%%%%%%%%%%%%%%%%%%%%%%%%%%%%%%%%%%%%%%%%%%%%%%%%%%%%%%%%%%%%%%%%%
\subsection{Related CTAN Packages}

There are several other packages which offer a similar functionality:
%
\begin{itemize}
\item
The packages
\href{http://ctan.org/pkg/docmute}{\textsf{docmute}},
\href{http://ctan.org/pkg/includex}{\textsf{includex}} and
\href{http://ctan.org/pkg/standalone}{\textsf{standalone}}
provide commands to include only the document body of
a child file thus allowing both files to be compiled individually.
\item
The packages \href{http://ctan.org/pkg/subdocs}{\textsf{subdocs}}
and \href{http://ctan.org/pkg/subfiles}{\textsf{subfiles}}
provide structures in which the main and child documents can be
encapsulated and allowing them to be compiled individually.
The inclusion mechanism is different from the conventional |\include|.
\item
The package \href{http://ctan.org/pkg/combine}{\textsf{combine}}
is an elaborate solution to combine several documents into one.
\end{itemize}
%
See also the CTAN topic \href{http://ctan.org/topic/subdocs}{\textsf{subdocs}}
for further related packages.
The present package differs from the above solutions in that
a document structure constructed with the conventional |\include| mechanism
just needs two extra commands at the top of every file
such that all constituent files can be compiled individually.

%%%%%%%%%%%%%%%%%%%%%%%%%%%%%%%%%%%%%%%%%%%%%%%%%%%%%%%%%%%%%%%%%%%%%%%%%%%%%%%%
%\subsection{Feature Suggestions}
%
%The following is a list of features which may be useful for future
%versions of this package:
%%
%\begin{itemize}
%\item
%\ldots
%\end{itemize}

%%%%%%%%%%%%%%%%%%%%%%%%%%%%%%%%%%%%%%%%%%%%%%%%%%%%%%%%%%%%%%%%%%%%%%%%%%%%%%%%
\subsection{Revision History}

%%%%%%%%%%%%%%%%%%%%%%%%%%%%%%%%%%%%%%%%
\paragraph{v2.0:} 2018/12/30

\begin{itemize}
\item
immediate forward processing
\item
added |\childdocby| mechanism
\item
manual restructured
\end{itemize}

%%%%%%%%%%%%%%%%%%%%%%%%%%%%%%%%%%%%%%%%
\paragraph{v1.6:} 2018/01/17

\begin{itemize}
\item
application for development of include files
\item
corrections to manual
\end{itemize}

%%%%%%%%%%%%%%%%%%%%%%%%%%%%%%%%%%%%%%%%
\paragraph{v1.5:} 2017/05/21

\begin{itemize}
\item
more complete structuring introduced
\item
|\childdocof| introduced
\item
|\childdoc| renamed to |\childdocmain|
\item
|\childredirect| renamed to |\childdocforward| and |\childdocforwardprefix|
and functionality expanded
\end{itemize}

%%%%%%%%%%%%%%%%%%%%%%%%%%%%%%%%%%%%%%%%
\paragraph{v1.0:} 2017/04/27

\begin{itemize}
\item
manual and install package
\item
first version published on CTAN
\end{itemize}

%%%%%%%%%%%%%%%%%%%%%%%%%%%%%%%%%%%%%%%%
\paragraph{v0.6:} 2017/04/26

\begin{itemize}
\item
redirection mechanism added
\end{itemize}

%%%%%%%%%%%%%%%%%%%%%%%%%%%%%%%%%%%%%%%%
\paragraph{v0.5:} 2017/04/26

\begin{itemize}
\item
functionality in definition file
\end{itemize}


%%%%%%%%%%%%%%%%%%%%%%%%%%%%%%%%%%%%%%%%%%%%%%%%%%%%%%%%%%%%%%%%%%%%%%%%%%%%%%%%
%%%%%%%%%%%%%%%%%%%%%%%%%%%%%%%%%%%%%%%%%%%%%%%%%%%%%%%%%%%%%%%%%%%%%%%%%%%%%%%%
%%%%%%%%%%%%%%%%%%%%%%%%%%%%%%%%%%%%%%%%%%%%%%%%%%%%%%%%%%%%%%%%%%%%%%%%%%%%%%%%
\appendix

\settowidth\MacroIndent{\rmfamily\scriptsize 000\ }

 \DocInput{childdoc.dtx}

\end{document}
%</driver>
% \fi
%
% %%%%%%%%%%%%%%%%%%%%%%%%%%%%%%%%%%%%%%%%%%%%%%%%%%%%%%%%%%%%%%%%%%%%%%%%%%%%%%
% %%%%%%%%%%%%%%%%%%%%%%%%%%%%%%%%%%%%%%%%%%%%%%%%%%%%%%%%%%%%%%%%%%%%%%%%%%%%%%
% \section{Sample}
%\iffalse
%<*samplemain>
%\fi
%
% The following presents a sample document
% with two chapters, two parts, a title page,
% a compile flag as well as three forwarding files to set the flag.
% It consists of eight |.tex| files:
% \begin{center}
% \begin{tabular}{ll}
% |cdocsamp.tex|&main file\\
% |cdocsch1.tex|&include file for chapter 1\\
% |cdocsch2.tex|&include file for chapter 2\\
% |cdocspt3.tex|&include file for part 3\\
% |cdocspt4.tex|&include file for part 4\\
% |cdocsdrf.tex|&forwarding file for main file in draft mode\\
% |cdocsfi1.tex|&forwarding file for final version of chapter 1\\
% |cdocsfi2.tex|&forwarding file for final version of chapter 2\\
% \end{tabular}
% \end{center}
% Each of the eight files can be compiled directly by the \LaTeX{} compiler.
%
% %%%%%%%%%%%%%%%%%%%%%%%%%%%%%%%%%%%%%%
% \paragraph{Main File.}
%
% The main file is called |cdocsamp.tex|.
%
% Load the \textsf{childdoc} definitions and
% declare the filename for the main document:
%    \begin{macrocode}
\input{childdoc.def}
\childdocmain{}
%    \end{macrocode}

% Optional override for |\version| flag:
%    \begin{macrocode}
%%\ifchilddoc\else\providecommand{\version}{draft}\fi
%    \end{macrocode}

% Define the default values for the |\version| flag
% (|final| for the main file and |draft| for childs):
%    \begin{macrocode}
\ifchilddoc
\providecommand{\version}{draft}
\else
\providecommand{\version}{final}
\fi
%    \end{macrocode}

% Load the standard document class:
%    \begin{macrocode}
\documentclass[12pt]{article}
%    \end{macrocode}

% Start the document body:
%    \begin{macrocode}
\begin{document}
%    \end{macrocode}

% Declare a title page.
% Print title, part of document being processed and version flag:
%    \begin{macrocode}
\addtocounter{page}{-1}
\begin{center}
{\LARGE\bfseries{}childdoc example\par}
\vspace{1cm}
\ifchilddoc
\ifchilddocmanual part\else chapter\fi:
`\childdocname' of `\childdocjob'\par
\else
main document: `\childdocjob'\par
\fi
version: \version\par
\end{center}
\newpage
%    \end{macrocode}

% Manually include selected file,
% otherwise process as usual:
%    \begin{macrocode}
\ifchilddocmanual
\section*{part `\childdocname'}
\input{\childdocname}
\else
%    \end{macrocode}

% Include the two chapters:
%    \begin{macrocode}
\include{cdocsch1}
\include{cdocsch2}
%    \end{macrocode}

% Include the two parts unless only chapters should be displayed:
%    \begin{macrocode}
\ifchilddoc\else
\section{part three}
\input{cdocspt3}
\section{part four}
\input{cdocspt4}
\fi
%    \end{macrocode}

% Process as usual until here:
%    \begin{macrocode}
\fi
%    \end{macrocode}

% End of document body:
%    \begin{macrocode}
\end{document}
%    \end{macrocode}
%\iffalse
%</samplemain>
%\fi
%
% %%%%%%%%%%%%%%%%%%%%%%%%%%%%%%%%%%%%%%
% \paragraph{Chapter Include Files.}
%
% The include files are called |cdocsch1.tex| and |cdocsch2.tex|.
%
%\iffalse
%<*samplechap1|samplechap2>
%\fi

% Optional override for |\version| flag:
%    \begin{macrocode}
%%\providecommand{\version}{final}
%    \end{macrocode}

% Include the main document:
%    \begin{macrocode}
\input{childdoc.def}
\childdocof{cdocsamp}
%    \end{macrocode}

%\iffalse
%</samplechap1|samplechap2>
%\fi
%
%\iffalse
%<*samplechap1>
%\fi
% Some text for chapter 1:
%    \begin{macrocode}
\section{one}
some text in chapter one
%    \end{macrocode}

%\iffalse
%</samplechap1>
%\fi
% Some text for chapter 2:
%\iffalse
%<*samplechap2>
%\fi
%    \begin{macrocode}
\section{two}
more text in chapter two
%    \end{macrocode}

%\iffalse
%</samplechap2>
%\fi
%
% %%%%%%%%%%%%%%%%%%%%%%%%%%%%%%%%%%%%%%
% \paragraph{Part Include Files.}
%
% The include files are called |cdocspt3.tex| and |cdocspt4.tex|.
%
%\iffalse
%<*samplepart3|samplepart4>
%\fi

% Optional override for |\version| flag:
%    \begin{macrocode}
%%\providecommand{\version}{final}
%    \end{macrocode}

% Include the main document:
%    \begin{macrocode}
\input{childdoc.def}
\childdocby{cdocsamp}
%    \end{macrocode}

%\iffalse
%</samplepart3|samplepart4>
%\fi
%
%\iffalse
%<*samplepart3>
%\fi
% Some text for part 3:
%    \begin{macrocode}
some text in part three
%    \end{macrocode}

%\iffalse
%</samplepart3>
%\fi
% Some text for part 4:
%\iffalse
%<*samplepart4>
%\fi
%    \begin{macrocode}
more text in part four
%    \end{macrocode}

%\iffalse
%</samplepart4>
%\fi
%
% %%%%%%%%%%%%%%%%%%%%%%%%%%%%%%%%%%%%%%
% \paragraph{Forwarding for a Complete Draft.}
%
% The following forwarding file |cdocsdrf.tex|
% compiles the main document in draft mode:
%\iffalse
%<*sampledraft>
%\fi
%    \begin{macrocode}
\def\version{draft}
\input{childdoc.def}
\childdocforward{cdocsamp}
%    \end{macrocode}

%\iffalse
%</sampledraft>
%\fi
%
% %%%%%%%%%%%%%%%%%%%%%%%%%%%%%%%%%%%%%%
% \paragraph{Forwarding for Final Version of the Chapters.}
%
% The following forwarding files |cdocsfn1.tex| and |cdocsfn2.tex|
% (with identical content)
% compile the final versions of the child documents
% |cdocsch1.tex| and |cdocsch2.tex|, respectively:
%\iffalse
%<*samplefinal>
%\fi
%    \begin{macrocode}
\def\version{final}
\input{childdoc.def}
\childdocforwardprefix[cdocsamp]{cdocsfn}{cdocsch}
%    \end{macrocode}

%\iffalse
%</samplefinal>
%\fi
%
% %%%%%%%%%%%%%%%%%%%%%%%%%%%%%%%%%%%%%%
% \paragraph{Command Line Processing.}
%
% The following three command lines generate the output files
% |cdocscld|, |cdocscl1| and |cdocscl2|
% which should be identical to
% |cdocsdrf|, |cdocsch1| and |cdocsfn2|, respectively:
% \begin{center}
% \begin{tabular}{l}
% |latex -jobname cdocscld \|\\
% |  "\def\version{draft}\input{childdoc.def}\childdocforward{cdocsamp}"|\\
% |latex -jobname cdocscl1 \|\\
% |  "\input{childdoc.def}\childdocforward[cdocsamp]{cdocsch1}"|\\
% |latex -jobname cdocscl2 \|\\
% |  "\def\version{final}\input{childdoc.def}\childdocforward{cdocsch2}"|
% \end{tabular}
% \end{center}
% Note that the trailing backslash on each first line
% merely continues the input to the second line
% (for convenient cut ant paste).
% Furthermore, the command |latex| can be replaced by any
% of its alternative versions such as |pdflatex|.
%
% %%%%%%%%%%%%%%%%%%%%%%%%%%%%%%%%%%%%%%%%%%%%%%%%%%%%%%%%%%%%%%%%%%%%%%%%%%%%%%
% %%%%%%%%%%%%%%%%%%%%%%%%%%%%%%%%%%%%%%%%%%%%%%%%%%%%%%%%%%%%%%%%%%%%%%%%%%%%%%
% \section{Implementation}
%\iffalse
%<*package>
%\fi
%
% This section describes the definitions file |childdoc.def|.

% The definitions cannot be loaded using |\usepackage| or |\RequirePackage|
% which has a mechanism to prevent loading a style file more than once.
% When loading the definitions by means of |\input|
% multiple instances have to be prevented manually:
%\iffalse
%This code needs to be before the `\ProvidesFile' directive
%which is defined at the beginning of this file.
%Therefore it is also placed there and commented out here.
%</package>
%<*discard>
%\fi
%    \begin{macrocode}
\ifdefined\childdocmain\endinput\fi
%    \end{macrocode}
%\iffalse
%</discard>
%<*package>
%\fi
%
% \macro{\ifchilddoc}
% \macro{\ifchilddocmanual}
% The conditional |\ifchilddoc| tells whether a
% child (true) or main (false) document is being compiled.
% The conditional |\ifchilddocmanual| tells whether
% the |\includeonly| mechanism is used (false) or
% the selection of child files must be performed manually (true).
% The definitions initialise to false:
%    \begin{macrocode}
\newif\ifchilddoc
\newif\ifchilddocmanual
%    \end{macrocode}

% \macro{\childdocname}
% \macro{\childdocjob}
% The macro |\childdocname| stores the name of the main document
% to be compiled. The macro |\childdocjob| stores the name of
% the document on which the \LaTeX{} compiler was originally invoked.
% The content of |\jobname| cannot be compared
% to filenames specified in the source due to different catcodes.
% The following code rescans |\jobname|, stores the result
% in |\childdocname| and saves a copy in |\childdocjob|:
%    \begin{macrocode}
\edef\childdocname{\scantokens\expandafter{\jobname\noexpand}}
\let\childdocjob\childdocname
%    \end{macrocode}

% \macro{\childdocdisable}
% The macro |\childdocdisable| prevents the main file
% from being processed more than once.
% At this stage, the main document command |\childdocmain|
% is assumed to be called once again where it should do nothing.
% Any subsequent call to it should prevent
% a secondary processing of the main document
% It overwrites the forwarding commands
% |\childdocof| and |\childdocforward|
% with empty macros to prevent further inclusions of the main document:
%    \begin{macrocode}
\newcommand{\childdocdisable}
{
  \renewcommand{\childdocmain}[1]{\renewcommand{\childdocmain}[1]{\endinput}}
  \renewcommand{\childdocof}[1]{}
  \renewcommand{\childdocby}[2][]{}
  \renewcommand{\childdocforward}[2][]{}
  \renewcommand{\childdocdisable}{}
}
%    \end{macrocode}

% \macro{\childdocmain}
% The macro |\childdocmain| is to be called at the top of the main file
% with nothing or the main filename (without extension) as argument.
% First, it breaks loops.
% If the argument is not empty and does not match |\childdocname|
% (which is set by the first inclusion of |childdoc.def|),
% |\ifchilddoc| is set to true, |\includeonly| is applied to the child file
% and |\jobname| is set to the main file
% (for proper handling of |.aux| files):
%    \begin{macrocode}
\newcommand{\childdocmain}[1]
{
  \childdocdisable\childdocmain{}
  \if?#1?\else
    \begingroup
      \def\childdoctmp{#1}
      \ifx\childdoctmp\childdocname
        \def\childdoctmp{}
      \else
        \def\childdoctmp
        {
          \childdoctrue
          \includeonly{\childdocname}
          \def\childdocjob{#1}
          \def\jobname{#1}
        }
      \fi
      \expandafter
    \endgroup
    \childdoctmp
  \fi
}
%    \end{macrocode}

% \macro{\childdocof}
% The command |\childdocof| redirects
% compilation to the main file |#1|.
%    \begin{macrocode}
\newcommand{\childdocof}[1]
{
  \childdocdisable
  \childdoctrue
  \includeonly{\childdocname}
  \def\jobname{#1}
  \def\childdocjob{#1}
  \input{#1}
}
%    \end{macrocode}

% \macro{\childdocby}
% The command |\childdocby| ....
%    \begin{macrocode}
\newcommand{\childdocby}[2][]
{
  \childdocdisable
  \childdoctrue
  \childdocmanualtrue
  \if?#1?\else
    \def\jobname{#2}
  \fi
  \def\childdocjob{#2}
  \input{#2}
  \endinput
}
%    \end{macrocode}

% \macro{\childdocforward}
% The command |\childdocforward| redirects
% compilation to the main file or
% (if the optional argument is given) a child file.
% Parameters are set as if the main file
% or a child file starting with |\childdocof| was compiled.
% Then compilation is handed over to the main file:
%    \begin{macrocode}
\newcommand{\childdocforward}[2][]
{
  \begingroup
    \if?#1?
      \def\childdoctmp
      {
        \def\childdocname{#2}
        \def\childdocjob{#2}
        \def\jobname{#2}
        \input{#2}
        \endinput
      }
    \else
      \def\childdoctmp
      {
        \childdocdisable
        \def\childdocname{#2}
        \childdoctrue
        \includeonly{#2}
        \def\childdocjob{#1}
        \def\jobname{#1}
        \input{#1}
        \endinput
      }
    \fi
    \expandafter
  \endgroup
  \childdoctmp
}
%    \end{macrocode}

% \macro{\childdocforwardprefix}
% The command |\childdocforwardprefix| redirects
% compilation to the main or a child file by means of a pattern.
% The prefix |#1| in the current filename is replaced by |#2|
% and the suffix of the current filename is kept
% (it is assumed that the filename does not contain the substring `|~~~|'
% which is used as a delimiter).
% Compilation is handed over to the new file by |\childdocforward|:
%    \begin{macrocode}
\newcommand{\childdocforwardprefix}[3][]
{
  \begingroup
    \def\childdocextract #2##1~~~{\def\childdoctmp{\childdocforward[#1]{#3##1}}}
    \expandafter\childdocextract\childdocname~~~
    \expandafter
  \endgroup
  \childdoctmp
}
%    \end{macrocode}

% \macro{\childdoc}
% The deprecated macro |\childdoc| is a legacy version of |\childdocmain|:
%    \begin{macrocode}
\newcommand{\childdoc}{\childdocmain}
%    \end{macrocode}

% \macro{\childdocredirect}
% The deprecated macro |\childdocredirect| is a legacy version
% of |\childdocforward| and |\childdocforwardprefix|:
%    \begin{macrocode}
\newcommand{\childdocredirect}[2][]
{
  \begingroup
    \if?#1?
      \def\childdoctmp{\childdocforward{#2}}
    \else
      \def\childdoctmp{\childdocforwardprefix{#1}{#2}}
    \fi
    \expandafter
  \endgroup
  \childdoctmp
}
%    \end{macrocode}

%\iffalse
%</package>
%\fi
%
\endinput

\childdocforwardprefix[cdocsamp]{cdocsfn}{cdocsch}
%    \end{macrocode}

%\iffalse
%</samplefinal>
%\fi
%
% %%%%%%%%%%%%%%%%%%%%%%%%%%%%%%%%%%%%%%
% \paragraph{Command Line Processing.}
%
% The following three command lines generate the output files
% |cdocscld|, |cdocscl1| and |cdocscl2|
% which should be identical to
% |cdocsdrf|, |cdocsch1| and |cdocsfn2|, respectively:
% \begin{center}
% \begin{tabular}{l}
% |latex -jobname cdocscld \|\\
% |  "\def\version{draft}% \iffalse
%
% childdoc.dtx Copyright (C) 2017-2018 Niklas Beisert
%
% This work may be distributed and/or modified under the
% conditions of the LaTeX Project Public License, either version 1.3
% of this license or (at your option) any later version.
% The latest version of this license is in
%   http://www.latex-project.org/lppl.txt
% and version 1.3 or later is part of all distributions of LaTeX
% version 2005/12/01 or later.
%
% This work has the LPPL maintenance status `maintained'.
%
% The Current Maintainer of this work is Niklas Beisert.
%
% This work consists of the files childdoc.dtx and childdoc.ins
% and the derived files childdoc.def and cdocsamp.tex with
% cdocsch1.tex, cdocsch2.tex, cdocsdrf.tex, cdocsfn1.tex, cdocsfn2.tex.
%
%<package>\ifdefined\childdocmain\endinput\fi
%<package>\ProvidesFile{childdoc.def}[2018/12/30 v2.0 child document driver]
%<samplemain>\ProvidesFile{cdocsamp.tex}[2018/12/30 v2.0 sample for childdoc]
%<*driver>
%\ProvidesFile{childdoc.drv}[2018/12/30 v2.0 childdoc reference manual file]
\PassOptionsToClass{10pt,a4paper}{article}
\documentclass{ltxdoc}

\usepackage[margin=35mm]{geometry}
\usepackage{hyperref}
\usepackage{hyperxmp}
\usepackage[usenames]{color}

\hypersetup{colorlinks=true}
\hypersetup{pdfstartview=FitH}
\hypersetup{pdfpagemode=UseNone}
\hypersetup{pdfsource={}}
\hypersetup{pdflang={en-UK}}
\hypersetup{pdfcopyright={Copyright 2017-2018 Niklas Beisert.
  This work may be distributed and/or modified under the
  conditions of the LaTeX Project Public License, either version 1.3
  of this license or (at your option) any later version.}}
\hypersetup{pdflicenseurl={http://www.latex-project.org/lppl.txt}}
\hypersetup{pdfcontactaddress={ETH Zurich, ITP, HIT K,
  Wolfgang-Pauli-Strasse 27}}
\hypersetup{pdfcontactpostcode={8093}}
\hypersetup{pdfcontactcity={Zurich}}
\hypersetup{pdfcontactcountry={Switzerland}}
\hypersetup{pdfcontactemail={nbeisert@itp.phys.ethz.ch}}
\hypersetup{pdfcontacturl={http://people.phys.ethz.ch/\xmptilde nbeisert/}}

\newcommand{\secref}[1]{\hyperref[#1]{section \ref*{#1}}}

\parskip1ex
\parindent0pt
\let\olditemize\itemize
\def\itemize{\olditemize\parskip0pt}

\begin{document}

\title{The \textsf{childdoc} Package}
\hypersetup{pdftitle={The childdoc Package}}
\author{Niklas Beisert\\[2ex]
  Institut f\"ur Theoretische Physik\\
  Eidgen\"ossische Technische Hochschule Z\"urich\\
  Wolfgang-Pauli-Strasse 27, 8093 Z\"urich, Switzerland\\[1ex]
  \href{mailto:nbeisert@itp.phys.ethz.ch}
  {\texttt{nbeisert@itp.phys.ethz.ch}}}
\hypersetup{pdfauthor={Niklas Beisert}}
\hypersetup{pdfsubject={Manual for the LaTeX2e Package childdoc}}
\date{30 December 2018, \textsf{v2.0}}
\maketitle

\begin{abstract}\noindent
\textsf{childdoc} is a \LaTeXe{} package
that enables the direct compilation
of document sections included by |\include|
to individual files.
\end{abstract}

\begingroup
\parskip0ex
\tableofcontents
\endgroup

%%%%%%%%%%%%%%%%%%%%%%%%%%%%%%%%%%%%%%%%%%%%%%%%%%%%%%%%%%%%%%%%%%%%%%%%%%%%%%%%
%%%%%%%%%%%%%%%%%%%%%%%%%%%%%%%%%%%%%%%%%%%%%%%%%%%%%%%%%%%%%%%%%%%%%%%%%%%%%%%%
\section{Introduction}

\LaTeX{} provides a mechanism to structure a large document (such as a book)
into a main file and several child files (containing the chapters)
using the |\include| command.
This mechanism is beneficial for documents
which span hundreds of pages in order to
make the source file(s) more manageable.
Moreover, compilation can be restricted to
selected child files by means of the |\includeonly| command.
The latter feature can be used to reduce the compilation time while editing
(this was significantly more useful in the earlier days of \LaTeX{})
or to generate a smaller document which is easier to navigate.
Another application of |\includeonly| is to generate
documents consisting of selected parts of the complete document.

However, there are a few drawbacks of the plain |\include| mechanism:
\begin{itemize}
\item
The child files cannot be compiled on their own,
they can only be compiled via the main file.
A naive editing environment
(such as a text editor with an option
to have the current file processed by \LaTeX)
may require one to switch to the main file before compiling;
attempting to compile the child file produces errors.
\item
The main file must be modified (each time)
to adjust the |\includeonly| command
to the present needs. This easily leaves the main file in a messy state.
\item
The generated document will always carry the filename
of the main document. This is inconvenient if
several child files are to be compiled and
to be kept for distribution.
\end{itemize}

The present package provides a simple interface
to make child files individually compilable by \LaTeX{}.
Compiling a child file then has the same effect as compiling
the main file with an |\includeonly| command
to select the appropriate child.
Moreover the generated document will carry the name of the child
rather than the main file.
This resolves all three above issues.

This feature is meant to make the editing of books,
thesis documents and lecture notes somewhat more convenient.
However, the package can also be used efficiently for
composing a series of documents (such as exercise sheets)
which are typically distributed individually.
It then assists the author in generating the individual documents
(potentially in different versions)
as well as a document containing the collected series.
Another application is in developing style files
or other kinds of included material
where compilation of the style file could redirect
to a sample or test file.

%%%%%%%%%%%%%%%%%%%%%%%%%%%%%%%%%%%%%%%%%%%%%%%%%%%%%%%%%%%%%%%%%%%%%%%%%%%%%%%%
%%%%%%%%%%%%%%%%%%%%%%%%%%%%%%%%%%%%%%%%%%%%%%%%%%%%%%%%%%%%%%%%%%%%%%%%%%%%%%%%
\section{Usage}

First of all, the package \textsf{childdoc} is \emph{not} a standard
\LaTeXe{} |.sty| style file! Therefore it needs to be invoked in
a non-standard way.

%%%%%%%%%%%%%%%%%%%%%%%%%%%%%%%%%%%%%%%%%%%%%%%%%%%%%%%%%%%%%%%%%%%%%%%%%%%%%%%%
\subsection{Included Files}
\label{sec:include}

%%%%%%%%%%%%%%%%%%%%%%%%%%%%%%%%%%%%%%%%
\DescribeMacro{\childdocmain}
To use the package, add the commands
\begin{center}
\begin{tabular}{l}
|\input{childdoc.def}|\\
|\childdocmain{}|\\
\end{tabular}
\end{center}
at the very top of the main \LaTeX{} file,
in particular \emph{before} the |\documentclass| statement!
The argument of |\childdocmain| should be left empty
(but it must be present).

%%%%%%%%%%%%%%%%%%%%%%%%%%%%%%%%%%%%%%%%
\DescribeMacro{\childdocof}
Furthermore, add the commands
\begin{center}
\begin{tabular}{l}
|\input{childdoc.def}|\\
|\childdocof{|\textit{main}|}|\\
\end{tabular}
\end{center}
at the top of every child file \textit{child}
which is included by |\include{|\textit{child}|}|
from within the main file
(or at least for those files to be compiled individually).
The argument \textit{main} must be the filename of the main file.

There are a couple of
considerations in setting up the main and child documents:

%%%%%%%%%%%%%%%%%%%%%%%%%%%%%%%%%%%%%%%%
\paragraph{Restrictions.}

Please note the following restrictions:
\begin{itemize}
\item
|\childdocmain| must be called with one argument \textit{main}
to ensure compatibility with earlier version of the package.
It must either be empty (|\childdocmain{}|)
or precisely match the filename of the main file in which it is specified.
See \secref{sec:detection} for further information.
\item
The filename \textit{main} must be specified without the |.tex| extension.
\item
The filename \textit{main} is case sensitive
(even in case-insensitive file systems)
due to internal string comparison.
\item
The argument \textit{main} should be fully expanded, it cannot be a macro.
\item
Subdirectories and special characters should be avoided in filenames.
\item
The command |\childdocmain{|\textit{main}|}| must be followed by a whitespace.
It should not be followed immediately by another command
or by a comment mark `|%|'.
This is because the \TeX{} parser reads the token immediately following
the argument of |\childdocmain| and puts it
at the beginning of every child section;
however, a white\-space is ignored.
\end{itemize}

%%%%%%%%%%%%%%%%%%%%%%%%%%%%%%%%%%%%%%%%
\paragraph{Content of Main File.}

It is advisable to place all content in the child files included by |\include|.
Any output contained in the main file will appear in all child documents
unless suppressed manually;
it cannot be suppressed automatically by the |\includeonly| directive
and thus should normally be avoided.
A method to include some content in the main file
by means of conditional processing is described in \secref{sec:conditional}.

%%%%%%%%%%%%%%%%%%%%%%%%%%%%%%%%%%%%%%%%
\paragraph{Page Numbering.}

When only a part of the document is compiled,
the appropriate numbering of pages
(as well as other status parameters)
is determined from the |.aux| files.
The latter contain information from previous passes.
However this information needs to propagate through
all intermediate child documents.
Therefore the page numbering in child documents may well
be inconsistent until the complete document is compiled at least once.

A useful (if unconventional) way to always ensure a consistent
page numbering is to restart the numbering in each child document
and denote the pages by `\textit{child}|.|\textit{page}'
where \textit{child} represents the chapter/section number of the child file.
This can be achieved by the command
|\numberwithin{page}{|\textit{child}|}|
of the \textsf{amsmath} package
where \textit{child} can be |chapter| or |section|
depending on the chosen structuring.
Alternatively, one can modify the macro |\thepage| appropriately
and reset the counter |page| at the start of each child file.

%%%%%%%%%%%%%%%%%%%%%%%%%%%%%%%%%%%%%%%%%%%%%%%%%%%%%%%%%%%%%%%%%%%%%%%%%%%%%%%%
\subsection{Conditional Processing}
\label{sec:conditional}

The package provides a mechanism to compile different versions
of a document. To customise the versions further some conditional processing
can come in handy to distinguish which version is being compiled.
The package provides two macros to describe the compilation context:

%%%%%%%%%%%%%%%%%%%%%%%%%%%%%%%%%%%%%%%%
\DescribeMacro{\ifchilddoc}
The conditional |\ifchilddoc| distinguishes between the compilation of
child documents and the main document:
%
\begin{center}
|\ifchilddoc |\textit{child-code}| |[|\||else |\textit{main-code}]| \||fi|
\end{center}

%%%%%%%%%%%%%%%%%%%%%%%%%%%%%%%%%%%%%%%%
\DescribeMacro{\childdocname}
\DescribeMacro{\childdocjob}
The macro |\childdocname| contains the filename (without extension)
of the main or child file being processed.
Note that |\childdocjob| will always contain the name of the main file.

%%%%%%%%%%%%%%%%%%%%%%%%%%%%%%%%%%%%%%%%
\paragraph{Title Page.}

Conditional processing can be used to include a title or banner page
in the main document when proper precautions are taken.
Importantly, the code in the main file should ensure that the page counter
(as well as other status parameters which are stored in the |.aux| files)
takes the same value after the conditional processing.
Otherwise the page numbers may take divergent values
depending on which part is compiled.

For example, a title page could be declared by:
%
\begin{center}
\begin{tabular}{l}
|\ifchilddoc\||else|\\
|\addtocounter{page}{-1}|\\
\textit{code for title page}\\
|\newpage|\\
|\||fi|
\end{tabular}
\end{center}
%
A banner page for the child documents can be generated by:
%
\begin{center}
\begin{tabular}{l}
|\ifchilddoc|\\
|\addtocounter{page}{-1}|\\
\textit{code for banner page}\\
|\newpage|\\
|\||fi|
\end{tabular}
\end{center}
%
Here one could write a message such as:
\begin{center}
|This is the part \childdocname{} of \childdocjob{}.|
\end{center}

%%%%%%%%%%%%%%%%%%%%%%%%%%%%%%%%%%%%%%%%%%%%%%%%%%%%%%%%%%%%%%%%%%%%%%%%%%%%%%%%
\subsection{Flags}
\label{sec:flags}

The package makes it easy to generate different versions
of the main or child documents.
To this end compilation flags can be defined
and assigned different default values.
They will be particularly useful in conjunction
with the forwarding mechanism described in \secref{sec:forward}.

For example, it may be useful to have a flag |\version|
which can be set to |draft| or |final|.
The document source will contain some conditional code
depending on the value of |\version|.
Suppose further, the flag should default to |final| for the main file
and to |draft| for child files
which is a natural assignment for editing the document.
This is achieved by placing the following code
in the preamble of the main document
(below the |\childdocmain| directive):
%
\begin{center}
\begin{tabular}{l}
|\ifchilddoc|\\
|\providecommand{\version}{draft}|\\
|\||else|\\
|\providecommand{\version}{final}|\\
|\||fi|
\end{tabular}
\end{center}
%
The definition by |\providecommand| makes sure
that previous definitions are not overwritten.
Further statements |\providecommand{\version}{...}|
can thus be added before the above code to override it.

For the main file, one might add a line
(between |\childdocmain| and the above block)
%
\begin{center}
|%\ifchilddoc\||else\providecommand{\version}{draft}\||fi|
\end{center}
%
which can be uncommented to produce a draft version.
Likewise one can add a line to the very top of a child file
(above the |\childdocof{|\textit{main}|}| directive)
%
\begin{center}
|%\providecommand{\version}{final}|
\end{center}
%
which can be uncommented to produce the final version of this child document.

%%%%%%%%%%%%%%%%%%%%%%%%%%%%%%%%%%%%%%%%%%%%%%%%%%%%%%%%%%%%%%%%%%%%%%%%%%%%%%%%
\subsection{Forwarding}
\label{sec:forward}

Different versions of the main or child documents
using compilation flags as described in \secref{sec:flags}
can be (permanently) stored in different files
for convenient compilation, viewing and distribution.
To this end, the package defines a command
to pass on compilation to a different file:

%%%%%%%%%%%%%%%%%%%%%%%%%%%%%%%%%%%%%%%%
\DescribeMacro{\childdocforward}
The command |\childdocforward| redirects processing to
another source file:
%
\begin{center}
\begin{tabular}{l}
|\input{childdoc.def}|\\
|\childdocforward[|\textit{main}|]{|\textit{dest}|}|\\
\end{tabular}
\end{center}
%
The argument \textit{dest} is the destination file
(without extension).
It should be the main file or one of the child files.
Note that further \textsf{childdoc} directives
such as |\childdocof| and |\childdocforward|
in the indicated file will be processed in this form.
The optional argument \textit{main}
passes on directly to the main file \textit{main}
while pretending to compile the child \textit{dest}.
This form behaves as if \textit{dest}
issues |\childdocof{|\textit{main}|}| right away,
and no further \textsf{childdoc} directives will be processed.

%%%%%%%%%%%%%%%%%%%%%%%%%%%%%%%%%%%%%%%%
\DescribeMacro{\...prefix}
In the alternative form |\childdocforwardprefix|,
%
\begin{center}
\begin{tabular}{l}
|\input{childdoc.def}|\\
|\childdocforwardprefix[|\textit{main}|]{|\textit{prefix}|}{|\textit{dest}|}|
\end{tabular}
\end{center}
%
the destination file is determined by a pattern
depending on the current file:
To make this work, the current file must be called
`{\textit{prefix}\hspace{0.2em}\textit{suffix}}'
with \textit{prefix} matching precisely the argument.
Processing is then passed on to the file
`{\textit{dest}\hspace{0.2em}\textit{suffix}}'.
Surely, the same effect is achieved by
directly specifying the
argument `{\textit{dest}\hspace{0.2em}\textit{suffix}}'
in the first form.
However, that requires to set up a different file
for each child. With the alternative form of the command
all these files can have exactly the same content
which simplifies setting them up and maintaining them.

For example, the following file |draft.tex|
with a compilation flag |\version| as described in \secref{sec:flags}
compiles the main document as a draft:
%
\begin{center}
\begin{tabular}{l}
|\def\version{draft}|\\
|\input{childdoc.def}|\\
|\childdocforward{|\textit{main}|}|
\end{tabular}
\end{center}
%
Likewise, the following files |final|\textit{nn}|.tex|
compile the final version of the child document
|child|\textit{nn}|.tex|:
%
\begin{center}
\begin{tabular}{l}
|\def\version{final}|\\
|\input{childdoc.def}|\\
|\childdocforwardprefix{final}{child}|
\end{tabular}
\end{center}
%

Note that when several versions of a main file and/or of each child file
are to be generated, it may be convenient to set up a |Makefile| or
shell script to automatise the process.

%%%%%%%%%%%%%%%%%%%%%%%%%%%%%%%%%%%%%%%%%%%%%%%%%%%%%%%%%%%%%%%%%%%%%%%%%%%%%%%%
\subsection{Command Line Processing}
\label{sec:commandline}

The effect of redirection files can also be achieved by invoking
the \LaTeX{} compiler with a more elaborate command line.
Most conveniently this should be done as part
of a shell script or a |Makefile|.

When using \textsf{childdoc} in the main file, the following
command lines effectively perform a redirection
(note that depending on the shell being used,
backslashes may have to be doubled: `|\|' $\to$ `|\\|'):
%
\begin{center}
|... -jobname "|\textit{target}|" |\\|"|[\textit{flags}]%
|\input{childdoc.def}\childdocforward[|\textit{main}|]{|\textit{dest}|}"|
\end{center}
%
Here \textit{target} is the name of the output file,
\textit{main} is the name of the main file
and \textit{dest} is the name of the main or child file to be processed
(all filenames without extensions).
The optional argument \textit{main} can be omitted
if \textit{main} matches \textit{dest}.
Optionally, compilation \textit{flags} can be defined via |\def| commands.
This command line makes the \TeX{} engine believe
it is compiling the file \textit{target}
whose content is specified as the latter parameter.
The provided code then forwards the processing to
\textit{main} or \textit{dest} as described in \secref{sec:forward}.

%%%%%%%%%%%%%%%%%%%%%%%%%%%%%%%%%%%%%%%%%%%%%%%%%%%%%%%%%%%%%%%%%%%%%%%%%%%%%%%%
\subsection{Include by Input}
\label{sec:input}

Including child documents by |\include| has some restrictions by design.
Most notably, the content of a child document always occupies
its own set of pages; pages cannot be shared between child documents.
Usually, this behaviour makes perfect sense
because each child document contain an essential part of the document.
However, in some situations it may be desirable to compose
a document from a collection of parts
without having mandatory page breaks between then.
For this case, the package
provides a mechanism to include parts
by |\input| which can also be processed individually.
However, by construction this mechanism
requires manual handling of the content to be output.

%%%%%%%%%%%%%%%%%%%%%%%%%%%%%%%%%%%%%%%%
\DescribeMacro{\ifchilddocmanual}
The main file should be prepared as usual, see \secref{sec:include}.
However, the document body must make a distinction
between processing of an individual part and of the main document, e.g.:
%
\begin{center}
\begin{tabular}{l}
|\ifchilddocmanual|\\
|\input{\childdocname}|\\
|\||else|\\
\textit{document body with }|\input{|\textit{part}|}|\\
|\||fi|
\end{tabular}
\end{center}
%
The conditional |\ifchilddocmanual| is true whenever
a part to be included by |\input| is being compiled,
and the name of the part is stored in |\childdocname|.

%%%%%%%%%%%%%%%%%%%%%%%%%%%%%%%%%%%%%%%%
\DescribeMacro{\childdocby}
Each part to be included by |\input| should start with:
%
\begin{center}
\begin{tabular}{l}
|\input{childdoc.def}|\\
|\childdocby{|\textit{main}|}|\\
\end{tabular}
\end{center}
%
The directive |\childdocby| is similar to |\childdocof|
described in \secref{sec:include},
but the subsequent selection of content must be done manually.
To that end, both |\ifchilddoc| and |\ifchilddocmanual|
will be true upon processing of a part,
and the name of the part is stored in |\childdocname|.
Note that |\jobname| will be set to the filename of the current part
so that each part receives an individual |.aux| file
that does not interfere with the |.aux| file(s) of the main document.
This behaviour can be altered by the alternative form
|\childdocby[*]{|\textit{main}|}| (with a non-empty optional argument)
which uses the |.aux| file of the main document
by setting |\jobname| to \textit{main}.

%%%%%%%%%%%%%%%%%%%%%%%%%%%%%%%%%%%%%%%%%%%%%%%%%%%%%%%%%%%%%%%%%%%%%%%%%%%%%%%%
\subsection{Driver Development}
\label{sec:driver}

The \textsf{childdoc} mechanism can also be use for the development
of definition files such as \LaTeX{} styles or classes.
This case differs from the above setup with multiple parts
included by |\include| in that no |\includeonly| should be invoked.
This can be achieved by starting the include file
(before |\ProvidesPackage|) with:
%
\begin{center}
\begin{tabular}{l}
|\input{childdoc.def}|\\
|\childdocforward{|\textit{main}|}|\\
\end{tabular}
\end{center}
%
or alternatively with:
%
\begin{center}
\begin{tabular}{l}
|\input{childdoc.def}|\\
|\childdocby{|\textit{main}|}|\\
\end{tabular}
\end{center}
%
Both forms have slightly different effects as described above.
The main file is prepared as usual, see \secref{sec:include}.

%%%%%%%%%%%%%%%%%%%%%%%%%%%%%%%%%%%%%%%%%%%%%%%%%%%%%%%%%%%%%%%%%%%%%%%%%%%%%%%%
\subsection{Legacy Detection}
\label{sec:detection}

The directive |\childdocmain| in the main file can detect
whether the complete document or merely a child is to be compiled
even without using the directive |\childdocof|.
This method is deprecated because it is less robust
and there is no compelling reason to use it;
it is merely provided for backward compatibility
and it may be removed in future versions.

If the detection mechanism is to be used,
it is mandatory to correctly specify
the filename of the main file as the argument of |\childdocmain|:
%
\begin{center}
\begin{tabular}{l}
|\input{childdoc.def}|\\
|\childdocmain{|\textit{main}|}|\\
\end{tabular}
\end{center}
%
If |\jobname| does not match the argument \textit{main} of |\childdocmain|,
it is assumed that |\jobname| points to the child file to be compiled.
When using |\childdocmain| with the main file specified as argument,
it suffices to start a child file
with just |\input{|\textit{main}|}|
without loading of the package and using |\childdocof|.
If instead all processing is done
with the appropriate \textsf{childdoc} directives,
the argument of \textit{main} of |\childdocmain| can be empty.

An alternative version of the command line processing described
in \secref{sec:commandline} using the detection mechanism reads:
%
\begin{center}
|... -jobname "|\textit{target}|" "|[\textit{flags}]%
[|\def\jobname{|\textit{dest}|}|]|\input{|\textit{main}|}"|
\end{center}

%%%%%%%%%%%%%%%%%%%%%%%%%%%%%%%%%%%%%%%%%%%%%%%%%%%%%%%%%%%%%%%%%%%%%%%%%%%%%%%%
\subsection{Manual Code}
\label{sec:manual}

In case one cannot be certain whether the definitions file |childdoc.def|
is installed on the target \TeX{} distribution
and one prefers not to ship it,
it is conceivable to paste a few relevant commands into the sources.

To that end, drop all statements |\input{childdoc.def}|
and perform the replacements as outlined below.
Instead of |\childdocmain{|\textit{main}|}| add the following code
to the top of the main file:
%
\begin{center}
\begin{tabular}{l}
|\||ifdefined\childdocname\endinput\||fi\newif\ifchilddoc|\\
|\edef\childdocname{\scantokens\expandafter{\jobname\noexpand}}|\\
|\def\childdocmain{|\textit{main}|}\||ifx\childdocmain\childdocname\||else|\\
|\childdoctrue\includeonly{\childdocname}\let\jobname\childdocmain\||fi|\\
\end{tabular}
\end{center}
%
Instead of |\childdocof{|\textit{main}|}| just include the main file
at the top of each child file:
%
\begin{center}
|\input{|\textit{main}|}|
\end{center}
%
A simple redirection |\childdocforward{|\textit{dest}|}| is achieved by:
%
\begin{center}
|\def\jobname{|\textit{dest}|}\input{\jobname}|
\end{center}
%
The redirection with prefix
|\childdocforwardprefix[|\textit{prefix}|]{|\textit{dest}|}|
is accomplished by:
%
\begin{center}
\begin{tabular}{l}
|{\edef\jobname{\scantokens\expandafter{\jobname\noexpand}}|\\
|\def\redirectjob |\textit{prefix}|#1~~~{\gdef\jobname{|\textit{dest}|#1}}|\\
|\expandafter\redirectjob\jobname~~~}\input{\jobname}|
\end{tabular}
\end{center}

In an alternative approach,
child documents can be compiled by a specific command line
without additional code or specific definitions:
%
\begin{center}
|... -jobname "|\textit{target}|" "|[\textit{flags}]%
|\includeonly{|\textit{dest}|}\input{|\textit{main}|}"|
\end{center}
%

%%%%%%%%%%%%%%%%%%%%%%%%%%%%%%%%%%%%%%%%%%%%%%%%%%%%%%%%%%%%%%%%%%%%%%%%%%%%%%%%
%%%%%%%%%%%%%%%%%%%%%%%%%%%%%%%%%%%%%%%%%%%%%%%%%%%%%%%%%%%%%%%%%%%%%%%%%%%%%%%%
\section{Information}

%%%%%%%%%%%%%%%%%%%%%%%%%%%%%%%%%%%%%%%%%%%%%%%%%%%%%%%%%%%%%%%%%%%%%%%%%%%%%%%%
\subsection{Copyright}

Copyright \copyright{} 2017--2018 Niklas Beisert

This work may be distributed and/or modified under the
conditions of the \LaTeX{} Project Public License, either version 1.3
of this license or (at your option) any later version.
The latest version of this license is in
  \url{http://www.latex-project.org/lppl.txt}
and version 1.3 or later is part of all distributions of \LaTeX{}
version 2005/12/01 or later.

This work has the LPPL maintenance status `maintained'.

The Current Maintainer of this work is Niklas Beisert.

This work consists of the files |README.txt|, |childdoc.ins| and |childdoc.dtx|
as well as the derived files |childdoc.def|, |cdocsamp.tex|
with |cdocsch1.tex|, |cdocsch2.tex|, |cdocspt3.tex|, |cdocspt4.tex|,
|cdocsdrf.tex|, |cdocsfn1.tex|, |cdocsfn2.tex|
as well as |childdoc.pdf|.

%%%%%%%%%%%%%%%%%%%%%%%%%%%%%%%%%%%%%%%%%%%%%%%%%%%%%%%%%%%%%%%%%%%%%%%%%%%%%%%%
\subsection{Files and Installation}

The package consists of the files:
%
\begin{center}
\begin{tabular}{ll}
    |README.txt|   & readme file \\
    |childdoc.ins| & installation file \\
    |childdoc.dtx| & source file \\
    |childdoc.def| & definition file \\
    |cdocsamp.tex| & sample main file \\
    |cdocsch1.tex| & sample include file \\
    |cdocsch2.tex| & sample include file \\
    |cdocspt3.tex| & sample part file \\
    |cdocspt4.tex| & sample part file \\
    |cdocsdrf.tex| & sample redirection file \\
    |cdocsfn1.tex| & sample redirection file \\
    |cdocsfn2.tex| & sample redirection file \\
    |childdoc.pdf| & manual
\end{tabular}
\end{center}
%
The distribution consists of the files
|README.txt|, |childdoc.ins| and |childdoc.dtx|.
%
\begin{itemize}
\item
Run (pdf)\LaTeX{} on |childdoc.dtx|
to compile the manual |childdoc.pdf| (this file).
\item
Run \LaTeX{} on |childdoc.ins| to create the definitions file |childdoc.def|
and the sample |cdocsamp.tex| with include files
|cdocsch1.tex|, |cdocsch2.tex|, |cdocspt3.tex|, |cdocspt4.tex|,
|cdocsdrf.tex|, |cdocsfn1.tex|, |cdocsfn2.tex|.
Then copy the file |childdoc.def| to an appropriate directory of your \LaTeX{}
distribution, e.g.\ \textit{texmf-root}|/tex/latex/childdoc|.
\end{itemize}

%%%%%%%%%%%%%%%%%%%%%%%%%%%%%%%%%%%%%%%%%%%%%%%%%%%%%%%%%%%%%%%%%%%%%%%%%%%%%%%%
\subsection{Related CTAN Packages}

There are several other packages which offer a similar functionality:
%
\begin{itemize}
\item
The packages
\href{http://ctan.org/pkg/docmute}{\textsf{docmute}},
\href{http://ctan.org/pkg/includex}{\textsf{includex}} and
\href{http://ctan.org/pkg/standalone}{\textsf{standalone}}
provide commands to include only the document body of
a child file thus allowing both files to be compiled individually.
\item
The packages \href{http://ctan.org/pkg/subdocs}{\textsf{subdocs}}
and \href{http://ctan.org/pkg/subfiles}{\textsf{subfiles}}
provide structures in which the main and child documents can be
encapsulated and allowing them to be compiled individually.
The inclusion mechanism is different from the conventional |\include|.
\item
The package \href{http://ctan.org/pkg/combine}{\textsf{combine}}
is an elaborate solution to combine several documents into one.
\end{itemize}
%
See also the CTAN topic \href{http://ctan.org/topic/subdocs}{\textsf{subdocs}}
for further related packages.
The present package differs from the above solutions in that
a document structure constructed with the conventional |\include| mechanism
just needs two extra commands at the top of every file
such that all constituent files can be compiled individually.

%%%%%%%%%%%%%%%%%%%%%%%%%%%%%%%%%%%%%%%%%%%%%%%%%%%%%%%%%%%%%%%%%%%%%%%%%%%%%%%%
%\subsection{Feature Suggestions}
%
%The following is a list of features which may be useful for future
%versions of this package:
%%
%\begin{itemize}
%\item
%\ldots
%\end{itemize}

%%%%%%%%%%%%%%%%%%%%%%%%%%%%%%%%%%%%%%%%%%%%%%%%%%%%%%%%%%%%%%%%%%%%%%%%%%%%%%%%
\subsection{Revision History}

%%%%%%%%%%%%%%%%%%%%%%%%%%%%%%%%%%%%%%%%
\paragraph{v2.0:} 2018/12/30

\begin{itemize}
\item
immediate forward processing
\item
added |\childdocby| mechanism
\item
manual restructured
\end{itemize}

%%%%%%%%%%%%%%%%%%%%%%%%%%%%%%%%%%%%%%%%
\paragraph{v1.6:} 2018/01/17

\begin{itemize}
\item
application for development of include files
\item
corrections to manual
\end{itemize}

%%%%%%%%%%%%%%%%%%%%%%%%%%%%%%%%%%%%%%%%
\paragraph{v1.5:} 2017/05/21

\begin{itemize}
\item
more complete structuring introduced
\item
|\childdocof| introduced
\item
|\childdoc| renamed to |\childdocmain|
\item
|\childredirect| renamed to |\childdocforward| and |\childdocforwardprefix|
and functionality expanded
\end{itemize}

%%%%%%%%%%%%%%%%%%%%%%%%%%%%%%%%%%%%%%%%
\paragraph{v1.0:} 2017/04/27

\begin{itemize}
\item
manual and install package
\item
first version published on CTAN
\end{itemize}

%%%%%%%%%%%%%%%%%%%%%%%%%%%%%%%%%%%%%%%%
\paragraph{v0.6:} 2017/04/26

\begin{itemize}
\item
redirection mechanism added
\end{itemize}

%%%%%%%%%%%%%%%%%%%%%%%%%%%%%%%%%%%%%%%%
\paragraph{v0.5:} 2017/04/26

\begin{itemize}
\item
functionality in definition file
\end{itemize}


%%%%%%%%%%%%%%%%%%%%%%%%%%%%%%%%%%%%%%%%%%%%%%%%%%%%%%%%%%%%%%%%%%%%%%%%%%%%%%%%
%%%%%%%%%%%%%%%%%%%%%%%%%%%%%%%%%%%%%%%%%%%%%%%%%%%%%%%%%%%%%%%%%%%%%%%%%%%%%%%%
%%%%%%%%%%%%%%%%%%%%%%%%%%%%%%%%%%%%%%%%%%%%%%%%%%%%%%%%%%%%%%%%%%%%%%%%%%%%%%%%
\appendix

\settowidth\MacroIndent{\rmfamily\scriptsize 000\ }

 \DocInput{childdoc.dtx}

\end{document}
%</driver>
% \fi
%
% %%%%%%%%%%%%%%%%%%%%%%%%%%%%%%%%%%%%%%%%%%%%%%%%%%%%%%%%%%%%%%%%%%%%%%%%%%%%%%
% %%%%%%%%%%%%%%%%%%%%%%%%%%%%%%%%%%%%%%%%%%%%%%%%%%%%%%%%%%%%%%%%%%%%%%%%%%%%%%
% \section{Sample}
%\iffalse
%<*samplemain>
%\fi
%
% The following presents a sample document
% with two chapters, two parts, a title page,
% a compile flag as well as three forwarding files to set the flag.
% It consists of eight |.tex| files:
% \begin{center}
% \begin{tabular}{ll}
% |cdocsamp.tex|&main file\\
% |cdocsch1.tex|&include file for chapter 1\\
% |cdocsch2.tex|&include file for chapter 2\\
% |cdocspt3.tex|&include file for part 3\\
% |cdocspt4.tex|&include file for part 4\\
% |cdocsdrf.tex|&forwarding file for main file in draft mode\\
% |cdocsfi1.tex|&forwarding file for final version of chapter 1\\
% |cdocsfi2.tex|&forwarding file for final version of chapter 2\\
% \end{tabular}
% \end{center}
% Each of the eight files can be compiled directly by the \LaTeX{} compiler.
%
% %%%%%%%%%%%%%%%%%%%%%%%%%%%%%%%%%%%%%%
% \paragraph{Main File.}
%
% The main file is called |cdocsamp.tex|.
%
% Load the \textsf{childdoc} definitions and
% declare the filename for the main document:
%    \begin{macrocode}
\input{childdoc.def}
\childdocmain{}
%    \end{macrocode}

% Optional override for |\version| flag:
%    \begin{macrocode}
%%\ifchilddoc\else\providecommand{\version}{draft}\fi
%    \end{macrocode}

% Define the default values for the |\version| flag
% (|final| for the main file and |draft| for childs):
%    \begin{macrocode}
\ifchilddoc
\providecommand{\version}{draft}
\else
\providecommand{\version}{final}
\fi
%    \end{macrocode}

% Load the standard document class:
%    \begin{macrocode}
\documentclass[12pt]{article}
%    \end{macrocode}

% Start the document body:
%    \begin{macrocode}
\begin{document}
%    \end{macrocode}

% Declare a title page.
% Print title, part of document being processed and version flag:
%    \begin{macrocode}
\addtocounter{page}{-1}
\begin{center}
{\LARGE\bfseries{}childdoc example\par}
\vspace{1cm}
\ifchilddoc
\ifchilddocmanual part\else chapter\fi:
`\childdocname' of `\childdocjob'\par
\else
main document: `\childdocjob'\par
\fi
version: \version\par
\end{center}
\newpage
%    \end{macrocode}

% Manually include selected file,
% otherwise process as usual:
%    \begin{macrocode}
\ifchilddocmanual
\section*{part `\childdocname'}
\input{\childdocname}
\else
%    \end{macrocode}

% Include the two chapters:
%    \begin{macrocode}
\include{cdocsch1}
\include{cdocsch2}
%    \end{macrocode}

% Include the two parts unless only chapters should be displayed:
%    \begin{macrocode}
\ifchilddoc\else
\section{part three}
\input{cdocspt3}
\section{part four}
\input{cdocspt4}
\fi
%    \end{macrocode}

% Process as usual until here:
%    \begin{macrocode}
\fi
%    \end{macrocode}

% End of document body:
%    \begin{macrocode}
\end{document}
%    \end{macrocode}
%\iffalse
%</samplemain>
%\fi
%
% %%%%%%%%%%%%%%%%%%%%%%%%%%%%%%%%%%%%%%
% \paragraph{Chapter Include Files.}
%
% The include files are called |cdocsch1.tex| and |cdocsch2.tex|.
%
%\iffalse
%<*samplechap1|samplechap2>
%\fi

% Optional override for |\version| flag:
%    \begin{macrocode}
%%\providecommand{\version}{final}
%    \end{macrocode}

% Include the main document:
%    \begin{macrocode}
\input{childdoc.def}
\childdocof{cdocsamp}
%    \end{macrocode}

%\iffalse
%</samplechap1|samplechap2>
%\fi
%
%\iffalse
%<*samplechap1>
%\fi
% Some text for chapter 1:
%    \begin{macrocode}
\section{one}
some text in chapter one
%    \end{macrocode}

%\iffalse
%</samplechap1>
%\fi
% Some text for chapter 2:
%\iffalse
%<*samplechap2>
%\fi
%    \begin{macrocode}
\section{two}
more text in chapter two
%    \end{macrocode}

%\iffalse
%</samplechap2>
%\fi
%
% %%%%%%%%%%%%%%%%%%%%%%%%%%%%%%%%%%%%%%
% \paragraph{Part Include Files.}
%
% The include files are called |cdocspt3.tex| and |cdocspt4.tex|.
%
%\iffalse
%<*samplepart3|samplepart4>
%\fi

% Optional override for |\version| flag:
%    \begin{macrocode}
%%\providecommand{\version}{final}
%    \end{macrocode}

% Include the main document:
%    \begin{macrocode}
\input{childdoc.def}
\childdocby{cdocsamp}
%    \end{macrocode}

%\iffalse
%</samplepart3|samplepart4>
%\fi
%
%\iffalse
%<*samplepart3>
%\fi
% Some text for part 3:
%    \begin{macrocode}
some text in part three
%    \end{macrocode}

%\iffalse
%</samplepart3>
%\fi
% Some text for part 4:
%\iffalse
%<*samplepart4>
%\fi
%    \begin{macrocode}
more text in part four
%    \end{macrocode}

%\iffalse
%</samplepart4>
%\fi
%
% %%%%%%%%%%%%%%%%%%%%%%%%%%%%%%%%%%%%%%
% \paragraph{Forwarding for a Complete Draft.}
%
% The following forwarding file |cdocsdrf.tex|
% compiles the main document in draft mode:
%\iffalse
%<*sampledraft>
%\fi
%    \begin{macrocode}
\def\version{draft}
\input{childdoc.def}
\childdocforward{cdocsamp}
%    \end{macrocode}

%\iffalse
%</sampledraft>
%\fi
%
% %%%%%%%%%%%%%%%%%%%%%%%%%%%%%%%%%%%%%%
% \paragraph{Forwarding for Final Version of the Chapters.}
%
% The following forwarding files |cdocsfn1.tex| and |cdocsfn2.tex|
% (with identical content)
% compile the final versions of the child documents
% |cdocsch1.tex| and |cdocsch2.tex|, respectively:
%\iffalse
%<*samplefinal>
%\fi
%    \begin{macrocode}
\def\version{final}
\input{childdoc.def}
\childdocforwardprefix[cdocsamp]{cdocsfn}{cdocsch}
%    \end{macrocode}

%\iffalse
%</samplefinal>
%\fi
%
% %%%%%%%%%%%%%%%%%%%%%%%%%%%%%%%%%%%%%%
% \paragraph{Command Line Processing.}
%
% The following three command lines generate the output files
% |cdocscld|, |cdocscl1| and |cdocscl2|
% which should be identical to
% |cdocsdrf|, |cdocsch1| and |cdocsfn2|, respectively:
% \begin{center}
% \begin{tabular}{l}
% |latex -jobname cdocscld \|\\
% |  "\def\version{draft}\input{childdoc.def}\childdocforward{cdocsamp}"|\\
% |latex -jobname cdocscl1 \|\\
% |  "\input{childdoc.def}\childdocforward[cdocsamp]{cdocsch1}"|\\
% |latex -jobname cdocscl2 \|\\
% |  "\def\version{final}\input{childdoc.def}\childdocforward{cdocsch2}"|
% \end{tabular}
% \end{center}
% Note that the trailing backslash on each first line
% merely continues the input to the second line
% (for convenient cut ant paste).
% Furthermore, the command |latex| can be replaced by any
% of its alternative versions such as |pdflatex|.
%
% %%%%%%%%%%%%%%%%%%%%%%%%%%%%%%%%%%%%%%%%%%%%%%%%%%%%%%%%%%%%%%%%%%%%%%%%%%%%%%
% %%%%%%%%%%%%%%%%%%%%%%%%%%%%%%%%%%%%%%%%%%%%%%%%%%%%%%%%%%%%%%%%%%%%%%%%%%%%%%
% \section{Implementation}
%\iffalse
%<*package>
%\fi
%
% This section describes the definitions file |childdoc.def|.

% The definitions cannot be loaded using |\usepackage| or |\RequirePackage|
% which has a mechanism to prevent loading a style file more than once.
% When loading the definitions by means of |\input|
% multiple instances have to be prevented manually:
%\iffalse
%This code needs to be before the `\ProvidesFile' directive
%which is defined at the beginning of this file.
%Therefore it is also placed there and commented out here.
%</package>
%<*discard>
%\fi
%    \begin{macrocode}
\ifdefined\childdocmain\endinput\fi
%    \end{macrocode}
%\iffalse
%</discard>
%<*package>
%\fi
%
% \macro{\ifchilddoc}
% \macro{\ifchilddocmanual}
% The conditional |\ifchilddoc| tells whether a
% child (true) or main (false) document is being compiled.
% The conditional |\ifchilddocmanual| tells whether
% the |\includeonly| mechanism is used (false) or
% the selection of child files must be performed manually (true).
% The definitions initialise to false:
%    \begin{macrocode}
\newif\ifchilddoc
\newif\ifchilddocmanual
%    \end{macrocode}

% \macro{\childdocname}
% \macro{\childdocjob}
% The macro |\childdocname| stores the name of the main document
% to be compiled. The macro |\childdocjob| stores the name of
% the document on which the \LaTeX{} compiler was originally invoked.
% The content of |\jobname| cannot be compared
% to filenames specified in the source due to different catcodes.
% The following code rescans |\jobname|, stores the result
% in |\childdocname| and saves a copy in |\childdocjob|:
%    \begin{macrocode}
\edef\childdocname{\scantokens\expandafter{\jobname\noexpand}}
\let\childdocjob\childdocname
%    \end{macrocode}

% \macro{\childdocdisable}
% The macro |\childdocdisable| prevents the main file
% from being processed more than once.
% At this stage, the main document command |\childdocmain|
% is assumed to be called once again where it should do nothing.
% Any subsequent call to it should prevent
% a secondary processing of the main document
% It overwrites the forwarding commands
% |\childdocof| and |\childdocforward|
% with empty macros to prevent further inclusions of the main document:
%    \begin{macrocode}
\newcommand{\childdocdisable}
{
  \renewcommand{\childdocmain}[1]{\renewcommand{\childdocmain}[1]{\endinput}}
  \renewcommand{\childdocof}[1]{}
  \renewcommand{\childdocby}[2][]{}
  \renewcommand{\childdocforward}[2][]{}
  \renewcommand{\childdocdisable}{}
}
%    \end{macrocode}

% \macro{\childdocmain}
% The macro |\childdocmain| is to be called at the top of the main file
% with nothing or the main filename (without extension) as argument.
% First, it breaks loops.
% If the argument is not empty and does not match |\childdocname|
% (which is set by the first inclusion of |childdoc.def|),
% |\ifchilddoc| is set to true, |\includeonly| is applied to the child file
% and |\jobname| is set to the main file
% (for proper handling of |.aux| files):
%    \begin{macrocode}
\newcommand{\childdocmain}[1]
{
  \childdocdisable\childdocmain{}
  \if?#1?\else
    \begingroup
      \def\childdoctmp{#1}
      \ifx\childdoctmp\childdocname
        \def\childdoctmp{}
      \else
        \def\childdoctmp
        {
          \childdoctrue
          \includeonly{\childdocname}
          \def\childdocjob{#1}
          \def\jobname{#1}
        }
      \fi
      \expandafter
    \endgroup
    \childdoctmp
  \fi
}
%    \end{macrocode}

% \macro{\childdocof}
% The command |\childdocof| redirects
% compilation to the main file |#1|.
%    \begin{macrocode}
\newcommand{\childdocof}[1]
{
  \childdocdisable
  \childdoctrue
  \includeonly{\childdocname}
  \def\jobname{#1}
  \def\childdocjob{#1}
  \input{#1}
}
%    \end{macrocode}

% \macro{\childdocby}
% The command |\childdocby| ....
%    \begin{macrocode}
\newcommand{\childdocby}[2][]
{
  \childdocdisable
  \childdoctrue
  \childdocmanualtrue
  \if?#1?\else
    \def\jobname{#2}
  \fi
  \def\childdocjob{#2}
  \input{#2}
  \endinput
}
%    \end{macrocode}

% \macro{\childdocforward}
% The command |\childdocforward| redirects
% compilation to the main file or
% (if the optional argument is given) a child file.
% Parameters are set as if the main file
% or a child file starting with |\childdocof| was compiled.
% Then compilation is handed over to the main file:
%    \begin{macrocode}
\newcommand{\childdocforward}[2][]
{
  \begingroup
    \if?#1?
      \def\childdoctmp
      {
        \def\childdocname{#2}
        \def\childdocjob{#2}
        \def\jobname{#2}
        \input{#2}
        \endinput
      }
    \else
      \def\childdoctmp
      {
        \childdocdisable
        \def\childdocname{#2}
        \childdoctrue
        \includeonly{#2}
        \def\childdocjob{#1}
        \def\jobname{#1}
        \input{#1}
        \endinput
      }
    \fi
    \expandafter
  \endgroup
  \childdoctmp
}
%    \end{macrocode}

% \macro{\childdocforwardprefix}
% The command |\childdocforwardprefix| redirects
% compilation to the main or a child file by means of a pattern.
% The prefix |#1| in the current filename is replaced by |#2|
% and the suffix of the current filename is kept
% (it is assumed that the filename does not contain the substring `|~~~|'
% which is used as a delimiter).
% Compilation is handed over to the new file by |\childdocforward|:
%    \begin{macrocode}
\newcommand{\childdocforwardprefix}[3][]
{
  \begingroup
    \def\childdocextract #2##1~~~{\def\childdoctmp{\childdocforward[#1]{#3##1}}}
    \expandafter\childdocextract\childdocname~~~
    \expandafter
  \endgroup
  \childdoctmp
}
%    \end{macrocode}

% \macro{\childdoc}
% The deprecated macro |\childdoc| is a legacy version of |\childdocmain|:
%    \begin{macrocode}
\newcommand{\childdoc}{\childdocmain}
%    \end{macrocode}

% \macro{\childdocredirect}
% The deprecated macro |\childdocredirect| is a legacy version
% of |\childdocforward| and |\childdocforwardprefix|:
%    \begin{macrocode}
\newcommand{\childdocredirect}[2][]
{
  \begingroup
    \if?#1?
      \def\childdoctmp{\childdocforward{#2}}
    \else
      \def\childdoctmp{\childdocforwardprefix{#1}{#2}}
    \fi
    \expandafter
  \endgroup
  \childdoctmp
}
%    \end{macrocode}

%\iffalse
%</package>
%\fi
%
\endinput
\childdocforward{cdocsamp}"|\\
% |latex -jobname cdocscl1 \|\\
% |  "% \iffalse
%
% childdoc.dtx Copyright (C) 2017-2018 Niklas Beisert
%
% This work may be distributed and/or modified under the
% conditions of the LaTeX Project Public License, either version 1.3
% of this license or (at your option) any later version.
% The latest version of this license is in
%   http://www.latex-project.org/lppl.txt
% and version 1.3 or later is part of all distributions of LaTeX
% version 2005/12/01 or later.
%
% This work has the LPPL maintenance status `maintained'.
%
% The Current Maintainer of this work is Niklas Beisert.
%
% This work consists of the files childdoc.dtx and childdoc.ins
% and the derived files childdoc.def and cdocsamp.tex with
% cdocsch1.tex, cdocsch2.tex, cdocsdrf.tex, cdocsfn1.tex, cdocsfn2.tex.
%
%<package>\ifdefined\childdocmain\endinput\fi
%<package>\ProvidesFile{childdoc.def}[2018/12/30 v2.0 child document driver]
%<samplemain>\ProvidesFile{cdocsamp.tex}[2018/12/30 v2.0 sample for childdoc]
%<*driver>
%\ProvidesFile{childdoc.drv}[2018/12/30 v2.0 childdoc reference manual file]
\PassOptionsToClass{10pt,a4paper}{article}
\documentclass{ltxdoc}

\usepackage[margin=35mm]{geometry}
\usepackage{hyperref}
\usepackage{hyperxmp}
\usepackage[usenames]{color}

\hypersetup{colorlinks=true}
\hypersetup{pdfstartview=FitH}
\hypersetup{pdfpagemode=UseNone}
\hypersetup{pdfsource={}}
\hypersetup{pdflang={en-UK}}
\hypersetup{pdfcopyright={Copyright 2017-2018 Niklas Beisert.
  This work may be distributed and/or modified under the
  conditions of the LaTeX Project Public License, either version 1.3
  of this license or (at your option) any later version.}}
\hypersetup{pdflicenseurl={http://www.latex-project.org/lppl.txt}}
\hypersetup{pdfcontactaddress={ETH Zurich, ITP, HIT K,
  Wolfgang-Pauli-Strasse 27}}
\hypersetup{pdfcontactpostcode={8093}}
\hypersetup{pdfcontactcity={Zurich}}
\hypersetup{pdfcontactcountry={Switzerland}}
\hypersetup{pdfcontactemail={nbeisert@itp.phys.ethz.ch}}
\hypersetup{pdfcontacturl={http://people.phys.ethz.ch/\xmptilde nbeisert/}}

\newcommand{\secref}[1]{\hyperref[#1]{section \ref*{#1}}}

\parskip1ex
\parindent0pt
\let\olditemize\itemize
\def\itemize{\olditemize\parskip0pt}

\begin{document}

\title{The \textsf{childdoc} Package}
\hypersetup{pdftitle={The childdoc Package}}
\author{Niklas Beisert\\[2ex]
  Institut f\"ur Theoretische Physik\\
  Eidgen\"ossische Technische Hochschule Z\"urich\\
  Wolfgang-Pauli-Strasse 27, 8093 Z\"urich, Switzerland\\[1ex]
  \href{mailto:nbeisert@itp.phys.ethz.ch}
  {\texttt{nbeisert@itp.phys.ethz.ch}}}
\hypersetup{pdfauthor={Niklas Beisert}}
\hypersetup{pdfsubject={Manual for the LaTeX2e Package childdoc}}
\date{30 December 2018, \textsf{v2.0}}
\maketitle

\begin{abstract}\noindent
\textsf{childdoc} is a \LaTeXe{} package
that enables the direct compilation
of document sections included by |\include|
to individual files.
\end{abstract}

\begingroup
\parskip0ex
\tableofcontents
\endgroup

%%%%%%%%%%%%%%%%%%%%%%%%%%%%%%%%%%%%%%%%%%%%%%%%%%%%%%%%%%%%%%%%%%%%%%%%%%%%%%%%
%%%%%%%%%%%%%%%%%%%%%%%%%%%%%%%%%%%%%%%%%%%%%%%%%%%%%%%%%%%%%%%%%%%%%%%%%%%%%%%%
\section{Introduction}

\LaTeX{} provides a mechanism to structure a large document (such as a book)
into a main file and several child files (containing the chapters)
using the |\include| command.
This mechanism is beneficial for documents
which span hundreds of pages in order to
make the source file(s) more manageable.
Moreover, compilation can be restricted to
selected child files by means of the |\includeonly| command.
The latter feature can be used to reduce the compilation time while editing
(this was significantly more useful in the earlier days of \LaTeX{})
or to generate a smaller document which is easier to navigate.
Another application of |\includeonly| is to generate
documents consisting of selected parts of the complete document.

However, there are a few drawbacks of the plain |\include| mechanism:
\begin{itemize}
\item
The child files cannot be compiled on their own,
they can only be compiled via the main file.
A naive editing environment
(such as a text editor with an option
to have the current file processed by \LaTeX)
may require one to switch to the main file before compiling;
attempting to compile the child file produces errors.
\item
The main file must be modified (each time)
to adjust the |\includeonly| command
to the present needs. This easily leaves the main file in a messy state.
\item
The generated document will always carry the filename
of the main document. This is inconvenient if
several child files are to be compiled and
to be kept for distribution.
\end{itemize}

The present package provides a simple interface
to make child files individually compilable by \LaTeX{}.
Compiling a child file then has the same effect as compiling
the main file with an |\includeonly| command
to select the appropriate child.
Moreover the generated document will carry the name of the child
rather than the main file.
This resolves all three above issues.

This feature is meant to make the editing of books,
thesis documents and lecture notes somewhat more convenient.
However, the package can also be used efficiently for
composing a series of documents (such as exercise sheets)
which are typically distributed individually.
It then assists the author in generating the individual documents
(potentially in different versions)
as well as a document containing the collected series.
Another application is in developing style files
or other kinds of included material
where compilation of the style file could redirect
to a sample or test file.

%%%%%%%%%%%%%%%%%%%%%%%%%%%%%%%%%%%%%%%%%%%%%%%%%%%%%%%%%%%%%%%%%%%%%%%%%%%%%%%%
%%%%%%%%%%%%%%%%%%%%%%%%%%%%%%%%%%%%%%%%%%%%%%%%%%%%%%%%%%%%%%%%%%%%%%%%%%%%%%%%
\section{Usage}

First of all, the package \textsf{childdoc} is \emph{not} a standard
\LaTeXe{} |.sty| style file! Therefore it needs to be invoked in
a non-standard way.

%%%%%%%%%%%%%%%%%%%%%%%%%%%%%%%%%%%%%%%%%%%%%%%%%%%%%%%%%%%%%%%%%%%%%%%%%%%%%%%%
\subsection{Included Files}
\label{sec:include}

%%%%%%%%%%%%%%%%%%%%%%%%%%%%%%%%%%%%%%%%
\DescribeMacro{\childdocmain}
To use the package, add the commands
\begin{center}
\begin{tabular}{l}
|\input{childdoc.def}|\\
|\childdocmain{}|\\
\end{tabular}
\end{center}
at the very top of the main \LaTeX{} file,
in particular \emph{before} the |\documentclass| statement!
The argument of |\childdocmain| should be left empty
(but it must be present).

%%%%%%%%%%%%%%%%%%%%%%%%%%%%%%%%%%%%%%%%
\DescribeMacro{\childdocof}
Furthermore, add the commands
\begin{center}
\begin{tabular}{l}
|\input{childdoc.def}|\\
|\childdocof{|\textit{main}|}|\\
\end{tabular}
\end{center}
at the top of every child file \textit{child}
which is included by |\include{|\textit{child}|}|
from within the main file
(or at least for those files to be compiled individually).
The argument \textit{main} must be the filename of the main file.

There are a couple of
considerations in setting up the main and child documents:

%%%%%%%%%%%%%%%%%%%%%%%%%%%%%%%%%%%%%%%%
\paragraph{Restrictions.}

Please note the following restrictions:
\begin{itemize}
\item
|\childdocmain| must be called with one argument \textit{main}
to ensure compatibility with earlier version of the package.
It must either be empty (|\childdocmain{}|)
or precisely match the filename of the main file in which it is specified.
See \secref{sec:detection} for further information.
\item
The filename \textit{main} must be specified without the |.tex| extension.
\item
The filename \textit{main} is case sensitive
(even in case-insensitive file systems)
due to internal string comparison.
\item
The argument \textit{main} should be fully expanded, it cannot be a macro.
\item
Subdirectories and special characters should be avoided in filenames.
\item
The command |\childdocmain{|\textit{main}|}| must be followed by a whitespace.
It should not be followed immediately by another command
or by a comment mark `|%|'.
This is because the \TeX{} parser reads the token immediately following
the argument of |\childdocmain| and puts it
at the beginning of every child section;
however, a white\-space is ignored.
\end{itemize}

%%%%%%%%%%%%%%%%%%%%%%%%%%%%%%%%%%%%%%%%
\paragraph{Content of Main File.}

It is advisable to place all content in the child files included by |\include|.
Any output contained in the main file will appear in all child documents
unless suppressed manually;
it cannot be suppressed automatically by the |\includeonly| directive
and thus should normally be avoided.
A method to include some content in the main file
by means of conditional processing is described in \secref{sec:conditional}.

%%%%%%%%%%%%%%%%%%%%%%%%%%%%%%%%%%%%%%%%
\paragraph{Page Numbering.}

When only a part of the document is compiled,
the appropriate numbering of pages
(as well as other status parameters)
is determined from the |.aux| files.
The latter contain information from previous passes.
However this information needs to propagate through
all intermediate child documents.
Therefore the page numbering in child documents may well
be inconsistent until the complete document is compiled at least once.

A useful (if unconventional) way to always ensure a consistent
page numbering is to restart the numbering in each child document
and denote the pages by `\textit{child}|.|\textit{page}'
where \textit{child} represents the chapter/section number of the child file.
This can be achieved by the command
|\numberwithin{page}{|\textit{child}|}|
of the \textsf{amsmath} package
where \textit{child} can be |chapter| or |section|
depending on the chosen structuring.
Alternatively, one can modify the macro |\thepage| appropriately
and reset the counter |page| at the start of each child file.

%%%%%%%%%%%%%%%%%%%%%%%%%%%%%%%%%%%%%%%%%%%%%%%%%%%%%%%%%%%%%%%%%%%%%%%%%%%%%%%%
\subsection{Conditional Processing}
\label{sec:conditional}

The package provides a mechanism to compile different versions
of a document. To customise the versions further some conditional processing
can come in handy to distinguish which version is being compiled.
The package provides two macros to describe the compilation context:

%%%%%%%%%%%%%%%%%%%%%%%%%%%%%%%%%%%%%%%%
\DescribeMacro{\ifchilddoc}
The conditional |\ifchilddoc| distinguishes between the compilation of
child documents and the main document:
%
\begin{center}
|\ifchilddoc |\textit{child-code}| |[|\||else |\textit{main-code}]| \||fi|
\end{center}

%%%%%%%%%%%%%%%%%%%%%%%%%%%%%%%%%%%%%%%%
\DescribeMacro{\childdocname}
\DescribeMacro{\childdocjob}
The macro |\childdocname| contains the filename (without extension)
of the main or child file being processed.
Note that |\childdocjob| will always contain the name of the main file.

%%%%%%%%%%%%%%%%%%%%%%%%%%%%%%%%%%%%%%%%
\paragraph{Title Page.}

Conditional processing can be used to include a title or banner page
in the main document when proper precautions are taken.
Importantly, the code in the main file should ensure that the page counter
(as well as other status parameters which are stored in the |.aux| files)
takes the same value after the conditional processing.
Otherwise the page numbers may take divergent values
depending on which part is compiled.

For example, a title page could be declared by:
%
\begin{center}
\begin{tabular}{l}
|\ifchilddoc\||else|\\
|\addtocounter{page}{-1}|\\
\textit{code for title page}\\
|\newpage|\\
|\||fi|
\end{tabular}
\end{center}
%
A banner page for the child documents can be generated by:
%
\begin{center}
\begin{tabular}{l}
|\ifchilddoc|\\
|\addtocounter{page}{-1}|\\
\textit{code for banner page}\\
|\newpage|\\
|\||fi|
\end{tabular}
\end{center}
%
Here one could write a message such as:
\begin{center}
|This is the part \childdocname{} of \childdocjob{}.|
\end{center}

%%%%%%%%%%%%%%%%%%%%%%%%%%%%%%%%%%%%%%%%%%%%%%%%%%%%%%%%%%%%%%%%%%%%%%%%%%%%%%%%
\subsection{Flags}
\label{sec:flags}

The package makes it easy to generate different versions
of the main or child documents.
To this end compilation flags can be defined
and assigned different default values.
They will be particularly useful in conjunction
with the forwarding mechanism described in \secref{sec:forward}.

For example, it may be useful to have a flag |\version|
which can be set to |draft| or |final|.
The document source will contain some conditional code
depending on the value of |\version|.
Suppose further, the flag should default to |final| for the main file
and to |draft| for child files
which is a natural assignment for editing the document.
This is achieved by placing the following code
in the preamble of the main document
(below the |\childdocmain| directive):
%
\begin{center}
\begin{tabular}{l}
|\ifchilddoc|\\
|\providecommand{\version}{draft}|\\
|\||else|\\
|\providecommand{\version}{final}|\\
|\||fi|
\end{tabular}
\end{center}
%
The definition by |\providecommand| makes sure
that previous definitions are not overwritten.
Further statements |\providecommand{\version}{...}|
can thus be added before the above code to override it.

For the main file, one might add a line
(between |\childdocmain| and the above block)
%
\begin{center}
|%\ifchilddoc\||else\providecommand{\version}{draft}\||fi|
\end{center}
%
which can be uncommented to produce a draft version.
Likewise one can add a line to the very top of a child file
(above the |\childdocof{|\textit{main}|}| directive)
%
\begin{center}
|%\providecommand{\version}{final}|
\end{center}
%
which can be uncommented to produce the final version of this child document.

%%%%%%%%%%%%%%%%%%%%%%%%%%%%%%%%%%%%%%%%%%%%%%%%%%%%%%%%%%%%%%%%%%%%%%%%%%%%%%%%
\subsection{Forwarding}
\label{sec:forward}

Different versions of the main or child documents
using compilation flags as described in \secref{sec:flags}
can be (permanently) stored in different files
for convenient compilation, viewing and distribution.
To this end, the package defines a command
to pass on compilation to a different file:

%%%%%%%%%%%%%%%%%%%%%%%%%%%%%%%%%%%%%%%%
\DescribeMacro{\childdocforward}
The command |\childdocforward| redirects processing to
another source file:
%
\begin{center}
\begin{tabular}{l}
|\input{childdoc.def}|\\
|\childdocforward[|\textit{main}|]{|\textit{dest}|}|\\
\end{tabular}
\end{center}
%
The argument \textit{dest} is the destination file
(without extension).
It should be the main file or one of the child files.
Note that further \textsf{childdoc} directives
such as |\childdocof| and |\childdocforward|
in the indicated file will be processed in this form.
The optional argument \textit{main}
passes on directly to the main file \textit{main}
while pretending to compile the child \textit{dest}.
This form behaves as if \textit{dest}
issues |\childdocof{|\textit{main}|}| right away,
and no further \textsf{childdoc} directives will be processed.

%%%%%%%%%%%%%%%%%%%%%%%%%%%%%%%%%%%%%%%%
\DescribeMacro{\...prefix}
In the alternative form |\childdocforwardprefix|,
%
\begin{center}
\begin{tabular}{l}
|\input{childdoc.def}|\\
|\childdocforwardprefix[|\textit{main}|]{|\textit{prefix}|}{|\textit{dest}|}|
\end{tabular}
\end{center}
%
the destination file is determined by a pattern
depending on the current file:
To make this work, the current file must be called
`{\textit{prefix}\hspace{0.2em}\textit{suffix}}'
with \textit{prefix} matching precisely the argument.
Processing is then passed on to the file
`{\textit{dest}\hspace{0.2em}\textit{suffix}}'.
Surely, the same effect is achieved by
directly specifying the
argument `{\textit{dest}\hspace{0.2em}\textit{suffix}}'
in the first form.
However, that requires to set up a different file
for each child. With the alternative form of the command
all these files can have exactly the same content
which simplifies setting them up and maintaining them.

For example, the following file |draft.tex|
with a compilation flag |\version| as described in \secref{sec:flags}
compiles the main document as a draft:
%
\begin{center}
\begin{tabular}{l}
|\def\version{draft}|\\
|\input{childdoc.def}|\\
|\childdocforward{|\textit{main}|}|
\end{tabular}
\end{center}
%
Likewise, the following files |final|\textit{nn}|.tex|
compile the final version of the child document
|child|\textit{nn}|.tex|:
%
\begin{center}
\begin{tabular}{l}
|\def\version{final}|\\
|\input{childdoc.def}|\\
|\childdocforwardprefix{final}{child}|
\end{tabular}
\end{center}
%

Note that when several versions of a main file and/or of each child file
are to be generated, it may be convenient to set up a |Makefile| or
shell script to automatise the process.

%%%%%%%%%%%%%%%%%%%%%%%%%%%%%%%%%%%%%%%%%%%%%%%%%%%%%%%%%%%%%%%%%%%%%%%%%%%%%%%%
\subsection{Command Line Processing}
\label{sec:commandline}

The effect of redirection files can also be achieved by invoking
the \LaTeX{} compiler with a more elaborate command line.
Most conveniently this should be done as part
of a shell script or a |Makefile|.

When using \textsf{childdoc} in the main file, the following
command lines effectively perform a redirection
(note that depending on the shell being used,
backslashes may have to be doubled: `|\|' $\to$ `|\\|'):
%
\begin{center}
|... -jobname "|\textit{target}|" |\\|"|[\textit{flags}]%
|\input{childdoc.def}\childdocforward[|\textit{main}|]{|\textit{dest}|}"|
\end{center}
%
Here \textit{target} is the name of the output file,
\textit{main} is the name of the main file
and \textit{dest} is the name of the main or child file to be processed
(all filenames without extensions).
The optional argument \textit{main} can be omitted
if \textit{main} matches \textit{dest}.
Optionally, compilation \textit{flags} can be defined via |\def| commands.
This command line makes the \TeX{} engine believe
it is compiling the file \textit{target}
whose content is specified as the latter parameter.
The provided code then forwards the processing to
\textit{main} or \textit{dest} as described in \secref{sec:forward}.

%%%%%%%%%%%%%%%%%%%%%%%%%%%%%%%%%%%%%%%%%%%%%%%%%%%%%%%%%%%%%%%%%%%%%%%%%%%%%%%%
\subsection{Include by Input}
\label{sec:input}

Including child documents by |\include| has some restrictions by design.
Most notably, the content of a child document always occupies
its own set of pages; pages cannot be shared between child documents.
Usually, this behaviour makes perfect sense
because each child document contain an essential part of the document.
However, in some situations it may be desirable to compose
a document from a collection of parts
without having mandatory page breaks between then.
For this case, the package
provides a mechanism to include parts
by |\input| which can also be processed individually.
However, by construction this mechanism
requires manual handling of the content to be output.

%%%%%%%%%%%%%%%%%%%%%%%%%%%%%%%%%%%%%%%%
\DescribeMacro{\ifchilddocmanual}
The main file should be prepared as usual, see \secref{sec:include}.
However, the document body must make a distinction
between processing of an individual part and of the main document, e.g.:
%
\begin{center}
\begin{tabular}{l}
|\ifchilddocmanual|\\
|\input{\childdocname}|\\
|\||else|\\
\textit{document body with }|\input{|\textit{part}|}|\\
|\||fi|
\end{tabular}
\end{center}
%
The conditional |\ifchilddocmanual| is true whenever
a part to be included by |\input| is being compiled,
and the name of the part is stored in |\childdocname|.

%%%%%%%%%%%%%%%%%%%%%%%%%%%%%%%%%%%%%%%%
\DescribeMacro{\childdocby}
Each part to be included by |\input| should start with:
%
\begin{center}
\begin{tabular}{l}
|\input{childdoc.def}|\\
|\childdocby{|\textit{main}|}|\\
\end{tabular}
\end{center}
%
The directive |\childdocby| is similar to |\childdocof|
described in \secref{sec:include},
but the subsequent selection of content must be done manually.
To that end, both |\ifchilddoc| and |\ifchilddocmanual|
will be true upon processing of a part,
and the name of the part is stored in |\childdocname|.
Note that |\jobname| will be set to the filename of the current part
so that each part receives an individual |.aux| file
that does not interfere with the |.aux| file(s) of the main document.
This behaviour can be altered by the alternative form
|\childdocby[*]{|\textit{main}|}| (with a non-empty optional argument)
which uses the |.aux| file of the main document
by setting |\jobname| to \textit{main}.

%%%%%%%%%%%%%%%%%%%%%%%%%%%%%%%%%%%%%%%%%%%%%%%%%%%%%%%%%%%%%%%%%%%%%%%%%%%%%%%%
\subsection{Driver Development}
\label{sec:driver}

The \textsf{childdoc} mechanism can also be use for the development
of definition files such as \LaTeX{} styles or classes.
This case differs from the above setup with multiple parts
included by |\include| in that no |\includeonly| should be invoked.
This can be achieved by starting the include file
(before |\ProvidesPackage|) with:
%
\begin{center}
\begin{tabular}{l}
|\input{childdoc.def}|\\
|\childdocforward{|\textit{main}|}|\\
\end{tabular}
\end{center}
%
or alternatively with:
%
\begin{center}
\begin{tabular}{l}
|\input{childdoc.def}|\\
|\childdocby{|\textit{main}|}|\\
\end{tabular}
\end{center}
%
Both forms have slightly different effects as described above.
The main file is prepared as usual, see \secref{sec:include}.

%%%%%%%%%%%%%%%%%%%%%%%%%%%%%%%%%%%%%%%%%%%%%%%%%%%%%%%%%%%%%%%%%%%%%%%%%%%%%%%%
\subsection{Legacy Detection}
\label{sec:detection}

The directive |\childdocmain| in the main file can detect
whether the complete document or merely a child is to be compiled
even without using the directive |\childdocof|.
This method is deprecated because it is less robust
and there is no compelling reason to use it;
it is merely provided for backward compatibility
and it may be removed in future versions.

If the detection mechanism is to be used,
it is mandatory to correctly specify
the filename of the main file as the argument of |\childdocmain|:
%
\begin{center}
\begin{tabular}{l}
|\input{childdoc.def}|\\
|\childdocmain{|\textit{main}|}|\\
\end{tabular}
\end{center}
%
If |\jobname| does not match the argument \textit{main} of |\childdocmain|,
it is assumed that |\jobname| points to the child file to be compiled.
When using |\childdocmain| with the main file specified as argument,
it suffices to start a child file
with just |\input{|\textit{main}|}|
without loading of the package and using |\childdocof|.
If instead all processing is done
with the appropriate \textsf{childdoc} directives,
the argument of \textit{main} of |\childdocmain| can be empty.

An alternative version of the command line processing described
in \secref{sec:commandline} using the detection mechanism reads:
%
\begin{center}
|... -jobname "|\textit{target}|" "|[\textit{flags}]%
[|\def\jobname{|\textit{dest}|}|]|\input{|\textit{main}|}"|
\end{center}

%%%%%%%%%%%%%%%%%%%%%%%%%%%%%%%%%%%%%%%%%%%%%%%%%%%%%%%%%%%%%%%%%%%%%%%%%%%%%%%%
\subsection{Manual Code}
\label{sec:manual}

In case one cannot be certain whether the definitions file |childdoc.def|
is installed on the target \TeX{} distribution
and one prefers not to ship it,
it is conceivable to paste a few relevant commands into the sources.

To that end, drop all statements |\input{childdoc.def}|
and perform the replacements as outlined below.
Instead of |\childdocmain{|\textit{main}|}| add the following code
to the top of the main file:
%
\begin{center}
\begin{tabular}{l}
|\||ifdefined\childdocname\endinput\||fi\newif\ifchilddoc|\\
|\edef\childdocname{\scantokens\expandafter{\jobname\noexpand}}|\\
|\def\childdocmain{|\textit{main}|}\||ifx\childdocmain\childdocname\||else|\\
|\childdoctrue\includeonly{\childdocname}\let\jobname\childdocmain\||fi|\\
\end{tabular}
\end{center}
%
Instead of |\childdocof{|\textit{main}|}| just include the main file
at the top of each child file:
%
\begin{center}
|\input{|\textit{main}|}|
\end{center}
%
A simple redirection |\childdocforward{|\textit{dest}|}| is achieved by:
%
\begin{center}
|\def\jobname{|\textit{dest}|}\input{\jobname}|
\end{center}
%
The redirection with prefix
|\childdocforwardprefix[|\textit{prefix}|]{|\textit{dest}|}|
is accomplished by:
%
\begin{center}
\begin{tabular}{l}
|{\edef\jobname{\scantokens\expandafter{\jobname\noexpand}}|\\
|\def\redirectjob |\textit{prefix}|#1~~~{\gdef\jobname{|\textit{dest}|#1}}|\\
|\expandafter\redirectjob\jobname~~~}\input{\jobname}|
\end{tabular}
\end{center}

In an alternative approach,
child documents can be compiled by a specific command line
without additional code or specific definitions:
%
\begin{center}
|... -jobname "|\textit{target}|" "|[\textit{flags}]%
|\includeonly{|\textit{dest}|}\input{|\textit{main}|}"|
\end{center}
%

%%%%%%%%%%%%%%%%%%%%%%%%%%%%%%%%%%%%%%%%%%%%%%%%%%%%%%%%%%%%%%%%%%%%%%%%%%%%%%%%
%%%%%%%%%%%%%%%%%%%%%%%%%%%%%%%%%%%%%%%%%%%%%%%%%%%%%%%%%%%%%%%%%%%%%%%%%%%%%%%%
\section{Information}

%%%%%%%%%%%%%%%%%%%%%%%%%%%%%%%%%%%%%%%%%%%%%%%%%%%%%%%%%%%%%%%%%%%%%%%%%%%%%%%%
\subsection{Copyright}

Copyright \copyright{} 2017--2018 Niklas Beisert

This work may be distributed and/or modified under the
conditions of the \LaTeX{} Project Public License, either version 1.3
of this license or (at your option) any later version.
The latest version of this license is in
  \url{http://www.latex-project.org/lppl.txt}
and version 1.3 or later is part of all distributions of \LaTeX{}
version 2005/12/01 or later.

This work has the LPPL maintenance status `maintained'.

The Current Maintainer of this work is Niklas Beisert.

This work consists of the files |README.txt|, |childdoc.ins| and |childdoc.dtx|
as well as the derived files |childdoc.def|, |cdocsamp.tex|
with |cdocsch1.tex|, |cdocsch2.tex|, |cdocspt3.tex|, |cdocspt4.tex|,
|cdocsdrf.tex|, |cdocsfn1.tex|, |cdocsfn2.tex|
as well as |childdoc.pdf|.

%%%%%%%%%%%%%%%%%%%%%%%%%%%%%%%%%%%%%%%%%%%%%%%%%%%%%%%%%%%%%%%%%%%%%%%%%%%%%%%%
\subsection{Files and Installation}

The package consists of the files:
%
\begin{center}
\begin{tabular}{ll}
    |README.txt|   & readme file \\
    |childdoc.ins| & installation file \\
    |childdoc.dtx| & source file \\
    |childdoc.def| & definition file \\
    |cdocsamp.tex| & sample main file \\
    |cdocsch1.tex| & sample include file \\
    |cdocsch2.tex| & sample include file \\
    |cdocspt3.tex| & sample part file \\
    |cdocspt4.tex| & sample part file \\
    |cdocsdrf.tex| & sample redirection file \\
    |cdocsfn1.tex| & sample redirection file \\
    |cdocsfn2.tex| & sample redirection file \\
    |childdoc.pdf| & manual
\end{tabular}
\end{center}
%
The distribution consists of the files
|README.txt|, |childdoc.ins| and |childdoc.dtx|.
%
\begin{itemize}
\item
Run (pdf)\LaTeX{} on |childdoc.dtx|
to compile the manual |childdoc.pdf| (this file).
\item
Run \LaTeX{} on |childdoc.ins| to create the definitions file |childdoc.def|
and the sample |cdocsamp.tex| with include files
|cdocsch1.tex|, |cdocsch2.tex|, |cdocspt3.tex|, |cdocspt4.tex|,
|cdocsdrf.tex|, |cdocsfn1.tex|, |cdocsfn2.tex|.
Then copy the file |childdoc.def| to an appropriate directory of your \LaTeX{}
distribution, e.g.\ \textit{texmf-root}|/tex/latex/childdoc|.
\end{itemize}

%%%%%%%%%%%%%%%%%%%%%%%%%%%%%%%%%%%%%%%%%%%%%%%%%%%%%%%%%%%%%%%%%%%%%%%%%%%%%%%%
\subsection{Related CTAN Packages}

There are several other packages which offer a similar functionality:
%
\begin{itemize}
\item
The packages
\href{http://ctan.org/pkg/docmute}{\textsf{docmute}},
\href{http://ctan.org/pkg/includex}{\textsf{includex}} and
\href{http://ctan.org/pkg/standalone}{\textsf{standalone}}
provide commands to include only the document body of
a child file thus allowing both files to be compiled individually.
\item
The packages \href{http://ctan.org/pkg/subdocs}{\textsf{subdocs}}
and \href{http://ctan.org/pkg/subfiles}{\textsf{subfiles}}
provide structures in which the main and child documents can be
encapsulated and allowing them to be compiled individually.
The inclusion mechanism is different from the conventional |\include|.
\item
The package \href{http://ctan.org/pkg/combine}{\textsf{combine}}
is an elaborate solution to combine several documents into one.
\end{itemize}
%
See also the CTAN topic \href{http://ctan.org/topic/subdocs}{\textsf{subdocs}}
for further related packages.
The present package differs from the above solutions in that
a document structure constructed with the conventional |\include| mechanism
just needs two extra commands at the top of every file
such that all constituent files can be compiled individually.

%%%%%%%%%%%%%%%%%%%%%%%%%%%%%%%%%%%%%%%%%%%%%%%%%%%%%%%%%%%%%%%%%%%%%%%%%%%%%%%%
%\subsection{Feature Suggestions}
%
%The following is a list of features which may be useful for future
%versions of this package:
%%
%\begin{itemize}
%\item
%\ldots
%\end{itemize}

%%%%%%%%%%%%%%%%%%%%%%%%%%%%%%%%%%%%%%%%%%%%%%%%%%%%%%%%%%%%%%%%%%%%%%%%%%%%%%%%
\subsection{Revision History}

%%%%%%%%%%%%%%%%%%%%%%%%%%%%%%%%%%%%%%%%
\paragraph{v2.0:} 2018/12/30

\begin{itemize}
\item
immediate forward processing
\item
added |\childdocby| mechanism
\item
manual restructured
\end{itemize}

%%%%%%%%%%%%%%%%%%%%%%%%%%%%%%%%%%%%%%%%
\paragraph{v1.6:} 2018/01/17

\begin{itemize}
\item
application for development of include files
\item
corrections to manual
\end{itemize}

%%%%%%%%%%%%%%%%%%%%%%%%%%%%%%%%%%%%%%%%
\paragraph{v1.5:} 2017/05/21

\begin{itemize}
\item
more complete structuring introduced
\item
|\childdocof| introduced
\item
|\childdoc| renamed to |\childdocmain|
\item
|\childredirect| renamed to |\childdocforward| and |\childdocforwardprefix|
and functionality expanded
\end{itemize}

%%%%%%%%%%%%%%%%%%%%%%%%%%%%%%%%%%%%%%%%
\paragraph{v1.0:} 2017/04/27

\begin{itemize}
\item
manual and install package
\item
first version published on CTAN
\end{itemize}

%%%%%%%%%%%%%%%%%%%%%%%%%%%%%%%%%%%%%%%%
\paragraph{v0.6:} 2017/04/26

\begin{itemize}
\item
redirection mechanism added
\end{itemize}

%%%%%%%%%%%%%%%%%%%%%%%%%%%%%%%%%%%%%%%%
\paragraph{v0.5:} 2017/04/26

\begin{itemize}
\item
functionality in definition file
\end{itemize}


%%%%%%%%%%%%%%%%%%%%%%%%%%%%%%%%%%%%%%%%%%%%%%%%%%%%%%%%%%%%%%%%%%%%%%%%%%%%%%%%
%%%%%%%%%%%%%%%%%%%%%%%%%%%%%%%%%%%%%%%%%%%%%%%%%%%%%%%%%%%%%%%%%%%%%%%%%%%%%%%%
%%%%%%%%%%%%%%%%%%%%%%%%%%%%%%%%%%%%%%%%%%%%%%%%%%%%%%%%%%%%%%%%%%%%%%%%%%%%%%%%
\appendix

\settowidth\MacroIndent{\rmfamily\scriptsize 000\ }

 \DocInput{childdoc.dtx}

\end{document}
%</driver>
% \fi
%
% %%%%%%%%%%%%%%%%%%%%%%%%%%%%%%%%%%%%%%%%%%%%%%%%%%%%%%%%%%%%%%%%%%%%%%%%%%%%%%
% %%%%%%%%%%%%%%%%%%%%%%%%%%%%%%%%%%%%%%%%%%%%%%%%%%%%%%%%%%%%%%%%%%%%%%%%%%%%%%
% \section{Sample}
%\iffalse
%<*samplemain>
%\fi
%
% The following presents a sample document
% with two chapters, two parts, a title page,
% a compile flag as well as three forwarding files to set the flag.
% It consists of eight |.tex| files:
% \begin{center}
% \begin{tabular}{ll}
% |cdocsamp.tex|&main file\\
% |cdocsch1.tex|&include file for chapter 1\\
% |cdocsch2.tex|&include file for chapter 2\\
% |cdocspt3.tex|&include file for part 3\\
% |cdocspt4.tex|&include file for part 4\\
% |cdocsdrf.tex|&forwarding file for main file in draft mode\\
% |cdocsfi1.tex|&forwarding file for final version of chapter 1\\
% |cdocsfi2.tex|&forwarding file for final version of chapter 2\\
% \end{tabular}
% \end{center}
% Each of the eight files can be compiled directly by the \LaTeX{} compiler.
%
% %%%%%%%%%%%%%%%%%%%%%%%%%%%%%%%%%%%%%%
% \paragraph{Main File.}
%
% The main file is called |cdocsamp.tex|.
%
% Load the \textsf{childdoc} definitions and
% declare the filename for the main document:
%    \begin{macrocode}
\input{childdoc.def}
\childdocmain{}
%    \end{macrocode}

% Optional override for |\version| flag:
%    \begin{macrocode}
%%\ifchilddoc\else\providecommand{\version}{draft}\fi
%    \end{macrocode}

% Define the default values for the |\version| flag
% (|final| for the main file and |draft| for childs):
%    \begin{macrocode}
\ifchilddoc
\providecommand{\version}{draft}
\else
\providecommand{\version}{final}
\fi
%    \end{macrocode}

% Load the standard document class:
%    \begin{macrocode}
\documentclass[12pt]{article}
%    \end{macrocode}

% Start the document body:
%    \begin{macrocode}
\begin{document}
%    \end{macrocode}

% Declare a title page.
% Print title, part of document being processed and version flag:
%    \begin{macrocode}
\addtocounter{page}{-1}
\begin{center}
{\LARGE\bfseries{}childdoc example\par}
\vspace{1cm}
\ifchilddoc
\ifchilddocmanual part\else chapter\fi:
`\childdocname' of `\childdocjob'\par
\else
main document: `\childdocjob'\par
\fi
version: \version\par
\end{center}
\newpage
%    \end{macrocode}

% Manually include selected file,
% otherwise process as usual:
%    \begin{macrocode}
\ifchilddocmanual
\section*{part `\childdocname'}
\input{\childdocname}
\else
%    \end{macrocode}

% Include the two chapters:
%    \begin{macrocode}
\include{cdocsch1}
\include{cdocsch2}
%    \end{macrocode}

% Include the two parts unless only chapters should be displayed:
%    \begin{macrocode}
\ifchilddoc\else
\section{part three}
\input{cdocspt3}
\section{part four}
\input{cdocspt4}
\fi
%    \end{macrocode}

% Process as usual until here:
%    \begin{macrocode}
\fi
%    \end{macrocode}

% End of document body:
%    \begin{macrocode}
\end{document}
%    \end{macrocode}
%\iffalse
%</samplemain>
%\fi
%
% %%%%%%%%%%%%%%%%%%%%%%%%%%%%%%%%%%%%%%
% \paragraph{Chapter Include Files.}
%
% The include files are called |cdocsch1.tex| and |cdocsch2.tex|.
%
%\iffalse
%<*samplechap1|samplechap2>
%\fi

% Optional override for |\version| flag:
%    \begin{macrocode}
%%\providecommand{\version}{final}
%    \end{macrocode}

% Include the main document:
%    \begin{macrocode}
\input{childdoc.def}
\childdocof{cdocsamp}
%    \end{macrocode}

%\iffalse
%</samplechap1|samplechap2>
%\fi
%
%\iffalse
%<*samplechap1>
%\fi
% Some text for chapter 1:
%    \begin{macrocode}
\section{one}
some text in chapter one
%    \end{macrocode}

%\iffalse
%</samplechap1>
%\fi
% Some text for chapter 2:
%\iffalse
%<*samplechap2>
%\fi
%    \begin{macrocode}
\section{two}
more text in chapter two
%    \end{macrocode}

%\iffalse
%</samplechap2>
%\fi
%
% %%%%%%%%%%%%%%%%%%%%%%%%%%%%%%%%%%%%%%
% \paragraph{Part Include Files.}
%
% The include files are called |cdocspt3.tex| and |cdocspt4.tex|.
%
%\iffalse
%<*samplepart3|samplepart4>
%\fi

% Optional override for |\version| flag:
%    \begin{macrocode}
%%\providecommand{\version}{final}
%    \end{macrocode}

% Include the main document:
%    \begin{macrocode}
\input{childdoc.def}
\childdocby{cdocsamp}
%    \end{macrocode}

%\iffalse
%</samplepart3|samplepart4>
%\fi
%
%\iffalse
%<*samplepart3>
%\fi
% Some text for part 3:
%    \begin{macrocode}
some text in part three
%    \end{macrocode}

%\iffalse
%</samplepart3>
%\fi
% Some text for part 4:
%\iffalse
%<*samplepart4>
%\fi
%    \begin{macrocode}
more text in part four
%    \end{macrocode}

%\iffalse
%</samplepart4>
%\fi
%
% %%%%%%%%%%%%%%%%%%%%%%%%%%%%%%%%%%%%%%
% \paragraph{Forwarding for a Complete Draft.}
%
% The following forwarding file |cdocsdrf.tex|
% compiles the main document in draft mode:
%\iffalse
%<*sampledraft>
%\fi
%    \begin{macrocode}
\def\version{draft}
\input{childdoc.def}
\childdocforward{cdocsamp}
%    \end{macrocode}

%\iffalse
%</sampledraft>
%\fi
%
% %%%%%%%%%%%%%%%%%%%%%%%%%%%%%%%%%%%%%%
% \paragraph{Forwarding for Final Version of the Chapters.}
%
% The following forwarding files |cdocsfn1.tex| and |cdocsfn2.tex|
% (with identical content)
% compile the final versions of the child documents
% |cdocsch1.tex| and |cdocsch2.tex|, respectively:
%\iffalse
%<*samplefinal>
%\fi
%    \begin{macrocode}
\def\version{final}
\input{childdoc.def}
\childdocforwardprefix[cdocsamp]{cdocsfn}{cdocsch}
%    \end{macrocode}

%\iffalse
%</samplefinal>
%\fi
%
% %%%%%%%%%%%%%%%%%%%%%%%%%%%%%%%%%%%%%%
% \paragraph{Command Line Processing.}
%
% The following three command lines generate the output files
% |cdocscld|, |cdocscl1| and |cdocscl2|
% which should be identical to
% |cdocsdrf|, |cdocsch1| and |cdocsfn2|, respectively:
% \begin{center}
% \begin{tabular}{l}
% |latex -jobname cdocscld \|\\
% |  "\def\version{draft}\input{childdoc.def}\childdocforward{cdocsamp}"|\\
% |latex -jobname cdocscl1 \|\\
% |  "\input{childdoc.def}\childdocforward[cdocsamp]{cdocsch1}"|\\
% |latex -jobname cdocscl2 \|\\
% |  "\def\version{final}\input{childdoc.def}\childdocforward{cdocsch2}"|
% \end{tabular}
% \end{center}
% Note that the trailing backslash on each first line
% merely continues the input to the second line
% (for convenient cut ant paste).
% Furthermore, the command |latex| can be replaced by any
% of its alternative versions such as |pdflatex|.
%
% %%%%%%%%%%%%%%%%%%%%%%%%%%%%%%%%%%%%%%%%%%%%%%%%%%%%%%%%%%%%%%%%%%%%%%%%%%%%%%
% %%%%%%%%%%%%%%%%%%%%%%%%%%%%%%%%%%%%%%%%%%%%%%%%%%%%%%%%%%%%%%%%%%%%%%%%%%%%%%
% \section{Implementation}
%\iffalse
%<*package>
%\fi
%
% This section describes the definitions file |childdoc.def|.

% The definitions cannot be loaded using |\usepackage| or |\RequirePackage|
% which has a mechanism to prevent loading a style file more than once.
% When loading the definitions by means of |\input|
% multiple instances have to be prevented manually:
%\iffalse
%This code needs to be before the `\ProvidesFile' directive
%which is defined at the beginning of this file.
%Therefore it is also placed there and commented out here.
%</package>
%<*discard>
%\fi
%    \begin{macrocode}
\ifdefined\childdocmain\endinput\fi
%    \end{macrocode}
%\iffalse
%</discard>
%<*package>
%\fi
%
% \macro{\ifchilddoc}
% \macro{\ifchilddocmanual}
% The conditional |\ifchilddoc| tells whether a
% child (true) or main (false) document is being compiled.
% The conditional |\ifchilddocmanual| tells whether
% the |\includeonly| mechanism is used (false) or
% the selection of child files must be performed manually (true).
% The definitions initialise to false:
%    \begin{macrocode}
\newif\ifchilddoc
\newif\ifchilddocmanual
%    \end{macrocode}

% \macro{\childdocname}
% \macro{\childdocjob}
% The macro |\childdocname| stores the name of the main document
% to be compiled. The macro |\childdocjob| stores the name of
% the document on which the \LaTeX{} compiler was originally invoked.
% The content of |\jobname| cannot be compared
% to filenames specified in the source due to different catcodes.
% The following code rescans |\jobname|, stores the result
% in |\childdocname| and saves a copy in |\childdocjob|:
%    \begin{macrocode}
\edef\childdocname{\scantokens\expandafter{\jobname\noexpand}}
\let\childdocjob\childdocname
%    \end{macrocode}

% \macro{\childdocdisable}
% The macro |\childdocdisable| prevents the main file
% from being processed more than once.
% At this stage, the main document command |\childdocmain|
% is assumed to be called once again where it should do nothing.
% Any subsequent call to it should prevent
% a secondary processing of the main document
% It overwrites the forwarding commands
% |\childdocof| and |\childdocforward|
% with empty macros to prevent further inclusions of the main document:
%    \begin{macrocode}
\newcommand{\childdocdisable}
{
  \renewcommand{\childdocmain}[1]{\renewcommand{\childdocmain}[1]{\endinput}}
  \renewcommand{\childdocof}[1]{}
  \renewcommand{\childdocby}[2][]{}
  \renewcommand{\childdocforward}[2][]{}
  \renewcommand{\childdocdisable}{}
}
%    \end{macrocode}

% \macro{\childdocmain}
% The macro |\childdocmain| is to be called at the top of the main file
% with nothing or the main filename (without extension) as argument.
% First, it breaks loops.
% If the argument is not empty and does not match |\childdocname|
% (which is set by the first inclusion of |childdoc.def|),
% |\ifchilddoc| is set to true, |\includeonly| is applied to the child file
% and |\jobname| is set to the main file
% (for proper handling of |.aux| files):
%    \begin{macrocode}
\newcommand{\childdocmain}[1]
{
  \childdocdisable\childdocmain{}
  \if?#1?\else
    \begingroup
      \def\childdoctmp{#1}
      \ifx\childdoctmp\childdocname
        \def\childdoctmp{}
      \else
        \def\childdoctmp
        {
          \childdoctrue
          \includeonly{\childdocname}
          \def\childdocjob{#1}
          \def\jobname{#1}
        }
      \fi
      \expandafter
    \endgroup
    \childdoctmp
  \fi
}
%    \end{macrocode}

% \macro{\childdocof}
% The command |\childdocof| redirects
% compilation to the main file |#1|.
%    \begin{macrocode}
\newcommand{\childdocof}[1]
{
  \childdocdisable
  \childdoctrue
  \includeonly{\childdocname}
  \def\jobname{#1}
  \def\childdocjob{#1}
  \input{#1}
}
%    \end{macrocode}

% \macro{\childdocby}
% The command |\childdocby| ....
%    \begin{macrocode}
\newcommand{\childdocby}[2][]
{
  \childdocdisable
  \childdoctrue
  \childdocmanualtrue
  \if?#1?\else
    \def\jobname{#2}
  \fi
  \def\childdocjob{#2}
  \input{#2}
  \endinput
}
%    \end{macrocode}

% \macro{\childdocforward}
% The command |\childdocforward| redirects
% compilation to the main file or
% (if the optional argument is given) a child file.
% Parameters are set as if the main file
% or a child file starting with |\childdocof| was compiled.
% Then compilation is handed over to the main file:
%    \begin{macrocode}
\newcommand{\childdocforward}[2][]
{
  \begingroup
    \if?#1?
      \def\childdoctmp
      {
        \def\childdocname{#2}
        \def\childdocjob{#2}
        \def\jobname{#2}
        \input{#2}
        \endinput
      }
    \else
      \def\childdoctmp
      {
        \childdocdisable
        \def\childdocname{#2}
        \childdoctrue
        \includeonly{#2}
        \def\childdocjob{#1}
        \def\jobname{#1}
        \input{#1}
        \endinput
      }
    \fi
    \expandafter
  \endgroup
  \childdoctmp
}
%    \end{macrocode}

% \macro{\childdocforwardprefix}
% The command |\childdocforwardprefix| redirects
% compilation to the main or a child file by means of a pattern.
% The prefix |#1| in the current filename is replaced by |#2|
% and the suffix of the current filename is kept
% (it is assumed that the filename does not contain the substring `|~~~|'
% which is used as a delimiter).
% Compilation is handed over to the new file by |\childdocforward|:
%    \begin{macrocode}
\newcommand{\childdocforwardprefix}[3][]
{
  \begingroup
    \def\childdocextract #2##1~~~{\def\childdoctmp{\childdocforward[#1]{#3##1}}}
    \expandafter\childdocextract\childdocname~~~
    \expandafter
  \endgroup
  \childdoctmp
}
%    \end{macrocode}

% \macro{\childdoc}
% The deprecated macro |\childdoc| is a legacy version of |\childdocmain|:
%    \begin{macrocode}
\newcommand{\childdoc}{\childdocmain}
%    \end{macrocode}

% \macro{\childdocredirect}
% The deprecated macro |\childdocredirect| is a legacy version
% of |\childdocforward| and |\childdocforwardprefix|:
%    \begin{macrocode}
\newcommand{\childdocredirect}[2][]
{
  \begingroup
    \if?#1?
      \def\childdoctmp{\childdocforward{#2}}
    \else
      \def\childdoctmp{\childdocforwardprefix{#1}{#2}}
    \fi
    \expandafter
  \endgroup
  \childdoctmp
}
%    \end{macrocode}

%\iffalse
%</package>
%\fi
%
\endinput
\childdocforward[cdocsamp]{cdocsch1}"|\\
% |latex -jobname cdocscl2 \|\\
% |  "\def\version{final}% \iffalse
%
% childdoc.dtx Copyright (C) 2017-2018 Niklas Beisert
%
% This work may be distributed and/or modified under the
% conditions of the LaTeX Project Public License, either version 1.3
% of this license or (at your option) any later version.
% The latest version of this license is in
%   http://www.latex-project.org/lppl.txt
% and version 1.3 or later is part of all distributions of LaTeX
% version 2005/12/01 or later.
%
% This work has the LPPL maintenance status `maintained'.
%
% The Current Maintainer of this work is Niklas Beisert.
%
% This work consists of the files childdoc.dtx and childdoc.ins
% and the derived files childdoc.def and cdocsamp.tex with
% cdocsch1.tex, cdocsch2.tex, cdocsdrf.tex, cdocsfn1.tex, cdocsfn2.tex.
%
%<package>\ifdefined\childdocmain\endinput\fi
%<package>\ProvidesFile{childdoc.def}[2018/12/30 v2.0 child document driver]
%<samplemain>\ProvidesFile{cdocsamp.tex}[2018/12/30 v2.0 sample for childdoc]
%<*driver>
%\ProvidesFile{childdoc.drv}[2018/12/30 v2.0 childdoc reference manual file]
\PassOptionsToClass{10pt,a4paper}{article}
\documentclass{ltxdoc}

\usepackage[margin=35mm]{geometry}
\usepackage{hyperref}
\usepackage{hyperxmp}
\usepackage[usenames]{color}

\hypersetup{colorlinks=true}
\hypersetup{pdfstartview=FitH}
\hypersetup{pdfpagemode=UseNone}
\hypersetup{pdfsource={}}
\hypersetup{pdflang={en-UK}}
\hypersetup{pdfcopyright={Copyright 2017-2018 Niklas Beisert.
  This work may be distributed and/or modified under the
  conditions of the LaTeX Project Public License, either version 1.3
  of this license or (at your option) any later version.}}
\hypersetup{pdflicenseurl={http://www.latex-project.org/lppl.txt}}
\hypersetup{pdfcontactaddress={ETH Zurich, ITP, HIT K,
  Wolfgang-Pauli-Strasse 27}}
\hypersetup{pdfcontactpostcode={8093}}
\hypersetup{pdfcontactcity={Zurich}}
\hypersetup{pdfcontactcountry={Switzerland}}
\hypersetup{pdfcontactemail={nbeisert@itp.phys.ethz.ch}}
\hypersetup{pdfcontacturl={http://people.phys.ethz.ch/\xmptilde nbeisert/}}

\newcommand{\secref}[1]{\hyperref[#1]{section \ref*{#1}}}

\parskip1ex
\parindent0pt
\let\olditemize\itemize
\def\itemize{\olditemize\parskip0pt}

\begin{document}

\title{The \textsf{childdoc} Package}
\hypersetup{pdftitle={The childdoc Package}}
\author{Niklas Beisert\\[2ex]
  Institut f\"ur Theoretische Physik\\
  Eidgen\"ossische Technische Hochschule Z\"urich\\
  Wolfgang-Pauli-Strasse 27, 8093 Z\"urich, Switzerland\\[1ex]
  \href{mailto:nbeisert@itp.phys.ethz.ch}
  {\texttt{nbeisert@itp.phys.ethz.ch}}}
\hypersetup{pdfauthor={Niklas Beisert}}
\hypersetup{pdfsubject={Manual for the LaTeX2e Package childdoc}}
\date{30 December 2018, \textsf{v2.0}}
\maketitle

\begin{abstract}\noindent
\textsf{childdoc} is a \LaTeXe{} package
that enables the direct compilation
of document sections included by |\include|
to individual files.
\end{abstract}

\begingroup
\parskip0ex
\tableofcontents
\endgroup

%%%%%%%%%%%%%%%%%%%%%%%%%%%%%%%%%%%%%%%%%%%%%%%%%%%%%%%%%%%%%%%%%%%%%%%%%%%%%%%%
%%%%%%%%%%%%%%%%%%%%%%%%%%%%%%%%%%%%%%%%%%%%%%%%%%%%%%%%%%%%%%%%%%%%%%%%%%%%%%%%
\section{Introduction}

\LaTeX{} provides a mechanism to structure a large document (such as a book)
into a main file and several child files (containing the chapters)
using the |\include| command.
This mechanism is beneficial for documents
which span hundreds of pages in order to
make the source file(s) more manageable.
Moreover, compilation can be restricted to
selected child files by means of the |\includeonly| command.
The latter feature can be used to reduce the compilation time while editing
(this was significantly more useful in the earlier days of \LaTeX{})
or to generate a smaller document which is easier to navigate.
Another application of |\includeonly| is to generate
documents consisting of selected parts of the complete document.

However, there are a few drawbacks of the plain |\include| mechanism:
\begin{itemize}
\item
The child files cannot be compiled on their own,
they can only be compiled via the main file.
A naive editing environment
(such as a text editor with an option
to have the current file processed by \LaTeX)
may require one to switch to the main file before compiling;
attempting to compile the child file produces errors.
\item
The main file must be modified (each time)
to adjust the |\includeonly| command
to the present needs. This easily leaves the main file in a messy state.
\item
The generated document will always carry the filename
of the main document. This is inconvenient if
several child files are to be compiled and
to be kept for distribution.
\end{itemize}

The present package provides a simple interface
to make child files individually compilable by \LaTeX{}.
Compiling a child file then has the same effect as compiling
the main file with an |\includeonly| command
to select the appropriate child.
Moreover the generated document will carry the name of the child
rather than the main file.
This resolves all three above issues.

This feature is meant to make the editing of books,
thesis documents and lecture notes somewhat more convenient.
However, the package can also be used efficiently for
composing a series of documents (such as exercise sheets)
which are typically distributed individually.
It then assists the author in generating the individual documents
(potentially in different versions)
as well as a document containing the collected series.
Another application is in developing style files
or other kinds of included material
where compilation of the style file could redirect
to a sample or test file.

%%%%%%%%%%%%%%%%%%%%%%%%%%%%%%%%%%%%%%%%%%%%%%%%%%%%%%%%%%%%%%%%%%%%%%%%%%%%%%%%
%%%%%%%%%%%%%%%%%%%%%%%%%%%%%%%%%%%%%%%%%%%%%%%%%%%%%%%%%%%%%%%%%%%%%%%%%%%%%%%%
\section{Usage}

First of all, the package \textsf{childdoc} is \emph{not} a standard
\LaTeXe{} |.sty| style file! Therefore it needs to be invoked in
a non-standard way.

%%%%%%%%%%%%%%%%%%%%%%%%%%%%%%%%%%%%%%%%%%%%%%%%%%%%%%%%%%%%%%%%%%%%%%%%%%%%%%%%
\subsection{Included Files}
\label{sec:include}

%%%%%%%%%%%%%%%%%%%%%%%%%%%%%%%%%%%%%%%%
\DescribeMacro{\childdocmain}
To use the package, add the commands
\begin{center}
\begin{tabular}{l}
|\input{childdoc.def}|\\
|\childdocmain{}|\\
\end{tabular}
\end{center}
at the very top of the main \LaTeX{} file,
in particular \emph{before} the |\documentclass| statement!
The argument of |\childdocmain| should be left empty
(but it must be present).

%%%%%%%%%%%%%%%%%%%%%%%%%%%%%%%%%%%%%%%%
\DescribeMacro{\childdocof}
Furthermore, add the commands
\begin{center}
\begin{tabular}{l}
|\input{childdoc.def}|\\
|\childdocof{|\textit{main}|}|\\
\end{tabular}
\end{center}
at the top of every child file \textit{child}
which is included by |\include{|\textit{child}|}|
from within the main file
(or at least for those files to be compiled individually).
The argument \textit{main} must be the filename of the main file.

There are a couple of
considerations in setting up the main and child documents:

%%%%%%%%%%%%%%%%%%%%%%%%%%%%%%%%%%%%%%%%
\paragraph{Restrictions.}

Please note the following restrictions:
\begin{itemize}
\item
|\childdocmain| must be called with one argument \textit{main}
to ensure compatibility with earlier version of the package.
It must either be empty (|\childdocmain{}|)
or precisely match the filename of the main file in which it is specified.
See \secref{sec:detection} for further information.
\item
The filename \textit{main} must be specified without the |.tex| extension.
\item
The filename \textit{main} is case sensitive
(even in case-insensitive file systems)
due to internal string comparison.
\item
The argument \textit{main} should be fully expanded, it cannot be a macro.
\item
Subdirectories and special characters should be avoided in filenames.
\item
The command |\childdocmain{|\textit{main}|}| must be followed by a whitespace.
It should not be followed immediately by another command
or by a comment mark `|%|'.
This is because the \TeX{} parser reads the token immediately following
the argument of |\childdocmain| and puts it
at the beginning of every child section;
however, a white\-space is ignored.
\end{itemize}

%%%%%%%%%%%%%%%%%%%%%%%%%%%%%%%%%%%%%%%%
\paragraph{Content of Main File.}

It is advisable to place all content in the child files included by |\include|.
Any output contained in the main file will appear in all child documents
unless suppressed manually;
it cannot be suppressed automatically by the |\includeonly| directive
and thus should normally be avoided.
A method to include some content in the main file
by means of conditional processing is described in \secref{sec:conditional}.

%%%%%%%%%%%%%%%%%%%%%%%%%%%%%%%%%%%%%%%%
\paragraph{Page Numbering.}

When only a part of the document is compiled,
the appropriate numbering of pages
(as well as other status parameters)
is determined from the |.aux| files.
The latter contain information from previous passes.
However this information needs to propagate through
all intermediate child documents.
Therefore the page numbering in child documents may well
be inconsistent until the complete document is compiled at least once.

A useful (if unconventional) way to always ensure a consistent
page numbering is to restart the numbering in each child document
and denote the pages by `\textit{child}|.|\textit{page}'
where \textit{child} represents the chapter/section number of the child file.
This can be achieved by the command
|\numberwithin{page}{|\textit{child}|}|
of the \textsf{amsmath} package
where \textit{child} can be |chapter| or |section|
depending on the chosen structuring.
Alternatively, one can modify the macro |\thepage| appropriately
and reset the counter |page| at the start of each child file.

%%%%%%%%%%%%%%%%%%%%%%%%%%%%%%%%%%%%%%%%%%%%%%%%%%%%%%%%%%%%%%%%%%%%%%%%%%%%%%%%
\subsection{Conditional Processing}
\label{sec:conditional}

The package provides a mechanism to compile different versions
of a document. To customise the versions further some conditional processing
can come in handy to distinguish which version is being compiled.
The package provides two macros to describe the compilation context:

%%%%%%%%%%%%%%%%%%%%%%%%%%%%%%%%%%%%%%%%
\DescribeMacro{\ifchilddoc}
The conditional |\ifchilddoc| distinguishes between the compilation of
child documents and the main document:
%
\begin{center}
|\ifchilddoc |\textit{child-code}| |[|\||else |\textit{main-code}]| \||fi|
\end{center}

%%%%%%%%%%%%%%%%%%%%%%%%%%%%%%%%%%%%%%%%
\DescribeMacro{\childdocname}
\DescribeMacro{\childdocjob}
The macro |\childdocname| contains the filename (without extension)
of the main or child file being processed.
Note that |\childdocjob| will always contain the name of the main file.

%%%%%%%%%%%%%%%%%%%%%%%%%%%%%%%%%%%%%%%%
\paragraph{Title Page.}

Conditional processing can be used to include a title or banner page
in the main document when proper precautions are taken.
Importantly, the code in the main file should ensure that the page counter
(as well as other status parameters which are stored in the |.aux| files)
takes the same value after the conditional processing.
Otherwise the page numbers may take divergent values
depending on which part is compiled.

For example, a title page could be declared by:
%
\begin{center}
\begin{tabular}{l}
|\ifchilddoc\||else|\\
|\addtocounter{page}{-1}|\\
\textit{code for title page}\\
|\newpage|\\
|\||fi|
\end{tabular}
\end{center}
%
A banner page for the child documents can be generated by:
%
\begin{center}
\begin{tabular}{l}
|\ifchilddoc|\\
|\addtocounter{page}{-1}|\\
\textit{code for banner page}\\
|\newpage|\\
|\||fi|
\end{tabular}
\end{center}
%
Here one could write a message such as:
\begin{center}
|This is the part \childdocname{} of \childdocjob{}.|
\end{center}

%%%%%%%%%%%%%%%%%%%%%%%%%%%%%%%%%%%%%%%%%%%%%%%%%%%%%%%%%%%%%%%%%%%%%%%%%%%%%%%%
\subsection{Flags}
\label{sec:flags}

The package makes it easy to generate different versions
of the main or child documents.
To this end compilation flags can be defined
and assigned different default values.
They will be particularly useful in conjunction
with the forwarding mechanism described in \secref{sec:forward}.

For example, it may be useful to have a flag |\version|
which can be set to |draft| or |final|.
The document source will contain some conditional code
depending on the value of |\version|.
Suppose further, the flag should default to |final| for the main file
and to |draft| for child files
which is a natural assignment for editing the document.
This is achieved by placing the following code
in the preamble of the main document
(below the |\childdocmain| directive):
%
\begin{center}
\begin{tabular}{l}
|\ifchilddoc|\\
|\providecommand{\version}{draft}|\\
|\||else|\\
|\providecommand{\version}{final}|\\
|\||fi|
\end{tabular}
\end{center}
%
The definition by |\providecommand| makes sure
that previous definitions are not overwritten.
Further statements |\providecommand{\version}{...}|
can thus be added before the above code to override it.

For the main file, one might add a line
(between |\childdocmain| and the above block)
%
\begin{center}
|%\ifchilddoc\||else\providecommand{\version}{draft}\||fi|
\end{center}
%
which can be uncommented to produce a draft version.
Likewise one can add a line to the very top of a child file
(above the |\childdocof{|\textit{main}|}| directive)
%
\begin{center}
|%\providecommand{\version}{final}|
\end{center}
%
which can be uncommented to produce the final version of this child document.

%%%%%%%%%%%%%%%%%%%%%%%%%%%%%%%%%%%%%%%%%%%%%%%%%%%%%%%%%%%%%%%%%%%%%%%%%%%%%%%%
\subsection{Forwarding}
\label{sec:forward}

Different versions of the main or child documents
using compilation flags as described in \secref{sec:flags}
can be (permanently) stored in different files
for convenient compilation, viewing and distribution.
To this end, the package defines a command
to pass on compilation to a different file:

%%%%%%%%%%%%%%%%%%%%%%%%%%%%%%%%%%%%%%%%
\DescribeMacro{\childdocforward}
The command |\childdocforward| redirects processing to
another source file:
%
\begin{center}
\begin{tabular}{l}
|\input{childdoc.def}|\\
|\childdocforward[|\textit{main}|]{|\textit{dest}|}|\\
\end{tabular}
\end{center}
%
The argument \textit{dest} is the destination file
(without extension).
It should be the main file or one of the child files.
Note that further \textsf{childdoc} directives
such as |\childdocof| and |\childdocforward|
in the indicated file will be processed in this form.
The optional argument \textit{main}
passes on directly to the main file \textit{main}
while pretending to compile the child \textit{dest}.
This form behaves as if \textit{dest}
issues |\childdocof{|\textit{main}|}| right away,
and no further \textsf{childdoc} directives will be processed.

%%%%%%%%%%%%%%%%%%%%%%%%%%%%%%%%%%%%%%%%
\DescribeMacro{\...prefix}
In the alternative form |\childdocforwardprefix|,
%
\begin{center}
\begin{tabular}{l}
|\input{childdoc.def}|\\
|\childdocforwardprefix[|\textit{main}|]{|\textit{prefix}|}{|\textit{dest}|}|
\end{tabular}
\end{center}
%
the destination file is determined by a pattern
depending on the current file:
To make this work, the current file must be called
`{\textit{prefix}\hspace{0.2em}\textit{suffix}}'
with \textit{prefix} matching precisely the argument.
Processing is then passed on to the file
`{\textit{dest}\hspace{0.2em}\textit{suffix}}'.
Surely, the same effect is achieved by
directly specifying the
argument `{\textit{dest}\hspace{0.2em}\textit{suffix}}'
in the first form.
However, that requires to set up a different file
for each child. With the alternative form of the command
all these files can have exactly the same content
which simplifies setting them up and maintaining them.

For example, the following file |draft.tex|
with a compilation flag |\version| as described in \secref{sec:flags}
compiles the main document as a draft:
%
\begin{center}
\begin{tabular}{l}
|\def\version{draft}|\\
|\input{childdoc.def}|\\
|\childdocforward{|\textit{main}|}|
\end{tabular}
\end{center}
%
Likewise, the following files |final|\textit{nn}|.tex|
compile the final version of the child document
|child|\textit{nn}|.tex|:
%
\begin{center}
\begin{tabular}{l}
|\def\version{final}|\\
|\input{childdoc.def}|\\
|\childdocforwardprefix{final}{child}|
\end{tabular}
\end{center}
%

Note that when several versions of a main file and/or of each child file
are to be generated, it may be convenient to set up a |Makefile| or
shell script to automatise the process.

%%%%%%%%%%%%%%%%%%%%%%%%%%%%%%%%%%%%%%%%%%%%%%%%%%%%%%%%%%%%%%%%%%%%%%%%%%%%%%%%
\subsection{Command Line Processing}
\label{sec:commandline}

The effect of redirection files can also be achieved by invoking
the \LaTeX{} compiler with a more elaborate command line.
Most conveniently this should be done as part
of a shell script or a |Makefile|.

When using \textsf{childdoc} in the main file, the following
command lines effectively perform a redirection
(note that depending on the shell being used,
backslashes may have to be doubled: `|\|' $\to$ `|\\|'):
%
\begin{center}
|... -jobname "|\textit{target}|" |\\|"|[\textit{flags}]%
|\input{childdoc.def}\childdocforward[|\textit{main}|]{|\textit{dest}|}"|
\end{center}
%
Here \textit{target} is the name of the output file,
\textit{main} is the name of the main file
and \textit{dest} is the name of the main or child file to be processed
(all filenames without extensions).
The optional argument \textit{main} can be omitted
if \textit{main} matches \textit{dest}.
Optionally, compilation \textit{flags} can be defined via |\def| commands.
This command line makes the \TeX{} engine believe
it is compiling the file \textit{target}
whose content is specified as the latter parameter.
The provided code then forwards the processing to
\textit{main} or \textit{dest} as described in \secref{sec:forward}.

%%%%%%%%%%%%%%%%%%%%%%%%%%%%%%%%%%%%%%%%%%%%%%%%%%%%%%%%%%%%%%%%%%%%%%%%%%%%%%%%
\subsection{Include by Input}
\label{sec:input}

Including child documents by |\include| has some restrictions by design.
Most notably, the content of a child document always occupies
its own set of pages; pages cannot be shared between child documents.
Usually, this behaviour makes perfect sense
because each child document contain an essential part of the document.
However, in some situations it may be desirable to compose
a document from a collection of parts
without having mandatory page breaks between then.
For this case, the package
provides a mechanism to include parts
by |\input| which can also be processed individually.
However, by construction this mechanism
requires manual handling of the content to be output.

%%%%%%%%%%%%%%%%%%%%%%%%%%%%%%%%%%%%%%%%
\DescribeMacro{\ifchilddocmanual}
The main file should be prepared as usual, see \secref{sec:include}.
However, the document body must make a distinction
between processing of an individual part and of the main document, e.g.:
%
\begin{center}
\begin{tabular}{l}
|\ifchilddocmanual|\\
|\input{\childdocname}|\\
|\||else|\\
\textit{document body with }|\input{|\textit{part}|}|\\
|\||fi|
\end{tabular}
\end{center}
%
The conditional |\ifchilddocmanual| is true whenever
a part to be included by |\input| is being compiled,
and the name of the part is stored in |\childdocname|.

%%%%%%%%%%%%%%%%%%%%%%%%%%%%%%%%%%%%%%%%
\DescribeMacro{\childdocby}
Each part to be included by |\input| should start with:
%
\begin{center}
\begin{tabular}{l}
|\input{childdoc.def}|\\
|\childdocby{|\textit{main}|}|\\
\end{tabular}
\end{center}
%
The directive |\childdocby| is similar to |\childdocof|
described in \secref{sec:include},
but the subsequent selection of content must be done manually.
To that end, both |\ifchilddoc| and |\ifchilddocmanual|
will be true upon processing of a part,
and the name of the part is stored in |\childdocname|.
Note that |\jobname| will be set to the filename of the current part
so that each part receives an individual |.aux| file
that does not interfere with the |.aux| file(s) of the main document.
This behaviour can be altered by the alternative form
|\childdocby[*]{|\textit{main}|}| (with a non-empty optional argument)
which uses the |.aux| file of the main document
by setting |\jobname| to \textit{main}.

%%%%%%%%%%%%%%%%%%%%%%%%%%%%%%%%%%%%%%%%%%%%%%%%%%%%%%%%%%%%%%%%%%%%%%%%%%%%%%%%
\subsection{Driver Development}
\label{sec:driver}

The \textsf{childdoc} mechanism can also be use for the development
of definition files such as \LaTeX{} styles or classes.
This case differs from the above setup with multiple parts
included by |\include| in that no |\includeonly| should be invoked.
This can be achieved by starting the include file
(before |\ProvidesPackage|) with:
%
\begin{center}
\begin{tabular}{l}
|\input{childdoc.def}|\\
|\childdocforward{|\textit{main}|}|\\
\end{tabular}
\end{center}
%
or alternatively with:
%
\begin{center}
\begin{tabular}{l}
|\input{childdoc.def}|\\
|\childdocby{|\textit{main}|}|\\
\end{tabular}
\end{center}
%
Both forms have slightly different effects as described above.
The main file is prepared as usual, see \secref{sec:include}.

%%%%%%%%%%%%%%%%%%%%%%%%%%%%%%%%%%%%%%%%%%%%%%%%%%%%%%%%%%%%%%%%%%%%%%%%%%%%%%%%
\subsection{Legacy Detection}
\label{sec:detection}

The directive |\childdocmain| in the main file can detect
whether the complete document or merely a child is to be compiled
even without using the directive |\childdocof|.
This method is deprecated because it is less robust
and there is no compelling reason to use it;
it is merely provided for backward compatibility
and it may be removed in future versions.

If the detection mechanism is to be used,
it is mandatory to correctly specify
the filename of the main file as the argument of |\childdocmain|:
%
\begin{center}
\begin{tabular}{l}
|\input{childdoc.def}|\\
|\childdocmain{|\textit{main}|}|\\
\end{tabular}
\end{center}
%
If |\jobname| does not match the argument \textit{main} of |\childdocmain|,
it is assumed that |\jobname| points to the child file to be compiled.
When using |\childdocmain| with the main file specified as argument,
it suffices to start a child file
with just |\input{|\textit{main}|}|
without loading of the package and using |\childdocof|.
If instead all processing is done
with the appropriate \textsf{childdoc} directives,
the argument of \textit{main} of |\childdocmain| can be empty.

An alternative version of the command line processing described
in \secref{sec:commandline} using the detection mechanism reads:
%
\begin{center}
|... -jobname "|\textit{target}|" "|[\textit{flags}]%
[|\def\jobname{|\textit{dest}|}|]|\input{|\textit{main}|}"|
\end{center}

%%%%%%%%%%%%%%%%%%%%%%%%%%%%%%%%%%%%%%%%%%%%%%%%%%%%%%%%%%%%%%%%%%%%%%%%%%%%%%%%
\subsection{Manual Code}
\label{sec:manual}

In case one cannot be certain whether the definitions file |childdoc.def|
is installed on the target \TeX{} distribution
and one prefers not to ship it,
it is conceivable to paste a few relevant commands into the sources.

To that end, drop all statements |\input{childdoc.def}|
and perform the replacements as outlined below.
Instead of |\childdocmain{|\textit{main}|}| add the following code
to the top of the main file:
%
\begin{center}
\begin{tabular}{l}
|\||ifdefined\childdocname\endinput\||fi\newif\ifchilddoc|\\
|\edef\childdocname{\scantokens\expandafter{\jobname\noexpand}}|\\
|\def\childdocmain{|\textit{main}|}\||ifx\childdocmain\childdocname\||else|\\
|\childdoctrue\includeonly{\childdocname}\let\jobname\childdocmain\||fi|\\
\end{tabular}
\end{center}
%
Instead of |\childdocof{|\textit{main}|}| just include the main file
at the top of each child file:
%
\begin{center}
|\input{|\textit{main}|}|
\end{center}
%
A simple redirection |\childdocforward{|\textit{dest}|}| is achieved by:
%
\begin{center}
|\def\jobname{|\textit{dest}|}\input{\jobname}|
\end{center}
%
The redirection with prefix
|\childdocforwardprefix[|\textit{prefix}|]{|\textit{dest}|}|
is accomplished by:
%
\begin{center}
\begin{tabular}{l}
|{\edef\jobname{\scantokens\expandafter{\jobname\noexpand}}|\\
|\def\redirectjob |\textit{prefix}|#1~~~{\gdef\jobname{|\textit{dest}|#1}}|\\
|\expandafter\redirectjob\jobname~~~}\input{\jobname}|
\end{tabular}
\end{center}

In an alternative approach,
child documents can be compiled by a specific command line
without additional code or specific definitions:
%
\begin{center}
|... -jobname "|\textit{target}|" "|[\textit{flags}]%
|\includeonly{|\textit{dest}|}\input{|\textit{main}|}"|
\end{center}
%

%%%%%%%%%%%%%%%%%%%%%%%%%%%%%%%%%%%%%%%%%%%%%%%%%%%%%%%%%%%%%%%%%%%%%%%%%%%%%%%%
%%%%%%%%%%%%%%%%%%%%%%%%%%%%%%%%%%%%%%%%%%%%%%%%%%%%%%%%%%%%%%%%%%%%%%%%%%%%%%%%
\section{Information}

%%%%%%%%%%%%%%%%%%%%%%%%%%%%%%%%%%%%%%%%%%%%%%%%%%%%%%%%%%%%%%%%%%%%%%%%%%%%%%%%
\subsection{Copyright}

Copyright \copyright{} 2017--2018 Niklas Beisert

This work may be distributed and/or modified under the
conditions of the \LaTeX{} Project Public License, either version 1.3
of this license or (at your option) any later version.
The latest version of this license is in
  \url{http://www.latex-project.org/lppl.txt}
and version 1.3 or later is part of all distributions of \LaTeX{}
version 2005/12/01 or later.

This work has the LPPL maintenance status `maintained'.

The Current Maintainer of this work is Niklas Beisert.

This work consists of the files |README.txt|, |childdoc.ins| and |childdoc.dtx|
as well as the derived files |childdoc.def|, |cdocsamp.tex|
with |cdocsch1.tex|, |cdocsch2.tex|, |cdocspt3.tex|, |cdocspt4.tex|,
|cdocsdrf.tex|, |cdocsfn1.tex|, |cdocsfn2.tex|
as well as |childdoc.pdf|.

%%%%%%%%%%%%%%%%%%%%%%%%%%%%%%%%%%%%%%%%%%%%%%%%%%%%%%%%%%%%%%%%%%%%%%%%%%%%%%%%
\subsection{Files and Installation}

The package consists of the files:
%
\begin{center}
\begin{tabular}{ll}
    |README.txt|   & readme file \\
    |childdoc.ins| & installation file \\
    |childdoc.dtx| & source file \\
    |childdoc.def| & definition file \\
    |cdocsamp.tex| & sample main file \\
    |cdocsch1.tex| & sample include file \\
    |cdocsch2.tex| & sample include file \\
    |cdocspt3.tex| & sample part file \\
    |cdocspt4.tex| & sample part file \\
    |cdocsdrf.tex| & sample redirection file \\
    |cdocsfn1.tex| & sample redirection file \\
    |cdocsfn2.tex| & sample redirection file \\
    |childdoc.pdf| & manual
\end{tabular}
\end{center}
%
The distribution consists of the files
|README.txt|, |childdoc.ins| and |childdoc.dtx|.
%
\begin{itemize}
\item
Run (pdf)\LaTeX{} on |childdoc.dtx|
to compile the manual |childdoc.pdf| (this file).
\item
Run \LaTeX{} on |childdoc.ins| to create the definitions file |childdoc.def|
and the sample |cdocsamp.tex| with include files
|cdocsch1.tex|, |cdocsch2.tex|, |cdocspt3.tex|, |cdocspt4.tex|,
|cdocsdrf.tex|, |cdocsfn1.tex|, |cdocsfn2.tex|.
Then copy the file |childdoc.def| to an appropriate directory of your \LaTeX{}
distribution, e.g.\ \textit{texmf-root}|/tex/latex/childdoc|.
\end{itemize}

%%%%%%%%%%%%%%%%%%%%%%%%%%%%%%%%%%%%%%%%%%%%%%%%%%%%%%%%%%%%%%%%%%%%%%%%%%%%%%%%
\subsection{Related CTAN Packages}

There are several other packages which offer a similar functionality:
%
\begin{itemize}
\item
The packages
\href{http://ctan.org/pkg/docmute}{\textsf{docmute}},
\href{http://ctan.org/pkg/includex}{\textsf{includex}} and
\href{http://ctan.org/pkg/standalone}{\textsf{standalone}}
provide commands to include only the document body of
a child file thus allowing both files to be compiled individually.
\item
The packages \href{http://ctan.org/pkg/subdocs}{\textsf{subdocs}}
and \href{http://ctan.org/pkg/subfiles}{\textsf{subfiles}}
provide structures in which the main and child documents can be
encapsulated and allowing them to be compiled individually.
The inclusion mechanism is different from the conventional |\include|.
\item
The package \href{http://ctan.org/pkg/combine}{\textsf{combine}}
is an elaborate solution to combine several documents into one.
\end{itemize}
%
See also the CTAN topic \href{http://ctan.org/topic/subdocs}{\textsf{subdocs}}
for further related packages.
The present package differs from the above solutions in that
a document structure constructed with the conventional |\include| mechanism
just needs two extra commands at the top of every file
such that all constituent files can be compiled individually.

%%%%%%%%%%%%%%%%%%%%%%%%%%%%%%%%%%%%%%%%%%%%%%%%%%%%%%%%%%%%%%%%%%%%%%%%%%%%%%%%
%\subsection{Feature Suggestions}
%
%The following is a list of features which may be useful for future
%versions of this package:
%%
%\begin{itemize}
%\item
%\ldots
%\end{itemize}

%%%%%%%%%%%%%%%%%%%%%%%%%%%%%%%%%%%%%%%%%%%%%%%%%%%%%%%%%%%%%%%%%%%%%%%%%%%%%%%%
\subsection{Revision History}

%%%%%%%%%%%%%%%%%%%%%%%%%%%%%%%%%%%%%%%%
\paragraph{v2.0:} 2018/12/30

\begin{itemize}
\item
immediate forward processing
\item
added |\childdocby| mechanism
\item
manual restructured
\end{itemize}

%%%%%%%%%%%%%%%%%%%%%%%%%%%%%%%%%%%%%%%%
\paragraph{v1.6:} 2018/01/17

\begin{itemize}
\item
application for development of include files
\item
corrections to manual
\end{itemize}

%%%%%%%%%%%%%%%%%%%%%%%%%%%%%%%%%%%%%%%%
\paragraph{v1.5:} 2017/05/21

\begin{itemize}
\item
more complete structuring introduced
\item
|\childdocof| introduced
\item
|\childdoc| renamed to |\childdocmain|
\item
|\childredirect| renamed to |\childdocforward| and |\childdocforwardprefix|
and functionality expanded
\end{itemize}

%%%%%%%%%%%%%%%%%%%%%%%%%%%%%%%%%%%%%%%%
\paragraph{v1.0:} 2017/04/27

\begin{itemize}
\item
manual and install package
\item
first version published on CTAN
\end{itemize}

%%%%%%%%%%%%%%%%%%%%%%%%%%%%%%%%%%%%%%%%
\paragraph{v0.6:} 2017/04/26

\begin{itemize}
\item
redirection mechanism added
\end{itemize}

%%%%%%%%%%%%%%%%%%%%%%%%%%%%%%%%%%%%%%%%
\paragraph{v0.5:} 2017/04/26

\begin{itemize}
\item
functionality in definition file
\end{itemize}


%%%%%%%%%%%%%%%%%%%%%%%%%%%%%%%%%%%%%%%%%%%%%%%%%%%%%%%%%%%%%%%%%%%%%%%%%%%%%%%%
%%%%%%%%%%%%%%%%%%%%%%%%%%%%%%%%%%%%%%%%%%%%%%%%%%%%%%%%%%%%%%%%%%%%%%%%%%%%%%%%
%%%%%%%%%%%%%%%%%%%%%%%%%%%%%%%%%%%%%%%%%%%%%%%%%%%%%%%%%%%%%%%%%%%%%%%%%%%%%%%%
\appendix

\settowidth\MacroIndent{\rmfamily\scriptsize 000\ }

 \DocInput{childdoc.dtx}

\end{document}
%</driver>
% \fi
%
% %%%%%%%%%%%%%%%%%%%%%%%%%%%%%%%%%%%%%%%%%%%%%%%%%%%%%%%%%%%%%%%%%%%%%%%%%%%%%%
% %%%%%%%%%%%%%%%%%%%%%%%%%%%%%%%%%%%%%%%%%%%%%%%%%%%%%%%%%%%%%%%%%%%%%%%%%%%%%%
% \section{Sample}
%\iffalse
%<*samplemain>
%\fi
%
% The following presents a sample document
% with two chapters, two parts, a title page,
% a compile flag as well as three forwarding files to set the flag.
% It consists of eight |.tex| files:
% \begin{center}
% \begin{tabular}{ll}
% |cdocsamp.tex|&main file\\
% |cdocsch1.tex|&include file for chapter 1\\
% |cdocsch2.tex|&include file for chapter 2\\
% |cdocspt3.tex|&include file for part 3\\
% |cdocspt4.tex|&include file for part 4\\
% |cdocsdrf.tex|&forwarding file for main file in draft mode\\
% |cdocsfi1.tex|&forwarding file for final version of chapter 1\\
% |cdocsfi2.tex|&forwarding file for final version of chapter 2\\
% \end{tabular}
% \end{center}
% Each of the eight files can be compiled directly by the \LaTeX{} compiler.
%
% %%%%%%%%%%%%%%%%%%%%%%%%%%%%%%%%%%%%%%
% \paragraph{Main File.}
%
% The main file is called |cdocsamp.tex|.
%
% Load the \textsf{childdoc} definitions and
% declare the filename for the main document:
%    \begin{macrocode}
\input{childdoc.def}
\childdocmain{}
%    \end{macrocode}

% Optional override for |\version| flag:
%    \begin{macrocode}
%%\ifchilddoc\else\providecommand{\version}{draft}\fi
%    \end{macrocode}

% Define the default values for the |\version| flag
% (|final| for the main file and |draft| for childs):
%    \begin{macrocode}
\ifchilddoc
\providecommand{\version}{draft}
\else
\providecommand{\version}{final}
\fi
%    \end{macrocode}

% Load the standard document class:
%    \begin{macrocode}
\documentclass[12pt]{article}
%    \end{macrocode}

% Start the document body:
%    \begin{macrocode}
\begin{document}
%    \end{macrocode}

% Declare a title page.
% Print title, part of document being processed and version flag:
%    \begin{macrocode}
\addtocounter{page}{-1}
\begin{center}
{\LARGE\bfseries{}childdoc example\par}
\vspace{1cm}
\ifchilddoc
\ifchilddocmanual part\else chapter\fi:
`\childdocname' of `\childdocjob'\par
\else
main document: `\childdocjob'\par
\fi
version: \version\par
\end{center}
\newpage
%    \end{macrocode}

% Manually include selected file,
% otherwise process as usual:
%    \begin{macrocode}
\ifchilddocmanual
\section*{part `\childdocname'}
\input{\childdocname}
\else
%    \end{macrocode}

% Include the two chapters:
%    \begin{macrocode}
\include{cdocsch1}
\include{cdocsch2}
%    \end{macrocode}

% Include the two parts unless only chapters should be displayed:
%    \begin{macrocode}
\ifchilddoc\else
\section{part three}
\input{cdocspt3}
\section{part four}
\input{cdocspt4}
\fi
%    \end{macrocode}

% Process as usual until here:
%    \begin{macrocode}
\fi
%    \end{macrocode}

% End of document body:
%    \begin{macrocode}
\end{document}
%    \end{macrocode}
%\iffalse
%</samplemain>
%\fi
%
% %%%%%%%%%%%%%%%%%%%%%%%%%%%%%%%%%%%%%%
% \paragraph{Chapter Include Files.}
%
% The include files are called |cdocsch1.tex| and |cdocsch2.tex|.
%
%\iffalse
%<*samplechap1|samplechap2>
%\fi

% Optional override for |\version| flag:
%    \begin{macrocode}
%%\providecommand{\version}{final}
%    \end{macrocode}

% Include the main document:
%    \begin{macrocode}
\input{childdoc.def}
\childdocof{cdocsamp}
%    \end{macrocode}

%\iffalse
%</samplechap1|samplechap2>
%\fi
%
%\iffalse
%<*samplechap1>
%\fi
% Some text for chapter 1:
%    \begin{macrocode}
\section{one}
some text in chapter one
%    \end{macrocode}

%\iffalse
%</samplechap1>
%\fi
% Some text for chapter 2:
%\iffalse
%<*samplechap2>
%\fi
%    \begin{macrocode}
\section{two}
more text in chapter two
%    \end{macrocode}

%\iffalse
%</samplechap2>
%\fi
%
% %%%%%%%%%%%%%%%%%%%%%%%%%%%%%%%%%%%%%%
% \paragraph{Part Include Files.}
%
% The include files are called |cdocspt3.tex| and |cdocspt4.tex|.
%
%\iffalse
%<*samplepart3|samplepart4>
%\fi

% Optional override for |\version| flag:
%    \begin{macrocode}
%%\providecommand{\version}{final}
%    \end{macrocode}

% Include the main document:
%    \begin{macrocode}
\input{childdoc.def}
\childdocby{cdocsamp}
%    \end{macrocode}

%\iffalse
%</samplepart3|samplepart4>
%\fi
%
%\iffalse
%<*samplepart3>
%\fi
% Some text for part 3:
%    \begin{macrocode}
some text in part three
%    \end{macrocode}

%\iffalse
%</samplepart3>
%\fi
% Some text for part 4:
%\iffalse
%<*samplepart4>
%\fi
%    \begin{macrocode}
more text in part four
%    \end{macrocode}

%\iffalse
%</samplepart4>
%\fi
%
% %%%%%%%%%%%%%%%%%%%%%%%%%%%%%%%%%%%%%%
% \paragraph{Forwarding for a Complete Draft.}
%
% The following forwarding file |cdocsdrf.tex|
% compiles the main document in draft mode:
%\iffalse
%<*sampledraft>
%\fi
%    \begin{macrocode}
\def\version{draft}
\input{childdoc.def}
\childdocforward{cdocsamp}
%    \end{macrocode}

%\iffalse
%</sampledraft>
%\fi
%
% %%%%%%%%%%%%%%%%%%%%%%%%%%%%%%%%%%%%%%
% \paragraph{Forwarding for Final Version of the Chapters.}
%
% The following forwarding files |cdocsfn1.tex| and |cdocsfn2.tex|
% (with identical content)
% compile the final versions of the child documents
% |cdocsch1.tex| and |cdocsch2.tex|, respectively:
%\iffalse
%<*samplefinal>
%\fi
%    \begin{macrocode}
\def\version{final}
\input{childdoc.def}
\childdocforwardprefix[cdocsamp]{cdocsfn}{cdocsch}
%    \end{macrocode}

%\iffalse
%</samplefinal>
%\fi
%
% %%%%%%%%%%%%%%%%%%%%%%%%%%%%%%%%%%%%%%
% \paragraph{Command Line Processing.}
%
% The following three command lines generate the output files
% |cdocscld|, |cdocscl1| and |cdocscl2|
% which should be identical to
% |cdocsdrf|, |cdocsch1| and |cdocsfn2|, respectively:
% \begin{center}
% \begin{tabular}{l}
% |latex -jobname cdocscld \|\\
% |  "\def\version{draft}\input{childdoc.def}\childdocforward{cdocsamp}"|\\
% |latex -jobname cdocscl1 \|\\
% |  "\input{childdoc.def}\childdocforward[cdocsamp]{cdocsch1}"|\\
% |latex -jobname cdocscl2 \|\\
% |  "\def\version{final}\input{childdoc.def}\childdocforward{cdocsch2}"|
% \end{tabular}
% \end{center}
% Note that the trailing backslash on each first line
% merely continues the input to the second line
% (for convenient cut ant paste).
% Furthermore, the command |latex| can be replaced by any
% of its alternative versions such as |pdflatex|.
%
% %%%%%%%%%%%%%%%%%%%%%%%%%%%%%%%%%%%%%%%%%%%%%%%%%%%%%%%%%%%%%%%%%%%%%%%%%%%%%%
% %%%%%%%%%%%%%%%%%%%%%%%%%%%%%%%%%%%%%%%%%%%%%%%%%%%%%%%%%%%%%%%%%%%%%%%%%%%%%%
% \section{Implementation}
%\iffalse
%<*package>
%\fi
%
% This section describes the definitions file |childdoc.def|.

% The definitions cannot be loaded using |\usepackage| or |\RequirePackage|
% which has a mechanism to prevent loading a style file more than once.
% When loading the definitions by means of |\input|
% multiple instances have to be prevented manually:
%\iffalse
%This code needs to be before the `\ProvidesFile' directive
%which is defined at the beginning of this file.
%Therefore it is also placed there and commented out here.
%</package>
%<*discard>
%\fi
%    \begin{macrocode}
\ifdefined\childdocmain\endinput\fi
%    \end{macrocode}
%\iffalse
%</discard>
%<*package>
%\fi
%
% \macro{\ifchilddoc}
% \macro{\ifchilddocmanual}
% The conditional |\ifchilddoc| tells whether a
% child (true) or main (false) document is being compiled.
% The conditional |\ifchilddocmanual| tells whether
% the |\includeonly| mechanism is used (false) or
% the selection of child files must be performed manually (true).
% The definitions initialise to false:
%    \begin{macrocode}
\newif\ifchilddoc
\newif\ifchilddocmanual
%    \end{macrocode}

% \macro{\childdocname}
% \macro{\childdocjob}
% The macro |\childdocname| stores the name of the main document
% to be compiled. The macro |\childdocjob| stores the name of
% the document on which the \LaTeX{} compiler was originally invoked.
% The content of |\jobname| cannot be compared
% to filenames specified in the source due to different catcodes.
% The following code rescans |\jobname|, stores the result
% in |\childdocname| and saves a copy in |\childdocjob|:
%    \begin{macrocode}
\edef\childdocname{\scantokens\expandafter{\jobname\noexpand}}
\let\childdocjob\childdocname
%    \end{macrocode}

% \macro{\childdocdisable}
% The macro |\childdocdisable| prevents the main file
% from being processed more than once.
% At this stage, the main document command |\childdocmain|
% is assumed to be called once again where it should do nothing.
% Any subsequent call to it should prevent
% a secondary processing of the main document
% It overwrites the forwarding commands
% |\childdocof| and |\childdocforward|
% with empty macros to prevent further inclusions of the main document:
%    \begin{macrocode}
\newcommand{\childdocdisable}
{
  \renewcommand{\childdocmain}[1]{\renewcommand{\childdocmain}[1]{\endinput}}
  \renewcommand{\childdocof}[1]{}
  \renewcommand{\childdocby}[2][]{}
  \renewcommand{\childdocforward}[2][]{}
  \renewcommand{\childdocdisable}{}
}
%    \end{macrocode}

% \macro{\childdocmain}
% The macro |\childdocmain| is to be called at the top of the main file
% with nothing or the main filename (without extension) as argument.
% First, it breaks loops.
% If the argument is not empty and does not match |\childdocname|
% (which is set by the first inclusion of |childdoc.def|),
% |\ifchilddoc| is set to true, |\includeonly| is applied to the child file
% and |\jobname| is set to the main file
% (for proper handling of |.aux| files):
%    \begin{macrocode}
\newcommand{\childdocmain}[1]
{
  \childdocdisable\childdocmain{}
  \if?#1?\else
    \begingroup
      \def\childdoctmp{#1}
      \ifx\childdoctmp\childdocname
        \def\childdoctmp{}
      \else
        \def\childdoctmp
        {
          \childdoctrue
          \includeonly{\childdocname}
          \def\childdocjob{#1}
          \def\jobname{#1}
        }
      \fi
      \expandafter
    \endgroup
    \childdoctmp
  \fi
}
%    \end{macrocode}

% \macro{\childdocof}
% The command |\childdocof| redirects
% compilation to the main file |#1|.
%    \begin{macrocode}
\newcommand{\childdocof}[1]
{
  \childdocdisable
  \childdoctrue
  \includeonly{\childdocname}
  \def\jobname{#1}
  \def\childdocjob{#1}
  \input{#1}
}
%    \end{macrocode}

% \macro{\childdocby}
% The command |\childdocby| ....
%    \begin{macrocode}
\newcommand{\childdocby}[2][]
{
  \childdocdisable
  \childdoctrue
  \childdocmanualtrue
  \if?#1?\else
    \def\jobname{#2}
  \fi
  \def\childdocjob{#2}
  \input{#2}
  \endinput
}
%    \end{macrocode}

% \macro{\childdocforward}
% The command |\childdocforward| redirects
% compilation to the main file or
% (if the optional argument is given) a child file.
% Parameters are set as if the main file
% or a child file starting with |\childdocof| was compiled.
% Then compilation is handed over to the main file:
%    \begin{macrocode}
\newcommand{\childdocforward}[2][]
{
  \begingroup
    \if?#1?
      \def\childdoctmp
      {
        \def\childdocname{#2}
        \def\childdocjob{#2}
        \def\jobname{#2}
        \input{#2}
        \endinput
      }
    \else
      \def\childdoctmp
      {
        \childdocdisable
        \def\childdocname{#2}
        \childdoctrue
        \includeonly{#2}
        \def\childdocjob{#1}
        \def\jobname{#1}
        \input{#1}
        \endinput
      }
    \fi
    \expandafter
  \endgroup
  \childdoctmp
}
%    \end{macrocode}

% \macro{\childdocforwardprefix}
% The command |\childdocforwardprefix| redirects
% compilation to the main or a child file by means of a pattern.
% The prefix |#1| in the current filename is replaced by |#2|
% and the suffix of the current filename is kept
% (it is assumed that the filename does not contain the substring `|~~~|'
% which is used as a delimiter).
% Compilation is handed over to the new file by |\childdocforward|:
%    \begin{macrocode}
\newcommand{\childdocforwardprefix}[3][]
{
  \begingroup
    \def\childdocextract #2##1~~~{\def\childdoctmp{\childdocforward[#1]{#3##1}}}
    \expandafter\childdocextract\childdocname~~~
    \expandafter
  \endgroup
  \childdoctmp
}
%    \end{macrocode}

% \macro{\childdoc}
% The deprecated macro |\childdoc| is a legacy version of |\childdocmain|:
%    \begin{macrocode}
\newcommand{\childdoc}{\childdocmain}
%    \end{macrocode}

% \macro{\childdocredirect}
% The deprecated macro |\childdocredirect| is a legacy version
% of |\childdocforward| and |\childdocforwardprefix|:
%    \begin{macrocode}
\newcommand{\childdocredirect}[2][]
{
  \begingroup
    \if?#1?
      \def\childdoctmp{\childdocforward{#2}}
    \else
      \def\childdoctmp{\childdocforwardprefix{#1}{#2}}
    \fi
    \expandafter
  \endgroup
  \childdoctmp
}
%    \end{macrocode}

%\iffalse
%</package>
%\fi
%
\endinput
\childdocforward{cdocsch2}"|
% \end{tabular}
% \end{center}
% Note that the trailing backslash on each first line
% merely continues the input to the second line
% (for convenient cut ant paste).
% Furthermore, the command |latex| can be replaced by any
% of its alternative versions such as |pdflatex|.
%
% %%%%%%%%%%%%%%%%%%%%%%%%%%%%%%%%%%%%%%%%%%%%%%%%%%%%%%%%%%%%%%%%%%%%%%%%%%%%%%
% %%%%%%%%%%%%%%%%%%%%%%%%%%%%%%%%%%%%%%%%%%%%%%%%%%%%%%%%%%%%%%%%%%%%%%%%%%%%%%
% \section{Implementation}
%\iffalse
%<*package>
%\fi
%
% This section describes the definitions file |childdoc.def|.

% The definitions cannot be loaded using |\usepackage| or |\RequirePackage|
% which has a mechanism to prevent loading a style file more than once.
% When loading the definitions by means of |\input|
% multiple instances have to be prevented manually:
%\iffalse
%This code needs to be before the `\ProvidesFile' directive
%which is defined at the beginning of this file.
%Therefore it is also placed there and commented out here.
%</package>
%<*discard>
%\fi
%    \begin{macrocode}
\ifdefined\childdocmain\endinput\fi
%    \end{macrocode}
%\iffalse
%</discard>
%<*package>
%\fi
%
% \macro{\ifchilddoc}
% \macro{\ifchilddocmanual}
% The conditional |\ifchilddoc| tells whether a
% child (true) or main (false) document is being compiled.
% The conditional |\ifchilddocmanual| tells whether
% the |\includeonly| mechanism is used (false) or
% the selection of child files must be performed manually (true).
% The definitions initialise to false:
%    \begin{macrocode}
\newif\ifchilddoc
\newif\ifchilddocmanual
%    \end{macrocode}

% \macro{\childdocname}
% \macro{\childdocjob}
% The macro |\childdocname| stores the name of the main document
% to be compiled. The macro |\childdocjob| stores the name of
% the document on which the \LaTeX{} compiler was originally invoked.
% The content of |\jobname| cannot be compared
% to filenames specified in the source due to different catcodes.
% The following code rescans |\jobname|, stores the result
% in |\childdocname| and saves a copy in |\childdocjob|:
%    \begin{macrocode}
\edef\childdocname{\scantokens\expandafter{\jobname\noexpand}}
\let\childdocjob\childdocname
%    \end{macrocode}

% \macro{\childdocdisable}
% The macro |\childdocdisable| prevents the main file
% from being processed more than once.
% At this stage, the main document command |\childdocmain|
% is assumed to be called once again where it should do nothing.
% Any subsequent call to it should prevent
% a secondary processing of the main document
% It overwrites the forwarding commands
% |\childdocof| and |\childdocforward|
% with empty macros to prevent further inclusions of the main document:
%    \begin{macrocode}
\newcommand{\childdocdisable}
{
  \renewcommand{\childdocmain}[1]{\renewcommand{\childdocmain}[1]{\endinput}}
  \renewcommand{\childdocof}[1]{}
  \renewcommand{\childdocby}[2][]{}
  \renewcommand{\childdocforward}[2][]{}
  \renewcommand{\childdocdisable}{}
}
%    \end{macrocode}

% \macro{\childdocmain}
% The macro |\childdocmain| is to be called at the top of the main file
% with nothing or the main filename (without extension) as argument.
% First, it breaks loops.
% If the argument is not empty and does not match |\childdocname|
% (which is set by the first inclusion of |childdoc.def|),
% |\ifchilddoc| is set to true, |\includeonly| is applied to the child file
% and |\jobname| is set to the main file
% (for proper handling of |.aux| files):
%    \begin{macrocode}
\newcommand{\childdocmain}[1]
{
  \childdocdisable\childdocmain{}
  \if?#1?\else
    \begingroup
      \def\childdoctmp{#1}
      \ifx\childdoctmp\childdocname
        \def\childdoctmp{}
      \else
        \def\childdoctmp
        {
          \childdoctrue
          \includeonly{\childdocname}
          \def\childdocjob{#1}
          \def\jobname{#1}
        }
      \fi
      \expandafter
    \endgroup
    \childdoctmp
  \fi
}
%    \end{macrocode}

% \macro{\childdocof}
% The command |\childdocof| redirects
% compilation to the main file |#1|.
%    \begin{macrocode}
\newcommand{\childdocof}[1]
{
  \childdocdisable
  \childdoctrue
  \includeonly{\childdocname}
  \def\jobname{#1}
  \def\childdocjob{#1}
  \input{#1}
}
%    \end{macrocode}

% \macro{\childdocby}
% The command |\childdocby| ....
%    \begin{macrocode}
\newcommand{\childdocby}[2][]
{
  \childdocdisable
  \childdoctrue
  \childdocmanualtrue
  \if?#1?\else
    \def\jobname{#2}
  \fi
  \def\childdocjob{#2}
  \input{#2}
  \endinput
}
%    \end{macrocode}

% \macro{\childdocforward}
% The command |\childdocforward| redirects
% compilation to the main file or
% (if the optional argument is given) a child file.
% Parameters are set as if the main file
% or a child file starting with |\childdocof| was compiled.
% Then compilation is handed over to the main file:
%    \begin{macrocode}
\newcommand{\childdocforward}[2][]
{
  \begingroup
    \if?#1?
      \def\childdoctmp
      {
        \def\childdocname{#2}
        \def\childdocjob{#2}
        \def\jobname{#2}
        \input{#2}
        \endinput
      }
    \else
      \def\childdoctmp
      {
        \childdocdisable
        \def\childdocname{#2}
        \childdoctrue
        \includeonly{#2}
        \def\childdocjob{#1}
        \def\jobname{#1}
        \input{#1}
        \endinput
      }
    \fi
    \expandafter
  \endgroup
  \childdoctmp
}
%    \end{macrocode}

% \macro{\childdocforwardprefix}
% The command |\childdocforwardprefix| redirects
% compilation to the main or a child file by means of a pattern.
% The prefix |#1| in the current filename is replaced by |#2|
% and the suffix of the current filename is kept
% (it is assumed that the filename does not contain the substring `|~~~|'
% which is used as a delimiter).
% Compilation is handed over to the new file by |\childdocforward|:
%    \begin{macrocode}
\newcommand{\childdocforwardprefix}[3][]
{
  \begingroup
    \def\childdocextract #2##1~~~{\def\childdoctmp{\childdocforward[#1]{#3##1}}}
    \expandafter\childdocextract\childdocname~~~
    \expandafter
  \endgroup
  \childdoctmp
}
%    \end{macrocode}

% \macro{\childdoc}
% The deprecated macro |\childdoc| is a legacy version of |\childdocmain|:
%    \begin{macrocode}
\newcommand{\childdoc}{\childdocmain}
%    \end{macrocode}

% \macro{\childdocredirect}
% The deprecated macro |\childdocredirect| is a legacy version
% of |\childdocforward| and |\childdocforwardprefix|:
%    \begin{macrocode}
\newcommand{\childdocredirect}[2][]
{
  \begingroup
    \if?#1?
      \def\childdoctmp{\childdocforward{#2}}
    \else
      \def\childdoctmp{\childdocforwardprefix{#1}{#2}}
    \fi
    \expandafter
  \endgroup
  \childdoctmp
}
%    \end{macrocode}

%\iffalse
%</package>
%\fi
%
\endinput
\childdocforward{cdocsch2}"|
% \end{tabular}
% \end{center}
% Note that the trailing backslash on each first line
% merely continues the input to the second line
% (for convenient cut ant paste).
% Furthermore, the command |latex| can be replaced by any
% of its alternative versions such as |pdflatex|.
%
% %%%%%%%%%%%%%%%%%%%%%%%%%%%%%%%%%%%%%%%%%%%%%%%%%%%%%%%%%%%%%%%%%%%%%%%%%%%%%%
% %%%%%%%%%%%%%%%%%%%%%%%%%%%%%%%%%%%%%%%%%%%%%%%%%%%%%%%%%%%%%%%%%%%%%%%%%%%%%%
% \section{Implementation}
%\iffalse
%<*package>
%\fi
%
% This section describes the definitions file |childdoc.def|.

% The definitions cannot be loaded using |\usepackage| or |\RequirePackage|
% which has a mechanism to prevent loading a style file more than once.
% When loading the definitions by means of |\input|
% multiple instances have to be prevented manually:
%\iffalse
%This code needs to be before the `\ProvidesFile' directive
%which is defined at the beginning of this file.
%Therefore it is also placed there and commented out here.
%</package>
%<*discard>
%\fi
%    \begin{macrocode}
\ifdefined\childdocmain\endinput\fi
%    \end{macrocode}
%\iffalse
%</discard>
%<*package>
%\fi
%
% \macro{\ifchilddoc}
% \macro{\ifchilddocmanual}
% The conditional |\ifchilddoc| tells whether a
% child (true) or main (false) document is being compiled.
% The conditional |\ifchilddocmanual| tells whether
% the |\includeonly| mechanism is used (false) or
% the selection of child files must be performed manually (true).
% The definitions initialise to false:
%    \begin{macrocode}
\newif\ifchilddoc
\newif\ifchilddocmanual
%    \end{macrocode}

% \macro{\childdocname}
% \macro{\childdocjob}
% The macro |\childdocname| stores the name of the main document
% to be compiled. The macro |\childdocjob| stores the name of
% the document on which the \LaTeX{} compiler was originally invoked.
% The content of |\jobname| cannot be compared
% to filenames specified in the source due to different catcodes.
% The following code rescans |\jobname|, stores the result
% in |\childdocname| and saves a copy in |\childdocjob|:
%    \begin{macrocode}
\edef\childdocname{\scantokens\expandafter{\jobname\noexpand}}
\let\childdocjob\childdocname
%    \end{macrocode}

% \macro{\childdocdisable}
% The macro |\childdocdisable| prevents the main file
% from being processed more than once.
% At this stage, the main document command |\childdocmain|
% is assumed to be called once again where it should do nothing.
% Any subsequent call to it should prevent
% a secondary processing of the main document
% It overwrites the forwarding commands
% |\childdocof| and |\childdocforward|
% with empty macros to prevent further inclusions of the main document:
%    \begin{macrocode}
\newcommand{\childdocdisable}
{
  \renewcommand{\childdocmain}[1]{\renewcommand{\childdocmain}[1]{\endinput}}
  \renewcommand{\childdocof}[1]{}
  \renewcommand{\childdocby}[2][]{}
  \renewcommand{\childdocforward}[2][]{}
  \renewcommand{\childdocdisable}{}
}
%    \end{macrocode}

% \macro{\childdocmain}
% The macro |\childdocmain| is to be called at the top of the main file
% with nothing or the main filename (without extension) as argument.
% First, it breaks loops.
% If the argument is not empty and does not match |\childdocname|
% (which is set by the first inclusion of |childdoc.def|),
% |\ifchilddoc| is set to true, |\includeonly| is applied to the child file
% and |\jobname| is set to the main file
% (for proper handling of |.aux| files):
%    \begin{macrocode}
\newcommand{\childdocmain}[1]
{
  \childdocdisable\childdocmain{}
  \if?#1?\else
    \begingroup
      \def\childdoctmp{#1}
      \ifx\childdoctmp\childdocname
        \def\childdoctmp{}
      \else
        \def\childdoctmp
        {
          \childdoctrue
          \includeonly{\childdocname}
          \def\childdocjob{#1}
          \def\jobname{#1}
        }
      \fi
      \expandafter
    \endgroup
    \childdoctmp
  \fi
}
%    \end{macrocode}

% \macro{\childdocof}
% The command |\childdocof| redirects
% compilation to the main file |#1|.
%    \begin{macrocode}
\newcommand{\childdocof}[1]
{
  \childdocdisable
  \childdoctrue
  \includeonly{\childdocname}
  \def\jobname{#1}
  \def\childdocjob{#1}
  \input{#1}
}
%    \end{macrocode}

% \macro{\childdocby}
% The command |\childdocby| ....
%    \begin{macrocode}
\newcommand{\childdocby}[2][]
{
  \childdocdisable
  \childdoctrue
  \childdocmanualtrue
  \if?#1?\else
    \def\jobname{#2}
  \fi
  \def\childdocjob{#2}
  \input{#2}
  \endinput
}
%    \end{macrocode}

% \macro{\childdocforward}
% The command |\childdocforward| redirects
% compilation to the main file or
% (if the optional argument is given) a child file.
% Parameters are set as if the main file
% or a child file starting with |\childdocof| was compiled.
% Then compilation is handed over to the main file:
%    \begin{macrocode}
\newcommand{\childdocforward}[2][]
{
  \begingroup
    \if?#1?
      \def\childdoctmp
      {
        \def\childdocname{#2}
        \def\childdocjob{#2}
        \def\jobname{#2}
        \input{#2}
        \endinput
      }
    \else
      \def\childdoctmp
      {
        \childdocdisable
        \def\childdocname{#2}
        \childdoctrue
        \includeonly{#2}
        \def\childdocjob{#1}
        \def\jobname{#1}
        \input{#1}
        \endinput
      }
    \fi
    \expandafter
  \endgroup
  \childdoctmp
}
%    \end{macrocode}

% \macro{\childdocforwardprefix}
% The command |\childdocforwardprefix| redirects
% compilation to the main or a child file by means of a pattern.
% The prefix |#1| in the current filename is replaced by |#2|
% and the suffix of the current filename is kept
% (it is assumed that the filename does not contain the substring `|~~~|'
% which is used as a delimiter).
% Compilation is handed over to the new file by |\childdocforward|:
%    \begin{macrocode}
\newcommand{\childdocforwardprefix}[3][]
{
  \begingroup
    \def\childdocextract #2##1~~~{\def\childdoctmp{\childdocforward[#1]{#3##1}}}
    \expandafter\childdocextract\childdocname~~~
    \expandafter
  \endgroup
  \childdoctmp
}
%    \end{macrocode}

% \macro{\childdoc}
% The deprecated macro |\childdoc| is a legacy version of |\childdocmain|:
%    \begin{macrocode}
\newcommand{\childdoc}{\childdocmain}
%    \end{macrocode}

% \macro{\childdocredirect}
% The deprecated macro |\childdocredirect| is a legacy version
% of |\childdocforward| and |\childdocforwardprefix|:
%    \begin{macrocode}
\newcommand{\childdocredirect}[2][]
{
  \begingroup
    \if?#1?
      \def\childdoctmp{\childdocforward{#2}}
    \else
      \def\childdoctmp{\childdocforwardprefix{#1}{#2}}
    \fi
    \expandafter
  \endgroup
  \childdoctmp
}
%    \end{macrocode}

%\iffalse
%</package>
%\fi
%
\endinput

%<samplemultiprobleme|samplemultiproblemf>\childdocby{exfserm}
%<*samplemultisheet1>
%\fi
% The parameter of |\childdocof| must match the main file name |exfserm|.
% Uncommenting the commented line suppressed printing of the solution.

% %%%%%%%%%%%%%%%%%%%%%%%%%%%%%%%%%%%%%%
% \paragraph{Sheet Environment.}
%
% Declare a sheet with intended due date:
%    \begin{macrocode}
\begin{sheet}[due={2019-04-29}]
%    \end{macrocode}
% Adjust due date for each sheet.
% For sheet containing unused problems |exfseraa.tex|,
% declare a sheet |title={unused problems}| instead of due date.
%\iffalse
%</samplemultisheet1>
%<samplemultisheet2>\begin{sheet}[due={2019-05-06}]
%<samplemultisheet3>\begin{sheet}[due={2019-05-13}]
%<samplemultisheeta>\begin{sheet}[title={unused problems}]
%<*samplemultisheet1>
%\fi

% %%%%%%%%%%%%%%%%%%%%%%%%%%%%%%%%%%%%%%
% \paragraph{Problems.}
%
% Start a problem:
%    \begin{macrocode}
\begin{problem}[title={Sample A}]
%    \end{macrocode}
%
%\iffalse
%</samplemultisheet1>
%<samplemultisheet2>\begin{problem}[title={Sample C}]
%<samplemultisheet3>\input{exfserpe}
%<samplemultisheeta>\begin{problem}[title={Sample X}]
%<samplemultiprobleme>\begin{problem}[title={Sample E}]
%<*samplemultisheet1|samplemultisheet2|samplemultisheeta|samplemultiprobleme>
%\fi

% Let us declare a figure using \textsf{mpostinl} (if available).
% Denote it by the label \textit{tag}|-fig|,
% where \textit{tag} is the tag of the problem
% (in order to avoid potential conflicts with other problems;
% \textit{tag} is assigned automatically or by specifying
% the option |tag| for the |problem| environment):
%    \begin{macrocode}
\ifdefined\mpostuse
\begin{mpostfig}[label={\problemtag-fig}]
interim xu:=1.5cm;
paths[1]:=fullcircle scaled 1xu;
fill paths[1] withgreyscale 0.7;
draw paths[1] withpen pencircle scaled 1pt;
label(btex \figure etex, center(paths[1]));
\end{mpostfig}
\fi
%    \end{macrocode}

% Write a problem body with figure, some subproblems and a solution:
%    \begin{macrocode}
\lorem

\begin{subproblem}
\lorem
\begin{center}
\ifdefined\mpostuse\mpostuse{\problemtag-fig}\else figure\fi
\end{center}
\lorem
\end{subproblem}

\begin{solution}
\lorem
\end{solution}

\begin{subproblem}
\lorem
\end{subproblem}

\begin{solution}
\lorem
\end{solution}

\lorem

\begin{subproblem}
\lorem
\end{subproblem}

\begin{solution}
\lorem
\end{solution}
%    \end{macrocode}

% End the problem:
%    \begin{macrocode}
\end{problem}
%    \end{macrocode}
%\iffalse
%</samplemultisheet1|samplemultisheet2|samplemultisheeta|samplemultiprobleme>
%<*samplemultisheet1|samplemultisheet2|samplemultisheet3|samplemultisheeta>
%\fi

% Start new page:
%    \begin{macrocode}
\turnover
%    \end{macrocode}

%\iffalse
%</samplemultisheet1|samplemultisheet2|samplemultisheet3|samplemultisheeta>
%<*samplemultisheet1>
%\fi
% Write a second problem to accompany the first one:
%    \begin{macrocode}
\begin{problem}[title={Sample B}]
%    \end{macrocode}
%
%\iffalse
%</samplemultisheet1>
%<samplemultisheet2>\begin{problem}[title={Sample D}]
%<samplemultisheet3>\input{exfserpf}
%<samplemultisheeta>\begin{problem}[title={Sample Y}]
%<samplemultiproblemf>\begin{problem}[title={Sample F}]
%<*samplemultisheet1|samplemultisheet2|samplemultisheeta|samplemultiproblemf>
%\fi

% Problem body without a figure;
% this time the |solution| environments are
% included in the |subproblem| environments:
%    \begin{macrocode}
\lorem

\begin{subproblem}
\lorem
\begin{solution}
\lorem
\end{solution}
\end{subproblem}

\begin{subproblem}
\lorem
\begin{solution}
\lorem
\end{solution}
\end{subproblem}
%    \end{macrocode}

% End the problem:
%    \begin{macrocode}
\end{problem}
%    \end{macrocode}

% %%%%%%%%%%%%%%%%%%%%%%%%%%%%%%%%%%%%%%
% \paragraph{End Sheet.}
%\iffalse
%</samplemultisheet1|samplemultisheet2|samplemultisheeta|samplemultiproblemf>
%<*samplemultisheet1|samplemultisheet2|samplemultisheet3|samplemultisheeta>
%\fi
% End the sheet:
%    \begin{macrocode}
\end{sheet}
%    \end{macrocode}

%\iffalse
%</samplemultisheet1|samplemultisheet2|samplemultisheet3|samplemultisheeta>
%\fi
%
% %%%%%%%%%%%%%%%%%%%%%%%%%%%%%%%%%%%%%%%%%%%%%%%%%%%%%%%%%%%%%%%%%%%%%%%%%%%%%%
% %%%%%%%%%%%%%%%%%%%%%%%%%%%%%%%%%%%%%%%%%%%%%%%%%%%%%%%%%%%%%%%%%%%%%%%%%%%%%%
% \subsection{Individual Problem Files}
% \label{sec:samplemultisheet3}
%\iffalse
%<*discard>
%\fi
%
% It may be more convenient to define each problem in an individual file,
% so that a sheet can be composed by including the appropriate
% problem files.
% In such a setup, the \textsf{childdoc} mechanism
% allows to compile each problem individually.
%
% To that end, prepare a file |exfserp|\textit{nn}|.tex|
% containing the |problem| environment.
% This file should start with:
%    \begin{macrocode}
%%\providecommand{\printsol}{n}
% \iffalse
%
% childdoc.dtx Copyright (C) 2017-2018 Niklas Beisert
%
% This work may be distributed and/or modified under the
% conditions of the LaTeX Project Public License, either version 1.3
% of this license or (at your option) any later version.
% The latest version of this license is in
%   http://www.latex-project.org/lppl.txt
% and version 1.3 or later is part of all distributions of LaTeX
% version 2005/12/01 or later.
%
% This work has the LPPL maintenance status `maintained'.
%
% The Current Maintainer of this work is Niklas Beisert.
%
% This work consists of the files childdoc.dtx and childdoc.ins
% and the derived files childdoc.def and cdocsamp.tex with
% cdocsch1.tex, cdocsch2.tex, cdocsdrf.tex, cdocsfn1.tex, cdocsfn2.tex.
%
%<package>\ifdefined\childdocmain\endinput\fi
%<package>\ProvidesFile{childdoc.def}[2018/12/30 v2.0 child document driver]
%<samplemain>\ProvidesFile{cdocsamp.tex}[2018/12/30 v2.0 sample for childdoc]
%<*driver>
%\ProvidesFile{childdoc.drv}[2018/12/30 v2.0 childdoc reference manual file]
\PassOptionsToClass{10pt,a4paper}{article}
\documentclass{ltxdoc}

\usepackage[margin=35mm]{geometry}
\usepackage{hyperref}
\usepackage{hyperxmp}
\usepackage[usenames]{color}

\hypersetup{colorlinks=true}
\hypersetup{pdfstartview=FitH}
\hypersetup{pdfpagemode=UseNone}
\hypersetup{pdfsource={}}
\hypersetup{pdflang={en-UK}}
\hypersetup{pdfcopyright={Copyright 2017-2018 Niklas Beisert.
  This work may be distributed and/or modified under the
  conditions of the LaTeX Project Public License, either version 1.3
  of this license or (at your option) any later version.}}
\hypersetup{pdflicenseurl={http://www.latex-project.org/lppl.txt}}
\hypersetup{pdfcontactaddress={ETH Zurich, ITP, HIT K,
  Wolfgang-Pauli-Strasse 27}}
\hypersetup{pdfcontactpostcode={8093}}
\hypersetup{pdfcontactcity={Zurich}}
\hypersetup{pdfcontactcountry={Switzerland}}
\hypersetup{pdfcontactemail={nbeisert@itp.phys.ethz.ch}}
\hypersetup{pdfcontacturl={http://people.phys.ethz.ch/\xmptilde nbeisert/}}

\newcommand{\secref}[1]{\hyperref[#1]{section \ref*{#1}}}

\parskip1ex
\parindent0pt
\let\olditemize\itemize
\def\itemize{\olditemize\parskip0pt}

\begin{document}

\title{The \textsf{childdoc} Package}
\hypersetup{pdftitle={The childdoc Package}}
\author{Niklas Beisert\\[2ex]
  Institut f\"ur Theoretische Physik\\
  Eidgen\"ossische Technische Hochschule Z\"urich\\
  Wolfgang-Pauli-Strasse 27, 8093 Z\"urich, Switzerland\\[1ex]
  \href{mailto:nbeisert@itp.phys.ethz.ch}
  {\texttt{nbeisert@itp.phys.ethz.ch}}}
\hypersetup{pdfauthor={Niklas Beisert}}
\hypersetup{pdfsubject={Manual for the LaTeX2e Package childdoc}}
\date{30 December 2018, \textsf{v2.0}}
\maketitle

\begin{abstract}\noindent
\textsf{childdoc} is a \LaTeXe{} package
that enables the direct compilation
of document sections included by |\include|
to individual files.
\end{abstract}

\begingroup
\parskip0ex
\tableofcontents
\endgroup

%%%%%%%%%%%%%%%%%%%%%%%%%%%%%%%%%%%%%%%%%%%%%%%%%%%%%%%%%%%%%%%%%%%%%%%%%%%%%%%%
%%%%%%%%%%%%%%%%%%%%%%%%%%%%%%%%%%%%%%%%%%%%%%%%%%%%%%%%%%%%%%%%%%%%%%%%%%%%%%%%
\section{Introduction}

\LaTeX{} provides a mechanism to structure a large document (such as a book)
into a main file and several child files (containing the chapters)
using the |\include| command.
This mechanism is beneficial for documents
which span hundreds of pages in order to
make the source file(s) more manageable.
Moreover, compilation can be restricted to
selected child files by means of the |\includeonly| command.
The latter feature can be used to reduce the compilation time while editing
(this was significantly more useful in the earlier days of \LaTeX{})
or to generate a smaller document which is easier to navigate.
Another application of |\includeonly| is to generate
documents consisting of selected parts of the complete document.

However, there are a few drawbacks of the plain |\include| mechanism:
\begin{itemize}
\item
The child files cannot be compiled on their own,
they can only be compiled via the main file.
A naive editing environment
(such as a text editor with an option
to have the current file processed by \LaTeX)
may require one to switch to the main file before compiling;
attempting to compile the child file produces errors.
\item
The main file must be modified (each time)
to adjust the |\includeonly| command
to the present needs. This easily leaves the main file in a messy state.
\item
The generated document will always carry the filename
of the main document. This is inconvenient if
several child files are to be compiled and
to be kept for distribution.
\end{itemize}

The present package provides a simple interface
to make child files individually compilable by \LaTeX{}.
Compiling a child file then has the same effect as compiling
the main file with an |\includeonly| command
to select the appropriate child.
Moreover the generated document will carry the name of the child
rather than the main file.
This resolves all three above issues.

This feature is meant to make the editing of books,
thesis documents and lecture notes somewhat more convenient.
However, the package can also be used efficiently for
composing a series of documents (such as exercise sheets)
which are typically distributed individually.
It then assists the author in generating the individual documents
(potentially in different versions)
as well as a document containing the collected series.
Another application is in developing style files
or other kinds of included material
where compilation of the style file could redirect
to a sample or test file.

%%%%%%%%%%%%%%%%%%%%%%%%%%%%%%%%%%%%%%%%%%%%%%%%%%%%%%%%%%%%%%%%%%%%%%%%%%%%%%%%
%%%%%%%%%%%%%%%%%%%%%%%%%%%%%%%%%%%%%%%%%%%%%%%%%%%%%%%%%%%%%%%%%%%%%%%%%%%%%%%%
\section{Usage}

First of all, the package \textsf{childdoc} is \emph{not} a standard
\LaTeXe{} |.sty| style file! Therefore it needs to be invoked in
a non-standard way.

%%%%%%%%%%%%%%%%%%%%%%%%%%%%%%%%%%%%%%%%%%%%%%%%%%%%%%%%%%%%%%%%%%%%%%%%%%%%%%%%
\subsection{Included Files}
\label{sec:include}

%%%%%%%%%%%%%%%%%%%%%%%%%%%%%%%%%%%%%%%%
\DescribeMacro{\childdocmain}
To use the package, add the commands
\begin{center}
\begin{tabular}{l}
|% \iffalse
%
% childdoc.dtx Copyright (C) 2017-2018 Niklas Beisert
%
% This work may be distributed and/or modified under the
% conditions of the LaTeX Project Public License, either version 1.3
% of this license or (at your option) any later version.
% The latest version of this license is in
%   http://www.latex-project.org/lppl.txt
% and version 1.3 or later is part of all distributions of LaTeX
% version 2005/12/01 or later.
%
% This work has the LPPL maintenance status `maintained'.
%
% The Current Maintainer of this work is Niklas Beisert.
%
% This work consists of the files childdoc.dtx and childdoc.ins
% and the derived files childdoc.def and cdocsamp.tex with
% cdocsch1.tex, cdocsch2.tex, cdocsdrf.tex, cdocsfn1.tex, cdocsfn2.tex.
%
%<package>\ifdefined\childdocmain\endinput\fi
%<package>\ProvidesFile{childdoc.def}[2018/12/30 v2.0 child document driver]
%<samplemain>\ProvidesFile{cdocsamp.tex}[2018/12/30 v2.0 sample for childdoc]
%<*driver>
%\ProvidesFile{childdoc.drv}[2018/12/30 v2.0 childdoc reference manual file]
\PassOptionsToClass{10pt,a4paper}{article}
\documentclass{ltxdoc}

\usepackage[margin=35mm]{geometry}
\usepackage{hyperref}
\usepackage{hyperxmp}
\usepackage[usenames]{color}

\hypersetup{colorlinks=true}
\hypersetup{pdfstartview=FitH}
\hypersetup{pdfpagemode=UseNone}
\hypersetup{pdfsource={}}
\hypersetup{pdflang={en-UK}}
\hypersetup{pdfcopyright={Copyright 2017-2018 Niklas Beisert.
  This work may be distributed and/or modified under the
  conditions of the LaTeX Project Public License, either version 1.3
  of this license or (at your option) any later version.}}
\hypersetup{pdflicenseurl={http://www.latex-project.org/lppl.txt}}
\hypersetup{pdfcontactaddress={ETH Zurich, ITP, HIT K,
  Wolfgang-Pauli-Strasse 27}}
\hypersetup{pdfcontactpostcode={8093}}
\hypersetup{pdfcontactcity={Zurich}}
\hypersetup{pdfcontactcountry={Switzerland}}
\hypersetup{pdfcontactemail={nbeisert@itp.phys.ethz.ch}}
\hypersetup{pdfcontacturl={http://people.phys.ethz.ch/\xmptilde nbeisert/}}

\newcommand{\secref}[1]{\hyperref[#1]{section \ref*{#1}}}

\parskip1ex
\parindent0pt
\let\olditemize\itemize
\def\itemize{\olditemize\parskip0pt}

\begin{document}

\title{The \textsf{childdoc} Package}
\hypersetup{pdftitle={The childdoc Package}}
\author{Niklas Beisert\\[2ex]
  Institut f\"ur Theoretische Physik\\
  Eidgen\"ossische Technische Hochschule Z\"urich\\
  Wolfgang-Pauli-Strasse 27, 8093 Z\"urich, Switzerland\\[1ex]
  \href{mailto:nbeisert@itp.phys.ethz.ch}
  {\texttt{nbeisert@itp.phys.ethz.ch}}}
\hypersetup{pdfauthor={Niklas Beisert}}
\hypersetup{pdfsubject={Manual for the LaTeX2e Package childdoc}}
\date{30 December 2018, \textsf{v2.0}}
\maketitle

\begin{abstract}\noindent
\textsf{childdoc} is a \LaTeXe{} package
that enables the direct compilation
of document sections included by |\include|
to individual files.
\end{abstract}

\begingroup
\parskip0ex
\tableofcontents
\endgroup

%%%%%%%%%%%%%%%%%%%%%%%%%%%%%%%%%%%%%%%%%%%%%%%%%%%%%%%%%%%%%%%%%%%%%%%%%%%%%%%%
%%%%%%%%%%%%%%%%%%%%%%%%%%%%%%%%%%%%%%%%%%%%%%%%%%%%%%%%%%%%%%%%%%%%%%%%%%%%%%%%
\section{Introduction}

\LaTeX{} provides a mechanism to structure a large document (such as a book)
into a main file and several child files (containing the chapters)
using the |\include| command.
This mechanism is beneficial for documents
which span hundreds of pages in order to
make the source file(s) more manageable.
Moreover, compilation can be restricted to
selected child files by means of the |\includeonly| command.
The latter feature can be used to reduce the compilation time while editing
(this was significantly more useful in the earlier days of \LaTeX{})
or to generate a smaller document which is easier to navigate.
Another application of |\includeonly| is to generate
documents consisting of selected parts of the complete document.

However, there are a few drawbacks of the plain |\include| mechanism:
\begin{itemize}
\item
The child files cannot be compiled on their own,
they can only be compiled via the main file.
A naive editing environment
(such as a text editor with an option
to have the current file processed by \LaTeX)
may require one to switch to the main file before compiling;
attempting to compile the child file produces errors.
\item
The main file must be modified (each time)
to adjust the |\includeonly| command
to the present needs. This easily leaves the main file in a messy state.
\item
The generated document will always carry the filename
of the main document. This is inconvenient if
several child files are to be compiled and
to be kept for distribution.
\end{itemize}

The present package provides a simple interface
to make child files individually compilable by \LaTeX{}.
Compiling a child file then has the same effect as compiling
the main file with an |\includeonly| command
to select the appropriate child.
Moreover the generated document will carry the name of the child
rather than the main file.
This resolves all three above issues.

This feature is meant to make the editing of books,
thesis documents and lecture notes somewhat more convenient.
However, the package can also be used efficiently for
composing a series of documents (such as exercise sheets)
which are typically distributed individually.
It then assists the author in generating the individual documents
(potentially in different versions)
as well as a document containing the collected series.
Another application is in developing style files
or other kinds of included material
where compilation of the style file could redirect
to a sample or test file.

%%%%%%%%%%%%%%%%%%%%%%%%%%%%%%%%%%%%%%%%%%%%%%%%%%%%%%%%%%%%%%%%%%%%%%%%%%%%%%%%
%%%%%%%%%%%%%%%%%%%%%%%%%%%%%%%%%%%%%%%%%%%%%%%%%%%%%%%%%%%%%%%%%%%%%%%%%%%%%%%%
\section{Usage}

First of all, the package \textsf{childdoc} is \emph{not} a standard
\LaTeXe{} |.sty| style file! Therefore it needs to be invoked in
a non-standard way.

%%%%%%%%%%%%%%%%%%%%%%%%%%%%%%%%%%%%%%%%%%%%%%%%%%%%%%%%%%%%%%%%%%%%%%%%%%%%%%%%
\subsection{Included Files}
\label{sec:include}

%%%%%%%%%%%%%%%%%%%%%%%%%%%%%%%%%%%%%%%%
\DescribeMacro{\childdocmain}
To use the package, add the commands
\begin{center}
\begin{tabular}{l}
|% \iffalse
%
% childdoc.dtx Copyright (C) 2017-2018 Niklas Beisert
%
% This work may be distributed and/or modified under the
% conditions of the LaTeX Project Public License, either version 1.3
% of this license or (at your option) any later version.
% The latest version of this license is in
%   http://www.latex-project.org/lppl.txt
% and version 1.3 or later is part of all distributions of LaTeX
% version 2005/12/01 or later.
%
% This work has the LPPL maintenance status `maintained'.
%
% The Current Maintainer of this work is Niklas Beisert.
%
% This work consists of the files childdoc.dtx and childdoc.ins
% and the derived files childdoc.def and cdocsamp.tex with
% cdocsch1.tex, cdocsch2.tex, cdocsdrf.tex, cdocsfn1.tex, cdocsfn2.tex.
%
%<package>\ifdefined\childdocmain\endinput\fi
%<package>\ProvidesFile{childdoc.def}[2018/12/30 v2.0 child document driver]
%<samplemain>\ProvidesFile{cdocsamp.tex}[2018/12/30 v2.0 sample for childdoc]
%<*driver>
%\ProvidesFile{childdoc.drv}[2018/12/30 v2.0 childdoc reference manual file]
\PassOptionsToClass{10pt,a4paper}{article}
\documentclass{ltxdoc}

\usepackage[margin=35mm]{geometry}
\usepackage{hyperref}
\usepackage{hyperxmp}
\usepackage[usenames]{color}

\hypersetup{colorlinks=true}
\hypersetup{pdfstartview=FitH}
\hypersetup{pdfpagemode=UseNone}
\hypersetup{pdfsource={}}
\hypersetup{pdflang={en-UK}}
\hypersetup{pdfcopyright={Copyright 2017-2018 Niklas Beisert.
  This work may be distributed and/or modified under the
  conditions of the LaTeX Project Public License, either version 1.3
  of this license or (at your option) any later version.}}
\hypersetup{pdflicenseurl={http://www.latex-project.org/lppl.txt}}
\hypersetup{pdfcontactaddress={ETH Zurich, ITP, HIT K,
  Wolfgang-Pauli-Strasse 27}}
\hypersetup{pdfcontactpostcode={8093}}
\hypersetup{pdfcontactcity={Zurich}}
\hypersetup{pdfcontactcountry={Switzerland}}
\hypersetup{pdfcontactemail={nbeisert@itp.phys.ethz.ch}}
\hypersetup{pdfcontacturl={http://people.phys.ethz.ch/\xmptilde nbeisert/}}

\newcommand{\secref}[1]{\hyperref[#1]{section \ref*{#1}}}

\parskip1ex
\parindent0pt
\let\olditemize\itemize
\def\itemize{\olditemize\parskip0pt}

\begin{document}

\title{The \textsf{childdoc} Package}
\hypersetup{pdftitle={The childdoc Package}}
\author{Niklas Beisert\\[2ex]
  Institut f\"ur Theoretische Physik\\
  Eidgen\"ossische Technische Hochschule Z\"urich\\
  Wolfgang-Pauli-Strasse 27, 8093 Z\"urich, Switzerland\\[1ex]
  \href{mailto:nbeisert@itp.phys.ethz.ch}
  {\texttt{nbeisert@itp.phys.ethz.ch}}}
\hypersetup{pdfauthor={Niklas Beisert}}
\hypersetup{pdfsubject={Manual for the LaTeX2e Package childdoc}}
\date{30 December 2018, \textsf{v2.0}}
\maketitle

\begin{abstract}\noindent
\textsf{childdoc} is a \LaTeXe{} package
that enables the direct compilation
of document sections included by |\include|
to individual files.
\end{abstract}

\begingroup
\parskip0ex
\tableofcontents
\endgroup

%%%%%%%%%%%%%%%%%%%%%%%%%%%%%%%%%%%%%%%%%%%%%%%%%%%%%%%%%%%%%%%%%%%%%%%%%%%%%%%%
%%%%%%%%%%%%%%%%%%%%%%%%%%%%%%%%%%%%%%%%%%%%%%%%%%%%%%%%%%%%%%%%%%%%%%%%%%%%%%%%
\section{Introduction}

\LaTeX{} provides a mechanism to structure a large document (such as a book)
into a main file and several child files (containing the chapters)
using the |\include| command.
This mechanism is beneficial for documents
which span hundreds of pages in order to
make the source file(s) more manageable.
Moreover, compilation can be restricted to
selected child files by means of the |\includeonly| command.
The latter feature can be used to reduce the compilation time while editing
(this was significantly more useful in the earlier days of \LaTeX{})
or to generate a smaller document which is easier to navigate.
Another application of |\includeonly| is to generate
documents consisting of selected parts of the complete document.

However, there are a few drawbacks of the plain |\include| mechanism:
\begin{itemize}
\item
The child files cannot be compiled on their own,
they can only be compiled via the main file.
A naive editing environment
(such as a text editor with an option
to have the current file processed by \LaTeX)
may require one to switch to the main file before compiling;
attempting to compile the child file produces errors.
\item
The main file must be modified (each time)
to adjust the |\includeonly| command
to the present needs. This easily leaves the main file in a messy state.
\item
The generated document will always carry the filename
of the main document. This is inconvenient if
several child files are to be compiled and
to be kept for distribution.
\end{itemize}

The present package provides a simple interface
to make child files individually compilable by \LaTeX{}.
Compiling a child file then has the same effect as compiling
the main file with an |\includeonly| command
to select the appropriate child.
Moreover the generated document will carry the name of the child
rather than the main file.
This resolves all three above issues.

This feature is meant to make the editing of books,
thesis documents and lecture notes somewhat more convenient.
However, the package can also be used efficiently for
composing a series of documents (such as exercise sheets)
which are typically distributed individually.
It then assists the author in generating the individual documents
(potentially in different versions)
as well as a document containing the collected series.
Another application is in developing style files
or other kinds of included material
where compilation of the style file could redirect
to a sample or test file.

%%%%%%%%%%%%%%%%%%%%%%%%%%%%%%%%%%%%%%%%%%%%%%%%%%%%%%%%%%%%%%%%%%%%%%%%%%%%%%%%
%%%%%%%%%%%%%%%%%%%%%%%%%%%%%%%%%%%%%%%%%%%%%%%%%%%%%%%%%%%%%%%%%%%%%%%%%%%%%%%%
\section{Usage}

First of all, the package \textsf{childdoc} is \emph{not} a standard
\LaTeXe{} |.sty| style file! Therefore it needs to be invoked in
a non-standard way.

%%%%%%%%%%%%%%%%%%%%%%%%%%%%%%%%%%%%%%%%%%%%%%%%%%%%%%%%%%%%%%%%%%%%%%%%%%%%%%%%
\subsection{Included Files}
\label{sec:include}

%%%%%%%%%%%%%%%%%%%%%%%%%%%%%%%%%%%%%%%%
\DescribeMacro{\childdocmain}
To use the package, add the commands
\begin{center}
\begin{tabular}{l}
|\input{childdoc.def}|\\
|\childdocmain{}|\\
\end{tabular}
\end{center}
at the very top of the main \LaTeX{} file,
in particular \emph{before} the |\documentclass| statement!
The argument of |\childdocmain| should be left empty
(but it must be present).

%%%%%%%%%%%%%%%%%%%%%%%%%%%%%%%%%%%%%%%%
\DescribeMacro{\childdocof}
Furthermore, add the commands
\begin{center}
\begin{tabular}{l}
|\input{childdoc.def}|\\
|\childdocof{|\textit{main}|}|\\
\end{tabular}
\end{center}
at the top of every child file \textit{child}
which is included by |\include{|\textit{child}|}|
from within the main file
(or at least for those files to be compiled individually).
The argument \textit{main} must be the filename of the main file.

There are a couple of
considerations in setting up the main and child documents:

%%%%%%%%%%%%%%%%%%%%%%%%%%%%%%%%%%%%%%%%
\paragraph{Restrictions.}

Please note the following restrictions:
\begin{itemize}
\item
|\childdocmain| must be called with one argument \textit{main}
to ensure compatibility with earlier version of the package.
It must either be empty (|\childdocmain{}|)
or precisely match the filename of the main file in which it is specified.
See \secref{sec:detection} for further information.
\item
The filename \textit{main} must be specified without the |.tex| extension.
\item
The filename \textit{main} is case sensitive
(even in case-insensitive file systems)
due to internal string comparison.
\item
The argument \textit{main} should be fully expanded, it cannot be a macro.
\item
Subdirectories and special characters should be avoided in filenames.
\item
The command |\childdocmain{|\textit{main}|}| must be followed by a whitespace.
It should not be followed immediately by another command
or by a comment mark `|%|'.
This is because the \TeX{} parser reads the token immediately following
the argument of |\childdocmain| and puts it
at the beginning of every child section;
however, a white\-space is ignored.
\end{itemize}

%%%%%%%%%%%%%%%%%%%%%%%%%%%%%%%%%%%%%%%%
\paragraph{Content of Main File.}

It is advisable to place all content in the child files included by |\include|.
Any output contained in the main file will appear in all child documents
unless suppressed manually;
it cannot be suppressed automatically by the |\includeonly| directive
and thus should normally be avoided.
A method to include some content in the main file
by means of conditional processing is described in \secref{sec:conditional}.

%%%%%%%%%%%%%%%%%%%%%%%%%%%%%%%%%%%%%%%%
\paragraph{Page Numbering.}

When only a part of the document is compiled,
the appropriate numbering of pages
(as well as other status parameters)
is determined from the |.aux| files.
The latter contain information from previous passes.
However this information needs to propagate through
all intermediate child documents.
Therefore the page numbering in child documents may well
be inconsistent until the complete document is compiled at least once.

A useful (if unconventional) way to always ensure a consistent
page numbering is to restart the numbering in each child document
and denote the pages by `\textit{child}|.|\textit{page}'
where \textit{child} represents the chapter/section number of the child file.
This can be achieved by the command
|\numberwithin{page}{|\textit{child}|}|
of the \textsf{amsmath} package
where \textit{child} can be |chapter| or |section|
depending on the chosen structuring.
Alternatively, one can modify the macro |\thepage| appropriately
and reset the counter |page| at the start of each child file.

%%%%%%%%%%%%%%%%%%%%%%%%%%%%%%%%%%%%%%%%%%%%%%%%%%%%%%%%%%%%%%%%%%%%%%%%%%%%%%%%
\subsection{Conditional Processing}
\label{sec:conditional}

The package provides a mechanism to compile different versions
of a document. To customise the versions further some conditional processing
can come in handy to distinguish which version is being compiled.
The package provides two macros to describe the compilation context:

%%%%%%%%%%%%%%%%%%%%%%%%%%%%%%%%%%%%%%%%
\DescribeMacro{\ifchilddoc}
The conditional |\ifchilddoc| distinguishes between the compilation of
child documents and the main document:
%
\begin{center}
|\ifchilddoc |\textit{child-code}| |[|\||else |\textit{main-code}]| \||fi|
\end{center}

%%%%%%%%%%%%%%%%%%%%%%%%%%%%%%%%%%%%%%%%
\DescribeMacro{\childdocname}
\DescribeMacro{\childdocjob}
The macro |\childdocname| contains the filename (without extension)
of the main or child file being processed.
Note that |\childdocjob| will always contain the name of the main file.

%%%%%%%%%%%%%%%%%%%%%%%%%%%%%%%%%%%%%%%%
\paragraph{Title Page.}

Conditional processing can be used to include a title or banner page
in the main document when proper precautions are taken.
Importantly, the code in the main file should ensure that the page counter
(as well as other status parameters which are stored in the |.aux| files)
takes the same value after the conditional processing.
Otherwise the page numbers may take divergent values
depending on which part is compiled.

For example, a title page could be declared by:
%
\begin{center}
\begin{tabular}{l}
|\ifchilddoc\||else|\\
|\addtocounter{page}{-1}|\\
\textit{code for title page}\\
|\newpage|\\
|\||fi|
\end{tabular}
\end{center}
%
A banner page for the child documents can be generated by:
%
\begin{center}
\begin{tabular}{l}
|\ifchilddoc|\\
|\addtocounter{page}{-1}|\\
\textit{code for banner page}\\
|\newpage|\\
|\||fi|
\end{tabular}
\end{center}
%
Here one could write a message such as:
\begin{center}
|This is the part \childdocname{} of \childdocjob{}.|
\end{center}

%%%%%%%%%%%%%%%%%%%%%%%%%%%%%%%%%%%%%%%%%%%%%%%%%%%%%%%%%%%%%%%%%%%%%%%%%%%%%%%%
\subsection{Flags}
\label{sec:flags}

The package makes it easy to generate different versions
of the main or child documents.
To this end compilation flags can be defined
and assigned different default values.
They will be particularly useful in conjunction
with the forwarding mechanism described in \secref{sec:forward}.

For example, it may be useful to have a flag |\version|
which can be set to |draft| or |final|.
The document source will contain some conditional code
depending on the value of |\version|.
Suppose further, the flag should default to |final| for the main file
and to |draft| for child files
which is a natural assignment for editing the document.
This is achieved by placing the following code
in the preamble of the main document
(below the |\childdocmain| directive):
%
\begin{center}
\begin{tabular}{l}
|\ifchilddoc|\\
|\providecommand{\version}{draft}|\\
|\||else|\\
|\providecommand{\version}{final}|\\
|\||fi|
\end{tabular}
\end{center}
%
The definition by |\providecommand| makes sure
that previous definitions are not overwritten.
Further statements |\providecommand{\version}{...}|
can thus be added before the above code to override it.

For the main file, one might add a line
(between |\childdocmain| and the above block)
%
\begin{center}
|%\ifchilddoc\||else\providecommand{\version}{draft}\||fi|
\end{center}
%
which can be uncommented to produce a draft version.
Likewise one can add a line to the very top of a child file
(above the |\childdocof{|\textit{main}|}| directive)
%
\begin{center}
|%\providecommand{\version}{final}|
\end{center}
%
which can be uncommented to produce the final version of this child document.

%%%%%%%%%%%%%%%%%%%%%%%%%%%%%%%%%%%%%%%%%%%%%%%%%%%%%%%%%%%%%%%%%%%%%%%%%%%%%%%%
\subsection{Forwarding}
\label{sec:forward}

Different versions of the main or child documents
using compilation flags as described in \secref{sec:flags}
can be (permanently) stored in different files
for convenient compilation, viewing and distribution.
To this end, the package defines a command
to pass on compilation to a different file:

%%%%%%%%%%%%%%%%%%%%%%%%%%%%%%%%%%%%%%%%
\DescribeMacro{\childdocforward}
The command |\childdocforward| redirects processing to
another source file:
%
\begin{center}
\begin{tabular}{l}
|\input{childdoc.def}|\\
|\childdocforward[|\textit{main}|]{|\textit{dest}|}|\\
\end{tabular}
\end{center}
%
The argument \textit{dest} is the destination file
(without extension).
It should be the main file or one of the child files.
Note that further \textsf{childdoc} directives
such as |\childdocof| and |\childdocforward|
in the indicated file will be processed in this form.
The optional argument \textit{main}
passes on directly to the main file \textit{main}
while pretending to compile the child \textit{dest}.
This form behaves as if \textit{dest}
issues |\childdocof{|\textit{main}|}| right away,
and no further \textsf{childdoc} directives will be processed.

%%%%%%%%%%%%%%%%%%%%%%%%%%%%%%%%%%%%%%%%
\DescribeMacro{\...prefix}
In the alternative form |\childdocforwardprefix|,
%
\begin{center}
\begin{tabular}{l}
|\input{childdoc.def}|\\
|\childdocforwardprefix[|\textit{main}|]{|\textit{prefix}|}{|\textit{dest}|}|
\end{tabular}
\end{center}
%
the destination file is determined by a pattern
depending on the current file:
To make this work, the current file must be called
`{\textit{prefix}\hspace{0.2em}\textit{suffix}}'
with \textit{prefix} matching precisely the argument.
Processing is then passed on to the file
`{\textit{dest}\hspace{0.2em}\textit{suffix}}'.
Surely, the same effect is achieved by
directly specifying the
argument `{\textit{dest}\hspace{0.2em}\textit{suffix}}'
in the first form.
However, that requires to set up a different file
for each child. With the alternative form of the command
all these files can have exactly the same content
which simplifies setting them up and maintaining them.

For example, the following file |draft.tex|
with a compilation flag |\version| as described in \secref{sec:flags}
compiles the main document as a draft:
%
\begin{center}
\begin{tabular}{l}
|\def\version{draft}|\\
|\input{childdoc.def}|\\
|\childdocforward{|\textit{main}|}|
\end{tabular}
\end{center}
%
Likewise, the following files |final|\textit{nn}|.tex|
compile the final version of the child document
|child|\textit{nn}|.tex|:
%
\begin{center}
\begin{tabular}{l}
|\def\version{final}|\\
|\input{childdoc.def}|\\
|\childdocforwardprefix{final}{child}|
\end{tabular}
\end{center}
%

Note that when several versions of a main file and/or of each child file
are to be generated, it may be convenient to set up a |Makefile| or
shell script to automatise the process.

%%%%%%%%%%%%%%%%%%%%%%%%%%%%%%%%%%%%%%%%%%%%%%%%%%%%%%%%%%%%%%%%%%%%%%%%%%%%%%%%
\subsection{Command Line Processing}
\label{sec:commandline}

The effect of redirection files can also be achieved by invoking
the \LaTeX{} compiler with a more elaborate command line.
Most conveniently this should be done as part
of a shell script or a |Makefile|.

When using \textsf{childdoc} in the main file, the following
command lines effectively perform a redirection
(note that depending on the shell being used,
backslashes may have to be doubled: `|\|' $\to$ `|\\|'):
%
\begin{center}
|... -jobname "|\textit{target}|" |\\|"|[\textit{flags}]%
|\input{childdoc.def}\childdocforward[|\textit{main}|]{|\textit{dest}|}"|
\end{center}
%
Here \textit{target} is the name of the output file,
\textit{main} is the name of the main file
and \textit{dest} is the name of the main or child file to be processed
(all filenames without extensions).
The optional argument \textit{main} can be omitted
if \textit{main} matches \textit{dest}.
Optionally, compilation \textit{flags} can be defined via |\def| commands.
This command line makes the \TeX{} engine believe
it is compiling the file \textit{target}
whose content is specified as the latter parameter.
The provided code then forwards the processing to
\textit{main} or \textit{dest} as described in \secref{sec:forward}.

%%%%%%%%%%%%%%%%%%%%%%%%%%%%%%%%%%%%%%%%%%%%%%%%%%%%%%%%%%%%%%%%%%%%%%%%%%%%%%%%
\subsection{Include by Input}
\label{sec:input}

Including child documents by |\include| has some restrictions by design.
Most notably, the content of a child document always occupies
its own set of pages; pages cannot be shared between child documents.
Usually, this behaviour makes perfect sense
because each child document contain an essential part of the document.
However, in some situations it may be desirable to compose
a document from a collection of parts
without having mandatory page breaks between then.
For this case, the package
provides a mechanism to include parts
by |\input| which can also be processed individually.
However, by construction this mechanism
requires manual handling of the content to be output.

%%%%%%%%%%%%%%%%%%%%%%%%%%%%%%%%%%%%%%%%
\DescribeMacro{\ifchilddocmanual}
The main file should be prepared as usual, see \secref{sec:include}.
However, the document body must make a distinction
between processing of an individual part and of the main document, e.g.:
%
\begin{center}
\begin{tabular}{l}
|\ifchilddocmanual|\\
|\input{\childdocname}|\\
|\||else|\\
\textit{document body with }|\input{|\textit{part}|}|\\
|\||fi|
\end{tabular}
\end{center}
%
The conditional |\ifchilddocmanual| is true whenever
a part to be included by |\input| is being compiled,
and the name of the part is stored in |\childdocname|.

%%%%%%%%%%%%%%%%%%%%%%%%%%%%%%%%%%%%%%%%
\DescribeMacro{\childdocby}
Each part to be included by |\input| should start with:
%
\begin{center}
\begin{tabular}{l}
|\input{childdoc.def}|\\
|\childdocby{|\textit{main}|}|\\
\end{tabular}
\end{center}
%
The directive |\childdocby| is similar to |\childdocof|
described in \secref{sec:include},
but the subsequent selection of content must be done manually.
To that end, both |\ifchilddoc| and |\ifchilddocmanual|
will be true upon processing of a part,
and the name of the part is stored in |\childdocname|.
Note that |\jobname| will be set to the filename of the current part
so that each part receives an individual |.aux| file
that does not interfere with the |.aux| file(s) of the main document.
This behaviour can be altered by the alternative form
|\childdocby[*]{|\textit{main}|}| (with a non-empty optional argument)
which uses the |.aux| file of the main document
by setting |\jobname| to \textit{main}.

%%%%%%%%%%%%%%%%%%%%%%%%%%%%%%%%%%%%%%%%%%%%%%%%%%%%%%%%%%%%%%%%%%%%%%%%%%%%%%%%
\subsection{Driver Development}
\label{sec:driver}

The \textsf{childdoc} mechanism can also be use for the development
of definition files such as \LaTeX{} styles or classes.
This case differs from the above setup with multiple parts
included by |\include| in that no |\includeonly| should be invoked.
This can be achieved by starting the include file
(before |\ProvidesPackage|) with:
%
\begin{center}
\begin{tabular}{l}
|\input{childdoc.def}|\\
|\childdocforward{|\textit{main}|}|\\
\end{tabular}
\end{center}
%
or alternatively with:
%
\begin{center}
\begin{tabular}{l}
|\input{childdoc.def}|\\
|\childdocby{|\textit{main}|}|\\
\end{tabular}
\end{center}
%
Both forms have slightly different effects as described above.
The main file is prepared as usual, see \secref{sec:include}.

%%%%%%%%%%%%%%%%%%%%%%%%%%%%%%%%%%%%%%%%%%%%%%%%%%%%%%%%%%%%%%%%%%%%%%%%%%%%%%%%
\subsection{Legacy Detection}
\label{sec:detection}

The directive |\childdocmain| in the main file can detect
whether the complete document or merely a child is to be compiled
even without using the directive |\childdocof|.
This method is deprecated because it is less robust
and there is no compelling reason to use it;
it is merely provided for backward compatibility
and it may be removed in future versions.

If the detection mechanism is to be used,
it is mandatory to correctly specify
the filename of the main file as the argument of |\childdocmain|:
%
\begin{center}
\begin{tabular}{l}
|\input{childdoc.def}|\\
|\childdocmain{|\textit{main}|}|\\
\end{tabular}
\end{center}
%
If |\jobname| does not match the argument \textit{main} of |\childdocmain|,
it is assumed that |\jobname| points to the child file to be compiled.
When using |\childdocmain| with the main file specified as argument,
it suffices to start a child file
with just |\input{|\textit{main}|}|
without loading of the package and using |\childdocof|.
If instead all processing is done
with the appropriate \textsf{childdoc} directives,
the argument of \textit{main} of |\childdocmain| can be empty.

An alternative version of the command line processing described
in \secref{sec:commandline} using the detection mechanism reads:
%
\begin{center}
|... -jobname "|\textit{target}|" "|[\textit{flags}]%
[|\def\jobname{|\textit{dest}|}|]|\input{|\textit{main}|}"|
\end{center}

%%%%%%%%%%%%%%%%%%%%%%%%%%%%%%%%%%%%%%%%%%%%%%%%%%%%%%%%%%%%%%%%%%%%%%%%%%%%%%%%
\subsection{Manual Code}
\label{sec:manual}

In case one cannot be certain whether the definitions file |childdoc.def|
is installed on the target \TeX{} distribution
and one prefers not to ship it,
it is conceivable to paste a few relevant commands into the sources.

To that end, drop all statements |\input{childdoc.def}|
and perform the replacements as outlined below.
Instead of |\childdocmain{|\textit{main}|}| add the following code
to the top of the main file:
%
\begin{center}
\begin{tabular}{l}
|\||ifdefined\childdocname\endinput\||fi\newif\ifchilddoc|\\
|\edef\childdocname{\scantokens\expandafter{\jobname\noexpand}}|\\
|\def\childdocmain{|\textit{main}|}\||ifx\childdocmain\childdocname\||else|\\
|\childdoctrue\includeonly{\childdocname}\let\jobname\childdocmain\||fi|\\
\end{tabular}
\end{center}
%
Instead of |\childdocof{|\textit{main}|}| just include the main file
at the top of each child file:
%
\begin{center}
|\input{|\textit{main}|}|
\end{center}
%
A simple redirection |\childdocforward{|\textit{dest}|}| is achieved by:
%
\begin{center}
|\def\jobname{|\textit{dest}|}\input{\jobname}|
\end{center}
%
The redirection with prefix
|\childdocforwardprefix[|\textit{prefix}|]{|\textit{dest}|}|
is accomplished by:
%
\begin{center}
\begin{tabular}{l}
|{\edef\jobname{\scantokens\expandafter{\jobname\noexpand}}|\\
|\def\redirectjob |\textit{prefix}|#1~~~{\gdef\jobname{|\textit{dest}|#1}}|\\
|\expandafter\redirectjob\jobname~~~}\input{\jobname}|
\end{tabular}
\end{center}

In an alternative approach,
child documents can be compiled by a specific command line
without additional code or specific definitions:
%
\begin{center}
|... -jobname "|\textit{target}|" "|[\textit{flags}]%
|\includeonly{|\textit{dest}|}\input{|\textit{main}|}"|
\end{center}
%

%%%%%%%%%%%%%%%%%%%%%%%%%%%%%%%%%%%%%%%%%%%%%%%%%%%%%%%%%%%%%%%%%%%%%%%%%%%%%%%%
%%%%%%%%%%%%%%%%%%%%%%%%%%%%%%%%%%%%%%%%%%%%%%%%%%%%%%%%%%%%%%%%%%%%%%%%%%%%%%%%
\section{Information}

%%%%%%%%%%%%%%%%%%%%%%%%%%%%%%%%%%%%%%%%%%%%%%%%%%%%%%%%%%%%%%%%%%%%%%%%%%%%%%%%
\subsection{Copyright}

Copyright \copyright{} 2017--2018 Niklas Beisert

This work may be distributed and/or modified under the
conditions of the \LaTeX{} Project Public License, either version 1.3
of this license or (at your option) any later version.
The latest version of this license is in
  \url{http://www.latex-project.org/lppl.txt}
and version 1.3 or later is part of all distributions of \LaTeX{}
version 2005/12/01 or later.

This work has the LPPL maintenance status `maintained'.

The Current Maintainer of this work is Niklas Beisert.

This work consists of the files |README.txt|, |childdoc.ins| and |childdoc.dtx|
as well as the derived files |childdoc.def|, |cdocsamp.tex|
with |cdocsch1.tex|, |cdocsch2.tex|, |cdocspt3.tex|, |cdocspt4.tex|,
|cdocsdrf.tex|, |cdocsfn1.tex|, |cdocsfn2.tex|
as well as |childdoc.pdf|.

%%%%%%%%%%%%%%%%%%%%%%%%%%%%%%%%%%%%%%%%%%%%%%%%%%%%%%%%%%%%%%%%%%%%%%%%%%%%%%%%
\subsection{Files and Installation}

The package consists of the files:
%
\begin{center}
\begin{tabular}{ll}
    |README.txt|   & readme file \\
    |childdoc.ins| & installation file \\
    |childdoc.dtx| & source file \\
    |childdoc.def| & definition file \\
    |cdocsamp.tex| & sample main file \\
    |cdocsch1.tex| & sample include file \\
    |cdocsch2.tex| & sample include file \\
    |cdocspt3.tex| & sample part file \\
    |cdocspt4.tex| & sample part file \\
    |cdocsdrf.tex| & sample redirection file \\
    |cdocsfn1.tex| & sample redirection file \\
    |cdocsfn2.tex| & sample redirection file \\
    |childdoc.pdf| & manual
\end{tabular}
\end{center}
%
The distribution consists of the files
|README.txt|, |childdoc.ins| and |childdoc.dtx|.
%
\begin{itemize}
\item
Run (pdf)\LaTeX{} on |childdoc.dtx|
to compile the manual |childdoc.pdf| (this file).
\item
Run \LaTeX{} on |childdoc.ins| to create the definitions file |childdoc.def|
and the sample |cdocsamp.tex| with include files
|cdocsch1.tex|, |cdocsch2.tex|, |cdocspt3.tex|, |cdocspt4.tex|,
|cdocsdrf.tex|, |cdocsfn1.tex|, |cdocsfn2.tex|.
Then copy the file |childdoc.def| to an appropriate directory of your \LaTeX{}
distribution, e.g.\ \textit{texmf-root}|/tex/latex/childdoc|.
\end{itemize}

%%%%%%%%%%%%%%%%%%%%%%%%%%%%%%%%%%%%%%%%%%%%%%%%%%%%%%%%%%%%%%%%%%%%%%%%%%%%%%%%
\subsection{Related CTAN Packages}

There are several other packages which offer a similar functionality:
%
\begin{itemize}
\item
The packages
\href{http://ctan.org/pkg/docmute}{\textsf{docmute}},
\href{http://ctan.org/pkg/includex}{\textsf{includex}} and
\href{http://ctan.org/pkg/standalone}{\textsf{standalone}}
provide commands to include only the document body of
a child file thus allowing both files to be compiled individually.
\item
The packages \href{http://ctan.org/pkg/subdocs}{\textsf{subdocs}}
and \href{http://ctan.org/pkg/subfiles}{\textsf{subfiles}}
provide structures in which the main and child documents can be
encapsulated and allowing them to be compiled individually.
The inclusion mechanism is different from the conventional |\include|.
\item
The package \href{http://ctan.org/pkg/combine}{\textsf{combine}}
is an elaborate solution to combine several documents into one.
\end{itemize}
%
See also the CTAN topic \href{http://ctan.org/topic/subdocs}{\textsf{subdocs}}
for further related packages.
The present package differs from the above solutions in that
a document structure constructed with the conventional |\include| mechanism
just needs two extra commands at the top of every file
such that all constituent files can be compiled individually.

%%%%%%%%%%%%%%%%%%%%%%%%%%%%%%%%%%%%%%%%%%%%%%%%%%%%%%%%%%%%%%%%%%%%%%%%%%%%%%%%
%\subsection{Feature Suggestions}
%
%The following is a list of features which may be useful for future
%versions of this package:
%%
%\begin{itemize}
%\item
%\ldots
%\end{itemize}

%%%%%%%%%%%%%%%%%%%%%%%%%%%%%%%%%%%%%%%%%%%%%%%%%%%%%%%%%%%%%%%%%%%%%%%%%%%%%%%%
\subsection{Revision History}

%%%%%%%%%%%%%%%%%%%%%%%%%%%%%%%%%%%%%%%%
\paragraph{v2.0:} 2018/12/30

\begin{itemize}
\item
immediate forward processing
\item
added |\childdocby| mechanism
\item
manual restructured
\end{itemize}

%%%%%%%%%%%%%%%%%%%%%%%%%%%%%%%%%%%%%%%%
\paragraph{v1.6:} 2018/01/17

\begin{itemize}
\item
application for development of include files
\item
corrections to manual
\end{itemize}

%%%%%%%%%%%%%%%%%%%%%%%%%%%%%%%%%%%%%%%%
\paragraph{v1.5:} 2017/05/21

\begin{itemize}
\item
more complete structuring introduced
\item
|\childdocof| introduced
\item
|\childdoc| renamed to |\childdocmain|
\item
|\childredirect| renamed to |\childdocforward| and |\childdocforwardprefix|
and functionality expanded
\end{itemize}

%%%%%%%%%%%%%%%%%%%%%%%%%%%%%%%%%%%%%%%%
\paragraph{v1.0:} 2017/04/27

\begin{itemize}
\item
manual and install package
\item
first version published on CTAN
\end{itemize}

%%%%%%%%%%%%%%%%%%%%%%%%%%%%%%%%%%%%%%%%
\paragraph{v0.6:} 2017/04/26

\begin{itemize}
\item
redirection mechanism added
\end{itemize}

%%%%%%%%%%%%%%%%%%%%%%%%%%%%%%%%%%%%%%%%
\paragraph{v0.5:} 2017/04/26

\begin{itemize}
\item
functionality in definition file
\end{itemize}


%%%%%%%%%%%%%%%%%%%%%%%%%%%%%%%%%%%%%%%%%%%%%%%%%%%%%%%%%%%%%%%%%%%%%%%%%%%%%%%%
%%%%%%%%%%%%%%%%%%%%%%%%%%%%%%%%%%%%%%%%%%%%%%%%%%%%%%%%%%%%%%%%%%%%%%%%%%%%%%%%
%%%%%%%%%%%%%%%%%%%%%%%%%%%%%%%%%%%%%%%%%%%%%%%%%%%%%%%%%%%%%%%%%%%%%%%%%%%%%%%%
\appendix

\settowidth\MacroIndent{\rmfamily\scriptsize 000\ }

 \DocInput{childdoc.dtx}

\end{document}
%</driver>
% \fi
%
% %%%%%%%%%%%%%%%%%%%%%%%%%%%%%%%%%%%%%%%%%%%%%%%%%%%%%%%%%%%%%%%%%%%%%%%%%%%%%%
% %%%%%%%%%%%%%%%%%%%%%%%%%%%%%%%%%%%%%%%%%%%%%%%%%%%%%%%%%%%%%%%%%%%%%%%%%%%%%%
% \section{Sample}
%\iffalse
%<*samplemain>
%\fi
%
% The following presents a sample document
% with two chapters, two parts, a title page,
% a compile flag as well as three forwarding files to set the flag.
% It consists of eight |.tex| files:
% \begin{center}
% \begin{tabular}{ll}
% |cdocsamp.tex|&main file\\
% |cdocsch1.tex|&include file for chapter 1\\
% |cdocsch2.tex|&include file for chapter 2\\
% |cdocspt3.tex|&include file for part 3\\
% |cdocspt4.tex|&include file for part 4\\
% |cdocsdrf.tex|&forwarding file for main file in draft mode\\
% |cdocsfi1.tex|&forwarding file for final version of chapter 1\\
% |cdocsfi2.tex|&forwarding file for final version of chapter 2\\
% \end{tabular}
% \end{center}
% Each of the eight files can be compiled directly by the \LaTeX{} compiler.
%
% %%%%%%%%%%%%%%%%%%%%%%%%%%%%%%%%%%%%%%
% \paragraph{Main File.}
%
% The main file is called |cdocsamp.tex|.
%
% Load the \textsf{childdoc} definitions and
% declare the filename for the main document:
%    \begin{macrocode}
\input{childdoc.def}
\childdocmain{}
%    \end{macrocode}

% Optional override for |\version| flag:
%    \begin{macrocode}
%%\ifchilddoc\else\providecommand{\version}{draft}\fi
%    \end{macrocode}

% Define the default values for the |\version| flag
% (|final| for the main file and |draft| for childs):
%    \begin{macrocode}
\ifchilddoc
\providecommand{\version}{draft}
\else
\providecommand{\version}{final}
\fi
%    \end{macrocode}

% Load the standard document class:
%    \begin{macrocode}
\documentclass[12pt]{article}
%    \end{macrocode}

% Start the document body:
%    \begin{macrocode}
\begin{document}
%    \end{macrocode}

% Declare a title page.
% Print title, part of document being processed and version flag:
%    \begin{macrocode}
\addtocounter{page}{-1}
\begin{center}
{\LARGE\bfseries{}childdoc example\par}
\vspace{1cm}
\ifchilddoc
\ifchilddocmanual part\else chapter\fi:
`\childdocname' of `\childdocjob'\par
\else
main document: `\childdocjob'\par
\fi
version: \version\par
\end{center}
\newpage
%    \end{macrocode}

% Manually include selected file,
% otherwise process as usual:
%    \begin{macrocode}
\ifchilddocmanual
\section*{part `\childdocname'}
\input{\childdocname}
\else
%    \end{macrocode}

% Include the two chapters:
%    \begin{macrocode}
\include{cdocsch1}
\include{cdocsch2}
%    \end{macrocode}

% Include the two parts unless only chapters should be displayed:
%    \begin{macrocode}
\ifchilddoc\else
\section{part three}
\input{cdocspt3}
\section{part four}
\input{cdocspt4}
\fi
%    \end{macrocode}

% Process as usual until here:
%    \begin{macrocode}
\fi
%    \end{macrocode}

% End of document body:
%    \begin{macrocode}
\end{document}
%    \end{macrocode}
%\iffalse
%</samplemain>
%\fi
%
% %%%%%%%%%%%%%%%%%%%%%%%%%%%%%%%%%%%%%%
% \paragraph{Chapter Include Files.}
%
% The include files are called |cdocsch1.tex| and |cdocsch2.tex|.
%
%\iffalse
%<*samplechap1|samplechap2>
%\fi

% Optional override for |\version| flag:
%    \begin{macrocode}
%%\providecommand{\version}{final}
%    \end{macrocode}

% Include the main document:
%    \begin{macrocode}
\input{childdoc.def}
\childdocof{cdocsamp}
%    \end{macrocode}

%\iffalse
%</samplechap1|samplechap2>
%\fi
%
%\iffalse
%<*samplechap1>
%\fi
% Some text for chapter 1:
%    \begin{macrocode}
\section{one}
some text in chapter one
%    \end{macrocode}

%\iffalse
%</samplechap1>
%\fi
% Some text for chapter 2:
%\iffalse
%<*samplechap2>
%\fi
%    \begin{macrocode}
\section{two}
more text in chapter two
%    \end{macrocode}

%\iffalse
%</samplechap2>
%\fi
%
% %%%%%%%%%%%%%%%%%%%%%%%%%%%%%%%%%%%%%%
% \paragraph{Part Include Files.}
%
% The include files are called |cdocspt3.tex| and |cdocspt4.tex|.
%
%\iffalse
%<*samplepart3|samplepart4>
%\fi

% Optional override for |\version| flag:
%    \begin{macrocode}
%%\providecommand{\version}{final}
%    \end{macrocode}

% Include the main document:
%    \begin{macrocode}
\input{childdoc.def}
\childdocby{cdocsamp}
%    \end{macrocode}

%\iffalse
%</samplepart3|samplepart4>
%\fi
%
%\iffalse
%<*samplepart3>
%\fi
% Some text for part 3:
%    \begin{macrocode}
some text in part three
%    \end{macrocode}

%\iffalse
%</samplepart3>
%\fi
% Some text for part 4:
%\iffalse
%<*samplepart4>
%\fi
%    \begin{macrocode}
more text in part four
%    \end{macrocode}

%\iffalse
%</samplepart4>
%\fi
%
% %%%%%%%%%%%%%%%%%%%%%%%%%%%%%%%%%%%%%%
% \paragraph{Forwarding for a Complete Draft.}
%
% The following forwarding file |cdocsdrf.tex|
% compiles the main document in draft mode:
%\iffalse
%<*sampledraft>
%\fi
%    \begin{macrocode}
\def\version{draft}
\input{childdoc.def}
\childdocforward{cdocsamp}
%    \end{macrocode}

%\iffalse
%</sampledraft>
%\fi
%
% %%%%%%%%%%%%%%%%%%%%%%%%%%%%%%%%%%%%%%
% \paragraph{Forwarding for Final Version of the Chapters.}
%
% The following forwarding files |cdocsfn1.tex| and |cdocsfn2.tex|
% (with identical content)
% compile the final versions of the child documents
% |cdocsch1.tex| and |cdocsch2.tex|, respectively:
%\iffalse
%<*samplefinal>
%\fi
%    \begin{macrocode}
\def\version{final}
\input{childdoc.def}
\childdocforwardprefix[cdocsamp]{cdocsfn}{cdocsch}
%    \end{macrocode}

%\iffalse
%</samplefinal>
%\fi
%
% %%%%%%%%%%%%%%%%%%%%%%%%%%%%%%%%%%%%%%
% \paragraph{Command Line Processing.}
%
% The following three command lines generate the output files
% |cdocscld|, |cdocscl1| and |cdocscl2|
% which should be identical to
% |cdocsdrf|, |cdocsch1| and |cdocsfn2|, respectively:
% \begin{center}
% \begin{tabular}{l}
% |latex -jobname cdocscld \|\\
% |  "\def\version{draft}\input{childdoc.def}\childdocforward{cdocsamp}"|\\
% |latex -jobname cdocscl1 \|\\
% |  "\input{childdoc.def}\childdocforward[cdocsamp]{cdocsch1}"|\\
% |latex -jobname cdocscl2 \|\\
% |  "\def\version{final}\input{childdoc.def}\childdocforward{cdocsch2}"|
% \end{tabular}
% \end{center}
% Note that the trailing backslash on each first line
% merely continues the input to the second line
% (for convenient cut ant paste).
% Furthermore, the command |latex| can be replaced by any
% of its alternative versions such as |pdflatex|.
%
% %%%%%%%%%%%%%%%%%%%%%%%%%%%%%%%%%%%%%%%%%%%%%%%%%%%%%%%%%%%%%%%%%%%%%%%%%%%%%%
% %%%%%%%%%%%%%%%%%%%%%%%%%%%%%%%%%%%%%%%%%%%%%%%%%%%%%%%%%%%%%%%%%%%%%%%%%%%%%%
% \section{Implementation}
%\iffalse
%<*package>
%\fi
%
% This section describes the definitions file |childdoc.def|.

% The definitions cannot be loaded using |\usepackage| or |\RequirePackage|
% which has a mechanism to prevent loading a style file more than once.
% When loading the definitions by means of |\input|
% multiple instances have to be prevented manually:
%\iffalse
%This code needs to be before the `\ProvidesFile' directive
%which is defined at the beginning of this file.
%Therefore it is also placed there and commented out here.
%</package>
%<*discard>
%\fi
%    \begin{macrocode}
\ifdefined\childdocmain\endinput\fi
%    \end{macrocode}
%\iffalse
%</discard>
%<*package>
%\fi
%
% \macro{\ifchilddoc}
% \macro{\ifchilddocmanual}
% The conditional |\ifchilddoc| tells whether a
% child (true) or main (false) document is being compiled.
% The conditional |\ifchilddocmanual| tells whether
% the |\includeonly| mechanism is used (false) or
% the selection of child files must be performed manually (true).
% The definitions initialise to false:
%    \begin{macrocode}
\newif\ifchilddoc
\newif\ifchilddocmanual
%    \end{macrocode}

% \macro{\childdocname}
% \macro{\childdocjob}
% The macro |\childdocname| stores the name of the main document
% to be compiled. The macro |\childdocjob| stores the name of
% the document on which the \LaTeX{} compiler was originally invoked.
% The content of |\jobname| cannot be compared
% to filenames specified in the source due to different catcodes.
% The following code rescans |\jobname|, stores the result
% in |\childdocname| and saves a copy in |\childdocjob|:
%    \begin{macrocode}
\edef\childdocname{\scantokens\expandafter{\jobname\noexpand}}
\let\childdocjob\childdocname
%    \end{macrocode}

% \macro{\childdocdisable}
% The macro |\childdocdisable| prevents the main file
% from being processed more than once.
% At this stage, the main document command |\childdocmain|
% is assumed to be called once again where it should do nothing.
% Any subsequent call to it should prevent
% a secondary processing of the main document
% It overwrites the forwarding commands
% |\childdocof| and |\childdocforward|
% with empty macros to prevent further inclusions of the main document:
%    \begin{macrocode}
\newcommand{\childdocdisable}
{
  \renewcommand{\childdocmain}[1]{\renewcommand{\childdocmain}[1]{\endinput}}
  \renewcommand{\childdocof}[1]{}
  \renewcommand{\childdocby}[2][]{}
  \renewcommand{\childdocforward}[2][]{}
  \renewcommand{\childdocdisable}{}
}
%    \end{macrocode}

% \macro{\childdocmain}
% The macro |\childdocmain| is to be called at the top of the main file
% with nothing or the main filename (without extension) as argument.
% First, it breaks loops.
% If the argument is not empty and does not match |\childdocname|
% (which is set by the first inclusion of |childdoc.def|),
% |\ifchilddoc| is set to true, |\includeonly| is applied to the child file
% and |\jobname| is set to the main file
% (for proper handling of |.aux| files):
%    \begin{macrocode}
\newcommand{\childdocmain}[1]
{
  \childdocdisable\childdocmain{}
  \if?#1?\else
    \begingroup
      \def\childdoctmp{#1}
      \ifx\childdoctmp\childdocname
        \def\childdoctmp{}
      \else
        \def\childdoctmp
        {
          \childdoctrue
          \includeonly{\childdocname}
          \def\childdocjob{#1}
          \def\jobname{#1}
        }
      \fi
      \expandafter
    \endgroup
    \childdoctmp
  \fi
}
%    \end{macrocode}

% \macro{\childdocof}
% The command |\childdocof| redirects
% compilation to the main file |#1|.
%    \begin{macrocode}
\newcommand{\childdocof}[1]
{
  \childdocdisable
  \childdoctrue
  \includeonly{\childdocname}
  \def\jobname{#1}
  \def\childdocjob{#1}
  \input{#1}
}
%    \end{macrocode}

% \macro{\childdocby}
% The command |\childdocby| ....
%    \begin{macrocode}
\newcommand{\childdocby}[2][]
{
  \childdocdisable
  \childdoctrue
  \childdocmanualtrue
  \if?#1?\else
    \def\jobname{#2}
  \fi
  \def\childdocjob{#2}
  \input{#2}
  \endinput
}
%    \end{macrocode}

% \macro{\childdocforward}
% The command |\childdocforward| redirects
% compilation to the main file or
% (if the optional argument is given) a child file.
% Parameters are set as if the main file
% or a child file starting with |\childdocof| was compiled.
% Then compilation is handed over to the main file:
%    \begin{macrocode}
\newcommand{\childdocforward}[2][]
{
  \begingroup
    \if?#1?
      \def\childdoctmp
      {
        \def\childdocname{#2}
        \def\childdocjob{#2}
        \def\jobname{#2}
        \input{#2}
        \endinput
      }
    \else
      \def\childdoctmp
      {
        \childdocdisable
        \def\childdocname{#2}
        \childdoctrue
        \includeonly{#2}
        \def\childdocjob{#1}
        \def\jobname{#1}
        \input{#1}
        \endinput
      }
    \fi
    \expandafter
  \endgroup
  \childdoctmp
}
%    \end{macrocode}

% \macro{\childdocforwardprefix}
% The command |\childdocforwardprefix| redirects
% compilation to the main or a child file by means of a pattern.
% The prefix |#1| in the current filename is replaced by |#2|
% and the suffix of the current filename is kept
% (it is assumed that the filename does not contain the substring `|~~~|'
% which is used as a delimiter).
% Compilation is handed over to the new file by |\childdocforward|:
%    \begin{macrocode}
\newcommand{\childdocforwardprefix}[3][]
{
  \begingroup
    \def\childdocextract #2##1~~~{\def\childdoctmp{\childdocforward[#1]{#3##1}}}
    \expandafter\childdocextract\childdocname~~~
    \expandafter
  \endgroup
  \childdoctmp
}
%    \end{macrocode}

% \macro{\childdoc}
% The deprecated macro |\childdoc| is a legacy version of |\childdocmain|:
%    \begin{macrocode}
\newcommand{\childdoc}{\childdocmain}
%    \end{macrocode}

% \macro{\childdocredirect}
% The deprecated macro |\childdocredirect| is a legacy version
% of |\childdocforward| and |\childdocforwardprefix|:
%    \begin{macrocode}
\newcommand{\childdocredirect}[2][]
{
  \begingroup
    \if?#1?
      \def\childdoctmp{\childdocforward{#2}}
    \else
      \def\childdoctmp{\childdocforwardprefix{#1}{#2}}
    \fi
    \expandafter
  \endgroup
  \childdoctmp
}
%    \end{macrocode}

%\iffalse
%</package>
%\fi
%
\endinput
|\\
|\childdocmain{}|\\
\end{tabular}
\end{center}
at the very top of the main \LaTeX{} file,
in particular \emph{before} the |\documentclass| statement!
The argument of |\childdocmain| should be left empty
(but it must be present).

%%%%%%%%%%%%%%%%%%%%%%%%%%%%%%%%%%%%%%%%
\DescribeMacro{\childdocof}
Furthermore, add the commands
\begin{center}
\begin{tabular}{l}
|% \iffalse
%
% childdoc.dtx Copyright (C) 2017-2018 Niklas Beisert
%
% This work may be distributed and/or modified under the
% conditions of the LaTeX Project Public License, either version 1.3
% of this license or (at your option) any later version.
% The latest version of this license is in
%   http://www.latex-project.org/lppl.txt
% and version 1.3 or later is part of all distributions of LaTeX
% version 2005/12/01 or later.
%
% This work has the LPPL maintenance status `maintained'.
%
% The Current Maintainer of this work is Niklas Beisert.
%
% This work consists of the files childdoc.dtx and childdoc.ins
% and the derived files childdoc.def and cdocsamp.tex with
% cdocsch1.tex, cdocsch2.tex, cdocsdrf.tex, cdocsfn1.tex, cdocsfn2.tex.
%
%<package>\ifdefined\childdocmain\endinput\fi
%<package>\ProvidesFile{childdoc.def}[2018/12/30 v2.0 child document driver]
%<samplemain>\ProvidesFile{cdocsamp.tex}[2018/12/30 v2.0 sample for childdoc]
%<*driver>
%\ProvidesFile{childdoc.drv}[2018/12/30 v2.0 childdoc reference manual file]
\PassOptionsToClass{10pt,a4paper}{article}
\documentclass{ltxdoc}

\usepackage[margin=35mm]{geometry}
\usepackage{hyperref}
\usepackage{hyperxmp}
\usepackage[usenames]{color}

\hypersetup{colorlinks=true}
\hypersetup{pdfstartview=FitH}
\hypersetup{pdfpagemode=UseNone}
\hypersetup{pdfsource={}}
\hypersetup{pdflang={en-UK}}
\hypersetup{pdfcopyright={Copyright 2017-2018 Niklas Beisert.
  This work may be distributed and/or modified under the
  conditions of the LaTeX Project Public License, either version 1.3
  of this license or (at your option) any later version.}}
\hypersetup{pdflicenseurl={http://www.latex-project.org/lppl.txt}}
\hypersetup{pdfcontactaddress={ETH Zurich, ITP, HIT K,
  Wolfgang-Pauli-Strasse 27}}
\hypersetup{pdfcontactpostcode={8093}}
\hypersetup{pdfcontactcity={Zurich}}
\hypersetup{pdfcontactcountry={Switzerland}}
\hypersetup{pdfcontactemail={nbeisert@itp.phys.ethz.ch}}
\hypersetup{pdfcontacturl={http://people.phys.ethz.ch/\xmptilde nbeisert/}}

\newcommand{\secref}[1]{\hyperref[#1]{section \ref*{#1}}}

\parskip1ex
\parindent0pt
\let\olditemize\itemize
\def\itemize{\olditemize\parskip0pt}

\begin{document}

\title{The \textsf{childdoc} Package}
\hypersetup{pdftitle={The childdoc Package}}
\author{Niklas Beisert\\[2ex]
  Institut f\"ur Theoretische Physik\\
  Eidgen\"ossische Technische Hochschule Z\"urich\\
  Wolfgang-Pauli-Strasse 27, 8093 Z\"urich, Switzerland\\[1ex]
  \href{mailto:nbeisert@itp.phys.ethz.ch}
  {\texttt{nbeisert@itp.phys.ethz.ch}}}
\hypersetup{pdfauthor={Niklas Beisert}}
\hypersetup{pdfsubject={Manual for the LaTeX2e Package childdoc}}
\date{30 December 2018, \textsf{v2.0}}
\maketitle

\begin{abstract}\noindent
\textsf{childdoc} is a \LaTeXe{} package
that enables the direct compilation
of document sections included by |\include|
to individual files.
\end{abstract}

\begingroup
\parskip0ex
\tableofcontents
\endgroup

%%%%%%%%%%%%%%%%%%%%%%%%%%%%%%%%%%%%%%%%%%%%%%%%%%%%%%%%%%%%%%%%%%%%%%%%%%%%%%%%
%%%%%%%%%%%%%%%%%%%%%%%%%%%%%%%%%%%%%%%%%%%%%%%%%%%%%%%%%%%%%%%%%%%%%%%%%%%%%%%%
\section{Introduction}

\LaTeX{} provides a mechanism to structure a large document (such as a book)
into a main file and several child files (containing the chapters)
using the |\include| command.
This mechanism is beneficial for documents
which span hundreds of pages in order to
make the source file(s) more manageable.
Moreover, compilation can be restricted to
selected child files by means of the |\includeonly| command.
The latter feature can be used to reduce the compilation time while editing
(this was significantly more useful in the earlier days of \LaTeX{})
or to generate a smaller document which is easier to navigate.
Another application of |\includeonly| is to generate
documents consisting of selected parts of the complete document.

However, there are a few drawbacks of the plain |\include| mechanism:
\begin{itemize}
\item
The child files cannot be compiled on their own,
they can only be compiled via the main file.
A naive editing environment
(such as a text editor with an option
to have the current file processed by \LaTeX)
may require one to switch to the main file before compiling;
attempting to compile the child file produces errors.
\item
The main file must be modified (each time)
to adjust the |\includeonly| command
to the present needs. This easily leaves the main file in a messy state.
\item
The generated document will always carry the filename
of the main document. This is inconvenient if
several child files are to be compiled and
to be kept for distribution.
\end{itemize}

The present package provides a simple interface
to make child files individually compilable by \LaTeX{}.
Compiling a child file then has the same effect as compiling
the main file with an |\includeonly| command
to select the appropriate child.
Moreover the generated document will carry the name of the child
rather than the main file.
This resolves all three above issues.

This feature is meant to make the editing of books,
thesis documents and lecture notes somewhat more convenient.
However, the package can also be used efficiently for
composing a series of documents (such as exercise sheets)
which are typically distributed individually.
It then assists the author in generating the individual documents
(potentially in different versions)
as well as a document containing the collected series.
Another application is in developing style files
or other kinds of included material
where compilation of the style file could redirect
to a sample or test file.

%%%%%%%%%%%%%%%%%%%%%%%%%%%%%%%%%%%%%%%%%%%%%%%%%%%%%%%%%%%%%%%%%%%%%%%%%%%%%%%%
%%%%%%%%%%%%%%%%%%%%%%%%%%%%%%%%%%%%%%%%%%%%%%%%%%%%%%%%%%%%%%%%%%%%%%%%%%%%%%%%
\section{Usage}

First of all, the package \textsf{childdoc} is \emph{not} a standard
\LaTeXe{} |.sty| style file! Therefore it needs to be invoked in
a non-standard way.

%%%%%%%%%%%%%%%%%%%%%%%%%%%%%%%%%%%%%%%%%%%%%%%%%%%%%%%%%%%%%%%%%%%%%%%%%%%%%%%%
\subsection{Included Files}
\label{sec:include}

%%%%%%%%%%%%%%%%%%%%%%%%%%%%%%%%%%%%%%%%
\DescribeMacro{\childdocmain}
To use the package, add the commands
\begin{center}
\begin{tabular}{l}
|\input{childdoc.def}|\\
|\childdocmain{}|\\
\end{tabular}
\end{center}
at the very top of the main \LaTeX{} file,
in particular \emph{before} the |\documentclass| statement!
The argument of |\childdocmain| should be left empty
(but it must be present).

%%%%%%%%%%%%%%%%%%%%%%%%%%%%%%%%%%%%%%%%
\DescribeMacro{\childdocof}
Furthermore, add the commands
\begin{center}
\begin{tabular}{l}
|\input{childdoc.def}|\\
|\childdocof{|\textit{main}|}|\\
\end{tabular}
\end{center}
at the top of every child file \textit{child}
which is included by |\include{|\textit{child}|}|
from within the main file
(or at least for those files to be compiled individually).
The argument \textit{main} must be the filename of the main file.

There are a couple of
considerations in setting up the main and child documents:

%%%%%%%%%%%%%%%%%%%%%%%%%%%%%%%%%%%%%%%%
\paragraph{Restrictions.}

Please note the following restrictions:
\begin{itemize}
\item
|\childdocmain| must be called with one argument \textit{main}
to ensure compatibility with earlier version of the package.
It must either be empty (|\childdocmain{}|)
or precisely match the filename of the main file in which it is specified.
See \secref{sec:detection} for further information.
\item
The filename \textit{main} must be specified without the |.tex| extension.
\item
The filename \textit{main} is case sensitive
(even in case-insensitive file systems)
due to internal string comparison.
\item
The argument \textit{main} should be fully expanded, it cannot be a macro.
\item
Subdirectories and special characters should be avoided in filenames.
\item
The command |\childdocmain{|\textit{main}|}| must be followed by a whitespace.
It should not be followed immediately by another command
or by a comment mark `|%|'.
This is because the \TeX{} parser reads the token immediately following
the argument of |\childdocmain| and puts it
at the beginning of every child section;
however, a white\-space is ignored.
\end{itemize}

%%%%%%%%%%%%%%%%%%%%%%%%%%%%%%%%%%%%%%%%
\paragraph{Content of Main File.}

It is advisable to place all content in the child files included by |\include|.
Any output contained in the main file will appear in all child documents
unless suppressed manually;
it cannot be suppressed automatically by the |\includeonly| directive
and thus should normally be avoided.
A method to include some content in the main file
by means of conditional processing is described in \secref{sec:conditional}.

%%%%%%%%%%%%%%%%%%%%%%%%%%%%%%%%%%%%%%%%
\paragraph{Page Numbering.}

When only a part of the document is compiled,
the appropriate numbering of pages
(as well as other status parameters)
is determined from the |.aux| files.
The latter contain information from previous passes.
However this information needs to propagate through
all intermediate child documents.
Therefore the page numbering in child documents may well
be inconsistent until the complete document is compiled at least once.

A useful (if unconventional) way to always ensure a consistent
page numbering is to restart the numbering in each child document
and denote the pages by `\textit{child}|.|\textit{page}'
where \textit{child} represents the chapter/section number of the child file.
This can be achieved by the command
|\numberwithin{page}{|\textit{child}|}|
of the \textsf{amsmath} package
where \textit{child} can be |chapter| or |section|
depending on the chosen structuring.
Alternatively, one can modify the macro |\thepage| appropriately
and reset the counter |page| at the start of each child file.

%%%%%%%%%%%%%%%%%%%%%%%%%%%%%%%%%%%%%%%%%%%%%%%%%%%%%%%%%%%%%%%%%%%%%%%%%%%%%%%%
\subsection{Conditional Processing}
\label{sec:conditional}

The package provides a mechanism to compile different versions
of a document. To customise the versions further some conditional processing
can come in handy to distinguish which version is being compiled.
The package provides two macros to describe the compilation context:

%%%%%%%%%%%%%%%%%%%%%%%%%%%%%%%%%%%%%%%%
\DescribeMacro{\ifchilddoc}
The conditional |\ifchilddoc| distinguishes between the compilation of
child documents and the main document:
%
\begin{center}
|\ifchilddoc |\textit{child-code}| |[|\||else |\textit{main-code}]| \||fi|
\end{center}

%%%%%%%%%%%%%%%%%%%%%%%%%%%%%%%%%%%%%%%%
\DescribeMacro{\childdocname}
\DescribeMacro{\childdocjob}
The macro |\childdocname| contains the filename (without extension)
of the main or child file being processed.
Note that |\childdocjob| will always contain the name of the main file.

%%%%%%%%%%%%%%%%%%%%%%%%%%%%%%%%%%%%%%%%
\paragraph{Title Page.}

Conditional processing can be used to include a title or banner page
in the main document when proper precautions are taken.
Importantly, the code in the main file should ensure that the page counter
(as well as other status parameters which are stored in the |.aux| files)
takes the same value after the conditional processing.
Otherwise the page numbers may take divergent values
depending on which part is compiled.

For example, a title page could be declared by:
%
\begin{center}
\begin{tabular}{l}
|\ifchilddoc\||else|\\
|\addtocounter{page}{-1}|\\
\textit{code for title page}\\
|\newpage|\\
|\||fi|
\end{tabular}
\end{center}
%
A banner page for the child documents can be generated by:
%
\begin{center}
\begin{tabular}{l}
|\ifchilddoc|\\
|\addtocounter{page}{-1}|\\
\textit{code for banner page}\\
|\newpage|\\
|\||fi|
\end{tabular}
\end{center}
%
Here one could write a message such as:
\begin{center}
|This is the part \childdocname{} of \childdocjob{}.|
\end{center}

%%%%%%%%%%%%%%%%%%%%%%%%%%%%%%%%%%%%%%%%%%%%%%%%%%%%%%%%%%%%%%%%%%%%%%%%%%%%%%%%
\subsection{Flags}
\label{sec:flags}

The package makes it easy to generate different versions
of the main or child documents.
To this end compilation flags can be defined
and assigned different default values.
They will be particularly useful in conjunction
with the forwarding mechanism described in \secref{sec:forward}.

For example, it may be useful to have a flag |\version|
which can be set to |draft| or |final|.
The document source will contain some conditional code
depending on the value of |\version|.
Suppose further, the flag should default to |final| for the main file
and to |draft| for child files
which is a natural assignment for editing the document.
This is achieved by placing the following code
in the preamble of the main document
(below the |\childdocmain| directive):
%
\begin{center}
\begin{tabular}{l}
|\ifchilddoc|\\
|\providecommand{\version}{draft}|\\
|\||else|\\
|\providecommand{\version}{final}|\\
|\||fi|
\end{tabular}
\end{center}
%
The definition by |\providecommand| makes sure
that previous definitions are not overwritten.
Further statements |\providecommand{\version}{...}|
can thus be added before the above code to override it.

For the main file, one might add a line
(between |\childdocmain| and the above block)
%
\begin{center}
|%\ifchilddoc\||else\providecommand{\version}{draft}\||fi|
\end{center}
%
which can be uncommented to produce a draft version.
Likewise one can add a line to the very top of a child file
(above the |\childdocof{|\textit{main}|}| directive)
%
\begin{center}
|%\providecommand{\version}{final}|
\end{center}
%
which can be uncommented to produce the final version of this child document.

%%%%%%%%%%%%%%%%%%%%%%%%%%%%%%%%%%%%%%%%%%%%%%%%%%%%%%%%%%%%%%%%%%%%%%%%%%%%%%%%
\subsection{Forwarding}
\label{sec:forward}

Different versions of the main or child documents
using compilation flags as described in \secref{sec:flags}
can be (permanently) stored in different files
for convenient compilation, viewing and distribution.
To this end, the package defines a command
to pass on compilation to a different file:

%%%%%%%%%%%%%%%%%%%%%%%%%%%%%%%%%%%%%%%%
\DescribeMacro{\childdocforward}
The command |\childdocforward| redirects processing to
another source file:
%
\begin{center}
\begin{tabular}{l}
|\input{childdoc.def}|\\
|\childdocforward[|\textit{main}|]{|\textit{dest}|}|\\
\end{tabular}
\end{center}
%
The argument \textit{dest} is the destination file
(without extension).
It should be the main file or one of the child files.
Note that further \textsf{childdoc} directives
such as |\childdocof| and |\childdocforward|
in the indicated file will be processed in this form.
The optional argument \textit{main}
passes on directly to the main file \textit{main}
while pretending to compile the child \textit{dest}.
This form behaves as if \textit{dest}
issues |\childdocof{|\textit{main}|}| right away,
and no further \textsf{childdoc} directives will be processed.

%%%%%%%%%%%%%%%%%%%%%%%%%%%%%%%%%%%%%%%%
\DescribeMacro{\...prefix}
In the alternative form |\childdocforwardprefix|,
%
\begin{center}
\begin{tabular}{l}
|\input{childdoc.def}|\\
|\childdocforwardprefix[|\textit{main}|]{|\textit{prefix}|}{|\textit{dest}|}|
\end{tabular}
\end{center}
%
the destination file is determined by a pattern
depending on the current file:
To make this work, the current file must be called
`{\textit{prefix}\hspace{0.2em}\textit{suffix}}'
with \textit{prefix} matching precisely the argument.
Processing is then passed on to the file
`{\textit{dest}\hspace{0.2em}\textit{suffix}}'.
Surely, the same effect is achieved by
directly specifying the
argument `{\textit{dest}\hspace{0.2em}\textit{suffix}}'
in the first form.
However, that requires to set up a different file
for each child. With the alternative form of the command
all these files can have exactly the same content
which simplifies setting them up and maintaining them.

For example, the following file |draft.tex|
with a compilation flag |\version| as described in \secref{sec:flags}
compiles the main document as a draft:
%
\begin{center}
\begin{tabular}{l}
|\def\version{draft}|\\
|\input{childdoc.def}|\\
|\childdocforward{|\textit{main}|}|
\end{tabular}
\end{center}
%
Likewise, the following files |final|\textit{nn}|.tex|
compile the final version of the child document
|child|\textit{nn}|.tex|:
%
\begin{center}
\begin{tabular}{l}
|\def\version{final}|\\
|\input{childdoc.def}|\\
|\childdocforwardprefix{final}{child}|
\end{tabular}
\end{center}
%

Note that when several versions of a main file and/or of each child file
are to be generated, it may be convenient to set up a |Makefile| or
shell script to automatise the process.

%%%%%%%%%%%%%%%%%%%%%%%%%%%%%%%%%%%%%%%%%%%%%%%%%%%%%%%%%%%%%%%%%%%%%%%%%%%%%%%%
\subsection{Command Line Processing}
\label{sec:commandline}

The effect of redirection files can also be achieved by invoking
the \LaTeX{} compiler with a more elaborate command line.
Most conveniently this should be done as part
of a shell script or a |Makefile|.

When using \textsf{childdoc} in the main file, the following
command lines effectively perform a redirection
(note that depending on the shell being used,
backslashes may have to be doubled: `|\|' $\to$ `|\\|'):
%
\begin{center}
|... -jobname "|\textit{target}|" |\\|"|[\textit{flags}]%
|\input{childdoc.def}\childdocforward[|\textit{main}|]{|\textit{dest}|}"|
\end{center}
%
Here \textit{target} is the name of the output file,
\textit{main} is the name of the main file
and \textit{dest} is the name of the main or child file to be processed
(all filenames without extensions).
The optional argument \textit{main} can be omitted
if \textit{main} matches \textit{dest}.
Optionally, compilation \textit{flags} can be defined via |\def| commands.
This command line makes the \TeX{} engine believe
it is compiling the file \textit{target}
whose content is specified as the latter parameter.
The provided code then forwards the processing to
\textit{main} or \textit{dest} as described in \secref{sec:forward}.

%%%%%%%%%%%%%%%%%%%%%%%%%%%%%%%%%%%%%%%%%%%%%%%%%%%%%%%%%%%%%%%%%%%%%%%%%%%%%%%%
\subsection{Include by Input}
\label{sec:input}

Including child documents by |\include| has some restrictions by design.
Most notably, the content of a child document always occupies
its own set of pages; pages cannot be shared between child documents.
Usually, this behaviour makes perfect sense
because each child document contain an essential part of the document.
However, in some situations it may be desirable to compose
a document from a collection of parts
without having mandatory page breaks between then.
For this case, the package
provides a mechanism to include parts
by |\input| which can also be processed individually.
However, by construction this mechanism
requires manual handling of the content to be output.

%%%%%%%%%%%%%%%%%%%%%%%%%%%%%%%%%%%%%%%%
\DescribeMacro{\ifchilddocmanual}
The main file should be prepared as usual, see \secref{sec:include}.
However, the document body must make a distinction
between processing of an individual part and of the main document, e.g.:
%
\begin{center}
\begin{tabular}{l}
|\ifchilddocmanual|\\
|\input{\childdocname}|\\
|\||else|\\
\textit{document body with }|\input{|\textit{part}|}|\\
|\||fi|
\end{tabular}
\end{center}
%
The conditional |\ifchilddocmanual| is true whenever
a part to be included by |\input| is being compiled,
and the name of the part is stored in |\childdocname|.

%%%%%%%%%%%%%%%%%%%%%%%%%%%%%%%%%%%%%%%%
\DescribeMacro{\childdocby}
Each part to be included by |\input| should start with:
%
\begin{center}
\begin{tabular}{l}
|\input{childdoc.def}|\\
|\childdocby{|\textit{main}|}|\\
\end{tabular}
\end{center}
%
The directive |\childdocby| is similar to |\childdocof|
described in \secref{sec:include},
but the subsequent selection of content must be done manually.
To that end, both |\ifchilddoc| and |\ifchilddocmanual|
will be true upon processing of a part,
and the name of the part is stored in |\childdocname|.
Note that |\jobname| will be set to the filename of the current part
so that each part receives an individual |.aux| file
that does not interfere with the |.aux| file(s) of the main document.
This behaviour can be altered by the alternative form
|\childdocby[*]{|\textit{main}|}| (with a non-empty optional argument)
which uses the |.aux| file of the main document
by setting |\jobname| to \textit{main}.

%%%%%%%%%%%%%%%%%%%%%%%%%%%%%%%%%%%%%%%%%%%%%%%%%%%%%%%%%%%%%%%%%%%%%%%%%%%%%%%%
\subsection{Driver Development}
\label{sec:driver}

The \textsf{childdoc} mechanism can also be use for the development
of definition files such as \LaTeX{} styles or classes.
This case differs from the above setup with multiple parts
included by |\include| in that no |\includeonly| should be invoked.
This can be achieved by starting the include file
(before |\ProvidesPackage|) with:
%
\begin{center}
\begin{tabular}{l}
|\input{childdoc.def}|\\
|\childdocforward{|\textit{main}|}|\\
\end{tabular}
\end{center}
%
or alternatively with:
%
\begin{center}
\begin{tabular}{l}
|\input{childdoc.def}|\\
|\childdocby{|\textit{main}|}|\\
\end{tabular}
\end{center}
%
Both forms have slightly different effects as described above.
The main file is prepared as usual, see \secref{sec:include}.

%%%%%%%%%%%%%%%%%%%%%%%%%%%%%%%%%%%%%%%%%%%%%%%%%%%%%%%%%%%%%%%%%%%%%%%%%%%%%%%%
\subsection{Legacy Detection}
\label{sec:detection}

The directive |\childdocmain| in the main file can detect
whether the complete document or merely a child is to be compiled
even without using the directive |\childdocof|.
This method is deprecated because it is less robust
and there is no compelling reason to use it;
it is merely provided for backward compatibility
and it may be removed in future versions.

If the detection mechanism is to be used,
it is mandatory to correctly specify
the filename of the main file as the argument of |\childdocmain|:
%
\begin{center}
\begin{tabular}{l}
|\input{childdoc.def}|\\
|\childdocmain{|\textit{main}|}|\\
\end{tabular}
\end{center}
%
If |\jobname| does not match the argument \textit{main} of |\childdocmain|,
it is assumed that |\jobname| points to the child file to be compiled.
When using |\childdocmain| with the main file specified as argument,
it suffices to start a child file
with just |\input{|\textit{main}|}|
without loading of the package and using |\childdocof|.
If instead all processing is done
with the appropriate \textsf{childdoc} directives,
the argument of \textit{main} of |\childdocmain| can be empty.

An alternative version of the command line processing described
in \secref{sec:commandline} using the detection mechanism reads:
%
\begin{center}
|... -jobname "|\textit{target}|" "|[\textit{flags}]%
[|\def\jobname{|\textit{dest}|}|]|\input{|\textit{main}|}"|
\end{center}

%%%%%%%%%%%%%%%%%%%%%%%%%%%%%%%%%%%%%%%%%%%%%%%%%%%%%%%%%%%%%%%%%%%%%%%%%%%%%%%%
\subsection{Manual Code}
\label{sec:manual}

In case one cannot be certain whether the definitions file |childdoc.def|
is installed on the target \TeX{} distribution
and one prefers not to ship it,
it is conceivable to paste a few relevant commands into the sources.

To that end, drop all statements |\input{childdoc.def}|
and perform the replacements as outlined below.
Instead of |\childdocmain{|\textit{main}|}| add the following code
to the top of the main file:
%
\begin{center}
\begin{tabular}{l}
|\||ifdefined\childdocname\endinput\||fi\newif\ifchilddoc|\\
|\edef\childdocname{\scantokens\expandafter{\jobname\noexpand}}|\\
|\def\childdocmain{|\textit{main}|}\||ifx\childdocmain\childdocname\||else|\\
|\childdoctrue\includeonly{\childdocname}\let\jobname\childdocmain\||fi|\\
\end{tabular}
\end{center}
%
Instead of |\childdocof{|\textit{main}|}| just include the main file
at the top of each child file:
%
\begin{center}
|\input{|\textit{main}|}|
\end{center}
%
A simple redirection |\childdocforward{|\textit{dest}|}| is achieved by:
%
\begin{center}
|\def\jobname{|\textit{dest}|}\input{\jobname}|
\end{center}
%
The redirection with prefix
|\childdocforwardprefix[|\textit{prefix}|]{|\textit{dest}|}|
is accomplished by:
%
\begin{center}
\begin{tabular}{l}
|{\edef\jobname{\scantokens\expandafter{\jobname\noexpand}}|\\
|\def\redirectjob |\textit{prefix}|#1~~~{\gdef\jobname{|\textit{dest}|#1}}|\\
|\expandafter\redirectjob\jobname~~~}\input{\jobname}|
\end{tabular}
\end{center}

In an alternative approach,
child documents can be compiled by a specific command line
without additional code or specific definitions:
%
\begin{center}
|... -jobname "|\textit{target}|" "|[\textit{flags}]%
|\includeonly{|\textit{dest}|}\input{|\textit{main}|}"|
\end{center}
%

%%%%%%%%%%%%%%%%%%%%%%%%%%%%%%%%%%%%%%%%%%%%%%%%%%%%%%%%%%%%%%%%%%%%%%%%%%%%%%%%
%%%%%%%%%%%%%%%%%%%%%%%%%%%%%%%%%%%%%%%%%%%%%%%%%%%%%%%%%%%%%%%%%%%%%%%%%%%%%%%%
\section{Information}

%%%%%%%%%%%%%%%%%%%%%%%%%%%%%%%%%%%%%%%%%%%%%%%%%%%%%%%%%%%%%%%%%%%%%%%%%%%%%%%%
\subsection{Copyright}

Copyright \copyright{} 2017--2018 Niklas Beisert

This work may be distributed and/or modified under the
conditions of the \LaTeX{} Project Public License, either version 1.3
of this license or (at your option) any later version.
The latest version of this license is in
  \url{http://www.latex-project.org/lppl.txt}
and version 1.3 or later is part of all distributions of \LaTeX{}
version 2005/12/01 or later.

This work has the LPPL maintenance status `maintained'.

The Current Maintainer of this work is Niklas Beisert.

This work consists of the files |README.txt|, |childdoc.ins| and |childdoc.dtx|
as well as the derived files |childdoc.def|, |cdocsamp.tex|
with |cdocsch1.tex|, |cdocsch2.tex|, |cdocspt3.tex|, |cdocspt4.tex|,
|cdocsdrf.tex|, |cdocsfn1.tex|, |cdocsfn2.tex|
as well as |childdoc.pdf|.

%%%%%%%%%%%%%%%%%%%%%%%%%%%%%%%%%%%%%%%%%%%%%%%%%%%%%%%%%%%%%%%%%%%%%%%%%%%%%%%%
\subsection{Files and Installation}

The package consists of the files:
%
\begin{center}
\begin{tabular}{ll}
    |README.txt|   & readme file \\
    |childdoc.ins| & installation file \\
    |childdoc.dtx| & source file \\
    |childdoc.def| & definition file \\
    |cdocsamp.tex| & sample main file \\
    |cdocsch1.tex| & sample include file \\
    |cdocsch2.tex| & sample include file \\
    |cdocspt3.tex| & sample part file \\
    |cdocspt4.tex| & sample part file \\
    |cdocsdrf.tex| & sample redirection file \\
    |cdocsfn1.tex| & sample redirection file \\
    |cdocsfn2.tex| & sample redirection file \\
    |childdoc.pdf| & manual
\end{tabular}
\end{center}
%
The distribution consists of the files
|README.txt|, |childdoc.ins| and |childdoc.dtx|.
%
\begin{itemize}
\item
Run (pdf)\LaTeX{} on |childdoc.dtx|
to compile the manual |childdoc.pdf| (this file).
\item
Run \LaTeX{} on |childdoc.ins| to create the definitions file |childdoc.def|
and the sample |cdocsamp.tex| with include files
|cdocsch1.tex|, |cdocsch2.tex|, |cdocspt3.tex|, |cdocspt4.tex|,
|cdocsdrf.tex|, |cdocsfn1.tex|, |cdocsfn2.tex|.
Then copy the file |childdoc.def| to an appropriate directory of your \LaTeX{}
distribution, e.g.\ \textit{texmf-root}|/tex/latex/childdoc|.
\end{itemize}

%%%%%%%%%%%%%%%%%%%%%%%%%%%%%%%%%%%%%%%%%%%%%%%%%%%%%%%%%%%%%%%%%%%%%%%%%%%%%%%%
\subsection{Related CTAN Packages}

There are several other packages which offer a similar functionality:
%
\begin{itemize}
\item
The packages
\href{http://ctan.org/pkg/docmute}{\textsf{docmute}},
\href{http://ctan.org/pkg/includex}{\textsf{includex}} and
\href{http://ctan.org/pkg/standalone}{\textsf{standalone}}
provide commands to include only the document body of
a child file thus allowing both files to be compiled individually.
\item
The packages \href{http://ctan.org/pkg/subdocs}{\textsf{subdocs}}
and \href{http://ctan.org/pkg/subfiles}{\textsf{subfiles}}
provide structures in which the main and child documents can be
encapsulated and allowing them to be compiled individually.
The inclusion mechanism is different from the conventional |\include|.
\item
The package \href{http://ctan.org/pkg/combine}{\textsf{combine}}
is an elaborate solution to combine several documents into one.
\end{itemize}
%
See also the CTAN topic \href{http://ctan.org/topic/subdocs}{\textsf{subdocs}}
for further related packages.
The present package differs from the above solutions in that
a document structure constructed with the conventional |\include| mechanism
just needs two extra commands at the top of every file
such that all constituent files can be compiled individually.

%%%%%%%%%%%%%%%%%%%%%%%%%%%%%%%%%%%%%%%%%%%%%%%%%%%%%%%%%%%%%%%%%%%%%%%%%%%%%%%%
%\subsection{Feature Suggestions}
%
%The following is a list of features which may be useful for future
%versions of this package:
%%
%\begin{itemize}
%\item
%\ldots
%\end{itemize}

%%%%%%%%%%%%%%%%%%%%%%%%%%%%%%%%%%%%%%%%%%%%%%%%%%%%%%%%%%%%%%%%%%%%%%%%%%%%%%%%
\subsection{Revision History}

%%%%%%%%%%%%%%%%%%%%%%%%%%%%%%%%%%%%%%%%
\paragraph{v2.0:} 2018/12/30

\begin{itemize}
\item
immediate forward processing
\item
added |\childdocby| mechanism
\item
manual restructured
\end{itemize}

%%%%%%%%%%%%%%%%%%%%%%%%%%%%%%%%%%%%%%%%
\paragraph{v1.6:} 2018/01/17

\begin{itemize}
\item
application for development of include files
\item
corrections to manual
\end{itemize}

%%%%%%%%%%%%%%%%%%%%%%%%%%%%%%%%%%%%%%%%
\paragraph{v1.5:} 2017/05/21

\begin{itemize}
\item
more complete structuring introduced
\item
|\childdocof| introduced
\item
|\childdoc| renamed to |\childdocmain|
\item
|\childredirect| renamed to |\childdocforward| and |\childdocforwardprefix|
and functionality expanded
\end{itemize}

%%%%%%%%%%%%%%%%%%%%%%%%%%%%%%%%%%%%%%%%
\paragraph{v1.0:} 2017/04/27

\begin{itemize}
\item
manual and install package
\item
first version published on CTAN
\end{itemize}

%%%%%%%%%%%%%%%%%%%%%%%%%%%%%%%%%%%%%%%%
\paragraph{v0.6:} 2017/04/26

\begin{itemize}
\item
redirection mechanism added
\end{itemize}

%%%%%%%%%%%%%%%%%%%%%%%%%%%%%%%%%%%%%%%%
\paragraph{v0.5:} 2017/04/26

\begin{itemize}
\item
functionality in definition file
\end{itemize}


%%%%%%%%%%%%%%%%%%%%%%%%%%%%%%%%%%%%%%%%%%%%%%%%%%%%%%%%%%%%%%%%%%%%%%%%%%%%%%%%
%%%%%%%%%%%%%%%%%%%%%%%%%%%%%%%%%%%%%%%%%%%%%%%%%%%%%%%%%%%%%%%%%%%%%%%%%%%%%%%%
%%%%%%%%%%%%%%%%%%%%%%%%%%%%%%%%%%%%%%%%%%%%%%%%%%%%%%%%%%%%%%%%%%%%%%%%%%%%%%%%
\appendix

\settowidth\MacroIndent{\rmfamily\scriptsize 000\ }

 \DocInput{childdoc.dtx}

\end{document}
%</driver>
% \fi
%
% %%%%%%%%%%%%%%%%%%%%%%%%%%%%%%%%%%%%%%%%%%%%%%%%%%%%%%%%%%%%%%%%%%%%%%%%%%%%%%
% %%%%%%%%%%%%%%%%%%%%%%%%%%%%%%%%%%%%%%%%%%%%%%%%%%%%%%%%%%%%%%%%%%%%%%%%%%%%%%
% \section{Sample}
%\iffalse
%<*samplemain>
%\fi
%
% The following presents a sample document
% with two chapters, two parts, a title page,
% a compile flag as well as three forwarding files to set the flag.
% It consists of eight |.tex| files:
% \begin{center}
% \begin{tabular}{ll}
% |cdocsamp.tex|&main file\\
% |cdocsch1.tex|&include file for chapter 1\\
% |cdocsch2.tex|&include file for chapter 2\\
% |cdocspt3.tex|&include file for part 3\\
% |cdocspt4.tex|&include file for part 4\\
% |cdocsdrf.tex|&forwarding file for main file in draft mode\\
% |cdocsfi1.tex|&forwarding file for final version of chapter 1\\
% |cdocsfi2.tex|&forwarding file for final version of chapter 2\\
% \end{tabular}
% \end{center}
% Each of the eight files can be compiled directly by the \LaTeX{} compiler.
%
% %%%%%%%%%%%%%%%%%%%%%%%%%%%%%%%%%%%%%%
% \paragraph{Main File.}
%
% The main file is called |cdocsamp.tex|.
%
% Load the \textsf{childdoc} definitions and
% declare the filename for the main document:
%    \begin{macrocode}
\input{childdoc.def}
\childdocmain{}
%    \end{macrocode}

% Optional override for |\version| flag:
%    \begin{macrocode}
%%\ifchilddoc\else\providecommand{\version}{draft}\fi
%    \end{macrocode}

% Define the default values for the |\version| flag
% (|final| for the main file and |draft| for childs):
%    \begin{macrocode}
\ifchilddoc
\providecommand{\version}{draft}
\else
\providecommand{\version}{final}
\fi
%    \end{macrocode}

% Load the standard document class:
%    \begin{macrocode}
\documentclass[12pt]{article}
%    \end{macrocode}

% Start the document body:
%    \begin{macrocode}
\begin{document}
%    \end{macrocode}

% Declare a title page.
% Print title, part of document being processed and version flag:
%    \begin{macrocode}
\addtocounter{page}{-1}
\begin{center}
{\LARGE\bfseries{}childdoc example\par}
\vspace{1cm}
\ifchilddoc
\ifchilddocmanual part\else chapter\fi:
`\childdocname' of `\childdocjob'\par
\else
main document: `\childdocjob'\par
\fi
version: \version\par
\end{center}
\newpage
%    \end{macrocode}

% Manually include selected file,
% otherwise process as usual:
%    \begin{macrocode}
\ifchilddocmanual
\section*{part `\childdocname'}
\input{\childdocname}
\else
%    \end{macrocode}

% Include the two chapters:
%    \begin{macrocode}
\include{cdocsch1}
\include{cdocsch2}
%    \end{macrocode}

% Include the two parts unless only chapters should be displayed:
%    \begin{macrocode}
\ifchilddoc\else
\section{part three}
\input{cdocspt3}
\section{part four}
\input{cdocspt4}
\fi
%    \end{macrocode}

% Process as usual until here:
%    \begin{macrocode}
\fi
%    \end{macrocode}

% End of document body:
%    \begin{macrocode}
\end{document}
%    \end{macrocode}
%\iffalse
%</samplemain>
%\fi
%
% %%%%%%%%%%%%%%%%%%%%%%%%%%%%%%%%%%%%%%
% \paragraph{Chapter Include Files.}
%
% The include files are called |cdocsch1.tex| and |cdocsch2.tex|.
%
%\iffalse
%<*samplechap1|samplechap2>
%\fi

% Optional override for |\version| flag:
%    \begin{macrocode}
%%\providecommand{\version}{final}
%    \end{macrocode}

% Include the main document:
%    \begin{macrocode}
\input{childdoc.def}
\childdocof{cdocsamp}
%    \end{macrocode}

%\iffalse
%</samplechap1|samplechap2>
%\fi
%
%\iffalse
%<*samplechap1>
%\fi
% Some text for chapter 1:
%    \begin{macrocode}
\section{one}
some text in chapter one
%    \end{macrocode}

%\iffalse
%</samplechap1>
%\fi
% Some text for chapter 2:
%\iffalse
%<*samplechap2>
%\fi
%    \begin{macrocode}
\section{two}
more text in chapter two
%    \end{macrocode}

%\iffalse
%</samplechap2>
%\fi
%
% %%%%%%%%%%%%%%%%%%%%%%%%%%%%%%%%%%%%%%
% \paragraph{Part Include Files.}
%
% The include files are called |cdocspt3.tex| and |cdocspt4.tex|.
%
%\iffalse
%<*samplepart3|samplepart4>
%\fi

% Optional override for |\version| flag:
%    \begin{macrocode}
%%\providecommand{\version}{final}
%    \end{macrocode}

% Include the main document:
%    \begin{macrocode}
\input{childdoc.def}
\childdocby{cdocsamp}
%    \end{macrocode}

%\iffalse
%</samplepart3|samplepart4>
%\fi
%
%\iffalse
%<*samplepart3>
%\fi
% Some text for part 3:
%    \begin{macrocode}
some text in part three
%    \end{macrocode}

%\iffalse
%</samplepart3>
%\fi
% Some text for part 4:
%\iffalse
%<*samplepart4>
%\fi
%    \begin{macrocode}
more text in part four
%    \end{macrocode}

%\iffalse
%</samplepart4>
%\fi
%
% %%%%%%%%%%%%%%%%%%%%%%%%%%%%%%%%%%%%%%
% \paragraph{Forwarding for a Complete Draft.}
%
% The following forwarding file |cdocsdrf.tex|
% compiles the main document in draft mode:
%\iffalse
%<*sampledraft>
%\fi
%    \begin{macrocode}
\def\version{draft}
\input{childdoc.def}
\childdocforward{cdocsamp}
%    \end{macrocode}

%\iffalse
%</sampledraft>
%\fi
%
% %%%%%%%%%%%%%%%%%%%%%%%%%%%%%%%%%%%%%%
% \paragraph{Forwarding for Final Version of the Chapters.}
%
% The following forwarding files |cdocsfn1.tex| and |cdocsfn2.tex|
% (with identical content)
% compile the final versions of the child documents
% |cdocsch1.tex| and |cdocsch2.tex|, respectively:
%\iffalse
%<*samplefinal>
%\fi
%    \begin{macrocode}
\def\version{final}
\input{childdoc.def}
\childdocforwardprefix[cdocsamp]{cdocsfn}{cdocsch}
%    \end{macrocode}

%\iffalse
%</samplefinal>
%\fi
%
% %%%%%%%%%%%%%%%%%%%%%%%%%%%%%%%%%%%%%%
% \paragraph{Command Line Processing.}
%
% The following three command lines generate the output files
% |cdocscld|, |cdocscl1| and |cdocscl2|
% which should be identical to
% |cdocsdrf|, |cdocsch1| and |cdocsfn2|, respectively:
% \begin{center}
% \begin{tabular}{l}
% |latex -jobname cdocscld \|\\
% |  "\def\version{draft}\input{childdoc.def}\childdocforward{cdocsamp}"|\\
% |latex -jobname cdocscl1 \|\\
% |  "\input{childdoc.def}\childdocforward[cdocsamp]{cdocsch1}"|\\
% |latex -jobname cdocscl2 \|\\
% |  "\def\version{final}\input{childdoc.def}\childdocforward{cdocsch2}"|
% \end{tabular}
% \end{center}
% Note that the trailing backslash on each first line
% merely continues the input to the second line
% (for convenient cut ant paste).
% Furthermore, the command |latex| can be replaced by any
% of its alternative versions such as |pdflatex|.
%
% %%%%%%%%%%%%%%%%%%%%%%%%%%%%%%%%%%%%%%%%%%%%%%%%%%%%%%%%%%%%%%%%%%%%%%%%%%%%%%
% %%%%%%%%%%%%%%%%%%%%%%%%%%%%%%%%%%%%%%%%%%%%%%%%%%%%%%%%%%%%%%%%%%%%%%%%%%%%%%
% \section{Implementation}
%\iffalse
%<*package>
%\fi
%
% This section describes the definitions file |childdoc.def|.

% The definitions cannot be loaded using |\usepackage| or |\RequirePackage|
% which has a mechanism to prevent loading a style file more than once.
% When loading the definitions by means of |\input|
% multiple instances have to be prevented manually:
%\iffalse
%This code needs to be before the `\ProvidesFile' directive
%which is defined at the beginning of this file.
%Therefore it is also placed there and commented out here.
%</package>
%<*discard>
%\fi
%    \begin{macrocode}
\ifdefined\childdocmain\endinput\fi
%    \end{macrocode}
%\iffalse
%</discard>
%<*package>
%\fi
%
% \macro{\ifchilddoc}
% \macro{\ifchilddocmanual}
% The conditional |\ifchilddoc| tells whether a
% child (true) or main (false) document is being compiled.
% The conditional |\ifchilddocmanual| tells whether
% the |\includeonly| mechanism is used (false) or
% the selection of child files must be performed manually (true).
% The definitions initialise to false:
%    \begin{macrocode}
\newif\ifchilddoc
\newif\ifchilddocmanual
%    \end{macrocode}

% \macro{\childdocname}
% \macro{\childdocjob}
% The macro |\childdocname| stores the name of the main document
% to be compiled. The macro |\childdocjob| stores the name of
% the document on which the \LaTeX{} compiler was originally invoked.
% The content of |\jobname| cannot be compared
% to filenames specified in the source due to different catcodes.
% The following code rescans |\jobname|, stores the result
% in |\childdocname| and saves a copy in |\childdocjob|:
%    \begin{macrocode}
\edef\childdocname{\scantokens\expandafter{\jobname\noexpand}}
\let\childdocjob\childdocname
%    \end{macrocode}

% \macro{\childdocdisable}
% The macro |\childdocdisable| prevents the main file
% from being processed more than once.
% At this stage, the main document command |\childdocmain|
% is assumed to be called once again where it should do nothing.
% Any subsequent call to it should prevent
% a secondary processing of the main document
% It overwrites the forwarding commands
% |\childdocof| and |\childdocforward|
% with empty macros to prevent further inclusions of the main document:
%    \begin{macrocode}
\newcommand{\childdocdisable}
{
  \renewcommand{\childdocmain}[1]{\renewcommand{\childdocmain}[1]{\endinput}}
  \renewcommand{\childdocof}[1]{}
  \renewcommand{\childdocby}[2][]{}
  \renewcommand{\childdocforward}[2][]{}
  \renewcommand{\childdocdisable}{}
}
%    \end{macrocode}

% \macro{\childdocmain}
% The macro |\childdocmain| is to be called at the top of the main file
% with nothing or the main filename (without extension) as argument.
% First, it breaks loops.
% If the argument is not empty and does not match |\childdocname|
% (which is set by the first inclusion of |childdoc.def|),
% |\ifchilddoc| is set to true, |\includeonly| is applied to the child file
% and |\jobname| is set to the main file
% (for proper handling of |.aux| files):
%    \begin{macrocode}
\newcommand{\childdocmain}[1]
{
  \childdocdisable\childdocmain{}
  \if?#1?\else
    \begingroup
      \def\childdoctmp{#1}
      \ifx\childdoctmp\childdocname
        \def\childdoctmp{}
      \else
        \def\childdoctmp
        {
          \childdoctrue
          \includeonly{\childdocname}
          \def\childdocjob{#1}
          \def\jobname{#1}
        }
      \fi
      \expandafter
    \endgroup
    \childdoctmp
  \fi
}
%    \end{macrocode}

% \macro{\childdocof}
% The command |\childdocof| redirects
% compilation to the main file |#1|.
%    \begin{macrocode}
\newcommand{\childdocof}[1]
{
  \childdocdisable
  \childdoctrue
  \includeonly{\childdocname}
  \def\jobname{#1}
  \def\childdocjob{#1}
  \input{#1}
}
%    \end{macrocode}

% \macro{\childdocby}
% The command |\childdocby| ....
%    \begin{macrocode}
\newcommand{\childdocby}[2][]
{
  \childdocdisable
  \childdoctrue
  \childdocmanualtrue
  \if?#1?\else
    \def\jobname{#2}
  \fi
  \def\childdocjob{#2}
  \input{#2}
  \endinput
}
%    \end{macrocode}

% \macro{\childdocforward}
% The command |\childdocforward| redirects
% compilation to the main file or
% (if the optional argument is given) a child file.
% Parameters are set as if the main file
% or a child file starting with |\childdocof| was compiled.
% Then compilation is handed over to the main file:
%    \begin{macrocode}
\newcommand{\childdocforward}[2][]
{
  \begingroup
    \if?#1?
      \def\childdoctmp
      {
        \def\childdocname{#2}
        \def\childdocjob{#2}
        \def\jobname{#2}
        \input{#2}
        \endinput
      }
    \else
      \def\childdoctmp
      {
        \childdocdisable
        \def\childdocname{#2}
        \childdoctrue
        \includeonly{#2}
        \def\childdocjob{#1}
        \def\jobname{#1}
        \input{#1}
        \endinput
      }
    \fi
    \expandafter
  \endgroup
  \childdoctmp
}
%    \end{macrocode}

% \macro{\childdocforwardprefix}
% The command |\childdocforwardprefix| redirects
% compilation to the main or a child file by means of a pattern.
% The prefix |#1| in the current filename is replaced by |#2|
% and the suffix of the current filename is kept
% (it is assumed that the filename does not contain the substring `|~~~|'
% which is used as a delimiter).
% Compilation is handed over to the new file by |\childdocforward|:
%    \begin{macrocode}
\newcommand{\childdocforwardprefix}[3][]
{
  \begingroup
    \def\childdocextract #2##1~~~{\def\childdoctmp{\childdocforward[#1]{#3##1}}}
    \expandafter\childdocextract\childdocname~~~
    \expandafter
  \endgroup
  \childdoctmp
}
%    \end{macrocode}

% \macro{\childdoc}
% The deprecated macro |\childdoc| is a legacy version of |\childdocmain|:
%    \begin{macrocode}
\newcommand{\childdoc}{\childdocmain}
%    \end{macrocode}

% \macro{\childdocredirect}
% The deprecated macro |\childdocredirect| is a legacy version
% of |\childdocforward| and |\childdocforwardprefix|:
%    \begin{macrocode}
\newcommand{\childdocredirect}[2][]
{
  \begingroup
    \if?#1?
      \def\childdoctmp{\childdocforward{#2}}
    \else
      \def\childdoctmp{\childdocforwardprefix{#1}{#2}}
    \fi
    \expandafter
  \endgroup
  \childdoctmp
}
%    \end{macrocode}

%\iffalse
%</package>
%\fi
%
\endinput
|\\
|\childdocof{|\textit{main}|}|\\
\end{tabular}
\end{center}
at the top of every child file \textit{child}
which is included by |\include{|\textit{child}|}|
from within the main file
(or at least for those files to be compiled individually).
The argument \textit{main} must be the filename of the main file.

There are a couple of
considerations in setting up the main and child documents:

%%%%%%%%%%%%%%%%%%%%%%%%%%%%%%%%%%%%%%%%
\paragraph{Restrictions.}

Please note the following restrictions:
\begin{itemize}
\item
|\childdocmain| must be called with one argument \textit{main}
to ensure compatibility with earlier version of the package.
It must either be empty (|\childdocmain{}|)
or precisely match the filename of the main file in which it is specified.
See \secref{sec:detection} for further information.
\item
The filename \textit{main} must be specified without the |.tex| extension.
\item
The filename \textit{main} is case sensitive
(even in case-insensitive file systems)
due to internal string comparison.
\item
The argument \textit{main} should be fully expanded, it cannot be a macro.
\item
Subdirectories and special characters should be avoided in filenames.
\item
The command |\childdocmain{|\textit{main}|}| must be followed by a whitespace.
It should not be followed immediately by another command
or by a comment mark `|%|'.
This is because the \TeX{} parser reads the token immediately following
the argument of |\childdocmain| and puts it
at the beginning of every child section;
however, a white\-space is ignored.
\end{itemize}

%%%%%%%%%%%%%%%%%%%%%%%%%%%%%%%%%%%%%%%%
\paragraph{Content of Main File.}

It is advisable to place all content in the child files included by |\include|.
Any output contained in the main file will appear in all child documents
unless suppressed manually;
it cannot be suppressed automatically by the |\includeonly| directive
and thus should normally be avoided.
A method to include some content in the main file
by means of conditional processing is described in \secref{sec:conditional}.

%%%%%%%%%%%%%%%%%%%%%%%%%%%%%%%%%%%%%%%%
\paragraph{Page Numbering.}

When only a part of the document is compiled,
the appropriate numbering of pages
(as well as other status parameters)
is determined from the |.aux| files.
The latter contain information from previous passes.
However this information needs to propagate through
all intermediate child documents.
Therefore the page numbering in child documents may well
be inconsistent until the complete document is compiled at least once.

A useful (if unconventional) way to always ensure a consistent
page numbering is to restart the numbering in each child document
and denote the pages by `\textit{child}|.|\textit{page}'
where \textit{child} represents the chapter/section number of the child file.
This can be achieved by the command
|\numberwithin{page}{|\textit{child}|}|
of the \textsf{amsmath} package
where \textit{child} can be |chapter| or |section|
depending on the chosen structuring.
Alternatively, one can modify the macro |\thepage| appropriately
and reset the counter |page| at the start of each child file.

%%%%%%%%%%%%%%%%%%%%%%%%%%%%%%%%%%%%%%%%%%%%%%%%%%%%%%%%%%%%%%%%%%%%%%%%%%%%%%%%
\subsection{Conditional Processing}
\label{sec:conditional}

The package provides a mechanism to compile different versions
of a document. To customise the versions further some conditional processing
can come in handy to distinguish which version is being compiled.
The package provides two macros to describe the compilation context:

%%%%%%%%%%%%%%%%%%%%%%%%%%%%%%%%%%%%%%%%
\DescribeMacro{\ifchilddoc}
The conditional |\ifchilddoc| distinguishes between the compilation of
child documents and the main document:
%
\begin{center}
|\ifchilddoc |\textit{child-code}| |[|\||else |\textit{main-code}]| \||fi|
\end{center}

%%%%%%%%%%%%%%%%%%%%%%%%%%%%%%%%%%%%%%%%
\DescribeMacro{\childdocname}
\DescribeMacro{\childdocjob}
The macro |\childdocname| contains the filename (without extension)
of the main or child file being processed.
Note that |\childdocjob| will always contain the name of the main file.

%%%%%%%%%%%%%%%%%%%%%%%%%%%%%%%%%%%%%%%%
\paragraph{Title Page.}

Conditional processing can be used to include a title or banner page
in the main document when proper precautions are taken.
Importantly, the code in the main file should ensure that the page counter
(as well as other status parameters which are stored in the |.aux| files)
takes the same value after the conditional processing.
Otherwise the page numbers may take divergent values
depending on which part is compiled.

For example, a title page could be declared by:
%
\begin{center}
\begin{tabular}{l}
|\ifchilddoc\||else|\\
|\addtocounter{page}{-1}|\\
\textit{code for title page}\\
|\newpage|\\
|\||fi|
\end{tabular}
\end{center}
%
A banner page for the child documents can be generated by:
%
\begin{center}
\begin{tabular}{l}
|\ifchilddoc|\\
|\addtocounter{page}{-1}|\\
\textit{code for banner page}\\
|\newpage|\\
|\||fi|
\end{tabular}
\end{center}
%
Here one could write a message such as:
\begin{center}
|This is the part \childdocname{} of \childdocjob{}.|
\end{center}

%%%%%%%%%%%%%%%%%%%%%%%%%%%%%%%%%%%%%%%%%%%%%%%%%%%%%%%%%%%%%%%%%%%%%%%%%%%%%%%%
\subsection{Flags}
\label{sec:flags}

The package makes it easy to generate different versions
of the main or child documents.
To this end compilation flags can be defined
and assigned different default values.
They will be particularly useful in conjunction
with the forwarding mechanism described in \secref{sec:forward}.

For example, it may be useful to have a flag |\version|
which can be set to |draft| or |final|.
The document source will contain some conditional code
depending on the value of |\version|.
Suppose further, the flag should default to |final| for the main file
and to |draft| for child files
which is a natural assignment for editing the document.
This is achieved by placing the following code
in the preamble of the main document
(below the |\childdocmain| directive):
%
\begin{center}
\begin{tabular}{l}
|\ifchilddoc|\\
|\providecommand{\version}{draft}|\\
|\||else|\\
|\providecommand{\version}{final}|\\
|\||fi|
\end{tabular}
\end{center}
%
The definition by |\providecommand| makes sure
that previous definitions are not overwritten.
Further statements |\providecommand{\version}{...}|
can thus be added before the above code to override it.

For the main file, one might add a line
(between |\childdocmain| and the above block)
%
\begin{center}
|%\ifchilddoc\||else\providecommand{\version}{draft}\||fi|
\end{center}
%
which can be uncommented to produce a draft version.
Likewise one can add a line to the very top of a child file
(above the |\childdocof{|\textit{main}|}| directive)
%
\begin{center}
|%\providecommand{\version}{final}|
\end{center}
%
which can be uncommented to produce the final version of this child document.

%%%%%%%%%%%%%%%%%%%%%%%%%%%%%%%%%%%%%%%%%%%%%%%%%%%%%%%%%%%%%%%%%%%%%%%%%%%%%%%%
\subsection{Forwarding}
\label{sec:forward}

Different versions of the main or child documents
using compilation flags as described in \secref{sec:flags}
can be (permanently) stored in different files
for convenient compilation, viewing and distribution.
To this end, the package defines a command
to pass on compilation to a different file:

%%%%%%%%%%%%%%%%%%%%%%%%%%%%%%%%%%%%%%%%
\DescribeMacro{\childdocforward}
The command |\childdocforward| redirects processing to
another source file:
%
\begin{center}
\begin{tabular}{l}
|% \iffalse
%
% childdoc.dtx Copyright (C) 2017-2018 Niklas Beisert
%
% This work may be distributed and/or modified under the
% conditions of the LaTeX Project Public License, either version 1.3
% of this license or (at your option) any later version.
% The latest version of this license is in
%   http://www.latex-project.org/lppl.txt
% and version 1.3 or later is part of all distributions of LaTeX
% version 2005/12/01 or later.
%
% This work has the LPPL maintenance status `maintained'.
%
% The Current Maintainer of this work is Niklas Beisert.
%
% This work consists of the files childdoc.dtx and childdoc.ins
% and the derived files childdoc.def and cdocsamp.tex with
% cdocsch1.tex, cdocsch2.tex, cdocsdrf.tex, cdocsfn1.tex, cdocsfn2.tex.
%
%<package>\ifdefined\childdocmain\endinput\fi
%<package>\ProvidesFile{childdoc.def}[2018/12/30 v2.0 child document driver]
%<samplemain>\ProvidesFile{cdocsamp.tex}[2018/12/30 v2.0 sample for childdoc]
%<*driver>
%\ProvidesFile{childdoc.drv}[2018/12/30 v2.0 childdoc reference manual file]
\PassOptionsToClass{10pt,a4paper}{article}
\documentclass{ltxdoc}

\usepackage[margin=35mm]{geometry}
\usepackage{hyperref}
\usepackage{hyperxmp}
\usepackage[usenames]{color}

\hypersetup{colorlinks=true}
\hypersetup{pdfstartview=FitH}
\hypersetup{pdfpagemode=UseNone}
\hypersetup{pdfsource={}}
\hypersetup{pdflang={en-UK}}
\hypersetup{pdfcopyright={Copyright 2017-2018 Niklas Beisert.
  This work may be distributed and/or modified under the
  conditions of the LaTeX Project Public License, either version 1.3
  of this license or (at your option) any later version.}}
\hypersetup{pdflicenseurl={http://www.latex-project.org/lppl.txt}}
\hypersetup{pdfcontactaddress={ETH Zurich, ITP, HIT K,
  Wolfgang-Pauli-Strasse 27}}
\hypersetup{pdfcontactpostcode={8093}}
\hypersetup{pdfcontactcity={Zurich}}
\hypersetup{pdfcontactcountry={Switzerland}}
\hypersetup{pdfcontactemail={nbeisert@itp.phys.ethz.ch}}
\hypersetup{pdfcontacturl={http://people.phys.ethz.ch/\xmptilde nbeisert/}}

\newcommand{\secref}[1]{\hyperref[#1]{section \ref*{#1}}}

\parskip1ex
\parindent0pt
\let\olditemize\itemize
\def\itemize{\olditemize\parskip0pt}

\begin{document}

\title{The \textsf{childdoc} Package}
\hypersetup{pdftitle={The childdoc Package}}
\author{Niklas Beisert\\[2ex]
  Institut f\"ur Theoretische Physik\\
  Eidgen\"ossische Technische Hochschule Z\"urich\\
  Wolfgang-Pauli-Strasse 27, 8093 Z\"urich, Switzerland\\[1ex]
  \href{mailto:nbeisert@itp.phys.ethz.ch}
  {\texttt{nbeisert@itp.phys.ethz.ch}}}
\hypersetup{pdfauthor={Niklas Beisert}}
\hypersetup{pdfsubject={Manual for the LaTeX2e Package childdoc}}
\date{30 December 2018, \textsf{v2.0}}
\maketitle

\begin{abstract}\noindent
\textsf{childdoc} is a \LaTeXe{} package
that enables the direct compilation
of document sections included by |\include|
to individual files.
\end{abstract}

\begingroup
\parskip0ex
\tableofcontents
\endgroup

%%%%%%%%%%%%%%%%%%%%%%%%%%%%%%%%%%%%%%%%%%%%%%%%%%%%%%%%%%%%%%%%%%%%%%%%%%%%%%%%
%%%%%%%%%%%%%%%%%%%%%%%%%%%%%%%%%%%%%%%%%%%%%%%%%%%%%%%%%%%%%%%%%%%%%%%%%%%%%%%%
\section{Introduction}

\LaTeX{} provides a mechanism to structure a large document (such as a book)
into a main file and several child files (containing the chapters)
using the |\include| command.
This mechanism is beneficial for documents
which span hundreds of pages in order to
make the source file(s) more manageable.
Moreover, compilation can be restricted to
selected child files by means of the |\includeonly| command.
The latter feature can be used to reduce the compilation time while editing
(this was significantly more useful in the earlier days of \LaTeX{})
or to generate a smaller document which is easier to navigate.
Another application of |\includeonly| is to generate
documents consisting of selected parts of the complete document.

However, there are a few drawbacks of the plain |\include| mechanism:
\begin{itemize}
\item
The child files cannot be compiled on their own,
they can only be compiled via the main file.
A naive editing environment
(such as a text editor with an option
to have the current file processed by \LaTeX)
may require one to switch to the main file before compiling;
attempting to compile the child file produces errors.
\item
The main file must be modified (each time)
to adjust the |\includeonly| command
to the present needs. This easily leaves the main file in a messy state.
\item
The generated document will always carry the filename
of the main document. This is inconvenient if
several child files are to be compiled and
to be kept for distribution.
\end{itemize}

The present package provides a simple interface
to make child files individually compilable by \LaTeX{}.
Compiling a child file then has the same effect as compiling
the main file with an |\includeonly| command
to select the appropriate child.
Moreover the generated document will carry the name of the child
rather than the main file.
This resolves all three above issues.

This feature is meant to make the editing of books,
thesis documents and lecture notes somewhat more convenient.
However, the package can also be used efficiently for
composing a series of documents (such as exercise sheets)
which are typically distributed individually.
It then assists the author in generating the individual documents
(potentially in different versions)
as well as a document containing the collected series.
Another application is in developing style files
or other kinds of included material
where compilation of the style file could redirect
to a sample or test file.

%%%%%%%%%%%%%%%%%%%%%%%%%%%%%%%%%%%%%%%%%%%%%%%%%%%%%%%%%%%%%%%%%%%%%%%%%%%%%%%%
%%%%%%%%%%%%%%%%%%%%%%%%%%%%%%%%%%%%%%%%%%%%%%%%%%%%%%%%%%%%%%%%%%%%%%%%%%%%%%%%
\section{Usage}

First of all, the package \textsf{childdoc} is \emph{not} a standard
\LaTeXe{} |.sty| style file! Therefore it needs to be invoked in
a non-standard way.

%%%%%%%%%%%%%%%%%%%%%%%%%%%%%%%%%%%%%%%%%%%%%%%%%%%%%%%%%%%%%%%%%%%%%%%%%%%%%%%%
\subsection{Included Files}
\label{sec:include}

%%%%%%%%%%%%%%%%%%%%%%%%%%%%%%%%%%%%%%%%
\DescribeMacro{\childdocmain}
To use the package, add the commands
\begin{center}
\begin{tabular}{l}
|\input{childdoc.def}|\\
|\childdocmain{}|\\
\end{tabular}
\end{center}
at the very top of the main \LaTeX{} file,
in particular \emph{before} the |\documentclass| statement!
The argument of |\childdocmain| should be left empty
(but it must be present).

%%%%%%%%%%%%%%%%%%%%%%%%%%%%%%%%%%%%%%%%
\DescribeMacro{\childdocof}
Furthermore, add the commands
\begin{center}
\begin{tabular}{l}
|\input{childdoc.def}|\\
|\childdocof{|\textit{main}|}|\\
\end{tabular}
\end{center}
at the top of every child file \textit{child}
which is included by |\include{|\textit{child}|}|
from within the main file
(or at least for those files to be compiled individually).
The argument \textit{main} must be the filename of the main file.

There are a couple of
considerations in setting up the main and child documents:

%%%%%%%%%%%%%%%%%%%%%%%%%%%%%%%%%%%%%%%%
\paragraph{Restrictions.}

Please note the following restrictions:
\begin{itemize}
\item
|\childdocmain| must be called with one argument \textit{main}
to ensure compatibility with earlier version of the package.
It must either be empty (|\childdocmain{}|)
or precisely match the filename of the main file in which it is specified.
See \secref{sec:detection} for further information.
\item
The filename \textit{main} must be specified without the |.tex| extension.
\item
The filename \textit{main} is case sensitive
(even in case-insensitive file systems)
due to internal string comparison.
\item
The argument \textit{main} should be fully expanded, it cannot be a macro.
\item
Subdirectories and special characters should be avoided in filenames.
\item
The command |\childdocmain{|\textit{main}|}| must be followed by a whitespace.
It should not be followed immediately by another command
or by a comment mark `|%|'.
This is because the \TeX{} parser reads the token immediately following
the argument of |\childdocmain| and puts it
at the beginning of every child section;
however, a white\-space is ignored.
\end{itemize}

%%%%%%%%%%%%%%%%%%%%%%%%%%%%%%%%%%%%%%%%
\paragraph{Content of Main File.}

It is advisable to place all content in the child files included by |\include|.
Any output contained in the main file will appear in all child documents
unless suppressed manually;
it cannot be suppressed automatically by the |\includeonly| directive
and thus should normally be avoided.
A method to include some content in the main file
by means of conditional processing is described in \secref{sec:conditional}.

%%%%%%%%%%%%%%%%%%%%%%%%%%%%%%%%%%%%%%%%
\paragraph{Page Numbering.}

When only a part of the document is compiled,
the appropriate numbering of pages
(as well as other status parameters)
is determined from the |.aux| files.
The latter contain information from previous passes.
However this information needs to propagate through
all intermediate child documents.
Therefore the page numbering in child documents may well
be inconsistent until the complete document is compiled at least once.

A useful (if unconventional) way to always ensure a consistent
page numbering is to restart the numbering in each child document
and denote the pages by `\textit{child}|.|\textit{page}'
where \textit{child} represents the chapter/section number of the child file.
This can be achieved by the command
|\numberwithin{page}{|\textit{child}|}|
of the \textsf{amsmath} package
where \textit{child} can be |chapter| or |section|
depending on the chosen structuring.
Alternatively, one can modify the macro |\thepage| appropriately
and reset the counter |page| at the start of each child file.

%%%%%%%%%%%%%%%%%%%%%%%%%%%%%%%%%%%%%%%%%%%%%%%%%%%%%%%%%%%%%%%%%%%%%%%%%%%%%%%%
\subsection{Conditional Processing}
\label{sec:conditional}

The package provides a mechanism to compile different versions
of a document. To customise the versions further some conditional processing
can come in handy to distinguish which version is being compiled.
The package provides two macros to describe the compilation context:

%%%%%%%%%%%%%%%%%%%%%%%%%%%%%%%%%%%%%%%%
\DescribeMacro{\ifchilddoc}
The conditional |\ifchilddoc| distinguishes between the compilation of
child documents and the main document:
%
\begin{center}
|\ifchilddoc |\textit{child-code}| |[|\||else |\textit{main-code}]| \||fi|
\end{center}

%%%%%%%%%%%%%%%%%%%%%%%%%%%%%%%%%%%%%%%%
\DescribeMacro{\childdocname}
\DescribeMacro{\childdocjob}
The macro |\childdocname| contains the filename (without extension)
of the main or child file being processed.
Note that |\childdocjob| will always contain the name of the main file.

%%%%%%%%%%%%%%%%%%%%%%%%%%%%%%%%%%%%%%%%
\paragraph{Title Page.}

Conditional processing can be used to include a title or banner page
in the main document when proper precautions are taken.
Importantly, the code in the main file should ensure that the page counter
(as well as other status parameters which are stored in the |.aux| files)
takes the same value after the conditional processing.
Otherwise the page numbers may take divergent values
depending on which part is compiled.

For example, a title page could be declared by:
%
\begin{center}
\begin{tabular}{l}
|\ifchilddoc\||else|\\
|\addtocounter{page}{-1}|\\
\textit{code for title page}\\
|\newpage|\\
|\||fi|
\end{tabular}
\end{center}
%
A banner page for the child documents can be generated by:
%
\begin{center}
\begin{tabular}{l}
|\ifchilddoc|\\
|\addtocounter{page}{-1}|\\
\textit{code for banner page}\\
|\newpage|\\
|\||fi|
\end{tabular}
\end{center}
%
Here one could write a message such as:
\begin{center}
|This is the part \childdocname{} of \childdocjob{}.|
\end{center}

%%%%%%%%%%%%%%%%%%%%%%%%%%%%%%%%%%%%%%%%%%%%%%%%%%%%%%%%%%%%%%%%%%%%%%%%%%%%%%%%
\subsection{Flags}
\label{sec:flags}

The package makes it easy to generate different versions
of the main or child documents.
To this end compilation flags can be defined
and assigned different default values.
They will be particularly useful in conjunction
with the forwarding mechanism described in \secref{sec:forward}.

For example, it may be useful to have a flag |\version|
which can be set to |draft| or |final|.
The document source will contain some conditional code
depending on the value of |\version|.
Suppose further, the flag should default to |final| for the main file
and to |draft| for child files
which is a natural assignment for editing the document.
This is achieved by placing the following code
in the preamble of the main document
(below the |\childdocmain| directive):
%
\begin{center}
\begin{tabular}{l}
|\ifchilddoc|\\
|\providecommand{\version}{draft}|\\
|\||else|\\
|\providecommand{\version}{final}|\\
|\||fi|
\end{tabular}
\end{center}
%
The definition by |\providecommand| makes sure
that previous definitions are not overwritten.
Further statements |\providecommand{\version}{...}|
can thus be added before the above code to override it.

For the main file, one might add a line
(between |\childdocmain| and the above block)
%
\begin{center}
|%\ifchilddoc\||else\providecommand{\version}{draft}\||fi|
\end{center}
%
which can be uncommented to produce a draft version.
Likewise one can add a line to the very top of a child file
(above the |\childdocof{|\textit{main}|}| directive)
%
\begin{center}
|%\providecommand{\version}{final}|
\end{center}
%
which can be uncommented to produce the final version of this child document.

%%%%%%%%%%%%%%%%%%%%%%%%%%%%%%%%%%%%%%%%%%%%%%%%%%%%%%%%%%%%%%%%%%%%%%%%%%%%%%%%
\subsection{Forwarding}
\label{sec:forward}

Different versions of the main or child documents
using compilation flags as described in \secref{sec:flags}
can be (permanently) stored in different files
for convenient compilation, viewing and distribution.
To this end, the package defines a command
to pass on compilation to a different file:

%%%%%%%%%%%%%%%%%%%%%%%%%%%%%%%%%%%%%%%%
\DescribeMacro{\childdocforward}
The command |\childdocforward| redirects processing to
another source file:
%
\begin{center}
\begin{tabular}{l}
|\input{childdoc.def}|\\
|\childdocforward[|\textit{main}|]{|\textit{dest}|}|\\
\end{tabular}
\end{center}
%
The argument \textit{dest} is the destination file
(without extension).
It should be the main file or one of the child files.
Note that further \textsf{childdoc} directives
such as |\childdocof| and |\childdocforward|
in the indicated file will be processed in this form.
The optional argument \textit{main}
passes on directly to the main file \textit{main}
while pretending to compile the child \textit{dest}.
This form behaves as if \textit{dest}
issues |\childdocof{|\textit{main}|}| right away,
and no further \textsf{childdoc} directives will be processed.

%%%%%%%%%%%%%%%%%%%%%%%%%%%%%%%%%%%%%%%%
\DescribeMacro{\...prefix}
In the alternative form |\childdocforwardprefix|,
%
\begin{center}
\begin{tabular}{l}
|\input{childdoc.def}|\\
|\childdocforwardprefix[|\textit{main}|]{|\textit{prefix}|}{|\textit{dest}|}|
\end{tabular}
\end{center}
%
the destination file is determined by a pattern
depending on the current file:
To make this work, the current file must be called
`{\textit{prefix}\hspace{0.2em}\textit{suffix}}'
with \textit{prefix} matching precisely the argument.
Processing is then passed on to the file
`{\textit{dest}\hspace{0.2em}\textit{suffix}}'.
Surely, the same effect is achieved by
directly specifying the
argument `{\textit{dest}\hspace{0.2em}\textit{suffix}}'
in the first form.
However, that requires to set up a different file
for each child. With the alternative form of the command
all these files can have exactly the same content
which simplifies setting them up and maintaining them.

For example, the following file |draft.tex|
with a compilation flag |\version| as described in \secref{sec:flags}
compiles the main document as a draft:
%
\begin{center}
\begin{tabular}{l}
|\def\version{draft}|\\
|\input{childdoc.def}|\\
|\childdocforward{|\textit{main}|}|
\end{tabular}
\end{center}
%
Likewise, the following files |final|\textit{nn}|.tex|
compile the final version of the child document
|child|\textit{nn}|.tex|:
%
\begin{center}
\begin{tabular}{l}
|\def\version{final}|\\
|\input{childdoc.def}|\\
|\childdocforwardprefix{final}{child}|
\end{tabular}
\end{center}
%

Note that when several versions of a main file and/or of each child file
are to be generated, it may be convenient to set up a |Makefile| or
shell script to automatise the process.

%%%%%%%%%%%%%%%%%%%%%%%%%%%%%%%%%%%%%%%%%%%%%%%%%%%%%%%%%%%%%%%%%%%%%%%%%%%%%%%%
\subsection{Command Line Processing}
\label{sec:commandline}

The effect of redirection files can also be achieved by invoking
the \LaTeX{} compiler with a more elaborate command line.
Most conveniently this should be done as part
of a shell script or a |Makefile|.

When using \textsf{childdoc} in the main file, the following
command lines effectively perform a redirection
(note that depending on the shell being used,
backslashes may have to be doubled: `|\|' $\to$ `|\\|'):
%
\begin{center}
|... -jobname "|\textit{target}|" |\\|"|[\textit{flags}]%
|\input{childdoc.def}\childdocforward[|\textit{main}|]{|\textit{dest}|}"|
\end{center}
%
Here \textit{target} is the name of the output file,
\textit{main} is the name of the main file
and \textit{dest} is the name of the main or child file to be processed
(all filenames without extensions).
The optional argument \textit{main} can be omitted
if \textit{main} matches \textit{dest}.
Optionally, compilation \textit{flags} can be defined via |\def| commands.
This command line makes the \TeX{} engine believe
it is compiling the file \textit{target}
whose content is specified as the latter parameter.
The provided code then forwards the processing to
\textit{main} or \textit{dest} as described in \secref{sec:forward}.

%%%%%%%%%%%%%%%%%%%%%%%%%%%%%%%%%%%%%%%%%%%%%%%%%%%%%%%%%%%%%%%%%%%%%%%%%%%%%%%%
\subsection{Include by Input}
\label{sec:input}

Including child documents by |\include| has some restrictions by design.
Most notably, the content of a child document always occupies
its own set of pages; pages cannot be shared between child documents.
Usually, this behaviour makes perfect sense
because each child document contain an essential part of the document.
However, in some situations it may be desirable to compose
a document from a collection of parts
without having mandatory page breaks between then.
For this case, the package
provides a mechanism to include parts
by |\input| which can also be processed individually.
However, by construction this mechanism
requires manual handling of the content to be output.

%%%%%%%%%%%%%%%%%%%%%%%%%%%%%%%%%%%%%%%%
\DescribeMacro{\ifchilddocmanual}
The main file should be prepared as usual, see \secref{sec:include}.
However, the document body must make a distinction
between processing of an individual part and of the main document, e.g.:
%
\begin{center}
\begin{tabular}{l}
|\ifchilddocmanual|\\
|\input{\childdocname}|\\
|\||else|\\
\textit{document body with }|\input{|\textit{part}|}|\\
|\||fi|
\end{tabular}
\end{center}
%
The conditional |\ifchilddocmanual| is true whenever
a part to be included by |\input| is being compiled,
and the name of the part is stored in |\childdocname|.

%%%%%%%%%%%%%%%%%%%%%%%%%%%%%%%%%%%%%%%%
\DescribeMacro{\childdocby}
Each part to be included by |\input| should start with:
%
\begin{center}
\begin{tabular}{l}
|\input{childdoc.def}|\\
|\childdocby{|\textit{main}|}|\\
\end{tabular}
\end{center}
%
The directive |\childdocby| is similar to |\childdocof|
described in \secref{sec:include},
but the subsequent selection of content must be done manually.
To that end, both |\ifchilddoc| and |\ifchilddocmanual|
will be true upon processing of a part,
and the name of the part is stored in |\childdocname|.
Note that |\jobname| will be set to the filename of the current part
so that each part receives an individual |.aux| file
that does not interfere with the |.aux| file(s) of the main document.
This behaviour can be altered by the alternative form
|\childdocby[*]{|\textit{main}|}| (with a non-empty optional argument)
which uses the |.aux| file of the main document
by setting |\jobname| to \textit{main}.

%%%%%%%%%%%%%%%%%%%%%%%%%%%%%%%%%%%%%%%%%%%%%%%%%%%%%%%%%%%%%%%%%%%%%%%%%%%%%%%%
\subsection{Driver Development}
\label{sec:driver}

The \textsf{childdoc} mechanism can also be use for the development
of definition files such as \LaTeX{} styles or classes.
This case differs from the above setup with multiple parts
included by |\include| in that no |\includeonly| should be invoked.
This can be achieved by starting the include file
(before |\ProvidesPackage|) with:
%
\begin{center}
\begin{tabular}{l}
|\input{childdoc.def}|\\
|\childdocforward{|\textit{main}|}|\\
\end{tabular}
\end{center}
%
or alternatively with:
%
\begin{center}
\begin{tabular}{l}
|\input{childdoc.def}|\\
|\childdocby{|\textit{main}|}|\\
\end{tabular}
\end{center}
%
Both forms have slightly different effects as described above.
The main file is prepared as usual, see \secref{sec:include}.

%%%%%%%%%%%%%%%%%%%%%%%%%%%%%%%%%%%%%%%%%%%%%%%%%%%%%%%%%%%%%%%%%%%%%%%%%%%%%%%%
\subsection{Legacy Detection}
\label{sec:detection}

The directive |\childdocmain| in the main file can detect
whether the complete document or merely a child is to be compiled
even without using the directive |\childdocof|.
This method is deprecated because it is less robust
and there is no compelling reason to use it;
it is merely provided for backward compatibility
and it may be removed in future versions.

If the detection mechanism is to be used,
it is mandatory to correctly specify
the filename of the main file as the argument of |\childdocmain|:
%
\begin{center}
\begin{tabular}{l}
|\input{childdoc.def}|\\
|\childdocmain{|\textit{main}|}|\\
\end{tabular}
\end{center}
%
If |\jobname| does not match the argument \textit{main} of |\childdocmain|,
it is assumed that |\jobname| points to the child file to be compiled.
When using |\childdocmain| with the main file specified as argument,
it suffices to start a child file
with just |\input{|\textit{main}|}|
without loading of the package and using |\childdocof|.
If instead all processing is done
with the appropriate \textsf{childdoc} directives,
the argument of \textit{main} of |\childdocmain| can be empty.

An alternative version of the command line processing described
in \secref{sec:commandline} using the detection mechanism reads:
%
\begin{center}
|... -jobname "|\textit{target}|" "|[\textit{flags}]%
[|\def\jobname{|\textit{dest}|}|]|\input{|\textit{main}|}"|
\end{center}

%%%%%%%%%%%%%%%%%%%%%%%%%%%%%%%%%%%%%%%%%%%%%%%%%%%%%%%%%%%%%%%%%%%%%%%%%%%%%%%%
\subsection{Manual Code}
\label{sec:manual}

In case one cannot be certain whether the definitions file |childdoc.def|
is installed on the target \TeX{} distribution
and one prefers not to ship it,
it is conceivable to paste a few relevant commands into the sources.

To that end, drop all statements |\input{childdoc.def}|
and perform the replacements as outlined below.
Instead of |\childdocmain{|\textit{main}|}| add the following code
to the top of the main file:
%
\begin{center}
\begin{tabular}{l}
|\||ifdefined\childdocname\endinput\||fi\newif\ifchilddoc|\\
|\edef\childdocname{\scantokens\expandafter{\jobname\noexpand}}|\\
|\def\childdocmain{|\textit{main}|}\||ifx\childdocmain\childdocname\||else|\\
|\childdoctrue\includeonly{\childdocname}\let\jobname\childdocmain\||fi|\\
\end{tabular}
\end{center}
%
Instead of |\childdocof{|\textit{main}|}| just include the main file
at the top of each child file:
%
\begin{center}
|\input{|\textit{main}|}|
\end{center}
%
A simple redirection |\childdocforward{|\textit{dest}|}| is achieved by:
%
\begin{center}
|\def\jobname{|\textit{dest}|}\input{\jobname}|
\end{center}
%
The redirection with prefix
|\childdocforwardprefix[|\textit{prefix}|]{|\textit{dest}|}|
is accomplished by:
%
\begin{center}
\begin{tabular}{l}
|{\edef\jobname{\scantokens\expandafter{\jobname\noexpand}}|\\
|\def\redirectjob |\textit{prefix}|#1~~~{\gdef\jobname{|\textit{dest}|#1}}|\\
|\expandafter\redirectjob\jobname~~~}\input{\jobname}|
\end{tabular}
\end{center}

In an alternative approach,
child documents can be compiled by a specific command line
without additional code or specific definitions:
%
\begin{center}
|... -jobname "|\textit{target}|" "|[\textit{flags}]%
|\includeonly{|\textit{dest}|}\input{|\textit{main}|}"|
\end{center}
%

%%%%%%%%%%%%%%%%%%%%%%%%%%%%%%%%%%%%%%%%%%%%%%%%%%%%%%%%%%%%%%%%%%%%%%%%%%%%%%%%
%%%%%%%%%%%%%%%%%%%%%%%%%%%%%%%%%%%%%%%%%%%%%%%%%%%%%%%%%%%%%%%%%%%%%%%%%%%%%%%%
\section{Information}

%%%%%%%%%%%%%%%%%%%%%%%%%%%%%%%%%%%%%%%%%%%%%%%%%%%%%%%%%%%%%%%%%%%%%%%%%%%%%%%%
\subsection{Copyright}

Copyright \copyright{} 2017--2018 Niklas Beisert

This work may be distributed and/or modified under the
conditions of the \LaTeX{} Project Public License, either version 1.3
of this license or (at your option) any later version.
The latest version of this license is in
  \url{http://www.latex-project.org/lppl.txt}
and version 1.3 or later is part of all distributions of \LaTeX{}
version 2005/12/01 or later.

This work has the LPPL maintenance status `maintained'.

The Current Maintainer of this work is Niklas Beisert.

This work consists of the files |README.txt|, |childdoc.ins| and |childdoc.dtx|
as well as the derived files |childdoc.def|, |cdocsamp.tex|
with |cdocsch1.tex|, |cdocsch2.tex|, |cdocspt3.tex|, |cdocspt4.tex|,
|cdocsdrf.tex|, |cdocsfn1.tex|, |cdocsfn2.tex|
as well as |childdoc.pdf|.

%%%%%%%%%%%%%%%%%%%%%%%%%%%%%%%%%%%%%%%%%%%%%%%%%%%%%%%%%%%%%%%%%%%%%%%%%%%%%%%%
\subsection{Files and Installation}

The package consists of the files:
%
\begin{center}
\begin{tabular}{ll}
    |README.txt|   & readme file \\
    |childdoc.ins| & installation file \\
    |childdoc.dtx| & source file \\
    |childdoc.def| & definition file \\
    |cdocsamp.tex| & sample main file \\
    |cdocsch1.tex| & sample include file \\
    |cdocsch2.tex| & sample include file \\
    |cdocspt3.tex| & sample part file \\
    |cdocspt4.tex| & sample part file \\
    |cdocsdrf.tex| & sample redirection file \\
    |cdocsfn1.tex| & sample redirection file \\
    |cdocsfn2.tex| & sample redirection file \\
    |childdoc.pdf| & manual
\end{tabular}
\end{center}
%
The distribution consists of the files
|README.txt|, |childdoc.ins| and |childdoc.dtx|.
%
\begin{itemize}
\item
Run (pdf)\LaTeX{} on |childdoc.dtx|
to compile the manual |childdoc.pdf| (this file).
\item
Run \LaTeX{} on |childdoc.ins| to create the definitions file |childdoc.def|
and the sample |cdocsamp.tex| with include files
|cdocsch1.tex|, |cdocsch2.tex|, |cdocspt3.tex|, |cdocspt4.tex|,
|cdocsdrf.tex|, |cdocsfn1.tex|, |cdocsfn2.tex|.
Then copy the file |childdoc.def| to an appropriate directory of your \LaTeX{}
distribution, e.g.\ \textit{texmf-root}|/tex/latex/childdoc|.
\end{itemize}

%%%%%%%%%%%%%%%%%%%%%%%%%%%%%%%%%%%%%%%%%%%%%%%%%%%%%%%%%%%%%%%%%%%%%%%%%%%%%%%%
\subsection{Related CTAN Packages}

There are several other packages which offer a similar functionality:
%
\begin{itemize}
\item
The packages
\href{http://ctan.org/pkg/docmute}{\textsf{docmute}},
\href{http://ctan.org/pkg/includex}{\textsf{includex}} and
\href{http://ctan.org/pkg/standalone}{\textsf{standalone}}
provide commands to include only the document body of
a child file thus allowing both files to be compiled individually.
\item
The packages \href{http://ctan.org/pkg/subdocs}{\textsf{subdocs}}
and \href{http://ctan.org/pkg/subfiles}{\textsf{subfiles}}
provide structures in which the main and child documents can be
encapsulated and allowing them to be compiled individually.
The inclusion mechanism is different from the conventional |\include|.
\item
The package \href{http://ctan.org/pkg/combine}{\textsf{combine}}
is an elaborate solution to combine several documents into one.
\end{itemize}
%
See also the CTAN topic \href{http://ctan.org/topic/subdocs}{\textsf{subdocs}}
for further related packages.
The present package differs from the above solutions in that
a document structure constructed with the conventional |\include| mechanism
just needs two extra commands at the top of every file
such that all constituent files can be compiled individually.

%%%%%%%%%%%%%%%%%%%%%%%%%%%%%%%%%%%%%%%%%%%%%%%%%%%%%%%%%%%%%%%%%%%%%%%%%%%%%%%%
%\subsection{Feature Suggestions}
%
%The following is a list of features which may be useful for future
%versions of this package:
%%
%\begin{itemize}
%\item
%\ldots
%\end{itemize}

%%%%%%%%%%%%%%%%%%%%%%%%%%%%%%%%%%%%%%%%%%%%%%%%%%%%%%%%%%%%%%%%%%%%%%%%%%%%%%%%
\subsection{Revision History}

%%%%%%%%%%%%%%%%%%%%%%%%%%%%%%%%%%%%%%%%
\paragraph{v2.0:} 2018/12/30

\begin{itemize}
\item
immediate forward processing
\item
added |\childdocby| mechanism
\item
manual restructured
\end{itemize}

%%%%%%%%%%%%%%%%%%%%%%%%%%%%%%%%%%%%%%%%
\paragraph{v1.6:} 2018/01/17

\begin{itemize}
\item
application for development of include files
\item
corrections to manual
\end{itemize}

%%%%%%%%%%%%%%%%%%%%%%%%%%%%%%%%%%%%%%%%
\paragraph{v1.5:} 2017/05/21

\begin{itemize}
\item
more complete structuring introduced
\item
|\childdocof| introduced
\item
|\childdoc| renamed to |\childdocmain|
\item
|\childredirect| renamed to |\childdocforward| and |\childdocforwardprefix|
and functionality expanded
\end{itemize}

%%%%%%%%%%%%%%%%%%%%%%%%%%%%%%%%%%%%%%%%
\paragraph{v1.0:} 2017/04/27

\begin{itemize}
\item
manual and install package
\item
first version published on CTAN
\end{itemize}

%%%%%%%%%%%%%%%%%%%%%%%%%%%%%%%%%%%%%%%%
\paragraph{v0.6:} 2017/04/26

\begin{itemize}
\item
redirection mechanism added
\end{itemize}

%%%%%%%%%%%%%%%%%%%%%%%%%%%%%%%%%%%%%%%%
\paragraph{v0.5:} 2017/04/26

\begin{itemize}
\item
functionality in definition file
\end{itemize}


%%%%%%%%%%%%%%%%%%%%%%%%%%%%%%%%%%%%%%%%%%%%%%%%%%%%%%%%%%%%%%%%%%%%%%%%%%%%%%%%
%%%%%%%%%%%%%%%%%%%%%%%%%%%%%%%%%%%%%%%%%%%%%%%%%%%%%%%%%%%%%%%%%%%%%%%%%%%%%%%%
%%%%%%%%%%%%%%%%%%%%%%%%%%%%%%%%%%%%%%%%%%%%%%%%%%%%%%%%%%%%%%%%%%%%%%%%%%%%%%%%
\appendix

\settowidth\MacroIndent{\rmfamily\scriptsize 000\ }

 \DocInput{childdoc.dtx}

\end{document}
%</driver>
% \fi
%
% %%%%%%%%%%%%%%%%%%%%%%%%%%%%%%%%%%%%%%%%%%%%%%%%%%%%%%%%%%%%%%%%%%%%%%%%%%%%%%
% %%%%%%%%%%%%%%%%%%%%%%%%%%%%%%%%%%%%%%%%%%%%%%%%%%%%%%%%%%%%%%%%%%%%%%%%%%%%%%
% \section{Sample}
%\iffalse
%<*samplemain>
%\fi
%
% The following presents a sample document
% with two chapters, two parts, a title page,
% a compile flag as well as three forwarding files to set the flag.
% It consists of eight |.tex| files:
% \begin{center}
% \begin{tabular}{ll}
% |cdocsamp.tex|&main file\\
% |cdocsch1.tex|&include file for chapter 1\\
% |cdocsch2.tex|&include file for chapter 2\\
% |cdocspt3.tex|&include file for part 3\\
% |cdocspt4.tex|&include file for part 4\\
% |cdocsdrf.tex|&forwarding file for main file in draft mode\\
% |cdocsfi1.tex|&forwarding file for final version of chapter 1\\
% |cdocsfi2.tex|&forwarding file for final version of chapter 2\\
% \end{tabular}
% \end{center}
% Each of the eight files can be compiled directly by the \LaTeX{} compiler.
%
% %%%%%%%%%%%%%%%%%%%%%%%%%%%%%%%%%%%%%%
% \paragraph{Main File.}
%
% The main file is called |cdocsamp.tex|.
%
% Load the \textsf{childdoc} definitions and
% declare the filename for the main document:
%    \begin{macrocode}
\input{childdoc.def}
\childdocmain{}
%    \end{macrocode}

% Optional override for |\version| flag:
%    \begin{macrocode}
%%\ifchilddoc\else\providecommand{\version}{draft}\fi
%    \end{macrocode}

% Define the default values for the |\version| flag
% (|final| for the main file and |draft| for childs):
%    \begin{macrocode}
\ifchilddoc
\providecommand{\version}{draft}
\else
\providecommand{\version}{final}
\fi
%    \end{macrocode}

% Load the standard document class:
%    \begin{macrocode}
\documentclass[12pt]{article}
%    \end{macrocode}

% Start the document body:
%    \begin{macrocode}
\begin{document}
%    \end{macrocode}

% Declare a title page.
% Print title, part of document being processed and version flag:
%    \begin{macrocode}
\addtocounter{page}{-1}
\begin{center}
{\LARGE\bfseries{}childdoc example\par}
\vspace{1cm}
\ifchilddoc
\ifchilddocmanual part\else chapter\fi:
`\childdocname' of `\childdocjob'\par
\else
main document: `\childdocjob'\par
\fi
version: \version\par
\end{center}
\newpage
%    \end{macrocode}

% Manually include selected file,
% otherwise process as usual:
%    \begin{macrocode}
\ifchilddocmanual
\section*{part `\childdocname'}
\input{\childdocname}
\else
%    \end{macrocode}

% Include the two chapters:
%    \begin{macrocode}
\include{cdocsch1}
\include{cdocsch2}
%    \end{macrocode}

% Include the two parts unless only chapters should be displayed:
%    \begin{macrocode}
\ifchilddoc\else
\section{part three}
\input{cdocspt3}
\section{part four}
\input{cdocspt4}
\fi
%    \end{macrocode}

% Process as usual until here:
%    \begin{macrocode}
\fi
%    \end{macrocode}

% End of document body:
%    \begin{macrocode}
\end{document}
%    \end{macrocode}
%\iffalse
%</samplemain>
%\fi
%
% %%%%%%%%%%%%%%%%%%%%%%%%%%%%%%%%%%%%%%
% \paragraph{Chapter Include Files.}
%
% The include files are called |cdocsch1.tex| and |cdocsch2.tex|.
%
%\iffalse
%<*samplechap1|samplechap2>
%\fi

% Optional override for |\version| flag:
%    \begin{macrocode}
%%\providecommand{\version}{final}
%    \end{macrocode}

% Include the main document:
%    \begin{macrocode}
\input{childdoc.def}
\childdocof{cdocsamp}
%    \end{macrocode}

%\iffalse
%</samplechap1|samplechap2>
%\fi
%
%\iffalse
%<*samplechap1>
%\fi
% Some text for chapter 1:
%    \begin{macrocode}
\section{one}
some text in chapter one
%    \end{macrocode}

%\iffalse
%</samplechap1>
%\fi
% Some text for chapter 2:
%\iffalse
%<*samplechap2>
%\fi
%    \begin{macrocode}
\section{two}
more text in chapter two
%    \end{macrocode}

%\iffalse
%</samplechap2>
%\fi
%
% %%%%%%%%%%%%%%%%%%%%%%%%%%%%%%%%%%%%%%
% \paragraph{Part Include Files.}
%
% The include files are called |cdocspt3.tex| and |cdocspt4.tex|.
%
%\iffalse
%<*samplepart3|samplepart4>
%\fi

% Optional override for |\version| flag:
%    \begin{macrocode}
%%\providecommand{\version}{final}
%    \end{macrocode}

% Include the main document:
%    \begin{macrocode}
\input{childdoc.def}
\childdocby{cdocsamp}
%    \end{macrocode}

%\iffalse
%</samplepart3|samplepart4>
%\fi
%
%\iffalse
%<*samplepart3>
%\fi
% Some text for part 3:
%    \begin{macrocode}
some text in part three
%    \end{macrocode}

%\iffalse
%</samplepart3>
%\fi
% Some text for part 4:
%\iffalse
%<*samplepart4>
%\fi
%    \begin{macrocode}
more text in part four
%    \end{macrocode}

%\iffalse
%</samplepart4>
%\fi
%
% %%%%%%%%%%%%%%%%%%%%%%%%%%%%%%%%%%%%%%
% \paragraph{Forwarding for a Complete Draft.}
%
% The following forwarding file |cdocsdrf.tex|
% compiles the main document in draft mode:
%\iffalse
%<*sampledraft>
%\fi
%    \begin{macrocode}
\def\version{draft}
\input{childdoc.def}
\childdocforward{cdocsamp}
%    \end{macrocode}

%\iffalse
%</sampledraft>
%\fi
%
% %%%%%%%%%%%%%%%%%%%%%%%%%%%%%%%%%%%%%%
% \paragraph{Forwarding for Final Version of the Chapters.}
%
% The following forwarding files |cdocsfn1.tex| and |cdocsfn2.tex|
% (with identical content)
% compile the final versions of the child documents
% |cdocsch1.tex| and |cdocsch2.tex|, respectively:
%\iffalse
%<*samplefinal>
%\fi
%    \begin{macrocode}
\def\version{final}
\input{childdoc.def}
\childdocforwardprefix[cdocsamp]{cdocsfn}{cdocsch}
%    \end{macrocode}

%\iffalse
%</samplefinal>
%\fi
%
% %%%%%%%%%%%%%%%%%%%%%%%%%%%%%%%%%%%%%%
% \paragraph{Command Line Processing.}
%
% The following three command lines generate the output files
% |cdocscld|, |cdocscl1| and |cdocscl2|
% which should be identical to
% |cdocsdrf|, |cdocsch1| and |cdocsfn2|, respectively:
% \begin{center}
% \begin{tabular}{l}
% |latex -jobname cdocscld \|\\
% |  "\def\version{draft}\input{childdoc.def}\childdocforward{cdocsamp}"|\\
% |latex -jobname cdocscl1 \|\\
% |  "\input{childdoc.def}\childdocforward[cdocsamp]{cdocsch1}"|\\
% |latex -jobname cdocscl2 \|\\
% |  "\def\version{final}\input{childdoc.def}\childdocforward{cdocsch2}"|
% \end{tabular}
% \end{center}
% Note that the trailing backslash on each first line
% merely continues the input to the second line
% (for convenient cut ant paste).
% Furthermore, the command |latex| can be replaced by any
% of its alternative versions such as |pdflatex|.
%
% %%%%%%%%%%%%%%%%%%%%%%%%%%%%%%%%%%%%%%%%%%%%%%%%%%%%%%%%%%%%%%%%%%%%%%%%%%%%%%
% %%%%%%%%%%%%%%%%%%%%%%%%%%%%%%%%%%%%%%%%%%%%%%%%%%%%%%%%%%%%%%%%%%%%%%%%%%%%%%
% \section{Implementation}
%\iffalse
%<*package>
%\fi
%
% This section describes the definitions file |childdoc.def|.

% The definitions cannot be loaded using |\usepackage| or |\RequirePackage|
% which has a mechanism to prevent loading a style file more than once.
% When loading the definitions by means of |\input|
% multiple instances have to be prevented manually:
%\iffalse
%This code needs to be before the `\ProvidesFile' directive
%which is defined at the beginning of this file.
%Therefore it is also placed there and commented out here.
%</package>
%<*discard>
%\fi
%    \begin{macrocode}
\ifdefined\childdocmain\endinput\fi
%    \end{macrocode}
%\iffalse
%</discard>
%<*package>
%\fi
%
% \macro{\ifchilddoc}
% \macro{\ifchilddocmanual}
% The conditional |\ifchilddoc| tells whether a
% child (true) or main (false) document is being compiled.
% The conditional |\ifchilddocmanual| tells whether
% the |\includeonly| mechanism is used (false) or
% the selection of child files must be performed manually (true).
% The definitions initialise to false:
%    \begin{macrocode}
\newif\ifchilddoc
\newif\ifchilddocmanual
%    \end{macrocode}

% \macro{\childdocname}
% \macro{\childdocjob}
% The macro |\childdocname| stores the name of the main document
% to be compiled. The macro |\childdocjob| stores the name of
% the document on which the \LaTeX{} compiler was originally invoked.
% The content of |\jobname| cannot be compared
% to filenames specified in the source due to different catcodes.
% The following code rescans |\jobname|, stores the result
% in |\childdocname| and saves a copy in |\childdocjob|:
%    \begin{macrocode}
\edef\childdocname{\scantokens\expandafter{\jobname\noexpand}}
\let\childdocjob\childdocname
%    \end{macrocode}

% \macro{\childdocdisable}
% The macro |\childdocdisable| prevents the main file
% from being processed more than once.
% At this stage, the main document command |\childdocmain|
% is assumed to be called once again where it should do nothing.
% Any subsequent call to it should prevent
% a secondary processing of the main document
% It overwrites the forwarding commands
% |\childdocof| and |\childdocforward|
% with empty macros to prevent further inclusions of the main document:
%    \begin{macrocode}
\newcommand{\childdocdisable}
{
  \renewcommand{\childdocmain}[1]{\renewcommand{\childdocmain}[1]{\endinput}}
  \renewcommand{\childdocof}[1]{}
  \renewcommand{\childdocby}[2][]{}
  \renewcommand{\childdocforward}[2][]{}
  \renewcommand{\childdocdisable}{}
}
%    \end{macrocode}

% \macro{\childdocmain}
% The macro |\childdocmain| is to be called at the top of the main file
% with nothing or the main filename (without extension) as argument.
% First, it breaks loops.
% If the argument is not empty and does not match |\childdocname|
% (which is set by the first inclusion of |childdoc.def|),
% |\ifchilddoc| is set to true, |\includeonly| is applied to the child file
% and |\jobname| is set to the main file
% (for proper handling of |.aux| files):
%    \begin{macrocode}
\newcommand{\childdocmain}[1]
{
  \childdocdisable\childdocmain{}
  \if?#1?\else
    \begingroup
      \def\childdoctmp{#1}
      \ifx\childdoctmp\childdocname
        \def\childdoctmp{}
      \else
        \def\childdoctmp
        {
          \childdoctrue
          \includeonly{\childdocname}
          \def\childdocjob{#1}
          \def\jobname{#1}
        }
      \fi
      \expandafter
    \endgroup
    \childdoctmp
  \fi
}
%    \end{macrocode}

% \macro{\childdocof}
% The command |\childdocof| redirects
% compilation to the main file |#1|.
%    \begin{macrocode}
\newcommand{\childdocof}[1]
{
  \childdocdisable
  \childdoctrue
  \includeonly{\childdocname}
  \def\jobname{#1}
  \def\childdocjob{#1}
  \input{#1}
}
%    \end{macrocode}

% \macro{\childdocby}
% The command |\childdocby| ....
%    \begin{macrocode}
\newcommand{\childdocby}[2][]
{
  \childdocdisable
  \childdoctrue
  \childdocmanualtrue
  \if?#1?\else
    \def\jobname{#2}
  \fi
  \def\childdocjob{#2}
  \input{#2}
  \endinput
}
%    \end{macrocode}

% \macro{\childdocforward}
% The command |\childdocforward| redirects
% compilation to the main file or
% (if the optional argument is given) a child file.
% Parameters are set as if the main file
% or a child file starting with |\childdocof| was compiled.
% Then compilation is handed over to the main file:
%    \begin{macrocode}
\newcommand{\childdocforward}[2][]
{
  \begingroup
    \if?#1?
      \def\childdoctmp
      {
        \def\childdocname{#2}
        \def\childdocjob{#2}
        \def\jobname{#2}
        \input{#2}
        \endinput
      }
    \else
      \def\childdoctmp
      {
        \childdocdisable
        \def\childdocname{#2}
        \childdoctrue
        \includeonly{#2}
        \def\childdocjob{#1}
        \def\jobname{#1}
        \input{#1}
        \endinput
      }
    \fi
    \expandafter
  \endgroup
  \childdoctmp
}
%    \end{macrocode}

% \macro{\childdocforwardprefix}
% The command |\childdocforwardprefix| redirects
% compilation to the main or a child file by means of a pattern.
% The prefix |#1| in the current filename is replaced by |#2|
% and the suffix of the current filename is kept
% (it is assumed that the filename does not contain the substring `|~~~|'
% which is used as a delimiter).
% Compilation is handed over to the new file by |\childdocforward|:
%    \begin{macrocode}
\newcommand{\childdocforwardprefix}[3][]
{
  \begingroup
    \def\childdocextract #2##1~~~{\def\childdoctmp{\childdocforward[#1]{#3##1}}}
    \expandafter\childdocextract\childdocname~~~
    \expandafter
  \endgroup
  \childdoctmp
}
%    \end{macrocode}

% \macro{\childdoc}
% The deprecated macro |\childdoc| is a legacy version of |\childdocmain|:
%    \begin{macrocode}
\newcommand{\childdoc}{\childdocmain}
%    \end{macrocode}

% \macro{\childdocredirect}
% The deprecated macro |\childdocredirect| is a legacy version
% of |\childdocforward| and |\childdocforwardprefix|:
%    \begin{macrocode}
\newcommand{\childdocredirect}[2][]
{
  \begingroup
    \if?#1?
      \def\childdoctmp{\childdocforward{#2}}
    \else
      \def\childdoctmp{\childdocforwardprefix{#1}{#2}}
    \fi
    \expandafter
  \endgroup
  \childdoctmp
}
%    \end{macrocode}

%\iffalse
%</package>
%\fi
%
\endinput
|\\
|\childdocforward[|\textit{main}|]{|\textit{dest}|}|\\
\end{tabular}
\end{center}
%
The argument \textit{dest} is the destination file
(without extension).
It should be the main file or one of the child files.
Note that further \textsf{childdoc} directives
such as |\childdocof| and |\childdocforward|
in the indicated file will be processed in this form.
The optional argument \textit{main}
passes on directly to the main file \textit{main}
while pretending to compile the child \textit{dest}.
This form behaves as if \textit{dest}
issues |\childdocof{|\textit{main}|}| right away,
and no further \textsf{childdoc} directives will be processed.

%%%%%%%%%%%%%%%%%%%%%%%%%%%%%%%%%%%%%%%%
\DescribeMacro{\...prefix}
In the alternative form |\childdocforwardprefix|,
%
\begin{center}
\begin{tabular}{l}
|% \iffalse
%
% childdoc.dtx Copyright (C) 2017-2018 Niklas Beisert
%
% This work may be distributed and/or modified under the
% conditions of the LaTeX Project Public License, either version 1.3
% of this license or (at your option) any later version.
% The latest version of this license is in
%   http://www.latex-project.org/lppl.txt
% and version 1.3 or later is part of all distributions of LaTeX
% version 2005/12/01 or later.
%
% This work has the LPPL maintenance status `maintained'.
%
% The Current Maintainer of this work is Niklas Beisert.
%
% This work consists of the files childdoc.dtx and childdoc.ins
% and the derived files childdoc.def and cdocsamp.tex with
% cdocsch1.tex, cdocsch2.tex, cdocsdrf.tex, cdocsfn1.tex, cdocsfn2.tex.
%
%<package>\ifdefined\childdocmain\endinput\fi
%<package>\ProvidesFile{childdoc.def}[2018/12/30 v2.0 child document driver]
%<samplemain>\ProvidesFile{cdocsamp.tex}[2018/12/30 v2.0 sample for childdoc]
%<*driver>
%\ProvidesFile{childdoc.drv}[2018/12/30 v2.0 childdoc reference manual file]
\PassOptionsToClass{10pt,a4paper}{article}
\documentclass{ltxdoc}

\usepackage[margin=35mm]{geometry}
\usepackage{hyperref}
\usepackage{hyperxmp}
\usepackage[usenames]{color}

\hypersetup{colorlinks=true}
\hypersetup{pdfstartview=FitH}
\hypersetup{pdfpagemode=UseNone}
\hypersetup{pdfsource={}}
\hypersetup{pdflang={en-UK}}
\hypersetup{pdfcopyright={Copyright 2017-2018 Niklas Beisert.
  This work may be distributed and/or modified under the
  conditions of the LaTeX Project Public License, either version 1.3
  of this license or (at your option) any later version.}}
\hypersetup{pdflicenseurl={http://www.latex-project.org/lppl.txt}}
\hypersetup{pdfcontactaddress={ETH Zurich, ITP, HIT K,
  Wolfgang-Pauli-Strasse 27}}
\hypersetup{pdfcontactpostcode={8093}}
\hypersetup{pdfcontactcity={Zurich}}
\hypersetup{pdfcontactcountry={Switzerland}}
\hypersetup{pdfcontactemail={nbeisert@itp.phys.ethz.ch}}
\hypersetup{pdfcontacturl={http://people.phys.ethz.ch/\xmptilde nbeisert/}}

\newcommand{\secref}[1]{\hyperref[#1]{section \ref*{#1}}}

\parskip1ex
\parindent0pt
\let\olditemize\itemize
\def\itemize{\olditemize\parskip0pt}

\begin{document}

\title{The \textsf{childdoc} Package}
\hypersetup{pdftitle={The childdoc Package}}
\author{Niklas Beisert\\[2ex]
  Institut f\"ur Theoretische Physik\\
  Eidgen\"ossische Technische Hochschule Z\"urich\\
  Wolfgang-Pauli-Strasse 27, 8093 Z\"urich, Switzerland\\[1ex]
  \href{mailto:nbeisert@itp.phys.ethz.ch}
  {\texttt{nbeisert@itp.phys.ethz.ch}}}
\hypersetup{pdfauthor={Niklas Beisert}}
\hypersetup{pdfsubject={Manual for the LaTeX2e Package childdoc}}
\date{30 December 2018, \textsf{v2.0}}
\maketitle

\begin{abstract}\noindent
\textsf{childdoc} is a \LaTeXe{} package
that enables the direct compilation
of document sections included by |\include|
to individual files.
\end{abstract}

\begingroup
\parskip0ex
\tableofcontents
\endgroup

%%%%%%%%%%%%%%%%%%%%%%%%%%%%%%%%%%%%%%%%%%%%%%%%%%%%%%%%%%%%%%%%%%%%%%%%%%%%%%%%
%%%%%%%%%%%%%%%%%%%%%%%%%%%%%%%%%%%%%%%%%%%%%%%%%%%%%%%%%%%%%%%%%%%%%%%%%%%%%%%%
\section{Introduction}

\LaTeX{} provides a mechanism to structure a large document (such as a book)
into a main file and several child files (containing the chapters)
using the |\include| command.
This mechanism is beneficial for documents
which span hundreds of pages in order to
make the source file(s) more manageable.
Moreover, compilation can be restricted to
selected child files by means of the |\includeonly| command.
The latter feature can be used to reduce the compilation time while editing
(this was significantly more useful in the earlier days of \LaTeX{})
or to generate a smaller document which is easier to navigate.
Another application of |\includeonly| is to generate
documents consisting of selected parts of the complete document.

However, there are a few drawbacks of the plain |\include| mechanism:
\begin{itemize}
\item
The child files cannot be compiled on their own,
they can only be compiled via the main file.
A naive editing environment
(such as a text editor with an option
to have the current file processed by \LaTeX)
may require one to switch to the main file before compiling;
attempting to compile the child file produces errors.
\item
The main file must be modified (each time)
to adjust the |\includeonly| command
to the present needs. This easily leaves the main file in a messy state.
\item
The generated document will always carry the filename
of the main document. This is inconvenient if
several child files are to be compiled and
to be kept for distribution.
\end{itemize}

The present package provides a simple interface
to make child files individually compilable by \LaTeX{}.
Compiling a child file then has the same effect as compiling
the main file with an |\includeonly| command
to select the appropriate child.
Moreover the generated document will carry the name of the child
rather than the main file.
This resolves all three above issues.

This feature is meant to make the editing of books,
thesis documents and lecture notes somewhat more convenient.
However, the package can also be used efficiently for
composing a series of documents (such as exercise sheets)
which are typically distributed individually.
It then assists the author in generating the individual documents
(potentially in different versions)
as well as a document containing the collected series.
Another application is in developing style files
or other kinds of included material
where compilation of the style file could redirect
to a sample or test file.

%%%%%%%%%%%%%%%%%%%%%%%%%%%%%%%%%%%%%%%%%%%%%%%%%%%%%%%%%%%%%%%%%%%%%%%%%%%%%%%%
%%%%%%%%%%%%%%%%%%%%%%%%%%%%%%%%%%%%%%%%%%%%%%%%%%%%%%%%%%%%%%%%%%%%%%%%%%%%%%%%
\section{Usage}

First of all, the package \textsf{childdoc} is \emph{not} a standard
\LaTeXe{} |.sty| style file! Therefore it needs to be invoked in
a non-standard way.

%%%%%%%%%%%%%%%%%%%%%%%%%%%%%%%%%%%%%%%%%%%%%%%%%%%%%%%%%%%%%%%%%%%%%%%%%%%%%%%%
\subsection{Included Files}
\label{sec:include}

%%%%%%%%%%%%%%%%%%%%%%%%%%%%%%%%%%%%%%%%
\DescribeMacro{\childdocmain}
To use the package, add the commands
\begin{center}
\begin{tabular}{l}
|\input{childdoc.def}|\\
|\childdocmain{}|\\
\end{tabular}
\end{center}
at the very top of the main \LaTeX{} file,
in particular \emph{before} the |\documentclass| statement!
The argument of |\childdocmain| should be left empty
(but it must be present).

%%%%%%%%%%%%%%%%%%%%%%%%%%%%%%%%%%%%%%%%
\DescribeMacro{\childdocof}
Furthermore, add the commands
\begin{center}
\begin{tabular}{l}
|\input{childdoc.def}|\\
|\childdocof{|\textit{main}|}|\\
\end{tabular}
\end{center}
at the top of every child file \textit{child}
which is included by |\include{|\textit{child}|}|
from within the main file
(or at least for those files to be compiled individually).
The argument \textit{main} must be the filename of the main file.

There are a couple of
considerations in setting up the main and child documents:

%%%%%%%%%%%%%%%%%%%%%%%%%%%%%%%%%%%%%%%%
\paragraph{Restrictions.}

Please note the following restrictions:
\begin{itemize}
\item
|\childdocmain| must be called with one argument \textit{main}
to ensure compatibility with earlier version of the package.
It must either be empty (|\childdocmain{}|)
or precisely match the filename of the main file in which it is specified.
See \secref{sec:detection} for further information.
\item
The filename \textit{main} must be specified without the |.tex| extension.
\item
The filename \textit{main} is case sensitive
(even in case-insensitive file systems)
due to internal string comparison.
\item
The argument \textit{main} should be fully expanded, it cannot be a macro.
\item
Subdirectories and special characters should be avoided in filenames.
\item
The command |\childdocmain{|\textit{main}|}| must be followed by a whitespace.
It should not be followed immediately by another command
or by a comment mark `|%|'.
This is because the \TeX{} parser reads the token immediately following
the argument of |\childdocmain| and puts it
at the beginning of every child section;
however, a white\-space is ignored.
\end{itemize}

%%%%%%%%%%%%%%%%%%%%%%%%%%%%%%%%%%%%%%%%
\paragraph{Content of Main File.}

It is advisable to place all content in the child files included by |\include|.
Any output contained in the main file will appear in all child documents
unless suppressed manually;
it cannot be suppressed automatically by the |\includeonly| directive
and thus should normally be avoided.
A method to include some content in the main file
by means of conditional processing is described in \secref{sec:conditional}.

%%%%%%%%%%%%%%%%%%%%%%%%%%%%%%%%%%%%%%%%
\paragraph{Page Numbering.}

When only a part of the document is compiled,
the appropriate numbering of pages
(as well as other status parameters)
is determined from the |.aux| files.
The latter contain information from previous passes.
However this information needs to propagate through
all intermediate child documents.
Therefore the page numbering in child documents may well
be inconsistent until the complete document is compiled at least once.

A useful (if unconventional) way to always ensure a consistent
page numbering is to restart the numbering in each child document
and denote the pages by `\textit{child}|.|\textit{page}'
where \textit{child} represents the chapter/section number of the child file.
This can be achieved by the command
|\numberwithin{page}{|\textit{child}|}|
of the \textsf{amsmath} package
where \textit{child} can be |chapter| or |section|
depending on the chosen structuring.
Alternatively, one can modify the macro |\thepage| appropriately
and reset the counter |page| at the start of each child file.

%%%%%%%%%%%%%%%%%%%%%%%%%%%%%%%%%%%%%%%%%%%%%%%%%%%%%%%%%%%%%%%%%%%%%%%%%%%%%%%%
\subsection{Conditional Processing}
\label{sec:conditional}

The package provides a mechanism to compile different versions
of a document. To customise the versions further some conditional processing
can come in handy to distinguish which version is being compiled.
The package provides two macros to describe the compilation context:

%%%%%%%%%%%%%%%%%%%%%%%%%%%%%%%%%%%%%%%%
\DescribeMacro{\ifchilddoc}
The conditional |\ifchilddoc| distinguishes between the compilation of
child documents and the main document:
%
\begin{center}
|\ifchilddoc |\textit{child-code}| |[|\||else |\textit{main-code}]| \||fi|
\end{center}

%%%%%%%%%%%%%%%%%%%%%%%%%%%%%%%%%%%%%%%%
\DescribeMacro{\childdocname}
\DescribeMacro{\childdocjob}
The macro |\childdocname| contains the filename (without extension)
of the main or child file being processed.
Note that |\childdocjob| will always contain the name of the main file.

%%%%%%%%%%%%%%%%%%%%%%%%%%%%%%%%%%%%%%%%
\paragraph{Title Page.}

Conditional processing can be used to include a title or banner page
in the main document when proper precautions are taken.
Importantly, the code in the main file should ensure that the page counter
(as well as other status parameters which are stored in the |.aux| files)
takes the same value after the conditional processing.
Otherwise the page numbers may take divergent values
depending on which part is compiled.

For example, a title page could be declared by:
%
\begin{center}
\begin{tabular}{l}
|\ifchilddoc\||else|\\
|\addtocounter{page}{-1}|\\
\textit{code for title page}\\
|\newpage|\\
|\||fi|
\end{tabular}
\end{center}
%
A banner page for the child documents can be generated by:
%
\begin{center}
\begin{tabular}{l}
|\ifchilddoc|\\
|\addtocounter{page}{-1}|\\
\textit{code for banner page}\\
|\newpage|\\
|\||fi|
\end{tabular}
\end{center}
%
Here one could write a message such as:
\begin{center}
|This is the part \childdocname{} of \childdocjob{}.|
\end{center}

%%%%%%%%%%%%%%%%%%%%%%%%%%%%%%%%%%%%%%%%%%%%%%%%%%%%%%%%%%%%%%%%%%%%%%%%%%%%%%%%
\subsection{Flags}
\label{sec:flags}

The package makes it easy to generate different versions
of the main or child documents.
To this end compilation flags can be defined
and assigned different default values.
They will be particularly useful in conjunction
with the forwarding mechanism described in \secref{sec:forward}.

For example, it may be useful to have a flag |\version|
which can be set to |draft| or |final|.
The document source will contain some conditional code
depending on the value of |\version|.
Suppose further, the flag should default to |final| for the main file
and to |draft| for child files
which is a natural assignment for editing the document.
This is achieved by placing the following code
in the preamble of the main document
(below the |\childdocmain| directive):
%
\begin{center}
\begin{tabular}{l}
|\ifchilddoc|\\
|\providecommand{\version}{draft}|\\
|\||else|\\
|\providecommand{\version}{final}|\\
|\||fi|
\end{tabular}
\end{center}
%
The definition by |\providecommand| makes sure
that previous definitions are not overwritten.
Further statements |\providecommand{\version}{...}|
can thus be added before the above code to override it.

For the main file, one might add a line
(between |\childdocmain| and the above block)
%
\begin{center}
|%\ifchilddoc\||else\providecommand{\version}{draft}\||fi|
\end{center}
%
which can be uncommented to produce a draft version.
Likewise one can add a line to the very top of a child file
(above the |\childdocof{|\textit{main}|}| directive)
%
\begin{center}
|%\providecommand{\version}{final}|
\end{center}
%
which can be uncommented to produce the final version of this child document.

%%%%%%%%%%%%%%%%%%%%%%%%%%%%%%%%%%%%%%%%%%%%%%%%%%%%%%%%%%%%%%%%%%%%%%%%%%%%%%%%
\subsection{Forwarding}
\label{sec:forward}

Different versions of the main or child documents
using compilation flags as described in \secref{sec:flags}
can be (permanently) stored in different files
for convenient compilation, viewing and distribution.
To this end, the package defines a command
to pass on compilation to a different file:

%%%%%%%%%%%%%%%%%%%%%%%%%%%%%%%%%%%%%%%%
\DescribeMacro{\childdocforward}
The command |\childdocforward| redirects processing to
another source file:
%
\begin{center}
\begin{tabular}{l}
|\input{childdoc.def}|\\
|\childdocforward[|\textit{main}|]{|\textit{dest}|}|\\
\end{tabular}
\end{center}
%
The argument \textit{dest} is the destination file
(without extension).
It should be the main file or one of the child files.
Note that further \textsf{childdoc} directives
such as |\childdocof| and |\childdocforward|
in the indicated file will be processed in this form.
The optional argument \textit{main}
passes on directly to the main file \textit{main}
while pretending to compile the child \textit{dest}.
This form behaves as if \textit{dest}
issues |\childdocof{|\textit{main}|}| right away,
and no further \textsf{childdoc} directives will be processed.

%%%%%%%%%%%%%%%%%%%%%%%%%%%%%%%%%%%%%%%%
\DescribeMacro{\...prefix}
In the alternative form |\childdocforwardprefix|,
%
\begin{center}
\begin{tabular}{l}
|\input{childdoc.def}|\\
|\childdocforwardprefix[|\textit{main}|]{|\textit{prefix}|}{|\textit{dest}|}|
\end{tabular}
\end{center}
%
the destination file is determined by a pattern
depending on the current file:
To make this work, the current file must be called
`{\textit{prefix}\hspace{0.2em}\textit{suffix}}'
with \textit{prefix} matching precisely the argument.
Processing is then passed on to the file
`{\textit{dest}\hspace{0.2em}\textit{suffix}}'.
Surely, the same effect is achieved by
directly specifying the
argument `{\textit{dest}\hspace{0.2em}\textit{suffix}}'
in the first form.
However, that requires to set up a different file
for each child. With the alternative form of the command
all these files can have exactly the same content
which simplifies setting them up and maintaining them.

For example, the following file |draft.tex|
with a compilation flag |\version| as described in \secref{sec:flags}
compiles the main document as a draft:
%
\begin{center}
\begin{tabular}{l}
|\def\version{draft}|\\
|\input{childdoc.def}|\\
|\childdocforward{|\textit{main}|}|
\end{tabular}
\end{center}
%
Likewise, the following files |final|\textit{nn}|.tex|
compile the final version of the child document
|child|\textit{nn}|.tex|:
%
\begin{center}
\begin{tabular}{l}
|\def\version{final}|\\
|\input{childdoc.def}|\\
|\childdocforwardprefix{final}{child}|
\end{tabular}
\end{center}
%

Note that when several versions of a main file and/or of each child file
are to be generated, it may be convenient to set up a |Makefile| or
shell script to automatise the process.

%%%%%%%%%%%%%%%%%%%%%%%%%%%%%%%%%%%%%%%%%%%%%%%%%%%%%%%%%%%%%%%%%%%%%%%%%%%%%%%%
\subsection{Command Line Processing}
\label{sec:commandline}

The effect of redirection files can also be achieved by invoking
the \LaTeX{} compiler with a more elaborate command line.
Most conveniently this should be done as part
of a shell script or a |Makefile|.

When using \textsf{childdoc} in the main file, the following
command lines effectively perform a redirection
(note that depending on the shell being used,
backslashes may have to be doubled: `|\|' $\to$ `|\\|'):
%
\begin{center}
|... -jobname "|\textit{target}|" |\\|"|[\textit{flags}]%
|\input{childdoc.def}\childdocforward[|\textit{main}|]{|\textit{dest}|}"|
\end{center}
%
Here \textit{target} is the name of the output file,
\textit{main} is the name of the main file
and \textit{dest} is the name of the main or child file to be processed
(all filenames without extensions).
The optional argument \textit{main} can be omitted
if \textit{main} matches \textit{dest}.
Optionally, compilation \textit{flags} can be defined via |\def| commands.
This command line makes the \TeX{} engine believe
it is compiling the file \textit{target}
whose content is specified as the latter parameter.
The provided code then forwards the processing to
\textit{main} or \textit{dest} as described in \secref{sec:forward}.

%%%%%%%%%%%%%%%%%%%%%%%%%%%%%%%%%%%%%%%%%%%%%%%%%%%%%%%%%%%%%%%%%%%%%%%%%%%%%%%%
\subsection{Include by Input}
\label{sec:input}

Including child documents by |\include| has some restrictions by design.
Most notably, the content of a child document always occupies
its own set of pages; pages cannot be shared between child documents.
Usually, this behaviour makes perfect sense
because each child document contain an essential part of the document.
However, in some situations it may be desirable to compose
a document from a collection of parts
without having mandatory page breaks between then.
For this case, the package
provides a mechanism to include parts
by |\input| which can also be processed individually.
However, by construction this mechanism
requires manual handling of the content to be output.

%%%%%%%%%%%%%%%%%%%%%%%%%%%%%%%%%%%%%%%%
\DescribeMacro{\ifchilddocmanual}
The main file should be prepared as usual, see \secref{sec:include}.
However, the document body must make a distinction
between processing of an individual part and of the main document, e.g.:
%
\begin{center}
\begin{tabular}{l}
|\ifchilddocmanual|\\
|\input{\childdocname}|\\
|\||else|\\
\textit{document body with }|\input{|\textit{part}|}|\\
|\||fi|
\end{tabular}
\end{center}
%
The conditional |\ifchilddocmanual| is true whenever
a part to be included by |\input| is being compiled,
and the name of the part is stored in |\childdocname|.

%%%%%%%%%%%%%%%%%%%%%%%%%%%%%%%%%%%%%%%%
\DescribeMacro{\childdocby}
Each part to be included by |\input| should start with:
%
\begin{center}
\begin{tabular}{l}
|\input{childdoc.def}|\\
|\childdocby{|\textit{main}|}|\\
\end{tabular}
\end{center}
%
The directive |\childdocby| is similar to |\childdocof|
described in \secref{sec:include},
but the subsequent selection of content must be done manually.
To that end, both |\ifchilddoc| and |\ifchilddocmanual|
will be true upon processing of a part,
and the name of the part is stored in |\childdocname|.
Note that |\jobname| will be set to the filename of the current part
so that each part receives an individual |.aux| file
that does not interfere with the |.aux| file(s) of the main document.
This behaviour can be altered by the alternative form
|\childdocby[*]{|\textit{main}|}| (with a non-empty optional argument)
which uses the |.aux| file of the main document
by setting |\jobname| to \textit{main}.

%%%%%%%%%%%%%%%%%%%%%%%%%%%%%%%%%%%%%%%%%%%%%%%%%%%%%%%%%%%%%%%%%%%%%%%%%%%%%%%%
\subsection{Driver Development}
\label{sec:driver}

The \textsf{childdoc} mechanism can also be use for the development
of definition files such as \LaTeX{} styles or classes.
This case differs from the above setup with multiple parts
included by |\include| in that no |\includeonly| should be invoked.
This can be achieved by starting the include file
(before |\ProvidesPackage|) with:
%
\begin{center}
\begin{tabular}{l}
|\input{childdoc.def}|\\
|\childdocforward{|\textit{main}|}|\\
\end{tabular}
\end{center}
%
or alternatively with:
%
\begin{center}
\begin{tabular}{l}
|\input{childdoc.def}|\\
|\childdocby{|\textit{main}|}|\\
\end{tabular}
\end{center}
%
Both forms have slightly different effects as described above.
The main file is prepared as usual, see \secref{sec:include}.

%%%%%%%%%%%%%%%%%%%%%%%%%%%%%%%%%%%%%%%%%%%%%%%%%%%%%%%%%%%%%%%%%%%%%%%%%%%%%%%%
\subsection{Legacy Detection}
\label{sec:detection}

The directive |\childdocmain| in the main file can detect
whether the complete document or merely a child is to be compiled
even without using the directive |\childdocof|.
This method is deprecated because it is less robust
and there is no compelling reason to use it;
it is merely provided for backward compatibility
and it may be removed in future versions.

If the detection mechanism is to be used,
it is mandatory to correctly specify
the filename of the main file as the argument of |\childdocmain|:
%
\begin{center}
\begin{tabular}{l}
|\input{childdoc.def}|\\
|\childdocmain{|\textit{main}|}|\\
\end{tabular}
\end{center}
%
If |\jobname| does not match the argument \textit{main} of |\childdocmain|,
it is assumed that |\jobname| points to the child file to be compiled.
When using |\childdocmain| with the main file specified as argument,
it suffices to start a child file
with just |\input{|\textit{main}|}|
without loading of the package and using |\childdocof|.
If instead all processing is done
with the appropriate \textsf{childdoc} directives,
the argument of \textit{main} of |\childdocmain| can be empty.

An alternative version of the command line processing described
in \secref{sec:commandline} using the detection mechanism reads:
%
\begin{center}
|... -jobname "|\textit{target}|" "|[\textit{flags}]%
[|\def\jobname{|\textit{dest}|}|]|\input{|\textit{main}|}"|
\end{center}

%%%%%%%%%%%%%%%%%%%%%%%%%%%%%%%%%%%%%%%%%%%%%%%%%%%%%%%%%%%%%%%%%%%%%%%%%%%%%%%%
\subsection{Manual Code}
\label{sec:manual}

In case one cannot be certain whether the definitions file |childdoc.def|
is installed on the target \TeX{} distribution
and one prefers not to ship it,
it is conceivable to paste a few relevant commands into the sources.

To that end, drop all statements |\input{childdoc.def}|
and perform the replacements as outlined below.
Instead of |\childdocmain{|\textit{main}|}| add the following code
to the top of the main file:
%
\begin{center}
\begin{tabular}{l}
|\||ifdefined\childdocname\endinput\||fi\newif\ifchilddoc|\\
|\edef\childdocname{\scantokens\expandafter{\jobname\noexpand}}|\\
|\def\childdocmain{|\textit{main}|}\||ifx\childdocmain\childdocname\||else|\\
|\childdoctrue\includeonly{\childdocname}\let\jobname\childdocmain\||fi|\\
\end{tabular}
\end{center}
%
Instead of |\childdocof{|\textit{main}|}| just include the main file
at the top of each child file:
%
\begin{center}
|\input{|\textit{main}|}|
\end{center}
%
A simple redirection |\childdocforward{|\textit{dest}|}| is achieved by:
%
\begin{center}
|\def\jobname{|\textit{dest}|}\input{\jobname}|
\end{center}
%
The redirection with prefix
|\childdocforwardprefix[|\textit{prefix}|]{|\textit{dest}|}|
is accomplished by:
%
\begin{center}
\begin{tabular}{l}
|{\edef\jobname{\scantokens\expandafter{\jobname\noexpand}}|\\
|\def\redirectjob |\textit{prefix}|#1~~~{\gdef\jobname{|\textit{dest}|#1}}|\\
|\expandafter\redirectjob\jobname~~~}\input{\jobname}|
\end{tabular}
\end{center}

In an alternative approach,
child documents can be compiled by a specific command line
without additional code or specific definitions:
%
\begin{center}
|... -jobname "|\textit{target}|" "|[\textit{flags}]%
|\includeonly{|\textit{dest}|}\input{|\textit{main}|}"|
\end{center}
%

%%%%%%%%%%%%%%%%%%%%%%%%%%%%%%%%%%%%%%%%%%%%%%%%%%%%%%%%%%%%%%%%%%%%%%%%%%%%%%%%
%%%%%%%%%%%%%%%%%%%%%%%%%%%%%%%%%%%%%%%%%%%%%%%%%%%%%%%%%%%%%%%%%%%%%%%%%%%%%%%%
\section{Information}

%%%%%%%%%%%%%%%%%%%%%%%%%%%%%%%%%%%%%%%%%%%%%%%%%%%%%%%%%%%%%%%%%%%%%%%%%%%%%%%%
\subsection{Copyright}

Copyright \copyright{} 2017--2018 Niklas Beisert

This work may be distributed and/or modified under the
conditions of the \LaTeX{} Project Public License, either version 1.3
of this license or (at your option) any later version.
The latest version of this license is in
  \url{http://www.latex-project.org/lppl.txt}
and version 1.3 or later is part of all distributions of \LaTeX{}
version 2005/12/01 or later.

This work has the LPPL maintenance status `maintained'.

The Current Maintainer of this work is Niklas Beisert.

This work consists of the files |README.txt|, |childdoc.ins| and |childdoc.dtx|
as well as the derived files |childdoc.def|, |cdocsamp.tex|
with |cdocsch1.tex|, |cdocsch2.tex|, |cdocspt3.tex|, |cdocspt4.tex|,
|cdocsdrf.tex|, |cdocsfn1.tex|, |cdocsfn2.tex|
as well as |childdoc.pdf|.

%%%%%%%%%%%%%%%%%%%%%%%%%%%%%%%%%%%%%%%%%%%%%%%%%%%%%%%%%%%%%%%%%%%%%%%%%%%%%%%%
\subsection{Files and Installation}

The package consists of the files:
%
\begin{center}
\begin{tabular}{ll}
    |README.txt|   & readme file \\
    |childdoc.ins| & installation file \\
    |childdoc.dtx| & source file \\
    |childdoc.def| & definition file \\
    |cdocsamp.tex| & sample main file \\
    |cdocsch1.tex| & sample include file \\
    |cdocsch2.tex| & sample include file \\
    |cdocspt3.tex| & sample part file \\
    |cdocspt4.tex| & sample part file \\
    |cdocsdrf.tex| & sample redirection file \\
    |cdocsfn1.tex| & sample redirection file \\
    |cdocsfn2.tex| & sample redirection file \\
    |childdoc.pdf| & manual
\end{tabular}
\end{center}
%
The distribution consists of the files
|README.txt|, |childdoc.ins| and |childdoc.dtx|.
%
\begin{itemize}
\item
Run (pdf)\LaTeX{} on |childdoc.dtx|
to compile the manual |childdoc.pdf| (this file).
\item
Run \LaTeX{} on |childdoc.ins| to create the definitions file |childdoc.def|
and the sample |cdocsamp.tex| with include files
|cdocsch1.tex|, |cdocsch2.tex|, |cdocspt3.tex|, |cdocspt4.tex|,
|cdocsdrf.tex|, |cdocsfn1.tex|, |cdocsfn2.tex|.
Then copy the file |childdoc.def| to an appropriate directory of your \LaTeX{}
distribution, e.g.\ \textit{texmf-root}|/tex/latex/childdoc|.
\end{itemize}

%%%%%%%%%%%%%%%%%%%%%%%%%%%%%%%%%%%%%%%%%%%%%%%%%%%%%%%%%%%%%%%%%%%%%%%%%%%%%%%%
\subsection{Related CTAN Packages}

There are several other packages which offer a similar functionality:
%
\begin{itemize}
\item
The packages
\href{http://ctan.org/pkg/docmute}{\textsf{docmute}},
\href{http://ctan.org/pkg/includex}{\textsf{includex}} and
\href{http://ctan.org/pkg/standalone}{\textsf{standalone}}
provide commands to include only the document body of
a child file thus allowing both files to be compiled individually.
\item
The packages \href{http://ctan.org/pkg/subdocs}{\textsf{subdocs}}
and \href{http://ctan.org/pkg/subfiles}{\textsf{subfiles}}
provide structures in which the main and child documents can be
encapsulated and allowing them to be compiled individually.
The inclusion mechanism is different from the conventional |\include|.
\item
The package \href{http://ctan.org/pkg/combine}{\textsf{combine}}
is an elaborate solution to combine several documents into one.
\end{itemize}
%
See also the CTAN topic \href{http://ctan.org/topic/subdocs}{\textsf{subdocs}}
for further related packages.
The present package differs from the above solutions in that
a document structure constructed with the conventional |\include| mechanism
just needs two extra commands at the top of every file
such that all constituent files can be compiled individually.

%%%%%%%%%%%%%%%%%%%%%%%%%%%%%%%%%%%%%%%%%%%%%%%%%%%%%%%%%%%%%%%%%%%%%%%%%%%%%%%%
%\subsection{Feature Suggestions}
%
%The following is a list of features which may be useful for future
%versions of this package:
%%
%\begin{itemize}
%\item
%\ldots
%\end{itemize}

%%%%%%%%%%%%%%%%%%%%%%%%%%%%%%%%%%%%%%%%%%%%%%%%%%%%%%%%%%%%%%%%%%%%%%%%%%%%%%%%
\subsection{Revision History}

%%%%%%%%%%%%%%%%%%%%%%%%%%%%%%%%%%%%%%%%
\paragraph{v2.0:} 2018/12/30

\begin{itemize}
\item
immediate forward processing
\item
added |\childdocby| mechanism
\item
manual restructured
\end{itemize}

%%%%%%%%%%%%%%%%%%%%%%%%%%%%%%%%%%%%%%%%
\paragraph{v1.6:} 2018/01/17

\begin{itemize}
\item
application for development of include files
\item
corrections to manual
\end{itemize}

%%%%%%%%%%%%%%%%%%%%%%%%%%%%%%%%%%%%%%%%
\paragraph{v1.5:} 2017/05/21

\begin{itemize}
\item
more complete structuring introduced
\item
|\childdocof| introduced
\item
|\childdoc| renamed to |\childdocmain|
\item
|\childredirect| renamed to |\childdocforward| and |\childdocforwardprefix|
and functionality expanded
\end{itemize}

%%%%%%%%%%%%%%%%%%%%%%%%%%%%%%%%%%%%%%%%
\paragraph{v1.0:} 2017/04/27

\begin{itemize}
\item
manual and install package
\item
first version published on CTAN
\end{itemize}

%%%%%%%%%%%%%%%%%%%%%%%%%%%%%%%%%%%%%%%%
\paragraph{v0.6:} 2017/04/26

\begin{itemize}
\item
redirection mechanism added
\end{itemize}

%%%%%%%%%%%%%%%%%%%%%%%%%%%%%%%%%%%%%%%%
\paragraph{v0.5:} 2017/04/26

\begin{itemize}
\item
functionality in definition file
\end{itemize}


%%%%%%%%%%%%%%%%%%%%%%%%%%%%%%%%%%%%%%%%%%%%%%%%%%%%%%%%%%%%%%%%%%%%%%%%%%%%%%%%
%%%%%%%%%%%%%%%%%%%%%%%%%%%%%%%%%%%%%%%%%%%%%%%%%%%%%%%%%%%%%%%%%%%%%%%%%%%%%%%%
%%%%%%%%%%%%%%%%%%%%%%%%%%%%%%%%%%%%%%%%%%%%%%%%%%%%%%%%%%%%%%%%%%%%%%%%%%%%%%%%
\appendix

\settowidth\MacroIndent{\rmfamily\scriptsize 000\ }

 \DocInput{childdoc.dtx}

\end{document}
%</driver>
% \fi
%
% %%%%%%%%%%%%%%%%%%%%%%%%%%%%%%%%%%%%%%%%%%%%%%%%%%%%%%%%%%%%%%%%%%%%%%%%%%%%%%
% %%%%%%%%%%%%%%%%%%%%%%%%%%%%%%%%%%%%%%%%%%%%%%%%%%%%%%%%%%%%%%%%%%%%%%%%%%%%%%
% \section{Sample}
%\iffalse
%<*samplemain>
%\fi
%
% The following presents a sample document
% with two chapters, two parts, a title page,
% a compile flag as well as three forwarding files to set the flag.
% It consists of eight |.tex| files:
% \begin{center}
% \begin{tabular}{ll}
% |cdocsamp.tex|&main file\\
% |cdocsch1.tex|&include file for chapter 1\\
% |cdocsch2.tex|&include file for chapter 2\\
% |cdocspt3.tex|&include file for part 3\\
% |cdocspt4.tex|&include file for part 4\\
% |cdocsdrf.tex|&forwarding file for main file in draft mode\\
% |cdocsfi1.tex|&forwarding file for final version of chapter 1\\
% |cdocsfi2.tex|&forwarding file for final version of chapter 2\\
% \end{tabular}
% \end{center}
% Each of the eight files can be compiled directly by the \LaTeX{} compiler.
%
% %%%%%%%%%%%%%%%%%%%%%%%%%%%%%%%%%%%%%%
% \paragraph{Main File.}
%
% The main file is called |cdocsamp.tex|.
%
% Load the \textsf{childdoc} definitions and
% declare the filename for the main document:
%    \begin{macrocode}
\input{childdoc.def}
\childdocmain{}
%    \end{macrocode}

% Optional override for |\version| flag:
%    \begin{macrocode}
%%\ifchilddoc\else\providecommand{\version}{draft}\fi
%    \end{macrocode}

% Define the default values for the |\version| flag
% (|final| for the main file and |draft| for childs):
%    \begin{macrocode}
\ifchilddoc
\providecommand{\version}{draft}
\else
\providecommand{\version}{final}
\fi
%    \end{macrocode}

% Load the standard document class:
%    \begin{macrocode}
\documentclass[12pt]{article}
%    \end{macrocode}

% Start the document body:
%    \begin{macrocode}
\begin{document}
%    \end{macrocode}

% Declare a title page.
% Print title, part of document being processed and version flag:
%    \begin{macrocode}
\addtocounter{page}{-1}
\begin{center}
{\LARGE\bfseries{}childdoc example\par}
\vspace{1cm}
\ifchilddoc
\ifchilddocmanual part\else chapter\fi:
`\childdocname' of `\childdocjob'\par
\else
main document: `\childdocjob'\par
\fi
version: \version\par
\end{center}
\newpage
%    \end{macrocode}

% Manually include selected file,
% otherwise process as usual:
%    \begin{macrocode}
\ifchilddocmanual
\section*{part `\childdocname'}
\input{\childdocname}
\else
%    \end{macrocode}

% Include the two chapters:
%    \begin{macrocode}
\include{cdocsch1}
\include{cdocsch2}
%    \end{macrocode}

% Include the two parts unless only chapters should be displayed:
%    \begin{macrocode}
\ifchilddoc\else
\section{part three}
\input{cdocspt3}
\section{part four}
\input{cdocspt4}
\fi
%    \end{macrocode}

% Process as usual until here:
%    \begin{macrocode}
\fi
%    \end{macrocode}

% End of document body:
%    \begin{macrocode}
\end{document}
%    \end{macrocode}
%\iffalse
%</samplemain>
%\fi
%
% %%%%%%%%%%%%%%%%%%%%%%%%%%%%%%%%%%%%%%
% \paragraph{Chapter Include Files.}
%
% The include files are called |cdocsch1.tex| and |cdocsch2.tex|.
%
%\iffalse
%<*samplechap1|samplechap2>
%\fi

% Optional override for |\version| flag:
%    \begin{macrocode}
%%\providecommand{\version}{final}
%    \end{macrocode}

% Include the main document:
%    \begin{macrocode}
\input{childdoc.def}
\childdocof{cdocsamp}
%    \end{macrocode}

%\iffalse
%</samplechap1|samplechap2>
%\fi
%
%\iffalse
%<*samplechap1>
%\fi
% Some text for chapter 1:
%    \begin{macrocode}
\section{one}
some text in chapter one
%    \end{macrocode}

%\iffalse
%</samplechap1>
%\fi
% Some text for chapter 2:
%\iffalse
%<*samplechap2>
%\fi
%    \begin{macrocode}
\section{two}
more text in chapter two
%    \end{macrocode}

%\iffalse
%</samplechap2>
%\fi
%
% %%%%%%%%%%%%%%%%%%%%%%%%%%%%%%%%%%%%%%
% \paragraph{Part Include Files.}
%
% The include files are called |cdocspt3.tex| and |cdocspt4.tex|.
%
%\iffalse
%<*samplepart3|samplepart4>
%\fi

% Optional override for |\version| flag:
%    \begin{macrocode}
%%\providecommand{\version}{final}
%    \end{macrocode}

% Include the main document:
%    \begin{macrocode}
\input{childdoc.def}
\childdocby{cdocsamp}
%    \end{macrocode}

%\iffalse
%</samplepart3|samplepart4>
%\fi
%
%\iffalse
%<*samplepart3>
%\fi
% Some text for part 3:
%    \begin{macrocode}
some text in part three
%    \end{macrocode}

%\iffalse
%</samplepart3>
%\fi
% Some text for part 4:
%\iffalse
%<*samplepart4>
%\fi
%    \begin{macrocode}
more text in part four
%    \end{macrocode}

%\iffalse
%</samplepart4>
%\fi
%
% %%%%%%%%%%%%%%%%%%%%%%%%%%%%%%%%%%%%%%
% \paragraph{Forwarding for a Complete Draft.}
%
% The following forwarding file |cdocsdrf.tex|
% compiles the main document in draft mode:
%\iffalse
%<*sampledraft>
%\fi
%    \begin{macrocode}
\def\version{draft}
\input{childdoc.def}
\childdocforward{cdocsamp}
%    \end{macrocode}

%\iffalse
%</sampledraft>
%\fi
%
% %%%%%%%%%%%%%%%%%%%%%%%%%%%%%%%%%%%%%%
% \paragraph{Forwarding for Final Version of the Chapters.}
%
% The following forwarding files |cdocsfn1.tex| and |cdocsfn2.tex|
% (with identical content)
% compile the final versions of the child documents
% |cdocsch1.tex| and |cdocsch2.tex|, respectively:
%\iffalse
%<*samplefinal>
%\fi
%    \begin{macrocode}
\def\version{final}
\input{childdoc.def}
\childdocforwardprefix[cdocsamp]{cdocsfn}{cdocsch}
%    \end{macrocode}

%\iffalse
%</samplefinal>
%\fi
%
% %%%%%%%%%%%%%%%%%%%%%%%%%%%%%%%%%%%%%%
% \paragraph{Command Line Processing.}
%
% The following three command lines generate the output files
% |cdocscld|, |cdocscl1| and |cdocscl2|
% which should be identical to
% |cdocsdrf|, |cdocsch1| and |cdocsfn2|, respectively:
% \begin{center}
% \begin{tabular}{l}
% |latex -jobname cdocscld \|\\
% |  "\def\version{draft}\input{childdoc.def}\childdocforward{cdocsamp}"|\\
% |latex -jobname cdocscl1 \|\\
% |  "\input{childdoc.def}\childdocforward[cdocsamp]{cdocsch1}"|\\
% |latex -jobname cdocscl2 \|\\
% |  "\def\version{final}\input{childdoc.def}\childdocforward{cdocsch2}"|
% \end{tabular}
% \end{center}
% Note that the trailing backslash on each first line
% merely continues the input to the second line
% (for convenient cut ant paste).
% Furthermore, the command |latex| can be replaced by any
% of its alternative versions such as |pdflatex|.
%
% %%%%%%%%%%%%%%%%%%%%%%%%%%%%%%%%%%%%%%%%%%%%%%%%%%%%%%%%%%%%%%%%%%%%%%%%%%%%%%
% %%%%%%%%%%%%%%%%%%%%%%%%%%%%%%%%%%%%%%%%%%%%%%%%%%%%%%%%%%%%%%%%%%%%%%%%%%%%%%
% \section{Implementation}
%\iffalse
%<*package>
%\fi
%
% This section describes the definitions file |childdoc.def|.

% The definitions cannot be loaded using |\usepackage| or |\RequirePackage|
% which has a mechanism to prevent loading a style file more than once.
% When loading the definitions by means of |\input|
% multiple instances have to be prevented manually:
%\iffalse
%This code needs to be before the `\ProvidesFile' directive
%which is defined at the beginning of this file.
%Therefore it is also placed there and commented out here.
%</package>
%<*discard>
%\fi
%    \begin{macrocode}
\ifdefined\childdocmain\endinput\fi
%    \end{macrocode}
%\iffalse
%</discard>
%<*package>
%\fi
%
% \macro{\ifchilddoc}
% \macro{\ifchilddocmanual}
% The conditional |\ifchilddoc| tells whether a
% child (true) or main (false) document is being compiled.
% The conditional |\ifchilddocmanual| tells whether
% the |\includeonly| mechanism is used (false) or
% the selection of child files must be performed manually (true).
% The definitions initialise to false:
%    \begin{macrocode}
\newif\ifchilddoc
\newif\ifchilddocmanual
%    \end{macrocode}

% \macro{\childdocname}
% \macro{\childdocjob}
% The macro |\childdocname| stores the name of the main document
% to be compiled. The macro |\childdocjob| stores the name of
% the document on which the \LaTeX{} compiler was originally invoked.
% The content of |\jobname| cannot be compared
% to filenames specified in the source due to different catcodes.
% The following code rescans |\jobname|, stores the result
% in |\childdocname| and saves a copy in |\childdocjob|:
%    \begin{macrocode}
\edef\childdocname{\scantokens\expandafter{\jobname\noexpand}}
\let\childdocjob\childdocname
%    \end{macrocode}

% \macro{\childdocdisable}
% The macro |\childdocdisable| prevents the main file
% from being processed more than once.
% At this stage, the main document command |\childdocmain|
% is assumed to be called once again where it should do nothing.
% Any subsequent call to it should prevent
% a secondary processing of the main document
% It overwrites the forwarding commands
% |\childdocof| and |\childdocforward|
% with empty macros to prevent further inclusions of the main document:
%    \begin{macrocode}
\newcommand{\childdocdisable}
{
  \renewcommand{\childdocmain}[1]{\renewcommand{\childdocmain}[1]{\endinput}}
  \renewcommand{\childdocof}[1]{}
  \renewcommand{\childdocby}[2][]{}
  \renewcommand{\childdocforward}[2][]{}
  \renewcommand{\childdocdisable}{}
}
%    \end{macrocode}

% \macro{\childdocmain}
% The macro |\childdocmain| is to be called at the top of the main file
% with nothing or the main filename (without extension) as argument.
% First, it breaks loops.
% If the argument is not empty and does not match |\childdocname|
% (which is set by the first inclusion of |childdoc.def|),
% |\ifchilddoc| is set to true, |\includeonly| is applied to the child file
% and |\jobname| is set to the main file
% (for proper handling of |.aux| files):
%    \begin{macrocode}
\newcommand{\childdocmain}[1]
{
  \childdocdisable\childdocmain{}
  \if?#1?\else
    \begingroup
      \def\childdoctmp{#1}
      \ifx\childdoctmp\childdocname
        \def\childdoctmp{}
      \else
        \def\childdoctmp
        {
          \childdoctrue
          \includeonly{\childdocname}
          \def\childdocjob{#1}
          \def\jobname{#1}
        }
      \fi
      \expandafter
    \endgroup
    \childdoctmp
  \fi
}
%    \end{macrocode}

% \macro{\childdocof}
% The command |\childdocof| redirects
% compilation to the main file |#1|.
%    \begin{macrocode}
\newcommand{\childdocof}[1]
{
  \childdocdisable
  \childdoctrue
  \includeonly{\childdocname}
  \def\jobname{#1}
  \def\childdocjob{#1}
  \input{#1}
}
%    \end{macrocode}

% \macro{\childdocby}
% The command |\childdocby| ....
%    \begin{macrocode}
\newcommand{\childdocby}[2][]
{
  \childdocdisable
  \childdoctrue
  \childdocmanualtrue
  \if?#1?\else
    \def\jobname{#2}
  \fi
  \def\childdocjob{#2}
  \input{#2}
  \endinput
}
%    \end{macrocode}

% \macro{\childdocforward}
% The command |\childdocforward| redirects
% compilation to the main file or
% (if the optional argument is given) a child file.
% Parameters are set as if the main file
% or a child file starting with |\childdocof| was compiled.
% Then compilation is handed over to the main file:
%    \begin{macrocode}
\newcommand{\childdocforward}[2][]
{
  \begingroup
    \if?#1?
      \def\childdoctmp
      {
        \def\childdocname{#2}
        \def\childdocjob{#2}
        \def\jobname{#2}
        \input{#2}
        \endinput
      }
    \else
      \def\childdoctmp
      {
        \childdocdisable
        \def\childdocname{#2}
        \childdoctrue
        \includeonly{#2}
        \def\childdocjob{#1}
        \def\jobname{#1}
        \input{#1}
        \endinput
      }
    \fi
    \expandafter
  \endgroup
  \childdoctmp
}
%    \end{macrocode}

% \macro{\childdocforwardprefix}
% The command |\childdocforwardprefix| redirects
% compilation to the main or a child file by means of a pattern.
% The prefix |#1| in the current filename is replaced by |#2|
% and the suffix of the current filename is kept
% (it is assumed that the filename does not contain the substring `|~~~|'
% which is used as a delimiter).
% Compilation is handed over to the new file by |\childdocforward|:
%    \begin{macrocode}
\newcommand{\childdocforwardprefix}[3][]
{
  \begingroup
    \def\childdocextract #2##1~~~{\def\childdoctmp{\childdocforward[#1]{#3##1}}}
    \expandafter\childdocextract\childdocname~~~
    \expandafter
  \endgroup
  \childdoctmp
}
%    \end{macrocode}

% \macro{\childdoc}
% The deprecated macro |\childdoc| is a legacy version of |\childdocmain|:
%    \begin{macrocode}
\newcommand{\childdoc}{\childdocmain}
%    \end{macrocode}

% \macro{\childdocredirect}
% The deprecated macro |\childdocredirect| is a legacy version
% of |\childdocforward| and |\childdocforwardprefix|:
%    \begin{macrocode}
\newcommand{\childdocredirect}[2][]
{
  \begingroup
    \if?#1?
      \def\childdoctmp{\childdocforward{#2}}
    \else
      \def\childdoctmp{\childdocforwardprefix{#1}{#2}}
    \fi
    \expandafter
  \endgroup
  \childdoctmp
}
%    \end{macrocode}

%\iffalse
%</package>
%\fi
%
\endinput
|\\
|\childdocforwardprefix[|\textit{main}|]{|\textit{prefix}|}{|\textit{dest}|}|
\end{tabular}
\end{center}
%
the destination file is determined by a pattern
depending on the current file:
To make this work, the current file must be called
`{\textit{prefix}\hspace{0.2em}\textit{suffix}}'
with \textit{prefix} matching precisely the argument.
Processing is then passed on to the file
`{\textit{dest}\hspace{0.2em}\textit{suffix}}'.
Surely, the same effect is achieved by
directly specifying the
argument `{\textit{dest}\hspace{0.2em}\textit{suffix}}'
in the first form.
However, that requires to set up a different file
for each child. With the alternative form of the command
all these files can have exactly the same content
which simplifies setting them up and maintaining them.

For example, the following file |draft.tex|
with a compilation flag |\version| as described in \secref{sec:flags}
compiles the main document as a draft:
%
\begin{center}
\begin{tabular}{l}
|\def\version{draft}|\\
|% \iffalse
%
% childdoc.dtx Copyright (C) 2017-2018 Niklas Beisert
%
% This work may be distributed and/or modified under the
% conditions of the LaTeX Project Public License, either version 1.3
% of this license or (at your option) any later version.
% The latest version of this license is in
%   http://www.latex-project.org/lppl.txt
% and version 1.3 or later is part of all distributions of LaTeX
% version 2005/12/01 or later.
%
% This work has the LPPL maintenance status `maintained'.
%
% The Current Maintainer of this work is Niklas Beisert.
%
% This work consists of the files childdoc.dtx and childdoc.ins
% and the derived files childdoc.def and cdocsamp.tex with
% cdocsch1.tex, cdocsch2.tex, cdocsdrf.tex, cdocsfn1.tex, cdocsfn2.tex.
%
%<package>\ifdefined\childdocmain\endinput\fi
%<package>\ProvidesFile{childdoc.def}[2018/12/30 v2.0 child document driver]
%<samplemain>\ProvidesFile{cdocsamp.tex}[2018/12/30 v2.0 sample for childdoc]
%<*driver>
%\ProvidesFile{childdoc.drv}[2018/12/30 v2.0 childdoc reference manual file]
\PassOptionsToClass{10pt,a4paper}{article}
\documentclass{ltxdoc}

\usepackage[margin=35mm]{geometry}
\usepackage{hyperref}
\usepackage{hyperxmp}
\usepackage[usenames]{color}

\hypersetup{colorlinks=true}
\hypersetup{pdfstartview=FitH}
\hypersetup{pdfpagemode=UseNone}
\hypersetup{pdfsource={}}
\hypersetup{pdflang={en-UK}}
\hypersetup{pdfcopyright={Copyright 2017-2018 Niklas Beisert.
  This work may be distributed and/or modified under the
  conditions of the LaTeX Project Public License, either version 1.3
  of this license or (at your option) any later version.}}
\hypersetup{pdflicenseurl={http://www.latex-project.org/lppl.txt}}
\hypersetup{pdfcontactaddress={ETH Zurich, ITP, HIT K,
  Wolfgang-Pauli-Strasse 27}}
\hypersetup{pdfcontactpostcode={8093}}
\hypersetup{pdfcontactcity={Zurich}}
\hypersetup{pdfcontactcountry={Switzerland}}
\hypersetup{pdfcontactemail={nbeisert@itp.phys.ethz.ch}}
\hypersetup{pdfcontacturl={http://people.phys.ethz.ch/\xmptilde nbeisert/}}

\newcommand{\secref}[1]{\hyperref[#1]{section \ref*{#1}}}

\parskip1ex
\parindent0pt
\let\olditemize\itemize
\def\itemize{\olditemize\parskip0pt}

\begin{document}

\title{The \textsf{childdoc} Package}
\hypersetup{pdftitle={The childdoc Package}}
\author{Niklas Beisert\\[2ex]
  Institut f\"ur Theoretische Physik\\
  Eidgen\"ossische Technische Hochschule Z\"urich\\
  Wolfgang-Pauli-Strasse 27, 8093 Z\"urich, Switzerland\\[1ex]
  \href{mailto:nbeisert@itp.phys.ethz.ch}
  {\texttt{nbeisert@itp.phys.ethz.ch}}}
\hypersetup{pdfauthor={Niklas Beisert}}
\hypersetup{pdfsubject={Manual for the LaTeX2e Package childdoc}}
\date{30 December 2018, \textsf{v2.0}}
\maketitle

\begin{abstract}\noindent
\textsf{childdoc} is a \LaTeXe{} package
that enables the direct compilation
of document sections included by |\include|
to individual files.
\end{abstract}

\begingroup
\parskip0ex
\tableofcontents
\endgroup

%%%%%%%%%%%%%%%%%%%%%%%%%%%%%%%%%%%%%%%%%%%%%%%%%%%%%%%%%%%%%%%%%%%%%%%%%%%%%%%%
%%%%%%%%%%%%%%%%%%%%%%%%%%%%%%%%%%%%%%%%%%%%%%%%%%%%%%%%%%%%%%%%%%%%%%%%%%%%%%%%
\section{Introduction}

\LaTeX{} provides a mechanism to structure a large document (such as a book)
into a main file and several child files (containing the chapters)
using the |\include| command.
This mechanism is beneficial for documents
which span hundreds of pages in order to
make the source file(s) more manageable.
Moreover, compilation can be restricted to
selected child files by means of the |\includeonly| command.
The latter feature can be used to reduce the compilation time while editing
(this was significantly more useful in the earlier days of \LaTeX{})
or to generate a smaller document which is easier to navigate.
Another application of |\includeonly| is to generate
documents consisting of selected parts of the complete document.

However, there are a few drawbacks of the plain |\include| mechanism:
\begin{itemize}
\item
The child files cannot be compiled on their own,
they can only be compiled via the main file.
A naive editing environment
(such as a text editor with an option
to have the current file processed by \LaTeX)
may require one to switch to the main file before compiling;
attempting to compile the child file produces errors.
\item
The main file must be modified (each time)
to adjust the |\includeonly| command
to the present needs. This easily leaves the main file in a messy state.
\item
The generated document will always carry the filename
of the main document. This is inconvenient if
several child files are to be compiled and
to be kept for distribution.
\end{itemize}

The present package provides a simple interface
to make child files individually compilable by \LaTeX{}.
Compiling a child file then has the same effect as compiling
the main file with an |\includeonly| command
to select the appropriate child.
Moreover the generated document will carry the name of the child
rather than the main file.
This resolves all three above issues.

This feature is meant to make the editing of books,
thesis documents and lecture notes somewhat more convenient.
However, the package can also be used efficiently for
composing a series of documents (such as exercise sheets)
which are typically distributed individually.
It then assists the author in generating the individual documents
(potentially in different versions)
as well as a document containing the collected series.
Another application is in developing style files
or other kinds of included material
where compilation of the style file could redirect
to a sample or test file.

%%%%%%%%%%%%%%%%%%%%%%%%%%%%%%%%%%%%%%%%%%%%%%%%%%%%%%%%%%%%%%%%%%%%%%%%%%%%%%%%
%%%%%%%%%%%%%%%%%%%%%%%%%%%%%%%%%%%%%%%%%%%%%%%%%%%%%%%%%%%%%%%%%%%%%%%%%%%%%%%%
\section{Usage}

First of all, the package \textsf{childdoc} is \emph{not} a standard
\LaTeXe{} |.sty| style file! Therefore it needs to be invoked in
a non-standard way.

%%%%%%%%%%%%%%%%%%%%%%%%%%%%%%%%%%%%%%%%%%%%%%%%%%%%%%%%%%%%%%%%%%%%%%%%%%%%%%%%
\subsection{Included Files}
\label{sec:include}

%%%%%%%%%%%%%%%%%%%%%%%%%%%%%%%%%%%%%%%%
\DescribeMacro{\childdocmain}
To use the package, add the commands
\begin{center}
\begin{tabular}{l}
|\input{childdoc.def}|\\
|\childdocmain{}|\\
\end{tabular}
\end{center}
at the very top of the main \LaTeX{} file,
in particular \emph{before} the |\documentclass| statement!
The argument of |\childdocmain| should be left empty
(but it must be present).

%%%%%%%%%%%%%%%%%%%%%%%%%%%%%%%%%%%%%%%%
\DescribeMacro{\childdocof}
Furthermore, add the commands
\begin{center}
\begin{tabular}{l}
|\input{childdoc.def}|\\
|\childdocof{|\textit{main}|}|\\
\end{tabular}
\end{center}
at the top of every child file \textit{child}
which is included by |\include{|\textit{child}|}|
from within the main file
(or at least for those files to be compiled individually).
The argument \textit{main} must be the filename of the main file.

There are a couple of
considerations in setting up the main and child documents:

%%%%%%%%%%%%%%%%%%%%%%%%%%%%%%%%%%%%%%%%
\paragraph{Restrictions.}

Please note the following restrictions:
\begin{itemize}
\item
|\childdocmain| must be called with one argument \textit{main}
to ensure compatibility with earlier version of the package.
It must either be empty (|\childdocmain{}|)
or precisely match the filename of the main file in which it is specified.
See \secref{sec:detection} for further information.
\item
The filename \textit{main} must be specified without the |.tex| extension.
\item
The filename \textit{main} is case sensitive
(even in case-insensitive file systems)
due to internal string comparison.
\item
The argument \textit{main} should be fully expanded, it cannot be a macro.
\item
Subdirectories and special characters should be avoided in filenames.
\item
The command |\childdocmain{|\textit{main}|}| must be followed by a whitespace.
It should not be followed immediately by another command
or by a comment mark `|%|'.
This is because the \TeX{} parser reads the token immediately following
the argument of |\childdocmain| and puts it
at the beginning of every child section;
however, a white\-space is ignored.
\end{itemize}

%%%%%%%%%%%%%%%%%%%%%%%%%%%%%%%%%%%%%%%%
\paragraph{Content of Main File.}

It is advisable to place all content in the child files included by |\include|.
Any output contained in the main file will appear in all child documents
unless suppressed manually;
it cannot be suppressed automatically by the |\includeonly| directive
and thus should normally be avoided.
A method to include some content in the main file
by means of conditional processing is described in \secref{sec:conditional}.

%%%%%%%%%%%%%%%%%%%%%%%%%%%%%%%%%%%%%%%%
\paragraph{Page Numbering.}

When only a part of the document is compiled,
the appropriate numbering of pages
(as well as other status parameters)
is determined from the |.aux| files.
The latter contain information from previous passes.
However this information needs to propagate through
all intermediate child documents.
Therefore the page numbering in child documents may well
be inconsistent until the complete document is compiled at least once.

A useful (if unconventional) way to always ensure a consistent
page numbering is to restart the numbering in each child document
and denote the pages by `\textit{child}|.|\textit{page}'
where \textit{child} represents the chapter/section number of the child file.
This can be achieved by the command
|\numberwithin{page}{|\textit{child}|}|
of the \textsf{amsmath} package
where \textit{child} can be |chapter| or |section|
depending on the chosen structuring.
Alternatively, one can modify the macro |\thepage| appropriately
and reset the counter |page| at the start of each child file.

%%%%%%%%%%%%%%%%%%%%%%%%%%%%%%%%%%%%%%%%%%%%%%%%%%%%%%%%%%%%%%%%%%%%%%%%%%%%%%%%
\subsection{Conditional Processing}
\label{sec:conditional}

The package provides a mechanism to compile different versions
of a document. To customise the versions further some conditional processing
can come in handy to distinguish which version is being compiled.
The package provides two macros to describe the compilation context:

%%%%%%%%%%%%%%%%%%%%%%%%%%%%%%%%%%%%%%%%
\DescribeMacro{\ifchilddoc}
The conditional |\ifchilddoc| distinguishes between the compilation of
child documents and the main document:
%
\begin{center}
|\ifchilddoc |\textit{child-code}| |[|\||else |\textit{main-code}]| \||fi|
\end{center}

%%%%%%%%%%%%%%%%%%%%%%%%%%%%%%%%%%%%%%%%
\DescribeMacro{\childdocname}
\DescribeMacro{\childdocjob}
The macro |\childdocname| contains the filename (without extension)
of the main or child file being processed.
Note that |\childdocjob| will always contain the name of the main file.

%%%%%%%%%%%%%%%%%%%%%%%%%%%%%%%%%%%%%%%%
\paragraph{Title Page.}

Conditional processing can be used to include a title or banner page
in the main document when proper precautions are taken.
Importantly, the code in the main file should ensure that the page counter
(as well as other status parameters which are stored in the |.aux| files)
takes the same value after the conditional processing.
Otherwise the page numbers may take divergent values
depending on which part is compiled.

For example, a title page could be declared by:
%
\begin{center}
\begin{tabular}{l}
|\ifchilddoc\||else|\\
|\addtocounter{page}{-1}|\\
\textit{code for title page}\\
|\newpage|\\
|\||fi|
\end{tabular}
\end{center}
%
A banner page for the child documents can be generated by:
%
\begin{center}
\begin{tabular}{l}
|\ifchilddoc|\\
|\addtocounter{page}{-1}|\\
\textit{code for banner page}\\
|\newpage|\\
|\||fi|
\end{tabular}
\end{center}
%
Here one could write a message such as:
\begin{center}
|This is the part \childdocname{} of \childdocjob{}.|
\end{center}

%%%%%%%%%%%%%%%%%%%%%%%%%%%%%%%%%%%%%%%%%%%%%%%%%%%%%%%%%%%%%%%%%%%%%%%%%%%%%%%%
\subsection{Flags}
\label{sec:flags}

The package makes it easy to generate different versions
of the main or child documents.
To this end compilation flags can be defined
and assigned different default values.
They will be particularly useful in conjunction
with the forwarding mechanism described in \secref{sec:forward}.

For example, it may be useful to have a flag |\version|
which can be set to |draft| or |final|.
The document source will contain some conditional code
depending on the value of |\version|.
Suppose further, the flag should default to |final| for the main file
and to |draft| for child files
which is a natural assignment for editing the document.
This is achieved by placing the following code
in the preamble of the main document
(below the |\childdocmain| directive):
%
\begin{center}
\begin{tabular}{l}
|\ifchilddoc|\\
|\providecommand{\version}{draft}|\\
|\||else|\\
|\providecommand{\version}{final}|\\
|\||fi|
\end{tabular}
\end{center}
%
The definition by |\providecommand| makes sure
that previous definitions are not overwritten.
Further statements |\providecommand{\version}{...}|
can thus be added before the above code to override it.

For the main file, one might add a line
(between |\childdocmain| and the above block)
%
\begin{center}
|%\ifchilddoc\||else\providecommand{\version}{draft}\||fi|
\end{center}
%
which can be uncommented to produce a draft version.
Likewise one can add a line to the very top of a child file
(above the |\childdocof{|\textit{main}|}| directive)
%
\begin{center}
|%\providecommand{\version}{final}|
\end{center}
%
which can be uncommented to produce the final version of this child document.

%%%%%%%%%%%%%%%%%%%%%%%%%%%%%%%%%%%%%%%%%%%%%%%%%%%%%%%%%%%%%%%%%%%%%%%%%%%%%%%%
\subsection{Forwarding}
\label{sec:forward}

Different versions of the main or child documents
using compilation flags as described in \secref{sec:flags}
can be (permanently) stored in different files
for convenient compilation, viewing and distribution.
To this end, the package defines a command
to pass on compilation to a different file:

%%%%%%%%%%%%%%%%%%%%%%%%%%%%%%%%%%%%%%%%
\DescribeMacro{\childdocforward}
The command |\childdocforward| redirects processing to
another source file:
%
\begin{center}
\begin{tabular}{l}
|\input{childdoc.def}|\\
|\childdocforward[|\textit{main}|]{|\textit{dest}|}|\\
\end{tabular}
\end{center}
%
The argument \textit{dest} is the destination file
(without extension).
It should be the main file or one of the child files.
Note that further \textsf{childdoc} directives
such as |\childdocof| and |\childdocforward|
in the indicated file will be processed in this form.
The optional argument \textit{main}
passes on directly to the main file \textit{main}
while pretending to compile the child \textit{dest}.
This form behaves as if \textit{dest}
issues |\childdocof{|\textit{main}|}| right away,
and no further \textsf{childdoc} directives will be processed.

%%%%%%%%%%%%%%%%%%%%%%%%%%%%%%%%%%%%%%%%
\DescribeMacro{\...prefix}
In the alternative form |\childdocforwardprefix|,
%
\begin{center}
\begin{tabular}{l}
|\input{childdoc.def}|\\
|\childdocforwardprefix[|\textit{main}|]{|\textit{prefix}|}{|\textit{dest}|}|
\end{tabular}
\end{center}
%
the destination file is determined by a pattern
depending on the current file:
To make this work, the current file must be called
`{\textit{prefix}\hspace{0.2em}\textit{suffix}}'
with \textit{prefix} matching precisely the argument.
Processing is then passed on to the file
`{\textit{dest}\hspace{0.2em}\textit{suffix}}'.
Surely, the same effect is achieved by
directly specifying the
argument `{\textit{dest}\hspace{0.2em}\textit{suffix}}'
in the first form.
However, that requires to set up a different file
for each child. With the alternative form of the command
all these files can have exactly the same content
which simplifies setting them up and maintaining them.

For example, the following file |draft.tex|
with a compilation flag |\version| as described in \secref{sec:flags}
compiles the main document as a draft:
%
\begin{center}
\begin{tabular}{l}
|\def\version{draft}|\\
|\input{childdoc.def}|\\
|\childdocforward{|\textit{main}|}|
\end{tabular}
\end{center}
%
Likewise, the following files |final|\textit{nn}|.tex|
compile the final version of the child document
|child|\textit{nn}|.tex|:
%
\begin{center}
\begin{tabular}{l}
|\def\version{final}|\\
|\input{childdoc.def}|\\
|\childdocforwardprefix{final}{child}|
\end{tabular}
\end{center}
%

Note that when several versions of a main file and/or of each child file
are to be generated, it may be convenient to set up a |Makefile| or
shell script to automatise the process.

%%%%%%%%%%%%%%%%%%%%%%%%%%%%%%%%%%%%%%%%%%%%%%%%%%%%%%%%%%%%%%%%%%%%%%%%%%%%%%%%
\subsection{Command Line Processing}
\label{sec:commandline}

The effect of redirection files can also be achieved by invoking
the \LaTeX{} compiler with a more elaborate command line.
Most conveniently this should be done as part
of a shell script or a |Makefile|.

When using \textsf{childdoc} in the main file, the following
command lines effectively perform a redirection
(note that depending on the shell being used,
backslashes may have to be doubled: `|\|' $\to$ `|\\|'):
%
\begin{center}
|... -jobname "|\textit{target}|" |\\|"|[\textit{flags}]%
|\input{childdoc.def}\childdocforward[|\textit{main}|]{|\textit{dest}|}"|
\end{center}
%
Here \textit{target} is the name of the output file,
\textit{main} is the name of the main file
and \textit{dest} is the name of the main or child file to be processed
(all filenames without extensions).
The optional argument \textit{main} can be omitted
if \textit{main} matches \textit{dest}.
Optionally, compilation \textit{flags} can be defined via |\def| commands.
This command line makes the \TeX{} engine believe
it is compiling the file \textit{target}
whose content is specified as the latter parameter.
The provided code then forwards the processing to
\textit{main} or \textit{dest} as described in \secref{sec:forward}.

%%%%%%%%%%%%%%%%%%%%%%%%%%%%%%%%%%%%%%%%%%%%%%%%%%%%%%%%%%%%%%%%%%%%%%%%%%%%%%%%
\subsection{Include by Input}
\label{sec:input}

Including child documents by |\include| has some restrictions by design.
Most notably, the content of a child document always occupies
its own set of pages; pages cannot be shared between child documents.
Usually, this behaviour makes perfect sense
because each child document contain an essential part of the document.
However, in some situations it may be desirable to compose
a document from a collection of parts
without having mandatory page breaks between then.
For this case, the package
provides a mechanism to include parts
by |\input| which can also be processed individually.
However, by construction this mechanism
requires manual handling of the content to be output.

%%%%%%%%%%%%%%%%%%%%%%%%%%%%%%%%%%%%%%%%
\DescribeMacro{\ifchilddocmanual}
The main file should be prepared as usual, see \secref{sec:include}.
However, the document body must make a distinction
between processing of an individual part and of the main document, e.g.:
%
\begin{center}
\begin{tabular}{l}
|\ifchilddocmanual|\\
|\input{\childdocname}|\\
|\||else|\\
\textit{document body with }|\input{|\textit{part}|}|\\
|\||fi|
\end{tabular}
\end{center}
%
The conditional |\ifchilddocmanual| is true whenever
a part to be included by |\input| is being compiled,
and the name of the part is stored in |\childdocname|.

%%%%%%%%%%%%%%%%%%%%%%%%%%%%%%%%%%%%%%%%
\DescribeMacro{\childdocby}
Each part to be included by |\input| should start with:
%
\begin{center}
\begin{tabular}{l}
|\input{childdoc.def}|\\
|\childdocby{|\textit{main}|}|\\
\end{tabular}
\end{center}
%
The directive |\childdocby| is similar to |\childdocof|
described in \secref{sec:include},
but the subsequent selection of content must be done manually.
To that end, both |\ifchilddoc| and |\ifchilddocmanual|
will be true upon processing of a part,
and the name of the part is stored in |\childdocname|.
Note that |\jobname| will be set to the filename of the current part
so that each part receives an individual |.aux| file
that does not interfere with the |.aux| file(s) of the main document.
This behaviour can be altered by the alternative form
|\childdocby[*]{|\textit{main}|}| (with a non-empty optional argument)
which uses the |.aux| file of the main document
by setting |\jobname| to \textit{main}.

%%%%%%%%%%%%%%%%%%%%%%%%%%%%%%%%%%%%%%%%%%%%%%%%%%%%%%%%%%%%%%%%%%%%%%%%%%%%%%%%
\subsection{Driver Development}
\label{sec:driver}

The \textsf{childdoc} mechanism can also be use for the development
of definition files such as \LaTeX{} styles or classes.
This case differs from the above setup with multiple parts
included by |\include| in that no |\includeonly| should be invoked.
This can be achieved by starting the include file
(before |\ProvidesPackage|) with:
%
\begin{center}
\begin{tabular}{l}
|\input{childdoc.def}|\\
|\childdocforward{|\textit{main}|}|\\
\end{tabular}
\end{center}
%
or alternatively with:
%
\begin{center}
\begin{tabular}{l}
|\input{childdoc.def}|\\
|\childdocby{|\textit{main}|}|\\
\end{tabular}
\end{center}
%
Both forms have slightly different effects as described above.
The main file is prepared as usual, see \secref{sec:include}.

%%%%%%%%%%%%%%%%%%%%%%%%%%%%%%%%%%%%%%%%%%%%%%%%%%%%%%%%%%%%%%%%%%%%%%%%%%%%%%%%
\subsection{Legacy Detection}
\label{sec:detection}

The directive |\childdocmain| in the main file can detect
whether the complete document or merely a child is to be compiled
even without using the directive |\childdocof|.
This method is deprecated because it is less robust
and there is no compelling reason to use it;
it is merely provided for backward compatibility
and it may be removed in future versions.

If the detection mechanism is to be used,
it is mandatory to correctly specify
the filename of the main file as the argument of |\childdocmain|:
%
\begin{center}
\begin{tabular}{l}
|\input{childdoc.def}|\\
|\childdocmain{|\textit{main}|}|\\
\end{tabular}
\end{center}
%
If |\jobname| does not match the argument \textit{main} of |\childdocmain|,
it is assumed that |\jobname| points to the child file to be compiled.
When using |\childdocmain| with the main file specified as argument,
it suffices to start a child file
with just |\input{|\textit{main}|}|
without loading of the package and using |\childdocof|.
If instead all processing is done
with the appropriate \textsf{childdoc} directives,
the argument of \textit{main} of |\childdocmain| can be empty.

An alternative version of the command line processing described
in \secref{sec:commandline} using the detection mechanism reads:
%
\begin{center}
|... -jobname "|\textit{target}|" "|[\textit{flags}]%
[|\def\jobname{|\textit{dest}|}|]|\input{|\textit{main}|}"|
\end{center}

%%%%%%%%%%%%%%%%%%%%%%%%%%%%%%%%%%%%%%%%%%%%%%%%%%%%%%%%%%%%%%%%%%%%%%%%%%%%%%%%
\subsection{Manual Code}
\label{sec:manual}

In case one cannot be certain whether the definitions file |childdoc.def|
is installed on the target \TeX{} distribution
and one prefers not to ship it,
it is conceivable to paste a few relevant commands into the sources.

To that end, drop all statements |\input{childdoc.def}|
and perform the replacements as outlined below.
Instead of |\childdocmain{|\textit{main}|}| add the following code
to the top of the main file:
%
\begin{center}
\begin{tabular}{l}
|\||ifdefined\childdocname\endinput\||fi\newif\ifchilddoc|\\
|\edef\childdocname{\scantokens\expandafter{\jobname\noexpand}}|\\
|\def\childdocmain{|\textit{main}|}\||ifx\childdocmain\childdocname\||else|\\
|\childdoctrue\includeonly{\childdocname}\let\jobname\childdocmain\||fi|\\
\end{tabular}
\end{center}
%
Instead of |\childdocof{|\textit{main}|}| just include the main file
at the top of each child file:
%
\begin{center}
|\input{|\textit{main}|}|
\end{center}
%
A simple redirection |\childdocforward{|\textit{dest}|}| is achieved by:
%
\begin{center}
|\def\jobname{|\textit{dest}|}\input{\jobname}|
\end{center}
%
The redirection with prefix
|\childdocforwardprefix[|\textit{prefix}|]{|\textit{dest}|}|
is accomplished by:
%
\begin{center}
\begin{tabular}{l}
|{\edef\jobname{\scantokens\expandafter{\jobname\noexpand}}|\\
|\def\redirectjob |\textit{prefix}|#1~~~{\gdef\jobname{|\textit{dest}|#1}}|\\
|\expandafter\redirectjob\jobname~~~}\input{\jobname}|
\end{tabular}
\end{center}

In an alternative approach,
child documents can be compiled by a specific command line
without additional code or specific definitions:
%
\begin{center}
|... -jobname "|\textit{target}|" "|[\textit{flags}]%
|\includeonly{|\textit{dest}|}\input{|\textit{main}|}"|
\end{center}
%

%%%%%%%%%%%%%%%%%%%%%%%%%%%%%%%%%%%%%%%%%%%%%%%%%%%%%%%%%%%%%%%%%%%%%%%%%%%%%%%%
%%%%%%%%%%%%%%%%%%%%%%%%%%%%%%%%%%%%%%%%%%%%%%%%%%%%%%%%%%%%%%%%%%%%%%%%%%%%%%%%
\section{Information}

%%%%%%%%%%%%%%%%%%%%%%%%%%%%%%%%%%%%%%%%%%%%%%%%%%%%%%%%%%%%%%%%%%%%%%%%%%%%%%%%
\subsection{Copyright}

Copyright \copyright{} 2017--2018 Niklas Beisert

This work may be distributed and/or modified under the
conditions of the \LaTeX{} Project Public License, either version 1.3
of this license or (at your option) any later version.
The latest version of this license is in
  \url{http://www.latex-project.org/lppl.txt}
and version 1.3 or later is part of all distributions of \LaTeX{}
version 2005/12/01 or later.

This work has the LPPL maintenance status `maintained'.

The Current Maintainer of this work is Niklas Beisert.

This work consists of the files |README.txt|, |childdoc.ins| and |childdoc.dtx|
as well as the derived files |childdoc.def|, |cdocsamp.tex|
with |cdocsch1.tex|, |cdocsch2.tex|, |cdocspt3.tex|, |cdocspt4.tex|,
|cdocsdrf.tex|, |cdocsfn1.tex|, |cdocsfn2.tex|
as well as |childdoc.pdf|.

%%%%%%%%%%%%%%%%%%%%%%%%%%%%%%%%%%%%%%%%%%%%%%%%%%%%%%%%%%%%%%%%%%%%%%%%%%%%%%%%
\subsection{Files and Installation}

The package consists of the files:
%
\begin{center}
\begin{tabular}{ll}
    |README.txt|   & readme file \\
    |childdoc.ins| & installation file \\
    |childdoc.dtx| & source file \\
    |childdoc.def| & definition file \\
    |cdocsamp.tex| & sample main file \\
    |cdocsch1.tex| & sample include file \\
    |cdocsch2.tex| & sample include file \\
    |cdocspt3.tex| & sample part file \\
    |cdocspt4.tex| & sample part file \\
    |cdocsdrf.tex| & sample redirection file \\
    |cdocsfn1.tex| & sample redirection file \\
    |cdocsfn2.tex| & sample redirection file \\
    |childdoc.pdf| & manual
\end{tabular}
\end{center}
%
The distribution consists of the files
|README.txt|, |childdoc.ins| and |childdoc.dtx|.
%
\begin{itemize}
\item
Run (pdf)\LaTeX{} on |childdoc.dtx|
to compile the manual |childdoc.pdf| (this file).
\item
Run \LaTeX{} on |childdoc.ins| to create the definitions file |childdoc.def|
and the sample |cdocsamp.tex| with include files
|cdocsch1.tex|, |cdocsch2.tex|, |cdocspt3.tex|, |cdocspt4.tex|,
|cdocsdrf.tex|, |cdocsfn1.tex|, |cdocsfn2.tex|.
Then copy the file |childdoc.def| to an appropriate directory of your \LaTeX{}
distribution, e.g.\ \textit{texmf-root}|/tex/latex/childdoc|.
\end{itemize}

%%%%%%%%%%%%%%%%%%%%%%%%%%%%%%%%%%%%%%%%%%%%%%%%%%%%%%%%%%%%%%%%%%%%%%%%%%%%%%%%
\subsection{Related CTAN Packages}

There are several other packages which offer a similar functionality:
%
\begin{itemize}
\item
The packages
\href{http://ctan.org/pkg/docmute}{\textsf{docmute}},
\href{http://ctan.org/pkg/includex}{\textsf{includex}} and
\href{http://ctan.org/pkg/standalone}{\textsf{standalone}}
provide commands to include only the document body of
a child file thus allowing both files to be compiled individually.
\item
The packages \href{http://ctan.org/pkg/subdocs}{\textsf{subdocs}}
and \href{http://ctan.org/pkg/subfiles}{\textsf{subfiles}}
provide structures in which the main and child documents can be
encapsulated and allowing them to be compiled individually.
The inclusion mechanism is different from the conventional |\include|.
\item
The package \href{http://ctan.org/pkg/combine}{\textsf{combine}}
is an elaborate solution to combine several documents into one.
\end{itemize}
%
See also the CTAN topic \href{http://ctan.org/topic/subdocs}{\textsf{subdocs}}
for further related packages.
The present package differs from the above solutions in that
a document structure constructed with the conventional |\include| mechanism
just needs two extra commands at the top of every file
such that all constituent files can be compiled individually.

%%%%%%%%%%%%%%%%%%%%%%%%%%%%%%%%%%%%%%%%%%%%%%%%%%%%%%%%%%%%%%%%%%%%%%%%%%%%%%%%
%\subsection{Feature Suggestions}
%
%The following is a list of features which may be useful for future
%versions of this package:
%%
%\begin{itemize}
%\item
%\ldots
%\end{itemize}

%%%%%%%%%%%%%%%%%%%%%%%%%%%%%%%%%%%%%%%%%%%%%%%%%%%%%%%%%%%%%%%%%%%%%%%%%%%%%%%%
\subsection{Revision History}

%%%%%%%%%%%%%%%%%%%%%%%%%%%%%%%%%%%%%%%%
\paragraph{v2.0:} 2018/12/30

\begin{itemize}
\item
immediate forward processing
\item
added |\childdocby| mechanism
\item
manual restructured
\end{itemize}

%%%%%%%%%%%%%%%%%%%%%%%%%%%%%%%%%%%%%%%%
\paragraph{v1.6:} 2018/01/17

\begin{itemize}
\item
application for development of include files
\item
corrections to manual
\end{itemize}

%%%%%%%%%%%%%%%%%%%%%%%%%%%%%%%%%%%%%%%%
\paragraph{v1.5:} 2017/05/21

\begin{itemize}
\item
more complete structuring introduced
\item
|\childdocof| introduced
\item
|\childdoc| renamed to |\childdocmain|
\item
|\childredirect| renamed to |\childdocforward| and |\childdocforwardprefix|
and functionality expanded
\end{itemize}

%%%%%%%%%%%%%%%%%%%%%%%%%%%%%%%%%%%%%%%%
\paragraph{v1.0:} 2017/04/27

\begin{itemize}
\item
manual and install package
\item
first version published on CTAN
\end{itemize}

%%%%%%%%%%%%%%%%%%%%%%%%%%%%%%%%%%%%%%%%
\paragraph{v0.6:} 2017/04/26

\begin{itemize}
\item
redirection mechanism added
\end{itemize}

%%%%%%%%%%%%%%%%%%%%%%%%%%%%%%%%%%%%%%%%
\paragraph{v0.5:} 2017/04/26

\begin{itemize}
\item
functionality in definition file
\end{itemize}


%%%%%%%%%%%%%%%%%%%%%%%%%%%%%%%%%%%%%%%%%%%%%%%%%%%%%%%%%%%%%%%%%%%%%%%%%%%%%%%%
%%%%%%%%%%%%%%%%%%%%%%%%%%%%%%%%%%%%%%%%%%%%%%%%%%%%%%%%%%%%%%%%%%%%%%%%%%%%%%%%
%%%%%%%%%%%%%%%%%%%%%%%%%%%%%%%%%%%%%%%%%%%%%%%%%%%%%%%%%%%%%%%%%%%%%%%%%%%%%%%%
\appendix

\settowidth\MacroIndent{\rmfamily\scriptsize 000\ }

 \DocInput{childdoc.dtx}

\end{document}
%</driver>
% \fi
%
% %%%%%%%%%%%%%%%%%%%%%%%%%%%%%%%%%%%%%%%%%%%%%%%%%%%%%%%%%%%%%%%%%%%%%%%%%%%%%%
% %%%%%%%%%%%%%%%%%%%%%%%%%%%%%%%%%%%%%%%%%%%%%%%%%%%%%%%%%%%%%%%%%%%%%%%%%%%%%%
% \section{Sample}
%\iffalse
%<*samplemain>
%\fi
%
% The following presents a sample document
% with two chapters, two parts, a title page,
% a compile flag as well as three forwarding files to set the flag.
% It consists of eight |.tex| files:
% \begin{center}
% \begin{tabular}{ll}
% |cdocsamp.tex|&main file\\
% |cdocsch1.tex|&include file for chapter 1\\
% |cdocsch2.tex|&include file for chapter 2\\
% |cdocspt3.tex|&include file for part 3\\
% |cdocspt4.tex|&include file for part 4\\
% |cdocsdrf.tex|&forwarding file for main file in draft mode\\
% |cdocsfi1.tex|&forwarding file for final version of chapter 1\\
% |cdocsfi2.tex|&forwarding file for final version of chapter 2\\
% \end{tabular}
% \end{center}
% Each of the eight files can be compiled directly by the \LaTeX{} compiler.
%
% %%%%%%%%%%%%%%%%%%%%%%%%%%%%%%%%%%%%%%
% \paragraph{Main File.}
%
% The main file is called |cdocsamp.tex|.
%
% Load the \textsf{childdoc} definitions and
% declare the filename for the main document:
%    \begin{macrocode}
\input{childdoc.def}
\childdocmain{}
%    \end{macrocode}

% Optional override for |\version| flag:
%    \begin{macrocode}
%%\ifchilddoc\else\providecommand{\version}{draft}\fi
%    \end{macrocode}

% Define the default values for the |\version| flag
% (|final| for the main file and |draft| for childs):
%    \begin{macrocode}
\ifchilddoc
\providecommand{\version}{draft}
\else
\providecommand{\version}{final}
\fi
%    \end{macrocode}

% Load the standard document class:
%    \begin{macrocode}
\documentclass[12pt]{article}
%    \end{macrocode}

% Start the document body:
%    \begin{macrocode}
\begin{document}
%    \end{macrocode}

% Declare a title page.
% Print title, part of document being processed and version flag:
%    \begin{macrocode}
\addtocounter{page}{-1}
\begin{center}
{\LARGE\bfseries{}childdoc example\par}
\vspace{1cm}
\ifchilddoc
\ifchilddocmanual part\else chapter\fi:
`\childdocname' of `\childdocjob'\par
\else
main document: `\childdocjob'\par
\fi
version: \version\par
\end{center}
\newpage
%    \end{macrocode}

% Manually include selected file,
% otherwise process as usual:
%    \begin{macrocode}
\ifchilddocmanual
\section*{part `\childdocname'}
\input{\childdocname}
\else
%    \end{macrocode}

% Include the two chapters:
%    \begin{macrocode}
\include{cdocsch1}
\include{cdocsch2}
%    \end{macrocode}

% Include the two parts unless only chapters should be displayed:
%    \begin{macrocode}
\ifchilddoc\else
\section{part three}
\input{cdocspt3}
\section{part four}
\input{cdocspt4}
\fi
%    \end{macrocode}

% Process as usual until here:
%    \begin{macrocode}
\fi
%    \end{macrocode}

% End of document body:
%    \begin{macrocode}
\end{document}
%    \end{macrocode}
%\iffalse
%</samplemain>
%\fi
%
% %%%%%%%%%%%%%%%%%%%%%%%%%%%%%%%%%%%%%%
% \paragraph{Chapter Include Files.}
%
% The include files are called |cdocsch1.tex| and |cdocsch2.tex|.
%
%\iffalse
%<*samplechap1|samplechap2>
%\fi

% Optional override for |\version| flag:
%    \begin{macrocode}
%%\providecommand{\version}{final}
%    \end{macrocode}

% Include the main document:
%    \begin{macrocode}
\input{childdoc.def}
\childdocof{cdocsamp}
%    \end{macrocode}

%\iffalse
%</samplechap1|samplechap2>
%\fi
%
%\iffalse
%<*samplechap1>
%\fi
% Some text for chapter 1:
%    \begin{macrocode}
\section{one}
some text in chapter one
%    \end{macrocode}

%\iffalse
%</samplechap1>
%\fi
% Some text for chapter 2:
%\iffalse
%<*samplechap2>
%\fi
%    \begin{macrocode}
\section{two}
more text in chapter two
%    \end{macrocode}

%\iffalse
%</samplechap2>
%\fi
%
% %%%%%%%%%%%%%%%%%%%%%%%%%%%%%%%%%%%%%%
% \paragraph{Part Include Files.}
%
% The include files are called |cdocspt3.tex| and |cdocspt4.tex|.
%
%\iffalse
%<*samplepart3|samplepart4>
%\fi

% Optional override for |\version| flag:
%    \begin{macrocode}
%%\providecommand{\version}{final}
%    \end{macrocode}

% Include the main document:
%    \begin{macrocode}
\input{childdoc.def}
\childdocby{cdocsamp}
%    \end{macrocode}

%\iffalse
%</samplepart3|samplepart4>
%\fi
%
%\iffalse
%<*samplepart3>
%\fi
% Some text for part 3:
%    \begin{macrocode}
some text in part three
%    \end{macrocode}

%\iffalse
%</samplepart3>
%\fi
% Some text for part 4:
%\iffalse
%<*samplepart4>
%\fi
%    \begin{macrocode}
more text in part four
%    \end{macrocode}

%\iffalse
%</samplepart4>
%\fi
%
% %%%%%%%%%%%%%%%%%%%%%%%%%%%%%%%%%%%%%%
% \paragraph{Forwarding for a Complete Draft.}
%
% The following forwarding file |cdocsdrf.tex|
% compiles the main document in draft mode:
%\iffalse
%<*sampledraft>
%\fi
%    \begin{macrocode}
\def\version{draft}
\input{childdoc.def}
\childdocforward{cdocsamp}
%    \end{macrocode}

%\iffalse
%</sampledraft>
%\fi
%
% %%%%%%%%%%%%%%%%%%%%%%%%%%%%%%%%%%%%%%
% \paragraph{Forwarding for Final Version of the Chapters.}
%
% The following forwarding files |cdocsfn1.tex| and |cdocsfn2.tex|
% (with identical content)
% compile the final versions of the child documents
% |cdocsch1.tex| and |cdocsch2.tex|, respectively:
%\iffalse
%<*samplefinal>
%\fi
%    \begin{macrocode}
\def\version{final}
\input{childdoc.def}
\childdocforwardprefix[cdocsamp]{cdocsfn}{cdocsch}
%    \end{macrocode}

%\iffalse
%</samplefinal>
%\fi
%
% %%%%%%%%%%%%%%%%%%%%%%%%%%%%%%%%%%%%%%
% \paragraph{Command Line Processing.}
%
% The following three command lines generate the output files
% |cdocscld|, |cdocscl1| and |cdocscl2|
% which should be identical to
% |cdocsdrf|, |cdocsch1| and |cdocsfn2|, respectively:
% \begin{center}
% \begin{tabular}{l}
% |latex -jobname cdocscld \|\\
% |  "\def\version{draft}\input{childdoc.def}\childdocforward{cdocsamp}"|\\
% |latex -jobname cdocscl1 \|\\
% |  "\input{childdoc.def}\childdocforward[cdocsamp]{cdocsch1}"|\\
% |latex -jobname cdocscl2 \|\\
% |  "\def\version{final}\input{childdoc.def}\childdocforward{cdocsch2}"|
% \end{tabular}
% \end{center}
% Note that the trailing backslash on each first line
% merely continues the input to the second line
% (for convenient cut ant paste).
% Furthermore, the command |latex| can be replaced by any
% of its alternative versions such as |pdflatex|.
%
% %%%%%%%%%%%%%%%%%%%%%%%%%%%%%%%%%%%%%%%%%%%%%%%%%%%%%%%%%%%%%%%%%%%%%%%%%%%%%%
% %%%%%%%%%%%%%%%%%%%%%%%%%%%%%%%%%%%%%%%%%%%%%%%%%%%%%%%%%%%%%%%%%%%%%%%%%%%%%%
% \section{Implementation}
%\iffalse
%<*package>
%\fi
%
% This section describes the definitions file |childdoc.def|.

% The definitions cannot be loaded using |\usepackage| or |\RequirePackage|
% which has a mechanism to prevent loading a style file more than once.
% When loading the definitions by means of |\input|
% multiple instances have to be prevented manually:
%\iffalse
%This code needs to be before the `\ProvidesFile' directive
%which is defined at the beginning of this file.
%Therefore it is also placed there and commented out here.
%</package>
%<*discard>
%\fi
%    \begin{macrocode}
\ifdefined\childdocmain\endinput\fi
%    \end{macrocode}
%\iffalse
%</discard>
%<*package>
%\fi
%
% \macro{\ifchilddoc}
% \macro{\ifchilddocmanual}
% The conditional |\ifchilddoc| tells whether a
% child (true) or main (false) document is being compiled.
% The conditional |\ifchilddocmanual| tells whether
% the |\includeonly| mechanism is used (false) or
% the selection of child files must be performed manually (true).
% The definitions initialise to false:
%    \begin{macrocode}
\newif\ifchilddoc
\newif\ifchilddocmanual
%    \end{macrocode}

% \macro{\childdocname}
% \macro{\childdocjob}
% The macro |\childdocname| stores the name of the main document
% to be compiled. The macro |\childdocjob| stores the name of
% the document on which the \LaTeX{} compiler was originally invoked.
% The content of |\jobname| cannot be compared
% to filenames specified in the source due to different catcodes.
% The following code rescans |\jobname|, stores the result
% in |\childdocname| and saves a copy in |\childdocjob|:
%    \begin{macrocode}
\edef\childdocname{\scantokens\expandafter{\jobname\noexpand}}
\let\childdocjob\childdocname
%    \end{macrocode}

% \macro{\childdocdisable}
% The macro |\childdocdisable| prevents the main file
% from being processed more than once.
% At this stage, the main document command |\childdocmain|
% is assumed to be called once again where it should do nothing.
% Any subsequent call to it should prevent
% a secondary processing of the main document
% It overwrites the forwarding commands
% |\childdocof| and |\childdocforward|
% with empty macros to prevent further inclusions of the main document:
%    \begin{macrocode}
\newcommand{\childdocdisable}
{
  \renewcommand{\childdocmain}[1]{\renewcommand{\childdocmain}[1]{\endinput}}
  \renewcommand{\childdocof}[1]{}
  \renewcommand{\childdocby}[2][]{}
  \renewcommand{\childdocforward}[2][]{}
  \renewcommand{\childdocdisable}{}
}
%    \end{macrocode}

% \macro{\childdocmain}
% The macro |\childdocmain| is to be called at the top of the main file
% with nothing or the main filename (without extension) as argument.
% First, it breaks loops.
% If the argument is not empty and does not match |\childdocname|
% (which is set by the first inclusion of |childdoc.def|),
% |\ifchilddoc| is set to true, |\includeonly| is applied to the child file
% and |\jobname| is set to the main file
% (for proper handling of |.aux| files):
%    \begin{macrocode}
\newcommand{\childdocmain}[1]
{
  \childdocdisable\childdocmain{}
  \if?#1?\else
    \begingroup
      \def\childdoctmp{#1}
      \ifx\childdoctmp\childdocname
        \def\childdoctmp{}
      \else
        \def\childdoctmp
        {
          \childdoctrue
          \includeonly{\childdocname}
          \def\childdocjob{#1}
          \def\jobname{#1}
        }
      \fi
      \expandafter
    \endgroup
    \childdoctmp
  \fi
}
%    \end{macrocode}

% \macro{\childdocof}
% The command |\childdocof| redirects
% compilation to the main file |#1|.
%    \begin{macrocode}
\newcommand{\childdocof}[1]
{
  \childdocdisable
  \childdoctrue
  \includeonly{\childdocname}
  \def\jobname{#1}
  \def\childdocjob{#1}
  \input{#1}
}
%    \end{macrocode}

% \macro{\childdocby}
% The command |\childdocby| ....
%    \begin{macrocode}
\newcommand{\childdocby}[2][]
{
  \childdocdisable
  \childdoctrue
  \childdocmanualtrue
  \if?#1?\else
    \def\jobname{#2}
  \fi
  \def\childdocjob{#2}
  \input{#2}
  \endinput
}
%    \end{macrocode}

% \macro{\childdocforward}
% The command |\childdocforward| redirects
% compilation to the main file or
% (if the optional argument is given) a child file.
% Parameters are set as if the main file
% or a child file starting with |\childdocof| was compiled.
% Then compilation is handed over to the main file:
%    \begin{macrocode}
\newcommand{\childdocforward}[2][]
{
  \begingroup
    \if?#1?
      \def\childdoctmp
      {
        \def\childdocname{#2}
        \def\childdocjob{#2}
        \def\jobname{#2}
        \input{#2}
        \endinput
      }
    \else
      \def\childdoctmp
      {
        \childdocdisable
        \def\childdocname{#2}
        \childdoctrue
        \includeonly{#2}
        \def\childdocjob{#1}
        \def\jobname{#1}
        \input{#1}
        \endinput
      }
    \fi
    \expandafter
  \endgroup
  \childdoctmp
}
%    \end{macrocode}

% \macro{\childdocforwardprefix}
% The command |\childdocforwardprefix| redirects
% compilation to the main or a child file by means of a pattern.
% The prefix |#1| in the current filename is replaced by |#2|
% and the suffix of the current filename is kept
% (it is assumed that the filename does not contain the substring `|~~~|'
% which is used as a delimiter).
% Compilation is handed over to the new file by |\childdocforward|:
%    \begin{macrocode}
\newcommand{\childdocforwardprefix}[3][]
{
  \begingroup
    \def\childdocextract #2##1~~~{\def\childdoctmp{\childdocforward[#1]{#3##1}}}
    \expandafter\childdocextract\childdocname~~~
    \expandafter
  \endgroup
  \childdoctmp
}
%    \end{macrocode}

% \macro{\childdoc}
% The deprecated macro |\childdoc| is a legacy version of |\childdocmain|:
%    \begin{macrocode}
\newcommand{\childdoc}{\childdocmain}
%    \end{macrocode}

% \macro{\childdocredirect}
% The deprecated macro |\childdocredirect| is a legacy version
% of |\childdocforward| and |\childdocforwardprefix|:
%    \begin{macrocode}
\newcommand{\childdocredirect}[2][]
{
  \begingroup
    \if?#1?
      \def\childdoctmp{\childdocforward{#2}}
    \else
      \def\childdoctmp{\childdocforwardprefix{#1}{#2}}
    \fi
    \expandafter
  \endgroup
  \childdoctmp
}
%    \end{macrocode}

%\iffalse
%</package>
%\fi
%
\endinput
|\\
|\childdocforward{|\textit{main}|}|
\end{tabular}
\end{center}
%
Likewise, the following files |final|\textit{nn}|.tex|
compile the final version of the child document
|child|\textit{nn}|.tex|:
%
\begin{center}
\begin{tabular}{l}
|\def\version{final}|\\
|% \iffalse
%
% childdoc.dtx Copyright (C) 2017-2018 Niklas Beisert
%
% This work may be distributed and/or modified under the
% conditions of the LaTeX Project Public License, either version 1.3
% of this license or (at your option) any later version.
% The latest version of this license is in
%   http://www.latex-project.org/lppl.txt
% and version 1.3 or later is part of all distributions of LaTeX
% version 2005/12/01 or later.
%
% This work has the LPPL maintenance status `maintained'.
%
% The Current Maintainer of this work is Niklas Beisert.
%
% This work consists of the files childdoc.dtx and childdoc.ins
% and the derived files childdoc.def and cdocsamp.tex with
% cdocsch1.tex, cdocsch2.tex, cdocsdrf.tex, cdocsfn1.tex, cdocsfn2.tex.
%
%<package>\ifdefined\childdocmain\endinput\fi
%<package>\ProvidesFile{childdoc.def}[2018/12/30 v2.0 child document driver]
%<samplemain>\ProvidesFile{cdocsamp.tex}[2018/12/30 v2.0 sample for childdoc]
%<*driver>
%\ProvidesFile{childdoc.drv}[2018/12/30 v2.0 childdoc reference manual file]
\PassOptionsToClass{10pt,a4paper}{article}
\documentclass{ltxdoc}

\usepackage[margin=35mm]{geometry}
\usepackage{hyperref}
\usepackage{hyperxmp}
\usepackage[usenames]{color}

\hypersetup{colorlinks=true}
\hypersetup{pdfstartview=FitH}
\hypersetup{pdfpagemode=UseNone}
\hypersetup{pdfsource={}}
\hypersetup{pdflang={en-UK}}
\hypersetup{pdfcopyright={Copyright 2017-2018 Niklas Beisert.
  This work may be distributed and/or modified under the
  conditions of the LaTeX Project Public License, either version 1.3
  of this license or (at your option) any later version.}}
\hypersetup{pdflicenseurl={http://www.latex-project.org/lppl.txt}}
\hypersetup{pdfcontactaddress={ETH Zurich, ITP, HIT K,
  Wolfgang-Pauli-Strasse 27}}
\hypersetup{pdfcontactpostcode={8093}}
\hypersetup{pdfcontactcity={Zurich}}
\hypersetup{pdfcontactcountry={Switzerland}}
\hypersetup{pdfcontactemail={nbeisert@itp.phys.ethz.ch}}
\hypersetup{pdfcontacturl={http://people.phys.ethz.ch/\xmptilde nbeisert/}}

\newcommand{\secref}[1]{\hyperref[#1]{section \ref*{#1}}}

\parskip1ex
\parindent0pt
\let\olditemize\itemize
\def\itemize{\olditemize\parskip0pt}

\begin{document}

\title{The \textsf{childdoc} Package}
\hypersetup{pdftitle={The childdoc Package}}
\author{Niklas Beisert\\[2ex]
  Institut f\"ur Theoretische Physik\\
  Eidgen\"ossische Technische Hochschule Z\"urich\\
  Wolfgang-Pauli-Strasse 27, 8093 Z\"urich, Switzerland\\[1ex]
  \href{mailto:nbeisert@itp.phys.ethz.ch}
  {\texttt{nbeisert@itp.phys.ethz.ch}}}
\hypersetup{pdfauthor={Niklas Beisert}}
\hypersetup{pdfsubject={Manual for the LaTeX2e Package childdoc}}
\date{30 December 2018, \textsf{v2.0}}
\maketitle

\begin{abstract}\noindent
\textsf{childdoc} is a \LaTeXe{} package
that enables the direct compilation
of document sections included by |\include|
to individual files.
\end{abstract}

\begingroup
\parskip0ex
\tableofcontents
\endgroup

%%%%%%%%%%%%%%%%%%%%%%%%%%%%%%%%%%%%%%%%%%%%%%%%%%%%%%%%%%%%%%%%%%%%%%%%%%%%%%%%
%%%%%%%%%%%%%%%%%%%%%%%%%%%%%%%%%%%%%%%%%%%%%%%%%%%%%%%%%%%%%%%%%%%%%%%%%%%%%%%%
\section{Introduction}

\LaTeX{} provides a mechanism to structure a large document (such as a book)
into a main file and several child files (containing the chapters)
using the |\include| command.
This mechanism is beneficial for documents
which span hundreds of pages in order to
make the source file(s) more manageable.
Moreover, compilation can be restricted to
selected child files by means of the |\includeonly| command.
The latter feature can be used to reduce the compilation time while editing
(this was significantly more useful in the earlier days of \LaTeX{})
or to generate a smaller document which is easier to navigate.
Another application of |\includeonly| is to generate
documents consisting of selected parts of the complete document.

However, there are a few drawbacks of the plain |\include| mechanism:
\begin{itemize}
\item
The child files cannot be compiled on their own,
they can only be compiled via the main file.
A naive editing environment
(such as a text editor with an option
to have the current file processed by \LaTeX)
may require one to switch to the main file before compiling;
attempting to compile the child file produces errors.
\item
The main file must be modified (each time)
to adjust the |\includeonly| command
to the present needs. This easily leaves the main file in a messy state.
\item
The generated document will always carry the filename
of the main document. This is inconvenient if
several child files are to be compiled and
to be kept for distribution.
\end{itemize}

The present package provides a simple interface
to make child files individually compilable by \LaTeX{}.
Compiling a child file then has the same effect as compiling
the main file with an |\includeonly| command
to select the appropriate child.
Moreover the generated document will carry the name of the child
rather than the main file.
This resolves all three above issues.

This feature is meant to make the editing of books,
thesis documents and lecture notes somewhat more convenient.
However, the package can also be used efficiently for
composing a series of documents (such as exercise sheets)
which are typically distributed individually.
It then assists the author in generating the individual documents
(potentially in different versions)
as well as a document containing the collected series.
Another application is in developing style files
or other kinds of included material
where compilation of the style file could redirect
to a sample or test file.

%%%%%%%%%%%%%%%%%%%%%%%%%%%%%%%%%%%%%%%%%%%%%%%%%%%%%%%%%%%%%%%%%%%%%%%%%%%%%%%%
%%%%%%%%%%%%%%%%%%%%%%%%%%%%%%%%%%%%%%%%%%%%%%%%%%%%%%%%%%%%%%%%%%%%%%%%%%%%%%%%
\section{Usage}

First of all, the package \textsf{childdoc} is \emph{not} a standard
\LaTeXe{} |.sty| style file! Therefore it needs to be invoked in
a non-standard way.

%%%%%%%%%%%%%%%%%%%%%%%%%%%%%%%%%%%%%%%%%%%%%%%%%%%%%%%%%%%%%%%%%%%%%%%%%%%%%%%%
\subsection{Included Files}
\label{sec:include}

%%%%%%%%%%%%%%%%%%%%%%%%%%%%%%%%%%%%%%%%
\DescribeMacro{\childdocmain}
To use the package, add the commands
\begin{center}
\begin{tabular}{l}
|\input{childdoc.def}|\\
|\childdocmain{}|\\
\end{tabular}
\end{center}
at the very top of the main \LaTeX{} file,
in particular \emph{before} the |\documentclass| statement!
The argument of |\childdocmain| should be left empty
(but it must be present).

%%%%%%%%%%%%%%%%%%%%%%%%%%%%%%%%%%%%%%%%
\DescribeMacro{\childdocof}
Furthermore, add the commands
\begin{center}
\begin{tabular}{l}
|\input{childdoc.def}|\\
|\childdocof{|\textit{main}|}|\\
\end{tabular}
\end{center}
at the top of every child file \textit{child}
which is included by |\include{|\textit{child}|}|
from within the main file
(or at least for those files to be compiled individually).
The argument \textit{main} must be the filename of the main file.

There are a couple of
considerations in setting up the main and child documents:

%%%%%%%%%%%%%%%%%%%%%%%%%%%%%%%%%%%%%%%%
\paragraph{Restrictions.}

Please note the following restrictions:
\begin{itemize}
\item
|\childdocmain| must be called with one argument \textit{main}
to ensure compatibility with earlier version of the package.
It must either be empty (|\childdocmain{}|)
or precisely match the filename of the main file in which it is specified.
See \secref{sec:detection} for further information.
\item
The filename \textit{main} must be specified without the |.tex| extension.
\item
The filename \textit{main} is case sensitive
(even in case-insensitive file systems)
due to internal string comparison.
\item
The argument \textit{main} should be fully expanded, it cannot be a macro.
\item
Subdirectories and special characters should be avoided in filenames.
\item
The command |\childdocmain{|\textit{main}|}| must be followed by a whitespace.
It should not be followed immediately by another command
or by a comment mark `|%|'.
This is because the \TeX{} parser reads the token immediately following
the argument of |\childdocmain| and puts it
at the beginning of every child section;
however, a white\-space is ignored.
\end{itemize}

%%%%%%%%%%%%%%%%%%%%%%%%%%%%%%%%%%%%%%%%
\paragraph{Content of Main File.}

It is advisable to place all content in the child files included by |\include|.
Any output contained in the main file will appear in all child documents
unless suppressed manually;
it cannot be suppressed automatically by the |\includeonly| directive
and thus should normally be avoided.
A method to include some content in the main file
by means of conditional processing is described in \secref{sec:conditional}.

%%%%%%%%%%%%%%%%%%%%%%%%%%%%%%%%%%%%%%%%
\paragraph{Page Numbering.}

When only a part of the document is compiled,
the appropriate numbering of pages
(as well as other status parameters)
is determined from the |.aux| files.
The latter contain information from previous passes.
However this information needs to propagate through
all intermediate child documents.
Therefore the page numbering in child documents may well
be inconsistent until the complete document is compiled at least once.

A useful (if unconventional) way to always ensure a consistent
page numbering is to restart the numbering in each child document
and denote the pages by `\textit{child}|.|\textit{page}'
where \textit{child} represents the chapter/section number of the child file.
This can be achieved by the command
|\numberwithin{page}{|\textit{child}|}|
of the \textsf{amsmath} package
where \textit{child} can be |chapter| or |section|
depending on the chosen structuring.
Alternatively, one can modify the macro |\thepage| appropriately
and reset the counter |page| at the start of each child file.

%%%%%%%%%%%%%%%%%%%%%%%%%%%%%%%%%%%%%%%%%%%%%%%%%%%%%%%%%%%%%%%%%%%%%%%%%%%%%%%%
\subsection{Conditional Processing}
\label{sec:conditional}

The package provides a mechanism to compile different versions
of a document. To customise the versions further some conditional processing
can come in handy to distinguish which version is being compiled.
The package provides two macros to describe the compilation context:

%%%%%%%%%%%%%%%%%%%%%%%%%%%%%%%%%%%%%%%%
\DescribeMacro{\ifchilddoc}
The conditional |\ifchilddoc| distinguishes between the compilation of
child documents and the main document:
%
\begin{center}
|\ifchilddoc |\textit{child-code}| |[|\||else |\textit{main-code}]| \||fi|
\end{center}

%%%%%%%%%%%%%%%%%%%%%%%%%%%%%%%%%%%%%%%%
\DescribeMacro{\childdocname}
\DescribeMacro{\childdocjob}
The macro |\childdocname| contains the filename (without extension)
of the main or child file being processed.
Note that |\childdocjob| will always contain the name of the main file.

%%%%%%%%%%%%%%%%%%%%%%%%%%%%%%%%%%%%%%%%
\paragraph{Title Page.}

Conditional processing can be used to include a title or banner page
in the main document when proper precautions are taken.
Importantly, the code in the main file should ensure that the page counter
(as well as other status parameters which are stored in the |.aux| files)
takes the same value after the conditional processing.
Otherwise the page numbers may take divergent values
depending on which part is compiled.

For example, a title page could be declared by:
%
\begin{center}
\begin{tabular}{l}
|\ifchilddoc\||else|\\
|\addtocounter{page}{-1}|\\
\textit{code for title page}\\
|\newpage|\\
|\||fi|
\end{tabular}
\end{center}
%
A banner page for the child documents can be generated by:
%
\begin{center}
\begin{tabular}{l}
|\ifchilddoc|\\
|\addtocounter{page}{-1}|\\
\textit{code for banner page}\\
|\newpage|\\
|\||fi|
\end{tabular}
\end{center}
%
Here one could write a message such as:
\begin{center}
|This is the part \childdocname{} of \childdocjob{}.|
\end{center}

%%%%%%%%%%%%%%%%%%%%%%%%%%%%%%%%%%%%%%%%%%%%%%%%%%%%%%%%%%%%%%%%%%%%%%%%%%%%%%%%
\subsection{Flags}
\label{sec:flags}

The package makes it easy to generate different versions
of the main or child documents.
To this end compilation flags can be defined
and assigned different default values.
They will be particularly useful in conjunction
with the forwarding mechanism described in \secref{sec:forward}.

For example, it may be useful to have a flag |\version|
which can be set to |draft| or |final|.
The document source will contain some conditional code
depending on the value of |\version|.
Suppose further, the flag should default to |final| for the main file
and to |draft| for child files
which is a natural assignment for editing the document.
This is achieved by placing the following code
in the preamble of the main document
(below the |\childdocmain| directive):
%
\begin{center}
\begin{tabular}{l}
|\ifchilddoc|\\
|\providecommand{\version}{draft}|\\
|\||else|\\
|\providecommand{\version}{final}|\\
|\||fi|
\end{tabular}
\end{center}
%
The definition by |\providecommand| makes sure
that previous definitions are not overwritten.
Further statements |\providecommand{\version}{...}|
can thus be added before the above code to override it.

For the main file, one might add a line
(between |\childdocmain| and the above block)
%
\begin{center}
|%\ifchilddoc\||else\providecommand{\version}{draft}\||fi|
\end{center}
%
which can be uncommented to produce a draft version.
Likewise one can add a line to the very top of a child file
(above the |\childdocof{|\textit{main}|}| directive)
%
\begin{center}
|%\providecommand{\version}{final}|
\end{center}
%
which can be uncommented to produce the final version of this child document.

%%%%%%%%%%%%%%%%%%%%%%%%%%%%%%%%%%%%%%%%%%%%%%%%%%%%%%%%%%%%%%%%%%%%%%%%%%%%%%%%
\subsection{Forwarding}
\label{sec:forward}

Different versions of the main or child documents
using compilation flags as described in \secref{sec:flags}
can be (permanently) stored in different files
for convenient compilation, viewing and distribution.
To this end, the package defines a command
to pass on compilation to a different file:

%%%%%%%%%%%%%%%%%%%%%%%%%%%%%%%%%%%%%%%%
\DescribeMacro{\childdocforward}
The command |\childdocforward| redirects processing to
another source file:
%
\begin{center}
\begin{tabular}{l}
|\input{childdoc.def}|\\
|\childdocforward[|\textit{main}|]{|\textit{dest}|}|\\
\end{tabular}
\end{center}
%
The argument \textit{dest} is the destination file
(without extension).
It should be the main file or one of the child files.
Note that further \textsf{childdoc} directives
such as |\childdocof| and |\childdocforward|
in the indicated file will be processed in this form.
The optional argument \textit{main}
passes on directly to the main file \textit{main}
while pretending to compile the child \textit{dest}.
This form behaves as if \textit{dest}
issues |\childdocof{|\textit{main}|}| right away,
and no further \textsf{childdoc} directives will be processed.

%%%%%%%%%%%%%%%%%%%%%%%%%%%%%%%%%%%%%%%%
\DescribeMacro{\...prefix}
In the alternative form |\childdocforwardprefix|,
%
\begin{center}
\begin{tabular}{l}
|\input{childdoc.def}|\\
|\childdocforwardprefix[|\textit{main}|]{|\textit{prefix}|}{|\textit{dest}|}|
\end{tabular}
\end{center}
%
the destination file is determined by a pattern
depending on the current file:
To make this work, the current file must be called
`{\textit{prefix}\hspace{0.2em}\textit{suffix}}'
with \textit{prefix} matching precisely the argument.
Processing is then passed on to the file
`{\textit{dest}\hspace{0.2em}\textit{suffix}}'.
Surely, the same effect is achieved by
directly specifying the
argument `{\textit{dest}\hspace{0.2em}\textit{suffix}}'
in the first form.
However, that requires to set up a different file
for each child. With the alternative form of the command
all these files can have exactly the same content
which simplifies setting them up and maintaining them.

For example, the following file |draft.tex|
with a compilation flag |\version| as described in \secref{sec:flags}
compiles the main document as a draft:
%
\begin{center}
\begin{tabular}{l}
|\def\version{draft}|\\
|\input{childdoc.def}|\\
|\childdocforward{|\textit{main}|}|
\end{tabular}
\end{center}
%
Likewise, the following files |final|\textit{nn}|.tex|
compile the final version of the child document
|child|\textit{nn}|.tex|:
%
\begin{center}
\begin{tabular}{l}
|\def\version{final}|\\
|\input{childdoc.def}|\\
|\childdocforwardprefix{final}{child}|
\end{tabular}
\end{center}
%

Note that when several versions of a main file and/or of each child file
are to be generated, it may be convenient to set up a |Makefile| or
shell script to automatise the process.

%%%%%%%%%%%%%%%%%%%%%%%%%%%%%%%%%%%%%%%%%%%%%%%%%%%%%%%%%%%%%%%%%%%%%%%%%%%%%%%%
\subsection{Command Line Processing}
\label{sec:commandline}

The effect of redirection files can also be achieved by invoking
the \LaTeX{} compiler with a more elaborate command line.
Most conveniently this should be done as part
of a shell script or a |Makefile|.

When using \textsf{childdoc} in the main file, the following
command lines effectively perform a redirection
(note that depending on the shell being used,
backslashes may have to be doubled: `|\|' $\to$ `|\\|'):
%
\begin{center}
|... -jobname "|\textit{target}|" |\\|"|[\textit{flags}]%
|\input{childdoc.def}\childdocforward[|\textit{main}|]{|\textit{dest}|}"|
\end{center}
%
Here \textit{target} is the name of the output file,
\textit{main} is the name of the main file
and \textit{dest} is the name of the main or child file to be processed
(all filenames without extensions).
The optional argument \textit{main} can be omitted
if \textit{main} matches \textit{dest}.
Optionally, compilation \textit{flags} can be defined via |\def| commands.
This command line makes the \TeX{} engine believe
it is compiling the file \textit{target}
whose content is specified as the latter parameter.
The provided code then forwards the processing to
\textit{main} or \textit{dest} as described in \secref{sec:forward}.

%%%%%%%%%%%%%%%%%%%%%%%%%%%%%%%%%%%%%%%%%%%%%%%%%%%%%%%%%%%%%%%%%%%%%%%%%%%%%%%%
\subsection{Include by Input}
\label{sec:input}

Including child documents by |\include| has some restrictions by design.
Most notably, the content of a child document always occupies
its own set of pages; pages cannot be shared between child documents.
Usually, this behaviour makes perfect sense
because each child document contain an essential part of the document.
However, in some situations it may be desirable to compose
a document from a collection of parts
without having mandatory page breaks between then.
For this case, the package
provides a mechanism to include parts
by |\input| which can also be processed individually.
However, by construction this mechanism
requires manual handling of the content to be output.

%%%%%%%%%%%%%%%%%%%%%%%%%%%%%%%%%%%%%%%%
\DescribeMacro{\ifchilddocmanual}
The main file should be prepared as usual, see \secref{sec:include}.
However, the document body must make a distinction
between processing of an individual part and of the main document, e.g.:
%
\begin{center}
\begin{tabular}{l}
|\ifchilddocmanual|\\
|\input{\childdocname}|\\
|\||else|\\
\textit{document body with }|\input{|\textit{part}|}|\\
|\||fi|
\end{tabular}
\end{center}
%
The conditional |\ifchilddocmanual| is true whenever
a part to be included by |\input| is being compiled,
and the name of the part is stored in |\childdocname|.

%%%%%%%%%%%%%%%%%%%%%%%%%%%%%%%%%%%%%%%%
\DescribeMacro{\childdocby}
Each part to be included by |\input| should start with:
%
\begin{center}
\begin{tabular}{l}
|\input{childdoc.def}|\\
|\childdocby{|\textit{main}|}|\\
\end{tabular}
\end{center}
%
The directive |\childdocby| is similar to |\childdocof|
described in \secref{sec:include},
but the subsequent selection of content must be done manually.
To that end, both |\ifchilddoc| and |\ifchilddocmanual|
will be true upon processing of a part,
and the name of the part is stored in |\childdocname|.
Note that |\jobname| will be set to the filename of the current part
so that each part receives an individual |.aux| file
that does not interfere with the |.aux| file(s) of the main document.
This behaviour can be altered by the alternative form
|\childdocby[*]{|\textit{main}|}| (with a non-empty optional argument)
which uses the |.aux| file of the main document
by setting |\jobname| to \textit{main}.

%%%%%%%%%%%%%%%%%%%%%%%%%%%%%%%%%%%%%%%%%%%%%%%%%%%%%%%%%%%%%%%%%%%%%%%%%%%%%%%%
\subsection{Driver Development}
\label{sec:driver}

The \textsf{childdoc} mechanism can also be use for the development
of definition files such as \LaTeX{} styles or classes.
This case differs from the above setup with multiple parts
included by |\include| in that no |\includeonly| should be invoked.
This can be achieved by starting the include file
(before |\ProvidesPackage|) with:
%
\begin{center}
\begin{tabular}{l}
|\input{childdoc.def}|\\
|\childdocforward{|\textit{main}|}|\\
\end{tabular}
\end{center}
%
or alternatively with:
%
\begin{center}
\begin{tabular}{l}
|\input{childdoc.def}|\\
|\childdocby{|\textit{main}|}|\\
\end{tabular}
\end{center}
%
Both forms have slightly different effects as described above.
The main file is prepared as usual, see \secref{sec:include}.

%%%%%%%%%%%%%%%%%%%%%%%%%%%%%%%%%%%%%%%%%%%%%%%%%%%%%%%%%%%%%%%%%%%%%%%%%%%%%%%%
\subsection{Legacy Detection}
\label{sec:detection}

The directive |\childdocmain| in the main file can detect
whether the complete document or merely a child is to be compiled
even without using the directive |\childdocof|.
This method is deprecated because it is less robust
and there is no compelling reason to use it;
it is merely provided for backward compatibility
and it may be removed in future versions.

If the detection mechanism is to be used,
it is mandatory to correctly specify
the filename of the main file as the argument of |\childdocmain|:
%
\begin{center}
\begin{tabular}{l}
|\input{childdoc.def}|\\
|\childdocmain{|\textit{main}|}|\\
\end{tabular}
\end{center}
%
If |\jobname| does not match the argument \textit{main} of |\childdocmain|,
it is assumed that |\jobname| points to the child file to be compiled.
When using |\childdocmain| with the main file specified as argument,
it suffices to start a child file
with just |\input{|\textit{main}|}|
without loading of the package and using |\childdocof|.
If instead all processing is done
with the appropriate \textsf{childdoc} directives,
the argument of \textit{main} of |\childdocmain| can be empty.

An alternative version of the command line processing described
in \secref{sec:commandline} using the detection mechanism reads:
%
\begin{center}
|... -jobname "|\textit{target}|" "|[\textit{flags}]%
[|\def\jobname{|\textit{dest}|}|]|\input{|\textit{main}|}"|
\end{center}

%%%%%%%%%%%%%%%%%%%%%%%%%%%%%%%%%%%%%%%%%%%%%%%%%%%%%%%%%%%%%%%%%%%%%%%%%%%%%%%%
\subsection{Manual Code}
\label{sec:manual}

In case one cannot be certain whether the definitions file |childdoc.def|
is installed on the target \TeX{} distribution
and one prefers not to ship it,
it is conceivable to paste a few relevant commands into the sources.

To that end, drop all statements |\input{childdoc.def}|
and perform the replacements as outlined below.
Instead of |\childdocmain{|\textit{main}|}| add the following code
to the top of the main file:
%
\begin{center}
\begin{tabular}{l}
|\||ifdefined\childdocname\endinput\||fi\newif\ifchilddoc|\\
|\edef\childdocname{\scantokens\expandafter{\jobname\noexpand}}|\\
|\def\childdocmain{|\textit{main}|}\||ifx\childdocmain\childdocname\||else|\\
|\childdoctrue\includeonly{\childdocname}\let\jobname\childdocmain\||fi|\\
\end{tabular}
\end{center}
%
Instead of |\childdocof{|\textit{main}|}| just include the main file
at the top of each child file:
%
\begin{center}
|\input{|\textit{main}|}|
\end{center}
%
A simple redirection |\childdocforward{|\textit{dest}|}| is achieved by:
%
\begin{center}
|\def\jobname{|\textit{dest}|}\input{\jobname}|
\end{center}
%
The redirection with prefix
|\childdocforwardprefix[|\textit{prefix}|]{|\textit{dest}|}|
is accomplished by:
%
\begin{center}
\begin{tabular}{l}
|{\edef\jobname{\scantokens\expandafter{\jobname\noexpand}}|\\
|\def\redirectjob |\textit{prefix}|#1~~~{\gdef\jobname{|\textit{dest}|#1}}|\\
|\expandafter\redirectjob\jobname~~~}\input{\jobname}|
\end{tabular}
\end{center}

In an alternative approach,
child documents can be compiled by a specific command line
without additional code or specific definitions:
%
\begin{center}
|... -jobname "|\textit{target}|" "|[\textit{flags}]%
|\includeonly{|\textit{dest}|}\input{|\textit{main}|}"|
\end{center}
%

%%%%%%%%%%%%%%%%%%%%%%%%%%%%%%%%%%%%%%%%%%%%%%%%%%%%%%%%%%%%%%%%%%%%%%%%%%%%%%%%
%%%%%%%%%%%%%%%%%%%%%%%%%%%%%%%%%%%%%%%%%%%%%%%%%%%%%%%%%%%%%%%%%%%%%%%%%%%%%%%%
\section{Information}

%%%%%%%%%%%%%%%%%%%%%%%%%%%%%%%%%%%%%%%%%%%%%%%%%%%%%%%%%%%%%%%%%%%%%%%%%%%%%%%%
\subsection{Copyright}

Copyright \copyright{} 2017--2018 Niklas Beisert

This work may be distributed and/or modified under the
conditions of the \LaTeX{} Project Public License, either version 1.3
of this license or (at your option) any later version.
The latest version of this license is in
  \url{http://www.latex-project.org/lppl.txt}
and version 1.3 or later is part of all distributions of \LaTeX{}
version 2005/12/01 or later.

This work has the LPPL maintenance status `maintained'.

The Current Maintainer of this work is Niklas Beisert.

This work consists of the files |README.txt|, |childdoc.ins| and |childdoc.dtx|
as well as the derived files |childdoc.def|, |cdocsamp.tex|
with |cdocsch1.tex|, |cdocsch2.tex|, |cdocspt3.tex|, |cdocspt4.tex|,
|cdocsdrf.tex|, |cdocsfn1.tex|, |cdocsfn2.tex|
as well as |childdoc.pdf|.

%%%%%%%%%%%%%%%%%%%%%%%%%%%%%%%%%%%%%%%%%%%%%%%%%%%%%%%%%%%%%%%%%%%%%%%%%%%%%%%%
\subsection{Files and Installation}

The package consists of the files:
%
\begin{center}
\begin{tabular}{ll}
    |README.txt|   & readme file \\
    |childdoc.ins| & installation file \\
    |childdoc.dtx| & source file \\
    |childdoc.def| & definition file \\
    |cdocsamp.tex| & sample main file \\
    |cdocsch1.tex| & sample include file \\
    |cdocsch2.tex| & sample include file \\
    |cdocspt3.tex| & sample part file \\
    |cdocspt4.tex| & sample part file \\
    |cdocsdrf.tex| & sample redirection file \\
    |cdocsfn1.tex| & sample redirection file \\
    |cdocsfn2.tex| & sample redirection file \\
    |childdoc.pdf| & manual
\end{tabular}
\end{center}
%
The distribution consists of the files
|README.txt|, |childdoc.ins| and |childdoc.dtx|.
%
\begin{itemize}
\item
Run (pdf)\LaTeX{} on |childdoc.dtx|
to compile the manual |childdoc.pdf| (this file).
\item
Run \LaTeX{} on |childdoc.ins| to create the definitions file |childdoc.def|
and the sample |cdocsamp.tex| with include files
|cdocsch1.tex|, |cdocsch2.tex|, |cdocspt3.tex|, |cdocspt4.tex|,
|cdocsdrf.tex|, |cdocsfn1.tex|, |cdocsfn2.tex|.
Then copy the file |childdoc.def| to an appropriate directory of your \LaTeX{}
distribution, e.g.\ \textit{texmf-root}|/tex/latex/childdoc|.
\end{itemize}

%%%%%%%%%%%%%%%%%%%%%%%%%%%%%%%%%%%%%%%%%%%%%%%%%%%%%%%%%%%%%%%%%%%%%%%%%%%%%%%%
\subsection{Related CTAN Packages}

There are several other packages which offer a similar functionality:
%
\begin{itemize}
\item
The packages
\href{http://ctan.org/pkg/docmute}{\textsf{docmute}},
\href{http://ctan.org/pkg/includex}{\textsf{includex}} and
\href{http://ctan.org/pkg/standalone}{\textsf{standalone}}
provide commands to include only the document body of
a child file thus allowing both files to be compiled individually.
\item
The packages \href{http://ctan.org/pkg/subdocs}{\textsf{subdocs}}
and \href{http://ctan.org/pkg/subfiles}{\textsf{subfiles}}
provide structures in which the main and child documents can be
encapsulated and allowing them to be compiled individually.
The inclusion mechanism is different from the conventional |\include|.
\item
The package \href{http://ctan.org/pkg/combine}{\textsf{combine}}
is an elaborate solution to combine several documents into one.
\end{itemize}
%
See also the CTAN topic \href{http://ctan.org/topic/subdocs}{\textsf{subdocs}}
for further related packages.
The present package differs from the above solutions in that
a document structure constructed with the conventional |\include| mechanism
just needs two extra commands at the top of every file
such that all constituent files can be compiled individually.

%%%%%%%%%%%%%%%%%%%%%%%%%%%%%%%%%%%%%%%%%%%%%%%%%%%%%%%%%%%%%%%%%%%%%%%%%%%%%%%%
%\subsection{Feature Suggestions}
%
%The following is a list of features which may be useful for future
%versions of this package:
%%
%\begin{itemize}
%\item
%\ldots
%\end{itemize}

%%%%%%%%%%%%%%%%%%%%%%%%%%%%%%%%%%%%%%%%%%%%%%%%%%%%%%%%%%%%%%%%%%%%%%%%%%%%%%%%
\subsection{Revision History}

%%%%%%%%%%%%%%%%%%%%%%%%%%%%%%%%%%%%%%%%
\paragraph{v2.0:} 2018/12/30

\begin{itemize}
\item
immediate forward processing
\item
added |\childdocby| mechanism
\item
manual restructured
\end{itemize}

%%%%%%%%%%%%%%%%%%%%%%%%%%%%%%%%%%%%%%%%
\paragraph{v1.6:} 2018/01/17

\begin{itemize}
\item
application for development of include files
\item
corrections to manual
\end{itemize}

%%%%%%%%%%%%%%%%%%%%%%%%%%%%%%%%%%%%%%%%
\paragraph{v1.5:} 2017/05/21

\begin{itemize}
\item
more complete structuring introduced
\item
|\childdocof| introduced
\item
|\childdoc| renamed to |\childdocmain|
\item
|\childredirect| renamed to |\childdocforward| and |\childdocforwardprefix|
and functionality expanded
\end{itemize}

%%%%%%%%%%%%%%%%%%%%%%%%%%%%%%%%%%%%%%%%
\paragraph{v1.0:} 2017/04/27

\begin{itemize}
\item
manual and install package
\item
first version published on CTAN
\end{itemize}

%%%%%%%%%%%%%%%%%%%%%%%%%%%%%%%%%%%%%%%%
\paragraph{v0.6:} 2017/04/26

\begin{itemize}
\item
redirection mechanism added
\end{itemize}

%%%%%%%%%%%%%%%%%%%%%%%%%%%%%%%%%%%%%%%%
\paragraph{v0.5:} 2017/04/26

\begin{itemize}
\item
functionality in definition file
\end{itemize}


%%%%%%%%%%%%%%%%%%%%%%%%%%%%%%%%%%%%%%%%%%%%%%%%%%%%%%%%%%%%%%%%%%%%%%%%%%%%%%%%
%%%%%%%%%%%%%%%%%%%%%%%%%%%%%%%%%%%%%%%%%%%%%%%%%%%%%%%%%%%%%%%%%%%%%%%%%%%%%%%%
%%%%%%%%%%%%%%%%%%%%%%%%%%%%%%%%%%%%%%%%%%%%%%%%%%%%%%%%%%%%%%%%%%%%%%%%%%%%%%%%
\appendix

\settowidth\MacroIndent{\rmfamily\scriptsize 000\ }

 \DocInput{childdoc.dtx}

\end{document}
%</driver>
% \fi
%
% %%%%%%%%%%%%%%%%%%%%%%%%%%%%%%%%%%%%%%%%%%%%%%%%%%%%%%%%%%%%%%%%%%%%%%%%%%%%%%
% %%%%%%%%%%%%%%%%%%%%%%%%%%%%%%%%%%%%%%%%%%%%%%%%%%%%%%%%%%%%%%%%%%%%%%%%%%%%%%
% \section{Sample}
%\iffalse
%<*samplemain>
%\fi
%
% The following presents a sample document
% with two chapters, two parts, a title page,
% a compile flag as well as three forwarding files to set the flag.
% It consists of eight |.tex| files:
% \begin{center}
% \begin{tabular}{ll}
% |cdocsamp.tex|&main file\\
% |cdocsch1.tex|&include file for chapter 1\\
% |cdocsch2.tex|&include file for chapter 2\\
% |cdocspt3.tex|&include file for part 3\\
% |cdocspt4.tex|&include file for part 4\\
% |cdocsdrf.tex|&forwarding file for main file in draft mode\\
% |cdocsfi1.tex|&forwarding file for final version of chapter 1\\
% |cdocsfi2.tex|&forwarding file for final version of chapter 2\\
% \end{tabular}
% \end{center}
% Each of the eight files can be compiled directly by the \LaTeX{} compiler.
%
% %%%%%%%%%%%%%%%%%%%%%%%%%%%%%%%%%%%%%%
% \paragraph{Main File.}
%
% The main file is called |cdocsamp.tex|.
%
% Load the \textsf{childdoc} definitions and
% declare the filename for the main document:
%    \begin{macrocode}
\input{childdoc.def}
\childdocmain{}
%    \end{macrocode}

% Optional override for |\version| flag:
%    \begin{macrocode}
%%\ifchilddoc\else\providecommand{\version}{draft}\fi
%    \end{macrocode}

% Define the default values for the |\version| flag
% (|final| for the main file and |draft| for childs):
%    \begin{macrocode}
\ifchilddoc
\providecommand{\version}{draft}
\else
\providecommand{\version}{final}
\fi
%    \end{macrocode}

% Load the standard document class:
%    \begin{macrocode}
\documentclass[12pt]{article}
%    \end{macrocode}

% Start the document body:
%    \begin{macrocode}
\begin{document}
%    \end{macrocode}

% Declare a title page.
% Print title, part of document being processed and version flag:
%    \begin{macrocode}
\addtocounter{page}{-1}
\begin{center}
{\LARGE\bfseries{}childdoc example\par}
\vspace{1cm}
\ifchilddoc
\ifchilddocmanual part\else chapter\fi:
`\childdocname' of `\childdocjob'\par
\else
main document: `\childdocjob'\par
\fi
version: \version\par
\end{center}
\newpage
%    \end{macrocode}

% Manually include selected file,
% otherwise process as usual:
%    \begin{macrocode}
\ifchilddocmanual
\section*{part `\childdocname'}
\input{\childdocname}
\else
%    \end{macrocode}

% Include the two chapters:
%    \begin{macrocode}
\include{cdocsch1}
\include{cdocsch2}
%    \end{macrocode}

% Include the two parts unless only chapters should be displayed:
%    \begin{macrocode}
\ifchilddoc\else
\section{part three}
\input{cdocspt3}
\section{part four}
\input{cdocspt4}
\fi
%    \end{macrocode}

% Process as usual until here:
%    \begin{macrocode}
\fi
%    \end{macrocode}

% End of document body:
%    \begin{macrocode}
\end{document}
%    \end{macrocode}
%\iffalse
%</samplemain>
%\fi
%
% %%%%%%%%%%%%%%%%%%%%%%%%%%%%%%%%%%%%%%
% \paragraph{Chapter Include Files.}
%
% The include files are called |cdocsch1.tex| and |cdocsch2.tex|.
%
%\iffalse
%<*samplechap1|samplechap2>
%\fi

% Optional override for |\version| flag:
%    \begin{macrocode}
%%\providecommand{\version}{final}
%    \end{macrocode}

% Include the main document:
%    \begin{macrocode}
\input{childdoc.def}
\childdocof{cdocsamp}
%    \end{macrocode}

%\iffalse
%</samplechap1|samplechap2>
%\fi
%
%\iffalse
%<*samplechap1>
%\fi
% Some text for chapter 1:
%    \begin{macrocode}
\section{one}
some text in chapter one
%    \end{macrocode}

%\iffalse
%</samplechap1>
%\fi
% Some text for chapter 2:
%\iffalse
%<*samplechap2>
%\fi
%    \begin{macrocode}
\section{two}
more text in chapter two
%    \end{macrocode}

%\iffalse
%</samplechap2>
%\fi
%
% %%%%%%%%%%%%%%%%%%%%%%%%%%%%%%%%%%%%%%
% \paragraph{Part Include Files.}
%
% The include files are called |cdocspt3.tex| and |cdocspt4.tex|.
%
%\iffalse
%<*samplepart3|samplepart4>
%\fi

% Optional override for |\version| flag:
%    \begin{macrocode}
%%\providecommand{\version}{final}
%    \end{macrocode}

% Include the main document:
%    \begin{macrocode}
\input{childdoc.def}
\childdocby{cdocsamp}
%    \end{macrocode}

%\iffalse
%</samplepart3|samplepart4>
%\fi
%
%\iffalse
%<*samplepart3>
%\fi
% Some text for part 3:
%    \begin{macrocode}
some text in part three
%    \end{macrocode}

%\iffalse
%</samplepart3>
%\fi
% Some text for part 4:
%\iffalse
%<*samplepart4>
%\fi
%    \begin{macrocode}
more text in part four
%    \end{macrocode}

%\iffalse
%</samplepart4>
%\fi
%
% %%%%%%%%%%%%%%%%%%%%%%%%%%%%%%%%%%%%%%
% \paragraph{Forwarding for a Complete Draft.}
%
% The following forwarding file |cdocsdrf.tex|
% compiles the main document in draft mode:
%\iffalse
%<*sampledraft>
%\fi
%    \begin{macrocode}
\def\version{draft}
\input{childdoc.def}
\childdocforward{cdocsamp}
%    \end{macrocode}

%\iffalse
%</sampledraft>
%\fi
%
% %%%%%%%%%%%%%%%%%%%%%%%%%%%%%%%%%%%%%%
% \paragraph{Forwarding for Final Version of the Chapters.}
%
% The following forwarding files |cdocsfn1.tex| and |cdocsfn2.tex|
% (with identical content)
% compile the final versions of the child documents
% |cdocsch1.tex| and |cdocsch2.tex|, respectively:
%\iffalse
%<*samplefinal>
%\fi
%    \begin{macrocode}
\def\version{final}
\input{childdoc.def}
\childdocforwardprefix[cdocsamp]{cdocsfn}{cdocsch}
%    \end{macrocode}

%\iffalse
%</samplefinal>
%\fi
%
% %%%%%%%%%%%%%%%%%%%%%%%%%%%%%%%%%%%%%%
% \paragraph{Command Line Processing.}
%
% The following three command lines generate the output files
% |cdocscld|, |cdocscl1| and |cdocscl2|
% which should be identical to
% |cdocsdrf|, |cdocsch1| and |cdocsfn2|, respectively:
% \begin{center}
% \begin{tabular}{l}
% |latex -jobname cdocscld \|\\
% |  "\def\version{draft}\input{childdoc.def}\childdocforward{cdocsamp}"|\\
% |latex -jobname cdocscl1 \|\\
% |  "\input{childdoc.def}\childdocforward[cdocsamp]{cdocsch1}"|\\
% |latex -jobname cdocscl2 \|\\
% |  "\def\version{final}\input{childdoc.def}\childdocforward{cdocsch2}"|
% \end{tabular}
% \end{center}
% Note that the trailing backslash on each first line
% merely continues the input to the second line
% (for convenient cut ant paste).
% Furthermore, the command |latex| can be replaced by any
% of its alternative versions such as |pdflatex|.
%
% %%%%%%%%%%%%%%%%%%%%%%%%%%%%%%%%%%%%%%%%%%%%%%%%%%%%%%%%%%%%%%%%%%%%%%%%%%%%%%
% %%%%%%%%%%%%%%%%%%%%%%%%%%%%%%%%%%%%%%%%%%%%%%%%%%%%%%%%%%%%%%%%%%%%%%%%%%%%%%
% \section{Implementation}
%\iffalse
%<*package>
%\fi
%
% This section describes the definitions file |childdoc.def|.

% The definitions cannot be loaded using |\usepackage| or |\RequirePackage|
% which has a mechanism to prevent loading a style file more than once.
% When loading the definitions by means of |\input|
% multiple instances have to be prevented manually:
%\iffalse
%This code needs to be before the `\ProvidesFile' directive
%which is defined at the beginning of this file.
%Therefore it is also placed there and commented out here.
%</package>
%<*discard>
%\fi
%    \begin{macrocode}
\ifdefined\childdocmain\endinput\fi
%    \end{macrocode}
%\iffalse
%</discard>
%<*package>
%\fi
%
% \macro{\ifchilddoc}
% \macro{\ifchilddocmanual}
% The conditional |\ifchilddoc| tells whether a
% child (true) or main (false) document is being compiled.
% The conditional |\ifchilddocmanual| tells whether
% the |\includeonly| mechanism is used (false) or
% the selection of child files must be performed manually (true).
% The definitions initialise to false:
%    \begin{macrocode}
\newif\ifchilddoc
\newif\ifchilddocmanual
%    \end{macrocode}

% \macro{\childdocname}
% \macro{\childdocjob}
% The macro |\childdocname| stores the name of the main document
% to be compiled. The macro |\childdocjob| stores the name of
% the document on which the \LaTeX{} compiler was originally invoked.
% The content of |\jobname| cannot be compared
% to filenames specified in the source due to different catcodes.
% The following code rescans |\jobname|, stores the result
% in |\childdocname| and saves a copy in |\childdocjob|:
%    \begin{macrocode}
\edef\childdocname{\scantokens\expandafter{\jobname\noexpand}}
\let\childdocjob\childdocname
%    \end{macrocode}

% \macro{\childdocdisable}
% The macro |\childdocdisable| prevents the main file
% from being processed more than once.
% At this stage, the main document command |\childdocmain|
% is assumed to be called once again where it should do nothing.
% Any subsequent call to it should prevent
% a secondary processing of the main document
% It overwrites the forwarding commands
% |\childdocof| and |\childdocforward|
% with empty macros to prevent further inclusions of the main document:
%    \begin{macrocode}
\newcommand{\childdocdisable}
{
  \renewcommand{\childdocmain}[1]{\renewcommand{\childdocmain}[1]{\endinput}}
  \renewcommand{\childdocof}[1]{}
  \renewcommand{\childdocby}[2][]{}
  \renewcommand{\childdocforward}[2][]{}
  \renewcommand{\childdocdisable}{}
}
%    \end{macrocode}

% \macro{\childdocmain}
% The macro |\childdocmain| is to be called at the top of the main file
% with nothing or the main filename (without extension) as argument.
% First, it breaks loops.
% If the argument is not empty and does not match |\childdocname|
% (which is set by the first inclusion of |childdoc.def|),
% |\ifchilddoc| is set to true, |\includeonly| is applied to the child file
% and |\jobname| is set to the main file
% (for proper handling of |.aux| files):
%    \begin{macrocode}
\newcommand{\childdocmain}[1]
{
  \childdocdisable\childdocmain{}
  \if?#1?\else
    \begingroup
      \def\childdoctmp{#1}
      \ifx\childdoctmp\childdocname
        \def\childdoctmp{}
      \else
        \def\childdoctmp
        {
          \childdoctrue
          \includeonly{\childdocname}
          \def\childdocjob{#1}
          \def\jobname{#1}
        }
      \fi
      \expandafter
    \endgroup
    \childdoctmp
  \fi
}
%    \end{macrocode}

% \macro{\childdocof}
% The command |\childdocof| redirects
% compilation to the main file |#1|.
%    \begin{macrocode}
\newcommand{\childdocof}[1]
{
  \childdocdisable
  \childdoctrue
  \includeonly{\childdocname}
  \def\jobname{#1}
  \def\childdocjob{#1}
  \input{#1}
}
%    \end{macrocode}

% \macro{\childdocby}
% The command |\childdocby| ....
%    \begin{macrocode}
\newcommand{\childdocby}[2][]
{
  \childdocdisable
  \childdoctrue
  \childdocmanualtrue
  \if?#1?\else
    \def\jobname{#2}
  \fi
  \def\childdocjob{#2}
  \input{#2}
  \endinput
}
%    \end{macrocode}

% \macro{\childdocforward}
% The command |\childdocforward| redirects
% compilation to the main file or
% (if the optional argument is given) a child file.
% Parameters are set as if the main file
% or a child file starting with |\childdocof| was compiled.
% Then compilation is handed over to the main file:
%    \begin{macrocode}
\newcommand{\childdocforward}[2][]
{
  \begingroup
    \if?#1?
      \def\childdoctmp
      {
        \def\childdocname{#2}
        \def\childdocjob{#2}
        \def\jobname{#2}
        \input{#2}
        \endinput
      }
    \else
      \def\childdoctmp
      {
        \childdocdisable
        \def\childdocname{#2}
        \childdoctrue
        \includeonly{#2}
        \def\childdocjob{#1}
        \def\jobname{#1}
        \input{#1}
        \endinput
      }
    \fi
    \expandafter
  \endgroup
  \childdoctmp
}
%    \end{macrocode}

% \macro{\childdocforwardprefix}
% The command |\childdocforwardprefix| redirects
% compilation to the main or a child file by means of a pattern.
% The prefix |#1| in the current filename is replaced by |#2|
% and the suffix of the current filename is kept
% (it is assumed that the filename does not contain the substring `|~~~|'
% which is used as a delimiter).
% Compilation is handed over to the new file by |\childdocforward|:
%    \begin{macrocode}
\newcommand{\childdocforwardprefix}[3][]
{
  \begingroup
    \def\childdocextract #2##1~~~{\def\childdoctmp{\childdocforward[#1]{#3##1}}}
    \expandafter\childdocextract\childdocname~~~
    \expandafter
  \endgroup
  \childdoctmp
}
%    \end{macrocode}

% \macro{\childdoc}
% The deprecated macro |\childdoc| is a legacy version of |\childdocmain|:
%    \begin{macrocode}
\newcommand{\childdoc}{\childdocmain}
%    \end{macrocode}

% \macro{\childdocredirect}
% The deprecated macro |\childdocredirect| is a legacy version
% of |\childdocforward| and |\childdocforwardprefix|:
%    \begin{macrocode}
\newcommand{\childdocredirect}[2][]
{
  \begingroup
    \if?#1?
      \def\childdoctmp{\childdocforward{#2}}
    \else
      \def\childdoctmp{\childdocforwardprefix{#1}{#2}}
    \fi
    \expandafter
  \endgroup
  \childdoctmp
}
%    \end{macrocode}

%\iffalse
%</package>
%\fi
%
\endinput
|\\
|\childdocforwardprefix{final}{child}|
\end{tabular}
\end{center}
%

Note that when several versions of a main file and/or of each child file
are to be generated, it may be convenient to set up a |Makefile| or
shell script to automatise the process.

%%%%%%%%%%%%%%%%%%%%%%%%%%%%%%%%%%%%%%%%%%%%%%%%%%%%%%%%%%%%%%%%%%%%%%%%%%%%%%%%
\subsection{Command Line Processing}
\label{sec:commandline}

The effect of redirection files can also be achieved by invoking
the \LaTeX{} compiler with a more elaborate command line.
Most conveniently this should be done as part
of a shell script or a |Makefile|.

When using \textsf{childdoc} in the main file, the following
command lines effectively perform a redirection
(note that depending on the shell being used,
backslashes may have to be doubled: `|\|' $\to$ `|\\|'):
%
\begin{center}
|... -jobname "|\textit{target}|" |\\|"|[\textit{flags}]%
|% \iffalse
%
% childdoc.dtx Copyright (C) 2017-2018 Niklas Beisert
%
% This work may be distributed and/or modified under the
% conditions of the LaTeX Project Public License, either version 1.3
% of this license or (at your option) any later version.
% The latest version of this license is in
%   http://www.latex-project.org/lppl.txt
% and version 1.3 or later is part of all distributions of LaTeX
% version 2005/12/01 or later.
%
% This work has the LPPL maintenance status `maintained'.
%
% The Current Maintainer of this work is Niklas Beisert.
%
% This work consists of the files childdoc.dtx and childdoc.ins
% and the derived files childdoc.def and cdocsamp.tex with
% cdocsch1.tex, cdocsch2.tex, cdocsdrf.tex, cdocsfn1.tex, cdocsfn2.tex.
%
%<package>\ifdefined\childdocmain\endinput\fi
%<package>\ProvidesFile{childdoc.def}[2018/12/30 v2.0 child document driver]
%<samplemain>\ProvidesFile{cdocsamp.tex}[2018/12/30 v2.0 sample for childdoc]
%<*driver>
%\ProvidesFile{childdoc.drv}[2018/12/30 v2.0 childdoc reference manual file]
\PassOptionsToClass{10pt,a4paper}{article}
\documentclass{ltxdoc}

\usepackage[margin=35mm]{geometry}
\usepackage{hyperref}
\usepackage{hyperxmp}
\usepackage[usenames]{color}

\hypersetup{colorlinks=true}
\hypersetup{pdfstartview=FitH}
\hypersetup{pdfpagemode=UseNone}
\hypersetup{pdfsource={}}
\hypersetup{pdflang={en-UK}}
\hypersetup{pdfcopyright={Copyright 2017-2018 Niklas Beisert.
  This work may be distributed and/or modified under the
  conditions of the LaTeX Project Public License, either version 1.3
  of this license or (at your option) any later version.}}
\hypersetup{pdflicenseurl={http://www.latex-project.org/lppl.txt}}
\hypersetup{pdfcontactaddress={ETH Zurich, ITP, HIT K,
  Wolfgang-Pauli-Strasse 27}}
\hypersetup{pdfcontactpostcode={8093}}
\hypersetup{pdfcontactcity={Zurich}}
\hypersetup{pdfcontactcountry={Switzerland}}
\hypersetup{pdfcontactemail={nbeisert@itp.phys.ethz.ch}}
\hypersetup{pdfcontacturl={http://people.phys.ethz.ch/\xmptilde nbeisert/}}

\newcommand{\secref}[1]{\hyperref[#1]{section \ref*{#1}}}

\parskip1ex
\parindent0pt
\let\olditemize\itemize
\def\itemize{\olditemize\parskip0pt}

\begin{document}

\title{The \textsf{childdoc} Package}
\hypersetup{pdftitle={The childdoc Package}}
\author{Niklas Beisert\\[2ex]
  Institut f\"ur Theoretische Physik\\
  Eidgen\"ossische Technische Hochschule Z\"urich\\
  Wolfgang-Pauli-Strasse 27, 8093 Z\"urich, Switzerland\\[1ex]
  \href{mailto:nbeisert@itp.phys.ethz.ch}
  {\texttt{nbeisert@itp.phys.ethz.ch}}}
\hypersetup{pdfauthor={Niklas Beisert}}
\hypersetup{pdfsubject={Manual for the LaTeX2e Package childdoc}}
\date{30 December 2018, \textsf{v2.0}}
\maketitle

\begin{abstract}\noindent
\textsf{childdoc} is a \LaTeXe{} package
that enables the direct compilation
of document sections included by |\include|
to individual files.
\end{abstract}

\begingroup
\parskip0ex
\tableofcontents
\endgroup

%%%%%%%%%%%%%%%%%%%%%%%%%%%%%%%%%%%%%%%%%%%%%%%%%%%%%%%%%%%%%%%%%%%%%%%%%%%%%%%%
%%%%%%%%%%%%%%%%%%%%%%%%%%%%%%%%%%%%%%%%%%%%%%%%%%%%%%%%%%%%%%%%%%%%%%%%%%%%%%%%
\section{Introduction}

\LaTeX{} provides a mechanism to structure a large document (such as a book)
into a main file and several child files (containing the chapters)
using the |\include| command.
This mechanism is beneficial for documents
which span hundreds of pages in order to
make the source file(s) more manageable.
Moreover, compilation can be restricted to
selected child files by means of the |\includeonly| command.
The latter feature can be used to reduce the compilation time while editing
(this was significantly more useful in the earlier days of \LaTeX{})
or to generate a smaller document which is easier to navigate.
Another application of |\includeonly| is to generate
documents consisting of selected parts of the complete document.

However, there are a few drawbacks of the plain |\include| mechanism:
\begin{itemize}
\item
The child files cannot be compiled on their own,
they can only be compiled via the main file.
A naive editing environment
(such as a text editor with an option
to have the current file processed by \LaTeX)
may require one to switch to the main file before compiling;
attempting to compile the child file produces errors.
\item
The main file must be modified (each time)
to adjust the |\includeonly| command
to the present needs. This easily leaves the main file in a messy state.
\item
The generated document will always carry the filename
of the main document. This is inconvenient if
several child files are to be compiled and
to be kept for distribution.
\end{itemize}

The present package provides a simple interface
to make child files individually compilable by \LaTeX{}.
Compiling a child file then has the same effect as compiling
the main file with an |\includeonly| command
to select the appropriate child.
Moreover the generated document will carry the name of the child
rather than the main file.
This resolves all three above issues.

This feature is meant to make the editing of books,
thesis documents and lecture notes somewhat more convenient.
However, the package can also be used efficiently for
composing a series of documents (such as exercise sheets)
which are typically distributed individually.
It then assists the author in generating the individual documents
(potentially in different versions)
as well as a document containing the collected series.
Another application is in developing style files
or other kinds of included material
where compilation of the style file could redirect
to a sample or test file.

%%%%%%%%%%%%%%%%%%%%%%%%%%%%%%%%%%%%%%%%%%%%%%%%%%%%%%%%%%%%%%%%%%%%%%%%%%%%%%%%
%%%%%%%%%%%%%%%%%%%%%%%%%%%%%%%%%%%%%%%%%%%%%%%%%%%%%%%%%%%%%%%%%%%%%%%%%%%%%%%%
\section{Usage}

First of all, the package \textsf{childdoc} is \emph{not} a standard
\LaTeXe{} |.sty| style file! Therefore it needs to be invoked in
a non-standard way.

%%%%%%%%%%%%%%%%%%%%%%%%%%%%%%%%%%%%%%%%%%%%%%%%%%%%%%%%%%%%%%%%%%%%%%%%%%%%%%%%
\subsection{Included Files}
\label{sec:include}

%%%%%%%%%%%%%%%%%%%%%%%%%%%%%%%%%%%%%%%%
\DescribeMacro{\childdocmain}
To use the package, add the commands
\begin{center}
\begin{tabular}{l}
|\input{childdoc.def}|\\
|\childdocmain{}|\\
\end{tabular}
\end{center}
at the very top of the main \LaTeX{} file,
in particular \emph{before} the |\documentclass| statement!
The argument of |\childdocmain| should be left empty
(but it must be present).

%%%%%%%%%%%%%%%%%%%%%%%%%%%%%%%%%%%%%%%%
\DescribeMacro{\childdocof}
Furthermore, add the commands
\begin{center}
\begin{tabular}{l}
|\input{childdoc.def}|\\
|\childdocof{|\textit{main}|}|\\
\end{tabular}
\end{center}
at the top of every child file \textit{child}
which is included by |\include{|\textit{child}|}|
from within the main file
(or at least for those files to be compiled individually).
The argument \textit{main} must be the filename of the main file.

There are a couple of
considerations in setting up the main and child documents:

%%%%%%%%%%%%%%%%%%%%%%%%%%%%%%%%%%%%%%%%
\paragraph{Restrictions.}

Please note the following restrictions:
\begin{itemize}
\item
|\childdocmain| must be called with one argument \textit{main}
to ensure compatibility with earlier version of the package.
It must either be empty (|\childdocmain{}|)
or precisely match the filename of the main file in which it is specified.
See \secref{sec:detection} for further information.
\item
The filename \textit{main} must be specified without the |.tex| extension.
\item
The filename \textit{main} is case sensitive
(even in case-insensitive file systems)
due to internal string comparison.
\item
The argument \textit{main} should be fully expanded, it cannot be a macro.
\item
Subdirectories and special characters should be avoided in filenames.
\item
The command |\childdocmain{|\textit{main}|}| must be followed by a whitespace.
It should not be followed immediately by another command
or by a comment mark `|%|'.
This is because the \TeX{} parser reads the token immediately following
the argument of |\childdocmain| and puts it
at the beginning of every child section;
however, a white\-space is ignored.
\end{itemize}

%%%%%%%%%%%%%%%%%%%%%%%%%%%%%%%%%%%%%%%%
\paragraph{Content of Main File.}

It is advisable to place all content in the child files included by |\include|.
Any output contained in the main file will appear in all child documents
unless suppressed manually;
it cannot be suppressed automatically by the |\includeonly| directive
and thus should normally be avoided.
A method to include some content in the main file
by means of conditional processing is described in \secref{sec:conditional}.

%%%%%%%%%%%%%%%%%%%%%%%%%%%%%%%%%%%%%%%%
\paragraph{Page Numbering.}

When only a part of the document is compiled,
the appropriate numbering of pages
(as well as other status parameters)
is determined from the |.aux| files.
The latter contain information from previous passes.
However this information needs to propagate through
all intermediate child documents.
Therefore the page numbering in child documents may well
be inconsistent until the complete document is compiled at least once.

A useful (if unconventional) way to always ensure a consistent
page numbering is to restart the numbering in each child document
and denote the pages by `\textit{child}|.|\textit{page}'
where \textit{child} represents the chapter/section number of the child file.
This can be achieved by the command
|\numberwithin{page}{|\textit{child}|}|
of the \textsf{amsmath} package
where \textit{child} can be |chapter| or |section|
depending on the chosen structuring.
Alternatively, one can modify the macro |\thepage| appropriately
and reset the counter |page| at the start of each child file.

%%%%%%%%%%%%%%%%%%%%%%%%%%%%%%%%%%%%%%%%%%%%%%%%%%%%%%%%%%%%%%%%%%%%%%%%%%%%%%%%
\subsection{Conditional Processing}
\label{sec:conditional}

The package provides a mechanism to compile different versions
of a document. To customise the versions further some conditional processing
can come in handy to distinguish which version is being compiled.
The package provides two macros to describe the compilation context:

%%%%%%%%%%%%%%%%%%%%%%%%%%%%%%%%%%%%%%%%
\DescribeMacro{\ifchilddoc}
The conditional |\ifchilddoc| distinguishes between the compilation of
child documents and the main document:
%
\begin{center}
|\ifchilddoc |\textit{child-code}| |[|\||else |\textit{main-code}]| \||fi|
\end{center}

%%%%%%%%%%%%%%%%%%%%%%%%%%%%%%%%%%%%%%%%
\DescribeMacro{\childdocname}
\DescribeMacro{\childdocjob}
The macro |\childdocname| contains the filename (without extension)
of the main or child file being processed.
Note that |\childdocjob| will always contain the name of the main file.

%%%%%%%%%%%%%%%%%%%%%%%%%%%%%%%%%%%%%%%%
\paragraph{Title Page.}

Conditional processing can be used to include a title or banner page
in the main document when proper precautions are taken.
Importantly, the code in the main file should ensure that the page counter
(as well as other status parameters which are stored in the |.aux| files)
takes the same value after the conditional processing.
Otherwise the page numbers may take divergent values
depending on which part is compiled.

For example, a title page could be declared by:
%
\begin{center}
\begin{tabular}{l}
|\ifchilddoc\||else|\\
|\addtocounter{page}{-1}|\\
\textit{code for title page}\\
|\newpage|\\
|\||fi|
\end{tabular}
\end{center}
%
A banner page for the child documents can be generated by:
%
\begin{center}
\begin{tabular}{l}
|\ifchilddoc|\\
|\addtocounter{page}{-1}|\\
\textit{code for banner page}\\
|\newpage|\\
|\||fi|
\end{tabular}
\end{center}
%
Here one could write a message such as:
\begin{center}
|This is the part \childdocname{} of \childdocjob{}.|
\end{center}

%%%%%%%%%%%%%%%%%%%%%%%%%%%%%%%%%%%%%%%%%%%%%%%%%%%%%%%%%%%%%%%%%%%%%%%%%%%%%%%%
\subsection{Flags}
\label{sec:flags}

The package makes it easy to generate different versions
of the main or child documents.
To this end compilation flags can be defined
and assigned different default values.
They will be particularly useful in conjunction
with the forwarding mechanism described in \secref{sec:forward}.

For example, it may be useful to have a flag |\version|
which can be set to |draft| or |final|.
The document source will contain some conditional code
depending on the value of |\version|.
Suppose further, the flag should default to |final| for the main file
and to |draft| for child files
which is a natural assignment for editing the document.
This is achieved by placing the following code
in the preamble of the main document
(below the |\childdocmain| directive):
%
\begin{center}
\begin{tabular}{l}
|\ifchilddoc|\\
|\providecommand{\version}{draft}|\\
|\||else|\\
|\providecommand{\version}{final}|\\
|\||fi|
\end{tabular}
\end{center}
%
The definition by |\providecommand| makes sure
that previous definitions are not overwritten.
Further statements |\providecommand{\version}{...}|
can thus be added before the above code to override it.

For the main file, one might add a line
(between |\childdocmain| and the above block)
%
\begin{center}
|%\ifchilddoc\||else\providecommand{\version}{draft}\||fi|
\end{center}
%
which can be uncommented to produce a draft version.
Likewise one can add a line to the very top of a child file
(above the |\childdocof{|\textit{main}|}| directive)
%
\begin{center}
|%\providecommand{\version}{final}|
\end{center}
%
which can be uncommented to produce the final version of this child document.

%%%%%%%%%%%%%%%%%%%%%%%%%%%%%%%%%%%%%%%%%%%%%%%%%%%%%%%%%%%%%%%%%%%%%%%%%%%%%%%%
\subsection{Forwarding}
\label{sec:forward}

Different versions of the main or child documents
using compilation flags as described in \secref{sec:flags}
can be (permanently) stored in different files
for convenient compilation, viewing and distribution.
To this end, the package defines a command
to pass on compilation to a different file:

%%%%%%%%%%%%%%%%%%%%%%%%%%%%%%%%%%%%%%%%
\DescribeMacro{\childdocforward}
The command |\childdocforward| redirects processing to
another source file:
%
\begin{center}
\begin{tabular}{l}
|\input{childdoc.def}|\\
|\childdocforward[|\textit{main}|]{|\textit{dest}|}|\\
\end{tabular}
\end{center}
%
The argument \textit{dest} is the destination file
(without extension).
It should be the main file or one of the child files.
Note that further \textsf{childdoc} directives
such as |\childdocof| and |\childdocforward|
in the indicated file will be processed in this form.
The optional argument \textit{main}
passes on directly to the main file \textit{main}
while pretending to compile the child \textit{dest}.
This form behaves as if \textit{dest}
issues |\childdocof{|\textit{main}|}| right away,
and no further \textsf{childdoc} directives will be processed.

%%%%%%%%%%%%%%%%%%%%%%%%%%%%%%%%%%%%%%%%
\DescribeMacro{\...prefix}
In the alternative form |\childdocforwardprefix|,
%
\begin{center}
\begin{tabular}{l}
|\input{childdoc.def}|\\
|\childdocforwardprefix[|\textit{main}|]{|\textit{prefix}|}{|\textit{dest}|}|
\end{tabular}
\end{center}
%
the destination file is determined by a pattern
depending on the current file:
To make this work, the current file must be called
`{\textit{prefix}\hspace{0.2em}\textit{suffix}}'
with \textit{prefix} matching precisely the argument.
Processing is then passed on to the file
`{\textit{dest}\hspace{0.2em}\textit{suffix}}'.
Surely, the same effect is achieved by
directly specifying the
argument `{\textit{dest}\hspace{0.2em}\textit{suffix}}'
in the first form.
However, that requires to set up a different file
for each child. With the alternative form of the command
all these files can have exactly the same content
which simplifies setting them up and maintaining them.

For example, the following file |draft.tex|
with a compilation flag |\version| as described in \secref{sec:flags}
compiles the main document as a draft:
%
\begin{center}
\begin{tabular}{l}
|\def\version{draft}|\\
|\input{childdoc.def}|\\
|\childdocforward{|\textit{main}|}|
\end{tabular}
\end{center}
%
Likewise, the following files |final|\textit{nn}|.tex|
compile the final version of the child document
|child|\textit{nn}|.tex|:
%
\begin{center}
\begin{tabular}{l}
|\def\version{final}|\\
|\input{childdoc.def}|\\
|\childdocforwardprefix{final}{child}|
\end{tabular}
\end{center}
%

Note that when several versions of a main file and/or of each child file
are to be generated, it may be convenient to set up a |Makefile| or
shell script to automatise the process.

%%%%%%%%%%%%%%%%%%%%%%%%%%%%%%%%%%%%%%%%%%%%%%%%%%%%%%%%%%%%%%%%%%%%%%%%%%%%%%%%
\subsection{Command Line Processing}
\label{sec:commandline}

The effect of redirection files can also be achieved by invoking
the \LaTeX{} compiler with a more elaborate command line.
Most conveniently this should be done as part
of a shell script or a |Makefile|.

When using \textsf{childdoc} in the main file, the following
command lines effectively perform a redirection
(note that depending on the shell being used,
backslashes may have to be doubled: `|\|' $\to$ `|\\|'):
%
\begin{center}
|... -jobname "|\textit{target}|" |\\|"|[\textit{flags}]%
|\input{childdoc.def}\childdocforward[|\textit{main}|]{|\textit{dest}|}"|
\end{center}
%
Here \textit{target} is the name of the output file,
\textit{main} is the name of the main file
and \textit{dest} is the name of the main or child file to be processed
(all filenames without extensions).
The optional argument \textit{main} can be omitted
if \textit{main} matches \textit{dest}.
Optionally, compilation \textit{flags} can be defined via |\def| commands.
This command line makes the \TeX{} engine believe
it is compiling the file \textit{target}
whose content is specified as the latter parameter.
The provided code then forwards the processing to
\textit{main} or \textit{dest} as described in \secref{sec:forward}.

%%%%%%%%%%%%%%%%%%%%%%%%%%%%%%%%%%%%%%%%%%%%%%%%%%%%%%%%%%%%%%%%%%%%%%%%%%%%%%%%
\subsection{Include by Input}
\label{sec:input}

Including child documents by |\include| has some restrictions by design.
Most notably, the content of a child document always occupies
its own set of pages; pages cannot be shared between child documents.
Usually, this behaviour makes perfect sense
because each child document contain an essential part of the document.
However, in some situations it may be desirable to compose
a document from a collection of parts
without having mandatory page breaks between then.
For this case, the package
provides a mechanism to include parts
by |\input| which can also be processed individually.
However, by construction this mechanism
requires manual handling of the content to be output.

%%%%%%%%%%%%%%%%%%%%%%%%%%%%%%%%%%%%%%%%
\DescribeMacro{\ifchilddocmanual}
The main file should be prepared as usual, see \secref{sec:include}.
However, the document body must make a distinction
between processing of an individual part and of the main document, e.g.:
%
\begin{center}
\begin{tabular}{l}
|\ifchilddocmanual|\\
|\input{\childdocname}|\\
|\||else|\\
\textit{document body with }|\input{|\textit{part}|}|\\
|\||fi|
\end{tabular}
\end{center}
%
The conditional |\ifchilddocmanual| is true whenever
a part to be included by |\input| is being compiled,
and the name of the part is stored in |\childdocname|.

%%%%%%%%%%%%%%%%%%%%%%%%%%%%%%%%%%%%%%%%
\DescribeMacro{\childdocby}
Each part to be included by |\input| should start with:
%
\begin{center}
\begin{tabular}{l}
|\input{childdoc.def}|\\
|\childdocby{|\textit{main}|}|\\
\end{tabular}
\end{center}
%
The directive |\childdocby| is similar to |\childdocof|
described in \secref{sec:include},
but the subsequent selection of content must be done manually.
To that end, both |\ifchilddoc| and |\ifchilddocmanual|
will be true upon processing of a part,
and the name of the part is stored in |\childdocname|.
Note that |\jobname| will be set to the filename of the current part
so that each part receives an individual |.aux| file
that does not interfere with the |.aux| file(s) of the main document.
This behaviour can be altered by the alternative form
|\childdocby[*]{|\textit{main}|}| (with a non-empty optional argument)
which uses the |.aux| file of the main document
by setting |\jobname| to \textit{main}.

%%%%%%%%%%%%%%%%%%%%%%%%%%%%%%%%%%%%%%%%%%%%%%%%%%%%%%%%%%%%%%%%%%%%%%%%%%%%%%%%
\subsection{Driver Development}
\label{sec:driver}

The \textsf{childdoc} mechanism can also be use for the development
of definition files such as \LaTeX{} styles or classes.
This case differs from the above setup with multiple parts
included by |\include| in that no |\includeonly| should be invoked.
This can be achieved by starting the include file
(before |\ProvidesPackage|) with:
%
\begin{center}
\begin{tabular}{l}
|\input{childdoc.def}|\\
|\childdocforward{|\textit{main}|}|\\
\end{tabular}
\end{center}
%
or alternatively with:
%
\begin{center}
\begin{tabular}{l}
|\input{childdoc.def}|\\
|\childdocby{|\textit{main}|}|\\
\end{tabular}
\end{center}
%
Both forms have slightly different effects as described above.
The main file is prepared as usual, see \secref{sec:include}.

%%%%%%%%%%%%%%%%%%%%%%%%%%%%%%%%%%%%%%%%%%%%%%%%%%%%%%%%%%%%%%%%%%%%%%%%%%%%%%%%
\subsection{Legacy Detection}
\label{sec:detection}

The directive |\childdocmain| in the main file can detect
whether the complete document or merely a child is to be compiled
even without using the directive |\childdocof|.
This method is deprecated because it is less robust
and there is no compelling reason to use it;
it is merely provided for backward compatibility
and it may be removed in future versions.

If the detection mechanism is to be used,
it is mandatory to correctly specify
the filename of the main file as the argument of |\childdocmain|:
%
\begin{center}
\begin{tabular}{l}
|\input{childdoc.def}|\\
|\childdocmain{|\textit{main}|}|\\
\end{tabular}
\end{center}
%
If |\jobname| does not match the argument \textit{main} of |\childdocmain|,
it is assumed that |\jobname| points to the child file to be compiled.
When using |\childdocmain| with the main file specified as argument,
it suffices to start a child file
with just |\input{|\textit{main}|}|
without loading of the package and using |\childdocof|.
If instead all processing is done
with the appropriate \textsf{childdoc} directives,
the argument of \textit{main} of |\childdocmain| can be empty.

An alternative version of the command line processing described
in \secref{sec:commandline} using the detection mechanism reads:
%
\begin{center}
|... -jobname "|\textit{target}|" "|[\textit{flags}]%
[|\def\jobname{|\textit{dest}|}|]|\input{|\textit{main}|}"|
\end{center}

%%%%%%%%%%%%%%%%%%%%%%%%%%%%%%%%%%%%%%%%%%%%%%%%%%%%%%%%%%%%%%%%%%%%%%%%%%%%%%%%
\subsection{Manual Code}
\label{sec:manual}

In case one cannot be certain whether the definitions file |childdoc.def|
is installed on the target \TeX{} distribution
and one prefers not to ship it,
it is conceivable to paste a few relevant commands into the sources.

To that end, drop all statements |\input{childdoc.def}|
and perform the replacements as outlined below.
Instead of |\childdocmain{|\textit{main}|}| add the following code
to the top of the main file:
%
\begin{center}
\begin{tabular}{l}
|\||ifdefined\childdocname\endinput\||fi\newif\ifchilddoc|\\
|\edef\childdocname{\scantokens\expandafter{\jobname\noexpand}}|\\
|\def\childdocmain{|\textit{main}|}\||ifx\childdocmain\childdocname\||else|\\
|\childdoctrue\includeonly{\childdocname}\let\jobname\childdocmain\||fi|\\
\end{tabular}
\end{center}
%
Instead of |\childdocof{|\textit{main}|}| just include the main file
at the top of each child file:
%
\begin{center}
|\input{|\textit{main}|}|
\end{center}
%
A simple redirection |\childdocforward{|\textit{dest}|}| is achieved by:
%
\begin{center}
|\def\jobname{|\textit{dest}|}\input{\jobname}|
\end{center}
%
The redirection with prefix
|\childdocforwardprefix[|\textit{prefix}|]{|\textit{dest}|}|
is accomplished by:
%
\begin{center}
\begin{tabular}{l}
|{\edef\jobname{\scantokens\expandafter{\jobname\noexpand}}|\\
|\def\redirectjob |\textit{prefix}|#1~~~{\gdef\jobname{|\textit{dest}|#1}}|\\
|\expandafter\redirectjob\jobname~~~}\input{\jobname}|
\end{tabular}
\end{center}

In an alternative approach,
child documents can be compiled by a specific command line
without additional code or specific definitions:
%
\begin{center}
|... -jobname "|\textit{target}|" "|[\textit{flags}]%
|\includeonly{|\textit{dest}|}\input{|\textit{main}|}"|
\end{center}
%

%%%%%%%%%%%%%%%%%%%%%%%%%%%%%%%%%%%%%%%%%%%%%%%%%%%%%%%%%%%%%%%%%%%%%%%%%%%%%%%%
%%%%%%%%%%%%%%%%%%%%%%%%%%%%%%%%%%%%%%%%%%%%%%%%%%%%%%%%%%%%%%%%%%%%%%%%%%%%%%%%
\section{Information}

%%%%%%%%%%%%%%%%%%%%%%%%%%%%%%%%%%%%%%%%%%%%%%%%%%%%%%%%%%%%%%%%%%%%%%%%%%%%%%%%
\subsection{Copyright}

Copyright \copyright{} 2017--2018 Niklas Beisert

This work may be distributed and/or modified under the
conditions of the \LaTeX{} Project Public License, either version 1.3
of this license or (at your option) any later version.
The latest version of this license is in
  \url{http://www.latex-project.org/lppl.txt}
and version 1.3 or later is part of all distributions of \LaTeX{}
version 2005/12/01 or later.

This work has the LPPL maintenance status `maintained'.

The Current Maintainer of this work is Niklas Beisert.

This work consists of the files |README.txt|, |childdoc.ins| and |childdoc.dtx|
as well as the derived files |childdoc.def|, |cdocsamp.tex|
with |cdocsch1.tex|, |cdocsch2.tex|, |cdocspt3.tex|, |cdocspt4.tex|,
|cdocsdrf.tex|, |cdocsfn1.tex|, |cdocsfn2.tex|
as well as |childdoc.pdf|.

%%%%%%%%%%%%%%%%%%%%%%%%%%%%%%%%%%%%%%%%%%%%%%%%%%%%%%%%%%%%%%%%%%%%%%%%%%%%%%%%
\subsection{Files and Installation}

The package consists of the files:
%
\begin{center}
\begin{tabular}{ll}
    |README.txt|   & readme file \\
    |childdoc.ins| & installation file \\
    |childdoc.dtx| & source file \\
    |childdoc.def| & definition file \\
    |cdocsamp.tex| & sample main file \\
    |cdocsch1.tex| & sample include file \\
    |cdocsch2.tex| & sample include file \\
    |cdocspt3.tex| & sample part file \\
    |cdocspt4.tex| & sample part file \\
    |cdocsdrf.tex| & sample redirection file \\
    |cdocsfn1.tex| & sample redirection file \\
    |cdocsfn2.tex| & sample redirection file \\
    |childdoc.pdf| & manual
\end{tabular}
\end{center}
%
The distribution consists of the files
|README.txt|, |childdoc.ins| and |childdoc.dtx|.
%
\begin{itemize}
\item
Run (pdf)\LaTeX{} on |childdoc.dtx|
to compile the manual |childdoc.pdf| (this file).
\item
Run \LaTeX{} on |childdoc.ins| to create the definitions file |childdoc.def|
and the sample |cdocsamp.tex| with include files
|cdocsch1.tex|, |cdocsch2.tex|, |cdocspt3.tex|, |cdocspt4.tex|,
|cdocsdrf.tex|, |cdocsfn1.tex|, |cdocsfn2.tex|.
Then copy the file |childdoc.def| to an appropriate directory of your \LaTeX{}
distribution, e.g.\ \textit{texmf-root}|/tex/latex/childdoc|.
\end{itemize}

%%%%%%%%%%%%%%%%%%%%%%%%%%%%%%%%%%%%%%%%%%%%%%%%%%%%%%%%%%%%%%%%%%%%%%%%%%%%%%%%
\subsection{Related CTAN Packages}

There are several other packages which offer a similar functionality:
%
\begin{itemize}
\item
The packages
\href{http://ctan.org/pkg/docmute}{\textsf{docmute}},
\href{http://ctan.org/pkg/includex}{\textsf{includex}} and
\href{http://ctan.org/pkg/standalone}{\textsf{standalone}}
provide commands to include only the document body of
a child file thus allowing both files to be compiled individually.
\item
The packages \href{http://ctan.org/pkg/subdocs}{\textsf{subdocs}}
and \href{http://ctan.org/pkg/subfiles}{\textsf{subfiles}}
provide structures in which the main and child documents can be
encapsulated and allowing them to be compiled individually.
The inclusion mechanism is different from the conventional |\include|.
\item
The package \href{http://ctan.org/pkg/combine}{\textsf{combine}}
is an elaborate solution to combine several documents into one.
\end{itemize}
%
See also the CTAN topic \href{http://ctan.org/topic/subdocs}{\textsf{subdocs}}
for further related packages.
The present package differs from the above solutions in that
a document structure constructed with the conventional |\include| mechanism
just needs two extra commands at the top of every file
such that all constituent files can be compiled individually.

%%%%%%%%%%%%%%%%%%%%%%%%%%%%%%%%%%%%%%%%%%%%%%%%%%%%%%%%%%%%%%%%%%%%%%%%%%%%%%%%
%\subsection{Feature Suggestions}
%
%The following is a list of features which may be useful for future
%versions of this package:
%%
%\begin{itemize}
%\item
%\ldots
%\end{itemize}

%%%%%%%%%%%%%%%%%%%%%%%%%%%%%%%%%%%%%%%%%%%%%%%%%%%%%%%%%%%%%%%%%%%%%%%%%%%%%%%%
\subsection{Revision History}

%%%%%%%%%%%%%%%%%%%%%%%%%%%%%%%%%%%%%%%%
\paragraph{v2.0:} 2018/12/30

\begin{itemize}
\item
immediate forward processing
\item
added |\childdocby| mechanism
\item
manual restructured
\end{itemize}

%%%%%%%%%%%%%%%%%%%%%%%%%%%%%%%%%%%%%%%%
\paragraph{v1.6:} 2018/01/17

\begin{itemize}
\item
application for development of include files
\item
corrections to manual
\end{itemize}

%%%%%%%%%%%%%%%%%%%%%%%%%%%%%%%%%%%%%%%%
\paragraph{v1.5:} 2017/05/21

\begin{itemize}
\item
more complete structuring introduced
\item
|\childdocof| introduced
\item
|\childdoc| renamed to |\childdocmain|
\item
|\childredirect| renamed to |\childdocforward| and |\childdocforwardprefix|
and functionality expanded
\end{itemize}

%%%%%%%%%%%%%%%%%%%%%%%%%%%%%%%%%%%%%%%%
\paragraph{v1.0:} 2017/04/27

\begin{itemize}
\item
manual and install package
\item
first version published on CTAN
\end{itemize}

%%%%%%%%%%%%%%%%%%%%%%%%%%%%%%%%%%%%%%%%
\paragraph{v0.6:} 2017/04/26

\begin{itemize}
\item
redirection mechanism added
\end{itemize}

%%%%%%%%%%%%%%%%%%%%%%%%%%%%%%%%%%%%%%%%
\paragraph{v0.5:} 2017/04/26

\begin{itemize}
\item
functionality in definition file
\end{itemize}


%%%%%%%%%%%%%%%%%%%%%%%%%%%%%%%%%%%%%%%%%%%%%%%%%%%%%%%%%%%%%%%%%%%%%%%%%%%%%%%%
%%%%%%%%%%%%%%%%%%%%%%%%%%%%%%%%%%%%%%%%%%%%%%%%%%%%%%%%%%%%%%%%%%%%%%%%%%%%%%%%
%%%%%%%%%%%%%%%%%%%%%%%%%%%%%%%%%%%%%%%%%%%%%%%%%%%%%%%%%%%%%%%%%%%%%%%%%%%%%%%%
\appendix

\settowidth\MacroIndent{\rmfamily\scriptsize 000\ }

 \DocInput{childdoc.dtx}

\end{document}
%</driver>
% \fi
%
% %%%%%%%%%%%%%%%%%%%%%%%%%%%%%%%%%%%%%%%%%%%%%%%%%%%%%%%%%%%%%%%%%%%%%%%%%%%%%%
% %%%%%%%%%%%%%%%%%%%%%%%%%%%%%%%%%%%%%%%%%%%%%%%%%%%%%%%%%%%%%%%%%%%%%%%%%%%%%%
% \section{Sample}
%\iffalse
%<*samplemain>
%\fi
%
% The following presents a sample document
% with two chapters, two parts, a title page,
% a compile flag as well as three forwarding files to set the flag.
% It consists of eight |.tex| files:
% \begin{center}
% \begin{tabular}{ll}
% |cdocsamp.tex|&main file\\
% |cdocsch1.tex|&include file for chapter 1\\
% |cdocsch2.tex|&include file for chapter 2\\
% |cdocspt3.tex|&include file for part 3\\
% |cdocspt4.tex|&include file for part 4\\
% |cdocsdrf.tex|&forwarding file for main file in draft mode\\
% |cdocsfi1.tex|&forwarding file for final version of chapter 1\\
% |cdocsfi2.tex|&forwarding file for final version of chapter 2\\
% \end{tabular}
% \end{center}
% Each of the eight files can be compiled directly by the \LaTeX{} compiler.
%
% %%%%%%%%%%%%%%%%%%%%%%%%%%%%%%%%%%%%%%
% \paragraph{Main File.}
%
% The main file is called |cdocsamp.tex|.
%
% Load the \textsf{childdoc} definitions and
% declare the filename for the main document:
%    \begin{macrocode}
\input{childdoc.def}
\childdocmain{}
%    \end{macrocode}

% Optional override for |\version| flag:
%    \begin{macrocode}
%%\ifchilddoc\else\providecommand{\version}{draft}\fi
%    \end{macrocode}

% Define the default values for the |\version| flag
% (|final| for the main file and |draft| for childs):
%    \begin{macrocode}
\ifchilddoc
\providecommand{\version}{draft}
\else
\providecommand{\version}{final}
\fi
%    \end{macrocode}

% Load the standard document class:
%    \begin{macrocode}
\documentclass[12pt]{article}
%    \end{macrocode}

% Start the document body:
%    \begin{macrocode}
\begin{document}
%    \end{macrocode}

% Declare a title page.
% Print title, part of document being processed and version flag:
%    \begin{macrocode}
\addtocounter{page}{-1}
\begin{center}
{\LARGE\bfseries{}childdoc example\par}
\vspace{1cm}
\ifchilddoc
\ifchilddocmanual part\else chapter\fi:
`\childdocname' of `\childdocjob'\par
\else
main document: `\childdocjob'\par
\fi
version: \version\par
\end{center}
\newpage
%    \end{macrocode}

% Manually include selected file,
% otherwise process as usual:
%    \begin{macrocode}
\ifchilddocmanual
\section*{part `\childdocname'}
\input{\childdocname}
\else
%    \end{macrocode}

% Include the two chapters:
%    \begin{macrocode}
\include{cdocsch1}
\include{cdocsch2}
%    \end{macrocode}

% Include the two parts unless only chapters should be displayed:
%    \begin{macrocode}
\ifchilddoc\else
\section{part three}
\input{cdocspt3}
\section{part four}
\input{cdocspt4}
\fi
%    \end{macrocode}

% Process as usual until here:
%    \begin{macrocode}
\fi
%    \end{macrocode}

% End of document body:
%    \begin{macrocode}
\end{document}
%    \end{macrocode}
%\iffalse
%</samplemain>
%\fi
%
% %%%%%%%%%%%%%%%%%%%%%%%%%%%%%%%%%%%%%%
% \paragraph{Chapter Include Files.}
%
% The include files are called |cdocsch1.tex| and |cdocsch2.tex|.
%
%\iffalse
%<*samplechap1|samplechap2>
%\fi

% Optional override for |\version| flag:
%    \begin{macrocode}
%%\providecommand{\version}{final}
%    \end{macrocode}

% Include the main document:
%    \begin{macrocode}
\input{childdoc.def}
\childdocof{cdocsamp}
%    \end{macrocode}

%\iffalse
%</samplechap1|samplechap2>
%\fi
%
%\iffalse
%<*samplechap1>
%\fi
% Some text for chapter 1:
%    \begin{macrocode}
\section{one}
some text in chapter one
%    \end{macrocode}

%\iffalse
%</samplechap1>
%\fi
% Some text for chapter 2:
%\iffalse
%<*samplechap2>
%\fi
%    \begin{macrocode}
\section{two}
more text in chapter two
%    \end{macrocode}

%\iffalse
%</samplechap2>
%\fi
%
% %%%%%%%%%%%%%%%%%%%%%%%%%%%%%%%%%%%%%%
% \paragraph{Part Include Files.}
%
% The include files are called |cdocspt3.tex| and |cdocspt4.tex|.
%
%\iffalse
%<*samplepart3|samplepart4>
%\fi

% Optional override for |\version| flag:
%    \begin{macrocode}
%%\providecommand{\version}{final}
%    \end{macrocode}

% Include the main document:
%    \begin{macrocode}
\input{childdoc.def}
\childdocby{cdocsamp}
%    \end{macrocode}

%\iffalse
%</samplepart3|samplepart4>
%\fi
%
%\iffalse
%<*samplepart3>
%\fi
% Some text for part 3:
%    \begin{macrocode}
some text in part three
%    \end{macrocode}

%\iffalse
%</samplepart3>
%\fi
% Some text for part 4:
%\iffalse
%<*samplepart4>
%\fi
%    \begin{macrocode}
more text in part four
%    \end{macrocode}

%\iffalse
%</samplepart4>
%\fi
%
% %%%%%%%%%%%%%%%%%%%%%%%%%%%%%%%%%%%%%%
% \paragraph{Forwarding for a Complete Draft.}
%
% The following forwarding file |cdocsdrf.tex|
% compiles the main document in draft mode:
%\iffalse
%<*sampledraft>
%\fi
%    \begin{macrocode}
\def\version{draft}
\input{childdoc.def}
\childdocforward{cdocsamp}
%    \end{macrocode}

%\iffalse
%</sampledraft>
%\fi
%
% %%%%%%%%%%%%%%%%%%%%%%%%%%%%%%%%%%%%%%
% \paragraph{Forwarding for Final Version of the Chapters.}
%
% The following forwarding files |cdocsfn1.tex| and |cdocsfn2.tex|
% (with identical content)
% compile the final versions of the child documents
% |cdocsch1.tex| and |cdocsch2.tex|, respectively:
%\iffalse
%<*samplefinal>
%\fi
%    \begin{macrocode}
\def\version{final}
\input{childdoc.def}
\childdocforwardprefix[cdocsamp]{cdocsfn}{cdocsch}
%    \end{macrocode}

%\iffalse
%</samplefinal>
%\fi
%
% %%%%%%%%%%%%%%%%%%%%%%%%%%%%%%%%%%%%%%
% \paragraph{Command Line Processing.}
%
% The following three command lines generate the output files
% |cdocscld|, |cdocscl1| and |cdocscl2|
% which should be identical to
% |cdocsdrf|, |cdocsch1| and |cdocsfn2|, respectively:
% \begin{center}
% \begin{tabular}{l}
% |latex -jobname cdocscld \|\\
% |  "\def\version{draft}\input{childdoc.def}\childdocforward{cdocsamp}"|\\
% |latex -jobname cdocscl1 \|\\
% |  "\input{childdoc.def}\childdocforward[cdocsamp]{cdocsch1}"|\\
% |latex -jobname cdocscl2 \|\\
% |  "\def\version{final}\input{childdoc.def}\childdocforward{cdocsch2}"|
% \end{tabular}
% \end{center}
% Note that the trailing backslash on each first line
% merely continues the input to the second line
% (for convenient cut ant paste).
% Furthermore, the command |latex| can be replaced by any
% of its alternative versions such as |pdflatex|.
%
% %%%%%%%%%%%%%%%%%%%%%%%%%%%%%%%%%%%%%%%%%%%%%%%%%%%%%%%%%%%%%%%%%%%%%%%%%%%%%%
% %%%%%%%%%%%%%%%%%%%%%%%%%%%%%%%%%%%%%%%%%%%%%%%%%%%%%%%%%%%%%%%%%%%%%%%%%%%%%%
% \section{Implementation}
%\iffalse
%<*package>
%\fi
%
% This section describes the definitions file |childdoc.def|.

% The definitions cannot be loaded using |\usepackage| or |\RequirePackage|
% which has a mechanism to prevent loading a style file more than once.
% When loading the definitions by means of |\input|
% multiple instances have to be prevented manually:
%\iffalse
%This code needs to be before the `\ProvidesFile' directive
%which is defined at the beginning of this file.
%Therefore it is also placed there and commented out here.
%</package>
%<*discard>
%\fi
%    \begin{macrocode}
\ifdefined\childdocmain\endinput\fi
%    \end{macrocode}
%\iffalse
%</discard>
%<*package>
%\fi
%
% \macro{\ifchilddoc}
% \macro{\ifchilddocmanual}
% The conditional |\ifchilddoc| tells whether a
% child (true) or main (false) document is being compiled.
% The conditional |\ifchilddocmanual| tells whether
% the |\includeonly| mechanism is used (false) or
% the selection of child files must be performed manually (true).
% The definitions initialise to false:
%    \begin{macrocode}
\newif\ifchilddoc
\newif\ifchilddocmanual
%    \end{macrocode}

% \macro{\childdocname}
% \macro{\childdocjob}
% The macro |\childdocname| stores the name of the main document
% to be compiled. The macro |\childdocjob| stores the name of
% the document on which the \LaTeX{} compiler was originally invoked.
% The content of |\jobname| cannot be compared
% to filenames specified in the source due to different catcodes.
% The following code rescans |\jobname|, stores the result
% in |\childdocname| and saves a copy in |\childdocjob|:
%    \begin{macrocode}
\edef\childdocname{\scantokens\expandafter{\jobname\noexpand}}
\let\childdocjob\childdocname
%    \end{macrocode}

% \macro{\childdocdisable}
% The macro |\childdocdisable| prevents the main file
% from being processed more than once.
% At this stage, the main document command |\childdocmain|
% is assumed to be called once again where it should do nothing.
% Any subsequent call to it should prevent
% a secondary processing of the main document
% It overwrites the forwarding commands
% |\childdocof| and |\childdocforward|
% with empty macros to prevent further inclusions of the main document:
%    \begin{macrocode}
\newcommand{\childdocdisable}
{
  \renewcommand{\childdocmain}[1]{\renewcommand{\childdocmain}[1]{\endinput}}
  \renewcommand{\childdocof}[1]{}
  \renewcommand{\childdocby}[2][]{}
  \renewcommand{\childdocforward}[2][]{}
  \renewcommand{\childdocdisable}{}
}
%    \end{macrocode}

% \macro{\childdocmain}
% The macro |\childdocmain| is to be called at the top of the main file
% with nothing or the main filename (without extension) as argument.
% First, it breaks loops.
% If the argument is not empty and does not match |\childdocname|
% (which is set by the first inclusion of |childdoc.def|),
% |\ifchilddoc| is set to true, |\includeonly| is applied to the child file
% and |\jobname| is set to the main file
% (for proper handling of |.aux| files):
%    \begin{macrocode}
\newcommand{\childdocmain}[1]
{
  \childdocdisable\childdocmain{}
  \if?#1?\else
    \begingroup
      \def\childdoctmp{#1}
      \ifx\childdoctmp\childdocname
        \def\childdoctmp{}
      \else
        \def\childdoctmp
        {
          \childdoctrue
          \includeonly{\childdocname}
          \def\childdocjob{#1}
          \def\jobname{#1}
        }
      \fi
      \expandafter
    \endgroup
    \childdoctmp
  \fi
}
%    \end{macrocode}

% \macro{\childdocof}
% The command |\childdocof| redirects
% compilation to the main file |#1|.
%    \begin{macrocode}
\newcommand{\childdocof}[1]
{
  \childdocdisable
  \childdoctrue
  \includeonly{\childdocname}
  \def\jobname{#1}
  \def\childdocjob{#1}
  \input{#1}
}
%    \end{macrocode}

% \macro{\childdocby}
% The command |\childdocby| ....
%    \begin{macrocode}
\newcommand{\childdocby}[2][]
{
  \childdocdisable
  \childdoctrue
  \childdocmanualtrue
  \if?#1?\else
    \def\jobname{#2}
  \fi
  \def\childdocjob{#2}
  \input{#2}
  \endinput
}
%    \end{macrocode}

% \macro{\childdocforward}
% The command |\childdocforward| redirects
% compilation to the main file or
% (if the optional argument is given) a child file.
% Parameters are set as if the main file
% or a child file starting with |\childdocof| was compiled.
% Then compilation is handed over to the main file:
%    \begin{macrocode}
\newcommand{\childdocforward}[2][]
{
  \begingroup
    \if?#1?
      \def\childdoctmp
      {
        \def\childdocname{#2}
        \def\childdocjob{#2}
        \def\jobname{#2}
        \input{#2}
        \endinput
      }
    \else
      \def\childdoctmp
      {
        \childdocdisable
        \def\childdocname{#2}
        \childdoctrue
        \includeonly{#2}
        \def\childdocjob{#1}
        \def\jobname{#1}
        \input{#1}
        \endinput
      }
    \fi
    \expandafter
  \endgroup
  \childdoctmp
}
%    \end{macrocode}

% \macro{\childdocforwardprefix}
% The command |\childdocforwardprefix| redirects
% compilation to the main or a child file by means of a pattern.
% The prefix |#1| in the current filename is replaced by |#2|
% and the suffix of the current filename is kept
% (it is assumed that the filename does not contain the substring `|~~~|'
% which is used as a delimiter).
% Compilation is handed over to the new file by |\childdocforward|:
%    \begin{macrocode}
\newcommand{\childdocforwardprefix}[3][]
{
  \begingroup
    \def\childdocextract #2##1~~~{\def\childdoctmp{\childdocforward[#1]{#3##1}}}
    \expandafter\childdocextract\childdocname~~~
    \expandafter
  \endgroup
  \childdoctmp
}
%    \end{macrocode}

% \macro{\childdoc}
% The deprecated macro |\childdoc| is a legacy version of |\childdocmain|:
%    \begin{macrocode}
\newcommand{\childdoc}{\childdocmain}
%    \end{macrocode}

% \macro{\childdocredirect}
% The deprecated macro |\childdocredirect| is a legacy version
% of |\childdocforward| and |\childdocforwardprefix|:
%    \begin{macrocode}
\newcommand{\childdocredirect}[2][]
{
  \begingroup
    \if?#1?
      \def\childdoctmp{\childdocforward{#2}}
    \else
      \def\childdoctmp{\childdocforwardprefix{#1}{#2}}
    \fi
    \expandafter
  \endgroup
  \childdoctmp
}
%    \end{macrocode}

%\iffalse
%</package>
%\fi
%
\endinput
\childdocforward[|\textit{main}|]{|\textit{dest}|}"|
\end{center}
%
Here \textit{target} is the name of the output file,
\textit{main} is the name of the main file
and \textit{dest} is the name of the main or child file to be processed
(all filenames without extensions).
The optional argument \textit{main} can be omitted
if \textit{main} matches \textit{dest}.
Optionally, compilation \textit{flags} can be defined via |\def| commands.
This command line makes the \TeX{} engine believe
it is compiling the file \textit{target}
whose content is specified as the latter parameter.
The provided code then forwards the processing to
\textit{main} or \textit{dest} as described in \secref{sec:forward}.

%%%%%%%%%%%%%%%%%%%%%%%%%%%%%%%%%%%%%%%%%%%%%%%%%%%%%%%%%%%%%%%%%%%%%%%%%%%%%%%%
\subsection{Include by Input}
\label{sec:input}

Including child documents by |\include| has some restrictions by design.
Most notably, the content of a child document always occupies
its own set of pages; pages cannot be shared between child documents.
Usually, this behaviour makes perfect sense
because each child document contain an essential part of the document.
However, in some situations it may be desirable to compose
a document from a collection of parts
without having mandatory page breaks between then.
For this case, the package
provides a mechanism to include parts
by |\input| which can also be processed individually.
However, by construction this mechanism
requires manual handling of the content to be output.

%%%%%%%%%%%%%%%%%%%%%%%%%%%%%%%%%%%%%%%%
\DescribeMacro{\ifchilddocmanual}
The main file should be prepared as usual, see \secref{sec:include}.
However, the document body must make a distinction
between processing of an individual part and of the main document, e.g.:
%
\begin{center}
\begin{tabular}{l}
|\ifchilddocmanual|\\
|\input{\childdocname}|\\
|\||else|\\
\textit{document body with }|\input{|\textit{part}|}|\\
|\||fi|
\end{tabular}
\end{center}
%
The conditional |\ifchilddocmanual| is true whenever
a part to be included by |\input| is being compiled,
and the name of the part is stored in |\childdocname|.

%%%%%%%%%%%%%%%%%%%%%%%%%%%%%%%%%%%%%%%%
\DescribeMacro{\childdocby}
Each part to be included by |\input| should start with:
%
\begin{center}
\begin{tabular}{l}
|% \iffalse
%
% childdoc.dtx Copyright (C) 2017-2018 Niklas Beisert
%
% This work may be distributed and/or modified under the
% conditions of the LaTeX Project Public License, either version 1.3
% of this license or (at your option) any later version.
% The latest version of this license is in
%   http://www.latex-project.org/lppl.txt
% and version 1.3 or later is part of all distributions of LaTeX
% version 2005/12/01 or later.
%
% This work has the LPPL maintenance status `maintained'.
%
% The Current Maintainer of this work is Niklas Beisert.
%
% This work consists of the files childdoc.dtx and childdoc.ins
% and the derived files childdoc.def and cdocsamp.tex with
% cdocsch1.tex, cdocsch2.tex, cdocsdrf.tex, cdocsfn1.tex, cdocsfn2.tex.
%
%<package>\ifdefined\childdocmain\endinput\fi
%<package>\ProvidesFile{childdoc.def}[2018/12/30 v2.0 child document driver]
%<samplemain>\ProvidesFile{cdocsamp.tex}[2018/12/30 v2.0 sample for childdoc]
%<*driver>
%\ProvidesFile{childdoc.drv}[2018/12/30 v2.0 childdoc reference manual file]
\PassOptionsToClass{10pt,a4paper}{article}
\documentclass{ltxdoc}

\usepackage[margin=35mm]{geometry}
\usepackage{hyperref}
\usepackage{hyperxmp}
\usepackage[usenames]{color}

\hypersetup{colorlinks=true}
\hypersetup{pdfstartview=FitH}
\hypersetup{pdfpagemode=UseNone}
\hypersetup{pdfsource={}}
\hypersetup{pdflang={en-UK}}
\hypersetup{pdfcopyright={Copyright 2017-2018 Niklas Beisert.
  This work may be distributed and/or modified under the
  conditions of the LaTeX Project Public License, either version 1.3
  of this license or (at your option) any later version.}}
\hypersetup{pdflicenseurl={http://www.latex-project.org/lppl.txt}}
\hypersetup{pdfcontactaddress={ETH Zurich, ITP, HIT K,
  Wolfgang-Pauli-Strasse 27}}
\hypersetup{pdfcontactpostcode={8093}}
\hypersetup{pdfcontactcity={Zurich}}
\hypersetup{pdfcontactcountry={Switzerland}}
\hypersetup{pdfcontactemail={nbeisert@itp.phys.ethz.ch}}
\hypersetup{pdfcontacturl={http://people.phys.ethz.ch/\xmptilde nbeisert/}}

\newcommand{\secref}[1]{\hyperref[#1]{section \ref*{#1}}}

\parskip1ex
\parindent0pt
\let\olditemize\itemize
\def\itemize{\olditemize\parskip0pt}

\begin{document}

\title{The \textsf{childdoc} Package}
\hypersetup{pdftitle={The childdoc Package}}
\author{Niklas Beisert\\[2ex]
  Institut f\"ur Theoretische Physik\\
  Eidgen\"ossische Technische Hochschule Z\"urich\\
  Wolfgang-Pauli-Strasse 27, 8093 Z\"urich, Switzerland\\[1ex]
  \href{mailto:nbeisert@itp.phys.ethz.ch}
  {\texttt{nbeisert@itp.phys.ethz.ch}}}
\hypersetup{pdfauthor={Niklas Beisert}}
\hypersetup{pdfsubject={Manual for the LaTeX2e Package childdoc}}
\date{30 December 2018, \textsf{v2.0}}
\maketitle

\begin{abstract}\noindent
\textsf{childdoc} is a \LaTeXe{} package
that enables the direct compilation
of document sections included by |\include|
to individual files.
\end{abstract}

\begingroup
\parskip0ex
\tableofcontents
\endgroup

%%%%%%%%%%%%%%%%%%%%%%%%%%%%%%%%%%%%%%%%%%%%%%%%%%%%%%%%%%%%%%%%%%%%%%%%%%%%%%%%
%%%%%%%%%%%%%%%%%%%%%%%%%%%%%%%%%%%%%%%%%%%%%%%%%%%%%%%%%%%%%%%%%%%%%%%%%%%%%%%%
\section{Introduction}

\LaTeX{} provides a mechanism to structure a large document (such as a book)
into a main file and several child files (containing the chapters)
using the |\include| command.
This mechanism is beneficial for documents
which span hundreds of pages in order to
make the source file(s) more manageable.
Moreover, compilation can be restricted to
selected child files by means of the |\includeonly| command.
The latter feature can be used to reduce the compilation time while editing
(this was significantly more useful in the earlier days of \LaTeX{})
or to generate a smaller document which is easier to navigate.
Another application of |\includeonly| is to generate
documents consisting of selected parts of the complete document.

However, there are a few drawbacks of the plain |\include| mechanism:
\begin{itemize}
\item
The child files cannot be compiled on their own,
they can only be compiled via the main file.
A naive editing environment
(such as a text editor with an option
to have the current file processed by \LaTeX)
may require one to switch to the main file before compiling;
attempting to compile the child file produces errors.
\item
The main file must be modified (each time)
to adjust the |\includeonly| command
to the present needs. This easily leaves the main file in a messy state.
\item
The generated document will always carry the filename
of the main document. This is inconvenient if
several child files are to be compiled and
to be kept for distribution.
\end{itemize}

The present package provides a simple interface
to make child files individually compilable by \LaTeX{}.
Compiling a child file then has the same effect as compiling
the main file with an |\includeonly| command
to select the appropriate child.
Moreover the generated document will carry the name of the child
rather than the main file.
This resolves all three above issues.

This feature is meant to make the editing of books,
thesis documents and lecture notes somewhat more convenient.
However, the package can also be used efficiently for
composing a series of documents (such as exercise sheets)
which are typically distributed individually.
It then assists the author in generating the individual documents
(potentially in different versions)
as well as a document containing the collected series.
Another application is in developing style files
or other kinds of included material
where compilation of the style file could redirect
to a sample or test file.

%%%%%%%%%%%%%%%%%%%%%%%%%%%%%%%%%%%%%%%%%%%%%%%%%%%%%%%%%%%%%%%%%%%%%%%%%%%%%%%%
%%%%%%%%%%%%%%%%%%%%%%%%%%%%%%%%%%%%%%%%%%%%%%%%%%%%%%%%%%%%%%%%%%%%%%%%%%%%%%%%
\section{Usage}

First of all, the package \textsf{childdoc} is \emph{not} a standard
\LaTeXe{} |.sty| style file! Therefore it needs to be invoked in
a non-standard way.

%%%%%%%%%%%%%%%%%%%%%%%%%%%%%%%%%%%%%%%%%%%%%%%%%%%%%%%%%%%%%%%%%%%%%%%%%%%%%%%%
\subsection{Included Files}
\label{sec:include}

%%%%%%%%%%%%%%%%%%%%%%%%%%%%%%%%%%%%%%%%
\DescribeMacro{\childdocmain}
To use the package, add the commands
\begin{center}
\begin{tabular}{l}
|\input{childdoc.def}|\\
|\childdocmain{}|\\
\end{tabular}
\end{center}
at the very top of the main \LaTeX{} file,
in particular \emph{before} the |\documentclass| statement!
The argument of |\childdocmain| should be left empty
(but it must be present).

%%%%%%%%%%%%%%%%%%%%%%%%%%%%%%%%%%%%%%%%
\DescribeMacro{\childdocof}
Furthermore, add the commands
\begin{center}
\begin{tabular}{l}
|\input{childdoc.def}|\\
|\childdocof{|\textit{main}|}|\\
\end{tabular}
\end{center}
at the top of every child file \textit{child}
which is included by |\include{|\textit{child}|}|
from within the main file
(or at least for those files to be compiled individually).
The argument \textit{main} must be the filename of the main file.

There are a couple of
considerations in setting up the main and child documents:

%%%%%%%%%%%%%%%%%%%%%%%%%%%%%%%%%%%%%%%%
\paragraph{Restrictions.}

Please note the following restrictions:
\begin{itemize}
\item
|\childdocmain| must be called with one argument \textit{main}
to ensure compatibility with earlier version of the package.
It must either be empty (|\childdocmain{}|)
or precisely match the filename of the main file in which it is specified.
See \secref{sec:detection} for further information.
\item
The filename \textit{main} must be specified without the |.tex| extension.
\item
The filename \textit{main} is case sensitive
(even in case-insensitive file systems)
due to internal string comparison.
\item
The argument \textit{main} should be fully expanded, it cannot be a macro.
\item
Subdirectories and special characters should be avoided in filenames.
\item
The command |\childdocmain{|\textit{main}|}| must be followed by a whitespace.
It should not be followed immediately by another command
or by a comment mark `|%|'.
This is because the \TeX{} parser reads the token immediately following
the argument of |\childdocmain| and puts it
at the beginning of every child section;
however, a white\-space is ignored.
\end{itemize}

%%%%%%%%%%%%%%%%%%%%%%%%%%%%%%%%%%%%%%%%
\paragraph{Content of Main File.}

It is advisable to place all content in the child files included by |\include|.
Any output contained in the main file will appear in all child documents
unless suppressed manually;
it cannot be suppressed automatically by the |\includeonly| directive
and thus should normally be avoided.
A method to include some content in the main file
by means of conditional processing is described in \secref{sec:conditional}.

%%%%%%%%%%%%%%%%%%%%%%%%%%%%%%%%%%%%%%%%
\paragraph{Page Numbering.}

When only a part of the document is compiled,
the appropriate numbering of pages
(as well as other status parameters)
is determined from the |.aux| files.
The latter contain information from previous passes.
However this information needs to propagate through
all intermediate child documents.
Therefore the page numbering in child documents may well
be inconsistent until the complete document is compiled at least once.

A useful (if unconventional) way to always ensure a consistent
page numbering is to restart the numbering in each child document
and denote the pages by `\textit{child}|.|\textit{page}'
where \textit{child} represents the chapter/section number of the child file.
This can be achieved by the command
|\numberwithin{page}{|\textit{child}|}|
of the \textsf{amsmath} package
where \textit{child} can be |chapter| or |section|
depending on the chosen structuring.
Alternatively, one can modify the macro |\thepage| appropriately
and reset the counter |page| at the start of each child file.

%%%%%%%%%%%%%%%%%%%%%%%%%%%%%%%%%%%%%%%%%%%%%%%%%%%%%%%%%%%%%%%%%%%%%%%%%%%%%%%%
\subsection{Conditional Processing}
\label{sec:conditional}

The package provides a mechanism to compile different versions
of a document. To customise the versions further some conditional processing
can come in handy to distinguish which version is being compiled.
The package provides two macros to describe the compilation context:

%%%%%%%%%%%%%%%%%%%%%%%%%%%%%%%%%%%%%%%%
\DescribeMacro{\ifchilddoc}
The conditional |\ifchilddoc| distinguishes between the compilation of
child documents and the main document:
%
\begin{center}
|\ifchilddoc |\textit{child-code}| |[|\||else |\textit{main-code}]| \||fi|
\end{center}

%%%%%%%%%%%%%%%%%%%%%%%%%%%%%%%%%%%%%%%%
\DescribeMacro{\childdocname}
\DescribeMacro{\childdocjob}
The macro |\childdocname| contains the filename (without extension)
of the main or child file being processed.
Note that |\childdocjob| will always contain the name of the main file.

%%%%%%%%%%%%%%%%%%%%%%%%%%%%%%%%%%%%%%%%
\paragraph{Title Page.}

Conditional processing can be used to include a title or banner page
in the main document when proper precautions are taken.
Importantly, the code in the main file should ensure that the page counter
(as well as other status parameters which are stored in the |.aux| files)
takes the same value after the conditional processing.
Otherwise the page numbers may take divergent values
depending on which part is compiled.

For example, a title page could be declared by:
%
\begin{center}
\begin{tabular}{l}
|\ifchilddoc\||else|\\
|\addtocounter{page}{-1}|\\
\textit{code for title page}\\
|\newpage|\\
|\||fi|
\end{tabular}
\end{center}
%
A banner page for the child documents can be generated by:
%
\begin{center}
\begin{tabular}{l}
|\ifchilddoc|\\
|\addtocounter{page}{-1}|\\
\textit{code for banner page}\\
|\newpage|\\
|\||fi|
\end{tabular}
\end{center}
%
Here one could write a message such as:
\begin{center}
|This is the part \childdocname{} of \childdocjob{}.|
\end{center}

%%%%%%%%%%%%%%%%%%%%%%%%%%%%%%%%%%%%%%%%%%%%%%%%%%%%%%%%%%%%%%%%%%%%%%%%%%%%%%%%
\subsection{Flags}
\label{sec:flags}

The package makes it easy to generate different versions
of the main or child documents.
To this end compilation flags can be defined
and assigned different default values.
They will be particularly useful in conjunction
with the forwarding mechanism described in \secref{sec:forward}.

For example, it may be useful to have a flag |\version|
which can be set to |draft| or |final|.
The document source will contain some conditional code
depending on the value of |\version|.
Suppose further, the flag should default to |final| for the main file
and to |draft| for child files
which is a natural assignment for editing the document.
This is achieved by placing the following code
in the preamble of the main document
(below the |\childdocmain| directive):
%
\begin{center}
\begin{tabular}{l}
|\ifchilddoc|\\
|\providecommand{\version}{draft}|\\
|\||else|\\
|\providecommand{\version}{final}|\\
|\||fi|
\end{tabular}
\end{center}
%
The definition by |\providecommand| makes sure
that previous definitions are not overwritten.
Further statements |\providecommand{\version}{...}|
can thus be added before the above code to override it.

For the main file, one might add a line
(between |\childdocmain| and the above block)
%
\begin{center}
|%\ifchilddoc\||else\providecommand{\version}{draft}\||fi|
\end{center}
%
which can be uncommented to produce a draft version.
Likewise one can add a line to the very top of a child file
(above the |\childdocof{|\textit{main}|}| directive)
%
\begin{center}
|%\providecommand{\version}{final}|
\end{center}
%
which can be uncommented to produce the final version of this child document.

%%%%%%%%%%%%%%%%%%%%%%%%%%%%%%%%%%%%%%%%%%%%%%%%%%%%%%%%%%%%%%%%%%%%%%%%%%%%%%%%
\subsection{Forwarding}
\label{sec:forward}

Different versions of the main or child documents
using compilation flags as described in \secref{sec:flags}
can be (permanently) stored in different files
for convenient compilation, viewing and distribution.
To this end, the package defines a command
to pass on compilation to a different file:

%%%%%%%%%%%%%%%%%%%%%%%%%%%%%%%%%%%%%%%%
\DescribeMacro{\childdocforward}
The command |\childdocforward| redirects processing to
another source file:
%
\begin{center}
\begin{tabular}{l}
|\input{childdoc.def}|\\
|\childdocforward[|\textit{main}|]{|\textit{dest}|}|\\
\end{tabular}
\end{center}
%
The argument \textit{dest} is the destination file
(without extension).
It should be the main file or one of the child files.
Note that further \textsf{childdoc} directives
such as |\childdocof| and |\childdocforward|
in the indicated file will be processed in this form.
The optional argument \textit{main}
passes on directly to the main file \textit{main}
while pretending to compile the child \textit{dest}.
This form behaves as if \textit{dest}
issues |\childdocof{|\textit{main}|}| right away,
and no further \textsf{childdoc} directives will be processed.

%%%%%%%%%%%%%%%%%%%%%%%%%%%%%%%%%%%%%%%%
\DescribeMacro{\...prefix}
In the alternative form |\childdocforwardprefix|,
%
\begin{center}
\begin{tabular}{l}
|\input{childdoc.def}|\\
|\childdocforwardprefix[|\textit{main}|]{|\textit{prefix}|}{|\textit{dest}|}|
\end{tabular}
\end{center}
%
the destination file is determined by a pattern
depending on the current file:
To make this work, the current file must be called
`{\textit{prefix}\hspace{0.2em}\textit{suffix}}'
with \textit{prefix} matching precisely the argument.
Processing is then passed on to the file
`{\textit{dest}\hspace{0.2em}\textit{suffix}}'.
Surely, the same effect is achieved by
directly specifying the
argument `{\textit{dest}\hspace{0.2em}\textit{suffix}}'
in the first form.
However, that requires to set up a different file
for each child. With the alternative form of the command
all these files can have exactly the same content
which simplifies setting them up and maintaining them.

For example, the following file |draft.tex|
with a compilation flag |\version| as described in \secref{sec:flags}
compiles the main document as a draft:
%
\begin{center}
\begin{tabular}{l}
|\def\version{draft}|\\
|\input{childdoc.def}|\\
|\childdocforward{|\textit{main}|}|
\end{tabular}
\end{center}
%
Likewise, the following files |final|\textit{nn}|.tex|
compile the final version of the child document
|child|\textit{nn}|.tex|:
%
\begin{center}
\begin{tabular}{l}
|\def\version{final}|\\
|\input{childdoc.def}|\\
|\childdocforwardprefix{final}{child}|
\end{tabular}
\end{center}
%

Note that when several versions of a main file and/or of each child file
are to be generated, it may be convenient to set up a |Makefile| or
shell script to automatise the process.

%%%%%%%%%%%%%%%%%%%%%%%%%%%%%%%%%%%%%%%%%%%%%%%%%%%%%%%%%%%%%%%%%%%%%%%%%%%%%%%%
\subsection{Command Line Processing}
\label{sec:commandline}

The effect of redirection files can also be achieved by invoking
the \LaTeX{} compiler with a more elaborate command line.
Most conveniently this should be done as part
of a shell script or a |Makefile|.

When using \textsf{childdoc} in the main file, the following
command lines effectively perform a redirection
(note that depending on the shell being used,
backslashes may have to be doubled: `|\|' $\to$ `|\\|'):
%
\begin{center}
|... -jobname "|\textit{target}|" |\\|"|[\textit{flags}]%
|\input{childdoc.def}\childdocforward[|\textit{main}|]{|\textit{dest}|}"|
\end{center}
%
Here \textit{target} is the name of the output file,
\textit{main} is the name of the main file
and \textit{dest} is the name of the main or child file to be processed
(all filenames without extensions).
The optional argument \textit{main} can be omitted
if \textit{main} matches \textit{dest}.
Optionally, compilation \textit{flags} can be defined via |\def| commands.
This command line makes the \TeX{} engine believe
it is compiling the file \textit{target}
whose content is specified as the latter parameter.
The provided code then forwards the processing to
\textit{main} or \textit{dest} as described in \secref{sec:forward}.

%%%%%%%%%%%%%%%%%%%%%%%%%%%%%%%%%%%%%%%%%%%%%%%%%%%%%%%%%%%%%%%%%%%%%%%%%%%%%%%%
\subsection{Include by Input}
\label{sec:input}

Including child documents by |\include| has some restrictions by design.
Most notably, the content of a child document always occupies
its own set of pages; pages cannot be shared between child documents.
Usually, this behaviour makes perfect sense
because each child document contain an essential part of the document.
However, in some situations it may be desirable to compose
a document from a collection of parts
without having mandatory page breaks between then.
For this case, the package
provides a mechanism to include parts
by |\input| which can also be processed individually.
However, by construction this mechanism
requires manual handling of the content to be output.

%%%%%%%%%%%%%%%%%%%%%%%%%%%%%%%%%%%%%%%%
\DescribeMacro{\ifchilddocmanual}
The main file should be prepared as usual, see \secref{sec:include}.
However, the document body must make a distinction
between processing of an individual part and of the main document, e.g.:
%
\begin{center}
\begin{tabular}{l}
|\ifchilddocmanual|\\
|\input{\childdocname}|\\
|\||else|\\
\textit{document body with }|\input{|\textit{part}|}|\\
|\||fi|
\end{tabular}
\end{center}
%
The conditional |\ifchilddocmanual| is true whenever
a part to be included by |\input| is being compiled,
and the name of the part is stored in |\childdocname|.

%%%%%%%%%%%%%%%%%%%%%%%%%%%%%%%%%%%%%%%%
\DescribeMacro{\childdocby}
Each part to be included by |\input| should start with:
%
\begin{center}
\begin{tabular}{l}
|\input{childdoc.def}|\\
|\childdocby{|\textit{main}|}|\\
\end{tabular}
\end{center}
%
The directive |\childdocby| is similar to |\childdocof|
described in \secref{sec:include},
but the subsequent selection of content must be done manually.
To that end, both |\ifchilddoc| and |\ifchilddocmanual|
will be true upon processing of a part,
and the name of the part is stored in |\childdocname|.
Note that |\jobname| will be set to the filename of the current part
so that each part receives an individual |.aux| file
that does not interfere with the |.aux| file(s) of the main document.
This behaviour can be altered by the alternative form
|\childdocby[*]{|\textit{main}|}| (with a non-empty optional argument)
which uses the |.aux| file of the main document
by setting |\jobname| to \textit{main}.

%%%%%%%%%%%%%%%%%%%%%%%%%%%%%%%%%%%%%%%%%%%%%%%%%%%%%%%%%%%%%%%%%%%%%%%%%%%%%%%%
\subsection{Driver Development}
\label{sec:driver}

The \textsf{childdoc} mechanism can also be use for the development
of definition files such as \LaTeX{} styles or classes.
This case differs from the above setup with multiple parts
included by |\include| in that no |\includeonly| should be invoked.
This can be achieved by starting the include file
(before |\ProvidesPackage|) with:
%
\begin{center}
\begin{tabular}{l}
|\input{childdoc.def}|\\
|\childdocforward{|\textit{main}|}|\\
\end{tabular}
\end{center}
%
or alternatively with:
%
\begin{center}
\begin{tabular}{l}
|\input{childdoc.def}|\\
|\childdocby{|\textit{main}|}|\\
\end{tabular}
\end{center}
%
Both forms have slightly different effects as described above.
The main file is prepared as usual, see \secref{sec:include}.

%%%%%%%%%%%%%%%%%%%%%%%%%%%%%%%%%%%%%%%%%%%%%%%%%%%%%%%%%%%%%%%%%%%%%%%%%%%%%%%%
\subsection{Legacy Detection}
\label{sec:detection}

The directive |\childdocmain| in the main file can detect
whether the complete document or merely a child is to be compiled
even without using the directive |\childdocof|.
This method is deprecated because it is less robust
and there is no compelling reason to use it;
it is merely provided for backward compatibility
and it may be removed in future versions.

If the detection mechanism is to be used,
it is mandatory to correctly specify
the filename of the main file as the argument of |\childdocmain|:
%
\begin{center}
\begin{tabular}{l}
|\input{childdoc.def}|\\
|\childdocmain{|\textit{main}|}|\\
\end{tabular}
\end{center}
%
If |\jobname| does not match the argument \textit{main} of |\childdocmain|,
it is assumed that |\jobname| points to the child file to be compiled.
When using |\childdocmain| with the main file specified as argument,
it suffices to start a child file
with just |\input{|\textit{main}|}|
without loading of the package and using |\childdocof|.
If instead all processing is done
with the appropriate \textsf{childdoc} directives,
the argument of \textit{main} of |\childdocmain| can be empty.

An alternative version of the command line processing described
in \secref{sec:commandline} using the detection mechanism reads:
%
\begin{center}
|... -jobname "|\textit{target}|" "|[\textit{flags}]%
[|\def\jobname{|\textit{dest}|}|]|\input{|\textit{main}|}"|
\end{center}

%%%%%%%%%%%%%%%%%%%%%%%%%%%%%%%%%%%%%%%%%%%%%%%%%%%%%%%%%%%%%%%%%%%%%%%%%%%%%%%%
\subsection{Manual Code}
\label{sec:manual}

In case one cannot be certain whether the definitions file |childdoc.def|
is installed on the target \TeX{} distribution
and one prefers not to ship it,
it is conceivable to paste a few relevant commands into the sources.

To that end, drop all statements |\input{childdoc.def}|
and perform the replacements as outlined below.
Instead of |\childdocmain{|\textit{main}|}| add the following code
to the top of the main file:
%
\begin{center}
\begin{tabular}{l}
|\||ifdefined\childdocname\endinput\||fi\newif\ifchilddoc|\\
|\edef\childdocname{\scantokens\expandafter{\jobname\noexpand}}|\\
|\def\childdocmain{|\textit{main}|}\||ifx\childdocmain\childdocname\||else|\\
|\childdoctrue\includeonly{\childdocname}\let\jobname\childdocmain\||fi|\\
\end{tabular}
\end{center}
%
Instead of |\childdocof{|\textit{main}|}| just include the main file
at the top of each child file:
%
\begin{center}
|\input{|\textit{main}|}|
\end{center}
%
A simple redirection |\childdocforward{|\textit{dest}|}| is achieved by:
%
\begin{center}
|\def\jobname{|\textit{dest}|}\input{\jobname}|
\end{center}
%
The redirection with prefix
|\childdocforwardprefix[|\textit{prefix}|]{|\textit{dest}|}|
is accomplished by:
%
\begin{center}
\begin{tabular}{l}
|{\edef\jobname{\scantokens\expandafter{\jobname\noexpand}}|\\
|\def\redirectjob |\textit{prefix}|#1~~~{\gdef\jobname{|\textit{dest}|#1}}|\\
|\expandafter\redirectjob\jobname~~~}\input{\jobname}|
\end{tabular}
\end{center}

In an alternative approach,
child documents can be compiled by a specific command line
without additional code or specific definitions:
%
\begin{center}
|... -jobname "|\textit{target}|" "|[\textit{flags}]%
|\includeonly{|\textit{dest}|}\input{|\textit{main}|}"|
\end{center}
%

%%%%%%%%%%%%%%%%%%%%%%%%%%%%%%%%%%%%%%%%%%%%%%%%%%%%%%%%%%%%%%%%%%%%%%%%%%%%%%%%
%%%%%%%%%%%%%%%%%%%%%%%%%%%%%%%%%%%%%%%%%%%%%%%%%%%%%%%%%%%%%%%%%%%%%%%%%%%%%%%%
\section{Information}

%%%%%%%%%%%%%%%%%%%%%%%%%%%%%%%%%%%%%%%%%%%%%%%%%%%%%%%%%%%%%%%%%%%%%%%%%%%%%%%%
\subsection{Copyright}

Copyright \copyright{} 2017--2018 Niklas Beisert

This work may be distributed and/or modified under the
conditions of the \LaTeX{} Project Public License, either version 1.3
of this license or (at your option) any later version.
The latest version of this license is in
  \url{http://www.latex-project.org/lppl.txt}
and version 1.3 or later is part of all distributions of \LaTeX{}
version 2005/12/01 or later.

This work has the LPPL maintenance status `maintained'.

The Current Maintainer of this work is Niklas Beisert.

This work consists of the files |README.txt|, |childdoc.ins| and |childdoc.dtx|
as well as the derived files |childdoc.def|, |cdocsamp.tex|
with |cdocsch1.tex|, |cdocsch2.tex|, |cdocspt3.tex|, |cdocspt4.tex|,
|cdocsdrf.tex|, |cdocsfn1.tex|, |cdocsfn2.tex|
as well as |childdoc.pdf|.

%%%%%%%%%%%%%%%%%%%%%%%%%%%%%%%%%%%%%%%%%%%%%%%%%%%%%%%%%%%%%%%%%%%%%%%%%%%%%%%%
\subsection{Files and Installation}

The package consists of the files:
%
\begin{center}
\begin{tabular}{ll}
    |README.txt|   & readme file \\
    |childdoc.ins| & installation file \\
    |childdoc.dtx| & source file \\
    |childdoc.def| & definition file \\
    |cdocsamp.tex| & sample main file \\
    |cdocsch1.tex| & sample include file \\
    |cdocsch2.tex| & sample include file \\
    |cdocspt3.tex| & sample part file \\
    |cdocspt4.tex| & sample part file \\
    |cdocsdrf.tex| & sample redirection file \\
    |cdocsfn1.tex| & sample redirection file \\
    |cdocsfn2.tex| & sample redirection file \\
    |childdoc.pdf| & manual
\end{tabular}
\end{center}
%
The distribution consists of the files
|README.txt|, |childdoc.ins| and |childdoc.dtx|.
%
\begin{itemize}
\item
Run (pdf)\LaTeX{} on |childdoc.dtx|
to compile the manual |childdoc.pdf| (this file).
\item
Run \LaTeX{} on |childdoc.ins| to create the definitions file |childdoc.def|
and the sample |cdocsamp.tex| with include files
|cdocsch1.tex|, |cdocsch2.tex|, |cdocspt3.tex|, |cdocspt4.tex|,
|cdocsdrf.tex|, |cdocsfn1.tex|, |cdocsfn2.tex|.
Then copy the file |childdoc.def| to an appropriate directory of your \LaTeX{}
distribution, e.g.\ \textit{texmf-root}|/tex/latex/childdoc|.
\end{itemize}

%%%%%%%%%%%%%%%%%%%%%%%%%%%%%%%%%%%%%%%%%%%%%%%%%%%%%%%%%%%%%%%%%%%%%%%%%%%%%%%%
\subsection{Related CTAN Packages}

There are several other packages which offer a similar functionality:
%
\begin{itemize}
\item
The packages
\href{http://ctan.org/pkg/docmute}{\textsf{docmute}},
\href{http://ctan.org/pkg/includex}{\textsf{includex}} and
\href{http://ctan.org/pkg/standalone}{\textsf{standalone}}
provide commands to include only the document body of
a child file thus allowing both files to be compiled individually.
\item
The packages \href{http://ctan.org/pkg/subdocs}{\textsf{subdocs}}
and \href{http://ctan.org/pkg/subfiles}{\textsf{subfiles}}
provide structures in which the main and child documents can be
encapsulated and allowing them to be compiled individually.
The inclusion mechanism is different from the conventional |\include|.
\item
The package \href{http://ctan.org/pkg/combine}{\textsf{combine}}
is an elaborate solution to combine several documents into one.
\end{itemize}
%
See also the CTAN topic \href{http://ctan.org/topic/subdocs}{\textsf{subdocs}}
for further related packages.
The present package differs from the above solutions in that
a document structure constructed with the conventional |\include| mechanism
just needs two extra commands at the top of every file
such that all constituent files can be compiled individually.

%%%%%%%%%%%%%%%%%%%%%%%%%%%%%%%%%%%%%%%%%%%%%%%%%%%%%%%%%%%%%%%%%%%%%%%%%%%%%%%%
%\subsection{Feature Suggestions}
%
%The following is a list of features which may be useful for future
%versions of this package:
%%
%\begin{itemize}
%\item
%\ldots
%\end{itemize}

%%%%%%%%%%%%%%%%%%%%%%%%%%%%%%%%%%%%%%%%%%%%%%%%%%%%%%%%%%%%%%%%%%%%%%%%%%%%%%%%
\subsection{Revision History}

%%%%%%%%%%%%%%%%%%%%%%%%%%%%%%%%%%%%%%%%
\paragraph{v2.0:} 2018/12/30

\begin{itemize}
\item
immediate forward processing
\item
added |\childdocby| mechanism
\item
manual restructured
\end{itemize}

%%%%%%%%%%%%%%%%%%%%%%%%%%%%%%%%%%%%%%%%
\paragraph{v1.6:} 2018/01/17

\begin{itemize}
\item
application for development of include files
\item
corrections to manual
\end{itemize}

%%%%%%%%%%%%%%%%%%%%%%%%%%%%%%%%%%%%%%%%
\paragraph{v1.5:} 2017/05/21

\begin{itemize}
\item
more complete structuring introduced
\item
|\childdocof| introduced
\item
|\childdoc| renamed to |\childdocmain|
\item
|\childredirect| renamed to |\childdocforward| and |\childdocforwardprefix|
and functionality expanded
\end{itemize}

%%%%%%%%%%%%%%%%%%%%%%%%%%%%%%%%%%%%%%%%
\paragraph{v1.0:} 2017/04/27

\begin{itemize}
\item
manual and install package
\item
first version published on CTAN
\end{itemize}

%%%%%%%%%%%%%%%%%%%%%%%%%%%%%%%%%%%%%%%%
\paragraph{v0.6:} 2017/04/26

\begin{itemize}
\item
redirection mechanism added
\end{itemize}

%%%%%%%%%%%%%%%%%%%%%%%%%%%%%%%%%%%%%%%%
\paragraph{v0.5:} 2017/04/26

\begin{itemize}
\item
functionality in definition file
\end{itemize}


%%%%%%%%%%%%%%%%%%%%%%%%%%%%%%%%%%%%%%%%%%%%%%%%%%%%%%%%%%%%%%%%%%%%%%%%%%%%%%%%
%%%%%%%%%%%%%%%%%%%%%%%%%%%%%%%%%%%%%%%%%%%%%%%%%%%%%%%%%%%%%%%%%%%%%%%%%%%%%%%%
%%%%%%%%%%%%%%%%%%%%%%%%%%%%%%%%%%%%%%%%%%%%%%%%%%%%%%%%%%%%%%%%%%%%%%%%%%%%%%%%
\appendix

\settowidth\MacroIndent{\rmfamily\scriptsize 000\ }

 \DocInput{childdoc.dtx}

\end{document}
%</driver>
% \fi
%
% %%%%%%%%%%%%%%%%%%%%%%%%%%%%%%%%%%%%%%%%%%%%%%%%%%%%%%%%%%%%%%%%%%%%%%%%%%%%%%
% %%%%%%%%%%%%%%%%%%%%%%%%%%%%%%%%%%%%%%%%%%%%%%%%%%%%%%%%%%%%%%%%%%%%%%%%%%%%%%
% \section{Sample}
%\iffalse
%<*samplemain>
%\fi
%
% The following presents a sample document
% with two chapters, two parts, a title page,
% a compile flag as well as three forwarding files to set the flag.
% It consists of eight |.tex| files:
% \begin{center}
% \begin{tabular}{ll}
% |cdocsamp.tex|&main file\\
% |cdocsch1.tex|&include file for chapter 1\\
% |cdocsch2.tex|&include file for chapter 2\\
% |cdocspt3.tex|&include file for part 3\\
% |cdocspt4.tex|&include file for part 4\\
% |cdocsdrf.tex|&forwarding file for main file in draft mode\\
% |cdocsfi1.tex|&forwarding file for final version of chapter 1\\
% |cdocsfi2.tex|&forwarding file for final version of chapter 2\\
% \end{tabular}
% \end{center}
% Each of the eight files can be compiled directly by the \LaTeX{} compiler.
%
% %%%%%%%%%%%%%%%%%%%%%%%%%%%%%%%%%%%%%%
% \paragraph{Main File.}
%
% The main file is called |cdocsamp.tex|.
%
% Load the \textsf{childdoc} definitions and
% declare the filename for the main document:
%    \begin{macrocode}
\input{childdoc.def}
\childdocmain{}
%    \end{macrocode}

% Optional override for |\version| flag:
%    \begin{macrocode}
%%\ifchilddoc\else\providecommand{\version}{draft}\fi
%    \end{macrocode}

% Define the default values for the |\version| flag
% (|final| for the main file and |draft| for childs):
%    \begin{macrocode}
\ifchilddoc
\providecommand{\version}{draft}
\else
\providecommand{\version}{final}
\fi
%    \end{macrocode}

% Load the standard document class:
%    \begin{macrocode}
\documentclass[12pt]{article}
%    \end{macrocode}

% Start the document body:
%    \begin{macrocode}
\begin{document}
%    \end{macrocode}

% Declare a title page.
% Print title, part of document being processed and version flag:
%    \begin{macrocode}
\addtocounter{page}{-1}
\begin{center}
{\LARGE\bfseries{}childdoc example\par}
\vspace{1cm}
\ifchilddoc
\ifchilddocmanual part\else chapter\fi:
`\childdocname' of `\childdocjob'\par
\else
main document: `\childdocjob'\par
\fi
version: \version\par
\end{center}
\newpage
%    \end{macrocode}

% Manually include selected file,
% otherwise process as usual:
%    \begin{macrocode}
\ifchilddocmanual
\section*{part `\childdocname'}
\input{\childdocname}
\else
%    \end{macrocode}

% Include the two chapters:
%    \begin{macrocode}
\include{cdocsch1}
\include{cdocsch2}
%    \end{macrocode}

% Include the two parts unless only chapters should be displayed:
%    \begin{macrocode}
\ifchilddoc\else
\section{part three}
\input{cdocspt3}
\section{part four}
\input{cdocspt4}
\fi
%    \end{macrocode}

% Process as usual until here:
%    \begin{macrocode}
\fi
%    \end{macrocode}

% End of document body:
%    \begin{macrocode}
\end{document}
%    \end{macrocode}
%\iffalse
%</samplemain>
%\fi
%
% %%%%%%%%%%%%%%%%%%%%%%%%%%%%%%%%%%%%%%
% \paragraph{Chapter Include Files.}
%
% The include files are called |cdocsch1.tex| and |cdocsch2.tex|.
%
%\iffalse
%<*samplechap1|samplechap2>
%\fi

% Optional override for |\version| flag:
%    \begin{macrocode}
%%\providecommand{\version}{final}
%    \end{macrocode}

% Include the main document:
%    \begin{macrocode}
\input{childdoc.def}
\childdocof{cdocsamp}
%    \end{macrocode}

%\iffalse
%</samplechap1|samplechap2>
%\fi
%
%\iffalse
%<*samplechap1>
%\fi
% Some text for chapter 1:
%    \begin{macrocode}
\section{one}
some text in chapter one
%    \end{macrocode}

%\iffalse
%</samplechap1>
%\fi
% Some text for chapter 2:
%\iffalse
%<*samplechap2>
%\fi
%    \begin{macrocode}
\section{two}
more text in chapter two
%    \end{macrocode}

%\iffalse
%</samplechap2>
%\fi
%
% %%%%%%%%%%%%%%%%%%%%%%%%%%%%%%%%%%%%%%
% \paragraph{Part Include Files.}
%
% The include files are called |cdocspt3.tex| and |cdocspt4.tex|.
%
%\iffalse
%<*samplepart3|samplepart4>
%\fi

% Optional override for |\version| flag:
%    \begin{macrocode}
%%\providecommand{\version}{final}
%    \end{macrocode}

% Include the main document:
%    \begin{macrocode}
\input{childdoc.def}
\childdocby{cdocsamp}
%    \end{macrocode}

%\iffalse
%</samplepart3|samplepart4>
%\fi
%
%\iffalse
%<*samplepart3>
%\fi
% Some text for part 3:
%    \begin{macrocode}
some text in part three
%    \end{macrocode}

%\iffalse
%</samplepart3>
%\fi
% Some text for part 4:
%\iffalse
%<*samplepart4>
%\fi
%    \begin{macrocode}
more text in part four
%    \end{macrocode}

%\iffalse
%</samplepart4>
%\fi
%
% %%%%%%%%%%%%%%%%%%%%%%%%%%%%%%%%%%%%%%
% \paragraph{Forwarding for a Complete Draft.}
%
% The following forwarding file |cdocsdrf.tex|
% compiles the main document in draft mode:
%\iffalse
%<*sampledraft>
%\fi
%    \begin{macrocode}
\def\version{draft}
\input{childdoc.def}
\childdocforward{cdocsamp}
%    \end{macrocode}

%\iffalse
%</sampledraft>
%\fi
%
% %%%%%%%%%%%%%%%%%%%%%%%%%%%%%%%%%%%%%%
% \paragraph{Forwarding for Final Version of the Chapters.}
%
% The following forwarding files |cdocsfn1.tex| and |cdocsfn2.tex|
% (with identical content)
% compile the final versions of the child documents
% |cdocsch1.tex| and |cdocsch2.tex|, respectively:
%\iffalse
%<*samplefinal>
%\fi
%    \begin{macrocode}
\def\version{final}
\input{childdoc.def}
\childdocforwardprefix[cdocsamp]{cdocsfn}{cdocsch}
%    \end{macrocode}

%\iffalse
%</samplefinal>
%\fi
%
% %%%%%%%%%%%%%%%%%%%%%%%%%%%%%%%%%%%%%%
% \paragraph{Command Line Processing.}
%
% The following three command lines generate the output files
% |cdocscld|, |cdocscl1| and |cdocscl2|
% which should be identical to
% |cdocsdrf|, |cdocsch1| and |cdocsfn2|, respectively:
% \begin{center}
% \begin{tabular}{l}
% |latex -jobname cdocscld \|\\
% |  "\def\version{draft}\input{childdoc.def}\childdocforward{cdocsamp}"|\\
% |latex -jobname cdocscl1 \|\\
% |  "\input{childdoc.def}\childdocforward[cdocsamp]{cdocsch1}"|\\
% |latex -jobname cdocscl2 \|\\
% |  "\def\version{final}\input{childdoc.def}\childdocforward{cdocsch2}"|
% \end{tabular}
% \end{center}
% Note that the trailing backslash on each first line
% merely continues the input to the second line
% (for convenient cut ant paste).
% Furthermore, the command |latex| can be replaced by any
% of its alternative versions such as |pdflatex|.
%
% %%%%%%%%%%%%%%%%%%%%%%%%%%%%%%%%%%%%%%%%%%%%%%%%%%%%%%%%%%%%%%%%%%%%%%%%%%%%%%
% %%%%%%%%%%%%%%%%%%%%%%%%%%%%%%%%%%%%%%%%%%%%%%%%%%%%%%%%%%%%%%%%%%%%%%%%%%%%%%
% \section{Implementation}
%\iffalse
%<*package>
%\fi
%
% This section describes the definitions file |childdoc.def|.

% The definitions cannot be loaded using |\usepackage| or |\RequirePackage|
% which has a mechanism to prevent loading a style file more than once.
% When loading the definitions by means of |\input|
% multiple instances have to be prevented manually:
%\iffalse
%This code needs to be before the `\ProvidesFile' directive
%which is defined at the beginning of this file.
%Therefore it is also placed there and commented out here.
%</package>
%<*discard>
%\fi
%    \begin{macrocode}
\ifdefined\childdocmain\endinput\fi
%    \end{macrocode}
%\iffalse
%</discard>
%<*package>
%\fi
%
% \macro{\ifchilddoc}
% \macro{\ifchilddocmanual}
% The conditional |\ifchilddoc| tells whether a
% child (true) or main (false) document is being compiled.
% The conditional |\ifchilddocmanual| tells whether
% the |\includeonly| mechanism is used (false) or
% the selection of child files must be performed manually (true).
% The definitions initialise to false:
%    \begin{macrocode}
\newif\ifchilddoc
\newif\ifchilddocmanual
%    \end{macrocode}

% \macro{\childdocname}
% \macro{\childdocjob}
% The macro |\childdocname| stores the name of the main document
% to be compiled. The macro |\childdocjob| stores the name of
% the document on which the \LaTeX{} compiler was originally invoked.
% The content of |\jobname| cannot be compared
% to filenames specified in the source due to different catcodes.
% The following code rescans |\jobname|, stores the result
% in |\childdocname| and saves a copy in |\childdocjob|:
%    \begin{macrocode}
\edef\childdocname{\scantokens\expandafter{\jobname\noexpand}}
\let\childdocjob\childdocname
%    \end{macrocode}

% \macro{\childdocdisable}
% The macro |\childdocdisable| prevents the main file
% from being processed more than once.
% At this stage, the main document command |\childdocmain|
% is assumed to be called once again where it should do nothing.
% Any subsequent call to it should prevent
% a secondary processing of the main document
% It overwrites the forwarding commands
% |\childdocof| and |\childdocforward|
% with empty macros to prevent further inclusions of the main document:
%    \begin{macrocode}
\newcommand{\childdocdisable}
{
  \renewcommand{\childdocmain}[1]{\renewcommand{\childdocmain}[1]{\endinput}}
  \renewcommand{\childdocof}[1]{}
  \renewcommand{\childdocby}[2][]{}
  \renewcommand{\childdocforward}[2][]{}
  \renewcommand{\childdocdisable}{}
}
%    \end{macrocode}

% \macro{\childdocmain}
% The macro |\childdocmain| is to be called at the top of the main file
% with nothing or the main filename (without extension) as argument.
% First, it breaks loops.
% If the argument is not empty and does not match |\childdocname|
% (which is set by the first inclusion of |childdoc.def|),
% |\ifchilddoc| is set to true, |\includeonly| is applied to the child file
% and |\jobname| is set to the main file
% (for proper handling of |.aux| files):
%    \begin{macrocode}
\newcommand{\childdocmain}[1]
{
  \childdocdisable\childdocmain{}
  \if?#1?\else
    \begingroup
      \def\childdoctmp{#1}
      \ifx\childdoctmp\childdocname
        \def\childdoctmp{}
      \else
        \def\childdoctmp
        {
          \childdoctrue
          \includeonly{\childdocname}
          \def\childdocjob{#1}
          \def\jobname{#1}
        }
      \fi
      \expandafter
    \endgroup
    \childdoctmp
  \fi
}
%    \end{macrocode}

% \macro{\childdocof}
% The command |\childdocof| redirects
% compilation to the main file |#1|.
%    \begin{macrocode}
\newcommand{\childdocof}[1]
{
  \childdocdisable
  \childdoctrue
  \includeonly{\childdocname}
  \def\jobname{#1}
  \def\childdocjob{#1}
  \input{#1}
}
%    \end{macrocode}

% \macro{\childdocby}
% The command |\childdocby| ....
%    \begin{macrocode}
\newcommand{\childdocby}[2][]
{
  \childdocdisable
  \childdoctrue
  \childdocmanualtrue
  \if?#1?\else
    \def\jobname{#2}
  \fi
  \def\childdocjob{#2}
  \input{#2}
  \endinput
}
%    \end{macrocode}

% \macro{\childdocforward}
% The command |\childdocforward| redirects
% compilation to the main file or
% (if the optional argument is given) a child file.
% Parameters are set as if the main file
% or a child file starting with |\childdocof| was compiled.
% Then compilation is handed over to the main file:
%    \begin{macrocode}
\newcommand{\childdocforward}[2][]
{
  \begingroup
    \if?#1?
      \def\childdoctmp
      {
        \def\childdocname{#2}
        \def\childdocjob{#2}
        \def\jobname{#2}
        \input{#2}
        \endinput
      }
    \else
      \def\childdoctmp
      {
        \childdocdisable
        \def\childdocname{#2}
        \childdoctrue
        \includeonly{#2}
        \def\childdocjob{#1}
        \def\jobname{#1}
        \input{#1}
        \endinput
      }
    \fi
    \expandafter
  \endgroup
  \childdoctmp
}
%    \end{macrocode}

% \macro{\childdocforwardprefix}
% The command |\childdocforwardprefix| redirects
% compilation to the main or a child file by means of a pattern.
% The prefix |#1| in the current filename is replaced by |#2|
% and the suffix of the current filename is kept
% (it is assumed that the filename does not contain the substring `|~~~|'
% which is used as a delimiter).
% Compilation is handed over to the new file by |\childdocforward|:
%    \begin{macrocode}
\newcommand{\childdocforwardprefix}[3][]
{
  \begingroup
    \def\childdocextract #2##1~~~{\def\childdoctmp{\childdocforward[#1]{#3##1}}}
    \expandafter\childdocextract\childdocname~~~
    \expandafter
  \endgroup
  \childdoctmp
}
%    \end{macrocode}

% \macro{\childdoc}
% The deprecated macro |\childdoc| is a legacy version of |\childdocmain|:
%    \begin{macrocode}
\newcommand{\childdoc}{\childdocmain}
%    \end{macrocode}

% \macro{\childdocredirect}
% The deprecated macro |\childdocredirect| is a legacy version
% of |\childdocforward| and |\childdocforwardprefix|:
%    \begin{macrocode}
\newcommand{\childdocredirect}[2][]
{
  \begingroup
    \if?#1?
      \def\childdoctmp{\childdocforward{#2}}
    \else
      \def\childdoctmp{\childdocforwardprefix{#1}{#2}}
    \fi
    \expandafter
  \endgroup
  \childdoctmp
}
%    \end{macrocode}

%\iffalse
%</package>
%\fi
%
\endinput
|\\
|\childdocby{|\textit{main}|}|\\
\end{tabular}
\end{center}
%
The directive |\childdocby| is similar to |\childdocof|
described in \secref{sec:include},
but the subsequent selection of content must be done manually.
To that end, both |\ifchilddoc| and |\ifchilddocmanual|
will be true upon processing of a part,
and the name of the part is stored in |\childdocname|.
Note that |\jobname| will be set to the filename of the current part
so that each part receives an individual |.aux| file
that does not interfere with the |.aux| file(s) of the main document.
This behaviour can be altered by the alternative form
|\childdocby[*]{|\textit{main}|}| (with a non-empty optional argument)
which uses the |.aux| file of the main document
by setting |\jobname| to \textit{main}.

%%%%%%%%%%%%%%%%%%%%%%%%%%%%%%%%%%%%%%%%%%%%%%%%%%%%%%%%%%%%%%%%%%%%%%%%%%%%%%%%
\subsection{Driver Development}
\label{sec:driver}

The \textsf{childdoc} mechanism can also be use for the development
of definition files such as \LaTeX{} styles or classes.
This case differs from the above setup with multiple parts
included by |\include| in that no |\includeonly| should be invoked.
This can be achieved by starting the include file
(before |\ProvidesPackage|) with:
%
\begin{center}
\begin{tabular}{l}
|% \iffalse
%
% childdoc.dtx Copyright (C) 2017-2018 Niklas Beisert
%
% This work may be distributed and/or modified under the
% conditions of the LaTeX Project Public License, either version 1.3
% of this license or (at your option) any later version.
% The latest version of this license is in
%   http://www.latex-project.org/lppl.txt
% and version 1.3 or later is part of all distributions of LaTeX
% version 2005/12/01 or later.
%
% This work has the LPPL maintenance status `maintained'.
%
% The Current Maintainer of this work is Niklas Beisert.
%
% This work consists of the files childdoc.dtx and childdoc.ins
% and the derived files childdoc.def and cdocsamp.tex with
% cdocsch1.tex, cdocsch2.tex, cdocsdrf.tex, cdocsfn1.tex, cdocsfn2.tex.
%
%<package>\ifdefined\childdocmain\endinput\fi
%<package>\ProvidesFile{childdoc.def}[2018/12/30 v2.0 child document driver]
%<samplemain>\ProvidesFile{cdocsamp.tex}[2018/12/30 v2.0 sample for childdoc]
%<*driver>
%\ProvidesFile{childdoc.drv}[2018/12/30 v2.0 childdoc reference manual file]
\PassOptionsToClass{10pt,a4paper}{article}
\documentclass{ltxdoc}

\usepackage[margin=35mm]{geometry}
\usepackage{hyperref}
\usepackage{hyperxmp}
\usepackage[usenames]{color}

\hypersetup{colorlinks=true}
\hypersetup{pdfstartview=FitH}
\hypersetup{pdfpagemode=UseNone}
\hypersetup{pdfsource={}}
\hypersetup{pdflang={en-UK}}
\hypersetup{pdfcopyright={Copyright 2017-2018 Niklas Beisert.
  This work may be distributed and/or modified under the
  conditions of the LaTeX Project Public License, either version 1.3
  of this license or (at your option) any later version.}}
\hypersetup{pdflicenseurl={http://www.latex-project.org/lppl.txt}}
\hypersetup{pdfcontactaddress={ETH Zurich, ITP, HIT K,
  Wolfgang-Pauli-Strasse 27}}
\hypersetup{pdfcontactpostcode={8093}}
\hypersetup{pdfcontactcity={Zurich}}
\hypersetup{pdfcontactcountry={Switzerland}}
\hypersetup{pdfcontactemail={nbeisert@itp.phys.ethz.ch}}
\hypersetup{pdfcontacturl={http://people.phys.ethz.ch/\xmptilde nbeisert/}}

\newcommand{\secref}[1]{\hyperref[#1]{section \ref*{#1}}}

\parskip1ex
\parindent0pt
\let\olditemize\itemize
\def\itemize{\olditemize\parskip0pt}

\begin{document}

\title{The \textsf{childdoc} Package}
\hypersetup{pdftitle={The childdoc Package}}
\author{Niklas Beisert\\[2ex]
  Institut f\"ur Theoretische Physik\\
  Eidgen\"ossische Technische Hochschule Z\"urich\\
  Wolfgang-Pauli-Strasse 27, 8093 Z\"urich, Switzerland\\[1ex]
  \href{mailto:nbeisert@itp.phys.ethz.ch}
  {\texttt{nbeisert@itp.phys.ethz.ch}}}
\hypersetup{pdfauthor={Niklas Beisert}}
\hypersetup{pdfsubject={Manual for the LaTeX2e Package childdoc}}
\date{30 December 2018, \textsf{v2.0}}
\maketitle

\begin{abstract}\noindent
\textsf{childdoc} is a \LaTeXe{} package
that enables the direct compilation
of document sections included by |\include|
to individual files.
\end{abstract}

\begingroup
\parskip0ex
\tableofcontents
\endgroup

%%%%%%%%%%%%%%%%%%%%%%%%%%%%%%%%%%%%%%%%%%%%%%%%%%%%%%%%%%%%%%%%%%%%%%%%%%%%%%%%
%%%%%%%%%%%%%%%%%%%%%%%%%%%%%%%%%%%%%%%%%%%%%%%%%%%%%%%%%%%%%%%%%%%%%%%%%%%%%%%%
\section{Introduction}

\LaTeX{} provides a mechanism to structure a large document (such as a book)
into a main file and several child files (containing the chapters)
using the |\include| command.
This mechanism is beneficial for documents
which span hundreds of pages in order to
make the source file(s) more manageable.
Moreover, compilation can be restricted to
selected child files by means of the |\includeonly| command.
The latter feature can be used to reduce the compilation time while editing
(this was significantly more useful in the earlier days of \LaTeX{})
or to generate a smaller document which is easier to navigate.
Another application of |\includeonly| is to generate
documents consisting of selected parts of the complete document.

However, there are a few drawbacks of the plain |\include| mechanism:
\begin{itemize}
\item
The child files cannot be compiled on their own,
they can only be compiled via the main file.
A naive editing environment
(such as a text editor with an option
to have the current file processed by \LaTeX)
may require one to switch to the main file before compiling;
attempting to compile the child file produces errors.
\item
The main file must be modified (each time)
to adjust the |\includeonly| command
to the present needs. This easily leaves the main file in a messy state.
\item
The generated document will always carry the filename
of the main document. This is inconvenient if
several child files are to be compiled and
to be kept for distribution.
\end{itemize}

The present package provides a simple interface
to make child files individually compilable by \LaTeX{}.
Compiling a child file then has the same effect as compiling
the main file with an |\includeonly| command
to select the appropriate child.
Moreover the generated document will carry the name of the child
rather than the main file.
This resolves all three above issues.

This feature is meant to make the editing of books,
thesis documents and lecture notes somewhat more convenient.
However, the package can also be used efficiently for
composing a series of documents (such as exercise sheets)
which are typically distributed individually.
It then assists the author in generating the individual documents
(potentially in different versions)
as well as a document containing the collected series.
Another application is in developing style files
or other kinds of included material
where compilation of the style file could redirect
to a sample or test file.

%%%%%%%%%%%%%%%%%%%%%%%%%%%%%%%%%%%%%%%%%%%%%%%%%%%%%%%%%%%%%%%%%%%%%%%%%%%%%%%%
%%%%%%%%%%%%%%%%%%%%%%%%%%%%%%%%%%%%%%%%%%%%%%%%%%%%%%%%%%%%%%%%%%%%%%%%%%%%%%%%
\section{Usage}

First of all, the package \textsf{childdoc} is \emph{not} a standard
\LaTeXe{} |.sty| style file! Therefore it needs to be invoked in
a non-standard way.

%%%%%%%%%%%%%%%%%%%%%%%%%%%%%%%%%%%%%%%%%%%%%%%%%%%%%%%%%%%%%%%%%%%%%%%%%%%%%%%%
\subsection{Included Files}
\label{sec:include}

%%%%%%%%%%%%%%%%%%%%%%%%%%%%%%%%%%%%%%%%
\DescribeMacro{\childdocmain}
To use the package, add the commands
\begin{center}
\begin{tabular}{l}
|\input{childdoc.def}|\\
|\childdocmain{}|\\
\end{tabular}
\end{center}
at the very top of the main \LaTeX{} file,
in particular \emph{before} the |\documentclass| statement!
The argument of |\childdocmain| should be left empty
(but it must be present).

%%%%%%%%%%%%%%%%%%%%%%%%%%%%%%%%%%%%%%%%
\DescribeMacro{\childdocof}
Furthermore, add the commands
\begin{center}
\begin{tabular}{l}
|\input{childdoc.def}|\\
|\childdocof{|\textit{main}|}|\\
\end{tabular}
\end{center}
at the top of every child file \textit{child}
which is included by |\include{|\textit{child}|}|
from within the main file
(or at least for those files to be compiled individually).
The argument \textit{main} must be the filename of the main file.

There are a couple of
considerations in setting up the main and child documents:

%%%%%%%%%%%%%%%%%%%%%%%%%%%%%%%%%%%%%%%%
\paragraph{Restrictions.}

Please note the following restrictions:
\begin{itemize}
\item
|\childdocmain| must be called with one argument \textit{main}
to ensure compatibility with earlier version of the package.
It must either be empty (|\childdocmain{}|)
or precisely match the filename of the main file in which it is specified.
See \secref{sec:detection} for further information.
\item
The filename \textit{main} must be specified without the |.tex| extension.
\item
The filename \textit{main} is case sensitive
(even in case-insensitive file systems)
due to internal string comparison.
\item
The argument \textit{main} should be fully expanded, it cannot be a macro.
\item
Subdirectories and special characters should be avoided in filenames.
\item
The command |\childdocmain{|\textit{main}|}| must be followed by a whitespace.
It should not be followed immediately by another command
or by a comment mark `|%|'.
This is because the \TeX{} parser reads the token immediately following
the argument of |\childdocmain| and puts it
at the beginning of every child section;
however, a white\-space is ignored.
\end{itemize}

%%%%%%%%%%%%%%%%%%%%%%%%%%%%%%%%%%%%%%%%
\paragraph{Content of Main File.}

It is advisable to place all content in the child files included by |\include|.
Any output contained in the main file will appear in all child documents
unless suppressed manually;
it cannot be suppressed automatically by the |\includeonly| directive
and thus should normally be avoided.
A method to include some content in the main file
by means of conditional processing is described in \secref{sec:conditional}.

%%%%%%%%%%%%%%%%%%%%%%%%%%%%%%%%%%%%%%%%
\paragraph{Page Numbering.}

When only a part of the document is compiled,
the appropriate numbering of pages
(as well as other status parameters)
is determined from the |.aux| files.
The latter contain information from previous passes.
However this information needs to propagate through
all intermediate child documents.
Therefore the page numbering in child documents may well
be inconsistent until the complete document is compiled at least once.

A useful (if unconventional) way to always ensure a consistent
page numbering is to restart the numbering in each child document
and denote the pages by `\textit{child}|.|\textit{page}'
where \textit{child} represents the chapter/section number of the child file.
This can be achieved by the command
|\numberwithin{page}{|\textit{child}|}|
of the \textsf{amsmath} package
where \textit{child} can be |chapter| or |section|
depending on the chosen structuring.
Alternatively, one can modify the macro |\thepage| appropriately
and reset the counter |page| at the start of each child file.

%%%%%%%%%%%%%%%%%%%%%%%%%%%%%%%%%%%%%%%%%%%%%%%%%%%%%%%%%%%%%%%%%%%%%%%%%%%%%%%%
\subsection{Conditional Processing}
\label{sec:conditional}

The package provides a mechanism to compile different versions
of a document. To customise the versions further some conditional processing
can come in handy to distinguish which version is being compiled.
The package provides two macros to describe the compilation context:

%%%%%%%%%%%%%%%%%%%%%%%%%%%%%%%%%%%%%%%%
\DescribeMacro{\ifchilddoc}
The conditional |\ifchilddoc| distinguishes between the compilation of
child documents and the main document:
%
\begin{center}
|\ifchilddoc |\textit{child-code}| |[|\||else |\textit{main-code}]| \||fi|
\end{center}

%%%%%%%%%%%%%%%%%%%%%%%%%%%%%%%%%%%%%%%%
\DescribeMacro{\childdocname}
\DescribeMacro{\childdocjob}
The macro |\childdocname| contains the filename (without extension)
of the main or child file being processed.
Note that |\childdocjob| will always contain the name of the main file.

%%%%%%%%%%%%%%%%%%%%%%%%%%%%%%%%%%%%%%%%
\paragraph{Title Page.}

Conditional processing can be used to include a title or banner page
in the main document when proper precautions are taken.
Importantly, the code in the main file should ensure that the page counter
(as well as other status parameters which are stored in the |.aux| files)
takes the same value after the conditional processing.
Otherwise the page numbers may take divergent values
depending on which part is compiled.

For example, a title page could be declared by:
%
\begin{center}
\begin{tabular}{l}
|\ifchilddoc\||else|\\
|\addtocounter{page}{-1}|\\
\textit{code for title page}\\
|\newpage|\\
|\||fi|
\end{tabular}
\end{center}
%
A banner page for the child documents can be generated by:
%
\begin{center}
\begin{tabular}{l}
|\ifchilddoc|\\
|\addtocounter{page}{-1}|\\
\textit{code for banner page}\\
|\newpage|\\
|\||fi|
\end{tabular}
\end{center}
%
Here one could write a message such as:
\begin{center}
|This is the part \childdocname{} of \childdocjob{}.|
\end{center}

%%%%%%%%%%%%%%%%%%%%%%%%%%%%%%%%%%%%%%%%%%%%%%%%%%%%%%%%%%%%%%%%%%%%%%%%%%%%%%%%
\subsection{Flags}
\label{sec:flags}

The package makes it easy to generate different versions
of the main or child documents.
To this end compilation flags can be defined
and assigned different default values.
They will be particularly useful in conjunction
with the forwarding mechanism described in \secref{sec:forward}.

For example, it may be useful to have a flag |\version|
which can be set to |draft| or |final|.
The document source will contain some conditional code
depending on the value of |\version|.
Suppose further, the flag should default to |final| for the main file
and to |draft| for child files
which is a natural assignment for editing the document.
This is achieved by placing the following code
in the preamble of the main document
(below the |\childdocmain| directive):
%
\begin{center}
\begin{tabular}{l}
|\ifchilddoc|\\
|\providecommand{\version}{draft}|\\
|\||else|\\
|\providecommand{\version}{final}|\\
|\||fi|
\end{tabular}
\end{center}
%
The definition by |\providecommand| makes sure
that previous definitions are not overwritten.
Further statements |\providecommand{\version}{...}|
can thus be added before the above code to override it.

For the main file, one might add a line
(between |\childdocmain| and the above block)
%
\begin{center}
|%\ifchilddoc\||else\providecommand{\version}{draft}\||fi|
\end{center}
%
which can be uncommented to produce a draft version.
Likewise one can add a line to the very top of a child file
(above the |\childdocof{|\textit{main}|}| directive)
%
\begin{center}
|%\providecommand{\version}{final}|
\end{center}
%
which can be uncommented to produce the final version of this child document.

%%%%%%%%%%%%%%%%%%%%%%%%%%%%%%%%%%%%%%%%%%%%%%%%%%%%%%%%%%%%%%%%%%%%%%%%%%%%%%%%
\subsection{Forwarding}
\label{sec:forward}

Different versions of the main or child documents
using compilation flags as described in \secref{sec:flags}
can be (permanently) stored in different files
for convenient compilation, viewing and distribution.
To this end, the package defines a command
to pass on compilation to a different file:

%%%%%%%%%%%%%%%%%%%%%%%%%%%%%%%%%%%%%%%%
\DescribeMacro{\childdocforward}
The command |\childdocforward| redirects processing to
another source file:
%
\begin{center}
\begin{tabular}{l}
|\input{childdoc.def}|\\
|\childdocforward[|\textit{main}|]{|\textit{dest}|}|\\
\end{tabular}
\end{center}
%
The argument \textit{dest} is the destination file
(without extension).
It should be the main file or one of the child files.
Note that further \textsf{childdoc} directives
such as |\childdocof| and |\childdocforward|
in the indicated file will be processed in this form.
The optional argument \textit{main}
passes on directly to the main file \textit{main}
while pretending to compile the child \textit{dest}.
This form behaves as if \textit{dest}
issues |\childdocof{|\textit{main}|}| right away,
and no further \textsf{childdoc} directives will be processed.

%%%%%%%%%%%%%%%%%%%%%%%%%%%%%%%%%%%%%%%%
\DescribeMacro{\...prefix}
In the alternative form |\childdocforwardprefix|,
%
\begin{center}
\begin{tabular}{l}
|\input{childdoc.def}|\\
|\childdocforwardprefix[|\textit{main}|]{|\textit{prefix}|}{|\textit{dest}|}|
\end{tabular}
\end{center}
%
the destination file is determined by a pattern
depending on the current file:
To make this work, the current file must be called
`{\textit{prefix}\hspace{0.2em}\textit{suffix}}'
with \textit{prefix} matching precisely the argument.
Processing is then passed on to the file
`{\textit{dest}\hspace{0.2em}\textit{suffix}}'.
Surely, the same effect is achieved by
directly specifying the
argument `{\textit{dest}\hspace{0.2em}\textit{suffix}}'
in the first form.
However, that requires to set up a different file
for each child. With the alternative form of the command
all these files can have exactly the same content
which simplifies setting them up and maintaining them.

For example, the following file |draft.tex|
with a compilation flag |\version| as described in \secref{sec:flags}
compiles the main document as a draft:
%
\begin{center}
\begin{tabular}{l}
|\def\version{draft}|\\
|\input{childdoc.def}|\\
|\childdocforward{|\textit{main}|}|
\end{tabular}
\end{center}
%
Likewise, the following files |final|\textit{nn}|.tex|
compile the final version of the child document
|child|\textit{nn}|.tex|:
%
\begin{center}
\begin{tabular}{l}
|\def\version{final}|\\
|\input{childdoc.def}|\\
|\childdocforwardprefix{final}{child}|
\end{tabular}
\end{center}
%

Note that when several versions of a main file and/or of each child file
are to be generated, it may be convenient to set up a |Makefile| or
shell script to automatise the process.

%%%%%%%%%%%%%%%%%%%%%%%%%%%%%%%%%%%%%%%%%%%%%%%%%%%%%%%%%%%%%%%%%%%%%%%%%%%%%%%%
\subsection{Command Line Processing}
\label{sec:commandline}

The effect of redirection files can also be achieved by invoking
the \LaTeX{} compiler with a more elaborate command line.
Most conveniently this should be done as part
of a shell script or a |Makefile|.

When using \textsf{childdoc} in the main file, the following
command lines effectively perform a redirection
(note that depending on the shell being used,
backslashes may have to be doubled: `|\|' $\to$ `|\\|'):
%
\begin{center}
|... -jobname "|\textit{target}|" |\\|"|[\textit{flags}]%
|\input{childdoc.def}\childdocforward[|\textit{main}|]{|\textit{dest}|}"|
\end{center}
%
Here \textit{target} is the name of the output file,
\textit{main} is the name of the main file
and \textit{dest} is the name of the main or child file to be processed
(all filenames without extensions).
The optional argument \textit{main} can be omitted
if \textit{main} matches \textit{dest}.
Optionally, compilation \textit{flags} can be defined via |\def| commands.
This command line makes the \TeX{} engine believe
it is compiling the file \textit{target}
whose content is specified as the latter parameter.
The provided code then forwards the processing to
\textit{main} or \textit{dest} as described in \secref{sec:forward}.

%%%%%%%%%%%%%%%%%%%%%%%%%%%%%%%%%%%%%%%%%%%%%%%%%%%%%%%%%%%%%%%%%%%%%%%%%%%%%%%%
\subsection{Include by Input}
\label{sec:input}

Including child documents by |\include| has some restrictions by design.
Most notably, the content of a child document always occupies
its own set of pages; pages cannot be shared between child documents.
Usually, this behaviour makes perfect sense
because each child document contain an essential part of the document.
However, in some situations it may be desirable to compose
a document from a collection of parts
without having mandatory page breaks between then.
For this case, the package
provides a mechanism to include parts
by |\input| which can also be processed individually.
However, by construction this mechanism
requires manual handling of the content to be output.

%%%%%%%%%%%%%%%%%%%%%%%%%%%%%%%%%%%%%%%%
\DescribeMacro{\ifchilddocmanual}
The main file should be prepared as usual, see \secref{sec:include}.
However, the document body must make a distinction
between processing of an individual part and of the main document, e.g.:
%
\begin{center}
\begin{tabular}{l}
|\ifchilddocmanual|\\
|\input{\childdocname}|\\
|\||else|\\
\textit{document body with }|\input{|\textit{part}|}|\\
|\||fi|
\end{tabular}
\end{center}
%
The conditional |\ifchilddocmanual| is true whenever
a part to be included by |\input| is being compiled,
and the name of the part is stored in |\childdocname|.

%%%%%%%%%%%%%%%%%%%%%%%%%%%%%%%%%%%%%%%%
\DescribeMacro{\childdocby}
Each part to be included by |\input| should start with:
%
\begin{center}
\begin{tabular}{l}
|\input{childdoc.def}|\\
|\childdocby{|\textit{main}|}|\\
\end{tabular}
\end{center}
%
The directive |\childdocby| is similar to |\childdocof|
described in \secref{sec:include},
but the subsequent selection of content must be done manually.
To that end, both |\ifchilddoc| and |\ifchilddocmanual|
will be true upon processing of a part,
and the name of the part is stored in |\childdocname|.
Note that |\jobname| will be set to the filename of the current part
so that each part receives an individual |.aux| file
that does not interfere with the |.aux| file(s) of the main document.
This behaviour can be altered by the alternative form
|\childdocby[*]{|\textit{main}|}| (with a non-empty optional argument)
which uses the |.aux| file of the main document
by setting |\jobname| to \textit{main}.

%%%%%%%%%%%%%%%%%%%%%%%%%%%%%%%%%%%%%%%%%%%%%%%%%%%%%%%%%%%%%%%%%%%%%%%%%%%%%%%%
\subsection{Driver Development}
\label{sec:driver}

The \textsf{childdoc} mechanism can also be use for the development
of definition files such as \LaTeX{} styles or classes.
This case differs from the above setup with multiple parts
included by |\include| in that no |\includeonly| should be invoked.
This can be achieved by starting the include file
(before |\ProvidesPackage|) with:
%
\begin{center}
\begin{tabular}{l}
|\input{childdoc.def}|\\
|\childdocforward{|\textit{main}|}|\\
\end{tabular}
\end{center}
%
or alternatively with:
%
\begin{center}
\begin{tabular}{l}
|\input{childdoc.def}|\\
|\childdocby{|\textit{main}|}|\\
\end{tabular}
\end{center}
%
Both forms have slightly different effects as described above.
The main file is prepared as usual, see \secref{sec:include}.

%%%%%%%%%%%%%%%%%%%%%%%%%%%%%%%%%%%%%%%%%%%%%%%%%%%%%%%%%%%%%%%%%%%%%%%%%%%%%%%%
\subsection{Legacy Detection}
\label{sec:detection}

The directive |\childdocmain| in the main file can detect
whether the complete document or merely a child is to be compiled
even without using the directive |\childdocof|.
This method is deprecated because it is less robust
and there is no compelling reason to use it;
it is merely provided for backward compatibility
and it may be removed in future versions.

If the detection mechanism is to be used,
it is mandatory to correctly specify
the filename of the main file as the argument of |\childdocmain|:
%
\begin{center}
\begin{tabular}{l}
|\input{childdoc.def}|\\
|\childdocmain{|\textit{main}|}|\\
\end{tabular}
\end{center}
%
If |\jobname| does not match the argument \textit{main} of |\childdocmain|,
it is assumed that |\jobname| points to the child file to be compiled.
When using |\childdocmain| with the main file specified as argument,
it suffices to start a child file
with just |\input{|\textit{main}|}|
without loading of the package and using |\childdocof|.
If instead all processing is done
with the appropriate \textsf{childdoc} directives,
the argument of \textit{main} of |\childdocmain| can be empty.

An alternative version of the command line processing described
in \secref{sec:commandline} using the detection mechanism reads:
%
\begin{center}
|... -jobname "|\textit{target}|" "|[\textit{flags}]%
[|\def\jobname{|\textit{dest}|}|]|\input{|\textit{main}|}"|
\end{center}

%%%%%%%%%%%%%%%%%%%%%%%%%%%%%%%%%%%%%%%%%%%%%%%%%%%%%%%%%%%%%%%%%%%%%%%%%%%%%%%%
\subsection{Manual Code}
\label{sec:manual}

In case one cannot be certain whether the definitions file |childdoc.def|
is installed on the target \TeX{} distribution
and one prefers not to ship it,
it is conceivable to paste a few relevant commands into the sources.

To that end, drop all statements |\input{childdoc.def}|
and perform the replacements as outlined below.
Instead of |\childdocmain{|\textit{main}|}| add the following code
to the top of the main file:
%
\begin{center}
\begin{tabular}{l}
|\||ifdefined\childdocname\endinput\||fi\newif\ifchilddoc|\\
|\edef\childdocname{\scantokens\expandafter{\jobname\noexpand}}|\\
|\def\childdocmain{|\textit{main}|}\||ifx\childdocmain\childdocname\||else|\\
|\childdoctrue\includeonly{\childdocname}\let\jobname\childdocmain\||fi|\\
\end{tabular}
\end{center}
%
Instead of |\childdocof{|\textit{main}|}| just include the main file
at the top of each child file:
%
\begin{center}
|\input{|\textit{main}|}|
\end{center}
%
A simple redirection |\childdocforward{|\textit{dest}|}| is achieved by:
%
\begin{center}
|\def\jobname{|\textit{dest}|}\input{\jobname}|
\end{center}
%
The redirection with prefix
|\childdocforwardprefix[|\textit{prefix}|]{|\textit{dest}|}|
is accomplished by:
%
\begin{center}
\begin{tabular}{l}
|{\edef\jobname{\scantokens\expandafter{\jobname\noexpand}}|\\
|\def\redirectjob |\textit{prefix}|#1~~~{\gdef\jobname{|\textit{dest}|#1}}|\\
|\expandafter\redirectjob\jobname~~~}\input{\jobname}|
\end{tabular}
\end{center}

In an alternative approach,
child documents can be compiled by a specific command line
without additional code or specific definitions:
%
\begin{center}
|... -jobname "|\textit{target}|" "|[\textit{flags}]%
|\includeonly{|\textit{dest}|}\input{|\textit{main}|}"|
\end{center}
%

%%%%%%%%%%%%%%%%%%%%%%%%%%%%%%%%%%%%%%%%%%%%%%%%%%%%%%%%%%%%%%%%%%%%%%%%%%%%%%%%
%%%%%%%%%%%%%%%%%%%%%%%%%%%%%%%%%%%%%%%%%%%%%%%%%%%%%%%%%%%%%%%%%%%%%%%%%%%%%%%%
\section{Information}

%%%%%%%%%%%%%%%%%%%%%%%%%%%%%%%%%%%%%%%%%%%%%%%%%%%%%%%%%%%%%%%%%%%%%%%%%%%%%%%%
\subsection{Copyright}

Copyright \copyright{} 2017--2018 Niklas Beisert

This work may be distributed and/or modified under the
conditions of the \LaTeX{} Project Public License, either version 1.3
of this license or (at your option) any later version.
The latest version of this license is in
  \url{http://www.latex-project.org/lppl.txt}
and version 1.3 or later is part of all distributions of \LaTeX{}
version 2005/12/01 or later.

This work has the LPPL maintenance status `maintained'.

The Current Maintainer of this work is Niklas Beisert.

This work consists of the files |README.txt|, |childdoc.ins| and |childdoc.dtx|
as well as the derived files |childdoc.def|, |cdocsamp.tex|
with |cdocsch1.tex|, |cdocsch2.tex|, |cdocspt3.tex|, |cdocspt4.tex|,
|cdocsdrf.tex|, |cdocsfn1.tex|, |cdocsfn2.tex|
as well as |childdoc.pdf|.

%%%%%%%%%%%%%%%%%%%%%%%%%%%%%%%%%%%%%%%%%%%%%%%%%%%%%%%%%%%%%%%%%%%%%%%%%%%%%%%%
\subsection{Files and Installation}

The package consists of the files:
%
\begin{center}
\begin{tabular}{ll}
    |README.txt|   & readme file \\
    |childdoc.ins| & installation file \\
    |childdoc.dtx| & source file \\
    |childdoc.def| & definition file \\
    |cdocsamp.tex| & sample main file \\
    |cdocsch1.tex| & sample include file \\
    |cdocsch2.tex| & sample include file \\
    |cdocspt3.tex| & sample part file \\
    |cdocspt4.tex| & sample part file \\
    |cdocsdrf.tex| & sample redirection file \\
    |cdocsfn1.tex| & sample redirection file \\
    |cdocsfn2.tex| & sample redirection file \\
    |childdoc.pdf| & manual
\end{tabular}
\end{center}
%
The distribution consists of the files
|README.txt|, |childdoc.ins| and |childdoc.dtx|.
%
\begin{itemize}
\item
Run (pdf)\LaTeX{} on |childdoc.dtx|
to compile the manual |childdoc.pdf| (this file).
\item
Run \LaTeX{} on |childdoc.ins| to create the definitions file |childdoc.def|
and the sample |cdocsamp.tex| with include files
|cdocsch1.tex|, |cdocsch2.tex|, |cdocspt3.tex|, |cdocspt4.tex|,
|cdocsdrf.tex|, |cdocsfn1.tex|, |cdocsfn2.tex|.
Then copy the file |childdoc.def| to an appropriate directory of your \LaTeX{}
distribution, e.g.\ \textit{texmf-root}|/tex/latex/childdoc|.
\end{itemize}

%%%%%%%%%%%%%%%%%%%%%%%%%%%%%%%%%%%%%%%%%%%%%%%%%%%%%%%%%%%%%%%%%%%%%%%%%%%%%%%%
\subsection{Related CTAN Packages}

There are several other packages which offer a similar functionality:
%
\begin{itemize}
\item
The packages
\href{http://ctan.org/pkg/docmute}{\textsf{docmute}},
\href{http://ctan.org/pkg/includex}{\textsf{includex}} and
\href{http://ctan.org/pkg/standalone}{\textsf{standalone}}
provide commands to include only the document body of
a child file thus allowing both files to be compiled individually.
\item
The packages \href{http://ctan.org/pkg/subdocs}{\textsf{subdocs}}
and \href{http://ctan.org/pkg/subfiles}{\textsf{subfiles}}
provide structures in which the main and child documents can be
encapsulated and allowing them to be compiled individually.
The inclusion mechanism is different from the conventional |\include|.
\item
The package \href{http://ctan.org/pkg/combine}{\textsf{combine}}
is an elaborate solution to combine several documents into one.
\end{itemize}
%
See also the CTAN topic \href{http://ctan.org/topic/subdocs}{\textsf{subdocs}}
for further related packages.
The present package differs from the above solutions in that
a document structure constructed with the conventional |\include| mechanism
just needs two extra commands at the top of every file
such that all constituent files can be compiled individually.

%%%%%%%%%%%%%%%%%%%%%%%%%%%%%%%%%%%%%%%%%%%%%%%%%%%%%%%%%%%%%%%%%%%%%%%%%%%%%%%%
%\subsection{Feature Suggestions}
%
%The following is a list of features which may be useful for future
%versions of this package:
%%
%\begin{itemize}
%\item
%\ldots
%\end{itemize}

%%%%%%%%%%%%%%%%%%%%%%%%%%%%%%%%%%%%%%%%%%%%%%%%%%%%%%%%%%%%%%%%%%%%%%%%%%%%%%%%
\subsection{Revision History}

%%%%%%%%%%%%%%%%%%%%%%%%%%%%%%%%%%%%%%%%
\paragraph{v2.0:} 2018/12/30

\begin{itemize}
\item
immediate forward processing
\item
added |\childdocby| mechanism
\item
manual restructured
\end{itemize}

%%%%%%%%%%%%%%%%%%%%%%%%%%%%%%%%%%%%%%%%
\paragraph{v1.6:} 2018/01/17

\begin{itemize}
\item
application for development of include files
\item
corrections to manual
\end{itemize}

%%%%%%%%%%%%%%%%%%%%%%%%%%%%%%%%%%%%%%%%
\paragraph{v1.5:} 2017/05/21

\begin{itemize}
\item
more complete structuring introduced
\item
|\childdocof| introduced
\item
|\childdoc| renamed to |\childdocmain|
\item
|\childredirect| renamed to |\childdocforward| and |\childdocforwardprefix|
and functionality expanded
\end{itemize}

%%%%%%%%%%%%%%%%%%%%%%%%%%%%%%%%%%%%%%%%
\paragraph{v1.0:} 2017/04/27

\begin{itemize}
\item
manual and install package
\item
first version published on CTAN
\end{itemize}

%%%%%%%%%%%%%%%%%%%%%%%%%%%%%%%%%%%%%%%%
\paragraph{v0.6:} 2017/04/26

\begin{itemize}
\item
redirection mechanism added
\end{itemize}

%%%%%%%%%%%%%%%%%%%%%%%%%%%%%%%%%%%%%%%%
\paragraph{v0.5:} 2017/04/26

\begin{itemize}
\item
functionality in definition file
\end{itemize}


%%%%%%%%%%%%%%%%%%%%%%%%%%%%%%%%%%%%%%%%%%%%%%%%%%%%%%%%%%%%%%%%%%%%%%%%%%%%%%%%
%%%%%%%%%%%%%%%%%%%%%%%%%%%%%%%%%%%%%%%%%%%%%%%%%%%%%%%%%%%%%%%%%%%%%%%%%%%%%%%%
%%%%%%%%%%%%%%%%%%%%%%%%%%%%%%%%%%%%%%%%%%%%%%%%%%%%%%%%%%%%%%%%%%%%%%%%%%%%%%%%
\appendix

\settowidth\MacroIndent{\rmfamily\scriptsize 000\ }

 \DocInput{childdoc.dtx}

\end{document}
%</driver>
% \fi
%
% %%%%%%%%%%%%%%%%%%%%%%%%%%%%%%%%%%%%%%%%%%%%%%%%%%%%%%%%%%%%%%%%%%%%%%%%%%%%%%
% %%%%%%%%%%%%%%%%%%%%%%%%%%%%%%%%%%%%%%%%%%%%%%%%%%%%%%%%%%%%%%%%%%%%%%%%%%%%%%
% \section{Sample}
%\iffalse
%<*samplemain>
%\fi
%
% The following presents a sample document
% with two chapters, two parts, a title page,
% a compile flag as well as three forwarding files to set the flag.
% It consists of eight |.tex| files:
% \begin{center}
% \begin{tabular}{ll}
% |cdocsamp.tex|&main file\\
% |cdocsch1.tex|&include file for chapter 1\\
% |cdocsch2.tex|&include file for chapter 2\\
% |cdocspt3.tex|&include file for part 3\\
% |cdocspt4.tex|&include file for part 4\\
% |cdocsdrf.tex|&forwarding file for main file in draft mode\\
% |cdocsfi1.tex|&forwarding file for final version of chapter 1\\
% |cdocsfi2.tex|&forwarding file for final version of chapter 2\\
% \end{tabular}
% \end{center}
% Each of the eight files can be compiled directly by the \LaTeX{} compiler.
%
% %%%%%%%%%%%%%%%%%%%%%%%%%%%%%%%%%%%%%%
% \paragraph{Main File.}
%
% The main file is called |cdocsamp.tex|.
%
% Load the \textsf{childdoc} definitions and
% declare the filename for the main document:
%    \begin{macrocode}
\input{childdoc.def}
\childdocmain{}
%    \end{macrocode}

% Optional override for |\version| flag:
%    \begin{macrocode}
%%\ifchilddoc\else\providecommand{\version}{draft}\fi
%    \end{macrocode}

% Define the default values for the |\version| flag
% (|final| for the main file and |draft| for childs):
%    \begin{macrocode}
\ifchilddoc
\providecommand{\version}{draft}
\else
\providecommand{\version}{final}
\fi
%    \end{macrocode}

% Load the standard document class:
%    \begin{macrocode}
\documentclass[12pt]{article}
%    \end{macrocode}

% Start the document body:
%    \begin{macrocode}
\begin{document}
%    \end{macrocode}

% Declare a title page.
% Print title, part of document being processed and version flag:
%    \begin{macrocode}
\addtocounter{page}{-1}
\begin{center}
{\LARGE\bfseries{}childdoc example\par}
\vspace{1cm}
\ifchilddoc
\ifchilddocmanual part\else chapter\fi:
`\childdocname' of `\childdocjob'\par
\else
main document: `\childdocjob'\par
\fi
version: \version\par
\end{center}
\newpage
%    \end{macrocode}

% Manually include selected file,
% otherwise process as usual:
%    \begin{macrocode}
\ifchilddocmanual
\section*{part `\childdocname'}
\input{\childdocname}
\else
%    \end{macrocode}

% Include the two chapters:
%    \begin{macrocode}
\include{cdocsch1}
\include{cdocsch2}
%    \end{macrocode}

% Include the two parts unless only chapters should be displayed:
%    \begin{macrocode}
\ifchilddoc\else
\section{part three}
\input{cdocspt3}
\section{part four}
\input{cdocspt4}
\fi
%    \end{macrocode}

% Process as usual until here:
%    \begin{macrocode}
\fi
%    \end{macrocode}

% End of document body:
%    \begin{macrocode}
\end{document}
%    \end{macrocode}
%\iffalse
%</samplemain>
%\fi
%
% %%%%%%%%%%%%%%%%%%%%%%%%%%%%%%%%%%%%%%
% \paragraph{Chapter Include Files.}
%
% The include files are called |cdocsch1.tex| and |cdocsch2.tex|.
%
%\iffalse
%<*samplechap1|samplechap2>
%\fi

% Optional override for |\version| flag:
%    \begin{macrocode}
%%\providecommand{\version}{final}
%    \end{macrocode}

% Include the main document:
%    \begin{macrocode}
\input{childdoc.def}
\childdocof{cdocsamp}
%    \end{macrocode}

%\iffalse
%</samplechap1|samplechap2>
%\fi
%
%\iffalse
%<*samplechap1>
%\fi
% Some text for chapter 1:
%    \begin{macrocode}
\section{one}
some text in chapter one
%    \end{macrocode}

%\iffalse
%</samplechap1>
%\fi
% Some text for chapter 2:
%\iffalse
%<*samplechap2>
%\fi
%    \begin{macrocode}
\section{two}
more text in chapter two
%    \end{macrocode}

%\iffalse
%</samplechap2>
%\fi
%
% %%%%%%%%%%%%%%%%%%%%%%%%%%%%%%%%%%%%%%
% \paragraph{Part Include Files.}
%
% The include files are called |cdocspt3.tex| and |cdocspt4.tex|.
%
%\iffalse
%<*samplepart3|samplepart4>
%\fi

% Optional override for |\version| flag:
%    \begin{macrocode}
%%\providecommand{\version}{final}
%    \end{macrocode}

% Include the main document:
%    \begin{macrocode}
\input{childdoc.def}
\childdocby{cdocsamp}
%    \end{macrocode}

%\iffalse
%</samplepart3|samplepart4>
%\fi
%
%\iffalse
%<*samplepart3>
%\fi
% Some text for part 3:
%    \begin{macrocode}
some text in part three
%    \end{macrocode}

%\iffalse
%</samplepart3>
%\fi
% Some text for part 4:
%\iffalse
%<*samplepart4>
%\fi
%    \begin{macrocode}
more text in part four
%    \end{macrocode}

%\iffalse
%</samplepart4>
%\fi
%
% %%%%%%%%%%%%%%%%%%%%%%%%%%%%%%%%%%%%%%
% \paragraph{Forwarding for a Complete Draft.}
%
% The following forwarding file |cdocsdrf.tex|
% compiles the main document in draft mode:
%\iffalse
%<*sampledraft>
%\fi
%    \begin{macrocode}
\def\version{draft}
\input{childdoc.def}
\childdocforward{cdocsamp}
%    \end{macrocode}

%\iffalse
%</sampledraft>
%\fi
%
% %%%%%%%%%%%%%%%%%%%%%%%%%%%%%%%%%%%%%%
% \paragraph{Forwarding for Final Version of the Chapters.}
%
% The following forwarding files |cdocsfn1.tex| and |cdocsfn2.tex|
% (with identical content)
% compile the final versions of the child documents
% |cdocsch1.tex| and |cdocsch2.tex|, respectively:
%\iffalse
%<*samplefinal>
%\fi
%    \begin{macrocode}
\def\version{final}
\input{childdoc.def}
\childdocforwardprefix[cdocsamp]{cdocsfn}{cdocsch}
%    \end{macrocode}

%\iffalse
%</samplefinal>
%\fi
%
% %%%%%%%%%%%%%%%%%%%%%%%%%%%%%%%%%%%%%%
% \paragraph{Command Line Processing.}
%
% The following three command lines generate the output files
% |cdocscld|, |cdocscl1| and |cdocscl2|
% which should be identical to
% |cdocsdrf|, |cdocsch1| and |cdocsfn2|, respectively:
% \begin{center}
% \begin{tabular}{l}
% |latex -jobname cdocscld \|\\
% |  "\def\version{draft}\input{childdoc.def}\childdocforward{cdocsamp}"|\\
% |latex -jobname cdocscl1 \|\\
% |  "\input{childdoc.def}\childdocforward[cdocsamp]{cdocsch1}"|\\
% |latex -jobname cdocscl2 \|\\
% |  "\def\version{final}\input{childdoc.def}\childdocforward{cdocsch2}"|
% \end{tabular}
% \end{center}
% Note that the trailing backslash on each first line
% merely continues the input to the second line
% (for convenient cut ant paste).
% Furthermore, the command |latex| can be replaced by any
% of its alternative versions such as |pdflatex|.
%
% %%%%%%%%%%%%%%%%%%%%%%%%%%%%%%%%%%%%%%%%%%%%%%%%%%%%%%%%%%%%%%%%%%%%%%%%%%%%%%
% %%%%%%%%%%%%%%%%%%%%%%%%%%%%%%%%%%%%%%%%%%%%%%%%%%%%%%%%%%%%%%%%%%%%%%%%%%%%%%
% \section{Implementation}
%\iffalse
%<*package>
%\fi
%
% This section describes the definitions file |childdoc.def|.

% The definitions cannot be loaded using |\usepackage| or |\RequirePackage|
% which has a mechanism to prevent loading a style file more than once.
% When loading the definitions by means of |\input|
% multiple instances have to be prevented manually:
%\iffalse
%This code needs to be before the `\ProvidesFile' directive
%which is defined at the beginning of this file.
%Therefore it is also placed there and commented out here.
%</package>
%<*discard>
%\fi
%    \begin{macrocode}
\ifdefined\childdocmain\endinput\fi
%    \end{macrocode}
%\iffalse
%</discard>
%<*package>
%\fi
%
% \macro{\ifchilddoc}
% \macro{\ifchilddocmanual}
% The conditional |\ifchilddoc| tells whether a
% child (true) or main (false) document is being compiled.
% The conditional |\ifchilddocmanual| tells whether
% the |\includeonly| mechanism is used (false) or
% the selection of child files must be performed manually (true).
% The definitions initialise to false:
%    \begin{macrocode}
\newif\ifchilddoc
\newif\ifchilddocmanual
%    \end{macrocode}

% \macro{\childdocname}
% \macro{\childdocjob}
% The macro |\childdocname| stores the name of the main document
% to be compiled. The macro |\childdocjob| stores the name of
% the document on which the \LaTeX{} compiler was originally invoked.
% The content of |\jobname| cannot be compared
% to filenames specified in the source due to different catcodes.
% The following code rescans |\jobname|, stores the result
% in |\childdocname| and saves a copy in |\childdocjob|:
%    \begin{macrocode}
\edef\childdocname{\scantokens\expandafter{\jobname\noexpand}}
\let\childdocjob\childdocname
%    \end{macrocode}

% \macro{\childdocdisable}
% The macro |\childdocdisable| prevents the main file
% from being processed more than once.
% At this stage, the main document command |\childdocmain|
% is assumed to be called once again where it should do nothing.
% Any subsequent call to it should prevent
% a secondary processing of the main document
% It overwrites the forwarding commands
% |\childdocof| and |\childdocforward|
% with empty macros to prevent further inclusions of the main document:
%    \begin{macrocode}
\newcommand{\childdocdisable}
{
  \renewcommand{\childdocmain}[1]{\renewcommand{\childdocmain}[1]{\endinput}}
  \renewcommand{\childdocof}[1]{}
  \renewcommand{\childdocby}[2][]{}
  \renewcommand{\childdocforward}[2][]{}
  \renewcommand{\childdocdisable}{}
}
%    \end{macrocode}

% \macro{\childdocmain}
% The macro |\childdocmain| is to be called at the top of the main file
% with nothing or the main filename (without extension) as argument.
% First, it breaks loops.
% If the argument is not empty and does not match |\childdocname|
% (which is set by the first inclusion of |childdoc.def|),
% |\ifchilddoc| is set to true, |\includeonly| is applied to the child file
% and |\jobname| is set to the main file
% (for proper handling of |.aux| files):
%    \begin{macrocode}
\newcommand{\childdocmain}[1]
{
  \childdocdisable\childdocmain{}
  \if?#1?\else
    \begingroup
      \def\childdoctmp{#1}
      \ifx\childdoctmp\childdocname
        \def\childdoctmp{}
      \else
        \def\childdoctmp
        {
          \childdoctrue
          \includeonly{\childdocname}
          \def\childdocjob{#1}
          \def\jobname{#1}
        }
      \fi
      \expandafter
    \endgroup
    \childdoctmp
  \fi
}
%    \end{macrocode}

% \macro{\childdocof}
% The command |\childdocof| redirects
% compilation to the main file |#1|.
%    \begin{macrocode}
\newcommand{\childdocof}[1]
{
  \childdocdisable
  \childdoctrue
  \includeonly{\childdocname}
  \def\jobname{#1}
  \def\childdocjob{#1}
  \input{#1}
}
%    \end{macrocode}

% \macro{\childdocby}
% The command |\childdocby| ....
%    \begin{macrocode}
\newcommand{\childdocby}[2][]
{
  \childdocdisable
  \childdoctrue
  \childdocmanualtrue
  \if?#1?\else
    \def\jobname{#2}
  \fi
  \def\childdocjob{#2}
  \input{#2}
  \endinput
}
%    \end{macrocode}

% \macro{\childdocforward}
% The command |\childdocforward| redirects
% compilation to the main file or
% (if the optional argument is given) a child file.
% Parameters are set as if the main file
% or a child file starting with |\childdocof| was compiled.
% Then compilation is handed over to the main file:
%    \begin{macrocode}
\newcommand{\childdocforward}[2][]
{
  \begingroup
    \if?#1?
      \def\childdoctmp
      {
        \def\childdocname{#2}
        \def\childdocjob{#2}
        \def\jobname{#2}
        \input{#2}
        \endinput
      }
    \else
      \def\childdoctmp
      {
        \childdocdisable
        \def\childdocname{#2}
        \childdoctrue
        \includeonly{#2}
        \def\childdocjob{#1}
        \def\jobname{#1}
        \input{#1}
        \endinput
      }
    \fi
    \expandafter
  \endgroup
  \childdoctmp
}
%    \end{macrocode}

% \macro{\childdocforwardprefix}
% The command |\childdocforwardprefix| redirects
% compilation to the main or a child file by means of a pattern.
% The prefix |#1| in the current filename is replaced by |#2|
% and the suffix of the current filename is kept
% (it is assumed that the filename does not contain the substring `|~~~|'
% which is used as a delimiter).
% Compilation is handed over to the new file by |\childdocforward|:
%    \begin{macrocode}
\newcommand{\childdocforwardprefix}[3][]
{
  \begingroup
    \def\childdocextract #2##1~~~{\def\childdoctmp{\childdocforward[#1]{#3##1}}}
    \expandafter\childdocextract\childdocname~~~
    \expandafter
  \endgroup
  \childdoctmp
}
%    \end{macrocode}

% \macro{\childdoc}
% The deprecated macro |\childdoc| is a legacy version of |\childdocmain|:
%    \begin{macrocode}
\newcommand{\childdoc}{\childdocmain}
%    \end{macrocode}

% \macro{\childdocredirect}
% The deprecated macro |\childdocredirect| is a legacy version
% of |\childdocforward| and |\childdocforwardprefix|:
%    \begin{macrocode}
\newcommand{\childdocredirect}[2][]
{
  \begingroup
    \if?#1?
      \def\childdoctmp{\childdocforward{#2}}
    \else
      \def\childdoctmp{\childdocforwardprefix{#1}{#2}}
    \fi
    \expandafter
  \endgroup
  \childdoctmp
}
%    \end{macrocode}

%\iffalse
%</package>
%\fi
%
\endinput
|\\
|\childdocforward{|\textit{main}|}|\\
\end{tabular}
\end{center}
%
or alternatively with:
%
\begin{center}
\begin{tabular}{l}
|% \iffalse
%
% childdoc.dtx Copyright (C) 2017-2018 Niklas Beisert
%
% This work may be distributed and/or modified under the
% conditions of the LaTeX Project Public License, either version 1.3
% of this license or (at your option) any later version.
% The latest version of this license is in
%   http://www.latex-project.org/lppl.txt
% and version 1.3 or later is part of all distributions of LaTeX
% version 2005/12/01 or later.
%
% This work has the LPPL maintenance status `maintained'.
%
% The Current Maintainer of this work is Niklas Beisert.
%
% This work consists of the files childdoc.dtx and childdoc.ins
% and the derived files childdoc.def and cdocsamp.tex with
% cdocsch1.tex, cdocsch2.tex, cdocsdrf.tex, cdocsfn1.tex, cdocsfn2.tex.
%
%<package>\ifdefined\childdocmain\endinput\fi
%<package>\ProvidesFile{childdoc.def}[2018/12/30 v2.0 child document driver]
%<samplemain>\ProvidesFile{cdocsamp.tex}[2018/12/30 v2.0 sample for childdoc]
%<*driver>
%\ProvidesFile{childdoc.drv}[2018/12/30 v2.0 childdoc reference manual file]
\PassOptionsToClass{10pt,a4paper}{article}
\documentclass{ltxdoc}

\usepackage[margin=35mm]{geometry}
\usepackage{hyperref}
\usepackage{hyperxmp}
\usepackage[usenames]{color}

\hypersetup{colorlinks=true}
\hypersetup{pdfstartview=FitH}
\hypersetup{pdfpagemode=UseNone}
\hypersetup{pdfsource={}}
\hypersetup{pdflang={en-UK}}
\hypersetup{pdfcopyright={Copyright 2017-2018 Niklas Beisert.
  This work may be distributed and/or modified under the
  conditions of the LaTeX Project Public License, either version 1.3
  of this license or (at your option) any later version.}}
\hypersetup{pdflicenseurl={http://www.latex-project.org/lppl.txt}}
\hypersetup{pdfcontactaddress={ETH Zurich, ITP, HIT K,
  Wolfgang-Pauli-Strasse 27}}
\hypersetup{pdfcontactpostcode={8093}}
\hypersetup{pdfcontactcity={Zurich}}
\hypersetup{pdfcontactcountry={Switzerland}}
\hypersetup{pdfcontactemail={nbeisert@itp.phys.ethz.ch}}
\hypersetup{pdfcontacturl={http://people.phys.ethz.ch/\xmptilde nbeisert/}}

\newcommand{\secref}[1]{\hyperref[#1]{section \ref*{#1}}}

\parskip1ex
\parindent0pt
\let\olditemize\itemize
\def\itemize{\olditemize\parskip0pt}

\begin{document}

\title{The \textsf{childdoc} Package}
\hypersetup{pdftitle={The childdoc Package}}
\author{Niklas Beisert\\[2ex]
  Institut f\"ur Theoretische Physik\\
  Eidgen\"ossische Technische Hochschule Z\"urich\\
  Wolfgang-Pauli-Strasse 27, 8093 Z\"urich, Switzerland\\[1ex]
  \href{mailto:nbeisert@itp.phys.ethz.ch}
  {\texttt{nbeisert@itp.phys.ethz.ch}}}
\hypersetup{pdfauthor={Niklas Beisert}}
\hypersetup{pdfsubject={Manual for the LaTeX2e Package childdoc}}
\date{30 December 2018, \textsf{v2.0}}
\maketitle

\begin{abstract}\noindent
\textsf{childdoc} is a \LaTeXe{} package
that enables the direct compilation
of document sections included by |\include|
to individual files.
\end{abstract}

\begingroup
\parskip0ex
\tableofcontents
\endgroup

%%%%%%%%%%%%%%%%%%%%%%%%%%%%%%%%%%%%%%%%%%%%%%%%%%%%%%%%%%%%%%%%%%%%%%%%%%%%%%%%
%%%%%%%%%%%%%%%%%%%%%%%%%%%%%%%%%%%%%%%%%%%%%%%%%%%%%%%%%%%%%%%%%%%%%%%%%%%%%%%%
\section{Introduction}

\LaTeX{} provides a mechanism to structure a large document (such as a book)
into a main file and several child files (containing the chapters)
using the |\include| command.
This mechanism is beneficial for documents
which span hundreds of pages in order to
make the source file(s) more manageable.
Moreover, compilation can be restricted to
selected child files by means of the |\includeonly| command.
The latter feature can be used to reduce the compilation time while editing
(this was significantly more useful in the earlier days of \LaTeX{})
or to generate a smaller document which is easier to navigate.
Another application of |\includeonly| is to generate
documents consisting of selected parts of the complete document.

However, there are a few drawbacks of the plain |\include| mechanism:
\begin{itemize}
\item
The child files cannot be compiled on their own,
they can only be compiled via the main file.
A naive editing environment
(such as a text editor with an option
to have the current file processed by \LaTeX)
may require one to switch to the main file before compiling;
attempting to compile the child file produces errors.
\item
The main file must be modified (each time)
to adjust the |\includeonly| command
to the present needs. This easily leaves the main file in a messy state.
\item
The generated document will always carry the filename
of the main document. This is inconvenient if
several child files are to be compiled and
to be kept for distribution.
\end{itemize}

The present package provides a simple interface
to make child files individually compilable by \LaTeX{}.
Compiling a child file then has the same effect as compiling
the main file with an |\includeonly| command
to select the appropriate child.
Moreover the generated document will carry the name of the child
rather than the main file.
This resolves all three above issues.

This feature is meant to make the editing of books,
thesis documents and lecture notes somewhat more convenient.
However, the package can also be used efficiently for
composing a series of documents (such as exercise sheets)
which are typically distributed individually.
It then assists the author in generating the individual documents
(potentially in different versions)
as well as a document containing the collected series.
Another application is in developing style files
or other kinds of included material
where compilation of the style file could redirect
to a sample or test file.

%%%%%%%%%%%%%%%%%%%%%%%%%%%%%%%%%%%%%%%%%%%%%%%%%%%%%%%%%%%%%%%%%%%%%%%%%%%%%%%%
%%%%%%%%%%%%%%%%%%%%%%%%%%%%%%%%%%%%%%%%%%%%%%%%%%%%%%%%%%%%%%%%%%%%%%%%%%%%%%%%
\section{Usage}

First of all, the package \textsf{childdoc} is \emph{not} a standard
\LaTeXe{} |.sty| style file! Therefore it needs to be invoked in
a non-standard way.

%%%%%%%%%%%%%%%%%%%%%%%%%%%%%%%%%%%%%%%%%%%%%%%%%%%%%%%%%%%%%%%%%%%%%%%%%%%%%%%%
\subsection{Included Files}
\label{sec:include}

%%%%%%%%%%%%%%%%%%%%%%%%%%%%%%%%%%%%%%%%
\DescribeMacro{\childdocmain}
To use the package, add the commands
\begin{center}
\begin{tabular}{l}
|\input{childdoc.def}|\\
|\childdocmain{}|\\
\end{tabular}
\end{center}
at the very top of the main \LaTeX{} file,
in particular \emph{before} the |\documentclass| statement!
The argument of |\childdocmain| should be left empty
(but it must be present).

%%%%%%%%%%%%%%%%%%%%%%%%%%%%%%%%%%%%%%%%
\DescribeMacro{\childdocof}
Furthermore, add the commands
\begin{center}
\begin{tabular}{l}
|\input{childdoc.def}|\\
|\childdocof{|\textit{main}|}|\\
\end{tabular}
\end{center}
at the top of every child file \textit{child}
which is included by |\include{|\textit{child}|}|
from within the main file
(or at least for those files to be compiled individually).
The argument \textit{main} must be the filename of the main file.

There are a couple of
considerations in setting up the main and child documents:

%%%%%%%%%%%%%%%%%%%%%%%%%%%%%%%%%%%%%%%%
\paragraph{Restrictions.}

Please note the following restrictions:
\begin{itemize}
\item
|\childdocmain| must be called with one argument \textit{main}
to ensure compatibility with earlier version of the package.
It must either be empty (|\childdocmain{}|)
or precisely match the filename of the main file in which it is specified.
See \secref{sec:detection} for further information.
\item
The filename \textit{main} must be specified without the |.tex| extension.
\item
The filename \textit{main} is case sensitive
(even in case-insensitive file systems)
due to internal string comparison.
\item
The argument \textit{main} should be fully expanded, it cannot be a macro.
\item
Subdirectories and special characters should be avoided in filenames.
\item
The command |\childdocmain{|\textit{main}|}| must be followed by a whitespace.
It should not be followed immediately by another command
or by a comment mark `|%|'.
This is because the \TeX{} parser reads the token immediately following
the argument of |\childdocmain| and puts it
at the beginning of every child section;
however, a white\-space is ignored.
\end{itemize}

%%%%%%%%%%%%%%%%%%%%%%%%%%%%%%%%%%%%%%%%
\paragraph{Content of Main File.}

It is advisable to place all content in the child files included by |\include|.
Any output contained in the main file will appear in all child documents
unless suppressed manually;
it cannot be suppressed automatically by the |\includeonly| directive
and thus should normally be avoided.
A method to include some content in the main file
by means of conditional processing is described in \secref{sec:conditional}.

%%%%%%%%%%%%%%%%%%%%%%%%%%%%%%%%%%%%%%%%
\paragraph{Page Numbering.}

When only a part of the document is compiled,
the appropriate numbering of pages
(as well as other status parameters)
is determined from the |.aux| files.
The latter contain information from previous passes.
However this information needs to propagate through
all intermediate child documents.
Therefore the page numbering in child documents may well
be inconsistent until the complete document is compiled at least once.

A useful (if unconventional) way to always ensure a consistent
page numbering is to restart the numbering in each child document
and denote the pages by `\textit{child}|.|\textit{page}'
where \textit{child} represents the chapter/section number of the child file.
This can be achieved by the command
|\numberwithin{page}{|\textit{child}|}|
of the \textsf{amsmath} package
where \textit{child} can be |chapter| or |section|
depending on the chosen structuring.
Alternatively, one can modify the macro |\thepage| appropriately
and reset the counter |page| at the start of each child file.

%%%%%%%%%%%%%%%%%%%%%%%%%%%%%%%%%%%%%%%%%%%%%%%%%%%%%%%%%%%%%%%%%%%%%%%%%%%%%%%%
\subsection{Conditional Processing}
\label{sec:conditional}

The package provides a mechanism to compile different versions
of a document. To customise the versions further some conditional processing
can come in handy to distinguish which version is being compiled.
The package provides two macros to describe the compilation context:

%%%%%%%%%%%%%%%%%%%%%%%%%%%%%%%%%%%%%%%%
\DescribeMacro{\ifchilddoc}
The conditional |\ifchilddoc| distinguishes between the compilation of
child documents and the main document:
%
\begin{center}
|\ifchilddoc |\textit{child-code}| |[|\||else |\textit{main-code}]| \||fi|
\end{center}

%%%%%%%%%%%%%%%%%%%%%%%%%%%%%%%%%%%%%%%%
\DescribeMacro{\childdocname}
\DescribeMacro{\childdocjob}
The macro |\childdocname| contains the filename (without extension)
of the main or child file being processed.
Note that |\childdocjob| will always contain the name of the main file.

%%%%%%%%%%%%%%%%%%%%%%%%%%%%%%%%%%%%%%%%
\paragraph{Title Page.}

Conditional processing can be used to include a title or banner page
in the main document when proper precautions are taken.
Importantly, the code in the main file should ensure that the page counter
(as well as other status parameters which are stored in the |.aux| files)
takes the same value after the conditional processing.
Otherwise the page numbers may take divergent values
depending on which part is compiled.

For example, a title page could be declared by:
%
\begin{center}
\begin{tabular}{l}
|\ifchilddoc\||else|\\
|\addtocounter{page}{-1}|\\
\textit{code for title page}\\
|\newpage|\\
|\||fi|
\end{tabular}
\end{center}
%
A banner page for the child documents can be generated by:
%
\begin{center}
\begin{tabular}{l}
|\ifchilddoc|\\
|\addtocounter{page}{-1}|\\
\textit{code for banner page}\\
|\newpage|\\
|\||fi|
\end{tabular}
\end{center}
%
Here one could write a message such as:
\begin{center}
|This is the part \childdocname{} of \childdocjob{}.|
\end{center}

%%%%%%%%%%%%%%%%%%%%%%%%%%%%%%%%%%%%%%%%%%%%%%%%%%%%%%%%%%%%%%%%%%%%%%%%%%%%%%%%
\subsection{Flags}
\label{sec:flags}

The package makes it easy to generate different versions
of the main or child documents.
To this end compilation flags can be defined
and assigned different default values.
They will be particularly useful in conjunction
with the forwarding mechanism described in \secref{sec:forward}.

For example, it may be useful to have a flag |\version|
which can be set to |draft| or |final|.
The document source will contain some conditional code
depending on the value of |\version|.
Suppose further, the flag should default to |final| for the main file
and to |draft| for child files
which is a natural assignment for editing the document.
This is achieved by placing the following code
in the preamble of the main document
(below the |\childdocmain| directive):
%
\begin{center}
\begin{tabular}{l}
|\ifchilddoc|\\
|\providecommand{\version}{draft}|\\
|\||else|\\
|\providecommand{\version}{final}|\\
|\||fi|
\end{tabular}
\end{center}
%
The definition by |\providecommand| makes sure
that previous definitions are not overwritten.
Further statements |\providecommand{\version}{...}|
can thus be added before the above code to override it.

For the main file, one might add a line
(between |\childdocmain| and the above block)
%
\begin{center}
|%\ifchilddoc\||else\providecommand{\version}{draft}\||fi|
\end{center}
%
which can be uncommented to produce a draft version.
Likewise one can add a line to the very top of a child file
(above the |\childdocof{|\textit{main}|}| directive)
%
\begin{center}
|%\providecommand{\version}{final}|
\end{center}
%
which can be uncommented to produce the final version of this child document.

%%%%%%%%%%%%%%%%%%%%%%%%%%%%%%%%%%%%%%%%%%%%%%%%%%%%%%%%%%%%%%%%%%%%%%%%%%%%%%%%
\subsection{Forwarding}
\label{sec:forward}

Different versions of the main or child documents
using compilation flags as described in \secref{sec:flags}
can be (permanently) stored in different files
for convenient compilation, viewing and distribution.
To this end, the package defines a command
to pass on compilation to a different file:

%%%%%%%%%%%%%%%%%%%%%%%%%%%%%%%%%%%%%%%%
\DescribeMacro{\childdocforward}
The command |\childdocforward| redirects processing to
another source file:
%
\begin{center}
\begin{tabular}{l}
|\input{childdoc.def}|\\
|\childdocforward[|\textit{main}|]{|\textit{dest}|}|\\
\end{tabular}
\end{center}
%
The argument \textit{dest} is the destination file
(without extension).
It should be the main file or one of the child files.
Note that further \textsf{childdoc} directives
such as |\childdocof| and |\childdocforward|
in the indicated file will be processed in this form.
The optional argument \textit{main}
passes on directly to the main file \textit{main}
while pretending to compile the child \textit{dest}.
This form behaves as if \textit{dest}
issues |\childdocof{|\textit{main}|}| right away,
and no further \textsf{childdoc} directives will be processed.

%%%%%%%%%%%%%%%%%%%%%%%%%%%%%%%%%%%%%%%%
\DescribeMacro{\...prefix}
In the alternative form |\childdocforwardprefix|,
%
\begin{center}
\begin{tabular}{l}
|\input{childdoc.def}|\\
|\childdocforwardprefix[|\textit{main}|]{|\textit{prefix}|}{|\textit{dest}|}|
\end{tabular}
\end{center}
%
the destination file is determined by a pattern
depending on the current file:
To make this work, the current file must be called
`{\textit{prefix}\hspace{0.2em}\textit{suffix}}'
with \textit{prefix} matching precisely the argument.
Processing is then passed on to the file
`{\textit{dest}\hspace{0.2em}\textit{suffix}}'.
Surely, the same effect is achieved by
directly specifying the
argument `{\textit{dest}\hspace{0.2em}\textit{suffix}}'
in the first form.
However, that requires to set up a different file
for each child. With the alternative form of the command
all these files can have exactly the same content
which simplifies setting them up and maintaining them.

For example, the following file |draft.tex|
with a compilation flag |\version| as described in \secref{sec:flags}
compiles the main document as a draft:
%
\begin{center}
\begin{tabular}{l}
|\def\version{draft}|\\
|\input{childdoc.def}|\\
|\childdocforward{|\textit{main}|}|
\end{tabular}
\end{center}
%
Likewise, the following files |final|\textit{nn}|.tex|
compile the final version of the child document
|child|\textit{nn}|.tex|:
%
\begin{center}
\begin{tabular}{l}
|\def\version{final}|\\
|\input{childdoc.def}|\\
|\childdocforwardprefix{final}{child}|
\end{tabular}
\end{center}
%

Note that when several versions of a main file and/or of each child file
are to be generated, it may be convenient to set up a |Makefile| or
shell script to automatise the process.

%%%%%%%%%%%%%%%%%%%%%%%%%%%%%%%%%%%%%%%%%%%%%%%%%%%%%%%%%%%%%%%%%%%%%%%%%%%%%%%%
\subsection{Command Line Processing}
\label{sec:commandline}

The effect of redirection files can also be achieved by invoking
the \LaTeX{} compiler with a more elaborate command line.
Most conveniently this should be done as part
of a shell script or a |Makefile|.

When using \textsf{childdoc} in the main file, the following
command lines effectively perform a redirection
(note that depending on the shell being used,
backslashes may have to be doubled: `|\|' $\to$ `|\\|'):
%
\begin{center}
|... -jobname "|\textit{target}|" |\\|"|[\textit{flags}]%
|\input{childdoc.def}\childdocforward[|\textit{main}|]{|\textit{dest}|}"|
\end{center}
%
Here \textit{target} is the name of the output file,
\textit{main} is the name of the main file
and \textit{dest} is the name of the main or child file to be processed
(all filenames without extensions).
The optional argument \textit{main} can be omitted
if \textit{main} matches \textit{dest}.
Optionally, compilation \textit{flags} can be defined via |\def| commands.
This command line makes the \TeX{} engine believe
it is compiling the file \textit{target}
whose content is specified as the latter parameter.
The provided code then forwards the processing to
\textit{main} or \textit{dest} as described in \secref{sec:forward}.

%%%%%%%%%%%%%%%%%%%%%%%%%%%%%%%%%%%%%%%%%%%%%%%%%%%%%%%%%%%%%%%%%%%%%%%%%%%%%%%%
\subsection{Include by Input}
\label{sec:input}

Including child documents by |\include| has some restrictions by design.
Most notably, the content of a child document always occupies
its own set of pages; pages cannot be shared between child documents.
Usually, this behaviour makes perfect sense
because each child document contain an essential part of the document.
However, in some situations it may be desirable to compose
a document from a collection of parts
without having mandatory page breaks between then.
For this case, the package
provides a mechanism to include parts
by |\input| which can also be processed individually.
However, by construction this mechanism
requires manual handling of the content to be output.

%%%%%%%%%%%%%%%%%%%%%%%%%%%%%%%%%%%%%%%%
\DescribeMacro{\ifchilddocmanual}
The main file should be prepared as usual, see \secref{sec:include}.
However, the document body must make a distinction
between processing of an individual part and of the main document, e.g.:
%
\begin{center}
\begin{tabular}{l}
|\ifchilddocmanual|\\
|\input{\childdocname}|\\
|\||else|\\
\textit{document body with }|\input{|\textit{part}|}|\\
|\||fi|
\end{tabular}
\end{center}
%
The conditional |\ifchilddocmanual| is true whenever
a part to be included by |\input| is being compiled,
and the name of the part is stored in |\childdocname|.

%%%%%%%%%%%%%%%%%%%%%%%%%%%%%%%%%%%%%%%%
\DescribeMacro{\childdocby}
Each part to be included by |\input| should start with:
%
\begin{center}
\begin{tabular}{l}
|\input{childdoc.def}|\\
|\childdocby{|\textit{main}|}|\\
\end{tabular}
\end{center}
%
The directive |\childdocby| is similar to |\childdocof|
described in \secref{sec:include},
but the subsequent selection of content must be done manually.
To that end, both |\ifchilddoc| and |\ifchilddocmanual|
will be true upon processing of a part,
and the name of the part is stored in |\childdocname|.
Note that |\jobname| will be set to the filename of the current part
so that each part receives an individual |.aux| file
that does not interfere with the |.aux| file(s) of the main document.
This behaviour can be altered by the alternative form
|\childdocby[*]{|\textit{main}|}| (with a non-empty optional argument)
which uses the |.aux| file of the main document
by setting |\jobname| to \textit{main}.

%%%%%%%%%%%%%%%%%%%%%%%%%%%%%%%%%%%%%%%%%%%%%%%%%%%%%%%%%%%%%%%%%%%%%%%%%%%%%%%%
\subsection{Driver Development}
\label{sec:driver}

The \textsf{childdoc} mechanism can also be use for the development
of definition files such as \LaTeX{} styles or classes.
This case differs from the above setup with multiple parts
included by |\include| in that no |\includeonly| should be invoked.
This can be achieved by starting the include file
(before |\ProvidesPackage|) with:
%
\begin{center}
\begin{tabular}{l}
|\input{childdoc.def}|\\
|\childdocforward{|\textit{main}|}|\\
\end{tabular}
\end{center}
%
or alternatively with:
%
\begin{center}
\begin{tabular}{l}
|\input{childdoc.def}|\\
|\childdocby{|\textit{main}|}|\\
\end{tabular}
\end{center}
%
Both forms have slightly different effects as described above.
The main file is prepared as usual, see \secref{sec:include}.

%%%%%%%%%%%%%%%%%%%%%%%%%%%%%%%%%%%%%%%%%%%%%%%%%%%%%%%%%%%%%%%%%%%%%%%%%%%%%%%%
\subsection{Legacy Detection}
\label{sec:detection}

The directive |\childdocmain| in the main file can detect
whether the complete document or merely a child is to be compiled
even without using the directive |\childdocof|.
This method is deprecated because it is less robust
and there is no compelling reason to use it;
it is merely provided for backward compatibility
and it may be removed in future versions.

If the detection mechanism is to be used,
it is mandatory to correctly specify
the filename of the main file as the argument of |\childdocmain|:
%
\begin{center}
\begin{tabular}{l}
|\input{childdoc.def}|\\
|\childdocmain{|\textit{main}|}|\\
\end{tabular}
\end{center}
%
If |\jobname| does not match the argument \textit{main} of |\childdocmain|,
it is assumed that |\jobname| points to the child file to be compiled.
When using |\childdocmain| with the main file specified as argument,
it suffices to start a child file
with just |\input{|\textit{main}|}|
without loading of the package and using |\childdocof|.
If instead all processing is done
with the appropriate \textsf{childdoc} directives,
the argument of \textit{main} of |\childdocmain| can be empty.

An alternative version of the command line processing described
in \secref{sec:commandline} using the detection mechanism reads:
%
\begin{center}
|... -jobname "|\textit{target}|" "|[\textit{flags}]%
[|\def\jobname{|\textit{dest}|}|]|\input{|\textit{main}|}"|
\end{center}

%%%%%%%%%%%%%%%%%%%%%%%%%%%%%%%%%%%%%%%%%%%%%%%%%%%%%%%%%%%%%%%%%%%%%%%%%%%%%%%%
\subsection{Manual Code}
\label{sec:manual}

In case one cannot be certain whether the definitions file |childdoc.def|
is installed on the target \TeX{} distribution
and one prefers not to ship it,
it is conceivable to paste a few relevant commands into the sources.

To that end, drop all statements |\input{childdoc.def}|
and perform the replacements as outlined below.
Instead of |\childdocmain{|\textit{main}|}| add the following code
to the top of the main file:
%
\begin{center}
\begin{tabular}{l}
|\||ifdefined\childdocname\endinput\||fi\newif\ifchilddoc|\\
|\edef\childdocname{\scantokens\expandafter{\jobname\noexpand}}|\\
|\def\childdocmain{|\textit{main}|}\||ifx\childdocmain\childdocname\||else|\\
|\childdoctrue\includeonly{\childdocname}\let\jobname\childdocmain\||fi|\\
\end{tabular}
\end{center}
%
Instead of |\childdocof{|\textit{main}|}| just include the main file
at the top of each child file:
%
\begin{center}
|\input{|\textit{main}|}|
\end{center}
%
A simple redirection |\childdocforward{|\textit{dest}|}| is achieved by:
%
\begin{center}
|\def\jobname{|\textit{dest}|}\input{\jobname}|
\end{center}
%
The redirection with prefix
|\childdocforwardprefix[|\textit{prefix}|]{|\textit{dest}|}|
is accomplished by:
%
\begin{center}
\begin{tabular}{l}
|{\edef\jobname{\scantokens\expandafter{\jobname\noexpand}}|\\
|\def\redirectjob |\textit{prefix}|#1~~~{\gdef\jobname{|\textit{dest}|#1}}|\\
|\expandafter\redirectjob\jobname~~~}\input{\jobname}|
\end{tabular}
\end{center}

In an alternative approach,
child documents can be compiled by a specific command line
without additional code or specific definitions:
%
\begin{center}
|... -jobname "|\textit{target}|" "|[\textit{flags}]%
|\includeonly{|\textit{dest}|}\input{|\textit{main}|}"|
\end{center}
%

%%%%%%%%%%%%%%%%%%%%%%%%%%%%%%%%%%%%%%%%%%%%%%%%%%%%%%%%%%%%%%%%%%%%%%%%%%%%%%%%
%%%%%%%%%%%%%%%%%%%%%%%%%%%%%%%%%%%%%%%%%%%%%%%%%%%%%%%%%%%%%%%%%%%%%%%%%%%%%%%%
\section{Information}

%%%%%%%%%%%%%%%%%%%%%%%%%%%%%%%%%%%%%%%%%%%%%%%%%%%%%%%%%%%%%%%%%%%%%%%%%%%%%%%%
\subsection{Copyright}

Copyright \copyright{} 2017--2018 Niklas Beisert

This work may be distributed and/or modified under the
conditions of the \LaTeX{} Project Public License, either version 1.3
of this license or (at your option) any later version.
The latest version of this license is in
  \url{http://www.latex-project.org/lppl.txt}
and version 1.3 or later is part of all distributions of \LaTeX{}
version 2005/12/01 or later.

This work has the LPPL maintenance status `maintained'.

The Current Maintainer of this work is Niklas Beisert.

This work consists of the files |README.txt|, |childdoc.ins| and |childdoc.dtx|
as well as the derived files |childdoc.def|, |cdocsamp.tex|
with |cdocsch1.tex|, |cdocsch2.tex|, |cdocspt3.tex|, |cdocspt4.tex|,
|cdocsdrf.tex|, |cdocsfn1.tex|, |cdocsfn2.tex|
as well as |childdoc.pdf|.

%%%%%%%%%%%%%%%%%%%%%%%%%%%%%%%%%%%%%%%%%%%%%%%%%%%%%%%%%%%%%%%%%%%%%%%%%%%%%%%%
\subsection{Files and Installation}

The package consists of the files:
%
\begin{center}
\begin{tabular}{ll}
    |README.txt|   & readme file \\
    |childdoc.ins| & installation file \\
    |childdoc.dtx| & source file \\
    |childdoc.def| & definition file \\
    |cdocsamp.tex| & sample main file \\
    |cdocsch1.tex| & sample include file \\
    |cdocsch2.tex| & sample include file \\
    |cdocspt3.tex| & sample part file \\
    |cdocspt4.tex| & sample part file \\
    |cdocsdrf.tex| & sample redirection file \\
    |cdocsfn1.tex| & sample redirection file \\
    |cdocsfn2.tex| & sample redirection file \\
    |childdoc.pdf| & manual
\end{tabular}
\end{center}
%
The distribution consists of the files
|README.txt|, |childdoc.ins| and |childdoc.dtx|.
%
\begin{itemize}
\item
Run (pdf)\LaTeX{} on |childdoc.dtx|
to compile the manual |childdoc.pdf| (this file).
\item
Run \LaTeX{} on |childdoc.ins| to create the definitions file |childdoc.def|
and the sample |cdocsamp.tex| with include files
|cdocsch1.tex|, |cdocsch2.tex|, |cdocspt3.tex|, |cdocspt4.tex|,
|cdocsdrf.tex|, |cdocsfn1.tex|, |cdocsfn2.tex|.
Then copy the file |childdoc.def| to an appropriate directory of your \LaTeX{}
distribution, e.g.\ \textit{texmf-root}|/tex/latex/childdoc|.
\end{itemize}

%%%%%%%%%%%%%%%%%%%%%%%%%%%%%%%%%%%%%%%%%%%%%%%%%%%%%%%%%%%%%%%%%%%%%%%%%%%%%%%%
\subsection{Related CTAN Packages}

There are several other packages which offer a similar functionality:
%
\begin{itemize}
\item
The packages
\href{http://ctan.org/pkg/docmute}{\textsf{docmute}},
\href{http://ctan.org/pkg/includex}{\textsf{includex}} and
\href{http://ctan.org/pkg/standalone}{\textsf{standalone}}
provide commands to include only the document body of
a child file thus allowing both files to be compiled individually.
\item
The packages \href{http://ctan.org/pkg/subdocs}{\textsf{subdocs}}
and \href{http://ctan.org/pkg/subfiles}{\textsf{subfiles}}
provide structures in which the main and child documents can be
encapsulated and allowing them to be compiled individually.
The inclusion mechanism is different from the conventional |\include|.
\item
The package \href{http://ctan.org/pkg/combine}{\textsf{combine}}
is an elaborate solution to combine several documents into one.
\end{itemize}
%
See also the CTAN topic \href{http://ctan.org/topic/subdocs}{\textsf{subdocs}}
for further related packages.
The present package differs from the above solutions in that
a document structure constructed with the conventional |\include| mechanism
just needs two extra commands at the top of every file
such that all constituent files can be compiled individually.

%%%%%%%%%%%%%%%%%%%%%%%%%%%%%%%%%%%%%%%%%%%%%%%%%%%%%%%%%%%%%%%%%%%%%%%%%%%%%%%%
%\subsection{Feature Suggestions}
%
%The following is a list of features which may be useful for future
%versions of this package:
%%
%\begin{itemize}
%\item
%\ldots
%\end{itemize}

%%%%%%%%%%%%%%%%%%%%%%%%%%%%%%%%%%%%%%%%%%%%%%%%%%%%%%%%%%%%%%%%%%%%%%%%%%%%%%%%
\subsection{Revision History}

%%%%%%%%%%%%%%%%%%%%%%%%%%%%%%%%%%%%%%%%
\paragraph{v2.0:} 2018/12/30

\begin{itemize}
\item
immediate forward processing
\item
added |\childdocby| mechanism
\item
manual restructured
\end{itemize}

%%%%%%%%%%%%%%%%%%%%%%%%%%%%%%%%%%%%%%%%
\paragraph{v1.6:} 2018/01/17

\begin{itemize}
\item
application for development of include files
\item
corrections to manual
\end{itemize}

%%%%%%%%%%%%%%%%%%%%%%%%%%%%%%%%%%%%%%%%
\paragraph{v1.5:} 2017/05/21

\begin{itemize}
\item
more complete structuring introduced
\item
|\childdocof| introduced
\item
|\childdoc| renamed to |\childdocmain|
\item
|\childredirect| renamed to |\childdocforward| and |\childdocforwardprefix|
and functionality expanded
\end{itemize}

%%%%%%%%%%%%%%%%%%%%%%%%%%%%%%%%%%%%%%%%
\paragraph{v1.0:} 2017/04/27

\begin{itemize}
\item
manual and install package
\item
first version published on CTAN
\end{itemize}

%%%%%%%%%%%%%%%%%%%%%%%%%%%%%%%%%%%%%%%%
\paragraph{v0.6:} 2017/04/26

\begin{itemize}
\item
redirection mechanism added
\end{itemize}

%%%%%%%%%%%%%%%%%%%%%%%%%%%%%%%%%%%%%%%%
\paragraph{v0.5:} 2017/04/26

\begin{itemize}
\item
functionality in definition file
\end{itemize}


%%%%%%%%%%%%%%%%%%%%%%%%%%%%%%%%%%%%%%%%%%%%%%%%%%%%%%%%%%%%%%%%%%%%%%%%%%%%%%%%
%%%%%%%%%%%%%%%%%%%%%%%%%%%%%%%%%%%%%%%%%%%%%%%%%%%%%%%%%%%%%%%%%%%%%%%%%%%%%%%%
%%%%%%%%%%%%%%%%%%%%%%%%%%%%%%%%%%%%%%%%%%%%%%%%%%%%%%%%%%%%%%%%%%%%%%%%%%%%%%%%
\appendix

\settowidth\MacroIndent{\rmfamily\scriptsize 000\ }

 \DocInput{childdoc.dtx}

\end{document}
%</driver>
% \fi
%
% %%%%%%%%%%%%%%%%%%%%%%%%%%%%%%%%%%%%%%%%%%%%%%%%%%%%%%%%%%%%%%%%%%%%%%%%%%%%%%
% %%%%%%%%%%%%%%%%%%%%%%%%%%%%%%%%%%%%%%%%%%%%%%%%%%%%%%%%%%%%%%%%%%%%%%%%%%%%%%
% \section{Sample}
%\iffalse
%<*samplemain>
%\fi
%
% The following presents a sample document
% with two chapters, two parts, a title page,
% a compile flag as well as three forwarding files to set the flag.
% It consists of eight |.tex| files:
% \begin{center}
% \begin{tabular}{ll}
% |cdocsamp.tex|&main file\\
% |cdocsch1.tex|&include file for chapter 1\\
% |cdocsch2.tex|&include file for chapter 2\\
% |cdocspt3.tex|&include file for part 3\\
% |cdocspt4.tex|&include file for part 4\\
% |cdocsdrf.tex|&forwarding file for main file in draft mode\\
% |cdocsfi1.tex|&forwarding file for final version of chapter 1\\
% |cdocsfi2.tex|&forwarding file for final version of chapter 2\\
% \end{tabular}
% \end{center}
% Each of the eight files can be compiled directly by the \LaTeX{} compiler.
%
% %%%%%%%%%%%%%%%%%%%%%%%%%%%%%%%%%%%%%%
% \paragraph{Main File.}
%
% The main file is called |cdocsamp.tex|.
%
% Load the \textsf{childdoc} definitions and
% declare the filename for the main document:
%    \begin{macrocode}
\input{childdoc.def}
\childdocmain{}
%    \end{macrocode}

% Optional override for |\version| flag:
%    \begin{macrocode}
%%\ifchilddoc\else\providecommand{\version}{draft}\fi
%    \end{macrocode}

% Define the default values for the |\version| flag
% (|final| for the main file and |draft| for childs):
%    \begin{macrocode}
\ifchilddoc
\providecommand{\version}{draft}
\else
\providecommand{\version}{final}
\fi
%    \end{macrocode}

% Load the standard document class:
%    \begin{macrocode}
\documentclass[12pt]{article}
%    \end{macrocode}

% Start the document body:
%    \begin{macrocode}
\begin{document}
%    \end{macrocode}

% Declare a title page.
% Print title, part of document being processed and version flag:
%    \begin{macrocode}
\addtocounter{page}{-1}
\begin{center}
{\LARGE\bfseries{}childdoc example\par}
\vspace{1cm}
\ifchilddoc
\ifchilddocmanual part\else chapter\fi:
`\childdocname' of `\childdocjob'\par
\else
main document: `\childdocjob'\par
\fi
version: \version\par
\end{center}
\newpage
%    \end{macrocode}

% Manually include selected file,
% otherwise process as usual:
%    \begin{macrocode}
\ifchilddocmanual
\section*{part `\childdocname'}
\input{\childdocname}
\else
%    \end{macrocode}

% Include the two chapters:
%    \begin{macrocode}
\include{cdocsch1}
\include{cdocsch2}
%    \end{macrocode}

% Include the two parts unless only chapters should be displayed:
%    \begin{macrocode}
\ifchilddoc\else
\section{part three}
\input{cdocspt3}
\section{part four}
\input{cdocspt4}
\fi
%    \end{macrocode}

% Process as usual until here:
%    \begin{macrocode}
\fi
%    \end{macrocode}

% End of document body:
%    \begin{macrocode}
\end{document}
%    \end{macrocode}
%\iffalse
%</samplemain>
%\fi
%
% %%%%%%%%%%%%%%%%%%%%%%%%%%%%%%%%%%%%%%
% \paragraph{Chapter Include Files.}
%
% The include files are called |cdocsch1.tex| and |cdocsch2.tex|.
%
%\iffalse
%<*samplechap1|samplechap2>
%\fi

% Optional override for |\version| flag:
%    \begin{macrocode}
%%\providecommand{\version}{final}
%    \end{macrocode}

% Include the main document:
%    \begin{macrocode}
\input{childdoc.def}
\childdocof{cdocsamp}
%    \end{macrocode}

%\iffalse
%</samplechap1|samplechap2>
%\fi
%
%\iffalse
%<*samplechap1>
%\fi
% Some text for chapter 1:
%    \begin{macrocode}
\section{one}
some text in chapter one
%    \end{macrocode}

%\iffalse
%</samplechap1>
%\fi
% Some text for chapter 2:
%\iffalse
%<*samplechap2>
%\fi
%    \begin{macrocode}
\section{two}
more text in chapter two
%    \end{macrocode}

%\iffalse
%</samplechap2>
%\fi
%
% %%%%%%%%%%%%%%%%%%%%%%%%%%%%%%%%%%%%%%
% \paragraph{Part Include Files.}
%
% The include files are called |cdocspt3.tex| and |cdocspt4.tex|.
%
%\iffalse
%<*samplepart3|samplepart4>
%\fi

% Optional override for |\version| flag:
%    \begin{macrocode}
%%\providecommand{\version}{final}
%    \end{macrocode}

% Include the main document:
%    \begin{macrocode}
\input{childdoc.def}
\childdocby{cdocsamp}
%    \end{macrocode}

%\iffalse
%</samplepart3|samplepart4>
%\fi
%
%\iffalse
%<*samplepart3>
%\fi
% Some text for part 3:
%    \begin{macrocode}
some text in part three
%    \end{macrocode}

%\iffalse
%</samplepart3>
%\fi
% Some text for part 4:
%\iffalse
%<*samplepart4>
%\fi
%    \begin{macrocode}
more text in part four
%    \end{macrocode}

%\iffalse
%</samplepart4>
%\fi
%
% %%%%%%%%%%%%%%%%%%%%%%%%%%%%%%%%%%%%%%
% \paragraph{Forwarding for a Complete Draft.}
%
% The following forwarding file |cdocsdrf.tex|
% compiles the main document in draft mode:
%\iffalse
%<*sampledraft>
%\fi
%    \begin{macrocode}
\def\version{draft}
\input{childdoc.def}
\childdocforward{cdocsamp}
%    \end{macrocode}

%\iffalse
%</sampledraft>
%\fi
%
% %%%%%%%%%%%%%%%%%%%%%%%%%%%%%%%%%%%%%%
% \paragraph{Forwarding for Final Version of the Chapters.}
%
% The following forwarding files |cdocsfn1.tex| and |cdocsfn2.tex|
% (with identical content)
% compile the final versions of the child documents
% |cdocsch1.tex| and |cdocsch2.tex|, respectively:
%\iffalse
%<*samplefinal>
%\fi
%    \begin{macrocode}
\def\version{final}
\input{childdoc.def}
\childdocforwardprefix[cdocsamp]{cdocsfn}{cdocsch}
%    \end{macrocode}

%\iffalse
%</samplefinal>
%\fi
%
% %%%%%%%%%%%%%%%%%%%%%%%%%%%%%%%%%%%%%%
% \paragraph{Command Line Processing.}
%
% The following three command lines generate the output files
% |cdocscld|, |cdocscl1| and |cdocscl2|
% which should be identical to
% |cdocsdrf|, |cdocsch1| and |cdocsfn2|, respectively:
% \begin{center}
% \begin{tabular}{l}
% |latex -jobname cdocscld \|\\
% |  "\def\version{draft}\input{childdoc.def}\childdocforward{cdocsamp}"|\\
% |latex -jobname cdocscl1 \|\\
% |  "\input{childdoc.def}\childdocforward[cdocsamp]{cdocsch1}"|\\
% |latex -jobname cdocscl2 \|\\
% |  "\def\version{final}\input{childdoc.def}\childdocforward{cdocsch2}"|
% \end{tabular}
% \end{center}
% Note that the trailing backslash on each first line
% merely continues the input to the second line
% (for convenient cut ant paste).
% Furthermore, the command |latex| can be replaced by any
% of its alternative versions such as |pdflatex|.
%
% %%%%%%%%%%%%%%%%%%%%%%%%%%%%%%%%%%%%%%%%%%%%%%%%%%%%%%%%%%%%%%%%%%%%%%%%%%%%%%
% %%%%%%%%%%%%%%%%%%%%%%%%%%%%%%%%%%%%%%%%%%%%%%%%%%%%%%%%%%%%%%%%%%%%%%%%%%%%%%
% \section{Implementation}
%\iffalse
%<*package>
%\fi
%
% This section describes the definitions file |childdoc.def|.

% The definitions cannot be loaded using |\usepackage| or |\RequirePackage|
% which has a mechanism to prevent loading a style file more than once.
% When loading the definitions by means of |\input|
% multiple instances have to be prevented manually:
%\iffalse
%This code needs to be before the `\ProvidesFile' directive
%which is defined at the beginning of this file.
%Therefore it is also placed there and commented out here.
%</package>
%<*discard>
%\fi
%    \begin{macrocode}
\ifdefined\childdocmain\endinput\fi
%    \end{macrocode}
%\iffalse
%</discard>
%<*package>
%\fi
%
% \macro{\ifchilddoc}
% \macro{\ifchilddocmanual}
% The conditional |\ifchilddoc| tells whether a
% child (true) or main (false) document is being compiled.
% The conditional |\ifchilddocmanual| tells whether
% the |\includeonly| mechanism is used (false) or
% the selection of child files must be performed manually (true).
% The definitions initialise to false:
%    \begin{macrocode}
\newif\ifchilddoc
\newif\ifchilddocmanual
%    \end{macrocode}

% \macro{\childdocname}
% \macro{\childdocjob}
% The macro |\childdocname| stores the name of the main document
% to be compiled. The macro |\childdocjob| stores the name of
% the document on which the \LaTeX{} compiler was originally invoked.
% The content of |\jobname| cannot be compared
% to filenames specified in the source due to different catcodes.
% The following code rescans |\jobname|, stores the result
% in |\childdocname| and saves a copy in |\childdocjob|:
%    \begin{macrocode}
\edef\childdocname{\scantokens\expandafter{\jobname\noexpand}}
\let\childdocjob\childdocname
%    \end{macrocode}

% \macro{\childdocdisable}
% The macro |\childdocdisable| prevents the main file
% from being processed more than once.
% At this stage, the main document command |\childdocmain|
% is assumed to be called once again where it should do nothing.
% Any subsequent call to it should prevent
% a secondary processing of the main document
% It overwrites the forwarding commands
% |\childdocof| and |\childdocforward|
% with empty macros to prevent further inclusions of the main document:
%    \begin{macrocode}
\newcommand{\childdocdisable}
{
  \renewcommand{\childdocmain}[1]{\renewcommand{\childdocmain}[1]{\endinput}}
  \renewcommand{\childdocof}[1]{}
  \renewcommand{\childdocby}[2][]{}
  \renewcommand{\childdocforward}[2][]{}
  \renewcommand{\childdocdisable}{}
}
%    \end{macrocode}

% \macro{\childdocmain}
% The macro |\childdocmain| is to be called at the top of the main file
% with nothing or the main filename (without extension) as argument.
% First, it breaks loops.
% If the argument is not empty and does not match |\childdocname|
% (which is set by the first inclusion of |childdoc.def|),
% |\ifchilddoc| is set to true, |\includeonly| is applied to the child file
% and |\jobname| is set to the main file
% (for proper handling of |.aux| files):
%    \begin{macrocode}
\newcommand{\childdocmain}[1]
{
  \childdocdisable\childdocmain{}
  \if?#1?\else
    \begingroup
      \def\childdoctmp{#1}
      \ifx\childdoctmp\childdocname
        \def\childdoctmp{}
      \else
        \def\childdoctmp
        {
          \childdoctrue
          \includeonly{\childdocname}
          \def\childdocjob{#1}
          \def\jobname{#1}
        }
      \fi
      \expandafter
    \endgroup
    \childdoctmp
  \fi
}
%    \end{macrocode}

% \macro{\childdocof}
% The command |\childdocof| redirects
% compilation to the main file |#1|.
%    \begin{macrocode}
\newcommand{\childdocof}[1]
{
  \childdocdisable
  \childdoctrue
  \includeonly{\childdocname}
  \def\jobname{#1}
  \def\childdocjob{#1}
  \input{#1}
}
%    \end{macrocode}

% \macro{\childdocby}
% The command |\childdocby| ....
%    \begin{macrocode}
\newcommand{\childdocby}[2][]
{
  \childdocdisable
  \childdoctrue
  \childdocmanualtrue
  \if?#1?\else
    \def\jobname{#2}
  \fi
  \def\childdocjob{#2}
  \input{#2}
  \endinput
}
%    \end{macrocode}

% \macro{\childdocforward}
% The command |\childdocforward| redirects
% compilation to the main file or
% (if the optional argument is given) a child file.
% Parameters are set as if the main file
% or a child file starting with |\childdocof| was compiled.
% Then compilation is handed over to the main file:
%    \begin{macrocode}
\newcommand{\childdocforward}[2][]
{
  \begingroup
    \if?#1?
      \def\childdoctmp
      {
        \def\childdocname{#2}
        \def\childdocjob{#2}
        \def\jobname{#2}
        \input{#2}
        \endinput
      }
    \else
      \def\childdoctmp
      {
        \childdocdisable
        \def\childdocname{#2}
        \childdoctrue
        \includeonly{#2}
        \def\childdocjob{#1}
        \def\jobname{#1}
        \input{#1}
        \endinput
      }
    \fi
    \expandafter
  \endgroup
  \childdoctmp
}
%    \end{macrocode}

% \macro{\childdocforwardprefix}
% The command |\childdocforwardprefix| redirects
% compilation to the main or a child file by means of a pattern.
% The prefix |#1| in the current filename is replaced by |#2|
% and the suffix of the current filename is kept
% (it is assumed that the filename does not contain the substring `|~~~|'
% which is used as a delimiter).
% Compilation is handed over to the new file by |\childdocforward|:
%    \begin{macrocode}
\newcommand{\childdocforwardprefix}[3][]
{
  \begingroup
    \def\childdocextract #2##1~~~{\def\childdoctmp{\childdocforward[#1]{#3##1}}}
    \expandafter\childdocextract\childdocname~~~
    \expandafter
  \endgroup
  \childdoctmp
}
%    \end{macrocode}

% \macro{\childdoc}
% The deprecated macro |\childdoc| is a legacy version of |\childdocmain|:
%    \begin{macrocode}
\newcommand{\childdoc}{\childdocmain}
%    \end{macrocode}

% \macro{\childdocredirect}
% The deprecated macro |\childdocredirect| is a legacy version
% of |\childdocforward| and |\childdocforwardprefix|:
%    \begin{macrocode}
\newcommand{\childdocredirect}[2][]
{
  \begingroup
    \if?#1?
      \def\childdoctmp{\childdocforward{#2}}
    \else
      \def\childdoctmp{\childdocforwardprefix{#1}{#2}}
    \fi
    \expandafter
  \endgroup
  \childdoctmp
}
%    \end{macrocode}

%\iffalse
%</package>
%\fi
%
\endinput
|\\
|\childdocby{|\textit{main}|}|\\
\end{tabular}
\end{center}
%
Both forms have slightly different effects as described above.
The main file is prepared as usual, see \secref{sec:include}.

%%%%%%%%%%%%%%%%%%%%%%%%%%%%%%%%%%%%%%%%%%%%%%%%%%%%%%%%%%%%%%%%%%%%%%%%%%%%%%%%
\subsection{Legacy Detection}
\label{sec:detection}

The directive |\childdocmain| in the main file can detect
whether the complete document or merely a child is to be compiled
even without using the directive |\childdocof|.
This method is deprecated because it is less robust
and there is no compelling reason to use it;
it is merely provided for backward compatibility
and it may be removed in future versions.

If the detection mechanism is to be used,
it is mandatory to correctly specify
the filename of the main file as the argument of |\childdocmain|:
%
\begin{center}
\begin{tabular}{l}
|% \iffalse
%
% childdoc.dtx Copyright (C) 2017-2018 Niklas Beisert
%
% This work may be distributed and/or modified under the
% conditions of the LaTeX Project Public License, either version 1.3
% of this license or (at your option) any later version.
% The latest version of this license is in
%   http://www.latex-project.org/lppl.txt
% and version 1.3 or later is part of all distributions of LaTeX
% version 2005/12/01 or later.
%
% This work has the LPPL maintenance status `maintained'.
%
% The Current Maintainer of this work is Niklas Beisert.
%
% This work consists of the files childdoc.dtx and childdoc.ins
% and the derived files childdoc.def and cdocsamp.tex with
% cdocsch1.tex, cdocsch2.tex, cdocsdrf.tex, cdocsfn1.tex, cdocsfn2.tex.
%
%<package>\ifdefined\childdocmain\endinput\fi
%<package>\ProvidesFile{childdoc.def}[2018/12/30 v2.0 child document driver]
%<samplemain>\ProvidesFile{cdocsamp.tex}[2018/12/30 v2.0 sample for childdoc]
%<*driver>
%\ProvidesFile{childdoc.drv}[2018/12/30 v2.0 childdoc reference manual file]
\PassOptionsToClass{10pt,a4paper}{article}
\documentclass{ltxdoc}

\usepackage[margin=35mm]{geometry}
\usepackage{hyperref}
\usepackage{hyperxmp}
\usepackage[usenames]{color}

\hypersetup{colorlinks=true}
\hypersetup{pdfstartview=FitH}
\hypersetup{pdfpagemode=UseNone}
\hypersetup{pdfsource={}}
\hypersetup{pdflang={en-UK}}
\hypersetup{pdfcopyright={Copyright 2017-2018 Niklas Beisert.
  This work may be distributed and/or modified under the
  conditions of the LaTeX Project Public License, either version 1.3
  of this license or (at your option) any later version.}}
\hypersetup{pdflicenseurl={http://www.latex-project.org/lppl.txt}}
\hypersetup{pdfcontactaddress={ETH Zurich, ITP, HIT K,
  Wolfgang-Pauli-Strasse 27}}
\hypersetup{pdfcontactpostcode={8093}}
\hypersetup{pdfcontactcity={Zurich}}
\hypersetup{pdfcontactcountry={Switzerland}}
\hypersetup{pdfcontactemail={nbeisert@itp.phys.ethz.ch}}
\hypersetup{pdfcontacturl={http://people.phys.ethz.ch/\xmptilde nbeisert/}}

\newcommand{\secref}[1]{\hyperref[#1]{section \ref*{#1}}}

\parskip1ex
\parindent0pt
\let\olditemize\itemize
\def\itemize{\olditemize\parskip0pt}

\begin{document}

\title{The \textsf{childdoc} Package}
\hypersetup{pdftitle={The childdoc Package}}
\author{Niklas Beisert\\[2ex]
  Institut f\"ur Theoretische Physik\\
  Eidgen\"ossische Technische Hochschule Z\"urich\\
  Wolfgang-Pauli-Strasse 27, 8093 Z\"urich, Switzerland\\[1ex]
  \href{mailto:nbeisert@itp.phys.ethz.ch}
  {\texttt{nbeisert@itp.phys.ethz.ch}}}
\hypersetup{pdfauthor={Niklas Beisert}}
\hypersetup{pdfsubject={Manual for the LaTeX2e Package childdoc}}
\date{30 December 2018, \textsf{v2.0}}
\maketitle

\begin{abstract}\noindent
\textsf{childdoc} is a \LaTeXe{} package
that enables the direct compilation
of document sections included by |\include|
to individual files.
\end{abstract}

\begingroup
\parskip0ex
\tableofcontents
\endgroup

%%%%%%%%%%%%%%%%%%%%%%%%%%%%%%%%%%%%%%%%%%%%%%%%%%%%%%%%%%%%%%%%%%%%%%%%%%%%%%%%
%%%%%%%%%%%%%%%%%%%%%%%%%%%%%%%%%%%%%%%%%%%%%%%%%%%%%%%%%%%%%%%%%%%%%%%%%%%%%%%%
\section{Introduction}

\LaTeX{} provides a mechanism to structure a large document (such as a book)
into a main file and several child files (containing the chapters)
using the |\include| command.
This mechanism is beneficial for documents
which span hundreds of pages in order to
make the source file(s) more manageable.
Moreover, compilation can be restricted to
selected child files by means of the |\includeonly| command.
The latter feature can be used to reduce the compilation time while editing
(this was significantly more useful in the earlier days of \LaTeX{})
or to generate a smaller document which is easier to navigate.
Another application of |\includeonly| is to generate
documents consisting of selected parts of the complete document.

However, there are a few drawbacks of the plain |\include| mechanism:
\begin{itemize}
\item
The child files cannot be compiled on their own,
they can only be compiled via the main file.
A naive editing environment
(such as a text editor with an option
to have the current file processed by \LaTeX)
may require one to switch to the main file before compiling;
attempting to compile the child file produces errors.
\item
The main file must be modified (each time)
to adjust the |\includeonly| command
to the present needs. This easily leaves the main file in a messy state.
\item
The generated document will always carry the filename
of the main document. This is inconvenient if
several child files are to be compiled and
to be kept for distribution.
\end{itemize}

The present package provides a simple interface
to make child files individually compilable by \LaTeX{}.
Compiling a child file then has the same effect as compiling
the main file with an |\includeonly| command
to select the appropriate child.
Moreover the generated document will carry the name of the child
rather than the main file.
This resolves all three above issues.

This feature is meant to make the editing of books,
thesis documents and lecture notes somewhat more convenient.
However, the package can also be used efficiently for
composing a series of documents (such as exercise sheets)
which are typically distributed individually.
It then assists the author in generating the individual documents
(potentially in different versions)
as well as a document containing the collected series.
Another application is in developing style files
or other kinds of included material
where compilation of the style file could redirect
to a sample or test file.

%%%%%%%%%%%%%%%%%%%%%%%%%%%%%%%%%%%%%%%%%%%%%%%%%%%%%%%%%%%%%%%%%%%%%%%%%%%%%%%%
%%%%%%%%%%%%%%%%%%%%%%%%%%%%%%%%%%%%%%%%%%%%%%%%%%%%%%%%%%%%%%%%%%%%%%%%%%%%%%%%
\section{Usage}

First of all, the package \textsf{childdoc} is \emph{not} a standard
\LaTeXe{} |.sty| style file! Therefore it needs to be invoked in
a non-standard way.

%%%%%%%%%%%%%%%%%%%%%%%%%%%%%%%%%%%%%%%%%%%%%%%%%%%%%%%%%%%%%%%%%%%%%%%%%%%%%%%%
\subsection{Included Files}
\label{sec:include}

%%%%%%%%%%%%%%%%%%%%%%%%%%%%%%%%%%%%%%%%
\DescribeMacro{\childdocmain}
To use the package, add the commands
\begin{center}
\begin{tabular}{l}
|\input{childdoc.def}|\\
|\childdocmain{}|\\
\end{tabular}
\end{center}
at the very top of the main \LaTeX{} file,
in particular \emph{before} the |\documentclass| statement!
The argument of |\childdocmain| should be left empty
(but it must be present).

%%%%%%%%%%%%%%%%%%%%%%%%%%%%%%%%%%%%%%%%
\DescribeMacro{\childdocof}
Furthermore, add the commands
\begin{center}
\begin{tabular}{l}
|\input{childdoc.def}|\\
|\childdocof{|\textit{main}|}|\\
\end{tabular}
\end{center}
at the top of every child file \textit{child}
which is included by |\include{|\textit{child}|}|
from within the main file
(or at least for those files to be compiled individually).
The argument \textit{main} must be the filename of the main file.

There are a couple of
considerations in setting up the main and child documents:

%%%%%%%%%%%%%%%%%%%%%%%%%%%%%%%%%%%%%%%%
\paragraph{Restrictions.}

Please note the following restrictions:
\begin{itemize}
\item
|\childdocmain| must be called with one argument \textit{main}
to ensure compatibility with earlier version of the package.
It must either be empty (|\childdocmain{}|)
or precisely match the filename of the main file in which it is specified.
See \secref{sec:detection} for further information.
\item
The filename \textit{main} must be specified without the |.tex| extension.
\item
The filename \textit{main} is case sensitive
(even in case-insensitive file systems)
due to internal string comparison.
\item
The argument \textit{main} should be fully expanded, it cannot be a macro.
\item
Subdirectories and special characters should be avoided in filenames.
\item
The command |\childdocmain{|\textit{main}|}| must be followed by a whitespace.
It should not be followed immediately by another command
or by a comment mark `|%|'.
This is because the \TeX{} parser reads the token immediately following
the argument of |\childdocmain| and puts it
at the beginning of every child section;
however, a white\-space is ignored.
\end{itemize}

%%%%%%%%%%%%%%%%%%%%%%%%%%%%%%%%%%%%%%%%
\paragraph{Content of Main File.}

It is advisable to place all content in the child files included by |\include|.
Any output contained in the main file will appear in all child documents
unless suppressed manually;
it cannot be suppressed automatically by the |\includeonly| directive
and thus should normally be avoided.
A method to include some content in the main file
by means of conditional processing is described in \secref{sec:conditional}.

%%%%%%%%%%%%%%%%%%%%%%%%%%%%%%%%%%%%%%%%
\paragraph{Page Numbering.}

When only a part of the document is compiled,
the appropriate numbering of pages
(as well as other status parameters)
is determined from the |.aux| files.
The latter contain information from previous passes.
However this information needs to propagate through
all intermediate child documents.
Therefore the page numbering in child documents may well
be inconsistent until the complete document is compiled at least once.

A useful (if unconventional) way to always ensure a consistent
page numbering is to restart the numbering in each child document
and denote the pages by `\textit{child}|.|\textit{page}'
where \textit{child} represents the chapter/section number of the child file.
This can be achieved by the command
|\numberwithin{page}{|\textit{child}|}|
of the \textsf{amsmath} package
where \textit{child} can be |chapter| or |section|
depending on the chosen structuring.
Alternatively, one can modify the macro |\thepage| appropriately
and reset the counter |page| at the start of each child file.

%%%%%%%%%%%%%%%%%%%%%%%%%%%%%%%%%%%%%%%%%%%%%%%%%%%%%%%%%%%%%%%%%%%%%%%%%%%%%%%%
\subsection{Conditional Processing}
\label{sec:conditional}

The package provides a mechanism to compile different versions
of a document. To customise the versions further some conditional processing
can come in handy to distinguish which version is being compiled.
The package provides two macros to describe the compilation context:

%%%%%%%%%%%%%%%%%%%%%%%%%%%%%%%%%%%%%%%%
\DescribeMacro{\ifchilddoc}
The conditional |\ifchilddoc| distinguishes between the compilation of
child documents and the main document:
%
\begin{center}
|\ifchilddoc |\textit{child-code}| |[|\||else |\textit{main-code}]| \||fi|
\end{center}

%%%%%%%%%%%%%%%%%%%%%%%%%%%%%%%%%%%%%%%%
\DescribeMacro{\childdocname}
\DescribeMacro{\childdocjob}
The macro |\childdocname| contains the filename (without extension)
of the main or child file being processed.
Note that |\childdocjob| will always contain the name of the main file.

%%%%%%%%%%%%%%%%%%%%%%%%%%%%%%%%%%%%%%%%
\paragraph{Title Page.}

Conditional processing can be used to include a title or banner page
in the main document when proper precautions are taken.
Importantly, the code in the main file should ensure that the page counter
(as well as other status parameters which are stored in the |.aux| files)
takes the same value after the conditional processing.
Otherwise the page numbers may take divergent values
depending on which part is compiled.

For example, a title page could be declared by:
%
\begin{center}
\begin{tabular}{l}
|\ifchilddoc\||else|\\
|\addtocounter{page}{-1}|\\
\textit{code for title page}\\
|\newpage|\\
|\||fi|
\end{tabular}
\end{center}
%
A banner page for the child documents can be generated by:
%
\begin{center}
\begin{tabular}{l}
|\ifchilddoc|\\
|\addtocounter{page}{-1}|\\
\textit{code for banner page}\\
|\newpage|\\
|\||fi|
\end{tabular}
\end{center}
%
Here one could write a message such as:
\begin{center}
|This is the part \childdocname{} of \childdocjob{}.|
\end{center}

%%%%%%%%%%%%%%%%%%%%%%%%%%%%%%%%%%%%%%%%%%%%%%%%%%%%%%%%%%%%%%%%%%%%%%%%%%%%%%%%
\subsection{Flags}
\label{sec:flags}

The package makes it easy to generate different versions
of the main or child documents.
To this end compilation flags can be defined
and assigned different default values.
They will be particularly useful in conjunction
with the forwarding mechanism described in \secref{sec:forward}.

For example, it may be useful to have a flag |\version|
which can be set to |draft| or |final|.
The document source will contain some conditional code
depending on the value of |\version|.
Suppose further, the flag should default to |final| for the main file
and to |draft| for child files
which is a natural assignment for editing the document.
This is achieved by placing the following code
in the preamble of the main document
(below the |\childdocmain| directive):
%
\begin{center}
\begin{tabular}{l}
|\ifchilddoc|\\
|\providecommand{\version}{draft}|\\
|\||else|\\
|\providecommand{\version}{final}|\\
|\||fi|
\end{tabular}
\end{center}
%
The definition by |\providecommand| makes sure
that previous definitions are not overwritten.
Further statements |\providecommand{\version}{...}|
can thus be added before the above code to override it.

For the main file, one might add a line
(between |\childdocmain| and the above block)
%
\begin{center}
|%\ifchilddoc\||else\providecommand{\version}{draft}\||fi|
\end{center}
%
which can be uncommented to produce a draft version.
Likewise one can add a line to the very top of a child file
(above the |\childdocof{|\textit{main}|}| directive)
%
\begin{center}
|%\providecommand{\version}{final}|
\end{center}
%
which can be uncommented to produce the final version of this child document.

%%%%%%%%%%%%%%%%%%%%%%%%%%%%%%%%%%%%%%%%%%%%%%%%%%%%%%%%%%%%%%%%%%%%%%%%%%%%%%%%
\subsection{Forwarding}
\label{sec:forward}

Different versions of the main or child documents
using compilation flags as described in \secref{sec:flags}
can be (permanently) stored in different files
for convenient compilation, viewing and distribution.
To this end, the package defines a command
to pass on compilation to a different file:

%%%%%%%%%%%%%%%%%%%%%%%%%%%%%%%%%%%%%%%%
\DescribeMacro{\childdocforward}
The command |\childdocforward| redirects processing to
another source file:
%
\begin{center}
\begin{tabular}{l}
|\input{childdoc.def}|\\
|\childdocforward[|\textit{main}|]{|\textit{dest}|}|\\
\end{tabular}
\end{center}
%
The argument \textit{dest} is the destination file
(without extension).
It should be the main file or one of the child files.
Note that further \textsf{childdoc} directives
such as |\childdocof| and |\childdocforward|
in the indicated file will be processed in this form.
The optional argument \textit{main}
passes on directly to the main file \textit{main}
while pretending to compile the child \textit{dest}.
This form behaves as if \textit{dest}
issues |\childdocof{|\textit{main}|}| right away,
and no further \textsf{childdoc} directives will be processed.

%%%%%%%%%%%%%%%%%%%%%%%%%%%%%%%%%%%%%%%%
\DescribeMacro{\...prefix}
In the alternative form |\childdocforwardprefix|,
%
\begin{center}
\begin{tabular}{l}
|\input{childdoc.def}|\\
|\childdocforwardprefix[|\textit{main}|]{|\textit{prefix}|}{|\textit{dest}|}|
\end{tabular}
\end{center}
%
the destination file is determined by a pattern
depending on the current file:
To make this work, the current file must be called
`{\textit{prefix}\hspace{0.2em}\textit{suffix}}'
with \textit{prefix} matching precisely the argument.
Processing is then passed on to the file
`{\textit{dest}\hspace{0.2em}\textit{suffix}}'.
Surely, the same effect is achieved by
directly specifying the
argument `{\textit{dest}\hspace{0.2em}\textit{suffix}}'
in the first form.
However, that requires to set up a different file
for each child. With the alternative form of the command
all these files can have exactly the same content
which simplifies setting them up and maintaining them.

For example, the following file |draft.tex|
with a compilation flag |\version| as described in \secref{sec:flags}
compiles the main document as a draft:
%
\begin{center}
\begin{tabular}{l}
|\def\version{draft}|\\
|\input{childdoc.def}|\\
|\childdocforward{|\textit{main}|}|
\end{tabular}
\end{center}
%
Likewise, the following files |final|\textit{nn}|.tex|
compile the final version of the child document
|child|\textit{nn}|.tex|:
%
\begin{center}
\begin{tabular}{l}
|\def\version{final}|\\
|\input{childdoc.def}|\\
|\childdocforwardprefix{final}{child}|
\end{tabular}
\end{center}
%

Note that when several versions of a main file and/or of each child file
are to be generated, it may be convenient to set up a |Makefile| or
shell script to automatise the process.

%%%%%%%%%%%%%%%%%%%%%%%%%%%%%%%%%%%%%%%%%%%%%%%%%%%%%%%%%%%%%%%%%%%%%%%%%%%%%%%%
\subsection{Command Line Processing}
\label{sec:commandline}

The effect of redirection files can also be achieved by invoking
the \LaTeX{} compiler with a more elaborate command line.
Most conveniently this should be done as part
of a shell script or a |Makefile|.

When using \textsf{childdoc} in the main file, the following
command lines effectively perform a redirection
(note that depending on the shell being used,
backslashes may have to be doubled: `|\|' $\to$ `|\\|'):
%
\begin{center}
|... -jobname "|\textit{target}|" |\\|"|[\textit{flags}]%
|\input{childdoc.def}\childdocforward[|\textit{main}|]{|\textit{dest}|}"|
\end{center}
%
Here \textit{target} is the name of the output file,
\textit{main} is the name of the main file
and \textit{dest} is the name of the main or child file to be processed
(all filenames without extensions).
The optional argument \textit{main} can be omitted
if \textit{main} matches \textit{dest}.
Optionally, compilation \textit{flags} can be defined via |\def| commands.
This command line makes the \TeX{} engine believe
it is compiling the file \textit{target}
whose content is specified as the latter parameter.
The provided code then forwards the processing to
\textit{main} or \textit{dest} as described in \secref{sec:forward}.

%%%%%%%%%%%%%%%%%%%%%%%%%%%%%%%%%%%%%%%%%%%%%%%%%%%%%%%%%%%%%%%%%%%%%%%%%%%%%%%%
\subsection{Include by Input}
\label{sec:input}

Including child documents by |\include| has some restrictions by design.
Most notably, the content of a child document always occupies
its own set of pages; pages cannot be shared between child documents.
Usually, this behaviour makes perfect sense
because each child document contain an essential part of the document.
However, in some situations it may be desirable to compose
a document from a collection of parts
without having mandatory page breaks between then.
For this case, the package
provides a mechanism to include parts
by |\input| which can also be processed individually.
However, by construction this mechanism
requires manual handling of the content to be output.

%%%%%%%%%%%%%%%%%%%%%%%%%%%%%%%%%%%%%%%%
\DescribeMacro{\ifchilddocmanual}
The main file should be prepared as usual, see \secref{sec:include}.
However, the document body must make a distinction
between processing of an individual part and of the main document, e.g.:
%
\begin{center}
\begin{tabular}{l}
|\ifchilddocmanual|\\
|\input{\childdocname}|\\
|\||else|\\
\textit{document body with }|\input{|\textit{part}|}|\\
|\||fi|
\end{tabular}
\end{center}
%
The conditional |\ifchilddocmanual| is true whenever
a part to be included by |\input| is being compiled,
and the name of the part is stored in |\childdocname|.

%%%%%%%%%%%%%%%%%%%%%%%%%%%%%%%%%%%%%%%%
\DescribeMacro{\childdocby}
Each part to be included by |\input| should start with:
%
\begin{center}
\begin{tabular}{l}
|\input{childdoc.def}|\\
|\childdocby{|\textit{main}|}|\\
\end{tabular}
\end{center}
%
The directive |\childdocby| is similar to |\childdocof|
described in \secref{sec:include},
but the subsequent selection of content must be done manually.
To that end, both |\ifchilddoc| and |\ifchilddocmanual|
will be true upon processing of a part,
and the name of the part is stored in |\childdocname|.
Note that |\jobname| will be set to the filename of the current part
so that each part receives an individual |.aux| file
that does not interfere with the |.aux| file(s) of the main document.
This behaviour can be altered by the alternative form
|\childdocby[*]{|\textit{main}|}| (with a non-empty optional argument)
which uses the |.aux| file of the main document
by setting |\jobname| to \textit{main}.

%%%%%%%%%%%%%%%%%%%%%%%%%%%%%%%%%%%%%%%%%%%%%%%%%%%%%%%%%%%%%%%%%%%%%%%%%%%%%%%%
\subsection{Driver Development}
\label{sec:driver}

The \textsf{childdoc} mechanism can also be use for the development
of definition files such as \LaTeX{} styles or classes.
This case differs from the above setup with multiple parts
included by |\include| in that no |\includeonly| should be invoked.
This can be achieved by starting the include file
(before |\ProvidesPackage|) with:
%
\begin{center}
\begin{tabular}{l}
|\input{childdoc.def}|\\
|\childdocforward{|\textit{main}|}|\\
\end{tabular}
\end{center}
%
or alternatively with:
%
\begin{center}
\begin{tabular}{l}
|\input{childdoc.def}|\\
|\childdocby{|\textit{main}|}|\\
\end{tabular}
\end{center}
%
Both forms have slightly different effects as described above.
The main file is prepared as usual, see \secref{sec:include}.

%%%%%%%%%%%%%%%%%%%%%%%%%%%%%%%%%%%%%%%%%%%%%%%%%%%%%%%%%%%%%%%%%%%%%%%%%%%%%%%%
\subsection{Legacy Detection}
\label{sec:detection}

The directive |\childdocmain| in the main file can detect
whether the complete document or merely a child is to be compiled
even without using the directive |\childdocof|.
This method is deprecated because it is less robust
and there is no compelling reason to use it;
it is merely provided for backward compatibility
and it may be removed in future versions.

If the detection mechanism is to be used,
it is mandatory to correctly specify
the filename of the main file as the argument of |\childdocmain|:
%
\begin{center}
\begin{tabular}{l}
|\input{childdoc.def}|\\
|\childdocmain{|\textit{main}|}|\\
\end{tabular}
\end{center}
%
If |\jobname| does not match the argument \textit{main} of |\childdocmain|,
it is assumed that |\jobname| points to the child file to be compiled.
When using |\childdocmain| with the main file specified as argument,
it suffices to start a child file
with just |\input{|\textit{main}|}|
without loading of the package and using |\childdocof|.
If instead all processing is done
with the appropriate \textsf{childdoc} directives,
the argument of \textit{main} of |\childdocmain| can be empty.

An alternative version of the command line processing described
in \secref{sec:commandline} using the detection mechanism reads:
%
\begin{center}
|... -jobname "|\textit{target}|" "|[\textit{flags}]%
[|\def\jobname{|\textit{dest}|}|]|\input{|\textit{main}|}"|
\end{center}

%%%%%%%%%%%%%%%%%%%%%%%%%%%%%%%%%%%%%%%%%%%%%%%%%%%%%%%%%%%%%%%%%%%%%%%%%%%%%%%%
\subsection{Manual Code}
\label{sec:manual}

In case one cannot be certain whether the definitions file |childdoc.def|
is installed on the target \TeX{} distribution
and one prefers not to ship it,
it is conceivable to paste a few relevant commands into the sources.

To that end, drop all statements |\input{childdoc.def}|
and perform the replacements as outlined below.
Instead of |\childdocmain{|\textit{main}|}| add the following code
to the top of the main file:
%
\begin{center}
\begin{tabular}{l}
|\||ifdefined\childdocname\endinput\||fi\newif\ifchilddoc|\\
|\edef\childdocname{\scantokens\expandafter{\jobname\noexpand}}|\\
|\def\childdocmain{|\textit{main}|}\||ifx\childdocmain\childdocname\||else|\\
|\childdoctrue\includeonly{\childdocname}\let\jobname\childdocmain\||fi|\\
\end{tabular}
\end{center}
%
Instead of |\childdocof{|\textit{main}|}| just include the main file
at the top of each child file:
%
\begin{center}
|\input{|\textit{main}|}|
\end{center}
%
A simple redirection |\childdocforward{|\textit{dest}|}| is achieved by:
%
\begin{center}
|\def\jobname{|\textit{dest}|}\input{\jobname}|
\end{center}
%
The redirection with prefix
|\childdocforwardprefix[|\textit{prefix}|]{|\textit{dest}|}|
is accomplished by:
%
\begin{center}
\begin{tabular}{l}
|{\edef\jobname{\scantokens\expandafter{\jobname\noexpand}}|\\
|\def\redirectjob |\textit{prefix}|#1~~~{\gdef\jobname{|\textit{dest}|#1}}|\\
|\expandafter\redirectjob\jobname~~~}\input{\jobname}|
\end{tabular}
\end{center}

In an alternative approach,
child documents can be compiled by a specific command line
without additional code or specific definitions:
%
\begin{center}
|... -jobname "|\textit{target}|" "|[\textit{flags}]%
|\includeonly{|\textit{dest}|}\input{|\textit{main}|}"|
\end{center}
%

%%%%%%%%%%%%%%%%%%%%%%%%%%%%%%%%%%%%%%%%%%%%%%%%%%%%%%%%%%%%%%%%%%%%%%%%%%%%%%%%
%%%%%%%%%%%%%%%%%%%%%%%%%%%%%%%%%%%%%%%%%%%%%%%%%%%%%%%%%%%%%%%%%%%%%%%%%%%%%%%%
\section{Information}

%%%%%%%%%%%%%%%%%%%%%%%%%%%%%%%%%%%%%%%%%%%%%%%%%%%%%%%%%%%%%%%%%%%%%%%%%%%%%%%%
\subsection{Copyright}

Copyright \copyright{} 2017--2018 Niklas Beisert

This work may be distributed and/or modified under the
conditions of the \LaTeX{} Project Public License, either version 1.3
of this license or (at your option) any later version.
The latest version of this license is in
  \url{http://www.latex-project.org/lppl.txt}
and version 1.3 or later is part of all distributions of \LaTeX{}
version 2005/12/01 or later.

This work has the LPPL maintenance status `maintained'.

The Current Maintainer of this work is Niklas Beisert.

This work consists of the files |README.txt|, |childdoc.ins| and |childdoc.dtx|
as well as the derived files |childdoc.def|, |cdocsamp.tex|
with |cdocsch1.tex|, |cdocsch2.tex|, |cdocspt3.tex|, |cdocspt4.tex|,
|cdocsdrf.tex|, |cdocsfn1.tex|, |cdocsfn2.tex|
as well as |childdoc.pdf|.

%%%%%%%%%%%%%%%%%%%%%%%%%%%%%%%%%%%%%%%%%%%%%%%%%%%%%%%%%%%%%%%%%%%%%%%%%%%%%%%%
\subsection{Files and Installation}

The package consists of the files:
%
\begin{center}
\begin{tabular}{ll}
    |README.txt|   & readme file \\
    |childdoc.ins| & installation file \\
    |childdoc.dtx| & source file \\
    |childdoc.def| & definition file \\
    |cdocsamp.tex| & sample main file \\
    |cdocsch1.tex| & sample include file \\
    |cdocsch2.tex| & sample include file \\
    |cdocspt3.tex| & sample part file \\
    |cdocspt4.tex| & sample part file \\
    |cdocsdrf.tex| & sample redirection file \\
    |cdocsfn1.tex| & sample redirection file \\
    |cdocsfn2.tex| & sample redirection file \\
    |childdoc.pdf| & manual
\end{tabular}
\end{center}
%
The distribution consists of the files
|README.txt|, |childdoc.ins| and |childdoc.dtx|.
%
\begin{itemize}
\item
Run (pdf)\LaTeX{} on |childdoc.dtx|
to compile the manual |childdoc.pdf| (this file).
\item
Run \LaTeX{} on |childdoc.ins| to create the definitions file |childdoc.def|
and the sample |cdocsamp.tex| with include files
|cdocsch1.tex|, |cdocsch2.tex|, |cdocspt3.tex|, |cdocspt4.tex|,
|cdocsdrf.tex|, |cdocsfn1.tex|, |cdocsfn2.tex|.
Then copy the file |childdoc.def| to an appropriate directory of your \LaTeX{}
distribution, e.g.\ \textit{texmf-root}|/tex/latex/childdoc|.
\end{itemize}

%%%%%%%%%%%%%%%%%%%%%%%%%%%%%%%%%%%%%%%%%%%%%%%%%%%%%%%%%%%%%%%%%%%%%%%%%%%%%%%%
\subsection{Related CTAN Packages}

There are several other packages which offer a similar functionality:
%
\begin{itemize}
\item
The packages
\href{http://ctan.org/pkg/docmute}{\textsf{docmute}},
\href{http://ctan.org/pkg/includex}{\textsf{includex}} and
\href{http://ctan.org/pkg/standalone}{\textsf{standalone}}
provide commands to include only the document body of
a child file thus allowing both files to be compiled individually.
\item
The packages \href{http://ctan.org/pkg/subdocs}{\textsf{subdocs}}
and \href{http://ctan.org/pkg/subfiles}{\textsf{subfiles}}
provide structures in which the main and child documents can be
encapsulated and allowing them to be compiled individually.
The inclusion mechanism is different from the conventional |\include|.
\item
The package \href{http://ctan.org/pkg/combine}{\textsf{combine}}
is an elaborate solution to combine several documents into one.
\end{itemize}
%
See also the CTAN topic \href{http://ctan.org/topic/subdocs}{\textsf{subdocs}}
for further related packages.
The present package differs from the above solutions in that
a document structure constructed with the conventional |\include| mechanism
just needs two extra commands at the top of every file
such that all constituent files can be compiled individually.

%%%%%%%%%%%%%%%%%%%%%%%%%%%%%%%%%%%%%%%%%%%%%%%%%%%%%%%%%%%%%%%%%%%%%%%%%%%%%%%%
%\subsection{Feature Suggestions}
%
%The following is a list of features which may be useful for future
%versions of this package:
%%
%\begin{itemize}
%\item
%\ldots
%\end{itemize}

%%%%%%%%%%%%%%%%%%%%%%%%%%%%%%%%%%%%%%%%%%%%%%%%%%%%%%%%%%%%%%%%%%%%%%%%%%%%%%%%
\subsection{Revision History}

%%%%%%%%%%%%%%%%%%%%%%%%%%%%%%%%%%%%%%%%
\paragraph{v2.0:} 2018/12/30

\begin{itemize}
\item
immediate forward processing
\item
added |\childdocby| mechanism
\item
manual restructured
\end{itemize}

%%%%%%%%%%%%%%%%%%%%%%%%%%%%%%%%%%%%%%%%
\paragraph{v1.6:} 2018/01/17

\begin{itemize}
\item
application for development of include files
\item
corrections to manual
\end{itemize}

%%%%%%%%%%%%%%%%%%%%%%%%%%%%%%%%%%%%%%%%
\paragraph{v1.5:} 2017/05/21

\begin{itemize}
\item
more complete structuring introduced
\item
|\childdocof| introduced
\item
|\childdoc| renamed to |\childdocmain|
\item
|\childredirect| renamed to |\childdocforward| and |\childdocforwardprefix|
and functionality expanded
\end{itemize}

%%%%%%%%%%%%%%%%%%%%%%%%%%%%%%%%%%%%%%%%
\paragraph{v1.0:} 2017/04/27

\begin{itemize}
\item
manual and install package
\item
first version published on CTAN
\end{itemize}

%%%%%%%%%%%%%%%%%%%%%%%%%%%%%%%%%%%%%%%%
\paragraph{v0.6:} 2017/04/26

\begin{itemize}
\item
redirection mechanism added
\end{itemize}

%%%%%%%%%%%%%%%%%%%%%%%%%%%%%%%%%%%%%%%%
\paragraph{v0.5:} 2017/04/26

\begin{itemize}
\item
functionality in definition file
\end{itemize}


%%%%%%%%%%%%%%%%%%%%%%%%%%%%%%%%%%%%%%%%%%%%%%%%%%%%%%%%%%%%%%%%%%%%%%%%%%%%%%%%
%%%%%%%%%%%%%%%%%%%%%%%%%%%%%%%%%%%%%%%%%%%%%%%%%%%%%%%%%%%%%%%%%%%%%%%%%%%%%%%%
%%%%%%%%%%%%%%%%%%%%%%%%%%%%%%%%%%%%%%%%%%%%%%%%%%%%%%%%%%%%%%%%%%%%%%%%%%%%%%%%
\appendix

\settowidth\MacroIndent{\rmfamily\scriptsize 000\ }

 \DocInput{childdoc.dtx}

\end{document}
%</driver>
% \fi
%
% %%%%%%%%%%%%%%%%%%%%%%%%%%%%%%%%%%%%%%%%%%%%%%%%%%%%%%%%%%%%%%%%%%%%%%%%%%%%%%
% %%%%%%%%%%%%%%%%%%%%%%%%%%%%%%%%%%%%%%%%%%%%%%%%%%%%%%%%%%%%%%%%%%%%%%%%%%%%%%
% \section{Sample}
%\iffalse
%<*samplemain>
%\fi
%
% The following presents a sample document
% with two chapters, two parts, a title page,
% a compile flag as well as three forwarding files to set the flag.
% It consists of eight |.tex| files:
% \begin{center}
% \begin{tabular}{ll}
% |cdocsamp.tex|&main file\\
% |cdocsch1.tex|&include file for chapter 1\\
% |cdocsch2.tex|&include file for chapter 2\\
% |cdocspt3.tex|&include file for part 3\\
% |cdocspt4.tex|&include file for part 4\\
% |cdocsdrf.tex|&forwarding file for main file in draft mode\\
% |cdocsfi1.tex|&forwarding file for final version of chapter 1\\
% |cdocsfi2.tex|&forwarding file for final version of chapter 2\\
% \end{tabular}
% \end{center}
% Each of the eight files can be compiled directly by the \LaTeX{} compiler.
%
% %%%%%%%%%%%%%%%%%%%%%%%%%%%%%%%%%%%%%%
% \paragraph{Main File.}
%
% The main file is called |cdocsamp.tex|.
%
% Load the \textsf{childdoc} definitions and
% declare the filename for the main document:
%    \begin{macrocode}
\input{childdoc.def}
\childdocmain{}
%    \end{macrocode}

% Optional override for |\version| flag:
%    \begin{macrocode}
%%\ifchilddoc\else\providecommand{\version}{draft}\fi
%    \end{macrocode}

% Define the default values for the |\version| flag
% (|final| for the main file and |draft| for childs):
%    \begin{macrocode}
\ifchilddoc
\providecommand{\version}{draft}
\else
\providecommand{\version}{final}
\fi
%    \end{macrocode}

% Load the standard document class:
%    \begin{macrocode}
\documentclass[12pt]{article}
%    \end{macrocode}

% Start the document body:
%    \begin{macrocode}
\begin{document}
%    \end{macrocode}

% Declare a title page.
% Print title, part of document being processed and version flag:
%    \begin{macrocode}
\addtocounter{page}{-1}
\begin{center}
{\LARGE\bfseries{}childdoc example\par}
\vspace{1cm}
\ifchilddoc
\ifchilddocmanual part\else chapter\fi:
`\childdocname' of `\childdocjob'\par
\else
main document: `\childdocjob'\par
\fi
version: \version\par
\end{center}
\newpage
%    \end{macrocode}

% Manually include selected file,
% otherwise process as usual:
%    \begin{macrocode}
\ifchilddocmanual
\section*{part `\childdocname'}
\input{\childdocname}
\else
%    \end{macrocode}

% Include the two chapters:
%    \begin{macrocode}
\include{cdocsch1}
\include{cdocsch2}
%    \end{macrocode}

% Include the two parts unless only chapters should be displayed:
%    \begin{macrocode}
\ifchilddoc\else
\section{part three}
\input{cdocspt3}
\section{part four}
\input{cdocspt4}
\fi
%    \end{macrocode}

% Process as usual until here:
%    \begin{macrocode}
\fi
%    \end{macrocode}

% End of document body:
%    \begin{macrocode}
\end{document}
%    \end{macrocode}
%\iffalse
%</samplemain>
%\fi
%
% %%%%%%%%%%%%%%%%%%%%%%%%%%%%%%%%%%%%%%
% \paragraph{Chapter Include Files.}
%
% The include files are called |cdocsch1.tex| and |cdocsch2.tex|.
%
%\iffalse
%<*samplechap1|samplechap2>
%\fi

% Optional override for |\version| flag:
%    \begin{macrocode}
%%\providecommand{\version}{final}
%    \end{macrocode}

% Include the main document:
%    \begin{macrocode}
\input{childdoc.def}
\childdocof{cdocsamp}
%    \end{macrocode}

%\iffalse
%</samplechap1|samplechap2>
%\fi
%
%\iffalse
%<*samplechap1>
%\fi
% Some text for chapter 1:
%    \begin{macrocode}
\section{one}
some text in chapter one
%    \end{macrocode}

%\iffalse
%</samplechap1>
%\fi
% Some text for chapter 2:
%\iffalse
%<*samplechap2>
%\fi
%    \begin{macrocode}
\section{two}
more text in chapter two
%    \end{macrocode}

%\iffalse
%</samplechap2>
%\fi
%
% %%%%%%%%%%%%%%%%%%%%%%%%%%%%%%%%%%%%%%
% \paragraph{Part Include Files.}
%
% The include files are called |cdocspt3.tex| and |cdocspt4.tex|.
%
%\iffalse
%<*samplepart3|samplepart4>
%\fi

% Optional override for |\version| flag:
%    \begin{macrocode}
%%\providecommand{\version}{final}
%    \end{macrocode}

% Include the main document:
%    \begin{macrocode}
\input{childdoc.def}
\childdocby{cdocsamp}
%    \end{macrocode}

%\iffalse
%</samplepart3|samplepart4>
%\fi
%
%\iffalse
%<*samplepart3>
%\fi
% Some text for part 3:
%    \begin{macrocode}
some text in part three
%    \end{macrocode}

%\iffalse
%</samplepart3>
%\fi
% Some text for part 4:
%\iffalse
%<*samplepart4>
%\fi
%    \begin{macrocode}
more text in part four
%    \end{macrocode}

%\iffalse
%</samplepart4>
%\fi
%
% %%%%%%%%%%%%%%%%%%%%%%%%%%%%%%%%%%%%%%
% \paragraph{Forwarding for a Complete Draft.}
%
% The following forwarding file |cdocsdrf.tex|
% compiles the main document in draft mode:
%\iffalse
%<*sampledraft>
%\fi
%    \begin{macrocode}
\def\version{draft}
\input{childdoc.def}
\childdocforward{cdocsamp}
%    \end{macrocode}

%\iffalse
%</sampledraft>
%\fi
%
% %%%%%%%%%%%%%%%%%%%%%%%%%%%%%%%%%%%%%%
% \paragraph{Forwarding for Final Version of the Chapters.}
%
% The following forwarding files |cdocsfn1.tex| and |cdocsfn2.tex|
% (with identical content)
% compile the final versions of the child documents
% |cdocsch1.tex| and |cdocsch2.tex|, respectively:
%\iffalse
%<*samplefinal>
%\fi
%    \begin{macrocode}
\def\version{final}
\input{childdoc.def}
\childdocforwardprefix[cdocsamp]{cdocsfn}{cdocsch}
%    \end{macrocode}

%\iffalse
%</samplefinal>
%\fi
%
% %%%%%%%%%%%%%%%%%%%%%%%%%%%%%%%%%%%%%%
% \paragraph{Command Line Processing.}
%
% The following three command lines generate the output files
% |cdocscld|, |cdocscl1| and |cdocscl2|
% which should be identical to
% |cdocsdrf|, |cdocsch1| and |cdocsfn2|, respectively:
% \begin{center}
% \begin{tabular}{l}
% |latex -jobname cdocscld \|\\
% |  "\def\version{draft}\input{childdoc.def}\childdocforward{cdocsamp}"|\\
% |latex -jobname cdocscl1 \|\\
% |  "\input{childdoc.def}\childdocforward[cdocsamp]{cdocsch1}"|\\
% |latex -jobname cdocscl2 \|\\
% |  "\def\version{final}\input{childdoc.def}\childdocforward{cdocsch2}"|
% \end{tabular}
% \end{center}
% Note that the trailing backslash on each first line
% merely continues the input to the second line
% (for convenient cut ant paste).
% Furthermore, the command |latex| can be replaced by any
% of its alternative versions such as |pdflatex|.
%
% %%%%%%%%%%%%%%%%%%%%%%%%%%%%%%%%%%%%%%%%%%%%%%%%%%%%%%%%%%%%%%%%%%%%%%%%%%%%%%
% %%%%%%%%%%%%%%%%%%%%%%%%%%%%%%%%%%%%%%%%%%%%%%%%%%%%%%%%%%%%%%%%%%%%%%%%%%%%%%
% \section{Implementation}
%\iffalse
%<*package>
%\fi
%
% This section describes the definitions file |childdoc.def|.

% The definitions cannot be loaded using |\usepackage| or |\RequirePackage|
% which has a mechanism to prevent loading a style file more than once.
% When loading the definitions by means of |\input|
% multiple instances have to be prevented manually:
%\iffalse
%This code needs to be before the `\ProvidesFile' directive
%which is defined at the beginning of this file.
%Therefore it is also placed there and commented out here.
%</package>
%<*discard>
%\fi
%    \begin{macrocode}
\ifdefined\childdocmain\endinput\fi
%    \end{macrocode}
%\iffalse
%</discard>
%<*package>
%\fi
%
% \macro{\ifchilddoc}
% \macro{\ifchilddocmanual}
% The conditional |\ifchilddoc| tells whether a
% child (true) or main (false) document is being compiled.
% The conditional |\ifchilddocmanual| tells whether
% the |\includeonly| mechanism is used (false) or
% the selection of child files must be performed manually (true).
% The definitions initialise to false:
%    \begin{macrocode}
\newif\ifchilddoc
\newif\ifchilddocmanual
%    \end{macrocode}

% \macro{\childdocname}
% \macro{\childdocjob}
% The macro |\childdocname| stores the name of the main document
% to be compiled. The macro |\childdocjob| stores the name of
% the document on which the \LaTeX{} compiler was originally invoked.
% The content of |\jobname| cannot be compared
% to filenames specified in the source due to different catcodes.
% The following code rescans |\jobname|, stores the result
% in |\childdocname| and saves a copy in |\childdocjob|:
%    \begin{macrocode}
\edef\childdocname{\scantokens\expandafter{\jobname\noexpand}}
\let\childdocjob\childdocname
%    \end{macrocode}

% \macro{\childdocdisable}
% The macro |\childdocdisable| prevents the main file
% from being processed more than once.
% At this stage, the main document command |\childdocmain|
% is assumed to be called once again where it should do nothing.
% Any subsequent call to it should prevent
% a secondary processing of the main document
% It overwrites the forwarding commands
% |\childdocof| and |\childdocforward|
% with empty macros to prevent further inclusions of the main document:
%    \begin{macrocode}
\newcommand{\childdocdisable}
{
  \renewcommand{\childdocmain}[1]{\renewcommand{\childdocmain}[1]{\endinput}}
  \renewcommand{\childdocof}[1]{}
  \renewcommand{\childdocby}[2][]{}
  \renewcommand{\childdocforward}[2][]{}
  \renewcommand{\childdocdisable}{}
}
%    \end{macrocode}

% \macro{\childdocmain}
% The macro |\childdocmain| is to be called at the top of the main file
% with nothing or the main filename (without extension) as argument.
% First, it breaks loops.
% If the argument is not empty and does not match |\childdocname|
% (which is set by the first inclusion of |childdoc.def|),
% |\ifchilddoc| is set to true, |\includeonly| is applied to the child file
% and |\jobname| is set to the main file
% (for proper handling of |.aux| files):
%    \begin{macrocode}
\newcommand{\childdocmain}[1]
{
  \childdocdisable\childdocmain{}
  \if?#1?\else
    \begingroup
      \def\childdoctmp{#1}
      \ifx\childdoctmp\childdocname
        \def\childdoctmp{}
      \else
        \def\childdoctmp
        {
          \childdoctrue
          \includeonly{\childdocname}
          \def\childdocjob{#1}
          \def\jobname{#1}
        }
      \fi
      \expandafter
    \endgroup
    \childdoctmp
  \fi
}
%    \end{macrocode}

% \macro{\childdocof}
% The command |\childdocof| redirects
% compilation to the main file |#1|.
%    \begin{macrocode}
\newcommand{\childdocof}[1]
{
  \childdocdisable
  \childdoctrue
  \includeonly{\childdocname}
  \def\jobname{#1}
  \def\childdocjob{#1}
  \input{#1}
}
%    \end{macrocode}

% \macro{\childdocby}
% The command |\childdocby| ....
%    \begin{macrocode}
\newcommand{\childdocby}[2][]
{
  \childdocdisable
  \childdoctrue
  \childdocmanualtrue
  \if?#1?\else
    \def\jobname{#2}
  \fi
  \def\childdocjob{#2}
  \input{#2}
  \endinput
}
%    \end{macrocode}

% \macro{\childdocforward}
% The command |\childdocforward| redirects
% compilation to the main file or
% (if the optional argument is given) a child file.
% Parameters are set as if the main file
% or a child file starting with |\childdocof| was compiled.
% Then compilation is handed over to the main file:
%    \begin{macrocode}
\newcommand{\childdocforward}[2][]
{
  \begingroup
    \if?#1?
      \def\childdoctmp
      {
        \def\childdocname{#2}
        \def\childdocjob{#2}
        \def\jobname{#2}
        \input{#2}
        \endinput
      }
    \else
      \def\childdoctmp
      {
        \childdocdisable
        \def\childdocname{#2}
        \childdoctrue
        \includeonly{#2}
        \def\childdocjob{#1}
        \def\jobname{#1}
        \input{#1}
        \endinput
      }
    \fi
    \expandafter
  \endgroup
  \childdoctmp
}
%    \end{macrocode}

% \macro{\childdocforwardprefix}
% The command |\childdocforwardprefix| redirects
% compilation to the main or a child file by means of a pattern.
% The prefix |#1| in the current filename is replaced by |#2|
% and the suffix of the current filename is kept
% (it is assumed that the filename does not contain the substring `|~~~|'
% which is used as a delimiter).
% Compilation is handed over to the new file by |\childdocforward|:
%    \begin{macrocode}
\newcommand{\childdocforwardprefix}[3][]
{
  \begingroup
    \def\childdocextract #2##1~~~{\def\childdoctmp{\childdocforward[#1]{#3##1}}}
    \expandafter\childdocextract\childdocname~~~
    \expandafter
  \endgroup
  \childdoctmp
}
%    \end{macrocode}

% \macro{\childdoc}
% The deprecated macro |\childdoc| is a legacy version of |\childdocmain|:
%    \begin{macrocode}
\newcommand{\childdoc}{\childdocmain}
%    \end{macrocode}

% \macro{\childdocredirect}
% The deprecated macro |\childdocredirect| is a legacy version
% of |\childdocforward| and |\childdocforwardprefix|:
%    \begin{macrocode}
\newcommand{\childdocredirect}[2][]
{
  \begingroup
    \if?#1?
      \def\childdoctmp{\childdocforward{#2}}
    \else
      \def\childdoctmp{\childdocforwardprefix{#1}{#2}}
    \fi
    \expandafter
  \endgroup
  \childdoctmp
}
%    \end{macrocode}

%\iffalse
%</package>
%\fi
%
\endinput
|\\
|\childdocmain{|\textit{main}|}|\\
\end{tabular}
\end{center}
%
If |\jobname| does not match the argument \textit{main} of |\childdocmain|,
it is assumed that |\jobname| points to the child file to be compiled.
When using |\childdocmain| with the main file specified as argument,
it suffices to start a child file
with just |\input{|\textit{main}|}|
without loading of the package and using |\childdocof|.
If instead all processing is done
with the appropriate \textsf{childdoc} directives,
the argument of \textit{main} of |\childdocmain| can be empty.

An alternative version of the command line processing described
in \secref{sec:commandline} using the detection mechanism reads:
%
\begin{center}
|... -jobname "|\textit{target}|" "|[\textit{flags}]%
[|\def\jobname{|\textit{dest}|}|]|\input{|\textit{main}|}"|
\end{center}

%%%%%%%%%%%%%%%%%%%%%%%%%%%%%%%%%%%%%%%%%%%%%%%%%%%%%%%%%%%%%%%%%%%%%%%%%%%%%%%%
\subsection{Manual Code}
\label{sec:manual}

In case one cannot be certain whether the definitions file |childdoc.def|
is installed on the target \TeX{} distribution
and one prefers not to ship it,
it is conceivable to paste a few relevant commands into the sources.

To that end, drop all statements |% \iffalse
%
% childdoc.dtx Copyright (C) 2017-2018 Niklas Beisert
%
% This work may be distributed and/or modified under the
% conditions of the LaTeX Project Public License, either version 1.3
% of this license or (at your option) any later version.
% The latest version of this license is in
%   http://www.latex-project.org/lppl.txt
% and version 1.3 or later is part of all distributions of LaTeX
% version 2005/12/01 or later.
%
% This work has the LPPL maintenance status `maintained'.
%
% The Current Maintainer of this work is Niklas Beisert.
%
% This work consists of the files childdoc.dtx and childdoc.ins
% and the derived files childdoc.def and cdocsamp.tex with
% cdocsch1.tex, cdocsch2.tex, cdocsdrf.tex, cdocsfn1.tex, cdocsfn2.tex.
%
%<package>\ifdefined\childdocmain\endinput\fi
%<package>\ProvidesFile{childdoc.def}[2018/12/30 v2.0 child document driver]
%<samplemain>\ProvidesFile{cdocsamp.tex}[2018/12/30 v2.0 sample for childdoc]
%<*driver>
%\ProvidesFile{childdoc.drv}[2018/12/30 v2.0 childdoc reference manual file]
\PassOptionsToClass{10pt,a4paper}{article}
\documentclass{ltxdoc}

\usepackage[margin=35mm]{geometry}
\usepackage{hyperref}
\usepackage{hyperxmp}
\usepackage[usenames]{color}

\hypersetup{colorlinks=true}
\hypersetup{pdfstartview=FitH}
\hypersetup{pdfpagemode=UseNone}
\hypersetup{pdfsource={}}
\hypersetup{pdflang={en-UK}}
\hypersetup{pdfcopyright={Copyright 2017-2018 Niklas Beisert.
  This work may be distributed and/or modified under the
  conditions of the LaTeX Project Public License, either version 1.3
  of this license or (at your option) any later version.}}
\hypersetup{pdflicenseurl={http://www.latex-project.org/lppl.txt}}
\hypersetup{pdfcontactaddress={ETH Zurich, ITP, HIT K,
  Wolfgang-Pauli-Strasse 27}}
\hypersetup{pdfcontactpostcode={8093}}
\hypersetup{pdfcontactcity={Zurich}}
\hypersetup{pdfcontactcountry={Switzerland}}
\hypersetup{pdfcontactemail={nbeisert@itp.phys.ethz.ch}}
\hypersetup{pdfcontacturl={http://people.phys.ethz.ch/\xmptilde nbeisert/}}

\newcommand{\secref}[1]{\hyperref[#1]{section \ref*{#1}}}

\parskip1ex
\parindent0pt
\let\olditemize\itemize
\def\itemize{\olditemize\parskip0pt}

\begin{document}

\title{The \textsf{childdoc} Package}
\hypersetup{pdftitle={The childdoc Package}}
\author{Niklas Beisert\\[2ex]
  Institut f\"ur Theoretische Physik\\
  Eidgen\"ossische Technische Hochschule Z\"urich\\
  Wolfgang-Pauli-Strasse 27, 8093 Z\"urich, Switzerland\\[1ex]
  \href{mailto:nbeisert@itp.phys.ethz.ch}
  {\texttt{nbeisert@itp.phys.ethz.ch}}}
\hypersetup{pdfauthor={Niklas Beisert}}
\hypersetup{pdfsubject={Manual for the LaTeX2e Package childdoc}}
\date{30 December 2018, \textsf{v2.0}}
\maketitle

\begin{abstract}\noindent
\textsf{childdoc} is a \LaTeXe{} package
that enables the direct compilation
of document sections included by |\include|
to individual files.
\end{abstract}

\begingroup
\parskip0ex
\tableofcontents
\endgroup

%%%%%%%%%%%%%%%%%%%%%%%%%%%%%%%%%%%%%%%%%%%%%%%%%%%%%%%%%%%%%%%%%%%%%%%%%%%%%%%%
%%%%%%%%%%%%%%%%%%%%%%%%%%%%%%%%%%%%%%%%%%%%%%%%%%%%%%%%%%%%%%%%%%%%%%%%%%%%%%%%
\section{Introduction}

\LaTeX{} provides a mechanism to structure a large document (such as a book)
into a main file and several child files (containing the chapters)
using the |\include| command.
This mechanism is beneficial for documents
which span hundreds of pages in order to
make the source file(s) more manageable.
Moreover, compilation can be restricted to
selected child files by means of the |\includeonly| command.
The latter feature can be used to reduce the compilation time while editing
(this was significantly more useful in the earlier days of \LaTeX{})
or to generate a smaller document which is easier to navigate.
Another application of |\includeonly| is to generate
documents consisting of selected parts of the complete document.

However, there are a few drawbacks of the plain |\include| mechanism:
\begin{itemize}
\item
The child files cannot be compiled on their own,
they can only be compiled via the main file.
A naive editing environment
(such as a text editor with an option
to have the current file processed by \LaTeX)
may require one to switch to the main file before compiling;
attempting to compile the child file produces errors.
\item
The main file must be modified (each time)
to adjust the |\includeonly| command
to the present needs. This easily leaves the main file in a messy state.
\item
The generated document will always carry the filename
of the main document. This is inconvenient if
several child files are to be compiled and
to be kept for distribution.
\end{itemize}

The present package provides a simple interface
to make child files individually compilable by \LaTeX{}.
Compiling a child file then has the same effect as compiling
the main file with an |\includeonly| command
to select the appropriate child.
Moreover the generated document will carry the name of the child
rather than the main file.
This resolves all three above issues.

This feature is meant to make the editing of books,
thesis documents and lecture notes somewhat more convenient.
However, the package can also be used efficiently for
composing a series of documents (such as exercise sheets)
which are typically distributed individually.
It then assists the author in generating the individual documents
(potentially in different versions)
as well as a document containing the collected series.
Another application is in developing style files
or other kinds of included material
where compilation of the style file could redirect
to a sample or test file.

%%%%%%%%%%%%%%%%%%%%%%%%%%%%%%%%%%%%%%%%%%%%%%%%%%%%%%%%%%%%%%%%%%%%%%%%%%%%%%%%
%%%%%%%%%%%%%%%%%%%%%%%%%%%%%%%%%%%%%%%%%%%%%%%%%%%%%%%%%%%%%%%%%%%%%%%%%%%%%%%%
\section{Usage}

First of all, the package \textsf{childdoc} is \emph{not} a standard
\LaTeXe{} |.sty| style file! Therefore it needs to be invoked in
a non-standard way.

%%%%%%%%%%%%%%%%%%%%%%%%%%%%%%%%%%%%%%%%%%%%%%%%%%%%%%%%%%%%%%%%%%%%%%%%%%%%%%%%
\subsection{Included Files}
\label{sec:include}

%%%%%%%%%%%%%%%%%%%%%%%%%%%%%%%%%%%%%%%%
\DescribeMacro{\childdocmain}
To use the package, add the commands
\begin{center}
\begin{tabular}{l}
|\input{childdoc.def}|\\
|\childdocmain{}|\\
\end{tabular}
\end{center}
at the very top of the main \LaTeX{} file,
in particular \emph{before} the |\documentclass| statement!
The argument of |\childdocmain| should be left empty
(but it must be present).

%%%%%%%%%%%%%%%%%%%%%%%%%%%%%%%%%%%%%%%%
\DescribeMacro{\childdocof}
Furthermore, add the commands
\begin{center}
\begin{tabular}{l}
|\input{childdoc.def}|\\
|\childdocof{|\textit{main}|}|\\
\end{tabular}
\end{center}
at the top of every child file \textit{child}
which is included by |\include{|\textit{child}|}|
from within the main file
(or at least for those files to be compiled individually).
The argument \textit{main} must be the filename of the main file.

There are a couple of
considerations in setting up the main and child documents:

%%%%%%%%%%%%%%%%%%%%%%%%%%%%%%%%%%%%%%%%
\paragraph{Restrictions.}

Please note the following restrictions:
\begin{itemize}
\item
|\childdocmain| must be called with one argument \textit{main}
to ensure compatibility with earlier version of the package.
It must either be empty (|\childdocmain{}|)
or precisely match the filename of the main file in which it is specified.
See \secref{sec:detection} for further information.
\item
The filename \textit{main} must be specified without the |.tex| extension.
\item
The filename \textit{main} is case sensitive
(even in case-insensitive file systems)
due to internal string comparison.
\item
The argument \textit{main} should be fully expanded, it cannot be a macro.
\item
Subdirectories and special characters should be avoided in filenames.
\item
The command |\childdocmain{|\textit{main}|}| must be followed by a whitespace.
It should not be followed immediately by another command
or by a comment mark `|%|'.
This is because the \TeX{} parser reads the token immediately following
the argument of |\childdocmain| and puts it
at the beginning of every child section;
however, a white\-space is ignored.
\end{itemize}

%%%%%%%%%%%%%%%%%%%%%%%%%%%%%%%%%%%%%%%%
\paragraph{Content of Main File.}

It is advisable to place all content in the child files included by |\include|.
Any output contained in the main file will appear in all child documents
unless suppressed manually;
it cannot be suppressed automatically by the |\includeonly| directive
and thus should normally be avoided.
A method to include some content in the main file
by means of conditional processing is described in \secref{sec:conditional}.

%%%%%%%%%%%%%%%%%%%%%%%%%%%%%%%%%%%%%%%%
\paragraph{Page Numbering.}

When only a part of the document is compiled,
the appropriate numbering of pages
(as well as other status parameters)
is determined from the |.aux| files.
The latter contain information from previous passes.
However this information needs to propagate through
all intermediate child documents.
Therefore the page numbering in child documents may well
be inconsistent until the complete document is compiled at least once.

A useful (if unconventional) way to always ensure a consistent
page numbering is to restart the numbering in each child document
and denote the pages by `\textit{child}|.|\textit{page}'
where \textit{child} represents the chapter/section number of the child file.
This can be achieved by the command
|\numberwithin{page}{|\textit{child}|}|
of the \textsf{amsmath} package
where \textit{child} can be |chapter| or |section|
depending on the chosen structuring.
Alternatively, one can modify the macro |\thepage| appropriately
and reset the counter |page| at the start of each child file.

%%%%%%%%%%%%%%%%%%%%%%%%%%%%%%%%%%%%%%%%%%%%%%%%%%%%%%%%%%%%%%%%%%%%%%%%%%%%%%%%
\subsection{Conditional Processing}
\label{sec:conditional}

The package provides a mechanism to compile different versions
of a document. To customise the versions further some conditional processing
can come in handy to distinguish which version is being compiled.
The package provides two macros to describe the compilation context:

%%%%%%%%%%%%%%%%%%%%%%%%%%%%%%%%%%%%%%%%
\DescribeMacro{\ifchilddoc}
The conditional |\ifchilddoc| distinguishes between the compilation of
child documents and the main document:
%
\begin{center}
|\ifchilddoc |\textit{child-code}| |[|\||else |\textit{main-code}]| \||fi|
\end{center}

%%%%%%%%%%%%%%%%%%%%%%%%%%%%%%%%%%%%%%%%
\DescribeMacro{\childdocname}
\DescribeMacro{\childdocjob}
The macro |\childdocname| contains the filename (without extension)
of the main or child file being processed.
Note that |\childdocjob| will always contain the name of the main file.

%%%%%%%%%%%%%%%%%%%%%%%%%%%%%%%%%%%%%%%%
\paragraph{Title Page.}

Conditional processing can be used to include a title or banner page
in the main document when proper precautions are taken.
Importantly, the code in the main file should ensure that the page counter
(as well as other status parameters which are stored in the |.aux| files)
takes the same value after the conditional processing.
Otherwise the page numbers may take divergent values
depending on which part is compiled.

For example, a title page could be declared by:
%
\begin{center}
\begin{tabular}{l}
|\ifchilddoc\||else|\\
|\addtocounter{page}{-1}|\\
\textit{code for title page}\\
|\newpage|\\
|\||fi|
\end{tabular}
\end{center}
%
A banner page for the child documents can be generated by:
%
\begin{center}
\begin{tabular}{l}
|\ifchilddoc|\\
|\addtocounter{page}{-1}|\\
\textit{code for banner page}\\
|\newpage|\\
|\||fi|
\end{tabular}
\end{center}
%
Here one could write a message such as:
\begin{center}
|This is the part \childdocname{} of \childdocjob{}.|
\end{center}

%%%%%%%%%%%%%%%%%%%%%%%%%%%%%%%%%%%%%%%%%%%%%%%%%%%%%%%%%%%%%%%%%%%%%%%%%%%%%%%%
\subsection{Flags}
\label{sec:flags}

The package makes it easy to generate different versions
of the main or child documents.
To this end compilation flags can be defined
and assigned different default values.
They will be particularly useful in conjunction
with the forwarding mechanism described in \secref{sec:forward}.

For example, it may be useful to have a flag |\version|
which can be set to |draft| or |final|.
The document source will contain some conditional code
depending on the value of |\version|.
Suppose further, the flag should default to |final| for the main file
and to |draft| for child files
which is a natural assignment for editing the document.
This is achieved by placing the following code
in the preamble of the main document
(below the |\childdocmain| directive):
%
\begin{center}
\begin{tabular}{l}
|\ifchilddoc|\\
|\providecommand{\version}{draft}|\\
|\||else|\\
|\providecommand{\version}{final}|\\
|\||fi|
\end{tabular}
\end{center}
%
The definition by |\providecommand| makes sure
that previous definitions are not overwritten.
Further statements |\providecommand{\version}{...}|
can thus be added before the above code to override it.

For the main file, one might add a line
(between |\childdocmain| and the above block)
%
\begin{center}
|%\ifchilddoc\||else\providecommand{\version}{draft}\||fi|
\end{center}
%
which can be uncommented to produce a draft version.
Likewise one can add a line to the very top of a child file
(above the |\childdocof{|\textit{main}|}| directive)
%
\begin{center}
|%\providecommand{\version}{final}|
\end{center}
%
which can be uncommented to produce the final version of this child document.

%%%%%%%%%%%%%%%%%%%%%%%%%%%%%%%%%%%%%%%%%%%%%%%%%%%%%%%%%%%%%%%%%%%%%%%%%%%%%%%%
\subsection{Forwarding}
\label{sec:forward}

Different versions of the main or child documents
using compilation flags as described in \secref{sec:flags}
can be (permanently) stored in different files
for convenient compilation, viewing and distribution.
To this end, the package defines a command
to pass on compilation to a different file:

%%%%%%%%%%%%%%%%%%%%%%%%%%%%%%%%%%%%%%%%
\DescribeMacro{\childdocforward}
The command |\childdocforward| redirects processing to
another source file:
%
\begin{center}
\begin{tabular}{l}
|\input{childdoc.def}|\\
|\childdocforward[|\textit{main}|]{|\textit{dest}|}|\\
\end{tabular}
\end{center}
%
The argument \textit{dest} is the destination file
(without extension).
It should be the main file or one of the child files.
Note that further \textsf{childdoc} directives
such as |\childdocof| and |\childdocforward|
in the indicated file will be processed in this form.
The optional argument \textit{main}
passes on directly to the main file \textit{main}
while pretending to compile the child \textit{dest}.
This form behaves as if \textit{dest}
issues |\childdocof{|\textit{main}|}| right away,
and no further \textsf{childdoc} directives will be processed.

%%%%%%%%%%%%%%%%%%%%%%%%%%%%%%%%%%%%%%%%
\DescribeMacro{\...prefix}
In the alternative form |\childdocforwardprefix|,
%
\begin{center}
\begin{tabular}{l}
|\input{childdoc.def}|\\
|\childdocforwardprefix[|\textit{main}|]{|\textit{prefix}|}{|\textit{dest}|}|
\end{tabular}
\end{center}
%
the destination file is determined by a pattern
depending on the current file:
To make this work, the current file must be called
`{\textit{prefix}\hspace{0.2em}\textit{suffix}}'
with \textit{prefix} matching precisely the argument.
Processing is then passed on to the file
`{\textit{dest}\hspace{0.2em}\textit{suffix}}'.
Surely, the same effect is achieved by
directly specifying the
argument `{\textit{dest}\hspace{0.2em}\textit{suffix}}'
in the first form.
However, that requires to set up a different file
for each child. With the alternative form of the command
all these files can have exactly the same content
which simplifies setting them up and maintaining them.

For example, the following file |draft.tex|
with a compilation flag |\version| as described in \secref{sec:flags}
compiles the main document as a draft:
%
\begin{center}
\begin{tabular}{l}
|\def\version{draft}|\\
|\input{childdoc.def}|\\
|\childdocforward{|\textit{main}|}|
\end{tabular}
\end{center}
%
Likewise, the following files |final|\textit{nn}|.tex|
compile the final version of the child document
|child|\textit{nn}|.tex|:
%
\begin{center}
\begin{tabular}{l}
|\def\version{final}|\\
|\input{childdoc.def}|\\
|\childdocforwardprefix{final}{child}|
\end{tabular}
\end{center}
%

Note that when several versions of a main file and/or of each child file
are to be generated, it may be convenient to set up a |Makefile| or
shell script to automatise the process.

%%%%%%%%%%%%%%%%%%%%%%%%%%%%%%%%%%%%%%%%%%%%%%%%%%%%%%%%%%%%%%%%%%%%%%%%%%%%%%%%
\subsection{Command Line Processing}
\label{sec:commandline}

The effect of redirection files can also be achieved by invoking
the \LaTeX{} compiler with a more elaborate command line.
Most conveniently this should be done as part
of a shell script or a |Makefile|.

When using \textsf{childdoc} in the main file, the following
command lines effectively perform a redirection
(note that depending on the shell being used,
backslashes may have to be doubled: `|\|' $\to$ `|\\|'):
%
\begin{center}
|... -jobname "|\textit{target}|" |\\|"|[\textit{flags}]%
|\input{childdoc.def}\childdocforward[|\textit{main}|]{|\textit{dest}|}"|
\end{center}
%
Here \textit{target} is the name of the output file,
\textit{main} is the name of the main file
and \textit{dest} is the name of the main or child file to be processed
(all filenames without extensions).
The optional argument \textit{main} can be omitted
if \textit{main} matches \textit{dest}.
Optionally, compilation \textit{flags} can be defined via |\def| commands.
This command line makes the \TeX{} engine believe
it is compiling the file \textit{target}
whose content is specified as the latter parameter.
The provided code then forwards the processing to
\textit{main} or \textit{dest} as described in \secref{sec:forward}.

%%%%%%%%%%%%%%%%%%%%%%%%%%%%%%%%%%%%%%%%%%%%%%%%%%%%%%%%%%%%%%%%%%%%%%%%%%%%%%%%
\subsection{Include by Input}
\label{sec:input}

Including child documents by |\include| has some restrictions by design.
Most notably, the content of a child document always occupies
its own set of pages; pages cannot be shared between child documents.
Usually, this behaviour makes perfect sense
because each child document contain an essential part of the document.
However, in some situations it may be desirable to compose
a document from a collection of parts
without having mandatory page breaks between then.
For this case, the package
provides a mechanism to include parts
by |\input| which can also be processed individually.
However, by construction this mechanism
requires manual handling of the content to be output.

%%%%%%%%%%%%%%%%%%%%%%%%%%%%%%%%%%%%%%%%
\DescribeMacro{\ifchilddocmanual}
The main file should be prepared as usual, see \secref{sec:include}.
However, the document body must make a distinction
between processing of an individual part and of the main document, e.g.:
%
\begin{center}
\begin{tabular}{l}
|\ifchilddocmanual|\\
|\input{\childdocname}|\\
|\||else|\\
\textit{document body with }|\input{|\textit{part}|}|\\
|\||fi|
\end{tabular}
\end{center}
%
The conditional |\ifchilddocmanual| is true whenever
a part to be included by |\input| is being compiled,
and the name of the part is stored in |\childdocname|.

%%%%%%%%%%%%%%%%%%%%%%%%%%%%%%%%%%%%%%%%
\DescribeMacro{\childdocby}
Each part to be included by |\input| should start with:
%
\begin{center}
\begin{tabular}{l}
|\input{childdoc.def}|\\
|\childdocby{|\textit{main}|}|\\
\end{tabular}
\end{center}
%
The directive |\childdocby| is similar to |\childdocof|
described in \secref{sec:include},
but the subsequent selection of content must be done manually.
To that end, both |\ifchilddoc| and |\ifchilddocmanual|
will be true upon processing of a part,
and the name of the part is stored in |\childdocname|.
Note that |\jobname| will be set to the filename of the current part
so that each part receives an individual |.aux| file
that does not interfere with the |.aux| file(s) of the main document.
This behaviour can be altered by the alternative form
|\childdocby[*]{|\textit{main}|}| (with a non-empty optional argument)
which uses the |.aux| file of the main document
by setting |\jobname| to \textit{main}.

%%%%%%%%%%%%%%%%%%%%%%%%%%%%%%%%%%%%%%%%%%%%%%%%%%%%%%%%%%%%%%%%%%%%%%%%%%%%%%%%
\subsection{Driver Development}
\label{sec:driver}

The \textsf{childdoc} mechanism can also be use for the development
of definition files such as \LaTeX{} styles or classes.
This case differs from the above setup with multiple parts
included by |\include| in that no |\includeonly| should be invoked.
This can be achieved by starting the include file
(before |\ProvidesPackage|) with:
%
\begin{center}
\begin{tabular}{l}
|\input{childdoc.def}|\\
|\childdocforward{|\textit{main}|}|\\
\end{tabular}
\end{center}
%
or alternatively with:
%
\begin{center}
\begin{tabular}{l}
|\input{childdoc.def}|\\
|\childdocby{|\textit{main}|}|\\
\end{tabular}
\end{center}
%
Both forms have slightly different effects as described above.
The main file is prepared as usual, see \secref{sec:include}.

%%%%%%%%%%%%%%%%%%%%%%%%%%%%%%%%%%%%%%%%%%%%%%%%%%%%%%%%%%%%%%%%%%%%%%%%%%%%%%%%
\subsection{Legacy Detection}
\label{sec:detection}

The directive |\childdocmain| in the main file can detect
whether the complete document or merely a child is to be compiled
even without using the directive |\childdocof|.
This method is deprecated because it is less robust
and there is no compelling reason to use it;
it is merely provided for backward compatibility
and it may be removed in future versions.

If the detection mechanism is to be used,
it is mandatory to correctly specify
the filename of the main file as the argument of |\childdocmain|:
%
\begin{center}
\begin{tabular}{l}
|\input{childdoc.def}|\\
|\childdocmain{|\textit{main}|}|\\
\end{tabular}
\end{center}
%
If |\jobname| does not match the argument \textit{main} of |\childdocmain|,
it is assumed that |\jobname| points to the child file to be compiled.
When using |\childdocmain| with the main file specified as argument,
it suffices to start a child file
with just |\input{|\textit{main}|}|
without loading of the package and using |\childdocof|.
If instead all processing is done
with the appropriate \textsf{childdoc} directives,
the argument of \textit{main} of |\childdocmain| can be empty.

An alternative version of the command line processing described
in \secref{sec:commandline} using the detection mechanism reads:
%
\begin{center}
|... -jobname "|\textit{target}|" "|[\textit{flags}]%
[|\def\jobname{|\textit{dest}|}|]|\input{|\textit{main}|}"|
\end{center}

%%%%%%%%%%%%%%%%%%%%%%%%%%%%%%%%%%%%%%%%%%%%%%%%%%%%%%%%%%%%%%%%%%%%%%%%%%%%%%%%
\subsection{Manual Code}
\label{sec:manual}

In case one cannot be certain whether the definitions file |childdoc.def|
is installed on the target \TeX{} distribution
and one prefers not to ship it,
it is conceivable to paste a few relevant commands into the sources.

To that end, drop all statements |\input{childdoc.def}|
and perform the replacements as outlined below.
Instead of |\childdocmain{|\textit{main}|}| add the following code
to the top of the main file:
%
\begin{center}
\begin{tabular}{l}
|\||ifdefined\childdocname\endinput\||fi\newif\ifchilddoc|\\
|\edef\childdocname{\scantokens\expandafter{\jobname\noexpand}}|\\
|\def\childdocmain{|\textit{main}|}\||ifx\childdocmain\childdocname\||else|\\
|\childdoctrue\includeonly{\childdocname}\let\jobname\childdocmain\||fi|\\
\end{tabular}
\end{center}
%
Instead of |\childdocof{|\textit{main}|}| just include the main file
at the top of each child file:
%
\begin{center}
|\input{|\textit{main}|}|
\end{center}
%
A simple redirection |\childdocforward{|\textit{dest}|}| is achieved by:
%
\begin{center}
|\def\jobname{|\textit{dest}|}\input{\jobname}|
\end{center}
%
The redirection with prefix
|\childdocforwardprefix[|\textit{prefix}|]{|\textit{dest}|}|
is accomplished by:
%
\begin{center}
\begin{tabular}{l}
|{\edef\jobname{\scantokens\expandafter{\jobname\noexpand}}|\\
|\def\redirectjob |\textit{prefix}|#1~~~{\gdef\jobname{|\textit{dest}|#1}}|\\
|\expandafter\redirectjob\jobname~~~}\input{\jobname}|
\end{tabular}
\end{center}

In an alternative approach,
child documents can be compiled by a specific command line
without additional code or specific definitions:
%
\begin{center}
|... -jobname "|\textit{target}|" "|[\textit{flags}]%
|\includeonly{|\textit{dest}|}\input{|\textit{main}|}"|
\end{center}
%

%%%%%%%%%%%%%%%%%%%%%%%%%%%%%%%%%%%%%%%%%%%%%%%%%%%%%%%%%%%%%%%%%%%%%%%%%%%%%%%%
%%%%%%%%%%%%%%%%%%%%%%%%%%%%%%%%%%%%%%%%%%%%%%%%%%%%%%%%%%%%%%%%%%%%%%%%%%%%%%%%
\section{Information}

%%%%%%%%%%%%%%%%%%%%%%%%%%%%%%%%%%%%%%%%%%%%%%%%%%%%%%%%%%%%%%%%%%%%%%%%%%%%%%%%
\subsection{Copyright}

Copyright \copyright{} 2017--2018 Niklas Beisert

This work may be distributed and/or modified under the
conditions of the \LaTeX{} Project Public License, either version 1.3
of this license or (at your option) any later version.
The latest version of this license is in
  \url{http://www.latex-project.org/lppl.txt}
and version 1.3 or later is part of all distributions of \LaTeX{}
version 2005/12/01 or later.

This work has the LPPL maintenance status `maintained'.

The Current Maintainer of this work is Niklas Beisert.

This work consists of the files |README.txt|, |childdoc.ins| and |childdoc.dtx|
as well as the derived files |childdoc.def|, |cdocsamp.tex|
with |cdocsch1.tex|, |cdocsch2.tex|, |cdocspt3.tex|, |cdocspt4.tex|,
|cdocsdrf.tex|, |cdocsfn1.tex|, |cdocsfn2.tex|
as well as |childdoc.pdf|.

%%%%%%%%%%%%%%%%%%%%%%%%%%%%%%%%%%%%%%%%%%%%%%%%%%%%%%%%%%%%%%%%%%%%%%%%%%%%%%%%
\subsection{Files and Installation}

The package consists of the files:
%
\begin{center}
\begin{tabular}{ll}
    |README.txt|   & readme file \\
    |childdoc.ins| & installation file \\
    |childdoc.dtx| & source file \\
    |childdoc.def| & definition file \\
    |cdocsamp.tex| & sample main file \\
    |cdocsch1.tex| & sample include file \\
    |cdocsch2.tex| & sample include file \\
    |cdocspt3.tex| & sample part file \\
    |cdocspt4.tex| & sample part file \\
    |cdocsdrf.tex| & sample redirection file \\
    |cdocsfn1.tex| & sample redirection file \\
    |cdocsfn2.tex| & sample redirection file \\
    |childdoc.pdf| & manual
\end{tabular}
\end{center}
%
The distribution consists of the files
|README.txt|, |childdoc.ins| and |childdoc.dtx|.
%
\begin{itemize}
\item
Run (pdf)\LaTeX{} on |childdoc.dtx|
to compile the manual |childdoc.pdf| (this file).
\item
Run \LaTeX{} on |childdoc.ins| to create the definitions file |childdoc.def|
and the sample |cdocsamp.tex| with include files
|cdocsch1.tex|, |cdocsch2.tex|, |cdocspt3.tex|, |cdocspt4.tex|,
|cdocsdrf.tex|, |cdocsfn1.tex|, |cdocsfn2.tex|.
Then copy the file |childdoc.def| to an appropriate directory of your \LaTeX{}
distribution, e.g.\ \textit{texmf-root}|/tex/latex/childdoc|.
\end{itemize}

%%%%%%%%%%%%%%%%%%%%%%%%%%%%%%%%%%%%%%%%%%%%%%%%%%%%%%%%%%%%%%%%%%%%%%%%%%%%%%%%
\subsection{Related CTAN Packages}

There are several other packages which offer a similar functionality:
%
\begin{itemize}
\item
The packages
\href{http://ctan.org/pkg/docmute}{\textsf{docmute}},
\href{http://ctan.org/pkg/includex}{\textsf{includex}} and
\href{http://ctan.org/pkg/standalone}{\textsf{standalone}}
provide commands to include only the document body of
a child file thus allowing both files to be compiled individually.
\item
The packages \href{http://ctan.org/pkg/subdocs}{\textsf{subdocs}}
and \href{http://ctan.org/pkg/subfiles}{\textsf{subfiles}}
provide structures in which the main and child documents can be
encapsulated and allowing them to be compiled individually.
The inclusion mechanism is different from the conventional |\include|.
\item
The package \href{http://ctan.org/pkg/combine}{\textsf{combine}}
is an elaborate solution to combine several documents into one.
\end{itemize}
%
See also the CTAN topic \href{http://ctan.org/topic/subdocs}{\textsf{subdocs}}
for further related packages.
The present package differs from the above solutions in that
a document structure constructed with the conventional |\include| mechanism
just needs two extra commands at the top of every file
such that all constituent files can be compiled individually.

%%%%%%%%%%%%%%%%%%%%%%%%%%%%%%%%%%%%%%%%%%%%%%%%%%%%%%%%%%%%%%%%%%%%%%%%%%%%%%%%
%\subsection{Feature Suggestions}
%
%The following is a list of features which may be useful for future
%versions of this package:
%%
%\begin{itemize}
%\item
%\ldots
%\end{itemize}

%%%%%%%%%%%%%%%%%%%%%%%%%%%%%%%%%%%%%%%%%%%%%%%%%%%%%%%%%%%%%%%%%%%%%%%%%%%%%%%%
\subsection{Revision History}

%%%%%%%%%%%%%%%%%%%%%%%%%%%%%%%%%%%%%%%%
\paragraph{v2.0:} 2018/12/30

\begin{itemize}
\item
immediate forward processing
\item
added |\childdocby| mechanism
\item
manual restructured
\end{itemize}

%%%%%%%%%%%%%%%%%%%%%%%%%%%%%%%%%%%%%%%%
\paragraph{v1.6:} 2018/01/17

\begin{itemize}
\item
application for development of include files
\item
corrections to manual
\end{itemize}

%%%%%%%%%%%%%%%%%%%%%%%%%%%%%%%%%%%%%%%%
\paragraph{v1.5:} 2017/05/21

\begin{itemize}
\item
more complete structuring introduced
\item
|\childdocof| introduced
\item
|\childdoc| renamed to |\childdocmain|
\item
|\childredirect| renamed to |\childdocforward| and |\childdocforwardprefix|
and functionality expanded
\end{itemize}

%%%%%%%%%%%%%%%%%%%%%%%%%%%%%%%%%%%%%%%%
\paragraph{v1.0:} 2017/04/27

\begin{itemize}
\item
manual and install package
\item
first version published on CTAN
\end{itemize}

%%%%%%%%%%%%%%%%%%%%%%%%%%%%%%%%%%%%%%%%
\paragraph{v0.6:} 2017/04/26

\begin{itemize}
\item
redirection mechanism added
\end{itemize}

%%%%%%%%%%%%%%%%%%%%%%%%%%%%%%%%%%%%%%%%
\paragraph{v0.5:} 2017/04/26

\begin{itemize}
\item
functionality in definition file
\end{itemize}


%%%%%%%%%%%%%%%%%%%%%%%%%%%%%%%%%%%%%%%%%%%%%%%%%%%%%%%%%%%%%%%%%%%%%%%%%%%%%%%%
%%%%%%%%%%%%%%%%%%%%%%%%%%%%%%%%%%%%%%%%%%%%%%%%%%%%%%%%%%%%%%%%%%%%%%%%%%%%%%%%
%%%%%%%%%%%%%%%%%%%%%%%%%%%%%%%%%%%%%%%%%%%%%%%%%%%%%%%%%%%%%%%%%%%%%%%%%%%%%%%%
\appendix

\settowidth\MacroIndent{\rmfamily\scriptsize 000\ }

 \DocInput{childdoc.dtx}

\end{document}
%</driver>
% \fi
%
% %%%%%%%%%%%%%%%%%%%%%%%%%%%%%%%%%%%%%%%%%%%%%%%%%%%%%%%%%%%%%%%%%%%%%%%%%%%%%%
% %%%%%%%%%%%%%%%%%%%%%%%%%%%%%%%%%%%%%%%%%%%%%%%%%%%%%%%%%%%%%%%%%%%%%%%%%%%%%%
% \section{Sample}
%\iffalse
%<*samplemain>
%\fi
%
% The following presents a sample document
% with two chapters, two parts, a title page,
% a compile flag as well as three forwarding files to set the flag.
% It consists of eight |.tex| files:
% \begin{center}
% \begin{tabular}{ll}
% |cdocsamp.tex|&main file\\
% |cdocsch1.tex|&include file for chapter 1\\
% |cdocsch2.tex|&include file for chapter 2\\
% |cdocspt3.tex|&include file for part 3\\
% |cdocspt4.tex|&include file for part 4\\
% |cdocsdrf.tex|&forwarding file for main file in draft mode\\
% |cdocsfi1.tex|&forwarding file for final version of chapter 1\\
% |cdocsfi2.tex|&forwarding file for final version of chapter 2\\
% \end{tabular}
% \end{center}
% Each of the eight files can be compiled directly by the \LaTeX{} compiler.
%
% %%%%%%%%%%%%%%%%%%%%%%%%%%%%%%%%%%%%%%
% \paragraph{Main File.}
%
% The main file is called |cdocsamp.tex|.
%
% Load the \textsf{childdoc} definitions and
% declare the filename for the main document:
%    \begin{macrocode}
\input{childdoc.def}
\childdocmain{}
%    \end{macrocode}

% Optional override for |\version| flag:
%    \begin{macrocode}
%%\ifchilddoc\else\providecommand{\version}{draft}\fi
%    \end{macrocode}

% Define the default values for the |\version| flag
% (|final| for the main file and |draft| for childs):
%    \begin{macrocode}
\ifchilddoc
\providecommand{\version}{draft}
\else
\providecommand{\version}{final}
\fi
%    \end{macrocode}

% Load the standard document class:
%    \begin{macrocode}
\documentclass[12pt]{article}
%    \end{macrocode}

% Start the document body:
%    \begin{macrocode}
\begin{document}
%    \end{macrocode}

% Declare a title page.
% Print title, part of document being processed and version flag:
%    \begin{macrocode}
\addtocounter{page}{-1}
\begin{center}
{\LARGE\bfseries{}childdoc example\par}
\vspace{1cm}
\ifchilddoc
\ifchilddocmanual part\else chapter\fi:
`\childdocname' of `\childdocjob'\par
\else
main document: `\childdocjob'\par
\fi
version: \version\par
\end{center}
\newpage
%    \end{macrocode}

% Manually include selected file,
% otherwise process as usual:
%    \begin{macrocode}
\ifchilddocmanual
\section*{part `\childdocname'}
\input{\childdocname}
\else
%    \end{macrocode}

% Include the two chapters:
%    \begin{macrocode}
\include{cdocsch1}
\include{cdocsch2}
%    \end{macrocode}

% Include the two parts unless only chapters should be displayed:
%    \begin{macrocode}
\ifchilddoc\else
\section{part three}
\input{cdocspt3}
\section{part four}
\input{cdocspt4}
\fi
%    \end{macrocode}

% Process as usual until here:
%    \begin{macrocode}
\fi
%    \end{macrocode}

% End of document body:
%    \begin{macrocode}
\end{document}
%    \end{macrocode}
%\iffalse
%</samplemain>
%\fi
%
% %%%%%%%%%%%%%%%%%%%%%%%%%%%%%%%%%%%%%%
% \paragraph{Chapter Include Files.}
%
% The include files are called |cdocsch1.tex| and |cdocsch2.tex|.
%
%\iffalse
%<*samplechap1|samplechap2>
%\fi

% Optional override for |\version| flag:
%    \begin{macrocode}
%%\providecommand{\version}{final}
%    \end{macrocode}

% Include the main document:
%    \begin{macrocode}
\input{childdoc.def}
\childdocof{cdocsamp}
%    \end{macrocode}

%\iffalse
%</samplechap1|samplechap2>
%\fi
%
%\iffalse
%<*samplechap1>
%\fi
% Some text for chapter 1:
%    \begin{macrocode}
\section{one}
some text in chapter one
%    \end{macrocode}

%\iffalse
%</samplechap1>
%\fi
% Some text for chapter 2:
%\iffalse
%<*samplechap2>
%\fi
%    \begin{macrocode}
\section{two}
more text in chapter two
%    \end{macrocode}

%\iffalse
%</samplechap2>
%\fi
%
% %%%%%%%%%%%%%%%%%%%%%%%%%%%%%%%%%%%%%%
% \paragraph{Part Include Files.}
%
% The include files are called |cdocspt3.tex| and |cdocspt4.tex|.
%
%\iffalse
%<*samplepart3|samplepart4>
%\fi

% Optional override for |\version| flag:
%    \begin{macrocode}
%%\providecommand{\version}{final}
%    \end{macrocode}

% Include the main document:
%    \begin{macrocode}
\input{childdoc.def}
\childdocby{cdocsamp}
%    \end{macrocode}

%\iffalse
%</samplepart3|samplepart4>
%\fi
%
%\iffalse
%<*samplepart3>
%\fi
% Some text for part 3:
%    \begin{macrocode}
some text in part three
%    \end{macrocode}

%\iffalse
%</samplepart3>
%\fi
% Some text for part 4:
%\iffalse
%<*samplepart4>
%\fi
%    \begin{macrocode}
more text in part four
%    \end{macrocode}

%\iffalse
%</samplepart4>
%\fi
%
% %%%%%%%%%%%%%%%%%%%%%%%%%%%%%%%%%%%%%%
% \paragraph{Forwarding for a Complete Draft.}
%
% The following forwarding file |cdocsdrf.tex|
% compiles the main document in draft mode:
%\iffalse
%<*sampledraft>
%\fi
%    \begin{macrocode}
\def\version{draft}
\input{childdoc.def}
\childdocforward{cdocsamp}
%    \end{macrocode}

%\iffalse
%</sampledraft>
%\fi
%
% %%%%%%%%%%%%%%%%%%%%%%%%%%%%%%%%%%%%%%
% \paragraph{Forwarding for Final Version of the Chapters.}
%
% The following forwarding files |cdocsfn1.tex| and |cdocsfn2.tex|
% (with identical content)
% compile the final versions of the child documents
% |cdocsch1.tex| and |cdocsch2.tex|, respectively:
%\iffalse
%<*samplefinal>
%\fi
%    \begin{macrocode}
\def\version{final}
\input{childdoc.def}
\childdocforwardprefix[cdocsamp]{cdocsfn}{cdocsch}
%    \end{macrocode}

%\iffalse
%</samplefinal>
%\fi
%
% %%%%%%%%%%%%%%%%%%%%%%%%%%%%%%%%%%%%%%
% \paragraph{Command Line Processing.}
%
% The following three command lines generate the output files
% |cdocscld|, |cdocscl1| and |cdocscl2|
% which should be identical to
% |cdocsdrf|, |cdocsch1| and |cdocsfn2|, respectively:
% \begin{center}
% \begin{tabular}{l}
% |latex -jobname cdocscld \|\\
% |  "\def\version{draft}\input{childdoc.def}\childdocforward{cdocsamp}"|\\
% |latex -jobname cdocscl1 \|\\
% |  "\input{childdoc.def}\childdocforward[cdocsamp]{cdocsch1}"|\\
% |latex -jobname cdocscl2 \|\\
% |  "\def\version{final}\input{childdoc.def}\childdocforward{cdocsch2}"|
% \end{tabular}
% \end{center}
% Note that the trailing backslash on each first line
% merely continues the input to the second line
% (for convenient cut ant paste).
% Furthermore, the command |latex| can be replaced by any
% of its alternative versions such as |pdflatex|.
%
% %%%%%%%%%%%%%%%%%%%%%%%%%%%%%%%%%%%%%%%%%%%%%%%%%%%%%%%%%%%%%%%%%%%%%%%%%%%%%%
% %%%%%%%%%%%%%%%%%%%%%%%%%%%%%%%%%%%%%%%%%%%%%%%%%%%%%%%%%%%%%%%%%%%%%%%%%%%%%%
% \section{Implementation}
%\iffalse
%<*package>
%\fi
%
% This section describes the definitions file |childdoc.def|.

% The definitions cannot be loaded using |\usepackage| or |\RequirePackage|
% which has a mechanism to prevent loading a style file more than once.
% When loading the definitions by means of |\input|
% multiple instances have to be prevented manually:
%\iffalse
%This code needs to be before the `\ProvidesFile' directive
%which is defined at the beginning of this file.
%Therefore it is also placed there and commented out here.
%</package>
%<*discard>
%\fi
%    \begin{macrocode}
\ifdefined\childdocmain\endinput\fi
%    \end{macrocode}
%\iffalse
%</discard>
%<*package>
%\fi
%
% \macro{\ifchilddoc}
% \macro{\ifchilddocmanual}
% The conditional |\ifchilddoc| tells whether a
% child (true) or main (false) document is being compiled.
% The conditional |\ifchilddocmanual| tells whether
% the |\includeonly| mechanism is used (false) or
% the selection of child files must be performed manually (true).
% The definitions initialise to false:
%    \begin{macrocode}
\newif\ifchilddoc
\newif\ifchilddocmanual
%    \end{macrocode}

% \macro{\childdocname}
% \macro{\childdocjob}
% The macro |\childdocname| stores the name of the main document
% to be compiled. The macro |\childdocjob| stores the name of
% the document on which the \LaTeX{} compiler was originally invoked.
% The content of |\jobname| cannot be compared
% to filenames specified in the source due to different catcodes.
% The following code rescans |\jobname|, stores the result
% in |\childdocname| and saves a copy in |\childdocjob|:
%    \begin{macrocode}
\edef\childdocname{\scantokens\expandafter{\jobname\noexpand}}
\let\childdocjob\childdocname
%    \end{macrocode}

% \macro{\childdocdisable}
% The macro |\childdocdisable| prevents the main file
% from being processed more than once.
% At this stage, the main document command |\childdocmain|
% is assumed to be called once again where it should do nothing.
% Any subsequent call to it should prevent
% a secondary processing of the main document
% It overwrites the forwarding commands
% |\childdocof| and |\childdocforward|
% with empty macros to prevent further inclusions of the main document:
%    \begin{macrocode}
\newcommand{\childdocdisable}
{
  \renewcommand{\childdocmain}[1]{\renewcommand{\childdocmain}[1]{\endinput}}
  \renewcommand{\childdocof}[1]{}
  \renewcommand{\childdocby}[2][]{}
  \renewcommand{\childdocforward}[2][]{}
  \renewcommand{\childdocdisable}{}
}
%    \end{macrocode}

% \macro{\childdocmain}
% The macro |\childdocmain| is to be called at the top of the main file
% with nothing or the main filename (without extension) as argument.
% First, it breaks loops.
% If the argument is not empty and does not match |\childdocname|
% (which is set by the first inclusion of |childdoc.def|),
% |\ifchilddoc| is set to true, |\includeonly| is applied to the child file
% and |\jobname| is set to the main file
% (for proper handling of |.aux| files):
%    \begin{macrocode}
\newcommand{\childdocmain}[1]
{
  \childdocdisable\childdocmain{}
  \if?#1?\else
    \begingroup
      \def\childdoctmp{#1}
      \ifx\childdoctmp\childdocname
        \def\childdoctmp{}
      \else
        \def\childdoctmp
        {
          \childdoctrue
          \includeonly{\childdocname}
          \def\childdocjob{#1}
          \def\jobname{#1}
        }
      \fi
      \expandafter
    \endgroup
    \childdoctmp
  \fi
}
%    \end{macrocode}

% \macro{\childdocof}
% The command |\childdocof| redirects
% compilation to the main file |#1|.
%    \begin{macrocode}
\newcommand{\childdocof}[1]
{
  \childdocdisable
  \childdoctrue
  \includeonly{\childdocname}
  \def\jobname{#1}
  \def\childdocjob{#1}
  \input{#1}
}
%    \end{macrocode}

% \macro{\childdocby}
% The command |\childdocby| ....
%    \begin{macrocode}
\newcommand{\childdocby}[2][]
{
  \childdocdisable
  \childdoctrue
  \childdocmanualtrue
  \if?#1?\else
    \def\jobname{#2}
  \fi
  \def\childdocjob{#2}
  \input{#2}
  \endinput
}
%    \end{macrocode}

% \macro{\childdocforward}
% The command |\childdocforward| redirects
% compilation to the main file or
% (if the optional argument is given) a child file.
% Parameters are set as if the main file
% or a child file starting with |\childdocof| was compiled.
% Then compilation is handed over to the main file:
%    \begin{macrocode}
\newcommand{\childdocforward}[2][]
{
  \begingroup
    \if?#1?
      \def\childdoctmp
      {
        \def\childdocname{#2}
        \def\childdocjob{#2}
        \def\jobname{#2}
        \input{#2}
        \endinput
      }
    \else
      \def\childdoctmp
      {
        \childdocdisable
        \def\childdocname{#2}
        \childdoctrue
        \includeonly{#2}
        \def\childdocjob{#1}
        \def\jobname{#1}
        \input{#1}
        \endinput
      }
    \fi
    \expandafter
  \endgroup
  \childdoctmp
}
%    \end{macrocode}

% \macro{\childdocforwardprefix}
% The command |\childdocforwardprefix| redirects
% compilation to the main or a child file by means of a pattern.
% The prefix |#1| in the current filename is replaced by |#2|
% and the suffix of the current filename is kept
% (it is assumed that the filename does not contain the substring `|~~~|'
% which is used as a delimiter).
% Compilation is handed over to the new file by |\childdocforward|:
%    \begin{macrocode}
\newcommand{\childdocforwardprefix}[3][]
{
  \begingroup
    \def\childdocextract #2##1~~~{\def\childdoctmp{\childdocforward[#1]{#3##1}}}
    \expandafter\childdocextract\childdocname~~~
    \expandafter
  \endgroup
  \childdoctmp
}
%    \end{macrocode}

% \macro{\childdoc}
% The deprecated macro |\childdoc| is a legacy version of |\childdocmain|:
%    \begin{macrocode}
\newcommand{\childdoc}{\childdocmain}
%    \end{macrocode}

% \macro{\childdocredirect}
% The deprecated macro |\childdocredirect| is a legacy version
% of |\childdocforward| and |\childdocforwardprefix|:
%    \begin{macrocode}
\newcommand{\childdocredirect}[2][]
{
  \begingroup
    \if?#1?
      \def\childdoctmp{\childdocforward{#2}}
    \else
      \def\childdoctmp{\childdocforwardprefix{#1}{#2}}
    \fi
    \expandafter
  \endgroup
  \childdoctmp
}
%    \end{macrocode}

%\iffalse
%</package>
%\fi
%
\endinput
|
and perform the replacements as outlined below.
Instead of |\childdocmain{|\textit{main}|}| add the following code
to the top of the main file:
%
\begin{center}
\begin{tabular}{l}
|\||ifdefined\childdocname\endinput\||fi\newif\ifchilddoc|\\
|\edef\childdocname{\scantokens\expandafter{\jobname\noexpand}}|\\
|\def\childdocmain{|\textit{main}|}\||ifx\childdocmain\childdocname\||else|\\
|\childdoctrue\includeonly{\childdocname}\let\jobname\childdocmain\||fi|\\
\end{tabular}
\end{center}
%
Instead of |\childdocof{|\textit{main}|}| just include the main file
at the top of each child file:
%
\begin{center}
|\input{|\textit{main}|}|
\end{center}
%
A simple redirection |\childdocforward{|\textit{dest}|}| is achieved by:
%
\begin{center}
|\def\jobname{|\textit{dest}|}\input{\jobname}|
\end{center}
%
The redirection with prefix
|\childdocforwardprefix[|\textit{prefix}|]{|\textit{dest}|}|
is accomplished by:
%
\begin{center}
\begin{tabular}{l}
|{\edef\jobname{\scantokens\expandafter{\jobname\noexpand}}|\\
|\def\redirectjob |\textit{prefix}|#1~~~{\gdef\jobname{|\textit{dest}|#1}}|\\
|\expandafter\redirectjob\jobname~~~}\input{\jobname}|
\end{tabular}
\end{center}

In an alternative approach,
child documents can be compiled by a specific command line
without additional code or specific definitions:
%
\begin{center}
|... -jobname "|\textit{target}|" "|[\textit{flags}]%
|\includeonly{|\textit{dest}|}\input{|\textit{main}|}"|
\end{center}
%

%%%%%%%%%%%%%%%%%%%%%%%%%%%%%%%%%%%%%%%%%%%%%%%%%%%%%%%%%%%%%%%%%%%%%%%%%%%%%%%%
%%%%%%%%%%%%%%%%%%%%%%%%%%%%%%%%%%%%%%%%%%%%%%%%%%%%%%%%%%%%%%%%%%%%%%%%%%%%%%%%
\section{Information}

%%%%%%%%%%%%%%%%%%%%%%%%%%%%%%%%%%%%%%%%%%%%%%%%%%%%%%%%%%%%%%%%%%%%%%%%%%%%%%%%
\subsection{Copyright}

Copyright \copyright{} 2017--2018 Niklas Beisert

This work may be distributed and/or modified under the
conditions of the \LaTeX{} Project Public License, either version 1.3
of this license or (at your option) any later version.
The latest version of this license is in
  \url{http://www.latex-project.org/lppl.txt}
and version 1.3 or later is part of all distributions of \LaTeX{}
version 2005/12/01 or later.

This work has the LPPL maintenance status `maintained'.

The Current Maintainer of this work is Niklas Beisert.

This work consists of the files |README.txt|, |childdoc.ins| and |childdoc.dtx|
as well as the derived files |childdoc.def|, |cdocsamp.tex|
with |cdocsch1.tex|, |cdocsch2.tex|, |cdocspt3.tex|, |cdocspt4.tex|,
|cdocsdrf.tex|, |cdocsfn1.tex|, |cdocsfn2.tex|
as well as |childdoc.pdf|.

%%%%%%%%%%%%%%%%%%%%%%%%%%%%%%%%%%%%%%%%%%%%%%%%%%%%%%%%%%%%%%%%%%%%%%%%%%%%%%%%
\subsection{Files and Installation}

The package consists of the files:
%
\begin{center}
\begin{tabular}{ll}
    |README.txt|   & readme file \\
    |childdoc.ins| & installation file \\
    |childdoc.dtx| & source file \\
    |childdoc.def| & definition file \\
    |cdocsamp.tex| & sample main file \\
    |cdocsch1.tex| & sample include file \\
    |cdocsch2.tex| & sample include file \\
    |cdocspt3.tex| & sample part file \\
    |cdocspt4.tex| & sample part file \\
    |cdocsdrf.tex| & sample redirection file \\
    |cdocsfn1.tex| & sample redirection file \\
    |cdocsfn2.tex| & sample redirection file \\
    |childdoc.pdf| & manual
\end{tabular}
\end{center}
%
The distribution consists of the files
|README.txt|, |childdoc.ins| and |childdoc.dtx|.
%
\begin{itemize}
\item
Run (pdf)\LaTeX{} on |childdoc.dtx|
to compile the manual |childdoc.pdf| (this file).
\item
Run \LaTeX{} on |childdoc.ins| to create the definitions file |childdoc.def|
and the sample |cdocsamp.tex| with include files
|cdocsch1.tex|, |cdocsch2.tex|, |cdocspt3.tex|, |cdocspt4.tex|,
|cdocsdrf.tex|, |cdocsfn1.tex|, |cdocsfn2.tex|.
Then copy the file |childdoc.def| to an appropriate directory of your \LaTeX{}
distribution, e.g.\ \textit{texmf-root}|/tex/latex/childdoc|.
\end{itemize}

%%%%%%%%%%%%%%%%%%%%%%%%%%%%%%%%%%%%%%%%%%%%%%%%%%%%%%%%%%%%%%%%%%%%%%%%%%%%%%%%
\subsection{Related CTAN Packages}

There are several other packages which offer a similar functionality:
%
\begin{itemize}
\item
The packages
\href{http://ctan.org/pkg/docmute}{\textsf{docmute}},
\href{http://ctan.org/pkg/includex}{\textsf{includex}} and
\href{http://ctan.org/pkg/standalone}{\textsf{standalone}}
provide commands to include only the document body of
a child file thus allowing both files to be compiled individually.
\item
The packages \href{http://ctan.org/pkg/subdocs}{\textsf{subdocs}}
and \href{http://ctan.org/pkg/subfiles}{\textsf{subfiles}}
provide structures in which the main and child documents can be
encapsulated and allowing them to be compiled individually.
The inclusion mechanism is different from the conventional |\include|.
\item
The package \href{http://ctan.org/pkg/combine}{\textsf{combine}}
is an elaborate solution to combine several documents into one.
\end{itemize}
%
See also the CTAN topic \href{http://ctan.org/topic/subdocs}{\textsf{subdocs}}
for further related packages.
The present package differs from the above solutions in that
a document structure constructed with the conventional |\include| mechanism
just needs two extra commands at the top of every file
such that all constituent files can be compiled individually.

%%%%%%%%%%%%%%%%%%%%%%%%%%%%%%%%%%%%%%%%%%%%%%%%%%%%%%%%%%%%%%%%%%%%%%%%%%%%%%%%
%\subsection{Feature Suggestions}
%
%The following is a list of features which may be useful for future
%versions of this package:
%%
%\begin{itemize}
%\item
%\ldots
%\end{itemize}

%%%%%%%%%%%%%%%%%%%%%%%%%%%%%%%%%%%%%%%%%%%%%%%%%%%%%%%%%%%%%%%%%%%%%%%%%%%%%%%%
\subsection{Revision History}

%%%%%%%%%%%%%%%%%%%%%%%%%%%%%%%%%%%%%%%%
\paragraph{v2.0:} 2018/12/30

\begin{itemize}
\item
immediate forward processing
\item
added |\childdocby| mechanism
\item
manual restructured
\end{itemize}

%%%%%%%%%%%%%%%%%%%%%%%%%%%%%%%%%%%%%%%%
\paragraph{v1.6:} 2018/01/17

\begin{itemize}
\item
application for development of include files
\item
corrections to manual
\end{itemize}

%%%%%%%%%%%%%%%%%%%%%%%%%%%%%%%%%%%%%%%%
\paragraph{v1.5:} 2017/05/21

\begin{itemize}
\item
more complete structuring introduced
\item
|\childdocof| introduced
\item
|\childdoc| renamed to |\childdocmain|
\item
|\childredirect| renamed to |\childdocforward| and |\childdocforwardprefix|
and functionality expanded
\end{itemize}

%%%%%%%%%%%%%%%%%%%%%%%%%%%%%%%%%%%%%%%%
\paragraph{v1.0:} 2017/04/27

\begin{itemize}
\item
manual and install package
\item
first version published on CTAN
\end{itemize}

%%%%%%%%%%%%%%%%%%%%%%%%%%%%%%%%%%%%%%%%
\paragraph{v0.6:} 2017/04/26

\begin{itemize}
\item
redirection mechanism added
\end{itemize}

%%%%%%%%%%%%%%%%%%%%%%%%%%%%%%%%%%%%%%%%
\paragraph{v0.5:} 2017/04/26

\begin{itemize}
\item
functionality in definition file
\end{itemize}


%%%%%%%%%%%%%%%%%%%%%%%%%%%%%%%%%%%%%%%%%%%%%%%%%%%%%%%%%%%%%%%%%%%%%%%%%%%%%%%%
%%%%%%%%%%%%%%%%%%%%%%%%%%%%%%%%%%%%%%%%%%%%%%%%%%%%%%%%%%%%%%%%%%%%%%%%%%%%%%%%
%%%%%%%%%%%%%%%%%%%%%%%%%%%%%%%%%%%%%%%%%%%%%%%%%%%%%%%%%%%%%%%%%%%%%%%%%%%%%%%%
\appendix

\settowidth\MacroIndent{\rmfamily\scriptsize 000\ }

 \DocInput{childdoc.dtx}

\end{document}
%</driver>
% \fi
%
% %%%%%%%%%%%%%%%%%%%%%%%%%%%%%%%%%%%%%%%%%%%%%%%%%%%%%%%%%%%%%%%%%%%%%%%%%%%%%%
% %%%%%%%%%%%%%%%%%%%%%%%%%%%%%%%%%%%%%%%%%%%%%%%%%%%%%%%%%%%%%%%%%%%%%%%%%%%%%%
% \section{Sample}
%\iffalse
%<*samplemain>
%\fi
%
% The following presents a sample document
% with two chapters, two parts, a title page,
% a compile flag as well as three forwarding files to set the flag.
% It consists of eight |.tex| files:
% \begin{center}
% \begin{tabular}{ll}
% |cdocsamp.tex|&main file\\
% |cdocsch1.tex|&include file for chapter 1\\
% |cdocsch2.tex|&include file for chapter 2\\
% |cdocspt3.tex|&include file for part 3\\
% |cdocspt4.tex|&include file for part 4\\
% |cdocsdrf.tex|&forwarding file for main file in draft mode\\
% |cdocsfi1.tex|&forwarding file for final version of chapter 1\\
% |cdocsfi2.tex|&forwarding file for final version of chapter 2\\
% \end{tabular}
% \end{center}
% Each of the eight files can be compiled directly by the \LaTeX{} compiler.
%
% %%%%%%%%%%%%%%%%%%%%%%%%%%%%%%%%%%%%%%
% \paragraph{Main File.}
%
% The main file is called |cdocsamp.tex|.
%
% Load the \textsf{childdoc} definitions and
% declare the filename for the main document:
%    \begin{macrocode}
% \iffalse
%
% childdoc.dtx Copyright (C) 2017-2018 Niklas Beisert
%
% This work may be distributed and/or modified under the
% conditions of the LaTeX Project Public License, either version 1.3
% of this license or (at your option) any later version.
% The latest version of this license is in
%   http://www.latex-project.org/lppl.txt
% and version 1.3 or later is part of all distributions of LaTeX
% version 2005/12/01 or later.
%
% This work has the LPPL maintenance status `maintained'.
%
% The Current Maintainer of this work is Niklas Beisert.
%
% This work consists of the files childdoc.dtx and childdoc.ins
% and the derived files childdoc.def and cdocsamp.tex with
% cdocsch1.tex, cdocsch2.tex, cdocsdrf.tex, cdocsfn1.tex, cdocsfn2.tex.
%
%<package>\ifdefined\childdocmain\endinput\fi
%<package>\ProvidesFile{childdoc.def}[2018/12/30 v2.0 child document driver]
%<samplemain>\ProvidesFile{cdocsamp.tex}[2018/12/30 v2.0 sample for childdoc]
%<*driver>
%\ProvidesFile{childdoc.drv}[2018/12/30 v2.0 childdoc reference manual file]
\PassOptionsToClass{10pt,a4paper}{article}
\documentclass{ltxdoc}

\usepackage[margin=35mm]{geometry}
\usepackage{hyperref}
\usepackage{hyperxmp}
\usepackage[usenames]{color}

\hypersetup{colorlinks=true}
\hypersetup{pdfstartview=FitH}
\hypersetup{pdfpagemode=UseNone}
\hypersetup{pdfsource={}}
\hypersetup{pdflang={en-UK}}
\hypersetup{pdfcopyright={Copyright 2017-2018 Niklas Beisert.
  This work may be distributed and/or modified under the
  conditions of the LaTeX Project Public License, either version 1.3
  of this license or (at your option) any later version.}}
\hypersetup{pdflicenseurl={http://www.latex-project.org/lppl.txt}}
\hypersetup{pdfcontactaddress={ETH Zurich, ITP, HIT K,
  Wolfgang-Pauli-Strasse 27}}
\hypersetup{pdfcontactpostcode={8093}}
\hypersetup{pdfcontactcity={Zurich}}
\hypersetup{pdfcontactcountry={Switzerland}}
\hypersetup{pdfcontactemail={nbeisert@itp.phys.ethz.ch}}
\hypersetup{pdfcontacturl={http://people.phys.ethz.ch/\xmptilde nbeisert/}}

\newcommand{\secref}[1]{\hyperref[#1]{section \ref*{#1}}}

\parskip1ex
\parindent0pt
\let\olditemize\itemize
\def\itemize{\olditemize\parskip0pt}

\begin{document}

\title{The \textsf{childdoc} Package}
\hypersetup{pdftitle={The childdoc Package}}
\author{Niklas Beisert\\[2ex]
  Institut f\"ur Theoretische Physik\\
  Eidgen\"ossische Technische Hochschule Z\"urich\\
  Wolfgang-Pauli-Strasse 27, 8093 Z\"urich, Switzerland\\[1ex]
  \href{mailto:nbeisert@itp.phys.ethz.ch}
  {\texttt{nbeisert@itp.phys.ethz.ch}}}
\hypersetup{pdfauthor={Niklas Beisert}}
\hypersetup{pdfsubject={Manual for the LaTeX2e Package childdoc}}
\date{30 December 2018, \textsf{v2.0}}
\maketitle

\begin{abstract}\noindent
\textsf{childdoc} is a \LaTeXe{} package
that enables the direct compilation
of document sections included by |\include|
to individual files.
\end{abstract}

\begingroup
\parskip0ex
\tableofcontents
\endgroup

%%%%%%%%%%%%%%%%%%%%%%%%%%%%%%%%%%%%%%%%%%%%%%%%%%%%%%%%%%%%%%%%%%%%%%%%%%%%%%%%
%%%%%%%%%%%%%%%%%%%%%%%%%%%%%%%%%%%%%%%%%%%%%%%%%%%%%%%%%%%%%%%%%%%%%%%%%%%%%%%%
\section{Introduction}

\LaTeX{} provides a mechanism to structure a large document (such as a book)
into a main file and several child files (containing the chapters)
using the |\include| command.
This mechanism is beneficial for documents
which span hundreds of pages in order to
make the source file(s) more manageable.
Moreover, compilation can be restricted to
selected child files by means of the |\includeonly| command.
The latter feature can be used to reduce the compilation time while editing
(this was significantly more useful in the earlier days of \LaTeX{})
or to generate a smaller document which is easier to navigate.
Another application of |\includeonly| is to generate
documents consisting of selected parts of the complete document.

However, there are a few drawbacks of the plain |\include| mechanism:
\begin{itemize}
\item
The child files cannot be compiled on their own,
they can only be compiled via the main file.
A naive editing environment
(such as a text editor with an option
to have the current file processed by \LaTeX)
may require one to switch to the main file before compiling;
attempting to compile the child file produces errors.
\item
The main file must be modified (each time)
to adjust the |\includeonly| command
to the present needs. This easily leaves the main file in a messy state.
\item
The generated document will always carry the filename
of the main document. This is inconvenient if
several child files are to be compiled and
to be kept for distribution.
\end{itemize}

The present package provides a simple interface
to make child files individually compilable by \LaTeX{}.
Compiling a child file then has the same effect as compiling
the main file with an |\includeonly| command
to select the appropriate child.
Moreover the generated document will carry the name of the child
rather than the main file.
This resolves all three above issues.

This feature is meant to make the editing of books,
thesis documents and lecture notes somewhat more convenient.
However, the package can also be used efficiently for
composing a series of documents (such as exercise sheets)
which are typically distributed individually.
It then assists the author in generating the individual documents
(potentially in different versions)
as well as a document containing the collected series.
Another application is in developing style files
or other kinds of included material
where compilation of the style file could redirect
to a sample or test file.

%%%%%%%%%%%%%%%%%%%%%%%%%%%%%%%%%%%%%%%%%%%%%%%%%%%%%%%%%%%%%%%%%%%%%%%%%%%%%%%%
%%%%%%%%%%%%%%%%%%%%%%%%%%%%%%%%%%%%%%%%%%%%%%%%%%%%%%%%%%%%%%%%%%%%%%%%%%%%%%%%
\section{Usage}

First of all, the package \textsf{childdoc} is \emph{not} a standard
\LaTeXe{} |.sty| style file! Therefore it needs to be invoked in
a non-standard way.

%%%%%%%%%%%%%%%%%%%%%%%%%%%%%%%%%%%%%%%%%%%%%%%%%%%%%%%%%%%%%%%%%%%%%%%%%%%%%%%%
\subsection{Included Files}
\label{sec:include}

%%%%%%%%%%%%%%%%%%%%%%%%%%%%%%%%%%%%%%%%
\DescribeMacro{\childdocmain}
To use the package, add the commands
\begin{center}
\begin{tabular}{l}
|\input{childdoc.def}|\\
|\childdocmain{}|\\
\end{tabular}
\end{center}
at the very top of the main \LaTeX{} file,
in particular \emph{before} the |\documentclass| statement!
The argument of |\childdocmain| should be left empty
(but it must be present).

%%%%%%%%%%%%%%%%%%%%%%%%%%%%%%%%%%%%%%%%
\DescribeMacro{\childdocof}
Furthermore, add the commands
\begin{center}
\begin{tabular}{l}
|\input{childdoc.def}|\\
|\childdocof{|\textit{main}|}|\\
\end{tabular}
\end{center}
at the top of every child file \textit{child}
which is included by |\include{|\textit{child}|}|
from within the main file
(or at least for those files to be compiled individually).
The argument \textit{main} must be the filename of the main file.

There are a couple of
considerations in setting up the main and child documents:

%%%%%%%%%%%%%%%%%%%%%%%%%%%%%%%%%%%%%%%%
\paragraph{Restrictions.}

Please note the following restrictions:
\begin{itemize}
\item
|\childdocmain| must be called with one argument \textit{main}
to ensure compatibility with earlier version of the package.
It must either be empty (|\childdocmain{}|)
or precisely match the filename of the main file in which it is specified.
See \secref{sec:detection} for further information.
\item
The filename \textit{main} must be specified without the |.tex| extension.
\item
The filename \textit{main} is case sensitive
(even in case-insensitive file systems)
due to internal string comparison.
\item
The argument \textit{main} should be fully expanded, it cannot be a macro.
\item
Subdirectories and special characters should be avoided in filenames.
\item
The command |\childdocmain{|\textit{main}|}| must be followed by a whitespace.
It should not be followed immediately by another command
or by a comment mark `|%|'.
This is because the \TeX{} parser reads the token immediately following
the argument of |\childdocmain| and puts it
at the beginning of every child section;
however, a white\-space is ignored.
\end{itemize}

%%%%%%%%%%%%%%%%%%%%%%%%%%%%%%%%%%%%%%%%
\paragraph{Content of Main File.}

It is advisable to place all content in the child files included by |\include|.
Any output contained in the main file will appear in all child documents
unless suppressed manually;
it cannot be suppressed automatically by the |\includeonly| directive
and thus should normally be avoided.
A method to include some content in the main file
by means of conditional processing is described in \secref{sec:conditional}.

%%%%%%%%%%%%%%%%%%%%%%%%%%%%%%%%%%%%%%%%
\paragraph{Page Numbering.}

When only a part of the document is compiled,
the appropriate numbering of pages
(as well as other status parameters)
is determined from the |.aux| files.
The latter contain information from previous passes.
However this information needs to propagate through
all intermediate child documents.
Therefore the page numbering in child documents may well
be inconsistent until the complete document is compiled at least once.

A useful (if unconventional) way to always ensure a consistent
page numbering is to restart the numbering in each child document
and denote the pages by `\textit{child}|.|\textit{page}'
where \textit{child} represents the chapter/section number of the child file.
This can be achieved by the command
|\numberwithin{page}{|\textit{child}|}|
of the \textsf{amsmath} package
where \textit{child} can be |chapter| or |section|
depending on the chosen structuring.
Alternatively, one can modify the macro |\thepage| appropriately
and reset the counter |page| at the start of each child file.

%%%%%%%%%%%%%%%%%%%%%%%%%%%%%%%%%%%%%%%%%%%%%%%%%%%%%%%%%%%%%%%%%%%%%%%%%%%%%%%%
\subsection{Conditional Processing}
\label{sec:conditional}

The package provides a mechanism to compile different versions
of a document. To customise the versions further some conditional processing
can come in handy to distinguish which version is being compiled.
The package provides two macros to describe the compilation context:

%%%%%%%%%%%%%%%%%%%%%%%%%%%%%%%%%%%%%%%%
\DescribeMacro{\ifchilddoc}
The conditional |\ifchilddoc| distinguishes between the compilation of
child documents and the main document:
%
\begin{center}
|\ifchilddoc |\textit{child-code}| |[|\||else |\textit{main-code}]| \||fi|
\end{center}

%%%%%%%%%%%%%%%%%%%%%%%%%%%%%%%%%%%%%%%%
\DescribeMacro{\childdocname}
\DescribeMacro{\childdocjob}
The macro |\childdocname| contains the filename (without extension)
of the main or child file being processed.
Note that |\childdocjob| will always contain the name of the main file.

%%%%%%%%%%%%%%%%%%%%%%%%%%%%%%%%%%%%%%%%
\paragraph{Title Page.}

Conditional processing can be used to include a title or banner page
in the main document when proper precautions are taken.
Importantly, the code in the main file should ensure that the page counter
(as well as other status parameters which are stored in the |.aux| files)
takes the same value after the conditional processing.
Otherwise the page numbers may take divergent values
depending on which part is compiled.

For example, a title page could be declared by:
%
\begin{center}
\begin{tabular}{l}
|\ifchilddoc\||else|\\
|\addtocounter{page}{-1}|\\
\textit{code for title page}\\
|\newpage|\\
|\||fi|
\end{tabular}
\end{center}
%
A banner page for the child documents can be generated by:
%
\begin{center}
\begin{tabular}{l}
|\ifchilddoc|\\
|\addtocounter{page}{-1}|\\
\textit{code for banner page}\\
|\newpage|\\
|\||fi|
\end{tabular}
\end{center}
%
Here one could write a message such as:
\begin{center}
|This is the part \childdocname{} of \childdocjob{}.|
\end{center}

%%%%%%%%%%%%%%%%%%%%%%%%%%%%%%%%%%%%%%%%%%%%%%%%%%%%%%%%%%%%%%%%%%%%%%%%%%%%%%%%
\subsection{Flags}
\label{sec:flags}

The package makes it easy to generate different versions
of the main or child documents.
To this end compilation flags can be defined
and assigned different default values.
They will be particularly useful in conjunction
with the forwarding mechanism described in \secref{sec:forward}.

For example, it may be useful to have a flag |\version|
which can be set to |draft| or |final|.
The document source will contain some conditional code
depending on the value of |\version|.
Suppose further, the flag should default to |final| for the main file
and to |draft| for child files
which is a natural assignment for editing the document.
This is achieved by placing the following code
in the preamble of the main document
(below the |\childdocmain| directive):
%
\begin{center}
\begin{tabular}{l}
|\ifchilddoc|\\
|\providecommand{\version}{draft}|\\
|\||else|\\
|\providecommand{\version}{final}|\\
|\||fi|
\end{tabular}
\end{center}
%
The definition by |\providecommand| makes sure
that previous definitions are not overwritten.
Further statements |\providecommand{\version}{...}|
can thus be added before the above code to override it.

For the main file, one might add a line
(between |\childdocmain| and the above block)
%
\begin{center}
|%\ifchilddoc\||else\providecommand{\version}{draft}\||fi|
\end{center}
%
which can be uncommented to produce a draft version.
Likewise one can add a line to the very top of a child file
(above the |\childdocof{|\textit{main}|}| directive)
%
\begin{center}
|%\providecommand{\version}{final}|
\end{center}
%
which can be uncommented to produce the final version of this child document.

%%%%%%%%%%%%%%%%%%%%%%%%%%%%%%%%%%%%%%%%%%%%%%%%%%%%%%%%%%%%%%%%%%%%%%%%%%%%%%%%
\subsection{Forwarding}
\label{sec:forward}

Different versions of the main or child documents
using compilation flags as described in \secref{sec:flags}
can be (permanently) stored in different files
for convenient compilation, viewing and distribution.
To this end, the package defines a command
to pass on compilation to a different file:

%%%%%%%%%%%%%%%%%%%%%%%%%%%%%%%%%%%%%%%%
\DescribeMacro{\childdocforward}
The command |\childdocforward| redirects processing to
another source file:
%
\begin{center}
\begin{tabular}{l}
|\input{childdoc.def}|\\
|\childdocforward[|\textit{main}|]{|\textit{dest}|}|\\
\end{tabular}
\end{center}
%
The argument \textit{dest} is the destination file
(without extension).
It should be the main file or one of the child files.
Note that further \textsf{childdoc} directives
such as |\childdocof| and |\childdocforward|
in the indicated file will be processed in this form.
The optional argument \textit{main}
passes on directly to the main file \textit{main}
while pretending to compile the child \textit{dest}.
This form behaves as if \textit{dest}
issues |\childdocof{|\textit{main}|}| right away,
and no further \textsf{childdoc} directives will be processed.

%%%%%%%%%%%%%%%%%%%%%%%%%%%%%%%%%%%%%%%%
\DescribeMacro{\...prefix}
In the alternative form |\childdocforwardprefix|,
%
\begin{center}
\begin{tabular}{l}
|\input{childdoc.def}|\\
|\childdocforwardprefix[|\textit{main}|]{|\textit{prefix}|}{|\textit{dest}|}|
\end{tabular}
\end{center}
%
the destination file is determined by a pattern
depending on the current file:
To make this work, the current file must be called
`{\textit{prefix}\hspace{0.2em}\textit{suffix}}'
with \textit{prefix} matching precisely the argument.
Processing is then passed on to the file
`{\textit{dest}\hspace{0.2em}\textit{suffix}}'.
Surely, the same effect is achieved by
directly specifying the
argument `{\textit{dest}\hspace{0.2em}\textit{suffix}}'
in the first form.
However, that requires to set up a different file
for each child. With the alternative form of the command
all these files can have exactly the same content
which simplifies setting them up and maintaining them.

For example, the following file |draft.tex|
with a compilation flag |\version| as described in \secref{sec:flags}
compiles the main document as a draft:
%
\begin{center}
\begin{tabular}{l}
|\def\version{draft}|\\
|\input{childdoc.def}|\\
|\childdocforward{|\textit{main}|}|
\end{tabular}
\end{center}
%
Likewise, the following files |final|\textit{nn}|.tex|
compile the final version of the child document
|child|\textit{nn}|.tex|:
%
\begin{center}
\begin{tabular}{l}
|\def\version{final}|\\
|\input{childdoc.def}|\\
|\childdocforwardprefix{final}{child}|
\end{tabular}
\end{center}
%

Note that when several versions of a main file and/or of each child file
are to be generated, it may be convenient to set up a |Makefile| or
shell script to automatise the process.

%%%%%%%%%%%%%%%%%%%%%%%%%%%%%%%%%%%%%%%%%%%%%%%%%%%%%%%%%%%%%%%%%%%%%%%%%%%%%%%%
\subsection{Command Line Processing}
\label{sec:commandline}

The effect of redirection files can also be achieved by invoking
the \LaTeX{} compiler with a more elaborate command line.
Most conveniently this should be done as part
of a shell script or a |Makefile|.

When using \textsf{childdoc} in the main file, the following
command lines effectively perform a redirection
(note that depending on the shell being used,
backslashes may have to be doubled: `|\|' $\to$ `|\\|'):
%
\begin{center}
|... -jobname "|\textit{target}|" |\\|"|[\textit{flags}]%
|\input{childdoc.def}\childdocforward[|\textit{main}|]{|\textit{dest}|}"|
\end{center}
%
Here \textit{target} is the name of the output file,
\textit{main} is the name of the main file
and \textit{dest} is the name of the main or child file to be processed
(all filenames without extensions).
The optional argument \textit{main} can be omitted
if \textit{main} matches \textit{dest}.
Optionally, compilation \textit{flags} can be defined via |\def| commands.
This command line makes the \TeX{} engine believe
it is compiling the file \textit{target}
whose content is specified as the latter parameter.
The provided code then forwards the processing to
\textit{main} or \textit{dest} as described in \secref{sec:forward}.

%%%%%%%%%%%%%%%%%%%%%%%%%%%%%%%%%%%%%%%%%%%%%%%%%%%%%%%%%%%%%%%%%%%%%%%%%%%%%%%%
\subsection{Include by Input}
\label{sec:input}

Including child documents by |\include| has some restrictions by design.
Most notably, the content of a child document always occupies
its own set of pages; pages cannot be shared between child documents.
Usually, this behaviour makes perfect sense
because each child document contain an essential part of the document.
However, in some situations it may be desirable to compose
a document from a collection of parts
without having mandatory page breaks between then.
For this case, the package
provides a mechanism to include parts
by |\input| which can also be processed individually.
However, by construction this mechanism
requires manual handling of the content to be output.

%%%%%%%%%%%%%%%%%%%%%%%%%%%%%%%%%%%%%%%%
\DescribeMacro{\ifchilddocmanual}
The main file should be prepared as usual, see \secref{sec:include}.
However, the document body must make a distinction
between processing of an individual part and of the main document, e.g.:
%
\begin{center}
\begin{tabular}{l}
|\ifchilddocmanual|\\
|\input{\childdocname}|\\
|\||else|\\
\textit{document body with }|\input{|\textit{part}|}|\\
|\||fi|
\end{tabular}
\end{center}
%
The conditional |\ifchilddocmanual| is true whenever
a part to be included by |\input| is being compiled,
and the name of the part is stored in |\childdocname|.

%%%%%%%%%%%%%%%%%%%%%%%%%%%%%%%%%%%%%%%%
\DescribeMacro{\childdocby}
Each part to be included by |\input| should start with:
%
\begin{center}
\begin{tabular}{l}
|\input{childdoc.def}|\\
|\childdocby{|\textit{main}|}|\\
\end{tabular}
\end{center}
%
The directive |\childdocby| is similar to |\childdocof|
described in \secref{sec:include},
but the subsequent selection of content must be done manually.
To that end, both |\ifchilddoc| and |\ifchilddocmanual|
will be true upon processing of a part,
and the name of the part is stored in |\childdocname|.
Note that |\jobname| will be set to the filename of the current part
so that each part receives an individual |.aux| file
that does not interfere with the |.aux| file(s) of the main document.
This behaviour can be altered by the alternative form
|\childdocby[*]{|\textit{main}|}| (with a non-empty optional argument)
which uses the |.aux| file of the main document
by setting |\jobname| to \textit{main}.

%%%%%%%%%%%%%%%%%%%%%%%%%%%%%%%%%%%%%%%%%%%%%%%%%%%%%%%%%%%%%%%%%%%%%%%%%%%%%%%%
\subsection{Driver Development}
\label{sec:driver}

The \textsf{childdoc} mechanism can also be use for the development
of definition files such as \LaTeX{} styles or classes.
This case differs from the above setup with multiple parts
included by |\include| in that no |\includeonly| should be invoked.
This can be achieved by starting the include file
(before |\ProvidesPackage|) with:
%
\begin{center}
\begin{tabular}{l}
|\input{childdoc.def}|\\
|\childdocforward{|\textit{main}|}|\\
\end{tabular}
\end{center}
%
or alternatively with:
%
\begin{center}
\begin{tabular}{l}
|\input{childdoc.def}|\\
|\childdocby{|\textit{main}|}|\\
\end{tabular}
\end{center}
%
Both forms have slightly different effects as described above.
The main file is prepared as usual, see \secref{sec:include}.

%%%%%%%%%%%%%%%%%%%%%%%%%%%%%%%%%%%%%%%%%%%%%%%%%%%%%%%%%%%%%%%%%%%%%%%%%%%%%%%%
\subsection{Legacy Detection}
\label{sec:detection}

The directive |\childdocmain| in the main file can detect
whether the complete document or merely a child is to be compiled
even without using the directive |\childdocof|.
This method is deprecated because it is less robust
and there is no compelling reason to use it;
it is merely provided for backward compatibility
and it may be removed in future versions.

If the detection mechanism is to be used,
it is mandatory to correctly specify
the filename of the main file as the argument of |\childdocmain|:
%
\begin{center}
\begin{tabular}{l}
|\input{childdoc.def}|\\
|\childdocmain{|\textit{main}|}|\\
\end{tabular}
\end{center}
%
If |\jobname| does not match the argument \textit{main} of |\childdocmain|,
it is assumed that |\jobname| points to the child file to be compiled.
When using |\childdocmain| with the main file specified as argument,
it suffices to start a child file
with just |\input{|\textit{main}|}|
without loading of the package and using |\childdocof|.
If instead all processing is done
with the appropriate \textsf{childdoc} directives,
the argument of \textit{main} of |\childdocmain| can be empty.

An alternative version of the command line processing described
in \secref{sec:commandline} using the detection mechanism reads:
%
\begin{center}
|... -jobname "|\textit{target}|" "|[\textit{flags}]%
[|\def\jobname{|\textit{dest}|}|]|\input{|\textit{main}|}"|
\end{center}

%%%%%%%%%%%%%%%%%%%%%%%%%%%%%%%%%%%%%%%%%%%%%%%%%%%%%%%%%%%%%%%%%%%%%%%%%%%%%%%%
\subsection{Manual Code}
\label{sec:manual}

In case one cannot be certain whether the definitions file |childdoc.def|
is installed on the target \TeX{} distribution
and one prefers not to ship it,
it is conceivable to paste a few relevant commands into the sources.

To that end, drop all statements |\input{childdoc.def}|
and perform the replacements as outlined below.
Instead of |\childdocmain{|\textit{main}|}| add the following code
to the top of the main file:
%
\begin{center}
\begin{tabular}{l}
|\||ifdefined\childdocname\endinput\||fi\newif\ifchilddoc|\\
|\edef\childdocname{\scantokens\expandafter{\jobname\noexpand}}|\\
|\def\childdocmain{|\textit{main}|}\||ifx\childdocmain\childdocname\||else|\\
|\childdoctrue\includeonly{\childdocname}\let\jobname\childdocmain\||fi|\\
\end{tabular}
\end{center}
%
Instead of |\childdocof{|\textit{main}|}| just include the main file
at the top of each child file:
%
\begin{center}
|\input{|\textit{main}|}|
\end{center}
%
A simple redirection |\childdocforward{|\textit{dest}|}| is achieved by:
%
\begin{center}
|\def\jobname{|\textit{dest}|}\input{\jobname}|
\end{center}
%
The redirection with prefix
|\childdocforwardprefix[|\textit{prefix}|]{|\textit{dest}|}|
is accomplished by:
%
\begin{center}
\begin{tabular}{l}
|{\edef\jobname{\scantokens\expandafter{\jobname\noexpand}}|\\
|\def\redirectjob |\textit{prefix}|#1~~~{\gdef\jobname{|\textit{dest}|#1}}|\\
|\expandafter\redirectjob\jobname~~~}\input{\jobname}|
\end{tabular}
\end{center}

In an alternative approach,
child documents can be compiled by a specific command line
without additional code or specific definitions:
%
\begin{center}
|... -jobname "|\textit{target}|" "|[\textit{flags}]%
|\includeonly{|\textit{dest}|}\input{|\textit{main}|}"|
\end{center}
%

%%%%%%%%%%%%%%%%%%%%%%%%%%%%%%%%%%%%%%%%%%%%%%%%%%%%%%%%%%%%%%%%%%%%%%%%%%%%%%%%
%%%%%%%%%%%%%%%%%%%%%%%%%%%%%%%%%%%%%%%%%%%%%%%%%%%%%%%%%%%%%%%%%%%%%%%%%%%%%%%%
\section{Information}

%%%%%%%%%%%%%%%%%%%%%%%%%%%%%%%%%%%%%%%%%%%%%%%%%%%%%%%%%%%%%%%%%%%%%%%%%%%%%%%%
\subsection{Copyright}

Copyright \copyright{} 2017--2018 Niklas Beisert

This work may be distributed and/or modified under the
conditions of the \LaTeX{} Project Public License, either version 1.3
of this license or (at your option) any later version.
The latest version of this license is in
  \url{http://www.latex-project.org/lppl.txt}
and version 1.3 or later is part of all distributions of \LaTeX{}
version 2005/12/01 or later.

This work has the LPPL maintenance status `maintained'.

The Current Maintainer of this work is Niklas Beisert.

This work consists of the files |README.txt|, |childdoc.ins| and |childdoc.dtx|
as well as the derived files |childdoc.def|, |cdocsamp.tex|
with |cdocsch1.tex|, |cdocsch2.tex|, |cdocspt3.tex|, |cdocspt4.tex|,
|cdocsdrf.tex|, |cdocsfn1.tex|, |cdocsfn2.tex|
as well as |childdoc.pdf|.

%%%%%%%%%%%%%%%%%%%%%%%%%%%%%%%%%%%%%%%%%%%%%%%%%%%%%%%%%%%%%%%%%%%%%%%%%%%%%%%%
\subsection{Files and Installation}

The package consists of the files:
%
\begin{center}
\begin{tabular}{ll}
    |README.txt|   & readme file \\
    |childdoc.ins| & installation file \\
    |childdoc.dtx| & source file \\
    |childdoc.def| & definition file \\
    |cdocsamp.tex| & sample main file \\
    |cdocsch1.tex| & sample include file \\
    |cdocsch2.tex| & sample include file \\
    |cdocspt3.tex| & sample part file \\
    |cdocspt4.tex| & sample part file \\
    |cdocsdrf.tex| & sample redirection file \\
    |cdocsfn1.tex| & sample redirection file \\
    |cdocsfn2.tex| & sample redirection file \\
    |childdoc.pdf| & manual
\end{tabular}
\end{center}
%
The distribution consists of the files
|README.txt|, |childdoc.ins| and |childdoc.dtx|.
%
\begin{itemize}
\item
Run (pdf)\LaTeX{} on |childdoc.dtx|
to compile the manual |childdoc.pdf| (this file).
\item
Run \LaTeX{} on |childdoc.ins| to create the definitions file |childdoc.def|
and the sample |cdocsamp.tex| with include files
|cdocsch1.tex|, |cdocsch2.tex|, |cdocspt3.tex|, |cdocspt4.tex|,
|cdocsdrf.tex|, |cdocsfn1.tex|, |cdocsfn2.tex|.
Then copy the file |childdoc.def| to an appropriate directory of your \LaTeX{}
distribution, e.g.\ \textit{texmf-root}|/tex/latex/childdoc|.
\end{itemize}

%%%%%%%%%%%%%%%%%%%%%%%%%%%%%%%%%%%%%%%%%%%%%%%%%%%%%%%%%%%%%%%%%%%%%%%%%%%%%%%%
\subsection{Related CTAN Packages}

There are several other packages which offer a similar functionality:
%
\begin{itemize}
\item
The packages
\href{http://ctan.org/pkg/docmute}{\textsf{docmute}},
\href{http://ctan.org/pkg/includex}{\textsf{includex}} and
\href{http://ctan.org/pkg/standalone}{\textsf{standalone}}
provide commands to include only the document body of
a child file thus allowing both files to be compiled individually.
\item
The packages \href{http://ctan.org/pkg/subdocs}{\textsf{subdocs}}
and \href{http://ctan.org/pkg/subfiles}{\textsf{subfiles}}
provide structures in which the main and child documents can be
encapsulated and allowing them to be compiled individually.
The inclusion mechanism is different from the conventional |\include|.
\item
The package \href{http://ctan.org/pkg/combine}{\textsf{combine}}
is an elaborate solution to combine several documents into one.
\end{itemize}
%
See also the CTAN topic \href{http://ctan.org/topic/subdocs}{\textsf{subdocs}}
for further related packages.
The present package differs from the above solutions in that
a document structure constructed with the conventional |\include| mechanism
just needs two extra commands at the top of every file
such that all constituent files can be compiled individually.

%%%%%%%%%%%%%%%%%%%%%%%%%%%%%%%%%%%%%%%%%%%%%%%%%%%%%%%%%%%%%%%%%%%%%%%%%%%%%%%%
%\subsection{Feature Suggestions}
%
%The following is a list of features which may be useful for future
%versions of this package:
%%
%\begin{itemize}
%\item
%\ldots
%\end{itemize}

%%%%%%%%%%%%%%%%%%%%%%%%%%%%%%%%%%%%%%%%%%%%%%%%%%%%%%%%%%%%%%%%%%%%%%%%%%%%%%%%
\subsection{Revision History}

%%%%%%%%%%%%%%%%%%%%%%%%%%%%%%%%%%%%%%%%
\paragraph{v2.0:} 2018/12/30

\begin{itemize}
\item
immediate forward processing
\item
added |\childdocby| mechanism
\item
manual restructured
\end{itemize}

%%%%%%%%%%%%%%%%%%%%%%%%%%%%%%%%%%%%%%%%
\paragraph{v1.6:} 2018/01/17

\begin{itemize}
\item
application for development of include files
\item
corrections to manual
\end{itemize}

%%%%%%%%%%%%%%%%%%%%%%%%%%%%%%%%%%%%%%%%
\paragraph{v1.5:} 2017/05/21

\begin{itemize}
\item
more complete structuring introduced
\item
|\childdocof| introduced
\item
|\childdoc| renamed to |\childdocmain|
\item
|\childredirect| renamed to |\childdocforward| and |\childdocforwardprefix|
and functionality expanded
\end{itemize}

%%%%%%%%%%%%%%%%%%%%%%%%%%%%%%%%%%%%%%%%
\paragraph{v1.0:} 2017/04/27

\begin{itemize}
\item
manual and install package
\item
first version published on CTAN
\end{itemize}

%%%%%%%%%%%%%%%%%%%%%%%%%%%%%%%%%%%%%%%%
\paragraph{v0.6:} 2017/04/26

\begin{itemize}
\item
redirection mechanism added
\end{itemize}

%%%%%%%%%%%%%%%%%%%%%%%%%%%%%%%%%%%%%%%%
\paragraph{v0.5:} 2017/04/26

\begin{itemize}
\item
functionality in definition file
\end{itemize}


%%%%%%%%%%%%%%%%%%%%%%%%%%%%%%%%%%%%%%%%%%%%%%%%%%%%%%%%%%%%%%%%%%%%%%%%%%%%%%%%
%%%%%%%%%%%%%%%%%%%%%%%%%%%%%%%%%%%%%%%%%%%%%%%%%%%%%%%%%%%%%%%%%%%%%%%%%%%%%%%%
%%%%%%%%%%%%%%%%%%%%%%%%%%%%%%%%%%%%%%%%%%%%%%%%%%%%%%%%%%%%%%%%%%%%%%%%%%%%%%%%
\appendix

\settowidth\MacroIndent{\rmfamily\scriptsize 000\ }

 \DocInput{childdoc.dtx}

\end{document}
%</driver>
% \fi
%
% %%%%%%%%%%%%%%%%%%%%%%%%%%%%%%%%%%%%%%%%%%%%%%%%%%%%%%%%%%%%%%%%%%%%%%%%%%%%%%
% %%%%%%%%%%%%%%%%%%%%%%%%%%%%%%%%%%%%%%%%%%%%%%%%%%%%%%%%%%%%%%%%%%%%%%%%%%%%%%
% \section{Sample}
%\iffalse
%<*samplemain>
%\fi
%
% The following presents a sample document
% with two chapters, two parts, a title page,
% a compile flag as well as three forwarding files to set the flag.
% It consists of eight |.tex| files:
% \begin{center}
% \begin{tabular}{ll}
% |cdocsamp.tex|&main file\\
% |cdocsch1.tex|&include file for chapter 1\\
% |cdocsch2.tex|&include file for chapter 2\\
% |cdocspt3.tex|&include file for part 3\\
% |cdocspt4.tex|&include file for part 4\\
% |cdocsdrf.tex|&forwarding file for main file in draft mode\\
% |cdocsfi1.tex|&forwarding file for final version of chapter 1\\
% |cdocsfi2.tex|&forwarding file for final version of chapter 2\\
% \end{tabular}
% \end{center}
% Each of the eight files can be compiled directly by the \LaTeX{} compiler.
%
% %%%%%%%%%%%%%%%%%%%%%%%%%%%%%%%%%%%%%%
% \paragraph{Main File.}
%
% The main file is called |cdocsamp.tex|.
%
% Load the \textsf{childdoc} definitions and
% declare the filename for the main document:
%    \begin{macrocode}
\input{childdoc.def}
\childdocmain{}
%    \end{macrocode}

% Optional override for |\version| flag:
%    \begin{macrocode}
%%\ifchilddoc\else\providecommand{\version}{draft}\fi
%    \end{macrocode}

% Define the default values for the |\version| flag
% (|final| for the main file and |draft| for childs):
%    \begin{macrocode}
\ifchilddoc
\providecommand{\version}{draft}
\else
\providecommand{\version}{final}
\fi
%    \end{macrocode}

% Load the standard document class:
%    \begin{macrocode}
\documentclass[12pt]{article}
%    \end{macrocode}

% Start the document body:
%    \begin{macrocode}
\begin{document}
%    \end{macrocode}

% Declare a title page.
% Print title, part of document being processed and version flag:
%    \begin{macrocode}
\addtocounter{page}{-1}
\begin{center}
{\LARGE\bfseries{}childdoc example\par}
\vspace{1cm}
\ifchilddoc
\ifchilddocmanual part\else chapter\fi:
`\childdocname' of `\childdocjob'\par
\else
main document: `\childdocjob'\par
\fi
version: \version\par
\end{center}
\newpage
%    \end{macrocode}

% Manually include selected file,
% otherwise process as usual:
%    \begin{macrocode}
\ifchilddocmanual
\section*{part `\childdocname'}
\input{\childdocname}
\else
%    \end{macrocode}

% Include the two chapters:
%    \begin{macrocode}
\include{cdocsch1}
\include{cdocsch2}
%    \end{macrocode}

% Include the two parts unless only chapters should be displayed:
%    \begin{macrocode}
\ifchilddoc\else
\section{part three}
\input{cdocspt3}
\section{part four}
\input{cdocspt4}
\fi
%    \end{macrocode}

% Process as usual until here:
%    \begin{macrocode}
\fi
%    \end{macrocode}

% End of document body:
%    \begin{macrocode}
\end{document}
%    \end{macrocode}
%\iffalse
%</samplemain>
%\fi
%
% %%%%%%%%%%%%%%%%%%%%%%%%%%%%%%%%%%%%%%
% \paragraph{Chapter Include Files.}
%
% The include files are called |cdocsch1.tex| and |cdocsch2.tex|.
%
%\iffalse
%<*samplechap1|samplechap2>
%\fi

% Optional override for |\version| flag:
%    \begin{macrocode}
%%\providecommand{\version}{final}
%    \end{macrocode}

% Include the main document:
%    \begin{macrocode}
\input{childdoc.def}
\childdocof{cdocsamp}
%    \end{macrocode}

%\iffalse
%</samplechap1|samplechap2>
%\fi
%
%\iffalse
%<*samplechap1>
%\fi
% Some text for chapter 1:
%    \begin{macrocode}
\section{one}
some text in chapter one
%    \end{macrocode}

%\iffalse
%</samplechap1>
%\fi
% Some text for chapter 2:
%\iffalse
%<*samplechap2>
%\fi
%    \begin{macrocode}
\section{two}
more text in chapter two
%    \end{macrocode}

%\iffalse
%</samplechap2>
%\fi
%
% %%%%%%%%%%%%%%%%%%%%%%%%%%%%%%%%%%%%%%
% \paragraph{Part Include Files.}
%
% The include files are called |cdocspt3.tex| and |cdocspt4.tex|.
%
%\iffalse
%<*samplepart3|samplepart4>
%\fi

% Optional override for |\version| flag:
%    \begin{macrocode}
%%\providecommand{\version}{final}
%    \end{macrocode}

% Include the main document:
%    \begin{macrocode}
\input{childdoc.def}
\childdocby{cdocsamp}
%    \end{macrocode}

%\iffalse
%</samplepart3|samplepart4>
%\fi
%
%\iffalse
%<*samplepart3>
%\fi
% Some text for part 3:
%    \begin{macrocode}
some text in part three
%    \end{macrocode}

%\iffalse
%</samplepart3>
%\fi
% Some text for part 4:
%\iffalse
%<*samplepart4>
%\fi
%    \begin{macrocode}
more text in part four
%    \end{macrocode}

%\iffalse
%</samplepart4>
%\fi
%
% %%%%%%%%%%%%%%%%%%%%%%%%%%%%%%%%%%%%%%
% \paragraph{Forwarding for a Complete Draft.}
%
% The following forwarding file |cdocsdrf.tex|
% compiles the main document in draft mode:
%\iffalse
%<*sampledraft>
%\fi
%    \begin{macrocode}
\def\version{draft}
\input{childdoc.def}
\childdocforward{cdocsamp}
%    \end{macrocode}

%\iffalse
%</sampledraft>
%\fi
%
% %%%%%%%%%%%%%%%%%%%%%%%%%%%%%%%%%%%%%%
% \paragraph{Forwarding for Final Version of the Chapters.}
%
% The following forwarding files |cdocsfn1.tex| and |cdocsfn2.tex|
% (with identical content)
% compile the final versions of the child documents
% |cdocsch1.tex| and |cdocsch2.tex|, respectively:
%\iffalse
%<*samplefinal>
%\fi
%    \begin{macrocode}
\def\version{final}
\input{childdoc.def}
\childdocforwardprefix[cdocsamp]{cdocsfn}{cdocsch}
%    \end{macrocode}

%\iffalse
%</samplefinal>
%\fi
%
% %%%%%%%%%%%%%%%%%%%%%%%%%%%%%%%%%%%%%%
% \paragraph{Command Line Processing.}
%
% The following three command lines generate the output files
% |cdocscld|, |cdocscl1| and |cdocscl2|
% which should be identical to
% |cdocsdrf|, |cdocsch1| and |cdocsfn2|, respectively:
% \begin{center}
% \begin{tabular}{l}
% |latex -jobname cdocscld \|\\
% |  "\def\version{draft}\input{childdoc.def}\childdocforward{cdocsamp}"|\\
% |latex -jobname cdocscl1 \|\\
% |  "\input{childdoc.def}\childdocforward[cdocsamp]{cdocsch1}"|\\
% |latex -jobname cdocscl2 \|\\
% |  "\def\version{final}\input{childdoc.def}\childdocforward{cdocsch2}"|
% \end{tabular}
% \end{center}
% Note that the trailing backslash on each first line
% merely continues the input to the second line
% (for convenient cut ant paste).
% Furthermore, the command |latex| can be replaced by any
% of its alternative versions such as |pdflatex|.
%
% %%%%%%%%%%%%%%%%%%%%%%%%%%%%%%%%%%%%%%%%%%%%%%%%%%%%%%%%%%%%%%%%%%%%%%%%%%%%%%
% %%%%%%%%%%%%%%%%%%%%%%%%%%%%%%%%%%%%%%%%%%%%%%%%%%%%%%%%%%%%%%%%%%%%%%%%%%%%%%
% \section{Implementation}
%\iffalse
%<*package>
%\fi
%
% This section describes the definitions file |childdoc.def|.

% The definitions cannot be loaded using |\usepackage| or |\RequirePackage|
% which has a mechanism to prevent loading a style file more than once.
% When loading the definitions by means of |\input|
% multiple instances have to be prevented manually:
%\iffalse
%This code needs to be before the `\ProvidesFile' directive
%which is defined at the beginning of this file.
%Therefore it is also placed there and commented out here.
%</package>
%<*discard>
%\fi
%    \begin{macrocode}
\ifdefined\childdocmain\endinput\fi
%    \end{macrocode}
%\iffalse
%</discard>
%<*package>
%\fi
%
% \macro{\ifchilddoc}
% \macro{\ifchilddocmanual}
% The conditional |\ifchilddoc| tells whether a
% child (true) or main (false) document is being compiled.
% The conditional |\ifchilddocmanual| tells whether
% the |\includeonly| mechanism is used (false) or
% the selection of child files must be performed manually (true).
% The definitions initialise to false:
%    \begin{macrocode}
\newif\ifchilddoc
\newif\ifchilddocmanual
%    \end{macrocode}

% \macro{\childdocname}
% \macro{\childdocjob}
% The macro |\childdocname| stores the name of the main document
% to be compiled. The macro |\childdocjob| stores the name of
% the document on which the \LaTeX{} compiler was originally invoked.
% The content of |\jobname| cannot be compared
% to filenames specified in the source due to different catcodes.
% The following code rescans |\jobname|, stores the result
% in |\childdocname| and saves a copy in |\childdocjob|:
%    \begin{macrocode}
\edef\childdocname{\scantokens\expandafter{\jobname\noexpand}}
\let\childdocjob\childdocname
%    \end{macrocode}

% \macro{\childdocdisable}
% The macro |\childdocdisable| prevents the main file
% from being processed more than once.
% At this stage, the main document command |\childdocmain|
% is assumed to be called once again where it should do nothing.
% Any subsequent call to it should prevent
% a secondary processing of the main document
% It overwrites the forwarding commands
% |\childdocof| and |\childdocforward|
% with empty macros to prevent further inclusions of the main document:
%    \begin{macrocode}
\newcommand{\childdocdisable}
{
  \renewcommand{\childdocmain}[1]{\renewcommand{\childdocmain}[1]{\endinput}}
  \renewcommand{\childdocof}[1]{}
  \renewcommand{\childdocby}[2][]{}
  \renewcommand{\childdocforward}[2][]{}
  \renewcommand{\childdocdisable}{}
}
%    \end{macrocode}

% \macro{\childdocmain}
% The macro |\childdocmain| is to be called at the top of the main file
% with nothing or the main filename (without extension) as argument.
% First, it breaks loops.
% If the argument is not empty and does not match |\childdocname|
% (which is set by the first inclusion of |childdoc.def|),
% |\ifchilddoc| is set to true, |\includeonly| is applied to the child file
% and |\jobname| is set to the main file
% (for proper handling of |.aux| files):
%    \begin{macrocode}
\newcommand{\childdocmain}[1]
{
  \childdocdisable\childdocmain{}
  \if?#1?\else
    \begingroup
      \def\childdoctmp{#1}
      \ifx\childdoctmp\childdocname
        \def\childdoctmp{}
      \else
        \def\childdoctmp
        {
          \childdoctrue
          \includeonly{\childdocname}
          \def\childdocjob{#1}
          \def\jobname{#1}
        }
      \fi
      \expandafter
    \endgroup
    \childdoctmp
  \fi
}
%    \end{macrocode}

% \macro{\childdocof}
% The command |\childdocof| redirects
% compilation to the main file |#1|.
%    \begin{macrocode}
\newcommand{\childdocof}[1]
{
  \childdocdisable
  \childdoctrue
  \includeonly{\childdocname}
  \def\jobname{#1}
  \def\childdocjob{#1}
  \input{#1}
}
%    \end{macrocode}

% \macro{\childdocby}
% The command |\childdocby| ....
%    \begin{macrocode}
\newcommand{\childdocby}[2][]
{
  \childdocdisable
  \childdoctrue
  \childdocmanualtrue
  \if?#1?\else
    \def\jobname{#2}
  \fi
  \def\childdocjob{#2}
  \input{#2}
  \endinput
}
%    \end{macrocode}

% \macro{\childdocforward}
% The command |\childdocforward| redirects
% compilation to the main file or
% (if the optional argument is given) a child file.
% Parameters are set as if the main file
% or a child file starting with |\childdocof| was compiled.
% Then compilation is handed over to the main file:
%    \begin{macrocode}
\newcommand{\childdocforward}[2][]
{
  \begingroup
    \if?#1?
      \def\childdoctmp
      {
        \def\childdocname{#2}
        \def\childdocjob{#2}
        \def\jobname{#2}
        \input{#2}
        \endinput
      }
    \else
      \def\childdoctmp
      {
        \childdocdisable
        \def\childdocname{#2}
        \childdoctrue
        \includeonly{#2}
        \def\childdocjob{#1}
        \def\jobname{#1}
        \input{#1}
        \endinput
      }
    \fi
    \expandafter
  \endgroup
  \childdoctmp
}
%    \end{macrocode}

% \macro{\childdocforwardprefix}
% The command |\childdocforwardprefix| redirects
% compilation to the main or a child file by means of a pattern.
% The prefix |#1| in the current filename is replaced by |#2|
% and the suffix of the current filename is kept
% (it is assumed that the filename does not contain the substring `|~~~|'
% which is used as a delimiter).
% Compilation is handed over to the new file by |\childdocforward|:
%    \begin{macrocode}
\newcommand{\childdocforwardprefix}[3][]
{
  \begingroup
    \def\childdocextract #2##1~~~{\def\childdoctmp{\childdocforward[#1]{#3##1}}}
    \expandafter\childdocextract\childdocname~~~
    \expandafter
  \endgroup
  \childdoctmp
}
%    \end{macrocode}

% \macro{\childdoc}
% The deprecated macro |\childdoc| is a legacy version of |\childdocmain|:
%    \begin{macrocode}
\newcommand{\childdoc}{\childdocmain}
%    \end{macrocode}

% \macro{\childdocredirect}
% The deprecated macro |\childdocredirect| is a legacy version
% of |\childdocforward| and |\childdocforwardprefix|:
%    \begin{macrocode}
\newcommand{\childdocredirect}[2][]
{
  \begingroup
    \if?#1?
      \def\childdoctmp{\childdocforward{#2}}
    \else
      \def\childdoctmp{\childdocforwardprefix{#1}{#2}}
    \fi
    \expandafter
  \endgroup
  \childdoctmp
}
%    \end{macrocode}

%\iffalse
%</package>
%\fi
%
\endinput

\childdocmain{}
%    \end{macrocode}

% Optional override for |\version| flag:
%    \begin{macrocode}
%%\ifchilddoc\else\providecommand{\version}{draft}\fi
%    \end{macrocode}

% Define the default values for the |\version| flag
% (|final| for the main file and |draft| for childs):
%    \begin{macrocode}
\ifchilddoc
\providecommand{\version}{draft}
\else
\providecommand{\version}{final}
\fi
%    \end{macrocode}

% Load the standard document class:
%    \begin{macrocode}
\documentclass[12pt]{article}
%    \end{macrocode}

% Start the document body:
%    \begin{macrocode}
\begin{document}
%    \end{macrocode}

% Declare a title page.
% Print title, part of document being processed and version flag:
%    \begin{macrocode}
\addtocounter{page}{-1}
\begin{center}
{\LARGE\bfseries{}childdoc example\par}
\vspace{1cm}
\ifchilddoc
\ifchilddocmanual part\else chapter\fi:
`\childdocname' of `\childdocjob'\par
\else
main document: `\childdocjob'\par
\fi
version: \version\par
\end{center}
\newpage
%    \end{macrocode}

% Manually include selected file,
% otherwise process as usual:
%    \begin{macrocode}
\ifchilddocmanual
\section*{part `\childdocname'}
\input{\childdocname}
\else
%    \end{macrocode}

% Include the two chapters:
%    \begin{macrocode}
\include{cdocsch1}
\include{cdocsch2}
%    \end{macrocode}

% Include the two parts unless only chapters should be displayed:
%    \begin{macrocode}
\ifchilddoc\else
\section{part three}
\input{cdocspt3}
\section{part four}
\input{cdocspt4}
\fi
%    \end{macrocode}

% Process as usual until here:
%    \begin{macrocode}
\fi
%    \end{macrocode}

% End of document body:
%    \begin{macrocode}
\end{document}
%    \end{macrocode}
%\iffalse
%</samplemain>
%\fi
%
% %%%%%%%%%%%%%%%%%%%%%%%%%%%%%%%%%%%%%%
% \paragraph{Chapter Include Files.}
%
% The include files are called |cdocsch1.tex| and |cdocsch2.tex|.
%
%\iffalse
%<*samplechap1|samplechap2>
%\fi

% Optional override for |\version| flag:
%    \begin{macrocode}
%%\providecommand{\version}{final}
%    \end{macrocode}

% Include the main document:
%    \begin{macrocode}
% \iffalse
%
% childdoc.dtx Copyright (C) 2017-2018 Niklas Beisert
%
% This work may be distributed and/or modified under the
% conditions of the LaTeX Project Public License, either version 1.3
% of this license or (at your option) any later version.
% The latest version of this license is in
%   http://www.latex-project.org/lppl.txt
% and version 1.3 or later is part of all distributions of LaTeX
% version 2005/12/01 or later.
%
% This work has the LPPL maintenance status `maintained'.
%
% The Current Maintainer of this work is Niklas Beisert.
%
% This work consists of the files childdoc.dtx and childdoc.ins
% and the derived files childdoc.def and cdocsamp.tex with
% cdocsch1.tex, cdocsch2.tex, cdocsdrf.tex, cdocsfn1.tex, cdocsfn2.tex.
%
%<package>\ifdefined\childdocmain\endinput\fi
%<package>\ProvidesFile{childdoc.def}[2018/12/30 v2.0 child document driver]
%<samplemain>\ProvidesFile{cdocsamp.tex}[2018/12/30 v2.0 sample for childdoc]
%<*driver>
%\ProvidesFile{childdoc.drv}[2018/12/30 v2.0 childdoc reference manual file]
\PassOptionsToClass{10pt,a4paper}{article}
\documentclass{ltxdoc}

\usepackage[margin=35mm]{geometry}
\usepackage{hyperref}
\usepackage{hyperxmp}
\usepackage[usenames]{color}

\hypersetup{colorlinks=true}
\hypersetup{pdfstartview=FitH}
\hypersetup{pdfpagemode=UseNone}
\hypersetup{pdfsource={}}
\hypersetup{pdflang={en-UK}}
\hypersetup{pdfcopyright={Copyright 2017-2018 Niklas Beisert.
  This work may be distributed and/or modified under the
  conditions of the LaTeX Project Public License, either version 1.3
  of this license or (at your option) any later version.}}
\hypersetup{pdflicenseurl={http://www.latex-project.org/lppl.txt}}
\hypersetup{pdfcontactaddress={ETH Zurich, ITP, HIT K,
  Wolfgang-Pauli-Strasse 27}}
\hypersetup{pdfcontactpostcode={8093}}
\hypersetup{pdfcontactcity={Zurich}}
\hypersetup{pdfcontactcountry={Switzerland}}
\hypersetup{pdfcontactemail={nbeisert@itp.phys.ethz.ch}}
\hypersetup{pdfcontacturl={http://people.phys.ethz.ch/\xmptilde nbeisert/}}

\newcommand{\secref}[1]{\hyperref[#1]{section \ref*{#1}}}

\parskip1ex
\parindent0pt
\let\olditemize\itemize
\def\itemize{\olditemize\parskip0pt}

\begin{document}

\title{The \textsf{childdoc} Package}
\hypersetup{pdftitle={The childdoc Package}}
\author{Niklas Beisert\\[2ex]
  Institut f\"ur Theoretische Physik\\
  Eidgen\"ossische Technische Hochschule Z\"urich\\
  Wolfgang-Pauli-Strasse 27, 8093 Z\"urich, Switzerland\\[1ex]
  \href{mailto:nbeisert@itp.phys.ethz.ch}
  {\texttt{nbeisert@itp.phys.ethz.ch}}}
\hypersetup{pdfauthor={Niklas Beisert}}
\hypersetup{pdfsubject={Manual for the LaTeX2e Package childdoc}}
\date{30 December 2018, \textsf{v2.0}}
\maketitle

\begin{abstract}\noindent
\textsf{childdoc} is a \LaTeXe{} package
that enables the direct compilation
of document sections included by |\include|
to individual files.
\end{abstract}

\begingroup
\parskip0ex
\tableofcontents
\endgroup

%%%%%%%%%%%%%%%%%%%%%%%%%%%%%%%%%%%%%%%%%%%%%%%%%%%%%%%%%%%%%%%%%%%%%%%%%%%%%%%%
%%%%%%%%%%%%%%%%%%%%%%%%%%%%%%%%%%%%%%%%%%%%%%%%%%%%%%%%%%%%%%%%%%%%%%%%%%%%%%%%
\section{Introduction}

\LaTeX{} provides a mechanism to structure a large document (such as a book)
into a main file and several child files (containing the chapters)
using the |\include| command.
This mechanism is beneficial for documents
which span hundreds of pages in order to
make the source file(s) more manageable.
Moreover, compilation can be restricted to
selected child files by means of the |\includeonly| command.
The latter feature can be used to reduce the compilation time while editing
(this was significantly more useful in the earlier days of \LaTeX{})
or to generate a smaller document which is easier to navigate.
Another application of |\includeonly| is to generate
documents consisting of selected parts of the complete document.

However, there are a few drawbacks of the plain |\include| mechanism:
\begin{itemize}
\item
The child files cannot be compiled on their own,
they can only be compiled via the main file.
A naive editing environment
(such as a text editor with an option
to have the current file processed by \LaTeX)
may require one to switch to the main file before compiling;
attempting to compile the child file produces errors.
\item
The main file must be modified (each time)
to adjust the |\includeonly| command
to the present needs. This easily leaves the main file in a messy state.
\item
The generated document will always carry the filename
of the main document. This is inconvenient if
several child files are to be compiled and
to be kept for distribution.
\end{itemize}

The present package provides a simple interface
to make child files individually compilable by \LaTeX{}.
Compiling a child file then has the same effect as compiling
the main file with an |\includeonly| command
to select the appropriate child.
Moreover the generated document will carry the name of the child
rather than the main file.
This resolves all three above issues.

This feature is meant to make the editing of books,
thesis documents and lecture notes somewhat more convenient.
However, the package can also be used efficiently for
composing a series of documents (such as exercise sheets)
which are typically distributed individually.
It then assists the author in generating the individual documents
(potentially in different versions)
as well as a document containing the collected series.
Another application is in developing style files
or other kinds of included material
where compilation of the style file could redirect
to a sample or test file.

%%%%%%%%%%%%%%%%%%%%%%%%%%%%%%%%%%%%%%%%%%%%%%%%%%%%%%%%%%%%%%%%%%%%%%%%%%%%%%%%
%%%%%%%%%%%%%%%%%%%%%%%%%%%%%%%%%%%%%%%%%%%%%%%%%%%%%%%%%%%%%%%%%%%%%%%%%%%%%%%%
\section{Usage}

First of all, the package \textsf{childdoc} is \emph{not} a standard
\LaTeXe{} |.sty| style file! Therefore it needs to be invoked in
a non-standard way.

%%%%%%%%%%%%%%%%%%%%%%%%%%%%%%%%%%%%%%%%%%%%%%%%%%%%%%%%%%%%%%%%%%%%%%%%%%%%%%%%
\subsection{Included Files}
\label{sec:include}

%%%%%%%%%%%%%%%%%%%%%%%%%%%%%%%%%%%%%%%%
\DescribeMacro{\childdocmain}
To use the package, add the commands
\begin{center}
\begin{tabular}{l}
|\input{childdoc.def}|\\
|\childdocmain{}|\\
\end{tabular}
\end{center}
at the very top of the main \LaTeX{} file,
in particular \emph{before} the |\documentclass| statement!
The argument of |\childdocmain| should be left empty
(but it must be present).

%%%%%%%%%%%%%%%%%%%%%%%%%%%%%%%%%%%%%%%%
\DescribeMacro{\childdocof}
Furthermore, add the commands
\begin{center}
\begin{tabular}{l}
|\input{childdoc.def}|\\
|\childdocof{|\textit{main}|}|\\
\end{tabular}
\end{center}
at the top of every child file \textit{child}
which is included by |\include{|\textit{child}|}|
from within the main file
(or at least for those files to be compiled individually).
The argument \textit{main} must be the filename of the main file.

There are a couple of
considerations in setting up the main and child documents:

%%%%%%%%%%%%%%%%%%%%%%%%%%%%%%%%%%%%%%%%
\paragraph{Restrictions.}

Please note the following restrictions:
\begin{itemize}
\item
|\childdocmain| must be called with one argument \textit{main}
to ensure compatibility with earlier version of the package.
It must either be empty (|\childdocmain{}|)
or precisely match the filename of the main file in which it is specified.
See \secref{sec:detection} for further information.
\item
The filename \textit{main} must be specified without the |.tex| extension.
\item
The filename \textit{main} is case sensitive
(even in case-insensitive file systems)
due to internal string comparison.
\item
The argument \textit{main} should be fully expanded, it cannot be a macro.
\item
Subdirectories and special characters should be avoided in filenames.
\item
The command |\childdocmain{|\textit{main}|}| must be followed by a whitespace.
It should not be followed immediately by another command
or by a comment mark `|%|'.
This is because the \TeX{} parser reads the token immediately following
the argument of |\childdocmain| and puts it
at the beginning of every child section;
however, a white\-space is ignored.
\end{itemize}

%%%%%%%%%%%%%%%%%%%%%%%%%%%%%%%%%%%%%%%%
\paragraph{Content of Main File.}

It is advisable to place all content in the child files included by |\include|.
Any output contained in the main file will appear in all child documents
unless suppressed manually;
it cannot be suppressed automatically by the |\includeonly| directive
and thus should normally be avoided.
A method to include some content in the main file
by means of conditional processing is described in \secref{sec:conditional}.

%%%%%%%%%%%%%%%%%%%%%%%%%%%%%%%%%%%%%%%%
\paragraph{Page Numbering.}

When only a part of the document is compiled,
the appropriate numbering of pages
(as well as other status parameters)
is determined from the |.aux| files.
The latter contain information from previous passes.
However this information needs to propagate through
all intermediate child documents.
Therefore the page numbering in child documents may well
be inconsistent until the complete document is compiled at least once.

A useful (if unconventional) way to always ensure a consistent
page numbering is to restart the numbering in each child document
and denote the pages by `\textit{child}|.|\textit{page}'
where \textit{child} represents the chapter/section number of the child file.
This can be achieved by the command
|\numberwithin{page}{|\textit{child}|}|
of the \textsf{amsmath} package
where \textit{child} can be |chapter| or |section|
depending on the chosen structuring.
Alternatively, one can modify the macro |\thepage| appropriately
and reset the counter |page| at the start of each child file.

%%%%%%%%%%%%%%%%%%%%%%%%%%%%%%%%%%%%%%%%%%%%%%%%%%%%%%%%%%%%%%%%%%%%%%%%%%%%%%%%
\subsection{Conditional Processing}
\label{sec:conditional}

The package provides a mechanism to compile different versions
of a document. To customise the versions further some conditional processing
can come in handy to distinguish which version is being compiled.
The package provides two macros to describe the compilation context:

%%%%%%%%%%%%%%%%%%%%%%%%%%%%%%%%%%%%%%%%
\DescribeMacro{\ifchilddoc}
The conditional |\ifchilddoc| distinguishes between the compilation of
child documents and the main document:
%
\begin{center}
|\ifchilddoc |\textit{child-code}| |[|\||else |\textit{main-code}]| \||fi|
\end{center}

%%%%%%%%%%%%%%%%%%%%%%%%%%%%%%%%%%%%%%%%
\DescribeMacro{\childdocname}
\DescribeMacro{\childdocjob}
The macro |\childdocname| contains the filename (without extension)
of the main or child file being processed.
Note that |\childdocjob| will always contain the name of the main file.

%%%%%%%%%%%%%%%%%%%%%%%%%%%%%%%%%%%%%%%%
\paragraph{Title Page.}

Conditional processing can be used to include a title or banner page
in the main document when proper precautions are taken.
Importantly, the code in the main file should ensure that the page counter
(as well as other status parameters which are stored in the |.aux| files)
takes the same value after the conditional processing.
Otherwise the page numbers may take divergent values
depending on which part is compiled.

For example, a title page could be declared by:
%
\begin{center}
\begin{tabular}{l}
|\ifchilddoc\||else|\\
|\addtocounter{page}{-1}|\\
\textit{code for title page}\\
|\newpage|\\
|\||fi|
\end{tabular}
\end{center}
%
A banner page for the child documents can be generated by:
%
\begin{center}
\begin{tabular}{l}
|\ifchilddoc|\\
|\addtocounter{page}{-1}|\\
\textit{code for banner page}\\
|\newpage|\\
|\||fi|
\end{tabular}
\end{center}
%
Here one could write a message such as:
\begin{center}
|This is the part \childdocname{} of \childdocjob{}.|
\end{center}

%%%%%%%%%%%%%%%%%%%%%%%%%%%%%%%%%%%%%%%%%%%%%%%%%%%%%%%%%%%%%%%%%%%%%%%%%%%%%%%%
\subsection{Flags}
\label{sec:flags}

The package makes it easy to generate different versions
of the main or child documents.
To this end compilation flags can be defined
and assigned different default values.
They will be particularly useful in conjunction
with the forwarding mechanism described in \secref{sec:forward}.

For example, it may be useful to have a flag |\version|
which can be set to |draft| or |final|.
The document source will contain some conditional code
depending on the value of |\version|.
Suppose further, the flag should default to |final| for the main file
and to |draft| for child files
which is a natural assignment for editing the document.
This is achieved by placing the following code
in the preamble of the main document
(below the |\childdocmain| directive):
%
\begin{center}
\begin{tabular}{l}
|\ifchilddoc|\\
|\providecommand{\version}{draft}|\\
|\||else|\\
|\providecommand{\version}{final}|\\
|\||fi|
\end{tabular}
\end{center}
%
The definition by |\providecommand| makes sure
that previous definitions are not overwritten.
Further statements |\providecommand{\version}{...}|
can thus be added before the above code to override it.

For the main file, one might add a line
(between |\childdocmain| and the above block)
%
\begin{center}
|%\ifchilddoc\||else\providecommand{\version}{draft}\||fi|
\end{center}
%
which can be uncommented to produce a draft version.
Likewise one can add a line to the very top of a child file
(above the |\childdocof{|\textit{main}|}| directive)
%
\begin{center}
|%\providecommand{\version}{final}|
\end{center}
%
which can be uncommented to produce the final version of this child document.

%%%%%%%%%%%%%%%%%%%%%%%%%%%%%%%%%%%%%%%%%%%%%%%%%%%%%%%%%%%%%%%%%%%%%%%%%%%%%%%%
\subsection{Forwarding}
\label{sec:forward}

Different versions of the main or child documents
using compilation flags as described in \secref{sec:flags}
can be (permanently) stored in different files
for convenient compilation, viewing and distribution.
To this end, the package defines a command
to pass on compilation to a different file:

%%%%%%%%%%%%%%%%%%%%%%%%%%%%%%%%%%%%%%%%
\DescribeMacro{\childdocforward}
The command |\childdocforward| redirects processing to
another source file:
%
\begin{center}
\begin{tabular}{l}
|\input{childdoc.def}|\\
|\childdocforward[|\textit{main}|]{|\textit{dest}|}|\\
\end{tabular}
\end{center}
%
The argument \textit{dest} is the destination file
(without extension).
It should be the main file or one of the child files.
Note that further \textsf{childdoc} directives
such as |\childdocof| and |\childdocforward|
in the indicated file will be processed in this form.
The optional argument \textit{main}
passes on directly to the main file \textit{main}
while pretending to compile the child \textit{dest}.
This form behaves as if \textit{dest}
issues |\childdocof{|\textit{main}|}| right away,
and no further \textsf{childdoc} directives will be processed.

%%%%%%%%%%%%%%%%%%%%%%%%%%%%%%%%%%%%%%%%
\DescribeMacro{\...prefix}
In the alternative form |\childdocforwardprefix|,
%
\begin{center}
\begin{tabular}{l}
|\input{childdoc.def}|\\
|\childdocforwardprefix[|\textit{main}|]{|\textit{prefix}|}{|\textit{dest}|}|
\end{tabular}
\end{center}
%
the destination file is determined by a pattern
depending on the current file:
To make this work, the current file must be called
`{\textit{prefix}\hspace{0.2em}\textit{suffix}}'
with \textit{prefix} matching precisely the argument.
Processing is then passed on to the file
`{\textit{dest}\hspace{0.2em}\textit{suffix}}'.
Surely, the same effect is achieved by
directly specifying the
argument `{\textit{dest}\hspace{0.2em}\textit{suffix}}'
in the first form.
However, that requires to set up a different file
for each child. With the alternative form of the command
all these files can have exactly the same content
which simplifies setting them up and maintaining them.

For example, the following file |draft.tex|
with a compilation flag |\version| as described in \secref{sec:flags}
compiles the main document as a draft:
%
\begin{center}
\begin{tabular}{l}
|\def\version{draft}|\\
|\input{childdoc.def}|\\
|\childdocforward{|\textit{main}|}|
\end{tabular}
\end{center}
%
Likewise, the following files |final|\textit{nn}|.tex|
compile the final version of the child document
|child|\textit{nn}|.tex|:
%
\begin{center}
\begin{tabular}{l}
|\def\version{final}|\\
|\input{childdoc.def}|\\
|\childdocforwardprefix{final}{child}|
\end{tabular}
\end{center}
%

Note that when several versions of a main file and/or of each child file
are to be generated, it may be convenient to set up a |Makefile| or
shell script to automatise the process.

%%%%%%%%%%%%%%%%%%%%%%%%%%%%%%%%%%%%%%%%%%%%%%%%%%%%%%%%%%%%%%%%%%%%%%%%%%%%%%%%
\subsection{Command Line Processing}
\label{sec:commandline}

The effect of redirection files can also be achieved by invoking
the \LaTeX{} compiler with a more elaborate command line.
Most conveniently this should be done as part
of a shell script or a |Makefile|.

When using \textsf{childdoc} in the main file, the following
command lines effectively perform a redirection
(note that depending on the shell being used,
backslashes may have to be doubled: `|\|' $\to$ `|\\|'):
%
\begin{center}
|... -jobname "|\textit{target}|" |\\|"|[\textit{flags}]%
|\input{childdoc.def}\childdocforward[|\textit{main}|]{|\textit{dest}|}"|
\end{center}
%
Here \textit{target} is the name of the output file,
\textit{main} is the name of the main file
and \textit{dest} is the name of the main or child file to be processed
(all filenames without extensions).
The optional argument \textit{main} can be omitted
if \textit{main} matches \textit{dest}.
Optionally, compilation \textit{flags} can be defined via |\def| commands.
This command line makes the \TeX{} engine believe
it is compiling the file \textit{target}
whose content is specified as the latter parameter.
The provided code then forwards the processing to
\textit{main} or \textit{dest} as described in \secref{sec:forward}.

%%%%%%%%%%%%%%%%%%%%%%%%%%%%%%%%%%%%%%%%%%%%%%%%%%%%%%%%%%%%%%%%%%%%%%%%%%%%%%%%
\subsection{Include by Input}
\label{sec:input}

Including child documents by |\include| has some restrictions by design.
Most notably, the content of a child document always occupies
its own set of pages; pages cannot be shared between child documents.
Usually, this behaviour makes perfect sense
because each child document contain an essential part of the document.
However, in some situations it may be desirable to compose
a document from a collection of parts
without having mandatory page breaks between then.
For this case, the package
provides a mechanism to include parts
by |\input| which can also be processed individually.
However, by construction this mechanism
requires manual handling of the content to be output.

%%%%%%%%%%%%%%%%%%%%%%%%%%%%%%%%%%%%%%%%
\DescribeMacro{\ifchilddocmanual}
The main file should be prepared as usual, see \secref{sec:include}.
However, the document body must make a distinction
between processing of an individual part and of the main document, e.g.:
%
\begin{center}
\begin{tabular}{l}
|\ifchilddocmanual|\\
|\input{\childdocname}|\\
|\||else|\\
\textit{document body with }|\input{|\textit{part}|}|\\
|\||fi|
\end{tabular}
\end{center}
%
The conditional |\ifchilddocmanual| is true whenever
a part to be included by |\input| is being compiled,
and the name of the part is stored in |\childdocname|.

%%%%%%%%%%%%%%%%%%%%%%%%%%%%%%%%%%%%%%%%
\DescribeMacro{\childdocby}
Each part to be included by |\input| should start with:
%
\begin{center}
\begin{tabular}{l}
|\input{childdoc.def}|\\
|\childdocby{|\textit{main}|}|\\
\end{tabular}
\end{center}
%
The directive |\childdocby| is similar to |\childdocof|
described in \secref{sec:include},
but the subsequent selection of content must be done manually.
To that end, both |\ifchilddoc| and |\ifchilddocmanual|
will be true upon processing of a part,
and the name of the part is stored in |\childdocname|.
Note that |\jobname| will be set to the filename of the current part
so that each part receives an individual |.aux| file
that does not interfere with the |.aux| file(s) of the main document.
This behaviour can be altered by the alternative form
|\childdocby[*]{|\textit{main}|}| (with a non-empty optional argument)
which uses the |.aux| file of the main document
by setting |\jobname| to \textit{main}.

%%%%%%%%%%%%%%%%%%%%%%%%%%%%%%%%%%%%%%%%%%%%%%%%%%%%%%%%%%%%%%%%%%%%%%%%%%%%%%%%
\subsection{Driver Development}
\label{sec:driver}

The \textsf{childdoc} mechanism can also be use for the development
of definition files such as \LaTeX{} styles or classes.
This case differs from the above setup with multiple parts
included by |\include| in that no |\includeonly| should be invoked.
This can be achieved by starting the include file
(before |\ProvidesPackage|) with:
%
\begin{center}
\begin{tabular}{l}
|\input{childdoc.def}|\\
|\childdocforward{|\textit{main}|}|\\
\end{tabular}
\end{center}
%
or alternatively with:
%
\begin{center}
\begin{tabular}{l}
|\input{childdoc.def}|\\
|\childdocby{|\textit{main}|}|\\
\end{tabular}
\end{center}
%
Both forms have slightly different effects as described above.
The main file is prepared as usual, see \secref{sec:include}.

%%%%%%%%%%%%%%%%%%%%%%%%%%%%%%%%%%%%%%%%%%%%%%%%%%%%%%%%%%%%%%%%%%%%%%%%%%%%%%%%
\subsection{Legacy Detection}
\label{sec:detection}

The directive |\childdocmain| in the main file can detect
whether the complete document or merely a child is to be compiled
even without using the directive |\childdocof|.
This method is deprecated because it is less robust
and there is no compelling reason to use it;
it is merely provided for backward compatibility
and it may be removed in future versions.

If the detection mechanism is to be used,
it is mandatory to correctly specify
the filename of the main file as the argument of |\childdocmain|:
%
\begin{center}
\begin{tabular}{l}
|\input{childdoc.def}|\\
|\childdocmain{|\textit{main}|}|\\
\end{tabular}
\end{center}
%
If |\jobname| does not match the argument \textit{main} of |\childdocmain|,
it is assumed that |\jobname| points to the child file to be compiled.
When using |\childdocmain| with the main file specified as argument,
it suffices to start a child file
with just |\input{|\textit{main}|}|
without loading of the package and using |\childdocof|.
If instead all processing is done
with the appropriate \textsf{childdoc} directives,
the argument of \textit{main} of |\childdocmain| can be empty.

An alternative version of the command line processing described
in \secref{sec:commandline} using the detection mechanism reads:
%
\begin{center}
|... -jobname "|\textit{target}|" "|[\textit{flags}]%
[|\def\jobname{|\textit{dest}|}|]|\input{|\textit{main}|}"|
\end{center}

%%%%%%%%%%%%%%%%%%%%%%%%%%%%%%%%%%%%%%%%%%%%%%%%%%%%%%%%%%%%%%%%%%%%%%%%%%%%%%%%
\subsection{Manual Code}
\label{sec:manual}

In case one cannot be certain whether the definitions file |childdoc.def|
is installed on the target \TeX{} distribution
and one prefers not to ship it,
it is conceivable to paste a few relevant commands into the sources.

To that end, drop all statements |\input{childdoc.def}|
and perform the replacements as outlined below.
Instead of |\childdocmain{|\textit{main}|}| add the following code
to the top of the main file:
%
\begin{center}
\begin{tabular}{l}
|\||ifdefined\childdocname\endinput\||fi\newif\ifchilddoc|\\
|\edef\childdocname{\scantokens\expandafter{\jobname\noexpand}}|\\
|\def\childdocmain{|\textit{main}|}\||ifx\childdocmain\childdocname\||else|\\
|\childdoctrue\includeonly{\childdocname}\let\jobname\childdocmain\||fi|\\
\end{tabular}
\end{center}
%
Instead of |\childdocof{|\textit{main}|}| just include the main file
at the top of each child file:
%
\begin{center}
|\input{|\textit{main}|}|
\end{center}
%
A simple redirection |\childdocforward{|\textit{dest}|}| is achieved by:
%
\begin{center}
|\def\jobname{|\textit{dest}|}\input{\jobname}|
\end{center}
%
The redirection with prefix
|\childdocforwardprefix[|\textit{prefix}|]{|\textit{dest}|}|
is accomplished by:
%
\begin{center}
\begin{tabular}{l}
|{\edef\jobname{\scantokens\expandafter{\jobname\noexpand}}|\\
|\def\redirectjob |\textit{prefix}|#1~~~{\gdef\jobname{|\textit{dest}|#1}}|\\
|\expandafter\redirectjob\jobname~~~}\input{\jobname}|
\end{tabular}
\end{center}

In an alternative approach,
child documents can be compiled by a specific command line
without additional code or specific definitions:
%
\begin{center}
|... -jobname "|\textit{target}|" "|[\textit{flags}]%
|\includeonly{|\textit{dest}|}\input{|\textit{main}|}"|
\end{center}
%

%%%%%%%%%%%%%%%%%%%%%%%%%%%%%%%%%%%%%%%%%%%%%%%%%%%%%%%%%%%%%%%%%%%%%%%%%%%%%%%%
%%%%%%%%%%%%%%%%%%%%%%%%%%%%%%%%%%%%%%%%%%%%%%%%%%%%%%%%%%%%%%%%%%%%%%%%%%%%%%%%
\section{Information}

%%%%%%%%%%%%%%%%%%%%%%%%%%%%%%%%%%%%%%%%%%%%%%%%%%%%%%%%%%%%%%%%%%%%%%%%%%%%%%%%
\subsection{Copyright}

Copyright \copyright{} 2017--2018 Niklas Beisert

This work may be distributed and/or modified under the
conditions of the \LaTeX{} Project Public License, either version 1.3
of this license or (at your option) any later version.
The latest version of this license is in
  \url{http://www.latex-project.org/lppl.txt}
and version 1.3 or later is part of all distributions of \LaTeX{}
version 2005/12/01 or later.

This work has the LPPL maintenance status `maintained'.

The Current Maintainer of this work is Niklas Beisert.

This work consists of the files |README.txt|, |childdoc.ins| and |childdoc.dtx|
as well as the derived files |childdoc.def|, |cdocsamp.tex|
with |cdocsch1.tex|, |cdocsch2.tex|, |cdocspt3.tex|, |cdocspt4.tex|,
|cdocsdrf.tex|, |cdocsfn1.tex|, |cdocsfn2.tex|
as well as |childdoc.pdf|.

%%%%%%%%%%%%%%%%%%%%%%%%%%%%%%%%%%%%%%%%%%%%%%%%%%%%%%%%%%%%%%%%%%%%%%%%%%%%%%%%
\subsection{Files and Installation}

The package consists of the files:
%
\begin{center}
\begin{tabular}{ll}
    |README.txt|   & readme file \\
    |childdoc.ins| & installation file \\
    |childdoc.dtx| & source file \\
    |childdoc.def| & definition file \\
    |cdocsamp.tex| & sample main file \\
    |cdocsch1.tex| & sample include file \\
    |cdocsch2.tex| & sample include file \\
    |cdocspt3.tex| & sample part file \\
    |cdocspt4.tex| & sample part file \\
    |cdocsdrf.tex| & sample redirection file \\
    |cdocsfn1.tex| & sample redirection file \\
    |cdocsfn2.tex| & sample redirection file \\
    |childdoc.pdf| & manual
\end{tabular}
\end{center}
%
The distribution consists of the files
|README.txt|, |childdoc.ins| and |childdoc.dtx|.
%
\begin{itemize}
\item
Run (pdf)\LaTeX{} on |childdoc.dtx|
to compile the manual |childdoc.pdf| (this file).
\item
Run \LaTeX{} on |childdoc.ins| to create the definitions file |childdoc.def|
and the sample |cdocsamp.tex| with include files
|cdocsch1.tex|, |cdocsch2.tex|, |cdocspt3.tex|, |cdocspt4.tex|,
|cdocsdrf.tex|, |cdocsfn1.tex|, |cdocsfn2.tex|.
Then copy the file |childdoc.def| to an appropriate directory of your \LaTeX{}
distribution, e.g.\ \textit{texmf-root}|/tex/latex/childdoc|.
\end{itemize}

%%%%%%%%%%%%%%%%%%%%%%%%%%%%%%%%%%%%%%%%%%%%%%%%%%%%%%%%%%%%%%%%%%%%%%%%%%%%%%%%
\subsection{Related CTAN Packages}

There are several other packages which offer a similar functionality:
%
\begin{itemize}
\item
The packages
\href{http://ctan.org/pkg/docmute}{\textsf{docmute}},
\href{http://ctan.org/pkg/includex}{\textsf{includex}} and
\href{http://ctan.org/pkg/standalone}{\textsf{standalone}}
provide commands to include only the document body of
a child file thus allowing both files to be compiled individually.
\item
The packages \href{http://ctan.org/pkg/subdocs}{\textsf{subdocs}}
and \href{http://ctan.org/pkg/subfiles}{\textsf{subfiles}}
provide structures in which the main and child documents can be
encapsulated and allowing them to be compiled individually.
The inclusion mechanism is different from the conventional |\include|.
\item
The package \href{http://ctan.org/pkg/combine}{\textsf{combine}}
is an elaborate solution to combine several documents into one.
\end{itemize}
%
See also the CTAN topic \href{http://ctan.org/topic/subdocs}{\textsf{subdocs}}
for further related packages.
The present package differs from the above solutions in that
a document structure constructed with the conventional |\include| mechanism
just needs two extra commands at the top of every file
such that all constituent files can be compiled individually.

%%%%%%%%%%%%%%%%%%%%%%%%%%%%%%%%%%%%%%%%%%%%%%%%%%%%%%%%%%%%%%%%%%%%%%%%%%%%%%%%
%\subsection{Feature Suggestions}
%
%The following is a list of features which may be useful for future
%versions of this package:
%%
%\begin{itemize}
%\item
%\ldots
%\end{itemize}

%%%%%%%%%%%%%%%%%%%%%%%%%%%%%%%%%%%%%%%%%%%%%%%%%%%%%%%%%%%%%%%%%%%%%%%%%%%%%%%%
\subsection{Revision History}

%%%%%%%%%%%%%%%%%%%%%%%%%%%%%%%%%%%%%%%%
\paragraph{v2.0:} 2018/12/30

\begin{itemize}
\item
immediate forward processing
\item
added |\childdocby| mechanism
\item
manual restructured
\end{itemize}

%%%%%%%%%%%%%%%%%%%%%%%%%%%%%%%%%%%%%%%%
\paragraph{v1.6:} 2018/01/17

\begin{itemize}
\item
application for development of include files
\item
corrections to manual
\end{itemize}

%%%%%%%%%%%%%%%%%%%%%%%%%%%%%%%%%%%%%%%%
\paragraph{v1.5:} 2017/05/21

\begin{itemize}
\item
more complete structuring introduced
\item
|\childdocof| introduced
\item
|\childdoc| renamed to |\childdocmain|
\item
|\childredirect| renamed to |\childdocforward| and |\childdocforwardprefix|
and functionality expanded
\end{itemize}

%%%%%%%%%%%%%%%%%%%%%%%%%%%%%%%%%%%%%%%%
\paragraph{v1.0:} 2017/04/27

\begin{itemize}
\item
manual and install package
\item
first version published on CTAN
\end{itemize}

%%%%%%%%%%%%%%%%%%%%%%%%%%%%%%%%%%%%%%%%
\paragraph{v0.6:} 2017/04/26

\begin{itemize}
\item
redirection mechanism added
\end{itemize}

%%%%%%%%%%%%%%%%%%%%%%%%%%%%%%%%%%%%%%%%
\paragraph{v0.5:} 2017/04/26

\begin{itemize}
\item
functionality in definition file
\end{itemize}


%%%%%%%%%%%%%%%%%%%%%%%%%%%%%%%%%%%%%%%%%%%%%%%%%%%%%%%%%%%%%%%%%%%%%%%%%%%%%%%%
%%%%%%%%%%%%%%%%%%%%%%%%%%%%%%%%%%%%%%%%%%%%%%%%%%%%%%%%%%%%%%%%%%%%%%%%%%%%%%%%
%%%%%%%%%%%%%%%%%%%%%%%%%%%%%%%%%%%%%%%%%%%%%%%%%%%%%%%%%%%%%%%%%%%%%%%%%%%%%%%%
\appendix

\settowidth\MacroIndent{\rmfamily\scriptsize 000\ }

 \DocInput{childdoc.dtx}

\end{document}
%</driver>
% \fi
%
% %%%%%%%%%%%%%%%%%%%%%%%%%%%%%%%%%%%%%%%%%%%%%%%%%%%%%%%%%%%%%%%%%%%%%%%%%%%%%%
% %%%%%%%%%%%%%%%%%%%%%%%%%%%%%%%%%%%%%%%%%%%%%%%%%%%%%%%%%%%%%%%%%%%%%%%%%%%%%%
% \section{Sample}
%\iffalse
%<*samplemain>
%\fi
%
% The following presents a sample document
% with two chapters, two parts, a title page,
% a compile flag as well as three forwarding files to set the flag.
% It consists of eight |.tex| files:
% \begin{center}
% \begin{tabular}{ll}
% |cdocsamp.tex|&main file\\
% |cdocsch1.tex|&include file for chapter 1\\
% |cdocsch2.tex|&include file for chapter 2\\
% |cdocspt3.tex|&include file for part 3\\
% |cdocspt4.tex|&include file for part 4\\
% |cdocsdrf.tex|&forwarding file for main file in draft mode\\
% |cdocsfi1.tex|&forwarding file for final version of chapter 1\\
% |cdocsfi2.tex|&forwarding file for final version of chapter 2\\
% \end{tabular}
% \end{center}
% Each of the eight files can be compiled directly by the \LaTeX{} compiler.
%
% %%%%%%%%%%%%%%%%%%%%%%%%%%%%%%%%%%%%%%
% \paragraph{Main File.}
%
% The main file is called |cdocsamp.tex|.
%
% Load the \textsf{childdoc} definitions and
% declare the filename for the main document:
%    \begin{macrocode}
\input{childdoc.def}
\childdocmain{}
%    \end{macrocode}

% Optional override for |\version| flag:
%    \begin{macrocode}
%%\ifchilddoc\else\providecommand{\version}{draft}\fi
%    \end{macrocode}

% Define the default values for the |\version| flag
% (|final| for the main file and |draft| for childs):
%    \begin{macrocode}
\ifchilddoc
\providecommand{\version}{draft}
\else
\providecommand{\version}{final}
\fi
%    \end{macrocode}

% Load the standard document class:
%    \begin{macrocode}
\documentclass[12pt]{article}
%    \end{macrocode}

% Start the document body:
%    \begin{macrocode}
\begin{document}
%    \end{macrocode}

% Declare a title page.
% Print title, part of document being processed and version flag:
%    \begin{macrocode}
\addtocounter{page}{-1}
\begin{center}
{\LARGE\bfseries{}childdoc example\par}
\vspace{1cm}
\ifchilddoc
\ifchilddocmanual part\else chapter\fi:
`\childdocname' of `\childdocjob'\par
\else
main document: `\childdocjob'\par
\fi
version: \version\par
\end{center}
\newpage
%    \end{macrocode}

% Manually include selected file,
% otherwise process as usual:
%    \begin{macrocode}
\ifchilddocmanual
\section*{part `\childdocname'}
\input{\childdocname}
\else
%    \end{macrocode}

% Include the two chapters:
%    \begin{macrocode}
\include{cdocsch1}
\include{cdocsch2}
%    \end{macrocode}

% Include the two parts unless only chapters should be displayed:
%    \begin{macrocode}
\ifchilddoc\else
\section{part three}
\input{cdocspt3}
\section{part four}
\input{cdocspt4}
\fi
%    \end{macrocode}

% Process as usual until here:
%    \begin{macrocode}
\fi
%    \end{macrocode}

% End of document body:
%    \begin{macrocode}
\end{document}
%    \end{macrocode}
%\iffalse
%</samplemain>
%\fi
%
% %%%%%%%%%%%%%%%%%%%%%%%%%%%%%%%%%%%%%%
% \paragraph{Chapter Include Files.}
%
% The include files are called |cdocsch1.tex| and |cdocsch2.tex|.
%
%\iffalse
%<*samplechap1|samplechap2>
%\fi

% Optional override for |\version| flag:
%    \begin{macrocode}
%%\providecommand{\version}{final}
%    \end{macrocode}

% Include the main document:
%    \begin{macrocode}
\input{childdoc.def}
\childdocof{cdocsamp}
%    \end{macrocode}

%\iffalse
%</samplechap1|samplechap2>
%\fi
%
%\iffalse
%<*samplechap1>
%\fi
% Some text for chapter 1:
%    \begin{macrocode}
\section{one}
some text in chapter one
%    \end{macrocode}

%\iffalse
%</samplechap1>
%\fi
% Some text for chapter 2:
%\iffalse
%<*samplechap2>
%\fi
%    \begin{macrocode}
\section{two}
more text in chapter two
%    \end{macrocode}

%\iffalse
%</samplechap2>
%\fi
%
% %%%%%%%%%%%%%%%%%%%%%%%%%%%%%%%%%%%%%%
% \paragraph{Part Include Files.}
%
% The include files are called |cdocspt3.tex| and |cdocspt4.tex|.
%
%\iffalse
%<*samplepart3|samplepart4>
%\fi

% Optional override for |\version| flag:
%    \begin{macrocode}
%%\providecommand{\version}{final}
%    \end{macrocode}

% Include the main document:
%    \begin{macrocode}
\input{childdoc.def}
\childdocby{cdocsamp}
%    \end{macrocode}

%\iffalse
%</samplepart3|samplepart4>
%\fi
%
%\iffalse
%<*samplepart3>
%\fi
% Some text for part 3:
%    \begin{macrocode}
some text in part three
%    \end{macrocode}

%\iffalse
%</samplepart3>
%\fi
% Some text for part 4:
%\iffalse
%<*samplepart4>
%\fi
%    \begin{macrocode}
more text in part four
%    \end{macrocode}

%\iffalse
%</samplepart4>
%\fi
%
% %%%%%%%%%%%%%%%%%%%%%%%%%%%%%%%%%%%%%%
% \paragraph{Forwarding for a Complete Draft.}
%
% The following forwarding file |cdocsdrf.tex|
% compiles the main document in draft mode:
%\iffalse
%<*sampledraft>
%\fi
%    \begin{macrocode}
\def\version{draft}
\input{childdoc.def}
\childdocforward{cdocsamp}
%    \end{macrocode}

%\iffalse
%</sampledraft>
%\fi
%
% %%%%%%%%%%%%%%%%%%%%%%%%%%%%%%%%%%%%%%
% \paragraph{Forwarding for Final Version of the Chapters.}
%
% The following forwarding files |cdocsfn1.tex| and |cdocsfn2.tex|
% (with identical content)
% compile the final versions of the child documents
% |cdocsch1.tex| and |cdocsch2.tex|, respectively:
%\iffalse
%<*samplefinal>
%\fi
%    \begin{macrocode}
\def\version{final}
\input{childdoc.def}
\childdocforwardprefix[cdocsamp]{cdocsfn}{cdocsch}
%    \end{macrocode}

%\iffalse
%</samplefinal>
%\fi
%
% %%%%%%%%%%%%%%%%%%%%%%%%%%%%%%%%%%%%%%
% \paragraph{Command Line Processing.}
%
% The following three command lines generate the output files
% |cdocscld|, |cdocscl1| and |cdocscl2|
% which should be identical to
% |cdocsdrf|, |cdocsch1| and |cdocsfn2|, respectively:
% \begin{center}
% \begin{tabular}{l}
% |latex -jobname cdocscld \|\\
% |  "\def\version{draft}\input{childdoc.def}\childdocforward{cdocsamp}"|\\
% |latex -jobname cdocscl1 \|\\
% |  "\input{childdoc.def}\childdocforward[cdocsamp]{cdocsch1}"|\\
% |latex -jobname cdocscl2 \|\\
% |  "\def\version{final}\input{childdoc.def}\childdocforward{cdocsch2}"|
% \end{tabular}
% \end{center}
% Note that the trailing backslash on each first line
% merely continues the input to the second line
% (for convenient cut ant paste).
% Furthermore, the command |latex| can be replaced by any
% of its alternative versions such as |pdflatex|.
%
% %%%%%%%%%%%%%%%%%%%%%%%%%%%%%%%%%%%%%%%%%%%%%%%%%%%%%%%%%%%%%%%%%%%%%%%%%%%%%%
% %%%%%%%%%%%%%%%%%%%%%%%%%%%%%%%%%%%%%%%%%%%%%%%%%%%%%%%%%%%%%%%%%%%%%%%%%%%%%%
% \section{Implementation}
%\iffalse
%<*package>
%\fi
%
% This section describes the definitions file |childdoc.def|.

% The definitions cannot be loaded using |\usepackage| or |\RequirePackage|
% which has a mechanism to prevent loading a style file more than once.
% When loading the definitions by means of |\input|
% multiple instances have to be prevented manually:
%\iffalse
%This code needs to be before the `\ProvidesFile' directive
%which is defined at the beginning of this file.
%Therefore it is also placed there and commented out here.
%</package>
%<*discard>
%\fi
%    \begin{macrocode}
\ifdefined\childdocmain\endinput\fi
%    \end{macrocode}
%\iffalse
%</discard>
%<*package>
%\fi
%
% \macro{\ifchilddoc}
% \macro{\ifchilddocmanual}
% The conditional |\ifchilddoc| tells whether a
% child (true) or main (false) document is being compiled.
% The conditional |\ifchilddocmanual| tells whether
% the |\includeonly| mechanism is used (false) or
% the selection of child files must be performed manually (true).
% The definitions initialise to false:
%    \begin{macrocode}
\newif\ifchilddoc
\newif\ifchilddocmanual
%    \end{macrocode}

% \macro{\childdocname}
% \macro{\childdocjob}
% The macro |\childdocname| stores the name of the main document
% to be compiled. The macro |\childdocjob| stores the name of
% the document on which the \LaTeX{} compiler was originally invoked.
% The content of |\jobname| cannot be compared
% to filenames specified in the source due to different catcodes.
% The following code rescans |\jobname|, stores the result
% in |\childdocname| and saves a copy in |\childdocjob|:
%    \begin{macrocode}
\edef\childdocname{\scantokens\expandafter{\jobname\noexpand}}
\let\childdocjob\childdocname
%    \end{macrocode}

% \macro{\childdocdisable}
% The macro |\childdocdisable| prevents the main file
% from being processed more than once.
% At this stage, the main document command |\childdocmain|
% is assumed to be called once again where it should do nothing.
% Any subsequent call to it should prevent
% a secondary processing of the main document
% It overwrites the forwarding commands
% |\childdocof| and |\childdocforward|
% with empty macros to prevent further inclusions of the main document:
%    \begin{macrocode}
\newcommand{\childdocdisable}
{
  \renewcommand{\childdocmain}[1]{\renewcommand{\childdocmain}[1]{\endinput}}
  \renewcommand{\childdocof}[1]{}
  \renewcommand{\childdocby}[2][]{}
  \renewcommand{\childdocforward}[2][]{}
  \renewcommand{\childdocdisable}{}
}
%    \end{macrocode}

% \macro{\childdocmain}
% The macro |\childdocmain| is to be called at the top of the main file
% with nothing or the main filename (without extension) as argument.
% First, it breaks loops.
% If the argument is not empty and does not match |\childdocname|
% (which is set by the first inclusion of |childdoc.def|),
% |\ifchilddoc| is set to true, |\includeonly| is applied to the child file
% and |\jobname| is set to the main file
% (for proper handling of |.aux| files):
%    \begin{macrocode}
\newcommand{\childdocmain}[1]
{
  \childdocdisable\childdocmain{}
  \if?#1?\else
    \begingroup
      \def\childdoctmp{#1}
      \ifx\childdoctmp\childdocname
        \def\childdoctmp{}
      \else
        \def\childdoctmp
        {
          \childdoctrue
          \includeonly{\childdocname}
          \def\childdocjob{#1}
          \def\jobname{#1}
        }
      \fi
      \expandafter
    \endgroup
    \childdoctmp
  \fi
}
%    \end{macrocode}

% \macro{\childdocof}
% The command |\childdocof| redirects
% compilation to the main file |#1|.
%    \begin{macrocode}
\newcommand{\childdocof}[1]
{
  \childdocdisable
  \childdoctrue
  \includeonly{\childdocname}
  \def\jobname{#1}
  \def\childdocjob{#1}
  \input{#1}
}
%    \end{macrocode}

% \macro{\childdocby}
% The command |\childdocby| ....
%    \begin{macrocode}
\newcommand{\childdocby}[2][]
{
  \childdocdisable
  \childdoctrue
  \childdocmanualtrue
  \if?#1?\else
    \def\jobname{#2}
  \fi
  \def\childdocjob{#2}
  \input{#2}
  \endinput
}
%    \end{macrocode}

% \macro{\childdocforward}
% The command |\childdocforward| redirects
% compilation to the main file or
% (if the optional argument is given) a child file.
% Parameters are set as if the main file
% or a child file starting with |\childdocof| was compiled.
% Then compilation is handed over to the main file:
%    \begin{macrocode}
\newcommand{\childdocforward}[2][]
{
  \begingroup
    \if?#1?
      \def\childdoctmp
      {
        \def\childdocname{#2}
        \def\childdocjob{#2}
        \def\jobname{#2}
        \input{#2}
        \endinput
      }
    \else
      \def\childdoctmp
      {
        \childdocdisable
        \def\childdocname{#2}
        \childdoctrue
        \includeonly{#2}
        \def\childdocjob{#1}
        \def\jobname{#1}
        \input{#1}
        \endinput
      }
    \fi
    \expandafter
  \endgroup
  \childdoctmp
}
%    \end{macrocode}

% \macro{\childdocforwardprefix}
% The command |\childdocforwardprefix| redirects
% compilation to the main or a child file by means of a pattern.
% The prefix |#1| in the current filename is replaced by |#2|
% and the suffix of the current filename is kept
% (it is assumed that the filename does not contain the substring `|~~~|'
% which is used as a delimiter).
% Compilation is handed over to the new file by |\childdocforward|:
%    \begin{macrocode}
\newcommand{\childdocforwardprefix}[3][]
{
  \begingroup
    \def\childdocextract #2##1~~~{\def\childdoctmp{\childdocforward[#1]{#3##1}}}
    \expandafter\childdocextract\childdocname~~~
    \expandafter
  \endgroup
  \childdoctmp
}
%    \end{macrocode}

% \macro{\childdoc}
% The deprecated macro |\childdoc| is a legacy version of |\childdocmain|:
%    \begin{macrocode}
\newcommand{\childdoc}{\childdocmain}
%    \end{macrocode}

% \macro{\childdocredirect}
% The deprecated macro |\childdocredirect| is a legacy version
% of |\childdocforward| and |\childdocforwardprefix|:
%    \begin{macrocode}
\newcommand{\childdocredirect}[2][]
{
  \begingroup
    \if?#1?
      \def\childdoctmp{\childdocforward{#2}}
    \else
      \def\childdoctmp{\childdocforwardprefix{#1}{#2}}
    \fi
    \expandafter
  \endgroup
  \childdoctmp
}
%    \end{macrocode}

%\iffalse
%</package>
%\fi
%
\endinput

\childdocof{cdocsamp}
%    \end{macrocode}

%\iffalse
%</samplechap1|samplechap2>
%\fi
%
%\iffalse
%<*samplechap1>
%\fi
% Some text for chapter 1:
%    \begin{macrocode}
\section{one}
some text in chapter one
%    \end{macrocode}

%\iffalse
%</samplechap1>
%\fi
% Some text for chapter 2:
%\iffalse
%<*samplechap2>
%\fi
%    \begin{macrocode}
\section{two}
more text in chapter two
%    \end{macrocode}

%\iffalse
%</samplechap2>
%\fi
%
% %%%%%%%%%%%%%%%%%%%%%%%%%%%%%%%%%%%%%%
% \paragraph{Part Include Files.}
%
% The include files are called |cdocspt3.tex| and |cdocspt4.tex|.
%
%\iffalse
%<*samplepart3|samplepart4>
%\fi

% Optional override for |\version| flag:
%    \begin{macrocode}
%%\providecommand{\version}{final}
%    \end{macrocode}

% Include the main document:
%    \begin{macrocode}
% \iffalse
%
% childdoc.dtx Copyright (C) 2017-2018 Niklas Beisert
%
% This work may be distributed and/or modified under the
% conditions of the LaTeX Project Public License, either version 1.3
% of this license or (at your option) any later version.
% The latest version of this license is in
%   http://www.latex-project.org/lppl.txt
% and version 1.3 or later is part of all distributions of LaTeX
% version 2005/12/01 or later.
%
% This work has the LPPL maintenance status `maintained'.
%
% The Current Maintainer of this work is Niklas Beisert.
%
% This work consists of the files childdoc.dtx and childdoc.ins
% and the derived files childdoc.def and cdocsamp.tex with
% cdocsch1.tex, cdocsch2.tex, cdocsdrf.tex, cdocsfn1.tex, cdocsfn2.tex.
%
%<package>\ifdefined\childdocmain\endinput\fi
%<package>\ProvidesFile{childdoc.def}[2018/12/30 v2.0 child document driver]
%<samplemain>\ProvidesFile{cdocsamp.tex}[2018/12/30 v2.0 sample for childdoc]
%<*driver>
%\ProvidesFile{childdoc.drv}[2018/12/30 v2.0 childdoc reference manual file]
\PassOptionsToClass{10pt,a4paper}{article}
\documentclass{ltxdoc}

\usepackage[margin=35mm]{geometry}
\usepackage{hyperref}
\usepackage{hyperxmp}
\usepackage[usenames]{color}

\hypersetup{colorlinks=true}
\hypersetup{pdfstartview=FitH}
\hypersetup{pdfpagemode=UseNone}
\hypersetup{pdfsource={}}
\hypersetup{pdflang={en-UK}}
\hypersetup{pdfcopyright={Copyright 2017-2018 Niklas Beisert.
  This work may be distributed and/or modified under the
  conditions of the LaTeX Project Public License, either version 1.3
  of this license or (at your option) any later version.}}
\hypersetup{pdflicenseurl={http://www.latex-project.org/lppl.txt}}
\hypersetup{pdfcontactaddress={ETH Zurich, ITP, HIT K,
  Wolfgang-Pauli-Strasse 27}}
\hypersetup{pdfcontactpostcode={8093}}
\hypersetup{pdfcontactcity={Zurich}}
\hypersetup{pdfcontactcountry={Switzerland}}
\hypersetup{pdfcontactemail={nbeisert@itp.phys.ethz.ch}}
\hypersetup{pdfcontacturl={http://people.phys.ethz.ch/\xmptilde nbeisert/}}

\newcommand{\secref}[1]{\hyperref[#1]{section \ref*{#1}}}

\parskip1ex
\parindent0pt
\let\olditemize\itemize
\def\itemize{\olditemize\parskip0pt}

\begin{document}

\title{The \textsf{childdoc} Package}
\hypersetup{pdftitle={The childdoc Package}}
\author{Niklas Beisert\\[2ex]
  Institut f\"ur Theoretische Physik\\
  Eidgen\"ossische Technische Hochschule Z\"urich\\
  Wolfgang-Pauli-Strasse 27, 8093 Z\"urich, Switzerland\\[1ex]
  \href{mailto:nbeisert@itp.phys.ethz.ch}
  {\texttt{nbeisert@itp.phys.ethz.ch}}}
\hypersetup{pdfauthor={Niklas Beisert}}
\hypersetup{pdfsubject={Manual for the LaTeX2e Package childdoc}}
\date{30 December 2018, \textsf{v2.0}}
\maketitle

\begin{abstract}\noindent
\textsf{childdoc} is a \LaTeXe{} package
that enables the direct compilation
of document sections included by |\include|
to individual files.
\end{abstract}

\begingroup
\parskip0ex
\tableofcontents
\endgroup

%%%%%%%%%%%%%%%%%%%%%%%%%%%%%%%%%%%%%%%%%%%%%%%%%%%%%%%%%%%%%%%%%%%%%%%%%%%%%%%%
%%%%%%%%%%%%%%%%%%%%%%%%%%%%%%%%%%%%%%%%%%%%%%%%%%%%%%%%%%%%%%%%%%%%%%%%%%%%%%%%
\section{Introduction}

\LaTeX{} provides a mechanism to structure a large document (such as a book)
into a main file and several child files (containing the chapters)
using the |\include| command.
This mechanism is beneficial for documents
which span hundreds of pages in order to
make the source file(s) more manageable.
Moreover, compilation can be restricted to
selected child files by means of the |\includeonly| command.
The latter feature can be used to reduce the compilation time while editing
(this was significantly more useful in the earlier days of \LaTeX{})
or to generate a smaller document which is easier to navigate.
Another application of |\includeonly| is to generate
documents consisting of selected parts of the complete document.

However, there are a few drawbacks of the plain |\include| mechanism:
\begin{itemize}
\item
The child files cannot be compiled on their own,
they can only be compiled via the main file.
A naive editing environment
(such as a text editor with an option
to have the current file processed by \LaTeX)
may require one to switch to the main file before compiling;
attempting to compile the child file produces errors.
\item
The main file must be modified (each time)
to adjust the |\includeonly| command
to the present needs. This easily leaves the main file in a messy state.
\item
The generated document will always carry the filename
of the main document. This is inconvenient if
several child files are to be compiled and
to be kept for distribution.
\end{itemize}

The present package provides a simple interface
to make child files individually compilable by \LaTeX{}.
Compiling a child file then has the same effect as compiling
the main file with an |\includeonly| command
to select the appropriate child.
Moreover the generated document will carry the name of the child
rather than the main file.
This resolves all three above issues.

This feature is meant to make the editing of books,
thesis documents and lecture notes somewhat more convenient.
However, the package can also be used efficiently for
composing a series of documents (such as exercise sheets)
which are typically distributed individually.
It then assists the author in generating the individual documents
(potentially in different versions)
as well as a document containing the collected series.
Another application is in developing style files
or other kinds of included material
where compilation of the style file could redirect
to a sample or test file.

%%%%%%%%%%%%%%%%%%%%%%%%%%%%%%%%%%%%%%%%%%%%%%%%%%%%%%%%%%%%%%%%%%%%%%%%%%%%%%%%
%%%%%%%%%%%%%%%%%%%%%%%%%%%%%%%%%%%%%%%%%%%%%%%%%%%%%%%%%%%%%%%%%%%%%%%%%%%%%%%%
\section{Usage}

First of all, the package \textsf{childdoc} is \emph{not} a standard
\LaTeXe{} |.sty| style file! Therefore it needs to be invoked in
a non-standard way.

%%%%%%%%%%%%%%%%%%%%%%%%%%%%%%%%%%%%%%%%%%%%%%%%%%%%%%%%%%%%%%%%%%%%%%%%%%%%%%%%
\subsection{Included Files}
\label{sec:include}

%%%%%%%%%%%%%%%%%%%%%%%%%%%%%%%%%%%%%%%%
\DescribeMacro{\childdocmain}
To use the package, add the commands
\begin{center}
\begin{tabular}{l}
|\input{childdoc.def}|\\
|\childdocmain{}|\\
\end{tabular}
\end{center}
at the very top of the main \LaTeX{} file,
in particular \emph{before} the |\documentclass| statement!
The argument of |\childdocmain| should be left empty
(but it must be present).

%%%%%%%%%%%%%%%%%%%%%%%%%%%%%%%%%%%%%%%%
\DescribeMacro{\childdocof}
Furthermore, add the commands
\begin{center}
\begin{tabular}{l}
|\input{childdoc.def}|\\
|\childdocof{|\textit{main}|}|\\
\end{tabular}
\end{center}
at the top of every child file \textit{child}
which is included by |\include{|\textit{child}|}|
from within the main file
(or at least for those files to be compiled individually).
The argument \textit{main} must be the filename of the main file.

There are a couple of
considerations in setting up the main and child documents:

%%%%%%%%%%%%%%%%%%%%%%%%%%%%%%%%%%%%%%%%
\paragraph{Restrictions.}

Please note the following restrictions:
\begin{itemize}
\item
|\childdocmain| must be called with one argument \textit{main}
to ensure compatibility with earlier version of the package.
It must either be empty (|\childdocmain{}|)
or precisely match the filename of the main file in which it is specified.
See \secref{sec:detection} for further information.
\item
The filename \textit{main} must be specified without the |.tex| extension.
\item
The filename \textit{main} is case sensitive
(even in case-insensitive file systems)
due to internal string comparison.
\item
The argument \textit{main} should be fully expanded, it cannot be a macro.
\item
Subdirectories and special characters should be avoided in filenames.
\item
The command |\childdocmain{|\textit{main}|}| must be followed by a whitespace.
It should not be followed immediately by another command
or by a comment mark `|%|'.
This is because the \TeX{} parser reads the token immediately following
the argument of |\childdocmain| and puts it
at the beginning of every child section;
however, a white\-space is ignored.
\end{itemize}

%%%%%%%%%%%%%%%%%%%%%%%%%%%%%%%%%%%%%%%%
\paragraph{Content of Main File.}

It is advisable to place all content in the child files included by |\include|.
Any output contained in the main file will appear in all child documents
unless suppressed manually;
it cannot be suppressed automatically by the |\includeonly| directive
and thus should normally be avoided.
A method to include some content in the main file
by means of conditional processing is described in \secref{sec:conditional}.

%%%%%%%%%%%%%%%%%%%%%%%%%%%%%%%%%%%%%%%%
\paragraph{Page Numbering.}

When only a part of the document is compiled,
the appropriate numbering of pages
(as well as other status parameters)
is determined from the |.aux| files.
The latter contain information from previous passes.
However this information needs to propagate through
all intermediate child documents.
Therefore the page numbering in child documents may well
be inconsistent until the complete document is compiled at least once.

A useful (if unconventional) way to always ensure a consistent
page numbering is to restart the numbering in each child document
and denote the pages by `\textit{child}|.|\textit{page}'
where \textit{child} represents the chapter/section number of the child file.
This can be achieved by the command
|\numberwithin{page}{|\textit{child}|}|
of the \textsf{amsmath} package
where \textit{child} can be |chapter| or |section|
depending on the chosen structuring.
Alternatively, one can modify the macro |\thepage| appropriately
and reset the counter |page| at the start of each child file.

%%%%%%%%%%%%%%%%%%%%%%%%%%%%%%%%%%%%%%%%%%%%%%%%%%%%%%%%%%%%%%%%%%%%%%%%%%%%%%%%
\subsection{Conditional Processing}
\label{sec:conditional}

The package provides a mechanism to compile different versions
of a document. To customise the versions further some conditional processing
can come in handy to distinguish which version is being compiled.
The package provides two macros to describe the compilation context:

%%%%%%%%%%%%%%%%%%%%%%%%%%%%%%%%%%%%%%%%
\DescribeMacro{\ifchilddoc}
The conditional |\ifchilddoc| distinguishes between the compilation of
child documents and the main document:
%
\begin{center}
|\ifchilddoc |\textit{child-code}| |[|\||else |\textit{main-code}]| \||fi|
\end{center}

%%%%%%%%%%%%%%%%%%%%%%%%%%%%%%%%%%%%%%%%
\DescribeMacro{\childdocname}
\DescribeMacro{\childdocjob}
The macro |\childdocname| contains the filename (without extension)
of the main or child file being processed.
Note that |\childdocjob| will always contain the name of the main file.

%%%%%%%%%%%%%%%%%%%%%%%%%%%%%%%%%%%%%%%%
\paragraph{Title Page.}

Conditional processing can be used to include a title or banner page
in the main document when proper precautions are taken.
Importantly, the code in the main file should ensure that the page counter
(as well as other status parameters which are stored in the |.aux| files)
takes the same value after the conditional processing.
Otherwise the page numbers may take divergent values
depending on which part is compiled.

For example, a title page could be declared by:
%
\begin{center}
\begin{tabular}{l}
|\ifchilddoc\||else|\\
|\addtocounter{page}{-1}|\\
\textit{code for title page}\\
|\newpage|\\
|\||fi|
\end{tabular}
\end{center}
%
A banner page for the child documents can be generated by:
%
\begin{center}
\begin{tabular}{l}
|\ifchilddoc|\\
|\addtocounter{page}{-1}|\\
\textit{code for banner page}\\
|\newpage|\\
|\||fi|
\end{tabular}
\end{center}
%
Here one could write a message such as:
\begin{center}
|This is the part \childdocname{} of \childdocjob{}.|
\end{center}

%%%%%%%%%%%%%%%%%%%%%%%%%%%%%%%%%%%%%%%%%%%%%%%%%%%%%%%%%%%%%%%%%%%%%%%%%%%%%%%%
\subsection{Flags}
\label{sec:flags}

The package makes it easy to generate different versions
of the main or child documents.
To this end compilation flags can be defined
and assigned different default values.
They will be particularly useful in conjunction
with the forwarding mechanism described in \secref{sec:forward}.

For example, it may be useful to have a flag |\version|
which can be set to |draft| or |final|.
The document source will contain some conditional code
depending on the value of |\version|.
Suppose further, the flag should default to |final| for the main file
and to |draft| for child files
which is a natural assignment for editing the document.
This is achieved by placing the following code
in the preamble of the main document
(below the |\childdocmain| directive):
%
\begin{center}
\begin{tabular}{l}
|\ifchilddoc|\\
|\providecommand{\version}{draft}|\\
|\||else|\\
|\providecommand{\version}{final}|\\
|\||fi|
\end{tabular}
\end{center}
%
The definition by |\providecommand| makes sure
that previous definitions are not overwritten.
Further statements |\providecommand{\version}{...}|
can thus be added before the above code to override it.

For the main file, one might add a line
(between |\childdocmain| and the above block)
%
\begin{center}
|%\ifchilddoc\||else\providecommand{\version}{draft}\||fi|
\end{center}
%
which can be uncommented to produce a draft version.
Likewise one can add a line to the very top of a child file
(above the |\childdocof{|\textit{main}|}| directive)
%
\begin{center}
|%\providecommand{\version}{final}|
\end{center}
%
which can be uncommented to produce the final version of this child document.

%%%%%%%%%%%%%%%%%%%%%%%%%%%%%%%%%%%%%%%%%%%%%%%%%%%%%%%%%%%%%%%%%%%%%%%%%%%%%%%%
\subsection{Forwarding}
\label{sec:forward}

Different versions of the main or child documents
using compilation flags as described in \secref{sec:flags}
can be (permanently) stored in different files
for convenient compilation, viewing and distribution.
To this end, the package defines a command
to pass on compilation to a different file:

%%%%%%%%%%%%%%%%%%%%%%%%%%%%%%%%%%%%%%%%
\DescribeMacro{\childdocforward}
The command |\childdocforward| redirects processing to
another source file:
%
\begin{center}
\begin{tabular}{l}
|\input{childdoc.def}|\\
|\childdocforward[|\textit{main}|]{|\textit{dest}|}|\\
\end{tabular}
\end{center}
%
The argument \textit{dest} is the destination file
(without extension).
It should be the main file or one of the child files.
Note that further \textsf{childdoc} directives
such as |\childdocof| and |\childdocforward|
in the indicated file will be processed in this form.
The optional argument \textit{main}
passes on directly to the main file \textit{main}
while pretending to compile the child \textit{dest}.
This form behaves as if \textit{dest}
issues |\childdocof{|\textit{main}|}| right away,
and no further \textsf{childdoc} directives will be processed.

%%%%%%%%%%%%%%%%%%%%%%%%%%%%%%%%%%%%%%%%
\DescribeMacro{\...prefix}
In the alternative form |\childdocforwardprefix|,
%
\begin{center}
\begin{tabular}{l}
|\input{childdoc.def}|\\
|\childdocforwardprefix[|\textit{main}|]{|\textit{prefix}|}{|\textit{dest}|}|
\end{tabular}
\end{center}
%
the destination file is determined by a pattern
depending on the current file:
To make this work, the current file must be called
`{\textit{prefix}\hspace{0.2em}\textit{suffix}}'
with \textit{prefix} matching precisely the argument.
Processing is then passed on to the file
`{\textit{dest}\hspace{0.2em}\textit{suffix}}'.
Surely, the same effect is achieved by
directly specifying the
argument `{\textit{dest}\hspace{0.2em}\textit{suffix}}'
in the first form.
However, that requires to set up a different file
for each child. With the alternative form of the command
all these files can have exactly the same content
which simplifies setting them up and maintaining them.

For example, the following file |draft.tex|
with a compilation flag |\version| as described in \secref{sec:flags}
compiles the main document as a draft:
%
\begin{center}
\begin{tabular}{l}
|\def\version{draft}|\\
|\input{childdoc.def}|\\
|\childdocforward{|\textit{main}|}|
\end{tabular}
\end{center}
%
Likewise, the following files |final|\textit{nn}|.tex|
compile the final version of the child document
|child|\textit{nn}|.tex|:
%
\begin{center}
\begin{tabular}{l}
|\def\version{final}|\\
|\input{childdoc.def}|\\
|\childdocforwardprefix{final}{child}|
\end{tabular}
\end{center}
%

Note that when several versions of a main file and/or of each child file
are to be generated, it may be convenient to set up a |Makefile| or
shell script to automatise the process.

%%%%%%%%%%%%%%%%%%%%%%%%%%%%%%%%%%%%%%%%%%%%%%%%%%%%%%%%%%%%%%%%%%%%%%%%%%%%%%%%
\subsection{Command Line Processing}
\label{sec:commandline}

The effect of redirection files can also be achieved by invoking
the \LaTeX{} compiler with a more elaborate command line.
Most conveniently this should be done as part
of a shell script or a |Makefile|.

When using \textsf{childdoc} in the main file, the following
command lines effectively perform a redirection
(note that depending on the shell being used,
backslashes may have to be doubled: `|\|' $\to$ `|\\|'):
%
\begin{center}
|... -jobname "|\textit{target}|" |\\|"|[\textit{flags}]%
|\input{childdoc.def}\childdocforward[|\textit{main}|]{|\textit{dest}|}"|
\end{center}
%
Here \textit{target} is the name of the output file,
\textit{main} is the name of the main file
and \textit{dest} is the name of the main or child file to be processed
(all filenames without extensions).
The optional argument \textit{main} can be omitted
if \textit{main} matches \textit{dest}.
Optionally, compilation \textit{flags} can be defined via |\def| commands.
This command line makes the \TeX{} engine believe
it is compiling the file \textit{target}
whose content is specified as the latter parameter.
The provided code then forwards the processing to
\textit{main} or \textit{dest} as described in \secref{sec:forward}.

%%%%%%%%%%%%%%%%%%%%%%%%%%%%%%%%%%%%%%%%%%%%%%%%%%%%%%%%%%%%%%%%%%%%%%%%%%%%%%%%
\subsection{Include by Input}
\label{sec:input}

Including child documents by |\include| has some restrictions by design.
Most notably, the content of a child document always occupies
its own set of pages; pages cannot be shared between child documents.
Usually, this behaviour makes perfect sense
because each child document contain an essential part of the document.
However, in some situations it may be desirable to compose
a document from a collection of parts
without having mandatory page breaks between then.
For this case, the package
provides a mechanism to include parts
by |\input| which can also be processed individually.
However, by construction this mechanism
requires manual handling of the content to be output.

%%%%%%%%%%%%%%%%%%%%%%%%%%%%%%%%%%%%%%%%
\DescribeMacro{\ifchilddocmanual}
The main file should be prepared as usual, see \secref{sec:include}.
However, the document body must make a distinction
between processing of an individual part and of the main document, e.g.:
%
\begin{center}
\begin{tabular}{l}
|\ifchilddocmanual|\\
|\input{\childdocname}|\\
|\||else|\\
\textit{document body with }|\input{|\textit{part}|}|\\
|\||fi|
\end{tabular}
\end{center}
%
The conditional |\ifchilddocmanual| is true whenever
a part to be included by |\input| is being compiled,
and the name of the part is stored in |\childdocname|.

%%%%%%%%%%%%%%%%%%%%%%%%%%%%%%%%%%%%%%%%
\DescribeMacro{\childdocby}
Each part to be included by |\input| should start with:
%
\begin{center}
\begin{tabular}{l}
|\input{childdoc.def}|\\
|\childdocby{|\textit{main}|}|\\
\end{tabular}
\end{center}
%
The directive |\childdocby| is similar to |\childdocof|
described in \secref{sec:include},
but the subsequent selection of content must be done manually.
To that end, both |\ifchilddoc| and |\ifchilddocmanual|
will be true upon processing of a part,
and the name of the part is stored in |\childdocname|.
Note that |\jobname| will be set to the filename of the current part
so that each part receives an individual |.aux| file
that does not interfere with the |.aux| file(s) of the main document.
This behaviour can be altered by the alternative form
|\childdocby[*]{|\textit{main}|}| (with a non-empty optional argument)
which uses the |.aux| file of the main document
by setting |\jobname| to \textit{main}.

%%%%%%%%%%%%%%%%%%%%%%%%%%%%%%%%%%%%%%%%%%%%%%%%%%%%%%%%%%%%%%%%%%%%%%%%%%%%%%%%
\subsection{Driver Development}
\label{sec:driver}

The \textsf{childdoc} mechanism can also be use for the development
of definition files such as \LaTeX{} styles or classes.
This case differs from the above setup with multiple parts
included by |\include| in that no |\includeonly| should be invoked.
This can be achieved by starting the include file
(before |\ProvidesPackage|) with:
%
\begin{center}
\begin{tabular}{l}
|\input{childdoc.def}|\\
|\childdocforward{|\textit{main}|}|\\
\end{tabular}
\end{center}
%
or alternatively with:
%
\begin{center}
\begin{tabular}{l}
|\input{childdoc.def}|\\
|\childdocby{|\textit{main}|}|\\
\end{tabular}
\end{center}
%
Both forms have slightly different effects as described above.
The main file is prepared as usual, see \secref{sec:include}.

%%%%%%%%%%%%%%%%%%%%%%%%%%%%%%%%%%%%%%%%%%%%%%%%%%%%%%%%%%%%%%%%%%%%%%%%%%%%%%%%
\subsection{Legacy Detection}
\label{sec:detection}

The directive |\childdocmain| in the main file can detect
whether the complete document or merely a child is to be compiled
even without using the directive |\childdocof|.
This method is deprecated because it is less robust
and there is no compelling reason to use it;
it is merely provided for backward compatibility
and it may be removed in future versions.

If the detection mechanism is to be used,
it is mandatory to correctly specify
the filename of the main file as the argument of |\childdocmain|:
%
\begin{center}
\begin{tabular}{l}
|\input{childdoc.def}|\\
|\childdocmain{|\textit{main}|}|\\
\end{tabular}
\end{center}
%
If |\jobname| does not match the argument \textit{main} of |\childdocmain|,
it is assumed that |\jobname| points to the child file to be compiled.
When using |\childdocmain| with the main file specified as argument,
it suffices to start a child file
with just |\input{|\textit{main}|}|
without loading of the package and using |\childdocof|.
If instead all processing is done
with the appropriate \textsf{childdoc} directives,
the argument of \textit{main} of |\childdocmain| can be empty.

An alternative version of the command line processing described
in \secref{sec:commandline} using the detection mechanism reads:
%
\begin{center}
|... -jobname "|\textit{target}|" "|[\textit{flags}]%
[|\def\jobname{|\textit{dest}|}|]|\input{|\textit{main}|}"|
\end{center}

%%%%%%%%%%%%%%%%%%%%%%%%%%%%%%%%%%%%%%%%%%%%%%%%%%%%%%%%%%%%%%%%%%%%%%%%%%%%%%%%
\subsection{Manual Code}
\label{sec:manual}

In case one cannot be certain whether the definitions file |childdoc.def|
is installed on the target \TeX{} distribution
and one prefers not to ship it,
it is conceivable to paste a few relevant commands into the sources.

To that end, drop all statements |\input{childdoc.def}|
and perform the replacements as outlined below.
Instead of |\childdocmain{|\textit{main}|}| add the following code
to the top of the main file:
%
\begin{center}
\begin{tabular}{l}
|\||ifdefined\childdocname\endinput\||fi\newif\ifchilddoc|\\
|\edef\childdocname{\scantokens\expandafter{\jobname\noexpand}}|\\
|\def\childdocmain{|\textit{main}|}\||ifx\childdocmain\childdocname\||else|\\
|\childdoctrue\includeonly{\childdocname}\let\jobname\childdocmain\||fi|\\
\end{tabular}
\end{center}
%
Instead of |\childdocof{|\textit{main}|}| just include the main file
at the top of each child file:
%
\begin{center}
|\input{|\textit{main}|}|
\end{center}
%
A simple redirection |\childdocforward{|\textit{dest}|}| is achieved by:
%
\begin{center}
|\def\jobname{|\textit{dest}|}\input{\jobname}|
\end{center}
%
The redirection with prefix
|\childdocforwardprefix[|\textit{prefix}|]{|\textit{dest}|}|
is accomplished by:
%
\begin{center}
\begin{tabular}{l}
|{\edef\jobname{\scantokens\expandafter{\jobname\noexpand}}|\\
|\def\redirectjob |\textit{prefix}|#1~~~{\gdef\jobname{|\textit{dest}|#1}}|\\
|\expandafter\redirectjob\jobname~~~}\input{\jobname}|
\end{tabular}
\end{center}

In an alternative approach,
child documents can be compiled by a specific command line
without additional code or specific definitions:
%
\begin{center}
|... -jobname "|\textit{target}|" "|[\textit{flags}]%
|\includeonly{|\textit{dest}|}\input{|\textit{main}|}"|
\end{center}
%

%%%%%%%%%%%%%%%%%%%%%%%%%%%%%%%%%%%%%%%%%%%%%%%%%%%%%%%%%%%%%%%%%%%%%%%%%%%%%%%%
%%%%%%%%%%%%%%%%%%%%%%%%%%%%%%%%%%%%%%%%%%%%%%%%%%%%%%%%%%%%%%%%%%%%%%%%%%%%%%%%
\section{Information}

%%%%%%%%%%%%%%%%%%%%%%%%%%%%%%%%%%%%%%%%%%%%%%%%%%%%%%%%%%%%%%%%%%%%%%%%%%%%%%%%
\subsection{Copyright}

Copyright \copyright{} 2017--2018 Niklas Beisert

This work may be distributed and/or modified under the
conditions of the \LaTeX{} Project Public License, either version 1.3
of this license or (at your option) any later version.
The latest version of this license is in
  \url{http://www.latex-project.org/lppl.txt}
and version 1.3 or later is part of all distributions of \LaTeX{}
version 2005/12/01 or later.

This work has the LPPL maintenance status `maintained'.

The Current Maintainer of this work is Niklas Beisert.

This work consists of the files |README.txt|, |childdoc.ins| and |childdoc.dtx|
as well as the derived files |childdoc.def|, |cdocsamp.tex|
with |cdocsch1.tex|, |cdocsch2.tex|, |cdocspt3.tex|, |cdocspt4.tex|,
|cdocsdrf.tex|, |cdocsfn1.tex|, |cdocsfn2.tex|
as well as |childdoc.pdf|.

%%%%%%%%%%%%%%%%%%%%%%%%%%%%%%%%%%%%%%%%%%%%%%%%%%%%%%%%%%%%%%%%%%%%%%%%%%%%%%%%
\subsection{Files and Installation}

The package consists of the files:
%
\begin{center}
\begin{tabular}{ll}
    |README.txt|   & readme file \\
    |childdoc.ins| & installation file \\
    |childdoc.dtx| & source file \\
    |childdoc.def| & definition file \\
    |cdocsamp.tex| & sample main file \\
    |cdocsch1.tex| & sample include file \\
    |cdocsch2.tex| & sample include file \\
    |cdocspt3.tex| & sample part file \\
    |cdocspt4.tex| & sample part file \\
    |cdocsdrf.tex| & sample redirection file \\
    |cdocsfn1.tex| & sample redirection file \\
    |cdocsfn2.tex| & sample redirection file \\
    |childdoc.pdf| & manual
\end{tabular}
\end{center}
%
The distribution consists of the files
|README.txt|, |childdoc.ins| and |childdoc.dtx|.
%
\begin{itemize}
\item
Run (pdf)\LaTeX{} on |childdoc.dtx|
to compile the manual |childdoc.pdf| (this file).
\item
Run \LaTeX{} on |childdoc.ins| to create the definitions file |childdoc.def|
and the sample |cdocsamp.tex| with include files
|cdocsch1.tex|, |cdocsch2.tex|, |cdocspt3.tex|, |cdocspt4.tex|,
|cdocsdrf.tex|, |cdocsfn1.tex|, |cdocsfn2.tex|.
Then copy the file |childdoc.def| to an appropriate directory of your \LaTeX{}
distribution, e.g.\ \textit{texmf-root}|/tex/latex/childdoc|.
\end{itemize}

%%%%%%%%%%%%%%%%%%%%%%%%%%%%%%%%%%%%%%%%%%%%%%%%%%%%%%%%%%%%%%%%%%%%%%%%%%%%%%%%
\subsection{Related CTAN Packages}

There are several other packages which offer a similar functionality:
%
\begin{itemize}
\item
The packages
\href{http://ctan.org/pkg/docmute}{\textsf{docmute}},
\href{http://ctan.org/pkg/includex}{\textsf{includex}} and
\href{http://ctan.org/pkg/standalone}{\textsf{standalone}}
provide commands to include only the document body of
a child file thus allowing both files to be compiled individually.
\item
The packages \href{http://ctan.org/pkg/subdocs}{\textsf{subdocs}}
and \href{http://ctan.org/pkg/subfiles}{\textsf{subfiles}}
provide structures in which the main and child documents can be
encapsulated and allowing them to be compiled individually.
The inclusion mechanism is different from the conventional |\include|.
\item
The package \href{http://ctan.org/pkg/combine}{\textsf{combine}}
is an elaborate solution to combine several documents into one.
\end{itemize}
%
See also the CTAN topic \href{http://ctan.org/topic/subdocs}{\textsf{subdocs}}
for further related packages.
The present package differs from the above solutions in that
a document structure constructed with the conventional |\include| mechanism
just needs two extra commands at the top of every file
such that all constituent files can be compiled individually.

%%%%%%%%%%%%%%%%%%%%%%%%%%%%%%%%%%%%%%%%%%%%%%%%%%%%%%%%%%%%%%%%%%%%%%%%%%%%%%%%
%\subsection{Feature Suggestions}
%
%The following is a list of features which may be useful for future
%versions of this package:
%%
%\begin{itemize}
%\item
%\ldots
%\end{itemize}

%%%%%%%%%%%%%%%%%%%%%%%%%%%%%%%%%%%%%%%%%%%%%%%%%%%%%%%%%%%%%%%%%%%%%%%%%%%%%%%%
\subsection{Revision History}

%%%%%%%%%%%%%%%%%%%%%%%%%%%%%%%%%%%%%%%%
\paragraph{v2.0:} 2018/12/30

\begin{itemize}
\item
immediate forward processing
\item
added |\childdocby| mechanism
\item
manual restructured
\end{itemize}

%%%%%%%%%%%%%%%%%%%%%%%%%%%%%%%%%%%%%%%%
\paragraph{v1.6:} 2018/01/17

\begin{itemize}
\item
application for development of include files
\item
corrections to manual
\end{itemize}

%%%%%%%%%%%%%%%%%%%%%%%%%%%%%%%%%%%%%%%%
\paragraph{v1.5:} 2017/05/21

\begin{itemize}
\item
more complete structuring introduced
\item
|\childdocof| introduced
\item
|\childdoc| renamed to |\childdocmain|
\item
|\childredirect| renamed to |\childdocforward| and |\childdocforwardprefix|
and functionality expanded
\end{itemize}

%%%%%%%%%%%%%%%%%%%%%%%%%%%%%%%%%%%%%%%%
\paragraph{v1.0:} 2017/04/27

\begin{itemize}
\item
manual and install package
\item
first version published on CTAN
\end{itemize}

%%%%%%%%%%%%%%%%%%%%%%%%%%%%%%%%%%%%%%%%
\paragraph{v0.6:} 2017/04/26

\begin{itemize}
\item
redirection mechanism added
\end{itemize}

%%%%%%%%%%%%%%%%%%%%%%%%%%%%%%%%%%%%%%%%
\paragraph{v0.5:} 2017/04/26

\begin{itemize}
\item
functionality in definition file
\end{itemize}


%%%%%%%%%%%%%%%%%%%%%%%%%%%%%%%%%%%%%%%%%%%%%%%%%%%%%%%%%%%%%%%%%%%%%%%%%%%%%%%%
%%%%%%%%%%%%%%%%%%%%%%%%%%%%%%%%%%%%%%%%%%%%%%%%%%%%%%%%%%%%%%%%%%%%%%%%%%%%%%%%
%%%%%%%%%%%%%%%%%%%%%%%%%%%%%%%%%%%%%%%%%%%%%%%%%%%%%%%%%%%%%%%%%%%%%%%%%%%%%%%%
\appendix

\settowidth\MacroIndent{\rmfamily\scriptsize 000\ }

 \DocInput{childdoc.dtx}

\end{document}
%</driver>
% \fi
%
% %%%%%%%%%%%%%%%%%%%%%%%%%%%%%%%%%%%%%%%%%%%%%%%%%%%%%%%%%%%%%%%%%%%%%%%%%%%%%%
% %%%%%%%%%%%%%%%%%%%%%%%%%%%%%%%%%%%%%%%%%%%%%%%%%%%%%%%%%%%%%%%%%%%%%%%%%%%%%%
% \section{Sample}
%\iffalse
%<*samplemain>
%\fi
%
% The following presents a sample document
% with two chapters, two parts, a title page,
% a compile flag as well as three forwarding files to set the flag.
% It consists of eight |.tex| files:
% \begin{center}
% \begin{tabular}{ll}
% |cdocsamp.tex|&main file\\
% |cdocsch1.tex|&include file for chapter 1\\
% |cdocsch2.tex|&include file for chapter 2\\
% |cdocspt3.tex|&include file for part 3\\
% |cdocspt4.tex|&include file for part 4\\
% |cdocsdrf.tex|&forwarding file for main file in draft mode\\
% |cdocsfi1.tex|&forwarding file for final version of chapter 1\\
% |cdocsfi2.tex|&forwarding file for final version of chapter 2\\
% \end{tabular}
% \end{center}
% Each of the eight files can be compiled directly by the \LaTeX{} compiler.
%
% %%%%%%%%%%%%%%%%%%%%%%%%%%%%%%%%%%%%%%
% \paragraph{Main File.}
%
% The main file is called |cdocsamp.tex|.
%
% Load the \textsf{childdoc} definitions and
% declare the filename for the main document:
%    \begin{macrocode}
\input{childdoc.def}
\childdocmain{}
%    \end{macrocode}

% Optional override for |\version| flag:
%    \begin{macrocode}
%%\ifchilddoc\else\providecommand{\version}{draft}\fi
%    \end{macrocode}

% Define the default values for the |\version| flag
% (|final| for the main file and |draft| for childs):
%    \begin{macrocode}
\ifchilddoc
\providecommand{\version}{draft}
\else
\providecommand{\version}{final}
\fi
%    \end{macrocode}

% Load the standard document class:
%    \begin{macrocode}
\documentclass[12pt]{article}
%    \end{macrocode}

% Start the document body:
%    \begin{macrocode}
\begin{document}
%    \end{macrocode}

% Declare a title page.
% Print title, part of document being processed and version flag:
%    \begin{macrocode}
\addtocounter{page}{-1}
\begin{center}
{\LARGE\bfseries{}childdoc example\par}
\vspace{1cm}
\ifchilddoc
\ifchilddocmanual part\else chapter\fi:
`\childdocname' of `\childdocjob'\par
\else
main document: `\childdocjob'\par
\fi
version: \version\par
\end{center}
\newpage
%    \end{macrocode}

% Manually include selected file,
% otherwise process as usual:
%    \begin{macrocode}
\ifchilddocmanual
\section*{part `\childdocname'}
\input{\childdocname}
\else
%    \end{macrocode}

% Include the two chapters:
%    \begin{macrocode}
\include{cdocsch1}
\include{cdocsch2}
%    \end{macrocode}

% Include the two parts unless only chapters should be displayed:
%    \begin{macrocode}
\ifchilddoc\else
\section{part three}
\input{cdocspt3}
\section{part four}
\input{cdocspt4}
\fi
%    \end{macrocode}

% Process as usual until here:
%    \begin{macrocode}
\fi
%    \end{macrocode}

% End of document body:
%    \begin{macrocode}
\end{document}
%    \end{macrocode}
%\iffalse
%</samplemain>
%\fi
%
% %%%%%%%%%%%%%%%%%%%%%%%%%%%%%%%%%%%%%%
% \paragraph{Chapter Include Files.}
%
% The include files are called |cdocsch1.tex| and |cdocsch2.tex|.
%
%\iffalse
%<*samplechap1|samplechap2>
%\fi

% Optional override for |\version| flag:
%    \begin{macrocode}
%%\providecommand{\version}{final}
%    \end{macrocode}

% Include the main document:
%    \begin{macrocode}
\input{childdoc.def}
\childdocof{cdocsamp}
%    \end{macrocode}

%\iffalse
%</samplechap1|samplechap2>
%\fi
%
%\iffalse
%<*samplechap1>
%\fi
% Some text for chapter 1:
%    \begin{macrocode}
\section{one}
some text in chapter one
%    \end{macrocode}

%\iffalse
%</samplechap1>
%\fi
% Some text for chapter 2:
%\iffalse
%<*samplechap2>
%\fi
%    \begin{macrocode}
\section{two}
more text in chapter two
%    \end{macrocode}

%\iffalse
%</samplechap2>
%\fi
%
% %%%%%%%%%%%%%%%%%%%%%%%%%%%%%%%%%%%%%%
% \paragraph{Part Include Files.}
%
% The include files are called |cdocspt3.tex| and |cdocspt4.tex|.
%
%\iffalse
%<*samplepart3|samplepart4>
%\fi

% Optional override for |\version| flag:
%    \begin{macrocode}
%%\providecommand{\version}{final}
%    \end{macrocode}

% Include the main document:
%    \begin{macrocode}
\input{childdoc.def}
\childdocby{cdocsamp}
%    \end{macrocode}

%\iffalse
%</samplepart3|samplepart4>
%\fi
%
%\iffalse
%<*samplepart3>
%\fi
% Some text for part 3:
%    \begin{macrocode}
some text in part three
%    \end{macrocode}

%\iffalse
%</samplepart3>
%\fi
% Some text for part 4:
%\iffalse
%<*samplepart4>
%\fi
%    \begin{macrocode}
more text in part four
%    \end{macrocode}

%\iffalse
%</samplepart4>
%\fi
%
% %%%%%%%%%%%%%%%%%%%%%%%%%%%%%%%%%%%%%%
% \paragraph{Forwarding for a Complete Draft.}
%
% The following forwarding file |cdocsdrf.tex|
% compiles the main document in draft mode:
%\iffalse
%<*sampledraft>
%\fi
%    \begin{macrocode}
\def\version{draft}
\input{childdoc.def}
\childdocforward{cdocsamp}
%    \end{macrocode}

%\iffalse
%</sampledraft>
%\fi
%
% %%%%%%%%%%%%%%%%%%%%%%%%%%%%%%%%%%%%%%
% \paragraph{Forwarding for Final Version of the Chapters.}
%
% The following forwarding files |cdocsfn1.tex| and |cdocsfn2.tex|
% (with identical content)
% compile the final versions of the child documents
% |cdocsch1.tex| and |cdocsch2.tex|, respectively:
%\iffalse
%<*samplefinal>
%\fi
%    \begin{macrocode}
\def\version{final}
\input{childdoc.def}
\childdocforwardprefix[cdocsamp]{cdocsfn}{cdocsch}
%    \end{macrocode}

%\iffalse
%</samplefinal>
%\fi
%
% %%%%%%%%%%%%%%%%%%%%%%%%%%%%%%%%%%%%%%
% \paragraph{Command Line Processing.}
%
% The following three command lines generate the output files
% |cdocscld|, |cdocscl1| and |cdocscl2|
% which should be identical to
% |cdocsdrf|, |cdocsch1| and |cdocsfn2|, respectively:
% \begin{center}
% \begin{tabular}{l}
% |latex -jobname cdocscld \|\\
% |  "\def\version{draft}\input{childdoc.def}\childdocforward{cdocsamp}"|\\
% |latex -jobname cdocscl1 \|\\
% |  "\input{childdoc.def}\childdocforward[cdocsamp]{cdocsch1}"|\\
% |latex -jobname cdocscl2 \|\\
% |  "\def\version{final}\input{childdoc.def}\childdocforward{cdocsch2}"|
% \end{tabular}
% \end{center}
% Note that the trailing backslash on each first line
% merely continues the input to the second line
% (for convenient cut ant paste).
% Furthermore, the command |latex| can be replaced by any
% of its alternative versions such as |pdflatex|.
%
% %%%%%%%%%%%%%%%%%%%%%%%%%%%%%%%%%%%%%%%%%%%%%%%%%%%%%%%%%%%%%%%%%%%%%%%%%%%%%%
% %%%%%%%%%%%%%%%%%%%%%%%%%%%%%%%%%%%%%%%%%%%%%%%%%%%%%%%%%%%%%%%%%%%%%%%%%%%%%%
% \section{Implementation}
%\iffalse
%<*package>
%\fi
%
% This section describes the definitions file |childdoc.def|.

% The definitions cannot be loaded using |\usepackage| or |\RequirePackage|
% which has a mechanism to prevent loading a style file more than once.
% When loading the definitions by means of |\input|
% multiple instances have to be prevented manually:
%\iffalse
%This code needs to be before the `\ProvidesFile' directive
%which is defined at the beginning of this file.
%Therefore it is also placed there and commented out here.
%</package>
%<*discard>
%\fi
%    \begin{macrocode}
\ifdefined\childdocmain\endinput\fi
%    \end{macrocode}
%\iffalse
%</discard>
%<*package>
%\fi
%
% \macro{\ifchilddoc}
% \macro{\ifchilddocmanual}
% The conditional |\ifchilddoc| tells whether a
% child (true) or main (false) document is being compiled.
% The conditional |\ifchilddocmanual| tells whether
% the |\includeonly| mechanism is used (false) or
% the selection of child files must be performed manually (true).
% The definitions initialise to false:
%    \begin{macrocode}
\newif\ifchilddoc
\newif\ifchilddocmanual
%    \end{macrocode}

% \macro{\childdocname}
% \macro{\childdocjob}
% The macro |\childdocname| stores the name of the main document
% to be compiled. The macro |\childdocjob| stores the name of
% the document on which the \LaTeX{} compiler was originally invoked.
% The content of |\jobname| cannot be compared
% to filenames specified in the source due to different catcodes.
% The following code rescans |\jobname|, stores the result
% in |\childdocname| and saves a copy in |\childdocjob|:
%    \begin{macrocode}
\edef\childdocname{\scantokens\expandafter{\jobname\noexpand}}
\let\childdocjob\childdocname
%    \end{macrocode}

% \macro{\childdocdisable}
% The macro |\childdocdisable| prevents the main file
% from being processed more than once.
% At this stage, the main document command |\childdocmain|
% is assumed to be called once again where it should do nothing.
% Any subsequent call to it should prevent
% a secondary processing of the main document
% It overwrites the forwarding commands
% |\childdocof| and |\childdocforward|
% with empty macros to prevent further inclusions of the main document:
%    \begin{macrocode}
\newcommand{\childdocdisable}
{
  \renewcommand{\childdocmain}[1]{\renewcommand{\childdocmain}[1]{\endinput}}
  \renewcommand{\childdocof}[1]{}
  \renewcommand{\childdocby}[2][]{}
  \renewcommand{\childdocforward}[2][]{}
  \renewcommand{\childdocdisable}{}
}
%    \end{macrocode}

% \macro{\childdocmain}
% The macro |\childdocmain| is to be called at the top of the main file
% with nothing or the main filename (without extension) as argument.
% First, it breaks loops.
% If the argument is not empty and does not match |\childdocname|
% (which is set by the first inclusion of |childdoc.def|),
% |\ifchilddoc| is set to true, |\includeonly| is applied to the child file
% and |\jobname| is set to the main file
% (for proper handling of |.aux| files):
%    \begin{macrocode}
\newcommand{\childdocmain}[1]
{
  \childdocdisable\childdocmain{}
  \if?#1?\else
    \begingroup
      \def\childdoctmp{#1}
      \ifx\childdoctmp\childdocname
        \def\childdoctmp{}
      \else
        \def\childdoctmp
        {
          \childdoctrue
          \includeonly{\childdocname}
          \def\childdocjob{#1}
          \def\jobname{#1}
        }
      \fi
      \expandafter
    \endgroup
    \childdoctmp
  \fi
}
%    \end{macrocode}

% \macro{\childdocof}
% The command |\childdocof| redirects
% compilation to the main file |#1|.
%    \begin{macrocode}
\newcommand{\childdocof}[1]
{
  \childdocdisable
  \childdoctrue
  \includeonly{\childdocname}
  \def\jobname{#1}
  \def\childdocjob{#1}
  \input{#1}
}
%    \end{macrocode}

% \macro{\childdocby}
% The command |\childdocby| ....
%    \begin{macrocode}
\newcommand{\childdocby}[2][]
{
  \childdocdisable
  \childdoctrue
  \childdocmanualtrue
  \if?#1?\else
    \def\jobname{#2}
  \fi
  \def\childdocjob{#2}
  \input{#2}
  \endinput
}
%    \end{macrocode}

% \macro{\childdocforward}
% The command |\childdocforward| redirects
% compilation to the main file or
% (if the optional argument is given) a child file.
% Parameters are set as if the main file
% or a child file starting with |\childdocof| was compiled.
% Then compilation is handed over to the main file:
%    \begin{macrocode}
\newcommand{\childdocforward}[2][]
{
  \begingroup
    \if?#1?
      \def\childdoctmp
      {
        \def\childdocname{#2}
        \def\childdocjob{#2}
        \def\jobname{#2}
        \input{#2}
        \endinput
      }
    \else
      \def\childdoctmp
      {
        \childdocdisable
        \def\childdocname{#2}
        \childdoctrue
        \includeonly{#2}
        \def\childdocjob{#1}
        \def\jobname{#1}
        \input{#1}
        \endinput
      }
    \fi
    \expandafter
  \endgroup
  \childdoctmp
}
%    \end{macrocode}

% \macro{\childdocforwardprefix}
% The command |\childdocforwardprefix| redirects
% compilation to the main or a child file by means of a pattern.
% The prefix |#1| in the current filename is replaced by |#2|
% and the suffix of the current filename is kept
% (it is assumed that the filename does not contain the substring `|~~~|'
% which is used as a delimiter).
% Compilation is handed over to the new file by |\childdocforward|:
%    \begin{macrocode}
\newcommand{\childdocforwardprefix}[3][]
{
  \begingroup
    \def\childdocextract #2##1~~~{\def\childdoctmp{\childdocforward[#1]{#3##1}}}
    \expandafter\childdocextract\childdocname~~~
    \expandafter
  \endgroup
  \childdoctmp
}
%    \end{macrocode}

% \macro{\childdoc}
% The deprecated macro |\childdoc| is a legacy version of |\childdocmain|:
%    \begin{macrocode}
\newcommand{\childdoc}{\childdocmain}
%    \end{macrocode}

% \macro{\childdocredirect}
% The deprecated macro |\childdocredirect| is a legacy version
% of |\childdocforward| and |\childdocforwardprefix|:
%    \begin{macrocode}
\newcommand{\childdocredirect}[2][]
{
  \begingroup
    \if?#1?
      \def\childdoctmp{\childdocforward{#2}}
    \else
      \def\childdoctmp{\childdocforwardprefix{#1}{#2}}
    \fi
    \expandafter
  \endgroup
  \childdoctmp
}
%    \end{macrocode}

%\iffalse
%</package>
%\fi
%
\endinput

\childdocby{cdocsamp}
%    \end{macrocode}

%\iffalse
%</samplepart3|samplepart4>
%\fi
%
%\iffalse
%<*samplepart3>
%\fi
% Some text for part 3:
%    \begin{macrocode}
some text in part three
%    \end{macrocode}

%\iffalse
%</samplepart3>
%\fi
% Some text for part 4:
%\iffalse
%<*samplepart4>
%\fi
%    \begin{macrocode}
more text in part four
%    \end{macrocode}

%\iffalse
%</samplepart4>
%\fi
%
% %%%%%%%%%%%%%%%%%%%%%%%%%%%%%%%%%%%%%%
% \paragraph{Forwarding for a Complete Draft.}
%
% The following forwarding file |cdocsdrf.tex|
% compiles the main document in draft mode:
%\iffalse
%<*sampledraft>
%\fi
%    \begin{macrocode}
\def\version{draft}
% \iffalse
%
% childdoc.dtx Copyright (C) 2017-2018 Niklas Beisert
%
% This work may be distributed and/or modified under the
% conditions of the LaTeX Project Public License, either version 1.3
% of this license or (at your option) any later version.
% The latest version of this license is in
%   http://www.latex-project.org/lppl.txt
% and version 1.3 or later is part of all distributions of LaTeX
% version 2005/12/01 or later.
%
% This work has the LPPL maintenance status `maintained'.
%
% The Current Maintainer of this work is Niklas Beisert.
%
% This work consists of the files childdoc.dtx and childdoc.ins
% and the derived files childdoc.def and cdocsamp.tex with
% cdocsch1.tex, cdocsch2.tex, cdocsdrf.tex, cdocsfn1.tex, cdocsfn2.tex.
%
%<package>\ifdefined\childdocmain\endinput\fi
%<package>\ProvidesFile{childdoc.def}[2018/12/30 v2.0 child document driver]
%<samplemain>\ProvidesFile{cdocsamp.tex}[2018/12/30 v2.0 sample for childdoc]
%<*driver>
%\ProvidesFile{childdoc.drv}[2018/12/30 v2.0 childdoc reference manual file]
\PassOptionsToClass{10pt,a4paper}{article}
\documentclass{ltxdoc}

\usepackage[margin=35mm]{geometry}
\usepackage{hyperref}
\usepackage{hyperxmp}
\usepackage[usenames]{color}

\hypersetup{colorlinks=true}
\hypersetup{pdfstartview=FitH}
\hypersetup{pdfpagemode=UseNone}
\hypersetup{pdfsource={}}
\hypersetup{pdflang={en-UK}}
\hypersetup{pdfcopyright={Copyright 2017-2018 Niklas Beisert.
  This work may be distributed and/or modified under the
  conditions of the LaTeX Project Public License, either version 1.3
  of this license or (at your option) any later version.}}
\hypersetup{pdflicenseurl={http://www.latex-project.org/lppl.txt}}
\hypersetup{pdfcontactaddress={ETH Zurich, ITP, HIT K,
  Wolfgang-Pauli-Strasse 27}}
\hypersetup{pdfcontactpostcode={8093}}
\hypersetup{pdfcontactcity={Zurich}}
\hypersetup{pdfcontactcountry={Switzerland}}
\hypersetup{pdfcontactemail={nbeisert@itp.phys.ethz.ch}}
\hypersetup{pdfcontacturl={http://people.phys.ethz.ch/\xmptilde nbeisert/}}

\newcommand{\secref}[1]{\hyperref[#1]{section \ref*{#1}}}

\parskip1ex
\parindent0pt
\let\olditemize\itemize
\def\itemize{\olditemize\parskip0pt}

\begin{document}

\title{The \textsf{childdoc} Package}
\hypersetup{pdftitle={The childdoc Package}}
\author{Niklas Beisert\\[2ex]
  Institut f\"ur Theoretische Physik\\
  Eidgen\"ossische Technische Hochschule Z\"urich\\
  Wolfgang-Pauli-Strasse 27, 8093 Z\"urich, Switzerland\\[1ex]
  \href{mailto:nbeisert@itp.phys.ethz.ch}
  {\texttt{nbeisert@itp.phys.ethz.ch}}}
\hypersetup{pdfauthor={Niklas Beisert}}
\hypersetup{pdfsubject={Manual for the LaTeX2e Package childdoc}}
\date{30 December 2018, \textsf{v2.0}}
\maketitle

\begin{abstract}\noindent
\textsf{childdoc} is a \LaTeXe{} package
that enables the direct compilation
of document sections included by |\include|
to individual files.
\end{abstract}

\begingroup
\parskip0ex
\tableofcontents
\endgroup

%%%%%%%%%%%%%%%%%%%%%%%%%%%%%%%%%%%%%%%%%%%%%%%%%%%%%%%%%%%%%%%%%%%%%%%%%%%%%%%%
%%%%%%%%%%%%%%%%%%%%%%%%%%%%%%%%%%%%%%%%%%%%%%%%%%%%%%%%%%%%%%%%%%%%%%%%%%%%%%%%
\section{Introduction}

\LaTeX{} provides a mechanism to structure a large document (such as a book)
into a main file and several child files (containing the chapters)
using the |\include| command.
This mechanism is beneficial for documents
which span hundreds of pages in order to
make the source file(s) more manageable.
Moreover, compilation can be restricted to
selected child files by means of the |\includeonly| command.
The latter feature can be used to reduce the compilation time while editing
(this was significantly more useful in the earlier days of \LaTeX{})
or to generate a smaller document which is easier to navigate.
Another application of |\includeonly| is to generate
documents consisting of selected parts of the complete document.

However, there are a few drawbacks of the plain |\include| mechanism:
\begin{itemize}
\item
The child files cannot be compiled on their own,
they can only be compiled via the main file.
A naive editing environment
(such as a text editor with an option
to have the current file processed by \LaTeX)
may require one to switch to the main file before compiling;
attempting to compile the child file produces errors.
\item
The main file must be modified (each time)
to adjust the |\includeonly| command
to the present needs. This easily leaves the main file in a messy state.
\item
The generated document will always carry the filename
of the main document. This is inconvenient if
several child files are to be compiled and
to be kept for distribution.
\end{itemize}

The present package provides a simple interface
to make child files individually compilable by \LaTeX{}.
Compiling a child file then has the same effect as compiling
the main file with an |\includeonly| command
to select the appropriate child.
Moreover the generated document will carry the name of the child
rather than the main file.
This resolves all three above issues.

This feature is meant to make the editing of books,
thesis documents and lecture notes somewhat more convenient.
However, the package can also be used efficiently for
composing a series of documents (such as exercise sheets)
which are typically distributed individually.
It then assists the author in generating the individual documents
(potentially in different versions)
as well as a document containing the collected series.
Another application is in developing style files
or other kinds of included material
where compilation of the style file could redirect
to a sample or test file.

%%%%%%%%%%%%%%%%%%%%%%%%%%%%%%%%%%%%%%%%%%%%%%%%%%%%%%%%%%%%%%%%%%%%%%%%%%%%%%%%
%%%%%%%%%%%%%%%%%%%%%%%%%%%%%%%%%%%%%%%%%%%%%%%%%%%%%%%%%%%%%%%%%%%%%%%%%%%%%%%%
\section{Usage}

First of all, the package \textsf{childdoc} is \emph{not} a standard
\LaTeXe{} |.sty| style file! Therefore it needs to be invoked in
a non-standard way.

%%%%%%%%%%%%%%%%%%%%%%%%%%%%%%%%%%%%%%%%%%%%%%%%%%%%%%%%%%%%%%%%%%%%%%%%%%%%%%%%
\subsection{Included Files}
\label{sec:include}

%%%%%%%%%%%%%%%%%%%%%%%%%%%%%%%%%%%%%%%%
\DescribeMacro{\childdocmain}
To use the package, add the commands
\begin{center}
\begin{tabular}{l}
|\input{childdoc.def}|\\
|\childdocmain{}|\\
\end{tabular}
\end{center}
at the very top of the main \LaTeX{} file,
in particular \emph{before} the |\documentclass| statement!
The argument of |\childdocmain| should be left empty
(but it must be present).

%%%%%%%%%%%%%%%%%%%%%%%%%%%%%%%%%%%%%%%%
\DescribeMacro{\childdocof}
Furthermore, add the commands
\begin{center}
\begin{tabular}{l}
|\input{childdoc.def}|\\
|\childdocof{|\textit{main}|}|\\
\end{tabular}
\end{center}
at the top of every child file \textit{child}
which is included by |\include{|\textit{child}|}|
from within the main file
(or at least for those files to be compiled individually).
The argument \textit{main} must be the filename of the main file.

There are a couple of
considerations in setting up the main and child documents:

%%%%%%%%%%%%%%%%%%%%%%%%%%%%%%%%%%%%%%%%
\paragraph{Restrictions.}

Please note the following restrictions:
\begin{itemize}
\item
|\childdocmain| must be called with one argument \textit{main}
to ensure compatibility with earlier version of the package.
It must either be empty (|\childdocmain{}|)
or precisely match the filename of the main file in which it is specified.
See \secref{sec:detection} for further information.
\item
The filename \textit{main} must be specified without the |.tex| extension.
\item
The filename \textit{main} is case sensitive
(even in case-insensitive file systems)
due to internal string comparison.
\item
The argument \textit{main} should be fully expanded, it cannot be a macro.
\item
Subdirectories and special characters should be avoided in filenames.
\item
The command |\childdocmain{|\textit{main}|}| must be followed by a whitespace.
It should not be followed immediately by another command
or by a comment mark `|%|'.
This is because the \TeX{} parser reads the token immediately following
the argument of |\childdocmain| and puts it
at the beginning of every child section;
however, a white\-space is ignored.
\end{itemize}

%%%%%%%%%%%%%%%%%%%%%%%%%%%%%%%%%%%%%%%%
\paragraph{Content of Main File.}

It is advisable to place all content in the child files included by |\include|.
Any output contained in the main file will appear in all child documents
unless suppressed manually;
it cannot be suppressed automatically by the |\includeonly| directive
and thus should normally be avoided.
A method to include some content in the main file
by means of conditional processing is described in \secref{sec:conditional}.

%%%%%%%%%%%%%%%%%%%%%%%%%%%%%%%%%%%%%%%%
\paragraph{Page Numbering.}

When only a part of the document is compiled,
the appropriate numbering of pages
(as well as other status parameters)
is determined from the |.aux| files.
The latter contain information from previous passes.
However this information needs to propagate through
all intermediate child documents.
Therefore the page numbering in child documents may well
be inconsistent until the complete document is compiled at least once.

A useful (if unconventional) way to always ensure a consistent
page numbering is to restart the numbering in each child document
and denote the pages by `\textit{child}|.|\textit{page}'
where \textit{child} represents the chapter/section number of the child file.
This can be achieved by the command
|\numberwithin{page}{|\textit{child}|}|
of the \textsf{amsmath} package
where \textit{child} can be |chapter| or |section|
depending on the chosen structuring.
Alternatively, one can modify the macro |\thepage| appropriately
and reset the counter |page| at the start of each child file.

%%%%%%%%%%%%%%%%%%%%%%%%%%%%%%%%%%%%%%%%%%%%%%%%%%%%%%%%%%%%%%%%%%%%%%%%%%%%%%%%
\subsection{Conditional Processing}
\label{sec:conditional}

The package provides a mechanism to compile different versions
of a document. To customise the versions further some conditional processing
can come in handy to distinguish which version is being compiled.
The package provides two macros to describe the compilation context:

%%%%%%%%%%%%%%%%%%%%%%%%%%%%%%%%%%%%%%%%
\DescribeMacro{\ifchilddoc}
The conditional |\ifchilddoc| distinguishes between the compilation of
child documents and the main document:
%
\begin{center}
|\ifchilddoc |\textit{child-code}| |[|\||else |\textit{main-code}]| \||fi|
\end{center}

%%%%%%%%%%%%%%%%%%%%%%%%%%%%%%%%%%%%%%%%
\DescribeMacro{\childdocname}
\DescribeMacro{\childdocjob}
The macro |\childdocname| contains the filename (without extension)
of the main or child file being processed.
Note that |\childdocjob| will always contain the name of the main file.

%%%%%%%%%%%%%%%%%%%%%%%%%%%%%%%%%%%%%%%%
\paragraph{Title Page.}

Conditional processing can be used to include a title or banner page
in the main document when proper precautions are taken.
Importantly, the code in the main file should ensure that the page counter
(as well as other status parameters which are stored in the |.aux| files)
takes the same value after the conditional processing.
Otherwise the page numbers may take divergent values
depending on which part is compiled.

For example, a title page could be declared by:
%
\begin{center}
\begin{tabular}{l}
|\ifchilddoc\||else|\\
|\addtocounter{page}{-1}|\\
\textit{code for title page}\\
|\newpage|\\
|\||fi|
\end{tabular}
\end{center}
%
A banner page for the child documents can be generated by:
%
\begin{center}
\begin{tabular}{l}
|\ifchilddoc|\\
|\addtocounter{page}{-1}|\\
\textit{code for banner page}\\
|\newpage|\\
|\||fi|
\end{tabular}
\end{center}
%
Here one could write a message such as:
\begin{center}
|This is the part \childdocname{} of \childdocjob{}.|
\end{center}

%%%%%%%%%%%%%%%%%%%%%%%%%%%%%%%%%%%%%%%%%%%%%%%%%%%%%%%%%%%%%%%%%%%%%%%%%%%%%%%%
\subsection{Flags}
\label{sec:flags}

The package makes it easy to generate different versions
of the main or child documents.
To this end compilation flags can be defined
and assigned different default values.
They will be particularly useful in conjunction
with the forwarding mechanism described in \secref{sec:forward}.

For example, it may be useful to have a flag |\version|
which can be set to |draft| or |final|.
The document source will contain some conditional code
depending on the value of |\version|.
Suppose further, the flag should default to |final| for the main file
and to |draft| for child files
which is a natural assignment for editing the document.
This is achieved by placing the following code
in the preamble of the main document
(below the |\childdocmain| directive):
%
\begin{center}
\begin{tabular}{l}
|\ifchilddoc|\\
|\providecommand{\version}{draft}|\\
|\||else|\\
|\providecommand{\version}{final}|\\
|\||fi|
\end{tabular}
\end{center}
%
The definition by |\providecommand| makes sure
that previous definitions are not overwritten.
Further statements |\providecommand{\version}{...}|
can thus be added before the above code to override it.

For the main file, one might add a line
(between |\childdocmain| and the above block)
%
\begin{center}
|%\ifchilddoc\||else\providecommand{\version}{draft}\||fi|
\end{center}
%
which can be uncommented to produce a draft version.
Likewise one can add a line to the very top of a child file
(above the |\childdocof{|\textit{main}|}| directive)
%
\begin{center}
|%\providecommand{\version}{final}|
\end{center}
%
which can be uncommented to produce the final version of this child document.

%%%%%%%%%%%%%%%%%%%%%%%%%%%%%%%%%%%%%%%%%%%%%%%%%%%%%%%%%%%%%%%%%%%%%%%%%%%%%%%%
\subsection{Forwarding}
\label{sec:forward}

Different versions of the main or child documents
using compilation flags as described in \secref{sec:flags}
can be (permanently) stored in different files
for convenient compilation, viewing and distribution.
To this end, the package defines a command
to pass on compilation to a different file:

%%%%%%%%%%%%%%%%%%%%%%%%%%%%%%%%%%%%%%%%
\DescribeMacro{\childdocforward}
The command |\childdocforward| redirects processing to
another source file:
%
\begin{center}
\begin{tabular}{l}
|\input{childdoc.def}|\\
|\childdocforward[|\textit{main}|]{|\textit{dest}|}|\\
\end{tabular}
\end{center}
%
The argument \textit{dest} is the destination file
(without extension).
It should be the main file or one of the child files.
Note that further \textsf{childdoc} directives
such as |\childdocof| and |\childdocforward|
in the indicated file will be processed in this form.
The optional argument \textit{main}
passes on directly to the main file \textit{main}
while pretending to compile the child \textit{dest}.
This form behaves as if \textit{dest}
issues |\childdocof{|\textit{main}|}| right away,
and no further \textsf{childdoc} directives will be processed.

%%%%%%%%%%%%%%%%%%%%%%%%%%%%%%%%%%%%%%%%
\DescribeMacro{\...prefix}
In the alternative form |\childdocforwardprefix|,
%
\begin{center}
\begin{tabular}{l}
|\input{childdoc.def}|\\
|\childdocforwardprefix[|\textit{main}|]{|\textit{prefix}|}{|\textit{dest}|}|
\end{tabular}
\end{center}
%
the destination file is determined by a pattern
depending on the current file:
To make this work, the current file must be called
`{\textit{prefix}\hspace{0.2em}\textit{suffix}}'
with \textit{prefix} matching precisely the argument.
Processing is then passed on to the file
`{\textit{dest}\hspace{0.2em}\textit{suffix}}'.
Surely, the same effect is achieved by
directly specifying the
argument `{\textit{dest}\hspace{0.2em}\textit{suffix}}'
in the first form.
However, that requires to set up a different file
for each child. With the alternative form of the command
all these files can have exactly the same content
which simplifies setting them up and maintaining them.

For example, the following file |draft.tex|
with a compilation flag |\version| as described in \secref{sec:flags}
compiles the main document as a draft:
%
\begin{center}
\begin{tabular}{l}
|\def\version{draft}|\\
|\input{childdoc.def}|\\
|\childdocforward{|\textit{main}|}|
\end{tabular}
\end{center}
%
Likewise, the following files |final|\textit{nn}|.tex|
compile the final version of the child document
|child|\textit{nn}|.tex|:
%
\begin{center}
\begin{tabular}{l}
|\def\version{final}|\\
|\input{childdoc.def}|\\
|\childdocforwardprefix{final}{child}|
\end{tabular}
\end{center}
%

Note that when several versions of a main file and/or of each child file
are to be generated, it may be convenient to set up a |Makefile| or
shell script to automatise the process.

%%%%%%%%%%%%%%%%%%%%%%%%%%%%%%%%%%%%%%%%%%%%%%%%%%%%%%%%%%%%%%%%%%%%%%%%%%%%%%%%
\subsection{Command Line Processing}
\label{sec:commandline}

The effect of redirection files can also be achieved by invoking
the \LaTeX{} compiler with a more elaborate command line.
Most conveniently this should be done as part
of a shell script or a |Makefile|.

When using \textsf{childdoc} in the main file, the following
command lines effectively perform a redirection
(note that depending on the shell being used,
backslashes may have to be doubled: `|\|' $\to$ `|\\|'):
%
\begin{center}
|... -jobname "|\textit{target}|" |\\|"|[\textit{flags}]%
|\input{childdoc.def}\childdocforward[|\textit{main}|]{|\textit{dest}|}"|
\end{center}
%
Here \textit{target} is the name of the output file,
\textit{main} is the name of the main file
and \textit{dest} is the name of the main or child file to be processed
(all filenames without extensions).
The optional argument \textit{main} can be omitted
if \textit{main} matches \textit{dest}.
Optionally, compilation \textit{flags} can be defined via |\def| commands.
This command line makes the \TeX{} engine believe
it is compiling the file \textit{target}
whose content is specified as the latter parameter.
The provided code then forwards the processing to
\textit{main} or \textit{dest} as described in \secref{sec:forward}.

%%%%%%%%%%%%%%%%%%%%%%%%%%%%%%%%%%%%%%%%%%%%%%%%%%%%%%%%%%%%%%%%%%%%%%%%%%%%%%%%
\subsection{Include by Input}
\label{sec:input}

Including child documents by |\include| has some restrictions by design.
Most notably, the content of a child document always occupies
its own set of pages; pages cannot be shared between child documents.
Usually, this behaviour makes perfect sense
because each child document contain an essential part of the document.
However, in some situations it may be desirable to compose
a document from a collection of parts
without having mandatory page breaks between then.
For this case, the package
provides a mechanism to include parts
by |\input| which can also be processed individually.
However, by construction this mechanism
requires manual handling of the content to be output.

%%%%%%%%%%%%%%%%%%%%%%%%%%%%%%%%%%%%%%%%
\DescribeMacro{\ifchilddocmanual}
The main file should be prepared as usual, see \secref{sec:include}.
However, the document body must make a distinction
between processing of an individual part and of the main document, e.g.:
%
\begin{center}
\begin{tabular}{l}
|\ifchilddocmanual|\\
|\input{\childdocname}|\\
|\||else|\\
\textit{document body with }|\input{|\textit{part}|}|\\
|\||fi|
\end{tabular}
\end{center}
%
The conditional |\ifchilddocmanual| is true whenever
a part to be included by |\input| is being compiled,
and the name of the part is stored in |\childdocname|.

%%%%%%%%%%%%%%%%%%%%%%%%%%%%%%%%%%%%%%%%
\DescribeMacro{\childdocby}
Each part to be included by |\input| should start with:
%
\begin{center}
\begin{tabular}{l}
|\input{childdoc.def}|\\
|\childdocby{|\textit{main}|}|\\
\end{tabular}
\end{center}
%
The directive |\childdocby| is similar to |\childdocof|
described in \secref{sec:include},
but the subsequent selection of content must be done manually.
To that end, both |\ifchilddoc| and |\ifchilddocmanual|
will be true upon processing of a part,
and the name of the part is stored in |\childdocname|.
Note that |\jobname| will be set to the filename of the current part
so that each part receives an individual |.aux| file
that does not interfere with the |.aux| file(s) of the main document.
This behaviour can be altered by the alternative form
|\childdocby[*]{|\textit{main}|}| (with a non-empty optional argument)
which uses the |.aux| file of the main document
by setting |\jobname| to \textit{main}.

%%%%%%%%%%%%%%%%%%%%%%%%%%%%%%%%%%%%%%%%%%%%%%%%%%%%%%%%%%%%%%%%%%%%%%%%%%%%%%%%
\subsection{Driver Development}
\label{sec:driver}

The \textsf{childdoc} mechanism can also be use for the development
of definition files such as \LaTeX{} styles or classes.
This case differs from the above setup with multiple parts
included by |\include| in that no |\includeonly| should be invoked.
This can be achieved by starting the include file
(before |\ProvidesPackage|) with:
%
\begin{center}
\begin{tabular}{l}
|\input{childdoc.def}|\\
|\childdocforward{|\textit{main}|}|\\
\end{tabular}
\end{center}
%
or alternatively with:
%
\begin{center}
\begin{tabular}{l}
|\input{childdoc.def}|\\
|\childdocby{|\textit{main}|}|\\
\end{tabular}
\end{center}
%
Both forms have slightly different effects as described above.
The main file is prepared as usual, see \secref{sec:include}.

%%%%%%%%%%%%%%%%%%%%%%%%%%%%%%%%%%%%%%%%%%%%%%%%%%%%%%%%%%%%%%%%%%%%%%%%%%%%%%%%
\subsection{Legacy Detection}
\label{sec:detection}

The directive |\childdocmain| in the main file can detect
whether the complete document or merely a child is to be compiled
even without using the directive |\childdocof|.
This method is deprecated because it is less robust
and there is no compelling reason to use it;
it is merely provided for backward compatibility
and it may be removed in future versions.

If the detection mechanism is to be used,
it is mandatory to correctly specify
the filename of the main file as the argument of |\childdocmain|:
%
\begin{center}
\begin{tabular}{l}
|\input{childdoc.def}|\\
|\childdocmain{|\textit{main}|}|\\
\end{tabular}
\end{center}
%
If |\jobname| does not match the argument \textit{main} of |\childdocmain|,
it is assumed that |\jobname| points to the child file to be compiled.
When using |\childdocmain| with the main file specified as argument,
it suffices to start a child file
with just |\input{|\textit{main}|}|
without loading of the package and using |\childdocof|.
If instead all processing is done
with the appropriate \textsf{childdoc} directives,
the argument of \textit{main} of |\childdocmain| can be empty.

An alternative version of the command line processing described
in \secref{sec:commandline} using the detection mechanism reads:
%
\begin{center}
|... -jobname "|\textit{target}|" "|[\textit{flags}]%
[|\def\jobname{|\textit{dest}|}|]|\input{|\textit{main}|}"|
\end{center}

%%%%%%%%%%%%%%%%%%%%%%%%%%%%%%%%%%%%%%%%%%%%%%%%%%%%%%%%%%%%%%%%%%%%%%%%%%%%%%%%
\subsection{Manual Code}
\label{sec:manual}

In case one cannot be certain whether the definitions file |childdoc.def|
is installed on the target \TeX{} distribution
and one prefers not to ship it,
it is conceivable to paste a few relevant commands into the sources.

To that end, drop all statements |\input{childdoc.def}|
and perform the replacements as outlined below.
Instead of |\childdocmain{|\textit{main}|}| add the following code
to the top of the main file:
%
\begin{center}
\begin{tabular}{l}
|\||ifdefined\childdocname\endinput\||fi\newif\ifchilddoc|\\
|\edef\childdocname{\scantokens\expandafter{\jobname\noexpand}}|\\
|\def\childdocmain{|\textit{main}|}\||ifx\childdocmain\childdocname\||else|\\
|\childdoctrue\includeonly{\childdocname}\let\jobname\childdocmain\||fi|\\
\end{tabular}
\end{center}
%
Instead of |\childdocof{|\textit{main}|}| just include the main file
at the top of each child file:
%
\begin{center}
|\input{|\textit{main}|}|
\end{center}
%
A simple redirection |\childdocforward{|\textit{dest}|}| is achieved by:
%
\begin{center}
|\def\jobname{|\textit{dest}|}\input{\jobname}|
\end{center}
%
The redirection with prefix
|\childdocforwardprefix[|\textit{prefix}|]{|\textit{dest}|}|
is accomplished by:
%
\begin{center}
\begin{tabular}{l}
|{\edef\jobname{\scantokens\expandafter{\jobname\noexpand}}|\\
|\def\redirectjob |\textit{prefix}|#1~~~{\gdef\jobname{|\textit{dest}|#1}}|\\
|\expandafter\redirectjob\jobname~~~}\input{\jobname}|
\end{tabular}
\end{center}

In an alternative approach,
child documents can be compiled by a specific command line
without additional code or specific definitions:
%
\begin{center}
|... -jobname "|\textit{target}|" "|[\textit{flags}]%
|\includeonly{|\textit{dest}|}\input{|\textit{main}|}"|
\end{center}
%

%%%%%%%%%%%%%%%%%%%%%%%%%%%%%%%%%%%%%%%%%%%%%%%%%%%%%%%%%%%%%%%%%%%%%%%%%%%%%%%%
%%%%%%%%%%%%%%%%%%%%%%%%%%%%%%%%%%%%%%%%%%%%%%%%%%%%%%%%%%%%%%%%%%%%%%%%%%%%%%%%
\section{Information}

%%%%%%%%%%%%%%%%%%%%%%%%%%%%%%%%%%%%%%%%%%%%%%%%%%%%%%%%%%%%%%%%%%%%%%%%%%%%%%%%
\subsection{Copyright}

Copyright \copyright{} 2017--2018 Niklas Beisert

This work may be distributed and/or modified under the
conditions of the \LaTeX{} Project Public License, either version 1.3
of this license or (at your option) any later version.
The latest version of this license is in
  \url{http://www.latex-project.org/lppl.txt}
and version 1.3 or later is part of all distributions of \LaTeX{}
version 2005/12/01 or later.

This work has the LPPL maintenance status `maintained'.

The Current Maintainer of this work is Niklas Beisert.

This work consists of the files |README.txt|, |childdoc.ins| and |childdoc.dtx|
as well as the derived files |childdoc.def|, |cdocsamp.tex|
with |cdocsch1.tex|, |cdocsch2.tex|, |cdocspt3.tex|, |cdocspt4.tex|,
|cdocsdrf.tex|, |cdocsfn1.tex|, |cdocsfn2.tex|
as well as |childdoc.pdf|.

%%%%%%%%%%%%%%%%%%%%%%%%%%%%%%%%%%%%%%%%%%%%%%%%%%%%%%%%%%%%%%%%%%%%%%%%%%%%%%%%
\subsection{Files and Installation}

The package consists of the files:
%
\begin{center}
\begin{tabular}{ll}
    |README.txt|   & readme file \\
    |childdoc.ins| & installation file \\
    |childdoc.dtx| & source file \\
    |childdoc.def| & definition file \\
    |cdocsamp.tex| & sample main file \\
    |cdocsch1.tex| & sample include file \\
    |cdocsch2.tex| & sample include file \\
    |cdocspt3.tex| & sample part file \\
    |cdocspt4.tex| & sample part file \\
    |cdocsdrf.tex| & sample redirection file \\
    |cdocsfn1.tex| & sample redirection file \\
    |cdocsfn2.tex| & sample redirection file \\
    |childdoc.pdf| & manual
\end{tabular}
\end{center}
%
The distribution consists of the files
|README.txt|, |childdoc.ins| and |childdoc.dtx|.
%
\begin{itemize}
\item
Run (pdf)\LaTeX{} on |childdoc.dtx|
to compile the manual |childdoc.pdf| (this file).
\item
Run \LaTeX{} on |childdoc.ins| to create the definitions file |childdoc.def|
and the sample |cdocsamp.tex| with include files
|cdocsch1.tex|, |cdocsch2.tex|, |cdocspt3.tex|, |cdocspt4.tex|,
|cdocsdrf.tex|, |cdocsfn1.tex|, |cdocsfn2.tex|.
Then copy the file |childdoc.def| to an appropriate directory of your \LaTeX{}
distribution, e.g.\ \textit{texmf-root}|/tex/latex/childdoc|.
\end{itemize}

%%%%%%%%%%%%%%%%%%%%%%%%%%%%%%%%%%%%%%%%%%%%%%%%%%%%%%%%%%%%%%%%%%%%%%%%%%%%%%%%
\subsection{Related CTAN Packages}

There are several other packages which offer a similar functionality:
%
\begin{itemize}
\item
The packages
\href{http://ctan.org/pkg/docmute}{\textsf{docmute}},
\href{http://ctan.org/pkg/includex}{\textsf{includex}} and
\href{http://ctan.org/pkg/standalone}{\textsf{standalone}}
provide commands to include only the document body of
a child file thus allowing both files to be compiled individually.
\item
The packages \href{http://ctan.org/pkg/subdocs}{\textsf{subdocs}}
and \href{http://ctan.org/pkg/subfiles}{\textsf{subfiles}}
provide structures in which the main and child documents can be
encapsulated and allowing them to be compiled individually.
The inclusion mechanism is different from the conventional |\include|.
\item
The package \href{http://ctan.org/pkg/combine}{\textsf{combine}}
is an elaborate solution to combine several documents into one.
\end{itemize}
%
See also the CTAN topic \href{http://ctan.org/topic/subdocs}{\textsf{subdocs}}
for further related packages.
The present package differs from the above solutions in that
a document structure constructed with the conventional |\include| mechanism
just needs two extra commands at the top of every file
such that all constituent files can be compiled individually.

%%%%%%%%%%%%%%%%%%%%%%%%%%%%%%%%%%%%%%%%%%%%%%%%%%%%%%%%%%%%%%%%%%%%%%%%%%%%%%%%
%\subsection{Feature Suggestions}
%
%The following is a list of features which may be useful for future
%versions of this package:
%%
%\begin{itemize}
%\item
%\ldots
%\end{itemize}

%%%%%%%%%%%%%%%%%%%%%%%%%%%%%%%%%%%%%%%%%%%%%%%%%%%%%%%%%%%%%%%%%%%%%%%%%%%%%%%%
\subsection{Revision History}

%%%%%%%%%%%%%%%%%%%%%%%%%%%%%%%%%%%%%%%%
\paragraph{v2.0:} 2018/12/30

\begin{itemize}
\item
immediate forward processing
\item
added |\childdocby| mechanism
\item
manual restructured
\end{itemize}

%%%%%%%%%%%%%%%%%%%%%%%%%%%%%%%%%%%%%%%%
\paragraph{v1.6:} 2018/01/17

\begin{itemize}
\item
application for development of include files
\item
corrections to manual
\end{itemize}

%%%%%%%%%%%%%%%%%%%%%%%%%%%%%%%%%%%%%%%%
\paragraph{v1.5:} 2017/05/21

\begin{itemize}
\item
more complete structuring introduced
\item
|\childdocof| introduced
\item
|\childdoc| renamed to |\childdocmain|
\item
|\childredirect| renamed to |\childdocforward| and |\childdocforwardprefix|
and functionality expanded
\end{itemize}

%%%%%%%%%%%%%%%%%%%%%%%%%%%%%%%%%%%%%%%%
\paragraph{v1.0:} 2017/04/27

\begin{itemize}
\item
manual and install package
\item
first version published on CTAN
\end{itemize}

%%%%%%%%%%%%%%%%%%%%%%%%%%%%%%%%%%%%%%%%
\paragraph{v0.6:} 2017/04/26

\begin{itemize}
\item
redirection mechanism added
\end{itemize}

%%%%%%%%%%%%%%%%%%%%%%%%%%%%%%%%%%%%%%%%
\paragraph{v0.5:} 2017/04/26

\begin{itemize}
\item
functionality in definition file
\end{itemize}


%%%%%%%%%%%%%%%%%%%%%%%%%%%%%%%%%%%%%%%%%%%%%%%%%%%%%%%%%%%%%%%%%%%%%%%%%%%%%%%%
%%%%%%%%%%%%%%%%%%%%%%%%%%%%%%%%%%%%%%%%%%%%%%%%%%%%%%%%%%%%%%%%%%%%%%%%%%%%%%%%
%%%%%%%%%%%%%%%%%%%%%%%%%%%%%%%%%%%%%%%%%%%%%%%%%%%%%%%%%%%%%%%%%%%%%%%%%%%%%%%%
\appendix

\settowidth\MacroIndent{\rmfamily\scriptsize 000\ }

 \DocInput{childdoc.dtx}

\end{document}
%</driver>
% \fi
%
% %%%%%%%%%%%%%%%%%%%%%%%%%%%%%%%%%%%%%%%%%%%%%%%%%%%%%%%%%%%%%%%%%%%%%%%%%%%%%%
% %%%%%%%%%%%%%%%%%%%%%%%%%%%%%%%%%%%%%%%%%%%%%%%%%%%%%%%%%%%%%%%%%%%%%%%%%%%%%%
% \section{Sample}
%\iffalse
%<*samplemain>
%\fi
%
% The following presents a sample document
% with two chapters, two parts, a title page,
% a compile flag as well as three forwarding files to set the flag.
% It consists of eight |.tex| files:
% \begin{center}
% \begin{tabular}{ll}
% |cdocsamp.tex|&main file\\
% |cdocsch1.tex|&include file for chapter 1\\
% |cdocsch2.tex|&include file for chapter 2\\
% |cdocspt3.tex|&include file for part 3\\
% |cdocspt4.tex|&include file for part 4\\
% |cdocsdrf.tex|&forwarding file for main file in draft mode\\
% |cdocsfi1.tex|&forwarding file for final version of chapter 1\\
% |cdocsfi2.tex|&forwarding file for final version of chapter 2\\
% \end{tabular}
% \end{center}
% Each of the eight files can be compiled directly by the \LaTeX{} compiler.
%
% %%%%%%%%%%%%%%%%%%%%%%%%%%%%%%%%%%%%%%
% \paragraph{Main File.}
%
% The main file is called |cdocsamp.tex|.
%
% Load the \textsf{childdoc} definitions and
% declare the filename for the main document:
%    \begin{macrocode}
\input{childdoc.def}
\childdocmain{}
%    \end{macrocode}

% Optional override for |\version| flag:
%    \begin{macrocode}
%%\ifchilddoc\else\providecommand{\version}{draft}\fi
%    \end{macrocode}

% Define the default values for the |\version| flag
% (|final| for the main file and |draft| for childs):
%    \begin{macrocode}
\ifchilddoc
\providecommand{\version}{draft}
\else
\providecommand{\version}{final}
\fi
%    \end{macrocode}

% Load the standard document class:
%    \begin{macrocode}
\documentclass[12pt]{article}
%    \end{macrocode}

% Start the document body:
%    \begin{macrocode}
\begin{document}
%    \end{macrocode}

% Declare a title page.
% Print title, part of document being processed and version flag:
%    \begin{macrocode}
\addtocounter{page}{-1}
\begin{center}
{\LARGE\bfseries{}childdoc example\par}
\vspace{1cm}
\ifchilddoc
\ifchilddocmanual part\else chapter\fi:
`\childdocname' of `\childdocjob'\par
\else
main document: `\childdocjob'\par
\fi
version: \version\par
\end{center}
\newpage
%    \end{macrocode}

% Manually include selected file,
% otherwise process as usual:
%    \begin{macrocode}
\ifchilddocmanual
\section*{part `\childdocname'}
\input{\childdocname}
\else
%    \end{macrocode}

% Include the two chapters:
%    \begin{macrocode}
\include{cdocsch1}
\include{cdocsch2}
%    \end{macrocode}

% Include the two parts unless only chapters should be displayed:
%    \begin{macrocode}
\ifchilddoc\else
\section{part three}
\input{cdocspt3}
\section{part four}
\input{cdocspt4}
\fi
%    \end{macrocode}

% Process as usual until here:
%    \begin{macrocode}
\fi
%    \end{macrocode}

% End of document body:
%    \begin{macrocode}
\end{document}
%    \end{macrocode}
%\iffalse
%</samplemain>
%\fi
%
% %%%%%%%%%%%%%%%%%%%%%%%%%%%%%%%%%%%%%%
% \paragraph{Chapter Include Files.}
%
% The include files are called |cdocsch1.tex| and |cdocsch2.tex|.
%
%\iffalse
%<*samplechap1|samplechap2>
%\fi

% Optional override for |\version| flag:
%    \begin{macrocode}
%%\providecommand{\version}{final}
%    \end{macrocode}

% Include the main document:
%    \begin{macrocode}
\input{childdoc.def}
\childdocof{cdocsamp}
%    \end{macrocode}

%\iffalse
%</samplechap1|samplechap2>
%\fi
%
%\iffalse
%<*samplechap1>
%\fi
% Some text for chapter 1:
%    \begin{macrocode}
\section{one}
some text in chapter one
%    \end{macrocode}

%\iffalse
%</samplechap1>
%\fi
% Some text for chapter 2:
%\iffalse
%<*samplechap2>
%\fi
%    \begin{macrocode}
\section{two}
more text in chapter two
%    \end{macrocode}

%\iffalse
%</samplechap2>
%\fi
%
% %%%%%%%%%%%%%%%%%%%%%%%%%%%%%%%%%%%%%%
% \paragraph{Part Include Files.}
%
% The include files are called |cdocspt3.tex| and |cdocspt4.tex|.
%
%\iffalse
%<*samplepart3|samplepart4>
%\fi

% Optional override for |\version| flag:
%    \begin{macrocode}
%%\providecommand{\version}{final}
%    \end{macrocode}

% Include the main document:
%    \begin{macrocode}
\input{childdoc.def}
\childdocby{cdocsamp}
%    \end{macrocode}

%\iffalse
%</samplepart3|samplepart4>
%\fi
%
%\iffalse
%<*samplepart3>
%\fi
% Some text for part 3:
%    \begin{macrocode}
some text in part three
%    \end{macrocode}

%\iffalse
%</samplepart3>
%\fi
% Some text for part 4:
%\iffalse
%<*samplepart4>
%\fi
%    \begin{macrocode}
more text in part four
%    \end{macrocode}

%\iffalse
%</samplepart4>
%\fi
%
% %%%%%%%%%%%%%%%%%%%%%%%%%%%%%%%%%%%%%%
% \paragraph{Forwarding for a Complete Draft.}
%
% The following forwarding file |cdocsdrf.tex|
% compiles the main document in draft mode:
%\iffalse
%<*sampledraft>
%\fi
%    \begin{macrocode}
\def\version{draft}
\input{childdoc.def}
\childdocforward{cdocsamp}
%    \end{macrocode}

%\iffalse
%</sampledraft>
%\fi
%
% %%%%%%%%%%%%%%%%%%%%%%%%%%%%%%%%%%%%%%
% \paragraph{Forwarding for Final Version of the Chapters.}
%
% The following forwarding files |cdocsfn1.tex| and |cdocsfn2.tex|
% (with identical content)
% compile the final versions of the child documents
% |cdocsch1.tex| and |cdocsch2.tex|, respectively:
%\iffalse
%<*samplefinal>
%\fi
%    \begin{macrocode}
\def\version{final}
\input{childdoc.def}
\childdocforwardprefix[cdocsamp]{cdocsfn}{cdocsch}
%    \end{macrocode}

%\iffalse
%</samplefinal>
%\fi
%
% %%%%%%%%%%%%%%%%%%%%%%%%%%%%%%%%%%%%%%
% \paragraph{Command Line Processing.}
%
% The following three command lines generate the output files
% |cdocscld|, |cdocscl1| and |cdocscl2|
% which should be identical to
% |cdocsdrf|, |cdocsch1| and |cdocsfn2|, respectively:
% \begin{center}
% \begin{tabular}{l}
% |latex -jobname cdocscld \|\\
% |  "\def\version{draft}\input{childdoc.def}\childdocforward{cdocsamp}"|\\
% |latex -jobname cdocscl1 \|\\
% |  "\input{childdoc.def}\childdocforward[cdocsamp]{cdocsch1}"|\\
% |latex -jobname cdocscl2 \|\\
% |  "\def\version{final}\input{childdoc.def}\childdocforward{cdocsch2}"|
% \end{tabular}
% \end{center}
% Note that the trailing backslash on each first line
% merely continues the input to the second line
% (for convenient cut ant paste).
% Furthermore, the command |latex| can be replaced by any
% of its alternative versions such as |pdflatex|.
%
% %%%%%%%%%%%%%%%%%%%%%%%%%%%%%%%%%%%%%%%%%%%%%%%%%%%%%%%%%%%%%%%%%%%%%%%%%%%%%%
% %%%%%%%%%%%%%%%%%%%%%%%%%%%%%%%%%%%%%%%%%%%%%%%%%%%%%%%%%%%%%%%%%%%%%%%%%%%%%%
% \section{Implementation}
%\iffalse
%<*package>
%\fi
%
% This section describes the definitions file |childdoc.def|.

% The definitions cannot be loaded using |\usepackage| or |\RequirePackage|
% which has a mechanism to prevent loading a style file more than once.
% When loading the definitions by means of |\input|
% multiple instances have to be prevented manually:
%\iffalse
%This code needs to be before the `\ProvidesFile' directive
%which is defined at the beginning of this file.
%Therefore it is also placed there and commented out here.
%</package>
%<*discard>
%\fi
%    \begin{macrocode}
\ifdefined\childdocmain\endinput\fi
%    \end{macrocode}
%\iffalse
%</discard>
%<*package>
%\fi
%
% \macro{\ifchilddoc}
% \macro{\ifchilddocmanual}
% The conditional |\ifchilddoc| tells whether a
% child (true) or main (false) document is being compiled.
% The conditional |\ifchilddocmanual| tells whether
% the |\includeonly| mechanism is used (false) or
% the selection of child files must be performed manually (true).
% The definitions initialise to false:
%    \begin{macrocode}
\newif\ifchilddoc
\newif\ifchilddocmanual
%    \end{macrocode}

% \macro{\childdocname}
% \macro{\childdocjob}
% The macro |\childdocname| stores the name of the main document
% to be compiled. The macro |\childdocjob| stores the name of
% the document on which the \LaTeX{} compiler was originally invoked.
% The content of |\jobname| cannot be compared
% to filenames specified in the source due to different catcodes.
% The following code rescans |\jobname|, stores the result
% in |\childdocname| and saves a copy in |\childdocjob|:
%    \begin{macrocode}
\edef\childdocname{\scantokens\expandafter{\jobname\noexpand}}
\let\childdocjob\childdocname
%    \end{macrocode}

% \macro{\childdocdisable}
% The macro |\childdocdisable| prevents the main file
% from being processed more than once.
% At this stage, the main document command |\childdocmain|
% is assumed to be called once again where it should do nothing.
% Any subsequent call to it should prevent
% a secondary processing of the main document
% It overwrites the forwarding commands
% |\childdocof| and |\childdocforward|
% with empty macros to prevent further inclusions of the main document:
%    \begin{macrocode}
\newcommand{\childdocdisable}
{
  \renewcommand{\childdocmain}[1]{\renewcommand{\childdocmain}[1]{\endinput}}
  \renewcommand{\childdocof}[1]{}
  \renewcommand{\childdocby}[2][]{}
  \renewcommand{\childdocforward}[2][]{}
  \renewcommand{\childdocdisable}{}
}
%    \end{macrocode}

% \macro{\childdocmain}
% The macro |\childdocmain| is to be called at the top of the main file
% with nothing or the main filename (without extension) as argument.
% First, it breaks loops.
% If the argument is not empty and does not match |\childdocname|
% (which is set by the first inclusion of |childdoc.def|),
% |\ifchilddoc| is set to true, |\includeonly| is applied to the child file
% and |\jobname| is set to the main file
% (for proper handling of |.aux| files):
%    \begin{macrocode}
\newcommand{\childdocmain}[1]
{
  \childdocdisable\childdocmain{}
  \if?#1?\else
    \begingroup
      \def\childdoctmp{#1}
      \ifx\childdoctmp\childdocname
        \def\childdoctmp{}
      \else
        \def\childdoctmp
        {
          \childdoctrue
          \includeonly{\childdocname}
          \def\childdocjob{#1}
          \def\jobname{#1}
        }
      \fi
      \expandafter
    \endgroup
    \childdoctmp
  \fi
}
%    \end{macrocode}

% \macro{\childdocof}
% The command |\childdocof| redirects
% compilation to the main file |#1|.
%    \begin{macrocode}
\newcommand{\childdocof}[1]
{
  \childdocdisable
  \childdoctrue
  \includeonly{\childdocname}
  \def\jobname{#1}
  \def\childdocjob{#1}
  \input{#1}
}
%    \end{macrocode}

% \macro{\childdocby}
% The command |\childdocby| ....
%    \begin{macrocode}
\newcommand{\childdocby}[2][]
{
  \childdocdisable
  \childdoctrue
  \childdocmanualtrue
  \if?#1?\else
    \def\jobname{#2}
  \fi
  \def\childdocjob{#2}
  \input{#2}
  \endinput
}
%    \end{macrocode}

% \macro{\childdocforward}
% The command |\childdocforward| redirects
% compilation to the main file or
% (if the optional argument is given) a child file.
% Parameters are set as if the main file
% or a child file starting with |\childdocof| was compiled.
% Then compilation is handed over to the main file:
%    \begin{macrocode}
\newcommand{\childdocforward}[2][]
{
  \begingroup
    \if?#1?
      \def\childdoctmp
      {
        \def\childdocname{#2}
        \def\childdocjob{#2}
        \def\jobname{#2}
        \input{#2}
        \endinput
      }
    \else
      \def\childdoctmp
      {
        \childdocdisable
        \def\childdocname{#2}
        \childdoctrue
        \includeonly{#2}
        \def\childdocjob{#1}
        \def\jobname{#1}
        \input{#1}
        \endinput
      }
    \fi
    \expandafter
  \endgroup
  \childdoctmp
}
%    \end{macrocode}

% \macro{\childdocforwardprefix}
% The command |\childdocforwardprefix| redirects
% compilation to the main or a child file by means of a pattern.
% The prefix |#1| in the current filename is replaced by |#2|
% and the suffix of the current filename is kept
% (it is assumed that the filename does not contain the substring `|~~~|'
% which is used as a delimiter).
% Compilation is handed over to the new file by |\childdocforward|:
%    \begin{macrocode}
\newcommand{\childdocforwardprefix}[3][]
{
  \begingroup
    \def\childdocextract #2##1~~~{\def\childdoctmp{\childdocforward[#1]{#3##1}}}
    \expandafter\childdocextract\childdocname~~~
    \expandafter
  \endgroup
  \childdoctmp
}
%    \end{macrocode}

% \macro{\childdoc}
% The deprecated macro |\childdoc| is a legacy version of |\childdocmain|:
%    \begin{macrocode}
\newcommand{\childdoc}{\childdocmain}
%    \end{macrocode}

% \macro{\childdocredirect}
% The deprecated macro |\childdocredirect| is a legacy version
% of |\childdocforward| and |\childdocforwardprefix|:
%    \begin{macrocode}
\newcommand{\childdocredirect}[2][]
{
  \begingroup
    \if?#1?
      \def\childdoctmp{\childdocforward{#2}}
    \else
      \def\childdoctmp{\childdocforwardprefix{#1}{#2}}
    \fi
    \expandafter
  \endgroup
  \childdoctmp
}
%    \end{macrocode}

%\iffalse
%</package>
%\fi
%
\endinput

\childdocforward{cdocsamp}
%    \end{macrocode}

%\iffalse
%</sampledraft>
%\fi
%
% %%%%%%%%%%%%%%%%%%%%%%%%%%%%%%%%%%%%%%
% \paragraph{Forwarding for Final Version of the Chapters.}
%
% The following forwarding files |cdocsfn1.tex| and |cdocsfn2.tex|
% (with identical content)
% compile the final versions of the child documents
% |cdocsch1.tex| and |cdocsch2.tex|, respectively:
%\iffalse
%<*samplefinal>
%\fi
%    \begin{macrocode}
\def\version{final}
% \iffalse
%
% childdoc.dtx Copyright (C) 2017-2018 Niklas Beisert
%
% This work may be distributed and/or modified under the
% conditions of the LaTeX Project Public License, either version 1.3
% of this license or (at your option) any later version.
% The latest version of this license is in
%   http://www.latex-project.org/lppl.txt
% and version 1.3 or later is part of all distributions of LaTeX
% version 2005/12/01 or later.
%
% This work has the LPPL maintenance status `maintained'.
%
% The Current Maintainer of this work is Niklas Beisert.
%
% This work consists of the files childdoc.dtx and childdoc.ins
% and the derived files childdoc.def and cdocsamp.tex with
% cdocsch1.tex, cdocsch2.tex, cdocsdrf.tex, cdocsfn1.tex, cdocsfn2.tex.
%
%<package>\ifdefined\childdocmain\endinput\fi
%<package>\ProvidesFile{childdoc.def}[2018/12/30 v2.0 child document driver]
%<samplemain>\ProvidesFile{cdocsamp.tex}[2018/12/30 v2.0 sample for childdoc]
%<*driver>
%\ProvidesFile{childdoc.drv}[2018/12/30 v2.0 childdoc reference manual file]
\PassOptionsToClass{10pt,a4paper}{article}
\documentclass{ltxdoc}

\usepackage[margin=35mm]{geometry}
\usepackage{hyperref}
\usepackage{hyperxmp}
\usepackage[usenames]{color}

\hypersetup{colorlinks=true}
\hypersetup{pdfstartview=FitH}
\hypersetup{pdfpagemode=UseNone}
\hypersetup{pdfsource={}}
\hypersetup{pdflang={en-UK}}
\hypersetup{pdfcopyright={Copyright 2017-2018 Niklas Beisert.
  This work may be distributed and/or modified under the
  conditions of the LaTeX Project Public License, either version 1.3
  of this license or (at your option) any later version.}}
\hypersetup{pdflicenseurl={http://www.latex-project.org/lppl.txt}}
\hypersetup{pdfcontactaddress={ETH Zurich, ITP, HIT K,
  Wolfgang-Pauli-Strasse 27}}
\hypersetup{pdfcontactpostcode={8093}}
\hypersetup{pdfcontactcity={Zurich}}
\hypersetup{pdfcontactcountry={Switzerland}}
\hypersetup{pdfcontactemail={nbeisert@itp.phys.ethz.ch}}
\hypersetup{pdfcontacturl={http://people.phys.ethz.ch/\xmptilde nbeisert/}}

\newcommand{\secref}[1]{\hyperref[#1]{section \ref*{#1}}}

\parskip1ex
\parindent0pt
\let\olditemize\itemize
\def\itemize{\olditemize\parskip0pt}

\begin{document}

\title{The \textsf{childdoc} Package}
\hypersetup{pdftitle={The childdoc Package}}
\author{Niklas Beisert\\[2ex]
  Institut f\"ur Theoretische Physik\\
  Eidgen\"ossische Technische Hochschule Z\"urich\\
  Wolfgang-Pauli-Strasse 27, 8093 Z\"urich, Switzerland\\[1ex]
  \href{mailto:nbeisert@itp.phys.ethz.ch}
  {\texttt{nbeisert@itp.phys.ethz.ch}}}
\hypersetup{pdfauthor={Niklas Beisert}}
\hypersetup{pdfsubject={Manual for the LaTeX2e Package childdoc}}
\date{30 December 2018, \textsf{v2.0}}
\maketitle

\begin{abstract}\noindent
\textsf{childdoc} is a \LaTeXe{} package
that enables the direct compilation
of document sections included by |\include|
to individual files.
\end{abstract}

\begingroup
\parskip0ex
\tableofcontents
\endgroup

%%%%%%%%%%%%%%%%%%%%%%%%%%%%%%%%%%%%%%%%%%%%%%%%%%%%%%%%%%%%%%%%%%%%%%%%%%%%%%%%
%%%%%%%%%%%%%%%%%%%%%%%%%%%%%%%%%%%%%%%%%%%%%%%%%%%%%%%%%%%%%%%%%%%%%%%%%%%%%%%%
\section{Introduction}

\LaTeX{} provides a mechanism to structure a large document (such as a book)
into a main file and several child files (containing the chapters)
using the |\include| command.
This mechanism is beneficial for documents
which span hundreds of pages in order to
make the source file(s) more manageable.
Moreover, compilation can be restricted to
selected child files by means of the |\includeonly| command.
The latter feature can be used to reduce the compilation time while editing
(this was significantly more useful in the earlier days of \LaTeX{})
or to generate a smaller document which is easier to navigate.
Another application of |\includeonly| is to generate
documents consisting of selected parts of the complete document.

However, there are a few drawbacks of the plain |\include| mechanism:
\begin{itemize}
\item
The child files cannot be compiled on their own,
they can only be compiled via the main file.
A naive editing environment
(such as a text editor with an option
to have the current file processed by \LaTeX)
may require one to switch to the main file before compiling;
attempting to compile the child file produces errors.
\item
The main file must be modified (each time)
to adjust the |\includeonly| command
to the present needs. This easily leaves the main file in a messy state.
\item
The generated document will always carry the filename
of the main document. This is inconvenient if
several child files are to be compiled and
to be kept for distribution.
\end{itemize}

The present package provides a simple interface
to make child files individually compilable by \LaTeX{}.
Compiling a child file then has the same effect as compiling
the main file with an |\includeonly| command
to select the appropriate child.
Moreover the generated document will carry the name of the child
rather than the main file.
This resolves all three above issues.

This feature is meant to make the editing of books,
thesis documents and lecture notes somewhat more convenient.
However, the package can also be used efficiently for
composing a series of documents (such as exercise sheets)
which are typically distributed individually.
It then assists the author in generating the individual documents
(potentially in different versions)
as well as a document containing the collected series.
Another application is in developing style files
or other kinds of included material
where compilation of the style file could redirect
to a sample or test file.

%%%%%%%%%%%%%%%%%%%%%%%%%%%%%%%%%%%%%%%%%%%%%%%%%%%%%%%%%%%%%%%%%%%%%%%%%%%%%%%%
%%%%%%%%%%%%%%%%%%%%%%%%%%%%%%%%%%%%%%%%%%%%%%%%%%%%%%%%%%%%%%%%%%%%%%%%%%%%%%%%
\section{Usage}

First of all, the package \textsf{childdoc} is \emph{not} a standard
\LaTeXe{} |.sty| style file! Therefore it needs to be invoked in
a non-standard way.

%%%%%%%%%%%%%%%%%%%%%%%%%%%%%%%%%%%%%%%%%%%%%%%%%%%%%%%%%%%%%%%%%%%%%%%%%%%%%%%%
\subsection{Included Files}
\label{sec:include}

%%%%%%%%%%%%%%%%%%%%%%%%%%%%%%%%%%%%%%%%
\DescribeMacro{\childdocmain}
To use the package, add the commands
\begin{center}
\begin{tabular}{l}
|\input{childdoc.def}|\\
|\childdocmain{}|\\
\end{tabular}
\end{center}
at the very top of the main \LaTeX{} file,
in particular \emph{before} the |\documentclass| statement!
The argument of |\childdocmain| should be left empty
(but it must be present).

%%%%%%%%%%%%%%%%%%%%%%%%%%%%%%%%%%%%%%%%
\DescribeMacro{\childdocof}
Furthermore, add the commands
\begin{center}
\begin{tabular}{l}
|\input{childdoc.def}|\\
|\childdocof{|\textit{main}|}|\\
\end{tabular}
\end{center}
at the top of every child file \textit{child}
which is included by |\include{|\textit{child}|}|
from within the main file
(or at least for those files to be compiled individually).
The argument \textit{main} must be the filename of the main file.

There are a couple of
considerations in setting up the main and child documents:

%%%%%%%%%%%%%%%%%%%%%%%%%%%%%%%%%%%%%%%%
\paragraph{Restrictions.}

Please note the following restrictions:
\begin{itemize}
\item
|\childdocmain| must be called with one argument \textit{main}
to ensure compatibility with earlier version of the package.
It must either be empty (|\childdocmain{}|)
or precisely match the filename of the main file in which it is specified.
See \secref{sec:detection} for further information.
\item
The filename \textit{main} must be specified without the |.tex| extension.
\item
The filename \textit{main} is case sensitive
(even in case-insensitive file systems)
due to internal string comparison.
\item
The argument \textit{main} should be fully expanded, it cannot be a macro.
\item
Subdirectories and special characters should be avoided in filenames.
\item
The command |\childdocmain{|\textit{main}|}| must be followed by a whitespace.
It should not be followed immediately by another command
or by a comment mark `|%|'.
This is because the \TeX{} parser reads the token immediately following
the argument of |\childdocmain| and puts it
at the beginning of every child section;
however, a white\-space is ignored.
\end{itemize}

%%%%%%%%%%%%%%%%%%%%%%%%%%%%%%%%%%%%%%%%
\paragraph{Content of Main File.}

It is advisable to place all content in the child files included by |\include|.
Any output contained in the main file will appear in all child documents
unless suppressed manually;
it cannot be suppressed automatically by the |\includeonly| directive
and thus should normally be avoided.
A method to include some content in the main file
by means of conditional processing is described in \secref{sec:conditional}.

%%%%%%%%%%%%%%%%%%%%%%%%%%%%%%%%%%%%%%%%
\paragraph{Page Numbering.}

When only a part of the document is compiled,
the appropriate numbering of pages
(as well as other status parameters)
is determined from the |.aux| files.
The latter contain information from previous passes.
However this information needs to propagate through
all intermediate child documents.
Therefore the page numbering in child documents may well
be inconsistent until the complete document is compiled at least once.

A useful (if unconventional) way to always ensure a consistent
page numbering is to restart the numbering in each child document
and denote the pages by `\textit{child}|.|\textit{page}'
where \textit{child} represents the chapter/section number of the child file.
This can be achieved by the command
|\numberwithin{page}{|\textit{child}|}|
of the \textsf{amsmath} package
where \textit{child} can be |chapter| or |section|
depending on the chosen structuring.
Alternatively, one can modify the macro |\thepage| appropriately
and reset the counter |page| at the start of each child file.

%%%%%%%%%%%%%%%%%%%%%%%%%%%%%%%%%%%%%%%%%%%%%%%%%%%%%%%%%%%%%%%%%%%%%%%%%%%%%%%%
\subsection{Conditional Processing}
\label{sec:conditional}

The package provides a mechanism to compile different versions
of a document. To customise the versions further some conditional processing
can come in handy to distinguish which version is being compiled.
The package provides two macros to describe the compilation context:

%%%%%%%%%%%%%%%%%%%%%%%%%%%%%%%%%%%%%%%%
\DescribeMacro{\ifchilddoc}
The conditional |\ifchilddoc| distinguishes between the compilation of
child documents and the main document:
%
\begin{center}
|\ifchilddoc |\textit{child-code}| |[|\||else |\textit{main-code}]| \||fi|
\end{center}

%%%%%%%%%%%%%%%%%%%%%%%%%%%%%%%%%%%%%%%%
\DescribeMacro{\childdocname}
\DescribeMacro{\childdocjob}
The macro |\childdocname| contains the filename (without extension)
of the main or child file being processed.
Note that |\childdocjob| will always contain the name of the main file.

%%%%%%%%%%%%%%%%%%%%%%%%%%%%%%%%%%%%%%%%
\paragraph{Title Page.}

Conditional processing can be used to include a title or banner page
in the main document when proper precautions are taken.
Importantly, the code in the main file should ensure that the page counter
(as well as other status parameters which are stored in the |.aux| files)
takes the same value after the conditional processing.
Otherwise the page numbers may take divergent values
depending on which part is compiled.

For example, a title page could be declared by:
%
\begin{center}
\begin{tabular}{l}
|\ifchilddoc\||else|\\
|\addtocounter{page}{-1}|\\
\textit{code for title page}\\
|\newpage|\\
|\||fi|
\end{tabular}
\end{center}
%
A banner page for the child documents can be generated by:
%
\begin{center}
\begin{tabular}{l}
|\ifchilddoc|\\
|\addtocounter{page}{-1}|\\
\textit{code for banner page}\\
|\newpage|\\
|\||fi|
\end{tabular}
\end{center}
%
Here one could write a message such as:
\begin{center}
|This is the part \childdocname{} of \childdocjob{}.|
\end{center}

%%%%%%%%%%%%%%%%%%%%%%%%%%%%%%%%%%%%%%%%%%%%%%%%%%%%%%%%%%%%%%%%%%%%%%%%%%%%%%%%
\subsection{Flags}
\label{sec:flags}

The package makes it easy to generate different versions
of the main or child documents.
To this end compilation flags can be defined
and assigned different default values.
They will be particularly useful in conjunction
with the forwarding mechanism described in \secref{sec:forward}.

For example, it may be useful to have a flag |\version|
which can be set to |draft| or |final|.
The document source will contain some conditional code
depending on the value of |\version|.
Suppose further, the flag should default to |final| for the main file
and to |draft| for child files
which is a natural assignment for editing the document.
This is achieved by placing the following code
in the preamble of the main document
(below the |\childdocmain| directive):
%
\begin{center}
\begin{tabular}{l}
|\ifchilddoc|\\
|\providecommand{\version}{draft}|\\
|\||else|\\
|\providecommand{\version}{final}|\\
|\||fi|
\end{tabular}
\end{center}
%
The definition by |\providecommand| makes sure
that previous definitions are not overwritten.
Further statements |\providecommand{\version}{...}|
can thus be added before the above code to override it.

For the main file, one might add a line
(between |\childdocmain| and the above block)
%
\begin{center}
|%\ifchilddoc\||else\providecommand{\version}{draft}\||fi|
\end{center}
%
which can be uncommented to produce a draft version.
Likewise one can add a line to the very top of a child file
(above the |\childdocof{|\textit{main}|}| directive)
%
\begin{center}
|%\providecommand{\version}{final}|
\end{center}
%
which can be uncommented to produce the final version of this child document.

%%%%%%%%%%%%%%%%%%%%%%%%%%%%%%%%%%%%%%%%%%%%%%%%%%%%%%%%%%%%%%%%%%%%%%%%%%%%%%%%
\subsection{Forwarding}
\label{sec:forward}

Different versions of the main or child documents
using compilation flags as described in \secref{sec:flags}
can be (permanently) stored in different files
for convenient compilation, viewing and distribution.
To this end, the package defines a command
to pass on compilation to a different file:

%%%%%%%%%%%%%%%%%%%%%%%%%%%%%%%%%%%%%%%%
\DescribeMacro{\childdocforward}
The command |\childdocforward| redirects processing to
another source file:
%
\begin{center}
\begin{tabular}{l}
|\input{childdoc.def}|\\
|\childdocforward[|\textit{main}|]{|\textit{dest}|}|\\
\end{tabular}
\end{center}
%
The argument \textit{dest} is the destination file
(without extension).
It should be the main file or one of the child files.
Note that further \textsf{childdoc} directives
such as |\childdocof| and |\childdocforward|
in the indicated file will be processed in this form.
The optional argument \textit{main}
passes on directly to the main file \textit{main}
while pretending to compile the child \textit{dest}.
This form behaves as if \textit{dest}
issues |\childdocof{|\textit{main}|}| right away,
and no further \textsf{childdoc} directives will be processed.

%%%%%%%%%%%%%%%%%%%%%%%%%%%%%%%%%%%%%%%%
\DescribeMacro{\...prefix}
In the alternative form |\childdocforwardprefix|,
%
\begin{center}
\begin{tabular}{l}
|\input{childdoc.def}|\\
|\childdocforwardprefix[|\textit{main}|]{|\textit{prefix}|}{|\textit{dest}|}|
\end{tabular}
\end{center}
%
the destination file is determined by a pattern
depending on the current file:
To make this work, the current file must be called
`{\textit{prefix}\hspace{0.2em}\textit{suffix}}'
with \textit{prefix} matching precisely the argument.
Processing is then passed on to the file
`{\textit{dest}\hspace{0.2em}\textit{suffix}}'.
Surely, the same effect is achieved by
directly specifying the
argument `{\textit{dest}\hspace{0.2em}\textit{suffix}}'
in the first form.
However, that requires to set up a different file
for each child. With the alternative form of the command
all these files can have exactly the same content
which simplifies setting them up and maintaining them.

For example, the following file |draft.tex|
with a compilation flag |\version| as described in \secref{sec:flags}
compiles the main document as a draft:
%
\begin{center}
\begin{tabular}{l}
|\def\version{draft}|\\
|\input{childdoc.def}|\\
|\childdocforward{|\textit{main}|}|
\end{tabular}
\end{center}
%
Likewise, the following files |final|\textit{nn}|.tex|
compile the final version of the child document
|child|\textit{nn}|.tex|:
%
\begin{center}
\begin{tabular}{l}
|\def\version{final}|\\
|\input{childdoc.def}|\\
|\childdocforwardprefix{final}{child}|
\end{tabular}
\end{center}
%

Note that when several versions of a main file and/or of each child file
are to be generated, it may be convenient to set up a |Makefile| or
shell script to automatise the process.

%%%%%%%%%%%%%%%%%%%%%%%%%%%%%%%%%%%%%%%%%%%%%%%%%%%%%%%%%%%%%%%%%%%%%%%%%%%%%%%%
\subsection{Command Line Processing}
\label{sec:commandline}

The effect of redirection files can also be achieved by invoking
the \LaTeX{} compiler with a more elaborate command line.
Most conveniently this should be done as part
of a shell script or a |Makefile|.

When using \textsf{childdoc} in the main file, the following
command lines effectively perform a redirection
(note that depending on the shell being used,
backslashes may have to be doubled: `|\|' $\to$ `|\\|'):
%
\begin{center}
|... -jobname "|\textit{target}|" |\\|"|[\textit{flags}]%
|\input{childdoc.def}\childdocforward[|\textit{main}|]{|\textit{dest}|}"|
\end{center}
%
Here \textit{target} is the name of the output file,
\textit{main} is the name of the main file
and \textit{dest} is the name of the main or child file to be processed
(all filenames without extensions).
The optional argument \textit{main} can be omitted
if \textit{main} matches \textit{dest}.
Optionally, compilation \textit{flags} can be defined via |\def| commands.
This command line makes the \TeX{} engine believe
it is compiling the file \textit{target}
whose content is specified as the latter parameter.
The provided code then forwards the processing to
\textit{main} or \textit{dest} as described in \secref{sec:forward}.

%%%%%%%%%%%%%%%%%%%%%%%%%%%%%%%%%%%%%%%%%%%%%%%%%%%%%%%%%%%%%%%%%%%%%%%%%%%%%%%%
\subsection{Include by Input}
\label{sec:input}

Including child documents by |\include| has some restrictions by design.
Most notably, the content of a child document always occupies
its own set of pages; pages cannot be shared between child documents.
Usually, this behaviour makes perfect sense
because each child document contain an essential part of the document.
However, in some situations it may be desirable to compose
a document from a collection of parts
without having mandatory page breaks between then.
For this case, the package
provides a mechanism to include parts
by |\input| which can also be processed individually.
However, by construction this mechanism
requires manual handling of the content to be output.

%%%%%%%%%%%%%%%%%%%%%%%%%%%%%%%%%%%%%%%%
\DescribeMacro{\ifchilddocmanual}
The main file should be prepared as usual, see \secref{sec:include}.
However, the document body must make a distinction
between processing of an individual part and of the main document, e.g.:
%
\begin{center}
\begin{tabular}{l}
|\ifchilddocmanual|\\
|\input{\childdocname}|\\
|\||else|\\
\textit{document body with }|\input{|\textit{part}|}|\\
|\||fi|
\end{tabular}
\end{center}
%
The conditional |\ifchilddocmanual| is true whenever
a part to be included by |\input| is being compiled,
and the name of the part is stored in |\childdocname|.

%%%%%%%%%%%%%%%%%%%%%%%%%%%%%%%%%%%%%%%%
\DescribeMacro{\childdocby}
Each part to be included by |\input| should start with:
%
\begin{center}
\begin{tabular}{l}
|\input{childdoc.def}|\\
|\childdocby{|\textit{main}|}|\\
\end{tabular}
\end{center}
%
The directive |\childdocby| is similar to |\childdocof|
described in \secref{sec:include},
but the subsequent selection of content must be done manually.
To that end, both |\ifchilddoc| and |\ifchilddocmanual|
will be true upon processing of a part,
and the name of the part is stored in |\childdocname|.
Note that |\jobname| will be set to the filename of the current part
so that each part receives an individual |.aux| file
that does not interfere with the |.aux| file(s) of the main document.
This behaviour can be altered by the alternative form
|\childdocby[*]{|\textit{main}|}| (with a non-empty optional argument)
which uses the |.aux| file of the main document
by setting |\jobname| to \textit{main}.

%%%%%%%%%%%%%%%%%%%%%%%%%%%%%%%%%%%%%%%%%%%%%%%%%%%%%%%%%%%%%%%%%%%%%%%%%%%%%%%%
\subsection{Driver Development}
\label{sec:driver}

The \textsf{childdoc} mechanism can also be use for the development
of definition files such as \LaTeX{} styles or classes.
This case differs from the above setup with multiple parts
included by |\include| in that no |\includeonly| should be invoked.
This can be achieved by starting the include file
(before |\ProvidesPackage|) with:
%
\begin{center}
\begin{tabular}{l}
|\input{childdoc.def}|\\
|\childdocforward{|\textit{main}|}|\\
\end{tabular}
\end{center}
%
or alternatively with:
%
\begin{center}
\begin{tabular}{l}
|\input{childdoc.def}|\\
|\childdocby{|\textit{main}|}|\\
\end{tabular}
\end{center}
%
Both forms have slightly different effects as described above.
The main file is prepared as usual, see \secref{sec:include}.

%%%%%%%%%%%%%%%%%%%%%%%%%%%%%%%%%%%%%%%%%%%%%%%%%%%%%%%%%%%%%%%%%%%%%%%%%%%%%%%%
\subsection{Legacy Detection}
\label{sec:detection}

The directive |\childdocmain| in the main file can detect
whether the complete document or merely a child is to be compiled
even without using the directive |\childdocof|.
This method is deprecated because it is less robust
and there is no compelling reason to use it;
it is merely provided for backward compatibility
and it may be removed in future versions.

If the detection mechanism is to be used,
it is mandatory to correctly specify
the filename of the main file as the argument of |\childdocmain|:
%
\begin{center}
\begin{tabular}{l}
|\input{childdoc.def}|\\
|\childdocmain{|\textit{main}|}|\\
\end{tabular}
\end{center}
%
If |\jobname| does not match the argument \textit{main} of |\childdocmain|,
it is assumed that |\jobname| points to the child file to be compiled.
When using |\childdocmain| with the main file specified as argument,
it suffices to start a child file
with just |\input{|\textit{main}|}|
without loading of the package and using |\childdocof|.
If instead all processing is done
with the appropriate \textsf{childdoc} directives,
the argument of \textit{main} of |\childdocmain| can be empty.

An alternative version of the command line processing described
in \secref{sec:commandline} using the detection mechanism reads:
%
\begin{center}
|... -jobname "|\textit{target}|" "|[\textit{flags}]%
[|\def\jobname{|\textit{dest}|}|]|\input{|\textit{main}|}"|
\end{center}

%%%%%%%%%%%%%%%%%%%%%%%%%%%%%%%%%%%%%%%%%%%%%%%%%%%%%%%%%%%%%%%%%%%%%%%%%%%%%%%%
\subsection{Manual Code}
\label{sec:manual}

In case one cannot be certain whether the definitions file |childdoc.def|
is installed on the target \TeX{} distribution
and one prefers not to ship it,
it is conceivable to paste a few relevant commands into the sources.

To that end, drop all statements |\input{childdoc.def}|
and perform the replacements as outlined below.
Instead of |\childdocmain{|\textit{main}|}| add the following code
to the top of the main file:
%
\begin{center}
\begin{tabular}{l}
|\||ifdefined\childdocname\endinput\||fi\newif\ifchilddoc|\\
|\edef\childdocname{\scantokens\expandafter{\jobname\noexpand}}|\\
|\def\childdocmain{|\textit{main}|}\||ifx\childdocmain\childdocname\||else|\\
|\childdoctrue\includeonly{\childdocname}\let\jobname\childdocmain\||fi|\\
\end{tabular}
\end{center}
%
Instead of |\childdocof{|\textit{main}|}| just include the main file
at the top of each child file:
%
\begin{center}
|\input{|\textit{main}|}|
\end{center}
%
A simple redirection |\childdocforward{|\textit{dest}|}| is achieved by:
%
\begin{center}
|\def\jobname{|\textit{dest}|}\input{\jobname}|
\end{center}
%
The redirection with prefix
|\childdocforwardprefix[|\textit{prefix}|]{|\textit{dest}|}|
is accomplished by:
%
\begin{center}
\begin{tabular}{l}
|{\edef\jobname{\scantokens\expandafter{\jobname\noexpand}}|\\
|\def\redirectjob |\textit{prefix}|#1~~~{\gdef\jobname{|\textit{dest}|#1}}|\\
|\expandafter\redirectjob\jobname~~~}\input{\jobname}|
\end{tabular}
\end{center}

In an alternative approach,
child documents can be compiled by a specific command line
without additional code or specific definitions:
%
\begin{center}
|... -jobname "|\textit{target}|" "|[\textit{flags}]%
|\includeonly{|\textit{dest}|}\input{|\textit{main}|}"|
\end{center}
%

%%%%%%%%%%%%%%%%%%%%%%%%%%%%%%%%%%%%%%%%%%%%%%%%%%%%%%%%%%%%%%%%%%%%%%%%%%%%%%%%
%%%%%%%%%%%%%%%%%%%%%%%%%%%%%%%%%%%%%%%%%%%%%%%%%%%%%%%%%%%%%%%%%%%%%%%%%%%%%%%%
\section{Information}

%%%%%%%%%%%%%%%%%%%%%%%%%%%%%%%%%%%%%%%%%%%%%%%%%%%%%%%%%%%%%%%%%%%%%%%%%%%%%%%%
\subsection{Copyright}

Copyright \copyright{} 2017--2018 Niklas Beisert

This work may be distributed and/or modified under the
conditions of the \LaTeX{} Project Public License, either version 1.3
of this license or (at your option) any later version.
The latest version of this license is in
  \url{http://www.latex-project.org/lppl.txt}
and version 1.3 or later is part of all distributions of \LaTeX{}
version 2005/12/01 or later.

This work has the LPPL maintenance status `maintained'.

The Current Maintainer of this work is Niklas Beisert.

This work consists of the files |README.txt|, |childdoc.ins| and |childdoc.dtx|
as well as the derived files |childdoc.def|, |cdocsamp.tex|
with |cdocsch1.tex|, |cdocsch2.tex|, |cdocspt3.tex|, |cdocspt4.tex|,
|cdocsdrf.tex|, |cdocsfn1.tex|, |cdocsfn2.tex|
as well as |childdoc.pdf|.

%%%%%%%%%%%%%%%%%%%%%%%%%%%%%%%%%%%%%%%%%%%%%%%%%%%%%%%%%%%%%%%%%%%%%%%%%%%%%%%%
\subsection{Files and Installation}

The package consists of the files:
%
\begin{center}
\begin{tabular}{ll}
    |README.txt|   & readme file \\
    |childdoc.ins| & installation file \\
    |childdoc.dtx| & source file \\
    |childdoc.def| & definition file \\
    |cdocsamp.tex| & sample main file \\
    |cdocsch1.tex| & sample include file \\
    |cdocsch2.tex| & sample include file \\
    |cdocspt3.tex| & sample part file \\
    |cdocspt4.tex| & sample part file \\
    |cdocsdrf.tex| & sample redirection file \\
    |cdocsfn1.tex| & sample redirection file \\
    |cdocsfn2.tex| & sample redirection file \\
    |childdoc.pdf| & manual
\end{tabular}
\end{center}
%
The distribution consists of the files
|README.txt|, |childdoc.ins| and |childdoc.dtx|.
%
\begin{itemize}
\item
Run (pdf)\LaTeX{} on |childdoc.dtx|
to compile the manual |childdoc.pdf| (this file).
\item
Run \LaTeX{} on |childdoc.ins| to create the definitions file |childdoc.def|
and the sample |cdocsamp.tex| with include files
|cdocsch1.tex|, |cdocsch2.tex|, |cdocspt3.tex|, |cdocspt4.tex|,
|cdocsdrf.tex|, |cdocsfn1.tex|, |cdocsfn2.tex|.
Then copy the file |childdoc.def| to an appropriate directory of your \LaTeX{}
distribution, e.g.\ \textit{texmf-root}|/tex/latex/childdoc|.
\end{itemize}

%%%%%%%%%%%%%%%%%%%%%%%%%%%%%%%%%%%%%%%%%%%%%%%%%%%%%%%%%%%%%%%%%%%%%%%%%%%%%%%%
\subsection{Related CTAN Packages}

There are several other packages which offer a similar functionality:
%
\begin{itemize}
\item
The packages
\href{http://ctan.org/pkg/docmute}{\textsf{docmute}},
\href{http://ctan.org/pkg/includex}{\textsf{includex}} and
\href{http://ctan.org/pkg/standalone}{\textsf{standalone}}
provide commands to include only the document body of
a child file thus allowing both files to be compiled individually.
\item
The packages \href{http://ctan.org/pkg/subdocs}{\textsf{subdocs}}
and \href{http://ctan.org/pkg/subfiles}{\textsf{subfiles}}
provide structures in which the main and child documents can be
encapsulated and allowing them to be compiled individually.
The inclusion mechanism is different from the conventional |\include|.
\item
The package \href{http://ctan.org/pkg/combine}{\textsf{combine}}
is an elaborate solution to combine several documents into one.
\end{itemize}
%
See also the CTAN topic \href{http://ctan.org/topic/subdocs}{\textsf{subdocs}}
for further related packages.
The present package differs from the above solutions in that
a document structure constructed with the conventional |\include| mechanism
just needs two extra commands at the top of every file
such that all constituent files can be compiled individually.

%%%%%%%%%%%%%%%%%%%%%%%%%%%%%%%%%%%%%%%%%%%%%%%%%%%%%%%%%%%%%%%%%%%%%%%%%%%%%%%%
%\subsection{Feature Suggestions}
%
%The following is a list of features which may be useful for future
%versions of this package:
%%
%\begin{itemize}
%\item
%\ldots
%\end{itemize}

%%%%%%%%%%%%%%%%%%%%%%%%%%%%%%%%%%%%%%%%%%%%%%%%%%%%%%%%%%%%%%%%%%%%%%%%%%%%%%%%
\subsection{Revision History}

%%%%%%%%%%%%%%%%%%%%%%%%%%%%%%%%%%%%%%%%
\paragraph{v2.0:} 2018/12/30

\begin{itemize}
\item
immediate forward processing
\item
added |\childdocby| mechanism
\item
manual restructured
\end{itemize}

%%%%%%%%%%%%%%%%%%%%%%%%%%%%%%%%%%%%%%%%
\paragraph{v1.6:} 2018/01/17

\begin{itemize}
\item
application for development of include files
\item
corrections to manual
\end{itemize}

%%%%%%%%%%%%%%%%%%%%%%%%%%%%%%%%%%%%%%%%
\paragraph{v1.5:} 2017/05/21

\begin{itemize}
\item
more complete structuring introduced
\item
|\childdocof| introduced
\item
|\childdoc| renamed to |\childdocmain|
\item
|\childredirect| renamed to |\childdocforward| and |\childdocforwardprefix|
and functionality expanded
\end{itemize}

%%%%%%%%%%%%%%%%%%%%%%%%%%%%%%%%%%%%%%%%
\paragraph{v1.0:} 2017/04/27

\begin{itemize}
\item
manual and install package
\item
first version published on CTAN
\end{itemize}

%%%%%%%%%%%%%%%%%%%%%%%%%%%%%%%%%%%%%%%%
\paragraph{v0.6:} 2017/04/26

\begin{itemize}
\item
redirection mechanism added
\end{itemize}

%%%%%%%%%%%%%%%%%%%%%%%%%%%%%%%%%%%%%%%%
\paragraph{v0.5:} 2017/04/26

\begin{itemize}
\item
functionality in definition file
\end{itemize}


%%%%%%%%%%%%%%%%%%%%%%%%%%%%%%%%%%%%%%%%%%%%%%%%%%%%%%%%%%%%%%%%%%%%%%%%%%%%%%%%
%%%%%%%%%%%%%%%%%%%%%%%%%%%%%%%%%%%%%%%%%%%%%%%%%%%%%%%%%%%%%%%%%%%%%%%%%%%%%%%%
%%%%%%%%%%%%%%%%%%%%%%%%%%%%%%%%%%%%%%%%%%%%%%%%%%%%%%%%%%%%%%%%%%%%%%%%%%%%%%%%
\appendix

\settowidth\MacroIndent{\rmfamily\scriptsize 000\ }

 \DocInput{childdoc.dtx}

\end{document}
%</driver>
% \fi
%
% %%%%%%%%%%%%%%%%%%%%%%%%%%%%%%%%%%%%%%%%%%%%%%%%%%%%%%%%%%%%%%%%%%%%%%%%%%%%%%
% %%%%%%%%%%%%%%%%%%%%%%%%%%%%%%%%%%%%%%%%%%%%%%%%%%%%%%%%%%%%%%%%%%%%%%%%%%%%%%
% \section{Sample}
%\iffalse
%<*samplemain>
%\fi
%
% The following presents a sample document
% with two chapters, two parts, a title page,
% a compile flag as well as three forwarding files to set the flag.
% It consists of eight |.tex| files:
% \begin{center}
% \begin{tabular}{ll}
% |cdocsamp.tex|&main file\\
% |cdocsch1.tex|&include file for chapter 1\\
% |cdocsch2.tex|&include file for chapter 2\\
% |cdocspt3.tex|&include file for part 3\\
% |cdocspt4.tex|&include file for part 4\\
% |cdocsdrf.tex|&forwarding file for main file in draft mode\\
% |cdocsfi1.tex|&forwarding file for final version of chapter 1\\
% |cdocsfi2.tex|&forwarding file for final version of chapter 2\\
% \end{tabular}
% \end{center}
% Each of the eight files can be compiled directly by the \LaTeX{} compiler.
%
% %%%%%%%%%%%%%%%%%%%%%%%%%%%%%%%%%%%%%%
% \paragraph{Main File.}
%
% The main file is called |cdocsamp.tex|.
%
% Load the \textsf{childdoc} definitions and
% declare the filename for the main document:
%    \begin{macrocode}
\input{childdoc.def}
\childdocmain{}
%    \end{macrocode}

% Optional override for |\version| flag:
%    \begin{macrocode}
%%\ifchilddoc\else\providecommand{\version}{draft}\fi
%    \end{macrocode}

% Define the default values for the |\version| flag
% (|final| for the main file and |draft| for childs):
%    \begin{macrocode}
\ifchilddoc
\providecommand{\version}{draft}
\else
\providecommand{\version}{final}
\fi
%    \end{macrocode}

% Load the standard document class:
%    \begin{macrocode}
\documentclass[12pt]{article}
%    \end{macrocode}

% Start the document body:
%    \begin{macrocode}
\begin{document}
%    \end{macrocode}

% Declare a title page.
% Print title, part of document being processed and version flag:
%    \begin{macrocode}
\addtocounter{page}{-1}
\begin{center}
{\LARGE\bfseries{}childdoc example\par}
\vspace{1cm}
\ifchilddoc
\ifchilddocmanual part\else chapter\fi:
`\childdocname' of `\childdocjob'\par
\else
main document: `\childdocjob'\par
\fi
version: \version\par
\end{center}
\newpage
%    \end{macrocode}

% Manually include selected file,
% otherwise process as usual:
%    \begin{macrocode}
\ifchilddocmanual
\section*{part `\childdocname'}
\input{\childdocname}
\else
%    \end{macrocode}

% Include the two chapters:
%    \begin{macrocode}
\include{cdocsch1}
\include{cdocsch2}
%    \end{macrocode}

% Include the two parts unless only chapters should be displayed:
%    \begin{macrocode}
\ifchilddoc\else
\section{part three}
\input{cdocspt3}
\section{part four}
\input{cdocspt4}
\fi
%    \end{macrocode}

% Process as usual until here:
%    \begin{macrocode}
\fi
%    \end{macrocode}

% End of document body:
%    \begin{macrocode}
\end{document}
%    \end{macrocode}
%\iffalse
%</samplemain>
%\fi
%
% %%%%%%%%%%%%%%%%%%%%%%%%%%%%%%%%%%%%%%
% \paragraph{Chapter Include Files.}
%
% The include files are called |cdocsch1.tex| and |cdocsch2.tex|.
%
%\iffalse
%<*samplechap1|samplechap2>
%\fi

% Optional override for |\version| flag:
%    \begin{macrocode}
%%\providecommand{\version}{final}
%    \end{macrocode}

% Include the main document:
%    \begin{macrocode}
\input{childdoc.def}
\childdocof{cdocsamp}
%    \end{macrocode}

%\iffalse
%</samplechap1|samplechap2>
%\fi
%
%\iffalse
%<*samplechap1>
%\fi
% Some text for chapter 1:
%    \begin{macrocode}
\section{one}
some text in chapter one
%    \end{macrocode}

%\iffalse
%</samplechap1>
%\fi
% Some text for chapter 2:
%\iffalse
%<*samplechap2>
%\fi
%    \begin{macrocode}
\section{two}
more text in chapter two
%    \end{macrocode}

%\iffalse
%</samplechap2>
%\fi
%
% %%%%%%%%%%%%%%%%%%%%%%%%%%%%%%%%%%%%%%
% \paragraph{Part Include Files.}
%
% The include files are called |cdocspt3.tex| and |cdocspt4.tex|.
%
%\iffalse
%<*samplepart3|samplepart4>
%\fi

% Optional override for |\version| flag:
%    \begin{macrocode}
%%\providecommand{\version}{final}
%    \end{macrocode}

% Include the main document:
%    \begin{macrocode}
\input{childdoc.def}
\childdocby{cdocsamp}
%    \end{macrocode}

%\iffalse
%</samplepart3|samplepart4>
%\fi
%
%\iffalse
%<*samplepart3>
%\fi
% Some text for part 3:
%    \begin{macrocode}
some text in part three
%    \end{macrocode}

%\iffalse
%</samplepart3>
%\fi
% Some text for part 4:
%\iffalse
%<*samplepart4>
%\fi
%    \begin{macrocode}
more text in part four
%    \end{macrocode}

%\iffalse
%</samplepart4>
%\fi
%
% %%%%%%%%%%%%%%%%%%%%%%%%%%%%%%%%%%%%%%
% \paragraph{Forwarding for a Complete Draft.}
%
% The following forwarding file |cdocsdrf.tex|
% compiles the main document in draft mode:
%\iffalse
%<*sampledraft>
%\fi
%    \begin{macrocode}
\def\version{draft}
\input{childdoc.def}
\childdocforward{cdocsamp}
%    \end{macrocode}

%\iffalse
%</sampledraft>
%\fi
%
% %%%%%%%%%%%%%%%%%%%%%%%%%%%%%%%%%%%%%%
% \paragraph{Forwarding for Final Version of the Chapters.}
%
% The following forwarding files |cdocsfn1.tex| and |cdocsfn2.tex|
% (with identical content)
% compile the final versions of the child documents
% |cdocsch1.tex| and |cdocsch2.tex|, respectively:
%\iffalse
%<*samplefinal>
%\fi
%    \begin{macrocode}
\def\version{final}
\input{childdoc.def}
\childdocforwardprefix[cdocsamp]{cdocsfn}{cdocsch}
%    \end{macrocode}

%\iffalse
%</samplefinal>
%\fi
%
% %%%%%%%%%%%%%%%%%%%%%%%%%%%%%%%%%%%%%%
% \paragraph{Command Line Processing.}
%
% The following three command lines generate the output files
% |cdocscld|, |cdocscl1| and |cdocscl2|
% which should be identical to
% |cdocsdrf|, |cdocsch1| and |cdocsfn2|, respectively:
% \begin{center}
% \begin{tabular}{l}
% |latex -jobname cdocscld \|\\
% |  "\def\version{draft}\input{childdoc.def}\childdocforward{cdocsamp}"|\\
% |latex -jobname cdocscl1 \|\\
% |  "\input{childdoc.def}\childdocforward[cdocsamp]{cdocsch1}"|\\
% |latex -jobname cdocscl2 \|\\
% |  "\def\version{final}\input{childdoc.def}\childdocforward{cdocsch2}"|
% \end{tabular}
% \end{center}
% Note that the trailing backslash on each first line
% merely continues the input to the second line
% (for convenient cut ant paste).
% Furthermore, the command |latex| can be replaced by any
% of its alternative versions such as |pdflatex|.
%
% %%%%%%%%%%%%%%%%%%%%%%%%%%%%%%%%%%%%%%%%%%%%%%%%%%%%%%%%%%%%%%%%%%%%%%%%%%%%%%
% %%%%%%%%%%%%%%%%%%%%%%%%%%%%%%%%%%%%%%%%%%%%%%%%%%%%%%%%%%%%%%%%%%%%%%%%%%%%%%
% \section{Implementation}
%\iffalse
%<*package>
%\fi
%
% This section describes the definitions file |childdoc.def|.

% The definitions cannot be loaded using |\usepackage| or |\RequirePackage|
% which has a mechanism to prevent loading a style file more than once.
% When loading the definitions by means of |\input|
% multiple instances have to be prevented manually:
%\iffalse
%This code needs to be before the `\ProvidesFile' directive
%which is defined at the beginning of this file.
%Therefore it is also placed there and commented out here.
%</package>
%<*discard>
%\fi
%    \begin{macrocode}
\ifdefined\childdocmain\endinput\fi
%    \end{macrocode}
%\iffalse
%</discard>
%<*package>
%\fi
%
% \macro{\ifchilddoc}
% \macro{\ifchilddocmanual}
% The conditional |\ifchilddoc| tells whether a
% child (true) or main (false) document is being compiled.
% The conditional |\ifchilddocmanual| tells whether
% the |\includeonly| mechanism is used (false) or
% the selection of child files must be performed manually (true).
% The definitions initialise to false:
%    \begin{macrocode}
\newif\ifchilddoc
\newif\ifchilddocmanual
%    \end{macrocode}

% \macro{\childdocname}
% \macro{\childdocjob}
% The macro |\childdocname| stores the name of the main document
% to be compiled. The macro |\childdocjob| stores the name of
% the document on which the \LaTeX{} compiler was originally invoked.
% The content of |\jobname| cannot be compared
% to filenames specified in the source due to different catcodes.
% The following code rescans |\jobname|, stores the result
% in |\childdocname| and saves a copy in |\childdocjob|:
%    \begin{macrocode}
\edef\childdocname{\scantokens\expandafter{\jobname\noexpand}}
\let\childdocjob\childdocname
%    \end{macrocode}

% \macro{\childdocdisable}
% The macro |\childdocdisable| prevents the main file
% from being processed more than once.
% At this stage, the main document command |\childdocmain|
% is assumed to be called once again where it should do nothing.
% Any subsequent call to it should prevent
% a secondary processing of the main document
% It overwrites the forwarding commands
% |\childdocof| and |\childdocforward|
% with empty macros to prevent further inclusions of the main document:
%    \begin{macrocode}
\newcommand{\childdocdisable}
{
  \renewcommand{\childdocmain}[1]{\renewcommand{\childdocmain}[1]{\endinput}}
  \renewcommand{\childdocof}[1]{}
  \renewcommand{\childdocby}[2][]{}
  \renewcommand{\childdocforward}[2][]{}
  \renewcommand{\childdocdisable}{}
}
%    \end{macrocode}

% \macro{\childdocmain}
% The macro |\childdocmain| is to be called at the top of the main file
% with nothing or the main filename (without extension) as argument.
% First, it breaks loops.
% If the argument is not empty and does not match |\childdocname|
% (which is set by the first inclusion of |childdoc.def|),
% |\ifchilddoc| is set to true, |\includeonly| is applied to the child file
% and |\jobname| is set to the main file
% (for proper handling of |.aux| files):
%    \begin{macrocode}
\newcommand{\childdocmain}[1]
{
  \childdocdisable\childdocmain{}
  \if?#1?\else
    \begingroup
      \def\childdoctmp{#1}
      \ifx\childdoctmp\childdocname
        \def\childdoctmp{}
      \else
        \def\childdoctmp
        {
          \childdoctrue
          \includeonly{\childdocname}
          \def\childdocjob{#1}
          \def\jobname{#1}
        }
      \fi
      \expandafter
    \endgroup
    \childdoctmp
  \fi
}
%    \end{macrocode}

% \macro{\childdocof}
% The command |\childdocof| redirects
% compilation to the main file |#1|.
%    \begin{macrocode}
\newcommand{\childdocof}[1]
{
  \childdocdisable
  \childdoctrue
  \includeonly{\childdocname}
  \def\jobname{#1}
  \def\childdocjob{#1}
  \input{#1}
}
%    \end{macrocode}

% \macro{\childdocby}
% The command |\childdocby| ....
%    \begin{macrocode}
\newcommand{\childdocby}[2][]
{
  \childdocdisable
  \childdoctrue
  \childdocmanualtrue
  \if?#1?\else
    \def\jobname{#2}
  \fi
  \def\childdocjob{#2}
  \input{#2}
  \endinput
}
%    \end{macrocode}

% \macro{\childdocforward}
% The command |\childdocforward| redirects
% compilation to the main file or
% (if the optional argument is given) a child file.
% Parameters are set as if the main file
% or a child file starting with |\childdocof| was compiled.
% Then compilation is handed over to the main file:
%    \begin{macrocode}
\newcommand{\childdocforward}[2][]
{
  \begingroup
    \if?#1?
      \def\childdoctmp
      {
        \def\childdocname{#2}
        \def\childdocjob{#2}
        \def\jobname{#2}
        \input{#2}
        \endinput
      }
    \else
      \def\childdoctmp
      {
        \childdocdisable
        \def\childdocname{#2}
        \childdoctrue
        \includeonly{#2}
        \def\childdocjob{#1}
        \def\jobname{#1}
        \input{#1}
        \endinput
      }
    \fi
    \expandafter
  \endgroup
  \childdoctmp
}
%    \end{macrocode}

% \macro{\childdocforwardprefix}
% The command |\childdocforwardprefix| redirects
% compilation to the main or a child file by means of a pattern.
% The prefix |#1| in the current filename is replaced by |#2|
% and the suffix of the current filename is kept
% (it is assumed that the filename does not contain the substring `|~~~|'
% which is used as a delimiter).
% Compilation is handed over to the new file by |\childdocforward|:
%    \begin{macrocode}
\newcommand{\childdocforwardprefix}[3][]
{
  \begingroup
    \def\childdocextract #2##1~~~{\def\childdoctmp{\childdocforward[#1]{#3##1}}}
    \expandafter\childdocextract\childdocname~~~
    \expandafter
  \endgroup
  \childdoctmp
}
%    \end{macrocode}

% \macro{\childdoc}
% The deprecated macro |\childdoc| is a legacy version of |\childdocmain|:
%    \begin{macrocode}
\newcommand{\childdoc}{\childdocmain}
%    \end{macrocode}

% \macro{\childdocredirect}
% The deprecated macro |\childdocredirect| is a legacy version
% of |\childdocforward| and |\childdocforwardprefix|:
%    \begin{macrocode}
\newcommand{\childdocredirect}[2][]
{
  \begingroup
    \if?#1?
      \def\childdoctmp{\childdocforward{#2}}
    \else
      \def\childdoctmp{\childdocforwardprefix{#1}{#2}}
    \fi
    \expandafter
  \endgroup
  \childdoctmp
}
%    \end{macrocode}

%\iffalse
%</package>
%\fi
%
\endinput

\childdocforwardprefix[cdocsamp]{cdocsfn}{cdocsch}
%    \end{macrocode}

%\iffalse
%</samplefinal>
%\fi
%
% %%%%%%%%%%%%%%%%%%%%%%%%%%%%%%%%%%%%%%
% \paragraph{Command Line Processing.}
%
% The following three command lines generate the output files
% |cdocscld|, |cdocscl1| and |cdocscl2|
% which should be identical to
% |cdocsdrf|, |cdocsch1| and |cdocsfn2|, respectively:
% \begin{center}
% \begin{tabular}{l}
% |latex -jobname cdocscld \|\\
% |  "\def\version{draft}% \iffalse
%
% childdoc.dtx Copyright (C) 2017-2018 Niklas Beisert
%
% This work may be distributed and/or modified under the
% conditions of the LaTeX Project Public License, either version 1.3
% of this license or (at your option) any later version.
% The latest version of this license is in
%   http://www.latex-project.org/lppl.txt
% and version 1.3 or later is part of all distributions of LaTeX
% version 2005/12/01 or later.
%
% This work has the LPPL maintenance status `maintained'.
%
% The Current Maintainer of this work is Niklas Beisert.
%
% This work consists of the files childdoc.dtx and childdoc.ins
% and the derived files childdoc.def and cdocsamp.tex with
% cdocsch1.tex, cdocsch2.tex, cdocsdrf.tex, cdocsfn1.tex, cdocsfn2.tex.
%
%<package>\ifdefined\childdocmain\endinput\fi
%<package>\ProvidesFile{childdoc.def}[2018/12/30 v2.0 child document driver]
%<samplemain>\ProvidesFile{cdocsamp.tex}[2018/12/30 v2.0 sample for childdoc]
%<*driver>
%\ProvidesFile{childdoc.drv}[2018/12/30 v2.0 childdoc reference manual file]
\PassOptionsToClass{10pt,a4paper}{article}
\documentclass{ltxdoc}

\usepackage[margin=35mm]{geometry}
\usepackage{hyperref}
\usepackage{hyperxmp}
\usepackage[usenames]{color}

\hypersetup{colorlinks=true}
\hypersetup{pdfstartview=FitH}
\hypersetup{pdfpagemode=UseNone}
\hypersetup{pdfsource={}}
\hypersetup{pdflang={en-UK}}
\hypersetup{pdfcopyright={Copyright 2017-2018 Niklas Beisert.
  This work may be distributed and/or modified under the
  conditions of the LaTeX Project Public License, either version 1.3
  of this license or (at your option) any later version.}}
\hypersetup{pdflicenseurl={http://www.latex-project.org/lppl.txt}}
\hypersetup{pdfcontactaddress={ETH Zurich, ITP, HIT K,
  Wolfgang-Pauli-Strasse 27}}
\hypersetup{pdfcontactpostcode={8093}}
\hypersetup{pdfcontactcity={Zurich}}
\hypersetup{pdfcontactcountry={Switzerland}}
\hypersetup{pdfcontactemail={nbeisert@itp.phys.ethz.ch}}
\hypersetup{pdfcontacturl={http://people.phys.ethz.ch/\xmptilde nbeisert/}}

\newcommand{\secref}[1]{\hyperref[#1]{section \ref*{#1}}}

\parskip1ex
\parindent0pt
\let\olditemize\itemize
\def\itemize{\olditemize\parskip0pt}

\begin{document}

\title{The \textsf{childdoc} Package}
\hypersetup{pdftitle={The childdoc Package}}
\author{Niklas Beisert\\[2ex]
  Institut f\"ur Theoretische Physik\\
  Eidgen\"ossische Technische Hochschule Z\"urich\\
  Wolfgang-Pauli-Strasse 27, 8093 Z\"urich, Switzerland\\[1ex]
  \href{mailto:nbeisert@itp.phys.ethz.ch}
  {\texttt{nbeisert@itp.phys.ethz.ch}}}
\hypersetup{pdfauthor={Niklas Beisert}}
\hypersetup{pdfsubject={Manual for the LaTeX2e Package childdoc}}
\date{30 December 2018, \textsf{v2.0}}
\maketitle

\begin{abstract}\noindent
\textsf{childdoc} is a \LaTeXe{} package
that enables the direct compilation
of document sections included by |\include|
to individual files.
\end{abstract}

\begingroup
\parskip0ex
\tableofcontents
\endgroup

%%%%%%%%%%%%%%%%%%%%%%%%%%%%%%%%%%%%%%%%%%%%%%%%%%%%%%%%%%%%%%%%%%%%%%%%%%%%%%%%
%%%%%%%%%%%%%%%%%%%%%%%%%%%%%%%%%%%%%%%%%%%%%%%%%%%%%%%%%%%%%%%%%%%%%%%%%%%%%%%%
\section{Introduction}

\LaTeX{} provides a mechanism to structure a large document (such as a book)
into a main file and several child files (containing the chapters)
using the |\include| command.
This mechanism is beneficial for documents
which span hundreds of pages in order to
make the source file(s) more manageable.
Moreover, compilation can be restricted to
selected child files by means of the |\includeonly| command.
The latter feature can be used to reduce the compilation time while editing
(this was significantly more useful in the earlier days of \LaTeX{})
or to generate a smaller document which is easier to navigate.
Another application of |\includeonly| is to generate
documents consisting of selected parts of the complete document.

However, there are a few drawbacks of the plain |\include| mechanism:
\begin{itemize}
\item
The child files cannot be compiled on their own,
they can only be compiled via the main file.
A naive editing environment
(such as a text editor with an option
to have the current file processed by \LaTeX)
may require one to switch to the main file before compiling;
attempting to compile the child file produces errors.
\item
The main file must be modified (each time)
to adjust the |\includeonly| command
to the present needs. This easily leaves the main file in a messy state.
\item
The generated document will always carry the filename
of the main document. This is inconvenient if
several child files are to be compiled and
to be kept for distribution.
\end{itemize}

The present package provides a simple interface
to make child files individually compilable by \LaTeX{}.
Compiling a child file then has the same effect as compiling
the main file with an |\includeonly| command
to select the appropriate child.
Moreover the generated document will carry the name of the child
rather than the main file.
This resolves all three above issues.

This feature is meant to make the editing of books,
thesis documents and lecture notes somewhat more convenient.
However, the package can also be used efficiently for
composing a series of documents (such as exercise sheets)
which are typically distributed individually.
It then assists the author in generating the individual documents
(potentially in different versions)
as well as a document containing the collected series.
Another application is in developing style files
or other kinds of included material
where compilation of the style file could redirect
to a sample or test file.

%%%%%%%%%%%%%%%%%%%%%%%%%%%%%%%%%%%%%%%%%%%%%%%%%%%%%%%%%%%%%%%%%%%%%%%%%%%%%%%%
%%%%%%%%%%%%%%%%%%%%%%%%%%%%%%%%%%%%%%%%%%%%%%%%%%%%%%%%%%%%%%%%%%%%%%%%%%%%%%%%
\section{Usage}

First of all, the package \textsf{childdoc} is \emph{not} a standard
\LaTeXe{} |.sty| style file! Therefore it needs to be invoked in
a non-standard way.

%%%%%%%%%%%%%%%%%%%%%%%%%%%%%%%%%%%%%%%%%%%%%%%%%%%%%%%%%%%%%%%%%%%%%%%%%%%%%%%%
\subsection{Included Files}
\label{sec:include}

%%%%%%%%%%%%%%%%%%%%%%%%%%%%%%%%%%%%%%%%
\DescribeMacro{\childdocmain}
To use the package, add the commands
\begin{center}
\begin{tabular}{l}
|\input{childdoc.def}|\\
|\childdocmain{}|\\
\end{tabular}
\end{center}
at the very top of the main \LaTeX{} file,
in particular \emph{before} the |\documentclass| statement!
The argument of |\childdocmain| should be left empty
(but it must be present).

%%%%%%%%%%%%%%%%%%%%%%%%%%%%%%%%%%%%%%%%
\DescribeMacro{\childdocof}
Furthermore, add the commands
\begin{center}
\begin{tabular}{l}
|\input{childdoc.def}|\\
|\childdocof{|\textit{main}|}|\\
\end{tabular}
\end{center}
at the top of every child file \textit{child}
which is included by |\include{|\textit{child}|}|
from within the main file
(or at least for those files to be compiled individually).
The argument \textit{main} must be the filename of the main file.

There are a couple of
considerations in setting up the main and child documents:

%%%%%%%%%%%%%%%%%%%%%%%%%%%%%%%%%%%%%%%%
\paragraph{Restrictions.}

Please note the following restrictions:
\begin{itemize}
\item
|\childdocmain| must be called with one argument \textit{main}
to ensure compatibility with earlier version of the package.
It must either be empty (|\childdocmain{}|)
or precisely match the filename of the main file in which it is specified.
See \secref{sec:detection} for further information.
\item
The filename \textit{main} must be specified without the |.tex| extension.
\item
The filename \textit{main} is case sensitive
(even in case-insensitive file systems)
due to internal string comparison.
\item
The argument \textit{main} should be fully expanded, it cannot be a macro.
\item
Subdirectories and special characters should be avoided in filenames.
\item
The command |\childdocmain{|\textit{main}|}| must be followed by a whitespace.
It should not be followed immediately by another command
or by a comment mark `|%|'.
This is because the \TeX{} parser reads the token immediately following
the argument of |\childdocmain| and puts it
at the beginning of every child section;
however, a white\-space is ignored.
\end{itemize}

%%%%%%%%%%%%%%%%%%%%%%%%%%%%%%%%%%%%%%%%
\paragraph{Content of Main File.}

It is advisable to place all content in the child files included by |\include|.
Any output contained in the main file will appear in all child documents
unless suppressed manually;
it cannot be suppressed automatically by the |\includeonly| directive
and thus should normally be avoided.
A method to include some content in the main file
by means of conditional processing is described in \secref{sec:conditional}.

%%%%%%%%%%%%%%%%%%%%%%%%%%%%%%%%%%%%%%%%
\paragraph{Page Numbering.}

When only a part of the document is compiled,
the appropriate numbering of pages
(as well as other status parameters)
is determined from the |.aux| files.
The latter contain information from previous passes.
However this information needs to propagate through
all intermediate child documents.
Therefore the page numbering in child documents may well
be inconsistent until the complete document is compiled at least once.

A useful (if unconventional) way to always ensure a consistent
page numbering is to restart the numbering in each child document
and denote the pages by `\textit{child}|.|\textit{page}'
where \textit{child} represents the chapter/section number of the child file.
This can be achieved by the command
|\numberwithin{page}{|\textit{child}|}|
of the \textsf{amsmath} package
where \textit{child} can be |chapter| or |section|
depending on the chosen structuring.
Alternatively, one can modify the macro |\thepage| appropriately
and reset the counter |page| at the start of each child file.

%%%%%%%%%%%%%%%%%%%%%%%%%%%%%%%%%%%%%%%%%%%%%%%%%%%%%%%%%%%%%%%%%%%%%%%%%%%%%%%%
\subsection{Conditional Processing}
\label{sec:conditional}

The package provides a mechanism to compile different versions
of a document. To customise the versions further some conditional processing
can come in handy to distinguish which version is being compiled.
The package provides two macros to describe the compilation context:

%%%%%%%%%%%%%%%%%%%%%%%%%%%%%%%%%%%%%%%%
\DescribeMacro{\ifchilddoc}
The conditional |\ifchilddoc| distinguishes between the compilation of
child documents and the main document:
%
\begin{center}
|\ifchilddoc |\textit{child-code}| |[|\||else |\textit{main-code}]| \||fi|
\end{center}

%%%%%%%%%%%%%%%%%%%%%%%%%%%%%%%%%%%%%%%%
\DescribeMacro{\childdocname}
\DescribeMacro{\childdocjob}
The macro |\childdocname| contains the filename (without extension)
of the main or child file being processed.
Note that |\childdocjob| will always contain the name of the main file.

%%%%%%%%%%%%%%%%%%%%%%%%%%%%%%%%%%%%%%%%
\paragraph{Title Page.}

Conditional processing can be used to include a title or banner page
in the main document when proper precautions are taken.
Importantly, the code in the main file should ensure that the page counter
(as well as other status parameters which are stored in the |.aux| files)
takes the same value after the conditional processing.
Otherwise the page numbers may take divergent values
depending on which part is compiled.

For example, a title page could be declared by:
%
\begin{center}
\begin{tabular}{l}
|\ifchilddoc\||else|\\
|\addtocounter{page}{-1}|\\
\textit{code for title page}\\
|\newpage|\\
|\||fi|
\end{tabular}
\end{center}
%
A banner page for the child documents can be generated by:
%
\begin{center}
\begin{tabular}{l}
|\ifchilddoc|\\
|\addtocounter{page}{-1}|\\
\textit{code for banner page}\\
|\newpage|\\
|\||fi|
\end{tabular}
\end{center}
%
Here one could write a message such as:
\begin{center}
|This is the part \childdocname{} of \childdocjob{}.|
\end{center}

%%%%%%%%%%%%%%%%%%%%%%%%%%%%%%%%%%%%%%%%%%%%%%%%%%%%%%%%%%%%%%%%%%%%%%%%%%%%%%%%
\subsection{Flags}
\label{sec:flags}

The package makes it easy to generate different versions
of the main or child documents.
To this end compilation flags can be defined
and assigned different default values.
They will be particularly useful in conjunction
with the forwarding mechanism described in \secref{sec:forward}.

For example, it may be useful to have a flag |\version|
which can be set to |draft| or |final|.
The document source will contain some conditional code
depending on the value of |\version|.
Suppose further, the flag should default to |final| for the main file
and to |draft| for child files
which is a natural assignment for editing the document.
This is achieved by placing the following code
in the preamble of the main document
(below the |\childdocmain| directive):
%
\begin{center}
\begin{tabular}{l}
|\ifchilddoc|\\
|\providecommand{\version}{draft}|\\
|\||else|\\
|\providecommand{\version}{final}|\\
|\||fi|
\end{tabular}
\end{center}
%
The definition by |\providecommand| makes sure
that previous definitions are not overwritten.
Further statements |\providecommand{\version}{...}|
can thus be added before the above code to override it.

For the main file, one might add a line
(between |\childdocmain| and the above block)
%
\begin{center}
|%\ifchilddoc\||else\providecommand{\version}{draft}\||fi|
\end{center}
%
which can be uncommented to produce a draft version.
Likewise one can add a line to the very top of a child file
(above the |\childdocof{|\textit{main}|}| directive)
%
\begin{center}
|%\providecommand{\version}{final}|
\end{center}
%
which can be uncommented to produce the final version of this child document.

%%%%%%%%%%%%%%%%%%%%%%%%%%%%%%%%%%%%%%%%%%%%%%%%%%%%%%%%%%%%%%%%%%%%%%%%%%%%%%%%
\subsection{Forwarding}
\label{sec:forward}

Different versions of the main or child documents
using compilation flags as described in \secref{sec:flags}
can be (permanently) stored in different files
for convenient compilation, viewing and distribution.
To this end, the package defines a command
to pass on compilation to a different file:

%%%%%%%%%%%%%%%%%%%%%%%%%%%%%%%%%%%%%%%%
\DescribeMacro{\childdocforward}
The command |\childdocforward| redirects processing to
another source file:
%
\begin{center}
\begin{tabular}{l}
|\input{childdoc.def}|\\
|\childdocforward[|\textit{main}|]{|\textit{dest}|}|\\
\end{tabular}
\end{center}
%
The argument \textit{dest} is the destination file
(without extension).
It should be the main file or one of the child files.
Note that further \textsf{childdoc} directives
such as |\childdocof| and |\childdocforward|
in the indicated file will be processed in this form.
The optional argument \textit{main}
passes on directly to the main file \textit{main}
while pretending to compile the child \textit{dest}.
This form behaves as if \textit{dest}
issues |\childdocof{|\textit{main}|}| right away,
and no further \textsf{childdoc} directives will be processed.

%%%%%%%%%%%%%%%%%%%%%%%%%%%%%%%%%%%%%%%%
\DescribeMacro{\...prefix}
In the alternative form |\childdocforwardprefix|,
%
\begin{center}
\begin{tabular}{l}
|\input{childdoc.def}|\\
|\childdocforwardprefix[|\textit{main}|]{|\textit{prefix}|}{|\textit{dest}|}|
\end{tabular}
\end{center}
%
the destination file is determined by a pattern
depending on the current file:
To make this work, the current file must be called
`{\textit{prefix}\hspace{0.2em}\textit{suffix}}'
with \textit{prefix} matching precisely the argument.
Processing is then passed on to the file
`{\textit{dest}\hspace{0.2em}\textit{suffix}}'.
Surely, the same effect is achieved by
directly specifying the
argument `{\textit{dest}\hspace{0.2em}\textit{suffix}}'
in the first form.
However, that requires to set up a different file
for each child. With the alternative form of the command
all these files can have exactly the same content
which simplifies setting them up and maintaining them.

For example, the following file |draft.tex|
with a compilation flag |\version| as described in \secref{sec:flags}
compiles the main document as a draft:
%
\begin{center}
\begin{tabular}{l}
|\def\version{draft}|\\
|\input{childdoc.def}|\\
|\childdocforward{|\textit{main}|}|
\end{tabular}
\end{center}
%
Likewise, the following files |final|\textit{nn}|.tex|
compile the final version of the child document
|child|\textit{nn}|.tex|:
%
\begin{center}
\begin{tabular}{l}
|\def\version{final}|\\
|\input{childdoc.def}|\\
|\childdocforwardprefix{final}{child}|
\end{tabular}
\end{center}
%

Note that when several versions of a main file and/or of each child file
are to be generated, it may be convenient to set up a |Makefile| or
shell script to automatise the process.

%%%%%%%%%%%%%%%%%%%%%%%%%%%%%%%%%%%%%%%%%%%%%%%%%%%%%%%%%%%%%%%%%%%%%%%%%%%%%%%%
\subsection{Command Line Processing}
\label{sec:commandline}

The effect of redirection files can also be achieved by invoking
the \LaTeX{} compiler with a more elaborate command line.
Most conveniently this should be done as part
of a shell script or a |Makefile|.

When using \textsf{childdoc} in the main file, the following
command lines effectively perform a redirection
(note that depending on the shell being used,
backslashes may have to be doubled: `|\|' $\to$ `|\\|'):
%
\begin{center}
|... -jobname "|\textit{target}|" |\\|"|[\textit{flags}]%
|\input{childdoc.def}\childdocforward[|\textit{main}|]{|\textit{dest}|}"|
\end{center}
%
Here \textit{target} is the name of the output file,
\textit{main} is the name of the main file
and \textit{dest} is the name of the main or child file to be processed
(all filenames without extensions).
The optional argument \textit{main} can be omitted
if \textit{main} matches \textit{dest}.
Optionally, compilation \textit{flags} can be defined via |\def| commands.
This command line makes the \TeX{} engine believe
it is compiling the file \textit{target}
whose content is specified as the latter parameter.
The provided code then forwards the processing to
\textit{main} or \textit{dest} as described in \secref{sec:forward}.

%%%%%%%%%%%%%%%%%%%%%%%%%%%%%%%%%%%%%%%%%%%%%%%%%%%%%%%%%%%%%%%%%%%%%%%%%%%%%%%%
\subsection{Include by Input}
\label{sec:input}

Including child documents by |\include| has some restrictions by design.
Most notably, the content of a child document always occupies
its own set of pages; pages cannot be shared between child documents.
Usually, this behaviour makes perfect sense
because each child document contain an essential part of the document.
However, in some situations it may be desirable to compose
a document from a collection of parts
without having mandatory page breaks between then.
For this case, the package
provides a mechanism to include parts
by |\input| which can also be processed individually.
However, by construction this mechanism
requires manual handling of the content to be output.

%%%%%%%%%%%%%%%%%%%%%%%%%%%%%%%%%%%%%%%%
\DescribeMacro{\ifchilddocmanual}
The main file should be prepared as usual, see \secref{sec:include}.
However, the document body must make a distinction
between processing of an individual part and of the main document, e.g.:
%
\begin{center}
\begin{tabular}{l}
|\ifchilddocmanual|\\
|\input{\childdocname}|\\
|\||else|\\
\textit{document body with }|\input{|\textit{part}|}|\\
|\||fi|
\end{tabular}
\end{center}
%
The conditional |\ifchilddocmanual| is true whenever
a part to be included by |\input| is being compiled,
and the name of the part is stored in |\childdocname|.

%%%%%%%%%%%%%%%%%%%%%%%%%%%%%%%%%%%%%%%%
\DescribeMacro{\childdocby}
Each part to be included by |\input| should start with:
%
\begin{center}
\begin{tabular}{l}
|\input{childdoc.def}|\\
|\childdocby{|\textit{main}|}|\\
\end{tabular}
\end{center}
%
The directive |\childdocby| is similar to |\childdocof|
described in \secref{sec:include},
but the subsequent selection of content must be done manually.
To that end, both |\ifchilddoc| and |\ifchilddocmanual|
will be true upon processing of a part,
and the name of the part is stored in |\childdocname|.
Note that |\jobname| will be set to the filename of the current part
so that each part receives an individual |.aux| file
that does not interfere with the |.aux| file(s) of the main document.
This behaviour can be altered by the alternative form
|\childdocby[*]{|\textit{main}|}| (with a non-empty optional argument)
which uses the |.aux| file of the main document
by setting |\jobname| to \textit{main}.

%%%%%%%%%%%%%%%%%%%%%%%%%%%%%%%%%%%%%%%%%%%%%%%%%%%%%%%%%%%%%%%%%%%%%%%%%%%%%%%%
\subsection{Driver Development}
\label{sec:driver}

The \textsf{childdoc} mechanism can also be use for the development
of definition files such as \LaTeX{} styles or classes.
This case differs from the above setup with multiple parts
included by |\include| in that no |\includeonly| should be invoked.
This can be achieved by starting the include file
(before |\ProvidesPackage|) with:
%
\begin{center}
\begin{tabular}{l}
|\input{childdoc.def}|\\
|\childdocforward{|\textit{main}|}|\\
\end{tabular}
\end{center}
%
or alternatively with:
%
\begin{center}
\begin{tabular}{l}
|\input{childdoc.def}|\\
|\childdocby{|\textit{main}|}|\\
\end{tabular}
\end{center}
%
Both forms have slightly different effects as described above.
The main file is prepared as usual, see \secref{sec:include}.

%%%%%%%%%%%%%%%%%%%%%%%%%%%%%%%%%%%%%%%%%%%%%%%%%%%%%%%%%%%%%%%%%%%%%%%%%%%%%%%%
\subsection{Legacy Detection}
\label{sec:detection}

The directive |\childdocmain| in the main file can detect
whether the complete document or merely a child is to be compiled
even without using the directive |\childdocof|.
This method is deprecated because it is less robust
and there is no compelling reason to use it;
it is merely provided for backward compatibility
and it may be removed in future versions.

If the detection mechanism is to be used,
it is mandatory to correctly specify
the filename of the main file as the argument of |\childdocmain|:
%
\begin{center}
\begin{tabular}{l}
|\input{childdoc.def}|\\
|\childdocmain{|\textit{main}|}|\\
\end{tabular}
\end{center}
%
If |\jobname| does not match the argument \textit{main} of |\childdocmain|,
it is assumed that |\jobname| points to the child file to be compiled.
When using |\childdocmain| with the main file specified as argument,
it suffices to start a child file
with just |\input{|\textit{main}|}|
without loading of the package and using |\childdocof|.
If instead all processing is done
with the appropriate \textsf{childdoc} directives,
the argument of \textit{main} of |\childdocmain| can be empty.

An alternative version of the command line processing described
in \secref{sec:commandline} using the detection mechanism reads:
%
\begin{center}
|... -jobname "|\textit{target}|" "|[\textit{flags}]%
[|\def\jobname{|\textit{dest}|}|]|\input{|\textit{main}|}"|
\end{center}

%%%%%%%%%%%%%%%%%%%%%%%%%%%%%%%%%%%%%%%%%%%%%%%%%%%%%%%%%%%%%%%%%%%%%%%%%%%%%%%%
\subsection{Manual Code}
\label{sec:manual}

In case one cannot be certain whether the definitions file |childdoc.def|
is installed on the target \TeX{} distribution
and one prefers not to ship it,
it is conceivable to paste a few relevant commands into the sources.

To that end, drop all statements |\input{childdoc.def}|
and perform the replacements as outlined below.
Instead of |\childdocmain{|\textit{main}|}| add the following code
to the top of the main file:
%
\begin{center}
\begin{tabular}{l}
|\||ifdefined\childdocname\endinput\||fi\newif\ifchilddoc|\\
|\edef\childdocname{\scantokens\expandafter{\jobname\noexpand}}|\\
|\def\childdocmain{|\textit{main}|}\||ifx\childdocmain\childdocname\||else|\\
|\childdoctrue\includeonly{\childdocname}\let\jobname\childdocmain\||fi|\\
\end{tabular}
\end{center}
%
Instead of |\childdocof{|\textit{main}|}| just include the main file
at the top of each child file:
%
\begin{center}
|\input{|\textit{main}|}|
\end{center}
%
A simple redirection |\childdocforward{|\textit{dest}|}| is achieved by:
%
\begin{center}
|\def\jobname{|\textit{dest}|}\input{\jobname}|
\end{center}
%
The redirection with prefix
|\childdocforwardprefix[|\textit{prefix}|]{|\textit{dest}|}|
is accomplished by:
%
\begin{center}
\begin{tabular}{l}
|{\edef\jobname{\scantokens\expandafter{\jobname\noexpand}}|\\
|\def\redirectjob |\textit{prefix}|#1~~~{\gdef\jobname{|\textit{dest}|#1}}|\\
|\expandafter\redirectjob\jobname~~~}\input{\jobname}|
\end{tabular}
\end{center}

In an alternative approach,
child documents can be compiled by a specific command line
without additional code or specific definitions:
%
\begin{center}
|... -jobname "|\textit{target}|" "|[\textit{flags}]%
|\includeonly{|\textit{dest}|}\input{|\textit{main}|}"|
\end{center}
%

%%%%%%%%%%%%%%%%%%%%%%%%%%%%%%%%%%%%%%%%%%%%%%%%%%%%%%%%%%%%%%%%%%%%%%%%%%%%%%%%
%%%%%%%%%%%%%%%%%%%%%%%%%%%%%%%%%%%%%%%%%%%%%%%%%%%%%%%%%%%%%%%%%%%%%%%%%%%%%%%%
\section{Information}

%%%%%%%%%%%%%%%%%%%%%%%%%%%%%%%%%%%%%%%%%%%%%%%%%%%%%%%%%%%%%%%%%%%%%%%%%%%%%%%%
\subsection{Copyright}

Copyright \copyright{} 2017--2018 Niklas Beisert

This work may be distributed and/or modified under the
conditions of the \LaTeX{} Project Public License, either version 1.3
of this license or (at your option) any later version.
The latest version of this license is in
  \url{http://www.latex-project.org/lppl.txt}
and version 1.3 or later is part of all distributions of \LaTeX{}
version 2005/12/01 or later.

This work has the LPPL maintenance status `maintained'.

The Current Maintainer of this work is Niklas Beisert.

This work consists of the files |README.txt|, |childdoc.ins| and |childdoc.dtx|
as well as the derived files |childdoc.def|, |cdocsamp.tex|
with |cdocsch1.tex|, |cdocsch2.tex|, |cdocspt3.tex|, |cdocspt4.tex|,
|cdocsdrf.tex|, |cdocsfn1.tex|, |cdocsfn2.tex|
as well as |childdoc.pdf|.

%%%%%%%%%%%%%%%%%%%%%%%%%%%%%%%%%%%%%%%%%%%%%%%%%%%%%%%%%%%%%%%%%%%%%%%%%%%%%%%%
\subsection{Files and Installation}

The package consists of the files:
%
\begin{center}
\begin{tabular}{ll}
    |README.txt|   & readme file \\
    |childdoc.ins| & installation file \\
    |childdoc.dtx| & source file \\
    |childdoc.def| & definition file \\
    |cdocsamp.tex| & sample main file \\
    |cdocsch1.tex| & sample include file \\
    |cdocsch2.tex| & sample include file \\
    |cdocspt3.tex| & sample part file \\
    |cdocspt4.tex| & sample part file \\
    |cdocsdrf.tex| & sample redirection file \\
    |cdocsfn1.tex| & sample redirection file \\
    |cdocsfn2.tex| & sample redirection file \\
    |childdoc.pdf| & manual
\end{tabular}
\end{center}
%
The distribution consists of the files
|README.txt|, |childdoc.ins| and |childdoc.dtx|.
%
\begin{itemize}
\item
Run (pdf)\LaTeX{} on |childdoc.dtx|
to compile the manual |childdoc.pdf| (this file).
\item
Run \LaTeX{} on |childdoc.ins| to create the definitions file |childdoc.def|
and the sample |cdocsamp.tex| with include files
|cdocsch1.tex|, |cdocsch2.tex|, |cdocspt3.tex|, |cdocspt4.tex|,
|cdocsdrf.tex|, |cdocsfn1.tex|, |cdocsfn2.tex|.
Then copy the file |childdoc.def| to an appropriate directory of your \LaTeX{}
distribution, e.g.\ \textit{texmf-root}|/tex/latex/childdoc|.
\end{itemize}

%%%%%%%%%%%%%%%%%%%%%%%%%%%%%%%%%%%%%%%%%%%%%%%%%%%%%%%%%%%%%%%%%%%%%%%%%%%%%%%%
\subsection{Related CTAN Packages}

There are several other packages which offer a similar functionality:
%
\begin{itemize}
\item
The packages
\href{http://ctan.org/pkg/docmute}{\textsf{docmute}},
\href{http://ctan.org/pkg/includex}{\textsf{includex}} and
\href{http://ctan.org/pkg/standalone}{\textsf{standalone}}
provide commands to include only the document body of
a child file thus allowing both files to be compiled individually.
\item
The packages \href{http://ctan.org/pkg/subdocs}{\textsf{subdocs}}
and \href{http://ctan.org/pkg/subfiles}{\textsf{subfiles}}
provide structures in which the main and child documents can be
encapsulated and allowing them to be compiled individually.
The inclusion mechanism is different from the conventional |\include|.
\item
The package \href{http://ctan.org/pkg/combine}{\textsf{combine}}
is an elaborate solution to combine several documents into one.
\end{itemize}
%
See also the CTAN topic \href{http://ctan.org/topic/subdocs}{\textsf{subdocs}}
for further related packages.
The present package differs from the above solutions in that
a document structure constructed with the conventional |\include| mechanism
just needs two extra commands at the top of every file
such that all constituent files can be compiled individually.

%%%%%%%%%%%%%%%%%%%%%%%%%%%%%%%%%%%%%%%%%%%%%%%%%%%%%%%%%%%%%%%%%%%%%%%%%%%%%%%%
%\subsection{Feature Suggestions}
%
%The following is a list of features which may be useful for future
%versions of this package:
%%
%\begin{itemize}
%\item
%\ldots
%\end{itemize}

%%%%%%%%%%%%%%%%%%%%%%%%%%%%%%%%%%%%%%%%%%%%%%%%%%%%%%%%%%%%%%%%%%%%%%%%%%%%%%%%
\subsection{Revision History}

%%%%%%%%%%%%%%%%%%%%%%%%%%%%%%%%%%%%%%%%
\paragraph{v2.0:} 2018/12/30

\begin{itemize}
\item
immediate forward processing
\item
added |\childdocby| mechanism
\item
manual restructured
\end{itemize}

%%%%%%%%%%%%%%%%%%%%%%%%%%%%%%%%%%%%%%%%
\paragraph{v1.6:} 2018/01/17

\begin{itemize}
\item
application for development of include files
\item
corrections to manual
\end{itemize}

%%%%%%%%%%%%%%%%%%%%%%%%%%%%%%%%%%%%%%%%
\paragraph{v1.5:} 2017/05/21

\begin{itemize}
\item
more complete structuring introduced
\item
|\childdocof| introduced
\item
|\childdoc| renamed to |\childdocmain|
\item
|\childredirect| renamed to |\childdocforward| and |\childdocforwardprefix|
and functionality expanded
\end{itemize}

%%%%%%%%%%%%%%%%%%%%%%%%%%%%%%%%%%%%%%%%
\paragraph{v1.0:} 2017/04/27

\begin{itemize}
\item
manual and install package
\item
first version published on CTAN
\end{itemize}

%%%%%%%%%%%%%%%%%%%%%%%%%%%%%%%%%%%%%%%%
\paragraph{v0.6:} 2017/04/26

\begin{itemize}
\item
redirection mechanism added
\end{itemize}

%%%%%%%%%%%%%%%%%%%%%%%%%%%%%%%%%%%%%%%%
\paragraph{v0.5:} 2017/04/26

\begin{itemize}
\item
functionality in definition file
\end{itemize}


%%%%%%%%%%%%%%%%%%%%%%%%%%%%%%%%%%%%%%%%%%%%%%%%%%%%%%%%%%%%%%%%%%%%%%%%%%%%%%%%
%%%%%%%%%%%%%%%%%%%%%%%%%%%%%%%%%%%%%%%%%%%%%%%%%%%%%%%%%%%%%%%%%%%%%%%%%%%%%%%%
%%%%%%%%%%%%%%%%%%%%%%%%%%%%%%%%%%%%%%%%%%%%%%%%%%%%%%%%%%%%%%%%%%%%%%%%%%%%%%%%
\appendix

\settowidth\MacroIndent{\rmfamily\scriptsize 000\ }

 \DocInput{childdoc.dtx}

\end{document}
%</driver>
% \fi
%
% %%%%%%%%%%%%%%%%%%%%%%%%%%%%%%%%%%%%%%%%%%%%%%%%%%%%%%%%%%%%%%%%%%%%%%%%%%%%%%
% %%%%%%%%%%%%%%%%%%%%%%%%%%%%%%%%%%%%%%%%%%%%%%%%%%%%%%%%%%%%%%%%%%%%%%%%%%%%%%
% \section{Sample}
%\iffalse
%<*samplemain>
%\fi
%
% The following presents a sample document
% with two chapters, two parts, a title page,
% a compile flag as well as three forwarding files to set the flag.
% It consists of eight |.tex| files:
% \begin{center}
% \begin{tabular}{ll}
% |cdocsamp.tex|&main file\\
% |cdocsch1.tex|&include file for chapter 1\\
% |cdocsch2.tex|&include file for chapter 2\\
% |cdocspt3.tex|&include file for part 3\\
% |cdocspt4.tex|&include file for part 4\\
% |cdocsdrf.tex|&forwarding file for main file in draft mode\\
% |cdocsfi1.tex|&forwarding file for final version of chapter 1\\
% |cdocsfi2.tex|&forwarding file for final version of chapter 2\\
% \end{tabular}
% \end{center}
% Each of the eight files can be compiled directly by the \LaTeX{} compiler.
%
% %%%%%%%%%%%%%%%%%%%%%%%%%%%%%%%%%%%%%%
% \paragraph{Main File.}
%
% The main file is called |cdocsamp.tex|.
%
% Load the \textsf{childdoc} definitions and
% declare the filename for the main document:
%    \begin{macrocode}
\input{childdoc.def}
\childdocmain{}
%    \end{macrocode}

% Optional override for |\version| flag:
%    \begin{macrocode}
%%\ifchilddoc\else\providecommand{\version}{draft}\fi
%    \end{macrocode}

% Define the default values for the |\version| flag
% (|final| for the main file and |draft| for childs):
%    \begin{macrocode}
\ifchilddoc
\providecommand{\version}{draft}
\else
\providecommand{\version}{final}
\fi
%    \end{macrocode}

% Load the standard document class:
%    \begin{macrocode}
\documentclass[12pt]{article}
%    \end{macrocode}

% Start the document body:
%    \begin{macrocode}
\begin{document}
%    \end{macrocode}

% Declare a title page.
% Print title, part of document being processed and version flag:
%    \begin{macrocode}
\addtocounter{page}{-1}
\begin{center}
{\LARGE\bfseries{}childdoc example\par}
\vspace{1cm}
\ifchilddoc
\ifchilddocmanual part\else chapter\fi:
`\childdocname' of `\childdocjob'\par
\else
main document: `\childdocjob'\par
\fi
version: \version\par
\end{center}
\newpage
%    \end{macrocode}

% Manually include selected file,
% otherwise process as usual:
%    \begin{macrocode}
\ifchilddocmanual
\section*{part `\childdocname'}
\input{\childdocname}
\else
%    \end{macrocode}

% Include the two chapters:
%    \begin{macrocode}
\include{cdocsch1}
\include{cdocsch2}
%    \end{macrocode}

% Include the two parts unless only chapters should be displayed:
%    \begin{macrocode}
\ifchilddoc\else
\section{part three}
\input{cdocspt3}
\section{part four}
\input{cdocspt4}
\fi
%    \end{macrocode}

% Process as usual until here:
%    \begin{macrocode}
\fi
%    \end{macrocode}

% End of document body:
%    \begin{macrocode}
\end{document}
%    \end{macrocode}
%\iffalse
%</samplemain>
%\fi
%
% %%%%%%%%%%%%%%%%%%%%%%%%%%%%%%%%%%%%%%
% \paragraph{Chapter Include Files.}
%
% The include files are called |cdocsch1.tex| and |cdocsch2.tex|.
%
%\iffalse
%<*samplechap1|samplechap2>
%\fi

% Optional override for |\version| flag:
%    \begin{macrocode}
%%\providecommand{\version}{final}
%    \end{macrocode}

% Include the main document:
%    \begin{macrocode}
\input{childdoc.def}
\childdocof{cdocsamp}
%    \end{macrocode}

%\iffalse
%</samplechap1|samplechap2>
%\fi
%
%\iffalse
%<*samplechap1>
%\fi
% Some text for chapter 1:
%    \begin{macrocode}
\section{one}
some text in chapter one
%    \end{macrocode}

%\iffalse
%</samplechap1>
%\fi
% Some text for chapter 2:
%\iffalse
%<*samplechap2>
%\fi
%    \begin{macrocode}
\section{two}
more text in chapter two
%    \end{macrocode}

%\iffalse
%</samplechap2>
%\fi
%
% %%%%%%%%%%%%%%%%%%%%%%%%%%%%%%%%%%%%%%
% \paragraph{Part Include Files.}
%
% The include files are called |cdocspt3.tex| and |cdocspt4.tex|.
%
%\iffalse
%<*samplepart3|samplepart4>
%\fi

% Optional override for |\version| flag:
%    \begin{macrocode}
%%\providecommand{\version}{final}
%    \end{macrocode}

% Include the main document:
%    \begin{macrocode}
\input{childdoc.def}
\childdocby{cdocsamp}
%    \end{macrocode}

%\iffalse
%</samplepart3|samplepart4>
%\fi
%
%\iffalse
%<*samplepart3>
%\fi
% Some text for part 3:
%    \begin{macrocode}
some text in part three
%    \end{macrocode}

%\iffalse
%</samplepart3>
%\fi
% Some text for part 4:
%\iffalse
%<*samplepart4>
%\fi
%    \begin{macrocode}
more text in part four
%    \end{macrocode}

%\iffalse
%</samplepart4>
%\fi
%
% %%%%%%%%%%%%%%%%%%%%%%%%%%%%%%%%%%%%%%
% \paragraph{Forwarding for a Complete Draft.}
%
% The following forwarding file |cdocsdrf.tex|
% compiles the main document in draft mode:
%\iffalse
%<*sampledraft>
%\fi
%    \begin{macrocode}
\def\version{draft}
\input{childdoc.def}
\childdocforward{cdocsamp}
%    \end{macrocode}

%\iffalse
%</sampledraft>
%\fi
%
% %%%%%%%%%%%%%%%%%%%%%%%%%%%%%%%%%%%%%%
% \paragraph{Forwarding for Final Version of the Chapters.}
%
% The following forwarding files |cdocsfn1.tex| and |cdocsfn2.tex|
% (with identical content)
% compile the final versions of the child documents
% |cdocsch1.tex| and |cdocsch2.tex|, respectively:
%\iffalse
%<*samplefinal>
%\fi
%    \begin{macrocode}
\def\version{final}
\input{childdoc.def}
\childdocforwardprefix[cdocsamp]{cdocsfn}{cdocsch}
%    \end{macrocode}

%\iffalse
%</samplefinal>
%\fi
%
% %%%%%%%%%%%%%%%%%%%%%%%%%%%%%%%%%%%%%%
% \paragraph{Command Line Processing.}
%
% The following three command lines generate the output files
% |cdocscld|, |cdocscl1| and |cdocscl2|
% which should be identical to
% |cdocsdrf|, |cdocsch1| and |cdocsfn2|, respectively:
% \begin{center}
% \begin{tabular}{l}
% |latex -jobname cdocscld \|\\
% |  "\def\version{draft}\input{childdoc.def}\childdocforward{cdocsamp}"|\\
% |latex -jobname cdocscl1 \|\\
% |  "\input{childdoc.def}\childdocforward[cdocsamp]{cdocsch1}"|\\
% |latex -jobname cdocscl2 \|\\
% |  "\def\version{final}\input{childdoc.def}\childdocforward{cdocsch2}"|
% \end{tabular}
% \end{center}
% Note that the trailing backslash on each first line
% merely continues the input to the second line
% (for convenient cut ant paste).
% Furthermore, the command |latex| can be replaced by any
% of its alternative versions such as |pdflatex|.
%
% %%%%%%%%%%%%%%%%%%%%%%%%%%%%%%%%%%%%%%%%%%%%%%%%%%%%%%%%%%%%%%%%%%%%%%%%%%%%%%
% %%%%%%%%%%%%%%%%%%%%%%%%%%%%%%%%%%%%%%%%%%%%%%%%%%%%%%%%%%%%%%%%%%%%%%%%%%%%%%
% \section{Implementation}
%\iffalse
%<*package>
%\fi
%
% This section describes the definitions file |childdoc.def|.

% The definitions cannot be loaded using |\usepackage| or |\RequirePackage|
% which has a mechanism to prevent loading a style file more than once.
% When loading the definitions by means of |\input|
% multiple instances have to be prevented manually:
%\iffalse
%This code needs to be before the `\ProvidesFile' directive
%which is defined at the beginning of this file.
%Therefore it is also placed there and commented out here.
%</package>
%<*discard>
%\fi
%    \begin{macrocode}
\ifdefined\childdocmain\endinput\fi
%    \end{macrocode}
%\iffalse
%</discard>
%<*package>
%\fi
%
% \macro{\ifchilddoc}
% \macro{\ifchilddocmanual}
% The conditional |\ifchilddoc| tells whether a
% child (true) or main (false) document is being compiled.
% The conditional |\ifchilddocmanual| tells whether
% the |\includeonly| mechanism is used (false) or
% the selection of child files must be performed manually (true).
% The definitions initialise to false:
%    \begin{macrocode}
\newif\ifchilddoc
\newif\ifchilddocmanual
%    \end{macrocode}

% \macro{\childdocname}
% \macro{\childdocjob}
% The macro |\childdocname| stores the name of the main document
% to be compiled. The macro |\childdocjob| stores the name of
% the document on which the \LaTeX{} compiler was originally invoked.
% The content of |\jobname| cannot be compared
% to filenames specified in the source due to different catcodes.
% The following code rescans |\jobname|, stores the result
% in |\childdocname| and saves a copy in |\childdocjob|:
%    \begin{macrocode}
\edef\childdocname{\scantokens\expandafter{\jobname\noexpand}}
\let\childdocjob\childdocname
%    \end{macrocode}

% \macro{\childdocdisable}
% The macro |\childdocdisable| prevents the main file
% from being processed more than once.
% At this stage, the main document command |\childdocmain|
% is assumed to be called once again where it should do nothing.
% Any subsequent call to it should prevent
% a secondary processing of the main document
% It overwrites the forwarding commands
% |\childdocof| and |\childdocforward|
% with empty macros to prevent further inclusions of the main document:
%    \begin{macrocode}
\newcommand{\childdocdisable}
{
  \renewcommand{\childdocmain}[1]{\renewcommand{\childdocmain}[1]{\endinput}}
  \renewcommand{\childdocof}[1]{}
  \renewcommand{\childdocby}[2][]{}
  \renewcommand{\childdocforward}[2][]{}
  \renewcommand{\childdocdisable}{}
}
%    \end{macrocode}

% \macro{\childdocmain}
% The macro |\childdocmain| is to be called at the top of the main file
% with nothing or the main filename (without extension) as argument.
% First, it breaks loops.
% If the argument is not empty and does not match |\childdocname|
% (which is set by the first inclusion of |childdoc.def|),
% |\ifchilddoc| is set to true, |\includeonly| is applied to the child file
% and |\jobname| is set to the main file
% (for proper handling of |.aux| files):
%    \begin{macrocode}
\newcommand{\childdocmain}[1]
{
  \childdocdisable\childdocmain{}
  \if?#1?\else
    \begingroup
      \def\childdoctmp{#1}
      \ifx\childdoctmp\childdocname
        \def\childdoctmp{}
      \else
        \def\childdoctmp
        {
          \childdoctrue
          \includeonly{\childdocname}
          \def\childdocjob{#1}
          \def\jobname{#1}
        }
      \fi
      \expandafter
    \endgroup
    \childdoctmp
  \fi
}
%    \end{macrocode}

% \macro{\childdocof}
% The command |\childdocof| redirects
% compilation to the main file |#1|.
%    \begin{macrocode}
\newcommand{\childdocof}[1]
{
  \childdocdisable
  \childdoctrue
  \includeonly{\childdocname}
  \def\jobname{#1}
  \def\childdocjob{#1}
  \input{#1}
}
%    \end{macrocode}

% \macro{\childdocby}
% The command |\childdocby| ....
%    \begin{macrocode}
\newcommand{\childdocby}[2][]
{
  \childdocdisable
  \childdoctrue
  \childdocmanualtrue
  \if?#1?\else
    \def\jobname{#2}
  \fi
  \def\childdocjob{#2}
  \input{#2}
  \endinput
}
%    \end{macrocode}

% \macro{\childdocforward}
% The command |\childdocforward| redirects
% compilation to the main file or
% (if the optional argument is given) a child file.
% Parameters are set as if the main file
% or a child file starting with |\childdocof| was compiled.
% Then compilation is handed over to the main file:
%    \begin{macrocode}
\newcommand{\childdocforward}[2][]
{
  \begingroup
    \if?#1?
      \def\childdoctmp
      {
        \def\childdocname{#2}
        \def\childdocjob{#2}
        \def\jobname{#2}
        \input{#2}
        \endinput
      }
    \else
      \def\childdoctmp
      {
        \childdocdisable
        \def\childdocname{#2}
        \childdoctrue
        \includeonly{#2}
        \def\childdocjob{#1}
        \def\jobname{#1}
        \input{#1}
        \endinput
      }
    \fi
    \expandafter
  \endgroup
  \childdoctmp
}
%    \end{macrocode}

% \macro{\childdocforwardprefix}
% The command |\childdocforwardprefix| redirects
% compilation to the main or a child file by means of a pattern.
% The prefix |#1| in the current filename is replaced by |#2|
% and the suffix of the current filename is kept
% (it is assumed that the filename does not contain the substring `|~~~|'
% which is used as a delimiter).
% Compilation is handed over to the new file by |\childdocforward|:
%    \begin{macrocode}
\newcommand{\childdocforwardprefix}[3][]
{
  \begingroup
    \def\childdocextract #2##1~~~{\def\childdoctmp{\childdocforward[#1]{#3##1}}}
    \expandafter\childdocextract\childdocname~~~
    \expandafter
  \endgroup
  \childdoctmp
}
%    \end{macrocode}

% \macro{\childdoc}
% The deprecated macro |\childdoc| is a legacy version of |\childdocmain|:
%    \begin{macrocode}
\newcommand{\childdoc}{\childdocmain}
%    \end{macrocode}

% \macro{\childdocredirect}
% The deprecated macro |\childdocredirect| is a legacy version
% of |\childdocforward| and |\childdocforwardprefix|:
%    \begin{macrocode}
\newcommand{\childdocredirect}[2][]
{
  \begingroup
    \if?#1?
      \def\childdoctmp{\childdocforward{#2}}
    \else
      \def\childdoctmp{\childdocforwardprefix{#1}{#2}}
    \fi
    \expandafter
  \endgroup
  \childdoctmp
}
%    \end{macrocode}

%\iffalse
%</package>
%\fi
%
\endinput
\childdocforward{cdocsamp}"|\\
% |latex -jobname cdocscl1 \|\\
% |  "% \iffalse
%
% childdoc.dtx Copyright (C) 2017-2018 Niklas Beisert
%
% This work may be distributed and/or modified under the
% conditions of the LaTeX Project Public License, either version 1.3
% of this license or (at your option) any later version.
% The latest version of this license is in
%   http://www.latex-project.org/lppl.txt
% and version 1.3 or later is part of all distributions of LaTeX
% version 2005/12/01 or later.
%
% This work has the LPPL maintenance status `maintained'.
%
% The Current Maintainer of this work is Niklas Beisert.
%
% This work consists of the files childdoc.dtx and childdoc.ins
% and the derived files childdoc.def and cdocsamp.tex with
% cdocsch1.tex, cdocsch2.tex, cdocsdrf.tex, cdocsfn1.tex, cdocsfn2.tex.
%
%<package>\ifdefined\childdocmain\endinput\fi
%<package>\ProvidesFile{childdoc.def}[2018/12/30 v2.0 child document driver]
%<samplemain>\ProvidesFile{cdocsamp.tex}[2018/12/30 v2.0 sample for childdoc]
%<*driver>
%\ProvidesFile{childdoc.drv}[2018/12/30 v2.0 childdoc reference manual file]
\PassOptionsToClass{10pt,a4paper}{article}
\documentclass{ltxdoc}

\usepackage[margin=35mm]{geometry}
\usepackage{hyperref}
\usepackage{hyperxmp}
\usepackage[usenames]{color}

\hypersetup{colorlinks=true}
\hypersetup{pdfstartview=FitH}
\hypersetup{pdfpagemode=UseNone}
\hypersetup{pdfsource={}}
\hypersetup{pdflang={en-UK}}
\hypersetup{pdfcopyright={Copyright 2017-2018 Niklas Beisert.
  This work may be distributed and/or modified under the
  conditions of the LaTeX Project Public License, either version 1.3
  of this license or (at your option) any later version.}}
\hypersetup{pdflicenseurl={http://www.latex-project.org/lppl.txt}}
\hypersetup{pdfcontactaddress={ETH Zurich, ITP, HIT K,
  Wolfgang-Pauli-Strasse 27}}
\hypersetup{pdfcontactpostcode={8093}}
\hypersetup{pdfcontactcity={Zurich}}
\hypersetup{pdfcontactcountry={Switzerland}}
\hypersetup{pdfcontactemail={nbeisert@itp.phys.ethz.ch}}
\hypersetup{pdfcontacturl={http://people.phys.ethz.ch/\xmptilde nbeisert/}}

\newcommand{\secref}[1]{\hyperref[#1]{section \ref*{#1}}}

\parskip1ex
\parindent0pt
\let\olditemize\itemize
\def\itemize{\olditemize\parskip0pt}

\begin{document}

\title{The \textsf{childdoc} Package}
\hypersetup{pdftitle={The childdoc Package}}
\author{Niklas Beisert\\[2ex]
  Institut f\"ur Theoretische Physik\\
  Eidgen\"ossische Technische Hochschule Z\"urich\\
  Wolfgang-Pauli-Strasse 27, 8093 Z\"urich, Switzerland\\[1ex]
  \href{mailto:nbeisert@itp.phys.ethz.ch}
  {\texttt{nbeisert@itp.phys.ethz.ch}}}
\hypersetup{pdfauthor={Niklas Beisert}}
\hypersetup{pdfsubject={Manual for the LaTeX2e Package childdoc}}
\date{30 December 2018, \textsf{v2.0}}
\maketitle

\begin{abstract}\noindent
\textsf{childdoc} is a \LaTeXe{} package
that enables the direct compilation
of document sections included by |\include|
to individual files.
\end{abstract}

\begingroup
\parskip0ex
\tableofcontents
\endgroup

%%%%%%%%%%%%%%%%%%%%%%%%%%%%%%%%%%%%%%%%%%%%%%%%%%%%%%%%%%%%%%%%%%%%%%%%%%%%%%%%
%%%%%%%%%%%%%%%%%%%%%%%%%%%%%%%%%%%%%%%%%%%%%%%%%%%%%%%%%%%%%%%%%%%%%%%%%%%%%%%%
\section{Introduction}

\LaTeX{} provides a mechanism to structure a large document (such as a book)
into a main file and several child files (containing the chapters)
using the |\include| command.
This mechanism is beneficial for documents
which span hundreds of pages in order to
make the source file(s) more manageable.
Moreover, compilation can be restricted to
selected child files by means of the |\includeonly| command.
The latter feature can be used to reduce the compilation time while editing
(this was significantly more useful in the earlier days of \LaTeX{})
or to generate a smaller document which is easier to navigate.
Another application of |\includeonly| is to generate
documents consisting of selected parts of the complete document.

However, there are a few drawbacks of the plain |\include| mechanism:
\begin{itemize}
\item
The child files cannot be compiled on their own,
they can only be compiled via the main file.
A naive editing environment
(such as a text editor with an option
to have the current file processed by \LaTeX)
may require one to switch to the main file before compiling;
attempting to compile the child file produces errors.
\item
The main file must be modified (each time)
to adjust the |\includeonly| command
to the present needs. This easily leaves the main file in a messy state.
\item
The generated document will always carry the filename
of the main document. This is inconvenient if
several child files are to be compiled and
to be kept for distribution.
\end{itemize}

The present package provides a simple interface
to make child files individually compilable by \LaTeX{}.
Compiling a child file then has the same effect as compiling
the main file with an |\includeonly| command
to select the appropriate child.
Moreover the generated document will carry the name of the child
rather than the main file.
This resolves all three above issues.

This feature is meant to make the editing of books,
thesis documents and lecture notes somewhat more convenient.
However, the package can also be used efficiently for
composing a series of documents (such as exercise sheets)
which are typically distributed individually.
It then assists the author in generating the individual documents
(potentially in different versions)
as well as a document containing the collected series.
Another application is in developing style files
or other kinds of included material
where compilation of the style file could redirect
to a sample or test file.

%%%%%%%%%%%%%%%%%%%%%%%%%%%%%%%%%%%%%%%%%%%%%%%%%%%%%%%%%%%%%%%%%%%%%%%%%%%%%%%%
%%%%%%%%%%%%%%%%%%%%%%%%%%%%%%%%%%%%%%%%%%%%%%%%%%%%%%%%%%%%%%%%%%%%%%%%%%%%%%%%
\section{Usage}

First of all, the package \textsf{childdoc} is \emph{not} a standard
\LaTeXe{} |.sty| style file! Therefore it needs to be invoked in
a non-standard way.

%%%%%%%%%%%%%%%%%%%%%%%%%%%%%%%%%%%%%%%%%%%%%%%%%%%%%%%%%%%%%%%%%%%%%%%%%%%%%%%%
\subsection{Included Files}
\label{sec:include}

%%%%%%%%%%%%%%%%%%%%%%%%%%%%%%%%%%%%%%%%
\DescribeMacro{\childdocmain}
To use the package, add the commands
\begin{center}
\begin{tabular}{l}
|\input{childdoc.def}|\\
|\childdocmain{}|\\
\end{tabular}
\end{center}
at the very top of the main \LaTeX{} file,
in particular \emph{before} the |\documentclass| statement!
The argument of |\childdocmain| should be left empty
(but it must be present).

%%%%%%%%%%%%%%%%%%%%%%%%%%%%%%%%%%%%%%%%
\DescribeMacro{\childdocof}
Furthermore, add the commands
\begin{center}
\begin{tabular}{l}
|\input{childdoc.def}|\\
|\childdocof{|\textit{main}|}|\\
\end{tabular}
\end{center}
at the top of every child file \textit{child}
which is included by |\include{|\textit{child}|}|
from within the main file
(or at least for those files to be compiled individually).
The argument \textit{main} must be the filename of the main file.

There are a couple of
considerations in setting up the main and child documents:

%%%%%%%%%%%%%%%%%%%%%%%%%%%%%%%%%%%%%%%%
\paragraph{Restrictions.}

Please note the following restrictions:
\begin{itemize}
\item
|\childdocmain| must be called with one argument \textit{main}
to ensure compatibility with earlier version of the package.
It must either be empty (|\childdocmain{}|)
or precisely match the filename of the main file in which it is specified.
See \secref{sec:detection} for further information.
\item
The filename \textit{main} must be specified without the |.tex| extension.
\item
The filename \textit{main} is case sensitive
(even in case-insensitive file systems)
due to internal string comparison.
\item
The argument \textit{main} should be fully expanded, it cannot be a macro.
\item
Subdirectories and special characters should be avoided in filenames.
\item
The command |\childdocmain{|\textit{main}|}| must be followed by a whitespace.
It should not be followed immediately by another command
or by a comment mark `|%|'.
This is because the \TeX{} parser reads the token immediately following
the argument of |\childdocmain| and puts it
at the beginning of every child section;
however, a white\-space is ignored.
\end{itemize}

%%%%%%%%%%%%%%%%%%%%%%%%%%%%%%%%%%%%%%%%
\paragraph{Content of Main File.}

It is advisable to place all content in the child files included by |\include|.
Any output contained in the main file will appear in all child documents
unless suppressed manually;
it cannot be suppressed automatically by the |\includeonly| directive
and thus should normally be avoided.
A method to include some content in the main file
by means of conditional processing is described in \secref{sec:conditional}.

%%%%%%%%%%%%%%%%%%%%%%%%%%%%%%%%%%%%%%%%
\paragraph{Page Numbering.}

When only a part of the document is compiled,
the appropriate numbering of pages
(as well as other status parameters)
is determined from the |.aux| files.
The latter contain information from previous passes.
However this information needs to propagate through
all intermediate child documents.
Therefore the page numbering in child documents may well
be inconsistent until the complete document is compiled at least once.

A useful (if unconventional) way to always ensure a consistent
page numbering is to restart the numbering in each child document
and denote the pages by `\textit{child}|.|\textit{page}'
where \textit{child} represents the chapter/section number of the child file.
This can be achieved by the command
|\numberwithin{page}{|\textit{child}|}|
of the \textsf{amsmath} package
where \textit{child} can be |chapter| or |section|
depending on the chosen structuring.
Alternatively, one can modify the macro |\thepage| appropriately
and reset the counter |page| at the start of each child file.

%%%%%%%%%%%%%%%%%%%%%%%%%%%%%%%%%%%%%%%%%%%%%%%%%%%%%%%%%%%%%%%%%%%%%%%%%%%%%%%%
\subsection{Conditional Processing}
\label{sec:conditional}

The package provides a mechanism to compile different versions
of a document. To customise the versions further some conditional processing
can come in handy to distinguish which version is being compiled.
The package provides two macros to describe the compilation context:

%%%%%%%%%%%%%%%%%%%%%%%%%%%%%%%%%%%%%%%%
\DescribeMacro{\ifchilddoc}
The conditional |\ifchilddoc| distinguishes between the compilation of
child documents and the main document:
%
\begin{center}
|\ifchilddoc |\textit{child-code}| |[|\||else |\textit{main-code}]| \||fi|
\end{center}

%%%%%%%%%%%%%%%%%%%%%%%%%%%%%%%%%%%%%%%%
\DescribeMacro{\childdocname}
\DescribeMacro{\childdocjob}
The macro |\childdocname| contains the filename (without extension)
of the main or child file being processed.
Note that |\childdocjob| will always contain the name of the main file.

%%%%%%%%%%%%%%%%%%%%%%%%%%%%%%%%%%%%%%%%
\paragraph{Title Page.}

Conditional processing can be used to include a title or banner page
in the main document when proper precautions are taken.
Importantly, the code in the main file should ensure that the page counter
(as well as other status parameters which are stored in the |.aux| files)
takes the same value after the conditional processing.
Otherwise the page numbers may take divergent values
depending on which part is compiled.

For example, a title page could be declared by:
%
\begin{center}
\begin{tabular}{l}
|\ifchilddoc\||else|\\
|\addtocounter{page}{-1}|\\
\textit{code for title page}\\
|\newpage|\\
|\||fi|
\end{tabular}
\end{center}
%
A banner page for the child documents can be generated by:
%
\begin{center}
\begin{tabular}{l}
|\ifchilddoc|\\
|\addtocounter{page}{-1}|\\
\textit{code for banner page}\\
|\newpage|\\
|\||fi|
\end{tabular}
\end{center}
%
Here one could write a message such as:
\begin{center}
|This is the part \childdocname{} of \childdocjob{}.|
\end{center}

%%%%%%%%%%%%%%%%%%%%%%%%%%%%%%%%%%%%%%%%%%%%%%%%%%%%%%%%%%%%%%%%%%%%%%%%%%%%%%%%
\subsection{Flags}
\label{sec:flags}

The package makes it easy to generate different versions
of the main or child documents.
To this end compilation flags can be defined
and assigned different default values.
They will be particularly useful in conjunction
with the forwarding mechanism described in \secref{sec:forward}.

For example, it may be useful to have a flag |\version|
which can be set to |draft| or |final|.
The document source will contain some conditional code
depending on the value of |\version|.
Suppose further, the flag should default to |final| for the main file
and to |draft| for child files
which is a natural assignment for editing the document.
This is achieved by placing the following code
in the preamble of the main document
(below the |\childdocmain| directive):
%
\begin{center}
\begin{tabular}{l}
|\ifchilddoc|\\
|\providecommand{\version}{draft}|\\
|\||else|\\
|\providecommand{\version}{final}|\\
|\||fi|
\end{tabular}
\end{center}
%
The definition by |\providecommand| makes sure
that previous definitions are not overwritten.
Further statements |\providecommand{\version}{...}|
can thus be added before the above code to override it.

For the main file, one might add a line
(between |\childdocmain| and the above block)
%
\begin{center}
|%\ifchilddoc\||else\providecommand{\version}{draft}\||fi|
\end{center}
%
which can be uncommented to produce a draft version.
Likewise one can add a line to the very top of a child file
(above the |\childdocof{|\textit{main}|}| directive)
%
\begin{center}
|%\providecommand{\version}{final}|
\end{center}
%
which can be uncommented to produce the final version of this child document.

%%%%%%%%%%%%%%%%%%%%%%%%%%%%%%%%%%%%%%%%%%%%%%%%%%%%%%%%%%%%%%%%%%%%%%%%%%%%%%%%
\subsection{Forwarding}
\label{sec:forward}

Different versions of the main or child documents
using compilation flags as described in \secref{sec:flags}
can be (permanently) stored in different files
for convenient compilation, viewing and distribution.
To this end, the package defines a command
to pass on compilation to a different file:

%%%%%%%%%%%%%%%%%%%%%%%%%%%%%%%%%%%%%%%%
\DescribeMacro{\childdocforward}
The command |\childdocforward| redirects processing to
another source file:
%
\begin{center}
\begin{tabular}{l}
|\input{childdoc.def}|\\
|\childdocforward[|\textit{main}|]{|\textit{dest}|}|\\
\end{tabular}
\end{center}
%
The argument \textit{dest} is the destination file
(without extension).
It should be the main file or one of the child files.
Note that further \textsf{childdoc} directives
such as |\childdocof| and |\childdocforward|
in the indicated file will be processed in this form.
The optional argument \textit{main}
passes on directly to the main file \textit{main}
while pretending to compile the child \textit{dest}.
This form behaves as if \textit{dest}
issues |\childdocof{|\textit{main}|}| right away,
and no further \textsf{childdoc} directives will be processed.

%%%%%%%%%%%%%%%%%%%%%%%%%%%%%%%%%%%%%%%%
\DescribeMacro{\...prefix}
In the alternative form |\childdocforwardprefix|,
%
\begin{center}
\begin{tabular}{l}
|\input{childdoc.def}|\\
|\childdocforwardprefix[|\textit{main}|]{|\textit{prefix}|}{|\textit{dest}|}|
\end{tabular}
\end{center}
%
the destination file is determined by a pattern
depending on the current file:
To make this work, the current file must be called
`{\textit{prefix}\hspace{0.2em}\textit{suffix}}'
with \textit{prefix} matching precisely the argument.
Processing is then passed on to the file
`{\textit{dest}\hspace{0.2em}\textit{suffix}}'.
Surely, the same effect is achieved by
directly specifying the
argument `{\textit{dest}\hspace{0.2em}\textit{suffix}}'
in the first form.
However, that requires to set up a different file
for each child. With the alternative form of the command
all these files can have exactly the same content
which simplifies setting them up and maintaining them.

For example, the following file |draft.tex|
with a compilation flag |\version| as described in \secref{sec:flags}
compiles the main document as a draft:
%
\begin{center}
\begin{tabular}{l}
|\def\version{draft}|\\
|\input{childdoc.def}|\\
|\childdocforward{|\textit{main}|}|
\end{tabular}
\end{center}
%
Likewise, the following files |final|\textit{nn}|.tex|
compile the final version of the child document
|child|\textit{nn}|.tex|:
%
\begin{center}
\begin{tabular}{l}
|\def\version{final}|\\
|\input{childdoc.def}|\\
|\childdocforwardprefix{final}{child}|
\end{tabular}
\end{center}
%

Note that when several versions of a main file and/or of each child file
are to be generated, it may be convenient to set up a |Makefile| or
shell script to automatise the process.

%%%%%%%%%%%%%%%%%%%%%%%%%%%%%%%%%%%%%%%%%%%%%%%%%%%%%%%%%%%%%%%%%%%%%%%%%%%%%%%%
\subsection{Command Line Processing}
\label{sec:commandline}

The effect of redirection files can also be achieved by invoking
the \LaTeX{} compiler with a more elaborate command line.
Most conveniently this should be done as part
of a shell script or a |Makefile|.

When using \textsf{childdoc} in the main file, the following
command lines effectively perform a redirection
(note that depending on the shell being used,
backslashes may have to be doubled: `|\|' $\to$ `|\\|'):
%
\begin{center}
|... -jobname "|\textit{target}|" |\\|"|[\textit{flags}]%
|\input{childdoc.def}\childdocforward[|\textit{main}|]{|\textit{dest}|}"|
\end{center}
%
Here \textit{target} is the name of the output file,
\textit{main} is the name of the main file
and \textit{dest} is the name of the main or child file to be processed
(all filenames without extensions).
The optional argument \textit{main} can be omitted
if \textit{main} matches \textit{dest}.
Optionally, compilation \textit{flags} can be defined via |\def| commands.
This command line makes the \TeX{} engine believe
it is compiling the file \textit{target}
whose content is specified as the latter parameter.
The provided code then forwards the processing to
\textit{main} or \textit{dest} as described in \secref{sec:forward}.

%%%%%%%%%%%%%%%%%%%%%%%%%%%%%%%%%%%%%%%%%%%%%%%%%%%%%%%%%%%%%%%%%%%%%%%%%%%%%%%%
\subsection{Include by Input}
\label{sec:input}

Including child documents by |\include| has some restrictions by design.
Most notably, the content of a child document always occupies
its own set of pages; pages cannot be shared between child documents.
Usually, this behaviour makes perfect sense
because each child document contain an essential part of the document.
However, in some situations it may be desirable to compose
a document from a collection of parts
without having mandatory page breaks between then.
For this case, the package
provides a mechanism to include parts
by |\input| which can also be processed individually.
However, by construction this mechanism
requires manual handling of the content to be output.

%%%%%%%%%%%%%%%%%%%%%%%%%%%%%%%%%%%%%%%%
\DescribeMacro{\ifchilddocmanual}
The main file should be prepared as usual, see \secref{sec:include}.
However, the document body must make a distinction
between processing of an individual part and of the main document, e.g.:
%
\begin{center}
\begin{tabular}{l}
|\ifchilddocmanual|\\
|\input{\childdocname}|\\
|\||else|\\
\textit{document body with }|\input{|\textit{part}|}|\\
|\||fi|
\end{tabular}
\end{center}
%
The conditional |\ifchilddocmanual| is true whenever
a part to be included by |\input| is being compiled,
and the name of the part is stored in |\childdocname|.

%%%%%%%%%%%%%%%%%%%%%%%%%%%%%%%%%%%%%%%%
\DescribeMacro{\childdocby}
Each part to be included by |\input| should start with:
%
\begin{center}
\begin{tabular}{l}
|\input{childdoc.def}|\\
|\childdocby{|\textit{main}|}|\\
\end{tabular}
\end{center}
%
The directive |\childdocby| is similar to |\childdocof|
described in \secref{sec:include},
but the subsequent selection of content must be done manually.
To that end, both |\ifchilddoc| and |\ifchilddocmanual|
will be true upon processing of a part,
and the name of the part is stored in |\childdocname|.
Note that |\jobname| will be set to the filename of the current part
so that each part receives an individual |.aux| file
that does not interfere with the |.aux| file(s) of the main document.
This behaviour can be altered by the alternative form
|\childdocby[*]{|\textit{main}|}| (with a non-empty optional argument)
which uses the |.aux| file of the main document
by setting |\jobname| to \textit{main}.

%%%%%%%%%%%%%%%%%%%%%%%%%%%%%%%%%%%%%%%%%%%%%%%%%%%%%%%%%%%%%%%%%%%%%%%%%%%%%%%%
\subsection{Driver Development}
\label{sec:driver}

The \textsf{childdoc} mechanism can also be use for the development
of definition files such as \LaTeX{} styles or classes.
This case differs from the above setup with multiple parts
included by |\include| in that no |\includeonly| should be invoked.
This can be achieved by starting the include file
(before |\ProvidesPackage|) with:
%
\begin{center}
\begin{tabular}{l}
|\input{childdoc.def}|\\
|\childdocforward{|\textit{main}|}|\\
\end{tabular}
\end{center}
%
or alternatively with:
%
\begin{center}
\begin{tabular}{l}
|\input{childdoc.def}|\\
|\childdocby{|\textit{main}|}|\\
\end{tabular}
\end{center}
%
Both forms have slightly different effects as described above.
The main file is prepared as usual, see \secref{sec:include}.

%%%%%%%%%%%%%%%%%%%%%%%%%%%%%%%%%%%%%%%%%%%%%%%%%%%%%%%%%%%%%%%%%%%%%%%%%%%%%%%%
\subsection{Legacy Detection}
\label{sec:detection}

The directive |\childdocmain| in the main file can detect
whether the complete document or merely a child is to be compiled
even without using the directive |\childdocof|.
This method is deprecated because it is less robust
and there is no compelling reason to use it;
it is merely provided for backward compatibility
and it may be removed in future versions.

If the detection mechanism is to be used,
it is mandatory to correctly specify
the filename of the main file as the argument of |\childdocmain|:
%
\begin{center}
\begin{tabular}{l}
|\input{childdoc.def}|\\
|\childdocmain{|\textit{main}|}|\\
\end{tabular}
\end{center}
%
If |\jobname| does not match the argument \textit{main} of |\childdocmain|,
it is assumed that |\jobname| points to the child file to be compiled.
When using |\childdocmain| with the main file specified as argument,
it suffices to start a child file
with just |\input{|\textit{main}|}|
without loading of the package and using |\childdocof|.
If instead all processing is done
with the appropriate \textsf{childdoc} directives,
the argument of \textit{main} of |\childdocmain| can be empty.

An alternative version of the command line processing described
in \secref{sec:commandline} using the detection mechanism reads:
%
\begin{center}
|... -jobname "|\textit{target}|" "|[\textit{flags}]%
[|\def\jobname{|\textit{dest}|}|]|\input{|\textit{main}|}"|
\end{center}

%%%%%%%%%%%%%%%%%%%%%%%%%%%%%%%%%%%%%%%%%%%%%%%%%%%%%%%%%%%%%%%%%%%%%%%%%%%%%%%%
\subsection{Manual Code}
\label{sec:manual}

In case one cannot be certain whether the definitions file |childdoc.def|
is installed on the target \TeX{} distribution
and one prefers not to ship it,
it is conceivable to paste a few relevant commands into the sources.

To that end, drop all statements |\input{childdoc.def}|
and perform the replacements as outlined below.
Instead of |\childdocmain{|\textit{main}|}| add the following code
to the top of the main file:
%
\begin{center}
\begin{tabular}{l}
|\||ifdefined\childdocname\endinput\||fi\newif\ifchilddoc|\\
|\edef\childdocname{\scantokens\expandafter{\jobname\noexpand}}|\\
|\def\childdocmain{|\textit{main}|}\||ifx\childdocmain\childdocname\||else|\\
|\childdoctrue\includeonly{\childdocname}\let\jobname\childdocmain\||fi|\\
\end{tabular}
\end{center}
%
Instead of |\childdocof{|\textit{main}|}| just include the main file
at the top of each child file:
%
\begin{center}
|\input{|\textit{main}|}|
\end{center}
%
A simple redirection |\childdocforward{|\textit{dest}|}| is achieved by:
%
\begin{center}
|\def\jobname{|\textit{dest}|}\input{\jobname}|
\end{center}
%
The redirection with prefix
|\childdocforwardprefix[|\textit{prefix}|]{|\textit{dest}|}|
is accomplished by:
%
\begin{center}
\begin{tabular}{l}
|{\edef\jobname{\scantokens\expandafter{\jobname\noexpand}}|\\
|\def\redirectjob |\textit{prefix}|#1~~~{\gdef\jobname{|\textit{dest}|#1}}|\\
|\expandafter\redirectjob\jobname~~~}\input{\jobname}|
\end{tabular}
\end{center}

In an alternative approach,
child documents can be compiled by a specific command line
without additional code or specific definitions:
%
\begin{center}
|... -jobname "|\textit{target}|" "|[\textit{flags}]%
|\includeonly{|\textit{dest}|}\input{|\textit{main}|}"|
\end{center}
%

%%%%%%%%%%%%%%%%%%%%%%%%%%%%%%%%%%%%%%%%%%%%%%%%%%%%%%%%%%%%%%%%%%%%%%%%%%%%%%%%
%%%%%%%%%%%%%%%%%%%%%%%%%%%%%%%%%%%%%%%%%%%%%%%%%%%%%%%%%%%%%%%%%%%%%%%%%%%%%%%%
\section{Information}

%%%%%%%%%%%%%%%%%%%%%%%%%%%%%%%%%%%%%%%%%%%%%%%%%%%%%%%%%%%%%%%%%%%%%%%%%%%%%%%%
\subsection{Copyright}

Copyright \copyright{} 2017--2018 Niklas Beisert

This work may be distributed and/or modified under the
conditions of the \LaTeX{} Project Public License, either version 1.3
of this license or (at your option) any later version.
The latest version of this license is in
  \url{http://www.latex-project.org/lppl.txt}
and version 1.3 or later is part of all distributions of \LaTeX{}
version 2005/12/01 or later.

This work has the LPPL maintenance status `maintained'.

The Current Maintainer of this work is Niklas Beisert.

This work consists of the files |README.txt|, |childdoc.ins| and |childdoc.dtx|
as well as the derived files |childdoc.def|, |cdocsamp.tex|
with |cdocsch1.tex|, |cdocsch2.tex|, |cdocspt3.tex|, |cdocspt4.tex|,
|cdocsdrf.tex|, |cdocsfn1.tex|, |cdocsfn2.tex|
as well as |childdoc.pdf|.

%%%%%%%%%%%%%%%%%%%%%%%%%%%%%%%%%%%%%%%%%%%%%%%%%%%%%%%%%%%%%%%%%%%%%%%%%%%%%%%%
\subsection{Files and Installation}

The package consists of the files:
%
\begin{center}
\begin{tabular}{ll}
    |README.txt|   & readme file \\
    |childdoc.ins| & installation file \\
    |childdoc.dtx| & source file \\
    |childdoc.def| & definition file \\
    |cdocsamp.tex| & sample main file \\
    |cdocsch1.tex| & sample include file \\
    |cdocsch2.tex| & sample include file \\
    |cdocspt3.tex| & sample part file \\
    |cdocspt4.tex| & sample part file \\
    |cdocsdrf.tex| & sample redirection file \\
    |cdocsfn1.tex| & sample redirection file \\
    |cdocsfn2.tex| & sample redirection file \\
    |childdoc.pdf| & manual
\end{tabular}
\end{center}
%
The distribution consists of the files
|README.txt|, |childdoc.ins| and |childdoc.dtx|.
%
\begin{itemize}
\item
Run (pdf)\LaTeX{} on |childdoc.dtx|
to compile the manual |childdoc.pdf| (this file).
\item
Run \LaTeX{} on |childdoc.ins| to create the definitions file |childdoc.def|
and the sample |cdocsamp.tex| with include files
|cdocsch1.tex|, |cdocsch2.tex|, |cdocspt3.tex|, |cdocspt4.tex|,
|cdocsdrf.tex|, |cdocsfn1.tex|, |cdocsfn2.tex|.
Then copy the file |childdoc.def| to an appropriate directory of your \LaTeX{}
distribution, e.g.\ \textit{texmf-root}|/tex/latex/childdoc|.
\end{itemize}

%%%%%%%%%%%%%%%%%%%%%%%%%%%%%%%%%%%%%%%%%%%%%%%%%%%%%%%%%%%%%%%%%%%%%%%%%%%%%%%%
\subsection{Related CTAN Packages}

There are several other packages which offer a similar functionality:
%
\begin{itemize}
\item
The packages
\href{http://ctan.org/pkg/docmute}{\textsf{docmute}},
\href{http://ctan.org/pkg/includex}{\textsf{includex}} and
\href{http://ctan.org/pkg/standalone}{\textsf{standalone}}
provide commands to include only the document body of
a child file thus allowing both files to be compiled individually.
\item
The packages \href{http://ctan.org/pkg/subdocs}{\textsf{subdocs}}
and \href{http://ctan.org/pkg/subfiles}{\textsf{subfiles}}
provide structures in which the main and child documents can be
encapsulated and allowing them to be compiled individually.
The inclusion mechanism is different from the conventional |\include|.
\item
The package \href{http://ctan.org/pkg/combine}{\textsf{combine}}
is an elaborate solution to combine several documents into one.
\end{itemize}
%
See also the CTAN topic \href{http://ctan.org/topic/subdocs}{\textsf{subdocs}}
for further related packages.
The present package differs from the above solutions in that
a document structure constructed with the conventional |\include| mechanism
just needs two extra commands at the top of every file
such that all constituent files can be compiled individually.

%%%%%%%%%%%%%%%%%%%%%%%%%%%%%%%%%%%%%%%%%%%%%%%%%%%%%%%%%%%%%%%%%%%%%%%%%%%%%%%%
%\subsection{Feature Suggestions}
%
%The following is a list of features which may be useful for future
%versions of this package:
%%
%\begin{itemize}
%\item
%\ldots
%\end{itemize}

%%%%%%%%%%%%%%%%%%%%%%%%%%%%%%%%%%%%%%%%%%%%%%%%%%%%%%%%%%%%%%%%%%%%%%%%%%%%%%%%
\subsection{Revision History}

%%%%%%%%%%%%%%%%%%%%%%%%%%%%%%%%%%%%%%%%
\paragraph{v2.0:} 2018/12/30

\begin{itemize}
\item
immediate forward processing
\item
added |\childdocby| mechanism
\item
manual restructured
\end{itemize}

%%%%%%%%%%%%%%%%%%%%%%%%%%%%%%%%%%%%%%%%
\paragraph{v1.6:} 2018/01/17

\begin{itemize}
\item
application for development of include files
\item
corrections to manual
\end{itemize}

%%%%%%%%%%%%%%%%%%%%%%%%%%%%%%%%%%%%%%%%
\paragraph{v1.5:} 2017/05/21

\begin{itemize}
\item
more complete structuring introduced
\item
|\childdocof| introduced
\item
|\childdoc| renamed to |\childdocmain|
\item
|\childredirect| renamed to |\childdocforward| and |\childdocforwardprefix|
and functionality expanded
\end{itemize}

%%%%%%%%%%%%%%%%%%%%%%%%%%%%%%%%%%%%%%%%
\paragraph{v1.0:} 2017/04/27

\begin{itemize}
\item
manual and install package
\item
first version published on CTAN
\end{itemize}

%%%%%%%%%%%%%%%%%%%%%%%%%%%%%%%%%%%%%%%%
\paragraph{v0.6:} 2017/04/26

\begin{itemize}
\item
redirection mechanism added
\end{itemize}

%%%%%%%%%%%%%%%%%%%%%%%%%%%%%%%%%%%%%%%%
\paragraph{v0.5:} 2017/04/26

\begin{itemize}
\item
functionality in definition file
\end{itemize}


%%%%%%%%%%%%%%%%%%%%%%%%%%%%%%%%%%%%%%%%%%%%%%%%%%%%%%%%%%%%%%%%%%%%%%%%%%%%%%%%
%%%%%%%%%%%%%%%%%%%%%%%%%%%%%%%%%%%%%%%%%%%%%%%%%%%%%%%%%%%%%%%%%%%%%%%%%%%%%%%%
%%%%%%%%%%%%%%%%%%%%%%%%%%%%%%%%%%%%%%%%%%%%%%%%%%%%%%%%%%%%%%%%%%%%%%%%%%%%%%%%
\appendix

\settowidth\MacroIndent{\rmfamily\scriptsize 000\ }

 \DocInput{childdoc.dtx}

\end{document}
%</driver>
% \fi
%
% %%%%%%%%%%%%%%%%%%%%%%%%%%%%%%%%%%%%%%%%%%%%%%%%%%%%%%%%%%%%%%%%%%%%%%%%%%%%%%
% %%%%%%%%%%%%%%%%%%%%%%%%%%%%%%%%%%%%%%%%%%%%%%%%%%%%%%%%%%%%%%%%%%%%%%%%%%%%%%
% \section{Sample}
%\iffalse
%<*samplemain>
%\fi
%
% The following presents a sample document
% with two chapters, two parts, a title page,
% a compile flag as well as three forwarding files to set the flag.
% It consists of eight |.tex| files:
% \begin{center}
% \begin{tabular}{ll}
% |cdocsamp.tex|&main file\\
% |cdocsch1.tex|&include file for chapter 1\\
% |cdocsch2.tex|&include file for chapter 2\\
% |cdocspt3.tex|&include file for part 3\\
% |cdocspt4.tex|&include file for part 4\\
% |cdocsdrf.tex|&forwarding file for main file in draft mode\\
% |cdocsfi1.tex|&forwarding file for final version of chapter 1\\
% |cdocsfi2.tex|&forwarding file for final version of chapter 2\\
% \end{tabular}
% \end{center}
% Each of the eight files can be compiled directly by the \LaTeX{} compiler.
%
% %%%%%%%%%%%%%%%%%%%%%%%%%%%%%%%%%%%%%%
% \paragraph{Main File.}
%
% The main file is called |cdocsamp.tex|.
%
% Load the \textsf{childdoc} definitions and
% declare the filename for the main document:
%    \begin{macrocode}
\input{childdoc.def}
\childdocmain{}
%    \end{macrocode}

% Optional override for |\version| flag:
%    \begin{macrocode}
%%\ifchilddoc\else\providecommand{\version}{draft}\fi
%    \end{macrocode}

% Define the default values for the |\version| flag
% (|final| for the main file and |draft| for childs):
%    \begin{macrocode}
\ifchilddoc
\providecommand{\version}{draft}
\else
\providecommand{\version}{final}
\fi
%    \end{macrocode}

% Load the standard document class:
%    \begin{macrocode}
\documentclass[12pt]{article}
%    \end{macrocode}

% Start the document body:
%    \begin{macrocode}
\begin{document}
%    \end{macrocode}

% Declare a title page.
% Print title, part of document being processed and version flag:
%    \begin{macrocode}
\addtocounter{page}{-1}
\begin{center}
{\LARGE\bfseries{}childdoc example\par}
\vspace{1cm}
\ifchilddoc
\ifchilddocmanual part\else chapter\fi:
`\childdocname' of `\childdocjob'\par
\else
main document: `\childdocjob'\par
\fi
version: \version\par
\end{center}
\newpage
%    \end{macrocode}

% Manually include selected file,
% otherwise process as usual:
%    \begin{macrocode}
\ifchilddocmanual
\section*{part `\childdocname'}
\input{\childdocname}
\else
%    \end{macrocode}

% Include the two chapters:
%    \begin{macrocode}
\include{cdocsch1}
\include{cdocsch2}
%    \end{macrocode}

% Include the two parts unless only chapters should be displayed:
%    \begin{macrocode}
\ifchilddoc\else
\section{part three}
\input{cdocspt3}
\section{part four}
\input{cdocspt4}
\fi
%    \end{macrocode}

% Process as usual until here:
%    \begin{macrocode}
\fi
%    \end{macrocode}

% End of document body:
%    \begin{macrocode}
\end{document}
%    \end{macrocode}
%\iffalse
%</samplemain>
%\fi
%
% %%%%%%%%%%%%%%%%%%%%%%%%%%%%%%%%%%%%%%
% \paragraph{Chapter Include Files.}
%
% The include files are called |cdocsch1.tex| and |cdocsch2.tex|.
%
%\iffalse
%<*samplechap1|samplechap2>
%\fi

% Optional override for |\version| flag:
%    \begin{macrocode}
%%\providecommand{\version}{final}
%    \end{macrocode}

% Include the main document:
%    \begin{macrocode}
\input{childdoc.def}
\childdocof{cdocsamp}
%    \end{macrocode}

%\iffalse
%</samplechap1|samplechap2>
%\fi
%
%\iffalse
%<*samplechap1>
%\fi
% Some text for chapter 1:
%    \begin{macrocode}
\section{one}
some text in chapter one
%    \end{macrocode}

%\iffalse
%</samplechap1>
%\fi
% Some text for chapter 2:
%\iffalse
%<*samplechap2>
%\fi
%    \begin{macrocode}
\section{two}
more text in chapter two
%    \end{macrocode}

%\iffalse
%</samplechap2>
%\fi
%
% %%%%%%%%%%%%%%%%%%%%%%%%%%%%%%%%%%%%%%
% \paragraph{Part Include Files.}
%
% The include files are called |cdocspt3.tex| and |cdocspt4.tex|.
%
%\iffalse
%<*samplepart3|samplepart4>
%\fi

% Optional override for |\version| flag:
%    \begin{macrocode}
%%\providecommand{\version}{final}
%    \end{macrocode}

% Include the main document:
%    \begin{macrocode}
\input{childdoc.def}
\childdocby{cdocsamp}
%    \end{macrocode}

%\iffalse
%</samplepart3|samplepart4>
%\fi
%
%\iffalse
%<*samplepart3>
%\fi
% Some text for part 3:
%    \begin{macrocode}
some text in part three
%    \end{macrocode}

%\iffalse
%</samplepart3>
%\fi
% Some text for part 4:
%\iffalse
%<*samplepart4>
%\fi
%    \begin{macrocode}
more text in part four
%    \end{macrocode}

%\iffalse
%</samplepart4>
%\fi
%
% %%%%%%%%%%%%%%%%%%%%%%%%%%%%%%%%%%%%%%
% \paragraph{Forwarding for a Complete Draft.}
%
% The following forwarding file |cdocsdrf.tex|
% compiles the main document in draft mode:
%\iffalse
%<*sampledraft>
%\fi
%    \begin{macrocode}
\def\version{draft}
\input{childdoc.def}
\childdocforward{cdocsamp}
%    \end{macrocode}

%\iffalse
%</sampledraft>
%\fi
%
% %%%%%%%%%%%%%%%%%%%%%%%%%%%%%%%%%%%%%%
% \paragraph{Forwarding for Final Version of the Chapters.}
%
% The following forwarding files |cdocsfn1.tex| and |cdocsfn2.tex|
% (with identical content)
% compile the final versions of the child documents
% |cdocsch1.tex| and |cdocsch2.tex|, respectively:
%\iffalse
%<*samplefinal>
%\fi
%    \begin{macrocode}
\def\version{final}
\input{childdoc.def}
\childdocforwardprefix[cdocsamp]{cdocsfn}{cdocsch}
%    \end{macrocode}

%\iffalse
%</samplefinal>
%\fi
%
% %%%%%%%%%%%%%%%%%%%%%%%%%%%%%%%%%%%%%%
% \paragraph{Command Line Processing.}
%
% The following three command lines generate the output files
% |cdocscld|, |cdocscl1| and |cdocscl2|
% which should be identical to
% |cdocsdrf|, |cdocsch1| and |cdocsfn2|, respectively:
% \begin{center}
% \begin{tabular}{l}
% |latex -jobname cdocscld \|\\
% |  "\def\version{draft}\input{childdoc.def}\childdocforward{cdocsamp}"|\\
% |latex -jobname cdocscl1 \|\\
% |  "\input{childdoc.def}\childdocforward[cdocsamp]{cdocsch1}"|\\
% |latex -jobname cdocscl2 \|\\
% |  "\def\version{final}\input{childdoc.def}\childdocforward{cdocsch2}"|
% \end{tabular}
% \end{center}
% Note that the trailing backslash on each first line
% merely continues the input to the second line
% (for convenient cut ant paste).
% Furthermore, the command |latex| can be replaced by any
% of its alternative versions such as |pdflatex|.
%
% %%%%%%%%%%%%%%%%%%%%%%%%%%%%%%%%%%%%%%%%%%%%%%%%%%%%%%%%%%%%%%%%%%%%%%%%%%%%%%
% %%%%%%%%%%%%%%%%%%%%%%%%%%%%%%%%%%%%%%%%%%%%%%%%%%%%%%%%%%%%%%%%%%%%%%%%%%%%%%
% \section{Implementation}
%\iffalse
%<*package>
%\fi
%
% This section describes the definitions file |childdoc.def|.

% The definitions cannot be loaded using |\usepackage| or |\RequirePackage|
% which has a mechanism to prevent loading a style file more than once.
% When loading the definitions by means of |\input|
% multiple instances have to be prevented manually:
%\iffalse
%This code needs to be before the `\ProvidesFile' directive
%which is defined at the beginning of this file.
%Therefore it is also placed there and commented out here.
%</package>
%<*discard>
%\fi
%    \begin{macrocode}
\ifdefined\childdocmain\endinput\fi
%    \end{macrocode}
%\iffalse
%</discard>
%<*package>
%\fi
%
% \macro{\ifchilddoc}
% \macro{\ifchilddocmanual}
% The conditional |\ifchilddoc| tells whether a
% child (true) or main (false) document is being compiled.
% The conditional |\ifchilddocmanual| tells whether
% the |\includeonly| mechanism is used (false) or
% the selection of child files must be performed manually (true).
% The definitions initialise to false:
%    \begin{macrocode}
\newif\ifchilddoc
\newif\ifchilddocmanual
%    \end{macrocode}

% \macro{\childdocname}
% \macro{\childdocjob}
% The macro |\childdocname| stores the name of the main document
% to be compiled. The macro |\childdocjob| stores the name of
% the document on which the \LaTeX{} compiler was originally invoked.
% The content of |\jobname| cannot be compared
% to filenames specified in the source due to different catcodes.
% The following code rescans |\jobname|, stores the result
% in |\childdocname| and saves a copy in |\childdocjob|:
%    \begin{macrocode}
\edef\childdocname{\scantokens\expandafter{\jobname\noexpand}}
\let\childdocjob\childdocname
%    \end{macrocode}

% \macro{\childdocdisable}
% The macro |\childdocdisable| prevents the main file
% from being processed more than once.
% At this stage, the main document command |\childdocmain|
% is assumed to be called once again where it should do nothing.
% Any subsequent call to it should prevent
% a secondary processing of the main document
% It overwrites the forwarding commands
% |\childdocof| and |\childdocforward|
% with empty macros to prevent further inclusions of the main document:
%    \begin{macrocode}
\newcommand{\childdocdisable}
{
  \renewcommand{\childdocmain}[1]{\renewcommand{\childdocmain}[1]{\endinput}}
  \renewcommand{\childdocof}[1]{}
  \renewcommand{\childdocby}[2][]{}
  \renewcommand{\childdocforward}[2][]{}
  \renewcommand{\childdocdisable}{}
}
%    \end{macrocode}

% \macro{\childdocmain}
% The macro |\childdocmain| is to be called at the top of the main file
% with nothing or the main filename (without extension) as argument.
% First, it breaks loops.
% If the argument is not empty and does not match |\childdocname|
% (which is set by the first inclusion of |childdoc.def|),
% |\ifchilddoc| is set to true, |\includeonly| is applied to the child file
% and |\jobname| is set to the main file
% (for proper handling of |.aux| files):
%    \begin{macrocode}
\newcommand{\childdocmain}[1]
{
  \childdocdisable\childdocmain{}
  \if?#1?\else
    \begingroup
      \def\childdoctmp{#1}
      \ifx\childdoctmp\childdocname
        \def\childdoctmp{}
      \else
        \def\childdoctmp
        {
          \childdoctrue
          \includeonly{\childdocname}
          \def\childdocjob{#1}
          \def\jobname{#1}
        }
      \fi
      \expandafter
    \endgroup
    \childdoctmp
  \fi
}
%    \end{macrocode}

% \macro{\childdocof}
% The command |\childdocof| redirects
% compilation to the main file |#1|.
%    \begin{macrocode}
\newcommand{\childdocof}[1]
{
  \childdocdisable
  \childdoctrue
  \includeonly{\childdocname}
  \def\jobname{#1}
  \def\childdocjob{#1}
  \input{#1}
}
%    \end{macrocode}

% \macro{\childdocby}
% The command |\childdocby| ....
%    \begin{macrocode}
\newcommand{\childdocby}[2][]
{
  \childdocdisable
  \childdoctrue
  \childdocmanualtrue
  \if?#1?\else
    \def\jobname{#2}
  \fi
  \def\childdocjob{#2}
  \input{#2}
  \endinput
}
%    \end{macrocode}

% \macro{\childdocforward}
% The command |\childdocforward| redirects
% compilation to the main file or
% (if the optional argument is given) a child file.
% Parameters are set as if the main file
% or a child file starting with |\childdocof| was compiled.
% Then compilation is handed over to the main file:
%    \begin{macrocode}
\newcommand{\childdocforward}[2][]
{
  \begingroup
    \if?#1?
      \def\childdoctmp
      {
        \def\childdocname{#2}
        \def\childdocjob{#2}
        \def\jobname{#2}
        \input{#2}
        \endinput
      }
    \else
      \def\childdoctmp
      {
        \childdocdisable
        \def\childdocname{#2}
        \childdoctrue
        \includeonly{#2}
        \def\childdocjob{#1}
        \def\jobname{#1}
        \input{#1}
        \endinput
      }
    \fi
    \expandafter
  \endgroup
  \childdoctmp
}
%    \end{macrocode}

% \macro{\childdocforwardprefix}
% The command |\childdocforwardprefix| redirects
% compilation to the main or a child file by means of a pattern.
% The prefix |#1| in the current filename is replaced by |#2|
% and the suffix of the current filename is kept
% (it is assumed that the filename does not contain the substring `|~~~|'
% which is used as a delimiter).
% Compilation is handed over to the new file by |\childdocforward|:
%    \begin{macrocode}
\newcommand{\childdocforwardprefix}[3][]
{
  \begingroup
    \def\childdocextract #2##1~~~{\def\childdoctmp{\childdocforward[#1]{#3##1}}}
    \expandafter\childdocextract\childdocname~~~
    \expandafter
  \endgroup
  \childdoctmp
}
%    \end{macrocode}

% \macro{\childdoc}
% The deprecated macro |\childdoc| is a legacy version of |\childdocmain|:
%    \begin{macrocode}
\newcommand{\childdoc}{\childdocmain}
%    \end{macrocode}

% \macro{\childdocredirect}
% The deprecated macro |\childdocredirect| is a legacy version
% of |\childdocforward| and |\childdocforwardprefix|:
%    \begin{macrocode}
\newcommand{\childdocredirect}[2][]
{
  \begingroup
    \if?#1?
      \def\childdoctmp{\childdocforward{#2}}
    \else
      \def\childdoctmp{\childdocforwardprefix{#1}{#2}}
    \fi
    \expandafter
  \endgroup
  \childdoctmp
}
%    \end{macrocode}

%\iffalse
%</package>
%\fi
%
\endinput
\childdocforward[cdocsamp]{cdocsch1}"|\\
% |latex -jobname cdocscl2 \|\\
% |  "\def\version{final}% \iffalse
%
% childdoc.dtx Copyright (C) 2017-2018 Niklas Beisert
%
% This work may be distributed and/or modified under the
% conditions of the LaTeX Project Public License, either version 1.3
% of this license or (at your option) any later version.
% The latest version of this license is in
%   http://www.latex-project.org/lppl.txt
% and version 1.3 or later is part of all distributions of LaTeX
% version 2005/12/01 or later.
%
% This work has the LPPL maintenance status `maintained'.
%
% The Current Maintainer of this work is Niklas Beisert.
%
% This work consists of the files childdoc.dtx and childdoc.ins
% and the derived files childdoc.def and cdocsamp.tex with
% cdocsch1.tex, cdocsch2.tex, cdocsdrf.tex, cdocsfn1.tex, cdocsfn2.tex.
%
%<package>\ifdefined\childdocmain\endinput\fi
%<package>\ProvidesFile{childdoc.def}[2018/12/30 v2.0 child document driver]
%<samplemain>\ProvidesFile{cdocsamp.tex}[2018/12/30 v2.0 sample for childdoc]
%<*driver>
%\ProvidesFile{childdoc.drv}[2018/12/30 v2.0 childdoc reference manual file]
\PassOptionsToClass{10pt,a4paper}{article}
\documentclass{ltxdoc}

\usepackage[margin=35mm]{geometry}
\usepackage{hyperref}
\usepackage{hyperxmp}
\usepackage[usenames]{color}

\hypersetup{colorlinks=true}
\hypersetup{pdfstartview=FitH}
\hypersetup{pdfpagemode=UseNone}
\hypersetup{pdfsource={}}
\hypersetup{pdflang={en-UK}}
\hypersetup{pdfcopyright={Copyright 2017-2018 Niklas Beisert.
  This work may be distributed and/or modified under the
  conditions of the LaTeX Project Public License, either version 1.3
  of this license or (at your option) any later version.}}
\hypersetup{pdflicenseurl={http://www.latex-project.org/lppl.txt}}
\hypersetup{pdfcontactaddress={ETH Zurich, ITP, HIT K,
  Wolfgang-Pauli-Strasse 27}}
\hypersetup{pdfcontactpostcode={8093}}
\hypersetup{pdfcontactcity={Zurich}}
\hypersetup{pdfcontactcountry={Switzerland}}
\hypersetup{pdfcontactemail={nbeisert@itp.phys.ethz.ch}}
\hypersetup{pdfcontacturl={http://people.phys.ethz.ch/\xmptilde nbeisert/}}

\newcommand{\secref}[1]{\hyperref[#1]{section \ref*{#1}}}

\parskip1ex
\parindent0pt
\let\olditemize\itemize
\def\itemize{\olditemize\parskip0pt}

\begin{document}

\title{The \textsf{childdoc} Package}
\hypersetup{pdftitle={The childdoc Package}}
\author{Niklas Beisert\\[2ex]
  Institut f\"ur Theoretische Physik\\
  Eidgen\"ossische Technische Hochschule Z\"urich\\
  Wolfgang-Pauli-Strasse 27, 8093 Z\"urich, Switzerland\\[1ex]
  \href{mailto:nbeisert@itp.phys.ethz.ch}
  {\texttt{nbeisert@itp.phys.ethz.ch}}}
\hypersetup{pdfauthor={Niklas Beisert}}
\hypersetup{pdfsubject={Manual for the LaTeX2e Package childdoc}}
\date{30 December 2018, \textsf{v2.0}}
\maketitle

\begin{abstract}\noindent
\textsf{childdoc} is a \LaTeXe{} package
that enables the direct compilation
of document sections included by |\include|
to individual files.
\end{abstract}

\begingroup
\parskip0ex
\tableofcontents
\endgroup

%%%%%%%%%%%%%%%%%%%%%%%%%%%%%%%%%%%%%%%%%%%%%%%%%%%%%%%%%%%%%%%%%%%%%%%%%%%%%%%%
%%%%%%%%%%%%%%%%%%%%%%%%%%%%%%%%%%%%%%%%%%%%%%%%%%%%%%%%%%%%%%%%%%%%%%%%%%%%%%%%
\section{Introduction}

\LaTeX{} provides a mechanism to structure a large document (such as a book)
into a main file and several child files (containing the chapters)
using the |\include| command.
This mechanism is beneficial for documents
which span hundreds of pages in order to
make the source file(s) more manageable.
Moreover, compilation can be restricted to
selected child files by means of the |\includeonly| command.
The latter feature can be used to reduce the compilation time while editing
(this was significantly more useful in the earlier days of \LaTeX{})
or to generate a smaller document which is easier to navigate.
Another application of |\includeonly| is to generate
documents consisting of selected parts of the complete document.

However, there are a few drawbacks of the plain |\include| mechanism:
\begin{itemize}
\item
The child files cannot be compiled on their own,
they can only be compiled via the main file.
A naive editing environment
(such as a text editor with an option
to have the current file processed by \LaTeX)
may require one to switch to the main file before compiling;
attempting to compile the child file produces errors.
\item
The main file must be modified (each time)
to adjust the |\includeonly| command
to the present needs. This easily leaves the main file in a messy state.
\item
The generated document will always carry the filename
of the main document. This is inconvenient if
several child files are to be compiled and
to be kept for distribution.
\end{itemize}

The present package provides a simple interface
to make child files individually compilable by \LaTeX{}.
Compiling a child file then has the same effect as compiling
the main file with an |\includeonly| command
to select the appropriate child.
Moreover the generated document will carry the name of the child
rather than the main file.
This resolves all three above issues.

This feature is meant to make the editing of books,
thesis documents and lecture notes somewhat more convenient.
However, the package can also be used efficiently for
composing a series of documents (such as exercise sheets)
which are typically distributed individually.
It then assists the author in generating the individual documents
(potentially in different versions)
as well as a document containing the collected series.
Another application is in developing style files
or other kinds of included material
where compilation of the style file could redirect
to a sample or test file.

%%%%%%%%%%%%%%%%%%%%%%%%%%%%%%%%%%%%%%%%%%%%%%%%%%%%%%%%%%%%%%%%%%%%%%%%%%%%%%%%
%%%%%%%%%%%%%%%%%%%%%%%%%%%%%%%%%%%%%%%%%%%%%%%%%%%%%%%%%%%%%%%%%%%%%%%%%%%%%%%%
\section{Usage}

First of all, the package \textsf{childdoc} is \emph{not} a standard
\LaTeXe{} |.sty| style file! Therefore it needs to be invoked in
a non-standard way.

%%%%%%%%%%%%%%%%%%%%%%%%%%%%%%%%%%%%%%%%%%%%%%%%%%%%%%%%%%%%%%%%%%%%%%%%%%%%%%%%
\subsection{Included Files}
\label{sec:include}

%%%%%%%%%%%%%%%%%%%%%%%%%%%%%%%%%%%%%%%%
\DescribeMacro{\childdocmain}
To use the package, add the commands
\begin{center}
\begin{tabular}{l}
|\input{childdoc.def}|\\
|\childdocmain{}|\\
\end{tabular}
\end{center}
at the very top of the main \LaTeX{} file,
in particular \emph{before} the |\documentclass| statement!
The argument of |\childdocmain| should be left empty
(but it must be present).

%%%%%%%%%%%%%%%%%%%%%%%%%%%%%%%%%%%%%%%%
\DescribeMacro{\childdocof}
Furthermore, add the commands
\begin{center}
\begin{tabular}{l}
|\input{childdoc.def}|\\
|\childdocof{|\textit{main}|}|\\
\end{tabular}
\end{center}
at the top of every child file \textit{child}
which is included by |\include{|\textit{child}|}|
from within the main file
(or at least for those files to be compiled individually).
The argument \textit{main} must be the filename of the main file.

There are a couple of
considerations in setting up the main and child documents:

%%%%%%%%%%%%%%%%%%%%%%%%%%%%%%%%%%%%%%%%
\paragraph{Restrictions.}

Please note the following restrictions:
\begin{itemize}
\item
|\childdocmain| must be called with one argument \textit{main}
to ensure compatibility with earlier version of the package.
It must either be empty (|\childdocmain{}|)
or precisely match the filename of the main file in which it is specified.
See \secref{sec:detection} for further information.
\item
The filename \textit{main} must be specified without the |.tex| extension.
\item
The filename \textit{main} is case sensitive
(even in case-insensitive file systems)
due to internal string comparison.
\item
The argument \textit{main} should be fully expanded, it cannot be a macro.
\item
Subdirectories and special characters should be avoided in filenames.
\item
The command |\childdocmain{|\textit{main}|}| must be followed by a whitespace.
It should not be followed immediately by another command
or by a comment mark `|%|'.
This is because the \TeX{} parser reads the token immediately following
the argument of |\childdocmain| and puts it
at the beginning of every child section;
however, a white\-space is ignored.
\end{itemize}

%%%%%%%%%%%%%%%%%%%%%%%%%%%%%%%%%%%%%%%%
\paragraph{Content of Main File.}

It is advisable to place all content in the child files included by |\include|.
Any output contained in the main file will appear in all child documents
unless suppressed manually;
it cannot be suppressed automatically by the |\includeonly| directive
and thus should normally be avoided.
A method to include some content in the main file
by means of conditional processing is described in \secref{sec:conditional}.

%%%%%%%%%%%%%%%%%%%%%%%%%%%%%%%%%%%%%%%%
\paragraph{Page Numbering.}

When only a part of the document is compiled,
the appropriate numbering of pages
(as well as other status parameters)
is determined from the |.aux| files.
The latter contain information from previous passes.
However this information needs to propagate through
all intermediate child documents.
Therefore the page numbering in child documents may well
be inconsistent until the complete document is compiled at least once.

A useful (if unconventional) way to always ensure a consistent
page numbering is to restart the numbering in each child document
and denote the pages by `\textit{child}|.|\textit{page}'
where \textit{child} represents the chapter/section number of the child file.
This can be achieved by the command
|\numberwithin{page}{|\textit{child}|}|
of the \textsf{amsmath} package
where \textit{child} can be |chapter| or |section|
depending on the chosen structuring.
Alternatively, one can modify the macro |\thepage| appropriately
and reset the counter |page| at the start of each child file.

%%%%%%%%%%%%%%%%%%%%%%%%%%%%%%%%%%%%%%%%%%%%%%%%%%%%%%%%%%%%%%%%%%%%%%%%%%%%%%%%
\subsection{Conditional Processing}
\label{sec:conditional}

The package provides a mechanism to compile different versions
of a document. To customise the versions further some conditional processing
can come in handy to distinguish which version is being compiled.
The package provides two macros to describe the compilation context:

%%%%%%%%%%%%%%%%%%%%%%%%%%%%%%%%%%%%%%%%
\DescribeMacro{\ifchilddoc}
The conditional |\ifchilddoc| distinguishes between the compilation of
child documents and the main document:
%
\begin{center}
|\ifchilddoc |\textit{child-code}| |[|\||else |\textit{main-code}]| \||fi|
\end{center}

%%%%%%%%%%%%%%%%%%%%%%%%%%%%%%%%%%%%%%%%
\DescribeMacro{\childdocname}
\DescribeMacro{\childdocjob}
The macro |\childdocname| contains the filename (without extension)
of the main or child file being processed.
Note that |\childdocjob| will always contain the name of the main file.

%%%%%%%%%%%%%%%%%%%%%%%%%%%%%%%%%%%%%%%%
\paragraph{Title Page.}

Conditional processing can be used to include a title or banner page
in the main document when proper precautions are taken.
Importantly, the code in the main file should ensure that the page counter
(as well as other status parameters which are stored in the |.aux| files)
takes the same value after the conditional processing.
Otherwise the page numbers may take divergent values
depending on which part is compiled.

For example, a title page could be declared by:
%
\begin{center}
\begin{tabular}{l}
|\ifchilddoc\||else|\\
|\addtocounter{page}{-1}|\\
\textit{code for title page}\\
|\newpage|\\
|\||fi|
\end{tabular}
\end{center}
%
A banner page for the child documents can be generated by:
%
\begin{center}
\begin{tabular}{l}
|\ifchilddoc|\\
|\addtocounter{page}{-1}|\\
\textit{code for banner page}\\
|\newpage|\\
|\||fi|
\end{tabular}
\end{center}
%
Here one could write a message such as:
\begin{center}
|This is the part \childdocname{} of \childdocjob{}.|
\end{center}

%%%%%%%%%%%%%%%%%%%%%%%%%%%%%%%%%%%%%%%%%%%%%%%%%%%%%%%%%%%%%%%%%%%%%%%%%%%%%%%%
\subsection{Flags}
\label{sec:flags}

The package makes it easy to generate different versions
of the main or child documents.
To this end compilation flags can be defined
and assigned different default values.
They will be particularly useful in conjunction
with the forwarding mechanism described in \secref{sec:forward}.

For example, it may be useful to have a flag |\version|
which can be set to |draft| or |final|.
The document source will contain some conditional code
depending on the value of |\version|.
Suppose further, the flag should default to |final| for the main file
and to |draft| for child files
which is a natural assignment for editing the document.
This is achieved by placing the following code
in the preamble of the main document
(below the |\childdocmain| directive):
%
\begin{center}
\begin{tabular}{l}
|\ifchilddoc|\\
|\providecommand{\version}{draft}|\\
|\||else|\\
|\providecommand{\version}{final}|\\
|\||fi|
\end{tabular}
\end{center}
%
The definition by |\providecommand| makes sure
that previous definitions are not overwritten.
Further statements |\providecommand{\version}{...}|
can thus be added before the above code to override it.

For the main file, one might add a line
(between |\childdocmain| and the above block)
%
\begin{center}
|%\ifchilddoc\||else\providecommand{\version}{draft}\||fi|
\end{center}
%
which can be uncommented to produce a draft version.
Likewise one can add a line to the very top of a child file
(above the |\childdocof{|\textit{main}|}| directive)
%
\begin{center}
|%\providecommand{\version}{final}|
\end{center}
%
which can be uncommented to produce the final version of this child document.

%%%%%%%%%%%%%%%%%%%%%%%%%%%%%%%%%%%%%%%%%%%%%%%%%%%%%%%%%%%%%%%%%%%%%%%%%%%%%%%%
\subsection{Forwarding}
\label{sec:forward}

Different versions of the main or child documents
using compilation flags as described in \secref{sec:flags}
can be (permanently) stored in different files
for convenient compilation, viewing and distribution.
To this end, the package defines a command
to pass on compilation to a different file:

%%%%%%%%%%%%%%%%%%%%%%%%%%%%%%%%%%%%%%%%
\DescribeMacro{\childdocforward}
The command |\childdocforward| redirects processing to
another source file:
%
\begin{center}
\begin{tabular}{l}
|\input{childdoc.def}|\\
|\childdocforward[|\textit{main}|]{|\textit{dest}|}|\\
\end{tabular}
\end{center}
%
The argument \textit{dest} is the destination file
(without extension).
It should be the main file or one of the child files.
Note that further \textsf{childdoc} directives
such as |\childdocof| and |\childdocforward|
in the indicated file will be processed in this form.
The optional argument \textit{main}
passes on directly to the main file \textit{main}
while pretending to compile the child \textit{dest}.
This form behaves as if \textit{dest}
issues |\childdocof{|\textit{main}|}| right away,
and no further \textsf{childdoc} directives will be processed.

%%%%%%%%%%%%%%%%%%%%%%%%%%%%%%%%%%%%%%%%
\DescribeMacro{\...prefix}
In the alternative form |\childdocforwardprefix|,
%
\begin{center}
\begin{tabular}{l}
|\input{childdoc.def}|\\
|\childdocforwardprefix[|\textit{main}|]{|\textit{prefix}|}{|\textit{dest}|}|
\end{tabular}
\end{center}
%
the destination file is determined by a pattern
depending on the current file:
To make this work, the current file must be called
`{\textit{prefix}\hspace{0.2em}\textit{suffix}}'
with \textit{prefix} matching precisely the argument.
Processing is then passed on to the file
`{\textit{dest}\hspace{0.2em}\textit{suffix}}'.
Surely, the same effect is achieved by
directly specifying the
argument `{\textit{dest}\hspace{0.2em}\textit{suffix}}'
in the first form.
However, that requires to set up a different file
for each child. With the alternative form of the command
all these files can have exactly the same content
which simplifies setting them up and maintaining them.

For example, the following file |draft.tex|
with a compilation flag |\version| as described in \secref{sec:flags}
compiles the main document as a draft:
%
\begin{center}
\begin{tabular}{l}
|\def\version{draft}|\\
|\input{childdoc.def}|\\
|\childdocforward{|\textit{main}|}|
\end{tabular}
\end{center}
%
Likewise, the following files |final|\textit{nn}|.tex|
compile the final version of the child document
|child|\textit{nn}|.tex|:
%
\begin{center}
\begin{tabular}{l}
|\def\version{final}|\\
|\input{childdoc.def}|\\
|\childdocforwardprefix{final}{child}|
\end{tabular}
\end{center}
%

Note that when several versions of a main file and/or of each child file
are to be generated, it may be convenient to set up a |Makefile| or
shell script to automatise the process.

%%%%%%%%%%%%%%%%%%%%%%%%%%%%%%%%%%%%%%%%%%%%%%%%%%%%%%%%%%%%%%%%%%%%%%%%%%%%%%%%
\subsection{Command Line Processing}
\label{sec:commandline}

The effect of redirection files can also be achieved by invoking
the \LaTeX{} compiler with a more elaborate command line.
Most conveniently this should be done as part
of a shell script or a |Makefile|.

When using \textsf{childdoc} in the main file, the following
command lines effectively perform a redirection
(note that depending on the shell being used,
backslashes may have to be doubled: `|\|' $\to$ `|\\|'):
%
\begin{center}
|... -jobname "|\textit{target}|" |\\|"|[\textit{flags}]%
|\input{childdoc.def}\childdocforward[|\textit{main}|]{|\textit{dest}|}"|
\end{center}
%
Here \textit{target} is the name of the output file,
\textit{main} is the name of the main file
and \textit{dest} is the name of the main or child file to be processed
(all filenames without extensions).
The optional argument \textit{main} can be omitted
if \textit{main} matches \textit{dest}.
Optionally, compilation \textit{flags} can be defined via |\def| commands.
This command line makes the \TeX{} engine believe
it is compiling the file \textit{target}
whose content is specified as the latter parameter.
The provided code then forwards the processing to
\textit{main} or \textit{dest} as described in \secref{sec:forward}.

%%%%%%%%%%%%%%%%%%%%%%%%%%%%%%%%%%%%%%%%%%%%%%%%%%%%%%%%%%%%%%%%%%%%%%%%%%%%%%%%
\subsection{Include by Input}
\label{sec:input}

Including child documents by |\include| has some restrictions by design.
Most notably, the content of a child document always occupies
its own set of pages; pages cannot be shared between child documents.
Usually, this behaviour makes perfect sense
because each child document contain an essential part of the document.
However, in some situations it may be desirable to compose
a document from a collection of parts
without having mandatory page breaks between then.
For this case, the package
provides a mechanism to include parts
by |\input| which can also be processed individually.
However, by construction this mechanism
requires manual handling of the content to be output.

%%%%%%%%%%%%%%%%%%%%%%%%%%%%%%%%%%%%%%%%
\DescribeMacro{\ifchilddocmanual}
The main file should be prepared as usual, see \secref{sec:include}.
However, the document body must make a distinction
between processing of an individual part and of the main document, e.g.:
%
\begin{center}
\begin{tabular}{l}
|\ifchilddocmanual|\\
|\input{\childdocname}|\\
|\||else|\\
\textit{document body with }|\input{|\textit{part}|}|\\
|\||fi|
\end{tabular}
\end{center}
%
The conditional |\ifchilddocmanual| is true whenever
a part to be included by |\input| is being compiled,
and the name of the part is stored in |\childdocname|.

%%%%%%%%%%%%%%%%%%%%%%%%%%%%%%%%%%%%%%%%
\DescribeMacro{\childdocby}
Each part to be included by |\input| should start with:
%
\begin{center}
\begin{tabular}{l}
|\input{childdoc.def}|\\
|\childdocby{|\textit{main}|}|\\
\end{tabular}
\end{center}
%
The directive |\childdocby| is similar to |\childdocof|
described in \secref{sec:include},
but the subsequent selection of content must be done manually.
To that end, both |\ifchilddoc| and |\ifchilddocmanual|
will be true upon processing of a part,
and the name of the part is stored in |\childdocname|.
Note that |\jobname| will be set to the filename of the current part
so that each part receives an individual |.aux| file
that does not interfere with the |.aux| file(s) of the main document.
This behaviour can be altered by the alternative form
|\childdocby[*]{|\textit{main}|}| (with a non-empty optional argument)
which uses the |.aux| file of the main document
by setting |\jobname| to \textit{main}.

%%%%%%%%%%%%%%%%%%%%%%%%%%%%%%%%%%%%%%%%%%%%%%%%%%%%%%%%%%%%%%%%%%%%%%%%%%%%%%%%
\subsection{Driver Development}
\label{sec:driver}

The \textsf{childdoc} mechanism can also be use for the development
of definition files such as \LaTeX{} styles or classes.
This case differs from the above setup with multiple parts
included by |\include| in that no |\includeonly| should be invoked.
This can be achieved by starting the include file
(before |\ProvidesPackage|) with:
%
\begin{center}
\begin{tabular}{l}
|\input{childdoc.def}|\\
|\childdocforward{|\textit{main}|}|\\
\end{tabular}
\end{center}
%
or alternatively with:
%
\begin{center}
\begin{tabular}{l}
|\input{childdoc.def}|\\
|\childdocby{|\textit{main}|}|\\
\end{tabular}
\end{center}
%
Both forms have slightly different effects as described above.
The main file is prepared as usual, see \secref{sec:include}.

%%%%%%%%%%%%%%%%%%%%%%%%%%%%%%%%%%%%%%%%%%%%%%%%%%%%%%%%%%%%%%%%%%%%%%%%%%%%%%%%
\subsection{Legacy Detection}
\label{sec:detection}

The directive |\childdocmain| in the main file can detect
whether the complete document or merely a child is to be compiled
even without using the directive |\childdocof|.
This method is deprecated because it is less robust
and there is no compelling reason to use it;
it is merely provided for backward compatibility
and it may be removed in future versions.

If the detection mechanism is to be used,
it is mandatory to correctly specify
the filename of the main file as the argument of |\childdocmain|:
%
\begin{center}
\begin{tabular}{l}
|\input{childdoc.def}|\\
|\childdocmain{|\textit{main}|}|\\
\end{tabular}
\end{center}
%
If |\jobname| does not match the argument \textit{main} of |\childdocmain|,
it is assumed that |\jobname| points to the child file to be compiled.
When using |\childdocmain| with the main file specified as argument,
it suffices to start a child file
with just |\input{|\textit{main}|}|
without loading of the package and using |\childdocof|.
If instead all processing is done
with the appropriate \textsf{childdoc} directives,
the argument of \textit{main} of |\childdocmain| can be empty.

An alternative version of the command line processing described
in \secref{sec:commandline} using the detection mechanism reads:
%
\begin{center}
|... -jobname "|\textit{target}|" "|[\textit{flags}]%
[|\def\jobname{|\textit{dest}|}|]|\input{|\textit{main}|}"|
\end{center}

%%%%%%%%%%%%%%%%%%%%%%%%%%%%%%%%%%%%%%%%%%%%%%%%%%%%%%%%%%%%%%%%%%%%%%%%%%%%%%%%
\subsection{Manual Code}
\label{sec:manual}

In case one cannot be certain whether the definitions file |childdoc.def|
is installed on the target \TeX{} distribution
and one prefers not to ship it,
it is conceivable to paste a few relevant commands into the sources.

To that end, drop all statements |\input{childdoc.def}|
and perform the replacements as outlined below.
Instead of |\childdocmain{|\textit{main}|}| add the following code
to the top of the main file:
%
\begin{center}
\begin{tabular}{l}
|\||ifdefined\childdocname\endinput\||fi\newif\ifchilddoc|\\
|\edef\childdocname{\scantokens\expandafter{\jobname\noexpand}}|\\
|\def\childdocmain{|\textit{main}|}\||ifx\childdocmain\childdocname\||else|\\
|\childdoctrue\includeonly{\childdocname}\let\jobname\childdocmain\||fi|\\
\end{tabular}
\end{center}
%
Instead of |\childdocof{|\textit{main}|}| just include the main file
at the top of each child file:
%
\begin{center}
|\input{|\textit{main}|}|
\end{center}
%
A simple redirection |\childdocforward{|\textit{dest}|}| is achieved by:
%
\begin{center}
|\def\jobname{|\textit{dest}|}\input{\jobname}|
\end{center}
%
The redirection with prefix
|\childdocforwardprefix[|\textit{prefix}|]{|\textit{dest}|}|
is accomplished by:
%
\begin{center}
\begin{tabular}{l}
|{\edef\jobname{\scantokens\expandafter{\jobname\noexpand}}|\\
|\def\redirectjob |\textit{prefix}|#1~~~{\gdef\jobname{|\textit{dest}|#1}}|\\
|\expandafter\redirectjob\jobname~~~}\input{\jobname}|
\end{tabular}
\end{center}

In an alternative approach,
child documents can be compiled by a specific command line
without additional code or specific definitions:
%
\begin{center}
|... -jobname "|\textit{target}|" "|[\textit{flags}]%
|\includeonly{|\textit{dest}|}\input{|\textit{main}|}"|
\end{center}
%

%%%%%%%%%%%%%%%%%%%%%%%%%%%%%%%%%%%%%%%%%%%%%%%%%%%%%%%%%%%%%%%%%%%%%%%%%%%%%%%%
%%%%%%%%%%%%%%%%%%%%%%%%%%%%%%%%%%%%%%%%%%%%%%%%%%%%%%%%%%%%%%%%%%%%%%%%%%%%%%%%
\section{Information}

%%%%%%%%%%%%%%%%%%%%%%%%%%%%%%%%%%%%%%%%%%%%%%%%%%%%%%%%%%%%%%%%%%%%%%%%%%%%%%%%
\subsection{Copyright}

Copyright \copyright{} 2017--2018 Niklas Beisert

This work may be distributed and/or modified under the
conditions of the \LaTeX{} Project Public License, either version 1.3
of this license or (at your option) any later version.
The latest version of this license is in
  \url{http://www.latex-project.org/lppl.txt}
and version 1.3 or later is part of all distributions of \LaTeX{}
version 2005/12/01 or later.

This work has the LPPL maintenance status `maintained'.

The Current Maintainer of this work is Niklas Beisert.

This work consists of the files |README.txt|, |childdoc.ins| and |childdoc.dtx|
as well as the derived files |childdoc.def|, |cdocsamp.tex|
with |cdocsch1.tex|, |cdocsch2.tex|, |cdocspt3.tex|, |cdocspt4.tex|,
|cdocsdrf.tex|, |cdocsfn1.tex|, |cdocsfn2.tex|
as well as |childdoc.pdf|.

%%%%%%%%%%%%%%%%%%%%%%%%%%%%%%%%%%%%%%%%%%%%%%%%%%%%%%%%%%%%%%%%%%%%%%%%%%%%%%%%
\subsection{Files and Installation}

The package consists of the files:
%
\begin{center}
\begin{tabular}{ll}
    |README.txt|   & readme file \\
    |childdoc.ins| & installation file \\
    |childdoc.dtx| & source file \\
    |childdoc.def| & definition file \\
    |cdocsamp.tex| & sample main file \\
    |cdocsch1.tex| & sample include file \\
    |cdocsch2.tex| & sample include file \\
    |cdocspt3.tex| & sample part file \\
    |cdocspt4.tex| & sample part file \\
    |cdocsdrf.tex| & sample redirection file \\
    |cdocsfn1.tex| & sample redirection file \\
    |cdocsfn2.tex| & sample redirection file \\
    |childdoc.pdf| & manual
\end{tabular}
\end{center}
%
The distribution consists of the files
|README.txt|, |childdoc.ins| and |childdoc.dtx|.
%
\begin{itemize}
\item
Run (pdf)\LaTeX{} on |childdoc.dtx|
to compile the manual |childdoc.pdf| (this file).
\item
Run \LaTeX{} on |childdoc.ins| to create the definitions file |childdoc.def|
and the sample |cdocsamp.tex| with include files
|cdocsch1.tex|, |cdocsch2.tex|, |cdocspt3.tex|, |cdocspt4.tex|,
|cdocsdrf.tex|, |cdocsfn1.tex|, |cdocsfn2.tex|.
Then copy the file |childdoc.def| to an appropriate directory of your \LaTeX{}
distribution, e.g.\ \textit{texmf-root}|/tex/latex/childdoc|.
\end{itemize}

%%%%%%%%%%%%%%%%%%%%%%%%%%%%%%%%%%%%%%%%%%%%%%%%%%%%%%%%%%%%%%%%%%%%%%%%%%%%%%%%
\subsection{Related CTAN Packages}

There are several other packages which offer a similar functionality:
%
\begin{itemize}
\item
The packages
\href{http://ctan.org/pkg/docmute}{\textsf{docmute}},
\href{http://ctan.org/pkg/includex}{\textsf{includex}} and
\href{http://ctan.org/pkg/standalone}{\textsf{standalone}}
provide commands to include only the document body of
a child file thus allowing both files to be compiled individually.
\item
The packages \href{http://ctan.org/pkg/subdocs}{\textsf{subdocs}}
and \href{http://ctan.org/pkg/subfiles}{\textsf{subfiles}}
provide structures in which the main and child documents can be
encapsulated and allowing them to be compiled individually.
The inclusion mechanism is different from the conventional |\include|.
\item
The package \href{http://ctan.org/pkg/combine}{\textsf{combine}}
is an elaborate solution to combine several documents into one.
\end{itemize}
%
See also the CTAN topic \href{http://ctan.org/topic/subdocs}{\textsf{subdocs}}
for further related packages.
The present package differs from the above solutions in that
a document structure constructed with the conventional |\include| mechanism
just needs two extra commands at the top of every file
such that all constituent files can be compiled individually.

%%%%%%%%%%%%%%%%%%%%%%%%%%%%%%%%%%%%%%%%%%%%%%%%%%%%%%%%%%%%%%%%%%%%%%%%%%%%%%%%
%\subsection{Feature Suggestions}
%
%The following is a list of features which may be useful for future
%versions of this package:
%%
%\begin{itemize}
%\item
%\ldots
%\end{itemize}

%%%%%%%%%%%%%%%%%%%%%%%%%%%%%%%%%%%%%%%%%%%%%%%%%%%%%%%%%%%%%%%%%%%%%%%%%%%%%%%%
\subsection{Revision History}

%%%%%%%%%%%%%%%%%%%%%%%%%%%%%%%%%%%%%%%%
\paragraph{v2.0:} 2018/12/30

\begin{itemize}
\item
immediate forward processing
\item
added |\childdocby| mechanism
\item
manual restructured
\end{itemize}

%%%%%%%%%%%%%%%%%%%%%%%%%%%%%%%%%%%%%%%%
\paragraph{v1.6:} 2018/01/17

\begin{itemize}
\item
application for development of include files
\item
corrections to manual
\end{itemize}

%%%%%%%%%%%%%%%%%%%%%%%%%%%%%%%%%%%%%%%%
\paragraph{v1.5:} 2017/05/21

\begin{itemize}
\item
more complete structuring introduced
\item
|\childdocof| introduced
\item
|\childdoc| renamed to |\childdocmain|
\item
|\childredirect| renamed to |\childdocforward| and |\childdocforwardprefix|
and functionality expanded
\end{itemize}

%%%%%%%%%%%%%%%%%%%%%%%%%%%%%%%%%%%%%%%%
\paragraph{v1.0:} 2017/04/27

\begin{itemize}
\item
manual and install package
\item
first version published on CTAN
\end{itemize}

%%%%%%%%%%%%%%%%%%%%%%%%%%%%%%%%%%%%%%%%
\paragraph{v0.6:} 2017/04/26

\begin{itemize}
\item
redirection mechanism added
\end{itemize}

%%%%%%%%%%%%%%%%%%%%%%%%%%%%%%%%%%%%%%%%
\paragraph{v0.5:} 2017/04/26

\begin{itemize}
\item
functionality in definition file
\end{itemize}


%%%%%%%%%%%%%%%%%%%%%%%%%%%%%%%%%%%%%%%%%%%%%%%%%%%%%%%%%%%%%%%%%%%%%%%%%%%%%%%%
%%%%%%%%%%%%%%%%%%%%%%%%%%%%%%%%%%%%%%%%%%%%%%%%%%%%%%%%%%%%%%%%%%%%%%%%%%%%%%%%
%%%%%%%%%%%%%%%%%%%%%%%%%%%%%%%%%%%%%%%%%%%%%%%%%%%%%%%%%%%%%%%%%%%%%%%%%%%%%%%%
\appendix

\settowidth\MacroIndent{\rmfamily\scriptsize 000\ }

 \DocInput{childdoc.dtx}

\end{document}
%</driver>
% \fi
%
% %%%%%%%%%%%%%%%%%%%%%%%%%%%%%%%%%%%%%%%%%%%%%%%%%%%%%%%%%%%%%%%%%%%%%%%%%%%%%%
% %%%%%%%%%%%%%%%%%%%%%%%%%%%%%%%%%%%%%%%%%%%%%%%%%%%%%%%%%%%%%%%%%%%%%%%%%%%%%%
% \section{Sample}
%\iffalse
%<*samplemain>
%\fi
%
% The following presents a sample document
% with two chapters, two parts, a title page,
% a compile flag as well as three forwarding files to set the flag.
% It consists of eight |.tex| files:
% \begin{center}
% \begin{tabular}{ll}
% |cdocsamp.tex|&main file\\
% |cdocsch1.tex|&include file for chapter 1\\
% |cdocsch2.tex|&include file for chapter 2\\
% |cdocspt3.tex|&include file for part 3\\
% |cdocspt4.tex|&include file for part 4\\
% |cdocsdrf.tex|&forwarding file for main file in draft mode\\
% |cdocsfi1.tex|&forwarding file for final version of chapter 1\\
% |cdocsfi2.tex|&forwarding file for final version of chapter 2\\
% \end{tabular}
% \end{center}
% Each of the eight files can be compiled directly by the \LaTeX{} compiler.
%
% %%%%%%%%%%%%%%%%%%%%%%%%%%%%%%%%%%%%%%
% \paragraph{Main File.}
%
% The main file is called |cdocsamp.tex|.
%
% Load the \textsf{childdoc} definitions and
% declare the filename for the main document:
%    \begin{macrocode}
\input{childdoc.def}
\childdocmain{}
%    \end{macrocode}

% Optional override for |\version| flag:
%    \begin{macrocode}
%%\ifchilddoc\else\providecommand{\version}{draft}\fi
%    \end{macrocode}

% Define the default values for the |\version| flag
% (|final| for the main file and |draft| for childs):
%    \begin{macrocode}
\ifchilddoc
\providecommand{\version}{draft}
\else
\providecommand{\version}{final}
\fi
%    \end{macrocode}

% Load the standard document class:
%    \begin{macrocode}
\documentclass[12pt]{article}
%    \end{macrocode}

% Start the document body:
%    \begin{macrocode}
\begin{document}
%    \end{macrocode}

% Declare a title page.
% Print title, part of document being processed and version flag:
%    \begin{macrocode}
\addtocounter{page}{-1}
\begin{center}
{\LARGE\bfseries{}childdoc example\par}
\vspace{1cm}
\ifchilddoc
\ifchilddocmanual part\else chapter\fi:
`\childdocname' of `\childdocjob'\par
\else
main document: `\childdocjob'\par
\fi
version: \version\par
\end{center}
\newpage
%    \end{macrocode}

% Manually include selected file,
% otherwise process as usual:
%    \begin{macrocode}
\ifchilddocmanual
\section*{part `\childdocname'}
\input{\childdocname}
\else
%    \end{macrocode}

% Include the two chapters:
%    \begin{macrocode}
\include{cdocsch1}
\include{cdocsch2}
%    \end{macrocode}

% Include the two parts unless only chapters should be displayed:
%    \begin{macrocode}
\ifchilddoc\else
\section{part three}
\input{cdocspt3}
\section{part four}
\input{cdocspt4}
\fi
%    \end{macrocode}

% Process as usual until here:
%    \begin{macrocode}
\fi
%    \end{macrocode}

% End of document body:
%    \begin{macrocode}
\end{document}
%    \end{macrocode}
%\iffalse
%</samplemain>
%\fi
%
% %%%%%%%%%%%%%%%%%%%%%%%%%%%%%%%%%%%%%%
% \paragraph{Chapter Include Files.}
%
% The include files are called |cdocsch1.tex| and |cdocsch2.tex|.
%
%\iffalse
%<*samplechap1|samplechap2>
%\fi

% Optional override for |\version| flag:
%    \begin{macrocode}
%%\providecommand{\version}{final}
%    \end{macrocode}

% Include the main document:
%    \begin{macrocode}
\input{childdoc.def}
\childdocof{cdocsamp}
%    \end{macrocode}

%\iffalse
%</samplechap1|samplechap2>
%\fi
%
%\iffalse
%<*samplechap1>
%\fi
% Some text for chapter 1:
%    \begin{macrocode}
\section{one}
some text in chapter one
%    \end{macrocode}

%\iffalse
%</samplechap1>
%\fi
% Some text for chapter 2:
%\iffalse
%<*samplechap2>
%\fi
%    \begin{macrocode}
\section{two}
more text in chapter two
%    \end{macrocode}

%\iffalse
%</samplechap2>
%\fi
%
% %%%%%%%%%%%%%%%%%%%%%%%%%%%%%%%%%%%%%%
% \paragraph{Part Include Files.}
%
% The include files are called |cdocspt3.tex| and |cdocspt4.tex|.
%
%\iffalse
%<*samplepart3|samplepart4>
%\fi

% Optional override for |\version| flag:
%    \begin{macrocode}
%%\providecommand{\version}{final}
%    \end{macrocode}

% Include the main document:
%    \begin{macrocode}
\input{childdoc.def}
\childdocby{cdocsamp}
%    \end{macrocode}

%\iffalse
%</samplepart3|samplepart4>
%\fi
%
%\iffalse
%<*samplepart3>
%\fi
% Some text for part 3:
%    \begin{macrocode}
some text in part three
%    \end{macrocode}

%\iffalse
%</samplepart3>
%\fi
% Some text for part 4:
%\iffalse
%<*samplepart4>
%\fi
%    \begin{macrocode}
more text in part four
%    \end{macrocode}

%\iffalse
%</samplepart4>
%\fi
%
% %%%%%%%%%%%%%%%%%%%%%%%%%%%%%%%%%%%%%%
% \paragraph{Forwarding for a Complete Draft.}
%
% The following forwarding file |cdocsdrf.tex|
% compiles the main document in draft mode:
%\iffalse
%<*sampledraft>
%\fi
%    \begin{macrocode}
\def\version{draft}
\input{childdoc.def}
\childdocforward{cdocsamp}
%    \end{macrocode}

%\iffalse
%</sampledraft>
%\fi
%
% %%%%%%%%%%%%%%%%%%%%%%%%%%%%%%%%%%%%%%
% \paragraph{Forwarding for Final Version of the Chapters.}
%
% The following forwarding files |cdocsfn1.tex| and |cdocsfn2.tex|
% (with identical content)
% compile the final versions of the child documents
% |cdocsch1.tex| and |cdocsch2.tex|, respectively:
%\iffalse
%<*samplefinal>
%\fi
%    \begin{macrocode}
\def\version{final}
\input{childdoc.def}
\childdocforwardprefix[cdocsamp]{cdocsfn}{cdocsch}
%    \end{macrocode}

%\iffalse
%</samplefinal>
%\fi
%
% %%%%%%%%%%%%%%%%%%%%%%%%%%%%%%%%%%%%%%
% \paragraph{Command Line Processing.}
%
% The following three command lines generate the output files
% |cdocscld|, |cdocscl1| and |cdocscl2|
% which should be identical to
% |cdocsdrf|, |cdocsch1| and |cdocsfn2|, respectively:
% \begin{center}
% \begin{tabular}{l}
% |latex -jobname cdocscld \|\\
% |  "\def\version{draft}\input{childdoc.def}\childdocforward{cdocsamp}"|\\
% |latex -jobname cdocscl1 \|\\
% |  "\input{childdoc.def}\childdocforward[cdocsamp]{cdocsch1}"|\\
% |latex -jobname cdocscl2 \|\\
% |  "\def\version{final}\input{childdoc.def}\childdocforward{cdocsch2}"|
% \end{tabular}
% \end{center}
% Note that the trailing backslash on each first line
% merely continues the input to the second line
% (for convenient cut ant paste).
% Furthermore, the command |latex| can be replaced by any
% of its alternative versions such as |pdflatex|.
%
% %%%%%%%%%%%%%%%%%%%%%%%%%%%%%%%%%%%%%%%%%%%%%%%%%%%%%%%%%%%%%%%%%%%%%%%%%%%%%%
% %%%%%%%%%%%%%%%%%%%%%%%%%%%%%%%%%%%%%%%%%%%%%%%%%%%%%%%%%%%%%%%%%%%%%%%%%%%%%%
% \section{Implementation}
%\iffalse
%<*package>
%\fi
%
% This section describes the definitions file |childdoc.def|.

% The definitions cannot be loaded using |\usepackage| or |\RequirePackage|
% which has a mechanism to prevent loading a style file more than once.
% When loading the definitions by means of |\input|
% multiple instances have to be prevented manually:
%\iffalse
%This code needs to be before the `\ProvidesFile' directive
%which is defined at the beginning of this file.
%Therefore it is also placed there and commented out here.
%</package>
%<*discard>
%\fi
%    \begin{macrocode}
\ifdefined\childdocmain\endinput\fi
%    \end{macrocode}
%\iffalse
%</discard>
%<*package>
%\fi
%
% \macro{\ifchilddoc}
% \macro{\ifchilddocmanual}
% The conditional |\ifchilddoc| tells whether a
% child (true) or main (false) document is being compiled.
% The conditional |\ifchilddocmanual| tells whether
% the |\includeonly| mechanism is used (false) or
% the selection of child files must be performed manually (true).
% The definitions initialise to false:
%    \begin{macrocode}
\newif\ifchilddoc
\newif\ifchilddocmanual
%    \end{macrocode}

% \macro{\childdocname}
% \macro{\childdocjob}
% The macro |\childdocname| stores the name of the main document
% to be compiled. The macro |\childdocjob| stores the name of
% the document on which the \LaTeX{} compiler was originally invoked.
% The content of |\jobname| cannot be compared
% to filenames specified in the source due to different catcodes.
% The following code rescans |\jobname|, stores the result
% in |\childdocname| and saves a copy in |\childdocjob|:
%    \begin{macrocode}
\edef\childdocname{\scantokens\expandafter{\jobname\noexpand}}
\let\childdocjob\childdocname
%    \end{macrocode}

% \macro{\childdocdisable}
% The macro |\childdocdisable| prevents the main file
% from being processed more than once.
% At this stage, the main document command |\childdocmain|
% is assumed to be called once again where it should do nothing.
% Any subsequent call to it should prevent
% a secondary processing of the main document
% It overwrites the forwarding commands
% |\childdocof| and |\childdocforward|
% with empty macros to prevent further inclusions of the main document:
%    \begin{macrocode}
\newcommand{\childdocdisable}
{
  \renewcommand{\childdocmain}[1]{\renewcommand{\childdocmain}[1]{\endinput}}
  \renewcommand{\childdocof}[1]{}
  \renewcommand{\childdocby}[2][]{}
  \renewcommand{\childdocforward}[2][]{}
  \renewcommand{\childdocdisable}{}
}
%    \end{macrocode}

% \macro{\childdocmain}
% The macro |\childdocmain| is to be called at the top of the main file
% with nothing or the main filename (without extension) as argument.
% First, it breaks loops.
% If the argument is not empty and does not match |\childdocname|
% (which is set by the first inclusion of |childdoc.def|),
% |\ifchilddoc| is set to true, |\includeonly| is applied to the child file
% and |\jobname| is set to the main file
% (for proper handling of |.aux| files):
%    \begin{macrocode}
\newcommand{\childdocmain}[1]
{
  \childdocdisable\childdocmain{}
  \if?#1?\else
    \begingroup
      \def\childdoctmp{#1}
      \ifx\childdoctmp\childdocname
        \def\childdoctmp{}
      \else
        \def\childdoctmp
        {
          \childdoctrue
          \includeonly{\childdocname}
          \def\childdocjob{#1}
          \def\jobname{#1}
        }
      \fi
      \expandafter
    \endgroup
    \childdoctmp
  \fi
}
%    \end{macrocode}

% \macro{\childdocof}
% The command |\childdocof| redirects
% compilation to the main file |#1|.
%    \begin{macrocode}
\newcommand{\childdocof}[1]
{
  \childdocdisable
  \childdoctrue
  \includeonly{\childdocname}
  \def\jobname{#1}
  \def\childdocjob{#1}
  \input{#1}
}
%    \end{macrocode}

% \macro{\childdocby}
% The command |\childdocby| ....
%    \begin{macrocode}
\newcommand{\childdocby}[2][]
{
  \childdocdisable
  \childdoctrue
  \childdocmanualtrue
  \if?#1?\else
    \def\jobname{#2}
  \fi
  \def\childdocjob{#2}
  \input{#2}
  \endinput
}
%    \end{macrocode}

% \macro{\childdocforward}
% The command |\childdocforward| redirects
% compilation to the main file or
% (if the optional argument is given) a child file.
% Parameters are set as if the main file
% or a child file starting with |\childdocof| was compiled.
% Then compilation is handed over to the main file:
%    \begin{macrocode}
\newcommand{\childdocforward}[2][]
{
  \begingroup
    \if?#1?
      \def\childdoctmp
      {
        \def\childdocname{#2}
        \def\childdocjob{#2}
        \def\jobname{#2}
        \input{#2}
        \endinput
      }
    \else
      \def\childdoctmp
      {
        \childdocdisable
        \def\childdocname{#2}
        \childdoctrue
        \includeonly{#2}
        \def\childdocjob{#1}
        \def\jobname{#1}
        \input{#1}
        \endinput
      }
    \fi
    \expandafter
  \endgroup
  \childdoctmp
}
%    \end{macrocode}

% \macro{\childdocforwardprefix}
% The command |\childdocforwardprefix| redirects
% compilation to the main or a child file by means of a pattern.
% The prefix |#1| in the current filename is replaced by |#2|
% and the suffix of the current filename is kept
% (it is assumed that the filename does not contain the substring `|~~~|'
% which is used as a delimiter).
% Compilation is handed over to the new file by |\childdocforward|:
%    \begin{macrocode}
\newcommand{\childdocforwardprefix}[3][]
{
  \begingroup
    \def\childdocextract #2##1~~~{\def\childdoctmp{\childdocforward[#1]{#3##1}}}
    \expandafter\childdocextract\childdocname~~~
    \expandafter
  \endgroup
  \childdoctmp
}
%    \end{macrocode}

% \macro{\childdoc}
% The deprecated macro |\childdoc| is a legacy version of |\childdocmain|:
%    \begin{macrocode}
\newcommand{\childdoc}{\childdocmain}
%    \end{macrocode}

% \macro{\childdocredirect}
% The deprecated macro |\childdocredirect| is a legacy version
% of |\childdocforward| and |\childdocforwardprefix|:
%    \begin{macrocode}
\newcommand{\childdocredirect}[2][]
{
  \begingroup
    \if?#1?
      \def\childdoctmp{\childdocforward{#2}}
    \else
      \def\childdoctmp{\childdocforwardprefix{#1}{#2}}
    \fi
    \expandafter
  \endgroup
  \childdoctmp
}
%    \end{macrocode}

%\iffalse
%</package>
%\fi
%
\endinput
\childdocforward{cdocsch2}"|
% \end{tabular}
% \end{center}
% Note that the trailing backslash on each first line
% merely continues the input to the second line
% (for convenient cut ant paste).
% Furthermore, the command |latex| can be replaced by any
% of its alternative versions such as |pdflatex|.
%
% %%%%%%%%%%%%%%%%%%%%%%%%%%%%%%%%%%%%%%%%%%%%%%%%%%%%%%%%%%%%%%%%%%%%%%%%%%%%%%
% %%%%%%%%%%%%%%%%%%%%%%%%%%%%%%%%%%%%%%%%%%%%%%%%%%%%%%%%%%%%%%%%%%%%%%%%%%%%%%
% \section{Implementation}
%\iffalse
%<*package>
%\fi
%
% This section describes the definitions file |childdoc.def|.

% The definitions cannot be loaded using |\usepackage| or |\RequirePackage|
% which has a mechanism to prevent loading a style file more than once.
% When loading the definitions by means of |\input|
% multiple instances have to be prevented manually:
%\iffalse
%This code needs to be before the `\ProvidesFile' directive
%which is defined at the beginning of this file.
%Therefore it is also placed there and commented out here.
%</package>
%<*discard>
%\fi
%    \begin{macrocode}
\ifdefined\childdocmain\endinput\fi
%    \end{macrocode}
%\iffalse
%</discard>
%<*package>
%\fi
%
% \macro{\ifchilddoc}
% \macro{\ifchilddocmanual}
% The conditional |\ifchilddoc| tells whether a
% child (true) or main (false) document is being compiled.
% The conditional |\ifchilddocmanual| tells whether
% the |\includeonly| mechanism is used (false) or
% the selection of child files must be performed manually (true).
% The definitions initialise to false:
%    \begin{macrocode}
\newif\ifchilddoc
\newif\ifchilddocmanual
%    \end{macrocode}

% \macro{\childdocname}
% \macro{\childdocjob}
% The macro |\childdocname| stores the name of the main document
% to be compiled. The macro |\childdocjob| stores the name of
% the document on which the \LaTeX{} compiler was originally invoked.
% The content of |\jobname| cannot be compared
% to filenames specified in the source due to different catcodes.
% The following code rescans |\jobname|, stores the result
% in |\childdocname| and saves a copy in |\childdocjob|:
%    \begin{macrocode}
\edef\childdocname{\scantokens\expandafter{\jobname\noexpand}}
\let\childdocjob\childdocname
%    \end{macrocode}

% \macro{\childdocdisable}
% The macro |\childdocdisable| prevents the main file
% from being processed more than once.
% At this stage, the main document command |\childdocmain|
% is assumed to be called once again where it should do nothing.
% Any subsequent call to it should prevent
% a secondary processing of the main document
% It overwrites the forwarding commands
% |\childdocof| and |\childdocforward|
% with empty macros to prevent further inclusions of the main document:
%    \begin{macrocode}
\newcommand{\childdocdisable}
{
  \renewcommand{\childdocmain}[1]{\renewcommand{\childdocmain}[1]{\endinput}}
  \renewcommand{\childdocof}[1]{}
  \renewcommand{\childdocby}[2][]{}
  \renewcommand{\childdocforward}[2][]{}
  \renewcommand{\childdocdisable}{}
}
%    \end{macrocode}

% \macro{\childdocmain}
% The macro |\childdocmain| is to be called at the top of the main file
% with nothing or the main filename (without extension) as argument.
% First, it breaks loops.
% If the argument is not empty and does not match |\childdocname|
% (which is set by the first inclusion of |childdoc.def|),
% |\ifchilddoc| is set to true, |\includeonly| is applied to the child file
% and |\jobname| is set to the main file
% (for proper handling of |.aux| files):
%    \begin{macrocode}
\newcommand{\childdocmain}[1]
{
  \childdocdisable\childdocmain{}
  \if?#1?\else
    \begingroup
      \def\childdoctmp{#1}
      \ifx\childdoctmp\childdocname
        \def\childdoctmp{}
      \else
        \def\childdoctmp
        {
          \childdoctrue
          \includeonly{\childdocname}
          \def\childdocjob{#1}
          \def\jobname{#1}
        }
      \fi
      \expandafter
    \endgroup
    \childdoctmp
  \fi
}
%    \end{macrocode}

% \macro{\childdocof}
% The command |\childdocof| redirects
% compilation to the main file |#1|.
%    \begin{macrocode}
\newcommand{\childdocof}[1]
{
  \childdocdisable
  \childdoctrue
  \includeonly{\childdocname}
  \def\jobname{#1}
  \def\childdocjob{#1}
  \input{#1}
}
%    \end{macrocode}

% \macro{\childdocby}
% The command |\childdocby| ....
%    \begin{macrocode}
\newcommand{\childdocby}[2][]
{
  \childdocdisable
  \childdoctrue
  \childdocmanualtrue
  \if?#1?\else
    \def\jobname{#2}
  \fi
  \def\childdocjob{#2}
  \input{#2}
  \endinput
}
%    \end{macrocode}

% \macro{\childdocforward}
% The command |\childdocforward| redirects
% compilation to the main file or
% (if the optional argument is given) a child file.
% Parameters are set as if the main file
% or a child file starting with |\childdocof| was compiled.
% Then compilation is handed over to the main file:
%    \begin{macrocode}
\newcommand{\childdocforward}[2][]
{
  \begingroup
    \if?#1?
      \def\childdoctmp
      {
        \def\childdocname{#2}
        \def\childdocjob{#2}
        \def\jobname{#2}
        \input{#2}
        \endinput
      }
    \else
      \def\childdoctmp
      {
        \childdocdisable
        \def\childdocname{#2}
        \childdoctrue
        \includeonly{#2}
        \def\childdocjob{#1}
        \def\jobname{#1}
        \input{#1}
        \endinput
      }
    \fi
    \expandafter
  \endgroup
  \childdoctmp
}
%    \end{macrocode}

% \macro{\childdocforwardprefix}
% The command |\childdocforwardprefix| redirects
% compilation to the main or a child file by means of a pattern.
% The prefix |#1| in the current filename is replaced by |#2|
% and the suffix of the current filename is kept
% (it is assumed that the filename does not contain the substring `|~~~|'
% which is used as a delimiter).
% Compilation is handed over to the new file by |\childdocforward|:
%    \begin{macrocode}
\newcommand{\childdocforwardprefix}[3][]
{
  \begingroup
    \def\childdocextract #2##1~~~{\def\childdoctmp{\childdocforward[#1]{#3##1}}}
    \expandafter\childdocextract\childdocname~~~
    \expandafter
  \endgroup
  \childdoctmp
}
%    \end{macrocode}

% \macro{\childdoc}
% The deprecated macro |\childdoc| is a legacy version of |\childdocmain|:
%    \begin{macrocode}
\newcommand{\childdoc}{\childdocmain}
%    \end{macrocode}

% \macro{\childdocredirect}
% The deprecated macro |\childdocredirect| is a legacy version
% of |\childdocforward| and |\childdocforwardprefix|:
%    \begin{macrocode}
\newcommand{\childdocredirect}[2][]
{
  \begingroup
    \if?#1?
      \def\childdoctmp{\childdocforward{#2}}
    \else
      \def\childdoctmp{\childdocforwardprefix{#1}{#2}}
    \fi
    \expandafter
  \endgroup
  \childdoctmp
}
%    \end{macrocode}

%\iffalse
%</package>
%\fi
%
\endinput
|\\
|\childdocmain{}|\\
\end{tabular}
\end{center}
at the very top of the main \LaTeX{} file,
in particular \emph{before} the |\documentclass| statement!
The argument of |\childdocmain| should be left empty
(but it must be present).

%%%%%%%%%%%%%%%%%%%%%%%%%%%%%%%%%%%%%%%%
\DescribeMacro{\childdocof}
Furthermore, add the commands
\begin{center}
\begin{tabular}{l}
|% \iffalse
%
% childdoc.dtx Copyright (C) 2017-2018 Niklas Beisert
%
% This work may be distributed and/or modified under the
% conditions of the LaTeX Project Public License, either version 1.3
% of this license or (at your option) any later version.
% The latest version of this license is in
%   http://www.latex-project.org/lppl.txt
% and version 1.3 or later is part of all distributions of LaTeX
% version 2005/12/01 or later.
%
% This work has the LPPL maintenance status `maintained'.
%
% The Current Maintainer of this work is Niklas Beisert.
%
% This work consists of the files childdoc.dtx and childdoc.ins
% and the derived files childdoc.def and cdocsamp.tex with
% cdocsch1.tex, cdocsch2.tex, cdocsdrf.tex, cdocsfn1.tex, cdocsfn2.tex.
%
%<package>\ifdefined\childdocmain\endinput\fi
%<package>\ProvidesFile{childdoc.def}[2018/12/30 v2.0 child document driver]
%<samplemain>\ProvidesFile{cdocsamp.tex}[2018/12/30 v2.0 sample for childdoc]
%<*driver>
%\ProvidesFile{childdoc.drv}[2018/12/30 v2.0 childdoc reference manual file]
\PassOptionsToClass{10pt,a4paper}{article}
\documentclass{ltxdoc}

\usepackage[margin=35mm]{geometry}
\usepackage{hyperref}
\usepackage{hyperxmp}
\usepackage[usenames]{color}

\hypersetup{colorlinks=true}
\hypersetup{pdfstartview=FitH}
\hypersetup{pdfpagemode=UseNone}
\hypersetup{pdfsource={}}
\hypersetup{pdflang={en-UK}}
\hypersetup{pdfcopyright={Copyright 2017-2018 Niklas Beisert.
  This work may be distributed and/or modified under the
  conditions of the LaTeX Project Public License, either version 1.3
  of this license or (at your option) any later version.}}
\hypersetup{pdflicenseurl={http://www.latex-project.org/lppl.txt}}
\hypersetup{pdfcontactaddress={ETH Zurich, ITP, HIT K,
  Wolfgang-Pauli-Strasse 27}}
\hypersetup{pdfcontactpostcode={8093}}
\hypersetup{pdfcontactcity={Zurich}}
\hypersetup{pdfcontactcountry={Switzerland}}
\hypersetup{pdfcontactemail={nbeisert@itp.phys.ethz.ch}}
\hypersetup{pdfcontacturl={http://people.phys.ethz.ch/\xmptilde nbeisert/}}

\newcommand{\secref}[1]{\hyperref[#1]{section \ref*{#1}}}

\parskip1ex
\parindent0pt
\let\olditemize\itemize
\def\itemize{\olditemize\parskip0pt}

\begin{document}

\title{The \textsf{childdoc} Package}
\hypersetup{pdftitle={The childdoc Package}}
\author{Niklas Beisert\\[2ex]
  Institut f\"ur Theoretische Physik\\
  Eidgen\"ossische Technische Hochschule Z\"urich\\
  Wolfgang-Pauli-Strasse 27, 8093 Z\"urich, Switzerland\\[1ex]
  \href{mailto:nbeisert@itp.phys.ethz.ch}
  {\texttt{nbeisert@itp.phys.ethz.ch}}}
\hypersetup{pdfauthor={Niklas Beisert}}
\hypersetup{pdfsubject={Manual for the LaTeX2e Package childdoc}}
\date{30 December 2018, \textsf{v2.0}}
\maketitle

\begin{abstract}\noindent
\textsf{childdoc} is a \LaTeXe{} package
that enables the direct compilation
of document sections included by |\include|
to individual files.
\end{abstract}

\begingroup
\parskip0ex
\tableofcontents
\endgroup

%%%%%%%%%%%%%%%%%%%%%%%%%%%%%%%%%%%%%%%%%%%%%%%%%%%%%%%%%%%%%%%%%%%%%%%%%%%%%%%%
%%%%%%%%%%%%%%%%%%%%%%%%%%%%%%%%%%%%%%%%%%%%%%%%%%%%%%%%%%%%%%%%%%%%%%%%%%%%%%%%
\section{Introduction}

\LaTeX{} provides a mechanism to structure a large document (such as a book)
into a main file and several child files (containing the chapters)
using the |\include| command.
This mechanism is beneficial for documents
which span hundreds of pages in order to
make the source file(s) more manageable.
Moreover, compilation can be restricted to
selected child files by means of the |\includeonly| command.
The latter feature can be used to reduce the compilation time while editing
(this was significantly more useful in the earlier days of \LaTeX{})
or to generate a smaller document which is easier to navigate.
Another application of |\includeonly| is to generate
documents consisting of selected parts of the complete document.

However, there are a few drawbacks of the plain |\include| mechanism:
\begin{itemize}
\item
The child files cannot be compiled on their own,
they can only be compiled via the main file.
A naive editing environment
(such as a text editor with an option
to have the current file processed by \LaTeX)
may require one to switch to the main file before compiling;
attempting to compile the child file produces errors.
\item
The main file must be modified (each time)
to adjust the |\includeonly| command
to the present needs. This easily leaves the main file in a messy state.
\item
The generated document will always carry the filename
of the main document. This is inconvenient if
several child files are to be compiled and
to be kept for distribution.
\end{itemize}

The present package provides a simple interface
to make child files individually compilable by \LaTeX{}.
Compiling a child file then has the same effect as compiling
the main file with an |\includeonly| command
to select the appropriate child.
Moreover the generated document will carry the name of the child
rather than the main file.
This resolves all three above issues.

This feature is meant to make the editing of books,
thesis documents and lecture notes somewhat more convenient.
However, the package can also be used efficiently for
composing a series of documents (such as exercise sheets)
which are typically distributed individually.
It then assists the author in generating the individual documents
(potentially in different versions)
as well as a document containing the collected series.
Another application is in developing style files
or other kinds of included material
where compilation of the style file could redirect
to a sample or test file.

%%%%%%%%%%%%%%%%%%%%%%%%%%%%%%%%%%%%%%%%%%%%%%%%%%%%%%%%%%%%%%%%%%%%%%%%%%%%%%%%
%%%%%%%%%%%%%%%%%%%%%%%%%%%%%%%%%%%%%%%%%%%%%%%%%%%%%%%%%%%%%%%%%%%%%%%%%%%%%%%%
\section{Usage}

First of all, the package \textsf{childdoc} is \emph{not} a standard
\LaTeXe{} |.sty| style file! Therefore it needs to be invoked in
a non-standard way.

%%%%%%%%%%%%%%%%%%%%%%%%%%%%%%%%%%%%%%%%%%%%%%%%%%%%%%%%%%%%%%%%%%%%%%%%%%%%%%%%
\subsection{Included Files}
\label{sec:include}

%%%%%%%%%%%%%%%%%%%%%%%%%%%%%%%%%%%%%%%%
\DescribeMacro{\childdocmain}
To use the package, add the commands
\begin{center}
\begin{tabular}{l}
|% \iffalse
%
% childdoc.dtx Copyright (C) 2017-2018 Niklas Beisert
%
% This work may be distributed and/or modified under the
% conditions of the LaTeX Project Public License, either version 1.3
% of this license or (at your option) any later version.
% The latest version of this license is in
%   http://www.latex-project.org/lppl.txt
% and version 1.3 or later is part of all distributions of LaTeX
% version 2005/12/01 or later.
%
% This work has the LPPL maintenance status `maintained'.
%
% The Current Maintainer of this work is Niklas Beisert.
%
% This work consists of the files childdoc.dtx and childdoc.ins
% and the derived files childdoc.def and cdocsamp.tex with
% cdocsch1.tex, cdocsch2.tex, cdocsdrf.tex, cdocsfn1.tex, cdocsfn2.tex.
%
%<package>\ifdefined\childdocmain\endinput\fi
%<package>\ProvidesFile{childdoc.def}[2018/12/30 v2.0 child document driver]
%<samplemain>\ProvidesFile{cdocsamp.tex}[2018/12/30 v2.0 sample for childdoc]
%<*driver>
%\ProvidesFile{childdoc.drv}[2018/12/30 v2.0 childdoc reference manual file]
\PassOptionsToClass{10pt,a4paper}{article}
\documentclass{ltxdoc}

\usepackage[margin=35mm]{geometry}
\usepackage{hyperref}
\usepackage{hyperxmp}
\usepackage[usenames]{color}

\hypersetup{colorlinks=true}
\hypersetup{pdfstartview=FitH}
\hypersetup{pdfpagemode=UseNone}
\hypersetup{pdfsource={}}
\hypersetup{pdflang={en-UK}}
\hypersetup{pdfcopyright={Copyright 2017-2018 Niklas Beisert.
  This work may be distributed and/or modified under the
  conditions of the LaTeX Project Public License, either version 1.3
  of this license or (at your option) any later version.}}
\hypersetup{pdflicenseurl={http://www.latex-project.org/lppl.txt}}
\hypersetup{pdfcontactaddress={ETH Zurich, ITP, HIT K,
  Wolfgang-Pauli-Strasse 27}}
\hypersetup{pdfcontactpostcode={8093}}
\hypersetup{pdfcontactcity={Zurich}}
\hypersetup{pdfcontactcountry={Switzerland}}
\hypersetup{pdfcontactemail={nbeisert@itp.phys.ethz.ch}}
\hypersetup{pdfcontacturl={http://people.phys.ethz.ch/\xmptilde nbeisert/}}

\newcommand{\secref}[1]{\hyperref[#1]{section \ref*{#1}}}

\parskip1ex
\parindent0pt
\let\olditemize\itemize
\def\itemize{\olditemize\parskip0pt}

\begin{document}

\title{The \textsf{childdoc} Package}
\hypersetup{pdftitle={The childdoc Package}}
\author{Niklas Beisert\\[2ex]
  Institut f\"ur Theoretische Physik\\
  Eidgen\"ossische Technische Hochschule Z\"urich\\
  Wolfgang-Pauli-Strasse 27, 8093 Z\"urich, Switzerland\\[1ex]
  \href{mailto:nbeisert@itp.phys.ethz.ch}
  {\texttt{nbeisert@itp.phys.ethz.ch}}}
\hypersetup{pdfauthor={Niklas Beisert}}
\hypersetup{pdfsubject={Manual for the LaTeX2e Package childdoc}}
\date{30 December 2018, \textsf{v2.0}}
\maketitle

\begin{abstract}\noindent
\textsf{childdoc} is a \LaTeXe{} package
that enables the direct compilation
of document sections included by |\include|
to individual files.
\end{abstract}

\begingroup
\parskip0ex
\tableofcontents
\endgroup

%%%%%%%%%%%%%%%%%%%%%%%%%%%%%%%%%%%%%%%%%%%%%%%%%%%%%%%%%%%%%%%%%%%%%%%%%%%%%%%%
%%%%%%%%%%%%%%%%%%%%%%%%%%%%%%%%%%%%%%%%%%%%%%%%%%%%%%%%%%%%%%%%%%%%%%%%%%%%%%%%
\section{Introduction}

\LaTeX{} provides a mechanism to structure a large document (such as a book)
into a main file and several child files (containing the chapters)
using the |\include| command.
This mechanism is beneficial for documents
which span hundreds of pages in order to
make the source file(s) more manageable.
Moreover, compilation can be restricted to
selected child files by means of the |\includeonly| command.
The latter feature can be used to reduce the compilation time while editing
(this was significantly more useful in the earlier days of \LaTeX{})
or to generate a smaller document which is easier to navigate.
Another application of |\includeonly| is to generate
documents consisting of selected parts of the complete document.

However, there are a few drawbacks of the plain |\include| mechanism:
\begin{itemize}
\item
The child files cannot be compiled on their own,
they can only be compiled via the main file.
A naive editing environment
(such as a text editor with an option
to have the current file processed by \LaTeX)
may require one to switch to the main file before compiling;
attempting to compile the child file produces errors.
\item
The main file must be modified (each time)
to adjust the |\includeonly| command
to the present needs. This easily leaves the main file in a messy state.
\item
The generated document will always carry the filename
of the main document. This is inconvenient if
several child files are to be compiled and
to be kept for distribution.
\end{itemize}

The present package provides a simple interface
to make child files individually compilable by \LaTeX{}.
Compiling a child file then has the same effect as compiling
the main file with an |\includeonly| command
to select the appropriate child.
Moreover the generated document will carry the name of the child
rather than the main file.
This resolves all three above issues.

This feature is meant to make the editing of books,
thesis documents and lecture notes somewhat more convenient.
However, the package can also be used efficiently for
composing a series of documents (such as exercise sheets)
which are typically distributed individually.
It then assists the author in generating the individual documents
(potentially in different versions)
as well as a document containing the collected series.
Another application is in developing style files
or other kinds of included material
where compilation of the style file could redirect
to a sample or test file.

%%%%%%%%%%%%%%%%%%%%%%%%%%%%%%%%%%%%%%%%%%%%%%%%%%%%%%%%%%%%%%%%%%%%%%%%%%%%%%%%
%%%%%%%%%%%%%%%%%%%%%%%%%%%%%%%%%%%%%%%%%%%%%%%%%%%%%%%%%%%%%%%%%%%%%%%%%%%%%%%%
\section{Usage}

First of all, the package \textsf{childdoc} is \emph{not} a standard
\LaTeXe{} |.sty| style file! Therefore it needs to be invoked in
a non-standard way.

%%%%%%%%%%%%%%%%%%%%%%%%%%%%%%%%%%%%%%%%%%%%%%%%%%%%%%%%%%%%%%%%%%%%%%%%%%%%%%%%
\subsection{Included Files}
\label{sec:include}

%%%%%%%%%%%%%%%%%%%%%%%%%%%%%%%%%%%%%%%%
\DescribeMacro{\childdocmain}
To use the package, add the commands
\begin{center}
\begin{tabular}{l}
|\input{childdoc.def}|\\
|\childdocmain{}|\\
\end{tabular}
\end{center}
at the very top of the main \LaTeX{} file,
in particular \emph{before} the |\documentclass| statement!
The argument of |\childdocmain| should be left empty
(but it must be present).

%%%%%%%%%%%%%%%%%%%%%%%%%%%%%%%%%%%%%%%%
\DescribeMacro{\childdocof}
Furthermore, add the commands
\begin{center}
\begin{tabular}{l}
|\input{childdoc.def}|\\
|\childdocof{|\textit{main}|}|\\
\end{tabular}
\end{center}
at the top of every child file \textit{child}
which is included by |\include{|\textit{child}|}|
from within the main file
(or at least for those files to be compiled individually).
The argument \textit{main} must be the filename of the main file.

There are a couple of
considerations in setting up the main and child documents:

%%%%%%%%%%%%%%%%%%%%%%%%%%%%%%%%%%%%%%%%
\paragraph{Restrictions.}

Please note the following restrictions:
\begin{itemize}
\item
|\childdocmain| must be called with one argument \textit{main}
to ensure compatibility with earlier version of the package.
It must either be empty (|\childdocmain{}|)
or precisely match the filename of the main file in which it is specified.
See \secref{sec:detection} for further information.
\item
The filename \textit{main} must be specified without the |.tex| extension.
\item
The filename \textit{main} is case sensitive
(even in case-insensitive file systems)
due to internal string comparison.
\item
The argument \textit{main} should be fully expanded, it cannot be a macro.
\item
Subdirectories and special characters should be avoided in filenames.
\item
The command |\childdocmain{|\textit{main}|}| must be followed by a whitespace.
It should not be followed immediately by another command
or by a comment mark `|%|'.
This is because the \TeX{} parser reads the token immediately following
the argument of |\childdocmain| and puts it
at the beginning of every child section;
however, a white\-space is ignored.
\end{itemize}

%%%%%%%%%%%%%%%%%%%%%%%%%%%%%%%%%%%%%%%%
\paragraph{Content of Main File.}

It is advisable to place all content in the child files included by |\include|.
Any output contained in the main file will appear in all child documents
unless suppressed manually;
it cannot be suppressed automatically by the |\includeonly| directive
and thus should normally be avoided.
A method to include some content in the main file
by means of conditional processing is described in \secref{sec:conditional}.

%%%%%%%%%%%%%%%%%%%%%%%%%%%%%%%%%%%%%%%%
\paragraph{Page Numbering.}

When only a part of the document is compiled,
the appropriate numbering of pages
(as well as other status parameters)
is determined from the |.aux| files.
The latter contain information from previous passes.
However this information needs to propagate through
all intermediate child documents.
Therefore the page numbering in child documents may well
be inconsistent until the complete document is compiled at least once.

A useful (if unconventional) way to always ensure a consistent
page numbering is to restart the numbering in each child document
and denote the pages by `\textit{child}|.|\textit{page}'
where \textit{child} represents the chapter/section number of the child file.
This can be achieved by the command
|\numberwithin{page}{|\textit{child}|}|
of the \textsf{amsmath} package
where \textit{child} can be |chapter| or |section|
depending on the chosen structuring.
Alternatively, one can modify the macro |\thepage| appropriately
and reset the counter |page| at the start of each child file.

%%%%%%%%%%%%%%%%%%%%%%%%%%%%%%%%%%%%%%%%%%%%%%%%%%%%%%%%%%%%%%%%%%%%%%%%%%%%%%%%
\subsection{Conditional Processing}
\label{sec:conditional}

The package provides a mechanism to compile different versions
of a document. To customise the versions further some conditional processing
can come in handy to distinguish which version is being compiled.
The package provides two macros to describe the compilation context:

%%%%%%%%%%%%%%%%%%%%%%%%%%%%%%%%%%%%%%%%
\DescribeMacro{\ifchilddoc}
The conditional |\ifchilddoc| distinguishes between the compilation of
child documents and the main document:
%
\begin{center}
|\ifchilddoc |\textit{child-code}| |[|\||else |\textit{main-code}]| \||fi|
\end{center}

%%%%%%%%%%%%%%%%%%%%%%%%%%%%%%%%%%%%%%%%
\DescribeMacro{\childdocname}
\DescribeMacro{\childdocjob}
The macro |\childdocname| contains the filename (without extension)
of the main or child file being processed.
Note that |\childdocjob| will always contain the name of the main file.

%%%%%%%%%%%%%%%%%%%%%%%%%%%%%%%%%%%%%%%%
\paragraph{Title Page.}

Conditional processing can be used to include a title or banner page
in the main document when proper precautions are taken.
Importantly, the code in the main file should ensure that the page counter
(as well as other status parameters which are stored in the |.aux| files)
takes the same value after the conditional processing.
Otherwise the page numbers may take divergent values
depending on which part is compiled.

For example, a title page could be declared by:
%
\begin{center}
\begin{tabular}{l}
|\ifchilddoc\||else|\\
|\addtocounter{page}{-1}|\\
\textit{code for title page}\\
|\newpage|\\
|\||fi|
\end{tabular}
\end{center}
%
A banner page for the child documents can be generated by:
%
\begin{center}
\begin{tabular}{l}
|\ifchilddoc|\\
|\addtocounter{page}{-1}|\\
\textit{code for banner page}\\
|\newpage|\\
|\||fi|
\end{tabular}
\end{center}
%
Here one could write a message such as:
\begin{center}
|This is the part \childdocname{} of \childdocjob{}.|
\end{center}

%%%%%%%%%%%%%%%%%%%%%%%%%%%%%%%%%%%%%%%%%%%%%%%%%%%%%%%%%%%%%%%%%%%%%%%%%%%%%%%%
\subsection{Flags}
\label{sec:flags}

The package makes it easy to generate different versions
of the main or child documents.
To this end compilation flags can be defined
and assigned different default values.
They will be particularly useful in conjunction
with the forwarding mechanism described in \secref{sec:forward}.

For example, it may be useful to have a flag |\version|
which can be set to |draft| or |final|.
The document source will contain some conditional code
depending on the value of |\version|.
Suppose further, the flag should default to |final| for the main file
and to |draft| for child files
which is a natural assignment for editing the document.
This is achieved by placing the following code
in the preamble of the main document
(below the |\childdocmain| directive):
%
\begin{center}
\begin{tabular}{l}
|\ifchilddoc|\\
|\providecommand{\version}{draft}|\\
|\||else|\\
|\providecommand{\version}{final}|\\
|\||fi|
\end{tabular}
\end{center}
%
The definition by |\providecommand| makes sure
that previous definitions are not overwritten.
Further statements |\providecommand{\version}{...}|
can thus be added before the above code to override it.

For the main file, one might add a line
(between |\childdocmain| and the above block)
%
\begin{center}
|%\ifchilddoc\||else\providecommand{\version}{draft}\||fi|
\end{center}
%
which can be uncommented to produce a draft version.
Likewise one can add a line to the very top of a child file
(above the |\childdocof{|\textit{main}|}| directive)
%
\begin{center}
|%\providecommand{\version}{final}|
\end{center}
%
which can be uncommented to produce the final version of this child document.

%%%%%%%%%%%%%%%%%%%%%%%%%%%%%%%%%%%%%%%%%%%%%%%%%%%%%%%%%%%%%%%%%%%%%%%%%%%%%%%%
\subsection{Forwarding}
\label{sec:forward}

Different versions of the main or child documents
using compilation flags as described in \secref{sec:flags}
can be (permanently) stored in different files
for convenient compilation, viewing and distribution.
To this end, the package defines a command
to pass on compilation to a different file:

%%%%%%%%%%%%%%%%%%%%%%%%%%%%%%%%%%%%%%%%
\DescribeMacro{\childdocforward}
The command |\childdocforward| redirects processing to
another source file:
%
\begin{center}
\begin{tabular}{l}
|\input{childdoc.def}|\\
|\childdocforward[|\textit{main}|]{|\textit{dest}|}|\\
\end{tabular}
\end{center}
%
The argument \textit{dest} is the destination file
(without extension).
It should be the main file or one of the child files.
Note that further \textsf{childdoc} directives
such as |\childdocof| and |\childdocforward|
in the indicated file will be processed in this form.
The optional argument \textit{main}
passes on directly to the main file \textit{main}
while pretending to compile the child \textit{dest}.
This form behaves as if \textit{dest}
issues |\childdocof{|\textit{main}|}| right away,
and no further \textsf{childdoc} directives will be processed.

%%%%%%%%%%%%%%%%%%%%%%%%%%%%%%%%%%%%%%%%
\DescribeMacro{\...prefix}
In the alternative form |\childdocforwardprefix|,
%
\begin{center}
\begin{tabular}{l}
|\input{childdoc.def}|\\
|\childdocforwardprefix[|\textit{main}|]{|\textit{prefix}|}{|\textit{dest}|}|
\end{tabular}
\end{center}
%
the destination file is determined by a pattern
depending on the current file:
To make this work, the current file must be called
`{\textit{prefix}\hspace{0.2em}\textit{suffix}}'
with \textit{prefix} matching precisely the argument.
Processing is then passed on to the file
`{\textit{dest}\hspace{0.2em}\textit{suffix}}'.
Surely, the same effect is achieved by
directly specifying the
argument `{\textit{dest}\hspace{0.2em}\textit{suffix}}'
in the first form.
However, that requires to set up a different file
for each child. With the alternative form of the command
all these files can have exactly the same content
which simplifies setting them up and maintaining them.

For example, the following file |draft.tex|
with a compilation flag |\version| as described in \secref{sec:flags}
compiles the main document as a draft:
%
\begin{center}
\begin{tabular}{l}
|\def\version{draft}|\\
|\input{childdoc.def}|\\
|\childdocforward{|\textit{main}|}|
\end{tabular}
\end{center}
%
Likewise, the following files |final|\textit{nn}|.tex|
compile the final version of the child document
|child|\textit{nn}|.tex|:
%
\begin{center}
\begin{tabular}{l}
|\def\version{final}|\\
|\input{childdoc.def}|\\
|\childdocforwardprefix{final}{child}|
\end{tabular}
\end{center}
%

Note that when several versions of a main file and/or of each child file
are to be generated, it may be convenient to set up a |Makefile| or
shell script to automatise the process.

%%%%%%%%%%%%%%%%%%%%%%%%%%%%%%%%%%%%%%%%%%%%%%%%%%%%%%%%%%%%%%%%%%%%%%%%%%%%%%%%
\subsection{Command Line Processing}
\label{sec:commandline}

The effect of redirection files can also be achieved by invoking
the \LaTeX{} compiler with a more elaborate command line.
Most conveniently this should be done as part
of a shell script or a |Makefile|.

When using \textsf{childdoc} in the main file, the following
command lines effectively perform a redirection
(note that depending on the shell being used,
backslashes may have to be doubled: `|\|' $\to$ `|\\|'):
%
\begin{center}
|... -jobname "|\textit{target}|" |\\|"|[\textit{flags}]%
|\input{childdoc.def}\childdocforward[|\textit{main}|]{|\textit{dest}|}"|
\end{center}
%
Here \textit{target} is the name of the output file,
\textit{main} is the name of the main file
and \textit{dest} is the name of the main or child file to be processed
(all filenames without extensions).
The optional argument \textit{main} can be omitted
if \textit{main} matches \textit{dest}.
Optionally, compilation \textit{flags} can be defined via |\def| commands.
This command line makes the \TeX{} engine believe
it is compiling the file \textit{target}
whose content is specified as the latter parameter.
The provided code then forwards the processing to
\textit{main} or \textit{dest} as described in \secref{sec:forward}.

%%%%%%%%%%%%%%%%%%%%%%%%%%%%%%%%%%%%%%%%%%%%%%%%%%%%%%%%%%%%%%%%%%%%%%%%%%%%%%%%
\subsection{Include by Input}
\label{sec:input}

Including child documents by |\include| has some restrictions by design.
Most notably, the content of a child document always occupies
its own set of pages; pages cannot be shared between child documents.
Usually, this behaviour makes perfect sense
because each child document contain an essential part of the document.
However, in some situations it may be desirable to compose
a document from a collection of parts
without having mandatory page breaks between then.
For this case, the package
provides a mechanism to include parts
by |\input| which can also be processed individually.
However, by construction this mechanism
requires manual handling of the content to be output.

%%%%%%%%%%%%%%%%%%%%%%%%%%%%%%%%%%%%%%%%
\DescribeMacro{\ifchilddocmanual}
The main file should be prepared as usual, see \secref{sec:include}.
However, the document body must make a distinction
between processing of an individual part and of the main document, e.g.:
%
\begin{center}
\begin{tabular}{l}
|\ifchilddocmanual|\\
|\input{\childdocname}|\\
|\||else|\\
\textit{document body with }|\input{|\textit{part}|}|\\
|\||fi|
\end{tabular}
\end{center}
%
The conditional |\ifchilddocmanual| is true whenever
a part to be included by |\input| is being compiled,
and the name of the part is stored in |\childdocname|.

%%%%%%%%%%%%%%%%%%%%%%%%%%%%%%%%%%%%%%%%
\DescribeMacro{\childdocby}
Each part to be included by |\input| should start with:
%
\begin{center}
\begin{tabular}{l}
|\input{childdoc.def}|\\
|\childdocby{|\textit{main}|}|\\
\end{tabular}
\end{center}
%
The directive |\childdocby| is similar to |\childdocof|
described in \secref{sec:include},
but the subsequent selection of content must be done manually.
To that end, both |\ifchilddoc| and |\ifchilddocmanual|
will be true upon processing of a part,
and the name of the part is stored in |\childdocname|.
Note that |\jobname| will be set to the filename of the current part
so that each part receives an individual |.aux| file
that does not interfere with the |.aux| file(s) of the main document.
This behaviour can be altered by the alternative form
|\childdocby[*]{|\textit{main}|}| (with a non-empty optional argument)
which uses the |.aux| file of the main document
by setting |\jobname| to \textit{main}.

%%%%%%%%%%%%%%%%%%%%%%%%%%%%%%%%%%%%%%%%%%%%%%%%%%%%%%%%%%%%%%%%%%%%%%%%%%%%%%%%
\subsection{Driver Development}
\label{sec:driver}

The \textsf{childdoc} mechanism can also be use for the development
of definition files such as \LaTeX{} styles or classes.
This case differs from the above setup with multiple parts
included by |\include| in that no |\includeonly| should be invoked.
This can be achieved by starting the include file
(before |\ProvidesPackage|) with:
%
\begin{center}
\begin{tabular}{l}
|\input{childdoc.def}|\\
|\childdocforward{|\textit{main}|}|\\
\end{tabular}
\end{center}
%
or alternatively with:
%
\begin{center}
\begin{tabular}{l}
|\input{childdoc.def}|\\
|\childdocby{|\textit{main}|}|\\
\end{tabular}
\end{center}
%
Both forms have slightly different effects as described above.
The main file is prepared as usual, see \secref{sec:include}.

%%%%%%%%%%%%%%%%%%%%%%%%%%%%%%%%%%%%%%%%%%%%%%%%%%%%%%%%%%%%%%%%%%%%%%%%%%%%%%%%
\subsection{Legacy Detection}
\label{sec:detection}

The directive |\childdocmain| in the main file can detect
whether the complete document or merely a child is to be compiled
even without using the directive |\childdocof|.
This method is deprecated because it is less robust
and there is no compelling reason to use it;
it is merely provided for backward compatibility
and it may be removed in future versions.

If the detection mechanism is to be used,
it is mandatory to correctly specify
the filename of the main file as the argument of |\childdocmain|:
%
\begin{center}
\begin{tabular}{l}
|\input{childdoc.def}|\\
|\childdocmain{|\textit{main}|}|\\
\end{tabular}
\end{center}
%
If |\jobname| does not match the argument \textit{main} of |\childdocmain|,
it is assumed that |\jobname| points to the child file to be compiled.
When using |\childdocmain| with the main file specified as argument,
it suffices to start a child file
with just |\input{|\textit{main}|}|
without loading of the package and using |\childdocof|.
If instead all processing is done
with the appropriate \textsf{childdoc} directives,
the argument of \textit{main} of |\childdocmain| can be empty.

An alternative version of the command line processing described
in \secref{sec:commandline} using the detection mechanism reads:
%
\begin{center}
|... -jobname "|\textit{target}|" "|[\textit{flags}]%
[|\def\jobname{|\textit{dest}|}|]|\input{|\textit{main}|}"|
\end{center}

%%%%%%%%%%%%%%%%%%%%%%%%%%%%%%%%%%%%%%%%%%%%%%%%%%%%%%%%%%%%%%%%%%%%%%%%%%%%%%%%
\subsection{Manual Code}
\label{sec:manual}

In case one cannot be certain whether the definitions file |childdoc.def|
is installed on the target \TeX{} distribution
and one prefers not to ship it,
it is conceivable to paste a few relevant commands into the sources.

To that end, drop all statements |\input{childdoc.def}|
and perform the replacements as outlined below.
Instead of |\childdocmain{|\textit{main}|}| add the following code
to the top of the main file:
%
\begin{center}
\begin{tabular}{l}
|\||ifdefined\childdocname\endinput\||fi\newif\ifchilddoc|\\
|\edef\childdocname{\scantokens\expandafter{\jobname\noexpand}}|\\
|\def\childdocmain{|\textit{main}|}\||ifx\childdocmain\childdocname\||else|\\
|\childdoctrue\includeonly{\childdocname}\let\jobname\childdocmain\||fi|\\
\end{tabular}
\end{center}
%
Instead of |\childdocof{|\textit{main}|}| just include the main file
at the top of each child file:
%
\begin{center}
|\input{|\textit{main}|}|
\end{center}
%
A simple redirection |\childdocforward{|\textit{dest}|}| is achieved by:
%
\begin{center}
|\def\jobname{|\textit{dest}|}\input{\jobname}|
\end{center}
%
The redirection with prefix
|\childdocforwardprefix[|\textit{prefix}|]{|\textit{dest}|}|
is accomplished by:
%
\begin{center}
\begin{tabular}{l}
|{\edef\jobname{\scantokens\expandafter{\jobname\noexpand}}|\\
|\def\redirectjob |\textit{prefix}|#1~~~{\gdef\jobname{|\textit{dest}|#1}}|\\
|\expandafter\redirectjob\jobname~~~}\input{\jobname}|
\end{tabular}
\end{center}

In an alternative approach,
child documents can be compiled by a specific command line
without additional code or specific definitions:
%
\begin{center}
|... -jobname "|\textit{target}|" "|[\textit{flags}]%
|\includeonly{|\textit{dest}|}\input{|\textit{main}|}"|
\end{center}
%

%%%%%%%%%%%%%%%%%%%%%%%%%%%%%%%%%%%%%%%%%%%%%%%%%%%%%%%%%%%%%%%%%%%%%%%%%%%%%%%%
%%%%%%%%%%%%%%%%%%%%%%%%%%%%%%%%%%%%%%%%%%%%%%%%%%%%%%%%%%%%%%%%%%%%%%%%%%%%%%%%
\section{Information}

%%%%%%%%%%%%%%%%%%%%%%%%%%%%%%%%%%%%%%%%%%%%%%%%%%%%%%%%%%%%%%%%%%%%%%%%%%%%%%%%
\subsection{Copyright}

Copyright \copyright{} 2017--2018 Niklas Beisert

This work may be distributed and/or modified under the
conditions of the \LaTeX{} Project Public License, either version 1.3
of this license or (at your option) any later version.
The latest version of this license is in
  \url{http://www.latex-project.org/lppl.txt}
and version 1.3 or later is part of all distributions of \LaTeX{}
version 2005/12/01 or later.

This work has the LPPL maintenance status `maintained'.

The Current Maintainer of this work is Niklas Beisert.

This work consists of the files |README.txt|, |childdoc.ins| and |childdoc.dtx|
as well as the derived files |childdoc.def|, |cdocsamp.tex|
with |cdocsch1.tex|, |cdocsch2.tex|, |cdocspt3.tex|, |cdocspt4.tex|,
|cdocsdrf.tex|, |cdocsfn1.tex|, |cdocsfn2.tex|
as well as |childdoc.pdf|.

%%%%%%%%%%%%%%%%%%%%%%%%%%%%%%%%%%%%%%%%%%%%%%%%%%%%%%%%%%%%%%%%%%%%%%%%%%%%%%%%
\subsection{Files and Installation}

The package consists of the files:
%
\begin{center}
\begin{tabular}{ll}
    |README.txt|   & readme file \\
    |childdoc.ins| & installation file \\
    |childdoc.dtx| & source file \\
    |childdoc.def| & definition file \\
    |cdocsamp.tex| & sample main file \\
    |cdocsch1.tex| & sample include file \\
    |cdocsch2.tex| & sample include file \\
    |cdocspt3.tex| & sample part file \\
    |cdocspt4.tex| & sample part file \\
    |cdocsdrf.tex| & sample redirection file \\
    |cdocsfn1.tex| & sample redirection file \\
    |cdocsfn2.tex| & sample redirection file \\
    |childdoc.pdf| & manual
\end{tabular}
\end{center}
%
The distribution consists of the files
|README.txt|, |childdoc.ins| and |childdoc.dtx|.
%
\begin{itemize}
\item
Run (pdf)\LaTeX{} on |childdoc.dtx|
to compile the manual |childdoc.pdf| (this file).
\item
Run \LaTeX{} on |childdoc.ins| to create the definitions file |childdoc.def|
and the sample |cdocsamp.tex| with include files
|cdocsch1.tex|, |cdocsch2.tex|, |cdocspt3.tex|, |cdocspt4.tex|,
|cdocsdrf.tex|, |cdocsfn1.tex|, |cdocsfn2.tex|.
Then copy the file |childdoc.def| to an appropriate directory of your \LaTeX{}
distribution, e.g.\ \textit{texmf-root}|/tex/latex/childdoc|.
\end{itemize}

%%%%%%%%%%%%%%%%%%%%%%%%%%%%%%%%%%%%%%%%%%%%%%%%%%%%%%%%%%%%%%%%%%%%%%%%%%%%%%%%
\subsection{Related CTAN Packages}

There are several other packages which offer a similar functionality:
%
\begin{itemize}
\item
The packages
\href{http://ctan.org/pkg/docmute}{\textsf{docmute}},
\href{http://ctan.org/pkg/includex}{\textsf{includex}} and
\href{http://ctan.org/pkg/standalone}{\textsf{standalone}}
provide commands to include only the document body of
a child file thus allowing both files to be compiled individually.
\item
The packages \href{http://ctan.org/pkg/subdocs}{\textsf{subdocs}}
and \href{http://ctan.org/pkg/subfiles}{\textsf{subfiles}}
provide structures in which the main and child documents can be
encapsulated and allowing them to be compiled individually.
The inclusion mechanism is different from the conventional |\include|.
\item
The package \href{http://ctan.org/pkg/combine}{\textsf{combine}}
is an elaborate solution to combine several documents into one.
\end{itemize}
%
See also the CTAN topic \href{http://ctan.org/topic/subdocs}{\textsf{subdocs}}
for further related packages.
The present package differs from the above solutions in that
a document structure constructed with the conventional |\include| mechanism
just needs two extra commands at the top of every file
such that all constituent files can be compiled individually.

%%%%%%%%%%%%%%%%%%%%%%%%%%%%%%%%%%%%%%%%%%%%%%%%%%%%%%%%%%%%%%%%%%%%%%%%%%%%%%%%
%\subsection{Feature Suggestions}
%
%The following is a list of features which may be useful for future
%versions of this package:
%%
%\begin{itemize}
%\item
%\ldots
%\end{itemize}

%%%%%%%%%%%%%%%%%%%%%%%%%%%%%%%%%%%%%%%%%%%%%%%%%%%%%%%%%%%%%%%%%%%%%%%%%%%%%%%%
\subsection{Revision History}

%%%%%%%%%%%%%%%%%%%%%%%%%%%%%%%%%%%%%%%%
\paragraph{v2.0:} 2018/12/30

\begin{itemize}
\item
immediate forward processing
\item
added |\childdocby| mechanism
\item
manual restructured
\end{itemize}

%%%%%%%%%%%%%%%%%%%%%%%%%%%%%%%%%%%%%%%%
\paragraph{v1.6:} 2018/01/17

\begin{itemize}
\item
application for development of include files
\item
corrections to manual
\end{itemize}

%%%%%%%%%%%%%%%%%%%%%%%%%%%%%%%%%%%%%%%%
\paragraph{v1.5:} 2017/05/21

\begin{itemize}
\item
more complete structuring introduced
\item
|\childdocof| introduced
\item
|\childdoc| renamed to |\childdocmain|
\item
|\childredirect| renamed to |\childdocforward| and |\childdocforwardprefix|
and functionality expanded
\end{itemize}

%%%%%%%%%%%%%%%%%%%%%%%%%%%%%%%%%%%%%%%%
\paragraph{v1.0:} 2017/04/27

\begin{itemize}
\item
manual and install package
\item
first version published on CTAN
\end{itemize}

%%%%%%%%%%%%%%%%%%%%%%%%%%%%%%%%%%%%%%%%
\paragraph{v0.6:} 2017/04/26

\begin{itemize}
\item
redirection mechanism added
\end{itemize}

%%%%%%%%%%%%%%%%%%%%%%%%%%%%%%%%%%%%%%%%
\paragraph{v0.5:} 2017/04/26

\begin{itemize}
\item
functionality in definition file
\end{itemize}


%%%%%%%%%%%%%%%%%%%%%%%%%%%%%%%%%%%%%%%%%%%%%%%%%%%%%%%%%%%%%%%%%%%%%%%%%%%%%%%%
%%%%%%%%%%%%%%%%%%%%%%%%%%%%%%%%%%%%%%%%%%%%%%%%%%%%%%%%%%%%%%%%%%%%%%%%%%%%%%%%
%%%%%%%%%%%%%%%%%%%%%%%%%%%%%%%%%%%%%%%%%%%%%%%%%%%%%%%%%%%%%%%%%%%%%%%%%%%%%%%%
\appendix

\settowidth\MacroIndent{\rmfamily\scriptsize 000\ }

 \DocInput{childdoc.dtx}

\end{document}
%</driver>
% \fi
%
% %%%%%%%%%%%%%%%%%%%%%%%%%%%%%%%%%%%%%%%%%%%%%%%%%%%%%%%%%%%%%%%%%%%%%%%%%%%%%%
% %%%%%%%%%%%%%%%%%%%%%%%%%%%%%%%%%%%%%%%%%%%%%%%%%%%%%%%%%%%%%%%%%%%%%%%%%%%%%%
% \section{Sample}
%\iffalse
%<*samplemain>
%\fi
%
% The following presents a sample document
% with two chapters, two parts, a title page,
% a compile flag as well as three forwarding files to set the flag.
% It consists of eight |.tex| files:
% \begin{center}
% \begin{tabular}{ll}
% |cdocsamp.tex|&main file\\
% |cdocsch1.tex|&include file for chapter 1\\
% |cdocsch2.tex|&include file for chapter 2\\
% |cdocspt3.tex|&include file for part 3\\
% |cdocspt4.tex|&include file for part 4\\
% |cdocsdrf.tex|&forwarding file for main file in draft mode\\
% |cdocsfi1.tex|&forwarding file for final version of chapter 1\\
% |cdocsfi2.tex|&forwarding file for final version of chapter 2\\
% \end{tabular}
% \end{center}
% Each of the eight files can be compiled directly by the \LaTeX{} compiler.
%
% %%%%%%%%%%%%%%%%%%%%%%%%%%%%%%%%%%%%%%
% \paragraph{Main File.}
%
% The main file is called |cdocsamp.tex|.
%
% Load the \textsf{childdoc} definitions and
% declare the filename for the main document:
%    \begin{macrocode}
\input{childdoc.def}
\childdocmain{}
%    \end{macrocode}

% Optional override for |\version| flag:
%    \begin{macrocode}
%%\ifchilddoc\else\providecommand{\version}{draft}\fi
%    \end{macrocode}

% Define the default values for the |\version| flag
% (|final| for the main file and |draft| for childs):
%    \begin{macrocode}
\ifchilddoc
\providecommand{\version}{draft}
\else
\providecommand{\version}{final}
\fi
%    \end{macrocode}

% Load the standard document class:
%    \begin{macrocode}
\documentclass[12pt]{article}
%    \end{macrocode}

% Start the document body:
%    \begin{macrocode}
\begin{document}
%    \end{macrocode}

% Declare a title page.
% Print title, part of document being processed and version flag:
%    \begin{macrocode}
\addtocounter{page}{-1}
\begin{center}
{\LARGE\bfseries{}childdoc example\par}
\vspace{1cm}
\ifchilddoc
\ifchilddocmanual part\else chapter\fi:
`\childdocname' of `\childdocjob'\par
\else
main document: `\childdocjob'\par
\fi
version: \version\par
\end{center}
\newpage
%    \end{macrocode}

% Manually include selected file,
% otherwise process as usual:
%    \begin{macrocode}
\ifchilddocmanual
\section*{part `\childdocname'}
\input{\childdocname}
\else
%    \end{macrocode}

% Include the two chapters:
%    \begin{macrocode}
\include{cdocsch1}
\include{cdocsch2}
%    \end{macrocode}

% Include the two parts unless only chapters should be displayed:
%    \begin{macrocode}
\ifchilddoc\else
\section{part three}
\input{cdocspt3}
\section{part four}
\input{cdocspt4}
\fi
%    \end{macrocode}

% Process as usual until here:
%    \begin{macrocode}
\fi
%    \end{macrocode}

% End of document body:
%    \begin{macrocode}
\end{document}
%    \end{macrocode}
%\iffalse
%</samplemain>
%\fi
%
% %%%%%%%%%%%%%%%%%%%%%%%%%%%%%%%%%%%%%%
% \paragraph{Chapter Include Files.}
%
% The include files are called |cdocsch1.tex| and |cdocsch2.tex|.
%
%\iffalse
%<*samplechap1|samplechap2>
%\fi

% Optional override for |\version| flag:
%    \begin{macrocode}
%%\providecommand{\version}{final}
%    \end{macrocode}

% Include the main document:
%    \begin{macrocode}
\input{childdoc.def}
\childdocof{cdocsamp}
%    \end{macrocode}

%\iffalse
%</samplechap1|samplechap2>
%\fi
%
%\iffalse
%<*samplechap1>
%\fi
% Some text for chapter 1:
%    \begin{macrocode}
\section{one}
some text in chapter one
%    \end{macrocode}

%\iffalse
%</samplechap1>
%\fi
% Some text for chapter 2:
%\iffalse
%<*samplechap2>
%\fi
%    \begin{macrocode}
\section{two}
more text in chapter two
%    \end{macrocode}

%\iffalse
%</samplechap2>
%\fi
%
% %%%%%%%%%%%%%%%%%%%%%%%%%%%%%%%%%%%%%%
% \paragraph{Part Include Files.}
%
% The include files are called |cdocspt3.tex| and |cdocspt4.tex|.
%
%\iffalse
%<*samplepart3|samplepart4>
%\fi

% Optional override for |\version| flag:
%    \begin{macrocode}
%%\providecommand{\version}{final}
%    \end{macrocode}

% Include the main document:
%    \begin{macrocode}
\input{childdoc.def}
\childdocby{cdocsamp}
%    \end{macrocode}

%\iffalse
%</samplepart3|samplepart4>
%\fi
%
%\iffalse
%<*samplepart3>
%\fi
% Some text for part 3:
%    \begin{macrocode}
some text in part three
%    \end{macrocode}

%\iffalse
%</samplepart3>
%\fi
% Some text for part 4:
%\iffalse
%<*samplepart4>
%\fi
%    \begin{macrocode}
more text in part four
%    \end{macrocode}

%\iffalse
%</samplepart4>
%\fi
%
% %%%%%%%%%%%%%%%%%%%%%%%%%%%%%%%%%%%%%%
% \paragraph{Forwarding for a Complete Draft.}
%
% The following forwarding file |cdocsdrf.tex|
% compiles the main document in draft mode:
%\iffalse
%<*sampledraft>
%\fi
%    \begin{macrocode}
\def\version{draft}
\input{childdoc.def}
\childdocforward{cdocsamp}
%    \end{macrocode}

%\iffalse
%</sampledraft>
%\fi
%
% %%%%%%%%%%%%%%%%%%%%%%%%%%%%%%%%%%%%%%
% \paragraph{Forwarding for Final Version of the Chapters.}
%
% The following forwarding files |cdocsfn1.tex| and |cdocsfn2.tex|
% (with identical content)
% compile the final versions of the child documents
% |cdocsch1.tex| and |cdocsch2.tex|, respectively:
%\iffalse
%<*samplefinal>
%\fi
%    \begin{macrocode}
\def\version{final}
\input{childdoc.def}
\childdocforwardprefix[cdocsamp]{cdocsfn}{cdocsch}
%    \end{macrocode}

%\iffalse
%</samplefinal>
%\fi
%
% %%%%%%%%%%%%%%%%%%%%%%%%%%%%%%%%%%%%%%
% \paragraph{Command Line Processing.}
%
% The following three command lines generate the output files
% |cdocscld|, |cdocscl1| and |cdocscl2|
% which should be identical to
% |cdocsdrf|, |cdocsch1| and |cdocsfn2|, respectively:
% \begin{center}
% \begin{tabular}{l}
% |latex -jobname cdocscld \|\\
% |  "\def\version{draft}\input{childdoc.def}\childdocforward{cdocsamp}"|\\
% |latex -jobname cdocscl1 \|\\
% |  "\input{childdoc.def}\childdocforward[cdocsamp]{cdocsch1}"|\\
% |latex -jobname cdocscl2 \|\\
% |  "\def\version{final}\input{childdoc.def}\childdocforward{cdocsch2}"|
% \end{tabular}
% \end{center}
% Note that the trailing backslash on each first line
% merely continues the input to the second line
% (for convenient cut ant paste).
% Furthermore, the command |latex| can be replaced by any
% of its alternative versions such as |pdflatex|.
%
% %%%%%%%%%%%%%%%%%%%%%%%%%%%%%%%%%%%%%%%%%%%%%%%%%%%%%%%%%%%%%%%%%%%%%%%%%%%%%%
% %%%%%%%%%%%%%%%%%%%%%%%%%%%%%%%%%%%%%%%%%%%%%%%%%%%%%%%%%%%%%%%%%%%%%%%%%%%%%%
% \section{Implementation}
%\iffalse
%<*package>
%\fi
%
% This section describes the definitions file |childdoc.def|.

% The definitions cannot be loaded using |\usepackage| or |\RequirePackage|
% which has a mechanism to prevent loading a style file more than once.
% When loading the definitions by means of |\input|
% multiple instances have to be prevented manually:
%\iffalse
%This code needs to be before the `\ProvidesFile' directive
%which is defined at the beginning of this file.
%Therefore it is also placed there and commented out here.
%</package>
%<*discard>
%\fi
%    \begin{macrocode}
\ifdefined\childdocmain\endinput\fi
%    \end{macrocode}
%\iffalse
%</discard>
%<*package>
%\fi
%
% \macro{\ifchilddoc}
% \macro{\ifchilddocmanual}
% The conditional |\ifchilddoc| tells whether a
% child (true) or main (false) document is being compiled.
% The conditional |\ifchilddocmanual| tells whether
% the |\includeonly| mechanism is used (false) or
% the selection of child files must be performed manually (true).
% The definitions initialise to false:
%    \begin{macrocode}
\newif\ifchilddoc
\newif\ifchilddocmanual
%    \end{macrocode}

% \macro{\childdocname}
% \macro{\childdocjob}
% The macro |\childdocname| stores the name of the main document
% to be compiled. The macro |\childdocjob| stores the name of
% the document on which the \LaTeX{} compiler was originally invoked.
% The content of |\jobname| cannot be compared
% to filenames specified in the source due to different catcodes.
% The following code rescans |\jobname|, stores the result
% in |\childdocname| and saves a copy in |\childdocjob|:
%    \begin{macrocode}
\edef\childdocname{\scantokens\expandafter{\jobname\noexpand}}
\let\childdocjob\childdocname
%    \end{macrocode}

% \macro{\childdocdisable}
% The macro |\childdocdisable| prevents the main file
% from being processed more than once.
% At this stage, the main document command |\childdocmain|
% is assumed to be called once again where it should do nothing.
% Any subsequent call to it should prevent
% a secondary processing of the main document
% It overwrites the forwarding commands
% |\childdocof| and |\childdocforward|
% with empty macros to prevent further inclusions of the main document:
%    \begin{macrocode}
\newcommand{\childdocdisable}
{
  \renewcommand{\childdocmain}[1]{\renewcommand{\childdocmain}[1]{\endinput}}
  \renewcommand{\childdocof}[1]{}
  \renewcommand{\childdocby}[2][]{}
  \renewcommand{\childdocforward}[2][]{}
  \renewcommand{\childdocdisable}{}
}
%    \end{macrocode}

% \macro{\childdocmain}
% The macro |\childdocmain| is to be called at the top of the main file
% with nothing or the main filename (without extension) as argument.
% First, it breaks loops.
% If the argument is not empty and does not match |\childdocname|
% (which is set by the first inclusion of |childdoc.def|),
% |\ifchilddoc| is set to true, |\includeonly| is applied to the child file
% and |\jobname| is set to the main file
% (for proper handling of |.aux| files):
%    \begin{macrocode}
\newcommand{\childdocmain}[1]
{
  \childdocdisable\childdocmain{}
  \if?#1?\else
    \begingroup
      \def\childdoctmp{#1}
      \ifx\childdoctmp\childdocname
        \def\childdoctmp{}
      \else
        \def\childdoctmp
        {
          \childdoctrue
          \includeonly{\childdocname}
          \def\childdocjob{#1}
          \def\jobname{#1}
        }
      \fi
      \expandafter
    \endgroup
    \childdoctmp
  \fi
}
%    \end{macrocode}

% \macro{\childdocof}
% The command |\childdocof| redirects
% compilation to the main file |#1|.
%    \begin{macrocode}
\newcommand{\childdocof}[1]
{
  \childdocdisable
  \childdoctrue
  \includeonly{\childdocname}
  \def\jobname{#1}
  \def\childdocjob{#1}
  \input{#1}
}
%    \end{macrocode}

% \macro{\childdocby}
% The command |\childdocby| ....
%    \begin{macrocode}
\newcommand{\childdocby}[2][]
{
  \childdocdisable
  \childdoctrue
  \childdocmanualtrue
  \if?#1?\else
    \def\jobname{#2}
  \fi
  \def\childdocjob{#2}
  \input{#2}
  \endinput
}
%    \end{macrocode}

% \macro{\childdocforward}
% The command |\childdocforward| redirects
% compilation to the main file or
% (if the optional argument is given) a child file.
% Parameters are set as if the main file
% or a child file starting with |\childdocof| was compiled.
% Then compilation is handed over to the main file:
%    \begin{macrocode}
\newcommand{\childdocforward}[2][]
{
  \begingroup
    \if?#1?
      \def\childdoctmp
      {
        \def\childdocname{#2}
        \def\childdocjob{#2}
        \def\jobname{#2}
        \input{#2}
        \endinput
      }
    \else
      \def\childdoctmp
      {
        \childdocdisable
        \def\childdocname{#2}
        \childdoctrue
        \includeonly{#2}
        \def\childdocjob{#1}
        \def\jobname{#1}
        \input{#1}
        \endinput
      }
    \fi
    \expandafter
  \endgroup
  \childdoctmp
}
%    \end{macrocode}

% \macro{\childdocforwardprefix}
% The command |\childdocforwardprefix| redirects
% compilation to the main or a child file by means of a pattern.
% The prefix |#1| in the current filename is replaced by |#2|
% and the suffix of the current filename is kept
% (it is assumed that the filename does not contain the substring `|~~~|'
% which is used as a delimiter).
% Compilation is handed over to the new file by |\childdocforward|:
%    \begin{macrocode}
\newcommand{\childdocforwardprefix}[3][]
{
  \begingroup
    \def\childdocextract #2##1~~~{\def\childdoctmp{\childdocforward[#1]{#3##1}}}
    \expandafter\childdocextract\childdocname~~~
    \expandafter
  \endgroup
  \childdoctmp
}
%    \end{macrocode}

% \macro{\childdoc}
% The deprecated macro |\childdoc| is a legacy version of |\childdocmain|:
%    \begin{macrocode}
\newcommand{\childdoc}{\childdocmain}
%    \end{macrocode}

% \macro{\childdocredirect}
% The deprecated macro |\childdocredirect| is a legacy version
% of |\childdocforward| and |\childdocforwardprefix|:
%    \begin{macrocode}
\newcommand{\childdocredirect}[2][]
{
  \begingroup
    \if?#1?
      \def\childdoctmp{\childdocforward{#2}}
    \else
      \def\childdoctmp{\childdocforwardprefix{#1}{#2}}
    \fi
    \expandafter
  \endgroup
  \childdoctmp
}
%    \end{macrocode}

%\iffalse
%</package>
%\fi
%
\endinput
|\\
|\childdocmain{}|\\
\end{tabular}
\end{center}
at the very top of the main \LaTeX{} file,
in particular \emph{before} the |\documentclass| statement!
The argument of |\childdocmain| should be left empty
(but it must be present).

%%%%%%%%%%%%%%%%%%%%%%%%%%%%%%%%%%%%%%%%
\DescribeMacro{\childdocof}
Furthermore, add the commands
\begin{center}
\begin{tabular}{l}
|% \iffalse
%
% childdoc.dtx Copyright (C) 2017-2018 Niklas Beisert
%
% This work may be distributed and/or modified under the
% conditions of the LaTeX Project Public License, either version 1.3
% of this license or (at your option) any later version.
% The latest version of this license is in
%   http://www.latex-project.org/lppl.txt
% and version 1.3 or later is part of all distributions of LaTeX
% version 2005/12/01 or later.
%
% This work has the LPPL maintenance status `maintained'.
%
% The Current Maintainer of this work is Niklas Beisert.
%
% This work consists of the files childdoc.dtx and childdoc.ins
% and the derived files childdoc.def and cdocsamp.tex with
% cdocsch1.tex, cdocsch2.tex, cdocsdrf.tex, cdocsfn1.tex, cdocsfn2.tex.
%
%<package>\ifdefined\childdocmain\endinput\fi
%<package>\ProvidesFile{childdoc.def}[2018/12/30 v2.0 child document driver]
%<samplemain>\ProvidesFile{cdocsamp.tex}[2018/12/30 v2.0 sample for childdoc]
%<*driver>
%\ProvidesFile{childdoc.drv}[2018/12/30 v2.0 childdoc reference manual file]
\PassOptionsToClass{10pt,a4paper}{article}
\documentclass{ltxdoc}

\usepackage[margin=35mm]{geometry}
\usepackage{hyperref}
\usepackage{hyperxmp}
\usepackage[usenames]{color}

\hypersetup{colorlinks=true}
\hypersetup{pdfstartview=FitH}
\hypersetup{pdfpagemode=UseNone}
\hypersetup{pdfsource={}}
\hypersetup{pdflang={en-UK}}
\hypersetup{pdfcopyright={Copyright 2017-2018 Niklas Beisert.
  This work may be distributed and/or modified under the
  conditions of the LaTeX Project Public License, either version 1.3
  of this license or (at your option) any later version.}}
\hypersetup{pdflicenseurl={http://www.latex-project.org/lppl.txt}}
\hypersetup{pdfcontactaddress={ETH Zurich, ITP, HIT K,
  Wolfgang-Pauli-Strasse 27}}
\hypersetup{pdfcontactpostcode={8093}}
\hypersetup{pdfcontactcity={Zurich}}
\hypersetup{pdfcontactcountry={Switzerland}}
\hypersetup{pdfcontactemail={nbeisert@itp.phys.ethz.ch}}
\hypersetup{pdfcontacturl={http://people.phys.ethz.ch/\xmptilde nbeisert/}}

\newcommand{\secref}[1]{\hyperref[#1]{section \ref*{#1}}}

\parskip1ex
\parindent0pt
\let\olditemize\itemize
\def\itemize{\olditemize\parskip0pt}

\begin{document}

\title{The \textsf{childdoc} Package}
\hypersetup{pdftitle={The childdoc Package}}
\author{Niklas Beisert\\[2ex]
  Institut f\"ur Theoretische Physik\\
  Eidgen\"ossische Technische Hochschule Z\"urich\\
  Wolfgang-Pauli-Strasse 27, 8093 Z\"urich, Switzerland\\[1ex]
  \href{mailto:nbeisert@itp.phys.ethz.ch}
  {\texttt{nbeisert@itp.phys.ethz.ch}}}
\hypersetup{pdfauthor={Niklas Beisert}}
\hypersetup{pdfsubject={Manual for the LaTeX2e Package childdoc}}
\date{30 December 2018, \textsf{v2.0}}
\maketitle

\begin{abstract}\noindent
\textsf{childdoc} is a \LaTeXe{} package
that enables the direct compilation
of document sections included by |\include|
to individual files.
\end{abstract}

\begingroup
\parskip0ex
\tableofcontents
\endgroup

%%%%%%%%%%%%%%%%%%%%%%%%%%%%%%%%%%%%%%%%%%%%%%%%%%%%%%%%%%%%%%%%%%%%%%%%%%%%%%%%
%%%%%%%%%%%%%%%%%%%%%%%%%%%%%%%%%%%%%%%%%%%%%%%%%%%%%%%%%%%%%%%%%%%%%%%%%%%%%%%%
\section{Introduction}

\LaTeX{} provides a mechanism to structure a large document (such as a book)
into a main file and several child files (containing the chapters)
using the |\include| command.
This mechanism is beneficial for documents
which span hundreds of pages in order to
make the source file(s) more manageable.
Moreover, compilation can be restricted to
selected child files by means of the |\includeonly| command.
The latter feature can be used to reduce the compilation time while editing
(this was significantly more useful in the earlier days of \LaTeX{})
or to generate a smaller document which is easier to navigate.
Another application of |\includeonly| is to generate
documents consisting of selected parts of the complete document.

However, there are a few drawbacks of the plain |\include| mechanism:
\begin{itemize}
\item
The child files cannot be compiled on their own,
they can only be compiled via the main file.
A naive editing environment
(such as a text editor with an option
to have the current file processed by \LaTeX)
may require one to switch to the main file before compiling;
attempting to compile the child file produces errors.
\item
The main file must be modified (each time)
to adjust the |\includeonly| command
to the present needs. This easily leaves the main file in a messy state.
\item
The generated document will always carry the filename
of the main document. This is inconvenient if
several child files are to be compiled and
to be kept for distribution.
\end{itemize}

The present package provides a simple interface
to make child files individually compilable by \LaTeX{}.
Compiling a child file then has the same effect as compiling
the main file with an |\includeonly| command
to select the appropriate child.
Moreover the generated document will carry the name of the child
rather than the main file.
This resolves all three above issues.

This feature is meant to make the editing of books,
thesis documents and lecture notes somewhat more convenient.
However, the package can also be used efficiently for
composing a series of documents (such as exercise sheets)
which are typically distributed individually.
It then assists the author in generating the individual documents
(potentially in different versions)
as well as a document containing the collected series.
Another application is in developing style files
or other kinds of included material
where compilation of the style file could redirect
to a sample or test file.

%%%%%%%%%%%%%%%%%%%%%%%%%%%%%%%%%%%%%%%%%%%%%%%%%%%%%%%%%%%%%%%%%%%%%%%%%%%%%%%%
%%%%%%%%%%%%%%%%%%%%%%%%%%%%%%%%%%%%%%%%%%%%%%%%%%%%%%%%%%%%%%%%%%%%%%%%%%%%%%%%
\section{Usage}

First of all, the package \textsf{childdoc} is \emph{not} a standard
\LaTeXe{} |.sty| style file! Therefore it needs to be invoked in
a non-standard way.

%%%%%%%%%%%%%%%%%%%%%%%%%%%%%%%%%%%%%%%%%%%%%%%%%%%%%%%%%%%%%%%%%%%%%%%%%%%%%%%%
\subsection{Included Files}
\label{sec:include}

%%%%%%%%%%%%%%%%%%%%%%%%%%%%%%%%%%%%%%%%
\DescribeMacro{\childdocmain}
To use the package, add the commands
\begin{center}
\begin{tabular}{l}
|\input{childdoc.def}|\\
|\childdocmain{}|\\
\end{tabular}
\end{center}
at the very top of the main \LaTeX{} file,
in particular \emph{before} the |\documentclass| statement!
The argument of |\childdocmain| should be left empty
(but it must be present).

%%%%%%%%%%%%%%%%%%%%%%%%%%%%%%%%%%%%%%%%
\DescribeMacro{\childdocof}
Furthermore, add the commands
\begin{center}
\begin{tabular}{l}
|\input{childdoc.def}|\\
|\childdocof{|\textit{main}|}|\\
\end{tabular}
\end{center}
at the top of every child file \textit{child}
which is included by |\include{|\textit{child}|}|
from within the main file
(or at least for those files to be compiled individually).
The argument \textit{main} must be the filename of the main file.

There are a couple of
considerations in setting up the main and child documents:

%%%%%%%%%%%%%%%%%%%%%%%%%%%%%%%%%%%%%%%%
\paragraph{Restrictions.}

Please note the following restrictions:
\begin{itemize}
\item
|\childdocmain| must be called with one argument \textit{main}
to ensure compatibility with earlier version of the package.
It must either be empty (|\childdocmain{}|)
or precisely match the filename of the main file in which it is specified.
See \secref{sec:detection} for further information.
\item
The filename \textit{main} must be specified without the |.tex| extension.
\item
The filename \textit{main} is case sensitive
(even in case-insensitive file systems)
due to internal string comparison.
\item
The argument \textit{main} should be fully expanded, it cannot be a macro.
\item
Subdirectories and special characters should be avoided in filenames.
\item
The command |\childdocmain{|\textit{main}|}| must be followed by a whitespace.
It should not be followed immediately by another command
or by a comment mark `|%|'.
This is because the \TeX{} parser reads the token immediately following
the argument of |\childdocmain| and puts it
at the beginning of every child section;
however, a white\-space is ignored.
\end{itemize}

%%%%%%%%%%%%%%%%%%%%%%%%%%%%%%%%%%%%%%%%
\paragraph{Content of Main File.}

It is advisable to place all content in the child files included by |\include|.
Any output contained in the main file will appear in all child documents
unless suppressed manually;
it cannot be suppressed automatically by the |\includeonly| directive
and thus should normally be avoided.
A method to include some content in the main file
by means of conditional processing is described in \secref{sec:conditional}.

%%%%%%%%%%%%%%%%%%%%%%%%%%%%%%%%%%%%%%%%
\paragraph{Page Numbering.}

When only a part of the document is compiled,
the appropriate numbering of pages
(as well as other status parameters)
is determined from the |.aux| files.
The latter contain information from previous passes.
However this information needs to propagate through
all intermediate child documents.
Therefore the page numbering in child documents may well
be inconsistent until the complete document is compiled at least once.

A useful (if unconventional) way to always ensure a consistent
page numbering is to restart the numbering in each child document
and denote the pages by `\textit{child}|.|\textit{page}'
where \textit{child} represents the chapter/section number of the child file.
This can be achieved by the command
|\numberwithin{page}{|\textit{child}|}|
of the \textsf{amsmath} package
where \textit{child} can be |chapter| or |section|
depending on the chosen structuring.
Alternatively, one can modify the macro |\thepage| appropriately
and reset the counter |page| at the start of each child file.

%%%%%%%%%%%%%%%%%%%%%%%%%%%%%%%%%%%%%%%%%%%%%%%%%%%%%%%%%%%%%%%%%%%%%%%%%%%%%%%%
\subsection{Conditional Processing}
\label{sec:conditional}

The package provides a mechanism to compile different versions
of a document. To customise the versions further some conditional processing
can come in handy to distinguish which version is being compiled.
The package provides two macros to describe the compilation context:

%%%%%%%%%%%%%%%%%%%%%%%%%%%%%%%%%%%%%%%%
\DescribeMacro{\ifchilddoc}
The conditional |\ifchilddoc| distinguishes between the compilation of
child documents and the main document:
%
\begin{center}
|\ifchilddoc |\textit{child-code}| |[|\||else |\textit{main-code}]| \||fi|
\end{center}

%%%%%%%%%%%%%%%%%%%%%%%%%%%%%%%%%%%%%%%%
\DescribeMacro{\childdocname}
\DescribeMacro{\childdocjob}
The macro |\childdocname| contains the filename (without extension)
of the main or child file being processed.
Note that |\childdocjob| will always contain the name of the main file.

%%%%%%%%%%%%%%%%%%%%%%%%%%%%%%%%%%%%%%%%
\paragraph{Title Page.}

Conditional processing can be used to include a title or banner page
in the main document when proper precautions are taken.
Importantly, the code in the main file should ensure that the page counter
(as well as other status parameters which are stored in the |.aux| files)
takes the same value after the conditional processing.
Otherwise the page numbers may take divergent values
depending on which part is compiled.

For example, a title page could be declared by:
%
\begin{center}
\begin{tabular}{l}
|\ifchilddoc\||else|\\
|\addtocounter{page}{-1}|\\
\textit{code for title page}\\
|\newpage|\\
|\||fi|
\end{tabular}
\end{center}
%
A banner page for the child documents can be generated by:
%
\begin{center}
\begin{tabular}{l}
|\ifchilddoc|\\
|\addtocounter{page}{-1}|\\
\textit{code for banner page}\\
|\newpage|\\
|\||fi|
\end{tabular}
\end{center}
%
Here one could write a message such as:
\begin{center}
|This is the part \childdocname{} of \childdocjob{}.|
\end{center}

%%%%%%%%%%%%%%%%%%%%%%%%%%%%%%%%%%%%%%%%%%%%%%%%%%%%%%%%%%%%%%%%%%%%%%%%%%%%%%%%
\subsection{Flags}
\label{sec:flags}

The package makes it easy to generate different versions
of the main or child documents.
To this end compilation flags can be defined
and assigned different default values.
They will be particularly useful in conjunction
with the forwarding mechanism described in \secref{sec:forward}.

For example, it may be useful to have a flag |\version|
which can be set to |draft| or |final|.
The document source will contain some conditional code
depending on the value of |\version|.
Suppose further, the flag should default to |final| for the main file
and to |draft| for child files
which is a natural assignment for editing the document.
This is achieved by placing the following code
in the preamble of the main document
(below the |\childdocmain| directive):
%
\begin{center}
\begin{tabular}{l}
|\ifchilddoc|\\
|\providecommand{\version}{draft}|\\
|\||else|\\
|\providecommand{\version}{final}|\\
|\||fi|
\end{tabular}
\end{center}
%
The definition by |\providecommand| makes sure
that previous definitions are not overwritten.
Further statements |\providecommand{\version}{...}|
can thus be added before the above code to override it.

For the main file, one might add a line
(between |\childdocmain| and the above block)
%
\begin{center}
|%\ifchilddoc\||else\providecommand{\version}{draft}\||fi|
\end{center}
%
which can be uncommented to produce a draft version.
Likewise one can add a line to the very top of a child file
(above the |\childdocof{|\textit{main}|}| directive)
%
\begin{center}
|%\providecommand{\version}{final}|
\end{center}
%
which can be uncommented to produce the final version of this child document.

%%%%%%%%%%%%%%%%%%%%%%%%%%%%%%%%%%%%%%%%%%%%%%%%%%%%%%%%%%%%%%%%%%%%%%%%%%%%%%%%
\subsection{Forwarding}
\label{sec:forward}

Different versions of the main or child documents
using compilation flags as described in \secref{sec:flags}
can be (permanently) stored in different files
for convenient compilation, viewing and distribution.
To this end, the package defines a command
to pass on compilation to a different file:

%%%%%%%%%%%%%%%%%%%%%%%%%%%%%%%%%%%%%%%%
\DescribeMacro{\childdocforward}
The command |\childdocforward| redirects processing to
another source file:
%
\begin{center}
\begin{tabular}{l}
|\input{childdoc.def}|\\
|\childdocforward[|\textit{main}|]{|\textit{dest}|}|\\
\end{tabular}
\end{center}
%
The argument \textit{dest} is the destination file
(without extension).
It should be the main file or one of the child files.
Note that further \textsf{childdoc} directives
such as |\childdocof| and |\childdocforward|
in the indicated file will be processed in this form.
The optional argument \textit{main}
passes on directly to the main file \textit{main}
while pretending to compile the child \textit{dest}.
This form behaves as if \textit{dest}
issues |\childdocof{|\textit{main}|}| right away,
and no further \textsf{childdoc} directives will be processed.

%%%%%%%%%%%%%%%%%%%%%%%%%%%%%%%%%%%%%%%%
\DescribeMacro{\...prefix}
In the alternative form |\childdocforwardprefix|,
%
\begin{center}
\begin{tabular}{l}
|\input{childdoc.def}|\\
|\childdocforwardprefix[|\textit{main}|]{|\textit{prefix}|}{|\textit{dest}|}|
\end{tabular}
\end{center}
%
the destination file is determined by a pattern
depending on the current file:
To make this work, the current file must be called
`{\textit{prefix}\hspace{0.2em}\textit{suffix}}'
with \textit{prefix} matching precisely the argument.
Processing is then passed on to the file
`{\textit{dest}\hspace{0.2em}\textit{suffix}}'.
Surely, the same effect is achieved by
directly specifying the
argument `{\textit{dest}\hspace{0.2em}\textit{suffix}}'
in the first form.
However, that requires to set up a different file
for each child. With the alternative form of the command
all these files can have exactly the same content
which simplifies setting them up and maintaining them.

For example, the following file |draft.tex|
with a compilation flag |\version| as described in \secref{sec:flags}
compiles the main document as a draft:
%
\begin{center}
\begin{tabular}{l}
|\def\version{draft}|\\
|\input{childdoc.def}|\\
|\childdocforward{|\textit{main}|}|
\end{tabular}
\end{center}
%
Likewise, the following files |final|\textit{nn}|.tex|
compile the final version of the child document
|child|\textit{nn}|.tex|:
%
\begin{center}
\begin{tabular}{l}
|\def\version{final}|\\
|\input{childdoc.def}|\\
|\childdocforwardprefix{final}{child}|
\end{tabular}
\end{center}
%

Note that when several versions of a main file and/or of each child file
are to be generated, it may be convenient to set up a |Makefile| or
shell script to automatise the process.

%%%%%%%%%%%%%%%%%%%%%%%%%%%%%%%%%%%%%%%%%%%%%%%%%%%%%%%%%%%%%%%%%%%%%%%%%%%%%%%%
\subsection{Command Line Processing}
\label{sec:commandline}

The effect of redirection files can also be achieved by invoking
the \LaTeX{} compiler with a more elaborate command line.
Most conveniently this should be done as part
of a shell script or a |Makefile|.

When using \textsf{childdoc} in the main file, the following
command lines effectively perform a redirection
(note that depending on the shell being used,
backslashes may have to be doubled: `|\|' $\to$ `|\\|'):
%
\begin{center}
|... -jobname "|\textit{target}|" |\\|"|[\textit{flags}]%
|\input{childdoc.def}\childdocforward[|\textit{main}|]{|\textit{dest}|}"|
\end{center}
%
Here \textit{target} is the name of the output file,
\textit{main} is the name of the main file
and \textit{dest} is the name of the main or child file to be processed
(all filenames without extensions).
The optional argument \textit{main} can be omitted
if \textit{main} matches \textit{dest}.
Optionally, compilation \textit{flags} can be defined via |\def| commands.
This command line makes the \TeX{} engine believe
it is compiling the file \textit{target}
whose content is specified as the latter parameter.
The provided code then forwards the processing to
\textit{main} or \textit{dest} as described in \secref{sec:forward}.

%%%%%%%%%%%%%%%%%%%%%%%%%%%%%%%%%%%%%%%%%%%%%%%%%%%%%%%%%%%%%%%%%%%%%%%%%%%%%%%%
\subsection{Include by Input}
\label{sec:input}

Including child documents by |\include| has some restrictions by design.
Most notably, the content of a child document always occupies
its own set of pages; pages cannot be shared between child documents.
Usually, this behaviour makes perfect sense
because each child document contain an essential part of the document.
However, in some situations it may be desirable to compose
a document from a collection of parts
without having mandatory page breaks between then.
For this case, the package
provides a mechanism to include parts
by |\input| which can also be processed individually.
However, by construction this mechanism
requires manual handling of the content to be output.

%%%%%%%%%%%%%%%%%%%%%%%%%%%%%%%%%%%%%%%%
\DescribeMacro{\ifchilddocmanual}
The main file should be prepared as usual, see \secref{sec:include}.
However, the document body must make a distinction
between processing of an individual part and of the main document, e.g.:
%
\begin{center}
\begin{tabular}{l}
|\ifchilddocmanual|\\
|\input{\childdocname}|\\
|\||else|\\
\textit{document body with }|\input{|\textit{part}|}|\\
|\||fi|
\end{tabular}
\end{center}
%
The conditional |\ifchilddocmanual| is true whenever
a part to be included by |\input| is being compiled,
and the name of the part is stored in |\childdocname|.

%%%%%%%%%%%%%%%%%%%%%%%%%%%%%%%%%%%%%%%%
\DescribeMacro{\childdocby}
Each part to be included by |\input| should start with:
%
\begin{center}
\begin{tabular}{l}
|\input{childdoc.def}|\\
|\childdocby{|\textit{main}|}|\\
\end{tabular}
\end{center}
%
The directive |\childdocby| is similar to |\childdocof|
described in \secref{sec:include},
but the subsequent selection of content must be done manually.
To that end, both |\ifchilddoc| and |\ifchilddocmanual|
will be true upon processing of a part,
and the name of the part is stored in |\childdocname|.
Note that |\jobname| will be set to the filename of the current part
so that each part receives an individual |.aux| file
that does not interfere with the |.aux| file(s) of the main document.
This behaviour can be altered by the alternative form
|\childdocby[*]{|\textit{main}|}| (with a non-empty optional argument)
which uses the |.aux| file of the main document
by setting |\jobname| to \textit{main}.

%%%%%%%%%%%%%%%%%%%%%%%%%%%%%%%%%%%%%%%%%%%%%%%%%%%%%%%%%%%%%%%%%%%%%%%%%%%%%%%%
\subsection{Driver Development}
\label{sec:driver}

The \textsf{childdoc} mechanism can also be use for the development
of definition files such as \LaTeX{} styles or classes.
This case differs from the above setup with multiple parts
included by |\include| in that no |\includeonly| should be invoked.
This can be achieved by starting the include file
(before |\ProvidesPackage|) with:
%
\begin{center}
\begin{tabular}{l}
|\input{childdoc.def}|\\
|\childdocforward{|\textit{main}|}|\\
\end{tabular}
\end{center}
%
or alternatively with:
%
\begin{center}
\begin{tabular}{l}
|\input{childdoc.def}|\\
|\childdocby{|\textit{main}|}|\\
\end{tabular}
\end{center}
%
Both forms have slightly different effects as described above.
The main file is prepared as usual, see \secref{sec:include}.

%%%%%%%%%%%%%%%%%%%%%%%%%%%%%%%%%%%%%%%%%%%%%%%%%%%%%%%%%%%%%%%%%%%%%%%%%%%%%%%%
\subsection{Legacy Detection}
\label{sec:detection}

The directive |\childdocmain| in the main file can detect
whether the complete document or merely a child is to be compiled
even without using the directive |\childdocof|.
This method is deprecated because it is less robust
and there is no compelling reason to use it;
it is merely provided for backward compatibility
and it may be removed in future versions.

If the detection mechanism is to be used,
it is mandatory to correctly specify
the filename of the main file as the argument of |\childdocmain|:
%
\begin{center}
\begin{tabular}{l}
|\input{childdoc.def}|\\
|\childdocmain{|\textit{main}|}|\\
\end{tabular}
\end{center}
%
If |\jobname| does not match the argument \textit{main} of |\childdocmain|,
it is assumed that |\jobname| points to the child file to be compiled.
When using |\childdocmain| with the main file specified as argument,
it suffices to start a child file
with just |\input{|\textit{main}|}|
without loading of the package and using |\childdocof|.
If instead all processing is done
with the appropriate \textsf{childdoc} directives,
the argument of \textit{main} of |\childdocmain| can be empty.

An alternative version of the command line processing described
in \secref{sec:commandline} using the detection mechanism reads:
%
\begin{center}
|... -jobname "|\textit{target}|" "|[\textit{flags}]%
[|\def\jobname{|\textit{dest}|}|]|\input{|\textit{main}|}"|
\end{center}

%%%%%%%%%%%%%%%%%%%%%%%%%%%%%%%%%%%%%%%%%%%%%%%%%%%%%%%%%%%%%%%%%%%%%%%%%%%%%%%%
\subsection{Manual Code}
\label{sec:manual}

In case one cannot be certain whether the definitions file |childdoc.def|
is installed on the target \TeX{} distribution
and one prefers not to ship it,
it is conceivable to paste a few relevant commands into the sources.

To that end, drop all statements |\input{childdoc.def}|
and perform the replacements as outlined below.
Instead of |\childdocmain{|\textit{main}|}| add the following code
to the top of the main file:
%
\begin{center}
\begin{tabular}{l}
|\||ifdefined\childdocname\endinput\||fi\newif\ifchilddoc|\\
|\edef\childdocname{\scantokens\expandafter{\jobname\noexpand}}|\\
|\def\childdocmain{|\textit{main}|}\||ifx\childdocmain\childdocname\||else|\\
|\childdoctrue\includeonly{\childdocname}\let\jobname\childdocmain\||fi|\\
\end{tabular}
\end{center}
%
Instead of |\childdocof{|\textit{main}|}| just include the main file
at the top of each child file:
%
\begin{center}
|\input{|\textit{main}|}|
\end{center}
%
A simple redirection |\childdocforward{|\textit{dest}|}| is achieved by:
%
\begin{center}
|\def\jobname{|\textit{dest}|}\input{\jobname}|
\end{center}
%
The redirection with prefix
|\childdocforwardprefix[|\textit{prefix}|]{|\textit{dest}|}|
is accomplished by:
%
\begin{center}
\begin{tabular}{l}
|{\edef\jobname{\scantokens\expandafter{\jobname\noexpand}}|\\
|\def\redirectjob |\textit{prefix}|#1~~~{\gdef\jobname{|\textit{dest}|#1}}|\\
|\expandafter\redirectjob\jobname~~~}\input{\jobname}|
\end{tabular}
\end{center}

In an alternative approach,
child documents can be compiled by a specific command line
without additional code or specific definitions:
%
\begin{center}
|... -jobname "|\textit{target}|" "|[\textit{flags}]%
|\includeonly{|\textit{dest}|}\input{|\textit{main}|}"|
\end{center}
%

%%%%%%%%%%%%%%%%%%%%%%%%%%%%%%%%%%%%%%%%%%%%%%%%%%%%%%%%%%%%%%%%%%%%%%%%%%%%%%%%
%%%%%%%%%%%%%%%%%%%%%%%%%%%%%%%%%%%%%%%%%%%%%%%%%%%%%%%%%%%%%%%%%%%%%%%%%%%%%%%%
\section{Information}

%%%%%%%%%%%%%%%%%%%%%%%%%%%%%%%%%%%%%%%%%%%%%%%%%%%%%%%%%%%%%%%%%%%%%%%%%%%%%%%%
\subsection{Copyright}

Copyright \copyright{} 2017--2018 Niklas Beisert

This work may be distributed and/or modified under the
conditions of the \LaTeX{} Project Public License, either version 1.3
of this license or (at your option) any later version.
The latest version of this license is in
  \url{http://www.latex-project.org/lppl.txt}
and version 1.3 or later is part of all distributions of \LaTeX{}
version 2005/12/01 or later.

This work has the LPPL maintenance status `maintained'.

The Current Maintainer of this work is Niklas Beisert.

This work consists of the files |README.txt|, |childdoc.ins| and |childdoc.dtx|
as well as the derived files |childdoc.def|, |cdocsamp.tex|
with |cdocsch1.tex|, |cdocsch2.tex|, |cdocspt3.tex|, |cdocspt4.tex|,
|cdocsdrf.tex|, |cdocsfn1.tex|, |cdocsfn2.tex|
as well as |childdoc.pdf|.

%%%%%%%%%%%%%%%%%%%%%%%%%%%%%%%%%%%%%%%%%%%%%%%%%%%%%%%%%%%%%%%%%%%%%%%%%%%%%%%%
\subsection{Files and Installation}

The package consists of the files:
%
\begin{center}
\begin{tabular}{ll}
    |README.txt|   & readme file \\
    |childdoc.ins| & installation file \\
    |childdoc.dtx| & source file \\
    |childdoc.def| & definition file \\
    |cdocsamp.tex| & sample main file \\
    |cdocsch1.tex| & sample include file \\
    |cdocsch2.tex| & sample include file \\
    |cdocspt3.tex| & sample part file \\
    |cdocspt4.tex| & sample part file \\
    |cdocsdrf.tex| & sample redirection file \\
    |cdocsfn1.tex| & sample redirection file \\
    |cdocsfn2.tex| & sample redirection file \\
    |childdoc.pdf| & manual
\end{tabular}
\end{center}
%
The distribution consists of the files
|README.txt|, |childdoc.ins| and |childdoc.dtx|.
%
\begin{itemize}
\item
Run (pdf)\LaTeX{} on |childdoc.dtx|
to compile the manual |childdoc.pdf| (this file).
\item
Run \LaTeX{} on |childdoc.ins| to create the definitions file |childdoc.def|
and the sample |cdocsamp.tex| with include files
|cdocsch1.tex|, |cdocsch2.tex|, |cdocspt3.tex|, |cdocspt4.tex|,
|cdocsdrf.tex|, |cdocsfn1.tex|, |cdocsfn2.tex|.
Then copy the file |childdoc.def| to an appropriate directory of your \LaTeX{}
distribution, e.g.\ \textit{texmf-root}|/tex/latex/childdoc|.
\end{itemize}

%%%%%%%%%%%%%%%%%%%%%%%%%%%%%%%%%%%%%%%%%%%%%%%%%%%%%%%%%%%%%%%%%%%%%%%%%%%%%%%%
\subsection{Related CTAN Packages}

There are several other packages which offer a similar functionality:
%
\begin{itemize}
\item
The packages
\href{http://ctan.org/pkg/docmute}{\textsf{docmute}},
\href{http://ctan.org/pkg/includex}{\textsf{includex}} and
\href{http://ctan.org/pkg/standalone}{\textsf{standalone}}
provide commands to include only the document body of
a child file thus allowing both files to be compiled individually.
\item
The packages \href{http://ctan.org/pkg/subdocs}{\textsf{subdocs}}
and \href{http://ctan.org/pkg/subfiles}{\textsf{subfiles}}
provide structures in which the main and child documents can be
encapsulated and allowing them to be compiled individually.
The inclusion mechanism is different from the conventional |\include|.
\item
The package \href{http://ctan.org/pkg/combine}{\textsf{combine}}
is an elaborate solution to combine several documents into one.
\end{itemize}
%
See also the CTAN topic \href{http://ctan.org/topic/subdocs}{\textsf{subdocs}}
for further related packages.
The present package differs from the above solutions in that
a document structure constructed with the conventional |\include| mechanism
just needs two extra commands at the top of every file
such that all constituent files can be compiled individually.

%%%%%%%%%%%%%%%%%%%%%%%%%%%%%%%%%%%%%%%%%%%%%%%%%%%%%%%%%%%%%%%%%%%%%%%%%%%%%%%%
%\subsection{Feature Suggestions}
%
%The following is a list of features which may be useful for future
%versions of this package:
%%
%\begin{itemize}
%\item
%\ldots
%\end{itemize}

%%%%%%%%%%%%%%%%%%%%%%%%%%%%%%%%%%%%%%%%%%%%%%%%%%%%%%%%%%%%%%%%%%%%%%%%%%%%%%%%
\subsection{Revision History}

%%%%%%%%%%%%%%%%%%%%%%%%%%%%%%%%%%%%%%%%
\paragraph{v2.0:} 2018/12/30

\begin{itemize}
\item
immediate forward processing
\item
added |\childdocby| mechanism
\item
manual restructured
\end{itemize}

%%%%%%%%%%%%%%%%%%%%%%%%%%%%%%%%%%%%%%%%
\paragraph{v1.6:} 2018/01/17

\begin{itemize}
\item
application for development of include files
\item
corrections to manual
\end{itemize}

%%%%%%%%%%%%%%%%%%%%%%%%%%%%%%%%%%%%%%%%
\paragraph{v1.5:} 2017/05/21

\begin{itemize}
\item
more complete structuring introduced
\item
|\childdocof| introduced
\item
|\childdoc| renamed to |\childdocmain|
\item
|\childredirect| renamed to |\childdocforward| and |\childdocforwardprefix|
and functionality expanded
\end{itemize}

%%%%%%%%%%%%%%%%%%%%%%%%%%%%%%%%%%%%%%%%
\paragraph{v1.0:} 2017/04/27

\begin{itemize}
\item
manual and install package
\item
first version published on CTAN
\end{itemize}

%%%%%%%%%%%%%%%%%%%%%%%%%%%%%%%%%%%%%%%%
\paragraph{v0.6:} 2017/04/26

\begin{itemize}
\item
redirection mechanism added
\end{itemize}

%%%%%%%%%%%%%%%%%%%%%%%%%%%%%%%%%%%%%%%%
\paragraph{v0.5:} 2017/04/26

\begin{itemize}
\item
functionality in definition file
\end{itemize}


%%%%%%%%%%%%%%%%%%%%%%%%%%%%%%%%%%%%%%%%%%%%%%%%%%%%%%%%%%%%%%%%%%%%%%%%%%%%%%%%
%%%%%%%%%%%%%%%%%%%%%%%%%%%%%%%%%%%%%%%%%%%%%%%%%%%%%%%%%%%%%%%%%%%%%%%%%%%%%%%%
%%%%%%%%%%%%%%%%%%%%%%%%%%%%%%%%%%%%%%%%%%%%%%%%%%%%%%%%%%%%%%%%%%%%%%%%%%%%%%%%
\appendix

\settowidth\MacroIndent{\rmfamily\scriptsize 000\ }

 \DocInput{childdoc.dtx}

\end{document}
%</driver>
% \fi
%
% %%%%%%%%%%%%%%%%%%%%%%%%%%%%%%%%%%%%%%%%%%%%%%%%%%%%%%%%%%%%%%%%%%%%%%%%%%%%%%
% %%%%%%%%%%%%%%%%%%%%%%%%%%%%%%%%%%%%%%%%%%%%%%%%%%%%%%%%%%%%%%%%%%%%%%%%%%%%%%
% \section{Sample}
%\iffalse
%<*samplemain>
%\fi
%
% The following presents a sample document
% with two chapters, two parts, a title page,
% a compile flag as well as three forwarding files to set the flag.
% It consists of eight |.tex| files:
% \begin{center}
% \begin{tabular}{ll}
% |cdocsamp.tex|&main file\\
% |cdocsch1.tex|&include file for chapter 1\\
% |cdocsch2.tex|&include file for chapter 2\\
% |cdocspt3.tex|&include file for part 3\\
% |cdocspt4.tex|&include file for part 4\\
% |cdocsdrf.tex|&forwarding file for main file in draft mode\\
% |cdocsfi1.tex|&forwarding file for final version of chapter 1\\
% |cdocsfi2.tex|&forwarding file for final version of chapter 2\\
% \end{tabular}
% \end{center}
% Each of the eight files can be compiled directly by the \LaTeX{} compiler.
%
% %%%%%%%%%%%%%%%%%%%%%%%%%%%%%%%%%%%%%%
% \paragraph{Main File.}
%
% The main file is called |cdocsamp.tex|.
%
% Load the \textsf{childdoc} definitions and
% declare the filename for the main document:
%    \begin{macrocode}
\input{childdoc.def}
\childdocmain{}
%    \end{macrocode}

% Optional override for |\version| flag:
%    \begin{macrocode}
%%\ifchilddoc\else\providecommand{\version}{draft}\fi
%    \end{macrocode}

% Define the default values for the |\version| flag
% (|final| for the main file and |draft| for childs):
%    \begin{macrocode}
\ifchilddoc
\providecommand{\version}{draft}
\else
\providecommand{\version}{final}
\fi
%    \end{macrocode}

% Load the standard document class:
%    \begin{macrocode}
\documentclass[12pt]{article}
%    \end{macrocode}

% Start the document body:
%    \begin{macrocode}
\begin{document}
%    \end{macrocode}

% Declare a title page.
% Print title, part of document being processed and version flag:
%    \begin{macrocode}
\addtocounter{page}{-1}
\begin{center}
{\LARGE\bfseries{}childdoc example\par}
\vspace{1cm}
\ifchilddoc
\ifchilddocmanual part\else chapter\fi:
`\childdocname' of `\childdocjob'\par
\else
main document: `\childdocjob'\par
\fi
version: \version\par
\end{center}
\newpage
%    \end{macrocode}

% Manually include selected file,
% otherwise process as usual:
%    \begin{macrocode}
\ifchilddocmanual
\section*{part `\childdocname'}
\input{\childdocname}
\else
%    \end{macrocode}

% Include the two chapters:
%    \begin{macrocode}
\include{cdocsch1}
\include{cdocsch2}
%    \end{macrocode}

% Include the two parts unless only chapters should be displayed:
%    \begin{macrocode}
\ifchilddoc\else
\section{part three}
\input{cdocspt3}
\section{part four}
\input{cdocspt4}
\fi
%    \end{macrocode}

% Process as usual until here:
%    \begin{macrocode}
\fi
%    \end{macrocode}

% End of document body:
%    \begin{macrocode}
\end{document}
%    \end{macrocode}
%\iffalse
%</samplemain>
%\fi
%
% %%%%%%%%%%%%%%%%%%%%%%%%%%%%%%%%%%%%%%
% \paragraph{Chapter Include Files.}
%
% The include files are called |cdocsch1.tex| and |cdocsch2.tex|.
%
%\iffalse
%<*samplechap1|samplechap2>
%\fi

% Optional override for |\version| flag:
%    \begin{macrocode}
%%\providecommand{\version}{final}
%    \end{macrocode}

% Include the main document:
%    \begin{macrocode}
\input{childdoc.def}
\childdocof{cdocsamp}
%    \end{macrocode}

%\iffalse
%</samplechap1|samplechap2>
%\fi
%
%\iffalse
%<*samplechap1>
%\fi
% Some text for chapter 1:
%    \begin{macrocode}
\section{one}
some text in chapter one
%    \end{macrocode}

%\iffalse
%</samplechap1>
%\fi
% Some text for chapter 2:
%\iffalse
%<*samplechap2>
%\fi
%    \begin{macrocode}
\section{two}
more text in chapter two
%    \end{macrocode}

%\iffalse
%</samplechap2>
%\fi
%
% %%%%%%%%%%%%%%%%%%%%%%%%%%%%%%%%%%%%%%
% \paragraph{Part Include Files.}
%
% The include files are called |cdocspt3.tex| and |cdocspt4.tex|.
%
%\iffalse
%<*samplepart3|samplepart4>
%\fi

% Optional override for |\version| flag:
%    \begin{macrocode}
%%\providecommand{\version}{final}
%    \end{macrocode}

% Include the main document:
%    \begin{macrocode}
\input{childdoc.def}
\childdocby{cdocsamp}
%    \end{macrocode}

%\iffalse
%</samplepart3|samplepart4>
%\fi
%
%\iffalse
%<*samplepart3>
%\fi
% Some text for part 3:
%    \begin{macrocode}
some text in part three
%    \end{macrocode}

%\iffalse
%</samplepart3>
%\fi
% Some text for part 4:
%\iffalse
%<*samplepart4>
%\fi
%    \begin{macrocode}
more text in part four
%    \end{macrocode}

%\iffalse
%</samplepart4>
%\fi
%
% %%%%%%%%%%%%%%%%%%%%%%%%%%%%%%%%%%%%%%
% \paragraph{Forwarding for a Complete Draft.}
%
% The following forwarding file |cdocsdrf.tex|
% compiles the main document in draft mode:
%\iffalse
%<*sampledraft>
%\fi
%    \begin{macrocode}
\def\version{draft}
\input{childdoc.def}
\childdocforward{cdocsamp}
%    \end{macrocode}

%\iffalse
%</sampledraft>
%\fi
%
% %%%%%%%%%%%%%%%%%%%%%%%%%%%%%%%%%%%%%%
% \paragraph{Forwarding for Final Version of the Chapters.}
%
% The following forwarding files |cdocsfn1.tex| and |cdocsfn2.tex|
% (with identical content)
% compile the final versions of the child documents
% |cdocsch1.tex| and |cdocsch2.tex|, respectively:
%\iffalse
%<*samplefinal>
%\fi
%    \begin{macrocode}
\def\version{final}
\input{childdoc.def}
\childdocforwardprefix[cdocsamp]{cdocsfn}{cdocsch}
%    \end{macrocode}

%\iffalse
%</samplefinal>
%\fi
%
% %%%%%%%%%%%%%%%%%%%%%%%%%%%%%%%%%%%%%%
% \paragraph{Command Line Processing.}
%
% The following three command lines generate the output files
% |cdocscld|, |cdocscl1| and |cdocscl2|
% which should be identical to
% |cdocsdrf|, |cdocsch1| and |cdocsfn2|, respectively:
% \begin{center}
% \begin{tabular}{l}
% |latex -jobname cdocscld \|\\
% |  "\def\version{draft}\input{childdoc.def}\childdocforward{cdocsamp}"|\\
% |latex -jobname cdocscl1 \|\\
% |  "\input{childdoc.def}\childdocforward[cdocsamp]{cdocsch1}"|\\
% |latex -jobname cdocscl2 \|\\
% |  "\def\version{final}\input{childdoc.def}\childdocforward{cdocsch2}"|
% \end{tabular}
% \end{center}
% Note that the trailing backslash on each first line
% merely continues the input to the second line
% (for convenient cut ant paste).
% Furthermore, the command |latex| can be replaced by any
% of its alternative versions such as |pdflatex|.
%
% %%%%%%%%%%%%%%%%%%%%%%%%%%%%%%%%%%%%%%%%%%%%%%%%%%%%%%%%%%%%%%%%%%%%%%%%%%%%%%
% %%%%%%%%%%%%%%%%%%%%%%%%%%%%%%%%%%%%%%%%%%%%%%%%%%%%%%%%%%%%%%%%%%%%%%%%%%%%%%
% \section{Implementation}
%\iffalse
%<*package>
%\fi
%
% This section describes the definitions file |childdoc.def|.

% The definitions cannot be loaded using |\usepackage| or |\RequirePackage|
% which has a mechanism to prevent loading a style file more than once.
% When loading the definitions by means of |\input|
% multiple instances have to be prevented manually:
%\iffalse
%This code needs to be before the `\ProvidesFile' directive
%which is defined at the beginning of this file.
%Therefore it is also placed there and commented out here.
%</package>
%<*discard>
%\fi
%    \begin{macrocode}
\ifdefined\childdocmain\endinput\fi
%    \end{macrocode}
%\iffalse
%</discard>
%<*package>
%\fi
%
% \macro{\ifchilddoc}
% \macro{\ifchilddocmanual}
% The conditional |\ifchilddoc| tells whether a
% child (true) or main (false) document is being compiled.
% The conditional |\ifchilddocmanual| tells whether
% the |\includeonly| mechanism is used (false) or
% the selection of child files must be performed manually (true).
% The definitions initialise to false:
%    \begin{macrocode}
\newif\ifchilddoc
\newif\ifchilddocmanual
%    \end{macrocode}

% \macro{\childdocname}
% \macro{\childdocjob}
% The macro |\childdocname| stores the name of the main document
% to be compiled. The macro |\childdocjob| stores the name of
% the document on which the \LaTeX{} compiler was originally invoked.
% The content of |\jobname| cannot be compared
% to filenames specified in the source due to different catcodes.
% The following code rescans |\jobname|, stores the result
% in |\childdocname| and saves a copy in |\childdocjob|:
%    \begin{macrocode}
\edef\childdocname{\scantokens\expandafter{\jobname\noexpand}}
\let\childdocjob\childdocname
%    \end{macrocode}

% \macro{\childdocdisable}
% The macro |\childdocdisable| prevents the main file
% from being processed more than once.
% At this stage, the main document command |\childdocmain|
% is assumed to be called once again where it should do nothing.
% Any subsequent call to it should prevent
% a secondary processing of the main document
% It overwrites the forwarding commands
% |\childdocof| and |\childdocforward|
% with empty macros to prevent further inclusions of the main document:
%    \begin{macrocode}
\newcommand{\childdocdisable}
{
  \renewcommand{\childdocmain}[1]{\renewcommand{\childdocmain}[1]{\endinput}}
  \renewcommand{\childdocof}[1]{}
  \renewcommand{\childdocby}[2][]{}
  \renewcommand{\childdocforward}[2][]{}
  \renewcommand{\childdocdisable}{}
}
%    \end{macrocode}

% \macro{\childdocmain}
% The macro |\childdocmain| is to be called at the top of the main file
% with nothing or the main filename (without extension) as argument.
% First, it breaks loops.
% If the argument is not empty and does not match |\childdocname|
% (which is set by the first inclusion of |childdoc.def|),
% |\ifchilddoc| is set to true, |\includeonly| is applied to the child file
% and |\jobname| is set to the main file
% (for proper handling of |.aux| files):
%    \begin{macrocode}
\newcommand{\childdocmain}[1]
{
  \childdocdisable\childdocmain{}
  \if?#1?\else
    \begingroup
      \def\childdoctmp{#1}
      \ifx\childdoctmp\childdocname
        \def\childdoctmp{}
      \else
        \def\childdoctmp
        {
          \childdoctrue
          \includeonly{\childdocname}
          \def\childdocjob{#1}
          \def\jobname{#1}
        }
      \fi
      \expandafter
    \endgroup
    \childdoctmp
  \fi
}
%    \end{macrocode}

% \macro{\childdocof}
% The command |\childdocof| redirects
% compilation to the main file |#1|.
%    \begin{macrocode}
\newcommand{\childdocof}[1]
{
  \childdocdisable
  \childdoctrue
  \includeonly{\childdocname}
  \def\jobname{#1}
  \def\childdocjob{#1}
  \input{#1}
}
%    \end{macrocode}

% \macro{\childdocby}
% The command |\childdocby| ....
%    \begin{macrocode}
\newcommand{\childdocby}[2][]
{
  \childdocdisable
  \childdoctrue
  \childdocmanualtrue
  \if?#1?\else
    \def\jobname{#2}
  \fi
  \def\childdocjob{#2}
  \input{#2}
  \endinput
}
%    \end{macrocode}

% \macro{\childdocforward}
% The command |\childdocforward| redirects
% compilation to the main file or
% (if the optional argument is given) a child file.
% Parameters are set as if the main file
% or a child file starting with |\childdocof| was compiled.
% Then compilation is handed over to the main file:
%    \begin{macrocode}
\newcommand{\childdocforward}[2][]
{
  \begingroup
    \if?#1?
      \def\childdoctmp
      {
        \def\childdocname{#2}
        \def\childdocjob{#2}
        \def\jobname{#2}
        \input{#2}
        \endinput
      }
    \else
      \def\childdoctmp
      {
        \childdocdisable
        \def\childdocname{#2}
        \childdoctrue
        \includeonly{#2}
        \def\childdocjob{#1}
        \def\jobname{#1}
        \input{#1}
        \endinput
      }
    \fi
    \expandafter
  \endgroup
  \childdoctmp
}
%    \end{macrocode}

% \macro{\childdocforwardprefix}
% The command |\childdocforwardprefix| redirects
% compilation to the main or a child file by means of a pattern.
% The prefix |#1| in the current filename is replaced by |#2|
% and the suffix of the current filename is kept
% (it is assumed that the filename does not contain the substring `|~~~|'
% which is used as a delimiter).
% Compilation is handed over to the new file by |\childdocforward|:
%    \begin{macrocode}
\newcommand{\childdocforwardprefix}[3][]
{
  \begingroup
    \def\childdocextract #2##1~~~{\def\childdoctmp{\childdocforward[#1]{#3##1}}}
    \expandafter\childdocextract\childdocname~~~
    \expandafter
  \endgroup
  \childdoctmp
}
%    \end{macrocode}

% \macro{\childdoc}
% The deprecated macro |\childdoc| is a legacy version of |\childdocmain|:
%    \begin{macrocode}
\newcommand{\childdoc}{\childdocmain}
%    \end{macrocode}

% \macro{\childdocredirect}
% The deprecated macro |\childdocredirect| is a legacy version
% of |\childdocforward| and |\childdocforwardprefix|:
%    \begin{macrocode}
\newcommand{\childdocredirect}[2][]
{
  \begingroup
    \if?#1?
      \def\childdoctmp{\childdocforward{#2}}
    \else
      \def\childdoctmp{\childdocforwardprefix{#1}{#2}}
    \fi
    \expandafter
  \endgroup
  \childdoctmp
}
%    \end{macrocode}

%\iffalse
%</package>
%\fi
%
\endinput
|\\
|\childdocof{|\textit{main}|}|\\
\end{tabular}
\end{center}
at the top of every child file \textit{child}
which is included by |\include{|\textit{child}|}|
from within the main file
(or at least for those files to be compiled individually).
The argument \textit{main} must be the filename of the main file.

There are a couple of
considerations in setting up the main and child documents:

%%%%%%%%%%%%%%%%%%%%%%%%%%%%%%%%%%%%%%%%
\paragraph{Restrictions.}

Please note the following restrictions:
\begin{itemize}
\item
|\childdocmain| must be called with one argument \textit{main}
to ensure compatibility with earlier version of the package.
It must either be empty (|\childdocmain{}|)
or precisely match the filename of the main file in which it is specified.
See \secref{sec:detection} for further information.
\item
The filename \textit{main} must be specified without the |.tex| extension.
\item
The filename \textit{main} is case sensitive
(even in case-insensitive file systems)
due to internal string comparison.
\item
The argument \textit{main} should be fully expanded, it cannot be a macro.
\item
Subdirectories and special characters should be avoided in filenames.
\item
The command |\childdocmain{|\textit{main}|}| must be followed by a whitespace.
It should not be followed immediately by another command
or by a comment mark `|%|'.
This is because the \TeX{} parser reads the token immediately following
the argument of |\childdocmain| and puts it
at the beginning of every child section;
however, a white\-space is ignored.
\end{itemize}

%%%%%%%%%%%%%%%%%%%%%%%%%%%%%%%%%%%%%%%%
\paragraph{Content of Main File.}

It is advisable to place all content in the child files included by |\include|.
Any output contained in the main file will appear in all child documents
unless suppressed manually;
it cannot be suppressed automatically by the |\includeonly| directive
and thus should normally be avoided.
A method to include some content in the main file
by means of conditional processing is described in \secref{sec:conditional}.

%%%%%%%%%%%%%%%%%%%%%%%%%%%%%%%%%%%%%%%%
\paragraph{Page Numbering.}

When only a part of the document is compiled,
the appropriate numbering of pages
(as well as other status parameters)
is determined from the |.aux| files.
The latter contain information from previous passes.
However this information needs to propagate through
all intermediate child documents.
Therefore the page numbering in child documents may well
be inconsistent until the complete document is compiled at least once.

A useful (if unconventional) way to always ensure a consistent
page numbering is to restart the numbering in each child document
and denote the pages by `\textit{child}|.|\textit{page}'
where \textit{child} represents the chapter/section number of the child file.
This can be achieved by the command
|\numberwithin{page}{|\textit{child}|}|
of the \textsf{amsmath} package
where \textit{child} can be |chapter| or |section|
depending on the chosen structuring.
Alternatively, one can modify the macro |\thepage| appropriately
and reset the counter |page| at the start of each child file.

%%%%%%%%%%%%%%%%%%%%%%%%%%%%%%%%%%%%%%%%%%%%%%%%%%%%%%%%%%%%%%%%%%%%%%%%%%%%%%%%
\subsection{Conditional Processing}
\label{sec:conditional}

The package provides a mechanism to compile different versions
of a document. To customise the versions further some conditional processing
can come in handy to distinguish which version is being compiled.
The package provides two macros to describe the compilation context:

%%%%%%%%%%%%%%%%%%%%%%%%%%%%%%%%%%%%%%%%
\DescribeMacro{\ifchilddoc}
The conditional |\ifchilddoc| distinguishes between the compilation of
child documents and the main document:
%
\begin{center}
|\ifchilddoc |\textit{child-code}| |[|\||else |\textit{main-code}]| \||fi|
\end{center}

%%%%%%%%%%%%%%%%%%%%%%%%%%%%%%%%%%%%%%%%
\DescribeMacro{\childdocname}
\DescribeMacro{\childdocjob}
The macro |\childdocname| contains the filename (without extension)
of the main or child file being processed.
Note that |\childdocjob| will always contain the name of the main file.

%%%%%%%%%%%%%%%%%%%%%%%%%%%%%%%%%%%%%%%%
\paragraph{Title Page.}

Conditional processing can be used to include a title or banner page
in the main document when proper precautions are taken.
Importantly, the code in the main file should ensure that the page counter
(as well as other status parameters which are stored in the |.aux| files)
takes the same value after the conditional processing.
Otherwise the page numbers may take divergent values
depending on which part is compiled.

For example, a title page could be declared by:
%
\begin{center}
\begin{tabular}{l}
|\ifchilddoc\||else|\\
|\addtocounter{page}{-1}|\\
\textit{code for title page}\\
|\newpage|\\
|\||fi|
\end{tabular}
\end{center}
%
A banner page for the child documents can be generated by:
%
\begin{center}
\begin{tabular}{l}
|\ifchilddoc|\\
|\addtocounter{page}{-1}|\\
\textit{code for banner page}\\
|\newpage|\\
|\||fi|
\end{tabular}
\end{center}
%
Here one could write a message such as:
\begin{center}
|This is the part \childdocname{} of \childdocjob{}.|
\end{center}

%%%%%%%%%%%%%%%%%%%%%%%%%%%%%%%%%%%%%%%%%%%%%%%%%%%%%%%%%%%%%%%%%%%%%%%%%%%%%%%%
\subsection{Flags}
\label{sec:flags}

The package makes it easy to generate different versions
of the main or child documents.
To this end compilation flags can be defined
and assigned different default values.
They will be particularly useful in conjunction
with the forwarding mechanism described in \secref{sec:forward}.

For example, it may be useful to have a flag |\version|
which can be set to |draft| or |final|.
The document source will contain some conditional code
depending on the value of |\version|.
Suppose further, the flag should default to |final| for the main file
and to |draft| for child files
which is a natural assignment for editing the document.
This is achieved by placing the following code
in the preamble of the main document
(below the |\childdocmain| directive):
%
\begin{center}
\begin{tabular}{l}
|\ifchilddoc|\\
|\providecommand{\version}{draft}|\\
|\||else|\\
|\providecommand{\version}{final}|\\
|\||fi|
\end{tabular}
\end{center}
%
The definition by |\providecommand| makes sure
that previous definitions are not overwritten.
Further statements |\providecommand{\version}{...}|
can thus be added before the above code to override it.

For the main file, one might add a line
(between |\childdocmain| and the above block)
%
\begin{center}
|%\ifchilddoc\||else\providecommand{\version}{draft}\||fi|
\end{center}
%
which can be uncommented to produce a draft version.
Likewise one can add a line to the very top of a child file
(above the |\childdocof{|\textit{main}|}| directive)
%
\begin{center}
|%\providecommand{\version}{final}|
\end{center}
%
which can be uncommented to produce the final version of this child document.

%%%%%%%%%%%%%%%%%%%%%%%%%%%%%%%%%%%%%%%%%%%%%%%%%%%%%%%%%%%%%%%%%%%%%%%%%%%%%%%%
\subsection{Forwarding}
\label{sec:forward}

Different versions of the main or child documents
using compilation flags as described in \secref{sec:flags}
can be (permanently) stored in different files
for convenient compilation, viewing and distribution.
To this end, the package defines a command
to pass on compilation to a different file:

%%%%%%%%%%%%%%%%%%%%%%%%%%%%%%%%%%%%%%%%
\DescribeMacro{\childdocforward}
The command |\childdocforward| redirects processing to
another source file:
%
\begin{center}
\begin{tabular}{l}
|% \iffalse
%
% childdoc.dtx Copyright (C) 2017-2018 Niklas Beisert
%
% This work may be distributed and/or modified under the
% conditions of the LaTeX Project Public License, either version 1.3
% of this license or (at your option) any later version.
% The latest version of this license is in
%   http://www.latex-project.org/lppl.txt
% and version 1.3 or later is part of all distributions of LaTeX
% version 2005/12/01 or later.
%
% This work has the LPPL maintenance status `maintained'.
%
% The Current Maintainer of this work is Niklas Beisert.
%
% This work consists of the files childdoc.dtx and childdoc.ins
% and the derived files childdoc.def and cdocsamp.tex with
% cdocsch1.tex, cdocsch2.tex, cdocsdrf.tex, cdocsfn1.tex, cdocsfn2.tex.
%
%<package>\ifdefined\childdocmain\endinput\fi
%<package>\ProvidesFile{childdoc.def}[2018/12/30 v2.0 child document driver]
%<samplemain>\ProvidesFile{cdocsamp.tex}[2018/12/30 v2.0 sample for childdoc]
%<*driver>
%\ProvidesFile{childdoc.drv}[2018/12/30 v2.0 childdoc reference manual file]
\PassOptionsToClass{10pt,a4paper}{article}
\documentclass{ltxdoc}

\usepackage[margin=35mm]{geometry}
\usepackage{hyperref}
\usepackage{hyperxmp}
\usepackage[usenames]{color}

\hypersetup{colorlinks=true}
\hypersetup{pdfstartview=FitH}
\hypersetup{pdfpagemode=UseNone}
\hypersetup{pdfsource={}}
\hypersetup{pdflang={en-UK}}
\hypersetup{pdfcopyright={Copyright 2017-2018 Niklas Beisert.
  This work may be distributed and/or modified under the
  conditions of the LaTeX Project Public License, either version 1.3
  of this license or (at your option) any later version.}}
\hypersetup{pdflicenseurl={http://www.latex-project.org/lppl.txt}}
\hypersetup{pdfcontactaddress={ETH Zurich, ITP, HIT K,
  Wolfgang-Pauli-Strasse 27}}
\hypersetup{pdfcontactpostcode={8093}}
\hypersetup{pdfcontactcity={Zurich}}
\hypersetup{pdfcontactcountry={Switzerland}}
\hypersetup{pdfcontactemail={nbeisert@itp.phys.ethz.ch}}
\hypersetup{pdfcontacturl={http://people.phys.ethz.ch/\xmptilde nbeisert/}}

\newcommand{\secref}[1]{\hyperref[#1]{section \ref*{#1}}}

\parskip1ex
\parindent0pt
\let\olditemize\itemize
\def\itemize{\olditemize\parskip0pt}

\begin{document}

\title{The \textsf{childdoc} Package}
\hypersetup{pdftitle={The childdoc Package}}
\author{Niklas Beisert\\[2ex]
  Institut f\"ur Theoretische Physik\\
  Eidgen\"ossische Technische Hochschule Z\"urich\\
  Wolfgang-Pauli-Strasse 27, 8093 Z\"urich, Switzerland\\[1ex]
  \href{mailto:nbeisert@itp.phys.ethz.ch}
  {\texttt{nbeisert@itp.phys.ethz.ch}}}
\hypersetup{pdfauthor={Niklas Beisert}}
\hypersetup{pdfsubject={Manual for the LaTeX2e Package childdoc}}
\date{30 December 2018, \textsf{v2.0}}
\maketitle

\begin{abstract}\noindent
\textsf{childdoc} is a \LaTeXe{} package
that enables the direct compilation
of document sections included by |\include|
to individual files.
\end{abstract}

\begingroup
\parskip0ex
\tableofcontents
\endgroup

%%%%%%%%%%%%%%%%%%%%%%%%%%%%%%%%%%%%%%%%%%%%%%%%%%%%%%%%%%%%%%%%%%%%%%%%%%%%%%%%
%%%%%%%%%%%%%%%%%%%%%%%%%%%%%%%%%%%%%%%%%%%%%%%%%%%%%%%%%%%%%%%%%%%%%%%%%%%%%%%%
\section{Introduction}

\LaTeX{} provides a mechanism to structure a large document (such as a book)
into a main file and several child files (containing the chapters)
using the |\include| command.
This mechanism is beneficial for documents
which span hundreds of pages in order to
make the source file(s) more manageable.
Moreover, compilation can be restricted to
selected child files by means of the |\includeonly| command.
The latter feature can be used to reduce the compilation time while editing
(this was significantly more useful in the earlier days of \LaTeX{})
or to generate a smaller document which is easier to navigate.
Another application of |\includeonly| is to generate
documents consisting of selected parts of the complete document.

However, there are a few drawbacks of the plain |\include| mechanism:
\begin{itemize}
\item
The child files cannot be compiled on their own,
they can only be compiled via the main file.
A naive editing environment
(such as a text editor with an option
to have the current file processed by \LaTeX)
may require one to switch to the main file before compiling;
attempting to compile the child file produces errors.
\item
The main file must be modified (each time)
to adjust the |\includeonly| command
to the present needs. This easily leaves the main file in a messy state.
\item
The generated document will always carry the filename
of the main document. This is inconvenient if
several child files are to be compiled and
to be kept for distribution.
\end{itemize}

The present package provides a simple interface
to make child files individually compilable by \LaTeX{}.
Compiling a child file then has the same effect as compiling
the main file with an |\includeonly| command
to select the appropriate child.
Moreover the generated document will carry the name of the child
rather than the main file.
This resolves all three above issues.

This feature is meant to make the editing of books,
thesis documents and lecture notes somewhat more convenient.
However, the package can also be used efficiently for
composing a series of documents (such as exercise sheets)
which are typically distributed individually.
It then assists the author in generating the individual documents
(potentially in different versions)
as well as a document containing the collected series.
Another application is in developing style files
or other kinds of included material
where compilation of the style file could redirect
to a sample or test file.

%%%%%%%%%%%%%%%%%%%%%%%%%%%%%%%%%%%%%%%%%%%%%%%%%%%%%%%%%%%%%%%%%%%%%%%%%%%%%%%%
%%%%%%%%%%%%%%%%%%%%%%%%%%%%%%%%%%%%%%%%%%%%%%%%%%%%%%%%%%%%%%%%%%%%%%%%%%%%%%%%
\section{Usage}

First of all, the package \textsf{childdoc} is \emph{not} a standard
\LaTeXe{} |.sty| style file! Therefore it needs to be invoked in
a non-standard way.

%%%%%%%%%%%%%%%%%%%%%%%%%%%%%%%%%%%%%%%%%%%%%%%%%%%%%%%%%%%%%%%%%%%%%%%%%%%%%%%%
\subsection{Included Files}
\label{sec:include}

%%%%%%%%%%%%%%%%%%%%%%%%%%%%%%%%%%%%%%%%
\DescribeMacro{\childdocmain}
To use the package, add the commands
\begin{center}
\begin{tabular}{l}
|\input{childdoc.def}|\\
|\childdocmain{}|\\
\end{tabular}
\end{center}
at the very top of the main \LaTeX{} file,
in particular \emph{before} the |\documentclass| statement!
The argument of |\childdocmain| should be left empty
(but it must be present).

%%%%%%%%%%%%%%%%%%%%%%%%%%%%%%%%%%%%%%%%
\DescribeMacro{\childdocof}
Furthermore, add the commands
\begin{center}
\begin{tabular}{l}
|\input{childdoc.def}|\\
|\childdocof{|\textit{main}|}|\\
\end{tabular}
\end{center}
at the top of every child file \textit{child}
which is included by |\include{|\textit{child}|}|
from within the main file
(or at least for those files to be compiled individually).
The argument \textit{main} must be the filename of the main file.

There are a couple of
considerations in setting up the main and child documents:

%%%%%%%%%%%%%%%%%%%%%%%%%%%%%%%%%%%%%%%%
\paragraph{Restrictions.}

Please note the following restrictions:
\begin{itemize}
\item
|\childdocmain| must be called with one argument \textit{main}
to ensure compatibility with earlier version of the package.
It must either be empty (|\childdocmain{}|)
or precisely match the filename of the main file in which it is specified.
See \secref{sec:detection} for further information.
\item
The filename \textit{main} must be specified without the |.tex| extension.
\item
The filename \textit{main} is case sensitive
(even in case-insensitive file systems)
due to internal string comparison.
\item
The argument \textit{main} should be fully expanded, it cannot be a macro.
\item
Subdirectories and special characters should be avoided in filenames.
\item
The command |\childdocmain{|\textit{main}|}| must be followed by a whitespace.
It should not be followed immediately by another command
or by a comment mark `|%|'.
This is because the \TeX{} parser reads the token immediately following
the argument of |\childdocmain| and puts it
at the beginning of every child section;
however, a white\-space is ignored.
\end{itemize}

%%%%%%%%%%%%%%%%%%%%%%%%%%%%%%%%%%%%%%%%
\paragraph{Content of Main File.}

It is advisable to place all content in the child files included by |\include|.
Any output contained in the main file will appear in all child documents
unless suppressed manually;
it cannot be suppressed automatically by the |\includeonly| directive
and thus should normally be avoided.
A method to include some content in the main file
by means of conditional processing is described in \secref{sec:conditional}.

%%%%%%%%%%%%%%%%%%%%%%%%%%%%%%%%%%%%%%%%
\paragraph{Page Numbering.}

When only a part of the document is compiled,
the appropriate numbering of pages
(as well as other status parameters)
is determined from the |.aux| files.
The latter contain information from previous passes.
However this information needs to propagate through
all intermediate child documents.
Therefore the page numbering in child documents may well
be inconsistent until the complete document is compiled at least once.

A useful (if unconventional) way to always ensure a consistent
page numbering is to restart the numbering in each child document
and denote the pages by `\textit{child}|.|\textit{page}'
where \textit{child} represents the chapter/section number of the child file.
This can be achieved by the command
|\numberwithin{page}{|\textit{child}|}|
of the \textsf{amsmath} package
where \textit{child} can be |chapter| or |section|
depending on the chosen structuring.
Alternatively, one can modify the macro |\thepage| appropriately
and reset the counter |page| at the start of each child file.

%%%%%%%%%%%%%%%%%%%%%%%%%%%%%%%%%%%%%%%%%%%%%%%%%%%%%%%%%%%%%%%%%%%%%%%%%%%%%%%%
\subsection{Conditional Processing}
\label{sec:conditional}

The package provides a mechanism to compile different versions
of a document. To customise the versions further some conditional processing
can come in handy to distinguish which version is being compiled.
The package provides two macros to describe the compilation context:

%%%%%%%%%%%%%%%%%%%%%%%%%%%%%%%%%%%%%%%%
\DescribeMacro{\ifchilddoc}
The conditional |\ifchilddoc| distinguishes between the compilation of
child documents and the main document:
%
\begin{center}
|\ifchilddoc |\textit{child-code}| |[|\||else |\textit{main-code}]| \||fi|
\end{center}

%%%%%%%%%%%%%%%%%%%%%%%%%%%%%%%%%%%%%%%%
\DescribeMacro{\childdocname}
\DescribeMacro{\childdocjob}
The macro |\childdocname| contains the filename (without extension)
of the main or child file being processed.
Note that |\childdocjob| will always contain the name of the main file.

%%%%%%%%%%%%%%%%%%%%%%%%%%%%%%%%%%%%%%%%
\paragraph{Title Page.}

Conditional processing can be used to include a title or banner page
in the main document when proper precautions are taken.
Importantly, the code in the main file should ensure that the page counter
(as well as other status parameters which are stored in the |.aux| files)
takes the same value after the conditional processing.
Otherwise the page numbers may take divergent values
depending on which part is compiled.

For example, a title page could be declared by:
%
\begin{center}
\begin{tabular}{l}
|\ifchilddoc\||else|\\
|\addtocounter{page}{-1}|\\
\textit{code for title page}\\
|\newpage|\\
|\||fi|
\end{tabular}
\end{center}
%
A banner page for the child documents can be generated by:
%
\begin{center}
\begin{tabular}{l}
|\ifchilddoc|\\
|\addtocounter{page}{-1}|\\
\textit{code for banner page}\\
|\newpage|\\
|\||fi|
\end{tabular}
\end{center}
%
Here one could write a message such as:
\begin{center}
|This is the part \childdocname{} of \childdocjob{}.|
\end{center}

%%%%%%%%%%%%%%%%%%%%%%%%%%%%%%%%%%%%%%%%%%%%%%%%%%%%%%%%%%%%%%%%%%%%%%%%%%%%%%%%
\subsection{Flags}
\label{sec:flags}

The package makes it easy to generate different versions
of the main or child documents.
To this end compilation flags can be defined
and assigned different default values.
They will be particularly useful in conjunction
with the forwarding mechanism described in \secref{sec:forward}.

For example, it may be useful to have a flag |\version|
which can be set to |draft| or |final|.
The document source will contain some conditional code
depending on the value of |\version|.
Suppose further, the flag should default to |final| for the main file
and to |draft| for child files
which is a natural assignment for editing the document.
This is achieved by placing the following code
in the preamble of the main document
(below the |\childdocmain| directive):
%
\begin{center}
\begin{tabular}{l}
|\ifchilddoc|\\
|\providecommand{\version}{draft}|\\
|\||else|\\
|\providecommand{\version}{final}|\\
|\||fi|
\end{tabular}
\end{center}
%
The definition by |\providecommand| makes sure
that previous definitions are not overwritten.
Further statements |\providecommand{\version}{...}|
can thus be added before the above code to override it.

For the main file, one might add a line
(between |\childdocmain| and the above block)
%
\begin{center}
|%\ifchilddoc\||else\providecommand{\version}{draft}\||fi|
\end{center}
%
which can be uncommented to produce a draft version.
Likewise one can add a line to the very top of a child file
(above the |\childdocof{|\textit{main}|}| directive)
%
\begin{center}
|%\providecommand{\version}{final}|
\end{center}
%
which can be uncommented to produce the final version of this child document.

%%%%%%%%%%%%%%%%%%%%%%%%%%%%%%%%%%%%%%%%%%%%%%%%%%%%%%%%%%%%%%%%%%%%%%%%%%%%%%%%
\subsection{Forwarding}
\label{sec:forward}

Different versions of the main or child documents
using compilation flags as described in \secref{sec:flags}
can be (permanently) stored in different files
for convenient compilation, viewing and distribution.
To this end, the package defines a command
to pass on compilation to a different file:

%%%%%%%%%%%%%%%%%%%%%%%%%%%%%%%%%%%%%%%%
\DescribeMacro{\childdocforward}
The command |\childdocforward| redirects processing to
another source file:
%
\begin{center}
\begin{tabular}{l}
|\input{childdoc.def}|\\
|\childdocforward[|\textit{main}|]{|\textit{dest}|}|\\
\end{tabular}
\end{center}
%
The argument \textit{dest} is the destination file
(without extension).
It should be the main file or one of the child files.
Note that further \textsf{childdoc} directives
such as |\childdocof| and |\childdocforward|
in the indicated file will be processed in this form.
The optional argument \textit{main}
passes on directly to the main file \textit{main}
while pretending to compile the child \textit{dest}.
This form behaves as if \textit{dest}
issues |\childdocof{|\textit{main}|}| right away,
and no further \textsf{childdoc} directives will be processed.

%%%%%%%%%%%%%%%%%%%%%%%%%%%%%%%%%%%%%%%%
\DescribeMacro{\...prefix}
In the alternative form |\childdocforwardprefix|,
%
\begin{center}
\begin{tabular}{l}
|\input{childdoc.def}|\\
|\childdocforwardprefix[|\textit{main}|]{|\textit{prefix}|}{|\textit{dest}|}|
\end{tabular}
\end{center}
%
the destination file is determined by a pattern
depending on the current file:
To make this work, the current file must be called
`{\textit{prefix}\hspace{0.2em}\textit{suffix}}'
with \textit{prefix} matching precisely the argument.
Processing is then passed on to the file
`{\textit{dest}\hspace{0.2em}\textit{suffix}}'.
Surely, the same effect is achieved by
directly specifying the
argument `{\textit{dest}\hspace{0.2em}\textit{suffix}}'
in the first form.
However, that requires to set up a different file
for each child. With the alternative form of the command
all these files can have exactly the same content
which simplifies setting them up and maintaining them.

For example, the following file |draft.tex|
with a compilation flag |\version| as described in \secref{sec:flags}
compiles the main document as a draft:
%
\begin{center}
\begin{tabular}{l}
|\def\version{draft}|\\
|\input{childdoc.def}|\\
|\childdocforward{|\textit{main}|}|
\end{tabular}
\end{center}
%
Likewise, the following files |final|\textit{nn}|.tex|
compile the final version of the child document
|child|\textit{nn}|.tex|:
%
\begin{center}
\begin{tabular}{l}
|\def\version{final}|\\
|\input{childdoc.def}|\\
|\childdocforwardprefix{final}{child}|
\end{tabular}
\end{center}
%

Note that when several versions of a main file and/or of each child file
are to be generated, it may be convenient to set up a |Makefile| or
shell script to automatise the process.

%%%%%%%%%%%%%%%%%%%%%%%%%%%%%%%%%%%%%%%%%%%%%%%%%%%%%%%%%%%%%%%%%%%%%%%%%%%%%%%%
\subsection{Command Line Processing}
\label{sec:commandline}

The effect of redirection files can also be achieved by invoking
the \LaTeX{} compiler with a more elaborate command line.
Most conveniently this should be done as part
of a shell script or a |Makefile|.

When using \textsf{childdoc} in the main file, the following
command lines effectively perform a redirection
(note that depending on the shell being used,
backslashes may have to be doubled: `|\|' $\to$ `|\\|'):
%
\begin{center}
|... -jobname "|\textit{target}|" |\\|"|[\textit{flags}]%
|\input{childdoc.def}\childdocforward[|\textit{main}|]{|\textit{dest}|}"|
\end{center}
%
Here \textit{target} is the name of the output file,
\textit{main} is the name of the main file
and \textit{dest} is the name of the main or child file to be processed
(all filenames without extensions).
The optional argument \textit{main} can be omitted
if \textit{main} matches \textit{dest}.
Optionally, compilation \textit{flags} can be defined via |\def| commands.
This command line makes the \TeX{} engine believe
it is compiling the file \textit{target}
whose content is specified as the latter parameter.
The provided code then forwards the processing to
\textit{main} or \textit{dest} as described in \secref{sec:forward}.

%%%%%%%%%%%%%%%%%%%%%%%%%%%%%%%%%%%%%%%%%%%%%%%%%%%%%%%%%%%%%%%%%%%%%%%%%%%%%%%%
\subsection{Include by Input}
\label{sec:input}

Including child documents by |\include| has some restrictions by design.
Most notably, the content of a child document always occupies
its own set of pages; pages cannot be shared between child documents.
Usually, this behaviour makes perfect sense
because each child document contain an essential part of the document.
However, in some situations it may be desirable to compose
a document from a collection of parts
without having mandatory page breaks between then.
For this case, the package
provides a mechanism to include parts
by |\input| which can also be processed individually.
However, by construction this mechanism
requires manual handling of the content to be output.

%%%%%%%%%%%%%%%%%%%%%%%%%%%%%%%%%%%%%%%%
\DescribeMacro{\ifchilddocmanual}
The main file should be prepared as usual, see \secref{sec:include}.
However, the document body must make a distinction
between processing of an individual part and of the main document, e.g.:
%
\begin{center}
\begin{tabular}{l}
|\ifchilddocmanual|\\
|\input{\childdocname}|\\
|\||else|\\
\textit{document body with }|\input{|\textit{part}|}|\\
|\||fi|
\end{tabular}
\end{center}
%
The conditional |\ifchilddocmanual| is true whenever
a part to be included by |\input| is being compiled,
and the name of the part is stored in |\childdocname|.

%%%%%%%%%%%%%%%%%%%%%%%%%%%%%%%%%%%%%%%%
\DescribeMacro{\childdocby}
Each part to be included by |\input| should start with:
%
\begin{center}
\begin{tabular}{l}
|\input{childdoc.def}|\\
|\childdocby{|\textit{main}|}|\\
\end{tabular}
\end{center}
%
The directive |\childdocby| is similar to |\childdocof|
described in \secref{sec:include},
but the subsequent selection of content must be done manually.
To that end, both |\ifchilddoc| and |\ifchilddocmanual|
will be true upon processing of a part,
and the name of the part is stored in |\childdocname|.
Note that |\jobname| will be set to the filename of the current part
so that each part receives an individual |.aux| file
that does not interfere with the |.aux| file(s) of the main document.
This behaviour can be altered by the alternative form
|\childdocby[*]{|\textit{main}|}| (with a non-empty optional argument)
which uses the |.aux| file of the main document
by setting |\jobname| to \textit{main}.

%%%%%%%%%%%%%%%%%%%%%%%%%%%%%%%%%%%%%%%%%%%%%%%%%%%%%%%%%%%%%%%%%%%%%%%%%%%%%%%%
\subsection{Driver Development}
\label{sec:driver}

The \textsf{childdoc} mechanism can also be use for the development
of definition files such as \LaTeX{} styles or classes.
This case differs from the above setup with multiple parts
included by |\include| in that no |\includeonly| should be invoked.
This can be achieved by starting the include file
(before |\ProvidesPackage|) with:
%
\begin{center}
\begin{tabular}{l}
|\input{childdoc.def}|\\
|\childdocforward{|\textit{main}|}|\\
\end{tabular}
\end{center}
%
or alternatively with:
%
\begin{center}
\begin{tabular}{l}
|\input{childdoc.def}|\\
|\childdocby{|\textit{main}|}|\\
\end{tabular}
\end{center}
%
Both forms have slightly different effects as described above.
The main file is prepared as usual, see \secref{sec:include}.

%%%%%%%%%%%%%%%%%%%%%%%%%%%%%%%%%%%%%%%%%%%%%%%%%%%%%%%%%%%%%%%%%%%%%%%%%%%%%%%%
\subsection{Legacy Detection}
\label{sec:detection}

The directive |\childdocmain| in the main file can detect
whether the complete document or merely a child is to be compiled
even without using the directive |\childdocof|.
This method is deprecated because it is less robust
and there is no compelling reason to use it;
it is merely provided for backward compatibility
and it may be removed in future versions.

If the detection mechanism is to be used,
it is mandatory to correctly specify
the filename of the main file as the argument of |\childdocmain|:
%
\begin{center}
\begin{tabular}{l}
|\input{childdoc.def}|\\
|\childdocmain{|\textit{main}|}|\\
\end{tabular}
\end{center}
%
If |\jobname| does not match the argument \textit{main} of |\childdocmain|,
it is assumed that |\jobname| points to the child file to be compiled.
When using |\childdocmain| with the main file specified as argument,
it suffices to start a child file
with just |\input{|\textit{main}|}|
without loading of the package and using |\childdocof|.
If instead all processing is done
with the appropriate \textsf{childdoc} directives,
the argument of \textit{main} of |\childdocmain| can be empty.

An alternative version of the command line processing described
in \secref{sec:commandline} using the detection mechanism reads:
%
\begin{center}
|... -jobname "|\textit{target}|" "|[\textit{flags}]%
[|\def\jobname{|\textit{dest}|}|]|\input{|\textit{main}|}"|
\end{center}

%%%%%%%%%%%%%%%%%%%%%%%%%%%%%%%%%%%%%%%%%%%%%%%%%%%%%%%%%%%%%%%%%%%%%%%%%%%%%%%%
\subsection{Manual Code}
\label{sec:manual}

In case one cannot be certain whether the definitions file |childdoc.def|
is installed on the target \TeX{} distribution
and one prefers not to ship it,
it is conceivable to paste a few relevant commands into the sources.

To that end, drop all statements |\input{childdoc.def}|
and perform the replacements as outlined below.
Instead of |\childdocmain{|\textit{main}|}| add the following code
to the top of the main file:
%
\begin{center}
\begin{tabular}{l}
|\||ifdefined\childdocname\endinput\||fi\newif\ifchilddoc|\\
|\edef\childdocname{\scantokens\expandafter{\jobname\noexpand}}|\\
|\def\childdocmain{|\textit{main}|}\||ifx\childdocmain\childdocname\||else|\\
|\childdoctrue\includeonly{\childdocname}\let\jobname\childdocmain\||fi|\\
\end{tabular}
\end{center}
%
Instead of |\childdocof{|\textit{main}|}| just include the main file
at the top of each child file:
%
\begin{center}
|\input{|\textit{main}|}|
\end{center}
%
A simple redirection |\childdocforward{|\textit{dest}|}| is achieved by:
%
\begin{center}
|\def\jobname{|\textit{dest}|}\input{\jobname}|
\end{center}
%
The redirection with prefix
|\childdocforwardprefix[|\textit{prefix}|]{|\textit{dest}|}|
is accomplished by:
%
\begin{center}
\begin{tabular}{l}
|{\edef\jobname{\scantokens\expandafter{\jobname\noexpand}}|\\
|\def\redirectjob |\textit{prefix}|#1~~~{\gdef\jobname{|\textit{dest}|#1}}|\\
|\expandafter\redirectjob\jobname~~~}\input{\jobname}|
\end{tabular}
\end{center}

In an alternative approach,
child documents can be compiled by a specific command line
without additional code or specific definitions:
%
\begin{center}
|... -jobname "|\textit{target}|" "|[\textit{flags}]%
|\includeonly{|\textit{dest}|}\input{|\textit{main}|}"|
\end{center}
%

%%%%%%%%%%%%%%%%%%%%%%%%%%%%%%%%%%%%%%%%%%%%%%%%%%%%%%%%%%%%%%%%%%%%%%%%%%%%%%%%
%%%%%%%%%%%%%%%%%%%%%%%%%%%%%%%%%%%%%%%%%%%%%%%%%%%%%%%%%%%%%%%%%%%%%%%%%%%%%%%%
\section{Information}

%%%%%%%%%%%%%%%%%%%%%%%%%%%%%%%%%%%%%%%%%%%%%%%%%%%%%%%%%%%%%%%%%%%%%%%%%%%%%%%%
\subsection{Copyright}

Copyright \copyright{} 2017--2018 Niklas Beisert

This work may be distributed and/or modified under the
conditions of the \LaTeX{} Project Public License, either version 1.3
of this license or (at your option) any later version.
The latest version of this license is in
  \url{http://www.latex-project.org/lppl.txt}
and version 1.3 or later is part of all distributions of \LaTeX{}
version 2005/12/01 or later.

This work has the LPPL maintenance status `maintained'.

The Current Maintainer of this work is Niklas Beisert.

This work consists of the files |README.txt|, |childdoc.ins| and |childdoc.dtx|
as well as the derived files |childdoc.def|, |cdocsamp.tex|
with |cdocsch1.tex|, |cdocsch2.tex|, |cdocspt3.tex|, |cdocspt4.tex|,
|cdocsdrf.tex|, |cdocsfn1.tex|, |cdocsfn2.tex|
as well as |childdoc.pdf|.

%%%%%%%%%%%%%%%%%%%%%%%%%%%%%%%%%%%%%%%%%%%%%%%%%%%%%%%%%%%%%%%%%%%%%%%%%%%%%%%%
\subsection{Files and Installation}

The package consists of the files:
%
\begin{center}
\begin{tabular}{ll}
    |README.txt|   & readme file \\
    |childdoc.ins| & installation file \\
    |childdoc.dtx| & source file \\
    |childdoc.def| & definition file \\
    |cdocsamp.tex| & sample main file \\
    |cdocsch1.tex| & sample include file \\
    |cdocsch2.tex| & sample include file \\
    |cdocspt3.tex| & sample part file \\
    |cdocspt4.tex| & sample part file \\
    |cdocsdrf.tex| & sample redirection file \\
    |cdocsfn1.tex| & sample redirection file \\
    |cdocsfn2.tex| & sample redirection file \\
    |childdoc.pdf| & manual
\end{tabular}
\end{center}
%
The distribution consists of the files
|README.txt|, |childdoc.ins| and |childdoc.dtx|.
%
\begin{itemize}
\item
Run (pdf)\LaTeX{} on |childdoc.dtx|
to compile the manual |childdoc.pdf| (this file).
\item
Run \LaTeX{} on |childdoc.ins| to create the definitions file |childdoc.def|
and the sample |cdocsamp.tex| with include files
|cdocsch1.tex|, |cdocsch2.tex|, |cdocspt3.tex|, |cdocspt4.tex|,
|cdocsdrf.tex|, |cdocsfn1.tex|, |cdocsfn2.tex|.
Then copy the file |childdoc.def| to an appropriate directory of your \LaTeX{}
distribution, e.g.\ \textit{texmf-root}|/tex/latex/childdoc|.
\end{itemize}

%%%%%%%%%%%%%%%%%%%%%%%%%%%%%%%%%%%%%%%%%%%%%%%%%%%%%%%%%%%%%%%%%%%%%%%%%%%%%%%%
\subsection{Related CTAN Packages}

There are several other packages which offer a similar functionality:
%
\begin{itemize}
\item
The packages
\href{http://ctan.org/pkg/docmute}{\textsf{docmute}},
\href{http://ctan.org/pkg/includex}{\textsf{includex}} and
\href{http://ctan.org/pkg/standalone}{\textsf{standalone}}
provide commands to include only the document body of
a child file thus allowing both files to be compiled individually.
\item
The packages \href{http://ctan.org/pkg/subdocs}{\textsf{subdocs}}
and \href{http://ctan.org/pkg/subfiles}{\textsf{subfiles}}
provide structures in which the main and child documents can be
encapsulated and allowing them to be compiled individually.
The inclusion mechanism is different from the conventional |\include|.
\item
The package \href{http://ctan.org/pkg/combine}{\textsf{combine}}
is an elaborate solution to combine several documents into one.
\end{itemize}
%
See also the CTAN topic \href{http://ctan.org/topic/subdocs}{\textsf{subdocs}}
for further related packages.
The present package differs from the above solutions in that
a document structure constructed with the conventional |\include| mechanism
just needs two extra commands at the top of every file
such that all constituent files can be compiled individually.

%%%%%%%%%%%%%%%%%%%%%%%%%%%%%%%%%%%%%%%%%%%%%%%%%%%%%%%%%%%%%%%%%%%%%%%%%%%%%%%%
%\subsection{Feature Suggestions}
%
%The following is a list of features which may be useful for future
%versions of this package:
%%
%\begin{itemize}
%\item
%\ldots
%\end{itemize}

%%%%%%%%%%%%%%%%%%%%%%%%%%%%%%%%%%%%%%%%%%%%%%%%%%%%%%%%%%%%%%%%%%%%%%%%%%%%%%%%
\subsection{Revision History}

%%%%%%%%%%%%%%%%%%%%%%%%%%%%%%%%%%%%%%%%
\paragraph{v2.0:} 2018/12/30

\begin{itemize}
\item
immediate forward processing
\item
added |\childdocby| mechanism
\item
manual restructured
\end{itemize}

%%%%%%%%%%%%%%%%%%%%%%%%%%%%%%%%%%%%%%%%
\paragraph{v1.6:} 2018/01/17

\begin{itemize}
\item
application for development of include files
\item
corrections to manual
\end{itemize}

%%%%%%%%%%%%%%%%%%%%%%%%%%%%%%%%%%%%%%%%
\paragraph{v1.5:} 2017/05/21

\begin{itemize}
\item
more complete structuring introduced
\item
|\childdocof| introduced
\item
|\childdoc| renamed to |\childdocmain|
\item
|\childredirect| renamed to |\childdocforward| and |\childdocforwardprefix|
and functionality expanded
\end{itemize}

%%%%%%%%%%%%%%%%%%%%%%%%%%%%%%%%%%%%%%%%
\paragraph{v1.0:} 2017/04/27

\begin{itemize}
\item
manual and install package
\item
first version published on CTAN
\end{itemize}

%%%%%%%%%%%%%%%%%%%%%%%%%%%%%%%%%%%%%%%%
\paragraph{v0.6:} 2017/04/26

\begin{itemize}
\item
redirection mechanism added
\end{itemize}

%%%%%%%%%%%%%%%%%%%%%%%%%%%%%%%%%%%%%%%%
\paragraph{v0.5:} 2017/04/26

\begin{itemize}
\item
functionality in definition file
\end{itemize}


%%%%%%%%%%%%%%%%%%%%%%%%%%%%%%%%%%%%%%%%%%%%%%%%%%%%%%%%%%%%%%%%%%%%%%%%%%%%%%%%
%%%%%%%%%%%%%%%%%%%%%%%%%%%%%%%%%%%%%%%%%%%%%%%%%%%%%%%%%%%%%%%%%%%%%%%%%%%%%%%%
%%%%%%%%%%%%%%%%%%%%%%%%%%%%%%%%%%%%%%%%%%%%%%%%%%%%%%%%%%%%%%%%%%%%%%%%%%%%%%%%
\appendix

\settowidth\MacroIndent{\rmfamily\scriptsize 000\ }

 \DocInput{childdoc.dtx}

\end{document}
%</driver>
% \fi
%
% %%%%%%%%%%%%%%%%%%%%%%%%%%%%%%%%%%%%%%%%%%%%%%%%%%%%%%%%%%%%%%%%%%%%%%%%%%%%%%
% %%%%%%%%%%%%%%%%%%%%%%%%%%%%%%%%%%%%%%%%%%%%%%%%%%%%%%%%%%%%%%%%%%%%%%%%%%%%%%
% \section{Sample}
%\iffalse
%<*samplemain>
%\fi
%
% The following presents a sample document
% with two chapters, two parts, a title page,
% a compile flag as well as three forwarding files to set the flag.
% It consists of eight |.tex| files:
% \begin{center}
% \begin{tabular}{ll}
% |cdocsamp.tex|&main file\\
% |cdocsch1.tex|&include file for chapter 1\\
% |cdocsch2.tex|&include file for chapter 2\\
% |cdocspt3.tex|&include file for part 3\\
% |cdocspt4.tex|&include file for part 4\\
% |cdocsdrf.tex|&forwarding file for main file in draft mode\\
% |cdocsfi1.tex|&forwarding file for final version of chapter 1\\
% |cdocsfi2.tex|&forwarding file for final version of chapter 2\\
% \end{tabular}
% \end{center}
% Each of the eight files can be compiled directly by the \LaTeX{} compiler.
%
% %%%%%%%%%%%%%%%%%%%%%%%%%%%%%%%%%%%%%%
% \paragraph{Main File.}
%
% The main file is called |cdocsamp.tex|.
%
% Load the \textsf{childdoc} definitions and
% declare the filename for the main document:
%    \begin{macrocode}
\input{childdoc.def}
\childdocmain{}
%    \end{macrocode}

% Optional override for |\version| flag:
%    \begin{macrocode}
%%\ifchilddoc\else\providecommand{\version}{draft}\fi
%    \end{macrocode}

% Define the default values for the |\version| flag
% (|final| for the main file and |draft| for childs):
%    \begin{macrocode}
\ifchilddoc
\providecommand{\version}{draft}
\else
\providecommand{\version}{final}
\fi
%    \end{macrocode}

% Load the standard document class:
%    \begin{macrocode}
\documentclass[12pt]{article}
%    \end{macrocode}

% Start the document body:
%    \begin{macrocode}
\begin{document}
%    \end{macrocode}

% Declare a title page.
% Print title, part of document being processed and version flag:
%    \begin{macrocode}
\addtocounter{page}{-1}
\begin{center}
{\LARGE\bfseries{}childdoc example\par}
\vspace{1cm}
\ifchilddoc
\ifchilddocmanual part\else chapter\fi:
`\childdocname' of `\childdocjob'\par
\else
main document: `\childdocjob'\par
\fi
version: \version\par
\end{center}
\newpage
%    \end{macrocode}

% Manually include selected file,
% otherwise process as usual:
%    \begin{macrocode}
\ifchilddocmanual
\section*{part `\childdocname'}
\input{\childdocname}
\else
%    \end{macrocode}

% Include the two chapters:
%    \begin{macrocode}
\include{cdocsch1}
\include{cdocsch2}
%    \end{macrocode}

% Include the two parts unless only chapters should be displayed:
%    \begin{macrocode}
\ifchilddoc\else
\section{part three}
\input{cdocspt3}
\section{part four}
\input{cdocspt4}
\fi
%    \end{macrocode}

% Process as usual until here:
%    \begin{macrocode}
\fi
%    \end{macrocode}

% End of document body:
%    \begin{macrocode}
\end{document}
%    \end{macrocode}
%\iffalse
%</samplemain>
%\fi
%
% %%%%%%%%%%%%%%%%%%%%%%%%%%%%%%%%%%%%%%
% \paragraph{Chapter Include Files.}
%
% The include files are called |cdocsch1.tex| and |cdocsch2.tex|.
%
%\iffalse
%<*samplechap1|samplechap2>
%\fi

% Optional override for |\version| flag:
%    \begin{macrocode}
%%\providecommand{\version}{final}
%    \end{macrocode}

% Include the main document:
%    \begin{macrocode}
\input{childdoc.def}
\childdocof{cdocsamp}
%    \end{macrocode}

%\iffalse
%</samplechap1|samplechap2>
%\fi
%
%\iffalse
%<*samplechap1>
%\fi
% Some text for chapter 1:
%    \begin{macrocode}
\section{one}
some text in chapter one
%    \end{macrocode}

%\iffalse
%</samplechap1>
%\fi
% Some text for chapter 2:
%\iffalse
%<*samplechap2>
%\fi
%    \begin{macrocode}
\section{two}
more text in chapter two
%    \end{macrocode}

%\iffalse
%</samplechap2>
%\fi
%
% %%%%%%%%%%%%%%%%%%%%%%%%%%%%%%%%%%%%%%
% \paragraph{Part Include Files.}
%
% The include files are called |cdocspt3.tex| and |cdocspt4.tex|.
%
%\iffalse
%<*samplepart3|samplepart4>
%\fi

% Optional override for |\version| flag:
%    \begin{macrocode}
%%\providecommand{\version}{final}
%    \end{macrocode}

% Include the main document:
%    \begin{macrocode}
\input{childdoc.def}
\childdocby{cdocsamp}
%    \end{macrocode}

%\iffalse
%</samplepart3|samplepart4>
%\fi
%
%\iffalse
%<*samplepart3>
%\fi
% Some text for part 3:
%    \begin{macrocode}
some text in part three
%    \end{macrocode}

%\iffalse
%</samplepart3>
%\fi
% Some text for part 4:
%\iffalse
%<*samplepart4>
%\fi
%    \begin{macrocode}
more text in part four
%    \end{macrocode}

%\iffalse
%</samplepart4>
%\fi
%
% %%%%%%%%%%%%%%%%%%%%%%%%%%%%%%%%%%%%%%
% \paragraph{Forwarding for a Complete Draft.}
%
% The following forwarding file |cdocsdrf.tex|
% compiles the main document in draft mode:
%\iffalse
%<*sampledraft>
%\fi
%    \begin{macrocode}
\def\version{draft}
\input{childdoc.def}
\childdocforward{cdocsamp}
%    \end{macrocode}

%\iffalse
%</sampledraft>
%\fi
%
% %%%%%%%%%%%%%%%%%%%%%%%%%%%%%%%%%%%%%%
% \paragraph{Forwarding for Final Version of the Chapters.}
%
% The following forwarding files |cdocsfn1.tex| and |cdocsfn2.tex|
% (with identical content)
% compile the final versions of the child documents
% |cdocsch1.tex| and |cdocsch2.tex|, respectively:
%\iffalse
%<*samplefinal>
%\fi
%    \begin{macrocode}
\def\version{final}
\input{childdoc.def}
\childdocforwardprefix[cdocsamp]{cdocsfn}{cdocsch}
%    \end{macrocode}

%\iffalse
%</samplefinal>
%\fi
%
% %%%%%%%%%%%%%%%%%%%%%%%%%%%%%%%%%%%%%%
% \paragraph{Command Line Processing.}
%
% The following three command lines generate the output files
% |cdocscld|, |cdocscl1| and |cdocscl2|
% which should be identical to
% |cdocsdrf|, |cdocsch1| and |cdocsfn2|, respectively:
% \begin{center}
% \begin{tabular}{l}
% |latex -jobname cdocscld \|\\
% |  "\def\version{draft}\input{childdoc.def}\childdocforward{cdocsamp}"|\\
% |latex -jobname cdocscl1 \|\\
% |  "\input{childdoc.def}\childdocforward[cdocsamp]{cdocsch1}"|\\
% |latex -jobname cdocscl2 \|\\
% |  "\def\version{final}\input{childdoc.def}\childdocforward{cdocsch2}"|
% \end{tabular}
% \end{center}
% Note that the trailing backslash on each first line
% merely continues the input to the second line
% (for convenient cut ant paste).
% Furthermore, the command |latex| can be replaced by any
% of its alternative versions such as |pdflatex|.
%
% %%%%%%%%%%%%%%%%%%%%%%%%%%%%%%%%%%%%%%%%%%%%%%%%%%%%%%%%%%%%%%%%%%%%%%%%%%%%%%
% %%%%%%%%%%%%%%%%%%%%%%%%%%%%%%%%%%%%%%%%%%%%%%%%%%%%%%%%%%%%%%%%%%%%%%%%%%%%%%
% \section{Implementation}
%\iffalse
%<*package>
%\fi
%
% This section describes the definitions file |childdoc.def|.

% The definitions cannot be loaded using |\usepackage| or |\RequirePackage|
% which has a mechanism to prevent loading a style file more than once.
% When loading the definitions by means of |\input|
% multiple instances have to be prevented manually:
%\iffalse
%This code needs to be before the `\ProvidesFile' directive
%which is defined at the beginning of this file.
%Therefore it is also placed there and commented out here.
%</package>
%<*discard>
%\fi
%    \begin{macrocode}
\ifdefined\childdocmain\endinput\fi
%    \end{macrocode}
%\iffalse
%</discard>
%<*package>
%\fi
%
% \macro{\ifchilddoc}
% \macro{\ifchilddocmanual}
% The conditional |\ifchilddoc| tells whether a
% child (true) or main (false) document is being compiled.
% The conditional |\ifchilddocmanual| tells whether
% the |\includeonly| mechanism is used (false) or
% the selection of child files must be performed manually (true).
% The definitions initialise to false:
%    \begin{macrocode}
\newif\ifchilddoc
\newif\ifchilddocmanual
%    \end{macrocode}

% \macro{\childdocname}
% \macro{\childdocjob}
% The macro |\childdocname| stores the name of the main document
% to be compiled. The macro |\childdocjob| stores the name of
% the document on which the \LaTeX{} compiler was originally invoked.
% The content of |\jobname| cannot be compared
% to filenames specified in the source due to different catcodes.
% The following code rescans |\jobname|, stores the result
% in |\childdocname| and saves a copy in |\childdocjob|:
%    \begin{macrocode}
\edef\childdocname{\scantokens\expandafter{\jobname\noexpand}}
\let\childdocjob\childdocname
%    \end{macrocode}

% \macro{\childdocdisable}
% The macro |\childdocdisable| prevents the main file
% from being processed more than once.
% At this stage, the main document command |\childdocmain|
% is assumed to be called once again where it should do nothing.
% Any subsequent call to it should prevent
% a secondary processing of the main document
% It overwrites the forwarding commands
% |\childdocof| and |\childdocforward|
% with empty macros to prevent further inclusions of the main document:
%    \begin{macrocode}
\newcommand{\childdocdisable}
{
  \renewcommand{\childdocmain}[1]{\renewcommand{\childdocmain}[1]{\endinput}}
  \renewcommand{\childdocof}[1]{}
  \renewcommand{\childdocby}[2][]{}
  \renewcommand{\childdocforward}[2][]{}
  \renewcommand{\childdocdisable}{}
}
%    \end{macrocode}

% \macro{\childdocmain}
% The macro |\childdocmain| is to be called at the top of the main file
% with nothing or the main filename (without extension) as argument.
% First, it breaks loops.
% If the argument is not empty and does not match |\childdocname|
% (which is set by the first inclusion of |childdoc.def|),
% |\ifchilddoc| is set to true, |\includeonly| is applied to the child file
% and |\jobname| is set to the main file
% (for proper handling of |.aux| files):
%    \begin{macrocode}
\newcommand{\childdocmain}[1]
{
  \childdocdisable\childdocmain{}
  \if?#1?\else
    \begingroup
      \def\childdoctmp{#1}
      \ifx\childdoctmp\childdocname
        \def\childdoctmp{}
      \else
        \def\childdoctmp
        {
          \childdoctrue
          \includeonly{\childdocname}
          \def\childdocjob{#1}
          \def\jobname{#1}
        }
      \fi
      \expandafter
    \endgroup
    \childdoctmp
  \fi
}
%    \end{macrocode}

% \macro{\childdocof}
% The command |\childdocof| redirects
% compilation to the main file |#1|.
%    \begin{macrocode}
\newcommand{\childdocof}[1]
{
  \childdocdisable
  \childdoctrue
  \includeonly{\childdocname}
  \def\jobname{#1}
  \def\childdocjob{#1}
  \input{#1}
}
%    \end{macrocode}

% \macro{\childdocby}
% The command |\childdocby| ....
%    \begin{macrocode}
\newcommand{\childdocby}[2][]
{
  \childdocdisable
  \childdoctrue
  \childdocmanualtrue
  \if?#1?\else
    \def\jobname{#2}
  \fi
  \def\childdocjob{#2}
  \input{#2}
  \endinput
}
%    \end{macrocode}

% \macro{\childdocforward}
% The command |\childdocforward| redirects
% compilation to the main file or
% (if the optional argument is given) a child file.
% Parameters are set as if the main file
% or a child file starting with |\childdocof| was compiled.
% Then compilation is handed over to the main file:
%    \begin{macrocode}
\newcommand{\childdocforward}[2][]
{
  \begingroup
    \if?#1?
      \def\childdoctmp
      {
        \def\childdocname{#2}
        \def\childdocjob{#2}
        \def\jobname{#2}
        \input{#2}
        \endinput
      }
    \else
      \def\childdoctmp
      {
        \childdocdisable
        \def\childdocname{#2}
        \childdoctrue
        \includeonly{#2}
        \def\childdocjob{#1}
        \def\jobname{#1}
        \input{#1}
        \endinput
      }
    \fi
    \expandafter
  \endgroup
  \childdoctmp
}
%    \end{macrocode}

% \macro{\childdocforwardprefix}
% The command |\childdocforwardprefix| redirects
% compilation to the main or a child file by means of a pattern.
% The prefix |#1| in the current filename is replaced by |#2|
% and the suffix of the current filename is kept
% (it is assumed that the filename does not contain the substring `|~~~|'
% which is used as a delimiter).
% Compilation is handed over to the new file by |\childdocforward|:
%    \begin{macrocode}
\newcommand{\childdocforwardprefix}[3][]
{
  \begingroup
    \def\childdocextract #2##1~~~{\def\childdoctmp{\childdocforward[#1]{#3##1}}}
    \expandafter\childdocextract\childdocname~~~
    \expandafter
  \endgroup
  \childdoctmp
}
%    \end{macrocode}

% \macro{\childdoc}
% The deprecated macro |\childdoc| is a legacy version of |\childdocmain|:
%    \begin{macrocode}
\newcommand{\childdoc}{\childdocmain}
%    \end{macrocode}

% \macro{\childdocredirect}
% The deprecated macro |\childdocredirect| is a legacy version
% of |\childdocforward| and |\childdocforwardprefix|:
%    \begin{macrocode}
\newcommand{\childdocredirect}[2][]
{
  \begingroup
    \if?#1?
      \def\childdoctmp{\childdocforward{#2}}
    \else
      \def\childdoctmp{\childdocforwardprefix{#1}{#2}}
    \fi
    \expandafter
  \endgroup
  \childdoctmp
}
%    \end{macrocode}

%\iffalse
%</package>
%\fi
%
\endinput
|\\
|\childdocforward[|\textit{main}|]{|\textit{dest}|}|\\
\end{tabular}
\end{center}
%
The argument \textit{dest} is the destination file
(without extension).
It should be the main file or one of the child files.
Note that further \textsf{childdoc} directives
such as |\childdocof| and |\childdocforward|
in the indicated file will be processed in this form.
The optional argument \textit{main}
passes on directly to the main file \textit{main}
while pretending to compile the child \textit{dest}.
This form behaves as if \textit{dest}
issues |\childdocof{|\textit{main}|}| right away,
and no further \textsf{childdoc} directives will be processed.

%%%%%%%%%%%%%%%%%%%%%%%%%%%%%%%%%%%%%%%%
\DescribeMacro{\...prefix}
In the alternative form |\childdocforwardprefix|,
%
\begin{center}
\begin{tabular}{l}
|% \iffalse
%
% childdoc.dtx Copyright (C) 2017-2018 Niklas Beisert
%
% This work may be distributed and/or modified under the
% conditions of the LaTeX Project Public License, either version 1.3
% of this license or (at your option) any later version.
% The latest version of this license is in
%   http://www.latex-project.org/lppl.txt
% and version 1.3 or later is part of all distributions of LaTeX
% version 2005/12/01 or later.
%
% This work has the LPPL maintenance status `maintained'.
%
% The Current Maintainer of this work is Niklas Beisert.
%
% This work consists of the files childdoc.dtx and childdoc.ins
% and the derived files childdoc.def and cdocsamp.tex with
% cdocsch1.tex, cdocsch2.tex, cdocsdrf.tex, cdocsfn1.tex, cdocsfn2.tex.
%
%<package>\ifdefined\childdocmain\endinput\fi
%<package>\ProvidesFile{childdoc.def}[2018/12/30 v2.0 child document driver]
%<samplemain>\ProvidesFile{cdocsamp.tex}[2018/12/30 v2.0 sample for childdoc]
%<*driver>
%\ProvidesFile{childdoc.drv}[2018/12/30 v2.0 childdoc reference manual file]
\PassOptionsToClass{10pt,a4paper}{article}
\documentclass{ltxdoc}

\usepackage[margin=35mm]{geometry}
\usepackage{hyperref}
\usepackage{hyperxmp}
\usepackage[usenames]{color}

\hypersetup{colorlinks=true}
\hypersetup{pdfstartview=FitH}
\hypersetup{pdfpagemode=UseNone}
\hypersetup{pdfsource={}}
\hypersetup{pdflang={en-UK}}
\hypersetup{pdfcopyright={Copyright 2017-2018 Niklas Beisert.
  This work may be distributed and/or modified under the
  conditions of the LaTeX Project Public License, either version 1.3
  of this license or (at your option) any later version.}}
\hypersetup{pdflicenseurl={http://www.latex-project.org/lppl.txt}}
\hypersetup{pdfcontactaddress={ETH Zurich, ITP, HIT K,
  Wolfgang-Pauli-Strasse 27}}
\hypersetup{pdfcontactpostcode={8093}}
\hypersetup{pdfcontactcity={Zurich}}
\hypersetup{pdfcontactcountry={Switzerland}}
\hypersetup{pdfcontactemail={nbeisert@itp.phys.ethz.ch}}
\hypersetup{pdfcontacturl={http://people.phys.ethz.ch/\xmptilde nbeisert/}}

\newcommand{\secref}[1]{\hyperref[#1]{section \ref*{#1}}}

\parskip1ex
\parindent0pt
\let\olditemize\itemize
\def\itemize{\olditemize\parskip0pt}

\begin{document}

\title{The \textsf{childdoc} Package}
\hypersetup{pdftitle={The childdoc Package}}
\author{Niklas Beisert\\[2ex]
  Institut f\"ur Theoretische Physik\\
  Eidgen\"ossische Technische Hochschule Z\"urich\\
  Wolfgang-Pauli-Strasse 27, 8093 Z\"urich, Switzerland\\[1ex]
  \href{mailto:nbeisert@itp.phys.ethz.ch}
  {\texttt{nbeisert@itp.phys.ethz.ch}}}
\hypersetup{pdfauthor={Niklas Beisert}}
\hypersetup{pdfsubject={Manual for the LaTeX2e Package childdoc}}
\date{30 December 2018, \textsf{v2.0}}
\maketitle

\begin{abstract}\noindent
\textsf{childdoc} is a \LaTeXe{} package
that enables the direct compilation
of document sections included by |\include|
to individual files.
\end{abstract}

\begingroup
\parskip0ex
\tableofcontents
\endgroup

%%%%%%%%%%%%%%%%%%%%%%%%%%%%%%%%%%%%%%%%%%%%%%%%%%%%%%%%%%%%%%%%%%%%%%%%%%%%%%%%
%%%%%%%%%%%%%%%%%%%%%%%%%%%%%%%%%%%%%%%%%%%%%%%%%%%%%%%%%%%%%%%%%%%%%%%%%%%%%%%%
\section{Introduction}

\LaTeX{} provides a mechanism to structure a large document (such as a book)
into a main file and several child files (containing the chapters)
using the |\include| command.
This mechanism is beneficial for documents
which span hundreds of pages in order to
make the source file(s) more manageable.
Moreover, compilation can be restricted to
selected child files by means of the |\includeonly| command.
The latter feature can be used to reduce the compilation time while editing
(this was significantly more useful in the earlier days of \LaTeX{})
or to generate a smaller document which is easier to navigate.
Another application of |\includeonly| is to generate
documents consisting of selected parts of the complete document.

However, there are a few drawbacks of the plain |\include| mechanism:
\begin{itemize}
\item
The child files cannot be compiled on their own,
they can only be compiled via the main file.
A naive editing environment
(such as a text editor with an option
to have the current file processed by \LaTeX)
may require one to switch to the main file before compiling;
attempting to compile the child file produces errors.
\item
The main file must be modified (each time)
to adjust the |\includeonly| command
to the present needs. This easily leaves the main file in a messy state.
\item
The generated document will always carry the filename
of the main document. This is inconvenient if
several child files are to be compiled and
to be kept for distribution.
\end{itemize}

The present package provides a simple interface
to make child files individually compilable by \LaTeX{}.
Compiling a child file then has the same effect as compiling
the main file with an |\includeonly| command
to select the appropriate child.
Moreover the generated document will carry the name of the child
rather than the main file.
This resolves all three above issues.

This feature is meant to make the editing of books,
thesis documents and lecture notes somewhat more convenient.
However, the package can also be used efficiently for
composing a series of documents (such as exercise sheets)
which are typically distributed individually.
It then assists the author in generating the individual documents
(potentially in different versions)
as well as a document containing the collected series.
Another application is in developing style files
or other kinds of included material
where compilation of the style file could redirect
to a sample or test file.

%%%%%%%%%%%%%%%%%%%%%%%%%%%%%%%%%%%%%%%%%%%%%%%%%%%%%%%%%%%%%%%%%%%%%%%%%%%%%%%%
%%%%%%%%%%%%%%%%%%%%%%%%%%%%%%%%%%%%%%%%%%%%%%%%%%%%%%%%%%%%%%%%%%%%%%%%%%%%%%%%
\section{Usage}

First of all, the package \textsf{childdoc} is \emph{not} a standard
\LaTeXe{} |.sty| style file! Therefore it needs to be invoked in
a non-standard way.

%%%%%%%%%%%%%%%%%%%%%%%%%%%%%%%%%%%%%%%%%%%%%%%%%%%%%%%%%%%%%%%%%%%%%%%%%%%%%%%%
\subsection{Included Files}
\label{sec:include}

%%%%%%%%%%%%%%%%%%%%%%%%%%%%%%%%%%%%%%%%
\DescribeMacro{\childdocmain}
To use the package, add the commands
\begin{center}
\begin{tabular}{l}
|\input{childdoc.def}|\\
|\childdocmain{}|\\
\end{tabular}
\end{center}
at the very top of the main \LaTeX{} file,
in particular \emph{before} the |\documentclass| statement!
The argument of |\childdocmain| should be left empty
(but it must be present).

%%%%%%%%%%%%%%%%%%%%%%%%%%%%%%%%%%%%%%%%
\DescribeMacro{\childdocof}
Furthermore, add the commands
\begin{center}
\begin{tabular}{l}
|\input{childdoc.def}|\\
|\childdocof{|\textit{main}|}|\\
\end{tabular}
\end{center}
at the top of every child file \textit{child}
which is included by |\include{|\textit{child}|}|
from within the main file
(or at least for those files to be compiled individually).
The argument \textit{main} must be the filename of the main file.

There are a couple of
considerations in setting up the main and child documents:

%%%%%%%%%%%%%%%%%%%%%%%%%%%%%%%%%%%%%%%%
\paragraph{Restrictions.}

Please note the following restrictions:
\begin{itemize}
\item
|\childdocmain| must be called with one argument \textit{main}
to ensure compatibility with earlier version of the package.
It must either be empty (|\childdocmain{}|)
or precisely match the filename of the main file in which it is specified.
See \secref{sec:detection} for further information.
\item
The filename \textit{main} must be specified without the |.tex| extension.
\item
The filename \textit{main} is case sensitive
(even in case-insensitive file systems)
due to internal string comparison.
\item
The argument \textit{main} should be fully expanded, it cannot be a macro.
\item
Subdirectories and special characters should be avoided in filenames.
\item
The command |\childdocmain{|\textit{main}|}| must be followed by a whitespace.
It should not be followed immediately by another command
or by a comment mark `|%|'.
This is because the \TeX{} parser reads the token immediately following
the argument of |\childdocmain| and puts it
at the beginning of every child section;
however, a white\-space is ignored.
\end{itemize}

%%%%%%%%%%%%%%%%%%%%%%%%%%%%%%%%%%%%%%%%
\paragraph{Content of Main File.}

It is advisable to place all content in the child files included by |\include|.
Any output contained in the main file will appear in all child documents
unless suppressed manually;
it cannot be suppressed automatically by the |\includeonly| directive
and thus should normally be avoided.
A method to include some content in the main file
by means of conditional processing is described in \secref{sec:conditional}.

%%%%%%%%%%%%%%%%%%%%%%%%%%%%%%%%%%%%%%%%
\paragraph{Page Numbering.}

When only a part of the document is compiled,
the appropriate numbering of pages
(as well as other status parameters)
is determined from the |.aux| files.
The latter contain information from previous passes.
However this information needs to propagate through
all intermediate child documents.
Therefore the page numbering in child documents may well
be inconsistent until the complete document is compiled at least once.

A useful (if unconventional) way to always ensure a consistent
page numbering is to restart the numbering in each child document
and denote the pages by `\textit{child}|.|\textit{page}'
where \textit{child} represents the chapter/section number of the child file.
This can be achieved by the command
|\numberwithin{page}{|\textit{child}|}|
of the \textsf{amsmath} package
where \textit{child} can be |chapter| or |section|
depending on the chosen structuring.
Alternatively, one can modify the macro |\thepage| appropriately
and reset the counter |page| at the start of each child file.

%%%%%%%%%%%%%%%%%%%%%%%%%%%%%%%%%%%%%%%%%%%%%%%%%%%%%%%%%%%%%%%%%%%%%%%%%%%%%%%%
\subsection{Conditional Processing}
\label{sec:conditional}

The package provides a mechanism to compile different versions
of a document. To customise the versions further some conditional processing
can come in handy to distinguish which version is being compiled.
The package provides two macros to describe the compilation context:

%%%%%%%%%%%%%%%%%%%%%%%%%%%%%%%%%%%%%%%%
\DescribeMacro{\ifchilddoc}
The conditional |\ifchilddoc| distinguishes between the compilation of
child documents and the main document:
%
\begin{center}
|\ifchilddoc |\textit{child-code}| |[|\||else |\textit{main-code}]| \||fi|
\end{center}

%%%%%%%%%%%%%%%%%%%%%%%%%%%%%%%%%%%%%%%%
\DescribeMacro{\childdocname}
\DescribeMacro{\childdocjob}
The macro |\childdocname| contains the filename (without extension)
of the main or child file being processed.
Note that |\childdocjob| will always contain the name of the main file.

%%%%%%%%%%%%%%%%%%%%%%%%%%%%%%%%%%%%%%%%
\paragraph{Title Page.}

Conditional processing can be used to include a title or banner page
in the main document when proper precautions are taken.
Importantly, the code in the main file should ensure that the page counter
(as well as other status parameters which are stored in the |.aux| files)
takes the same value after the conditional processing.
Otherwise the page numbers may take divergent values
depending on which part is compiled.

For example, a title page could be declared by:
%
\begin{center}
\begin{tabular}{l}
|\ifchilddoc\||else|\\
|\addtocounter{page}{-1}|\\
\textit{code for title page}\\
|\newpage|\\
|\||fi|
\end{tabular}
\end{center}
%
A banner page for the child documents can be generated by:
%
\begin{center}
\begin{tabular}{l}
|\ifchilddoc|\\
|\addtocounter{page}{-1}|\\
\textit{code for banner page}\\
|\newpage|\\
|\||fi|
\end{tabular}
\end{center}
%
Here one could write a message such as:
\begin{center}
|This is the part \childdocname{} of \childdocjob{}.|
\end{center}

%%%%%%%%%%%%%%%%%%%%%%%%%%%%%%%%%%%%%%%%%%%%%%%%%%%%%%%%%%%%%%%%%%%%%%%%%%%%%%%%
\subsection{Flags}
\label{sec:flags}

The package makes it easy to generate different versions
of the main or child documents.
To this end compilation flags can be defined
and assigned different default values.
They will be particularly useful in conjunction
with the forwarding mechanism described in \secref{sec:forward}.

For example, it may be useful to have a flag |\version|
which can be set to |draft| or |final|.
The document source will contain some conditional code
depending on the value of |\version|.
Suppose further, the flag should default to |final| for the main file
and to |draft| for child files
which is a natural assignment for editing the document.
This is achieved by placing the following code
in the preamble of the main document
(below the |\childdocmain| directive):
%
\begin{center}
\begin{tabular}{l}
|\ifchilddoc|\\
|\providecommand{\version}{draft}|\\
|\||else|\\
|\providecommand{\version}{final}|\\
|\||fi|
\end{tabular}
\end{center}
%
The definition by |\providecommand| makes sure
that previous definitions are not overwritten.
Further statements |\providecommand{\version}{...}|
can thus be added before the above code to override it.

For the main file, one might add a line
(between |\childdocmain| and the above block)
%
\begin{center}
|%\ifchilddoc\||else\providecommand{\version}{draft}\||fi|
\end{center}
%
which can be uncommented to produce a draft version.
Likewise one can add a line to the very top of a child file
(above the |\childdocof{|\textit{main}|}| directive)
%
\begin{center}
|%\providecommand{\version}{final}|
\end{center}
%
which can be uncommented to produce the final version of this child document.

%%%%%%%%%%%%%%%%%%%%%%%%%%%%%%%%%%%%%%%%%%%%%%%%%%%%%%%%%%%%%%%%%%%%%%%%%%%%%%%%
\subsection{Forwarding}
\label{sec:forward}

Different versions of the main or child documents
using compilation flags as described in \secref{sec:flags}
can be (permanently) stored in different files
for convenient compilation, viewing and distribution.
To this end, the package defines a command
to pass on compilation to a different file:

%%%%%%%%%%%%%%%%%%%%%%%%%%%%%%%%%%%%%%%%
\DescribeMacro{\childdocforward}
The command |\childdocforward| redirects processing to
another source file:
%
\begin{center}
\begin{tabular}{l}
|\input{childdoc.def}|\\
|\childdocforward[|\textit{main}|]{|\textit{dest}|}|\\
\end{tabular}
\end{center}
%
The argument \textit{dest} is the destination file
(without extension).
It should be the main file or one of the child files.
Note that further \textsf{childdoc} directives
such as |\childdocof| and |\childdocforward|
in the indicated file will be processed in this form.
The optional argument \textit{main}
passes on directly to the main file \textit{main}
while pretending to compile the child \textit{dest}.
This form behaves as if \textit{dest}
issues |\childdocof{|\textit{main}|}| right away,
and no further \textsf{childdoc} directives will be processed.

%%%%%%%%%%%%%%%%%%%%%%%%%%%%%%%%%%%%%%%%
\DescribeMacro{\...prefix}
In the alternative form |\childdocforwardprefix|,
%
\begin{center}
\begin{tabular}{l}
|\input{childdoc.def}|\\
|\childdocforwardprefix[|\textit{main}|]{|\textit{prefix}|}{|\textit{dest}|}|
\end{tabular}
\end{center}
%
the destination file is determined by a pattern
depending on the current file:
To make this work, the current file must be called
`{\textit{prefix}\hspace{0.2em}\textit{suffix}}'
with \textit{prefix} matching precisely the argument.
Processing is then passed on to the file
`{\textit{dest}\hspace{0.2em}\textit{suffix}}'.
Surely, the same effect is achieved by
directly specifying the
argument `{\textit{dest}\hspace{0.2em}\textit{suffix}}'
in the first form.
However, that requires to set up a different file
for each child. With the alternative form of the command
all these files can have exactly the same content
which simplifies setting them up and maintaining them.

For example, the following file |draft.tex|
with a compilation flag |\version| as described in \secref{sec:flags}
compiles the main document as a draft:
%
\begin{center}
\begin{tabular}{l}
|\def\version{draft}|\\
|\input{childdoc.def}|\\
|\childdocforward{|\textit{main}|}|
\end{tabular}
\end{center}
%
Likewise, the following files |final|\textit{nn}|.tex|
compile the final version of the child document
|child|\textit{nn}|.tex|:
%
\begin{center}
\begin{tabular}{l}
|\def\version{final}|\\
|\input{childdoc.def}|\\
|\childdocforwardprefix{final}{child}|
\end{tabular}
\end{center}
%

Note that when several versions of a main file and/or of each child file
are to be generated, it may be convenient to set up a |Makefile| or
shell script to automatise the process.

%%%%%%%%%%%%%%%%%%%%%%%%%%%%%%%%%%%%%%%%%%%%%%%%%%%%%%%%%%%%%%%%%%%%%%%%%%%%%%%%
\subsection{Command Line Processing}
\label{sec:commandline}

The effect of redirection files can also be achieved by invoking
the \LaTeX{} compiler with a more elaborate command line.
Most conveniently this should be done as part
of a shell script or a |Makefile|.

When using \textsf{childdoc} in the main file, the following
command lines effectively perform a redirection
(note that depending on the shell being used,
backslashes may have to be doubled: `|\|' $\to$ `|\\|'):
%
\begin{center}
|... -jobname "|\textit{target}|" |\\|"|[\textit{flags}]%
|\input{childdoc.def}\childdocforward[|\textit{main}|]{|\textit{dest}|}"|
\end{center}
%
Here \textit{target} is the name of the output file,
\textit{main} is the name of the main file
and \textit{dest} is the name of the main or child file to be processed
(all filenames without extensions).
The optional argument \textit{main} can be omitted
if \textit{main} matches \textit{dest}.
Optionally, compilation \textit{flags} can be defined via |\def| commands.
This command line makes the \TeX{} engine believe
it is compiling the file \textit{target}
whose content is specified as the latter parameter.
The provided code then forwards the processing to
\textit{main} or \textit{dest} as described in \secref{sec:forward}.

%%%%%%%%%%%%%%%%%%%%%%%%%%%%%%%%%%%%%%%%%%%%%%%%%%%%%%%%%%%%%%%%%%%%%%%%%%%%%%%%
\subsection{Include by Input}
\label{sec:input}

Including child documents by |\include| has some restrictions by design.
Most notably, the content of a child document always occupies
its own set of pages; pages cannot be shared between child documents.
Usually, this behaviour makes perfect sense
because each child document contain an essential part of the document.
However, in some situations it may be desirable to compose
a document from a collection of parts
without having mandatory page breaks between then.
For this case, the package
provides a mechanism to include parts
by |\input| which can also be processed individually.
However, by construction this mechanism
requires manual handling of the content to be output.

%%%%%%%%%%%%%%%%%%%%%%%%%%%%%%%%%%%%%%%%
\DescribeMacro{\ifchilddocmanual}
The main file should be prepared as usual, see \secref{sec:include}.
However, the document body must make a distinction
between processing of an individual part and of the main document, e.g.:
%
\begin{center}
\begin{tabular}{l}
|\ifchilddocmanual|\\
|\input{\childdocname}|\\
|\||else|\\
\textit{document body with }|\input{|\textit{part}|}|\\
|\||fi|
\end{tabular}
\end{center}
%
The conditional |\ifchilddocmanual| is true whenever
a part to be included by |\input| is being compiled,
and the name of the part is stored in |\childdocname|.

%%%%%%%%%%%%%%%%%%%%%%%%%%%%%%%%%%%%%%%%
\DescribeMacro{\childdocby}
Each part to be included by |\input| should start with:
%
\begin{center}
\begin{tabular}{l}
|\input{childdoc.def}|\\
|\childdocby{|\textit{main}|}|\\
\end{tabular}
\end{center}
%
The directive |\childdocby| is similar to |\childdocof|
described in \secref{sec:include},
but the subsequent selection of content must be done manually.
To that end, both |\ifchilddoc| and |\ifchilddocmanual|
will be true upon processing of a part,
and the name of the part is stored in |\childdocname|.
Note that |\jobname| will be set to the filename of the current part
so that each part receives an individual |.aux| file
that does not interfere with the |.aux| file(s) of the main document.
This behaviour can be altered by the alternative form
|\childdocby[*]{|\textit{main}|}| (with a non-empty optional argument)
which uses the |.aux| file of the main document
by setting |\jobname| to \textit{main}.

%%%%%%%%%%%%%%%%%%%%%%%%%%%%%%%%%%%%%%%%%%%%%%%%%%%%%%%%%%%%%%%%%%%%%%%%%%%%%%%%
\subsection{Driver Development}
\label{sec:driver}

The \textsf{childdoc} mechanism can also be use for the development
of definition files such as \LaTeX{} styles or classes.
This case differs from the above setup with multiple parts
included by |\include| in that no |\includeonly| should be invoked.
This can be achieved by starting the include file
(before |\ProvidesPackage|) with:
%
\begin{center}
\begin{tabular}{l}
|\input{childdoc.def}|\\
|\childdocforward{|\textit{main}|}|\\
\end{tabular}
\end{center}
%
or alternatively with:
%
\begin{center}
\begin{tabular}{l}
|\input{childdoc.def}|\\
|\childdocby{|\textit{main}|}|\\
\end{tabular}
\end{center}
%
Both forms have slightly different effects as described above.
The main file is prepared as usual, see \secref{sec:include}.

%%%%%%%%%%%%%%%%%%%%%%%%%%%%%%%%%%%%%%%%%%%%%%%%%%%%%%%%%%%%%%%%%%%%%%%%%%%%%%%%
\subsection{Legacy Detection}
\label{sec:detection}

The directive |\childdocmain| in the main file can detect
whether the complete document or merely a child is to be compiled
even without using the directive |\childdocof|.
This method is deprecated because it is less robust
and there is no compelling reason to use it;
it is merely provided for backward compatibility
and it may be removed in future versions.

If the detection mechanism is to be used,
it is mandatory to correctly specify
the filename of the main file as the argument of |\childdocmain|:
%
\begin{center}
\begin{tabular}{l}
|\input{childdoc.def}|\\
|\childdocmain{|\textit{main}|}|\\
\end{tabular}
\end{center}
%
If |\jobname| does not match the argument \textit{main} of |\childdocmain|,
it is assumed that |\jobname| points to the child file to be compiled.
When using |\childdocmain| with the main file specified as argument,
it suffices to start a child file
with just |\input{|\textit{main}|}|
without loading of the package and using |\childdocof|.
If instead all processing is done
with the appropriate \textsf{childdoc} directives,
the argument of \textit{main} of |\childdocmain| can be empty.

An alternative version of the command line processing described
in \secref{sec:commandline} using the detection mechanism reads:
%
\begin{center}
|... -jobname "|\textit{target}|" "|[\textit{flags}]%
[|\def\jobname{|\textit{dest}|}|]|\input{|\textit{main}|}"|
\end{center}

%%%%%%%%%%%%%%%%%%%%%%%%%%%%%%%%%%%%%%%%%%%%%%%%%%%%%%%%%%%%%%%%%%%%%%%%%%%%%%%%
\subsection{Manual Code}
\label{sec:manual}

In case one cannot be certain whether the definitions file |childdoc.def|
is installed on the target \TeX{} distribution
and one prefers not to ship it,
it is conceivable to paste a few relevant commands into the sources.

To that end, drop all statements |\input{childdoc.def}|
and perform the replacements as outlined below.
Instead of |\childdocmain{|\textit{main}|}| add the following code
to the top of the main file:
%
\begin{center}
\begin{tabular}{l}
|\||ifdefined\childdocname\endinput\||fi\newif\ifchilddoc|\\
|\edef\childdocname{\scantokens\expandafter{\jobname\noexpand}}|\\
|\def\childdocmain{|\textit{main}|}\||ifx\childdocmain\childdocname\||else|\\
|\childdoctrue\includeonly{\childdocname}\let\jobname\childdocmain\||fi|\\
\end{tabular}
\end{center}
%
Instead of |\childdocof{|\textit{main}|}| just include the main file
at the top of each child file:
%
\begin{center}
|\input{|\textit{main}|}|
\end{center}
%
A simple redirection |\childdocforward{|\textit{dest}|}| is achieved by:
%
\begin{center}
|\def\jobname{|\textit{dest}|}\input{\jobname}|
\end{center}
%
The redirection with prefix
|\childdocforwardprefix[|\textit{prefix}|]{|\textit{dest}|}|
is accomplished by:
%
\begin{center}
\begin{tabular}{l}
|{\edef\jobname{\scantokens\expandafter{\jobname\noexpand}}|\\
|\def\redirectjob |\textit{prefix}|#1~~~{\gdef\jobname{|\textit{dest}|#1}}|\\
|\expandafter\redirectjob\jobname~~~}\input{\jobname}|
\end{tabular}
\end{center}

In an alternative approach,
child documents can be compiled by a specific command line
without additional code or specific definitions:
%
\begin{center}
|... -jobname "|\textit{target}|" "|[\textit{flags}]%
|\includeonly{|\textit{dest}|}\input{|\textit{main}|}"|
\end{center}
%

%%%%%%%%%%%%%%%%%%%%%%%%%%%%%%%%%%%%%%%%%%%%%%%%%%%%%%%%%%%%%%%%%%%%%%%%%%%%%%%%
%%%%%%%%%%%%%%%%%%%%%%%%%%%%%%%%%%%%%%%%%%%%%%%%%%%%%%%%%%%%%%%%%%%%%%%%%%%%%%%%
\section{Information}

%%%%%%%%%%%%%%%%%%%%%%%%%%%%%%%%%%%%%%%%%%%%%%%%%%%%%%%%%%%%%%%%%%%%%%%%%%%%%%%%
\subsection{Copyright}

Copyright \copyright{} 2017--2018 Niklas Beisert

This work may be distributed and/or modified under the
conditions of the \LaTeX{} Project Public License, either version 1.3
of this license or (at your option) any later version.
The latest version of this license is in
  \url{http://www.latex-project.org/lppl.txt}
and version 1.3 or later is part of all distributions of \LaTeX{}
version 2005/12/01 or later.

This work has the LPPL maintenance status `maintained'.

The Current Maintainer of this work is Niklas Beisert.

This work consists of the files |README.txt|, |childdoc.ins| and |childdoc.dtx|
as well as the derived files |childdoc.def|, |cdocsamp.tex|
with |cdocsch1.tex|, |cdocsch2.tex|, |cdocspt3.tex|, |cdocspt4.tex|,
|cdocsdrf.tex|, |cdocsfn1.tex|, |cdocsfn2.tex|
as well as |childdoc.pdf|.

%%%%%%%%%%%%%%%%%%%%%%%%%%%%%%%%%%%%%%%%%%%%%%%%%%%%%%%%%%%%%%%%%%%%%%%%%%%%%%%%
\subsection{Files and Installation}

The package consists of the files:
%
\begin{center}
\begin{tabular}{ll}
    |README.txt|   & readme file \\
    |childdoc.ins| & installation file \\
    |childdoc.dtx| & source file \\
    |childdoc.def| & definition file \\
    |cdocsamp.tex| & sample main file \\
    |cdocsch1.tex| & sample include file \\
    |cdocsch2.tex| & sample include file \\
    |cdocspt3.tex| & sample part file \\
    |cdocspt4.tex| & sample part file \\
    |cdocsdrf.tex| & sample redirection file \\
    |cdocsfn1.tex| & sample redirection file \\
    |cdocsfn2.tex| & sample redirection file \\
    |childdoc.pdf| & manual
\end{tabular}
\end{center}
%
The distribution consists of the files
|README.txt|, |childdoc.ins| and |childdoc.dtx|.
%
\begin{itemize}
\item
Run (pdf)\LaTeX{} on |childdoc.dtx|
to compile the manual |childdoc.pdf| (this file).
\item
Run \LaTeX{} on |childdoc.ins| to create the definitions file |childdoc.def|
and the sample |cdocsamp.tex| with include files
|cdocsch1.tex|, |cdocsch2.tex|, |cdocspt3.tex|, |cdocspt4.tex|,
|cdocsdrf.tex|, |cdocsfn1.tex|, |cdocsfn2.tex|.
Then copy the file |childdoc.def| to an appropriate directory of your \LaTeX{}
distribution, e.g.\ \textit{texmf-root}|/tex/latex/childdoc|.
\end{itemize}

%%%%%%%%%%%%%%%%%%%%%%%%%%%%%%%%%%%%%%%%%%%%%%%%%%%%%%%%%%%%%%%%%%%%%%%%%%%%%%%%
\subsection{Related CTAN Packages}

There are several other packages which offer a similar functionality:
%
\begin{itemize}
\item
The packages
\href{http://ctan.org/pkg/docmute}{\textsf{docmute}},
\href{http://ctan.org/pkg/includex}{\textsf{includex}} and
\href{http://ctan.org/pkg/standalone}{\textsf{standalone}}
provide commands to include only the document body of
a child file thus allowing both files to be compiled individually.
\item
The packages \href{http://ctan.org/pkg/subdocs}{\textsf{subdocs}}
and \href{http://ctan.org/pkg/subfiles}{\textsf{subfiles}}
provide structures in which the main and child documents can be
encapsulated and allowing them to be compiled individually.
The inclusion mechanism is different from the conventional |\include|.
\item
The package \href{http://ctan.org/pkg/combine}{\textsf{combine}}
is an elaborate solution to combine several documents into one.
\end{itemize}
%
See also the CTAN topic \href{http://ctan.org/topic/subdocs}{\textsf{subdocs}}
for further related packages.
The present package differs from the above solutions in that
a document structure constructed with the conventional |\include| mechanism
just needs two extra commands at the top of every file
such that all constituent files can be compiled individually.

%%%%%%%%%%%%%%%%%%%%%%%%%%%%%%%%%%%%%%%%%%%%%%%%%%%%%%%%%%%%%%%%%%%%%%%%%%%%%%%%
%\subsection{Feature Suggestions}
%
%The following is a list of features which may be useful for future
%versions of this package:
%%
%\begin{itemize}
%\item
%\ldots
%\end{itemize}

%%%%%%%%%%%%%%%%%%%%%%%%%%%%%%%%%%%%%%%%%%%%%%%%%%%%%%%%%%%%%%%%%%%%%%%%%%%%%%%%
\subsection{Revision History}

%%%%%%%%%%%%%%%%%%%%%%%%%%%%%%%%%%%%%%%%
\paragraph{v2.0:} 2018/12/30

\begin{itemize}
\item
immediate forward processing
\item
added |\childdocby| mechanism
\item
manual restructured
\end{itemize}

%%%%%%%%%%%%%%%%%%%%%%%%%%%%%%%%%%%%%%%%
\paragraph{v1.6:} 2018/01/17

\begin{itemize}
\item
application for development of include files
\item
corrections to manual
\end{itemize}

%%%%%%%%%%%%%%%%%%%%%%%%%%%%%%%%%%%%%%%%
\paragraph{v1.5:} 2017/05/21

\begin{itemize}
\item
more complete structuring introduced
\item
|\childdocof| introduced
\item
|\childdoc| renamed to |\childdocmain|
\item
|\childredirect| renamed to |\childdocforward| and |\childdocforwardprefix|
and functionality expanded
\end{itemize}

%%%%%%%%%%%%%%%%%%%%%%%%%%%%%%%%%%%%%%%%
\paragraph{v1.0:} 2017/04/27

\begin{itemize}
\item
manual and install package
\item
first version published on CTAN
\end{itemize}

%%%%%%%%%%%%%%%%%%%%%%%%%%%%%%%%%%%%%%%%
\paragraph{v0.6:} 2017/04/26

\begin{itemize}
\item
redirection mechanism added
\end{itemize}

%%%%%%%%%%%%%%%%%%%%%%%%%%%%%%%%%%%%%%%%
\paragraph{v0.5:} 2017/04/26

\begin{itemize}
\item
functionality in definition file
\end{itemize}


%%%%%%%%%%%%%%%%%%%%%%%%%%%%%%%%%%%%%%%%%%%%%%%%%%%%%%%%%%%%%%%%%%%%%%%%%%%%%%%%
%%%%%%%%%%%%%%%%%%%%%%%%%%%%%%%%%%%%%%%%%%%%%%%%%%%%%%%%%%%%%%%%%%%%%%%%%%%%%%%%
%%%%%%%%%%%%%%%%%%%%%%%%%%%%%%%%%%%%%%%%%%%%%%%%%%%%%%%%%%%%%%%%%%%%%%%%%%%%%%%%
\appendix

\settowidth\MacroIndent{\rmfamily\scriptsize 000\ }

 \DocInput{childdoc.dtx}

\end{document}
%</driver>
% \fi
%
% %%%%%%%%%%%%%%%%%%%%%%%%%%%%%%%%%%%%%%%%%%%%%%%%%%%%%%%%%%%%%%%%%%%%%%%%%%%%%%
% %%%%%%%%%%%%%%%%%%%%%%%%%%%%%%%%%%%%%%%%%%%%%%%%%%%%%%%%%%%%%%%%%%%%%%%%%%%%%%
% \section{Sample}
%\iffalse
%<*samplemain>
%\fi
%
% The following presents a sample document
% with two chapters, two parts, a title page,
% a compile flag as well as three forwarding files to set the flag.
% It consists of eight |.tex| files:
% \begin{center}
% \begin{tabular}{ll}
% |cdocsamp.tex|&main file\\
% |cdocsch1.tex|&include file for chapter 1\\
% |cdocsch2.tex|&include file for chapter 2\\
% |cdocspt3.tex|&include file for part 3\\
% |cdocspt4.tex|&include file for part 4\\
% |cdocsdrf.tex|&forwarding file for main file in draft mode\\
% |cdocsfi1.tex|&forwarding file for final version of chapter 1\\
% |cdocsfi2.tex|&forwarding file for final version of chapter 2\\
% \end{tabular}
% \end{center}
% Each of the eight files can be compiled directly by the \LaTeX{} compiler.
%
% %%%%%%%%%%%%%%%%%%%%%%%%%%%%%%%%%%%%%%
% \paragraph{Main File.}
%
% The main file is called |cdocsamp.tex|.
%
% Load the \textsf{childdoc} definitions and
% declare the filename for the main document:
%    \begin{macrocode}
\input{childdoc.def}
\childdocmain{}
%    \end{macrocode}

% Optional override for |\version| flag:
%    \begin{macrocode}
%%\ifchilddoc\else\providecommand{\version}{draft}\fi
%    \end{macrocode}

% Define the default values for the |\version| flag
% (|final| for the main file and |draft| for childs):
%    \begin{macrocode}
\ifchilddoc
\providecommand{\version}{draft}
\else
\providecommand{\version}{final}
\fi
%    \end{macrocode}

% Load the standard document class:
%    \begin{macrocode}
\documentclass[12pt]{article}
%    \end{macrocode}

% Start the document body:
%    \begin{macrocode}
\begin{document}
%    \end{macrocode}

% Declare a title page.
% Print title, part of document being processed and version flag:
%    \begin{macrocode}
\addtocounter{page}{-1}
\begin{center}
{\LARGE\bfseries{}childdoc example\par}
\vspace{1cm}
\ifchilddoc
\ifchilddocmanual part\else chapter\fi:
`\childdocname' of `\childdocjob'\par
\else
main document: `\childdocjob'\par
\fi
version: \version\par
\end{center}
\newpage
%    \end{macrocode}

% Manually include selected file,
% otherwise process as usual:
%    \begin{macrocode}
\ifchilddocmanual
\section*{part `\childdocname'}
\input{\childdocname}
\else
%    \end{macrocode}

% Include the two chapters:
%    \begin{macrocode}
\include{cdocsch1}
\include{cdocsch2}
%    \end{macrocode}

% Include the two parts unless only chapters should be displayed:
%    \begin{macrocode}
\ifchilddoc\else
\section{part three}
\input{cdocspt3}
\section{part four}
\input{cdocspt4}
\fi
%    \end{macrocode}

% Process as usual until here:
%    \begin{macrocode}
\fi
%    \end{macrocode}

% End of document body:
%    \begin{macrocode}
\end{document}
%    \end{macrocode}
%\iffalse
%</samplemain>
%\fi
%
% %%%%%%%%%%%%%%%%%%%%%%%%%%%%%%%%%%%%%%
% \paragraph{Chapter Include Files.}
%
% The include files are called |cdocsch1.tex| and |cdocsch2.tex|.
%
%\iffalse
%<*samplechap1|samplechap2>
%\fi

% Optional override for |\version| flag:
%    \begin{macrocode}
%%\providecommand{\version}{final}
%    \end{macrocode}

% Include the main document:
%    \begin{macrocode}
\input{childdoc.def}
\childdocof{cdocsamp}
%    \end{macrocode}

%\iffalse
%</samplechap1|samplechap2>
%\fi
%
%\iffalse
%<*samplechap1>
%\fi
% Some text for chapter 1:
%    \begin{macrocode}
\section{one}
some text in chapter one
%    \end{macrocode}

%\iffalse
%</samplechap1>
%\fi
% Some text for chapter 2:
%\iffalse
%<*samplechap2>
%\fi
%    \begin{macrocode}
\section{two}
more text in chapter two
%    \end{macrocode}

%\iffalse
%</samplechap2>
%\fi
%
% %%%%%%%%%%%%%%%%%%%%%%%%%%%%%%%%%%%%%%
% \paragraph{Part Include Files.}
%
% The include files are called |cdocspt3.tex| and |cdocspt4.tex|.
%
%\iffalse
%<*samplepart3|samplepart4>
%\fi

% Optional override for |\version| flag:
%    \begin{macrocode}
%%\providecommand{\version}{final}
%    \end{macrocode}

% Include the main document:
%    \begin{macrocode}
\input{childdoc.def}
\childdocby{cdocsamp}
%    \end{macrocode}

%\iffalse
%</samplepart3|samplepart4>
%\fi
%
%\iffalse
%<*samplepart3>
%\fi
% Some text for part 3:
%    \begin{macrocode}
some text in part three
%    \end{macrocode}

%\iffalse
%</samplepart3>
%\fi
% Some text for part 4:
%\iffalse
%<*samplepart4>
%\fi
%    \begin{macrocode}
more text in part four
%    \end{macrocode}

%\iffalse
%</samplepart4>
%\fi
%
% %%%%%%%%%%%%%%%%%%%%%%%%%%%%%%%%%%%%%%
% \paragraph{Forwarding for a Complete Draft.}
%
% The following forwarding file |cdocsdrf.tex|
% compiles the main document in draft mode:
%\iffalse
%<*sampledraft>
%\fi
%    \begin{macrocode}
\def\version{draft}
\input{childdoc.def}
\childdocforward{cdocsamp}
%    \end{macrocode}

%\iffalse
%</sampledraft>
%\fi
%
% %%%%%%%%%%%%%%%%%%%%%%%%%%%%%%%%%%%%%%
% \paragraph{Forwarding for Final Version of the Chapters.}
%
% The following forwarding files |cdocsfn1.tex| and |cdocsfn2.tex|
% (with identical content)
% compile the final versions of the child documents
% |cdocsch1.tex| and |cdocsch2.tex|, respectively:
%\iffalse
%<*samplefinal>
%\fi
%    \begin{macrocode}
\def\version{final}
\input{childdoc.def}
\childdocforwardprefix[cdocsamp]{cdocsfn}{cdocsch}
%    \end{macrocode}

%\iffalse
%</samplefinal>
%\fi
%
% %%%%%%%%%%%%%%%%%%%%%%%%%%%%%%%%%%%%%%
% \paragraph{Command Line Processing.}
%
% The following three command lines generate the output files
% |cdocscld|, |cdocscl1| and |cdocscl2|
% which should be identical to
% |cdocsdrf|, |cdocsch1| and |cdocsfn2|, respectively:
% \begin{center}
% \begin{tabular}{l}
% |latex -jobname cdocscld \|\\
% |  "\def\version{draft}\input{childdoc.def}\childdocforward{cdocsamp}"|\\
% |latex -jobname cdocscl1 \|\\
% |  "\input{childdoc.def}\childdocforward[cdocsamp]{cdocsch1}"|\\
% |latex -jobname cdocscl2 \|\\
% |  "\def\version{final}\input{childdoc.def}\childdocforward{cdocsch2}"|
% \end{tabular}
% \end{center}
% Note that the trailing backslash on each first line
% merely continues the input to the second line
% (for convenient cut ant paste).
% Furthermore, the command |latex| can be replaced by any
% of its alternative versions such as |pdflatex|.
%
% %%%%%%%%%%%%%%%%%%%%%%%%%%%%%%%%%%%%%%%%%%%%%%%%%%%%%%%%%%%%%%%%%%%%%%%%%%%%%%
% %%%%%%%%%%%%%%%%%%%%%%%%%%%%%%%%%%%%%%%%%%%%%%%%%%%%%%%%%%%%%%%%%%%%%%%%%%%%%%
% \section{Implementation}
%\iffalse
%<*package>
%\fi
%
% This section describes the definitions file |childdoc.def|.

% The definitions cannot be loaded using |\usepackage| or |\RequirePackage|
% which has a mechanism to prevent loading a style file more than once.
% When loading the definitions by means of |\input|
% multiple instances have to be prevented manually:
%\iffalse
%This code needs to be before the `\ProvidesFile' directive
%which is defined at the beginning of this file.
%Therefore it is also placed there and commented out here.
%</package>
%<*discard>
%\fi
%    \begin{macrocode}
\ifdefined\childdocmain\endinput\fi
%    \end{macrocode}
%\iffalse
%</discard>
%<*package>
%\fi
%
% \macro{\ifchilddoc}
% \macro{\ifchilddocmanual}
% The conditional |\ifchilddoc| tells whether a
% child (true) or main (false) document is being compiled.
% The conditional |\ifchilddocmanual| tells whether
% the |\includeonly| mechanism is used (false) or
% the selection of child files must be performed manually (true).
% The definitions initialise to false:
%    \begin{macrocode}
\newif\ifchilddoc
\newif\ifchilddocmanual
%    \end{macrocode}

% \macro{\childdocname}
% \macro{\childdocjob}
% The macro |\childdocname| stores the name of the main document
% to be compiled. The macro |\childdocjob| stores the name of
% the document on which the \LaTeX{} compiler was originally invoked.
% The content of |\jobname| cannot be compared
% to filenames specified in the source due to different catcodes.
% The following code rescans |\jobname|, stores the result
% in |\childdocname| and saves a copy in |\childdocjob|:
%    \begin{macrocode}
\edef\childdocname{\scantokens\expandafter{\jobname\noexpand}}
\let\childdocjob\childdocname
%    \end{macrocode}

% \macro{\childdocdisable}
% The macro |\childdocdisable| prevents the main file
% from being processed more than once.
% At this stage, the main document command |\childdocmain|
% is assumed to be called once again where it should do nothing.
% Any subsequent call to it should prevent
% a secondary processing of the main document
% It overwrites the forwarding commands
% |\childdocof| and |\childdocforward|
% with empty macros to prevent further inclusions of the main document:
%    \begin{macrocode}
\newcommand{\childdocdisable}
{
  \renewcommand{\childdocmain}[1]{\renewcommand{\childdocmain}[1]{\endinput}}
  \renewcommand{\childdocof}[1]{}
  \renewcommand{\childdocby}[2][]{}
  \renewcommand{\childdocforward}[2][]{}
  \renewcommand{\childdocdisable}{}
}
%    \end{macrocode}

% \macro{\childdocmain}
% The macro |\childdocmain| is to be called at the top of the main file
% with nothing or the main filename (without extension) as argument.
% First, it breaks loops.
% If the argument is not empty and does not match |\childdocname|
% (which is set by the first inclusion of |childdoc.def|),
% |\ifchilddoc| is set to true, |\includeonly| is applied to the child file
% and |\jobname| is set to the main file
% (for proper handling of |.aux| files):
%    \begin{macrocode}
\newcommand{\childdocmain}[1]
{
  \childdocdisable\childdocmain{}
  \if?#1?\else
    \begingroup
      \def\childdoctmp{#1}
      \ifx\childdoctmp\childdocname
        \def\childdoctmp{}
      \else
        \def\childdoctmp
        {
          \childdoctrue
          \includeonly{\childdocname}
          \def\childdocjob{#1}
          \def\jobname{#1}
        }
      \fi
      \expandafter
    \endgroup
    \childdoctmp
  \fi
}
%    \end{macrocode}

% \macro{\childdocof}
% The command |\childdocof| redirects
% compilation to the main file |#1|.
%    \begin{macrocode}
\newcommand{\childdocof}[1]
{
  \childdocdisable
  \childdoctrue
  \includeonly{\childdocname}
  \def\jobname{#1}
  \def\childdocjob{#1}
  \input{#1}
}
%    \end{macrocode}

% \macro{\childdocby}
% The command |\childdocby| ....
%    \begin{macrocode}
\newcommand{\childdocby}[2][]
{
  \childdocdisable
  \childdoctrue
  \childdocmanualtrue
  \if?#1?\else
    \def\jobname{#2}
  \fi
  \def\childdocjob{#2}
  \input{#2}
  \endinput
}
%    \end{macrocode}

% \macro{\childdocforward}
% The command |\childdocforward| redirects
% compilation to the main file or
% (if the optional argument is given) a child file.
% Parameters are set as if the main file
% or a child file starting with |\childdocof| was compiled.
% Then compilation is handed over to the main file:
%    \begin{macrocode}
\newcommand{\childdocforward}[2][]
{
  \begingroup
    \if?#1?
      \def\childdoctmp
      {
        \def\childdocname{#2}
        \def\childdocjob{#2}
        \def\jobname{#2}
        \input{#2}
        \endinput
      }
    \else
      \def\childdoctmp
      {
        \childdocdisable
        \def\childdocname{#2}
        \childdoctrue
        \includeonly{#2}
        \def\childdocjob{#1}
        \def\jobname{#1}
        \input{#1}
        \endinput
      }
    \fi
    \expandafter
  \endgroup
  \childdoctmp
}
%    \end{macrocode}

% \macro{\childdocforwardprefix}
% The command |\childdocforwardprefix| redirects
% compilation to the main or a child file by means of a pattern.
% The prefix |#1| in the current filename is replaced by |#2|
% and the suffix of the current filename is kept
% (it is assumed that the filename does not contain the substring `|~~~|'
% which is used as a delimiter).
% Compilation is handed over to the new file by |\childdocforward|:
%    \begin{macrocode}
\newcommand{\childdocforwardprefix}[3][]
{
  \begingroup
    \def\childdocextract #2##1~~~{\def\childdoctmp{\childdocforward[#1]{#3##1}}}
    \expandafter\childdocextract\childdocname~~~
    \expandafter
  \endgroup
  \childdoctmp
}
%    \end{macrocode}

% \macro{\childdoc}
% The deprecated macro |\childdoc| is a legacy version of |\childdocmain|:
%    \begin{macrocode}
\newcommand{\childdoc}{\childdocmain}
%    \end{macrocode}

% \macro{\childdocredirect}
% The deprecated macro |\childdocredirect| is a legacy version
% of |\childdocforward| and |\childdocforwardprefix|:
%    \begin{macrocode}
\newcommand{\childdocredirect}[2][]
{
  \begingroup
    \if?#1?
      \def\childdoctmp{\childdocforward{#2}}
    \else
      \def\childdoctmp{\childdocforwardprefix{#1}{#2}}
    \fi
    \expandafter
  \endgroup
  \childdoctmp
}
%    \end{macrocode}

%\iffalse
%</package>
%\fi
%
\endinput
|\\
|\childdocforwardprefix[|\textit{main}|]{|\textit{prefix}|}{|\textit{dest}|}|
\end{tabular}
\end{center}
%
the destination file is determined by a pattern
depending on the current file:
To make this work, the current file must be called
`{\textit{prefix}\hspace{0.2em}\textit{suffix}}'
with \textit{prefix} matching precisely the argument.
Processing is then passed on to the file
`{\textit{dest}\hspace{0.2em}\textit{suffix}}'.
Surely, the same effect is achieved by
directly specifying the
argument `{\textit{dest}\hspace{0.2em}\textit{suffix}}'
in the first form.
However, that requires to set up a different file
for each child. With the alternative form of the command
all these files can have exactly the same content
which simplifies setting them up and maintaining them.

For example, the following file |draft.tex|
with a compilation flag |\version| as described in \secref{sec:flags}
compiles the main document as a draft:
%
\begin{center}
\begin{tabular}{l}
|\def\version{draft}|\\
|% \iffalse
%
% childdoc.dtx Copyright (C) 2017-2018 Niklas Beisert
%
% This work may be distributed and/or modified under the
% conditions of the LaTeX Project Public License, either version 1.3
% of this license or (at your option) any later version.
% The latest version of this license is in
%   http://www.latex-project.org/lppl.txt
% and version 1.3 or later is part of all distributions of LaTeX
% version 2005/12/01 or later.
%
% This work has the LPPL maintenance status `maintained'.
%
% The Current Maintainer of this work is Niklas Beisert.
%
% This work consists of the files childdoc.dtx and childdoc.ins
% and the derived files childdoc.def and cdocsamp.tex with
% cdocsch1.tex, cdocsch2.tex, cdocsdrf.tex, cdocsfn1.tex, cdocsfn2.tex.
%
%<package>\ifdefined\childdocmain\endinput\fi
%<package>\ProvidesFile{childdoc.def}[2018/12/30 v2.0 child document driver]
%<samplemain>\ProvidesFile{cdocsamp.tex}[2018/12/30 v2.0 sample for childdoc]
%<*driver>
%\ProvidesFile{childdoc.drv}[2018/12/30 v2.0 childdoc reference manual file]
\PassOptionsToClass{10pt,a4paper}{article}
\documentclass{ltxdoc}

\usepackage[margin=35mm]{geometry}
\usepackage{hyperref}
\usepackage{hyperxmp}
\usepackage[usenames]{color}

\hypersetup{colorlinks=true}
\hypersetup{pdfstartview=FitH}
\hypersetup{pdfpagemode=UseNone}
\hypersetup{pdfsource={}}
\hypersetup{pdflang={en-UK}}
\hypersetup{pdfcopyright={Copyright 2017-2018 Niklas Beisert.
  This work may be distributed and/or modified under the
  conditions of the LaTeX Project Public License, either version 1.3
  of this license or (at your option) any later version.}}
\hypersetup{pdflicenseurl={http://www.latex-project.org/lppl.txt}}
\hypersetup{pdfcontactaddress={ETH Zurich, ITP, HIT K,
  Wolfgang-Pauli-Strasse 27}}
\hypersetup{pdfcontactpostcode={8093}}
\hypersetup{pdfcontactcity={Zurich}}
\hypersetup{pdfcontactcountry={Switzerland}}
\hypersetup{pdfcontactemail={nbeisert@itp.phys.ethz.ch}}
\hypersetup{pdfcontacturl={http://people.phys.ethz.ch/\xmptilde nbeisert/}}

\newcommand{\secref}[1]{\hyperref[#1]{section \ref*{#1}}}

\parskip1ex
\parindent0pt
\let\olditemize\itemize
\def\itemize{\olditemize\parskip0pt}

\begin{document}

\title{The \textsf{childdoc} Package}
\hypersetup{pdftitle={The childdoc Package}}
\author{Niklas Beisert\\[2ex]
  Institut f\"ur Theoretische Physik\\
  Eidgen\"ossische Technische Hochschule Z\"urich\\
  Wolfgang-Pauli-Strasse 27, 8093 Z\"urich, Switzerland\\[1ex]
  \href{mailto:nbeisert@itp.phys.ethz.ch}
  {\texttt{nbeisert@itp.phys.ethz.ch}}}
\hypersetup{pdfauthor={Niklas Beisert}}
\hypersetup{pdfsubject={Manual for the LaTeX2e Package childdoc}}
\date{30 December 2018, \textsf{v2.0}}
\maketitle

\begin{abstract}\noindent
\textsf{childdoc} is a \LaTeXe{} package
that enables the direct compilation
of document sections included by |\include|
to individual files.
\end{abstract}

\begingroup
\parskip0ex
\tableofcontents
\endgroup

%%%%%%%%%%%%%%%%%%%%%%%%%%%%%%%%%%%%%%%%%%%%%%%%%%%%%%%%%%%%%%%%%%%%%%%%%%%%%%%%
%%%%%%%%%%%%%%%%%%%%%%%%%%%%%%%%%%%%%%%%%%%%%%%%%%%%%%%%%%%%%%%%%%%%%%%%%%%%%%%%
\section{Introduction}

\LaTeX{} provides a mechanism to structure a large document (such as a book)
into a main file and several child files (containing the chapters)
using the |\include| command.
This mechanism is beneficial for documents
which span hundreds of pages in order to
make the source file(s) more manageable.
Moreover, compilation can be restricted to
selected child files by means of the |\includeonly| command.
The latter feature can be used to reduce the compilation time while editing
(this was significantly more useful in the earlier days of \LaTeX{})
or to generate a smaller document which is easier to navigate.
Another application of |\includeonly| is to generate
documents consisting of selected parts of the complete document.

However, there are a few drawbacks of the plain |\include| mechanism:
\begin{itemize}
\item
The child files cannot be compiled on their own,
they can only be compiled via the main file.
A naive editing environment
(such as a text editor with an option
to have the current file processed by \LaTeX)
may require one to switch to the main file before compiling;
attempting to compile the child file produces errors.
\item
The main file must be modified (each time)
to adjust the |\includeonly| command
to the present needs. This easily leaves the main file in a messy state.
\item
The generated document will always carry the filename
of the main document. This is inconvenient if
several child files are to be compiled and
to be kept for distribution.
\end{itemize}

The present package provides a simple interface
to make child files individually compilable by \LaTeX{}.
Compiling a child file then has the same effect as compiling
the main file with an |\includeonly| command
to select the appropriate child.
Moreover the generated document will carry the name of the child
rather than the main file.
This resolves all three above issues.

This feature is meant to make the editing of books,
thesis documents and lecture notes somewhat more convenient.
However, the package can also be used efficiently for
composing a series of documents (such as exercise sheets)
which are typically distributed individually.
It then assists the author in generating the individual documents
(potentially in different versions)
as well as a document containing the collected series.
Another application is in developing style files
or other kinds of included material
where compilation of the style file could redirect
to a sample or test file.

%%%%%%%%%%%%%%%%%%%%%%%%%%%%%%%%%%%%%%%%%%%%%%%%%%%%%%%%%%%%%%%%%%%%%%%%%%%%%%%%
%%%%%%%%%%%%%%%%%%%%%%%%%%%%%%%%%%%%%%%%%%%%%%%%%%%%%%%%%%%%%%%%%%%%%%%%%%%%%%%%
\section{Usage}

First of all, the package \textsf{childdoc} is \emph{not} a standard
\LaTeXe{} |.sty| style file! Therefore it needs to be invoked in
a non-standard way.

%%%%%%%%%%%%%%%%%%%%%%%%%%%%%%%%%%%%%%%%%%%%%%%%%%%%%%%%%%%%%%%%%%%%%%%%%%%%%%%%
\subsection{Included Files}
\label{sec:include}

%%%%%%%%%%%%%%%%%%%%%%%%%%%%%%%%%%%%%%%%
\DescribeMacro{\childdocmain}
To use the package, add the commands
\begin{center}
\begin{tabular}{l}
|\input{childdoc.def}|\\
|\childdocmain{}|\\
\end{tabular}
\end{center}
at the very top of the main \LaTeX{} file,
in particular \emph{before} the |\documentclass| statement!
The argument of |\childdocmain| should be left empty
(but it must be present).

%%%%%%%%%%%%%%%%%%%%%%%%%%%%%%%%%%%%%%%%
\DescribeMacro{\childdocof}
Furthermore, add the commands
\begin{center}
\begin{tabular}{l}
|\input{childdoc.def}|\\
|\childdocof{|\textit{main}|}|\\
\end{tabular}
\end{center}
at the top of every child file \textit{child}
which is included by |\include{|\textit{child}|}|
from within the main file
(or at least for those files to be compiled individually).
The argument \textit{main} must be the filename of the main file.

There are a couple of
considerations in setting up the main and child documents:

%%%%%%%%%%%%%%%%%%%%%%%%%%%%%%%%%%%%%%%%
\paragraph{Restrictions.}

Please note the following restrictions:
\begin{itemize}
\item
|\childdocmain| must be called with one argument \textit{main}
to ensure compatibility with earlier version of the package.
It must either be empty (|\childdocmain{}|)
or precisely match the filename of the main file in which it is specified.
See \secref{sec:detection} for further information.
\item
The filename \textit{main} must be specified without the |.tex| extension.
\item
The filename \textit{main} is case sensitive
(even in case-insensitive file systems)
due to internal string comparison.
\item
The argument \textit{main} should be fully expanded, it cannot be a macro.
\item
Subdirectories and special characters should be avoided in filenames.
\item
The command |\childdocmain{|\textit{main}|}| must be followed by a whitespace.
It should not be followed immediately by another command
or by a comment mark `|%|'.
This is because the \TeX{} parser reads the token immediately following
the argument of |\childdocmain| and puts it
at the beginning of every child section;
however, a white\-space is ignored.
\end{itemize}

%%%%%%%%%%%%%%%%%%%%%%%%%%%%%%%%%%%%%%%%
\paragraph{Content of Main File.}

It is advisable to place all content in the child files included by |\include|.
Any output contained in the main file will appear in all child documents
unless suppressed manually;
it cannot be suppressed automatically by the |\includeonly| directive
and thus should normally be avoided.
A method to include some content in the main file
by means of conditional processing is described in \secref{sec:conditional}.

%%%%%%%%%%%%%%%%%%%%%%%%%%%%%%%%%%%%%%%%
\paragraph{Page Numbering.}

When only a part of the document is compiled,
the appropriate numbering of pages
(as well as other status parameters)
is determined from the |.aux| files.
The latter contain information from previous passes.
However this information needs to propagate through
all intermediate child documents.
Therefore the page numbering in child documents may well
be inconsistent until the complete document is compiled at least once.

A useful (if unconventional) way to always ensure a consistent
page numbering is to restart the numbering in each child document
and denote the pages by `\textit{child}|.|\textit{page}'
where \textit{child} represents the chapter/section number of the child file.
This can be achieved by the command
|\numberwithin{page}{|\textit{child}|}|
of the \textsf{amsmath} package
where \textit{child} can be |chapter| or |section|
depending on the chosen structuring.
Alternatively, one can modify the macro |\thepage| appropriately
and reset the counter |page| at the start of each child file.

%%%%%%%%%%%%%%%%%%%%%%%%%%%%%%%%%%%%%%%%%%%%%%%%%%%%%%%%%%%%%%%%%%%%%%%%%%%%%%%%
\subsection{Conditional Processing}
\label{sec:conditional}

The package provides a mechanism to compile different versions
of a document. To customise the versions further some conditional processing
can come in handy to distinguish which version is being compiled.
The package provides two macros to describe the compilation context:

%%%%%%%%%%%%%%%%%%%%%%%%%%%%%%%%%%%%%%%%
\DescribeMacro{\ifchilddoc}
The conditional |\ifchilddoc| distinguishes between the compilation of
child documents and the main document:
%
\begin{center}
|\ifchilddoc |\textit{child-code}| |[|\||else |\textit{main-code}]| \||fi|
\end{center}

%%%%%%%%%%%%%%%%%%%%%%%%%%%%%%%%%%%%%%%%
\DescribeMacro{\childdocname}
\DescribeMacro{\childdocjob}
The macro |\childdocname| contains the filename (without extension)
of the main or child file being processed.
Note that |\childdocjob| will always contain the name of the main file.

%%%%%%%%%%%%%%%%%%%%%%%%%%%%%%%%%%%%%%%%
\paragraph{Title Page.}

Conditional processing can be used to include a title or banner page
in the main document when proper precautions are taken.
Importantly, the code in the main file should ensure that the page counter
(as well as other status parameters which are stored in the |.aux| files)
takes the same value after the conditional processing.
Otherwise the page numbers may take divergent values
depending on which part is compiled.

For example, a title page could be declared by:
%
\begin{center}
\begin{tabular}{l}
|\ifchilddoc\||else|\\
|\addtocounter{page}{-1}|\\
\textit{code for title page}\\
|\newpage|\\
|\||fi|
\end{tabular}
\end{center}
%
A banner page for the child documents can be generated by:
%
\begin{center}
\begin{tabular}{l}
|\ifchilddoc|\\
|\addtocounter{page}{-1}|\\
\textit{code for banner page}\\
|\newpage|\\
|\||fi|
\end{tabular}
\end{center}
%
Here one could write a message such as:
\begin{center}
|This is the part \childdocname{} of \childdocjob{}.|
\end{center}

%%%%%%%%%%%%%%%%%%%%%%%%%%%%%%%%%%%%%%%%%%%%%%%%%%%%%%%%%%%%%%%%%%%%%%%%%%%%%%%%
\subsection{Flags}
\label{sec:flags}

The package makes it easy to generate different versions
of the main or child documents.
To this end compilation flags can be defined
and assigned different default values.
They will be particularly useful in conjunction
with the forwarding mechanism described in \secref{sec:forward}.

For example, it may be useful to have a flag |\version|
which can be set to |draft| or |final|.
The document source will contain some conditional code
depending on the value of |\version|.
Suppose further, the flag should default to |final| for the main file
and to |draft| for child files
which is a natural assignment for editing the document.
This is achieved by placing the following code
in the preamble of the main document
(below the |\childdocmain| directive):
%
\begin{center}
\begin{tabular}{l}
|\ifchilddoc|\\
|\providecommand{\version}{draft}|\\
|\||else|\\
|\providecommand{\version}{final}|\\
|\||fi|
\end{tabular}
\end{center}
%
The definition by |\providecommand| makes sure
that previous definitions are not overwritten.
Further statements |\providecommand{\version}{...}|
can thus be added before the above code to override it.

For the main file, one might add a line
(between |\childdocmain| and the above block)
%
\begin{center}
|%\ifchilddoc\||else\providecommand{\version}{draft}\||fi|
\end{center}
%
which can be uncommented to produce a draft version.
Likewise one can add a line to the very top of a child file
(above the |\childdocof{|\textit{main}|}| directive)
%
\begin{center}
|%\providecommand{\version}{final}|
\end{center}
%
which can be uncommented to produce the final version of this child document.

%%%%%%%%%%%%%%%%%%%%%%%%%%%%%%%%%%%%%%%%%%%%%%%%%%%%%%%%%%%%%%%%%%%%%%%%%%%%%%%%
\subsection{Forwarding}
\label{sec:forward}

Different versions of the main or child documents
using compilation flags as described in \secref{sec:flags}
can be (permanently) stored in different files
for convenient compilation, viewing and distribution.
To this end, the package defines a command
to pass on compilation to a different file:

%%%%%%%%%%%%%%%%%%%%%%%%%%%%%%%%%%%%%%%%
\DescribeMacro{\childdocforward}
The command |\childdocforward| redirects processing to
another source file:
%
\begin{center}
\begin{tabular}{l}
|\input{childdoc.def}|\\
|\childdocforward[|\textit{main}|]{|\textit{dest}|}|\\
\end{tabular}
\end{center}
%
The argument \textit{dest} is the destination file
(without extension).
It should be the main file or one of the child files.
Note that further \textsf{childdoc} directives
such as |\childdocof| and |\childdocforward|
in the indicated file will be processed in this form.
The optional argument \textit{main}
passes on directly to the main file \textit{main}
while pretending to compile the child \textit{dest}.
This form behaves as if \textit{dest}
issues |\childdocof{|\textit{main}|}| right away,
and no further \textsf{childdoc} directives will be processed.

%%%%%%%%%%%%%%%%%%%%%%%%%%%%%%%%%%%%%%%%
\DescribeMacro{\...prefix}
In the alternative form |\childdocforwardprefix|,
%
\begin{center}
\begin{tabular}{l}
|\input{childdoc.def}|\\
|\childdocforwardprefix[|\textit{main}|]{|\textit{prefix}|}{|\textit{dest}|}|
\end{tabular}
\end{center}
%
the destination file is determined by a pattern
depending on the current file:
To make this work, the current file must be called
`{\textit{prefix}\hspace{0.2em}\textit{suffix}}'
with \textit{prefix} matching precisely the argument.
Processing is then passed on to the file
`{\textit{dest}\hspace{0.2em}\textit{suffix}}'.
Surely, the same effect is achieved by
directly specifying the
argument `{\textit{dest}\hspace{0.2em}\textit{suffix}}'
in the first form.
However, that requires to set up a different file
for each child. With the alternative form of the command
all these files can have exactly the same content
which simplifies setting them up and maintaining them.

For example, the following file |draft.tex|
with a compilation flag |\version| as described in \secref{sec:flags}
compiles the main document as a draft:
%
\begin{center}
\begin{tabular}{l}
|\def\version{draft}|\\
|\input{childdoc.def}|\\
|\childdocforward{|\textit{main}|}|
\end{tabular}
\end{center}
%
Likewise, the following files |final|\textit{nn}|.tex|
compile the final version of the child document
|child|\textit{nn}|.tex|:
%
\begin{center}
\begin{tabular}{l}
|\def\version{final}|\\
|\input{childdoc.def}|\\
|\childdocforwardprefix{final}{child}|
\end{tabular}
\end{center}
%

Note that when several versions of a main file and/or of each child file
are to be generated, it may be convenient to set up a |Makefile| or
shell script to automatise the process.

%%%%%%%%%%%%%%%%%%%%%%%%%%%%%%%%%%%%%%%%%%%%%%%%%%%%%%%%%%%%%%%%%%%%%%%%%%%%%%%%
\subsection{Command Line Processing}
\label{sec:commandline}

The effect of redirection files can also be achieved by invoking
the \LaTeX{} compiler with a more elaborate command line.
Most conveniently this should be done as part
of a shell script or a |Makefile|.

When using \textsf{childdoc} in the main file, the following
command lines effectively perform a redirection
(note that depending on the shell being used,
backslashes may have to be doubled: `|\|' $\to$ `|\\|'):
%
\begin{center}
|... -jobname "|\textit{target}|" |\\|"|[\textit{flags}]%
|\input{childdoc.def}\childdocforward[|\textit{main}|]{|\textit{dest}|}"|
\end{center}
%
Here \textit{target} is the name of the output file,
\textit{main} is the name of the main file
and \textit{dest} is the name of the main or child file to be processed
(all filenames without extensions).
The optional argument \textit{main} can be omitted
if \textit{main} matches \textit{dest}.
Optionally, compilation \textit{flags} can be defined via |\def| commands.
This command line makes the \TeX{} engine believe
it is compiling the file \textit{target}
whose content is specified as the latter parameter.
The provided code then forwards the processing to
\textit{main} or \textit{dest} as described in \secref{sec:forward}.

%%%%%%%%%%%%%%%%%%%%%%%%%%%%%%%%%%%%%%%%%%%%%%%%%%%%%%%%%%%%%%%%%%%%%%%%%%%%%%%%
\subsection{Include by Input}
\label{sec:input}

Including child documents by |\include| has some restrictions by design.
Most notably, the content of a child document always occupies
its own set of pages; pages cannot be shared between child documents.
Usually, this behaviour makes perfect sense
because each child document contain an essential part of the document.
However, in some situations it may be desirable to compose
a document from a collection of parts
without having mandatory page breaks between then.
For this case, the package
provides a mechanism to include parts
by |\input| which can also be processed individually.
However, by construction this mechanism
requires manual handling of the content to be output.

%%%%%%%%%%%%%%%%%%%%%%%%%%%%%%%%%%%%%%%%
\DescribeMacro{\ifchilddocmanual}
The main file should be prepared as usual, see \secref{sec:include}.
However, the document body must make a distinction
between processing of an individual part and of the main document, e.g.:
%
\begin{center}
\begin{tabular}{l}
|\ifchilddocmanual|\\
|\input{\childdocname}|\\
|\||else|\\
\textit{document body with }|\input{|\textit{part}|}|\\
|\||fi|
\end{tabular}
\end{center}
%
The conditional |\ifchilddocmanual| is true whenever
a part to be included by |\input| is being compiled,
and the name of the part is stored in |\childdocname|.

%%%%%%%%%%%%%%%%%%%%%%%%%%%%%%%%%%%%%%%%
\DescribeMacro{\childdocby}
Each part to be included by |\input| should start with:
%
\begin{center}
\begin{tabular}{l}
|\input{childdoc.def}|\\
|\childdocby{|\textit{main}|}|\\
\end{tabular}
\end{center}
%
The directive |\childdocby| is similar to |\childdocof|
described in \secref{sec:include},
but the subsequent selection of content must be done manually.
To that end, both |\ifchilddoc| and |\ifchilddocmanual|
will be true upon processing of a part,
and the name of the part is stored in |\childdocname|.
Note that |\jobname| will be set to the filename of the current part
so that each part receives an individual |.aux| file
that does not interfere with the |.aux| file(s) of the main document.
This behaviour can be altered by the alternative form
|\childdocby[*]{|\textit{main}|}| (with a non-empty optional argument)
which uses the |.aux| file of the main document
by setting |\jobname| to \textit{main}.

%%%%%%%%%%%%%%%%%%%%%%%%%%%%%%%%%%%%%%%%%%%%%%%%%%%%%%%%%%%%%%%%%%%%%%%%%%%%%%%%
\subsection{Driver Development}
\label{sec:driver}

The \textsf{childdoc} mechanism can also be use for the development
of definition files such as \LaTeX{} styles or classes.
This case differs from the above setup with multiple parts
included by |\include| in that no |\includeonly| should be invoked.
This can be achieved by starting the include file
(before |\ProvidesPackage|) with:
%
\begin{center}
\begin{tabular}{l}
|\input{childdoc.def}|\\
|\childdocforward{|\textit{main}|}|\\
\end{tabular}
\end{center}
%
or alternatively with:
%
\begin{center}
\begin{tabular}{l}
|\input{childdoc.def}|\\
|\childdocby{|\textit{main}|}|\\
\end{tabular}
\end{center}
%
Both forms have slightly different effects as described above.
The main file is prepared as usual, see \secref{sec:include}.

%%%%%%%%%%%%%%%%%%%%%%%%%%%%%%%%%%%%%%%%%%%%%%%%%%%%%%%%%%%%%%%%%%%%%%%%%%%%%%%%
\subsection{Legacy Detection}
\label{sec:detection}

The directive |\childdocmain| in the main file can detect
whether the complete document or merely a child is to be compiled
even without using the directive |\childdocof|.
This method is deprecated because it is less robust
and there is no compelling reason to use it;
it is merely provided for backward compatibility
and it may be removed in future versions.

If the detection mechanism is to be used,
it is mandatory to correctly specify
the filename of the main file as the argument of |\childdocmain|:
%
\begin{center}
\begin{tabular}{l}
|\input{childdoc.def}|\\
|\childdocmain{|\textit{main}|}|\\
\end{tabular}
\end{center}
%
If |\jobname| does not match the argument \textit{main} of |\childdocmain|,
it is assumed that |\jobname| points to the child file to be compiled.
When using |\childdocmain| with the main file specified as argument,
it suffices to start a child file
with just |\input{|\textit{main}|}|
without loading of the package and using |\childdocof|.
If instead all processing is done
with the appropriate \textsf{childdoc} directives,
the argument of \textit{main} of |\childdocmain| can be empty.

An alternative version of the command line processing described
in \secref{sec:commandline} using the detection mechanism reads:
%
\begin{center}
|... -jobname "|\textit{target}|" "|[\textit{flags}]%
[|\def\jobname{|\textit{dest}|}|]|\input{|\textit{main}|}"|
\end{center}

%%%%%%%%%%%%%%%%%%%%%%%%%%%%%%%%%%%%%%%%%%%%%%%%%%%%%%%%%%%%%%%%%%%%%%%%%%%%%%%%
\subsection{Manual Code}
\label{sec:manual}

In case one cannot be certain whether the definitions file |childdoc.def|
is installed on the target \TeX{} distribution
and one prefers not to ship it,
it is conceivable to paste a few relevant commands into the sources.

To that end, drop all statements |\input{childdoc.def}|
and perform the replacements as outlined below.
Instead of |\childdocmain{|\textit{main}|}| add the following code
to the top of the main file:
%
\begin{center}
\begin{tabular}{l}
|\||ifdefined\childdocname\endinput\||fi\newif\ifchilddoc|\\
|\edef\childdocname{\scantokens\expandafter{\jobname\noexpand}}|\\
|\def\childdocmain{|\textit{main}|}\||ifx\childdocmain\childdocname\||else|\\
|\childdoctrue\includeonly{\childdocname}\let\jobname\childdocmain\||fi|\\
\end{tabular}
\end{center}
%
Instead of |\childdocof{|\textit{main}|}| just include the main file
at the top of each child file:
%
\begin{center}
|\input{|\textit{main}|}|
\end{center}
%
A simple redirection |\childdocforward{|\textit{dest}|}| is achieved by:
%
\begin{center}
|\def\jobname{|\textit{dest}|}\input{\jobname}|
\end{center}
%
The redirection with prefix
|\childdocforwardprefix[|\textit{prefix}|]{|\textit{dest}|}|
is accomplished by:
%
\begin{center}
\begin{tabular}{l}
|{\edef\jobname{\scantokens\expandafter{\jobname\noexpand}}|\\
|\def\redirectjob |\textit{prefix}|#1~~~{\gdef\jobname{|\textit{dest}|#1}}|\\
|\expandafter\redirectjob\jobname~~~}\input{\jobname}|
\end{tabular}
\end{center}

In an alternative approach,
child documents can be compiled by a specific command line
without additional code or specific definitions:
%
\begin{center}
|... -jobname "|\textit{target}|" "|[\textit{flags}]%
|\includeonly{|\textit{dest}|}\input{|\textit{main}|}"|
\end{center}
%

%%%%%%%%%%%%%%%%%%%%%%%%%%%%%%%%%%%%%%%%%%%%%%%%%%%%%%%%%%%%%%%%%%%%%%%%%%%%%%%%
%%%%%%%%%%%%%%%%%%%%%%%%%%%%%%%%%%%%%%%%%%%%%%%%%%%%%%%%%%%%%%%%%%%%%%%%%%%%%%%%
\section{Information}

%%%%%%%%%%%%%%%%%%%%%%%%%%%%%%%%%%%%%%%%%%%%%%%%%%%%%%%%%%%%%%%%%%%%%%%%%%%%%%%%
\subsection{Copyright}

Copyright \copyright{} 2017--2018 Niklas Beisert

This work may be distributed and/or modified under the
conditions of the \LaTeX{} Project Public License, either version 1.3
of this license or (at your option) any later version.
The latest version of this license is in
  \url{http://www.latex-project.org/lppl.txt}
and version 1.3 or later is part of all distributions of \LaTeX{}
version 2005/12/01 or later.

This work has the LPPL maintenance status `maintained'.

The Current Maintainer of this work is Niklas Beisert.

This work consists of the files |README.txt|, |childdoc.ins| and |childdoc.dtx|
as well as the derived files |childdoc.def|, |cdocsamp.tex|
with |cdocsch1.tex|, |cdocsch2.tex|, |cdocspt3.tex|, |cdocspt4.tex|,
|cdocsdrf.tex|, |cdocsfn1.tex|, |cdocsfn2.tex|
as well as |childdoc.pdf|.

%%%%%%%%%%%%%%%%%%%%%%%%%%%%%%%%%%%%%%%%%%%%%%%%%%%%%%%%%%%%%%%%%%%%%%%%%%%%%%%%
\subsection{Files and Installation}

The package consists of the files:
%
\begin{center}
\begin{tabular}{ll}
    |README.txt|   & readme file \\
    |childdoc.ins| & installation file \\
    |childdoc.dtx| & source file \\
    |childdoc.def| & definition file \\
    |cdocsamp.tex| & sample main file \\
    |cdocsch1.tex| & sample include file \\
    |cdocsch2.tex| & sample include file \\
    |cdocspt3.tex| & sample part file \\
    |cdocspt4.tex| & sample part file \\
    |cdocsdrf.tex| & sample redirection file \\
    |cdocsfn1.tex| & sample redirection file \\
    |cdocsfn2.tex| & sample redirection file \\
    |childdoc.pdf| & manual
\end{tabular}
\end{center}
%
The distribution consists of the files
|README.txt|, |childdoc.ins| and |childdoc.dtx|.
%
\begin{itemize}
\item
Run (pdf)\LaTeX{} on |childdoc.dtx|
to compile the manual |childdoc.pdf| (this file).
\item
Run \LaTeX{} on |childdoc.ins| to create the definitions file |childdoc.def|
and the sample |cdocsamp.tex| with include files
|cdocsch1.tex|, |cdocsch2.tex|, |cdocspt3.tex|, |cdocspt4.tex|,
|cdocsdrf.tex|, |cdocsfn1.tex|, |cdocsfn2.tex|.
Then copy the file |childdoc.def| to an appropriate directory of your \LaTeX{}
distribution, e.g.\ \textit{texmf-root}|/tex/latex/childdoc|.
\end{itemize}

%%%%%%%%%%%%%%%%%%%%%%%%%%%%%%%%%%%%%%%%%%%%%%%%%%%%%%%%%%%%%%%%%%%%%%%%%%%%%%%%
\subsection{Related CTAN Packages}

There are several other packages which offer a similar functionality:
%
\begin{itemize}
\item
The packages
\href{http://ctan.org/pkg/docmute}{\textsf{docmute}},
\href{http://ctan.org/pkg/includex}{\textsf{includex}} and
\href{http://ctan.org/pkg/standalone}{\textsf{standalone}}
provide commands to include only the document body of
a child file thus allowing both files to be compiled individually.
\item
The packages \href{http://ctan.org/pkg/subdocs}{\textsf{subdocs}}
and \href{http://ctan.org/pkg/subfiles}{\textsf{subfiles}}
provide structures in which the main and child documents can be
encapsulated and allowing them to be compiled individually.
The inclusion mechanism is different from the conventional |\include|.
\item
The package \href{http://ctan.org/pkg/combine}{\textsf{combine}}
is an elaborate solution to combine several documents into one.
\end{itemize}
%
See also the CTAN topic \href{http://ctan.org/topic/subdocs}{\textsf{subdocs}}
for further related packages.
The present package differs from the above solutions in that
a document structure constructed with the conventional |\include| mechanism
just needs two extra commands at the top of every file
such that all constituent files can be compiled individually.

%%%%%%%%%%%%%%%%%%%%%%%%%%%%%%%%%%%%%%%%%%%%%%%%%%%%%%%%%%%%%%%%%%%%%%%%%%%%%%%%
%\subsection{Feature Suggestions}
%
%The following is a list of features which may be useful for future
%versions of this package:
%%
%\begin{itemize}
%\item
%\ldots
%\end{itemize}

%%%%%%%%%%%%%%%%%%%%%%%%%%%%%%%%%%%%%%%%%%%%%%%%%%%%%%%%%%%%%%%%%%%%%%%%%%%%%%%%
\subsection{Revision History}

%%%%%%%%%%%%%%%%%%%%%%%%%%%%%%%%%%%%%%%%
\paragraph{v2.0:} 2018/12/30

\begin{itemize}
\item
immediate forward processing
\item
added |\childdocby| mechanism
\item
manual restructured
\end{itemize}

%%%%%%%%%%%%%%%%%%%%%%%%%%%%%%%%%%%%%%%%
\paragraph{v1.6:} 2018/01/17

\begin{itemize}
\item
application for development of include files
\item
corrections to manual
\end{itemize}

%%%%%%%%%%%%%%%%%%%%%%%%%%%%%%%%%%%%%%%%
\paragraph{v1.5:} 2017/05/21

\begin{itemize}
\item
more complete structuring introduced
\item
|\childdocof| introduced
\item
|\childdoc| renamed to |\childdocmain|
\item
|\childredirect| renamed to |\childdocforward| and |\childdocforwardprefix|
and functionality expanded
\end{itemize}

%%%%%%%%%%%%%%%%%%%%%%%%%%%%%%%%%%%%%%%%
\paragraph{v1.0:} 2017/04/27

\begin{itemize}
\item
manual and install package
\item
first version published on CTAN
\end{itemize}

%%%%%%%%%%%%%%%%%%%%%%%%%%%%%%%%%%%%%%%%
\paragraph{v0.6:} 2017/04/26

\begin{itemize}
\item
redirection mechanism added
\end{itemize}

%%%%%%%%%%%%%%%%%%%%%%%%%%%%%%%%%%%%%%%%
\paragraph{v0.5:} 2017/04/26

\begin{itemize}
\item
functionality in definition file
\end{itemize}


%%%%%%%%%%%%%%%%%%%%%%%%%%%%%%%%%%%%%%%%%%%%%%%%%%%%%%%%%%%%%%%%%%%%%%%%%%%%%%%%
%%%%%%%%%%%%%%%%%%%%%%%%%%%%%%%%%%%%%%%%%%%%%%%%%%%%%%%%%%%%%%%%%%%%%%%%%%%%%%%%
%%%%%%%%%%%%%%%%%%%%%%%%%%%%%%%%%%%%%%%%%%%%%%%%%%%%%%%%%%%%%%%%%%%%%%%%%%%%%%%%
\appendix

\settowidth\MacroIndent{\rmfamily\scriptsize 000\ }

 \DocInput{childdoc.dtx}

\end{document}
%</driver>
% \fi
%
% %%%%%%%%%%%%%%%%%%%%%%%%%%%%%%%%%%%%%%%%%%%%%%%%%%%%%%%%%%%%%%%%%%%%%%%%%%%%%%
% %%%%%%%%%%%%%%%%%%%%%%%%%%%%%%%%%%%%%%%%%%%%%%%%%%%%%%%%%%%%%%%%%%%%%%%%%%%%%%
% \section{Sample}
%\iffalse
%<*samplemain>
%\fi
%
% The following presents a sample document
% with two chapters, two parts, a title page,
% a compile flag as well as three forwarding files to set the flag.
% It consists of eight |.tex| files:
% \begin{center}
% \begin{tabular}{ll}
% |cdocsamp.tex|&main file\\
% |cdocsch1.tex|&include file for chapter 1\\
% |cdocsch2.tex|&include file for chapter 2\\
% |cdocspt3.tex|&include file for part 3\\
% |cdocspt4.tex|&include file for part 4\\
% |cdocsdrf.tex|&forwarding file for main file in draft mode\\
% |cdocsfi1.tex|&forwarding file for final version of chapter 1\\
% |cdocsfi2.tex|&forwarding file for final version of chapter 2\\
% \end{tabular}
% \end{center}
% Each of the eight files can be compiled directly by the \LaTeX{} compiler.
%
% %%%%%%%%%%%%%%%%%%%%%%%%%%%%%%%%%%%%%%
% \paragraph{Main File.}
%
% The main file is called |cdocsamp.tex|.
%
% Load the \textsf{childdoc} definitions and
% declare the filename for the main document:
%    \begin{macrocode}
\input{childdoc.def}
\childdocmain{}
%    \end{macrocode}

% Optional override for |\version| flag:
%    \begin{macrocode}
%%\ifchilddoc\else\providecommand{\version}{draft}\fi
%    \end{macrocode}

% Define the default values for the |\version| flag
% (|final| for the main file and |draft| for childs):
%    \begin{macrocode}
\ifchilddoc
\providecommand{\version}{draft}
\else
\providecommand{\version}{final}
\fi
%    \end{macrocode}

% Load the standard document class:
%    \begin{macrocode}
\documentclass[12pt]{article}
%    \end{macrocode}

% Start the document body:
%    \begin{macrocode}
\begin{document}
%    \end{macrocode}

% Declare a title page.
% Print title, part of document being processed and version flag:
%    \begin{macrocode}
\addtocounter{page}{-1}
\begin{center}
{\LARGE\bfseries{}childdoc example\par}
\vspace{1cm}
\ifchilddoc
\ifchilddocmanual part\else chapter\fi:
`\childdocname' of `\childdocjob'\par
\else
main document: `\childdocjob'\par
\fi
version: \version\par
\end{center}
\newpage
%    \end{macrocode}

% Manually include selected file,
% otherwise process as usual:
%    \begin{macrocode}
\ifchilddocmanual
\section*{part `\childdocname'}
\input{\childdocname}
\else
%    \end{macrocode}

% Include the two chapters:
%    \begin{macrocode}
\include{cdocsch1}
\include{cdocsch2}
%    \end{macrocode}

% Include the two parts unless only chapters should be displayed:
%    \begin{macrocode}
\ifchilddoc\else
\section{part three}
\input{cdocspt3}
\section{part four}
\input{cdocspt4}
\fi
%    \end{macrocode}

% Process as usual until here:
%    \begin{macrocode}
\fi
%    \end{macrocode}

% End of document body:
%    \begin{macrocode}
\end{document}
%    \end{macrocode}
%\iffalse
%</samplemain>
%\fi
%
% %%%%%%%%%%%%%%%%%%%%%%%%%%%%%%%%%%%%%%
% \paragraph{Chapter Include Files.}
%
% The include files are called |cdocsch1.tex| and |cdocsch2.tex|.
%
%\iffalse
%<*samplechap1|samplechap2>
%\fi

% Optional override for |\version| flag:
%    \begin{macrocode}
%%\providecommand{\version}{final}
%    \end{macrocode}

% Include the main document:
%    \begin{macrocode}
\input{childdoc.def}
\childdocof{cdocsamp}
%    \end{macrocode}

%\iffalse
%</samplechap1|samplechap2>
%\fi
%
%\iffalse
%<*samplechap1>
%\fi
% Some text for chapter 1:
%    \begin{macrocode}
\section{one}
some text in chapter one
%    \end{macrocode}

%\iffalse
%</samplechap1>
%\fi
% Some text for chapter 2:
%\iffalse
%<*samplechap2>
%\fi
%    \begin{macrocode}
\section{two}
more text in chapter two
%    \end{macrocode}

%\iffalse
%</samplechap2>
%\fi
%
% %%%%%%%%%%%%%%%%%%%%%%%%%%%%%%%%%%%%%%
% \paragraph{Part Include Files.}
%
% The include files are called |cdocspt3.tex| and |cdocspt4.tex|.
%
%\iffalse
%<*samplepart3|samplepart4>
%\fi

% Optional override for |\version| flag:
%    \begin{macrocode}
%%\providecommand{\version}{final}
%    \end{macrocode}

% Include the main document:
%    \begin{macrocode}
\input{childdoc.def}
\childdocby{cdocsamp}
%    \end{macrocode}

%\iffalse
%</samplepart3|samplepart4>
%\fi
%
%\iffalse
%<*samplepart3>
%\fi
% Some text for part 3:
%    \begin{macrocode}
some text in part three
%    \end{macrocode}

%\iffalse
%</samplepart3>
%\fi
% Some text for part 4:
%\iffalse
%<*samplepart4>
%\fi
%    \begin{macrocode}
more text in part four
%    \end{macrocode}

%\iffalse
%</samplepart4>
%\fi
%
% %%%%%%%%%%%%%%%%%%%%%%%%%%%%%%%%%%%%%%
% \paragraph{Forwarding for a Complete Draft.}
%
% The following forwarding file |cdocsdrf.tex|
% compiles the main document in draft mode:
%\iffalse
%<*sampledraft>
%\fi
%    \begin{macrocode}
\def\version{draft}
\input{childdoc.def}
\childdocforward{cdocsamp}
%    \end{macrocode}

%\iffalse
%</sampledraft>
%\fi
%
% %%%%%%%%%%%%%%%%%%%%%%%%%%%%%%%%%%%%%%
% \paragraph{Forwarding for Final Version of the Chapters.}
%
% The following forwarding files |cdocsfn1.tex| and |cdocsfn2.tex|
% (with identical content)
% compile the final versions of the child documents
% |cdocsch1.tex| and |cdocsch2.tex|, respectively:
%\iffalse
%<*samplefinal>
%\fi
%    \begin{macrocode}
\def\version{final}
\input{childdoc.def}
\childdocforwardprefix[cdocsamp]{cdocsfn}{cdocsch}
%    \end{macrocode}

%\iffalse
%</samplefinal>
%\fi
%
% %%%%%%%%%%%%%%%%%%%%%%%%%%%%%%%%%%%%%%
% \paragraph{Command Line Processing.}
%
% The following three command lines generate the output files
% |cdocscld|, |cdocscl1| and |cdocscl2|
% which should be identical to
% |cdocsdrf|, |cdocsch1| and |cdocsfn2|, respectively:
% \begin{center}
% \begin{tabular}{l}
% |latex -jobname cdocscld \|\\
% |  "\def\version{draft}\input{childdoc.def}\childdocforward{cdocsamp}"|\\
% |latex -jobname cdocscl1 \|\\
% |  "\input{childdoc.def}\childdocforward[cdocsamp]{cdocsch1}"|\\
% |latex -jobname cdocscl2 \|\\
% |  "\def\version{final}\input{childdoc.def}\childdocforward{cdocsch2}"|
% \end{tabular}
% \end{center}
% Note that the trailing backslash on each first line
% merely continues the input to the second line
% (for convenient cut ant paste).
% Furthermore, the command |latex| can be replaced by any
% of its alternative versions such as |pdflatex|.
%
% %%%%%%%%%%%%%%%%%%%%%%%%%%%%%%%%%%%%%%%%%%%%%%%%%%%%%%%%%%%%%%%%%%%%%%%%%%%%%%
% %%%%%%%%%%%%%%%%%%%%%%%%%%%%%%%%%%%%%%%%%%%%%%%%%%%%%%%%%%%%%%%%%%%%%%%%%%%%%%
% \section{Implementation}
%\iffalse
%<*package>
%\fi
%
% This section describes the definitions file |childdoc.def|.

% The definitions cannot be loaded using |\usepackage| or |\RequirePackage|
% which has a mechanism to prevent loading a style file more than once.
% When loading the definitions by means of |\input|
% multiple instances have to be prevented manually:
%\iffalse
%This code needs to be before the `\ProvidesFile' directive
%which is defined at the beginning of this file.
%Therefore it is also placed there and commented out here.
%</package>
%<*discard>
%\fi
%    \begin{macrocode}
\ifdefined\childdocmain\endinput\fi
%    \end{macrocode}
%\iffalse
%</discard>
%<*package>
%\fi
%
% \macro{\ifchilddoc}
% \macro{\ifchilddocmanual}
% The conditional |\ifchilddoc| tells whether a
% child (true) or main (false) document is being compiled.
% The conditional |\ifchilddocmanual| tells whether
% the |\includeonly| mechanism is used (false) or
% the selection of child files must be performed manually (true).
% The definitions initialise to false:
%    \begin{macrocode}
\newif\ifchilddoc
\newif\ifchilddocmanual
%    \end{macrocode}

% \macro{\childdocname}
% \macro{\childdocjob}
% The macro |\childdocname| stores the name of the main document
% to be compiled. The macro |\childdocjob| stores the name of
% the document on which the \LaTeX{} compiler was originally invoked.
% The content of |\jobname| cannot be compared
% to filenames specified in the source due to different catcodes.
% The following code rescans |\jobname|, stores the result
% in |\childdocname| and saves a copy in |\childdocjob|:
%    \begin{macrocode}
\edef\childdocname{\scantokens\expandafter{\jobname\noexpand}}
\let\childdocjob\childdocname
%    \end{macrocode}

% \macro{\childdocdisable}
% The macro |\childdocdisable| prevents the main file
% from being processed more than once.
% At this stage, the main document command |\childdocmain|
% is assumed to be called once again where it should do nothing.
% Any subsequent call to it should prevent
% a secondary processing of the main document
% It overwrites the forwarding commands
% |\childdocof| and |\childdocforward|
% with empty macros to prevent further inclusions of the main document:
%    \begin{macrocode}
\newcommand{\childdocdisable}
{
  \renewcommand{\childdocmain}[1]{\renewcommand{\childdocmain}[1]{\endinput}}
  \renewcommand{\childdocof}[1]{}
  \renewcommand{\childdocby}[2][]{}
  \renewcommand{\childdocforward}[2][]{}
  \renewcommand{\childdocdisable}{}
}
%    \end{macrocode}

% \macro{\childdocmain}
% The macro |\childdocmain| is to be called at the top of the main file
% with nothing or the main filename (without extension) as argument.
% First, it breaks loops.
% If the argument is not empty and does not match |\childdocname|
% (which is set by the first inclusion of |childdoc.def|),
% |\ifchilddoc| is set to true, |\includeonly| is applied to the child file
% and |\jobname| is set to the main file
% (for proper handling of |.aux| files):
%    \begin{macrocode}
\newcommand{\childdocmain}[1]
{
  \childdocdisable\childdocmain{}
  \if?#1?\else
    \begingroup
      \def\childdoctmp{#1}
      \ifx\childdoctmp\childdocname
        \def\childdoctmp{}
      \else
        \def\childdoctmp
        {
          \childdoctrue
          \includeonly{\childdocname}
          \def\childdocjob{#1}
          \def\jobname{#1}
        }
      \fi
      \expandafter
    \endgroup
    \childdoctmp
  \fi
}
%    \end{macrocode}

% \macro{\childdocof}
% The command |\childdocof| redirects
% compilation to the main file |#1|.
%    \begin{macrocode}
\newcommand{\childdocof}[1]
{
  \childdocdisable
  \childdoctrue
  \includeonly{\childdocname}
  \def\jobname{#1}
  \def\childdocjob{#1}
  \input{#1}
}
%    \end{macrocode}

% \macro{\childdocby}
% The command |\childdocby| ....
%    \begin{macrocode}
\newcommand{\childdocby}[2][]
{
  \childdocdisable
  \childdoctrue
  \childdocmanualtrue
  \if?#1?\else
    \def\jobname{#2}
  \fi
  \def\childdocjob{#2}
  \input{#2}
  \endinput
}
%    \end{macrocode}

% \macro{\childdocforward}
% The command |\childdocforward| redirects
% compilation to the main file or
% (if the optional argument is given) a child file.
% Parameters are set as if the main file
% or a child file starting with |\childdocof| was compiled.
% Then compilation is handed over to the main file:
%    \begin{macrocode}
\newcommand{\childdocforward}[2][]
{
  \begingroup
    \if?#1?
      \def\childdoctmp
      {
        \def\childdocname{#2}
        \def\childdocjob{#2}
        \def\jobname{#2}
        \input{#2}
        \endinput
      }
    \else
      \def\childdoctmp
      {
        \childdocdisable
        \def\childdocname{#2}
        \childdoctrue
        \includeonly{#2}
        \def\childdocjob{#1}
        \def\jobname{#1}
        \input{#1}
        \endinput
      }
    \fi
    \expandafter
  \endgroup
  \childdoctmp
}
%    \end{macrocode}

% \macro{\childdocforwardprefix}
% The command |\childdocforwardprefix| redirects
% compilation to the main or a child file by means of a pattern.
% The prefix |#1| in the current filename is replaced by |#2|
% and the suffix of the current filename is kept
% (it is assumed that the filename does not contain the substring `|~~~|'
% which is used as a delimiter).
% Compilation is handed over to the new file by |\childdocforward|:
%    \begin{macrocode}
\newcommand{\childdocforwardprefix}[3][]
{
  \begingroup
    \def\childdocextract #2##1~~~{\def\childdoctmp{\childdocforward[#1]{#3##1}}}
    \expandafter\childdocextract\childdocname~~~
    \expandafter
  \endgroup
  \childdoctmp
}
%    \end{macrocode}

% \macro{\childdoc}
% The deprecated macro |\childdoc| is a legacy version of |\childdocmain|:
%    \begin{macrocode}
\newcommand{\childdoc}{\childdocmain}
%    \end{macrocode}

% \macro{\childdocredirect}
% The deprecated macro |\childdocredirect| is a legacy version
% of |\childdocforward| and |\childdocforwardprefix|:
%    \begin{macrocode}
\newcommand{\childdocredirect}[2][]
{
  \begingroup
    \if?#1?
      \def\childdoctmp{\childdocforward{#2}}
    \else
      \def\childdoctmp{\childdocforwardprefix{#1}{#2}}
    \fi
    \expandafter
  \endgroup
  \childdoctmp
}
%    \end{macrocode}

%\iffalse
%</package>
%\fi
%
\endinput
|\\
|\childdocforward{|\textit{main}|}|
\end{tabular}
\end{center}
%
Likewise, the following files |final|\textit{nn}|.tex|
compile the final version of the child document
|child|\textit{nn}|.tex|:
%
\begin{center}
\begin{tabular}{l}
|\def\version{final}|\\
|% \iffalse
%
% childdoc.dtx Copyright (C) 2017-2018 Niklas Beisert
%
% This work may be distributed and/or modified under the
% conditions of the LaTeX Project Public License, either version 1.3
% of this license or (at your option) any later version.
% The latest version of this license is in
%   http://www.latex-project.org/lppl.txt
% and version 1.3 or later is part of all distributions of LaTeX
% version 2005/12/01 or later.
%
% This work has the LPPL maintenance status `maintained'.
%
% The Current Maintainer of this work is Niklas Beisert.
%
% This work consists of the files childdoc.dtx and childdoc.ins
% and the derived files childdoc.def and cdocsamp.tex with
% cdocsch1.tex, cdocsch2.tex, cdocsdrf.tex, cdocsfn1.tex, cdocsfn2.tex.
%
%<package>\ifdefined\childdocmain\endinput\fi
%<package>\ProvidesFile{childdoc.def}[2018/12/30 v2.0 child document driver]
%<samplemain>\ProvidesFile{cdocsamp.tex}[2018/12/30 v2.0 sample for childdoc]
%<*driver>
%\ProvidesFile{childdoc.drv}[2018/12/30 v2.0 childdoc reference manual file]
\PassOptionsToClass{10pt,a4paper}{article}
\documentclass{ltxdoc}

\usepackage[margin=35mm]{geometry}
\usepackage{hyperref}
\usepackage{hyperxmp}
\usepackage[usenames]{color}

\hypersetup{colorlinks=true}
\hypersetup{pdfstartview=FitH}
\hypersetup{pdfpagemode=UseNone}
\hypersetup{pdfsource={}}
\hypersetup{pdflang={en-UK}}
\hypersetup{pdfcopyright={Copyright 2017-2018 Niklas Beisert.
  This work may be distributed and/or modified under the
  conditions of the LaTeX Project Public License, either version 1.3
  of this license or (at your option) any later version.}}
\hypersetup{pdflicenseurl={http://www.latex-project.org/lppl.txt}}
\hypersetup{pdfcontactaddress={ETH Zurich, ITP, HIT K,
  Wolfgang-Pauli-Strasse 27}}
\hypersetup{pdfcontactpostcode={8093}}
\hypersetup{pdfcontactcity={Zurich}}
\hypersetup{pdfcontactcountry={Switzerland}}
\hypersetup{pdfcontactemail={nbeisert@itp.phys.ethz.ch}}
\hypersetup{pdfcontacturl={http://people.phys.ethz.ch/\xmptilde nbeisert/}}

\newcommand{\secref}[1]{\hyperref[#1]{section \ref*{#1}}}

\parskip1ex
\parindent0pt
\let\olditemize\itemize
\def\itemize{\olditemize\parskip0pt}

\begin{document}

\title{The \textsf{childdoc} Package}
\hypersetup{pdftitle={The childdoc Package}}
\author{Niklas Beisert\\[2ex]
  Institut f\"ur Theoretische Physik\\
  Eidgen\"ossische Technische Hochschule Z\"urich\\
  Wolfgang-Pauli-Strasse 27, 8093 Z\"urich, Switzerland\\[1ex]
  \href{mailto:nbeisert@itp.phys.ethz.ch}
  {\texttt{nbeisert@itp.phys.ethz.ch}}}
\hypersetup{pdfauthor={Niklas Beisert}}
\hypersetup{pdfsubject={Manual for the LaTeX2e Package childdoc}}
\date{30 December 2018, \textsf{v2.0}}
\maketitle

\begin{abstract}\noindent
\textsf{childdoc} is a \LaTeXe{} package
that enables the direct compilation
of document sections included by |\include|
to individual files.
\end{abstract}

\begingroup
\parskip0ex
\tableofcontents
\endgroup

%%%%%%%%%%%%%%%%%%%%%%%%%%%%%%%%%%%%%%%%%%%%%%%%%%%%%%%%%%%%%%%%%%%%%%%%%%%%%%%%
%%%%%%%%%%%%%%%%%%%%%%%%%%%%%%%%%%%%%%%%%%%%%%%%%%%%%%%%%%%%%%%%%%%%%%%%%%%%%%%%
\section{Introduction}

\LaTeX{} provides a mechanism to structure a large document (such as a book)
into a main file and several child files (containing the chapters)
using the |\include| command.
This mechanism is beneficial for documents
which span hundreds of pages in order to
make the source file(s) more manageable.
Moreover, compilation can be restricted to
selected child files by means of the |\includeonly| command.
The latter feature can be used to reduce the compilation time while editing
(this was significantly more useful in the earlier days of \LaTeX{})
or to generate a smaller document which is easier to navigate.
Another application of |\includeonly| is to generate
documents consisting of selected parts of the complete document.

However, there are a few drawbacks of the plain |\include| mechanism:
\begin{itemize}
\item
The child files cannot be compiled on their own,
they can only be compiled via the main file.
A naive editing environment
(such as a text editor with an option
to have the current file processed by \LaTeX)
may require one to switch to the main file before compiling;
attempting to compile the child file produces errors.
\item
The main file must be modified (each time)
to adjust the |\includeonly| command
to the present needs. This easily leaves the main file in a messy state.
\item
The generated document will always carry the filename
of the main document. This is inconvenient if
several child files are to be compiled and
to be kept for distribution.
\end{itemize}

The present package provides a simple interface
to make child files individually compilable by \LaTeX{}.
Compiling a child file then has the same effect as compiling
the main file with an |\includeonly| command
to select the appropriate child.
Moreover the generated document will carry the name of the child
rather than the main file.
This resolves all three above issues.

This feature is meant to make the editing of books,
thesis documents and lecture notes somewhat more convenient.
However, the package can also be used efficiently for
composing a series of documents (such as exercise sheets)
which are typically distributed individually.
It then assists the author in generating the individual documents
(potentially in different versions)
as well as a document containing the collected series.
Another application is in developing style files
or other kinds of included material
where compilation of the style file could redirect
to a sample or test file.

%%%%%%%%%%%%%%%%%%%%%%%%%%%%%%%%%%%%%%%%%%%%%%%%%%%%%%%%%%%%%%%%%%%%%%%%%%%%%%%%
%%%%%%%%%%%%%%%%%%%%%%%%%%%%%%%%%%%%%%%%%%%%%%%%%%%%%%%%%%%%%%%%%%%%%%%%%%%%%%%%
\section{Usage}

First of all, the package \textsf{childdoc} is \emph{not} a standard
\LaTeXe{} |.sty| style file! Therefore it needs to be invoked in
a non-standard way.

%%%%%%%%%%%%%%%%%%%%%%%%%%%%%%%%%%%%%%%%%%%%%%%%%%%%%%%%%%%%%%%%%%%%%%%%%%%%%%%%
\subsection{Included Files}
\label{sec:include}

%%%%%%%%%%%%%%%%%%%%%%%%%%%%%%%%%%%%%%%%
\DescribeMacro{\childdocmain}
To use the package, add the commands
\begin{center}
\begin{tabular}{l}
|\input{childdoc.def}|\\
|\childdocmain{}|\\
\end{tabular}
\end{center}
at the very top of the main \LaTeX{} file,
in particular \emph{before} the |\documentclass| statement!
The argument of |\childdocmain| should be left empty
(but it must be present).

%%%%%%%%%%%%%%%%%%%%%%%%%%%%%%%%%%%%%%%%
\DescribeMacro{\childdocof}
Furthermore, add the commands
\begin{center}
\begin{tabular}{l}
|\input{childdoc.def}|\\
|\childdocof{|\textit{main}|}|\\
\end{tabular}
\end{center}
at the top of every child file \textit{child}
which is included by |\include{|\textit{child}|}|
from within the main file
(or at least for those files to be compiled individually).
The argument \textit{main} must be the filename of the main file.

There are a couple of
considerations in setting up the main and child documents:

%%%%%%%%%%%%%%%%%%%%%%%%%%%%%%%%%%%%%%%%
\paragraph{Restrictions.}

Please note the following restrictions:
\begin{itemize}
\item
|\childdocmain| must be called with one argument \textit{main}
to ensure compatibility with earlier version of the package.
It must either be empty (|\childdocmain{}|)
or precisely match the filename of the main file in which it is specified.
See \secref{sec:detection} for further information.
\item
The filename \textit{main} must be specified without the |.tex| extension.
\item
The filename \textit{main} is case sensitive
(even in case-insensitive file systems)
due to internal string comparison.
\item
The argument \textit{main} should be fully expanded, it cannot be a macro.
\item
Subdirectories and special characters should be avoided in filenames.
\item
The command |\childdocmain{|\textit{main}|}| must be followed by a whitespace.
It should not be followed immediately by another command
or by a comment mark `|%|'.
This is because the \TeX{} parser reads the token immediately following
the argument of |\childdocmain| and puts it
at the beginning of every child section;
however, a white\-space is ignored.
\end{itemize}

%%%%%%%%%%%%%%%%%%%%%%%%%%%%%%%%%%%%%%%%
\paragraph{Content of Main File.}

It is advisable to place all content in the child files included by |\include|.
Any output contained in the main file will appear in all child documents
unless suppressed manually;
it cannot be suppressed automatically by the |\includeonly| directive
and thus should normally be avoided.
A method to include some content in the main file
by means of conditional processing is described in \secref{sec:conditional}.

%%%%%%%%%%%%%%%%%%%%%%%%%%%%%%%%%%%%%%%%
\paragraph{Page Numbering.}

When only a part of the document is compiled,
the appropriate numbering of pages
(as well as other status parameters)
is determined from the |.aux| files.
The latter contain information from previous passes.
However this information needs to propagate through
all intermediate child documents.
Therefore the page numbering in child documents may well
be inconsistent until the complete document is compiled at least once.

A useful (if unconventional) way to always ensure a consistent
page numbering is to restart the numbering in each child document
and denote the pages by `\textit{child}|.|\textit{page}'
where \textit{child} represents the chapter/section number of the child file.
This can be achieved by the command
|\numberwithin{page}{|\textit{child}|}|
of the \textsf{amsmath} package
where \textit{child} can be |chapter| or |section|
depending on the chosen structuring.
Alternatively, one can modify the macro |\thepage| appropriately
and reset the counter |page| at the start of each child file.

%%%%%%%%%%%%%%%%%%%%%%%%%%%%%%%%%%%%%%%%%%%%%%%%%%%%%%%%%%%%%%%%%%%%%%%%%%%%%%%%
\subsection{Conditional Processing}
\label{sec:conditional}

The package provides a mechanism to compile different versions
of a document. To customise the versions further some conditional processing
can come in handy to distinguish which version is being compiled.
The package provides two macros to describe the compilation context:

%%%%%%%%%%%%%%%%%%%%%%%%%%%%%%%%%%%%%%%%
\DescribeMacro{\ifchilddoc}
The conditional |\ifchilddoc| distinguishes between the compilation of
child documents and the main document:
%
\begin{center}
|\ifchilddoc |\textit{child-code}| |[|\||else |\textit{main-code}]| \||fi|
\end{center}

%%%%%%%%%%%%%%%%%%%%%%%%%%%%%%%%%%%%%%%%
\DescribeMacro{\childdocname}
\DescribeMacro{\childdocjob}
The macro |\childdocname| contains the filename (without extension)
of the main or child file being processed.
Note that |\childdocjob| will always contain the name of the main file.

%%%%%%%%%%%%%%%%%%%%%%%%%%%%%%%%%%%%%%%%
\paragraph{Title Page.}

Conditional processing can be used to include a title or banner page
in the main document when proper precautions are taken.
Importantly, the code in the main file should ensure that the page counter
(as well as other status parameters which are stored in the |.aux| files)
takes the same value after the conditional processing.
Otherwise the page numbers may take divergent values
depending on which part is compiled.

For example, a title page could be declared by:
%
\begin{center}
\begin{tabular}{l}
|\ifchilddoc\||else|\\
|\addtocounter{page}{-1}|\\
\textit{code for title page}\\
|\newpage|\\
|\||fi|
\end{tabular}
\end{center}
%
A banner page for the child documents can be generated by:
%
\begin{center}
\begin{tabular}{l}
|\ifchilddoc|\\
|\addtocounter{page}{-1}|\\
\textit{code for banner page}\\
|\newpage|\\
|\||fi|
\end{tabular}
\end{center}
%
Here one could write a message such as:
\begin{center}
|This is the part \childdocname{} of \childdocjob{}.|
\end{center}

%%%%%%%%%%%%%%%%%%%%%%%%%%%%%%%%%%%%%%%%%%%%%%%%%%%%%%%%%%%%%%%%%%%%%%%%%%%%%%%%
\subsection{Flags}
\label{sec:flags}

The package makes it easy to generate different versions
of the main or child documents.
To this end compilation flags can be defined
and assigned different default values.
They will be particularly useful in conjunction
with the forwarding mechanism described in \secref{sec:forward}.

For example, it may be useful to have a flag |\version|
which can be set to |draft| or |final|.
The document source will contain some conditional code
depending on the value of |\version|.
Suppose further, the flag should default to |final| for the main file
and to |draft| for child files
which is a natural assignment for editing the document.
This is achieved by placing the following code
in the preamble of the main document
(below the |\childdocmain| directive):
%
\begin{center}
\begin{tabular}{l}
|\ifchilddoc|\\
|\providecommand{\version}{draft}|\\
|\||else|\\
|\providecommand{\version}{final}|\\
|\||fi|
\end{tabular}
\end{center}
%
The definition by |\providecommand| makes sure
that previous definitions are not overwritten.
Further statements |\providecommand{\version}{...}|
can thus be added before the above code to override it.

For the main file, one might add a line
(between |\childdocmain| and the above block)
%
\begin{center}
|%\ifchilddoc\||else\providecommand{\version}{draft}\||fi|
\end{center}
%
which can be uncommented to produce a draft version.
Likewise one can add a line to the very top of a child file
(above the |\childdocof{|\textit{main}|}| directive)
%
\begin{center}
|%\providecommand{\version}{final}|
\end{center}
%
which can be uncommented to produce the final version of this child document.

%%%%%%%%%%%%%%%%%%%%%%%%%%%%%%%%%%%%%%%%%%%%%%%%%%%%%%%%%%%%%%%%%%%%%%%%%%%%%%%%
\subsection{Forwarding}
\label{sec:forward}

Different versions of the main or child documents
using compilation flags as described in \secref{sec:flags}
can be (permanently) stored in different files
for convenient compilation, viewing and distribution.
To this end, the package defines a command
to pass on compilation to a different file:

%%%%%%%%%%%%%%%%%%%%%%%%%%%%%%%%%%%%%%%%
\DescribeMacro{\childdocforward}
The command |\childdocforward| redirects processing to
another source file:
%
\begin{center}
\begin{tabular}{l}
|\input{childdoc.def}|\\
|\childdocforward[|\textit{main}|]{|\textit{dest}|}|\\
\end{tabular}
\end{center}
%
The argument \textit{dest} is the destination file
(without extension).
It should be the main file or one of the child files.
Note that further \textsf{childdoc} directives
such as |\childdocof| and |\childdocforward|
in the indicated file will be processed in this form.
The optional argument \textit{main}
passes on directly to the main file \textit{main}
while pretending to compile the child \textit{dest}.
This form behaves as if \textit{dest}
issues |\childdocof{|\textit{main}|}| right away,
and no further \textsf{childdoc} directives will be processed.

%%%%%%%%%%%%%%%%%%%%%%%%%%%%%%%%%%%%%%%%
\DescribeMacro{\...prefix}
In the alternative form |\childdocforwardprefix|,
%
\begin{center}
\begin{tabular}{l}
|\input{childdoc.def}|\\
|\childdocforwardprefix[|\textit{main}|]{|\textit{prefix}|}{|\textit{dest}|}|
\end{tabular}
\end{center}
%
the destination file is determined by a pattern
depending on the current file:
To make this work, the current file must be called
`{\textit{prefix}\hspace{0.2em}\textit{suffix}}'
with \textit{prefix} matching precisely the argument.
Processing is then passed on to the file
`{\textit{dest}\hspace{0.2em}\textit{suffix}}'.
Surely, the same effect is achieved by
directly specifying the
argument `{\textit{dest}\hspace{0.2em}\textit{suffix}}'
in the first form.
However, that requires to set up a different file
for each child. With the alternative form of the command
all these files can have exactly the same content
which simplifies setting them up and maintaining them.

For example, the following file |draft.tex|
with a compilation flag |\version| as described in \secref{sec:flags}
compiles the main document as a draft:
%
\begin{center}
\begin{tabular}{l}
|\def\version{draft}|\\
|\input{childdoc.def}|\\
|\childdocforward{|\textit{main}|}|
\end{tabular}
\end{center}
%
Likewise, the following files |final|\textit{nn}|.tex|
compile the final version of the child document
|child|\textit{nn}|.tex|:
%
\begin{center}
\begin{tabular}{l}
|\def\version{final}|\\
|\input{childdoc.def}|\\
|\childdocforwardprefix{final}{child}|
\end{tabular}
\end{center}
%

Note that when several versions of a main file and/or of each child file
are to be generated, it may be convenient to set up a |Makefile| or
shell script to automatise the process.

%%%%%%%%%%%%%%%%%%%%%%%%%%%%%%%%%%%%%%%%%%%%%%%%%%%%%%%%%%%%%%%%%%%%%%%%%%%%%%%%
\subsection{Command Line Processing}
\label{sec:commandline}

The effect of redirection files can also be achieved by invoking
the \LaTeX{} compiler with a more elaborate command line.
Most conveniently this should be done as part
of a shell script or a |Makefile|.

When using \textsf{childdoc} in the main file, the following
command lines effectively perform a redirection
(note that depending on the shell being used,
backslashes may have to be doubled: `|\|' $\to$ `|\\|'):
%
\begin{center}
|... -jobname "|\textit{target}|" |\\|"|[\textit{flags}]%
|\input{childdoc.def}\childdocforward[|\textit{main}|]{|\textit{dest}|}"|
\end{center}
%
Here \textit{target} is the name of the output file,
\textit{main} is the name of the main file
and \textit{dest} is the name of the main or child file to be processed
(all filenames without extensions).
The optional argument \textit{main} can be omitted
if \textit{main} matches \textit{dest}.
Optionally, compilation \textit{flags} can be defined via |\def| commands.
This command line makes the \TeX{} engine believe
it is compiling the file \textit{target}
whose content is specified as the latter parameter.
The provided code then forwards the processing to
\textit{main} or \textit{dest} as described in \secref{sec:forward}.

%%%%%%%%%%%%%%%%%%%%%%%%%%%%%%%%%%%%%%%%%%%%%%%%%%%%%%%%%%%%%%%%%%%%%%%%%%%%%%%%
\subsection{Include by Input}
\label{sec:input}

Including child documents by |\include| has some restrictions by design.
Most notably, the content of a child document always occupies
its own set of pages; pages cannot be shared between child documents.
Usually, this behaviour makes perfect sense
because each child document contain an essential part of the document.
However, in some situations it may be desirable to compose
a document from a collection of parts
without having mandatory page breaks between then.
For this case, the package
provides a mechanism to include parts
by |\input| which can also be processed individually.
However, by construction this mechanism
requires manual handling of the content to be output.

%%%%%%%%%%%%%%%%%%%%%%%%%%%%%%%%%%%%%%%%
\DescribeMacro{\ifchilddocmanual}
The main file should be prepared as usual, see \secref{sec:include}.
However, the document body must make a distinction
between processing of an individual part and of the main document, e.g.:
%
\begin{center}
\begin{tabular}{l}
|\ifchilddocmanual|\\
|\input{\childdocname}|\\
|\||else|\\
\textit{document body with }|\input{|\textit{part}|}|\\
|\||fi|
\end{tabular}
\end{center}
%
The conditional |\ifchilddocmanual| is true whenever
a part to be included by |\input| is being compiled,
and the name of the part is stored in |\childdocname|.

%%%%%%%%%%%%%%%%%%%%%%%%%%%%%%%%%%%%%%%%
\DescribeMacro{\childdocby}
Each part to be included by |\input| should start with:
%
\begin{center}
\begin{tabular}{l}
|\input{childdoc.def}|\\
|\childdocby{|\textit{main}|}|\\
\end{tabular}
\end{center}
%
The directive |\childdocby| is similar to |\childdocof|
described in \secref{sec:include},
but the subsequent selection of content must be done manually.
To that end, both |\ifchilddoc| and |\ifchilddocmanual|
will be true upon processing of a part,
and the name of the part is stored in |\childdocname|.
Note that |\jobname| will be set to the filename of the current part
so that each part receives an individual |.aux| file
that does not interfere with the |.aux| file(s) of the main document.
This behaviour can be altered by the alternative form
|\childdocby[*]{|\textit{main}|}| (with a non-empty optional argument)
which uses the |.aux| file of the main document
by setting |\jobname| to \textit{main}.

%%%%%%%%%%%%%%%%%%%%%%%%%%%%%%%%%%%%%%%%%%%%%%%%%%%%%%%%%%%%%%%%%%%%%%%%%%%%%%%%
\subsection{Driver Development}
\label{sec:driver}

The \textsf{childdoc} mechanism can also be use for the development
of definition files such as \LaTeX{} styles or classes.
This case differs from the above setup with multiple parts
included by |\include| in that no |\includeonly| should be invoked.
This can be achieved by starting the include file
(before |\ProvidesPackage|) with:
%
\begin{center}
\begin{tabular}{l}
|\input{childdoc.def}|\\
|\childdocforward{|\textit{main}|}|\\
\end{tabular}
\end{center}
%
or alternatively with:
%
\begin{center}
\begin{tabular}{l}
|\input{childdoc.def}|\\
|\childdocby{|\textit{main}|}|\\
\end{tabular}
\end{center}
%
Both forms have slightly different effects as described above.
The main file is prepared as usual, see \secref{sec:include}.

%%%%%%%%%%%%%%%%%%%%%%%%%%%%%%%%%%%%%%%%%%%%%%%%%%%%%%%%%%%%%%%%%%%%%%%%%%%%%%%%
\subsection{Legacy Detection}
\label{sec:detection}

The directive |\childdocmain| in the main file can detect
whether the complete document or merely a child is to be compiled
even without using the directive |\childdocof|.
This method is deprecated because it is less robust
and there is no compelling reason to use it;
it is merely provided for backward compatibility
and it may be removed in future versions.

If the detection mechanism is to be used,
it is mandatory to correctly specify
the filename of the main file as the argument of |\childdocmain|:
%
\begin{center}
\begin{tabular}{l}
|\input{childdoc.def}|\\
|\childdocmain{|\textit{main}|}|\\
\end{tabular}
\end{center}
%
If |\jobname| does not match the argument \textit{main} of |\childdocmain|,
it is assumed that |\jobname| points to the child file to be compiled.
When using |\childdocmain| with the main file specified as argument,
it suffices to start a child file
with just |\input{|\textit{main}|}|
without loading of the package and using |\childdocof|.
If instead all processing is done
with the appropriate \textsf{childdoc} directives,
the argument of \textit{main} of |\childdocmain| can be empty.

An alternative version of the command line processing described
in \secref{sec:commandline} using the detection mechanism reads:
%
\begin{center}
|... -jobname "|\textit{target}|" "|[\textit{flags}]%
[|\def\jobname{|\textit{dest}|}|]|\input{|\textit{main}|}"|
\end{center}

%%%%%%%%%%%%%%%%%%%%%%%%%%%%%%%%%%%%%%%%%%%%%%%%%%%%%%%%%%%%%%%%%%%%%%%%%%%%%%%%
\subsection{Manual Code}
\label{sec:manual}

In case one cannot be certain whether the definitions file |childdoc.def|
is installed on the target \TeX{} distribution
and one prefers not to ship it,
it is conceivable to paste a few relevant commands into the sources.

To that end, drop all statements |\input{childdoc.def}|
and perform the replacements as outlined below.
Instead of |\childdocmain{|\textit{main}|}| add the following code
to the top of the main file:
%
\begin{center}
\begin{tabular}{l}
|\||ifdefined\childdocname\endinput\||fi\newif\ifchilddoc|\\
|\edef\childdocname{\scantokens\expandafter{\jobname\noexpand}}|\\
|\def\childdocmain{|\textit{main}|}\||ifx\childdocmain\childdocname\||else|\\
|\childdoctrue\includeonly{\childdocname}\let\jobname\childdocmain\||fi|\\
\end{tabular}
\end{center}
%
Instead of |\childdocof{|\textit{main}|}| just include the main file
at the top of each child file:
%
\begin{center}
|\input{|\textit{main}|}|
\end{center}
%
A simple redirection |\childdocforward{|\textit{dest}|}| is achieved by:
%
\begin{center}
|\def\jobname{|\textit{dest}|}\input{\jobname}|
\end{center}
%
The redirection with prefix
|\childdocforwardprefix[|\textit{prefix}|]{|\textit{dest}|}|
is accomplished by:
%
\begin{center}
\begin{tabular}{l}
|{\edef\jobname{\scantokens\expandafter{\jobname\noexpand}}|\\
|\def\redirectjob |\textit{prefix}|#1~~~{\gdef\jobname{|\textit{dest}|#1}}|\\
|\expandafter\redirectjob\jobname~~~}\input{\jobname}|
\end{tabular}
\end{center}

In an alternative approach,
child documents can be compiled by a specific command line
without additional code or specific definitions:
%
\begin{center}
|... -jobname "|\textit{target}|" "|[\textit{flags}]%
|\includeonly{|\textit{dest}|}\input{|\textit{main}|}"|
\end{center}
%

%%%%%%%%%%%%%%%%%%%%%%%%%%%%%%%%%%%%%%%%%%%%%%%%%%%%%%%%%%%%%%%%%%%%%%%%%%%%%%%%
%%%%%%%%%%%%%%%%%%%%%%%%%%%%%%%%%%%%%%%%%%%%%%%%%%%%%%%%%%%%%%%%%%%%%%%%%%%%%%%%
\section{Information}

%%%%%%%%%%%%%%%%%%%%%%%%%%%%%%%%%%%%%%%%%%%%%%%%%%%%%%%%%%%%%%%%%%%%%%%%%%%%%%%%
\subsection{Copyright}

Copyright \copyright{} 2017--2018 Niklas Beisert

This work may be distributed and/or modified under the
conditions of the \LaTeX{} Project Public License, either version 1.3
of this license or (at your option) any later version.
The latest version of this license is in
  \url{http://www.latex-project.org/lppl.txt}
and version 1.3 or later is part of all distributions of \LaTeX{}
version 2005/12/01 or later.

This work has the LPPL maintenance status `maintained'.

The Current Maintainer of this work is Niklas Beisert.

This work consists of the files |README.txt|, |childdoc.ins| and |childdoc.dtx|
as well as the derived files |childdoc.def|, |cdocsamp.tex|
with |cdocsch1.tex|, |cdocsch2.tex|, |cdocspt3.tex|, |cdocspt4.tex|,
|cdocsdrf.tex|, |cdocsfn1.tex|, |cdocsfn2.tex|
as well as |childdoc.pdf|.

%%%%%%%%%%%%%%%%%%%%%%%%%%%%%%%%%%%%%%%%%%%%%%%%%%%%%%%%%%%%%%%%%%%%%%%%%%%%%%%%
\subsection{Files and Installation}

The package consists of the files:
%
\begin{center}
\begin{tabular}{ll}
    |README.txt|   & readme file \\
    |childdoc.ins| & installation file \\
    |childdoc.dtx| & source file \\
    |childdoc.def| & definition file \\
    |cdocsamp.tex| & sample main file \\
    |cdocsch1.tex| & sample include file \\
    |cdocsch2.tex| & sample include file \\
    |cdocspt3.tex| & sample part file \\
    |cdocspt4.tex| & sample part file \\
    |cdocsdrf.tex| & sample redirection file \\
    |cdocsfn1.tex| & sample redirection file \\
    |cdocsfn2.tex| & sample redirection file \\
    |childdoc.pdf| & manual
\end{tabular}
\end{center}
%
The distribution consists of the files
|README.txt|, |childdoc.ins| and |childdoc.dtx|.
%
\begin{itemize}
\item
Run (pdf)\LaTeX{} on |childdoc.dtx|
to compile the manual |childdoc.pdf| (this file).
\item
Run \LaTeX{} on |childdoc.ins| to create the definitions file |childdoc.def|
and the sample |cdocsamp.tex| with include files
|cdocsch1.tex|, |cdocsch2.tex|, |cdocspt3.tex|, |cdocspt4.tex|,
|cdocsdrf.tex|, |cdocsfn1.tex|, |cdocsfn2.tex|.
Then copy the file |childdoc.def| to an appropriate directory of your \LaTeX{}
distribution, e.g.\ \textit{texmf-root}|/tex/latex/childdoc|.
\end{itemize}

%%%%%%%%%%%%%%%%%%%%%%%%%%%%%%%%%%%%%%%%%%%%%%%%%%%%%%%%%%%%%%%%%%%%%%%%%%%%%%%%
\subsection{Related CTAN Packages}

There are several other packages which offer a similar functionality:
%
\begin{itemize}
\item
The packages
\href{http://ctan.org/pkg/docmute}{\textsf{docmute}},
\href{http://ctan.org/pkg/includex}{\textsf{includex}} and
\href{http://ctan.org/pkg/standalone}{\textsf{standalone}}
provide commands to include only the document body of
a child file thus allowing both files to be compiled individually.
\item
The packages \href{http://ctan.org/pkg/subdocs}{\textsf{subdocs}}
and \href{http://ctan.org/pkg/subfiles}{\textsf{subfiles}}
provide structures in which the main and child documents can be
encapsulated and allowing them to be compiled individually.
The inclusion mechanism is different from the conventional |\include|.
\item
The package \href{http://ctan.org/pkg/combine}{\textsf{combine}}
is an elaborate solution to combine several documents into one.
\end{itemize}
%
See also the CTAN topic \href{http://ctan.org/topic/subdocs}{\textsf{subdocs}}
for further related packages.
The present package differs from the above solutions in that
a document structure constructed with the conventional |\include| mechanism
just needs two extra commands at the top of every file
such that all constituent files can be compiled individually.

%%%%%%%%%%%%%%%%%%%%%%%%%%%%%%%%%%%%%%%%%%%%%%%%%%%%%%%%%%%%%%%%%%%%%%%%%%%%%%%%
%\subsection{Feature Suggestions}
%
%The following is a list of features which may be useful for future
%versions of this package:
%%
%\begin{itemize}
%\item
%\ldots
%\end{itemize}

%%%%%%%%%%%%%%%%%%%%%%%%%%%%%%%%%%%%%%%%%%%%%%%%%%%%%%%%%%%%%%%%%%%%%%%%%%%%%%%%
\subsection{Revision History}

%%%%%%%%%%%%%%%%%%%%%%%%%%%%%%%%%%%%%%%%
\paragraph{v2.0:} 2018/12/30

\begin{itemize}
\item
immediate forward processing
\item
added |\childdocby| mechanism
\item
manual restructured
\end{itemize}

%%%%%%%%%%%%%%%%%%%%%%%%%%%%%%%%%%%%%%%%
\paragraph{v1.6:} 2018/01/17

\begin{itemize}
\item
application for development of include files
\item
corrections to manual
\end{itemize}

%%%%%%%%%%%%%%%%%%%%%%%%%%%%%%%%%%%%%%%%
\paragraph{v1.5:} 2017/05/21

\begin{itemize}
\item
more complete structuring introduced
\item
|\childdocof| introduced
\item
|\childdoc| renamed to |\childdocmain|
\item
|\childredirect| renamed to |\childdocforward| and |\childdocforwardprefix|
and functionality expanded
\end{itemize}

%%%%%%%%%%%%%%%%%%%%%%%%%%%%%%%%%%%%%%%%
\paragraph{v1.0:} 2017/04/27

\begin{itemize}
\item
manual and install package
\item
first version published on CTAN
\end{itemize}

%%%%%%%%%%%%%%%%%%%%%%%%%%%%%%%%%%%%%%%%
\paragraph{v0.6:} 2017/04/26

\begin{itemize}
\item
redirection mechanism added
\end{itemize}

%%%%%%%%%%%%%%%%%%%%%%%%%%%%%%%%%%%%%%%%
\paragraph{v0.5:} 2017/04/26

\begin{itemize}
\item
functionality in definition file
\end{itemize}


%%%%%%%%%%%%%%%%%%%%%%%%%%%%%%%%%%%%%%%%%%%%%%%%%%%%%%%%%%%%%%%%%%%%%%%%%%%%%%%%
%%%%%%%%%%%%%%%%%%%%%%%%%%%%%%%%%%%%%%%%%%%%%%%%%%%%%%%%%%%%%%%%%%%%%%%%%%%%%%%%
%%%%%%%%%%%%%%%%%%%%%%%%%%%%%%%%%%%%%%%%%%%%%%%%%%%%%%%%%%%%%%%%%%%%%%%%%%%%%%%%
\appendix

\settowidth\MacroIndent{\rmfamily\scriptsize 000\ }

 \DocInput{childdoc.dtx}

\end{document}
%</driver>
% \fi
%
% %%%%%%%%%%%%%%%%%%%%%%%%%%%%%%%%%%%%%%%%%%%%%%%%%%%%%%%%%%%%%%%%%%%%%%%%%%%%%%
% %%%%%%%%%%%%%%%%%%%%%%%%%%%%%%%%%%%%%%%%%%%%%%%%%%%%%%%%%%%%%%%%%%%%%%%%%%%%%%
% \section{Sample}
%\iffalse
%<*samplemain>
%\fi
%
% The following presents a sample document
% with two chapters, two parts, a title page,
% a compile flag as well as three forwarding files to set the flag.
% It consists of eight |.tex| files:
% \begin{center}
% \begin{tabular}{ll}
% |cdocsamp.tex|&main file\\
% |cdocsch1.tex|&include file for chapter 1\\
% |cdocsch2.tex|&include file for chapter 2\\
% |cdocspt3.tex|&include file for part 3\\
% |cdocspt4.tex|&include file for part 4\\
% |cdocsdrf.tex|&forwarding file for main file in draft mode\\
% |cdocsfi1.tex|&forwarding file for final version of chapter 1\\
% |cdocsfi2.tex|&forwarding file for final version of chapter 2\\
% \end{tabular}
% \end{center}
% Each of the eight files can be compiled directly by the \LaTeX{} compiler.
%
% %%%%%%%%%%%%%%%%%%%%%%%%%%%%%%%%%%%%%%
% \paragraph{Main File.}
%
% The main file is called |cdocsamp.tex|.
%
% Load the \textsf{childdoc} definitions and
% declare the filename for the main document:
%    \begin{macrocode}
\input{childdoc.def}
\childdocmain{}
%    \end{macrocode}

% Optional override for |\version| flag:
%    \begin{macrocode}
%%\ifchilddoc\else\providecommand{\version}{draft}\fi
%    \end{macrocode}

% Define the default values for the |\version| flag
% (|final| for the main file and |draft| for childs):
%    \begin{macrocode}
\ifchilddoc
\providecommand{\version}{draft}
\else
\providecommand{\version}{final}
\fi
%    \end{macrocode}

% Load the standard document class:
%    \begin{macrocode}
\documentclass[12pt]{article}
%    \end{macrocode}

% Start the document body:
%    \begin{macrocode}
\begin{document}
%    \end{macrocode}

% Declare a title page.
% Print title, part of document being processed and version flag:
%    \begin{macrocode}
\addtocounter{page}{-1}
\begin{center}
{\LARGE\bfseries{}childdoc example\par}
\vspace{1cm}
\ifchilddoc
\ifchilddocmanual part\else chapter\fi:
`\childdocname' of `\childdocjob'\par
\else
main document: `\childdocjob'\par
\fi
version: \version\par
\end{center}
\newpage
%    \end{macrocode}

% Manually include selected file,
% otherwise process as usual:
%    \begin{macrocode}
\ifchilddocmanual
\section*{part `\childdocname'}
\input{\childdocname}
\else
%    \end{macrocode}

% Include the two chapters:
%    \begin{macrocode}
\include{cdocsch1}
\include{cdocsch2}
%    \end{macrocode}

% Include the two parts unless only chapters should be displayed:
%    \begin{macrocode}
\ifchilddoc\else
\section{part three}
\input{cdocspt3}
\section{part four}
\input{cdocspt4}
\fi
%    \end{macrocode}

% Process as usual until here:
%    \begin{macrocode}
\fi
%    \end{macrocode}

% End of document body:
%    \begin{macrocode}
\end{document}
%    \end{macrocode}
%\iffalse
%</samplemain>
%\fi
%
% %%%%%%%%%%%%%%%%%%%%%%%%%%%%%%%%%%%%%%
% \paragraph{Chapter Include Files.}
%
% The include files are called |cdocsch1.tex| and |cdocsch2.tex|.
%
%\iffalse
%<*samplechap1|samplechap2>
%\fi

% Optional override for |\version| flag:
%    \begin{macrocode}
%%\providecommand{\version}{final}
%    \end{macrocode}

% Include the main document:
%    \begin{macrocode}
\input{childdoc.def}
\childdocof{cdocsamp}
%    \end{macrocode}

%\iffalse
%</samplechap1|samplechap2>
%\fi
%
%\iffalse
%<*samplechap1>
%\fi
% Some text for chapter 1:
%    \begin{macrocode}
\section{one}
some text in chapter one
%    \end{macrocode}

%\iffalse
%</samplechap1>
%\fi
% Some text for chapter 2:
%\iffalse
%<*samplechap2>
%\fi
%    \begin{macrocode}
\section{two}
more text in chapter two
%    \end{macrocode}

%\iffalse
%</samplechap2>
%\fi
%
% %%%%%%%%%%%%%%%%%%%%%%%%%%%%%%%%%%%%%%
% \paragraph{Part Include Files.}
%
% The include files are called |cdocspt3.tex| and |cdocspt4.tex|.
%
%\iffalse
%<*samplepart3|samplepart4>
%\fi

% Optional override for |\version| flag:
%    \begin{macrocode}
%%\providecommand{\version}{final}
%    \end{macrocode}

% Include the main document:
%    \begin{macrocode}
\input{childdoc.def}
\childdocby{cdocsamp}
%    \end{macrocode}

%\iffalse
%</samplepart3|samplepart4>
%\fi
%
%\iffalse
%<*samplepart3>
%\fi
% Some text for part 3:
%    \begin{macrocode}
some text in part three
%    \end{macrocode}

%\iffalse
%</samplepart3>
%\fi
% Some text for part 4:
%\iffalse
%<*samplepart4>
%\fi
%    \begin{macrocode}
more text in part four
%    \end{macrocode}

%\iffalse
%</samplepart4>
%\fi
%
% %%%%%%%%%%%%%%%%%%%%%%%%%%%%%%%%%%%%%%
% \paragraph{Forwarding for a Complete Draft.}
%
% The following forwarding file |cdocsdrf.tex|
% compiles the main document in draft mode:
%\iffalse
%<*sampledraft>
%\fi
%    \begin{macrocode}
\def\version{draft}
\input{childdoc.def}
\childdocforward{cdocsamp}
%    \end{macrocode}

%\iffalse
%</sampledraft>
%\fi
%
% %%%%%%%%%%%%%%%%%%%%%%%%%%%%%%%%%%%%%%
% \paragraph{Forwarding for Final Version of the Chapters.}
%
% The following forwarding files |cdocsfn1.tex| and |cdocsfn2.tex|
% (with identical content)
% compile the final versions of the child documents
% |cdocsch1.tex| and |cdocsch2.tex|, respectively:
%\iffalse
%<*samplefinal>
%\fi
%    \begin{macrocode}
\def\version{final}
\input{childdoc.def}
\childdocforwardprefix[cdocsamp]{cdocsfn}{cdocsch}
%    \end{macrocode}

%\iffalse
%</samplefinal>
%\fi
%
% %%%%%%%%%%%%%%%%%%%%%%%%%%%%%%%%%%%%%%
% \paragraph{Command Line Processing.}
%
% The following three command lines generate the output files
% |cdocscld|, |cdocscl1| and |cdocscl2|
% which should be identical to
% |cdocsdrf|, |cdocsch1| and |cdocsfn2|, respectively:
% \begin{center}
% \begin{tabular}{l}
% |latex -jobname cdocscld \|\\
% |  "\def\version{draft}\input{childdoc.def}\childdocforward{cdocsamp}"|\\
% |latex -jobname cdocscl1 \|\\
% |  "\input{childdoc.def}\childdocforward[cdocsamp]{cdocsch1}"|\\
% |latex -jobname cdocscl2 \|\\
% |  "\def\version{final}\input{childdoc.def}\childdocforward{cdocsch2}"|
% \end{tabular}
% \end{center}
% Note that the trailing backslash on each first line
% merely continues the input to the second line
% (for convenient cut ant paste).
% Furthermore, the command |latex| can be replaced by any
% of its alternative versions such as |pdflatex|.
%
% %%%%%%%%%%%%%%%%%%%%%%%%%%%%%%%%%%%%%%%%%%%%%%%%%%%%%%%%%%%%%%%%%%%%%%%%%%%%%%
% %%%%%%%%%%%%%%%%%%%%%%%%%%%%%%%%%%%%%%%%%%%%%%%%%%%%%%%%%%%%%%%%%%%%%%%%%%%%%%
% \section{Implementation}
%\iffalse
%<*package>
%\fi
%
% This section describes the definitions file |childdoc.def|.

% The definitions cannot be loaded using |\usepackage| or |\RequirePackage|
% which has a mechanism to prevent loading a style file more than once.
% When loading the definitions by means of |\input|
% multiple instances have to be prevented manually:
%\iffalse
%This code needs to be before the `\ProvidesFile' directive
%which is defined at the beginning of this file.
%Therefore it is also placed there and commented out here.
%</package>
%<*discard>
%\fi
%    \begin{macrocode}
\ifdefined\childdocmain\endinput\fi
%    \end{macrocode}
%\iffalse
%</discard>
%<*package>
%\fi
%
% \macro{\ifchilddoc}
% \macro{\ifchilddocmanual}
% The conditional |\ifchilddoc| tells whether a
% child (true) or main (false) document is being compiled.
% The conditional |\ifchilddocmanual| tells whether
% the |\includeonly| mechanism is used (false) or
% the selection of child files must be performed manually (true).
% The definitions initialise to false:
%    \begin{macrocode}
\newif\ifchilddoc
\newif\ifchilddocmanual
%    \end{macrocode}

% \macro{\childdocname}
% \macro{\childdocjob}
% The macro |\childdocname| stores the name of the main document
% to be compiled. The macro |\childdocjob| stores the name of
% the document on which the \LaTeX{} compiler was originally invoked.
% The content of |\jobname| cannot be compared
% to filenames specified in the source due to different catcodes.
% The following code rescans |\jobname|, stores the result
% in |\childdocname| and saves a copy in |\childdocjob|:
%    \begin{macrocode}
\edef\childdocname{\scantokens\expandafter{\jobname\noexpand}}
\let\childdocjob\childdocname
%    \end{macrocode}

% \macro{\childdocdisable}
% The macro |\childdocdisable| prevents the main file
% from being processed more than once.
% At this stage, the main document command |\childdocmain|
% is assumed to be called once again where it should do nothing.
% Any subsequent call to it should prevent
% a secondary processing of the main document
% It overwrites the forwarding commands
% |\childdocof| and |\childdocforward|
% with empty macros to prevent further inclusions of the main document:
%    \begin{macrocode}
\newcommand{\childdocdisable}
{
  \renewcommand{\childdocmain}[1]{\renewcommand{\childdocmain}[1]{\endinput}}
  \renewcommand{\childdocof}[1]{}
  \renewcommand{\childdocby}[2][]{}
  \renewcommand{\childdocforward}[2][]{}
  \renewcommand{\childdocdisable}{}
}
%    \end{macrocode}

% \macro{\childdocmain}
% The macro |\childdocmain| is to be called at the top of the main file
% with nothing or the main filename (without extension) as argument.
% First, it breaks loops.
% If the argument is not empty and does not match |\childdocname|
% (which is set by the first inclusion of |childdoc.def|),
% |\ifchilddoc| is set to true, |\includeonly| is applied to the child file
% and |\jobname| is set to the main file
% (for proper handling of |.aux| files):
%    \begin{macrocode}
\newcommand{\childdocmain}[1]
{
  \childdocdisable\childdocmain{}
  \if?#1?\else
    \begingroup
      \def\childdoctmp{#1}
      \ifx\childdoctmp\childdocname
        \def\childdoctmp{}
      \else
        \def\childdoctmp
        {
          \childdoctrue
          \includeonly{\childdocname}
          \def\childdocjob{#1}
          \def\jobname{#1}
        }
      \fi
      \expandafter
    \endgroup
    \childdoctmp
  \fi
}
%    \end{macrocode}

% \macro{\childdocof}
% The command |\childdocof| redirects
% compilation to the main file |#1|.
%    \begin{macrocode}
\newcommand{\childdocof}[1]
{
  \childdocdisable
  \childdoctrue
  \includeonly{\childdocname}
  \def\jobname{#1}
  \def\childdocjob{#1}
  \input{#1}
}
%    \end{macrocode}

% \macro{\childdocby}
% The command |\childdocby| ....
%    \begin{macrocode}
\newcommand{\childdocby}[2][]
{
  \childdocdisable
  \childdoctrue
  \childdocmanualtrue
  \if?#1?\else
    \def\jobname{#2}
  \fi
  \def\childdocjob{#2}
  \input{#2}
  \endinput
}
%    \end{macrocode}

% \macro{\childdocforward}
% The command |\childdocforward| redirects
% compilation to the main file or
% (if the optional argument is given) a child file.
% Parameters are set as if the main file
% or a child file starting with |\childdocof| was compiled.
% Then compilation is handed over to the main file:
%    \begin{macrocode}
\newcommand{\childdocforward}[2][]
{
  \begingroup
    \if?#1?
      \def\childdoctmp
      {
        \def\childdocname{#2}
        \def\childdocjob{#2}
        \def\jobname{#2}
        \input{#2}
        \endinput
      }
    \else
      \def\childdoctmp
      {
        \childdocdisable
        \def\childdocname{#2}
        \childdoctrue
        \includeonly{#2}
        \def\childdocjob{#1}
        \def\jobname{#1}
        \input{#1}
        \endinput
      }
    \fi
    \expandafter
  \endgroup
  \childdoctmp
}
%    \end{macrocode}

% \macro{\childdocforwardprefix}
% The command |\childdocforwardprefix| redirects
% compilation to the main or a child file by means of a pattern.
% The prefix |#1| in the current filename is replaced by |#2|
% and the suffix of the current filename is kept
% (it is assumed that the filename does not contain the substring `|~~~|'
% which is used as a delimiter).
% Compilation is handed over to the new file by |\childdocforward|:
%    \begin{macrocode}
\newcommand{\childdocforwardprefix}[3][]
{
  \begingroup
    \def\childdocextract #2##1~~~{\def\childdoctmp{\childdocforward[#1]{#3##1}}}
    \expandafter\childdocextract\childdocname~~~
    \expandafter
  \endgroup
  \childdoctmp
}
%    \end{macrocode}

% \macro{\childdoc}
% The deprecated macro |\childdoc| is a legacy version of |\childdocmain|:
%    \begin{macrocode}
\newcommand{\childdoc}{\childdocmain}
%    \end{macrocode}

% \macro{\childdocredirect}
% The deprecated macro |\childdocredirect| is a legacy version
% of |\childdocforward| and |\childdocforwardprefix|:
%    \begin{macrocode}
\newcommand{\childdocredirect}[2][]
{
  \begingroup
    \if?#1?
      \def\childdoctmp{\childdocforward{#2}}
    \else
      \def\childdoctmp{\childdocforwardprefix{#1}{#2}}
    \fi
    \expandafter
  \endgroup
  \childdoctmp
}
%    \end{macrocode}

%\iffalse
%</package>
%\fi
%
\endinput
|\\
|\childdocforwardprefix{final}{child}|
\end{tabular}
\end{center}
%

Note that when several versions of a main file and/or of each child file
are to be generated, it may be convenient to set up a |Makefile| or
shell script to automatise the process.

%%%%%%%%%%%%%%%%%%%%%%%%%%%%%%%%%%%%%%%%%%%%%%%%%%%%%%%%%%%%%%%%%%%%%%%%%%%%%%%%
\subsection{Command Line Processing}
\label{sec:commandline}

The effect of redirection files can also be achieved by invoking
the \LaTeX{} compiler with a more elaborate command line.
Most conveniently this should be done as part
of a shell script or a |Makefile|.

When using \textsf{childdoc} in the main file, the following
command lines effectively perform a redirection
(note that depending on the shell being used,
backslashes may have to be doubled: `|\|' $\to$ `|\\|'):
%
\begin{center}
|... -jobname "|\textit{target}|" |\\|"|[\textit{flags}]%
|% \iffalse
%
% childdoc.dtx Copyright (C) 2017-2018 Niklas Beisert
%
% This work may be distributed and/or modified under the
% conditions of the LaTeX Project Public License, either version 1.3
% of this license or (at your option) any later version.
% The latest version of this license is in
%   http://www.latex-project.org/lppl.txt
% and version 1.3 or later is part of all distributions of LaTeX
% version 2005/12/01 or later.
%
% This work has the LPPL maintenance status `maintained'.
%
% The Current Maintainer of this work is Niklas Beisert.
%
% This work consists of the files childdoc.dtx and childdoc.ins
% and the derived files childdoc.def and cdocsamp.tex with
% cdocsch1.tex, cdocsch2.tex, cdocsdrf.tex, cdocsfn1.tex, cdocsfn2.tex.
%
%<package>\ifdefined\childdocmain\endinput\fi
%<package>\ProvidesFile{childdoc.def}[2018/12/30 v2.0 child document driver]
%<samplemain>\ProvidesFile{cdocsamp.tex}[2018/12/30 v2.0 sample for childdoc]
%<*driver>
%\ProvidesFile{childdoc.drv}[2018/12/30 v2.0 childdoc reference manual file]
\PassOptionsToClass{10pt,a4paper}{article}
\documentclass{ltxdoc}

\usepackage[margin=35mm]{geometry}
\usepackage{hyperref}
\usepackage{hyperxmp}
\usepackage[usenames]{color}

\hypersetup{colorlinks=true}
\hypersetup{pdfstartview=FitH}
\hypersetup{pdfpagemode=UseNone}
\hypersetup{pdfsource={}}
\hypersetup{pdflang={en-UK}}
\hypersetup{pdfcopyright={Copyright 2017-2018 Niklas Beisert.
  This work may be distributed and/or modified under the
  conditions of the LaTeX Project Public License, either version 1.3
  of this license or (at your option) any later version.}}
\hypersetup{pdflicenseurl={http://www.latex-project.org/lppl.txt}}
\hypersetup{pdfcontactaddress={ETH Zurich, ITP, HIT K,
  Wolfgang-Pauli-Strasse 27}}
\hypersetup{pdfcontactpostcode={8093}}
\hypersetup{pdfcontactcity={Zurich}}
\hypersetup{pdfcontactcountry={Switzerland}}
\hypersetup{pdfcontactemail={nbeisert@itp.phys.ethz.ch}}
\hypersetup{pdfcontacturl={http://people.phys.ethz.ch/\xmptilde nbeisert/}}

\newcommand{\secref}[1]{\hyperref[#1]{section \ref*{#1}}}

\parskip1ex
\parindent0pt
\let\olditemize\itemize
\def\itemize{\olditemize\parskip0pt}

\begin{document}

\title{The \textsf{childdoc} Package}
\hypersetup{pdftitle={The childdoc Package}}
\author{Niklas Beisert\\[2ex]
  Institut f\"ur Theoretische Physik\\
  Eidgen\"ossische Technische Hochschule Z\"urich\\
  Wolfgang-Pauli-Strasse 27, 8093 Z\"urich, Switzerland\\[1ex]
  \href{mailto:nbeisert@itp.phys.ethz.ch}
  {\texttt{nbeisert@itp.phys.ethz.ch}}}
\hypersetup{pdfauthor={Niklas Beisert}}
\hypersetup{pdfsubject={Manual for the LaTeX2e Package childdoc}}
\date{30 December 2018, \textsf{v2.0}}
\maketitle

\begin{abstract}\noindent
\textsf{childdoc} is a \LaTeXe{} package
that enables the direct compilation
of document sections included by |\include|
to individual files.
\end{abstract}

\begingroup
\parskip0ex
\tableofcontents
\endgroup

%%%%%%%%%%%%%%%%%%%%%%%%%%%%%%%%%%%%%%%%%%%%%%%%%%%%%%%%%%%%%%%%%%%%%%%%%%%%%%%%
%%%%%%%%%%%%%%%%%%%%%%%%%%%%%%%%%%%%%%%%%%%%%%%%%%%%%%%%%%%%%%%%%%%%%%%%%%%%%%%%
\section{Introduction}

\LaTeX{} provides a mechanism to structure a large document (such as a book)
into a main file and several child files (containing the chapters)
using the |\include| command.
This mechanism is beneficial for documents
which span hundreds of pages in order to
make the source file(s) more manageable.
Moreover, compilation can be restricted to
selected child files by means of the |\includeonly| command.
The latter feature can be used to reduce the compilation time while editing
(this was significantly more useful in the earlier days of \LaTeX{})
or to generate a smaller document which is easier to navigate.
Another application of |\includeonly| is to generate
documents consisting of selected parts of the complete document.

However, there are a few drawbacks of the plain |\include| mechanism:
\begin{itemize}
\item
The child files cannot be compiled on their own,
they can only be compiled via the main file.
A naive editing environment
(such as a text editor with an option
to have the current file processed by \LaTeX)
may require one to switch to the main file before compiling;
attempting to compile the child file produces errors.
\item
The main file must be modified (each time)
to adjust the |\includeonly| command
to the present needs. This easily leaves the main file in a messy state.
\item
The generated document will always carry the filename
of the main document. This is inconvenient if
several child files are to be compiled and
to be kept for distribution.
\end{itemize}

The present package provides a simple interface
to make child files individually compilable by \LaTeX{}.
Compiling a child file then has the same effect as compiling
the main file with an |\includeonly| command
to select the appropriate child.
Moreover the generated document will carry the name of the child
rather than the main file.
This resolves all three above issues.

This feature is meant to make the editing of books,
thesis documents and lecture notes somewhat more convenient.
However, the package can also be used efficiently for
composing a series of documents (such as exercise sheets)
which are typically distributed individually.
It then assists the author in generating the individual documents
(potentially in different versions)
as well as a document containing the collected series.
Another application is in developing style files
or other kinds of included material
where compilation of the style file could redirect
to a sample or test file.

%%%%%%%%%%%%%%%%%%%%%%%%%%%%%%%%%%%%%%%%%%%%%%%%%%%%%%%%%%%%%%%%%%%%%%%%%%%%%%%%
%%%%%%%%%%%%%%%%%%%%%%%%%%%%%%%%%%%%%%%%%%%%%%%%%%%%%%%%%%%%%%%%%%%%%%%%%%%%%%%%
\section{Usage}

First of all, the package \textsf{childdoc} is \emph{not} a standard
\LaTeXe{} |.sty| style file! Therefore it needs to be invoked in
a non-standard way.

%%%%%%%%%%%%%%%%%%%%%%%%%%%%%%%%%%%%%%%%%%%%%%%%%%%%%%%%%%%%%%%%%%%%%%%%%%%%%%%%
\subsection{Included Files}
\label{sec:include}

%%%%%%%%%%%%%%%%%%%%%%%%%%%%%%%%%%%%%%%%
\DescribeMacro{\childdocmain}
To use the package, add the commands
\begin{center}
\begin{tabular}{l}
|\input{childdoc.def}|\\
|\childdocmain{}|\\
\end{tabular}
\end{center}
at the very top of the main \LaTeX{} file,
in particular \emph{before} the |\documentclass| statement!
The argument of |\childdocmain| should be left empty
(but it must be present).

%%%%%%%%%%%%%%%%%%%%%%%%%%%%%%%%%%%%%%%%
\DescribeMacro{\childdocof}
Furthermore, add the commands
\begin{center}
\begin{tabular}{l}
|\input{childdoc.def}|\\
|\childdocof{|\textit{main}|}|\\
\end{tabular}
\end{center}
at the top of every child file \textit{child}
which is included by |\include{|\textit{child}|}|
from within the main file
(or at least for those files to be compiled individually).
The argument \textit{main} must be the filename of the main file.

There are a couple of
considerations in setting up the main and child documents:

%%%%%%%%%%%%%%%%%%%%%%%%%%%%%%%%%%%%%%%%
\paragraph{Restrictions.}

Please note the following restrictions:
\begin{itemize}
\item
|\childdocmain| must be called with one argument \textit{main}
to ensure compatibility with earlier version of the package.
It must either be empty (|\childdocmain{}|)
or precisely match the filename of the main file in which it is specified.
See \secref{sec:detection} for further information.
\item
The filename \textit{main} must be specified without the |.tex| extension.
\item
The filename \textit{main} is case sensitive
(even in case-insensitive file systems)
due to internal string comparison.
\item
The argument \textit{main} should be fully expanded, it cannot be a macro.
\item
Subdirectories and special characters should be avoided in filenames.
\item
The command |\childdocmain{|\textit{main}|}| must be followed by a whitespace.
It should not be followed immediately by another command
or by a comment mark `|%|'.
This is because the \TeX{} parser reads the token immediately following
the argument of |\childdocmain| and puts it
at the beginning of every child section;
however, a white\-space is ignored.
\end{itemize}

%%%%%%%%%%%%%%%%%%%%%%%%%%%%%%%%%%%%%%%%
\paragraph{Content of Main File.}

It is advisable to place all content in the child files included by |\include|.
Any output contained in the main file will appear in all child documents
unless suppressed manually;
it cannot be suppressed automatically by the |\includeonly| directive
and thus should normally be avoided.
A method to include some content in the main file
by means of conditional processing is described in \secref{sec:conditional}.

%%%%%%%%%%%%%%%%%%%%%%%%%%%%%%%%%%%%%%%%
\paragraph{Page Numbering.}

When only a part of the document is compiled,
the appropriate numbering of pages
(as well as other status parameters)
is determined from the |.aux| files.
The latter contain information from previous passes.
However this information needs to propagate through
all intermediate child documents.
Therefore the page numbering in child documents may well
be inconsistent until the complete document is compiled at least once.

A useful (if unconventional) way to always ensure a consistent
page numbering is to restart the numbering in each child document
and denote the pages by `\textit{child}|.|\textit{page}'
where \textit{child} represents the chapter/section number of the child file.
This can be achieved by the command
|\numberwithin{page}{|\textit{child}|}|
of the \textsf{amsmath} package
where \textit{child} can be |chapter| or |section|
depending on the chosen structuring.
Alternatively, one can modify the macro |\thepage| appropriately
and reset the counter |page| at the start of each child file.

%%%%%%%%%%%%%%%%%%%%%%%%%%%%%%%%%%%%%%%%%%%%%%%%%%%%%%%%%%%%%%%%%%%%%%%%%%%%%%%%
\subsection{Conditional Processing}
\label{sec:conditional}

The package provides a mechanism to compile different versions
of a document. To customise the versions further some conditional processing
can come in handy to distinguish which version is being compiled.
The package provides two macros to describe the compilation context:

%%%%%%%%%%%%%%%%%%%%%%%%%%%%%%%%%%%%%%%%
\DescribeMacro{\ifchilddoc}
The conditional |\ifchilddoc| distinguishes between the compilation of
child documents and the main document:
%
\begin{center}
|\ifchilddoc |\textit{child-code}| |[|\||else |\textit{main-code}]| \||fi|
\end{center}

%%%%%%%%%%%%%%%%%%%%%%%%%%%%%%%%%%%%%%%%
\DescribeMacro{\childdocname}
\DescribeMacro{\childdocjob}
The macro |\childdocname| contains the filename (without extension)
of the main or child file being processed.
Note that |\childdocjob| will always contain the name of the main file.

%%%%%%%%%%%%%%%%%%%%%%%%%%%%%%%%%%%%%%%%
\paragraph{Title Page.}

Conditional processing can be used to include a title or banner page
in the main document when proper precautions are taken.
Importantly, the code in the main file should ensure that the page counter
(as well as other status parameters which are stored in the |.aux| files)
takes the same value after the conditional processing.
Otherwise the page numbers may take divergent values
depending on which part is compiled.

For example, a title page could be declared by:
%
\begin{center}
\begin{tabular}{l}
|\ifchilddoc\||else|\\
|\addtocounter{page}{-1}|\\
\textit{code for title page}\\
|\newpage|\\
|\||fi|
\end{tabular}
\end{center}
%
A banner page for the child documents can be generated by:
%
\begin{center}
\begin{tabular}{l}
|\ifchilddoc|\\
|\addtocounter{page}{-1}|\\
\textit{code for banner page}\\
|\newpage|\\
|\||fi|
\end{tabular}
\end{center}
%
Here one could write a message such as:
\begin{center}
|This is the part \childdocname{} of \childdocjob{}.|
\end{center}

%%%%%%%%%%%%%%%%%%%%%%%%%%%%%%%%%%%%%%%%%%%%%%%%%%%%%%%%%%%%%%%%%%%%%%%%%%%%%%%%
\subsection{Flags}
\label{sec:flags}

The package makes it easy to generate different versions
of the main or child documents.
To this end compilation flags can be defined
and assigned different default values.
They will be particularly useful in conjunction
with the forwarding mechanism described in \secref{sec:forward}.

For example, it may be useful to have a flag |\version|
which can be set to |draft| or |final|.
The document source will contain some conditional code
depending on the value of |\version|.
Suppose further, the flag should default to |final| for the main file
and to |draft| for child files
which is a natural assignment for editing the document.
This is achieved by placing the following code
in the preamble of the main document
(below the |\childdocmain| directive):
%
\begin{center}
\begin{tabular}{l}
|\ifchilddoc|\\
|\providecommand{\version}{draft}|\\
|\||else|\\
|\providecommand{\version}{final}|\\
|\||fi|
\end{tabular}
\end{center}
%
The definition by |\providecommand| makes sure
that previous definitions are not overwritten.
Further statements |\providecommand{\version}{...}|
can thus be added before the above code to override it.

For the main file, one might add a line
(between |\childdocmain| and the above block)
%
\begin{center}
|%\ifchilddoc\||else\providecommand{\version}{draft}\||fi|
\end{center}
%
which can be uncommented to produce a draft version.
Likewise one can add a line to the very top of a child file
(above the |\childdocof{|\textit{main}|}| directive)
%
\begin{center}
|%\providecommand{\version}{final}|
\end{center}
%
which can be uncommented to produce the final version of this child document.

%%%%%%%%%%%%%%%%%%%%%%%%%%%%%%%%%%%%%%%%%%%%%%%%%%%%%%%%%%%%%%%%%%%%%%%%%%%%%%%%
\subsection{Forwarding}
\label{sec:forward}

Different versions of the main or child documents
using compilation flags as described in \secref{sec:flags}
can be (permanently) stored in different files
for convenient compilation, viewing and distribution.
To this end, the package defines a command
to pass on compilation to a different file:

%%%%%%%%%%%%%%%%%%%%%%%%%%%%%%%%%%%%%%%%
\DescribeMacro{\childdocforward}
The command |\childdocforward| redirects processing to
another source file:
%
\begin{center}
\begin{tabular}{l}
|\input{childdoc.def}|\\
|\childdocforward[|\textit{main}|]{|\textit{dest}|}|\\
\end{tabular}
\end{center}
%
The argument \textit{dest} is the destination file
(without extension).
It should be the main file or one of the child files.
Note that further \textsf{childdoc} directives
such as |\childdocof| and |\childdocforward|
in the indicated file will be processed in this form.
The optional argument \textit{main}
passes on directly to the main file \textit{main}
while pretending to compile the child \textit{dest}.
This form behaves as if \textit{dest}
issues |\childdocof{|\textit{main}|}| right away,
and no further \textsf{childdoc} directives will be processed.

%%%%%%%%%%%%%%%%%%%%%%%%%%%%%%%%%%%%%%%%
\DescribeMacro{\...prefix}
In the alternative form |\childdocforwardprefix|,
%
\begin{center}
\begin{tabular}{l}
|\input{childdoc.def}|\\
|\childdocforwardprefix[|\textit{main}|]{|\textit{prefix}|}{|\textit{dest}|}|
\end{tabular}
\end{center}
%
the destination file is determined by a pattern
depending on the current file:
To make this work, the current file must be called
`{\textit{prefix}\hspace{0.2em}\textit{suffix}}'
with \textit{prefix} matching precisely the argument.
Processing is then passed on to the file
`{\textit{dest}\hspace{0.2em}\textit{suffix}}'.
Surely, the same effect is achieved by
directly specifying the
argument `{\textit{dest}\hspace{0.2em}\textit{suffix}}'
in the first form.
However, that requires to set up a different file
for each child. With the alternative form of the command
all these files can have exactly the same content
which simplifies setting them up and maintaining them.

For example, the following file |draft.tex|
with a compilation flag |\version| as described in \secref{sec:flags}
compiles the main document as a draft:
%
\begin{center}
\begin{tabular}{l}
|\def\version{draft}|\\
|\input{childdoc.def}|\\
|\childdocforward{|\textit{main}|}|
\end{tabular}
\end{center}
%
Likewise, the following files |final|\textit{nn}|.tex|
compile the final version of the child document
|child|\textit{nn}|.tex|:
%
\begin{center}
\begin{tabular}{l}
|\def\version{final}|\\
|\input{childdoc.def}|\\
|\childdocforwardprefix{final}{child}|
\end{tabular}
\end{center}
%

Note that when several versions of a main file and/or of each child file
are to be generated, it may be convenient to set up a |Makefile| or
shell script to automatise the process.

%%%%%%%%%%%%%%%%%%%%%%%%%%%%%%%%%%%%%%%%%%%%%%%%%%%%%%%%%%%%%%%%%%%%%%%%%%%%%%%%
\subsection{Command Line Processing}
\label{sec:commandline}

The effect of redirection files can also be achieved by invoking
the \LaTeX{} compiler with a more elaborate command line.
Most conveniently this should be done as part
of a shell script or a |Makefile|.

When using \textsf{childdoc} in the main file, the following
command lines effectively perform a redirection
(note that depending on the shell being used,
backslashes may have to be doubled: `|\|' $\to$ `|\\|'):
%
\begin{center}
|... -jobname "|\textit{target}|" |\\|"|[\textit{flags}]%
|\input{childdoc.def}\childdocforward[|\textit{main}|]{|\textit{dest}|}"|
\end{center}
%
Here \textit{target} is the name of the output file,
\textit{main} is the name of the main file
and \textit{dest} is the name of the main or child file to be processed
(all filenames without extensions).
The optional argument \textit{main} can be omitted
if \textit{main} matches \textit{dest}.
Optionally, compilation \textit{flags} can be defined via |\def| commands.
This command line makes the \TeX{} engine believe
it is compiling the file \textit{target}
whose content is specified as the latter parameter.
The provided code then forwards the processing to
\textit{main} or \textit{dest} as described in \secref{sec:forward}.

%%%%%%%%%%%%%%%%%%%%%%%%%%%%%%%%%%%%%%%%%%%%%%%%%%%%%%%%%%%%%%%%%%%%%%%%%%%%%%%%
\subsection{Include by Input}
\label{sec:input}

Including child documents by |\include| has some restrictions by design.
Most notably, the content of a child document always occupies
its own set of pages; pages cannot be shared between child documents.
Usually, this behaviour makes perfect sense
because each child document contain an essential part of the document.
However, in some situations it may be desirable to compose
a document from a collection of parts
without having mandatory page breaks between then.
For this case, the package
provides a mechanism to include parts
by |\input| which can also be processed individually.
However, by construction this mechanism
requires manual handling of the content to be output.

%%%%%%%%%%%%%%%%%%%%%%%%%%%%%%%%%%%%%%%%
\DescribeMacro{\ifchilddocmanual}
The main file should be prepared as usual, see \secref{sec:include}.
However, the document body must make a distinction
between processing of an individual part and of the main document, e.g.:
%
\begin{center}
\begin{tabular}{l}
|\ifchilddocmanual|\\
|\input{\childdocname}|\\
|\||else|\\
\textit{document body with }|\input{|\textit{part}|}|\\
|\||fi|
\end{tabular}
\end{center}
%
The conditional |\ifchilddocmanual| is true whenever
a part to be included by |\input| is being compiled,
and the name of the part is stored in |\childdocname|.

%%%%%%%%%%%%%%%%%%%%%%%%%%%%%%%%%%%%%%%%
\DescribeMacro{\childdocby}
Each part to be included by |\input| should start with:
%
\begin{center}
\begin{tabular}{l}
|\input{childdoc.def}|\\
|\childdocby{|\textit{main}|}|\\
\end{tabular}
\end{center}
%
The directive |\childdocby| is similar to |\childdocof|
described in \secref{sec:include},
but the subsequent selection of content must be done manually.
To that end, both |\ifchilddoc| and |\ifchilddocmanual|
will be true upon processing of a part,
and the name of the part is stored in |\childdocname|.
Note that |\jobname| will be set to the filename of the current part
so that each part receives an individual |.aux| file
that does not interfere with the |.aux| file(s) of the main document.
This behaviour can be altered by the alternative form
|\childdocby[*]{|\textit{main}|}| (with a non-empty optional argument)
which uses the |.aux| file of the main document
by setting |\jobname| to \textit{main}.

%%%%%%%%%%%%%%%%%%%%%%%%%%%%%%%%%%%%%%%%%%%%%%%%%%%%%%%%%%%%%%%%%%%%%%%%%%%%%%%%
\subsection{Driver Development}
\label{sec:driver}

The \textsf{childdoc} mechanism can also be use for the development
of definition files such as \LaTeX{} styles or classes.
This case differs from the above setup with multiple parts
included by |\include| in that no |\includeonly| should be invoked.
This can be achieved by starting the include file
(before |\ProvidesPackage|) with:
%
\begin{center}
\begin{tabular}{l}
|\input{childdoc.def}|\\
|\childdocforward{|\textit{main}|}|\\
\end{tabular}
\end{center}
%
or alternatively with:
%
\begin{center}
\begin{tabular}{l}
|\input{childdoc.def}|\\
|\childdocby{|\textit{main}|}|\\
\end{tabular}
\end{center}
%
Both forms have slightly different effects as described above.
The main file is prepared as usual, see \secref{sec:include}.

%%%%%%%%%%%%%%%%%%%%%%%%%%%%%%%%%%%%%%%%%%%%%%%%%%%%%%%%%%%%%%%%%%%%%%%%%%%%%%%%
\subsection{Legacy Detection}
\label{sec:detection}

The directive |\childdocmain| in the main file can detect
whether the complete document or merely a child is to be compiled
even without using the directive |\childdocof|.
This method is deprecated because it is less robust
and there is no compelling reason to use it;
it is merely provided for backward compatibility
and it may be removed in future versions.

If the detection mechanism is to be used,
it is mandatory to correctly specify
the filename of the main file as the argument of |\childdocmain|:
%
\begin{center}
\begin{tabular}{l}
|\input{childdoc.def}|\\
|\childdocmain{|\textit{main}|}|\\
\end{tabular}
\end{center}
%
If |\jobname| does not match the argument \textit{main} of |\childdocmain|,
it is assumed that |\jobname| points to the child file to be compiled.
When using |\childdocmain| with the main file specified as argument,
it suffices to start a child file
with just |\input{|\textit{main}|}|
without loading of the package and using |\childdocof|.
If instead all processing is done
with the appropriate \textsf{childdoc} directives,
the argument of \textit{main} of |\childdocmain| can be empty.

An alternative version of the command line processing described
in \secref{sec:commandline} using the detection mechanism reads:
%
\begin{center}
|... -jobname "|\textit{target}|" "|[\textit{flags}]%
[|\def\jobname{|\textit{dest}|}|]|\input{|\textit{main}|}"|
\end{center}

%%%%%%%%%%%%%%%%%%%%%%%%%%%%%%%%%%%%%%%%%%%%%%%%%%%%%%%%%%%%%%%%%%%%%%%%%%%%%%%%
\subsection{Manual Code}
\label{sec:manual}

In case one cannot be certain whether the definitions file |childdoc.def|
is installed on the target \TeX{} distribution
and one prefers not to ship it,
it is conceivable to paste a few relevant commands into the sources.

To that end, drop all statements |\input{childdoc.def}|
and perform the replacements as outlined below.
Instead of |\childdocmain{|\textit{main}|}| add the following code
to the top of the main file:
%
\begin{center}
\begin{tabular}{l}
|\||ifdefined\childdocname\endinput\||fi\newif\ifchilddoc|\\
|\edef\childdocname{\scantokens\expandafter{\jobname\noexpand}}|\\
|\def\childdocmain{|\textit{main}|}\||ifx\childdocmain\childdocname\||else|\\
|\childdoctrue\includeonly{\childdocname}\let\jobname\childdocmain\||fi|\\
\end{tabular}
\end{center}
%
Instead of |\childdocof{|\textit{main}|}| just include the main file
at the top of each child file:
%
\begin{center}
|\input{|\textit{main}|}|
\end{center}
%
A simple redirection |\childdocforward{|\textit{dest}|}| is achieved by:
%
\begin{center}
|\def\jobname{|\textit{dest}|}\input{\jobname}|
\end{center}
%
The redirection with prefix
|\childdocforwardprefix[|\textit{prefix}|]{|\textit{dest}|}|
is accomplished by:
%
\begin{center}
\begin{tabular}{l}
|{\edef\jobname{\scantokens\expandafter{\jobname\noexpand}}|\\
|\def\redirectjob |\textit{prefix}|#1~~~{\gdef\jobname{|\textit{dest}|#1}}|\\
|\expandafter\redirectjob\jobname~~~}\input{\jobname}|
\end{tabular}
\end{center}

In an alternative approach,
child documents can be compiled by a specific command line
without additional code or specific definitions:
%
\begin{center}
|... -jobname "|\textit{target}|" "|[\textit{flags}]%
|\includeonly{|\textit{dest}|}\input{|\textit{main}|}"|
\end{center}
%

%%%%%%%%%%%%%%%%%%%%%%%%%%%%%%%%%%%%%%%%%%%%%%%%%%%%%%%%%%%%%%%%%%%%%%%%%%%%%%%%
%%%%%%%%%%%%%%%%%%%%%%%%%%%%%%%%%%%%%%%%%%%%%%%%%%%%%%%%%%%%%%%%%%%%%%%%%%%%%%%%
\section{Information}

%%%%%%%%%%%%%%%%%%%%%%%%%%%%%%%%%%%%%%%%%%%%%%%%%%%%%%%%%%%%%%%%%%%%%%%%%%%%%%%%
\subsection{Copyright}

Copyright \copyright{} 2017--2018 Niklas Beisert

This work may be distributed and/or modified under the
conditions of the \LaTeX{} Project Public License, either version 1.3
of this license or (at your option) any later version.
The latest version of this license is in
  \url{http://www.latex-project.org/lppl.txt}
and version 1.3 or later is part of all distributions of \LaTeX{}
version 2005/12/01 or later.

This work has the LPPL maintenance status `maintained'.

The Current Maintainer of this work is Niklas Beisert.

This work consists of the files |README.txt|, |childdoc.ins| and |childdoc.dtx|
as well as the derived files |childdoc.def|, |cdocsamp.tex|
with |cdocsch1.tex|, |cdocsch2.tex|, |cdocspt3.tex|, |cdocspt4.tex|,
|cdocsdrf.tex|, |cdocsfn1.tex|, |cdocsfn2.tex|
as well as |childdoc.pdf|.

%%%%%%%%%%%%%%%%%%%%%%%%%%%%%%%%%%%%%%%%%%%%%%%%%%%%%%%%%%%%%%%%%%%%%%%%%%%%%%%%
\subsection{Files and Installation}

The package consists of the files:
%
\begin{center}
\begin{tabular}{ll}
    |README.txt|   & readme file \\
    |childdoc.ins| & installation file \\
    |childdoc.dtx| & source file \\
    |childdoc.def| & definition file \\
    |cdocsamp.tex| & sample main file \\
    |cdocsch1.tex| & sample include file \\
    |cdocsch2.tex| & sample include file \\
    |cdocspt3.tex| & sample part file \\
    |cdocspt4.tex| & sample part file \\
    |cdocsdrf.tex| & sample redirection file \\
    |cdocsfn1.tex| & sample redirection file \\
    |cdocsfn2.tex| & sample redirection file \\
    |childdoc.pdf| & manual
\end{tabular}
\end{center}
%
The distribution consists of the files
|README.txt|, |childdoc.ins| and |childdoc.dtx|.
%
\begin{itemize}
\item
Run (pdf)\LaTeX{} on |childdoc.dtx|
to compile the manual |childdoc.pdf| (this file).
\item
Run \LaTeX{} on |childdoc.ins| to create the definitions file |childdoc.def|
and the sample |cdocsamp.tex| with include files
|cdocsch1.tex|, |cdocsch2.tex|, |cdocspt3.tex|, |cdocspt4.tex|,
|cdocsdrf.tex|, |cdocsfn1.tex|, |cdocsfn2.tex|.
Then copy the file |childdoc.def| to an appropriate directory of your \LaTeX{}
distribution, e.g.\ \textit{texmf-root}|/tex/latex/childdoc|.
\end{itemize}

%%%%%%%%%%%%%%%%%%%%%%%%%%%%%%%%%%%%%%%%%%%%%%%%%%%%%%%%%%%%%%%%%%%%%%%%%%%%%%%%
\subsection{Related CTAN Packages}

There are several other packages which offer a similar functionality:
%
\begin{itemize}
\item
The packages
\href{http://ctan.org/pkg/docmute}{\textsf{docmute}},
\href{http://ctan.org/pkg/includex}{\textsf{includex}} and
\href{http://ctan.org/pkg/standalone}{\textsf{standalone}}
provide commands to include only the document body of
a child file thus allowing both files to be compiled individually.
\item
The packages \href{http://ctan.org/pkg/subdocs}{\textsf{subdocs}}
and \href{http://ctan.org/pkg/subfiles}{\textsf{subfiles}}
provide structures in which the main and child documents can be
encapsulated and allowing them to be compiled individually.
The inclusion mechanism is different from the conventional |\include|.
\item
The package \href{http://ctan.org/pkg/combine}{\textsf{combine}}
is an elaborate solution to combine several documents into one.
\end{itemize}
%
See also the CTAN topic \href{http://ctan.org/topic/subdocs}{\textsf{subdocs}}
for further related packages.
The present package differs from the above solutions in that
a document structure constructed with the conventional |\include| mechanism
just needs two extra commands at the top of every file
such that all constituent files can be compiled individually.

%%%%%%%%%%%%%%%%%%%%%%%%%%%%%%%%%%%%%%%%%%%%%%%%%%%%%%%%%%%%%%%%%%%%%%%%%%%%%%%%
%\subsection{Feature Suggestions}
%
%The following is a list of features which may be useful for future
%versions of this package:
%%
%\begin{itemize}
%\item
%\ldots
%\end{itemize}

%%%%%%%%%%%%%%%%%%%%%%%%%%%%%%%%%%%%%%%%%%%%%%%%%%%%%%%%%%%%%%%%%%%%%%%%%%%%%%%%
\subsection{Revision History}

%%%%%%%%%%%%%%%%%%%%%%%%%%%%%%%%%%%%%%%%
\paragraph{v2.0:} 2018/12/30

\begin{itemize}
\item
immediate forward processing
\item
added |\childdocby| mechanism
\item
manual restructured
\end{itemize}

%%%%%%%%%%%%%%%%%%%%%%%%%%%%%%%%%%%%%%%%
\paragraph{v1.6:} 2018/01/17

\begin{itemize}
\item
application for development of include files
\item
corrections to manual
\end{itemize}

%%%%%%%%%%%%%%%%%%%%%%%%%%%%%%%%%%%%%%%%
\paragraph{v1.5:} 2017/05/21

\begin{itemize}
\item
more complete structuring introduced
\item
|\childdocof| introduced
\item
|\childdoc| renamed to |\childdocmain|
\item
|\childredirect| renamed to |\childdocforward| and |\childdocforwardprefix|
and functionality expanded
\end{itemize}

%%%%%%%%%%%%%%%%%%%%%%%%%%%%%%%%%%%%%%%%
\paragraph{v1.0:} 2017/04/27

\begin{itemize}
\item
manual and install package
\item
first version published on CTAN
\end{itemize}

%%%%%%%%%%%%%%%%%%%%%%%%%%%%%%%%%%%%%%%%
\paragraph{v0.6:} 2017/04/26

\begin{itemize}
\item
redirection mechanism added
\end{itemize}

%%%%%%%%%%%%%%%%%%%%%%%%%%%%%%%%%%%%%%%%
\paragraph{v0.5:} 2017/04/26

\begin{itemize}
\item
functionality in definition file
\end{itemize}


%%%%%%%%%%%%%%%%%%%%%%%%%%%%%%%%%%%%%%%%%%%%%%%%%%%%%%%%%%%%%%%%%%%%%%%%%%%%%%%%
%%%%%%%%%%%%%%%%%%%%%%%%%%%%%%%%%%%%%%%%%%%%%%%%%%%%%%%%%%%%%%%%%%%%%%%%%%%%%%%%
%%%%%%%%%%%%%%%%%%%%%%%%%%%%%%%%%%%%%%%%%%%%%%%%%%%%%%%%%%%%%%%%%%%%%%%%%%%%%%%%
\appendix

\settowidth\MacroIndent{\rmfamily\scriptsize 000\ }

 \DocInput{childdoc.dtx}

\end{document}
%</driver>
% \fi
%
% %%%%%%%%%%%%%%%%%%%%%%%%%%%%%%%%%%%%%%%%%%%%%%%%%%%%%%%%%%%%%%%%%%%%%%%%%%%%%%
% %%%%%%%%%%%%%%%%%%%%%%%%%%%%%%%%%%%%%%%%%%%%%%%%%%%%%%%%%%%%%%%%%%%%%%%%%%%%%%
% \section{Sample}
%\iffalse
%<*samplemain>
%\fi
%
% The following presents a sample document
% with two chapters, two parts, a title page,
% a compile flag as well as three forwarding files to set the flag.
% It consists of eight |.tex| files:
% \begin{center}
% \begin{tabular}{ll}
% |cdocsamp.tex|&main file\\
% |cdocsch1.tex|&include file for chapter 1\\
% |cdocsch2.tex|&include file for chapter 2\\
% |cdocspt3.tex|&include file for part 3\\
% |cdocspt4.tex|&include file for part 4\\
% |cdocsdrf.tex|&forwarding file for main file in draft mode\\
% |cdocsfi1.tex|&forwarding file for final version of chapter 1\\
% |cdocsfi2.tex|&forwarding file for final version of chapter 2\\
% \end{tabular}
% \end{center}
% Each of the eight files can be compiled directly by the \LaTeX{} compiler.
%
% %%%%%%%%%%%%%%%%%%%%%%%%%%%%%%%%%%%%%%
% \paragraph{Main File.}
%
% The main file is called |cdocsamp.tex|.
%
% Load the \textsf{childdoc} definitions and
% declare the filename for the main document:
%    \begin{macrocode}
\input{childdoc.def}
\childdocmain{}
%    \end{macrocode}

% Optional override for |\version| flag:
%    \begin{macrocode}
%%\ifchilddoc\else\providecommand{\version}{draft}\fi
%    \end{macrocode}

% Define the default values for the |\version| flag
% (|final| for the main file and |draft| for childs):
%    \begin{macrocode}
\ifchilddoc
\providecommand{\version}{draft}
\else
\providecommand{\version}{final}
\fi
%    \end{macrocode}

% Load the standard document class:
%    \begin{macrocode}
\documentclass[12pt]{article}
%    \end{macrocode}

% Start the document body:
%    \begin{macrocode}
\begin{document}
%    \end{macrocode}

% Declare a title page.
% Print title, part of document being processed and version flag:
%    \begin{macrocode}
\addtocounter{page}{-1}
\begin{center}
{\LARGE\bfseries{}childdoc example\par}
\vspace{1cm}
\ifchilddoc
\ifchilddocmanual part\else chapter\fi:
`\childdocname' of `\childdocjob'\par
\else
main document: `\childdocjob'\par
\fi
version: \version\par
\end{center}
\newpage
%    \end{macrocode}

% Manually include selected file,
% otherwise process as usual:
%    \begin{macrocode}
\ifchilddocmanual
\section*{part `\childdocname'}
\input{\childdocname}
\else
%    \end{macrocode}

% Include the two chapters:
%    \begin{macrocode}
\include{cdocsch1}
\include{cdocsch2}
%    \end{macrocode}

% Include the two parts unless only chapters should be displayed:
%    \begin{macrocode}
\ifchilddoc\else
\section{part three}
\input{cdocspt3}
\section{part four}
\input{cdocspt4}
\fi
%    \end{macrocode}

% Process as usual until here:
%    \begin{macrocode}
\fi
%    \end{macrocode}

% End of document body:
%    \begin{macrocode}
\end{document}
%    \end{macrocode}
%\iffalse
%</samplemain>
%\fi
%
% %%%%%%%%%%%%%%%%%%%%%%%%%%%%%%%%%%%%%%
% \paragraph{Chapter Include Files.}
%
% The include files are called |cdocsch1.tex| and |cdocsch2.tex|.
%
%\iffalse
%<*samplechap1|samplechap2>
%\fi

% Optional override for |\version| flag:
%    \begin{macrocode}
%%\providecommand{\version}{final}
%    \end{macrocode}

% Include the main document:
%    \begin{macrocode}
\input{childdoc.def}
\childdocof{cdocsamp}
%    \end{macrocode}

%\iffalse
%</samplechap1|samplechap2>
%\fi
%
%\iffalse
%<*samplechap1>
%\fi
% Some text for chapter 1:
%    \begin{macrocode}
\section{one}
some text in chapter one
%    \end{macrocode}

%\iffalse
%</samplechap1>
%\fi
% Some text for chapter 2:
%\iffalse
%<*samplechap2>
%\fi
%    \begin{macrocode}
\section{two}
more text in chapter two
%    \end{macrocode}

%\iffalse
%</samplechap2>
%\fi
%
% %%%%%%%%%%%%%%%%%%%%%%%%%%%%%%%%%%%%%%
% \paragraph{Part Include Files.}
%
% The include files are called |cdocspt3.tex| and |cdocspt4.tex|.
%
%\iffalse
%<*samplepart3|samplepart4>
%\fi

% Optional override for |\version| flag:
%    \begin{macrocode}
%%\providecommand{\version}{final}
%    \end{macrocode}

% Include the main document:
%    \begin{macrocode}
\input{childdoc.def}
\childdocby{cdocsamp}
%    \end{macrocode}

%\iffalse
%</samplepart3|samplepart4>
%\fi
%
%\iffalse
%<*samplepart3>
%\fi
% Some text for part 3:
%    \begin{macrocode}
some text in part three
%    \end{macrocode}

%\iffalse
%</samplepart3>
%\fi
% Some text for part 4:
%\iffalse
%<*samplepart4>
%\fi
%    \begin{macrocode}
more text in part four
%    \end{macrocode}

%\iffalse
%</samplepart4>
%\fi
%
% %%%%%%%%%%%%%%%%%%%%%%%%%%%%%%%%%%%%%%
% \paragraph{Forwarding for a Complete Draft.}
%
% The following forwarding file |cdocsdrf.tex|
% compiles the main document in draft mode:
%\iffalse
%<*sampledraft>
%\fi
%    \begin{macrocode}
\def\version{draft}
\input{childdoc.def}
\childdocforward{cdocsamp}
%    \end{macrocode}

%\iffalse
%</sampledraft>
%\fi
%
% %%%%%%%%%%%%%%%%%%%%%%%%%%%%%%%%%%%%%%
% \paragraph{Forwarding for Final Version of the Chapters.}
%
% The following forwarding files |cdocsfn1.tex| and |cdocsfn2.tex|
% (with identical content)
% compile the final versions of the child documents
% |cdocsch1.tex| and |cdocsch2.tex|, respectively:
%\iffalse
%<*samplefinal>
%\fi
%    \begin{macrocode}
\def\version{final}
\input{childdoc.def}
\childdocforwardprefix[cdocsamp]{cdocsfn}{cdocsch}
%    \end{macrocode}

%\iffalse
%</samplefinal>
%\fi
%
% %%%%%%%%%%%%%%%%%%%%%%%%%%%%%%%%%%%%%%
% \paragraph{Command Line Processing.}
%
% The following three command lines generate the output files
% |cdocscld|, |cdocscl1| and |cdocscl2|
% which should be identical to
% |cdocsdrf|, |cdocsch1| and |cdocsfn2|, respectively:
% \begin{center}
% \begin{tabular}{l}
% |latex -jobname cdocscld \|\\
% |  "\def\version{draft}\input{childdoc.def}\childdocforward{cdocsamp}"|\\
% |latex -jobname cdocscl1 \|\\
% |  "\input{childdoc.def}\childdocforward[cdocsamp]{cdocsch1}"|\\
% |latex -jobname cdocscl2 \|\\
% |  "\def\version{final}\input{childdoc.def}\childdocforward{cdocsch2}"|
% \end{tabular}
% \end{center}
% Note that the trailing backslash on each first line
% merely continues the input to the second line
% (for convenient cut ant paste).
% Furthermore, the command |latex| can be replaced by any
% of its alternative versions such as |pdflatex|.
%
% %%%%%%%%%%%%%%%%%%%%%%%%%%%%%%%%%%%%%%%%%%%%%%%%%%%%%%%%%%%%%%%%%%%%%%%%%%%%%%
% %%%%%%%%%%%%%%%%%%%%%%%%%%%%%%%%%%%%%%%%%%%%%%%%%%%%%%%%%%%%%%%%%%%%%%%%%%%%%%
% \section{Implementation}
%\iffalse
%<*package>
%\fi
%
% This section describes the definitions file |childdoc.def|.

% The definitions cannot be loaded using |\usepackage| or |\RequirePackage|
% which has a mechanism to prevent loading a style file more than once.
% When loading the definitions by means of |\input|
% multiple instances have to be prevented manually:
%\iffalse
%This code needs to be before the `\ProvidesFile' directive
%which is defined at the beginning of this file.
%Therefore it is also placed there and commented out here.
%</package>
%<*discard>
%\fi
%    \begin{macrocode}
\ifdefined\childdocmain\endinput\fi
%    \end{macrocode}
%\iffalse
%</discard>
%<*package>
%\fi
%
% \macro{\ifchilddoc}
% \macro{\ifchilddocmanual}
% The conditional |\ifchilddoc| tells whether a
% child (true) or main (false) document is being compiled.
% The conditional |\ifchilddocmanual| tells whether
% the |\includeonly| mechanism is used (false) or
% the selection of child files must be performed manually (true).
% The definitions initialise to false:
%    \begin{macrocode}
\newif\ifchilddoc
\newif\ifchilddocmanual
%    \end{macrocode}

% \macro{\childdocname}
% \macro{\childdocjob}
% The macro |\childdocname| stores the name of the main document
% to be compiled. The macro |\childdocjob| stores the name of
% the document on which the \LaTeX{} compiler was originally invoked.
% The content of |\jobname| cannot be compared
% to filenames specified in the source due to different catcodes.
% The following code rescans |\jobname|, stores the result
% in |\childdocname| and saves a copy in |\childdocjob|:
%    \begin{macrocode}
\edef\childdocname{\scantokens\expandafter{\jobname\noexpand}}
\let\childdocjob\childdocname
%    \end{macrocode}

% \macro{\childdocdisable}
% The macro |\childdocdisable| prevents the main file
% from being processed more than once.
% At this stage, the main document command |\childdocmain|
% is assumed to be called once again where it should do nothing.
% Any subsequent call to it should prevent
% a secondary processing of the main document
% It overwrites the forwarding commands
% |\childdocof| and |\childdocforward|
% with empty macros to prevent further inclusions of the main document:
%    \begin{macrocode}
\newcommand{\childdocdisable}
{
  \renewcommand{\childdocmain}[1]{\renewcommand{\childdocmain}[1]{\endinput}}
  \renewcommand{\childdocof}[1]{}
  \renewcommand{\childdocby}[2][]{}
  \renewcommand{\childdocforward}[2][]{}
  \renewcommand{\childdocdisable}{}
}
%    \end{macrocode}

% \macro{\childdocmain}
% The macro |\childdocmain| is to be called at the top of the main file
% with nothing or the main filename (without extension) as argument.
% First, it breaks loops.
% If the argument is not empty and does not match |\childdocname|
% (which is set by the first inclusion of |childdoc.def|),
% |\ifchilddoc| is set to true, |\includeonly| is applied to the child file
% and |\jobname| is set to the main file
% (for proper handling of |.aux| files):
%    \begin{macrocode}
\newcommand{\childdocmain}[1]
{
  \childdocdisable\childdocmain{}
  \if?#1?\else
    \begingroup
      \def\childdoctmp{#1}
      \ifx\childdoctmp\childdocname
        \def\childdoctmp{}
      \else
        \def\childdoctmp
        {
          \childdoctrue
          \includeonly{\childdocname}
          \def\childdocjob{#1}
          \def\jobname{#1}
        }
      \fi
      \expandafter
    \endgroup
    \childdoctmp
  \fi
}
%    \end{macrocode}

% \macro{\childdocof}
% The command |\childdocof| redirects
% compilation to the main file |#1|.
%    \begin{macrocode}
\newcommand{\childdocof}[1]
{
  \childdocdisable
  \childdoctrue
  \includeonly{\childdocname}
  \def\jobname{#1}
  \def\childdocjob{#1}
  \input{#1}
}
%    \end{macrocode}

% \macro{\childdocby}
% The command |\childdocby| ....
%    \begin{macrocode}
\newcommand{\childdocby}[2][]
{
  \childdocdisable
  \childdoctrue
  \childdocmanualtrue
  \if?#1?\else
    \def\jobname{#2}
  \fi
  \def\childdocjob{#2}
  \input{#2}
  \endinput
}
%    \end{macrocode}

% \macro{\childdocforward}
% The command |\childdocforward| redirects
% compilation to the main file or
% (if the optional argument is given) a child file.
% Parameters are set as if the main file
% or a child file starting with |\childdocof| was compiled.
% Then compilation is handed over to the main file:
%    \begin{macrocode}
\newcommand{\childdocforward}[2][]
{
  \begingroup
    \if?#1?
      \def\childdoctmp
      {
        \def\childdocname{#2}
        \def\childdocjob{#2}
        \def\jobname{#2}
        \input{#2}
        \endinput
      }
    \else
      \def\childdoctmp
      {
        \childdocdisable
        \def\childdocname{#2}
        \childdoctrue
        \includeonly{#2}
        \def\childdocjob{#1}
        \def\jobname{#1}
        \input{#1}
        \endinput
      }
    \fi
    \expandafter
  \endgroup
  \childdoctmp
}
%    \end{macrocode}

% \macro{\childdocforwardprefix}
% The command |\childdocforwardprefix| redirects
% compilation to the main or a child file by means of a pattern.
% The prefix |#1| in the current filename is replaced by |#2|
% and the suffix of the current filename is kept
% (it is assumed that the filename does not contain the substring `|~~~|'
% which is used as a delimiter).
% Compilation is handed over to the new file by |\childdocforward|:
%    \begin{macrocode}
\newcommand{\childdocforwardprefix}[3][]
{
  \begingroup
    \def\childdocextract #2##1~~~{\def\childdoctmp{\childdocforward[#1]{#3##1}}}
    \expandafter\childdocextract\childdocname~~~
    \expandafter
  \endgroup
  \childdoctmp
}
%    \end{macrocode}

% \macro{\childdoc}
% The deprecated macro |\childdoc| is a legacy version of |\childdocmain|:
%    \begin{macrocode}
\newcommand{\childdoc}{\childdocmain}
%    \end{macrocode}

% \macro{\childdocredirect}
% The deprecated macro |\childdocredirect| is a legacy version
% of |\childdocforward| and |\childdocforwardprefix|:
%    \begin{macrocode}
\newcommand{\childdocredirect}[2][]
{
  \begingroup
    \if?#1?
      \def\childdoctmp{\childdocforward{#2}}
    \else
      \def\childdoctmp{\childdocforwardprefix{#1}{#2}}
    \fi
    \expandafter
  \endgroup
  \childdoctmp
}
%    \end{macrocode}

%\iffalse
%</package>
%\fi
%
\endinput
\childdocforward[|\textit{main}|]{|\textit{dest}|}"|
\end{center}
%
Here \textit{target} is the name of the output file,
\textit{main} is the name of the main file
and \textit{dest} is the name of the main or child file to be processed
(all filenames without extensions).
The optional argument \textit{main} can be omitted
if \textit{main} matches \textit{dest}.
Optionally, compilation \textit{flags} can be defined via |\def| commands.
This command line makes the \TeX{} engine believe
it is compiling the file \textit{target}
whose content is specified as the latter parameter.
The provided code then forwards the processing to
\textit{main} or \textit{dest} as described in \secref{sec:forward}.

%%%%%%%%%%%%%%%%%%%%%%%%%%%%%%%%%%%%%%%%%%%%%%%%%%%%%%%%%%%%%%%%%%%%%%%%%%%%%%%%
\subsection{Include by Input}
\label{sec:input}

Including child documents by |\include| has some restrictions by design.
Most notably, the content of a child document always occupies
its own set of pages; pages cannot be shared between child documents.
Usually, this behaviour makes perfect sense
because each child document contain an essential part of the document.
However, in some situations it may be desirable to compose
a document from a collection of parts
without having mandatory page breaks between then.
For this case, the package
provides a mechanism to include parts
by |\input| which can also be processed individually.
However, by construction this mechanism
requires manual handling of the content to be output.

%%%%%%%%%%%%%%%%%%%%%%%%%%%%%%%%%%%%%%%%
\DescribeMacro{\ifchilddocmanual}
The main file should be prepared as usual, see \secref{sec:include}.
However, the document body must make a distinction
between processing of an individual part and of the main document, e.g.:
%
\begin{center}
\begin{tabular}{l}
|\ifchilddocmanual|\\
|\input{\childdocname}|\\
|\||else|\\
\textit{document body with }|\input{|\textit{part}|}|\\
|\||fi|
\end{tabular}
\end{center}
%
The conditional |\ifchilddocmanual| is true whenever
a part to be included by |\input| is being compiled,
and the name of the part is stored in |\childdocname|.

%%%%%%%%%%%%%%%%%%%%%%%%%%%%%%%%%%%%%%%%
\DescribeMacro{\childdocby}
Each part to be included by |\input| should start with:
%
\begin{center}
\begin{tabular}{l}
|% \iffalse
%
% childdoc.dtx Copyright (C) 2017-2018 Niklas Beisert
%
% This work may be distributed and/or modified under the
% conditions of the LaTeX Project Public License, either version 1.3
% of this license or (at your option) any later version.
% The latest version of this license is in
%   http://www.latex-project.org/lppl.txt
% and version 1.3 or later is part of all distributions of LaTeX
% version 2005/12/01 or later.
%
% This work has the LPPL maintenance status `maintained'.
%
% The Current Maintainer of this work is Niklas Beisert.
%
% This work consists of the files childdoc.dtx and childdoc.ins
% and the derived files childdoc.def and cdocsamp.tex with
% cdocsch1.tex, cdocsch2.tex, cdocsdrf.tex, cdocsfn1.tex, cdocsfn2.tex.
%
%<package>\ifdefined\childdocmain\endinput\fi
%<package>\ProvidesFile{childdoc.def}[2018/12/30 v2.0 child document driver]
%<samplemain>\ProvidesFile{cdocsamp.tex}[2018/12/30 v2.0 sample for childdoc]
%<*driver>
%\ProvidesFile{childdoc.drv}[2018/12/30 v2.0 childdoc reference manual file]
\PassOptionsToClass{10pt,a4paper}{article}
\documentclass{ltxdoc}

\usepackage[margin=35mm]{geometry}
\usepackage{hyperref}
\usepackage{hyperxmp}
\usepackage[usenames]{color}

\hypersetup{colorlinks=true}
\hypersetup{pdfstartview=FitH}
\hypersetup{pdfpagemode=UseNone}
\hypersetup{pdfsource={}}
\hypersetup{pdflang={en-UK}}
\hypersetup{pdfcopyright={Copyright 2017-2018 Niklas Beisert.
  This work may be distributed and/or modified under the
  conditions of the LaTeX Project Public License, either version 1.3
  of this license or (at your option) any later version.}}
\hypersetup{pdflicenseurl={http://www.latex-project.org/lppl.txt}}
\hypersetup{pdfcontactaddress={ETH Zurich, ITP, HIT K,
  Wolfgang-Pauli-Strasse 27}}
\hypersetup{pdfcontactpostcode={8093}}
\hypersetup{pdfcontactcity={Zurich}}
\hypersetup{pdfcontactcountry={Switzerland}}
\hypersetup{pdfcontactemail={nbeisert@itp.phys.ethz.ch}}
\hypersetup{pdfcontacturl={http://people.phys.ethz.ch/\xmptilde nbeisert/}}

\newcommand{\secref}[1]{\hyperref[#1]{section \ref*{#1}}}

\parskip1ex
\parindent0pt
\let\olditemize\itemize
\def\itemize{\olditemize\parskip0pt}

\begin{document}

\title{The \textsf{childdoc} Package}
\hypersetup{pdftitle={The childdoc Package}}
\author{Niklas Beisert\\[2ex]
  Institut f\"ur Theoretische Physik\\
  Eidgen\"ossische Technische Hochschule Z\"urich\\
  Wolfgang-Pauli-Strasse 27, 8093 Z\"urich, Switzerland\\[1ex]
  \href{mailto:nbeisert@itp.phys.ethz.ch}
  {\texttt{nbeisert@itp.phys.ethz.ch}}}
\hypersetup{pdfauthor={Niklas Beisert}}
\hypersetup{pdfsubject={Manual for the LaTeX2e Package childdoc}}
\date{30 December 2018, \textsf{v2.0}}
\maketitle

\begin{abstract}\noindent
\textsf{childdoc} is a \LaTeXe{} package
that enables the direct compilation
of document sections included by |\include|
to individual files.
\end{abstract}

\begingroup
\parskip0ex
\tableofcontents
\endgroup

%%%%%%%%%%%%%%%%%%%%%%%%%%%%%%%%%%%%%%%%%%%%%%%%%%%%%%%%%%%%%%%%%%%%%%%%%%%%%%%%
%%%%%%%%%%%%%%%%%%%%%%%%%%%%%%%%%%%%%%%%%%%%%%%%%%%%%%%%%%%%%%%%%%%%%%%%%%%%%%%%
\section{Introduction}

\LaTeX{} provides a mechanism to structure a large document (such as a book)
into a main file and several child files (containing the chapters)
using the |\include| command.
This mechanism is beneficial for documents
which span hundreds of pages in order to
make the source file(s) more manageable.
Moreover, compilation can be restricted to
selected child files by means of the |\includeonly| command.
The latter feature can be used to reduce the compilation time while editing
(this was significantly more useful in the earlier days of \LaTeX{})
or to generate a smaller document which is easier to navigate.
Another application of |\includeonly| is to generate
documents consisting of selected parts of the complete document.

However, there are a few drawbacks of the plain |\include| mechanism:
\begin{itemize}
\item
The child files cannot be compiled on their own,
they can only be compiled via the main file.
A naive editing environment
(such as a text editor with an option
to have the current file processed by \LaTeX)
may require one to switch to the main file before compiling;
attempting to compile the child file produces errors.
\item
The main file must be modified (each time)
to adjust the |\includeonly| command
to the present needs. This easily leaves the main file in a messy state.
\item
The generated document will always carry the filename
of the main document. This is inconvenient if
several child files are to be compiled and
to be kept for distribution.
\end{itemize}

The present package provides a simple interface
to make child files individually compilable by \LaTeX{}.
Compiling a child file then has the same effect as compiling
the main file with an |\includeonly| command
to select the appropriate child.
Moreover the generated document will carry the name of the child
rather than the main file.
This resolves all three above issues.

This feature is meant to make the editing of books,
thesis documents and lecture notes somewhat more convenient.
However, the package can also be used efficiently for
composing a series of documents (such as exercise sheets)
which are typically distributed individually.
It then assists the author in generating the individual documents
(potentially in different versions)
as well as a document containing the collected series.
Another application is in developing style files
or other kinds of included material
where compilation of the style file could redirect
to a sample or test file.

%%%%%%%%%%%%%%%%%%%%%%%%%%%%%%%%%%%%%%%%%%%%%%%%%%%%%%%%%%%%%%%%%%%%%%%%%%%%%%%%
%%%%%%%%%%%%%%%%%%%%%%%%%%%%%%%%%%%%%%%%%%%%%%%%%%%%%%%%%%%%%%%%%%%%%%%%%%%%%%%%
\section{Usage}

First of all, the package \textsf{childdoc} is \emph{not} a standard
\LaTeXe{} |.sty| style file! Therefore it needs to be invoked in
a non-standard way.

%%%%%%%%%%%%%%%%%%%%%%%%%%%%%%%%%%%%%%%%%%%%%%%%%%%%%%%%%%%%%%%%%%%%%%%%%%%%%%%%
\subsection{Included Files}
\label{sec:include}

%%%%%%%%%%%%%%%%%%%%%%%%%%%%%%%%%%%%%%%%
\DescribeMacro{\childdocmain}
To use the package, add the commands
\begin{center}
\begin{tabular}{l}
|\input{childdoc.def}|\\
|\childdocmain{}|\\
\end{tabular}
\end{center}
at the very top of the main \LaTeX{} file,
in particular \emph{before} the |\documentclass| statement!
The argument of |\childdocmain| should be left empty
(but it must be present).

%%%%%%%%%%%%%%%%%%%%%%%%%%%%%%%%%%%%%%%%
\DescribeMacro{\childdocof}
Furthermore, add the commands
\begin{center}
\begin{tabular}{l}
|\input{childdoc.def}|\\
|\childdocof{|\textit{main}|}|\\
\end{tabular}
\end{center}
at the top of every child file \textit{child}
which is included by |\include{|\textit{child}|}|
from within the main file
(or at least for those files to be compiled individually).
The argument \textit{main} must be the filename of the main file.

There are a couple of
considerations in setting up the main and child documents:

%%%%%%%%%%%%%%%%%%%%%%%%%%%%%%%%%%%%%%%%
\paragraph{Restrictions.}

Please note the following restrictions:
\begin{itemize}
\item
|\childdocmain| must be called with one argument \textit{main}
to ensure compatibility with earlier version of the package.
It must either be empty (|\childdocmain{}|)
or precisely match the filename of the main file in which it is specified.
See \secref{sec:detection} for further information.
\item
The filename \textit{main} must be specified without the |.tex| extension.
\item
The filename \textit{main} is case sensitive
(even in case-insensitive file systems)
due to internal string comparison.
\item
The argument \textit{main} should be fully expanded, it cannot be a macro.
\item
Subdirectories and special characters should be avoided in filenames.
\item
The command |\childdocmain{|\textit{main}|}| must be followed by a whitespace.
It should not be followed immediately by another command
or by a comment mark `|%|'.
This is because the \TeX{} parser reads the token immediately following
the argument of |\childdocmain| and puts it
at the beginning of every child section;
however, a white\-space is ignored.
\end{itemize}

%%%%%%%%%%%%%%%%%%%%%%%%%%%%%%%%%%%%%%%%
\paragraph{Content of Main File.}

It is advisable to place all content in the child files included by |\include|.
Any output contained in the main file will appear in all child documents
unless suppressed manually;
it cannot be suppressed automatically by the |\includeonly| directive
and thus should normally be avoided.
A method to include some content in the main file
by means of conditional processing is described in \secref{sec:conditional}.

%%%%%%%%%%%%%%%%%%%%%%%%%%%%%%%%%%%%%%%%
\paragraph{Page Numbering.}

When only a part of the document is compiled,
the appropriate numbering of pages
(as well as other status parameters)
is determined from the |.aux| files.
The latter contain information from previous passes.
However this information needs to propagate through
all intermediate child documents.
Therefore the page numbering in child documents may well
be inconsistent until the complete document is compiled at least once.

A useful (if unconventional) way to always ensure a consistent
page numbering is to restart the numbering in each child document
and denote the pages by `\textit{child}|.|\textit{page}'
where \textit{child} represents the chapter/section number of the child file.
This can be achieved by the command
|\numberwithin{page}{|\textit{child}|}|
of the \textsf{amsmath} package
where \textit{child} can be |chapter| or |section|
depending on the chosen structuring.
Alternatively, one can modify the macro |\thepage| appropriately
and reset the counter |page| at the start of each child file.

%%%%%%%%%%%%%%%%%%%%%%%%%%%%%%%%%%%%%%%%%%%%%%%%%%%%%%%%%%%%%%%%%%%%%%%%%%%%%%%%
\subsection{Conditional Processing}
\label{sec:conditional}

The package provides a mechanism to compile different versions
of a document. To customise the versions further some conditional processing
can come in handy to distinguish which version is being compiled.
The package provides two macros to describe the compilation context:

%%%%%%%%%%%%%%%%%%%%%%%%%%%%%%%%%%%%%%%%
\DescribeMacro{\ifchilddoc}
The conditional |\ifchilddoc| distinguishes between the compilation of
child documents and the main document:
%
\begin{center}
|\ifchilddoc |\textit{child-code}| |[|\||else |\textit{main-code}]| \||fi|
\end{center}

%%%%%%%%%%%%%%%%%%%%%%%%%%%%%%%%%%%%%%%%
\DescribeMacro{\childdocname}
\DescribeMacro{\childdocjob}
The macro |\childdocname| contains the filename (without extension)
of the main or child file being processed.
Note that |\childdocjob| will always contain the name of the main file.

%%%%%%%%%%%%%%%%%%%%%%%%%%%%%%%%%%%%%%%%
\paragraph{Title Page.}

Conditional processing can be used to include a title or banner page
in the main document when proper precautions are taken.
Importantly, the code in the main file should ensure that the page counter
(as well as other status parameters which are stored in the |.aux| files)
takes the same value after the conditional processing.
Otherwise the page numbers may take divergent values
depending on which part is compiled.

For example, a title page could be declared by:
%
\begin{center}
\begin{tabular}{l}
|\ifchilddoc\||else|\\
|\addtocounter{page}{-1}|\\
\textit{code for title page}\\
|\newpage|\\
|\||fi|
\end{tabular}
\end{center}
%
A banner page for the child documents can be generated by:
%
\begin{center}
\begin{tabular}{l}
|\ifchilddoc|\\
|\addtocounter{page}{-1}|\\
\textit{code for banner page}\\
|\newpage|\\
|\||fi|
\end{tabular}
\end{center}
%
Here one could write a message such as:
\begin{center}
|This is the part \childdocname{} of \childdocjob{}.|
\end{center}

%%%%%%%%%%%%%%%%%%%%%%%%%%%%%%%%%%%%%%%%%%%%%%%%%%%%%%%%%%%%%%%%%%%%%%%%%%%%%%%%
\subsection{Flags}
\label{sec:flags}

The package makes it easy to generate different versions
of the main or child documents.
To this end compilation flags can be defined
and assigned different default values.
They will be particularly useful in conjunction
with the forwarding mechanism described in \secref{sec:forward}.

For example, it may be useful to have a flag |\version|
which can be set to |draft| or |final|.
The document source will contain some conditional code
depending on the value of |\version|.
Suppose further, the flag should default to |final| for the main file
and to |draft| for child files
which is a natural assignment for editing the document.
This is achieved by placing the following code
in the preamble of the main document
(below the |\childdocmain| directive):
%
\begin{center}
\begin{tabular}{l}
|\ifchilddoc|\\
|\providecommand{\version}{draft}|\\
|\||else|\\
|\providecommand{\version}{final}|\\
|\||fi|
\end{tabular}
\end{center}
%
The definition by |\providecommand| makes sure
that previous definitions are not overwritten.
Further statements |\providecommand{\version}{...}|
can thus be added before the above code to override it.

For the main file, one might add a line
(between |\childdocmain| and the above block)
%
\begin{center}
|%\ifchilddoc\||else\providecommand{\version}{draft}\||fi|
\end{center}
%
which can be uncommented to produce a draft version.
Likewise one can add a line to the very top of a child file
(above the |\childdocof{|\textit{main}|}| directive)
%
\begin{center}
|%\providecommand{\version}{final}|
\end{center}
%
which can be uncommented to produce the final version of this child document.

%%%%%%%%%%%%%%%%%%%%%%%%%%%%%%%%%%%%%%%%%%%%%%%%%%%%%%%%%%%%%%%%%%%%%%%%%%%%%%%%
\subsection{Forwarding}
\label{sec:forward}

Different versions of the main or child documents
using compilation flags as described in \secref{sec:flags}
can be (permanently) stored in different files
for convenient compilation, viewing and distribution.
To this end, the package defines a command
to pass on compilation to a different file:

%%%%%%%%%%%%%%%%%%%%%%%%%%%%%%%%%%%%%%%%
\DescribeMacro{\childdocforward}
The command |\childdocforward| redirects processing to
another source file:
%
\begin{center}
\begin{tabular}{l}
|\input{childdoc.def}|\\
|\childdocforward[|\textit{main}|]{|\textit{dest}|}|\\
\end{tabular}
\end{center}
%
The argument \textit{dest} is the destination file
(without extension).
It should be the main file or one of the child files.
Note that further \textsf{childdoc} directives
such as |\childdocof| and |\childdocforward|
in the indicated file will be processed in this form.
The optional argument \textit{main}
passes on directly to the main file \textit{main}
while pretending to compile the child \textit{dest}.
This form behaves as if \textit{dest}
issues |\childdocof{|\textit{main}|}| right away,
and no further \textsf{childdoc} directives will be processed.

%%%%%%%%%%%%%%%%%%%%%%%%%%%%%%%%%%%%%%%%
\DescribeMacro{\...prefix}
In the alternative form |\childdocforwardprefix|,
%
\begin{center}
\begin{tabular}{l}
|\input{childdoc.def}|\\
|\childdocforwardprefix[|\textit{main}|]{|\textit{prefix}|}{|\textit{dest}|}|
\end{tabular}
\end{center}
%
the destination file is determined by a pattern
depending on the current file:
To make this work, the current file must be called
`{\textit{prefix}\hspace{0.2em}\textit{suffix}}'
with \textit{prefix} matching precisely the argument.
Processing is then passed on to the file
`{\textit{dest}\hspace{0.2em}\textit{suffix}}'.
Surely, the same effect is achieved by
directly specifying the
argument `{\textit{dest}\hspace{0.2em}\textit{suffix}}'
in the first form.
However, that requires to set up a different file
for each child. With the alternative form of the command
all these files can have exactly the same content
which simplifies setting them up and maintaining them.

For example, the following file |draft.tex|
with a compilation flag |\version| as described in \secref{sec:flags}
compiles the main document as a draft:
%
\begin{center}
\begin{tabular}{l}
|\def\version{draft}|\\
|\input{childdoc.def}|\\
|\childdocforward{|\textit{main}|}|
\end{tabular}
\end{center}
%
Likewise, the following files |final|\textit{nn}|.tex|
compile the final version of the child document
|child|\textit{nn}|.tex|:
%
\begin{center}
\begin{tabular}{l}
|\def\version{final}|\\
|\input{childdoc.def}|\\
|\childdocforwardprefix{final}{child}|
\end{tabular}
\end{center}
%

Note that when several versions of a main file and/or of each child file
are to be generated, it may be convenient to set up a |Makefile| or
shell script to automatise the process.

%%%%%%%%%%%%%%%%%%%%%%%%%%%%%%%%%%%%%%%%%%%%%%%%%%%%%%%%%%%%%%%%%%%%%%%%%%%%%%%%
\subsection{Command Line Processing}
\label{sec:commandline}

The effect of redirection files can also be achieved by invoking
the \LaTeX{} compiler with a more elaborate command line.
Most conveniently this should be done as part
of a shell script or a |Makefile|.

When using \textsf{childdoc} in the main file, the following
command lines effectively perform a redirection
(note that depending on the shell being used,
backslashes may have to be doubled: `|\|' $\to$ `|\\|'):
%
\begin{center}
|... -jobname "|\textit{target}|" |\\|"|[\textit{flags}]%
|\input{childdoc.def}\childdocforward[|\textit{main}|]{|\textit{dest}|}"|
\end{center}
%
Here \textit{target} is the name of the output file,
\textit{main} is the name of the main file
and \textit{dest} is the name of the main or child file to be processed
(all filenames without extensions).
The optional argument \textit{main} can be omitted
if \textit{main} matches \textit{dest}.
Optionally, compilation \textit{flags} can be defined via |\def| commands.
This command line makes the \TeX{} engine believe
it is compiling the file \textit{target}
whose content is specified as the latter parameter.
The provided code then forwards the processing to
\textit{main} or \textit{dest} as described in \secref{sec:forward}.

%%%%%%%%%%%%%%%%%%%%%%%%%%%%%%%%%%%%%%%%%%%%%%%%%%%%%%%%%%%%%%%%%%%%%%%%%%%%%%%%
\subsection{Include by Input}
\label{sec:input}

Including child documents by |\include| has some restrictions by design.
Most notably, the content of a child document always occupies
its own set of pages; pages cannot be shared between child documents.
Usually, this behaviour makes perfect sense
because each child document contain an essential part of the document.
However, in some situations it may be desirable to compose
a document from a collection of parts
without having mandatory page breaks between then.
For this case, the package
provides a mechanism to include parts
by |\input| which can also be processed individually.
However, by construction this mechanism
requires manual handling of the content to be output.

%%%%%%%%%%%%%%%%%%%%%%%%%%%%%%%%%%%%%%%%
\DescribeMacro{\ifchilddocmanual}
The main file should be prepared as usual, see \secref{sec:include}.
However, the document body must make a distinction
between processing of an individual part and of the main document, e.g.:
%
\begin{center}
\begin{tabular}{l}
|\ifchilddocmanual|\\
|\input{\childdocname}|\\
|\||else|\\
\textit{document body with }|\input{|\textit{part}|}|\\
|\||fi|
\end{tabular}
\end{center}
%
The conditional |\ifchilddocmanual| is true whenever
a part to be included by |\input| is being compiled,
and the name of the part is stored in |\childdocname|.

%%%%%%%%%%%%%%%%%%%%%%%%%%%%%%%%%%%%%%%%
\DescribeMacro{\childdocby}
Each part to be included by |\input| should start with:
%
\begin{center}
\begin{tabular}{l}
|\input{childdoc.def}|\\
|\childdocby{|\textit{main}|}|\\
\end{tabular}
\end{center}
%
The directive |\childdocby| is similar to |\childdocof|
described in \secref{sec:include},
but the subsequent selection of content must be done manually.
To that end, both |\ifchilddoc| and |\ifchilddocmanual|
will be true upon processing of a part,
and the name of the part is stored in |\childdocname|.
Note that |\jobname| will be set to the filename of the current part
so that each part receives an individual |.aux| file
that does not interfere with the |.aux| file(s) of the main document.
This behaviour can be altered by the alternative form
|\childdocby[*]{|\textit{main}|}| (with a non-empty optional argument)
which uses the |.aux| file of the main document
by setting |\jobname| to \textit{main}.

%%%%%%%%%%%%%%%%%%%%%%%%%%%%%%%%%%%%%%%%%%%%%%%%%%%%%%%%%%%%%%%%%%%%%%%%%%%%%%%%
\subsection{Driver Development}
\label{sec:driver}

The \textsf{childdoc} mechanism can also be use for the development
of definition files such as \LaTeX{} styles or classes.
This case differs from the above setup with multiple parts
included by |\include| in that no |\includeonly| should be invoked.
This can be achieved by starting the include file
(before |\ProvidesPackage|) with:
%
\begin{center}
\begin{tabular}{l}
|\input{childdoc.def}|\\
|\childdocforward{|\textit{main}|}|\\
\end{tabular}
\end{center}
%
or alternatively with:
%
\begin{center}
\begin{tabular}{l}
|\input{childdoc.def}|\\
|\childdocby{|\textit{main}|}|\\
\end{tabular}
\end{center}
%
Both forms have slightly different effects as described above.
The main file is prepared as usual, see \secref{sec:include}.

%%%%%%%%%%%%%%%%%%%%%%%%%%%%%%%%%%%%%%%%%%%%%%%%%%%%%%%%%%%%%%%%%%%%%%%%%%%%%%%%
\subsection{Legacy Detection}
\label{sec:detection}

The directive |\childdocmain| in the main file can detect
whether the complete document or merely a child is to be compiled
even without using the directive |\childdocof|.
This method is deprecated because it is less robust
and there is no compelling reason to use it;
it is merely provided for backward compatibility
and it may be removed in future versions.

If the detection mechanism is to be used,
it is mandatory to correctly specify
the filename of the main file as the argument of |\childdocmain|:
%
\begin{center}
\begin{tabular}{l}
|\input{childdoc.def}|\\
|\childdocmain{|\textit{main}|}|\\
\end{tabular}
\end{center}
%
If |\jobname| does not match the argument \textit{main} of |\childdocmain|,
it is assumed that |\jobname| points to the child file to be compiled.
When using |\childdocmain| with the main file specified as argument,
it suffices to start a child file
with just |\input{|\textit{main}|}|
without loading of the package and using |\childdocof|.
If instead all processing is done
with the appropriate \textsf{childdoc} directives,
the argument of \textit{main} of |\childdocmain| can be empty.

An alternative version of the command line processing described
in \secref{sec:commandline} using the detection mechanism reads:
%
\begin{center}
|... -jobname "|\textit{target}|" "|[\textit{flags}]%
[|\def\jobname{|\textit{dest}|}|]|\input{|\textit{main}|}"|
\end{center}

%%%%%%%%%%%%%%%%%%%%%%%%%%%%%%%%%%%%%%%%%%%%%%%%%%%%%%%%%%%%%%%%%%%%%%%%%%%%%%%%
\subsection{Manual Code}
\label{sec:manual}

In case one cannot be certain whether the definitions file |childdoc.def|
is installed on the target \TeX{} distribution
and one prefers not to ship it,
it is conceivable to paste a few relevant commands into the sources.

To that end, drop all statements |\input{childdoc.def}|
and perform the replacements as outlined below.
Instead of |\childdocmain{|\textit{main}|}| add the following code
to the top of the main file:
%
\begin{center}
\begin{tabular}{l}
|\||ifdefined\childdocname\endinput\||fi\newif\ifchilddoc|\\
|\edef\childdocname{\scantokens\expandafter{\jobname\noexpand}}|\\
|\def\childdocmain{|\textit{main}|}\||ifx\childdocmain\childdocname\||else|\\
|\childdoctrue\includeonly{\childdocname}\let\jobname\childdocmain\||fi|\\
\end{tabular}
\end{center}
%
Instead of |\childdocof{|\textit{main}|}| just include the main file
at the top of each child file:
%
\begin{center}
|\input{|\textit{main}|}|
\end{center}
%
A simple redirection |\childdocforward{|\textit{dest}|}| is achieved by:
%
\begin{center}
|\def\jobname{|\textit{dest}|}\input{\jobname}|
\end{center}
%
The redirection with prefix
|\childdocforwardprefix[|\textit{prefix}|]{|\textit{dest}|}|
is accomplished by:
%
\begin{center}
\begin{tabular}{l}
|{\edef\jobname{\scantokens\expandafter{\jobname\noexpand}}|\\
|\def\redirectjob |\textit{prefix}|#1~~~{\gdef\jobname{|\textit{dest}|#1}}|\\
|\expandafter\redirectjob\jobname~~~}\input{\jobname}|
\end{tabular}
\end{center}

In an alternative approach,
child documents can be compiled by a specific command line
without additional code or specific definitions:
%
\begin{center}
|... -jobname "|\textit{target}|" "|[\textit{flags}]%
|\includeonly{|\textit{dest}|}\input{|\textit{main}|}"|
\end{center}
%

%%%%%%%%%%%%%%%%%%%%%%%%%%%%%%%%%%%%%%%%%%%%%%%%%%%%%%%%%%%%%%%%%%%%%%%%%%%%%%%%
%%%%%%%%%%%%%%%%%%%%%%%%%%%%%%%%%%%%%%%%%%%%%%%%%%%%%%%%%%%%%%%%%%%%%%%%%%%%%%%%
\section{Information}

%%%%%%%%%%%%%%%%%%%%%%%%%%%%%%%%%%%%%%%%%%%%%%%%%%%%%%%%%%%%%%%%%%%%%%%%%%%%%%%%
\subsection{Copyright}

Copyright \copyright{} 2017--2018 Niklas Beisert

This work may be distributed and/or modified under the
conditions of the \LaTeX{} Project Public License, either version 1.3
of this license or (at your option) any later version.
The latest version of this license is in
  \url{http://www.latex-project.org/lppl.txt}
and version 1.3 or later is part of all distributions of \LaTeX{}
version 2005/12/01 or later.

This work has the LPPL maintenance status `maintained'.

The Current Maintainer of this work is Niklas Beisert.

This work consists of the files |README.txt|, |childdoc.ins| and |childdoc.dtx|
as well as the derived files |childdoc.def|, |cdocsamp.tex|
with |cdocsch1.tex|, |cdocsch2.tex|, |cdocspt3.tex|, |cdocspt4.tex|,
|cdocsdrf.tex|, |cdocsfn1.tex|, |cdocsfn2.tex|
as well as |childdoc.pdf|.

%%%%%%%%%%%%%%%%%%%%%%%%%%%%%%%%%%%%%%%%%%%%%%%%%%%%%%%%%%%%%%%%%%%%%%%%%%%%%%%%
\subsection{Files and Installation}

The package consists of the files:
%
\begin{center}
\begin{tabular}{ll}
    |README.txt|   & readme file \\
    |childdoc.ins| & installation file \\
    |childdoc.dtx| & source file \\
    |childdoc.def| & definition file \\
    |cdocsamp.tex| & sample main file \\
    |cdocsch1.tex| & sample include file \\
    |cdocsch2.tex| & sample include file \\
    |cdocspt3.tex| & sample part file \\
    |cdocspt4.tex| & sample part file \\
    |cdocsdrf.tex| & sample redirection file \\
    |cdocsfn1.tex| & sample redirection file \\
    |cdocsfn2.tex| & sample redirection file \\
    |childdoc.pdf| & manual
\end{tabular}
\end{center}
%
The distribution consists of the files
|README.txt|, |childdoc.ins| and |childdoc.dtx|.
%
\begin{itemize}
\item
Run (pdf)\LaTeX{} on |childdoc.dtx|
to compile the manual |childdoc.pdf| (this file).
\item
Run \LaTeX{} on |childdoc.ins| to create the definitions file |childdoc.def|
and the sample |cdocsamp.tex| with include files
|cdocsch1.tex|, |cdocsch2.tex|, |cdocspt3.tex|, |cdocspt4.tex|,
|cdocsdrf.tex|, |cdocsfn1.tex|, |cdocsfn2.tex|.
Then copy the file |childdoc.def| to an appropriate directory of your \LaTeX{}
distribution, e.g.\ \textit{texmf-root}|/tex/latex/childdoc|.
\end{itemize}

%%%%%%%%%%%%%%%%%%%%%%%%%%%%%%%%%%%%%%%%%%%%%%%%%%%%%%%%%%%%%%%%%%%%%%%%%%%%%%%%
\subsection{Related CTAN Packages}

There are several other packages which offer a similar functionality:
%
\begin{itemize}
\item
The packages
\href{http://ctan.org/pkg/docmute}{\textsf{docmute}},
\href{http://ctan.org/pkg/includex}{\textsf{includex}} and
\href{http://ctan.org/pkg/standalone}{\textsf{standalone}}
provide commands to include only the document body of
a child file thus allowing both files to be compiled individually.
\item
The packages \href{http://ctan.org/pkg/subdocs}{\textsf{subdocs}}
and \href{http://ctan.org/pkg/subfiles}{\textsf{subfiles}}
provide structures in which the main and child documents can be
encapsulated and allowing them to be compiled individually.
The inclusion mechanism is different from the conventional |\include|.
\item
The package \href{http://ctan.org/pkg/combine}{\textsf{combine}}
is an elaborate solution to combine several documents into one.
\end{itemize}
%
See also the CTAN topic \href{http://ctan.org/topic/subdocs}{\textsf{subdocs}}
for further related packages.
The present package differs from the above solutions in that
a document structure constructed with the conventional |\include| mechanism
just needs two extra commands at the top of every file
such that all constituent files can be compiled individually.

%%%%%%%%%%%%%%%%%%%%%%%%%%%%%%%%%%%%%%%%%%%%%%%%%%%%%%%%%%%%%%%%%%%%%%%%%%%%%%%%
%\subsection{Feature Suggestions}
%
%The following is a list of features which may be useful for future
%versions of this package:
%%
%\begin{itemize}
%\item
%\ldots
%\end{itemize}

%%%%%%%%%%%%%%%%%%%%%%%%%%%%%%%%%%%%%%%%%%%%%%%%%%%%%%%%%%%%%%%%%%%%%%%%%%%%%%%%
\subsection{Revision History}

%%%%%%%%%%%%%%%%%%%%%%%%%%%%%%%%%%%%%%%%
\paragraph{v2.0:} 2018/12/30

\begin{itemize}
\item
immediate forward processing
\item
added |\childdocby| mechanism
\item
manual restructured
\end{itemize}

%%%%%%%%%%%%%%%%%%%%%%%%%%%%%%%%%%%%%%%%
\paragraph{v1.6:} 2018/01/17

\begin{itemize}
\item
application for development of include files
\item
corrections to manual
\end{itemize}

%%%%%%%%%%%%%%%%%%%%%%%%%%%%%%%%%%%%%%%%
\paragraph{v1.5:} 2017/05/21

\begin{itemize}
\item
more complete structuring introduced
\item
|\childdocof| introduced
\item
|\childdoc| renamed to |\childdocmain|
\item
|\childredirect| renamed to |\childdocforward| and |\childdocforwardprefix|
and functionality expanded
\end{itemize}

%%%%%%%%%%%%%%%%%%%%%%%%%%%%%%%%%%%%%%%%
\paragraph{v1.0:} 2017/04/27

\begin{itemize}
\item
manual and install package
\item
first version published on CTAN
\end{itemize}

%%%%%%%%%%%%%%%%%%%%%%%%%%%%%%%%%%%%%%%%
\paragraph{v0.6:} 2017/04/26

\begin{itemize}
\item
redirection mechanism added
\end{itemize}

%%%%%%%%%%%%%%%%%%%%%%%%%%%%%%%%%%%%%%%%
\paragraph{v0.5:} 2017/04/26

\begin{itemize}
\item
functionality in definition file
\end{itemize}


%%%%%%%%%%%%%%%%%%%%%%%%%%%%%%%%%%%%%%%%%%%%%%%%%%%%%%%%%%%%%%%%%%%%%%%%%%%%%%%%
%%%%%%%%%%%%%%%%%%%%%%%%%%%%%%%%%%%%%%%%%%%%%%%%%%%%%%%%%%%%%%%%%%%%%%%%%%%%%%%%
%%%%%%%%%%%%%%%%%%%%%%%%%%%%%%%%%%%%%%%%%%%%%%%%%%%%%%%%%%%%%%%%%%%%%%%%%%%%%%%%
\appendix

\settowidth\MacroIndent{\rmfamily\scriptsize 000\ }

 \DocInput{childdoc.dtx}

\end{document}
%</driver>
% \fi
%
% %%%%%%%%%%%%%%%%%%%%%%%%%%%%%%%%%%%%%%%%%%%%%%%%%%%%%%%%%%%%%%%%%%%%%%%%%%%%%%
% %%%%%%%%%%%%%%%%%%%%%%%%%%%%%%%%%%%%%%%%%%%%%%%%%%%%%%%%%%%%%%%%%%%%%%%%%%%%%%
% \section{Sample}
%\iffalse
%<*samplemain>
%\fi
%
% The following presents a sample document
% with two chapters, two parts, a title page,
% a compile flag as well as three forwarding files to set the flag.
% It consists of eight |.tex| files:
% \begin{center}
% \begin{tabular}{ll}
% |cdocsamp.tex|&main file\\
% |cdocsch1.tex|&include file for chapter 1\\
% |cdocsch2.tex|&include file for chapter 2\\
% |cdocspt3.tex|&include file for part 3\\
% |cdocspt4.tex|&include file for part 4\\
% |cdocsdrf.tex|&forwarding file for main file in draft mode\\
% |cdocsfi1.tex|&forwarding file for final version of chapter 1\\
% |cdocsfi2.tex|&forwarding file for final version of chapter 2\\
% \end{tabular}
% \end{center}
% Each of the eight files can be compiled directly by the \LaTeX{} compiler.
%
% %%%%%%%%%%%%%%%%%%%%%%%%%%%%%%%%%%%%%%
% \paragraph{Main File.}
%
% The main file is called |cdocsamp.tex|.
%
% Load the \textsf{childdoc} definitions and
% declare the filename for the main document:
%    \begin{macrocode}
\input{childdoc.def}
\childdocmain{}
%    \end{macrocode}

% Optional override for |\version| flag:
%    \begin{macrocode}
%%\ifchilddoc\else\providecommand{\version}{draft}\fi
%    \end{macrocode}

% Define the default values for the |\version| flag
% (|final| for the main file and |draft| for childs):
%    \begin{macrocode}
\ifchilddoc
\providecommand{\version}{draft}
\else
\providecommand{\version}{final}
\fi
%    \end{macrocode}

% Load the standard document class:
%    \begin{macrocode}
\documentclass[12pt]{article}
%    \end{macrocode}

% Start the document body:
%    \begin{macrocode}
\begin{document}
%    \end{macrocode}

% Declare a title page.
% Print title, part of document being processed and version flag:
%    \begin{macrocode}
\addtocounter{page}{-1}
\begin{center}
{\LARGE\bfseries{}childdoc example\par}
\vspace{1cm}
\ifchilddoc
\ifchilddocmanual part\else chapter\fi:
`\childdocname' of `\childdocjob'\par
\else
main document: `\childdocjob'\par
\fi
version: \version\par
\end{center}
\newpage
%    \end{macrocode}

% Manually include selected file,
% otherwise process as usual:
%    \begin{macrocode}
\ifchilddocmanual
\section*{part `\childdocname'}
\input{\childdocname}
\else
%    \end{macrocode}

% Include the two chapters:
%    \begin{macrocode}
\include{cdocsch1}
\include{cdocsch2}
%    \end{macrocode}

% Include the two parts unless only chapters should be displayed:
%    \begin{macrocode}
\ifchilddoc\else
\section{part three}
\input{cdocspt3}
\section{part four}
\input{cdocspt4}
\fi
%    \end{macrocode}

% Process as usual until here:
%    \begin{macrocode}
\fi
%    \end{macrocode}

% End of document body:
%    \begin{macrocode}
\end{document}
%    \end{macrocode}
%\iffalse
%</samplemain>
%\fi
%
% %%%%%%%%%%%%%%%%%%%%%%%%%%%%%%%%%%%%%%
% \paragraph{Chapter Include Files.}
%
% The include files are called |cdocsch1.tex| and |cdocsch2.tex|.
%
%\iffalse
%<*samplechap1|samplechap2>
%\fi

% Optional override for |\version| flag:
%    \begin{macrocode}
%%\providecommand{\version}{final}
%    \end{macrocode}

% Include the main document:
%    \begin{macrocode}
\input{childdoc.def}
\childdocof{cdocsamp}
%    \end{macrocode}

%\iffalse
%</samplechap1|samplechap2>
%\fi
%
%\iffalse
%<*samplechap1>
%\fi
% Some text for chapter 1:
%    \begin{macrocode}
\section{one}
some text in chapter one
%    \end{macrocode}

%\iffalse
%</samplechap1>
%\fi
% Some text for chapter 2:
%\iffalse
%<*samplechap2>
%\fi
%    \begin{macrocode}
\section{two}
more text in chapter two
%    \end{macrocode}

%\iffalse
%</samplechap2>
%\fi
%
% %%%%%%%%%%%%%%%%%%%%%%%%%%%%%%%%%%%%%%
% \paragraph{Part Include Files.}
%
% The include files are called |cdocspt3.tex| and |cdocspt4.tex|.
%
%\iffalse
%<*samplepart3|samplepart4>
%\fi

% Optional override for |\version| flag:
%    \begin{macrocode}
%%\providecommand{\version}{final}
%    \end{macrocode}

% Include the main document:
%    \begin{macrocode}
\input{childdoc.def}
\childdocby{cdocsamp}
%    \end{macrocode}

%\iffalse
%</samplepart3|samplepart4>
%\fi
%
%\iffalse
%<*samplepart3>
%\fi
% Some text for part 3:
%    \begin{macrocode}
some text in part three
%    \end{macrocode}

%\iffalse
%</samplepart3>
%\fi
% Some text for part 4:
%\iffalse
%<*samplepart4>
%\fi
%    \begin{macrocode}
more text in part four
%    \end{macrocode}

%\iffalse
%</samplepart4>
%\fi
%
% %%%%%%%%%%%%%%%%%%%%%%%%%%%%%%%%%%%%%%
% \paragraph{Forwarding for a Complete Draft.}
%
% The following forwarding file |cdocsdrf.tex|
% compiles the main document in draft mode:
%\iffalse
%<*sampledraft>
%\fi
%    \begin{macrocode}
\def\version{draft}
\input{childdoc.def}
\childdocforward{cdocsamp}
%    \end{macrocode}

%\iffalse
%</sampledraft>
%\fi
%
% %%%%%%%%%%%%%%%%%%%%%%%%%%%%%%%%%%%%%%
% \paragraph{Forwarding for Final Version of the Chapters.}
%
% The following forwarding files |cdocsfn1.tex| and |cdocsfn2.tex|
% (with identical content)
% compile the final versions of the child documents
% |cdocsch1.tex| and |cdocsch2.tex|, respectively:
%\iffalse
%<*samplefinal>
%\fi
%    \begin{macrocode}
\def\version{final}
\input{childdoc.def}
\childdocforwardprefix[cdocsamp]{cdocsfn}{cdocsch}
%    \end{macrocode}

%\iffalse
%</samplefinal>
%\fi
%
% %%%%%%%%%%%%%%%%%%%%%%%%%%%%%%%%%%%%%%
% \paragraph{Command Line Processing.}
%
% The following three command lines generate the output files
% |cdocscld|, |cdocscl1| and |cdocscl2|
% which should be identical to
% |cdocsdrf|, |cdocsch1| and |cdocsfn2|, respectively:
% \begin{center}
% \begin{tabular}{l}
% |latex -jobname cdocscld \|\\
% |  "\def\version{draft}\input{childdoc.def}\childdocforward{cdocsamp}"|\\
% |latex -jobname cdocscl1 \|\\
% |  "\input{childdoc.def}\childdocforward[cdocsamp]{cdocsch1}"|\\
% |latex -jobname cdocscl2 \|\\
% |  "\def\version{final}\input{childdoc.def}\childdocforward{cdocsch2}"|
% \end{tabular}
% \end{center}
% Note that the trailing backslash on each first line
% merely continues the input to the second line
% (for convenient cut ant paste).
% Furthermore, the command |latex| can be replaced by any
% of its alternative versions such as |pdflatex|.
%
% %%%%%%%%%%%%%%%%%%%%%%%%%%%%%%%%%%%%%%%%%%%%%%%%%%%%%%%%%%%%%%%%%%%%%%%%%%%%%%
% %%%%%%%%%%%%%%%%%%%%%%%%%%%%%%%%%%%%%%%%%%%%%%%%%%%%%%%%%%%%%%%%%%%%%%%%%%%%%%
% \section{Implementation}
%\iffalse
%<*package>
%\fi
%
% This section describes the definitions file |childdoc.def|.

% The definitions cannot be loaded using |\usepackage| or |\RequirePackage|
% which has a mechanism to prevent loading a style file more than once.
% When loading the definitions by means of |\input|
% multiple instances have to be prevented manually:
%\iffalse
%This code needs to be before the `\ProvidesFile' directive
%which is defined at the beginning of this file.
%Therefore it is also placed there and commented out here.
%</package>
%<*discard>
%\fi
%    \begin{macrocode}
\ifdefined\childdocmain\endinput\fi
%    \end{macrocode}
%\iffalse
%</discard>
%<*package>
%\fi
%
% \macro{\ifchilddoc}
% \macro{\ifchilddocmanual}
% The conditional |\ifchilddoc| tells whether a
% child (true) or main (false) document is being compiled.
% The conditional |\ifchilddocmanual| tells whether
% the |\includeonly| mechanism is used (false) or
% the selection of child files must be performed manually (true).
% The definitions initialise to false:
%    \begin{macrocode}
\newif\ifchilddoc
\newif\ifchilddocmanual
%    \end{macrocode}

% \macro{\childdocname}
% \macro{\childdocjob}
% The macro |\childdocname| stores the name of the main document
% to be compiled. The macro |\childdocjob| stores the name of
% the document on which the \LaTeX{} compiler was originally invoked.
% The content of |\jobname| cannot be compared
% to filenames specified in the source due to different catcodes.
% The following code rescans |\jobname|, stores the result
% in |\childdocname| and saves a copy in |\childdocjob|:
%    \begin{macrocode}
\edef\childdocname{\scantokens\expandafter{\jobname\noexpand}}
\let\childdocjob\childdocname
%    \end{macrocode}

% \macro{\childdocdisable}
% The macro |\childdocdisable| prevents the main file
% from being processed more than once.
% At this stage, the main document command |\childdocmain|
% is assumed to be called once again where it should do nothing.
% Any subsequent call to it should prevent
% a secondary processing of the main document
% It overwrites the forwarding commands
% |\childdocof| and |\childdocforward|
% with empty macros to prevent further inclusions of the main document:
%    \begin{macrocode}
\newcommand{\childdocdisable}
{
  \renewcommand{\childdocmain}[1]{\renewcommand{\childdocmain}[1]{\endinput}}
  \renewcommand{\childdocof}[1]{}
  \renewcommand{\childdocby}[2][]{}
  \renewcommand{\childdocforward}[2][]{}
  \renewcommand{\childdocdisable}{}
}
%    \end{macrocode}

% \macro{\childdocmain}
% The macro |\childdocmain| is to be called at the top of the main file
% with nothing or the main filename (without extension) as argument.
% First, it breaks loops.
% If the argument is not empty and does not match |\childdocname|
% (which is set by the first inclusion of |childdoc.def|),
% |\ifchilddoc| is set to true, |\includeonly| is applied to the child file
% and |\jobname| is set to the main file
% (for proper handling of |.aux| files):
%    \begin{macrocode}
\newcommand{\childdocmain}[1]
{
  \childdocdisable\childdocmain{}
  \if?#1?\else
    \begingroup
      \def\childdoctmp{#1}
      \ifx\childdoctmp\childdocname
        \def\childdoctmp{}
      \else
        \def\childdoctmp
        {
          \childdoctrue
          \includeonly{\childdocname}
          \def\childdocjob{#1}
          \def\jobname{#1}
        }
      \fi
      \expandafter
    \endgroup
    \childdoctmp
  \fi
}
%    \end{macrocode}

% \macro{\childdocof}
% The command |\childdocof| redirects
% compilation to the main file |#1|.
%    \begin{macrocode}
\newcommand{\childdocof}[1]
{
  \childdocdisable
  \childdoctrue
  \includeonly{\childdocname}
  \def\jobname{#1}
  \def\childdocjob{#1}
  \input{#1}
}
%    \end{macrocode}

% \macro{\childdocby}
% The command |\childdocby| ....
%    \begin{macrocode}
\newcommand{\childdocby}[2][]
{
  \childdocdisable
  \childdoctrue
  \childdocmanualtrue
  \if?#1?\else
    \def\jobname{#2}
  \fi
  \def\childdocjob{#2}
  \input{#2}
  \endinput
}
%    \end{macrocode}

% \macro{\childdocforward}
% The command |\childdocforward| redirects
% compilation to the main file or
% (if the optional argument is given) a child file.
% Parameters are set as if the main file
% or a child file starting with |\childdocof| was compiled.
% Then compilation is handed over to the main file:
%    \begin{macrocode}
\newcommand{\childdocforward}[2][]
{
  \begingroup
    \if?#1?
      \def\childdoctmp
      {
        \def\childdocname{#2}
        \def\childdocjob{#2}
        \def\jobname{#2}
        \input{#2}
        \endinput
      }
    \else
      \def\childdoctmp
      {
        \childdocdisable
        \def\childdocname{#2}
        \childdoctrue
        \includeonly{#2}
        \def\childdocjob{#1}
        \def\jobname{#1}
        \input{#1}
        \endinput
      }
    \fi
    \expandafter
  \endgroup
  \childdoctmp
}
%    \end{macrocode}

% \macro{\childdocforwardprefix}
% The command |\childdocforwardprefix| redirects
% compilation to the main or a child file by means of a pattern.
% The prefix |#1| in the current filename is replaced by |#2|
% and the suffix of the current filename is kept
% (it is assumed that the filename does not contain the substring `|~~~|'
% which is used as a delimiter).
% Compilation is handed over to the new file by |\childdocforward|:
%    \begin{macrocode}
\newcommand{\childdocforwardprefix}[3][]
{
  \begingroup
    \def\childdocextract #2##1~~~{\def\childdoctmp{\childdocforward[#1]{#3##1}}}
    \expandafter\childdocextract\childdocname~~~
    \expandafter
  \endgroup
  \childdoctmp
}
%    \end{macrocode}

% \macro{\childdoc}
% The deprecated macro |\childdoc| is a legacy version of |\childdocmain|:
%    \begin{macrocode}
\newcommand{\childdoc}{\childdocmain}
%    \end{macrocode}

% \macro{\childdocredirect}
% The deprecated macro |\childdocredirect| is a legacy version
% of |\childdocforward| and |\childdocforwardprefix|:
%    \begin{macrocode}
\newcommand{\childdocredirect}[2][]
{
  \begingroup
    \if?#1?
      \def\childdoctmp{\childdocforward{#2}}
    \else
      \def\childdoctmp{\childdocforwardprefix{#1}{#2}}
    \fi
    \expandafter
  \endgroup
  \childdoctmp
}
%    \end{macrocode}

%\iffalse
%</package>
%\fi
%
\endinput
|\\
|\childdocby{|\textit{main}|}|\\
\end{tabular}
\end{center}
%
The directive |\childdocby| is similar to |\childdocof|
described in \secref{sec:include},
but the subsequent selection of content must be done manually.
To that end, both |\ifchilddoc| and |\ifchilddocmanual|
will be true upon processing of a part,
and the name of the part is stored in |\childdocname|.
Note that |\jobname| will be set to the filename of the current part
so that each part receives an individual |.aux| file
that does not interfere with the |.aux| file(s) of the main document.
This behaviour can be altered by the alternative form
|\childdocby[*]{|\textit{main}|}| (with a non-empty optional argument)
which uses the |.aux| file of the main document
by setting |\jobname| to \textit{main}.

%%%%%%%%%%%%%%%%%%%%%%%%%%%%%%%%%%%%%%%%%%%%%%%%%%%%%%%%%%%%%%%%%%%%%%%%%%%%%%%%
\subsection{Driver Development}
\label{sec:driver}

The \textsf{childdoc} mechanism can also be use for the development
of definition files such as \LaTeX{} styles or classes.
This case differs from the above setup with multiple parts
included by |\include| in that no |\includeonly| should be invoked.
This can be achieved by starting the include file
(before |\ProvidesPackage|) with:
%
\begin{center}
\begin{tabular}{l}
|% \iffalse
%
% childdoc.dtx Copyright (C) 2017-2018 Niklas Beisert
%
% This work may be distributed and/or modified under the
% conditions of the LaTeX Project Public License, either version 1.3
% of this license or (at your option) any later version.
% The latest version of this license is in
%   http://www.latex-project.org/lppl.txt
% and version 1.3 or later is part of all distributions of LaTeX
% version 2005/12/01 or later.
%
% This work has the LPPL maintenance status `maintained'.
%
% The Current Maintainer of this work is Niklas Beisert.
%
% This work consists of the files childdoc.dtx and childdoc.ins
% and the derived files childdoc.def and cdocsamp.tex with
% cdocsch1.tex, cdocsch2.tex, cdocsdrf.tex, cdocsfn1.tex, cdocsfn2.tex.
%
%<package>\ifdefined\childdocmain\endinput\fi
%<package>\ProvidesFile{childdoc.def}[2018/12/30 v2.0 child document driver]
%<samplemain>\ProvidesFile{cdocsamp.tex}[2018/12/30 v2.0 sample for childdoc]
%<*driver>
%\ProvidesFile{childdoc.drv}[2018/12/30 v2.0 childdoc reference manual file]
\PassOptionsToClass{10pt,a4paper}{article}
\documentclass{ltxdoc}

\usepackage[margin=35mm]{geometry}
\usepackage{hyperref}
\usepackage{hyperxmp}
\usepackage[usenames]{color}

\hypersetup{colorlinks=true}
\hypersetup{pdfstartview=FitH}
\hypersetup{pdfpagemode=UseNone}
\hypersetup{pdfsource={}}
\hypersetup{pdflang={en-UK}}
\hypersetup{pdfcopyright={Copyright 2017-2018 Niklas Beisert.
  This work may be distributed and/or modified under the
  conditions of the LaTeX Project Public License, either version 1.3
  of this license or (at your option) any later version.}}
\hypersetup{pdflicenseurl={http://www.latex-project.org/lppl.txt}}
\hypersetup{pdfcontactaddress={ETH Zurich, ITP, HIT K,
  Wolfgang-Pauli-Strasse 27}}
\hypersetup{pdfcontactpostcode={8093}}
\hypersetup{pdfcontactcity={Zurich}}
\hypersetup{pdfcontactcountry={Switzerland}}
\hypersetup{pdfcontactemail={nbeisert@itp.phys.ethz.ch}}
\hypersetup{pdfcontacturl={http://people.phys.ethz.ch/\xmptilde nbeisert/}}

\newcommand{\secref}[1]{\hyperref[#1]{section \ref*{#1}}}

\parskip1ex
\parindent0pt
\let\olditemize\itemize
\def\itemize{\olditemize\parskip0pt}

\begin{document}

\title{The \textsf{childdoc} Package}
\hypersetup{pdftitle={The childdoc Package}}
\author{Niklas Beisert\\[2ex]
  Institut f\"ur Theoretische Physik\\
  Eidgen\"ossische Technische Hochschule Z\"urich\\
  Wolfgang-Pauli-Strasse 27, 8093 Z\"urich, Switzerland\\[1ex]
  \href{mailto:nbeisert@itp.phys.ethz.ch}
  {\texttt{nbeisert@itp.phys.ethz.ch}}}
\hypersetup{pdfauthor={Niklas Beisert}}
\hypersetup{pdfsubject={Manual for the LaTeX2e Package childdoc}}
\date{30 December 2018, \textsf{v2.0}}
\maketitle

\begin{abstract}\noindent
\textsf{childdoc} is a \LaTeXe{} package
that enables the direct compilation
of document sections included by |\include|
to individual files.
\end{abstract}

\begingroup
\parskip0ex
\tableofcontents
\endgroup

%%%%%%%%%%%%%%%%%%%%%%%%%%%%%%%%%%%%%%%%%%%%%%%%%%%%%%%%%%%%%%%%%%%%%%%%%%%%%%%%
%%%%%%%%%%%%%%%%%%%%%%%%%%%%%%%%%%%%%%%%%%%%%%%%%%%%%%%%%%%%%%%%%%%%%%%%%%%%%%%%
\section{Introduction}

\LaTeX{} provides a mechanism to structure a large document (such as a book)
into a main file and several child files (containing the chapters)
using the |\include| command.
This mechanism is beneficial for documents
which span hundreds of pages in order to
make the source file(s) more manageable.
Moreover, compilation can be restricted to
selected child files by means of the |\includeonly| command.
The latter feature can be used to reduce the compilation time while editing
(this was significantly more useful in the earlier days of \LaTeX{})
or to generate a smaller document which is easier to navigate.
Another application of |\includeonly| is to generate
documents consisting of selected parts of the complete document.

However, there are a few drawbacks of the plain |\include| mechanism:
\begin{itemize}
\item
The child files cannot be compiled on their own,
they can only be compiled via the main file.
A naive editing environment
(such as a text editor with an option
to have the current file processed by \LaTeX)
may require one to switch to the main file before compiling;
attempting to compile the child file produces errors.
\item
The main file must be modified (each time)
to adjust the |\includeonly| command
to the present needs. This easily leaves the main file in a messy state.
\item
The generated document will always carry the filename
of the main document. This is inconvenient if
several child files are to be compiled and
to be kept for distribution.
\end{itemize}

The present package provides a simple interface
to make child files individually compilable by \LaTeX{}.
Compiling a child file then has the same effect as compiling
the main file with an |\includeonly| command
to select the appropriate child.
Moreover the generated document will carry the name of the child
rather than the main file.
This resolves all three above issues.

This feature is meant to make the editing of books,
thesis documents and lecture notes somewhat more convenient.
However, the package can also be used efficiently for
composing a series of documents (such as exercise sheets)
which are typically distributed individually.
It then assists the author in generating the individual documents
(potentially in different versions)
as well as a document containing the collected series.
Another application is in developing style files
or other kinds of included material
where compilation of the style file could redirect
to a sample or test file.

%%%%%%%%%%%%%%%%%%%%%%%%%%%%%%%%%%%%%%%%%%%%%%%%%%%%%%%%%%%%%%%%%%%%%%%%%%%%%%%%
%%%%%%%%%%%%%%%%%%%%%%%%%%%%%%%%%%%%%%%%%%%%%%%%%%%%%%%%%%%%%%%%%%%%%%%%%%%%%%%%
\section{Usage}

First of all, the package \textsf{childdoc} is \emph{not} a standard
\LaTeXe{} |.sty| style file! Therefore it needs to be invoked in
a non-standard way.

%%%%%%%%%%%%%%%%%%%%%%%%%%%%%%%%%%%%%%%%%%%%%%%%%%%%%%%%%%%%%%%%%%%%%%%%%%%%%%%%
\subsection{Included Files}
\label{sec:include}

%%%%%%%%%%%%%%%%%%%%%%%%%%%%%%%%%%%%%%%%
\DescribeMacro{\childdocmain}
To use the package, add the commands
\begin{center}
\begin{tabular}{l}
|\input{childdoc.def}|\\
|\childdocmain{}|\\
\end{tabular}
\end{center}
at the very top of the main \LaTeX{} file,
in particular \emph{before} the |\documentclass| statement!
The argument of |\childdocmain| should be left empty
(but it must be present).

%%%%%%%%%%%%%%%%%%%%%%%%%%%%%%%%%%%%%%%%
\DescribeMacro{\childdocof}
Furthermore, add the commands
\begin{center}
\begin{tabular}{l}
|\input{childdoc.def}|\\
|\childdocof{|\textit{main}|}|\\
\end{tabular}
\end{center}
at the top of every child file \textit{child}
which is included by |\include{|\textit{child}|}|
from within the main file
(or at least for those files to be compiled individually).
The argument \textit{main} must be the filename of the main file.

There are a couple of
considerations in setting up the main and child documents:

%%%%%%%%%%%%%%%%%%%%%%%%%%%%%%%%%%%%%%%%
\paragraph{Restrictions.}

Please note the following restrictions:
\begin{itemize}
\item
|\childdocmain| must be called with one argument \textit{main}
to ensure compatibility with earlier version of the package.
It must either be empty (|\childdocmain{}|)
or precisely match the filename of the main file in which it is specified.
See \secref{sec:detection} for further information.
\item
The filename \textit{main} must be specified without the |.tex| extension.
\item
The filename \textit{main} is case sensitive
(even in case-insensitive file systems)
due to internal string comparison.
\item
The argument \textit{main} should be fully expanded, it cannot be a macro.
\item
Subdirectories and special characters should be avoided in filenames.
\item
The command |\childdocmain{|\textit{main}|}| must be followed by a whitespace.
It should not be followed immediately by another command
or by a comment mark `|%|'.
This is because the \TeX{} parser reads the token immediately following
the argument of |\childdocmain| and puts it
at the beginning of every child section;
however, a white\-space is ignored.
\end{itemize}

%%%%%%%%%%%%%%%%%%%%%%%%%%%%%%%%%%%%%%%%
\paragraph{Content of Main File.}

It is advisable to place all content in the child files included by |\include|.
Any output contained in the main file will appear in all child documents
unless suppressed manually;
it cannot be suppressed automatically by the |\includeonly| directive
and thus should normally be avoided.
A method to include some content in the main file
by means of conditional processing is described in \secref{sec:conditional}.

%%%%%%%%%%%%%%%%%%%%%%%%%%%%%%%%%%%%%%%%
\paragraph{Page Numbering.}

When only a part of the document is compiled,
the appropriate numbering of pages
(as well as other status parameters)
is determined from the |.aux| files.
The latter contain information from previous passes.
However this information needs to propagate through
all intermediate child documents.
Therefore the page numbering in child documents may well
be inconsistent until the complete document is compiled at least once.

A useful (if unconventional) way to always ensure a consistent
page numbering is to restart the numbering in each child document
and denote the pages by `\textit{child}|.|\textit{page}'
where \textit{child} represents the chapter/section number of the child file.
This can be achieved by the command
|\numberwithin{page}{|\textit{child}|}|
of the \textsf{amsmath} package
where \textit{child} can be |chapter| or |section|
depending on the chosen structuring.
Alternatively, one can modify the macro |\thepage| appropriately
and reset the counter |page| at the start of each child file.

%%%%%%%%%%%%%%%%%%%%%%%%%%%%%%%%%%%%%%%%%%%%%%%%%%%%%%%%%%%%%%%%%%%%%%%%%%%%%%%%
\subsection{Conditional Processing}
\label{sec:conditional}

The package provides a mechanism to compile different versions
of a document. To customise the versions further some conditional processing
can come in handy to distinguish which version is being compiled.
The package provides two macros to describe the compilation context:

%%%%%%%%%%%%%%%%%%%%%%%%%%%%%%%%%%%%%%%%
\DescribeMacro{\ifchilddoc}
The conditional |\ifchilddoc| distinguishes between the compilation of
child documents and the main document:
%
\begin{center}
|\ifchilddoc |\textit{child-code}| |[|\||else |\textit{main-code}]| \||fi|
\end{center}

%%%%%%%%%%%%%%%%%%%%%%%%%%%%%%%%%%%%%%%%
\DescribeMacro{\childdocname}
\DescribeMacro{\childdocjob}
The macro |\childdocname| contains the filename (without extension)
of the main or child file being processed.
Note that |\childdocjob| will always contain the name of the main file.

%%%%%%%%%%%%%%%%%%%%%%%%%%%%%%%%%%%%%%%%
\paragraph{Title Page.}

Conditional processing can be used to include a title or banner page
in the main document when proper precautions are taken.
Importantly, the code in the main file should ensure that the page counter
(as well as other status parameters which are stored in the |.aux| files)
takes the same value after the conditional processing.
Otherwise the page numbers may take divergent values
depending on which part is compiled.

For example, a title page could be declared by:
%
\begin{center}
\begin{tabular}{l}
|\ifchilddoc\||else|\\
|\addtocounter{page}{-1}|\\
\textit{code for title page}\\
|\newpage|\\
|\||fi|
\end{tabular}
\end{center}
%
A banner page for the child documents can be generated by:
%
\begin{center}
\begin{tabular}{l}
|\ifchilddoc|\\
|\addtocounter{page}{-1}|\\
\textit{code for banner page}\\
|\newpage|\\
|\||fi|
\end{tabular}
\end{center}
%
Here one could write a message such as:
\begin{center}
|This is the part \childdocname{} of \childdocjob{}.|
\end{center}

%%%%%%%%%%%%%%%%%%%%%%%%%%%%%%%%%%%%%%%%%%%%%%%%%%%%%%%%%%%%%%%%%%%%%%%%%%%%%%%%
\subsection{Flags}
\label{sec:flags}

The package makes it easy to generate different versions
of the main or child documents.
To this end compilation flags can be defined
and assigned different default values.
They will be particularly useful in conjunction
with the forwarding mechanism described in \secref{sec:forward}.

For example, it may be useful to have a flag |\version|
which can be set to |draft| or |final|.
The document source will contain some conditional code
depending on the value of |\version|.
Suppose further, the flag should default to |final| for the main file
and to |draft| for child files
which is a natural assignment for editing the document.
This is achieved by placing the following code
in the preamble of the main document
(below the |\childdocmain| directive):
%
\begin{center}
\begin{tabular}{l}
|\ifchilddoc|\\
|\providecommand{\version}{draft}|\\
|\||else|\\
|\providecommand{\version}{final}|\\
|\||fi|
\end{tabular}
\end{center}
%
The definition by |\providecommand| makes sure
that previous definitions are not overwritten.
Further statements |\providecommand{\version}{...}|
can thus be added before the above code to override it.

For the main file, one might add a line
(between |\childdocmain| and the above block)
%
\begin{center}
|%\ifchilddoc\||else\providecommand{\version}{draft}\||fi|
\end{center}
%
which can be uncommented to produce a draft version.
Likewise one can add a line to the very top of a child file
(above the |\childdocof{|\textit{main}|}| directive)
%
\begin{center}
|%\providecommand{\version}{final}|
\end{center}
%
which can be uncommented to produce the final version of this child document.

%%%%%%%%%%%%%%%%%%%%%%%%%%%%%%%%%%%%%%%%%%%%%%%%%%%%%%%%%%%%%%%%%%%%%%%%%%%%%%%%
\subsection{Forwarding}
\label{sec:forward}

Different versions of the main or child documents
using compilation flags as described in \secref{sec:flags}
can be (permanently) stored in different files
for convenient compilation, viewing and distribution.
To this end, the package defines a command
to pass on compilation to a different file:

%%%%%%%%%%%%%%%%%%%%%%%%%%%%%%%%%%%%%%%%
\DescribeMacro{\childdocforward}
The command |\childdocforward| redirects processing to
another source file:
%
\begin{center}
\begin{tabular}{l}
|\input{childdoc.def}|\\
|\childdocforward[|\textit{main}|]{|\textit{dest}|}|\\
\end{tabular}
\end{center}
%
The argument \textit{dest} is the destination file
(without extension).
It should be the main file or one of the child files.
Note that further \textsf{childdoc} directives
such as |\childdocof| and |\childdocforward|
in the indicated file will be processed in this form.
The optional argument \textit{main}
passes on directly to the main file \textit{main}
while pretending to compile the child \textit{dest}.
This form behaves as if \textit{dest}
issues |\childdocof{|\textit{main}|}| right away,
and no further \textsf{childdoc} directives will be processed.

%%%%%%%%%%%%%%%%%%%%%%%%%%%%%%%%%%%%%%%%
\DescribeMacro{\...prefix}
In the alternative form |\childdocforwardprefix|,
%
\begin{center}
\begin{tabular}{l}
|\input{childdoc.def}|\\
|\childdocforwardprefix[|\textit{main}|]{|\textit{prefix}|}{|\textit{dest}|}|
\end{tabular}
\end{center}
%
the destination file is determined by a pattern
depending on the current file:
To make this work, the current file must be called
`{\textit{prefix}\hspace{0.2em}\textit{suffix}}'
with \textit{prefix} matching precisely the argument.
Processing is then passed on to the file
`{\textit{dest}\hspace{0.2em}\textit{suffix}}'.
Surely, the same effect is achieved by
directly specifying the
argument `{\textit{dest}\hspace{0.2em}\textit{suffix}}'
in the first form.
However, that requires to set up a different file
for each child. With the alternative form of the command
all these files can have exactly the same content
which simplifies setting them up and maintaining them.

For example, the following file |draft.tex|
with a compilation flag |\version| as described in \secref{sec:flags}
compiles the main document as a draft:
%
\begin{center}
\begin{tabular}{l}
|\def\version{draft}|\\
|\input{childdoc.def}|\\
|\childdocforward{|\textit{main}|}|
\end{tabular}
\end{center}
%
Likewise, the following files |final|\textit{nn}|.tex|
compile the final version of the child document
|child|\textit{nn}|.tex|:
%
\begin{center}
\begin{tabular}{l}
|\def\version{final}|\\
|\input{childdoc.def}|\\
|\childdocforwardprefix{final}{child}|
\end{tabular}
\end{center}
%

Note that when several versions of a main file and/or of each child file
are to be generated, it may be convenient to set up a |Makefile| or
shell script to automatise the process.

%%%%%%%%%%%%%%%%%%%%%%%%%%%%%%%%%%%%%%%%%%%%%%%%%%%%%%%%%%%%%%%%%%%%%%%%%%%%%%%%
\subsection{Command Line Processing}
\label{sec:commandline}

The effect of redirection files can also be achieved by invoking
the \LaTeX{} compiler with a more elaborate command line.
Most conveniently this should be done as part
of a shell script or a |Makefile|.

When using \textsf{childdoc} in the main file, the following
command lines effectively perform a redirection
(note that depending on the shell being used,
backslashes may have to be doubled: `|\|' $\to$ `|\\|'):
%
\begin{center}
|... -jobname "|\textit{target}|" |\\|"|[\textit{flags}]%
|\input{childdoc.def}\childdocforward[|\textit{main}|]{|\textit{dest}|}"|
\end{center}
%
Here \textit{target} is the name of the output file,
\textit{main} is the name of the main file
and \textit{dest} is the name of the main or child file to be processed
(all filenames without extensions).
The optional argument \textit{main} can be omitted
if \textit{main} matches \textit{dest}.
Optionally, compilation \textit{flags} can be defined via |\def| commands.
This command line makes the \TeX{} engine believe
it is compiling the file \textit{target}
whose content is specified as the latter parameter.
The provided code then forwards the processing to
\textit{main} or \textit{dest} as described in \secref{sec:forward}.

%%%%%%%%%%%%%%%%%%%%%%%%%%%%%%%%%%%%%%%%%%%%%%%%%%%%%%%%%%%%%%%%%%%%%%%%%%%%%%%%
\subsection{Include by Input}
\label{sec:input}

Including child documents by |\include| has some restrictions by design.
Most notably, the content of a child document always occupies
its own set of pages; pages cannot be shared between child documents.
Usually, this behaviour makes perfect sense
because each child document contain an essential part of the document.
However, in some situations it may be desirable to compose
a document from a collection of parts
without having mandatory page breaks between then.
For this case, the package
provides a mechanism to include parts
by |\input| which can also be processed individually.
However, by construction this mechanism
requires manual handling of the content to be output.

%%%%%%%%%%%%%%%%%%%%%%%%%%%%%%%%%%%%%%%%
\DescribeMacro{\ifchilddocmanual}
The main file should be prepared as usual, see \secref{sec:include}.
However, the document body must make a distinction
between processing of an individual part and of the main document, e.g.:
%
\begin{center}
\begin{tabular}{l}
|\ifchilddocmanual|\\
|\input{\childdocname}|\\
|\||else|\\
\textit{document body with }|\input{|\textit{part}|}|\\
|\||fi|
\end{tabular}
\end{center}
%
The conditional |\ifchilddocmanual| is true whenever
a part to be included by |\input| is being compiled,
and the name of the part is stored in |\childdocname|.

%%%%%%%%%%%%%%%%%%%%%%%%%%%%%%%%%%%%%%%%
\DescribeMacro{\childdocby}
Each part to be included by |\input| should start with:
%
\begin{center}
\begin{tabular}{l}
|\input{childdoc.def}|\\
|\childdocby{|\textit{main}|}|\\
\end{tabular}
\end{center}
%
The directive |\childdocby| is similar to |\childdocof|
described in \secref{sec:include},
but the subsequent selection of content must be done manually.
To that end, both |\ifchilddoc| and |\ifchilddocmanual|
will be true upon processing of a part,
and the name of the part is stored in |\childdocname|.
Note that |\jobname| will be set to the filename of the current part
so that each part receives an individual |.aux| file
that does not interfere with the |.aux| file(s) of the main document.
This behaviour can be altered by the alternative form
|\childdocby[*]{|\textit{main}|}| (with a non-empty optional argument)
which uses the |.aux| file of the main document
by setting |\jobname| to \textit{main}.

%%%%%%%%%%%%%%%%%%%%%%%%%%%%%%%%%%%%%%%%%%%%%%%%%%%%%%%%%%%%%%%%%%%%%%%%%%%%%%%%
\subsection{Driver Development}
\label{sec:driver}

The \textsf{childdoc} mechanism can also be use for the development
of definition files such as \LaTeX{} styles or classes.
This case differs from the above setup with multiple parts
included by |\include| in that no |\includeonly| should be invoked.
This can be achieved by starting the include file
(before |\ProvidesPackage|) with:
%
\begin{center}
\begin{tabular}{l}
|\input{childdoc.def}|\\
|\childdocforward{|\textit{main}|}|\\
\end{tabular}
\end{center}
%
or alternatively with:
%
\begin{center}
\begin{tabular}{l}
|\input{childdoc.def}|\\
|\childdocby{|\textit{main}|}|\\
\end{tabular}
\end{center}
%
Both forms have slightly different effects as described above.
The main file is prepared as usual, see \secref{sec:include}.

%%%%%%%%%%%%%%%%%%%%%%%%%%%%%%%%%%%%%%%%%%%%%%%%%%%%%%%%%%%%%%%%%%%%%%%%%%%%%%%%
\subsection{Legacy Detection}
\label{sec:detection}

The directive |\childdocmain| in the main file can detect
whether the complete document or merely a child is to be compiled
even without using the directive |\childdocof|.
This method is deprecated because it is less robust
and there is no compelling reason to use it;
it is merely provided for backward compatibility
and it may be removed in future versions.

If the detection mechanism is to be used,
it is mandatory to correctly specify
the filename of the main file as the argument of |\childdocmain|:
%
\begin{center}
\begin{tabular}{l}
|\input{childdoc.def}|\\
|\childdocmain{|\textit{main}|}|\\
\end{tabular}
\end{center}
%
If |\jobname| does not match the argument \textit{main} of |\childdocmain|,
it is assumed that |\jobname| points to the child file to be compiled.
When using |\childdocmain| with the main file specified as argument,
it suffices to start a child file
with just |\input{|\textit{main}|}|
without loading of the package and using |\childdocof|.
If instead all processing is done
with the appropriate \textsf{childdoc} directives,
the argument of \textit{main} of |\childdocmain| can be empty.

An alternative version of the command line processing described
in \secref{sec:commandline} using the detection mechanism reads:
%
\begin{center}
|... -jobname "|\textit{target}|" "|[\textit{flags}]%
[|\def\jobname{|\textit{dest}|}|]|\input{|\textit{main}|}"|
\end{center}

%%%%%%%%%%%%%%%%%%%%%%%%%%%%%%%%%%%%%%%%%%%%%%%%%%%%%%%%%%%%%%%%%%%%%%%%%%%%%%%%
\subsection{Manual Code}
\label{sec:manual}

In case one cannot be certain whether the definitions file |childdoc.def|
is installed on the target \TeX{} distribution
and one prefers not to ship it,
it is conceivable to paste a few relevant commands into the sources.

To that end, drop all statements |\input{childdoc.def}|
and perform the replacements as outlined below.
Instead of |\childdocmain{|\textit{main}|}| add the following code
to the top of the main file:
%
\begin{center}
\begin{tabular}{l}
|\||ifdefined\childdocname\endinput\||fi\newif\ifchilddoc|\\
|\edef\childdocname{\scantokens\expandafter{\jobname\noexpand}}|\\
|\def\childdocmain{|\textit{main}|}\||ifx\childdocmain\childdocname\||else|\\
|\childdoctrue\includeonly{\childdocname}\let\jobname\childdocmain\||fi|\\
\end{tabular}
\end{center}
%
Instead of |\childdocof{|\textit{main}|}| just include the main file
at the top of each child file:
%
\begin{center}
|\input{|\textit{main}|}|
\end{center}
%
A simple redirection |\childdocforward{|\textit{dest}|}| is achieved by:
%
\begin{center}
|\def\jobname{|\textit{dest}|}\input{\jobname}|
\end{center}
%
The redirection with prefix
|\childdocforwardprefix[|\textit{prefix}|]{|\textit{dest}|}|
is accomplished by:
%
\begin{center}
\begin{tabular}{l}
|{\edef\jobname{\scantokens\expandafter{\jobname\noexpand}}|\\
|\def\redirectjob |\textit{prefix}|#1~~~{\gdef\jobname{|\textit{dest}|#1}}|\\
|\expandafter\redirectjob\jobname~~~}\input{\jobname}|
\end{tabular}
\end{center}

In an alternative approach,
child documents can be compiled by a specific command line
without additional code or specific definitions:
%
\begin{center}
|... -jobname "|\textit{target}|" "|[\textit{flags}]%
|\includeonly{|\textit{dest}|}\input{|\textit{main}|}"|
\end{center}
%

%%%%%%%%%%%%%%%%%%%%%%%%%%%%%%%%%%%%%%%%%%%%%%%%%%%%%%%%%%%%%%%%%%%%%%%%%%%%%%%%
%%%%%%%%%%%%%%%%%%%%%%%%%%%%%%%%%%%%%%%%%%%%%%%%%%%%%%%%%%%%%%%%%%%%%%%%%%%%%%%%
\section{Information}

%%%%%%%%%%%%%%%%%%%%%%%%%%%%%%%%%%%%%%%%%%%%%%%%%%%%%%%%%%%%%%%%%%%%%%%%%%%%%%%%
\subsection{Copyright}

Copyright \copyright{} 2017--2018 Niklas Beisert

This work may be distributed and/or modified under the
conditions of the \LaTeX{} Project Public License, either version 1.3
of this license or (at your option) any later version.
The latest version of this license is in
  \url{http://www.latex-project.org/lppl.txt}
and version 1.3 or later is part of all distributions of \LaTeX{}
version 2005/12/01 or later.

This work has the LPPL maintenance status `maintained'.

The Current Maintainer of this work is Niklas Beisert.

This work consists of the files |README.txt|, |childdoc.ins| and |childdoc.dtx|
as well as the derived files |childdoc.def|, |cdocsamp.tex|
with |cdocsch1.tex|, |cdocsch2.tex|, |cdocspt3.tex|, |cdocspt4.tex|,
|cdocsdrf.tex|, |cdocsfn1.tex|, |cdocsfn2.tex|
as well as |childdoc.pdf|.

%%%%%%%%%%%%%%%%%%%%%%%%%%%%%%%%%%%%%%%%%%%%%%%%%%%%%%%%%%%%%%%%%%%%%%%%%%%%%%%%
\subsection{Files and Installation}

The package consists of the files:
%
\begin{center}
\begin{tabular}{ll}
    |README.txt|   & readme file \\
    |childdoc.ins| & installation file \\
    |childdoc.dtx| & source file \\
    |childdoc.def| & definition file \\
    |cdocsamp.tex| & sample main file \\
    |cdocsch1.tex| & sample include file \\
    |cdocsch2.tex| & sample include file \\
    |cdocspt3.tex| & sample part file \\
    |cdocspt4.tex| & sample part file \\
    |cdocsdrf.tex| & sample redirection file \\
    |cdocsfn1.tex| & sample redirection file \\
    |cdocsfn2.tex| & sample redirection file \\
    |childdoc.pdf| & manual
\end{tabular}
\end{center}
%
The distribution consists of the files
|README.txt|, |childdoc.ins| and |childdoc.dtx|.
%
\begin{itemize}
\item
Run (pdf)\LaTeX{} on |childdoc.dtx|
to compile the manual |childdoc.pdf| (this file).
\item
Run \LaTeX{} on |childdoc.ins| to create the definitions file |childdoc.def|
and the sample |cdocsamp.tex| with include files
|cdocsch1.tex|, |cdocsch2.tex|, |cdocspt3.tex|, |cdocspt4.tex|,
|cdocsdrf.tex|, |cdocsfn1.tex|, |cdocsfn2.tex|.
Then copy the file |childdoc.def| to an appropriate directory of your \LaTeX{}
distribution, e.g.\ \textit{texmf-root}|/tex/latex/childdoc|.
\end{itemize}

%%%%%%%%%%%%%%%%%%%%%%%%%%%%%%%%%%%%%%%%%%%%%%%%%%%%%%%%%%%%%%%%%%%%%%%%%%%%%%%%
\subsection{Related CTAN Packages}

There are several other packages which offer a similar functionality:
%
\begin{itemize}
\item
The packages
\href{http://ctan.org/pkg/docmute}{\textsf{docmute}},
\href{http://ctan.org/pkg/includex}{\textsf{includex}} and
\href{http://ctan.org/pkg/standalone}{\textsf{standalone}}
provide commands to include only the document body of
a child file thus allowing both files to be compiled individually.
\item
The packages \href{http://ctan.org/pkg/subdocs}{\textsf{subdocs}}
and \href{http://ctan.org/pkg/subfiles}{\textsf{subfiles}}
provide structures in which the main and child documents can be
encapsulated and allowing them to be compiled individually.
The inclusion mechanism is different from the conventional |\include|.
\item
The package \href{http://ctan.org/pkg/combine}{\textsf{combine}}
is an elaborate solution to combine several documents into one.
\end{itemize}
%
See also the CTAN topic \href{http://ctan.org/topic/subdocs}{\textsf{subdocs}}
for further related packages.
The present package differs from the above solutions in that
a document structure constructed with the conventional |\include| mechanism
just needs two extra commands at the top of every file
such that all constituent files can be compiled individually.

%%%%%%%%%%%%%%%%%%%%%%%%%%%%%%%%%%%%%%%%%%%%%%%%%%%%%%%%%%%%%%%%%%%%%%%%%%%%%%%%
%\subsection{Feature Suggestions}
%
%The following is a list of features which may be useful for future
%versions of this package:
%%
%\begin{itemize}
%\item
%\ldots
%\end{itemize}

%%%%%%%%%%%%%%%%%%%%%%%%%%%%%%%%%%%%%%%%%%%%%%%%%%%%%%%%%%%%%%%%%%%%%%%%%%%%%%%%
\subsection{Revision History}

%%%%%%%%%%%%%%%%%%%%%%%%%%%%%%%%%%%%%%%%
\paragraph{v2.0:} 2018/12/30

\begin{itemize}
\item
immediate forward processing
\item
added |\childdocby| mechanism
\item
manual restructured
\end{itemize}

%%%%%%%%%%%%%%%%%%%%%%%%%%%%%%%%%%%%%%%%
\paragraph{v1.6:} 2018/01/17

\begin{itemize}
\item
application for development of include files
\item
corrections to manual
\end{itemize}

%%%%%%%%%%%%%%%%%%%%%%%%%%%%%%%%%%%%%%%%
\paragraph{v1.5:} 2017/05/21

\begin{itemize}
\item
more complete structuring introduced
\item
|\childdocof| introduced
\item
|\childdoc| renamed to |\childdocmain|
\item
|\childredirect| renamed to |\childdocforward| and |\childdocforwardprefix|
and functionality expanded
\end{itemize}

%%%%%%%%%%%%%%%%%%%%%%%%%%%%%%%%%%%%%%%%
\paragraph{v1.0:} 2017/04/27

\begin{itemize}
\item
manual and install package
\item
first version published on CTAN
\end{itemize}

%%%%%%%%%%%%%%%%%%%%%%%%%%%%%%%%%%%%%%%%
\paragraph{v0.6:} 2017/04/26

\begin{itemize}
\item
redirection mechanism added
\end{itemize}

%%%%%%%%%%%%%%%%%%%%%%%%%%%%%%%%%%%%%%%%
\paragraph{v0.5:} 2017/04/26

\begin{itemize}
\item
functionality in definition file
\end{itemize}


%%%%%%%%%%%%%%%%%%%%%%%%%%%%%%%%%%%%%%%%%%%%%%%%%%%%%%%%%%%%%%%%%%%%%%%%%%%%%%%%
%%%%%%%%%%%%%%%%%%%%%%%%%%%%%%%%%%%%%%%%%%%%%%%%%%%%%%%%%%%%%%%%%%%%%%%%%%%%%%%%
%%%%%%%%%%%%%%%%%%%%%%%%%%%%%%%%%%%%%%%%%%%%%%%%%%%%%%%%%%%%%%%%%%%%%%%%%%%%%%%%
\appendix

\settowidth\MacroIndent{\rmfamily\scriptsize 000\ }

 \DocInput{childdoc.dtx}

\end{document}
%</driver>
% \fi
%
% %%%%%%%%%%%%%%%%%%%%%%%%%%%%%%%%%%%%%%%%%%%%%%%%%%%%%%%%%%%%%%%%%%%%%%%%%%%%%%
% %%%%%%%%%%%%%%%%%%%%%%%%%%%%%%%%%%%%%%%%%%%%%%%%%%%%%%%%%%%%%%%%%%%%%%%%%%%%%%
% \section{Sample}
%\iffalse
%<*samplemain>
%\fi
%
% The following presents a sample document
% with two chapters, two parts, a title page,
% a compile flag as well as three forwarding files to set the flag.
% It consists of eight |.tex| files:
% \begin{center}
% \begin{tabular}{ll}
% |cdocsamp.tex|&main file\\
% |cdocsch1.tex|&include file for chapter 1\\
% |cdocsch2.tex|&include file for chapter 2\\
% |cdocspt3.tex|&include file for part 3\\
% |cdocspt4.tex|&include file for part 4\\
% |cdocsdrf.tex|&forwarding file for main file in draft mode\\
% |cdocsfi1.tex|&forwarding file for final version of chapter 1\\
% |cdocsfi2.tex|&forwarding file for final version of chapter 2\\
% \end{tabular}
% \end{center}
% Each of the eight files can be compiled directly by the \LaTeX{} compiler.
%
% %%%%%%%%%%%%%%%%%%%%%%%%%%%%%%%%%%%%%%
% \paragraph{Main File.}
%
% The main file is called |cdocsamp.tex|.
%
% Load the \textsf{childdoc} definitions and
% declare the filename for the main document:
%    \begin{macrocode}
\input{childdoc.def}
\childdocmain{}
%    \end{macrocode}

% Optional override for |\version| flag:
%    \begin{macrocode}
%%\ifchilddoc\else\providecommand{\version}{draft}\fi
%    \end{macrocode}

% Define the default values for the |\version| flag
% (|final| for the main file and |draft| for childs):
%    \begin{macrocode}
\ifchilddoc
\providecommand{\version}{draft}
\else
\providecommand{\version}{final}
\fi
%    \end{macrocode}

% Load the standard document class:
%    \begin{macrocode}
\documentclass[12pt]{article}
%    \end{macrocode}

% Start the document body:
%    \begin{macrocode}
\begin{document}
%    \end{macrocode}

% Declare a title page.
% Print title, part of document being processed and version flag:
%    \begin{macrocode}
\addtocounter{page}{-1}
\begin{center}
{\LARGE\bfseries{}childdoc example\par}
\vspace{1cm}
\ifchilddoc
\ifchilddocmanual part\else chapter\fi:
`\childdocname' of `\childdocjob'\par
\else
main document: `\childdocjob'\par
\fi
version: \version\par
\end{center}
\newpage
%    \end{macrocode}

% Manually include selected file,
% otherwise process as usual:
%    \begin{macrocode}
\ifchilddocmanual
\section*{part `\childdocname'}
\input{\childdocname}
\else
%    \end{macrocode}

% Include the two chapters:
%    \begin{macrocode}
\include{cdocsch1}
\include{cdocsch2}
%    \end{macrocode}

% Include the two parts unless only chapters should be displayed:
%    \begin{macrocode}
\ifchilddoc\else
\section{part three}
\input{cdocspt3}
\section{part four}
\input{cdocspt4}
\fi
%    \end{macrocode}

% Process as usual until here:
%    \begin{macrocode}
\fi
%    \end{macrocode}

% End of document body:
%    \begin{macrocode}
\end{document}
%    \end{macrocode}
%\iffalse
%</samplemain>
%\fi
%
% %%%%%%%%%%%%%%%%%%%%%%%%%%%%%%%%%%%%%%
% \paragraph{Chapter Include Files.}
%
% The include files are called |cdocsch1.tex| and |cdocsch2.tex|.
%
%\iffalse
%<*samplechap1|samplechap2>
%\fi

% Optional override for |\version| flag:
%    \begin{macrocode}
%%\providecommand{\version}{final}
%    \end{macrocode}

% Include the main document:
%    \begin{macrocode}
\input{childdoc.def}
\childdocof{cdocsamp}
%    \end{macrocode}

%\iffalse
%</samplechap1|samplechap2>
%\fi
%
%\iffalse
%<*samplechap1>
%\fi
% Some text for chapter 1:
%    \begin{macrocode}
\section{one}
some text in chapter one
%    \end{macrocode}

%\iffalse
%</samplechap1>
%\fi
% Some text for chapter 2:
%\iffalse
%<*samplechap2>
%\fi
%    \begin{macrocode}
\section{two}
more text in chapter two
%    \end{macrocode}

%\iffalse
%</samplechap2>
%\fi
%
% %%%%%%%%%%%%%%%%%%%%%%%%%%%%%%%%%%%%%%
% \paragraph{Part Include Files.}
%
% The include files are called |cdocspt3.tex| and |cdocspt4.tex|.
%
%\iffalse
%<*samplepart3|samplepart4>
%\fi

% Optional override for |\version| flag:
%    \begin{macrocode}
%%\providecommand{\version}{final}
%    \end{macrocode}

% Include the main document:
%    \begin{macrocode}
\input{childdoc.def}
\childdocby{cdocsamp}
%    \end{macrocode}

%\iffalse
%</samplepart3|samplepart4>
%\fi
%
%\iffalse
%<*samplepart3>
%\fi
% Some text for part 3:
%    \begin{macrocode}
some text in part three
%    \end{macrocode}

%\iffalse
%</samplepart3>
%\fi
% Some text for part 4:
%\iffalse
%<*samplepart4>
%\fi
%    \begin{macrocode}
more text in part four
%    \end{macrocode}

%\iffalse
%</samplepart4>
%\fi
%
% %%%%%%%%%%%%%%%%%%%%%%%%%%%%%%%%%%%%%%
% \paragraph{Forwarding for a Complete Draft.}
%
% The following forwarding file |cdocsdrf.tex|
% compiles the main document in draft mode:
%\iffalse
%<*sampledraft>
%\fi
%    \begin{macrocode}
\def\version{draft}
\input{childdoc.def}
\childdocforward{cdocsamp}
%    \end{macrocode}

%\iffalse
%</sampledraft>
%\fi
%
% %%%%%%%%%%%%%%%%%%%%%%%%%%%%%%%%%%%%%%
% \paragraph{Forwarding for Final Version of the Chapters.}
%
% The following forwarding files |cdocsfn1.tex| and |cdocsfn2.tex|
% (with identical content)
% compile the final versions of the child documents
% |cdocsch1.tex| and |cdocsch2.tex|, respectively:
%\iffalse
%<*samplefinal>
%\fi
%    \begin{macrocode}
\def\version{final}
\input{childdoc.def}
\childdocforwardprefix[cdocsamp]{cdocsfn}{cdocsch}
%    \end{macrocode}

%\iffalse
%</samplefinal>
%\fi
%
% %%%%%%%%%%%%%%%%%%%%%%%%%%%%%%%%%%%%%%
% \paragraph{Command Line Processing.}
%
% The following three command lines generate the output files
% |cdocscld|, |cdocscl1| and |cdocscl2|
% which should be identical to
% |cdocsdrf|, |cdocsch1| and |cdocsfn2|, respectively:
% \begin{center}
% \begin{tabular}{l}
% |latex -jobname cdocscld \|\\
% |  "\def\version{draft}\input{childdoc.def}\childdocforward{cdocsamp}"|\\
% |latex -jobname cdocscl1 \|\\
% |  "\input{childdoc.def}\childdocforward[cdocsamp]{cdocsch1}"|\\
% |latex -jobname cdocscl2 \|\\
% |  "\def\version{final}\input{childdoc.def}\childdocforward{cdocsch2}"|
% \end{tabular}
% \end{center}
% Note that the trailing backslash on each first line
% merely continues the input to the second line
% (for convenient cut ant paste).
% Furthermore, the command |latex| can be replaced by any
% of its alternative versions such as |pdflatex|.
%
% %%%%%%%%%%%%%%%%%%%%%%%%%%%%%%%%%%%%%%%%%%%%%%%%%%%%%%%%%%%%%%%%%%%%%%%%%%%%%%
% %%%%%%%%%%%%%%%%%%%%%%%%%%%%%%%%%%%%%%%%%%%%%%%%%%%%%%%%%%%%%%%%%%%%%%%%%%%%%%
% \section{Implementation}
%\iffalse
%<*package>
%\fi
%
% This section describes the definitions file |childdoc.def|.

% The definitions cannot be loaded using |\usepackage| or |\RequirePackage|
% which has a mechanism to prevent loading a style file more than once.
% When loading the definitions by means of |\input|
% multiple instances have to be prevented manually:
%\iffalse
%This code needs to be before the `\ProvidesFile' directive
%which is defined at the beginning of this file.
%Therefore it is also placed there and commented out here.
%</package>
%<*discard>
%\fi
%    \begin{macrocode}
\ifdefined\childdocmain\endinput\fi
%    \end{macrocode}
%\iffalse
%</discard>
%<*package>
%\fi
%
% \macro{\ifchilddoc}
% \macro{\ifchilddocmanual}
% The conditional |\ifchilddoc| tells whether a
% child (true) or main (false) document is being compiled.
% The conditional |\ifchilddocmanual| tells whether
% the |\includeonly| mechanism is used (false) or
% the selection of child files must be performed manually (true).
% The definitions initialise to false:
%    \begin{macrocode}
\newif\ifchilddoc
\newif\ifchilddocmanual
%    \end{macrocode}

% \macro{\childdocname}
% \macro{\childdocjob}
% The macro |\childdocname| stores the name of the main document
% to be compiled. The macro |\childdocjob| stores the name of
% the document on which the \LaTeX{} compiler was originally invoked.
% The content of |\jobname| cannot be compared
% to filenames specified in the source due to different catcodes.
% The following code rescans |\jobname|, stores the result
% in |\childdocname| and saves a copy in |\childdocjob|:
%    \begin{macrocode}
\edef\childdocname{\scantokens\expandafter{\jobname\noexpand}}
\let\childdocjob\childdocname
%    \end{macrocode}

% \macro{\childdocdisable}
% The macro |\childdocdisable| prevents the main file
% from being processed more than once.
% At this stage, the main document command |\childdocmain|
% is assumed to be called once again where it should do nothing.
% Any subsequent call to it should prevent
% a secondary processing of the main document
% It overwrites the forwarding commands
% |\childdocof| and |\childdocforward|
% with empty macros to prevent further inclusions of the main document:
%    \begin{macrocode}
\newcommand{\childdocdisable}
{
  \renewcommand{\childdocmain}[1]{\renewcommand{\childdocmain}[1]{\endinput}}
  \renewcommand{\childdocof}[1]{}
  \renewcommand{\childdocby}[2][]{}
  \renewcommand{\childdocforward}[2][]{}
  \renewcommand{\childdocdisable}{}
}
%    \end{macrocode}

% \macro{\childdocmain}
% The macro |\childdocmain| is to be called at the top of the main file
% with nothing or the main filename (without extension) as argument.
% First, it breaks loops.
% If the argument is not empty and does not match |\childdocname|
% (which is set by the first inclusion of |childdoc.def|),
% |\ifchilddoc| is set to true, |\includeonly| is applied to the child file
% and |\jobname| is set to the main file
% (for proper handling of |.aux| files):
%    \begin{macrocode}
\newcommand{\childdocmain}[1]
{
  \childdocdisable\childdocmain{}
  \if?#1?\else
    \begingroup
      \def\childdoctmp{#1}
      \ifx\childdoctmp\childdocname
        \def\childdoctmp{}
      \else
        \def\childdoctmp
        {
          \childdoctrue
          \includeonly{\childdocname}
          \def\childdocjob{#1}
          \def\jobname{#1}
        }
      \fi
      \expandafter
    \endgroup
    \childdoctmp
  \fi
}
%    \end{macrocode}

% \macro{\childdocof}
% The command |\childdocof| redirects
% compilation to the main file |#1|.
%    \begin{macrocode}
\newcommand{\childdocof}[1]
{
  \childdocdisable
  \childdoctrue
  \includeonly{\childdocname}
  \def\jobname{#1}
  \def\childdocjob{#1}
  \input{#1}
}
%    \end{macrocode}

% \macro{\childdocby}
% The command |\childdocby| ....
%    \begin{macrocode}
\newcommand{\childdocby}[2][]
{
  \childdocdisable
  \childdoctrue
  \childdocmanualtrue
  \if?#1?\else
    \def\jobname{#2}
  \fi
  \def\childdocjob{#2}
  \input{#2}
  \endinput
}
%    \end{macrocode}

% \macro{\childdocforward}
% The command |\childdocforward| redirects
% compilation to the main file or
% (if the optional argument is given) a child file.
% Parameters are set as if the main file
% or a child file starting with |\childdocof| was compiled.
% Then compilation is handed over to the main file:
%    \begin{macrocode}
\newcommand{\childdocforward}[2][]
{
  \begingroup
    \if?#1?
      \def\childdoctmp
      {
        \def\childdocname{#2}
        \def\childdocjob{#2}
        \def\jobname{#2}
        \input{#2}
        \endinput
      }
    \else
      \def\childdoctmp
      {
        \childdocdisable
        \def\childdocname{#2}
        \childdoctrue
        \includeonly{#2}
        \def\childdocjob{#1}
        \def\jobname{#1}
        \input{#1}
        \endinput
      }
    \fi
    \expandafter
  \endgroup
  \childdoctmp
}
%    \end{macrocode}

% \macro{\childdocforwardprefix}
% The command |\childdocforwardprefix| redirects
% compilation to the main or a child file by means of a pattern.
% The prefix |#1| in the current filename is replaced by |#2|
% and the suffix of the current filename is kept
% (it is assumed that the filename does not contain the substring `|~~~|'
% which is used as a delimiter).
% Compilation is handed over to the new file by |\childdocforward|:
%    \begin{macrocode}
\newcommand{\childdocforwardprefix}[3][]
{
  \begingroup
    \def\childdocextract #2##1~~~{\def\childdoctmp{\childdocforward[#1]{#3##1}}}
    \expandafter\childdocextract\childdocname~~~
    \expandafter
  \endgroup
  \childdoctmp
}
%    \end{macrocode}

% \macro{\childdoc}
% The deprecated macro |\childdoc| is a legacy version of |\childdocmain|:
%    \begin{macrocode}
\newcommand{\childdoc}{\childdocmain}
%    \end{macrocode}

% \macro{\childdocredirect}
% The deprecated macro |\childdocredirect| is a legacy version
% of |\childdocforward| and |\childdocforwardprefix|:
%    \begin{macrocode}
\newcommand{\childdocredirect}[2][]
{
  \begingroup
    \if?#1?
      \def\childdoctmp{\childdocforward{#2}}
    \else
      \def\childdoctmp{\childdocforwardprefix{#1}{#2}}
    \fi
    \expandafter
  \endgroup
  \childdoctmp
}
%    \end{macrocode}

%\iffalse
%</package>
%\fi
%
\endinput
|\\
|\childdocforward{|\textit{main}|}|\\
\end{tabular}
\end{center}
%
or alternatively with:
%
\begin{center}
\begin{tabular}{l}
|% \iffalse
%
% childdoc.dtx Copyright (C) 2017-2018 Niklas Beisert
%
% This work may be distributed and/or modified under the
% conditions of the LaTeX Project Public License, either version 1.3
% of this license or (at your option) any later version.
% The latest version of this license is in
%   http://www.latex-project.org/lppl.txt
% and version 1.3 or later is part of all distributions of LaTeX
% version 2005/12/01 or later.
%
% This work has the LPPL maintenance status `maintained'.
%
% The Current Maintainer of this work is Niklas Beisert.
%
% This work consists of the files childdoc.dtx and childdoc.ins
% and the derived files childdoc.def and cdocsamp.tex with
% cdocsch1.tex, cdocsch2.tex, cdocsdrf.tex, cdocsfn1.tex, cdocsfn2.tex.
%
%<package>\ifdefined\childdocmain\endinput\fi
%<package>\ProvidesFile{childdoc.def}[2018/12/30 v2.0 child document driver]
%<samplemain>\ProvidesFile{cdocsamp.tex}[2018/12/30 v2.0 sample for childdoc]
%<*driver>
%\ProvidesFile{childdoc.drv}[2018/12/30 v2.0 childdoc reference manual file]
\PassOptionsToClass{10pt,a4paper}{article}
\documentclass{ltxdoc}

\usepackage[margin=35mm]{geometry}
\usepackage{hyperref}
\usepackage{hyperxmp}
\usepackage[usenames]{color}

\hypersetup{colorlinks=true}
\hypersetup{pdfstartview=FitH}
\hypersetup{pdfpagemode=UseNone}
\hypersetup{pdfsource={}}
\hypersetup{pdflang={en-UK}}
\hypersetup{pdfcopyright={Copyright 2017-2018 Niklas Beisert.
  This work may be distributed and/or modified under the
  conditions of the LaTeX Project Public License, either version 1.3
  of this license or (at your option) any later version.}}
\hypersetup{pdflicenseurl={http://www.latex-project.org/lppl.txt}}
\hypersetup{pdfcontactaddress={ETH Zurich, ITP, HIT K,
  Wolfgang-Pauli-Strasse 27}}
\hypersetup{pdfcontactpostcode={8093}}
\hypersetup{pdfcontactcity={Zurich}}
\hypersetup{pdfcontactcountry={Switzerland}}
\hypersetup{pdfcontactemail={nbeisert@itp.phys.ethz.ch}}
\hypersetup{pdfcontacturl={http://people.phys.ethz.ch/\xmptilde nbeisert/}}

\newcommand{\secref}[1]{\hyperref[#1]{section \ref*{#1}}}

\parskip1ex
\parindent0pt
\let\olditemize\itemize
\def\itemize{\olditemize\parskip0pt}

\begin{document}

\title{The \textsf{childdoc} Package}
\hypersetup{pdftitle={The childdoc Package}}
\author{Niklas Beisert\\[2ex]
  Institut f\"ur Theoretische Physik\\
  Eidgen\"ossische Technische Hochschule Z\"urich\\
  Wolfgang-Pauli-Strasse 27, 8093 Z\"urich, Switzerland\\[1ex]
  \href{mailto:nbeisert@itp.phys.ethz.ch}
  {\texttt{nbeisert@itp.phys.ethz.ch}}}
\hypersetup{pdfauthor={Niklas Beisert}}
\hypersetup{pdfsubject={Manual for the LaTeX2e Package childdoc}}
\date{30 December 2018, \textsf{v2.0}}
\maketitle

\begin{abstract}\noindent
\textsf{childdoc} is a \LaTeXe{} package
that enables the direct compilation
of document sections included by |\include|
to individual files.
\end{abstract}

\begingroup
\parskip0ex
\tableofcontents
\endgroup

%%%%%%%%%%%%%%%%%%%%%%%%%%%%%%%%%%%%%%%%%%%%%%%%%%%%%%%%%%%%%%%%%%%%%%%%%%%%%%%%
%%%%%%%%%%%%%%%%%%%%%%%%%%%%%%%%%%%%%%%%%%%%%%%%%%%%%%%%%%%%%%%%%%%%%%%%%%%%%%%%
\section{Introduction}

\LaTeX{} provides a mechanism to structure a large document (such as a book)
into a main file and several child files (containing the chapters)
using the |\include| command.
This mechanism is beneficial for documents
which span hundreds of pages in order to
make the source file(s) more manageable.
Moreover, compilation can be restricted to
selected child files by means of the |\includeonly| command.
The latter feature can be used to reduce the compilation time while editing
(this was significantly more useful in the earlier days of \LaTeX{})
or to generate a smaller document which is easier to navigate.
Another application of |\includeonly| is to generate
documents consisting of selected parts of the complete document.

However, there are a few drawbacks of the plain |\include| mechanism:
\begin{itemize}
\item
The child files cannot be compiled on their own,
they can only be compiled via the main file.
A naive editing environment
(such as a text editor with an option
to have the current file processed by \LaTeX)
may require one to switch to the main file before compiling;
attempting to compile the child file produces errors.
\item
The main file must be modified (each time)
to adjust the |\includeonly| command
to the present needs. This easily leaves the main file in a messy state.
\item
The generated document will always carry the filename
of the main document. This is inconvenient if
several child files are to be compiled and
to be kept for distribution.
\end{itemize}

The present package provides a simple interface
to make child files individually compilable by \LaTeX{}.
Compiling a child file then has the same effect as compiling
the main file with an |\includeonly| command
to select the appropriate child.
Moreover the generated document will carry the name of the child
rather than the main file.
This resolves all three above issues.

This feature is meant to make the editing of books,
thesis documents and lecture notes somewhat more convenient.
However, the package can also be used efficiently for
composing a series of documents (such as exercise sheets)
which are typically distributed individually.
It then assists the author in generating the individual documents
(potentially in different versions)
as well as a document containing the collected series.
Another application is in developing style files
or other kinds of included material
where compilation of the style file could redirect
to a sample or test file.

%%%%%%%%%%%%%%%%%%%%%%%%%%%%%%%%%%%%%%%%%%%%%%%%%%%%%%%%%%%%%%%%%%%%%%%%%%%%%%%%
%%%%%%%%%%%%%%%%%%%%%%%%%%%%%%%%%%%%%%%%%%%%%%%%%%%%%%%%%%%%%%%%%%%%%%%%%%%%%%%%
\section{Usage}

First of all, the package \textsf{childdoc} is \emph{not} a standard
\LaTeXe{} |.sty| style file! Therefore it needs to be invoked in
a non-standard way.

%%%%%%%%%%%%%%%%%%%%%%%%%%%%%%%%%%%%%%%%%%%%%%%%%%%%%%%%%%%%%%%%%%%%%%%%%%%%%%%%
\subsection{Included Files}
\label{sec:include}

%%%%%%%%%%%%%%%%%%%%%%%%%%%%%%%%%%%%%%%%
\DescribeMacro{\childdocmain}
To use the package, add the commands
\begin{center}
\begin{tabular}{l}
|\input{childdoc.def}|\\
|\childdocmain{}|\\
\end{tabular}
\end{center}
at the very top of the main \LaTeX{} file,
in particular \emph{before} the |\documentclass| statement!
The argument of |\childdocmain| should be left empty
(but it must be present).

%%%%%%%%%%%%%%%%%%%%%%%%%%%%%%%%%%%%%%%%
\DescribeMacro{\childdocof}
Furthermore, add the commands
\begin{center}
\begin{tabular}{l}
|\input{childdoc.def}|\\
|\childdocof{|\textit{main}|}|\\
\end{tabular}
\end{center}
at the top of every child file \textit{child}
which is included by |\include{|\textit{child}|}|
from within the main file
(or at least for those files to be compiled individually).
The argument \textit{main} must be the filename of the main file.

There are a couple of
considerations in setting up the main and child documents:

%%%%%%%%%%%%%%%%%%%%%%%%%%%%%%%%%%%%%%%%
\paragraph{Restrictions.}

Please note the following restrictions:
\begin{itemize}
\item
|\childdocmain| must be called with one argument \textit{main}
to ensure compatibility with earlier version of the package.
It must either be empty (|\childdocmain{}|)
or precisely match the filename of the main file in which it is specified.
See \secref{sec:detection} for further information.
\item
The filename \textit{main} must be specified without the |.tex| extension.
\item
The filename \textit{main} is case sensitive
(even in case-insensitive file systems)
due to internal string comparison.
\item
The argument \textit{main} should be fully expanded, it cannot be a macro.
\item
Subdirectories and special characters should be avoided in filenames.
\item
The command |\childdocmain{|\textit{main}|}| must be followed by a whitespace.
It should not be followed immediately by another command
or by a comment mark `|%|'.
This is because the \TeX{} parser reads the token immediately following
the argument of |\childdocmain| and puts it
at the beginning of every child section;
however, a white\-space is ignored.
\end{itemize}

%%%%%%%%%%%%%%%%%%%%%%%%%%%%%%%%%%%%%%%%
\paragraph{Content of Main File.}

It is advisable to place all content in the child files included by |\include|.
Any output contained in the main file will appear in all child documents
unless suppressed manually;
it cannot be suppressed automatically by the |\includeonly| directive
and thus should normally be avoided.
A method to include some content in the main file
by means of conditional processing is described in \secref{sec:conditional}.

%%%%%%%%%%%%%%%%%%%%%%%%%%%%%%%%%%%%%%%%
\paragraph{Page Numbering.}

When only a part of the document is compiled,
the appropriate numbering of pages
(as well as other status parameters)
is determined from the |.aux| files.
The latter contain information from previous passes.
However this information needs to propagate through
all intermediate child documents.
Therefore the page numbering in child documents may well
be inconsistent until the complete document is compiled at least once.

A useful (if unconventional) way to always ensure a consistent
page numbering is to restart the numbering in each child document
and denote the pages by `\textit{child}|.|\textit{page}'
where \textit{child} represents the chapter/section number of the child file.
This can be achieved by the command
|\numberwithin{page}{|\textit{child}|}|
of the \textsf{amsmath} package
where \textit{child} can be |chapter| or |section|
depending on the chosen structuring.
Alternatively, one can modify the macro |\thepage| appropriately
and reset the counter |page| at the start of each child file.

%%%%%%%%%%%%%%%%%%%%%%%%%%%%%%%%%%%%%%%%%%%%%%%%%%%%%%%%%%%%%%%%%%%%%%%%%%%%%%%%
\subsection{Conditional Processing}
\label{sec:conditional}

The package provides a mechanism to compile different versions
of a document. To customise the versions further some conditional processing
can come in handy to distinguish which version is being compiled.
The package provides two macros to describe the compilation context:

%%%%%%%%%%%%%%%%%%%%%%%%%%%%%%%%%%%%%%%%
\DescribeMacro{\ifchilddoc}
The conditional |\ifchilddoc| distinguishes between the compilation of
child documents and the main document:
%
\begin{center}
|\ifchilddoc |\textit{child-code}| |[|\||else |\textit{main-code}]| \||fi|
\end{center}

%%%%%%%%%%%%%%%%%%%%%%%%%%%%%%%%%%%%%%%%
\DescribeMacro{\childdocname}
\DescribeMacro{\childdocjob}
The macro |\childdocname| contains the filename (without extension)
of the main or child file being processed.
Note that |\childdocjob| will always contain the name of the main file.

%%%%%%%%%%%%%%%%%%%%%%%%%%%%%%%%%%%%%%%%
\paragraph{Title Page.}

Conditional processing can be used to include a title or banner page
in the main document when proper precautions are taken.
Importantly, the code in the main file should ensure that the page counter
(as well as other status parameters which are stored in the |.aux| files)
takes the same value after the conditional processing.
Otherwise the page numbers may take divergent values
depending on which part is compiled.

For example, a title page could be declared by:
%
\begin{center}
\begin{tabular}{l}
|\ifchilddoc\||else|\\
|\addtocounter{page}{-1}|\\
\textit{code for title page}\\
|\newpage|\\
|\||fi|
\end{tabular}
\end{center}
%
A banner page for the child documents can be generated by:
%
\begin{center}
\begin{tabular}{l}
|\ifchilddoc|\\
|\addtocounter{page}{-1}|\\
\textit{code for banner page}\\
|\newpage|\\
|\||fi|
\end{tabular}
\end{center}
%
Here one could write a message such as:
\begin{center}
|This is the part \childdocname{} of \childdocjob{}.|
\end{center}

%%%%%%%%%%%%%%%%%%%%%%%%%%%%%%%%%%%%%%%%%%%%%%%%%%%%%%%%%%%%%%%%%%%%%%%%%%%%%%%%
\subsection{Flags}
\label{sec:flags}

The package makes it easy to generate different versions
of the main or child documents.
To this end compilation flags can be defined
and assigned different default values.
They will be particularly useful in conjunction
with the forwarding mechanism described in \secref{sec:forward}.

For example, it may be useful to have a flag |\version|
which can be set to |draft| or |final|.
The document source will contain some conditional code
depending on the value of |\version|.
Suppose further, the flag should default to |final| for the main file
and to |draft| for child files
which is a natural assignment for editing the document.
This is achieved by placing the following code
in the preamble of the main document
(below the |\childdocmain| directive):
%
\begin{center}
\begin{tabular}{l}
|\ifchilddoc|\\
|\providecommand{\version}{draft}|\\
|\||else|\\
|\providecommand{\version}{final}|\\
|\||fi|
\end{tabular}
\end{center}
%
The definition by |\providecommand| makes sure
that previous definitions are not overwritten.
Further statements |\providecommand{\version}{...}|
can thus be added before the above code to override it.

For the main file, one might add a line
(between |\childdocmain| and the above block)
%
\begin{center}
|%\ifchilddoc\||else\providecommand{\version}{draft}\||fi|
\end{center}
%
which can be uncommented to produce a draft version.
Likewise one can add a line to the very top of a child file
(above the |\childdocof{|\textit{main}|}| directive)
%
\begin{center}
|%\providecommand{\version}{final}|
\end{center}
%
which can be uncommented to produce the final version of this child document.

%%%%%%%%%%%%%%%%%%%%%%%%%%%%%%%%%%%%%%%%%%%%%%%%%%%%%%%%%%%%%%%%%%%%%%%%%%%%%%%%
\subsection{Forwarding}
\label{sec:forward}

Different versions of the main or child documents
using compilation flags as described in \secref{sec:flags}
can be (permanently) stored in different files
for convenient compilation, viewing and distribution.
To this end, the package defines a command
to pass on compilation to a different file:

%%%%%%%%%%%%%%%%%%%%%%%%%%%%%%%%%%%%%%%%
\DescribeMacro{\childdocforward}
The command |\childdocforward| redirects processing to
another source file:
%
\begin{center}
\begin{tabular}{l}
|\input{childdoc.def}|\\
|\childdocforward[|\textit{main}|]{|\textit{dest}|}|\\
\end{tabular}
\end{center}
%
The argument \textit{dest} is the destination file
(without extension).
It should be the main file or one of the child files.
Note that further \textsf{childdoc} directives
such as |\childdocof| and |\childdocforward|
in the indicated file will be processed in this form.
The optional argument \textit{main}
passes on directly to the main file \textit{main}
while pretending to compile the child \textit{dest}.
This form behaves as if \textit{dest}
issues |\childdocof{|\textit{main}|}| right away,
and no further \textsf{childdoc} directives will be processed.

%%%%%%%%%%%%%%%%%%%%%%%%%%%%%%%%%%%%%%%%
\DescribeMacro{\...prefix}
In the alternative form |\childdocforwardprefix|,
%
\begin{center}
\begin{tabular}{l}
|\input{childdoc.def}|\\
|\childdocforwardprefix[|\textit{main}|]{|\textit{prefix}|}{|\textit{dest}|}|
\end{tabular}
\end{center}
%
the destination file is determined by a pattern
depending on the current file:
To make this work, the current file must be called
`{\textit{prefix}\hspace{0.2em}\textit{suffix}}'
with \textit{prefix} matching precisely the argument.
Processing is then passed on to the file
`{\textit{dest}\hspace{0.2em}\textit{suffix}}'.
Surely, the same effect is achieved by
directly specifying the
argument `{\textit{dest}\hspace{0.2em}\textit{suffix}}'
in the first form.
However, that requires to set up a different file
for each child. With the alternative form of the command
all these files can have exactly the same content
which simplifies setting them up and maintaining them.

For example, the following file |draft.tex|
with a compilation flag |\version| as described in \secref{sec:flags}
compiles the main document as a draft:
%
\begin{center}
\begin{tabular}{l}
|\def\version{draft}|\\
|\input{childdoc.def}|\\
|\childdocforward{|\textit{main}|}|
\end{tabular}
\end{center}
%
Likewise, the following files |final|\textit{nn}|.tex|
compile the final version of the child document
|child|\textit{nn}|.tex|:
%
\begin{center}
\begin{tabular}{l}
|\def\version{final}|\\
|\input{childdoc.def}|\\
|\childdocforwardprefix{final}{child}|
\end{tabular}
\end{center}
%

Note that when several versions of a main file and/or of each child file
are to be generated, it may be convenient to set up a |Makefile| or
shell script to automatise the process.

%%%%%%%%%%%%%%%%%%%%%%%%%%%%%%%%%%%%%%%%%%%%%%%%%%%%%%%%%%%%%%%%%%%%%%%%%%%%%%%%
\subsection{Command Line Processing}
\label{sec:commandline}

The effect of redirection files can also be achieved by invoking
the \LaTeX{} compiler with a more elaborate command line.
Most conveniently this should be done as part
of a shell script or a |Makefile|.

When using \textsf{childdoc} in the main file, the following
command lines effectively perform a redirection
(note that depending on the shell being used,
backslashes may have to be doubled: `|\|' $\to$ `|\\|'):
%
\begin{center}
|... -jobname "|\textit{target}|" |\\|"|[\textit{flags}]%
|\input{childdoc.def}\childdocforward[|\textit{main}|]{|\textit{dest}|}"|
\end{center}
%
Here \textit{target} is the name of the output file,
\textit{main} is the name of the main file
and \textit{dest} is the name of the main or child file to be processed
(all filenames without extensions).
The optional argument \textit{main} can be omitted
if \textit{main} matches \textit{dest}.
Optionally, compilation \textit{flags} can be defined via |\def| commands.
This command line makes the \TeX{} engine believe
it is compiling the file \textit{target}
whose content is specified as the latter parameter.
The provided code then forwards the processing to
\textit{main} or \textit{dest} as described in \secref{sec:forward}.

%%%%%%%%%%%%%%%%%%%%%%%%%%%%%%%%%%%%%%%%%%%%%%%%%%%%%%%%%%%%%%%%%%%%%%%%%%%%%%%%
\subsection{Include by Input}
\label{sec:input}

Including child documents by |\include| has some restrictions by design.
Most notably, the content of a child document always occupies
its own set of pages; pages cannot be shared between child documents.
Usually, this behaviour makes perfect sense
because each child document contain an essential part of the document.
However, in some situations it may be desirable to compose
a document from a collection of parts
without having mandatory page breaks between then.
For this case, the package
provides a mechanism to include parts
by |\input| which can also be processed individually.
However, by construction this mechanism
requires manual handling of the content to be output.

%%%%%%%%%%%%%%%%%%%%%%%%%%%%%%%%%%%%%%%%
\DescribeMacro{\ifchilddocmanual}
The main file should be prepared as usual, see \secref{sec:include}.
However, the document body must make a distinction
between processing of an individual part and of the main document, e.g.:
%
\begin{center}
\begin{tabular}{l}
|\ifchilddocmanual|\\
|\input{\childdocname}|\\
|\||else|\\
\textit{document body with }|\input{|\textit{part}|}|\\
|\||fi|
\end{tabular}
\end{center}
%
The conditional |\ifchilddocmanual| is true whenever
a part to be included by |\input| is being compiled,
and the name of the part is stored in |\childdocname|.

%%%%%%%%%%%%%%%%%%%%%%%%%%%%%%%%%%%%%%%%
\DescribeMacro{\childdocby}
Each part to be included by |\input| should start with:
%
\begin{center}
\begin{tabular}{l}
|\input{childdoc.def}|\\
|\childdocby{|\textit{main}|}|\\
\end{tabular}
\end{center}
%
The directive |\childdocby| is similar to |\childdocof|
described in \secref{sec:include},
but the subsequent selection of content must be done manually.
To that end, both |\ifchilddoc| and |\ifchilddocmanual|
will be true upon processing of a part,
and the name of the part is stored in |\childdocname|.
Note that |\jobname| will be set to the filename of the current part
so that each part receives an individual |.aux| file
that does not interfere with the |.aux| file(s) of the main document.
This behaviour can be altered by the alternative form
|\childdocby[*]{|\textit{main}|}| (with a non-empty optional argument)
which uses the |.aux| file of the main document
by setting |\jobname| to \textit{main}.

%%%%%%%%%%%%%%%%%%%%%%%%%%%%%%%%%%%%%%%%%%%%%%%%%%%%%%%%%%%%%%%%%%%%%%%%%%%%%%%%
\subsection{Driver Development}
\label{sec:driver}

The \textsf{childdoc} mechanism can also be use for the development
of definition files such as \LaTeX{} styles or classes.
This case differs from the above setup with multiple parts
included by |\include| in that no |\includeonly| should be invoked.
This can be achieved by starting the include file
(before |\ProvidesPackage|) with:
%
\begin{center}
\begin{tabular}{l}
|\input{childdoc.def}|\\
|\childdocforward{|\textit{main}|}|\\
\end{tabular}
\end{center}
%
or alternatively with:
%
\begin{center}
\begin{tabular}{l}
|\input{childdoc.def}|\\
|\childdocby{|\textit{main}|}|\\
\end{tabular}
\end{center}
%
Both forms have slightly different effects as described above.
The main file is prepared as usual, see \secref{sec:include}.

%%%%%%%%%%%%%%%%%%%%%%%%%%%%%%%%%%%%%%%%%%%%%%%%%%%%%%%%%%%%%%%%%%%%%%%%%%%%%%%%
\subsection{Legacy Detection}
\label{sec:detection}

The directive |\childdocmain| in the main file can detect
whether the complete document or merely a child is to be compiled
even without using the directive |\childdocof|.
This method is deprecated because it is less robust
and there is no compelling reason to use it;
it is merely provided for backward compatibility
and it may be removed in future versions.

If the detection mechanism is to be used,
it is mandatory to correctly specify
the filename of the main file as the argument of |\childdocmain|:
%
\begin{center}
\begin{tabular}{l}
|\input{childdoc.def}|\\
|\childdocmain{|\textit{main}|}|\\
\end{tabular}
\end{center}
%
If |\jobname| does not match the argument \textit{main} of |\childdocmain|,
it is assumed that |\jobname| points to the child file to be compiled.
When using |\childdocmain| with the main file specified as argument,
it suffices to start a child file
with just |\input{|\textit{main}|}|
without loading of the package and using |\childdocof|.
If instead all processing is done
with the appropriate \textsf{childdoc} directives,
the argument of \textit{main} of |\childdocmain| can be empty.

An alternative version of the command line processing described
in \secref{sec:commandline} using the detection mechanism reads:
%
\begin{center}
|... -jobname "|\textit{target}|" "|[\textit{flags}]%
[|\def\jobname{|\textit{dest}|}|]|\input{|\textit{main}|}"|
\end{center}

%%%%%%%%%%%%%%%%%%%%%%%%%%%%%%%%%%%%%%%%%%%%%%%%%%%%%%%%%%%%%%%%%%%%%%%%%%%%%%%%
\subsection{Manual Code}
\label{sec:manual}

In case one cannot be certain whether the definitions file |childdoc.def|
is installed on the target \TeX{} distribution
and one prefers not to ship it,
it is conceivable to paste a few relevant commands into the sources.

To that end, drop all statements |\input{childdoc.def}|
and perform the replacements as outlined below.
Instead of |\childdocmain{|\textit{main}|}| add the following code
to the top of the main file:
%
\begin{center}
\begin{tabular}{l}
|\||ifdefined\childdocname\endinput\||fi\newif\ifchilddoc|\\
|\edef\childdocname{\scantokens\expandafter{\jobname\noexpand}}|\\
|\def\childdocmain{|\textit{main}|}\||ifx\childdocmain\childdocname\||else|\\
|\childdoctrue\includeonly{\childdocname}\let\jobname\childdocmain\||fi|\\
\end{tabular}
\end{center}
%
Instead of |\childdocof{|\textit{main}|}| just include the main file
at the top of each child file:
%
\begin{center}
|\input{|\textit{main}|}|
\end{center}
%
A simple redirection |\childdocforward{|\textit{dest}|}| is achieved by:
%
\begin{center}
|\def\jobname{|\textit{dest}|}\input{\jobname}|
\end{center}
%
The redirection with prefix
|\childdocforwardprefix[|\textit{prefix}|]{|\textit{dest}|}|
is accomplished by:
%
\begin{center}
\begin{tabular}{l}
|{\edef\jobname{\scantokens\expandafter{\jobname\noexpand}}|\\
|\def\redirectjob |\textit{prefix}|#1~~~{\gdef\jobname{|\textit{dest}|#1}}|\\
|\expandafter\redirectjob\jobname~~~}\input{\jobname}|
\end{tabular}
\end{center}

In an alternative approach,
child documents can be compiled by a specific command line
without additional code or specific definitions:
%
\begin{center}
|... -jobname "|\textit{target}|" "|[\textit{flags}]%
|\includeonly{|\textit{dest}|}\input{|\textit{main}|}"|
\end{center}
%

%%%%%%%%%%%%%%%%%%%%%%%%%%%%%%%%%%%%%%%%%%%%%%%%%%%%%%%%%%%%%%%%%%%%%%%%%%%%%%%%
%%%%%%%%%%%%%%%%%%%%%%%%%%%%%%%%%%%%%%%%%%%%%%%%%%%%%%%%%%%%%%%%%%%%%%%%%%%%%%%%
\section{Information}

%%%%%%%%%%%%%%%%%%%%%%%%%%%%%%%%%%%%%%%%%%%%%%%%%%%%%%%%%%%%%%%%%%%%%%%%%%%%%%%%
\subsection{Copyright}

Copyright \copyright{} 2017--2018 Niklas Beisert

This work may be distributed and/or modified under the
conditions of the \LaTeX{} Project Public License, either version 1.3
of this license or (at your option) any later version.
The latest version of this license is in
  \url{http://www.latex-project.org/lppl.txt}
and version 1.3 or later is part of all distributions of \LaTeX{}
version 2005/12/01 or later.

This work has the LPPL maintenance status `maintained'.

The Current Maintainer of this work is Niklas Beisert.

This work consists of the files |README.txt|, |childdoc.ins| and |childdoc.dtx|
as well as the derived files |childdoc.def|, |cdocsamp.tex|
with |cdocsch1.tex|, |cdocsch2.tex|, |cdocspt3.tex|, |cdocspt4.tex|,
|cdocsdrf.tex|, |cdocsfn1.tex|, |cdocsfn2.tex|
as well as |childdoc.pdf|.

%%%%%%%%%%%%%%%%%%%%%%%%%%%%%%%%%%%%%%%%%%%%%%%%%%%%%%%%%%%%%%%%%%%%%%%%%%%%%%%%
\subsection{Files and Installation}

The package consists of the files:
%
\begin{center}
\begin{tabular}{ll}
    |README.txt|   & readme file \\
    |childdoc.ins| & installation file \\
    |childdoc.dtx| & source file \\
    |childdoc.def| & definition file \\
    |cdocsamp.tex| & sample main file \\
    |cdocsch1.tex| & sample include file \\
    |cdocsch2.tex| & sample include file \\
    |cdocspt3.tex| & sample part file \\
    |cdocspt4.tex| & sample part file \\
    |cdocsdrf.tex| & sample redirection file \\
    |cdocsfn1.tex| & sample redirection file \\
    |cdocsfn2.tex| & sample redirection file \\
    |childdoc.pdf| & manual
\end{tabular}
\end{center}
%
The distribution consists of the files
|README.txt|, |childdoc.ins| and |childdoc.dtx|.
%
\begin{itemize}
\item
Run (pdf)\LaTeX{} on |childdoc.dtx|
to compile the manual |childdoc.pdf| (this file).
\item
Run \LaTeX{} on |childdoc.ins| to create the definitions file |childdoc.def|
and the sample |cdocsamp.tex| with include files
|cdocsch1.tex|, |cdocsch2.tex|, |cdocspt3.tex|, |cdocspt4.tex|,
|cdocsdrf.tex|, |cdocsfn1.tex|, |cdocsfn2.tex|.
Then copy the file |childdoc.def| to an appropriate directory of your \LaTeX{}
distribution, e.g.\ \textit{texmf-root}|/tex/latex/childdoc|.
\end{itemize}

%%%%%%%%%%%%%%%%%%%%%%%%%%%%%%%%%%%%%%%%%%%%%%%%%%%%%%%%%%%%%%%%%%%%%%%%%%%%%%%%
\subsection{Related CTAN Packages}

There are several other packages which offer a similar functionality:
%
\begin{itemize}
\item
The packages
\href{http://ctan.org/pkg/docmute}{\textsf{docmute}},
\href{http://ctan.org/pkg/includex}{\textsf{includex}} and
\href{http://ctan.org/pkg/standalone}{\textsf{standalone}}
provide commands to include only the document body of
a child file thus allowing both files to be compiled individually.
\item
The packages \href{http://ctan.org/pkg/subdocs}{\textsf{subdocs}}
and \href{http://ctan.org/pkg/subfiles}{\textsf{subfiles}}
provide structures in which the main and child documents can be
encapsulated and allowing them to be compiled individually.
The inclusion mechanism is different from the conventional |\include|.
\item
The package \href{http://ctan.org/pkg/combine}{\textsf{combine}}
is an elaborate solution to combine several documents into one.
\end{itemize}
%
See also the CTAN topic \href{http://ctan.org/topic/subdocs}{\textsf{subdocs}}
for further related packages.
The present package differs from the above solutions in that
a document structure constructed with the conventional |\include| mechanism
just needs two extra commands at the top of every file
such that all constituent files can be compiled individually.

%%%%%%%%%%%%%%%%%%%%%%%%%%%%%%%%%%%%%%%%%%%%%%%%%%%%%%%%%%%%%%%%%%%%%%%%%%%%%%%%
%\subsection{Feature Suggestions}
%
%The following is a list of features which may be useful for future
%versions of this package:
%%
%\begin{itemize}
%\item
%\ldots
%\end{itemize}

%%%%%%%%%%%%%%%%%%%%%%%%%%%%%%%%%%%%%%%%%%%%%%%%%%%%%%%%%%%%%%%%%%%%%%%%%%%%%%%%
\subsection{Revision History}

%%%%%%%%%%%%%%%%%%%%%%%%%%%%%%%%%%%%%%%%
\paragraph{v2.0:} 2018/12/30

\begin{itemize}
\item
immediate forward processing
\item
added |\childdocby| mechanism
\item
manual restructured
\end{itemize}

%%%%%%%%%%%%%%%%%%%%%%%%%%%%%%%%%%%%%%%%
\paragraph{v1.6:} 2018/01/17

\begin{itemize}
\item
application for development of include files
\item
corrections to manual
\end{itemize}

%%%%%%%%%%%%%%%%%%%%%%%%%%%%%%%%%%%%%%%%
\paragraph{v1.5:} 2017/05/21

\begin{itemize}
\item
more complete structuring introduced
\item
|\childdocof| introduced
\item
|\childdoc| renamed to |\childdocmain|
\item
|\childredirect| renamed to |\childdocforward| and |\childdocforwardprefix|
and functionality expanded
\end{itemize}

%%%%%%%%%%%%%%%%%%%%%%%%%%%%%%%%%%%%%%%%
\paragraph{v1.0:} 2017/04/27

\begin{itemize}
\item
manual and install package
\item
first version published on CTAN
\end{itemize}

%%%%%%%%%%%%%%%%%%%%%%%%%%%%%%%%%%%%%%%%
\paragraph{v0.6:} 2017/04/26

\begin{itemize}
\item
redirection mechanism added
\end{itemize}

%%%%%%%%%%%%%%%%%%%%%%%%%%%%%%%%%%%%%%%%
\paragraph{v0.5:} 2017/04/26

\begin{itemize}
\item
functionality in definition file
\end{itemize}


%%%%%%%%%%%%%%%%%%%%%%%%%%%%%%%%%%%%%%%%%%%%%%%%%%%%%%%%%%%%%%%%%%%%%%%%%%%%%%%%
%%%%%%%%%%%%%%%%%%%%%%%%%%%%%%%%%%%%%%%%%%%%%%%%%%%%%%%%%%%%%%%%%%%%%%%%%%%%%%%%
%%%%%%%%%%%%%%%%%%%%%%%%%%%%%%%%%%%%%%%%%%%%%%%%%%%%%%%%%%%%%%%%%%%%%%%%%%%%%%%%
\appendix

\settowidth\MacroIndent{\rmfamily\scriptsize 000\ }

 \DocInput{childdoc.dtx}

\end{document}
%</driver>
% \fi
%
% %%%%%%%%%%%%%%%%%%%%%%%%%%%%%%%%%%%%%%%%%%%%%%%%%%%%%%%%%%%%%%%%%%%%%%%%%%%%%%
% %%%%%%%%%%%%%%%%%%%%%%%%%%%%%%%%%%%%%%%%%%%%%%%%%%%%%%%%%%%%%%%%%%%%%%%%%%%%%%
% \section{Sample}
%\iffalse
%<*samplemain>
%\fi
%
% The following presents a sample document
% with two chapters, two parts, a title page,
% a compile flag as well as three forwarding files to set the flag.
% It consists of eight |.tex| files:
% \begin{center}
% \begin{tabular}{ll}
% |cdocsamp.tex|&main file\\
% |cdocsch1.tex|&include file for chapter 1\\
% |cdocsch2.tex|&include file for chapter 2\\
% |cdocspt3.tex|&include file for part 3\\
% |cdocspt4.tex|&include file for part 4\\
% |cdocsdrf.tex|&forwarding file for main file in draft mode\\
% |cdocsfi1.tex|&forwarding file for final version of chapter 1\\
% |cdocsfi2.tex|&forwarding file for final version of chapter 2\\
% \end{tabular}
% \end{center}
% Each of the eight files can be compiled directly by the \LaTeX{} compiler.
%
% %%%%%%%%%%%%%%%%%%%%%%%%%%%%%%%%%%%%%%
% \paragraph{Main File.}
%
% The main file is called |cdocsamp.tex|.
%
% Load the \textsf{childdoc} definitions and
% declare the filename for the main document:
%    \begin{macrocode}
\input{childdoc.def}
\childdocmain{}
%    \end{macrocode}

% Optional override for |\version| flag:
%    \begin{macrocode}
%%\ifchilddoc\else\providecommand{\version}{draft}\fi
%    \end{macrocode}

% Define the default values for the |\version| flag
% (|final| for the main file and |draft| for childs):
%    \begin{macrocode}
\ifchilddoc
\providecommand{\version}{draft}
\else
\providecommand{\version}{final}
\fi
%    \end{macrocode}

% Load the standard document class:
%    \begin{macrocode}
\documentclass[12pt]{article}
%    \end{macrocode}

% Start the document body:
%    \begin{macrocode}
\begin{document}
%    \end{macrocode}

% Declare a title page.
% Print title, part of document being processed and version flag:
%    \begin{macrocode}
\addtocounter{page}{-1}
\begin{center}
{\LARGE\bfseries{}childdoc example\par}
\vspace{1cm}
\ifchilddoc
\ifchilddocmanual part\else chapter\fi:
`\childdocname' of `\childdocjob'\par
\else
main document: `\childdocjob'\par
\fi
version: \version\par
\end{center}
\newpage
%    \end{macrocode}

% Manually include selected file,
% otherwise process as usual:
%    \begin{macrocode}
\ifchilddocmanual
\section*{part `\childdocname'}
\input{\childdocname}
\else
%    \end{macrocode}

% Include the two chapters:
%    \begin{macrocode}
\include{cdocsch1}
\include{cdocsch2}
%    \end{macrocode}

% Include the two parts unless only chapters should be displayed:
%    \begin{macrocode}
\ifchilddoc\else
\section{part three}
\input{cdocspt3}
\section{part four}
\input{cdocspt4}
\fi
%    \end{macrocode}

% Process as usual until here:
%    \begin{macrocode}
\fi
%    \end{macrocode}

% End of document body:
%    \begin{macrocode}
\end{document}
%    \end{macrocode}
%\iffalse
%</samplemain>
%\fi
%
% %%%%%%%%%%%%%%%%%%%%%%%%%%%%%%%%%%%%%%
% \paragraph{Chapter Include Files.}
%
% The include files are called |cdocsch1.tex| and |cdocsch2.tex|.
%
%\iffalse
%<*samplechap1|samplechap2>
%\fi

% Optional override for |\version| flag:
%    \begin{macrocode}
%%\providecommand{\version}{final}
%    \end{macrocode}

% Include the main document:
%    \begin{macrocode}
\input{childdoc.def}
\childdocof{cdocsamp}
%    \end{macrocode}

%\iffalse
%</samplechap1|samplechap2>
%\fi
%
%\iffalse
%<*samplechap1>
%\fi
% Some text for chapter 1:
%    \begin{macrocode}
\section{one}
some text in chapter one
%    \end{macrocode}

%\iffalse
%</samplechap1>
%\fi
% Some text for chapter 2:
%\iffalse
%<*samplechap2>
%\fi
%    \begin{macrocode}
\section{two}
more text in chapter two
%    \end{macrocode}

%\iffalse
%</samplechap2>
%\fi
%
% %%%%%%%%%%%%%%%%%%%%%%%%%%%%%%%%%%%%%%
% \paragraph{Part Include Files.}
%
% The include files are called |cdocspt3.tex| and |cdocspt4.tex|.
%
%\iffalse
%<*samplepart3|samplepart4>
%\fi

% Optional override for |\version| flag:
%    \begin{macrocode}
%%\providecommand{\version}{final}
%    \end{macrocode}

% Include the main document:
%    \begin{macrocode}
\input{childdoc.def}
\childdocby{cdocsamp}
%    \end{macrocode}

%\iffalse
%</samplepart3|samplepart4>
%\fi
%
%\iffalse
%<*samplepart3>
%\fi
% Some text for part 3:
%    \begin{macrocode}
some text in part three
%    \end{macrocode}

%\iffalse
%</samplepart3>
%\fi
% Some text for part 4:
%\iffalse
%<*samplepart4>
%\fi
%    \begin{macrocode}
more text in part four
%    \end{macrocode}

%\iffalse
%</samplepart4>
%\fi
%
% %%%%%%%%%%%%%%%%%%%%%%%%%%%%%%%%%%%%%%
% \paragraph{Forwarding for a Complete Draft.}
%
% The following forwarding file |cdocsdrf.tex|
% compiles the main document in draft mode:
%\iffalse
%<*sampledraft>
%\fi
%    \begin{macrocode}
\def\version{draft}
\input{childdoc.def}
\childdocforward{cdocsamp}
%    \end{macrocode}

%\iffalse
%</sampledraft>
%\fi
%
% %%%%%%%%%%%%%%%%%%%%%%%%%%%%%%%%%%%%%%
% \paragraph{Forwarding for Final Version of the Chapters.}
%
% The following forwarding files |cdocsfn1.tex| and |cdocsfn2.tex|
% (with identical content)
% compile the final versions of the child documents
% |cdocsch1.tex| and |cdocsch2.tex|, respectively:
%\iffalse
%<*samplefinal>
%\fi
%    \begin{macrocode}
\def\version{final}
\input{childdoc.def}
\childdocforwardprefix[cdocsamp]{cdocsfn}{cdocsch}
%    \end{macrocode}

%\iffalse
%</samplefinal>
%\fi
%
% %%%%%%%%%%%%%%%%%%%%%%%%%%%%%%%%%%%%%%
% \paragraph{Command Line Processing.}
%
% The following three command lines generate the output files
% |cdocscld|, |cdocscl1| and |cdocscl2|
% which should be identical to
% |cdocsdrf|, |cdocsch1| and |cdocsfn2|, respectively:
% \begin{center}
% \begin{tabular}{l}
% |latex -jobname cdocscld \|\\
% |  "\def\version{draft}\input{childdoc.def}\childdocforward{cdocsamp}"|\\
% |latex -jobname cdocscl1 \|\\
% |  "\input{childdoc.def}\childdocforward[cdocsamp]{cdocsch1}"|\\
% |latex -jobname cdocscl2 \|\\
% |  "\def\version{final}\input{childdoc.def}\childdocforward{cdocsch2}"|
% \end{tabular}
% \end{center}
% Note that the trailing backslash on each first line
% merely continues the input to the second line
% (for convenient cut ant paste).
% Furthermore, the command |latex| can be replaced by any
% of its alternative versions such as |pdflatex|.
%
% %%%%%%%%%%%%%%%%%%%%%%%%%%%%%%%%%%%%%%%%%%%%%%%%%%%%%%%%%%%%%%%%%%%%%%%%%%%%%%
% %%%%%%%%%%%%%%%%%%%%%%%%%%%%%%%%%%%%%%%%%%%%%%%%%%%%%%%%%%%%%%%%%%%%%%%%%%%%%%
% \section{Implementation}
%\iffalse
%<*package>
%\fi
%
% This section describes the definitions file |childdoc.def|.

% The definitions cannot be loaded using |\usepackage| or |\RequirePackage|
% which has a mechanism to prevent loading a style file more than once.
% When loading the definitions by means of |\input|
% multiple instances have to be prevented manually:
%\iffalse
%This code needs to be before the `\ProvidesFile' directive
%which is defined at the beginning of this file.
%Therefore it is also placed there and commented out here.
%</package>
%<*discard>
%\fi
%    \begin{macrocode}
\ifdefined\childdocmain\endinput\fi
%    \end{macrocode}
%\iffalse
%</discard>
%<*package>
%\fi
%
% \macro{\ifchilddoc}
% \macro{\ifchilddocmanual}
% The conditional |\ifchilddoc| tells whether a
% child (true) or main (false) document is being compiled.
% The conditional |\ifchilddocmanual| tells whether
% the |\includeonly| mechanism is used (false) or
% the selection of child files must be performed manually (true).
% The definitions initialise to false:
%    \begin{macrocode}
\newif\ifchilddoc
\newif\ifchilddocmanual
%    \end{macrocode}

% \macro{\childdocname}
% \macro{\childdocjob}
% The macro |\childdocname| stores the name of the main document
% to be compiled. The macro |\childdocjob| stores the name of
% the document on which the \LaTeX{} compiler was originally invoked.
% The content of |\jobname| cannot be compared
% to filenames specified in the source due to different catcodes.
% The following code rescans |\jobname|, stores the result
% in |\childdocname| and saves a copy in |\childdocjob|:
%    \begin{macrocode}
\edef\childdocname{\scantokens\expandafter{\jobname\noexpand}}
\let\childdocjob\childdocname
%    \end{macrocode}

% \macro{\childdocdisable}
% The macro |\childdocdisable| prevents the main file
% from being processed more than once.
% At this stage, the main document command |\childdocmain|
% is assumed to be called once again where it should do nothing.
% Any subsequent call to it should prevent
% a secondary processing of the main document
% It overwrites the forwarding commands
% |\childdocof| and |\childdocforward|
% with empty macros to prevent further inclusions of the main document:
%    \begin{macrocode}
\newcommand{\childdocdisable}
{
  \renewcommand{\childdocmain}[1]{\renewcommand{\childdocmain}[1]{\endinput}}
  \renewcommand{\childdocof}[1]{}
  \renewcommand{\childdocby}[2][]{}
  \renewcommand{\childdocforward}[2][]{}
  \renewcommand{\childdocdisable}{}
}
%    \end{macrocode}

% \macro{\childdocmain}
% The macro |\childdocmain| is to be called at the top of the main file
% with nothing or the main filename (without extension) as argument.
% First, it breaks loops.
% If the argument is not empty and does not match |\childdocname|
% (which is set by the first inclusion of |childdoc.def|),
% |\ifchilddoc| is set to true, |\includeonly| is applied to the child file
% and |\jobname| is set to the main file
% (for proper handling of |.aux| files):
%    \begin{macrocode}
\newcommand{\childdocmain}[1]
{
  \childdocdisable\childdocmain{}
  \if?#1?\else
    \begingroup
      \def\childdoctmp{#1}
      \ifx\childdoctmp\childdocname
        \def\childdoctmp{}
      \else
        \def\childdoctmp
        {
          \childdoctrue
          \includeonly{\childdocname}
          \def\childdocjob{#1}
          \def\jobname{#1}
        }
      \fi
      \expandafter
    \endgroup
    \childdoctmp
  \fi
}
%    \end{macrocode}

% \macro{\childdocof}
% The command |\childdocof| redirects
% compilation to the main file |#1|.
%    \begin{macrocode}
\newcommand{\childdocof}[1]
{
  \childdocdisable
  \childdoctrue
  \includeonly{\childdocname}
  \def\jobname{#1}
  \def\childdocjob{#1}
  \input{#1}
}
%    \end{macrocode}

% \macro{\childdocby}
% The command |\childdocby| ....
%    \begin{macrocode}
\newcommand{\childdocby}[2][]
{
  \childdocdisable
  \childdoctrue
  \childdocmanualtrue
  \if?#1?\else
    \def\jobname{#2}
  \fi
  \def\childdocjob{#2}
  \input{#2}
  \endinput
}
%    \end{macrocode}

% \macro{\childdocforward}
% The command |\childdocforward| redirects
% compilation to the main file or
% (if the optional argument is given) a child file.
% Parameters are set as if the main file
% or a child file starting with |\childdocof| was compiled.
% Then compilation is handed over to the main file:
%    \begin{macrocode}
\newcommand{\childdocforward}[2][]
{
  \begingroup
    \if?#1?
      \def\childdoctmp
      {
        \def\childdocname{#2}
        \def\childdocjob{#2}
        \def\jobname{#2}
        \input{#2}
        \endinput
      }
    \else
      \def\childdoctmp
      {
        \childdocdisable
        \def\childdocname{#2}
        \childdoctrue
        \includeonly{#2}
        \def\childdocjob{#1}
        \def\jobname{#1}
        \input{#1}
        \endinput
      }
    \fi
    \expandafter
  \endgroup
  \childdoctmp
}
%    \end{macrocode}

% \macro{\childdocforwardprefix}
% The command |\childdocforwardprefix| redirects
% compilation to the main or a child file by means of a pattern.
% The prefix |#1| in the current filename is replaced by |#2|
% and the suffix of the current filename is kept
% (it is assumed that the filename does not contain the substring `|~~~|'
% which is used as a delimiter).
% Compilation is handed over to the new file by |\childdocforward|:
%    \begin{macrocode}
\newcommand{\childdocforwardprefix}[3][]
{
  \begingroup
    \def\childdocextract #2##1~~~{\def\childdoctmp{\childdocforward[#1]{#3##1}}}
    \expandafter\childdocextract\childdocname~~~
    \expandafter
  \endgroup
  \childdoctmp
}
%    \end{macrocode}

% \macro{\childdoc}
% The deprecated macro |\childdoc| is a legacy version of |\childdocmain|:
%    \begin{macrocode}
\newcommand{\childdoc}{\childdocmain}
%    \end{macrocode}

% \macro{\childdocredirect}
% The deprecated macro |\childdocredirect| is a legacy version
% of |\childdocforward| and |\childdocforwardprefix|:
%    \begin{macrocode}
\newcommand{\childdocredirect}[2][]
{
  \begingroup
    \if?#1?
      \def\childdoctmp{\childdocforward{#2}}
    \else
      \def\childdoctmp{\childdocforwardprefix{#1}{#2}}
    \fi
    \expandafter
  \endgroup
  \childdoctmp
}
%    \end{macrocode}

%\iffalse
%</package>
%\fi
%
\endinput
|\\
|\childdocby{|\textit{main}|}|\\
\end{tabular}
\end{center}
%
Both forms have slightly different effects as described above.
The main file is prepared as usual, see \secref{sec:include}.

%%%%%%%%%%%%%%%%%%%%%%%%%%%%%%%%%%%%%%%%%%%%%%%%%%%%%%%%%%%%%%%%%%%%%%%%%%%%%%%%
\subsection{Legacy Detection}
\label{sec:detection}

The directive |\childdocmain| in the main file can detect
whether the complete document or merely a child is to be compiled
even without using the directive |\childdocof|.
This method is deprecated because it is less robust
and there is no compelling reason to use it;
it is merely provided for backward compatibility
and it may be removed in future versions.

If the detection mechanism is to be used,
it is mandatory to correctly specify
the filename of the main file as the argument of |\childdocmain|:
%
\begin{center}
\begin{tabular}{l}
|% \iffalse
%
% childdoc.dtx Copyright (C) 2017-2018 Niklas Beisert
%
% This work may be distributed and/or modified under the
% conditions of the LaTeX Project Public License, either version 1.3
% of this license or (at your option) any later version.
% The latest version of this license is in
%   http://www.latex-project.org/lppl.txt
% and version 1.3 or later is part of all distributions of LaTeX
% version 2005/12/01 or later.
%
% This work has the LPPL maintenance status `maintained'.
%
% The Current Maintainer of this work is Niklas Beisert.
%
% This work consists of the files childdoc.dtx and childdoc.ins
% and the derived files childdoc.def and cdocsamp.tex with
% cdocsch1.tex, cdocsch2.tex, cdocsdrf.tex, cdocsfn1.tex, cdocsfn2.tex.
%
%<package>\ifdefined\childdocmain\endinput\fi
%<package>\ProvidesFile{childdoc.def}[2018/12/30 v2.0 child document driver]
%<samplemain>\ProvidesFile{cdocsamp.tex}[2018/12/30 v2.0 sample for childdoc]
%<*driver>
%\ProvidesFile{childdoc.drv}[2018/12/30 v2.0 childdoc reference manual file]
\PassOptionsToClass{10pt,a4paper}{article}
\documentclass{ltxdoc}

\usepackage[margin=35mm]{geometry}
\usepackage{hyperref}
\usepackage{hyperxmp}
\usepackage[usenames]{color}

\hypersetup{colorlinks=true}
\hypersetup{pdfstartview=FitH}
\hypersetup{pdfpagemode=UseNone}
\hypersetup{pdfsource={}}
\hypersetup{pdflang={en-UK}}
\hypersetup{pdfcopyright={Copyright 2017-2018 Niklas Beisert.
  This work may be distributed and/or modified under the
  conditions of the LaTeX Project Public License, either version 1.3
  of this license or (at your option) any later version.}}
\hypersetup{pdflicenseurl={http://www.latex-project.org/lppl.txt}}
\hypersetup{pdfcontactaddress={ETH Zurich, ITP, HIT K,
  Wolfgang-Pauli-Strasse 27}}
\hypersetup{pdfcontactpostcode={8093}}
\hypersetup{pdfcontactcity={Zurich}}
\hypersetup{pdfcontactcountry={Switzerland}}
\hypersetup{pdfcontactemail={nbeisert@itp.phys.ethz.ch}}
\hypersetup{pdfcontacturl={http://people.phys.ethz.ch/\xmptilde nbeisert/}}

\newcommand{\secref}[1]{\hyperref[#1]{section \ref*{#1}}}

\parskip1ex
\parindent0pt
\let\olditemize\itemize
\def\itemize{\olditemize\parskip0pt}

\begin{document}

\title{The \textsf{childdoc} Package}
\hypersetup{pdftitle={The childdoc Package}}
\author{Niklas Beisert\\[2ex]
  Institut f\"ur Theoretische Physik\\
  Eidgen\"ossische Technische Hochschule Z\"urich\\
  Wolfgang-Pauli-Strasse 27, 8093 Z\"urich, Switzerland\\[1ex]
  \href{mailto:nbeisert@itp.phys.ethz.ch}
  {\texttt{nbeisert@itp.phys.ethz.ch}}}
\hypersetup{pdfauthor={Niklas Beisert}}
\hypersetup{pdfsubject={Manual for the LaTeX2e Package childdoc}}
\date{30 December 2018, \textsf{v2.0}}
\maketitle

\begin{abstract}\noindent
\textsf{childdoc} is a \LaTeXe{} package
that enables the direct compilation
of document sections included by |\include|
to individual files.
\end{abstract}

\begingroup
\parskip0ex
\tableofcontents
\endgroup

%%%%%%%%%%%%%%%%%%%%%%%%%%%%%%%%%%%%%%%%%%%%%%%%%%%%%%%%%%%%%%%%%%%%%%%%%%%%%%%%
%%%%%%%%%%%%%%%%%%%%%%%%%%%%%%%%%%%%%%%%%%%%%%%%%%%%%%%%%%%%%%%%%%%%%%%%%%%%%%%%
\section{Introduction}

\LaTeX{} provides a mechanism to structure a large document (such as a book)
into a main file and several child files (containing the chapters)
using the |\include| command.
This mechanism is beneficial for documents
which span hundreds of pages in order to
make the source file(s) more manageable.
Moreover, compilation can be restricted to
selected child files by means of the |\includeonly| command.
The latter feature can be used to reduce the compilation time while editing
(this was significantly more useful in the earlier days of \LaTeX{})
or to generate a smaller document which is easier to navigate.
Another application of |\includeonly| is to generate
documents consisting of selected parts of the complete document.

However, there are a few drawbacks of the plain |\include| mechanism:
\begin{itemize}
\item
The child files cannot be compiled on their own,
they can only be compiled via the main file.
A naive editing environment
(such as a text editor with an option
to have the current file processed by \LaTeX)
may require one to switch to the main file before compiling;
attempting to compile the child file produces errors.
\item
The main file must be modified (each time)
to adjust the |\includeonly| command
to the present needs. This easily leaves the main file in a messy state.
\item
The generated document will always carry the filename
of the main document. This is inconvenient if
several child files are to be compiled and
to be kept for distribution.
\end{itemize}

The present package provides a simple interface
to make child files individually compilable by \LaTeX{}.
Compiling a child file then has the same effect as compiling
the main file with an |\includeonly| command
to select the appropriate child.
Moreover the generated document will carry the name of the child
rather than the main file.
This resolves all three above issues.

This feature is meant to make the editing of books,
thesis documents and lecture notes somewhat more convenient.
However, the package can also be used efficiently for
composing a series of documents (such as exercise sheets)
which are typically distributed individually.
It then assists the author in generating the individual documents
(potentially in different versions)
as well as a document containing the collected series.
Another application is in developing style files
or other kinds of included material
where compilation of the style file could redirect
to a sample or test file.

%%%%%%%%%%%%%%%%%%%%%%%%%%%%%%%%%%%%%%%%%%%%%%%%%%%%%%%%%%%%%%%%%%%%%%%%%%%%%%%%
%%%%%%%%%%%%%%%%%%%%%%%%%%%%%%%%%%%%%%%%%%%%%%%%%%%%%%%%%%%%%%%%%%%%%%%%%%%%%%%%
\section{Usage}

First of all, the package \textsf{childdoc} is \emph{not} a standard
\LaTeXe{} |.sty| style file! Therefore it needs to be invoked in
a non-standard way.

%%%%%%%%%%%%%%%%%%%%%%%%%%%%%%%%%%%%%%%%%%%%%%%%%%%%%%%%%%%%%%%%%%%%%%%%%%%%%%%%
\subsection{Included Files}
\label{sec:include}

%%%%%%%%%%%%%%%%%%%%%%%%%%%%%%%%%%%%%%%%
\DescribeMacro{\childdocmain}
To use the package, add the commands
\begin{center}
\begin{tabular}{l}
|\input{childdoc.def}|\\
|\childdocmain{}|\\
\end{tabular}
\end{center}
at the very top of the main \LaTeX{} file,
in particular \emph{before} the |\documentclass| statement!
The argument of |\childdocmain| should be left empty
(but it must be present).

%%%%%%%%%%%%%%%%%%%%%%%%%%%%%%%%%%%%%%%%
\DescribeMacro{\childdocof}
Furthermore, add the commands
\begin{center}
\begin{tabular}{l}
|\input{childdoc.def}|\\
|\childdocof{|\textit{main}|}|\\
\end{tabular}
\end{center}
at the top of every child file \textit{child}
which is included by |\include{|\textit{child}|}|
from within the main file
(or at least for those files to be compiled individually).
The argument \textit{main} must be the filename of the main file.

There are a couple of
considerations in setting up the main and child documents:

%%%%%%%%%%%%%%%%%%%%%%%%%%%%%%%%%%%%%%%%
\paragraph{Restrictions.}

Please note the following restrictions:
\begin{itemize}
\item
|\childdocmain| must be called with one argument \textit{main}
to ensure compatibility with earlier version of the package.
It must either be empty (|\childdocmain{}|)
or precisely match the filename of the main file in which it is specified.
See \secref{sec:detection} for further information.
\item
The filename \textit{main} must be specified without the |.tex| extension.
\item
The filename \textit{main} is case sensitive
(even in case-insensitive file systems)
due to internal string comparison.
\item
The argument \textit{main} should be fully expanded, it cannot be a macro.
\item
Subdirectories and special characters should be avoided in filenames.
\item
The command |\childdocmain{|\textit{main}|}| must be followed by a whitespace.
It should not be followed immediately by another command
or by a comment mark `|%|'.
This is because the \TeX{} parser reads the token immediately following
the argument of |\childdocmain| and puts it
at the beginning of every child section;
however, a white\-space is ignored.
\end{itemize}

%%%%%%%%%%%%%%%%%%%%%%%%%%%%%%%%%%%%%%%%
\paragraph{Content of Main File.}

It is advisable to place all content in the child files included by |\include|.
Any output contained in the main file will appear in all child documents
unless suppressed manually;
it cannot be suppressed automatically by the |\includeonly| directive
and thus should normally be avoided.
A method to include some content in the main file
by means of conditional processing is described in \secref{sec:conditional}.

%%%%%%%%%%%%%%%%%%%%%%%%%%%%%%%%%%%%%%%%
\paragraph{Page Numbering.}

When only a part of the document is compiled,
the appropriate numbering of pages
(as well as other status parameters)
is determined from the |.aux| files.
The latter contain information from previous passes.
However this information needs to propagate through
all intermediate child documents.
Therefore the page numbering in child documents may well
be inconsistent until the complete document is compiled at least once.

A useful (if unconventional) way to always ensure a consistent
page numbering is to restart the numbering in each child document
and denote the pages by `\textit{child}|.|\textit{page}'
where \textit{child} represents the chapter/section number of the child file.
This can be achieved by the command
|\numberwithin{page}{|\textit{child}|}|
of the \textsf{amsmath} package
where \textit{child} can be |chapter| or |section|
depending on the chosen structuring.
Alternatively, one can modify the macro |\thepage| appropriately
and reset the counter |page| at the start of each child file.

%%%%%%%%%%%%%%%%%%%%%%%%%%%%%%%%%%%%%%%%%%%%%%%%%%%%%%%%%%%%%%%%%%%%%%%%%%%%%%%%
\subsection{Conditional Processing}
\label{sec:conditional}

The package provides a mechanism to compile different versions
of a document. To customise the versions further some conditional processing
can come in handy to distinguish which version is being compiled.
The package provides two macros to describe the compilation context:

%%%%%%%%%%%%%%%%%%%%%%%%%%%%%%%%%%%%%%%%
\DescribeMacro{\ifchilddoc}
The conditional |\ifchilddoc| distinguishes between the compilation of
child documents and the main document:
%
\begin{center}
|\ifchilddoc |\textit{child-code}| |[|\||else |\textit{main-code}]| \||fi|
\end{center}

%%%%%%%%%%%%%%%%%%%%%%%%%%%%%%%%%%%%%%%%
\DescribeMacro{\childdocname}
\DescribeMacro{\childdocjob}
The macro |\childdocname| contains the filename (without extension)
of the main or child file being processed.
Note that |\childdocjob| will always contain the name of the main file.

%%%%%%%%%%%%%%%%%%%%%%%%%%%%%%%%%%%%%%%%
\paragraph{Title Page.}

Conditional processing can be used to include a title or banner page
in the main document when proper precautions are taken.
Importantly, the code in the main file should ensure that the page counter
(as well as other status parameters which are stored in the |.aux| files)
takes the same value after the conditional processing.
Otherwise the page numbers may take divergent values
depending on which part is compiled.

For example, a title page could be declared by:
%
\begin{center}
\begin{tabular}{l}
|\ifchilddoc\||else|\\
|\addtocounter{page}{-1}|\\
\textit{code for title page}\\
|\newpage|\\
|\||fi|
\end{tabular}
\end{center}
%
A banner page for the child documents can be generated by:
%
\begin{center}
\begin{tabular}{l}
|\ifchilddoc|\\
|\addtocounter{page}{-1}|\\
\textit{code for banner page}\\
|\newpage|\\
|\||fi|
\end{tabular}
\end{center}
%
Here one could write a message such as:
\begin{center}
|This is the part \childdocname{} of \childdocjob{}.|
\end{center}

%%%%%%%%%%%%%%%%%%%%%%%%%%%%%%%%%%%%%%%%%%%%%%%%%%%%%%%%%%%%%%%%%%%%%%%%%%%%%%%%
\subsection{Flags}
\label{sec:flags}

The package makes it easy to generate different versions
of the main or child documents.
To this end compilation flags can be defined
and assigned different default values.
They will be particularly useful in conjunction
with the forwarding mechanism described in \secref{sec:forward}.

For example, it may be useful to have a flag |\version|
which can be set to |draft| or |final|.
The document source will contain some conditional code
depending on the value of |\version|.
Suppose further, the flag should default to |final| for the main file
and to |draft| for child files
which is a natural assignment for editing the document.
This is achieved by placing the following code
in the preamble of the main document
(below the |\childdocmain| directive):
%
\begin{center}
\begin{tabular}{l}
|\ifchilddoc|\\
|\providecommand{\version}{draft}|\\
|\||else|\\
|\providecommand{\version}{final}|\\
|\||fi|
\end{tabular}
\end{center}
%
The definition by |\providecommand| makes sure
that previous definitions are not overwritten.
Further statements |\providecommand{\version}{...}|
can thus be added before the above code to override it.

For the main file, one might add a line
(between |\childdocmain| and the above block)
%
\begin{center}
|%\ifchilddoc\||else\providecommand{\version}{draft}\||fi|
\end{center}
%
which can be uncommented to produce a draft version.
Likewise one can add a line to the very top of a child file
(above the |\childdocof{|\textit{main}|}| directive)
%
\begin{center}
|%\providecommand{\version}{final}|
\end{center}
%
which can be uncommented to produce the final version of this child document.

%%%%%%%%%%%%%%%%%%%%%%%%%%%%%%%%%%%%%%%%%%%%%%%%%%%%%%%%%%%%%%%%%%%%%%%%%%%%%%%%
\subsection{Forwarding}
\label{sec:forward}

Different versions of the main or child documents
using compilation flags as described in \secref{sec:flags}
can be (permanently) stored in different files
for convenient compilation, viewing and distribution.
To this end, the package defines a command
to pass on compilation to a different file:

%%%%%%%%%%%%%%%%%%%%%%%%%%%%%%%%%%%%%%%%
\DescribeMacro{\childdocforward}
The command |\childdocforward| redirects processing to
another source file:
%
\begin{center}
\begin{tabular}{l}
|\input{childdoc.def}|\\
|\childdocforward[|\textit{main}|]{|\textit{dest}|}|\\
\end{tabular}
\end{center}
%
The argument \textit{dest} is the destination file
(without extension).
It should be the main file or one of the child files.
Note that further \textsf{childdoc} directives
such as |\childdocof| and |\childdocforward|
in the indicated file will be processed in this form.
The optional argument \textit{main}
passes on directly to the main file \textit{main}
while pretending to compile the child \textit{dest}.
This form behaves as if \textit{dest}
issues |\childdocof{|\textit{main}|}| right away,
and no further \textsf{childdoc} directives will be processed.

%%%%%%%%%%%%%%%%%%%%%%%%%%%%%%%%%%%%%%%%
\DescribeMacro{\...prefix}
In the alternative form |\childdocforwardprefix|,
%
\begin{center}
\begin{tabular}{l}
|\input{childdoc.def}|\\
|\childdocforwardprefix[|\textit{main}|]{|\textit{prefix}|}{|\textit{dest}|}|
\end{tabular}
\end{center}
%
the destination file is determined by a pattern
depending on the current file:
To make this work, the current file must be called
`{\textit{prefix}\hspace{0.2em}\textit{suffix}}'
with \textit{prefix} matching precisely the argument.
Processing is then passed on to the file
`{\textit{dest}\hspace{0.2em}\textit{suffix}}'.
Surely, the same effect is achieved by
directly specifying the
argument `{\textit{dest}\hspace{0.2em}\textit{suffix}}'
in the first form.
However, that requires to set up a different file
for each child. With the alternative form of the command
all these files can have exactly the same content
which simplifies setting them up and maintaining them.

For example, the following file |draft.tex|
with a compilation flag |\version| as described in \secref{sec:flags}
compiles the main document as a draft:
%
\begin{center}
\begin{tabular}{l}
|\def\version{draft}|\\
|\input{childdoc.def}|\\
|\childdocforward{|\textit{main}|}|
\end{tabular}
\end{center}
%
Likewise, the following files |final|\textit{nn}|.tex|
compile the final version of the child document
|child|\textit{nn}|.tex|:
%
\begin{center}
\begin{tabular}{l}
|\def\version{final}|\\
|\input{childdoc.def}|\\
|\childdocforwardprefix{final}{child}|
\end{tabular}
\end{center}
%

Note that when several versions of a main file and/or of each child file
are to be generated, it may be convenient to set up a |Makefile| or
shell script to automatise the process.

%%%%%%%%%%%%%%%%%%%%%%%%%%%%%%%%%%%%%%%%%%%%%%%%%%%%%%%%%%%%%%%%%%%%%%%%%%%%%%%%
\subsection{Command Line Processing}
\label{sec:commandline}

The effect of redirection files can also be achieved by invoking
the \LaTeX{} compiler with a more elaborate command line.
Most conveniently this should be done as part
of a shell script or a |Makefile|.

When using \textsf{childdoc} in the main file, the following
command lines effectively perform a redirection
(note that depending on the shell being used,
backslashes may have to be doubled: `|\|' $\to$ `|\\|'):
%
\begin{center}
|... -jobname "|\textit{target}|" |\\|"|[\textit{flags}]%
|\input{childdoc.def}\childdocforward[|\textit{main}|]{|\textit{dest}|}"|
\end{center}
%
Here \textit{target} is the name of the output file,
\textit{main} is the name of the main file
and \textit{dest} is the name of the main or child file to be processed
(all filenames without extensions).
The optional argument \textit{main} can be omitted
if \textit{main} matches \textit{dest}.
Optionally, compilation \textit{flags} can be defined via |\def| commands.
This command line makes the \TeX{} engine believe
it is compiling the file \textit{target}
whose content is specified as the latter parameter.
The provided code then forwards the processing to
\textit{main} or \textit{dest} as described in \secref{sec:forward}.

%%%%%%%%%%%%%%%%%%%%%%%%%%%%%%%%%%%%%%%%%%%%%%%%%%%%%%%%%%%%%%%%%%%%%%%%%%%%%%%%
\subsection{Include by Input}
\label{sec:input}

Including child documents by |\include| has some restrictions by design.
Most notably, the content of a child document always occupies
its own set of pages; pages cannot be shared between child documents.
Usually, this behaviour makes perfect sense
because each child document contain an essential part of the document.
However, in some situations it may be desirable to compose
a document from a collection of parts
without having mandatory page breaks between then.
For this case, the package
provides a mechanism to include parts
by |\input| which can also be processed individually.
However, by construction this mechanism
requires manual handling of the content to be output.

%%%%%%%%%%%%%%%%%%%%%%%%%%%%%%%%%%%%%%%%
\DescribeMacro{\ifchilddocmanual}
The main file should be prepared as usual, see \secref{sec:include}.
However, the document body must make a distinction
between processing of an individual part and of the main document, e.g.:
%
\begin{center}
\begin{tabular}{l}
|\ifchilddocmanual|\\
|\input{\childdocname}|\\
|\||else|\\
\textit{document body with }|\input{|\textit{part}|}|\\
|\||fi|
\end{tabular}
\end{center}
%
The conditional |\ifchilddocmanual| is true whenever
a part to be included by |\input| is being compiled,
and the name of the part is stored in |\childdocname|.

%%%%%%%%%%%%%%%%%%%%%%%%%%%%%%%%%%%%%%%%
\DescribeMacro{\childdocby}
Each part to be included by |\input| should start with:
%
\begin{center}
\begin{tabular}{l}
|\input{childdoc.def}|\\
|\childdocby{|\textit{main}|}|\\
\end{tabular}
\end{center}
%
The directive |\childdocby| is similar to |\childdocof|
described in \secref{sec:include},
but the subsequent selection of content must be done manually.
To that end, both |\ifchilddoc| and |\ifchilddocmanual|
will be true upon processing of a part,
and the name of the part is stored in |\childdocname|.
Note that |\jobname| will be set to the filename of the current part
so that each part receives an individual |.aux| file
that does not interfere with the |.aux| file(s) of the main document.
This behaviour can be altered by the alternative form
|\childdocby[*]{|\textit{main}|}| (with a non-empty optional argument)
which uses the |.aux| file of the main document
by setting |\jobname| to \textit{main}.

%%%%%%%%%%%%%%%%%%%%%%%%%%%%%%%%%%%%%%%%%%%%%%%%%%%%%%%%%%%%%%%%%%%%%%%%%%%%%%%%
\subsection{Driver Development}
\label{sec:driver}

The \textsf{childdoc} mechanism can also be use for the development
of definition files such as \LaTeX{} styles or classes.
This case differs from the above setup with multiple parts
included by |\include| in that no |\includeonly| should be invoked.
This can be achieved by starting the include file
(before |\ProvidesPackage|) with:
%
\begin{center}
\begin{tabular}{l}
|\input{childdoc.def}|\\
|\childdocforward{|\textit{main}|}|\\
\end{tabular}
\end{center}
%
or alternatively with:
%
\begin{center}
\begin{tabular}{l}
|\input{childdoc.def}|\\
|\childdocby{|\textit{main}|}|\\
\end{tabular}
\end{center}
%
Both forms have slightly different effects as described above.
The main file is prepared as usual, see \secref{sec:include}.

%%%%%%%%%%%%%%%%%%%%%%%%%%%%%%%%%%%%%%%%%%%%%%%%%%%%%%%%%%%%%%%%%%%%%%%%%%%%%%%%
\subsection{Legacy Detection}
\label{sec:detection}

The directive |\childdocmain| in the main file can detect
whether the complete document or merely a child is to be compiled
even without using the directive |\childdocof|.
This method is deprecated because it is less robust
and there is no compelling reason to use it;
it is merely provided for backward compatibility
and it may be removed in future versions.

If the detection mechanism is to be used,
it is mandatory to correctly specify
the filename of the main file as the argument of |\childdocmain|:
%
\begin{center}
\begin{tabular}{l}
|\input{childdoc.def}|\\
|\childdocmain{|\textit{main}|}|\\
\end{tabular}
\end{center}
%
If |\jobname| does not match the argument \textit{main} of |\childdocmain|,
it is assumed that |\jobname| points to the child file to be compiled.
When using |\childdocmain| with the main file specified as argument,
it suffices to start a child file
with just |\input{|\textit{main}|}|
without loading of the package and using |\childdocof|.
If instead all processing is done
with the appropriate \textsf{childdoc} directives,
the argument of \textit{main} of |\childdocmain| can be empty.

An alternative version of the command line processing described
in \secref{sec:commandline} using the detection mechanism reads:
%
\begin{center}
|... -jobname "|\textit{target}|" "|[\textit{flags}]%
[|\def\jobname{|\textit{dest}|}|]|\input{|\textit{main}|}"|
\end{center}

%%%%%%%%%%%%%%%%%%%%%%%%%%%%%%%%%%%%%%%%%%%%%%%%%%%%%%%%%%%%%%%%%%%%%%%%%%%%%%%%
\subsection{Manual Code}
\label{sec:manual}

In case one cannot be certain whether the definitions file |childdoc.def|
is installed on the target \TeX{} distribution
and one prefers not to ship it,
it is conceivable to paste a few relevant commands into the sources.

To that end, drop all statements |\input{childdoc.def}|
and perform the replacements as outlined below.
Instead of |\childdocmain{|\textit{main}|}| add the following code
to the top of the main file:
%
\begin{center}
\begin{tabular}{l}
|\||ifdefined\childdocname\endinput\||fi\newif\ifchilddoc|\\
|\edef\childdocname{\scantokens\expandafter{\jobname\noexpand}}|\\
|\def\childdocmain{|\textit{main}|}\||ifx\childdocmain\childdocname\||else|\\
|\childdoctrue\includeonly{\childdocname}\let\jobname\childdocmain\||fi|\\
\end{tabular}
\end{center}
%
Instead of |\childdocof{|\textit{main}|}| just include the main file
at the top of each child file:
%
\begin{center}
|\input{|\textit{main}|}|
\end{center}
%
A simple redirection |\childdocforward{|\textit{dest}|}| is achieved by:
%
\begin{center}
|\def\jobname{|\textit{dest}|}\input{\jobname}|
\end{center}
%
The redirection with prefix
|\childdocforwardprefix[|\textit{prefix}|]{|\textit{dest}|}|
is accomplished by:
%
\begin{center}
\begin{tabular}{l}
|{\edef\jobname{\scantokens\expandafter{\jobname\noexpand}}|\\
|\def\redirectjob |\textit{prefix}|#1~~~{\gdef\jobname{|\textit{dest}|#1}}|\\
|\expandafter\redirectjob\jobname~~~}\input{\jobname}|
\end{tabular}
\end{center}

In an alternative approach,
child documents can be compiled by a specific command line
without additional code or specific definitions:
%
\begin{center}
|... -jobname "|\textit{target}|" "|[\textit{flags}]%
|\includeonly{|\textit{dest}|}\input{|\textit{main}|}"|
\end{center}
%

%%%%%%%%%%%%%%%%%%%%%%%%%%%%%%%%%%%%%%%%%%%%%%%%%%%%%%%%%%%%%%%%%%%%%%%%%%%%%%%%
%%%%%%%%%%%%%%%%%%%%%%%%%%%%%%%%%%%%%%%%%%%%%%%%%%%%%%%%%%%%%%%%%%%%%%%%%%%%%%%%
\section{Information}

%%%%%%%%%%%%%%%%%%%%%%%%%%%%%%%%%%%%%%%%%%%%%%%%%%%%%%%%%%%%%%%%%%%%%%%%%%%%%%%%
\subsection{Copyright}

Copyright \copyright{} 2017--2018 Niklas Beisert

This work may be distributed and/or modified under the
conditions of the \LaTeX{} Project Public License, either version 1.3
of this license or (at your option) any later version.
The latest version of this license is in
  \url{http://www.latex-project.org/lppl.txt}
and version 1.3 or later is part of all distributions of \LaTeX{}
version 2005/12/01 or later.

This work has the LPPL maintenance status `maintained'.

The Current Maintainer of this work is Niklas Beisert.

This work consists of the files |README.txt|, |childdoc.ins| and |childdoc.dtx|
as well as the derived files |childdoc.def|, |cdocsamp.tex|
with |cdocsch1.tex|, |cdocsch2.tex|, |cdocspt3.tex|, |cdocspt4.tex|,
|cdocsdrf.tex|, |cdocsfn1.tex|, |cdocsfn2.tex|
as well as |childdoc.pdf|.

%%%%%%%%%%%%%%%%%%%%%%%%%%%%%%%%%%%%%%%%%%%%%%%%%%%%%%%%%%%%%%%%%%%%%%%%%%%%%%%%
\subsection{Files and Installation}

The package consists of the files:
%
\begin{center}
\begin{tabular}{ll}
    |README.txt|   & readme file \\
    |childdoc.ins| & installation file \\
    |childdoc.dtx| & source file \\
    |childdoc.def| & definition file \\
    |cdocsamp.tex| & sample main file \\
    |cdocsch1.tex| & sample include file \\
    |cdocsch2.tex| & sample include file \\
    |cdocspt3.tex| & sample part file \\
    |cdocspt4.tex| & sample part file \\
    |cdocsdrf.tex| & sample redirection file \\
    |cdocsfn1.tex| & sample redirection file \\
    |cdocsfn2.tex| & sample redirection file \\
    |childdoc.pdf| & manual
\end{tabular}
\end{center}
%
The distribution consists of the files
|README.txt|, |childdoc.ins| and |childdoc.dtx|.
%
\begin{itemize}
\item
Run (pdf)\LaTeX{} on |childdoc.dtx|
to compile the manual |childdoc.pdf| (this file).
\item
Run \LaTeX{} on |childdoc.ins| to create the definitions file |childdoc.def|
and the sample |cdocsamp.tex| with include files
|cdocsch1.tex|, |cdocsch2.tex|, |cdocspt3.tex|, |cdocspt4.tex|,
|cdocsdrf.tex|, |cdocsfn1.tex|, |cdocsfn2.tex|.
Then copy the file |childdoc.def| to an appropriate directory of your \LaTeX{}
distribution, e.g.\ \textit{texmf-root}|/tex/latex/childdoc|.
\end{itemize}

%%%%%%%%%%%%%%%%%%%%%%%%%%%%%%%%%%%%%%%%%%%%%%%%%%%%%%%%%%%%%%%%%%%%%%%%%%%%%%%%
\subsection{Related CTAN Packages}

There are several other packages which offer a similar functionality:
%
\begin{itemize}
\item
The packages
\href{http://ctan.org/pkg/docmute}{\textsf{docmute}},
\href{http://ctan.org/pkg/includex}{\textsf{includex}} and
\href{http://ctan.org/pkg/standalone}{\textsf{standalone}}
provide commands to include only the document body of
a child file thus allowing both files to be compiled individually.
\item
The packages \href{http://ctan.org/pkg/subdocs}{\textsf{subdocs}}
and \href{http://ctan.org/pkg/subfiles}{\textsf{subfiles}}
provide structures in which the main and child documents can be
encapsulated and allowing them to be compiled individually.
The inclusion mechanism is different from the conventional |\include|.
\item
The package \href{http://ctan.org/pkg/combine}{\textsf{combine}}
is an elaborate solution to combine several documents into one.
\end{itemize}
%
See also the CTAN topic \href{http://ctan.org/topic/subdocs}{\textsf{subdocs}}
for further related packages.
The present package differs from the above solutions in that
a document structure constructed with the conventional |\include| mechanism
just needs two extra commands at the top of every file
such that all constituent files can be compiled individually.

%%%%%%%%%%%%%%%%%%%%%%%%%%%%%%%%%%%%%%%%%%%%%%%%%%%%%%%%%%%%%%%%%%%%%%%%%%%%%%%%
%\subsection{Feature Suggestions}
%
%The following is a list of features which may be useful for future
%versions of this package:
%%
%\begin{itemize}
%\item
%\ldots
%\end{itemize}

%%%%%%%%%%%%%%%%%%%%%%%%%%%%%%%%%%%%%%%%%%%%%%%%%%%%%%%%%%%%%%%%%%%%%%%%%%%%%%%%
\subsection{Revision History}

%%%%%%%%%%%%%%%%%%%%%%%%%%%%%%%%%%%%%%%%
\paragraph{v2.0:} 2018/12/30

\begin{itemize}
\item
immediate forward processing
\item
added |\childdocby| mechanism
\item
manual restructured
\end{itemize}

%%%%%%%%%%%%%%%%%%%%%%%%%%%%%%%%%%%%%%%%
\paragraph{v1.6:} 2018/01/17

\begin{itemize}
\item
application for development of include files
\item
corrections to manual
\end{itemize}

%%%%%%%%%%%%%%%%%%%%%%%%%%%%%%%%%%%%%%%%
\paragraph{v1.5:} 2017/05/21

\begin{itemize}
\item
more complete structuring introduced
\item
|\childdocof| introduced
\item
|\childdoc| renamed to |\childdocmain|
\item
|\childredirect| renamed to |\childdocforward| and |\childdocforwardprefix|
and functionality expanded
\end{itemize}

%%%%%%%%%%%%%%%%%%%%%%%%%%%%%%%%%%%%%%%%
\paragraph{v1.0:} 2017/04/27

\begin{itemize}
\item
manual and install package
\item
first version published on CTAN
\end{itemize}

%%%%%%%%%%%%%%%%%%%%%%%%%%%%%%%%%%%%%%%%
\paragraph{v0.6:} 2017/04/26

\begin{itemize}
\item
redirection mechanism added
\end{itemize}

%%%%%%%%%%%%%%%%%%%%%%%%%%%%%%%%%%%%%%%%
\paragraph{v0.5:} 2017/04/26

\begin{itemize}
\item
functionality in definition file
\end{itemize}


%%%%%%%%%%%%%%%%%%%%%%%%%%%%%%%%%%%%%%%%%%%%%%%%%%%%%%%%%%%%%%%%%%%%%%%%%%%%%%%%
%%%%%%%%%%%%%%%%%%%%%%%%%%%%%%%%%%%%%%%%%%%%%%%%%%%%%%%%%%%%%%%%%%%%%%%%%%%%%%%%
%%%%%%%%%%%%%%%%%%%%%%%%%%%%%%%%%%%%%%%%%%%%%%%%%%%%%%%%%%%%%%%%%%%%%%%%%%%%%%%%
\appendix

\settowidth\MacroIndent{\rmfamily\scriptsize 000\ }

 \DocInput{childdoc.dtx}

\end{document}
%</driver>
% \fi
%
% %%%%%%%%%%%%%%%%%%%%%%%%%%%%%%%%%%%%%%%%%%%%%%%%%%%%%%%%%%%%%%%%%%%%%%%%%%%%%%
% %%%%%%%%%%%%%%%%%%%%%%%%%%%%%%%%%%%%%%%%%%%%%%%%%%%%%%%%%%%%%%%%%%%%%%%%%%%%%%
% \section{Sample}
%\iffalse
%<*samplemain>
%\fi
%
% The following presents a sample document
% with two chapters, two parts, a title page,
% a compile flag as well as three forwarding files to set the flag.
% It consists of eight |.tex| files:
% \begin{center}
% \begin{tabular}{ll}
% |cdocsamp.tex|&main file\\
% |cdocsch1.tex|&include file for chapter 1\\
% |cdocsch2.tex|&include file for chapter 2\\
% |cdocspt3.tex|&include file for part 3\\
% |cdocspt4.tex|&include file for part 4\\
% |cdocsdrf.tex|&forwarding file for main file in draft mode\\
% |cdocsfi1.tex|&forwarding file for final version of chapter 1\\
% |cdocsfi2.tex|&forwarding file for final version of chapter 2\\
% \end{tabular}
% \end{center}
% Each of the eight files can be compiled directly by the \LaTeX{} compiler.
%
% %%%%%%%%%%%%%%%%%%%%%%%%%%%%%%%%%%%%%%
% \paragraph{Main File.}
%
% The main file is called |cdocsamp.tex|.
%
% Load the \textsf{childdoc} definitions and
% declare the filename for the main document:
%    \begin{macrocode}
\input{childdoc.def}
\childdocmain{}
%    \end{macrocode}

% Optional override for |\version| flag:
%    \begin{macrocode}
%%\ifchilddoc\else\providecommand{\version}{draft}\fi
%    \end{macrocode}

% Define the default values for the |\version| flag
% (|final| for the main file and |draft| for childs):
%    \begin{macrocode}
\ifchilddoc
\providecommand{\version}{draft}
\else
\providecommand{\version}{final}
\fi
%    \end{macrocode}

% Load the standard document class:
%    \begin{macrocode}
\documentclass[12pt]{article}
%    \end{macrocode}

% Start the document body:
%    \begin{macrocode}
\begin{document}
%    \end{macrocode}

% Declare a title page.
% Print title, part of document being processed and version flag:
%    \begin{macrocode}
\addtocounter{page}{-1}
\begin{center}
{\LARGE\bfseries{}childdoc example\par}
\vspace{1cm}
\ifchilddoc
\ifchilddocmanual part\else chapter\fi:
`\childdocname' of `\childdocjob'\par
\else
main document: `\childdocjob'\par
\fi
version: \version\par
\end{center}
\newpage
%    \end{macrocode}

% Manually include selected file,
% otherwise process as usual:
%    \begin{macrocode}
\ifchilddocmanual
\section*{part `\childdocname'}
\input{\childdocname}
\else
%    \end{macrocode}

% Include the two chapters:
%    \begin{macrocode}
\include{cdocsch1}
\include{cdocsch2}
%    \end{macrocode}

% Include the two parts unless only chapters should be displayed:
%    \begin{macrocode}
\ifchilddoc\else
\section{part three}
\input{cdocspt3}
\section{part four}
\input{cdocspt4}
\fi
%    \end{macrocode}

% Process as usual until here:
%    \begin{macrocode}
\fi
%    \end{macrocode}

% End of document body:
%    \begin{macrocode}
\end{document}
%    \end{macrocode}
%\iffalse
%</samplemain>
%\fi
%
% %%%%%%%%%%%%%%%%%%%%%%%%%%%%%%%%%%%%%%
% \paragraph{Chapter Include Files.}
%
% The include files are called |cdocsch1.tex| and |cdocsch2.tex|.
%
%\iffalse
%<*samplechap1|samplechap2>
%\fi

% Optional override for |\version| flag:
%    \begin{macrocode}
%%\providecommand{\version}{final}
%    \end{macrocode}

% Include the main document:
%    \begin{macrocode}
\input{childdoc.def}
\childdocof{cdocsamp}
%    \end{macrocode}

%\iffalse
%</samplechap1|samplechap2>
%\fi
%
%\iffalse
%<*samplechap1>
%\fi
% Some text for chapter 1:
%    \begin{macrocode}
\section{one}
some text in chapter one
%    \end{macrocode}

%\iffalse
%</samplechap1>
%\fi
% Some text for chapter 2:
%\iffalse
%<*samplechap2>
%\fi
%    \begin{macrocode}
\section{two}
more text in chapter two
%    \end{macrocode}

%\iffalse
%</samplechap2>
%\fi
%
% %%%%%%%%%%%%%%%%%%%%%%%%%%%%%%%%%%%%%%
% \paragraph{Part Include Files.}
%
% The include files are called |cdocspt3.tex| and |cdocspt4.tex|.
%
%\iffalse
%<*samplepart3|samplepart4>
%\fi

% Optional override for |\version| flag:
%    \begin{macrocode}
%%\providecommand{\version}{final}
%    \end{macrocode}

% Include the main document:
%    \begin{macrocode}
\input{childdoc.def}
\childdocby{cdocsamp}
%    \end{macrocode}

%\iffalse
%</samplepart3|samplepart4>
%\fi
%
%\iffalse
%<*samplepart3>
%\fi
% Some text for part 3:
%    \begin{macrocode}
some text in part three
%    \end{macrocode}

%\iffalse
%</samplepart3>
%\fi
% Some text for part 4:
%\iffalse
%<*samplepart4>
%\fi
%    \begin{macrocode}
more text in part four
%    \end{macrocode}

%\iffalse
%</samplepart4>
%\fi
%
% %%%%%%%%%%%%%%%%%%%%%%%%%%%%%%%%%%%%%%
% \paragraph{Forwarding for a Complete Draft.}
%
% The following forwarding file |cdocsdrf.tex|
% compiles the main document in draft mode:
%\iffalse
%<*sampledraft>
%\fi
%    \begin{macrocode}
\def\version{draft}
\input{childdoc.def}
\childdocforward{cdocsamp}
%    \end{macrocode}

%\iffalse
%</sampledraft>
%\fi
%
% %%%%%%%%%%%%%%%%%%%%%%%%%%%%%%%%%%%%%%
% \paragraph{Forwarding for Final Version of the Chapters.}
%
% The following forwarding files |cdocsfn1.tex| and |cdocsfn2.tex|
% (with identical content)
% compile the final versions of the child documents
% |cdocsch1.tex| and |cdocsch2.tex|, respectively:
%\iffalse
%<*samplefinal>
%\fi
%    \begin{macrocode}
\def\version{final}
\input{childdoc.def}
\childdocforwardprefix[cdocsamp]{cdocsfn}{cdocsch}
%    \end{macrocode}

%\iffalse
%</samplefinal>
%\fi
%
% %%%%%%%%%%%%%%%%%%%%%%%%%%%%%%%%%%%%%%
% \paragraph{Command Line Processing.}
%
% The following three command lines generate the output files
% |cdocscld|, |cdocscl1| and |cdocscl2|
% which should be identical to
% |cdocsdrf|, |cdocsch1| and |cdocsfn2|, respectively:
% \begin{center}
% \begin{tabular}{l}
% |latex -jobname cdocscld \|\\
% |  "\def\version{draft}\input{childdoc.def}\childdocforward{cdocsamp}"|\\
% |latex -jobname cdocscl1 \|\\
% |  "\input{childdoc.def}\childdocforward[cdocsamp]{cdocsch1}"|\\
% |latex -jobname cdocscl2 \|\\
% |  "\def\version{final}\input{childdoc.def}\childdocforward{cdocsch2}"|
% \end{tabular}
% \end{center}
% Note that the trailing backslash on each first line
% merely continues the input to the second line
% (for convenient cut ant paste).
% Furthermore, the command |latex| can be replaced by any
% of its alternative versions such as |pdflatex|.
%
% %%%%%%%%%%%%%%%%%%%%%%%%%%%%%%%%%%%%%%%%%%%%%%%%%%%%%%%%%%%%%%%%%%%%%%%%%%%%%%
% %%%%%%%%%%%%%%%%%%%%%%%%%%%%%%%%%%%%%%%%%%%%%%%%%%%%%%%%%%%%%%%%%%%%%%%%%%%%%%
% \section{Implementation}
%\iffalse
%<*package>
%\fi
%
% This section describes the definitions file |childdoc.def|.

% The definitions cannot be loaded using |\usepackage| or |\RequirePackage|
% which has a mechanism to prevent loading a style file more than once.
% When loading the definitions by means of |\input|
% multiple instances have to be prevented manually:
%\iffalse
%This code needs to be before the `\ProvidesFile' directive
%which is defined at the beginning of this file.
%Therefore it is also placed there and commented out here.
%</package>
%<*discard>
%\fi
%    \begin{macrocode}
\ifdefined\childdocmain\endinput\fi
%    \end{macrocode}
%\iffalse
%</discard>
%<*package>
%\fi
%
% \macro{\ifchilddoc}
% \macro{\ifchilddocmanual}
% The conditional |\ifchilddoc| tells whether a
% child (true) or main (false) document is being compiled.
% The conditional |\ifchilddocmanual| tells whether
% the |\includeonly| mechanism is used (false) or
% the selection of child files must be performed manually (true).
% The definitions initialise to false:
%    \begin{macrocode}
\newif\ifchilddoc
\newif\ifchilddocmanual
%    \end{macrocode}

% \macro{\childdocname}
% \macro{\childdocjob}
% The macro |\childdocname| stores the name of the main document
% to be compiled. The macro |\childdocjob| stores the name of
% the document on which the \LaTeX{} compiler was originally invoked.
% The content of |\jobname| cannot be compared
% to filenames specified in the source due to different catcodes.
% The following code rescans |\jobname|, stores the result
% in |\childdocname| and saves a copy in |\childdocjob|:
%    \begin{macrocode}
\edef\childdocname{\scantokens\expandafter{\jobname\noexpand}}
\let\childdocjob\childdocname
%    \end{macrocode}

% \macro{\childdocdisable}
% The macro |\childdocdisable| prevents the main file
% from being processed more than once.
% At this stage, the main document command |\childdocmain|
% is assumed to be called once again where it should do nothing.
% Any subsequent call to it should prevent
% a secondary processing of the main document
% It overwrites the forwarding commands
% |\childdocof| and |\childdocforward|
% with empty macros to prevent further inclusions of the main document:
%    \begin{macrocode}
\newcommand{\childdocdisable}
{
  \renewcommand{\childdocmain}[1]{\renewcommand{\childdocmain}[1]{\endinput}}
  \renewcommand{\childdocof}[1]{}
  \renewcommand{\childdocby}[2][]{}
  \renewcommand{\childdocforward}[2][]{}
  \renewcommand{\childdocdisable}{}
}
%    \end{macrocode}

% \macro{\childdocmain}
% The macro |\childdocmain| is to be called at the top of the main file
% with nothing or the main filename (without extension) as argument.
% First, it breaks loops.
% If the argument is not empty and does not match |\childdocname|
% (which is set by the first inclusion of |childdoc.def|),
% |\ifchilddoc| is set to true, |\includeonly| is applied to the child file
% and |\jobname| is set to the main file
% (for proper handling of |.aux| files):
%    \begin{macrocode}
\newcommand{\childdocmain}[1]
{
  \childdocdisable\childdocmain{}
  \if?#1?\else
    \begingroup
      \def\childdoctmp{#1}
      \ifx\childdoctmp\childdocname
        \def\childdoctmp{}
      \else
        \def\childdoctmp
        {
          \childdoctrue
          \includeonly{\childdocname}
          \def\childdocjob{#1}
          \def\jobname{#1}
        }
      \fi
      \expandafter
    \endgroup
    \childdoctmp
  \fi
}
%    \end{macrocode}

% \macro{\childdocof}
% The command |\childdocof| redirects
% compilation to the main file |#1|.
%    \begin{macrocode}
\newcommand{\childdocof}[1]
{
  \childdocdisable
  \childdoctrue
  \includeonly{\childdocname}
  \def\jobname{#1}
  \def\childdocjob{#1}
  \input{#1}
}
%    \end{macrocode}

% \macro{\childdocby}
% The command |\childdocby| ....
%    \begin{macrocode}
\newcommand{\childdocby}[2][]
{
  \childdocdisable
  \childdoctrue
  \childdocmanualtrue
  \if?#1?\else
    \def\jobname{#2}
  \fi
  \def\childdocjob{#2}
  \input{#2}
  \endinput
}
%    \end{macrocode}

% \macro{\childdocforward}
% The command |\childdocforward| redirects
% compilation to the main file or
% (if the optional argument is given) a child file.
% Parameters are set as if the main file
% or a child file starting with |\childdocof| was compiled.
% Then compilation is handed over to the main file:
%    \begin{macrocode}
\newcommand{\childdocforward}[2][]
{
  \begingroup
    \if?#1?
      \def\childdoctmp
      {
        \def\childdocname{#2}
        \def\childdocjob{#2}
        \def\jobname{#2}
        \input{#2}
        \endinput
      }
    \else
      \def\childdoctmp
      {
        \childdocdisable
        \def\childdocname{#2}
        \childdoctrue
        \includeonly{#2}
        \def\childdocjob{#1}
        \def\jobname{#1}
        \input{#1}
        \endinput
      }
    \fi
    \expandafter
  \endgroup
  \childdoctmp
}
%    \end{macrocode}

% \macro{\childdocforwardprefix}
% The command |\childdocforwardprefix| redirects
% compilation to the main or a child file by means of a pattern.
% The prefix |#1| in the current filename is replaced by |#2|
% and the suffix of the current filename is kept
% (it is assumed that the filename does not contain the substring `|~~~|'
% which is used as a delimiter).
% Compilation is handed over to the new file by |\childdocforward|:
%    \begin{macrocode}
\newcommand{\childdocforwardprefix}[3][]
{
  \begingroup
    \def\childdocextract #2##1~~~{\def\childdoctmp{\childdocforward[#1]{#3##1}}}
    \expandafter\childdocextract\childdocname~~~
    \expandafter
  \endgroup
  \childdoctmp
}
%    \end{macrocode}

% \macro{\childdoc}
% The deprecated macro |\childdoc| is a legacy version of |\childdocmain|:
%    \begin{macrocode}
\newcommand{\childdoc}{\childdocmain}
%    \end{macrocode}

% \macro{\childdocredirect}
% The deprecated macro |\childdocredirect| is a legacy version
% of |\childdocforward| and |\childdocforwardprefix|:
%    \begin{macrocode}
\newcommand{\childdocredirect}[2][]
{
  \begingroup
    \if?#1?
      \def\childdoctmp{\childdocforward{#2}}
    \else
      \def\childdoctmp{\childdocforwardprefix{#1}{#2}}
    \fi
    \expandafter
  \endgroup
  \childdoctmp
}
%    \end{macrocode}

%\iffalse
%</package>
%\fi
%
\endinput
|\\
|\childdocmain{|\textit{main}|}|\\
\end{tabular}
\end{center}
%
If |\jobname| does not match the argument \textit{main} of |\childdocmain|,
it is assumed that |\jobname| points to the child file to be compiled.
When using |\childdocmain| with the main file specified as argument,
it suffices to start a child file
with just |\input{|\textit{main}|}|
without loading of the package and using |\childdocof|.
If instead all processing is done
with the appropriate \textsf{childdoc} directives,
the argument of \textit{main} of |\childdocmain| can be empty.

An alternative version of the command line processing described
in \secref{sec:commandline} using the detection mechanism reads:
%
\begin{center}
|... -jobname "|\textit{target}|" "|[\textit{flags}]%
[|\def\jobname{|\textit{dest}|}|]|\input{|\textit{main}|}"|
\end{center}

%%%%%%%%%%%%%%%%%%%%%%%%%%%%%%%%%%%%%%%%%%%%%%%%%%%%%%%%%%%%%%%%%%%%%%%%%%%%%%%%
\subsection{Manual Code}
\label{sec:manual}

In case one cannot be certain whether the definitions file |childdoc.def|
is installed on the target \TeX{} distribution
and one prefers not to ship it,
it is conceivable to paste a few relevant commands into the sources.

To that end, drop all statements |% \iffalse
%
% childdoc.dtx Copyright (C) 2017-2018 Niklas Beisert
%
% This work may be distributed and/or modified under the
% conditions of the LaTeX Project Public License, either version 1.3
% of this license or (at your option) any later version.
% The latest version of this license is in
%   http://www.latex-project.org/lppl.txt
% and version 1.3 or later is part of all distributions of LaTeX
% version 2005/12/01 or later.
%
% This work has the LPPL maintenance status `maintained'.
%
% The Current Maintainer of this work is Niklas Beisert.
%
% This work consists of the files childdoc.dtx and childdoc.ins
% and the derived files childdoc.def and cdocsamp.tex with
% cdocsch1.tex, cdocsch2.tex, cdocsdrf.tex, cdocsfn1.tex, cdocsfn2.tex.
%
%<package>\ifdefined\childdocmain\endinput\fi
%<package>\ProvidesFile{childdoc.def}[2018/12/30 v2.0 child document driver]
%<samplemain>\ProvidesFile{cdocsamp.tex}[2018/12/30 v2.0 sample for childdoc]
%<*driver>
%\ProvidesFile{childdoc.drv}[2018/12/30 v2.0 childdoc reference manual file]
\PassOptionsToClass{10pt,a4paper}{article}
\documentclass{ltxdoc}

\usepackage[margin=35mm]{geometry}
\usepackage{hyperref}
\usepackage{hyperxmp}
\usepackage[usenames]{color}

\hypersetup{colorlinks=true}
\hypersetup{pdfstartview=FitH}
\hypersetup{pdfpagemode=UseNone}
\hypersetup{pdfsource={}}
\hypersetup{pdflang={en-UK}}
\hypersetup{pdfcopyright={Copyright 2017-2018 Niklas Beisert.
  This work may be distributed and/or modified under the
  conditions of the LaTeX Project Public License, either version 1.3
  of this license or (at your option) any later version.}}
\hypersetup{pdflicenseurl={http://www.latex-project.org/lppl.txt}}
\hypersetup{pdfcontactaddress={ETH Zurich, ITP, HIT K,
  Wolfgang-Pauli-Strasse 27}}
\hypersetup{pdfcontactpostcode={8093}}
\hypersetup{pdfcontactcity={Zurich}}
\hypersetup{pdfcontactcountry={Switzerland}}
\hypersetup{pdfcontactemail={nbeisert@itp.phys.ethz.ch}}
\hypersetup{pdfcontacturl={http://people.phys.ethz.ch/\xmptilde nbeisert/}}

\newcommand{\secref}[1]{\hyperref[#1]{section \ref*{#1}}}

\parskip1ex
\parindent0pt
\let\olditemize\itemize
\def\itemize{\olditemize\parskip0pt}

\begin{document}

\title{The \textsf{childdoc} Package}
\hypersetup{pdftitle={The childdoc Package}}
\author{Niklas Beisert\\[2ex]
  Institut f\"ur Theoretische Physik\\
  Eidgen\"ossische Technische Hochschule Z\"urich\\
  Wolfgang-Pauli-Strasse 27, 8093 Z\"urich, Switzerland\\[1ex]
  \href{mailto:nbeisert@itp.phys.ethz.ch}
  {\texttt{nbeisert@itp.phys.ethz.ch}}}
\hypersetup{pdfauthor={Niklas Beisert}}
\hypersetup{pdfsubject={Manual for the LaTeX2e Package childdoc}}
\date{30 December 2018, \textsf{v2.0}}
\maketitle

\begin{abstract}\noindent
\textsf{childdoc} is a \LaTeXe{} package
that enables the direct compilation
of document sections included by |\include|
to individual files.
\end{abstract}

\begingroup
\parskip0ex
\tableofcontents
\endgroup

%%%%%%%%%%%%%%%%%%%%%%%%%%%%%%%%%%%%%%%%%%%%%%%%%%%%%%%%%%%%%%%%%%%%%%%%%%%%%%%%
%%%%%%%%%%%%%%%%%%%%%%%%%%%%%%%%%%%%%%%%%%%%%%%%%%%%%%%%%%%%%%%%%%%%%%%%%%%%%%%%
\section{Introduction}

\LaTeX{} provides a mechanism to structure a large document (such as a book)
into a main file and several child files (containing the chapters)
using the |\include| command.
This mechanism is beneficial for documents
which span hundreds of pages in order to
make the source file(s) more manageable.
Moreover, compilation can be restricted to
selected child files by means of the |\includeonly| command.
The latter feature can be used to reduce the compilation time while editing
(this was significantly more useful in the earlier days of \LaTeX{})
or to generate a smaller document which is easier to navigate.
Another application of |\includeonly| is to generate
documents consisting of selected parts of the complete document.

However, there are a few drawbacks of the plain |\include| mechanism:
\begin{itemize}
\item
The child files cannot be compiled on their own,
they can only be compiled via the main file.
A naive editing environment
(such as a text editor with an option
to have the current file processed by \LaTeX)
may require one to switch to the main file before compiling;
attempting to compile the child file produces errors.
\item
The main file must be modified (each time)
to adjust the |\includeonly| command
to the present needs. This easily leaves the main file in a messy state.
\item
The generated document will always carry the filename
of the main document. This is inconvenient if
several child files are to be compiled and
to be kept for distribution.
\end{itemize}

The present package provides a simple interface
to make child files individually compilable by \LaTeX{}.
Compiling a child file then has the same effect as compiling
the main file with an |\includeonly| command
to select the appropriate child.
Moreover the generated document will carry the name of the child
rather than the main file.
This resolves all three above issues.

This feature is meant to make the editing of books,
thesis documents and lecture notes somewhat more convenient.
However, the package can also be used efficiently for
composing a series of documents (such as exercise sheets)
which are typically distributed individually.
It then assists the author in generating the individual documents
(potentially in different versions)
as well as a document containing the collected series.
Another application is in developing style files
or other kinds of included material
where compilation of the style file could redirect
to a sample or test file.

%%%%%%%%%%%%%%%%%%%%%%%%%%%%%%%%%%%%%%%%%%%%%%%%%%%%%%%%%%%%%%%%%%%%%%%%%%%%%%%%
%%%%%%%%%%%%%%%%%%%%%%%%%%%%%%%%%%%%%%%%%%%%%%%%%%%%%%%%%%%%%%%%%%%%%%%%%%%%%%%%
\section{Usage}

First of all, the package \textsf{childdoc} is \emph{not} a standard
\LaTeXe{} |.sty| style file! Therefore it needs to be invoked in
a non-standard way.

%%%%%%%%%%%%%%%%%%%%%%%%%%%%%%%%%%%%%%%%%%%%%%%%%%%%%%%%%%%%%%%%%%%%%%%%%%%%%%%%
\subsection{Included Files}
\label{sec:include}

%%%%%%%%%%%%%%%%%%%%%%%%%%%%%%%%%%%%%%%%
\DescribeMacro{\childdocmain}
To use the package, add the commands
\begin{center}
\begin{tabular}{l}
|\input{childdoc.def}|\\
|\childdocmain{}|\\
\end{tabular}
\end{center}
at the very top of the main \LaTeX{} file,
in particular \emph{before} the |\documentclass| statement!
The argument of |\childdocmain| should be left empty
(but it must be present).

%%%%%%%%%%%%%%%%%%%%%%%%%%%%%%%%%%%%%%%%
\DescribeMacro{\childdocof}
Furthermore, add the commands
\begin{center}
\begin{tabular}{l}
|\input{childdoc.def}|\\
|\childdocof{|\textit{main}|}|\\
\end{tabular}
\end{center}
at the top of every child file \textit{child}
which is included by |\include{|\textit{child}|}|
from within the main file
(or at least for those files to be compiled individually).
The argument \textit{main} must be the filename of the main file.

There are a couple of
considerations in setting up the main and child documents:

%%%%%%%%%%%%%%%%%%%%%%%%%%%%%%%%%%%%%%%%
\paragraph{Restrictions.}

Please note the following restrictions:
\begin{itemize}
\item
|\childdocmain| must be called with one argument \textit{main}
to ensure compatibility with earlier version of the package.
It must either be empty (|\childdocmain{}|)
or precisely match the filename of the main file in which it is specified.
See \secref{sec:detection} for further information.
\item
The filename \textit{main} must be specified without the |.tex| extension.
\item
The filename \textit{main} is case sensitive
(even in case-insensitive file systems)
due to internal string comparison.
\item
The argument \textit{main} should be fully expanded, it cannot be a macro.
\item
Subdirectories and special characters should be avoided in filenames.
\item
The command |\childdocmain{|\textit{main}|}| must be followed by a whitespace.
It should not be followed immediately by another command
or by a comment mark `|%|'.
This is because the \TeX{} parser reads the token immediately following
the argument of |\childdocmain| and puts it
at the beginning of every child section;
however, a white\-space is ignored.
\end{itemize}

%%%%%%%%%%%%%%%%%%%%%%%%%%%%%%%%%%%%%%%%
\paragraph{Content of Main File.}

It is advisable to place all content in the child files included by |\include|.
Any output contained in the main file will appear in all child documents
unless suppressed manually;
it cannot be suppressed automatically by the |\includeonly| directive
and thus should normally be avoided.
A method to include some content in the main file
by means of conditional processing is described in \secref{sec:conditional}.

%%%%%%%%%%%%%%%%%%%%%%%%%%%%%%%%%%%%%%%%
\paragraph{Page Numbering.}

When only a part of the document is compiled,
the appropriate numbering of pages
(as well as other status parameters)
is determined from the |.aux| files.
The latter contain information from previous passes.
However this information needs to propagate through
all intermediate child documents.
Therefore the page numbering in child documents may well
be inconsistent until the complete document is compiled at least once.

A useful (if unconventional) way to always ensure a consistent
page numbering is to restart the numbering in each child document
and denote the pages by `\textit{child}|.|\textit{page}'
where \textit{child} represents the chapter/section number of the child file.
This can be achieved by the command
|\numberwithin{page}{|\textit{child}|}|
of the \textsf{amsmath} package
where \textit{child} can be |chapter| or |section|
depending on the chosen structuring.
Alternatively, one can modify the macro |\thepage| appropriately
and reset the counter |page| at the start of each child file.

%%%%%%%%%%%%%%%%%%%%%%%%%%%%%%%%%%%%%%%%%%%%%%%%%%%%%%%%%%%%%%%%%%%%%%%%%%%%%%%%
\subsection{Conditional Processing}
\label{sec:conditional}

The package provides a mechanism to compile different versions
of a document. To customise the versions further some conditional processing
can come in handy to distinguish which version is being compiled.
The package provides two macros to describe the compilation context:

%%%%%%%%%%%%%%%%%%%%%%%%%%%%%%%%%%%%%%%%
\DescribeMacro{\ifchilddoc}
The conditional |\ifchilddoc| distinguishes between the compilation of
child documents and the main document:
%
\begin{center}
|\ifchilddoc |\textit{child-code}| |[|\||else |\textit{main-code}]| \||fi|
\end{center}

%%%%%%%%%%%%%%%%%%%%%%%%%%%%%%%%%%%%%%%%
\DescribeMacro{\childdocname}
\DescribeMacro{\childdocjob}
The macro |\childdocname| contains the filename (without extension)
of the main or child file being processed.
Note that |\childdocjob| will always contain the name of the main file.

%%%%%%%%%%%%%%%%%%%%%%%%%%%%%%%%%%%%%%%%
\paragraph{Title Page.}

Conditional processing can be used to include a title or banner page
in the main document when proper precautions are taken.
Importantly, the code in the main file should ensure that the page counter
(as well as other status parameters which are stored in the |.aux| files)
takes the same value after the conditional processing.
Otherwise the page numbers may take divergent values
depending on which part is compiled.

For example, a title page could be declared by:
%
\begin{center}
\begin{tabular}{l}
|\ifchilddoc\||else|\\
|\addtocounter{page}{-1}|\\
\textit{code for title page}\\
|\newpage|\\
|\||fi|
\end{tabular}
\end{center}
%
A banner page for the child documents can be generated by:
%
\begin{center}
\begin{tabular}{l}
|\ifchilddoc|\\
|\addtocounter{page}{-1}|\\
\textit{code for banner page}\\
|\newpage|\\
|\||fi|
\end{tabular}
\end{center}
%
Here one could write a message such as:
\begin{center}
|This is the part \childdocname{} of \childdocjob{}.|
\end{center}

%%%%%%%%%%%%%%%%%%%%%%%%%%%%%%%%%%%%%%%%%%%%%%%%%%%%%%%%%%%%%%%%%%%%%%%%%%%%%%%%
\subsection{Flags}
\label{sec:flags}

The package makes it easy to generate different versions
of the main or child documents.
To this end compilation flags can be defined
and assigned different default values.
They will be particularly useful in conjunction
with the forwarding mechanism described in \secref{sec:forward}.

For example, it may be useful to have a flag |\version|
which can be set to |draft| or |final|.
The document source will contain some conditional code
depending on the value of |\version|.
Suppose further, the flag should default to |final| for the main file
and to |draft| for child files
which is a natural assignment for editing the document.
This is achieved by placing the following code
in the preamble of the main document
(below the |\childdocmain| directive):
%
\begin{center}
\begin{tabular}{l}
|\ifchilddoc|\\
|\providecommand{\version}{draft}|\\
|\||else|\\
|\providecommand{\version}{final}|\\
|\||fi|
\end{tabular}
\end{center}
%
The definition by |\providecommand| makes sure
that previous definitions are not overwritten.
Further statements |\providecommand{\version}{...}|
can thus be added before the above code to override it.

For the main file, one might add a line
(between |\childdocmain| and the above block)
%
\begin{center}
|%\ifchilddoc\||else\providecommand{\version}{draft}\||fi|
\end{center}
%
which can be uncommented to produce a draft version.
Likewise one can add a line to the very top of a child file
(above the |\childdocof{|\textit{main}|}| directive)
%
\begin{center}
|%\providecommand{\version}{final}|
\end{center}
%
which can be uncommented to produce the final version of this child document.

%%%%%%%%%%%%%%%%%%%%%%%%%%%%%%%%%%%%%%%%%%%%%%%%%%%%%%%%%%%%%%%%%%%%%%%%%%%%%%%%
\subsection{Forwarding}
\label{sec:forward}

Different versions of the main or child documents
using compilation flags as described in \secref{sec:flags}
can be (permanently) stored in different files
for convenient compilation, viewing and distribution.
To this end, the package defines a command
to pass on compilation to a different file:

%%%%%%%%%%%%%%%%%%%%%%%%%%%%%%%%%%%%%%%%
\DescribeMacro{\childdocforward}
The command |\childdocforward| redirects processing to
another source file:
%
\begin{center}
\begin{tabular}{l}
|\input{childdoc.def}|\\
|\childdocforward[|\textit{main}|]{|\textit{dest}|}|\\
\end{tabular}
\end{center}
%
The argument \textit{dest} is the destination file
(without extension).
It should be the main file or one of the child files.
Note that further \textsf{childdoc} directives
such as |\childdocof| and |\childdocforward|
in the indicated file will be processed in this form.
The optional argument \textit{main}
passes on directly to the main file \textit{main}
while pretending to compile the child \textit{dest}.
This form behaves as if \textit{dest}
issues |\childdocof{|\textit{main}|}| right away,
and no further \textsf{childdoc} directives will be processed.

%%%%%%%%%%%%%%%%%%%%%%%%%%%%%%%%%%%%%%%%
\DescribeMacro{\...prefix}
In the alternative form |\childdocforwardprefix|,
%
\begin{center}
\begin{tabular}{l}
|\input{childdoc.def}|\\
|\childdocforwardprefix[|\textit{main}|]{|\textit{prefix}|}{|\textit{dest}|}|
\end{tabular}
\end{center}
%
the destination file is determined by a pattern
depending on the current file:
To make this work, the current file must be called
`{\textit{prefix}\hspace{0.2em}\textit{suffix}}'
with \textit{prefix} matching precisely the argument.
Processing is then passed on to the file
`{\textit{dest}\hspace{0.2em}\textit{suffix}}'.
Surely, the same effect is achieved by
directly specifying the
argument `{\textit{dest}\hspace{0.2em}\textit{suffix}}'
in the first form.
However, that requires to set up a different file
for each child. With the alternative form of the command
all these files can have exactly the same content
which simplifies setting them up and maintaining them.

For example, the following file |draft.tex|
with a compilation flag |\version| as described in \secref{sec:flags}
compiles the main document as a draft:
%
\begin{center}
\begin{tabular}{l}
|\def\version{draft}|\\
|\input{childdoc.def}|\\
|\childdocforward{|\textit{main}|}|
\end{tabular}
\end{center}
%
Likewise, the following files |final|\textit{nn}|.tex|
compile the final version of the child document
|child|\textit{nn}|.tex|:
%
\begin{center}
\begin{tabular}{l}
|\def\version{final}|\\
|\input{childdoc.def}|\\
|\childdocforwardprefix{final}{child}|
\end{tabular}
\end{center}
%

Note that when several versions of a main file and/or of each child file
are to be generated, it may be convenient to set up a |Makefile| or
shell script to automatise the process.

%%%%%%%%%%%%%%%%%%%%%%%%%%%%%%%%%%%%%%%%%%%%%%%%%%%%%%%%%%%%%%%%%%%%%%%%%%%%%%%%
\subsection{Command Line Processing}
\label{sec:commandline}

The effect of redirection files can also be achieved by invoking
the \LaTeX{} compiler with a more elaborate command line.
Most conveniently this should be done as part
of a shell script or a |Makefile|.

When using \textsf{childdoc} in the main file, the following
command lines effectively perform a redirection
(note that depending on the shell being used,
backslashes may have to be doubled: `|\|' $\to$ `|\\|'):
%
\begin{center}
|... -jobname "|\textit{target}|" |\\|"|[\textit{flags}]%
|\input{childdoc.def}\childdocforward[|\textit{main}|]{|\textit{dest}|}"|
\end{center}
%
Here \textit{target} is the name of the output file,
\textit{main} is the name of the main file
and \textit{dest} is the name of the main or child file to be processed
(all filenames without extensions).
The optional argument \textit{main} can be omitted
if \textit{main} matches \textit{dest}.
Optionally, compilation \textit{flags} can be defined via |\def| commands.
This command line makes the \TeX{} engine believe
it is compiling the file \textit{target}
whose content is specified as the latter parameter.
The provided code then forwards the processing to
\textit{main} or \textit{dest} as described in \secref{sec:forward}.

%%%%%%%%%%%%%%%%%%%%%%%%%%%%%%%%%%%%%%%%%%%%%%%%%%%%%%%%%%%%%%%%%%%%%%%%%%%%%%%%
\subsection{Include by Input}
\label{sec:input}

Including child documents by |\include| has some restrictions by design.
Most notably, the content of a child document always occupies
its own set of pages; pages cannot be shared between child documents.
Usually, this behaviour makes perfect sense
because each child document contain an essential part of the document.
However, in some situations it may be desirable to compose
a document from a collection of parts
without having mandatory page breaks between then.
For this case, the package
provides a mechanism to include parts
by |\input| which can also be processed individually.
However, by construction this mechanism
requires manual handling of the content to be output.

%%%%%%%%%%%%%%%%%%%%%%%%%%%%%%%%%%%%%%%%
\DescribeMacro{\ifchilddocmanual}
The main file should be prepared as usual, see \secref{sec:include}.
However, the document body must make a distinction
between processing of an individual part and of the main document, e.g.:
%
\begin{center}
\begin{tabular}{l}
|\ifchilddocmanual|\\
|\input{\childdocname}|\\
|\||else|\\
\textit{document body with }|\input{|\textit{part}|}|\\
|\||fi|
\end{tabular}
\end{center}
%
The conditional |\ifchilddocmanual| is true whenever
a part to be included by |\input| is being compiled,
and the name of the part is stored in |\childdocname|.

%%%%%%%%%%%%%%%%%%%%%%%%%%%%%%%%%%%%%%%%
\DescribeMacro{\childdocby}
Each part to be included by |\input| should start with:
%
\begin{center}
\begin{tabular}{l}
|\input{childdoc.def}|\\
|\childdocby{|\textit{main}|}|\\
\end{tabular}
\end{center}
%
The directive |\childdocby| is similar to |\childdocof|
described in \secref{sec:include},
but the subsequent selection of content must be done manually.
To that end, both |\ifchilddoc| and |\ifchilddocmanual|
will be true upon processing of a part,
and the name of the part is stored in |\childdocname|.
Note that |\jobname| will be set to the filename of the current part
so that each part receives an individual |.aux| file
that does not interfere with the |.aux| file(s) of the main document.
This behaviour can be altered by the alternative form
|\childdocby[*]{|\textit{main}|}| (with a non-empty optional argument)
which uses the |.aux| file of the main document
by setting |\jobname| to \textit{main}.

%%%%%%%%%%%%%%%%%%%%%%%%%%%%%%%%%%%%%%%%%%%%%%%%%%%%%%%%%%%%%%%%%%%%%%%%%%%%%%%%
\subsection{Driver Development}
\label{sec:driver}

The \textsf{childdoc} mechanism can also be use for the development
of definition files such as \LaTeX{} styles or classes.
This case differs from the above setup with multiple parts
included by |\include| in that no |\includeonly| should be invoked.
This can be achieved by starting the include file
(before |\ProvidesPackage|) with:
%
\begin{center}
\begin{tabular}{l}
|\input{childdoc.def}|\\
|\childdocforward{|\textit{main}|}|\\
\end{tabular}
\end{center}
%
or alternatively with:
%
\begin{center}
\begin{tabular}{l}
|\input{childdoc.def}|\\
|\childdocby{|\textit{main}|}|\\
\end{tabular}
\end{center}
%
Both forms have slightly different effects as described above.
The main file is prepared as usual, see \secref{sec:include}.

%%%%%%%%%%%%%%%%%%%%%%%%%%%%%%%%%%%%%%%%%%%%%%%%%%%%%%%%%%%%%%%%%%%%%%%%%%%%%%%%
\subsection{Legacy Detection}
\label{sec:detection}

The directive |\childdocmain| in the main file can detect
whether the complete document or merely a child is to be compiled
even without using the directive |\childdocof|.
This method is deprecated because it is less robust
and there is no compelling reason to use it;
it is merely provided for backward compatibility
and it may be removed in future versions.

If the detection mechanism is to be used,
it is mandatory to correctly specify
the filename of the main file as the argument of |\childdocmain|:
%
\begin{center}
\begin{tabular}{l}
|\input{childdoc.def}|\\
|\childdocmain{|\textit{main}|}|\\
\end{tabular}
\end{center}
%
If |\jobname| does not match the argument \textit{main} of |\childdocmain|,
it is assumed that |\jobname| points to the child file to be compiled.
When using |\childdocmain| with the main file specified as argument,
it suffices to start a child file
with just |\input{|\textit{main}|}|
without loading of the package and using |\childdocof|.
If instead all processing is done
with the appropriate \textsf{childdoc} directives,
the argument of \textit{main} of |\childdocmain| can be empty.

An alternative version of the command line processing described
in \secref{sec:commandline} using the detection mechanism reads:
%
\begin{center}
|... -jobname "|\textit{target}|" "|[\textit{flags}]%
[|\def\jobname{|\textit{dest}|}|]|\input{|\textit{main}|}"|
\end{center}

%%%%%%%%%%%%%%%%%%%%%%%%%%%%%%%%%%%%%%%%%%%%%%%%%%%%%%%%%%%%%%%%%%%%%%%%%%%%%%%%
\subsection{Manual Code}
\label{sec:manual}

In case one cannot be certain whether the definitions file |childdoc.def|
is installed on the target \TeX{} distribution
and one prefers not to ship it,
it is conceivable to paste a few relevant commands into the sources.

To that end, drop all statements |\input{childdoc.def}|
and perform the replacements as outlined below.
Instead of |\childdocmain{|\textit{main}|}| add the following code
to the top of the main file:
%
\begin{center}
\begin{tabular}{l}
|\||ifdefined\childdocname\endinput\||fi\newif\ifchilddoc|\\
|\edef\childdocname{\scantokens\expandafter{\jobname\noexpand}}|\\
|\def\childdocmain{|\textit{main}|}\||ifx\childdocmain\childdocname\||else|\\
|\childdoctrue\includeonly{\childdocname}\let\jobname\childdocmain\||fi|\\
\end{tabular}
\end{center}
%
Instead of |\childdocof{|\textit{main}|}| just include the main file
at the top of each child file:
%
\begin{center}
|\input{|\textit{main}|}|
\end{center}
%
A simple redirection |\childdocforward{|\textit{dest}|}| is achieved by:
%
\begin{center}
|\def\jobname{|\textit{dest}|}\input{\jobname}|
\end{center}
%
The redirection with prefix
|\childdocforwardprefix[|\textit{prefix}|]{|\textit{dest}|}|
is accomplished by:
%
\begin{center}
\begin{tabular}{l}
|{\edef\jobname{\scantokens\expandafter{\jobname\noexpand}}|\\
|\def\redirectjob |\textit{prefix}|#1~~~{\gdef\jobname{|\textit{dest}|#1}}|\\
|\expandafter\redirectjob\jobname~~~}\input{\jobname}|
\end{tabular}
\end{center}

In an alternative approach,
child documents can be compiled by a specific command line
without additional code or specific definitions:
%
\begin{center}
|... -jobname "|\textit{target}|" "|[\textit{flags}]%
|\includeonly{|\textit{dest}|}\input{|\textit{main}|}"|
\end{center}
%

%%%%%%%%%%%%%%%%%%%%%%%%%%%%%%%%%%%%%%%%%%%%%%%%%%%%%%%%%%%%%%%%%%%%%%%%%%%%%%%%
%%%%%%%%%%%%%%%%%%%%%%%%%%%%%%%%%%%%%%%%%%%%%%%%%%%%%%%%%%%%%%%%%%%%%%%%%%%%%%%%
\section{Information}

%%%%%%%%%%%%%%%%%%%%%%%%%%%%%%%%%%%%%%%%%%%%%%%%%%%%%%%%%%%%%%%%%%%%%%%%%%%%%%%%
\subsection{Copyright}

Copyright \copyright{} 2017--2018 Niklas Beisert

This work may be distributed and/or modified under the
conditions of the \LaTeX{} Project Public License, either version 1.3
of this license or (at your option) any later version.
The latest version of this license is in
  \url{http://www.latex-project.org/lppl.txt}
and version 1.3 or later is part of all distributions of \LaTeX{}
version 2005/12/01 or later.

This work has the LPPL maintenance status `maintained'.

The Current Maintainer of this work is Niklas Beisert.

This work consists of the files |README.txt|, |childdoc.ins| and |childdoc.dtx|
as well as the derived files |childdoc.def|, |cdocsamp.tex|
with |cdocsch1.tex|, |cdocsch2.tex|, |cdocspt3.tex|, |cdocspt4.tex|,
|cdocsdrf.tex|, |cdocsfn1.tex|, |cdocsfn2.tex|
as well as |childdoc.pdf|.

%%%%%%%%%%%%%%%%%%%%%%%%%%%%%%%%%%%%%%%%%%%%%%%%%%%%%%%%%%%%%%%%%%%%%%%%%%%%%%%%
\subsection{Files and Installation}

The package consists of the files:
%
\begin{center}
\begin{tabular}{ll}
    |README.txt|   & readme file \\
    |childdoc.ins| & installation file \\
    |childdoc.dtx| & source file \\
    |childdoc.def| & definition file \\
    |cdocsamp.tex| & sample main file \\
    |cdocsch1.tex| & sample include file \\
    |cdocsch2.tex| & sample include file \\
    |cdocspt3.tex| & sample part file \\
    |cdocspt4.tex| & sample part file \\
    |cdocsdrf.tex| & sample redirection file \\
    |cdocsfn1.tex| & sample redirection file \\
    |cdocsfn2.tex| & sample redirection file \\
    |childdoc.pdf| & manual
\end{tabular}
\end{center}
%
The distribution consists of the files
|README.txt|, |childdoc.ins| and |childdoc.dtx|.
%
\begin{itemize}
\item
Run (pdf)\LaTeX{} on |childdoc.dtx|
to compile the manual |childdoc.pdf| (this file).
\item
Run \LaTeX{} on |childdoc.ins| to create the definitions file |childdoc.def|
and the sample |cdocsamp.tex| with include files
|cdocsch1.tex|, |cdocsch2.tex|, |cdocspt3.tex|, |cdocspt4.tex|,
|cdocsdrf.tex|, |cdocsfn1.tex|, |cdocsfn2.tex|.
Then copy the file |childdoc.def| to an appropriate directory of your \LaTeX{}
distribution, e.g.\ \textit{texmf-root}|/tex/latex/childdoc|.
\end{itemize}

%%%%%%%%%%%%%%%%%%%%%%%%%%%%%%%%%%%%%%%%%%%%%%%%%%%%%%%%%%%%%%%%%%%%%%%%%%%%%%%%
\subsection{Related CTAN Packages}

There are several other packages which offer a similar functionality:
%
\begin{itemize}
\item
The packages
\href{http://ctan.org/pkg/docmute}{\textsf{docmute}},
\href{http://ctan.org/pkg/includex}{\textsf{includex}} and
\href{http://ctan.org/pkg/standalone}{\textsf{standalone}}
provide commands to include only the document body of
a child file thus allowing both files to be compiled individually.
\item
The packages \href{http://ctan.org/pkg/subdocs}{\textsf{subdocs}}
and \href{http://ctan.org/pkg/subfiles}{\textsf{subfiles}}
provide structures in which the main and child documents can be
encapsulated and allowing them to be compiled individually.
The inclusion mechanism is different from the conventional |\include|.
\item
The package \href{http://ctan.org/pkg/combine}{\textsf{combine}}
is an elaborate solution to combine several documents into one.
\end{itemize}
%
See also the CTAN topic \href{http://ctan.org/topic/subdocs}{\textsf{subdocs}}
for further related packages.
The present package differs from the above solutions in that
a document structure constructed with the conventional |\include| mechanism
just needs two extra commands at the top of every file
such that all constituent files can be compiled individually.

%%%%%%%%%%%%%%%%%%%%%%%%%%%%%%%%%%%%%%%%%%%%%%%%%%%%%%%%%%%%%%%%%%%%%%%%%%%%%%%%
%\subsection{Feature Suggestions}
%
%The following is a list of features which may be useful for future
%versions of this package:
%%
%\begin{itemize}
%\item
%\ldots
%\end{itemize}

%%%%%%%%%%%%%%%%%%%%%%%%%%%%%%%%%%%%%%%%%%%%%%%%%%%%%%%%%%%%%%%%%%%%%%%%%%%%%%%%
\subsection{Revision History}

%%%%%%%%%%%%%%%%%%%%%%%%%%%%%%%%%%%%%%%%
\paragraph{v2.0:} 2018/12/30

\begin{itemize}
\item
immediate forward processing
\item
added |\childdocby| mechanism
\item
manual restructured
\end{itemize}

%%%%%%%%%%%%%%%%%%%%%%%%%%%%%%%%%%%%%%%%
\paragraph{v1.6:} 2018/01/17

\begin{itemize}
\item
application for development of include files
\item
corrections to manual
\end{itemize}

%%%%%%%%%%%%%%%%%%%%%%%%%%%%%%%%%%%%%%%%
\paragraph{v1.5:} 2017/05/21

\begin{itemize}
\item
more complete structuring introduced
\item
|\childdocof| introduced
\item
|\childdoc| renamed to |\childdocmain|
\item
|\childredirect| renamed to |\childdocforward| and |\childdocforwardprefix|
and functionality expanded
\end{itemize}

%%%%%%%%%%%%%%%%%%%%%%%%%%%%%%%%%%%%%%%%
\paragraph{v1.0:} 2017/04/27

\begin{itemize}
\item
manual and install package
\item
first version published on CTAN
\end{itemize}

%%%%%%%%%%%%%%%%%%%%%%%%%%%%%%%%%%%%%%%%
\paragraph{v0.6:} 2017/04/26

\begin{itemize}
\item
redirection mechanism added
\end{itemize}

%%%%%%%%%%%%%%%%%%%%%%%%%%%%%%%%%%%%%%%%
\paragraph{v0.5:} 2017/04/26

\begin{itemize}
\item
functionality in definition file
\end{itemize}


%%%%%%%%%%%%%%%%%%%%%%%%%%%%%%%%%%%%%%%%%%%%%%%%%%%%%%%%%%%%%%%%%%%%%%%%%%%%%%%%
%%%%%%%%%%%%%%%%%%%%%%%%%%%%%%%%%%%%%%%%%%%%%%%%%%%%%%%%%%%%%%%%%%%%%%%%%%%%%%%%
%%%%%%%%%%%%%%%%%%%%%%%%%%%%%%%%%%%%%%%%%%%%%%%%%%%%%%%%%%%%%%%%%%%%%%%%%%%%%%%%
\appendix

\settowidth\MacroIndent{\rmfamily\scriptsize 000\ }

 \DocInput{childdoc.dtx}

\end{document}
%</driver>
% \fi
%
% %%%%%%%%%%%%%%%%%%%%%%%%%%%%%%%%%%%%%%%%%%%%%%%%%%%%%%%%%%%%%%%%%%%%%%%%%%%%%%
% %%%%%%%%%%%%%%%%%%%%%%%%%%%%%%%%%%%%%%%%%%%%%%%%%%%%%%%%%%%%%%%%%%%%%%%%%%%%%%
% \section{Sample}
%\iffalse
%<*samplemain>
%\fi
%
% The following presents a sample document
% with two chapters, two parts, a title page,
% a compile flag as well as three forwarding files to set the flag.
% It consists of eight |.tex| files:
% \begin{center}
% \begin{tabular}{ll}
% |cdocsamp.tex|&main file\\
% |cdocsch1.tex|&include file for chapter 1\\
% |cdocsch2.tex|&include file for chapter 2\\
% |cdocspt3.tex|&include file for part 3\\
% |cdocspt4.tex|&include file for part 4\\
% |cdocsdrf.tex|&forwarding file for main file in draft mode\\
% |cdocsfi1.tex|&forwarding file for final version of chapter 1\\
% |cdocsfi2.tex|&forwarding file for final version of chapter 2\\
% \end{tabular}
% \end{center}
% Each of the eight files can be compiled directly by the \LaTeX{} compiler.
%
% %%%%%%%%%%%%%%%%%%%%%%%%%%%%%%%%%%%%%%
% \paragraph{Main File.}
%
% The main file is called |cdocsamp.tex|.
%
% Load the \textsf{childdoc} definitions and
% declare the filename for the main document:
%    \begin{macrocode}
\input{childdoc.def}
\childdocmain{}
%    \end{macrocode}

% Optional override for |\version| flag:
%    \begin{macrocode}
%%\ifchilddoc\else\providecommand{\version}{draft}\fi
%    \end{macrocode}

% Define the default values for the |\version| flag
% (|final| for the main file and |draft| for childs):
%    \begin{macrocode}
\ifchilddoc
\providecommand{\version}{draft}
\else
\providecommand{\version}{final}
\fi
%    \end{macrocode}

% Load the standard document class:
%    \begin{macrocode}
\documentclass[12pt]{article}
%    \end{macrocode}

% Start the document body:
%    \begin{macrocode}
\begin{document}
%    \end{macrocode}

% Declare a title page.
% Print title, part of document being processed and version flag:
%    \begin{macrocode}
\addtocounter{page}{-1}
\begin{center}
{\LARGE\bfseries{}childdoc example\par}
\vspace{1cm}
\ifchilddoc
\ifchilddocmanual part\else chapter\fi:
`\childdocname' of `\childdocjob'\par
\else
main document: `\childdocjob'\par
\fi
version: \version\par
\end{center}
\newpage
%    \end{macrocode}

% Manually include selected file,
% otherwise process as usual:
%    \begin{macrocode}
\ifchilddocmanual
\section*{part `\childdocname'}
\input{\childdocname}
\else
%    \end{macrocode}

% Include the two chapters:
%    \begin{macrocode}
\include{cdocsch1}
\include{cdocsch2}
%    \end{macrocode}

% Include the two parts unless only chapters should be displayed:
%    \begin{macrocode}
\ifchilddoc\else
\section{part three}
\input{cdocspt3}
\section{part four}
\input{cdocspt4}
\fi
%    \end{macrocode}

% Process as usual until here:
%    \begin{macrocode}
\fi
%    \end{macrocode}

% End of document body:
%    \begin{macrocode}
\end{document}
%    \end{macrocode}
%\iffalse
%</samplemain>
%\fi
%
% %%%%%%%%%%%%%%%%%%%%%%%%%%%%%%%%%%%%%%
% \paragraph{Chapter Include Files.}
%
% The include files are called |cdocsch1.tex| and |cdocsch2.tex|.
%
%\iffalse
%<*samplechap1|samplechap2>
%\fi

% Optional override for |\version| flag:
%    \begin{macrocode}
%%\providecommand{\version}{final}
%    \end{macrocode}

% Include the main document:
%    \begin{macrocode}
\input{childdoc.def}
\childdocof{cdocsamp}
%    \end{macrocode}

%\iffalse
%</samplechap1|samplechap2>
%\fi
%
%\iffalse
%<*samplechap1>
%\fi
% Some text for chapter 1:
%    \begin{macrocode}
\section{one}
some text in chapter one
%    \end{macrocode}

%\iffalse
%</samplechap1>
%\fi
% Some text for chapter 2:
%\iffalse
%<*samplechap2>
%\fi
%    \begin{macrocode}
\section{two}
more text in chapter two
%    \end{macrocode}

%\iffalse
%</samplechap2>
%\fi
%
% %%%%%%%%%%%%%%%%%%%%%%%%%%%%%%%%%%%%%%
% \paragraph{Part Include Files.}
%
% The include files are called |cdocspt3.tex| and |cdocspt4.tex|.
%
%\iffalse
%<*samplepart3|samplepart4>
%\fi

% Optional override for |\version| flag:
%    \begin{macrocode}
%%\providecommand{\version}{final}
%    \end{macrocode}

% Include the main document:
%    \begin{macrocode}
\input{childdoc.def}
\childdocby{cdocsamp}
%    \end{macrocode}

%\iffalse
%</samplepart3|samplepart4>
%\fi
%
%\iffalse
%<*samplepart3>
%\fi
% Some text for part 3:
%    \begin{macrocode}
some text in part three
%    \end{macrocode}

%\iffalse
%</samplepart3>
%\fi
% Some text for part 4:
%\iffalse
%<*samplepart4>
%\fi
%    \begin{macrocode}
more text in part four
%    \end{macrocode}

%\iffalse
%</samplepart4>
%\fi
%
% %%%%%%%%%%%%%%%%%%%%%%%%%%%%%%%%%%%%%%
% \paragraph{Forwarding for a Complete Draft.}
%
% The following forwarding file |cdocsdrf.tex|
% compiles the main document in draft mode:
%\iffalse
%<*sampledraft>
%\fi
%    \begin{macrocode}
\def\version{draft}
\input{childdoc.def}
\childdocforward{cdocsamp}
%    \end{macrocode}

%\iffalse
%</sampledraft>
%\fi
%
% %%%%%%%%%%%%%%%%%%%%%%%%%%%%%%%%%%%%%%
% \paragraph{Forwarding for Final Version of the Chapters.}
%
% The following forwarding files |cdocsfn1.tex| and |cdocsfn2.tex|
% (with identical content)
% compile the final versions of the child documents
% |cdocsch1.tex| and |cdocsch2.tex|, respectively:
%\iffalse
%<*samplefinal>
%\fi
%    \begin{macrocode}
\def\version{final}
\input{childdoc.def}
\childdocforwardprefix[cdocsamp]{cdocsfn}{cdocsch}
%    \end{macrocode}

%\iffalse
%</samplefinal>
%\fi
%
% %%%%%%%%%%%%%%%%%%%%%%%%%%%%%%%%%%%%%%
% \paragraph{Command Line Processing.}
%
% The following three command lines generate the output files
% |cdocscld|, |cdocscl1| and |cdocscl2|
% which should be identical to
% |cdocsdrf|, |cdocsch1| and |cdocsfn2|, respectively:
% \begin{center}
% \begin{tabular}{l}
% |latex -jobname cdocscld \|\\
% |  "\def\version{draft}\input{childdoc.def}\childdocforward{cdocsamp}"|\\
% |latex -jobname cdocscl1 \|\\
% |  "\input{childdoc.def}\childdocforward[cdocsamp]{cdocsch1}"|\\
% |latex -jobname cdocscl2 \|\\
% |  "\def\version{final}\input{childdoc.def}\childdocforward{cdocsch2}"|
% \end{tabular}
% \end{center}
% Note that the trailing backslash on each first line
% merely continues the input to the second line
% (for convenient cut ant paste).
% Furthermore, the command |latex| can be replaced by any
% of its alternative versions such as |pdflatex|.
%
% %%%%%%%%%%%%%%%%%%%%%%%%%%%%%%%%%%%%%%%%%%%%%%%%%%%%%%%%%%%%%%%%%%%%%%%%%%%%%%
% %%%%%%%%%%%%%%%%%%%%%%%%%%%%%%%%%%%%%%%%%%%%%%%%%%%%%%%%%%%%%%%%%%%%%%%%%%%%%%
% \section{Implementation}
%\iffalse
%<*package>
%\fi
%
% This section describes the definitions file |childdoc.def|.

% The definitions cannot be loaded using |\usepackage| or |\RequirePackage|
% which has a mechanism to prevent loading a style file more than once.
% When loading the definitions by means of |\input|
% multiple instances have to be prevented manually:
%\iffalse
%This code needs to be before the `\ProvidesFile' directive
%which is defined at the beginning of this file.
%Therefore it is also placed there and commented out here.
%</package>
%<*discard>
%\fi
%    \begin{macrocode}
\ifdefined\childdocmain\endinput\fi
%    \end{macrocode}
%\iffalse
%</discard>
%<*package>
%\fi
%
% \macro{\ifchilddoc}
% \macro{\ifchilddocmanual}
% The conditional |\ifchilddoc| tells whether a
% child (true) or main (false) document is being compiled.
% The conditional |\ifchilddocmanual| tells whether
% the |\includeonly| mechanism is used (false) or
% the selection of child files must be performed manually (true).
% The definitions initialise to false:
%    \begin{macrocode}
\newif\ifchilddoc
\newif\ifchilddocmanual
%    \end{macrocode}

% \macro{\childdocname}
% \macro{\childdocjob}
% The macro |\childdocname| stores the name of the main document
% to be compiled. The macro |\childdocjob| stores the name of
% the document on which the \LaTeX{} compiler was originally invoked.
% The content of |\jobname| cannot be compared
% to filenames specified in the source due to different catcodes.
% The following code rescans |\jobname|, stores the result
% in |\childdocname| and saves a copy in |\childdocjob|:
%    \begin{macrocode}
\edef\childdocname{\scantokens\expandafter{\jobname\noexpand}}
\let\childdocjob\childdocname
%    \end{macrocode}

% \macro{\childdocdisable}
% The macro |\childdocdisable| prevents the main file
% from being processed more than once.
% At this stage, the main document command |\childdocmain|
% is assumed to be called once again where it should do nothing.
% Any subsequent call to it should prevent
% a secondary processing of the main document
% It overwrites the forwarding commands
% |\childdocof| and |\childdocforward|
% with empty macros to prevent further inclusions of the main document:
%    \begin{macrocode}
\newcommand{\childdocdisable}
{
  \renewcommand{\childdocmain}[1]{\renewcommand{\childdocmain}[1]{\endinput}}
  \renewcommand{\childdocof}[1]{}
  \renewcommand{\childdocby}[2][]{}
  \renewcommand{\childdocforward}[2][]{}
  \renewcommand{\childdocdisable}{}
}
%    \end{macrocode}

% \macro{\childdocmain}
% The macro |\childdocmain| is to be called at the top of the main file
% with nothing or the main filename (without extension) as argument.
% First, it breaks loops.
% If the argument is not empty and does not match |\childdocname|
% (which is set by the first inclusion of |childdoc.def|),
% |\ifchilddoc| is set to true, |\includeonly| is applied to the child file
% and |\jobname| is set to the main file
% (for proper handling of |.aux| files):
%    \begin{macrocode}
\newcommand{\childdocmain}[1]
{
  \childdocdisable\childdocmain{}
  \if?#1?\else
    \begingroup
      \def\childdoctmp{#1}
      \ifx\childdoctmp\childdocname
        \def\childdoctmp{}
      \else
        \def\childdoctmp
        {
          \childdoctrue
          \includeonly{\childdocname}
          \def\childdocjob{#1}
          \def\jobname{#1}
        }
      \fi
      \expandafter
    \endgroup
    \childdoctmp
  \fi
}
%    \end{macrocode}

% \macro{\childdocof}
% The command |\childdocof| redirects
% compilation to the main file |#1|.
%    \begin{macrocode}
\newcommand{\childdocof}[1]
{
  \childdocdisable
  \childdoctrue
  \includeonly{\childdocname}
  \def\jobname{#1}
  \def\childdocjob{#1}
  \input{#1}
}
%    \end{macrocode}

% \macro{\childdocby}
% The command |\childdocby| ....
%    \begin{macrocode}
\newcommand{\childdocby}[2][]
{
  \childdocdisable
  \childdoctrue
  \childdocmanualtrue
  \if?#1?\else
    \def\jobname{#2}
  \fi
  \def\childdocjob{#2}
  \input{#2}
  \endinput
}
%    \end{macrocode}

% \macro{\childdocforward}
% The command |\childdocforward| redirects
% compilation to the main file or
% (if the optional argument is given) a child file.
% Parameters are set as if the main file
% or a child file starting with |\childdocof| was compiled.
% Then compilation is handed over to the main file:
%    \begin{macrocode}
\newcommand{\childdocforward}[2][]
{
  \begingroup
    \if?#1?
      \def\childdoctmp
      {
        \def\childdocname{#2}
        \def\childdocjob{#2}
        \def\jobname{#2}
        \input{#2}
        \endinput
      }
    \else
      \def\childdoctmp
      {
        \childdocdisable
        \def\childdocname{#2}
        \childdoctrue
        \includeonly{#2}
        \def\childdocjob{#1}
        \def\jobname{#1}
        \input{#1}
        \endinput
      }
    \fi
    \expandafter
  \endgroup
  \childdoctmp
}
%    \end{macrocode}

% \macro{\childdocforwardprefix}
% The command |\childdocforwardprefix| redirects
% compilation to the main or a child file by means of a pattern.
% The prefix |#1| in the current filename is replaced by |#2|
% and the suffix of the current filename is kept
% (it is assumed that the filename does not contain the substring `|~~~|'
% which is used as a delimiter).
% Compilation is handed over to the new file by |\childdocforward|:
%    \begin{macrocode}
\newcommand{\childdocforwardprefix}[3][]
{
  \begingroup
    \def\childdocextract #2##1~~~{\def\childdoctmp{\childdocforward[#1]{#3##1}}}
    \expandafter\childdocextract\childdocname~~~
    \expandafter
  \endgroup
  \childdoctmp
}
%    \end{macrocode}

% \macro{\childdoc}
% The deprecated macro |\childdoc| is a legacy version of |\childdocmain|:
%    \begin{macrocode}
\newcommand{\childdoc}{\childdocmain}
%    \end{macrocode}

% \macro{\childdocredirect}
% The deprecated macro |\childdocredirect| is a legacy version
% of |\childdocforward| and |\childdocforwardprefix|:
%    \begin{macrocode}
\newcommand{\childdocredirect}[2][]
{
  \begingroup
    \if?#1?
      \def\childdoctmp{\childdocforward{#2}}
    \else
      \def\childdoctmp{\childdocforwardprefix{#1}{#2}}
    \fi
    \expandafter
  \endgroup
  \childdoctmp
}
%    \end{macrocode}

%\iffalse
%</package>
%\fi
%
\endinput
|
and perform the replacements as outlined below.
Instead of |\childdocmain{|\textit{main}|}| add the following code
to the top of the main file:
%
\begin{center}
\begin{tabular}{l}
|\||ifdefined\childdocname\endinput\||fi\newif\ifchilddoc|\\
|\edef\childdocname{\scantokens\expandafter{\jobname\noexpand}}|\\
|\def\childdocmain{|\textit{main}|}\||ifx\childdocmain\childdocname\||else|\\
|\childdoctrue\includeonly{\childdocname}\let\jobname\childdocmain\||fi|\\
\end{tabular}
\end{center}
%
Instead of |\childdocof{|\textit{main}|}| just include the main file
at the top of each child file:
%
\begin{center}
|\input{|\textit{main}|}|
\end{center}
%
A simple redirection |\childdocforward{|\textit{dest}|}| is achieved by:
%
\begin{center}
|\def\jobname{|\textit{dest}|}\input{\jobname}|
\end{center}
%
The redirection with prefix
|\childdocforwardprefix[|\textit{prefix}|]{|\textit{dest}|}|
is accomplished by:
%
\begin{center}
\begin{tabular}{l}
|{\edef\jobname{\scantokens\expandafter{\jobname\noexpand}}|\\
|\def\redirectjob |\textit{prefix}|#1~~~{\gdef\jobname{|\textit{dest}|#1}}|\\
|\expandafter\redirectjob\jobname~~~}\input{\jobname}|
\end{tabular}
\end{center}

In an alternative approach,
child documents can be compiled by a specific command line
without additional code or specific definitions:
%
\begin{center}
|... -jobname "|\textit{target}|" "|[\textit{flags}]%
|\includeonly{|\textit{dest}|}\input{|\textit{main}|}"|
\end{center}
%

%%%%%%%%%%%%%%%%%%%%%%%%%%%%%%%%%%%%%%%%%%%%%%%%%%%%%%%%%%%%%%%%%%%%%%%%%%%%%%%%
%%%%%%%%%%%%%%%%%%%%%%%%%%%%%%%%%%%%%%%%%%%%%%%%%%%%%%%%%%%%%%%%%%%%%%%%%%%%%%%%
\section{Information}

%%%%%%%%%%%%%%%%%%%%%%%%%%%%%%%%%%%%%%%%%%%%%%%%%%%%%%%%%%%%%%%%%%%%%%%%%%%%%%%%
\subsection{Copyright}

Copyright \copyright{} 2017--2018 Niklas Beisert

This work may be distributed and/or modified under the
conditions of the \LaTeX{} Project Public License, either version 1.3
of this license or (at your option) any later version.
The latest version of this license is in
  \url{http://www.latex-project.org/lppl.txt}
and version 1.3 or later is part of all distributions of \LaTeX{}
version 2005/12/01 or later.

This work has the LPPL maintenance status `maintained'.

The Current Maintainer of this work is Niklas Beisert.

This work consists of the files |README.txt|, |childdoc.ins| and |childdoc.dtx|
as well as the derived files |childdoc.def|, |cdocsamp.tex|
with |cdocsch1.tex|, |cdocsch2.tex|, |cdocspt3.tex|, |cdocspt4.tex|,
|cdocsdrf.tex|, |cdocsfn1.tex|, |cdocsfn2.tex|
as well as |childdoc.pdf|.

%%%%%%%%%%%%%%%%%%%%%%%%%%%%%%%%%%%%%%%%%%%%%%%%%%%%%%%%%%%%%%%%%%%%%%%%%%%%%%%%
\subsection{Files and Installation}

The package consists of the files:
%
\begin{center}
\begin{tabular}{ll}
    |README.txt|   & readme file \\
    |childdoc.ins| & installation file \\
    |childdoc.dtx| & source file \\
    |childdoc.def| & definition file \\
    |cdocsamp.tex| & sample main file \\
    |cdocsch1.tex| & sample include file \\
    |cdocsch2.tex| & sample include file \\
    |cdocspt3.tex| & sample part file \\
    |cdocspt4.tex| & sample part file \\
    |cdocsdrf.tex| & sample redirection file \\
    |cdocsfn1.tex| & sample redirection file \\
    |cdocsfn2.tex| & sample redirection file \\
    |childdoc.pdf| & manual
\end{tabular}
\end{center}
%
The distribution consists of the files
|README.txt|, |childdoc.ins| and |childdoc.dtx|.
%
\begin{itemize}
\item
Run (pdf)\LaTeX{} on |childdoc.dtx|
to compile the manual |childdoc.pdf| (this file).
\item
Run \LaTeX{} on |childdoc.ins| to create the definitions file |childdoc.def|
and the sample |cdocsamp.tex| with include files
|cdocsch1.tex|, |cdocsch2.tex|, |cdocspt3.tex|, |cdocspt4.tex|,
|cdocsdrf.tex|, |cdocsfn1.tex|, |cdocsfn2.tex|.
Then copy the file |childdoc.def| to an appropriate directory of your \LaTeX{}
distribution, e.g.\ \textit{texmf-root}|/tex/latex/childdoc|.
\end{itemize}

%%%%%%%%%%%%%%%%%%%%%%%%%%%%%%%%%%%%%%%%%%%%%%%%%%%%%%%%%%%%%%%%%%%%%%%%%%%%%%%%
\subsection{Related CTAN Packages}

There are several other packages which offer a similar functionality:
%
\begin{itemize}
\item
The packages
\href{http://ctan.org/pkg/docmute}{\textsf{docmute}},
\href{http://ctan.org/pkg/includex}{\textsf{includex}} and
\href{http://ctan.org/pkg/standalone}{\textsf{standalone}}
provide commands to include only the document body of
a child file thus allowing both files to be compiled individually.
\item
The packages \href{http://ctan.org/pkg/subdocs}{\textsf{subdocs}}
and \href{http://ctan.org/pkg/subfiles}{\textsf{subfiles}}
provide structures in which the main and child documents can be
encapsulated and allowing them to be compiled individually.
The inclusion mechanism is different from the conventional |\include|.
\item
The package \href{http://ctan.org/pkg/combine}{\textsf{combine}}
is an elaborate solution to combine several documents into one.
\end{itemize}
%
See also the CTAN topic \href{http://ctan.org/topic/subdocs}{\textsf{subdocs}}
for further related packages.
The present package differs from the above solutions in that
a document structure constructed with the conventional |\include| mechanism
just needs two extra commands at the top of every file
such that all constituent files can be compiled individually.

%%%%%%%%%%%%%%%%%%%%%%%%%%%%%%%%%%%%%%%%%%%%%%%%%%%%%%%%%%%%%%%%%%%%%%%%%%%%%%%%
%\subsection{Feature Suggestions}
%
%The following is a list of features which may be useful for future
%versions of this package:
%%
%\begin{itemize}
%\item
%\ldots
%\end{itemize}

%%%%%%%%%%%%%%%%%%%%%%%%%%%%%%%%%%%%%%%%%%%%%%%%%%%%%%%%%%%%%%%%%%%%%%%%%%%%%%%%
\subsection{Revision History}

%%%%%%%%%%%%%%%%%%%%%%%%%%%%%%%%%%%%%%%%
\paragraph{v2.0:} 2018/12/30

\begin{itemize}
\item
immediate forward processing
\item
added |\childdocby| mechanism
\item
manual restructured
\end{itemize}

%%%%%%%%%%%%%%%%%%%%%%%%%%%%%%%%%%%%%%%%
\paragraph{v1.6:} 2018/01/17

\begin{itemize}
\item
application for development of include files
\item
corrections to manual
\end{itemize}

%%%%%%%%%%%%%%%%%%%%%%%%%%%%%%%%%%%%%%%%
\paragraph{v1.5:} 2017/05/21

\begin{itemize}
\item
more complete structuring introduced
\item
|\childdocof| introduced
\item
|\childdoc| renamed to |\childdocmain|
\item
|\childredirect| renamed to |\childdocforward| and |\childdocforwardprefix|
and functionality expanded
\end{itemize}

%%%%%%%%%%%%%%%%%%%%%%%%%%%%%%%%%%%%%%%%
\paragraph{v1.0:} 2017/04/27

\begin{itemize}
\item
manual and install package
\item
first version published on CTAN
\end{itemize}

%%%%%%%%%%%%%%%%%%%%%%%%%%%%%%%%%%%%%%%%
\paragraph{v0.6:} 2017/04/26

\begin{itemize}
\item
redirection mechanism added
\end{itemize}

%%%%%%%%%%%%%%%%%%%%%%%%%%%%%%%%%%%%%%%%
\paragraph{v0.5:} 2017/04/26

\begin{itemize}
\item
functionality in definition file
\end{itemize}


%%%%%%%%%%%%%%%%%%%%%%%%%%%%%%%%%%%%%%%%%%%%%%%%%%%%%%%%%%%%%%%%%%%%%%%%%%%%%%%%
%%%%%%%%%%%%%%%%%%%%%%%%%%%%%%%%%%%%%%%%%%%%%%%%%%%%%%%%%%%%%%%%%%%%%%%%%%%%%%%%
%%%%%%%%%%%%%%%%%%%%%%%%%%%%%%%%%%%%%%%%%%%%%%%%%%%%%%%%%%%%%%%%%%%%%%%%%%%%%%%%
\appendix

\settowidth\MacroIndent{\rmfamily\scriptsize 000\ }

 \DocInput{childdoc.dtx}

\end{document}
%</driver>
% \fi
%
% %%%%%%%%%%%%%%%%%%%%%%%%%%%%%%%%%%%%%%%%%%%%%%%%%%%%%%%%%%%%%%%%%%%%%%%%%%%%%%
% %%%%%%%%%%%%%%%%%%%%%%%%%%%%%%%%%%%%%%%%%%%%%%%%%%%%%%%%%%%%%%%%%%%%%%%%%%%%%%
% \section{Sample}
%\iffalse
%<*samplemain>
%\fi
%
% The following presents a sample document
% with two chapters, two parts, a title page,
% a compile flag as well as three forwarding files to set the flag.
% It consists of eight |.tex| files:
% \begin{center}
% \begin{tabular}{ll}
% |cdocsamp.tex|&main file\\
% |cdocsch1.tex|&include file for chapter 1\\
% |cdocsch2.tex|&include file for chapter 2\\
% |cdocspt3.tex|&include file for part 3\\
% |cdocspt4.tex|&include file for part 4\\
% |cdocsdrf.tex|&forwarding file for main file in draft mode\\
% |cdocsfi1.tex|&forwarding file for final version of chapter 1\\
% |cdocsfi2.tex|&forwarding file for final version of chapter 2\\
% \end{tabular}
% \end{center}
% Each of the eight files can be compiled directly by the \LaTeX{} compiler.
%
% %%%%%%%%%%%%%%%%%%%%%%%%%%%%%%%%%%%%%%
% \paragraph{Main File.}
%
% The main file is called |cdocsamp.tex|.
%
% Load the \textsf{childdoc} definitions and
% declare the filename for the main document:
%    \begin{macrocode}
% \iffalse
%
% childdoc.dtx Copyright (C) 2017-2018 Niklas Beisert
%
% This work may be distributed and/or modified under the
% conditions of the LaTeX Project Public License, either version 1.3
% of this license or (at your option) any later version.
% The latest version of this license is in
%   http://www.latex-project.org/lppl.txt
% and version 1.3 or later is part of all distributions of LaTeX
% version 2005/12/01 or later.
%
% This work has the LPPL maintenance status `maintained'.
%
% The Current Maintainer of this work is Niklas Beisert.
%
% This work consists of the files childdoc.dtx and childdoc.ins
% and the derived files childdoc.def and cdocsamp.tex with
% cdocsch1.tex, cdocsch2.tex, cdocsdrf.tex, cdocsfn1.tex, cdocsfn2.tex.
%
%<package>\ifdefined\childdocmain\endinput\fi
%<package>\ProvidesFile{childdoc.def}[2018/12/30 v2.0 child document driver]
%<samplemain>\ProvidesFile{cdocsamp.tex}[2018/12/30 v2.0 sample for childdoc]
%<*driver>
%\ProvidesFile{childdoc.drv}[2018/12/30 v2.0 childdoc reference manual file]
\PassOptionsToClass{10pt,a4paper}{article}
\documentclass{ltxdoc}

\usepackage[margin=35mm]{geometry}
\usepackage{hyperref}
\usepackage{hyperxmp}
\usepackage[usenames]{color}

\hypersetup{colorlinks=true}
\hypersetup{pdfstartview=FitH}
\hypersetup{pdfpagemode=UseNone}
\hypersetup{pdfsource={}}
\hypersetup{pdflang={en-UK}}
\hypersetup{pdfcopyright={Copyright 2017-2018 Niklas Beisert.
  This work may be distributed and/or modified under the
  conditions of the LaTeX Project Public License, either version 1.3
  of this license or (at your option) any later version.}}
\hypersetup{pdflicenseurl={http://www.latex-project.org/lppl.txt}}
\hypersetup{pdfcontactaddress={ETH Zurich, ITP, HIT K,
  Wolfgang-Pauli-Strasse 27}}
\hypersetup{pdfcontactpostcode={8093}}
\hypersetup{pdfcontactcity={Zurich}}
\hypersetup{pdfcontactcountry={Switzerland}}
\hypersetup{pdfcontactemail={nbeisert@itp.phys.ethz.ch}}
\hypersetup{pdfcontacturl={http://people.phys.ethz.ch/\xmptilde nbeisert/}}

\newcommand{\secref}[1]{\hyperref[#1]{section \ref*{#1}}}

\parskip1ex
\parindent0pt
\let\olditemize\itemize
\def\itemize{\olditemize\parskip0pt}

\begin{document}

\title{The \textsf{childdoc} Package}
\hypersetup{pdftitle={The childdoc Package}}
\author{Niklas Beisert\\[2ex]
  Institut f\"ur Theoretische Physik\\
  Eidgen\"ossische Technische Hochschule Z\"urich\\
  Wolfgang-Pauli-Strasse 27, 8093 Z\"urich, Switzerland\\[1ex]
  \href{mailto:nbeisert@itp.phys.ethz.ch}
  {\texttt{nbeisert@itp.phys.ethz.ch}}}
\hypersetup{pdfauthor={Niklas Beisert}}
\hypersetup{pdfsubject={Manual for the LaTeX2e Package childdoc}}
\date{30 December 2018, \textsf{v2.0}}
\maketitle

\begin{abstract}\noindent
\textsf{childdoc} is a \LaTeXe{} package
that enables the direct compilation
of document sections included by |\include|
to individual files.
\end{abstract}

\begingroup
\parskip0ex
\tableofcontents
\endgroup

%%%%%%%%%%%%%%%%%%%%%%%%%%%%%%%%%%%%%%%%%%%%%%%%%%%%%%%%%%%%%%%%%%%%%%%%%%%%%%%%
%%%%%%%%%%%%%%%%%%%%%%%%%%%%%%%%%%%%%%%%%%%%%%%%%%%%%%%%%%%%%%%%%%%%%%%%%%%%%%%%
\section{Introduction}

\LaTeX{} provides a mechanism to structure a large document (such as a book)
into a main file and several child files (containing the chapters)
using the |\include| command.
This mechanism is beneficial for documents
which span hundreds of pages in order to
make the source file(s) more manageable.
Moreover, compilation can be restricted to
selected child files by means of the |\includeonly| command.
The latter feature can be used to reduce the compilation time while editing
(this was significantly more useful in the earlier days of \LaTeX{})
or to generate a smaller document which is easier to navigate.
Another application of |\includeonly| is to generate
documents consisting of selected parts of the complete document.

However, there are a few drawbacks of the plain |\include| mechanism:
\begin{itemize}
\item
The child files cannot be compiled on their own,
they can only be compiled via the main file.
A naive editing environment
(such as a text editor with an option
to have the current file processed by \LaTeX)
may require one to switch to the main file before compiling;
attempting to compile the child file produces errors.
\item
The main file must be modified (each time)
to adjust the |\includeonly| command
to the present needs. This easily leaves the main file in a messy state.
\item
The generated document will always carry the filename
of the main document. This is inconvenient if
several child files are to be compiled and
to be kept for distribution.
\end{itemize}

The present package provides a simple interface
to make child files individually compilable by \LaTeX{}.
Compiling a child file then has the same effect as compiling
the main file with an |\includeonly| command
to select the appropriate child.
Moreover the generated document will carry the name of the child
rather than the main file.
This resolves all three above issues.

This feature is meant to make the editing of books,
thesis documents and lecture notes somewhat more convenient.
However, the package can also be used efficiently for
composing a series of documents (such as exercise sheets)
which are typically distributed individually.
It then assists the author in generating the individual documents
(potentially in different versions)
as well as a document containing the collected series.
Another application is in developing style files
or other kinds of included material
where compilation of the style file could redirect
to a sample or test file.

%%%%%%%%%%%%%%%%%%%%%%%%%%%%%%%%%%%%%%%%%%%%%%%%%%%%%%%%%%%%%%%%%%%%%%%%%%%%%%%%
%%%%%%%%%%%%%%%%%%%%%%%%%%%%%%%%%%%%%%%%%%%%%%%%%%%%%%%%%%%%%%%%%%%%%%%%%%%%%%%%
\section{Usage}

First of all, the package \textsf{childdoc} is \emph{not} a standard
\LaTeXe{} |.sty| style file! Therefore it needs to be invoked in
a non-standard way.

%%%%%%%%%%%%%%%%%%%%%%%%%%%%%%%%%%%%%%%%%%%%%%%%%%%%%%%%%%%%%%%%%%%%%%%%%%%%%%%%
\subsection{Included Files}
\label{sec:include}

%%%%%%%%%%%%%%%%%%%%%%%%%%%%%%%%%%%%%%%%
\DescribeMacro{\childdocmain}
To use the package, add the commands
\begin{center}
\begin{tabular}{l}
|\input{childdoc.def}|\\
|\childdocmain{}|\\
\end{tabular}
\end{center}
at the very top of the main \LaTeX{} file,
in particular \emph{before} the |\documentclass| statement!
The argument of |\childdocmain| should be left empty
(but it must be present).

%%%%%%%%%%%%%%%%%%%%%%%%%%%%%%%%%%%%%%%%
\DescribeMacro{\childdocof}
Furthermore, add the commands
\begin{center}
\begin{tabular}{l}
|\input{childdoc.def}|\\
|\childdocof{|\textit{main}|}|\\
\end{tabular}
\end{center}
at the top of every child file \textit{child}
which is included by |\include{|\textit{child}|}|
from within the main file
(or at least for those files to be compiled individually).
The argument \textit{main} must be the filename of the main file.

There are a couple of
considerations in setting up the main and child documents:

%%%%%%%%%%%%%%%%%%%%%%%%%%%%%%%%%%%%%%%%
\paragraph{Restrictions.}

Please note the following restrictions:
\begin{itemize}
\item
|\childdocmain| must be called with one argument \textit{main}
to ensure compatibility with earlier version of the package.
It must either be empty (|\childdocmain{}|)
or precisely match the filename of the main file in which it is specified.
See \secref{sec:detection} for further information.
\item
The filename \textit{main} must be specified without the |.tex| extension.
\item
The filename \textit{main} is case sensitive
(even in case-insensitive file systems)
due to internal string comparison.
\item
The argument \textit{main} should be fully expanded, it cannot be a macro.
\item
Subdirectories and special characters should be avoided in filenames.
\item
The command |\childdocmain{|\textit{main}|}| must be followed by a whitespace.
It should not be followed immediately by another command
or by a comment mark `|%|'.
This is because the \TeX{} parser reads the token immediately following
the argument of |\childdocmain| and puts it
at the beginning of every child section;
however, a white\-space is ignored.
\end{itemize}

%%%%%%%%%%%%%%%%%%%%%%%%%%%%%%%%%%%%%%%%
\paragraph{Content of Main File.}

It is advisable to place all content in the child files included by |\include|.
Any output contained in the main file will appear in all child documents
unless suppressed manually;
it cannot be suppressed automatically by the |\includeonly| directive
and thus should normally be avoided.
A method to include some content in the main file
by means of conditional processing is described in \secref{sec:conditional}.

%%%%%%%%%%%%%%%%%%%%%%%%%%%%%%%%%%%%%%%%
\paragraph{Page Numbering.}

When only a part of the document is compiled,
the appropriate numbering of pages
(as well as other status parameters)
is determined from the |.aux| files.
The latter contain information from previous passes.
However this information needs to propagate through
all intermediate child documents.
Therefore the page numbering in child documents may well
be inconsistent until the complete document is compiled at least once.

A useful (if unconventional) way to always ensure a consistent
page numbering is to restart the numbering in each child document
and denote the pages by `\textit{child}|.|\textit{page}'
where \textit{child} represents the chapter/section number of the child file.
This can be achieved by the command
|\numberwithin{page}{|\textit{child}|}|
of the \textsf{amsmath} package
where \textit{child} can be |chapter| or |section|
depending on the chosen structuring.
Alternatively, one can modify the macro |\thepage| appropriately
and reset the counter |page| at the start of each child file.

%%%%%%%%%%%%%%%%%%%%%%%%%%%%%%%%%%%%%%%%%%%%%%%%%%%%%%%%%%%%%%%%%%%%%%%%%%%%%%%%
\subsection{Conditional Processing}
\label{sec:conditional}

The package provides a mechanism to compile different versions
of a document. To customise the versions further some conditional processing
can come in handy to distinguish which version is being compiled.
The package provides two macros to describe the compilation context:

%%%%%%%%%%%%%%%%%%%%%%%%%%%%%%%%%%%%%%%%
\DescribeMacro{\ifchilddoc}
The conditional |\ifchilddoc| distinguishes between the compilation of
child documents and the main document:
%
\begin{center}
|\ifchilddoc |\textit{child-code}| |[|\||else |\textit{main-code}]| \||fi|
\end{center}

%%%%%%%%%%%%%%%%%%%%%%%%%%%%%%%%%%%%%%%%
\DescribeMacro{\childdocname}
\DescribeMacro{\childdocjob}
The macro |\childdocname| contains the filename (without extension)
of the main or child file being processed.
Note that |\childdocjob| will always contain the name of the main file.

%%%%%%%%%%%%%%%%%%%%%%%%%%%%%%%%%%%%%%%%
\paragraph{Title Page.}

Conditional processing can be used to include a title or banner page
in the main document when proper precautions are taken.
Importantly, the code in the main file should ensure that the page counter
(as well as other status parameters which are stored in the |.aux| files)
takes the same value after the conditional processing.
Otherwise the page numbers may take divergent values
depending on which part is compiled.

For example, a title page could be declared by:
%
\begin{center}
\begin{tabular}{l}
|\ifchilddoc\||else|\\
|\addtocounter{page}{-1}|\\
\textit{code for title page}\\
|\newpage|\\
|\||fi|
\end{tabular}
\end{center}
%
A banner page for the child documents can be generated by:
%
\begin{center}
\begin{tabular}{l}
|\ifchilddoc|\\
|\addtocounter{page}{-1}|\\
\textit{code for banner page}\\
|\newpage|\\
|\||fi|
\end{tabular}
\end{center}
%
Here one could write a message such as:
\begin{center}
|This is the part \childdocname{} of \childdocjob{}.|
\end{center}

%%%%%%%%%%%%%%%%%%%%%%%%%%%%%%%%%%%%%%%%%%%%%%%%%%%%%%%%%%%%%%%%%%%%%%%%%%%%%%%%
\subsection{Flags}
\label{sec:flags}

The package makes it easy to generate different versions
of the main or child documents.
To this end compilation flags can be defined
and assigned different default values.
They will be particularly useful in conjunction
with the forwarding mechanism described in \secref{sec:forward}.

For example, it may be useful to have a flag |\version|
which can be set to |draft| or |final|.
The document source will contain some conditional code
depending on the value of |\version|.
Suppose further, the flag should default to |final| for the main file
and to |draft| for child files
which is a natural assignment for editing the document.
This is achieved by placing the following code
in the preamble of the main document
(below the |\childdocmain| directive):
%
\begin{center}
\begin{tabular}{l}
|\ifchilddoc|\\
|\providecommand{\version}{draft}|\\
|\||else|\\
|\providecommand{\version}{final}|\\
|\||fi|
\end{tabular}
\end{center}
%
The definition by |\providecommand| makes sure
that previous definitions are not overwritten.
Further statements |\providecommand{\version}{...}|
can thus be added before the above code to override it.

For the main file, one might add a line
(between |\childdocmain| and the above block)
%
\begin{center}
|%\ifchilddoc\||else\providecommand{\version}{draft}\||fi|
\end{center}
%
which can be uncommented to produce a draft version.
Likewise one can add a line to the very top of a child file
(above the |\childdocof{|\textit{main}|}| directive)
%
\begin{center}
|%\providecommand{\version}{final}|
\end{center}
%
which can be uncommented to produce the final version of this child document.

%%%%%%%%%%%%%%%%%%%%%%%%%%%%%%%%%%%%%%%%%%%%%%%%%%%%%%%%%%%%%%%%%%%%%%%%%%%%%%%%
\subsection{Forwarding}
\label{sec:forward}

Different versions of the main or child documents
using compilation flags as described in \secref{sec:flags}
can be (permanently) stored in different files
for convenient compilation, viewing and distribution.
To this end, the package defines a command
to pass on compilation to a different file:

%%%%%%%%%%%%%%%%%%%%%%%%%%%%%%%%%%%%%%%%
\DescribeMacro{\childdocforward}
The command |\childdocforward| redirects processing to
another source file:
%
\begin{center}
\begin{tabular}{l}
|\input{childdoc.def}|\\
|\childdocforward[|\textit{main}|]{|\textit{dest}|}|\\
\end{tabular}
\end{center}
%
The argument \textit{dest} is the destination file
(without extension).
It should be the main file or one of the child files.
Note that further \textsf{childdoc} directives
such as |\childdocof| and |\childdocforward|
in the indicated file will be processed in this form.
The optional argument \textit{main}
passes on directly to the main file \textit{main}
while pretending to compile the child \textit{dest}.
This form behaves as if \textit{dest}
issues |\childdocof{|\textit{main}|}| right away,
and no further \textsf{childdoc} directives will be processed.

%%%%%%%%%%%%%%%%%%%%%%%%%%%%%%%%%%%%%%%%
\DescribeMacro{\...prefix}
In the alternative form |\childdocforwardprefix|,
%
\begin{center}
\begin{tabular}{l}
|\input{childdoc.def}|\\
|\childdocforwardprefix[|\textit{main}|]{|\textit{prefix}|}{|\textit{dest}|}|
\end{tabular}
\end{center}
%
the destination file is determined by a pattern
depending on the current file:
To make this work, the current file must be called
`{\textit{prefix}\hspace{0.2em}\textit{suffix}}'
with \textit{prefix} matching precisely the argument.
Processing is then passed on to the file
`{\textit{dest}\hspace{0.2em}\textit{suffix}}'.
Surely, the same effect is achieved by
directly specifying the
argument `{\textit{dest}\hspace{0.2em}\textit{suffix}}'
in the first form.
However, that requires to set up a different file
for each child. With the alternative form of the command
all these files can have exactly the same content
which simplifies setting them up and maintaining them.

For example, the following file |draft.tex|
with a compilation flag |\version| as described in \secref{sec:flags}
compiles the main document as a draft:
%
\begin{center}
\begin{tabular}{l}
|\def\version{draft}|\\
|\input{childdoc.def}|\\
|\childdocforward{|\textit{main}|}|
\end{tabular}
\end{center}
%
Likewise, the following files |final|\textit{nn}|.tex|
compile the final version of the child document
|child|\textit{nn}|.tex|:
%
\begin{center}
\begin{tabular}{l}
|\def\version{final}|\\
|\input{childdoc.def}|\\
|\childdocforwardprefix{final}{child}|
\end{tabular}
\end{center}
%

Note that when several versions of a main file and/or of each child file
are to be generated, it may be convenient to set up a |Makefile| or
shell script to automatise the process.

%%%%%%%%%%%%%%%%%%%%%%%%%%%%%%%%%%%%%%%%%%%%%%%%%%%%%%%%%%%%%%%%%%%%%%%%%%%%%%%%
\subsection{Command Line Processing}
\label{sec:commandline}

The effect of redirection files can also be achieved by invoking
the \LaTeX{} compiler with a more elaborate command line.
Most conveniently this should be done as part
of a shell script or a |Makefile|.

When using \textsf{childdoc} in the main file, the following
command lines effectively perform a redirection
(note that depending on the shell being used,
backslashes may have to be doubled: `|\|' $\to$ `|\\|'):
%
\begin{center}
|... -jobname "|\textit{target}|" |\\|"|[\textit{flags}]%
|\input{childdoc.def}\childdocforward[|\textit{main}|]{|\textit{dest}|}"|
\end{center}
%
Here \textit{target} is the name of the output file,
\textit{main} is the name of the main file
and \textit{dest} is the name of the main or child file to be processed
(all filenames without extensions).
The optional argument \textit{main} can be omitted
if \textit{main} matches \textit{dest}.
Optionally, compilation \textit{flags} can be defined via |\def| commands.
This command line makes the \TeX{} engine believe
it is compiling the file \textit{target}
whose content is specified as the latter parameter.
The provided code then forwards the processing to
\textit{main} or \textit{dest} as described in \secref{sec:forward}.

%%%%%%%%%%%%%%%%%%%%%%%%%%%%%%%%%%%%%%%%%%%%%%%%%%%%%%%%%%%%%%%%%%%%%%%%%%%%%%%%
\subsection{Include by Input}
\label{sec:input}

Including child documents by |\include| has some restrictions by design.
Most notably, the content of a child document always occupies
its own set of pages; pages cannot be shared between child documents.
Usually, this behaviour makes perfect sense
because each child document contain an essential part of the document.
However, in some situations it may be desirable to compose
a document from a collection of parts
without having mandatory page breaks between then.
For this case, the package
provides a mechanism to include parts
by |\input| which can also be processed individually.
However, by construction this mechanism
requires manual handling of the content to be output.

%%%%%%%%%%%%%%%%%%%%%%%%%%%%%%%%%%%%%%%%
\DescribeMacro{\ifchilddocmanual}
The main file should be prepared as usual, see \secref{sec:include}.
However, the document body must make a distinction
between processing of an individual part and of the main document, e.g.:
%
\begin{center}
\begin{tabular}{l}
|\ifchilddocmanual|\\
|\input{\childdocname}|\\
|\||else|\\
\textit{document body with }|\input{|\textit{part}|}|\\
|\||fi|
\end{tabular}
\end{center}
%
The conditional |\ifchilddocmanual| is true whenever
a part to be included by |\input| is being compiled,
and the name of the part is stored in |\childdocname|.

%%%%%%%%%%%%%%%%%%%%%%%%%%%%%%%%%%%%%%%%
\DescribeMacro{\childdocby}
Each part to be included by |\input| should start with:
%
\begin{center}
\begin{tabular}{l}
|\input{childdoc.def}|\\
|\childdocby{|\textit{main}|}|\\
\end{tabular}
\end{center}
%
The directive |\childdocby| is similar to |\childdocof|
described in \secref{sec:include},
but the subsequent selection of content must be done manually.
To that end, both |\ifchilddoc| and |\ifchilddocmanual|
will be true upon processing of a part,
and the name of the part is stored in |\childdocname|.
Note that |\jobname| will be set to the filename of the current part
so that each part receives an individual |.aux| file
that does not interfere with the |.aux| file(s) of the main document.
This behaviour can be altered by the alternative form
|\childdocby[*]{|\textit{main}|}| (with a non-empty optional argument)
which uses the |.aux| file of the main document
by setting |\jobname| to \textit{main}.

%%%%%%%%%%%%%%%%%%%%%%%%%%%%%%%%%%%%%%%%%%%%%%%%%%%%%%%%%%%%%%%%%%%%%%%%%%%%%%%%
\subsection{Driver Development}
\label{sec:driver}

The \textsf{childdoc} mechanism can also be use for the development
of definition files such as \LaTeX{} styles or classes.
This case differs from the above setup with multiple parts
included by |\include| in that no |\includeonly| should be invoked.
This can be achieved by starting the include file
(before |\ProvidesPackage|) with:
%
\begin{center}
\begin{tabular}{l}
|\input{childdoc.def}|\\
|\childdocforward{|\textit{main}|}|\\
\end{tabular}
\end{center}
%
or alternatively with:
%
\begin{center}
\begin{tabular}{l}
|\input{childdoc.def}|\\
|\childdocby{|\textit{main}|}|\\
\end{tabular}
\end{center}
%
Both forms have slightly different effects as described above.
The main file is prepared as usual, see \secref{sec:include}.

%%%%%%%%%%%%%%%%%%%%%%%%%%%%%%%%%%%%%%%%%%%%%%%%%%%%%%%%%%%%%%%%%%%%%%%%%%%%%%%%
\subsection{Legacy Detection}
\label{sec:detection}

The directive |\childdocmain| in the main file can detect
whether the complete document or merely a child is to be compiled
even without using the directive |\childdocof|.
This method is deprecated because it is less robust
and there is no compelling reason to use it;
it is merely provided for backward compatibility
and it may be removed in future versions.

If the detection mechanism is to be used,
it is mandatory to correctly specify
the filename of the main file as the argument of |\childdocmain|:
%
\begin{center}
\begin{tabular}{l}
|\input{childdoc.def}|\\
|\childdocmain{|\textit{main}|}|\\
\end{tabular}
\end{center}
%
If |\jobname| does not match the argument \textit{main} of |\childdocmain|,
it is assumed that |\jobname| points to the child file to be compiled.
When using |\childdocmain| with the main file specified as argument,
it suffices to start a child file
with just |\input{|\textit{main}|}|
without loading of the package and using |\childdocof|.
If instead all processing is done
with the appropriate \textsf{childdoc} directives,
the argument of \textit{main} of |\childdocmain| can be empty.

An alternative version of the command line processing described
in \secref{sec:commandline} using the detection mechanism reads:
%
\begin{center}
|... -jobname "|\textit{target}|" "|[\textit{flags}]%
[|\def\jobname{|\textit{dest}|}|]|\input{|\textit{main}|}"|
\end{center}

%%%%%%%%%%%%%%%%%%%%%%%%%%%%%%%%%%%%%%%%%%%%%%%%%%%%%%%%%%%%%%%%%%%%%%%%%%%%%%%%
\subsection{Manual Code}
\label{sec:manual}

In case one cannot be certain whether the definitions file |childdoc.def|
is installed on the target \TeX{} distribution
and one prefers not to ship it,
it is conceivable to paste a few relevant commands into the sources.

To that end, drop all statements |\input{childdoc.def}|
and perform the replacements as outlined below.
Instead of |\childdocmain{|\textit{main}|}| add the following code
to the top of the main file:
%
\begin{center}
\begin{tabular}{l}
|\||ifdefined\childdocname\endinput\||fi\newif\ifchilddoc|\\
|\edef\childdocname{\scantokens\expandafter{\jobname\noexpand}}|\\
|\def\childdocmain{|\textit{main}|}\||ifx\childdocmain\childdocname\||else|\\
|\childdoctrue\includeonly{\childdocname}\let\jobname\childdocmain\||fi|\\
\end{tabular}
\end{center}
%
Instead of |\childdocof{|\textit{main}|}| just include the main file
at the top of each child file:
%
\begin{center}
|\input{|\textit{main}|}|
\end{center}
%
A simple redirection |\childdocforward{|\textit{dest}|}| is achieved by:
%
\begin{center}
|\def\jobname{|\textit{dest}|}\input{\jobname}|
\end{center}
%
The redirection with prefix
|\childdocforwardprefix[|\textit{prefix}|]{|\textit{dest}|}|
is accomplished by:
%
\begin{center}
\begin{tabular}{l}
|{\edef\jobname{\scantokens\expandafter{\jobname\noexpand}}|\\
|\def\redirectjob |\textit{prefix}|#1~~~{\gdef\jobname{|\textit{dest}|#1}}|\\
|\expandafter\redirectjob\jobname~~~}\input{\jobname}|
\end{tabular}
\end{center}

In an alternative approach,
child documents can be compiled by a specific command line
without additional code or specific definitions:
%
\begin{center}
|... -jobname "|\textit{target}|" "|[\textit{flags}]%
|\includeonly{|\textit{dest}|}\input{|\textit{main}|}"|
\end{center}
%

%%%%%%%%%%%%%%%%%%%%%%%%%%%%%%%%%%%%%%%%%%%%%%%%%%%%%%%%%%%%%%%%%%%%%%%%%%%%%%%%
%%%%%%%%%%%%%%%%%%%%%%%%%%%%%%%%%%%%%%%%%%%%%%%%%%%%%%%%%%%%%%%%%%%%%%%%%%%%%%%%
\section{Information}

%%%%%%%%%%%%%%%%%%%%%%%%%%%%%%%%%%%%%%%%%%%%%%%%%%%%%%%%%%%%%%%%%%%%%%%%%%%%%%%%
\subsection{Copyright}

Copyright \copyright{} 2017--2018 Niklas Beisert

This work may be distributed and/or modified under the
conditions of the \LaTeX{} Project Public License, either version 1.3
of this license or (at your option) any later version.
The latest version of this license is in
  \url{http://www.latex-project.org/lppl.txt}
and version 1.3 or later is part of all distributions of \LaTeX{}
version 2005/12/01 or later.

This work has the LPPL maintenance status `maintained'.

The Current Maintainer of this work is Niklas Beisert.

This work consists of the files |README.txt|, |childdoc.ins| and |childdoc.dtx|
as well as the derived files |childdoc.def|, |cdocsamp.tex|
with |cdocsch1.tex|, |cdocsch2.tex|, |cdocspt3.tex|, |cdocspt4.tex|,
|cdocsdrf.tex|, |cdocsfn1.tex|, |cdocsfn2.tex|
as well as |childdoc.pdf|.

%%%%%%%%%%%%%%%%%%%%%%%%%%%%%%%%%%%%%%%%%%%%%%%%%%%%%%%%%%%%%%%%%%%%%%%%%%%%%%%%
\subsection{Files and Installation}

The package consists of the files:
%
\begin{center}
\begin{tabular}{ll}
    |README.txt|   & readme file \\
    |childdoc.ins| & installation file \\
    |childdoc.dtx| & source file \\
    |childdoc.def| & definition file \\
    |cdocsamp.tex| & sample main file \\
    |cdocsch1.tex| & sample include file \\
    |cdocsch2.tex| & sample include file \\
    |cdocspt3.tex| & sample part file \\
    |cdocspt4.tex| & sample part file \\
    |cdocsdrf.tex| & sample redirection file \\
    |cdocsfn1.tex| & sample redirection file \\
    |cdocsfn2.tex| & sample redirection file \\
    |childdoc.pdf| & manual
\end{tabular}
\end{center}
%
The distribution consists of the files
|README.txt|, |childdoc.ins| and |childdoc.dtx|.
%
\begin{itemize}
\item
Run (pdf)\LaTeX{} on |childdoc.dtx|
to compile the manual |childdoc.pdf| (this file).
\item
Run \LaTeX{} on |childdoc.ins| to create the definitions file |childdoc.def|
and the sample |cdocsamp.tex| with include files
|cdocsch1.tex|, |cdocsch2.tex|, |cdocspt3.tex|, |cdocspt4.tex|,
|cdocsdrf.tex|, |cdocsfn1.tex|, |cdocsfn2.tex|.
Then copy the file |childdoc.def| to an appropriate directory of your \LaTeX{}
distribution, e.g.\ \textit{texmf-root}|/tex/latex/childdoc|.
\end{itemize}

%%%%%%%%%%%%%%%%%%%%%%%%%%%%%%%%%%%%%%%%%%%%%%%%%%%%%%%%%%%%%%%%%%%%%%%%%%%%%%%%
\subsection{Related CTAN Packages}

There are several other packages which offer a similar functionality:
%
\begin{itemize}
\item
The packages
\href{http://ctan.org/pkg/docmute}{\textsf{docmute}},
\href{http://ctan.org/pkg/includex}{\textsf{includex}} and
\href{http://ctan.org/pkg/standalone}{\textsf{standalone}}
provide commands to include only the document body of
a child file thus allowing both files to be compiled individually.
\item
The packages \href{http://ctan.org/pkg/subdocs}{\textsf{subdocs}}
and \href{http://ctan.org/pkg/subfiles}{\textsf{subfiles}}
provide structures in which the main and child documents can be
encapsulated and allowing them to be compiled individually.
The inclusion mechanism is different from the conventional |\include|.
\item
The package \href{http://ctan.org/pkg/combine}{\textsf{combine}}
is an elaborate solution to combine several documents into one.
\end{itemize}
%
See also the CTAN topic \href{http://ctan.org/topic/subdocs}{\textsf{subdocs}}
for further related packages.
The present package differs from the above solutions in that
a document structure constructed with the conventional |\include| mechanism
just needs two extra commands at the top of every file
such that all constituent files can be compiled individually.

%%%%%%%%%%%%%%%%%%%%%%%%%%%%%%%%%%%%%%%%%%%%%%%%%%%%%%%%%%%%%%%%%%%%%%%%%%%%%%%%
%\subsection{Feature Suggestions}
%
%The following is a list of features which may be useful for future
%versions of this package:
%%
%\begin{itemize}
%\item
%\ldots
%\end{itemize}

%%%%%%%%%%%%%%%%%%%%%%%%%%%%%%%%%%%%%%%%%%%%%%%%%%%%%%%%%%%%%%%%%%%%%%%%%%%%%%%%
\subsection{Revision History}

%%%%%%%%%%%%%%%%%%%%%%%%%%%%%%%%%%%%%%%%
\paragraph{v2.0:} 2018/12/30

\begin{itemize}
\item
immediate forward processing
\item
added |\childdocby| mechanism
\item
manual restructured
\end{itemize}

%%%%%%%%%%%%%%%%%%%%%%%%%%%%%%%%%%%%%%%%
\paragraph{v1.6:} 2018/01/17

\begin{itemize}
\item
application for development of include files
\item
corrections to manual
\end{itemize}

%%%%%%%%%%%%%%%%%%%%%%%%%%%%%%%%%%%%%%%%
\paragraph{v1.5:} 2017/05/21

\begin{itemize}
\item
more complete structuring introduced
\item
|\childdocof| introduced
\item
|\childdoc| renamed to |\childdocmain|
\item
|\childredirect| renamed to |\childdocforward| and |\childdocforwardprefix|
and functionality expanded
\end{itemize}

%%%%%%%%%%%%%%%%%%%%%%%%%%%%%%%%%%%%%%%%
\paragraph{v1.0:} 2017/04/27

\begin{itemize}
\item
manual and install package
\item
first version published on CTAN
\end{itemize}

%%%%%%%%%%%%%%%%%%%%%%%%%%%%%%%%%%%%%%%%
\paragraph{v0.6:} 2017/04/26

\begin{itemize}
\item
redirection mechanism added
\end{itemize}

%%%%%%%%%%%%%%%%%%%%%%%%%%%%%%%%%%%%%%%%
\paragraph{v0.5:} 2017/04/26

\begin{itemize}
\item
functionality in definition file
\end{itemize}


%%%%%%%%%%%%%%%%%%%%%%%%%%%%%%%%%%%%%%%%%%%%%%%%%%%%%%%%%%%%%%%%%%%%%%%%%%%%%%%%
%%%%%%%%%%%%%%%%%%%%%%%%%%%%%%%%%%%%%%%%%%%%%%%%%%%%%%%%%%%%%%%%%%%%%%%%%%%%%%%%
%%%%%%%%%%%%%%%%%%%%%%%%%%%%%%%%%%%%%%%%%%%%%%%%%%%%%%%%%%%%%%%%%%%%%%%%%%%%%%%%
\appendix

\settowidth\MacroIndent{\rmfamily\scriptsize 000\ }

 \DocInput{childdoc.dtx}

\end{document}
%</driver>
% \fi
%
% %%%%%%%%%%%%%%%%%%%%%%%%%%%%%%%%%%%%%%%%%%%%%%%%%%%%%%%%%%%%%%%%%%%%%%%%%%%%%%
% %%%%%%%%%%%%%%%%%%%%%%%%%%%%%%%%%%%%%%%%%%%%%%%%%%%%%%%%%%%%%%%%%%%%%%%%%%%%%%
% \section{Sample}
%\iffalse
%<*samplemain>
%\fi
%
% The following presents a sample document
% with two chapters, two parts, a title page,
% a compile flag as well as three forwarding files to set the flag.
% It consists of eight |.tex| files:
% \begin{center}
% \begin{tabular}{ll}
% |cdocsamp.tex|&main file\\
% |cdocsch1.tex|&include file for chapter 1\\
% |cdocsch2.tex|&include file for chapter 2\\
% |cdocspt3.tex|&include file for part 3\\
% |cdocspt4.tex|&include file for part 4\\
% |cdocsdrf.tex|&forwarding file for main file in draft mode\\
% |cdocsfi1.tex|&forwarding file for final version of chapter 1\\
% |cdocsfi2.tex|&forwarding file for final version of chapter 2\\
% \end{tabular}
% \end{center}
% Each of the eight files can be compiled directly by the \LaTeX{} compiler.
%
% %%%%%%%%%%%%%%%%%%%%%%%%%%%%%%%%%%%%%%
% \paragraph{Main File.}
%
% The main file is called |cdocsamp.tex|.
%
% Load the \textsf{childdoc} definitions and
% declare the filename for the main document:
%    \begin{macrocode}
\input{childdoc.def}
\childdocmain{}
%    \end{macrocode}

% Optional override for |\version| flag:
%    \begin{macrocode}
%%\ifchilddoc\else\providecommand{\version}{draft}\fi
%    \end{macrocode}

% Define the default values for the |\version| flag
% (|final| for the main file and |draft| for childs):
%    \begin{macrocode}
\ifchilddoc
\providecommand{\version}{draft}
\else
\providecommand{\version}{final}
\fi
%    \end{macrocode}

% Load the standard document class:
%    \begin{macrocode}
\documentclass[12pt]{article}
%    \end{macrocode}

% Start the document body:
%    \begin{macrocode}
\begin{document}
%    \end{macrocode}

% Declare a title page.
% Print title, part of document being processed and version flag:
%    \begin{macrocode}
\addtocounter{page}{-1}
\begin{center}
{\LARGE\bfseries{}childdoc example\par}
\vspace{1cm}
\ifchilddoc
\ifchilddocmanual part\else chapter\fi:
`\childdocname' of `\childdocjob'\par
\else
main document: `\childdocjob'\par
\fi
version: \version\par
\end{center}
\newpage
%    \end{macrocode}

% Manually include selected file,
% otherwise process as usual:
%    \begin{macrocode}
\ifchilddocmanual
\section*{part `\childdocname'}
\input{\childdocname}
\else
%    \end{macrocode}

% Include the two chapters:
%    \begin{macrocode}
\include{cdocsch1}
\include{cdocsch2}
%    \end{macrocode}

% Include the two parts unless only chapters should be displayed:
%    \begin{macrocode}
\ifchilddoc\else
\section{part three}
\input{cdocspt3}
\section{part four}
\input{cdocspt4}
\fi
%    \end{macrocode}

% Process as usual until here:
%    \begin{macrocode}
\fi
%    \end{macrocode}

% End of document body:
%    \begin{macrocode}
\end{document}
%    \end{macrocode}
%\iffalse
%</samplemain>
%\fi
%
% %%%%%%%%%%%%%%%%%%%%%%%%%%%%%%%%%%%%%%
% \paragraph{Chapter Include Files.}
%
% The include files are called |cdocsch1.tex| and |cdocsch2.tex|.
%
%\iffalse
%<*samplechap1|samplechap2>
%\fi

% Optional override for |\version| flag:
%    \begin{macrocode}
%%\providecommand{\version}{final}
%    \end{macrocode}

% Include the main document:
%    \begin{macrocode}
\input{childdoc.def}
\childdocof{cdocsamp}
%    \end{macrocode}

%\iffalse
%</samplechap1|samplechap2>
%\fi
%
%\iffalse
%<*samplechap1>
%\fi
% Some text for chapter 1:
%    \begin{macrocode}
\section{one}
some text in chapter one
%    \end{macrocode}

%\iffalse
%</samplechap1>
%\fi
% Some text for chapter 2:
%\iffalse
%<*samplechap2>
%\fi
%    \begin{macrocode}
\section{two}
more text in chapter two
%    \end{macrocode}

%\iffalse
%</samplechap2>
%\fi
%
% %%%%%%%%%%%%%%%%%%%%%%%%%%%%%%%%%%%%%%
% \paragraph{Part Include Files.}
%
% The include files are called |cdocspt3.tex| and |cdocspt4.tex|.
%
%\iffalse
%<*samplepart3|samplepart4>
%\fi

% Optional override for |\version| flag:
%    \begin{macrocode}
%%\providecommand{\version}{final}
%    \end{macrocode}

% Include the main document:
%    \begin{macrocode}
\input{childdoc.def}
\childdocby{cdocsamp}
%    \end{macrocode}

%\iffalse
%</samplepart3|samplepart4>
%\fi
%
%\iffalse
%<*samplepart3>
%\fi
% Some text for part 3:
%    \begin{macrocode}
some text in part three
%    \end{macrocode}

%\iffalse
%</samplepart3>
%\fi
% Some text for part 4:
%\iffalse
%<*samplepart4>
%\fi
%    \begin{macrocode}
more text in part four
%    \end{macrocode}

%\iffalse
%</samplepart4>
%\fi
%
% %%%%%%%%%%%%%%%%%%%%%%%%%%%%%%%%%%%%%%
% \paragraph{Forwarding for a Complete Draft.}
%
% The following forwarding file |cdocsdrf.tex|
% compiles the main document in draft mode:
%\iffalse
%<*sampledraft>
%\fi
%    \begin{macrocode}
\def\version{draft}
\input{childdoc.def}
\childdocforward{cdocsamp}
%    \end{macrocode}

%\iffalse
%</sampledraft>
%\fi
%
% %%%%%%%%%%%%%%%%%%%%%%%%%%%%%%%%%%%%%%
% \paragraph{Forwarding for Final Version of the Chapters.}
%
% The following forwarding files |cdocsfn1.tex| and |cdocsfn2.tex|
% (with identical content)
% compile the final versions of the child documents
% |cdocsch1.tex| and |cdocsch2.tex|, respectively:
%\iffalse
%<*samplefinal>
%\fi
%    \begin{macrocode}
\def\version{final}
\input{childdoc.def}
\childdocforwardprefix[cdocsamp]{cdocsfn}{cdocsch}
%    \end{macrocode}

%\iffalse
%</samplefinal>
%\fi
%
% %%%%%%%%%%%%%%%%%%%%%%%%%%%%%%%%%%%%%%
% \paragraph{Command Line Processing.}
%
% The following three command lines generate the output files
% |cdocscld|, |cdocscl1| and |cdocscl2|
% which should be identical to
% |cdocsdrf|, |cdocsch1| and |cdocsfn2|, respectively:
% \begin{center}
% \begin{tabular}{l}
% |latex -jobname cdocscld \|\\
% |  "\def\version{draft}\input{childdoc.def}\childdocforward{cdocsamp}"|\\
% |latex -jobname cdocscl1 \|\\
% |  "\input{childdoc.def}\childdocforward[cdocsamp]{cdocsch1}"|\\
% |latex -jobname cdocscl2 \|\\
% |  "\def\version{final}\input{childdoc.def}\childdocforward{cdocsch2}"|
% \end{tabular}
% \end{center}
% Note that the trailing backslash on each first line
% merely continues the input to the second line
% (for convenient cut ant paste).
% Furthermore, the command |latex| can be replaced by any
% of its alternative versions such as |pdflatex|.
%
% %%%%%%%%%%%%%%%%%%%%%%%%%%%%%%%%%%%%%%%%%%%%%%%%%%%%%%%%%%%%%%%%%%%%%%%%%%%%%%
% %%%%%%%%%%%%%%%%%%%%%%%%%%%%%%%%%%%%%%%%%%%%%%%%%%%%%%%%%%%%%%%%%%%%%%%%%%%%%%
% \section{Implementation}
%\iffalse
%<*package>
%\fi
%
% This section describes the definitions file |childdoc.def|.

% The definitions cannot be loaded using |\usepackage| or |\RequirePackage|
% which has a mechanism to prevent loading a style file more than once.
% When loading the definitions by means of |\input|
% multiple instances have to be prevented manually:
%\iffalse
%This code needs to be before the `\ProvidesFile' directive
%which is defined at the beginning of this file.
%Therefore it is also placed there and commented out here.
%</package>
%<*discard>
%\fi
%    \begin{macrocode}
\ifdefined\childdocmain\endinput\fi
%    \end{macrocode}
%\iffalse
%</discard>
%<*package>
%\fi
%
% \macro{\ifchilddoc}
% \macro{\ifchilddocmanual}
% The conditional |\ifchilddoc| tells whether a
% child (true) or main (false) document is being compiled.
% The conditional |\ifchilddocmanual| tells whether
% the |\includeonly| mechanism is used (false) or
% the selection of child files must be performed manually (true).
% The definitions initialise to false:
%    \begin{macrocode}
\newif\ifchilddoc
\newif\ifchilddocmanual
%    \end{macrocode}

% \macro{\childdocname}
% \macro{\childdocjob}
% The macro |\childdocname| stores the name of the main document
% to be compiled. The macro |\childdocjob| stores the name of
% the document on which the \LaTeX{} compiler was originally invoked.
% The content of |\jobname| cannot be compared
% to filenames specified in the source due to different catcodes.
% The following code rescans |\jobname|, stores the result
% in |\childdocname| and saves a copy in |\childdocjob|:
%    \begin{macrocode}
\edef\childdocname{\scantokens\expandafter{\jobname\noexpand}}
\let\childdocjob\childdocname
%    \end{macrocode}

% \macro{\childdocdisable}
% The macro |\childdocdisable| prevents the main file
% from being processed more than once.
% At this stage, the main document command |\childdocmain|
% is assumed to be called once again where it should do nothing.
% Any subsequent call to it should prevent
% a secondary processing of the main document
% It overwrites the forwarding commands
% |\childdocof| and |\childdocforward|
% with empty macros to prevent further inclusions of the main document:
%    \begin{macrocode}
\newcommand{\childdocdisable}
{
  \renewcommand{\childdocmain}[1]{\renewcommand{\childdocmain}[1]{\endinput}}
  \renewcommand{\childdocof}[1]{}
  \renewcommand{\childdocby}[2][]{}
  \renewcommand{\childdocforward}[2][]{}
  \renewcommand{\childdocdisable}{}
}
%    \end{macrocode}

% \macro{\childdocmain}
% The macro |\childdocmain| is to be called at the top of the main file
% with nothing or the main filename (without extension) as argument.
% First, it breaks loops.
% If the argument is not empty and does not match |\childdocname|
% (which is set by the first inclusion of |childdoc.def|),
% |\ifchilddoc| is set to true, |\includeonly| is applied to the child file
% and |\jobname| is set to the main file
% (for proper handling of |.aux| files):
%    \begin{macrocode}
\newcommand{\childdocmain}[1]
{
  \childdocdisable\childdocmain{}
  \if?#1?\else
    \begingroup
      \def\childdoctmp{#1}
      \ifx\childdoctmp\childdocname
        \def\childdoctmp{}
      \else
        \def\childdoctmp
        {
          \childdoctrue
          \includeonly{\childdocname}
          \def\childdocjob{#1}
          \def\jobname{#1}
        }
      \fi
      \expandafter
    \endgroup
    \childdoctmp
  \fi
}
%    \end{macrocode}

% \macro{\childdocof}
% The command |\childdocof| redirects
% compilation to the main file |#1|.
%    \begin{macrocode}
\newcommand{\childdocof}[1]
{
  \childdocdisable
  \childdoctrue
  \includeonly{\childdocname}
  \def\jobname{#1}
  \def\childdocjob{#1}
  \input{#1}
}
%    \end{macrocode}

% \macro{\childdocby}
% The command |\childdocby| ....
%    \begin{macrocode}
\newcommand{\childdocby}[2][]
{
  \childdocdisable
  \childdoctrue
  \childdocmanualtrue
  \if?#1?\else
    \def\jobname{#2}
  \fi
  \def\childdocjob{#2}
  \input{#2}
  \endinput
}
%    \end{macrocode}

% \macro{\childdocforward}
% The command |\childdocforward| redirects
% compilation to the main file or
% (if the optional argument is given) a child file.
% Parameters are set as if the main file
% or a child file starting with |\childdocof| was compiled.
% Then compilation is handed over to the main file:
%    \begin{macrocode}
\newcommand{\childdocforward}[2][]
{
  \begingroup
    \if?#1?
      \def\childdoctmp
      {
        \def\childdocname{#2}
        \def\childdocjob{#2}
        \def\jobname{#2}
        \input{#2}
        \endinput
      }
    \else
      \def\childdoctmp
      {
        \childdocdisable
        \def\childdocname{#2}
        \childdoctrue
        \includeonly{#2}
        \def\childdocjob{#1}
        \def\jobname{#1}
        \input{#1}
        \endinput
      }
    \fi
    \expandafter
  \endgroup
  \childdoctmp
}
%    \end{macrocode}

% \macro{\childdocforwardprefix}
% The command |\childdocforwardprefix| redirects
% compilation to the main or a child file by means of a pattern.
% The prefix |#1| in the current filename is replaced by |#2|
% and the suffix of the current filename is kept
% (it is assumed that the filename does not contain the substring `|~~~|'
% which is used as a delimiter).
% Compilation is handed over to the new file by |\childdocforward|:
%    \begin{macrocode}
\newcommand{\childdocforwardprefix}[3][]
{
  \begingroup
    \def\childdocextract #2##1~~~{\def\childdoctmp{\childdocforward[#1]{#3##1}}}
    \expandafter\childdocextract\childdocname~~~
    \expandafter
  \endgroup
  \childdoctmp
}
%    \end{macrocode}

% \macro{\childdoc}
% The deprecated macro |\childdoc| is a legacy version of |\childdocmain|:
%    \begin{macrocode}
\newcommand{\childdoc}{\childdocmain}
%    \end{macrocode}

% \macro{\childdocredirect}
% The deprecated macro |\childdocredirect| is a legacy version
% of |\childdocforward| and |\childdocforwardprefix|:
%    \begin{macrocode}
\newcommand{\childdocredirect}[2][]
{
  \begingroup
    \if?#1?
      \def\childdoctmp{\childdocforward{#2}}
    \else
      \def\childdoctmp{\childdocforwardprefix{#1}{#2}}
    \fi
    \expandafter
  \endgroup
  \childdoctmp
}
%    \end{macrocode}

%\iffalse
%</package>
%\fi
%
\endinput

\childdocmain{}
%    \end{macrocode}

% Optional override for |\version| flag:
%    \begin{macrocode}
%%\ifchilddoc\else\providecommand{\version}{draft}\fi
%    \end{macrocode}

% Define the default values for the |\version| flag
% (|final| for the main file and |draft| for childs):
%    \begin{macrocode}
\ifchilddoc
\providecommand{\version}{draft}
\else
\providecommand{\version}{final}
\fi
%    \end{macrocode}

% Load the standard document class:
%    \begin{macrocode}
\documentclass[12pt]{article}
%    \end{macrocode}

% Start the document body:
%    \begin{macrocode}
\begin{document}
%    \end{macrocode}

% Declare a title page.
% Print title, part of document being processed and version flag:
%    \begin{macrocode}
\addtocounter{page}{-1}
\begin{center}
{\LARGE\bfseries{}childdoc example\par}
\vspace{1cm}
\ifchilddoc
\ifchilddocmanual part\else chapter\fi:
`\childdocname' of `\childdocjob'\par
\else
main document: `\childdocjob'\par
\fi
version: \version\par
\end{center}
\newpage
%    \end{macrocode}

% Manually include selected file,
% otherwise process as usual:
%    \begin{macrocode}
\ifchilddocmanual
\section*{part `\childdocname'}
\input{\childdocname}
\else
%    \end{macrocode}

% Include the two chapters:
%    \begin{macrocode}
\include{cdocsch1}
\include{cdocsch2}
%    \end{macrocode}

% Include the two parts unless only chapters should be displayed:
%    \begin{macrocode}
\ifchilddoc\else
\section{part three}
\input{cdocspt3}
\section{part four}
\input{cdocspt4}
\fi
%    \end{macrocode}

% Process as usual until here:
%    \begin{macrocode}
\fi
%    \end{macrocode}

% End of document body:
%    \begin{macrocode}
\end{document}
%    \end{macrocode}
%\iffalse
%</samplemain>
%\fi
%
% %%%%%%%%%%%%%%%%%%%%%%%%%%%%%%%%%%%%%%
% \paragraph{Chapter Include Files.}
%
% The include files are called |cdocsch1.tex| and |cdocsch2.tex|.
%
%\iffalse
%<*samplechap1|samplechap2>
%\fi

% Optional override for |\version| flag:
%    \begin{macrocode}
%%\providecommand{\version}{final}
%    \end{macrocode}

% Include the main document:
%    \begin{macrocode}
% \iffalse
%
% childdoc.dtx Copyright (C) 2017-2018 Niklas Beisert
%
% This work may be distributed and/or modified under the
% conditions of the LaTeX Project Public License, either version 1.3
% of this license or (at your option) any later version.
% The latest version of this license is in
%   http://www.latex-project.org/lppl.txt
% and version 1.3 or later is part of all distributions of LaTeX
% version 2005/12/01 or later.
%
% This work has the LPPL maintenance status `maintained'.
%
% The Current Maintainer of this work is Niklas Beisert.
%
% This work consists of the files childdoc.dtx and childdoc.ins
% and the derived files childdoc.def and cdocsamp.tex with
% cdocsch1.tex, cdocsch2.tex, cdocsdrf.tex, cdocsfn1.tex, cdocsfn2.tex.
%
%<package>\ifdefined\childdocmain\endinput\fi
%<package>\ProvidesFile{childdoc.def}[2018/12/30 v2.0 child document driver]
%<samplemain>\ProvidesFile{cdocsamp.tex}[2018/12/30 v2.0 sample for childdoc]
%<*driver>
%\ProvidesFile{childdoc.drv}[2018/12/30 v2.0 childdoc reference manual file]
\PassOptionsToClass{10pt,a4paper}{article}
\documentclass{ltxdoc}

\usepackage[margin=35mm]{geometry}
\usepackage{hyperref}
\usepackage{hyperxmp}
\usepackage[usenames]{color}

\hypersetup{colorlinks=true}
\hypersetup{pdfstartview=FitH}
\hypersetup{pdfpagemode=UseNone}
\hypersetup{pdfsource={}}
\hypersetup{pdflang={en-UK}}
\hypersetup{pdfcopyright={Copyright 2017-2018 Niklas Beisert.
  This work may be distributed and/or modified under the
  conditions of the LaTeX Project Public License, either version 1.3
  of this license or (at your option) any later version.}}
\hypersetup{pdflicenseurl={http://www.latex-project.org/lppl.txt}}
\hypersetup{pdfcontactaddress={ETH Zurich, ITP, HIT K,
  Wolfgang-Pauli-Strasse 27}}
\hypersetup{pdfcontactpostcode={8093}}
\hypersetup{pdfcontactcity={Zurich}}
\hypersetup{pdfcontactcountry={Switzerland}}
\hypersetup{pdfcontactemail={nbeisert@itp.phys.ethz.ch}}
\hypersetup{pdfcontacturl={http://people.phys.ethz.ch/\xmptilde nbeisert/}}

\newcommand{\secref}[1]{\hyperref[#1]{section \ref*{#1}}}

\parskip1ex
\parindent0pt
\let\olditemize\itemize
\def\itemize{\olditemize\parskip0pt}

\begin{document}

\title{The \textsf{childdoc} Package}
\hypersetup{pdftitle={The childdoc Package}}
\author{Niklas Beisert\\[2ex]
  Institut f\"ur Theoretische Physik\\
  Eidgen\"ossische Technische Hochschule Z\"urich\\
  Wolfgang-Pauli-Strasse 27, 8093 Z\"urich, Switzerland\\[1ex]
  \href{mailto:nbeisert@itp.phys.ethz.ch}
  {\texttt{nbeisert@itp.phys.ethz.ch}}}
\hypersetup{pdfauthor={Niklas Beisert}}
\hypersetup{pdfsubject={Manual for the LaTeX2e Package childdoc}}
\date{30 December 2018, \textsf{v2.0}}
\maketitle

\begin{abstract}\noindent
\textsf{childdoc} is a \LaTeXe{} package
that enables the direct compilation
of document sections included by |\include|
to individual files.
\end{abstract}

\begingroup
\parskip0ex
\tableofcontents
\endgroup

%%%%%%%%%%%%%%%%%%%%%%%%%%%%%%%%%%%%%%%%%%%%%%%%%%%%%%%%%%%%%%%%%%%%%%%%%%%%%%%%
%%%%%%%%%%%%%%%%%%%%%%%%%%%%%%%%%%%%%%%%%%%%%%%%%%%%%%%%%%%%%%%%%%%%%%%%%%%%%%%%
\section{Introduction}

\LaTeX{} provides a mechanism to structure a large document (such as a book)
into a main file and several child files (containing the chapters)
using the |\include| command.
This mechanism is beneficial for documents
which span hundreds of pages in order to
make the source file(s) more manageable.
Moreover, compilation can be restricted to
selected child files by means of the |\includeonly| command.
The latter feature can be used to reduce the compilation time while editing
(this was significantly more useful in the earlier days of \LaTeX{})
or to generate a smaller document which is easier to navigate.
Another application of |\includeonly| is to generate
documents consisting of selected parts of the complete document.

However, there are a few drawbacks of the plain |\include| mechanism:
\begin{itemize}
\item
The child files cannot be compiled on their own,
they can only be compiled via the main file.
A naive editing environment
(such as a text editor with an option
to have the current file processed by \LaTeX)
may require one to switch to the main file before compiling;
attempting to compile the child file produces errors.
\item
The main file must be modified (each time)
to adjust the |\includeonly| command
to the present needs. This easily leaves the main file in a messy state.
\item
The generated document will always carry the filename
of the main document. This is inconvenient if
several child files are to be compiled and
to be kept for distribution.
\end{itemize}

The present package provides a simple interface
to make child files individually compilable by \LaTeX{}.
Compiling a child file then has the same effect as compiling
the main file with an |\includeonly| command
to select the appropriate child.
Moreover the generated document will carry the name of the child
rather than the main file.
This resolves all three above issues.

This feature is meant to make the editing of books,
thesis documents and lecture notes somewhat more convenient.
However, the package can also be used efficiently for
composing a series of documents (such as exercise sheets)
which are typically distributed individually.
It then assists the author in generating the individual documents
(potentially in different versions)
as well as a document containing the collected series.
Another application is in developing style files
or other kinds of included material
where compilation of the style file could redirect
to a sample or test file.

%%%%%%%%%%%%%%%%%%%%%%%%%%%%%%%%%%%%%%%%%%%%%%%%%%%%%%%%%%%%%%%%%%%%%%%%%%%%%%%%
%%%%%%%%%%%%%%%%%%%%%%%%%%%%%%%%%%%%%%%%%%%%%%%%%%%%%%%%%%%%%%%%%%%%%%%%%%%%%%%%
\section{Usage}

First of all, the package \textsf{childdoc} is \emph{not} a standard
\LaTeXe{} |.sty| style file! Therefore it needs to be invoked in
a non-standard way.

%%%%%%%%%%%%%%%%%%%%%%%%%%%%%%%%%%%%%%%%%%%%%%%%%%%%%%%%%%%%%%%%%%%%%%%%%%%%%%%%
\subsection{Included Files}
\label{sec:include}

%%%%%%%%%%%%%%%%%%%%%%%%%%%%%%%%%%%%%%%%
\DescribeMacro{\childdocmain}
To use the package, add the commands
\begin{center}
\begin{tabular}{l}
|\input{childdoc.def}|\\
|\childdocmain{}|\\
\end{tabular}
\end{center}
at the very top of the main \LaTeX{} file,
in particular \emph{before} the |\documentclass| statement!
The argument of |\childdocmain| should be left empty
(but it must be present).

%%%%%%%%%%%%%%%%%%%%%%%%%%%%%%%%%%%%%%%%
\DescribeMacro{\childdocof}
Furthermore, add the commands
\begin{center}
\begin{tabular}{l}
|\input{childdoc.def}|\\
|\childdocof{|\textit{main}|}|\\
\end{tabular}
\end{center}
at the top of every child file \textit{child}
which is included by |\include{|\textit{child}|}|
from within the main file
(or at least for those files to be compiled individually).
The argument \textit{main} must be the filename of the main file.

There are a couple of
considerations in setting up the main and child documents:

%%%%%%%%%%%%%%%%%%%%%%%%%%%%%%%%%%%%%%%%
\paragraph{Restrictions.}

Please note the following restrictions:
\begin{itemize}
\item
|\childdocmain| must be called with one argument \textit{main}
to ensure compatibility with earlier version of the package.
It must either be empty (|\childdocmain{}|)
or precisely match the filename of the main file in which it is specified.
See \secref{sec:detection} for further information.
\item
The filename \textit{main} must be specified without the |.tex| extension.
\item
The filename \textit{main} is case sensitive
(even in case-insensitive file systems)
due to internal string comparison.
\item
The argument \textit{main} should be fully expanded, it cannot be a macro.
\item
Subdirectories and special characters should be avoided in filenames.
\item
The command |\childdocmain{|\textit{main}|}| must be followed by a whitespace.
It should not be followed immediately by another command
or by a comment mark `|%|'.
This is because the \TeX{} parser reads the token immediately following
the argument of |\childdocmain| and puts it
at the beginning of every child section;
however, a white\-space is ignored.
\end{itemize}

%%%%%%%%%%%%%%%%%%%%%%%%%%%%%%%%%%%%%%%%
\paragraph{Content of Main File.}

It is advisable to place all content in the child files included by |\include|.
Any output contained in the main file will appear in all child documents
unless suppressed manually;
it cannot be suppressed automatically by the |\includeonly| directive
and thus should normally be avoided.
A method to include some content in the main file
by means of conditional processing is described in \secref{sec:conditional}.

%%%%%%%%%%%%%%%%%%%%%%%%%%%%%%%%%%%%%%%%
\paragraph{Page Numbering.}

When only a part of the document is compiled,
the appropriate numbering of pages
(as well as other status parameters)
is determined from the |.aux| files.
The latter contain information from previous passes.
However this information needs to propagate through
all intermediate child documents.
Therefore the page numbering in child documents may well
be inconsistent until the complete document is compiled at least once.

A useful (if unconventional) way to always ensure a consistent
page numbering is to restart the numbering in each child document
and denote the pages by `\textit{child}|.|\textit{page}'
where \textit{child} represents the chapter/section number of the child file.
This can be achieved by the command
|\numberwithin{page}{|\textit{child}|}|
of the \textsf{amsmath} package
where \textit{child} can be |chapter| or |section|
depending on the chosen structuring.
Alternatively, one can modify the macro |\thepage| appropriately
and reset the counter |page| at the start of each child file.

%%%%%%%%%%%%%%%%%%%%%%%%%%%%%%%%%%%%%%%%%%%%%%%%%%%%%%%%%%%%%%%%%%%%%%%%%%%%%%%%
\subsection{Conditional Processing}
\label{sec:conditional}

The package provides a mechanism to compile different versions
of a document. To customise the versions further some conditional processing
can come in handy to distinguish which version is being compiled.
The package provides two macros to describe the compilation context:

%%%%%%%%%%%%%%%%%%%%%%%%%%%%%%%%%%%%%%%%
\DescribeMacro{\ifchilddoc}
The conditional |\ifchilddoc| distinguishes between the compilation of
child documents and the main document:
%
\begin{center}
|\ifchilddoc |\textit{child-code}| |[|\||else |\textit{main-code}]| \||fi|
\end{center}

%%%%%%%%%%%%%%%%%%%%%%%%%%%%%%%%%%%%%%%%
\DescribeMacro{\childdocname}
\DescribeMacro{\childdocjob}
The macro |\childdocname| contains the filename (without extension)
of the main or child file being processed.
Note that |\childdocjob| will always contain the name of the main file.

%%%%%%%%%%%%%%%%%%%%%%%%%%%%%%%%%%%%%%%%
\paragraph{Title Page.}

Conditional processing can be used to include a title or banner page
in the main document when proper precautions are taken.
Importantly, the code in the main file should ensure that the page counter
(as well as other status parameters which are stored in the |.aux| files)
takes the same value after the conditional processing.
Otherwise the page numbers may take divergent values
depending on which part is compiled.

For example, a title page could be declared by:
%
\begin{center}
\begin{tabular}{l}
|\ifchilddoc\||else|\\
|\addtocounter{page}{-1}|\\
\textit{code for title page}\\
|\newpage|\\
|\||fi|
\end{tabular}
\end{center}
%
A banner page for the child documents can be generated by:
%
\begin{center}
\begin{tabular}{l}
|\ifchilddoc|\\
|\addtocounter{page}{-1}|\\
\textit{code for banner page}\\
|\newpage|\\
|\||fi|
\end{tabular}
\end{center}
%
Here one could write a message such as:
\begin{center}
|This is the part \childdocname{} of \childdocjob{}.|
\end{center}

%%%%%%%%%%%%%%%%%%%%%%%%%%%%%%%%%%%%%%%%%%%%%%%%%%%%%%%%%%%%%%%%%%%%%%%%%%%%%%%%
\subsection{Flags}
\label{sec:flags}

The package makes it easy to generate different versions
of the main or child documents.
To this end compilation flags can be defined
and assigned different default values.
They will be particularly useful in conjunction
with the forwarding mechanism described in \secref{sec:forward}.

For example, it may be useful to have a flag |\version|
which can be set to |draft| or |final|.
The document source will contain some conditional code
depending on the value of |\version|.
Suppose further, the flag should default to |final| for the main file
and to |draft| for child files
which is a natural assignment for editing the document.
This is achieved by placing the following code
in the preamble of the main document
(below the |\childdocmain| directive):
%
\begin{center}
\begin{tabular}{l}
|\ifchilddoc|\\
|\providecommand{\version}{draft}|\\
|\||else|\\
|\providecommand{\version}{final}|\\
|\||fi|
\end{tabular}
\end{center}
%
The definition by |\providecommand| makes sure
that previous definitions are not overwritten.
Further statements |\providecommand{\version}{...}|
can thus be added before the above code to override it.

For the main file, one might add a line
(between |\childdocmain| and the above block)
%
\begin{center}
|%\ifchilddoc\||else\providecommand{\version}{draft}\||fi|
\end{center}
%
which can be uncommented to produce a draft version.
Likewise one can add a line to the very top of a child file
(above the |\childdocof{|\textit{main}|}| directive)
%
\begin{center}
|%\providecommand{\version}{final}|
\end{center}
%
which can be uncommented to produce the final version of this child document.

%%%%%%%%%%%%%%%%%%%%%%%%%%%%%%%%%%%%%%%%%%%%%%%%%%%%%%%%%%%%%%%%%%%%%%%%%%%%%%%%
\subsection{Forwarding}
\label{sec:forward}

Different versions of the main or child documents
using compilation flags as described in \secref{sec:flags}
can be (permanently) stored in different files
for convenient compilation, viewing and distribution.
To this end, the package defines a command
to pass on compilation to a different file:

%%%%%%%%%%%%%%%%%%%%%%%%%%%%%%%%%%%%%%%%
\DescribeMacro{\childdocforward}
The command |\childdocforward| redirects processing to
another source file:
%
\begin{center}
\begin{tabular}{l}
|\input{childdoc.def}|\\
|\childdocforward[|\textit{main}|]{|\textit{dest}|}|\\
\end{tabular}
\end{center}
%
The argument \textit{dest} is the destination file
(without extension).
It should be the main file or one of the child files.
Note that further \textsf{childdoc} directives
such as |\childdocof| and |\childdocforward|
in the indicated file will be processed in this form.
The optional argument \textit{main}
passes on directly to the main file \textit{main}
while pretending to compile the child \textit{dest}.
This form behaves as if \textit{dest}
issues |\childdocof{|\textit{main}|}| right away,
and no further \textsf{childdoc} directives will be processed.

%%%%%%%%%%%%%%%%%%%%%%%%%%%%%%%%%%%%%%%%
\DescribeMacro{\...prefix}
In the alternative form |\childdocforwardprefix|,
%
\begin{center}
\begin{tabular}{l}
|\input{childdoc.def}|\\
|\childdocforwardprefix[|\textit{main}|]{|\textit{prefix}|}{|\textit{dest}|}|
\end{tabular}
\end{center}
%
the destination file is determined by a pattern
depending on the current file:
To make this work, the current file must be called
`{\textit{prefix}\hspace{0.2em}\textit{suffix}}'
with \textit{prefix} matching precisely the argument.
Processing is then passed on to the file
`{\textit{dest}\hspace{0.2em}\textit{suffix}}'.
Surely, the same effect is achieved by
directly specifying the
argument `{\textit{dest}\hspace{0.2em}\textit{suffix}}'
in the first form.
However, that requires to set up a different file
for each child. With the alternative form of the command
all these files can have exactly the same content
which simplifies setting them up and maintaining them.

For example, the following file |draft.tex|
with a compilation flag |\version| as described in \secref{sec:flags}
compiles the main document as a draft:
%
\begin{center}
\begin{tabular}{l}
|\def\version{draft}|\\
|\input{childdoc.def}|\\
|\childdocforward{|\textit{main}|}|
\end{tabular}
\end{center}
%
Likewise, the following files |final|\textit{nn}|.tex|
compile the final version of the child document
|child|\textit{nn}|.tex|:
%
\begin{center}
\begin{tabular}{l}
|\def\version{final}|\\
|\input{childdoc.def}|\\
|\childdocforwardprefix{final}{child}|
\end{tabular}
\end{center}
%

Note that when several versions of a main file and/or of each child file
are to be generated, it may be convenient to set up a |Makefile| or
shell script to automatise the process.

%%%%%%%%%%%%%%%%%%%%%%%%%%%%%%%%%%%%%%%%%%%%%%%%%%%%%%%%%%%%%%%%%%%%%%%%%%%%%%%%
\subsection{Command Line Processing}
\label{sec:commandline}

The effect of redirection files can also be achieved by invoking
the \LaTeX{} compiler with a more elaborate command line.
Most conveniently this should be done as part
of a shell script or a |Makefile|.

When using \textsf{childdoc} in the main file, the following
command lines effectively perform a redirection
(note that depending on the shell being used,
backslashes may have to be doubled: `|\|' $\to$ `|\\|'):
%
\begin{center}
|... -jobname "|\textit{target}|" |\\|"|[\textit{flags}]%
|\input{childdoc.def}\childdocforward[|\textit{main}|]{|\textit{dest}|}"|
\end{center}
%
Here \textit{target} is the name of the output file,
\textit{main} is the name of the main file
and \textit{dest} is the name of the main or child file to be processed
(all filenames without extensions).
The optional argument \textit{main} can be omitted
if \textit{main} matches \textit{dest}.
Optionally, compilation \textit{flags} can be defined via |\def| commands.
This command line makes the \TeX{} engine believe
it is compiling the file \textit{target}
whose content is specified as the latter parameter.
The provided code then forwards the processing to
\textit{main} or \textit{dest} as described in \secref{sec:forward}.

%%%%%%%%%%%%%%%%%%%%%%%%%%%%%%%%%%%%%%%%%%%%%%%%%%%%%%%%%%%%%%%%%%%%%%%%%%%%%%%%
\subsection{Include by Input}
\label{sec:input}

Including child documents by |\include| has some restrictions by design.
Most notably, the content of a child document always occupies
its own set of pages; pages cannot be shared between child documents.
Usually, this behaviour makes perfect sense
because each child document contain an essential part of the document.
However, in some situations it may be desirable to compose
a document from a collection of parts
without having mandatory page breaks between then.
For this case, the package
provides a mechanism to include parts
by |\input| which can also be processed individually.
However, by construction this mechanism
requires manual handling of the content to be output.

%%%%%%%%%%%%%%%%%%%%%%%%%%%%%%%%%%%%%%%%
\DescribeMacro{\ifchilddocmanual}
The main file should be prepared as usual, see \secref{sec:include}.
However, the document body must make a distinction
between processing of an individual part and of the main document, e.g.:
%
\begin{center}
\begin{tabular}{l}
|\ifchilddocmanual|\\
|\input{\childdocname}|\\
|\||else|\\
\textit{document body with }|\input{|\textit{part}|}|\\
|\||fi|
\end{tabular}
\end{center}
%
The conditional |\ifchilddocmanual| is true whenever
a part to be included by |\input| is being compiled,
and the name of the part is stored in |\childdocname|.

%%%%%%%%%%%%%%%%%%%%%%%%%%%%%%%%%%%%%%%%
\DescribeMacro{\childdocby}
Each part to be included by |\input| should start with:
%
\begin{center}
\begin{tabular}{l}
|\input{childdoc.def}|\\
|\childdocby{|\textit{main}|}|\\
\end{tabular}
\end{center}
%
The directive |\childdocby| is similar to |\childdocof|
described in \secref{sec:include},
but the subsequent selection of content must be done manually.
To that end, both |\ifchilddoc| and |\ifchilddocmanual|
will be true upon processing of a part,
and the name of the part is stored in |\childdocname|.
Note that |\jobname| will be set to the filename of the current part
so that each part receives an individual |.aux| file
that does not interfere with the |.aux| file(s) of the main document.
This behaviour can be altered by the alternative form
|\childdocby[*]{|\textit{main}|}| (with a non-empty optional argument)
which uses the |.aux| file of the main document
by setting |\jobname| to \textit{main}.

%%%%%%%%%%%%%%%%%%%%%%%%%%%%%%%%%%%%%%%%%%%%%%%%%%%%%%%%%%%%%%%%%%%%%%%%%%%%%%%%
\subsection{Driver Development}
\label{sec:driver}

The \textsf{childdoc} mechanism can also be use for the development
of definition files such as \LaTeX{} styles or classes.
This case differs from the above setup with multiple parts
included by |\include| in that no |\includeonly| should be invoked.
This can be achieved by starting the include file
(before |\ProvidesPackage|) with:
%
\begin{center}
\begin{tabular}{l}
|\input{childdoc.def}|\\
|\childdocforward{|\textit{main}|}|\\
\end{tabular}
\end{center}
%
or alternatively with:
%
\begin{center}
\begin{tabular}{l}
|\input{childdoc.def}|\\
|\childdocby{|\textit{main}|}|\\
\end{tabular}
\end{center}
%
Both forms have slightly different effects as described above.
The main file is prepared as usual, see \secref{sec:include}.

%%%%%%%%%%%%%%%%%%%%%%%%%%%%%%%%%%%%%%%%%%%%%%%%%%%%%%%%%%%%%%%%%%%%%%%%%%%%%%%%
\subsection{Legacy Detection}
\label{sec:detection}

The directive |\childdocmain| in the main file can detect
whether the complete document or merely a child is to be compiled
even without using the directive |\childdocof|.
This method is deprecated because it is less robust
and there is no compelling reason to use it;
it is merely provided for backward compatibility
and it may be removed in future versions.

If the detection mechanism is to be used,
it is mandatory to correctly specify
the filename of the main file as the argument of |\childdocmain|:
%
\begin{center}
\begin{tabular}{l}
|\input{childdoc.def}|\\
|\childdocmain{|\textit{main}|}|\\
\end{tabular}
\end{center}
%
If |\jobname| does not match the argument \textit{main} of |\childdocmain|,
it is assumed that |\jobname| points to the child file to be compiled.
When using |\childdocmain| with the main file specified as argument,
it suffices to start a child file
with just |\input{|\textit{main}|}|
without loading of the package and using |\childdocof|.
If instead all processing is done
with the appropriate \textsf{childdoc} directives,
the argument of \textit{main} of |\childdocmain| can be empty.

An alternative version of the command line processing described
in \secref{sec:commandline} using the detection mechanism reads:
%
\begin{center}
|... -jobname "|\textit{target}|" "|[\textit{flags}]%
[|\def\jobname{|\textit{dest}|}|]|\input{|\textit{main}|}"|
\end{center}

%%%%%%%%%%%%%%%%%%%%%%%%%%%%%%%%%%%%%%%%%%%%%%%%%%%%%%%%%%%%%%%%%%%%%%%%%%%%%%%%
\subsection{Manual Code}
\label{sec:manual}

In case one cannot be certain whether the definitions file |childdoc.def|
is installed on the target \TeX{} distribution
and one prefers not to ship it,
it is conceivable to paste a few relevant commands into the sources.

To that end, drop all statements |\input{childdoc.def}|
and perform the replacements as outlined below.
Instead of |\childdocmain{|\textit{main}|}| add the following code
to the top of the main file:
%
\begin{center}
\begin{tabular}{l}
|\||ifdefined\childdocname\endinput\||fi\newif\ifchilddoc|\\
|\edef\childdocname{\scantokens\expandafter{\jobname\noexpand}}|\\
|\def\childdocmain{|\textit{main}|}\||ifx\childdocmain\childdocname\||else|\\
|\childdoctrue\includeonly{\childdocname}\let\jobname\childdocmain\||fi|\\
\end{tabular}
\end{center}
%
Instead of |\childdocof{|\textit{main}|}| just include the main file
at the top of each child file:
%
\begin{center}
|\input{|\textit{main}|}|
\end{center}
%
A simple redirection |\childdocforward{|\textit{dest}|}| is achieved by:
%
\begin{center}
|\def\jobname{|\textit{dest}|}\input{\jobname}|
\end{center}
%
The redirection with prefix
|\childdocforwardprefix[|\textit{prefix}|]{|\textit{dest}|}|
is accomplished by:
%
\begin{center}
\begin{tabular}{l}
|{\edef\jobname{\scantokens\expandafter{\jobname\noexpand}}|\\
|\def\redirectjob |\textit{prefix}|#1~~~{\gdef\jobname{|\textit{dest}|#1}}|\\
|\expandafter\redirectjob\jobname~~~}\input{\jobname}|
\end{tabular}
\end{center}

In an alternative approach,
child documents can be compiled by a specific command line
without additional code or specific definitions:
%
\begin{center}
|... -jobname "|\textit{target}|" "|[\textit{flags}]%
|\includeonly{|\textit{dest}|}\input{|\textit{main}|}"|
\end{center}
%

%%%%%%%%%%%%%%%%%%%%%%%%%%%%%%%%%%%%%%%%%%%%%%%%%%%%%%%%%%%%%%%%%%%%%%%%%%%%%%%%
%%%%%%%%%%%%%%%%%%%%%%%%%%%%%%%%%%%%%%%%%%%%%%%%%%%%%%%%%%%%%%%%%%%%%%%%%%%%%%%%
\section{Information}

%%%%%%%%%%%%%%%%%%%%%%%%%%%%%%%%%%%%%%%%%%%%%%%%%%%%%%%%%%%%%%%%%%%%%%%%%%%%%%%%
\subsection{Copyright}

Copyright \copyright{} 2017--2018 Niklas Beisert

This work may be distributed and/or modified under the
conditions of the \LaTeX{} Project Public License, either version 1.3
of this license or (at your option) any later version.
The latest version of this license is in
  \url{http://www.latex-project.org/lppl.txt}
and version 1.3 or later is part of all distributions of \LaTeX{}
version 2005/12/01 or later.

This work has the LPPL maintenance status `maintained'.

The Current Maintainer of this work is Niklas Beisert.

This work consists of the files |README.txt|, |childdoc.ins| and |childdoc.dtx|
as well as the derived files |childdoc.def|, |cdocsamp.tex|
with |cdocsch1.tex|, |cdocsch2.tex|, |cdocspt3.tex|, |cdocspt4.tex|,
|cdocsdrf.tex|, |cdocsfn1.tex|, |cdocsfn2.tex|
as well as |childdoc.pdf|.

%%%%%%%%%%%%%%%%%%%%%%%%%%%%%%%%%%%%%%%%%%%%%%%%%%%%%%%%%%%%%%%%%%%%%%%%%%%%%%%%
\subsection{Files and Installation}

The package consists of the files:
%
\begin{center}
\begin{tabular}{ll}
    |README.txt|   & readme file \\
    |childdoc.ins| & installation file \\
    |childdoc.dtx| & source file \\
    |childdoc.def| & definition file \\
    |cdocsamp.tex| & sample main file \\
    |cdocsch1.tex| & sample include file \\
    |cdocsch2.tex| & sample include file \\
    |cdocspt3.tex| & sample part file \\
    |cdocspt4.tex| & sample part file \\
    |cdocsdrf.tex| & sample redirection file \\
    |cdocsfn1.tex| & sample redirection file \\
    |cdocsfn2.tex| & sample redirection file \\
    |childdoc.pdf| & manual
\end{tabular}
\end{center}
%
The distribution consists of the files
|README.txt|, |childdoc.ins| and |childdoc.dtx|.
%
\begin{itemize}
\item
Run (pdf)\LaTeX{} on |childdoc.dtx|
to compile the manual |childdoc.pdf| (this file).
\item
Run \LaTeX{} on |childdoc.ins| to create the definitions file |childdoc.def|
and the sample |cdocsamp.tex| with include files
|cdocsch1.tex|, |cdocsch2.tex|, |cdocspt3.tex|, |cdocspt4.tex|,
|cdocsdrf.tex|, |cdocsfn1.tex|, |cdocsfn2.tex|.
Then copy the file |childdoc.def| to an appropriate directory of your \LaTeX{}
distribution, e.g.\ \textit{texmf-root}|/tex/latex/childdoc|.
\end{itemize}

%%%%%%%%%%%%%%%%%%%%%%%%%%%%%%%%%%%%%%%%%%%%%%%%%%%%%%%%%%%%%%%%%%%%%%%%%%%%%%%%
\subsection{Related CTAN Packages}

There are several other packages which offer a similar functionality:
%
\begin{itemize}
\item
The packages
\href{http://ctan.org/pkg/docmute}{\textsf{docmute}},
\href{http://ctan.org/pkg/includex}{\textsf{includex}} and
\href{http://ctan.org/pkg/standalone}{\textsf{standalone}}
provide commands to include only the document body of
a child file thus allowing both files to be compiled individually.
\item
The packages \href{http://ctan.org/pkg/subdocs}{\textsf{subdocs}}
and \href{http://ctan.org/pkg/subfiles}{\textsf{subfiles}}
provide structures in which the main and child documents can be
encapsulated and allowing them to be compiled individually.
The inclusion mechanism is different from the conventional |\include|.
\item
The package \href{http://ctan.org/pkg/combine}{\textsf{combine}}
is an elaborate solution to combine several documents into one.
\end{itemize}
%
See also the CTAN topic \href{http://ctan.org/topic/subdocs}{\textsf{subdocs}}
for further related packages.
The present package differs from the above solutions in that
a document structure constructed with the conventional |\include| mechanism
just needs two extra commands at the top of every file
such that all constituent files can be compiled individually.

%%%%%%%%%%%%%%%%%%%%%%%%%%%%%%%%%%%%%%%%%%%%%%%%%%%%%%%%%%%%%%%%%%%%%%%%%%%%%%%%
%\subsection{Feature Suggestions}
%
%The following is a list of features which may be useful for future
%versions of this package:
%%
%\begin{itemize}
%\item
%\ldots
%\end{itemize}

%%%%%%%%%%%%%%%%%%%%%%%%%%%%%%%%%%%%%%%%%%%%%%%%%%%%%%%%%%%%%%%%%%%%%%%%%%%%%%%%
\subsection{Revision History}

%%%%%%%%%%%%%%%%%%%%%%%%%%%%%%%%%%%%%%%%
\paragraph{v2.0:} 2018/12/30

\begin{itemize}
\item
immediate forward processing
\item
added |\childdocby| mechanism
\item
manual restructured
\end{itemize}

%%%%%%%%%%%%%%%%%%%%%%%%%%%%%%%%%%%%%%%%
\paragraph{v1.6:} 2018/01/17

\begin{itemize}
\item
application for development of include files
\item
corrections to manual
\end{itemize}

%%%%%%%%%%%%%%%%%%%%%%%%%%%%%%%%%%%%%%%%
\paragraph{v1.5:} 2017/05/21

\begin{itemize}
\item
more complete structuring introduced
\item
|\childdocof| introduced
\item
|\childdoc| renamed to |\childdocmain|
\item
|\childredirect| renamed to |\childdocforward| and |\childdocforwardprefix|
and functionality expanded
\end{itemize}

%%%%%%%%%%%%%%%%%%%%%%%%%%%%%%%%%%%%%%%%
\paragraph{v1.0:} 2017/04/27

\begin{itemize}
\item
manual and install package
\item
first version published on CTAN
\end{itemize}

%%%%%%%%%%%%%%%%%%%%%%%%%%%%%%%%%%%%%%%%
\paragraph{v0.6:} 2017/04/26

\begin{itemize}
\item
redirection mechanism added
\end{itemize}

%%%%%%%%%%%%%%%%%%%%%%%%%%%%%%%%%%%%%%%%
\paragraph{v0.5:} 2017/04/26

\begin{itemize}
\item
functionality in definition file
\end{itemize}


%%%%%%%%%%%%%%%%%%%%%%%%%%%%%%%%%%%%%%%%%%%%%%%%%%%%%%%%%%%%%%%%%%%%%%%%%%%%%%%%
%%%%%%%%%%%%%%%%%%%%%%%%%%%%%%%%%%%%%%%%%%%%%%%%%%%%%%%%%%%%%%%%%%%%%%%%%%%%%%%%
%%%%%%%%%%%%%%%%%%%%%%%%%%%%%%%%%%%%%%%%%%%%%%%%%%%%%%%%%%%%%%%%%%%%%%%%%%%%%%%%
\appendix

\settowidth\MacroIndent{\rmfamily\scriptsize 000\ }

 \DocInput{childdoc.dtx}

\end{document}
%</driver>
% \fi
%
% %%%%%%%%%%%%%%%%%%%%%%%%%%%%%%%%%%%%%%%%%%%%%%%%%%%%%%%%%%%%%%%%%%%%%%%%%%%%%%
% %%%%%%%%%%%%%%%%%%%%%%%%%%%%%%%%%%%%%%%%%%%%%%%%%%%%%%%%%%%%%%%%%%%%%%%%%%%%%%
% \section{Sample}
%\iffalse
%<*samplemain>
%\fi
%
% The following presents a sample document
% with two chapters, two parts, a title page,
% a compile flag as well as three forwarding files to set the flag.
% It consists of eight |.tex| files:
% \begin{center}
% \begin{tabular}{ll}
% |cdocsamp.tex|&main file\\
% |cdocsch1.tex|&include file for chapter 1\\
% |cdocsch2.tex|&include file for chapter 2\\
% |cdocspt3.tex|&include file for part 3\\
% |cdocspt4.tex|&include file for part 4\\
% |cdocsdrf.tex|&forwarding file for main file in draft mode\\
% |cdocsfi1.tex|&forwarding file for final version of chapter 1\\
% |cdocsfi2.tex|&forwarding file for final version of chapter 2\\
% \end{tabular}
% \end{center}
% Each of the eight files can be compiled directly by the \LaTeX{} compiler.
%
% %%%%%%%%%%%%%%%%%%%%%%%%%%%%%%%%%%%%%%
% \paragraph{Main File.}
%
% The main file is called |cdocsamp.tex|.
%
% Load the \textsf{childdoc} definitions and
% declare the filename for the main document:
%    \begin{macrocode}
\input{childdoc.def}
\childdocmain{}
%    \end{macrocode}

% Optional override for |\version| flag:
%    \begin{macrocode}
%%\ifchilddoc\else\providecommand{\version}{draft}\fi
%    \end{macrocode}

% Define the default values for the |\version| flag
% (|final| for the main file and |draft| for childs):
%    \begin{macrocode}
\ifchilddoc
\providecommand{\version}{draft}
\else
\providecommand{\version}{final}
\fi
%    \end{macrocode}

% Load the standard document class:
%    \begin{macrocode}
\documentclass[12pt]{article}
%    \end{macrocode}

% Start the document body:
%    \begin{macrocode}
\begin{document}
%    \end{macrocode}

% Declare a title page.
% Print title, part of document being processed and version flag:
%    \begin{macrocode}
\addtocounter{page}{-1}
\begin{center}
{\LARGE\bfseries{}childdoc example\par}
\vspace{1cm}
\ifchilddoc
\ifchilddocmanual part\else chapter\fi:
`\childdocname' of `\childdocjob'\par
\else
main document: `\childdocjob'\par
\fi
version: \version\par
\end{center}
\newpage
%    \end{macrocode}

% Manually include selected file,
% otherwise process as usual:
%    \begin{macrocode}
\ifchilddocmanual
\section*{part `\childdocname'}
\input{\childdocname}
\else
%    \end{macrocode}

% Include the two chapters:
%    \begin{macrocode}
\include{cdocsch1}
\include{cdocsch2}
%    \end{macrocode}

% Include the two parts unless only chapters should be displayed:
%    \begin{macrocode}
\ifchilddoc\else
\section{part three}
\input{cdocspt3}
\section{part four}
\input{cdocspt4}
\fi
%    \end{macrocode}

% Process as usual until here:
%    \begin{macrocode}
\fi
%    \end{macrocode}

% End of document body:
%    \begin{macrocode}
\end{document}
%    \end{macrocode}
%\iffalse
%</samplemain>
%\fi
%
% %%%%%%%%%%%%%%%%%%%%%%%%%%%%%%%%%%%%%%
% \paragraph{Chapter Include Files.}
%
% The include files are called |cdocsch1.tex| and |cdocsch2.tex|.
%
%\iffalse
%<*samplechap1|samplechap2>
%\fi

% Optional override for |\version| flag:
%    \begin{macrocode}
%%\providecommand{\version}{final}
%    \end{macrocode}

% Include the main document:
%    \begin{macrocode}
\input{childdoc.def}
\childdocof{cdocsamp}
%    \end{macrocode}

%\iffalse
%</samplechap1|samplechap2>
%\fi
%
%\iffalse
%<*samplechap1>
%\fi
% Some text for chapter 1:
%    \begin{macrocode}
\section{one}
some text in chapter one
%    \end{macrocode}

%\iffalse
%</samplechap1>
%\fi
% Some text for chapter 2:
%\iffalse
%<*samplechap2>
%\fi
%    \begin{macrocode}
\section{two}
more text in chapter two
%    \end{macrocode}

%\iffalse
%</samplechap2>
%\fi
%
% %%%%%%%%%%%%%%%%%%%%%%%%%%%%%%%%%%%%%%
% \paragraph{Part Include Files.}
%
% The include files are called |cdocspt3.tex| and |cdocspt4.tex|.
%
%\iffalse
%<*samplepart3|samplepart4>
%\fi

% Optional override for |\version| flag:
%    \begin{macrocode}
%%\providecommand{\version}{final}
%    \end{macrocode}

% Include the main document:
%    \begin{macrocode}
\input{childdoc.def}
\childdocby{cdocsamp}
%    \end{macrocode}

%\iffalse
%</samplepart3|samplepart4>
%\fi
%
%\iffalse
%<*samplepart3>
%\fi
% Some text for part 3:
%    \begin{macrocode}
some text in part three
%    \end{macrocode}

%\iffalse
%</samplepart3>
%\fi
% Some text for part 4:
%\iffalse
%<*samplepart4>
%\fi
%    \begin{macrocode}
more text in part four
%    \end{macrocode}

%\iffalse
%</samplepart4>
%\fi
%
% %%%%%%%%%%%%%%%%%%%%%%%%%%%%%%%%%%%%%%
% \paragraph{Forwarding for a Complete Draft.}
%
% The following forwarding file |cdocsdrf.tex|
% compiles the main document in draft mode:
%\iffalse
%<*sampledraft>
%\fi
%    \begin{macrocode}
\def\version{draft}
\input{childdoc.def}
\childdocforward{cdocsamp}
%    \end{macrocode}

%\iffalse
%</sampledraft>
%\fi
%
% %%%%%%%%%%%%%%%%%%%%%%%%%%%%%%%%%%%%%%
% \paragraph{Forwarding for Final Version of the Chapters.}
%
% The following forwarding files |cdocsfn1.tex| and |cdocsfn2.tex|
% (with identical content)
% compile the final versions of the child documents
% |cdocsch1.tex| and |cdocsch2.tex|, respectively:
%\iffalse
%<*samplefinal>
%\fi
%    \begin{macrocode}
\def\version{final}
\input{childdoc.def}
\childdocforwardprefix[cdocsamp]{cdocsfn}{cdocsch}
%    \end{macrocode}

%\iffalse
%</samplefinal>
%\fi
%
% %%%%%%%%%%%%%%%%%%%%%%%%%%%%%%%%%%%%%%
% \paragraph{Command Line Processing.}
%
% The following three command lines generate the output files
% |cdocscld|, |cdocscl1| and |cdocscl2|
% which should be identical to
% |cdocsdrf|, |cdocsch1| and |cdocsfn2|, respectively:
% \begin{center}
% \begin{tabular}{l}
% |latex -jobname cdocscld \|\\
% |  "\def\version{draft}\input{childdoc.def}\childdocforward{cdocsamp}"|\\
% |latex -jobname cdocscl1 \|\\
% |  "\input{childdoc.def}\childdocforward[cdocsamp]{cdocsch1}"|\\
% |latex -jobname cdocscl2 \|\\
% |  "\def\version{final}\input{childdoc.def}\childdocforward{cdocsch2}"|
% \end{tabular}
% \end{center}
% Note that the trailing backslash on each first line
% merely continues the input to the second line
% (for convenient cut ant paste).
% Furthermore, the command |latex| can be replaced by any
% of its alternative versions such as |pdflatex|.
%
% %%%%%%%%%%%%%%%%%%%%%%%%%%%%%%%%%%%%%%%%%%%%%%%%%%%%%%%%%%%%%%%%%%%%%%%%%%%%%%
% %%%%%%%%%%%%%%%%%%%%%%%%%%%%%%%%%%%%%%%%%%%%%%%%%%%%%%%%%%%%%%%%%%%%%%%%%%%%%%
% \section{Implementation}
%\iffalse
%<*package>
%\fi
%
% This section describes the definitions file |childdoc.def|.

% The definitions cannot be loaded using |\usepackage| or |\RequirePackage|
% which has a mechanism to prevent loading a style file more than once.
% When loading the definitions by means of |\input|
% multiple instances have to be prevented manually:
%\iffalse
%This code needs to be before the `\ProvidesFile' directive
%which is defined at the beginning of this file.
%Therefore it is also placed there and commented out here.
%</package>
%<*discard>
%\fi
%    \begin{macrocode}
\ifdefined\childdocmain\endinput\fi
%    \end{macrocode}
%\iffalse
%</discard>
%<*package>
%\fi
%
% \macro{\ifchilddoc}
% \macro{\ifchilddocmanual}
% The conditional |\ifchilddoc| tells whether a
% child (true) or main (false) document is being compiled.
% The conditional |\ifchilddocmanual| tells whether
% the |\includeonly| mechanism is used (false) or
% the selection of child files must be performed manually (true).
% The definitions initialise to false:
%    \begin{macrocode}
\newif\ifchilddoc
\newif\ifchilddocmanual
%    \end{macrocode}

% \macro{\childdocname}
% \macro{\childdocjob}
% The macro |\childdocname| stores the name of the main document
% to be compiled. The macro |\childdocjob| stores the name of
% the document on which the \LaTeX{} compiler was originally invoked.
% The content of |\jobname| cannot be compared
% to filenames specified in the source due to different catcodes.
% The following code rescans |\jobname|, stores the result
% in |\childdocname| and saves a copy in |\childdocjob|:
%    \begin{macrocode}
\edef\childdocname{\scantokens\expandafter{\jobname\noexpand}}
\let\childdocjob\childdocname
%    \end{macrocode}

% \macro{\childdocdisable}
% The macro |\childdocdisable| prevents the main file
% from being processed more than once.
% At this stage, the main document command |\childdocmain|
% is assumed to be called once again where it should do nothing.
% Any subsequent call to it should prevent
% a secondary processing of the main document
% It overwrites the forwarding commands
% |\childdocof| and |\childdocforward|
% with empty macros to prevent further inclusions of the main document:
%    \begin{macrocode}
\newcommand{\childdocdisable}
{
  \renewcommand{\childdocmain}[1]{\renewcommand{\childdocmain}[1]{\endinput}}
  \renewcommand{\childdocof}[1]{}
  \renewcommand{\childdocby}[2][]{}
  \renewcommand{\childdocforward}[2][]{}
  \renewcommand{\childdocdisable}{}
}
%    \end{macrocode}

% \macro{\childdocmain}
% The macro |\childdocmain| is to be called at the top of the main file
% with nothing or the main filename (without extension) as argument.
% First, it breaks loops.
% If the argument is not empty and does not match |\childdocname|
% (which is set by the first inclusion of |childdoc.def|),
% |\ifchilddoc| is set to true, |\includeonly| is applied to the child file
% and |\jobname| is set to the main file
% (for proper handling of |.aux| files):
%    \begin{macrocode}
\newcommand{\childdocmain}[1]
{
  \childdocdisable\childdocmain{}
  \if?#1?\else
    \begingroup
      \def\childdoctmp{#1}
      \ifx\childdoctmp\childdocname
        \def\childdoctmp{}
      \else
        \def\childdoctmp
        {
          \childdoctrue
          \includeonly{\childdocname}
          \def\childdocjob{#1}
          \def\jobname{#1}
        }
      \fi
      \expandafter
    \endgroup
    \childdoctmp
  \fi
}
%    \end{macrocode}

% \macro{\childdocof}
% The command |\childdocof| redirects
% compilation to the main file |#1|.
%    \begin{macrocode}
\newcommand{\childdocof}[1]
{
  \childdocdisable
  \childdoctrue
  \includeonly{\childdocname}
  \def\jobname{#1}
  \def\childdocjob{#1}
  \input{#1}
}
%    \end{macrocode}

% \macro{\childdocby}
% The command |\childdocby| ....
%    \begin{macrocode}
\newcommand{\childdocby}[2][]
{
  \childdocdisable
  \childdoctrue
  \childdocmanualtrue
  \if?#1?\else
    \def\jobname{#2}
  \fi
  \def\childdocjob{#2}
  \input{#2}
  \endinput
}
%    \end{macrocode}

% \macro{\childdocforward}
% The command |\childdocforward| redirects
% compilation to the main file or
% (if the optional argument is given) a child file.
% Parameters are set as if the main file
% or a child file starting with |\childdocof| was compiled.
% Then compilation is handed over to the main file:
%    \begin{macrocode}
\newcommand{\childdocforward}[2][]
{
  \begingroup
    \if?#1?
      \def\childdoctmp
      {
        \def\childdocname{#2}
        \def\childdocjob{#2}
        \def\jobname{#2}
        \input{#2}
        \endinput
      }
    \else
      \def\childdoctmp
      {
        \childdocdisable
        \def\childdocname{#2}
        \childdoctrue
        \includeonly{#2}
        \def\childdocjob{#1}
        \def\jobname{#1}
        \input{#1}
        \endinput
      }
    \fi
    \expandafter
  \endgroup
  \childdoctmp
}
%    \end{macrocode}

% \macro{\childdocforwardprefix}
% The command |\childdocforwardprefix| redirects
% compilation to the main or a child file by means of a pattern.
% The prefix |#1| in the current filename is replaced by |#2|
% and the suffix of the current filename is kept
% (it is assumed that the filename does not contain the substring `|~~~|'
% which is used as a delimiter).
% Compilation is handed over to the new file by |\childdocforward|:
%    \begin{macrocode}
\newcommand{\childdocforwardprefix}[3][]
{
  \begingroup
    \def\childdocextract #2##1~~~{\def\childdoctmp{\childdocforward[#1]{#3##1}}}
    \expandafter\childdocextract\childdocname~~~
    \expandafter
  \endgroup
  \childdoctmp
}
%    \end{macrocode}

% \macro{\childdoc}
% The deprecated macro |\childdoc| is a legacy version of |\childdocmain|:
%    \begin{macrocode}
\newcommand{\childdoc}{\childdocmain}
%    \end{macrocode}

% \macro{\childdocredirect}
% The deprecated macro |\childdocredirect| is a legacy version
% of |\childdocforward| and |\childdocforwardprefix|:
%    \begin{macrocode}
\newcommand{\childdocredirect}[2][]
{
  \begingroup
    \if?#1?
      \def\childdoctmp{\childdocforward{#2}}
    \else
      \def\childdoctmp{\childdocforwardprefix{#1}{#2}}
    \fi
    \expandafter
  \endgroup
  \childdoctmp
}
%    \end{macrocode}

%\iffalse
%</package>
%\fi
%
\endinput

\childdocof{cdocsamp}
%    \end{macrocode}

%\iffalse
%</samplechap1|samplechap2>
%\fi
%
%\iffalse
%<*samplechap1>
%\fi
% Some text for chapter 1:
%    \begin{macrocode}
\section{one}
some text in chapter one
%    \end{macrocode}

%\iffalse
%</samplechap1>
%\fi
% Some text for chapter 2:
%\iffalse
%<*samplechap2>
%\fi
%    \begin{macrocode}
\section{two}
more text in chapter two
%    \end{macrocode}

%\iffalse
%</samplechap2>
%\fi
%
% %%%%%%%%%%%%%%%%%%%%%%%%%%%%%%%%%%%%%%
% \paragraph{Part Include Files.}
%
% The include files are called |cdocspt3.tex| and |cdocspt4.tex|.
%
%\iffalse
%<*samplepart3|samplepart4>
%\fi

% Optional override for |\version| flag:
%    \begin{macrocode}
%%\providecommand{\version}{final}
%    \end{macrocode}

% Include the main document:
%    \begin{macrocode}
% \iffalse
%
% childdoc.dtx Copyright (C) 2017-2018 Niklas Beisert
%
% This work may be distributed and/or modified under the
% conditions of the LaTeX Project Public License, either version 1.3
% of this license or (at your option) any later version.
% The latest version of this license is in
%   http://www.latex-project.org/lppl.txt
% and version 1.3 or later is part of all distributions of LaTeX
% version 2005/12/01 or later.
%
% This work has the LPPL maintenance status `maintained'.
%
% The Current Maintainer of this work is Niklas Beisert.
%
% This work consists of the files childdoc.dtx and childdoc.ins
% and the derived files childdoc.def and cdocsamp.tex with
% cdocsch1.tex, cdocsch2.tex, cdocsdrf.tex, cdocsfn1.tex, cdocsfn2.tex.
%
%<package>\ifdefined\childdocmain\endinput\fi
%<package>\ProvidesFile{childdoc.def}[2018/12/30 v2.0 child document driver]
%<samplemain>\ProvidesFile{cdocsamp.tex}[2018/12/30 v2.0 sample for childdoc]
%<*driver>
%\ProvidesFile{childdoc.drv}[2018/12/30 v2.0 childdoc reference manual file]
\PassOptionsToClass{10pt,a4paper}{article}
\documentclass{ltxdoc}

\usepackage[margin=35mm]{geometry}
\usepackage{hyperref}
\usepackage{hyperxmp}
\usepackage[usenames]{color}

\hypersetup{colorlinks=true}
\hypersetup{pdfstartview=FitH}
\hypersetup{pdfpagemode=UseNone}
\hypersetup{pdfsource={}}
\hypersetup{pdflang={en-UK}}
\hypersetup{pdfcopyright={Copyright 2017-2018 Niklas Beisert.
  This work may be distributed and/or modified under the
  conditions of the LaTeX Project Public License, either version 1.3
  of this license or (at your option) any later version.}}
\hypersetup{pdflicenseurl={http://www.latex-project.org/lppl.txt}}
\hypersetup{pdfcontactaddress={ETH Zurich, ITP, HIT K,
  Wolfgang-Pauli-Strasse 27}}
\hypersetup{pdfcontactpostcode={8093}}
\hypersetup{pdfcontactcity={Zurich}}
\hypersetup{pdfcontactcountry={Switzerland}}
\hypersetup{pdfcontactemail={nbeisert@itp.phys.ethz.ch}}
\hypersetup{pdfcontacturl={http://people.phys.ethz.ch/\xmptilde nbeisert/}}

\newcommand{\secref}[1]{\hyperref[#1]{section \ref*{#1}}}

\parskip1ex
\parindent0pt
\let\olditemize\itemize
\def\itemize{\olditemize\parskip0pt}

\begin{document}

\title{The \textsf{childdoc} Package}
\hypersetup{pdftitle={The childdoc Package}}
\author{Niklas Beisert\\[2ex]
  Institut f\"ur Theoretische Physik\\
  Eidgen\"ossische Technische Hochschule Z\"urich\\
  Wolfgang-Pauli-Strasse 27, 8093 Z\"urich, Switzerland\\[1ex]
  \href{mailto:nbeisert@itp.phys.ethz.ch}
  {\texttt{nbeisert@itp.phys.ethz.ch}}}
\hypersetup{pdfauthor={Niklas Beisert}}
\hypersetup{pdfsubject={Manual for the LaTeX2e Package childdoc}}
\date{30 December 2018, \textsf{v2.0}}
\maketitle

\begin{abstract}\noindent
\textsf{childdoc} is a \LaTeXe{} package
that enables the direct compilation
of document sections included by |\include|
to individual files.
\end{abstract}

\begingroup
\parskip0ex
\tableofcontents
\endgroup

%%%%%%%%%%%%%%%%%%%%%%%%%%%%%%%%%%%%%%%%%%%%%%%%%%%%%%%%%%%%%%%%%%%%%%%%%%%%%%%%
%%%%%%%%%%%%%%%%%%%%%%%%%%%%%%%%%%%%%%%%%%%%%%%%%%%%%%%%%%%%%%%%%%%%%%%%%%%%%%%%
\section{Introduction}

\LaTeX{} provides a mechanism to structure a large document (such as a book)
into a main file and several child files (containing the chapters)
using the |\include| command.
This mechanism is beneficial for documents
which span hundreds of pages in order to
make the source file(s) more manageable.
Moreover, compilation can be restricted to
selected child files by means of the |\includeonly| command.
The latter feature can be used to reduce the compilation time while editing
(this was significantly more useful in the earlier days of \LaTeX{})
or to generate a smaller document which is easier to navigate.
Another application of |\includeonly| is to generate
documents consisting of selected parts of the complete document.

However, there are a few drawbacks of the plain |\include| mechanism:
\begin{itemize}
\item
The child files cannot be compiled on their own,
they can only be compiled via the main file.
A naive editing environment
(such as a text editor with an option
to have the current file processed by \LaTeX)
may require one to switch to the main file before compiling;
attempting to compile the child file produces errors.
\item
The main file must be modified (each time)
to adjust the |\includeonly| command
to the present needs. This easily leaves the main file in a messy state.
\item
The generated document will always carry the filename
of the main document. This is inconvenient if
several child files are to be compiled and
to be kept for distribution.
\end{itemize}

The present package provides a simple interface
to make child files individually compilable by \LaTeX{}.
Compiling a child file then has the same effect as compiling
the main file with an |\includeonly| command
to select the appropriate child.
Moreover the generated document will carry the name of the child
rather than the main file.
This resolves all three above issues.

This feature is meant to make the editing of books,
thesis documents and lecture notes somewhat more convenient.
However, the package can also be used efficiently for
composing a series of documents (such as exercise sheets)
which are typically distributed individually.
It then assists the author in generating the individual documents
(potentially in different versions)
as well as a document containing the collected series.
Another application is in developing style files
or other kinds of included material
where compilation of the style file could redirect
to a sample or test file.

%%%%%%%%%%%%%%%%%%%%%%%%%%%%%%%%%%%%%%%%%%%%%%%%%%%%%%%%%%%%%%%%%%%%%%%%%%%%%%%%
%%%%%%%%%%%%%%%%%%%%%%%%%%%%%%%%%%%%%%%%%%%%%%%%%%%%%%%%%%%%%%%%%%%%%%%%%%%%%%%%
\section{Usage}

First of all, the package \textsf{childdoc} is \emph{not} a standard
\LaTeXe{} |.sty| style file! Therefore it needs to be invoked in
a non-standard way.

%%%%%%%%%%%%%%%%%%%%%%%%%%%%%%%%%%%%%%%%%%%%%%%%%%%%%%%%%%%%%%%%%%%%%%%%%%%%%%%%
\subsection{Included Files}
\label{sec:include}

%%%%%%%%%%%%%%%%%%%%%%%%%%%%%%%%%%%%%%%%
\DescribeMacro{\childdocmain}
To use the package, add the commands
\begin{center}
\begin{tabular}{l}
|\input{childdoc.def}|\\
|\childdocmain{}|\\
\end{tabular}
\end{center}
at the very top of the main \LaTeX{} file,
in particular \emph{before} the |\documentclass| statement!
The argument of |\childdocmain| should be left empty
(but it must be present).

%%%%%%%%%%%%%%%%%%%%%%%%%%%%%%%%%%%%%%%%
\DescribeMacro{\childdocof}
Furthermore, add the commands
\begin{center}
\begin{tabular}{l}
|\input{childdoc.def}|\\
|\childdocof{|\textit{main}|}|\\
\end{tabular}
\end{center}
at the top of every child file \textit{child}
which is included by |\include{|\textit{child}|}|
from within the main file
(or at least for those files to be compiled individually).
The argument \textit{main} must be the filename of the main file.

There are a couple of
considerations in setting up the main and child documents:

%%%%%%%%%%%%%%%%%%%%%%%%%%%%%%%%%%%%%%%%
\paragraph{Restrictions.}

Please note the following restrictions:
\begin{itemize}
\item
|\childdocmain| must be called with one argument \textit{main}
to ensure compatibility with earlier version of the package.
It must either be empty (|\childdocmain{}|)
or precisely match the filename of the main file in which it is specified.
See \secref{sec:detection} for further information.
\item
The filename \textit{main} must be specified without the |.tex| extension.
\item
The filename \textit{main} is case sensitive
(even in case-insensitive file systems)
due to internal string comparison.
\item
The argument \textit{main} should be fully expanded, it cannot be a macro.
\item
Subdirectories and special characters should be avoided in filenames.
\item
The command |\childdocmain{|\textit{main}|}| must be followed by a whitespace.
It should not be followed immediately by another command
or by a comment mark `|%|'.
This is because the \TeX{} parser reads the token immediately following
the argument of |\childdocmain| and puts it
at the beginning of every child section;
however, a white\-space is ignored.
\end{itemize}

%%%%%%%%%%%%%%%%%%%%%%%%%%%%%%%%%%%%%%%%
\paragraph{Content of Main File.}

It is advisable to place all content in the child files included by |\include|.
Any output contained in the main file will appear in all child documents
unless suppressed manually;
it cannot be suppressed automatically by the |\includeonly| directive
and thus should normally be avoided.
A method to include some content in the main file
by means of conditional processing is described in \secref{sec:conditional}.

%%%%%%%%%%%%%%%%%%%%%%%%%%%%%%%%%%%%%%%%
\paragraph{Page Numbering.}

When only a part of the document is compiled,
the appropriate numbering of pages
(as well as other status parameters)
is determined from the |.aux| files.
The latter contain information from previous passes.
However this information needs to propagate through
all intermediate child documents.
Therefore the page numbering in child documents may well
be inconsistent until the complete document is compiled at least once.

A useful (if unconventional) way to always ensure a consistent
page numbering is to restart the numbering in each child document
and denote the pages by `\textit{child}|.|\textit{page}'
where \textit{child} represents the chapter/section number of the child file.
This can be achieved by the command
|\numberwithin{page}{|\textit{child}|}|
of the \textsf{amsmath} package
where \textit{child} can be |chapter| or |section|
depending on the chosen structuring.
Alternatively, one can modify the macro |\thepage| appropriately
and reset the counter |page| at the start of each child file.

%%%%%%%%%%%%%%%%%%%%%%%%%%%%%%%%%%%%%%%%%%%%%%%%%%%%%%%%%%%%%%%%%%%%%%%%%%%%%%%%
\subsection{Conditional Processing}
\label{sec:conditional}

The package provides a mechanism to compile different versions
of a document. To customise the versions further some conditional processing
can come in handy to distinguish which version is being compiled.
The package provides two macros to describe the compilation context:

%%%%%%%%%%%%%%%%%%%%%%%%%%%%%%%%%%%%%%%%
\DescribeMacro{\ifchilddoc}
The conditional |\ifchilddoc| distinguishes between the compilation of
child documents and the main document:
%
\begin{center}
|\ifchilddoc |\textit{child-code}| |[|\||else |\textit{main-code}]| \||fi|
\end{center}

%%%%%%%%%%%%%%%%%%%%%%%%%%%%%%%%%%%%%%%%
\DescribeMacro{\childdocname}
\DescribeMacro{\childdocjob}
The macro |\childdocname| contains the filename (without extension)
of the main or child file being processed.
Note that |\childdocjob| will always contain the name of the main file.

%%%%%%%%%%%%%%%%%%%%%%%%%%%%%%%%%%%%%%%%
\paragraph{Title Page.}

Conditional processing can be used to include a title or banner page
in the main document when proper precautions are taken.
Importantly, the code in the main file should ensure that the page counter
(as well as other status parameters which are stored in the |.aux| files)
takes the same value after the conditional processing.
Otherwise the page numbers may take divergent values
depending on which part is compiled.

For example, a title page could be declared by:
%
\begin{center}
\begin{tabular}{l}
|\ifchilddoc\||else|\\
|\addtocounter{page}{-1}|\\
\textit{code for title page}\\
|\newpage|\\
|\||fi|
\end{tabular}
\end{center}
%
A banner page for the child documents can be generated by:
%
\begin{center}
\begin{tabular}{l}
|\ifchilddoc|\\
|\addtocounter{page}{-1}|\\
\textit{code for banner page}\\
|\newpage|\\
|\||fi|
\end{tabular}
\end{center}
%
Here one could write a message such as:
\begin{center}
|This is the part \childdocname{} of \childdocjob{}.|
\end{center}

%%%%%%%%%%%%%%%%%%%%%%%%%%%%%%%%%%%%%%%%%%%%%%%%%%%%%%%%%%%%%%%%%%%%%%%%%%%%%%%%
\subsection{Flags}
\label{sec:flags}

The package makes it easy to generate different versions
of the main or child documents.
To this end compilation flags can be defined
and assigned different default values.
They will be particularly useful in conjunction
with the forwarding mechanism described in \secref{sec:forward}.

For example, it may be useful to have a flag |\version|
which can be set to |draft| or |final|.
The document source will contain some conditional code
depending on the value of |\version|.
Suppose further, the flag should default to |final| for the main file
and to |draft| for child files
which is a natural assignment for editing the document.
This is achieved by placing the following code
in the preamble of the main document
(below the |\childdocmain| directive):
%
\begin{center}
\begin{tabular}{l}
|\ifchilddoc|\\
|\providecommand{\version}{draft}|\\
|\||else|\\
|\providecommand{\version}{final}|\\
|\||fi|
\end{tabular}
\end{center}
%
The definition by |\providecommand| makes sure
that previous definitions are not overwritten.
Further statements |\providecommand{\version}{...}|
can thus be added before the above code to override it.

For the main file, one might add a line
(between |\childdocmain| and the above block)
%
\begin{center}
|%\ifchilddoc\||else\providecommand{\version}{draft}\||fi|
\end{center}
%
which can be uncommented to produce a draft version.
Likewise one can add a line to the very top of a child file
(above the |\childdocof{|\textit{main}|}| directive)
%
\begin{center}
|%\providecommand{\version}{final}|
\end{center}
%
which can be uncommented to produce the final version of this child document.

%%%%%%%%%%%%%%%%%%%%%%%%%%%%%%%%%%%%%%%%%%%%%%%%%%%%%%%%%%%%%%%%%%%%%%%%%%%%%%%%
\subsection{Forwarding}
\label{sec:forward}

Different versions of the main or child documents
using compilation flags as described in \secref{sec:flags}
can be (permanently) stored in different files
for convenient compilation, viewing and distribution.
To this end, the package defines a command
to pass on compilation to a different file:

%%%%%%%%%%%%%%%%%%%%%%%%%%%%%%%%%%%%%%%%
\DescribeMacro{\childdocforward}
The command |\childdocforward| redirects processing to
another source file:
%
\begin{center}
\begin{tabular}{l}
|\input{childdoc.def}|\\
|\childdocforward[|\textit{main}|]{|\textit{dest}|}|\\
\end{tabular}
\end{center}
%
The argument \textit{dest} is the destination file
(without extension).
It should be the main file or one of the child files.
Note that further \textsf{childdoc} directives
such as |\childdocof| and |\childdocforward|
in the indicated file will be processed in this form.
The optional argument \textit{main}
passes on directly to the main file \textit{main}
while pretending to compile the child \textit{dest}.
This form behaves as if \textit{dest}
issues |\childdocof{|\textit{main}|}| right away,
and no further \textsf{childdoc} directives will be processed.

%%%%%%%%%%%%%%%%%%%%%%%%%%%%%%%%%%%%%%%%
\DescribeMacro{\...prefix}
In the alternative form |\childdocforwardprefix|,
%
\begin{center}
\begin{tabular}{l}
|\input{childdoc.def}|\\
|\childdocforwardprefix[|\textit{main}|]{|\textit{prefix}|}{|\textit{dest}|}|
\end{tabular}
\end{center}
%
the destination file is determined by a pattern
depending on the current file:
To make this work, the current file must be called
`{\textit{prefix}\hspace{0.2em}\textit{suffix}}'
with \textit{prefix} matching precisely the argument.
Processing is then passed on to the file
`{\textit{dest}\hspace{0.2em}\textit{suffix}}'.
Surely, the same effect is achieved by
directly specifying the
argument `{\textit{dest}\hspace{0.2em}\textit{suffix}}'
in the first form.
However, that requires to set up a different file
for each child. With the alternative form of the command
all these files can have exactly the same content
which simplifies setting them up and maintaining them.

For example, the following file |draft.tex|
with a compilation flag |\version| as described in \secref{sec:flags}
compiles the main document as a draft:
%
\begin{center}
\begin{tabular}{l}
|\def\version{draft}|\\
|\input{childdoc.def}|\\
|\childdocforward{|\textit{main}|}|
\end{tabular}
\end{center}
%
Likewise, the following files |final|\textit{nn}|.tex|
compile the final version of the child document
|child|\textit{nn}|.tex|:
%
\begin{center}
\begin{tabular}{l}
|\def\version{final}|\\
|\input{childdoc.def}|\\
|\childdocforwardprefix{final}{child}|
\end{tabular}
\end{center}
%

Note that when several versions of a main file and/or of each child file
are to be generated, it may be convenient to set up a |Makefile| or
shell script to automatise the process.

%%%%%%%%%%%%%%%%%%%%%%%%%%%%%%%%%%%%%%%%%%%%%%%%%%%%%%%%%%%%%%%%%%%%%%%%%%%%%%%%
\subsection{Command Line Processing}
\label{sec:commandline}

The effect of redirection files can also be achieved by invoking
the \LaTeX{} compiler with a more elaborate command line.
Most conveniently this should be done as part
of a shell script or a |Makefile|.

When using \textsf{childdoc} in the main file, the following
command lines effectively perform a redirection
(note that depending on the shell being used,
backslashes may have to be doubled: `|\|' $\to$ `|\\|'):
%
\begin{center}
|... -jobname "|\textit{target}|" |\\|"|[\textit{flags}]%
|\input{childdoc.def}\childdocforward[|\textit{main}|]{|\textit{dest}|}"|
\end{center}
%
Here \textit{target} is the name of the output file,
\textit{main} is the name of the main file
and \textit{dest} is the name of the main or child file to be processed
(all filenames without extensions).
The optional argument \textit{main} can be omitted
if \textit{main} matches \textit{dest}.
Optionally, compilation \textit{flags} can be defined via |\def| commands.
This command line makes the \TeX{} engine believe
it is compiling the file \textit{target}
whose content is specified as the latter parameter.
The provided code then forwards the processing to
\textit{main} or \textit{dest} as described in \secref{sec:forward}.

%%%%%%%%%%%%%%%%%%%%%%%%%%%%%%%%%%%%%%%%%%%%%%%%%%%%%%%%%%%%%%%%%%%%%%%%%%%%%%%%
\subsection{Include by Input}
\label{sec:input}

Including child documents by |\include| has some restrictions by design.
Most notably, the content of a child document always occupies
its own set of pages; pages cannot be shared between child documents.
Usually, this behaviour makes perfect sense
because each child document contain an essential part of the document.
However, in some situations it may be desirable to compose
a document from a collection of parts
without having mandatory page breaks between then.
For this case, the package
provides a mechanism to include parts
by |\input| which can also be processed individually.
However, by construction this mechanism
requires manual handling of the content to be output.

%%%%%%%%%%%%%%%%%%%%%%%%%%%%%%%%%%%%%%%%
\DescribeMacro{\ifchilddocmanual}
The main file should be prepared as usual, see \secref{sec:include}.
However, the document body must make a distinction
between processing of an individual part and of the main document, e.g.:
%
\begin{center}
\begin{tabular}{l}
|\ifchilddocmanual|\\
|\input{\childdocname}|\\
|\||else|\\
\textit{document body with }|\input{|\textit{part}|}|\\
|\||fi|
\end{tabular}
\end{center}
%
The conditional |\ifchilddocmanual| is true whenever
a part to be included by |\input| is being compiled,
and the name of the part is stored in |\childdocname|.

%%%%%%%%%%%%%%%%%%%%%%%%%%%%%%%%%%%%%%%%
\DescribeMacro{\childdocby}
Each part to be included by |\input| should start with:
%
\begin{center}
\begin{tabular}{l}
|\input{childdoc.def}|\\
|\childdocby{|\textit{main}|}|\\
\end{tabular}
\end{center}
%
The directive |\childdocby| is similar to |\childdocof|
described in \secref{sec:include},
but the subsequent selection of content must be done manually.
To that end, both |\ifchilddoc| and |\ifchilddocmanual|
will be true upon processing of a part,
and the name of the part is stored in |\childdocname|.
Note that |\jobname| will be set to the filename of the current part
so that each part receives an individual |.aux| file
that does not interfere with the |.aux| file(s) of the main document.
This behaviour can be altered by the alternative form
|\childdocby[*]{|\textit{main}|}| (with a non-empty optional argument)
which uses the |.aux| file of the main document
by setting |\jobname| to \textit{main}.

%%%%%%%%%%%%%%%%%%%%%%%%%%%%%%%%%%%%%%%%%%%%%%%%%%%%%%%%%%%%%%%%%%%%%%%%%%%%%%%%
\subsection{Driver Development}
\label{sec:driver}

The \textsf{childdoc} mechanism can also be use for the development
of definition files such as \LaTeX{} styles or classes.
This case differs from the above setup with multiple parts
included by |\include| in that no |\includeonly| should be invoked.
This can be achieved by starting the include file
(before |\ProvidesPackage|) with:
%
\begin{center}
\begin{tabular}{l}
|\input{childdoc.def}|\\
|\childdocforward{|\textit{main}|}|\\
\end{tabular}
\end{center}
%
or alternatively with:
%
\begin{center}
\begin{tabular}{l}
|\input{childdoc.def}|\\
|\childdocby{|\textit{main}|}|\\
\end{tabular}
\end{center}
%
Both forms have slightly different effects as described above.
The main file is prepared as usual, see \secref{sec:include}.

%%%%%%%%%%%%%%%%%%%%%%%%%%%%%%%%%%%%%%%%%%%%%%%%%%%%%%%%%%%%%%%%%%%%%%%%%%%%%%%%
\subsection{Legacy Detection}
\label{sec:detection}

The directive |\childdocmain| in the main file can detect
whether the complete document or merely a child is to be compiled
even without using the directive |\childdocof|.
This method is deprecated because it is less robust
and there is no compelling reason to use it;
it is merely provided for backward compatibility
and it may be removed in future versions.

If the detection mechanism is to be used,
it is mandatory to correctly specify
the filename of the main file as the argument of |\childdocmain|:
%
\begin{center}
\begin{tabular}{l}
|\input{childdoc.def}|\\
|\childdocmain{|\textit{main}|}|\\
\end{tabular}
\end{center}
%
If |\jobname| does not match the argument \textit{main} of |\childdocmain|,
it is assumed that |\jobname| points to the child file to be compiled.
When using |\childdocmain| with the main file specified as argument,
it suffices to start a child file
with just |\input{|\textit{main}|}|
without loading of the package and using |\childdocof|.
If instead all processing is done
with the appropriate \textsf{childdoc} directives,
the argument of \textit{main} of |\childdocmain| can be empty.

An alternative version of the command line processing described
in \secref{sec:commandline} using the detection mechanism reads:
%
\begin{center}
|... -jobname "|\textit{target}|" "|[\textit{flags}]%
[|\def\jobname{|\textit{dest}|}|]|\input{|\textit{main}|}"|
\end{center}

%%%%%%%%%%%%%%%%%%%%%%%%%%%%%%%%%%%%%%%%%%%%%%%%%%%%%%%%%%%%%%%%%%%%%%%%%%%%%%%%
\subsection{Manual Code}
\label{sec:manual}

In case one cannot be certain whether the definitions file |childdoc.def|
is installed on the target \TeX{} distribution
and one prefers not to ship it,
it is conceivable to paste a few relevant commands into the sources.

To that end, drop all statements |\input{childdoc.def}|
and perform the replacements as outlined below.
Instead of |\childdocmain{|\textit{main}|}| add the following code
to the top of the main file:
%
\begin{center}
\begin{tabular}{l}
|\||ifdefined\childdocname\endinput\||fi\newif\ifchilddoc|\\
|\edef\childdocname{\scantokens\expandafter{\jobname\noexpand}}|\\
|\def\childdocmain{|\textit{main}|}\||ifx\childdocmain\childdocname\||else|\\
|\childdoctrue\includeonly{\childdocname}\let\jobname\childdocmain\||fi|\\
\end{tabular}
\end{center}
%
Instead of |\childdocof{|\textit{main}|}| just include the main file
at the top of each child file:
%
\begin{center}
|\input{|\textit{main}|}|
\end{center}
%
A simple redirection |\childdocforward{|\textit{dest}|}| is achieved by:
%
\begin{center}
|\def\jobname{|\textit{dest}|}\input{\jobname}|
\end{center}
%
The redirection with prefix
|\childdocforwardprefix[|\textit{prefix}|]{|\textit{dest}|}|
is accomplished by:
%
\begin{center}
\begin{tabular}{l}
|{\edef\jobname{\scantokens\expandafter{\jobname\noexpand}}|\\
|\def\redirectjob |\textit{prefix}|#1~~~{\gdef\jobname{|\textit{dest}|#1}}|\\
|\expandafter\redirectjob\jobname~~~}\input{\jobname}|
\end{tabular}
\end{center}

In an alternative approach,
child documents can be compiled by a specific command line
without additional code or specific definitions:
%
\begin{center}
|... -jobname "|\textit{target}|" "|[\textit{flags}]%
|\includeonly{|\textit{dest}|}\input{|\textit{main}|}"|
\end{center}
%

%%%%%%%%%%%%%%%%%%%%%%%%%%%%%%%%%%%%%%%%%%%%%%%%%%%%%%%%%%%%%%%%%%%%%%%%%%%%%%%%
%%%%%%%%%%%%%%%%%%%%%%%%%%%%%%%%%%%%%%%%%%%%%%%%%%%%%%%%%%%%%%%%%%%%%%%%%%%%%%%%
\section{Information}

%%%%%%%%%%%%%%%%%%%%%%%%%%%%%%%%%%%%%%%%%%%%%%%%%%%%%%%%%%%%%%%%%%%%%%%%%%%%%%%%
\subsection{Copyright}

Copyright \copyright{} 2017--2018 Niklas Beisert

This work may be distributed and/or modified under the
conditions of the \LaTeX{} Project Public License, either version 1.3
of this license or (at your option) any later version.
The latest version of this license is in
  \url{http://www.latex-project.org/lppl.txt}
and version 1.3 or later is part of all distributions of \LaTeX{}
version 2005/12/01 or later.

This work has the LPPL maintenance status `maintained'.

The Current Maintainer of this work is Niklas Beisert.

This work consists of the files |README.txt|, |childdoc.ins| and |childdoc.dtx|
as well as the derived files |childdoc.def|, |cdocsamp.tex|
with |cdocsch1.tex|, |cdocsch2.tex|, |cdocspt3.tex|, |cdocspt4.tex|,
|cdocsdrf.tex|, |cdocsfn1.tex|, |cdocsfn2.tex|
as well as |childdoc.pdf|.

%%%%%%%%%%%%%%%%%%%%%%%%%%%%%%%%%%%%%%%%%%%%%%%%%%%%%%%%%%%%%%%%%%%%%%%%%%%%%%%%
\subsection{Files and Installation}

The package consists of the files:
%
\begin{center}
\begin{tabular}{ll}
    |README.txt|   & readme file \\
    |childdoc.ins| & installation file \\
    |childdoc.dtx| & source file \\
    |childdoc.def| & definition file \\
    |cdocsamp.tex| & sample main file \\
    |cdocsch1.tex| & sample include file \\
    |cdocsch2.tex| & sample include file \\
    |cdocspt3.tex| & sample part file \\
    |cdocspt4.tex| & sample part file \\
    |cdocsdrf.tex| & sample redirection file \\
    |cdocsfn1.tex| & sample redirection file \\
    |cdocsfn2.tex| & sample redirection file \\
    |childdoc.pdf| & manual
\end{tabular}
\end{center}
%
The distribution consists of the files
|README.txt|, |childdoc.ins| and |childdoc.dtx|.
%
\begin{itemize}
\item
Run (pdf)\LaTeX{} on |childdoc.dtx|
to compile the manual |childdoc.pdf| (this file).
\item
Run \LaTeX{} on |childdoc.ins| to create the definitions file |childdoc.def|
and the sample |cdocsamp.tex| with include files
|cdocsch1.tex|, |cdocsch2.tex|, |cdocspt3.tex|, |cdocspt4.tex|,
|cdocsdrf.tex|, |cdocsfn1.tex|, |cdocsfn2.tex|.
Then copy the file |childdoc.def| to an appropriate directory of your \LaTeX{}
distribution, e.g.\ \textit{texmf-root}|/tex/latex/childdoc|.
\end{itemize}

%%%%%%%%%%%%%%%%%%%%%%%%%%%%%%%%%%%%%%%%%%%%%%%%%%%%%%%%%%%%%%%%%%%%%%%%%%%%%%%%
\subsection{Related CTAN Packages}

There are several other packages which offer a similar functionality:
%
\begin{itemize}
\item
The packages
\href{http://ctan.org/pkg/docmute}{\textsf{docmute}},
\href{http://ctan.org/pkg/includex}{\textsf{includex}} and
\href{http://ctan.org/pkg/standalone}{\textsf{standalone}}
provide commands to include only the document body of
a child file thus allowing both files to be compiled individually.
\item
The packages \href{http://ctan.org/pkg/subdocs}{\textsf{subdocs}}
and \href{http://ctan.org/pkg/subfiles}{\textsf{subfiles}}
provide structures in which the main and child documents can be
encapsulated and allowing them to be compiled individually.
The inclusion mechanism is different from the conventional |\include|.
\item
The package \href{http://ctan.org/pkg/combine}{\textsf{combine}}
is an elaborate solution to combine several documents into one.
\end{itemize}
%
See also the CTAN topic \href{http://ctan.org/topic/subdocs}{\textsf{subdocs}}
for further related packages.
The present package differs from the above solutions in that
a document structure constructed with the conventional |\include| mechanism
just needs two extra commands at the top of every file
such that all constituent files can be compiled individually.

%%%%%%%%%%%%%%%%%%%%%%%%%%%%%%%%%%%%%%%%%%%%%%%%%%%%%%%%%%%%%%%%%%%%%%%%%%%%%%%%
%\subsection{Feature Suggestions}
%
%The following is a list of features which may be useful for future
%versions of this package:
%%
%\begin{itemize}
%\item
%\ldots
%\end{itemize}

%%%%%%%%%%%%%%%%%%%%%%%%%%%%%%%%%%%%%%%%%%%%%%%%%%%%%%%%%%%%%%%%%%%%%%%%%%%%%%%%
\subsection{Revision History}

%%%%%%%%%%%%%%%%%%%%%%%%%%%%%%%%%%%%%%%%
\paragraph{v2.0:} 2018/12/30

\begin{itemize}
\item
immediate forward processing
\item
added |\childdocby| mechanism
\item
manual restructured
\end{itemize}

%%%%%%%%%%%%%%%%%%%%%%%%%%%%%%%%%%%%%%%%
\paragraph{v1.6:} 2018/01/17

\begin{itemize}
\item
application for development of include files
\item
corrections to manual
\end{itemize}

%%%%%%%%%%%%%%%%%%%%%%%%%%%%%%%%%%%%%%%%
\paragraph{v1.5:} 2017/05/21

\begin{itemize}
\item
more complete structuring introduced
\item
|\childdocof| introduced
\item
|\childdoc| renamed to |\childdocmain|
\item
|\childredirect| renamed to |\childdocforward| and |\childdocforwardprefix|
and functionality expanded
\end{itemize}

%%%%%%%%%%%%%%%%%%%%%%%%%%%%%%%%%%%%%%%%
\paragraph{v1.0:} 2017/04/27

\begin{itemize}
\item
manual and install package
\item
first version published on CTAN
\end{itemize}

%%%%%%%%%%%%%%%%%%%%%%%%%%%%%%%%%%%%%%%%
\paragraph{v0.6:} 2017/04/26

\begin{itemize}
\item
redirection mechanism added
\end{itemize}

%%%%%%%%%%%%%%%%%%%%%%%%%%%%%%%%%%%%%%%%
\paragraph{v0.5:} 2017/04/26

\begin{itemize}
\item
functionality in definition file
\end{itemize}


%%%%%%%%%%%%%%%%%%%%%%%%%%%%%%%%%%%%%%%%%%%%%%%%%%%%%%%%%%%%%%%%%%%%%%%%%%%%%%%%
%%%%%%%%%%%%%%%%%%%%%%%%%%%%%%%%%%%%%%%%%%%%%%%%%%%%%%%%%%%%%%%%%%%%%%%%%%%%%%%%
%%%%%%%%%%%%%%%%%%%%%%%%%%%%%%%%%%%%%%%%%%%%%%%%%%%%%%%%%%%%%%%%%%%%%%%%%%%%%%%%
\appendix

\settowidth\MacroIndent{\rmfamily\scriptsize 000\ }

 \DocInput{childdoc.dtx}

\end{document}
%</driver>
% \fi
%
% %%%%%%%%%%%%%%%%%%%%%%%%%%%%%%%%%%%%%%%%%%%%%%%%%%%%%%%%%%%%%%%%%%%%%%%%%%%%%%
% %%%%%%%%%%%%%%%%%%%%%%%%%%%%%%%%%%%%%%%%%%%%%%%%%%%%%%%%%%%%%%%%%%%%%%%%%%%%%%
% \section{Sample}
%\iffalse
%<*samplemain>
%\fi
%
% The following presents a sample document
% with two chapters, two parts, a title page,
% a compile flag as well as three forwarding files to set the flag.
% It consists of eight |.tex| files:
% \begin{center}
% \begin{tabular}{ll}
% |cdocsamp.tex|&main file\\
% |cdocsch1.tex|&include file for chapter 1\\
% |cdocsch2.tex|&include file for chapter 2\\
% |cdocspt3.tex|&include file for part 3\\
% |cdocspt4.tex|&include file for part 4\\
% |cdocsdrf.tex|&forwarding file for main file in draft mode\\
% |cdocsfi1.tex|&forwarding file for final version of chapter 1\\
% |cdocsfi2.tex|&forwarding file for final version of chapter 2\\
% \end{tabular}
% \end{center}
% Each of the eight files can be compiled directly by the \LaTeX{} compiler.
%
% %%%%%%%%%%%%%%%%%%%%%%%%%%%%%%%%%%%%%%
% \paragraph{Main File.}
%
% The main file is called |cdocsamp.tex|.
%
% Load the \textsf{childdoc} definitions and
% declare the filename for the main document:
%    \begin{macrocode}
\input{childdoc.def}
\childdocmain{}
%    \end{macrocode}

% Optional override for |\version| flag:
%    \begin{macrocode}
%%\ifchilddoc\else\providecommand{\version}{draft}\fi
%    \end{macrocode}

% Define the default values for the |\version| flag
% (|final| for the main file and |draft| for childs):
%    \begin{macrocode}
\ifchilddoc
\providecommand{\version}{draft}
\else
\providecommand{\version}{final}
\fi
%    \end{macrocode}

% Load the standard document class:
%    \begin{macrocode}
\documentclass[12pt]{article}
%    \end{macrocode}

% Start the document body:
%    \begin{macrocode}
\begin{document}
%    \end{macrocode}

% Declare a title page.
% Print title, part of document being processed and version flag:
%    \begin{macrocode}
\addtocounter{page}{-1}
\begin{center}
{\LARGE\bfseries{}childdoc example\par}
\vspace{1cm}
\ifchilddoc
\ifchilddocmanual part\else chapter\fi:
`\childdocname' of `\childdocjob'\par
\else
main document: `\childdocjob'\par
\fi
version: \version\par
\end{center}
\newpage
%    \end{macrocode}

% Manually include selected file,
% otherwise process as usual:
%    \begin{macrocode}
\ifchilddocmanual
\section*{part `\childdocname'}
\input{\childdocname}
\else
%    \end{macrocode}

% Include the two chapters:
%    \begin{macrocode}
\include{cdocsch1}
\include{cdocsch2}
%    \end{macrocode}

% Include the two parts unless only chapters should be displayed:
%    \begin{macrocode}
\ifchilddoc\else
\section{part three}
\input{cdocspt3}
\section{part four}
\input{cdocspt4}
\fi
%    \end{macrocode}

% Process as usual until here:
%    \begin{macrocode}
\fi
%    \end{macrocode}

% End of document body:
%    \begin{macrocode}
\end{document}
%    \end{macrocode}
%\iffalse
%</samplemain>
%\fi
%
% %%%%%%%%%%%%%%%%%%%%%%%%%%%%%%%%%%%%%%
% \paragraph{Chapter Include Files.}
%
% The include files are called |cdocsch1.tex| and |cdocsch2.tex|.
%
%\iffalse
%<*samplechap1|samplechap2>
%\fi

% Optional override for |\version| flag:
%    \begin{macrocode}
%%\providecommand{\version}{final}
%    \end{macrocode}

% Include the main document:
%    \begin{macrocode}
\input{childdoc.def}
\childdocof{cdocsamp}
%    \end{macrocode}

%\iffalse
%</samplechap1|samplechap2>
%\fi
%
%\iffalse
%<*samplechap1>
%\fi
% Some text for chapter 1:
%    \begin{macrocode}
\section{one}
some text in chapter one
%    \end{macrocode}

%\iffalse
%</samplechap1>
%\fi
% Some text for chapter 2:
%\iffalse
%<*samplechap2>
%\fi
%    \begin{macrocode}
\section{two}
more text in chapter two
%    \end{macrocode}

%\iffalse
%</samplechap2>
%\fi
%
% %%%%%%%%%%%%%%%%%%%%%%%%%%%%%%%%%%%%%%
% \paragraph{Part Include Files.}
%
% The include files are called |cdocspt3.tex| and |cdocspt4.tex|.
%
%\iffalse
%<*samplepart3|samplepart4>
%\fi

% Optional override for |\version| flag:
%    \begin{macrocode}
%%\providecommand{\version}{final}
%    \end{macrocode}

% Include the main document:
%    \begin{macrocode}
\input{childdoc.def}
\childdocby{cdocsamp}
%    \end{macrocode}

%\iffalse
%</samplepart3|samplepart4>
%\fi
%
%\iffalse
%<*samplepart3>
%\fi
% Some text for part 3:
%    \begin{macrocode}
some text in part three
%    \end{macrocode}

%\iffalse
%</samplepart3>
%\fi
% Some text for part 4:
%\iffalse
%<*samplepart4>
%\fi
%    \begin{macrocode}
more text in part four
%    \end{macrocode}

%\iffalse
%</samplepart4>
%\fi
%
% %%%%%%%%%%%%%%%%%%%%%%%%%%%%%%%%%%%%%%
% \paragraph{Forwarding for a Complete Draft.}
%
% The following forwarding file |cdocsdrf.tex|
% compiles the main document in draft mode:
%\iffalse
%<*sampledraft>
%\fi
%    \begin{macrocode}
\def\version{draft}
\input{childdoc.def}
\childdocforward{cdocsamp}
%    \end{macrocode}

%\iffalse
%</sampledraft>
%\fi
%
% %%%%%%%%%%%%%%%%%%%%%%%%%%%%%%%%%%%%%%
% \paragraph{Forwarding for Final Version of the Chapters.}
%
% The following forwarding files |cdocsfn1.tex| and |cdocsfn2.tex|
% (with identical content)
% compile the final versions of the child documents
% |cdocsch1.tex| and |cdocsch2.tex|, respectively:
%\iffalse
%<*samplefinal>
%\fi
%    \begin{macrocode}
\def\version{final}
\input{childdoc.def}
\childdocforwardprefix[cdocsamp]{cdocsfn}{cdocsch}
%    \end{macrocode}

%\iffalse
%</samplefinal>
%\fi
%
% %%%%%%%%%%%%%%%%%%%%%%%%%%%%%%%%%%%%%%
% \paragraph{Command Line Processing.}
%
% The following three command lines generate the output files
% |cdocscld|, |cdocscl1| and |cdocscl2|
% which should be identical to
% |cdocsdrf|, |cdocsch1| and |cdocsfn2|, respectively:
% \begin{center}
% \begin{tabular}{l}
% |latex -jobname cdocscld \|\\
% |  "\def\version{draft}\input{childdoc.def}\childdocforward{cdocsamp}"|\\
% |latex -jobname cdocscl1 \|\\
% |  "\input{childdoc.def}\childdocforward[cdocsamp]{cdocsch1}"|\\
% |latex -jobname cdocscl2 \|\\
% |  "\def\version{final}\input{childdoc.def}\childdocforward{cdocsch2}"|
% \end{tabular}
% \end{center}
% Note that the trailing backslash on each first line
% merely continues the input to the second line
% (for convenient cut ant paste).
% Furthermore, the command |latex| can be replaced by any
% of its alternative versions such as |pdflatex|.
%
% %%%%%%%%%%%%%%%%%%%%%%%%%%%%%%%%%%%%%%%%%%%%%%%%%%%%%%%%%%%%%%%%%%%%%%%%%%%%%%
% %%%%%%%%%%%%%%%%%%%%%%%%%%%%%%%%%%%%%%%%%%%%%%%%%%%%%%%%%%%%%%%%%%%%%%%%%%%%%%
% \section{Implementation}
%\iffalse
%<*package>
%\fi
%
% This section describes the definitions file |childdoc.def|.

% The definitions cannot be loaded using |\usepackage| or |\RequirePackage|
% which has a mechanism to prevent loading a style file more than once.
% When loading the definitions by means of |\input|
% multiple instances have to be prevented manually:
%\iffalse
%This code needs to be before the `\ProvidesFile' directive
%which is defined at the beginning of this file.
%Therefore it is also placed there and commented out here.
%</package>
%<*discard>
%\fi
%    \begin{macrocode}
\ifdefined\childdocmain\endinput\fi
%    \end{macrocode}
%\iffalse
%</discard>
%<*package>
%\fi
%
% \macro{\ifchilddoc}
% \macro{\ifchilddocmanual}
% The conditional |\ifchilddoc| tells whether a
% child (true) or main (false) document is being compiled.
% The conditional |\ifchilddocmanual| tells whether
% the |\includeonly| mechanism is used (false) or
% the selection of child files must be performed manually (true).
% The definitions initialise to false:
%    \begin{macrocode}
\newif\ifchilddoc
\newif\ifchilddocmanual
%    \end{macrocode}

% \macro{\childdocname}
% \macro{\childdocjob}
% The macro |\childdocname| stores the name of the main document
% to be compiled. The macro |\childdocjob| stores the name of
% the document on which the \LaTeX{} compiler was originally invoked.
% The content of |\jobname| cannot be compared
% to filenames specified in the source due to different catcodes.
% The following code rescans |\jobname|, stores the result
% in |\childdocname| and saves a copy in |\childdocjob|:
%    \begin{macrocode}
\edef\childdocname{\scantokens\expandafter{\jobname\noexpand}}
\let\childdocjob\childdocname
%    \end{macrocode}

% \macro{\childdocdisable}
% The macro |\childdocdisable| prevents the main file
% from being processed more than once.
% At this stage, the main document command |\childdocmain|
% is assumed to be called once again where it should do nothing.
% Any subsequent call to it should prevent
% a secondary processing of the main document
% It overwrites the forwarding commands
% |\childdocof| and |\childdocforward|
% with empty macros to prevent further inclusions of the main document:
%    \begin{macrocode}
\newcommand{\childdocdisable}
{
  \renewcommand{\childdocmain}[1]{\renewcommand{\childdocmain}[1]{\endinput}}
  \renewcommand{\childdocof}[1]{}
  \renewcommand{\childdocby}[2][]{}
  \renewcommand{\childdocforward}[2][]{}
  \renewcommand{\childdocdisable}{}
}
%    \end{macrocode}

% \macro{\childdocmain}
% The macro |\childdocmain| is to be called at the top of the main file
% with nothing or the main filename (without extension) as argument.
% First, it breaks loops.
% If the argument is not empty and does not match |\childdocname|
% (which is set by the first inclusion of |childdoc.def|),
% |\ifchilddoc| is set to true, |\includeonly| is applied to the child file
% and |\jobname| is set to the main file
% (for proper handling of |.aux| files):
%    \begin{macrocode}
\newcommand{\childdocmain}[1]
{
  \childdocdisable\childdocmain{}
  \if?#1?\else
    \begingroup
      \def\childdoctmp{#1}
      \ifx\childdoctmp\childdocname
        \def\childdoctmp{}
      \else
        \def\childdoctmp
        {
          \childdoctrue
          \includeonly{\childdocname}
          \def\childdocjob{#1}
          \def\jobname{#1}
        }
      \fi
      \expandafter
    \endgroup
    \childdoctmp
  \fi
}
%    \end{macrocode}

% \macro{\childdocof}
% The command |\childdocof| redirects
% compilation to the main file |#1|.
%    \begin{macrocode}
\newcommand{\childdocof}[1]
{
  \childdocdisable
  \childdoctrue
  \includeonly{\childdocname}
  \def\jobname{#1}
  \def\childdocjob{#1}
  \input{#1}
}
%    \end{macrocode}

% \macro{\childdocby}
% The command |\childdocby| ....
%    \begin{macrocode}
\newcommand{\childdocby}[2][]
{
  \childdocdisable
  \childdoctrue
  \childdocmanualtrue
  \if?#1?\else
    \def\jobname{#2}
  \fi
  \def\childdocjob{#2}
  \input{#2}
  \endinput
}
%    \end{macrocode}

% \macro{\childdocforward}
% The command |\childdocforward| redirects
% compilation to the main file or
% (if the optional argument is given) a child file.
% Parameters are set as if the main file
% or a child file starting with |\childdocof| was compiled.
% Then compilation is handed over to the main file:
%    \begin{macrocode}
\newcommand{\childdocforward}[2][]
{
  \begingroup
    \if?#1?
      \def\childdoctmp
      {
        \def\childdocname{#2}
        \def\childdocjob{#2}
        \def\jobname{#2}
        \input{#2}
        \endinput
      }
    \else
      \def\childdoctmp
      {
        \childdocdisable
        \def\childdocname{#2}
        \childdoctrue
        \includeonly{#2}
        \def\childdocjob{#1}
        \def\jobname{#1}
        \input{#1}
        \endinput
      }
    \fi
    \expandafter
  \endgroup
  \childdoctmp
}
%    \end{macrocode}

% \macro{\childdocforwardprefix}
% The command |\childdocforwardprefix| redirects
% compilation to the main or a child file by means of a pattern.
% The prefix |#1| in the current filename is replaced by |#2|
% and the suffix of the current filename is kept
% (it is assumed that the filename does not contain the substring `|~~~|'
% which is used as a delimiter).
% Compilation is handed over to the new file by |\childdocforward|:
%    \begin{macrocode}
\newcommand{\childdocforwardprefix}[3][]
{
  \begingroup
    \def\childdocextract #2##1~~~{\def\childdoctmp{\childdocforward[#1]{#3##1}}}
    \expandafter\childdocextract\childdocname~~~
    \expandafter
  \endgroup
  \childdoctmp
}
%    \end{macrocode}

% \macro{\childdoc}
% The deprecated macro |\childdoc| is a legacy version of |\childdocmain|:
%    \begin{macrocode}
\newcommand{\childdoc}{\childdocmain}
%    \end{macrocode}

% \macro{\childdocredirect}
% The deprecated macro |\childdocredirect| is a legacy version
% of |\childdocforward| and |\childdocforwardprefix|:
%    \begin{macrocode}
\newcommand{\childdocredirect}[2][]
{
  \begingroup
    \if?#1?
      \def\childdoctmp{\childdocforward{#2}}
    \else
      \def\childdoctmp{\childdocforwardprefix{#1}{#2}}
    \fi
    \expandafter
  \endgroup
  \childdoctmp
}
%    \end{macrocode}

%\iffalse
%</package>
%\fi
%
\endinput

\childdocby{cdocsamp}
%    \end{macrocode}

%\iffalse
%</samplepart3|samplepart4>
%\fi
%
%\iffalse
%<*samplepart3>
%\fi
% Some text for part 3:
%    \begin{macrocode}
some text in part three
%    \end{macrocode}

%\iffalse
%</samplepart3>
%\fi
% Some text for part 4:
%\iffalse
%<*samplepart4>
%\fi
%    \begin{macrocode}
more text in part four
%    \end{macrocode}

%\iffalse
%</samplepart4>
%\fi
%
% %%%%%%%%%%%%%%%%%%%%%%%%%%%%%%%%%%%%%%
% \paragraph{Forwarding for a Complete Draft.}
%
% The following forwarding file |cdocsdrf.tex|
% compiles the main document in draft mode:
%\iffalse
%<*sampledraft>
%\fi
%    \begin{macrocode}
\def\version{draft}
% \iffalse
%
% childdoc.dtx Copyright (C) 2017-2018 Niklas Beisert
%
% This work may be distributed and/or modified under the
% conditions of the LaTeX Project Public License, either version 1.3
% of this license or (at your option) any later version.
% The latest version of this license is in
%   http://www.latex-project.org/lppl.txt
% and version 1.3 or later is part of all distributions of LaTeX
% version 2005/12/01 or later.
%
% This work has the LPPL maintenance status `maintained'.
%
% The Current Maintainer of this work is Niklas Beisert.
%
% This work consists of the files childdoc.dtx and childdoc.ins
% and the derived files childdoc.def and cdocsamp.tex with
% cdocsch1.tex, cdocsch2.tex, cdocsdrf.tex, cdocsfn1.tex, cdocsfn2.tex.
%
%<package>\ifdefined\childdocmain\endinput\fi
%<package>\ProvidesFile{childdoc.def}[2018/12/30 v2.0 child document driver]
%<samplemain>\ProvidesFile{cdocsamp.tex}[2018/12/30 v2.0 sample for childdoc]
%<*driver>
%\ProvidesFile{childdoc.drv}[2018/12/30 v2.0 childdoc reference manual file]
\PassOptionsToClass{10pt,a4paper}{article}
\documentclass{ltxdoc}

\usepackage[margin=35mm]{geometry}
\usepackage{hyperref}
\usepackage{hyperxmp}
\usepackage[usenames]{color}

\hypersetup{colorlinks=true}
\hypersetup{pdfstartview=FitH}
\hypersetup{pdfpagemode=UseNone}
\hypersetup{pdfsource={}}
\hypersetup{pdflang={en-UK}}
\hypersetup{pdfcopyright={Copyright 2017-2018 Niklas Beisert.
  This work may be distributed and/or modified under the
  conditions of the LaTeX Project Public License, either version 1.3
  of this license or (at your option) any later version.}}
\hypersetup{pdflicenseurl={http://www.latex-project.org/lppl.txt}}
\hypersetup{pdfcontactaddress={ETH Zurich, ITP, HIT K,
  Wolfgang-Pauli-Strasse 27}}
\hypersetup{pdfcontactpostcode={8093}}
\hypersetup{pdfcontactcity={Zurich}}
\hypersetup{pdfcontactcountry={Switzerland}}
\hypersetup{pdfcontactemail={nbeisert@itp.phys.ethz.ch}}
\hypersetup{pdfcontacturl={http://people.phys.ethz.ch/\xmptilde nbeisert/}}

\newcommand{\secref}[1]{\hyperref[#1]{section \ref*{#1}}}

\parskip1ex
\parindent0pt
\let\olditemize\itemize
\def\itemize{\olditemize\parskip0pt}

\begin{document}

\title{The \textsf{childdoc} Package}
\hypersetup{pdftitle={The childdoc Package}}
\author{Niklas Beisert\\[2ex]
  Institut f\"ur Theoretische Physik\\
  Eidgen\"ossische Technische Hochschule Z\"urich\\
  Wolfgang-Pauli-Strasse 27, 8093 Z\"urich, Switzerland\\[1ex]
  \href{mailto:nbeisert@itp.phys.ethz.ch}
  {\texttt{nbeisert@itp.phys.ethz.ch}}}
\hypersetup{pdfauthor={Niklas Beisert}}
\hypersetup{pdfsubject={Manual for the LaTeX2e Package childdoc}}
\date{30 December 2018, \textsf{v2.0}}
\maketitle

\begin{abstract}\noindent
\textsf{childdoc} is a \LaTeXe{} package
that enables the direct compilation
of document sections included by |\include|
to individual files.
\end{abstract}

\begingroup
\parskip0ex
\tableofcontents
\endgroup

%%%%%%%%%%%%%%%%%%%%%%%%%%%%%%%%%%%%%%%%%%%%%%%%%%%%%%%%%%%%%%%%%%%%%%%%%%%%%%%%
%%%%%%%%%%%%%%%%%%%%%%%%%%%%%%%%%%%%%%%%%%%%%%%%%%%%%%%%%%%%%%%%%%%%%%%%%%%%%%%%
\section{Introduction}

\LaTeX{} provides a mechanism to structure a large document (such as a book)
into a main file and several child files (containing the chapters)
using the |\include| command.
This mechanism is beneficial for documents
which span hundreds of pages in order to
make the source file(s) more manageable.
Moreover, compilation can be restricted to
selected child files by means of the |\includeonly| command.
The latter feature can be used to reduce the compilation time while editing
(this was significantly more useful in the earlier days of \LaTeX{})
or to generate a smaller document which is easier to navigate.
Another application of |\includeonly| is to generate
documents consisting of selected parts of the complete document.

However, there are a few drawbacks of the plain |\include| mechanism:
\begin{itemize}
\item
The child files cannot be compiled on their own,
they can only be compiled via the main file.
A naive editing environment
(such as a text editor with an option
to have the current file processed by \LaTeX)
may require one to switch to the main file before compiling;
attempting to compile the child file produces errors.
\item
The main file must be modified (each time)
to adjust the |\includeonly| command
to the present needs. This easily leaves the main file in a messy state.
\item
The generated document will always carry the filename
of the main document. This is inconvenient if
several child files are to be compiled and
to be kept for distribution.
\end{itemize}

The present package provides a simple interface
to make child files individually compilable by \LaTeX{}.
Compiling a child file then has the same effect as compiling
the main file with an |\includeonly| command
to select the appropriate child.
Moreover the generated document will carry the name of the child
rather than the main file.
This resolves all three above issues.

This feature is meant to make the editing of books,
thesis documents and lecture notes somewhat more convenient.
However, the package can also be used efficiently for
composing a series of documents (such as exercise sheets)
which are typically distributed individually.
It then assists the author in generating the individual documents
(potentially in different versions)
as well as a document containing the collected series.
Another application is in developing style files
or other kinds of included material
where compilation of the style file could redirect
to a sample or test file.

%%%%%%%%%%%%%%%%%%%%%%%%%%%%%%%%%%%%%%%%%%%%%%%%%%%%%%%%%%%%%%%%%%%%%%%%%%%%%%%%
%%%%%%%%%%%%%%%%%%%%%%%%%%%%%%%%%%%%%%%%%%%%%%%%%%%%%%%%%%%%%%%%%%%%%%%%%%%%%%%%
\section{Usage}

First of all, the package \textsf{childdoc} is \emph{not} a standard
\LaTeXe{} |.sty| style file! Therefore it needs to be invoked in
a non-standard way.

%%%%%%%%%%%%%%%%%%%%%%%%%%%%%%%%%%%%%%%%%%%%%%%%%%%%%%%%%%%%%%%%%%%%%%%%%%%%%%%%
\subsection{Included Files}
\label{sec:include}

%%%%%%%%%%%%%%%%%%%%%%%%%%%%%%%%%%%%%%%%
\DescribeMacro{\childdocmain}
To use the package, add the commands
\begin{center}
\begin{tabular}{l}
|\input{childdoc.def}|\\
|\childdocmain{}|\\
\end{tabular}
\end{center}
at the very top of the main \LaTeX{} file,
in particular \emph{before} the |\documentclass| statement!
The argument of |\childdocmain| should be left empty
(but it must be present).

%%%%%%%%%%%%%%%%%%%%%%%%%%%%%%%%%%%%%%%%
\DescribeMacro{\childdocof}
Furthermore, add the commands
\begin{center}
\begin{tabular}{l}
|\input{childdoc.def}|\\
|\childdocof{|\textit{main}|}|\\
\end{tabular}
\end{center}
at the top of every child file \textit{child}
which is included by |\include{|\textit{child}|}|
from within the main file
(or at least for those files to be compiled individually).
The argument \textit{main} must be the filename of the main file.

There are a couple of
considerations in setting up the main and child documents:

%%%%%%%%%%%%%%%%%%%%%%%%%%%%%%%%%%%%%%%%
\paragraph{Restrictions.}

Please note the following restrictions:
\begin{itemize}
\item
|\childdocmain| must be called with one argument \textit{main}
to ensure compatibility with earlier version of the package.
It must either be empty (|\childdocmain{}|)
or precisely match the filename of the main file in which it is specified.
See \secref{sec:detection} for further information.
\item
The filename \textit{main} must be specified without the |.tex| extension.
\item
The filename \textit{main} is case sensitive
(even in case-insensitive file systems)
due to internal string comparison.
\item
The argument \textit{main} should be fully expanded, it cannot be a macro.
\item
Subdirectories and special characters should be avoided in filenames.
\item
The command |\childdocmain{|\textit{main}|}| must be followed by a whitespace.
It should not be followed immediately by another command
or by a comment mark `|%|'.
This is because the \TeX{} parser reads the token immediately following
the argument of |\childdocmain| and puts it
at the beginning of every child section;
however, a white\-space is ignored.
\end{itemize}

%%%%%%%%%%%%%%%%%%%%%%%%%%%%%%%%%%%%%%%%
\paragraph{Content of Main File.}

It is advisable to place all content in the child files included by |\include|.
Any output contained in the main file will appear in all child documents
unless suppressed manually;
it cannot be suppressed automatically by the |\includeonly| directive
and thus should normally be avoided.
A method to include some content in the main file
by means of conditional processing is described in \secref{sec:conditional}.

%%%%%%%%%%%%%%%%%%%%%%%%%%%%%%%%%%%%%%%%
\paragraph{Page Numbering.}

When only a part of the document is compiled,
the appropriate numbering of pages
(as well as other status parameters)
is determined from the |.aux| files.
The latter contain information from previous passes.
However this information needs to propagate through
all intermediate child documents.
Therefore the page numbering in child documents may well
be inconsistent until the complete document is compiled at least once.

A useful (if unconventional) way to always ensure a consistent
page numbering is to restart the numbering in each child document
and denote the pages by `\textit{child}|.|\textit{page}'
where \textit{child} represents the chapter/section number of the child file.
This can be achieved by the command
|\numberwithin{page}{|\textit{child}|}|
of the \textsf{amsmath} package
where \textit{child} can be |chapter| or |section|
depending on the chosen structuring.
Alternatively, one can modify the macro |\thepage| appropriately
and reset the counter |page| at the start of each child file.

%%%%%%%%%%%%%%%%%%%%%%%%%%%%%%%%%%%%%%%%%%%%%%%%%%%%%%%%%%%%%%%%%%%%%%%%%%%%%%%%
\subsection{Conditional Processing}
\label{sec:conditional}

The package provides a mechanism to compile different versions
of a document. To customise the versions further some conditional processing
can come in handy to distinguish which version is being compiled.
The package provides two macros to describe the compilation context:

%%%%%%%%%%%%%%%%%%%%%%%%%%%%%%%%%%%%%%%%
\DescribeMacro{\ifchilddoc}
The conditional |\ifchilddoc| distinguishes between the compilation of
child documents and the main document:
%
\begin{center}
|\ifchilddoc |\textit{child-code}| |[|\||else |\textit{main-code}]| \||fi|
\end{center}

%%%%%%%%%%%%%%%%%%%%%%%%%%%%%%%%%%%%%%%%
\DescribeMacro{\childdocname}
\DescribeMacro{\childdocjob}
The macro |\childdocname| contains the filename (without extension)
of the main or child file being processed.
Note that |\childdocjob| will always contain the name of the main file.

%%%%%%%%%%%%%%%%%%%%%%%%%%%%%%%%%%%%%%%%
\paragraph{Title Page.}

Conditional processing can be used to include a title or banner page
in the main document when proper precautions are taken.
Importantly, the code in the main file should ensure that the page counter
(as well as other status parameters which are stored in the |.aux| files)
takes the same value after the conditional processing.
Otherwise the page numbers may take divergent values
depending on which part is compiled.

For example, a title page could be declared by:
%
\begin{center}
\begin{tabular}{l}
|\ifchilddoc\||else|\\
|\addtocounter{page}{-1}|\\
\textit{code for title page}\\
|\newpage|\\
|\||fi|
\end{tabular}
\end{center}
%
A banner page for the child documents can be generated by:
%
\begin{center}
\begin{tabular}{l}
|\ifchilddoc|\\
|\addtocounter{page}{-1}|\\
\textit{code for banner page}\\
|\newpage|\\
|\||fi|
\end{tabular}
\end{center}
%
Here one could write a message such as:
\begin{center}
|This is the part \childdocname{} of \childdocjob{}.|
\end{center}

%%%%%%%%%%%%%%%%%%%%%%%%%%%%%%%%%%%%%%%%%%%%%%%%%%%%%%%%%%%%%%%%%%%%%%%%%%%%%%%%
\subsection{Flags}
\label{sec:flags}

The package makes it easy to generate different versions
of the main or child documents.
To this end compilation flags can be defined
and assigned different default values.
They will be particularly useful in conjunction
with the forwarding mechanism described in \secref{sec:forward}.

For example, it may be useful to have a flag |\version|
which can be set to |draft| or |final|.
The document source will contain some conditional code
depending on the value of |\version|.
Suppose further, the flag should default to |final| for the main file
and to |draft| for child files
which is a natural assignment for editing the document.
This is achieved by placing the following code
in the preamble of the main document
(below the |\childdocmain| directive):
%
\begin{center}
\begin{tabular}{l}
|\ifchilddoc|\\
|\providecommand{\version}{draft}|\\
|\||else|\\
|\providecommand{\version}{final}|\\
|\||fi|
\end{tabular}
\end{center}
%
The definition by |\providecommand| makes sure
that previous definitions are not overwritten.
Further statements |\providecommand{\version}{...}|
can thus be added before the above code to override it.

For the main file, one might add a line
(between |\childdocmain| and the above block)
%
\begin{center}
|%\ifchilddoc\||else\providecommand{\version}{draft}\||fi|
\end{center}
%
which can be uncommented to produce a draft version.
Likewise one can add a line to the very top of a child file
(above the |\childdocof{|\textit{main}|}| directive)
%
\begin{center}
|%\providecommand{\version}{final}|
\end{center}
%
which can be uncommented to produce the final version of this child document.

%%%%%%%%%%%%%%%%%%%%%%%%%%%%%%%%%%%%%%%%%%%%%%%%%%%%%%%%%%%%%%%%%%%%%%%%%%%%%%%%
\subsection{Forwarding}
\label{sec:forward}

Different versions of the main or child documents
using compilation flags as described in \secref{sec:flags}
can be (permanently) stored in different files
for convenient compilation, viewing and distribution.
To this end, the package defines a command
to pass on compilation to a different file:

%%%%%%%%%%%%%%%%%%%%%%%%%%%%%%%%%%%%%%%%
\DescribeMacro{\childdocforward}
The command |\childdocforward| redirects processing to
another source file:
%
\begin{center}
\begin{tabular}{l}
|\input{childdoc.def}|\\
|\childdocforward[|\textit{main}|]{|\textit{dest}|}|\\
\end{tabular}
\end{center}
%
The argument \textit{dest} is the destination file
(without extension).
It should be the main file or one of the child files.
Note that further \textsf{childdoc} directives
such as |\childdocof| and |\childdocforward|
in the indicated file will be processed in this form.
The optional argument \textit{main}
passes on directly to the main file \textit{main}
while pretending to compile the child \textit{dest}.
This form behaves as if \textit{dest}
issues |\childdocof{|\textit{main}|}| right away,
and no further \textsf{childdoc} directives will be processed.

%%%%%%%%%%%%%%%%%%%%%%%%%%%%%%%%%%%%%%%%
\DescribeMacro{\...prefix}
In the alternative form |\childdocforwardprefix|,
%
\begin{center}
\begin{tabular}{l}
|\input{childdoc.def}|\\
|\childdocforwardprefix[|\textit{main}|]{|\textit{prefix}|}{|\textit{dest}|}|
\end{tabular}
\end{center}
%
the destination file is determined by a pattern
depending on the current file:
To make this work, the current file must be called
`{\textit{prefix}\hspace{0.2em}\textit{suffix}}'
with \textit{prefix} matching precisely the argument.
Processing is then passed on to the file
`{\textit{dest}\hspace{0.2em}\textit{suffix}}'.
Surely, the same effect is achieved by
directly specifying the
argument `{\textit{dest}\hspace{0.2em}\textit{suffix}}'
in the first form.
However, that requires to set up a different file
for each child. With the alternative form of the command
all these files can have exactly the same content
which simplifies setting them up and maintaining them.

For example, the following file |draft.tex|
with a compilation flag |\version| as described in \secref{sec:flags}
compiles the main document as a draft:
%
\begin{center}
\begin{tabular}{l}
|\def\version{draft}|\\
|\input{childdoc.def}|\\
|\childdocforward{|\textit{main}|}|
\end{tabular}
\end{center}
%
Likewise, the following files |final|\textit{nn}|.tex|
compile the final version of the child document
|child|\textit{nn}|.tex|:
%
\begin{center}
\begin{tabular}{l}
|\def\version{final}|\\
|\input{childdoc.def}|\\
|\childdocforwardprefix{final}{child}|
\end{tabular}
\end{center}
%

Note that when several versions of a main file and/or of each child file
are to be generated, it may be convenient to set up a |Makefile| or
shell script to automatise the process.

%%%%%%%%%%%%%%%%%%%%%%%%%%%%%%%%%%%%%%%%%%%%%%%%%%%%%%%%%%%%%%%%%%%%%%%%%%%%%%%%
\subsection{Command Line Processing}
\label{sec:commandline}

The effect of redirection files can also be achieved by invoking
the \LaTeX{} compiler with a more elaborate command line.
Most conveniently this should be done as part
of a shell script or a |Makefile|.

When using \textsf{childdoc} in the main file, the following
command lines effectively perform a redirection
(note that depending on the shell being used,
backslashes may have to be doubled: `|\|' $\to$ `|\\|'):
%
\begin{center}
|... -jobname "|\textit{target}|" |\\|"|[\textit{flags}]%
|\input{childdoc.def}\childdocforward[|\textit{main}|]{|\textit{dest}|}"|
\end{center}
%
Here \textit{target} is the name of the output file,
\textit{main} is the name of the main file
and \textit{dest} is the name of the main or child file to be processed
(all filenames without extensions).
The optional argument \textit{main} can be omitted
if \textit{main} matches \textit{dest}.
Optionally, compilation \textit{flags} can be defined via |\def| commands.
This command line makes the \TeX{} engine believe
it is compiling the file \textit{target}
whose content is specified as the latter parameter.
The provided code then forwards the processing to
\textit{main} or \textit{dest} as described in \secref{sec:forward}.

%%%%%%%%%%%%%%%%%%%%%%%%%%%%%%%%%%%%%%%%%%%%%%%%%%%%%%%%%%%%%%%%%%%%%%%%%%%%%%%%
\subsection{Include by Input}
\label{sec:input}

Including child documents by |\include| has some restrictions by design.
Most notably, the content of a child document always occupies
its own set of pages; pages cannot be shared between child documents.
Usually, this behaviour makes perfect sense
because each child document contain an essential part of the document.
However, in some situations it may be desirable to compose
a document from a collection of parts
without having mandatory page breaks between then.
For this case, the package
provides a mechanism to include parts
by |\input| which can also be processed individually.
However, by construction this mechanism
requires manual handling of the content to be output.

%%%%%%%%%%%%%%%%%%%%%%%%%%%%%%%%%%%%%%%%
\DescribeMacro{\ifchilddocmanual}
The main file should be prepared as usual, see \secref{sec:include}.
However, the document body must make a distinction
between processing of an individual part and of the main document, e.g.:
%
\begin{center}
\begin{tabular}{l}
|\ifchilddocmanual|\\
|\input{\childdocname}|\\
|\||else|\\
\textit{document body with }|\input{|\textit{part}|}|\\
|\||fi|
\end{tabular}
\end{center}
%
The conditional |\ifchilddocmanual| is true whenever
a part to be included by |\input| is being compiled,
and the name of the part is stored in |\childdocname|.

%%%%%%%%%%%%%%%%%%%%%%%%%%%%%%%%%%%%%%%%
\DescribeMacro{\childdocby}
Each part to be included by |\input| should start with:
%
\begin{center}
\begin{tabular}{l}
|\input{childdoc.def}|\\
|\childdocby{|\textit{main}|}|\\
\end{tabular}
\end{center}
%
The directive |\childdocby| is similar to |\childdocof|
described in \secref{sec:include},
but the subsequent selection of content must be done manually.
To that end, both |\ifchilddoc| and |\ifchilddocmanual|
will be true upon processing of a part,
and the name of the part is stored in |\childdocname|.
Note that |\jobname| will be set to the filename of the current part
so that each part receives an individual |.aux| file
that does not interfere with the |.aux| file(s) of the main document.
This behaviour can be altered by the alternative form
|\childdocby[*]{|\textit{main}|}| (with a non-empty optional argument)
which uses the |.aux| file of the main document
by setting |\jobname| to \textit{main}.

%%%%%%%%%%%%%%%%%%%%%%%%%%%%%%%%%%%%%%%%%%%%%%%%%%%%%%%%%%%%%%%%%%%%%%%%%%%%%%%%
\subsection{Driver Development}
\label{sec:driver}

The \textsf{childdoc} mechanism can also be use for the development
of definition files such as \LaTeX{} styles or classes.
This case differs from the above setup with multiple parts
included by |\include| in that no |\includeonly| should be invoked.
This can be achieved by starting the include file
(before |\ProvidesPackage|) with:
%
\begin{center}
\begin{tabular}{l}
|\input{childdoc.def}|\\
|\childdocforward{|\textit{main}|}|\\
\end{tabular}
\end{center}
%
or alternatively with:
%
\begin{center}
\begin{tabular}{l}
|\input{childdoc.def}|\\
|\childdocby{|\textit{main}|}|\\
\end{tabular}
\end{center}
%
Both forms have slightly different effects as described above.
The main file is prepared as usual, see \secref{sec:include}.

%%%%%%%%%%%%%%%%%%%%%%%%%%%%%%%%%%%%%%%%%%%%%%%%%%%%%%%%%%%%%%%%%%%%%%%%%%%%%%%%
\subsection{Legacy Detection}
\label{sec:detection}

The directive |\childdocmain| in the main file can detect
whether the complete document or merely a child is to be compiled
even without using the directive |\childdocof|.
This method is deprecated because it is less robust
and there is no compelling reason to use it;
it is merely provided for backward compatibility
and it may be removed in future versions.

If the detection mechanism is to be used,
it is mandatory to correctly specify
the filename of the main file as the argument of |\childdocmain|:
%
\begin{center}
\begin{tabular}{l}
|\input{childdoc.def}|\\
|\childdocmain{|\textit{main}|}|\\
\end{tabular}
\end{center}
%
If |\jobname| does not match the argument \textit{main} of |\childdocmain|,
it is assumed that |\jobname| points to the child file to be compiled.
When using |\childdocmain| with the main file specified as argument,
it suffices to start a child file
with just |\input{|\textit{main}|}|
without loading of the package and using |\childdocof|.
If instead all processing is done
with the appropriate \textsf{childdoc} directives,
the argument of \textit{main} of |\childdocmain| can be empty.

An alternative version of the command line processing described
in \secref{sec:commandline} using the detection mechanism reads:
%
\begin{center}
|... -jobname "|\textit{target}|" "|[\textit{flags}]%
[|\def\jobname{|\textit{dest}|}|]|\input{|\textit{main}|}"|
\end{center}

%%%%%%%%%%%%%%%%%%%%%%%%%%%%%%%%%%%%%%%%%%%%%%%%%%%%%%%%%%%%%%%%%%%%%%%%%%%%%%%%
\subsection{Manual Code}
\label{sec:manual}

In case one cannot be certain whether the definitions file |childdoc.def|
is installed on the target \TeX{} distribution
and one prefers not to ship it,
it is conceivable to paste a few relevant commands into the sources.

To that end, drop all statements |\input{childdoc.def}|
and perform the replacements as outlined below.
Instead of |\childdocmain{|\textit{main}|}| add the following code
to the top of the main file:
%
\begin{center}
\begin{tabular}{l}
|\||ifdefined\childdocname\endinput\||fi\newif\ifchilddoc|\\
|\edef\childdocname{\scantokens\expandafter{\jobname\noexpand}}|\\
|\def\childdocmain{|\textit{main}|}\||ifx\childdocmain\childdocname\||else|\\
|\childdoctrue\includeonly{\childdocname}\let\jobname\childdocmain\||fi|\\
\end{tabular}
\end{center}
%
Instead of |\childdocof{|\textit{main}|}| just include the main file
at the top of each child file:
%
\begin{center}
|\input{|\textit{main}|}|
\end{center}
%
A simple redirection |\childdocforward{|\textit{dest}|}| is achieved by:
%
\begin{center}
|\def\jobname{|\textit{dest}|}\input{\jobname}|
\end{center}
%
The redirection with prefix
|\childdocforwardprefix[|\textit{prefix}|]{|\textit{dest}|}|
is accomplished by:
%
\begin{center}
\begin{tabular}{l}
|{\edef\jobname{\scantokens\expandafter{\jobname\noexpand}}|\\
|\def\redirectjob |\textit{prefix}|#1~~~{\gdef\jobname{|\textit{dest}|#1}}|\\
|\expandafter\redirectjob\jobname~~~}\input{\jobname}|
\end{tabular}
\end{center}

In an alternative approach,
child documents can be compiled by a specific command line
without additional code or specific definitions:
%
\begin{center}
|... -jobname "|\textit{target}|" "|[\textit{flags}]%
|\includeonly{|\textit{dest}|}\input{|\textit{main}|}"|
\end{center}
%

%%%%%%%%%%%%%%%%%%%%%%%%%%%%%%%%%%%%%%%%%%%%%%%%%%%%%%%%%%%%%%%%%%%%%%%%%%%%%%%%
%%%%%%%%%%%%%%%%%%%%%%%%%%%%%%%%%%%%%%%%%%%%%%%%%%%%%%%%%%%%%%%%%%%%%%%%%%%%%%%%
\section{Information}

%%%%%%%%%%%%%%%%%%%%%%%%%%%%%%%%%%%%%%%%%%%%%%%%%%%%%%%%%%%%%%%%%%%%%%%%%%%%%%%%
\subsection{Copyright}

Copyright \copyright{} 2017--2018 Niklas Beisert

This work may be distributed and/or modified under the
conditions of the \LaTeX{} Project Public License, either version 1.3
of this license or (at your option) any later version.
The latest version of this license is in
  \url{http://www.latex-project.org/lppl.txt}
and version 1.3 or later is part of all distributions of \LaTeX{}
version 2005/12/01 or later.

This work has the LPPL maintenance status `maintained'.

The Current Maintainer of this work is Niklas Beisert.

This work consists of the files |README.txt|, |childdoc.ins| and |childdoc.dtx|
as well as the derived files |childdoc.def|, |cdocsamp.tex|
with |cdocsch1.tex|, |cdocsch2.tex|, |cdocspt3.tex|, |cdocspt4.tex|,
|cdocsdrf.tex|, |cdocsfn1.tex|, |cdocsfn2.tex|
as well as |childdoc.pdf|.

%%%%%%%%%%%%%%%%%%%%%%%%%%%%%%%%%%%%%%%%%%%%%%%%%%%%%%%%%%%%%%%%%%%%%%%%%%%%%%%%
\subsection{Files and Installation}

The package consists of the files:
%
\begin{center}
\begin{tabular}{ll}
    |README.txt|   & readme file \\
    |childdoc.ins| & installation file \\
    |childdoc.dtx| & source file \\
    |childdoc.def| & definition file \\
    |cdocsamp.tex| & sample main file \\
    |cdocsch1.tex| & sample include file \\
    |cdocsch2.tex| & sample include file \\
    |cdocspt3.tex| & sample part file \\
    |cdocspt4.tex| & sample part file \\
    |cdocsdrf.tex| & sample redirection file \\
    |cdocsfn1.tex| & sample redirection file \\
    |cdocsfn2.tex| & sample redirection file \\
    |childdoc.pdf| & manual
\end{tabular}
\end{center}
%
The distribution consists of the files
|README.txt|, |childdoc.ins| and |childdoc.dtx|.
%
\begin{itemize}
\item
Run (pdf)\LaTeX{} on |childdoc.dtx|
to compile the manual |childdoc.pdf| (this file).
\item
Run \LaTeX{} on |childdoc.ins| to create the definitions file |childdoc.def|
and the sample |cdocsamp.tex| with include files
|cdocsch1.tex|, |cdocsch2.tex|, |cdocspt3.tex|, |cdocspt4.tex|,
|cdocsdrf.tex|, |cdocsfn1.tex|, |cdocsfn2.tex|.
Then copy the file |childdoc.def| to an appropriate directory of your \LaTeX{}
distribution, e.g.\ \textit{texmf-root}|/tex/latex/childdoc|.
\end{itemize}

%%%%%%%%%%%%%%%%%%%%%%%%%%%%%%%%%%%%%%%%%%%%%%%%%%%%%%%%%%%%%%%%%%%%%%%%%%%%%%%%
\subsection{Related CTAN Packages}

There are several other packages which offer a similar functionality:
%
\begin{itemize}
\item
The packages
\href{http://ctan.org/pkg/docmute}{\textsf{docmute}},
\href{http://ctan.org/pkg/includex}{\textsf{includex}} and
\href{http://ctan.org/pkg/standalone}{\textsf{standalone}}
provide commands to include only the document body of
a child file thus allowing both files to be compiled individually.
\item
The packages \href{http://ctan.org/pkg/subdocs}{\textsf{subdocs}}
and \href{http://ctan.org/pkg/subfiles}{\textsf{subfiles}}
provide structures in which the main and child documents can be
encapsulated and allowing them to be compiled individually.
The inclusion mechanism is different from the conventional |\include|.
\item
The package \href{http://ctan.org/pkg/combine}{\textsf{combine}}
is an elaborate solution to combine several documents into one.
\end{itemize}
%
See also the CTAN topic \href{http://ctan.org/topic/subdocs}{\textsf{subdocs}}
for further related packages.
The present package differs from the above solutions in that
a document structure constructed with the conventional |\include| mechanism
just needs two extra commands at the top of every file
such that all constituent files can be compiled individually.

%%%%%%%%%%%%%%%%%%%%%%%%%%%%%%%%%%%%%%%%%%%%%%%%%%%%%%%%%%%%%%%%%%%%%%%%%%%%%%%%
%\subsection{Feature Suggestions}
%
%The following is a list of features which may be useful for future
%versions of this package:
%%
%\begin{itemize}
%\item
%\ldots
%\end{itemize}

%%%%%%%%%%%%%%%%%%%%%%%%%%%%%%%%%%%%%%%%%%%%%%%%%%%%%%%%%%%%%%%%%%%%%%%%%%%%%%%%
\subsection{Revision History}

%%%%%%%%%%%%%%%%%%%%%%%%%%%%%%%%%%%%%%%%
\paragraph{v2.0:} 2018/12/30

\begin{itemize}
\item
immediate forward processing
\item
added |\childdocby| mechanism
\item
manual restructured
\end{itemize}

%%%%%%%%%%%%%%%%%%%%%%%%%%%%%%%%%%%%%%%%
\paragraph{v1.6:} 2018/01/17

\begin{itemize}
\item
application for development of include files
\item
corrections to manual
\end{itemize}

%%%%%%%%%%%%%%%%%%%%%%%%%%%%%%%%%%%%%%%%
\paragraph{v1.5:} 2017/05/21

\begin{itemize}
\item
more complete structuring introduced
\item
|\childdocof| introduced
\item
|\childdoc| renamed to |\childdocmain|
\item
|\childredirect| renamed to |\childdocforward| and |\childdocforwardprefix|
and functionality expanded
\end{itemize}

%%%%%%%%%%%%%%%%%%%%%%%%%%%%%%%%%%%%%%%%
\paragraph{v1.0:} 2017/04/27

\begin{itemize}
\item
manual and install package
\item
first version published on CTAN
\end{itemize}

%%%%%%%%%%%%%%%%%%%%%%%%%%%%%%%%%%%%%%%%
\paragraph{v0.6:} 2017/04/26

\begin{itemize}
\item
redirection mechanism added
\end{itemize}

%%%%%%%%%%%%%%%%%%%%%%%%%%%%%%%%%%%%%%%%
\paragraph{v0.5:} 2017/04/26

\begin{itemize}
\item
functionality in definition file
\end{itemize}


%%%%%%%%%%%%%%%%%%%%%%%%%%%%%%%%%%%%%%%%%%%%%%%%%%%%%%%%%%%%%%%%%%%%%%%%%%%%%%%%
%%%%%%%%%%%%%%%%%%%%%%%%%%%%%%%%%%%%%%%%%%%%%%%%%%%%%%%%%%%%%%%%%%%%%%%%%%%%%%%%
%%%%%%%%%%%%%%%%%%%%%%%%%%%%%%%%%%%%%%%%%%%%%%%%%%%%%%%%%%%%%%%%%%%%%%%%%%%%%%%%
\appendix

\settowidth\MacroIndent{\rmfamily\scriptsize 000\ }

 \DocInput{childdoc.dtx}

\end{document}
%</driver>
% \fi
%
% %%%%%%%%%%%%%%%%%%%%%%%%%%%%%%%%%%%%%%%%%%%%%%%%%%%%%%%%%%%%%%%%%%%%%%%%%%%%%%
% %%%%%%%%%%%%%%%%%%%%%%%%%%%%%%%%%%%%%%%%%%%%%%%%%%%%%%%%%%%%%%%%%%%%%%%%%%%%%%
% \section{Sample}
%\iffalse
%<*samplemain>
%\fi
%
% The following presents a sample document
% with two chapters, two parts, a title page,
% a compile flag as well as three forwarding files to set the flag.
% It consists of eight |.tex| files:
% \begin{center}
% \begin{tabular}{ll}
% |cdocsamp.tex|&main file\\
% |cdocsch1.tex|&include file for chapter 1\\
% |cdocsch2.tex|&include file for chapter 2\\
% |cdocspt3.tex|&include file for part 3\\
% |cdocspt4.tex|&include file for part 4\\
% |cdocsdrf.tex|&forwarding file for main file in draft mode\\
% |cdocsfi1.tex|&forwarding file for final version of chapter 1\\
% |cdocsfi2.tex|&forwarding file for final version of chapter 2\\
% \end{tabular}
% \end{center}
% Each of the eight files can be compiled directly by the \LaTeX{} compiler.
%
% %%%%%%%%%%%%%%%%%%%%%%%%%%%%%%%%%%%%%%
% \paragraph{Main File.}
%
% The main file is called |cdocsamp.tex|.
%
% Load the \textsf{childdoc} definitions and
% declare the filename for the main document:
%    \begin{macrocode}
\input{childdoc.def}
\childdocmain{}
%    \end{macrocode}

% Optional override for |\version| flag:
%    \begin{macrocode}
%%\ifchilddoc\else\providecommand{\version}{draft}\fi
%    \end{macrocode}

% Define the default values for the |\version| flag
% (|final| for the main file and |draft| for childs):
%    \begin{macrocode}
\ifchilddoc
\providecommand{\version}{draft}
\else
\providecommand{\version}{final}
\fi
%    \end{macrocode}

% Load the standard document class:
%    \begin{macrocode}
\documentclass[12pt]{article}
%    \end{macrocode}

% Start the document body:
%    \begin{macrocode}
\begin{document}
%    \end{macrocode}

% Declare a title page.
% Print title, part of document being processed and version flag:
%    \begin{macrocode}
\addtocounter{page}{-1}
\begin{center}
{\LARGE\bfseries{}childdoc example\par}
\vspace{1cm}
\ifchilddoc
\ifchilddocmanual part\else chapter\fi:
`\childdocname' of `\childdocjob'\par
\else
main document: `\childdocjob'\par
\fi
version: \version\par
\end{center}
\newpage
%    \end{macrocode}

% Manually include selected file,
% otherwise process as usual:
%    \begin{macrocode}
\ifchilddocmanual
\section*{part `\childdocname'}
\input{\childdocname}
\else
%    \end{macrocode}

% Include the two chapters:
%    \begin{macrocode}
\include{cdocsch1}
\include{cdocsch2}
%    \end{macrocode}

% Include the two parts unless only chapters should be displayed:
%    \begin{macrocode}
\ifchilddoc\else
\section{part three}
\input{cdocspt3}
\section{part four}
\input{cdocspt4}
\fi
%    \end{macrocode}

% Process as usual until here:
%    \begin{macrocode}
\fi
%    \end{macrocode}

% End of document body:
%    \begin{macrocode}
\end{document}
%    \end{macrocode}
%\iffalse
%</samplemain>
%\fi
%
% %%%%%%%%%%%%%%%%%%%%%%%%%%%%%%%%%%%%%%
% \paragraph{Chapter Include Files.}
%
% The include files are called |cdocsch1.tex| and |cdocsch2.tex|.
%
%\iffalse
%<*samplechap1|samplechap2>
%\fi

% Optional override for |\version| flag:
%    \begin{macrocode}
%%\providecommand{\version}{final}
%    \end{macrocode}

% Include the main document:
%    \begin{macrocode}
\input{childdoc.def}
\childdocof{cdocsamp}
%    \end{macrocode}

%\iffalse
%</samplechap1|samplechap2>
%\fi
%
%\iffalse
%<*samplechap1>
%\fi
% Some text for chapter 1:
%    \begin{macrocode}
\section{one}
some text in chapter one
%    \end{macrocode}

%\iffalse
%</samplechap1>
%\fi
% Some text for chapter 2:
%\iffalse
%<*samplechap2>
%\fi
%    \begin{macrocode}
\section{two}
more text in chapter two
%    \end{macrocode}

%\iffalse
%</samplechap2>
%\fi
%
% %%%%%%%%%%%%%%%%%%%%%%%%%%%%%%%%%%%%%%
% \paragraph{Part Include Files.}
%
% The include files are called |cdocspt3.tex| and |cdocspt4.tex|.
%
%\iffalse
%<*samplepart3|samplepart4>
%\fi

% Optional override for |\version| flag:
%    \begin{macrocode}
%%\providecommand{\version}{final}
%    \end{macrocode}

% Include the main document:
%    \begin{macrocode}
\input{childdoc.def}
\childdocby{cdocsamp}
%    \end{macrocode}

%\iffalse
%</samplepart3|samplepart4>
%\fi
%
%\iffalse
%<*samplepart3>
%\fi
% Some text for part 3:
%    \begin{macrocode}
some text in part three
%    \end{macrocode}

%\iffalse
%</samplepart3>
%\fi
% Some text for part 4:
%\iffalse
%<*samplepart4>
%\fi
%    \begin{macrocode}
more text in part four
%    \end{macrocode}

%\iffalse
%</samplepart4>
%\fi
%
% %%%%%%%%%%%%%%%%%%%%%%%%%%%%%%%%%%%%%%
% \paragraph{Forwarding for a Complete Draft.}
%
% The following forwarding file |cdocsdrf.tex|
% compiles the main document in draft mode:
%\iffalse
%<*sampledraft>
%\fi
%    \begin{macrocode}
\def\version{draft}
\input{childdoc.def}
\childdocforward{cdocsamp}
%    \end{macrocode}

%\iffalse
%</sampledraft>
%\fi
%
% %%%%%%%%%%%%%%%%%%%%%%%%%%%%%%%%%%%%%%
% \paragraph{Forwarding for Final Version of the Chapters.}
%
% The following forwarding files |cdocsfn1.tex| and |cdocsfn2.tex|
% (with identical content)
% compile the final versions of the child documents
% |cdocsch1.tex| and |cdocsch2.tex|, respectively:
%\iffalse
%<*samplefinal>
%\fi
%    \begin{macrocode}
\def\version{final}
\input{childdoc.def}
\childdocforwardprefix[cdocsamp]{cdocsfn}{cdocsch}
%    \end{macrocode}

%\iffalse
%</samplefinal>
%\fi
%
% %%%%%%%%%%%%%%%%%%%%%%%%%%%%%%%%%%%%%%
% \paragraph{Command Line Processing.}
%
% The following three command lines generate the output files
% |cdocscld|, |cdocscl1| and |cdocscl2|
% which should be identical to
% |cdocsdrf|, |cdocsch1| and |cdocsfn2|, respectively:
% \begin{center}
% \begin{tabular}{l}
% |latex -jobname cdocscld \|\\
% |  "\def\version{draft}\input{childdoc.def}\childdocforward{cdocsamp}"|\\
% |latex -jobname cdocscl1 \|\\
% |  "\input{childdoc.def}\childdocforward[cdocsamp]{cdocsch1}"|\\
% |latex -jobname cdocscl2 \|\\
% |  "\def\version{final}\input{childdoc.def}\childdocforward{cdocsch2}"|
% \end{tabular}
% \end{center}
% Note that the trailing backslash on each first line
% merely continues the input to the second line
% (for convenient cut ant paste).
% Furthermore, the command |latex| can be replaced by any
% of its alternative versions such as |pdflatex|.
%
% %%%%%%%%%%%%%%%%%%%%%%%%%%%%%%%%%%%%%%%%%%%%%%%%%%%%%%%%%%%%%%%%%%%%%%%%%%%%%%
% %%%%%%%%%%%%%%%%%%%%%%%%%%%%%%%%%%%%%%%%%%%%%%%%%%%%%%%%%%%%%%%%%%%%%%%%%%%%%%
% \section{Implementation}
%\iffalse
%<*package>
%\fi
%
% This section describes the definitions file |childdoc.def|.

% The definitions cannot be loaded using |\usepackage| or |\RequirePackage|
% which has a mechanism to prevent loading a style file more than once.
% When loading the definitions by means of |\input|
% multiple instances have to be prevented manually:
%\iffalse
%This code needs to be before the `\ProvidesFile' directive
%which is defined at the beginning of this file.
%Therefore it is also placed there and commented out here.
%</package>
%<*discard>
%\fi
%    \begin{macrocode}
\ifdefined\childdocmain\endinput\fi
%    \end{macrocode}
%\iffalse
%</discard>
%<*package>
%\fi
%
% \macro{\ifchilddoc}
% \macro{\ifchilddocmanual}
% The conditional |\ifchilddoc| tells whether a
% child (true) or main (false) document is being compiled.
% The conditional |\ifchilddocmanual| tells whether
% the |\includeonly| mechanism is used (false) or
% the selection of child files must be performed manually (true).
% The definitions initialise to false:
%    \begin{macrocode}
\newif\ifchilddoc
\newif\ifchilddocmanual
%    \end{macrocode}

% \macro{\childdocname}
% \macro{\childdocjob}
% The macro |\childdocname| stores the name of the main document
% to be compiled. The macro |\childdocjob| stores the name of
% the document on which the \LaTeX{} compiler was originally invoked.
% The content of |\jobname| cannot be compared
% to filenames specified in the source due to different catcodes.
% The following code rescans |\jobname|, stores the result
% in |\childdocname| and saves a copy in |\childdocjob|:
%    \begin{macrocode}
\edef\childdocname{\scantokens\expandafter{\jobname\noexpand}}
\let\childdocjob\childdocname
%    \end{macrocode}

% \macro{\childdocdisable}
% The macro |\childdocdisable| prevents the main file
% from being processed more than once.
% At this stage, the main document command |\childdocmain|
% is assumed to be called once again where it should do nothing.
% Any subsequent call to it should prevent
% a secondary processing of the main document
% It overwrites the forwarding commands
% |\childdocof| and |\childdocforward|
% with empty macros to prevent further inclusions of the main document:
%    \begin{macrocode}
\newcommand{\childdocdisable}
{
  \renewcommand{\childdocmain}[1]{\renewcommand{\childdocmain}[1]{\endinput}}
  \renewcommand{\childdocof}[1]{}
  \renewcommand{\childdocby}[2][]{}
  \renewcommand{\childdocforward}[2][]{}
  \renewcommand{\childdocdisable}{}
}
%    \end{macrocode}

% \macro{\childdocmain}
% The macro |\childdocmain| is to be called at the top of the main file
% with nothing or the main filename (without extension) as argument.
% First, it breaks loops.
% If the argument is not empty and does not match |\childdocname|
% (which is set by the first inclusion of |childdoc.def|),
% |\ifchilddoc| is set to true, |\includeonly| is applied to the child file
% and |\jobname| is set to the main file
% (for proper handling of |.aux| files):
%    \begin{macrocode}
\newcommand{\childdocmain}[1]
{
  \childdocdisable\childdocmain{}
  \if?#1?\else
    \begingroup
      \def\childdoctmp{#1}
      \ifx\childdoctmp\childdocname
        \def\childdoctmp{}
      \else
        \def\childdoctmp
        {
          \childdoctrue
          \includeonly{\childdocname}
          \def\childdocjob{#1}
          \def\jobname{#1}
        }
      \fi
      \expandafter
    \endgroup
    \childdoctmp
  \fi
}
%    \end{macrocode}

% \macro{\childdocof}
% The command |\childdocof| redirects
% compilation to the main file |#1|.
%    \begin{macrocode}
\newcommand{\childdocof}[1]
{
  \childdocdisable
  \childdoctrue
  \includeonly{\childdocname}
  \def\jobname{#1}
  \def\childdocjob{#1}
  \input{#1}
}
%    \end{macrocode}

% \macro{\childdocby}
% The command |\childdocby| ....
%    \begin{macrocode}
\newcommand{\childdocby}[2][]
{
  \childdocdisable
  \childdoctrue
  \childdocmanualtrue
  \if?#1?\else
    \def\jobname{#2}
  \fi
  \def\childdocjob{#2}
  \input{#2}
  \endinput
}
%    \end{macrocode}

% \macro{\childdocforward}
% The command |\childdocforward| redirects
% compilation to the main file or
% (if the optional argument is given) a child file.
% Parameters are set as if the main file
% or a child file starting with |\childdocof| was compiled.
% Then compilation is handed over to the main file:
%    \begin{macrocode}
\newcommand{\childdocforward}[2][]
{
  \begingroup
    \if?#1?
      \def\childdoctmp
      {
        \def\childdocname{#2}
        \def\childdocjob{#2}
        \def\jobname{#2}
        \input{#2}
        \endinput
      }
    \else
      \def\childdoctmp
      {
        \childdocdisable
        \def\childdocname{#2}
        \childdoctrue
        \includeonly{#2}
        \def\childdocjob{#1}
        \def\jobname{#1}
        \input{#1}
        \endinput
      }
    \fi
    \expandafter
  \endgroup
  \childdoctmp
}
%    \end{macrocode}

% \macro{\childdocforwardprefix}
% The command |\childdocforwardprefix| redirects
% compilation to the main or a child file by means of a pattern.
% The prefix |#1| in the current filename is replaced by |#2|
% and the suffix of the current filename is kept
% (it is assumed that the filename does not contain the substring `|~~~|'
% which is used as a delimiter).
% Compilation is handed over to the new file by |\childdocforward|:
%    \begin{macrocode}
\newcommand{\childdocforwardprefix}[3][]
{
  \begingroup
    \def\childdocextract #2##1~~~{\def\childdoctmp{\childdocforward[#1]{#3##1}}}
    \expandafter\childdocextract\childdocname~~~
    \expandafter
  \endgroup
  \childdoctmp
}
%    \end{macrocode}

% \macro{\childdoc}
% The deprecated macro |\childdoc| is a legacy version of |\childdocmain|:
%    \begin{macrocode}
\newcommand{\childdoc}{\childdocmain}
%    \end{macrocode}

% \macro{\childdocredirect}
% The deprecated macro |\childdocredirect| is a legacy version
% of |\childdocforward| and |\childdocforwardprefix|:
%    \begin{macrocode}
\newcommand{\childdocredirect}[2][]
{
  \begingroup
    \if?#1?
      \def\childdoctmp{\childdocforward{#2}}
    \else
      \def\childdoctmp{\childdocforwardprefix{#1}{#2}}
    \fi
    \expandafter
  \endgroup
  \childdoctmp
}
%    \end{macrocode}

%\iffalse
%</package>
%\fi
%
\endinput

\childdocforward{cdocsamp}
%    \end{macrocode}

%\iffalse
%</sampledraft>
%\fi
%
% %%%%%%%%%%%%%%%%%%%%%%%%%%%%%%%%%%%%%%
% \paragraph{Forwarding for Final Version of the Chapters.}
%
% The following forwarding files |cdocsfn1.tex| and |cdocsfn2.tex|
% (with identical content)
% compile the final versions of the child documents
% |cdocsch1.tex| and |cdocsch2.tex|, respectively:
%\iffalse
%<*samplefinal>
%\fi
%    \begin{macrocode}
\def\version{final}
% \iffalse
%
% childdoc.dtx Copyright (C) 2017-2018 Niklas Beisert
%
% This work may be distributed and/or modified under the
% conditions of the LaTeX Project Public License, either version 1.3
% of this license or (at your option) any later version.
% The latest version of this license is in
%   http://www.latex-project.org/lppl.txt
% and version 1.3 or later is part of all distributions of LaTeX
% version 2005/12/01 or later.
%
% This work has the LPPL maintenance status `maintained'.
%
% The Current Maintainer of this work is Niklas Beisert.
%
% This work consists of the files childdoc.dtx and childdoc.ins
% and the derived files childdoc.def and cdocsamp.tex with
% cdocsch1.tex, cdocsch2.tex, cdocsdrf.tex, cdocsfn1.tex, cdocsfn2.tex.
%
%<package>\ifdefined\childdocmain\endinput\fi
%<package>\ProvidesFile{childdoc.def}[2018/12/30 v2.0 child document driver]
%<samplemain>\ProvidesFile{cdocsamp.tex}[2018/12/30 v2.0 sample for childdoc]
%<*driver>
%\ProvidesFile{childdoc.drv}[2018/12/30 v2.0 childdoc reference manual file]
\PassOptionsToClass{10pt,a4paper}{article}
\documentclass{ltxdoc}

\usepackage[margin=35mm]{geometry}
\usepackage{hyperref}
\usepackage{hyperxmp}
\usepackage[usenames]{color}

\hypersetup{colorlinks=true}
\hypersetup{pdfstartview=FitH}
\hypersetup{pdfpagemode=UseNone}
\hypersetup{pdfsource={}}
\hypersetup{pdflang={en-UK}}
\hypersetup{pdfcopyright={Copyright 2017-2018 Niklas Beisert.
  This work may be distributed and/or modified under the
  conditions of the LaTeX Project Public License, either version 1.3
  of this license or (at your option) any later version.}}
\hypersetup{pdflicenseurl={http://www.latex-project.org/lppl.txt}}
\hypersetup{pdfcontactaddress={ETH Zurich, ITP, HIT K,
  Wolfgang-Pauli-Strasse 27}}
\hypersetup{pdfcontactpostcode={8093}}
\hypersetup{pdfcontactcity={Zurich}}
\hypersetup{pdfcontactcountry={Switzerland}}
\hypersetup{pdfcontactemail={nbeisert@itp.phys.ethz.ch}}
\hypersetup{pdfcontacturl={http://people.phys.ethz.ch/\xmptilde nbeisert/}}

\newcommand{\secref}[1]{\hyperref[#1]{section \ref*{#1}}}

\parskip1ex
\parindent0pt
\let\olditemize\itemize
\def\itemize{\olditemize\parskip0pt}

\begin{document}

\title{The \textsf{childdoc} Package}
\hypersetup{pdftitle={The childdoc Package}}
\author{Niklas Beisert\\[2ex]
  Institut f\"ur Theoretische Physik\\
  Eidgen\"ossische Technische Hochschule Z\"urich\\
  Wolfgang-Pauli-Strasse 27, 8093 Z\"urich, Switzerland\\[1ex]
  \href{mailto:nbeisert@itp.phys.ethz.ch}
  {\texttt{nbeisert@itp.phys.ethz.ch}}}
\hypersetup{pdfauthor={Niklas Beisert}}
\hypersetup{pdfsubject={Manual for the LaTeX2e Package childdoc}}
\date{30 December 2018, \textsf{v2.0}}
\maketitle

\begin{abstract}\noindent
\textsf{childdoc} is a \LaTeXe{} package
that enables the direct compilation
of document sections included by |\include|
to individual files.
\end{abstract}

\begingroup
\parskip0ex
\tableofcontents
\endgroup

%%%%%%%%%%%%%%%%%%%%%%%%%%%%%%%%%%%%%%%%%%%%%%%%%%%%%%%%%%%%%%%%%%%%%%%%%%%%%%%%
%%%%%%%%%%%%%%%%%%%%%%%%%%%%%%%%%%%%%%%%%%%%%%%%%%%%%%%%%%%%%%%%%%%%%%%%%%%%%%%%
\section{Introduction}

\LaTeX{} provides a mechanism to structure a large document (such as a book)
into a main file and several child files (containing the chapters)
using the |\include| command.
This mechanism is beneficial for documents
which span hundreds of pages in order to
make the source file(s) more manageable.
Moreover, compilation can be restricted to
selected child files by means of the |\includeonly| command.
The latter feature can be used to reduce the compilation time while editing
(this was significantly more useful in the earlier days of \LaTeX{})
or to generate a smaller document which is easier to navigate.
Another application of |\includeonly| is to generate
documents consisting of selected parts of the complete document.

However, there are a few drawbacks of the plain |\include| mechanism:
\begin{itemize}
\item
The child files cannot be compiled on their own,
they can only be compiled via the main file.
A naive editing environment
(such as a text editor with an option
to have the current file processed by \LaTeX)
may require one to switch to the main file before compiling;
attempting to compile the child file produces errors.
\item
The main file must be modified (each time)
to adjust the |\includeonly| command
to the present needs. This easily leaves the main file in a messy state.
\item
The generated document will always carry the filename
of the main document. This is inconvenient if
several child files are to be compiled and
to be kept for distribution.
\end{itemize}

The present package provides a simple interface
to make child files individually compilable by \LaTeX{}.
Compiling a child file then has the same effect as compiling
the main file with an |\includeonly| command
to select the appropriate child.
Moreover the generated document will carry the name of the child
rather than the main file.
This resolves all three above issues.

This feature is meant to make the editing of books,
thesis documents and lecture notes somewhat more convenient.
However, the package can also be used efficiently for
composing a series of documents (such as exercise sheets)
which are typically distributed individually.
It then assists the author in generating the individual documents
(potentially in different versions)
as well as a document containing the collected series.
Another application is in developing style files
or other kinds of included material
where compilation of the style file could redirect
to a sample or test file.

%%%%%%%%%%%%%%%%%%%%%%%%%%%%%%%%%%%%%%%%%%%%%%%%%%%%%%%%%%%%%%%%%%%%%%%%%%%%%%%%
%%%%%%%%%%%%%%%%%%%%%%%%%%%%%%%%%%%%%%%%%%%%%%%%%%%%%%%%%%%%%%%%%%%%%%%%%%%%%%%%
\section{Usage}

First of all, the package \textsf{childdoc} is \emph{not} a standard
\LaTeXe{} |.sty| style file! Therefore it needs to be invoked in
a non-standard way.

%%%%%%%%%%%%%%%%%%%%%%%%%%%%%%%%%%%%%%%%%%%%%%%%%%%%%%%%%%%%%%%%%%%%%%%%%%%%%%%%
\subsection{Included Files}
\label{sec:include}

%%%%%%%%%%%%%%%%%%%%%%%%%%%%%%%%%%%%%%%%
\DescribeMacro{\childdocmain}
To use the package, add the commands
\begin{center}
\begin{tabular}{l}
|\input{childdoc.def}|\\
|\childdocmain{}|\\
\end{tabular}
\end{center}
at the very top of the main \LaTeX{} file,
in particular \emph{before} the |\documentclass| statement!
The argument of |\childdocmain| should be left empty
(but it must be present).

%%%%%%%%%%%%%%%%%%%%%%%%%%%%%%%%%%%%%%%%
\DescribeMacro{\childdocof}
Furthermore, add the commands
\begin{center}
\begin{tabular}{l}
|\input{childdoc.def}|\\
|\childdocof{|\textit{main}|}|\\
\end{tabular}
\end{center}
at the top of every child file \textit{child}
which is included by |\include{|\textit{child}|}|
from within the main file
(or at least for those files to be compiled individually).
The argument \textit{main} must be the filename of the main file.

There are a couple of
considerations in setting up the main and child documents:

%%%%%%%%%%%%%%%%%%%%%%%%%%%%%%%%%%%%%%%%
\paragraph{Restrictions.}

Please note the following restrictions:
\begin{itemize}
\item
|\childdocmain| must be called with one argument \textit{main}
to ensure compatibility with earlier version of the package.
It must either be empty (|\childdocmain{}|)
or precisely match the filename of the main file in which it is specified.
See \secref{sec:detection} for further information.
\item
The filename \textit{main} must be specified without the |.tex| extension.
\item
The filename \textit{main} is case sensitive
(even in case-insensitive file systems)
due to internal string comparison.
\item
The argument \textit{main} should be fully expanded, it cannot be a macro.
\item
Subdirectories and special characters should be avoided in filenames.
\item
The command |\childdocmain{|\textit{main}|}| must be followed by a whitespace.
It should not be followed immediately by another command
or by a comment mark `|%|'.
This is because the \TeX{} parser reads the token immediately following
the argument of |\childdocmain| and puts it
at the beginning of every child section;
however, a white\-space is ignored.
\end{itemize}

%%%%%%%%%%%%%%%%%%%%%%%%%%%%%%%%%%%%%%%%
\paragraph{Content of Main File.}

It is advisable to place all content in the child files included by |\include|.
Any output contained in the main file will appear in all child documents
unless suppressed manually;
it cannot be suppressed automatically by the |\includeonly| directive
and thus should normally be avoided.
A method to include some content in the main file
by means of conditional processing is described in \secref{sec:conditional}.

%%%%%%%%%%%%%%%%%%%%%%%%%%%%%%%%%%%%%%%%
\paragraph{Page Numbering.}

When only a part of the document is compiled,
the appropriate numbering of pages
(as well as other status parameters)
is determined from the |.aux| files.
The latter contain information from previous passes.
However this information needs to propagate through
all intermediate child documents.
Therefore the page numbering in child documents may well
be inconsistent until the complete document is compiled at least once.

A useful (if unconventional) way to always ensure a consistent
page numbering is to restart the numbering in each child document
and denote the pages by `\textit{child}|.|\textit{page}'
where \textit{child} represents the chapter/section number of the child file.
This can be achieved by the command
|\numberwithin{page}{|\textit{child}|}|
of the \textsf{amsmath} package
where \textit{child} can be |chapter| or |section|
depending on the chosen structuring.
Alternatively, one can modify the macro |\thepage| appropriately
and reset the counter |page| at the start of each child file.

%%%%%%%%%%%%%%%%%%%%%%%%%%%%%%%%%%%%%%%%%%%%%%%%%%%%%%%%%%%%%%%%%%%%%%%%%%%%%%%%
\subsection{Conditional Processing}
\label{sec:conditional}

The package provides a mechanism to compile different versions
of a document. To customise the versions further some conditional processing
can come in handy to distinguish which version is being compiled.
The package provides two macros to describe the compilation context:

%%%%%%%%%%%%%%%%%%%%%%%%%%%%%%%%%%%%%%%%
\DescribeMacro{\ifchilddoc}
The conditional |\ifchilddoc| distinguishes between the compilation of
child documents and the main document:
%
\begin{center}
|\ifchilddoc |\textit{child-code}| |[|\||else |\textit{main-code}]| \||fi|
\end{center}

%%%%%%%%%%%%%%%%%%%%%%%%%%%%%%%%%%%%%%%%
\DescribeMacro{\childdocname}
\DescribeMacro{\childdocjob}
The macro |\childdocname| contains the filename (without extension)
of the main or child file being processed.
Note that |\childdocjob| will always contain the name of the main file.

%%%%%%%%%%%%%%%%%%%%%%%%%%%%%%%%%%%%%%%%
\paragraph{Title Page.}

Conditional processing can be used to include a title or banner page
in the main document when proper precautions are taken.
Importantly, the code in the main file should ensure that the page counter
(as well as other status parameters which are stored in the |.aux| files)
takes the same value after the conditional processing.
Otherwise the page numbers may take divergent values
depending on which part is compiled.

For example, a title page could be declared by:
%
\begin{center}
\begin{tabular}{l}
|\ifchilddoc\||else|\\
|\addtocounter{page}{-1}|\\
\textit{code for title page}\\
|\newpage|\\
|\||fi|
\end{tabular}
\end{center}
%
A banner page for the child documents can be generated by:
%
\begin{center}
\begin{tabular}{l}
|\ifchilddoc|\\
|\addtocounter{page}{-1}|\\
\textit{code for banner page}\\
|\newpage|\\
|\||fi|
\end{tabular}
\end{center}
%
Here one could write a message such as:
\begin{center}
|This is the part \childdocname{} of \childdocjob{}.|
\end{center}

%%%%%%%%%%%%%%%%%%%%%%%%%%%%%%%%%%%%%%%%%%%%%%%%%%%%%%%%%%%%%%%%%%%%%%%%%%%%%%%%
\subsection{Flags}
\label{sec:flags}

The package makes it easy to generate different versions
of the main or child documents.
To this end compilation flags can be defined
and assigned different default values.
They will be particularly useful in conjunction
with the forwarding mechanism described in \secref{sec:forward}.

For example, it may be useful to have a flag |\version|
which can be set to |draft| or |final|.
The document source will contain some conditional code
depending on the value of |\version|.
Suppose further, the flag should default to |final| for the main file
and to |draft| for child files
which is a natural assignment for editing the document.
This is achieved by placing the following code
in the preamble of the main document
(below the |\childdocmain| directive):
%
\begin{center}
\begin{tabular}{l}
|\ifchilddoc|\\
|\providecommand{\version}{draft}|\\
|\||else|\\
|\providecommand{\version}{final}|\\
|\||fi|
\end{tabular}
\end{center}
%
The definition by |\providecommand| makes sure
that previous definitions are not overwritten.
Further statements |\providecommand{\version}{...}|
can thus be added before the above code to override it.

For the main file, one might add a line
(between |\childdocmain| and the above block)
%
\begin{center}
|%\ifchilddoc\||else\providecommand{\version}{draft}\||fi|
\end{center}
%
which can be uncommented to produce a draft version.
Likewise one can add a line to the very top of a child file
(above the |\childdocof{|\textit{main}|}| directive)
%
\begin{center}
|%\providecommand{\version}{final}|
\end{center}
%
which can be uncommented to produce the final version of this child document.

%%%%%%%%%%%%%%%%%%%%%%%%%%%%%%%%%%%%%%%%%%%%%%%%%%%%%%%%%%%%%%%%%%%%%%%%%%%%%%%%
\subsection{Forwarding}
\label{sec:forward}

Different versions of the main or child documents
using compilation flags as described in \secref{sec:flags}
can be (permanently) stored in different files
for convenient compilation, viewing and distribution.
To this end, the package defines a command
to pass on compilation to a different file:

%%%%%%%%%%%%%%%%%%%%%%%%%%%%%%%%%%%%%%%%
\DescribeMacro{\childdocforward}
The command |\childdocforward| redirects processing to
another source file:
%
\begin{center}
\begin{tabular}{l}
|\input{childdoc.def}|\\
|\childdocforward[|\textit{main}|]{|\textit{dest}|}|\\
\end{tabular}
\end{center}
%
The argument \textit{dest} is the destination file
(without extension).
It should be the main file or one of the child files.
Note that further \textsf{childdoc} directives
such as |\childdocof| and |\childdocforward|
in the indicated file will be processed in this form.
The optional argument \textit{main}
passes on directly to the main file \textit{main}
while pretending to compile the child \textit{dest}.
This form behaves as if \textit{dest}
issues |\childdocof{|\textit{main}|}| right away,
and no further \textsf{childdoc} directives will be processed.

%%%%%%%%%%%%%%%%%%%%%%%%%%%%%%%%%%%%%%%%
\DescribeMacro{\...prefix}
In the alternative form |\childdocforwardprefix|,
%
\begin{center}
\begin{tabular}{l}
|\input{childdoc.def}|\\
|\childdocforwardprefix[|\textit{main}|]{|\textit{prefix}|}{|\textit{dest}|}|
\end{tabular}
\end{center}
%
the destination file is determined by a pattern
depending on the current file:
To make this work, the current file must be called
`{\textit{prefix}\hspace{0.2em}\textit{suffix}}'
with \textit{prefix} matching precisely the argument.
Processing is then passed on to the file
`{\textit{dest}\hspace{0.2em}\textit{suffix}}'.
Surely, the same effect is achieved by
directly specifying the
argument `{\textit{dest}\hspace{0.2em}\textit{suffix}}'
in the first form.
However, that requires to set up a different file
for each child. With the alternative form of the command
all these files can have exactly the same content
which simplifies setting them up and maintaining them.

For example, the following file |draft.tex|
with a compilation flag |\version| as described in \secref{sec:flags}
compiles the main document as a draft:
%
\begin{center}
\begin{tabular}{l}
|\def\version{draft}|\\
|\input{childdoc.def}|\\
|\childdocforward{|\textit{main}|}|
\end{tabular}
\end{center}
%
Likewise, the following files |final|\textit{nn}|.tex|
compile the final version of the child document
|child|\textit{nn}|.tex|:
%
\begin{center}
\begin{tabular}{l}
|\def\version{final}|\\
|\input{childdoc.def}|\\
|\childdocforwardprefix{final}{child}|
\end{tabular}
\end{center}
%

Note that when several versions of a main file and/or of each child file
are to be generated, it may be convenient to set up a |Makefile| or
shell script to automatise the process.

%%%%%%%%%%%%%%%%%%%%%%%%%%%%%%%%%%%%%%%%%%%%%%%%%%%%%%%%%%%%%%%%%%%%%%%%%%%%%%%%
\subsection{Command Line Processing}
\label{sec:commandline}

The effect of redirection files can also be achieved by invoking
the \LaTeX{} compiler with a more elaborate command line.
Most conveniently this should be done as part
of a shell script or a |Makefile|.

When using \textsf{childdoc} in the main file, the following
command lines effectively perform a redirection
(note that depending on the shell being used,
backslashes may have to be doubled: `|\|' $\to$ `|\\|'):
%
\begin{center}
|... -jobname "|\textit{target}|" |\\|"|[\textit{flags}]%
|\input{childdoc.def}\childdocforward[|\textit{main}|]{|\textit{dest}|}"|
\end{center}
%
Here \textit{target} is the name of the output file,
\textit{main} is the name of the main file
and \textit{dest} is the name of the main or child file to be processed
(all filenames without extensions).
The optional argument \textit{main} can be omitted
if \textit{main} matches \textit{dest}.
Optionally, compilation \textit{flags} can be defined via |\def| commands.
This command line makes the \TeX{} engine believe
it is compiling the file \textit{target}
whose content is specified as the latter parameter.
The provided code then forwards the processing to
\textit{main} or \textit{dest} as described in \secref{sec:forward}.

%%%%%%%%%%%%%%%%%%%%%%%%%%%%%%%%%%%%%%%%%%%%%%%%%%%%%%%%%%%%%%%%%%%%%%%%%%%%%%%%
\subsection{Include by Input}
\label{sec:input}

Including child documents by |\include| has some restrictions by design.
Most notably, the content of a child document always occupies
its own set of pages; pages cannot be shared between child documents.
Usually, this behaviour makes perfect sense
because each child document contain an essential part of the document.
However, in some situations it may be desirable to compose
a document from a collection of parts
without having mandatory page breaks between then.
For this case, the package
provides a mechanism to include parts
by |\input| which can also be processed individually.
However, by construction this mechanism
requires manual handling of the content to be output.

%%%%%%%%%%%%%%%%%%%%%%%%%%%%%%%%%%%%%%%%
\DescribeMacro{\ifchilddocmanual}
The main file should be prepared as usual, see \secref{sec:include}.
However, the document body must make a distinction
between processing of an individual part and of the main document, e.g.:
%
\begin{center}
\begin{tabular}{l}
|\ifchilddocmanual|\\
|\input{\childdocname}|\\
|\||else|\\
\textit{document body with }|\input{|\textit{part}|}|\\
|\||fi|
\end{tabular}
\end{center}
%
The conditional |\ifchilddocmanual| is true whenever
a part to be included by |\input| is being compiled,
and the name of the part is stored in |\childdocname|.

%%%%%%%%%%%%%%%%%%%%%%%%%%%%%%%%%%%%%%%%
\DescribeMacro{\childdocby}
Each part to be included by |\input| should start with:
%
\begin{center}
\begin{tabular}{l}
|\input{childdoc.def}|\\
|\childdocby{|\textit{main}|}|\\
\end{tabular}
\end{center}
%
The directive |\childdocby| is similar to |\childdocof|
described in \secref{sec:include},
but the subsequent selection of content must be done manually.
To that end, both |\ifchilddoc| and |\ifchilddocmanual|
will be true upon processing of a part,
and the name of the part is stored in |\childdocname|.
Note that |\jobname| will be set to the filename of the current part
so that each part receives an individual |.aux| file
that does not interfere with the |.aux| file(s) of the main document.
This behaviour can be altered by the alternative form
|\childdocby[*]{|\textit{main}|}| (with a non-empty optional argument)
which uses the |.aux| file of the main document
by setting |\jobname| to \textit{main}.

%%%%%%%%%%%%%%%%%%%%%%%%%%%%%%%%%%%%%%%%%%%%%%%%%%%%%%%%%%%%%%%%%%%%%%%%%%%%%%%%
\subsection{Driver Development}
\label{sec:driver}

The \textsf{childdoc} mechanism can also be use for the development
of definition files such as \LaTeX{} styles or classes.
This case differs from the above setup with multiple parts
included by |\include| in that no |\includeonly| should be invoked.
This can be achieved by starting the include file
(before |\ProvidesPackage|) with:
%
\begin{center}
\begin{tabular}{l}
|\input{childdoc.def}|\\
|\childdocforward{|\textit{main}|}|\\
\end{tabular}
\end{center}
%
or alternatively with:
%
\begin{center}
\begin{tabular}{l}
|\input{childdoc.def}|\\
|\childdocby{|\textit{main}|}|\\
\end{tabular}
\end{center}
%
Both forms have slightly different effects as described above.
The main file is prepared as usual, see \secref{sec:include}.

%%%%%%%%%%%%%%%%%%%%%%%%%%%%%%%%%%%%%%%%%%%%%%%%%%%%%%%%%%%%%%%%%%%%%%%%%%%%%%%%
\subsection{Legacy Detection}
\label{sec:detection}

The directive |\childdocmain| in the main file can detect
whether the complete document or merely a child is to be compiled
even without using the directive |\childdocof|.
This method is deprecated because it is less robust
and there is no compelling reason to use it;
it is merely provided for backward compatibility
and it may be removed in future versions.

If the detection mechanism is to be used,
it is mandatory to correctly specify
the filename of the main file as the argument of |\childdocmain|:
%
\begin{center}
\begin{tabular}{l}
|\input{childdoc.def}|\\
|\childdocmain{|\textit{main}|}|\\
\end{tabular}
\end{center}
%
If |\jobname| does not match the argument \textit{main} of |\childdocmain|,
it is assumed that |\jobname| points to the child file to be compiled.
When using |\childdocmain| with the main file specified as argument,
it suffices to start a child file
with just |\input{|\textit{main}|}|
without loading of the package and using |\childdocof|.
If instead all processing is done
with the appropriate \textsf{childdoc} directives,
the argument of \textit{main} of |\childdocmain| can be empty.

An alternative version of the command line processing described
in \secref{sec:commandline} using the detection mechanism reads:
%
\begin{center}
|... -jobname "|\textit{target}|" "|[\textit{flags}]%
[|\def\jobname{|\textit{dest}|}|]|\input{|\textit{main}|}"|
\end{center}

%%%%%%%%%%%%%%%%%%%%%%%%%%%%%%%%%%%%%%%%%%%%%%%%%%%%%%%%%%%%%%%%%%%%%%%%%%%%%%%%
\subsection{Manual Code}
\label{sec:manual}

In case one cannot be certain whether the definitions file |childdoc.def|
is installed on the target \TeX{} distribution
and one prefers not to ship it,
it is conceivable to paste a few relevant commands into the sources.

To that end, drop all statements |\input{childdoc.def}|
and perform the replacements as outlined below.
Instead of |\childdocmain{|\textit{main}|}| add the following code
to the top of the main file:
%
\begin{center}
\begin{tabular}{l}
|\||ifdefined\childdocname\endinput\||fi\newif\ifchilddoc|\\
|\edef\childdocname{\scantokens\expandafter{\jobname\noexpand}}|\\
|\def\childdocmain{|\textit{main}|}\||ifx\childdocmain\childdocname\||else|\\
|\childdoctrue\includeonly{\childdocname}\let\jobname\childdocmain\||fi|\\
\end{tabular}
\end{center}
%
Instead of |\childdocof{|\textit{main}|}| just include the main file
at the top of each child file:
%
\begin{center}
|\input{|\textit{main}|}|
\end{center}
%
A simple redirection |\childdocforward{|\textit{dest}|}| is achieved by:
%
\begin{center}
|\def\jobname{|\textit{dest}|}\input{\jobname}|
\end{center}
%
The redirection with prefix
|\childdocforwardprefix[|\textit{prefix}|]{|\textit{dest}|}|
is accomplished by:
%
\begin{center}
\begin{tabular}{l}
|{\edef\jobname{\scantokens\expandafter{\jobname\noexpand}}|\\
|\def\redirectjob |\textit{prefix}|#1~~~{\gdef\jobname{|\textit{dest}|#1}}|\\
|\expandafter\redirectjob\jobname~~~}\input{\jobname}|
\end{tabular}
\end{center}

In an alternative approach,
child documents can be compiled by a specific command line
without additional code or specific definitions:
%
\begin{center}
|... -jobname "|\textit{target}|" "|[\textit{flags}]%
|\includeonly{|\textit{dest}|}\input{|\textit{main}|}"|
\end{center}
%

%%%%%%%%%%%%%%%%%%%%%%%%%%%%%%%%%%%%%%%%%%%%%%%%%%%%%%%%%%%%%%%%%%%%%%%%%%%%%%%%
%%%%%%%%%%%%%%%%%%%%%%%%%%%%%%%%%%%%%%%%%%%%%%%%%%%%%%%%%%%%%%%%%%%%%%%%%%%%%%%%
\section{Information}

%%%%%%%%%%%%%%%%%%%%%%%%%%%%%%%%%%%%%%%%%%%%%%%%%%%%%%%%%%%%%%%%%%%%%%%%%%%%%%%%
\subsection{Copyright}

Copyright \copyright{} 2017--2018 Niklas Beisert

This work may be distributed and/or modified under the
conditions of the \LaTeX{} Project Public License, either version 1.3
of this license or (at your option) any later version.
The latest version of this license is in
  \url{http://www.latex-project.org/lppl.txt}
and version 1.3 or later is part of all distributions of \LaTeX{}
version 2005/12/01 or later.

This work has the LPPL maintenance status `maintained'.

The Current Maintainer of this work is Niklas Beisert.

This work consists of the files |README.txt|, |childdoc.ins| and |childdoc.dtx|
as well as the derived files |childdoc.def|, |cdocsamp.tex|
with |cdocsch1.tex|, |cdocsch2.tex|, |cdocspt3.tex|, |cdocspt4.tex|,
|cdocsdrf.tex|, |cdocsfn1.tex|, |cdocsfn2.tex|
as well as |childdoc.pdf|.

%%%%%%%%%%%%%%%%%%%%%%%%%%%%%%%%%%%%%%%%%%%%%%%%%%%%%%%%%%%%%%%%%%%%%%%%%%%%%%%%
\subsection{Files and Installation}

The package consists of the files:
%
\begin{center}
\begin{tabular}{ll}
    |README.txt|   & readme file \\
    |childdoc.ins| & installation file \\
    |childdoc.dtx| & source file \\
    |childdoc.def| & definition file \\
    |cdocsamp.tex| & sample main file \\
    |cdocsch1.tex| & sample include file \\
    |cdocsch2.tex| & sample include file \\
    |cdocspt3.tex| & sample part file \\
    |cdocspt4.tex| & sample part file \\
    |cdocsdrf.tex| & sample redirection file \\
    |cdocsfn1.tex| & sample redirection file \\
    |cdocsfn2.tex| & sample redirection file \\
    |childdoc.pdf| & manual
\end{tabular}
\end{center}
%
The distribution consists of the files
|README.txt|, |childdoc.ins| and |childdoc.dtx|.
%
\begin{itemize}
\item
Run (pdf)\LaTeX{} on |childdoc.dtx|
to compile the manual |childdoc.pdf| (this file).
\item
Run \LaTeX{} on |childdoc.ins| to create the definitions file |childdoc.def|
and the sample |cdocsamp.tex| with include files
|cdocsch1.tex|, |cdocsch2.tex|, |cdocspt3.tex|, |cdocspt4.tex|,
|cdocsdrf.tex|, |cdocsfn1.tex|, |cdocsfn2.tex|.
Then copy the file |childdoc.def| to an appropriate directory of your \LaTeX{}
distribution, e.g.\ \textit{texmf-root}|/tex/latex/childdoc|.
\end{itemize}

%%%%%%%%%%%%%%%%%%%%%%%%%%%%%%%%%%%%%%%%%%%%%%%%%%%%%%%%%%%%%%%%%%%%%%%%%%%%%%%%
\subsection{Related CTAN Packages}

There are several other packages which offer a similar functionality:
%
\begin{itemize}
\item
The packages
\href{http://ctan.org/pkg/docmute}{\textsf{docmute}},
\href{http://ctan.org/pkg/includex}{\textsf{includex}} and
\href{http://ctan.org/pkg/standalone}{\textsf{standalone}}
provide commands to include only the document body of
a child file thus allowing both files to be compiled individually.
\item
The packages \href{http://ctan.org/pkg/subdocs}{\textsf{subdocs}}
and \href{http://ctan.org/pkg/subfiles}{\textsf{subfiles}}
provide structures in which the main and child documents can be
encapsulated and allowing them to be compiled individually.
The inclusion mechanism is different from the conventional |\include|.
\item
The package \href{http://ctan.org/pkg/combine}{\textsf{combine}}
is an elaborate solution to combine several documents into one.
\end{itemize}
%
See also the CTAN topic \href{http://ctan.org/topic/subdocs}{\textsf{subdocs}}
for further related packages.
The present package differs from the above solutions in that
a document structure constructed with the conventional |\include| mechanism
just needs two extra commands at the top of every file
such that all constituent files can be compiled individually.

%%%%%%%%%%%%%%%%%%%%%%%%%%%%%%%%%%%%%%%%%%%%%%%%%%%%%%%%%%%%%%%%%%%%%%%%%%%%%%%%
%\subsection{Feature Suggestions}
%
%The following is a list of features which may be useful for future
%versions of this package:
%%
%\begin{itemize}
%\item
%\ldots
%\end{itemize}

%%%%%%%%%%%%%%%%%%%%%%%%%%%%%%%%%%%%%%%%%%%%%%%%%%%%%%%%%%%%%%%%%%%%%%%%%%%%%%%%
\subsection{Revision History}

%%%%%%%%%%%%%%%%%%%%%%%%%%%%%%%%%%%%%%%%
\paragraph{v2.0:} 2018/12/30

\begin{itemize}
\item
immediate forward processing
\item
added |\childdocby| mechanism
\item
manual restructured
\end{itemize}

%%%%%%%%%%%%%%%%%%%%%%%%%%%%%%%%%%%%%%%%
\paragraph{v1.6:} 2018/01/17

\begin{itemize}
\item
application for development of include files
\item
corrections to manual
\end{itemize}

%%%%%%%%%%%%%%%%%%%%%%%%%%%%%%%%%%%%%%%%
\paragraph{v1.5:} 2017/05/21

\begin{itemize}
\item
more complete structuring introduced
\item
|\childdocof| introduced
\item
|\childdoc| renamed to |\childdocmain|
\item
|\childredirect| renamed to |\childdocforward| and |\childdocforwardprefix|
and functionality expanded
\end{itemize}

%%%%%%%%%%%%%%%%%%%%%%%%%%%%%%%%%%%%%%%%
\paragraph{v1.0:} 2017/04/27

\begin{itemize}
\item
manual and install package
\item
first version published on CTAN
\end{itemize}

%%%%%%%%%%%%%%%%%%%%%%%%%%%%%%%%%%%%%%%%
\paragraph{v0.6:} 2017/04/26

\begin{itemize}
\item
redirection mechanism added
\end{itemize}

%%%%%%%%%%%%%%%%%%%%%%%%%%%%%%%%%%%%%%%%
\paragraph{v0.5:} 2017/04/26

\begin{itemize}
\item
functionality in definition file
\end{itemize}


%%%%%%%%%%%%%%%%%%%%%%%%%%%%%%%%%%%%%%%%%%%%%%%%%%%%%%%%%%%%%%%%%%%%%%%%%%%%%%%%
%%%%%%%%%%%%%%%%%%%%%%%%%%%%%%%%%%%%%%%%%%%%%%%%%%%%%%%%%%%%%%%%%%%%%%%%%%%%%%%%
%%%%%%%%%%%%%%%%%%%%%%%%%%%%%%%%%%%%%%%%%%%%%%%%%%%%%%%%%%%%%%%%%%%%%%%%%%%%%%%%
\appendix

\settowidth\MacroIndent{\rmfamily\scriptsize 000\ }

 \DocInput{childdoc.dtx}

\end{document}
%</driver>
% \fi
%
% %%%%%%%%%%%%%%%%%%%%%%%%%%%%%%%%%%%%%%%%%%%%%%%%%%%%%%%%%%%%%%%%%%%%%%%%%%%%%%
% %%%%%%%%%%%%%%%%%%%%%%%%%%%%%%%%%%%%%%%%%%%%%%%%%%%%%%%%%%%%%%%%%%%%%%%%%%%%%%
% \section{Sample}
%\iffalse
%<*samplemain>
%\fi
%
% The following presents a sample document
% with two chapters, two parts, a title page,
% a compile flag as well as three forwarding files to set the flag.
% It consists of eight |.tex| files:
% \begin{center}
% \begin{tabular}{ll}
% |cdocsamp.tex|&main file\\
% |cdocsch1.tex|&include file for chapter 1\\
% |cdocsch2.tex|&include file for chapter 2\\
% |cdocspt3.tex|&include file for part 3\\
% |cdocspt4.tex|&include file for part 4\\
% |cdocsdrf.tex|&forwarding file for main file in draft mode\\
% |cdocsfi1.tex|&forwarding file for final version of chapter 1\\
% |cdocsfi2.tex|&forwarding file for final version of chapter 2\\
% \end{tabular}
% \end{center}
% Each of the eight files can be compiled directly by the \LaTeX{} compiler.
%
% %%%%%%%%%%%%%%%%%%%%%%%%%%%%%%%%%%%%%%
% \paragraph{Main File.}
%
% The main file is called |cdocsamp.tex|.
%
% Load the \textsf{childdoc} definitions and
% declare the filename for the main document:
%    \begin{macrocode}
\input{childdoc.def}
\childdocmain{}
%    \end{macrocode}

% Optional override for |\version| flag:
%    \begin{macrocode}
%%\ifchilddoc\else\providecommand{\version}{draft}\fi
%    \end{macrocode}

% Define the default values for the |\version| flag
% (|final| for the main file and |draft| for childs):
%    \begin{macrocode}
\ifchilddoc
\providecommand{\version}{draft}
\else
\providecommand{\version}{final}
\fi
%    \end{macrocode}

% Load the standard document class:
%    \begin{macrocode}
\documentclass[12pt]{article}
%    \end{macrocode}

% Start the document body:
%    \begin{macrocode}
\begin{document}
%    \end{macrocode}

% Declare a title page.
% Print title, part of document being processed and version flag:
%    \begin{macrocode}
\addtocounter{page}{-1}
\begin{center}
{\LARGE\bfseries{}childdoc example\par}
\vspace{1cm}
\ifchilddoc
\ifchilddocmanual part\else chapter\fi:
`\childdocname' of `\childdocjob'\par
\else
main document: `\childdocjob'\par
\fi
version: \version\par
\end{center}
\newpage
%    \end{macrocode}

% Manually include selected file,
% otherwise process as usual:
%    \begin{macrocode}
\ifchilddocmanual
\section*{part `\childdocname'}
\input{\childdocname}
\else
%    \end{macrocode}

% Include the two chapters:
%    \begin{macrocode}
\include{cdocsch1}
\include{cdocsch2}
%    \end{macrocode}

% Include the two parts unless only chapters should be displayed:
%    \begin{macrocode}
\ifchilddoc\else
\section{part three}
\input{cdocspt3}
\section{part four}
\input{cdocspt4}
\fi
%    \end{macrocode}

% Process as usual until here:
%    \begin{macrocode}
\fi
%    \end{macrocode}

% End of document body:
%    \begin{macrocode}
\end{document}
%    \end{macrocode}
%\iffalse
%</samplemain>
%\fi
%
% %%%%%%%%%%%%%%%%%%%%%%%%%%%%%%%%%%%%%%
% \paragraph{Chapter Include Files.}
%
% The include files are called |cdocsch1.tex| and |cdocsch2.tex|.
%
%\iffalse
%<*samplechap1|samplechap2>
%\fi

% Optional override for |\version| flag:
%    \begin{macrocode}
%%\providecommand{\version}{final}
%    \end{macrocode}

% Include the main document:
%    \begin{macrocode}
\input{childdoc.def}
\childdocof{cdocsamp}
%    \end{macrocode}

%\iffalse
%</samplechap1|samplechap2>
%\fi
%
%\iffalse
%<*samplechap1>
%\fi
% Some text for chapter 1:
%    \begin{macrocode}
\section{one}
some text in chapter one
%    \end{macrocode}

%\iffalse
%</samplechap1>
%\fi
% Some text for chapter 2:
%\iffalse
%<*samplechap2>
%\fi
%    \begin{macrocode}
\section{two}
more text in chapter two
%    \end{macrocode}

%\iffalse
%</samplechap2>
%\fi
%
% %%%%%%%%%%%%%%%%%%%%%%%%%%%%%%%%%%%%%%
% \paragraph{Part Include Files.}
%
% The include files are called |cdocspt3.tex| and |cdocspt4.tex|.
%
%\iffalse
%<*samplepart3|samplepart4>
%\fi

% Optional override for |\version| flag:
%    \begin{macrocode}
%%\providecommand{\version}{final}
%    \end{macrocode}

% Include the main document:
%    \begin{macrocode}
\input{childdoc.def}
\childdocby{cdocsamp}
%    \end{macrocode}

%\iffalse
%</samplepart3|samplepart4>
%\fi
%
%\iffalse
%<*samplepart3>
%\fi
% Some text for part 3:
%    \begin{macrocode}
some text in part three
%    \end{macrocode}

%\iffalse
%</samplepart3>
%\fi
% Some text for part 4:
%\iffalse
%<*samplepart4>
%\fi
%    \begin{macrocode}
more text in part four
%    \end{macrocode}

%\iffalse
%</samplepart4>
%\fi
%
% %%%%%%%%%%%%%%%%%%%%%%%%%%%%%%%%%%%%%%
% \paragraph{Forwarding for a Complete Draft.}
%
% The following forwarding file |cdocsdrf.tex|
% compiles the main document in draft mode:
%\iffalse
%<*sampledraft>
%\fi
%    \begin{macrocode}
\def\version{draft}
\input{childdoc.def}
\childdocforward{cdocsamp}
%    \end{macrocode}

%\iffalse
%</sampledraft>
%\fi
%
% %%%%%%%%%%%%%%%%%%%%%%%%%%%%%%%%%%%%%%
% \paragraph{Forwarding for Final Version of the Chapters.}
%
% The following forwarding files |cdocsfn1.tex| and |cdocsfn2.tex|
% (with identical content)
% compile the final versions of the child documents
% |cdocsch1.tex| and |cdocsch2.tex|, respectively:
%\iffalse
%<*samplefinal>
%\fi
%    \begin{macrocode}
\def\version{final}
\input{childdoc.def}
\childdocforwardprefix[cdocsamp]{cdocsfn}{cdocsch}
%    \end{macrocode}

%\iffalse
%</samplefinal>
%\fi
%
% %%%%%%%%%%%%%%%%%%%%%%%%%%%%%%%%%%%%%%
% \paragraph{Command Line Processing.}
%
% The following three command lines generate the output files
% |cdocscld|, |cdocscl1| and |cdocscl2|
% which should be identical to
% |cdocsdrf|, |cdocsch1| and |cdocsfn2|, respectively:
% \begin{center}
% \begin{tabular}{l}
% |latex -jobname cdocscld \|\\
% |  "\def\version{draft}\input{childdoc.def}\childdocforward{cdocsamp}"|\\
% |latex -jobname cdocscl1 \|\\
% |  "\input{childdoc.def}\childdocforward[cdocsamp]{cdocsch1}"|\\
% |latex -jobname cdocscl2 \|\\
% |  "\def\version{final}\input{childdoc.def}\childdocforward{cdocsch2}"|
% \end{tabular}
% \end{center}
% Note that the trailing backslash on each first line
% merely continues the input to the second line
% (for convenient cut ant paste).
% Furthermore, the command |latex| can be replaced by any
% of its alternative versions such as |pdflatex|.
%
% %%%%%%%%%%%%%%%%%%%%%%%%%%%%%%%%%%%%%%%%%%%%%%%%%%%%%%%%%%%%%%%%%%%%%%%%%%%%%%
% %%%%%%%%%%%%%%%%%%%%%%%%%%%%%%%%%%%%%%%%%%%%%%%%%%%%%%%%%%%%%%%%%%%%%%%%%%%%%%
% \section{Implementation}
%\iffalse
%<*package>
%\fi
%
% This section describes the definitions file |childdoc.def|.

% The definitions cannot be loaded using |\usepackage| or |\RequirePackage|
% which has a mechanism to prevent loading a style file more than once.
% When loading the definitions by means of |\input|
% multiple instances have to be prevented manually:
%\iffalse
%This code needs to be before the `\ProvidesFile' directive
%which is defined at the beginning of this file.
%Therefore it is also placed there and commented out here.
%</package>
%<*discard>
%\fi
%    \begin{macrocode}
\ifdefined\childdocmain\endinput\fi
%    \end{macrocode}
%\iffalse
%</discard>
%<*package>
%\fi
%
% \macro{\ifchilddoc}
% \macro{\ifchilddocmanual}
% The conditional |\ifchilddoc| tells whether a
% child (true) or main (false) document is being compiled.
% The conditional |\ifchilddocmanual| tells whether
% the |\includeonly| mechanism is used (false) or
% the selection of child files must be performed manually (true).
% The definitions initialise to false:
%    \begin{macrocode}
\newif\ifchilddoc
\newif\ifchilddocmanual
%    \end{macrocode}

% \macro{\childdocname}
% \macro{\childdocjob}
% The macro |\childdocname| stores the name of the main document
% to be compiled. The macro |\childdocjob| stores the name of
% the document on which the \LaTeX{} compiler was originally invoked.
% The content of |\jobname| cannot be compared
% to filenames specified in the source due to different catcodes.
% The following code rescans |\jobname|, stores the result
% in |\childdocname| and saves a copy in |\childdocjob|:
%    \begin{macrocode}
\edef\childdocname{\scantokens\expandafter{\jobname\noexpand}}
\let\childdocjob\childdocname
%    \end{macrocode}

% \macro{\childdocdisable}
% The macro |\childdocdisable| prevents the main file
% from being processed more than once.
% At this stage, the main document command |\childdocmain|
% is assumed to be called once again where it should do nothing.
% Any subsequent call to it should prevent
% a secondary processing of the main document
% It overwrites the forwarding commands
% |\childdocof| and |\childdocforward|
% with empty macros to prevent further inclusions of the main document:
%    \begin{macrocode}
\newcommand{\childdocdisable}
{
  \renewcommand{\childdocmain}[1]{\renewcommand{\childdocmain}[1]{\endinput}}
  \renewcommand{\childdocof}[1]{}
  \renewcommand{\childdocby}[2][]{}
  \renewcommand{\childdocforward}[2][]{}
  \renewcommand{\childdocdisable}{}
}
%    \end{macrocode}

% \macro{\childdocmain}
% The macro |\childdocmain| is to be called at the top of the main file
% with nothing or the main filename (without extension) as argument.
% First, it breaks loops.
% If the argument is not empty and does not match |\childdocname|
% (which is set by the first inclusion of |childdoc.def|),
% |\ifchilddoc| is set to true, |\includeonly| is applied to the child file
% and |\jobname| is set to the main file
% (for proper handling of |.aux| files):
%    \begin{macrocode}
\newcommand{\childdocmain}[1]
{
  \childdocdisable\childdocmain{}
  \if?#1?\else
    \begingroup
      \def\childdoctmp{#1}
      \ifx\childdoctmp\childdocname
        \def\childdoctmp{}
      \else
        \def\childdoctmp
        {
          \childdoctrue
          \includeonly{\childdocname}
          \def\childdocjob{#1}
          \def\jobname{#1}
        }
      \fi
      \expandafter
    \endgroup
    \childdoctmp
  \fi
}
%    \end{macrocode}

% \macro{\childdocof}
% The command |\childdocof| redirects
% compilation to the main file |#1|.
%    \begin{macrocode}
\newcommand{\childdocof}[1]
{
  \childdocdisable
  \childdoctrue
  \includeonly{\childdocname}
  \def\jobname{#1}
  \def\childdocjob{#1}
  \input{#1}
}
%    \end{macrocode}

% \macro{\childdocby}
% The command |\childdocby| ....
%    \begin{macrocode}
\newcommand{\childdocby}[2][]
{
  \childdocdisable
  \childdoctrue
  \childdocmanualtrue
  \if?#1?\else
    \def\jobname{#2}
  \fi
  \def\childdocjob{#2}
  \input{#2}
  \endinput
}
%    \end{macrocode}

% \macro{\childdocforward}
% The command |\childdocforward| redirects
% compilation to the main file or
% (if the optional argument is given) a child file.
% Parameters are set as if the main file
% or a child file starting with |\childdocof| was compiled.
% Then compilation is handed over to the main file:
%    \begin{macrocode}
\newcommand{\childdocforward}[2][]
{
  \begingroup
    \if?#1?
      \def\childdoctmp
      {
        \def\childdocname{#2}
        \def\childdocjob{#2}
        \def\jobname{#2}
        \input{#2}
        \endinput
      }
    \else
      \def\childdoctmp
      {
        \childdocdisable
        \def\childdocname{#2}
        \childdoctrue
        \includeonly{#2}
        \def\childdocjob{#1}
        \def\jobname{#1}
        \input{#1}
        \endinput
      }
    \fi
    \expandafter
  \endgroup
  \childdoctmp
}
%    \end{macrocode}

% \macro{\childdocforwardprefix}
% The command |\childdocforwardprefix| redirects
% compilation to the main or a child file by means of a pattern.
% The prefix |#1| in the current filename is replaced by |#2|
% and the suffix of the current filename is kept
% (it is assumed that the filename does not contain the substring `|~~~|'
% which is used as a delimiter).
% Compilation is handed over to the new file by |\childdocforward|:
%    \begin{macrocode}
\newcommand{\childdocforwardprefix}[3][]
{
  \begingroup
    \def\childdocextract #2##1~~~{\def\childdoctmp{\childdocforward[#1]{#3##1}}}
    \expandafter\childdocextract\childdocname~~~
    \expandafter
  \endgroup
  \childdoctmp
}
%    \end{macrocode}

% \macro{\childdoc}
% The deprecated macro |\childdoc| is a legacy version of |\childdocmain|:
%    \begin{macrocode}
\newcommand{\childdoc}{\childdocmain}
%    \end{macrocode}

% \macro{\childdocredirect}
% The deprecated macro |\childdocredirect| is a legacy version
% of |\childdocforward| and |\childdocforwardprefix|:
%    \begin{macrocode}
\newcommand{\childdocredirect}[2][]
{
  \begingroup
    \if?#1?
      \def\childdoctmp{\childdocforward{#2}}
    \else
      \def\childdoctmp{\childdocforwardprefix{#1}{#2}}
    \fi
    \expandafter
  \endgroup
  \childdoctmp
}
%    \end{macrocode}

%\iffalse
%</package>
%\fi
%
\endinput

\childdocforwardprefix[cdocsamp]{cdocsfn}{cdocsch}
%    \end{macrocode}

%\iffalse
%</samplefinal>
%\fi
%
% %%%%%%%%%%%%%%%%%%%%%%%%%%%%%%%%%%%%%%
% \paragraph{Command Line Processing.}
%
% The following three command lines generate the output files
% |cdocscld|, |cdocscl1| and |cdocscl2|
% which should be identical to
% |cdocsdrf|, |cdocsch1| and |cdocsfn2|, respectively:
% \begin{center}
% \begin{tabular}{l}
% |latex -jobname cdocscld \|\\
% |  "\def\version{draft}% \iffalse
%
% childdoc.dtx Copyright (C) 2017-2018 Niklas Beisert
%
% This work may be distributed and/or modified under the
% conditions of the LaTeX Project Public License, either version 1.3
% of this license or (at your option) any later version.
% The latest version of this license is in
%   http://www.latex-project.org/lppl.txt
% and version 1.3 or later is part of all distributions of LaTeX
% version 2005/12/01 or later.
%
% This work has the LPPL maintenance status `maintained'.
%
% The Current Maintainer of this work is Niklas Beisert.
%
% This work consists of the files childdoc.dtx and childdoc.ins
% and the derived files childdoc.def and cdocsamp.tex with
% cdocsch1.tex, cdocsch2.tex, cdocsdrf.tex, cdocsfn1.tex, cdocsfn2.tex.
%
%<package>\ifdefined\childdocmain\endinput\fi
%<package>\ProvidesFile{childdoc.def}[2018/12/30 v2.0 child document driver]
%<samplemain>\ProvidesFile{cdocsamp.tex}[2018/12/30 v2.0 sample for childdoc]
%<*driver>
%\ProvidesFile{childdoc.drv}[2018/12/30 v2.0 childdoc reference manual file]
\PassOptionsToClass{10pt,a4paper}{article}
\documentclass{ltxdoc}

\usepackage[margin=35mm]{geometry}
\usepackage{hyperref}
\usepackage{hyperxmp}
\usepackage[usenames]{color}

\hypersetup{colorlinks=true}
\hypersetup{pdfstartview=FitH}
\hypersetup{pdfpagemode=UseNone}
\hypersetup{pdfsource={}}
\hypersetup{pdflang={en-UK}}
\hypersetup{pdfcopyright={Copyright 2017-2018 Niklas Beisert.
  This work may be distributed and/or modified under the
  conditions of the LaTeX Project Public License, either version 1.3
  of this license or (at your option) any later version.}}
\hypersetup{pdflicenseurl={http://www.latex-project.org/lppl.txt}}
\hypersetup{pdfcontactaddress={ETH Zurich, ITP, HIT K,
  Wolfgang-Pauli-Strasse 27}}
\hypersetup{pdfcontactpostcode={8093}}
\hypersetup{pdfcontactcity={Zurich}}
\hypersetup{pdfcontactcountry={Switzerland}}
\hypersetup{pdfcontactemail={nbeisert@itp.phys.ethz.ch}}
\hypersetup{pdfcontacturl={http://people.phys.ethz.ch/\xmptilde nbeisert/}}

\newcommand{\secref}[1]{\hyperref[#1]{section \ref*{#1}}}

\parskip1ex
\parindent0pt
\let\olditemize\itemize
\def\itemize{\olditemize\parskip0pt}

\begin{document}

\title{The \textsf{childdoc} Package}
\hypersetup{pdftitle={The childdoc Package}}
\author{Niklas Beisert\\[2ex]
  Institut f\"ur Theoretische Physik\\
  Eidgen\"ossische Technische Hochschule Z\"urich\\
  Wolfgang-Pauli-Strasse 27, 8093 Z\"urich, Switzerland\\[1ex]
  \href{mailto:nbeisert@itp.phys.ethz.ch}
  {\texttt{nbeisert@itp.phys.ethz.ch}}}
\hypersetup{pdfauthor={Niklas Beisert}}
\hypersetup{pdfsubject={Manual for the LaTeX2e Package childdoc}}
\date{30 December 2018, \textsf{v2.0}}
\maketitle

\begin{abstract}\noindent
\textsf{childdoc} is a \LaTeXe{} package
that enables the direct compilation
of document sections included by |\include|
to individual files.
\end{abstract}

\begingroup
\parskip0ex
\tableofcontents
\endgroup

%%%%%%%%%%%%%%%%%%%%%%%%%%%%%%%%%%%%%%%%%%%%%%%%%%%%%%%%%%%%%%%%%%%%%%%%%%%%%%%%
%%%%%%%%%%%%%%%%%%%%%%%%%%%%%%%%%%%%%%%%%%%%%%%%%%%%%%%%%%%%%%%%%%%%%%%%%%%%%%%%
\section{Introduction}

\LaTeX{} provides a mechanism to structure a large document (such as a book)
into a main file and several child files (containing the chapters)
using the |\include| command.
This mechanism is beneficial for documents
which span hundreds of pages in order to
make the source file(s) more manageable.
Moreover, compilation can be restricted to
selected child files by means of the |\includeonly| command.
The latter feature can be used to reduce the compilation time while editing
(this was significantly more useful in the earlier days of \LaTeX{})
or to generate a smaller document which is easier to navigate.
Another application of |\includeonly| is to generate
documents consisting of selected parts of the complete document.

However, there are a few drawbacks of the plain |\include| mechanism:
\begin{itemize}
\item
The child files cannot be compiled on their own,
they can only be compiled via the main file.
A naive editing environment
(such as a text editor with an option
to have the current file processed by \LaTeX)
may require one to switch to the main file before compiling;
attempting to compile the child file produces errors.
\item
The main file must be modified (each time)
to adjust the |\includeonly| command
to the present needs. This easily leaves the main file in a messy state.
\item
The generated document will always carry the filename
of the main document. This is inconvenient if
several child files are to be compiled and
to be kept for distribution.
\end{itemize}

The present package provides a simple interface
to make child files individually compilable by \LaTeX{}.
Compiling a child file then has the same effect as compiling
the main file with an |\includeonly| command
to select the appropriate child.
Moreover the generated document will carry the name of the child
rather than the main file.
This resolves all three above issues.

This feature is meant to make the editing of books,
thesis documents and lecture notes somewhat more convenient.
However, the package can also be used efficiently for
composing a series of documents (such as exercise sheets)
which are typically distributed individually.
It then assists the author in generating the individual documents
(potentially in different versions)
as well as a document containing the collected series.
Another application is in developing style files
or other kinds of included material
where compilation of the style file could redirect
to a sample or test file.

%%%%%%%%%%%%%%%%%%%%%%%%%%%%%%%%%%%%%%%%%%%%%%%%%%%%%%%%%%%%%%%%%%%%%%%%%%%%%%%%
%%%%%%%%%%%%%%%%%%%%%%%%%%%%%%%%%%%%%%%%%%%%%%%%%%%%%%%%%%%%%%%%%%%%%%%%%%%%%%%%
\section{Usage}

First of all, the package \textsf{childdoc} is \emph{not} a standard
\LaTeXe{} |.sty| style file! Therefore it needs to be invoked in
a non-standard way.

%%%%%%%%%%%%%%%%%%%%%%%%%%%%%%%%%%%%%%%%%%%%%%%%%%%%%%%%%%%%%%%%%%%%%%%%%%%%%%%%
\subsection{Included Files}
\label{sec:include}

%%%%%%%%%%%%%%%%%%%%%%%%%%%%%%%%%%%%%%%%
\DescribeMacro{\childdocmain}
To use the package, add the commands
\begin{center}
\begin{tabular}{l}
|\input{childdoc.def}|\\
|\childdocmain{}|\\
\end{tabular}
\end{center}
at the very top of the main \LaTeX{} file,
in particular \emph{before} the |\documentclass| statement!
The argument of |\childdocmain| should be left empty
(but it must be present).

%%%%%%%%%%%%%%%%%%%%%%%%%%%%%%%%%%%%%%%%
\DescribeMacro{\childdocof}
Furthermore, add the commands
\begin{center}
\begin{tabular}{l}
|\input{childdoc.def}|\\
|\childdocof{|\textit{main}|}|\\
\end{tabular}
\end{center}
at the top of every child file \textit{child}
which is included by |\include{|\textit{child}|}|
from within the main file
(or at least for those files to be compiled individually).
The argument \textit{main} must be the filename of the main file.

There are a couple of
considerations in setting up the main and child documents:

%%%%%%%%%%%%%%%%%%%%%%%%%%%%%%%%%%%%%%%%
\paragraph{Restrictions.}

Please note the following restrictions:
\begin{itemize}
\item
|\childdocmain| must be called with one argument \textit{main}
to ensure compatibility with earlier version of the package.
It must either be empty (|\childdocmain{}|)
or precisely match the filename of the main file in which it is specified.
See \secref{sec:detection} for further information.
\item
The filename \textit{main} must be specified without the |.tex| extension.
\item
The filename \textit{main} is case sensitive
(even in case-insensitive file systems)
due to internal string comparison.
\item
The argument \textit{main} should be fully expanded, it cannot be a macro.
\item
Subdirectories and special characters should be avoided in filenames.
\item
The command |\childdocmain{|\textit{main}|}| must be followed by a whitespace.
It should not be followed immediately by another command
or by a comment mark `|%|'.
This is because the \TeX{} parser reads the token immediately following
the argument of |\childdocmain| and puts it
at the beginning of every child section;
however, a white\-space is ignored.
\end{itemize}

%%%%%%%%%%%%%%%%%%%%%%%%%%%%%%%%%%%%%%%%
\paragraph{Content of Main File.}

It is advisable to place all content in the child files included by |\include|.
Any output contained in the main file will appear in all child documents
unless suppressed manually;
it cannot be suppressed automatically by the |\includeonly| directive
and thus should normally be avoided.
A method to include some content in the main file
by means of conditional processing is described in \secref{sec:conditional}.

%%%%%%%%%%%%%%%%%%%%%%%%%%%%%%%%%%%%%%%%
\paragraph{Page Numbering.}

When only a part of the document is compiled,
the appropriate numbering of pages
(as well as other status parameters)
is determined from the |.aux| files.
The latter contain information from previous passes.
However this information needs to propagate through
all intermediate child documents.
Therefore the page numbering in child documents may well
be inconsistent until the complete document is compiled at least once.

A useful (if unconventional) way to always ensure a consistent
page numbering is to restart the numbering in each child document
and denote the pages by `\textit{child}|.|\textit{page}'
where \textit{child} represents the chapter/section number of the child file.
This can be achieved by the command
|\numberwithin{page}{|\textit{child}|}|
of the \textsf{amsmath} package
where \textit{child} can be |chapter| or |section|
depending on the chosen structuring.
Alternatively, one can modify the macro |\thepage| appropriately
and reset the counter |page| at the start of each child file.

%%%%%%%%%%%%%%%%%%%%%%%%%%%%%%%%%%%%%%%%%%%%%%%%%%%%%%%%%%%%%%%%%%%%%%%%%%%%%%%%
\subsection{Conditional Processing}
\label{sec:conditional}

The package provides a mechanism to compile different versions
of a document. To customise the versions further some conditional processing
can come in handy to distinguish which version is being compiled.
The package provides two macros to describe the compilation context:

%%%%%%%%%%%%%%%%%%%%%%%%%%%%%%%%%%%%%%%%
\DescribeMacro{\ifchilddoc}
The conditional |\ifchilddoc| distinguishes between the compilation of
child documents and the main document:
%
\begin{center}
|\ifchilddoc |\textit{child-code}| |[|\||else |\textit{main-code}]| \||fi|
\end{center}

%%%%%%%%%%%%%%%%%%%%%%%%%%%%%%%%%%%%%%%%
\DescribeMacro{\childdocname}
\DescribeMacro{\childdocjob}
The macro |\childdocname| contains the filename (without extension)
of the main or child file being processed.
Note that |\childdocjob| will always contain the name of the main file.

%%%%%%%%%%%%%%%%%%%%%%%%%%%%%%%%%%%%%%%%
\paragraph{Title Page.}

Conditional processing can be used to include a title or banner page
in the main document when proper precautions are taken.
Importantly, the code in the main file should ensure that the page counter
(as well as other status parameters which are stored in the |.aux| files)
takes the same value after the conditional processing.
Otherwise the page numbers may take divergent values
depending on which part is compiled.

For example, a title page could be declared by:
%
\begin{center}
\begin{tabular}{l}
|\ifchilddoc\||else|\\
|\addtocounter{page}{-1}|\\
\textit{code for title page}\\
|\newpage|\\
|\||fi|
\end{tabular}
\end{center}
%
A banner page for the child documents can be generated by:
%
\begin{center}
\begin{tabular}{l}
|\ifchilddoc|\\
|\addtocounter{page}{-1}|\\
\textit{code for banner page}\\
|\newpage|\\
|\||fi|
\end{tabular}
\end{center}
%
Here one could write a message such as:
\begin{center}
|This is the part \childdocname{} of \childdocjob{}.|
\end{center}

%%%%%%%%%%%%%%%%%%%%%%%%%%%%%%%%%%%%%%%%%%%%%%%%%%%%%%%%%%%%%%%%%%%%%%%%%%%%%%%%
\subsection{Flags}
\label{sec:flags}

The package makes it easy to generate different versions
of the main or child documents.
To this end compilation flags can be defined
and assigned different default values.
They will be particularly useful in conjunction
with the forwarding mechanism described in \secref{sec:forward}.

For example, it may be useful to have a flag |\version|
which can be set to |draft| or |final|.
The document source will contain some conditional code
depending on the value of |\version|.
Suppose further, the flag should default to |final| for the main file
and to |draft| for child files
which is a natural assignment for editing the document.
This is achieved by placing the following code
in the preamble of the main document
(below the |\childdocmain| directive):
%
\begin{center}
\begin{tabular}{l}
|\ifchilddoc|\\
|\providecommand{\version}{draft}|\\
|\||else|\\
|\providecommand{\version}{final}|\\
|\||fi|
\end{tabular}
\end{center}
%
The definition by |\providecommand| makes sure
that previous definitions are not overwritten.
Further statements |\providecommand{\version}{...}|
can thus be added before the above code to override it.

For the main file, one might add a line
(between |\childdocmain| and the above block)
%
\begin{center}
|%\ifchilddoc\||else\providecommand{\version}{draft}\||fi|
\end{center}
%
which can be uncommented to produce a draft version.
Likewise one can add a line to the very top of a child file
(above the |\childdocof{|\textit{main}|}| directive)
%
\begin{center}
|%\providecommand{\version}{final}|
\end{center}
%
which can be uncommented to produce the final version of this child document.

%%%%%%%%%%%%%%%%%%%%%%%%%%%%%%%%%%%%%%%%%%%%%%%%%%%%%%%%%%%%%%%%%%%%%%%%%%%%%%%%
\subsection{Forwarding}
\label{sec:forward}

Different versions of the main or child documents
using compilation flags as described in \secref{sec:flags}
can be (permanently) stored in different files
for convenient compilation, viewing and distribution.
To this end, the package defines a command
to pass on compilation to a different file:

%%%%%%%%%%%%%%%%%%%%%%%%%%%%%%%%%%%%%%%%
\DescribeMacro{\childdocforward}
The command |\childdocforward| redirects processing to
another source file:
%
\begin{center}
\begin{tabular}{l}
|\input{childdoc.def}|\\
|\childdocforward[|\textit{main}|]{|\textit{dest}|}|\\
\end{tabular}
\end{center}
%
The argument \textit{dest} is the destination file
(without extension).
It should be the main file or one of the child files.
Note that further \textsf{childdoc} directives
such as |\childdocof| and |\childdocforward|
in the indicated file will be processed in this form.
The optional argument \textit{main}
passes on directly to the main file \textit{main}
while pretending to compile the child \textit{dest}.
This form behaves as if \textit{dest}
issues |\childdocof{|\textit{main}|}| right away,
and no further \textsf{childdoc} directives will be processed.

%%%%%%%%%%%%%%%%%%%%%%%%%%%%%%%%%%%%%%%%
\DescribeMacro{\...prefix}
In the alternative form |\childdocforwardprefix|,
%
\begin{center}
\begin{tabular}{l}
|\input{childdoc.def}|\\
|\childdocforwardprefix[|\textit{main}|]{|\textit{prefix}|}{|\textit{dest}|}|
\end{tabular}
\end{center}
%
the destination file is determined by a pattern
depending on the current file:
To make this work, the current file must be called
`{\textit{prefix}\hspace{0.2em}\textit{suffix}}'
with \textit{prefix} matching precisely the argument.
Processing is then passed on to the file
`{\textit{dest}\hspace{0.2em}\textit{suffix}}'.
Surely, the same effect is achieved by
directly specifying the
argument `{\textit{dest}\hspace{0.2em}\textit{suffix}}'
in the first form.
However, that requires to set up a different file
for each child. With the alternative form of the command
all these files can have exactly the same content
which simplifies setting them up and maintaining them.

For example, the following file |draft.tex|
with a compilation flag |\version| as described in \secref{sec:flags}
compiles the main document as a draft:
%
\begin{center}
\begin{tabular}{l}
|\def\version{draft}|\\
|\input{childdoc.def}|\\
|\childdocforward{|\textit{main}|}|
\end{tabular}
\end{center}
%
Likewise, the following files |final|\textit{nn}|.tex|
compile the final version of the child document
|child|\textit{nn}|.tex|:
%
\begin{center}
\begin{tabular}{l}
|\def\version{final}|\\
|\input{childdoc.def}|\\
|\childdocforwardprefix{final}{child}|
\end{tabular}
\end{center}
%

Note that when several versions of a main file and/or of each child file
are to be generated, it may be convenient to set up a |Makefile| or
shell script to automatise the process.

%%%%%%%%%%%%%%%%%%%%%%%%%%%%%%%%%%%%%%%%%%%%%%%%%%%%%%%%%%%%%%%%%%%%%%%%%%%%%%%%
\subsection{Command Line Processing}
\label{sec:commandline}

The effect of redirection files can also be achieved by invoking
the \LaTeX{} compiler with a more elaborate command line.
Most conveniently this should be done as part
of a shell script or a |Makefile|.

When using \textsf{childdoc} in the main file, the following
command lines effectively perform a redirection
(note that depending on the shell being used,
backslashes may have to be doubled: `|\|' $\to$ `|\\|'):
%
\begin{center}
|... -jobname "|\textit{target}|" |\\|"|[\textit{flags}]%
|\input{childdoc.def}\childdocforward[|\textit{main}|]{|\textit{dest}|}"|
\end{center}
%
Here \textit{target} is the name of the output file,
\textit{main} is the name of the main file
and \textit{dest} is the name of the main or child file to be processed
(all filenames without extensions).
The optional argument \textit{main} can be omitted
if \textit{main} matches \textit{dest}.
Optionally, compilation \textit{flags} can be defined via |\def| commands.
This command line makes the \TeX{} engine believe
it is compiling the file \textit{target}
whose content is specified as the latter parameter.
The provided code then forwards the processing to
\textit{main} or \textit{dest} as described in \secref{sec:forward}.

%%%%%%%%%%%%%%%%%%%%%%%%%%%%%%%%%%%%%%%%%%%%%%%%%%%%%%%%%%%%%%%%%%%%%%%%%%%%%%%%
\subsection{Include by Input}
\label{sec:input}

Including child documents by |\include| has some restrictions by design.
Most notably, the content of a child document always occupies
its own set of pages; pages cannot be shared between child documents.
Usually, this behaviour makes perfect sense
because each child document contain an essential part of the document.
However, in some situations it may be desirable to compose
a document from a collection of parts
without having mandatory page breaks between then.
For this case, the package
provides a mechanism to include parts
by |\input| which can also be processed individually.
However, by construction this mechanism
requires manual handling of the content to be output.

%%%%%%%%%%%%%%%%%%%%%%%%%%%%%%%%%%%%%%%%
\DescribeMacro{\ifchilddocmanual}
The main file should be prepared as usual, see \secref{sec:include}.
However, the document body must make a distinction
between processing of an individual part and of the main document, e.g.:
%
\begin{center}
\begin{tabular}{l}
|\ifchilddocmanual|\\
|\input{\childdocname}|\\
|\||else|\\
\textit{document body with }|\input{|\textit{part}|}|\\
|\||fi|
\end{tabular}
\end{center}
%
The conditional |\ifchilddocmanual| is true whenever
a part to be included by |\input| is being compiled,
and the name of the part is stored in |\childdocname|.

%%%%%%%%%%%%%%%%%%%%%%%%%%%%%%%%%%%%%%%%
\DescribeMacro{\childdocby}
Each part to be included by |\input| should start with:
%
\begin{center}
\begin{tabular}{l}
|\input{childdoc.def}|\\
|\childdocby{|\textit{main}|}|\\
\end{tabular}
\end{center}
%
The directive |\childdocby| is similar to |\childdocof|
described in \secref{sec:include},
but the subsequent selection of content must be done manually.
To that end, both |\ifchilddoc| and |\ifchilddocmanual|
will be true upon processing of a part,
and the name of the part is stored in |\childdocname|.
Note that |\jobname| will be set to the filename of the current part
so that each part receives an individual |.aux| file
that does not interfere with the |.aux| file(s) of the main document.
This behaviour can be altered by the alternative form
|\childdocby[*]{|\textit{main}|}| (with a non-empty optional argument)
which uses the |.aux| file of the main document
by setting |\jobname| to \textit{main}.

%%%%%%%%%%%%%%%%%%%%%%%%%%%%%%%%%%%%%%%%%%%%%%%%%%%%%%%%%%%%%%%%%%%%%%%%%%%%%%%%
\subsection{Driver Development}
\label{sec:driver}

The \textsf{childdoc} mechanism can also be use for the development
of definition files such as \LaTeX{} styles or classes.
This case differs from the above setup with multiple parts
included by |\include| in that no |\includeonly| should be invoked.
This can be achieved by starting the include file
(before |\ProvidesPackage|) with:
%
\begin{center}
\begin{tabular}{l}
|\input{childdoc.def}|\\
|\childdocforward{|\textit{main}|}|\\
\end{tabular}
\end{center}
%
or alternatively with:
%
\begin{center}
\begin{tabular}{l}
|\input{childdoc.def}|\\
|\childdocby{|\textit{main}|}|\\
\end{tabular}
\end{center}
%
Both forms have slightly different effects as described above.
The main file is prepared as usual, see \secref{sec:include}.

%%%%%%%%%%%%%%%%%%%%%%%%%%%%%%%%%%%%%%%%%%%%%%%%%%%%%%%%%%%%%%%%%%%%%%%%%%%%%%%%
\subsection{Legacy Detection}
\label{sec:detection}

The directive |\childdocmain| in the main file can detect
whether the complete document or merely a child is to be compiled
even without using the directive |\childdocof|.
This method is deprecated because it is less robust
and there is no compelling reason to use it;
it is merely provided for backward compatibility
and it may be removed in future versions.

If the detection mechanism is to be used,
it is mandatory to correctly specify
the filename of the main file as the argument of |\childdocmain|:
%
\begin{center}
\begin{tabular}{l}
|\input{childdoc.def}|\\
|\childdocmain{|\textit{main}|}|\\
\end{tabular}
\end{center}
%
If |\jobname| does not match the argument \textit{main} of |\childdocmain|,
it is assumed that |\jobname| points to the child file to be compiled.
When using |\childdocmain| with the main file specified as argument,
it suffices to start a child file
with just |\input{|\textit{main}|}|
without loading of the package and using |\childdocof|.
If instead all processing is done
with the appropriate \textsf{childdoc} directives,
the argument of \textit{main} of |\childdocmain| can be empty.

An alternative version of the command line processing described
in \secref{sec:commandline} using the detection mechanism reads:
%
\begin{center}
|... -jobname "|\textit{target}|" "|[\textit{flags}]%
[|\def\jobname{|\textit{dest}|}|]|\input{|\textit{main}|}"|
\end{center}

%%%%%%%%%%%%%%%%%%%%%%%%%%%%%%%%%%%%%%%%%%%%%%%%%%%%%%%%%%%%%%%%%%%%%%%%%%%%%%%%
\subsection{Manual Code}
\label{sec:manual}

In case one cannot be certain whether the definitions file |childdoc.def|
is installed on the target \TeX{} distribution
and one prefers not to ship it,
it is conceivable to paste a few relevant commands into the sources.

To that end, drop all statements |\input{childdoc.def}|
and perform the replacements as outlined below.
Instead of |\childdocmain{|\textit{main}|}| add the following code
to the top of the main file:
%
\begin{center}
\begin{tabular}{l}
|\||ifdefined\childdocname\endinput\||fi\newif\ifchilddoc|\\
|\edef\childdocname{\scantokens\expandafter{\jobname\noexpand}}|\\
|\def\childdocmain{|\textit{main}|}\||ifx\childdocmain\childdocname\||else|\\
|\childdoctrue\includeonly{\childdocname}\let\jobname\childdocmain\||fi|\\
\end{tabular}
\end{center}
%
Instead of |\childdocof{|\textit{main}|}| just include the main file
at the top of each child file:
%
\begin{center}
|\input{|\textit{main}|}|
\end{center}
%
A simple redirection |\childdocforward{|\textit{dest}|}| is achieved by:
%
\begin{center}
|\def\jobname{|\textit{dest}|}\input{\jobname}|
\end{center}
%
The redirection with prefix
|\childdocforwardprefix[|\textit{prefix}|]{|\textit{dest}|}|
is accomplished by:
%
\begin{center}
\begin{tabular}{l}
|{\edef\jobname{\scantokens\expandafter{\jobname\noexpand}}|\\
|\def\redirectjob |\textit{prefix}|#1~~~{\gdef\jobname{|\textit{dest}|#1}}|\\
|\expandafter\redirectjob\jobname~~~}\input{\jobname}|
\end{tabular}
\end{center}

In an alternative approach,
child documents can be compiled by a specific command line
without additional code or specific definitions:
%
\begin{center}
|... -jobname "|\textit{target}|" "|[\textit{flags}]%
|\includeonly{|\textit{dest}|}\input{|\textit{main}|}"|
\end{center}
%

%%%%%%%%%%%%%%%%%%%%%%%%%%%%%%%%%%%%%%%%%%%%%%%%%%%%%%%%%%%%%%%%%%%%%%%%%%%%%%%%
%%%%%%%%%%%%%%%%%%%%%%%%%%%%%%%%%%%%%%%%%%%%%%%%%%%%%%%%%%%%%%%%%%%%%%%%%%%%%%%%
\section{Information}

%%%%%%%%%%%%%%%%%%%%%%%%%%%%%%%%%%%%%%%%%%%%%%%%%%%%%%%%%%%%%%%%%%%%%%%%%%%%%%%%
\subsection{Copyright}

Copyright \copyright{} 2017--2018 Niklas Beisert

This work may be distributed and/or modified under the
conditions of the \LaTeX{} Project Public License, either version 1.3
of this license or (at your option) any later version.
The latest version of this license is in
  \url{http://www.latex-project.org/lppl.txt}
and version 1.3 or later is part of all distributions of \LaTeX{}
version 2005/12/01 or later.

This work has the LPPL maintenance status `maintained'.

The Current Maintainer of this work is Niklas Beisert.

This work consists of the files |README.txt|, |childdoc.ins| and |childdoc.dtx|
as well as the derived files |childdoc.def|, |cdocsamp.tex|
with |cdocsch1.tex|, |cdocsch2.tex|, |cdocspt3.tex|, |cdocspt4.tex|,
|cdocsdrf.tex|, |cdocsfn1.tex|, |cdocsfn2.tex|
as well as |childdoc.pdf|.

%%%%%%%%%%%%%%%%%%%%%%%%%%%%%%%%%%%%%%%%%%%%%%%%%%%%%%%%%%%%%%%%%%%%%%%%%%%%%%%%
\subsection{Files and Installation}

The package consists of the files:
%
\begin{center}
\begin{tabular}{ll}
    |README.txt|   & readme file \\
    |childdoc.ins| & installation file \\
    |childdoc.dtx| & source file \\
    |childdoc.def| & definition file \\
    |cdocsamp.tex| & sample main file \\
    |cdocsch1.tex| & sample include file \\
    |cdocsch2.tex| & sample include file \\
    |cdocspt3.tex| & sample part file \\
    |cdocspt4.tex| & sample part file \\
    |cdocsdrf.tex| & sample redirection file \\
    |cdocsfn1.tex| & sample redirection file \\
    |cdocsfn2.tex| & sample redirection file \\
    |childdoc.pdf| & manual
\end{tabular}
\end{center}
%
The distribution consists of the files
|README.txt|, |childdoc.ins| and |childdoc.dtx|.
%
\begin{itemize}
\item
Run (pdf)\LaTeX{} on |childdoc.dtx|
to compile the manual |childdoc.pdf| (this file).
\item
Run \LaTeX{} on |childdoc.ins| to create the definitions file |childdoc.def|
and the sample |cdocsamp.tex| with include files
|cdocsch1.tex|, |cdocsch2.tex|, |cdocspt3.tex|, |cdocspt4.tex|,
|cdocsdrf.tex|, |cdocsfn1.tex|, |cdocsfn2.tex|.
Then copy the file |childdoc.def| to an appropriate directory of your \LaTeX{}
distribution, e.g.\ \textit{texmf-root}|/tex/latex/childdoc|.
\end{itemize}

%%%%%%%%%%%%%%%%%%%%%%%%%%%%%%%%%%%%%%%%%%%%%%%%%%%%%%%%%%%%%%%%%%%%%%%%%%%%%%%%
\subsection{Related CTAN Packages}

There are several other packages which offer a similar functionality:
%
\begin{itemize}
\item
The packages
\href{http://ctan.org/pkg/docmute}{\textsf{docmute}},
\href{http://ctan.org/pkg/includex}{\textsf{includex}} and
\href{http://ctan.org/pkg/standalone}{\textsf{standalone}}
provide commands to include only the document body of
a child file thus allowing both files to be compiled individually.
\item
The packages \href{http://ctan.org/pkg/subdocs}{\textsf{subdocs}}
and \href{http://ctan.org/pkg/subfiles}{\textsf{subfiles}}
provide structures in which the main and child documents can be
encapsulated and allowing them to be compiled individually.
The inclusion mechanism is different from the conventional |\include|.
\item
The package \href{http://ctan.org/pkg/combine}{\textsf{combine}}
is an elaborate solution to combine several documents into one.
\end{itemize}
%
See also the CTAN topic \href{http://ctan.org/topic/subdocs}{\textsf{subdocs}}
for further related packages.
The present package differs from the above solutions in that
a document structure constructed with the conventional |\include| mechanism
just needs two extra commands at the top of every file
such that all constituent files can be compiled individually.

%%%%%%%%%%%%%%%%%%%%%%%%%%%%%%%%%%%%%%%%%%%%%%%%%%%%%%%%%%%%%%%%%%%%%%%%%%%%%%%%
%\subsection{Feature Suggestions}
%
%The following is a list of features which may be useful for future
%versions of this package:
%%
%\begin{itemize}
%\item
%\ldots
%\end{itemize}

%%%%%%%%%%%%%%%%%%%%%%%%%%%%%%%%%%%%%%%%%%%%%%%%%%%%%%%%%%%%%%%%%%%%%%%%%%%%%%%%
\subsection{Revision History}

%%%%%%%%%%%%%%%%%%%%%%%%%%%%%%%%%%%%%%%%
\paragraph{v2.0:} 2018/12/30

\begin{itemize}
\item
immediate forward processing
\item
added |\childdocby| mechanism
\item
manual restructured
\end{itemize}

%%%%%%%%%%%%%%%%%%%%%%%%%%%%%%%%%%%%%%%%
\paragraph{v1.6:} 2018/01/17

\begin{itemize}
\item
application for development of include files
\item
corrections to manual
\end{itemize}

%%%%%%%%%%%%%%%%%%%%%%%%%%%%%%%%%%%%%%%%
\paragraph{v1.5:} 2017/05/21

\begin{itemize}
\item
more complete structuring introduced
\item
|\childdocof| introduced
\item
|\childdoc| renamed to |\childdocmain|
\item
|\childredirect| renamed to |\childdocforward| and |\childdocforwardprefix|
and functionality expanded
\end{itemize}

%%%%%%%%%%%%%%%%%%%%%%%%%%%%%%%%%%%%%%%%
\paragraph{v1.0:} 2017/04/27

\begin{itemize}
\item
manual and install package
\item
first version published on CTAN
\end{itemize}

%%%%%%%%%%%%%%%%%%%%%%%%%%%%%%%%%%%%%%%%
\paragraph{v0.6:} 2017/04/26

\begin{itemize}
\item
redirection mechanism added
\end{itemize}

%%%%%%%%%%%%%%%%%%%%%%%%%%%%%%%%%%%%%%%%
\paragraph{v0.5:} 2017/04/26

\begin{itemize}
\item
functionality in definition file
\end{itemize}


%%%%%%%%%%%%%%%%%%%%%%%%%%%%%%%%%%%%%%%%%%%%%%%%%%%%%%%%%%%%%%%%%%%%%%%%%%%%%%%%
%%%%%%%%%%%%%%%%%%%%%%%%%%%%%%%%%%%%%%%%%%%%%%%%%%%%%%%%%%%%%%%%%%%%%%%%%%%%%%%%
%%%%%%%%%%%%%%%%%%%%%%%%%%%%%%%%%%%%%%%%%%%%%%%%%%%%%%%%%%%%%%%%%%%%%%%%%%%%%%%%
\appendix

\settowidth\MacroIndent{\rmfamily\scriptsize 000\ }

 \DocInput{childdoc.dtx}

\end{document}
%</driver>
% \fi
%
% %%%%%%%%%%%%%%%%%%%%%%%%%%%%%%%%%%%%%%%%%%%%%%%%%%%%%%%%%%%%%%%%%%%%%%%%%%%%%%
% %%%%%%%%%%%%%%%%%%%%%%%%%%%%%%%%%%%%%%%%%%%%%%%%%%%%%%%%%%%%%%%%%%%%%%%%%%%%%%
% \section{Sample}
%\iffalse
%<*samplemain>
%\fi
%
% The following presents a sample document
% with two chapters, two parts, a title page,
% a compile flag as well as three forwarding files to set the flag.
% It consists of eight |.tex| files:
% \begin{center}
% \begin{tabular}{ll}
% |cdocsamp.tex|&main file\\
% |cdocsch1.tex|&include file for chapter 1\\
% |cdocsch2.tex|&include file for chapter 2\\
% |cdocspt3.tex|&include file for part 3\\
% |cdocspt4.tex|&include file for part 4\\
% |cdocsdrf.tex|&forwarding file for main file in draft mode\\
% |cdocsfi1.tex|&forwarding file for final version of chapter 1\\
% |cdocsfi2.tex|&forwarding file for final version of chapter 2\\
% \end{tabular}
% \end{center}
% Each of the eight files can be compiled directly by the \LaTeX{} compiler.
%
% %%%%%%%%%%%%%%%%%%%%%%%%%%%%%%%%%%%%%%
% \paragraph{Main File.}
%
% The main file is called |cdocsamp.tex|.
%
% Load the \textsf{childdoc} definitions and
% declare the filename for the main document:
%    \begin{macrocode}
\input{childdoc.def}
\childdocmain{}
%    \end{macrocode}

% Optional override for |\version| flag:
%    \begin{macrocode}
%%\ifchilddoc\else\providecommand{\version}{draft}\fi
%    \end{macrocode}

% Define the default values for the |\version| flag
% (|final| for the main file and |draft| for childs):
%    \begin{macrocode}
\ifchilddoc
\providecommand{\version}{draft}
\else
\providecommand{\version}{final}
\fi
%    \end{macrocode}

% Load the standard document class:
%    \begin{macrocode}
\documentclass[12pt]{article}
%    \end{macrocode}

% Start the document body:
%    \begin{macrocode}
\begin{document}
%    \end{macrocode}

% Declare a title page.
% Print title, part of document being processed and version flag:
%    \begin{macrocode}
\addtocounter{page}{-1}
\begin{center}
{\LARGE\bfseries{}childdoc example\par}
\vspace{1cm}
\ifchilddoc
\ifchilddocmanual part\else chapter\fi:
`\childdocname' of `\childdocjob'\par
\else
main document: `\childdocjob'\par
\fi
version: \version\par
\end{center}
\newpage
%    \end{macrocode}

% Manually include selected file,
% otherwise process as usual:
%    \begin{macrocode}
\ifchilddocmanual
\section*{part `\childdocname'}
\input{\childdocname}
\else
%    \end{macrocode}

% Include the two chapters:
%    \begin{macrocode}
\include{cdocsch1}
\include{cdocsch2}
%    \end{macrocode}

% Include the two parts unless only chapters should be displayed:
%    \begin{macrocode}
\ifchilddoc\else
\section{part three}
\input{cdocspt3}
\section{part four}
\input{cdocspt4}
\fi
%    \end{macrocode}

% Process as usual until here:
%    \begin{macrocode}
\fi
%    \end{macrocode}

% End of document body:
%    \begin{macrocode}
\end{document}
%    \end{macrocode}
%\iffalse
%</samplemain>
%\fi
%
% %%%%%%%%%%%%%%%%%%%%%%%%%%%%%%%%%%%%%%
% \paragraph{Chapter Include Files.}
%
% The include files are called |cdocsch1.tex| and |cdocsch2.tex|.
%
%\iffalse
%<*samplechap1|samplechap2>
%\fi

% Optional override for |\version| flag:
%    \begin{macrocode}
%%\providecommand{\version}{final}
%    \end{macrocode}

% Include the main document:
%    \begin{macrocode}
\input{childdoc.def}
\childdocof{cdocsamp}
%    \end{macrocode}

%\iffalse
%</samplechap1|samplechap2>
%\fi
%
%\iffalse
%<*samplechap1>
%\fi
% Some text for chapter 1:
%    \begin{macrocode}
\section{one}
some text in chapter one
%    \end{macrocode}

%\iffalse
%</samplechap1>
%\fi
% Some text for chapter 2:
%\iffalse
%<*samplechap2>
%\fi
%    \begin{macrocode}
\section{two}
more text in chapter two
%    \end{macrocode}

%\iffalse
%</samplechap2>
%\fi
%
% %%%%%%%%%%%%%%%%%%%%%%%%%%%%%%%%%%%%%%
% \paragraph{Part Include Files.}
%
% The include files are called |cdocspt3.tex| and |cdocspt4.tex|.
%
%\iffalse
%<*samplepart3|samplepart4>
%\fi

% Optional override for |\version| flag:
%    \begin{macrocode}
%%\providecommand{\version}{final}
%    \end{macrocode}

% Include the main document:
%    \begin{macrocode}
\input{childdoc.def}
\childdocby{cdocsamp}
%    \end{macrocode}

%\iffalse
%</samplepart3|samplepart4>
%\fi
%
%\iffalse
%<*samplepart3>
%\fi
% Some text for part 3:
%    \begin{macrocode}
some text in part three
%    \end{macrocode}

%\iffalse
%</samplepart3>
%\fi
% Some text for part 4:
%\iffalse
%<*samplepart4>
%\fi
%    \begin{macrocode}
more text in part four
%    \end{macrocode}

%\iffalse
%</samplepart4>
%\fi
%
% %%%%%%%%%%%%%%%%%%%%%%%%%%%%%%%%%%%%%%
% \paragraph{Forwarding for a Complete Draft.}
%
% The following forwarding file |cdocsdrf.tex|
% compiles the main document in draft mode:
%\iffalse
%<*sampledraft>
%\fi
%    \begin{macrocode}
\def\version{draft}
\input{childdoc.def}
\childdocforward{cdocsamp}
%    \end{macrocode}

%\iffalse
%</sampledraft>
%\fi
%
% %%%%%%%%%%%%%%%%%%%%%%%%%%%%%%%%%%%%%%
% \paragraph{Forwarding for Final Version of the Chapters.}
%
% The following forwarding files |cdocsfn1.tex| and |cdocsfn2.tex|
% (with identical content)
% compile the final versions of the child documents
% |cdocsch1.tex| and |cdocsch2.tex|, respectively:
%\iffalse
%<*samplefinal>
%\fi
%    \begin{macrocode}
\def\version{final}
\input{childdoc.def}
\childdocforwardprefix[cdocsamp]{cdocsfn}{cdocsch}
%    \end{macrocode}

%\iffalse
%</samplefinal>
%\fi
%
% %%%%%%%%%%%%%%%%%%%%%%%%%%%%%%%%%%%%%%
% \paragraph{Command Line Processing.}
%
% The following three command lines generate the output files
% |cdocscld|, |cdocscl1| and |cdocscl2|
% which should be identical to
% |cdocsdrf|, |cdocsch1| and |cdocsfn2|, respectively:
% \begin{center}
% \begin{tabular}{l}
% |latex -jobname cdocscld \|\\
% |  "\def\version{draft}\input{childdoc.def}\childdocforward{cdocsamp}"|\\
% |latex -jobname cdocscl1 \|\\
% |  "\input{childdoc.def}\childdocforward[cdocsamp]{cdocsch1}"|\\
% |latex -jobname cdocscl2 \|\\
% |  "\def\version{final}\input{childdoc.def}\childdocforward{cdocsch2}"|
% \end{tabular}
% \end{center}
% Note that the trailing backslash on each first line
% merely continues the input to the second line
% (for convenient cut ant paste).
% Furthermore, the command |latex| can be replaced by any
% of its alternative versions such as |pdflatex|.
%
% %%%%%%%%%%%%%%%%%%%%%%%%%%%%%%%%%%%%%%%%%%%%%%%%%%%%%%%%%%%%%%%%%%%%%%%%%%%%%%
% %%%%%%%%%%%%%%%%%%%%%%%%%%%%%%%%%%%%%%%%%%%%%%%%%%%%%%%%%%%%%%%%%%%%%%%%%%%%%%
% \section{Implementation}
%\iffalse
%<*package>
%\fi
%
% This section describes the definitions file |childdoc.def|.

% The definitions cannot be loaded using |\usepackage| or |\RequirePackage|
% which has a mechanism to prevent loading a style file more than once.
% When loading the definitions by means of |\input|
% multiple instances have to be prevented manually:
%\iffalse
%This code needs to be before the `\ProvidesFile' directive
%which is defined at the beginning of this file.
%Therefore it is also placed there and commented out here.
%</package>
%<*discard>
%\fi
%    \begin{macrocode}
\ifdefined\childdocmain\endinput\fi
%    \end{macrocode}
%\iffalse
%</discard>
%<*package>
%\fi
%
% \macro{\ifchilddoc}
% \macro{\ifchilddocmanual}
% The conditional |\ifchilddoc| tells whether a
% child (true) or main (false) document is being compiled.
% The conditional |\ifchilddocmanual| tells whether
% the |\includeonly| mechanism is used (false) or
% the selection of child files must be performed manually (true).
% The definitions initialise to false:
%    \begin{macrocode}
\newif\ifchilddoc
\newif\ifchilddocmanual
%    \end{macrocode}

% \macro{\childdocname}
% \macro{\childdocjob}
% The macro |\childdocname| stores the name of the main document
% to be compiled. The macro |\childdocjob| stores the name of
% the document on which the \LaTeX{} compiler was originally invoked.
% The content of |\jobname| cannot be compared
% to filenames specified in the source due to different catcodes.
% The following code rescans |\jobname|, stores the result
% in |\childdocname| and saves a copy in |\childdocjob|:
%    \begin{macrocode}
\edef\childdocname{\scantokens\expandafter{\jobname\noexpand}}
\let\childdocjob\childdocname
%    \end{macrocode}

% \macro{\childdocdisable}
% The macro |\childdocdisable| prevents the main file
% from being processed more than once.
% At this stage, the main document command |\childdocmain|
% is assumed to be called once again where it should do nothing.
% Any subsequent call to it should prevent
% a secondary processing of the main document
% It overwrites the forwarding commands
% |\childdocof| and |\childdocforward|
% with empty macros to prevent further inclusions of the main document:
%    \begin{macrocode}
\newcommand{\childdocdisable}
{
  \renewcommand{\childdocmain}[1]{\renewcommand{\childdocmain}[1]{\endinput}}
  \renewcommand{\childdocof}[1]{}
  \renewcommand{\childdocby}[2][]{}
  \renewcommand{\childdocforward}[2][]{}
  \renewcommand{\childdocdisable}{}
}
%    \end{macrocode}

% \macro{\childdocmain}
% The macro |\childdocmain| is to be called at the top of the main file
% with nothing or the main filename (without extension) as argument.
% First, it breaks loops.
% If the argument is not empty and does not match |\childdocname|
% (which is set by the first inclusion of |childdoc.def|),
% |\ifchilddoc| is set to true, |\includeonly| is applied to the child file
% and |\jobname| is set to the main file
% (for proper handling of |.aux| files):
%    \begin{macrocode}
\newcommand{\childdocmain}[1]
{
  \childdocdisable\childdocmain{}
  \if?#1?\else
    \begingroup
      \def\childdoctmp{#1}
      \ifx\childdoctmp\childdocname
        \def\childdoctmp{}
      \else
        \def\childdoctmp
        {
          \childdoctrue
          \includeonly{\childdocname}
          \def\childdocjob{#1}
          \def\jobname{#1}
        }
      \fi
      \expandafter
    \endgroup
    \childdoctmp
  \fi
}
%    \end{macrocode}

% \macro{\childdocof}
% The command |\childdocof| redirects
% compilation to the main file |#1|.
%    \begin{macrocode}
\newcommand{\childdocof}[1]
{
  \childdocdisable
  \childdoctrue
  \includeonly{\childdocname}
  \def\jobname{#1}
  \def\childdocjob{#1}
  \input{#1}
}
%    \end{macrocode}

% \macro{\childdocby}
% The command |\childdocby| ....
%    \begin{macrocode}
\newcommand{\childdocby}[2][]
{
  \childdocdisable
  \childdoctrue
  \childdocmanualtrue
  \if?#1?\else
    \def\jobname{#2}
  \fi
  \def\childdocjob{#2}
  \input{#2}
  \endinput
}
%    \end{macrocode}

% \macro{\childdocforward}
% The command |\childdocforward| redirects
% compilation to the main file or
% (if the optional argument is given) a child file.
% Parameters are set as if the main file
% or a child file starting with |\childdocof| was compiled.
% Then compilation is handed over to the main file:
%    \begin{macrocode}
\newcommand{\childdocforward}[2][]
{
  \begingroup
    \if?#1?
      \def\childdoctmp
      {
        \def\childdocname{#2}
        \def\childdocjob{#2}
        \def\jobname{#2}
        \input{#2}
        \endinput
      }
    \else
      \def\childdoctmp
      {
        \childdocdisable
        \def\childdocname{#2}
        \childdoctrue
        \includeonly{#2}
        \def\childdocjob{#1}
        \def\jobname{#1}
        \input{#1}
        \endinput
      }
    \fi
    \expandafter
  \endgroup
  \childdoctmp
}
%    \end{macrocode}

% \macro{\childdocforwardprefix}
% The command |\childdocforwardprefix| redirects
% compilation to the main or a child file by means of a pattern.
% The prefix |#1| in the current filename is replaced by |#2|
% and the suffix of the current filename is kept
% (it is assumed that the filename does not contain the substring `|~~~|'
% which is used as a delimiter).
% Compilation is handed over to the new file by |\childdocforward|:
%    \begin{macrocode}
\newcommand{\childdocforwardprefix}[3][]
{
  \begingroup
    \def\childdocextract #2##1~~~{\def\childdoctmp{\childdocforward[#1]{#3##1}}}
    \expandafter\childdocextract\childdocname~~~
    \expandafter
  \endgroup
  \childdoctmp
}
%    \end{macrocode}

% \macro{\childdoc}
% The deprecated macro |\childdoc| is a legacy version of |\childdocmain|:
%    \begin{macrocode}
\newcommand{\childdoc}{\childdocmain}
%    \end{macrocode}

% \macro{\childdocredirect}
% The deprecated macro |\childdocredirect| is a legacy version
% of |\childdocforward| and |\childdocforwardprefix|:
%    \begin{macrocode}
\newcommand{\childdocredirect}[2][]
{
  \begingroup
    \if?#1?
      \def\childdoctmp{\childdocforward{#2}}
    \else
      \def\childdoctmp{\childdocforwardprefix{#1}{#2}}
    \fi
    \expandafter
  \endgroup
  \childdoctmp
}
%    \end{macrocode}

%\iffalse
%</package>
%\fi
%
\endinput
\childdocforward{cdocsamp}"|\\
% |latex -jobname cdocscl1 \|\\
% |  "% \iffalse
%
% childdoc.dtx Copyright (C) 2017-2018 Niklas Beisert
%
% This work may be distributed and/or modified under the
% conditions of the LaTeX Project Public License, either version 1.3
% of this license or (at your option) any later version.
% The latest version of this license is in
%   http://www.latex-project.org/lppl.txt
% and version 1.3 or later is part of all distributions of LaTeX
% version 2005/12/01 or later.
%
% This work has the LPPL maintenance status `maintained'.
%
% The Current Maintainer of this work is Niklas Beisert.
%
% This work consists of the files childdoc.dtx and childdoc.ins
% and the derived files childdoc.def and cdocsamp.tex with
% cdocsch1.tex, cdocsch2.tex, cdocsdrf.tex, cdocsfn1.tex, cdocsfn2.tex.
%
%<package>\ifdefined\childdocmain\endinput\fi
%<package>\ProvidesFile{childdoc.def}[2018/12/30 v2.0 child document driver]
%<samplemain>\ProvidesFile{cdocsamp.tex}[2018/12/30 v2.0 sample for childdoc]
%<*driver>
%\ProvidesFile{childdoc.drv}[2018/12/30 v2.0 childdoc reference manual file]
\PassOptionsToClass{10pt,a4paper}{article}
\documentclass{ltxdoc}

\usepackage[margin=35mm]{geometry}
\usepackage{hyperref}
\usepackage{hyperxmp}
\usepackage[usenames]{color}

\hypersetup{colorlinks=true}
\hypersetup{pdfstartview=FitH}
\hypersetup{pdfpagemode=UseNone}
\hypersetup{pdfsource={}}
\hypersetup{pdflang={en-UK}}
\hypersetup{pdfcopyright={Copyright 2017-2018 Niklas Beisert.
  This work may be distributed and/or modified under the
  conditions of the LaTeX Project Public License, either version 1.3
  of this license or (at your option) any later version.}}
\hypersetup{pdflicenseurl={http://www.latex-project.org/lppl.txt}}
\hypersetup{pdfcontactaddress={ETH Zurich, ITP, HIT K,
  Wolfgang-Pauli-Strasse 27}}
\hypersetup{pdfcontactpostcode={8093}}
\hypersetup{pdfcontactcity={Zurich}}
\hypersetup{pdfcontactcountry={Switzerland}}
\hypersetup{pdfcontactemail={nbeisert@itp.phys.ethz.ch}}
\hypersetup{pdfcontacturl={http://people.phys.ethz.ch/\xmptilde nbeisert/}}

\newcommand{\secref}[1]{\hyperref[#1]{section \ref*{#1}}}

\parskip1ex
\parindent0pt
\let\olditemize\itemize
\def\itemize{\olditemize\parskip0pt}

\begin{document}

\title{The \textsf{childdoc} Package}
\hypersetup{pdftitle={The childdoc Package}}
\author{Niklas Beisert\\[2ex]
  Institut f\"ur Theoretische Physik\\
  Eidgen\"ossische Technische Hochschule Z\"urich\\
  Wolfgang-Pauli-Strasse 27, 8093 Z\"urich, Switzerland\\[1ex]
  \href{mailto:nbeisert@itp.phys.ethz.ch}
  {\texttt{nbeisert@itp.phys.ethz.ch}}}
\hypersetup{pdfauthor={Niklas Beisert}}
\hypersetup{pdfsubject={Manual for the LaTeX2e Package childdoc}}
\date{30 December 2018, \textsf{v2.0}}
\maketitle

\begin{abstract}\noindent
\textsf{childdoc} is a \LaTeXe{} package
that enables the direct compilation
of document sections included by |\include|
to individual files.
\end{abstract}

\begingroup
\parskip0ex
\tableofcontents
\endgroup

%%%%%%%%%%%%%%%%%%%%%%%%%%%%%%%%%%%%%%%%%%%%%%%%%%%%%%%%%%%%%%%%%%%%%%%%%%%%%%%%
%%%%%%%%%%%%%%%%%%%%%%%%%%%%%%%%%%%%%%%%%%%%%%%%%%%%%%%%%%%%%%%%%%%%%%%%%%%%%%%%
\section{Introduction}

\LaTeX{} provides a mechanism to structure a large document (such as a book)
into a main file and several child files (containing the chapters)
using the |\include| command.
This mechanism is beneficial for documents
which span hundreds of pages in order to
make the source file(s) more manageable.
Moreover, compilation can be restricted to
selected child files by means of the |\includeonly| command.
The latter feature can be used to reduce the compilation time while editing
(this was significantly more useful in the earlier days of \LaTeX{})
or to generate a smaller document which is easier to navigate.
Another application of |\includeonly| is to generate
documents consisting of selected parts of the complete document.

However, there are a few drawbacks of the plain |\include| mechanism:
\begin{itemize}
\item
The child files cannot be compiled on their own,
they can only be compiled via the main file.
A naive editing environment
(such as a text editor with an option
to have the current file processed by \LaTeX)
may require one to switch to the main file before compiling;
attempting to compile the child file produces errors.
\item
The main file must be modified (each time)
to adjust the |\includeonly| command
to the present needs. This easily leaves the main file in a messy state.
\item
The generated document will always carry the filename
of the main document. This is inconvenient if
several child files are to be compiled and
to be kept for distribution.
\end{itemize}

The present package provides a simple interface
to make child files individually compilable by \LaTeX{}.
Compiling a child file then has the same effect as compiling
the main file with an |\includeonly| command
to select the appropriate child.
Moreover the generated document will carry the name of the child
rather than the main file.
This resolves all three above issues.

This feature is meant to make the editing of books,
thesis documents and lecture notes somewhat more convenient.
However, the package can also be used efficiently for
composing a series of documents (such as exercise sheets)
which are typically distributed individually.
It then assists the author in generating the individual documents
(potentially in different versions)
as well as a document containing the collected series.
Another application is in developing style files
or other kinds of included material
where compilation of the style file could redirect
to a sample or test file.

%%%%%%%%%%%%%%%%%%%%%%%%%%%%%%%%%%%%%%%%%%%%%%%%%%%%%%%%%%%%%%%%%%%%%%%%%%%%%%%%
%%%%%%%%%%%%%%%%%%%%%%%%%%%%%%%%%%%%%%%%%%%%%%%%%%%%%%%%%%%%%%%%%%%%%%%%%%%%%%%%
\section{Usage}

First of all, the package \textsf{childdoc} is \emph{not} a standard
\LaTeXe{} |.sty| style file! Therefore it needs to be invoked in
a non-standard way.

%%%%%%%%%%%%%%%%%%%%%%%%%%%%%%%%%%%%%%%%%%%%%%%%%%%%%%%%%%%%%%%%%%%%%%%%%%%%%%%%
\subsection{Included Files}
\label{sec:include}

%%%%%%%%%%%%%%%%%%%%%%%%%%%%%%%%%%%%%%%%
\DescribeMacro{\childdocmain}
To use the package, add the commands
\begin{center}
\begin{tabular}{l}
|\input{childdoc.def}|\\
|\childdocmain{}|\\
\end{tabular}
\end{center}
at the very top of the main \LaTeX{} file,
in particular \emph{before} the |\documentclass| statement!
The argument of |\childdocmain| should be left empty
(but it must be present).

%%%%%%%%%%%%%%%%%%%%%%%%%%%%%%%%%%%%%%%%
\DescribeMacro{\childdocof}
Furthermore, add the commands
\begin{center}
\begin{tabular}{l}
|\input{childdoc.def}|\\
|\childdocof{|\textit{main}|}|\\
\end{tabular}
\end{center}
at the top of every child file \textit{child}
which is included by |\include{|\textit{child}|}|
from within the main file
(or at least for those files to be compiled individually).
The argument \textit{main} must be the filename of the main file.

There are a couple of
considerations in setting up the main and child documents:

%%%%%%%%%%%%%%%%%%%%%%%%%%%%%%%%%%%%%%%%
\paragraph{Restrictions.}

Please note the following restrictions:
\begin{itemize}
\item
|\childdocmain| must be called with one argument \textit{main}
to ensure compatibility with earlier version of the package.
It must either be empty (|\childdocmain{}|)
or precisely match the filename of the main file in which it is specified.
See \secref{sec:detection} for further information.
\item
The filename \textit{main} must be specified without the |.tex| extension.
\item
The filename \textit{main} is case sensitive
(even in case-insensitive file systems)
due to internal string comparison.
\item
The argument \textit{main} should be fully expanded, it cannot be a macro.
\item
Subdirectories and special characters should be avoided in filenames.
\item
The command |\childdocmain{|\textit{main}|}| must be followed by a whitespace.
It should not be followed immediately by another command
or by a comment mark `|%|'.
This is because the \TeX{} parser reads the token immediately following
the argument of |\childdocmain| and puts it
at the beginning of every child section;
however, a white\-space is ignored.
\end{itemize}

%%%%%%%%%%%%%%%%%%%%%%%%%%%%%%%%%%%%%%%%
\paragraph{Content of Main File.}

It is advisable to place all content in the child files included by |\include|.
Any output contained in the main file will appear in all child documents
unless suppressed manually;
it cannot be suppressed automatically by the |\includeonly| directive
and thus should normally be avoided.
A method to include some content in the main file
by means of conditional processing is described in \secref{sec:conditional}.

%%%%%%%%%%%%%%%%%%%%%%%%%%%%%%%%%%%%%%%%
\paragraph{Page Numbering.}

When only a part of the document is compiled,
the appropriate numbering of pages
(as well as other status parameters)
is determined from the |.aux| files.
The latter contain information from previous passes.
However this information needs to propagate through
all intermediate child documents.
Therefore the page numbering in child documents may well
be inconsistent until the complete document is compiled at least once.

A useful (if unconventional) way to always ensure a consistent
page numbering is to restart the numbering in each child document
and denote the pages by `\textit{child}|.|\textit{page}'
where \textit{child} represents the chapter/section number of the child file.
This can be achieved by the command
|\numberwithin{page}{|\textit{child}|}|
of the \textsf{amsmath} package
where \textit{child} can be |chapter| or |section|
depending on the chosen structuring.
Alternatively, one can modify the macro |\thepage| appropriately
and reset the counter |page| at the start of each child file.

%%%%%%%%%%%%%%%%%%%%%%%%%%%%%%%%%%%%%%%%%%%%%%%%%%%%%%%%%%%%%%%%%%%%%%%%%%%%%%%%
\subsection{Conditional Processing}
\label{sec:conditional}

The package provides a mechanism to compile different versions
of a document. To customise the versions further some conditional processing
can come in handy to distinguish which version is being compiled.
The package provides two macros to describe the compilation context:

%%%%%%%%%%%%%%%%%%%%%%%%%%%%%%%%%%%%%%%%
\DescribeMacro{\ifchilddoc}
The conditional |\ifchilddoc| distinguishes between the compilation of
child documents and the main document:
%
\begin{center}
|\ifchilddoc |\textit{child-code}| |[|\||else |\textit{main-code}]| \||fi|
\end{center}

%%%%%%%%%%%%%%%%%%%%%%%%%%%%%%%%%%%%%%%%
\DescribeMacro{\childdocname}
\DescribeMacro{\childdocjob}
The macro |\childdocname| contains the filename (without extension)
of the main or child file being processed.
Note that |\childdocjob| will always contain the name of the main file.

%%%%%%%%%%%%%%%%%%%%%%%%%%%%%%%%%%%%%%%%
\paragraph{Title Page.}

Conditional processing can be used to include a title or banner page
in the main document when proper precautions are taken.
Importantly, the code in the main file should ensure that the page counter
(as well as other status parameters which are stored in the |.aux| files)
takes the same value after the conditional processing.
Otherwise the page numbers may take divergent values
depending on which part is compiled.

For example, a title page could be declared by:
%
\begin{center}
\begin{tabular}{l}
|\ifchilddoc\||else|\\
|\addtocounter{page}{-1}|\\
\textit{code for title page}\\
|\newpage|\\
|\||fi|
\end{tabular}
\end{center}
%
A banner page for the child documents can be generated by:
%
\begin{center}
\begin{tabular}{l}
|\ifchilddoc|\\
|\addtocounter{page}{-1}|\\
\textit{code for banner page}\\
|\newpage|\\
|\||fi|
\end{tabular}
\end{center}
%
Here one could write a message such as:
\begin{center}
|This is the part \childdocname{} of \childdocjob{}.|
\end{center}

%%%%%%%%%%%%%%%%%%%%%%%%%%%%%%%%%%%%%%%%%%%%%%%%%%%%%%%%%%%%%%%%%%%%%%%%%%%%%%%%
\subsection{Flags}
\label{sec:flags}

The package makes it easy to generate different versions
of the main or child documents.
To this end compilation flags can be defined
and assigned different default values.
They will be particularly useful in conjunction
with the forwarding mechanism described in \secref{sec:forward}.

For example, it may be useful to have a flag |\version|
which can be set to |draft| or |final|.
The document source will contain some conditional code
depending on the value of |\version|.
Suppose further, the flag should default to |final| for the main file
and to |draft| for child files
which is a natural assignment for editing the document.
This is achieved by placing the following code
in the preamble of the main document
(below the |\childdocmain| directive):
%
\begin{center}
\begin{tabular}{l}
|\ifchilddoc|\\
|\providecommand{\version}{draft}|\\
|\||else|\\
|\providecommand{\version}{final}|\\
|\||fi|
\end{tabular}
\end{center}
%
The definition by |\providecommand| makes sure
that previous definitions are not overwritten.
Further statements |\providecommand{\version}{...}|
can thus be added before the above code to override it.

For the main file, one might add a line
(between |\childdocmain| and the above block)
%
\begin{center}
|%\ifchilddoc\||else\providecommand{\version}{draft}\||fi|
\end{center}
%
which can be uncommented to produce a draft version.
Likewise one can add a line to the very top of a child file
(above the |\childdocof{|\textit{main}|}| directive)
%
\begin{center}
|%\providecommand{\version}{final}|
\end{center}
%
which can be uncommented to produce the final version of this child document.

%%%%%%%%%%%%%%%%%%%%%%%%%%%%%%%%%%%%%%%%%%%%%%%%%%%%%%%%%%%%%%%%%%%%%%%%%%%%%%%%
\subsection{Forwarding}
\label{sec:forward}

Different versions of the main or child documents
using compilation flags as described in \secref{sec:flags}
can be (permanently) stored in different files
for convenient compilation, viewing and distribution.
To this end, the package defines a command
to pass on compilation to a different file:

%%%%%%%%%%%%%%%%%%%%%%%%%%%%%%%%%%%%%%%%
\DescribeMacro{\childdocforward}
The command |\childdocforward| redirects processing to
another source file:
%
\begin{center}
\begin{tabular}{l}
|\input{childdoc.def}|\\
|\childdocforward[|\textit{main}|]{|\textit{dest}|}|\\
\end{tabular}
\end{center}
%
The argument \textit{dest} is the destination file
(without extension).
It should be the main file or one of the child files.
Note that further \textsf{childdoc} directives
such as |\childdocof| and |\childdocforward|
in the indicated file will be processed in this form.
The optional argument \textit{main}
passes on directly to the main file \textit{main}
while pretending to compile the child \textit{dest}.
This form behaves as if \textit{dest}
issues |\childdocof{|\textit{main}|}| right away,
and no further \textsf{childdoc} directives will be processed.

%%%%%%%%%%%%%%%%%%%%%%%%%%%%%%%%%%%%%%%%
\DescribeMacro{\...prefix}
In the alternative form |\childdocforwardprefix|,
%
\begin{center}
\begin{tabular}{l}
|\input{childdoc.def}|\\
|\childdocforwardprefix[|\textit{main}|]{|\textit{prefix}|}{|\textit{dest}|}|
\end{tabular}
\end{center}
%
the destination file is determined by a pattern
depending on the current file:
To make this work, the current file must be called
`{\textit{prefix}\hspace{0.2em}\textit{suffix}}'
with \textit{prefix} matching precisely the argument.
Processing is then passed on to the file
`{\textit{dest}\hspace{0.2em}\textit{suffix}}'.
Surely, the same effect is achieved by
directly specifying the
argument `{\textit{dest}\hspace{0.2em}\textit{suffix}}'
in the first form.
However, that requires to set up a different file
for each child. With the alternative form of the command
all these files can have exactly the same content
which simplifies setting them up and maintaining them.

For example, the following file |draft.tex|
with a compilation flag |\version| as described in \secref{sec:flags}
compiles the main document as a draft:
%
\begin{center}
\begin{tabular}{l}
|\def\version{draft}|\\
|\input{childdoc.def}|\\
|\childdocforward{|\textit{main}|}|
\end{tabular}
\end{center}
%
Likewise, the following files |final|\textit{nn}|.tex|
compile the final version of the child document
|child|\textit{nn}|.tex|:
%
\begin{center}
\begin{tabular}{l}
|\def\version{final}|\\
|\input{childdoc.def}|\\
|\childdocforwardprefix{final}{child}|
\end{tabular}
\end{center}
%

Note that when several versions of a main file and/or of each child file
are to be generated, it may be convenient to set up a |Makefile| or
shell script to automatise the process.

%%%%%%%%%%%%%%%%%%%%%%%%%%%%%%%%%%%%%%%%%%%%%%%%%%%%%%%%%%%%%%%%%%%%%%%%%%%%%%%%
\subsection{Command Line Processing}
\label{sec:commandline}

The effect of redirection files can also be achieved by invoking
the \LaTeX{} compiler with a more elaborate command line.
Most conveniently this should be done as part
of a shell script or a |Makefile|.

When using \textsf{childdoc} in the main file, the following
command lines effectively perform a redirection
(note that depending on the shell being used,
backslashes may have to be doubled: `|\|' $\to$ `|\\|'):
%
\begin{center}
|... -jobname "|\textit{target}|" |\\|"|[\textit{flags}]%
|\input{childdoc.def}\childdocforward[|\textit{main}|]{|\textit{dest}|}"|
\end{center}
%
Here \textit{target} is the name of the output file,
\textit{main} is the name of the main file
and \textit{dest} is the name of the main or child file to be processed
(all filenames without extensions).
The optional argument \textit{main} can be omitted
if \textit{main} matches \textit{dest}.
Optionally, compilation \textit{flags} can be defined via |\def| commands.
This command line makes the \TeX{} engine believe
it is compiling the file \textit{target}
whose content is specified as the latter parameter.
The provided code then forwards the processing to
\textit{main} or \textit{dest} as described in \secref{sec:forward}.

%%%%%%%%%%%%%%%%%%%%%%%%%%%%%%%%%%%%%%%%%%%%%%%%%%%%%%%%%%%%%%%%%%%%%%%%%%%%%%%%
\subsection{Include by Input}
\label{sec:input}

Including child documents by |\include| has some restrictions by design.
Most notably, the content of a child document always occupies
its own set of pages; pages cannot be shared between child documents.
Usually, this behaviour makes perfect sense
because each child document contain an essential part of the document.
However, in some situations it may be desirable to compose
a document from a collection of parts
without having mandatory page breaks between then.
For this case, the package
provides a mechanism to include parts
by |\input| which can also be processed individually.
However, by construction this mechanism
requires manual handling of the content to be output.

%%%%%%%%%%%%%%%%%%%%%%%%%%%%%%%%%%%%%%%%
\DescribeMacro{\ifchilddocmanual}
The main file should be prepared as usual, see \secref{sec:include}.
However, the document body must make a distinction
between processing of an individual part and of the main document, e.g.:
%
\begin{center}
\begin{tabular}{l}
|\ifchilddocmanual|\\
|\input{\childdocname}|\\
|\||else|\\
\textit{document body with }|\input{|\textit{part}|}|\\
|\||fi|
\end{tabular}
\end{center}
%
The conditional |\ifchilddocmanual| is true whenever
a part to be included by |\input| is being compiled,
and the name of the part is stored in |\childdocname|.

%%%%%%%%%%%%%%%%%%%%%%%%%%%%%%%%%%%%%%%%
\DescribeMacro{\childdocby}
Each part to be included by |\input| should start with:
%
\begin{center}
\begin{tabular}{l}
|\input{childdoc.def}|\\
|\childdocby{|\textit{main}|}|\\
\end{tabular}
\end{center}
%
The directive |\childdocby| is similar to |\childdocof|
described in \secref{sec:include},
but the subsequent selection of content must be done manually.
To that end, both |\ifchilddoc| and |\ifchilddocmanual|
will be true upon processing of a part,
and the name of the part is stored in |\childdocname|.
Note that |\jobname| will be set to the filename of the current part
so that each part receives an individual |.aux| file
that does not interfere with the |.aux| file(s) of the main document.
This behaviour can be altered by the alternative form
|\childdocby[*]{|\textit{main}|}| (with a non-empty optional argument)
which uses the |.aux| file of the main document
by setting |\jobname| to \textit{main}.

%%%%%%%%%%%%%%%%%%%%%%%%%%%%%%%%%%%%%%%%%%%%%%%%%%%%%%%%%%%%%%%%%%%%%%%%%%%%%%%%
\subsection{Driver Development}
\label{sec:driver}

The \textsf{childdoc} mechanism can also be use for the development
of definition files such as \LaTeX{} styles or classes.
This case differs from the above setup with multiple parts
included by |\include| in that no |\includeonly| should be invoked.
This can be achieved by starting the include file
(before |\ProvidesPackage|) with:
%
\begin{center}
\begin{tabular}{l}
|\input{childdoc.def}|\\
|\childdocforward{|\textit{main}|}|\\
\end{tabular}
\end{center}
%
or alternatively with:
%
\begin{center}
\begin{tabular}{l}
|\input{childdoc.def}|\\
|\childdocby{|\textit{main}|}|\\
\end{tabular}
\end{center}
%
Both forms have slightly different effects as described above.
The main file is prepared as usual, see \secref{sec:include}.

%%%%%%%%%%%%%%%%%%%%%%%%%%%%%%%%%%%%%%%%%%%%%%%%%%%%%%%%%%%%%%%%%%%%%%%%%%%%%%%%
\subsection{Legacy Detection}
\label{sec:detection}

The directive |\childdocmain| in the main file can detect
whether the complete document or merely a child is to be compiled
even without using the directive |\childdocof|.
This method is deprecated because it is less robust
and there is no compelling reason to use it;
it is merely provided for backward compatibility
and it may be removed in future versions.

If the detection mechanism is to be used,
it is mandatory to correctly specify
the filename of the main file as the argument of |\childdocmain|:
%
\begin{center}
\begin{tabular}{l}
|\input{childdoc.def}|\\
|\childdocmain{|\textit{main}|}|\\
\end{tabular}
\end{center}
%
If |\jobname| does not match the argument \textit{main} of |\childdocmain|,
it is assumed that |\jobname| points to the child file to be compiled.
When using |\childdocmain| with the main file specified as argument,
it suffices to start a child file
with just |\input{|\textit{main}|}|
without loading of the package and using |\childdocof|.
If instead all processing is done
with the appropriate \textsf{childdoc} directives,
the argument of \textit{main} of |\childdocmain| can be empty.

An alternative version of the command line processing described
in \secref{sec:commandline} using the detection mechanism reads:
%
\begin{center}
|... -jobname "|\textit{target}|" "|[\textit{flags}]%
[|\def\jobname{|\textit{dest}|}|]|\input{|\textit{main}|}"|
\end{center}

%%%%%%%%%%%%%%%%%%%%%%%%%%%%%%%%%%%%%%%%%%%%%%%%%%%%%%%%%%%%%%%%%%%%%%%%%%%%%%%%
\subsection{Manual Code}
\label{sec:manual}

In case one cannot be certain whether the definitions file |childdoc.def|
is installed on the target \TeX{} distribution
and one prefers not to ship it,
it is conceivable to paste a few relevant commands into the sources.

To that end, drop all statements |\input{childdoc.def}|
and perform the replacements as outlined below.
Instead of |\childdocmain{|\textit{main}|}| add the following code
to the top of the main file:
%
\begin{center}
\begin{tabular}{l}
|\||ifdefined\childdocname\endinput\||fi\newif\ifchilddoc|\\
|\edef\childdocname{\scantokens\expandafter{\jobname\noexpand}}|\\
|\def\childdocmain{|\textit{main}|}\||ifx\childdocmain\childdocname\||else|\\
|\childdoctrue\includeonly{\childdocname}\let\jobname\childdocmain\||fi|\\
\end{tabular}
\end{center}
%
Instead of |\childdocof{|\textit{main}|}| just include the main file
at the top of each child file:
%
\begin{center}
|\input{|\textit{main}|}|
\end{center}
%
A simple redirection |\childdocforward{|\textit{dest}|}| is achieved by:
%
\begin{center}
|\def\jobname{|\textit{dest}|}\input{\jobname}|
\end{center}
%
The redirection with prefix
|\childdocforwardprefix[|\textit{prefix}|]{|\textit{dest}|}|
is accomplished by:
%
\begin{center}
\begin{tabular}{l}
|{\edef\jobname{\scantokens\expandafter{\jobname\noexpand}}|\\
|\def\redirectjob |\textit{prefix}|#1~~~{\gdef\jobname{|\textit{dest}|#1}}|\\
|\expandafter\redirectjob\jobname~~~}\input{\jobname}|
\end{tabular}
\end{center}

In an alternative approach,
child documents can be compiled by a specific command line
without additional code or specific definitions:
%
\begin{center}
|... -jobname "|\textit{target}|" "|[\textit{flags}]%
|\includeonly{|\textit{dest}|}\input{|\textit{main}|}"|
\end{center}
%

%%%%%%%%%%%%%%%%%%%%%%%%%%%%%%%%%%%%%%%%%%%%%%%%%%%%%%%%%%%%%%%%%%%%%%%%%%%%%%%%
%%%%%%%%%%%%%%%%%%%%%%%%%%%%%%%%%%%%%%%%%%%%%%%%%%%%%%%%%%%%%%%%%%%%%%%%%%%%%%%%
\section{Information}

%%%%%%%%%%%%%%%%%%%%%%%%%%%%%%%%%%%%%%%%%%%%%%%%%%%%%%%%%%%%%%%%%%%%%%%%%%%%%%%%
\subsection{Copyright}

Copyright \copyright{} 2017--2018 Niklas Beisert

This work may be distributed and/or modified under the
conditions of the \LaTeX{} Project Public License, either version 1.3
of this license or (at your option) any later version.
The latest version of this license is in
  \url{http://www.latex-project.org/lppl.txt}
and version 1.3 or later is part of all distributions of \LaTeX{}
version 2005/12/01 or later.

This work has the LPPL maintenance status `maintained'.

The Current Maintainer of this work is Niklas Beisert.

This work consists of the files |README.txt|, |childdoc.ins| and |childdoc.dtx|
as well as the derived files |childdoc.def|, |cdocsamp.tex|
with |cdocsch1.tex|, |cdocsch2.tex|, |cdocspt3.tex|, |cdocspt4.tex|,
|cdocsdrf.tex|, |cdocsfn1.tex|, |cdocsfn2.tex|
as well as |childdoc.pdf|.

%%%%%%%%%%%%%%%%%%%%%%%%%%%%%%%%%%%%%%%%%%%%%%%%%%%%%%%%%%%%%%%%%%%%%%%%%%%%%%%%
\subsection{Files and Installation}

The package consists of the files:
%
\begin{center}
\begin{tabular}{ll}
    |README.txt|   & readme file \\
    |childdoc.ins| & installation file \\
    |childdoc.dtx| & source file \\
    |childdoc.def| & definition file \\
    |cdocsamp.tex| & sample main file \\
    |cdocsch1.tex| & sample include file \\
    |cdocsch2.tex| & sample include file \\
    |cdocspt3.tex| & sample part file \\
    |cdocspt4.tex| & sample part file \\
    |cdocsdrf.tex| & sample redirection file \\
    |cdocsfn1.tex| & sample redirection file \\
    |cdocsfn2.tex| & sample redirection file \\
    |childdoc.pdf| & manual
\end{tabular}
\end{center}
%
The distribution consists of the files
|README.txt|, |childdoc.ins| and |childdoc.dtx|.
%
\begin{itemize}
\item
Run (pdf)\LaTeX{} on |childdoc.dtx|
to compile the manual |childdoc.pdf| (this file).
\item
Run \LaTeX{} on |childdoc.ins| to create the definitions file |childdoc.def|
and the sample |cdocsamp.tex| with include files
|cdocsch1.tex|, |cdocsch2.tex|, |cdocspt3.tex|, |cdocspt4.tex|,
|cdocsdrf.tex|, |cdocsfn1.tex|, |cdocsfn2.tex|.
Then copy the file |childdoc.def| to an appropriate directory of your \LaTeX{}
distribution, e.g.\ \textit{texmf-root}|/tex/latex/childdoc|.
\end{itemize}

%%%%%%%%%%%%%%%%%%%%%%%%%%%%%%%%%%%%%%%%%%%%%%%%%%%%%%%%%%%%%%%%%%%%%%%%%%%%%%%%
\subsection{Related CTAN Packages}

There are several other packages which offer a similar functionality:
%
\begin{itemize}
\item
The packages
\href{http://ctan.org/pkg/docmute}{\textsf{docmute}},
\href{http://ctan.org/pkg/includex}{\textsf{includex}} and
\href{http://ctan.org/pkg/standalone}{\textsf{standalone}}
provide commands to include only the document body of
a child file thus allowing both files to be compiled individually.
\item
The packages \href{http://ctan.org/pkg/subdocs}{\textsf{subdocs}}
and \href{http://ctan.org/pkg/subfiles}{\textsf{subfiles}}
provide structures in which the main and child documents can be
encapsulated and allowing them to be compiled individually.
The inclusion mechanism is different from the conventional |\include|.
\item
The package \href{http://ctan.org/pkg/combine}{\textsf{combine}}
is an elaborate solution to combine several documents into one.
\end{itemize}
%
See also the CTAN topic \href{http://ctan.org/topic/subdocs}{\textsf{subdocs}}
for further related packages.
The present package differs from the above solutions in that
a document structure constructed with the conventional |\include| mechanism
just needs two extra commands at the top of every file
such that all constituent files can be compiled individually.

%%%%%%%%%%%%%%%%%%%%%%%%%%%%%%%%%%%%%%%%%%%%%%%%%%%%%%%%%%%%%%%%%%%%%%%%%%%%%%%%
%\subsection{Feature Suggestions}
%
%The following is a list of features which may be useful for future
%versions of this package:
%%
%\begin{itemize}
%\item
%\ldots
%\end{itemize}

%%%%%%%%%%%%%%%%%%%%%%%%%%%%%%%%%%%%%%%%%%%%%%%%%%%%%%%%%%%%%%%%%%%%%%%%%%%%%%%%
\subsection{Revision History}

%%%%%%%%%%%%%%%%%%%%%%%%%%%%%%%%%%%%%%%%
\paragraph{v2.0:} 2018/12/30

\begin{itemize}
\item
immediate forward processing
\item
added |\childdocby| mechanism
\item
manual restructured
\end{itemize}

%%%%%%%%%%%%%%%%%%%%%%%%%%%%%%%%%%%%%%%%
\paragraph{v1.6:} 2018/01/17

\begin{itemize}
\item
application for development of include files
\item
corrections to manual
\end{itemize}

%%%%%%%%%%%%%%%%%%%%%%%%%%%%%%%%%%%%%%%%
\paragraph{v1.5:} 2017/05/21

\begin{itemize}
\item
more complete structuring introduced
\item
|\childdocof| introduced
\item
|\childdoc| renamed to |\childdocmain|
\item
|\childredirect| renamed to |\childdocforward| and |\childdocforwardprefix|
and functionality expanded
\end{itemize}

%%%%%%%%%%%%%%%%%%%%%%%%%%%%%%%%%%%%%%%%
\paragraph{v1.0:} 2017/04/27

\begin{itemize}
\item
manual and install package
\item
first version published on CTAN
\end{itemize}

%%%%%%%%%%%%%%%%%%%%%%%%%%%%%%%%%%%%%%%%
\paragraph{v0.6:} 2017/04/26

\begin{itemize}
\item
redirection mechanism added
\end{itemize}

%%%%%%%%%%%%%%%%%%%%%%%%%%%%%%%%%%%%%%%%
\paragraph{v0.5:} 2017/04/26

\begin{itemize}
\item
functionality in definition file
\end{itemize}


%%%%%%%%%%%%%%%%%%%%%%%%%%%%%%%%%%%%%%%%%%%%%%%%%%%%%%%%%%%%%%%%%%%%%%%%%%%%%%%%
%%%%%%%%%%%%%%%%%%%%%%%%%%%%%%%%%%%%%%%%%%%%%%%%%%%%%%%%%%%%%%%%%%%%%%%%%%%%%%%%
%%%%%%%%%%%%%%%%%%%%%%%%%%%%%%%%%%%%%%%%%%%%%%%%%%%%%%%%%%%%%%%%%%%%%%%%%%%%%%%%
\appendix

\settowidth\MacroIndent{\rmfamily\scriptsize 000\ }

 \DocInput{childdoc.dtx}

\end{document}
%</driver>
% \fi
%
% %%%%%%%%%%%%%%%%%%%%%%%%%%%%%%%%%%%%%%%%%%%%%%%%%%%%%%%%%%%%%%%%%%%%%%%%%%%%%%
% %%%%%%%%%%%%%%%%%%%%%%%%%%%%%%%%%%%%%%%%%%%%%%%%%%%%%%%%%%%%%%%%%%%%%%%%%%%%%%
% \section{Sample}
%\iffalse
%<*samplemain>
%\fi
%
% The following presents a sample document
% with two chapters, two parts, a title page,
% a compile flag as well as three forwarding files to set the flag.
% It consists of eight |.tex| files:
% \begin{center}
% \begin{tabular}{ll}
% |cdocsamp.tex|&main file\\
% |cdocsch1.tex|&include file for chapter 1\\
% |cdocsch2.tex|&include file for chapter 2\\
% |cdocspt3.tex|&include file for part 3\\
% |cdocspt4.tex|&include file for part 4\\
% |cdocsdrf.tex|&forwarding file for main file in draft mode\\
% |cdocsfi1.tex|&forwarding file for final version of chapter 1\\
% |cdocsfi2.tex|&forwarding file for final version of chapter 2\\
% \end{tabular}
% \end{center}
% Each of the eight files can be compiled directly by the \LaTeX{} compiler.
%
% %%%%%%%%%%%%%%%%%%%%%%%%%%%%%%%%%%%%%%
% \paragraph{Main File.}
%
% The main file is called |cdocsamp.tex|.
%
% Load the \textsf{childdoc} definitions and
% declare the filename for the main document:
%    \begin{macrocode}
\input{childdoc.def}
\childdocmain{}
%    \end{macrocode}

% Optional override for |\version| flag:
%    \begin{macrocode}
%%\ifchilddoc\else\providecommand{\version}{draft}\fi
%    \end{macrocode}

% Define the default values for the |\version| flag
% (|final| for the main file and |draft| for childs):
%    \begin{macrocode}
\ifchilddoc
\providecommand{\version}{draft}
\else
\providecommand{\version}{final}
\fi
%    \end{macrocode}

% Load the standard document class:
%    \begin{macrocode}
\documentclass[12pt]{article}
%    \end{macrocode}

% Start the document body:
%    \begin{macrocode}
\begin{document}
%    \end{macrocode}

% Declare a title page.
% Print title, part of document being processed and version flag:
%    \begin{macrocode}
\addtocounter{page}{-1}
\begin{center}
{\LARGE\bfseries{}childdoc example\par}
\vspace{1cm}
\ifchilddoc
\ifchilddocmanual part\else chapter\fi:
`\childdocname' of `\childdocjob'\par
\else
main document: `\childdocjob'\par
\fi
version: \version\par
\end{center}
\newpage
%    \end{macrocode}

% Manually include selected file,
% otherwise process as usual:
%    \begin{macrocode}
\ifchilddocmanual
\section*{part `\childdocname'}
\input{\childdocname}
\else
%    \end{macrocode}

% Include the two chapters:
%    \begin{macrocode}
\include{cdocsch1}
\include{cdocsch2}
%    \end{macrocode}

% Include the two parts unless only chapters should be displayed:
%    \begin{macrocode}
\ifchilddoc\else
\section{part three}
\input{cdocspt3}
\section{part four}
\input{cdocspt4}
\fi
%    \end{macrocode}

% Process as usual until here:
%    \begin{macrocode}
\fi
%    \end{macrocode}

% End of document body:
%    \begin{macrocode}
\end{document}
%    \end{macrocode}
%\iffalse
%</samplemain>
%\fi
%
% %%%%%%%%%%%%%%%%%%%%%%%%%%%%%%%%%%%%%%
% \paragraph{Chapter Include Files.}
%
% The include files are called |cdocsch1.tex| and |cdocsch2.tex|.
%
%\iffalse
%<*samplechap1|samplechap2>
%\fi

% Optional override for |\version| flag:
%    \begin{macrocode}
%%\providecommand{\version}{final}
%    \end{macrocode}

% Include the main document:
%    \begin{macrocode}
\input{childdoc.def}
\childdocof{cdocsamp}
%    \end{macrocode}

%\iffalse
%</samplechap1|samplechap2>
%\fi
%
%\iffalse
%<*samplechap1>
%\fi
% Some text for chapter 1:
%    \begin{macrocode}
\section{one}
some text in chapter one
%    \end{macrocode}

%\iffalse
%</samplechap1>
%\fi
% Some text for chapter 2:
%\iffalse
%<*samplechap2>
%\fi
%    \begin{macrocode}
\section{two}
more text in chapter two
%    \end{macrocode}

%\iffalse
%</samplechap2>
%\fi
%
% %%%%%%%%%%%%%%%%%%%%%%%%%%%%%%%%%%%%%%
% \paragraph{Part Include Files.}
%
% The include files are called |cdocspt3.tex| and |cdocspt4.tex|.
%
%\iffalse
%<*samplepart3|samplepart4>
%\fi

% Optional override for |\version| flag:
%    \begin{macrocode}
%%\providecommand{\version}{final}
%    \end{macrocode}

% Include the main document:
%    \begin{macrocode}
\input{childdoc.def}
\childdocby{cdocsamp}
%    \end{macrocode}

%\iffalse
%</samplepart3|samplepart4>
%\fi
%
%\iffalse
%<*samplepart3>
%\fi
% Some text for part 3:
%    \begin{macrocode}
some text in part three
%    \end{macrocode}

%\iffalse
%</samplepart3>
%\fi
% Some text for part 4:
%\iffalse
%<*samplepart4>
%\fi
%    \begin{macrocode}
more text in part four
%    \end{macrocode}

%\iffalse
%</samplepart4>
%\fi
%
% %%%%%%%%%%%%%%%%%%%%%%%%%%%%%%%%%%%%%%
% \paragraph{Forwarding for a Complete Draft.}
%
% The following forwarding file |cdocsdrf.tex|
% compiles the main document in draft mode:
%\iffalse
%<*sampledraft>
%\fi
%    \begin{macrocode}
\def\version{draft}
\input{childdoc.def}
\childdocforward{cdocsamp}
%    \end{macrocode}

%\iffalse
%</sampledraft>
%\fi
%
% %%%%%%%%%%%%%%%%%%%%%%%%%%%%%%%%%%%%%%
% \paragraph{Forwarding for Final Version of the Chapters.}
%
% The following forwarding files |cdocsfn1.tex| and |cdocsfn2.tex|
% (with identical content)
% compile the final versions of the child documents
% |cdocsch1.tex| and |cdocsch2.tex|, respectively:
%\iffalse
%<*samplefinal>
%\fi
%    \begin{macrocode}
\def\version{final}
\input{childdoc.def}
\childdocforwardprefix[cdocsamp]{cdocsfn}{cdocsch}
%    \end{macrocode}

%\iffalse
%</samplefinal>
%\fi
%
% %%%%%%%%%%%%%%%%%%%%%%%%%%%%%%%%%%%%%%
% \paragraph{Command Line Processing.}
%
% The following three command lines generate the output files
% |cdocscld|, |cdocscl1| and |cdocscl2|
% which should be identical to
% |cdocsdrf|, |cdocsch1| and |cdocsfn2|, respectively:
% \begin{center}
% \begin{tabular}{l}
% |latex -jobname cdocscld \|\\
% |  "\def\version{draft}\input{childdoc.def}\childdocforward{cdocsamp}"|\\
% |latex -jobname cdocscl1 \|\\
% |  "\input{childdoc.def}\childdocforward[cdocsamp]{cdocsch1}"|\\
% |latex -jobname cdocscl2 \|\\
% |  "\def\version{final}\input{childdoc.def}\childdocforward{cdocsch2}"|
% \end{tabular}
% \end{center}
% Note that the trailing backslash on each first line
% merely continues the input to the second line
% (for convenient cut ant paste).
% Furthermore, the command |latex| can be replaced by any
% of its alternative versions such as |pdflatex|.
%
% %%%%%%%%%%%%%%%%%%%%%%%%%%%%%%%%%%%%%%%%%%%%%%%%%%%%%%%%%%%%%%%%%%%%%%%%%%%%%%
% %%%%%%%%%%%%%%%%%%%%%%%%%%%%%%%%%%%%%%%%%%%%%%%%%%%%%%%%%%%%%%%%%%%%%%%%%%%%%%
% \section{Implementation}
%\iffalse
%<*package>
%\fi
%
% This section describes the definitions file |childdoc.def|.

% The definitions cannot be loaded using |\usepackage| or |\RequirePackage|
% which has a mechanism to prevent loading a style file more than once.
% When loading the definitions by means of |\input|
% multiple instances have to be prevented manually:
%\iffalse
%This code needs to be before the `\ProvidesFile' directive
%which is defined at the beginning of this file.
%Therefore it is also placed there and commented out here.
%</package>
%<*discard>
%\fi
%    \begin{macrocode}
\ifdefined\childdocmain\endinput\fi
%    \end{macrocode}
%\iffalse
%</discard>
%<*package>
%\fi
%
% \macro{\ifchilddoc}
% \macro{\ifchilddocmanual}
% The conditional |\ifchilddoc| tells whether a
% child (true) or main (false) document is being compiled.
% The conditional |\ifchilddocmanual| tells whether
% the |\includeonly| mechanism is used (false) or
% the selection of child files must be performed manually (true).
% The definitions initialise to false:
%    \begin{macrocode}
\newif\ifchilddoc
\newif\ifchilddocmanual
%    \end{macrocode}

% \macro{\childdocname}
% \macro{\childdocjob}
% The macro |\childdocname| stores the name of the main document
% to be compiled. The macro |\childdocjob| stores the name of
% the document on which the \LaTeX{} compiler was originally invoked.
% The content of |\jobname| cannot be compared
% to filenames specified in the source due to different catcodes.
% The following code rescans |\jobname|, stores the result
% in |\childdocname| and saves a copy in |\childdocjob|:
%    \begin{macrocode}
\edef\childdocname{\scantokens\expandafter{\jobname\noexpand}}
\let\childdocjob\childdocname
%    \end{macrocode}

% \macro{\childdocdisable}
% The macro |\childdocdisable| prevents the main file
% from being processed more than once.
% At this stage, the main document command |\childdocmain|
% is assumed to be called once again where it should do nothing.
% Any subsequent call to it should prevent
% a secondary processing of the main document
% It overwrites the forwarding commands
% |\childdocof| and |\childdocforward|
% with empty macros to prevent further inclusions of the main document:
%    \begin{macrocode}
\newcommand{\childdocdisable}
{
  \renewcommand{\childdocmain}[1]{\renewcommand{\childdocmain}[1]{\endinput}}
  \renewcommand{\childdocof}[1]{}
  \renewcommand{\childdocby}[2][]{}
  \renewcommand{\childdocforward}[2][]{}
  \renewcommand{\childdocdisable}{}
}
%    \end{macrocode}

% \macro{\childdocmain}
% The macro |\childdocmain| is to be called at the top of the main file
% with nothing or the main filename (without extension) as argument.
% First, it breaks loops.
% If the argument is not empty and does not match |\childdocname|
% (which is set by the first inclusion of |childdoc.def|),
% |\ifchilddoc| is set to true, |\includeonly| is applied to the child file
% and |\jobname| is set to the main file
% (for proper handling of |.aux| files):
%    \begin{macrocode}
\newcommand{\childdocmain}[1]
{
  \childdocdisable\childdocmain{}
  \if?#1?\else
    \begingroup
      \def\childdoctmp{#1}
      \ifx\childdoctmp\childdocname
        \def\childdoctmp{}
      \else
        \def\childdoctmp
        {
          \childdoctrue
          \includeonly{\childdocname}
          \def\childdocjob{#1}
          \def\jobname{#1}
        }
      \fi
      \expandafter
    \endgroup
    \childdoctmp
  \fi
}
%    \end{macrocode}

% \macro{\childdocof}
% The command |\childdocof| redirects
% compilation to the main file |#1|.
%    \begin{macrocode}
\newcommand{\childdocof}[1]
{
  \childdocdisable
  \childdoctrue
  \includeonly{\childdocname}
  \def\jobname{#1}
  \def\childdocjob{#1}
  \input{#1}
}
%    \end{macrocode}

% \macro{\childdocby}
% The command |\childdocby| ....
%    \begin{macrocode}
\newcommand{\childdocby}[2][]
{
  \childdocdisable
  \childdoctrue
  \childdocmanualtrue
  \if?#1?\else
    \def\jobname{#2}
  \fi
  \def\childdocjob{#2}
  \input{#2}
  \endinput
}
%    \end{macrocode}

% \macro{\childdocforward}
% The command |\childdocforward| redirects
% compilation to the main file or
% (if the optional argument is given) a child file.
% Parameters are set as if the main file
% or a child file starting with |\childdocof| was compiled.
% Then compilation is handed over to the main file:
%    \begin{macrocode}
\newcommand{\childdocforward}[2][]
{
  \begingroup
    \if?#1?
      \def\childdoctmp
      {
        \def\childdocname{#2}
        \def\childdocjob{#2}
        \def\jobname{#2}
        \input{#2}
        \endinput
      }
    \else
      \def\childdoctmp
      {
        \childdocdisable
        \def\childdocname{#2}
        \childdoctrue
        \includeonly{#2}
        \def\childdocjob{#1}
        \def\jobname{#1}
        \input{#1}
        \endinput
      }
    \fi
    \expandafter
  \endgroup
  \childdoctmp
}
%    \end{macrocode}

% \macro{\childdocforwardprefix}
% The command |\childdocforwardprefix| redirects
% compilation to the main or a child file by means of a pattern.
% The prefix |#1| in the current filename is replaced by |#2|
% and the suffix of the current filename is kept
% (it is assumed that the filename does not contain the substring `|~~~|'
% which is used as a delimiter).
% Compilation is handed over to the new file by |\childdocforward|:
%    \begin{macrocode}
\newcommand{\childdocforwardprefix}[3][]
{
  \begingroup
    \def\childdocextract #2##1~~~{\def\childdoctmp{\childdocforward[#1]{#3##1}}}
    \expandafter\childdocextract\childdocname~~~
    \expandafter
  \endgroup
  \childdoctmp
}
%    \end{macrocode}

% \macro{\childdoc}
% The deprecated macro |\childdoc| is a legacy version of |\childdocmain|:
%    \begin{macrocode}
\newcommand{\childdoc}{\childdocmain}
%    \end{macrocode}

% \macro{\childdocredirect}
% The deprecated macro |\childdocredirect| is a legacy version
% of |\childdocforward| and |\childdocforwardprefix|:
%    \begin{macrocode}
\newcommand{\childdocredirect}[2][]
{
  \begingroup
    \if?#1?
      \def\childdoctmp{\childdocforward{#2}}
    \else
      \def\childdoctmp{\childdocforwardprefix{#1}{#2}}
    \fi
    \expandafter
  \endgroup
  \childdoctmp
}
%    \end{macrocode}

%\iffalse
%</package>
%\fi
%
\endinput
\childdocforward[cdocsamp]{cdocsch1}"|\\
% |latex -jobname cdocscl2 \|\\
% |  "\def\version{final}% \iffalse
%
% childdoc.dtx Copyright (C) 2017-2018 Niklas Beisert
%
% This work may be distributed and/or modified under the
% conditions of the LaTeX Project Public License, either version 1.3
% of this license or (at your option) any later version.
% The latest version of this license is in
%   http://www.latex-project.org/lppl.txt
% and version 1.3 or later is part of all distributions of LaTeX
% version 2005/12/01 or later.
%
% This work has the LPPL maintenance status `maintained'.
%
% The Current Maintainer of this work is Niklas Beisert.
%
% This work consists of the files childdoc.dtx and childdoc.ins
% and the derived files childdoc.def and cdocsamp.tex with
% cdocsch1.tex, cdocsch2.tex, cdocsdrf.tex, cdocsfn1.tex, cdocsfn2.tex.
%
%<package>\ifdefined\childdocmain\endinput\fi
%<package>\ProvidesFile{childdoc.def}[2018/12/30 v2.0 child document driver]
%<samplemain>\ProvidesFile{cdocsamp.tex}[2018/12/30 v2.0 sample for childdoc]
%<*driver>
%\ProvidesFile{childdoc.drv}[2018/12/30 v2.0 childdoc reference manual file]
\PassOptionsToClass{10pt,a4paper}{article}
\documentclass{ltxdoc}

\usepackage[margin=35mm]{geometry}
\usepackage{hyperref}
\usepackage{hyperxmp}
\usepackage[usenames]{color}

\hypersetup{colorlinks=true}
\hypersetup{pdfstartview=FitH}
\hypersetup{pdfpagemode=UseNone}
\hypersetup{pdfsource={}}
\hypersetup{pdflang={en-UK}}
\hypersetup{pdfcopyright={Copyright 2017-2018 Niklas Beisert.
  This work may be distributed and/or modified under the
  conditions of the LaTeX Project Public License, either version 1.3
  of this license or (at your option) any later version.}}
\hypersetup{pdflicenseurl={http://www.latex-project.org/lppl.txt}}
\hypersetup{pdfcontactaddress={ETH Zurich, ITP, HIT K,
  Wolfgang-Pauli-Strasse 27}}
\hypersetup{pdfcontactpostcode={8093}}
\hypersetup{pdfcontactcity={Zurich}}
\hypersetup{pdfcontactcountry={Switzerland}}
\hypersetup{pdfcontactemail={nbeisert@itp.phys.ethz.ch}}
\hypersetup{pdfcontacturl={http://people.phys.ethz.ch/\xmptilde nbeisert/}}

\newcommand{\secref}[1]{\hyperref[#1]{section \ref*{#1}}}

\parskip1ex
\parindent0pt
\let\olditemize\itemize
\def\itemize{\olditemize\parskip0pt}

\begin{document}

\title{The \textsf{childdoc} Package}
\hypersetup{pdftitle={The childdoc Package}}
\author{Niklas Beisert\\[2ex]
  Institut f\"ur Theoretische Physik\\
  Eidgen\"ossische Technische Hochschule Z\"urich\\
  Wolfgang-Pauli-Strasse 27, 8093 Z\"urich, Switzerland\\[1ex]
  \href{mailto:nbeisert@itp.phys.ethz.ch}
  {\texttt{nbeisert@itp.phys.ethz.ch}}}
\hypersetup{pdfauthor={Niklas Beisert}}
\hypersetup{pdfsubject={Manual for the LaTeX2e Package childdoc}}
\date{30 December 2018, \textsf{v2.0}}
\maketitle

\begin{abstract}\noindent
\textsf{childdoc} is a \LaTeXe{} package
that enables the direct compilation
of document sections included by |\include|
to individual files.
\end{abstract}

\begingroup
\parskip0ex
\tableofcontents
\endgroup

%%%%%%%%%%%%%%%%%%%%%%%%%%%%%%%%%%%%%%%%%%%%%%%%%%%%%%%%%%%%%%%%%%%%%%%%%%%%%%%%
%%%%%%%%%%%%%%%%%%%%%%%%%%%%%%%%%%%%%%%%%%%%%%%%%%%%%%%%%%%%%%%%%%%%%%%%%%%%%%%%
\section{Introduction}

\LaTeX{} provides a mechanism to structure a large document (such as a book)
into a main file and several child files (containing the chapters)
using the |\include| command.
This mechanism is beneficial for documents
which span hundreds of pages in order to
make the source file(s) more manageable.
Moreover, compilation can be restricted to
selected child files by means of the |\includeonly| command.
The latter feature can be used to reduce the compilation time while editing
(this was significantly more useful in the earlier days of \LaTeX{})
or to generate a smaller document which is easier to navigate.
Another application of |\includeonly| is to generate
documents consisting of selected parts of the complete document.

However, there are a few drawbacks of the plain |\include| mechanism:
\begin{itemize}
\item
The child files cannot be compiled on their own,
they can only be compiled via the main file.
A naive editing environment
(such as a text editor with an option
to have the current file processed by \LaTeX)
may require one to switch to the main file before compiling;
attempting to compile the child file produces errors.
\item
The main file must be modified (each time)
to adjust the |\includeonly| command
to the present needs. This easily leaves the main file in a messy state.
\item
The generated document will always carry the filename
of the main document. This is inconvenient if
several child files are to be compiled and
to be kept for distribution.
\end{itemize}

The present package provides a simple interface
to make child files individually compilable by \LaTeX{}.
Compiling a child file then has the same effect as compiling
the main file with an |\includeonly| command
to select the appropriate child.
Moreover the generated document will carry the name of the child
rather than the main file.
This resolves all three above issues.

This feature is meant to make the editing of books,
thesis documents and lecture notes somewhat more convenient.
However, the package can also be used efficiently for
composing a series of documents (such as exercise sheets)
which are typically distributed individually.
It then assists the author in generating the individual documents
(potentially in different versions)
as well as a document containing the collected series.
Another application is in developing style files
or other kinds of included material
where compilation of the style file could redirect
to a sample or test file.

%%%%%%%%%%%%%%%%%%%%%%%%%%%%%%%%%%%%%%%%%%%%%%%%%%%%%%%%%%%%%%%%%%%%%%%%%%%%%%%%
%%%%%%%%%%%%%%%%%%%%%%%%%%%%%%%%%%%%%%%%%%%%%%%%%%%%%%%%%%%%%%%%%%%%%%%%%%%%%%%%
\section{Usage}

First of all, the package \textsf{childdoc} is \emph{not} a standard
\LaTeXe{} |.sty| style file! Therefore it needs to be invoked in
a non-standard way.

%%%%%%%%%%%%%%%%%%%%%%%%%%%%%%%%%%%%%%%%%%%%%%%%%%%%%%%%%%%%%%%%%%%%%%%%%%%%%%%%
\subsection{Included Files}
\label{sec:include}

%%%%%%%%%%%%%%%%%%%%%%%%%%%%%%%%%%%%%%%%
\DescribeMacro{\childdocmain}
To use the package, add the commands
\begin{center}
\begin{tabular}{l}
|\input{childdoc.def}|\\
|\childdocmain{}|\\
\end{tabular}
\end{center}
at the very top of the main \LaTeX{} file,
in particular \emph{before} the |\documentclass| statement!
The argument of |\childdocmain| should be left empty
(but it must be present).

%%%%%%%%%%%%%%%%%%%%%%%%%%%%%%%%%%%%%%%%
\DescribeMacro{\childdocof}
Furthermore, add the commands
\begin{center}
\begin{tabular}{l}
|\input{childdoc.def}|\\
|\childdocof{|\textit{main}|}|\\
\end{tabular}
\end{center}
at the top of every child file \textit{child}
which is included by |\include{|\textit{child}|}|
from within the main file
(or at least for those files to be compiled individually).
The argument \textit{main} must be the filename of the main file.

There are a couple of
considerations in setting up the main and child documents:

%%%%%%%%%%%%%%%%%%%%%%%%%%%%%%%%%%%%%%%%
\paragraph{Restrictions.}

Please note the following restrictions:
\begin{itemize}
\item
|\childdocmain| must be called with one argument \textit{main}
to ensure compatibility with earlier version of the package.
It must either be empty (|\childdocmain{}|)
or precisely match the filename of the main file in which it is specified.
See \secref{sec:detection} for further information.
\item
The filename \textit{main} must be specified without the |.tex| extension.
\item
The filename \textit{main} is case sensitive
(even in case-insensitive file systems)
due to internal string comparison.
\item
The argument \textit{main} should be fully expanded, it cannot be a macro.
\item
Subdirectories and special characters should be avoided in filenames.
\item
The command |\childdocmain{|\textit{main}|}| must be followed by a whitespace.
It should not be followed immediately by another command
or by a comment mark `|%|'.
This is because the \TeX{} parser reads the token immediately following
the argument of |\childdocmain| and puts it
at the beginning of every child section;
however, a white\-space is ignored.
\end{itemize}

%%%%%%%%%%%%%%%%%%%%%%%%%%%%%%%%%%%%%%%%
\paragraph{Content of Main File.}

It is advisable to place all content in the child files included by |\include|.
Any output contained in the main file will appear in all child documents
unless suppressed manually;
it cannot be suppressed automatically by the |\includeonly| directive
and thus should normally be avoided.
A method to include some content in the main file
by means of conditional processing is described in \secref{sec:conditional}.

%%%%%%%%%%%%%%%%%%%%%%%%%%%%%%%%%%%%%%%%
\paragraph{Page Numbering.}

When only a part of the document is compiled,
the appropriate numbering of pages
(as well as other status parameters)
is determined from the |.aux| files.
The latter contain information from previous passes.
However this information needs to propagate through
all intermediate child documents.
Therefore the page numbering in child documents may well
be inconsistent until the complete document is compiled at least once.

A useful (if unconventional) way to always ensure a consistent
page numbering is to restart the numbering in each child document
and denote the pages by `\textit{child}|.|\textit{page}'
where \textit{child} represents the chapter/section number of the child file.
This can be achieved by the command
|\numberwithin{page}{|\textit{child}|}|
of the \textsf{amsmath} package
where \textit{child} can be |chapter| or |section|
depending on the chosen structuring.
Alternatively, one can modify the macro |\thepage| appropriately
and reset the counter |page| at the start of each child file.

%%%%%%%%%%%%%%%%%%%%%%%%%%%%%%%%%%%%%%%%%%%%%%%%%%%%%%%%%%%%%%%%%%%%%%%%%%%%%%%%
\subsection{Conditional Processing}
\label{sec:conditional}

The package provides a mechanism to compile different versions
of a document. To customise the versions further some conditional processing
can come in handy to distinguish which version is being compiled.
The package provides two macros to describe the compilation context:

%%%%%%%%%%%%%%%%%%%%%%%%%%%%%%%%%%%%%%%%
\DescribeMacro{\ifchilddoc}
The conditional |\ifchilddoc| distinguishes between the compilation of
child documents and the main document:
%
\begin{center}
|\ifchilddoc |\textit{child-code}| |[|\||else |\textit{main-code}]| \||fi|
\end{center}

%%%%%%%%%%%%%%%%%%%%%%%%%%%%%%%%%%%%%%%%
\DescribeMacro{\childdocname}
\DescribeMacro{\childdocjob}
The macro |\childdocname| contains the filename (without extension)
of the main or child file being processed.
Note that |\childdocjob| will always contain the name of the main file.

%%%%%%%%%%%%%%%%%%%%%%%%%%%%%%%%%%%%%%%%
\paragraph{Title Page.}

Conditional processing can be used to include a title or banner page
in the main document when proper precautions are taken.
Importantly, the code in the main file should ensure that the page counter
(as well as other status parameters which are stored in the |.aux| files)
takes the same value after the conditional processing.
Otherwise the page numbers may take divergent values
depending on which part is compiled.

For example, a title page could be declared by:
%
\begin{center}
\begin{tabular}{l}
|\ifchilddoc\||else|\\
|\addtocounter{page}{-1}|\\
\textit{code for title page}\\
|\newpage|\\
|\||fi|
\end{tabular}
\end{center}
%
A banner page for the child documents can be generated by:
%
\begin{center}
\begin{tabular}{l}
|\ifchilddoc|\\
|\addtocounter{page}{-1}|\\
\textit{code for banner page}\\
|\newpage|\\
|\||fi|
\end{tabular}
\end{center}
%
Here one could write a message such as:
\begin{center}
|This is the part \childdocname{} of \childdocjob{}.|
\end{center}

%%%%%%%%%%%%%%%%%%%%%%%%%%%%%%%%%%%%%%%%%%%%%%%%%%%%%%%%%%%%%%%%%%%%%%%%%%%%%%%%
\subsection{Flags}
\label{sec:flags}

The package makes it easy to generate different versions
of the main or child documents.
To this end compilation flags can be defined
and assigned different default values.
They will be particularly useful in conjunction
with the forwarding mechanism described in \secref{sec:forward}.

For example, it may be useful to have a flag |\version|
which can be set to |draft| or |final|.
The document source will contain some conditional code
depending on the value of |\version|.
Suppose further, the flag should default to |final| for the main file
and to |draft| for child files
which is a natural assignment for editing the document.
This is achieved by placing the following code
in the preamble of the main document
(below the |\childdocmain| directive):
%
\begin{center}
\begin{tabular}{l}
|\ifchilddoc|\\
|\providecommand{\version}{draft}|\\
|\||else|\\
|\providecommand{\version}{final}|\\
|\||fi|
\end{tabular}
\end{center}
%
The definition by |\providecommand| makes sure
that previous definitions are not overwritten.
Further statements |\providecommand{\version}{...}|
can thus be added before the above code to override it.

For the main file, one might add a line
(between |\childdocmain| and the above block)
%
\begin{center}
|%\ifchilddoc\||else\providecommand{\version}{draft}\||fi|
\end{center}
%
which can be uncommented to produce a draft version.
Likewise one can add a line to the very top of a child file
(above the |\childdocof{|\textit{main}|}| directive)
%
\begin{center}
|%\providecommand{\version}{final}|
\end{center}
%
which can be uncommented to produce the final version of this child document.

%%%%%%%%%%%%%%%%%%%%%%%%%%%%%%%%%%%%%%%%%%%%%%%%%%%%%%%%%%%%%%%%%%%%%%%%%%%%%%%%
\subsection{Forwarding}
\label{sec:forward}

Different versions of the main or child documents
using compilation flags as described in \secref{sec:flags}
can be (permanently) stored in different files
for convenient compilation, viewing and distribution.
To this end, the package defines a command
to pass on compilation to a different file:

%%%%%%%%%%%%%%%%%%%%%%%%%%%%%%%%%%%%%%%%
\DescribeMacro{\childdocforward}
The command |\childdocforward| redirects processing to
another source file:
%
\begin{center}
\begin{tabular}{l}
|\input{childdoc.def}|\\
|\childdocforward[|\textit{main}|]{|\textit{dest}|}|\\
\end{tabular}
\end{center}
%
The argument \textit{dest} is the destination file
(without extension).
It should be the main file or one of the child files.
Note that further \textsf{childdoc} directives
such as |\childdocof| and |\childdocforward|
in the indicated file will be processed in this form.
The optional argument \textit{main}
passes on directly to the main file \textit{main}
while pretending to compile the child \textit{dest}.
This form behaves as if \textit{dest}
issues |\childdocof{|\textit{main}|}| right away,
and no further \textsf{childdoc} directives will be processed.

%%%%%%%%%%%%%%%%%%%%%%%%%%%%%%%%%%%%%%%%
\DescribeMacro{\...prefix}
In the alternative form |\childdocforwardprefix|,
%
\begin{center}
\begin{tabular}{l}
|\input{childdoc.def}|\\
|\childdocforwardprefix[|\textit{main}|]{|\textit{prefix}|}{|\textit{dest}|}|
\end{tabular}
\end{center}
%
the destination file is determined by a pattern
depending on the current file:
To make this work, the current file must be called
`{\textit{prefix}\hspace{0.2em}\textit{suffix}}'
with \textit{prefix} matching precisely the argument.
Processing is then passed on to the file
`{\textit{dest}\hspace{0.2em}\textit{suffix}}'.
Surely, the same effect is achieved by
directly specifying the
argument `{\textit{dest}\hspace{0.2em}\textit{suffix}}'
in the first form.
However, that requires to set up a different file
for each child. With the alternative form of the command
all these files can have exactly the same content
which simplifies setting them up and maintaining them.

For example, the following file |draft.tex|
with a compilation flag |\version| as described in \secref{sec:flags}
compiles the main document as a draft:
%
\begin{center}
\begin{tabular}{l}
|\def\version{draft}|\\
|\input{childdoc.def}|\\
|\childdocforward{|\textit{main}|}|
\end{tabular}
\end{center}
%
Likewise, the following files |final|\textit{nn}|.tex|
compile the final version of the child document
|child|\textit{nn}|.tex|:
%
\begin{center}
\begin{tabular}{l}
|\def\version{final}|\\
|\input{childdoc.def}|\\
|\childdocforwardprefix{final}{child}|
\end{tabular}
\end{center}
%

Note that when several versions of a main file and/or of each child file
are to be generated, it may be convenient to set up a |Makefile| or
shell script to automatise the process.

%%%%%%%%%%%%%%%%%%%%%%%%%%%%%%%%%%%%%%%%%%%%%%%%%%%%%%%%%%%%%%%%%%%%%%%%%%%%%%%%
\subsection{Command Line Processing}
\label{sec:commandline}

The effect of redirection files can also be achieved by invoking
the \LaTeX{} compiler with a more elaborate command line.
Most conveniently this should be done as part
of a shell script or a |Makefile|.

When using \textsf{childdoc} in the main file, the following
command lines effectively perform a redirection
(note that depending on the shell being used,
backslashes may have to be doubled: `|\|' $\to$ `|\\|'):
%
\begin{center}
|... -jobname "|\textit{target}|" |\\|"|[\textit{flags}]%
|\input{childdoc.def}\childdocforward[|\textit{main}|]{|\textit{dest}|}"|
\end{center}
%
Here \textit{target} is the name of the output file,
\textit{main} is the name of the main file
and \textit{dest} is the name of the main or child file to be processed
(all filenames without extensions).
The optional argument \textit{main} can be omitted
if \textit{main} matches \textit{dest}.
Optionally, compilation \textit{flags} can be defined via |\def| commands.
This command line makes the \TeX{} engine believe
it is compiling the file \textit{target}
whose content is specified as the latter parameter.
The provided code then forwards the processing to
\textit{main} or \textit{dest} as described in \secref{sec:forward}.

%%%%%%%%%%%%%%%%%%%%%%%%%%%%%%%%%%%%%%%%%%%%%%%%%%%%%%%%%%%%%%%%%%%%%%%%%%%%%%%%
\subsection{Include by Input}
\label{sec:input}

Including child documents by |\include| has some restrictions by design.
Most notably, the content of a child document always occupies
its own set of pages; pages cannot be shared between child documents.
Usually, this behaviour makes perfect sense
because each child document contain an essential part of the document.
However, in some situations it may be desirable to compose
a document from a collection of parts
without having mandatory page breaks between then.
For this case, the package
provides a mechanism to include parts
by |\input| which can also be processed individually.
However, by construction this mechanism
requires manual handling of the content to be output.

%%%%%%%%%%%%%%%%%%%%%%%%%%%%%%%%%%%%%%%%
\DescribeMacro{\ifchilddocmanual}
The main file should be prepared as usual, see \secref{sec:include}.
However, the document body must make a distinction
between processing of an individual part and of the main document, e.g.:
%
\begin{center}
\begin{tabular}{l}
|\ifchilddocmanual|\\
|\input{\childdocname}|\\
|\||else|\\
\textit{document body with }|\input{|\textit{part}|}|\\
|\||fi|
\end{tabular}
\end{center}
%
The conditional |\ifchilddocmanual| is true whenever
a part to be included by |\input| is being compiled,
and the name of the part is stored in |\childdocname|.

%%%%%%%%%%%%%%%%%%%%%%%%%%%%%%%%%%%%%%%%
\DescribeMacro{\childdocby}
Each part to be included by |\input| should start with:
%
\begin{center}
\begin{tabular}{l}
|\input{childdoc.def}|\\
|\childdocby{|\textit{main}|}|\\
\end{tabular}
\end{center}
%
The directive |\childdocby| is similar to |\childdocof|
described in \secref{sec:include},
but the subsequent selection of content must be done manually.
To that end, both |\ifchilddoc| and |\ifchilddocmanual|
will be true upon processing of a part,
and the name of the part is stored in |\childdocname|.
Note that |\jobname| will be set to the filename of the current part
so that each part receives an individual |.aux| file
that does not interfere with the |.aux| file(s) of the main document.
This behaviour can be altered by the alternative form
|\childdocby[*]{|\textit{main}|}| (with a non-empty optional argument)
which uses the |.aux| file of the main document
by setting |\jobname| to \textit{main}.

%%%%%%%%%%%%%%%%%%%%%%%%%%%%%%%%%%%%%%%%%%%%%%%%%%%%%%%%%%%%%%%%%%%%%%%%%%%%%%%%
\subsection{Driver Development}
\label{sec:driver}

The \textsf{childdoc} mechanism can also be use for the development
of definition files such as \LaTeX{} styles or classes.
This case differs from the above setup with multiple parts
included by |\include| in that no |\includeonly| should be invoked.
This can be achieved by starting the include file
(before |\ProvidesPackage|) with:
%
\begin{center}
\begin{tabular}{l}
|\input{childdoc.def}|\\
|\childdocforward{|\textit{main}|}|\\
\end{tabular}
\end{center}
%
or alternatively with:
%
\begin{center}
\begin{tabular}{l}
|\input{childdoc.def}|\\
|\childdocby{|\textit{main}|}|\\
\end{tabular}
\end{center}
%
Both forms have slightly different effects as described above.
The main file is prepared as usual, see \secref{sec:include}.

%%%%%%%%%%%%%%%%%%%%%%%%%%%%%%%%%%%%%%%%%%%%%%%%%%%%%%%%%%%%%%%%%%%%%%%%%%%%%%%%
\subsection{Legacy Detection}
\label{sec:detection}

The directive |\childdocmain| in the main file can detect
whether the complete document or merely a child is to be compiled
even without using the directive |\childdocof|.
This method is deprecated because it is less robust
and there is no compelling reason to use it;
it is merely provided for backward compatibility
and it may be removed in future versions.

If the detection mechanism is to be used,
it is mandatory to correctly specify
the filename of the main file as the argument of |\childdocmain|:
%
\begin{center}
\begin{tabular}{l}
|\input{childdoc.def}|\\
|\childdocmain{|\textit{main}|}|\\
\end{tabular}
\end{center}
%
If |\jobname| does not match the argument \textit{main} of |\childdocmain|,
it is assumed that |\jobname| points to the child file to be compiled.
When using |\childdocmain| with the main file specified as argument,
it suffices to start a child file
with just |\input{|\textit{main}|}|
without loading of the package and using |\childdocof|.
If instead all processing is done
with the appropriate \textsf{childdoc} directives,
the argument of \textit{main} of |\childdocmain| can be empty.

An alternative version of the command line processing described
in \secref{sec:commandline} using the detection mechanism reads:
%
\begin{center}
|... -jobname "|\textit{target}|" "|[\textit{flags}]%
[|\def\jobname{|\textit{dest}|}|]|\input{|\textit{main}|}"|
\end{center}

%%%%%%%%%%%%%%%%%%%%%%%%%%%%%%%%%%%%%%%%%%%%%%%%%%%%%%%%%%%%%%%%%%%%%%%%%%%%%%%%
\subsection{Manual Code}
\label{sec:manual}

In case one cannot be certain whether the definitions file |childdoc.def|
is installed on the target \TeX{} distribution
and one prefers not to ship it,
it is conceivable to paste a few relevant commands into the sources.

To that end, drop all statements |\input{childdoc.def}|
and perform the replacements as outlined below.
Instead of |\childdocmain{|\textit{main}|}| add the following code
to the top of the main file:
%
\begin{center}
\begin{tabular}{l}
|\||ifdefined\childdocname\endinput\||fi\newif\ifchilddoc|\\
|\edef\childdocname{\scantokens\expandafter{\jobname\noexpand}}|\\
|\def\childdocmain{|\textit{main}|}\||ifx\childdocmain\childdocname\||else|\\
|\childdoctrue\includeonly{\childdocname}\let\jobname\childdocmain\||fi|\\
\end{tabular}
\end{center}
%
Instead of |\childdocof{|\textit{main}|}| just include the main file
at the top of each child file:
%
\begin{center}
|\input{|\textit{main}|}|
\end{center}
%
A simple redirection |\childdocforward{|\textit{dest}|}| is achieved by:
%
\begin{center}
|\def\jobname{|\textit{dest}|}\input{\jobname}|
\end{center}
%
The redirection with prefix
|\childdocforwardprefix[|\textit{prefix}|]{|\textit{dest}|}|
is accomplished by:
%
\begin{center}
\begin{tabular}{l}
|{\edef\jobname{\scantokens\expandafter{\jobname\noexpand}}|\\
|\def\redirectjob |\textit{prefix}|#1~~~{\gdef\jobname{|\textit{dest}|#1}}|\\
|\expandafter\redirectjob\jobname~~~}\input{\jobname}|
\end{tabular}
\end{center}

In an alternative approach,
child documents can be compiled by a specific command line
without additional code or specific definitions:
%
\begin{center}
|... -jobname "|\textit{target}|" "|[\textit{flags}]%
|\includeonly{|\textit{dest}|}\input{|\textit{main}|}"|
\end{center}
%

%%%%%%%%%%%%%%%%%%%%%%%%%%%%%%%%%%%%%%%%%%%%%%%%%%%%%%%%%%%%%%%%%%%%%%%%%%%%%%%%
%%%%%%%%%%%%%%%%%%%%%%%%%%%%%%%%%%%%%%%%%%%%%%%%%%%%%%%%%%%%%%%%%%%%%%%%%%%%%%%%
\section{Information}

%%%%%%%%%%%%%%%%%%%%%%%%%%%%%%%%%%%%%%%%%%%%%%%%%%%%%%%%%%%%%%%%%%%%%%%%%%%%%%%%
\subsection{Copyright}

Copyright \copyright{} 2017--2018 Niklas Beisert

This work may be distributed and/or modified under the
conditions of the \LaTeX{} Project Public License, either version 1.3
of this license or (at your option) any later version.
The latest version of this license is in
  \url{http://www.latex-project.org/lppl.txt}
and version 1.3 or later is part of all distributions of \LaTeX{}
version 2005/12/01 or later.

This work has the LPPL maintenance status `maintained'.

The Current Maintainer of this work is Niklas Beisert.

This work consists of the files |README.txt|, |childdoc.ins| and |childdoc.dtx|
as well as the derived files |childdoc.def|, |cdocsamp.tex|
with |cdocsch1.tex|, |cdocsch2.tex|, |cdocspt3.tex|, |cdocspt4.tex|,
|cdocsdrf.tex|, |cdocsfn1.tex|, |cdocsfn2.tex|
as well as |childdoc.pdf|.

%%%%%%%%%%%%%%%%%%%%%%%%%%%%%%%%%%%%%%%%%%%%%%%%%%%%%%%%%%%%%%%%%%%%%%%%%%%%%%%%
\subsection{Files and Installation}

The package consists of the files:
%
\begin{center}
\begin{tabular}{ll}
    |README.txt|   & readme file \\
    |childdoc.ins| & installation file \\
    |childdoc.dtx| & source file \\
    |childdoc.def| & definition file \\
    |cdocsamp.tex| & sample main file \\
    |cdocsch1.tex| & sample include file \\
    |cdocsch2.tex| & sample include file \\
    |cdocspt3.tex| & sample part file \\
    |cdocspt4.tex| & sample part file \\
    |cdocsdrf.tex| & sample redirection file \\
    |cdocsfn1.tex| & sample redirection file \\
    |cdocsfn2.tex| & sample redirection file \\
    |childdoc.pdf| & manual
\end{tabular}
\end{center}
%
The distribution consists of the files
|README.txt|, |childdoc.ins| and |childdoc.dtx|.
%
\begin{itemize}
\item
Run (pdf)\LaTeX{} on |childdoc.dtx|
to compile the manual |childdoc.pdf| (this file).
\item
Run \LaTeX{} on |childdoc.ins| to create the definitions file |childdoc.def|
and the sample |cdocsamp.tex| with include files
|cdocsch1.tex|, |cdocsch2.tex|, |cdocspt3.tex|, |cdocspt4.tex|,
|cdocsdrf.tex|, |cdocsfn1.tex|, |cdocsfn2.tex|.
Then copy the file |childdoc.def| to an appropriate directory of your \LaTeX{}
distribution, e.g.\ \textit{texmf-root}|/tex/latex/childdoc|.
\end{itemize}

%%%%%%%%%%%%%%%%%%%%%%%%%%%%%%%%%%%%%%%%%%%%%%%%%%%%%%%%%%%%%%%%%%%%%%%%%%%%%%%%
\subsection{Related CTAN Packages}

There are several other packages which offer a similar functionality:
%
\begin{itemize}
\item
The packages
\href{http://ctan.org/pkg/docmute}{\textsf{docmute}},
\href{http://ctan.org/pkg/includex}{\textsf{includex}} and
\href{http://ctan.org/pkg/standalone}{\textsf{standalone}}
provide commands to include only the document body of
a child file thus allowing both files to be compiled individually.
\item
The packages \href{http://ctan.org/pkg/subdocs}{\textsf{subdocs}}
and \href{http://ctan.org/pkg/subfiles}{\textsf{subfiles}}
provide structures in which the main and child documents can be
encapsulated and allowing them to be compiled individually.
The inclusion mechanism is different from the conventional |\include|.
\item
The package \href{http://ctan.org/pkg/combine}{\textsf{combine}}
is an elaborate solution to combine several documents into one.
\end{itemize}
%
See also the CTAN topic \href{http://ctan.org/topic/subdocs}{\textsf{subdocs}}
for further related packages.
The present package differs from the above solutions in that
a document structure constructed with the conventional |\include| mechanism
just needs two extra commands at the top of every file
such that all constituent files can be compiled individually.

%%%%%%%%%%%%%%%%%%%%%%%%%%%%%%%%%%%%%%%%%%%%%%%%%%%%%%%%%%%%%%%%%%%%%%%%%%%%%%%%
%\subsection{Feature Suggestions}
%
%The following is a list of features which may be useful for future
%versions of this package:
%%
%\begin{itemize}
%\item
%\ldots
%\end{itemize}

%%%%%%%%%%%%%%%%%%%%%%%%%%%%%%%%%%%%%%%%%%%%%%%%%%%%%%%%%%%%%%%%%%%%%%%%%%%%%%%%
\subsection{Revision History}

%%%%%%%%%%%%%%%%%%%%%%%%%%%%%%%%%%%%%%%%
\paragraph{v2.0:} 2018/12/30

\begin{itemize}
\item
immediate forward processing
\item
added |\childdocby| mechanism
\item
manual restructured
\end{itemize}

%%%%%%%%%%%%%%%%%%%%%%%%%%%%%%%%%%%%%%%%
\paragraph{v1.6:} 2018/01/17

\begin{itemize}
\item
application for development of include files
\item
corrections to manual
\end{itemize}

%%%%%%%%%%%%%%%%%%%%%%%%%%%%%%%%%%%%%%%%
\paragraph{v1.5:} 2017/05/21

\begin{itemize}
\item
more complete structuring introduced
\item
|\childdocof| introduced
\item
|\childdoc| renamed to |\childdocmain|
\item
|\childredirect| renamed to |\childdocforward| and |\childdocforwardprefix|
and functionality expanded
\end{itemize}

%%%%%%%%%%%%%%%%%%%%%%%%%%%%%%%%%%%%%%%%
\paragraph{v1.0:} 2017/04/27

\begin{itemize}
\item
manual and install package
\item
first version published on CTAN
\end{itemize}

%%%%%%%%%%%%%%%%%%%%%%%%%%%%%%%%%%%%%%%%
\paragraph{v0.6:} 2017/04/26

\begin{itemize}
\item
redirection mechanism added
\end{itemize}

%%%%%%%%%%%%%%%%%%%%%%%%%%%%%%%%%%%%%%%%
\paragraph{v0.5:} 2017/04/26

\begin{itemize}
\item
functionality in definition file
\end{itemize}


%%%%%%%%%%%%%%%%%%%%%%%%%%%%%%%%%%%%%%%%%%%%%%%%%%%%%%%%%%%%%%%%%%%%%%%%%%%%%%%%
%%%%%%%%%%%%%%%%%%%%%%%%%%%%%%%%%%%%%%%%%%%%%%%%%%%%%%%%%%%%%%%%%%%%%%%%%%%%%%%%
%%%%%%%%%%%%%%%%%%%%%%%%%%%%%%%%%%%%%%%%%%%%%%%%%%%%%%%%%%%%%%%%%%%%%%%%%%%%%%%%
\appendix

\settowidth\MacroIndent{\rmfamily\scriptsize 000\ }

 \DocInput{childdoc.dtx}

\end{document}
%</driver>
% \fi
%
% %%%%%%%%%%%%%%%%%%%%%%%%%%%%%%%%%%%%%%%%%%%%%%%%%%%%%%%%%%%%%%%%%%%%%%%%%%%%%%
% %%%%%%%%%%%%%%%%%%%%%%%%%%%%%%%%%%%%%%%%%%%%%%%%%%%%%%%%%%%%%%%%%%%%%%%%%%%%%%
% \section{Sample}
%\iffalse
%<*samplemain>
%\fi
%
% The following presents a sample document
% with two chapters, two parts, a title page,
% a compile flag as well as three forwarding files to set the flag.
% It consists of eight |.tex| files:
% \begin{center}
% \begin{tabular}{ll}
% |cdocsamp.tex|&main file\\
% |cdocsch1.tex|&include file for chapter 1\\
% |cdocsch2.tex|&include file for chapter 2\\
% |cdocspt3.tex|&include file for part 3\\
% |cdocspt4.tex|&include file for part 4\\
% |cdocsdrf.tex|&forwarding file for main file in draft mode\\
% |cdocsfi1.tex|&forwarding file for final version of chapter 1\\
% |cdocsfi2.tex|&forwarding file for final version of chapter 2\\
% \end{tabular}
% \end{center}
% Each of the eight files can be compiled directly by the \LaTeX{} compiler.
%
% %%%%%%%%%%%%%%%%%%%%%%%%%%%%%%%%%%%%%%
% \paragraph{Main File.}
%
% The main file is called |cdocsamp.tex|.
%
% Load the \textsf{childdoc} definitions and
% declare the filename for the main document:
%    \begin{macrocode}
\input{childdoc.def}
\childdocmain{}
%    \end{macrocode}

% Optional override for |\version| flag:
%    \begin{macrocode}
%%\ifchilddoc\else\providecommand{\version}{draft}\fi
%    \end{macrocode}

% Define the default values for the |\version| flag
% (|final| for the main file and |draft| for childs):
%    \begin{macrocode}
\ifchilddoc
\providecommand{\version}{draft}
\else
\providecommand{\version}{final}
\fi
%    \end{macrocode}

% Load the standard document class:
%    \begin{macrocode}
\documentclass[12pt]{article}
%    \end{macrocode}

% Start the document body:
%    \begin{macrocode}
\begin{document}
%    \end{macrocode}

% Declare a title page.
% Print title, part of document being processed and version flag:
%    \begin{macrocode}
\addtocounter{page}{-1}
\begin{center}
{\LARGE\bfseries{}childdoc example\par}
\vspace{1cm}
\ifchilddoc
\ifchilddocmanual part\else chapter\fi:
`\childdocname' of `\childdocjob'\par
\else
main document: `\childdocjob'\par
\fi
version: \version\par
\end{center}
\newpage
%    \end{macrocode}

% Manually include selected file,
% otherwise process as usual:
%    \begin{macrocode}
\ifchilddocmanual
\section*{part `\childdocname'}
\input{\childdocname}
\else
%    \end{macrocode}

% Include the two chapters:
%    \begin{macrocode}
\include{cdocsch1}
\include{cdocsch2}
%    \end{macrocode}

% Include the two parts unless only chapters should be displayed:
%    \begin{macrocode}
\ifchilddoc\else
\section{part three}
\input{cdocspt3}
\section{part four}
\input{cdocspt4}
\fi
%    \end{macrocode}

% Process as usual until here:
%    \begin{macrocode}
\fi
%    \end{macrocode}

% End of document body:
%    \begin{macrocode}
\end{document}
%    \end{macrocode}
%\iffalse
%</samplemain>
%\fi
%
% %%%%%%%%%%%%%%%%%%%%%%%%%%%%%%%%%%%%%%
% \paragraph{Chapter Include Files.}
%
% The include files are called |cdocsch1.tex| and |cdocsch2.tex|.
%
%\iffalse
%<*samplechap1|samplechap2>
%\fi

% Optional override for |\version| flag:
%    \begin{macrocode}
%%\providecommand{\version}{final}
%    \end{macrocode}

% Include the main document:
%    \begin{macrocode}
\input{childdoc.def}
\childdocof{cdocsamp}
%    \end{macrocode}

%\iffalse
%</samplechap1|samplechap2>
%\fi
%
%\iffalse
%<*samplechap1>
%\fi
% Some text for chapter 1:
%    \begin{macrocode}
\section{one}
some text in chapter one
%    \end{macrocode}

%\iffalse
%</samplechap1>
%\fi
% Some text for chapter 2:
%\iffalse
%<*samplechap2>
%\fi
%    \begin{macrocode}
\section{two}
more text in chapter two
%    \end{macrocode}

%\iffalse
%</samplechap2>
%\fi
%
% %%%%%%%%%%%%%%%%%%%%%%%%%%%%%%%%%%%%%%
% \paragraph{Part Include Files.}
%
% The include files are called |cdocspt3.tex| and |cdocspt4.tex|.
%
%\iffalse
%<*samplepart3|samplepart4>
%\fi

% Optional override for |\version| flag:
%    \begin{macrocode}
%%\providecommand{\version}{final}
%    \end{macrocode}

% Include the main document:
%    \begin{macrocode}
\input{childdoc.def}
\childdocby{cdocsamp}
%    \end{macrocode}

%\iffalse
%</samplepart3|samplepart4>
%\fi
%
%\iffalse
%<*samplepart3>
%\fi
% Some text for part 3:
%    \begin{macrocode}
some text in part three
%    \end{macrocode}

%\iffalse
%</samplepart3>
%\fi
% Some text for part 4:
%\iffalse
%<*samplepart4>
%\fi
%    \begin{macrocode}
more text in part four
%    \end{macrocode}

%\iffalse
%</samplepart4>
%\fi
%
% %%%%%%%%%%%%%%%%%%%%%%%%%%%%%%%%%%%%%%
% \paragraph{Forwarding for a Complete Draft.}
%
% The following forwarding file |cdocsdrf.tex|
% compiles the main document in draft mode:
%\iffalse
%<*sampledraft>
%\fi
%    \begin{macrocode}
\def\version{draft}
\input{childdoc.def}
\childdocforward{cdocsamp}
%    \end{macrocode}

%\iffalse
%</sampledraft>
%\fi
%
% %%%%%%%%%%%%%%%%%%%%%%%%%%%%%%%%%%%%%%
% \paragraph{Forwarding for Final Version of the Chapters.}
%
% The following forwarding files |cdocsfn1.tex| and |cdocsfn2.tex|
% (with identical content)
% compile the final versions of the child documents
% |cdocsch1.tex| and |cdocsch2.tex|, respectively:
%\iffalse
%<*samplefinal>
%\fi
%    \begin{macrocode}
\def\version{final}
\input{childdoc.def}
\childdocforwardprefix[cdocsamp]{cdocsfn}{cdocsch}
%    \end{macrocode}

%\iffalse
%</samplefinal>
%\fi
%
% %%%%%%%%%%%%%%%%%%%%%%%%%%%%%%%%%%%%%%
% \paragraph{Command Line Processing.}
%
% The following three command lines generate the output files
% |cdocscld|, |cdocscl1| and |cdocscl2|
% which should be identical to
% |cdocsdrf|, |cdocsch1| and |cdocsfn2|, respectively:
% \begin{center}
% \begin{tabular}{l}
% |latex -jobname cdocscld \|\\
% |  "\def\version{draft}\input{childdoc.def}\childdocforward{cdocsamp}"|\\
% |latex -jobname cdocscl1 \|\\
% |  "\input{childdoc.def}\childdocforward[cdocsamp]{cdocsch1}"|\\
% |latex -jobname cdocscl2 \|\\
% |  "\def\version{final}\input{childdoc.def}\childdocforward{cdocsch2}"|
% \end{tabular}
% \end{center}
% Note that the trailing backslash on each first line
% merely continues the input to the second line
% (for convenient cut ant paste).
% Furthermore, the command |latex| can be replaced by any
% of its alternative versions such as |pdflatex|.
%
% %%%%%%%%%%%%%%%%%%%%%%%%%%%%%%%%%%%%%%%%%%%%%%%%%%%%%%%%%%%%%%%%%%%%%%%%%%%%%%
% %%%%%%%%%%%%%%%%%%%%%%%%%%%%%%%%%%%%%%%%%%%%%%%%%%%%%%%%%%%%%%%%%%%%%%%%%%%%%%
% \section{Implementation}
%\iffalse
%<*package>
%\fi
%
% This section describes the definitions file |childdoc.def|.

% The definitions cannot be loaded using |\usepackage| or |\RequirePackage|
% which has a mechanism to prevent loading a style file more than once.
% When loading the definitions by means of |\input|
% multiple instances have to be prevented manually:
%\iffalse
%This code needs to be before the `\ProvidesFile' directive
%which is defined at the beginning of this file.
%Therefore it is also placed there and commented out here.
%</package>
%<*discard>
%\fi
%    \begin{macrocode}
\ifdefined\childdocmain\endinput\fi
%    \end{macrocode}
%\iffalse
%</discard>
%<*package>
%\fi
%
% \macro{\ifchilddoc}
% \macro{\ifchilddocmanual}
% The conditional |\ifchilddoc| tells whether a
% child (true) or main (false) document is being compiled.
% The conditional |\ifchilddocmanual| tells whether
% the |\includeonly| mechanism is used (false) or
% the selection of child files must be performed manually (true).
% The definitions initialise to false:
%    \begin{macrocode}
\newif\ifchilddoc
\newif\ifchilddocmanual
%    \end{macrocode}

% \macro{\childdocname}
% \macro{\childdocjob}
% The macro |\childdocname| stores the name of the main document
% to be compiled. The macro |\childdocjob| stores the name of
% the document on which the \LaTeX{} compiler was originally invoked.
% The content of |\jobname| cannot be compared
% to filenames specified in the source due to different catcodes.
% The following code rescans |\jobname|, stores the result
% in |\childdocname| and saves a copy in |\childdocjob|:
%    \begin{macrocode}
\edef\childdocname{\scantokens\expandafter{\jobname\noexpand}}
\let\childdocjob\childdocname
%    \end{macrocode}

% \macro{\childdocdisable}
% The macro |\childdocdisable| prevents the main file
% from being processed more than once.
% At this stage, the main document command |\childdocmain|
% is assumed to be called once again where it should do nothing.
% Any subsequent call to it should prevent
% a secondary processing of the main document
% It overwrites the forwarding commands
% |\childdocof| and |\childdocforward|
% with empty macros to prevent further inclusions of the main document:
%    \begin{macrocode}
\newcommand{\childdocdisable}
{
  \renewcommand{\childdocmain}[1]{\renewcommand{\childdocmain}[1]{\endinput}}
  \renewcommand{\childdocof}[1]{}
  \renewcommand{\childdocby}[2][]{}
  \renewcommand{\childdocforward}[2][]{}
  \renewcommand{\childdocdisable}{}
}
%    \end{macrocode}

% \macro{\childdocmain}
% The macro |\childdocmain| is to be called at the top of the main file
% with nothing or the main filename (without extension) as argument.
% First, it breaks loops.
% If the argument is not empty and does not match |\childdocname|
% (which is set by the first inclusion of |childdoc.def|),
% |\ifchilddoc| is set to true, |\includeonly| is applied to the child file
% and |\jobname| is set to the main file
% (for proper handling of |.aux| files):
%    \begin{macrocode}
\newcommand{\childdocmain}[1]
{
  \childdocdisable\childdocmain{}
  \if?#1?\else
    \begingroup
      \def\childdoctmp{#1}
      \ifx\childdoctmp\childdocname
        \def\childdoctmp{}
      \else
        \def\childdoctmp
        {
          \childdoctrue
          \includeonly{\childdocname}
          \def\childdocjob{#1}
          \def\jobname{#1}
        }
      \fi
      \expandafter
    \endgroup
    \childdoctmp
  \fi
}
%    \end{macrocode}

% \macro{\childdocof}
% The command |\childdocof| redirects
% compilation to the main file |#1|.
%    \begin{macrocode}
\newcommand{\childdocof}[1]
{
  \childdocdisable
  \childdoctrue
  \includeonly{\childdocname}
  \def\jobname{#1}
  \def\childdocjob{#1}
  \input{#1}
}
%    \end{macrocode}

% \macro{\childdocby}
% The command |\childdocby| ....
%    \begin{macrocode}
\newcommand{\childdocby}[2][]
{
  \childdocdisable
  \childdoctrue
  \childdocmanualtrue
  \if?#1?\else
    \def\jobname{#2}
  \fi
  \def\childdocjob{#2}
  \input{#2}
  \endinput
}
%    \end{macrocode}

% \macro{\childdocforward}
% The command |\childdocforward| redirects
% compilation to the main file or
% (if the optional argument is given) a child file.
% Parameters are set as if the main file
% or a child file starting with |\childdocof| was compiled.
% Then compilation is handed over to the main file:
%    \begin{macrocode}
\newcommand{\childdocforward}[2][]
{
  \begingroup
    \if?#1?
      \def\childdoctmp
      {
        \def\childdocname{#2}
        \def\childdocjob{#2}
        \def\jobname{#2}
        \input{#2}
        \endinput
      }
    \else
      \def\childdoctmp
      {
        \childdocdisable
        \def\childdocname{#2}
        \childdoctrue
        \includeonly{#2}
        \def\childdocjob{#1}
        \def\jobname{#1}
        \input{#1}
        \endinput
      }
    \fi
    \expandafter
  \endgroup
  \childdoctmp
}
%    \end{macrocode}

% \macro{\childdocforwardprefix}
% The command |\childdocforwardprefix| redirects
% compilation to the main or a child file by means of a pattern.
% The prefix |#1| in the current filename is replaced by |#2|
% and the suffix of the current filename is kept
% (it is assumed that the filename does not contain the substring `|~~~|'
% which is used as a delimiter).
% Compilation is handed over to the new file by |\childdocforward|:
%    \begin{macrocode}
\newcommand{\childdocforwardprefix}[3][]
{
  \begingroup
    \def\childdocextract #2##1~~~{\def\childdoctmp{\childdocforward[#1]{#3##1}}}
    \expandafter\childdocextract\childdocname~~~
    \expandafter
  \endgroup
  \childdoctmp
}
%    \end{macrocode}

% \macro{\childdoc}
% The deprecated macro |\childdoc| is a legacy version of |\childdocmain|:
%    \begin{macrocode}
\newcommand{\childdoc}{\childdocmain}
%    \end{macrocode}

% \macro{\childdocredirect}
% The deprecated macro |\childdocredirect| is a legacy version
% of |\childdocforward| and |\childdocforwardprefix|:
%    \begin{macrocode}
\newcommand{\childdocredirect}[2][]
{
  \begingroup
    \if?#1?
      \def\childdoctmp{\childdocforward{#2}}
    \else
      \def\childdoctmp{\childdocforwardprefix{#1}{#2}}
    \fi
    \expandafter
  \endgroup
  \childdoctmp
}
%    \end{macrocode}

%\iffalse
%</package>
%\fi
%
\endinput
\childdocforward{cdocsch2}"|
% \end{tabular}
% \end{center}
% Note that the trailing backslash on each first line
% merely continues the input to the second line
% (for convenient cut ant paste).
% Furthermore, the command |latex| can be replaced by any
% of its alternative versions such as |pdflatex|.
%
% %%%%%%%%%%%%%%%%%%%%%%%%%%%%%%%%%%%%%%%%%%%%%%%%%%%%%%%%%%%%%%%%%%%%%%%%%%%%%%
% %%%%%%%%%%%%%%%%%%%%%%%%%%%%%%%%%%%%%%%%%%%%%%%%%%%%%%%%%%%%%%%%%%%%%%%%%%%%%%
% \section{Implementation}
%\iffalse
%<*package>
%\fi
%
% This section describes the definitions file |childdoc.def|.

% The definitions cannot be loaded using |\usepackage| or |\RequirePackage|
% which has a mechanism to prevent loading a style file more than once.
% When loading the definitions by means of |\input|
% multiple instances have to be prevented manually:
%\iffalse
%This code needs to be before the `\ProvidesFile' directive
%which is defined at the beginning of this file.
%Therefore it is also placed there and commented out here.
%</package>
%<*discard>
%\fi
%    \begin{macrocode}
\ifdefined\childdocmain\endinput\fi
%    \end{macrocode}
%\iffalse
%</discard>
%<*package>
%\fi
%
% \macro{\ifchilddoc}
% \macro{\ifchilddocmanual}
% The conditional |\ifchilddoc| tells whether a
% child (true) or main (false) document is being compiled.
% The conditional |\ifchilddocmanual| tells whether
% the |\includeonly| mechanism is used (false) or
% the selection of child files must be performed manually (true).
% The definitions initialise to false:
%    \begin{macrocode}
\newif\ifchilddoc
\newif\ifchilddocmanual
%    \end{macrocode}

% \macro{\childdocname}
% \macro{\childdocjob}
% The macro |\childdocname| stores the name of the main document
% to be compiled. The macro |\childdocjob| stores the name of
% the document on which the \LaTeX{} compiler was originally invoked.
% The content of |\jobname| cannot be compared
% to filenames specified in the source due to different catcodes.
% The following code rescans |\jobname|, stores the result
% in |\childdocname| and saves a copy in |\childdocjob|:
%    \begin{macrocode}
\edef\childdocname{\scantokens\expandafter{\jobname\noexpand}}
\let\childdocjob\childdocname
%    \end{macrocode}

% \macro{\childdocdisable}
% The macro |\childdocdisable| prevents the main file
% from being processed more than once.
% At this stage, the main document command |\childdocmain|
% is assumed to be called once again where it should do nothing.
% Any subsequent call to it should prevent
% a secondary processing of the main document
% It overwrites the forwarding commands
% |\childdocof| and |\childdocforward|
% with empty macros to prevent further inclusions of the main document:
%    \begin{macrocode}
\newcommand{\childdocdisable}
{
  \renewcommand{\childdocmain}[1]{\renewcommand{\childdocmain}[1]{\endinput}}
  \renewcommand{\childdocof}[1]{}
  \renewcommand{\childdocby}[2][]{}
  \renewcommand{\childdocforward}[2][]{}
  \renewcommand{\childdocdisable}{}
}
%    \end{macrocode}

% \macro{\childdocmain}
% The macro |\childdocmain| is to be called at the top of the main file
% with nothing or the main filename (without extension) as argument.
% First, it breaks loops.
% If the argument is not empty and does not match |\childdocname|
% (which is set by the first inclusion of |childdoc.def|),
% |\ifchilddoc| is set to true, |\includeonly| is applied to the child file
% and |\jobname| is set to the main file
% (for proper handling of |.aux| files):
%    \begin{macrocode}
\newcommand{\childdocmain}[1]
{
  \childdocdisable\childdocmain{}
  \if?#1?\else
    \begingroup
      \def\childdoctmp{#1}
      \ifx\childdoctmp\childdocname
        \def\childdoctmp{}
      \else
        \def\childdoctmp
        {
          \childdoctrue
          \includeonly{\childdocname}
          \def\childdocjob{#1}
          \def\jobname{#1}
        }
      \fi
      \expandafter
    \endgroup
    \childdoctmp
  \fi
}
%    \end{macrocode}

% \macro{\childdocof}
% The command |\childdocof| redirects
% compilation to the main file |#1|.
%    \begin{macrocode}
\newcommand{\childdocof}[1]
{
  \childdocdisable
  \childdoctrue
  \includeonly{\childdocname}
  \def\jobname{#1}
  \def\childdocjob{#1}
  \input{#1}
}
%    \end{macrocode}

% \macro{\childdocby}
% The command |\childdocby| ....
%    \begin{macrocode}
\newcommand{\childdocby}[2][]
{
  \childdocdisable
  \childdoctrue
  \childdocmanualtrue
  \if?#1?\else
    \def\jobname{#2}
  \fi
  \def\childdocjob{#2}
  \input{#2}
  \endinput
}
%    \end{macrocode}

% \macro{\childdocforward}
% The command |\childdocforward| redirects
% compilation to the main file or
% (if the optional argument is given) a child file.
% Parameters are set as if the main file
% or a child file starting with |\childdocof| was compiled.
% Then compilation is handed over to the main file:
%    \begin{macrocode}
\newcommand{\childdocforward}[2][]
{
  \begingroup
    \if?#1?
      \def\childdoctmp
      {
        \def\childdocname{#2}
        \def\childdocjob{#2}
        \def\jobname{#2}
        \input{#2}
        \endinput
      }
    \else
      \def\childdoctmp
      {
        \childdocdisable
        \def\childdocname{#2}
        \childdoctrue
        \includeonly{#2}
        \def\childdocjob{#1}
        \def\jobname{#1}
        \input{#1}
        \endinput
      }
    \fi
    \expandafter
  \endgroup
  \childdoctmp
}
%    \end{macrocode}

% \macro{\childdocforwardprefix}
% The command |\childdocforwardprefix| redirects
% compilation to the main or a child file by means of a pattern.
% The prefix |#1| in the current filename is replaced by |#2|
% and the suffix of the current filename is kept
% (it is assumed that the filename does not contain the substring `|~~~|'
% which is used as a delimiter).
% Compilation is handed over to the new file by |\childdocforward|:
%    \begin{macrocode}
\newcommand{\childdocforwardprefix}[3][]
{
  \begingroup
    \def\childdocextract #2##1~~~{\def\childdoctmp{\childdocforward[#1]{#3##1}}}
    \expandafter\childdocextract\childdocname~~~
    \expandafter
  \endgroup
  \childdoctmp
}
%    \end{macrocode}

% \macro{\childdoc}
% The deprecated macro |\childdoc| is a legacy version of |\childdocmain|:
%    \begin{macrocode}
\newcommand{\childdoc}{\childdocmain}
%    \end{macrocode}

% \macro{\childdocredirect}
% The deprecated macro |\childdocredirect| is a legacy version
% of |\childdocforward| and |\childdocforwardprefix|:
%    \begin{macrocode}
\newcommand{\childdocredirect}[2][]
{
  \begingroup
    \if?#1?
      \def\childdoctmp{\childdocforward{#2}}
    \else
      \def\childdoctmp{\childdocforwardprefix{#1}{#2}}
    \fi
    \expandafter
  \endgroup
  \childdoctmp
}
%    \end{macrocode}

%\iffalse
%</package>
%\fi
%
\endinput
|\\
|\childdocof{|\textit{main}|}|\\
\end{tabular}
\end{center}
at the top of every child file \textit{child}
which is included by |\include{|\textit{child}|}|
from within the main file
(or at least for those files to be compiled individually).
The argument \textit{main} must be the filename of the main file.

There are a couple of
considerations in setting up the main and child documents:

%%%%%%%%%%%%%%%%%%%%%%%%%%%%%%%%%%%%%%%%
\paragraph{Restrictions.}

Please note the following restrictions:
\begin{itemize}
\item
|\childdocmain| must be called with one argument \textit{main}
to ensure compatibility with earlier version of the package.
It must either be empty (|\childdocmain{}|)
or precisely match the filename of the main file in which it is specified.
See \secref{sec:detection} for further information.
\item
The filename \textit{main} must be specified without the |.tex| extension.
\item
The filename \textit{main} is case sensitive
(even in case-insensitive file systems)
due to internal string comparison.
\item
The argument \textit{main} should be fully expanded, it cannot be a macro.
\item
Subdirectories and special characters should be avoided in filenames.
\item
The command |\childdocmain{|\textit{main}|}| must be followed by a whitespace.
It should not be followed immediately by another command
or by a comment mark `|%|'.
This is because the \TeX{} parser reads the token immediately following
the argument of |\childdocmain| and puts it
at the beginning of every child section;
however, a white\-space is ignored.
\end{itemize}

%%%%%%%%%%%%%%%%%%%%%%%%%%%%%%%%%%%%%%%%
\paragraph{Content of Main File.}

It is advisable to place all content in the child files included by |\include|.
Any output contained in the main file will appear in all child documents
unless suppressed manually;
it cannot be suppressed automatically by the |\includeonly| directive
and thus should normally be avoided.
A method to include some content in the main file
by means of conditional processing is described in \secref{sec:conditional}.

%%%%%%%%%%%%%%%%%%%%%%%%%%%%%%%%%%%%%%%%
\paragraph{Page Numbering.}

When only a part of the document is compiled,
the appropriate numbering of pages
(as well as other status parameters)
is determined from the |.aux| files.
The latter contain information from previous passes.
However this information needs to propagate through
all intermediate child documents.
Therefore the page numbering in child documents may well
be inconsistent until the complete document is compiled at least once.

A useful (if unconventional) way to always ensure a consistent
page numbering is to restart the numbering in each child document
and denote the pages by `\textit{child}|.|\textit{page}'
where \textit{child} represents the chapter/section number of the child file.
This can be achieved by the command
|\numberwithin{page}{|\textit{child}|}|
of the \textsf{amsmath} package
where \textit{child} can be |chapter| or |section|
depending on the chosen structuring.
Alternatively, one can modify the macro |\thepage| appropriately
and reset the counter |page| at the start of each child file.

%%%%%%%%%%%%%%%%%%%%%%%%%%%%%%%%%%%%%%%%%%%%%%%%%%%%%%%%%%%%%%%%%%%%%%%%%%%%%%%%
\subsection{Conditional Processing}
\label{sec:conditional}

The package provides a mechanism to compile different versions
of a document. To customise the versions further some conditional processing
can come in handy to distinguish which version is being compiled.
The package provides two macros to describe the compilation context:

%%%%%%%%%%%%%%%%%%%%%%%%%%%%%%%%%%%%%%%%
\DescribeMacro{\ifchilddoc}
The conditional |\ifchilddoc| distinguishes between the compilation of
child documents and the main document:
%
\begin{center}
|\ifchilddoc |\textit{child-code}| |[|\||else |\textit{main-code}]| \||fi|
\end{center}

%%%%%%%%%%%%%%%%%%%%%%%%%%%%%%%%%%%%%%%%
\DescribeMacro{\childdocname}
\DescribeMacro{\childdocjob}
The macro |\childdocname| contains the filename (without extension)
of the main or child file being processed.
Note that |\childdocjob| will always contain the name of the main file.

%%%%%%%%%%%%%%%%%%%%%%%%%%%%%%%%%%%%%%%%
\paragraph{Title Page.}

Conditional processing can be used to include a title or banner page
in the main document when proper precautions are taken.
Importantly, the code in the main file should ensure that the page counter
(as well as other status parameters which are stored in the |.aux| files)
takes the same value after the conditional processing.
Otherwise the page numbers may take divergent values
depending on which part is compiled.

For example, a title page could be declared by:
%
\begin{center}
\begin{tabular}{l}
|\ifchilddoc\||else|\\
|\addtocounter{page}{-1}|\\
\textit{code for title page}\\
|\newpage|\\
|\||fi|
\end{tabular}
\end{center}
%
A banner page for the child documents can be generated by:
%
\begin{center}
\begin{tabular}{l}
|\ifchilddoc|\\
|\addtocounter{page}{-1}|\\
\textit{code for banner page}\\
|\newpage|\\
|\||fi|
\end{tabular}
\end{center}
%
Here one could write a message such as:
\begin{center}
|This is the part \childdocname{} of \childdocjob{}.|
\end{center}

%%%%%%%%%%%%%%%%%%%%%%%%%%%%%%%%%%%%%%%%%%%%%%%%%%%%%%%%%%%%%%%%%%%%%%%%%%%%%%%%
\subsection{Flags}
\label{sec:flags}

The package makes it easy to generate different versions
of the main or child documents.
To this end compilation flags can be defined
and assigned different default values.
They will be particularly useful in conjunction
with the forwarding mechanism described in \secref{sec:forward}.

For example, it may be useful to have a flag |\version|
which can be set to |draft| or |final|.
The document source will contain some conditional code
depending on the value of |\version|.
Suppose further, the flag should default to |final| for the main file
and to |draft| for child files
which is a natural assignment for editing the document.
This is achieved by placing the following code
in the preamble of the main document
(below the |\childdocmain| directive):
%
\begin{center}
\begin{tabular}{l}
|\ifchilddoc|\\
|\providecommand{\version}{draft}|\\
|\||else|\\
|\providecommand{\version}{final}|\\
|\||fi|
\end{tabular}
\end{center}
%
The definition by |\providecommand| makes sure
that previous definitions are not overwritten.
Further statements |\providecommand{\version}{...}|
can thus be added before the above code to override it.

For the main file, one might add a line
(between |\childdocmain| and the above block)
%
\begin{center}
|%\ifchilddoc\||else\providecommand{\version}{draft}\||fi|
\end{center}
%
which can be uncommented to produce a draft version.
Likewise one can add a line to the very top of a child file
(above the |\childdocof{|\textit{main}|}| directive)
%
\begin{center}
|%\providecommand{\version}{final}|
\end{center}
%
which can be uncommented to produce the final version of this child document.

%%%%%%%%%%%%%%%%%%%%%%%%%%%%%%%%%%%%%%%%%%%%%%%%%%%%%%%%%%%%%%%%%%%%%%%%%%%%%%%%
\subsection{Forwarding}
\label{sec:forward}

Different versions of the main or child documents
using compilation flags as described in \secref{sec:flags}
can be (permanently) stored in different files
for convenient compilation, viewing and distribution.
To this end, the package defines a command
to pass on compilation to a different file:

%%%%%%%%%%%%%%%%%%%%%%%%%%%%%%%%%%%%%%%%
\DescribeMacro{\childdocforward}
The command |\childdocforward| redirects processing to
another source file:
%
\begin{center}
\begin{tabular}{l}
|% \iffalse
%
% childdoc.dtx Copyright (C) 2017-2018 Niklas Beisert
%
% This work may be distributed and/or modified under the
% conditions of the LaTeX Project Public License, either version 1.3
% of this license or (at your option) any later version.
% The latest version of this license is in
%   http://www.latex-project.org/lppl.txt
% and version 1.3 or later is part of all distributions of LaTeX
% version 2005/12/01 or later.
%
% This work has the LPPL maintenance status `maintained'.
%
% The Current Maintainer of this work is Niklas Beisert.
%
% This work consists of the files childdoc.dtx and childdoc.ins
% and the derived files childdoc.def and cdocsamp.tex with
% cdocsch1.tex, cdocsch2.tex, cdocsdrf.tex, cdocsfn1.tex, cdocsfn2.tex.
%
%<package>\ifdefined\childdocmain\endinput\fi
%<package>\ProvidesFile{childdoc.def}[2018/12/30 v2.0 child document driver]
%<samplemain>\ProvidesFile{cdocsamp.tex}[2018/12/30 v2.0 sample for childdoc]
%<*driver>
%\ProvidesFile{childdoc.drv}[2018/12/30 v2.0 childdoc reference manual file]
\PassOptionsToClass{10pt,a4paper}{article}
\documentclass{ltxdoc}

\usepackage[margin=35mm]{geometry}
\usepackage{hyperref}
\usepackage{hyperxmp}
\usepackage[usenames]{color}

\hypersetup{colorlinks=true}
\hypersetup{pdfstartview=FitH}
\hypersetup{pdfpagemode=UseNone}
\hypersetup{pdfsource={}}
\hypersetup{pdflang={en-UK}}
\hypersetup{pdfcopyright={Copyright 2017-2018 Niklas Beisert.
  This work may be distributed and/or modified under the
  conditions of the LaTeX Project Public License, either version 1.3
  of this license or (at your option) any later version.}}
\hypersetup{pdflicenseurl={http://www.latex-project.org/lppl.txt}}
\hypersetup{pdfcontactaddress={ETH Zurich, ITP, HIT K,
  Wolfgang-Pauli-Strasse 27}}
\hypersetup{pdfcontactpostcode={8093}}
\hypersetup{pdfcontactcity={Zurich}}
\hypersetup{pdfcontactcountry={Switzerland}}
\hypersetup{pdfcontactemail={nbeisert@itp.phys.ethz.ch}}
\hypersetup{pdfcontacturl={http://people.phys.ethz.ch/\xmptilde nbeisert/}}

\newcommand{\secref}[1]{\hyperref[#1]{section \ref*{#1}}}

\parskip1ex
\parindent0pt
\let\olditemize\itemize
\def\itemize{\olditemize\parskip0pt}

\begin{document}

\title{The \textsf{childdoc} Package}
\hypersetup{pdftitle={The childdoc Package}}
\author{Niklas Beisert\\[2ex]
  Institut f\"ur Theoretische Physik\\
  Eidgen\"ossische Technische Hochschule Z\"urich\\
  Wolfgang-Pauli-Strasse 27, 8093 Z\"urich, Switzerland\\[1ex]
  \href{mailto:nbeisert@itp.phys.ethz.ch}
  {\texttt{nbeisert@itp.phys.ethz.ch}}}
\hypersetup{pdfauthor={Niklas Beisert}}
\hypersetup{pdfsubject={Manual for the LaTeX2e Package childdoc}}
\date{30 December 2018, \textsf{v2.0}}
\maketitle

\begin{abstract}\noindent
\textsf{childdoc} is a \LaTeXe{} package
that enables the direct compilation
of document sections included by |\include|
to individual files.
\end{abstract}

\begingroup
\parskip0ex
\tableofcontents
\endgroup

%%%%%%%%%%%%%%%%%%%%%%%%%%%%%%%%%%%%%%%%%%%%%%%%%%%%%%%%%%%%%%%%%%%%%%%%%%%%%%%%
%%%%%%%%%%%%%%%%%%%%%%%%%%%%%%%%%%%%%%%%%%%%%%%%%%%%%%%%%%%%%%%%%%%%%%%%%%%%%%%%
\section{Introduction}

\LaTeX{} provides a mechanism to structure a large document (such as a book)
into a main file and several child files (containing the chapters)
using the |\include| command.
This mechanism is beneficial for documents
which span hundreds of pages in order to
make the source file(s) more manageable.
Moreover, compilation can be restricted to
selected child files by means of the |\includeonly| command.
The latter feature can be used to reduce the compilation time while editing
(this was significantly more useful in the earlier days of \LaTeX{})
or to generate a smaller document which is easier to navigate.
Another application of |\includeonly| is to generate
documents consisting of selected parts of the complete document.

However, there are a few drawbacks of the plain |\include| mechanism:
\begin{itemize}
\item
The child files cannot be compiled on their own,
they can only be compiled via the main file.
A naive editing environment
(such as a text editor with an option
to have the current file processed by \LaTeX)
may require one to switch to the main file before compiling;
attempting to compile the child file produces errors.
\item
The main file must be modified (each time)
to adjust the |\includeonly| command
to the present needs. This easily leaves the main file in a messy state.
\item
The generated document will always carry the filename
of the main document. This is inconvenient if
several child files are to be compiled and
to be kept for distribution.
\end{itemize}

The present package provides a simple interface
to make child files individually compilable by \LaTeX{}.
Compiling a child file then has the same effect as compiling
the main file with an |\includeonly| command
to select the appropriate child.
Moreover the generated document will carry the name of the child
rather than the main file.
This resolves all three above issues.

This feature is meant to make the editing of books,
thesis documents and lecture notes somewhat more convenient.
However, the package can also be used efficiently for
composing a series of documents (such as exercise sheets)
which are typically distributed individually.
It then assists the author in generating the individual documents
(potentially in different versions)
as well as a document containing the collected series.
Another application is in developing style files
or other kinds of included material
where compilation of the style file could redirect
to a sample or test file.

%%%%%%%%%%%%%%%%%%%%%%%%%%%%%%%%%%%%%%%%%%%%%%%%%%%%%%%%%%%%%%%%%%%%%%%%%%%%%%%%
%%%%%%%%%%%%%%%%%%%%%%%%%%%%%%%%%%%%%%%%%%%%%%%%%%%%%%%%%%%%%%%%%%%%%%%%%%%%%%%%
\section{Usage}

First of all, the package \textsf{childdoc} is \emph{not} a standard
\LaTeXe{} |.sty| style file! Therefore it needs to be invoked in
a non-standard way.

%%%%%%%%%%%%%%%%%%%%%%%%%%%%%%%%%%%%%%%%%%%%%%%%%%%%%%%%%%%%%%%%%%%%%%%%%%%%%%%%
\subsection{Included Files}
\label{sec:include}

%%%%%%%%%%%%%%%%%%%%%%%%%%%%%%%%%%%%%%%%
\DescribeMacro{\childdocmain}
To use the package, add the commands
\begin{center}
\begin{tabular}{l}
|% \iffalse
%
% childdoc.dtx Copyright (C) 2017-2018 Niklas Beisert
%
% This work may be distributed and/or modified under the
% conditions of the LaTeX Project Public License, either version 1.3
% of this license or (at your option) any later version.
% The latest version of this license is in
%   http://www.latex-project.org/lppl.txt
% and version 1.3 or later is part of all distributions of LaTeX
% version 2005/12/01 or later.
%
% This work has the LPPL maintenance status `maintained'.
%
% The Current Maintainer of this work is Niklas Beisert.
%
% This work consists of the files childdoc.dtx and childdoc.ins
% and the derived files childdoc.def and cdocsamp.tex with
% cdocsch1.tex, cdocsch2.tex, cdocsdrf.tex, cdocsfn1.tex, cdocsfn2.tex.
%
%<package>\ifdefined\childdocmain\endinput\fi
%<package>\ProvidesFile{childdoc.def}[2018/12/30 v2.0 child document driver]
%<samplemain>\ProvidesFile{cdocsamp.tex}[2018/12/30 v2.0 sample for childdoc]
%<*driver>
%\ProvidesFile{childdoc.drv}[2018/12/30 v2.0 childdoc reference manual file]
\PassOptionsToClass{10pt,a4paper}{article}
\documentclass{ltxdoc}

\usepackage[margin=35mm]{geometry}
\usepackage{hyperref}
\usepackage{hyperxmp}
\usepackage[usenames]{color}

\hypersetup{colorlinks=true}
\hypersetup{pdfstartview=FitH}
\hypersetup{pdfpagemode=UseNone}
\hypersetup{pdfsource={}}
\hypersetup{pdflang={en-UK}}
\hypersetup{pdfcopyright={Copyright 2017-2018 Niklas Beisert.
  This work may be distributed and/or modified under the
  conditions of the LaTeX Project Public License, either version 1.3
  of this license or (at your option) any later version.}}
\hypersetup{pdflicenseurl={http://www.latex-project.org/lppl.txt}}
\hypersetup{pdfcontactaddress={ETH Zurich, ITP, HIT K,
  Wolfgang-Pauli-Strasse 27}}
\hypersetup{pdfcontactpostcode={8093}}
\hypersetup{pdfcontactcity={Zurich}}
\hypersetup{pdfcontactcountry={Switzerland}}
\hypersetup{pdfcontactemail={nbeisert@itp.phys.ethz.ch}}
\hypersetup{pdfcontacturl={http://people.phys.ethz.ch/\xmptilde nbeisert/}}

\newcommand{\secref}[1]{\hyperref[#1]{section \ref*{#1}}}

\parskip1ex
\parindent0pt
\let\olditemize\itemize
\def\itemize{\olditemize\parskip0pt}

\begin{document}

\title{The \textsf{childdoc} Package}
\hypersetup{pdftitle={The childdoc Package}}
\author{Niklas Beisert\\[2ex]
  Institut f\"ur Theoretische Physik\\
  Eidgen\"ossische Technische Hochschule Z\"urich\\
  Wolfgang-Pauli-Strasse 27, 8093 Z\"urich, Switzerland\\[1ex]
  \href{mailto:nbeisert@itp.phys.ethz.ch}
  {\texttt{nbeisert@itp.phys.ethz.ch}}}
\hypersetup{pdfauthor={Niklas Beisert}}
\hypersetup{pdfsubject={Manual for the LaTeX2e Package childdoc}}
\date{30 December 2018, \textsf{v2.0}}
\maketitle

\begin{abstract}\noindent
\textsf{childdoc} is a \LaTeXe{} package
that enables the direct compilation
of document sections included by |\include|
to individual files.
\end{abstract}

\begingroup
\parskip0ex
\tableofcontents
\endgroup

%%%%%%%%%%%%%%%%%%%%%%%%%%%%%%%%%%%%%%%%%%%%%%%%%%%%%%%%%%%%%%%%%%%%%%%%%%%%%%%%
%%%%%%%%%%%%%%%%%%%%%%%%%%%%%%%%%%%%%%%%%%%%%%%%%%%%%%%%%%%%%%%%%%%%%%%%%%%%%%%%
\section{Introduction}

\LaTeX{} provides a mechanism to structure a large document (such as a book)
into a main file and several child files (containing the chapters)
using the |\include| command.
This mechanism is beneficial for documents
which span hundreds of pages in order to
make the source file(s) more manageable.
Moreover, compilation can be restricted to
selected child files by means of the |\includeonly| command.
The latter feature can be used to reduce the compilation time while editing
(this was significantly more useful in the earlier days of \LaTeX{})
or to generate a smaller document which is easier to navigate.
Another application of |\includeonly| is to generate
documents consisting of selected parts of the complete document.

However, there are a few drawbacks of the plain |\include| mechanism:
\begin{itemize}
\item
The child files cannot be compiled on their own,
they can only be compiled via the main file.
A naive editing environment
(such as a text editor with an option
to have the current file processed by \LaTeX)
may require one to switch to the main file before compiling;
attempting to compile the child file produces errors.
\item
The main file must be modified (each time)
to adjust the |\includeonly| command
to the present needs. This easily leaves the main file in a messy state.
\item
The generated document will always carry the filename
of the main document. This is inconvenient if
several child files are to be compiled and
to be kept for distribution.
\end{itemize}

The present package provides a simple interface
to make child files individually compilable by \LaTeX{}.
Compiling a child file then has the same effect as compiling
the main file with an |\includeonly| command
to select the appropriate child.
Moreover the generated document will carry the name of the child
rather than the main file.
This resolves all three above issues.

This feature is meant to make the editing of books,
thesis documents and lecture notes somewhat more convenient.
However, the package can also be used efficiently for
composing a series of documents (such as exercise sheets)
which are typically distributed individually.
It then assists the author in generating the individual documents
(potentially in different versions)
as well as a document containing the collected series.
Another application is in developing style files
or other kinds of included material
where compilation of the style file could redirect
to a sample or test file.

%%%%%%%%%%%%%%%%%%%%%%%%%%%%%%%%%%%%%%%%%%%%%%%%%%%%%%%%%%%%%%%%%%%%%%%%%%%%%%%%
%%%%%%%%%%%%%%%%%%%%%%%%%%%%%%%%%%%%%%%%%%%%%%%%%%%%%%%%%%%%%%%%%%%%%%%%%%%%%%%%
\section{Usage}

First of all, the package \textsf{childdoc} is \emph{not} a standard
\LaTeXe{} |.sty| style file! Therefore it needs to be invoked in
a non-standard way.

%%%%%%%%%%%%%%%%%%%%%%%%%%%%%%%%%%%%%%%%%%%%%%%%%%%%%%%%%%%%%%%%%%%%%%%%%%%%%%%%
\subsection{Included Files}
\label{sec:include}

%%%%%%%%%%%%%%%%%%%%%%%%%%%%%%%%%%%%%%%%
\DescribeMacro{\childdocmain}
To use the package, add the commands
\begin{center}
\begin{tabular}{l}
|\input{childdoc.def}|\\
|\childdocmain{}|\\
\end{tabular}
\end{center}
at the very top of the main \LaTeX{} file,
in particular \emph{before} the |\documentclass| statement!
The argument of |\childdocmain| should be left empty
(but it must be present).

%%%%%%%%%%%%%%%%%%%%%%%%%%%%%%%%%%%%%%%%
\DescribeMacro{\childdocof}
Furthermore, add the commands
\begin{center}
\begin{tabular}{l}
|\input{childdoc.def}|\\
|\childdocof{|\textit{main}|}|\\
\end{tabular}
\end{center}
at the top of every child file \textit{child}
which is included by |\include{|\textit{child}|}|
from within the main file
(or at least for those files to be compiled individually).
The argument \textit{main} must be the filename of the main file.

There are a couple of
considerations in setting up the main and child documents:

%%%%%%%%%%%%%%%%%%%%%%%%%%%%%%%%%%%%%%%%
\paragraph{Restrictions.}

Please note the following restrictions:
\begin{itemize}
\item
|\childdocmain| must be called with one argument \textit{main}
to ensure compatibility with earlier version of the package.
It must either be empty (|\childdocmain{}|)
or precisely match the filename of the main file in which it is specified.
See \secref{sec:detection} for further information.
\item
The filename \textit{main} must be specified without the |.tex| extension.
\item
The filename \textit{main} is case sensitive
(even in case-insensitive file systems)
due to internal string comparison.
\item
The argument \textit{main} should be fully expanded, it cannot be a macro.
\item
Subdirectories and special characters should be avoided in filenames.
\item
The command |\childdocmain{|\textit{main}|}| must be followed by a whitespace.
It should not be followed immediately by another command
or by a comment mark `|%|'.
This is because the \TeX{} parser reads the token immediately following
the argument of |\childdocmain| and puts it
at the beginning of every child section;
however, a white\-space is ignored.
\end{itemize}

%%%%%%%%%%%%%%%%%%%%%%%%%%%%%%%%%%%%%%%%
\paragraph{Content of Main File.}

It is advisable to place all content in the child files included by |\include|.
Any output contained in the main file will appear in all child documents
unless suppressed manually;
it cannot be suppressed automatically by the |\includeonly| directive
and thus should normally be avoided.
A method to include some content in the main file
by means of conditional processing is described in \secref{sec:conditional}.

%%%%%%%%%%%%%%%%%%%%%%%%%%%%%%%%%%%%%%%%
\paragraph{Page Numbering.}

When only a part of the document is compiled,
the appropriate numbering of pages
(as well as other status parameters)
is determined from the |.aux| files.
The latter contain information from previous passes.
However this information needs to propagate through
all intermediate child documents.
Therefore the page numbering in child documents may well
be inconsistent until the complete document is compiled at least once.

A useful (if unconventional) way to always ensure a consistent
page numbering is to restart the numbering in each child document
and denote the pages by `\textit{child}|.|\textit{page}'
where \textit{child} represents the chapter/section number of the child file.
This can be achieved by the command
|\numberwithin{page}{|\textit{child}|}|
of the \textsf{amsmath} package
where \textit{child} can be |chapter| or |section|
depending on the chosen structuring.
Alternatively, one can modify the macro |\thepage| appropriately
and reset the counter |page| at the start of each child file.

%%%%%%%%%%%%%%%%%%%%%%%%%%%%%%%%%%%%%%%%%%%%%%%%%%%%%%%%%%%%%%%%%%%%%%%%%%%%%%%%
\subsection{Conditional Processing}
\label{sec:conditional}

The package provides a mechanism to compile different versions
of a document. To customise the versions further some conditional processing
can come in handy to distinguish which version is being compiled.
The package provides two macros to describe the compilation context:

%%%%%%%%%%%%%%%%%%%%%%%%%%%%%%%%%%%%%%%%
\DescribeMacro{\ifchilddoc}
The conditional |\ifchilddoc| distinguishes between the compilation of
child documents and the main document:
%
\begin{center}
|\ifchilddoc |\textit{child-code}| |[|\||else |\textit{main-code}]| \||fi|
\end{center}

%%%%%%%%%%%%%%%%%%%%%%%%%%%%%%%%%%%%%%%%
\DescribeMacro{\childdocname}
\DescribeMacro{\childdocjob}
The macro |\childdocname| contains the filename (without extension)
of the main or child file being processed.
Note that |\childdocjob| will always contain the name of the main file.

%%%%%%%%%%%%%%%%%%%%%%%%%%%%%%%%%%%%%%%%
\paragraph{Title Page.}

Conditional processing can be used to include a title or banner page
in the main document when proper precautions are taken.
Importantly, the code in the main file should ensure that the page counter
(as well as other status parameters which are stored in the |.aux| files)
takes the same value after the conditional processing.
Otherwise the page numbers may take divergent values
depending on which part is compiled.

For example, a title page could be declared by:
%
\begin{center}
\begin{tabular}{l}
|\ifchilddoc\||else|\\
|\addtocounter{page}{-1}|\\
\textit{code for title page}\\
|\newpage|\\
|\||fi|
\end{tabular}
\end{center}
%
A banner page for the child documents can be generated by:
%
\begin{center}
\begin{tabular}{l}
|\ifchilddoc|\\
|\addtocounter{page}{-1}|\\
\textit{code for banner page}\\
|\newpage|\\
|\||fi|
\end{tabular}
\end{center}
%
Here one could write a message such as:
\begin{center}
|This is the part \childdocname{} of \childdocjob{}.|
\end{center}

%%%%%%%%%%%%%%%%%%%%%%%%%%%%%%%%%%%%%%%%%%%%%%%%%%%%%%%%%%%%%%%%%%%%%%%%%%%%%%%%
\subsection{Flags}
\label{sec:flags}

The package makes it easy to generate different versions
of the main or child documents.
To this end compilation flags can be defined
and assigned different default values.
They will be particularly useful in conjunction
with the forwarding mechanism described in \secref{sec:forward}.

For example, it may be useful to have a flag |\version|
which can be set to |draft| or |final|.
The document source will contain some conditional code
depending on the value of |\version|.
Suppose further, the flag should default to |final| for the main file
and to |draft| for child files
which is a natural assignment for editing the document.
This is achieved by placing the following code
in the preamble of the main document
(below the |\childdocmain| directive):
%
\begin{center}
\begin{tabular}{l}
|\ifchilddoc|\\
|\providecommand{\version}{draft}|\\
|\||else|\\
|\providecommand{\version}{final}|\\
|\||fi|
\end{tabular}
\end{center}
%
The definition by |\providecommand| makes sure
that previous definitions are not overwritten.
Further statements |\providecommand{\version}{...}|
can thus be added before the above code to override it.

For the main file, one might add a line
(between |\childdocmain| and the above block)
%
\begin{center}
|%\ifchilddoc\||else\providecommand{\version}{draft}\||fi|
\end{center}
%
which can be uncommented to produce a draft version.
Likewise one can add a line to the very top of a child file
(above the |\childdocof{|\textit{main}|}| directive)
%
\begin{center}
|%\providecommand{\version}{final}|
\end{center}
%
which can be uncommented to produce the final version of this child document.

%%%%%%%%%%%%%%%%%%%%%%%%%%%%%%%%%%%%%%%%%%%%%%%%%%%%%%%%%%%%%%%%%%%%%%%%%%%%%%%%
\subsection{Forwarding}
\label{sec:forward}

Different versions of the main or child documents
using compilation flags as described in \secref{sec:flags}
can be (permanently) stored in different files
for convenient compilation, viewing and distribution.
To this end, the package defines a command
to pass on compilation to a different file:

%%%%%%%%%%%%%%%%%%%%%%%%%%%%%%%%%%%%%%%%
\DescribeMacro{\childdocforward}
The command |\childdocforward| redirects processing to
another source file:
%
\begin{center}
\begin{tabular}{l}
|\input{childdoc.def}|\\
|\childdocforward[|\textit{main}|]{|\textit{dest}|}|\\
\end{tabular}
\end{center}
%
The argument \textit{dest} is the destination file
(without extension).
It should be the main file or one of the child files.
Note that further \textsf{childdoc} directives
such as |\childdocof| and |\childdocforward|
in the indicated file will be processed in this form.
The optional argument \textit{main}
passes on directly to the main file \textit{main}
while pretending to compile the child \textit{dest}.
This form behaves as if \textit{dest}
issues |\childdocof{|\textit{main}|}| right away,
and no further \textsf{childdoc} directives will be processed.

%%%%%%%%%%%%%%%%%%%%%%%%%%%%%%%%%%%%%%%%
\DescribeMacro{\...prefix}
In the alternative form |\childdocforwardprefix|,
%
\begin{center}
\begin{tabular}{l}
|\input{childdoc.def}|\\
|\childdocforwardprefix[|\textit{main}|]{|\textit{prefix}|}{|\textit{dest}|}|
\end{tabular}
\end{center}
%
the destination file is determined by a pattern
depending on the current file:
To make this work, the current file must be called
`{\textit{prefix}\hspace{0.2em}\textit{suffix}}'
with \textit{prefix} matching precisely the argument.
Processing is then passed on to the file
`{\textit{dest}\hspace{0.2em}\textit{suffix}}'.
Surely, the same effect is achieved by
directly specifying the
argument `{\textit{dest}\hspace{0.2em}\textit{suffix}}'
in the first form.
However, that requires to set up a different file
for each child. With the alternative form of the command
all these files can have exactly the same content
which simplifies setting them up and maintaining them.

For example, the following file |draft.tex|
with a compilation flag |\version| as described in \secref{sec:flags}
compiles the main document as a draft:
%
\begin{center}
\begin{tabular}{l}
|\def\version{draft}|\\
|\input{childdoc.def}|\\
|\childdocforward{|\textit{main}|}|
\end{tabular}
\end{center}
%
Likewise, the following files |final|\textit{nn}|.tex|
compile the final version of the child document
|child|\textit{nn}|.tex|:
%
\begin{center}
\begin{tabular}{l}
|\def\version{final}|\\
|\input{childdoc.def}|\\
|\childdocforwardprefix{final}{child}|
\end{tabular}
\end{center}
%

Note that when several versions of a main file and/or of each child file
are to be generated, it may be convenient to set up a |Makefile| or
shell script to automatise the process.

%%%%%%%%%%%%%%%%%%%%%%%%%%%%%%%%%%%%%%%%%%%%%%%%%%%%%%%%%%%%%%%%%%%%%%%%%%%%%%%%
\subsection{Command Line Processing}
\label{sec:commandline}

The effect of redirection files can also be achieved by invoking
the \LaTeX{} compiler with a more elaborate command line.
Most conveniently this should be done as part
of a shell script or a |Makefile|.

When using \textsf{childdoc} in the main file, the following
command lines effectively perform a redirection
(note that depending on the shell being used,
backslashes may have to be doubled: `|\|' $\to$ `|\\|'):
%
\begin{center}
|... -jobname "|\textit{target}|" |\\|"|[\textit{flags}]%
|\input{childdoc.def}\childdocforward[|\textit{main}|]{|\textit{dest}|}"|
\end{center}
%
Here \textit{target} is the name of the output file,
\textit{main} is the name of the main file
and \textit{dest} is the name of the main or child file to be processed
(all filenames without extensions).
The optional argument \textit{main} can be omitted
if \textit{main} matches \textit{dest}.
Optionally, compilation \textit{flags} can be defined via |\def| commands.
This command line makes the \TeX{} engine believe
it is compiling the file \textit{target}
whose content is specified as the latter parameter.
The provided code then forwards the processing to
\textit{main} or \textit{dest} as described in \secref{sec:forward}.

%%%%%%%%%%%%%%%%%%%%%%%%%%%%%%%%%%%%%%%%%%%%%%%%%%%%%%%%%%%%%%%%%%%%%%%%%%%%%%%%
\subsection{Include by Input}
\label{sec:input}

Including child documents by |\include| has some restrictions by design.
Most notably, the content of a child document always occupies
its own set of pages; pages cannot be shared between child documents.
Usually, this behaviour makes perfect sense
because each child document contain an essential part of the document.
However, in some situations it may be desirable to compose
a document from a collection of parts
without having mandatory page breaks between then.
For this case, the package
provides a mechanism to include parts
by |\input| which can also be processed individually.
However, by construction this mechanism
requires manual handling of the content to be output.

%%%%%%%%%%%%%%%%%%%%%%%%%%%%%%%%%%%%%%%%
\DescribeMacro{\ifchilddocmanual}
The main file should be prepared as usual, see \secref{sec:include}.
However, the document body must make a distinction
between processing of an individual part and of the main document, e.g.:
%
\begin{center}
\begin{tabular}{l}
|\ifchilddocmanual|\\
|\input{\childdocname}|\\
|\||else|\\
\textit{document body with }|\input{|\textit{part}|}|\\
|\||fi|
\end{tabular}
\end{center}
%
The conditional |\ifchilddocmanual| is true whenever
a part to be included by |\input| is being compiled,
and the name of the part is stored in |\childdocname|.

%%%%%%%%%%%%%%%%%%%%%%%%%%%%%%%%%%%%%%%%
\DescribeMacro{\childdocby}
Each part to be included by |\input| should start with:
%
\begin{center}
\begin{tabular}{l}
|\input{childdoc.def}|\\
|\childdocby{|\textit{main}|}|\\
\end{tabular}
\end{center}
%
The directive |\childdocby| is similar to |\childdocof|
described in \secref{sec:include},
but the subsequent selection of content must be done manually.
To that end, both |\ifchilddoc| and |\ifchilddocmanual|
will be true upon processing of a part,
and the name of the part is stored in |\childdocname|.
Note that |\jobname| will be set to the filename of the current part
so that each part receives an individual |.aux| file
that does not interfere with the |.aux| file(s) of the main document.
This behaviour can be altered by the alternative form
|\childdocby[*]{|\textit{main}|}| (with a non-empty optional argument)
which uses the |.aux| file of the main document
by setting |\jobname| to \textit{main}.

%%%%%%%%%%%%%%%%%%%%%%%%%%%%%%%%%%%%%%%%%%%%%%%%%%%%%%%%%%%%%%%%%%%%%%%%%%%%%%%%
\subsection{Driver Development}
\label{sec:driver}

The \textsf{childdoc} mechanism can also be use for the development
of definition files such as \LaTeX{} styles or classes.
This case differs from the above setup with multiple parts
included by |\include| in that no |\includeonly| should be invoked.
This can be achieved by starting the include file
(before |\ProvidesPackage|) with:
%
\begin{center}
\begin{tabular}{l}
|\input{childdoc.def}|\\
|\childdocforward{|\textit{main}|}|\\
\end{tabular}
\end{center}
%
or alternatively with:
%
\begin{center}
\begin{tabular}{l}
|\input{childdoc.def}|\\
|\childdocby{|\textit{main}|}|\\
\end{tabular}
\end{center}
%
Both forms have slightly different effects as described above.
The main file is prepared as usual, see \secref{sec:include}.

%%%%%%%%%%%%%%%%%%%%%%%%%%%%%%%%%%%%%%%%%%%%%%%%%%%%%%%%%%%%%%%%%%%%%%%%%%%%%%%%
\subsection{Legacy Detection}
\label{sec:detection}

The directive |\childdocmain| in the main file can detect
whether the complete document or merely a child is to be compiled
even without using the directive |\childdocof|.
This method is deprecated because it is less robust
and there is no compelling reason to use it;
it is merely provided for backward compatibility
and it may be removed in future versions.

If the detection mechanism is to be used,
it is mandatory to correctly specify
the filename of the main file as the argument of |\childdocmain|:
%
\begin{center}
\begin{tabular}{l}
|\input{childdoc.def}|\\
|\childdocmain{|\textit{main}|}|\\
\end{tabular}
\end{center}
%
If |\jobname| does not match the argument \textit{main} of |\childdocmain|,
it is assumed that |\jobname| points to the child file to be compiled.
When using |\childdocmain| with the main file specified as argument,
it suffices to start a child file
with just |\input{|\textit{main}|}|
without loading of the package and using |\childdocof|.
If instead all processing is done
with the appropriate \textsf{childdoc} directives,
the argument of \textit{main} of |\childdocmain| can be empty.

An alternative version of the command line processing described
in \secref{sec:commandline} using the detection mechanism reads:
%
\begin{center}
|... -jobname "|\textit{target}|" "|[\textit{flags}]%
[|\def\jobname{|\textit{dest}|}|]|\input{|\textit{main}|}"|
\end{center}

%%%%%%%%%%%%%%%%%%%%%%%%%%%%%%%%%%%%%%%%%%%%%%%%%%%%%%%%%%%%%%%%%%%%%%%%%%%%%%%%
\subsection{Manual Code}
\label{sec:manual}

In case one cannot be certain whether the definitions file |childdoc.def|
is installed on the target \TeX{} distribution
and one prefers not to ship it,
it is conceivable to paste a few relevant commands into the sources.

To that end, drop all statements |\input{childdoc.def}|
and perform the replacements as outlined below.
Instead of |\childdocmain{|\textit{main}|}| add the following code
to the top of the main file:
%
\begin{center}
\begin{tabular}{l}
|\||ifdefined\childdocname\endinput\||fi\newif\ifchilddoc|\\
|\edef\childdocname{\scantokens\expandafter{\jobname\noexpand}}|\\
|\def\childdocmain{|\textit{main}|}\||ifx\childdocmain\childdocname\||else|\\
|\childdoctrue\includeonly{\childdocname}\let\jobname\childdocmain\||fi|\\
\end{tabular}
\end{center}
%
Instead of |\childdocof{|\textit{main}|}| just include the main file
at the top of each child file:
%
\begin{center}
|\input{|\textit{main}|}|
\end{center}
%
A simple redirection |\childdocforward{|\textit{dest}|}| is achieved by:
%
\begin{center}
|\def\jobname{|\textit{dest}|}\input{\jobname}|
\end{center}
%
The redirection with prefix
|\childdocforwardprefix[|\textit{prefix}|]{|\textit{dest}|}|
is accomplished by:
%
\begin{center}
\begin{tabular}{l}
|{\edef\jobname{\scantokens\expandafter{\jobname\noexpand}}|\\
|\def\redirectjob |\textit{prefix}|#1~~~{\gdef\jobname{|\textit{dest}|#1}}|\\
|\expandafter\redirectjob\jobname~~~}\input{\jobname}|
\end{tabular}
\end{center}

In an alternative approach,
child documents can be compiled by a specific command line
without additional code or specific definitions:
%
\begin{center}
|... -jobname "|\textit{target}|" "|[\textit{flags}]%
|\includeonly{|\textit{dest}|}\input{|\textit{main}|}"|
\end{center}
%

%%%%%%%%%%%%%%%%%%%%%%%%%%%%%%%%%%%%%%%%%%%%%%%%%%%%%%%%%%%%%%%%%%%%%%%%%%%%%%%%
%%%%%%%%%%%%%%%%%%%%%%%%%%%%%%%%%%%%%%%%%%%%%%%%%%%%%%%%%%%%%%%%%%%%%%%%%%%%%%%%
\section{Information}

%%%%%%%%%%%%%%%%%%%%%%%%%%%%%%%%%%%%%%%%%%%%%%%%%%%%%%%%%%%%%%%%%%%%%%%%%%%%%%%%
\subsection{Copyright}

Copyright \copyright{} 2017--2018 Niklas Beisert

This work may be distributed and/or modified under the
conditions of the \LaTeX{} Project Public License, either version 1.3
of this license or (at your option) any later version.
The latest version of this license is in
  \url{http://www.latex-project.org/lppl.txt}
and version 1.3 or later is part of all distributions of \LaTeX{}
version 2005/12/01 or later.

This work has the LPPL maintenance status `maintained'.

The Current Maintainer of this work is Niklas Beisert.

This work consists of the files |README.txt|, |childdoc.ins| and |childdoc.dtx|
as well as the derived files |childdoc.def|, |cdocsamp.tex|
with |cdocsch1.tex|, |cdocsch2.tex|, |cdocspt3.tex|, |cdocspt4.tex|,
|cdocsdrf.tex|, |cdocsfn1.tex|, |cdocsfn2.tex|
as well as |childdoc.pdf|.

%%%%%%%%%%%%%%%%%%%%%%%%%%%%%%%%%%%%%%%%%%%%%%%%%%%%%%%%%%%%%%%%%%%%%%%%%%%%%%%%
\subsection{Files and Installation}

The package consists of the files:
%
\begin{center}
\begin{tabular}{ll}
    |README.txt|   & readme file \\
    |childdoc.ins| & installation file \\
    |childdoc.dtx| & source file \\
    |childdoc.def| & definition file \\
    |cdocsamp.tex| & sample main file \\
    |cdocsch1.tex| & sample include file \\
    |cdocsch2.tex| & sample include file \\
    |cdocspt3.tex| & sample part file \\
    |cdocspt4.tex| & sample part file \\
    |cdocsdrf.tex| & sample redirection file \\
    |cdocsfn1.tex| & sample redirection file \\
    |cdocsfn2.tex| & sample redirection file \\
    |childdoc.pdf| & manual
\end{tabular}
\end{center}
%
The distribution consists of the files
|README.txt|, |childdoc.ins| and |childdoc.dtx|.
%
\begin{itemize}
\item
Run (pdf)\LaTeX{} on |childdoc.dtx|
to compile the manual |childdoc.pdf| (this file).
\item
Run \LaTeX{} on |childdoc.ins| to create the definitions file |childdoc.def|
and the sample |cdocsamp.tex| with include files
|cdocsch1.tex|, |cdocsch2.tex|, |cdocspt3.tex|, |cdocspt4.tex|,
|cdocsdrf.tex|, |cdocsfn1.tex|, |cdocsfn2.tex|.
Then copy the file |childdoc.def| to an appropriate directory of your \LaTeX{}
distribution, e.g.\ \textit{texmf-root}|/tex/latex/childdoc|.
\end{itemize}

%%%%%%%%%%%%%%%%%%%%%%%%%%%%%%%%%%%%%%%%%%%%%%%%%%%%%%%%%%%%%%%%%%%%%%%%%%%%%%%%
\subsection{Related CTAN Packages}

There are several other packages which offer a similar functionality:
%
\begin{itemize}
\item
The packages
\href{http://ctan.org/pkg/docmute}{\textsf{docmute}},
\href{http://ctan.org/pkg/includex}{\textsf{includex}} and
\href{http://ctan.org/pkg/standalone}{\textsf{standalone}}
provide commands to include only the document body of
a child file thus allowing both files to be compiled individually.
\item
The packages \href{http://ctan.org/pkg/subdocs}{\textsf{subdocs}}
and \href{http://ctan.org/pkg/subfiles}{\textsf{subfiles}}
provide structures in which the main and child documents can be
encapsulated and allowing them to be compiled individually.
The inclusion mechanism is different from the conventional |\include|.
\item
The package \href{http://ctan.org/pkg/combine}{\textsf{combine}}
is an elaborate solution to combine several documents into one.
\end{itemize}
%
See also the CTAN topic \href{http://ctan.org/topic/subdocs}{\textsf{subdocs}}
for further related packages.
The present package differs from the above solutions in that
a document structure constructed with the conventional |\include| mechanism
just needs two extra commands at the top of every file
such that all constituent files can be compiled individually.

%%%%%%%%%%%%%%%%%%%%%%%%%%%%%%%%%%%%%%%%%%%%%%%%%%%%%%%%%%%%%%%%%%%%%%%%%%%%%%%%
%\subsection{Feature Suggestions}
%
%The following is a list of features which may be useful for future
%versions of this package:
%%
%\begin{itemize}
%\item
%\ldots
%\end{itemize}

%%%%%%%%%%%%%%%%%%%%%%%%%%%%%%%%%%%%%%%%%%%%%%%%%%%%%%%%%%%%%%%%%%%%%%%%%%%%%%%%
\subsection{Revision History}

%%%%%%%%%%%%%%%%%%%%%%%%%%%%%%%%%%%%%%%%
\paragraph{v2.0:} 2018/12/30

\begin{itemize}
\item
immediate forward processing
\item
added |\childdocby| mechanism
\item
manual restructured
\end{itemize}

%%%%%%%%%%%%%%%%%%%%%%%%%%%%%%%%%%%%%%%%
\paragraph{v1.6:} 2018/01/17

\begin{itemize}
\item
application for development of include files
\item
corrections to manual
\end{itemize}

%%%%%%%%%%%%%%%%%%%%%%%%%%%%%%%%%%%%%%%%
\paragraph{v1.5:} 2017/05/21

\begin{itemize}
\item
more complete structuring introduced
\item
|\childdocof| introduced
\item
|\childdoc| renamed to |\childdocmain|
\item
|\childredirect| renamed to |\childdocforward| and |\childdocforwardprefix|
and functionality expanded
\end{itemize}

%%%%%%%%%%%%%%%%%%%%%%%%%%%%%%%%%%%%%%%%
\paragraph{v1.0:} 2017/04/27

\begin{itemize}
\item
manual and install package
\item
first version published on CTAN
\end{itemize}

%%%%%%%%%%%%%%%%%%%%%%%%%%%%%%%%%%%%%%%%
\paragraph{v0.6:} 2017/04/26

\begin{itemize}
\item
redirection mechanism added
\end{itemize}

%%%%%%%%%%%%%%%%%%%%%%%%%%%%%%%%%%%%%%%%
\paragraph{v0.5:} 2017/04/26

\begin{itemize}
\item
functionality in definition file
\end{itemize}


%%%%%%%%%%%%%%%%%%%%%%%%%%%%%%%%%%%%%%%%%%%%%%%%%%%%%%%%%%%%%%%%%%%%%%%%%%%%%%%%
%%%%%%%%%%%%%%%%%%%%%%%%%%%%%%%%%%%%%%%%%%%%%%%%%%%%%%%%%%%%%%%%%%%%%%%%%%%%%%%%
%%%%%%%%%%%%%%%%%%%%%%%%%%%%%%%%%%%%%%%%%%%%%%%%%%%%%%%%%%%%%%%%%%%%%%%%%%%%%%%%
\appendix

\settowidth\MacroIndent{\rmfamily\scriptsize 000\ }

 \DocInput{childdoc.dtx}

\end{document}
%</driver>
% \fi
%
% %%%%%%%%%%%%%%%%%%%%%%%%%%%%%%%%%%%%%%%%%%%%%%%%%%%%%%%%%%%%%%%%%%%%%%%%%%%%%%
% %%%%%%%%%%%%%%%%%%%%%%%%%%%%%%%%%%%%%%%%%%%%%%%%%%%%%%%%%%%%%%%%%%%%%%%%%%%%%%
% \section{Sample}
%\iffalse
%<*samplemain>
%\fi
%
% The following presents a sample document
% with two chapters, two parts, a title page,
% a compile flag as well as three forwarding files to set the flag.
% It consists of eight |.tex| files:
% \begin{center}
% \begin{tabular}{ll}
% |cdocsamp.tex|&main file\\
% |cdocsch1.tex|&include file for chapter 1\\
% |cdocsch2.tex|&include file for chapter 2\\
% |cdocspt3.tex|&include file for part 3\\
% |cdocspt4.tex|&include file for part 4\\
% |cdocsdrf.tex|&forwarding file for main file in draft mode\\
% |cdocsfi1.tex|&forwarding file for final version of chapter 1\\
% |cdocsfi2.tex|&forwarding file for final version of chapter 2\\
% \end{tabular}
% \end{center}
% Each of the eight files can be compiled directly by the \LaTeX{} compiler.
%
% %%%%%%%%%%%%%%%%%%%%%%%%%%%%%%%%%%%%%%
% \paragraph{Main File.}
%
% The main file is called |cdocsamp.tex|.
%
% Load the \textsf{childdoc} definitions and
% declare the filename for the main document:
%    \begin{macrocode}
\input{childdoc.def}
\childdocmain{}
%    \end{macrocode}

% Optional override for |\version| flag:
%    \begin{macrocode}
%%\ifchilddoc\else\providecommand{\version}{draft}\fi
%    \end{macrocode}

% Define the default values for the |\version| flag
% (|final| for the main file and |draft| for childs):
%    \begin{macrocode}
\ifchilddoc
\providecommand{\version}{draft}
\else
\providecommand{\version}{final}
\fi
%    \end{macrocode}

% Load the standard document class:
%    \begin{macrocode}
\documentclass[12pt]{article}
%    \end{macrocode}

% Start the document body:
%    \begin{macrocode}
\begin{document}
%    \end{macrocode}

% Declare a title page.
% Print title, part of document being processed and version flag:
%    \begin{macrocode}
\addtocounter{page}{-1}
\begin{center}
{\LARGE\bfseries{}childdoc example\par}
\vspace{1cm}
\ifchilddoc
\ifchilddocmanual part\else chapter\fi:
`\childdocname' of `\childdocjob'\par
\else
main document: `\childdocjob'\par
\fi
version: \version\par
\end{center}
\newpage
%    \end{macrocode}

% Manually include selected file,
% otherwise process as usual:
%    \begin{macrocode}
\ifchilddocmanual
\section*{part `\childdocname'}
\input{\childdocname}
\else
%    \end{macrocode}

% Include the two chapters:
%    \begin{macrocode}
\include{cdocsch1}
\include{cdocsch2}
%    \end{macrocode}

% Include the two parts unless only chapters should be displayed:
%    \begin{macrocode}
\ifchilddoc\else
\section{part three}
\input{cdocspt3}
\section{part four}
\input{cdocspt4}
\fi
%    \end{macrocode}

% Process as usual until here:
%    \begin{macrocode}
\fi
%    \end{macrocode}

% End of document body:
%    \begin{macrocode}
\end{document}
%    \end{macrocode}
%\iffalse
%</samplemain>
%\fi
%
% %%%%%%%%%%%%%%%%%%%%%%%%%%%%%%%%%%%%%%
% \paragraph{Chapter Include Files.}
%
% The include files are called |cdocsch1.tex| and |cdocsch2.tex|.
%
%\iffalse
%<*samplechap1|samplechap2>
%\fi

% Optional override for |\version| flag:
%    \begin{macrocode}
%%\providecommand{\version}{final}
%    \end{macrocode}

% Include the main document:
%    \begin{macrocode}
\input{childdoc.def}
\childdocof{cdocsamp}
%    \end{macrocode}

%\iffalse
%</samplechap1|samplechap2>
%\fi
%
%\iffalse
%<*samplechap1>
%\fi
% Some text for chapter 1:
%    \begin{macrocode}
\section{one}
some text in chapter one
%    \end{macrocode}

%\iffalse
%</samplechap1>
%\fi
% Some text for chapter 2:
%\iffalse
%<*samplechap2>
%\fi
%    \begin{macrocode}
\section{two}
more text in chapter two
%    \end{macrocode}

%\iffalse
%</samplechap2>
%\fi
%
% %%%%%%%%%%%%%%%%%%%%%%%%%%%%%%%%%%%%%%
% \paragraph{Part Include Files.}
%
% The include files are called |cdocspt3.tex| and |cdocspt4.tex|.
%
%\iffalse
%<*samplepart3|samplepart4>
%\fi

% Optional override for |\version| flag:
%    \begin{macrocode}
%%\providecommand{\version}{final}
%    \end{macrocode}

% Include the main document:
%    \begin{macrocode}
\input{childdoc.def}
\childdocby{cdocsamp}
%    \end{macrocode}

%\iffalse
%</samplepart3|samplepart4>
%\fi
%
%\iffalse
%<*samplepart3>
%\fi
% Some text for part 3:
%    \begin{macrocode}
some text in part three
%    \end{macrocode}

%\iffalse
%</samplepart3>
%\fi
% Some text for part 4:
%\iffalse
%<*samplepart4>
%\fi
%    \begin{macrocode}
more text in part four
%    \end{macrocode}

%\iffalse
%</samplepart4>
%\fi
%
% %%%%%%%%%%%%%%%%%%%%%%%%%%%%%%%%%%%%%%
% \paragraph{Forwarding for a Complete Draft.}
%
% The following forwarding file |cdocsdrf.tex|
% compiles the main document in draft mode:
%\iffalse
%<*sampledraft>
%\fi
%    \begin{macrocode}
\def\version{draft}
\input{childdoc.def}
\childdocforward{cdocsamp}
%    \end{macrocode}

%\iffalse
%</sampledraft>
%\fi
%
% %%%%%%%%%%%%%%%%%%%%%%%%%%%%%%%%%%%%%%
% \paragraph{Forwarding for Final Version of the Chapters.}
%
% The following forwarding files |cdocsfn1.tex| and |cdocsfn2.tex|
% (with identical content)
% compile the final versions of the child documents
% |cdocsch1.tex| and |cdocsch2.tex|, respectively:
%\iffalse
%<*samplefinal>
%\fi
%    \begin{macrocode}
\def\version{final}
\input{childdoc.def}
\childdocforwardprefix[cdocsamp]{cdocsfn}{cdocsch}
%    \end{macrocode}

%\iffalse
%</samplefinal>
%\fi
%
% %%%%%%%%%%%%%%%%%%%%%%%%%%%%%%%%%%%%%%
% \paragraph{Command Line Processing.}
%
% The following three command lines generate the output files
% |cdocscld|, |cdocscl1| and |cdocscl2|
% which should be identical to
% |cdocsdrf|, |cdocsch1| and |cdocsfn2|, respectively:
% \begin{center}
% \begin{tabular}{l}
% |latex -jobname cdocscld \|\\
% |  "\def\version{draft}\input{childdoc.def}\childdocforward{cdocsamp}"|\\
% |latex -jobname cdocscl1 \|\\
% |  "\input{childdoc.def}\childdocforward[cdocsamp]{cdocsch1}"|\\
% |latex -jobname cdocscl2 \|\\
% |  "\def\version{final}\input{childdoc.def}\childdocforward{cdocsch2}"|
% \end{tabular}
% \end{center}
% Note that the trailing backslash on each first line
% merely continues the input to the second line
% (for convenient cut ant paste).
% Furthermore, the command |latex| can be replaced by any
% of its alternative versions such as |pdflatex|.
%
% %%%%%%%%%%%%%%%%%%%%%%%%%%%%%%%%%%%%%%%%%%%%%%%%%%%%%%%%%%%%%%%%%%%%%%%%%%%%%%
% %%%%%%%%%%%%%%%%%%%%%%%%%%%%%%%%%%%%%%%%%%%%%%%%%%%%%%%%%%%%%%%%%%%%%%%%%%%%%%
% \section{Implementation}
%\iffalse
%<*package>
%\fi
%
% This section describes the definitions file |childdoc.def|.

% The definitions cannot be loaded using |\usepackage| or |\RequirePackage|
% which has a mechanism to prevent loading a style file more than once.
% When loading the definitions by means of |\input|
% multiple instances have to be prevented manually:
%\iffalse
%This code needs to be before the `\ProvidesFile' directive
%which is defined at the beginning of this file.
%Therefore it is also placed there and commented out here.
%</package>
%<*discard>
%\fi
%    \begin{macrocode}
\ifdefined\childdocmain\endinput\fi
%    \end{macrocode}
%\iffalse
%</discard>
%<*package>
%\fi
%
% \macro{\ifchilddoc}
% \macro{\ifchilddocmanual}
% The conditional |\ifchilddoc| tells whether a
% child (true) or main (false) document is being compiled.
% The conditional |\ifchilddocmanual| tells whether
% the |\includeonly| mechanism is used (false) or
% the selection of child files must be performed manually (true).
% The definitions initialise to false:
%    \begin{macrocode}
\newif\ifchilddoc
\newif\ifchilddocmanual
%    \end{macrocode}

% \macro{\childdocname}
% \macro{\childdocjob}
% The macro |\childdocname| stores the name of the main document
% to be compiled. The macro |\childdocjob| stores the name of
% the document on which the \LaTeX{} compiler was originally invoked.
% The content of |\jobname| cannot be compared
% to filenames specified in the source due to different catcodes.
% The following code rescans |\jobname|, stores the result
% in |\childdocname| and saves a copy in |\childdocjob|:
%    \begin{macrocode}
\edef\childdocname{\scantokens\expandafter{\jobname\noexpand}}
\let\childdocjob\childdocname
%    \end{macrocode}

% \macro{\childdocdisable}
% The macro |\childdocdisable| prevents the main file
% from being processed more than once.
% At this stage, the main document command |\childdocmain|
% is assumed to be called once again where it should do nothing.
% Any subsequent call to it should prevent
% a secondary processing of the main document
% It overwrites the forwarding commands
% |\childdocof| and |\childdocforward|
% with empty macros to prevent further inclusions of the main document:
%    \begin{macrocode}
\newcommand{\childdocdisable}
{
  \renewcommand{\childdocmain}[1]{\renewcommand{\childdocmain}[1]{\endinput}}
  \renewcommand{\childdocof}[1]{}
  \renewcommand{\childdocby}[2][]{}
  \renewcommand{\childdocforward}[2][]{}
  \renewcommand{\childdocdisable}{}
}
%    \end{macrocode}

% \macro{\childdocmain}
% The macro |\childdocmain| is to be called at the top of the main file
% with nothing or the main filename (without extension) as argument.
% First, it breaks loops.
% If the argument is not empty and does not match |\childdocname|
% (which is set by the first inclusion of |childdoc.def|),
% |\ifchilddoc| is set to true, |\includeonly| is applied to the child file
% and |\jobname| is set to the main file
% (for proper handling of |.aux| files):
%    \begin{macrocode}
\newcommand{\childdocmain}[1]
{
  \childdocdisable\childdocmain{}
  \if?#1?\else
    \begingroup
      \def\childdoctmp{#1}
      \ifx\childdoctmp\childdocname
        \def\childdoctmp{}
      \else
        \def\childdoctmp
        {
          \childdoctrue
          \includeonly{\childdocname}
          \def\childdocjob{#1}
          \def\jobname{#1}
        }
      \fi
      \expandafter
    \endgroup
    \childdoctmp
  \fi
}
%    \end{macrocode}

% \macro{\childdocof}
% The command |\childdocof| redirects
% compilation to the main file |#1|.
%    \begin{macrocode}
\newcommand{\childdocof}[1]
{
  \childdocdisable
  \childdoctrue
  \includeonly{\childdocname}
  \def\jobname{#1}
  \def\childdocjob{#1}
  \input{#1}
}
%    \end{macrocode}

% \macro{\childdocby}
% The command |\childdocby| ....
%    \begin{macrocode}
\newcommand{\childdocby}[2][]
{
  \childdocdisable
  \childdoctrue
  \childdocmanualtrue
  \if?#1?\else
    \def\jobname{#2}
  \fi
  \def\childdocjob{#2}
  \input{#2}
  \endinput
}
%    \end{macrocode}

% \macro{\childdocforward}
% The command |\childdocforward| redirects
% compilation to the main file or
% (if the optional argument is given) a child file.
% Parameters are set as if the main file
% or a child file starting with |\childdocof| was compiled.
% Then compilation is handed over to the main file:
%    \begin{macrocode}
\newcommand{\childdocforward}[2][]
{
  \begingroup
    \if?#1?
      \def\childdoctmp
      {
        \def\childdocname{#2}
        \def\childdocjob{#2}
        \def\jobname{#2}
        \input{#2}
        \endinput
      }
    \else
      \def\childdoctmp
      {
        \childdocdisable
        \def\childdocname{#2}
        \childdoctrue
        \includeonly{#2}
        \def\childdocjob{#1}
        \def\jobname{#1}
        \input{#1}
        \endinput
      }
    \fi
    \expandafter
  \endgroup
  \childdoctmp
}
%    \end{macrocode}

% \macro{\childdocforwardprefix}
% The command |\childdocforwardprefix| redirects
% compilation to the main or a child file by means of a pattern.
% The prefix |#1| in the current filename is replaced by |#2|
% and the suffix of the current filename is kept
% (it is assumed that the filename does not contain the substring `|~~~|'
% which is used as a delimiter).
% Compilation is handed over to the new file by |\childdocforward|:
%    \begin{macrocode}
\newcommand{\childdocforwardprefix}[3][]
{
  \begingroup
    \def\childdocextract #2##1~~~{\def\childdoctmp{\childdocforward[#1]{#3##1}}}
    \expandafter\childdocextract\childdocname~~~
    \expandafter
  \endgroup
  \childdoctmp
}
%    \end{macrocode}

% \macro{\childdoc}
% The deprecated macro |\childdoc| is a legacy version of |\childdocmain|:
%    \begin{macrocode}
\newcommand{\childdoc}{\childdocmain}
%    \end{macrocode}

% \macro{\childdocredirect}
% The deprecated macro |\childdocredirect| is a legacy version
% of |\childdocforward| and |\childdocforwardprefix|:
%    \begin{macrocode}
\newcommand{\childdocredirect}[2][]
{
  \begingroup
    \if?#1?
      \def\childdoctmp{\childdocforward{#2}}
    \else
      \def\childdoctmp{\childdocforwardprefix{#1}{#2}}
    \fi
    \expandafter
  \endgroup
  \childdoctmp
}
%    \end{macrocode}

%\iffalse
%</package>
%\fi
%
\endinput
|\\
|\childdocmain{}|\\
\end{tabular}
\end{center}
at the very top of the main \LaTeX{} file,
in particular \emph{before} the |\documentclass| statement!
The argument of |\childdocmain| should be left empty
(but it must be present).

%%%%%%%%%%%%%%%%%%%%%%%%%%%%%%%%%%%%%%%%
\DescribeMacro{\childdocof}
Furthermore, add the commands
\begin{center}
\begin{tabular}{l}
|% \iffalse
%
% childdoc.dtx Copyright (C) 2017-2018 Niklas Beisert
%
% This work may be distributed and/or modified under the
% conditions of the LaTeX Project Public License, either version 1.3
% of this license or (at your option) any later version.
% The latest version of this license is in
%   http://www.latex-project.org/lppl.txt
% and version 1.3 or later is part of all distributions of LaTeX
% version 2005/12/01 or later.
%
% This work has the LPPL maintenance status `maintained'.
%
% The Current Maintainer of this work is Niklas Beisert.
%
% This work consists of the files childdoc.dtx and childdoc.ins
% and the derived files childdoc.def and cdocsamp.tex with
% cdocsch1.tex, cdocsch2.tex, cdocsdrf.tex, cdocsfn1.tex, cdocsfn2.tex.
%
%<package>\ifdefined\childdocmain\endinput\fi
%<package>\ProvidesFile{childdoc.def}[2018/12/30 v2.0 child document driver]
%<samplemain>\ProvidesFile{cdocsamp.tex}[2018/12/30 v2.0 sample for childdoc]
%<*driver>
%\ProvidesFile{childdoc.drv}[2018/12/30 v2.0 childdoc reference manual file]
\PassOptionsToClass{10pt,a4paper}{article}
\documentclass{ltxdoc}

\usepackage[margin=35mm]{geometry}
\usepackage{hyperref}
\usepackage{hyperxmp}
\usepackage[usenames]{color}

\hypersetup{colorlinks=true}
\hypersetup{pdfstartview=FitH}
\hypersetup{pdfpagemode=UseNone}
\hypersetup{pdfsource={}}
\hypersetup{pdflang={en-UK}}
\hypersetup{pdfcopyright={Copyright 2017-2018 Niklas Beisert.
  This work may be distributed and/or modified under the
  conditions of the LaTeX Project Public License, either version 1.3
  of this license or (at your option) any later version.}}
\hypersetup{pdflicenseurl={http://www.latex-project.org/lppl.txt}}
\hypersetup{pdfcontactaddress={ETH Zurich, ITP, HIT K,
  Wolfgang-Pauli-Strasse 27}}
\hypersetup{pdfcontactpostcode={8093}}
\hypersetup{pdfcontactcity={Zurich}}
\hypersetup{pdfcontactcountry={Switzerland}}
\hypersetup{pdfcontactemail={nbeisert@itp.phys.ethz.ch}}
\hypersetup{pdfcontacturl={http://people.phys.ethz.ch/\xmptilde nbeisert/}}

\newcommand{\secref}[1]{\hyperref[#1]{section \ref*{#1}}}

\parskip1ex
\parindent0pt
\let\olditemize\itemize
\def\itemize{\olditemize\parskip0pt}

\begin{document}

\title{The \textsf{childdoc} Package}
\hypersetup{pdftitle={The childdoc Package}}
\author{Niklas Beisert\\[2ex]
  Institut f\"ur Theoretische Physik\\
  Eidgen\"ossische Technische Hochschule Z\"urich\\
  Wolfgang-Pauli-Strasse 27, 8093 Z\"urich, Switzerland\\[1ex]
  \href{mailto:nbeisert@itp.phys.ethz.ch}
  {\texttt{nbeisert@itp.phys.ethz.ch}}}
\hypersetup{pdfauthor={Niklas Beisert}}
\hypersetup{pdfsubject={Manual for the LaTeX2e Package childdoc}}
\date{30 December 2018, \textsf{v2.0}}
\maketitle

\begin{abstract}\noindent
\textsf{childdoc} is a \LaTeXe{} package
that enables the direct compilation
of document sections included by |\include|
to individual files.
\end{abstract}

\begingroup
\parskip0ex
\tableofcontents
\endgroup

%%%%%%%%%%%%%%%%%%%%%%%%%%%%%%%%%%%%%%%%%%%%%%%%%%%%%%%%%%%%%%%%%%%%%%%%%%%%%%%%
%%%%%%%%%%%%%%%%%%%%%%%%%%%%%%%%%%%%%%%%%%%%%%%%%%%%%%%%%%%%%%%%%%%%%%%%%%%%%%%%
\section{Introduction}

\LaTeX{} provides a mechanism to structure a large document (such as a book)
into a main file and several child files (containing the chapters)
using the |\include| command.
This mechanism is beneficial for documents
which span hundreds of pages in order to
make the source file(s) more manageable.
Moreover, compilation can be restricted to
selected child files by means of the |\includeonly| command.
The latter feature can be used to reduce the compilation time while editing
(this was significantly more useful in the earlier days of \LaTeX{})
or to generate a smaller document which is easier to navigate.
Another application of |\includeonly| is to generate
documents consisting of selected parts of the complete document.

However, there are a few drawbacks of the plain |\include| mechanism:
\begin{itemize}
\item
The child files cannot be compiled on their own,
they can only be compiled via the main file.
A naive editing environment
(such as a text editor with an option
to have the current file processed by \LaTeX)
may require one to switch to the main file before compiling;
attempting to compile the child file produces errors.
\item
The main file must be modified (each time)
to adjust the |\includeonly| command
to the present needs. This easily leaves the main file in a messy state.
\item
The generated document will always carry the filename
of the main document. This is inconvenient if
several child files are to be compiled and
to be kept for distribution.
\end{itemize}

The present package provides a simple interface
to make child files individually compilable by \LaTeX{}.
Compiling a child file then has the same effect as compiling
the main file with an |\includeonly| command
to select the appropriate child.
Moreover the generated document will carry the name of the child
rather than the main file.
This resolves all three above issues.

This feature is meant to make the editing of books,
thesis documents and lecture notes somewhat more convenient.
However, the package can also be used efficiently for
composing a series of documents (such as exercise sheets)
which are typically distributed individually.
It then assists the author in generating the individual documents
(potentially in different versions)
as well as a document containing the collected series.
Another application is in developing style files
or other kinds of included material
where compilation of the style file could redirect
to a sample or test file.

%%%%%%%%%%%%%%%%%%%%%%%%%%%%%%%%%%%%%%%%%%%%%%%%%%%%%%%%%%%%%%%%%%%%%%%%%%%%%%%%
%%%%%%%%%%%%%%%%%%%%%%%%%%%%%%%%%%%%%%%%%%%%%%%%%%%%%%%%%%%%%%%%%%%%%%%%%%%%%%%%
\section{Usage}

First of all, the package \textsf{childdoc} is \emph{not} a standard
\LaTeXe{} |.sty| style file! Therefore it needs to be invoked in
a non-standard way.

%%%%%%%%%%%%%%%%%%%%%%%%%%%%%%%%%%%%%%%%%%%%%%%%%%%%%%%%%%%%%%%%%%%%%%%%%%%%%%%%
\subsection{Included Files}
\label{sec:include}

%%%%%%%%%%%%%%%%%%%%%%%%%%%%%%%%%%%%%%%%
\DescribeMacro{\childdocmain}
To use the package, add the commands
\begin{center}
\begin{tabular}{l}
|\input{childdoc.def}|\\
|\childdocmain{}|\\
\end{tabular}
\end{center}
at the very top of the main \LaTeX{} file,
in particular \emph{before} the |\documentclass| statement!
The argument of |\childdocmain| should be left empty
(but it must be present).

%%%%%%%%%%%%%%%%%%%%%%%%%%%%%%%%%%%%%%%%
\DescribeMacro{\childdocof}
Furthermore, add the commands
\begin{center}
\begin{tabular}{l}
|\input{childdoc.def}|\\
|\childdocof{|\textit{main}|}|\\
\end{tabular}
\end{center}
at the top of every child file \textit{child}
which is included by |\include{|\textit{child}|}|
from within the main file
(or at least for those files to be compiled individually).
The argument \textit{main} must be the filename of the main file.

There are a couple of
considerations in setting up the main and child documents:

%%%%%%%%%%%%%%%%%%%%%%%%%%%%%%%%%%%%%%%%
\paragraph{Restrictions.}

Please note the following restrictions:
\begin{itemize}
\item
|\childdocmain| must be called with one argument \textit{main}
to ensure compatibility with earlier version of the package.
It must either be empty (|\childdocmain{}|)
or precisely match the filename of the main file in which it is specified.
See \secref{sec:detection} for further information.
\item
The filename \textit{main} must be specified without the |.tex| extension.
\item
The filename \textit{main} is case sensitive
(even in case-insensitive file systems)
due to internal string comparison.
\item
The argument \textit{main} should be fully expanded, it cannot be a macro.
\item
Subdirectories and special characters should be avoided in filenames.
\item
The command |\childdocmain{|\textit{main}|}| must be followed by a whitespace.
It should not be followed immediately by another command
or by a comment mark `|%|'.
This is because the \TeX{} parser reads the token immediately following
the argument of |\childdocmain| and puts it
at the beginning of every child section;
however, a white\-space is ignored.
\end{itemize}

%%%%%%%%%%%%%%%%%%%%%%%%%%%%%%%%%%%%%%%%
\paragraph{Content of Main File.}

It is advisable to place all content in the child files included by |\include|.
Any output contained in the main file will appear in all child documents
unless suppressed manually;
it cannot be suppressed automatically by the |\includeonly| directive
and thus should normally be avoided.
A method to include some content in the main file
by means of conditional processing is described in \secref{sec:conditional}.

%%%%%%%%%%%%%%%%%%%%%%%%%%%%%%%%%%%%%%%%
\paragraph{Page Numbering.}

When only a part of the document is compiled,
the appropriate numbering of pages
(as well as other status parameters)
is determined from the |.aux| files.
The latter contain information from previous passes.
However this information needs to propagate through
all intermediate child documents.
Therefore the page numbering in child documents may well
be inconsistent until the complete document is compiled at least once.

A useful (if unconventional) way to always ensure a consistent
page numbering is to restart the numbering in each child document
and denote the pages by `\textit{child}|.|\textit{page}'
where \textit{child} represents the chapter/section number of the child file.
This can be achieved by the command
|\numberwithin{page}{|\textit{child}|}|
of the \textsf{amsmath} package
where \textit{child} can be |chapter| or |section|
depending on the chosen structuring.
Alternatively, one can modify the macro |\thepage| appropriately
and reset the counter |page| at the start of each child file.

%%%%%%%%%%%%%%%%%%%%%%%%%%%%%%%%%%%%%%%%%%%%%%%%%%%%%%%%%%%%%%%%%%%%%%%%%%%%%%%%
\subsection{Conditional Processing}
\label{sec:conditional}

The package provides a mechanism to compile different versions
of a document. To customise the versions further some conditional processing
can come in handy to distinguish which version is being compiled.
The package provides two macros to describe the compilation context:

%%%%%%%%%%%%%%%%%%%%%%%%%%%%%%%%%%%%%%%%
\DescribeMacro{\ifchilddoc}
The conditional |\ifchilddoc| distinguishes between the compilation of
child documents and the main document:
%
\begin{center}
|\ifchilddoc |\textit{child-code}| |[|\||else |\textit{main-code}]| \||fi|
\end{center}

%%%%%%%%%%%%%%%%%%%%%%%%%%%%%%%%%%%%%%%%
\DescribeMacro{\childdocname}
\DescribeMacro{\childdocjob}
The macro |\childdocname| contains the filename (without extension)
of the main or child file being processed.
Note that |\childdocjob| will always contain the name of the main file.

%%%%%%%%%%%%%%%%%%%%%%%%%%%%%%%%%%%%%%%%
\paragraph{Title Page.}

Conditional processing can be used to include a title or banner page
in the main document when proper precautions are taken.
Importantly, the code in the main file should ensure that the page counter
(as well as other status parameters which are stored in the |.aux| files)
takes the same value after the conditional processing.
Otherwise the page numbers may take divergent values
depending on which part is compiled.

For example, a title page could be declared by:
%
\begin{center}
\begin{tabular}{l}
|\ifchilddoc\||else|\\
|\addtocounter{page}{-1}|\\
\textit{code for title page}\\
|\newpage|\\
|\||fi|
\end{tabular}
\end{center}
%
A banner page for the child documents can be generated by:
%
\begin{center}
\begin{tabular}{l}
|\ifchilddoc|\\
|\addtocounter{page}{-1}|\\
\textit{code for banner page}\\
|\newpage|\\
|\||fi|
\end{tabular}
\end{center}
%
Here one could write a message such as:
\begin{center}
|This is the part \childdocname{} of \childdocjob{}.|
\end{center}

%%%%%%%%%%%%%%%%%%%%%%%%%%%%%%%%%%%%%%%%%%%%%%%%%%%%%%%%%%%%%%%%%%%%%%%%%%%%%%%%
\subsection{Flags}
\label{sec:flags}

The package makes it easy to generate different versions
of the main or child documents.
To this end compilation flags can be defined
and assigned different default values.
They will be particularly useful in conjunction
with the forwarding mechanism described in \secref{sec:forward}.

For example, it may be useful to have a flag |\version|
which can be set to |draft| or |final|.
The document source will contain some conditional code
depending on the value of |\version|.
Suppose further, the flag should default to |final| for the main file
and to |draft| for child files
which is a natural assignment for editing the document.
This is achieved by placing the following code
in the preamble of the main document
(below the |\childdocmain| directive):
%
\begin{center}
\begin{tabular}{l}
|\ifchilddoc|\\
|\providecommand{\version}{draft}|\\
|\||else|\\
|\providecommand{\version}{final}|\\
|\||fi|
\end{tabular}
\end{center}
%
The definition by |\providecommand| makes sure
that previous definitions are not overwritten.
Further statements |\providecommand{\version}{...}|
can thus be added before the above code to override it.

For the main file, one might add a line
(between |\childdocmain| and the above block)
%
\begin{center}
|%\ifchilddoc\||else\providecommand{\version}{draft}\||fi|
\end{center}
%
which can be uncommented to produce a draft version.
Likewise one can add a line to the very top of a child file
(above the |\childdocof{|\textit{main}|}| directive)
%
\begin{center}
|%\providecommand{\version}{final}|
\end{center}
%
which can be uncommented to produce the final version of this child document.

%%%%%%%%%%%%%%%%%%%%%%%%%%%%%%%%%%%%%%%%%%%%%%%%%%%%%%%%%%%%%%%%%%%%%%%%%%%%%%%%
\subsection{Forwarding}
\label{sec:forward}

Different versions of the main or child documents
using compilation flags as described in \secref{sec:flags}
can be (permanently) stored in different files
for convenient compilation, viewing and distribution.
To this end, the package defines a command
to pass on compilation to a different file:

%%%%%%%%%%%%%%%%%%%%%%%%%%%%%%%%%%%%%%%%
\DescribeMacro{\childdocforward}
The command |\childdocforward| redirects processing to
another source file:
%
\begin{center}
\begin{tabular}{l}
|\input{childdoc.def}|\\
|\childdocforward[|\textit{main}|]{|\textit{dest}|}|\\
\end{tabular}
\end{center}
%
The argument \textit{dest} is the destination file
(without extension).
It should be the main file or one of the child files.
Note that further \textsf{childdoc} directives
such as |\childdocof| and |\childdocforward|
in the indicated file will be processed in this form.
The optional argument \textit{main}
passes on directly to the main file \textit{main}
while pretending to compile the child \textit{dest}.
This form behaves as if \textit{dest}
issues |\childdocof{|\textit{main}|}| right away,
and no further \textsf{childdoc} directives will be processed.

%%%%%%%%%%%%%%%%%%%%%%%%%%%%%%%%%%%%%%%%
\DescribeMacro{\...prefix}
In the alternative form |\childdocforwardprefix|,
%
\begin{center}
\begin{tabular}{l}
|\input{childdoc.def}|\\
|\childdocforwardprefix[|\textit{main}|]{|\textit{prefix}|}{|\textit{dest}|}|
\end{tabular}
\end{center}
%
the destination file is determined by a pattern
depending on the current file:
To make this work, the current file must be called
`{\textit{prefix}\hspace{0.2em}\textit{suffix}}'
with \textit{prefix} matching precisely the argument.
Processing is then passed on to the file
`{\textit{dest}\hspace{0.2em}\textit{suffix}}'.
Surely, the same effect is achieved by
directly specifying the
argument `{\textit{dest}\hspace{0.2em}\textit{suffix}}'
in the first form.
However, that requires to set up a different file
for each child. With the alternative form of the command
all these files can have exactly the same content
which simplifies setting them up and maintaining them.

For example, the following file |draft.tex|
with a compilation flag |\version| as described in \secref{sec:flags}
compiles the main document as a draft:
%
\begin{center}
\begin{tabular}{l}
|\def\version{draft}|\\
|\input{childdoc.def}|\\
|\childdocforward{|\textit{main}|}|
\end{tabular}
\end{center}
%
Likewise, the following files |final|\textit{nn}|.tex|
compile the final version of the child document
|child|\textit{nn}|.tex|:
%
\begin{center}
\begin{tabular}{l}
|\def\version{final}|\\
|\input{childdoc.def}|\\
|\childdocforwardprefix{final}{child}|
\end{tabular}
\end{center}
%

Note that when several versions of a main file and/or of each child file
are to be generated, it may be convenient to set up a |Makefile| or
shell script to automatise the process.

%%%%%%%%%%%%%%%%%%%%%%%%%%%%%%%%%%%%%%%%%%%%%%%%%%%%%%%%%%%%%%%%%%%%%%%%%%%%%%%%
\subsection{Command Line Processing}
\label{sec:commandline}

The effect of redirection files can also be achieved by invoking
the \LaTeX{} compiler with a more elaborate command line.
Most conveniently this should be done as part
of a shell script or a |Makefile|.

When using \textsf{childdoc} in the main file, the following
command lines effectively perform a redirection
(note that depending on the shell being used,
backslashes may have to be doubled: `|\|' $\to$ `|\\|'):
%
\begin{center}
|... -jobname "|\textit{target}|" |\\|"|[\textit{flags}]%
|\input{childdoc.def}\childdocforward[|\textit{main}|]{|\textit{dest}|}"|
\end{center}
%
Here \textit{target} is the name of the output file,
\textit{main} is the name of the main file
and \textit{dest} is the name of the main or child file to be processed
(all filenames without extensions).
The optional argument \textit{main} can be omitted
if \textit{main} matches \textit{dest}.
Optionally, compilation \textit{flags} can be defined via |\def| commands.
This command line makes the \TeX{} engine believe
it is compiling the file \textit{target}
whose content is specified as the latter parameter.
The provided code then forwards the processing to
\textit{main} or \textit{dest} as described in \secref{sec:forward}.

%%%%%%%%%%%%%%%%%%%%%%%%%%%%%%%%%%%%%%%%%%%%%%%%%%%%%%%%%%%%%%%%%%%%%%%%%%%%%%%%
\subsection{Include by Input}
\label{sec:input}

Including child documents by |\include| has some restrictions by design.
Most notably, the content of a child document always occupies
its own set of pages; pages cannot be shared between child documents.
Usually, this behaviour makes perfect sense
because each child document contain an essential part of the document.
However, in some situations it may be desirable to compose
a document from a collection of parts
without having mandatory page breaks between then.
For this case, the package
provides a mechanism to include parts
by |\input| which can also be processed individually.
However, by construction this mechanism
requires manual handling of the content to be output.

%%%%%%%%%%%%%%%%%%%%%%%%%%%%%%%%%%%%%%%%
\DescribeMacro{\ifchilddocmanual}
The main file should be prepared as usual, see \secref{sec:include}.
However, the document body must make a distinction
between processing of an individual part and of the main document, e.g.:
%
\begin{center}
\begin{tabular}{l}
|\ifchilddocmanual|\\
|\input{\childdocname}|\\
|\||else|\\
\textit{document body with }|\input{|\textit{part}|}|\\
|\||fi|
\end{tabular}
\end{center}
%
The conditional |\ifchilddocmanual| is true whenever
a part to be included by |\input| is being compiled,
and the name of the part is stored in |\childdocname|.

%%%%%%%%%%%%%%%%%%%%%%%%%%%%%%%%%%%%%%%%
\DescribeMacro{\childdocby}
Each part to be included by |\input| should start with:
%
\begin{center}
\begin{tabular}{l}
|\input{childdoc.def}|\\
|\childdocby{|\textit{main}|}|\\
\end{tabular}
\end{center}
%
The directive |\childdocby| is similar to |\childdocof|
described in \secref{sec:include},
but the subsequent selection of content must be done manually.
To that end, both |\ifchilddoc| and |\ifchilddocmanual|
will be true upon processing of a part,
and the name of the part is stored in |\childdocname|.
Note that |\jobname| will be set to the filename of the current part
so that each part receives an individual |.aux| file
that does not interfere with the |.aux| file(s) of the main document.
This behaviour can be altered by the alternative form
|\childdocby[*]{|\textit{main}|}| (with a non-empty optional argument)
which uses the |.aux| file of the main document
by setting |\jobname| to \textit{main}.

%%%%%%%%%%%%%%%%%%%%%%%%%%%%%%%%%%%%%%%%%%%%%%%%%%%%%%%%%%%%%%%%%%%%%%%%%%%%%%%%
\subsection{Driver Development}
\label{sec:driver}

The \textsf{childdoc} mechanism can also be use for the development
of definition files such as \LaTeX{} styles or classes.
This case differs from the above setup with multiple parts
included by |\include| in that no |\includeonly| should be invoked.
This can be achieved by starting the include file
(before |\ProvidesPackage|) with:
%
\begin{center}
\begin{tabular}{l}
|\input{childdoc.def}|\\
|\childdocforward{|\textit{main}|}|\\
\end{tabular}
\end{center}
%
or alternatively with:
%
\begin{center}
\begin{tabular}{l}
|\input{childdoc.def}|\\
|\childdocby{|\textit{main}|}|\\
\end{tabular}
\end{center}
%
Both forms have slightly different effects as described above.
The main file is prepared as usual, see \secref{sec:include}.

%%%%%%%%%%%%%%%%%%%%%%%%%%%%%%%%%%%%%%%%%%%%%%%%%%%%%%%%%%%%%%%%%%%%%%%%%%%%%%%%
\subsection{Legacy Detection}
\label{sec:detection}

The directive |\childdocmain| in the main file can detect
whether the complete document or merely a child is to be compiled
even without using the directive |\childdocof|.
This method is deprecated because it is less robust
and there is no compelling reason to use it;
it is merely provided for backward compatibility
and it may be removed in future versions.

If the detection mechanism is to be used,
it is mandatory to correctly specify
the filename of the main file as the argument of |\childdocmain|:
%
\begin{center}
\begin{tabular}{l}
|\input{childdoc.def}|\\
|\childdocmain{|\textit{main}|}|\\
\end{tabular}
\end{center}
%
If |\jobname| does not match the argument \textit{main} of |\childdocmain|,
it is assumed that |\jobname| points to the child file to be compiled.
When using |\childdocmain| with the main file specified as argument,
it suffices to start a child file
with just |\input{|\textit{main}|}|
without loading of the package and using |\childdocof|.
If instead all processing is done
with the appropriate \textsf{childdoc} directives,
the argument of \textit{main} of |\childdocmain| can be empty.

An alternative version of the command line processing described
in \secref{sec:commandline} using the detection mechanism reads:
%
\begin{center}
|... -jobname "|\textit{target}|" "|[\textit{flags}]%
[|\def\jobname{|\textit{dest}|}|]|\input{|\textit{main}|}"|
\end{center}

%%%%%%%%%%%%%%%%%%%%%%%%%%%%%%%%%%%%%%%%%%%%%%%%%%%%%%%%%%%%%%%%%%%%%%%%%%%%%%%%
\subsection{Manual Code}
\label{sec:manual}

In case one cannot be certain whether the definitions file |childdoc.def|
is installed on the target \TeX{} distribution
and one prefers not to ship it,
it is conceivable to paste a few relevant commands into the sources.

To that end, drop all statements |\input{childdoc.def}|
and perform the replacements as outlined below.
Instead of |\childdocmain{|\textit{main}|}| add the following code
to the top of the main file:
%
\begin{center}
\begin{tabular}{l}
|\||ifdefined\childdocname\endinput\||fi\newif\ifchilddoc|\\
|\edef\childdocname{\scantokens\expandafter{\jobname\noexpand}}|\\
|\def\childdocmain{|\textit{main}|}\||ifx\childdocmain\childdocname\||else|\\
|\childdoctrue\includeonly{\childdocname}\let\jobname\childdocmain\||fi|\\
\end{tabular}
\end{center}
%
Instead of |\childdocof{|\textit{main}|}| just include the main file
at the top of each child file:
%
\begin{center}
|\input{|\textit{main}|}|
\end{center}
%
A simple redirection |\childdocforward{|\textit{dest}|}| is achieved by:
%
\begin{center}
|\def\jobname{|\textit{dest}|}\input{\jobname}|
\end{center}
%
The redirection with prefix
|\childdocforwardprefix[|\textit{prefix}|]{|\textit{dest}|}|
is accomplished by:
%
\begin{center}
\begin{tabular}{l}
|{\edef\jobname{\scantokens\expandafter{\jobname\noexpand}}|\\
|\def\redirectjob |\textit{prefix}|#1~~~{\gdef\jobname{|\textit{dest}|#1}}|\\
|\expandafter\redirectjob\jobname~~~}\input{\jobname}|
\end{tabular}
\end{center}

In an alternative approach,
child documents can be compiled by a specific command line
without additional code or specific definitions:
%
\begin{center}
|... -jobname "|\textit{target}|" "|[\textit{flags}]%
|\includeonly{|\textit{dest}|}\input{|\textit{main}|}"|
\end{center}
%

%%%%%%%%%%%%%%%%%%%%%%%%%%%%%%%%%%%%%%%%%%%%%%%%%%%%%%%%%%%%%%%%%%%%%%%%%%%%%%%%
%%%%%%%%%%%%%%%%%%%%%%%%%%%%%%%%%%%%%%%%%%%%%%%%%%%%%%%%%%%%%%%%%%%%%%%%%%%%%%%%
\section{Information}

%%%%%%%%%%%%%%%%%%%%%%%%%%%%%%%%%%%%%%%%%%%%%%%%%%%%%%%%%%%%%%%%%%%%%%%%%%%%%%%%
\subsection{Copyright}

Copyright \copyright{} 2017--2018 Niklas Beisert

This work may be distributed and/or modified under the
conditions of the \LaTeX{} Project Public License, either version 1.3
of this license or (at your option) any later version.
The latest version of this license is in
  \url{http://www.latex-project.org/lppl.txt}
and version 1.3 or later is part of all distributions of \LaTeX{}
version 2005/12/01 or later.

This work has the LPPL maintenance status `maintained'.

The Current Maintainer of this work is Niklas Beisert.

This work consists of the files |README.txt|, |childdoc.ins| and |childdoc.dtx|
as well as the derived files |childdoc.def|, |cdocsamp.tex|
with |cdocsch1.tex|, |cdocsch2.tex|, |cdocspt3.tex|, |cdocspt4.tex|,
|cdocsdrf.tex|, |cdocsfn1.tex|, |cdocsfn2.tex|
as well as |childdoc.pdf|.

%%%%%%%%%%%%%%%%%%%%%%%%%%%%%%%%%%%%%%%%%%%%%%%%%%%%%%%%%%%%%%%%%%%%%%%%%%%%%%%%
\subsection{Files and Installation}

The package consists of the files:
%
\begin{center}
\begin{tabular}{ll}
    |README.txt|   & readme file \\
    |childdoc.ins| & installation file \\
    |childdoc.dtx| & source file \\
    |childdoc.def| & definition file \\
    |cdocsamp.tex| & sample main file \\
    |cdocsch1.tex| & sample include file \\
    |cdocsch2.tex| & sample include file \\
    |cdocspt3.tex| & sample part file \\
    |cdocspt4.tex| & sample part file \\
    |cdocsdrf.tex| & sample redirection file \\
    |cdocsfn1.tex| & sample redirection file \\
    |cdocsfn2.tex| & sample redirection file \\
    |childdoc.pdf| & manual
\end{tabular}
\end{center}
%
The distribution consists of the files
|README.txt|, |childdoc.ins| and |childdoc.dtx|.
%
\begin{itemize}
\item
Run (pdf)\LaTeX{} on |childdoc.dtx|
to compile the manual |childdoc.pdf| (this file).
\item
Run \LaTeX{} on |childdoc.ins| to create the definitions file |childdoc.def|
and the sample |cdocsamp.tex| with include files
|cdocsch1.tex|, |cdocsch2.tex|, |cdocspt3.tex|, |cdocspt4.tex|,
|cdocsdrf.tex|, |cdocsfn1.tex|, |cdocsfn2.tex|.
Then copy the file |childdoc.def| to an appropriate directory of your \LaTeX{}
distribution, e.g.\ \textit{texmf-root}|/tex/latex/childdoc|.
\end{itemize}

%%%%%%%%%%%%%%%%%%%%%%%%%%%%%%%%%%%%%%%%%%%%%%%%%%%%%%%%%%%%%%%%%%%%%%%%%%%%%%%%
\subsection{Related CTAN Packages}

There are several other packages which offer a similar functionality:
%
\begin{itemize}
\item
The packages
\href{http://ctan.org/pkg/docmute}{\textsf{docmute}},
\href{http://ctan.org/pkg/includex}{\textsf{includex}} and
\href{http://ctan.org/pkg/standalone}{\textsf{standalone}}
provide commands to include only the document body of
a child file thus allowing both files to be compiled individually.
\item
The packages \href{http://ctan.org/pkg/subdocs}{\textsf{subdocs}}
and \href{http://ctan.org/pkg/subfiles}{\textsf{subfiles}}
provide structures in which the main and child documents can be
encapsulated and allowing them to be compiled individually.
The inclusion mechanism is different from the conventional |\include|.
\item
The package \href{http://ctan.org/pkg/combine}{\textsf{combine}}
is an elaborate solution to combine several documents into one.
\end{itemize}
%
See also the CTAN topic \href{http://ctan.org/topic/subdocs}{\textsf{subdocs}}
for further related packages.
The present package differs from the above solutions in that
a document structure constructed with the conventional |\include| mechanism
just needs two extra commands at the top of every file
such that all constituent files can be compiled individually.

%%%%%%%%%%%%%%%%%%%%%%%%%%%%%%%%%%%%%%%%%%%%%%%%%%%%%%%%%%%%%%%%%%%%%%%%%%%%%%%%
%\subsection{Feature Suggestions}
%
%The following is a list of features which may be useful for future
%versions of this package:
%%
%\begin{itemize}
%\item
%\ldots
%\end{itemize}

%%%%%%%%%%%%%%%%%%%%%%%%%%%%%%%%%%%%%%%%%%%%%%%%%%%%%%%%%%%%%%%%%%%%%%%%%%%%%%%%
\subsection{Revision History}

%%%%%%%%%%%%%%%%%%%%%%%%%%%%%%%%%%%%%%%%
\paragraph{v2.0:} 2018/12/30

\begin{itemize}
\item
immediate forward processing
\item
added |\childdocby| mechanism
\item
manual restructured
\end{itemize}

%%%%%%%%%%%%%%%%%%%%%%%%%%%%%%%%%%%%%%%%
\paragraph{v1.6:} 2018/01/17

\begin{itemize}
\item
application for development of include files
\item
corrections to manual
\end{itemize}

%%%%%%%%%%%%%%%%%%%%%%%%%%%%%%%%%%%%%%%%
\paragraph{v1.5:} 2017/05/21

\begin{itemize}
\item
more complete structuring introduced
\item
|\childdocof| introduced
\item
|\childdoc| renamed to |\childdocmain|
\item
|\childredirect| renamed to |\childdocforward| and |\childdocforwardprefix|
and functionality expanded
\end{itemize}

%%%%%%%%%%%%%%%%%%%%%%%%%%%%%%%%%%%%%%%%
\paragraph{v1.0:} 2017/04/27

\begin{itemize}
\item
manual and install package
\item
first version published on CTAN
\end{itemize}

%%%%%%%%%%%%%%%%%%%%%%%%%%%%%%%%%%%%%%%%
\paragraph{v0.6:} 2017/04/26

\begin{itemize}
\item
redirection mechanism added
\end{itemize}

%%%%%%%%%%%%%%%%%%%%%%%%%%%%%%%%%%%%%%%%
\paragraph{v0.5:} 2017/04/26

\begin{itemize}
\item
functionality in definition file
\end{itemize}


%%%%%%%%%%%%%%%%%%%%%%%%%%%%%%%%%%%%%%%%%%%%%%%%%%%%%%%%%%%%%%%%%%%%%%%%%%%%%%%%
%%%%%%%%%%%%%%%%%%%%%%%%%%%%%%%%%%%%%%%%%%%%%%%%%%%%%%%%%%%%%%%%%%%%%%%%%%%%%%%%
%%%%%%%%%%%%%%%%%%%%%%%%%%%%%%%%%%%%%%%%%%%%%%%%%%%%%%%%%%%%%%%%%%%%%%%%%%%%%%%%
\appendix

\settowidth\MacroIndent{\rmfamily\scriptsize 000\ }

 \DocInput{childdoc.dtx}

\end{document}
%</driver>
% \fi
%
% %%%%%%%%%%%%%%%%%%%%%%%%%%%%%%%%%%%%%%%%%%%%%%%%%%%%%%%%%%%%%%%%%%%%%%%%%%%%%%
% %%%%%%%%%%%%%%%%%%%%%%%%%%%%%%%%%%%%%%%%%%%%%%%%%%%%%%%%%%%%%%%%%%%%%%%%%%%%%%
% \section{Sample}
%\iffalse
%<*samplemain>
%\fi
%
% The following presents a sample document
% with two chapters, two parts, a title page,
% a compile flag as well as three forwarding files to set the flag.
% It consists of eight |.tex| files:
% \begin{center}
% \begin{tabular}{ll}
% |cdocsamp.tex|&main file\\
% |cdocsch1.tex|&include file for chapter 1\\
% |cdocsch2.tex|&include file for chapter 2\\
% |cdocspt3.tex|&include file for part 3\\
% |cdocspt4.tex|&include file for part 4\\
% |cdocsdrf.tex|&forwarding file for main file in draft mode\\
% |cdocsfi1.tex|&forwarding file for final version of chapter 1\\
% |cdocsfi2.tex|&forwarding file for final version of chapter 2\\
% \end{tabular}
% \end{center}
% Each of the eight files can be compiled directly by the \LaTeX{} compiler.
%
% %%%%%%%%%%%%%%%%%%%%%%%%%%%%%%%%%%%%%%
% \paragraph{Main File.}
%
% The main file is called |cdocsamp.tex|.
%
% Load the \textsf{childdoc} definitions and
% declare the filename for the main document:
%    \begin{macrocode}
\input{childdoc.def}
\childdocmain{}
%    \end{macrocode}

% Optional override for |\version| flag:
%    \begin{macrocode}
%%\ifchilddoc\else\providecommand{\version}{draft}\fi
%    \end{macrocode}

% Define the default values for the |\version| flag
% (|final| for the main file and |draft| for childs):
%    \begin{macrocode}
\ifchilddoc
\providecommand{\version}{draft}
\else
\providecommand{\version}{final}
\fi
%    \end{macrocode}

% Load the standard document class:
%    \begin{macrocode}
\documentclass[12pt]{article}
%    \end{macrocode}

% Start the document body:
%    \begin{macrocode}
\begin{document}
%    \end{macrocode}

% Declare a title page.
% Print title, part of document being processed and version flag:
%    \begin{macrocode}
\addtocounter{page}{-1}
\begin{center}
{\LARGE\bfseries{}childdoc example\par}
\vspace{1cm}
\ifchilddoc
\ifchilddocmanual part\else chapter\fi:
`\childdocname' of `\childdocjob'\par
\else
main document: `\childdocjob'\par
\fi
version: \version\par
\end{center}
\newpage
%    \end{macrocode}

% Manually include selected file,
% otherwise process as usual:
%    \begin{macrocode}
\ifchilddocmanual
\section*{part `\childdocname'}
\input{\childdocname}
\else
%    \end{macrocode}

% Include the two chapters:
%    \begin{macrocode}
\include{cdocsch1}
\include{cdocsch2}
%    \end{macrocode}

% Include the two parts unless only chapters should be displayed:
%    \begin{macrocode}
\ifchilddoc\else
\section{part three}
\input{cdocspt3}
\section{part four}
\input{cdocspt4}
\fi
%    \end{macrocode}

% Process as usual until here:
%    \begin{macrocode}
\fi
%    \end{macrocode}

% End of document body:
%    \begin{macrocode}
\end{document}
%    \end{macrocode}
%\iffalse
%</samplemain>
%\fi
%
% %%%%%%%%%%%%%%%%%%%%%%%%%%%%%%%%%%%%%%
% \paragraph{Chapter Include Files.}
%
% The include files are called |cdocsch1.tex| and |cdocsch2.tex|.
%
%\iffalse
%<*samplechap1|samplechap2>
%\fi

% Optional override for |\version| flag:
%    \begin{macrocode}
%%\providecommand{\version}{final}
%    \end{macrocode}

% Include the main document:
%    \begin{macrocode}
\input{childdoc.def}
\childdocof{cdocsamp}
%    \end{macrocode}

%\iffalse
%</samplechap1|samplechap2>
%\fi
%
%\iffalse
%<*samplechap1>
%\fi
% Some text for chapter 1:
%    \begin{macrocode}
\section{one}
some text in chapter one
%    \end{macrocode}

%\iffalse
%</samplechap1>
%\fi
% Some text for chapter 2:
%\iffalse
%<*samplechap2>
%\fi
%    \begin{macrocode}
\section{two}
more text in chapter two
%    \end{macrocode}

%\iffalse
%</samplechap2>
%\fi
%
% %%%%%%%%%%%%%%%%%%%%%%%%%%%%%%%%%%%%%%
% \paragraph{Part Include Files.}
%
% The include files are called |cdocspt3.tex| and |cdocspt4.tex|.
%
%\iffalse
%<*samplepart3|samplepart4>
%\fi

% Optional override for |\version| flag:
%    \begin{macrocode}
%%\providecommand{\version}{final}
%    \end{macrocode}

% Include the main document:
%    \begin{macrocode}
\input{childdoc.def}
\childdocby{cdocsamp}
%    \end{macrocode}

%\iffalse
%</samplepart3|samplepart4>
%\fi
%
%\iffalse
%<*samplepart3>
%\fi
% Some text for part 3:
%    \begin{macrocode}
some text in part three
%    \end{macrocode}

%\iffalse
%</samplepart3>
%\fi
% Some text for part 4:
%\iffalse
%<*samplepart4>
%\fi
%    \begin{macrocode}
more text in part four
%    \end{macrocode}

%\iffalse
%</samplepart4>
%\fi
%
% %%%%%%%%%%%%%%%%%%%%%%%%%%%%%%%%%%%%%%
% \paragraph{Forwarding for a Complete Draft.}
%
% The following forwarding file |cdocsdrf.tex|
% compiles the main document in draft mode:
%\iffalse
%<*sampledraft>
%\fi
%    \begin{macrocode}
\def\version{draft}
\input{childdoc.def}
\childdocforward{cdocsamp}
%    \end{macrocode}

%\iffalse
%</sampledraft>
%\fi
%
% %%%%%%%%%%%%%%%%%%%%%%%%%%%%%%%%%%%%%%
% \paragraph{Forwarding for Final Version of the Chapters.}
%
% The following forwarding files |cdocsfn1.tex| and |cdocsfn2.tex|
% (with identical content)
% compile the final versions of the child documents
% |cdocsch1.tex| and |cdocsch2.tex|, respectively:
%\iffalse
%<*samplefinal>
%\fi
%    \begin{macrocode}
\def\version{final}
\input{childdoc.def}
\childdocforwardprefix[cdocsamp]{cdocsfn}{cdocsch}
%    \end{macrocode}

%\iffalse
%</samplefinal>
%\fi
%
% %%%%%%%%%%%%%%%%%%%%%%%%%%%%%%%%%%%%%%
% \paragraph{Command Line Processing.}
%
% The following three command lines generate the output files
% |cdocscld|, |cdocscl1| and |cdocscl2|
% which should be identical to
% |cdocsdrf|, |cdocsch1| and |cdocsfn2|, respectively:
% \begin{center}
% \begin{tabular}{l}
% |latex -jobname cdocscld \|\\
% |  "\def\version{draft}\input{childdoc.def}\childdocforward{cdocsamp}"|\\
% |latex -jobname cdocscl1 \|\\
% |  "\input{childdoc.def}\childdocforward[cdocsamp]{cdocsch1}"|\\
% |latex -jobname cdocscl2 \|\\
% |  "\def\version{final}\input{childdoc.def}\childdocforward{cdocsch2}"|
% \end{tabular}
% \end{center}
% Note that the trailing backslash on each first line
% merely continues the input to the second line
% (for convenient cut ant paste).
% Furthermore, the command |latex| can be replaced by any
% of its alternative versions such as |pdflatex|.
%
% %%%%%%%%%%%%%%%%%%%%%%%%%%%%%%%%%%%%%%%%%%%%%%%%%%%%%%%%%%%%%%%%%%%%%%%%%%%%%%
% %%%%%%%%%%%%%%%%%%%%%%%%%%%%%%%%%%%%%%%%%%%%%%%%%%%%%%%%%%%%%%%%%%%%%%%%%%%%%%
% \section{Implementation}
%\iffalse
%<*package>
%\fi
%
% This section describes the definitions file |childdoc.def|.

% The definitions cannot be loaded using |\usepackage| or |\RequirePackage|
% which has a mechanism to prevent loading a style file more than once.
% When loading the definitions by means of |\input|
% multiple instances have to be prevented manually:
%\iffalse
%This code needs to be before the `\ProvidesFile' directive
%which is defined at the beginning of this file.
%Therefore it is also placed there and commented out here.
%</package>
%<*discard>
%\fi
%    \begin{macrocode}
\ifdefined\childdocmain\endinput\fi
%    \end{macrocode}
%\iffalse
%</discard>
%<*package>
%\fi
%
% \macro{\ifchilddoc}
% \macro{\ifchilddocmanual}
% The conditional |\ifchilddoc| tells whether a
% child (true) or main (false) document is being compiled.
% The conditional |\ifchilddocmanual| tells whether
% the |\includeonly| mechanism is used (false) or
% the selection of child files must be performed manually (true).
% The definitions initialise to false:
%    \begin{macrocode}
\newif\ifchilddoc
\newif\ifchilddocmanual
%    \end{macrocode}

% \macro{\childdocname}
% \macro{\childdocjob}
% The macro |\childdocname| stores the name of the main document
% to be compiled. The macro |\childdocjob| stores the name of
% the document on which the \LaTeX{} compiler was originally invoked.
% The content of |\jobname| cannot be compared
% to filenames specified in the source due to different catcodes.
% The following code rescans |\jobname|, stores the result
% in |\childdocname| and saves a copy in |\childdocjob|:
%    \begin{macrocode}
\edef\childdocname{\scantokens\expandafter{\jobname\noexpand}}
\let\childdocjob\childdocname
%    \end{macrocode}

% \macro{\childdocdisable}
% The macro |\childdocdisable| prevents the main file
% from being processed more than once.
% At this stage, the main document command |\childdocmain|
% is assumed to be called once again where it should do nothing.
% Any subsequent call to it should prevent
% a secondary processing of the main document
% It overwrites the forwarding commands
% |\childdocof| and |\childdocforward|
% with empty macros to prevent further inclusions of the main document:
%    \begin{macrocode}
\newcommand{\childdocdisable}
{
  \renewcommand{\childdocmain}[1]{\renewcommand{\childdocmain}[1]{\endinput}}
  \renewcommand{\childdocof}[1]{}
  \renewcommand{\childdocby}[2][]{}
  \renewcommand{\childdocforward}[2][]{}
  \renewcommand{\childdocdisable}{}
}
%    \end{macrocode}

% \macro{\childdocmain}
% The macro |\childdocmain| is to be called at the top of the main file
% with nothing or the main filename (without extension) as argument.
% First, it breaks loops.
% If the argument is not empty and does not match |\childdocname|
% (which is set by the first inclusion of |childdoc.def|),
% |\ifchilddoc| is set to true, |\includeonly| is applied to the child file
% and |\jobname| is set to the main file
% (for proper handling of |.aux| files):
%    \begin{macrocode}
\newcommand{\childdocmain}[1]
{
  \childdocdisable\childdocmain{}
  \if?#1?\else
    \begingroup
      \def\childdoctmp{#1}
      \ifx\childdoctmp\childdocname
        \def\childdoctmp{}
      \else
        \def\childdoctmp
        {
          \childdoctrue
          \includeonly{\childdocname}
          \def\childdocjob{#1}
          \def\jobname{#1}
        }
      \fi
      \expandafter
    \endgroup
    \childdoctmp
  \fi
}
%    \end{macrocode}

% \macro{\childdocof}
% The command |\childdocof| redirects
% compilation to the main file |#1|.
%    \begin{macrocode}
\newcommand{\childdocof}[1]
{
  \childdocdisable
  \childdoctrue
  \includeonly{\childdocname}
  \def\jobname{#1}
  \def\childdocjob{#1}
  \input{#1}
}
%    \end{macrocode}

% \macro{\childdocby}
% The command |\childdocby| ....
%    \begin{macrocode}
\newcommand{\childdocby}[2][]
{
  \childdocdisable
  \childdoctrue
  \childdocmanualtrue
  \if?#1?\else
    \def\jobname{#2}
  \fi
  \def\childdocjob{#2}
  \input{#2}
  \endinput
}
%    \end{macrocode}

% \macro{\childdocforward}
% The command |\childdocforward| redirects
% compilation to the main file or
% (if the optional argument is given) a child file.
% Parameters are set as if the main file
% or a child file starting with |\childdocof| was compiled.
% Then compilation is handed over to the main file:
%    \begin{macrocode}
\newcommand{\childdocforward}[2][]
{
  \begingroup
    \if?#1?
      \def\childdoctmp
      {
        \def\childdocname{#2}
        \def\childdocjob{#2}
        \def\jobname{#2}
        \input{#2}
        \endinput
      }
    \else
      \def\childdoctmp
      {
        \childdocdisable
        \def\childdocname{#2}
        \childdoctrue
        \includeonly{#2}
        \def\childdocjob{#1}
        \def\jobname{#1}
        \input{#1}
        \endinput
      }
    \fi
    \expandafter
  \endgroup
  \childdoctmp
}
%    \end{macrocode}

% \macro{\childdocforwardprefix}
% The command |\childdocforwardprefix| redirects
% compilation to the main or a child file by means of a pattern.
% The prefix |#1| in the current filename is replaced by |#2|
% and the suffix of the current filename is kept
% (it is assumed that the filename does not contain the substring `|~~~|'
% which is used as a delimiter).
% Compilation is handed over to the new file by |\childdocforward|:
%    \begin{macrocode}
\newcommand{\childdocforwardprefix}[3][]
{
  \begingroup
    \def\childdocextract #2##1~~~{\def\childdoctmp{\childdocforward[#1]{#3##1}}}
    \expandafter\childdocextract\childdocname~~~
    \expandafter
  \endgroup
  \childdoctmp
}
%    \end{macrocode}

% \macro{\childdoc}
% The deprecated macro |\childdoc| is a legacy version of |\childdocmain|:
%    \begin{macrocode}
\newcommand{\childdoc}{\childdocmain}
%    \end{macrocode}

% \macro{\childdocredirect}
% The deprecated macro |\childdocredirect| is a legacy version
% of |\childdocforward| and |\childdocforwardprefix|:
%    \begin{macrocode}
\newcommand{\childdocredirect}[2][]
{
  \begingroup
    \if?#1?
      \def\childdoctmp{\childdocforward{#2}}
    \else
      \def\childdoctmp{\childdocforwardprefix{#1}{#2}}
    \fi
    \expandafter
  \endgroup
  \childdoctmp
}
%    \end{macrocode}

%\iffalse
%</package>
%\fi
%
\endinput
|\\
|\childdocof{|\textit{main}|}|\\
\end{tabular}
\end{center}
at the top of every child file \textit{child}
which is included by |\include{|\textit{child}|}|
from within the main file
(or at least for those files to be compiled individually).
The argument \textit{main} must be the filename of the main file.

There are a couple of
considerations in setting up the main and child documents:

%%%%%%%%%%%%%%%%%%%%%%%%%%%%%%%%%%%%%%%%
\paragraph{Restrictions.}

Please note the following restrictions:
\begin{itemize}
\item
|\childdocmain| must be called with one argument \textit{main}
to ensure compatibility with earlier version of the package.
It must either be empty (|\childdocmain{}|)
or precisely match the filename of the main file in which it is specified.
See \secref{sec:detection} for further information.
\item
The filename \textit{main} must be specified without the |.tex| extension.
\item
The filename \textit{main} is case sensitive
(even in case-insensitive file systems)
due to internal string comparison.
\item
The argument \textit{main} should be fully expanded, it cannot be a macro.
\item
Subdirectories and special characters should be avoided in filenames.
\item
The command |\childdocmain{|\textit{main}|}| must be followed by a whitespace.
It should not be followed immediately by another command
or by a comment mark `|%|'.
This is because the \TeX{} parser reads the token immediately following
the argument of |\childdocmain| and puts it
at the beginning of every child section;
however, a white\-space is ignored.
\end{itemize}

%%%%%%%%%%%%%%%%%%%%%%%%%%%%%%%%%%%%%%%%
\paragraph{Content of Main File.}

It is advisable to place all content in the child files included by |\include|.
Any output contained in the main file will appear in all child documents
unless suppressed manually;
it cannot be suppressed automatically by the |\includeonly| directive
and thus should normally be avoided.
A method to include some content in the main file
by means of conditional processing is described in \secref{sec:conditional}.

%%%%%%%%%%%%%%%%%%%%%%%%%%%%%%%%%%%%%%%%
\paragraph{Page Numbering.}

When only a part of the document is compiled,
the appropriate numbering of pages
(as well as other status parameters)
is determined from the |.aux| files.
The latter contain information from previous passes.
However this information needs to propagate through
all intermediate child documents.
Therefore the page numbering in child documents may well
be inconsistent until the complete document is compiled at least once.

A useful (if unconventional) way to always ensure a consistent
page numbering is to restart the numbering in each child document
and denote the pages by `\textit{child}|.|\textit{page}'
where \textit{child} represents the chapter/section number of the child file.
This can be achieved by the command
|\numberwithin{page}{|\textit{child}|}|
of the \textsf{amsmath} package
where \textit{child} can be |chapter| or |section|
depending on the chosen structuring.
Alternatively, one can modify the macro |\thepage| appropriately
and reset the counter |page| at the start of each child file.

%%%%%%%%%%%%%%%%%%%%%%%%%%%%%%%%%%%%%%%%%%%%%%%%%%%%%%%%%%%%%%%%%%%%%%%%%%%%%%%%
\subsection{Conditional Processing}
\label{sec:conditional}

The package provides a mechanism to compile different versions
of a document. To customise the versions further some conditional processing
can come in handy to distinguish which version is being compiled.
The package provides two macros to describe the compilation context:

%%%%%%%%%%%%%%%%%%%%%%%%%%%%%%%%%%%%%%%%
\DescribeMacro{\ifchilddoc}
The conditional |\ifchilddoc| distinguishes between the compilation of
child documents and the main document:
%
\begin{center}
|\ifchilddoc |\textit{child-code}| |[|\||else |\textit{main-code}]| \||fi|
\end{center}

%%%%%%%%%%%%%%%%%%%%%%%%%%%%%%%%%%%%%%%%
\DescribeMacro{\childdocname}
\DescribeMacro{\childdocjob}
The macro |\childdocname| contains the filename (without extension)
of the main or child file being processed.
Note that |\childdocjob| will always contain the name of the main file.

%%%%%%%%%%%%%%%%%%%%%%%%%%%%%%%%%%%%%%%%
\paragraph{Title Page.}

Conditional processing can be used to include a title or banner page
in the main document when proper precautions are taken.
Importantly, the code in the main file should ensure that the page counter
(as well as other status parameters which are stored in the |.aux| files)
takes the same value after the conditional processing.
Otherwise the page numbers may take divergent values
depending on which part is compiled.

For example, a title page could be declared by:
%
\begin{center}
\begin{tabular}{l}
|\ifchilddoc\||else|\\
|\addtocounter{page}{-1}|\\
\textit{code for title page}\\
|\newpage|\\
|\||fi|
\end{tabular}
\end{center}
%
A banner page for the child documents can be generated by:
%
\begin{center}
\begin{tabular}{l}
|\ifchilddoc|\\
|\addtocounter{page}{-1}|\\
\textit{code for banner page}\\
|\newpage|\\
|\||fi|
\end{tabular}
\end{center}
%
Here one could write a message such as:
\begin{center}
|This is the part \childdocname{} of \childdocjob{}.|
\end{center}

%%%%%%%%%%%%%%%%%%%%%%%%%%%%%%%%%%%%%%%%%%%%%%%%%%%%%%%%%%%%%%%%%%%%%%%%%%%%%%%%
\subsection{Flags}
\label{sec:flags}

The package makes it easy to generate different versions
of the main or child documents.
To this end compilation flags can be defined
and assigned different default values.
They will be particularly useful in conjunction
with the forwarding mechanism described in \secref{sec:forward}.

For example, it may be useful to have a flag |\version|
which can be set to |draft| or |final|.
The document source will contain some conditional code
depending on the value of |\version|.
Suppose further, the flag should default to |final| for the main file
and to |draft| for child files
which is a natural assignment for editing the document.
This is achieved by placing the following code
in the preamble of the main document
(below the |\childdocmain| directive):
%
\begin{center}
\begin{tabular}{l}
|\ifchilddoc|\\
|\providecommand{\version}{draft}|\\
|\||else|\\
|\providecommand{\version}{final}|\\
|\||fi|
\end{tabular}
\end{center}
%
The definition by |\providecommand| makes sure
that previous definitions are not overwritten.
Further statements |\providecommand{\version}{...}|
can thus be added before the above code to override it.

For the main file, one might add a line
(between |\childdocmain| and the above block)
%
\begin{center}
|%\ifchilddoc\||else\providecommand{\version}{draft}\||fi|
\end{center}
%
which can be uncommented to produce a draft version.
Likewise one can add a line to the very top of a child file
(above the |\childdocof{|\textit{main}|}| directive)
%
\begin{center}
|%\providecommand{\version}{final}|
\end{center}
%
which can be uncommented to produce the final version of this child document.

%%%%%%%%%%%%%%%%%%%%%%%%%%%%%%%%%%%%%%%%%%%%%%%%%%%%%%%%%%%%%%%%%%%%%%%%%%%%%%%%
\subsection{Forwarding}
\label{sec:forward}

Different versions of the main or child documents
using compilation flags as described in \secref{sec:flags}
can be (permanently) stored in different files
for convenient compilation, viewing and distribution.
To this end, the package defines a command
to pass on compilation to a different file:

%%%%%%%%%%%%%%%%%%%%%%%%%%%%%%%%%%%%%%%%
\DescribeMacro{\childdocforward}
The command |\childdocforward| redirects processing to
another source file:
%
\begin{center}
\begin{tabular}{l}
|% \iffalse
%
% childdoc.dtx Copyright (C) 2017-2018 Niklas Beisert
%
% This work may be distributed and/or modified under the
% conditions of the LaTeX Project Public License, either version 1.3
% of this license or (at your option) any later version.
% The latest version of this license is in
%   http://www.latex-project.org/lppl.txt
% and version 1.3 or later is part of all distributions of LaTeX
% version 2005/12/01 or later.
%
% This work has the LPPL maintenance status `maintained'.
%
% The Current Maintainer of this work is Niklas Beisert.
%
% This work consists of the files childdoc.dtx and childdoc.ins
% and the derived files childdoc.def and cdocsamp.tex with
% cdocsch1.tex, cdocsch2.tex, cdocsdrf.tex, cdocsfn1.tex, cdocsfn2.tex.
%
%<package>\ifdefined\childdocmain\endinput\fi
%<package>\ProvidesFile{childdoc.def}[2018/12/30 v2.0 child document driver]
%<samplemain>\ProvidesFile{cdocsamp.tex}[2018/12/30 v2.0 sample for childdoc]
%<*driver>
%\ProvidesFile{childdoc.drv}[2018/12/30 v2.0 childdoc reference manual file]
\PassOptionsToClass{10pt,a4paper}{article}
\documentclass{ltxdoc}

\usepackage[margin=35mm]{geometry}
\usepackage{hyperref}
\usepackage{hyperxmp}
\usepackage[usenames]{color}

\hypersetup{colorlinks=true}
\hypersetup{pdfstartview=FitH}
\hypersetup{pdfpagemode=UseNone}
\hypersetup{pdfsource={}}
\hypersetup{pdflang={en-UK}}
\hypersetup{pdfcopyright={Copyright 2017-2018 Niklas Beisert.
  This work may be distributed and/or modified under the
  conditions of the LaTeX Project Public License, either version 1.3
  of this license or (at your option) any later version.}}
\hypersetup{pdflicenseurl={http://www.latex-project.org/lppl.txt}}
\hypersetup{pdfcontactaddress={ETH Zurich, ITP, HIT K,
  Wolfgang-Pauli-Strasse 27}}
\hypersetup{pdfcontactpostcode={8093}}
\hypersetup{pdfcontactcity={Zurich}}
\hypersetup{pdfcontactcountry={Switzerland}}
\hypersetup{pdfcontactemail={nbeisert@itp.phys.ethz.ch}}
\hypersetup{pdfcontacturl={http://people.phys.ethz.ch/\xmptilde nbeisert/}}

\newcommand{\secref}[1]{\hyperref[#1]{section \ref*{#1}}}

\parskip1ex
\parindent0pt
\let\olditemize\itemize
\def\itemize{\olditemize\parskip0pt}

\begin{document}

\title{The \textsf{childdoc} Package}
\hypersetup{pdftitle={The childdoc Package}}
\author{Niklas Beisert\\[2ex]
  Institut f\"ur Theoretische Physik\\
  Eidgen\"ossische Technische Hochschule Z\"urich\\
  Wolfgang-Pauli-Strasse 27, 8093 Z\"urich, Switzerland\\[1ex]
  \href{mailto:nbeisert@itp.phys.ethz.ch}
  {\texttt{nbeisert@itp.phys.ethz.ch}}}
\hypersetup{pdfauthor={Niklas Beisert}}
\hypersetup{pdfsubject={Manual for the LaTeX2e Package childdoc}}
\date{30 December 2018, \textsf{v2.0}}
\maketitle

\begin{abstract}\noindent
\textsf{childdoc} is a \LaTeXe{} package
that enables the direct compilation
of document sections included by |\include|
to individual files.
\end{abstract}

\begingroup
\parskip0ex
\tableofcontents
\endgroup

%%%%%%%%%%%%%%%%%%%%%%%%%%%%%%%%%%%%%%%%%%%%%%%%%%%%%%%%%%%%%%%%%%%%%%%%%%%%%%%%
%%%%%%%%%%%%%%%%%%%%%%%%%%%%%%%%%%%%%%%%%%%%%%%%%%%%%%%%%%%%%%%%%%%%%%%%%%%%%%%%
\section{Introduction}

\LaTeX{} provides a mechanism to structure a large document (such as a book)
into a main file and several child files (containing the chapters)
using the |\include| command.
This mechanism is beneficial for documents
which span hundreds of pages in order to
make the source file(s) more manageable.
Moreover, compilation can be restricted to
selected child files by means of the |\includeonly| command.
The latter feature can be used to reduce the compilation time while editing
(this was significantly more useful in the earlier days of \LaTeX{})
or to generate a smaller document which is easier to navigate.
Another application of |\includeonly| is to generate
documents consisting of selected parts of the complete document.

However, there are a few drawbacks of the plain |\include| mechanism:
\begin{itemize}
\item
The child files cannot be compiled on their own,
they can only be compiled via the main file.
A naive editing environment
(such as a text editor with an option
to have the current file processed by \LaTeX)
may require one to switch to the main file before compiling;
attempting to compile the child file produces errors.
\item
The main file must be modified (each time)
to adjust the |\includeonly| command
to the present needs. This easily leaves the main file in a messy state.
\item
The generated document will always carry the filename
of the main document. This is inconvenient if
several child files are to be compiled and
to be kept for distribution.
\end{itemize}

The present package provides a simple interface
to make child files individually compilable by \LaTeX{}.
Compiling a child file then has the same effect as compiling
the main file with an |\includeonly| command
to select the appropriate child.
Moreover the generated document will carry the name of the child
rather than the main file.
This resolves all three above issues.

This feature is meant to make the editing of books,
thesis documents and lecture notes somewhat more convenient.
However, the package can also be used efficiently for
composing a series of documents (such as exercise sheets)
which are typically distributed individually.
It then assists the author in generating the individual documents
(potentially in different versions)
as well as a document containing the collected series.
Another application is in developing style files
or other kinds of included material
where compilation of the style file could redirect
to a sample or test file.

%%%%%%%%%%%%%%%%%%%%%%%%%%%%%%%%%%%%%%%%%%%%%%%%%%%%%%%%%%%%%%%%%%%%%%%%%%%%%%%%
%%%%%%%%%%%%%%%%%%%%%%%%%%%%%%%%%%%%%%%%%%%%%%%%%%%%%%%%%%%%%%%%%%%%%%%%%%%%%%%%
\section{Usage}

First of all, the package \textsf{childdoc} is \emph{not} a standard
\LaTeXe{} |.sty| style file! Therefore it needs to be invoked in
a non-standard way.

%%%%%%%%%%%%%%%%%%%%%%%%%%%%%%%%%%%%%%%%%%%%%%%%%%%%%%%%%%%%%%%%%%%%%%%%%%%%%%%%
\subsection{Included Files}
\label{sec:include}

%%%%%%%%%%%%%%%%%%%%%%%%%%%%%%%%%%%%%%%%
\DescribeMacro{\childdocmain}
To use the package, add the commands
\begin{center}
\begin{tabular}{l}
|\input{childdoc.def}|\\
|\childdocmain{}|\\
\end{tabular}
\end{center}
at the very top of the main \LaTeX{} file,
in particular \emph{before} the |\documentclass| statement!
The argument of |\childdocmain| should be left empty
(but it must be present).

%%%%%%%%%%%%%%%%%%%%%%%%%%%%%%%%%%%%%%%%
\DescribeMacro{\childdocof}
Furthermore, add the commands
\begin{center}
\begin{tabular}{l}
|\input{childdoc.def}|\\
|\childdocof{|\textit{main}|}|\\
\end{tabular}
\end{center}
at the top of every child file \textit{child}
which is included by |\include{|\textit{child}|}|
from within the main file
(or at least for those files to be compiled individually).
The argument \textit{main} must be the filename of the main file.

There are a couple of
considerations in setting up the main and child documents:

%%%%%%%%%%%%%%%%%%%%%%%%%%%%%%%%%%%%%%%%
\paragraph{Restrictions.}

Please note the following restrictions:
\begin{itemize}
\item
|\childdocmain| must be called with one argument \textit{main}
to ensure compatibility with earlier version of the package.
It must either be empty (|\childdocmain{}|)
or precisely match the filename of the main file in which it is specified.
See \secref{sec:detection} for further information.
\item
The filename \textit{main} must be specified without the |.tex| extension.
\item
The filename \textit{main} is case sensitive
(even in case-insensitive file systems)
due to internal string comparison.
\item
The argument \textit{main} should be fully expanded, it cannot be a macro.
\item
Subdirectories and special characters should be avoided in filenames.
\item
The command |\childdocmain{|\textit{main}|}| must be followed by a whitespace.
It should not be followed immediately by another command
or by a comment mark `|%|'.
This is because the \TeX{} parser reads the token immediately following
the argument of |\childdocmain| and puts it
at the beginning of every child section;
however, a white\-space is ignored.
\end{itemize}

%%%%%%%%%%%%%%%%%%%%%%%%%%%%%%%%%%%%%%%%
\paragraph{Content of Main File.}

It is advisable to place all content in the child files included by |\include|.
Any output contained in the main file will appear in all child documents
unless suppressed manually;
it cannot be suppressed automatically by the |\includeonly| directive
and thus should normally be avoided.
A method to include some content in the main file
by means of conditional processing is described in \secref{sec:conditional}.

%%%%%%%%%%%%%%%%%%%%%%%%%%%%%%%%%%%%%%%%
\paragraph{Page Numbering.}

When only a part of the document is compiled,
the appropriate numbering of pages
(as well as other status parameters)
is determined from the |.aux| files.
The latter contain information from previous passes.
However this information needs to propagate through
all intermediate child documents.
Therefore the page numbering in child documents may well
be inconsistent until the complete document is compiled at least once.

A useful (if unconventional) way to always ensure a consistent
page numbering is to restart the numbering in each child document
and denote the pages by `\textit{child}|.|\textit{page}'
where \textit{child} represents the chapter/section number of the child file.
This can be achieved by the command
|\numberwithin{page}{|\textit{child}|}|
of the \textsf{amsmath} package
where \textit{child} can be |chapter| or |section|
depending on the chosen structuring.
Alternatively, one can modify the macro |\thepage| appropriately
and reset the counter |page| at the start of each child file.

%%%%%%%%%%%%%%%%%%%%%%%%%%%%%%%%%%%%%%%%%%%%%%%%%%%%%%%%%%%%%%%%%%%%%%%%%%%%%%%%
\subsection{Conditional Processing}
\label{sec:conditional}

The package provides a mechanism to compile different versions
of a document. To customise the versions further some conditional processing
can come in handy to distinguish which version is being compiled.
The package provides two macros to describe the compilation context:

%%%%%%%%%%%%%%%%%%%%%%%%%%%%%%%%%%%%%%%%
\DescribeMacro{\ifchilddoc}
The conditional |\ifchilddoc| distinguishes between the compilation of
child documents and the main document:
%
\begin{center}
|\ifchilddoc |\textit{child-code}| |[|\||else |\textit{main-code}]| \||fi|
\end{center}

%%%%%%%%%%%%%%%%%%%%%%%%%%%%%%%%%%%%%%%%
\DescribeMacro{\childdocname}
\DescribeMacro{\childdocjob}
The macro |\childdocname| contains the filename (without extension)
of the main or child file being processed.
Note that |\childdocjob| will always contain the name of the main file.

%%%%%%%%%%%%%%%%%%%%%%%%%%%%%%%%%%%%%%%%
\paragraph{Title Page.}

Conditional processing can be used to include a title or banner page
in the main document when proper precautions are taken.
Importantly, the code in the main file should ensure that the page counter
(as well as other status parameters which are stored in the |.aux| files)
takes the same value after the conditional processing.
Otherwise the page numbers may take divergent values
depending on which part is compiled.

For example, a title page could be declared by:
%
\begin{center}
\begin{tabular}{l}
|\ifchilddoc\||else|\\
|\addtocounter{page}{-1}|\\
\textit{code for title page}\\
|\newpage|\\
|\||fi|
\end{tabular}
\end{center}
%
A banner page for the child documents can be generated by:
%
\begin{center}
\begin{tabular}{l}
|\ifchilddoc|\\
|\addtocounter{page}{-1}|\\
\textit{code for banner page}\\
|\newpage|\\
|\||fi|
\end{tabular}
\end{center}
%
Here one could write a message such as:
\begin{center}
|This is the part \childdocname{} of \childdocjob{}.|
\end{center}

%%%%%%%%%%%%%%%%%%%%%%%%%%%%%%%%%%%%%%%%%%%%%%%%%%%%%%%%%%%%%%%%%%%%%%%%%%%%%%%%
\subsection{Flags}
\label{sec:flags}

The package makes it easy to generate different versions
of the main or child documents.
To this end compilation flags can be defined
and assigned different default values.
They will be particularly useful in conjunction
with the forwarding mechanism described in \secref{sec:forward}.

For example, it may be useful to have a flag |\version|
which can be set to |draft| or |final|.
The document source will contain some conditional code
depending on the value of |\version|.
Suppose further, the flag should default to |final| for the main file
and to |draft| for child files
which is a natural assignment for editing the document.
This is achieved by placing the following code
in the preamble of the main document
(below the |\childdocmain| directive):
%
\begin{center}
\begin{tabular}{l}
|\ifchilddoc|\\
|\providecommand{\version}{draft}|\\
|\||else|\\
|\providecommand{\version}{final}|\\
|\||fi|
\end{tabular}
\end{center}
%
The definition by |\providecommand| makes sure
that previous definitions are not overwritten.
Further statements |\providecommand{\version}{...}|
can thus be added before the above code to override it.

For the main file, one might add a line
(between |\childdocmain| and the above block)
%
\begin{center}
|%\ifchilddoc\||else\providecommand{\version}{draft}\||fi|
\end{center}
%
which can be uncommented to produce a draft version.
Likewise one can add a line to the very top of a child file
(above the |\childdocof{|\textit{main}|}| directive)
%
\begin{center}
|%\providecommand{\version}{final}|
\end{center}
%
which can be uncommented to produce the final version of this child document.

%%%%%%%%%%%%%%%%%%%%%%%%%%%%%%%%%%%%%%%%%%%%%%%%%%%%%%%%%%%%%%%%%%%%%%%%%%%%%%%%
\subsection{Forwarding}
\label{sec:forward}

Different versions of the main or child documents
using compilation flags as described in \secref{sec:flags}
can be (permanently) stored in different files
for convenient compilation, viewing and distribution.
To this end, the package defines a command
to pass on compilation to a different file:

%%%%%%%%%%%%%%%%%%%%%%%%%%%%%%%%%%%%%%%%
\DescribeMacro{\childdocforward}
The command |\childdocforward| redirects processing to
another source file:
%
\begin{center}
\begin{tabular}{l}
|\input{childdoc.def}|\\
|\childdocforward[|\textit{main}|]{|\textit{dest}|}|\\
\end{tabular}
\end{center}
%
The argument \textit{dest} is the destination file
(without extension).
It should be the main file or one of the child files.
Note that further \textsf{childdoc} directives
such as |\childdocof| and |\childdocforward|
in the indicated file will be processed in this form.
The optional argument \textit{main}
passes on directly to the main file \textit{main}
while pretending to compile the child \textit{dest}.
This form behaves as if \textit{dest}
issues |\childdocof{|\textit{main}|}| right away,
and no further \textsf{childdoc} directives will be processed.

%%%%%%%%%%%%%%%%%%%%%%%%%%%%%%%%%%%%%%%%
\DescribeMacro{\...prefix}
In the alternative form |\childdocforwardprefix|,
%
\begin{center}
\begin{tabular}{l}
|\input{childdoc.def}|\\
|\childdocforwardprefix[|\textit{main}|]{|\textit{prefix}|}{|\textit{dest}|}|
\end{tabular}
\end{center}
%
the destination file is determined by a pattern
depending on the current file:
To make this work, the current file must be called
`{\textit{prefix}\hspace{0.2em}\textit{suffix}}'
with \textit{prefix} matching precisely the argument.
Processing is then passed on to the file
`{\textit{dest}\hspace{0.2em}\textit{suffix}}'.
Surely, the same effect is achieved by
directly specifying the
argument `{\textit{dest}\hspace{0.2em}\textit{suffix}}'
in the first form.
However, that requires to set up a different file
for each child. With the alternative form of the command
all these files can have exactly the same content
which simplifies setting them up and maintaining them.

For example, the following file |draft.tex|
with a compilation flag |\version| as described in \secref{sec:flags}
compiles the main document as a draft:
%
\begin{center}
\begin{tabular}{l}
|\def\version{draft}|\\
|\input{childdoc.def}|\\
|\childdocforward{|\textit{main}|}|
\end{tabular}
\end{center}
%
Likewise, the following files |final|\textit{nn}|.tex|
compile the final version of the child document
|child|\textit{nn}|.tex|:
%
\begin{center}
\begin{tabular}{l}
|\def\version{final}|\\
|\input{childdoc.def}|\\
|\childdocforwardprefix{final}{child}|
\end{tabular}
\end{center}
%

Note that when several versions of a main file and/or of each child file
are to be generated, it may be convenient to set up a |Makefile| or
shell script to automatise the process.

%%%%%%%%%%%%%%%%%%%%%%%%%%%%%%%%%%%%%%%%%%%%%%%%%%%%%%%%%%%%%%%%%%%%%%%%%%%%%%%%
\subsection{Command Line Processing}
\label{sec:commandline}

The effect of redirection files can also be achieved by invoking
the \LaTeX{} compiler with a more elaborate command line.
Most conveniently this should be done as part
of a shell script or a |Makefile|.

When using \textsf{childdoc} in the main file, the following
command lines effectively perform a redirection
(note that depending on the shell being used,
backslashes may have to be doubled: `|\|' $\to$ `|\\|'):
%
\begin{center}
|... -jobname "|\textit{target}|" |\\|"|[\textit{flags}]%
|\input{childdoc.def}\childdocforward[|\textit{main}|]{|\textit{dest}|}"|
\end{center}
%
Here \textit{target} is the name of the output file,
\textit{main} is the name of the main file
and \textit{dest} is the name of the main or child file to be processed
(all filenames without extensions).
The optional argument \textit{main} can be omitted
if \textit{main} matches \textit{dest}.
Optionally, compilation \textit{flags} can be defined via |\def| commands.
This command line makes the \TeX{} engine believe
it is compiling the file \textit{target}
whose content is specified as the latter parameter.
The provided code then forwards the processing to
\textit{main} or \textit{dest} as described in \secref{sec:forward}.

%%%%%%%%%%%%%%%%%%%%%%%%%%%%%%%%%%%%%%%%%%%%%%%%%%%%%%%%%%%%%%%%%%%%%%%%%%%%%%%%
\subsection{Include by Input}
\label{sec:input}

Including child documents by |\include| has some restrictions by design.
Most notably, the content of a child document always occupies
its own set of pages; pages cannot be shared between child documents.
Usually, this behaviour makes perfect sense
because each child document contain an essential part of the document.
However, in some situations it may be desirable to compose
a document from a collection of parts
without having mandatory page breaks between then.
For this case, the package
provides a mechanism to include parts
by |\input| which can also be processed individually.
However, by construction this mechanism
requires manual handling of the content to be output.

%%%%%%%%%%%%%%%%%%%%%%%%%%%%%%%%%%%%%%%%
\DescribeMacro{\ifchilddocmanual}
The main file should be prepared as usual, see \secref{sec:include}.
However, the document body must make a distinction
between processing of an individual part and of the main document, e.g.:
%
\begin{center}
\begin{tabular}{l}
|\ifchilddocmanual|\\
|\input{\childdocname}|\\
|\||else|\\
\textit{document body with }|\input{|\textit{part}|}|\\
|\||fi|
\end{tabular}
\end{center}
%
The conditional |\ifchilddocmanual| is true whenever
a part to be included by |\input| is being compiled,
and the name of the part is stored in |\childdocname|.

%%%%%%%%%%%%%%%%%%%%%%%%%%%%%%%%%%%%%%%%
\DescribeMacro{\childdocby}
Each part to be included by |\input| should start with:
%
\begin{center}
\begin{tabular}{l}
|\input{childdoc.def}|\\
|\childdocby{|\textit{main}|}|\\
\end{tabular}
\end{center}
%
The directive |\childdocby| is similar to |\childdocof|
described in \secref{sec:include},
but the subsequent selection of content must be done manually.
To that end, both |\ifchilddoc| and |\ifchilddocmanual|
will be true upon processing of a part,
and the name of the part is stored in |\childdocname|.
Note that |\jobname| will be set to the filename of the current part
so that each part receives an individual |.aux| file
that does not interfere with the |.aux| file(s) of the main document.
This behaviour can be altered by the alternative form
|\childdocby[*]{|\textit{main}|}| (with a non-empty optional argument)
which uses the |.aux| file of the main document
by setting |\jobname| to \textit{main}.

%%%%%%%%%%%%%%%%%%%%%%%%%%%%%%%%%%%%%%%%%%%%%%%%%%%%%%%%%%%%%%%%%%%%%%%%%%%%%%%%
\subsection{Driver Development}
\label{sec:driver}

The \textsf{childdoc} mechanism can also be use for the development
of definition files such as \LaTeX{} styles or classes.
This case differs from the above setup with multiple parts
included by |\include| in that no |\includeonly| should be invoked.
This can be achieved by starting the include file
(before |\ProvidesPackage|) with:
%
\begin{center}
\begin{tabular}{l}
|\input{childdoc.def}|\\
|\childdocforward{|\textit{main}|}|\\
\end{tabular}
\end{center}
%
or alternatively with:
%
\begin{center}
\begin{tabular}{l}
|\input{childdoc.def}|\\
|\childdocby{|\textit{main}|}|\\
\end{tabular}
\end{center}
%
Both forms have slightly different effects as described above.
The main file is prepared as usual, see \secref{sec:include}.

%%%%%%%%%%%%%%%%%%%%%%%%%%%%%%%%%%%%%%%%%%%%%%%%%%%%%%%%%%%%%%%%%%%%%%%%%%%%%%%%
\subsection{Legacy Detection}
\label{sec:detection}

The directive |\childdocmain| in the main file can detect
whether the complete document or merely a child is to be compiled
even without using the directive |\childdocof|.
This method is deprecated because it is less robust
and there is no compelling reason to use it;
it is merely provided for backward compatibility
and it may be removed in future versions.

If the detection mechanism is to be used,
it is mandatory to correctly specify
the filename of the main file as the argument of |\childdocmain|:
%
\begin{center}
\begin{tabular}{l}
|\input{childdoc.def}|\\
|\childdocmain{|\textit{main}|}|\\
\end{tabular}
\end{center}
%
If |\jobname| does not match the argument \textit{main} of |\childdocmain|,
it is assumed that |\jobname| points to the child file to be compiled.
When using |\childdocmain| with the main file specified as argument,
it suffices to start a child file
with just |\input{|\textit{main}|}|
without loading of the package and using |\childdocof|.
If instead all processing is done
with the appropriate \textsf{childdoc} directives,
the argument of \textit{main} of |\childdocmain| can be empty.

An alternative version of the command line processing described
in \secref{sec:commandline} using the detection mechanism reads:
%
\begin{center}
|... -jobname "|\textit{target}|" "|[\textit{flags}]%
[|\def\jobname{|\textit{dest}|}|]|\input{|\textit{main}|}"|
\end{center}

%%%%%%%%%%%%%%%%%%%%%%%%%%%%%%%%%%%%%%%%%%%%%%%%%%%%%%%%%%%%%%%%%%%%%%%%%%%%%%%%
\subsection{Manual Code}
\label{sec:manual}

In case one cannot be certain whether the definitions file |childdoc.def|
is installed on the target \TeX{} distribution
and one prefers not to ship it,
it is conceivable to paste a few relevant commands into the sources.

To that end, drop all statements |\input{childdoc.def}|
and perform the replacements as outlined below.
Instead of |\childdocmain{|\textit{main}|}| add the following code
to the top of the main file:
%
\begin{center}
\begin{tabular}{l}
|\||ifdefined\childdocname\endinput\||fi\newif\ifchilddoc|\\
|\edef\childdocname{\scantokens\expandafter{\jobname\noexpand}}|\\
|\def\childdocmain{|\textit{main}|}\||ifx\childdocmain\childdocname\||else|\\
|\childdoctrue\includeonly{\childdocname}\let\jobname\childdocmain\||fi|\\
\end{tabular}
\end{center}
%
Instead of |\childdocof{|\textit{main}|}| just include the main file
at the top of each child file:
%
\begin{center}
|\input{|\textit{main}|}|
\end{center}
%
A simple redirection |\childdocforward{|\textit{dest}|}| is achieved by:
%
\begin{center}
|\def\jobname{|\textit{dest}|}\input{\jobname}|
\end{center}
%
The redirection with prefix
|\childdocforwardprefix[|\textit{prefix}|]{|\textit{dest}|}|
is accomplished by:
%
\begin{center}
\begin{tabular}{l}
|{\edef\jobname{\scantokens\expandafter{\jobname\noexpand}}|\\
|\def\redirectjob |\textit{prefix}|#1~~~{\gdef\jobname{|\textit{dest}|#1}}|\\
|\expandafter\redirectjob\jobname~~~}\input{\jobname}|
\end{tabular}
\end{center}

In an alternative approach,
child documents can be compiled by a specific command line
without additional code or specific definitions:
%
\begin{center}
|... -jobname "|\textit{target}|" "|[\textit{flags}]%
|\includeonly{|\textit{dest}|}\input{|\textit{main}|}"|
\end{center}
%

%%%%%%%%%%%%%%%%%%%%%%%%%%%%%%%%%%%%%%%%%%%%%%%%%%%%%%%%%%%%%%%%%%%%%%%%%%%%%%%%
%%%%%%%%%%%%%%%%%%%%%%%%%%%%%%%%%%%%%%%%%%%%%%%%%%%%%%%%%%%%%%%%%%%%%%%%%%%%%%%%
\section{Information}

%%%%%%%%%%%%%%%%%%%%%%%%%%%%%%%%%%%%%%%%%%%%%%%%%%%%%%%%%%%%%%%%%%%%%%%%%%%%%%%%
\subsection{Copyright}

Copyright \copyright{} 2017--2018 Niklas Beisert

This work may be distributed and/or modified under the
conditions of the \LaTeX{} Project Public License, either version 1.3
of this license or (at your option) any later version.
The latest version of this license is in
  \url{http://www.latex-project.org/lppl.txt}
and version 1.3 or later is part of all distributions of \LaTeX{}
version 2005/12/01 or later.

This work has the LPPL maintenance status `maintained'.

The Current Maintainer of this work is Niklas Beisert.

This work consists of the files |README.txt|, |childdoc.ins| and |childdoc.dtx|
as well as the derived files |childdoc.def|, |cdocsamp.tex|
with |cdocsch1.tex|, |cdocsch2.tex|, |cdocspt3.tex|, |cdocspt4.tex|,
|cdocsdrf.tex|, |cdocsfn1.tex|, |cdocsfn2.tex|
as well as |childdoc.pdf|.

%%%%%%%%%%%%%%%%%%%%%%%%%%%%%%%%%%%%%%%%%%%%%%%%%%%%%%%%%%%%%%%%%%%%%%%%%%%%%%%%
\subsection{Files and Installation}

The package consists of the files:
%
\begin{center}
\begin{tabular}{ll}
    |README.txt|   & readme file \\
    |childdoc.ins| & installation file \\
    |childdoc.dtx| & source file \\
    |childdoc.def| & definition file \\
    |cdocsamp.tex| & sample main file \\
    |cdocsch1.tex| & sample include file \\
    |cdocsch2.tex| & sample include file \\
    |cdocspt3.tex| & sample part file \\
    |cdocspt4.tex| & sample part file \\
    |cdocsdrf.tex| & sample redirection file \\
    |cdocsfn1.tex| & sample redirection file \\
    |cdocsfn2.tex| & sample redirection file \\
    |childdoc.pdf| & manual
\end{tabular}
\end{center}
%
The distribution consists of the files
|README.txt|, |childdoc.ins| and |childdoc.dtx|.
%
\begin{itemize}
\item
Run (pdf)\LaTeX{} on |childdoc.dtx|
to compile the manual |childdoc.pdf| (this file).
\item
Run \LaTeX{} on |childdoc.ins| to create the definitions file |childdoc.def|
and the sample |cdocsamp.tex| with include files
|cdocsch1.tex|, |cdocsch2.tex|, |cdocspt3.tex|, |cdocspt4.tex|,
|cdocsdrf.tex|, |cdocsfn1.tex|, |cdocsfn2.tex|.
Then copy the file |childdoc.def| to an appropriate directory of your \LaTeX{}
distribution, e.g.\ \textit{texmf-root}|/tex/latex/childdoc|.
\end{itemize}

%%%%%%%%%%%%%%%%%%%%%%%%%%%%%%%%%%%%%%%%%%%%%%%%%%%%%%%%%%%%%%%%%%%%%%%%%%%%%%%%
\subsection{Related CTAN Packages}

There are several other packages which offer a similar functionality:
%
\begin{itemize}
\item
The packages
\href{http://ctan.org/pkg/docmute}{\textsf{docmute}},
\href{http://ctan.org/pkg/includex}{\textsf{includex}} and
\href{http://ctan.org/pkg/standalone}{\textsf{standalone}}
provide commands to include only the document body of
a child file thus allowing both files to be compiled individually.
\item
The packages \href{http://ctan.org/pkg/subdocs}{\textsf{subdocs}}
and \href{http://ctan.org/pkg/subfiles}{\textsf{subfiles}}
provide structures in which the main and child documents can be
encapsulated and allowing them to be compiled individually.
The inclusion mechanism is different from the conventional |\include|.
\item
The package \href{http://ctan.org/pkg/combine}{\textsf{combine}}
is an elaborate solution to combine several documents into one.
\end{itemize}
%
See also the CTAN topic \href{http://ctan.org/topic/subdocs}{\textsf{subdocs}}
for further related packages.
The present package differs from the above solutions in that
a document structure constructed with the conventional |\include| mechanism
just needs two extra commands at the top of every file
such that all constituent files can be compiled individually.

%%%%%%%%%%%%%%%%%%%%%%%%%%%%%%%%%%%%%%%%%%%%%%%%%%%%%%%%%%%%%%%%%%%%%%%%%%%%%%%%
%\subsection{Feature Suggestions}
%
%The following is a list of features which may be useful for future
%versions of this package:
%%
%\begin{itemize}
%\item
%\ldots
%\end{itemize}

%%%%%%%%%%%%%%%%%%%%%%%%%%%%%%%%%%%%%%%%%%%%%%%%%%%%%%%%%%%%%%%%%%%%%%%%%%%%%%%%
\subsection{Revision History}

%%%%%%%%%%%%%%%%%%%%%%%%%%%%%%%%%%%%%%%%
\paragraph{v2.0:} 2018/12/30

\begin{itemize}
\item
immediate forward processing
\item
added |\childdocby| mechanism
\item
manual restructured
\end{itemize}

%%%%%%%%%%%%%%%%%%%%%%%%%%%%%%%%%%%%%%%%
\paragraph{v1.6:} 2018/01/17

\begin{itemize}
\item
application for development of include files
\item
corrections to manual
\end{itemize}

%%%%%%%%%%%%%%%%%%%%%%%%%%%%%%%%%%%%%%%%
\paragraph{v1.5:} 2017/05/21

\begin{itemize}
\item
more complete structuring introduced
\item
|\childdocof| introduced
\item
|\childdoc| renamed to |\childdocmain|
\item
|\childredirect| renamed to |\childdocforward| and |\childdocforwardprefix|
and functionality expanded
\end{itemize}

%%%%%%%%%%%%%%%%%%%%%%%%%%%%%%%%%%%%%%%%
\paragraph{v1.0:} 2017/04/27

\begin{itemize}
\item
manual and install package
\item
first version published on CTAN
\end{itemize}

%%%%%%%%%%%%%%%%%%%%%%%%%%%%%%%%%%%%%%%%
\paragraph{v0.6:} 2017/04/26

\begin{itemize}
\item
redirection mechanism added
\end{itemize}

%%%%%%%%%%%%%%%%%%%%%%%%%%%%%%%%%%%%%%%%
\paragraph{v0.5:} 2017/04/26

\begin{itemize}
\item
functionality in definition file
\end{itemize}


%%%%%%%%%%%%%%%%%%%%%%%%%%%%%%%%%%%%%%%%%%%%%%%%%%%%%%%%%%%%%%%%%%%%%%%%%%%%%%%%
%%%%%%%%%%%%%%%%%%%%%%%%%%%%%%%%%%%%%%%%%%%%%%%%%%%%%%%%%%%%%%%%%%%%%%%%%%%%%%%%
%%%%%%%%%%%%%%%%%%%%%%%%%%%%%%%%%%%%%%%%%%%%%%%%%%%%%%%%%%%%%%%%%%%%%%%%%%%%%%%%
\appendix

\settowidth\MacroIndent{\rmfamily\scriptsize 000\ }

 \DocInput{childdoc.dtx}

\end{document}
%</driver>
% \fi
%
% %%%%%%%%%%%%%%%%%%%%%%%%%%%%%%%%%%%%%%%%%%%%%%%%%%%%%%%%%%%%%%%%%%%%%%%%%%%%%%
% %%%%%%%%%%%%%%%%%%%%%%%%%%%%%%%%%%%%%%%%%%%%%%%%%%%%%%%%%%%%%%%%%%%%%%%%%%%%%%
% \section{Sample}
%\iffalse
%<*samplemain>
%\fi
%
% The following presents a sample document
% with two chapters, two parts, a title page,
% a compile flag as well as three forwarding files to set the flag.
% It consists of eight |.tex| files:
% \begin{center}
% \begin{tabular}{ll}
% |cdocsamp.tex|&main file\\
% |cdocsch1.tex|&include file for chapter 1\\
% |cdocsch2.tex|&include file for chapter 2\\
% |cdocspt3.tex|&include file for part 3\\
% |cdocspt4.tex|&include file for part 4\\
% |cdocsdrf.tex|&forwarding file for main file in draft mode\\
% |cdocsfi1.tex|&forwarding file for final version of chapter 1\\
% |cdocsfi2.tex|&forwarding file for final version of chapter 2\\
% \end{tabular}
% \end{center}
% Each of the eight files can be compiled directly by the \LaTeX{} compiler.
%
% %%%%%%%%%%%%%%%%%%%%%%%%%%%%%%%%%%%%%%
% \paragraph{Main File.}
%
% The main file is called |cdocsamp.tex|.
%
% Load the \textsf{childdoc} definitions and
% declare the filename for the main document:
%    \begin{macrocode}
\input{childdoc.def}
\childdocmain{}
%    \end{macrocode}

% Optional override for |\version| flag:
%    \begin{macrocode}
%%\ifchilddoc\else\providecommand{\version}{draft}\fi
%    \end{macrocode}

% Define the default values for the |\version| flag
% (|final| for the main file and |draft| for childs):
%    \begin{macrocode}
\ifchilddoc
\providecommand{\version}{draft}
\else
\providecommand{\version}{final}
\fi
%    \end{macrocode}

% Load the standard document class:
%    \begin{macrocode}
\documentclass[12pt]{article}
%    \end{macrocode}

% Start the document body:
%    \begin{macrocode}
\begin{document}
%    \end{macrocode}

% Declare a title page.
% Print title, part of document being processed and version flag:
%    \begin{macrocode}
\addtocounter{page}{-1}
\begin{center}
{\LARGE\bfseries{}childdoc example\par}
\vspace{1cm}
\ifchilddoc
\ifchilddocmanual part\else chapter\fi:
`\childdocname' of `\childdocjob'\par
\else
main document: `\childdocjob'\par
\fi
version: \version\par
\end{center}
\newpage
%    \end{macrocode}

% Manually include selected file,
% otherwise process as usual:
%    \begin{macrocode}
\ifchilddocmanual
\section*{part `\childdocname'}
\input{\childdocname}
\else
%    \end{macrocode}

% Include the two chapters:
%    \begin{macrocode}
\include{cdocsch1}
\include{cdocsch2}
%    \end{macrocode}

% Include the two parts unless only chapters should be displayed:
%    \begin{macrocode}
\ifchilddoc\else
\section{part three}
\input{cdocspt3}
\section{part four}
\input{cdocspt4}
\fi
%    \end{macrocode}

% Process as usual until here:
%    \begin{macrocode}
\fi
%    \end{macrocode}

% End of document body:
%    \begin{macrocode}
\end{document}
%    \end{macrocode}
%\iffalse
%</samplemain>
%\fi
%
% %%%%%%%%%%%%%%%%%%%%%%%%%%%%%%%%%%%%%%
% \paragraph{Chapter Include Files.}
%
% The include files are called |cdocsch1.tex| and |cdocsch2.tex|.
%
%\iffalse
%<*samplechap1|samplechap2>
%\fi

% Optional override for |\version| flag:
%    \begin{macrocode}
%%\providecommand{\version}{final}
%    \end{macrocode}

% Include the main document:
%    \begin{macrocode}
\input{childdoc.def}
\childdocof{cdocsamp}
%    \end{macrocode}

%\iffalse
%</samplechap1|samplechap2>
%\fi
%
%\iffalse
%<*samplechap1>
%\fi
% Some text for chapter 1:
%    \begin{macrocode}
\section{one}
some text in chapter one
%    \end{macrocode}

%\iffalse
%</samplechap1>
%\fi
% Some text for chapter 2:
%\iffalse
%<*samplechap2>
%\fi
%    \begin{macrocode}
\section{two}
more text in chapter two
%    \end{macrocode}

%\iffalse
%</samplechap2>
%\fi
%
% %%%%%%%%%%%%%%%%%%%%%%%%%%%%%%%%%%%%%%
% \paragraph{Part Include Files.}
%
% The include files are called |cdocspt3.tex| and |cdocspt4.tex|.
%
%\iffalse
%<*samplepart3|samplepart4>
%\fi

% Optional override for |\version| flag:
%    \begin{macrocode}
%%\providecommand{\version}{final}
%    \end{macrocode}

% Include the main document:
%    \begin{macrocode}
\input{childdoc.def}
\childdocby{cdocsamp}
%    \end{macrocode}

%\iffalse
%</samplepart3|samplepart4>
%\fi
%
%\iffalse
%<*samplepart3>
%\fi
% Some text for part 3:
%    \begin{macrocode}
some text in part three
%    \end{macrocode}

%\iffalse
%</samplepart3>
%\fi
% Some text for part 4:
%\iffalse
%<*samplepart4>
%\fi
%    \begin{macrocode}
more text in part four
%    \end{macrocode}

%\iffalse
%</samplepart4>
%\fi
%
% %%%%%%%%%%%%%%%%%%%%%%%%%%%%%%%%%%%%%%
% \paragraph{Forwarding for a Complete Draft.}
%
% The following forwarding file |cdocsdrf.tex|
% compiles the main document in draft mode:
%\iffalse
%<*sampledraft>
%\fi
%    \begin{macrocode}
\def\version{draft}
\input{childdoc.def}
\childdocforward{cdocsamp}
%    \end{macrocode}

%\iffalse
%</sampledraft>
%\fi
%
% %%%%%%%%%%%%%%%%%%%%%%%%%%%%%%%%%%%%%%
% \paragraph{Forwarding for Final Version of the Chapters.}
%
% The following forwarding files |cdocsfn1.tex| and |cdocsfn2.tex|
% (with identical content)
% compile the final versions of the child documents
% |cdocsch1.tex| and |cdocsch2.tex|, respectively:
%\iffalse
%<*samplefinal>
%\fi
%    \begin{macrocode}
\def\version{final}
\input{childdoc.def}
\childdocforwardprefix[cdocsamp]{cdocsfn}{cdocsch}
%    \end{macrocode}

%\iffalse
%</samplefinal>
%\fi
%
% %%%%%%%%%%%%%%%%%%%%%%%%%%%%%%%%%%%%%%
% \paragraph{Command Line Processing.}
%
% The following three command lines generate the output files
% |cdocscld|, |cdocscl1| and |cdocscl2|
% which should be identical to
% |cdocsdrf|, |cdocsch1| and |cdocsfn2|, respectively:
% \begin{center}
% \begin{tabular}{l}
% |latex -jobname cdocscld \|\\
% |  "\def\version{draft}\input{childdoc.def}\childdocforward{cdocsamp}"|\\
% |latex -jobname cdocscl1 \|\\
% |  "\input{childdoc.def}\childdocforward[cdocsamp]{cdocsch1}"|\\
% |latex -jobname cdocscl2 \|\\
% |  "\def\version{final}\input{childdoc.def}\childdocforward{cdocsch2}"|
% \end{tabular}
% \end{center}
% Note that the trailing backslash on each first line
% merely continues the input to the second line
% (for convenient cut ant paste).
% Furthermore, the command |latex| can be replaced by any
% of its alternative versions such as |pdflatex|.
%
% %%%%%%%%%%%%%%%%%%%%%%%%%%%%%%%%%%%%%%%%%%%%%%%%%%%%%%%%%%%%%%%%%%%%%%%%%%%%%%
% %%%%%%%%%%%%%%%%%%%%%%%%%%%%%%%%%%%%%%%%%%%%%%%%%%%%%%%%%%%%%%%%%%%%%%%%%%%%%%
% \section{Implementation}
%\iffalse
%<*package>
%\fi
%
% This section describes the definitions file |childdoc.def|.

% The definitions cannot be loaded using |\usepackage| or |\RequirePackage|
% which has a mechanism to prevent loading a style file more than once.
% When loading the definitions by means of |\input|
% multiple instances have to be prevented manually:
%\iffalse
%This code needs to be before the `\ProvidesFile' directive
%which is defined at the beginning of this file.
%Therefore it is also placed there and commented out here.
%</package>
%<*discard>
%\fi
%    \begin{macrocode}
\ifdefined\childdocmain\endinput\fi
%    \end{macrocode}
%\iffalse
%</discard>
%<*package>
%\fi
%
% \macro{\ifchilddoc}
% \macro{\ifchilddocmanual}
% The conditional |\ifchilddoc| tells whether a
% child (true) or main (false) document is being compiled.
% The conditional |\ifchilddocmanual| tells whether
% the |\includeonly| mechanism is used (false) or
% the selection of child files must be performed manually (true).
% The definitions initialise to false:
%    \begin{macrocode}
\newif\ifchilddoc
\newif\ifchilddocmanual
%    \end{macrocode}

% \macro{\childdocname}
% \macro{\childdocjob}
% The macro |\childdocname| stores the name of the main document
% to be compiled. The macro |\childdocjob| stores the name of
% the document on which the \LaTeX{} compiler was originally invoked.
% The content of |\jobname| cannot be compared
% to filenames specified in the source due to different catcodes.
% The following code rescans |\jobname|, stores the result
% in |\childdocname| and saves a copy in |\childdocjob|:
%    \begin{macrocode}
\edef\childdocname{\scantokens\expandafter{\jobname\noexpand}}
\let\childdocjob\childdocname
%    \end{macrocode}

% \macro{\childdocdisable}
% The macro |\childdocdisable| prevents the main file
% from being processed more than once.
% At this stage, the main document command |\childdocmain|
% is assumed to be called once again where it should do nothing.
% Any subsequent call to it should prevent
% a secondary processing of the main document
% It overwrites the forwarding commands
% |\childdocof| and |\childdocforward|
% with empty macros to prevent further inclusions of the main document:
%    \begin{macrocode}
\newcommand{\childdocdisable}
{
  \renewcommand{\childdocmain}[1]{\renewcommand{\childdocmain}[1]{\endinput}}
  \renewcommand{\childdocof}[1]{}
  \renewcommand{\childdocby}[2][]{}
  \renewcommand{\childdocforward}[2][]{}
  \renewcommand{\childdocdisable}{}
}
%    \end{macrocode}

% \macro{\childdocmain}
% The macro |\childdocmain| is to be called at the top of the main file
% with nothing or the main filename (without extension) as argument.
% First, it breaks loops.
% If the argument is not empty and does not match |\childdocname|
% (which is set by the first inclusion of |childdoc.def|),
% |\ifchilddoc| is set to true, |\includeonly| is applied to the child file
% and |\jobname| is set to the main file
% (for proper handling of |.aux| files):
%    \begin{macrocode}
\newcommand{\childdocmain}[1]
{
  \childdocdisable\childdocmain{}
  \if?#1?\else
    \begingroup
      \def\childdoctmp{#1}
      \ifx\childdoctmp\childdocname
        \def\childdoctmp{}
      \else
        \def\childdoctmp
        {
          \childdoctrue
          \includeonly{\childdocname}
          \def\childdocjob{#1}
          \def\jobname{#1}
        }
      \fi
      \expandafter
    \endgroup
    \childdoctmp
  \fi
}
%    \end{macrocode}

% \macro{\childdocof}
% The command |\childdocof| redirects
% compilation to the main file |#1|.
%    \begin{macrocode}
\newcommand{\childdocof}[1]
{
  \childdocdisable
  \childdoctrue
  \includeonly{\childdocname}
  \def\jobname{#1}
  \def\childdocjob{#1}
  \input{#1}
}
%    \end{macrocode}

% \macro{\childdocby}
% The command |\childdocby| ....
%    \begin{macrocode}
\newcommand{\childdocby}[2][]
{
  \childdocdisable
  \childdoctrue
  \childdocmanualtrue
  \if?#1?\else
    \def\jobname{#2}
  \fi
  \def\childdocjob{#2}
  \input{#2}
  \endinput
}
%    \end{macrocode}

% \macro{\childdocforward}
% The command |\childdocforward| redirects
% compilation to the main file or
% (if the optional argument is given) a child file.
% Parameters are set as if the main file
% or a child file starting with |\childdocof| was compiled.
% Then compilation is handed over to the main file:
%    \begin{macrocode}
\newcommand{\childdocforward}[2][]
{
  \begingroup
    \if?#1?
      \def\childdoctmp
      {
        \def\childdocname{#2}
        \def\childdocjob{#2}
        \def\jobname{#2}
        \input{#2}
        \endinput
      }
    \else
      \def\childdoctmp
      {
        \childdocdisable
        \def\childdocname{#2}
        \childdoctrue
        \includeonly{#2}
        \def\childdocjob{#1}
        \def\jobname{#1}
        \input{#1}
        \endinput
      }
    \fi
    \expandafter
  \endgroup
  \childdoctmp
}
%    \end{macrocode}

% \macro{\childdocforwardprefix}
% The command |\childdocforwardprefix| redirects
% compilation to the main or a child file by means of a pattern.
% The prefix |#1| in the current filename is replaced by |#2|
% and the suffix of the current filename is kept
% (it is assumed that the filename does not contain the substring `|~~~|'
% which is used as a delimiter).
% Compilation is handed over to the new file by |\childdocforward|:
%    \begin{macrocode}
\newcommand{\childdocforwardprefix}[3][]
{
  \begingroup
    \def\childdocextract #2##1~~~{\def\childdoctmp{\childdocforward[#1]{#3##1}}}
    \expandafter\childdocextract\childdocname~~~
    \expandafter
  \endgroup
  \childdoctmp
}
%    \end{macrocode}

% \macro{\childdoc}
% The deprecated macro |\childdoc| is a legacy version of |\childdocmain|:
%    \begin{macrocode}
\newcommand{\childdoc}{\childdocmain}
%    \end{macrocode}

% \macro{\childdocredirect}
% The deprecated macro |\childdocredirect| is a legacy version
% of |\childdocforward| and |\childdocforwardprefix|:
%    \begin{macrocode}
\newcommand{\childdocredirect}[2][]
{
  \begingroup
    \if?#1?
      \def\childdoctmp{\childdocforward{#2}}
    \else
      \def\childdoctmp{\childdocforwardprefix{#1}{#2}}
    \fi
    \expandafter
  \endgroup
  \childdoctmp
}
%    \end{macrocode}

%\iffalse
%</package>
%\fi
%
\endinput
|\\
|\childdocforward[|\textit{main}|]{|\textit{dest}|}|\\
\end{tabular}
\end{center}
%
The argument \textit{dest} is the destination file
(without extension).
It should be the main file or one of the child files.
Note that further \textsf{childdoc} directives
such as |\childdocof| and |\childdocforward|
in the indicated file will be processed in this form.
The optional argument \textit{main}
passes on directly to the main file \textit{main}
while pretending to compile the child \textit{dest}.
This form behaves as if \textit{dest}
issues |\childdocof{|\textit{main}|}| right away,
and no further \textsf{childdoc} directives will be processed.

%%%%%%%%%%%%%%%%%%%%%%%%%%%%%%%%%%%%%%%%
\DescribeMacro{\...prefix}
In the alternative form |\childdocforwardprefix|,
%
\begin{center}
\begin{tabular}{l}
|% \iffalse
%
% childdoc.dtx Copyright (C) 2017-2018 Niklas Beisert
%
% This work may be distributed and/or modified under the
% conditions of the LaTeX Project Public License, either version 1.3
% of this license or (at your option) any later version.
% The latest version of this license is in
%   http://www.latex-project.org/lppl.txt
% and version 1.3 or later is part of all distributions of LaTeX
% version 2005/12/01 or later.
%
% This work has the LPPL maintenance status `maintained'.
%
% The Current Maintainer of this work is Niklas Beisert.
%
% This work consists of the files childdoc.dtx and childdoc.ins
% and the derived files childdoc.def and cdocsamp.tex with
% cdocsch1.tex, cdocsch2.tex, cdocsdrf.tex, cdocsfn1.tex, cdocsfn2.tex.
%
%<package>\ifdefined\childdocmain\endinput\fi
%<package>\ProvidesFile{childdoc.def}[2018/12/30 v2.0 child document driver]
%<samplemain>\ProvidesFile{cdocsamp.tex}[2018/12/30 v2.0 sample for childdoc]
%<*driver>
%\ProvidesFile{childdoc.drv}[2018/12/30 v2.0 childdoc reference manual file]
\PassOptionsToClass{10pt,a4paper}{article}
\documentclass{ltxdoc}

\usepackage[margin=35mm]{geometry}
\usepackage{hyperref}
\usepackage{hyperxmp}
\usepackage[usenames]{color}

\hypersetup{colorlinks=true}
\hypersetup{pdfstartview=FitH}
\hypersetup{pdfpagemode=UseNone}
\hypersetup{pdfsource={}}
\hypersetup{pdflang={en-UK}}
\hypersetup{pdfcopyright={Copyright 2017-2018 Niklas Beisert.
  This work may be distributed and/or modified under the
  conditions of the LaTeX Project Public License, either version 1.3
  of this license or (at your option) any later version.}}
\hypersetup{pdflicenseurl={http://www.latex-project.org/lppl.txt}}
\hypersetup{pdfcontactaddress={ETH Zurich, ITP, HIT K,
  Wolfgang-Pauli-Strasse 27}}
\hypersetup{pdfcontactpostcode={8093}}
\hypersetup{pdfcontactcity={Zurich}}
\hypersetup{pdfcontactcountry={Switzerland}}
\hypersetup{pdfcontactemail={nbeisert@itp.phys.ethz.ch}}
\hypersetup{pdfcontacturl={http://people.phys.ethz.ch/\xmptilde nbeisert/}}

\newcommand{\secref}[1]{\hyperref[#1]{section \ref*{#1}}}

\parskip1ex
\parindent0pt
\let\olditemize\itemize
\def\itemize{\olditemize\parskip0pt}

\begin{document}

\title{The \textsf{childdoc} Package}
\hypersetup{pdftitle={The childdoc Package}}
\author{Niklas Beisert\\[2ex]
  Institut f\"ur Theoretische Physik\\
  Eidgen\"ossische Technische Hochschule Z\"urich\\
  Wolfgang-Pauli-Strasse 27, 8093 Z\"urich, Switzerland\\[1ex]
  \href{mailto:nbeisert@itp.phys.ethz.ch}
  {\texttt{nbeisert@itp.phys.ethz.ch}}}
\hypersetup{pdfauthor={Niklas Beisert}}
\hypersetup{pdfsubject={Manual for the LaTeX2e Package childdoc}}
\date{30 December 2018, \textsf{v2.0}}
\maketitle

\begin{abstract}\noindent
\textsf{childdoc} is a \LaTeXe{} package
that enables the direct compilation
of document sections included by |\include|
to individual files.
\end{abstract}

\begingroup
\parskip0ex
\tableofcontents
\endgroup

%%%%%%%%%%%%%%%%%%%%%%%%%%%%%%%%%%%%%%%%%%%%%%%%%%%%%%%%%%%%%%%%%%%%%%%%%%%%%%%%
%%%%%%%%%%%%%%%%%%%%%%%%%%%%%%%%%%%%%%%%%%%%%%%%%%%%%%%%%%%%%%%%%%%%%%%%%%%%%%%%
\section{Introduction}

\LaTeX{} provides a mechanism to structure a large document (such as a book)
into a main file and several child files (containing the chapters)
using the |\include| command.
This mechanism is beneficial for documents
which span hundreds of pages in order to
make the source file(s) more manageable.
Moreover, compilation can be restricted to
selected child files by means of the |\includeonly| command.
The latter feature can be used to reduce the compilation time while editing
(this was significantly more useful in the earlier days of \LaTeX{})
or to generate a smaller document which is easier to navigate.
Another application of |\includeonly| is to generate
documents consisting of selected parts of the complete document.

However, there are a few drawbacks of the plain |\include| mechanism:
\begin{itemize}
\item
The child files cannot be compiled on their own,
they can only be compiled via the main file.
A naive editing environment
(such as a text editor with an option
to have the current file processed by \LaTeX)
may require one to switch to the main file before compiling;
attempting to compile the child file produces errors.
\item
The main file must be modified (each time)
to adjust the |\includeonly| command
to the present needs. This easily leaves the main file in a messy state.
\item
The generated document will always carry the filename
of the main document. This is inconvenient if
several child files are to be compiled and
to be kept for distribution.
\end{itemize}

The present package provides a simple interface
to make child files individually compilable by \LaTeX{}.
Compiling a child file then has the same effect as compiling
the main file with an |\includeonly| command
to select the appropriate child.
Moreover the generated document will carry the name of the child
rather than the main file.
This resolves all three above issues.

This feature is meant to make the editing of books,
thesis documents and lecture notes somewhat more convenient.
However, the package can also be used efficiently for
composing a series of documents (such as exercise sheets)
which are typically distributed individually.
It then assists the author in generating the individual documents
(potentially in different versions)
as well as a document containing the collected series.
Another application is in developing style files
or other kinds of included material
where compilation of the style file could redirect
to a sample or test file.

%%%%%%%%%%%%%%%%%%%%%%%%%%%%%%%%%%%%%%%%%%%%%%%%%%%%%%%%%%%%%%%%%%%%%%%%%%%%%%%%
%%%%%%%%%%%%%%%%%%%%%%%%%%%%%%%%%%%%%%%%%%%%%%%%%%%%%%%%%%%%%%%%%%%%%%%%%%%%%%%%
\section{Usage}

First of all, the package \textsf{childdoc} is \emph{not} a standard
\LaTeXe{} |.sty| style file! Therefore it needs to be invoked in
a non-standard way.

%%%%%%%%%%%%%%%%%%%%%%%%%%%%%%%%%%%%%%%%%%%%%%%%%%%%%%%%%%%%%%%%%%%%%%%%%%%%%%%%
\subsection{Included Files}
\label{sec:include}

%%%%%%%%%%%%%%%%%%%%%%%%%%%%%%%%%%%%%%%%
\DescribeMacro{\childdocmain}
To use the package, add the commands
\begin{center}
\begin{tabular}{l}
|\input{childdoc.def}|\\
|\childdocmain{}|\\
\end{tabular}
\end{center}
at the very top of the main \LaTeX{} file,
in particular \emph{before} the |\documentclass| statement!
The argument of |\childdocmain| should be left empty
(but it must be present).

%%%%%%%%%%%%%%%%%%%%%%%%%%%%%%%%%%%%%%%%
\DescribeMacro{\childdocof}
Furthermore, add the commands
\begin{center}
\begin{tabular}{l}
|\input{childdoc.def}|\\
|\childdocof{|\textit{main}|}|\\
\end{tabular}
\end{center}
at the top of every child file \textit{child}
which is included by |\include{|\textit{child}|}|
from within the main file
(or at least for those files to be compiled individually).
The argument \textit{main} must be the filename of the main file.

There are a couple of
considerations in setting up the main and child documents:

%%%%%%%%%%%%%%%%%%%%%%%%%%%%%%%%%%%%%%%%
\paragraph{Restrictions.}

Please note the following restrictions:
\begin{itemize}
\item
|\childdocmain| must be called with one argument \textit{main}
to ensure compatibility with earlier version of the package.
It must either be empty (|\childdocmain{}|)
or precisely match the filename of the main file in which it is specified.
See \secref{sec:detection} for further information.
\item
The filename \textit{main} must be specified without the |.tex| extension.
\item
The filename \textit{main} is case sensitive
(even in case-insensitive file systems)
due to internal string comparison.
\item
The argument \textit{main} should be fully expanded, it cannot be a macro.
\item
Subdirectories and special characters should be avoided in filenames.
\item
The command |\childdocmain{|\textit{main}|}| must be followed by a whitespace.
It should not be followed immediately by another command
or by a comment mark `|%|'.
This is because the \TeX{} parser reads the token immediately following
the argument of |\childdocmain| and puts it
at the beginning of every child section;
however, a white\-space is ignored.
\end{itemize}

%%%%%%%%%%%%%%%%%%%%%%%%%%%%%%%%%%%%%%%%
\paragraph{Content of Main File.}

It is advisable to place all content in the child files included by |\include|.
Any output contained in the main file will appear in all child documents
unless suppressed manually;
it cannot be suppressed automatically by the |\includeonly| directive
and thus should normally be avoided.
A method to include some content in the main file
by means of conditional processing is described in \secref{sec:conditional}.

%%%%%%%%%%%%%%%%%%%%%%%%%%%%%%%%%%%%%%%%
\paragraph{Page Numbering.}

When only a part of the document is compiled,
the appropriate numbering of pages
(as well as other status parameters)
is determined from the |.aux| files.
The latter contain information from previous passes.
However this information needs to propagate through
all intermediate child documents.
Therefore the page numbering in child documents may well
be inconsistent until the complete document is compiled at least once.

A useful (if unconventional) way to always ensure a consistent
page numbering is to restart the numbering in each child document
and denote the pages by `\textit{child}|.|\textit{page}'
where \textit{child} represents the chapter/section number of the child file.
This can be achieved by the command
|\numberwithin{page}{|\textit{child}|}|
of the \textsf{amsmath} package
where \textit{child} can be |chapter| or |section|
depending on the chosen structuring.
Alternatively, one can modify the macro |\thepage| appropriately
and reset the counter |page| at the start of each child file.

%%%%%%%%%%%%%%%%%%%%%%%%%%%%%%%%%%%%%%%%%%%%%%%%%%%%%%%%%%%%%%%%%%%%%%%%%%%%%%%%
\subsection{Conditional Processing}
\label{sec:conditional}

The package provides a mechanism to compile different versions
of a document. To customise the versions further some conditional processing
can come in handy to distinguish which version is being compiled.
The package provides two macros to describe the compilation context:

%%%%%%%%%%%%%%%%%%%%%%%%%%%%%%%%%%%%%%%%
\DescribeMacro{\ifchilddoc}
The conditional |\ifchilddoc| distinguishes between the compilation of
child documents and the main document:
%
\begin{center}
|\ifchilddoc |\textit{child-code}| |[|\||else |\textit{main-code}]| \||fi|
\end{center}

%%%%%%%%%%%%%%%%%%%%%%%%%%%%%%%%%%%%%%%%
\DescribeMacro{\childdocname}
\DescribeMacro{\childdocjob}
The macro |\childdocname| contains the filename (without extension)
of the main or child file being processed.
Note that |\childdocjob| will always contain the name of the main file.

%%%%%%%%%%%%%%%%%%%%%%%%%%%%%%%%%%%%%%%%
\paragraph{Title Page.}

Conditional processing can be used to include a title or banner page
in the main document when proper precautions are taken.
Importantly, the code in the main file should ensure that the page counter
(as well as other status parameters which are stored in the |.aux| files)
takes the same value after the conditional processing.
Otherwise the page numbers may take divergent values
depending on which part is compiled.

For example, a title page could be declared by:
%
\begin{center}
\begin{tabular}{l}
|\ifchilddoc\||else|\\
|\addtocounter{page}{-1}|\\
\textit{code for title page}\\
|\newpage|\\
|\||fi|
\end{tabular}
\end{center}
%
A banner page for the child documents can be generated by:
%
\begin{center}
\begin{tabular}{l}
|\ifchilddoc|\\
|\addtocounter{page}{-1}|\\
\textit{code for banner page}\\
|\newpage|\\
|\||fi|
\end{tabular}
\end{center}
%
Here one could write a message such as:
\begin{center}
|This is the part \childdocname{} of \childdocjob{}.|
\end{center}

%%%%%%%%%%%%%%%%%%%%%%%%%%%%%%%%%%%%%%%%%%%%%%%%%%%%%%%%%%%%%%%%%%%%%%%%%%%%%%%%
\subsection{Flags}
\label{sec:flags}

The package makes it easy to generate different versions
of the main or child documents.
To this end compilation flags can be defined
and assigned different default values.
They will be particularly useful in conjunction
with the forwarding mechanism described in \secref{sec:forward}.

For example, it may be useful to have a flag |\version|
which can be set to |draft| or |final|.
The document source will contain some conditional code
depending on the value of |\version|.
Suppose further, the flag should default to |final| for the main file
and to |draft| for child files
which is a natural assignment for editing the document.
This is achieved by placing the following code
in the preamble of the main document
(below the |\childdocmain| directive):
%
\begin{center}
\begin{tabular}{l}
|\ifchilddoc|\\
|\providecommand{\version}{draft}|\\
|\||else|\\
|\providecommand{\version}{final}|\\
|\||fi|
\end{tabular}
\end{center}
%
The definition by |\providecommand| makes sure
that previous definitions are not overwritten.
Further statements |\providecommand{\version}{...}|
can thus be added before the above code to override it.

For the main file, one might add a line
(between |\childdocmain| and the above block)
%
\begin{center}
|%\ifchilddoc\||else\providecommand{\version}{draft}\||fi|
\end{center}
%
which can be uncommented to produce a draft version.
Likewise one can add a line to the very top of a child file
(above the |\childdocof{|\textit{main}|}| directive)
%
\begin{center}
|%\providecommand{\version}{final}|
\end{center}
%
which can be uncommented to produce the final version of this child document.

%%%%%%%%%%%%%%%%%%%%%%%%%%%%%%%%%%%%%%%%%%%%%%%%%%%%%%%%%%%%%%%%%%%%%%%%%%%%%%%%
\subsection{Forwarding}
\label{sec:forward}

Different versions of the main or child documents
using compilation flags as described in \secref{sec:flags}
can be (permanently) stored in different files
for convenient compilation, viewing and distribution.
To this end, the package defines a command
to pass on compilation to a different file:

%%%%%%%%%%%%%%%%%%%%%%%%%%%%%%%%%%%%%%%%
\DescribeMacro{\childdocforward}
The command |\childdocforward| redirects processing to
another source file:
%
\begin{center}
\begin{tabular}{l}
|\input{childdoc.def}|\\
|\childdocforward[|\textit{main}|]{|\textit{dest}|}|\\
\end{tabular}
\end{center}
%
The argument \textit{dest} is the destination file
(without extension).
It should be the main file or one of the child files.
Note that further \textsf{childdoc} directives
such as |\childdocof| and |\childdocforward|
in the indicated file will be processed in this form.
The optional argument \textit{main}
passes on directly to the main file \textit{main}
while pretending to compile the child \textit{dest}.
This form behaves as if \textit{dest}
issues |\childdocof{|\textit{main}|}| right away,
and no further \textsf{childdoc} directives will be processed.

%%%%%%%%%%%%%%%%%%%%%%%%%%%%%%%%%%%%%%%%
\DescribeMacro{\...prefix}
In the alternative form |\childdocforwardprefix|,
%
\begin{center}
\begin{tabular}{l}
|\input{childdoc.def}|\\
|\childdocforwardprefix[|\textit{main}|]{|\textit{prefix}|}{|\textit{dest}|}|
\end{tabular}
\end{center}
%
the destination file is determined by a pattern
depending on the current file:
To make this work, the current file must be called
`{\textit{prefix}\hspace{0.2em}\textit{suffix}}'
with \textit{prefix} matching precisely the argument.
Processing is then passed on to the file
`{\textit{dest}\hspace{0.2em}\textit{suffix}}'.
Surely, the same effect is achieved by
directly specifying the
argument `{\textit{dest}\hspace{0.2em}\textit{suffix}}'
in the first form.
However, that requires to set up a different file
for each child. With the alternative form of the command
all these files can have exactly the same content
which simplifies setting them up and maintaining them.

For example, the following file |draft.tex|
with a compilation flag |\version| as described in \secref{sec:flags}
compiles the main document as a draft:
%
\begin{center}
\begin{tabular}{l}
|\def\version{draft}|\\
|\input{childdoc.def}|\\
|\childdocforward{|\textit{main}|}|
\end{tabular}
\end{center}
%
Likewise, the following files |final|\textit{nn}|.tex|
compile the final version of the child document
|child|\textit{nn}|.tex|:
%
\begin{center}
\begin{tabular}{l}
|\def\version{final}|\\
|\input{childdoc.def}|\\
|\childdocforwardprefix{final}{child}|
\end{tabular}
\end{center}
%

Note that when several versions of a main file and/or of each child file
are to be generated, it may be convenient to set up a |Makefile| or
shell script to automatise the process.

%%%%%%%%%%%%%%%%%%%%%%%%%%%%%%%%%%%%%%%%%%%%%%%%%%%%%%%%%%%%%%%%%%%%%%%%%%%%%%%%
\subsection{Command Line Processing}
\label{sec:commandline}

The effect of redirection files can also be achieved by invoking
the \LaTeX{} compiler with a more elaborate command line.
Most conveniently this should be done as part
of a shell script or a |Makefile|.

When using \textsf{childdoc} in the main file, the following
command lines effectively perform a redirection
(note that depending on the shell being used,
backslashes may have to be doubled: `|\|' $\to$ `|\\|'):
%
\begin{center}
|... -jobname "|\textit{target}|" |\\|"|[\textit{flags}]%
|\input{childdoc.def}\childdocforward[|\textit{main}|]{|\textit{dest}|}"|
\end{center}
%
Here \textit{target} is the name of the output file,
\textit{main} is the name of the main file
and \textit{dest} is the name of the main or child file to be processed
(all filenames without extensions).
The optional argument \textit{main} can be omitted
if \textit{main} matches \textit{dest}.
Optionally, compilation \textit{flags} can be defined via |\def| commands.
This command line makes the \TeX{} engine believe
it is compiling the file \textit{target}
whose content is specified as the latter parameter.
The provided code then forwards the processing to
\textit{main} or \textit{dest} as described in \secref{sec:forward}.

%%%%%%%%%%%%%%%%%%%%%%%%%%%%%%%%%%%%%%%%%%%%%%%%%%%%%%%%%%%%%%%%%%%%%%%%%%%%%%%%
\subsection{Include by Input}
\label{sec:input}

Including child documents by |\include| has some restrictions by design.
Most notably, the content of a child document always occupies
its own set of pages; pages cannot be shared between child documents.
Usually, this behaviour makes perfect sense
because each child document contain an essential part of the document.
However, in some situations it may be desirable to compose
a document from a collection of parts
without having mandatory page breaks between then.
For this case, the package
provides a mechanism to include parts
by |\input| which can also be processed individually.
However, by construction this mechanism
requires manual handling of the content to be output.

%%%%%%%%%%%%%%%%%%%%%%%%%%%%%%%%%%%%%%%%
\DescribeMacro{\ifchilddocmanual}
The main file should be prepared as usual, see \secref{sec:include}.
However, the document body must make a distinction
between processing of an individual part and of the main document, e.g.:
%
\begin{center}
\begin{tabular}{l}
|\ifchilddocmanual|\\
|\input{\childdocname}|\\
|\||else|\\
\textit{document body with }|\input{|\textit{part}|}|\\
|\||fi|
\end{tabular}
\end{center}
%
The conditional |\ifchilddocmanual| is true whenever
a part to be included by |\input| is being compiled,
and the name of the part is stored in |\childdocname|.

%%%%%%%%%%%%%%%%%%%%%%%%%%%%%%%%%%%%%%%%
\DescribeMacro{\childdocby}
Each part to be included by |\input| should start with:
%
\begin{center}
\begin{tabular}{l}
|\input{childdoc.def}|\\
|\childdocby{|\textit{main}|}|\\
\end{tabular}
\end{center}
%
The directive |\childdocby| is similar to |\childdocof|
described in \secref{sec:include},
but the subsequent selection of content must be done manually.
To that end, both |\ifchilddoc| and |\ifchilddocmanual|
will be true upon processing of a part,
and the name of the part is stored in |\childdocname|.
Note that |\jobname| will be set to the filename of the current part
so that each part receives an individual |.aux| file
that does not interfere with the |.aux| file(s) of the main document.
This behaviour can be altered by the alternative form
|\childdocby[*]{|\textit{main}|}| (with a non-empty optional argument)
which uses the |.aux| file of the main document
by setting |\jobname| to \textit{main}.

%%%%%%%%%%%%%%%%%%%%%%%%%%%%%%%%%%%%%%%%%%%%%%%%%%%%%%%%%%%%%%%%%%%%%%%%%%%%%%%%
\subsection{Driver Development}
\label{sec:driver}

The \textsf{childdoc} mechanism can also be use for the development
of definition files such as \LaTeX{} styles or classes.
This case differs from the above setup with multiple parts
included by |\include| in that no |\includeonly| should be invoked.
This can be achieved by starting the include file
(before |\ProvidesPackage|) with:
%
\begin{center}
\begin{tabular}{l}
|\input{childdoc.def}|\\
|\childdocforward{|\textit{main}|}|\\
\end{tabular}
\end{center}
%
or alternatively with:
%
\begin{center}
\begin{tabular}{l}
|\input{childdoc.def}|\\
|\childdocby{|\textit{main}|}|\\
\end{tabular}
\end{center}
%
Both forms have slightly different effects as described above.
The main file is prepared as usual, see \secref{sec:include}.

%%%%%%%%%%%%%%%%%%%%%%%%%%%%%%%%%%%%%%%%%%%%%%%%%%%%%%%%%%%%%%%%%%%%%%%%%%%%%%%%
\subsection{Legacy Detection}
\label{sec:detection}

The directive |\childdocmain| in the main file can detect
whether the complete document or merely a child is to be compiled
even without using the directive |\childdocof|.
This method is deprecated because it is less robust
and there is no compelling reason to use it;
it is merely provided for backward compatibility
and it may be removed in future versions.

If the detection mechanism is to be used,
it is mandatory to correctly specify
the filename of the main file as the argument of |\childdocmain|:
%
\begin{center}
\begin{tabular}{l}
|\input{childdoc.def}|\\
|\childdocmain{|\textit{main}|}|\\
\end{tabular}
\end{center}
%
If |\jobname| does not match the argument \textit{main} of |\childdocmain|,
it is assumed that |\jobname| points to the child file to be compiled.
When using |\childdocmain| with the main file specified as argument,
it suffices to start a child file
with just |\input{|\textit{main}|}|
without loading of the package and using |\childdocof|.
If instead all processing is done
with the appropriate \textsf{childdoc} directives,
the argument of \textit{main} of |\childdocmain| can be empty.

An alternative version of the command line processing described
in \secref{sec:commandline} using the detection mechanism reads:
%
\begin{center}
|... -jobname "|\textit{target}|" "|[\textit{flags}]%
[|\def\jobname{|\textit{dest}|}|]|\input{|\textit{main}|}"|
\end{center}

%%%%%%%%%%%%%%%%%%%%%%%%%%%%%%%%%%%%%%%%%%%%%%%%%%%%%%%%%%%%%%%%%%%%%%%%%%%%%%%%
\subsection{Manual Code}
\label{sec:manual}

In case one cannot be certain whether the definitions file |childdoc.def|
is installed on the target \TeX{} distribution
and one prefers not to ship it,
it is conceivable to paste a few relevant commands into the sources.

To that end, drop all statements |\input{childdoc.def}|
and perform the replacements as outlined below.
Instead of |\childdocmain{|\textit{main}|}| add the following code
to the top of the main file:
%
\begin{center}
\begin{tabular}{l}
|\||ifdefined\childdocname\endinput\||fi\newif\ifchilddoc|\\
|\edef\childdocname{\scantokens\expandafter{\jobname\noexpand}}|\\
|\def\childdocmain{|\textit{main}|}\||ifx\childdocmain\childdocname\||else|\\
|\childdoctrue\includeonly{\childdocname}\let\jobname\childdocmain\||fi|\\
\end{tabular}
\end{center}
%
Instead of |\childdocof{|\textit{main}|}| just include the main file
at the top of each child file:
%
\begin{center}
|\input{|\textit{main}|}|
\end{center}
%
A simple redirection |\childdocforward{|\textit{dest}|}| is achieved by:
%
\begin{center}
|\def\jobname{|\textit{dest}|}\input{\jobname}|
\end{center}
%
The redirection with prefix
|\childdocforwardprefix[|\textit{prefix}|]{|\textit{dest}|}|
is accomplished by:
%
\begin{center}
\begin{tabular}{l}
|{\edef\jobname{\scantokens\expandafter{\jobname\noexpand}}|\\
|\def\redirectjob |\textit{prefix}|#1~~~{\gdef\jobname{|\textit{dest}|#1}}|\\
|\expandafter\redirectjob\jobname~~~}\input{\jobname}|
\end{tabular}
\end{center}

In an alternative approach,
child documents can be compiled by a specific command line
without additional code or specific definitions:
%
\begin{center}
|... -jobname "|\textit{target}|" "|[\textit{flags}]%
|\includeonly{|\textit{dest}|}\input{|\textit{main}|}"|
\end{center}
%

%%%%%%%%%%%%%%%%%%%%%%%%%%%%%%%%%%%%%%%%%%%%%%%%%%%%%%%%%%%%%%%%%%%%%%%%%%%%%%%%
%%%%%%%%%%%%%%%%%%%%%%%%%%%%%%%%%%%%%%%%%%%%%%%%%%%%%%%%%%%%%%%%%%%%%%%%%%%%%%%%
\section{Information}

%%%%%%%%%%%%%%%%%%%%%%%%%%%%%%%%%%%%%%%%%%%%%%%%%%%%%%%%%%%%%%%%%%%%%%%%%%%%%%%%
\subsection{Copyright}

Copyright \copyright{} 2017--2018 Niklas Beisert

This work may be distributed and/or modified under the
conditions of the \LaTeX{} Project Public License, either version 1.3
of this license or (at your option) any later version.
The latest version of this license is in
  \url{http://www.latex-project.org/lppl.txt}
and version 1.3 or later is part of all distributions of \LaTeX{}
version 2005/12/01 or later.

This work has the LPPL maintenance status `maintained'.

The Current Maintainer of this work is Niklas Beisert.

This work consists of the files |README.txt|, |childdoc.ins| and |childdoc.dtx|
as well as the derived files |childdoc.def|, |cdocsamp.tex|
with |cdocsch1.tex|, |cdocsch2.tex|, |cdocspt3.tex|, |cdocspt4.tex|,
|cdocsdrf.tex|, |cdocsfn1.tex|, |cdocsfn2.tex|
as well as |childdoc.pdf|.

%%%%%%%%%%%%%%%%%%%%%%%%%%%%%%%%%%%%%%%%%%%%%%%%%%%%%%%%%%%%%%%%%%%%%%%%%%%%%%%%
\subsection{Files and Installation}

The package consists of the files:
%
\begin{center}
\begin{tabular}{ll}
    |README.txt|   & readme file \\
    |childdoc.ins| & installation file \\
    |childdoc.dtx| & source file \\
    |childdoc.def| & definition file \\
    |cdocsamp.tex| & sample main file \\
    |cdocsch1.tex| & sample include file \\
    |cdocsch2.tex| & sample include file \\
    |cdocspt3.tex| & sample part file \\
    |cdocspt4.tex| & sample part file \\
    |cdocsdrf.tex| & sample redirection file \\
    |cdocsfn1.tex| & sample redirection file \\
    |cdocsfn2.tex| & sample redirection file \\
    |childdoc.pdf| & manual
\end{tabular}
\end{center}
%
The distribution consists of the files
|README.txt|, |childdoc.ins| and |childdoc.dtx|.
%
\begin{itemize}
\item
Run (pdf)\LaTeX{} on |childdoc.dtx|
to compile the manual |childdoc.pdf| (this file).
\item
Run \LaTeX{} on |childdoc.ins| to create the definitions file |childdoc.def|
and the sample |cdocsamp.tex| with include files
|cdocsch1.tex|, |cdocsch2.tex|, |cdocspt3.tex|, |cdocspt4.tex|,
|cdocsdrf.tex|, |cdocsfn1.tex|, |cdocsfn2.tex|.
Then copy the file |childdoc.def| to an appropriate directory of your \LaTeX{}
distribution, e.g.\ \textit{texmf-root}|/tex/latex/childdoc|.
\end{itemize}

%%%%%%%%%%%%%%%%%%%%%%%%%%%%%%%%%%%%%%%%%%%%%%%%%%%%%%%%%%%%%%%%%%%%%%%%%%%%%%%%
\subsection{Related CTAN Packages}

There are several other packages which offer a similar functionality:
%
\begin{itemize}
\item
The packages
\href{http://ctan.org/pkg/docmute}{\textsf{docmute}},
\href{http://ctan.org/pkg/includex}{\textsf{includex}} and
\href{http://ctan.org/pkg/standalone}{\textsf{standalone}}
provide commands to include only the document body of
a child file thus allowing both files to be compiled individually.
\item
The packages \href{http://ctan.org/pkg/subdocs}{\textsf{subdocs}}
and \href{http://ctan.org/pkg/subfiles}{\textsf{subfiles}}
provide structures in which the main and child documents can be
encapsulated and allowing them to be compiled individually.
The inclusion mechanism is different from the conventional |\include|.
\item
The package \href{http://ctan.org/pkg/combine}{\textsf{combine}}
is an elaborate solution to combine several documents into one.
\end{itemize}
%
See also the CTAN topic \href{http://ctan.org/topic/subdocs}{\textsf{subdocs}}
for further related packages.
The present package differs from the above solutions in that
a document structure constructed with the conventional |\include| mechanism
just needs two extra commands at the top of every file
such that all constituent files can be compiled individually.

%%%%%%%%%%%%%%%%%%%%%%%%%%%%%%%%%%%%%%%%%%%%%%%%%%%%%%%%%%%%%%%%%%%%%%%%%%%%%%%%
%\subsection{Feature Suggestions}
%
%The following is a list of features which may be useful for future
%versions of this package:
%%
%\begin{itemize}
%\item
%\ldots
%\end{itemize}

%%%%%%%%%%%%%%%%%%%%%%%%%%%%%%%%%%%%%%%%%%%%%%%%%%%%%%%%%%%%%%%%%%%%%%%%%%%%%%%%
\subsection{Revision History}

%%%%%%%%%%%%%%%%%%%%%%%%%%%%%%%%%%%%%%%%
\paragraph{v2.0:} 2018/12/30

\begin{itemize}
\item
immediate forward processing
\item
added |\childdocby| mechanism
\item
manual restructured
\end{itemize}

%%%%%%%%%%%%%%%%%%%%%%%%%%%%%%%%%%%%%%%%
\paragraph{v1.6:} 2018/01/17

\begin{itemize}
\item
application for development of include files
\item
corrections to manual
\end{itemize}

%%%%%%%%%%%%%%%%%%%%%%%%%%%%%%%%%%%%%%%%
\paragraph{v1.5:} 2017/05/21

\begin{itemize}
\item
more complete structuring introduced
\item
|\childdocof| introduced
\item
|\childdoc| renamed to |\childdocmain|
\item
|\childredirect| renamed to |\childdocforward| and |\childdocforwardprefix|
and functionality expanded
\end{itemize}

%%%%%%%%%%%%%%%%%%%%%%%%%%%%%%%%%%%%%%%%
\paragraph{v1.0:} 2017/04/27

\begin{itemize}
\item
manual and install package
\item
first version published on CTAN
\end{itemize}

%%%%%%%%%%%%%%%%%%%%%%%%%%%%%%%%%%%%%%%%
\paragraph{v0.6:} 2017/04/26

\begin{itemize}
\item
redirection mechanism added
\end{itemize}

%%%%%%%%%%%%%%%%%%%%%%%%%%%%%%%%%%%%%%%%
\paragraph{v0.5:} 2017/04/26

\begin{itemize}
\item
functionality in definition file
\end{itemize}


%%%%%%%%%%%%%%%%%%%%%%%%%%%%%%%%%%%%%%%%%%%%%%%%%%%%%%%%%%%%%%%%%%%%%%%%%%%%%%%%
%%%%%%%%%%%%%%%%%%%%%%%%%%%%%%%%%%%%%%%%%%%%%%%%%%%%%%%%%%%%%%%%%%%%%%%%%%%%%%%%
%%%%%%%%%%%%%%%%%%%%%%%%%%%%%%%%%%%%%%%%%%%%%%%%%%%%%%%%%%%%%%%%%%%%%%%%%%%%%%%%
\appendix

\settowidth\MacroIndent{\rmfamily\scriptsize 000\ }

 \DocInput{childdoc.dtx}

\end{document}
%</driver>
% \fi
%
% %%%%%%%%%%%%%%%%%%%%%%%%%%%%%%%%%%%%%%%%%%%%%%%%%%%%%%%%%%%%%%%%%%%%%%%%%%%%%%
% %%%%%%%%%%%%%%%%%%%%%%%%%%%%%%%%%%%%%%%%%%%%%%%%%%%%%%%%%%%%%%%%%%%%%%%%%%%%%%
% \section{Sample}
%\iffalse
%<*samplemain>
%\fi
%
% The following presents a sample document
% with two chapters, two parts, a title page,
% a compile flag as well as three forwarding files to set the flag.
% It consists of eight |.tex| files:
% \begin{center}
% \begin{tabular}{ll}
% |cdocsamp.tex|&main file\\
% |cdocsch1.tex|&include file for chapter 1\\
% |cdocsch2.tex|&include file for chapter 2\\
% |cdocspt3.tex|&include file for part 3\\
% |cdocspt4.tex|&include file for part 4\\
% |cdocsdrf.tex|&forwarding file for main file in draft mode\\
% |cdocsfi1.tex|&forwarding file for final version of chapter 1\\
% |cdocsfi2.tex|&forwarding file for final version of chapter 2\\
% \end{tabular}
% \end{center}
% Each of the eight files can be compiled directly by the \LaTeX{} compiler.
%
% %%%%%%%%%%%%%%%%%%%%%%%%%%%%%%%%%%%%%%
% \paragraph{Main File.}
%
% The main file is called |cdocsamp.tex|.
%
% Load the \textsf{childdoc} definitions and
% declare the filename for the main document:
%    \begin{macrocode}
\input{childdoc.def}
\childdocmain{}
%    \end{macrocode}

% Optional override for |\version| flag:
%    \begin{macrocode}
%%\ifchilddoc\else\providecommand{\version}{draft}\fi
%    \end{macrocode}

% Define the default values for the |\version| flag
% (|final| for the main file and |draft| for childs):
%    \begin{macrocode}
\ifchilddoc
\providecommand{\version}{draft}
\else
\providecommand{\version}{final}
\fi
%    \end{macrocode}

% Load the standard document class:
%    \begin{macrocode}
\documentclass[12pt]{article}
%    \end{macrocode}

% Start the document body:
%    \begin{macrocode}
\begin{document}
%    \end{macrocode}

% Declare a title page.
% Print title, part of document being processed and version flag:
%    \begin{macrocode}
\addtocounter{page}{-1}
\begin{center}
{\LARGE\bfseries{}childdoc example\par}
\vspace{1cm}
\ifchilddoc
\ifchilddocmanual part\else chapter\fi:
`\childdocname' of `\childdocjob'\par
\else
main document: `\childdocjob'\par
\fi
version: \version\par
\end{center}
\newpage
%    \end{macrocode}

% Manually include selected file,
% otherwise process as usual:
%    \begin{macrocode}
\ifchilddocmanual
\section*{part `\childdocname'}
\input{\childdocname}
\else
%    \end{macrocode}

% Include the two chapters:
%    \begin{macrocode}
\include{cdocsch1}
\include{cdocsch2}
%    \end{macrocode}

% Include the two parts unless only chapters should be displayed:
%    \begin{macrocode}
\ifchilddoc\else
\section{part three}
\input{cdocspt3}
\section{part four}
\input{cdocspt4}
\fi
%    \end{macrocode}

% Process as usual until here:
%    \begin{macrocode}
\fi
%    \end{macrocode}

% End of document body:
%    \begin{macrocode}
\end{document}
%    \end{macrocode}
%\iffalse
%</samplemain>
%\fi
%
% %%%%%%%%%%%%%%%%%%%%%%%%%%%%%%%%%%%%%%
% \paragraph{Chapter Include Files.}
%
% The include files are called |cdocsch1.tex| and |cdocsch2.tex|.
%
%\iffalse
%<*samplechap1|samplechap2>
%\fi

% Optional override for |\version| flag:
%    \begin{macrocode}
%%\providecommand{\version}{final}
%    \end{macrocode}

% Include the main document:
%    \begin{macrocode}
\input{childdoc.def}
\childdocof{cdocsamp}
%    \end{macrocode}

%\iffalse
%</samplechap1|samplechap2>
%\fi
%
%\iffalse
%<*samplechap1>
%\fi
% Some text for chapter 1:
%    \begin{macrocode}
\section{one}
some text in chapter one
%    \end{macrocode}

%\iffalse
%</samplechap1>
%\fi
% Some text for chapter 2:
%\iffalse
%<*samplechap2>
%\fi
%    \begin{macrocode}
\section{two}
more text in chapter two
%    \end{macrocode}

%\iffalse
%</samplechap2>
%\fi
%
% %%%%%%%%%%%%%%%%%%%%%%%%%%%%%%%%%%%%%%
% \paragraph{Part Include Files.}
%
% The include files are called |cdocspt3.tex| and |cdocspt4.tex|.
%
%\iffalse
%<*samplepart3|samplepart4>
%\fi

% Optional override for |\version| flag:
%    \begin{macrocode}
%%\providecommand{\version}{final}
%    \end{macrocode}

% Include the main document:
%    \begin{macrocode}
\input{childdoc.def}
\childdocby{cdocsamp}
%    \end{macrocode}

%\iffalse
%</samplepart3|samplepart4>
%\fi
%
%\iffalse
%<*samplepart3>
%\fi
% Some text for part 3:
%    \begin{macrocode}
some text in part three
%    \end{macrocode}

%\iffalse
%</samplepart3>
%\fi
% Some text for part 4:
%\iffalse
%<*samplepart4>
%\fi
%    \begin{macrocode}
more text in part four
%    \end{macrocode}

%\iffalse
%</samplepart4>
%\fi
%
% %%%%%%%%%%%%%%%%%%%%%%%%%%%%%%%%%%%%%%
% \paragraph{Forwarding for a Complete Draft.}
%
% The following forwarding file |cdocsdrf.tex|
% compiles the main document in draft mode:
%\iffalse
%<*sampledraft>
%\fi
%    \begin{macrocode}
\def\version{draft}
\input{childdoc.def}
\childdocforward{cdocsamp}
%    \end{macrocode}

%\iffalse
%</sampledraft>
%\fi
%
% %%%%%%%%%%%%%%%%%%%%%%%%%%%%%%%%%%%%%%
% \paragraph{Forwarding for Final Version of the Chapters.}
%
% The following forwarding files |cdocsfn1.tex| and |cdocsfn2.tex|
% (with identical content)
% compile the final versions of the child documents
% |cdocsch1.tex| and |cdocsch2.tex|, respectively:
%\iffalse
%<*samplefinal>
%\fi
%    \begin{macrocode}
\def\version{final}
\input{childdoc.def}
\childdocforwardprefix[cdocsamp]{cdocsfn}{cdocsch}
%    \end{macrocode}

%\iffalse
%</samplefinal>
%\fi
%
% %%%%%%%%%%%%%%%%%%%%%%%%%%%%%%%%%%%%%%
% \paragraph{Command Line Processing.}
%
% The following three command lines generate the output files
% |cdocscld|, |cdocscl1| and |cdocscl2|
% which should be identical to
% |cdocsdrf|, |cdocsch1| and |cdocsfn2|, respectively:
% \begin{center}
% \begin{tabular}{l}
% |latex -jobname cdocscld \|\\
% |  "\def\version{draft}\input{childdoc.def}\childdocforward{cdocsamp}"|\\
% |latex -jobname cdocscl1 \|\\
% |  "\input{childdoc.def}\childdocforward[cdocsamp]{cdocsch1}"|\\
% |latex -jobname cdocscl2 \|\\
% |  "\def\version{final}\input{childdoc.def}\childdocforward{cdocsch2}"|
% \end{tabular}
% \end{center}
% Note that the trailing backslash on each first line
% merely continues the input to the second line
% (for convenient cut ant paste).
% Furthermore, the command |latex| can be replaced by any
% of its alternative versions such as |pdflatex|.
%
% %%%%%%%%%%%%%%%%%%%%%%%%%%%%%%%%%%%%%%%%%%%%%%%%%%%%%%%%%%%%%%%%%%%%%%%%%%%%%%
% %%%%%%%%%%%%%%%%%%%%%%%%%%%%%%%%%%%%%%%%%%%%%%%%%%%%%%%%%%%%%%%%%%%%%%%%%%%%%%
% \section{Implementation}
%\iffalse
%<*package>
%\fi
%
% This section describes the definitions file |childdoc.def|.

% The definitions cannot be loaded using |\usepackage| or |\RequirePackage|
% which has a mechanism to prevent loading a style file more than once.
% When loading the definitions by means of |\input|
% multiple instances have to be prevented manually:
%\iffalse
%This code needs to be before the `\ProvidesFile' directive
%which is defined at the beginning of this file.
%Therefore it is also placed there and commented out here.
%</package>
%<*discard>
%\fi
%    \begin{macrocode}
\ifdefined\childdocmain\endinput\fi
%    \end{macrocode}
%\iffalse
%</discard>
%<*package>
%\fi
%
% \macro{\ifchilddoc}
% \macro{\ifchilddocmanual}
% The conditional |\ifchilddoc| tells whether a
% child (true) or main (false) document is being compiled.
% The conditional |\ifchilddocmanual| tells whether
% the |\includeonly| mechanism is used (false) or
% the selection of child files must be performed manually (true).
% The definitions initialise to false:
%    \begin{macrocode}
\newif\ifchilddoc
\newif\ifchilddocmanual
%    \end{macrocode}

% \macro{\childdocname}
% \macro{\childdocjob}
% The macro |\childdocname| stores the name of the main document
% to be compiled. The macro |\childdocjob| stores the name of
% the document on which the \LaTeX{} compiler was originally invoked.
% The content of |\jobname| cannot be compared
% to filenames specified in the source due to different catcodes.
% The following code rescans |\jobname|, stores the result
% in |\childdocname| and saves a copy in |\childdocjob|:
%    \begin{macrocode}
\edef\childdocname{\scantokens\expandafter{\jobname\noexpand}}
\let\childdocjob\childdocname
%    \end{macrocode}

% \macro{\childdocdisable}
% The macro |\childdocdisable| prevents the main file
% from being processed more than once.
% At this stage, the main document command |\childdocmain|
% is assumed to be called once again where it should do nothing.
% Any subsequent call to it should prevent
% a secondary processing of the main document
% It overwrites the forwarding commands
% |\childdocof| and |\childdocforward|
% with empty macros to prevent further inclusions of the main document:
%    \begin{macrocode}
\newcommand{\childdocdisable}
{
  \renewcommand{\childdocmain}[1]{\renewcommand{\childdocmain}[1]{\endinput}}
  \renewcommand{\childdocof}[1]{}
  \renewcommand{\childdocby}[2][]{}
  \renewcommand{\childdocforward}[2][]{}
  \renewcommand{\childdocdisable}{}
}
%    \end{macrocode}

% \macro{\childdocmain}
% The macro |\childdocmain| is to be called at the top of the main file
% with nothing or the main filename (without extension) as argument.
% First, it breaks loops.
% If the argument is not empty and does not match |\childdocname|
% (which is set by the first inclusion of |childdoc.def|),
% |\ifchilddoc| is set to true, |\includeonly| is applied to the child file
% and |\jobname| is set to the main file
% (for proper handling of |.aux| files):
%    \begin{macrocode}
\newcommand{\childdocmain}[1]
{
  \childdocdisable\childdocmain{}
  \if?#1?\else
    \begingroup
      \def\childdoctmp{#1}
      \ifx\childdoctmp\childdocname
        \def\childdoctmp{}
      \else
        \def\childdoctmp
        {
          \childdoctrue
          \includeonly{\childdocname}
          \def\childdocjob{#1}
          \def\jobname{#1}
        }
      \fi
      \expandafter
    \endgroup
    \childdoctmp
  \fi
}
%    \end{macrocode}

% \macro{\childdocof}
% The command |\childdocof| redirects
% compilation to the main file |#1|.
%    \begin{macrocode}
\newcommand{\childdocof}[1]
{
  \childdocdisable
  \childdoctrue
  \includeonly{\childdocname}
  \def\jobname{#1}
  \def\childdocjob{#1}
  \input{#1}
}
%    \end{macrocode}

% \macro{\childdocby}
% The command |\childdocby| ....
%    \begin{macrocode}
\newcommand{\childdocby}[2][]
{
  \childdocdisable
  \childdoctrue
  \childdocmanualtrue
  \if?#1?\else
    \def\jobname{#2}
  \fi
  \def\childdocjob{#2}
  \input{#2}
  \endinput
}
%    \end{macrocode}

% \macro{\childdocforward}
% The command |\childdocforward| redirects
% compilation to the main file or
% (if the optional argument is given) a child file.
% Parameters are set as if the main file
% or a child file starting with |\childdocof| was compiled.
% Then compilation is handed over to the main file:
%    \begin{macrocode}
\newcommand{\childdocforward}[2][]
{
  \begingroup
    \if?#1?
      \def\childdoctmp
      {
        \def\childdocname{#2}
        \def\childdocjob{#2}
        \def\jobname{#2}
        \input{#2}
        \endinput
      }
    \else
      \def\childdoctmp
      {
        \childdocdisable
        \def\childdocname{#2}
        \childdoctrue
        \includeonly{#2}
        \def\childdocjob{#1}
        \def\jobname{#1}
        \input{#1}
        \endinput
      }
    \fi
    \expandafter
  \endgroup
  \childdoctmp
}
%    \end{macrocode}

% \macro{\childdocforwardprefix}
% The command |\childdocforwardprefix| redirects
% compilation to the main or a child file by means of a pattern.
% The prefix |#1| in the current filename is replaced by |#2|
% and the suffix of the current filename is kept
% (it is assumed that the filename does not contain the substring `|~~~|'
% which is used as a delimiter).
% Compilation is handed over to the new file by |\childdocforward|:
%    \begin{macrocode}
\newcommand{\childdocforwardprefix}[3][]
{
  \begingroup
    \def\childdocextract #2##1~~~{\def\childdoctmp{\childdocforward[#1]{#3##1}}}
    \expandafter\childdocextract\childdocname~~~
    \expandafter
  \endgroup
  \childdoctmp
}
%    \end{macrocode}

% \macro{\childdoc}
% The deprecated macro |\childdoc| is a legacy version of |\childdocmain|:
%    \begin{macrocode}
\newcommand{\childdoc}{\childdocmain}
%    \end{macrocode}

% \macro{\childdocredirect}
% The deprecated macro |\childdocredirect| is a legacy version
% of |\childdocforward| and |\childdocforwardprefix|:
%    \begin{macrocode}
\newcommand{\childdocredirect}[2][]
{
  \begingroup
    \if?#1?
      \def\childdoctmp{\childdocforward{#2}}
    \else
      \def\childdoctmp{\childdocforwardprefix{#1}{#2}}
    \fi
    \expandafter
  \endgroup
  \childdoctmp
}
%    \end{macrocode}

%\iffalse
%</package>
%\fi
%
\endinput
|\\
|\childdocforwardprefix[|\textit{main}|]{|\textit{prefix}|}{|\textit{dest}|}|
\end{tabular}
\end{center}
%
the destination file is determined by a pattern
depending on the current file:
To make this work, the current file must be called
`{\textit{prefix}\hspace{0.2em}\textit{suffix}}'
with \textit{prefix} matching precisely the argument.
Processing is then passed on to the file
`{\textit{dest}\hspace{0.2em}\textit{suffix}}'.
Surely, the same effect is achieved by
directly specifying the
argument `{\textit{dest}\hspace{0.2em}\textit{suffix}}'
in the first form.
However, that requires to set up a different file
for each child. With the alternative form of the command
all these files can have exactly the same content
which simplifies setting them up and maintaining them.

For example, the following file |draft.tex|
with a compilation flag |\version| as described in \secref{sec:flags}
compiles the main document as a draft:
%
\begin{center}
\begin{tabular}{l}
|\def\version{draft}|\\
|% \iffalse
%
% childdoc.dtx Copyright (C) 2017-2018 Niklas Beisert
%
% This work may be distributed and/or modified under the
% conditions of the LaTeX Project Public License, either version 1.3
% of this license or (at your option) any later version.
% The latest version of this license is in
%   http://www.latex-project.org/lppl.txt
% and version 1.3 or later is part of all distributions of LaTeX
% version 2005/12/01 or later.
%
% This work has the LPPL maintenance status `maintained'.
%
% The Current Maintainer of this work is Niklas Beisert.
%
% This work consists of the files childdoc.dtx and childdoc.ins
% and the derived files childdoc.def and cdocsamp.tex with
% cdocsch1.tex, cdocsch2.tex, cdocsdrf.tex, cdocsfn1.tex, cdocsfn2.tex.
%
%<package>\ifdefined\childdocmain\endinput\fi
%<package>\ProvidesFile{childdoc.def}[2018/12/30 v2.0 child document driver]
%<samplemain>\ProvidesFile{cdocsamp.tex}[2018/12/30 v2.0 sample for childdoc]
%<*driver>
%\ProvidesFile{childdoc.drv}[2018/12/30 v2.0 childdoc reference manual file]
\PassOptionsToClass{10pt,a4paper}{article}
\documentclass{ltxdoc}

\usepackage[margin=35mm]{geometry}
\usepackage{hyperref}
\usepackage{hyperxmp}
\usepackage[usenames]{color}

\hypersetup{colorlinks=true}
\hypersetup{pdfstartview=FitH}
\hypersetup{pdfpagemode=UseNone}
\hypersetup{pdfsource={}}
\hypersetup{pdflang={en-UK}}
\hypersetup{pdfcopyright={Copyright 2017-2018 Niklas Beisert.
  This work may be distributed and/or modified under the
  conditions of the LaTeX Project Public License, either version 1.3
  of this license or (at your option) any later version.}}
\hypersetup{pdflicenseurl={http://www.latex-project.org/lppl.txt}}
\hypersetup{pdfcontactaddress={ETH Zurich, ITP, HIT K,
  Wolfgang-Pauli-Strasse 27}}
\hypersetup{pdfcontactpostcode={8093}}
\hypersetup{pdfcontactcity={Zurich}}
\hypersetup{pdfcontactcountry={Switzerland}}
\hypersetup{pdfcontactemail={nbeisert@itp.phys.ethz.ch}}
\hypersetup{pdfcontacturl={http://people.phys.ethz.ch/\xmptilde nbeisert/}}

\newcommand{\secref}[1]{\hyperref[#1]{section \ref*{#1}}}

\parskip1ex
\parindent0pt
\let\olditemize\itemize
\def\itemize{\olditemize\parskip0pt}

\begin{document}

\title{The \textsf{childdoc} Package}
\hypersetup{pdftitle={The childdoc Package}}
\author{Niklas Beisert\\[2ex]
  Institut f\"ur Theoretische Physik\\
  Eidgen\"ossische Technische Hochschule Z\"urich\\
  Wolfgang-Pauli-Strasse 27, 8093 Z\"urich, Switzerland\\[1ex]
  \href{mailto:nbeisert@itp.phys.ethz.ch}
  {\texttt{nbeisert@itp.phys.ethz.ch}}}
\hypersetup{pdfauthor={Niklas Beisert}}
\hypersetup{pdfsubject={Manual for the LaTeX2e Package childdoc}}
\date{30 December 2018, \textsf{v2.0}}
\maketitle

\begin{abstract}\noindent
\textsf{childdoc} is a \LaTeXe{} package
that enables the direct compilation
of document sections included by |\include|
to individual files.
\end{abstract}

\begingroup
\parskip0ex
\tableofcontents
\endgroup

%%%%%%%%%%%%%%%%%%%%%%%%%%%%%%%%%%%%%%%%%%%%%%%%%%%%%%%%%%%%%%%%%%%%%%%%%%%%%%%%
%%%%%%%%%%%%%%%%%%%%%%%%%%%%%%%%%%%%%%%%%%%%%%%%%%%%%%%%%%%%%%%%%%%%%%%%%%%%%%%%
\section{Introduction}

\LaTeX{} provides a mechanism to structure a large document (such as a book)
into a main file and several child files (containing the chapters)
using the |\include| command.
This mechanism is beneficial for documents
which span hundreds of pages in order to
make the source file(s) more manageable.
Moreover, compilation can be restricted to
selected child files by means of the |\includeonly| command.
The latter feature can be used to reduce the compilation time while editing
(this was significantly more useful in the earlier days of \LaTeX{})
or to generate a smaller document which is easier to navigate.
Another application of |\includeonly| is to generate
documents consisting of selected parts of the complete document.

However, there are a few drawbacks of the plain |\include| mechanism:
\begin{itemize}
\item
The child files cannot be compiled on their own,
they can only be compiled via the main file.
A naive editing environment
(such as a text editor with an option
to have the current file processed by \LaTeX)
may require one to switch to the main file before compiling;
attempting to compile the child file produces errors.
\item
The main file must be modified (each time)
to adjust the |\includeonly| command
to the present needs. This easily leaves the main file in a messy state.
\item
The generated document will always carry the filename
of the main document. This is inconvenient if
several child files are to be compiled and
to be kept for distribution.
\end{itemize}

The present package provides a simple interface
to make child files individually compilable by \LaTeX{}.
Compiling a child file then has the same effect as compiling
the main file with an |\includeonly| command
to select the appropriate child.
Moreover the generated document will carry the name of the child
rather than the main file.
This resolves all three above issues.

This feature is meant to make the editing of books,
thesis documents and lecture notes somewhat more convenient.
However, the package can also be used efficiently for
composing a series of documents (such as exercise sheets)
which are typically distributed individually.
It then assists the author in generating the individual documents
(potentially in different versions)
as well as a document containing the collected series.
Another application is in developing style files
or other kinds of included material
where compilation of the style file could redirect
to a sample or test file.

%%%%%%%%%%%%%%%%%%%%%%%%%%%%%%%%%%%%%%%%%%%%%%%%%%%%%%%%%%%%%%%%%%%%%%%%%%%%%%%%
%%%%%%%%%%%%%%%%%%%%%%%%%%%%%%%%%%%%%%%%%%%%%%%%%%%%%%%%%%%%%%%%%%%%%%%%%%%%%%%%
\section{Usage}

First of all, the package \textsf{childdoc} is \emph{not} a standard
\LaTeXe{} |.sty| style file! Therefore it needs to be invoked in
a non-standard way.

%%%%%%%%%%%%%%%%%%%%%%%%%%%%%%%%%%%%%%%%%%%%%%%%%%%%%%%%%%%%%%%%%%%%%%%%%%%%%%%%
\subsection{Included Files}
\label{sec:include}

%%%%%%%%%%%%%%%%%%%%%%%%%%%%%%%%%%%%%%%%
\DescribeMacro{\childdocmain}
To use the package, add the commands
\begin{center}
\begin{tabular}{l}
|\input{childdoc.def}|\\
|\childdocmain{}|\\
\end{tabular}
\end{center}
at the very top of the main \LaTeX{} file,
in particular \emph{before} the |\documentclass| statement!
The argument of |\childdocmain| should be left empty
(but it must be present).

%%%%%%%%%%%%%%%%%%%%%%%%%%%%%%%%%%%%%%%%
\DescribeMacro{\childdocof}
Furthermore, add the commands
\begin{center}
\begin{tabular}{l}
|\input{childdoc.def}|\\
|\childdocof{|\textit{main}|}|\\
\end{tabular}
\end{center}
at the top of every child file \textit{child}
which is included by |\include{|\textit{child}|}|
from within the main file
(or at least for those files to be compiled individually).
The argument \textit{main} must be the filename of the main file.

There are a couple of
considerations in setting up the main and child documents:

%%%%%%%%%%%%%%%%%%%%%%%%%%%%%%%%%%%%%%%%
\paragraph{Restrictions.}

Please note the following restrictions:
\begin{itemize}
\item
|\childdocmain| must be called with one argument \textit{main}
to ensure compatibility with earlier version of the package.
It must either be empty (|\childdocmain{}|)
or precisely match the filename of the main file in which it is specified.
See \secref{sec:detection} for further information.
\item
The filename \textit{main} must be specified without the |.tex| extension.
\item
The filename \textit{main} is case sensitive
(even in case-insensitive file systems)
due to internal string comparison.
\item
The argument \textit{main} should be fully expanded, it cannot be a macro.
\item
Subdirectories and special characters should be avoided in filenames.
\item
The command |\childdocmain{|\textit{main}|}| must be followed by a whitespace.
It should not be followed immediately by another command
or by a comment mark `|%|'.
This is because the \TeX{} parser reads the token immediately following
the argument of |\childdocmain| and puts it
at the beginning of every child section;
however, a white\-space is ignored.
\end{itemize}

%%%%%%%%%%%%%%%%%%%%%%%%%%%%%%%%%%%%%%%%
\paragraph{Content of Main File.}

It is advisable to place all content in the child files included by |\include|.
Any output contained in the main file will appear in all child documents
unless suppressed manually;
it cannot be suppressed automatically by the |\includeonly| directive
and thus should normally be avoided.
A method to include some content in the main file
by means of conditional processing is described in \secref{sec:conditional}.

%%%%%%%%%%%%%%%%%%%%%%%%%%%%%%%%%%%%%%%%
\paragraph{Page Numbering.}

When only a part of the document is compiled,
the appropriate numbering of pages
(as well as other status parameters)
is determined from the |.aux| files.
The latter contain information from previous passes.
However this information needs to propagate through
all intermediate child documents.
Therefore the page numbering in child documents may well
be inconsistent until the complete document is compiled at least once.

A useful (if unconventional) way to always ensure a consistent
page numbering is to restart the numbering in each child document
and denote the pages by `\textit{child}|.|\textit{page}'
where \textit{child} represents the chapter/section number of the child file.
This can be achieved by the command
|\numberwithin{page}{|\textit{child}|}|
of the \textsf{amsmath} package
where \textit{child} can be |chapter| or |section|
depending on the chosen structuring.
Alternatively, one can modify the macro |\thepage| appropriately
and reset the counter |page| at the start of each child file.

%%%%%%%%%%%%%%%%%%%%%%%%%%%%%%%%%%%%%%%%%%%%%%%%%%%%%%%%%%%%%%%%%%%%%%%%%%%%%%%%
\subsection{Conditional Processing}
\label{sec:conditional}

The package provides a mechanism to compile different versions
of a document. To customise the versions further some conditional processing
can come in handy to distinguish which version is being compiled.
The package provides two macros to describe the compilation context:

%%%%%%%%%%%%%%%%%%%%%%%%%%%%%%%%%%%%%%%%
\DescribeMacro{\ifchilddoc}
The conditional |\ifchilddoc| distinguishes between the compilation of
child documents and the main document:
%
\begin{center}
|\ifchilddoc |\textit{child-code}| |[|\||else |\textit{main-code}]| \||fi|
\end{center}

%%%%%%%%%%%%%%%%%%%%%%%%%%%%%%%%%%%%%%%%
\DescribeMacro{\childdocname}
\DescribeMacro{\childdocjob}
The macro |\childdocname| contains the filename (without extension)
of the main or child file being processed.
Note that |\childdocjob| will always contain the name of the main file.

%%%%%%%%%%%%%%%%%%%%%%%%%%%%%%%%%%%%%%%%
\paragraph{Title Page.}

Conditional processing can be used to include a title or banner page
in the main document when proper precautions are taken.
Importantly, the code in the main file should ensure that the page counter
(as well as other status parameters which are stored in the |.aux| files)
takes the same value after the conditional processing.
Otherwise the page numbers may take divergent values
depending on which part is compiled.

For example, a title page could be declared by:
%
\begin{center}
\begin{tabular}{l}
|\ifchilddoc\||else|\\
|\addtocounter{page}{-1}|\\
\textit{code for title page}\\
|\newpage|\\
|\||fi|
\end{tabular}
\end{center}
%
A banner page for the child documents can be generated by:
%
\begin{center}
\begin{tabular}{l}
|\ifchilddoc|\\
|\addtocounter{page}{-1}|\\
\textit{code for banner page}\\
|\newpage|\\
|\||fi|
\end{tabular}
\end{center}
%
Here one could write a message such as:
\begin{center}
|This is the part \childdocname{} of \childdocjob{}.|
\end{center}

%%%%%%%%%%%%%%%%%%%%%%%%%%%%%%%%%%%%%%%%%%%%%%%%%%%%%%%%%%%%%%%%%%%%%%%%%%%%%%%%
\subsection{Flags}
\label{sec:flags}

The package makes it easy to generate different versions
of the main or child documents.
To this end compilation flags can be defined
and assigned different default values.
They will be particularly useful in conjunction
with the forwarding mechanism described in \secref{sec:forward}.

For example, it may be useful to have a flag |\version|
which can be set to |draft| or |final|.
The document source will contain some conditional code
depending on the value of |\version|.
Suppose further, the flag should default to |final| for the main file
and to |draft| for child files
which is a natural assignment for editing the document.
This is achieved by placing the following code
in the preamble of the main document
(below the |\childdocmain| directive):
%
\begin{center}
\begin{tabular}{l}
|\ifchilddoc|\\
|\providecommand{\version}{draft}|\\
|\||else|\\
|\providecommand{\version}{final}|\\
|\||fi|
\end{tabular}
\end{center}
%
The definition by |\providecommand| makes sure
that previous definitions are not overwritten.
Further statements |\providecommand{\version}{...}|
can thus be added before the above code to override it.

For the main file, one might add a line
(between |\childdocmain| and the above block)
%
\begin{center}
|%\ifchilddoc\||else\providecommand{\version}{draft}\||fi|
\end{center}
%
which can be uncommented to produce a draft version.
Likewise one can add a line to the very top of a child file
(above the |\childdocof{|\textit{main}|}| directive)
%
\begin{center}
|%\providecommand{\version}{final}|
\end{center}
%
which can be uncommented to produce the final version of this child document.

%%%%%%%%%%%%%%%%%%%%%%%%%%%%%%%%%%%%%%%%%%%%%%%%%%%%%%%%%%%%%%%%%%%%%%%%%%%%%%%%
\subsection{Forwarding}
\label{sec:forward}

Different versions of the main or child documents
using compilation flags as described in \secref{sec:flags}
can be (permanently) stored in different files
for convenient compilation, viewing and distribution.
To this end, the package defines a command
to pass on compilation to a different file:

%%%%%%%%%%%%%%%%%%%%%%%%%%%%%%%%%%%%%%%%
\DescribeMacro{\childdocforward}
The command |\childdocforward| redirects processing to
another source file:
%
\begin{center}
\begin{tabular}{l}
|\input{childdoc.def}|\\
|\childdocforward[|\textit{main}|]{|\textit{dest}|}|\\
\end{tabular}
\end{center}
%
The argument \textit{dest} is the destination file
(without extension).
It should be the main file or one of the child files.
Note that further \textsf{childdoc} directives
such as |\childdocof| and |\childdocforward|
in the indicated file will be processed in this form.
The optional argument \textit{main}
passes on directly to the main file \textit{main}
while pretending to compile the child \textit{dest}.
This form behaves as if \textit{dest}
issues |\childdocof{|\textit{main}|}| right away,
and no further \textsf{childdoc} directives will be processed.

%%%%%%%%%%%%%%%%%%%%%%%%%%%%%%%%%%%%%%%%
\DescribeMacro{\...prefix}
In the alternative form |\childdocforwardprefix|,
%
\begin{center}
\begin{tabular}{l}
|\input{childdoc.def}|\\
|\childdocforwardprefix[|\textit{main}|]{|\textit{prefix}|}{|\textit{dest}|}|
\end{tabular}
\end{center}
%
the destination file is determined by a pattern
depending on the current file:
To make this work, the current file must be called
`{\textit{prefix}\hspace{0.2em}\textit{suffix}}'
with \textit{prefix} matching precisely the argument.
Processing is then passed on to the file
`{\textit{dest}\hspace{0.2em}\textit{suffix}}'.
Surely, the same effect is achieved by
directly specifying the
argument `{\textit{dest}\hspace{0.2em}\textit{suffix}}'
in the first form.
However, that requires to set up a different file
for each child. With the alternative form of the command
all these files can have exactly the same content
which simplifies setting them up and maintaining them.

For example, the following file |draft.tex|
with a compilation flag |\version| as described in \secref{sec:flags}
compiles the main document as a draft:
%
\begin{center}
\begin{tabular}{l}
|\def\version{draft}|\\
|\input{childdoc.def}|\\
|\childdocforward{|\textit{main}|}|
\end{tabular}
\end{center}
%
Likewise, the following files |final|\textit{nn}|.tex|
compile the final version of the child document
|child|\textit{nn}|.tex|:
%
\begin{center}
\begin{tabular}{l}
|\def\version{final}|\\
|\input{childdoc.def}|\\
|\childdocforwardprefix{final}{child}|
\end{tabular}
\end{center}
%

Note that when several versions of a main file and/or of each child file
are to be generated, it may be convenient to set up a |Makefile| or
shell script to automatise the process.

%%%%%%%%%%%%%%%%%%%%%%%%%%%%%%%%%%%%%%%%%%%%%%%%%%%%%%%%%%%%%%%%%%%%%%%%%%%%%%%%
\subsection{Command Line Processing}
\label{sec:commandline}

The effect of redirection files can also be achieved by invoking
the \LaTeX{} compiler with a more elaborate command line.
Most conveniently this should be done as part
of a shell script or a |Makefile|.

When using \textsf{childdoc} in the main file, the following
command lines effectively perform a redirection
(note that depending on the shell being used,
backslashes may have to be doubled: `|\|' $\to$ `|\\|'):
%
\begin{center}
|... -jobname "|\textit{target}|" |\\|"|[\textit{flags}]%
|\input{childdoc.def}\childdocforward[|\textit{main}|]{|\textit{dest}|}"|
\end{center}
%
Here \textit{target} is the name of the output file,
\textit{main} is the name of the main file
and \textit{dest} is the name of the main or child file to be processed
(all filenames without extensions).
The optional argument \textit{main} can be omitted
if \textit{main} matches \textit{dest}.
Optionally, compilation \textit{flags} can be defined via |\def| commands.
This command line makes the \TeX{} engine believe
it is compiling the file \textit{target}
whose content is specified as the latter parameter.
The provided code then forwards the processing to
\textit{main} or \textit{dest} as described in \secref{sec:forward}.

%%%%%%%%%%%%%%%%%%%%%%%%%%%%%%%%%%%%%%%%%%%%%%%%%%%%%%%%%%%%%%%%%%%%%%%%%%%%%%%%
\subsection{Include by Input}
\label{sec:input}

Including child documents by |\include| has some restrictions by design.
Most notably, the content of a child document always occupies
its own set of pages; pages cannot be shared between child documents.
Usually, this behaviour makes perfect sense
because each child document contain an essential part of the document.
However, in some situations it may be desirable to compose
a document from a collection of parts
without having mandatory page breaks between then.
For this case, the package
provides a mechanism to include parts
by |\input| which can also be processed individually.
However, by construction this mechanism
requires manual handling of the content to be output.

%%%%%%%%%%%%%%%%%%%%%%%%%%%%%%%%%%%%%%%%
\DescribeMacro{\ifchilddocmanual}
The main file should be prepared as usual, see \secref{sec:include}.
However, the document body must make a distinction
between processing of an individual part and of the main document, e.g.:
%
\begin{center}
\begin{tabular}{l}
|\ifchilddocmanual|\\
|\input{\childdocname}|\\
|\||else|\\
\textit{document body with }|\input{|\textit{part}|}|\\
|\||fi|
\end{tabular}
\end{center}
%
The conditional |\ifchilddocmanual| is true whenever
a part to be included by |\input| is being compiled,
and the name of the part is stored in |\childdocname|.

%%%%%%%%%%%%%%%%%%%%%%%%%%%%%%%%%%%%%%%%
\DescribeMacro{\childdocby}
Each part to be included by |\input| should start with:
%
\begin{center}
\begin{tabular}{l}
|\input{childdoc.def}|\\
|\childdocby{|\textit{main}|}|\\
\end{tabular}
\end{center}
%
The directive |\childdocby| is similar to |\childdocof|
described in \secref{sec:include},
but the subsequent selection of content must be done manually.
To that end, both |\ifchilddoc| and |\ifchilddocmanual|
will be true upon processing of a part,
and the name of the part is stored in |\childdocname|.
Note that |\jobname| will be set to the filename of the current part
so that each part receives an individual |.aux| file
that does not interfere with the |.aux| file(s) of the main document.
This behaviour can be altered by the alternative form
|\childdocby[*]{|\textit{main}|}| (with a non-empty optional argument)
which uses the |.aux| file of the main document
by setting |\jobname| to \textit{main}.

%%%%%%%%%%%%%%%%%%%%%%%%%%%%%%%%%%%%%%%%%%%%%%%%%%%%%%%%%%%%%%%%%%%%%%%%%%%%%%%%
\subsection{Driver Development}
\label{sec:driver}

The \textsf{childdoc} mechanism can also be use for the development
of definition files such as \LaTeX{} styles or classes.
This case differs from the above setup with multiple parts
included by |\include| in that no |\includeonly| should be invoked.
This can be achieved by starting the include file
(before |\ProvidesPackage|) with:
%
\begin{center}
\begin{tabular}{l}
|\input{childdoc.def}|\\
|\childdocforward{|\textit{main}|}|\\
\end{tabular}
\end{center}
%
or alternatively with:
%
\begin{center}
\begin{tabular}{l}
|\input{childdoc.def}|\\
|\childdocby{|\textit{main}|}|\\
\end{tabular}
\end{center}
%
Both forms have slightly different effects as described above.
The main file is prepared as usual, see \secref{sec:include}.

%%%%%%%%%%%%%%%%%%%%%%%%%%%%%%%%%%%%%%%%%%%%%%%%%%%%%%%%%%%%%%%%%%%%%%%%%%%%%%%%
\subsection{Legacy Detection}
\label{sec:detection}

The directive |\childdocmain| in the main file can detect
whether the complete document or merely a child is to be compiled
even without using the directive |\childdocof|.
This method is deprecated because it is less robust
and there is no compelling reason to use it;
it is merely provided for backward compatibility
and it may be removed in future versions.

If the detection mechanism is to be used,
it is mandatory to correctly specify
the filename of the main file as the argument of |\childdocmain|:
%
\begin{center}
\begin{tabular}{l}
|\input{childdoc.def}|\\
|\childdocmain{|\textit{main}|}|\\
\end{tabular}
\end{center}
%
If |\jobname| does not match the argument \textit{main} of |\childdocmain|,
it is assumed that |\jobname| points to the child file to be compiled.
When using |\childdocmain| with the main file specified as argument,
it suffices to start a child file
with just |\input{|\textit{main}|}|
without loading of the package and using |\childdocof|.
If instead all processing is done
with the appropriate \textsf{childdoc} directives,
the argument of \textit{main} of |\childdocmain| can be empty.

An alternative version of the command line processing described
in \secref{sec:commandline} using the detection mechanism reads:
%
\begin{center}
|... -jobname "|\textit{target}|" "|[\textit{flags}]%
[|\def\jobname{|\textit{dest}|}|]|\input{|\textit{main}|}"|
\end{center}

%%%%%%%%%%%%%%%%%%%%%%%%%%%%%%%%%%%%%%%%%%%%%%%%%%%%%%%%%%%%%%%%%%%%%%%%%%%%%%%%
\subsection{Manual Code}
\label{sec:manual}

In case one cannot be certain whether the definitions file |childdoc.def|
is installed on the target \TeX{} distribution
and one prefers not to ship it,
it is conceivable to paste a few relevant commands into the sources.

To that end, drop all statements |\input{childdoc.def}|
and perform the replacements as outlined below.
Instead of |\childdocmain{|\textit{main}|}| add the following code
to the top of the main file:
%
\begin{center}
\begin{tabular}{l}
|\||ifdefined\childdocname\endinput\||fi\newif\ifchilddoc|\\
|\edef\childdocname{\scantokens\expandafter{\jobname\noexpand}}|\\
|\def\childdocmain{|\textit{main}|}\||ifx\childdocmain\childdocname\||else|\\
|\childdoctrue\includeonly{\childdocname}\let\jobname\childdocmain\||fi|\\
\end{tabular}
\end{center}
%
Instead of |\childdocof{|\textit{main}|}| just include the main file
at the top of each child file:
%
\begin{center}
|\input{|\textit{main}|}|
\end{center}
%
A simple redirection |\childdocforward{|\textit{dest}|}| is achieved by:
%
\begin{center}
|\def\jobname{|\textit{dest}|}\input{\jobname}|
\end{center}
%
The redirection with prefix
|\childdocforwardprefix[|\textit{prefix}|]{|\textit{dest}|}|
is accomplished by:
%
\begin{center}
\begin{tabular}{l}
|{\edef\jobname{\scantokens\expandafter{\jobname\noexpand}}|\\
|\def\redirectjob |\textit{prefix}|#1~~~{\gdef\jobname{|\textit{dest}|#1}}|\\
|\expandafter\redirectjob\jobname~~~}\input{\jobname}|
\end{tabular}
\end{center}

In an alternative approach,
child documents can be compiled by a specific command line
without additional code or specific definitions:
%
\begin{center}
|... -jobname "|\textit{target}|" "|[\textit{flags}]%
|\includeonly{|\textit{dest}|}\input{|\textit{main}|}"|
\end{center}
%

%%%%%%%%%%%%%%%%%%%%%%%%%%%%%%%%%%%%%%%%%%%%%%%%%%%%%%%%%%%%%%%%%%%%%%%%%%%%%%%%
%%%%%%%%%%%%%%%%%%%%%%%%%%%%%%%%%%%%%%%%%%%%%%%%%%%%%%%%%%%%%%%%%%%%%%%%%%%%%%%%
\section{Information}

%%%%%%%%%%%%%%%%%%%%%%%%%%%%%%%%%%%%%%%%%%%%%%%%%%%%%%%%%%%%%%%%%%%%%%%%%%%%%%%%
\subsection{Copyright}

Copyright \copyright{} 2017--2018 Niklas Beisert

This work may be distributed and/or modified under the
conditions of the \LaTeX{} Project Public License, either version 1.3
of this license or (at your option) any later version.
The latest version of this license is in
  \url{http://www.latex-project.org/lppl.txt}
and version 1.3 or later is part of all distributions of \LaTeX{}
version 2005/12/01 or later.

This work has the LPPL maintenance status `maintained'.

The Current Maintainer of this work is Niklas Beisert.

This work consists of the files |README.txt|, |childdoc.ins| and |childdoc.dtx|
as well as the derived files |childdoc.def|, |cdocsamp.tex|
with |cdocsch1.tex|, |cdocsch2.tex|, |cdocspt3.tex|, |cdocspt4.tex|,
|cdocsdrf.tex|, |cdocsfn1.tex|, |cdocsfn2.tex|
as well as |childdoc.pdf|.

%%%%%%%%%%%%%%%%%%%%%%%%%%%%%%%%%%%%%%%%%%%%%%%%%%%%%%%%%%%%%%%%%%%%%%%%%%%%%%%%
\subsection{Files and Installation}

The package consists of the files:
%
\begin{center}
\begin{tabular}{ll}
    |README.txt|   & readme file \\
    |childdoc.ins| & installation file \\
    |childdoc.dtx| & source file \\
    |childdoc.def| & definition file \\
    |cdocsamp.tex| & sample main file \\
    |cdocsch1.tex| & sample include file \\
    |cdocsch2.tex| & sample include file \\
    |cdocspt3.tex| & sample part file \\
    |cdocspt4.tex| & sample part file \\
    |cdocsdrf.tex| & sample redirection file \\
    |cdocsfn1.tex| & sample redirection file \\
    |cdocsfn2.tex| & sample redirection file \\
    |childdoc.pdf| & manual
\end{tabular}
\end{center}
%
The distribution consists of the files
|README.txt|, |childdoc.ins| and |childdoc.dtx|.
%
\begin{itemize}
\item
Run (pdf)\LaTeX{} on |childdoc.dtx|
to compile the manual |childdoc.pdf| (this file).
\item
Run \LaTeX{} on |childdoc.ins| to create the definitions file |childdoc.def|
and the sample |cdocsamp.tex| with include files
|cdocsch1.tex|, |cdocsch2.tex|, |cdocspt3.tex|, |cdocspt4.tex|,
|cdocsdrf.tex|, |cdocsfn1.tex|, |cdocsfn2.tex|.
Then copy the file |childdoc.def| to an appropriate directory of your \LaTeX{}
distribution, e.g.\ \textit{texmf-root}|/tex/latex/childdoc|.
\end{itemize}

%%%%%%%%%%%%%%%%%%%%%%%%%%%%%%%%%%%%%%%%%%%%%%%%%%%%%%%%%%%%%%%%%%%%%%%%%%%%%%%%
\subsection{Related CTAN Packages}

There are several other packages which offer a similar functionality:
%
\begin{itemize}
\item
The packages
\href{http://ctan.org/pkg/docmute}{\textsf{docmute}},
\href{http://ctan.org/pkg/includex}{\textsf{includex}} and
\href{http://ctan.org/pkg/standalone}{\textsf{standalone}}
provide commands to include only the document body of
a child file thus allowing both files to be compiled individually.
\item
The packages \href{http://ctan.org/pkg/subdocs}{\textsf{subdocs}}
and \href{http://ctan.org/pkg/subfiles}{\textsf{subfiles}}
provide structures in which the main and child documents can be
encapsulated and allowing them to be compiled individually.
The inclusion mechanism is different from the conventional |\include|.
\item
The package \href{http://ctan.org/pkg/combine}{\textsf{combine}}
is an elaborate solution to combine several documents into one.
\end{itemize}
%
See also the CTAN topic \href{http://ctan.org/topic/subdocs}{\textsf{subdocs}}
for further related packages.
The present package differs from the above solutions in that
a document structure constructed with the conventional |\include| mechanism
just needs two extra commands at the top of every file
such that all constituent files can be compiled individually.

%%%%%%%%%%%%%%%%%%%%%%%%%%%%%%%%%%%%%%%%%%%%%%%%%%%%%%%%%%%%%%%%%%%%%%%%%%%%%%%%
%\subsection{Feature Suggestions}
%
%The following is a list of features which may be useful for future
%versions of this package:
%%
%\begin{itemize}
%\item
%\ldots
%\end{itemize}

%%%%%%%%%%%%%%%%%%%%%%%%%%%%%%%%%%%%%%%%%%%%%%%%%%%%%%%%%%%%%%%%%%%%%%%%%%%%%%%%
\subsection{Revision History}

%%%%%%%%%%%%%%%%%%%%%%%%%%%%%%%%%%%%%%%%
\paragraph{v2.0:} 2018/12/30

\begin{itemize}
\item
immediate forward processing
\item
added |\childdocby| mechanism
\item
manual restructured
\end{itemize}

%%%%%%%%%%%%%%%%%%%%%%%%%%%%%%%%%%%%%%%%
\paragraph{v1.6:} 2018/01/17

\begin{itemize}
\item
application for development of include files
\item
corrections to manual
\end{itemize}

%%%%%%%%%%%%%%%%%%%%%%%%%%%%%%%%%%%%%%%%
\paragraph{v1.5:} 2017/05/21

\begin{itemize}
\item
more complete structuring introduced
\item
|\childdocof| introduced
\item
|\childdoc| renamed to |\childdocmain|
\item
|\childredirect| renamed to |\childdocforward| and |\childdocforwardprefix|
and functionality expanded
\end{itemize}

%%%%%%%%%%%%%%%%%%%%%%%%%%%%%%%%%%%%%%%%
\paragraph{v1.0:} 2017/04/27

\begin{itemize}
\item
manual and install package
\item
first version published on CTAN
\end{itemize}

%%%%%%%%%%%%%%%%%%%%%%%%%%%%%%%%%%%%%%%%
\paragraph{v0.6:} 2017/04/26

\begin{itemize}
\item
redirection mechanism added
\end{itemize}

%%%%%%%%%%%%%%%%%%%%%%%%%%%%%%%%%%%%%%%%
\paragraph{v0.5:} 2017/04/26

\begin{itemize}
\item
functionality in definition file
\end{itemize}


%%%%%%%%%%%%%%%%%%%%%%%%%%%%%%%%%%%%%%%%%%%%%%%%%%%%%%%%%%%%%%%%%%%%%%%%%%%%%%%%
%%%%%%%%%%%%%%%%%%%%%%%%%%%%%%%%%%%%%%%%%%%%%%%%%%%%%%%%%%%%%%%%%%%%%%%%%%%%%%%%
%%%%%%%%%%%%%%%%%%%%%%%%%%%%%%%%%%%%%%%%%%%%%%%%%%%%%%%%%%%%%%%%%%%%%%%%%%%%%%%%
\appendix

\settowidth\MacroIndent{\rmfamily\scriptsize 000\ }

 \DocInput{childdoc.dtx}

\end{document}
%</driver>
% \fi
%
% %%%%%%%%%%%%%%%%%%%%%%%%%%%%%%%%%%%%%%%%%%%%%%%%%%%%%%%%%%%%%%%%%%%%%%%%%%%%%%
% %%%%%%%%%%%%%%%%%%%%%%%%%%%%%%%%%%%%%%%%%%%%%%%%%%%%%%%%%%%%%%%%%%%%%%%%%%%%%%
% \section{Sample}
%\iffalse
%<*samplemain>
%\fi
%
% The following presents a sample document
% with two chapters, two parts, a title page,
% a compile flag as well as three forwarding files to set the flag.
% It consists of eight |.tex| files:
% \begin{center}
% \begin{tabular}{ll}
% |cdocsamp.tex|&main file\\
% |cdocsch1.tex|&include file for chapter 1\\
% |cdocsch2.tex|&include file for chapter 2\\
% |cdocspt3.tex|&include file for part 3\\
% |cdocspt4.tex|&include file for part 4\\
% |cdocsdrf.tex|&forwarding file for main file in draft mode\\
% |cdocsfi1.tex|&forwarding file for final version of chapter 1\\
% |cdocsfi2.tex|&forwarding file for final version of chapter 2\\
% \end{tabular}
% \end{center}
% Each of the eight files can be compiled directly by the \LaTeX{} compiler.
%
% %%%%%%%%%%%%%%%%%%%%%%%%%%%%%%%%%%%%%%
% \paragraph{Main File.}
%
% The main file is called |cdocsamp.tex|.
%
% Load the \textsf{childdoc} definitions and
% declare the filename for the main document:
%    \begin{macrocode}
\input{childdoc.def}
\childdocmain{}
%    \end{macrocode}

% Optional override for |\version| flag:
%    \begin{macrocode}
%%\ifchilddoc\else\providecommand{\version}{draft}\fi
%    \end{macrocode}

% Define the default values for the |\version| flag
% (|final| for the main file and |draft| for childs):
%    \begin{macrocode}
\ifchilddoc
\providecommand{\version}{draft}
\else
\providecommand{\version}{final}
\fi
%    \end{macrocode}

% Load the standard document class:
%    \begin{macrocode}
\documentclass[12pt]{article}
%    \end{macrocode}

% Start the document body:
%    \begin{macrocode}
\begin{document}
%    \end{macrocode}

% Declare a title page.
% Print title, part of document being processed and version flag:
%    \begin{macrocode}
\addtocounter{page}{-1}
\begin{center}
{\LARGE\bfseries{}childdoc example\par}
\vspace{1cm}
\ifchilddoc
\ifchilddocmanual part\else chapter\fi:
`\childdocname' of `\childdocjob'\par
\else
main document: `\childdocjob'\par
\fi
version: \version\par
\end{center}
\newpage
%    \end{macrocode}

% Manually include selected file,
% otherwise process as usual:
%    \begin{macrocode}
\ifchilddocmanual
\section*{part `\childdocname'}
\input{\childdocname}
\else
%    \end{macrocode}

% Include the two chapters:
%    \begin{macrocode}
\include{cdocsch1}
\include{cdocsch2}
%    \end{macrocode}

% Include the two parts unless only chapters should be displayed:
%    \begin{macrocode}
\ifchilddoc\else
\section{part three}
\input{cdocspt3}
\section{part four}
\input{cdocspt4}
\fi
%    \end{macrocode}

% Process as usual until here:
%    \begin{macrocode}
\fi
%    \end{macrocode}

% End of document body:
%    \begin{macrocode}
\end{document}
%    \end{macrocode}
%\iffalse
%</samplemain>
%\fi
%
% %%%%%%%%%%%%%%%%%%%%%%%%%%%%%%%%%%%%%%
% \paragraph{Chapter Include Files.}
%
% The include files are called |cdocsch1.tex| and |cdocsch2.tex|.
%
%\iffalse
%<*samplechap1|samplechap2>
%\fi

% Optional override for |\version| flag:
%    \begin{macrocode}
%%\providecommand{\version}{final}
%    \end{macrocode}

% Include the main document:
%    \begin{macrocode}
\input{childdoc.def}
\childdocof{cdocsamp}
%    \end{macrocode}

%\iffalse
%</samplechap1|samplechap2>
%\fi
%
%\iffalse
%<*samplechap1>
%\fi
% Some text for chapter 1:
%    \begin{macrocode}
\section{one}
some text in chapter one
%    \end{macrocode}

%\iffalse
%</samplechap1>
%\fi
% Some text for chapter 2:
%\iffalse
%<*samplechap2>
%\fi
%    \begin{macrocode}
\section{two}
more text in chapter two
%    \end{macrocode}

%\iffalse
%</samplechap2>
%\fi
%
% %%%%%%%%%%%%%%%%%%%%%%%%%%%%%%%%%%%%%%
% \paragraph{Part Include Files.}
%
% The include files are called |cdocspt3.tex| and |cdocspt4.tex|.
%
%\iffalse
%<*samplepart3|samplepart4>
%\fi

% Optional override for |\version| flag:
%    \begin{macrocode}
%%\providecommand{\version}{final}
%    \end{macrocode}

% Include the main document:
%    \begin{macrocode}
\input{childdoc.def}
\childdocby{cdocsamp}
%    \end{macrocode}

%\iffalse
%</samplepart3|samplepart4>
%\fi
%
%\iffalse
%<*samplepart3>
%\fi
% Some text for part 3:
%    \begin{macrocode}
some text in part three
%    \end{macrocode}

%\iffalse
%</samplepart3>
%\fi
% Some text for part 4:
%\iffalse
%<*samplepart4>
%\fi
%    \begin{macrocode}
more text in part four
%    \end{macrocode}

%\iffalse
%</samplepart4>
%\fi
%
% %%%%%%%%%%%%%%%%%%%%%%%%%%%%%%%%%%%%%%
% \paragraph{Forwarding for a Complete Draft.}
%
% The following forwarding file |cdocsdrf.tex|
% compiles the main document in draft mode:
%\iffalse
%<*sampledraft>
%\fi
%    \begin{macrocode}
\def\version{draft}
\input{childdoc.def}
\childdocforward{cdocsamp}
%    \end{macrocode}

%\iffalse
%</sampledraft>
%\fi
%
% %%%%%%%%%%%%%%%%%%%%%%%%%%%%%%%%%%%%%%
% \paragraph{Forwarding for Final Version of the Chapters.}
%
% The following forwarding files |cdocsfn1.tex| and |cdocsfn2.tex|
% (with identical content)
% compile the final versions of the child documents
% |cdocsch1.tex| and |cdocsch2.tex|, respectively:
%\iffalse
%<*samplefinal>
%\fi
%    \begin{macrocode}
\def\version{final}
\input{childdoc.def}
\childdocforwardprefix[cdocsamp]{cdocsfn}{cdocsch}
%    \end{macrocode}

%\iffalse
%</samplefinal>
%\fi
%
% %%%%%%%%%%%%%%%%%%%%%%%%%%%%%%%%%%%%%%
% \paragraph{Command Line Processing.}
%
% The following three command lines generate the output files
% |cdocscld|, |cdocscl1| and |cdocscl2|
% which should be identical to
% |cdocsdrf|, |cdocsch1| and |cdocsfn2|, respectively:
% \begin{center}
% \begin{tabular}{l}
% |latex -jobname cdocscld \|\\
% |  "\def\version{draft}\input{childdoc.def}\childdocforward{cdocsamp}"|\\
% |latex -jobname cdocscl1 \|\\
% |  "\input{childdoc.def}\childdocforward[cdocsamp]{cdocsch1}"|\\
% |latex -jobname cdocscl2 \|\\
% |  "\def\version{final}\input{childdoc.def}\childdocforward{cdocsch2}"|
% \end{tabular}
% \end{center}
% Note that the trailing backslash on each first line
% merely continues the input to the second line
% (for convenient cut ant paste).
% Furthermore, the command |latex| can be replaced by any
% of its alternative versions such as |pdflatex|.
%
% %%%%%%%%%%%%%%%%%%%%%%%%%%%%%%%%%%%%%%%%%%%%%%%%%%%%%%%%%%%%%%%%%%%%%%%%%%%%%%
% %%%%%%%%%%%%%%%%%%%%%%%%%%%%%%%%%%%%%%%%%%%%%%%%%%%%%%%%%%%%%%%%%%%%%%%%%%%%%%
% \section{Implementation}
%\iffalse
%<*package>
%\fi
%
% This section describes the definitions file |childdoc.def|.

% The definitions cannot be loaded using |\usepackage| or |\RequirePackage|
% which has a mechanism to prevent loading a style file more than once.
% When loading the definitions by means of |\input|
% multiple instances have to be prevented manually:
%\iffalse
%This code needs to be before the `\ProvidesFile' directive
%which is defined at the beginning of this file.
%Therefore it is also placed there and commented out here.
%</package>
%<*discard>
%\fi
%    \begin{macrocode}
\ifdefined\childdocmain\endinput\fi
%    \end{macrocode}
%\iffalse
%</discard>
%<*package>
%\fi
%
% \macro{\ifchilddoc}
% \macro{\ifchilddocmanual}
% The conditional |\ifchilddoc| tells whether a
% child (true) or main (false) document is being compiled.
% The conditional |\ifchilddocmanual| tells whether
% the |\includeonly| mechanism is used (false) or
% the selection of child files must be performed manually (true).
% The definitions initialise to false:
%    \begin{macrocode}
\newif\ifchilddoc
\newif\ifchilddocmanual
%    \end{macrocode}

% \macro{\childdocname}
% \macro{\childdocjob}
% The macro |\childdocname| stores the name of the main document
% to be compiled. The macro |\childdocjob| stores the name of
% the document on which the \LaTeX{} compiler was originally invoked.
% The content of |\jobname| cannot be compared
% to filenames specified in the source due to different catcodes.
% The following code rescans |\jobname|, stores the result
% in |\childdocname| and saves a copy in |\childdocjob|:
%    \begin{macrocode}
\edef\childdocname{\scantokens\expandafter{\jobname\noexpand}}
\let\childdocjob\childdocname
%    \end{macrocode}

% \macro{\childdocdisable}
% The macro |\childdocdisable| prevents the main file
% from being processed more than once.
% At this stage, the main document command |\childdocmain|
% is assumed to be called once again where it should do nothing.
% Any subsequent call to it should prevent
% a secondary processing of the main document
% It overwrites the forwarding commands
% |\childdocof| and |\childdocforward|
% with empty macros to prevent further inclusions of the main document:
%    \begin{macrocode}
\newcommand{\childdocdisable}
{
  \renewcommand{\childdocmain}[1]{\renewcommand{\childdocmain}[1]{\endinput}}
  \renewcommand{\childdocof}[1]{}
  \renewcommand{\childdocby}[2][]{}
  \renewcommand{\childdocforward}[2][]{}
  \renewcommand{\childdocdisable}{}
}
%    \end{macrocode}

% \macro{\childdocmain}
% The macro |\childdocmain| is to be called at the top of the main file
% with nothing or the main filename (without extension) as argument.
% First, it breaks loops.
% If the argument is not empty and does not match |\childdocname|
% (which is set by the first inclusion of |childdoc.def|),
% |\ifchilddoc| is set to true, |\includeonly| is applied to the child file
% and |\jobname| is set to the main file
% (for proper handling of |.aux| files):
%    \begin{macrocode}
\newcommand{\childdocmain}[1]
{
  \childdocdisable\childdocmain{}
  \if?#1?\else
    \begingroup
      \def\childdoctmp{#1}
      \ifx\childdoctmp\childdocname
        \def\childdoctmp{}
      \else
        \def\childdoctmp
        {
          \childdoctrue
          \includeonly{\childdocname}
          \def\childdocjob{#1}
          \def\jobname{#1}
        }
      \fi
      \expandafter
    \endgroup
    \childdoctmp
  \fi
}
%    \end{macrocode}

% \macro{\childdocof}
% The command |\childdocof| redirects
% compilation to the main file |#1|.
%    \begin{macrocode}
\newcommand{\childdocof}[1]
{
  \childdocdisable
  \childdoctrue
  \includeonly{\childdocname}
  \def\jobname{#1}
  \def\childdocjob{#1}
  \input{#1}
}
%    \end{macrocode}

% \macro{\childdocby}
% The command |\childdocby| ....
%    \begin{macrocode}
\newcommand{\childdocby}[2][]
{
  \childdocdisable
  \childdoctrue
  \childdocmanualtrue
  \if?#1?\else
    \def\jobname{#2}
  \fi
  \def\childdocjob{#2}
  \input{#2}
  \endinput
}
%    \end{macrocode}

% \macro{\childdocforward}
% The command |\childdocforward| redirects
% compilation to the main file or
% (if the optional argument is given) a child file.
% Parameters are set as if the main file
% or a child file starting with |\childdocof| was compiled.
% Then compilation is handed over to the main file:
%    \begin{macrocode}
\newcommand{\childdocforward}[2][]
{
  \begingroup
    \if?#1?
      \def\childdoctmp
      {
        \def\childdocname{#2}
        \def\childdocjob{#2}
        \def\jobname{#2}
        \input{#2}
        \endinput
      }
    \else
      \def\childdoctmp
      {
        \childdocdisable
        \def\childdocname{#2}
        \childdoctrue
        \includeonly{#2}
        \def\childdocjob{#1}
        \def\jobname{#1}
        \input{#1}
        \endinput
      }
    \fi
    \expandafter
  \endgroup
  \childdoctmp
}
%    \end{macrocode}

% \macro{\childdocforwardprefix}
% The command |\childdocforwardprefix| redirects
% compilation to the main or a child file by means of a pattern.
% The prefix |#1| in the current filename is replaced by |#2|
% and the suffix of the current filename is kept
% (it is assumed that the filename does not contain the substring `|~~~|'
% which is used as a delimiter).
% Compilation is handed over to the new file by |\childdocforward|:
%    \begin{macrocode}
\newcommand{\childdocforwardprefix}[3][]
{
  \begingroup
    \def\childdocextract #2##1~~~{\def\childdoctmp{\childdocforward[#1]{#3##1}}}
    \expandafter\childdocextract\childdocname~~~
    \expandafter
  \endgroup
  \childdoctmp
}
%    \end{macrocode}

% \macro{\childdoc}
% The deprecated macro |\childdoc| is a legacy version of |\childdocmain|:
%    \begin{macrocode}
\newcommand{\childdoc}{\childdocmain}
%    \end{macrocode}

% \macro{\childdocredirect}
% The deprecated macro |\childdocredirect| is a legacy version
% of |\childdocforward| and |\childdocforwardprefix|:
%    \begin{macrocode}
\newcommand{\childdocredirect}[2][]
{
  \begingroup
    \if?#1?
      \def\childdoctmp{\childdocforward{#2}}
    \else
      \def\childdoctmp{\childdocforwardprefix{#1}{#2}}
    \fi
    \expandafter
  \endgroup
  \childdoctmp
}
%    \end{macrocode}

%\iffalse
%</package>
%\fi
%
\endinput
|\\
|\childdocforward{|\textit{main}|}|
\end{tabular}
\end{center}
%
Likewise, the following files |final|\textit{nn}|.tex|
compile the final version of the child document
|child|\textit{nn}|.tex|:
%
\begin{center}
\begin{tabular}{l}
|\def\version{final}|\\
|% \iffalse
%
% childdoc.dtx Copyright (C) 2017-2018 Niklas Beisert
%
% This work may be distributed and/or modified under the
% conditions of the LaTeX Project Public License, either version 1.3
% of this license or (at your option) any later version.
% The latest version of this license is in
%   http://www.latex-project.org/lppl.txt
% and version 1.3 or later is part of all distributions of LaTeX
% version 2005/12/01 or later.
%
% This work has the LPPL maintenance status `maintained'.
%
% The Current Maintainer of this work is Niklas Beisert.
%
% This work consists of the files childdoc.dtx and childdoc.ins
% and the derived files childdoc.def and cdocsamp.tex with
% cdocsch1.tex, cdocsch2.tex, cdocsdrf.tex, cdocsfn1.tex, cdocsfn2.tex.
%
%<package>\ifdefined\childdocmain\endinput\fi
%<package>\ProvidesFile{childdoc.def}[2018/12/30 v2.0 child document driver]
%<samplemain>\ProvidesFile{cdocsamp.tex}[2018/12/30 v2.0 sample for childdoc]
%<*driver>
%\ProvidesFile{childdoc.drv}[2018/12/30 v2.0 childdoc reference manual file]
\PassOptionsToClass{10pt,a4paper}{article}
\documentclass{ltxdoc}

\usepackage[margin=35mm]{geometry}
\usepackage{hyperref}
\usepackage{hyperxmp}
\usepackage[usenames]{color}

\hypersetup{colorlinks=true}
\hypersetup{pdfstartview=FitH}
\hypersetup{pdfpagemode=UseNone}
\hypersetup{pdfsource={}}
\hypersetup{pdflang={en-UK}}
\hypersetup{pdfcopyright={Copyright 2017-2018 Niklas Beisert.
  This work may be distributed and/or modified under the
  conditions of the LaTeX Project Public License, either version 1.3
  of this license or (at your option) any later version.}}
\hypersetup{pdflicenseurl={http://www.latex-project.org/lppl.txt}}
\hypersetup{pdfcontactaddress={ETH Zurich, ITP, HIT K,
  Wolfgang-Pauli-Strasse 27}}
\hypersetup{pdfcontactpostcode={8093}}
\hypersetup{pdfcontactcity={Zurich}}
\hypersetup{pdfcontactcountry={Switzerland}}
\hypersetup{pdfcontactemail={nbeisert@itp.phys.ethz.ch}}
\hypersetup{pdfcontacturl={http://people.phys.ethz.ch/\xmptilde nbeisert/}}

\newcommand{\secref}[1]{\hyperref[#1]{section \ref*{#1}}}

\parskip1ex
\parindent0pt
\let\olditemize\itemize
\def\itemize{\olditemize\parskip0pt}

\begin{document}

\title{The \textsf{childdoc} Package}
\hypersetup{pdftitle={The childdoc Package}}
\author{Niklas Beisert\\[2ex]
  Institut f\"ur Theoretische Physik\\
  Eidgen\"ossische Technische Hochschule Z\"urich\\
  Wolfgang-Pauli-Strasse 27, 8093 Z\"urich, Switzerland\\[1ex]
  \href{mailto:nbeisert@itp.phys.ethz.ch}
  {\texttt{nbeisert@itp.phys.ethz.ch}}}
\hypersetup{pdfauthor={Niklas Beisert}}
\hypersetup{pdfsubject={Manual for the LaTeX2e Package childdoc}}
\date{30 December 2018, \textsf{v2.0}}
\maketitle

\begin{abstract}\noindent
\textsf{childdoc} is a \LaTeXe{} package
that enables the direct compilation
of document sections included by |\include|
to individual files.
\end{abstract}

\begingroup
\parskip0ex
\tableofcontents
\endgroup

%%%%%%%%%%%%%%%%%%%%%%%%%%%%%%%%%%%%%%%%%%%%%%%%%%%%%%%%%%%%%%%%%%%%%%%%%%%%%%%%
%%%%%%%%%%%%%%%%%%%%%%%%%%%%%%%%%%%%%%%%%%%%%%%%%%%%%%%%%%%%%%%%%%%%%%%%%%%%%%%%
\section{Introduction}

\LaTeX{} provides a mechanism to structure a large document (such as a book)
into a main file and several child files (containing the chapters)
using the |\include| command.
This mechanism is beneficial for documents
which span hundreds of pages in order to
make the source file(s) more manageable.
Moreover, compilation can be restricted to
selected child files by means of the |\includeonly| command.
The latter feature can be used to reduce the compilation time while editing
(this was significantly more useful in the earlier days of \LaTeX{})
or to generate a smaller document which is easier to navigate.
Another application of |\includeonly| is to generate
documents consisting of selected parts of the complete document.

However, there are a few drawbacks of the plain |\include| mechanism:
\begin{itemize}
\item
The child files cannot be compiled on their own,
they can only be compiled via the main file.
A naive editing environment
(such as a text editor with an option
to have the current file processed by \LaTeX)
may require one to switch to the main file before compiling;
attempting to compile the child file produces errors.
\item
The main file must be modified (each time)
to adjust the |\includeonly| command
to the present needs. This easily leaves the main file in a messy state.
\item
The generated document will always carry the filename
of the main document. This is inconvenient if
several child files are to be compiled and
to be kept for distribution.
\end{itemize}

The present package provides a simple interface
to make child files individually compilable by \LaTeX{}.
Compiling a child file then has the same effect as compiling
the main file with an |\includeonly| command
to select the appropriate child.
Moreover the generated document will carry the name of the child
rather than the main file.
This resolves all three above issues.

This feature is meant to make the editing of books,
thesis documents and lecture notes somewhat more convenient.
However, the package can also be used efficiently for
composing a series of documents (such as exercise sheets)
which are typically distributed individually.
It then assists the author in generating the individual documents
(potentially in different versions)
as well as a document containing the collected series.
Another application is in developing style files
or other kinds of included material
where compilation of the style file could redirect
to a sample or test file.

%%%%%%%%%%%%%%%%%%%%%%%%%%%%%%%%%%%%%%%%%%%%%%%%%%%%%%%%%%%%%%%%%%%%%%%%%%%%%%%%
%%%%%%%%%%%%%%%%%%%%%%%%%%%%%%%%%%%%%%%%%%%%%%%%%%%%%%%%%%%%%%%%%%%%%%%%%%%%%%%%
\section{Usage}

First of all, the package \textsf{childdoc} is \emph{not} a standard
\LaTeXe{} |.sty| style file! Therefore it needs to be invoked in
a non-standard way.

%%%%%%%%%%%%%%%%%%%%%%%%%%%%%%%%%%%%%%%%%%%%%%%%%%%%%%%%%%%%%%%%%%%%%%%%%%%%%%%%
\subsection{Included Files}
\label{sec:include}

%%%%%%%%%%%%%%%%%%%%%%%%%%%%%%%%%%%%%%%%
\DescribeMacro{\childdocmain}
To use the package, add the commands
\begin{center}
\begin{tabular}{l}
|\input{childdoc.def}|\\
|\childdocmain{}|\\
\end{tabular}
\end{center}
at the very top of the main \LaTeX{} file,
in particular \emph{before} the |\documentclass| statement!
The argument of |\childdocmain| should be left empty
(but it must be present).

%%%%%%%%%%%%%%%%%%%%%%%%%%%%%%%%%%%%%%%%
\DescribeMacro{\childdocof}
Furthermore, add the commands
\begin{center}
\begin{tabular}{l}
|\input{childdoc.def}|\\
|\childdocof{|\textit{main}|}|\\
\end{tabular}
\end{center}
at the top of every child file \textit{child}
which is included by |\include{|\textit{child}|}|
from within the main file
(or at least for those files to be compiled individually).
The argument \textit{main} must be the filename of the main file.

There are a couple of
considerations in setting up the main and child documents:

%%%%%%%%%%%%%%%%%%%%%%%%%%%%%%%%%%%%%%%%
\paragraph{Restrictions.}

Please note the following restrictions:
\begin{itemize}
\item
|\childdocmain| must be called with one argument \textit{main}
to ensure compatibility with earlier version of the package.
It must either be empty (|\childdocmain{}|)
or precisely match the filename of the main file in which it is specified.
See \secref{sec:detection} for further information.
\item
The filename \textit{main} must be specified without the |.tex| extension.
\item
The filename \textit{main} is case sensitive
(even in case-insensitive file systems)
due to internal string comparison.
\item
The argument \textit{main} should be fully expanded, it cannot be a macro.
\item
Subdirectories and special characters should be avoided in filenames.
\item
The command |\childdocmain{|\textit{main}|}| must be followed by a whitespace.
It should not be followed immediately by another command
or by a comment mark `|%|'.
This is because the \TeX{} parser reads the token immediately following
the argument of |\childdocmain| and puts it
at the beginning of every child section;
however, a white\-space is ignored.
\end{itemize}

%%%%%%%%%%%%%%%%%%%%%%%%%%%%%%%%%%%%%%%%
\paragraph{Content of Main File.}

It is advisable to place all content in the child files included by |\include|.
Any output contained in the main file will appear in all child documents
unless suppressed manually;
it cannot be suppressed automatically by the |\includeonly| directive
and thus should normally be avoided.
A method to include some content in the main file
by means of conditional processing is described in \secref{sec:conditional}.

%%%%%%%%%%%%%%%%%%%%%%%%%%%%%%%%%%%%%%%%
\paragraph{Page Numbering.}

When only a part of the document is compiled,
the appropriate numbering of pages
(as well as other status parameters)
is determined from the |.aux| files.
The latter contain information from previous passes.
However this information needs to propagate through
all intermediate child documents.
Therefore the page numbering in child documents may well
be inconsistent until the complete document is compiled at least once.

A useful (if unconventional) way to always ensure a consistent
page numbering is to restart the numbering in each child document
and denote the pages by `\textit{child}|.|\textit{page}'
where \textit{child} represents the chapter/section number of the child file.
This can be achieved by the command
|\numberwithin{page}{|\textit{child}|}|
of the \textsf{amsmath} package
where \textit{child} can be |chapter| or |section|
depending on the chosen structuring.
Alternatively, one can modify the macro |\thepage| appropriately
and reset the counter |page| at the start of each child file.

%%%%%%%%%%%%%%%%%%%%%%%%%%%%%%%%%%%%%%%%%%%%%%%%%%%%%%%%%%%%%%%%%%%%%%%%%%%%%%%%
\subsection{Conditional Processing}
\label{sec:conditional}

The package provides a mechanism to compile different versions
of a document. To customise the versions further some conditional processing
can come in handy to distinguish which version is being compiled.
The package provides two macros to describe the compilation context:

%%%%%%%%%%%%%%%%%%%%%%%%%%%%%%%%%%%%%%%%
\DescribeMacro{\ifchilddoc}
The conditional |\ifchilddoc| distinguishes between the compilation of
child documents and the main document:
%
\begin{center}
|\ifchilddoc |\textit{child-code}| |[|\||else |\textit{main-code}]| \||fi|
\end{center}

%%%%%%%%%%%%%%%%%%%%%%%%%%%%%%%%%%%%%%%%
\DescribeMacro{\childdocname}
\DescribeMacro{\childdocjob}
The macro |\childdocname| contains the filename (without extension)
of the main or child file being processed.
Note that |\childdocjob| will always contain the name of the main file.

%%%%%%%%%%%%%%%%%%%%%%%%%%%%%%%%%%%%%%%%
\paragraph{Title Page.}

Conditional processing can be used to include a title or banner page
in the main document when proper precautions are taken.
Importantly, the code in the main file should ensure that the page counter
(as well as other status parameters which are stored in the |.aux| files)
takes the same value after the conditional processing.
Otherwise the page numbers may take divergent values
depending on which part is compiled.

For example, a title page could be declared by:
%
\begin{center}
\begin{tabular}{l}
|\ifchilddoc\||else|\\
|\addtocounter{page}{-1}|\\
\textit{code for title page}\\
|\newpage|\\
|\||fi|
\end{tabular}
\end{center}
%
A banner page for the child documents can be generated by:
%
\begin{center}
\begin{tabular}{l}
|\ifchilddoc|\\
|\addtocounter{page}{-1}|\\
\textit{code for banner page}\\
|\newpage|\\
|\||fi|
\end{tabular}
\end{center}
%
Here one could write a message such as:
\begin{center}
|This is the part \childdocname{} of \childdocjob{}.|
\end{center}

%%%%%%%%%%%%%%%%%%%%%%%%%%%%%%%%%%%%%%%%%%%%%%%%%%%%%%%%%%%%%%%%%%%%%%%%%%%%%%%%
\subsection{Flags}
\label{sec:flags}

The package makes it easy to generate different versions
of the main or child documents.
To this end compilation flags can be defined
and assigned different default values.
They will be particularly useful in conjunction
with the forwarding mechanism described in \secref{sec:forward}.

For example, it may be useful to have a flag |\version|
which can be set to |draft| or |final|.
The document source will contain some conditional code
depending on the value of |\version|.
Suppose further, the flag should default to |final| for the main file
and to |draft| for child files
which is a natural assignment for editing the document.
This is achieved by placing the following code
in the preamble of the main document
(below the |\childdocmain| directive):
%
\begin{center}
\begin{tabular}{l}
|\ifchilddoc|\\
|\providecommand{\version}{draft}|\\
|\||else|\\
|\providecommand{\version}{final}|\\
|\||fi|
\end{tabular}
\end{center}
%
The definition by |\providecommand| makes sure
that previous definitions are not overwritten.
Further statements |\providecommand{\version}{...}|
can thus be added before the above code to override it.

For the main file, one might add a line
(between |\childdocmain| and the above block)
%
\begin{center}
|%\ifchilddoc\||else\providecommand{\version}{draft}\||fi|
\end{center}
%
which can be uncommented to produce a draft version.
Likewise one can add a line to the very top of a child file
(above the |\childdocof{|\textit{main}|}| directive)
%
\begin{center}
|%\providecommand{\version}{final}|
\end{center}
%
which can be uncommented to produce the final version of this child document.

%%%%%%%%%%%%%%%%%%%%%%%%%%%%%%%%%%%%%%%%%%%%%%%%%%%%%%%%%%%%%%%%%%%%%%%%%%%%%%%%
\subsection{Forwarding}
\label{sec:forward}

Different versions of the main or child documents
using compilation flags as described in \secref{sec:flags}
can be (permanently) stored in different files
for convenient compilation, viewing and distribution.
To this end, the package defines a command
to pass on compilation to a different file:

%%%%%%%%%%%%%%%%%%%%%%%%%%%%%%%%%%%%%%%%
\DescribeMacro{\childdocforward}
The command |\childdocforward| redirects processing to
another source file:
%
\begin{center}
\begin{tabular}{l}
|\input{childdoc.def}|\\
|\childdocforward[|\textit{main}|]{|\textit{dest}|}|\\
\end{tabular}
\end{center}
%
The argument \textit{dest} is the destination file
(without extension).
It should be the main file or one of the child files.
Note that further \textsf{childdoc} directives
such as |\childdocof| and |\childdocforward|
in the indicated file will be processed in this form.
The optional argument \textit{main}
passes on directly to the main file \textit{main}
while pretending to compile the child \textit{dest}.
This form behaves as if \textit{dest}
issues |\childdocof{|\textit{main}|}| right away,
and no further \textsf{childdoc} directives will be processed.

%%%%%%%%%%%%%%%%%%%%%%%%%%%%%%%%%%%%%%%%
\DescribeMacro{\...prefix}
In the alternative form |\childdocforwardprefix|,
%
\begin{center}
\begin{tabular}{l}
|\input{childdoc.def}|\\
|\childdocforwardprefix[|\textit{main}|]{|\textit{prefix}|}{|\textit{dest}|}|
\end{tabular}
\end{center}
%
the destination file is determined by a pattern
depending on the current file:
To make this work, the current file must be called
`{\textit{prefix}\hspace{0.2em}\textit{suffix}}'
with \textit{prefix} matching precisely the argument.
Processing is then passed on to the file
`{\textit{dest}\hspace{0.2em}\textit{suffix}}'.
Surely, the same effect is achieved by
directly specifying the
argument `{\textit{dest}\hspace{0.2em}\textit{suffix}}'
in the first form.
However, that requires to set up a different file
for each child. With the alternative form of the command
all these files can have exactly the same content
which simplifies setting them up and maintaining them.

For example, the following file |draft.tex|
with a compilation flag |\version| as described in \secref{sec:flags}
compiles the main document as a draft:
%
\begin{center}
\begin{tabular}{l}
|\def\version{draft}|\\
|\input{childdoc.def}|\\
|\childdocforward{|\textit{main}|}|
\end{tabular}
\end{center}
%
Likewise, the following files |final|\textit{nn}|.tex|
compile the final version of the child document
|child|\textit{nn}|.tex|:
%
\begin{center}
\begin{tabular}{l}
|\def\version{final}|\\
|\input{childdoc.def}|\\
|\childdocforwardprefix{final}{child}|
\end{tabular}
\end{center}
%

Note that when several versions of a main file and/or of each child file
are to be generated, it may be convenient to set up a |Makefile| or
shell script to automatise the process.

%%%%%%%%%%%%%%%%%%%%%%%%%%%%%%%%%%%%%%%%%%%%%%%%%%%%%%%%%%%%%%%%%%%%%%%%%%%%%%%%
\subsection{Command Line Processing}
\label{sec:commandline}

The effect of redirection files can also be achieved by invoking
the \LaTeX{} compiler with a more elaborate command line.
Most conveniently this should be done as part
of a shell script or a |Makefile|.

When using \textsf{childdoc} in the main file, the following
command lines effectively perform a redirection
(note that depending on the shell being used,
backslashes may have to be doubled: `|\|' $\to$ `|\\|'):
%
\begin{center}
|... -jobname "|\textit{target}|" |\\|"|[\textit{flags}]%
|\input{childdoc.def}\childdocforward[|\textit{main}|]{|\textit{dest}|}"|
\end{center}
%
Here \textit{target} is the name of the output file,
\textit{main} is the name of the main file
and \textit{dest} is the name of the main or child file to be processed
(all filenames without extensions).
The optional argument \textit{main} can be omitted
if \textit{main} matches \textit{dest}.
Optionally, compilation \textit{flags} can be defined via |\def| commands.
This command line makes the \TeX{} engine believe
it is compiling the file \textit{target}
whose content is specified as the latter parameter.
The provided code then forwards the processing to
\textit{main} or \textit{dest} as described in \secref{sec:forward}.

%%%%%%%%%%%%%%%%%%%%%%%%%%%%%%%%%%%%%%%%%%%%%%%%%%%%%%%%%%%%%%%%%%%%%%%%%%%%%%%%
\subsection{Include by Input}
\label{sec:input}

Including child documents by |\include| has some restrictions by design.
Most notably, the content of a child document always occupies
its own set of pages; pages cannot be shared between child documents.
Usually, this behaviour makes perfect sense
because each child document contain an essential part of the document.
However, in some situations it may be desirable to compose
a document from a collection of parts
without having mandatory page breaks between then.
For this case, the package
provides a mechanism to include parts
by |\input| which can also be processed individually.
However, by construction this mechanism
requires manual handling of the content to be output.

%%%%%%%%%%%%%%%%%%%%%%%%%%%%%%%%%%%%%%%%
\DescribeMacro{\ifchilddocmanual}
The main file should be prepared as usual, see \secref{sec:include}.
However, the document body must make a distinction
between processing of an individual part and of the main document, e.g.:
%
\begin{center}
\begin{tabular}{l}
|\ifchilddocmanual|\\
|\input{\childdocname}|\\
|\||else|\\
\textit{document body with }|\input{|\textit{part}|}|\\
|\||fi|
\end{tabular}
\end{center}
%
The conditional |\ifchilddocmanual| is true whenever
a part to be included by |\input| is being compiled,
and the name of the part is stored in |\childdocname|.

%%%%%%%%%%%%%%%%%%%%%%%%%%%%%%%%%%%%%%%%
\DescribeMacro{\childdocby}
Each part to be included by |\input| should start with:
%
\begin{center}
\begin{tabular}{l}
|\input{childdoc.def}|\\
|\childdocby{|\textit{main}|}|\\
\end{tabular}
\end{center}
%
The directive |\childdocby| is similar to |\childdocof|
described in \secref{sec:include},
but the subsequent selection of content must be done manually.
To that end, both |\ifchilddoc| and |\ifchilddocmanual|
will be true upon processing of a part,
and the name of the part is stored in |\childdocname|.
Note that |\jobname| will be set to the filename of the current part
so that each part receives an individual |.aux| file
that does not interfere with the |.aux| file(s) of the main document.
This behaviour can be altered by the alternative form
|\childdocby[*]{|\textit{main}|}| (with a non-empty optional argument)
which uses the |.aux| file of the main document
by setting |\jobname| to \textit{main}.

%%%%%%%%%%%%%%%%%%%%%%%%%%%%%%%%%%%%%%%%%%%%%%%%%%%%%%%%%%%%%%%%%%%%%%%%%%%%%%%%
\subsection{Driver Development}
\label{sec:driver}

The \textsf{childdoc} mechanism can also be use for the development
of definition files such as \LaTeX{} styles or classes.
This case differs from the above setup with multiple parts
included by |\include| in that no |\includeonly| should be invoked.
This can be achieved by starting the include file
(before |\ProvidesPackage|) with:
%
\begin{center}
\begin{tabular}{l}
|\input{childdoc.def}|\\
|\childdocforward{|\textit{main}|}|\\
\end{tabular}
\end{center}
%
or alternatively with:
%
\begin{center}
\begin{tabular}{l}
|\input{childdoc.def}|\\
|\childdocby{|\textit{main}|}|\\
\end{tabular}
\end{center}
%
Both forms have slightly different effects as described above.
The main file is prepared as usual, see \secref{sec:include}.

%%%%%%%%%%%%%%%%%%%%%%%%%%%%%%%%%%%%%%%%%%%%%%%%%%%%%%%%%%%%%%%%%%%%%%%%%%%%%%%%
\subsection{Legacy Detection}
\label{sec:detection}

The directive |\childdocmain| in the main file can detect
whether the complete document or merely a child is to be compiled
even without using the directive |\childdocof|.
This method is deprecated because it is less robust
and there is no compelling reason to use it;
it is merely provided for backward compatibility
and it may be removed in future versions.

If the detection mechanism is to be used,
it is mandatory to correctly specify
the filename of the main file as the argument of |\childdocmain|:
%
\begin{center}
\begin{tabular}{l}
|\input{childdoc.def}|\\
|\childdocmain{|\textit{main}|}|\\
\end{tabular}
\end{center}
%
If |\jobname| does not match the argument \textit{main} of |\childdocmain|,
it is assumed that |\jobname| points to the child file to be compiled.
When using |\childdocmain| with the main file specified as argument,
it suffices to start a child file
with just |\input{|\textit{main}|}|
without loading of the package and using |\childdocof|.
If instead all processing is done
with the appropriate \textsf{childdoc} directives,
the argument of \textit{main} of |\childdocmain| can be empty.

An alternative version of the command line processing described
in \secref{sec:commandline} using the detection mechanism reads:
%
\begin{center}
|... -jobname "|\textit{target}|" "|[\textit{flags}]%
[|\def\jobname{|\textit{dest}|}|]|\input{|\textit{main}|}"|
\end{center}

%%%%%%%%%%%%%%%%%%%%%%%%%%%%%%%%%%%%%%%%%%%%%%%%%%%%%%%%%%%%%%%%%%%%%%%%%%%%%%%%
\subsection{Manual Code}
\label{sec:manual}

In case one cannot be certain whether the definitions file |childdoc.def|
is installed on the target \TeX{} distribution
and one prefers not to ship it,
it is conceivable to paste a few relevant commands into the sources.

To that end, drop all statements |\input{childdoc.def}|
and perform the replacements as outlined below.
Instead of |\childdocmain{|\textit{main}|}| add the following code
to the top of the main file:
%
\begin{center}
\begin{tabular}{l}
|\||ifdefined\childdocname\endinput\||fi\newif\ifchilddoc|\\
|\edef\childdocname{\scantokens\expandafter{\jobname\noexpand}}|\\
|\def\childdocmain{|\textit{main}|}\||ifx\childdocmain\childdocname\||else|\\
|\childdoctrue\includeonly{\childdocname}\let\jobname\childdocmain\||fi|\\
\end{tabular}
\end{center}
%
Instead of |\childdocof{|\textit{main}|}| just include the main file
at the top of each child file:
%
\begin{center}
|\input{|\textit{main}|}|
\end{center}
%
A simple redirection |\childdocforward{|\textit{dest}|}| is achieved by:
%
\begin{center}
|\def\jobname{|\textit{dest}|}\input{\jobname}|
\end{center}
%
The redirection with prefix
|\childdocforwardprefix[|\textit{prefix}|]{|\textit{dest}|}|
is accomplished by:
%
\begin{center}
\begin{tabular}{l}
|{\edef\jobname{\scantokens\expandafter{\jobname\noexpand}}|\\
|\def\redirectjob |\textit{prefix}|#1~~~{\gdef\jobname{|\textit{dest}|#1}}|\\
|\expandafter\redirectjob\jobname~~~}\input{\jobname}|
\end{tabular}
\end{center}

In an alternative approach,
child documents can be compiled by a specific command line
without additional code or specific definitions:
%
\begin{center}
|... -jobname "|\textit{target}|" "|[\textit{flags}]%
|\includeonly{|\textit{dest}|}\input{|\textit{main}|}"|
\end{center}
%

%%%%%%%%%%%%%%%%%%%%%%%%%%%%%%%%%%%%%%%%%%%%%%%%%%%%%%%%%%%%%%%%%%%%%%%%%%%%%%%%
%%%%%%%%%%%%%%%%%%%%%%%%%%%%%%%%%%%%%%%%%%%%%%%%%%%%%%%%%%%%%%%%%%%%%%%%%%%%%%%%
\section{Information}

%%%%%%%%%%%%%%%%%%%%%%%%%%%%%%%%%%%%%%%%%%%%%%%%%%%%%%%%%%%%%%%%%%%%%%%%%%%%%%%%
\subsection{Copyright}

Copyright \copyright{} 2017--2018 Niklas Beisert

This work may be distributed and/or modified under the
conditions of the \LaTeX{} Project Public License, either version 1.3
of this license or (at your option) any later version.
The latest version of this license is in
  \url{http://www.latex-project.org/lppl.txt}
and version 1.3 or later is part of all distributions of \LaTeX{}
version 2005/12/01 or later.

This work has the LPPL maintenance status `maintained'.

The Current Maintainer of this work is Niklas Beisert.

This work consists of the files |README.txt|, |childdoc.ins| and |childdoc.dtx|
as well as the derived files |childdoc.def|, |cdocsamp.tex|
with |cdocsch1.tex|, |cdocsch2.tex|, |cdocspt3.tex|, |cdocspt4.tex|,
|cdocsdrf.tex|, |cdocsfn1.tex|, |cdocsfn2.tex|
as well as |childdoc.pdf|.

%%%%%%%%%%%%%%%%%%%%%%%%%%%%%%%%%%%%%%%%%%%%%%%%%%%%%%%%%%%%%%%%%%%%%%%%%%%%%%%%
\subsection{Files and Installation}

The package consists of the files:
%
\begin{center}
\begin{tabular}{ll}
    |README.txt|   & readme file \\
    |childdoc.ins| & installation file \\
    |childdoc.dtx| & source file \\
    |childdoc.def| & definition file \\
    |cdocsamp.tex| & sample main file \\
    |cdocsch1.tex| & sample include file \\
    |cdocsch2.tex| & sample include file \\
    |cdocspt3.tex| & sample part file \\
    |cdocspt4.tex| & sample part file \\
    |cdocsdrf.tex| & sample redirection file \\
    |cdocsfn1.tex| & sample redirection file \\
    |cdocsfn2.tex| & sample redirection file \\
    |childdoc.pdf| & manual
\end{tabular}
\end{center}
%
The distribution consists of the files
|README.txt|, |childdoc.ins| and |childdoc.dtx|.
%
\begin{itemize}
\item
Run (pdf)\LaTeX{} on |childdoc.dtx|
to compile the manual |childdoc.pdf| (this file).
\item
Run \LaTeX{} on |childdoc.ins| to create the definitions file |childdoc.def|
and the sample |cdocsamp.tex| with include files
|cdocsch1.tex|, |cdocsch2.tex|, |cdocspt3.tex|, |cdocspt4.tex|,
|cdocsdrf.tex|, |cdocsfn1.tex|, |cdocsfn2.tex|.
Then copy the file |childdoc.def| to an appropriate directory of your \LaTeX{}
distribution, e.g.\ \textit{texmf-root}|/tex/latex/childdoc|.
\end{itemize}

%%%%%%%%%%%%%%%%%%%%%%%%%%%%%%%%%%%%%%%%%%%%%%%%%%%%%%%%%%%%%%%%%%%%%%%%%%%%%%%%
\subsection{Related CTAN Packages}

There are several other packages which offer a similar functionality:
%
\begin{itemize}
\item
The packages
\href{http://ctan.org/pkg/docmute}{\textsf{docmute}},
\href{http://ctan.org/pkg/includex}{\textsf{includex}} and
\href{http://ctan.org/pkg/standalone}{\textsf{standalone}}
provide commands to include only the document body of
a child file thus allowing both files to be compiled individually.
\item
The packages \href{http://ctan.org/pkg/subdocs}{\textsf{subdocs}}
and \href{http://ctan.org/pkg/subfiles}{\textsf{subfiles}}
provide structures in which the main and child documents can be
encapsulated and allowing them to be compiled individually.
The inclusion mechanism is different from the conventional |\include|.
\item
The package \href{http://ctan.org/pkg/combine}{\textsf{combine}}
is an elaborate solution to combine several documents into one.
\end{itemize}
%
See also the CTAN topic \href{http://ctan.org/topic/subdocs}{\textsf{subdocs}}
for further related packages.
The present package differs from the above solutions in that
a document structure constructed with the conventional |\include| mechanism
just needs two extra commands at the top of every file
such that all constituent files can be compiled individually.

%%%%%%%%%%%%%%%%%%%%%%%%%%%%%%%%%%%%%%%%%%%%%%%%%%%%%%%%%%%%%%%%%%%%%%%%%%%%%%%%
%\subsection{Feature Suggestions}
%
%The following is a list of features which may be useful for future
%versions of this package:
%%
%\begin{itemize}
%\item
%\ldots
%\end{itemize}

%%%%%%%%%%%%%%%%%%%%%%%%%%%%%%%%%%%%%%%%%%%%%%%%%%%%%%%%%%%%%%%%%%%%%%%%%%%%%%%%
\subsection{Revision History}

%%%%%%%%%%%%%%%%%%%%%%%%%%%%%%%%%%%%%%%%
\paragraph{v2.0:} 2018/12/30

\begin{itemize}
\item
immediate forward processing
\item
added |\childdocby| mechanism
\item
manual restructured
\end{itemize}

%%%%%%%%%%%%%%%%%%%%%%%%%%%%%%%%%%%%%%%%
\paragraph{v1.6:} 2018/01/17

\begin{itemize}
\item
application for development of include files
\item
corrections to manual
\end{itemize}

%%%%%%%%%%%%%%%%%%%%%%%%%%%%%%%%%%%%%%%%
\paragraph{v1.5:} 2017/05/21

\begin{itemize}
\item
more complete structuring introduced
\item
|\childdocof| introduced
\item
|\childdoc| renamed to |\childdocmain|
\item
|\childredirect| renamed to |\childdocforward| and |\childdocforwardprefix|
and functionality expanded
\end{itemize}

%%%%%%%%%%%%%%%%%%%%%%%%%%%%%%%%%%%%%%%%
\paragraph{v1.0:} 2017/04/27

\begin{itemize}
\item
manual and install package
\item
first version published on CTAN
\end{itemize}

%%%%%%%%%%%%%%%%%%%%%%%%%%%%%%%%%%%%%%%%
\paragraph{v0.6:} 2017/04/26

\begin{itemize}
\item
redirection mechanism added
\end{itemize}

%%%%%%%%%%%%%%%%%%%%%%%%%%%%%%%%%%%%%%%%
\paragraph{v0.5:} 2017/04/26

\begin{itemize}
\item
functionality in definition file
\end{itemize}


%%%%%%%%%%%%%%%%%%%%%%%%%%%%%%%%%%%%%%%%%%%%%%%%%%%%%%%%%%%%%%%%%%%%%%%%%%%%%%%%
%%%%%%%%%%%%%%%%%%%%%%%%%%%%%%%%%%%%%%%%%%%%%%%%%%%%%%%%%%%%%%%%%%%%%%%%%%%%%%%%
%%%%%%%%%%%%%%%%%%%%%%%%%%%%%%%%%%%%%%%%%%%%%%%%%%%%%%%%%%%%%%%%%%%%%%%%%%%%%%%%
\appendix

\settowidth\MacroIndent{\rmfamily\scriptsize 000\ }

 \DocInput{childdoc.dtx}

\end{document}
%</driver>
% \fi
%
% %%%%%%%%%%%%%%%%%%%%%%%%%%%%%%%%%%%%%%%%%%%%%%%%%%%%%%%%%%%%%%%%%%%%%%%%%%%%%%
% %%%%%%%%%%%%%%%%%%%%%%%%%%%%%%%%%%%%%%%%%%%%%%%%%%%%%%%%%%%%%%%%%%%%%%%%%%%%%%
% \section{Sample}
%\iffalse
%<*samplemain>
%\fi
%
% The following presents a sample document
% with two chapters, two parts, a title page,
% a compile flag as well as three forwarding files to set the flag.
% It consists of eight |.tex| files:
% \begin{center}
% \begin{tabular}{ll}
% |cdocsamp.tex|&main file\\
% |cdocsch1.tex|&include file for chapter 1\\
% |cdocsch2.tex|&include file for chapter 2\\
% |cdocspt3.tex|&include file for part 3\\
% |cdocspt4.tex|&include file for part 4\\
% |cdocsdrf.tex|&forwarding file for main file in draft mode\\
% |cdocsfi1.tex|&forwarding file for final version of chapter 1\\
% |cdocsfi2.tex|&forwarding file for final version of chapter 2\\
% \end{tabular}
% \end{center}
% Each of the eight files can be compiled directly by the \LaTeX{} compiler.
%
% %%%%%%%%%%%%%%%%%%%%%%%%%%%%%%%%%%%%%%
% \paragraph{Main File.}
%
% The main file is called |cdocsamp.tex|.
%
% Load the \textsf{childdoc} definitions and
% declare the filename for the main document:
%    \begin{macrocode}
\input{childdoc.def}
\childdocmain{}
%    \end{macrocode}

% Optional override for |\version| flag:
%    \begin{macrocode}
%%\ifchilddoc\else\providecommand{\version}{draft}\fi
%    \end{macrocode}

% Define the default values for the |\version| flag
% (|final| for the main file and |draft| for childs):
%    \begin{macrocode}
\ifchilddoc
\providecommand{\version}{draft}
\else
\providecommand{\version}{final}
\fi
%    \end{macrocode}

% Load the standard document class:
%    \begin{macrocode}
\documentclass[12pt]{article}
%    \end{macrocode}

% Start the document body:
%    \begin{macrocode}
\begin{document}
%    \end{macrocode}

% Declare a title page.
% Print title, part of document being processed and version flag:
%    \begin{macrocode}
\addtocounter{page}{-1}
\begin{center}
{\LARGE\bfseries{}childdoc example\par}
\vspace{1cm}
\ifchilddoc
\ifchilddocmanual part\else chapter\fi:
`\childdocname' of `\childdocjob'\par
\else
main document: `\childdocjob'\par
\fi
version: \version\par
\end{center}
\newpage
%    \end{macrocode}

% Manually include selected file,
% otherwise process as usual:
%    \begin{macrocode}
\ifchilddocmanual
\section*{part `\childdocname'}
\input{\childdocname}
\else
%    \end{macrocode}

% Include the two chapters:
%    \begin{macrocode}
\include{cdocsch1}
\include{cdocsch2}
%    \end{macrocode}

% Include the two parts unless only chapters should be displayed:
%    \begin{macrocode}
\ifchilddoc\else
\section{part three}
\input{cdocspt3}
\section{part four}
\input{cdocspt4}
\fi
%    \end{macrocode}

% Process as usual until here:
%    \begin{macrocode}
\fi
%    \end{macrocode}

% End of document body:
%    \begin{macrocode}
\end{document}
%    \end{macrocode}
%\iffalse
%</samplemain>
%\fi
%
% %%%%%%%%%%%%%%%%%%%%%%%%%%%%%%%%%%%%%%
% \paragraph{Chapter Include Files.}
%
% The include files are called |cdocsch1.tex| and |cdocsch2.tex|.
%
%\iffalse
%<*samplechap1|samplechap2>
%\fi

% Optional override for |\version| flag:
%    \begin{macrocode}
%%\providecommand{\version}{final}
%    \end{macrocode}

% Include the main document:
%    \begin{macrocode}
\input{childdoc.def}
\childdocof{cdocsamp}
%    \end{macrocode}

%\iffalse
%</samplechap1|samplechap2>
%\fi
%
%\iffalse
%<*samplechap1>
%\fi
% Some text for chapter 1:
%    \begin{macrocode}
\section{one}
some text in chapter one
%    \end{macrocode}

%\iffalse
%</samplechap1>
%\fi
% Some text for chapter 2:
%\iffalse
%<*samplechap2>
%\fi
%    \begin{macrocode}
\section{two}
more text in chapter two
%    \end{macrocode}

%\iffalse
%</samplechap2>
%\fi
%
% %%%%%%%%%%%%%%%%%%%%%%%%%%%%%%%%%%%%%%
% \paragraph{Part Include Files.}
%
% The include files are called |cdocspt3.tex| and |cdocspt4.tex|.
%
%\iffalse
%<*samplepart3|samplepart4>
%\fi

% Optional override for |\version| flag:
%    \begin{macrocode}
%%\providecommand{\version}{final}
%    \end{macrocode}

% Include the main document:
%    \begin{macrocode}
\input{childdoc.def}
\childdocby{cdocsamp}
%    \end{macrocode}

%\iffalse
%</samplepart3|samplepart4>
%\fi
%
%\iffalse
%<*samplepart3>
%\fi
% Some text for part 3:
%    \begin{macrocode}
some text in part three
%    \end{macrocode}

%\iffalse
%</samplepart3>
%\fi
% Some text for part 4:
%\iffalse
%<*samplepart4>
%\fi
%    \begin{macrocode}
more text in part four
%    \end{macrocode}

%\iffalse
%</samplepart4>
%\fi
%
% %%%%%%%%%%%%%%%%%%%%%%%%%%%%%%%%%%%%%%
% \paragraph{Forwarding for a Complete Draft.}
%
% The following forwarding file |cdocsdrf.tex|
% compiles the main document in draft mode:
%\iffalse
%<*sampledraft>
%\fi
%    \begin{macrocode}
\def\version{draft}
\input{childdoc.def}
\childdocforward{cdocsamp}
%    \end{macrocode}

%\iffalse
%</sampledraft>
%\fi
%
% %%%%%%%%%%%%%%%%%%%%%%%%%%%%%%%%%%%%%%
% \paragraph{Forwarding for Final Version of the Chapters.}
%
% The following forwarding files |cdocsfn1.tex| and |cdocsfn2.tex|
% (with identical content)
% compile the final versions of the child documents
% |cdocsch1.tex| and |cdocsch2.tex|, respectively:
%\iffalse
%<*samplefinal>
%\fi
%    \begin{macrocode}
\def\version{final}
\input{childdoc.def}
\childdocforwardprefix[cdocsamp]{cdocsfn}{cdocsch}
%    \end{macrocode}

%\iffalse
%</samplefinal>
%\fi
%
% %%%%%%%%%%%%%%%%%%%%%%%%%%%%%%%%%%%%%%
% \paragraph{Command Line Processing.}
%
% The following three command lines generate the output files
% |cdocscld|, |cdocscl1| and |cdocscl2|
% which should be identical to
% |cdocsdrf|, |cdocsch1| and |cdocsfn2|, respectively:
% \begin{center}
% \begin{tabular}{l}
% |latex -jobname cdocscld \|\\
% |  "\def\version{draft}\input{childdoc.def}\childdocforward{cdocsamp}"|\\
% |latex -jobname cdocscl1 \|\\
% |  "\input{childdoc.def}\childdocforward[cdocsamp]{cdocsch1}"|\\
% |latex -jobname cdocscl2 \|\\
% |  "\def\version{final}\input{childdoc.def}\childdocforward{cdocsch2}"|
% \end{tabular}
% \end{center}
% Note that the trailing backslash on each first line
% merely continues the input to the second line
% (for convenient cut ant paste).
% Furthermore, the command |latex| can be replaced by any
% of its alternative versions such as |pdflatex|.
%
% %%%%%%%%%%%%%%%%%%%%%%%%%%%%%%%%%%%%%%%%%%%%%%%%%%%%%%%%%%%%%%%%%%%%%%%%%%%%%%
% %%%%%%%%%%%%%%%%%%%%%%%%%%%%%%%%%%%%%%%%%%%%%%%%%%%%%%%%%%%%%%%%%%%%%%%%%%%%%%
% \section{Implementation}
%\iffalse
%<*package>
%\fi
%
% This section describes the definitions file |childdoc.def|.

% The definitions cannot be loaded using |\usepackage| or |\RequirePackage|
% which has a mechanism to prevent loading a style file more than once.
% When loading the definitions by means of |\input|
% multiple instances have to be prevented manually:
%\iffalse
%This code needs to be before the `\ProvidesFile' directive
%which is defined at the beginning of this file.
%Therefore it is also placed there and commented out here.
%</package>
%<*discard>
%\fi
%    \begin{macrocode}
\ifdefined\childdocmain\endinput\fi
%    \end{macrocode}
%\iffalse
%</discard>
%<*package>
%\fi
%
% \macro{\ifchilddoc}
% \macro{\ifchilddocmanual}
% The conditional |\ifchilddoc| tells whether a
% child (true) or main (false) document is being compiled.
% The conditional |\ifchilddocmanual| tells whether
% the |\includeonly| mechanism is used (false) or
% the selection of child files must be performed manually (true).
% The definitions initialise to false:
%    \begin{macrocode}
\newif\ifchilddoc
\newif\ifchilddocmanual
%    \end{macrocode}

% \macro{\childdocname}
% \macro{\childdocjob}
% The macro |\childdocname| stores the name of the main document
% to be compiled. The macro |\childdocjob| stores the name of
% the document on which the \LaTeX{} compiler was originally invoked.
% The content of |\jobname| cannot be compared
% to filenames specified in the source due to different catcodes.
% The following code rescans |\jobname|, stores the result
% in |\childdocname| and saves a copy in |\childdocjob|:
%    \begin{macrocode}
\edef\childdocname{\scantokens\expandafter{\jobname\noexpand}}
\let\childdocjob\childdocname
%    \end{macrocode}

% \macro{\childdocdisable}
% The macro |\childdocdisable| prevents the main file
% from being processed more than once.
% At this stage, the main document command |\childdocmain|
% is assumed to be called once again where it should do nothing.
% Any subsequent call to it should prevent
% a secondary processing of the main document
% It overwrites the forwarding commands
% |\childdocof| and |\childdocforward|
% with empty macros to prevent further inclusions of the main document:
%    \begin{macrocode}
\newcommand{\childdocdisable}
{
  \renewcommand{\childdocmain}[1]{\renewcommand{\childdocmain}[1]{\endinput}}
  \renewcommand{\childdocof}[1]{}
  \renewcommand{\childdocby}[2][]{}
  \renewcommand{\childdocforward}[2][]{}
  \renewcommand{\childdocdisable}{}
}
%    \end{macrocode}

% \macro{\childdocmain}
% The macro |\childdocmain| is to be called at the top of the main file
% with nothing or the main filename (without extension) as argument.
% First, it breaks loops.
% If the argument is not empty and does not match |\childdocname|
% (which is set by the first inclusion of |childdoc.def|),
% |\ifchilddoc| is set to true, |\includeonly| is applied to the child file
% and |\jobname| is set to the main file
% (for proper handling of |.aux| files):
%    \begin{macrocode}
\newcommand{\childdocmain}[1]
{
  \childdocdisable\childdocmain{}
  \if?#1?\else
    \begingroup
      \def\childdoctmp{#1}
      \ifx\childdoctmp\childdocname
        \def\childdoctmp{}
      \else
        \def\childdoctmp
        {
          \childdoctrue
          \includeonly{\childdocname}
          \def\childdocjob{#1}
          \def\jobname{#1}
        }
      \fi
      \expandafter
    \endgroup
    \childdoctmp
  \fi
}
%    \end{macrocode}

% \macro{\childdocof}
% The command |\childdocof| redirects
% compilation to the main file |#1|.
%    \begin{macrocode}
\newcommand{\childdocof}[1]
{
  \childdocdisable
  \childdoctrue
  \includeonly{\childdocname}
  \def\jobname{#1}
  \def\childdocjob{#1}
  \input{#1}
}
%    \end{macrocode}

% \macro{\childdocby}
% The command |\childdocby| ....
%    \begin{macrocode}
\newcommand{\childdocby}[2][]
{
  \childdocdisable
  \childdoctrue
  \childdocmanualtrue
  \if?#1?\else
    \def\jobname{#2}
  \fi
  \def\childdocjob{#2}
  \input{#2}
  \endinput
}
%    \end{macrocode}

% \macro{\childdocforward}
% The command |\childdocforward| redirects
% compilation to the main file or
% (if the optional argument is given) a child file.
% Parameters are set as if the main file
% or a child file starting with |\childdocof| was compiled.
% Then compilation is handed over to the main file:
%    \begin{macrocode}
\newcommand{\childdocforward}[2][]
{
  \begingroup
    \if?#1?
      \def\childdoctmp
      {
        \def\childdocname{#2}
        \def\childdocjob{#2}
        \def\jobname{#2}
        \input{#2}
        \endinput
      }
    \else
      \def\childdoctmp
      {
        \childdocdisable
        \def\childdocname{#2}
        \childdoctrue
        \includeonly{#2}
        \def\childdocjob{#1}
        \def\jobname{#1}
        \input{#1}
        \endinput
      }
    \fi
    \expandafter
  \endgroup
  \childdoctmp
}
%    \end{macrocode}

% \macro{\childdocforwardprefix}
% The command |\childdocforwardprefix| redirects
% compilation to the main or a child file by means of a pattern.
% The prefix |#1| in the current filename is replaced by |#2|
% and the suffix of the current filename is kept
% (it is assumed that the filename does not contain the substring `|~~~|'
% which is used as a delimiter).
% Compilation is handed over to the new file by |\childdocforward|:
%    \begin{macrocode}
\newcommand{\childdocforwardprefix}[3][]
{
  \begingroup
    \def\childdocextract #2##1~~~{\def\childdoctmp{\childdocforward[#1]{#3##1}}}
    \expandafter\childdocextract\childdocname~~~
    \expandafter
  \endgroup
  \childdoctmp
}
%    \end{macrocode}

% \macro{\childdoc}
% The deprecated macro |\childdoc| is a legacy version of |\childdocmain|:
%    \begin{macrocode}
\newcommand{\childdoc}{\childdocmain}
%    \end{macrocode}

% \macro{\childdocredirect}
% The deprecated macro |\childdocredirect| is a legacy version
% of |\childdocforward| and |\childdocforwardprefix|:
%    \begin{macrocode}
\newcommand{\childdocredirect}[2][]
{
  \begingroup
    \if?#1?
      \def\childdoctmp{\childdocforward{#2}}
    \else
      \def\childdoctmp{\childdocforwardprefix{#1}{#2}}
    \fi
    \expandafter
  \endgroup
  \childdoctmp
}
%    \end{macrocode}

%\iffalse
%</package>
%\fi
%
\endinput
|\\
|\childdocforwardprefix{final}{child}|
\end{tabular}
\end{center}
%

Note that when several versions of a main file and/or of each child file
are to be generated, it may be convenient to set up a |Makefile| or
shell script to automatise the process.

%%%%%%%%%%%%%%%%%%%%%%%%%%%%%%%%%%%%%%%%%%%%%%%%%%%%%%%%%%%%%%%%%%%%%%%%%%%%%%%%
\subsection{Command Line Processing}
\label{sec:commandline}

The effect of redirection files can also be achieved by invoking
the \LaTeX{} compiler with a more elaborate command line.
Most conveniently this should be done as part
of a shell script or a |Makefile|.

When using \textsf{childdoc} in the main file, the following
command lines effectively perform a redirection
(note that depending on the shell being used,
backslashes may have to be doubled: `|\|' $\to$ `|\\|'):
%
\begin{center}
|... -jobname "|\textit{target}|" |\\|"|[\textit{flags}]%
|% \iffalse
%
% childdoc.dtx Copyright (C) 2017-2018 Niklas Beisert
%
% This work may be distributed and/or modified under the
% conditions of the LaTeX Project Public License, either version 1.3
% of this license or (at your option) any later version.
% The latest version of this license is in
%   http://www.latex-project.org/lppl.txt
% and version 1.3 or later is part of all distributions of LaTeX
% version 2005/12/01 or later.
%
% This work has the LPPL maintenance status `maintained'.
%
% The Current Maintainer of this work is Niklas Beisert.
%
% This work consists of the files childdoc.dtx and childdoc.ins
% and the derived files childdoc.def and cdocsamp.tex with
% cdocsch1.tex, cdocsch2.tex, cdocsdrf.tex, cdocsfn1.tex, cdocsfn2.tex.
%
%<package>\ifdefined\childdocmain\endinput\fi
%<package>\ProvidesFile{childdoc.def}[2018/12/30 v2.0 child document driver]
%<samplemain>\ProvidesFile{cdocsamp.tex}[2018/12/30 v2.0 sample for childdoc]
%<*driver>
%\ProvidesFile{childdoc.drv}[2018/12/30 v2.0 childdoc reference manual file]
\PassOptionsToClass{10pt,a4paper}{article}
\documentclass{ltxdoc}

\usepackage[margin=35mm]{geometry}
\usepackage{hyperref}
\usepackage{hyperxmp}
\usepackage[usenames]{color}

\hypersetup{colorlinks=true}
\hypersetup{pdfstartview=FitH}
\hypersetup{pdfpagemode=UseNone}
\hypersetup{pdfsource={}}
\hypersetup{pdflang={en-UK}}
\hypersetup{pdfcopyright={Copyright 2017-2018 Niklas Beisert.
  This work may be distributed and/or modified under the
  conditions of the LaTeX Project Public License, either version 1.3
  of this license or (at your option) any later version.}}
\hypersetup{pdflicenseurl={http://www.latex-project.org/lppl.txt}}
\hypersetup{pdfcontactaddress={ETH Zurich, ITP, HIT K,
  Wolfgang-Pauli-Strasse 27}}
\hypersetup{pdfcontactpostcode={8093}}
\hypersetup{pdfcontactcity={Zurich}}
\hypersetup{pdfcontactcountry={Switzerland}}
\hypersetup{pdfcontactemail={nbeisert@itp.phys.ethz.ch}}
\hypersetup{pdfcontacturl={http://people.phys.ethz.ch/\xmptilde nbeisert/}}

\newcommand{\secref}[1]{\hyperref[#1]{section \ref*{#1}}}

\parskip1ex
\parindent0pt
\let\olditemize\itemize
\def\itemize{\olditemize\parskip0pt}

\begin{document}

\title{The \textsf{childdoc} Package}
\hypersetup{pdftitle={The childdoc Package}}
\author{Niklas Beisert\\[2ex]
  Institut f\"ur Theoretische Physik\\
  Eidgen\"ossische Technische Hochschule Z\"urich\\
  Wolfgang-Pauli-Strasse 27, 8093 Z\"urich, Switzerland\\[1ex]
  \href{mailto:nbeisert@itp.phys.ethz.ch}
  {\texttt{nbeisert@itp.phys.ethz.ch}}}
\hypersetup{pdfauthor={Niklas Beisert}}
\hypersetup{pdfsubject={Manual for the LaTeX2e Package childdoc}}
\date{30 December 2018, \textsf{v2.0}}
\maketitle

\begin{abstract}\noindent
\textsf{childdoc} is a \LaTeXe{} package
that enables the direct compilation
of document sections included by |\include|
to individual files.
\end{abstract}

\begingroup
\parskip0ex
\tableofcontents
\endgroup

%%%%%%%%%%%%%%%%%%%%%%%%%%%%%%%%%%%%%%%%%%%%%%%%%%%%%%%%%%%%%%%%%%%%%%%%%%%%%%%%
%%%%%%%%%%%%%%%%%%%%%%%%%%%%%%%%%%%%%%%%%%%%%%%%%%%%%%%%%%%%%%%%%%%%%%%%%%%%%%%%
\section{Introduction}

\LaTeX{} provides a mechanism to structure a large document (such as a book)
into a main file and several child files (containing the chapters)
using the |\include| command.
This mechanism is beneficial for documents
which span hundreds of pages in order to
make the source file(s) more manageable.
Moreover, compilation can be restricted to
selected child files by means of the |\includeonly| command.
The latter feature can be used to reduce the compilation time while editing
(this was significantly more useful in the earlier days of \LaTeX{})
or to generate a smaller document which is easier to navigate.
Another application of |\includeonly| is to generate
documents consisting of selected parts of the complete document.

However, there are a few drawbacks of the plain |\include| mechanism:
\begin{itemize}
\item
The child files cannot be compiled on their own,
they can only be compiled via the main file.
A naive editing environment
(such as a text editor with an option
to have the current file processed by \LaTeX)
may require one to switch to the main file before compiling;
attempting to compile the child file produces errors.
\item
The main file must be modified (each time)
to adjust the |\includeonly| command
to the present needs. This easily leaves the main file in a messy state.
\item
The generated document will always carry the filename
of the main document. This is inconvenient if
several child files are to be compiled and
to be kept for distribution.
\end{itemize}

The present package provides a simple interface
to make child files individually compilable by \LaTeX{}.
Compiling a child file then has the same effect as compiling
the main file with an |\includeonly| command
to select the appropriate child.
Moreover the generated document will carry the name of the child
rather than the main file.
This resolves all three above issues.

This feature is meant to make the editing of books,
thesis documents and lecture notes somewhat more convenient.
However, the package can also be used efficiently for
composing a series of documents (such as exercise sheets)
which are typically distributed individually.
It then assists the author in generating the individual documents
(potentially in different versions)
as well as a document containing the collected series.
Another application is in developing style files
or other kinds of included material
where compilation of the style file could redirect
to a sample or test file.

%%%%%%%%%%%%%%%%%%%%%%%%%%%%%%%%%%%%%%%%%%%%%%%%%%%%%%%%%%%%%%%%%%%%%%%%%%%%%%%%
%%%%%%%%%%%%%%%%%%%%%%%%%%%%%%%%%%%%%%%%%%%%%%%%%%%%%%%%%%%%%%%%%%%%%%%%%%%%%%%%
\section{Usage}

First of all, the package \textsf{childdoc} is \emph{not} a standard
\LaTeXe{} |.sty| style file! Therefore it needs to be invoked in
a non-standard way.

%%%%%%%%%%%%%%%%%%%%%%%%%%%%%%%%%%%%%%%%%%%%%%%%%%%%%%%%%%%%%%%%%%%%%%%%%%%%%%%%
\subsection{Included Files}
\label{sec:include}

%%%%%%%%%%%%%%%%%%%%%%%%%%%%%%%%%%%%%%%%
\DescribeMacro{\childdocmain}
To use the package, add the commands
\begin{center}
\begin{tabular}{l}
|\input{childdoc.def}|\\
|\childdocmain{}|\\
\end{tabular}
\end{center}
at the very top of the main \LaTeX{} file,
in particular \emph{before} the |\documentclass| statement!
The argument of |\childdocmain| should be left empty
(but it must be present).

%%%%%%%%%%%%%%%%%%%%%%%%%%%%%%%%%%%%%%%%
\DescribeMacro{\childdocof}
Furthermore, add the commands
\begin{center}
\begin{tabular}{l}
|\input{childdoc.def}|\\
|\childdocof{|\textit{main}|}|\\
\end{tabular}
\end{center}
at the top of every child file \textit{child}
which is included by |\include{|\textit{child}|}|
from within the main file
(or at least for those files to be compiled individually).
The argument \textit{main} must be the filename of the main file.

There are a couple of
considerations in setting up the main and child documents:

%%%%%%%%%%%%%%%%%%%%%%%%%%%%%%%%%%%%%%%%
\paragraph{Restrictions.}

Please note the following restrictions:
\begin{itemize}
\item
|\childdocmain| must be called with one argument \textit{main}
to ensure compatibility with earlier version of the package.
It must either be empty (|\childdocmain{}|)
or precisely match the filename of the main file in which it is specified.
See \secref{sec:detection} for further information.
\item
The filename \textit{main} must be specified without the |.tex| extension.
\item
The filename \textit{main} is case sensitive
(even in case-insensitive file systems)
due to internal string comparison.
\item
The argument \textit{main} should be fully expanded, it cannot be a macro.
\item
Subdirectories and special characters should be avoided in filenames.
\item
The command |\childdocmain{|\textit{main}|}| must be followed by a whitespace.
It should not be followed immediately by another command
or by a comment mark `|%|'.
This is because the \TeX{} parser reads the token immediately following
the argument of |\childdocmain| and puts it
at the beginning of every child section;
however, a white\-space is ignored.
\end{itemize}

%%%%%%%%%%%%%%%%%%%%%%%%%%%%%%%%%%%%%%%%
\paragraph{Content of Main File.}

It is advisable to place all content in the child files included by |\include|.
Any output contained in the main file will appear in all child documents
unless suppressed manually;
it cannot be suppressed automatically by the |\includeonly| directive
and thus should normally be avoided.
A method to include some content in the main file
by means of conditional processing is described in \secref{sec:conditional}.

%%%%%%%%%%%%%%%%%%%%%%%%%%%%%%%%%%%%%%%%
\paragraph{Page Numbering.}

When only a part of the document is compiled,
the appropriate numbering of pages
(as well as other status parameters)
is determined from the |.aux| files.
The latter contain information from previous passes.
However this information needs to propagate through
all intermediate child documents.
Therefore the page numbering in child documents may well
be inconsistent until the complete document is compiled at least once.

A useful (if unconventional) way to always ensure a consistent
page numbering is to restart the numbering in each child document
and denote the pages by `\textit{child}|.|\textit{page}'
where \textit{child} represents the chapter/section number of the child file.
This can be achieved by the command
|\numberwithin{page}{|\textit{child}|}|
of the \textsf{amsmath} package
where \textit{child} can be |chapter| or |section|
depending on the chosen structuring.
Alternatively, one can modify the macro |\thepage| appropriately
and reset the counter |page| at the start of each child file.

%%%%%%%%%%%%%%%%%%%%%%%%%%%%%%%%%%%%%%%%%%%%%%%%%%%%%%%%%%%%%%%%%%%%%%%%%%%%%%%%
\subsection{Conditional Processing}
\label{sec:conditional}

The package provides a mechanism to compile different versions
of a document. To customise the versions further some conditional processing
can come in handy to distinguish which version is being compiled.
The package provides two macros to describe the compilation context:

%%%%%%%%%%%%%%%%%%%%%%%%%%%%%%%%%%%%%%%%
\DescribeMacro{\ifchilddoc}
The conditional |\ifchilddoc| distinguishes between the compilation of
child documents and the main document:
%
\begin{center}
|\ifchilddoc |\textit{child-code}| |[|\||else |\textit{main-code}]| \||fi|
\end{center}

%%%%%%%%%%%%%%%%%%%%%%%%%%%%%%%%%%%%%%%%
\DescribeMacro{\childdocname}
\DescribeMacro{\childdocjob}
The macro |\childdocname| contains the filename (without extension)
of the main or child file being processed.
Note that |\childdocjob| will always contain the name of the main file.

%%%%%%%%%%%%%%%%%%%%%%%%%%%%%%%%%%%%%%%%
\paragraph{Title Page.}

Conditional processing can be used to include a title or banner page
in the main document when proper precautions are taken.
Importantly, the code in the main file should ensure that the page counter
(as well as other status parameters which are stored in the |.aux| files)
takes the same value after the conditional processing.
Otherwise the page numbers may take divergent values
depending on which part is compiled.

For example, a title page could be declared by:
%
\begin{center}
\begin{tabular}{l}
|\ifchilddoc\||else|\\
|\addtocounter{page}{-1}|\\
\textit{code for title page}\\
|\newpage|\\
|\||fi|
\end{tabular}
\end{center}
%
A banner page for the child documents can be generated by:
%
\begin{center}
\begin{tabular}{l}
|\ifchilddoc|\\
|\addtocounter{page}{-1}|\\
\textit{code for banner page}\\
|\newpage|\\
|\||fi|
\end{tabular}
\end{center}
%
Here one could write a message such as:
\begin{center}
|This is the part \childdocname{} of \childdocjob{}.|
\end{center}

%%%%%%%%%%%%%%%%%%%%%%%%%%%%%%%%%%%%%%%%%%%%%%%%%%%%%%%%%%%%%%%%%%%%%%%%%%%%%%%%
\subsection{Flags}
\label{sec:flags}

The package makes it easy to generate different versions
of the main or child documents.
To this end compilation flags can be defined
and assigned different default values.
They will be particularly useful in conjunction
with the forwarding mechanism described in \secref{sec:forward}.

For example, it may be useful to have a flag |\version|
which can be set to |draft| or |final|.
The document source will contain some conditional code
depending on the value of |\version|.
Suppose further, the flag should default to |final| for the main file
and to |draft| for child files
which is a natural assignment for editing the document.
This is achieved by placing the following code
in the preamble of the main document
(below the |\childdocmain| directive):
%
\begin{center}
\begin{tabular}{l}
|\ifchilddoc|\\
|\providecommand{\version}{draft}|\\
|\||else|\\
|\providecommand{\version}{final}|\\
|\||fi|
\end{tabular}
\end{center}
%
The definition by |\providecommand| makes sure
that previous definitions are not overwritten.
Further statements |\providecommand{\version}{...}|
can thus be added before the above code to override it.

For the main file, one might add a line
(between |\childdocmain| and the above block)
%
\begin{center}
|%\ifchilddoc\||else\providecommand{\version}{draft}\||fi|
\end{center}
%
which can be uncommented to produce a draft version.
Likewise one can add a line to the very top of a child file
(above the |\childdocof{|\textit{main}|}| directive)
%
\begin{center}
|%\providecommand{\version}{final}|
\end{center}
%
which can be uncommented to produce the final version of this child document.

%%%%%%%%%%%%%%%%%%%%%%%%%%%%%%%%%%%%%%%%%%%%%%%%%%%%%%%%%%%%%%%%%%%%%%%%%%%%%%%%
\subsection{Forwarding}
\label{sec:forward}

Different versions of the main or child documents
using compilation flags as described in \secref{sec:flags}
can be (permanently) stored in different files
for convenient compilation, viewing and distribution.
To this end, the package defines a command
to pass on compilation to a different file:

%%%%%%%%%%%%%%%%%%%%%%%%%%%%%%%%%%%%%%%%
\DescribeMacro{\childdocforward}
The command |\childdocforward| redirects processing to
another source file:
%
\begin{center}
\begin{tabular}{l}
|\input{childdoc.def}|\\
|\childdocforward[|\textit{main}|]{|\textit{dest}|}|\\
\end{tabular}
\end{center}
%
The argument \textit{dest} is the destination file
(without extension).
It should be the main file or one of the child files.
Note that further \textsf{childdoc} directives
such as |\childdocof| and |\childdocforward|
in the indicated file will be processed in this form.
The optional argument \textit{main}
passes on directly to the main file \textit{main}
while pretending to compile the child \textit{dest}.
This form behaves as if \textit{dest}
issues |\childdocof{|\textit{main}|}| right away,
and no further \textsf{childdoc} directives will be processed.

%%%%%%%%%%%%%%%%%%%%%%%%%%%%%%%%%%%%%%%%
\DescribeMacro{\...prefix}
In the alternative form |\childdocforwardprefix|,
%
\begin{center}
\begin{tabular}{l}
|\input{childdoc.def}|\\
|\childdocforwardprefix[|\textit{main}|]{|\textit{prefix}|}{|\textit{dest}|}|
\end{tabular}
\end{center}
%
the destination file is determined by a pattern
depending on the current file:
To make this work, the current file must be called
`{\textit{prefix}\hspace{0.2em}\textit{suffix}}'
with \textit{prefix} matching precisely the argument.
Processing is then passed on to the file
`{\textit{dest}\hspace{0.2em}\textit{suffix}}'.
Surely, the same effect is achieved by
directly specifying the
argument `{\textit{dest}\hspace{0.2em}\textit{suffix}}'
in the first form.
However, that requires to set up a different file
for each child. With the alternative form of the command
all these files can have exactly the same content
which simplifies setting them up and maintaining them.

For example, the following file |draft.tex|
with a compilation flag |\version| as described in \secref{sec:flags}
compiles the main document as a draft:
%
\begin{center}
\begin{tabular}{l}
|\def\version{draft}|\\
|\input{childdoc.def}|\\
|\childdocforward{|\textit{main}|}|
\end{tabular}
\end{center}
%
Likewise, the following files |final|\textit{nn}|.tex|
compile the final version of the child document
|child|\textit{nn}|.tex|:
%
\begin{center}
\begin{tabular}{l}
|\def\version{final}|\\
|\input{childdoc.def}|\\
|\childdocforwardprefix{final}{child}|
\end{tabular}
\end{center}
%

Note that when several versions of a main file and/or of each child file
are to be generated, it may be convenient to set up a |Makefile| or
shell script to automatise the process.

%%%%%%%%%%%%%%%%%%%%%%%%%%%%%%%%%%%%%%%%%%%%%%%%%%%%%%%%%%%%%%%%%%%%%%%%%%%%%%%%
\subsection{Command Line Processing}
\label{sec:commandline}

The effect of redirection files can also be achieved by invoking
the \LaTeX{} compiler with a more elaborate command line.
Most conveniently this should be done as part
of a shell script or a |Makefile|.

When using \textsf{childdoc} in the main file, the following
command lines effectively perform a redirection
(note that depending on the shell being used,
backslashes may have to be doubled: `|\|' $\to$ `|\\|'):
%
\begin{center}
|... -jobname "|\textit{target}|" |\\|"|[\textit{flags}]%
|\input{childdoc.def}\childdocforward[|\textit{main}|]{|\textit{dest}|}"|
\end{center}
%
Here \textit{target} is the name of the output file,
\textit{main} is the name of the main file
and \textit{dest} is the name of the main or child file to be processed
(all filenames without extensions).
The optional argument \textit{main} can be omitted
if \textit{main} matches \textit{dest}.
Optionally, compilation \textit{flags} can be defined via |\def| commands.
This command line makes the \TeX{} engine believe
it is compiling the file \textit{target}
whose content is specified as the latter parameter.
The provided code then forwards the processing to
\textit{main} or \textit{dest} as described in \secref{sec:forward}.

%%%%%%%%%%%%%%%%%%%%%%%%%%%%%%%%%%%%%%%%%%%%%%%%%%%%%%%%%%%%%%%%%%%%%%%%%%%%%%%%
\subsection{Include by Input}
\label{sec:input}

Including child documents by |\include| has some restrictions by design.
Most notably, the content of a child document always occupies
its own set of pages; pages cannot be shared between child documents.
Usually, this behaviour makes perfect sense
because each child document contain an essential part of the document.
However, in some situations it may be desirable to compose
a document from a collection of parts
without having mandatory page breaks between then.
For this case, the package
provides a mechanism to include parts
by |\input| which can also be processed individually.
However, by construction this mechanism
requires manual handling of the content to be output.

%%%%%%%%%%%%%%%%%%%%%%%%%%%%%%%%%%%%%%%%
\DescribeMacro{\ifchilddocmanual}
The main file should be prepared as usual, see \secref{sec:include}.
However, the document body must make a distinction
between processing of an individual part and of the main document, e.g.:
%
\begin{center}
\begin{tabular}{l}
|\ifchilddocmanual|\\
|\input{\childdocname}|\\
|\||else|\\
\textit{document body with }|\input{|\textit{part}|}|\\
|\||fi|
\end{tabular}
\end{center}
%
The conditional |\ifchilddocmanual| is true whenever
a part to be included by |\input| is being compiled,
and the name of the part is stored in |\childdocname|.

%%%%%%%%%%%%%%%%%%%%%%%%%%%%%%%%%%%%%%%%
\DescribeMacro{\childdocby}
Each part to be included by |\input| should start with:
%
\begin{center}
\begin{tabular}{l}
|\input{childdoc.def}|\\
|\childdocby{|\textit{main}|}|\\
\end{tabular}
\end{center}
%
The directive |\childdocby| is similar to |\childdocof|
described in \secref{sec:include},
but the subsequent selection of content must be done manually.
To that end, both |\ifchilddoc| and |\ifchilddocmanual|
will be true upon processing of a part,
and the name of the part is stored in |\childdocname|.
Note that |\jobname| will be set to the filename of the current part
so that each part receives an individual |.aux| file
that does not interfere with the |.aux| file(s) of the main document.
This behaviour can be altered by the alternative form
|\childdocby[*]{|\textit{main}|}| (with a non-empty optional argument)
which uses the |.aux| file of the main document
by setting |\jobname| to \textit{main}.

%%%%%%%%%%%%%%%%%%%%%%%%%%%%%%%%%%%%%%%%%%%%%%%%%%%%%%%%%%%%%%%%%%%%%%%%%%%%%%%%
\subsection{Driver Development}
\label{sec:driver}

The \textsf{childdoc} mechanism can also be use for the development
of definition files such as \LaTeX{} styles or classes.
This case differs from the above setup with multiple parts
included by |\include| in that no |\includeonly| should be invoked.
This can be achieved by starting the include file
(before |\ProvidesPackage|) with:
%
\begin{center}
\begin{tabular}{l}
|\input{childdoc.def}|\\
|\childdocforward{|\textit{main}|}|\\
\end{tabular}
\end{center}
%
or alternatively with:
%
\begin{center}
\begin{tabular}{l}
|\input{childdoc.def}|\\
|\childdocby{|\textit{main}|}|\\
\end{tabular}
\end{center}
%
Both forms have slightly different effects as described above.
The main file is prepared as usual, see \secref{sec:include}.

%%%%%%%%%%%%%%%%%%%%%%%%%%%%%%%%%%%%%%%%%%%%%%%%%%%%%%%%%%%%%%%%%%%%%%%%%%%%%%%%
\subsection{Legacy Detection}
\label{sec:detection}

The directive |\childdocmain| in the main file can detect
whether the complete document or merely a child is to be compiled
even without using the directive |\childdocof|.
This method is deprecated because it is less robust
and there is no compelling reason to use it;
it is merely provided for backward compatibility
and it may be removed in future versions.

If the detection mechanism is to be used,
it is mandatory to correctly specify
the filename of the main file as the argument of |\childdocmain|:
%
\begin{center}
\begin{tabular}{l}
|\input{childdoc.def}|\\
|\childdocmain{|\textit{main}|}|\\
\end{tabular}
\end{center}
%
If |\jobname| does not match the argument \textit{main} of |\childdocmain|,
it is assumed that |\jobname| points to the child file to be compiled.
When using |\childdocmain| with the main file specified as argument,
it suffices to start a child file
with just |\input{|\textit{main}|}|
without loading of the package and using |\childdocof|.
If instead all processing is done
with the appropriate \textsf{childdoc} directives,
the argument of \textit{main} of |\childdocmain| can be empty.

An alternative version of the command line processing described
in \secref{sec:commandline} using the detection mechanism reads:
%
\begin{center}
|... -jobname "|\textit{target}|" "|[\textit{flags}]%
[|\def\jobname{|\textit{dest}|}|]|\input{|\textit{main}|}"|
\end{center}

%%%%%%%%%%%%%%%%%%%%%%%%%%%%%%%%%%%%%%%%%%%%%%%%%%%%%%%%%%%%%%%%%%%%%%%%%%%%%%%%
\subsection{Manual Code}
\label{sec:manual}

In case one cannot be certain whether the definitions file |childdoc.def|
is installed on the target \TeX{} distribution
and one prefers not to ship it,
it is conceivable to paste a few relevant commands into the sources.

To that end, drop all statements |\input{childdoc.def}|
and perform the replacements as outlined below.
Instead of |\childdocmain{|\textit{main}|}| add the following code
to the top of the main file:
%
\begin{center}
\begin{tabular}{l}
|\||ifdefined\childdocname\endinput\||fi\newif\ifchilddoc|\\
|\edef\childdocname{\scantokens\expandafter{\jobname\noexpand}}|\\
|\def\childdocmain{|\textit{main}|}\||ifx\childdocmain\childdocname\||else|\\
|\childdoctrue\includeonly{\childdocname}\let\jobname\childdocmain\||fi|\\
\end{tabular}
\end{center}
%
Instead of |\childdocof{|\textit{main}|}| just include the main file
at the top of each child file:
%
\begin{center}
|\input{|\textit{main}|}|
\end{center}
%
A simple redirection |\childdocforward{|\textit{dest}|}| is achieved by:
%
\begin{center}
|\def\jobname{|\textit{dest}|}\input{\jobname}|
\end{center}
%
The redirection with prefix
|\childdocforwardprefix[|\textit{prefix}|]{|\textit{dest}|}|
is accomplished by:
%
\begin{center}
\begin{tabular}{l}
|{\edef\jobname{\scantokens\expandafter{\jobname\noexpand}}|\\
|\def\redirectjob |\textit{prefix}|#1~~~{\gdef\jobname{|\textit{dest}|#1}}|\\
|\expandafter\redirectjob\jobname~~~}\input{\jobname}|
\end{tabular}
\end{center}

In an alternative approach,
child documents can be compiled by a specific command line
without additional code or specific definitions:
%
\begin{center}
|... -jobname "|\textit{target}|" "|[\textit{flags}]%
|\includeonly{|\textit{dest}|}\input{|\textit{main}|}"|
\end{center}
%

%%%%%%%%%%%%%%%%%%%%%%%%%%%%%%%%%%%%%%%%%%%%%%%%%%%%%%%%%%%%%%%%%%%%%%%%%%%%%%%%
%%%%%%%%%%%%%%%%%%%%%%%%%%%%%%%%%%%%%%%%%%%%%%%%%%%%%%%%%%%%%%%%%%%%%%%%%%%%%%%%
\section{Information}

%%%%%%%%%%%%%%%%%%%%%%%%%%%%%%%%%%%%%%%%%%%%%%%%%%%%%%%%%%%%%%%%%%%%%%%%%%%%%%%%
\subsection{Copyright}

Copyright \copyright{} 2017--2018 Niklas Beisert

This work may be distributed and/or modified under the
conditions of the \LaTeX{} Project Public License, either version 1.3
of this license or (at your option) any later version.
The latest version of this license is in
  \url{http://www.latex-project.org/lppl.txt}
and version 1.3 or later is part of all distributions of \LaTeX{}
version 2005/12/01 or later.

This work has the LPPL maintenance status `maintained'.

The Current Maintainer of this work is Niklas Beisert.

This work consists of the files |README.txt|, |childdoc.ins| and |childdoc.dtx|
as well as the derived files |childdoc.def|, |cdocsamp.tex|
with |cdocsch1.tex|, |cdocsch2.tex|, |cdocspt3.tex|, |cdocspt4.tex|,
|cdocsdrf.tex|, |cdocsfn1.tex|, |cdocsfn2.tex|
as well as |childdoc.pdf|.

%%%%%%%%%%%%%%%%%%%%%%%%%%%%%%%%%%%%%%%%%%%%%%%%%%%%%%%%%%%%%%%%%%%%%%%%%%%%%%%%
\subsection{Files and Installation}

The package consists of the files:
%
\begin{center}
\begin{tabular}{ll}
    |README.txt|   & readme file \\
    |childdoc.ins| & installation file \\
    |childdoc.dtx| & source file \\
    |childdoc.def| & definition file \\
    |cdocsamp.tex| & sample main file \\
    |cdocsch1.tex| & sample include file \\
    |cdocsch2.tex| & sample include file \\
    |cdocspt3.tex| & sample part file \\
    |cdocspt4.tex| & sample part file \\
    |cdocsdrf.tex| & sample redirection file \\
    |cdocsfn1.tex| & sample redirection file \\
    |cdocsfn2.tex| & sample redirection file \\
    |childdoc.pdf| & manual
\end{tabular}
\end{center}
%
The distribution consists of the files
|README.txt|, |childdoc.ins| and |childdoc.dtx|.
%
\begin{itemize}
\item
Run (pdf)\LaTeX{} on |childdoc.dtx|
to compile the manual |childdoc.pdf| (this file).
\item
Run \LaTeX{} on |childdoc.ins| to create the definitions file |childdoc.def|
and the sample |cdocsamp.tex| with include files
|cdocsch1.tex|, |cdocsch2.tex|, |cdocspt3.tex|, |cdocspt4.tex|,
|cdocsdrf.tex|, |cdocsfn1.tex|, |cdocsfn2.tex|.
Then copy the file |childdoc.def| to an appropriate directory of your \LaTeX{}
distribution, e.g.\ \textit{texmf-root}|/tex/latex/childdoc|.
\end{itemize}

%%%%%%%%%%%%%%%%%%%%%%%%%%%%%%%%%%%%%%%%%%%%%%%%%%%%%%%%%%%%%%%%%%%%%%%%%%%%%%%%
\subsection{Related CTAN Packages}

There are several other packages which offer a similar functionality:
%
\begin{itemize}
\item
The packages
\href{http://ctan.org/pkg/docmute}{\textsf{docmute}},
\href{http://ctan.org/pkg/includex}{\textsf{includex}} and
\href{http://ctan.org/pkg/standalone}{\textsf{standalone}}
provide commands to include only the document body of
a child file thus allowing both files to be compiled individually.
\item
The packages \href{http://ctan.org/pkg/subdocs}{\textsf{subdocs}}
and \href{http://ctan.org/pkg/subfiles}{\textsf{subfiles}}
provide structures in which the main and child documents can be
encapsulated and allowing them to be compiled individually.
The inclusion mechanism is different from the conventional |\include|.
\item
The package \href{http://ctan.org/pkg/combine}{\textsf{combine}}
is an elaborate solution to combine several documents into one.
\end{itemize}
%
See also the CTAN topic \href{http://ctan.org/topic/subdocs}{\textsf{subdocs}}
for further related packages.
The present package differs from the above solutions in that
a document structure constructed with the conventional |\include| mechanism
just needs two extra commands at the top of every file
such that all constituent files can be compiled individually.

%%%%%%%%%%%%%%%%%%%%%%%%%%%%%%%%%%%%%%%%%%%%%%%%%%%%%%%%%%%%%%%%%%%%%%%%%%%%%%%%
%\subsection{Feature Suggestions}
%
%The following is a list of features which may be useful for future
%versions of this package:
%%
%\begin{itemize}
%\item
%\ldots
%\end{itemize}

%%%%%%%%%%%%%%%%%%%%%%%%%%%%%%%%%%%%%%%%%%%%%%%%%%%%%%%%%%%%%%%%%%%%%%%%%%%%%%%%
\subsection{Revision History}

%%%%%%%%%%%%%%%%%%%%%%%%%%%%%%%%%%%%%%%%
\paragraph{v2.0:} 2018/12/30

\begin{itemize}
\item
immediate forward processing
\item
added |\childdocby| mechanism
\item
manual restructured
\end{itemize}

%%%%%%%%%%%%%%%%%%%%%%%%%%%%%%%%%%%%%%%%
\paragraph{v1.6:} 2018/01/17

\begin{itemize}
\item
application for development of include files
\item
corrections to manual
\end{itemize}

%%%%%%%%%%%%%%%%%%%%%%%%%%%%%%%%%%%%%%%%
\paragraph{v1.5:} 2017/05/21

\begin{itemize}
\item
more complete structuring introduced
\item
|\childdocof| introduced
\item
|\childdoc| renamed to |\childdocmain|
\item
|\childredirect| renamed to |\childdocforward| and |\childdocforwardprefix|
and functionality expanded
\end{itemize}

%%%%%%%%%%%%%%%%%%%%%%%%%%%%%%%%%%%%%%%%
\paragraph{v1.0:} 2017/04/27

\begin{itemize}
\item
manual and install package
\item
first version published on CTAN
\end{itemize}

%%%%%%%%%%%%%%%%%%%%%%%%%%%%%%%%%%%%%%%%
\paragraph{v0.6:} 2017/04/26

\begin{itemize}
\item
redirection mechanism added
\end{itemize}

%%%%%%%%%%%%%%%%%%%%%%%%%%%%%%%%%%%%%%%%
\paragraph{v0.5:} 2017/04/26

\begin{itemize}
\item
functionality in definition file
\end{itemize}


%%%%%%%%%%%%%%%%%%%%%%%%%%%%%%%%%%%%%%%%%%%%%%%%%%%%%%%%%%%%%%%%%%%%%%%%%%%%%%%%
%%%%%%%%%%%%%%%%%%%%%%%%%%%%%%%%%%%%%%%%%%%%%%%%%%%%%%%%%%%%%%%%%%%%%%%%%%%%%%%%
%%%%%%%%%%%%%%%%%%%%%%%%%%%%%%%%%%%%%%%%%%%%%%%%%%%%%%%%%%%%%%%%%%%%%%%%%%%%%%%%
\appendix

\settowidth\MacroIndent{\rmfamily\scriptsize 000\ }

 \DocInput{childdoc.dtx}

\end{document}
%</driver>
% \fi
%
% %%%%%%%%%%%%%%%%%%%%%%%%%%%%%%%%%%%%%%%%%%%%%%%%%%%%%%%%%%%%%%%%%%%%%%%%%%%%%%
% %%%%%%%%%%%%%%%%%%%%%%%%%%%%%%%%%%%%%%%%%%%%%%%%%%%%%%%%%%%%%%%%%%%%%%%%%%%%%%
% \section{Sample}
%\iffalse
%<*samplemain>
%\fi
%
% The following presents a sample document
% with two chapters, two parts, a title page,
% a compile flag as well as three forwarding files to set the flag.
% It consists of eight |.tex| files:
% \begin{center}
% \begin{tabular}{ll}
% |cdocsamp.tex|&main file\\
% |cdocsch1.tex|&include file for chapter 1\\
% |cdocsch2.tex|&include file for chapter 2\\
% |cdocspt3.tex|&include file for part 3\\
% |cdocspt4.tex|&include file for part 4\\
% |cdocsdrf.tex|&forwarding file for main file in draft mode\\
% |cdocsfi1.tex|&forwarding file for final version of chapter 1\\
% |cdocsfi2.tex|&forwarding file for final version of chapter 2\\
% \end{tabular}
% \end{center}
% Each of the eight files can be compiled directly by the \LaTeX{} compiler.
%
% %%%%%%%%%%%%%%%%%%%%%%%%%%%%%%%%%%%%%%
% \paragraph{Main File.}
%
% The main file is called |cdocsamp.tex|.
%
% Load the \textsf{childdoc} definitions and
% declare the filename for the main document:
%    \begin{macrocode}
\input{childdoc.def}
\childdocmain{}
%    \end{macrocode}

% Optional override for |\version| flag:
%    \begin{macrocode}
%%\ifchilddoc\else\providecommand{\version}{draft}\fi
%    \end{macrocode}

% Define the default values for the |\version| flag
% (|final| for the main file and |draft| for childs):
%    \begin{macrocode}
\ifchilddoc
\providecommand{\version}{draft}
\else
\providecommand{\version}{final}
\fi
%    \end{macrocode}

% Load the standard document class:
%    \begin{macrocode}
\documentclass[12pt]{article}
%    \end{macrocode}

% Start the document body:
%    \begin{macrocode}
\begin{document}
%    \end{macrocode}

% Declare a title page.
% Print title, part of document being processed and version flag:
%    \begin{macrocode}
\addtocounter{page}{-1}
\begin{center}
{\LARGE\bfseries{}childdoc example\par}
\vspace{1cm}
\ifchilddoc
\ifchilddocmanual part\else chapter\fi:
`\childdocname' of `\childdocjob'\par
\else
main document: `\childdocjob'\par
\fi
version: \version\par
\end{center}
\newpage
%    \end{macrocode}

% Manually include selected file,
% otherwise process as usual:
%    \begin{macrocode}
\ifchilddocmanual
\section*{part `\childdocname'}
\input{\childdocname}
\else
%    \end{macrocode}

% Include the two chapters:
%    \begin{macrocode}
\include{cdocsch1}
\include{cdocsch2}
%    \end{macrocode}

% Include the two parts unless only chapters should be displayed:
%    \begin{macrocode}
\ifchilddoc\else
\section{part three}
\input{cdocspt3}
\section{part four}
\input{cdocspt4}
\fi
%    \end{macrocode}

% Process as usual until here:
%    \begin{macrocode}
\fi
%    \end{macrocode}

% End of document body:
%    \begin{macrocode}
\end{document}
%    \end{macrocode}
%\iffalse
%</samplemain>
%\fi
%
% %%%%%%%%%%%%%%%%%%%%%%%%%%%%%%%%%%%%%%
% \paragraph{Chapter Include Files.}
%
% The include files are called |cdocsch1.tex| and |cdocsch2.tex|.
%
%\iffalse
%<*samplechap1|samplechap2>
%\fi

% Optional override for |\version| flag:
%    \begin{macrocode}
%%\providecommand{\version}{final}
%    \end{macrocode}

% Include the main document:
%    \begin{macrocode}
\input{childdoc.def}
\childdocof{cdocsamp}
%    \end{macrocode}

%\iffalse
%</samplechap1|samplechap2>
%\fi
%
%\iffalse
%<*samplechap1>
%\fi
% Some text for chapter 1:
%    \begin{macrocode}
\section{one}
some text in chapter one
%    \end{macrocode}

%\iffalse
%</samplechap1>
%\fi
% Some text for chapter 2:
%\iffalse
%<*samplechap2>
%\fi
%    \begin{macrocode}
\section{two}
more text in chapter two
%    \end{macrocode}

%\iffalse
%</samplechap2>
%\fi
%
% %%%%%%%%%%%%%%%%%%%%%%%%%%%%%%%%%%%%%%
% \paragraph{Part Include Files.}
%
% The include files are called |cdocspt3.tex| and |cdocspt4.tex|.
%
%\iffalse
%<*samplepart3|samplepart4>
%\fi

% Optional override for |\version| flag:
%    \begin{macrocode}
%%\providecommand{\version}{final}
%    \end{macrocode}

% Include the main document:
%    \begin{macrocode}
\input{childdoc.def}
\childdocby{cdocsamp}
%    \end{macrocode}

%\iffalse
%</samplepart3|samplepart4>
%\fi
%
%\iffalse
%<*samplepart3>
%\fi
% Some text for part 3:
%    \begin{macrocode}
some text in part three
%    \end{macrocode}

%\iffalse
%</samplepart3>
%\fi
% Some text for part 4:
%\iffalse
%<*samplepart4>
%\fi
%    \begin{macrocode}
more text in part four
%    \end{macrocode}

%\iffalse
%</samplepart4>
%\fi
%
% %%%%%%%%%%%%%%%%%%%%%%%%%%%%%%%%%%%%%%
% \paragraph{Forwarding for a Complete Draft.}
%
% The following forwarding file |cdocsdrf.tex|
% compiles the main document in draft mode:
%\iffalse
%<*sampledraft>
%\fi
%    \begin{macrocode}
\def\version{draft}
\input{childdoc.def}
\childdocforward{cdocsamp}
%    \end{macrocode}

%\iffalse
%</sampledraft>
%\fi
%
% %%%%%%%%%%%%%%%%%%%%%%%%%%%%%%%%%%%%%%
% \paragraph{Forwarding for Final Version of the Chapters.}
%
% The following forwarding files |cdocsfn1.tex| and |cdocsfn2.tex|
% (with identical content)
% compile the final versions of the child documents
% |cdocsch1.tex| and |cdocsch2.tex|, respectively:
%\iffalse
%<*samplefinal>
%\fi
%    \begin{macrocode}
\def\version{final}
\input{childdoc.def}
\childdocforwardprefix[cdocsamp]{cdocsfn}{cdocsch}
%    \end{macrocode}

%\iffalse
%</samplefinal>
%\fi
%
% %%%%%%%%%%%%%%%%%%%%%%%%%%%%%%%%%%%%%%
% \paragraph{Command Line Processing.}
%
% The following three command lines generate the output files
% |cdocscld|, |cdocscl1| and |cdocscl2|
% which should be identical to
% |cdocsdrf|, |cdocsch1| and |cdocsfn2|, respectively:
% \begin{center}
% \begin{tabular}{l}
% |latex -jobname cdocscld \|\\
% |  "\def\version{draft}\input{childdoc.def}\childdocforward{cdocsamp}"|\\
% |latex -jobname cdocscl1 \|\\
% |  "\input{childdoc.def}\childdocforward[cdocsamp]{cdocsch1}"|\\
% |latex -jobname cdocscl2 \|\\
% |  "\def\version{final}\input{childdoc.def}\childdocforward{cdocsch2}"|
% \end{tabular}
% \end{center}
% Note that the trailing backslash on each first line
% merely continues the input to the second line
% (for convenient cut ant paste).
% Furthermore, the command |latex| can be replaced by any
% of its alternative versions such as |pdflatex|.
%
% %%%%%%%%%%%%%%%%%%%%%%%%%%%%%%%%%%%%%%%%%%%%%%%%%%%%%%%%%%%%%%%%%%%%%%%%%%%%%%
% %%%%%%%%%%%%%%%%%%%%%%%%%%%%%%%%%%%%%%%%%%%%%%%%%%%%%%%%%%%%%%%%%%%%%%%%%%%%%%
% \section{Implementation}
%\iffalse
%<*package>
%\fi
%
% This section describes the definitions file |childdoc.def|.

% The definitions cannot be loaded using |\usepackage| or |\RequirePackage|
% which has a mechanism to prevent loading a style file more than once.
% When loading the definitions by means of |\input|
% multiple instances have to be prevented manually:
%\iffalse
%This code needs to be before the `\ProvidesFile' directive
%which is defined at the beginning of this file.
%Therefore it is also placed there and commented out here.
%</package>
%<*discard>
%\fi
%    \begin{macrocode}
\ifdefined\childdocmain\endinput\fi
%    \end{macrocode}
%\iffalse
%</discard>
%<*package>
%\fi
%
% \macro{\ifchilddoc}
% \macro{\ifchilddocmanual}
% The conditional |\ifchilddoc| tells whether a
% child (true) or main (false) document is being compiled.
% The conditional |\ifchilddocmanual| tells whether
% the |\includeonly| mechanism is used (false) or
% the selection of child files must be performed manually (true).
% The definitions initialise to false:
%    \begin{macrocode}
\newif\ifchilddoc
\newif\ifchilddocmanual
%    \end{macrocode}

% \macro{\childdocname}
% \macro{\childdocjob}
% The macro |\childdocname| stores the name of the main document
% to be compiled. The macro |\childdocjob| stores the name of
% the document on which the \LaTeX{} compiler was originally invoked.
% The content of |\jobname| cannot be compared
% to filenames specified in the source due to different catcodes.
% The following code rescans |\jobname|, stores the result
% in |\childdocname| and saves a copy in |\childdocjob|:
%    \begin{macrocode}
\edef\childdocname{\scantokens\expandafter{\jobname\noexpand}}
\let\childdocjob\childdocname
%    \end{macrocode}

% \macro{\childdocdisable}
% The macro |\childdocdisable| prevents the main file
% from being processed more than once.
% At this stage, the main document command |\childdocmain|
% is assumed to be called once again where it should do nothing.
% Any subsequent call to it should prevent
% a secondary processing of the main document
% It overwrites the forwarding commands
% |\childdocof| and |\childdocforward|
% with empty macros to prevent further inclusions of the main document:
%    \begin{macrocode}
\newcommand{\childdocdisable}
{
  \renewcommand{\childdocmain}[1]{\renewcommand{\childdocmain}[1]{\endinput}}
  \renewcommand{\childdocof}[1]{}
  \renewcommand{\childdocby}[2][]{}
  \renewcommand{\childdocforward}[2][]{}
  \renewcommand{\childdocdisable}{}
}
%    \end{macrocode}

% \macro{\childdocmain}
% The macro |\childdocmain| is to be called at the top of the main file
% with nothing or the main filename (without extension) as argument.
% First, it breaks loops.
% If the argument is not empty and does not match |\childdocname|
% (which is set by the first inclusion of |childdoc.def|),
% |\ifchilddoc| is set to true, |\includeonly| is applied to the child file
% and |\jobname| is set to the main file
% (for proper handling of |.aux| files):
%    \begin{macrocode}
\newcommand{\childdocmain}[1]
{
  \childdocdisable\childdocmain{}
  \if?#1?\else
    \begingroup
      \def\childdoctmp{#1}
      \ifx\childdoctmp\childdocname
        \def\childdoctmp{}
      \else
        \def\childdoctmp
        {
          \childdoctrue
          \includeonly{\childdocname}
          \def\childdocjob{#1}
          \def\jobname{#1}
        }
      \fi
      \expandafter
    \endgroup
    \childdoctmp
  \fi
}
%    \end{macrocode}

% \macro{\childdocof}
% The command |\childdocof| redirects
% compilation to the main file |#1|.
%    \begin{macrocode}
\newcommand{\childdocof}[1]
{
  \childdocdisable
  \childdoctrue
  \includeonly{\childdocname}
  \def\jobname{#1}
  \def\childdocjob{#1}
  \input{#1}
}
%    \end{macrocode}

% \macro{\childdocby}
% The command |\childdocby| ....
%    \begin{macrocode}
\newcommand{\childdocby}[2][]
{
  \childdocdisable
  \childdoctrue
  \childdocmanualtrue
  \if?#1?\else
    \def\jobname{#2}
  \fi
  \def\childdocjob{#2}
  \input{#2}
  \endinput
}
%    \end{macrocode}

% \macro{\childdocforward}
% The command |\childdocforward| redirects
% compilation to the main file or
% (if the optional argument is given) a child file.
% Parameters are set as if the main file
% or a child file starting with |\childdocof| was compiled.
% Then compilation is handed over to the main file:
%    \begin{macrocode}
\newcommand{\childdocforward}[2][]
{
  \begingroup
    \if?#1?
      \def\childdoctmp
      {
        \def\childdocname{#2}
        \def\childdocjob{#2}
        \def\jobname{#2}
        \input{#2}
        \endinput
      }
    \else
      \def\childdoctmp
      {
        \childdocdisable
        \def\childdocname{#2}
        \childdoctrue
        \includeonly{#2}
        \def\childdocjob{#1}
        \def\jobname{#1}
        \input{#1}
        \endinput
      }
    \fi
    \expandafter
  \endgroup
  \childdoctmp
}
%    \end{macrocode}

% \macro{\childdocforwardprefix}
% The command |\childdocforwardprefix| redirects
% compilation to the main or a child file by means of a pattern.
% The prefix |#1| in the current filename is replaced by |#2|
% and the suffix of the current filename is kept
% (it is assumed that the filename does not contain the substring `|~~~|'
% which is used as a delimiter).
% Compilation is handed over to the new file by |\childdocforward|:
%    \begin{macrocode}
\newcommand{\childdocforwardprefix}[3][]
{
  \begingroup
    \def\childdocextract #2##1~~~{\def\childdoctmp{\childdocforward[#1]{#3##1}}}
    \expandafter\childdocextract\childdocname~~~
    \expandafter
  \endgroup
  \childdoctmp
}
%    \end{macrocode}

% \macro{\childdoc}
% The deprecated macro |\childdoc| is a legacy version of |\childdocmain|:
%    \begin{macrocode}
\newcommand{\childdoc}{\childdocmain}
%    \end{macrocode}

% \macro{\childdocredirect}
% The deprecated macro |\childdocredirect| is a legacy version
% of |\childdocforward| and |\childdocforwardprefix|:
%    \begin{macrocode}
\newcommand{\childdocredirect}[2][]
{
  \begingroup
    \if?#1?
      \def\childdoctmp{\childdocforward{#2}}
    \else
      \def\childdoctmp{\childdocforwardprefix{#1}{#2}}
    \fi
    \expandafter
  \endgroup
  \childdoctmp
}
%    \end{macrocode}

%\iffalse
%</package>
%\fi
%
\endinput
\childdocforward[|\textit{main}|]{|\textit{dest}|}"|
\end{center}
%
Here \textit{target} is the name of the output file,
\textit{main} is the name of the main file
and \textit{dest} is the name of the main or child file to be processed
(all filenames without extensions).
The optional argument \textit{main} can be omitted
if \textit{main} matches \textit{dest}.
Optionally, compilation \textit{flags} can be defined via |\def| commands.
This command line makes the \TeX{} engine believe
it is compiling the file \textit{target}
whose content is specified as the latter parameter.
The provided code then forwards the processing to
\textit{main} or \textit{dest} as described in \secref{sec:forward}.

%%%%%%%%%%%%%%%%%%%%%%%%%%%%%%%%%%%%%%%%%%%%%%%%%%%%%%%%%%%%%%%%%%%%%%%%%%%%%%%%
\subsection{Include by Input}
\label{sec:input}

Including child documents by |\include| has some restrictions by design.
Most notably, the content of a child document always occupies
its own set of pages; pages cannot be shared between child documents.
Usually, this behaviour makes perfect sense
because each child document contain an essential part of the document.
However, in some situations it may be desirable to compose
a document from a collection of parts
without having mandatory page breaks between then.
For this case, the package
provides a mechanism to include parts
by |\input| which can also be processed individually.
However, by construction this mechanism
requires manual handling of the content to be output.

%%%%%%%%%%%%%%%%%%%%%%%%%%%%%%%%%%%%%%%%
\DescribeMacro{\ifchilddocmanual}
The main file should be prepared as usual, see \secref{sec:include}.
However, the document body must make a distinction
between processing of an individual part and of the main document, e.g.:
%
\begin{center}
\begin{tabular}{l}
|\ifchilddocmanual|\\
|\input{\childdocname}|\\
|\||else|\\
\textit{document body with }|\input{|\textit{part}|}|\\
|\||fi|
\end{tabular}
\end{center}
%
The conditional |\ifchilddocmanual| is true whenever
a part to be included by |\input| is being compiled,
and the name of the part is stored in |\childdocname|.

%%%%%%%%%%%%%%%%%%%%%%%%%%%%%%%%%%%%%%%%
\DescribeMacro{\childdocby}
Each part to be included by |\input| should start with:
%
\begin{center}
\begin{tabular}{l}
|% \iffalse
%
% childdoc.dtx Copyright (C) 2017-2018 Niklas Beisert
%
% This work may be distributed and/or modified under the
% conditions of the LaTeX Project Public License, either version 1.3
% of this license or (at your option) any later version.
% The latest version of this license is in
%   http://www.latex-project.org/lppl.txt
% and version 1.3 or later is part of all distributions of LaTeX
% version 2005/12/01 or later.
%
% This work has the LPPL maintenance status `maintained'.
%
% The Current Maintainer of this work is Niklas Beisert.
%
% This work consists of the files childdoc.dtx and childdoc.ins
% and the derived files childdoc.def and cdocsamp.tex with
% cdocsch1.tex, cdocsch2.tex, cdocsdrf.tex, cdocsfn1.tex, cdocsfn2.tex.
%
%<package>\ifdefined\childdocmain\endinput\fi
%<package>\ProvidesFile{childdoc.def}[2018/12/30 v2.0 child document driver]
%<samplemain>\ProvidesFile{cdocsamp.tex}[2018/12/30 v2.0 sample for childdoc]
%<*driver>
%\ProvidesFile{childdoc.drv}[2018/12/30 v2.0 childdoc reference manual file]
\PassOptionsToClass{10pt,a4paper}{article}
\documentclass{ltxdoc}

\usepackage[margin=35mm]{geometry}
\usepackage{hyperref}
\usepackage{hyperxmp}
\usepackage[usenames]{color}

\hypersetup{colorlinks=true}
\hypersetup{pdfstartview=FitH}
\hypersetup{pdfpagemode=UseNone}
\hypersetup{pdfsource={}}
\hypersetup{pdflang={en-UK}}
\hypersetup{pdfcopyright={Copyright 2017-2018 Niklas Beisert.
  This work may be distributed and/or modified under the
  conditions of the LaTeX Project Public License, either version 1.3
  of this license or (at your option) any later version.}}
\hypersetup{pdflicenseurl={http://www.latex-project.org/lppl.txt}}
\hypersetup{pdfcontactaddress={ETH Zurich, ITP, HIT K,
  Wolfgang-Pauli-Strasse 27}}
\hypersetup{pdfcontactpostcode={8093}}
\hypersetup{pdfcontactcity={Zurich}}
\hypersetup{pdfcontactcountry={Switzerland}}
\hypersetup{pdfcontactemail={nbeisert@itp.phys.ethz.ch}}
\hypersetup{pdfcontacturl={http://people.phys.ethz.ch/\xmptilde nbeisert/}}

\newcommand{\secref}[1]{\hyperref[#1]{section \ref*{#1}}}

\parskip1ex
\parindent0pt
\let\olditemize\itemize
\def\itemize{\olditemize\parskip0pt}

\begin{document}

\title{The \textsf{childdoc} Package}
\hypersetup{pdftitle={The childdoc Package}}
\author{Niklas Beisert\\[2ex]
  Institut f\"ur Theoretische Physik\\
  Eidgen\"ossische Technische Hochschule Z\"urich\\
  Wolfgang-Pauli-Strasse 27, 8093 Z\"urich, Switzerland\\[1ex]
  \href{mailto:nbeisert@itp.phys.ethz.ch}
  {\texttt{nbeisert@itp.phys.ethz.ch}}}
\hypersetup{pdfauthor={Niklas Beisert}}
\hypersetup{pdfsubject={Manual for the LaTeX2e Package childdoc}}
\date{30 December 2018, \textsf{v2.0}}
\maketitle

\begin{abstract}\noindent
\textsf{childdoc} is a \LaTeXe{} package
that enables the direct compilation
of document sections included by |\include|
to individual files.
\end{abstract}

\begingroup
\parskip0ex
\tableofcontents
\endgroup

%%%%%%%%%%%%%%%%%%%%%%%%%%%%%%%%%%%%%%%%%%%%%%%%%%%%%%%%%%%%%%%%%%%%%%%%%%%%%%%%
%%%%%%%%%%%%%%%%%%%%%%%%%%%%%%%%%%%%%%%%%%%%%%%%%%%%%%%%%%%%%%%%%%%%%%%%%%%%%%%%
\section{Introduction}

\LaTeX{} provides a mechanism to structure a large document (such as a book)
into a main file and several child files (containing the chapters)
using the |\include| command.
This mechanism is beneficial for documents
which span hundreds of pages in order to
make the source file(s) more manageable.
Moreover, compilation can be restricted to
selected child files by means of the |\includeonly| command.
The latter feature can be used to reduce the compilation time while editing
(this was significantly more useful in the earlier days of \LaTeX{})
or to generate a smaller document which is easier to navigate.
Another application of |\includeonly| is to generate
documents consisting of selected parts of the complete document.

However, there are a few drawbacks of the plain |\include| mechanism:
\begin{itemize}
\item
The child files cannot be compiled on their own,
they can only be compiled via the main file.
A naive editing environment
(such as a text editor with an option
to have the current file processed by \LaTeX)
may require one to switch to the main file before compiling;
attempting to compile the child file produces errors.
\item
The main file must be modified (each time)
to adjust the |\includeonly| command
to the present needs. This easily leaves the main file in a messy state.
\item
The generated document will always carry the filename
of the main document. This is inconvenient if
several child files are to be compiled and
to be kept for distribution.
\end{itemize}

The present package provides a simple interface
to make child files individually compilable by \LaTeX{}.
Compiling a child file then has the same effect as compiling
the main file with an |\includeonly| command
to select the appropriate child.
Moreover the generated document will carry the name of the child
rather than the main file.
This resolves all three above issues.

This feature is meant to make the editing of books,
thesis documents and lecture notes somewhat more convenient.
However, the package can also be used efficiently for
composing a series of documents (such as exercise sheets)
which are typically distributed individually.
It then assists the author in generating the individual documents
(potentially in different versions)
as well as a document containing the collected series.
Another application is in developing style files
or other kinds of included material
where compilation of the style file could redirect
to a sample or test file.

%%%%%%%%%%%%%%%%%%%%%%%%%%%%%%%%%%%%%%%%%%%%%%%%%%%%%%%%%%%%%%%%%%%%%%%%%%%%%%%%
%%%%%%%%%%%%%%%%%%%%%%%%%%%%%%%%%%%%%%%%%%%%%%%%%%%%%%%%%%%%%%%%%%%%%%%%%%%%%%%%
\section{Usage}

First of all, the package \textsf{childdoc} is \emph{not} a standard
\LaTeXe{} |.sty| style file! Therefore it needs to be invoked in
a non-standard way.

%%%%%%%%%%%%%%%%%%%%%%%%%%%%%%%%%%%%%%%%%%%%%%%%%%%%%%%%%%%%%%%%%%%%%%%%%%%%%%%%
\subsection{Included Files}
\label{sec:include}

%%%%%%%%%%%%%%%%%%%%%%%%%%%%%%%%%%%%%%%%
\DescribeMacro{\childdocmain}
To use the package, add the commands
\begin{center}
\begin{tabular}{l}
|\input{childdoc.def}|\\
|\childdocmain{}|\\
\end{tabular}
\end{center}
at the very top of the main \LaTeX{} file,
in particular \emph{before} the |\documentclass| statement!
The argument of |\childdocmain| should be left empty
(but it must be present).

%%%%%%%%%%%%%%%%%%%%%%%%%%%%%%%%%%%%%%%%
\DescribeMacro{\childdocof}
Furthermore, add the commands
\begin{center}
\begin{tabular}{l}
|\input{childdoc.def}|\\
|\childdocof{|\textit{main}|}|\\
\end{tabular}
\end{center}
at the top of every child file \textit{child}
which is included by |\include{|\textit{child}|}|
from within the main file
(or at least for those files to be compiled individually).
The argument \textit{main} must be the filename of the main file.

There are a couple of
considerations in setting up the main and child documents:

%%%%%%%%%%%%%%%%%%%%%%%%%%%%%%%%%%%%%%%%
\paragraph{Restrictions.}

Please note the following restrictions:
\begin{itemize}
\item
|\childdocmain| must be called with one argument \textit{main}
to ensure compatibility with earlier version of the package.
It must either be empty (|\childdocmain{}|)
or precisely match the filename of the main file in which it is specified.
See \secref{sec:detection} for further information.
\item
The filename \textit{main} must be specified without the |.tex| extension.
\item
The filename \textit{main} is case sensitive
(even in case-insensitive file systems)
due to internal string comparison.
\item
The argument \textit{main} should be fully expanded, it cannot be a macro.
\item
Subdirectories and special characters should be avoided in filenames.
\item
The command |\childdocmain{|\textit{main}|}| must be followed by a whitespace.
It should not be followed immediately by another command
or by a comment mark `|%|'.
This is because the \TeX{} parser reads the token immediately following
the argument of |\childdocmain| and puts it
at the beginning of every child section;
however, a white\-space is ignored.
\end{itemize}

%%%%%%%%%%%%%%%%%%%%%%%%%%%%%%%%%%%%%%%%
\paragraph{Content of Main File.}

It is advisable to place all content in the child files included by |\include|.
Any output contained in the main file will appear in all child documents
unless suppressed manually;
it cannot be suppressed automatically by the |\includeonly| directive
and thus should normally be avoided.
A method to include some content in the main file
by means of conditional processing is described in \secref{sec:conditional}.

%%%%%%%%%%%%%%%%%%%%%%%%%%%%%%%%%%%%%%%%
\paragraph{Page Numbering.}

When only a part of the document is compiled,
the appropriate numbering of pages
(as well as other status parameters)
is determined from the |.aux| files.
The latter contain information from previous passes.
However this information needs to propagate through
all intermediate child documents.
Therefore the page numbering in child documents may well
be inconsistent until the complete document is compiled at least once.

A useful (if unconventional) way to always ensure a consistent
page numbering is to restart the numbering in each child document
and denote the pages by `\textit{child}|.|\textit{page}'
where \textit{child} represents the chapter/section number of the child file.
This can be achieved by the command
|\numberwithin{page}{|\textit{child}|}|
of the \textsf{amsmath} package
where \textit{child} can be |chapter| or |section|
depending on the chosen structuring.
Alternatively, one can modify the macro |\thepage| appropriately
and reset the counter |page| at the start of each child file.

%%%%%%%%%%%%%%%%%%%%%%%%%%%%%%%%%%%%%%%%%%%%%%%%%%%%%%%%%%%%%%%%%%%%%%%%%%%%%%%%
\subsection{Conditional Processing}
\label{sec:conditional}

The package provides a mechanism to compile different versions
of a document. To customise the versions further some conditional processing
can come in handy to distinguish which version is being compiled.
The package provides two macros to describe the compilation context:

%%%%%%%%%%%%%%%%%%%%%%%%%%%%%%%%%%%%%%%%
\DescribeMacro{\ifchilddoc}
The conditional |\ifchilddoc| distinguishes between the compilation of
child documents and the main document:
%
\begin{center}
|\ifchilddoc |\textit{child-code}| |[|\||else |\textit{main-code}]| \||fi|
\end{center}

%%%%%%%%%%%%%%%%%%%%%%%%%%%%%%%%%%%%%%%%
\DescribeMacro{\childdocname}
\DescribeMacro{\childdocjob}
The macro |\childdocname| contains the filename (without extension)
of the main or child file being processed.
Note that |\childdocjob| will always contain the name of the main file.

%%%%%%%%%%%%%%%%%%%%%%%%%%%%%%%%%%%%%%%%
\paragraph{Title Page.}

Conditional processing can be used to include a title or banner page
in the main document when proper precautions are taken.
Importantly, the code in the main file should ensure that the page counter
(as well as other status parameters which are stored in the |.aux| files)
takes the same value after the conditional processing.
Otherwise the page numbers may take divergent values
depending on which part is compiled.

For example, a title page could be declared by:
%
\begin{center}
\begin{tabular}{l}
|\ifchilddoc\||else|\\
|\addtocounter{page}{-1}|\\
\textit{code for title page}\\
|\newpage|\\
|\||fi|
\end{tabular}
\end{center}
%
A banner page for the child documents can be generated by:
%
\begin{center}
\begin{tabular}{l}
|\ifchilddoc|\\
|\addtocounter{page}{-1}|\\
\textit{code for banner page}\\
|\newpage|\\
|\||fi|
\end{tabular}
\end{center}
%
Here one could write a message such as:
\begin{center}
|This is the part \childdocname{} of \childdocjob{}.|
\end{center}

%%%%%%%%%%%%%%%%%%%%%%%%%%%%%%%%%%%%%%%%%%%%%%%%%%%%%%%%%%%%%%%%%%%%%%%%%%%%%%%%
\subsection{Flags}
\label{sec:flags}

The package makes it easy to generate different versions
of the main or child documents.
To this end compilation flags can be defined
and assigned different default values.
They will be particularly useful in conjunction
with the forwarding mechanism described in \secref{sec:forward}.

For example, it may be useful to have a flag |\version|
which can be set to |draft| or |final|.
The document source will contain some conditional code
depending on the value of |\version|.
Suppose further, the flag should default to |final| for the main file
and to |draft| for child files
which is a natural assignment for editing the document.
This is achieved by placing the following code
in the preamble of the main document
(below the |\childdocmain| directive):
%
\begin{center}
\begin{tabular}{l}
|\ifchilddoc|\\
|\providecommand{\version}{draft}|\\
|\||else|\\
|\providecommand{\version}{final}|\\
|\||fi|
\end{tabular}
\end{center}
%
The definition by |\providecommand| makes sure
that previous definitions are not overwritten.
Further statements |\providecommand{\version}{...}|
can thus be added before the above code to override it.

For the main file, one might add a line
(between |\childdocmain| and the above block)
%
\begin{center}
|%\ifchilddoc\||else\providecommand{\version}{draft}\||fi|
\end{center}
%
which can be uncommented to produce a draft version.
Likewise one can add a line to the very top of a child file
(above the |\childdocof{|\textit{main}|}| directive)
%
\begin{center}
|%\providecommand{\version}{final}|
\end{center}
%
which can be uncommented to produce the final version of this child document.

%%%%%%%%%%%%%%%%%%%%%%%%%%%%%%%%%%%%%%%%%%%%%%%%%%%%%%%%%%%%%%%%%%%%%%%%%%%%%%%%
\subsection{Forwarding}
\label{sec:forward}

Different versions of the main or child documents
using compilation flags as described in \secref{sec:flags}
can be (permanently) stored in different files
for convenient compilation, viewing and distribution.
To this end, the package defines a command
to pass on compilation to a different file:

%%%%%%%%%%%%%%%%%%%%%%%%%%%%%%%%%%%%%%%%
\DescribeMacro{\childdocforward}
The command |\childdocforward| redirects processing to
another source file:
%
\begin{center}
\begin{tabular}{l}
|\input{childdoc.def}|\\
|\childdocforward[|\textit{main}|]{|\textit{dest}|}|\\
\end{tabular}
\end{center}
%
The argument \textit{dest} is the destination file
(without extension).
It should be the main file or one of the child files.
Note that further \textsf{childdoc} directives
such as |\childdocof| and |\childdocforward|
in the indicated file will be processed in this form.
The optional argument \textit{main}
passes on directly to the main file \textit{main}
while pretending to compile the child \textit{dest}.
This form behaves as if \textit{dest}
issues |\childdocof{|\textit{main}|}| right away,
and no further \textsf{childdoc} directives will be processed.

%%%%%%%%%%%%%%%%%%%%%%%%%%%%%%%%%%%%%%%%
\DescribeMacro{\...prefix}
In the alternative form |\childdocforwardprefix|,
%
\begin{center}
\begin{tabular}{l}
|\input{childdoc.def}|\\
|\childdocforwardprefix[|\textit{main}|]{|\textit{prefix}|}{|\textit{dest}|}|
\end{tabular}
\end{center}
%
the destination file is determined by a pattern
depending on the current file:
To make this work, the current file must be called
`{\textit{prefix}\hspace{0.2em}\textit{suffix}}'
with \textit{prefix} matching precisely the argument.
Processing is then passed on to the file
`{\textit{dest}\hspace{0.2em}\textit{suffix}}'.
Surely, the same effect is achieved by
directly specifying the
argument `{\textit{dest}\hspace{0.2em}\textit{suffix}}'
in the first form.
However, that requires to set up a different file
for each child. With the alternative form of the command
all these files can have exactly the same content
which simplifies setting them up and maintaining them.

For example, the following file |draft.tex|
with a compilation flag |\version| as described in \secref{sec:flags}
compiles the main document as a draft:
%
\begin{center}
\begin{tabular}{l}
|\def\version{draft}|\\
|\input{childdoc.def}|\\
|\childdocforward{|\textit{main}|}|
\end{tabular}
\end{center}
%
Likewise, the following files |final|\textit{nn}|.tex|
compile the final version of the child document
|child|\textit{nn}|.tex|:
%
\begin{center}
\begin{tabular}{l}
|\def\version{final}|\\
|\input{childdoc.def}|\\
|\childdocforwardprefix{final}{child}|
\end{tabular}
\end{center}
%

Note that when several versions of a main file and/or of each child file
are to be generated, it may be convenient to set up a |Makefile| or
shell script to automatise the process.

%%%%%%%%%%%%%%%%%%%%%%%%%%%%%%%%%%%%%%%%%%%%%%%%%%%%%%%%%%%%%%%%%%%%%%%%%%%%%%%%
\subsection{Command Line Processing}
\label{sec:commandline}

The effect of redirection files can also be achieved by invoking
the \LaTeX{} compiler with a more elaborate command line.
Most conveniently this should be done as part
of a shell script or a |Makefile|.

When using \textsf{childdoc} in the main file, the following
command lines effectively perform a redirection
(note that depending on the shell being used,
backslashes may have to be doubled: `|\|' $\to$ `|\\|'):
%
\begin{center}
|... -jobname "|\textit{target}|" |\\|"|[\textit{flags}]%
|\input{childdoc.def}\childdocforward[|\textit{main}|]{|\textit{dest}|}"|
\end{center}
%
Here \textit{target} is the name of the output file,
\textit{main} is the name of the main file
and \textit{dest} is the name of the main or child file to be processed
(all filenames without extensions).
The optional argument \textit{main} can be omitted
if \textit{main} matches \textit{dest}.
Optionally, compilation \textit{flags} can be defined via |\def| commands.
This command line makes the \TeX{} engine believe
it is compiling the file \textit{target}
whose content is specified as the latter parameter.
The provided code then forwards the processing to
\textit{main} or \textit{dest} as described in \secref{sec:forward}.

%%%%%%%%%%%%%%%%%%%%%%%%%%%%%%%%%%%%%%%%%%%%%%%%%%%%%%%%%%%%%%%%%%%%%%%%%%%%%%%%
\subsection{Include by Input}
\label{sec:input}

Including child documents by |\include| has some restrictions by design.
Most notably, the content of a child document always occupies
its own set of pages; pages cannot be shared between child documents.
Usually, this behaviour makes perfect sense
because each child document contain an essential part of the document.
However, in some situations it may be desirable to compose
a document from a collection of parts
without having mandatory page breaks between then.
For this case, the package
provides a mechanism to include parts
by |\input| which can also be processed individually.
However, by construction this mechanism
requires manual handling of the content to be output.

%%%%%%%%%%%%%%%%%%%%%%%%%%%%%%%%%%%%%%%%
\DescribeMacro{\ifchilddocmanual}
The main file should be prepared as usual, see \secref{sec:include}.
However, the document body must make a distinction
between processing of an individual part and of the main document, e.g.:
%
\begin{center}
\begin{tabular}{l}
|\ifchilddocmanual|\\
|\input{\childdocname}|\\
|\||else|\\
\textit{document body with }|\input{|\textit{part}|}|\\
|\||fi|
\end{tabular}
\end{center}
%
The conditional |\ifchilddocmanual| is true whenever
a part to be included by |\input| is being compiled,
and the name of the part is stored in |\childdocname|.

%%%%%%%%%%%%%%%%%%%%%%%%%%%%%%%%%%%%%%%%
\DescribeMacro{\childdocby}
Each part to be included by |\input| should start with:
%
\begin{center}
\begin{tabular}{l}
|\input{childdoc.def}|\\
|\childdocby{|\textit{main}|}|\\
\end{tabular}
\end{center}
%
The directive |\childdocby| is similar to |\childdocof|
described in \secref{sec:include},
but the subsequent selection of content must be done manually.
To that end, both |\ifchilddoc| and |\ifchilddocmanual|
will be true upon processing of a part,
and the name of the part is stored in |\childdocname|.
Note that |\jobname| will be set to the filename of the current part
so that each part receives an individual |.aux| file
that does not interfere with the |.aux| file(s) of the main document.
This behaviour can be altered by the alternative form
|\childdocby[*]{|\textit{main}|}| (with a non-empty optional argument)
which uses the |.aux| file of the main document
by setting |\jobname| to \textit{main}.

%%%%%%%%%%%%%%%%%%%%%%%%%%%%%%%%%%%%%%%%%%%%%%%%%%%%%%%%%%%%%%%%%%%%%%%%%%%%%%%%
\subsection{Driver Development}
\label{sec:driver}

The \textsf{childdoc} mechanism can also be use for the development
of definition files such as \LaTeX{} styles or classes.
This case differs from the above setup with multiple parts
included by |\include| in that no |\includeonly| should be invoked.
This can be achieved by starting the include file
(before |\ProvidesPackage|) with:
%
\begin{center}
\begin{tabular}{l}
|\input{childdoc.def}|\\
|\childdocforward{|\textit{main}|}|\\
\end{tabular}
\end{center}
%
or alternatively with:
%
\begin{center}
\begin{tabular}{l}
|\input{childdoc.def}|\\
|\childdocby{|\textit{main}|}|\\
\end{tabular}
\end{center}
%
Both forms have slightly different effects as described above.
The main file is prepared as usual, see \secref{sec:include}.

%%%%%%%%%%%%%%%%%%%%%%%%%%%%%%%%%%%%%%%%%%%%%%%%%%%%%%%%%%%%%%%%%%%%%%%%%%%%%%%%
\subsection{Legacy Detection}
\label{sec:detection}

The directive |\childdocmain| in the main file can detect
whether the complete document or merely a child is to be compiled
even without using the directive |\childdocof|.
This method is deprecated because it is less robust
and there is no compelling reason to use it;
it is merely provided for backward compatibility
and it may be removed in future versions.

If the detection mechanism is to be used,
it is mandatory to correctly specify
the filename of the main file as the argument of |\childdocmain|:
%
\begin{center}
\begin{tabular}{l}
|\input{childdoc.def}|\\
|\childdocmain{|\textit{main}|}|\\
\end{tabular}
\end{center}
%
If |\jobname| does not match the argument \textit{main} of |\childdocmain|,
it is assumed that |\jobname| points to the child file to be compiled.
When using |\childdocmain| with the main file specified as argument,
it suffices to start a child file
with just |\input{|\textit{main}|}|
without loading of the package and using |\childdocof|.
If instead all processing is done
with the appropriate \textsf{childdoc} directives,
the argument of \textit{main} of |\childdocmain| can be empty.

An alternative version of the command line processing described
in \secref{sec:commandline} using the detection mechanism reads:
%
\begin{center}
|... -jobname "|\textit{target}|" "|[\textit{flags}]%
[|\def\jobname{|\textit{dest}|}|]|\input{|\textit{main}|}"|
\end{center}

%%%%%%%%%%%%%%%%%%%%%%%%%%%%%%%%%%%%%%%%%%%%%%%%%%%%%%%%%%%%%%%%%%%%%%%%%%%%%%%%
\subsection{Manual Code}
\label{sec:manual}

In case one cannot be certain whether the definitions file |childdoc.def|
is installed on the target \TeX{} distribution
and one prefers not to ship it,
it is conceivable to paste a few relevant commands into the sources.

To that end, drop all statements |\input{childdoc.def}|
and perform the replacements as outlined below.
Instead of |\childdocmain{|\textit{main}|}| add the following code
to the top of the main file:
%
\begin{center}
\begin{tabular}{l}
|\||ifdefined\childdocname\endinput\||fi\newif\ifchilddoc|\\
|\edef\childdocname{\scantokens\expandafter{\jobname\noexpand}}|\\
|\def\childdocmain{|\textit{main}|}\||ifx\childdocmain\childdocname\||else|\\
|\childdoctrue\includeonly{\childdocname}\let\jobname\childdocmain\||fi|\\
\end{tabular}
\end{center}
%
Instead of |\childdocof{|\textit{main}|}| just include the main file
at the top of each child file:
%
\begin{center}
|\input{|\textit{main}|}|
\end{center}
%
A simple redirection |\childdocforward{|\textit{dest}|}| is achieved by:
%
\begin{center}
|\def\jobname{|\textit{dest}|}\input{\jobname}|
\end{center}
%
The redirection with prefix
|\childdocforwardprefix[|\textit{prefix}|]{|\textit{dest}|}|
is accomplished by:
%
\begin{center}
\begin{tabular}{l}
|{\edef\jobname{\scantokens\expandafter{\jobname\noexpand}}|\\
|\def\redirectjob |\textit{prefix}|#1~~~{\gdef\jobname{|\textit{dest}|#1}}|\\
|\expandafter\redirectjob\jobname~~~}\input{\jobname}|
\end{tabular}
\end{center}

In an alternative approach,
child documents can be compiled by a specific command line
without additional code or specific definitions:
%
\begin{center}
|... -jobname "|\textit{target}|" "|[\textit{flags}]%
|\includeonly{|\textit{dest}|}\input{|\textit{main}|}"|
\end{center}
%

%%%%%%%%%%%%%%%%%%%%%%%%%%%%%%%%%%%%%%%%%%%%%%%%%%%%%%%%%%%%%%%%%%%%%%%%%%%%%%%%
%%%%%%%%%%%%%%%%%%%%%%%%%%%%%%%%%%%%%%%%%%%%%%%%%%%%%%%%%%%%%%%%%%%%%%%%%%%%%%%%
\section{Information}

%%%%%%%%%%%%%%%%%%%%%%%%%%%%%%%%%%%%%%%%%%%%%%%%%%%%%%%%%%%%%%%%%%%%%%%%%%%%%%%%
\subsection{Copyright}

Copyright \copyright{} 2017--2018 Niklas Beisert

This work may be distributed and/or modified under the
conditions of the \LaTeX{} Project Public License, either version 1.3
of this license or (at your option) any later version.
The latest version of this license is in
  \url{http://www.latex-project.org/lppl.txt}
and version 1.3 or later is part of all distributions of \LaTeX{}
version 2005/12/01 or later.

This work has the LPPL maintenance status `maintained'.

The Current Maintainer of this work is Niklas Beisert.

This work consists of the files |README.txt|, |childdoc.ins| and |childdoc.dtx|
as well as the derived files |childdoc.def|, |cdocsamp.tex|
with |cdocsch1.tex|, |cdocsch2.tex|, |cdocspt3.tex|, |cdocspt4.tex|,
|cdocsdrf.tex|, |cdocsfn1.tex|, |cdocsfn2.tex|
as well as |childdoc.pdf|.

%%%%%%%%%%%%%%%%%%%%%%%%%%%%%%%%%%%%%%%%%%%%%%%%%%%%%%%%%%%%%%%%%%%%%%%%%%%%%%%%
\subsection{Files and Installation}

The package consists of the files:
%
\begin{center}
\begin{tabular}{ll}
    |README.txt|   & readme file \\
    |childdoc.ins| & installation file \\
    |childdoc.dtx| & source file \\
    |childdoc.def| & definition file \\
    |cdocsamp.tex| & sample main file \\
    |cdocsch1.tex| & sample include file \\
    |cdocsch2.tex| & sample include file \\
    |cdocspt3.tex| & sample part file \\
    |cdocspt4.tex| & sample part file \\
    |cdocsdrf.tex| & sample redirection file \\
    |cdocsfn1.tex| & sample redirection file \\
    |cdocsfn2.tex| & sample redirection file \\
    |childdoc.pdf| & manual
\end{tabular}
\end{center}
%
The distribution consists of the files
|README.txt|, |childdoc.ins| and |childdoc.dtx|.
%
\begin{itemize}
\item
Run (pdf)\LaTeX{} on |childdoc.dtx|
to compile the manual |childdoc.pdf| (this file).
\item
Run \LaTeX{} on |childdoc.ins| to create the definitions file |childdoc.def|
and the sample |cdocsamp.tex| with include files
|cdocsch1.tex|, |cdocsch2.tex|, |cdocspt3.tex|, |cdocspt4.tex|,
|cdocsdrf.tex|, |cdocsfn1.tex|, |cdocsfn2.tex|.
Then copy the file |childdoc.def| to an appropriate directory of your \LaTeX{}
distribution, e.g.\ \textit{texmf-root}|/tex/latex/childdoc|.
\end{itemize}

%%%%%%%%%%%%%%%%%%%%%%%%%%%%%%%%%%%%%%%%%%%%%%%%%%%%%%%%%%%%%%%%%%%%%%%%%%%%%%%%
\subsection{Related CTAN Packages}

There are several other packages which offer a similar functionality:
%
\begin{itemize}
\item
The packages
\href{http://ctan.org/pkg/docmute}{\textsf{docmute}},
\href{http://ctan.org/pkg/includex}{\textsf{includex}} and
\href{http://ctan.org/pkg/standalone}{\textsf{standalone}}
provide commands to include only the document body of
a child file thus allowing both files to be compiled individually.
\item
The packages \href{http://ctan.org/pkg/subdocs}{\textsf{subdocs}}
and \href{http://ctan.org/pkg/subfiles}{\textsf{subfiles}}
provide structures in which the main and child documents can be
encapsulated and allowing them to be compiled individually.
The inclusion mechanism is different from the conventional |\include|.
\item
The package \href{http://ctan.org/pkg/combine}{\textsf{combine}}
is an elaborate solution to combine several documents into one.
\end{itemize}
%
See also the CTAN topic \href{http://ctan.org/topic/subdocs}{\textsf{subdocs}}
for further related packages.
The present package differs from the above solutions in that
a document structure constructed with the conventional |\include| mechanism
just needs two extra commands at the top of every file
such that all constituent files can be compiled individually.

%%%%%%%%%%%%%%%%%%%%%%%%%%%%%%%%%%%%%%%%%%%%%%%%%%%%%%%%%%%%%%%%%%%%%%%%%%%%%%%%
%\subsection{Feature Suggestions}
%
%The following is a list of features which may be useful for future
%versions of this package:
%%
%\begin{itemize}
%\item
%\ldots
%\end{itemize}

%%%%%%%%%%%%%%%%%%%%%%%%%%%%%%%%%%%%%%%%%%%%%%%%%%%%%%%%%%%%%%%%%%%%%%%%%%%%%%%%
\subsection{Revision History}

%%%%%%%%%%%%%%%%%%%%%%%%%%%%%%%%%%%%%%%%
\paragraph{v2.0:} 2018/12/30

\begin{itemize}
\item
immediate forward processing
\item
added |\childdocby| mechanism
\item
manual restructured
\end{itemize}

%%%%%%%%%%%%%%%%%%%%%%%%%%%%%%%%%%%%%%%%
\paragraph{v1.6:} 2018/01/17

\begin{itemize}
\item
application for development of include files
\item
corrections to manual
\end{itemize}

%%%%%%%%%%%%%%%%%%%%%%%%%%%%%%%%%%%%%%%%
\paragraph{v1.5:} 2017/05/21

\begin{itemize}
\item
more complete structuring introduced
\item
|\childdocof| introduced
\item
|\childdoc| renamed to |\childdocmain|
\item
|\childredirect| renamed to |\childdocforward| and |\childdocforwardprefix|
and functionality expanded
\end{itemize}

%%%%%%%%%%%%%%%%%%%%%%%%%%%%%%%%%%%%%%%%
\paragraph{v1.0:} 2017/04/27

\begin{itemize}
\item
manual and install package
\item
first version published on CTAN
\end{itemize}

%%%%%%%%%%%%%%%%%%%%%%%%%%%%%%%%%%%%%%%%
\paragraph{v0.6:} 2017/04/26

\begin{itemize}
\item
redirection mechanism added
\end{itemize}

%%%%%%%%%%%%%%%%%%%%%%%%%%%%%%%%%%%%%%%%
\paragraph{v0.5:} 2017/04/26

\begin{itemize}
\item
functionality in definition file
\end{itemize}


%%%%%%%%%%%%%%%%%%%%%%%%%%%%%%%%%%%%%%%%%%%%%%%%%%%%%%%%%%%%%%%%%%%%%%%%%%%%%%%%
%%%%%%%%%%%%%%%%%%%%%%%%%%%%%%%%%%%%%%%%%%%%%%%%%%%%%%%%%%%%%%%%%%%%%%%%%%%%%%%%
%%%%%%%%%%%%%%%%%%%%%%%%%%%%%%%%%%%%%%%%%%%%%%%%%%%%%%%%%%%%%%%%%%%%%%%%%%%%%%%%
\appendix

\settowidth\MacroIndent{\rmfamily\scriptsize 000\ }

 \DocInput{childdoc.dtx}

\end{document}
%</driver>
% \fi
%
% %%%%%%%%%%%%%%%%%%%%%%%%%%%%%%%%%%%%%%%%%%%%%%%%%%%%%%%%%%%%%%%%%%%%%%%%%%%%%%
% %%%%%%%%%%%%%%%%%%%%%%%%%%%%%%%%%%%%%%%%%%%%%%%%%%%%%%%%%%%%%%%%%%%%%%%%%%%%%%
% \section{Sample}
%\iffalse
%<*samplemain>
%\fi
%
% The following presents a sample document
% with two chapters, two parts, a title page,
% a compile flag as well as three forwarding files to set the flag.
% It consists of eight |.tex| files:
% \begin{center}
% \begin{tabular}{ll}
% |cdocsamp.tex|&main file\\
% |cdocsch1.tex|&include file for chapter 1\\
% |cdocsch2.tex|&include file for chapter 2\\
% |cdocspt3.tex|&include file for part 3\\
% |cdocspt4.tex|&include file for part 4\\
% |cdocsdrf.tex|&forwarding file for main file in draft mode\\
% |cdocsfi1.tex|&forwarding file for final version of chapter 1\\
% |cdocsfi2.tex|&forwarding file for final version of chapter 2\\
% \end{tabular}
% \end{center}
% Each of the eight files can be compiled directly by the \LaTeX{} compiler.
%
% %%%%%%%%%%%%%%%%%%%%%%%%%%%%%%%%%%%%%%
% \paragraph{Main File.}
%
% The main file is called |cdocsamp.tex|.
%
% Load the \textsf{childdoc} definitions and
% declare the filename for the main document:
%    \begin{macrocode}
\input{childdoc.def}
\childdocmain{}
%    \end{macrocode}

% Optional override for |\version| flag:
%    \begin{macrocode}
%%\ifchilddoc\else\providecommand{\version}{draft}\fi
%    \end{macrocode}

% Define the default values for the |\version| flag
% (|final| for the main file and |draft| for childs):
%    \begin{macrocode}
\ifchilddoc
\providecommand{\version}{draft}
\else
\providecommand{\version}{final}
\fi
%    \end{macrocode}

% Load the standard document class:
%    \begin{macrocode}
\documentclass[12pt]{article}
%    \end{macrocode}

% Start the document body:
%    \begin{macrocode}
\begin{document}
%    \end{macrocode}

% Declare a title page.
% Print title, part of document being processed and version flag:
%    \begin{macrocode}
\addtocounter{page}{-1}
\begin{center}
{\LARGE\bfseries{}childdoc example\par}
\vspace{1cm}
\ifchilddoc
\ifchilddocmanual part\else chapter\fi:
`\childdocname' of `\childdocjob'\par
\else
main document: `\childdocjob'\par
\fi
version: \version\par
\end{center}
\newpage
%    \end{macrocode}

% Manually include selected file,
% otherwise process as usual:
%    \begin{macrocode}
\ifchilddocmanual
\section*{part `\childdocname'}
\input{\childdocname}
\else
%    \end{macrocode}

% Include the two chapters:
%    \begin{macrocode}
\include{cdocsch1}
\include{cdocsch2}
%    \end{macrocode}

% Include the two parts unless only chapters should be displayed:
%    \begin{macrocode}
\ifchilddoc\else
\section{part three}
\input{cdocspt3}
\section{part four}
\input{cdocspt4}
\fi
%    \end{macrocode}

% Process as usual until here:
%    \begin{macrocode}
\fi
%    \end{macrocode}

% End of document body:
%    \begin{macrocode}
\end{document}
%    \end{macrocode}
%\iffalse
%</samplemain>
%\fi
%
% %%%%%%%%%%%%%%%%%%%%%%%%%%%%%%%%%%%%%%
% \paragraph{Chapter Include Files.}
%
% The include files are called |cdocsch1.tex| and |cdocsch2.tex|.
%
%\iffalse
%<*samplechap1|samplechap2>
%\fi

% Optional override for |\version| flag:
%    \begin{macrocode}
%%\providecommand{\version}{final}
%    \end{macrocode}

% Include the main document:
%    \begin{macrocode}
\input{childdoc.def}
\childdocof{cdocsamp}
%    \end{macrocode}

%\iffalse
%</samplechap1|samplechap2>
%\fi
%
%\iffalse
%<*samplechap1>
%\fi
% Some text for chapter 1:
%    \begin{macrocode}
\section{one}
some text in chapter one
%    \end{macrocode}

%\iffalse
%</samplechap1>
%\fi
% Some text for chapter 2:
%\iffalse
%<*samplechap2>
%\fi
%    \begin{macrocode}
\section{two}
more text in chapter two
%    \end{macrocode}

%\iffalse
%</samplechap2>
%\fi
%
% %%%%%%%%%%%%%%%%%%%%%%%%%%%%%%%%%%%%%%
% \paragraph{Part Include Files.}
%
% The include files are called |cdocspt3.tex| and |cdocspt4.tex|.
%
%\iffalse
%<*samplepart3|samplepart4>
%\fi

% Optional override for |\version| flag:
%    \begin{macrocode}
%%\providecommand{\version}{final}
%    \end{macrocode}

% Include the main document:
%    \begin{macrocode}
\input{childdoc.def}
\childdocby{cdocsamp}
%    \end{macrocode}

%\iffalse
%</samplepart3|samplepart4>
%\fi
%
%\iffalse
%<*samplepart3>
%\fi
% Some text for part 3:
%    \begin{macrocode}
some text in part three
%    \end{macrocode}

%\iffalse
%</samplepart3>
%\fi
% Some text for part 4:
%\iffalse
%<*samplepart4>
%\fi
%    \begin{macrocode}
more text in part four
%    \end{macrocode}

%\iffalse
%</samplepart4>
%\fi
%
% %%%%%%%%%%%%%%%%%%%%%%%%%%%%%%%%%%%%%%
% \paragraph{Forwarding for a Complete Draft.}
%
% The following forwarding file |cdocsdrf.tex|
% compiles the main document in draft mode:
%\iffalse
%<*sampledraft>
%\fi
%    \begin{macrocode}
\def\version{draft}
\input{childdoc.def}
\childdocforward{cdocsamp}
%    \end{macrocode}

%\iffalse
%</sampledraft>
%\fi
%
% %%%%%%%%%%%%%%%%%%%%%%%%%%%%%%%%%%%%%%
% \paragraph{Forwarding for Final Version of the Chapters.}
%
% The following forwarding files |cdocsfn1.tex| and |cdocsfn2.tex|
% (with identical content)
% compile the final versions of the child documents
% |cdocsch1.tex| and |cdocsch2.tex|, respectively:
%\iffalse
%<*samplefinal>
%\fi
%    \begin{macrocode}
\def\version{final}
\input{childdoc.def}
\childdocforwardprefix[cdocsamp]{cdocsfn}{cdocsch}
%    \end{macrocode}

%\iffalse
%</samplefinal>
%\fi
%
% %%%%%%%%%%%%%%%%%%%%%%%%%%%%%%%%%%%%%%
% \paragraph{Command Line Processing.}
%
% The following three command lines generate the output files
% |cdocscld|, |cdocscl1| and |cdocscl2|
% which should be identical to
% |cdocsdrf|, |cdocsch1| and |cdocsfn2|, respectively:
% \begin{center}
% \begin{tabular}{l}
% |latex -jobname cdocscld \|\\
% |  "\def\version{draft}\input{childdoc.def}\childdocforward{cdocsamp}"|\\
% |latex -jobname cdocscl1 \|\\
% |  "\input{childdoc.def}\childdocforward[cdocsamp]{cdocsch1}"|\\
% |latex -jobname cdocscl2 \|\\
% |  "\def\version{final}\input{childdoc.def}\childdocforward{cdocsch2}"|
% \end{tabular}
% \end{center}
% Note that the trailing backslash on each first line
% merely continues the input to the second line
% (for convenient cut ant paste).
% Furthermore, the command |latex| can be replaced by any
% of its alternative versions such as |pdflatex|.
%
% %%%%%%%%%%%%%%%%%%%%%%%%%%%%%%%%%%%%%%%%%%%%%%%%%%%%%%%%%%%%%%%%%%%%%%%%%%%%%%
% %%%%%%%%%%%%%%%%%%%%%%%%%%%%%%%%%%%%%%%%%%%%%%%%%%%%%%%%%%%%%%%%%%%%%%%%%%%%%%
% \section{Implementation}
%\iffalse
%<*package>
%\fi
%
% This section describes the definitions file |childdoc.def|.

% The definitions cannot be loaded using |\usepackage| or |\RequirePackage|
% which has a mechanism to prevent loading a style file more than once.
% When loading the definitions by means of |\input|
% multiple instances have to be prevented manually:
%\iffalse
%This code needs to be before the `\ProvidesFile' directive
%which is defined at the beginning of this file.
%Therefore it is also placed there and commented out here.
%</package>
%<*discard>
%\fi
%    \begin{macrocode}
\ifdefined\childdocmain\endinput\fi
%    \end{macrocode}
%\iffalse
%</discard>
%<*package>
%\fi
%
% \macro{\ifchilddoc}
% \macro{\ifchilddocmanual}
% The conditional |\ifchilddoc| tells whether a
% child (true) or main (false) document is being compiled.
% The conditional |\ifchilddocmanual| tells whether
% the |\includeonly| mechanism is used (false) or
% the selection of child files must be performed manually (true).
% The definitions initialise to false:
%    \begin{macrocode}
\newif\ifchilddoc
\newif\ifchilddocmanual
%    \end{macrocode}

% \macro{\childdocname}
% \macro{\childdocjob}
% The macro |\childdocname| stores the name of the main document
% to be compiled. The macro |\childdocjob| stores the name of
% the document on which the \LaTeX{} compiler was originally invoked.
% The content of |\jobname| cannot be compared
% to filenames specified in the source due to different catcodes.
% The following code rescans |\jobname|, stores the result
% in |\childdocname| and saves a copy in |\childdocjob|:
%    \begin{macrocode}
\edef\childdocname{\scantokens\expandafter{\jobname\noexpand}}
\let\childdocjob\childdocname
%    \end{macrocode}

% \macro{\childdocdisable}
% The macro |\childdocdisable| prevents the main file
% from being processed more than once.
% At this stage, the main document command |\childdocmain|
% is assumed to be called once again where it should do nothing.
% Any subsequent call to it should prevent
% a secondary processing of the main document
% It overwrites the forwarding commands
% |\childdocof| and |\childdocforward|
% with empty macros to prevent further inclusions of the main document:
%    \begin{macrocode}
\newcommand{\childdocdisable}
{
  \renewcommand{\childdocmain}[1]{\renewcommand{\childdocmain}[1]{\endinput}}
  \renewcommand{\childdocof}[1]{}
  \renewcommand{\childdocby}[2][]{}
  \renewcommand{\childdocforward}[2][]{}
  \renewcommand{\childdocdisable}{}
}
%    \end{macrocode}

% \macro{\childdocmain}
% The macro |\childdocmain| is to be called at the top of the main file
% with nothing or the main filename (without extension) as argument.
% First, it breaks loops.
% If the argument is not empty and does not match |\childdocname|
% (which is set by the first inclusion of |childdoc.def|),
% |\ifchilddoc| is set to true, |\includeonly| is applied to the child file
% and |\jobname| is set to the main file
% (for proper handling of |.aux| files):
%    \begin{macrocode}
\newcommand{\childdocmain}[1]
{
  \childdocdisable\childdocmain{}
  \if?#1?\else
    \begingroup
      \def\childdoctmp{#1}
      \ifx\childdoctmp\childdocname
        \def\childdoctmp{}
      \else
        \def\childdoctmp
        {
          \childdoctrue
          \includeonly{\childdocname}
          \def\childdocjob{#1}
          \def\jobname{#1}
        }
      \fi
      \expandafter
    \endgroup
    \childdoctmp
  \fi
}
%    \end{macrocode}

% \macro{\childdocof}
% The command |\childdocof| redirects
% compilation to the main file |#1|.
%    \begin{macrocode}
\newcommand{\childdocof}[1]
{
  \childdocdisable
  \childdoctrue
  \includeonly{\childdocname}
  \def\jobname{#1}
  \def\childdocjob{#1}
  \input{#1}
}
%    \end{macrocode}

% \macro{\childdocby}
% The command |\childdocby| ....
%    \begin{macrocode}
\newcommand{\childdocby}[2][]
{
  \childdocdisable
  \childdoctrue
  \childdocmanualtrue
  \if?#1?\else
    \def\jobname{#2}
  \fi
  \def\childdocjob{#2}
  \input{#2}
  \endinput
}
%    \end{macrocode}

% \macro{\childdocforward}
% The command |\childdocforward| redirects
% compilation to the main file or
% (if the optional argument is given) a child file.
% Parameters are set as if the main file
% or a child file starting with |\childdocof| was compiled.
% Then compilation is handed over to the main file:
%    \begin{macrocode}
\newcommand{\childdocforward}[2][]
{
  \begingroup
    \if?#1?
      \def\childdoctmp
      {
        \def\childdocname{#2}
        \def\childdocjob{#2}
        \def\jobname{#2}
        \input{#2}
        \endinput
      }
    \else
      \def\childdoctmp
      {
        \childdocdisable
        \def\childdocname{#2}
        \childdoctrue
        \includeonly{#2}
        \def\childdocjob{#1}
        \def\jobname{#1}
        \input{#1}
        \endinput
      }
    \fi
    \expandafter
  \endgroup
  \childdoctmp
}
%    \end{macrocode}

% \macro{\childdocforwardprefix}
% The command |\childdocforwardprefix| redirects
% compilation to the main or a child file by means of a pattern.
% The prefix |#1| in the current filename is replaced by |#2|
% and the suffix of the current filename is kept
% (it is assumed that the filename does not contain the substring `|~~~|'
% which is used as a delimiter).
% Compilation is handed over to the new file by |\childdocforward|:
%    \begin{macrocode}
\newcommand{\childdocforwardprefix}[3][]
{
  \begingroup
    \def\childdocextract #2##1~~~{\def\childdoctmp{\childdocforward[#1]{#3##1}}}
    \expandafter\childdocextract\childdocname~~~
    \expandafter
  \endgroup
  \childdoctmp
}
%    \end{macrocode}

% \macro{\childdoc}
% The deprecated macro |\childdoc| is a legacy version of |\childdocmain|:
%    \begin{macrocode}
\newcommand{\childdoc}{\childdocmain}
%    \end{macrocode}

% \macro{\childdocredirect}
% The deprecated macro |\childdocredirect| is a legacy version
% of |\childdocforward| and |\childdocforwardprefix|:
%    \begin{macrocode}
\newcommand{\childdocredirect}[2][]
{
  \begingroup
    \if?#1?
      \def\childdoctmp{\childdocforward{#2}}
    \else
      \def\childdoctmp{\childdocforwardprefix{#1}{#2}}
    \fi
    \expandafter
  \endgroup
  \childdoctmp
}
%    \end{macrocode}

%\iffalse
%</package>
%\fi
%
\endinput
|\\
|\childdocby{|\textit{main}|}|\\
\end{tabular}
\end{center}
%
The directive |\childdocby| is similar to |\childdocof|
described in \secref{sec:include},
but the subsequent selection of content must be done manually.
To that end, both |\ifchilddoc| and |\ifchilddocmanual|
will be true upon processing of a part,
and the name of the part is stored in |\childdocname|.
Note that |\jobname| will be set to the filename of the current part
so that each part receives an individual |.aux| file
that does not interfere with the |.aux| file(s) of the main document.
This behaviour can be altered by the alternative form
|\childdocby[*]{|\textit{main}|}| (with a non-empty optional argument)
which uses the |.aux| file of the main document
by setting |\jobname| to \textit{main}.

%%%%%%%%%%%%%%%%%%%%%%%%%%%%%%%%%%%%%%%%%%%%%%%%%%%%%%%%%%%%%%%%%%%%%%%%%%%%%%%%
\subsection{Driver Development}
\label{sec:driver}

The \textsf{childdoc} mechanism can also be use for the development
of definition files such as \LaTeX{} styles or classes.
This case differs from the above setup with multiple parts
included by |\include| in that no |\includeonly| should be invoked.
This can be achieved by starting the include file
(before |\ProvidesPackage|) with:
%
\begin{center}
\begin{tabular}{l}
|% \iffalse
%
% childdoc.dtx Copyright (C) 2017-2018 Niklas Beisert
%
% This work may be distributed and/or modified under the
% conditions of the LaTeX Project Public License, either version 1.3
% of this license or (at your option) any later version.
% The latest version of this license is in
%   http://www.latex-project.org/lppl.txt
% and version 1.3 or later is part of all distributions of LaTeX
% version 2005/12/01 or later.
%
% This work has the LPPL maintenance status `maintained'.
%
% The Current Maintainer of this work is Niklas Beisert.
%
% This work consists of the files childdoc.dtx and childdoc.ins
% and the derived files childdoc.def and cdocsamp.tex with
% cdocsch1.tex, cdocsch2.tex, cdocsdrf.tex, cdocsfn1.tex, cdocsfn2.tex.
%
%<package>\ifdefined\childdocmain\endinput\fi
%<package>\ProvidesFile{childdoc.def}[2018/12/30 v2.0 child document driver]
%<samplemain>\ProvidesFile{cdocsamp.tex}[2018/12/30 v2.0 sample for childdoc]
%<*driver>
%\ProvidesFile{childdoc.drv}[2018/12/30 v2.0 childdoc reference manual file]
\PassOptionsToClass{10pt,a4paper}{article}
\documentclass{ltxdoc}

\usepackage[margin=35mm]{geometry}
\usepackage{hyperref}
\usepackage{hyperxmp}
\usepackage[usenames]{color}

\hypersetup{colorlinks=true}
\hypersetup{pdfstartview=FitH}
\hypersetup{pdfpagemode=UseNone}
\hypersetup{pdfsource={}}
\hypersetup{pdflang={en-UK}}
\hypersetup{pdfcopyright={Copyright 2017-2018 Niklas Beisert.
  This work may be distributed and/or modified under the
  conditions of the LaTeX Project Public License, either version 1.3
  of this license or (at your option) any later version.}}
\hypersetup{pdflicenseurl={http://www.latex-project.org/lppl.txt}}
\hypersetup{pdfcontactaddress={ETH Zurich, ITP, HIT K,
  Wolfgang-Pauli-Strasse 27}}
\hypersetup{pdfcontactpostcode={8093}}
\hypersetup{pdfcontactcity={Zurich}}
\hypersetup{pdfcontactcountry={Switzerland}}
\hypersetup{pdfcontactemail={nbeisert@itp.phys.ethz.ch}}
\hypersetup{pdfcontacturl={http://people.phys.ethz.ch/\xmptilde nbeisert/}}

\newcommand{\secref}[1]{\hyperref[#1]{section \ref*{#1}}}

\parskip1ex
\parindent0pt
\let\olditemize\itemize
\def\itemize{\olditemize\parskip0pt}

\begin{document}

\title{The \textsf{childdoc} Package}
\hypersetup{pdftitle={The childdoc Package}}
\author{Niklas Beisert\\[2ex]
  Institut f\"ur Theoretische Physik\\
  Eidgen\"ossische Technische Hochschule Z\"urich\\
  Wolfgang-Pauli-Strasse 27, 8093 Z\"urich, Switzerland\\[1ex]
  \href{mailto:nbeisert@itp.phys.ethz.ch}
  {\texttt{nbeisert@itp.phys.ethz.ch}}}
\hypersetup{pdfauthor={Niklas Beisert}}
\hypersetup{pdfsubject={Manual for the LaTeX2e Package childdoc}}
\date{30 December 2018, \textsf{v2.0}}
\maketitle

\begin{abstract}\noindent
\textsf{childdoc} is a \LaTeXe{} package
that enables the direct compilation
of document sections included by |\include|
to individual files.
\end{abstract}

\begingroup
\parskip0ex
\tableofcontents
\endgroup

%%%%%%%%%%%%%%%%%%%%%%%%%%%%%%%%%%%%%%%%%%%%%%%%%%%%%%%%%%%%%%%%%%%%%%%%%%%%%%%%
%%%%%%%%%%%%%%%%%%%%%%%%%%%%%%%%%%%%%%%%%%%%%%%%%%%%%%%%%%%%%%%%%%%%%%%%%%%%%%%%
\section{Introduction}

\LaTeX{} provides a mechanism to structure a large document (such as a book)
into a main file and several child files (containing the chapters)
using the |\include| command.
This mechanism is beneficial for documents
which span hundreds of pages in order to
make the source file(s) more manageable.
Moreover, compilation can be restricted to
selected child files by means of the |\includeonly| command.
The latter feature can be used to reduce the compilation time while editing
(this was significantly more useful in the earlier days of \LaTeX{})
or to generate a smaller document which is easier to navigate.
Another application of |\includeonly| is to generate
documents consisting of selected parts of the complete document.

However, there are a few drawbacks of the plain |\include| mechanism:
\begin{itemize}
\item
The child files cannot be compiled on their own,
they can only be compiled via the main file.
A naive editing environment
(such as a text editor with an option
to have the current file processed by \LaTeX)
may require one to switch to the main file before compiling;
attempting to compile the child file produces errors.
\item
The main file must be modified (each time)
to adjust the |\includeonly| command
to the present needs. This easily leaves the main file in a messy state.
\item
The generated document will always carry the filename
of the main document. This is inconvenient if
several child files are to be compiled and
to be kept for distribution.
\end{itemize}

The present package provides a simple interface
to make child files individually compilable by \LaTeX{}.
Compiling a child file then has the same effect as compiling
the main file with an |\includeonly| command
to select the appropriate child.
Moreover the generated document will carry the name of the child
rather than the main file.
This resolves all three above issues.

This feature is meant to make the editing of books,
thesis documents and lecture notes somewhat more convenient.
However, the package can also be used efficiently for
composing a series of documents (such as exercise sheets)
which are typically distributed individually.
It then assists the author in generating the individual documents
(potentially in different versions)
as well as a document containing the collected series.
Another application is in developing style files
or other kinds of included material
where compilation of the style file could redirect
to a sample or test file.

%%%%%%%%%%%%%%%%%%%%%%%%%%%%%%%%%%%%%%%%%%%%%%%%%%%%%%%%%%%%%%%%%%%%%%%%%%%%%%%%
%%%%%%%%%%%%%%%%%%%%%%%%%%%%%%%%%%%%%%%%%%%%%%%%%%%%%%%%%%%%%%%%%%%%%%%%%%%%%%%%
\section{Usage}

First of all, the package \textsf{childdoc} is \emph{not} a standard
\LaTeXe{} |.sty| style file! Therefore it needs to be invoked in
a non-standard way.

%%%%%%%%%%%%%%%%%%%%%%%%%%%%%%%%%%%%%%%%%%%%%%%%%%%%%%%%%%%%%%%%%%%%%%%%%%%%%%%%
\subsection{Included Files}
\label{sec:include}

%%%%%%%%%%%%%%%%%%%%%%%%%%%%%%%%%%%%%%%%
\DescribeMacro{\childdocmain}
To use the package, add the commands
\begin{center}
\begin{tabular}{l}
|\input{childdoc.def}|\\
|\childdocmain{}|\\
\end{tabular}
\end{center}
at the very top of the main \LaTeX{} file,
in particular \emph{before} the |\documentclass| statement!
The argument of |\childdocmain| should be left empty
(but it must be present).

%%%%%%%%%%%%%%%%%%%%%%%%%%%%%%%%%%%%%%%%
\DescribeMacro{\childdocof}
Furthermore, add the commands
\begin{center}
\begin{tabular}{l}
|\input{childdoc.def}|\\
|\childdocof{|\textit{main}|}|\\
\end{tabular}
\end{center}
at the top of every child file \textit{child}
which is included by |\include{|\textit{child}|}|
from within the main file
(or at least for those files to be compiled individually).
The argument \textit{main} must be the filename of the main file.

There are a couple of
considerations in setting up the main and child documents:

%%%%%%%%%%%%%%%%%%%%%%%%%%%%%%%%%%%%%%%%
\paragraph{Restrictions.}

Please note the following restrictions:
\begin{itemize}
\item
|\childdocmain| must be called with one argument \textit{main}
to ensure compatibility with earlier version of the package.
It must either be empty (|\childdocmain{}|)
or precisely match the filename of the main file in which it is specified.
See \secref{sec:detection} for further information.
\item
The filename \textit{main} must be specified without the |.tex| extension.
\item
The filename \textit{main} is case sensitive
(even in case-insensitive file systems)
due to internal string comparison.
\item
The argument \textit{main} should be fully expanded, it cannot be a macro.
\item
Subdirectories and special characters should be avoided in filenames.
\item
The command |\childdocmain{|\textit{main}|}| must be followed by a whitespace.
It should not be followed immediately by another command
or by a comment mark `|%|'.
This is because the \TeX{} parser reads the token immediately following
the argument of |\childdocmain| and puts it
at the beginning of every child section;
however, a white\-space is ignored.
\end{itemize}

%%%%%%%%%%%%%%%%%%%%%%%%%%%%%%%%%%%%%%%%
\paragraph{Content of Main File.}

It is advisable to place all content in the child files included by |\include|.
Any output contained in the main file will appear in all child documents
unless suppressed manually;
it cannot be suppressed automatically by the |\includeonly| directive
and thus should normally be avoided.
A method to include some content in the main file
by means of conditional processing is described in \secref{sec:conditional}.

%%%%%%%%%%%%%%%%%%%%%%%%%%%%%%%%%%%%%%%%
\paragraph{Page Numbering.}

When only a part of the document is compiled,
the appropriate numbering of pages
(as well as other status parameters)
is determined from the |.aux| files.
The latter contain information from previous passes.
However this information needs to propagate through
all intermediate child documents.
Therefore the page numbering in child documents may well
be inconsistent until the complete document is compiled at least once.

A useful (if unconventional) way to always ensure a consistent
page numbering is to restart the numbering in each child document
and denote the pages by `\textit{child}|.|\textit{page}'
where \textit{child} represents the chapter/section number of the child file.
This can be achieved by the command
|\numberwithin{page}{|\textit{child}|}|
of the \textsf{amsmath} package
where \textit{child} can be |chapter| or |section|
depending on the chosen structuring.
Alternatively, one can modify the macro |\thepage| appropriately
and reset the counter |page| at the start of each child file.

%%%%%%%%%%%%%%%%%%%%%%%%%%%%%%%%%%%%%%%%%%%%%%%%%%%%%%%%%%%%%%%%%%%%%%%%%%%%%%%%
\subsection{Conditional Processing}
\label{sec:conditional}

The package provides a mechanism to compile different versions
of a document. To customise the versions further some conditional processing
can come in handy to distinguish which version is being compiled.
The package provides two macros to describe the compilation context:

%%%%%%%%%%%%%%%%%%%%%%%%%%%%%%%%%%%%%%%%
\DescribeMacro{\ifchilddoc}
The conditional |\ifchilddoc| distinguishes between the compilation of
child documents and the main document:
%
\begin{center}
|\ifchilddoc |\textit{child-code}| |[|\||else |\textit{main-code}]| \||fi|
\end{center}

%%%%%%%%%%%%%%%%%%%%%%%%%%%%%%%%%%%%%%%%
\DescribeMacro{\childdocname}
\DescribeMacro{\childdocjob}
The macro |\childdocname| contains the filename (without extension)
of the main or child file being processed.
Note that |\childdocjob| will always contain the name of the main file.

%%%%%%%%%%%%%%%%%%%%%%%%%%%%%%%%%%%%%%%%
\paragraph{Title Page.}

Conditional processing can be used to include a title or banner page
in the main document when proper precautions are taken.
Importantly, the code in the main file should ensure that the page counter
(as well as other status parameters which are stored in the |.aux| files)
takes the same value after the conditional processing.
Otherwise the page numbers may take divergent values
depending on which part is compiled.

For example, a title page could be declared by:
%
\begin{center}
\begin{tabular}{l}
|\ifchilddoc\||else|\\
|\addtocounter{page}{-1}|\\
\textit{code for title page}\\
|\newpage|\\
|\||fi|
\end{tabular}
\end{center}
%
A banner page for the child documents can be generated by:
%
\begin{center}
\begin{tabular}{l}
|\ifchilddoc|\\
|\addtocounter{page}{-1}|\\
\textit{code for banner page}\\
|\newpage|\\
|\||fi|
\end{tabular}
\end{center}
%
Here one could write a message such as:
\begin{center}
|This is the part \childdocname{} of \childdocjob{}.|
\end{center}

%%%%%%%%%%%%%%%%%%%%%%%%%%%%%%%%%%%%%%%%%%%%%%%%%%%%%%%%%%%%%%%%%%%%%%%%%%%%%%%%
\subsection{Flags}
\label{sec:flags}

The package makes it easy to generate different versions
of the main or child documents.
To this end compilation flags can be defined
and assigned different default values.
They will be particularly useful in conjunction
with the forwarding mechanism described in \secref{sec:forward}.

For example, it may be useful to have a flag |\version|
which can be set to |draft| or |final|.
The document source will contain some conditional code
depending on the value of |\version|.
Suppose further, the flag should default to |final| for the main file
and to |draft| for child files
which is a natural assignment for editing the document.
This is achieved by placing the following code
in the preamble of the main document
(below the |\childdocmain| directive):
%
\begin{center}
\begin{tabular}{l}
|\ifchilddoc|\\
|\providecommand{\version}{draft}|\\
|\||else|\\
|\providecommand{\version}{final}|\\
|\||fi|
\end{tabular}
\end{center}
%
The definition by |\providecommand| makes sure
that previous definitions are not overwritten.
Further statements |\providecommand{\version}{...}|
can thus be added before the above code to override it.

For the main file, one might add a line
(between |\childdocmain| and the above block)
%
\begin{center}
|%\ifchilddoc\||else\providecommand{\version}{draft}\||fi|
\end{center}
%
which can be uncommented to produce a draft version.
Likewise one can add a line to the very top of a child file
(above the |\childdocof{|\textit{main}|}| directive)
%
\begin{center}
|%\providecommand{\version}{final}|
\end{center}
%
which can be uncommented to produce the final version of this child document.

%%%%%%%%%%%%%%%%%%%%%%%%%%%%%%%%%%%%%%%%%%%%%%%%%%%%%%%%%%%%%%%%%%%%%%%%%%%%%%%%
\subsection{Forwarding}
\label{sec:forward}

Different versions of the main or child documents
using compilation flags as described in \secref{sec:flags}
can be (permanently) stored in different files
for convenient compilation, viewing and distribution.
To this end, the package defines a command
to pass on compilation to a different file:

%%%%%%%%%%%%%%%%%%%%%%%%%%%%%%%%%%%%%%%%
\DescribeMacro{\childdocforward}
The command |\childdocforward| redirects processing to
another source file:
%
\begin{center}
\begin{tabular}{l}
|\input{childdoc.def}|\\
|\childdocforward[|\textit{main}|]{|\textit{dest}|}|\\
\end{tabular}
\end{center}
%
The argument \textit{dest} is the destination file
(without extension).
It should be the main file or one of the child files.
Note that further \textsf{childdoc} directives
such as |\childdocof| and |\childdocforward|
in the indicated file will be processed in this form.
The optional argument \textit{main}
passes on directly to the main file \textit{main}
while pretending to compile the child \textit{dest}.
This form behaves as if \textit{dest}
issues |\childdocof{|\textit{main}|}| right away,
and no further \textsf{childdoc} directives will be processed.

%%%%%%%%%%%%%%%%%%%%%%%%%%%%%%%%%%%%%%%%
\DescribeMacro{\...prefix}
In the alternative form |\childdocforwardprefix|,
%
\begin{center}
\begin{tabular}{l}
|\input{childdoc.def}|\\
|\childdocforwardprefix[|\textit{main}|]{|\textit{prefix}|}{|\textit{dest}|}|
\end{tabular}
\end{center}
%
the destination file is determined by a pattern
depending on the current file:
To make this work, the current file must be called
`{\textit{prefix}\hspace{0.2em}\textit{suffix}}'
with \textit{prefix} matching precisely the argument.
Processing is then passed on to the file
`{\textit{dest}\hspace{0.2em}\textit{suffix}}'.
Surely, the same effect is achieved by
directly specifying the
argument `{\textit{dest}\hspace{0.2em}\textit{suffix}}'
in the first form.
However, that requires to set up a different file
for each child. With the alternative form of the command
all these files can have exactly the same content
which simplifies setting them up and maintaining them.

For example, the following file |draft.tex|
with a compilation flag |\version| as described in \secref{sec:flags}
compiles the main document as a draft:
%
\begin{center}
\begin{tabular}{l}
|\def\version{draft}|\\
|\input{childdoc.def}|\\
|\childdocforward{|\textit{main}|}|
\end{tabular}
\end{center}
%
Likewise, the following files |final|\textit{nn}|.tex|
compile the final version of the child document
|child|\textit{nn}|.tex|:
%
\begin{center}
\begin{tabular}{l}
|\def\version{final}|\\
|\input{childdoc.def}|\\
|\childdocforwardprefix{final}{child}|
\end{tabular}
\end{center}
%

Note that when several versions of a main file and/or of each child file
are to be generated, it may be convenient to set up a |Makefile| or
shell script to automatise the process.

%%%%%%%%%%%%%%%%%%%%%%%%%%%%%%%%%%%%%%%%%%%%%%%%%%%%%%%%%%%%%%%%%%%%%%%%%%%%%%%%
\subsection{Command Line Processing}
\label{sec:commandline}

The effect of redirection files can also be achieved by invoking
the \LaTeX{} compiler with a more elaborate command line.
Most conveniently this should be done as part
of a shell script or a |Makefile|.

When using \textsf{childdoc} in the main file, the following
command lines effectively perform a redirection
(note that depending on the shell being used,
backslashes may have to be doubled: `|\|' $\to$ `|\\|'):
%
\begin{center}
|... -jobname "|\textit{target}|" |\\|"|[\textit{flags}]%
|\input{childdoc.def}\childdocforward[|\textit{main}|]{|\textit{dest}|}"|
\end{center}
%
Here \textit{target} is the name of the output file,
\textit{main} is the name of the main file
and \textit{dest} is the name of the main or child file to be processed
(all filenames without extensions).
The optional argument \textit{main} can be omitted
if \textit{main} matches \textit{dest}.
Optionally, compilation \textit{flags} can be defined via |\def| commands.
This command line makes the \TeX{} engine believe
it is compiling the file \textit{target}
whose content is specified as the latter parameter.
The provided code then forwards the processing to
\textit{main} or \textit{dest} as described in \secref{sec:forward}.

%%%%%%%%%%%%%%%%%%%%%%%%%%%%%%%%%%%%%%%%%%%%%%%%%%%%%%%%%%%%%%%%%%%%%%%%%%%%%%%%
\subsection{Include by Input}
\label{sec:input}

Including child documents by |\include| has some restrictions by design.
Most notably, the content of a child document always occupies
its own set of pages; pages cannot be shared between child documents.
Usually, this behaviour makes perfect sense
because each child document contain an essential part of the document.
However, in some situations it may be desirable to compose
a document from a collection of parts
without having mandatory page breaks between then.
For this case, the package
provides a mechanism to include parts
by |\input| which can also be processed individually.
However, by construction this mechanism
requires manual handling of the content to be output.

%%%%%%%%%%%%%%%%%%%%%%%%%%%%%%%%%%%%%%%%
\DescribeMacro{\ifchilddocmanual}
The main file should be prepared as usual, see \secref{sec:include}.
However, the document body must make a distinction
between processing of an individual part and of the main document, e.g.:
%
\begin{center}
\begin{tabular}{l}
|\ifchilddocmanual|\\
|\input{\childdocname}|\\
|\||else|\\
\textit{document body with }|\input{|\textit{part}|}|\\
|\||fi|
\end{tabular}
\end{center}
%
The conditional |\ifchilddocmanual| is true whenever
a part to be included by |\input| is being compiled,
and the name of the part is stored in |\childdocname|.

%%%%%%%%%%%%%%%%%%%%%%%%%%%%%%%%%%%%%%%%
\DescribeMacro{\childdocby}
Each part to be included by |\input| should start with:
%
\begin{center}
\begin{tabular}{l}
|\input{childdoc.def}|\\
|\childdocby{|\textit{main}|}|\\
\end{tabular}
\end{center}
%
The directive |\childdocby| is similar to |\childdocof|
described in \secref{sec:include},
but the subsequent selection of content must be done manually.
To that end, both |\ifchilddoc| and |\ifchilddocmanual|
will be true upon processing of a part,
and the name of the part is stored in |\childdocname|.
Note that |\jobname| will be set to the filename of the current part
so that each part receives an individual |.aux| file
that does not interfere with the |.aux| file(s) of the main document.
This behaviour can be altered by the alternative form
|\childdocby[*]{|\textit{main}|}| (with a non-empty optional argument)
which uses the |.aux| file of the main document
by setting |\jobname| to \textit{main}.

%%%%%%%%%%%%%%%%%%%%%%%%%%%%%%%%%%%%%%%%%%%%%%%%%%%%%%%%%%%%%%%%%%%%%%%%%%%%%%%%
\subsection{Driver Development}
\label{sec:driver}

The \textsf{childdoc} mechanism can also be use for the development
of definition files such as \LaTeX{} styles or classes.
This case differs from the above setup with multiple parts
included by |\include| in that no |\includeonly| should be invoked.
This can be achieved by starting the include file
(before |\ProvidesPackage|) with:
%
\begin{center}
\begin{tabular}{l}
|\input{childdoc.def}|\\
|\childdocforward{|\textit{main}|}|\\
\end{tabular}
\end{center}
%
or alternatively with:
%
\begin{center}
\begin{tabular}{l}
|\input{childdoc.def}|\\
|\childdocby{|\textit{main}|}|\\
\end{tabular}
\end{center}
%
Both forms have slightly different effects as described above.
The main file is prepared as usual, see \secref{sec:include}.

%%%%%%%%%%%%%%%%%%%%%%%%%%%%%%%%%%%%%%%%%%%%%%%%%%%%%%%%%%%%%%%%%%%%%%%%%%%%%%%%
\subsection{Legacy Detection}
\label{sec:detection}

The directive |\childdocmain| in the main file can detect
whether the complete document or merely a child is to be compiled
even without using the directive |\childdocof|.
This method is deprecated because it is less robust
and there is no compelling reason to use it;
it is merely provided for backward compatibility
and it may be removed in future versions.

If the detection mechanism is to be used,
it is mandatory to correctly specify
the filename of the main file as the argument of |\childdocmain|:
%
\begin{center}
\begin{tabular}{l}
|\input{childdoc.def}|\\
|\childdocmain{|\textit{main}|}|\\
\end{tabular}
\end{center}
%
If |\jobname| does not match the argument \textit{main} of |\childdocmain|,
it is assumed that |\jobname| points to the child file to be compiled.
When using |\childdocmain| with the main file specified as argument,
it suffices to start a child file
with just |\input{|\textit{main}|}|
without loading of the package and using |\childdocof|.
If instead all processing is done
with the appropriate \textsf{childdoc} directives,
the argument of \textit{main} of |\childdocmain| can be empty.

An alternative version of the command line processing described
in \secref{sec:commandline} using the detection mechanism reads:
%
\begin{center}
|... -jobname "|\textit{target}|" "|[\textit{flags}]%
[|\def\jobname{|\textit{dest}|}|]|\input{|\textit{main}|}"|
\end{center}

%%%%%%%%%%%%%%%%%%%%%%%%%%%%%%%%%%%%%%%%%%%%%%%%%%%%%%%%%%%%%%%%%%%%%%%%%%%%%%%%
\subsection{Manual Code}
\label{sec:manual}

In case one cannot be certain whether the definitions file |childdoc.def|
is installed on the target \TeX{} distribution
and one prefers not to ship it,
it is conceivable to paste a few relevant commands into the sources.

To that end, drop all statements |\input{childdoc.def}|
and perform the replacements as outlined below.
Instead of |\childdocmain{|\textit{main}|}| add the following code
to the top of the main file:
%
\begin{center}
\begin{tabular}{l}
|\||ifdefined\childdocname\endinput\||fi\newif\ifchilddoc|\\
|\edef\childdocname{\scantokens\expandafter{\jobname\noexpand}}|\\
|\def\childdocmain{|\textit{main}|}\||ifx\childdocmain\childdocname\||else|\\
|\childdoctrue\includeonly{\childdocname}\let\jobname\childdocmain\||fi|\\
\end{tabular}
\end{center}
%
Instead of |\childdocof{|\textit{main}|}| just include the main file
at the top of each child file:
%
\begin{center}
|\input{|\textit{main}|}|
\end{center}
%
A simple redirection |\childdocforward{|\textit{dest}|}| is achieved by:
%
\begin{center}
|\def\jobname{|\textit{dest}|}\input{\jobname}|
\end{center}
%
The redirection with prefix
|\childdocforwardprefix[|\textit{prefix}|]{|\textit{dest}|}|
is accomplished by:
%
\begin{center}
\begin{tabular}{l}
|{\edef\jobname{\scantokens\expandafter{\jobname\noexpand}}|\\
|\def\redirectjob |\textit{prefix}|#1~~~{\gdef\jobname{|\textit{dest}|#1}}|\\
|\expandafter\redirectjob\jobname~~~}\input{\jobname}|
\end{tabular}
\end{center}

In an alternative approach,
child documents can be compiled by a specific command line
without additional code or specific definitions:
%
\begin{center}
|... -jobname "|\textit{target}|" "|[\textit{flags}]%
|\includeonly{|\textit{dest}|}\input{|\textit{main}|}"|
\end{center}
%

%%%%%%%%%%%%%%%%%%%%%%%%%%%%%%%%%%%%%%%%%%%%%%%%%%%%%%%%%%%%%%%%%%%%%%%%%%%%%%%%
%%%%%%%%%%%%%%%%%%%%%%%%%%%%%%%%%%%%%%%%%%%%%%%%%%%%%%%%%%%%%%%%%%%%%%%%%%%%%%%%
\section{Information}

%%%%%%%%%%%%%%%%%%%%%%%%%%%%%%%%%%%%%%%%%%%%%%%%%%%%%%%%%%%%%%%%%%%%%%%%%%%%%%%%
\subsection{Copyright}

Copyright \copyright{} 2017--2018 Niklas Beisert

This work may be distributed and/or modified under the
conditions of the \LaTeX{} Project Public License, either version 1.3
of this license or (at your option) any later version.
The latest version of this license is in
  \url{http://www.latex-project.org/lppl.txt}
and version 1.3 or later is part of all distributions of \LaTeX{}
version 2005/12/01 or later.

This work has the LPPL maintenance status `maintained'.

The Current Maintainer of this work is Niklas Beisert.

This work consists of the files |README.txt|, |childdoc.ins| and |childdoc.dtx|
as well as the derived files |childdoc.def|, |cdocsamp.tex|
with |cdocsch1.tex|, |cdocsch2.tex|, |cdocspt3.tex|, |cdocspt4.tex|,
|cdocsdrf.tex|, |cdocsfn1.tex|, |cdocsfn2.tex|
as well as |childdoc.pdf|.

%%%%%%%%%%%%%%%%%%%%%%%%%%%%%%%%%%%%%%%%%%%%%%%%%%%%%%%%%%%%%%%%%%%%%%%%%%%%%%%%
\subsection{Files and Installation}

The package consists of the files:
%
\begin{center}
\begin{tabular}{ll}
    |README.txt|   & readme file \\
    |childdoc.ins| & installation file \\
    |childdoc.dtx| & source file \\
    |childdoc.def| & definition file \\
    |cdocsamp.tex| & sample main file \\
    |cdocsch1.tex| & sample include file \\
    |cdocsch2.tex| & sample include file \\
    |cdocspt3.tex| & sample part file \\
    |cdocspt4.tex| & sample part file \\
    |cdocsdrf.tex| & sample redirection file \\
    |cdocsfn1.tex| & sample redirection file \\
    |cdocsfn2.tex| & sample redirection file \\
    |childdoc.pdf| & manual
\end{tabular}
\end{center}
%
The distribution consists of the files
|README.txt|, |childdoc.ins| and |childdoc.dtx|.
%
\begin{itemize}
\item
Run (pdf)\LaTeX{} on |childdoc.dtx|
to compile the manual |childdoc.pdf| (this file).
\item
Run \LaTeX{} on |childdoc.ins| to create the definitions file |childdoc.def|
and the sample |cdocsamp.tex| with include files
|cdocsch1.tex|, |cdocsch2.tex|, |cdocspt3.tex|, |cdocspt4.tex|,
|cdocsdrf.tex|, |cdocsfn1.tex|, |cdocsfn2.tex|.
Then copy the file |childdoc.def| to an appropriate directory of your \LaTeX{}
distribution, e.g.\ \textit{texmf-root}|/tex/latex/childdoc|.
\end{itemize}

%%%%%%%%%%%%%%%%%%%%%%%%%%%%%%%%%%%%%%%%%%%%%%%%%%%%%%%%%%%%%%%%%%%%%%%%%%%%%%%%
\subsection{Related CTAN Packages}

There are several other packages which offer a similar functionality:
%
\begin{itemize}
\item
The packages
\href{http://ctan.org/pkg/docmute}{\textsf{docmute}},
\href{http://ctan.org/pkg/includex}{\textsf{includex}} and
\href{http://ctan.org/pkg/standalone}{\textsf{standalone}}
provide commands to include only the document body of
a child file thus allowing both files to be compiled individually.
\item
The packages \href{http://ctan.org/pkg/subdocs}{\textsf{subdocs}}
and \href{http://ctan.org/pkg/subfiles}{\textsf{subfiles}}
provide structures in which the main and child documents can be
encapsulated and allowing them to be compiled individually.
The inclusion mechanism is different from the conventional |\include|.
\item
The package \href{http://ctan.org/pkg/combine}{\textsf{combine}}
is an elaborate solution to combine several documents into one.
\end{itemize}
%
See also the CTAN topic \href{http://ctan.org/topic/subdocs}{\textsf{subdocs}}
for further related packages.
The present package differs from the above solutions in that
a document structure constructed with the conventional |\include| mechanism
just needs two extra commands at the top of every file
such that all constituent files can be compiled individually.

%%%%%%%%%%%%%%%%%%%%%%%%%%%%%%%%%%%%%%%%%%%%%%%%%%%%%%%%%%%%%%%%%%%%%%%%%%%%%%%%
%\subsection{Feature Suggestions}
%
%The following is a list of features which may be useful for future
%versions of this package:
%%
%\begin{itemize}
%\item
%\ldots
%\end{itemize}

%%%%%%%%%%%%%%%%%%%%%%%%%%%%%%%%%%%%%%%%%%%%%%%%%%%%%%%%%%%%%%%%%%%%%%%%%%%%%%%%
\subsection{Revision History}

%%%%%%%%%%%%%%%%%%%%%%%%%%%%%%%%%%%%%%%%
\paragraph{v2.0:} 2018/12/30

\begin{itemize}
\item
immediate forward processing
\item
added |\childdocby| mechanism
\item
manual restructured
\end{itemize}

%%%%%%%%%%%%%%%%%%%%%%%%%%%%%%%%%%%%%%%%
\paragraph{v1.6:} 2018/01/17

\begin{itemize}
\item
application for development of include files
\item
corrections to manual
\end{itemize}

%%%%%%%%%%%%%%%%%%%%%%%%%%%%%%%%%%%%%%%%
\paragraph{v1.5:} 2017/05/21

\begin{itemize}
\item
more complete structuring introduced
\item
|\childdocof| introduced
\item
|\childdoc| renamed to |\childdocmain|
\item
|\childredirect| renamed to |\childdocforward| and |\childdocforwardprefix|
and functionality expanded
\end{itemize}

%%%%%%%%%%%%%%%%%%%%%%%%%%%%%%%%%%%%%%%%
\paragraph{v1.0:} 2017/04/27

\begin{itemize}
\item
manual and install package
\item
first version published on CTAN
\end{itemize}

%%%%%%%%%%%%%%%%%%%%%%%%%%%%%%%%%%%%%%%%
\paragraph{v0.6:} 2017/04/26

\begin{itemize}
\item
redirection mechanism added
\end{itemize}

%%%%%%%%%%%%%%%%%%%%%%%%%%%%%%%%%%%%%%%%
\paragraph{v0.5:} 2017/04/26

\begin{itemize}
\item
functionality in definition file
\end{itemize}


%%%%%%%%%%%%%%%%%%%%%%%%%%%%%%%%%%%%%%%%%%%%%%%%%%%%%%%%%%%%%%%%%%%%%%%%%%%%%%%%
%%%%%%%%%%%%%%%%%%%%%%%%%%%%%%%%%%%%%%%%%%%%%%%%%%%%%%%%%%%%%%%%%%%%%%%%%%%%%%%%
%%%%%%%%%%%%%%%%%%%%%%%%%%%%%%%%%%%%%%%%%%%%%%%%%%%%%%%%%%%%%%%%%%%%%%%%%%%%%%%%
\appendix

\settowidth\MacroIndent{\rmfamily\scriptsize 000\ }

 \DocInput{childdoc.dtx}

\end{document}
%</driver>
% \fi
%
% %%%%%%%%%%%%%%%%%%%%%%%%%%%%%%%%%%%%%%%%%%%%%%%%%%%%%%%%%%%%%%%%%%%%%%%%%%%%%%
% %%%%%%%%%%%%%%%%%%%%%%%%%%%%%%%%%%%%%%%%%%%%%%%%%%%%%%%%%%%%%%%%%%%%%%%%%%%%%%
% \section{Sample}
%\iffalse
%<*samplemain>
%\fi
%
% The following presents a sample document
% with two chapters, two parts, a title page,
% a compile flag as well as three forwarding files to set the flag.
% It consists of eight |.tex| files:
% \begin{center}
% \begin{tabular}{ll}
% |cdocsamp.tex|&main file\\
% |cdocsch1.tex|&include file for chapter 1\\
% |cdocsch2.tex|&include file for chapter 2\\
% |cdocspt3.tex|&include file for part 3\\
% |cdocspt4.tex|&include file for part 4\\
% |cdocsdrf.tex|&forwarding file for main file in draft mode\\
% |cdocsfi1.tex|&forwarding file for final version of chapter 1\\
% |cdocsfi2.tex|&forwarding file for final version of chapter 2\\
% \end{tabular}
% \end{center}
% Each of the eight files can be compiled directly by the \LaTeX{} compiler.
%
% %%%%%%%%%%%%%%%%%%%%%%%%%%%%%%%%%%%%%%
% \paragraph{Main File.}
%
% The main file is called |cdocsamp.tex|.
%
% Load the \textsf{childdoc} definitions and
% declare the filename for the main document:
%    \begin{macrocode}
\input{childdoc.def}
\childdocmain{}
%    \end{macrocode}

% Optional override for |\version| flag:
%    \begin{macrocode}
%%\ifchilddoc\else\providecommand{\version}{draft}\fi
%    \end{macrocode}

% Define the default values for the |\version| flag
% (|final| for the main file and |draft| for childs):
%    \begin{macrocode}
\ifchilddoc
\providecommand{\version}{draft}
\else
\providecommand{\version}{final}
\fi
%    \end{macrocode}

% Load the standard document class:
%    \begin{macrocode}
\documentclass[12pt]{article}
%    \end{macrocode}

% Start the document body:
%    \begin{macrocode}
\begin{document}
%    \end{macrocode}

% Declare a title page.
% Print title, part of document being processed and version flag:
%    \begin{macrocode}
\addtocounter{page}{-1}
\begin{center}
{\LARGE\bfseries{}childdoc example\par}
\vspace{1cm}
\ifchilddoc
\ifchilddocmanual part\else chapter\fi:
`\childdocname' of `\childdocjob'\par
\else
main document: `\childdocjob'\par
\fi
version: \version\par
\end{center}
\newpage
%    \end{macrocode}

% Manually include selected file,
% otherwise process as usual:
%    \begin{macrocode}
\ifchilddocmanual
\section*{part `\childdocname'}
\input{\childdocname}
\else
%    \end{macrocode}

% Include the two chapters:
%    \begin{macrocode}
\include{cdocsch1}
\include{cdocsch2}
%    \end{macrocode}

% Include the two parts unless only chapters should be displayed:
%    \begin{macrocode}
\ifchilddoc\else
\section{part three}
\input{cdocspt3}
\section{part four}
\input{cdocspt4}
\fi
%    \end{macrocode}

% Process as usual until here:
%    \begin{macrocode}
\fi
%    \end{macrocode}

% End of document body:
%    \begin{macrocode}
\end{document}
%    \end{macrocode}
%\iffalse
%</samplemain>
%\fi
%
% %%%%%%%%%%%%%%%%%%%%%%%%%%%%%%%%%%%%%%
% \paragraph{Chapter Include Files.}
%
% The include files are called |cdocsch1.tex| and |cdocsch2.tex|.
%
%\iffalse
%<*samplechap1|samplechap2>
%\fi

% Optional override for |\version| flag:
%    \begin{macrocode}
%%\providecommand{\version}{final}
%    \end{macrocode}

% Include the main document:
%    \begin{macrocode}
\input{childdoc.def}
\childdocof{cdocsamp}
%    \end{macrocode}

%\iffalse
%</samplechap1|samplechap2>
%\fi
%
%\iffalse
%<*samplechap1>
%\fi
% Some text for chapter 1:
%    \begin{macrocode}
\section{one}
some text in chapter one
%    \end{macrocode}

%\iffalse
%</samplechap1>
%\fi
% Some text for chapter 2:
%\iffalse
%<*samplechap2>
%\fi
%    \begin{macrocode}
\section{two}
more text in chapter two
%    \end{macrocode}

%\iffalse
%</samplechap2>
%\fi
%
% %%%%%%%%%%%%%%%%%%%%%%%%%%%%%%%%%%%%%%
% \paragraph{Part Include Files.}
%
% The include files are called |cdocspt3.tex| and |cdocspt4.tex|.
%
%\iffalse
%<*samplepart3|samplepart4>
%\fi

% Optional override for |\version| flag:
%    \begin{macrocode}
%%\providecommand{\version}{final}
%    \end{macrocode}

% Include the main document:
%    \begin{macrocode}
\input{childdoc.def}
\childdocby{cdocsamp}
%    \end{macrocode}

%\iffalse
%</samplepart3|samplepart4>
%\fi
%
%\iffalse
%<*samplepart3>
%\fi
% Some text for part 3:
%    \begin{macrocode}
some text in part three
%    \end{macrocode}

%\iffalse
%</samplepart3>
%\fi
% Some text for part 4:
%\iffalse
%<*samplepart4>
%\fi
%    \begin{macrocode}
more text in part four
%    \end{macrocode}

%\iffalse
%</samplepart4>
%\fi
%
% %%%%%%%%%%%%%%%%%%%%%%%%%%%%%%%%%%%%%%
% \paragraph{Forwarding for a Complete Draft.}
%
% The following forwarding file |cdocsdrf.tex|
% compiles the main document in draft mode:
%\iffalse
%<*sampledraft>
%\fi
%    \begin{macrocode}
\def\version{draft}
\input{childdoc.def}
\childdocforward{cdocsamp}
%    \end{macrocode}

%\iffalse
%</sampledraft>
%\fi
%
% %%%%%%%%%%%%%%%%%%%%%%%%%%%%%%%%%%%%%%
% \paragraph{Forwarding for Final Version of the Chapters.}
%
% The following forwarding files |cdocsfn1.tex| and |cdocsfn2.tex|
% (with identical content)
% compile the final versions of the child documents
% |cdocsch1.tex| and |cdocsch2.tex|, respectively:
%\iffalse
%<*samplefinal>
%\fi
%    \begin{macrocode}
\def\version{final}
\input{childdoc.def}
\childdocforwardprefix[cdocsamp]{cdocsfn}{cdocsch}
%    \end{macrocode}

%\iffalse
%</samplefinal>
%\fi
%
% %%%%%%%%%%%%%%%%%%%%%%%%%%%%%%%%%%%%%%
% \paragraph{Command Line Processing.}
%
% The following three command lines generate the output files
% |cdocscld|, |cdocscl1| and |cdocscl2|
% which should be identical to
% |cdocsdrf|, |cdocsch1| and |cdocsfn2|, respectively:
% \begin{center}
% \begin{tabular}{l}
% |latex -jobname cdocscld \|\\
% |  "\def\version{draft}\input{childdoc.def}\childdocforward{cdocsamp}"|\\
% |latex -jobname cdocscl1 \|\\
% |  "\input{childdoc.def}\childdocforward[cdocsamp]{cdocsch1}"|\\
% |latex -jobname cdocscl2 \|\\
% |  "\def\version{final}\input{childdoc.def}\childdocforward{cdocsch2}"|
% \end{tabular}
% \end{center}
% Note that the trailing backslash on each first line
% merely continues the input to the second line
% (for convenient cut ant paste).
% Furthermore, the command |latex| can be replaced by any
% of its alternative versions such as |pdflatex|.
%
% %%%%%%%%%%%%%%%%%%%%%%%%%%%%%%%%%%%%%%%%%%%%%%%%%%%%%%%%%%%%%%%%%%%%%%%%%%%%%%
% %%%%%%%%%%%%%%%%%%%%%%%%%%%%%%%%%%%%%%%%%%%%%%%%%%%%%%%%%%%%%%%%%%%%%%%%%%%%%%
% \section{Implementation}
%\iffalse
%<*package>
%\fi
%
% This section describes the definitions file |childdoc.def|.

% The definitions cannot be loaded using |\usepackage| or |\RequirePackage|
% which has a mechanism to prevent loading a style file more than once.
% When loading the definitions by means of |\input|
% multiple instances have to be prevented manually:
%\iffalse
%This code needs to be before the `\ProvidesFile' directive
%which is defined at the beginning of this file.
%Therefore it is also placed there and commented out here.
%</package>
%<*discard>
%\fi
%    \begin{macrocode}
\ifdefined\childdocmain\endinput\fi
%    \end{macrocode}
%\iffalse
%</discard>
%<*package>
%\fi
%
% \macro{\ifchilddoc}
% \macro{\ifchilddocmanual}
% The conditional |\ifchilddoc| tells whether a
% child (true) or main (false) document is being compiled.
% The conditional |\ifchilddocmanual| tells whether
% the |\includeonly| mechanism is used (false) or
% the selection of child files must be performed manually (true).
% The definitions initialise to false:
%    \begin{macrocode}
\newif\ifchilddoc
\newif\ifchilddocmanual
%    \end{macrocode}

% \macro{\childdocname}
% \macro{\childdocjob}
% The macro |\childdocname| stores the name of the main document
% to be compiled. The macro |\childdocjob| stores the name of
% the document on which the \LaTeX{} compiler was originally invoked.
% The content of |\jobname| cannot be compared
% to filenames specified in the source due to different catcodes.
% The following code rescans |\jobname|, stores the result
% in |\childdocname| and saves a copy in |\childdocjob|:
%    \begin{macrocode}
\edef\childdocname{\scantokens\expandafter{\jobname\noexpand}}
\let\childdocjob\childdocname
%    \end{macrocode}

% \macro{\childdocdisable}
% The macro |\childdocdisable| prevents the main file
% from being processed more than once.
% At this stage, the main document command |\childdocmain|
% is assumed to be called once again where it should do nothing.
% Any subsequent call to it should prevent
% a secondary processing of the main document
% It overwrites the forwarding commands
% |\childdocof| and |\childdocforward|
% with empty macros to prevent further inclusions of the main document:
%    \begin{macrocode}
\newcommand{\childdocdisable}
{
  \renewcommand{\childdocmain}[1]{\renewcommand{\childdocmain}[1]{\endinput}}
  \renewcommand{\childdocof}[1]{}
  \renewcommand{\childdocby}[2][]{}
  \renewcommand{\childdocforward}[2][]{}
  \renewcommand{\childdocdisable}{}
}
%    \end{macrocode}

% \macro{\childdocmain}
% The macro |\childdocmain| is to be called at the top of the main file
% with nothing or the main filename (without extension) as argument.
% First, it breaks loops.
% If the argument is not empty and does not match |\childdocname|
% (which is set by the first inclusion of |childdoc.def|),
% |\ifchilddoc| is set to true, |\includeonly| is applied to the child file
% and |\jobname| is set to the main file
% (for proper handling of |.aux| files):
%    \begin{macrocode}
\newcommand{\childdocmain}[1]
{
  \childdocdisable\childdocmain{}
  \if?#1?\else
    \begingroup
      \def\childdoctmp{#1}
      \ifx\childdoctmp\childdocname
        \def\childdoctmp{}
      \else
        \def\childdoctmp
        {
          \childdoctrue
          \includeonly{\childdocname}
          \def\childdocjob{#1}
          \def\jobname{#1}
        }
      \fi
      \expandafter
    \endgroup
    \childdoctmp
  \fi
}
%    \end{macrocode}

% \macro{\childdocof}
% The command |\childdocof| redirects
% compilation to the main file |#1|.
%    \begin{macrocode}
\newcommand{\childdocof}[1]
{
  \childdocdisable
  \childdoctrue
  \includeonly{\childdocname}
  \def\jobname{#1}
  \def\childdocjob{#1}
  \input{#1}
}
%    \end{macrocode}

% \macro{\childdocby}
% The command |\childdocby| ....
%    \begin{macrocode}
\newcommand{\childdocby}[2][]
{
  \childdocdisable
  \childdoctrue
  \childdocmanualtrue
  \if?#1?\else
    \def\jobname{#2}
  \fi
  \def\childdocjob{#2}
  \input{#2}
  \endinput
}
%    \end{macrocode}

% \macro{\childdocforward}
% The command |\childdocforward| redirects
% compilation to the main file or
% (if the optional argument is given) a child file.
% Parameters are set as if the main file
% or a child file starting with |\childdocof| was compiled.
% Then compilation is handed over to the main file:
%    \begin{macrocode}
\newcommand{\childdocforward}[2][]
{
  \begingroup
    \if?#1?
      \def\childdoctmp
      {
        \def\childdocname{#2}
        \def\childdocjob{#2}
        \def\jobname{#2}
        \input{#2}
        \endinput
      }
    \else
      \def\childdoctmp
      {
        \childdocdisable
        \def\childdocname{#2}
        \childdoctrue
        \includeonly{#2}
        \def\childdocjob{#1}
        \def\jobname{#1}
        \input{#1}
        \endinput
      }
    \fi
    \expandafter
  \endgroup
  \childdoctmp
}
%    \end{macrocode}

% \macro{\childdocforwardprefix}
% The command |\childdocforwardprefix| redirects
% compilation to the main or a child file by means of a pattern.
% The prefix |#1| in the current filename is replaced by |#2|
% and the suffix of the current filename is kept
% (it is assumed that the filename does not contain the substring `|~~~|'
% which is used as a delimiter).
% Compilation is handed over to the new file by |\childdocforward|:
%    \begin{macrocode}
\newcommand{\childdocforwardprefix}[3][]
{
  \begingroup
    \def\childdocextract #2##1~~~{\def\childdoctmp{\childdocforward[#1]{#3##1}}}
    \expandafter\childdocextract\childdocname~~~
    \expandafter
  \endgroup
  \childdoctmp
}
%    \end{macrocode}

% \macro{\childdoc}
% The deprecated macro |\childdoc| is a legacy version of |\childdocmain|:
%    \begin{macrocode}
\newcommand{\childdoc}{\childdocmain}
%    \end{macrocode}

% \macro{\childdocredirect}
% The deprecated macro |\childdocredirect| is a legacy version
% of |\childdocforward| and |\childdocforwardprefix|:
%    \begin{macrocode}
\newcommand{\childdocredirect}[2][]
{
  \begingroup
    \if?#1?
      \def\childdoctmp{\childdocforward{#2}}
    \else
      \def\childdoctmp{\childdocforwardprefix{#1}{#2}}
    \fi
    \expandafter
  \endgroup
  \childdoctmp
}
%    \end{macrocode}

%\iffalse
%</package>
%\fi
%
\endinput
|\\
|\childdocforward{|\textit{main}|}|\\
\end{tabular}
\end{center}
%
or alternatively with:
%
\begin{center}
\begin{tabular}{l}
|% \iffalse
%
% childdoc.dtx Copyright (C) 2017-2018 Niklas Beisert
%
% This work may be distributed and/or modified under the
% conditions of the LaTeX Project Public License, either version 1.3
% of this license or (at your option) any later version.
% The latest version of this license is in
%   http://www.latex-project.org/lppl.txt
% and version 1.3 or later is part of all distributions of LaTeX
% version 2005/12/01 or later.
%
% This work has the LPPL maintenance status `maintained'.
%
% The Current Maintainer of this work is Niklas Beisert.
%
% This work consists of the files childdoc.dtx and childdoc.ins
% and the derived files childdoc.def and cdocsamp.tex with
% cdocsch1.tex, cdocsch2.tex, cdocsdrf.tex, cdocsfn1.tex, cdocsfn2.tex.
%
%<package>\ifdefined\childdocmain\endinput\fi
%<package>\ProvidesFile{childdoc.def}[2018/12/30 v2.0 child document driver]
%<samplemain>\ProvidesFile{cdocsamp.tex}[2018/12/30 v2.0 sample for childdoc]
%<*driver>
%\ProvidesFile{childdoc.drv}[2018/12/30 v2.0 childdoc reference manual file]
\PassOptionsToClass{10pt,a4paper}{article}
\documentclass{ltxdoc}

\usepackage[margin=35mm]{geometry}
\usepackage{hyperref}
\usepackage{hyperxmp}
\usepackage[usenames]{color}

\hypersetup{colorlinks=true}
\hypersetup{pdfstartview=FitH}
\hypersetup{pdfpagemode=UseNone}
\hypersetup{pdfsource={}}
\hypersetup{pdflang={en-UK}}
\hypersetup{pdfcopyright={Copyright 2017-2018 Niklas Beisert.
  This work may be distributed and/or modified under the
  conditions of the LaTeX Project Public License, either version 1.3
  of this license or (at your option) any later version.}}
\hypersetup{pdflicenseurl={http://www.latex-project.org/lppl.txt}}
\hypersetup{pdfcontactaddress={ETH Zurich, ITP, HIT K,
  Wolfgang-Pauli-Strasse 27}}
\hypersetup{pdfcontactpostcode={8093}}
\hypersetup{pdfcontactcity={Zurich}}
\hypersetup{pdfcontactcountry={Switzerland}}
\hypersetup{pdfcontactemail={nbeisert@itp.phys.ethz.ch}}
\hypersetup{pdfcontacturl={http://people.phys.ethz.ch/\xmptilde nbeisert/}}

\newcommand{\secref}[1]{\hyperref[#1]{section \ref*{#1}}}

\parskip1ex
\parindent0pt
\let\olditemize\itemize
\def\itemize{\olditemize\parskip0pt}

\begin{document}

\title{The \textsf{childdoc} Package}
\hypersetup{pdftitle={The childdoc Package}}
\author{Niklas Beisert\\[2ex]
  Institut f\"ur Theoretische Physik\\
  Eidgen\"ossische Technische Hochschule Z\"urich\\
  Wolfgang-Pauli-Strasse 27, 8093 Z\"urich, Switzerland\\[1ex]
  \href{mailto:nbeisert@itp.phys.ethz.ch}
  {\texttt{nbeisert@itp.phys.ethz.ch}}}
\hypersetup{pdfauthor={Niklas Beisert}}
\hypersetup{pdfsubject={Manual for the LaTeX2e Package childdoc}}
\date{30 December 2018, \textsf{v2.0}}
\maketitle

\begin{abstract}\noindent
\textsf{childdoc} is a \LaTeXe{} package
that enables the direct compilation
of document sections included by |\include|
to individual files.
\end{abstract}

\begingroup
\parskip0ex
\tableofcontents
\endgroup

%%%%%%%%%%%%%%%%%%%%%%%%%%%%%%%%%%%%%%%%%%%%%%%%%%%%%%%%%%%%%%%%%%%%%%%%%%%%%%%%
%%%%%%%%%%%%%%%%%%%%%%%%%%%%%%%%%%%%%%%%%%%%%%%%%%%%%%%%%%%%%%%%%%%%%%%%%%%%%%%%
\section{Introduction}

\LaTeX{} provides a mechanism to structure a large document (such as a book)
into a main file and several child files (containing the chapters)
using the |\include| command.
This mechanism is beneficial for documents
which span hundreds of pages in order to
make the source file(s) more manageable.
Moreover, compilation can be restricted to
selected child files by means of the |\includeonly| command.
The latter feature can be used to reduce the compilation time while editing
(this was significantly more useful in the earlier days of \LaTeX{})
or to generate a smaller document which is easier to navigate.
Another application of |\includeonly| is to generate
documents consisting of selected parts of the complete document.

However, there are a few drawbacks of the plain |\include| mechanism:
\begin{itemize}
\item
The child files cannot be compiled on their own,
they can only be compiled via the main file.
A naive editing environment
(such as a text editor with an option
to have the current file processed by \LaTeX)
may require one to switch to the main file before compiling;
attempting to compile the child file produces errors.
\item
The main file must be modified (each time)
to adjust the |\includeonly| command
to the present needs. This easily leaves the main file in a messy state.
\item
The generated document will always carry the filename
of the main document. This is inconvenient if
several child files are to be compiled and
to be kept for distribution.
\end{itemize}

The present package provides a simple interface
to make child files individually compilable by \LaTeX{}.
Compiling a child file then has the same effect as compiling
the main file with an |\includeonly| command
to select the appropriate child.
Moreover the generated document will carry the name of the child
rather than the main file.
This resolves all three above issues.

This feature is meant to make the editing of books,
thesis documents and lecture notes somewhat more convenient.
However, the package can also be used efficiently for
composing a series of documents (such as exercise sheets)
which are typically distributed individually.
It then assists the author in generating the individual documents
(potentially in different versions)
as well as a document containing the collected series.
Another application is in developing style files
or other kinds of included material
where compilation of the style file could redirect
to a sample or test file.

%%%%%%%%%%%%%%%%%%%%%%%%%%%%%%%%%%%%%%%%%%%%%%%%%%%%%%%%%%%%%%%%%%%%%%%%%%%%%%%%
%%%%%%%%%%%%%%%%%%%%%%%%%%%%%%%%%%%%%%%%%%%%%%%%%%%%%%%%%%%%%%%%%%%%%%%%%%%%%%%%
\section{Usage}

First of all, the package \textsf{childdoc} is \emph{not} a standard
\LaTeXe{} |.sty| style file! Therefore it needs to be invoked in
a non-standard way.

%%%%%%%%%%%%%%%%%%%%%%%%%%%%%%%%%%%%%%%%%%%%%%%%%%%%%%%%%%%%%%%%%%%%%%%%%%%%%%%%
\subsection{Included Files}
\label{sec:include}

%%%%%%%%%%%%%%%%%%%%%%%%%%%%%%%%%%%%%%%%
\DescribeMacro{\childdocmain}
To use the package, add the commands
\begin{center}
\begin{tabular}{l}
|\input{childdoc.def}|\\
|\childdocmain{}|\\
\end{tabular}
\end{center}
at the very top of the main \LaTeX{} file,
in particular \emph{before} the |\documentclass| statement!
The argument of |\childdocmain| should be left empty
(but it must be present).

%%%%%%%%%%%%%%%%%%%%%%%%%%%%%%%%%%%%%%%%
\DescribeMacro{\childdocof}
Furthermore, add the commands
\begin{center}
\begin{tabular}{l}
|\input{childdoc.def}|\\
|\childdocof{|\textit{main}|}|\\
\end{tabular}
\end{center}
at the top of every child file \textit{child}
which is included by |\include{|\textit{child}|}|
from within the main file
(or at least for those files to be compiled individually).
The argument \textit{main} must be the filename of the main file.

There are a couple of
considerations in setting up the main and child documents:

%%%%%%%%%%%%%%%%%%%%%%%%%%%%%%%%%%%%%%%%
\paragraph{Restrictions.}

Please note the following restrictions:
\begin{itemize}
\item
|\childdocmain| must be called with one argument \textit{main}
to ensure compatibility with earlier version of the package.
It must either be empty (|\childdocmain{}|)
or precisely match the filename of the main file in which it is specified.
See \secref{sec:detection} for further information.
\item
The filename \textit{main} must be specified without the |.tex| extension.
\item
The filename \textit{main} is case sensitive
(even in case-insensitive file systems)
due to internal string comparison.
\item
The argument \textit{main} should be fully expanded, it cannot be a macro.
\item
Subdirectories and special characters should be avoided in filenames.
\item
The command |\childdocmain{|\textit{main}|}| must be followed by a whitespace.
It should not be followed immediately by another command
or by a comment mark `|%|'.
This is because the \TeX{} parser reads the token immediately following
the argument of |\childdocmain| and puts it
at the beginning of every child section;
however, a white\-space is ignored.
\end{itemize}

%%%%%%%%%%%%%%%%%%%%%%%%%%%%%%%%%%%%%%%%
\paragraph{Content of Main File.}

It is advisable to place all content in the child files included by |\include|.
Any output contained in the main file will appear in all child documents
unless suppressed manually;
it cannot be suppressed automatically by the |\includeonly| directive
and thus should normally be avoided.
A method to include some content in the main file
by means of conditional processing is described in \secref{sec:conditional}.

%%%%%%%%%%%%%%%%%%%%%%%%%%%%%%%%%%%%%%%%
\paragraph{Page Numbering.}

When only a part of the document is compiled,
the appropriate numbering of pages
(as well as other status parameters)
is determined from the |.aux| files.
The latter contain information from previous passes.
However this information needs to propagate through
all intermediate child documents.
Therefore the page numbering in child documents may well
be inconsistent until the complete document is compiled at least once.

A useful (if unconventional) way to always ensure a consistent
page numbering is to restart the numbering in each child document
and denote the pages by `\textit{child}|.|\textit{page}'
where \textit{child} represents the chapter/section number of the child file.
This can be achieved by the command
|\numberwithin{page}{|\textit{child}|}|
of the \textsf{amsmath} package
where \textit{child} can be |chapter| or |section|
depending on the chosen structuring.
Alternatively, one can modify the macro |\thepage| appropriately
and reset the counter |page| at the start of each child file.

%%%%%%%%%%%%%%%%%%%%%%%%%%%%%%%%%%%%%%%%%%%%%%%%%%%%%%%%%%%%%%%%%%%%%%%%%%%%%%%%
\subsection{Conditional Processing}
\label{sec:conditional}

The package provides a mechanism to compile different versions
of a document. To customise the versions further some conditional processing
can come in handy to distinguish which version is being compiled.
The package provides two macros to describe the compilation context:

%%%%%%%%%%%%%%%%%%%%%%%%%%%%%%%%%%%%%%%%
\DescribeMacro{\ifchilddoc}
The conditional |\ifchilddoc| distinguishes between the compilation of
child documents and the main document:
%
\begin{center}
|\ifchilddoc |\textit{child-code}| |[|\||else |\textit{main-code}]| \||fi|
\end{center}

%%%%%%%%%%%%%%%%%%%%%%%%%%%%%%%%%%%%%%%%
\DescribeMacro{\childdocname}
\DescribeMacro{\childdocjob}
The macro |\childdocname| contains the filename (without extension)
of the main or child file being processed.
Note that |\childdocjob| will always contain the name of the main file.

%%%%%%%%%%%%%%%%%%%%%%%%%%%%%%%%%%%%%%%%
\paragraph{Title Page.}

Conditional processing can be used to include a title or banner page
in the main document when proper precautions are taken.
Importantly, the code in the main file should ensure that the page counter
(as well as other status parameters which are stored in the |.aux| files)
takes the same value after the conditional processing.
Otherwise the page numbers may take divergent values
depending on which part is compiled.

For example, a title page could be declared by:
%
\begin{center}
\begin{tabular}{l}
|\ifchilddoc\||else|\\
|\addtocounter{page}{-1}|\\
\textit{code for title page}\\
|\newpage|\\
|\||fi|
\end{tabular}
\end{center}
%
A banner page for the child documents can be generated by:
%
\begin{center}
\begin{tabular}{l}
|\ifchilddoc|\\
|\addtocounter{page}{-1}|\\
\textit{code for banner page}\\
|\newpage|\\
|\||fi|
\end{tabular}
\end{center}
%
Here one could write a message such as:
\begin{center}
|This is the part \childdocname{} of \childdocjob{}.|
\end{center}

%%%%%%%%%%%%%%%%%%%%%%%%%%%%%%%%%%%%%%%%%%%%%%%%%%%%%%%%%%%%%%%%%%%%%%%%%%%%%%%%
\subsection{Flags}
\label{sec:flags}

The package makes it easy to generate different versions
of the main or child documents.
To this end compilation flags can be defined
and assigned different default values.
They will be particularly useful in conjunction
with the forwarding mechanism described in \secref{sec:forward}.

For example, it may be useful to have a flag |\version|
which can be set to |draft| or |final|.
The document source will contain some conditional code
depending on the value of |\version|.
Suppose further, the flag should default to |final| for the main file
and to |draft| for child files
which is a natural assignment for editing the document.
This is achieved by placing the following code
in the preamble of the main document
(below the |\childdocmain| directive):
%
\begin{center}
\begin{tabular}{l}
|\ifchilddoc|\\
|\providecommand{\version}{draft}|\\
|\||else|\\
|\providecommand{\version}{final}|\\
|\||fi|
\end{tabular}
\end{center}
%
The definition by |\providecommand| makes sure
that previous definitions are not overwritten.
Further statements |\providecommand{\version}{...}|
can thus be added before the above code to override it.

For the main file, one might add a line
(between |\childdocmain| and the above block)
%
\begin{center}
|%\ifchilddoc\||else\providecommand{\version}{draft}\||fi|
\end{center}
%
which can be uncommented to produce a draft version.
Likewise one can add a line to the very top of a child file
(above the |\childdocof{|\textit{main}|}| directive)
%
\begin{center}
|%\providecommand{\version}{final}|
\end{center}
%
which can be uncommented to produce the final version of this child document.

%%%%%%%%%%%%%%%%%%%%%%%%%%%%%%%%%%%%%%%%%%%%%%%%%%%%%%%%%%%%%%%%%%%%%%%%%%%%%%%%
\subsection{Forwarding}
\label{sec:forward}

Different versions of the main or child documents
using compilation flags as described in \secref{sec:flags}
can be (permanently) stored in different files
for convenient compilation, viewing and distribution.
To this end, the package defines a command
to pass on compilation to a different file:

%%%%%%%%%%%%%%%%%%%%%%%%%%%%%%%%%%%%%%%%
\DescribeMacro{\childdocforward}
The command |\childdocforward| redirects processing to
another source file:
%
\begin{center}
\begin{tabular}{l}
|\input{childdoc.def}|\\
|\childdocforward[|\textit{main}|]{|\textit{dest}|}|\\
\end{tabular}
\end{center}
%
The argument \textit{dest} is the destination file
(without extension).
It should be the main file or one of the child files.
Note that further \textsf{childdoc} directives
such as |\childdocof| and |\childdocforward|
in the indicated file will be processed in this form.
The optional argument \textit{main}
passes on directly to the main file \textit{main}
while pretending to compile the child \textit{dest}.
This form behaves as if \textit{dest}
issues |\childdocof{|\textit{main}|}| right away,
and no further \textsf{childdoc} directives will be processed.

%%%%%%%%%%%%%%%%%%%%%%%%%%%%%%%%%%%%%%%%
\DescribeMacro{\...prefix}
In the alternative form |\childdocforwardprefix|,
%
\begin{center}
\begin{tabular}{l}
|\input{childdoc.def}|\\
|\childdocforwardprefix[|\textit{main}|]{|\textit{prefix}|}{|\textit{dest}|}|
\end{tabular}
\end{center}
%
the destination file is determined by a pattern
depending on the current file:
To make this work, the current file must be called
`{\textit{prefix}\hspace{0.2em}\textit{suffix}}'
with \textit{prefix} matching precisely the argument.
Processing is then passed on to the file
`{\textit{dest}\hspace{0.2em}\textit{suffix}}'.
Surely, the same effect is achieved by
directly specifying the
argument `{\textit{dest}\hspace{0.2em}\textit{suffix}}'
in the first form.
However, that requires to set up a different file
for each child. With the alternative form of the command
all these files can have exactly the same content
which simplifies setting them up and maintaining them.

For example, the following file |draft.tex|
with a compilation flag |\version| as described in \secref{sec:flags}
compiles the main document as a draft:
%
\begin{center}
\begin{tabular}{l}
|\def\version{draft}|\\
|\input{childdoc.def}|\\
|\childdocforward{|\textit{main}|}|
\end{tabular}
\end{center}
%
Likewise, the following files |final|\textit{nn}|.tex|
compile the final version of the child document
|child|\textit{nn}|.tex|:
%
\begin{center}
\begin{tabular}{l}
|\def\version{final}|\\
|\input{childdoc.def}|\\
|\childdocforwardprefix{final}{child}|
\end{tabular}
\end{center}
%

Note that when several versions of a main file and/or of each child file
are to be generated, it may be convenient to set up a |Makefile| or
shell script to automatise the process.

%%%%%%%%%%%%%%%%%%%%%%%%%%%%%%%%%%%%%%%%%%%%%%%%%%%%%%%%%%%%%%%%%%%%%%%%%%%%%%%%
\subsection{Command Line Processing}
\label{sec:commandline}

The effect of redirection files can also be achieved by invoking
the \LaTeX{} compiler with a more elaborate command line.
Most conveniently this should be done as part
of a shell script or a |Makefile|.

When using \textsf{childdoc} in the main file, the following
command lines effectively perform a redirection
(note that depending on the shell being used,
backslashes may have to be doubled: `|\|' $\to$ `|\\|'):
%
\begin{center}
|... -jobname "|\textit{target}|" |\\|"|[\textit{flags}]%
|\input{childdoc.def}\childdocforward[|\textit{main}|]{|\textit{dest}|}"|
\end{center}
%
Here \textit{target} is the name of the output file,
\textit{main} is the name of the main file
and \textit{dest} is the name of the main or child file to be processed
(all filenames without extensions).
The optional argument \textit{main} can be omitted
if \textit{main} matches \textit{dest}.
Optionally, compilation \textit{flags} can be defined via |\def| commands.
This command line makes the \TeX{} engine believe
it is compiling the file \textit{target}
whose content is specified as the latter parameter.
The provided code then forwards the processing to
\textit{main} or \textit{dest} as described in \secref{sec:forward}.

%%%%%%%%%%%%%%%%%%%%%%%%%%%%%%%%%%%%%%%%%%%%%%%%%%%%%%%%%%%%%%%%%%%%%%%%%%%%%%%%
\subsection{Include by Input}
\label{sec:input}

Including child documents by |\include| has some restrictions by design.
Most notably, the content of a child document always occupies
its own set of pages; pages cannot be shared between child documents.
Usually, this behaviour makes perfect sense
because each child document contain an essential part of the document.
However, in some situations it may be desirable to compose
a document from a collection of parts
without having mandatory page breaks between then.
For this case, the package
provides a mechanism to include parts
by |\input| which can also be processed individually.
However, by construction this mechanism
requires manual handling of the content to be output.

%%%%%%%%%%%%%%%%%%%%%%%%%%%%%%%%%%%%%%%%
\DescribeMacro{\ifchilddocmanual}
The main file should be prepared as usual, see \secref{sec:include}.
However, the document body must make a distinction
between processing of an individual part and of the main document, e.g.:
%
\begin{center}
\begin{tabular}{l}
|\ifchilddocmanual|\\
|\input{\childdocname}|\\
|\||else|\\
\textit{document body with }|\input{|\textit{part}|}|\\
|\||fi|
\end{tabular}
\end{center}
%
The conditional |\ifchilddocmanual| is true whenever
a part to be included by |\input| is being compiled,
and the name of the part is stored in |\childdocname|.

%%%%%%%%%%%%%%%%%%%%%%%%%%%%%%%%%%%%%%%%
\DescribeMacro{\childdocby}
Each part to be included by |\input| should start with:
%
\begin{center}
\begin{tabular}{l}
|\input{childdoc.def}|\\
|\childdocby{|\textit{main}|}|\\
\end{tabular}
\end{center}
%
The directive |\childdocby| is similar to |\childdocof|
described in \secref{sec:include},
but the subsequent selection of content must be done manually.
To that end, both |\ifchilddoc| and |\ifchilddocmanual|
will be true upon processing of a part,
and the name of the part is stored in |\childdocname|.
Note that |\jobname| will be set to the filename of the current part
so that each part receives an individual |.aux| file
that does not interfere with the |.aux| file(s) of the main document.
This behaviour can be altered by the alternative form
|\childdocby[*]{|\textit{main}|}| (with a non-empty optional argument)
which uses the |.aux| file of the main document
by setting |\jobname| to \textit{main}.

%%%%%%%%%%%%%%%%%%%%%%%%%%%%%%%%%%%%%%%%%%%%%%%%%%%%%%%%%%%%%%%%%%%%%%%%%%%%%%%%
\subsection{Driver Development}
\label{sec:driver}

The \textsf{childdoc} mechanism can also be use for the development
of definition files such as \LaTeX{} styles or classes.
This case differs from the above setup with multiple parts
included by |\include| in that no |\includeonly| should be invoked.
This can be achieved by starting the include file
(before |\ProvidesPackage|) with:
%
\begin{center}
\begin{tabular}{l}
|\input{childdoc.def}|\\
|\childdocforward{|\textit{main}|}|\\
\end{tabular}
\end{center}
%
or alternatively with:
%
\begin{center}
\begin{tabular}{l}
|\input{childdoc.def}|\\
|\childdocby{|\textit{main}|}|\\
\end{tabular}
\end{center}
%
Both forms have slightly different effects as described above.
The main file is prepared as usual, see \secref{sec:include}.

%%%%%%%%%%%%%%%%%%%%%%%%%%%%%%%%%%%%%%%%%%%%%%%%%%%%%%%%%%%%%%%%%%%%%%%%%%%%%%%%
\subsection{Legacy Detection}
\label{sec:detection}

The directive |\childdocmain| in the main file can detect
whether the complete document or merely a child is to be compiled
even without using the directive |\childdocof|.
This method is deprecated because it is less robust
and there is no compelling reason to use it;
it is merely provided for backward compatibility
and it may be removed in future versions.

If the detection mechanism is to be used,
it is mandatory to correctly specify
the filename of the main file as the argument of |\childdocmain|:
%
\begin{center}
\begin{tabular}{l}
|\input{childdoc.def}|\\
|\childdocmain{|\textit{main}|}|\\
\end{tabular}
\end{center}
%
If |\jobname| does not match the argument \textit{main} of |\childdocmain|,
it is assumed that |\jobname| points to the child file to be compiled.
When using |\childdocmain| with the main file specified as argument,
it suffices to start a child file
with just |\input{|\textit{main}|}|
without loading of the package and using |\childdocof|.
If instead all processing is done
with the appropriate \textsf{childdoc} directives,
the argument of \textit{main} of |\childdocmain| can be empty.

An alternative version of the command line processing described
in \secref{sec:commandline} using the detection mechanism reads:
%
\begin{center}
|... -jobname "|\textit{target}|" "|[\textit{flags}]%
[|\def\jobname{|\textit{dest}|}|]|\input{|\textit{main}|}"|
\end{center}

%%%%%%%%%%%%%%%%%%%%%%%%%%%%%%%%%%%%%%%%%%%%%%%%%%%%%%%%%%%%%%%%%%%%%%%%%%%%%%%%
\subsection{Manual Code}
\label{sec:manual}

In case one cannot be certain whether the definitions file |childdoc.def|
is installed on the target \TeX{} distribution
and one prefers not to ship it,
it is conceivable to paste a few relevant commands into the sources.

To that end, drop all statements |\input{childdoc.def}|
and perform the replacements as outlined below.
Instead of |\childdocmain{|\textit{main}|}| add the following code
to the top of the main file:
%
\begin{center}
\begin{tabular}{l}
|\||ifdefined\childdocname\endinput\||fi\newif\ifchilddoc|\\
|\edef\childdocname{\scantokens\expandafter{\jobname\noexpand}}|\\
|\def\childdocmain{|\textit{main}|}\||ifx\childdocmain\childdocname\||else|\\
|\childdoctrue\includeonly{\childdocname}\let\jobname\childdocmain\||fi|\\
\end{tabular}
\end{center}
%
Instead of |\childdocof{|\textit{main}|}| just include the main file
at the top of each child file:
%
\begin{center}
|\input{|\textit{main}|}|
\end{center}
%
A simple redirection |\childdocforward{|\textit{dest}|}| is achieved by:
%
\begin{center}
|\def\jobname{|\textit{dest}|}\input{\jobname}|
\end{center}
%
The redirection with prefix
|\childdocforwardprefix[|\textit{prefix}|]{|\textit{dest}|}|
is accomplished by:
%
\begin{center}
\begin{tabular}{l}
|{\edef\jobname{\scantokens\expandafter{\jobname\noexpand}}|\\
|\def\redirectjob |\textit{prefix}|#1~~~{\gdef\jobname{|\textit{dest}|#1}}|\\
|\expandafter\redirectjob\jobname~~~}\input{\jobname}|
\end{tabular}
\end{center}

In an alternative approach,
child documents can be compiled by a specific command line
without additional code or specific definitions:
%
\begin{center}
|... -jobname "|\textit{target}|" "|[\textit{flags}]%
|\includeonly{|\textit{dest}|}\input{|\textit{main}|}"|
\end{center}
%

%%%%%%%%%%%%%%%%%%%%%%%%%%%%%%%%%%%%%%%%%%%%%%%%%%%%%%%%%%%%%%%%%%%%%%%%%%%%%%%%
%%%%%%%%%%%%%%%%%%%%%%%%%%%%%%%%%%%%%%%%%%%%%%%%%%%%%%%%%%%%%%%%%%%%%%%%%%%%%%%%
\section{Information}

%%%%%%%%%%%%%%%%%%%%%%%%%%%%%%%%%%%%%%%%%%%%%%%%%%%%%%%%%%%%%%%%%%%%%%%%%%%%%%%%
\subsection{Copyright}

Copyright \copyright{} 2017--2018 Niklas Beisert

This work may be distributed and/or modified under the
conditions of the \LaTeX{} Project Public License, either version 1.3
of this license or (at your option) any later version.
The latest version of this license is in
  \url{http://www.latex-project.org/lppl.txt}
and version 1.3 or later is part of all distributions of \LaTeX{}
version 2005/12/01 or later.

This work has the LPPL maintenance status `maintained'.

The Current Maintainer of this work is Niklas Beisert.

This work consists of the files |README.txt|, |childdoc.ins| and |childdoc.dtx|
as well as the derived files |childdoc.def|, |cdocsamp.tex|
with |cdocsch1.tex|, |cdocsch2.tex|, |cdocspt3.tex|, |cdocspt4.tex|,
|cdocsdrf.tex|, |cdocsfn1.tex|, |cdocsfn2.tex|
as well as |childdoc.pdf|.

%%%%%%%%%%%%%%%%%%%%%%%%%%%%%%%%%%%%%%%%%%%%%%%%%%%%%%%%%%%%%%%%%%%%%%%%%%%%%%%%
\subsection{Files and Installation}

The package consists of the files:
%
\begin{center}
\begin{tabular}{ll}
    |README.txt|   & readme file \\
    |childdoc.ins| & installation file \\
    |childdoc.dtx| & source file \\
    |childdoc.def| & definition file \\
    |cdocsamp.tex| & sample main file \\
    |cdocsch1.tex| & sample include file \\
    |cdocsch2.tex| & sample include file \\
    |cdocspt3.tex| & sample part file \\
    |cdocspt4.tex| & sample part file \\
    |cdocsdrf.tex| & sample redirection file \\
    |cdocsfn1.tex| & sample redirection file \\
    |cdocsfn2.tex| & sample redirection file \\
    |childdoc.pdf| & manual
\end{tabular}
\end{center}
%
The distribution consists of the files
|README.txt|, |childdoc.ins| and |childdoc.dtx|.
%
\begin{itemize}
\item
Run (pdf)\LaTeX{} on |childdoc.dtx|
to compile the manual |childdoc.pdf| (this file).
\item
Run \LaTeX{} on |childdoc.ins| to create the definitions file |childdoc.def|
and the sample |cdocsamp.tex| with include files
|cdocsch1.tex|, |cdocsch2.tex|, |cdocspt3.tex|, |cdocspt4.tex|,
|cdocsdrf.tex|, |cdocsfn1.tex|, |cdocsfn2.tex|.
Then copy the file |childdoc.def| to an appropriate directory of your \LaTeX{}
distribution, e.g.\ \textit{texmf-root}|/tex/latex/childdoc|.
\end{itemize}

%%%%%%%%%%%%%%%%%%%%%%%%%%%%%%%%%%%%%%%%%%%%%%%%%%%%%%%%%%%%%%%%%%%%%%%%%%%%%%%%
\subsection{Related CTAN Packages}

There are several other packages which offer a similar functionality:
%
\begin{itemize}
\item
The packages
\href{http://ctan.org/pkg/docmute}{\textsf{docmute}},
\href{http://ctan.org/pkg/includex}{\textsf{includex}} and
\href{http://ctan.org/pkg/standalone}{\textsf{standalone}}
provide commands to include only the document body of
a child file thus allowing both files to be compiled individually.
\item
The packages \href{http://ctan.org/pkg/subdocs}{\textsf{subdocs}}
and \href{http://ctan.org/pkg/subfiles}{\textsf{subfiles}}
provide structures in which the main and child documents can be
encapsulated and allowing them to be compiled individually.
The inclusion mechanism is different from the conventional |\include|.
\item
The package \href{http://ctan.org/pkg/combine}{\textsf{combine}}
is an elaborate solution to combine several documents into one.
\end{itemize}
%
See also the CTAN topic \href{http://ctan.org/topic/subdocs}{\textsf{subdocs}}
for further related packages.
The present package differs from the above solutions in that
a document structure constructed with the conventional |\include| mechanism
just needs two extra commands at the top of every file
such that all constituent files can be compiled individually.

%%%%%%%%%%%%%%%%%%%%%%%%%%%%%%%%%%%%%%%%%%%%%%%%%%%%%%%%%%%%%%%%%%%%%%%%%%%%%%%%
%\subsection{Feature Suggestions}
%
%The following is a list of features which may be useful for future
%versions of this package:
%%
%\begin{itemize}
%\item
%\ldots
%\end{itemize}

%%%%%%%%%%%%%%%%%%%%%%%%%%%%%%%%%%%%%%%%%%%%%%%%%%%%%%%%%%%%%%%%%%%%%%%%%%%%%%%%
\subsection{Revision History}

%%%%%%%%%%%%%%%%%%%%%%%%%%%%%%%%%%%%%%%%
\paragraph{v2.0:} 2018/12/30

\begin{itemize}
\item
immediate forward processing
\item
added |\childdocby| mechanism
\item
manual restructured
\end{itemize}

%%%%%%%%%%%%%%%%%%%%%%%%%%%%%%%%%%%%%%%%
\paragraph{v1.6:} 2018/01/17

\begin{itemize}
\item
application for development of include files
\item
corrections to manual
\end{itemize}

%%%%%%%%%%%%%%%%%%%%%%%%%%%%%%%%%%%%%%%%
\paragraph{v1.5:} 2017/05/21

\begin{itemize}
\item
more complete structuring introduced
\item
|\childdocof| introduced
\item
|\childdoc| renamed to |\childdocmain|
\item
|\childredirect| renamed to |\childdocforward| and |\childdocforwardprefix|
and functionality expanded
\end{itemize}

%%%%%%%%%%%%%%%%%%%%%%%%%%%%%%%%%%%%%%%%
\paragraph{v1.0:} 2017/04/27

\begin{itemize}
\item
manual and install package
\item
first version published on CTAN
\end{itemize}

%%%%%%%%%%%%%%%%%%%%%%%%%%%%%%%%%%%%%%%%
\paragraph{v0.6:} 2017/04/26

\begin{itemize}
\item
redirection mechanism added
\end{itemize}

%%%%%%%%%%%%%%%%%%%%%%%%%%%%%%%%%%%%%%%%
\paragraph{v0.5:} 2017/04/26

\begin{itemize}
\item
functionality in definition file
\end{itemize}


%%%%%%%%%%%%%%%%%%%%%%%%%%%%%%%%%%%%%%%%%%%%%%%%%%%%%%%%%%%%%%%%%%%%%%%%%%%%%%%%
%%%%%%%%%%%%%%%%%%%%%%%%%%%%%%%%%%%%%%%%%%%%%%%%%%%%%%%%%%%%%%%%%%%%%%%%%%%%%%%%
%%%%%%%%%%%%%%%%%%%%%%%%%%%%%%%%%%%%%%%%%%%%%%%%%%%%%%%%%%%%%%%%%%%%%%%%%%%%%%%%
\appendix

\settowidth\MacroIndent{\rmfamily\scriptsize 000\ }

 \DocInput{childdoc.dtx}

\end{document}
%</driver>
% \fi
%
% %%%%%%%%%%%%%%%%%%%%%%%%%%%%%%%%%%%%%%%%%%%%%%%%%%%%%%%%%%%%%%%%%%%%%%%%%%%%%%
% %%%%%%%%%%%%%%%%%%%%%%%%%%%%%%%%%%%%%%%%%%%%%%%%%%%%%%%%%%%%%%%%%%%%%%%%%%%%%%
% \section{Sample}
%\iffalse
%<*samplemain>
%\fi
%
% The following presents a sample document
% with two chapters, two parts, a title page,
% a compile flag as well as three forwarding files to set the flag.
% It consists of eight |.tex| files:
% \begin{center}
% \begin{tabular}{ll}
% |cdocsamp.tex|&main file\\
% |cdocsch1.tex|&include file for chapter 1\\
% |cdocsch2.tex|&include file for chapter 2\\
% |cdocspt3.tex|&include file for part 3\\
% |cdocspt4.tex|&include file for part 4\\
% |cdocsdrf.tex|&forwarding file for main file in draft mode\\
% |cdocsfi1.tex|&forwarding file for final version of chapter 1\\
% |cdocsfi2.tex|&forwarding file for final version of chapter 2\\
% \end{tabular}
% \end{center}
% Each of the eight files can be compiled directly by the \LaTeX{} compiler.
%
% %%%%%%%%%%%%%%%%%%%%%%%%%%%%%%%%%%%%%%
% \paragraph{Main File.}
%
% The main file is called |cdocsamp.tex|.
%
% Load the \textsf{childdoc} definitions and
% declare the filename for the main document:
%    \begin{macrocode}
\input{childdoc.def}
\childdocmain{}
%    \end{macrocode}

% Optional override for |\version| flag:
%    \begin{macrocode}
%%\ifchilddoc\else\providecommand{\version}{draft}\fi
%    \end{macrocode}

% Define the default values for the |\version| flag
% (|final| for the main file and |draft| for childs):
%    \begin{macrocode}
\ifchilddoc
\providecommand{\version}{draft}
\else
\providecommand{\version}{final}
\fi
%    \end{macrocode}

% Load the standard document class:
%    \begin{macrocode}
\documentclass[12pt]{article}
%    \end{macrocode}

% Start the document body:
%    \begin{macrocode}
\begin{document}
%    \end{macrocode}

% Declare a title page.
% Print title, part of document being processed and version flag:
%    \begin{macrocode}
\addtocounter{page}{-1}
\begin{center}
{\LARGE\bfseries{}childdoc example\par}
\vspace{1cm}
\ifchilddoc
\ifchilddocmanual part\else chapter\fi:
`\childdocname' of `\childdocjob'\par
\else
main document: `\childdocjob'\par
\fi
version: \version\par
\end{center}
\newpage
%    \end{macrocode}

% Manually include selected file,
% otherwise process as usual:
%    \begin{macrocode}
\ifchilddocmanual
\section*{part `\childdocname'}
\input{\childdocname}
\else
%    \end{macrocode}

% Include the two chapters:
%    \begin{macrocode}
\include{cdocsch1}
\include{cdocsch2}
%    \end{macrocode}

% Include the two parts unless only chapters should be displayed:
%    \begin{macrocode}
\ifchilddoc\else
\section{part three}
\input{cdocspt3}
\section{part four}
\input{cdocspt4}
\fi
%    \end{macrocode}

% Process as usual until here:
%    \begin{macrocode}
\fi
%    \end{macrocode}

% End of document body:
%    \begin{macrocode}
\end{document}
%    \end{macrocode}
%\iffalse
%</samplemain>
%\fi
%
% %%%%%%%%%%%%%%%%%%%%%%%%%%%%%%%%%%%%%%
% \paragraph{Chapter Include Files.}
%
% The include files are called |cdocsch1.tex| and |cdocsch2.tex|.
%
%\iffalse
%<*samplechap1|samplechap2>
%\fi

% Optional override for |\version| flag:
%    \begin{macrocode}
%%\providecommand{\version}{final}
%    \end{macrocode}

% Include the main document:
%    \begin{macrocode}
\input{childdoc.def}
\childdocof{cdocsamp}
%    \end{macrocode}

%\iffalse
%</samplechap1|samplechap2>
%\fi
%
%\iffalse
%<*samplechap1>
%\fi
% Some text for chapter 1:
%    \begin{macrocode}
\section{one}
some text in chapter one
%    \end{macrocode}

%\iffalse
%</samplechap1>
%\fi
% Some text for chapter 2:
%\iffalse
%<*samplechap2>
%\fi
%    \begin{macrocode}
\section{two}
more text in chapter two
%    \end{macrocode}

%\iffalse
%</samplechap2>
%\fi
%
% %%%%%%%%%%%%%%%%%%%%%%%%%%%%%%%%%%%%%%
% \paragraph{Part Include Files.}
%
% The include files are called |cdocspt3.tex| and |cdocspt4.tex|.
%
%\iffalse
%<*samplepart3|samplepart4>
%\fi

% Optional override for |\version| flag:
%    \begin{macrocode}
%%\providecommand{\version}{final}
%    \end{macrocode}

% Include the main document:
%    \begin{macrocode}
\input{childdoc.def}
\childdocby{cdocsamp}
%    \end{macrocode}

%\iffalse
%</samplepart3|samplepart4>
%\fi
%
%\iffalse
%<*samplepart3>
%\fi
% Some text for part 3:
%    \begin{macrocode}
some text in part three
%    \end{macrocode}

%\iffalse
%</samplepart3>
%\fi
% Some text for part 4:
%\iffalse
%<*samplepart4>
%\fi
%    \begin{macrocode}
more text in part four
%    \end{macrocode}

%\iffalse
%</samplepart4>
%\fi
%
% %%%%%%%%%%%%%%%%%%%%%%%%%%%%%%%%%%%%%%
% \paragraph{Forwarding for a Complete Draft.}
%
% The following forwarding file |cdocsdrf.tex|
% compiles the main document in draft mode:
%\iffalse
%<*sampledraft>
%\fi
%    \begin{macrocode}
\def\version{draft}
\input{childdoc.def}
\childdocforward{cdocsamp}
%    \end{macrocode}

%\iffalse
%</sampledraft>
%\fi
%
% %%%%%%%%%%%%%%%%%%%%%%%%%%%%%%%%%%%%%%
% \paragraph{Forwarding for Final Version of the Chapters.}
%
% The following forwarding files |cdocsfn1.tex| and |cdocsfn2.tex|
% (with identical content)
% compile the final versions of the child documents
% |cdocsch1.tex| and |cdocsch2.tex|, respectively:
%\iffalse
%<*samplefinal>
%\fi
%    \begin{macrocode}
\def\version{final}
\input{childdoc.def}
\childdocforwardprefix[cdocsamp]{cdocsfn}{cdocsch}
%    \end{macrocode}

%\iffalse
%</samplefinal>
%\fi
%
% %%%%%%%%%%%%%%%%%%%%%%%%%%%%%%%%%%%%%%
% \paragraph{Command Line Processing.}
%
% The following three command lines generate the output files
% |cdocscld|, |cdocscl1| and |cdocscl2|
% which should be identical to
% |cdocsdrf|, |cdocsch1| and |cdocsfn2|, respectively:
% \begin{center}
% \begin{tabular}{l}
% |latex -jobname cdocscld \|\\
% |  "\def\version{draft}\input{childdoc.def}\childdocforward{cdocsamp}"|\\
% |latex -jobname cdocscl1 \|\\
% |  "\input{childdoc.def}\childdocforward[cdocsamp]{cdocsch1}"|\\
% |latex -jobname cdocscl2 \|\\
% |  "\def\version{final}\input{childdoc.def}\childdocforward{cdocsch2}"|
% \end{tabular}
% \end{center}
% Note that the trailing backslash on each first line
% merely continues the input to the second line
% (for convenient cut ant paste).
% Furthermore, the command |latex| can be replaced by any
% of its alternative versions such as |pdflatex|.
%
% %%%%%%%%%%%%%%%%%%%%%%%%%%%%%%%%%%%%%%%%%%%%%%%%%%%%%%%%%%%%%%%%%%%%%%%%%%%%%%
% %%%%%%%%%%%%%%%%%%%%%%%%%%%%%%%%%%%%%%%%%%%%%%%%%%%%%%%%%%%%%%%%%%%%%%%%%%%%%%
% \section{Implementation}
%\iffalse
%<*package>
%\fi
%
% This section describes the definitions file |childdoc.def|.

% The definitions cannot be loaded using |\usepackage| or |\RequirePackage|
% which has a mechanism to prevent loading a style file more than once.
% When loading the definitions by means of |\input|
% multiple instances have to be prevented manually:
%\iffalse
%This code needs to be before the `\ProvidesFile' directive
%which is defined at the beginning of this file.
%Therefore it is also placed there and commented out here.
%</package>
%<*discard>
%\fi
%    \begin{macrocode}
\ifdefined\childdocmain\endinput\fi
%    \end{macrocode}
%\iffalse
%</discard>
%<*package>
%\fi
%
% \macro{\ifchilddoc}
% \macro{\ifchilddocmanual}
% The conditional |\ifchilddoc| tells whether a
% child (true) or main (false) document is being compiled.
% The conditional |\ifchilddocmanual| tells whether
% the |\includeonly| mechanism is used (false) or
% the selection of child files must be performed manually (true).
% The definitions initialise to false:
%    \begin{macrocode}
\newif\ifchilddoc
\newif\ifchilddocmanual
%    \end{macrocode}

% \macro{\childdocname}
% \macro{\childdocjob}
% The macro |\childdocname| stores the name of the main document
% to be compiled. The macro |\childdocjob| stores the name of
% the document on which the \LaTeX{} compiler was originally invoked.
% The content of |\jobname| cannot be compared
% to filenames specified in the source due to different catcodes.
% The following code rescans |\jobname|, stores the result
% in |\childdocname| and saves a copy in |\childdocjob|:
%    \begin{macrocode}
\edef\childdocname{\scantokens\expandafter{\jobname\noexpand}}
\let\childdocjob\childdocname
%    \end{macrocode}

% \macro{\childdocdisable}
% The macro |\childdocdisable| prevents the main file
% from being processed more than once.
% At this stage, the main document command |\childdocmain|
% is assumed to be called once again where it should do nothing.
% Any subsequent call to it should prevent
% a secondary processing of the main document
% It overwrites the forwarding commands
% |\childdocof| and |\childdocforward|
% with empty macros to prevent further inclusions of the main document:
%    \begin{macrocode}
\newcommand{\childdocdisable}
{
  \renewcommand{\childdocmain}[1]{\renewcommand{\childdocmain}[1]{\endinput}}
  \renewcommand{\childdocof}[1]{}
  \renewcommand{\childdocby}[2][]{}
  \renewcommand{\childdocforward}[2][]{}
  \renewcommand{\childdocdisable}{}
}
%    \end{macrocode}

% \macro{\childdocmain}
% The macro |\childdocmain| is to be called at the top of the main file
% with nothing or the main filename (without extension) as argument.
% First, it breaks loops.
% If the argument is not empty and does not match |\childdocname|
% (which is set by the first inclusion of |childdoc.def|),
% |\ifchilddoc| is set to true, |\includeonly| is applied to the child file
% and |\jobname| is set to the main file
% (for proper handling of |.aux| files):
%    \begin{macrocode}
\newcommand{\childdocmain}[1]
{
  \childdocdisable\childdocmain{}
  \if?#1?\else
    \begingroup
      \def\childdoctmp{#1}
      \ifx\childdoctmp\childdocname
        \def\childdoctmp{}
      \else
        \def\childdoctmp
        {
          \childdoctrue
          \includeonly{\childdocname}
          \def\childdocjob{#1}
          \def\jobname{#1}
        }
      \fi
      \expandafter
    \endgroup
    \childdoctmp
  \fi
}
%    \end{macrocode}

% \macro{\childdocof}
% The command |\childdocof| redirects
% compilation to the main file |#1|.
%    \begin{macrocode}
\newcommand{\childdocof}[1]
{
  \childdocdisable
  \childdoctrue
  \includeonly{\childdocname}
  \def\jobname{#1}
  \def\childdocjob{#1}
  \input{#1}
}
%    \end{macrocode}

% \macro{\childdocby}
% The command |\childdocby| ....
%    \begin{macrocode}
\newcommand{\childdocby}[2][]
{
  \childdocdisable
  \childdoctrue
  \childdocmanualtrue
  \if?#1?\else
    \def\jobname{#2}
  \fi
  \def\childdocjob{#2}
  \input{#2}
  \endinput
}
%    \end{macrocode}

% \macro{\childdocforward}
% The command |\childdocforward| redirects
% compilation to the main file or
% (if the optional argument is given) a child file.
% Parameters are set as if the main file
% or a child file starting with |\childdocof| was compiled.
% Then compilation is handed over to the main file:
%    \begin{macrocode}
\newcommand{\childdocforward}[2][]
{
  \begingroup
    \if?#1?
      \def\childdoctmp
      {
        \def\childdocname{#2}
        \def\childdocjob{#2}
        \def\jobname{#2}
        \input{#2}
        \endinput
      }
    \else
      \def\childdoctmp
      {
        \childdocdisable
        \def\childdocname{#2}
        \childdoctrue
        \includeonly{#2}
        \def\childdocjob{#1}
        \def\jobname{#1}
        \input{#1}
        \endinput
      }
    \fi
    \expandafter
  \endgroup
  \childdoctmp
}
%    \end{macrocode}

% \macro{\childdocforwardprefix}
% The command |\childdocforwardprefix| redirects
% compilation to the main or a child file by means of a pattern.
% The prefix |#1| in the current filename is replaced by |#2|
% and the suffix of the current filename is kept
% (it is assumed that the filename does not contain the substring `|~~~|'
% which is used as a delimiter).
% Compilation is handed over to the new file by |\childdocforward|:
%    \begin{macrocode}
\newcommand{\childdocforwardprefix}[3][]
{
  \begingroup
    \def\childdocextract #2##1~~~{\def\childdoctmp{\childdocforward[#1]{#3##1}}}
    \expandafter\childdocextract\childdocname~~~
    \expandafter
  \endgroup
  \childdoctmp
}
%    \end{macrocode}

% \macro{\childdoc}
% The deprecated macro |\childdoc| is a legacy version of |\childdocmain|:
%    \begin{macrocode}
\newcommand{\childdoc}{\childdocmain}
%    \end{macrocode}

% \macro{\childdocredirect}
% The deprecated macro |\childdocredirect| is a legacy version
% of |\childdocforward| and |\childdocforwardprefix|:
%    \begin{macrocode}
\newcommand{\childdocredirect}[2][]
{
  \begingroup
    \if?#1?
      \def\childdoctmp{\childdocforward{#2}}
    \else
      \def\childdoctmp{\childdocforwardprefix{#1}{#2}}
    \fi
    \expandafter
  \endgroup
  \childdoctmp
}
%    \end{macrocode}

%\iffalse
%</package>
%\fi
%
\endinput
|\\
|\childdocby{|\textit{main}|}|\\
\end{tabular}
\end{center}
%
Both forms have slightly different effects as described above.
The main file is prepared as usual, see \secref{sec:include}.

%%%%%%%%%%%%%%%%%%%%%%%%%%%%%%%%%%%%%%%%%%%%%%%%%%%%%%%%%%%%%%%%%%%%%%%%%%%%%%%%
\subsection{Legacy Detection}
\label{sec:detection}

The directive |\childdocmain| in the main file can detect
whether the complete document or merely a child is to be compiled
even without using the directive |\childdocof|.
This method is deprecated because it is less robust
and there is no compelling reason to use it;
it is merely provided for backward compatibility
and it may be removed in future versions.

If the detection mechanism is to be used,
it is mandatory to correctly specify
the filename of the main file as the argument of |\childdocmain|:
%
\begin{center}
\begin{tabular}{l}
|% \iffalse
%
% childdoc.dtx Copyright (C) 2017-2018 Niklas Beisert
%
% This work may be distributed and/or modified under the
% conditions of the LaTeX Project Public License, either version 1.3
% of this license or (at your option) any later version.
% The latest version of this license is in
%   http://www.latex-project.org/lppl.txt
% and version 1.3 or later is part of all distributions of LaTeX
% version 2005/12/01 or later.
%
% This work has the LPPL maintenance status `maintained'.
%
% The Current Maintainer of this work is Niklas Beisert.
%
% This work consists of the files childdoc.dtx and childdoc.ins
% and the derived files childdoc.def and cdocsamp.tex with
% cdocsch1.tex, cdocsch2.tex, cdocsdrf.tex, cdocsfn1.tex, cdocsfn2.tex.
%
%<package>\ifdefined\childdocmain\endinput\fi
%<package>\ProvidesFile{childdoc.def}[2018/12/30 v2.0 child document driver]
%<samplemain>\ProvidesFile{cdocsamp.tex}[2018/12/30 v2.0 sample for childdoc]
%<*driver>
%\ProvidesFile{childdoc.drv}[2018/12/30 v2.0 childdoc reference manual file]
\PassOptionsToClass{10pt,a4paper}{article}
\documentclass{ltxdoc}

\usepackage[margin=35mm]{geometry}
\usepackage{hyperref}
\usepackage{hyperxmp}
\usepackage[usenames]{color}

\hypersetup{colorlinks=true}
\hypersetup{pdfstartview=FitH}
\hypersetup{pdfpagemode=UseNone}
\hypersetup{pdfsource={}}
\hypersetup{pdflang={en-UK}}
\hypersetup{pdfcopyright={Copyright 2017-2018 Niklas Beisert.
  This work may be distributed and/or modified under the
  conditions of the LaTeX Project Public License, either version 1.3
  of this license or (at your option) any later version.}}
\hypersetup{pdflicenseurl={http://www.latex-project.org/lppl.txt}}
\hypersetup{pdfcontactaddress={ETH Zurich, ITP, HIT K,
  Wolfgang-Pauli-Strasse 27}}
\hypersetup{pdfcontactpostcode={8093}}
\hypersetup{pdfcontactcity={Zurich}}
\hypersetup{pdfcontactcountry={Switzerland}}
\hypersetup{pdfcontactemail={nbeisert@itp.phys.ethz.ch}}
\hypersetup{pdfcontacturl={http://people.phys.ethz.ch/\xmptilde nbeisert/}}

\newcommand{\secref}[1]{\hyperref[#1]{section \ref*{#1}}}

\parskip1ex
\parindent0pt
\let\olditemize\itemize
\def\itemize{\olditemize\parskip0pt}

\begin{document}

\title{The \textsf{childdoc} Package}
\hypersetup{pdftitle={The childdoc Package}}
\author{Niklas Beisert\\[2ex]
  Institut f\"ur Theoretische Physik\\
  Eidgen\"ossische Technische Hochschule Z\"urich\\
  Wolfgang-Pauli-Strasse 27, 8093 Z\"urich, Switzerland\\[1ex]
  \href{mailto:nbeisert@itp.phys.ethz.ch}
  {\texttt{nbeisert@itp.phys.ethz.ch}}}
\hypersetup{pdfauthor={Niklas Beisert}}
\hypersetup{pdfsubject={Manual for the LaTeX2e Package childdoc}}
\date{30 December 2018, \textsf{v2.0}}
\maketitle

\begin{abstract}\noindent
\textsf{childdoc} is a \LaTeXe{} package
that enables the direct compilation
of document sections included by |\include|
to individual files.
\end{abstract}

\begingroup
\parskip0ex
\tableofcontents
\endgroup

%%%%%%%%%%%%%%%%%%%%%%%%%%%%%%%%%%%%%%%%%%%%%%%%%%%%%%%%%%%%%%%%%%%%%%%%%%%%%%%%
%%%%%%%%%%%%%%%%%%%%%%%%%%%%%%%%%%%%%%%%%%%%%%%%%%%%%%%%%%%%%%%%%%%%%%%%%%%%%%%%
\section{Introduction}

\LaTeX{} provides a mechanism to structure a large document (such as a book)
into a main file and several child files (containing the chapters)
using the |\include| command.
This mechanism is beneficial for documents
which span hundreds of pages in order to
make the source file(s) more manageable.
Moreover, compilation can be restricted to
selected child files by means of the |\includeonly| command.
The latter feature can be used to reduce the compilation time while editing
(this was significantly more useful in the earlier days of \LaTeX{})
or to generate a smaller document which is easier to navigate.
Another application of |\includeonly| is to generate
documents consisting of selected parts of the complete document.

However, there are a few drawbacks of the plain |\include| mechanism:
\begin{itemize}
\item
The child files cannot be compiled on their own,
they can only be compiled via the main file.
A naive editing environment
(such as a text editor with an option
to have the current file processed by \LaTeX)
may require one to switch to the main file before compiling;
attempting to compile the child file produces errors.
\item
The main file must be modified (each time)
to adjust the |\includeonly| command
to the present needs. This easily leaves the main file in a messy state.
\item
The generated document will always carry the filename
of the main document. This is inconvenient if
several child files are to be compiled and
to be kept for distribution.
\end{itemize}

The present package provides a simple interface
to make child files individually compilable by \LaTeX{}.
Compiling a child file then has the same effect as compiling
the main file with an |\includeonly| command
to select the appropriate child.
Moreover the generated document will carry the name of the child
rather than the main file.
This resolves all three above issues.

This feature is meant to make the editing of books,
thesis documents and lecture notes somewhat more convenient.
However, the package can also be used efficiently for
composing a series of documents (such as exercise sheets)
which are typically distributed individually.
It then assists the author in generating the individual documents
(potentially in different versions)
as well as a document containing the collected series.
Another application is in developing style files
or other kinds of included material
where compilation of the style file could redirect
to a sample or test file.

%%%%%%%%%%%%%%%%%%%%%%%%%%%%%%%%%%%%%%%%%%%%%%%%%%%%%%%%%%%%%%%%%%%%%%%%%%%%%%%%
%%%%%%%%%%%%%%%%%%%%%%%%%%%%%%%%%%%%%%%%%%%%%%%%%%%%%%%%%%%%%%%%%%%%%%%%%%%%%%%%
\section{Usage}

First of all, the package \textsf{childdoc} is \emph{not} a standard
\LaTeXe{} |.sty| style file! Therefore it needs to be invoked in
a non-standard way.

%%%%%%%%%%%%%%%%%%%%%%%%%%%%%%%%%%%%%%%%%%%%%%%%%%%%%%%%%%%%%%%%%%%%%%%%%%%%%%%%
\subsection{Included Files}
\label{sec:include}

%%%%%%%%%%%%%%%%%%%%%%%%%%%%%%%%%%%%%%%%
\DescribeMacro{\childdocmain}
To use the package, add the commands
\begin{center}
\begin{tabular}{l}
|\input{childdoc.def}|\\
|\childdocmain{}|\\
\end{tabular}
\end{center}
at the very top of the main \LaTeX{} file,
in particular \emph{before} the |\documentclass| statement!
The argument of |\childdocmain| should be left empty
(but it must be present).

%%%%%%%%%%%%%%%%%%%%%%%%%%%%%%%%%%%%%%%%
\DescribeMacro{\childdocof}
Furthermore, add the commands
\begin{center}
\begin{tabular}{l}
|\input{childdoc.def}|\\
|\childdocof{|\textit{main}|}|\\
\end{tabular}
\end{center}
at the top of every child file \textit{child}
which is included by |\include{|\textit{child}|}|
from within the main file
(or at least for those files to be compiled individually).
The argument \textit{main} must be the filename of the main file.

There are a couple of
considerations in setting up the main and child documents:

%%%%%%%%%%%%%%%%%%%%%%%%%%%%%%%%%%%%%%%%
\paragraph{Restrictions.}

Please note the following restrictions:
\begin{itemize}
\item
|\childdocmain| must be called with one argument \textit{main}
to ensure compatibility with earlier version of the package.
It must either be empty (|\childdocmain{}|)
or precisely match the filename of the main file in which it is specified.
See \secref{sec:detection} for further information.
\item
The filename \textit{main} must be specified without the |.tex| extension.
\item
The filename \textit{main} is case sensitive
(even in case-insensitive file systems)
due to internal string comparison.
\item
The argument \textit{main} should be fully expanded, it cannot be a macro.
\item
Subdirectories and special characters should be avoided in filenames.
\item
The command |\childdocmain{|\textit{main}|}| must be followed by a whitespace.
It should not be followed immediately by another command
or by a comment mark `|%|'.
This is because the \TeX{} parser reads the token immediately following
the argument of |\childdocmain| and puts it
at the beginning of every child section;
however, a white\-space is ignored.
\end{itemize}

%%%%%%%%%%%%%%%%%%%%%%%%%%%%%%%%%%%%%%%%
\paragraph{Content of Main File.}

It is advisable to place all content in the child files included by |\include|.
Any output contained in the main file will appear in all child documents
unless suppressed manually;
it cannot be suppressed automatically by the |\includeonly| directive
and thus should normally be avoided.
A method to include some content in the main file
by means of conditional processing is described in \secref{sec:conditional}.

%%%%%%%%%%%%%%%%%%%%%%%%%%%%%%%%%%%%%%%%
\paragraph{Page Numbering.}

When only a part of the document is compiled,
the appropriate numbering of pages
(as well as other status parameters)
is determined from the |.aux| files.
The latter contain information from previous passes.
However this information needs to propagate through
all intermediate child documents.
Therefore the page numbering in child documents may well
be inconsistent until the complete document is compiled at least once.

A useful (if unconventional) way to always ensure a consistent
page numbering is to restart the numbering in each child document
and denote the pages by `\textit{child}|.|\textit{page}'
where \textit{child} represents the chapter/section number of the child file.
This can be achieved by the command
|\numberwithin{page}{|\textit{child}|}|
of the \textsf{amsmath} package
where \textit{child} can be |chapter| or |section|
depending on the chosen structuring.
Alternatively, one can modify the macro |\thepage| appropriately
and reset the counter |page| at the start of each child file.

%%%%%%%%%%%%%%%%%%%%%%%%%%%%%%%%%%%%%%%%%%%%%%%%%%%%%%%%%%%%%%%%%%%%%%%%%%%%%%%%
\subsection{Conditional Processing}
\label{sec:conditional}

The package provides a mechanism to compile different versions
of a document. To customise the versions further some conditional processing
can come in handy to distinguish which version is being compiled.
The package provides two macros to describe the compilation context:

%%%%%%%%%%%%%%%%%%%%%%%%%%%%%%%%%%%%%%%%
\DescribeMacro{\ifchilddoc}
The conditional |\ifchilddoc| distinguishes between the compilation of
child documents and the main document:
%
\begin{center}
|\ifchilddoc |\textit{child-code}| |[|\||else |\textit{main-code}]| \||fi|
\end{center}

%%%%%%%%%%%%%%%%%%%%%%%%%%%%%%%%%%%%%%%%
\DescribeMacro{\childdocname}
\DescribeMacro{\childdocjob}
The macro |\childdocname| contains the filename (without extension)
of the main or child file being processed.
Note that |\childdocjob| will always contain the name of the main file.

%%%%%%%%%%%%%%%%%%%%%%%%%%%%%%%%%%%%%%%%
\paragraph{Title Page.}

Conditional processing can be used to include a title or banner page
in the main document when proper precautions are taken.
Importantly, the code in the main file should ensure that the page counter
(as well as other status parameters which are stored in the |.aux| files)
takes the same value after the conditional processing.
Otherwise the page numbers may take divergent values
depending on which part is compiled.

For example, a title page could be declared by:
%
\begin{center}
\begin{tabular}{l}
|\ifchilddoc\||else|\\
|\addtocounter{page}{-1}|\\
\textit{code for title page}\\
|\newpage|\\
|\||fi|
\end{tabular}
\end{center}
%
A banner page for the child documents can be generated by:
%
\begin{center}
\begin{tabular}{l}
|\ifchilddoc|\\
|\addtocounter{page}{-1}|\\
\textit{code for banner page}\\
|\newpage|\\
|\||fi|
\end{tabular}
\end{center}
%
Here one could write a message such as:
\begin{center}
|This is the part \childdocname{} of \childdocjob{}.|
\end{center}

%%%%%%%%%%%%%%%%%%%%%%%%%%%%%%%%%%%%%%%%%%%%%%%%%%%%%%%%%%%%%%%%%%%%%%%%%%%%%%%%
\subsection{Flags}
\label{sec:flags}

The package makes it easy to generate different versions
of the main or child documents.
To this end compilation flags can be defined
and assigned different default values.
They will be particularly useful in conjunction
with the forwarding mechanism described in \secref{sec:forward}.

For example, it may be useful to have a flag |\version|
which can be set to |draft| or |final|.
The document source will contain some conditional code
depending on the value of |\version|.
Suppose further, the flag should default to |final| for the main file
and to |draft| for child files
which is a natural assignment for editing the document.
This is achieved by placing the following code
in the preamble of the main document
(below the |\childdocmain| directive):
%
\begin{center}
\begin{tabular}{l}
|\ifchilddoc|\\
|\providecommand{\version}{draft}|\\
|\||else|\\
|\providecommand{\version}{final}|\\
|\||fi|
\end{tabular}
\end{center}
%
The definition by |\providecommand| makes sure
that previous definitions are not overwritten.
Further statements |\providecommand{\version}{...}|
can thus be added before the above code to override it.

For the main file, one might add a line
(between |\childdocmain| and the above block)
%
\begin{center}
|%\ifchilddoc\||else\providecommand{\version}{draft}\||fi|
\end{center}
%
which can be uncommented to produce a draft version.
Likewise one can add a line to the very top of a child file
(above the |\childdocof{|\textit{main}|}| directive)
%
\begin{center}
|%\providecommand{\version}{final}|
\end{center}
%
which can be uncommented to produce the final version of this child document.

%%%%%%%%%%%%%%%%%%%%%%%%%%%%%%%%%%%%%%%%%%%%%%%%%%%%%%%%%%%%%%%%%%%%%%%%%%%%%%%%
\subsection{Forwarding}
\label{sec:forward}

Different versions of the main or child documents
using compilation flags as described in \secref{sec:flags}
can be (permanently) stored in different files
for convenient compilation, viewing and distribution.
To this end, the package defines a command
to pass on compilation to a different file:

%%%%%%%%%%%%%%%%%%%%%%%%%%%%%%%%%%%%%%%%
\DescribeMacro{\childdocforward}
The command |\childdocforward| redirects processing to
another source file:
%
\begin{center}
\begin{tabular}{l}
|\input{childdoc.def}|\\
|\childdocforward[|\textit{main}|]{|\textit{dest}|}|\\
\end{tabular}
\end{center}
%
The argument \textit{dest} is the destination file
(without extension).
It should be the main file or one of the child files.
Note that further \textsf{childdoc} directives
such as |\childdocof| and |\childdocforward|
in the indicated file will be processed in this form.
The optional argument \textit{main}
passes on directly to the main file \textit{main}
while pretending to compile the child \textit{dest}.
This form behaves as if \textit{dest}
issues |\childdocof{|\textit{main}|}| right away,
and no further \textsf{childdoc} directives will be processed.

%%%%%%%%%%%%%%%%%%%%%%%%%%%%%%%%%%%%%%%%
\DescribeMacro{\...prefix}
In the alternative form |\childdocforwardprefix|,
%
\begin{center}
\begin{tabular}{l}
|\input{childdoc.def}|\\
|\childdocforwardprefix[|\textit{main}|]{|\textit{prefix}|}{|\textit{dest}|}|
\end{tabular}
\end{center}
%
the destination file is determined by a pattern
depending on the current file:
To make this work, the current file must be called
`{\textit{prefix}\hspace{0.2em}\textit{suffix}}'
with \textit{prefix} matching precisely the argument.
Processing is then passed on to the file
`{\textit{dest}\hspace{0.2em}\textit{suffix}}'.
Surely, the same effect is achieved by
directly specifying the
argument `{\textit{dest}\hspace{0.2em}\textit{suffix}}'
in the first form.
However, that requires to set up a different file
for each child. With the alternative form of the command
all these files can have exactly the same content
which simplifies setting them up and maintaining them.

For example, the following file |draft.tex|
with a compilation flag |\version| as described in \secref{sec:flags}
compiles the main document as a draft:
%
\begin{center}
\begin{tabular}{l}
|\def\version{draft}|\\
|\input{childdoc.def}|\\
|\childdocforward{|\textit{main}|}|
\end{tabular}
\end{center}
%
Likewise, the following files |final|\textit{nn}|.tex|
compile the final version of the child document
|child|\textit{nn}|.tex|:
%
\begin{center}
\begin{tabular}{l}
|\def\version{final}|\\
|\input{childdoc.def}|\\
|\childdocforwardprefix{final}{child}|
\end{tabular}
\end{center}
%

Note that when several versions of a main file and/or of each child file
are to be generated, it may be convenient to set up a |Makefile| or
shell script to automatise the process.

%%%%%%%%%%%%%%%%%%%%%%%%%%%%%%%%%%%%%%%%%%%%%%%%%%%%%%%%%%%%%%%%%%%%%%%%%%%%%%%%
\subsection{Command Line Processing}
\label{sec:commandline}

The effect of redirection files can also be achieved by invoking
the \LaTeX{} compiler with a more elaborate command line.
Most conveniently this should be done as part
of a shell script or a |Makefile|.

When using \textsf{childdoc} in the main file, the following
command lines effectively perform a redirection
(note that depending on the shell being used,
backslashes may have to be doubled: `|\|' $\to$ `|\\|'):
%
\begin{center}
|... -jobname "|\textit{target}|" |\\|"|[\textit{flags}]%
|\input{childdoc.def}\childdocforward[|\textit{main}|]{|\textit{dest}|}"|
\end{center}
%
Here \textit{target} is the name of the output file,
\textit{main} is the name of the main file
and \textit{dest} is the name of the main or child file to be processed
(all filenames without extensions).
The optional argument \textit{main} can be omitted
if \textit{main} matches \textit{dest}.
Optionally, compilation \textit{flags} can be defined via |\def| commands.
This command line makes the \TeX{} engine believe
it is compiling the file \textit{target}
whose content is specified as the latter parameter.
The provided code then forwards the processing to
\textit{main} or \textit{dest} as described in \secref{sec:forward}.

%%%%%%%%%%%%%%%%%%%%%%%%%%%%%%%%%%%%%%%%%%%%%%%%%%%%%%%%%%%%%%%%%%%%%%%%%%%%%%%%
\subsection{Include by Input}
\label{sec:input}

Including child documents by |\include| has some restrictions by design.
Most notably, the content of a child document always occupies
its own set of pages; pages cannot be shared between child documents.
Usually, this behaviour makes perfect sense
because each child document contain an essential part of the document.
However, in some situations it may be desirable to compose
a document from a collection of parts
without having mandatory page breaks between then.
For this case, the package
provides a mechanism to include parts
by |\input| which can also be processed individually.
However, by construction this mechanism
requires manual handling of the content to be output.

%%%%%%%%%%%%%%%%%%%%%%%%%%%%%%%%%%%%%%%%
\DescribeMacro{\ifchilddocmanual}
The main file should be prepared as usual, see \secref{sec:include}.
However, the document body must make a distinction
between processing of an individual part and of the main document, e.g.:
%
\begin{center}
\begin{tabular}{l}
|\ifchilddocmanual|\\
|\input{\childdocname}|\\
|\||else|\\
\textit{document body with }|\input{|\textit{part}|}|\\
|\||fi|
\end{tabular}
\end{center}
%
The conditional |\ifchilddocmanual| is true whenever
a part to be included by |\input| is being compiled,
and the name of the part is stored in |\childdocname|.

%%%%%%%%%%%%%%%%%%%%%%%%%%%%%%%%%%%%%%%%
\DescribeMacro{\childdocby}
Each part to be included by |\input| should start with:
%
\begin{center}
\begin{tabular}{l}
|\input{childdoc.def}|\\
|\childdocby{|\textit{main}|}|\\
\end{tabular}
\end{center}
%
The directive |\childdocby| is similar to |\childdocof|
described in \secref{sec:include},
but the subsequent selection of content must be done manually.
To that end, both |\ifchilddoc| and |\ifchilddocmanual|
will be true upon processing of a part,
and the name of the part is stored in |\childdocname|.
Note that |\jobname| will be set to the filename of the current part
so that each part receives an individual |.aux| file
that does not interfere with the |.aux| file(s) of the main document.
This behaviour can be altered by the alternative form
|\childdocby[*]{|\textit{main}|}| (with a non-empty optional argument)
which uses the |.aux| file of the main document
by setting |\jobname| to \textit{main}.

%%%%%%%%%%%%%%%%%%%%%%%%%%%%%%%%%%%%%%%%%%%%%%%%%%%%%%%%%%%%%%%%%%%%%%%%%%%%%%%%
\subsection{Driver Development}
\label{sec:driver}

The \textsf{childdoc} mechanism can also be use for the development
of definition files such as \LaTeX{} styles or classes.
This case differs from the above setup with multiple parts
included by |\include| in that no |\includeonly| should be invoked.
This can be achieved by starting the include file
(before |\ProvidesPackage|) with:
%
\begin{center}
\begin{tabular}{l}
|\input{childdoc.def}|\\
|\childdocforward{|\textit{main}|}|\\
\end{tabular}
\end{center}
%
or alternatively with:
%
\begin{center}
\begin{tabular}{l}
|\input{childdoc.def}|\\
|\childdocby{|\textit{main}|}|\\
\end{tabular}
\end{center}
%
Both forms have slightly different effects as described above.
The main file is prepared as usual, see \secref{sec:include}.

%%%%%%%%%%%%%%%%%%%%%%%%%%%%%%%%%%%%%%%%%%%%%%%%%%%%%%%%%%%%%%%%%%%%%%%%%%%%%%%%
\subsection{Legacy Detection}
\label{sec:detection}

The directive |\childdocmain| in the main file can detect
whether the complete document or merely a child is to be compiled
even without using the directive |\childdocof|.
This method is deprecated because it is less robust
and there is no compelling reason to use it;
it is merely provided for backward compatibility
and it may be removed in future versions.

If the detection mechanism is to be used,
it is mandatory to correctly specify
the filename of the main file as the argument of |\childdocmain|:
%
\begin{center}
\begin{tabular}{l}
|\input{childdoc.def}|\\
|\childdocmain{|\textit{main}|}|\\
\end{tabular}
\end{center}
%
If |\jobname| does not match the argument \textit{main} of |\childdocmain|,
it is assumed that |\jobname| points to the child file to be compiled.
When using |\childdocmain| with the main file specified as argument,
it suffices to start a child file
with just |\input{|\textit{main}|}|
without loading of the package and using |\childdocof|.
If instead all processing is done
with the appropriate \textsf{childdoc} directives,
the argument of \textit{main} of |\childdocmain| can be empty.

An alternative version of the command line processing described
in \secref{sec:commandline} using the detection mechanism reads:
%
\begin{center}
|... -jobname "|\textit{target}|" "|[\textit{flags}]%
[|\def\jobname{|\textit{dest}|}|]|\input{|\textit{main}|}"|
\end{center}

%%%%%%%%%%%%%%%%%%%%%%%%%%%%%%%%%%%%%%%%%%%%%%%%%%%%%%%%%%%%%%%%%%%%%%%%%%%%%%%%
\subsection{Manual Code}
\label{sec:manual}

In case one cannot be certain whether the definitions file |childdoc.def|
is installed on the target \TeX{} distribution
and one prefers not to ship it,
it is conceivable to paste a few relevant commands into the sources.

To that end, drop all statements |\input{childdoc.def}|
and perform the replacements as outlined below.
Instead of |\childdocmain{|\textit{main}|}| add the following code
to the top of the main file:
%
\begin{center}
\begin{tabular}{l}
|\||ifdefined\childdocname\endinput\||fi\newif\ifchilddoc|\\
|\edef\childdocname{\scantokens\expandafter{\jobname\noexpand}}|\\
|\def\childdocmain{|\textit{main}|}\||ifx\childdocmain\childdocname\||else|\\
|\childdoctrue\includeonly{\childdocname}\let\jobname\childdocmain\||fi|\\
\end{tabular}
\end{center}
%
Instead of |\childdocof{|\textit{main}|}| just include the main file
at the top of each child file:
%
\begin{center}
|\input{|\textit{main}|}|
\end{center}
%
A simple redirection |\childdocforward{|\textit{dest}|}| is achieved by:
%
\begin{center}
|\def\jobname{|\textit{dest}|}\input{\jobname}|
\end{center}
%
The redirection with prefix
|\childdocforwardprefix[|\textit{prefix}|]{|\textit{dest}|}|
is accomplished by:
%
\begin{center}
\begin{tabular}{l}
|{\edef\jobname{\scantokens\expandafter{\jobname\noexpand}}|\\
|\def\redirectjob |\textit{prefix}|#1~~~{\gdef\jobname{|\textit{dest}|#1}}|\\
|\expandafter\redirectjob\jobname~~~}\input{\jobname}|
\end{tabular}
\end{center}

In an alternative approach,
child documents can be compiled by a specific command line
without additional code or specific definitions:
%
\begin{center}
|... -jobname "|\textit{target}|" "|[\textit{flags}]%
|\includeonly{|\textit{dest}|}\input{|\textit{main}|}"|
\end{center}
%

%%%%%%%%%%%%%%%%%%%%%%%%%%%%%%%%%%%%%%%%%%%%%%%%%%%%%%%%%%%%%%%%%%%%%%%%%%%%%%%%
%%%%%%%%%%%%%%%%%%%%%%%%%%%%%%%%%%%%%%%%%%%%%%%%%%%%%%%%%%%%%%%%%%%%%%%%%%%%%%%%
\section{Information}

%%%%%%%%%%%%%%%%%%%%%%%%%%%%%%%%%%%%%%%%%%%%%%%%%%%%%%%%%%%%%%%%%%%%%%%%%%%%%%%%
\subsection{Copyright}

Copyright \copyright{} 2017--2018 Niklas Beisert

This work may be distributed and/or modified under the
conditions of the \LaTeX{} Project Public License, either version 1.3
of this license or (at your option) any later version.
The latest version of this license is in
  \url{http://www.latex-project.org/lppl.txt}
and version 1.3 or later is part of all distributions of \LaTeX{}
version 2005/12/01 or later.

This work has the LPPL maintenance status `maintained'.

The Current Maintainer of this work is Niklas Beisert.

This work consists of the files |README.txt|, |childdoc.ins| and |childdoc.dtx|
as well as the derived files |childdoc.def|, |cdocsamp.tex|
with |cdocsch1.tex|, |cdocsch2.tex|, |cdocspt3.tex|, |cdocspt4.tex|,
|cdocsdrf.tex|, |cdocsfn1.tex|, |cdocsfn2.tex|
as well as |childdoc.pdf|.

%%%%%%%%%%%%%%%%%%%%%%%%%%%%%%%%%%%%%%%%%%%%%%%%%%%%%%%%%%%%%%%%%%%%%%%%%%%%%%%%
\subsection{Files and Installation}

The package consists of the files:
%
\begin{center}
\begin{tabular}{ll}
    |README.txt|   & readme file \\
    |childdoc.ins| & installation file \\
    |childdoc.dtx| & source file \\
    |childdoc.def| & definition file \\
    |cdocsamp.tex| & sample main file \\
    |cdocsch1.tex| & sample include file \\
    |cdocsch2.tex| & sample include file \\
    |cdocspt3.tex| & sample part file \\
    |cdocspt4.tex| & sample part file \\
    |cdocsdrf.tex| & sample redirection file \\
    |cdocsfn1.tex| & sample redirection file \\
    |cdocsfn2.tex| & sample redirection file \\
    |childdoc.pdf| & manual
\end{tabular}
\end{center}
%
The distribution consists of the files
|README.txt|, |childdoc.ins| and |childdoc.dtx|.
%
\begin{itemize}
\item
Run (pdf)\LaTeX{} on |childdoc.dtx|
to compile the manual |childdoc.pdf| (this file).
\item
Run \LaTeX{} on |childdoc.ins| to create the definitions file |childdoc.def|
and the sample |cdocsamp.tex| with include files
|cdocsch1.tex|, |cdocsch2.tex|, |cdocspt3.tex|, |cdocspt4.tex|,
|cdocsdrf.tex|, |cdocsfn1.tex|, |cdocsfn2.tex|.
Then copy the file |childdoc.def| to an appropriate directory of your \LaTeX{}
distribution, e.g.\ \textit{texmf-root}|/tex/latex/childdoc|.
\end{itemize}

%%%%%%%%%%%%%%%%%%%%%%%%%%%%%%%%%%%%%%%%%%%%%%%%%%%%%%%%%%%%%%%%%%%%%%%%%%%%%%%%
\subsection{Related CTAN Packages}

There are several other packages which offer a similar functionality:
%
\begin{itemize}
\item
The packages
\href{http://ctan.org/pkg/docmute}{\textsf{docmute}},
\href{http://ctan.org/pkg/includex}{\textsf{includex}} and
\href{http://ctan.org/pkg/standalone}{\textsf{standalone}}
provide commands to include only the document body of
a child file thus allowing both files to be compiled individually.
\item
The packages \href{http://ctan.org/pkg/subdocs}{\textsf{subdocs}}
and \href{http://ctan.org/pkg/subfiles}{\textsf{subfiles}}
provide structures in which the main and child documents can be
encapsulated and allowing them to be compiled individually.
The inclusion mechanism is different from the conventional |\include|.
\item
The package \href{http://ctan.org/pkg/combine}{\textsf{combine}}
is an elaborate solution to combine several documents into one.
\end{itemize}
%
See also the CTAN topic \href{http://ctan.org/topic/subdocs}{\textsf{subdocs}}
for further related packages.
The present package differs from the above solutions in that
a document structure constructed with the conventional |\include| mechanism
just needs two extra commands at the top of every file
such that all constituent files can be compiled individually.

%%%%%%%%%%%%%%%%%%%%%%%%%%%%%%%%%%%%%%%%%%%%%%%%%%%%%%%%%%%%%%%%%%%%%%%%%%%%%%%%
%\subsection{Feature Suggestions}
%
%The following is a list of features which may be useful for future
%versions of this package:
%%
%\begin{itemize}
%\item
%\ldots
%\end{itemize}

%%%%%%%%%%%%%%%%%%%%%%%%%%%%%%%%%%%%%%%%%%%%%%%%%%%%%%%%%%%%%%%%%%%%%%%%%%%%%%%%
\subsection{Revision History}

%%%%%%%%%%%%%%%%%%%%%%%%%%%%%%%%%%%%%%%%
\paragraph{v2.0:} 2018/12/30

\begin{itemize}
\item
immediate forward processing
\item
added |\childdocby| mechanism
\item
manual restructured
\end{itemize}

%%%%%%%%%%%%%%%%%%%%%%%%%%%%%%%%%%%%%%%%
\paragraph{v1.6:} 2018/01/17

\begin{itemize}
\item
application for development of include files
\item
corrections to manual
\end{itemize}

%%%%%%%%%%%%%%%%%%%%%%%%%%%%%%%%%%%%%%%%
\paragraph{v1.5:} 2017/05/21

\begin{itemize}
\item
more complete structuring introduced
\item
|\childdocof| introduced
\item
|\childdoc| renamed to |\childdocmain|
\item
|\childredirect| renamed to |\childdocforward| and |\childdocforwardprefix|
and functionality expanded
\end{itemize}

%%%%%%%%%%%%%%%%%%%%%%%%%%%%%%%%%%%%%%%%
\paragraph{v1.0:} 2017/04/27

\begin{itemize}
\item
manual and install package
\item
first version published on CTAN
\end{itemize}

%%%%%%%%%%%%%%%%%%%%%%%%%%%%%%%%%%%%%%%%
\paragraph{v0.6:} 2017/04/26

\begin{itemize}
\item
redirection mechanism added
\end{itemize}

%%%%%%%%%%%%%%%%%%%%%%%%%%%%%%%%%%%%%%%%
\paragraph{v0.5:} 2017/04/26

\begin{itemize}
\item
functionality in definition file
\end{itemize}


%%%%%%%%%%%%%%%%%%%%%%%%%%%%%%%%%%%%%%%%%%%%%%%%%%%%%%%%%%%%%%%%%%%%%%%%%%%%%%%%
%%%%%%%%%%%%%%%%%%%%%%%%%%%%%%%%%%%%%%%%%%%%%%%%%%%%%%%%%%%%%%%%%%%%%%%%%%%%%%%%
%%%%%%%%%%%%%%%%%%%%%%%%%%%%%%%%%%%%%%%%%%%%%%%%%%%%%%%%%%%%%%%%%%%%%%%%%%%%%%%%
\appendix

\settowidth\MacroIndent{\rmfamily\scriptsize 000\ }

 \DocInput{childdoc.dtx}

\end{document}
%</driver>
% \fi
%
% %%%%%%%%%%%%%%%%%%%%%%%%%%%%%%%%%%%%%%%%%%%%%%%%%%%%%%%%%%%%%%%%%%%%%%%%%%%%%%
% %%%%%%%%%%%%%%%%%%%%%%%%%%%%%%%%%%%%%%%%%%%%%%%%%%%%%%%%%%%%%%%%%%%%%%%%%%%%%%
% \section{Sample}
%\iffalse
%<*samplemain>
%\fi
%
% The following presents a sample document
% with two chapters, two parts, a title page,
% a compile flag as well as three forwarding files to set the flag.
% It consists of eight |.tex| files:
% \begin{center}
% \begin{tabular}{ll}
% |cdocsamp.tex|&main file\\
% |cdocsch1.tex|&include file for chapter 1\\
% |cdocsch2.tex|&include file for chapter 2\\
% |cdocspt3.tex|&include file for part 3\\
% |cdocspt4.tex|&include file for part 4\\
% |cdocsdrf.tex|&forwarding file for main file in draft mode\\
% |cdocsfi1.tex|&forwarding file for final version of chapter 1\\
% |cdocsfi2.tex|&forwarding file for final version of chapter 2\\
% \end{tabular}
% \end{center}
% Each of the eight files can be compiled directly by the \LaTeX{} compiler.
%
% %%%%%%%%%%%%%%%%%%%%%%%%%%%%%%%%%%%%%%
% \paragraph{Main File.}
%
% The main file is called |cdocsamp.tex|.
%
% Load the \textsf{childdoc} definitions and
% declare the filename for the main document:
%    \begin{macrocode}
\input{childdoc.def}
\childdocmain{}
%    \end{macrocode}

% Optional override for |\version| flag:
%    \begin{macrocode}
%%\ifchilddoc\else\providecommand{\version}{draft}\fi
%    \end{macrocode}

% Define the default values for the |\version| flag
% (|final| for the main file and |draft| for childs):
%    \begin{macrocode}
\ifchilddoc
\providecommand{\version}{draft}
\else
\providecommand{\version}{final}
\fi
%    \end{macrocode}

% Load the standard document class:
%    \begin{macrocode}
\documentclass[12pt]{article}
%    \end{macrocode}

% Start the document body:
%    \begin{macrocode}
\begin{document}
%    \end{macrocode}

% Declare a title page.
% Print title, part of document being processed and version flag:
%    \begin{macrocode}
\addtocounter{page}{-1}
\begin{center}
{\LARGE\bfseries{}childdoc example\par}
\vspace{1cm}
\ifchilddoc
\ifchilddocmanual part\else chapter\fi:
`\childdocname' of `\childdocjob'\par
\else
main document: `\childdocjob'\par
\fi
version: \version\par
\end{center}
\newpage
%    \end{macrocode}

% Manually include selected file,
% otherwise process as usual:
%    \begin{macrocode}
\ifchilddocmanual
\section*{part `\childdocname'}
\input{\childdocname}
\else
%    \end{macrocode}

% Include the two chapters:
%    \begin{macrocode}
\include{cdocsch1}
\include{cdocsch2}
%    \end{macrocode}

% Include the two parts unless only chapters should be displayed:
%    \begin{macrocode}
\ifchilddoc\else
\section{part three}
\input{cdocspt3}
\section{part four}
\input{cdocspt4}
\fi
%    \end{macrocode}

% Process as usual until here:
%    \begin{macrocode}
\fi
%    \end{macrocode}

% End of document body:
%    \begin{macrocode}
\end{document}
%    \end{macrocode}
%\iffalse
%</samplemain>
%\fi
%
% %%%%%%%%%%%%%%%%%%%%%%%%%%%%%%%%%%%%%%
% \paragraph{Chapter Include Files.}
%
% The include files are called |cdocsch1.tex| and |cdocsch2.tex|.
%
%\iffalse
%<*samplechap1|samplechap2>
%\fi

% Optional override for |\version| flag:
%    \begin{macrocode}
%%\providecommand{\version}{final}
%    \end{macrocode}

% Include the main document:
%    \begin{macrocode}
\input{childdoc.def}
\childdocof{cdocsamp}
%    \end{macrocode}

%\iffalse
%</samplechap1|samplechap2>
%\fi
%
%\iffalse
%<*samplechap1>
%\fi
% Some text for chapter 1:
%    \begin{macrocode}
\section{one}
some text in chapter one
%    \end{macrocode}

%\iffalse
%</samplechap1>
%\fi
% Some text for chapter 2:
%\iffalse
%<*samplechap2>
%\fi
%    \begin{macrocode}
\section{two}
more text in chapter two
%    \end{macrocode}

%\iffalse
%</samplechap2>
%\fi
%
% %%%%%%%%%%%%%%%%%%%%%%%%%%%%%%%%%%%%%%
% \paragraph{Part Include Files.}
%
% The include files are called |cdocspt3.tex| and |cdocspt4.tex|.
%
%\iffalse
%<*samplepart3|samplepart4>
%\fi

% Optional override for |\version| flag:
%    \begin{macrocode}
%%\providecommand{\version}{final}
%    \end{macrocode}

% Include the main document:
%    \begin{macrocode}
\input{childdoc.def}
\childdocby{cdocsamp}
%    \end{macrocode}

%\iffalse
%</samplepart3|samplepart4>
%\fi
%
%\iffalse
%<*samplepart3>
%\fi
% Some text for part 3:
%    \begin{macrocode}
some text in part three
%    \end{macrocode}

%\iffalse
%</samplepart3>
%\fi
% Some text for part 4:
%\iffalse
%<*samplepart4>
%\fi
%    \begin{macrocode}
more text in part four
%    \end{macrocode}

%\iffalse
%</samplepart4>
%\fi
%
% %%%%%%%%%%%%%%%%%%%%%%%%%%%%%%%%%%%%%%
% \paragraph{Forwarding for a Complete Draft.}
%
% The following forwarding file |cdocsdrf.tex|
% compiles the main document in draft mode:
%\iffalse
%<*sampledraft>
%\fi
%    \begin{macrocode}
\def\version{draft}
\input{childdoc.def}
\childdocforward{cdocsamp}
%    \end{macrocode}

%\iffalse
%</sampledraft>
%\fi
%
% %%%%%%%%%%%%%%%%%%%%%%%%%%%%%%%%%%%%%%
% \paragraph{Forwarding for Final Version of the Chapters.}
%
% The following forwarding files |cdocsfn1.tex| and |cdocsfn2.tex|
% (with identical content)
% compile the final versions of the child documents
% |cdocsch1.tex| and |cdocsch2.tex|, respectively:
%\iffalse
%<*samplefinal>
%\fi
%    \begin{macrocode}
\def\version{final}
\input{childdoc.def}
\childdocforwardprefix[cdocsamp]{cdocsfn}{cdocsch}
%    \end{macrocode}

%\iffalse
%</samplefinal>
%\fi
%
% %%%%%%%%%%%%%%%%%%%%%%%%%%%%%%%%%%%%%%
% \paragraph{Command Line Processing.}
%
% The following three command lines generate the output files
% |cdocscld|, |cdocscl1| and |cdocscl2|
% which should be identical to
% |cdocsdrf|, |cdocsch1| and |cdocsfn2|, respectively:
% \begin{center}
% \begin{tabular}{l}
% |latex -jobname cdocscld \|\\
% |  "\def\version{draft}\input{childdoc.def}\childdocforward{cdocsamp}"|\\
% |latex -jobname cdocscl1 \|\\
% |  "\input{childdoc.def}\childdocforward[cdocsamp]{cdocsch1}"|\\
% |latex -jobname cdocscl2 \|\\
% |  "\def\version{final}\input{childdoc.def}\childdocforward{cdocsch2}"|
% \end{tabular}
% \end{center}
% Note that the trailing backslash on each first line
% merely continues the input to the second line
% (for convenient cut ant paste).
% Furthermore, the command |latex| can be replaced by any
% of its alternative versions such as |pdflatex|.
%
% %%%%%%%%%%%%%%%%%%%%%%%%%%%%%%%%%%%%%%%%%%%%%%%%%%%%%%%%%%%%%%%%%%%%%%%%%%%%%%
% %%%%%%%%%%%%%%%%%%%%%%%%%%%%%%%%%%%%%%%%%%%%%%%%%%%%%%%%%%%%%%%%%%%%%%%%%%%%%%
% \section{Implementation}
%\iffalse
%<*package>
%\fi
%
% This section describes the definitions file |childdoc.def|.

% The definitions cannot be loaded using |\usepackage| or |\RequirePackage|
% which has a mechanism to prevent loading a style file more than once.
% When loading the definitions by means of |\input|
% multiple instances have to be prevented manually:
%\iffalse
%This code needs to be before the `\ProvidesFile' directive
%which is defined at the beginning of this file.
%Therefore it is also placed there and commented out here.
%</package>
%<*discard>
%\fi
%    \begin{macrocode}
\ifdefined\childdocmain\endinput\fi
%    \end{macrocode}
%\iffalse
%</discard>
%<*package>
%\fi
%
% \macro{\ifchilddoc}
% \macro{\ifchilddocmanual}
% The conditional |\ifchilddoc| tells whether a
% child (true) or main (false) document is being compiled.
% The conditional |\ifchilddocmanual| tells whether
% the |\includeonly| mechanism is used (false) or
% the selection of child files must be performed manually (true).
% The definitions initialise to false:
%    \begin{macrocode}
\newif\ifchilddoc
\newif\ifchilddocmanual
%    \end{macrocode}

% \macro{\childdocname}
% \macro{\childdocjob}
% The macro |\childdocname| stores the name of the main document
% to be compiled. The macro |\childdocjob| stores the name of
% the document on which the \LaTeX{} compiler was originally invoked.
% The content of |\jobname| cannot be compared
% to filenames specified in the source due to different catcodes.
% The following code rescans |\jobname|, stores the result
% in |\childdocname| and saves a copy in |\childdocjob|:
%    \begin{macrocode}
\edef\childdocname{\scantokens\expandafter{\jobname\noexpand}}
\let\childdocjob\childdocname
%    \end{macrocode}

% \macro{\childdocdisable}
% The macro |\childdocdisable| prevents the main file
% from being processed more than once.
% At this stage, the main document command |\childdocmain|
% is assumed to be called once again where it should do nothing.
% Any subsequent call to it should prevent
% a secondary processing of the main document
% It overwrites the forwarding commands
% |\childdocof| and |\childdocforward|
% with empty macros to prevent further inclusions of the main document:
%    \begin{macrocode}
\newcommand{\childdocdisable}
{
  \renewcommand{\childdocmain}[1]{\renewcommand{\childdocmain}[1]{\endinput}}
  \renewcommand{\childdocof}[1]{}
  \renewcommand{\childdocby}[2][]{}
  \renewcommand{\childdocforward}[2][]{}
  \renewcommand{\childdocdisable}{}
}
%    \end{macrocode}

% \macro{\childdocmain}
% The macro |\childdocmain| is to be called at the top of the main file
% with nothing or the main filename (without extension) as argument.
% First, it breaks loops.
% If the argument is not empty and does not match |\childdocname|
% (which is set by the first inclusion of |childdoc.def|),
% |\ifchilddoc| is set to true, |\includeonly| is applied to the child file
% and |\jobname| is set to the main file
% (for proper handling of |.aux| files):
%    \begin{macrocode}
\newcommand{\childdocmain}[1]
{
  \childdocdisable\childdocmain{}
  \if?#1?\else
    \begingroup
      \def\childdoctmp{#1}
      \ifx\childdoctmp\childdocname
        \def\childdoctmp{}
      \else
        \def\childdoctmp
        {
          \childdoctrue
          \includeonly{\childdocname}
          \def\childdocjob{#1}
          \def\jobname{#1}
        }
      \fi
      \expandafter
    \endgroup
    \childdoctmp
  \fi
}
%    \end{macrocode}

% \macro{\childdocof}
% The command |\childdocof| redirects
% compilation to the main file |#1|.
%    \begin{macrocode}
\newcommand{\childdocof}[1]
{
  \childdocdisable
  \childdoctrue
  \includeonly{\childdocname}
  \def\jobname{#1}
  \def\childdocjob{#1}
  \input{#1}
}
%    \end{macrocode}

% \macro{\childdocby}
% The command |\childdocby| ....
%    \begin{macrocode}
\newcommand{\childdocby}[2][]
{
  \childdocdisable
  \childdoctrue
  \childdocmanualtrue
  \if?#1?\else
    \def\jobname{#2}
  \fi
  \def\childdocjob{#2}
  \input{#2}
  \endinput
}
%    \end{macrocode}

% \macro{\childdocforward}
% The command |\childdocforward| redirects
% compilation to the main file or
% (if the optional argument is given) a child file.
% Parameters are set as if the main file
% or a child file starting with |\childdocof| was compiled.
% Then compilation is handed over to the main file:
%    \begin{macrocode}
\newcommand{\childdocforward}[2][]
{
  \begingroup
    \if?#1?
      \def\childdoctmp
      {
        \def\childdocname{#2}
        \def\childdocjob{#2}
        \def\jobname{#2}
        \input{#2}
        \endinput
      }
    \else
      \def\childdoctmp
      {
        \childdocdisable
        \def\childdocname{#2}
        \childdoctrue
        \includeonly{#2}
        \def\childdocjob{#1}
        \def\jobname{#1}
        \input{#1}
        \endinput
      }
    \fi
    \expandafter
  \endgroup
  \childdoctmp
}
%    \end{macrocode}

% \macro{\childdocforwardprefix}
% The command |\childdocforwardprefix| redirects
% compilation to the main or a child file by means of a pattern.
% The prefix |#1| in the current filename is replaced by |#2|
% and the suffix of the current filename is kept
% (it is assumed that the filename does not contain the substring `|~~~|'
% which is used as a delimiter).
% Compilation is handed over to the new file by |\childdocforward|:
%    \begin{macrocode}
\newcommand{\childdocforwardprefix}[3][]
{
  \begingroup
    \def\childdocextract #2##1~~~{\def\childdoctmp{\childdocforward[#1]{#3##1}}}
    \expandafter\childdocextract\childdocname~~~
    \expandafter
  \endgroup
  \childdoctmp
}
%    \end{macrocode}

% \macro{\childdoc}
% The deprecated macro |\childdoc| is a legacy version of |\childdocmain|:
%    \begin{macrocode}
\newcommand{\childdoc}{\childdocmain}
%    \end{macrocode}

% \macro{\childdocredirect}
% The deprecated macro |\childdocredirect| is a legacy version
% of |\childdocforward| and |\childdocforwardprefix|:
%    \begin{macrocode}
\newcommand{\childdocredirect}[2][]
{
  \begingroup
    \if?#1?
      \def\childdoctmp{\childdocforward{#2}}
    \else
      \def\childdoctmp{\childdocforwardprefix{#1}{#2}}
    \fi
    \expandafter
  \endgroup
  \childdoctmp
}
%    \end{macrocode}

%\iffalse
%</package>
%\fi
%
\endinput
|\\
|\childdocmain{|\textit{main}|}|\\
\end{tabular}
\end{center}
%
If |\jobname| does not match the argument \textit{main} of |\childdocmain|,
it is assumed that |\jobname| points to the child file to be compiled.
When using |\childdocmain| with the main file specified as argument,
it suffices to start a child file
with just |\input{|\textit{main}|}|
without loading of the package and using |\childdocof|.
If instead all processing is done
with the appropriate \textsf{childdoc} directives,
the argument of \textit{main} of |\childdocmain| can be empty.

An alternative version of the command line processing described
in \secref{sec:commandline} using the detection mechanism reads:
%
\begin{center}
|... -jobname "|\textit{target}|" "|[\textit{flags}]%
[|\def\jobname{|\textit{dest}|}|]|\input{|\textit{main}|}"|
\end{center}

%%%%%%%%%%%%%%%%%%%%%%%%%%%%%%%%%%%%%%%%%%%%%%%%%%%%%%%%%%%%%%%%%%%%%%%%%%%%%%%%
\subsection{Manual Code}
\label{sec:manual}

In case one cannot be certain whether the definitions file |childdoc.def|
is installed on the target \TeX{} distribution
and one prefers not to ship it,
it is conceivable to paste a few relevant commands into the sources.

To that end, drop all statements |% \iffalse
%
% childdoc.dtx Copyright (C) 2017-2018 Niklas Beisert
%
% This work may be distributed and/or modified under the
% conditions of the LaTeX Project Public License, either version 1.3
% of this license or (at your option) any later version.
% The latest version of this license is in
%   http://www.latex-project.org/lppl.txt
% and version 1.3 or later is part of all distributions of LaTeX
% version 2005/12/01 or later.
%
% This work has the LPPL maintenance status `maintained'.
%
% The Current Maintainer of this work is Niklas Beisert.
%
% This work consists of the files childdoc.dtx and childdoc.ins
% and the derived files childdoc.def and cdocsamp.tex with
% cdocsch1.tex, cdocsch2.tex, cdocsdrf.tex, cdocsfn1.tex, cdocsfn2.tex.
%
%<package>\ifdefined\childdocmain\endinput\fi
%<package>\ProvidesFile{childdoc.def}[2018/12/30 v2.0 child document driver]
%<samplemain>\ProvidesFile{cdocsamp.tex}[2018/12/30 v2.0 sample for childdoc]
%<*driver>
%\ProvidesFile{childdoc.drv}[2018/12/30 v2.0 childdoc reference manual file]
\PassOptionsToClass{10pt,a4paper}{article}
\documentclass{ltxdoc}

\usepackage[margin=35mm]{geometry}
\usepackage{hyperref}
\usepackage{hyperxmp}
\usepackage[usenames]{color}

\hypersetup{colorlinks=true}
\hypersetup{pdfstartview=FitH}
\hypersetup{pdfpagemode=UseNone}
\hypersetup{pdfsource={}}
\hypersetup{pdflang={en-UK}}
\hypersetup{pdfcopyright={Copyright 2017-2018 Niklas Beisert.
  This work may be distributed and/or modified under the
  conditions of the LaTeX Project Public License, either version 1.3
  of this license or (at your option) any later version.}}
\hypersetup{pdflicenseurl={http://www.latex-project.org/lppl.txt}}
\hypersetup{pdfcontactaddress={ETH Zurich, ITP, HIT K,
  Wolfgang-Pauli-Strasse 27}}
\hypersetup{pdfcontactpostcode={8093}}
\hypersetup{pdfcontactcity={Zurich}}
\hypersetup{pdfcontactcountry={Switzerland}}
\hypersetup{pdfcontactemail={nbeisert@itp.phys.ethz.ch}}
\hypersetup{pdfcontacturl={http://people.phys.ethz.ch/\xmptilde nbeisert/}}

\newcommand{\secref}[1]{\hyperref[#1]{section \ref*{#1}}}

\parskip1ex
\parindent0pt
\let\olditemize\itemize
\def\itemize{\olditemize\parskip0pt}

\begin{document}

\title{The \textsf{childdoc} Package}
\hypersetup{pdftitle={The childdoc Package}}
\author{Niklas Beisert\\[2ex]
  Institut f\"ur Theoretische Physik\\
  Eidgen\"ossische Technische Hochschule Z\"urich\\
  Wolfgang-Pauli-Strasse 27, 8093 Z\"urich, Switzerland\\[1ex]
  \href{mailto:nbeisert@itp.phys.ethz.ch}
  {\texttt{nbeisert@itp.phys.ethz.ch}}}
\hypersetup{pdfauthor={Niklas Beisert}}
\hypersetup{pdfsubject={Manual for the LaTeX2e Package childdoc}}
\date{30 December 2018, \textsf{v2.0}}
\maketitle

\begin{abstract}\noindent
\textsf{childdoc} is a \LaTeXe{} package
that enables the direct compilation
of document sections included by |\include|
to individual files.
\end{abstract}

\begingroup
\parskip0ex
\tableofcontents
\endgroup

%%%%%%%%%%%%%%%%%%%%%%%%%%%%%%%%%%%%%%%%%%%%%%%%%%%%%%%%%%%%%%%%%%%%%%%%%%%%%%%%
%%%%%%%%%%%%%%%%%%%%%%%%%%%%%%%%%%%%%%%%%%%%%%%%%%%%%%%%%%%%%%%%%%%%%%%%%%%%%%%%
\section{Introduction}

\LaTeX{} provides a mechanism to structure a large document (such as a book)
into a main file and several child files (containing the chapters)
using the |\include| command.
This mechanism is beneficial for documents
which span hundreds of pages in order to
make the source file(s) more manageable.
Moreover, compilation can be restricted to
selected child files by means of the |\includeonly| command.
The latter feature can be used to reduce the compilation time while editing
(this was significantly more useful in the earlier days of \LaTeX{})
or to generate a smaller document which is easier to navigate.
Another application of |\includeonly| is to generate
documents consisting of selected parts of the complete document.

However, there are a few drawbacks of the plain |\include| mechanism:
\begin{itemize}
\item
The child files cannot be compiled on their own,
they can only be compiled via the main file.
A naive editing environment
(such as a text editor with an option
to have the current file processed by \LaTeX)
may require one to switch to the main file before compiling;
attempting to compile the child file produces errors.
\item
The main file must be modified (each time)
to adjust the |\includeonly| command
to the present needs. This easily leaves the main file in a messy state.
\item
The generated document will always carry the filename
of the main document. This is inconvenient if
several child files are to be compiled and
to be kept for distribution.
\end{itemize}

The present package provides a simple interface
to make child files individually compilable by \LaTeX{}.
Compiling a child file then has the same effect as compiling
the main file with an |\includeonly| command
to select the appropriate child.
Moreover the generated document will carry the name of the child
rather than the main file.
This resolves all three above issues.

This feature is meant to make the editing of books,
thesis documents and lecture notes somewhat more convenient.
However, the package can also be used efficiently for
composing a series of documents (such as exercise sheets)
which are typically distributed individually.
It then assists the author in generating the individual documents
(potentially in different versions)
as well as a document containing the collected series.
Another application is in developing style files
or other kinds of included material
where compilation of the style file could redirect
to a sample or test file.

%%%%%%%%%%%%%%%%%%%%%%%%%%%%%%%%%%%%%%%%%%%%%%%%%%%%%%%%%%%%%%%%%%%%%%%%%%%%%%%%
%%%%%%%%%%%%%%%%%%%%%%%%%%%%%%%%%%%%%%%%%%%%%%%%%%%%%%%%%%%%%%%%%%%%%%%%%%%%%%%%
\section{Usage}

First of all, the package \textsf{childdoc} is \emph{not} a standard
\LaTeXe{} |.sty| style file! Therefore it needs to be invoked in
a non-standard way.

%%%%%%%%%%%%%%%%%%%%%%%%%%%%%%%%%%%%%%%%%%%%%%%%%%%%%%%%%%%%%%%%%%%%%%%%%%%%%%%%
\subsection{Included Files}
\label{sec:include}

%%%%%%%%%%%%%%%%%%%%%%%%%%%%%%%%%%%%%%%%
\DescribeMacro{\childdocmain}
To use the package, add the commands
\begin{center}
\begin{tabular}{l}
|\input{childdoc.def}|\\
|\childdocmain{}|\\
\end{tabular}
\end{center}
at the very top of the main \LaTeX{} file,
in particular \emph{before} the |\documentclass| statement!
The argument of |\childdocmain| should be left empty
(but it must be present).

%%%%%%%%%%%%%%%%%%%%%%%%%%%%%%%%%%%%%%%%
\DescribeMacro{\childdocof}
Furthermore, add the commands
\begin{center}
\begin{tabular}{l}
|\input{childdoc.def}|\\
|\childdocof{|\textit{main}|}|\\
\end{tabular}
\end{center}
at the top of every child file \textit{child}
which is included by |\include{|\textit{child}|}|
from within the main file
(or at least for those files to be compiled individually).
The argument \textit{main} must be the filename of the main file.

There are a couple of
considerations in setting up the main and child documents:

%%%%%%%%%%%%%%%%%%%%%%%%%%%%%%%%%%%%%%%%
\paragraph{Restrictions.}

Please note the following restrictions:
\begin{itemize}
\item
|\childdocmain| must be called with one argument \textit{main}
to ensure compatibility with earlier version of the package.
It must either be empty (|\childdocmain{}|)
or precisely match the filename of the main file in which it is specified.
See \secref{sec:detection} for further information.
\item
The filename \textit{main} must be specified without the |.tex| extension.
\item
The filename \textit{main} is case sensitive
(even in case-insensitive file systems)
due to internal string comparison.
\item
The argument \textit{main} should be fully expanded, it cannot be a macro.
\item
Subdirectories and special characters should be avoided in filenames.
\item
The command |\childdocmain{|\textit{main}|}| must be followed by a whitespace.
It should not be followed immediately by another command
or by a comment mark `|%|'.
This is because the \TeX{} parser reads the token immediately following
the argument of |\childdocmain| and puts it
at the beginning of every child section;
however, a white\-space is ignored.
\end{itemize}

%%%%%%%%%%%%%%%%%%%%%%%%%%%%%%%%%%%%%%%%
\paragraph{Content of Main File.}

It is advisable to place all content in the child files included by |\include|.
Any output contained in the main file will appear in all child documents
unless suppressed manually;
it cannot be suppressed automatically by the |\includeonly| directive
and thus should normally be avoided.
A method to include some content in the main file
by means of conditional processing is described in \secref{sec:conditional}.

%%%%%%%%%%%%%%%%%%%%%%%%%%%%%%%%%%%%%%%%
\paragraph{Page Numbering.}

When only a part of the document is compiled,
the appropriate numbering of pages
(as well as other status parameters)
is determined from the |.aux| files.
The latter contain information from previous passes.
However this information needs to propagate through
all intermediate child documents.
Therefore the page numbering in child documents may well
be inconsistent until the complete document is compiled at least once.

A useful (if unconventional) way to always ensure a consistent
page numbering is to restart the numbering in each child document
and denote the pages by `\textit{child}|.|\textit{page}'
where \textit{child} represents the chapter/section number of the child file.
This can be achieved by the command
|\numberwithin{page}{|\textit{child}|}|
of the \textsf{amsmath} package
where \textit{child} can be |chapter| or |section|
depending on the chosen structuring.
Alternatively, one can modify the macro |\thepage| appropriately
and reset the counter |page| at the start of each child file.

%%%%%%%%%%%%%%%%%%%%%%%%%%%%%%%%%%%%%%%%%%%%%%%%%%%%%%%%%%%%%%%%%%%%%%%%%%%%%%%%
\subsection{Conditional Processing}
\label{sec:conditional}

The package provides a mechanism to compile different versions
of a document. To customise the versions further some conditional processing
can come in handy to distinguish which version is being compiled.
The package provides two macros to describe the compilation context:

%%%%%%%%%%%%%%%%%%%%%%%%%%%%%%%%%%%%%%%%
\DescribeMacro{\ifchilddoc}
The conditional |\ifchilddoc| distinguishes between the compilation of
child documents and the main document:
%
\begin{center}
|\ifchilddoc |\textit{child-code}| |[|\||else |\textit{main-code}]| \||fi|
\end{center}

%%%%%%%%%%%%%%%%%%%%%%%%%%%%%%%%%%%%%%%%
\DescribeMacro{\childdocname}
\DescribeMacro{\childdocjob}
The macro |\childdocname| contains the filename (without extension)
of the main or child file being processed.
Note that |\childdocjob| will always contain the name of the main file.

%%%%%%%%%%%%%%%%%%%%%%%%%%%%%%%%%%%%%%%%
\paragraph{Title Page.}

Conditional processing can be used to include a title or banner page
in the main document when proper precautions are taken.
Importantly, the code in the main file should ensure that the page counter
(as well as other status parameters which are stored in the |.aux| files)
takes the same value after the conditional processing.
Otherwise the page numbers may take divergent values
depending on which part is compiled.

For example, a title page could be declared by:
%
\begin{center}
\begin{tabular}{l}
|\ifchilddoc\||else|\\
|\addtocounter{page}{-1}|\\
\textit{code for title page}\\
|\newpage|\\
|\||fi|
\end{tabular}
\end{center}
%
A banner page for the child documents can be generated by:
%
\begin{center}
\begin{tabular}{l}
|\ifchilddoc|\\
|\addtocounter{page}{-1}|\\
\textit{code for banner page}\\
|\newpage|\\
|\||fi|
\end{tabular}
\end{center}
%
Here one could write a message such as:
\begin{center}
|This is the part \childdocname{} of \childdocjob{}.|
\end{center}

%%%%%%%%%%%%%%%%%%%%%%%%%%%%%%%%%%%%%%%%%%%%%%%%%%%%%%%%%%%%%%%%%%%%%%%%%%%%%%%%
\subsection{Flags}
\label{sec:flags}

The package makes it easy to generate different versions
of the main or child documents.
To this end compilation flags can be defined
and assigned different default values.
They will be particularly useful in conjunction
with the forwarding mechanism described in \secref{sec:forward}.

For example, it may be useful to have a flag |\version|
which can be set to |draft| or |final|.
The document source will contain some conditional code
depending on the value of |\version|.
Suppose further, the flag should default to |final| for the main file
and to |draft| for child files
which is a natural assignment for editing the document.
This is achieved by placing the following code
in the preamble of the main document
(below the |\childdocmain| directive):
%
\begin{center}
\begin{tabular}{l}
|\ifchilddoc|\\
|\providecommand{\version}{draft}|\\
|\||else|\\
|\providecommand{\version}{final}|\\
|\||fi|
\end{tabular}
\end{center}
%
The definition by |\providecommand| makes sure
that previous definitions are not overwritten.
Further statements |\providecommand{\version}{...}|
can thus be added before the above code to override it.

For the main file, one might add a line
(between |\childdocmain| and the above block)
%
\begin{center}
|%\ifchilddoc\||else\providecommand{\version}{draft}\||fi|
\end{center}
%
which can be uncommented to produce a draft version.
Likewise one can add a line to the very top of a child file
(above the |\childdocof{|\textit{main}|}| directive)
%
\begin{center}
|%\providecommand{\version}{final}|
\end{center}
%
which can be uncommented to produce the final version of this child document.

%%%%%%%%%%%%%%%%%%%%%%%%%%%%%%%%%%%%%%%%%%%%%%%%%%%%%%%%%%%%%%%%%%%%%%%%%%%%%%%%
\subsection{Forwarding}
\label{sec:forward}

Different versions of the main or child documents
using compilation flags as described in \secref{sec:flags}
can be (permanently) stored in different files
for convenient compilation, viewing and distribution.
To this end, the package defines a command
to pass on compilation to a different file:

%%%%%%%%%%%%%%%%%%%%%%%%%%%%%%%%%%%%%%%%
\DescribeMacro{\childdocforward}
The command |\childdocforward| redirects processing to
another source file:
%
\begin{center}
\begin{tabular}{l}
|\input{childdoc.def}|\\
|\childdocforward[|\textit{main}|]{|\textit{dest}|}|\\
\end{tabular}
\end{center}
%
The argument \textit{dest} is the destination file
(without extension).
It should be the main file or one of the child files.
Note that further \textsf{childdoc} directives
such as |\childdocof| and |\childdocforward|
in the indicated file will be processed in this form.
The optional argument \textit{main}
passes on directly to the main file \textit{main}
while pretending to compile the child \textit{dest}.
This form behaves as if \textit{dest}
issues |\childdocof{|\textit{main}|}| right away,
and no further \textsf{childdoc} directives will be processed.

%%%%%%%%%%%%%%%%%%%%%%%%%%%%%%%%%%%%%%%%
\DescribeMacro{\...prefix}
In the alternative form |\childdocforwardprefix|,
%
\begin{center}
\begin{tabular}{l}
|\input{childdoc.def}|\\
|\childdocforwardprefix[|\textit{main}|]{|\textit{prefix}|}{|\textit{dest}|}|
\end{tabular}
\end{center}
%
the destination file is determined by a pattern
depending on the current file:
To make this work, the current file must be called
`{\textit{prefix}\hspace{0.2em}\textit{suffix}}'
with \textit{prefix} matching precisely the argument.
Processing is then passed on to the file
`{\textit{dest}\hspace{0.2em}\textit{suffix}}'.
Surely, the same effect is achieved by
directly specifying the
argument `{\textit{dest}\hspace{0.2em}\textit{suffix}}'
in the first form.
However, that requires to set up a different file
for each child. With the alternative form of the command
all these files can have exactly the same content
which simplifies setting them up and maintaining them.

For example, the following file |draft.tex|
with a compilation flag |\version| as described in \secref{sec:flags}
compiles the main document as a draft:
%
\begin{center}
\begin{tabular}{l}
|\def\version{draft}|\\
|\input{childdoc.def}|\\
|\childdocforward{|\textit{main}|}|
\end{tabular}
\end{center}
%
Likewise, the following files |final|\textit{nn}|.tex|
compile the final version of the child document
|child|\textit{nn}|.tex|:
%
\begin{center}
\begin{tabular}{l}
|\def\version{final}|\\
|\input{childdoc.def}|\\
|\childdocforwardprefix{final}{child}|
\end{tabular}
\end{center}
%

Note that when several versions of a main file and/or of each child file
are to be generated, it may be convenient to set up a |Makefile| or
shell script to automatise the process.

%%%%%%%%%%%%%%%%%%%%%%%%%%%%%%%%%%%%%%%%%%%%%%%%%%%%%%%%%%%%%%%%%%%%%%%%%%%%%%%%
\subsection{Command Line Processing}
\label{sec:commandline}

The effect of redirection files can also be achieved by invoking
the \LaTeX{} compiler with a more elaborate command line.
Most conveniently this should be done as part
of a shell script or a |Makefile|.

When using \textsf{childdoc} in the main file, the following
command lines effectively perform a redirection
(note that depending on the shell being used,
backslashes may have to be doubled: `|\|' $\to$ `|\\|'):
%
\begin{center}
|... -jobname "|\textit{target}|" |\\|"|[\textit{flags}]%
|\input{childdoc.def}\childdocforward[|\textit{main}|]{|\textit{dest}|}"|
\end{center}
%
Here \textit{target} is the name of the output file,
\textit{main} is the name of the main file
and \textit{dest} is the name of the main or child file to be processed
(all filenames without extensions).
The optional argument \textit{main} can be omitted
if \textit{main} matches \textit{dest}.
Optionally, compilation \textit{flags} can be defined via |\def| commands.
This command line makes the \TeX{} engine believe
it is compiling the file \textit{target}
whose content is specified as the latter parameter.
The provided code then forwards the processing to
\textit{main} or \textit{dest} as described in \secref{sec:forward}.

%%%%%%%%%%%%%%%%%%%%%%%%%%%%%%%%%%%%%%%%%%%%%%%%%%%%%%%%%%%%%%%%%%%%%%%%%%%%%%%%
\subsection{Include by Input}
\label{sec:input}

Including child documents by |\include| has some restrictions by design.
Most notably, the content of a child document always occupies
its own set of pages; pages cannot be shared between child documents.
Usually, this behaviour makes perfect sense
because each child document contain an essential part of the document.
However, in some situations it may be desirable to compose
a document from a collection of parts
without having mandatory page breaks between then.
For this case, the package
provides a mechanism to include parts
by |\input| which can also be processed individually.
However, by construction this mechanism
requires manual handling of the content to be output.

%%%%%%%%%%%%%%%%%%%%%%%%%%%%%%%%%%%%%%%%
\DescribeMacro{\ifchilddocmanual}
The main file should be prepared as usual, see \secref{sec:include}.
However, the document body must make a distinction
between processing of an individual part and of the main document, e.g.:
%
\begin{center}
\begin{tabular}{l}
|\ifchilddocmanual|\\
|\input{\childdocname}|\\
|\||else|\\
\textit{document body with }|\input{|\textit{part}|}|\\
|\||fi|
\end{tabular}
\end{center}
%
The conditional |\ifchilddocmanual| is true whenever
a part to be included by |\input| is being compiled,
and the name of the part is stored in |\childdocname|.

%%%%%%%%%%%%%%%%%%%%%%%%%%%%%%%%%%%%%%%%
\DescribeMacro{\childdocby}
Each part to be included by |\input| should start with:
%
\begin{center}
\begin{tabular}{l}
|\input{childdoc.def}|\\
|\childdocby{|\textit{main}|}|\\
\end{tabular}
\end{center}
%
The directive |\childdocby| is similar to |\childdocof|
described in \secref{sec:include},
but the subsequent selection of content must be done manually.
To that end, both |\ifchilddoc| and |\ifchilddocmanual|
will be true upon processing of a part,
and the name of the part is stored in |\childdocname|.
Note that |\jobname| will be set to the filename of the current part
so that each part receives an individual |.aux| file
that does not interfere with the |.aux| file(s) of the main document.
This behaviour can be altered by the alternative form
|\childdocby[*]{|\textit{main}|}| (with a non-empty optional argument)
which uses the |.aux| file of the main document
by setting |\jobname| to \textit{main}.

%%%%%%%%%%%%%%%%%%%%%%%%%%%%%%%%%%%%%%%%%%%%%%%%%%%%%%%%%%%%%%%%%%%%%%%%%%%%%%%%
\subsection{Driver Development}
\label{sec:driver}

The \textsf{childdoc} mechanism can also be use for the development
of definition files such as \LaTeX{} styles or classes.
This case differs from the above setup with multiple parts
included by |\include| in that no |\includeonly| should be invoked.
This can be achieved by starting the include file
(before |\ProvidesPackage|) with:
%
\begin{center}
\begin{tabular}{l}
|\input{childdoc.def}|\\
|\childdocforward{|\textit{main}|}|\\
\end{tabular}
\end{center}
%
or alternatively with:
%
\begin{center}
\begin{tabular}{l}
|\input{childdoc.def}|\\
|\childdocby{|\textit{main}|}|\\
\end{tabular}
\end{center}
%
Both forms have slightly different effects as described above.
The main file is prepared as usual, see \secref{sec:include}.

%%%%%%%%%%%%%%%%%%%%%%%%%%%%%%%%%%%%%%%%%%%%%%%%%%%%%%%%%%%%%%%%%%%%%%%%%%%%%%%%
\subsection{Legacy Detection}
\label{sec:detection}

The directive |\childdocmain| in the main file can detect
whether the complete document or merely a child is to be compiled
even without using the directive |\childdocof|.
This method is deprecated because it is less robust
and there is no compelling reason to use it;
it is merely provided for backward compatibility
and it may be removed in future versions.

If the detection mechanism is to be used,
it is mandatory to correctly specify
the filename of the main file as the argument of |\childdocmain|:
%
\begin{center}
\begin{tabular}{l}
|\input{childdoc.def}|\\
|\childdocmain{|\textit{main}|}|\\
\end{tabular}
\end{center}
%
If |\jobname| does not match the argument \textit{main} of |\childdocmain|,
it is assumed that |\jobname| points to the child file to be compiled.
When using |\childdocmain| with the main file specified as argument,
it suffices to start a child file
with just |\input{|\textit{main}|}|
without loading of the package and using |\childdocof|.
If instead all processing is done
with the appropriate \textsf{childdoc} directives,
the argument of \textit{main} of |\childdocmain| can be empty.

An alternative version of the command line processing described
in \secref{sec:commandline} using the detection mechanism reads:
%
\begin{center}
|... -jobname "|\textit{target}|" "|[\textit{flags}]%
[|\def\jobname{|\textit{dest}|}|]|\input{|\textit{main}|}"|
\end{center}

%%%%%%%%%%%%%%%%%%%%%%%%%%%%%%%%%%%%%%%%%%%%%%%%%%%%%%%%%%%%%%%%%%%%%%%%%%%%%%%%
\subsection{Manual Code}
\label{sec:manual}

In case one cannot be certain whether the definitions file |childdoc.def|
is installed on the target \TeX{} distribution
and one prefers not to ship it,
it is conceivable to paste a few relevant commands into the sources.

To that end, drop all statements |\input{childdoc.def}|
and perform the replacements as outlined below.
Instead of |\childdocmain{|\textit{main}|}| add the following code
to the top of the main file:
%
\begin{center}
\begin{tabular}{l}
|\||ifdefined\childdocname\endinput\||fi\newif\ifchilddoc|\\
|\edef\childdocname{\scantokens\expandafter{\jobname\noexpand}}|\\
|\def\childdocmain{|\textit{main}|}\||ifx\childdocmain\childdocname\||else|\\
|\childdoctrue\includeonly{\childdocname}\let\jobname\childdocmain\||fi|\\
\end{tabular}
\end{center}
%
Instead of |\childdocof{|\textit{main}|}| just include the main file
at the top of each child file:
%
\begin{center}
|\input{|\textit{main}|}|
\end{center}
%
A simple redirection |\childdocforward{|\textit{dest}|}| is achieved by:
%
\begin{center}
|\def\jobname{|\textit{dest}|}\input{\jobname}|
\end{center}
%
The redirection with prefix
|\childdocforwardprefix[|\textit{prefix}|]{|\textit{dest}|}|
is accomplished by:
%
\begin{center}
\begin{tabular}{l}
|{\edef\jobname{\scantokens\expandafter{\jobname\noexpand}}|\\
|\def\redirectjob |\textit{prefix}|#1~~~{\gdef\jobname{|\textit{dest}|#1}}|\\
|\expandafter\redirectjob\jobname~~~}\input{\jobname}|
\end{tabular}
\end{center}

In an alternative approach,
child documents can be compiled by a specific command line
without additional code or specific definitions:
%
\begin{center}
|... -jobname "|\textit{target}|" "|[\textit{flags}]%
|\includeonly{|\textit{dest}|}\input{|\textit{main}|}"|
\end{center}
%

%%%%%%%%%%%%%%%%%%%%%%%%%%%%%%%%%%%%%%%%%%%%%%%%%%%%%%%%%%%%%%%%%%%%%%%%%%%%%%%%
%%%%%%%%%%%%%%%%%%%%%%%%%%%%%%%%%%%%%%%%%%%%%%%%%%%%%%%%%%%%%%%%%%%%%%%%%%%%%%%%
\section{Information}

%%%%%%%%%%%%%%%%%%%%%%%%%%%%%%%%%%%%%%%%%%%%%%%%%%%%%%%%%%%%%%%%%%%%%%%%%%%%%%%%
\subsection{Copyright}

Copyright \copyright{} 2017--2018 Niklas Beisert

This work may be distributed and/or modified under the
conditions of the \LaTeX{} Project Public License, either version 1.3
of this license or (at your option) any later version.
The latest version of this license is in
  \url{http://www.latex-project.org/lppl.txt}
and version 1.3 or later is part of all distributions of \LaTeX{}
version 2005/12/01 or later.

This work has the LPPL maintenance status `maintained'.

The Current Maintainer of this work is Niklas Beisert.

This work consists of the files |README.txt|, |childdoc.ins| and |childdoc.dtx|
as well as the derived files |childdoc.def|, |cdocsamp.tex|
with |cdocsch1.tex|, |cdocsch2.tex|, |cdocspt3.tex|, |cdocspt4.tex|,
|cdocsdrf.tex|, |cdocsfn1.tex|, |cdocsfn2.tex|
as well as |childdoc.pdf|.

%%%%%%%%%%%%%%%%%%%%%%%%%%%%%%%%%%%%%%%%%%%%%%%%%%%%%%%%%%%%%%%%%%%%%%%%%%%%%%%%
\subsection{Files and Installation}

The package consists of the files:
%
\begin{center}
\begin{tabular}{ll}
    |README.txt|   & readme file \\
    |childdoc.ins| & installation file \\
    |childdoc.dtx| & source file \\
    |childdoc.def| & definition file \\
    |cdocsamp.tex| & sample main file \\
    |cdocsch1.tex| & sample include file \\
    |cdocsch2.tex| & sample include file \\
    |cdocspt3.tex| & sample part file \\
    |cdocspt4.tex| & sample part file \\
    |cdocsdrf.tex| & sample redirection file \\
    |cdocsfn1.tex| & sample redirection file \\
    |cdocsfn2.tex| & sample redirection file \\
    |childdoc.pdf| & manual
\end{tabular}
\end{center}
%
The distribution consists of the files
|README.txt|, |childdoc.ins| and |childdoc.dtx|.
%
\begin{itemize}
\item
Run (pdf)\LaTeX{} on |childdoc.dtx|
to compile the manual |childdoc.pdf| (this file).
\item
Run \LaTeX{} on |childdoc.ins| to create the definitions file |childdoc.def|
and the sample |cdocsamp.tex| with include files
|cdocsch1.tex|, |cdocsch2.tex|, |cdocspt3.tex|, |cdocspt4.tex|,
|cdocsdrf.tex|, |cdocsfn1.tex|, |cdocsfn2.tex|.
Then copy the file |childdoc.def| to an appropriate directory of your \LaTeX{}
distribution, e.g.\ \textit{texmf-root}|/tex/latex/childdoc|.
\end{itemize}

%%%%%%%%%%%%%%%%%%%%%%%%%%%%%%%%%%%%%%%%%%%%%%%%%%%%%%%%%%%%%%%%%%%%%%%%%%%%%%%%
\subsection{Related CTAN Packages}

There are several other packages which offer a similar functionality:
%
\begin{itemize}
\item
The packages
\href{http://ctan.org/pkg/docmute}{\textsf{docmute}},
\href{http://ctan.org/pkg/includex}{\textsf{includex}} and
\href{http://ctan.org/pkg/standalone}{\textsf{standalone}}
provide commands to include only the document body of
a child file thus allowing both files to be compiled individually.
\item
The packages \href{http://ctan.org/pkg/subdocs}{\textsf{subdocs}}
and \href{http://ctan.org/pkg/subfiles}{\textsf{subfiles}}
provide structures in which the main and child documents can be
encapsulated and allowing them to be compiled individually.
The inclusion mechanism is different from the conventional |\include|.
\item
The package \href{http://ctan.org/pkg/combine}{\textsf{combine}}
is an elaborate solution to combine several documents into one.
\end{itemize}
%
See also the CTAN topic \href{http://ctan.org/topic/subdocs}{\textsf{subdocs}}
for further related packages.
The present package differs from the above solutions in that
a document structure constructed with the conventional |\include| mechanism
just needs two extra commands at the top of every file
such that all constituent files can be compiled individually.

%%%%%%%%%%%%%%%%%%%%%%%%%%%%%%%%%%%%%%%%%%%%%%%%%%%%%%%%%%%%%%%%%%%%%%%%%%%%%%%%
%\subsection{Feature Suggestions}
%
%The following is a list of features which may be useful for future
%versions of this package:
%%
%\begin{itemize}
%\item
%\ldots
%\end{itemize}

%%%%%%%%%%%%%%%%%%%%%%%%%%%%%%%%%%%%%%%%%%%%%%%%%%%%%%%%%%%%%%%%%%%%%%%%%%%%%%%%
\subsection{Revision History}

%%%%%%%%%%%%%%%%%%%%%%%%%%%%%%%%%%%%%%%%
\paragraph{v2.0:} 2018/12/30

\begin{itemize}
\item
immediate forward processing
\item
added |\childdocby| mechanism
\item
manual restructured
\end{itemize}

%%%%%%%%%%%%%%%%%%%%%%%%%%%%%%%%%%%%%%%%
\paragraph{v1.6:} 2018/01/17

\begin{itemize}
\item
application for development of include files
\item
corrections to manual
\end{itemize}

%%%%%%%%%%%%%%%%%%%%%%%%%%%%%%%%%%%%%%%%
\paragraph{v1.5:} 2017/05/21

\begin{itemize}
\item
more complete structuring introduced
\item
|\childdocof| introduced
\item
|\childdoc| renamed to |\childdocmain|
\item
|\childredirect| renamed to |\childdocforward| and |\childdocforwardprefix|
and functionality expanded
\end{itemize}

%%%%%%%%%%%%%%%%%%%%%%%%%%%%%%%%%%%%%%%%
\paragraph{v1.0:} 2017/04/27

\begin{itemize}
\item
manual and install package
\item
first version published on CTAN
\end{itemize}

%%%%%%%%%%%%%%%%%%%%%%%%%%%%%%%%%%%%%%%%
\paragraph{v0.6:} 2017/04/26

\begin{itemize}
\item
redirection mechanism added
\end{itemize}

%%%%%%%%%%%%%%%%%%%%%%%%%%%%%%%%%%%%%%%%
\paragraph{v0.5:} 2017/04/26

\begin{itemize}
\item
functionality in definition file
\end{itemize}


%%%%%%%%%%%%%%%%%%%%%%%%%%%%%%%%%%%%%%%%%%%%%%%%%%%%%%%%%%%%%%%%%%%%%%%%%%%%%%%%
%%%%%%%%%%%%%%%%%%%%%%%%%%%%%%%%%%%%%%%%%%%%%%%%%%%%%%%%%%%%%%%%%%%%%%%%%%%%%%%%
%%%%%%%%%%%%%%%%%%%%%%%%%%%%%%%%%%%%%%%%%%%%%%%%%%%%%%%%%%%%%%%%%%%%%%%%%%%%%%%%
\appendix

\settowidth\MacroIndent{\rmfamily\scriptsize 000\ }

 \DocInput{childdoc.dtx}

\end{document}
%</driver>
% \fi
%
% %%%%%%%%%%%%%%%%%%%%%%%%%%%%%%%%%%%%%%%%%%%%%%%%%%%%%%%%%%%%%%%%%%%%%%%%%%%%%%
% %%%%%%%%%%%%%%%%%%%%%%%%%%%%%%%%%%%%%%%%%%%%%%%%%%%%%%%%%%%%%%%%%%%%%%%%%%%%%%
% \section{Sample}
%\iffalse
%<*samplemain>
%\fi
%
% The following presents a sample document
% with two chapters, two parts, a title page,
% a compile flag as well as three forwarding files to set the flag.
% It consists of eight |.tex| files:
% \begin{center}
% \begin{tabular}{ll}
% |cdocsamp.tex|&main file\\
% |cdocsch1.tex|&include file for chapter 1\\
% |cdocsch2.tex|&include file for chapter 2\\
% |cdocspt3.tex|&include file for part 3\\
% |cdocspt4.tex|&include file for part 4\\
% |cdocsdrf.tex|&forwarding file for main file in draft mode\\
% |cdocsfi1.tex|&forwarding file for final version of chapter 1\\
% |cdocsfi2.tex|&forwarding file for final version of chapter 2\\
% \end{tabular}
% \end{center}
% Each of the eight files can be compiled directly by the \LaTeX{} compiler.
%
% %%%%%%%%%%%%%%%%%%%%%%%%%%%%%%%%%%%%%%
% \paragraph{Main File.}
%
% The main file is called |cdocsamp.tex|.
%
% Load the \textsf{childdoc} definitions and
% declare the filename for the main document:
%    \begin{macrocode}
\input{childdoc.def}
\childdocmain{}
%    \end{macrocode}

% Optional override for |\version| flag:
%    \begin{macrocode}
%%\ifchilddoc\else\providecommand{\version}{draft}\fi
%    \end{macrocode}

% Define the default values for the |\version| flag
% (|final| for the main file and |draft| for childs):
%    \begin{macrocode}
\ifchilddoc
\providecommand{\version}{draft}
\else
\providecommand{\version}{final}
\fi
%    \end{macrocode}

% Load the standard document class:
%    \begin{macrocode}
\documentclass[12pt]{article}
%    \end{macrocode}

% Start the document body:
%    \begin{macrocode}
\begin{document}
%    \end{macrocode}

% Declare a title page.
% Print title, part of document being processed and version flag:
%    \begin{macrocode}
\addtocounter{page}{-1}
\begin{center}
{\LARGE\bfseries{}childdoc example\par}
\vspace{1cm}
\ifchilddoc
\ifchilddocmanual part\else chapter\fi:
`\childdocname' of `\childdocjob'\par
\else
main document: `\childdocjob'\par
\fi
version: \version\par
\end{center}
\newpage
%    \end{macrocode}

% Manually include selected file,
% otherwise process as usual:
%    \begin{macrocode}
\ifchilddocmanual
\section*{part `\childdocname'}
\input{\childdocname}
\else
%    \end{macrocode}

% Include the two chapters:
%    \begin{macrocode}
\include{cdocsch1}
\include{cdocsch2}
%    \end{macrocode}

% Include the two parts unless only chapters should be displayed:
%    \begin{macrocode}
\ifchilddoc\else
\section{part three}
\input{cdocspt3}
\section{part four}
\input{cdocspt4}
\fi
%    \end{macrocode}

% Process as usual until here:
%    \begin{macrocode}
\fi
%    \end{macrocode}

% End of document body:
%    \begin{macrocode}
\end{document}
%    \end{macrocode}
%\iffalse
%</samplemain>
%\fi
%
% %%%%%%%%%%%%%%%%%%%%%%%%%%%%%%%%%%%%%%
% \paragraph{Chapter Include Files.}
%
% The include files are called |cdocsch1.tex| and |cdocsch2.tex|.
%
%\iffalse
%<*samplechap1|samplechap2>
%\fi

% Optional override for |\version| flag:
%    \begin{macrocode}
%%\providecommand{\version}{final}
%    \end{macrocode}

% Include the main document:
%    \begin{macrocode}
\input{childdoc.def}
\childdocof{cdocsamp}
%    \end{macrocode}

%\iffalse
%</samplechap1|samplechap2>
%\fi
%
%\iffalse
%<*samplechap1>
%\fi
% Some text for chapter 1:
%    \begin{macrocode}
\section{one}
some text in chapter one
%    \end{macrocode}

%\iffalse
%</samplechap1>
%\fi
% Some text for chapter 2:
%\iffalse
%<*samplechap2>
%\fi
%    \begin{macrocode}
\section{two}
more text in chapter two
%    \end{macrocode}

%\iffalse
%</samplechap2>
%\fi
%
% %%%%%%%%%%%%%%%%%%%%%%%%%%%%%%%%%%%%%%
% \paragraph{Part Include Files.}
%
% The include files are called |cdocspt3.tex| and |cdocspt4.tex|.
%
%\iffalse
%<*samplepart3|samplepart4>
%\fi

% Optional override for |\version| flag:
%    \begin{macrocode}
%%\providecommand{\version}{final}
%    \end{macrocode}

% Include the main document:
%    \begin{macrocode}
\input{childdoc.def}
\childdocby{cdocsamp}
%    \end{macrocode}

%\iffalse
%</samplepart3|samplepart4>
%\fi
%
%\iffalse
%<*samplepart3>
%\fi
% Some text for part 3:
%    \begin{macrocode}
some text in part three
%    \end{macrocode}

%\iffalse
%</samplepart3>
%\fi
% Some text for part 4:
%\iffalse
%<*samplepart4>
%\fi
%    \begin{macrocode}
more text in part four
%    \end{macrocode}

%\iffalse
%</samplepart4>
%\fi
%
% %%%%%%%%%%%%%%%%%%%%%%%%%%%%%%%%%%%%%%
% \paragraph{Forwarding for a Complete Draft.}
%
% The following forwarding file |cdocsdrf.tex|
% compiles the main document in draft mode:
%\iffalse
%<*sampledraft>
%\fi
%    \begin{macrocode}
\def\version{draft}
\input{childdoc.def}
\childdocforward{cdocsamp}
%    \end{macrocode}

%\iffalse
%</sampledraft>
%\fi
%
% %%%%%%%%%%%%%%%%%%%%%%%%%%%%%%%%%%%%%%
% \paragraph{Forwarding for Final Version of the Chapters.}
%
% The following forwarding files |cdocsfn1.tex| and |cdocsfn2.tex|
% (with identical content)
% compile the final versions of the child documents
% |cdocsch1.tex| and |cdocsch2.tex|, respectively:
%\iffalse
%<*samplefinal>
%\fi
%    \begin{macrocode}
\def\version{final}
\input{childdoc.def}
\childdocforwardprefix[cdocsamp]{cdocsfn}{cdocsch}
%    \end{macrocode}

%\iffalse
%</samplefinal>
%\fi
%
% %%%%%%%%%%%%%%%%%%%%%%%%%%%%%%%%%%%%%%
% \paragraph{Command Line Processing.}
%
% The following three command lines generate the output files
% |cdocscld|, |cdocscl1| and |cdocscl2|
% which should be identical to
% |cdocsdrf|, |cdocsch1| and |cdocsfn2|, respectively:
% \begin{center}
% \begin{tabular}{l}
% |latex -jobname cdocscld \|\\
% |  "\def\version{draft}\input{childdoc.def}\childdocforward{cdocsamp}"|\\
% |latex -jobname cdocscl1 \|\\
% |  "\input{childdoc.def}\childdocforward[cdocsamp]{cdocsch1}"|\\
% |latex -jobname cdocscl2 \|\\
% |  "\def\version{final}\input{childdoc.def}\childdocforward{cdocsch2}"|
% \end{tabular}
% \end{center}
% Note that the trailing backslash on each first line
% merely continues the input to the second line
% (for convenient cut ant paste).
% Furthermore, the command |latex| can be replaced by any
% of its alternative versions such as |pdflatex|.
%
% %%%%%%%%%%%%%%%%%%%%%%%%%%%%%%%%%%%%%%%%%%%%%%%%%%%%%%%%%%%%%%%%%%%%%%%%%%%%%%
% %%%%%%%%%%%%%%%%%%%%%%%%%%%%%%%%%%%%%%%%%%%%%%%%%%%%%%%%%%%%%%%%%%%%%%%%%%%%%%
% \section{Implementation}
%\iffalse
%<*package>
%\fi
%
% This section describes the definitions file |childdoc.def|.

% The definitions cannot be loaded using |\usepackage| or |\RequirePackage|
% which has a mechanism to prevent loading a style file more than once.
% When loading the definitions by means of |\input|
% multiple instances have to be prevented manually:
%\iffalse
%This code needs to be before the `\ProvidesFile' directive
%which is defined at the beginning of this file.
%Therefore it is also placed there and commented out here.
%</package>
%<*discard>
%\fi
%    \begin{macrocode}
\ifdefined\childdocmain\endinput\fi
%    \end{macrocode}
%\iffalse
%</discard>
%<*package>
%\fi
%
% \macro{\ifchilddoc}
% \macro{\ifchilddocmanual}
% The conditional |\ifchilddoc| tells whether a
% child (true) or main (false) document is being compiled.
% The conditional |\ifchilddocmanual| tells whether
% the |\includeonly| mechanism is used (false) or
% the selection of child files must be performed manually (true).
% The definitions initialise to false:
%    \begin{macrocode}
\newif\ifchilddoc
\newif\ifchilddocmanual
%    \end{macrocode}

% \macro{\childdocname}
% \macro{\childdocjob}
% The macro |\childdocname| stores the name of the main document
% to be compiled. The macro |\childdocjob| stores the name of
% the document on which the \LaTeX{} compiler was originally invoked.
% The content of |\jobname| cannot be compared
% to filenames specified in the source due to different catcodes.
% The following code rescans |\jobname|, stores the result
% in |\childdocname| and saves a copy in |\childdocjob|:
%    \begin{macrocode}
\edef\childdocname{\scantokens\expandafter{\jobname\noexpand}}
\let\childdocjob\childdocname
%    \end{macrocode}

% \macro{\childdocdisable}
% The macro |\childdocdisable| prevents the main file
% from being processed more than once.
% At this stage, the main document command |\childdocmain|
% is assumed to be called once again where it should do nothing.
% Any subsequent call to it should prevent
% a secondary processing of the main document
% It overwrites the forwarding commands
% |\childdocof| and |\childdocforward|
% with empty macros to prevent further inclusions of the main document:
%    \begin{macrocode}
\newcommand{\childdocdisable}
{
  \renewcommand{\childdocmain}[1]{\renewcommand{\childdocmain}[1]{\endinput}}
  \renewcommand{\childdocof}[1]{}
  \renewcommand{\childdocby}[2][]{}
  \renewcommand{\childdocforward}[2][]{}
  \renewcommand{\childdocdisable}{}
}
%    \end{macrocode}

% \macro{\childdocmain}
% The macro |\childdocmain| is to be called at the top of the main file
% with nothing or the main filename (without extension) as argument.
% First, it breaks loops.
% If the argument is not empty and does not match |\childdocname|
% (which is set by the first inclusion of |childdoc.def|),
% |\ifchilddoc| is set to true, |\includeonly| is applied to the child file
% and |\jobname| is set to the main file
% (for proper handling of |.aux| files):
%    \begin{macrocode}
\newcommand{\childdocmain}[1]
{
  \childdocdisable\childdocmain{}
  \if?#1?\else
    \begingroup
      \def\childdoctmp{#1}
      \ifx\childdoctmp\childdocname
        \def\childdoctmp{}
      \else
        \def\childdoctmp
        {
          \childdoctrue
          \includeonly{\childdocname}
          \def\childdocjob{#1}
          \def\jobname{#1}
        }
      \fi
      \expandafter
    \endgroup
    \childdoctmp
  \fi
}
%    \end{macrocode}

% \macro{\childdocof}
% The command |\childdocof| redirects
% compilation to the main file |#1|.
%    \begin{macrocode}
\newcommand{\childdocof}[1]
{
  \childdocdisable
  \childdoctrue
  \includeonly{\childdocname}
  \def\jobname{#1}
  \def\childdocjob{#1}
  \input{#1}
}
%    \end{macrocode}

% \macro{\childdocby}
% The command |\childdocby| ....
%    \begin{macrocode}
\newcommand{\childdocby}[2][]
{
  \childdocdisable
  \childdoctrue
  \childdocmanualtrue
  \if?#1?\else
    \def\jobname{#2}
  \fi
  \def\childdocjob{#2}
  \input{#2}
  \endinput
}
%    \end{macrocode}

% \macro{\childdocforward}
% The command |\childdocforward| redirects
% compilation to the main file or
% (if the optional argument is given) a child file.
% Parameters are set as if the main file
% or a child file starting with |\childdocof| was compiled.
% Then compilation is handed over to the main file:
%    \begin{macrocode}
\newcommand{\childdocforward}[2][]
{
  \begingroup
    \if?#1?
      \def\childdoctmp
      {
        \def\childdocname{#2}
        \def\childdocjob{#2}
        \def\jobname{#2}
        \input{#2}
        \endinput
      }
    \else
      \def\childdoctmp
      {
        \childdocdisable
        \def\childdocname{#2}
        \childdoctrue
        \includeonly{#2}
        \def\childdocjob{#1}
        \def\jobname{#1}
        \input{#1}
        \endinput
      }
    \fi
    \expandafter
  \endgroup
  \childdoctmp
}
%    \end{macrocode}

% \macro{\childdocforwardprefix}
% The command |\childdocforwardprefix| redirects
% compilation to the main or a child file by means of a pattern.
% The prefix |#1| in the current filename is replaced by |#2|
% and the suffix of the current filename is kept
% (it is assumed that the filename does not contain the substring `|~~~|'
% which is used as a delimiter).
% Compilation is handed over to the new file by |\childdocforward|:
%    \begin{macrocode}
\newcommand{\childdocforwardprefix}[3][]
{
  \begingroup
    \def\childdocextract #2##1~~~{\def\childdoctmp{\childdocforward[#1]{#3##1}}}
    \expandafter\childdocextract\childdocname~~~
    \expandafter
  \endgroup
  \childdoctmp
}
%    \end{macrocode}

% \macro{\childdoc}
% The deprecated macro |\childdoc| is a legacy version of |\childdocmain|:
%    \begin{macrocode}
\newcommand{\childdoc}{\childdocmain}
%    \end{macrocode}

% \macro{\childdocredirect}
% The deprecated macro |\childdocredirect| is a legacy version
% of |\childdocforward| and |\childdocforwardprefix|:
%    \begin{macrocode}
\newcommand{\childdocredirect}[2][]
{
  \begingroup
    \if?#1?
      \def\childdoctmp{\childdocforward{#2}}
    \else
      \def\childdoctmp{\childdocforwardprefix{#1}{#2}}
    \fi
    \expandafter
  \endgroup
  \childdoctmp
}
%    \end{macrocode}

%\iffalse
%</package>
%\fi
%
\endinput
|
and perform the replacements as outlined below.
Instead of |\childdocmain{|\textit{main}|}| add the following code
to the top of the main file:
%
\begin{center}
\begin{tabular}{l}
|\||ifdefined\childdocname\endinput\||fi\newif\ifchilddoc|\\
|\edef\childdocname{\scantokens\expandafter{\jobname\noexpand}}|\\
|\def\childdocmain{|\textit{main}|}\||ifx\childdocmain\childdocname\||else|\\
|\childdoctrue\includeonly{\childdocname}\let\jobname\childdocmain\||fi|\\
\end{tabular}
\end{center}
%
Instead of |\childdocof{|\textit{main}|}| just include the main file
at the top of each child file:
%
\begin{center}
|\input{|\textit{main}|}|
\end{center}
%
A simple redirection |\childdocforward{|\textit{dest}|}| is achieved by:
%
\begin{center}
|\def\jobname{|\textit{dest}|}\input{\jobname}|
\end{center}
%
The redirection with prefix
|\childdocforwardprefix[|\textit{prefix}|]{|\textit{dest}|}|
is accomplished by:
%
\begin{center}
\begin{tabular}{l}
|{\edef\jobname{\scantokens\expandafter{\jobname\noexpand}}|\\
|\def\redirectjob |\textit{prefix}|#1~~~{\gdef\jobname{|\textit{dest}|#1}}|\\
|\expandafter\redirectjob\jobname~~~}\input{\jobname}|
\end{tabular}
\end{center}

In an alternative approach,
child documents can be compiled by a specific command line
without additional code or specific definitions:
%
\begin{center}
|... -jobname "|\textit{target}|" "|[\textit{flags}]%
|\includeonly{|\textit{dest}|}\input{|\textit{main}|}"|
\end{center}
%

%%%%%%%%%%%%%%%%%%%%%%%%%%%%%%%%%%%%%%%%%%%%%%%%%%%%%%%%%%%%%%%%%%%%%%%%%%%%%%%%
%%%%%%%%%%%%%%%%%%%%%%%%%%%%%%%%%%%%%%%%%%%%%%%%%%%%%%%%%%%%%%%%%%%%%%%%%%%%%%%%
\section{Information}

%%%%%%%%%%%%%%%%%%%%%%%%%%%%%%%%%%%%%%%%%%%%%%%%%%%%%%%%%%%%%%%%%%%%%%%%%%%%%%%%
\subsection{Copyright}

Copyright \copyright{} 2017--2018 Niklas Beisert

This work may be distributed and/or modified under the
conditions of the \LaTeX{} Project Public License, either version 1.3
of this license or (at your option) any later version.
The latest version of this license is in
  \url{http://www.latex-project.org/lppl.txt}
and version 1.3 or later is part of all distributions of \LaTeX{}
version 2005/12/01 or later.

This work has the LPPL maintenance status `maintained'.

The Current Maintainer of this work is Niklas Beisert.

This work consists of the files |README.txt|, |childdoc.ins| and |childdoc.dtx|
as well as the derived files |childdoc.def|, |cdocsamp.tex|
with |cdocsch1.tex|, |cdocsch2.tex|, |cdocspt3.tex|, |cdocspt4.tex|,
|cdocsdrf.tex|, |cdocsfn1.tex|, |cdocsfn2.tex|
as well as |childdoc.pdf|.

%%%%%%%%%%%%%%%%%%%%%%%%%%%%%%%%%%%%%%%%%%%%%%%%%%%%%%%%%%%%%%%%%%%%%%%%%%%%%%%%
\subsection{Files and Installation}

The package consists of the files:
%
\begin{center}
\begin{tabular}{ll}
    |README.txt|   & readme file \\
    |childdoc.ins| & installation file \\
    |childdoc.dtx| & source file \\
    |childdoc.def| & definition file \\
    |cdocsamp.tex| & sample main file \\
    |cdocsch1.tex| & sample include file \\
    |cdocsch2.tex| & sample include file \\
    |cdocspt3.tex| & sample part file \\
    |cdocspt4.tex| & sample part file \\
    |cdocsdrf.tex| & sample redirection file \\
    |cdocsfn1.tex| & sample redirection file \\
    |cdocsfn2.tex| & sample redirection file \\
    |childdoc.pdf| & manual
\end{tabular}
\end{center}
%
The distribution consists of the files
|README.txt|, |childdoc.ins| and |childdoc.dtx|.
%
\begin{itemize}
\item
Run (pdf)\LaTeX{} on |childdoc.dtx|
to compile the manual |childdoc.pdf| (this file).
\item
Run \LaTeX{} on |childdoc.ins| to create the definitions file |childdoc.def|
and the sample |cdocsamp.tex| with include files
|cdocsch1.tex|, |cdocsch2.tex|, |cdocspt3.tex|, |cdocspt4.tex|,
|cdocsdrf.tex|, |cdocsfn1.tex|, |cdocsfn2.tex|.
Then copy the file |childdoc.def| to an appropriate directory of your \LaTeX{}
distribution, e.g.\ \textit{texmf-root}|/tex/latex/childdoc|.
\end{itemize}

%%%%%%%%%%%%%%%%%%%%%%%%%%%%%%%%%%%%%%%%%%%%%%%%%%%%%%%%%%%%%%%%%%%%%%%%%%%%%%%%
\subsection{Related CTAN Packages}

There are several other packages which offer a similar functionality:
%
\begin{itemize}
\item
The packages
\href{http://ctan.org/pkg/docmute}{\textsf{docmute}},
\href{http://ctan.org/pkg/includex}{\textsf{includex}} and
\href{http://ctan.org/pkg/standalone}{\textsf{standalone}}
provide commands to include only the document body of
a child file thus allowing both files to be compiled individually.
\item
The packages \href{http://ctan.org/pkg/subdocs}{\textsf{subdocs}}
and \href{http://ctan.org/pkg/subfiles}{\textsf{subfiles}}
provide structures in which the main and child documents can be
encapsulated and allowing them to be compiled individually.
The inclusion mechanism is different from the conventional |\include|.
\item
The package \href{http://ctan.org/pkg/combine}{\textsf{combine}}
is an elaborate solution to combine several documents into one.
\end{itemize}
%
See also the CTAN topic \href{http://ctan.org/topic/subdocs}{\textsf{subdocs}}
for further related packages.
The present package differs from the above solutions in that
a document structure constructed with the conventional |\include| mechanism
just needs two extra commands at the top of every file
such that all constituent files can be compiled individually.

%%%%%%%%%%%%%%%%%%%%%%%%%%%%%%%%%%%%%%%%%%%%%%%%%%%%%%%%%%%%%%%%%%%%%%%%%%%%%%%%
%\subsection{Feature Suggestions}
%
%The following is a list of features which may be useful for future
%versions of this package:
%%
%\begin{itemize}
%\item
%\ldots
%\end{itemize}

%%%%%%%%%%%%%%%%%%%%%%%%%%%%%%%%%%%%%%%%%%%%%%%%%%%%%%%%%%%%%%%%%%%%%%%%%%%%%%%%
\subsection{Revision History}

%%%%%%%%%%%%%%%%%%%%%%%%%%%%%%%%%%%%%%%%
\paragraph{v2.0:} 2018/12/30

\begin{itemize}
\item
immediate forward processing
\item
added |\childdocby| mechanism
\item
manual restructured
\end{itemize}

%%%%%%%%%%%%%%%%%%%%%%%%%%%%%%%%%%%%%%%%
\paragraph{v1.6:} 2018/01/17

\begin{itemize}
\item
application for development of include files
\item
corrections to manual
\end{itemize}

%%%%%%%%%%%%%%%%%%%%%%%%%%%%%%%%%%%%%%%%
\paragraph{v1.5:} 2017/05/21

\begin{itemize}
\item
more complete structuring introduced
\item
|\childdocof| introduced
\item
|\childdoc| renamed to |\childdocmain|
\item
|\childredirect| renamed to |\childdocforward| and |\childdocforwardprefix|
and functionality expanded
\end{itemize}

%%%%%%%%%%%%%%%%%%%%%%%%%%%%%%%%%%%%%%%%
\paragraph{v1.0:} 2017/04/27

\begin{itemize}
\item
manual and install package
\item
first version published on CTAN
\end{itemize}

%%%%%%%%%%%%%%%%%%%%%%%%%%%%%%%%%%%%%%%%
\paragraph{v0.6:} 2017/04/26

\begin{itemize}
\item
redirection mechanism added
\end{itemize}

%%%%%%%%%%%%%%%%%%%%%%%%%%%%%%%%%%%%%%%%
\paragraph{v0.5:} 2017/04/26

\begin{itemize}
\item
functionality in definition file
\end{itemize}


%%%%%%%%%%%%%%%%%%%%%%%%%%%%%%%%%%%%%%%%%%%%%%%%%%%%%%%%%%%%%%%%%%%%%%%%%%%%%%%%
%%%%%%%%%%%%%%%%%%%%%%%%%%%%%%%%%%%%%%%%%%%%%%%%%%%%%%%%%%%%%%%%%%%%%%%%%%%%%%%%
%%%%%%%%%%%%%%%%%%%%%%%%%%%%%%%%%%%%%%%%%%%%%%%%%%%%%%%%%%%%%%%%%%%%%%%%%%%%%%%%
\appendix

\settowidth\MacroIndent{\rmfamily\scriptsize 000\ }

 \DocInput{childdoc.dtx}

\end{document}
%</driver>
% \fi
%
% %%%%%%%%%%%%%%%%%%%%%%%%%%%%%%%%%%%%%%%%%%%%%%%%%%%%%%%%%%%%%%%%%%%%%%%%%%%%%%
% %%%%%%%%%%%%%%%%%%%%%%%%%%%%%%%%%%%%%%%%%%%%%%%%%%%%%%%%%%%%%%%%%%%%%%%%%%%%%%
% \section{Sample}
%\iffalse
%<*samplemain>
%\fi
%
% The following presents a sample document
% with two chapters, two parts, a title page,
% a compile flag as well as three forwarding files to set the flag.
% It consists of eight |.tex| files:
% \begin{center}
% \begin{tabular}{ll}
% |cdocsamp.tex|&main file\\
% |cdocsch1.tex|&include file for chapter 1\\
% |cdocsch2.tex|&include file for chapter 2\\
% |cdocspt3.tex|&include file for part 3\\
% |cdocspt4.tex|&include file for part 4\\
% |cdocsdrf.tex|&forwarding file for main file in draft mode\\
% |cdocsfi1.tex|&forwarding file for final version of chapter 1\\
% |cdocsfi2.tex|&forwarding file for final version of chapter 2\\
% \end{tabular}
% \end{center}
% Each of the eight files can be compiled directly by the \LaTeX{} compiler.
%
% %%%%%%%%%%%%%%%%%%%%%%%%%%%%%%%%%%%%%%
% \paragraph{Main File.}
%
% The main file is called |cdocsamp.tex|.
%
% Load the \textsf{childdoc} definitions and
% declare the filename for the main document:
%    \begin{macrocode}
% \iffalse
%
% childdoc.dtx Copyright (C) 2017-2018 Niklas Beisert
%
% This work may be distributed and/or modified under the
% conditions of the LaTeX Project Public License, either version 1.3
% of this license or (at your option) any later version.
% The latest version of this license is in
%   http://www.latex-project.org/lppl.txt
% and version 1.3 or later is part of all distributions of LaTeX
% version 2005/12/01 or later.
%
% This work has the LPPL maintenance status `maintained'.
%
% The Current Maintainer of this work is Niklas Beisert.
%
% This work consists of the files childdoc.dtx and childdoc.ins
% and the derived files childdoc.def and cdocsamp.tex with
% cdocsch1.tex, cdocsch2.tex, cdocsdrf.tex, cdocsfn1.tex, cdocsfn2.tex.
%
%<package>\ifdefined\childdocmain\endinput\fi
%<package>\ProvidesFile{childdoc.def}[2018/12/30 v2.0 child document driver]
%<samplemain>\ProvidesFile{cdocsamp.tex}[2018/12/30 v2.0 sample for childdoc]
%<*driver>
%\ProvidesFile{childdoc.drv}[2018/12/30 v2.0 childdoc reference manual file]
\PassOptionsToClass{10pt,a4paper}{article}
\documentclass{ltxdoc}

\usepackage[margin=35mm]{geometry}
\usepackage{hyperref}
\usepackage{hyperxmp}
\usepackage[usenames]{color}

\hypersetup{colorlinks=true}
\hypersetup{pdfstartview=FitH}
\hypersetup{pdfpagemode=UseNone}
\hypersetup{pdfsource={}}
\hypersetup{pdflang={en-UK}}
\hypersetup{pdfcopyright={Copyright 2017-2018 Niklas Beisert.
  This work may be distributed and/or modified under the
  conditions of the LaTeX Project Public License, either version 1.3
  of this license or (at your option) any later version.}}
\hypersetup{pdflicenseurl={http://www.latex-project.org/lppl.txt}}
\hypersetup{pdfcontactaddress={ETH Zurich, ITP, HIT K,
  Wolfgang-Pauli-Strasse 27}}
\hypersetup{pdfcontactpostcode={8093}}
\hypersetup{pdfcontactcity={Zurich}}
\hypersetup{pdfcontactcountry={Switzerland}}
\hypersetup{pdfcontactemail={nbeisert@itp.phys.ethz.ch}}
\hypersetup{pdfcontacturl={http://people.phys.ethz.ch/\xmptilde nbeisert/}}

\newcommand{\secref}[1]{\hyperref[#1]{section \ref*{#1}}}

\parskip1ex
\parindent0pt
\let\olditemize\itemize
\def\itemize{\olditemize\parskip0pt}

\begin{document}

\title{The \textsf{childdoc} Package}
\hypersetup{pdftitle={The childdoc Package}}
\author{Niklas Beisert\\[2ex]
  Institut f\"ur Theoretische Physik\\
  Eidgen\"ossische Technische Hochschule Z\"urich\\
  Wolfgang-Pauli-Strasse 27, 8093 Z\"urich, Switzerland\\[1ex]
  \href{mailto:nbeisert@itp.phys.ethz.ch}
  {\texttt{nbeisert@itp.phys.ethz.ch}}}
\hypersetup{pdfauthor={Niklas Beisert}}
\hypersetup{pdfsubject={Manual for the LaTeX2e Package childdoc}}
\date{30 December 2018, \textsf{v2.0}}
\maketitle

\begin{abstract}\noindent
\textsf{childdoc} is a \LaTeXe{} package
that enables the direct compilation
of document sections included by |\include|
to individual files.
\end{abstract}

\begingroup
\parskip0ex
\tableofcontents
\endgroup

%%%%%%%%%%%%%%%%%%%%%%%%%%%%%%%%%%%%%%%%%%%%%%%%%%%%%%%%%%%%%%%%%%%%%%%%%%%%%%%%
%%%%%%%%%%%%%%%%%%%%%%%%%%%%%%%%%%%%%%%%%%%%%%%%%%%%%%%%%%%%%%%%%%%%%%%%%%%%%%%%
\section{Introduction}

\LaTeX{} provides a mechanism to structure a large document (such as a book)
into a main file and several child files (containing the chapters)
using the |\include| command.
This mechanism is beneficial for documents
which span hundreds of pages in order to
make the source file(s) more manageable.
Moreover, compilation can be restricted to
selected child files by means of the |\includeonly| command.
The latter feature can be used to reduce the compilation time while editing
(this was significantly more useful in the earlier days of \LaTeX{})
or to generate a smaller document which is easier to navigate.
Another application of |\includeonly| is to generate
documents consisting of selected parts of the complete document.

However, there are a few drawbacks of the plain |\include| mechanism:
\begin{itemize}
\item
The child files cannot be compiled on their own,
they can only be compiled via the main file.
A naive editing environment
(such as a text editor with an option
to have the current file processed by \LaTeX)
may require one to switch to the main file before compiling;
attempting to compile the child file produces errors.
\item
The main file must be modified (each time)
to adjust the |\includeonly| command
to the present needs. This easily leaves the main file in a messy state.
\item
The generated document will always carry the filename
of the main document. This is inconvenient if
several child files are to be compiled and
to be kept for distribution.
\end{itemize}

The present package provides a simple interface
to make child files individually compilable by \LaTeX{}.
Compiling a child file then has the same effect as compiling
the main file with an |\includeonly| command
to select the appropriate child.
Moreover the generated document will carry the name of the child
rather than the main file.
This resolves all three above issues.

This feature is meant to make the editing of books,
thesis documents and lecture notes somewhat more convenient.
However, the package can also be used efficiently for
composing a series of documents (such as exercise sheets)
which are typically distributed individually.
It then assists the author in generating the individual documents
(potentially in different versions)
as well as a document containing the collected series.
Another application is in developing style files
or other kinds of included material
where compilation of the style file could redirect
to a sample or test file.

%%%%%%%%%%%%%%%%%%%%%%%%%%%%%%%%%%%%%%%%%%%%%%%%%%%%%%%%%%%%%%%%%%%%%%%%%%%%%%%%
%%%%%%%%%%%%%%%%%%%%%%%%%%%%%%%%%%%%%%%%%%%%%%%%%%%%%%%%%%%%%%%%%%%%%%%%%%%%%%%%
\section{Usage}

First of all, the package \textsf{childdoc} is \emph{not} a standard
\LaTeXe{} |.sty| style file! Therefore it needs to be invoked in
a non-standard way.

%%%%%%%%%%%%%%%%%%%%%%%%%%%%%%%%%%%%%%%%%%%%%%%%%%%%%%%%%%%%%%%%%%%%%%%%%%%%%%%%
\subsection{Included Files}
\label{sec:include}

%%%%%%%%%%%%%%%%%%%%%%%%%%%%%%%%%%%%%%%%
\DescribeMacro{\childdocmain}
To use the package, add the commands
\begin{center}
\begin{tabular}{l}
|\input{childdoc.def}|\\
|\childdocmain{}|\\
\end{tabular}
\end{center}
at the very top of the main \LaTeX{} file,
in particular \emph{before} the |\documentclass| statement!
The argument of |\childdocmain| should be left empty
(but it must be present).

%%%%%%%%%%%%%%%%%%%%%%%%%%%%%%%%%%%%%%%%
\DescribeMacro{\childdocof}
Furthermore, add the commands
\begin{center}
\begin{tabular}{l}
|\input{childdoc.def}|\\
|\childdocof{|\textit{main}|}|\\
\end{tabular}
\end{center}
at the top of every child file \textit{child}
which is included by |\include{|\textit{child}|}|
from within the main file
(or at least for those files to be compiled individually).
The argument \textit{main} must be the filename of the main file.

There are a couple of
considerations in setting up the main and child documents:

%%%%%%%%%%%%%%%%%%%%%%%%%%%%%%%%%%%%%%%%
\paragraph{Restrictions.}

Please note the following restrictions:
\begin{itemize}
\item
|\childdocmain| must be called with one argument \textit{main}
to ensure compatibility with earlier version of the package.
It must either be empty (|\childdocmain{}|)
or precisely match the filename of the main file in which it is specified.
See \secref{sec:detection} for further information.
\item
The filename \textit{main} must be specified without the |.tex| extension.
\item
The filename \textit{main} is case sensitive
(even in case-insensitive file systems)
due to internal string comparison.
\item
The argument \textit{main} should be fully expanded, it cannot be a macro.
\item
Subdirectories and special characters should be avoided in filenames.
\item
The command |\childdocmain{|\textit{main}|}| must be followed by a whitespace.
It should not be followed immediately by another command
or by a comment mark `|%|'.
This is because the \TeX{} parser reads the token immediately following
the argument of |\childdocmain| and puts it
at the beginning of every child section;
however, a white\-space is ignored.
\end{itemize}

%%%%%%%%%%%%%%%%%%%%%%%%%%%%%%%%%%%%%%%%
\paragraph{Content of Main File.}

It is advisable to place all content in the child files included by |\include|.
Any output contained in the main file will appear in all child documents
unless suppressed manually;
it cannot be suppressed automatically by the |\includeonly| directive
and thus should normally be avoided.
A method to include some content in the main file
by means of conditional processing is described in \secref{sec:conditional}.

%%%%%%%%%%%%%%%%%%%%%%%%%%%%%%%%%%%%%%%%
\paragraph{Page Numbering.}

When only a part of the document is compiled,
the appropriate numbering of pages
(as well as other status parameters)
is determined from the |.aux| files.
The latter contain information from previous passes.
However this information needs to propagate through
all intermediate child documents.
Therefore the page numbering in child documents may well
be inconsistent until the complete document is compiled at least once.

A useful (if unconventional) way to always ensure a consistent
page numbering is to restart the numbering in each child document
and denote the pages by `\textit{child}|.|\textit{page}'
where \textit{child} represents the chapter/section number of the child file.
This can be achieved by the command
|\numberwithin{page}{|\textit{child}|}|
of the \textsf{amsmath} package
where \textit{child} can be |chapter| or |section|
depending on the chosen structuring.
Alternatively, one can modify the macro |\thepage| appropriately
and reset the counter |page| at the start of each child file.

%%%%%%%%%%%%%%%%%%%%%%%%%%%%%%%%%%%%%%%%%%%%%%%%%%%%%%%%%%%%%%%%%%%%%%%%%%%%%%%%
\subsection{Conditional Processing}
\label{sec:conditional}

The package provides a mechanism to compile different versions
of a document. To customise the versions further some conditional processing
can come in handy to distinguish which version is being compiled.
The package provides two macros to describe the compilation context:

%%%%%%%%%%%%%%%%%%%%%%%%%%%%%%%%%%%%%%%%
\DescribeMacro{\ifchilddoc}
The conditional |\ifchilddoc| distinguishes between the compilation of
child documents and the main document:
%
\begin{center}
|\ifchilddoc |\textit{child-code}| |[|\||else |\textit{main-code}]| \||fi|
\end{center}

%%%%%%%%%%%%%%%%%%%%%%%%%%%%%%%%%%%%%%%%
\DescribeMacro{\childdocname}
\DescribeMacro{\childdocjob}
The macro |\childdocname| contains the filename (without extension)
of the main or child file being processed.
Note that |\childdocjob| will always contain the name of the main file.

%%%%%%%%%%%%%%%%%%%%%%%%%%%%%%%%%%%%%%%%
\paragraph{Title Page.}

Conditional processing can be used to include a title or banner page
in the main document when proper precautions are taken.
Importantly, the code in the main file should ensure that the page counter
(as well as other status parameters which are stored in the |.aux| files)
takes the same value after the conditional processing.
Otherwise the page numbers may take divergent values
depending on which part is compiled.

For example, a title page could be declared by:
%
\begin{center}
\begin{tabular}{l}
|\ifchilddoc\||else|\\
|\addtocounter{page}{-1}|\\
\textit{code for title page}\\
|\newpage|\\
|\||fi|
\end{tabular}
\end{center}
%
A banner page for the child documents can be generated by:
%
\begin{center}
\begin{tabular}{l}
|\ifchilddoc|\\
|\addtocounter{page}{-1}|\\
\textit{code for banner page}\\
|\newpage|\\
|\||fi|
\end{tabular}
\end{center}
%
Here one could write a message such as:
\begin{center}
|This is the part \childdocname{} of \childdocjob{}.|
\end{center}

%%%%%%%%%%%%%%%%%%%%%%%%%%%%%%%%%%%%%%%%%%%%%%%%%%%%%%%%%%%%%%%%%%%%%%%%%%%%%%%%
\subsection{Flags}
\label{sec:flags}

The package makes it easy to generate different versions
of the main or child documents.
To this end compilation flags can be defined
and assigned different default values.
They will be particularly useful in conjunction
with the forwarding mechanism described in \secref{sec:forward}.

For example, it may be useful to have a flag |\version|
which can be set to |draft| or |final|.
The document source will contain some conditional code
depending on the value of |\version|.
Suppose further, the flag should default to |final| for the main file
and to |draft| for child files
which is a natural assignment for editing the document.
This is achieved by placing the following code
in the preamble of the main document
(below the |\childdocmain| directive):
%
\begin{center}
\begin{tabular}{l}
|\ifchilddoc|\\
|\providecommand{\version}{draft}|\\
|\||else|\\
|\providecommand{\version}{final}|\\
|\||fi|
\end{tabular}
\end{center}
%
The definition by |\providecommand| makes sure
that previous definitions are not overwritten.
Further statements |\providecommand{\version}{...}|
can thus be added before the above code to override it.

For the main file, one might add a line
(between |\childdocmain| and the above block)
%
\begin{center}
|%\ifchilddoc\||else\providecommand{\version}{draft}\||fi|
\end{center}
%
which can be uncommented to produce a draft version.
Likewise one can add a line to the very top of a child file
(above the |\childdocof{|\textit{main}|}| directive)
%
\begin{center}
|%\providecommand{\version}{final}|
\end{center}
%
which can be uncommented to produce the final version of this child document.

%%%%%%%%%%%%%%%%%%%%%%%%%%%%%%%%%%%%%%%%%%%%%%%%%%%%%%%%%%%%%%%%%%%%%%%%%%%%%%%%
\subsection{Forwarding}
\label{sec:forward}

Different versions of the main or child documents
using compilation flags as described in \secref{sec:flags}
can be (permanently) stored in different files
for convenient compilation, viewing and distribution.
To this end, the package defines a command
to pass on compilation to a different file:

%%%%%%%%%%%%%%%%%%%%%%%%%%%%%%%%%%%%%%%%
\DescribeMacro{\childdocforward}
The command |\childdocforward| redirects processing to
another source file:
%
\begin{center}
\begin{tabular}{l}
|\input{childdoc.def}|\\
|\childdocforward[|\textit{main}|]{|\textit{dest}|}|\\
\end{tabular}
\end{center}
%
The argument \textit{dest} is the destination file
(without extension).
It should be the main file or one of the child files.
Note that further \textsf{childdoc} directives
such as |\childdocof| and |\childdocforward|
in the indicated file will be processed in this form.
The optional argument \textit{main}
passes on directly to the main file \textit{main}
while pretending to compile the child \textit{dest}.
This form behaves as if \textit{dest}
issues |\childdocof{|\textit{main}|}| right away,
and no further \textsf{childdoc} directives will be processed.

%%%%%%%%%%%%%%%%%%%%%%%%%%%%%%%%%%%%%%%%
\DescribeMacro{\...prefix}
In the alternative form |\childdocforwardprefix|,
%
\begin{center}
\begin{tabular}{l}
|\input{childdoc.def}|\\
|\childdocforwardprefix[|\textit{main}|]{|\textit{prefix}|}{|\textit{dest}|}|
\end{tabular}
\end{center}
%
the destination file is determined by a pattern
depending on the current file:
To make this work, the current file must be called
`{\textit{prefix}\hspace{0.2em}\textit{suffix}}'
with \textit{prefix} matching precisely the argument.
Processing is then passed on to the file
`{\textit{dest}\hspace{0.2em}\textit{suffix}}'.
Surely, the same effect is achieved by
directly specifying the
argument `{\textit{dest}\hspace{0.2em}\textit{suffix}}'
in the first form.
However, that requires to set up a different file
for each child. With the alternative form of the command
all these files can have exactly the same content
which simplifies setting them up and maintaining them.

For example, the following file |draft.tex|
with a compilation flag |\version| as described in \secref{sec:flags}
compiles the main document as a draft:
%
\begin{center}
\begin{tabular}{l}
|\def\version{draft}|\\
|\input{childdoc.def}|\\
|\childdocforward{|\textit{main}|}|
\end{tabular}
\end{center}
%
Likewise, the following files |final|\textit{nn}|.tex|
compile the final version of the child document
|child|\textit{nn}|.tex|:
%
\begin{center}
\begin{tabular}{l}
|\def\version{final}|\\
|\input{childdoc.def}|\\
|\childdocforwardprefix{final}{child}|
\end{tabular}
\end{center}
%

Note that when several versions of a main file and/or of each child file
are to be generated, it may be convenient to set up a |Makefile| or
shell script to automatise the process.

%%%%%%%%%%%%%%%%%%%%%%%%%%%%%%%%%%%%%%%%%%%%%%%%%%%%%%%%%%%%%%%%%%%%%%%%%%%%%%%%
\subsection{Command Line Processing}
\label{sec:commandline}

The effect of redirection files can also be achieved by invoking
the \LaTeX{} compiler with a more elaborate command line.
Most conveniently this should be done as part
of a shell script or a |Makefile|.

When using \textsf{childdoc} in the main file, the following
command lines effectively perform a redirection
(note that depending on the shell being used,
backslashes may have to be doubled: `|\|' $\to$ `|\\|'):
%
\begin{center}
|... -jobname "|\textit{target}|" |\\|"|[\textit{flags}]%
|\input{childdoc.def}\childdocforward[|\textit{main}|]{|\textit{dest}|}"|
\end{center}
%
Here \textit{target} is the name of the output file,
\textit{main} is the name of the main file
and \textit{dest} is the name of the main or child file to be processed
(all filenames without extensions).
The optional argument \textit{main} can be omitted
if \textit{main} matches \textit{dest}.
Optionally, compilation \textit{flags} can be defined via |\def| commands.
This command line makes the \TeX{} engine believe
it is compiling the file \textit{target}
whose content is specified as the latter parameter.
The provided code then forwards the processing to
\textit{main} or \textit{dest} as described in \secref{sec:forward}.

%%%%%%%%%%%%%%%%%%%%%%%%%%%%%%%%%%%%%%%%%%%%%%%%%%%%%%%%%%%%%%%%%%%%%%%%%%%%%%%%
\subsection{Include by Input}
\label{sec:input}

Including child documents by |\include| has some restrictions by design.
Most notably, the content of a child document always occupies
its own set of pages; pages cannot be shared between child documents.
Usually, this behaviour makes perfect sense
because each child document contain an essential part of the document.
However, in some situations it may be desirable to compose
a document from a collection of parts
without having mandatory page breaks between then.
For this case, the package
provides a mechanism to include parts
by |\input| which can also be processed individually.
However, by construction this mechanism
requires manual handling of the content to be output.

%%%%%%%%%%%%%%%%%%%%%%%%%%%%%%%%%%%%%%%%
\DescribeMacro{\ifchilddocmanual}
The main file should be prepared as usual, see \secref{sec:include}.
However, the document body must make a distinction
between processing of an individual part and of the main document, e.g.:
%
\begin{center}
\begin{tabular}{l}
|\ifchilddocmanual|\\
|\input{\childdocname}|\\
|\||else|\\
\textit{document body with }|\input{|\textit{part}|}|\\
|\||fi|
\end{tabular}
\end{center}
%
The conditional |\ifchilddocmanual| is true whenever
a part to be included by |\input| is being compiled,
and the name of the part is stored in |\childdocname|.

%%%%%%%%%%%%%%%%%%%%%%%%%%%%%%%%%%%%%%%%
\DescribeMacro{\childdocby}
Each part to be included by |\input| should start with:
%
\begin{center}
\begin{tabular}{l}
|\input{childdoc.def}|\\
|\childdocby{|\textit{main}|}|\\
\end{tabular}
\end{center}
%
The directive |\childdocby| is similar to |\childdocof|
described in \secref{sec:include},
but the subsequent selection of content must be done manually.
To that end, both |\ifchilddoc| and |\ifchilddocmanual|
will be true upon processing of a part,
and the name of the part is stored in |\childdocname|.
Note that |\jobname| will be set to the filename of the current part
so that each part receives an individual |.aux| file
that does not interfere with the |.aux| file(s) of the main document.
This behaviour can be altered by the alternative form
|\childdocby[*]{|\textit{main}|}| (with a non-empty optional argument)
which uses the |.aux| file of the main document
by setting |\jobname| to \textit{main}.

%%%%%%%%%%%%%%%%%%%%%%%%%%%%%%%%%%%%%%%%%%%%%%%%%%%%%%%%%%%%%%%%%%%%%%%%%%%%%%%%
\subsection{Driver Development}
\label{sec:driver}

The \textsf{childdoc} mechanism can also be use for the development
of definition files such as \LaTeX{} styles or classes.
This case differs from the above setup with multiple parts
included by |\include| in that no |\includeonly| should be invoked.
This can be achieved by starting the include file
(before |\ProvidesPackage|) with:
%
\begin{center}
\begin{tabular}{l}
|\input{childdoc.def}|\\
|\childdocforward{|\textit{main}|}|\\
\end{tabular}
\end{center}
%
or alternatively with:
%
\begin{center}
\begin{tabular}{l}
|\input{childdoc.def}|\\
|\childdocby{|\textit{main}|}|\\
\end{tabular}
\end{center}
%
Both forms have slightly different effects as described above.
The main file is prepared as usual, see \secref{sec:include}.

%%%%%%%%%%%%%%%%%%%%%%%%%%%%%%%%%%%%%%%%%%%%%%%%%%%%%%%%%%%%%%%%%%%%%%%%%%%%%%%%
\subsection{Legacy Detection}
\label{sec:detection}

The directive |\childdocmain| in the main file can detect
whether the complete document or merely a child is to be compiled
even without using the directive |\childdocof|.
This method is deprecated because it is less robust
and there is no compelling reason to use it;
it is merely provided for backward compatibility
and it may be removed in future versions.

If the detection mechanism is to be used,
it is mandatory to correctly specify
the filename of the main file as the argument of |\childdocmain|:
%
\begin{center}
\begin{tabular}{l}
|\input{childdoc.def}|\\
|\childdocmain{|\textit{main}|}|\\
\end{tabular}
\end{center}
%
If |\jobname| does not match the argument \textit{main} of |\childdocmain|,
it is assumed that |\jobname| points to the child file to be compiled.
When using |\childdocmain| with the main file specified as argument,
it suffices to start a child file
with just |\input{|\textit{main}|}|
without loading of the package and using |\childdocof|.
If instead all processing is done
with the appropriate \textsf{childdoc} directives,
the argument of \textit{main} of |\childdocmain| can be empty.

An alternative version of the command line processing described
in \secref{sec:commandline} using the detection mechanism reads:
%
\begin{center}
|... -jobname "|\textit{target}|" "|[\textit{flags}]%
[|\def\jobname{|\textit{dest}|}|]|\input{|\textit{main}|}"|
\end{center}

%%%%%%%%%%%%%%%%%%%%%%%%%%%%%%%%%%%%%%%%%%%%%%%%%%%%%%%%%%%%%%%%%%%%%%%%%%%%%%%%
\subsection{Manual Code}
\label{sec:manual}

In case one cannot be certain whether the definitions file |childdoc.def|
is installed on the target \TeX{} distribution
and one prefers not to ship it,
it is conceivable to paste a few relevant commands into the sources.

To that end, drop all statements |\input{childdoc.def}|
and perform the replacements as outlined below.
Instead of |\childdocmain{|\textit{main}|}| add the following code
to the top of the main file:
%
\begin{center}
\begin{tabular}{l}
|\||ifdefined\childdocname\endinput\||fi\newif\ifchilddoc|\\
|\edef\childdocname{\scantokens\expandafter{\jobname\noexpand}}|\\
|\def\childdocmain{|\textit{main}|}\||ifx\childdocmain\childdocname\||else|\\
|\childdoctrue\includeonly{\childdocname}\let\jobname\childdocmain\||fi|\\
\end{tabular}
\end{center}
%
Instead of |\childdocof{|\textit{main}|}| just include the main file
at the top of each child file:
%
\begin{center}
|\input{|\textit{main}|}|
\end{center}
%
A simple redirection |\childdocforward{|\textit{dest}|}| is achieved by:
%
\begin{center}
|\def\jobname{|\textit{dest}|}\input{\jobname}|
\end{center}
%
The redirection with prefix
|\childdocforwardprefix[|\textit{prefix}|]{|\textit{dest}|}|
is accomplished by:
%
\begin{center}
\begin{tabular}{l}
|{\edef\jobname{\scantokens\expandafter{\jobname\noexpand}}|\\
|\def\redirectjob |\textit{prefix}|#1~~~{\gdef\jobname{|\textit{dest}|#1}}|\\
|\expandafter\redirectjob\jobname~~~}\input{\jobname}|
\end{tabular}
\end{center}

In an alternative approach,
child documents can be compiled by a specific command line
without additional code or specific definitions:
%
\begin{center}
|... -jobname "|\textit{target}|" "|[\textit{flags}]%
|\includeonly{|\textit{dest}|}\input{|\textit{main}|}"|
\end{center}
%

%%%%%%%%%%%%%%%%%%%%%%%%%%%%%%%%%%%%%%%%%%%%%%%%%%%%%%%%%%%%%%%%%%%%%%%%%%%%%%%%
%%%%%%%%%%%%%%%%%%%%%%%%%%%%%%%%%%%%%%%%%%%%%%%%%%%%%%%%%%%%%%%%%%%%%%%%%%%%%%%%
\section{Information}

%%%%%%%%%%%%%%%%%%%%%%%%%%%%%%%%%%%%%%%%%%%%%%%%%%%%%%%%%%%%%%%%%%%%%%%%%%%%%%%%
\subsection{Copyright}

Copyright \copyright{} 2017--2018 Niklas Beisert

This work may be distributed and/or modified under the
conditions of the \LaTeX{} Project Public License, either version 1.3
of this license or (at your option) any later version.
The latest version of this license is in
  \url{http://www.latex-project.org/lppl.txt}
and version 1.3 or later is part of all distributions of \LaTeX{}
version 2005/12/01 or later.

This work has the LPPL maintenance status `maintained'.

The Current Maintainer of this work is Niklas Beisert.

This work consists of the files |README.txt|, |childdoc.ins| and |childdoc.dtx|
as well as the derived files |childdoc.def|, |cdocsamp.tex|
with |cdocsch1.tex|, |cdocsch2.tex|, |cdocspt3.tex|, |cdocspt4.tex|,
|cdocsdrf.tex|, |cdocsfn1.tex|, |cdocsfn2.tex|
as well as |childdoc.pdf|.

%%%%%%%%%%%%%%%%%%%%%%%%%%%%%%%%%%%%%%%%%%%%%%%%%%%%%%%%%%%%%%%%%%%%%%%%%%%%%%%%
\subsection{Files and Installation}

The package consists of the files:
%
\begin{center}
\begin{tabular}{ll}
    |README.txt|   & readme file \\
    |childdoc.ins| & installation file \\
    |childdoc.dtx| & source file \\
    |childdoc.def| & definition file \\
    |cdocsamp.tex| & sample main file \\
    |cdocsch1.tex| & sample include file \\
    |cdocsch2.tex| & sample include file \\
    |cdocspt3.tex| & sample part file \\
    |cdocspt4.tex| & sample part file \\
    |cdocsdrf.tex| & sample redirection file \\
    |cdocsfn1.tex| & sample redirection file \\
    |cdocsfn2.tex| & sample redirection file \\
    |childdoc.pdf| & manual
\end{tabular}
\end{center}
%
The distribution consists of the files
|README.txt|, |childdoc.ins| and |childdoc.dtx|.
%
\begin{itemize}
\item
Run (pdf)\LaTeX{} on |childdoc.dtx|
to compile the manual |childdoc.pdf| (this file).
\item
Run \LaTeX{} on |childdoc.ins| to create the definitions file |childdoc.def|
and the sample |cdocsamp.tex| with include files
|cdocsch1.tex|, |cdocsch2.tex|, |cdocspt3.tex|, |cdocspt4.tex|,
|cdocsdrf.tex|, |cdocsfn1.tex|, |cdocsfn2.tex|.
Then copy the file |childdoc.def| to an appropriate directory of your \LaTeX{}
distribution, e.g.\ \textit{texmf-root}|/tex/latex/childdoc|.
\end{itemize}

%%%%%%%%%%%%%%%%%%%%%%%%%%%%%%%%%%%%%%%%%%%%%%%%%%%%%%%%%%%%%%%%%%%%%%%%%%%%%%%%
\subsection{Related CTAN Packages}

There are several other packages which offer a similar functionality:
%
\begin{itemize}
\item
The packages
\href{http://ctan.org/pkg/docmute}{\textsf{docmute}},
\href{http://ctan.org/pkg/includex}{\textsf{includex}} and
\href{http://ctan.org/pkg/standalone}{\textsf{standalone}}
provide commands to include only the document body of
a child file thus allowing both files to be compiled individually.
\item
The packages \href{http://ctan.org/pkg/subdocs}{\textsf{subdocs}}
and \href{http://ctan.org/pkg/subfiles}{\textsf{subfiles}}
provide structures in which the main and child documents can be
encapsulated and allowing them to be compiled individually.
The inclusion mechanism is different from the conventional |\include|.
\item
The package \href{http://ctan.org/pkg/combine}{\textsf{combine}}
is an elaborate solution to combine several documents into one.
\end{itemize}
%
See also the CTAN topic \href{http://ctan.org/topic/subdocs}{\textsf{subdocs}}
for further related packages.
The present package differs from the above solutions in that
a document structure constructed with the conventional |\include| mechanism
just needs two extra commands at the top of every file
such that all constituent files can be compiled individually.

%%%%%%%%%%%%%%%%%%%%%%%%%%%%%%%%%%%%%%%%%%%%%%%%%%%%%%%%%%%%%%%%%%%%%%%%%%%%%%%%
%\subsection{Feature Suggestions}
%
%The following is a list of features which may be useful for future
%versions of this package:
%%
%\begin{itemize}
%\item
%\ldots
%\end{itemize}

%%%%%%%%%%%%%%%%%%%%%%%%%%%%%%%%%%%%%%%%%%%%%%%%%%%%%%%%%%%%%%%%%%%%%%%%%%%%%%%%
\subsection{Revision History}

%%%%%%%%%%%%%%%%%%%%%%%%%%%%%%%%%%%%%%%%
\paragraph{v2.0:} 2018/12/30

\begin{itemize}
\item
immediate forward processing
\item
added |\childdocby| mechanism
\item
manual restructured
\end{itemize}

%%%%%%%%%%%%%%%%%%%%%%%%%%%%%%%%%%%%%%%%
\paragraph{v1.6:} 2018/01/17

\begin{itemize}
\item
application for development of include files
\item
corrections to manual
\end{itemize}

%%%%%%%%%%%%%%%%%%%%%%%%%%%%%%%%%%%%%%%%
\paragraph{v1.5:} 2017/05/21

\begin{itemize}
\item
more complete structuring introduced
\item
|\childdocof| introduced
\item
|\childdoc| renamed to |\childdocmain|
\item
|\childredirect| renamed to |\childdocforward| and |\childdocforwardprefix|
and functionality expanded
\end{itemize}

%%%%%%%%%%%%%%%%%%%%%%%%%%%%%%%%%%%%%%%%
\paragraph{v1.0:} 2017/04/27

\begin{itemize}
\item
manual and install package
\item
first version published on CTAN
\end{itemize}

%%%%%%%%%%%%%%%%%%%%%%%%%%%%%%%%%%%%%%%%
\paragraph{v0.6:} 2017/04/26

\begin{itemize}
\item
redirection mechanism added
\end{itemize}

%%%%%%%%%%%%%%%%%%%%%%%%%%%%%%%%%%%%%%%%
\paragraph{v0.5:} 2017/04/26

\begin{itemize}
\item
functionality in definition file
\end{itemize}


%%%%%%%%%%%%%%%%%%%%%%%%%%%%%%%%%%%%%%%%%%%%%%%%%%%%%%%%%%%%%%%%%%%%%%%%%%%%%%%%
%%%%%%%%%%%%%%%%%%%%%%%%%%%%%%%%%%%%%%%%%%%%%%%%%%%%%%%%%%%%%%%%%%%%%%%%%%%%%%%%
%%%%%%%%%%%%%%%%%%%%%%%%%%%%%%%%%%%%%%%%%%%%%%%%%%%%%%%%%%%%%%%%%%%%%%%%%%%%%%%%
\appendix

\settowidth\MacroIndent{\rmfamily\scriptsize 000\ }

 \DocInput{childdoc.dtx}

\end{document}
%</driver>
% \fi
%
% %%%%%%%%%%%%%%%%%%%%%%%%%%%%%%%%%%%%%%%%%%%%%%%%%%%%%%%%%%%%%%%%%%%%%%%%%%%%%%
% %%%%%%%%%%%%%%%%%%%%%%%%%%%%%%%%%%%%%%%%%%%%%%%%%%%%%%%%%%%%%%%%%%%%%%%%%%%%%%
% \section{Sample}
%\iffalse
%<*samplemain>
%\fi
%
% The following presents a sample document
% with two chapters, two parts, a title page,
% a compile flag as well as three forwarding files to set the flag.
% It consists of eight |.tex| files:
% \begin{center}
% \begin{tabular}{ll}
% |cdocsamp.tex|&main file\\
% |cdocsch1.tex|&include file for chapter 1\\
% |cdocsch2.tex|&include file for chapter 2\\
% |cdocspt3.tex|&include file for part 3\\
% |cdocspt4.tex|&include file for part 4\\
% |cdocsdrf.tex|&forwarding file for main file in draft mode\\
% |cdocsfi1.tex|&forwarding file for final version of chapter 1\\
% |cdocsfi2.tex|&forwarding file for final version of chapter 2\\
% \end{tabular}
% \end{center}
% Each of the eight files can be compiled directly by the \LaTeX{} compiler.
%
% %%%%%%%%%%%%%%%%%%%%%%%%%%%%%%%%%%%%%%
% \paragraph{Main File.}
%
% The main file is called |cdocsamp.tex|.
%
% Load the \textsf{childdoc} definitions and
% declare the filename for the main document:
%    \begin{macrocode}
\input{childdoc.def}
\childdocmain{}
%    \end{macrocode}

% Optional override for |\version| flag:
%    \begin{macrocode}
%%\ifchilddoc\else\providecommand{\version}{draft}\fi
%    \end{macrocode}

% Define the default values for the |\version| flag
% (|final| for the main file and |draft| for childs):
%    \begin{macrocode}
\ifchilddoc
\providecommand{\version}{draft}
\else
\providecommand{\version}{final}
\fi
%    \end{macrocode}

% Load the standard document class:
%    \begin{macrocode}
\documentclass[12pt]{article}
%    \end{macrocode}

% Start the document body:
%    \begin{macrocode}
\begin{document}
%    \end{macrocode}

% Declare a title page.
% Print title, part of document being processed and version flag:
%    \begin{macrocode}
\addtocounter{page}{-1}
\begin{center}
{\LARGE\bfseries{}childdoc example\par}
\vspace{1cm}
\ifchilddoc
\ifchilddocmanual part\else chapter\fi:
`\childdocname' of `\childdocjob'\par
\else
main document: `\childdocjob'\par
\fi
version: \version\par
\end{center}
\newpage
%    \end{macrocode}

% Manually include selected file,
% otherwise process as usual:
%    \begin{macrocode}
\ifchilddocmanual
\section*{part `\childdocname'}
\input{\childdocname}
\else
%    \end{macrocode}

% Include the two chapters:
%    \begin{macrocode}
\include{cdocsch1}
\include{cdocsch2}
%    \end{macrocode}

% Include the two parts unless only chapters should be displayed:
%    \begin{macrocode}
\ifchilddoc\else
\section{part three}
\input{cdocspt3}
\section{part four}
\input{cdocspt4}
\fi
%    \end{macrocode}

% Process as usual until here:
%    \begin{macrocode}
\fi
%    \end{macrocode}

% End of document body:
%    \begin{macrocode}
\end{document}
%    \end{macrocode}
%\iffalse
%</samplemain>
%\fi
%
% %%%%%%%%%%%%%%%%%%%%%%%%%%%%%%%%%%%%%%
% \paragraph{Chapter Include Files.}
%
% The include files are called |cdocsch1.tex| and |cdocsch2.tex|.
%
%\iffalse
%<*samplechap1|samplechap2>
%\fi

% Optional override for |\version| flag:
%    \begin{macrocode}
%%\providecommand{\version}{final}
%    \end{macrocode}

% Include the main document:
%    \begin{macrocode}
\input{childdoc.def}
\childdocof{cdocsamp}
%    \end{macrocode}

%\iffalse
%</samplechap1|samplechap2>
%\fi
%
%\iffalse
%<*samplechap1>
%\fi
% Some text for chapter 1:
%    \begin{macrocode}
\section{one}
some text in chapter one
%    \end{macrocode}

%\iffalse
%</samplechap1>
%\fi
% Some text for chapter 2:
%\iffalse
%<*samplechap2>
%\fi
%    \begin{macrocode}
\section{two}
more text in chapter two
%    \end{macrocode}

%\iffalse
%</samplechap2>
%\fi
%
% %%%%%%%%%%%%%%%%%%%%%%%%%%%%%%%%%%%%%%
% \paragraph{Part Include Files.}
%
% The include files are called |cdocspt3.tex| and |cdocspt4.tex|.
%
%\iffalse
%<*samplepart3|samplepart4>
%\fi

% Optional override for |\version| flag:
%    \begin{macrocode}
%%\providecommand{\version}{final}
%    \end{macrocode}

% Include the main document:
%    \begin{macrocode}
\input{childdoc.def}
\childdocby{cdocsamp}
%    \end{macrocode}

%\iffalse
%</samplepart3|samplepart4>
%\fi
%
%\iffalse
%<*samplepart3>
%\fi
% Some text for part 3:
%    \begin{macrocode}
some text in part three
%    \end{macrocode}

%\iffalse
%</samplepart3>
%\fi
% Some text for part 4:
%\iffalse
%<*samplepart4>
%\fi
%    \begin{macrocode}
more text in part four
%    \end{macrocode}

%\iffalse
%</samplepart4>
%\fi
%
% %%%%%%%%%%%%%%%%%%%%%%%%%%%%%%%%%%%%%%
% \paragraph{Forwarding for a Complete Draft.}
%
% The following forwarding file |cdocsdrf.tex|
% compiles the main document in draft mode:
%\iffalse
%<*sampledraft>
%\fi
%    \begin{macrocode}
\def\version{draft}
\input{childdoc.def}
\childdocforward{cdocsamp}
%    \end{macrocode}

%\iffalse
%</sampledraft>
%\fi
%
% %%%%%%%%%%%%%%%%%%%%%%%%%%%%%%%%%%%%%%
% \paragraph{Forwarding for Final Version of the Chapters.}
%
% The following forwarding files |cdocsfn1.tex| and |cdocsfn2.tex|
% (with identical content)
% compile the final versions of the child documents
% |cdocsch1.tex| and |cdocsch2.tex|, respectively:
%\iffalse
%<*samplefinal>
%\fi
%    \begin{macrocode}
\def\version{final}
\input{childdoc.def}
\childdocforwardprefix[cdocsamp]{cdocsfn}{cdocsch}
%    \end{macrocode}

%\iffalse
%</samplefinal>
%\fi
%
% %%%%%%%%%%%%%%%%%%%%%%%%%%%%%%%%%%%%%%
% \paragraph{Command Line Processing.}
%
% The following three command lines generate the output files
% |cdocscld|, |cdocscl1| and |cdocscl2|
% which should be identical to
% |cdocsdrf|, |cdocsch1| and |cdocsfn2|, respectively:
% \begin{center}
% \begin{tabular}{l}
% |latex -jobname cdocscld \|\\
% |  "\def\version{draft}\input{childdoc.def}\childdocforward{cdocsamp}"|\\
% |latex -jobname cdocscl1 \|\\
% |  "\input{childdoc.def}\childdocforward[cdocsamp]{cdocsch1}"|\\
% |latex -jobname cdocscl2 \|\\
% |  "\def\version{final}\input{childdoc.def}\childdocforward{cdocsch2}"|
% \end{tabular}
% \end{center}
% Note that the trailing backslash on each first line
% merely continues the input to the second line
% (for convenient cut ant paste).
% Furthermore, the command |latex| can be replaced by any
% of its alternative versions such as |pdflatex|.
%
% %%%%%%%%%%%%%%%%%%%%%%%%%%%%%%%%%%%%%%%%%%%%%%%%%%%%%%%%%%%%%%%%%%%%%%%%%%%%%%
% %%%%%%%%%%%%%%%%%%%%%%%%%%%%%%%%%%%%%%%%%%%%%%%%%%%%%%%%%%%%%%%%%%%%%%%%%%%%%%
% \section{Implementation}
%\iffalse
%<*package>
%\fi
%
% This section describes the definitions file |childdoc.def|.

% The definitions cannot be loaded using |\usepackage| or |\RequirePackage|
% which has a mechanism to prevent loading a style file more than once.
% When loading the definitions by means of |\input|
% multiple instances have to be prevented manually:
%\iffalse
%This code needs to be before the `\ProvidesFile' directive
%which is defined at the beginning of this file.
%Therefore it is also placed there and commented out here.
%</package>
%<*discard>
%\fi
%    \begin{macrocode}
\ifdefined\childdocmain\endinput\fi
%    \end{macrocode}
%\iffalse
%</discard>
%<*package>
%\fi
%
% \macro{\ifchilddoc}
% \macro{\ifchilddocmanual}
% The conditional |\ifchilddoc| tells whether a
% child (true) or main (false) document is being compiled.
% The conditional |\ifchilddocmanual| tells whether
% the |\includeonly| mechanism is used (false) or
% the selection of child files must be performed manually (true).
% The definitions initialise to false:
%    \begin{macrocode}
\newif\ifchilddoc
\newif\ifchilddocmanual
%    \end{macrocode}

% \macro{\childdocname}
% \macro{\childdocjob}
% The macro |\childdocname| stores the name of the main document
% to be compiled. The macro |\childdocjob| stores the name of
% the document on which the \LaTeX{} compiler was originally invoked.
% The content of |\jobname| cannot be compared
% to filenames specified in the source due to different catcodes.
% The following code rescans |\jobname|, stores the result
% in |\childdocname| and saves a copy in |\childdocjob|:
%    \begin{macrocode}
\edef\childdocname{\scantokens\expandafter{\jobname\noexpand}}
\let\childdocjob\childdocname
%    \end{macrocode}

% \macro{\childdocdisable}
% The macro |\childdocdisable| prevents the main file
% from being processed more than once.
% At this stage, the main document command |\childdocmain|
% is assumed to be called once again where it should do nothing.
% Any subsequent call to it should prevent
% a secondary processing of the main document
% It overwrites the forwarding commands
% |\childdocof| and |\childdocforward|
% with empty macros to prevent further inclusions of the main document:
%    \begin{macrocode}
\newcommand{\childdocdisable}
{
  \renewcommand{\childdocmain}[1]{\renewcommand{\childdocmain}[1]{\endinput}}
  \renewcommand{\childdocof}[1]{}
  \renewcommand{\childdocby}[2][]{}
  \renewcommand{\childdocforward}[2][]{}
  \renewcommand{\childdocdisable}{}
}
%    \end{macrocode}

% \macro{\childdocmain}
% The macro |\childdocmain| is to be called at the top of the main file
% with nothing or the main filename (without extension) as argument.
% First, it breaks loops.
% If the argument is not empty and does not match |\childdocname|
% (which is set by the first inclusion of |childdoc.def|),
% |\ifchilddoc| is set to true, |\includeonly| is applied to the child file
% and |\jobname| is set to the main file
% (for proper handling of |.aux| files):
%    \begin{macrocode}
\newcommand{\childdocmain}[1]
{
  \childdocdisable\childdocmain{}
  \if?#1?\else
    \begingroup
      \def\childdoctmp{#1}
      \ifx\childdoctmp\childdocname
        \def\childdoctmp{}
      \else
        \def\childdoctmp
        {
          \childdoctrue
          \includeonly{\childdocname}
          \def\childdocjob{#1}
          \def\jobname{#1}
        }
      \fi
      \expandafter
    \endgroup
    \childdoctmp
  \fi
}
%    \end{macrocode}

% \macro{\childdocof}
% The command |\childdocof| redirects
% compilation to the main file |#1|.
%    \begin{macrocode}
\newcommand{\childdocof}[1]
{
  \childdocdisable
  \childdoctrue
  \includeonly{\childdocname}
  \def\jobname{#1}
  \def\childdocjob{#1}
  \input{#1}
}
%    \end{macrocode}

% \macro{\childdocby}
% The command |\childdocby| ....
%    \begin{macrocode}
\newcommand{\childdocby}[2][]
{
  \childdocdisable
  \childdoctrue
  \childdocmanualtrue
  \if?#1?\else
    \def\jobname{#2}
  \fi
  \def\childdocjob{#2}
  \input{#2}
  \endinput
}
%    \end{macrocode}

% \macro{\childdocforward}
% The command |\childdocforward| redirects
% compilation to the main file or
% (if the optional argument is given) a child file.
% Parameters are set as if the main file
% or a child file starting with |\childdocof| was compiled.
% Then compilation is handed over to the main file:
%    \begin{macrocode}
\newcommand{\childdocforward}[2][]
{
  \begingroup
    \if?#1?
      \def\childdoctmp
      {
        \def\childdocname{#2}
        \def\childdocjob{#2}
        \def\jobname{#2}
        \input{#2}
        \endinput
      }
    \else
      \def\childdoctmp
      {
        \childdocdisable
        \def\childdocname{#2}
        \childdoctrue
        \includeonly{#2}
        \def\childdocjob{#1}
        \def\jobname{#1}
        \input{#1}
        \endinput
      }
    \fi
    \expandafter
  \endgroup
  \childdoctmp
}
%    \end{macrocode}

% \macro{\childdocforwardprefix}
% The command |\childdocforwardprefix| redirects
% compilation to the main or a child file by means of a pattern.
% The prefix |#1| in the current filename is replaced by |#2|
% and the suffix of the current filename is kept
% (it is assumed that the filename does not contain the substring `|~~~|'
% which is used as a delimiter).
% Compilation is handed over to the new file by |\childdocforward|:
%    \begin{macrocode}
\newcommand{\childdocforwardprefix}[3][]
{
  \begingroup
    \def\childdocextract #2##1~~~{\def\childdoctmp{\childdocforward[#1]{#3##1}}}
    \expandafter\childdocextract\childdocname~~~
    \expandafter
  \endgroup
  \childdoctmp
}
%    \end{macrocode}

% \macro{\childdoc}
% The deprecated macro |\childdoc| is a legacy version of |\childdocmain|:
%    \begin{macrocode}
\newcommand{\childdoc}{\childdocmain}
%    \end{macrocode}

% \macro{\childdocredirect}
% The deprecated macro |\childdocredirect| is a legacy version
% of |\childdocforward| and |\childdocforwardprefix|:
%    \begin{macrocode}
\newcommand{\childdocredirect}[2][]
{
  \begingroup
    \if?#1?
      \def\childdoctmp{\childdocforward{#2}}
    \else
      \def\childdoctmp{\childdocforwardprefix{#1}{#2}}
    \fi
    \expandafter
  \endgroup
  \childdoctmp
}
%    \end{macrocode}

%\iffalse
%</package>
%\fi
%
\endinput

\childdocmain{}
%    \end{macrocode}

% Optional override for |\version| flag:
%    \begin{macrocode}
%%\ifchilddoc\else\providecommand{\version}{draft}\fi
%    \end{macrocode}

% Define the default values for the |\version| flag
% (|final| for the main file and |draft| for childs):
%    \begin{macrocode}
\ifchilddoc
\providecommand{\version}{draft}
\else
\providecommand{\version}{final}
\fi
%    \end{macrocode}

% Load the standard document class:
%    \begin{macrocode}
\documentclass[12pt]{article}
%    \end{macrocode}

% Start the document body:
%    \begin{macrocode}
\begin{document}
%    \end{macrocode}

% Declare a title page.
% Print title, part of document being processed and version flag:
%    \begin{macrocode}
\addtocounter{page}{-1}
\begin{center}
{\LARGE\bfseries{}childdoc example\par}
\vspace{1cm}
\ifchilddoc
\ifchilddocmanual part\else chapter\fi:
`\childdocname' of `\childdocjob'\par
\else
main document: `\childdocjob'\par
\fi
version: \version\par
\end{center}
\newpage
%    \end{macrocode}

% Manually include selected file,
% otherwise process as usual:
%    \begin{macrocode}
\ifchilddocmanual
\section*{part `\childdocname'}
\input{\childdocname}
\else
%    \end{macrocode}

% Include the two chapters:
%    \begin{macrocode}
\include{cdocsch1}
\include{cdocsch2}
%    \end{macrocode}

% Include the two parts unless only chapters should be displayed:
%    \begin{macrocode}
\ifchilddoc\else
\section{part three}
\input{cdocspt3}
\section{part four}
\input{cdocspt4}
\fi
%    \end{macrocode}

% Process as usual until here:
%    \begin{macrocode}
\fi
%    \end{macrocode}

% End of document body:
%    \begin{macrocode}
\end{document}
%    \end{macrocode}
%\iffalse
%</samplemain>
%\fi
%
% %%%%%%%%%%%%%%%%%%%%%%%%%%%%%%%%%%%%%%
% \paragraph{Chapter Include Files.}
%
% The include files are called |cdocsch1.tex| and |cdocsch2.tex|.
%
%\iffalse
%<*samplechap1|samplechap2>
%\fi

% Optional override for |\version| flag:
%    \begin{macrocode}
%%\providecommand{\version}{final}
%    \end{macrocode}

% Include the main document:
%    \begin{macrocode}
% \iffalse
%
% childdoc.dtx Copyright (C) 2017-2018 Niklas Beisert
%
% This work may be distributed and/or modified under the
% conditions of the LaTeX Project Public License, either version 1.3
% of this license or (at your option) any later version.
% The latest version of this license is in
%   http://www.latex-project.org/lppl.txt
% and version 1.3 or later is part of all distributions of LaTeX
% version 2005/12/01 or later.
%
% This work has the LPPL maintenance status `maintained'.
%
% The Current Maintainer of this work is Niklas Beisert.
%
% This work consists of the files childdoc.dtx and childdoc.ins
% and the derived files childdoc.def and cdocsamp.tex with
% cdocsch1.tex, cdocsch2.tex, cdocsdrf.tex, cdocsfn1.tex, cdocsfn2.tex.
%
%<package>\ifdefined\childdocmain\endinput\fi
%<package>\ProvidesFile{childdoc.def}[2018/12/30 v2.0 child document driver]
%<samplemain>\ProvidesFile{cdocsamp.tex}[2018/12/30 v2.0 sample for childdoc]
%<*driver>
%\ProvidesFile{childdoc.drv}[2018/12/30 v2.0 childdoc reference manual file]
\PassOptionsToClass{10pt,a4paper}{article}
\documentclass{ltxdoc}

\usepackage[margin=35mm]{geometry}
\usepackage{hyperref}
\usepackage{hyperxmp}
\usepackage[usenames]{color}

\hypersetup{colorlinks=true}
\hypersetup{pdfstartview=FitH}
\hypersetup{pdfpagemode=UseNone}
\hypersetup{pdfsource={}}
\hypersetup{pdflang={en-UK}}
\hypersetup{pdfcopyright={Copyright 2017-2018 Niklas Beisert.
  This work may be distributed and/or modified under the
  conditions of the LaTeX Project Public License, either version 1.3
  of this license or (at your option) any later version.}}
\hypersetup{pdflicenseurl={http://www.latex-project.org/lppl.txt}}
\hypersetup{pdfcontactaddress={ETH Zurich, ITP, HIT K,
  Wolfgang-Pauli-Strasse 27}}
\hypersetup{pdfcontactpostcode={8093}}
\hypersetup{pdfcontactcity={Zurich}}
\hypersetup{pdfcontactcountry={Switzerland}}
\hypersetup{pdfcontactemail={nbeisert@itp.phys.ethz.ch}}
\hypersetup{pdfcontacturl={http://people.phys.ethz.ch/\xmptilde nbeisert/}}

\newcommand{\secref}[1]{\hyperref[#1]{section \ref*{#1}}}

\parskip1ex
\parindent0pt
\let\olditemize\itemize
\def\itemize{\olditemize\parskip0pt}

\begin{document}

\title{The \textsf{childdoc} Package}
\hypersetup{pdftitle={The childdoc Package}}
\author{Niklas Beisert\\[2ex]
  Institut f\"ur Theoretische Physik\\
  Eidgen\"ossische Technische Hochschule Z\"urich\\
  Wolfgang-Pauli-Strasse 27, 8093 Z\"urich, Switzerland\\[1ex]
  \href{mailto:nbeisert@itp.phys.ethz.ch}
  {\texttt{nbeisert@itp.phys.ethz.ch}}}
\hypersetup{pdfauthor={Niklas Beisert}}
\hypersetup{pdfsubject={Manual for the LaTeX2e Package childdoc}}
\date{30 December 2018, \textsf{v2.0}}
\maketitle

\begin{abstract}\noindent
\textsf{childdoc} is a \LaTeXe{} package
that enables the direct compilation
of document sections included by |\include|
to individual files.
\end{abstract}

\begingroup
\parskip0ex
\tableofcontents
\endgroup

%%%%%%%%%%%%%%%%%%%%%%%%%%%%%%%%%%%%%%%%%%%%%%%%%%%%%%%%%%%%%%%%%%%%%%%%%%%%%%%%
%%%%%%%%%%%%%%%%%%%%%%%%%%%%%%%%%%%%%%%%%%%%%%%%%%%%%%%%%%%%%%%%%%%%%%%%%%%%%%%%
\section{Introduction}

\LaTeX{} provides a mechanism to structure a large document (such as a book)
into a main file and several child files (containing the chapters)
using the |\include| command.
This mechanism is beneficial for documents
which span hundreds of pages in order to
make the source file(s) more manageable.
Moreover, compilation can be restricted to
selected child files by means of the |\includeonly| command.
The latter feature can be used to reduce the compilation time while editing
(this was significantly more useful in the earlier days of \LaTeX{})
or to generate a smaller document which is easier to navigate.
Another application of |\includeonly| is to generate
documents consisting of selected parts of the complete document.

However, there are a few drawbacks of the plain |\include| mechanism:
\begin{itemize}
\item
The child files cannot be compiled on their own,
they can only be compiled via the main file.
A naive editing environment
(such as a text editor with an option
to have the current file processed by \LaTeX)
may require one to switch to the main file before compiling;
attempting to compile the child file produces errors.
\item
The main file must be modified (each time)
to adjust the |\includeonly| command
to the present needs. This easily leaves the main file in a messy state.
\item
The generated document will always carry the filename
of the main document. This is inconvenient if
several child files are to be compiled and
to be kept for distribution.
\end{itemize}

The present package provides a simple interface
to make child files individually compilable by \LaTeX{}.
Compiling a child file then has the same effect as compiling
the main file with an |\includeonly| command
to select the appropriate child.
Moreover the generated document will carry the name of the child
rather than the main file.
This resolves all three above issues.

This feature is meant to make the editing of books,
thesis documents and lecture notes somewhat more convenient.
However, the package can also be used efficiently for
composing a series of documents (such as exercise sheets)
which are typically distributed individually.
It then assists the author in generating the individual documents
(potentially in different versions)
as well as a document containing the collected series.
Another application is in developing style files
or other kinds of included material
where compilation of the style file could redirect
to a sample or test file.

%%%%%%%%%%%%%%%%%%%%%%%%%%%%%%%%%%%%%%%%%%%%%%%%%%%%%%%%%%%%%%%%%%%%%%%%%%%%%%%%
%%%%%%%%%%%%%%%%%%%%%%%%%%%%%%%%%%%%%%%%%%%%%%%%%%%%%%%%%%%%%%%%%%%%%%%%%%%%%%%%
\section{Usage}

First of all, the package \textsf{childdoc} is \emph{not} a standard
\LaTeXe{} |.sty| style file! Therefore it needs to be invoked in
a non-standard way.

%%%%%%%%%%%%%%%%%%%%%%%%%%%%%%%%%%%%%%%%%%%%%%%%%%%%%%%%%%%%%%%%%%%%%%%%%%%%%%%%
\subsection{Included Files}
\label{sec:include}

%%%%%%%%%%%%%%%%%%%%%%%%%%%%%%%%%%%%%%%%
\DescribeMacro{\childdocmain}
To use the package, add the commands
\begin{center}
\begin{tabular}{l}
|\input{childdoc.def}|\\
|\childdocmain{}|\\
\end{tabular}
\end{center}
at the very top of the main \LaTeX{} file,
in particular \emph{before} the |\documentclass| statement!
The argument of |\childdocmain| should be left empty
(but it must be present).

%%%%%%%%%%%%%%%%%%%%%%%%%%%%%%%%%%%%%%%%
\DescribeMacro{\childdocof}
Furthermore, add the commands
\begin{center}
\begin{tabular}{l}
|\input{childdoc.def}|\\
|\childdocof{|\textit{main}|}|\\
\end{tabular}
\end{center}
at the top of every child file \textit{child}
which is included by |\include{|\textit{child}|}|
from within the main file
(or at least for those files to be compiled individually).
The argument \textit{main} must be the filename of the main file.

There are a couple of
considerations in setting up the main and child documents:

%%%%%%%%%%%%%%%%%%%%%%%%%%%%%%%%%%%%%%%%
\paragraph{Restrictions.}

Please note the following restrictions:
\begin{itemize}
\item
|\childdocmain| must be called with one argument \textit{main}
to ensure compatibility with earlier version of the package.
It must either be empty (|\childdocmain{}|)
or precisely match the filename of the main file in which it is specified.
See \secref{sec:detection} for further information.
\item
The filename \textit{main} must be specified without the |.tex| extension.
\item
The filename \textit{main} is case sensitive
(even in case-insensitive file systems)
due to internal string comparison.
\item
The argument \textit{main} should be fully expanded, it cannot be a macro.
\item
Subdirectories and special characters should be avoided in filenames.
\item
The command |\childdocmain{|\textit{main}|}| must be followed by a whitespace.
It should not be followed immediately by another command
or by a comment mark `|%|'.
This is because the \TeX{} parser reads the token immediately following
the argument of |\childdocmain| and puts it
at the beginning of every child section;
however, a white\-space is ignored.
\end{itemize}

%%%%%%%%%%%%%%%%%%%%%%%%%%%%%%%%%%%%%%%%
\paragraph{Content of Main File.}

It is advisable to place all content in the child files included by |\include|.
Any output contained in the main file will appear in all child documents
unless suppressed manually;
it cannot be suppressed automatically by the |\includeonly| directive
and thus should normally be avoided.
A method to include some content in the main file
by means of conditional processing is described in \secref{sec:conditional}.

%%%%%%%%%%%%%%%%%%%%%%%%%%%%%%%%%%%%%%%%
\paragraph{Page Numbering.}

When only a part of the document is compiled,
the appropriate numbering of pages
(as well as other status parameters)
is determined from the |.aux| files.
The latter contain information from previous passes.
However this information needs to propagate through
all intermediate child documents.
Therefore the page numbering in child documents may well
be inconsistent until the complete document is compiled at least once.

A useful (if unconventional) way to always ensure a consistent
page numbering is to restart the numbering in each child document
and denote the pages by `\textit{child}|.|\textit{page}'
where \textit{child} represents the chapter/section number of the child file.
This can be achieved by the command
|\numberwithin{page}{|\textit{child}|}|
of the \textsf{amsmath} package
where \textit{child} can be |chapter| or |section|
depending on the chosen structuring.
Alternatively, one can modify the macro |\thepage| appropriately
and reset the counter |page| at the start of each child file.

%%%%%%%%%%%%%%%%%%%%%%%%%%%%%%%%%%%%%%%%%%%%%%%%%%%%%%%%%%%%%%%%%%%%%%%%%%%%%%%%
\subsection{Conditional Processing}
\label{sec:conditional}

The package provides a mechanism to compile different versions
of a document. To customise the versions further some conditional processing
can come in handy to distinguish which version is being compiled.
The package provides two macros to describe the compilation context:

%%%%%%%%%%%%%%%%%%%%%%%%%%%%%%%%%%%%%%%%
\DescribeMacro{\ifchilddoc}
The conditional |\ifchilddoc| distinguishes between the compilation of
child documents and the main document:
%
\begin{center}
|\ifchilddoc |\textit{child-code}| |[|\||else |\textit{main-code}]| \||fi|
\end{center}

%%%%%%%%%%%%%%%%%%%%%%%%%%%%%%%%%%%%%%%%
\DescribeMacro{\childdocname}
\DescribeMacro{\childdocjob}
The macro |\childdocname| contains the filename (without extension)
of the main or child file being processed.
Note that |\childdocjob| will always contain the name of the main file.

%%%%%%%%%%%%%%%%%%%%%%%%%%%%%%%%%%%%%%%%
\paragraph{Title Page.}

Conditional processing can be used to include a title or banner page
in the main document when proper precautions are taken.
Importantly, the code in the main file should ensure that the page counter
(as well as other status parameters which are stored in the |.aux| files)
takes the same value after the conditional processing.
Otherwise the page numbers may take divergent values
depending on which part is compiled.

For example, a title page could be declared by:
%
\begin{center}
\begin{tabular}{l}
|\ifchilddoc\||else|\\
|\addtocounter{page}{-1}|\\
\textit{code for title page}\\
|\newpage|\\
|\||fi|
\end{tabular}
\end{center}
%
A banner page for the child documents can be generated by:
%
\begin{center}
\begin{tabular}{l}
|\ifchilddoc|\\
|\addtocounter{page}{-1}|\\
\textit{code for banner page}\\
|\newpage|\\
|\||fi|
\end{tabular}
\end{center}
%
Here one could write a message such as:
\begin{center}
|This is the part \childdocname{} of \childdocjob{}.|
\end{center}

%%%%%%%%%%%%%%%%%%%%%%%%%%%%%%%%%%%%%%%%%%%%%%%%%%%%%%%%%%%%%%%%%%%%%%%%%%%%%%%%
\subsection{Flags}
\label{sec:flags}

The package makes it easy to generate different versions
of the main or child documents.
To this end compilation flags can be defined
and assigned different default values.
They will be particularly useful in conjunction
with the forwarding mechanism described in \secref{sec:forward}.

For example, it may be useful to have a flag |\version|
which can be set to |draft| or |final|.
The document source will contain some conditional code
depending on the value of |\version|.
Suppose further, the flag should default to |final| for the main file
and to |draft| for child files
which is a natural assignment for editing the document.
This is achieved by placing the following code
in the preamble of the main document
(below the |\childdocmain| directive):
%
\begin{center}
\begin{tabular}{l}
|\ifchilddoc|\\
|\providecommand{\version}{draft}|\\
|\||else|\\
|\providecommand{\version}{final}|\\
|\||fi|
\end{tabular}
\end{center}
%
The definition by |\providecommand| makes sure
that previous definitions are not overwritten.
Further statements |\providecommand{\version}{...}|
can thus be added before the above code to override it.

For the main file, one might add a line
(between |\childdocmain| and the above block)
%
\begin{center}
|%\ifchilddoc\||else\providecommand{\version}{draft}\||fi|
\end{center}
%
which can be uncommented to produce a draft version.
Likewise one can add a line to the very top of a child file
(above the |\childdocof{|\textit{main}|}| directive)
%
\begin{center}
|%\providecommand{\version}{final}|
\end{center}
%
which can be uncommented to produce the final version of this child document.

%%%%%%%%%%%%%%%%%%%%%%%%%%%%%%%%%%%%%%%%%%%%%%%%%%%%%%%%%%%%%%%%%%%%%%%%%%%%%%%%
\subsection{Forwarding}
\label{sec:forward}

Different versions of the main or child documents
using compilation flags as described in \secref{sec:flags}
can be (permanently) stored in different files
for convenient compilation, viewing and distribution.
To this end, the package defines a command
to pass on compilation to a different file:

%%%%%%%%%%%%%%%%%%%%%%%%%%%%%%%%%%%%%%%%
\DescribeMacro{\childdocforward}
The command |\childdocforward| redirects processing to
another source file:
%
\begin{center}
\begin{tabular}{l}
|\input{childdoc.def}|\\
|\childdocforward[|\textit{main}|]{|\textit{dest}|}|\\
\end{tabular}
\end{center}
%
The argument \textit{dest} is the destination file
(without extension).
It should be the main file or one of the child files.
Note that further \textsf{childdoc} directives
such as |\childdocof| and |\childdocforward|
in the indicated file will be processed in this form.
The optional argument \textit{main}
passes on directly to the main file \textit{main}
while pretending to compile the child \textit{dest}.
This form behaves as if \textit{dest}
issues |\childdocof{|\textit{main}|}| right away,
and no further \textsf{childdoc} directives will be processed.

%%%%%%%%%%%%%%%%%%%%%%%%%%%%%%%%%%%%%%%%
\DescribeMacro{\...prefix}
In the alternative form |\childdocforwardprefix|,
%
\begin{center}
\begin{tabular}{l}
|\input{childdoc.def}|\\
|\childdocforwardprefix[|\textit{main}|]{|\textit{prefix}|}{|\textit{dest}|}|
\end{tabular}
\end{center}
%
the destination file is determined by a pattern
depending on the current file:
To make this work, the current file must be called
`{\textit{prefix}\hspace{0.2em}\textit{suffix}}'
with \textit{prefix} matching precisely the argument.
Processing is then passed on to the file
`{\textit{dest}\hspace{0.2em}\textit{suffix}}'.
Surely, the same effect is achieved by
directly specifying the
argument `{\textit{dest}\hspace{0.2em}\textit{suffix}}'
in the first form.
However, that requires to set up a different file
for each child. With the alternative form of the command
all these files can have exactly the same content
which simplifies setting them up and maintaining them.

For example, the following file |draft.tex|
with a compilation flag |\version| as described in \secref{sec:flags}
compiles the main document as a draft:
%
\begin{center}
\begin{tabular}{l}
|\def\version{draft}|\\
|\input{childdoc.def}|\\
|\childdocforward{|\textit{main}|}|
\end{tabular}
\end{center}
%
Likewise, the following files |final|\textit{nn}|.tex|
compile the final version of the child document
|child|\textit{nn}|.tex|:
%
\begin{center}
\begin{tabular}{l}
|\def\version{final}|\\
|\input{childdoc.def}|\\
|\childdocforwardprefix{final}{child}|
\end{tabular}
\end{center}
%

Note that when several versions of a main file and/or of each child file
are to be generated, it may be convenient to set up a |Makefile| or
shell script to automatise the process.

%%%%%%%%%%%%%%%%%%%%%%%%%%%%%%%%%%%%%%%%%%%%%%%%%%%%%%%%%%%%%%%%%%%%%%%%%%%%%%%%
\subsection{Command Line Processing}
\label{sec:commandline}

The effect of redirection files can also be achieved by invoking
the \LaTeX{} compiler with a more elaborate command line.
Most conveniently this should be done as part
of a shell script or a |Makefile|.

When using \textsf{childdoc} in the main file, the following
command lines effectively perform a redirection
(note that depending on the shell being used,
backslashes may have to be doubled: `|\|' $\to$ `|\\|'):
%
\begin{center}
|... -jobname "|\textit{target}|" |\\|"|[\textit{flags}]%
|\input{childdoc.def}\childdocforward[|\textit{main}|]{|\textit{dest}|}"|
\end{center}
%
Here \textit{target} is the name of the output file,
\textit{main} is the name of the main file
and \textit{dest} is the name of the main or child file to be processed
(all filenames without extensions).
The optional argument \textit{main} can be omitted
if \textit{main} matches \textit{dest}.
Optionally, compilation \textit{flags} can be defined via |\def| commands.
This command line makes the \TeX{} engine believe
it is compiling the file \textit{target}
whose content is specified as the latter parameter.
The provided code then forwards the processing to
\textit{main} or \textit{dest} as described in \secref{sec:forward}.

%%%%%%%%%%%%%%%%%%%%%%%%%%%%%%%%%%%%%%%%%%%%%%%%%%%%%%%%%%%%%%%%%%%%%%%%%%%%%%%%
\subsection{Include by Input}
\label{sec:input}

Including child documents by |\include| has some restrictions by design.
Most notably, the content of a child document always occupies
its own set of pages; pages cannot be shared between child documents.
Usually, this behaviour makes perfect sense
because each child document contain an essential part of the document.
However, in some situations it may be desirable to compose
a document from a collection of parts
without having mandatory page breaks between then.
For this case, the package
provides a mechanism to include parts
by |\input| which can also be processed individually.
However, by construction this mechanism
requires manual handling of the content to be output.

%%%%%%%%%%%%%%%%%%%%%%%%%%%%%%%%%%%%%%%%
\DescribeMacro{\ifchilddocmanual}
The main file should be prepared as usual, see \secref{sec:include}.
However, the document body must make a distinction
between processing of an individual part and of the main document, e.g.:
%
\begin{center}
\begin{tabular}{l}
|\ifchilddocmanual|\\
|\input{\childdocname}|\\
|\||else|\\
\textit{document body with }|\input{|\textit{part}|}|\\
|\||fi|
\end{tabular}
\end{center}
%
The conditional |\ifchilddocmanual| is true whenever
a part to be included by |\input| is being compiled,
and the name of the part is stored in |\childdocname|.

%%%%%%%%%%%%%%%%%%%%%%%%%%%%%%%%%%%%%%%%
\DescribeMacro{\childdocby}
Each part to be included by |\input| should start with:
%
\begin{center}
\begin{tabular}{l}
|\input{childdoc.def}|\\
|\childdocby{|\textit{main}|}|\\
\end{tabular}
\end{center}
%
The directive |\childdocby| is similar to |\childdocof|
described in \secref{sec:include},
but the subsequent selection of content must be done manually.
To that end, both |\ifchilddoc| and |\ifchilddocmanual|
will be true upon processing of a part,
and the name of the part is stored in |\childdocname|.
Note that |\jobname| will be set to the filename of the current part
so that each part receives an individual |.aux| file
that does not interfere with the |.aux| file(s) of the main document.
This behaviour can be altered by the alternative form
|\childdocby[*]{|\textit{main}|}| (with a non-empty optional argument)
which uses the |.aux| file of the main document
by setting |\jobname| to \textit{main}.

%%%%%%%%%%%%%%%%%%%%%%%%%%%%%%%%%%%%%%%%%%%%%%%%%%%%%%%%%%%%%%%%%%%%%%%%%%%%%%%%
\subsection{Driver Development}
\label{sec:driver}

The \textsf{childdoc} mechanism can also be use for the development
of definition files such as \LaTeX{} styles or classes.
This case differs from the above setup with multiple parts
included by |\include| in that no |\includeonly| should be invoked.
This can be achieved by starting the include file
(before |\ProvidesPackage|) with:
%
\begin{center}
\begin{tabular}{l}
|\input{childdoc.def}|\\
|\childdocforward{|\textit{main}|}|\\
\end{tabular}
\end{center}
%
or alternatively with:
%
\begin{center}
\begin{tabular}{l}
|\input{childdoc.def}|\\
|\childdocby{|\textit{main}|}|\\
\end{tabular}
\end{center}
%
Both forms have slightly different effects as described above.
The main file is prepared as usual, see \secref{sec:include}.

%%%%%%%%%%%%%%%%%%%%%%%%%%%%%%%%%%%%%%%%%%%%%%%%%%%%%%%%%%%%%%%%%%%%%%%%%%%%%%%%
\subsection{Legacy Detection}
\label{sec:detection}

The directive |\childdocmain| in the main file can detect
whether the complete document or merely a child is to be compiled
even without using the directive |\childdocof|.
This method is deprecated because it is less robust
and there is no compelling reason to use it;
it is merely provided for backward compatibility
and it may be removed in future versions.

If the detection mechanism is to be used,
it is mandatory to correctly specify
the filename of the main file as the argument of |\childdocmain|:
%
\begin{center}
\begin{tabular}{l}
|\input{childdoc.def}|\\
|\childdocmain{|\textit{main}|}|\\
\end{tabular}
\end{center}
%
If |\jobname| does not match the argument \textit{main} of |\childdocmain|,
it is assumed that |\jobname| points to the child file to be compiled.
When using |\childdocmain| with the main file specified as argument,
it suffices to start a child file
with just |\input{|\textit{main}|}|
without loading of the package and using |\childdocof|.
If instead all processing is done
with the appropriate \textsf{childdoc} directives,
the argument of \textit{main} of |\childdocmain| can be empty.

An alternative version of the command line processing described
in \secref{sec:commandline} using the detection mechanism reads:
%
\begin{center}
|... -jobname "|\textit{target}|" "|[\textit{flags}]%
[|\def\jobname{|\textit{dest}|}|]|\input{|\textit{main}|}"|
\end{center}

%%%%%%%%%%%%%%%%%%%%%%%%%%%%%%%%%%%%%%%%%%%%%%%%%%%%%%%%%%%%%%%%%%%%%%%%%%%%%%%%
\subsection{Manual Code}
\label{sec:manual}

In case one cannot be certain whether the definitions file |childdoc.def|
is installed on the target \TeX{} distribution
and one prefers not to ship it,
it is conceivable to paste a few relevant commands into the sources.

To that end, drop all statements |\input{childdoc.def}|
and perform the replacements as outlined below.
Instead of |\childdocmain{|\textit{main}|}| add the following code
to the top of the main file:
%
\begin{center}
\begin{tabular}{l}
|\||ifdefined\childdocname\endinput\||fi\newif\ifchilddoc|\\
|\edef\childdocname{\scantokens\expandafter{\jobname\noexpand}}|\\
|\def\childdocmain{|\textit{main}|}\||ifx\childdocmain\childdocname\||else|\\
|\childdoctrue\includeonly{\childdocname}\let\jobname\childdocmain\||fi|\\
\end{tabular}
\end{center}
%
Instead of |\childdocof{|\textit{main}|}| just include the main file
at the top of each child file:
%
\begin{center}
|\input{|\textit{main}|}|
\end{center}
%
A simple redirection |\childdocforward{|\textit{dest}|}| is achieved by:
%
\begin{center}
|\def\jobname{|\textit{dest}|}\input{\jobname}|
\end{center}
%
The redirection with prefix
|\childdocforwardprefix[|\textit{prefix}|]{|\textit{dest}|}|
is accomplished by:
%
\begin{center}
\begin{tabular}{l}
|{\edef\jobname{\scantokens\expandafter{\jobname\noexpand}}|\\
|\def\redirectjob |\textit{prefix}|#1~~~{\gdef\jobname{|\textit{dest}|#1}}|\\
|\expandafter\redirectjob\jobname~~~}\input{\jobname}|
\end{tabular}
\end{center}

In an alternative approach,
child documents can be compiled by a specific command line
without additional code or specific definitions:
%
\begin{center}
|... -jobname "|\textit{target}|" "|[\textit{flags}]%
|\includeonly{|\textit{dest}|}\input{|\textit{main}|}"|
\end{center}
%

%%%%%%%%%%%%%%%%%%%%%%%%%%%%%%%%%%%%%%%%%%%%%%%%%%%%%%%%%%%%%%%%%%%%%%%%%%%%%%%%
%%%%%%%%%%%%%%%%%%%%%%%%%%%%%%%%%%%%%%%%%%%%%%%%%%%%%%%%%%%%%%%%%%%%%%%%%%%%%%%%
\section{Information}

%%%%%%%%%%%%%%%%%%%%%%%%%%%%%%%%%%%%%%%%%%%%%%%%%%%%%%%%%%%%%%%%%%%%%%%%%%%%%%%%
\subsection{Copyright}

Copyright \copyright{} 2017--2018 Niklas Beisert

This work may be distributed and/or modified under the
conditions of the \LaTeX{} Project Public License, either version 1.3
of this license or (at your option) any later version.
The latest version of this license is in
  \url{http://www.latex-project.org/lppl.txt}
and version 1.3 or later is part of all distributions of \LaTeX{}
version 2005/12/01 or later.

This work has the LPPL maintenance status `maintained'.

The Current Maintainer of this work is Niklas Beisert.

This work consists of the files |README.txt|, |childdoc.ins| and |childdoc.dtx|
as well as the derived files |childdoc.def|, |cdocsamp.tex|
with |cdocsch1.tex|, |cdocsch2.tex|, |cdocspt3.tex|, |cdocspt4.tex|,
|cdocsdrf.tex|, |cdocsfn1.tex|, |cdocsfn2.tex|
as well as |childdoc.pdf|.

%%%%%%%%%%%%%%%%%%%%%%%%%%%%%%%%%%%%%%%%%%%%%%%%%%%%%%%%%%%%%%%%%%%%%%%%%%%%%%%%
\subsection{Files and Installation}

The package consists of the files:
%
\begin{center}
\begin{tabular}{ll}
    |README.txt|   & readme file \\
    |childdoc.ins| & installation file \\
    |childdoc.dtx| & source file \\
    |childdoc.def| & definition file \\
    |cdocsamp.tex| & sample main file \\
    |cdocsch1.tex| & sample include file \\
    |cdocsch2.tex| & sample include file \\
    |cdocspt3.tex| & sample part file \\
    |cdocspt4.tex| & sample part file \\
    |cdocsdrf.tex| & sample redirection file \\
    |cdocsfn1.tex| & sample redirection file \\
    |cdocsfn2.tex| & sample redirection file \\
    |childdoc.pdf| & manual
\end{tabular}
\end{center}
%
The distribution consists of the files
|README.txt|, |childdoc.ins| and |childdoc.dtx|.
%
\begin{itemize}
\item
Run (pdf)\LaTeX{} on |childdoc.dtx|
to compile the manual |childdoc.pdf| (this file).
\item
Run \LaTeX{} on |childdoc.ins| to create the definitions file |childdoc.def|
and the sample |cdocsamp.tex| with include files
|cdocsch1.tex|, |cdocsch2.tex|, |cdocspt3.tex|, |cdocspt4.tex|,
|cdocsdrf.tex|, |cdocsfn1.tex|, |cdocsfn2.tex|.
Then copy the file |childdoc.def| to an appropriate directory of your \LaTeX{}
distribution, e.g.\ \textit{texmf-root}|/tex/latex/childdoc|.
\end{itemize}

%%%%%%%%%%%%%%%%%%%%%%%%%%%%%%%%%%%%%%%%%%%%%%%%%%%%%%%%%%%%%%%%%%%%%%%%%%%%%%%%
\subsection{Related CTAN Packages}

There are several other packages which offer a similar functionality:
%
\begin{itemize}
\item
The packages
\href{http://ctan.org/pkg/docmute}{\textsf{docmute}},
\href{http://ctan.org/pkg/includex}{\textsf{includex}} and
\href{http://ctan.org/pkg/standalone}{\textsf{standalone}}
provide commands to include only the document body of
a child file thus allowing both files to be compiled individually.
\item
The packages \href{http://ctan.org/pkg/subdocs}{\textsf{subdocs}}
and \href{http://ctan.org/pkg/subfiles}{\textsf{subfiles}}
provide structures in which the main and child documents can be
encapsulated and allowing them to be compiled individually.
The inclusion mechanism is different from the conventional |\include|.
\item
The package \href{http://ctan.org/pkg/combine}{\textsf{combine}}
is an elaborate solution to combine several documents into one.
\end{itemize}
%
See also the CTAN topic \href{http://ctan.org/topic/subdocs}{\textsf{subdocs}}
for further related packages.
The present package differs from the above solutions in that
a document structure constructed with the conventional |\include| mechanism
just needs two extra commands at the top of every file
such that all constituent files can be compiled individually.

%%%%%%%%%%%%%%%%%%%%%%%%%%%%%%%%%%%%%%%%%%%%%%%%%%%%%%%%%%%%%%%%%%%%%%%%%%%%%%%%
%\subsection{Feature Suggestions}
%
%The following is a list of features which may be useful for future
%versions of this package:
%%
%\begin{itemize}
%\item
%\ldots
%\end{itemize}

%%%%%%%%%%%%%%%%%%%%%%%%%%%%%%%%%%%%%%%%%%%%%%%%%%%%%%%%%%%%%%%%%%%%%%%%%%%%%%%%
\subsection{Revision History}

%%%%%%%%%%%%%%%%%%%%%%%%%%%%%%%%%%%%%%%%
\paragraph{v2.0:} 2018/12/30

\begin{itemize}
\item
immediate forward processing
\item
added |\childdocby| mechanism
\item
manual restructured
\end{itemize}

%%%%%%%%%%%%%%%%%%%%%%%%%%%%%%%%%%%%%%%%
\paragraph{v1.6:} 2018/01/17

\begin{itemize}
\item
application for development of include files
\item
corrections to manual
\end{itemize}

%%%%%%%%%%%%%%%%%%%%%%%%%%%%%%%%%%%%%%%%
\paragraph{v1.5:} 2017/05/21

\begin{itemize}
\item
more complete structuring introduced
\item
|\childdocof| introduced
\item
|\childdoc| renamed to |\childdocmain|
\item
|\childredirect| renamed to |\childdocforward| and |\childdocforwardprefix|
and functionality expanded
\end{itemize}

%%%%%%%%%%%%%%%%%%%%%%%%%%%%%%%%%%%%%%%%
\paragraph{v1.0:} 2017/04/27

\begin{itemize}
\item
manual and install package
\item
first version published on CTAN
\end{itemize}

%%%%%%%%%%%%%%%%%%%%%%%%%%%%%%%%%%%%%%%%
\paragraph{v0.6:} 2017/04/26

\begin{itemize}
\item
redirection mechanism added
\end{itemize}

%%%%%%%%%%%%%%%%%%%%%%%%%%%%%%%%%%%%%%%%
\paragraph{v0.5:} 2017/04/26

\begin{itemize}
\item
functionality in definition file
\end{itemize}


%%%%%%%%%%%%%%%%%%%%%%%%%%%%%%%%%%%%%%%%%%%%%%%%%%%%%%%%%%%%%%%%%%%%%%%%%%%%%%%%
%%%%%%%%%%%%%%%%%%%%%%%%%%%%%%%%%%%%%%%%%%%%%%%%%%%%%%%%%%%%%%%%%%%%%%%%%%%%%%%%
%%%%%%%%%%%%%%%%%%%%%%%%%%%%%%%%%%%%%%%%%%%%%%%%%%%%%%%%%%%%%%%%%%%%%%%%%%%%%%%%
\appendix

\settowidth\MacroIndent{\rmfamily\scriptsize 000\ }

 \DocInput{childdoc.dtx}

\end{document}
%</driver>
% \fi
%
% %%%%%%%%%%%%%%%%%%%%%%%%%%%%%%%%%%%%%%%%%%%%%%%%%%%%%%%%%%%%%%%%%%%%%%%%%%%%%%
% %%%%%%%%%%%%%%%%%%%%%%%%%%%%%%%%%%%%%%%%%%%%%%%%%%%%%%%%%%%%%%%%%%%%%%%%%%%%%%
% \section{Sample}
%\iffalse
%<*samplemain>
%\fi
%
% The following presents a sample document
% with two chapters, two parts, a title page,
% a compile flag as well as three forwarding files to set the flag.
% It consists of eight |.tex| files:
% \begin{center}
% \begin{tabular}{ll}
% |cdocsamp.tex|&main file\\
% |cdocsch1.tex|&include file for chapter 1\\
% |cdocsch2.tex|&include file for chapter 2\\
% |cdocspt3.tex|&include file for part 3\\
% |cdocspt4.tex|&include file for part 4\\
% |cdocsdrf.tex|&forwarding file for main file in draft mode\\
% |cdocsfi1.tex|&forwarding file for final version of chapter 1\\
% |cdocsfi2.tex|&forwarding file for final version of chapter 2\\
% \end{tabular}
% \end{center}
% Each of the eight files can be compiled directly by the \LaTeX{} compiler.
%
% %%%%%%%%%%%%%%%%%%%%%%%%%%%%%%%%%%%%%%
% \paragraph{Main File.}
%
% The main file is called |cdocsamp.tex|.
%
% Load the \textsf{childdoc} definitions and
% declare the filename for the main document:
%    \begin{macrocode}
\input{childdoc.def}
\childdocmain{}
%    \end{macrocode}

% Optional override for |\version| flag:
%    \begin{macrocode}
%%\ifchilddoc\else\providecommand{\version}{draft}\fi
%    \end{macrocode}

% Define the default values for the |\version| flag
% (|final| for the main file and |draft| for childs):
%    \begin{macrocode}
\ifchilddoc
\providecommand{\version}{draft}
\else
\providecommand{\version}{final}
\fi
%    \end{macrocode}

% Load the standard document class:
%    \begin{macrocode}
\documentclass[12pt]{article}
%    \end{macrocode}

% Start the document body:
%    \begin{macrocode}
\begin{document}
%    \end{macrocode}

% Declare a title page.
% Print title, part of document being processed and version flag:
%    \begin{macrocode}
\addtocounter{page}{-1}
\begin{center}
{\LARGE\bfseries{}childdoc example\par}
\vspace{1cm}
\ifchilddoc
\ifchilddocmanual part\else chapter\fi:
`\childdocname' of `\childdocjob'\par
\else
main document: `\childdocjob'\par
\fi
version: \version\par
\end{center}
\newpage
%    \end{macrocode}

% Manually include selected file,
% otherwise process as usual:
%    \begin{macrocode}
\ifchilddocmanual
\section*{part `\childdocname'}
\input{\childdocname}
\else
%    \end{macrocode}

% Include the two chapters:
%    \begin{macrocode}
\include{cdocsch1}
\include{cdocsch2}
%    \end{macrocode}

% Include the two parts unless only chapters should be displayed:
%    \begin{macrocode}
\ifchilddoc\else
\section{part three}
\input{cdocspt3}
\section{part four}
\input{cdocspt4}
\fi
%    \end{macrocode}

% Process as usual until here:
%    \begin{macrocode}
\fi
%    \end{macrocode}

% End of document body:
%    \begin{macrocode}
\end{document}
%    \end{macrocode}
%\iffalse
%</samplemain>
%\fi
%
% %%%%%%%%%%%%%%%%%%%%%%%%%%%%%%%%%%%%%%
% \paragraph{Chapter Include Files.}
%
% The include files are called |cdocsch1.tex| and |cdocsch2.tex|.
%
%\iffalse
%<*samplechap1|samplechap2>
%\fi

% Optional override for |\version| flag:
%    \begin{macrocode}
%%\providecommand{\version}{final}
%    \end{macrocode}

% Include the main document:
%    \begin{macrocode}
\input{childdoc.def}
\childdocof{cdocsamp}
%    \end{macrocode}

%\iffalse
%</samplechap1|samplechap2>
%\fi
%
%\iffalse
%<*samplechap1>
%\fi
% Some text for chapter 1:
%    \begin{macrocode}
\section{one}
some text in chapter one
%    \end{macrocode}

%\iffalse
%</samplechap1>
%\fi
% Some text for chapter 2:
%\iffalse
%<*samplechap2>
%\fi
%    \begin{macrocode}
\section{two}
more text in chapter two
%    \end{macrocode}

%\iffalse
%</samplechap2>
%\fi
%
% %%%%%%%%%%%%%%%%%%%%%%%%%%%%%%%%%%%%%%
% \paragraph{Part Include Files.}
%
% The include files are called |cdocspt3.tex| and |cdocspt4.tex|.
%
%\iffalse
%<*samplepart3|samplepart4>
%\fi

% Optional override for |\version| flag:
%    \begin{macrocode}
%%\providecommand{\version}{final}
%    \end{macrocode}

% Include the main document:
%    \begin{macrocode}
\input{childdoc.def}
\childdocby{cdocsamp}
%    \end{macrocode}

%\iffalse
%</samplepart3|samplepart4>
%\fi
%
%\iffalse
%<*samplepart3>
%\fi
% Some text for part 3:
%    \begin{macrocode}
some text in part three
%    \end{macrocode}

%\iffalse
%</samplepart3>
%\fi
% Some text for part 4:
%\iffalse
%<*samplepart4>
%\fi
%    \begin{macrocode}
more text in part four
%    \end{macrocode}

%\iffalse
%</samplepart4>
%\fi
%
% %%%%%%%%%%%%%%%%%%%%%%%%%%%%%%%%%%%%%%
% \paragraph{Forwarding for a Complete Draft.}
%
% The following forwarding file |cdocsdrf.tex|
% compiles the main document in draft mode:
%\iffalse
%<*sampledraft>
%\fi
%    \begin{macrocode}
\def\version{draft}
\input{childdoc.def}
\childdocforward{cdocsamp}
%    \end{macrocode}

%\iffalse
%</sampledraft>
%\fi
%
% %%%%%%%%%%%%%%%%%%%%%%%%%%%%%%%%%%%%%%
% \paragraph{Forwarding for Final Version of the Chapters.}
%
% The following forwarding files |cdocsfn1.tex| and |cdocsfn2.tex|
% (with identical content)
% compile the final versions of the child documents
% |cdocsch1.tex| and |cdocsch2.tex|, respectively:
%\iffalse
%<*samplefinal>
%\fi
%    \begin{macrocode}
\def\version{final}
\input{childdoc.def}
\childdocforwardprefix[cdocsamp]{cdocsfn}{cdocsch}
%    \end{macrocode}

%\iffalse
%</samplefinal>
%\fi
%
% %%%%%%%%%%%%%%%%%%%%%%%%%%%%%%%%%%%%%%
% \paragraph{Command Line Processing.}
%
% The following three command lines generate the output files
% |cdocscld|, |cdocscl1| and |cdocscl2|
% which should be identical to
% |cdocsdrf|, |cdocsch1| and |cdocsfn2|, respectively:
% \begin{center}
% \begin{tabular}{l}
% |latex -jobname cdocscld \|\\
% |  "\def\version{draft}\input{childdoc.def}\childdocforward{cdocsamp}"|\\
% |latex -jobname cdocscl1 \|\\
% |  "\input{childdoc.def}\childdocforward[cdocsamp]{cdocsch1}"|\\
% |latex -jobname cdocscl2 \|\\
% |  "\def\version{final}\input{childdoc.def}\childdocforward{cdocsch2}"|
% \end{tabular}
% \end{center}
% Note that the trailing backslash on each first line
% merely continues the input to the second line
% (for convenient cut ant paste).
% Furthermore, the command |latex| can be replaced by any
% of its alternative versions such as |pdflatex|.
%
% %%%%%%%%%%%%%%%%%%%%%%%%%%%%%%%%%%%%%%%%%%%%%%%%%%%%%%%%%%%%%%%%%%%%%%%%%%%%%%
% %%%%%%%%%%%%%%%%%%%%%%%%%%%%%%%%%%%%%%%%%%%%%%%%%%%%%%%%%%%%%%%%%%%%%%%%%%%%%%
% \section{Implementation}
%\iffalse
%<*package>
%\fi
%
% This section describes the definitions file |childdoc.def|.

% The definitions cannot be loaded using |\usepackage| or |\RequirePackage|
% which has a mechanism to prevent loading a style file more than once.
% When loading the definitions by means of |\input|
% multiple instances have to be prevented manually:
%\iffalse
%This code needs to be before the `\ProvidesFile' directive
%which is defined at the beginning of this file.
%Therefore it is also placed there and commented out here.
%</package>
%<*discard>
%\fi
%    \begin{macrocode}
\ifdefined\childdocmain\endinput\fi
%    \end{macrocode}
%\iffalse
%</discard>
%<*package>
%\fi
%
% \macro{\ifchilddoc}
% \macro{\ifchilddocmanual}
% The conditional |\ifchilddoc| tells whether a
% child (true) or main (false) document is being compiled.
% The conditional |\ifchilddocmanual| tells whether
% the |\includeonly| mechanism is used (false) or
% the selection of child files must be performed manually (true).
% The definitions initialise to false:
%    \begin{macrocode}
\newif\ifchilddoc
\newif\ifchilddocmanual
%    \end{macrocode}

% \macro{\childdocname}
% \macro{\childdocjob}
% The macro |\childdocname| stores the name of the main document
% to be compiled. The macro |\childdocjob| stores the name of
% the document on which the \LaTeX{} compiler was originally invoked.
% The content of |\jobname| cannot be compared
% to filenames specified in the source due to different catcodes.
% The following code rescans |\jobname|, stores the result
% in |\childdocname| and saves a copy in |\childdocjob|:
%    \begin{macrocode}
\edef\childdocname{\scantokens\expandafter{\jobname\noexpand}}
\let\childdocjob\childdocname
%    \end{macrocode}

% \macro{\childdocdisable}
% The macro |\childdocdisable| prevents the main file
% from being processed more than once.
% At this stage, the main document command |\childdocmain|
% is assumed to be called once again where it should do nothing.
% Any subsequent call to it should prevent
% a secondary processing of the main document
% It overwrites the forwarding commands
% |\childdocof| and |\childdocforward|
% with empty macros to prevent further inclusions of the main document:
%    \begin{macrocode}
\newcommand{\childdocdisable}
{
  \renewcommand{\childdocmain}[1]{\renewcommand{\childdocmain}[1]{\endinput}}
  \renewcommand{\childdocof}[1]{}
  \renewcommand{\childdocby}[2][]{}
  \renewcommand{\childdocforward}[2][]{}
  \renewcommand{\childdocdisable}{}
}
%    \end{macrocode}

% \macro{\childdocmain}
% The macro |\childdocmain| is to be called at the top of the main file
% with nothing or the main filename (without extension) as argument.
% First, it breaks loops.
% If the argument is not empty and does not match |\childdocname|
% (which is set by the first inclusion of |childdoc.def|),
% |\ifchilddoc| is set to true, |\includeonly| is applied to the child file
% and |\jobname| is set to the main file
% (for proper handling of |.aux| files):
%    \begin{macrocode}
\newcommand{\childdocmain}[1]
{
  \childdocdisable\childdocmain{}
  \if?#1?\else
    \begingroup
      \def\childdoctmp{#1}
      \ifx\childdoctmp\childdocname
        \def\childdoctmp{}
      \else
        \def\childdoctmp
        {
          \childdoctrue
          \includeonly{\childdocname}
          \def\childdocjob{#1}
          \def\jobname{#1}
        }
      \fi
      \expandafter
    \endgroup
    \childdoctmp
  \fi
}
%    \end{macrocode}

% \macro{\childdocof}
% The command |\childdocof| redirects
% compilation to the main file |#1|.
%    \begin{macrocode}
\newcommand{\childdocof}[1]
{
  \childdocdisable
  \childdoctrue
  \includeonly{\childdocname}
  \def\jobname{#1}
  \def\childdocjob{#1}
  \input{#1}
}
%    \end{macrocode}

% \macro{\childdocby}
% The command |\childdocby| ....
%    \begin{macrocode}
\newcommand{\childdocby}[2][]
{
  \childdocdisable
  \childdoctrue
  \childdocmanualtrue
  \if?#1?\else
    \def\jobname{#2}
  \fi
  \def\childdocjob{#2}
  \input{#2}
  \endinput
}
%    \end{macrocode}

% \macro{\childdocforward}
% The command |\childdocforward| redirects
% compilation to the main file or
% (if the optional argument is given) a child file.
% Parameters are set as if the main file
% or a child file starting with |\childdocof| was compiled.
% Then compilation is handed over to the main file:
%    \begin{macrocode}
\newcommand{\childdocforward}[2][]
{
  \begingroup
    \if?#1?
      \def\childdoctmp
      {
        \def\childdocname{#2}
        \def\childdocjob{#2}
        \def\jobname{#2}
        \input{#2}
        \endinput
      }
    \else
      \def\childdoctmp
      {
        \childdocdisable
        \def\childdocname{#2}
        \childdoctrue
        \includeonly{#2}
        \def\childdocjob{#1}
        \def\jobname{#1}
        \input{#1}
        \endinput
      }
    \fi
    \expandafter
  \endgroup
  \childdoctmp
}
%    \end{macrocode}

% \macro{\childdocforwardprefix}
% The command |\childdocforwardprefix| redirects
% compilation to the main or a child file by means of a pattern.
% The prefix |#1| in the current filename is replaced by |#2|
% and the suffix of the current filename is kept
% (it is assumed that the filename does not contain the substring `|~~~|'
% which is used as a delimiter).
% Compilation is handed over to the new file by |\childdocforward|:
%    \begin{macrocode}
\newcommand{\childdocforwardprefix}[3][]
{
  \begingroup
    \def\childdocextract #2##1~~~{\def\childdoctmp{\childdocforward[#1]{#3##1}}}
    \expandafter\childdocextract\childdocname~~~
    \expandafter
  \endgroup
  \childdoctmp
}
%    \end{macrocode}

% \macro{\childdoc}
% The deprecated macro |\childdoc| is a legacy version of |\childdocmain|:
%    \begin{macrocode}
\newcommand{\childdoc}{\childdocmain}
%    \end{macrocode}

% \macro{\childdocredirect}
% The deprecated macro |\childdocredirect| is a legacy version
% of |\childdocforward| and |\childdocforwardprefix|:
%    \begin{macrocode}
\newcommand{\childdocredirect}[2][]
{
  \begingroup
    \if?#1?
      \def\childdoctmp{\childdocforward{#2}}
    \else
      \def\childdoctmp{\childdocforwardprefix{#1}{#2}}
    \fi
    \expandafter
  \endgroup
  \childdoctmp
}
%    \end{macrocode}

%\iffalse
%</package>
%\fi
%
\endinput

\childdocof{cdocsamp}
%    \end{macrocode}

%\iffalse
%</samplechap1|samplechap2>
%\fi
%
%\iffalse
%<*samplechap1>
%\fi
% Some text for chapter 1:
%    \begin{macrocode}
\section{one}
some text in chapter one
%    \end{macrocode}

%\iffalse
%</samplechap1>
%\fi
% Some text for chapter 2:
%\iffalse
%<*samplechap2>
%\fi
%    \begin{macrocode}
\section{two}
more text in chapter two
%    \end{macrocode}

%\iffalse
%</samplechap2>
%\fi
%
% %%%%%%%%%%%%%%%%%%%%%%%%%%%%%%%%%%%%%%
% \paragraph{Part Include Files.}
%
% The include files are called |cdocspt3.tex| and |cdocspt4.tex|.
%
%\iffalse
%<*samplepart3|samplepart4>
%\fi

% Optional override for |\version| flag:
%    \begin{macrocode}
%%\providecommand{\version}{final}
%    \end{macrocode}

% Include the main document:
%    \begin{macrocode}
% \iffalse
%
% childdoc.dtx Copyright (C) 2017-2018 Niklas Beisert
%
% This work may be distributed and/or modified under the
% conditions of the LaTeX Project Public License, either version 1.3
% of this license or (at your option) any later version.
% The latest version of this license is in
%   http://www.latex-project.org/lppl.txt
% and version 1.3 or later is part of all distributions of LaTeX
% version 2005/12/01 or later.
%
% This work has the LPPL maintenance status `maintained'.
%
% The Current Maintainer of this work is Niklas Beisert.
%
% This work consists of the files childdoc.dtx and childdoc.ins
% and the derived files childdoc.def and cdocsamp.tex with
% cdocsch1.tex, cdocsch2.tex, cdocsdrf.tex, cdocsfn1.tex, cdocsfn2.tex.
%
%<package>\ifdefined\childdocmain\endinput\fi
%<package>\ProvidesFile{childdoc.def}[2018/12/30 v2.0 child document driver]
%<samplemain>\ProvidesFile{cdocsamp.tex}[2018/12/30 v2.0 sample for childdoc]
%<*driver>
%\ProvidesFile{childdoc.drv}[2018/12/30 v2.0 childdoc reference manual file]
\PassOptionsToClass{10pt,a4paper}{article}
\documentclass{ltxdoc}

\usepackage[margin=35mm]{geometry}
\usepackage{hyperref}
\usepackage{hyperxmp}
\usepackage[usenames]{color}

\hypersetup{colorlinks=true}
\hypersetup{pdfstartview=FitH}
\hypersetup{pdfpagemode=UseNone}
\hypersetup{pdfsource={}}
\hypersetup{pdflang={en-UK}}
\hypersetup{pdfcopyright={Copyright 2017-2018 Niklas Beisert.
  This work may be distributed and/or modified under the
  conditions of the LaTeX Project Public License, either version 1.3
  of this license or (at your option) any later version.}}
\hypersetup{pdflicenseurl={http://www.latex-project.org/lppl.txt}}
\hypersetup{pdfcontactaddress={ETH Zurich, ITP, HIT K,
  Wolfgang-Pauli-Strasse 27}}
\hypersetup{pdfcontactpostcode={8093}}
\hypersetup{pdfcontactcity={Zurich}}
\hypersetup{pdfcontactcountry={Switzerland}}
\hypersetup{pdfcontactemail={nbeisert@itp.phys.ethz.ch}}
\hypersetup{pdfcontacturl={http://people.phys.ethz.ch/\xmptilde nbeisert/}}

\newcommand{\secref}[1]{\hyperref[#1]{section \ref*{#1}}}

\parskip1ex
\parindent0pt
\let\olditemize\itemize
\def\itemize{\olditemize\parskip0pt}

\begin{document}

\title{The \textsf{childdoc} Package}
\hypersetup{pdftitle={The childdoc Package}}
\author{Niklas Beisert\\[2ex]
  Institut f\"ur Theoretische Physik\\
  Eidgen\"ossische Technische Hochschule Z\"urich\\
  Wolfgang-Pauli-Strasse 27, 8093 Z\"urich, Switzerland\\[1ex]
  \href{mailto:nbeisert@itp.phys.ethz.ch}
  {\texttt{nbeisert@itp.phys.ethz.ch}}}
\hypersetup{pdfauthor={Niklas Beisert}}
\hypersetup{pdfsubject={Manual for the LaTeX2e Package childdoc}}
\date{30 December 2018, \textsf{v2.0}}
\maketitle

\begin{abstract}\noindent
\textsf{childdoc} is a \LaTeXe{} package
that enables the direct compilation
of document sections included by |\include|
to individual files.
\end{abstract}

\begingroup
\parskip0ex
\tableofcontents
\endgroup

%%%%%%%%%%%%%%%%%%%%%%%%%%%%%%%%%%%%%%%%%%%%%%%%%%%%%%%%%%%%%%%%%%%%%%%%%%%%%%%%
%%%%%%%%%%%%%%%%%%%%%%%%%%%%%%%%%%%%%%%%%%%%%%%%%%%%%%%%%%%%%%%%%%%%%%%%%%%%%%%%
\section{Introduction}

\LaTeX{} provides a mechanism to structure a large document (such as a book)
into a main file and several child files (containing the chapters)
using the |\include| command.
This mechanism is beneficial for documents
which span hundreds of pages in order to
make the source file(s) more manageable.
Moreover, compilation can be restricted to
selected child files by means of the |\includeonly| command.
The latter feature can be used to reduce the compilation time while editing
(this was significantly more useful in the earlier days of \LaTeX{})
or to generate a smaller document which is easier to navigate.
Another application of |\includeonly| is to generate
documents consisting of selected parts of the complete document.

However, there are a few drawbacks of the plain |\include| mechanism:
\begin{itemize}
\item
The child files cannot be compiled on their own,
they can only be compiled via the main file.
A naive editing environment
(such as a text editor with an option
to have the current file processed by \LaTeX)
may require one to switch to the main file before compiling;
attempting to compile the child file produces errors.
\item
The main file must be modified (each time)
to adjust the |\includeonly| command
to the present needs. This easily leaves the main file in a messy state.
\item
The generated document will always carry the filename
of the main document. This is inconvenient if
several child files are to be compiled and
to be kept for distribution.
\end{itemize}

The present package provides a simple interface
to make child files individually compilable by \LaTeX{}.
Compiling a child file then has the same effect as compiling
the main file with an |\includeonly| command
to select the appropriate child.
Moreover the generated document will carry the name of the child
rather than the main file.
This resolves all three above issues.

This feature is meant to make the editing of books,
thesis documents and lecture notes somewhat more convenient.
However, the package can also be used efficiently for
composing a series of documents (such as exercise sheets)
which are typically distributed individually.
It then assists the author in generating the individual documents
(potentially in different versions)
as well as a document containing the collected series.
Another application is in developing style files
or other kinds of included material
where compilation of the style file could redirect
to a sample or test file.

%%%%%%%%%%%%%%%%%%%%%%%%%%%%%%%%%%%%%%%%%%%%%%%%%%%%%%%%%%%%%%%%%%%%%%%%%%%%%%%%
%%%%%%%%%%%%%%%%%%%%%%%%%%%%%%%%%%%%%%%%%%%%%%%%%%%%%%%%%%%%%%%%%%%%%%%%%%%%%%%%
\section{Usage}

First of all, the package \textsf{childdoc} is \emph{not} a standard
\LaTeXe{} |.sty| style file! Therefore it needs to be invoked in
a non-standard way.

%%%%%%%%%%%%%%%%%%%%%%%%%%%%%%%%%%%%%%%%%%%%%%%%%%%%%%%%%%%%%%%%%%%%%%%%%%%%%%%%
\subsection{Included Files}
\label{sec:include}

%%%%%%%%%%%%%%%%%%%%%%%%%%%%%%%%%%%%%%%%
\DescribeMacro{\childdocmain}
To use the package, add the commands
\begin{center}
\begin{tabular}{l}
|\input{childdoc.def}|\\
|\childdocmain{}|\\
\end{tabular}
\end{center}
at the very top of the main \LaTeX{} file,
in particular \emph{before} the |\documentclass| statement!
The argument of |\childdocmain| should be left empty
(but it must be present).

%%%%%%%%%%%%%%%%%%%%%%%%%%%%%%%%%%%%%%%%
\DescribeMacro{\childdocof}
Furthermore, add the commands
\begin{center}
\begin{tabular}{l}
|\input{childdoc.def}|\\
|\childdocof{|\textit{main}|}|\\
\end{tabular}
\end{center}
at the top of every child file \textit{child}
which is included by |\include{|\textit{child}|}|
from within the main file
(or at least for those files to be compiled individually).
The argument \textit{main} must be the filename of the main file.

There are a couple of
considerations in setting up the main and child documents:

%%%%%%%%%%%%%%%%%%%%%%%%%%%%%%%%%%%%%%%%
\paragraph{Restrictions.}

Please note the following restrictions:
\begin{itemize}
\item
|\childdocmain| must be called with one argument \textit{main}
to ensure compatibility with earlier version of the package.
It must either be empty (|\childdocmain{}|)
or precisely match the filename of the main file in which it is specified.
See \secref{sec:detection} for further information.
\item
The filename \textit{main} must be specified without the |.tex| extension.
\item
The filename \textit{main} is case sensitive
(even in case-insensitive file systems)
due to internal string comparison.
\item
The argument \textit{main} should be fully expanded, it cannot be a macro.
\item
Subdirectories and special characters should be avoided in filenames.
\item
The command |\childdocmain{|\textit{main}|}| must be followed by a whitespace.
It should not be followed immediately by another command
or by a comment mark `|%|'.
This is because the \TeX{} parser reads the token immediately following
the argument of |\childdocmain| and puts it
at the beginning of every child section;
however, a white\-space is ignored.
\end{itemize}

%%%%%%%%%%%%%%%%%%%%%%%%%%%%%%%%%%%%%%%%
\paragraph{Content of Main File.}

It is advisable to place all content in the child files included by |\include|.
Any output contained in the main file will appear in all child documents
unless suppressed manually;
it cannot be suppressed automatically by the |\includeonly| directive
and thus should normally be avoided.
A method to include some content in the main file
by means of conditional processing is described in \secref{sec:conditional}.

%%%%%%%%%%%%%%%%%%%%%%%%%%%%%%%%%%%%%%%%
\paragraph{Page Numbering.}

When only a part of the document is compiled,
the appropriate numbering of pages
(as well as other status parameters)
is determined from the |.aux| files.
The latter contain information from previous passes.
However this information needs to propagate through
all intermediate child documents.
Therefore the page numbering in child documents may well
be inconsistent until the complete document is compiled at least once.

A useful (if unconventional) way to always ensure a consistent
page numbering is to restart the numbering in each child document
and denote the pages by `\textit{child}|.|\textit{page}'
where \textit{child} represents the chapter/section number of the child file.
This can be achieved by the command
|\numberwithin{page}{|\textit{child}|}|
of the \textsf{amsmath} package
where \textit{child} can be |chapter| or |section|
depending on the chosen structuring.
Alternatively, one can modify the macro |\thepage| appropriately
and reset the counter |page| at the start of each child file.

%%%%%%%%%%%%%%%%%%%%%%%%%%%%%%%%%%%%%%%%%%%%%%%%%%%%%%%%%%%%%%%%%%%%%%%%%%%%%%%%
\subsection{Conditional Processing}
\label{sec:conditional}

The package provides a mechanism to compile different versions
of a document. To customise the versions further some conditional processing
can come in handy to distinguish which version is being compiled.
The package provides two macros to describe the compilation context:

%%%%%%%%%%%%%%%%%%%%%%%%%%%%%%%%%%%%%%%%
\DescribeMacro{\ifchilddoc}
The conditional |\ifchilddoc| distinguishes between the compilation of
child documents and the main document:
%
\begin{center}
|\ifchilddoc |\textit{child-code}| |[|\||else |\textit{main-code}]| \||fi|
\end{center}

%%%%%%%%%%%%%%%%%%%%%%%%%%%%%%%%%%%%%%%%
\DescribeMacro{\childdocname}
\DescribeMacro{\childdocjob}
The macro |\childdocname| contains the filename (without extension)
of the main or child file being processed.
Note that |\childdocjob| will always contain the name of the main file.

%%%%%%%%%%%%%%%%%%%%%%%%%%%%%%%%%%%%%%%%
\paragraph{Title Page.}

Conditional processing can be used to include a title or banner page
in the main document when proper precautions are taken.
Importantly, the code in the main file should ensure that the page counter
(as well as other status parameters which are stored in the |.aux| files)
takes the same value after the conditional processing.
Otherwise the page numbers may take divergent values
depending on which part is compiled.

For example, a title page could be declared by:
%
\begin{center}
\begin{tabular}{l}
|\ifchilddoc\||else|\\
|\addtocounter{page}{-1}|\\
\textit{code for title page}\\
|\newpage|\\
|\||fi|
\end{tabular}
\end{center}
%
A banner page for the child documents can be generated by:
%
\begin{center}
\begin{tabular}{l}
|\ifchilddoc|\\
|\addtocounter{page}{-1}|\\
\textit{code for banner page}\\
|\newpage|\\
|\||fi|
\end{tabular}
\end{center}
%
Here one could write a message such as:
\begin{center}
|This is the part \childdocname{} of \childdocjob{}.|
\end{center}

%%%%%%%%%%%%%%%%%%%%%%%%%%%%%%%%%%%%%%%%%%%%%%%%%%%%%%%%%%%%%%%%%%%%%%%%%%%%%%%%
\subsection{Flags}
\label{sec:flags}

The package makes it easy to generate different versions
of the main or child documents.
To this end compilation flags can be defined
and assigned different default values.
They will be particularly useful in conjunction
with the forwarding mechanism described in \secref{sec:forward}.

For example, it may be useful to have a flag |\version|
which can be set to |draft| or |final|.
The document source will contain some conditional code
depending on the value of |\version|.
Suppose further, the flag should default to |final| for the main file
and to |draft| for child files
which is a natural assignment for editing the document.
This is achieved by placing the following code
in the preamble of the main document
(below the |\childdocmain| directive):
%
\begin{center}
\begin{tabular}{l}
|\ifchilddoc|\\
|\providecommand{\version}{draft}|\\
|\||else|\\
|\providecommand{\version}{final}|\\
|\||fi|
\end{tabular}
\end{center}
%
The definition by |\providecommand| makes sure
that previous definitions are not overwritten.
Further statements |\providecommand{\version}{...}|
can thus be added before the above code to override it.

For the main file, one might add a line
(between |\childdocmain| and the above block)
%
\begin{center}
|%\ifchilddoc\||else\providecommand{\version}{draft}\||fi|
\end{center}
%
which can be uncommented to produce a draft version.
Likewise one can add a line to the very top of a child file
(above the |\childdocof{|\textit{main}|}| directive)
%
\begin{center}
|%\providecommand{\version}{final}|
\end{center}
%
which can be uncommented to produce the final version of this child document.

%%%%%%%%%%%%%%%%%%%%%%%%%%%%%%%%%%%%%%%%%%%%%%%%%%%%%%%%%%%%%%%%%%%%%%%%%%%%%%%%
\subsection{Forwarding}
\label{sec:forward}

Different versions of the main or child documents
using compilation flags as described in \secref{sec:flags}
can be (permanently) stored in different files
for convenient compilation, viewing and distribution.
To this end, the package defines a command
to pass on compilation to a different file:

%%%%%%%%%%%%%%%%%%%%%%%%%%%%%%%%%%%%%%%%
\DescribeMacro{\childdocforward}
The command |\childdocforward| redirects processing to
another source file:
%
\begin{center}
\begin{tabular}{l}
|\input{childdoc.def}|\\
|\childdocforward[|\textit{main}|]{|\textit{dest}|}|\\
\end{tabular}
\end{center}
%
The argument \textit{dest} is the destination file
(without extension).
It should be the main file or one of the child files.
Note that further \textsf{childdoc} directives
such as |\childdocof| and |\childdocforward|
in the indicated file will be processed in this form.
The optional argument \textit{main}
passes on directly to the main file \textit{main}
while pretending to compile the child \textit{dest}.
This form behaves as if \textit{dest}
issues |\childdocof{|\textit{main}|}| right away,
and no further \textsf{childdoc} directives will be processed.

%%%%%%%%%%%%%%%%%%%%%%%%%%%%%%%%%%%%%%%%
\DescribeMacro{\...prefix}
In the alternative form |\childdocforwardprefix|,
%
\begin{center}
\begin{tabular}{l}
|\input{childdoc.def}|\\
|\childdocforwardprefix[|\textit{main}|]{|\textit{prefix}|}{|\textit{dest}|}|
\end{tabular}
\end{center}
%
the destination file is determined by a pattern
depending on the current file:
To make this work, the current file must be called
`{\textit{prefix}\hspace{0.2em}\textit{suffix}}'
with \textit{prefix} matching precisely the argument.
Processing is then passed on to the file
`{\textit{dest}\hspace{0.2em}\textit{suffix}}'.
Surely, the same effect is achieved by
directly specifying the
argument `{\textit{dest}\hspace{0.2em}\textit{suffix}}'
in the first form.
However, that requires to set up a different file
for each child. With the alternative form of the command
all these files can have exactly the same content
which simplifies setting them up and maintaining them.

For example, the following file |draft.tex|
with a compilation flag |\version| as described in \secref{sec:flags}
compiles the main document as a draft:
%
\begin{center}
\begin{tabular}{l}
|\def\version{draft}|\\
|\input{childdoc.def}|\\
|\childdocforward{|\textit{main}|}|
\end{tabular}
\end{center}
%
Likewise, the following files |final|\textit{nn}|.tex|
compile the final version of the child document
|child|\textit{nn}|.tex|:
%
\begin{center}
\begin{tabular}{l}
|\def\version{final}|\\
|\input{childdoc.def}|\\
|\childdocforwardprefix{final}{child}|
\end{tabular}
\end{center}
%

Note that when several versions of a main file and/or of each child file
are to be generated, it may be convenient to set up a |Makefile| or
shell script to automatise the process.

%%%%%%%%%%%%%%%%%%%%%%%%%%%%%%%%%%%%%%%%%%%%%%%%%%%%%%%%%%%%%%%%%%%%%%%%%%%%%%%%
\subsection{Command Line Processing}
\label{sec:commandline}

The effect of redirection files can also be achieved by invoking
the \LaTeX{} compiler with a more elaborate command line.
Most conveniently this should be done as part
of a shell script or a |Makefile|.

When using \textsf{childdoc} in the main file, the following
command lines effectively perform a redirection
(note that depending on the shell being used,
backslashes may have to be doubled: `|\|' $\to$ `|\\|'):
%
\begin{center}
|... -jobname "|\textit{target}|" |\\|"|[\textit{flags}]%
|\input{childdoc.def}\childdocforward[|\textit{main}|]{|\textit{dest}|}"|
\end{center}
%
Here \textit{target} is the name of the output file,
\textit{main} is the name of the main file
and \textit{dest} is the name of the main or child file to be processed
(all filenames without extensions).
The optional argument \textit{main} can be omitted
if \textit{main} matches \textit{dest}.
Optionally, compilation \textit{flags} can be defined via |\def| commands.
This command line makes the \TeX{} engine believe
it is compiling the file \textit{target}
whose content is specified as the latter parameter.
The provided code then forwards the processing to
\textit{main} or \textit{dest} as described in \secref{sec:forward}.

%%%%%%%%%%%%%%%%%%%%%%%%%%%%%%%%%%%%%%%%%%%%%%%%%%%%%%%%%%%%%%%%%%%%%%%%%%%%%%%%
\subsection{Include by Input}
\label{sec:input}

Including child documents by |\include| has some restrictions by design.
Most notably, the content of a child document always occupies
its own set of pages; pages cannot be shared between child documents.
Usually, this behaviour makes perfect sense
because each child document contain an essential part of the document.
However, in some situations it may be desirable to compose
a document from a collection of parts
without having mandatory page breaks between then.
For this case, the package
provides a mechanism to include parts
by |\input| which can also be processed individually.
However, by construction this mechanism
requires manual handling of the content to be output.

%%%%%%%%%%%%%%%%%%%%%%%%%%%%%%%%%%%%%%%%
\DescribeMacro{\ifchilddocmanual}
The main file should be prepared as usual, see \secref{sec:include}.
However, the document body must make a distinction
between processing of an individual part and of the main document, e.g.:
%
\begin{center}
\begin{tabular}{l}
|\ifchilddocmanual|\\
|\input{\childdocname}|\\
|\||else|\\
\textit{document body with }|\input{|\textit{part}|}|\\
|\||fi|
\end{tabular}
\end{center}
%
The conditional |\ifchilddocmanual| is true whenever
a part to be included by |\input| is being compiled,
and the name of the part is stored in |\childdocname|.

%%%%%%%%%%%%%%%%%%%%%%%%%%%%%%%%%%%%%%%%
\DescribeMacro{\childdocby}
Each part to be included by |\input| should start with:
%
\begin{center}
\begin{tabular}{l}
|\input{childdoc.def}|\\
|\childdocby{|\textit{main}|}|\\
\end{tabular}
\end{center}
%
The directive |\childdocby| is similar to |\childdocof|
described in \secref{sec:include},
but the subsequent selection of content must be done manually.
To that end, both |\ifchilddoc| and |\ifchilddocmanual|
will be true upon processing of a part,
and the name of the part is stored in |\childdocname|.
Note that |\jobname| will be set to the filename of the current part
so that each part receives an individual |.aux| file
that does not interfere with the |.aux| file(s) of the main document.
This behaviour can be altered by the alternative form
|\childdocby[*]{|\textit{main}|}| (with a non-empty optional argument)
which uses the |.aux| file of the main document
by setting |\jobname| to \textit{main}.

%%%%%%%%%%%%%%%%%%%%%%%%%%%%%%%%%%%%%%%%%%%%%%%%%%%%%%%%%%%%%%%%%%%%%%%%%%%%%%%%
\subsection{Driver Development}
\label{sec:driver}

The \textsf{childdoc} mechanism can also be use for the development
of definition files such as \LaTeX{} styles or classes.
This case differs from the above setup with multiple parts
included by |\include| in that no |\includeonly| should be invoked.
This can be achieved by starting the include file
(before |\ProvidesPackage|) with:
%
\begin{center}
\begin{tabular}{l}
|\input{childdoc.def}|\\
|\childdocforward{|\textit{main}|}|\\
\end{tabular}
\end{center}
%
or alternatively with:
%
\begin{center}
\begin{tabular}{l}
|\input{childdoc.def}|\\
|\childdocby{|\textit{main}|}|\\
\end{tabular}
\end{center}
%
Both forms have slightly different effects as described above.
The main file is prepared as usual, see \secref{sec:include}.

%%%%%%%%%%%%%%%%%%%%%%%%%%%%%%%%%%%%%%%%%%%%%%%%%%%%%%%%%%%%%%%%%%%%%%%%%%%%%%%%
\subsection{Legacy Detection}
\label{sec:detection}

The directive |\childdocmain| in the main file can detect
whether the complete document or merely a child is to be compiled
even without using the directive |\childdocof|.
This method is deprecated because it is less robust
and there is no compelling reason to use it;
it is merely provided for backward compatibility
and it may be removed in future versions.

If the detection mechanism is to be used,
it is mandatory to correctly specify
the filename of the main file as the argument of |\childdocmain|:
%
\begin{center}
\begin{tabular}{l}
|\input{childdoc.def}|\\
|\childdocmain{|\textit{main}|}|\\
\end{tabular}
\end{center}
%
If |\jobname| does not match the argument \textit{main} of |\childdocmain|,
it is assumed that |\jobname| points to the child file to be compiled.
When using |\childdocmain| with the main file specified as argument,
it suffices to start a child file
with just |\input{|\textit{main}|}|
without loading of the package and using |\childdocof|.
If instead all processing is done
with the appropriate \textsf{childdoc} directives,
the argument of \textit{main} of |\childdocmain| can be empty.

An alternative version of the command line processing described
in \secref{sec:commandline} using the detection mechanism reads:
%
\begin{center}
|... -jobname "|\textit{target}|" "|[\textit{flags}]%
[|\def\jobname{|\textit{dest}|}|]|\input{|\textit{main}|}"|
\end{center}

%%%%%%%%%%%%%%%%%%%%%%%%%%%%%%%%%%%%%%%%%%%%%%%%%%%%%%%%%%%%%%%%%%%%%%%%%%%%%%%%
\subsection{Manual Code}
\label{sec:manual}

In case one cannot be certain whether the definitions file |childdoc.def|
is installed on the target \TeX{} distribution
and one prefers not to ship it,
it is conceivable to paste a few relevant commands into the sources.

To that end, drop all statements |\input{childdoc.def}|
and perform the replacements as outlined below.
Instead of |\childdocmain{|\textit{main}|}| add the following code
to the top of the main file:
%
\begin{center}
\begin{tabular}{l}
|\||ifdefined\childdocname\endinput\||fi\newif\ifchilddoc|\\
|\edef\childdocname{\scantokens\expandafter{\jobname\noexpand}}|\\
|\def\childdocmain{|\textit{main}|}\||ifx\childdocmain\childdocname\||else|\\
|\childdoctrue\includeonly{\childdocname}\let\jobname\childdocmain\||fi|\\
\end{tabular}
\end{center}
%
Instead of |\childdocof{|\textit{main}|}| just include the main file
at the top of each child file:
%
\begin{center}
|\input{|\textit{main}|}|
\end{center}
%
A simple redirection |\childdocforward{|\textit{dest}|}| is achieved by:
%
\begin{center}
|\def\jobname{|\textit{dest}|}\input{\jobname}|
\end{center}
%
The redirection with prefix
|\childdocforwardprefix[|\textit{prefix}|]{|\textit{dest}|}|
is accomplished by:
%
\begin{center}
\begin{tabular}{l}
|{\edef\jobname{\scantokens\expandafter{\jobname\noexpand}}|\\
|\def\redirectjob |\textit{prefix}|#1~~~{\gdef\jobname{|\textit{dest}|#1}}|\\
|\expandafter\redirectjob\jobname~~~}\input{\jobname}|
\end{tabular}
\end{center}

In an alternative approach,
child documents can be compiled by a specific command line
without additional code or specific definitions:
%
\begin{center}
|... -jobname "|\textit{target}|" "|[\textit{flags}]%
|\includeonly{|\textit{dest}|}\input{|\textit{main}|}"|
\end{center}
%

%%%%%%%%%%%%%%%%%%%%%%%%%%%%%%%%%%%%%%%%%%%%%%%%%%%%%%%%%%%%%%%%%%%%%%%%%%%%%%%%
%%%%%%%%%%%%%%%%%%%%%%%%%%%%%%%%%%%%%%%%%%%%%%%%%%%%%%%%%%%%%%%%%%%%%%%%%%%%%%%%
\section{Information}

%%%%%%%%%%%%%%%%%%%%%%%%%%%%%%%%%%%%%%%%%%%%%%%%%%%%%%%%%%%%%%%%%%%%%%%%%%%%%%%%
\subsection{Copyright}

Copyright \copyright{} 2017--2018 Niklas Beisert

This work may be distributed and/or modified under the
conditions of the \LaTeX{} Project Public License, either version 1.3
of this license or (at your option) any later version.
The latest version of this license is in
  \url{http://www.latex-project.org/lppl.txt}
and version 1.3 or later is part of all distributions of \LaTeX{}
version 2005/12/01 or later.

This work has the LPPL maintenance status `maintained'.

The Current Maintainer of this work is Niklas Beisert.

This work consists of the files |README.txt|, |childdoc.ins| and |childdoc.dtx|
as well as the derived files |childdoc.def|, |cdocsamp.tex|
with |cdocsch1.tex|, |cdocsch2.tex|, |cdocspt3.tex|, |cdocspt4.tex|,
|cdocsdrf.tex|, |cdocsfn1.tex|, |cdocsfn2.tex|
as well as |childdoc.pdf|.

%%%%%%%%%%%%%%%%%%%%%%%%%%%%%%%%%%%%%%%%%%%%%%%%%%%%%%%%%%%%%%%%%%%%%%%%%%%%%%%%
\subsection{Files and Installation}

The package consists of the files:
%
\begin{center}
\begin{tabular}{ll}
    |README.txt|   & readme file \\
    |childdoc.ins| & installation file \\
    |childdoc.dtx| & source file \\
    |childdoc.def| & definition file \\
    |cdocsamp.tex| & sample main file \\
    |cdocsch1.tex| & sample include file \\
    |cdocsch2.tex| & sample include file \\
    |cdocspt3.tex| & sample part file \\
    |cdocspt4.tex| & sample part file \\
    |cdocsdrf.tex| & sample redirection file \\
    |cdocsfn1.tex| & sample redirection file \\
    |cdocsfn2.tex| & sample redirection file \\
    |childdoc.pdf| & manual
\end{tabular}
\end{center}
%
The distribution consists of the files
|README.txt|, |childdoc.ins| and |childdoc.dtx|.
%
\begin{itemize}
\item
Run (pdf)\LaTeX{} on |childdoc.dtx|
to compile the manual |childdoc.pdf| (this file).
\item
Run \LaTeX{} on |childdoc.ins| to create the definitions file |childdoc.def|
and the sample |cdocsamp.tex| with include files
|cdocsch1.tex|, |cdocsch2.tex|, |cdocspt3.tex|, |cdocspt4.tex|,
|cdocsdrf.tex|, |cdocsfn1.tex|, |cdocsfn2.tex|.
Then copy the file |childdoc.def| to an appropriate directory of your \LaTeX{}
distribution, e.g.\ \textit{texmf-root}|/tex/latex/childdoc|.
\end{itemize}

%%%%%%%%%%%%%%%%%%%%%%%%%%%%%%%%%%%%%%%%%%%%%%%%%%%%%%%%%%%%%%%%%%%%%%%%%%%%%%%%
\subsection{Related CTAN Packages}

There are several other packages which offer a similar functionality:
%
\begin{itemize}
\item
The packages
\href{http://ctan.org/pkg/docmute}{\textsf{docmute}},
\href{http://ctan.org/pkg/includex}{\textsf{includex}} and
\href{http://ctan.org/pkg/standalone}{\textsf{standalone}}
provide commands to include only the document body of
a child file thus allowing both files to be compiled individually.
\item
The packages \href{http://ctan.org/pkg/subdocs}{\textsf{subdocs}}
and \href{http://ctan.org/pkg/subfiles}{\textsf{subfiles}}
provide structures in which the main and child documents can be
encapsulated and allowing them to be compiled individually.
The inclusion mechanism is different from the conventional |\include|.
\item
The package \href{http://ctan.org/pkg/combine}{\textsf{combine}}
is an elaborate solution to combine several documents into one.
\end{itemize}
%
See also the CTAN topic \href{http://ctan.org/topic/subdocs}{\textsf{subdocs}}
for further related packages.
The present package differs from the above solutions in that
a document structure constructed with the conventional |\include| mechanism
just needs two extra commands at the top of every file
such that all constituent files can be compiled individually.

%%%%%%%%%%%%%%%%%%%%%%%%%%%%%%%%%%%%%%%%%%%%%%%%%%%%%%%%%%%%%%%%%%%%%%%%%%%%%%%%
%\subsection{Feature Suggestions}
%
%The following is a list of features which may be useful for future
%versions of this package:
%%
%\begin{itemize}
%\item
%\ldots
%\end{itemize}

%%%%%%%%%%%%%%%%%%%%%%%%%%%%%%%%%%%%%%%%%%%%%%%%%%%%%%%%%%%%%%%%%%%%%%%%%%%%%%%%
\subsection{Revision History}

%%%%%%%%%%%%%%%%%%%%%%%%%%%%%%%%%%%%%%%%
\paragraph{v2.0:} 2018/12/30

\begin{itemize}
\item
immediate forward processing
\item
added |\childdocby| mechanism
\item
manual restructured
\end{itemize}

%%%%%%%%%%%%%%%%%%%%%%%%%%%%%%%%%%%%%%%%
\paragraph{v1.6:} 2018/01/17

\begin{itemize}
\item
application for development of include files
\item
corrections to manual
\end{itemize}

%%%%%%%%%%%%%%%%%%%%%%%%%%%%%%%%%%%%%%%%
\paragraph{v1.5:} 2017/05/21

\begin{itemize}
\item
more complete structuring introduced
\item
|\childdocof| introduced
\item
|\childdoc| renamed to |\childdocmain|
\item
|\childredirect| renamed to |\childdocforward| and |\childdocforwardprefix|
and functionality expanded
\end{itemize}

%%%%%%%%%%%%%%%%%%%%%%%%%%%%%%%%%%%%%%%%
\paragraph{v1.0:} 2017/04/27

\begin{itemize}
\item
manual and install package
\item
first version published on CTAN
\end{itemize}

%%%%%%%%%%%%%%%%%%%%%%%%%%%%%%%%%%%%%%%%
\paragraph{v0.6:} 2017/04/26

\begin{itemize}
\item
redirection mechanism added
\end{itemize}

%%%%%%%%%%%%%%%%%%%%%%%%%%%%%%%%%%%%%%%%
\paragraph{v0.5:} 2017/04/26

\begin{itemize}
\item
functionality in definition file
\end{itemize}


%%%%%%%%%%%%%%%%%%%%%%%%%%%%%%%%%%%%%%%%%%%%%%%%%%%%%%%%%%%%%%%%%%%%%%%%%%%%%%%%
%%%%%%%%%%%%%%%%%%%%%%%%%%%%%%%%%%%%%%%%%%%%%%%%%%%%%%%%%%%%%%%%%%%%%%%%%%%%%%%%
%%%%%%%%%%%%%%%%%%%%%%%%%%%%%%%%%%%%%%%%%%%%%%%%%%%%%%%%%%%%%%%%%%%%%%%%%%%%%%%%
\appendix

\settowidth\MacroIndent{\rmfamily\scriptsize 000\ }

 \DocInput{childdoc.dtx}

\end{document}
%</driver>
% \fi
%
% %%%%%%%%%%%%%%%%%%%%%%%%%%%%%%%%%%%%%%%%%%%%%%%%%%%%%%%%%%%%%%%%%%%%%%%%%%%%%%
% %%%%%%%%%%%%%%%%%%%%%%%%%%%%%%%%%%%%%%%%%%%%%%%%%%%%%%%%%%%%%%%%%%%%%%%%%%%%%%
% \section{Sample}
%\iffalse
%<*samplemain>
%\fi
%
% The following presents a sample document
% with two chapters, two parts, a title page,
% a compile flag as well as three forwarding files to set the flag.
% It consists of eight |.tex| files:
% \begin{center}
% \begin{tabular}{ll}
% |cdocsamp.tex|&main file\\
% |cdocsch1.tex|&include file for chapter 1\\
% |cdocsch2.tex|&include file for chapter 2\\
% |cdocspt3.tex|&include file for part 3\\
% |cdocspt4.tex|&include file for part 4\\
% |cdocsdrf.tex|&forwarding file for main file in draft mode\\
% |cdocsfi1.tex|&forwarding file for final version of chapter 1\\
% |cdocsfi2.tex|&forwarding file for final version of chapter 2\\
% \end{tabular}
% \end{center}
% Each of the eight files can be compiled directly by the \LaTeX{} compiler.
%
% %%%%%%%%%%%%%%%%%%%%%%%%%%%%%%%%%%%%%%
% \paragraph{Main File.}
%
% The main file is called |cdocsamp.tex|.
%
% Load the \textsf{childdoc} definitions and
% declare the filename for the main document:
%    \begin{macrocode}
\input{childdoc.def}
\childdocmain{}
%    \end{macrocode}

% Optional override for |\version| flag:
%    \begin{macrocode}
%%\ifchilddoc\else\providecommand{\version}{draft}\fi
%    \end{macrocode}

% Define the default values for the |\version| flag
% (|final| for the main file and |draft| for childs):
%    \begin{macrocode}
\ifchilddoc
\providecommand{\version}{draft}
\else
\providecommand{\version}{final}
\fi
%    \end{macrocode}

% Load the standard document class:
%    \begin{macrocode}
\documentclass[12pt]{article}
%    \end{macrocode}

% Start the document body:
%    \begin{macrocode}
\begin{document}
%    \end{macrocode}

% Declare a title page.
% Print title, part of document being processed and version flag:
%    \begin{macrocode}
\addtocounter{page}{-1}
\begin{center}
{\LARGE\bfseries{}childdoc example\par}
\vspace{1cm}
\ifchilddoc
\ifchilddocmanual part\else chapter\fi:
`\childdocname' of `\childdocjob'\par
\else
main document: `\childdocjob'\par
\fi
version: \version\par
\end{center}
\newpage
%    \end{macrocode}

% Manually include selected file,
% otherwise process as usual:
%    \begin{macrocode}
\ifchilddocmanual
\section*{part `\childdocname'}
\input{\childdocname}
\else
%    \end{macrocode}

% Include the two chapters:
%    \begin{macrocode}
\include{cdocsch1}
\include{cdocsch2}
%    \end{macrocode}

% Include the two parts unless only chapters should be displayed:
%    \begin{macrocode}
\ifchilddoc\else
\section{part three}
\input{cdocspt3}
\section{part four}
\input{cdocspt4}
\fi
%    \end{macrocode}

% Process as usual until here:
%    \begin{macrocode}
\fi
%    \end{macrocode}

% End of document body:
%    \begin{macrocode}
\end{document}
%    \end{macrocode}
%\iffalse
%</samplemain>
%\fi
%
% %%%%%%%%%%%%%%%%%%%%%%%%%%%%%%%%%%%%%%
% \paragraph{Chapter Include Files.}
%
% The include files are called |cdocsch1.tex| and |cdocsch2.tex|.
%
%\iffalse
%<*samplechap1|samplechap2>
%\fi

% Optional override for |\version| flag:
%    \begin{macrocode}
%%\providecommand{\version}{final}
%    \end{macrocode}

% Include the main document:
%    \begin{macrocode}
\input{childdoc.def}
\childdocof{cdocsamp}
%    \end{macrocode}

%\iffalse
%</samplechap1|samplechap2>
%\fi
%
%\iffalse
%<*samplechap1>
%\fi
% Some text for chapter 1:
%    \begin{macrocode}
\section{one}
some text in chapter one
%    \end{macrocode}

%\iffalse
%</samplechap1>
%\fi
% Some text for chapter 2:
%\iffalse
%<*samplechap2>
%\fi
%    \begin{macrocode}
\section{two}
more text in chapter two
%    \end{macrocode}

%\iffalse
%</samplechap2>
%\fi
%
% %%%%%%%%%%%%%%%%%%%%%%%%%%%%%%%%%%%%%%
% \paragraph{Part Include Files.}
%
% The include files are called |cdocspt3.tex| and |cdocspt4.tex|.
%
%\iffalse
%<*samplepart3|samplepart4>
%\fi

% Optional override for |\version| flag:
%    \begin{macrocode}
%%\providecommand{\version}{final}
%    \end{macrocode}

% Include the main document:
%    \begin{macrocode}
\input{childdoc.def}
\childdocby{cdocsamp}
%    \end{macrocode}

%\iffalse
%</samplepart3|samplepart4>
%\fi
%
%\iffalse
%<*samplepart3>
%\fi
% Some text for part 3:
%    \begin{macrocode}
some text in part three
%    \end{macrocode}

%\iffalse
%</samplepart3>
%\fi
% Some text for part 4:
%\iffalse
%<*samplepart4>
%\fi
%    \begin{macrocode}
more text in part four
%    \end{macrocode}

%\iffalse
%</samplepart4>
%\fi
%
% %%%%%%%%%%%%%%%%%%%%%%%%%%%%%%%%%%%%%%
% \paragraph{Forwarding for a Complete Draft.}
%
% The following forwarding file |cdocsdrf.tex|
% compiles the main document in draft mode:
%\iffalse
%<*sampledraft>
%\fi
%    \begin{macrocode}
\def\version{draft}
\input{childdoc.def}
\childdocforward{cdocsamp}
%    \end{macrocode}

%\iffalse
%</sampledraft>
%\fi
%
% %%%%%%%%%%%%%%%%%%%%%%%%%%%%%%%%%%%%%%
% \paragraph{Forwarding for Final Version of the Chapters.}
%
% The following forwarding files |cdocsfn1.tex| and |cdocsfn2.tex|
% (with identical content)
% compile the final versions of the child documents
% |cdocsch1.tex| and |cdocsch2.tex|, respectively:
%\iffalse
%<*samplefinal>
%\fi
%    \begin{macrocode}
\def\version{final}
\input{childdoc.def}
\childdocforwardprefix[cdocsamp]{cdocsfn}{cdocsch}
%    \end{macrocode}

%\iffalse
%</samplefinal>
%\fi
%
% %%%%%%%%%%%%%%%%%%%%%%%%%%%%%%%%%%%%%%
% \paragraph{Command Line Processing.}
%
% The following three command lines generate the output files
% |cdocscld|, |cdocscl1| and |cdocscl2|
% which should be identical to
% |cdocsdrf|, |cdocsch1| and |cdocsfn2|, respectively:
% \begin{center}
% \begin{tabular}{l}
% |latex -jobname cdocscld \|\\
% |  "\def\version{draft}\input{childdoc.def}\childdocforward{cdocsamp}"|\\
% |latex -jobname cdocscl1 \|\\
% |  "\input{childdoc.def}\childdocforward[cdocsamp]{cdocsch1}"|\\
% |latex -jobname cdocscl2 \|\\
% |  "\def\version{final}\input{childdoc.def}\childdocforward{cdocsch2}"|
% \end{tabular}
% \end{center}
% Note that the trailing backslash on each first line
% merely continues the input to the second line
% (for convenient cut ant paste).
% Furthermore, the command |latex| can be replaced by any
% of its alternative versions such as |pdflatex|.
%
% %%%%%%%%%%%%%%%%%%%%%%%%%%%%%%%%%%%%%%%%%%%%%%%%%%%%%%%%%%%%%%%%%%%%%%%%%%%%%%
% %%%%%%%%%%%%%%%%%%%%%%%%%%%%%%%%%%%%%%%%%%%%%%%%%%%%%%%%%%%%%%%%%%%%%%%%%%%%%%
% \section{Implementation}
%\iffalse
%<*package>
%\fi
%
% This section describes the definitions file |childdoc.def|.

% The definitions cannot be loaded using |\usepackage| or |\RequirePackage|
% which has a mechanism to prevent loading a style file more than once.
% When loading the definitions by means of |\input|
% multiple instances have to be prevented manually:
%\iffalse
%This code needs to be before the `\ProvidesFile' directive
%which is defined at the beginning of this file.
%Therefore it is also placed there and commented out here.
%</package>
%<*discard>
%\fi
%    \begin{macrocode}
\ifdefined\childdocmain\endinput\fi
%    \end{macrocode}
%\iffalse
%</discard>
%<*package>
%\fi
%
% \macro{\ifchilddoc}
% \macro{\ifchilddocmanual}
% The conditional |\ifchilddoc| tells whether a
% child (true) or main (false) document is being compiled.
% The conditional |\ifchilddocmanual| tells whether
% the |\includeonly| mechanism is used (false) or
% the selection of child files must be performed manually (true).
% The definitions initialise to false:
%    \begin{macrocode}
\newif\ifchilddoc
\newif\ifchilddocmanual
%    \end{macrocode}

% \macro{\childdocname}
% \macro{\childdocjob}
% The macro |\childdocname| stores the name of the main document
% to be compiled. The macro |\childdocjob| stores the name of
% the document on which the \LaTeX{} compiler was originally invoked.
% The content of |\jobname| cannot be compared
% to filenames specified in the source due to different catcodes.
% The following code rescans |\jobname|, stores the result
% in |\childdocname| and saves a copy in |\childdocjob|:
%    \begin{macrocode}
\edef\childdocname{\scantokens\expandafter{\jobname\noexpand}}
\let\childdocjob\childdocname
%    \end{macrocode}

% \macro{\childdocdisable}
% The macro |\childdocdisable| prevents the main file
% from being processed more than once.
% At this stage, the main document command |\childdocmain|
% is assumed to be called once again where it should do nothing.
% Any subsequent call to it should prevent
% a secondary processing of the main document
% It overwrites the forwarding commands
% |\childdocof| and |\childdocforward|
% with empty macros to prevent further inclusions of the main document:
%    \begin{macrocode}
\newcommand{\childdocdisable}
{
  \renewcommand{\childdocmain}[1]{\renewcommand{\childdocmain}[1]{\endinput}}
  \renewcommand{\childdocof}[1]{}
  \renewcommand{\childdocby}[2][]{}
  \renewcommand{\childdocforward}[2][]{}
  \renewcommand{\childdocdisable}{}
}
%    \end{macrocode}

% \macro{\childdocmain}
% The macro |\childdocmain| is to be called at the top of the main file
% with nothing or the main filename (without extension) as argument.
% First, it breaks loops.
% If the argument is not empty and does not match |\childdocname|
% (which is set by the first inclusion of |childdoc.def|),
% |\ifchilddoc| is set to true, |\includeonly| is applied to the child file
% and |\jobname| is set to the main file
% (for proper handling of |.aux| files):
%    \begin{macrocode}
\newcommand{\childdocmain}[1]
{
  \childdocdisable\childdocmain{}
  \if?#1?\else
    \begingroup
      \def\childdoctmp{#1}
      \ifx\childdoctmp\childdocname
        \def\childdoctmp{}
      \else
        \def\childdoctmp
        {
          \childdoctrue
          \includeonly{\childdocname}
          \def\childdocjob{#1}
          \def\jobname{#1}
        }
      \fi
      \expandafter
    \endgroup
    \childdoctmp
  \fi
}
%    \end{macrocode}

% \macro{\childdocof}
% The command |\childdocof| redirects
% compilation to the main file |#1|.
%    \begin{macrocode}
\newcommand{\childdocof}[1]
{
  \childdocdisable
  \childdoctrue
  \includeonly{\childdocname}
  \def\jobname{#1}
  \def\childdocjob{#1}
  \input{#1}
}
%    \end{macrocode}

% \macro{\childdocby}
% The command |\childdocby| ....
%    \begin{macrocode}
\newcommand{\childdocby}[2][]
{
  \childdocdisable
  \childdoctrue
  \childdocmanualtrue
  \if?#1?\else
    \def\jobname{#2}
  \fi
  \def\childdocjob{#2}
  \input{#2}
  \endinput
}
%    \end{macrocode}

% \macro{\childdocforward}
% The command |\childdocforward| redirects
% compilation to the main file or
% (if the optional argument is given) a child file.
% Parameters are set as if the main file
% or a child file starting with |\childdocof| was compiled.
% Then compilation is handed over to the main file:
%    \begin{macrocode}
\newcommand{\childdocforward}[2][]
{
  \begingroup
    \if?#1?
      \def\childdoctmp
      {
        \def\childdocname{#2}
        \def\childdocjob{#2}
        \def\jobname{#2}
        \input{#2}
        \endinput
      }
    \else
      \def\childdoctmp
      {
        \childdocdisable
        \def\childdocname{#2}
        \childdoctrue
        \includeonly{#2}
        \def\childdocjob{#1}
        \def\jobname{#1}
        \input{#1}
        \endinput
      }
    \fi
    \expandafter
  \endgroup
  \childdoctmp
}
%    \end{macrocode}

% \macro{\childdocforwardprefix}
% The command |\childdocforwardprefix| redirects
% compilation to the main or a child file by means of a pattern.
% The prefix |#1| in the current filename is replaced by |#2|
% and the suffix of the current filename is kept
% (it is assumed that the filename does not contain the substring `|~~~|'
% which is used as a delimiter).
% Compilation is handed over to the new file by |\childdocforward|:
%    \begin{macrocode}
\newcommand{\childdocforwardprefix}[3][]
{
  \begingroup
    \def\childdocextract #2##1~~~{\def\childdoctmp{\childdocforward[#1]{#3##1}}}
    \expandafter\childdocextract\childdocname~~~
    \expandafter
  \endgroup
  \childdoctmp
}
%    \end{macrocode}

% \macro{\childdoc}
% The deprecated macro |\childdoc| is a legacy version of |\childdocmain|:
%    \begin{macrocode}
\newcommand{\childdoc}{\childdocmain}
%    \end{macrocode}

% \macro{\childdocredirect}
% The deprecated macro |\childdocredirect| is a legacy version
% of |\childdocforward| and |\childdocforwardprefix|:
%    \begin{macrocode}
\newcommand{\childdocredirect}[2][]
{
  \begingroup
    \if?#1?
      \def\childdoctmp{\childdocforward{#2}}
    \else
      \def\childdoctmp{\childdocforwardprefix{#1}{#2}}
    \fi
    \expandafter
  \endgroup
  \childdoctmp
}
%    \end{macrocode}

%\iffalse
%</package>
%\fi
%
\endinput

\childdocby{cdocsamp}
%    \end{macrocode}

%\iffalse
%</samplepart3|samplepart4>
%\fi
%
%\iffalse
%<*samplepart3>
%\fi
% Some text for part 3:
%    \begin{macrocode}
some text in part three
%    \end{macrocode}

%\iffalse
%</samplepart3>
%\fi
% Some text for part 4:
%\iffalse
%<*samplepart4>
%\fi
%    \begin{macrocode}
more text in part four
%    \end{macrocode}

%\iffalse
%</samplepart4>
%\fi
%
% %%%%%%%%%%%%%%%%%%%%%%%%%%%%%%%%%%%%%%
% \paragraph{Forwarding for a Complete Draft.}
%
% The following forwarding file |cdocsdrf.tex|
% compiles the main document in draft mode:
%\iffalse
%<*sampledraft>
%\fi
%    \begin{macrocode}
\def\version{draft}
% \iffalse
%
% childdoc.dtx Copyright (C) 2017-2018 Niklas Beisert
%
% This work may be distributed and/or modified under the
% conditions of the LaTeX Project Public License, either version 1.3
% of this license or (at your option) any later version.
% The latest version of this license is in
%   http://www.latex-project.org/lppl.txt
% and version 1.3 or later is part of all distributions of LaTeX
% version 2005/12/01 or later.
%
% This work has the LPPL maintenance status `maintained'.
%
% The Current Maintainer of this work is Niklas Beisert.
%
% This work consists of the files childdoc.dtx and childdoc.ins
% and the derived files childdoc.def and cdocsamp.tex with
% cdocsch1.tex, cdocsch2.tex, cdocsdrf.tex, cdocsfn1.tex, cdocsfn2.tex.
%
%<package>\ifdefined\childdocmain\endinput\fi
%<package>\ProvidesFile{childdoc.def}[2018/12/30 v2.0 child document driver]
%<samplemain>\ProvidesFile{cdocsamp.tex}[2018/12/30 v2.0 sample for childdoc]
%<*driver>
%\ProvidesFile{childdoc.drv}[2018/12/30 v2.0 childdoc reference manual file]
\PassOptionsToClass{10pt,a4paper}{article}
\documentclass{ltxdoc}

\usepackage[margin=35mm]{geometry}
\usepackage{hyperref}
\usepackage{hyperxmp}
\usepackage[usenames]{color}

\hypersetup{colorlinks=true}
\hypersetup{pdfstartview=FitH}
\hypersetup{pdfpagemode=UseNone}
\hypersetup{pdfsource={}}
\hypersetup{pdflang={en-UK}}
\hypersetup{pdfcopyright={Copyright 2017-2018 Niklas Beisert.
  This work may be distributed and/or modified under the
  conditions of the LaTeX Project Public License, either version 1.3
  of this license or (at your option) any later version.}}
\hypersetup{pdflicenseurl={http://www.latex-project.org/lppl.txt}}
\hypersetup{pdfcontactaddress={ETH Zurich, ITP, HIT K,
  Wolfgang-Pauli-Strasse 27}}
\hypersetup{pdfcontactpostcode={8093}}
\hypersetup{pdfcontactcity={Zurich}}
\hypersetup{pdfcontactcountry={Switzerland}}
\hypersetup{pdfcontactemail={nbeisert@itp.phys.ethz.ch}}
\hypersetup{pdfcontacturl={http://people.phys.ethz.ch/\xmptilde nbeisert/}}

\newcommand{\secref}[1]{\hyperref[#1]{section \ref*{#1}}}

\parskip1ex
\parindent0pt
\let\olditemize\itemize
\def\itemize{\olditemize\parskip0pt}

\begin{document}

\title{The \textsf{childdoc} Package}
\hypersetup{pdftitle={The childdoc Package}}
\author{Niklas Beisert\\[2ex]
  Institut f\"ur Theoretische Physik\\
  Eidgen\"ossische Technische Hochschule Z\"urich\\
  Wolfgang-Pauli-Strasse 27, 8093 Z\"urich, Switzerland\\[1ex]
  \href{mailto:nbeisert@itp.phys.ethz.ch}
  {\texttt{nbeisert@itp.phys.ethz.ch}}}
\hypersetup{pdfauthor={Niklas Beisert}}
\hypersetup{pdfsubject={Manual for the LaTeX2e Package childdoc}}
\date{30 December 2018, \textsf{v2.0}}
\maketitle

\begin{abstract}\noindent
\textsf{childdoc} is a \LaTeXe{} package
that enables the direct compilation
of document sections included by |\include|
to individual files.
\end{abstract}

\begingroup
\parskip0ex
\tableofcontents
\endgroup

%%%%%%%%%%%%%%%%%%%%%%%%%%%%%%%%%%%%%%%%%%%%%%%%%%%%%%%%%%%%%%%%%%%%%%%%%%%%%%%%
%%%%%%%%%%%%%%%%%%%%%%%%%%%%%%%%%%%%%%%%%%%%%%%%%%%%%%%%%%%%%%%%%%%%%%%%%%%%%%%%
\section{Introduction}

\LaTeX{} provides a mechanism to structure a large document (such as a book)
into a main file and several child files (containing the chapters)
using the |\include| command.
This mechanism is beneficial for documents
which span hundreds of pages in order to
make the source file(s) more manageable.
Moreover, compilation can be restricted to
selected child files by means of the |\includeonly| command.
The latter feature can be used to reduce the compilation time while editing
(this was significantly more useful in the earlier days of \LaTeX{})
or to generate a smaller document which is easier to navigate.
Another application of |\includeonly| is to generate
documents consisting of selected parts of the complete document.

However, there are a few drawbacks of the plain |\include| mechanism:
\begin{itemize}
\item
The child files cannot be compiled on their own,
they can only be compiled via the main file.
A naive editing environment
(such as a text editor with an option
to have the current file processed by \LaTeX)
may require one to switch to the main file before compiling;
attempting to compile the child file produces errors.
\item
The main file must be modified (each time)
to adjust the |\includeonly| command
to the present needs. This easily leaves the main file in a messy state.
\item
The generated document will always carry the filename
of the main document. This is inconvenient if
several child files are to be compiled and
to be kept for distribution.
\end{itemize}

The present package provides a simple interface
to make child files individually compilable by \LaTeX{}.
Compiling a child file then has the same effect as compiling
the main file with an |\includeonly| command
to select the appropriate child.
Moreover the generated document will carry the name of the child
rather than the main file.
This resolves all three above issues.

This feature is meant to make the editing of books,
thesis documents and lecture notes somewhat more convenient.
However, the package can also be used efficiently for
composing a series of documents (such as exercise sheets)
which are typically distributed individually.
It then assists the author in generating the individual documents
(potentially in different versions)
as well as a document containing the collected series.
Another application is in developing style files
or other kinds of included material
where compilation of the style file could redirect
to a sample or test file.

%%%%%%%%%%%%%%%%%%%%%%%%%%%%%%%%%%%%%%%%%%%%%%%%%%%%%%%%%%%%%%%%%%%%%%%%%%%%%%%%
%%%%%%%%%%%%%%%%%%%%%%%%%%%%%%%%%%%%%%%%%%%%%%%%%%%%%%%%%%%%%%%%%%%%%%%%%%%%%%%%
\section{Usage}

First of all, the package \textsf{childdoc} is \emph{not} a standard
\LaTeXe{} |.sty| style file! Therefore it needs to be invoked in
a non-standard way.

%%%%%%%%%%%%%%%%%%%%%%%%%%%%%%%%%%%%%%%%%%%%%%%%%%%%%%%%%%%%%%%%%%%%%%%%%%%%%%%%
\subsection{Included Files}
\label{sec:include}

%%%%%%%%%%%%%%%%%%%%%%%%%%%%%%%%%%%%%%%%
\DescribeMacro{\childdocmain}
To use the package, add the commands
\begin{center}
\begin{tabular}{l}
|\input{childdoc.def}|\\
|\childdocmain{}|\\
\end{tabular}
\end{center}
at the very top of the main \LaTeX{} file,
in particular \emph{before} the |\documentclass| statement!
The argument of |\childdocmain| should be left empty
(but it must be present).

%%%%%%%%%%%%%%%%%%%%%%%%%%%%%%%%%%%%%%%%
\DescribeMacro{\childdocof}
Furthermore, add the commands
\begin{center}
\begin{tabular}{l}
|\input{childdoc.def}|\\
|\childdocof{|\textit{main}|}|\\
\end{tabular}
\end{center}
at the top of every child file \textit{child}
which is included by |\include{|\textit{child}|}|
from within the main file
(or at least for those files to be compiled individually).
The argument \textit{main} must be the filename of the main file.

There are a couple of
considerations in setting up the main and child documents:

%%%%%%%%%%%%%%%%%%%%%%%%%%%%%%%%%%%%%%%%
\paragraph{Restrictions.}

Please note the following restrictions:
\begin{itemize}
\item
|\childdocmain| must be called with one argument \textit{main}
to ensure compatibility with earlier version of the package.
It must either be empty (|\childdocmain{}|)
or precisely match the filename of the main file in which it is specified.
See \secref{sec:detection} for further information.
\item
The filename \textit{main} must be specified without the |.tex| extension.
\item
The filename \textit{main} is case sensitive
(even in case-insensitive file systems)
due to internal string comparison.
\item
The argument \textit{main} should be fully expanded, it cannot be a macro.
\item
Subdirectories and special characters should be avoided in filenames.
\item
The command |\childdocmain{|\textit{main}|}| must be followed by a whitespace.
It should not be followed immediately by another command
or by a comment mark `|%|'.
This is because the \TeX{} parser reads the token immediately following
the argument of |\childdocmain| and puts it
at the beginning of every child section;
however, a white\-space is ignored.
\end{itemize}

%%%%%%%%%%%%%%%%%%%%%%%%%%%%%%%%%%%%%%%%
\paragraph{Content of Main File.}

It is advisable to place all content in the child files included by |\include|.
Any output contained in the main file will appear in all child documents
unless suppressed manually;
it cannot be suppressed automatically by the |\includeonly| directive
and thus should normally be avoided.
A method to include some content in the main file
by means of conditional processing is described in \secref{sec:conditional}.

%%%%%%%%%%%%%%%%%%%%%%%%%%%%%%%%%%%%%%%%
\paragraph{Page Numbering.}

When only a part of the document is compiled,
the appropriate numbering of pages
(as well as other status parameters)
is determined from the |.aux| files.
The latter contain information from previous passes.
However this information needs to propagate through
all intermediate child documents.
Therefore the page numbering in child documents may well
be inconsistent until the complete document is compiled at least once.

A useful (if unconventional) way to always ensure a consistent
page numbering is to restart the numbering in each child document
and denote the pages by `\textit{child}|.|\textit{page}'
where \textit{child} represents the chapter/section number of the child file.
This can be achieved by the command
|\numberwithin{page}{|\textit{child}|}|
of the \textsf{amsmath} package
where \textit{child} can be |chapter| or |section|
depending on the chosen structuring.
Alternatively, one can modify the macro |\thepage| appropriately
and reset the counter |page| at the start of each child file.

%%%%%%%%%%%%%%%%%%%%%%%%%%%%%%%%%%%%%%%%%%%%%%%%%%%%%%%%%%%%%%%%%%%%%%%%%%%%%%%%
\subsection{Conditional Processing}
\label{sec:conditional}

The package provides a mechanism to compile different versions
of a document. To customise the versions further some conditional processing
can come in handy to distinguish which version is being compiled.
The package provides two macros to describe the compilation context:

%%%%%%%%%%%%%%%%%%%%%%%%%%%%%%%%%%%%%%%%
\DescribeMacro{\ifchilddoc}
The conditional |\ifchilddoc| distinguishes between the compilation of
child documents and the main document:
%
\begin{center}
|\ifchilddoc |\textit{child-code}| |[|\||else |\textit{main-code}]| \||fi|
\end{center}

%%%%%%%%%%%%%%%%%%%%%%%%%%%%%%%%%%%%%%%%
\DescribeMacro{\childdocname}
\DescribeMacro{\childdocjob}
The macro |\childdocname| contains the filename (without extension)
of the main or child file being processed.
Note that |\childdocjob| will always contain the name of the main file.

%%%%%%%%%%%%%%%%%%%%%%%%%%%%%%%%%%%%%%%%
\paragraph{Title Page.}

Conditional processing can be used to include a title or banner page
in the main document when proper precautions are taken.
Importantly, the code in the main file should ensure that the page counter
(as well as other status parameters which are stored in the |.aux| files)
takes the same value after the conditional processing.
Otherwise the page numbers may take divergent values
depending on which part is compiled.

For example, a title page could be declared by:
%
\begin{center}
\begin{tabular}{l}
|\ifchilddoc\||else|\\
|\addtocounter{page}{-1}|\\
\textit{code for title page}\\
|\newpage|\\
|\||fi|
\end{tabular}
\end{center}
%
A banner page for the child documents can be generated by:
%
\begin{center}
\begin{tabular}{l}
|\ifchilddoc|\\
|\addtocounter{page}{-1}|\\
\textit{code for banner page}\\
|\newpage|\\
|\||fi|
\end{tabular}
\end{center}
%
Here one could write a message such as:
\begin{center}
|This is the part \childdocname{} of \childdocjob{}.|
\end{center}

%%%%%%%%%%%%%%%%%%%%%%%%%%%%%%%%%%%%%%%%%%%%%%%%%%%%%%%%%%%%%%%%%%%%%%%%%%%%%%%%
\subsection{Flags}
\label{sec:flags}

The package makes it easy to generate different versions
of the main or child documents.
To this end compilation flags can be defined
and assigned different default values.
They will be particularly useful in conjunction
with the forwarding mechanism described in \secref{sec:forward}.

For example, it may be useful to have a flag |\version|
which can be set to |draft| or |final|.
The document source will contain some conditional code
depending on the value of |\version|.
Suppose further, the flag should default to |final| for the main file
and to |draft| for child files
which is a natural assignment for editing the document.
This is achieved by placing the following code
in the preamble of the main document
(below the |\childdocmain| directive):
%
\begin{center}
\begin{tabular}{l}
|\ifchilddoc|\\
|\providecommand{\version}{draft}|\\
|\||else|\\
|\providecommand{\version}{final}|\\
|\||fi|
\end{tabular}
\end{center}
%
The definition by |\providecommand| makes sure
that previous definitions are not overwritten.
Further statements |\providecommand{\version}{...}|
can thus be added before the above code to override it.

For the main file, one might add a line
(between |\childdocmain| and the above block)
%
\begin{center}
|%\ifchilddoc\||else\providecommand{\version}{draft}\||fi|
\end{center}
%
which can be uncommented to produce a draft version.
Likewise one can add a line to the very top of a child file
(above the |\childdocof{|\textit{main}|}| directive)
%
\begin{center}
|%\providecommand{\version}{final}|
\end{center}
%
which can be uncommented to produce the final version of this child document.

%%%%%%%%%%%%%%%%%%%%%%%%%%%%%%%%%%%%%%%%%%%%%%%%%%%%%%%%%%%%%%%%%%%%%%%%%%%%%%%%
\subsection{Forwarding}
\label{sec:forward}

Different versions of the main or child documents
using compilation flags as described in \secref{sec:flags}
can be (permanently) stored in different files
for convenient compilation, viewing and distribution.
To this end, the package defines a command
to pass on compilation to a different file:

%%%%%%%%%%%%%%%%%%%%%%%%%%%%%%%%%%%%%%%%
\DescribeMacro{\childdocforward}
The command |\childdocforward| redirects processing to
another source file:
%
\begin{center}
\begin{tabular}{l}
|\input{childdoc.def}|\\
|\childdocforward[|\textit{main}|]{|\textit{dest}|}|\\
\end{tabular}
\end{center}
%
The argument \textit{dest} is the destination file
(without extension).
It should be the main file or one of the child files.
Note that further \textsf{childdoc} directives
such as |\childdocof| and |\childdocforward|
in the indicated file will be processed in this form.
The optional argument \textit{main}
passes on directly to the main file \textit{main}
while pretending to compile the child \textit{dest}.
This form behaves as if \textit{dest}
issues |\childdocof{|\textit{main}|}| right away,
and no further \textsf{childdoc} directives will be processed.

%%%%%%%%%%%%%%%%%%%%%%%%%%%%%%%%%%%%%%%%
\DescribeMacro{\...prefix}
In the alternative form |\childdocforwardprefix|,
%
\begin{center}
\begin{tabular}{l}
|\input{childdoc.def}|\\
|\childdocforwardprefix[|\textit{main}|]{|\textit{prefix}|}{|\textit{dest}|}|
\end{tabular}
\end{center}
%
the destination file is determined by a pattern
depending on the current file:
To make this work, the current file must be called
`{\textit{prefix}\hspace{0.2em}\textit{suffix}}'
with \textit{prefix} matching precisely the argument.
Processing is then passed on to the file
`{\textit{dest}\hspace{0.2em}\textit{suffix}}'.
Surely, the same effect is achieved by
directly specifying the
argument `{\textit{dest}\hspace{0.2em}\textit{suffix}}'
in the first form.
However, that requires to set up a different file
for each child. With the alternative form of the command
all these files can have exactly the same content
which simplifies setting them up and maintaining them.

For example, the following file |draft.tex|
with a compilation flag |\version| as described in \secref{sec:flags}
compiles the main document as a draft:
%
\begin{center}
\begin{tabular}{l}
|\def\version{draft}|\\
|\input{childdoc.def}|\\
|\childdocforward{|\textit{main}|}|
\end{tabular}
\end{center}
%
Likewise, the following files |final|\textit{nn}|.tex|
compile the final version of the child document
|child|\textit{nn}|.tex|:
%
\begin{center}
\begin{tabular}{l}
|\def\version{final}|\\
|\input{childdoc.def}|\\
|\childdocforwardprefix{final}{child}|
\end{tabular}
\end{center}
%

Note that when several versions of a main file and/or of each child file
are to be generated, it may be convenient to set up a |Makefile| or
shell script to automatise the process.

%%%%%%%%%%%%%%%%%%%%%%%%%%%%%%%%%%%%%%%%%%%%%%%%%%%%%%%%%%%%%%%%%%%%%%%%%%%%%%%%
\subsection{Command Line Processing}
\label{sec:commandline}

The effect of redirection files can also be achieved by invoking
the \LaTeX{} compiler with a more elaborate command line.
Most conveniently this should be done as part
of a shell script or a |Makefile|.

When using \textsf{childdoc} in the main file, the following
command lines effectively perform a redirection
(note that depending on the shell being used,
backslashes may have to be doubled: `|\|' $\to$ `|\\|'):
%
\begin{center}
|... -jobname "|\textit{target}|" |\\|"|[\textit{flags}]%
|\input{childdoc.def}\childdocforward[|\textit{main}|]{|\textit{dest}|}"|
\end{center}
%
Here \textit{target} is the name of the output file,
\textit{main} is the name of the main file
and \textit{dest} is the name of the main or child file to be processed
(all filenames without extensions).
The optional argument \textit{main} can be omitted
if \textit{main} matches \textit{dest}.
Optionally, compilation \textit{flags} can be defined via |\def| commands.
This command line makes the \TeX{} engine believe
it is compiling the file \textit{target}
whose content is specified as the latter parameter.
The provided code then forwards the processing to
\textit{main} or \textit{dest} as described in \secref{sec:forward}.

%%%%%%%%%%%%%%%%%%%%%%%%%%%%%%%%%%%%%%%%%%%%%%%%%%%%%%%%%%%%%%%%%%%%%%%%%%%%%%%%
\subsection{Include by Input}
\label{sec:input}

Including child documents by |\include| has some restrictions by design.
Most notably, the content of a child document always occupies
its own set of pages; pages cannot be shared between child documents.
Usually, this behaviour makes perfect sense
because each child document contain an essential part of the document.
However, in some situations it may be desirable to compose
a document from a collection of parts
without having mandatory page breaks between then.
For this case, the package
provides a mechanism to include parts
by |\input| which can also be processed individually.
However, by construction this mechanism
requires manual handling of the content to be output.

%%%%%%%%%%%%%%%%%%%%%%%%%%%%%%%%%%%%%%%%
\DescribeMacro{\ifchilddocmanual}
The main file should be prepared as usual, see \secref{sec:include}.
However, the document body must make a distinction
between processing of an individual part and of the main document, e.g.:
%
\begin{center}
\begin{tabular}{l}
|\ifchilddocmanual|\\
|\input{\childdocname}|\\
|\||else|\\
\textit{document body with }|\input{|\textit{part}|}|\\
|\||fi|
\end{tabular}
\end{center}
%
The conditional |\ifchilddocmanual| is true whenever
a part to be included by |\input| is being compiled,
and the name of the part is stored in |\childdocname|.

%%%%%%%%%%%%%%%%%%%%%%%%%%%%%%%%%%%%%%%%
\DescribeMacro{\childdocby}
Each part to be included by |\input| should start with:
%
\begin{center}
\begin{tabular}{l}
|\input{childdoc.def}|\\
|\childdocby{|\textit{main}|}|\\
\end{tabular}
\end{center}
%
The directive |\childdocby| is similar to |\childdocof|
described in \secref{sec:include},
but the subsequent selection of content must be done manually.
To that end, both |\ifchilddoc| and |\ifchilddocmanual|
will be true upon processing of a part,
and the name of the part is stored in |\childdocname|.
Note that |\jobname| will be set to the filename of the current part
so that each part receives an individual |.aux| file
that does not interfere with the |.aux| file(s) of the main document.
This behaviour can be altered by the alternative form
|\childdocby[*]{|\textit{main}|}| (with a non-empty optional argument)
which uses the |.aux| file of the main document
by setting |\jobname| to \textit{main}.

%%%%%%%%%%%%%%%%%%%%%%%%%%%%%%%%%%%%%%%%%%%%%%%%%%%%%%%%%%%%%%%%%%%%%%%%%%%%%%%%
\subsection{Driver Development}
\label{sec:driver}

The \textsf{childdoc} mechanism can also be use for the development
of definition files such as \LaTeX{} styles or classes.
This case differs from the above setup with multiple parts
included by |\include| in that no |\includeonly| should be invoked.
This can be achieved by starting the include file
(before |\ProvidesPackage|) with:
%
\begin{center}
\begin{tabular}{l}
|\input{childdoc.def}|\\
|\childdocforward{|\textit{main}|}|\\
\end{tabular}
\end{center}
%
or alternatively with:
%
\begin{center}
\begin{tabular}{l}
|\input{childdoc.def}|\\
|\childdocby{|\textit{main}|}|\\
\end{tabular}
\end{center}
%
Both forms have slightly different effects as described above.
The main file is prepared as usual, see \secref{sec:include}.

%%%%%%%%%%%%%%%%%%%%%%%%%%%%%%%%%%%%%%%%%%%%%%%%%%%%%%%%%%%%%%%%%%%%%%%%%%%%%%%%
\subsection{Legacy Detection}
\label{sec:detection}

The directive |\childdocmain| in the main file can detect
whether the complete document or merely a child is to be compiled
even without using the directive |\childdocof|.
This method is deprecated because it is less robust
and there is no compelling reason to use it;
it is merely provided for backward compatibility
and it may be removed in future versions.

If the detection mechanism is to be used,
it is mandatory to correctly specify
the filename of the main file as the argument of |\childdocmain|:
%
\begin{center}
\begin{tabular}{l}
|\input{childdoc.def}|\\
|\childdocmain{|\textit{main}|}|\\
\end{tabular}
\end{center}
%
If |\jobname| does not match the argument \textit{main} of |\childdocmain|,
it is assumed that |\jobname| points to the child file to be compiled.
When using |\childdocmain| with the main file specified as argument,
it suffices to start a child file
with just |\input{|\textit{main}|}|
without loading of the package and using |\childdocof|.
If instead all processing is done
with the appropriate \textsf{childdoc} directives,
the argument of \textit{main} of |\childdocmain| can be empty.

An alternative version of the command line processing described
in \secref{sec:commandline} using the detection mechanism reads:
%
\begin{center}
|... -jobname "|\textit{target}|" "|[\textit{flags}]%
[|\def\jobname{|\textit{dest}|}|]|\input{|\textit{main}|}"|
\end{center}

%%%%%%%%%%%%%%%%%%%%%%%%%%%%%%%%%%%%%%%%%%%%%%%%%%%%%%%%%%%%%%%%%%%%%%%%%%%%%%%%
\subsection{Manual Code}
\label{sec:manual}

In case one cannot be certain whether the definitions file |childdoc.def|
is installed on the target \TeX{} distribution
and one prefers not to ship it,
it is conceivable to paste a few relevant commands into the sources.

To that end, drop all statements |\input{childdoc.def}|
and perform the replacements as outlined below.
Instead of |\childdocmain{|\textit{main}|}| add the following code
to the top of the main file:
%
\begin{center}
\begin{tabular}{l}
|\||ifdefined\childdocname\endinput\||fi\newif\ifchilddoc|\\
|\edef\childdocname{\scantokens\expandafter{\jobname\noexpand}}|\\
|\def\childdocmain{|\textit{main}|}\||ifx\childdocmain\childdocname\||else|\\
|\childdoctrue\includeonly{\childdocname}\let\jobname\childdocmain\||fi|\\
\end{tabular}
\end{center}
%
Instead of |\childdocof{|\textit{main}|}| just include the main file
at the top of each child file:
%
\begin{center}
|\input{|\textit{main}|}|
\end{center}
%
A simple redirection |\childdocforward{|\textit{dest}|}| is achieved by:
%
\begin{center}
|\def\jobname{|\textit{dest}|}\input{\jobname}|
\end{center}
%
The redirection with prefix
|\childdocforwardprefix[|\textit{prefix}|]{|\textit{dest}|}|
is accomplished by:
%
\begin{center}
\begin{tabular}{l}
|{\edef\jobname{\scantokens\expandafter{\jobname\noexpand}}|\\
|\def\redirectjob |\textit{prefix}|#1~~~{\gdef\jobname{|\textit{dest}|#1}}|\\
|\expandafter\redirectjob\jobname~~~}\input{\jobname}|
\end{tabular}
\end{center}

In an alternative approach,
child documents can be compiled by a specific command line
without additional code or specific definitions:
%
\begin{center}
|... -jobname "|\textit{target}|" "|[\textit{flags}]%
|\includeonly{|\textit{dest}|}\input{|\textit{main}|}"|
\end{center}
%

%%%%%%%%%%%%%%%%%%%%%%%%%%%%%%%%%%%%%%%%%%%%%%%%%%%%%%%%%%%%%%%%%%%%%%%%%%%%%%%%
%%%%%%%%%%%%%%%%%%%%%%%%%%%%%%%%%%%%%%%%%%%%%%%%%%%%%%%%%%%%%%%%%%%%%%%%%%%%%%%%
\section{Information}

%%%%%%%%%%%%%%%%%%%%%%%%%%%%%%%%%%%%%%%%%%%%%%%%%%%%%%%%%%%%%%%%%%%%%%%%%%%%%%%%
\subsection{Copyright}

Copyright \copyright{} 2017--2018 Niklas Beisert

This work may be distributed and/or modified under the
conditions of the \LaTeX{} Project Public License, either version 1.3
of this license or (at your option) any later version.
The latest version of this license is in
  \url{http://www.latex-project.org/lppl.txt}
and version 1.3 or later is part of all distributions of \LaTeX{}
version 2005/12/01 or later.

This work has the LPPL maintenance status `maintained'.

The Current Maintainer of this work is Niklas Beisert.

This work consists of the files |README.txt|, |childdoc.ins| and |childdoc.dtx|
as well as the derived files |childdoc.def|, |cdocsamp.tex|
with |cdocsch1.tex|, |cdocsch2.tex|, |cdocspt3.tex|, |cdocspt4.tex|,
|cdocsdrf.tex|, |cdocsfn1.tex|, |cdocsfn2.tex|
as well as |childdoc.pdf|.

%%%%%%%%%%%%%%%%%%%%%%%%%%%%%%%%%%%%%%%%%%%%%%%%%%%%%%%%%%%%%%%%%%%%%%%%%%%%%%%%
\subsection{Files and Installation}

The package consists of the files:
%
\begin{center}
\begin{tabular}{ll}
    |README.txt|   & readme file \\
    |childdoc.ins| & installation file \\
    |childdoc.dtx| & source file \\
    |childdoc.def| & definition file \\
    |cdocsamp.tex| & sample main file \\
    |cdocsch1.tex| & sample include file \\
    |cdocsch2.tex| & sample include file \\
    |cdocspt3.tex| & sample part file \\
    |cdocspt4.tex| & sample part file \\
    |cdocsdrf.tex| & sample redirection file \\
    |cdocsfn1.tex| & sample redirection file \\
    |cdocsfn2.tex| & sample redirection file \\
    |childdoc.pdf| & manual
\end{tabular}
\end{center}
%
The distribution consists of the files
|README.txt|, |childdoc.ins| and |childdoc.dtx|.
%
\begin{itemize}
\item
Run (pdf)\LaTeX{} on |childdoc.dtx|
to compile the manual |childdoc.pdf| (this file).
\item
Run \LaTeX{} on |childdoc.ins| to create the definitions file |childdoc.def|
and the sample |cdocsamp.tex| with include files
|cdocsch1.tex|, |cdocsch2.tex|, |cdocspt3.tex|, |cdocspt4.tex|,
|cdocsdrf.tex|, |cdocsfn1.tex|, |cdocsfn2.tex|.
Then copy the file |childdoc.def| to an appropriate directory of your \LaTeX{}
distribution, e.g.\ \textit{texmf-root}|/tex/latex/childdoc|.
\end{itemize}

%%%%%%%%%%%%%%%%%%%%%%%%%%%%%%%%%%%%%%%%%%%%%%%%%%%%%%%%%%%%%%%%%%%%%%%%%%%%%%%%
\subsection{Related CTAN Packages}

There are several other packages which offer a similar functionality:
%
\begin{itemize}
\item
The packages
\href{http://ctan.org/pkg/docmute}{\textsf{docmute}},
\href{http://ctan.org/pkg/includex}{\textsf{includex}} and
\href{http://ctan.org/pkg/standalone}{\textsf{standalone}}
provide commands to include only the document body of
a child file thus allowing both files to be compiled individually.
\item
The packages \href{http://ctan.org/pkg/subdocs}{\textsf{subdocs}}
and \href{http://ctan.org/pkg/subfiles}{\textsf{subfiles}}
provide structures in which the main and child documents can be
encapsulated and allowing them to be compiled individually.
The inclusion mechanism is different from the conventional |\include|.
\item
The package \href{http://ctan.org/pkg/combine}{\textsf{combine}}
is an elaborate solution to combine several documents into one.
\end{itemize}
%
See also the CTAN topic \href{http://ctan.org/topic/subdocs}{\textsf{subdocs}}
for further related packages.
The present package differs from the above solutions in that
a document structure constructed with the conventional |\include| mechanism
just needs two extra commands at the top of every file
such that all constituent files can be compiled individually.

%%%%%%%%%%%%%%%%%%%%%%%%%%%%%%%%%%%%%%%%%%%%%%%%%%%%%%%%%%%%%%%%%%%%%%%%%%%%%%%%
%\subsection{Feature Suggestions}
%
%The following is a list of features which may be useful for future
%versions of this package:
%%
%\begin{itemize}
%\item
%\ldots
%\end{itemize}

%%%%%%%%%%%%%%%%%%%%%%%%%%%%%%%%%%%%%%%%%%%%%%%%%%%%%%%%%%%%%%%%%%%%%%%%%%%%%%%%
\subsection{Revision History}

%%%%%%%%%%%%%%%%%%%%%%%%%%%%%%%%%%%%%%%%
\paragraph{v2.0:} 2018/12/30

\begin{itemize}
\item
immediate forward processing
\item
added |\childdocby| mechanism
\item
manual restructured
\end{itemize}

%%%%%%%%%%%%%%%%%%%%%%%%%%%%%%%%%%%%%%%%
\paragraph{v1.6:} 2018/01/17

\begin{itemize}
\item
application for development of include files
\item
corrections to manual
\end{itemize}

%%%%%%%%%%%%%%%%%%%%%%%%%%%%%%%%%%%%%%%%
\paragraph{v1.5:} 2017/05/21

\begin{itemize}
\item
more complete structuring introduced
\item
|\childdocof| introduced
\item
|\childdoc| renamed to |\childdocmain|
\item
|\childredirect| renamed to |\childdocforward| and |\childdocforwardprefix|
and functionality expanded
\end{itemize}

%%%%%%%%%%%%%%%%%%%%%%%%%%%%%%%%%%%%%%%%
\paragraph{v1.0:} 2017/04/27

\begin{itemize}
\item
manual and install package
\item
first version published on CTAN
\end{itemize}

%%%%%%%%%%%%%%%%%%%%%%%%%%%%%%%%%%%%%%%%
\paragraph{v0.6:} 2017/04/26

\begin{itemize}
\item
redirection mechanism added
\end{itemize}

%%%%%%%%%%%%%%%%%%%%%%%%%%%%%%%%%%%%%%%%
\paragraph{v0.5:} 2017/04/26

\begin{itemize}
\item
functionality in definition file
\end{itemize}


%%%%%%%%%%%%%%%%%%%%%%%%%%%%%%%%%%%%%%%%%%%%%%%%%%%%%%%%%%%%%%%%%%%%%%%%%%%%%%%%
%%%%%%%%%%%%%%%%%%%%%%%%%%%%%%%%%%%%%%%%%%%%%%%%%%%%%%%%%%%%%%%%%%%%%%%%%%%%%%%%
%%%%%%%%%%%%%%%%%%%%%%%%%%%%%%%%%%%%%%%%%%%%%%%%%%%%%%%%%%%%%%%%%%%%%%%%%%%%%%%%
\appendix

\settowidth\MacroIndent{\rmfamily\scriptsize 000\ }

 \DocInput{childdoc.dtx}

\end{document}
%</driver>
% \fi
%
% %%%%%%%%%%%%%%%%%%%%%%%%%%%%%%%%%%%%%%%%%%%%%%%%%%%%%%%%%%%%%%%%%%%%%%%%%%%%%%
% %%%%%%%%%%%%%%%%%%%%%%%%%%%%%%%%%%%%%%%%%%%%%%%%%%%%%%%%%%%%%%%%%%%%%%%%%%%%%%
% \section{Sample}
%\iffalse
%<*samplemain>
%\fi
%
% The following presents a sample document
% with two chapters, two parts, a title page,
% a compile flag as well as three forwarding files to set the flag.
% It consists of eight |.tex| files:
% \begin{center}
% \begin{tabular}{ll}
% |cdocsamp.tex|&main file\\
% |cdocsch1.tex|&include file for chapter 1\\
% |cdocsch2.tex|&include file for chapter 2\\
% |cdocspt3.tex|&include file for part 3\\
% |cdocspt4.tex|&include file for part 4\\
% |cdocsdrf.tex|&forwarding file for main file in draft mode\\
% |cdocsfi1.tex|&forwarding file for final version of chapter 1\\
% |cdocsfi2.tex|&forwarding file for final version of chapter 2\\
% \end{tabular}
% \end{center}
% Each of the eight files can be compiled directly by the \LaTeX{} compiler.
%
% %%%%%%%%%%%%%%%%%%%%%%%%%%%%%%%%%%%%%%
% \paragraph{Main File.}
%
% The main file is called |cdocsamp.tex|.
%
% Load the \textsf{childdoc} definitions and
% declare the filename for the main document:
%    \begin{macrocode}
\input{childdoc.def}
\childdocmain{}
%    \end{macrocode}

% Optional override for |\version| flag:
%    \begin{macrocode}
%%\ifchilddoc\else\providecommand{\version}{draft}\fi
%    \end{macrocode}

% Define the default values for the |\version| flag
% (|final| for the main file and |draft| for childs):
%    \begin{macrocode}
\ifchilddoc
\providecommand{\version}{draft}
\else
\providecommand{\version}{final}
\fi
%    \end{macrocode}

% Load the standard document class:
%    \begin{macrocode}
\documentclass[12pt]{article}
%    \end{macrocode}

% Start the document body:
%    \begin{macrocode}
\begin{document}
%    \end{macrocode}

% Declare a title page.
% Print title, part of document being processed and version flag:
%    \begin{macrocode}
\addtocounter{page}{-1}
\begin{center}
{\LARGE\bfseries{}childdoc example\par}
\vspace{1cm}
\ifchilddoc
\ifchilddocmanual part\else chapter\fi:
`\childdocname' of `\childdocjob'\par
\else
main document: `\childdocjob'\par
\fi
version: \version\par
\end{center}
\newpage
%    \end{macrocode}

% Manually include selected file,
% otherwise process as usual:
%    \begin{macrocode}
\ifchilddocmanual
\section*{part `\childdocname'}
\input{\childdocname}
\else
%    \end{macrocode}

% Include the two chapters:
%    \begin{macrocode}
\include{cdocsch1}
\include{cdocsch2}
%    \end{macrocode}

% Include the two parts unless only chapters should be displayed:
%    \begin{macrocode}
\ifchilddoc\else
\section{part three}
\input{cdocspt3}
\section{part four}
\input{cdocspt4}
\fi
%    \end{macrocode}

% Process as usual until here:
%    \begin{macrocode}
\fi
%    \end{macrocode}

% End of document body:
%    \begin{macrocode}
\end{document}
%    \end{macrocode}
%\iffalse
%</samplemain>
%\fi
%
% %%%%%%%%%%%%%%%%%%%%%%%%%%%%%%%%%%%%%%
% \paragraph{Chapter Include Files.}
%
% The include files are called |cdocsch1.tex| and |cdocsch2.tex|.
%
%\iffalse
%<*samplechap1|samplechap2>
%\fi

% Optional override for |\version| flag:
%    \begin{macrocode}
%%\providecommand{\version}{final}
%    \end{macrocode}

% Include the main document:
%    \begin{macrocode}
\input{childdoc.def}
\childdocof{cdocsamp}
%    \end{macrocode}

%\iffalse
%</samplechap1|samplechap2>
%\fi
%
%\iffalse
%<*samplechap1>
%\fi
% Some text for chapter 1:
%    \begin{macrocode}
\section{one}
some text in chapter one
%    \end{macrocode}

%\iffalse
%</samplechap1>
%\fi
% Some text for chapter 2:
%\iffalse
%<*samplechap2>
%\fi
%    \begin{macrocode}
\section{two}
more text in chapter two
%    \end{macrocode}

%\iffalse
%</samplechap2>
%\fi
%
% %%%%%%%%%%%%%%%%%%%%%%%%%%%%%%%%%%%%%%
% \paragraph{Part Include Files.}
%
% The include files are called |cdocspt3.tex| and |cdocspt4.tex|.
%
%\iffalse
%<*samplepart3|samplepart4>
%\fi

% Optional override for |\version| flag:
%    \begin{macrocode}
%%\providecommand{\version}{final}
%    \end{macrocode}

% Include the main document:
%    \begin{macrocode}
\input{childdoc.def}
\childdocby{cdocsamp}
%    \end{macrocode}

%\iffalse
%</samplepart3|samplepart4>
%\fi
%
%\iffalse
%<*samplepart3>
%\fi
% Some text for part 3:
%    \begin{macrocode}
some text in part three
%    \end{macrocode}

%\iffalse
%</samplepart3>
%\fi
% Some text for part 4:
%\iffalse
%<*samplepart4>
%\fi
%    \begin{macrocode}
more text in part four
%    \end{macrocode}

%\iffalse
%</samplepart4>
%\fi
%
% %%%%%%%%%%%%%%%%%%%%%%%%%%%%%%%%%%%%%%
% \paragraph{Forwarding for a Complete Draft.}
%
% The following forwarding file |cdocsdrf.tex|
% compiles the main document in draft mode:
%\iffalse
%<*sampledraft>
%\fi
%    \begin{macrocode}
\def\version{draft}
\input{childdoc.def}
\childdocforward{cdocsamp}
%    \end{macrocode}

%\iffalse
%</sampledraft>
%\fi
%
% %%%%%%%%%%%%%%%%%%%%%%%%%%%%%%%%%%%%%%
% \paragraph{Forwarding for Final Version of the Chapters.}
%
% The following forwarding files |cdocsfn1.tex| and |cdocsfn2.tex|
% (with identical content)
% compile the final versions of the child documents
% |cdocsch1.tex| and |cdocsch2.tex|, respectively:
%\iffalse
%<*samplefinal>
%\fi
%    \begin{macrocode}
\def\version{final}
\input{childdoc.def}
\childdocforwardprefix[cdocsamp]{cdocsfn}{cdocsch}
%    \end{macrocode}

%\iffalse
%</samplefinal>
%\fi
%
% %%%%%%%%%%%%%%%%%%%%%%%%%%%%%%%%%%%%%%
% \paragraph{Command Line Processing.}
%
% The following three command lines generate the output files
% |cdocscld|, |cdocscl1| and |cdocscl2|
% which should be identical to
% |cdocsdrf|, |cdocsch1| and |cdocsfn2|, respectively:
% \begin{center}
% \begin{tabular}{l}
% |latex -jobname cdocscld \|\\
% |  "\def\version{draft}\input{childdoc.def}\childdocforward{cdocsamp}"|\\
% |latex -jobname cdocscl1 \|\\
% |  "\input{childdoc.def}\childdocforward[cdocsamp]{cdocsch1}"|\\
% |latex -jobname cdocscl2 \|\\
% |  "\def\version{final}\input{childdoc.def}\childdocforward{cdocsch2}"|
% \end{tabular}
% \end{center}
% Note that the trailing backslash on each first line
% merely continues the input to the second line
% (for convenient cut ant paste).
% Furthermore, the command |latex| can be replaced by any
% of its alternative versions such as |pdflatex|.
%
% %%%%%%%%%%%%%%%%%%%%%%%%%%%%%%%%%%%%%%%%%%%%%%%%%%%%%%%%%%%%%%%%%%%%%%%%%%%%%%
% %%%%%%%%%%%%%%%%%%%%%%%%%%%%%%%%%%%%%%%%%%%%%%%%%%%%%%%%%%%%%%%%%%%%%%%%%%%%%%
% \section{Implementation}
%\iffalse
%<*package>
%\fi
%
% This section describes the definitions file |childdoc.def|.

% The definitions cannot be loaded using |\usepackage| or |\RequirePackage|
% which has a mechanism to prevent loading a style file more than once.
% When loading the definitions by means of |\input|
% multiple instances have to be prevented manually:
%\iffalse
%This code needs to be before the `\ProvidesFile' directive
%which is defined at the beginning of this file.
%Therefore it is also placed there and commented out here.
%</package>
%<*discard>
%\fi
%    \begin{macrocode}
\ifdefined\childdocmain\endinput\fi
%    \end{macrocode}
%\iffalse
%</discard>
%<*package>
%\fi
%
% \macro{\ifchilddoc}
% \macro{\ifchilddocmanual}
% The conditional |\ifchilddoc| tells whether a
% child (true) or main (false) document is being compiled.
% The conditional |\ifchilddocmanual| tells whether
% the |\includeonly| mechanism is used (false) or
% the selection of child files must be performed manually (true).
% The definitions initialise to false:
%    \begin{macrocode}
\newif\ifchilddoc
\newif\ifchilddocmanual
%    \end{macrocode}

% \macro{\childdocname}
% \macro{\childdocjob}
% The macro |\childdocname| stores the name of the main document
% to be compiled. The macro |\childdocjob| stores the name of
% the document on which the \LaTeX{} compiler was originally invoked.
% The content of |\jobname| cannot be compared
% to filenames specified in the source due to different catcodes.
% The following code rescans |\jobname|, stores the result
% in |\childdocname| and saves a copy in |\childdocjob|:
%    \begin{macrocode}
\edef\childdocname{\scantokens\expandafter{\jobname\noexpand}}
\let\childdocjob\childdocname
%    \end{macrocode}

% \macro{\childdocdisable}
% The macro |\childdocdisable| prevents the main file
% from being processed more than once.
% At this stage, the main document command |\childdocmain|
% is assumed to be called once again where it should do nothing.
% Any subsequent call to it should prevent
% a secondary processing of the main document
% It overwrites the forwarding commands
% |\childdocof| and |\childdocforward|
% with empty macros to prevent further inclusions of the main document:
%    \begin{macrocode}
\newcommand{\childdocdisable}
{
  \renewcommand{\childdocmain}[1]{\renewcommand{\childdocmain}[1]{\endinput}}
  \renewcommand{\childdocof}[1]{}
  \renewcommand{\childdocby}[2][]{}
  \renewcommand{\childdocforward}[2][]{}
  \renewcommand{\childdocdisable}{}
}
%    \end{macrocode}

% \macro{\childdocmain}
% The macro |\childdocmain| is to be called at the top of the main file
% with nothing or the main filename (without extension) as argument.
% First, it breaks loops.
% If the argument is not empty and does not match |\childdocname|
% (which is set by the first inclusion of |childdoc.def|),
% |\ifchilddoc| is set to true, |\includeonly| is applied to the child file
% and |\jobname| is set to the main file
% (for proper handling of |.aux| files):
%    \begin{macrocode}
\newcommand{\childdocmain}[1]
{
  \childdocdisable\childdocmain{}
  \if?#1?\else
    \begingroup
      \def\childdoctmp{#1}
      \ifx\childdoctmp\childdocname
        \def\childdoctmp{}
      \else
        \def\childdoctmp
        {
          \childdoctrue
          \includeonly{\childdocname}
          \def\childdocjob{#1}
          \def\jobname{#1}
        }
      \fi
      \expandafter
    \endgroup
    \childdoctmp
  \fi
}
%    \end{macrocode}

% \macro{\childdocof}
% The command |\childdocof| redirects
% compilation to the main file |#1|.
%    \begin{macrocode}
\newcommand{\childdocof}[1]
{
  \childdocdisable
  \childdoctrue
  \includeonly{\childdocname}
  \def\jobname{#1}
  \def\childdocjob{#1}
  \input{#1}
}
%    \end{macrocode}

% \macro{\childdocby}
% The command |\childdocby| ....
%    \begin{macrocode}
\newcommand{\childdocby}[2][]
{
  \childdocdisable
  \childdoctrue
  \childdocmanualtrue
  \if?#1?\else
    \def\jobname{#2}
  \fi
  \def\childdocjob{#2}
  \input{#2}
  \endinput
}
%    \end{macrocode}

% \macro{\childdocforward}
% The command |\childdocforward| redirects
% compilation to the main file or
% (if the optional argument is given) a child file.
% Parameters are set as if the main file
% or a child file starting with |\childdocof| was compiled.
% Then compilation is handed over to the main file:
%    \begin{macrocode}
\newcommand{\childdocforward}[2][]
{
  \begingroup
    \if?#1?
      \def\childdoctmp
      {
        \def\childdocname{#2}
        \def\childdocjob{#2}
        \def\jobname{#2}
        \input{#2}
        \endinput
      }
    \else
      \def\childdoctmp
      {
        \childdocdisable
        \def\childdocname{#2}
        \childdoctrue
        \includeonly{#2}
        \def\childdocjob{#1}
        \def\jobname{#1}
        \input{#1}
        \endinput
      }
    \fi
    \expandafter
  \endgroup
  \childdoctmp
}
%    \end{macrocode}

% \macro{\childdocforwardprefix}
% The command |\childdocforwardprefix| redirects
% compilation to the main or a child file by means of a pattern.
% The prefix |#1| in the current filename is replaced by |#2|
% and the suffix of the current filename is kept
% (it is assumed that the filename does not contain the substring `|~~~|'
% which is used as a delimiter).
% Compilation is handed over to the new file by |\childdocforward|:
%    \begin{macrocode}
\newcommand{\childdocforwardprefix}[3][]
{
  \begingroup
    \def\childdocextract #2##1~~~{\def\childdoctmp{\childdocforward[#1]{#3##1}}}
    \expandafter\childdocextract\childdocname~~~
    \expandafter
  \endgroup
  \childdoctmp
}
%    \end{macrocode}

% \macro{\childdoc}
% The deprecated macro |\childdoc| is a legacy version of |\childdocmain|:
%    \begin{macrocode}
\newcommand{\childdoc}{\childdocmain}
%    \end{macrocode}

% \macro{\childdocredirect}
% The deprecated macro |\childdocredirect| is a legacy version
% of |\childdocforward| and |\childdocforwardprefix|:
%    \begin{macrocode}
\newcommand{\childdocredirect}[2][]
{
  \begingroup
    \if?#1?
      \def\childdoctmp{\childdocforward{#2}}
    \else
      \def\childdoctmp{\childdocforwardprefix{#1}{#2}}
    \fi
    \expandafter
  \endgroup
  \childdoctmp
}
%    \end{macrocode}

%\iffalse
%</package>
%\fi
%
\endinput

\childdocforward{cdocsamp}
%    \end{macrocode}

%\iffalse
%</sampledraft>
%\fi
%
% %%%%%%%%%%%%%%%%%%%%%%%%%%%%%%%%%%%%%%
% \paragraph{Forwarding for Final Version of the Chapters.}
%
% The following forwarding files |cdocsfn1.tex| and |cdocsfn2.tex|
% (with identical content)
% compile the final versions of the child documents
% |cdocsch1.tex| and |cdocsch2.tex|, respectively:
%\iffalse
%<*samplefinal>
%\fi
%    \begin{macrocode}
\def\version{final}
% \iffalse
%
% childdoc.dtx Copyright (C) 2017-2018 Niklas Beisert
%
% This work may be distributed and/or modified under the
% conditions of the LaTeX Project Public License, either version 1.3
% of this license or (at your option) any later version.
% The latest version of this license is in
%   http://www.latex-project.org/lppl.txt
% and version 1.3 or later is part of all distributions of LaTeX
% version 2005/12/01 or later.
%
% This work has the LPPL maintenance status `maintained'.
%
% The Current Maintainer of this work is Niklas Beisert.
%
% This work consists of the files childdoc.dtx and childdoc.ins
% and the derived files childdoc.def and cdocsamp.tex with
% cdocsch1.tex, cdocsch2.tex, cdocsdrf.tex, cdocsfn1.tex, cdocsfn2.tex.
%
%<package>\ifdefined\childdocmain\endinput\fi
%<package>\ProvidesFile{childdoc.def}[2018/12/30 v2.0 child document driver]
%<samplemain>\ProvidesFile{cdocsamp.tex}[2018/12/30 v2.0 sample for childdoc]
%<*driver>
%\ProvidesFile{childdoc.drv}[2018/12/30 v2.0 childdoc reference manual file]
\PassOptionsToClass{10pt,a4paper}{article}
\documentclass{ltxdoc}

\usepackage[margin=35mm]{geometry}
\usepackage{hyperref}
\usepackage{hyperxmp}
\usepackage[usenames]{color}

\hypersetup{colorlinks=true}
\hypersetup{pdfstartview=FitH}
\hypersetup{pdfpagemode=UseNone}
\hypersetup{pdfsource={}}
\hypersetup{pdflang={en-UK}}
\hypersetup{pdfcopyright={Copyright 2017-2018 Niklas Beisert.
  This work may be distributed and/or modified under the
  conditions of the LaTeX Project Public License, either version 1.3
  of this license or (at your option) any later version.}}
\hypersetup{pdflicenseurl={http://www.latex-project.org/lppl.txt}}
\hypersetup{pdfcontactaddress={ETH Zurich, ITP, HIT K,
  Wolfgang-Pauli-Strasse 27}}
\hypersetup{pdfcontactpostcode={8093}}
\hypersetup{pdfcontactcity={Zurich}}
\hypersetup{pdfcontactcountry={Switzerland}}
\hypersetup{pdfcontactemail={nbeisert@itp.phys.ethz.ch}}
\hypersetup{pdfcontacturl={http://people.phys.ethz.ch/\xmptilde nbeisert/}}

\newcommand{\secref}[1]{\hyperref[#1]{section \ref*{#1}}}

\parskip1ex
\parindent0pt
\let\olditemize\itemize
\def\itemize{\olditemize\parskip0pt}

\begin{document}

\title{The \textsf{childdoc} Package}
\hypersetup{pdftitle={The childdoc Package}}
\author{Niklas Beisert\\[2ex]
  Institut f\"ur Theoretische Physik\\
  Eidgen\"ossische Technische Hochschule Z\"urich\\
  Wolfgang-Pauli-Strasse 27, 8093 Z\"urich, Switzerland\\[1ex]
  \href{mailto:nbeisert@itp.phys.ethz.ch}
  {\texttt{nbeisert@itp.phys.ethz.ch}}}
\hypersetup{pdfauthor={Niklas Beisert}}
\hypersetup{pdfsubject={Manual for the LaTeX2e Package childdoc}}
\date{30 December 2018, \textsf{v2.0}}
\maketitle

\begin{abstract}\noindent
\textsf{childdoc} is a \LaTeXe{} package
that enables the direct compilation
of document sections included by |\include|
to individual files.
\end{abstract}

\begingroup
\parskip0ex
\tableofcontents
\endgroup

%%%%%%%%%%%%%%%%%%%%%%%%%%%%%%%%%%%%%%%%%%%%%%%%%%%%%%%%%%%%%%%%%%%%%%%%%%%%%%%%
%%%%%%%%%%%%%%%%%%%%%%%%%%%%%%%%%%%%%%%%%%%%%%%%%%%%%%%%%%%%%%%%%%%%%%%%%%%%%%%%
\section{Introduction}

\LaTeX{} provides a mechanism to structure a large document (such as a book)
into a main file and several child files (containing the chapters)
using the |\include| command.
This mechanism is beneficial for documents
which span hundreds of pages in order to
make the source file(s) more manageable.
Moreover, compilation can be restricted to
selected child files by means of the |\includeonly| command.
The latter feature can be used to reduce the compilation time while editing
(this was significantly more useful in the earlier days of \LaTeX{})
or to generate a smaller document which is easier to navigate.
Another application of |\includeonly| is to generate
documents consisting of selected parts of the complete document.

However, there are a few drawbacks of the plain |\include| mechanism:
\begin{itemize}
\item
The child files cannot be compiled on their own,
they can only be compiled via the main file.
A naive editing environment
(such as a text editor with an option
to have the current file processed by \LaTeX)
may require one to switch to the main file before compiling;
attempting to compile the child file produces errors.
\item
The main file must be modified (each time)
to adjust the |\includeonly| command
to the present needs. This easily leaves the main file in a messy state.
\item
The generated document will always carry the filename
of the main document. This is inconvenient if
several child files are to be compiled and
to be kept for distribution.
\end{itemize}

The present package provides a simple interface
to make child files individually compilable by \LaTeX{}.
Compiling a child file then has the same effect as compiling
the main file with an |\includeonly| command
to select the appropriate child.
Moreover the generated document will carry the name of the child
rather than the main file.
This resolves all three above issues.

This feature is meant to make the editing of books,
thesis documents and lecture notes somewhat more convenient.
However, the package can also be used efficiently for
composing a series of documents (such as exercise sheets)
which are typically distributed individually.
It then assists the author in generating the individual documents
(potentially in different versions)
as well as a document containing the collected series.
Another application is in developing style files
or other kinds of included material
where compilation of the style file could redirect
to a sample or test file.

%%%%%%%%%%%%%%%%%%%%%%%%%%%%%%%%%%%%%%%%%%%%%%%%%%%%%%%%%%%%%%%%%%%%%%%%%%%%%%%%
%%%%%%%%%%%%%%%%%%%%%%%%%%%%%%%%%%%%%%%%%%%%%%%%%%%%%%%%%%%%%%%%%%%%%%%%%%%%%%%%
\section{Usage}

First of all, the package \textsf{childdoc} is \emph{not} a standard
\LaTeXe{} |.sty| style file! Therefore it needs to be invoked in
a non-standard way.

%%%%%%%%%%%%%%%%%%%%%%%%%%%%%%%%%%%%%%%%%%%%%%%%%%%%%%%%%%%%%%%%%%%%%%%%%%%%%%%%
\subsection{Included Files}
\label{sec:include}

%%%%%%%%%%%%%%%%%%%%%%%%%%%%%%%%%%%%%%%%
\DescribeMacro{\childdocmain}
To use the package, add the commands
\begin{center}
\begin{tabular}{l}
|\input{childdoc.def}|\\
|\childdocmain{}|\\
\end{tabular}
\end{center}
at the very top of the main \LaTeX{} file,
in particular \emph{before} the |\documentclass| statement!
The argument of |\childdocmain| should be left empty
(but it must be present).

%%%%%%%%%%%%%%%%%%%%%%%%%%%%%%%%%%%%%%%%
\DescribeMacro{\childdocof}
Furthermore, add the commands
\begin{center}
\begin{tabular}{l}
|\input{childdoc.def}|\\
|\childdocof{|\textit{main}|}|\\
\end{tabular}
\end{center}
at the top of every child file \textit{child}
which is included by |\include{|\textit{child}|}|
from within the main file
(or at least for those files to be compiled individually).
The argument \textit{main} must be the filename of the main file.

There are a couple of
considerations in setting up the main and child documents:

%%%%%%%%%%%%%%%%%%%%%%%%%%%%%%%%%%%%%%%%
\paragraph{Restrictions.}

Please note the following restrictions:
\begin{itemize}
\item
|\childdocmain| must be called with one argument \textit{main}
to ensure compatibility with earlier version of the package.
It must either be empty (|\childdocmain{}|)
or precisely match the filename of the main file in which it is specified.
See \secref{sec:detection} for further information.
\item
The filename \textit{main} must be specified without the |.tex| extension.
\item
The filename \textit{main} is case sensitive
(even in case-insensitive file systems)
due to internal string comparison.
\item
The argument \textit{main} should be fully expanded, it cannot be a macro.
\item
Subdirectories and special characters should be avoided in filenames.
\item
The command |\childdocmain{|\textit{main}|}| must be followed by a whitespace.
It should not be followed immediately by another command
or by a comment mark `|%|'.
This is because the \TeX{} parser reads the token immediately following
the argument of |\childdocmain| and puts it
at the beginning of every child section;
however, a white\-space is ignored.
\end{itemize}

%%%%%%%%%%%%%%%%%%%%%%%%%%%%%%%%%%%%%%%%
\paragraph{Content of Main File.}

It is advisable to place all content in the child files included by |\include|.
Any output contained in the main file will appear in all child documents
unless suppressed manually;
it cannot be suppressed automatically by the |\includeonly| directive
and thus should normally be avoided.
A method to include some content in the main file
by means of conditional processing is described in \secref{sec:conditional}.

%%%%%%%%%%%%%%%%%%%%%%%%%%%%%%%%%%%%%%%%
\paragraph{Page Numbering.}

When only a part of the document is compiled,
the appropriate numbering of pages
(as well as other status parameters)
is determined from the |.aux| files.
The latter contain information from previous passes.
However this information needs to propagate through
all intermediate child documents.
Therefore the page numbering in child documents may well
be inconsistent until the complete document is compiled at least once.

A useful (if unconventional) way to always ensure a consistent
page numbering is to restart the numbering in each child document
and denote the pages by `\textit{child}|.|\textit{page}'
where \textit{child} represents the chapter/section number of the child file.
This can be achieved by the command
|\numberwithin{page}{|\textit{child}|}|
of the \textsf{amsmath} package
where \textit{child} can be |chapter| or |section|
depending on the chosen structuring.
Alternatively, one can modify the macro |\thepage| appropriately
and reset the counter |page| at the start of each child file.

%%%%%%%%%%%%%%%%%%%%%%%%%%%%%%%%%%%%%%%%%%%%%%%%%%%%%%%%%%%%%%%%%%%%%%%%%%%%%%%%
\subsection{Conditional Processing}
\label{sec:conditional}

The package provides a mechanism to compile different versions
of a document. To customise the versions further some conditional processing
can come in handy to distinguish which version is being compiled.
The package provides two macros to describe the compilation context:

%%%%%%%%%%%%%%%%%%%%%%%%%%%%%%%%%%%%%%%%
\DescribeMacro{\ifchilddoc}
The conditional |\ifchilddoc| distinguishes between the compilation of
child documents and the main document:
%
\begin{center}
|\ifchilddoc |\textit{child-code}| |[|\||else |\textit{main-code}]| \||fi|
\end{center}

%%%%%%%%%%%%%%%%%%%%%%%%%%%%%%%%%%%%%%%%
\DescribeMacro{\childdocname}
\DescribeMacro{\childdocjob}
The macro |\childdocname| contains the filename (without extension)
of the main or child file being processed.
Note that |\childdocjob| will always contain the name of the main file.

%%%%%%%%%%%%%%%%%%%%%%%%%%%%%%%%%%%%%%%%
\paragraph{Title Page.}

Conditional processing can be used to include a title or banner page
in the main document when proper precautions are taken.
Importantly, the code in the main file should ensure that the page counter
(as well as other status parameters which are stored in the |.aux| files)
takes the same value after the conditional processing.
Otherwise the page numbers may take divergent values
depending on which part is compiled.

For example, a title page could be declared by:
%
\begin{center}
\begin{tabular}{l}
|\ifchilddoc\||else|\\
|\addtocounter{page}{-1}|\\
\textit{code for title page}\\
|\newpage|\\
|\||fi|
\end{tabular}
\end{center}
%
A banner page for the child documents can be generated by:
%
\begin{center}
\begin{tabular}{l}
|\ifchilddoc|\\
|\addtocounter{page}{-1}|\\
\textit{code for banner page}\\
|\newpage|\\
|\||fi|
\end{tabular}
\end{center}
%
Here one could write a message such as:
\begin{center}
|This is the part \childdocname{} of \childdocjob{}.|
\end{center}

%%%%%%%%%%%%%%%%%%%%%%%%%%%%%%%%%%%%%%%%%%%%%%%%%%%%%%%%%%%%%%%%%%%%%%%%%%%%%%%%
\subsection{Flags}
\label{sec:flags}

The package makes it easy to generate different versions
of the main or child documents.
To this end compilation flags can be defined
and assigned different default values.
They will be particularly useful in conjunction
with the forwarding mechanism described in \secref{sec:forward}.

For example, it may be useful to have a flag |\version|
which can be set to |draft| or |final|.
The document source will contain some conditional code
depending on the value of |\version|.
Suppose further, the flag should default to |final| for the main file
and to |draft| for child files
which is a natural assignment for editing the document.
This is achieved by placing the following code
in the preamble of the main document
(below the |\childdocmain| directive):
%
\begin{center}
\begin{tabular}{l}
|\ifchilddoc|\\
|\providecommand{\version}{draft}|\\
|\||else|\\
|\providecommand{\version}{final}|\\
|\||fi|
\end{tabular}
\end{center}
%
The definition by |\providecommand| makes sure
that previous definitions are not overwritten.
Further statements |\providecommand{\version}{...}|
can thus be added before the above code to override it.

For the main file, one might add a line
(between |\childdocmain| and the above block)
%
\begin{center}
|%\ifchilddoc\||else\providecommand{\version}{draft}\||fi|
\end{center}
%
which can be uncommented to produce a draft version.
Likewise one can add a line to the very top of a child file
(above the |\childdocof{|\textit{main}|}| directive)
%
\begin{center}
|%\providecommand{\version}{final}|
\end{center}
%
which can be uncommented to produce the final version of this child document.

%%%%%%%%%%%%%%%%%%%%%%%%%%%%%%%%%%%%%%%%%%%%%%%%%%%%%%%%%%%%%%%%%%%%%%%%%%%%%%%%
\subsection{Forwarding}
\label{sec:forward}

Different versions of the main or child documents
using compilation flags as described in \secref{sec:flags}
can be (permanently) stored in different files
for convenient compilation, viewing and distribution.
To this end, the package defines a command
to pass on compilation to a different file:

%%%%%%%%%%%%%%%%%%%%%%%%%%%%%%%%%%%%%%%%
\DescribeMacro{\childdocforward}
The command |\childdocforward| redirects processing to
another source file:
%
\begin{center}
\begin{tabular}{l}
|\input{childdoc.def}|\\
|\childdocforward[|\textit{main}|]{|\textit{dest}|}|\\
\end{tabular}
\end{center}
%
The argument \textit{dest} is the destination file
(without extension).
It should be the main file or one of the child files.
Note that further \textsf{childdoc} directives
such as |\childdocof| and |\childdocforward|
in the indicated file will be processed in this form.
The optional argument \textit{main}
passes on directly to the main file \textit{main}
while pretending to compile the child \textit{dest}.
This form behaves as if \textit{dest}
issues |\childdocof{|\textit{main}|}| right away,
and no further \textsf{childdoc} directives will be processed.

%%%%%%%%%%%%%%%%%%%%%%%%%%%%%%%%%%%%%%%%
\DescribeMacro{\...prefix}
In the alternative form |\childdocforwardprefix|,
%
\begin{center}
\begin{tabular}{l}
|\input{childdoc.def}|\\
|\childdocforwardprefix[|\textit{main}|]{|\textit{prefix}|}{|\textit{dest}|}|
\end{tabular}
\end{center}
%
the destination file is determined by a pattern
depending on the current file:
To make this work, the current file must be called
`{\textit{prefix}\hspace{0.2em}\textit{suffix}}'
with \textit{prefix} matching precisely the argument.
Processing is then passed on to the file
`{\textit{dest}\hspace{0.2em}\textit{suffix}}'.
Surely, the same effect is achieved by
directly specifying the
argument `{\textit{dest}\hspace{0.2em}\textit{suffix}}'
in the first form.
However, that requires to set up a different file
for each child. With the alternative form of the command
all these files can have exactly the same content
which simplifies setting them up and maintaining them.

For example, the following file |draft.tex|
with a compilation flag |\version| as described in \secref{sec:flags}
compiles the main document as a draft:
%
\begin{center}
\begin{tabular}{l}
|\def\version{draft}|\\
|\input{childdoc.def}|\\
|\childdocforward{|\textit{main}|}|
\end{tabular}
\end{center}
%
Likewise, the following files |final|\textit{nn}|.tex|
compile the final version of the child document
|child|\textit{nn}|.tex|:
%
\begin{center}
\begin{tabular}{l}
|\def\version{final}|\\
|\input{childdoc.def}|\\
|\childdocforwardprefix{final}{child}|
\end{tabular}
\end{center}
%

Note that when several versions of a main file and/or of each child file
are to be generated, it may be convenient to set up a |Makefile| or
shell script to automatise the process.

%%%%%%%%%%%%%%%%%%%%%%%%%%%%%%%%%%%%%%%%%%%%%%%%%%%%%%%%%%%%%%%%%%%%%%%%%%%%%%%%
\subsection{Command Line Processing}
\label{sec:commandline}

The effect of redirection files can also be achieved by invoking
the \LaTeX{} compiler with a more elaborate command line.
Most conveniently this should be done as part
of a shell script or a |Makefile|.

When using \textsf{childdoc} in the main file, the following
command lines effectively perform a redirection
(note that depending on the shell being used,
backslashes may have to be doubled: `|\|' $\to$ `|\\|'):
%
\begin{center}
|... -jobname "|\textit{target}|" |\\|"|[\textit{flags}]%
|\input{childdoc.def}\childdocforward[|\textit{main}|]{|\textit{dest}|}"|
\end{center}
%
Here \textit{target} is the name of the output file,
\textit{main} is the name of the main file
and \textit{dest} is the name of the main or child file to be processed
(all filenames without extensions).
The optional argument \textit{main} can be omitted
if \textit{main} matches \textit{dest}.
Optionally, compilation \textit{flags} can be defined via |\def| commands.
This command line makes the \TeX{} engine believe
it is compiling the file \textit{target}
whose content is specified as the latter parameter.
The provided code then forwards the processing to
\textit{main} or \textit{dest} as described in \secref{sec:forward}.

%%%%%%%%%%%%%%%%%%%%%%%%%%%%%%%%%%%%%%%%%%%%%%%%%%%%%%%%%%%%%%%%%%%%%%%%%%%%%%%%
\subsection{Include by Input}
\label{sec:input}

Including child documents by |\include| has some restrictions by design.
Most notably, the content of a child document always occupies
its own set of pages; pages cannot be shared between child documents.
Usually, this behaviour makes perfect sense
because each child document contain an essential part of the document.
However, in some situations it may be desirable to compose
a document from a collection of parts
without having mandatory page breaks between then.
For this case, the package
provides a mechanism to include parts
by |\input| which can also be processed individually.
However, by construction this mechanism
requires manual handling of the content to be output.

%%%%%%%%%%%%%%%%%%%%%%%%%%%%%%%%%%%%%%%%
\DescribeMacro{\ifchilddocmanual}
The main file should be prepared as usual, see \secref{sec:include}.
However, the document body must make a distinction
between processing of an individual part and of the main document, e.g.:
%
\begin{center}
\begin{tabular}{l}
|\ifchilddocmanual|\\
|\input{\childdocname}|\\
|\||else|\\
\textit{document body with }|\input{|\textit{part}|}|\\
|\||fi|
\end{tabular}
\end{center}
%
The conditional |\ifchilddocmanual| is true whenever
a part to be included by |\input| is being compiled,
and the name of the part is stored in |\childdocname|.

%%%%%%%%%%%%%%%%%%%%%%%%%%%%%%%%%%%%%%%%
\DescribeMacro{\childdocby}
Each part to be included by |\input| should start with:
%
\begin{center}
\begin{tabular}{l}
|\input{childdoc.def}|\\
|\childdocby{|\textit{main}|}|\\
\end{tabular}
\end{center}
%
The directive |\childdocby| is similar to |\childdocof|
described in \secref{sec:include},
but the subsequent selection of content must be done manually.
To that end, both |\ifchilddoc| and |\ifchilddocmanual|
will be true upon processing of a part,
and the name of the part is stored in |\childdocname|.
Note that |\jobname| will be set to the filename of the current part
so that each part receives an individual |.aux| file
that does not interfere with the |.aux| file(s) of the main document.
This behaviour can be altered by the alternative form
|\childdocby[*]{|\textit{main}|}| (with a non-empty optional argument)
which uses the |.aux| file of the main document
by setting |\jobname| to \textit{main}.

%%%%%%%%%%%%%%%%%%%%%%%%%%%%%%%%%%%%%%%%%%%%%%%%%%%%%%%%%%%%%%%%%%%%%%%%%%%%%%%%
\subsection{Driver Development}
\label{sec:driver}

The \textsf{childdoc} mechanism can also be use for the development
of definition files such as \LaTeX{} styles or classes.
This case differs from the above setup with multiple parts
included by |\include| in that no |\includeonly| should be invoked.
This can be achieved by starting the include file
(before |\ProvidesPackage|) with:
%
\begin{center}
\begin{tabular}{l}
|\input{childdoc.def}|\\
|\childdocforward{|\textit{main}|}|\\
\end{tabular}
\end{center}
%
or alternatively with:
%
\begin{center}
\begin{tabular}{l}
|\input{childdoc.def}|\\
|\childdocby{|\textit{main}|}|\\
\end{tabular}
\end{center}
%
Both forms have slightly different effects as described above.
The main file is prepared as usual, see \secref{sec:include}.

%%%%%%%%%%%%%%%%%%%%%%%%%%%%%%%%%%%%%%%%%%%%%%%%%%%%%%%%%%%%%%%%%%%%%%%%%%%%%%%%
\subsection{Legacy Detection}
\label{sec:detection}

The directive |\childdocmain| in the main file can detect
whether the complete document or merely a child is to be compiled
even without using the directive |\childdocof|.
This method is deprecated because it is less robust
and there is no compelling reason to use it;
it is merely provided for backward compatibility
and it may be removed in future versions.

If the detection mechanism is to be used,
it is mandatory to correctly specify
the filename of the main file as the argument of |\childdocmain|:
%
\begin{center}
\begin{tabular}{l}
|\input{childdoc.def}|\\
|\childdocmain{|\textit{main}|}|\\
\end{tabular}
\end{center}
%
If |\jobname| does not match the argument \textit{main} of |\childdocmain|,
it is assumed that |\jobname| points to the child file to be compiled.
When using |\childdocmain| with the main file specified as argument,
it suffices to start a child file
with just |\input{|\textit{main}|}|
without loading of the package and using |\childdocof|.
If instead all processing is done
with the appropriate \textsf{childdoc} directives,
the argument of \textit{main} of |\childdocmain| can be empty.

An alternative version of the command line processing described
in \secref{sec:commandline} using the detection mechanism reads:
%
\begin{center}
|... -jobname "|\textit{target}|" "|[\textit{flags}]%
[|\def\jobname{|\textit{dest}|}|]|\input{|\textit{main}|}"|
\end{center}

%%%%%%%%%%%%%%%%%%%%%%%%%%%%%%%%%%%%%%%%%%%%%%%%%%%%%%%%%%%%%%%%%%%%%%%%%%%%%%%%
\subsection{Manual Code}
\label{sec:manual}

In case one cannot be certain whether the definitions file |childdoc.def|
is installed on the target \TeX{} distribution
and one prefers not to ship it,
it is conceivable to paste a few relevant commands into the sources.

To that end, drop all statements |\input{childdoc.def}|
and perform the replacements as outlined below.
Instead of |\childdocmain{|\textit{main}|}| add the following code
to the top of the main file:
%
\begin{center}
\begin{tabular}{l}
|\||ifdefined\childdocname\endinput\||fi\newif\ifchilddoc|\\
|\edef\childdocname{\scantokens\expandafter{\jobname\noexpand}}|\\
|\def\childdocmain{|\textit{main}|}\||ifx\childdocmain\childdocname\||else|\\
|\childdoctrue\includeonly{\childdocname}\let\jobname\childdocmain\||fi|\\
\end{tabular}
\end{center}
%
Instead of |\childdocof{|\textit{main}|}| just include the main file
at the top of each child file:
%
\begin{center}
|\input{|\textit{main}|}|
\end{center}
%
A simple redirection |\childdocforward{|\textit{dest}|}| is achieved by:
%
\begin{center}
|\def\jobname{|\textit{dest}|}\input{\jobname}|
\end{center}
%
The redirection with prefix
|\childdocforwardprefix[|\textit{prefix}|]{|\textit{dest}|}|
is accomplished by:
%
\begin{center}
\begin{tabular}{l}
|{\edef\jobname{\scantokens\expandafter{\jobname\noexpand}}|\\
|\def\redirectjob |\textit{prefix}|#1~~~{\gdef\jobname{|\textit{dest}|#1}}|\\
|\expandafter\redirectjob\jobname~~~}\input{\jobname}|
\end{tabular}
\end{center}

In an alternative approach,
child documents can be compiled by a specific command line
without additional code or specific definitions:
%
\begin{center}
|... -jobname "|\textit{target}|" "|[\textit{flags}]%
|\includeonly{|\textit{dest}|}\input{|\textit{main}|}"|
\end{center}
%

%%%%%%%%%%%%%%%%%%%%%%%%%%%%%%%%%%%%%%%%%%%%%%%%%%%%%%%%%%%%%%%%%%%%%%%%%%%%%%%%
%%%%%%%%%%%%%%%%%%%%%%%%%%%%%%%%%%%%%%%%%%%%%%%%%%%%%%%%%%%%%%%%%%%%%%%%%%%%%%%%
\section{Information}

%%%%%%%%%%%%%%%%%%%%%%%%%%%%%%%%%%%%%%%%%%%%%%%%%%%%%%%%%%%%%%%%%%%%%%%%%%%%%%%%
\subsection{Copyright}

Copyright \copyright{} 2017--2018 Niklas Beisert

This work may be distributed and/or modified under the
conditions of the \LaTeX{} Project Public License, either version 1.3
of this license or (at your option) any later version.
The latest version of this license is in
  \url{http://www.latex-project.org/lppl.txt}
and version 1.3 or later is part of all distributions of \LaTeX{}
version 2005/12/01 or later.

This work has the LPPL maintenance status `maintained'.

The Current Maintainer of this work is Niklas Beisert.

This work consists of the files |README.txt|, |childdoc.ins| and |childdoc.dtx|
as well as the derived files |childdoc.def|, |cdocsamp.tex|
with |cdocsch1.tex|, |cdocsch2.tex|, |cdocspt3.tex|, |cdocspt4.tex|,
|cdocsdrf.tex|, |cdocsfn1.tex|, |cdocsfn2.tex|
as well as |childdoc.pdf|.

%%%%%%%%%%%%%%%%%%%%%%%%%%%%%%%%%%%%%%%%%%%%%%%%%%%%%%%%%%%%%%%%%%%%%%%%%%%%%%%%
\subsection{Files and Installation}

The package consists of the files:
%
\begin{center}
\begin{tabular}{ll}
    |README.txt|   & readme file \\
    |childdoc.ins| & installation file \\
    |childdoc.dtx| & source file \\
    |childdoc.def| & definition file \\
    |cdocsamp.tex| & sample main file \\
    |cdocsch1.tex| & sample include file \\
    |cdocsch2.tex| & sample include file \\
    |cdocspt3.tex| & sample part file \\
    |cdocspt4.tex| & sample part file \\
    |cdocsdrf.tex| & sample redirection file \\
    |cdocsfn1.tex| & sample redirection file \\
    |cdocsfn2.tex| & sample redirection file \\
    |childdoc.pdf| & manual
\end{tabular}
\end{center}
%
The distribution consists of the files
|README.txt|, |childdoc.ins| and |childdoc.dtx|.
%
\begin{itemize}
\item
Run (pdf)\LaTeX{} on |childdoc.dtx|
to compile the manual |childdoc.pdf| (this file).
\item
Run \LaTeX{} on |childdoc.ins| to create the definitions file |childdoc.def|
and the sample |cdocsamp.tex| with include files
|cdocsch1.tex|, |cdocsch2.tex|, |cdocspt3.tex|, |cdocspt4.tex|,
|cdocsdrf.tex|, |cdocsfn1.tex|, |cdocsfn2.tex|.
Then copy the file |childdoc.def| to an appropriate directory of your \LaTeX{}
distribution, e.g.\ \textit{texmf-root}|/tex/latex/childdoc|.
\end{itemize}

%%%%%%%%%%%%%%%%%%%%%%%%%%%%%%%%%%%%%%%%%%%%%%%%%%%%%%%%%%%%%%%%%%%%%%%%%%%%%%%%
\subsection{Related CTAN Packages}

There are several other packages which offer a similar functionality:
%
\begin{itemize}
\item
The packages
\href{http://ctan.org/pkg/docmute}{\textsf{docmute}},
\href{http://ctan.org/pkg/includex}{\textsf{includex}} and
\href{http://ctan.org/pkg/standalone}{\textsf{standalone}}
provide commands to include only the document body of
a child file thus allowing both files to be compiled individually.
\item
The packages \href{http://ctan.org/pkg/subdocs}{\textsf{subdocs}}
and \href{http://ctan.org/pkg/subfiles}{\textsf{subfiles}}
provide structures in which the main and child documents can be
encapsulated and allowing them to be compiled individually.
The inclusion mechanism is different from the conventional |\include|.
\item
The package \href{http://ctan.org/pkg/combine}{\textsf{combine}}
is an elaborate solution to combine several documents into one.
\end{itemize}
%
See also the CTAN topic \href{http://ctan.org/topic/subdocs}{\textsf{subdocs}}
for further related packages.
The present package differs from the above solutions in that
a document structure constructed with the conventional |\include| mechanism
just needs two extra commands at the top of every file
such that all constituent files can be compiled individually.

%%%%%%%%%%%%%%%%%%%%%%%%%%%%%%%%%%%%%%%%%%%%%%%%%%%%%%%%%%%%%%%%%%%%%%%%%%%%%%%%
%\subsection{Feature Suggestions}
%
%The following is a list of features which may be useful for future
%versions of this package:
%%
%\begin{itemize}
%\item
%\ldots
%\end{itemize}

%%%%%%%%%%%%%%%%%%%%%%%%%%%%%%%%%%%%%%%%%%%%%%%%%%%%%%%%%%%%%%%%%%%%%%%%%%%%%%%%
\subsection{Revision History}

%%%%%%%%%%%%%%%%%%%%%%%%%%%%%%%%%%%%%%%%
\paragraph{v2.0:} 2018/12/30

\begin{itemize}
\item
immediate forward processing
\item
added |\childdocby| mechanism
\item
manual restructured
\end{itemize}

%%%%%%%%%%%%%%%%%%%%%%%%%%%%%%%%%%%%%%%%
\paragraph{v1.6:} 2018/01/17

\begin{itemize}
\item
application for development of include files
\item
corrections to manual
\end{itemize}

%%%%%%%%%%%%%%%%%%%%%%%%%%%%%%%%%%%%%%%%
\paragraph{v1.5:} 2017/05/21

\begin{itemize}
\item
more complete structuring introduced
\item
|\childdocof| introduced
\item
|\childdoc| renamed to |\childdocmain|
\item
|\childredirect| renamed to |\childdocforward| and |\childdocforwardprefix|
and functionality expanded
\end{itemize}

%%%%%%%%%%%%%%%%%%%%%%%%%%%%%%%%%%%%%%%%
\paragraph{v1.0:} 2017/04/27

\begin{itemize}
\item
manual and install package
\item
first version published on CTAN
\end{itemize}

%%%%%%%%%%%%%%%%%%%%%%%%%%%%%%%%%%%%%%%%
\paragraph{v0.6:} 2017/04/26

\begin{itemize}
\item
redirection mechanism added
\end{itemize}

%%%%%%%%%%%%%%%%%%%%%%%%%%%%%%%%%%%%%%%%
\paragraph{v0.5:} 2017/04/26

\begin{itemize}
\item
functionality in definition file
\end{itemize}


%%%%%%%%%%%%%%%%%%%%%%%%%%%%%%%%%%%%%%%%%%%%%%%%%%%%%%%%%%%%%%%%%%%%%%%%%%%%%%%%
%%%%%%%%%%%%%%%%%%%%%%%%%%%%%%%%%%%%%%%%%%%%%%%%%%%%%%%%%%%%%%%%%%%%%%%%%%%%%%%%
%%%%%%%%%%%%%%%%%%%%%%%%%%%%%%%%%%%%%%%%%%%%%%%%%%%%%%%%%%%%%%%%%%%%%%%%%%%%%%%%
\appendix

\settowidth\MacroIndent{\rmfamily\scriptsize 000\ }

 \DocInput{childdoc.dtx}

\end{document}
%</driver>
% \fi
%
% %%%%%%%%%%%%%%%%%%%%%%%%%%%%%%%%%%%%%%%%%%%%%%%%%%%%%%%%%%%%%%%%%%%%%%%%%%%%%%
% %%%%%%%%%%%%%%%%%%%%%%%%%%%%%%%%%%%%%%%%%%%%%%%%%%%%%%%%%%%%%%%%%%%%%%%%%%%%%%
% \section{Sample}
%\iffalse
%<*samplemain>
%\fi
%
% The following presents a sample document
% with two chapters, two parts, a title page,
% a compile flag as well as three forwarding files to set the flag.
% It consists of eight |.tex| files:
% \begin{center}
% \begin{tabular}{ll}
% |cdocsamp.tex|&main file\\
% |cdocsch1.tex|&include file for chapter 1\\
% |cdocsch2.tex|&include file for chapter 2\\
% |cdocspt3.tex|&include file for part 3\\
% |cdocspt4.tex|&include file for part 4\\
% |cdocsdrf.tex|&forwarding file for main file in draft mode\\
% |cdocsfi1.tex|&forwarding file for final version of chapter 1\\
% |cdocsfi2.tex|&forwarding file for final version of chapter 2\\
% \end{tabular}
% \end{center}
% Each of the eight files can be compiled directly by the \LaTeX{} compiler.
%
% %%%%%%%%%%%%%%%%%%%%%%%%%%%%%%%%%%%%%%
% \paragraph{Main File.}
%
% The main file is called |cdocsamp.tex|.
%
% Load the \textsf{childdoc} definitions and
% declare the filename for the main document:
%    \begin{macrocode}
\input{childdoc.def}
\childdocmain{}
%    \end{macrocode}

% Optional override for |\version| flag:
%    \begin{macrocode}
%%\ifchilddoc\else\providecommand{\version}{draft}\fi
%    \end{macrocode}

% Define the default values for the |\version| flag
% (|final| for the main file and |draft| for childs):
%    \begin{macrocode}
\ifchilddoc
\providecommand{\version}{draft}
\else
\providecommand{\version}{final}
\fi
%    \end{macrocode}

% Load the standard document class:
%    \begin{macrocode}
\documentclass[12pt]{article}
%    \end{macrocode}

% Start the document body:
%    \begin{macrocode}
\begin{document}
%    \end{macrocode}

% Declare a title page.
% Print title, part of document being processed and version flag:
%    \begin{macrocode}
\addtocounter{page}{-1}
\begin{center}
{\LARGE\bfseries{}childdoc example\par}
\vspace{1cm}
\ifchilddoc
\ifchilddocmanual part\else chapter\fi:
`\childdocname' of `\childdocjob'\par
\else
main document: `\childdocjob'\par
\fi
version: \version\par
\end{center}
\newpage
%    \end{macrocode}

% Manually include selected file,
% otherwise process as usual:
%    \begin{macrocode}
\ifchilddocmanual
\section*{part `\childdocname'}
\input{\childdocname}
\else
%    \end{macrocode}

% Include the two chapters:
%    \begin{macrocode}
\include{cdocsch1}
\include{cdocsch2}
%    \end{macrocode}

% Include the two parts unless only chapters should be displayed:
%    \begin{macrocode}
\ifchilddoc\else
\section{part three}
\input{cdocspt3}
\section{part four}
\input{cdocspt4}
\fi
%    \end{macrocode}

% Process as usual until here:
%    \begin{macrocode}
\fi
%    \end{macrocode}

% End of document body:
%    \begin{macrocode}
\end{document}
%    \end{macrocode}
%\iffalse
%</samplemain>
%\fi
%
% %%%%%%%%%%%%%%%%%%%%%%%%%%%%%%%%%%%%%%
% \paragraph{Chapter Include Files.}
%
% The include files are called |cdocsch1.tex| and |cdocsch2.tex|.
%
%\iffalse
%<*samplechap1|samplechap2>
%\fi

% Optional override for |\version| flag:
%    \begin{macrocode}
%%\providecommand{\version}{final}
%    \end{macrocode}

% Include the main document:
%    \begin{macrocode}
\input{childdoc.def}
\childdocof{cdocsamp}
%    \end{macrocode}

%\iffalse
%</samplechap1|samplechap2>
%\fi
%
%\iffalse
%<*samplechap1>
%\fi
% Some text for chapter 1:
%    \begin{macrocode}
\section{one}
some text in chapter one
%    \end{macrocode}

%\iffalse
%</samplechap1>
%\fi
% Some text for chapter 2:
%\iffalse
%<*samplechap2>
%\fi
%    \begin{macrocode}
\section{two}
more text in chapter two
%    \end{macrocode}

%\iffalse
%</samplechap2>
%\fi
%
% %%%%%%%%%%%%%%%%%%%%%%%%%%%%%%%%%%%%%%
% \paragraph{Part Include Files.}
%
% The include files are called |cdocspt3.tex| and |cdocspt4.tex|.
%
%\iffalse
%<*samplepart3|samplepart4>
%\fi

% Optional override for |\version| flag:
%    \begin{macrocode}
%%\providecommand{\version}{final}
%    \end{macrocode}

% Include the main document:
%    \begin{macrocode}
\input{childdoc.def}
\childdocby{cdocsamp}
%    \end{macrocode}

%\iffalse
%</samplepart3|samplepart4>
%\fi
%
%\iffalse
%<*samplepart3>
%\fi
% Some text for part 3:
%    \begin{macrocode}
some text in part three
%    \end{macrocode}

%\iffalse
%</samplepart3>
%\fi
% Some text for part 4:
%\iffalse
%<*samplepart4>
%\fi
%    \begin{macrocode}
more text in part four
%    \end{macrocode}

%\iffalse
%</samplepart4>
%\fi
%
% %%%%%%%%%%%%%%%%%%%%%%%%%%%%%%%%%%%%%%
% \paragraph{Forwarding for a Complete Draft.}
%
% The following forwarding file |cdocsdrf.tex|
% compiles the main document in draft mode:
%\iffalse
%<*sampledraft>
%\fi
%    \begin{macrocode}
\def\version{draft}
\input{childdoc.def}
\childdocforward{cdocsamp}
%    \end{macrocode}

%\iffalse
%</sampledraft>
%\fi
%
% %%%%%%%%%%%%%%%%%%%%%%%%%%%%%%%%%%%%%%
% \paragraph{Forwarding for Final Version of the Chapters.}
%
% The following forwarding files |cdocsfn1.tex| and |cdocsfn2.tex|
% (with identical content)
% compile the final versions of the child documents
% |cdocsch1.tex| and |cdocsch2.tex|, respectively:
%\iffalse
%<*samplefinal>
%\fi
%    \begin{macrocode}
\def\version{final}
\input{childdoc.def}
\childdocforwardprefix[cdocsamp]{cdocsfn}{cdocsch}
%    \end{macrocode}

%\iffalse
%</samplefinal>
%\fi
%
% %%%%%%%%%%%%%%%%%%%%%%%%%%%%%%%%%%%%%%
% \paragraph{Command Line Processing.}
%
% The following three command lines generate the output files
% |cdocscld|, |cdocscl1| and |cdocscl2|
% which should be identical to
% |cdocsdrf|, |cdocsch1| and |cdocsfn2|, respectively:
% \begin{center}
% \begin{tabular}{l}
% |latex -jobname cdocscld \|\\
% |  "\def\version{draft}\input{childdoc.def}\childdocforward{cdocsamp}"|\\
% |latex -jobname cdocscl1 \|\\
% |  "\input{childdoc.def}\childdocforward[cdocsamp]{cdocsch1}"|\\
% |latex -jobname cdocscl2 \|\\
% |  "\def\version{final}\input{childdoc.def}\childdocforward{cdocsch2}"|
% \end{tabular}
% \end{center}
% Note that the trailing backslash on each first line
% merely continues the input to the second line
% (for convenient cut ant paste).
% Furthermore, the command |latex| can be replaced by any
% of its alternative versions such as |pdflatex|.
%
% %%%%%%%%%%%%%%%%%%%%%%%%%%%%%%%%%%%%%%%%%%%%%%%%%%%%%%%%%%%%%%%%%%%%%%%%%%%%%%
% %%%%%%%%%%%%%%%%%%%%%%%%%%%%%%%%%%%%%%%%%%%%%%%%%%%%%%%%%%%%%%%%%%%%%%%%%%%%%%
% \section{Implementation}
%\iffalse
%<*package>
%\fi
%
% This section describes the definitions file |childdoc.def|.

% The definitions cannot be loaded using |\usepackage| or |\RequirePackage|
% which has a mechanism to prevent loading a style file more than once.
% When loading the definitions by means of |\input|
% multiple instances have to be prevented manually:
%\iffalse
%This code needs to be before the `\ProvidesFile' directive
%which is defined at the beginning of this file.
%Therefore it is also placed there and commented out here.
%</package>
%<*discard>
%\fi
%    \begin{macrocode}
\ifdefined\childdocmain\endinput\fi
%    \end{macrocode}
%\iffalse
%</discard>
%<*package>
%\fi
%
% \macro{\ifchilddoc}
% \macro{\ifchilddocmanual}
% The conditional |\ifchilddoc| tells whether a
% child (true) or main (false) document is being compiled.
% The conditional |\ifchilddocmanual| tells whether
% the |\includeonly| mechanism is used (false) or
% the selection of child files must be performed manually (true).
% The definitions initialise to false:
%    \begin{macrocode}
\newif\ifchilddoc
\newif\ifchilddocmanual
%    \end{macrocode}

% \macro{\childdocname}
% \macro{\childdocjob}
% The macro |\childdocname| stores the name of the main document
% to be compiled. The macro |\childdocjob| stores the name of
% the document on which the \LaTeX{} compiler was originally invoked.
% The content of |\jobname| cannot be compared
% to filenames specified in the source due to different catcodes.
% The following code rescans |\jobname|, stores the result
% in |\childdocname| and saves a copy in |\childdocjob|:
%    \begin{macrocode}
\edef\childdocname{\scantokens\expandafter{\jobname\noexpand}}
\let\childdocjob\childdocname
%    \end{macrocode}

% \macro{\childdocdisable}
% The macro |\childdocdisable| prevents the main file
% from being processed more than once.
% At this stage, the main document command |\childdocmain|
% is assumed to be called once again where it should do nothing.
% Any subsequent call to it should prevent
% a secondary processing of the main document
% It overwrites the forwarding commands
% |\childdocof| and |\childdocforward|
% with empty macros to prevent further inclusions of the main document:
%    \begin{macrocode}
\newcommand{\childdocdisable}
{
  \renewcommand{\childdocmain}[1]{\renewcommand{\childdocmain}[1]{\endinput}}
  \renewcommand{\childdocof}[1]{}
  \renewcommand{\childdocby}[2][]{}
  \renewcommand{\childdocforward}[2][]{}
  \renewcommand{\childdocdisable}{}
}
%    \end{macrocode}

% \macro{\childdocmain}
% The macro |\childdocmain| is to be called at the top of the main file
% with nothing or the main filename (without extension) as argument.
% First, it breaks loops.
% If the argument is not empty and does not match |\childdocname|
% (which is set by the first inclusion of |childdoc.def|),
% |\ifchilddoc| is set to true, |\includeonly| is applied to the child file
% and |\jobname| is set to the main file
% (for proper handling of |.aux| files):
%    \begin{macrocode}
\newcommand{\childdocmain}[1]
{
  \childdocdisable\childdocmain{}
  \if?#1?\else
    \begingroup
      \def\childdoctmp{#1}
      \ifx\childdoctmp\childdocname
        \def\childdoctmp{}
      \else
        \def\childdoctmp
        {
          \childdoctrue
          \includeonly{\childdocname}
          \def\childdocjob{#1}
          \def\jobname{#1}
        }
      \fi
      \expandafter
    \endgroup
    \childdoctmp
  \fi
}
%    \end{macrocode}

% \macro{\childdocof}
% The command |\childdocof| redirects
% compilation to the main file |#1|.
%    \begin{macrocode}
\newcommand{\childdocof}[1]
{
  \childdocdisable
  \childdoctrue
  \includeonly{\childdocname}
  \def\jobname{#1}
  \def\childdocjob{#1}
  \input{#1}
}
%    \end{macrocode}

% \macro{\childdocby}
% The command |\childdocby| ....
%    \begin{macrocode}
\newcommand{\childdocby}[2][]
{
  \childdocdisable
  \childdoctrue
  \childdocmanualtrue
  \if?#1?\else
    \def\jobname{#2}
  \fi
  \def\childdocjob{#2}
  \input{#2}
  \endinput
}
%    \end{macrocode}

% \macro{\childdocforward}
% The command |\childdocforward| redirects
% compilation to the main file or
% (if the optional argument is given) a child file.
% Parameters are set as if the main file
% or a child file starting with |\childdocof| was compiled.
% Then compilation is handed over to the main file:
%    \begin{macrocode}
\newcommand{\childdocforward}[2][]
{
  \begingroup
    \if?#1?
      \def\childdoctmp
      {
        \def\childdocname{#2}
        \def\childdocjob{#2}
        \def\jobname{#2}
        \input{#2}
        \endinput
      }
    \else
      \def\childdoctmp
      {
        \childdocdisable
        \def\childdocname{#2}
        \childdoctrue
        \includeonly{#2}
        \def\childdocjob{#1}
        \def\jobname{#1}
        \input{#1}
        \endinput
      }
    \fi
    \expandafter
  \endgroup
  \childdoctmp
}
%    \end{macrocode}

% \macro{\childdocforwardprefix}
% The command |\childdocforwardprefix| redirects
% compilation to the main or a child file by means of a pattern.
% The prefix |#1| in the current filename is replaced by |#2|
% and the suffix of the current filename is kept
% (it is assumed that the filename does not contain the substring `|~~~|'
% which is used as a delimiter).
% Compilation is handed over to the new file by |\childdocforward|:
%    \begin{macrocode}
\newcommand{\childdocforwardprefix}[3][]
{
  \begingroup
    \def\childdocextract #2##1~~~{\def\childdoctmp{\childdocforward[#1]{#3##1}}}
    \expandafter\childdocextract\childdocname~~~
    \expandafter
  \endgroup
  \childdoctmp
}
%    \end{macrocode}

% \macro{\childdoc}
% The deprecated macro |\childdoc| is a legacy version of |\childdocmain|:
%    \begin{macrocode}
\newcommand{\childdoc}{\childdocmain}
%    \end{macrocode}

% \macro{\childdocredirect}
% The deprecated macro |\childdocredirect| is a legacy version
% of |\childdocforward| and |\childdocforwardprefix|:
%    \begin{macrocode}
\newcommand{\childdocredirect}[2][]
{
  \begingroup
    \if?#1?
      \def\childdoctmp{\childdocforward{#2}}
    \else
      \def\childdoctmp{\childdocforwardprefix{#1}{#2}}
    \fi
    \expandafter
  \endgroup
  \childdoctmp
}
%    \end{macrocode}

%\iffalse
%</package>
%\fi
%
\endinput

\childdocforwardprefix[cdocsamp]{cdocsfn}{cdocsch}
%    \end{macrocode}

%\iffalse
%</samplefinal>
%\fi
%
% %%%%%%%%%%%%%%%%%%%%%%%%%%%%%%%%%%%%%%
% \paragraph{Command Line Processing.}
%
% The following three command lines generate the output files
% |cdocscld|, |cdocscl1| and |cdocscl2|
% which should be identical to
% |cdocsdrf|, |cdocsch1| and |cdocsfn2|, respectively:
% \begin{center}
% \begin{tabular}{l}
% |latex -jobname cdocscld \|\\
% |  "\def\version{draft}% \iffalse
%
% childdoc.dtx Copyright (C) 2017-2018 Niklas Beisert
%
% This work may be distributed and/or modified under the
% conditions of the LaTeX Project Public License, either version 1.3
% of this license or (at your option) any later version.
% The latest version of this license is in
%   http://www.latex-project.org/lppl.txt
% and version 1.3 or later is part of all distributions of LaTeX
% version 2005/12/01 or later.
%
% This work has the LPPL maintenance status `maintained'.
%
% The Current Maintainer of this work is Niklas Beisert.
%
% This work consists of the files childdoc.dtx and childdoc.ins
% and the derived files childdoc.def and cdocsamp.tex with
% cdocsch1.tex, cdocsch2.tex, cdocsdrf.tex, cdocsfn1.tex, cdocsfn2.tex.
%
%<package>\ifdefined\childdocmain\endinput\fi
%<package>\ProvidesFile{childdoc.def}[2018/12/30 v2.0 child document driver]
%<samplemain>\ProvidesFile{cdocsamp.tex}[2018/12/30 v2.0 sample for childdoc]
%<*driver>
%\ProvidesFile{childdoc.drv}[2018/12/30 v2.0 childdoc reference manual file]
\PassOptionsToClass{10pt,a4paper}{article}
\documentclass{ltxdoc}

\usepackage[margin=35mm]{geometry}
\usepackage{hyperref}
\usepackage{hyperxmp}
\usepackage[usenames]{color}

\hypersetup{colorlinks=true}
\hypersetup{pdfstartview=FitH}
\hypersetup{pdfpagemode=UseNone}
\hypersetup{pdfsource={}}
\hypersetup{pdflang={en-UK}}
\hypersetup{pdfcopyright={Copyright 2017-2018 Niklas Beisert.
  This work may be distributed and/or modified under the
  conditions of the LaTeX Project Public License, either version 1.3
  of this license or (at your option) any later version.}}
\hypersetup{pdflicenseurl={http://www.latex-project.org/lppl.txt}}
\hypersetup{pdfcontactaddress={ETH Zurich, ITP, HIT K,
  Wolfgang-Pauli-Strasse 27}}
\hypersetup{pdfcontactpostcode={8093}}
\hypersetup{pdfcontactcity={Zurich}}
\hypersetup{pdfcontactcountry={Switzerland}}
\hypersetup{pdfcontactemail={nbeisert@itp.phys.ethz.ch}}
\hypersetup{pdfcontacturl={http://people.phys.ethz.ch/\xmptilde nbeisert/}}

\newcommand{\secref}[1]{\hyperref[#1]{section \ref*{#1}}}

\parskip1ex
\parindent0pt
\let\olditemize\itemize
\def\itemize{\olditemize\parskip0pt}

\begin{document}

\title{The \textsf{childdoc} Package}
\hypersetup{pdftitle={The childdoc Package}}
\author{Niklas Beisert\\[2ex]
  Institut f\"ur Theoretische Physik\\
  Eidgen\"ossische Technische Hochschule Z\"urich\\
  Wolfgang-Pauli-Strasse 27, 8093 Z\"urich, Switzerland\\[1ex]
  \href{mailto:nbeisert@itp.phys.ethz.ch}
  {\texttt{nbeisert@itp.phys.ethz.ch}}}
\hypersetup{pdfauthor={Niklas Beisert}}
\hypersetup{pdfsubject={Manual for the LaTeX2e Package childdoc}}
\date{30 December 2018, \textsf{v2.0}}
\maketitle

\begin{abstract}\noindent
\textsf{childdoc} is a \LaTeXe{} package
that enables the direct compilation
of document sections included by |\include|
to individual files.
\end{abstract}

\begingroup
\parskip0ex
\tableofcontents
\endgroup

%%%%%%%%%%%%%%%%%%%%%%%%%%%%%%%%%%%%%%%%%%%%%%%%%%%%%%%%%%%%%%%%%%%%%%%%%%%%%%%%
%%%%%%%%%%%%%%%%%%%%%%%%%%%%%%%%%%%%%%%%%%%%%%%%%%%%%%%%%%%%%%%%%%%%%%%%%%%%%%%%
\section{Introduction}

\LaTeX{} provides a mechanism to structure a large document (such as a book)
into a main file and several child files (containing the chapters)
using the |\include| command.
This mechanism is beneficial for documents
which span hundreds of pages in order to
make the source file(s) more manageable.
Moreover, compilation can be restricted to
selected child files by means of the |\includeonly| command.
The latter feature can be used to reduce the compilation time while editing
(this was significantly more useful in the earlier days of \LaTeX{})
or to generate a smaller document which is easier to navigate.
Another application of |\includeonly| is to generate
documents consisting of selected parts of the complete document.

However, there are a few drawbacks of the plain |\include| mechanism:
\begin{itemize}
\item
The child files cannot be compiled on their own,
they can only be compiled via the main file.
A naive editing environment
(such as a text editor with an option
to have the current file processed by \LaTeX)
may require one to switch to the main file before compiling;
attempting to compile the child file produces errors.
\item
The main file must be modified (each time)
to adjust the |\includeonly| command
to the present needs. This easily leaves the main file in a messy state.
\item
The generated document will always carry the filename
of the main document. This is inconvenient if
several child files are to be compiled and
to be kept for distribution.
\end{itemize}

The present package provides a simple interface
to make child files individually compilable by \LaTeX{}.
Compiling a child file then has the same effect as compiling
the main file with an |\includeonly| command
to select the appropriate child.
Moreover the generated document will carry the name of the child
rather than the main file.
This resolves all three above issues.

This feature is meant to make the editing of books,
thesis documents and lecture notes somewhat more convenient.
However, the package can also be used efficiently for
composing a series of documents (such as exercise sheets)
which are typically distributed individually.
It then assists the author in generating the individual documents
(potentially in different versions)
as well as a document containing the collected series.
Another application is in developing style files
or other kinds of included material
where compilation of the style file could redirect
to a sample or test file.

%%%%%%%%%%%%%%%%%%%%%%%%%%%%%%%%%%%%%%%%%%%%%%%%%%%%%%%%%%%%%%%%%%%%%%%%%%%%%%%%
%%%%%%%%%%%%%%%%%%%%%%%%%%%%%%%%%%%%%%%%%%%%%%%%%%%%%%%%%%%%%%%%%%%%%%%%%%%%%%%%
\section{Usage}

First of all, the package \textsf{childdoc} is \emph{not} a standard
\LaTeXe{} |.sty| style file! Therefore it needs to be invoked in
a non-standard way.

%%%%%%%%%%%%%%%%%%%%%%%%%%%%%%%%%%%%%%%%%%%%%%%%%%%%%%%%%%%%%%%%%%%%%%%%%%%%%%%%
\subsection{Included Files}
\label{sec:include}

%%%%%%%%%%%%%%%%%%%%%%%%%%%%%%%%%%%%%%%%
\DescribeMacro{\childdocmain}
To use the package, add the commands
\begin{center}
\begin{tabular}{l}
|\input{childdoc.def}|\\
|\childdocmain{}|\\
\end{tabular}
\end{center}
at the very top of the main \LaTeX{} file,
in particular \emph{before} the |\documentclass| statement!
The argument of |\childdocmain| should be left empty
(but it must be present).

%%%%%%%%%%%%%%%%%%%%%%%%%%%%%%%%%%%%%%%%
\DescribeMacro{\childdocof}
Furthermore, add the commands
\begin{center}
\begin{tabular}{l}
|\input{childdoc.def}|\\
|\childdocof{|\textit{main}|}|\\
\end{tabular}
\end{center}
at the top of every child file \textit{child}
which is included by |\include{|\textit{child}|}|
from within the main file
(or at least for those files to be compiled individually).
The argument \textit{main} must be the filename of the main file.

There are a couple of
considerations in setting up the main and child documents:

%%%%%%%%%%%%%%%%%%%%%%%%%%%%%%%%%%%%%%%%
\paragraph{Restrictions.}

Please note the following restrictions:
\begin{itemize}
\item
|\childdocmain| must be called with one argument \textit{main}
to ensure compatibility with earlier version of the package.
It must either be empty (|\childdocmain{}|)
or precisely match the filename of the main file in which it is specified.
See \secref{sec:detection} for further information.
\item
The filename \textit{main} must be specified without the |.tex| extension.
\item
The filename \textit{main} is case sensitive
(even in case-insensitive file systems)
due to internal string comparison.
\item
The argument \textit{main} should be fully expanded, it cannot be a macro.
\item
Subdirectories and special characters should be avoided in filenames.
\item
The command |\childdocmain{|\textit{main}|}| must be followed by a whitespace.
It should not be followed immediately by another command
or by a comment mark `|%|'.
This is because the \TeX{} parser reads the token immediately following
the argument of |\childdocmain| and puts it
at the beginning of every child section;
however, a white\-space is ignored.
\end{itemize}

%%%%%%%%%%%%%%%%%%%%%%%%%%%%%%%%%%%%%%%%
\paragraph{Content of Main File.}

It is advisable to place all content in the child files included by |\include|.
Any output contained in the main file will appear in all child documents
unless suppressed manually;
it cannot be suppressed automatically by the |\includeonly| directive
and thus should normally be avoided.
A method to include some content in the main file
by means of conditional processing is described in \secref{sec:conditional}.

%%%%%%%%%%%%%%%%%%%%%%%%%%%%%%%%%%%%%%%%
\paragraph{Page Numbering.}

When only a part of the document is compiled,
the appropriate numbering of pages
(as well as other status parameters)
is determined from the |.aux| files.
The latter contain information from previous passes.
However this information needs to propagate through
all intermediate child documents.
Therefore the page numbering in child documents may well
be inconsistent until the complete document is compiled at least once.

A useful (if unconventional) way to always ensure a consistent
page numbering is to restart the numbering in each child document
and denote the pages by `\textit{child}|.|\textit{page}'
where \textit{child} represents the chapter/section number of the child file.
This can be achieved by the command
|\numberwithin{page}{|\textit{child}|}|
of the \textsf{amsmath} package
where \textit{child} can be |chapter| or |section|
depending on the chosen structuring.
Alternatively, one can modify the macro |\thepage| appropriately
and reset the counter |page| at the start of each child file.

%%%%%%%%%%%%%%%%%%%%%%%%%%%%%%%%%%%%%%%%%%%%%%%%%%%%%%%%%%%%%%%%%%%%%%%%%%%%%%%%
\subsection{Conditional Processing}
\label{sec:conditional}

The package provides a mechanism to compile different versions
of a document. To customise the versions further some conditional processing
can come in handy to distinguish which version is being compiled.
The package provides two macros to describe the compilation context:

%%%%%%%%%%%%%%%%%%%%%%%%%%%%%%%%%%%%%%%%
\DescribeMacro{\ifchilddoc}
The conditional |\ifchilddoc| distinguishes between the compilation of
child documents and the main document:
%
\begin{center}
|\ifchilddoc |\textit{child-code}| |[|\||else |\textit{main-code}]| \||fi|
\end{center}

%%%%%%%%%%%%%%%%%%%%%%%%%%%%%%%%%%%%%%%%
\DescribeMacro{\childdocname}
\DescribeMacro{\childdocjob}
The macro |\childdocname| contains the filename (without extension)
of the main or child file being processed.
Note that |\childdocjob| will always contain the name of the main file.

%%%%%%%%%%%%%%%%%%%%%%%%%%%%%%%%%%%%%%%%
\paragraph{Title Page.}

Conditional processing can be used to include a title or banner page
in the main document when proper precautions are taken.
Importantly, the code in the main file should ensure that the page counter
(as well as other status parameters which are stored in the |.aux| files)
takes the same value after the conditional processing.
Otherwise the page numbers may take divergent values
depending on which part is compiled.

For example, a title page could be declared by:
%
\begin{center}
\begin{tabular}{l}
|\ifchilddoc\||else|\\
|\addtocounter{page}{-1}|\\
\textit{code for title page}\\
|\newpage|\\
|\||fi|
\end{tabular}
\end{center}
%
A banner page for the child documents can be generated by:
%
\begin{center}
\begin{tabular}{l}
|\ifchilddoc|\\
|\addtocounter{page}{-1}|\\
\textit{code for banner page}\\
|\newpage|\\
|\||fi|
\end{tabular}
\end{center}
%
Here one could write a message such as:
\begin{center}
|This is the part \childdocname{} of \childdocjob{}.|
\end{center}

%%%%%%%%%%%%%%%%%%%%%%%%%%%%%%%%%%%%%%%%%%%%%%%%%%%%%%%%%%%%%%%%%%%%%%%%%%%%%%%%
\subsection{Flags}
\label{sec:flags}

The package makes it easy to generate different versions
of the main or child documents.
To this end compilation flags can be defined
and assigned different default values.
They will be particularly useful in conjunction
with the forwarding mechanism described in \secref{sec:forward}.

For example, it may be useful to have a flag |\version|
which can be set to |draft| or |final|.
The document source will contain some conditional code
depending on the value of |\version|.
Suppose further, the flag should default to |final| for the main file
and to |draft| for child files
which is a natural assignment for editing the document.
This is achieved by placing the following code
in the preamble of the main document
(below the |\childdocmain| directive):
%
\begin{center}
\begin{tabular}{l}
|\ifchilddoc|\\
|\providecommand{\version}{draft}|\\
|\||else|\\
|\providecommand{\version}{final}|\\
|\||fi|
\end{tabular}
\end{center}
%
The definition by |\providecommand| makes sure
that previous definitions are not overwritten.
Further statements |\providecommand{\version}{...}|
can thus be added before the above code to override it.

For the main file, one might add a line
(between |\childdocmain| and the above block)
%
\begin{center}
|%\ifchilddoc\||else\providecommand{\version}{draft}\||fi|
\end{center}
%
which can be uncommented to produce a draft version.
Likewise one can add a line to the very top of a child file
(above the |\childdocof{|\textit{main}|}| directive)
%
\begin{center}
|%\providecommand{\version}{final}|
\end{center}
%
which can be uncommented to produce the final version of this child document.

%%%%%%%%%%%%%%%%%%%%%%%%%%%%%%%%%%%%%%%%%%%%%%%%%%%%%%%%%%%%%%%%%%%%%%%%%%%%%%%%
\subsection{Forwarding}
\label{sec:forward}

Different versions of the main or child documents
using compilation flags as described in \secref{sec:flags}
can be (permanently) stored in different files
for convenient compilation, viewing and distribution.
To this end, the package defines a command
to pass on compilation to a different file:

%%%%%%%%%%%%%%%%%%%%%%%%%%%%%%%%%%%%%%%%
\DescribeMacro{\childdocforward}
The command |\childdocforward| redirects processing to
another source file:
%
\begin{center}
\begin{tabular}{l}
|\input{childdoc.def}|\\
|\childdocforward[|\textit{main}|]{|\textit{dest}|}|\\
\end{tabular}
\end{center}
%
The argument \textit{dest} is the destination file
(without extension).
It should be the main file or one of the child files.
Note that further \textsf{childdoc} directives
such as |\childdocof| and |\childdocforward|
in the indicated file will be processed in this form.
The optional argument \textit{main}
passes on directly to the main file \textit{main}
while pretending to compile the child \textit{dest}.
This form behaves as if \textit{dest}
issues |\childdocof{|\textit{main}|}| right away,
and no further \textsf{childdoc} directives will be processed.

%%%%%%%%%%%%%%%%%%%%%%%%%%%%%%%%%%%%%%%%
\DescribeMacro{\...prefix}
In the alternative form |\childdocforwardprefix|,
%
\begin{center}
\begin{tabular}{l}
|\input{childdoc.def}|\\
|\childdocforwardprefix[|\textit{main}|]{|\textit{prefix}|}{|\textit{dest}|}|
\end{tabular}
\end{center}
%
the destination file is determined by a pattern
depending on the current file:
To make this work, the current file must be called
`{\textit{prefix}\hspace{0.2em}\textit{suffix}}'
with \textit{prefix} matching precisely the argument.
Processing is then passed on to the file
`{\textit{dest}\hspace{0.2em}\textit{suffix}}'.
Surely, the same effect is achieved by
directly specifying the
argument `{\textit{dest}\hspace{0.2em}\textit{suffix}}'
in the first form.
However, that requires to set up a different file
for each child. With the alternative form of the command
all these files can have exactly the same content
which simplifies setting them up and maintaining them.

For example, the following file |draft.tex|
with a compilation flag |\version| as described in \secref{sec:flags}
compiles the main document as a draft:
%
\begin{center}
\begin{tabular}{l}
|\def\version{draft}|\\
|\input{childdoc.def}|\\
|\childdocforward{|\textit{main}|}|
\end{tabular}
\end{center}
%
Likewise, the following files |final|\textit{nn}|.tex|
compile the final version of the child document
|child|\textit{nn}|.tex|:
%
\begin{center}
\begin{tabular}{l}
|\def\version{final}|\\
|\input{childdoc.def}|\\
|\childdocforwardprefix{final}{child}|
\end{tabular}
\end{center}
%

Note that when several versions of a main file and/or of each child file
are to be generated, it may be convenient to set up a |Makefile| or
shell script to automatise the process.

%%%%%%%%%%%%%%%%%%%%%%%%%%%%%%%%%%%%%%%%%%%%%%%%%%%%%%%%%%%%%%%%%%%%%%%%%%%%%%%%
\subsection{Command Line Processing}
\label{sec:commandline}

The effect of redirection files can also be achieved by invoking
the \LaTeX{} compiler with a more elaborate command line.
Most conveniently this should be done as part
of a shell script or a |Makefile|.

When using \textsf{childdoc} in the main file, the following
command lines effectively perform a redirection
(note that depending on the shell being used,
backslashes may have to be doubled: `|\|' $\to$ `|\\|'):
%
\begin{center}
|... -jobname "|\textit{target}|" |\\|"|[\textit{flags}]%
|\input{childdoc.def}\childdocforward[|\textit{main}|]{|\textit{dest}|}"|
\end{center}
%
Here \textit{target} is the name of the output file,
\textit{main} is the name of the main file
and \textit{dest} is the name of the main or child file to be processed
(all filenames without extensions).
The optional argument \textit{main} can be omitted
if \textit{main} matches \textit{dest}.
Optionally, compilation \textit{flags} can be defined via |\def| commands.
This command line makes the \TeX{} engine believe
it is compiling the file \textit{target}
whose content is specified as the latter parameter.
The provided code then forwards the processing to
\textit{main} or \textit{dest} as described in \secref{sec:forward}.

%%%%%%%%%%%%%%%%%%%%%%%%%%%%%%%%%%%%%%%%%%%%%%%%%%%%%%%%%%%%%%%%%%%%%%%%%%%%%%%%
\subsection{Include by Input}
\label{sec:input}

Including child documents by |\include| has some restrictions by design.
Most notably, the content of a child document always occupies
its own set of pages; pages cannot be shared between child documents.
Usually, this behaviour makes perfect sense
because each child document contain an essential part of the document.
However, in some situations it may be desirable to compose
a document from a collection of parts
without having mandatory page breaks between then.
For this case, the package
provides a mechanism to include parts
by |\input| which can also be processed individually.
However, by construction this mechanism
requires manual handling of the content to be output.

%%%%%%%%%%%%%%%%%%%%%%%%%%%%%%%%%%%%%%%%
\DescribeMacro{\ifchilddocmanual}
The main file should be prepared as usual, see \secref{sec:include}.
However, the document body must make a distinction
between processing of an individual part and of the main document, e.g.:
%
\begin{center}
\begin{tabular}{l}
|\ifchilddocmanual|\\
|\input{\childdocname}|\\
|\||else|\\
\textit{document body with }|\input{|\textit{part}|}|\\
|\||fi|
\end{tabular}
\end{center}
%
The conditional |\ifchilddocmanual| is true whenever
a part to be included by |\input| is being compiled,
and the name of the part is stored in |\childdocname|.

%%%%%%%%%%%%%%%%%%%%%%%%%%%%%%%%%%%%%%%%
\DescribeMacro{\childdocby}
Each part to be included by |\input| should start with:
%
\begin{center}
\begin{tabular}{l}
|\input{childdoc.def}|\\
|\childdocby{|\textit{main}|}|\\
\end{tabular}
\end{center}
%
The directive |\childdocby| is similar to |\childdocof|
described in \secref{sec:include},
but the subsequent selection of content must be done manually.
To that end, both |\ifchilddoc| and |\ifchilddocmanual|
will be true upon processing of a part,
and the name of the part is stored in |\childdocname|.
Note that |\jobname| will be set to the filename of the current part
so that each part receives an individual |.aux| file
that does not interfere with the |.aux| file(s) of the main document.
This behaviour can be altered by the alternative form
|\childdocby[*]{|\textit{main}|}| (with a non-empty optional argument)
which uses the |.aux| file of the main document
by setting |\jobname| to \textit{main}.

%%%%%%%%%%%%%%%%%%%%%%%%%%%%%%%%%%%%%%%%%%%%%%%%%%%%%%%%%%%%%%%%%%%%%%%%%%%%%%%%
\subsection{Driver Development}
\label{sec:driver}

The \textsf{childdoc} mechanism can also be use for the development
of definition files such as \LaTeX{} styles or classes.
This case differs from the above setup with multiple parts
included by |\include| in that no |\includeonly| should be invoked.
This can be achieved by starting the include file
(before |\ProvidesPackage|) with:
%
\begin{center}
\begin{tabular}{l}
|\input{childdoc.def}|\\
|\childdocforward{|\textit{main}|}|\\
\end{tabular}
\end{center}
%
or alternatively with:
%
\begin{center}
\begin{tabular}{l}
|\input{childdoc.def}|\\
|\childdocby{|\textit{main}|}|\\
\end{tabular}
\end{center}
%
Both forms have slightly different effects as described above.
The main file is prepared as usual, see \secref{sec:include}.

%%%%%%%%%%%%%%%%%%%%%%%%%%%%%%%%%%%%%%%%%%%%%%%%%%%%%%%%%%%%%%%%%%%%%%%%%%%%%%%%
\subsection{Legacy Detection}
\label{sec:detection}

The directive |\childdocmain| in the main file can detect
whether the complete document or merely a child is to be compiled
even without using the directive |\childdocof|.
This method is deprecated because it is less robust
and there is no compelling reason to use it;
it is merely provided for backward compatibility
and it may be removed in future versions.

If the detection mechanism is to be used,
it is mandatory to correctly specify
the filename of the main file as the argument of |\childdocmain|:
%
\begin{center}
\begin{tabular}{l}
|\input{childdoc.def}|\\
|\childdocmain{|\textit{main}|}|\\
\end{tabular}
\end{center}
%
If |\jobname| does not match the argument \textit{main} of |\childdocmain|,
it is assumed that |\jobname| points to the child file to be compiled.
When using |\childdocmain| with the main file specified as argument,
it suffices to start a child file
with just |\input{|\textit{main}|}|
without loading of the package and using |\childdocof|.
If instead all processing is done
with the appropriate \textsf{childdoc} directives,
the argument of \textit{main} of |\childdocmain| can be empty.

An alternative version of the command line processing described
in \secref{sec:commandline} using the detection mechanism reads:
%
\begin{center}
|... -jobname "|\textit{target}|" "|[\textit{flags}]%
[|\def\jobname{|\textit{dest}|}|]|\input{|\textit{main}|}"|
\end{center}

%%%%%%%%%%%%%%%%%%%%%%%%%%%%%%%%%%%%%%%%%%%%%%%%%%%%%%%%%%%%%%%%%%%%%%%%%%%%%%%%
\subsection{Manual Code}
\label{sec:manual}

In case one cannot be certain whether the definitions file |childdoc.def|
is installed on the target \TeX{} distribution
and one prefers not to ship it,
it is conceivable to paste a few relevant commands into the sources.

To that end, drop all statements |\input{childdoc.def}|
and perform the replacements as outlined below.
Instead of |\childdocmain{|\textit{main}|}| add the following code
to the top of the main file:
%
\begin{center}
\begin{tabular}{l}
|\||ifdefined\childdocname\endinput\||fi\newif\ifchilddoc|\\
|\edef\childdocname{\scantokens\expandafter{\jobname\noexpand}}|\\
|\def\childdocmain{|\textit{main}|}\||ifx\childdocmain\childdocname\||else|\\
|\childdoctrue\includeonly{\childdocname}\let\jobname\childdocmain\||fi|\\
\end{tabular}
\end{center}
%
Instead of |\childdocof{|\textit{main}|}| just include the main file
at the top of each child file:
%
\begin{center}
|\input{|\textit{main}|}|
\end{center}
%
A simple redirection |\childdocforward{|\textit{dest}|}| is achieved by:
%
\begin{center}
|\def\jobname{|\textit{dest}|}\input{\jobname}|
\end{center}
%
The redirection with prefix
|\childdocforwardprefix[|\textit{prefix}|]{|\textit{dest}|}|
is accomplished by:
%
\begin{center}
\begin{tabular}{l}
|{\edef\jobname{\scantokens\expandafter{\jobname\noexpand}}|\\
|\def\redirectjob |\textit{prefix}|#1~~~{\gdef\jobname{|\textit{dest}|#1}}|\\
|\expandafter\redirectjob\jobname~~~}\input{\jobname}|
\end{tabular}
\end{center}

In an alternative approach,
child documents can be compiled by a specific command line
without additional code or specific definitions:
%
\begin{center}
|... -jobname "|\textit{target}|" "|[\textit{flags}]%
|\includeonly{|\textit{dest}|}\input{|\textit{main}|}"|
\end{center}
%

%%%%%%%%%%%%%%%%%%%%%%%%%%%%%%%%%%%%%%%%%%%%%%%%%%%%%%%%%%%%%%%%%%%%%%%%%%%%%%%%
%%%%%%%%%%%%%%%%%%%%%%%%%%%%%%%%%%%%%%%%%%%%%%%%%%%%%%%%%%%%%%%%%%%%%%%%%%%%%%%%
\section{Information}

%%%%%%%%%%%%%%%%%%%%%%%%%%%%%%%%%%%%%%%%%%%%%%%%%%%%%%%%%%%%%%%%%%%%%%%%%%%%%%%%
\subsection{Copyright}

Copyright \copyright{} 2017--2018 Niklas Beisert

This work may be distributed and/or modified under the
conditions of the \LaTeX{} Project Public License, either version 1.3
of this license or (at your option) any later version.
The latest version of this license is in
  \url{http://www.latex-project.org/lppl.txt}
and version 1.3 or later is part of all distributions of \LaTeX{}
version 2005/12/01 or later.

This work has the LPPL maintenance status `maintained'.

The Current Maintainer of this work is Niklas Beisert.

This work consists of the files |README.txt|, |childdoc.ins| and |childdoc.dtx|
as well as the derived files |childdoc.def|, |cdocsamp.tex|
with |cdocsch1.tex|, |cdocsch2.tex|, |cdocspt3.tex|, |cdocspt4.tex|,
|cdocsdrf.tex|, |cdocsfn1.tex|, |cdocsfn2.tex|
as well as |childdoc.pdf|.

%%%%%%%%%%%%%%%%%%%%%%%%%%%%%%%%%%%%%%%%%%%%%%%%%%%%%%%%%%%%%%%%%%%%%%%%%%%%%%%%
\subsection{Files and Installation}

The package consists of the files:
%
\begin{center}
\begin{tabular}{ll}
    |README.txt|   & readme file \\
    |childdoc.ins| & installation file \\
    |childdoc.dtx| & source file \\
    |childdoc.def| & definition file \\
    |cdocsamp.tex| & sample main file \\
    |cdocsch1.tex| & sample include file \\
    |cdocsch2.tex| & sample include file \\
    |cdocspt3.tex| & sample part file \\
    |cdocspt4.tex| & sample part file \\
    |cdocsdrf.tex| & sample redirection file \\
    |cdocsfn1.tex| & sample redirection file \\
    |cdocsfn2.tex| & sample redirection file \\
    |childdoc.pdf| & manual
\end{tabular}
\end{center}
%
The distribution consists of the files
|README.txt|, |childdoc.ins| and |childdoc.dtx|.
%
\begin{itemize}
\item
Run (pdf)\LaTeX{} on |childdoc.dtx|
to compile the manual |childdoc.pdf| (this file).
\item
Run \LaTeX{} on |childdoc.ins| to create the definitions file |childdoc.def|
and the sample |cdocsamp.tex| with include files
|cdocsch1.tex|, |cdocsch2.tex|, |cdocspt3.tex|, |cdocspt4.tex|,
|cdocsdrf.tex|, |cdocsfn1.tex|, |cdocsfn2.tex|.
Then copy the file |childdoc.def| to an appropriate directory of your \LaTeX{}
distribution, e.g.\ \textit{texmf-root}|/tex/latex/childdoc|.
\end{itemize}

%%%%%%%%%%%%%%%%%%%%%%%%%%%%%%%%%%%%%%%%%%%%%%%%%%%%%%%%%%%%%%%%%%%%%%%%%%%%%%%%
\subsection{Related CTAN Packages}

There are several other packages which offer a similar functionality:
%
\begin{itemize}
\item
The packages
\href{http://ctan.org/pkg/docmute}{\textsf{docmute}},
\href{http://ctan.org/pkg/includex}{\textsf{includex}} and
\href{http://ctan.org/pkg/standalone}{\textsf{standalone}}
provide commands to include only the document body of
a child file thus allowing both files to be compiled individually.
\item
The packages \href{http://ctan.org/pkg/subdocs}{\textsf{subdocs}}
and \href{http://ctan.org/pkg/subfiles}{\textsf{subfiles}}
provide structures in which the main and child documents can be
encapsulated and allowing them to be compiled individually.
The inclusion mechanism is different from the conventional |\include|.
\item
The package \href{http://ctan.org/pkg/combine}{\textsf{combine}}
is an elaborate solution to combine several documents into one.
\end{itemize}
%
See also the CTAN topic \href{http://ctan.org/topic/subdocs}{\textsf{subdocs}}
for further related packages.
The present package differs from the above solutions in that
a document structure constructed with the conventional |\include| mechanism
just needs two extra commands at the top of every file
such that all constituent files can be compiled individually.

%%%%%%%%%%%%%%%%%%%%%%%%%%%%%%%%%%%%%%%%%%%%%%%%%%%%%%%%%%%%%%%%%%%%%%%%%%%%%%%%
%\subsection{Feature Suggestions}
%
%The following is a list of features which may be useful for future
%versions of this package:
%%
%\begin{itemize}
%\item
%\ldots
%\end{itemize}

%%%%%%%%%%%%%%%%%%%%%%%%%%%%%%%%%%%%%%%%%%%%%%%%%%%%%%%%%%%%%%%%%%%%%%%%%%%%%%%%
\subsection{Revision History}

%%%%%%%%%%%%%%%%%%%%%%%%%%%%%%%%%%%%%%%%
\paragraph{v2.0:} 2018/12/30

\begin{itemize}
\item
immediate forward processing
\item
added |\childdocby| mechanism
\item
manual restructured
\end{itemize}

%%%%%%%%%%%%%%%%%%%%%%%%%%%%%%%%%%%%%%%%
\paragraph{v1.6:} 2018/01/17

\begin{itemize}
\item
application for development of include files
\item
corrections to manual
\end{itemize}

%%%%%%%%%%%%%%%%%%%%%%%%%%%%%%%%%%%%%%%%
\paragraph{v1.5:} 2017/05/21

\begin{itemize}
\item
more complete structuring introduced
\item
|\childdocof| introduced
\item
|\childdoc| renamed to |\childdocmain|
\item
|\childredirect| renamed to |\childdocforward| and |\childdocforwardprefix|
and functionality expanded
\end{itemize}

%%%%%%%%%%%%%%%%%%%%%%%%%%%%%%%%%%%%%%%%
\paragraph{v1.0:} 2017/04/27

\begin{itemize}
\item
manual and install package
\item
first version published on CTAN
\end{itemize}

%%%%%%%%%%%%%%%%%%%%%%%%%%%%%%%%%%%%%%%%
\paragraph{v0.6:} 2017/04/26

\begin{itemize}
\item
redirection mechanism added
\end{itemize}

%%%%%%%%%%%%%%%%%%%%%%%%%%%%%%%%%%%%%%%%
\paragraph{v0.5:} 2017/04/26

\begin{itemize}
\item
functionality in definition file
\end{itemize}


%%%%%%%%%%%%%%%%%%%%%%%%%%%%%%%%%%%%%%%%%%%%%%%%%%%%%%%%%%%%%%%%%%%%%%%%%%%%%%%%
%%%%%%%%%%%%%%%%%%%%%%%%%%%%%%%%%%%%%%%%%%%%%%%%%%%%%%%%%%%%%%%%%%%%%%%%%%%%%%%%
%%%%%%%%%%%%%%%%%%%%%%%%%%%%%%%%%%%%%%%%%%%%%%%%%%%%%%%%%%%%%%%%%%%%%%%%%%%%%%%%
\appendix

\settowidth\MacroIndent{\rmfamily\scriptsize 000\ }

 \DocInput{childdoc.dtx}

\end{document}
%</driver>
% \fi
%
% %%%%%%%%%%%%%%%%%%%%%%%%%%%%%%%%%%%%%%%%%%%%%%%%%%%%%%%%%%%%%%%%%%%%%%%%%%%%%%
% %%%%%%%%%%%%%%%%%%%%%%%%%%%%%%%%%%%%%%%%%%%%%%%%%%%%%%%%%%%%%%%%%%%%%%%%%%%%%%
% \section{Sample}
%\iffalse
%<*samplemain>
%\fi
%
% The following presents a sample document
% with two chapters, two parts, a title page,
% a compile flag as well as three forwarding files to set the flag.
% It consists of eight |.tex| files:
% \begin{center}
% \begin{tabular}{ll}
% |cdocsamp.tex|&main file\\
% |cdocsch1.tex|&include file for chapter 1\\
% |cdocsch2.tex|&include file for chapter 2\\
% |cdocspt3.tex|&include file for part 3\\
% |cdocspt4.tex|&include file for part 4\\
% |cdocsdrf.tex|&forwarding file for main file in draft mode\\
% |cdocsfi1.tex|&forwarding file for final version of chapter 1\\
% |cdocsfi2.tex|&forwarding file for final version of chapter 2\\
% \end{tabular}
% \end{center}
% Each of the eight files can be compiled directly by the \LaTeX{} compiler.
%
% %%%%%%%%%%%%%%%%%%%%%%%%%%%%%%%%%%%%%%
% \paragraph{Main File.}
%
% The main file is called |cdocsamp.tex|.
%
% Load the \textsf{childdoc} definitions and
% declare the filename for the main document:
%    \begin{macrocode}
\input{childdoc.def}
\childdocmain{}
%    \end{macrocode}

% Optional override for |\version| flag:
%    \begin{macrocode}
%%\ifchilddoc\else\providecommand{\version}{draft}\fi
%    \end{macrocode}

% Define the default values for the |\version| flag
% (|final| for the main file and |draft| for childs):
%    \begin{macrocode}
\ifchilddoc
\providecommand{\version}{draft}
\else
\providecommand{\version}{final}
\fi
%    \end{macrocode}

% Load the standard document class:
%    \begin{macrocode}
\documentclass[12pt]{article}
%    \end{macrocode}

% Start the document body:
%    \begin{macrocode}
\begin{document}
%    \end{macrocode}

% Declare a title page.
% Print title, part of document being processed and version flag:
%    \begin{macrocode}
\addtocounter{page}{-1}
\begin{center}
{\LARGE\bfseries{}childdoc example\par}
\vspace{1cm}
\ifchilddoc
\ifchilddocmanual part\else chapter\fi:
`\childdocname' of `\childdocjob'\par
\else
main document: `\childdocjob'\par
\fi
version: \version\par
\end{center}
\newpage
%    \end{macrocode}

% Manually include selected file,
% otherwise process as usual:
%    \begin{macrocode}
\ifchilddocmanual
\section*{part `\childdocname'}
\input{\childdocname}
\else
%    \end{macrocode}

% Include the two chapters:
%    \begin{macrocode}
\include{cdocsch1}
\include{cdocsch2}
%    \end{macrocode}

% Include the two parts unless only chapters should be displayed:
%    \begin{macrocode}
\ifchilddoc\else
\section{part three}
\input{cdocspt3}
\section{part four}
\input{cdocspt4}
\fi
%    \end{macrocode}

% Process as usual until here:
%    \begin{macrocode}
\fi
%    \end{macrocode}

% End of document body:
%    \begin{macrocode}
\end{document}
%    \end{macrocode}
%\iffalse
%</samplemain>
%\fi
%
% %%%%%%%%%%%%%%%%%%%%%%%%%%%%%%%%%%%%%%
% \paragraph{Chapter Include Files.}
%
% The include files are called |cdocsch1.tex| and |cdocsch2.tex|.
%
%\iffalse
%<*samplechap1|samplechap2>
%\fi

% Optional override for |\version| flag:
%    \begin{macrocode}
%%\providecommand{\version}{final}
%    \end{macrocode}

% Include the main document:
%    \begin{macrocode}
\input{childdoc.def}
\childdocof{cdocsamp}
%    \end{macrocode}

%\iffalse
%</samplechap1|samplechap2>
%\fi
%
%\iffalse
%<*samplechap1>
%\fi
% Some text for chapter 1:
%    \begin{macrocode}
\section{one}
some text in chapter one
%    \end{macrocode}

%\iffalse
%</samplechap1>
%\fi
% Some text for chapter 2:
%\iffalse
%<*samplechap2>
%\fi
%    \begin{macrocode}
\section{two}
more text in chapter two
%    \end{macrocode}

%\iffalse
%</samplechap2>
%\fi
%
% %%%%%%%%%%%%%%%%%%%%%%%%%%%%%%%%%%%%%%
% \paragraph{Part Include Files.}
%
% The include files are called |cdocspt3.tex| and |cdocspt4.tex|.
%
%\iffalse
%<*samplepart3|samplepart4>
%\fi

% Optional override for |\version| flag:
%    \begin{macrocode}
%%\providecommand{\version}{final}
%    \end{macrocode}

% Include the main document:
%    \begin{macrocode}
\input{childdoc.def}
\childdocby{cdocsamp}
%    \end{macrocode}

%\iffalse
%</samplepart3|samplepart4>
%\fi
%
%\iffalse
%<*samplepart3>
%\fi
% Some text for part 3:
%    \begin{macrocode}
some text in part three
%    \end{macrocode}

%\iffalse
%</samplepart3>
%\fi
% Some text for part 4:
%\iffalse
%<*samplepart4>
%\fi
%    \begin{macrocode}
more text in part four
%    \end{macrocode}

%\iffalse
%</samplepart4>
%\fi
%
% %%%%%%%%%%%%%%%%%%%%%%%%%%%%%%%%%%%%%%
% \paragraph{Forwarding for a Complete Draft.}
%
% The following forwarding file |cdocsdrf.tex|
% compiles the main document in draft mode:
%\iffalse
%<*sampledraft>
%\fi
%    \begin{macrocode}
\def\version{draft}
\input{childdoc.def}
\childdocforward{cdocsamp}
%    \end{macrocode}

%\iffalse
%</sampledraft>
%\fi
%
% %%%%%%%%%%%%%%%%%%%%%%%%%%%%%%%%%%%%%%
% \paragraph{Forwarding for Final Version of the Chapters.}
%
% The following forwarding files |cdocsfn1.tex| and |cdocsfn2.tex|
% (with identical content)
% compile the final versions of the child documents
% |cdocsch1.tex| and |cdocsch2.tex|, respectively:
%\iffalse
%<*samplefinal>
%\fi
%    \begin{macrocode}
\def\version{final}
\input{childdoc.def}
\childdocforwardprefix[cdocsamp]{cdocsfn}{cdocsch}
%    \end{macrocode}

%\iffalse
%</samplefinal>
%\fi
%
% %%%%%%%%%%%%%%%%%%%%%%%%%%%%%%%%%%%%%%
% \paragraph{Command Line Processing.}
%
% The following three command lines generate the output files
% |cdocscld|, |cdocscl1| and |cdocscl2|
% which should be identical to
% |cdocsdrf|, |cdocsch1| and |cdocsfn2|, respectively:
% \begin{center}
% \begin{tabular}{l}
% |latex -jobname cdocscld \|\\
% |  "\def\version{draft}\input{childdoc.def}\childdocforward{cdocsamp}"|\\
% |latex -jobname cdocscl1 \|\\
% |  "\input{childdoc.def}\childdocforward[cdocsamp]{cdocsch1}"|\\
% |latex -jobname cdocscl2 \|\\
% |  "\def\version{final}\input{childdoc.def}\childdocforward{cdocsch2}"|
% \end{tabular}
% \end{center}
% Note that the trailing backslash on each first line
% merely continues the input to the second line
% (for convenient cut ant paste).
% Furthermore, the command |latex| can be replaced by any
% of its alternative versions such as |pdflatex|.
%
% %%%%%%%%%%%%%%%%%%%%%%%%%%%%%%%%%%%%%%%%%%%%%%%%%%%%%%%%%%%%%%%%%%%%%%%%%%%%%%
% %%%%%%%%%%%%%%%%%%%%%%%%%%%%%%%%%%%%%%%%%%%%%%%%%%%%%%%%%%%%%%%%%%%%%%%%%%%%%%
% \section{Implementation}
%\iffalse
%<*package>
%\fi
%
% This section describes the definitions file |childdoc.def|.

% The definitions cannot be loaded using |\usepackage| or |\RequirePackage|
% which has a mechanism to prevent loading a style file more than once.
% When loading the definitions by means of |\input|
% multiple instances have to be prevented manually:
%\iffalse
%This code needs to be before the `\ProvidesFile' directive
%which is defined at the beginning of this file.
%Therefore it is also placed there and commented out here.
%</package>
%<*discard>
%\fi
%    \begin{macrocode}
\ifdefined\childdocmain\endinput\fi
%    \end{macrocode}
%\iffalse
%</discard>
%<*package>
%\fi
%
% \macro{\ifchilddoc}
% \macro{\ifchilddocmanual}
% The conditional |\ifchilddoc| tells whether a
% child (true) or main (false) document is being compiled.
% The conditional |\ifchilddocmanual| tells whether
% the |\includeonly| mechanism is used (false) or
% the selection of child files must be performed manually (true).
% The definitions initialise to false:
%    \begin{macrocode}
\newif\ifchilddoc
\newif\ifchilddocmanual
%    \end{macrocode}

% \macro{\childdocname}
% \macro{\childdocjob}
% The macro |\childdocname| stores the name of the main document
% to be compiled. The macro |\childdocjob| stores the name of
% the document on which the \LaTeX{} compiler was originally invoked.
% The content of |\jobname| cannot be compared
% to filenames specified in the source due to different catcodes.
% The following code rescans |\jobname|, stores the result
% in |\childdocname| and saves a copy in |\childdocjob|:
%    \begin{macrocode}
\edef\childdocname{\scantokens\expandafter{\jobname\noexpand}}
\let\childdocjob\childdocname
%    \end{macrocode}

% \macro{\childdocdisable}
% The macro |\childdocdisable| prevents the main file
% from being processed more than once.
% At this stage, the main document command |\childdocmain|
% is assumed to be called once again where it should do nothing.
% Any subsequent call to it should prevent
% a secondary processing of the main document
% It overwrites the forwarding commands
% |\childdocof| and |\childdocforward|
% with empty macros to prevent further inclusions of the main document:
%    \begin{macrocode}
\newcommand{\childdocdisable}
{
  \renewcommand{\childdocmain}[1]{\renewcommand{\childdocmain}[1]{\endinput}}
  \renewcommand{\childdocof}[1]{}
  \renewcommand{\childdocby}[2][]{}
  \renewcommand{\childdocforward}[2][]{}
  \renewcommand{\childdocdisable}{}
}
%    \end{macrocode}

% \macro{\childdocmain}
% The macro |\childdocmain| is to be called at the top of the main file
% with nothing or the main filename (without extension) as argument.
% First, it breaks loops.
% If the argument is not empty and does not match |\childdocname|
% (which is set by the first inclusion of |childdoc.def|),
% |\ifchilddoc| is set to true, |\includeonly| is applied to the child file
% and |\jobname| is set to the main file
% (for proper handling of |.aux| files):
%    \begin{macrocode}
\newcommand{\childdocmain}[1]
{
  \childdocdisable\childdocmain{}
  \if?#1?\else
    \begingroup
      \def\childdoctmp{#1}
      \ifx\childdoctmp\childdocname
        \def\childdoctmp{}
      \else
        \def\childdoctmp
        {
          \childdoctrue
          \includeonly{\childdocname}
          \def\childdocjob{#1}
          \def\jobname{#1}
        }
      \fi
      \expandafter
    \endgroup
    \childdoctmp
  \fi
}
%    \end{macrocode}

% \macro{\childdocof}
% The command |\childdocof| redirects
% compilation to the main file |#1|.
%    \begin{macrocode}
\newcommand{\childdocof}[1]
{
  \childdocdisable
  \childdoctrue
  \includeonly{\childdocname}
  \def\jobname{#1}
  \def\childdocjob{#1}
  \input{#1}
}
%    \end{macrocode}

% \macro{\childdocby}
% The command |\childdocby| ....
%    \begin{macrocode}
\newcommand{\childdocby}[2][]
{
  \childdocdisable
  \childdoctrue
  \childdocmanualtrue
  \if?#1?\else
    \def\jobname{#2}
  \fi
  \def\childdocjob{#2}
  \input{#2}
  \endinput
}
%    \end{macrocode}

% \macro{\childdocforward}
% The command |\childdocforward| redirects
% compilation to the main file or
% (if the optional argument is given) a child file.
% Parameters are set as if the main file
% or a child file starting with |\childdocof| was compiled.
% Then compilation is handed over to the main file:
%    \begin{macrocode}
\newcommand{\childdocforward}[2][]
{
  \begingroup
    \if?#1?
      \def\childdoctmp
      {
        \def\childdocname{#2}
        \def\childdocjob{#2}
        \def\jobname{#2}
        \input{#2}
        \endinput
      }
    \else
      \def\childdoctmp
      {
        \childdocdisable
        \def\childdocname{#2}
        \childdoctrue
        \includeonly{#2}
        \def\childdocjob{#1}
        \def\jobname{#1}
        \input{#1}
        \endinput
      }
    \fi
    \expandafter
  \endgroup
  \childdoctmp
}
%    \end{macrocode}

% \macro{\childdocforwardprefix}
% The command |\childdocforwardprefix| redirects
% compilation to the main or a child file by means of a pattern.
% The prefix |#1| in the current filename is replaced by |#2|
% and the suffix of the current filename is kept
% (it is assumed that the filename does not contain the substring `|~~~|'
% which is used as a delimiter).
% Compilation is handed over to the new file by |\childdocforward|:
%    \begin{macrocode}
\newcommand{\childdocforwardprefix}[3][]
{
  \begingroup
    \def\childdocextract #2##1~~~{\def\childdoctmp{\childdocforward[#1]{#3##1}}}
    \expandafter\childdocextract\childdocname~~~
    \expandafter
  \endgroup
  \childdoctmp
}
%    \end{macrocode}

% \macro{\childdoc}
% The deprecated macro |\childdoc| is a legacy version of |\childdocmain|:
%    \begin{macrocode}
\newcommand{\childdoc}{\childdocmain}
%    \end{macrocode}

% \macro{\childdocredirect}
% The deprecated macro |\childdocredirect| is a legacy version
% of |\childdocforward| and |\childdocforwardprefix|:
%    \begin{macrocode}
\newcommand{\childdocredirect}[2][]
{
  \begingroup
    \if?#1?
      \def\childdoctmp{\childdocforward{#2}}
    \else
      \def\childdoctmp{\childdocforwardprefix{#1}{#2}}
    \fi
    \expandafter
  \endgroup
  \childdoctmp
}
%    \end{macrocode}

%\iffalse
%</package>
%\fi
%
\endinput
\childdocforward{cdocsamp}"|\\
% |latex -jobname cdocscl1 \|\\
% |  "% \iffalse
%
% childdoc.dtx Copyright (C) 2017-2018 Niklas Beisert
%
% This work may be distributed and/or modified under the
% conditions of the LaTeX Project Public License, either version 1.3
% of this license or (at your option) any later version.
% The latest version of this license is in
%   http://www.latex-project.org/lppl.txt
% and version 1.3 or later is part of all distributions of LaTeX
% version 2005/12/01 or later.
%
% This work has the LPPL maintenance status `maintained'.
%
% The Current Maintainer of this work is Niklas Beisert.
%
% This work consists of the files childdoc.dtx and childdoc.ins
% and the derived files childdoc.def and cdocsamp.tex with
% cdocsch1.tex, cdocsch2.tex, cdocsdrf.tex, cdocsfn1.tex, cdocsfn2.tex.
%
%<package>\ifdefined\childdocmain\endinput\fi
%<package>\ProvidesFile{childdoc.def}[2018/12/30 v2.0 child document driver]
%<samplemain>\ProvidesFile{cdocsamp.tex}[2018/12/30 v2.0 sample for childdoc]
%<*driver>
%\ProvidesFile{childdoc.drv}[2018/12/30 v2.0 childdoc reference manual file]
\PassOptionsToClass{10pt,a4paper}{article}
\documentclass{ltxdoc}

\usepackage[margin=35mm]{geometry}
\usepackage{hyperref}
\usepackage{hyperxmp}
\usepackage[usenames]{color}

\hypersetup{colorlinks=true}
\hypersetup{pdfstartview=FitH}
\hypersetup{pdfpagemode=UseNone}
\hypersetup{pdfsource={}}
\hypersetup{pdflang={en-UK}}
\hypersetup{pdfcopyright={Copyright 2017-2018 Niklas Beisert.
  This work may be distributed and/or modified under the
  conditions of the LaTeX Project Public License, either version 1.3
  of this license or (at your option) any later version.}}
\hypersetup{pdflicenseurl={http://www.latex-project.org/lppl.txt}}
\hypersetup{pdfcontactaddress={ETH Zurich, ITP, HIT K,
  Wolfgang-Pauli-Strasse 27}}
\hypersetup{pdfcontactpostcode={8093}}
\hypersetup{pdfcontactcity={Zurich}}
\hypersetup{pdfcontactcountry={Switzerland}}
\hypersetup{pdfcontactemail={nbeisert@itp.phys.ethz.ch}}
\hypersetup{pdfcontacturl={http://people.phys.ethz.ch/\xmptilde nbeisert/}}

\newcommand{\secref}[1]{\hyperref[#1]{section \ref*{#1}}}

\parskip1ex
\parindent0pt
\let\olditemize\itemize
\def\itemize{\olditemize\parskip0pt}

\begin{document}

\title{The \textsf{childdoc} Package}
\hypersetup{pdftitle={The childdoc Package}}
\author{Niklas Beisert\\[2ex]
  Institut f\"ur Theoretische Physik\\
  Eidgen\"ossische Technische Hochschule Z\"urich\\
  Wolfgang-Pauli-Strasse 27, 8093 Z\"urich, Switzerland\\[1ex]
  \href{mailto:nbeisert@itp.phys.ethz.ch}
  {\texttt{nbeisert@itp.phys.ethz.ch}}}
\hypersetup{pdfauthor={Niklas Beisert}}
\hypersetup{pdfsubject={Manual for the LaTeX2e Package childdoc}}
\date{30 December 2018, \textsf{v2.0}}
\maketitle

\begin{abstract}\noindent
\textsf{childdoc} is a \LaTeXe{} package
that enables the direct compilation
of document sections included by |\include|
to individual files.
\end{abstract}

\begingroup
\parskip0ex
\tableofcontents
\endgroup

%%%%%%%%%%%%%%%%%%%%%%%%%%%%%%%%%%%%%%%%%%%%%%%%%%%%%%%%%%%%%%%%%%%%%%%%%%%%%%%%
%%%%%%%%%%%%%%%%%%%%%%%%%%%%%%%%%%%%%%%%%%%%%%%%%%%%%%%%%%%%%%%%%%%%%%%%%%%%%%%%
\section{Introduction}

\LaTeX{} provides a mechanism to structure a large document (such as a book)
into a main file and several child files (containing the chapters)
using the |\include| command.
This mechanism is beneficial for documents
which span hundreds of pages in order to
make the source file(s) more manageable.
Moreover, compilation can be restricted to
selected child files by means of the |\includeonly| command.
The latter feature can be used to reduce the compilation time while editing
(this was significantly more useful in the earlier days of \LaTeX{})
or to generate a smaller document which is easier to navigate.
Another application of |\includeonly| is to generate
documents consisting of selected parts of the complete document.

However, there are a few drawbacks of the plain |\include| mechanism:
\begin{itemize}
\item
The child files cannot be compiled on their own,
they can only be compiled via the main file.
A naive editing environment
(such as a text editor with an option
to have the current file processed by \LaTeX)
may require one to switch to the main file before compiling;
attempting to compile the child file produces errors.
\item
The main file must be modified (each time)
to adjust the |\includeonly| command
to the present needs. This easily leaves the main file in a messy state.
\item
The generated document will always carry the filename
of the main document. This is inconvenient if
several child files are to be compiled and
to be kept for distribution.
\end{itemize}

The present package provides a simple interface
to make child files individually compilable by \LaTeX{}.
Compiling a child file then has the same effect as compiling
the main file with an |\includeonly| command
to select the appropriate child.
Moreover the generated document will carry the name of the child
rather than the main file.
This resolves all three above issues.

This feature is meant to make the editing of books,
thesis documents and lecture notes somewhat more convenient.
However, the package can also be used efficiently for
composing a series of documents (such as exercise sheets)
which are typically distributed individually.
It then assists the author in generating the individual documents
(potentially in different versions)
as well as a document containing the collected series.
Another application is in developing style files
or other kinds of included material
where compilation of the style file could redirect
to a sample or test file.

%%%%%%%%%%%%%%%%%%%%%%%%%%%%%%%%%%%%%%%%%%%%%%%%%%%%%%%%%%%%%%%%%%%%%%%%%%%%%%%%
%%%%%%%%%%%%%%%%%%%%%%%%%%%%%%%%%%%%%%%%%%%%%%%%%%%%%%%%%%%%%%%%%%%%%%%%%%%%%%%%
\section{Usage}

First of all, the package \textsf{childdoc} is \emph{not} a standard
\LaTeXe{} |.sty| style file! Therefore it needs to be invoked in
a non-standard way.

%%%%%%%%%%%%%%%%%%%%%%%%%%%%%%%%%%%%%%%%%%%%%%%%%%%%%%%%%%%%%%%%%%%%%%%%%%%%%%%%
\subsection{Included Files}
\label{sec:include}

%%%%%%%%%%%%%%%%%%%%%%%%%%%%%%%%%%%%%%%%
\DescribeMacro{\childdocmain}
To use the package, add the commands
\begin{center}
\begin{tabular}{l}
|\input{childdoc.def}|\\
|\childdocmain{}|\\
\end{tabular}
\end{center}
at the very top of the main \LaTeX{} file,
in particular \emph{before} the |\documentclass| statement!
The argument of |\childdocmain| should be left empty
(but it must be present).

%%%%%%%%%%%%%%%%%%%%%%%%%%%%%%%%%%%%%%%%
\DescribeMacro{\childdocof}
Furthermore, add the commands
\begin{center}
\begin{tabular}{l}
|\input{childdoc.def}|\\
|\childdocof{|\textit{main}|}|\\
\end{tabular}
\end{center}
at the top of every child file \textit{child}
which is included by |\include{|\textit{child}|}|
from within the main file
(or at least for those files to be compiled individually).
The argument \textit{main} must be the filename of the main file.

There are a couple of
considerations in setting up the main and child documents:

%%%%%%%%%%%%%%%%%%%%%%%%%%%%%%%%%%%%%%%%
\paragraph{Restrictions.}

Please note the following restrictions:
\begin{itemize}
\item
|\childdocmain| must be called with one argument \textit{main}
to ensure compatibility with earlier version of the package.
It must either be empty (|\childdocmain{}|)
or precisely match the filename of the main file in which it is specified.
See \secref{sec:detection} for further information.
\item
The filename \textit{main} must be specified without the |.tex| extension.
\item
The filename \textit{main} is case sensitive
(even in case-insensitive file systems)
due to internal string comparison.
\item
The argument \textit{main} should be fully expanded, it cannot be a macro.
\item
Subdirectories and special characters should be avoided in filenames.
\item
The command |\childdocmain{|\textit{main}|}| must be followed by a whitespace.
It should not be followed immediately by another command
or by a comment mark `|%|'.
This is because the \TeX{} parser reads the token immediately following
the argument of |\childdocmain| and puts it
at the beginning of every child section;
however, a white\-space is ignored.
\end{itemize}

%%%%%%%%%%%%%%%%%%%%%%%%%%%%%%%%%%%%%%%%
\paragraph{Content of Main File.}

It is advisable to place all content in the child files included by |\include|.
Any output contained in the main file will appear in all child documents
unless suppressed manually;
it cannot be suppressed automatically by the |\includeonly| directive
and thus should normally be avoided.
A method to include some content in the main file
by means of conditional processing is described in \secref{sec:conditional}.

%%%%%%%%%%%%%%%%%%%%%%%%%%%%%%%%%%%%%%%%
\paragraph{Page Numbering.}

When only a part of the document is compiled,
the appropriate numbering of pages
(as well as other status parameters)
is determined from the |.aux| files.
The latter contain information from previous passes.
However this information needs to propagate through
all intermediate child documents.
Therefore the page numbering in child documents may well
be inconsistent until the complete document is compiled at least once.

A useful (if unconventional) way to always ensure a consistent
page numbering is to restart the numbering in each child document
and denote the pages by `\textit{child}|.|\textit{page}'
where \textit{child} represents the chapter/section number of the child file.
This can be achieved by the command
|\numberwithin{page}{|\textit{child}|}|
of the \textsf{amsmath} package
where \textit{child} can be |chapter| or |section|
depending on the chosen structuring.
Alternatively, one can modify the macro |\thepage| appropriately
and reset the counter |page| at the start of each child file.

%%%%%%%%%%%%%%%%%%%%%%%%%%%%%%%%%%%%%%%%%%%%%%%%%%%%%%%%%%%%%%%%%%%%%%%%%%%%%%%%
\subsection{Conditional Processing}
\label{sec:conditional}

The package provides a mechanism to compile different versions
of a document. To customise the versions further some conditional processing
can come in handy to distinguish which version is being compiled.
The package provides two macros to describe the compilation context:

%%%%%%%%%%%%%%%%%%%%%%%%%%%%%%%%%%%%%%%%
\DescribeMacro{\ifchilddoc}
The conditional |\ifchilddoc| distinguishes between the compilation of
child documents and the main document:
%
\begin{center}
|\ifchilddoc |\textit{child-code}| |[|\||else |\textit{main-code}]| \||fi|
\end{center}

%%%%%%%%%%%%%%%%%%%%%%%%%%%%%%%%%%%%%%%%
\DescribeMacro{\childdocname}
\DescribeMacro{\childdocjob}
The macro |\childdocname| contains the filename (without extension)
of the main or child file being processed.
Note that |\childdocjob| will always contain the name of the main file.

%%%%%%%%%%%%%%%%%%%%%%%%%%%%%%%%%%%%%%%%
\paragraph{Title Page.}

Conditional processing can be used to include a title or banner page
in the main document when proper precautions are taken.
Importantly, the code in the main file should ensure that the page counter
(as well as other status parameters which are stored in the |.aux| files)
takes the same value after the conditional processing.
Otherwise the page numbers may take divergent values
depending on which part is compiled.

For example, a title page could be declared by:
%
\begin{center}
\begin{tabular}{l}
|\ifchilddoc\||else|\\
|\addtocounter{page}{-1}|\\
\textit{code for title page}\\
|\newpage|\\
|\||fi|
\end{tabular}
\end{center}
%
A banner page for the child documents can be generated by:
%
\begin{center}
\begin{tabular}{l}
|\ifchilddoc|\\
|\addtocounter{page}{-1}|\\
\textit{code for banner page}\\
|\newpage|\\
|\||fi|
\end{tabular}
\end{center}
%
Here one could write a message such as:
\begin{center}
|This is the part \childdocname{} of \childdocjob{}.|
\end{center}

%%%%%%%%%%%%%%%%%%%%%%%%%%%%%%%%%%%%%%%%%%%%%%%%%%%%%%%%%%%%%%%%%%%%%%%%%%%%%%%%
\subsection{Flags}
\label{sec:flags}

The package makes it easy to generate different versions
of the main or child documents.
To this end compilation flags can be defined
and assigned different default values.
They will be particularly useful in conjunction
with the forwarding mechanism described in \secref{sec:forward}.

For example, it may be useful to have a flag |\version|
which can be set to |draft| or |final|.
The document source will contain some conditional code
depending on the value of |\version|.
Suppose further, the flag should default to |final| for the main file
and to |draft| for child files
which is a natural assignment for editing the document.
This is achieved by placing the following code
in the preamble of the main document
(below the |\childdocmain| directive):
%
\begin{center}
\begin{tabular}{l}
|\ifchilddoc|\\
|\providecommand{\version}{draft}|\\
|\||else|\\
|\providecommand{\version}{final}|\\
|\||fi|
\end{tabular}
\end{center}
%
The definition by |\providecommand| makes sure
that previous definitions are not overwritten.
Further statements |\providecommand{\version}{...}|
can thus be added before the above code to override it.

For the main file, one might add a line
(between |\childdocmain| and the above block)
%
\begin{center}
|%\ifchilddoc\||else\providecommand{\version}{draft}\||fi|
\end{center}
%
which can be uncommented to produce a draft version.
Likewise one can add a line to the very top of a child file
(above the |\childdocof{|\textit{main}|}| directive)
%
\begin{center}
|%\providecommand{\version}{final}|
\end{center}
%
which can be uncommented to produce the final version of this child document.

%%%%%%%%%%%%%%%%%%%%%%%%%%%%%%%%%%%%%%%%%%%%%%%%%%%%%%%%%%%%%%%%%%%%%%%%%%%%%%%%
\subsection{Forwarding}
\label{sec:forward}

Different versions of the main or child documents
using compilation flags as described in \secref{sec:flags}
can be (permanently) stored in different files
for convenient compilation, viewing and distribution.
To this end, the package defines a command
to pass on compilation to a different file:

%%%%%%%%%%%%%%%%%%%%%%%%%%%%%%%%%%%%%%%%
\DescribeMacro{\childdocforward}
The command |\childdocforward| redirects processing to
another source file:
%
\begin{center}
\begin{tabular}{l}
|\input{childdoc.def}|\\
|\childdocforward[|\textit{main}|]{|\textit{dest}|}|\\
\end{tabular}
\end{center}
%
The argument \textit{dest} is the destination file
(without extension).
It should be the main file or one of the child files.
Note that further \textsf{childdoc} directives
such as |\childdocof| and |\childdocforward|
in the indicated file will be processed in this form.
The optional argument \textit{main}
passes on directly to the main file \textit{main}
while pretending to compile the child \textit{dest}.
This form behaves as if \textit{dest}
issues |\childdocof{|\textit{main}|}| right away,
and no further \textsf{childdoc} directives will be processed.

%%%%%%%%%%%%%%%%%%%%%%%%%%%%%%%%%%%%%%%%
\DescribeMacro{\...prefix}
In the alternative form |\childdocforwardprefix|,
%
\begin{center}
\begin{tabular}{l}
|\input{childdoc.def}|\\
|\childdocforwardprefix[|\textit{main}|]{|\textit{prefix}|}{|\textit{dest}|}|
\end{tabular}
\end{center}
%
the destination file is determined by a pattern
depending on the current file:
To make this work, the current file must be called
`{\textit{prefix}\hspace{0.2em}\textit{suffix}}'
with \textit{prefix} matching precisely the argument.
Processing is then passed on to the file
`{\textit{dest}\hspace{0.2em}\textit{suffix}}'.
Surely, the same effect is achieved by
directly specifying the
argument `{\textit{dest}\hspace{0.2em}\textit{suffix}}'
in the first form.
However, that requires to set up a different file
for each child. With the alternative form of the command
all these files can have exactly the same content
which simplifies setting them up and maintaining them.

For example, the following file |draft.tex|
with a compilation flag |\version| as described in \secref{sec:flags}
compiles the main document as a draft:
%
\begin{center}
\begin{tabular}{l}
|\def\version{draft}|\\
|\input{childdoc.def}|\\
|\childdocforward{|\textit{main}|}|
\end{tabular}
\end{center}
%
Likewise, the following files |final|\textit{nn}|.tex|
compile the final version of the child document
|child|\textit{nn}|.tex|:
%
\begin{center}
\begin{tabular}{l}
|\def\version{final}|\\
|\input{childdoc.def}|\\
|\childdocforwardprefix{final}{child}|
\end{tabular}
\end{center}
%

Note that when several versions of a main file and/or of each child file
are to be generated, it may be convenient to set up a |Makefile| or
shell script to automatise the process.

%%%%%%%%%%%%%%%%%%%%%%%%%%%%%%%%%%%%%%%%%%%%%%%%%%%%%%%%%%%%%%%%%%%%%%%%%%%%%%%%
\subsection{Command Line Processing}
\label{sec:commandline}

The effect of redirection files can also be achieved by invoking
the \LaTeX{} compiler with a more elaborate command line.
Most conveniently this should be done as part
of a shell script or a |Makefile|.

When using \textsf{childdoc} in the main file, the following
command lines effectively perform a redirection
(note that depending on the shell being used,
backslashes may have to be doubled: `|\|' $\to$ `|\\|'):
%
\begin{center}
|... -jobname "|\textit{target}|" |\\|"|[\textit{flags}]%
|\input{childdoc.def}\childdocforward[|\textit{main}|]{|\textit{dest}|}"|
\end{center}
%
Here \textit{target} is the name of the output file,
\textit{main} is the name of the main file
and \textit{dest} is the name of the main or child file to be processed
(all filenames without extensions).
The optional argument \textit{main} can be omitted
if \textit{main} matches \textit{dest}.
Optionally, compilation \textit{flags} can be defined via |\def| commands.
This command line makes the \TeX{} engine believe
it is compiling the file \textit{target}
whose content is specified as the latter parameter.
The provided code then forwards the processing to
\textit{main} or \textit{dest} as described in \secref{sec:forward}.

%%%%%%%%%%%%%%%%%%%%%%%%%%%%%%%%%%%%%%%%%%%%%%%%%%%%%%%%%%%%%%%%%%%%%%%%%%%%%%%%
\subsection{Include by Input}
\label{sec:input}

Including child documents by |\include| has some restrictions by design.
Most notably, the content of a child document always occupies
its own set of pages; pages cannot be shared between child documents.
Usually, this behaviour makes perfect sense
because each child document contain an essential part of the document.
However, in some situations it may be desirable to compose
a document from a collection of parts
without having mandatory page breaks between then.
For this case, the package
provides a mechanism to include parts
by |\input| which can also be processed individually.
However, by construction this mechanism
requires manual handling of the content to be output.

%%%%%%%%%%%%%%%%%%%%%%%%%%%%%%%%%%%%%%%%
\DescribeMacro{\ifchilddocmanual}
The main file should be prepared as usual, see \secref{sec:include}.
However, the document body must make a distinction
between processing of an individual part and of the main document, e.g.:
%
\begin{center}
\begin{tabular}{l}
|\ifchilddocmanual|\\
|\input{\childdocname}|\\
|\||else|\\
\textit{document body with }|\input{|\textit{part}|}|\\
|\||fi|
\end{tabular}
\end{center}
%
The conditional |\ifchilddocmanual| is true whenever
a part to be included by |\input| is being compiled,
and the name of the part is stored in |\childdocname|.

%%%%%%%%%%%%%%%%%%%%%%%%%%%%%%%%%%%%%%%%
\DescribeMacro{\childdocby}
Each part to be included by |\input| should start with:
%
\begin{center}
\begin{tabular}{l}
|\input{childdoc.def}|\\
|\childdocby{|\textit{main}|}|\\
\end{tabular}
\end{center}
%
The directive |\childdocby| is similar to |\childdocof|
described in \secref{sec:include},
but the subsequent selection of content must be done manually.
To that end, both |\ifchilddoc| and |\ifchilddocmanual|
will be true upon processing of a part,
and the name of the part is stored in |\childdocname|.
Note that |\jobname| will be set to the filename of the current part
so that each part receives an individual |.aux| file
that does not interfere with the |.aux| file(s) of the main document.
This behaviour can be altered by the alternative form
|\childdocby[*]{|\textit{main}|}| (with a non-empty optional argument)
which uses the |.aux| file of the main document
by setting |\jobname| to \textit{main}.

%%%%%%%%%%%%%%%%%%%%%%%%%%%%%%%%%%%%%%%%%%%%%%%%%%%%%%%%%%%%%%%%%%%%%%%%%%%%%%%%
\subsection{Driver Development}
\label{sec:driver}

The \textsf{childdoc} mechanism can also be use for the development
of definition files such as \LaTeX{} styles or classes.
This case differs from the above setup with multiple parts
included by |\include| in that no |\includeonly| should be invoked.
This can be achieved by starting the include file
(before |\ProvidesPackage|) with:
%
\begin{center}
\begin{tabular}{l}
|\input{childdoc.def}|\\
|\childdocforward{|\textit{main}|}|\\
\end{tabular}
\end{center}
%
or alternatively with:
%
\begin{center}
\begin{tabular}{l}
|\input{childdoc.def}|\\
|\childdocby{|\textit{main}|}|\\
\end{tabular}
\end{center}
%
Both forms have slightly different effects as described above.
The main file is prepared as usual, see \secref{sec:include}.

%%%%%%%%%%%%%%%%%%%%%%%%%%%%%%%%%%%%%%%%%%%%%%%%%%%%%%%%%%%%%%%%%%%%%%%%%%%%%%%%
\subsection{Legacy Detection}
\label{sec:detection}

The directive |\childdocmain| in the main file can detect
whether the complete document or merely a child is to be compiled
even without using the directive |\childdocof|.
This method is deprecated because it is less robust
and there is no compelling reason to use it;
it is merely provided for backward compatibility
and it may be removed in future versions.

If the detection mechanism is to be used,
it is mandatory to correctly specify
the filename of the main file as the argument of |\childdocmain|:
%
\begin{center}
\begin{tabular}{l}
|\input{childdoc.def}|\\
|\childdocmain{|\textit{main}|}|\\
\end{tabular}
\end{center}
%
If |\jobname| does not match the argument \textit{main} of |\childdocmain|,
it is assumed that |\jobname| points to the child file to be compiled.
When using |\childdocmain| with the main file specified as argument,
it suffices to start a child file
with just |\input{|\textit{main}|}|
without loading of the package and using |\childdocof|.
If instead all processing is done
with the appropriate \textsf{childdoc} directives,
the argument of \textit{main} of |\childdocmain| can be empty.

An alternative version of the command line processing described
in \secref{sec:commandline} using the detection mechanism reads:
%
\begin{center}
|... -jobname "|\textit{target}|" "|[\textit{flags}]%
[|\def\jobname{|\textit{dest}|}|]|\input{|\textit{main}|}"|
\end{center}

%%%%%%%%%%%%%%%%%%%%%%%%%%%%%%%%%%%%%%%%%%%%%%%%%%%%%%%%%%%%%%%%%%%%%%%%%%%%%%%%
\subsection{Manual Code}
\label{sec:manual}

In case one cannot be certain whether the definitions file |childdoc.def|
is installed on the target \TeX{} distribution
and one prefers not to ship it,
it is conceivable to paste a few relevant commands into the sources.

To that end, drop all statements |\input{childdoc.def}|
and perform the replacements as outlined below.
Instead of |\childdocmain{|\textit{main}|}| add the following code
to the top of the main file:
%
\begin{center}
\begin{tabular}{l}
|\||ifdefined\childdocname\endinput\||fi\newif\ifchilddoc|\\
|\edef\childdocname{\scantokens\expandafter{\jobname\noexpand}}|\\
|\def\childdocmain{|\textit{main}|}\||ifx\childdocmain\childdocname\||else|\\
|\childdoctrue\includeonly{\childdocname}\let\jobname\childdocmain\||fi|\\
\end{tabular}
\end{center}
%
Instead of |\childdocof{|\textit{main}|}| just include the main file
at the top of each child file:
%
\begin{center}
|\input{|\textit{main}|}|
\end{center}
%
A simple redirection |\childdocforward{|\textit{dest}|}| is achieved by:
%
\begin{center}
|\def\jobname{|\textit{dest}|}\input{\jobname}|
\end{center}
%
The redirection with prefix
|\childdocforwardprefix[|\textit{prefix}|]{|\textit{dest}|}|
is accomplished by:
%
\begin{center}
\begin{tabular}{l}
|{\edef\jobname{\scantokens\expandafter{\jobname\noexpand}}|\\
|\def\redirectjob |\textit{prefix}|#1~~~{\gdef\jobname{|\textit{dest}|#1}}|\\
|\expandafter\redirectjob\jobname~~~}\input{\jobname}|
\end{tabular}
\end{center}

In an alternative approach,
child documents can be compiled by a specific command line
without additional code or specific definitions:
%
\begin{center}
|... -jobname "|\textit{target}|" "|[\textit{flags}]%
|\includeonly{|\textit{dest}|}\input{|\textit{main}|}"|
\end{center}
%

%%%%%%%%%%%%%%%%%%%%%%%%%%%%%%%%%%%%%%%%%%%%%%%%%%%%%%%%%%%%%%%%%%%%%%%%%%%%%%%%
%%%%%%%%%%%%%%%%%%%%%%%%%%%%%%%%%%%%%%%%%%%%%%%%%%%%%%%%%%%%%%%%%%%%%%%%%%%%%%%%
\section{Information}

%%%%%%%%%%%%%%%%%%%%%%%%%%%%%%%%%%%%%%%%%%%%%%%%%%%%%%%%%%%%%%%%%%%%%%%%%%%%%%%%
\subsection{Copyright}

Copyright \copyright{} 2017--2018 Niklas Beisert

This work may be distributed and/or modified under the
conditions of the \LaTeX{} Project Public License, either version 1.3
of this license or (at your option) any later version.
The latest version of this license is in
  \url{http://www.latex-project.org/lppl.txt}
and version 1.3 or later is part of all distributions of \LaTeX{}
version 2005/12/01 or later.

This work has the LPPL maintenance status `maintained'.

The Current Maintainer of this work is Niklas Beisert.

This work consists of the files |README.txt|, |childdoc.ins| and |childdoc.dtx|
as well as the derived files |childdoc.def|, |cdocsamp.tex|
with |cdocsch1.tex|, |cdocsch2.tex|, |cdocspt3.tex|, |cdocspt4.tex|,
|cdocsdrf.tex|, |cdocsfn1.tex|, |cdocsfn2.tex|
as well as |childdoc.pdf|.

%%%%%%%%%%%%%%%%%%%%%%%%%%%%%%%%%%%%%%%%%%%%%%%%%%%%%%%%%%%%%%%%%%%%%%%%%%%%%%%%
\subsection{Files and Installation}

The package consists of the files:
%
\begin{center}
\begin{tabular}{ll}
    |README.txt|   & readme file \\
    |childdoc.ins| & installation file \\
    |childdoc.dtx| & source file \\
    |childdoc.def| & definition file \\
    |cdocsamp.tex| & sample main file \\
    |cdocsch1.tex| & sample include file \\
    |cdocsch2.tex| & sample include file \\
    |cdocspt3.tex| & sample part file \\
    |cdocspt4.tex| & sample part file \\
    |cdocsdrf.tex| & sample redirection file \\
    |cdocsfn1.tex| & sample redirection file \\
    |cdocsfn2.tex| & sample redirection file \\
    |childdoc.pdf| & manual
\end{tabular}
\end{center}
%
The distribution consists of the files
|README.txt|, |childdoc.ins| and |childdoc.dtx|.
%
\begin{itemize}
\item
Run (pdf)\LaTeX{} on |childdoc.dtx|
to compile the manual |childdoc.pdf| (this file).
\item
Run \LaTeX{} on |childdoc.ins| to create the definitions file |childdoc.def|
and the sample |cdocsamp.tex| with include files
|cdocsch1.tex|, |cdocsch2.tex|, |cdocspt3.tex|, |cdocspt4.tex|,
|cdocsdrf.tex|, |cdocsfn1.tex|, |cdocsfn2.tex|.
Then copy the file |childdoc.def| to an appropriate directory of your \LaTeX{}
distribution, e.g.\ \textit{texmf-root}|/tex/latex/childdoc|.
\end{itemize}

%%%%%%%%%%%%%%%%%%%%%%%%%%%%%%%%%%%%%%%%%%%%%%%%%%%%%%%%%%%%%%%%%%%%%%%%%%%%%%%%
\subsection{Related CTAN Packages}

There are several other packages which offer a similar functionality:
%
\begin{itemize}
\item
The packages
\href{http://ctan.org/pkg/docmute}{\textsf{docmute}},
\href{http://ctan.org/pkg/includex}{\textsf{includex}} and
\href{http://ctan.org/pkg/standalone}{\textsf{standalone}}
provide commands to include only the document body of
a child file thus allowing both files to be compiled individually.
\item
The packages \href{http://ctan.org/pkg/subdocs}{\textsf{subdocs}}
and \href{http://ctan.org/pkg/subfiles}{\textsf{subfiles}}
provide structures in which the main and child documents can be
encapsulated and allowing them to be compiled individually.
The inclusion mechanism is different from the conventional |\include|.
\item
The package \href{http://ctan.org/pkg/combine}{\textsf{combine}}
is an elaborate solution to combine several documents into one.
\end{itemize}
%
See also the CTAN topic \href{http://ctan.org/topic/subdocs}{\textsf{subdocs}}
for further related packages.
The present package differs from the above solutions in that
a document structure constructed with the conventional |\include| mechanism
just needs two extra commands at the top of every file
such that all constituent files can be compiled individually.

%%%%%%%%%%%%%%%%%%%%%%%%%%%%%%%%%%%%%%%%%%%%%%%%%%%%%%%%%%%%%%%%%%%%%%%%%%%%%%%%
%\subsection{Feature Suggestions}
%
%The following is a list of features which may be useful for future
%versions of this package:
%%
%\begin{itemize}
%\item
%\ldots
%\end{itemize}

%%%%%%%%%%%%%%%%%%%%%%%%%%%%%%%%%%%%%%%%%%%%%%%%%%%%%%%%%%%%%%%%%%%%%%%%%%%%%%%%
\subsection{Revision History}

%%%%%%%%%%%%%%%%%%%%%%%%%%%%%%%%%%%%%%%%
\paragraph{v2.0:} 2018/12/30

\begin{itemize}
\item
immediate forward processing
\item
added |\childdocby| mechanism
\item
manual restructured
\end{itemize}

%%%%%%%%%%%%%%%%%%%%%%%%%%%%%%%%%%%%%%%%
\paragraph{v1.6:} 2018/01/17

\begin{itemize}
\item
application for development of include files
\item
corrections to manual
\end{itemize}

%%%%%%%%%%%%%%%%%%%%%%%%%%%%%%%%%%%%%%%%
\paragraph{v1.5:} 2017/05/21

\begin{itemize}
\item
more complete structuring introduced
\item
|\childdocof| introduced
\item
|\childdoc| renamed to |\childdocmain|
\item
|\childredirect| renamed to |\childdocforward| and |\childdocforwardprefix|
and functionality expanded
\end{itemize}

%%%%%%%%%%%%%%%%%%%%%%%%%%%%%%%%%%%%%%%%
\paragraph{v1.0:} 2017/04/27

\begin{itemize}
\item
manual and install package
\item
first version published on CTAN
\end{itemize}

%%%%%%%%%%%%%%%%%%%%%%%%%%%%%%%%%%%%%%%%
\paragraph{v0.6:} 2017/04/26

\begin{itemize}
\item
redirection mechanism added
\end{itemize}

%%%%%%%%%%%%%%%%%%%%%%%%%%%%%%%%%%%%%%%%
\paragraph{v0.5:} 2017/04/26

\begin{itemize}
\item
functionality in definition file
\end{itemize}


%%%%%%%%%%%%%%%%%%%%%%%%%%%%%%%%%%%%%%%%%%%%%%%%%%%%%%%%%%%%%%%%%%%%%%%%%%%%%%%%
%%%%%%%%%%%%%%%%%%%%%%%%%%%%%%%%%%%%%%%%%%%%%%%%%%%%%%%%%%%%%%%%%%%%%%%%%%%%%%%%
%%%%%%%%%%%%%%%%%%%%%%%%%%%%%%%%%%%%%%%%%%%%%%%%%%%%%%%%%%%%%%%%%%%%%%%%%%%%%%%%
\appendix

\settowidth\MacroIndent{\rmfamily\scriptsize 000\ }

 \DocInput{childdoc.dtx}

\end{document}
%</driver>
% \fi
%
% %%%%%%%%%%%%%%%%%%%%%%%%%%%%%%%%%%%%%%%%%%%%%%%%%%%%%%%%%%%%%%%%%%%%%%%%%%%%%%
% %%%%%%%%%%%%%%%%%%%%%%%%%%%%%%%%%%%%%%%%%%%%%%%%%%%%%%%%%%%%%%%%%%%%%%%%%%%%%%
% \section{Sample}
%\iffalse
%<*samplemain>
%\fi
%
% The following presents a sample document
% with two chapters, two parts, a title page,
% a compile flag as well as three forwarding files to set the flag.
% It consists of eight |.tex| files:
% \begin{center}
% \begin{tabular}{ll}
% |cdocsamp.tex|&main file\\
% |cdocsch1.tex|&include file for chapter 1\\
% |cdocsch2.tex|&include file for chapter 2\\
% |cdocspt3.tex|&include file for part 3\\
% |cdocspt4.tex|&include file for part 4\\
% |cdocsdrf.tex|&forwarding file for main file in draft mode\\
% |cdocsfi1.tex|&forwarding file for final version of chapter 1\\
% |cdocsfi2.tex|&forwarding file for final version of chapter 2\\
% \end{tabular}
% \end{center}
% Each of the eight files can be compiled directly by the \LaTeX{} compiler.
%
% %%%%%%%%%%%%%%%%%%%%%%%%%%%%%%%%%%%%%%
% \paragraph{Main File.}
%
% The main file is called |cdocsamp.tex|.
%
% Load the \textsf{childdoc} definitions and
% declare the filename for the main document:
%    \begin{macrocode}
\input{childdoc.def}
\childdocmain{}
%    \end{macrocode}

% Optional override for |\version| flag:
%    \begin{macrocode}
%%\ifchilddoc\else\providecommand{\version}{draft}\fi
%    \end{macrocode}

% Define the default values for the |\version| flag
% (|final| for the main file and |draft| for childs):
%    \begin{macrocode}
\ifchilddoc
\providecommand{\version}{draft}
\else
\providecommand{\version}{final}
\fi
%    \end{macrocode}

% Load the standard document class:
%    \begin{macrocode}
\documentclass[12pt]{article}
%    \end{macrocode}

% Start the document body:
%    \begin{macrocode}
\begin{document}
%    \end{macrocode}

% Declare a title page.
% Print title, part of document being processed and version flag:
%    \begin{macrocode}
\addtocounter{page}{-1}
\begin{center}
{\LARGE\bfseries{}childdoc example\par}
\vspace{1cm}
\ifchilddoc
\ifchilddocmanual part\else chapter\fi:
`\childdocname' of `\childdocjob'\par
\else
main document: `\childdocjob'\par
\fi
version: \version\par
\end{center}
\newpage
%    \end{macrocode}

% Manually include selected file,
% otherwise process as usual:
%    \begin{macrocode}
\ifchilddocmanual
\section*{part `\childdocname'}
\input{\childdocname}
\else
%    \end{macrocode}

% Include the two chapters:
%    \begin{macrocode}
\include{cdocsch1}
\include{cdocsch2}
%    \end{macrocode}

% Include the two parts unless only chapters should be displayed:
%    \begin{macrocode}
\ifchilddoc\else
\section{part three}
\input{cdocspt3}
\section{part four}
\input{cdocspt4}
\fi
%    \end{macrocode}

% Process as usual until here:
%    \begin{macrocode}
\fi
%    \end{macrocode}

% End of document body:
%    \begin{macrocode}
\end{document}
%    \end{macrocode}
%\iffalse
%</samplemain>
%\fi
%
% %%%%%%%%%%%%%%%%%%%%%%%%%%%%%%%%%%%%%%
% \paragraph{Chapter Include Files.}
%
% The include files are called |cdocsch1.tex| and |cdocsch2.tex|.
%
%\iffalse
%<*samplechap1|samplechap2>
%\fi

% Optional override for |\version| flag:
%    \begin{macrocode}
%%\providecommand{\version}{final}
%    \end{macrocode}

% Include the main document:
%    \begin{macrocode}
\input{childdoc.def}
\childdocof{cdocsamp}
%    \end{macrocode}

%\iffalse
%</samplechap1|samplechap2>
%\fi
%
%\iffalse
%<*samplechap1>
%\fi
% Some text for chapter 1:
%    \begin{macrocode}
\section{one}
some text in chapter one
%    \end{macrocode}

%\iffalse
%</samplechap1>
%\fi
% Some text for chapter 2:
%\iffalse
%<*samplechap2>
%\fi
%    \begin{macrocode}
\section{two}
more text in chapter two
%    \end{macrocode}

%\iffalse
%</samplechap2>
%\fi
%
% %%%%%%%%%%%%%%%%%%%%%%%%%%%%%%%%%%%%%%
% \paragraph{Part Include Files.}
%
% The include files are called |cdocspt3.tex| and |cdocspt4.tex|.
%
%\iffalse
%<*samplepart3|samplepart4>
%\fi

% Optional override for |\version| flag:
%    \begin{macrocode}
%%\providecommand{\version}{final}
%    \end{macrocode}

% Include the main document:
%    \begin{macrocode}
\input{childdoc.def}
\childdocby{cdocsamp}
%    \end{macrocode}

%\iffalse
%</samplepart3|samplepart4>
%\fi
%
%\iffalse
%<*samplepart3>
%\fi
% Some text for part 3:
%    \begin{macrocode}
some text in part three
%    \end{macrocode}

%\iffalse
%</samplepart3>
%\fi
% Some text for part 4:
%\iffalse
%<*samplepart4>
%\fi
%    \begin{macrocode}
more text in part four
%    \end{macrocode}

%\iffalse
%</samplepart4>
%\fi
%
% %%%%%%%%%%%%%%%%%%%%%%%%%%%%%%%%%%%%%%
% \paragraph{Forwarding for a Complete Draft.}
%
% The following forwarding file |cdocsdrf.tex|
% compiles the main document in draft mode:
%\iffalse
%<*sampledraft>
%\fi
%    \begin{macrocode}
\def\version{draft}
\input{childdoc.def}
\childdocforward{cdocsamp}
%    \end{macrocode}

%\iffalse
%</sampledraft>
%\fi
%
% %%%%%%%%%%%%%%%%%%%%%%%%%%%%%%%%%%%%%%
% \paragraph{Forwarding for Final Version of the Chapters.}
%
% The following forwarding files |cdocsfn1.tex| and |cdocsfn2.tex|
% (with identical content)
% compile the final versions of the child documents
% |cdocsch1.tex| and |cdocsch2.tex|, respectively:
%\iffalse
%<*samplefinal>
%\fi
%    \begin{macrocode}
\def\version{final}
\input{childdoc.def}
\childdocforwardprefix[cdocsamp]{cdocsfn}{cdocsch}
%    \end{macrocode}

%\iffalse
%</samplefinal>
%\fi
%
% %%%%%%%%%%%%%%%%%%%%%%%%%%%%%%%%%%%%%%
% \paragraph{Command Line Processing.}
%
% The following three command lines generate the output files
% |cdocscld|, |cdocscl1| and |cdocscl2|
% which should be identical to
% |cdocsdrf|, |cdocsch1| and |cdocsfn2|, respectively:
% \begin{center}
% \begin{tabular}{l}
% |latex -jobname cdocscld \|\\
% |  "\def\version{draft}\input{childdoc.def}\childdocforward{cdocsamp}"|\\
% |latex -jobname cdocscl1 \|\\
% |  "\input{childdoc.def}\childdocforward[cdocsamp]{cdocsch1}"|\\
% |latex -jobname cdocscl2 \|\\
% |  "\def\version{final}\input{childdoc.def}\childdocforward{cdocsch2}"|
% \end{tabular}
% \end{center}
% Note that the trailing backslash on each first line
% merely continues the input to the second line
% (for convenient cut ant paste).
% Furthermore, the command |latex| can be replaced by any
% of its alternative versions such as |pdflatex|.
%
% %%%%%%%%%%%%%%%%%%%%%%%%%%%%%%%%%%%%%%%%%%%%%%%%%%%%%%%%%%%%%%%%%%%%%%%%%%%%%%
% %%%%%%%%%%%%%%%%%%%%%%%%%%%%%%%%%%%%%%%%%%%%%%%%%%%%%%%%%%%%%%%%%%%%%%%%%%%%%%
% \section{Implementation}
%\iffalse
%<*package>
%\fi
%
% This section describes the definitions file |childdoc.def|.

% The definitions cannot be loaded using |\usepackage| or |\RequirePackage|
% which has a mechanism to prevent loading a style file more than once.
% When loading the definitions by means of |\input|
% multiple instances have to be prevented manually:
%\iffalse
%This code needs to be before the `\ProvidesFile' directive
%which is defined at the beginning of this file.
%Therefore it is also placed there and commented out here.
%</package>
%<*discard>
%\fi
%    \begin{macrocode}
\ifdefined\childdocmain\endinput\fi
%    \end{macrocode}
%\iffalse
%</discard>
%<*package>
%\fi
%
% \macro{\ifchilddoc}
% \macro{\ifchilddocmanual}
% The conditional |\ifchilddoc| tells whether a
% child (true) or main (false) document is being compiled.
% The conditional |\ifchilddocmanual| tells whether
% the |\includeonly| mechanism is used (false) or
% the selection of child files must be performed manually (true).
% The definitions initialise to false:
%    \begin{macrocode}
\newif\ifchilddoc
\newif\ifchilddocmanual
%    \end{macrocode}

% \macro{\childdocname}
% \macro{\childdocjob}
% The macro |\childdocname| stores the name of the main document
% to be compiled. The macro |\childdocjob| stores the name of
% the document on which the \LaTeX{} compiler was originally invoked.
% The content of |\jobname| cannot be compared
% to filenames specified in the source due to different catcodes.
% The following code rescans |\jobname|, stores the result
% in |\childdocname| and saves a copy in |\childdocjob|:
%    \begin{macrocode}
\edef\childdocname{\scantokens\expandafter{\jobname\noexpand}}
\let\childdocjob\childdocname
%    \end{macrocode}

% \macro{\childdocdisable}
% The macro |\childdocdisable| prevents the main file
% from being processed more than once.
% At this stage, the main document command |\childdocmain|
% is assumed to be called once again where it should do nothing.
% Any subsequent call to it should prevent
% a secondary processing of the main document
% It overwrites the forwarding commands
% |\childdocof| and |\childdocforward|
% with empty macros to prevent further inclusions of the main document:
%    \begin{macrocode}
\newcommand{\childdocdisable}
{
  \renewcommand{\childdocmain}[1]{\renewcommand{\childdocmain}[1]{\endinput}}
  \renewcommand{\childdocof}[1]{}
  \renewcommand{\childdocby}[2][]{}
  \renewcommand{\childdocforward}[2][]{}
  \renewcommand{\childdocdisable}{}
}
%    \end{macrocode}

% \macro{\childdocmain}
% The macro |\childdocmain| is to be called at the top of the main file
% with nothing or the main filename (without extension) as argument.
% First, it breaks loops.
% If the argument is not empty and does not match |\childdocname|
% (which is set by the first inclusion of |childdoc.def|),
% |\ifchilddoc| is set to true, |\includeonly| is applied to the child file
% and |\jobname| is set to the main file
% (for proper handling of |.aux| files):
%    \begin{macrocode}
\newcommand{\childdocmain}[1]
{
  \childdocdisable\childdocmain{}
  \if?#1?\else
    \begingroup
      \def\childdoctmp{#1}
      \ifx\childdoctmp\childdocname
        \def\childdoctmp{}
      \else
        \def\childdoctmp
        {
          \childdoctrue
          \includeonly{\childdocname}
          \def\childdocjob{#1}
          \def\jobname{#1}
        }
      \fi
      \expandafter
    \endgroup
    \childdoctmp
  \fi
}
%    \end{macrocode}

% \macro{\childdocof}
% The command |\childdocof| redirects
% compilation to the main file |#1|.
%    \begin{macrocode}
\newcommand{\childdocof}[1]
{
  \childdocdisable
  \childdoctrue
  \includeonly{\childdocname}
  \def\jobname{#1}
  \def\childdocjob{#1}
  \input{#1}
}
%    \end{macrocode}

% \macro{\childdocby}
% The command |\childdocby| ....
%    \begin{macrocode}
\newcommand{\childdocby}[2][]
{
  \childdocdisable
  \childdoctrue
  \childdocmanualtrue
  \if?#1?\else
    \def\jobname{#2}
  \fi
  \def\childdocjob{#2}
  \input{#2}
  \endinput
}
%    \end{macrocode}

% \macro{\childdocforward}
% The command |\childdocforward| redirects
% compilation to the main file or
% (if the optional argument is given) a child file.
% Parameters are set as if the main file
% or a child file starting with |\childdocof| was compiled.
% Then compilation is handed over to the main file:
%    \begin{macrocode}
\newcommand{\childdocforward}[2][]
{
  \begingroup
    \if?#1?
      \def\childdoctmp
      {
        \def\childdocname{#2}
        \def\childdocjob{#2}
        \def\jobname{#2}
        \input{#2}
        \endinput
      }
    \else
      \def\childdoctmp
      {
        \childdocdisable
        \def\childdocname{#2}
        \childdoctrue
        \includeonly{#2}
        \def\childdocjob{#1}
        \def\jobname{#1}
        \input{#1}
        \endinput
      }
    \fi
    \expandafter
  \endgroup
  \childdoctmp
}
%    \end{macrocode}

% \macro{\childdocforwardprefix}
% The command |\childdocforwardprefix| redirects
% compilation to the main or a child file by means of a pattern.
% The prefix |#1| in the current filename is replaced by |#2|
% and the suffix of the current filename is kept
% (it is assumed that the filename does not contain the substring `|~~~|'
% which is used as a delimiter).
% Compilation is handed over to the new file by |\childdocforward|:
%    \begin{macrocode}
\newcommand{\childdocforwardprefix}[3][]
{
  \begingroup
    \def\childdocextract #2##1~~~{\def\childdoctmp{\childdocforward[#1]{#3##1}}}
    \expandafter\childdocextract\childdocname~~~
    \expandafter
  \endgroup
  \childdoctmp
}
%    \end{macrocode}

% \macro{\childdoc}
% The deprecated macro |\childdoc| is a legacy version of |\childdocmain|:
%    \begin{macrocode}
\newcommand{\childdoc}{\childdocmain}
%    \end{macrocode}

% \macro{\childdocredirect}
% The deprecated macro |\childdocredirect| is a legacy version
% of |\childdocforward| and |\childdocforwardprefix|:
%    \begin{macrocode}
\newcommand{\childdocredirect}[2][]
{
  \begingroup
    \if?#1?
      \def\childdoctmp{\childdocforward{#2}}
    \else
      \def\childdoctmp{\childdocforwardprefix{#1}{#2}}
    \fi
    \expandafter
  \endgroup
  \childdoctmp
}
%    \end{macrocode}

%\iffalse
%</package>
%\fi
%
\endinput
\childdocforward[cdocsamp]{cdocsch1}"|\\
% |latex -jobname cdocscl2 \|\\
% |  "\def\version{final}% \iffalse
%
% childdoc.dtx Copyright (C) 2017-2018 Niklas Beisert
%
% This work may be distributed and/or modified under the
% conditions of the LaTeX Project Public License, either version 1.3
% of this license or (at your option) any later version.
% The latest version of this license is in
%   http://www.latex-project.org/lppl.txt
% and version 1.3 or later is part of all distributions of LaTeX
% version 2005/12/01 or later.
%
% This work has the LPPL maintenance status `maintained'.
%
% The Current Maintainer of this work is Niklas Beisert.
%
% This work consists of the files childdoc.dtx and childdoc.ins
% and the derived files childdoc.def and cdocsamp.tex with
% cdocsch1.tex, cdocsch2.tex, cdocsdrf.tex, cdocsfn1.tex, cdocsfn2.tex.
%
%<package>\ifdefined\childdocmain\endinput\fi
%<package>\ProvidesFile{childdoc.def}[2018/12/30 v2.0 child document driver]
%<samplemain>\ProvidesFile{cdocsamp.tex}[2018/12/30 v2.0 sample for childdoc]
%<*driver>
%\ProvidesFile{childdoc.drv}[2018/12/30 v2.0 childdoc reference manual file]
\PassOptionsToClass{10pt,a4paper}{article}
\documentclass{ltxdoc}

\usepackage[margin=35mm]{geometry}
\usepackage{hyperref}
\usepackage{hyperxmp}
\usepackage[usenames]{color}

\hypersetup{colorlinks=true}
\hypersetup{pdfstartview=FitH}
\hypersetup{pdfpagemode=UseNone}
\hypersetup{pdfsource={}}
\hypersetup{pdflang={en-UK}}
\hypersetup{pdfcopyright={Copyright 2017-2018 Niklas Beisert.
  This work may be distributed and/or modified under the
  conditions of the LaTeX Project Public License, either version 1.3
  of this license or (at your option) any later version.}}
\hypersetup{pdflicenseurl={http://www.latex-project.org/lppl.txt}}
\hypersetup{pdfcontactaddress={ETH Zurich, ITP, HIT K,
  Wolfgang-Pauli-Strasse 27}}
\hypersetup{pdfcontactpostcode={8093}}
\hypersetup{pdfcontactcity={Zurich}}
\hypersetup{pdfcontactcountry={Switzerland}}
\hypersetup{pdfcontactemail={nbeisert@itp.phys.ethz.ch}}
\hypersetup{pdfcontacturl={http://people.phys.ethz.ch/\xmptilde nbeisert/}}

\newcommand{\secref}[1]{\hyperref[#1]{section \ref*{#1}}}

\parskip1ex
\parindent0pt
\let\olditemize\itemize
\def\itemize{\olditemize\parskip0pt}

\begin{document}

\title{The \textsf{childdoc} Package}
\hypersetup{pdftitle={The childdoc Package}}
\author{Niklas Beisert\\[2ex]
  Institut f\"ur Theoretische Physik\\
  Eidgen\"ossische Technische Hochschule Z\"urich\\
  Wolfgang-Pauli-Strasse 27, 8093 Z\"urich, Switzerland\\[1ex]
  \href{mailto:nbeisert@itp.phys.ethz.ch}
  {\texttt{nbeisert@itp.phys.ethz.ch}}}
\hypersetup{pdfauthor={Niklas Beisert}}
\hypersetup{pdfsubject={Manual for the LaTeX2e Package childdoc}}
\date{30 December 2018, \textsf{v2.0}}
\maketitle

\begin{abstract}\noindent
\textsf{childdoc} is a \LaTeXe{} package
that enables the direct compilation
of document sections included by |\include|
to individual files.
\end{abstract}

\begingroup
\parskip0ex
\tableofcontents
\endgroup

%%%%%%%%%%%%%%%%%%%%%%%%%%%%%%%%%%%%%%%%%%%%%%%%%%%%%%%%%%%%%%%%%%%%%%%%%%%%%%%%
%%%%%%%%%%%%%%%%%%%%%%%%%%%%%%%%%%%%%%%%%%%%%%%%%%%%%%%%%%%%%%%%%%%%%%%%%%%%%%%%
\section{Introduction}

\LaTeX{} provides a mechanism to structure a large document (such as a book)
into a main file and several child files (containing the chapters)
using the |\include| command.
This mechanism is beneficial for documents
which span hundreds of pages in order to
make the source file(s) more manageable.
Moreover, compilation can be restricted to
selected child files by means of the |\includeonly| command.
The latter feature can be used to reduce the compilation time while editing
(this was significantly more useful in the earlier days of \LaTeX{})
or to generate a smaller document which is easier to navigate.
Another application of |\includeonly| is to generate
documents consisting of selected parts of the complete document.

However, there are a few drawbacks of the plain |\include| mechanism:
\begin{itemize}
\item
The child files cannot be compiled on their own,
they can only be compiled via the main file.
A naive editing environment
(such as a text editor with an option
to have the current file processed by \LaTeX)
may require one to switch to the main file before compiling;
attempting to compile the child file produces errors.
\item
The main file must be modified (each time)
to adjust the |\includeonly| command
to the present needs. This easily leaves the main file in a messy state.
\item
The generated document will always carry the filename
of the main document. This is inconvenient if
several child files are to be compiled and
to be kept for distribution.
\end{itemize}

The present package provides a simple interface
to make child files individually compilable by \LaTeX{}.
Compiling a child file then has the same effect as compiling
the main file with an |\includeonly| command
to select the appropriate child.
Moreover the generated document will carry the name of the child
rather than the main file.
This resolves all three above issues.

This feature is meant to make the editing of books,
thesis documents and lecture notes somewhat more convenient.
However, the package can also be used efficiently for
composing a series of documents (such as exercise sheets)
which are typically distributed individually.
It then assists the author in generating the individual documents
(potentially in different versions)
as well as a document containing the collected series.
Another application is in developing style files
or other kinds of included material
where compilation of the style file could redirect
to a sample or test file.

%%%%%%%%%%%%%%%%%%%%%%%%%%%%%%%%%%%%%%%%%%%%%%%%%%%%%%%%%%%%%%%%%%%%%%%%%%%%%%%%
%%%%%%%%%%%%%%%%%%%%%%%%%%%%%%%%%%%%%%%%%%%%%%%%%%%%%%%%%%%%%%%%%%%%%%%%%%%%%%%%
\section{Usage}

First of all, the package \textsf{childdoc} is \emph{not} a standard
\LaTeXe{} |.sty| style file! Therefore it needs to be invoked in
a non-standard way.

%%%%%%%%%%%%%%%%%%%%%%%%%%%%%%%%%%%%%%%%%%%%%%%%%%%%%%%%%%%%%%%%%%%%%%%%%%%%%%%%
\subsection{Included Files}
\label{sec:include}

%%%%%%%%%%%%%%%%%%%%%%%%%%%%%%%%%%%%%%%%
\DescribeMacro{\childdocmain}
To use the package, add the commands
\begin{center}
\begin{tabular}{l}
|\input{childdoc.def}|\\
|\childdocmain{}|\\
\end{tabular}
\end{center}
at the very top of the main \LaTeX{} file,
in particular \emph{before} the |\documentclass| statement!
The argument of |\childdocmain| should be left empty
(but it must be present).

%%%%%%%%%%%%%%%%%%%%%%%%%%%%%%%%%%%%%%%%
\DescribeMacro{\childdocof}
Furthermore, add the commands
\begin{center}
\begin{tabular}{l}
|\input{childdoc.def}|\\
|\childdocof{|\textit{main}|}|\\
\end{tabular}
\end{center}
at the top of every child file \textit{child}
which is included by |\include{|\textit{child}|}|
from within the main file
(or at least for those files to be compiled individually).
The argument \textit{main} must be the filename of the main file.

There are a couple of
considerations in setting up the main and child documents:

%%%%%%%%%%%%%%%%%%%%%%%%%%%%%%%%%%%%%%%%
\paragraph{Restrictions.}

Please note the following restrictions:
\begin{itemize}
\item
|\childdocmain| must be called with one argument \textit{main}
to ensure compatibility with earlier version of the package.
It must either be empty (|\childdocmain{}|)
or precisely match the filename of the main file in which it is specified.
See \secref{sec:detection} for further information.
\item
The filename \textit{main} must be specified without the |.tex| extension.
\item
The filename \textit{main} is case sensitive
(even in case-insensitive file systems)
due to internal string comparison.
\item
The argument \textit{main} should be fully expanded, it cannot be a macro.
\item
Subdirectories and special characters should be avoided in filenames.
\item
The command |\childdocmain{|\textit{main}|}| must be followed by a whitespace.
It should not be followed immediately by another command
or by a comment mark `|%|'.
This is because the \TeX{} parser reads the token immediately following
the argument of |\childdocmain| and puts it
at the beginning of every child section;
however, a white\-space is ignored.
\end{itemize}

%%%%%%%%%%%%%%%%%%%%%%%%%%%%%%%%%%%%%%%%
\paragraph{Content of Main File.}

It is advisable to place all content in the child files included by |\include|.
Any output contained in the main file will appear in all child documents
unless suppressed manually;
it cannot be suppressed automatically by the |\includeonly| directive
and thus should normally be avoided.
A method to include some content in the main file
by means of conditional processing is described in \secref{sec:conditional}.

%%%%%%%%%%%%%%%%%%%%%%%%%%%%%%%%%%%%%%%%
\paragraph{Page Numbering.}

When only a part of the document is compiled,
the appropriate numbering of pages
(as well as other status parameters)
is determined from the |.aux| files.
The latter contain information from previous passes.
However this information needs to propagate through
all intermediate child documents.
Therefore the page numbering in child documents may well
be inconsistent until the complete document is compiled at least once.

A useful (if unconventional) way to always ensure a consistent
page numbering is to restart the numbering in each child document
and denote the pages by `\textit{child}|.|\textit{page}'
where \textit{child} represents the chapter/section number of the child file.
This can be achieved by the command
|\numberwithin{page}{|\textit{child}|}|
of the \textsf{amsmath} package
where \textit{child} can be |chapter| or |section|
depending on the chosen structuring.
Alternatively, one can modify the macro |\thepage| appropriately
and reset the counter |page| at the start of each child file.

%%%%%%%%%%%%%%%%%%%%%%%%%%%%%%%%%%%%%%%%%%%%%%%%%%%%%%%%%%%%%%%%%%%%%%%%%%%%%%%%
\subsection{Conditional Processing}
\label{sec:conditional}

The package provides a mechanism to compile different versions
of a document. To customise the versions further some conditional processing
can come in handy to distinguish which version is being compiled.
The package provides two macros to describe the compilation context:

%%%%%%%%%%%%%%%%%%%%%%%%%%%%%%%%%%%%%%%%
\DescribeMacro{\ifchilddoc}
The conditional |\ifchilddoc| distinguishes between the compilation of
child documents and the main document:
%
\begin{center}
|\ifchilddoc |\textit{child-code}| |[|\||else |\textit{main-code}]| \||fi|
\end{center}

%%%%%%%%%%%%%%%%%%%%%%%%%%%%%%%%%%%%%%%%
\DescribeMacro{\childdocname}
\DescribeMacro{\childdocjob}
The macro |\childdocname| contains the filename (without extension)
of the main or child file being processed.
Note that |\childdocjob| will always contain the name of the main file.

%%%%%%%%%%%%%%%%%%%%%%%%%%%%%%%%%%%%%%%%
\paragraph{Title Page.}

Conditional processing can be used to include a title or banner page
in the main document when proper precautions are taken.
Importantly, the code in the main file should ensure that the page counter
(as well as other status parameters which are stored in the |.aux| files)
takes the same value after the conditional processing.
Otherwise the page numbers may take divergent values
depending on which part is compiled.

For example, a title page could be declared by:
%
\begin{center}
\begin{tabular}{l}
|\ifchilddoc\||else|\\
|\addtocounter{page}{-1}|\\
\textit{code for title page}\\
|\newpage|\\
|\||fi|
\end{tabular}
\end{center}
%
A banner page for the child documents can be generated by:
%
\begin{center}
\begin{tabular}{l}
|\ifchilddoc|\\
|\addtocounter{page}{-1}|\\
\textit{code for banner page}\\
|\newpage|\\
|\||fi|
\end{tabular}
\end{center}
%
Here one could write a message such as:
\begin{center}
|This is the part \childdocname{} of \childdocjob{}.|
\end{center}

%%%%%%%%%%%%%%%%%%%%%%%%%%%%%%%%%%%%%%%%%%%%%%%%%%%%%%%%%%%%%%%%%%%%%%%%%%%%%%%%
\subsection{Flags}
\label{sec:flags}

The package makes it easy to generate different versions
of the main or child documents.
To this end compilation flags can be defined
and assigned different default values.
They will be particularly useful in conjunction
with the forwarding mechanism described in \secref{sec:forward}.

For example, it may be useful to have a flag |\version|
which can be set to |draft| or |final|.
The document source will contain some conditional code
depending on the value of |\version|.
Suppose further, the flag should default to |final| for the main file
and to |draft| for child files
which is a natural assignment for editing the document.
This is achieved by placing the following code
in the preamble of the main document
(below the |\childdocmain| directive):
%
\begin{center}
\begin{tabular}{l}
|\ifchilddoc|\\
|\providecommand{\version}{draft}|\\
|\||else|\\
|\providecommand{\version}{final}|\\
|\||fi|
\end{tabular}
\end{center}
%
The definition by |\providecommand| makes sure
that previous definitions are not overwritten.
Further statements |\providecommand{\version}{...}|
can thus be added before the above code to override it.

For the main file, one might add a line
(between |\childdocmain| and the above block)
%
\begin{center}
|%\ifchilddoc\||else\providecommand{\version}{draft}\||fi|
\end{center}
%
which can be uncommented to produce a draft version.
Likewise one can add a line to the very top of a child file
(above the |\childdocof{|\textit{main}|}| directive)
%
\begin{center}
|%\providecommand{\version}{final}|
\end{center}
%
which can be uncommented to produce the final version of this child document.

%%%%%%%%%%%%%%%%%%%%%%%%%%%%%%%%%%%%%%%%%%%%%%%%%%%%%%%%%%%%%%%%%%%%%%%%%%%%%%%%
\subsection{Forwarding}
\label{sec:forward}

Different versions of the main or child documents
using compilation flags as described in \secref{sec:flags}
can be (permanently) stored in different files
for convenient compilation, viewing and distribution.
To this end, the package defines a command
to pass on compilation to a different file:

%%%%%%%%%%%%%%%%%%%%%%%%%%%%%%%%%%%%%%%%
\DescribeMacro{\childdocforward}
The command |\childdocforward| redirects processing to
another source file:
%
\begin{center}
\begin{tabular}{l}
|\input{childdoc.def}|\\
|\childdocforward[|\textit{main}|]{|\textit{dest}|}|\\
\end{tabular}
\end{center}
%
The argument \textit{dest} is the destination file
(without extension).
It should be the main file or one of the child files.
Note that further \textsf{childdoc} directives
such as |\childdocof| and |\childdocforward|
in the indicated file will be processed in this form.
The optional argument \textit{main}
passes on directly to the main file \textit{main}
while pretending to compile the child \textit{dest}.
This form behaves as if \textit{dest}
issues |\childdocof{|\textit{main}|}| right away,
and no further \textsf{childdoc} directives will be processed.

%%%%%%%%%%%%%%%%%%%%%%%%%%%%%%%%%%%%%%%%
\DescribeMacro{\...prefix}
In the alternative form |\childdocforwardprefix|,
%
\begin{center}
\begin{tabular}{l}
|\input{childdoc.def}|\\
|\childdocforwardprefix[|\textit{main}|]{|\textit{prefix}|}{|\textit{dest}|}|
\end{tabular}
\end{center}
%
the destination file is determined by a pattern
depending on the current file:
To make this work, the current file must be called
`{\textit{prefix}\hspace{0.2em}\textit{suffix}}'
with \textit{prefix} matching precisely the argument.
Processing is then passed on to the file
`{\textit{dest}\hspace{0.2em}\textit{suffix}}'.
Surely, the same effect is achieved by
directly specifying the
argument `{\textit{dest}\hspace{0.2em}\textit{suffix}}'
in the first form.
However, that requires to set up a different file
for each child. With the alternative form of the command
all these files can have exactly the same content
which simplifies setting them up and maintaining them.

For example, the following file |draft.tex|
with a compilation flag |\version| as described in \secref{sec:flags}
compiles the main document as a draft:
%
\begin{center}
\begin{tabular}{l}
|\def\version{draft}|\\
|\input{childdoc.def}|\\
|\childdocforward{|\textit{main}|}|
\end{tabular}
\end{center}
%
Likewise, the following files |final|\textit{nn}|.tex|
compile the final version of the child document
|child|\textit{nn}|.tex|:
%
\begin{center}
\begin{tabular}{l}
|\def\version{final}|\\
|\input{childdoc.def}|\\
|\childdocforwardprefix{final}{child}|
\end{tabular}
\end{center}
%

Note that when several versions of a main file and/or of each child file
are to be generated, it may be convenient to set up a |Makefile| or
shell script to automatise the process.

%%%%%%%%%%%%%%%%%%%%%%%%%%%%%%%%%%%%%%%%%%%%%%%%%%%%%%%%%%%%%%%%%%%%%%%%%%%%%%%%
\subsection{Command Line Processing}
\label{sec:commandline}

The effect of redirection files can also be achieved by invoking
the \LaTeX{} compiler with a more elaborate command line.
Most conveniently this should be done as part
of a shell script or a |Makefile|.

When using \textsf{childdoc} in the main file, the following
command lines effectively perform a redirection
(note that depending on the shell being used,
backslashes may have to be doubled: `|\|' $\to$ `|\\|'):
%
\begin{center}
|... -jobname "|\textit{target}|" |\\|"|[\textit{flags}]%
|\input{childdoc.def}\childdocforward[|\textit{main}|]{|\textit{dest}|}"|
\end{center}
%
Here \textit{target} is the name of the output file,
\textit{main} is the name of the main file
and \textit{dest} is the name of the main or child file to be processed
(all filenames without extensions).
The optional argument \textit{main} can be omitted
if \textit{main} matches \textit{dest}.
Optionally, compilation \textit{flags} can be defined via |\def| commands.
This command line makes the \TeX{} engine believe
it is compiling the file \textit{target}
whose content is specified as the latter parameter.
The provided code then forwards the processing to
\textit{main} or \textit{dest} as described in \secref{sec:forward}.

%%%%%%%%%%%%%%%%%%%%%%%%%%%%%%%%%%%%%%%%%%%%%%%%%%%%%%%%%%%%%%%%%%%%%%%%%%%%%%%%
\subsection{Include by Input}
\label{sec:input}

Including child documents by |\include| has some restrictions by design.
Most notably, the content of a child document always occupies
its own set of pages; pages cannot be shared between child documents.
Usually, this behaviour makes perfect sense
because each child document contain an essential part of the document.
However, in some situations it may be desirable to compose
a document from a collection of parts
without having mandatory page breaks between then.
For this case, the package
provides a mechanism to include parts
by |\input| which can also be processed individually.
However, by construction this mechanism
requires manual handling of the content to be output.

%%%%%%%%%%%%%%%%%%%%%%%%%%%%%%%%%%%%%%%%
\DescribeMacro{\ifchilddocmanual}
The main file should be prepared as usual, see \secref{sec:include}.
However, the document body must make a distinction
between processing of an individual part and of the main document, e.g.:
%
\begin{center}
\begin{tabular}{l}
|\ifchilddocmanual|\\
|\input{\childdocname}|\\
|\||else|\\
\textit{document body with }|\input{|\textit{part}|}|\\
|\||fi|
\end{tabular}
\end{center}
%
The conditional |\ifchilddocmanual| is true whenever
a part to be included by |\input| is being compiled,
and the name of the part is stored in |\childdocname|.

%%%%%%%%%%%%%%%%%%%%%%%%%%%%%%%%%%%%%%%%
\DescribeMacro{\childdocby}
Each part to be included by |\input| should start with:
%
\begin{center}
\begin{tabular}{l}
|\input{childdoc.def}|\\
|\childdocby{|\textit{main}|}|\\
\end{tabular}
\end{center}
%
The directive |\childdocby| is similar to |\childdocof|
described in \secref{sec:include},
but the subsequent selection of content must be done manually.
To that end, both |\ifchilddoc| and |\ifchilddocmanual|
will be true upon processing of a part,
and the name of the part is stored in |\childdocname|.
Note that |\jobname| will be set to the filename of the current part
so that each part receives an individual |.aux| file
that does not interfere with the |.aux| file(s) of the main document.
This behaviour can be altered by the alternative form
|\childdocby[*]{|\textit{main}|}| (with a non-empty optional argument)
which uses the |.aux| file of the main document
by setting |\jobname| to \textit{main}.

%%%%%%%%%%%%%%%%%%%%%%%%%%%%%%%%%%%%%%%%%%%%%%%%%%%%%%%%%%%%%%%%%%%%%%%%%%%%%%%%
\subsection{Driver Development}
\label{sec:driver}

The \textsf{childdoc} mechanism can also be use for the development
of definition files such as \LaTeX{} styles or classes.
This case differs from the above setup with multiple parts
included by |\include| in that no |\includeonly| should be invoked.
This can be achieved by starting the include file
(before |\ProvidesPackage|) with:
%
\begin{center}
\begin{tabular}{l}
|\input{childdoc.def}|\\
|\childdocforward{|\textit{main}|}|\\
\end{tabular}
\end{center}
%
or alternatively with:
%
\begin{center}
\begin{tabular}{l}
|\input{childdoc.def}|\\
|\childdocby{|\textit{main}|}|\\
\end{tabular}
\end{center}
%
Both forms have slightly different effects as described above.
The main file is prepared as usual, see \secref{sec:include}.

%%%%%%%%%%%%%%%%%%%%%%%%%%%%%%%%%%%%%%%%%%%%%%%%%%%%%%%%%%%%%%%%%%%%%%%%%%%%%%%%
\subsection{Legacy Detection}
\label{sec:detection}

The directive |\childdocmain| in the main file can detect
whether the complete document or merely a child is to be compiled
even without using the directive |\childdocof|.
This method is deprecated because it is less robust
and there is no compelling reason to use it;
it is merely provided for backward compatibility
and it may be removed in future versions.

If the detection mechanism is to be used,
it is mandatory to correctly specify
the filename of the main file as the argument of |\childdocmain|:
%
\begin{center}
\begin{tabular}{l}
|\input{childdoc.def}|\\
|\childdocmain{|\textit{main}|}|\\
\end{tabular}
\end{center}
%
If |\jobname| does not match the argument \textit{main} of |\childdocmain|,
it is assumed that |\jobname| points to the child file to be compiled.
When using |\childdocmain| with the main file specified as argument,
it suffices to start a child file
with just |\input{|\textit{main}|}|
without loading of the package and using |\childdocof|.
If instead all processing is done
with the appropriate \textsf{childdoc} directives,
the argument of \textit{main} of |\childdocmain| can be empty.

An alternative version of the command line processing described
in \secref{sec:commandline} using the detection mechanism reads:
%
\begin{center}
|... -jobname "|\textit{target}|" "|[\textit{flags}]%
[|\def\jobname{|\textit{dest}|}|]|\input{|\textit{main}|}"|
\end{center}

%%%%%%%%%%%%%%%%%%%%%%%%%%%%%%%%%%%%%%%%%%%%%%%%%%%%%%%%%%%%%%%%%%%%%%%%%%%%%%%%
\subsection{Manual Code}
\label{sec:manual}

In case one cannot be certain whether the definitions file |childdoc.def|
is installed on the target \TeX{} distribution
and one prefers not to ship it,
it is conceivable to paste a few relevant commands into the sources.

To that end, drop all statements |\input{childdoc.def}|
and perform the replacements as outlined below.
Instead of |\childdocmain{|\textit{main}|}| add the following code
to the top of the main file:
%
\begin{center}
\begin{tabular}{l}
|\||ifdefined\childdocname\endinput\||fi\newif\ifchilddoc|\\
|\edef\childdocname{\scantokens\expandafter{\jobname\noexpand}}|\\
|\def\childdocmain{|\textit{main}|}\||ifx\childdocmain\childdocname\||else|\\
|\childdoctrue\includeonly{\childdocname}\let\jobname\childdocmain\||fi|\\
\end{tabular}
\end{center}
%
Instead of |\childdocof{|\textit{main}|}| just include the main file
at the top of each child file:
%
\begin{center}
|\input{|\textit{main}|}|
\end{center}
%
A simple redirection |\childdocforward{|\textit{dest}|}| is achieved by:
%
\begin{center}
|\def\jobname{|\textit{dest}|}\input{\jobname}|
\end{center}
%
The redirection with prefix
|\childdocforwardprefix[|\textit{prefix}|]{|\textit{dest}|}|
is accomplished by:
%
\begin{center}
\begin{tabular}{l}
|{\edef\jobname{\scantokens\expandafter{\jobname\noexpand}}|\\
|\def\redirectjob |\textit{prefix}|#1~~~{\gdef\jobname{|\textit{dest}|#1}}|\\
|\expandafter\redirectjob\jobname~~~}\input{\jobname}|
\end{tabular}
\end{center}

In an alternative approach,
child documents can be compiled by a specific command line
without additional code or specific definitions:
%
\begin{center}
|... -jobname "|\textit{target}|" "|[\textit{flags}]%
|\includeonly{|\textit{dest}|}\input{|\textit{main}|}"|
\end{center}
%

%%%%%%%%%%%%%%%%%%%%%%%%%%%%%%%%%%%%%%%%%%%%%%%%%%%%%%%%%%%%%%%%%%%%%%%%%%%%%%%%
%%%%%%%%%%%%%%%%%%%%%%%%%%%%%%%%%%%%%%%%%%%%%%%%%%%%%%%%%%%%%%%%%%%%%%%%%%%%%%%%
\section{Information}

%%%%%%%%%%%%%%%%%%%%%%%%%%%%%%%%%%%%%%%%%%%%%%%%%%%%%%%%%%%%%%%%%%%%%%%%%%%%%%%%
\subsection{Copyright}

Copyright \copyright{} 2017--2018 Niklas Beisert

This work may be distributed and/or modified under the
conditions of the \LaTeX{} Project Public License, either version 1.3
of this license or (at your option) any later version.
The latest version of this license is in
  \url{http://www.latex-project.org/lppl.txt}
and version 1.3 or later is part of all distributions of \LaTeX{}
version 2005/12/01 or later.

This work has the LPPL maintenance status `maintained'.

The Current Maintainer of this work is Niklas Beisert.

This work consists of the files |README.txt|, |childdoc.ins| and |childdoc.dtx|
as well as the derived files |childdoc.def|, |cdocsamp.tex|
with |cdocsch1.tex|, |cdocsch2.tex|, |cdocspt3.tex|, |cdocspt4.tex|,
|cdocsdrf.tex|, |cdocsfn1.tex|, |cdocsfn2.tex|
as well as |childdoc.pdf|.

%%%%%%%%%%%%%%%%%%%%%%%%%%%%%%%%%%%%%%%%%%%%%%%%%%%%%%%%%%%%%%%%%%%%%%%%%%%%%%%%
\subsection{Files and Installation}

The package consists of the files:
%
\begin{center}
\begin{tabular}{ll}
    |README.txt|   & readme file \\
    |childdoc.ins| & installation file \\
    |childdoc.dtx| & source file \\
    |childdoc.def| & definition file \\
    |cdocsamp.tex| & sample main file \\
    |cdocsch1.tex| & sample include file \\
    |cdocsch2.tex| & sample include file \\
    |cdocspt3.tex| & sample part file \\
    |cdocspt4.tex| & sample part file \\
    |cdocsdrf.tex| & sample redirection file \\
    |cdocsfn1.tex| & sample redirection file \\
    |cdocsfn2.tex| & sample redirection file \\
    |childdoc.pdf| & manual
\end{tabular}
\end{center}
%
The distribution consists of the files
|README.txt|, |childdoc.ins| and |childdoc.dtx|.
%
\begin{itemize}
\item
Run (pdf)\LaTeX{} on |childdoc.dtx|
to compile the manual |childdoc.pdf| (this file).
\item
Run \LaTeX{} on |childdoc.ins| to create the definitions file |childdoc.def|
and the sample |cdocsamp.tex| with include files
|cdocsch1.tex|, |cdocsch2.tex|, |cdocspt3.tex|, |cdocspt4.tex|,
|cdocsdrf.tex|, |cdocsfn1.tex|, |cdocsfn2.tex|.
Then copy the file |childdoc.def| to an appropriate directory of your \LaTeX{}
distribution, e.g.\ \textit{texmf-root}|/tex/latex/childdoc|.
\end{itemize}

%%%%%%%%%%%%%%%%%%%%%%%%%%%%%%%%%%%%%%%%%%%%%%%%%%%%%%%%%%%%%%%%%%%%%%%%%%%%%%%%
\subsection{Related CTAN Packages}

There are several other packages which offer a similar functionality:
%
\begin{itemize}
\item
The packages
\href{http://ctan.org/pkg/docmute}{\textsf{docmute}},
\href{http://ctan.org/pkg/includex}{\textsf{includex}} and
\href{http://ctan.org/pkg/standalone}{\textsf{standalone}}
provide commands to include only the document body of
a child file thus allowing both files to be compiled individually.
\item
The packages \href{http://ctan.org/pkg/subdocs}{\textsf{subdocs}}
and \href{http://ctan.org/pkg/subfiles}{\textsf{subfiles}}
provide structures in which the main and child documents can be
encapsulated and allowing them to be compiled individually.
The inclusion mechanism is different from the conventional |\include|.
\item
The package \href{http://ctan.org/pkg/combine}{\textsf{combine}}
is an elaborate solution to combine several documents into one.
\end{itemize}
%
See also the CTAN topic \href{http://ctan.org/topic/subdocs}{\textsf{subdocs}}
for further related packages.
The present package differs from the above solutions in that
a document structure constructed with the conventional |\include| mechanism
just needs two extra commands at the top of every file
such that all constituent files can be compiled individually.

%%%%%%%%%%%%%%%%%%%%%%%%%%%%%%%%%%%%%%%%%%%%%%%%%%%%%%%%%%%%%%%%%%%%%%%%%%%%%%%%
%\subsection{Feature Suggestions}
%
%The following is a list of features which may be useful for future
%versions of this package:
%%
%\begin{itemize}
%\item
%\ldots
%\end{itemize}

%%%%%%%%%%%%%%%%%%%%%%%%%%%%%%%%%%%%%%%%%%%%%%%%%%%%%%%%%%%%%%%%%%%%%%%%%%%%%%%%
\subsection{Revision History}

%%%%%%%%%%%%%%%%%%%%%%%%%%%%%%%%%%%%%%%%
\paragraph{v2.0:} 2018/12/30

\begin{itemize}
\item
immediate forward processing
\item
added |\childdocby| mechanism
\item
manual restructured
\end{itemize}

%%%%%%%%%%%%%%%%%%%%%%%%%%%%%%%%%%%%%%%%
\paragraph{v1.6:} 2018/01/17

\begin{itemize}
\item
application for development of include files
\item
corrections to manual
\end{itemize}

%%%%%%%%%%%%%%%%%%%%%%%%%%%%%%%%%%%%%%%%
\paragraph{v1.5:} 2017/05/21

\begin{itemize}
\item
more complete structuring introduced
\item
|\childdocof| introduced
\item
|\childdoc| renamed to |\childdocmain|
\item
|\childredirect| renamed to |\childdocforward| and |\childdocforwardprefix|
and functionality expanded
\end{itemize}

%%%%%%%%%%%%%%%%%%%%%%%%%%%%%%%%%%%%%%%%
\paragraph{v1.0:} 2017/04/27

\begin{itemize}
\item
manual and install package
\item
first version published on CTAN
\end{itemize}

%%%%%%%%%%%%%%%%%%%%%%%%%%%%%%%%%%%%%%%%
\paragraph{v0.6:} 2017/04/26

\begin{itemize}
\item
redirection mechanism added
\end{itemize}

%%%%%%%%%%%%%%%%%%%%%%%%%%%%%%%%%%%%%%%%
\paragraph{v0.5:} 2017/04/26

\begin{itemize}
\item
functionality in definition file
\end{itemize}


%%%%%%%%%%%%%%%%%%%%%%%%%%%%%%%%%%%%%%%%%%%%%%%%%%%%%%%%%%%%%%%%%%%%%%%%%%%%%%%%
%%%%%%%%%%%%%%%%%%%%%%%%%%%%%%%%%%%%%%%%%%%%%%%%%%%%%%%%%%%%%%%%%%%%%%%%%%%%%%%%
%%%%%%%%%%%%%%%%%%%%%%%%%%%%%%%%%%%%%%%%%%%%%%%%%%%%%%%%%%%%%%%%%%%%%%%%%%%%%%%%
\appendix

\settowidth\MacroIndent{\rmfamily\scriptsize 000\ }

 \DocInput{childdoc.dtx}

\end{document}
%</driver>
% \fi
%
% %%%%%%%%%%%%%%%%%%%%%%%%%%%%%%%%%%%%%%%%%%%%%%%%%%%%%%%%%%%%%%%%%%%%%%%%%%%%%%
% %%%%%%%%%%%%%%%%%%%%%%%%%%%%%%%%%%%%%%%%%%%%%%%%%%%%%%%%%%%%%%%%%%%%%%%%%%%%%%
% \section{Sample}
%\iffalse
%<*samplemain>
%\fi
%
% The following presents a sample document
% with two chapters, two parts, a title page,
% a compile flag as well as three forwarding files to set the flag.
% It consists of eight |.tex| files:
% \begin{center}
% \begin{tabular}{ll}
% |cdocsamp.tex|&main file\\
% |cdocsch1.tex|&include file for chapter 1\\
% |cdocsch2.tex|&include file for chapter 2\\
% |cdocspt3.tex|&include file for part 3\\
% |cdocspt4.tex|&include file for part 4\\
% |cdocsdrf.tex|&forwarding file for main file in draft mode\\
% |cdocsfi1.tex|&forwarding file for final version of chapter 1\\
% |cdocsfi2.tex|&forwarding file for final version of chapter 2\\
% \end{tabular}
% \end{center}
% Each of the eight files can be compiled directly by the \LaTeX{} compiler.
%
% %%%%%%%%%%%%%%%%%%%%%%%%%%%%%%%%%%%%%%
% \paragraph{Main File.}
%
% The main file is called |cdocsamp.tex|.
%
% Load the \textsf{childdoc} definitions and
% declare the filename for the main document:
%    \begin{macrocode}
\input{childdoc.def}
\childdocmain{}
%    \end{macrocode}

% Optional override for |\version| flag:
%    \begin{macrocode}
%%\ifchilddoc\else\providecommand{\version}{draft}\fi
%    \end{macrocode}

% Define the default values for the |\version| flag
% (|final| for the main file and |draft| for childs):
%    \begin{macrocode}
\ifchilddoc
\providecommand{\version}{draft}
\else
\providecommand{\version}{final}
\fi
%    \end{macrocode}

% Load the standard document class:
%    \begin{macrocode}
\documentclass[12pt]{article}
%    \end{macrocode}

% Start the document body:
%    \begin{macrocode}
\begin{document}
%    \end{macrocode}

% Declare a title page.
% Print title, part of document being processed and version flag:
%    \begin{macrocode}
\addtocounter{page}{-1}
\begin{center}
{\LARGE\bfseries{}childdoc example\par}
\vspace{1cm}
\ifchilddoc
\ifchilddocmanual part\else chapter\fi:
`\childdocname' of `\childdocjob'\par
\else
main document: `\childdocjob'\par
\fi
version: \version\par
\end{center}
\newpage
%    \end{macrocode}

% Manually include selected file,
% otherwise process as usual:
%    \begin{macrocode}
\ifchilddocmanual
\section*{part `\childdocname'}
\input{\childdocname}
\else
%    \end{macrocode}

% Include the two chapters:
%    \begin{macrocode}
\include{cdocsch1}
\include{cdocsch2}
%    \end{macrocode}

% Include the two parts unless only chapters should be displayed:
%    \begin{macrocode}
\ifchilddoc\else
\section{part three}
\input{cdocspt3}
\section{part four}
\input{cdocspt4}
\fi
%    \end{macrocode}

% Process as usual until here:
%    \begin{macrocode}
\fi
%    \end{macrocode}

% End of document body:
%    \begin{macrocode}
\end{document}
%    \end{macrocode}
%\iffalse
%</samplemain>
%\fi
%
% %%%%%%%%%%%%%%%%%%%%%%%%%%%%%%%%%%%%%%
% \paragraph{Chapter Include Files.}
%
% The include files are called |cdocsch1.tex| and |cdocsch2.tex|.
%
%\iffalse
%<*samplechap1|samplechap2>
%\fi

% Optional override for |\version| flag:
%    \begin{macrocode}
%%\providecommand{\version}{final}
%    \end{macrocode}

% Include the main document:
%    \begin{macrocode}
\input{childdoc.def}
\childdocof{cdocsamp}
%    \end{macrocode}

%\iffalse
%</samplechap1|samplechap2>
%\fi
%
%\iffalse
%<*samplechap1>
%\fi
% Some text for chapter 1:
%    \begin{macrocode}
\section{one}
some text in chapter one
%    \end{macrocode}

%\iffalse
%</samplechap1>
%\fi
% Some text for chapter 2:
%\iffalse
%<*samplechap2>
%\fi
%    \begin{macrocode}
\section{two}
more text in chapter two
%    \end{macrocode}

%\iffalse
%</samplechap2>
%\fi
%
% %%%%%%%%%%%%%%%%%%%%%%%%%%%%%%%%%%%%%%
% \paragraph{Part Include Files.}
%
% The include files are called |cdocspt3.tex| and |cdocspt4.tex|.
%
%\iffalse
%<*samplepart3|samplepart4>
%\fi

% Optional override for |\version| flag:
%    \begin{macrocode}
%%\providecommand{\version}{final}
%    \end{macrocode}

% Include the main document:
%    \begin{macrocode}
\input{childdoc.def}
\childdocby{cdocsamp}
%    \end{macrocode}

%\iffalse
%</samplepart3|samplepart4>
%\fi
%
%\iffalse
%<*samplepart3>
%\fi
% Some text for part 3:
%    \begin{macrocode}
some text in part three
%    \end{macrocode}

%\iffalse
%</samplepart3>
%\fi
% Some text for part 4:
%\iffalse
%<*samplepart4>
%\fi
%    \begin{macrocode}
more text in part four
%    \end{macrocode}

%\iffalse
%</samplepart4>
%\fi
%
% %%%%%%%%%%%%%%%%%%%%%%%%%%%%%%%%%%%%%%
% \paragraph{Forwarding for a Complete Draft.}
%
% The following forwarding file |cdocsdrf.tex|
% compiles the main document in draft mode:
%\iffalse
%<*sampledraft>
%\fi
%    \begin{macrocode}
\def\version{draft}
\input{childdoc.def}
\childdocforward{cdocsamp}
%    \end{macrocode}

%\iffalse
%</sampledraft>
%\fi
%
% %%%%%%%%%%%%%%%%%%%%%%%%%%%%%%%%%%%%%%
% \paragraph{Forwarding for Final Version of the Chapters.}
%
% The following forwarding files |cdocsfn1.tex| and |cdocsfn2.tex|
% (with identical content)
% compile the final versions of the child documents
% |cdocsch1.tex| and |cdocsch2.tex|, respectively:
%\iffalse
%<*samplefinal>
%\fi
%    \begin{macrocode}
\def\version{final}
\input{childdoc.def}
\childdocforwardprefix[cdocsamp]{cdocsfn}{cdocsch}
%    \end{macrocode}

%\iffalse
%</samplefinal>
%\fi
%
% %%%%%%%%%%%%%%%%%%%%%%%%%%%%%%%%%%%%%%
% \paragraph{Command Line Processing.}
%
% The following three command lines generate the output files
% |cdocscld|, |cdocscl1| and |cdocscl2|
% which should be identical to
% |cdocsdrf|, |cdocsch1| and |cdocsfn2|, respectively:
% \begin{center}
% \begin{tabular}{l}
% |latex -jobname cdocscld \|\\
% |  "\def\version{draft}\input{childdoc.def}\childdocforward{cdocsamp}"|\\
% |latex -jobname cdocscl1 \|\\
% |  "\input{childdoc.def}\childdocforward[cdocsamp]{cdocsch1}"|\\
% |latex -jobname cdocscl2 \|\\
% |  "\def\version{final}\input{childdoc.def}\childdocforward{cdocsch2}"|
% \end{tabular}
% \end{center}
% Note that the trailing backslash on each first line
% merely continues the input to the second line
% (for convenient cut ant paste).
% Furthermore, the command |latex| can be replaced by any
% of its alternative versions such as |pdflatex|.
%
% %%%%%%%%%%%%%%%%%%%%%%%%%%%%%%%%%%%%%%%%%%%%%%%%%%%%%%%%%%%%%%%%%%%%%%%%%%%%%%
% %%%%%%%%%%%%%%%%%%%%%%%%%%%%%%%%%%%%%%%%%%%%%%%%%%%%%%%%%%%%%%%%%%%%%%%%%%%%%%
% \section{Implementation}
%\iffalse
%<*package>
%\fi
%
% This section describes the definitions file |childdoc.def|.

% The definitions cannot be loaded using |\usepackage| or |\RequirePackage|
% which has a mechanism to prevent loading a style file more than once.
% When loading the definitions by means of |\input|
% multiple instances have to be prevented manually:
%\iffalse
%This code needs to be before the `\ProvidesFile' directive
%which is defined at the beginning of this file.
%Therefore it is also placed there and commented out here.
%</package>
%<*discard>
%\fi
%    \begin{macrocode}
\ifdefined\childdocmain\endinput\fi
%    \end{macrocode}
%\iffalse
%</discard>
%<*package>
%\fi
%
% \macro{\ifchilddoc}
% \macro{\ifchilddocmanual}
% The conditional |\ifchilddoc| tells whether a
% child (true) or main (false) document is being compiled.
% The conditional |\ifchilddocmanual| tells whether
% the |\includeonly| mechanism is used (false) or
% the selection of child files must be performed manually (true).
% The definitions initialise to false:
%    \begin{macrocode}
\newif\ifchilddoc
\newif\ifchilddocmanual
%    \end{macrocode}

% \macro{\childdocname}
% \macro{\childdocjob}
% The macro |\childdocname| stores the name of the main document
% to be compiled. The macro |\childdocjob| stores the name of
% the document on which the \LaTeX{} compiler was originally invoked.
% The content of |\jobname| cannot be compared
% to filenames specified in the source due to different catcodes.
% The following code rescans |\jobname|, stores the result
% in |\childdocname| and saves a copy in |\childdocjob|:
%    \begin{macrocode}
\edef\childdocname{\scantokens\expandafter{\jobname\noexpand}}
\let\childdocjob\childdocname
%    \end{macrocode}

% \macro{\childdocdisable}
% The macro |\childdocdisable| prevents the main file
% from being processed more than once.
% At this stage, the main document command |\childdocmain|
% is assumed to be called once again where it should do nothing.
% Any subsequent call to it should prevent
% a secondary processing of the main document
% It overwrites the forwarding commands
% |\childdocof| and |\childdocforward|
% with empty macros to prevent further inclusions of the main document:
%    \begin{macrocode}
\newcommand{\childdocdisable}
{
  \renewcommand{\childdocmain}[1]{\renewcommand{\childdocmain}[1]{\endinput}}
  \renewcommand{\childdocof}[1]{}
  \renewcommand{\childdocby}[2][]{}
  \renewcommand{\childdocforward}[2][]{}
  \renewcommand{\childdocdisable}{}
}
%    \end{macrocode}

% \macro{\childdocmain}
% The macro |\childdocmain| is to be called at the top of the main file
% with nothing or the main filename (without extension) as argument.
% First, it breaks loops.
% If the argument is not empty and does not match |\childdocname|
% (which is set by the first inclusion of |childdoc.def|),
% |\ifchilddoc| is set to true, |\includeonly| is applied to the child file
% and |\jobname| is set to the main file
% (for proper handling of |.aux| files):
%    \begin{macrocode}
\newcommand{\childdocmain}[1]
{
  \childdocdisable\childdocmain{}
  \if?#1?\else
    \begingroup
      \def\childdoctmp{#1}
      \ifx\childdoctmp\childdocname
        \def\childdoctmp{}
      \else
        \def\childdoctmp
        {
          \childdoctrue
          \includeonly{\childdocname}
          \def\childdocjob{#1}
          \def\jobname{#1}
        }
      \fi
      \expandafter
    \endgroup
    \childdoctmp
  \fi
}
%    \end{macrocode}

% \macro{\childdocof}
% The command |\childdocof| redirects
% compilation to the main file |#1|.
%    \begin{macrocode}
\newcommand{\childdocof}[1]
{
  \childdocdisable
  \childdoctrue
  \includeonly{\childdocname}
  \def\jobname{#1}
  \def\childdocjob{#1}
  \input{#1}
}
%    \end{macrocode}

% \macro{\childdocby}
% The command |\childdocby| ....
%    \begin{macrocode}
\newcommand{\childdocby}[2][]
{
  \childdocdisable
  \childdoctrue
  \childdocmanualtrue
  \if?#1?\else
    \def\jobname{#2}
  \fi
  \def\childdocjob{#2}
  \input{#2}
  \endinput
}
%    \end{macrocode}

% \macro{\childdocforward}
% The command |\childdocforward| redirects
% compilation to the main file or
% (if the optional argument is given) a child file.
% Parameters are set as if the main file
% or a child file starting with |\childdocof| was compiled.
% Then compilation is handed over to the main file:
%    \begin{macrocode}
\newcommand{\childdocforward}[2][]
{
  \begingroup
    \if?#1?
      \def\childdoctmp
      {
        \def\childdocname{#2}
        \def\childdocjob{#2}
        \def\jobname{#2}
        \input{#2}
        \endinput
      }
    \else
      \def\childdoctmp
      {
        \childdocdisable
        \def\childdocname{#2}
        \childdoctrue
        \includeonly{#2}
        \def\childdocjob{#1}
        \def\jobname{#1}
        \input{#1}
        \endinput
      }
    \fi
    \expandafter
  \endgroup
  \childdoctmp
}
%    \end{macrocode}

% \macro{\childdocforwardprefix}
% The command |\childdocforwardprefix| redirects
% compilation to the main or a child file by means of a pattern.
% The prefix |#1| in the current filename is replaced by |#2|
% and the suffix of the current filename is kept
% (it is assumed that the filename does not contain the substring `|~~~|'
% which is used as a delimiter).
% Compilation is handed over to the new file by |\childdocforward|:
%    \begin{macrocode}
\newcommand{\childdocforwardprefix}[3][]
{
  \begingroup
    \def\childdocextract #2##1~~~{\def\childdoctmp{\childdocforward[#1]{#3##1}}}
    \expandafter\childdocextract\childdocname~~~
    \expandafter
  \endgroup
  \childdoctmp
}
%    \end{macrocode}

% \macro{\childdoc}
% The deprecated macro |\childdoc| is a legacy version of |\childdocmain|:
%    \begin{macrocode}
\newcommand{\childdoc}{\childdocmain}
%    \end{macrocode}

% \macro{\childdocredirect}
% The deprecated macro |\childdocredirect| is a legacy version
% of |\childdocforward| and |\childdocforwardprefix|:
%    \begin{macrocode}
\newcommand{\childdocredirect}[2][]
{
  \begingroup
    \if?#1?
      \def\childdoctmp{\childdocforward{#2}}
    \else
      \def\childdoctmp{\childdocforwardprefix{#1}{#2}}
    \fi
    \expandafter
  \endgroup
  \childdoctmp
}
%    \end{macrocode}

%\iffalse
%</package>
%\fi
%
\endinput
\childdocforward{cdocsch2}"|
% \end{tabular}
% \end{center}
% Note that the trailing backslash on each first line
% merely continues the input to the second line
% (for convenient cut ant paste).
% Furthermore, the command |latex| can be replaced by any
% of its alternative versions such as |pdflatex|.
%
% %%%%%%%%%%%%%%%%%%%%%%%%%%%%%%%%%%%%%%%%%%%%%%%%%%%%%%%%%%%%%%%%%%%%%%%%%%%%%%
% %%%%%%%%%%%%%%%%%%%%%%%%%%%%%%%%%%%%%%%%%%%%%%%%%%%%%%%%%%%%%%%%%%%%%%%%%%%%%%
% \section{Implementation}
%\iffalse
%<*package>
%\fi
%
% This section describes the definitions file |childdoc.def|.

% The definitions cannot be loaded using |\usepackage| or |\RequirePackage|
% which has a mechanism to prevent loading a style file more than once.
% When loading the definitions by means of |\input|
% multiple instances have to be prevented manually:
%\iffalse
%This code needs to be before the `\ProvidesFile' directive
%which is defined at the beginning of this file.
%Therefore it is also placed there and commented out here.
%</package>
%<*discard>
%\fi
%    \begin{macrocode}
\ifdefined\childdocmain\endinput\fi
%    \end{macrocode}
%\iffalse
%</discard>
%<*package>
%\fi
%
% \macro{\ifchilddoc}
% \macro{\ifchilddocmanual}
% The conditional |\ifchilddoc| tells whether a
% child (true) or main (false) document is being compiled.
% The conditional |\ifchilddocmanual| tells whether
% the |\includeonly| mechanism is used (false) or
% the selection of child files must be performed manually (true).
% The definitions initialise to false:
%    \begin{macrocode}
\newif\ifchilddoc
\newif\ifchilddocmanual
%    \end{macrocode}

% \macro{\childdocname}
% \macro{\childdocjob}
% The macro |\childdocname| stores the name of the main document
% to be compiled. The macro |\childdocjob| stores the name of
% the document on which the \LaTeX{} compiler was originally invoked.
% The content of |\jobname| cannot be compared
% to filenames specified in the source due to different catcodes.
% The following code rescans |\jobname|, stores the result
% in |\childdocname| and saves a copy in |\childdocjob|:
%    \begin{macrocode}
\edef\childdocname{\scantokens\expandafter{\jobname\noexpand}}
\let\childdocjob\childdocname
%    \end{macrocode}

% \macro{\childdocdisable}
% The macro |\childdocdisable| prevents the main file
% from being processed more than once.
% At this stage, the main document command |\childdocmain|
% is assumed to be called once again where it should do nothing.
% Any subsequent call to it should prevent
% a secondary processing of the main document
% It overwrites the forwarding commands
% |\childdocof| and |\childdocforward|
% with empty macros to prevent further inclusions of the main document:
%    \begin{macrocode}
\newcommand{\childdocdisable}
{
  \renewcommand{\childdocmain}[1]{\renewcommand{\childdocmain}[1]{\endinput}}
  \renewcommand{\childdocof}[1]{}
  \renewcommand{\childdocby}[2][]{}
  \renewcommand{\childdocforward}[2][]{}
  \renewcommand{\childdocdisable}{}
}
%    \end{macrocode}

% \macro{\childdocmain}
% The macro |\childdocmain| is to be called at the top of the main file
% with nothing or the main filename (without extension) as argument.
% First, it breaks loops.
% If the argument is not empty and does not match |\childdocname|
% (which is set by the first inclusion of |childdoc.def|),
% |\ifchilddoc| is set to true, |\includeonly| is applied to the child file
% and |\jobname| is set to the main file
% (for proper handling of |.aux| files):
%    \begin{macrocode}
\newcommand{\childdocmain}[1]
{
  \childdocdisable\childdocmain{}
  \if?#1?\else
    \begingroup
      \def\childdoctmp{#1}
      \ifx\childdoctmp\childdocname
        \def\childdoctmp{}
      \else
        \def\childdoctmp
        {
          \childdoctrue
          \includeonly{\childdocname}
          \def\childdocjob{#1}
          \def\jobname{#1}
        }
      \fi
      \expandafter
    \endgroup
    \childdoctmp
  \fi
}
%    \end{macrocode}

% \macro{\childdocof}
% The command |\childdocof| redirects
% compilation to the main file |#1|.
%    \begin{macrocode}
\newcommand{\childdocof}[1]
{
  \childdocdisable
  \childdoctrue
  \includeonly{\childdocname}
  \def\jobname{#1}
  \def\childdocjob{#1}
  \input{#1}
}
%    \end{macrocode}

% \macro{\childdocby}
% The command |\childdocby| ....
%    \begin{macrocode}
\newcommand{\childdocby}[2][]
{
  \childdocdisable
  \childdoctrue
  \childdocmanualtrue
  \if?#1?\else
    \def\jobname{#2}
  \fi
  \def\childdocjob{#2}
  \input{#2}
  \endinput
}
%    \end{macrocode}

% \macro{\childdocforward}
% The command |\childdocforward| redirects
% compilation to the main file or
% (if the optional argument is given) a child file.
% Parameters are set as if the main file
% or a child file starting with |\childdocof| was compiled.
% Then compilation is handed over to the main file:
%    \begin{macrocode}
\newcommand{\childdocforward}[2][]
{
  \begingroup
    \if?#1?
      \def\childdoctmp
      {
        \def\childdocname{#2}
        \def\childdocjob{#2}
        \def\jobname{#2}
        \input{#2}
        \endinput
      }
    \else
      \def\childdoctmp
      {
        \childdocdisable
        \def\childdocname{#2}
        \childdoctrue
        \includeonly{#2}
        \def\childdocjob{#1}
        \def\jobname{#1}
        \input{#1}
        \endinput
      }
    \fi
    \expandafter
  \endgroup
  \childdoctmp
}
%    \end{macrocode}

% \macro{\childdocforwardprefix}
% The command |\childdocforwardprefix| redirects
% compilation to the main or a child file by means of a pattern.
% The prefix |#1| in the current filename is replaced by |#2|
% and the suffix of the current filename is kept
% (it is assumed that the filename does not contain the substring `|~~~|'
% which is used as a delimiter).
% Compilation is handed over to the new file by |\childdocforward|:
%    \begin{macrocode}
\newcommand{\childdocforwardprefix}[3][]
{
  \begingroup
    \def\childdocextract #2##1~~~{\def\childdoctmp{\childdocforward[#1]{#3##1}}}
    \expandafter\childdocextract\childdocname~~~
    \expandafter
  \endgroup
  \childdoctmp
}
%    \end{macrocode}

% \macro{\childdoc}
% The deprecated macro |\childdoc| is a legacy version of |\childdocmain|:
%    \begin{macrocode}
\newcommand{\childdoc}{\childdocmain}
%    \end{macrocode}

% \macro{\childdocredirect}
% The deprecated macro |\childdocredirect| is a legacy version
% of |\childdocforward| and |\childdocforwardprefix|:
%    \begin{macrocode}
\newcommand{\childdocredirect}[2][]
{
  \begingroup
    \if?#1?
      \def\childdoctmp{\childdocforward{#2}}
    \else
      \def\childdoctmp{\childdocforwardprefix{#1}{#2}}
    \fi
    \expandafter
  \endgroup
  \childdoctmp
}
%    \end{macrocode}

%\iffalse
%</package>
%\fi
%
\endinput
|\\
|\childdocforward[|\textit{main}|]{|\textit{dest}|}|\\
\end{tabular}
\end{center}
%
The argument \textit{dest} is the destination file
(without extension).
It should be the main file or one of the child files.
Note that further \textsf{childdoc} directives
such as |\childdocof| and |\childdocforward|
in the indicated file will be processed in this form.
The optional argument \textit{main}
passes on directly to the main file \textit{main}
while pretending to compile the child \textit{dest}.
This form behaves as if \textit{dest}
issues |\childdocof{|\textit{main}|}| right away,
and no further \textsf{childdoc} directives will be processed.

%%%%%%%%%%%%%%%%%%%%%%%%%%%%%%%%%%%%%%%%
\DescribeMacro{\...prefix}
In the alternative form |\childdocforwardprefix|,
%
\begin{center}
\begin{tabular}{l}
|% \iffalse
%
% childdoc.dtx Copyright (C) 2017-2018 Niklas Beisert
%
% This work may be distributed and/or modified under the
% conditions of the LaTeX Project Public License, either version 1.3
% of this license or (at your option) any later version.
% The latest version of this license is in
%   http://www.latex-project.org/lppl.txt
% and version 1.3 or later is part of all distributions of LaTeX
% version 2005/12/01 or later.
%
% This work has the LPPL maintenance status `maintained'.
%
% The Current Maintainer of this work is Niklas Beisert.
%
% This work consists of the files childdoc.dtx and childdoc.ins
% and the derived files childdoc.def and cdocsamp.tex with
% cdocsch1.tex, cdocsch2.tex, cdocsdrf.tex, cdocsfn1.tex, cdocsfn2.tex.
%
%<package>\ifdefined\childdocmain\endinput\fi
%<package>\ProvidesFile{childdoc.def}[2018/12/30 v2.0 child document driver]
%<samplemain>\ProvidesFile{cdocsamp.tex}[2018/12/30 v2.0 sample for childdoc]
%<*driver>
%\ProvidesFile{childdoc.drv}[2018/12/30 v2.0 childdoc reference manual file]
\PassOptionsToClass{10pt,a4paper}{article}
\documentclass{ltxdoc}

\usepackage[margin=35mm]{geometry}
\usepackage{hyperref}
\usepackage{hyperxmp}
\usepackage[usenames]{color}

\hypersetup{colorlinks=true}
\hypersetup{pdfstartview=FitH}
\hypersetup{pdfpagemode=UseNone}
\hypersetup{pdfsource={}}
\hypersetup{pdflang={en-UK}}
\hypersetup{pdfcopyright={Copyright 2017-2018 Niklas Beisert.
  This work may be distributed and/or modified under the
  conditions of the LaTeX Project Public License, either version 1.3
  of this license or (at your option) any later version.}}
\hypersetup{pdflicenseurl={http://www.latex-project.org/lppl.txt}}
\hypersetup{pdfcontactaddress={ETH Zurich, ITP, HIT K,
  Wolfgang-Pauli-Strasse 27}}
\hypersetup{pdfcontactpostcode={8093}}
\hypersetup{pdfcontactcity={Zurich}}
\hypersetup{pdfcontactcountry={Switzerland}}
\hypersetup{pdfcontactemail={nbeisert@itp.phys.ethz.ch}}
\hypersetup{pdfcontacturl={http://people.phys.ethz.ch/\xmptilde nbeisert/}}

\newcommand{\secref}[1]{\hyperref[#1]{section \ref*{#1}}}

\parskip1ex
\parindent0pt
\let\olditemize\itemize
\def\itemize{\olditemize\parskip0pt}

\begin{document}

\title{The \textsf{childdoc} Package}
\hypersetup{pdftitle={The childdoc Package}}
\author{Niklas Beisert\\[2ex]
  Institut f\"ur Theoretische Physik\\
  Eidgen\"ossische Technische Hochschule Z\"urich\\
  Wolfgang-Pauli-Strasse 27, 8093 Z\"urich, Switzerland\\[1ex]
  \href{mailto:nbeisert@itp.phys.ethz.ch}
  {\texttt{nbeisert@itp.phys.ethz.ch}}}
\hypersetup{pdfauthor={Niklas Beisert}}
\hypersetup{pdfsubject={Manual for the LaTeX2e Package childdoc}}
\date{30 December 2018, \textsf{v2.0}}
\maketitle

\begin{abstract}\noindent
\textsf{childdoc} is a \LaTeXe{} package
that enables the direct compilation
of document sections included by |\include|
to individual files.
\end{abstract}

\begingroup
\parskip0ex
\tableofcontents
\endgroup

%%%%%%%%%%%%%%%%%%%%%%%%%%%%%%%%%%%%%%%%%%%%%%%%%%%%%%%%%%%%%%%%%%%%%%%%%%%%%%%%
%%%%%%%%%%%%%%%%%%%%%%%%%%%%%%%%%%%%%%%%%%%%%%%%%%%%%%%%%%%%%%%%%%%%%%%%%%%%%%%%
\section{Introduction}

\LaTeX{} provides a mechanism to structure a large document (such as a book)
into a main file and several child files (containing the chapters)
using the |\include| command.
This mechanism is beneficial for documents
which span hundreds of pages in order to
make the source file(s) more manageable.
Moreover, compilation can be restricted to
selected child files by means of the |\includeonly| command.
The latter feature can be used to reduce the compilation time while editing
(this was significantly more useful in the earlier days of \LaTeX{})
or to generate a smaller document which is easier to navigate.
Another application of |\includeonly| is to generate
documents consisting of selected parts of the complete document.

However, there are a few drawbacks of the plain |\include| mechanism:
\begin{itemize}
\item
The child files cannot be compiled on their own,
they can only be compiled via the main file.
A naive editing environment
(such as a text editor with an option
to have the current file processed by \LaTeX)
may require one to switch to the main file before compiling;
attempting to compile the child file produces errors.
\item
The main file must be modified (each time)
to adjust the |\includeonly| command
to the present needs. This easily leaves the main file in a messy state.
\item
The generated document will always carry the filename
of the main document. This is inconvenient if
several child files are to be compiled and
to be kept for distribution.
\end{itemize}

The present package provides a simple interface
to make child files individually compilable by \LaTeX{}.
Compiling a child file then has the same effect as compiling
the main file with an |\includeonly| command
to select the appropriate child.
Moreover the generated document will carry the name of the child
rather than the main file.
This resolves all three above issues.

This feature is meant to make the editing of books,
thesis documents and lecture notes somewhat more convenient.
However, the package can also be used efficiently for
composing a series of documents (such as exercise sheets)
which are typically distributed individually.
It then assists the author in generating the individual documents
(potentially in different versions)
as well as a document containing the collected series.
Another application is in developing style files
or other kinds of included material
where compilation of the style file could redirect
to a sample or test file.

%%%%%%%%%%%%%%%%%%%%%%%%%%%%%%%%%%%%%%%%%%%%%%%%%%%%%%%%%%%%%%%%%%%%%%%%%%%%%%%%
%%%%%%%%%%%%%%%%%%%%%%%%%%%%%%%%%%%%%%%%%%%%%%%%%%%%%%%%%%%%%%%%%%%%%%%%%%%%%%%%
\section{Usage}

First of all, the package \textsf{childdoc} is \emph{not} a standard
\LaTeXe{} |.sty| style file! Therefore it needs to be invoked in
a non-standard way.

%%%%%%%%%%%%%%%%%%%%%%%%%%%%%%%%%%%%%%%%%%%%%%%%%%%%%%%%%%%%%%%%%%%%%%%%%%%%%%%%
\subsection{Included Files}
\label{sec:include}

%%%%%%%%%%%%%%%%%%%%%%%%%%%%%%%%%%%%%%%%
\DescribeMacro{\childdocmain}
To use the package, add the commands
\begin{center}
\begin{tabular}{l}
|% \iffalse
%
% childdoc.dtx Copyright (C) 2017-2018 Niklas Beisert
%
% This work may be distributed and/or modified under the
% conditions of the LaTeX Project Public License, either version 1.3
% of this license or (at your option) any later version.
% The latest version of this license is in
%   http://www.latex-project.org/lppl.txt
% and version 1.3 or later is part of all distributions of LaTeX
% version 2005/12/01 or later.
%
% This work has the LPPL maintenance status `maintained'.
%
% The Current Maintainer of this work is Niklas Beisert.
%
% This work consists of the files childdoc.dtx and childdoc.ins
% and the derived files childdoc.def and cdocsamp.tex with
% cdocsch1.tex, cdocsch2.tex, cdocsdrf.tex, cdocsfn1.tex, cdocsfn2.tex.
%
%<package>\ifdefined\childdocmain\endinput\fi
%<package>\ProvidesFile{childdoc.def}[2018/12/30 v2.0 child document driver]
%<samplemain>\ProvidesFile{cdocsamp.tex}[2018/12/30 v2.0 sample for childdoc]
%<*driver>
%\ProvidesFile{childdoc.drv}[2018/12/30 v2.0 childdoc reference manual file]
\PassOptionsToClass{10pt,a4paper}{article}
\documentclass{ltxdoc}

\usepackage[margin=35mm]{geometry}
\usepackage{hyperref}
\usepackage{hyperxmp}
\usepackage[usenames]{color}

\hypersetup{colorlinks=true}
\hypersetup{pdfstartview=FitH}
\hypersetup{pdfpagemode=UseNone}
\hypersetup{pdfsource={}}
\hypersetup{pdflang={en-UK}}
\hypersetup{pdfcopyright={Copyright 2017-2018 Niklas Beisert.
  This work may be distributed and/or modified under the
  conditions of the LaTeX Project Public License, either version 1.3
  of this license or (at your option) any later version.}}
\hypersetup{pdflicenseurl={http://www.latex-project.org/lppl.txt}}
\hypersetup{pdfcontactaddress={ETH Zurich, ITP, HIT K,
  Wolfgang-Pauli-Strasse 27}}
\hypersetup{pdfcontactpostcode={8093}}
\hypersetup{pdfcontactcity={Zurich}}
\hypersetup{pdfcontactcountry={Switzerland}}
\hypersetup{pdfcontactemail={nbeisert@itp.phys.ethz.ch}}
\hypersetup{pdfcontacturl={http://people.phys.ethz.ch/\xmptilde nbeisert/}}

\newcommand{\secref}[1]{\hyperref[#1]{section \ref*{#1}}}

\parskip1ex
\parindent0pt
\let\olditemize\itemize
\def\itemize{\olditemize\parskip0pt}

\begin{document}

\title{The \textsf{childdoc} Package}
\hypersetup{pdftitle={The childdoc Package}}
\author{Niklas Beisert\\[2ex]
  Institut f\"ur Theoretische Physik\\
  Eidgen\"ossische Technische Hochschule Z\"urich\\
  Wolfgang-Pauli-Strasse 27, 8093 Z\"urich, Switzerland\\[1ex]
  \href{mailto:nbeisert@itp.phys.ethz.ch}
  {\texttt{nbeisert@itp.phys.ethz.ch}}}
\hypersetup{pdfauthor={Niklas Beisert}}
\hypersetup{pdfsubject={Manual for the LaTeX2e Package childdoc}}
\date{30 December 2018, \textsf{v2.0}}
\maketitle

\begin{abstract}\noindent
\textsf{childdoc} is a \LaTeXe{} package
that enables the direct compilation
of document sections included by |\include|
to individual files.
\end{abstract}

\begingroup
\parskip0ex
\tableofcontents
\endgroup

%%%%%%%%%%%%%%%%%%%%%%%%%%%%%%%%%%%%%%%%%%%%%%%%%%%%%%%%%%%%%%%%%%%%%%%%%%%%%%%%
%%%%%%%%%%%%%%%%%%%%%%%%%%%%%%%%%%%%%%%%%%%%%%%%%%%%%%%%%%%%%%%%%%%%%%%%%%%%%%%%
\section{Introduction}

\LaTeX{} provides a mechanism to structure a large document (such as a book)
into a main file and several child files (containing the chapters)
using the |\include| command.
This mechanism is beneficial for documents
which span hundreds of pages in order to
make the source file(s) more manageable.
Moreover, compilation can be restricted to
selected child files by means of the |\includeonly| command.
The latter feature can be used to reduce the compilation time while editing
(this was significantly more useful in the earlier days of \LaTeX{})
or to generate a smaller document which is easier to navigate.
Another application of |\includeonly| is to generate
documents consisting of selected parts of the complete document.

However, there are a few drawbacks of the plain |\include| mechanism:
\begin{itemize}
\item
The child files cannot be compiled on their own,
they can only be compiled via the main file.
A naive editing environment
(such as a text editor with an option
to have the current file processed by \LaTeX)
may require one to switch to the main file before compiling;
attempting to compile the child file produces errors.
\item
The main file must be modified (each time)
to adjust the |\includeonly| command
to the present needs. This easily leaves the main file in a messy state.
\item
The generated document will always carry the filename
of the main document. This is inconvenient if
several child files are to be compiled and
to be kept for distribution.
\end{itemize}

The present package provides a simple interface
to make child files individually compilable by \LaTeX{}.
Compiling a child file then has the same effect as compiling
the main file with an |\includeonly| command
to select the appropriate child.
Moreover the generated document will carry the name of the child
rather than the main file.
This resolves all three above issues.

This feature is meant to make the editing of books,
thesis documents and lecture notes somewhat more convenient.
However, the package can also be used efficiently for
composing a series of documents (such as exercise sheets)
which are typically distributed individually.
It then assists the author in generating the individual documents
(potentially in different versions)
as well as a document containing the collected series.
Another application is in developing style files
or other kinds of included material
where compilation of the style file could redirect
to a sample or test file.

%%%%%%%%%%%%%%%%%%%%%%%%%%%%%%%%%%%%%%%%%%%%%%%%%%%%%%%%%%%%%%%%%%%%%%%%%%%%%%%%
%%%%%%%%%%%%%%%%%%%%%%%%%%%%%%%%%%%%%%%%%%%%%%%%%%%%%%%%%%%%%%%%%%%%%%%%%%%%%%%%
\section{Usage}

First of all, the package \textsf{childdoc} is \emph{not} a standard
\LaTeXe{} |.sty| style file! Therefore it needs to be invoked in
a non-standard way.

%%%%%%%%%%%%%%%%%%%%%%%%%%%%%%%%%%%%%%%%%%%%%%%%%%%%%%%%%%%%%%%%%%%%%%%%%%%%%%%%
\subsection{Included Files}
\label{sec:include}

%%%%%%%%%%%%%%%%%%%%%%%%%%%%%%%%%%%%%%%%
\DescribeMacro{\childdocmain}
To use the package, add the commands
\begin{center}
\begin{tabular}{l}
|\input{childdoc.def}|\\
|\childdocmain{}|\\
\end{tabular}
\end{center}
at the very top of the main \LaTeX{} file,
in particular \emph{before} the |\documentclass| statement!
The argument of |\childdocmain| should be left empty
(but it must be present).

%%%%%%%%%%%%%%%%%%%%%%%%%%%%%%%%%%%%%%%%
\DescribeMacro{\childdocof}
Furthermore, add the commands
\begin{center}
\begin{tabular}{l}
|\input{childdoc.def}|\\
|\childdocof{|\textit{main}|}|\\
\end{tabular}
\end{center}
at the top of every child file \textit{child}
which is included by |\include{|\textit{child}|}|
from within the main file
(or at least for those files to be compiled individually).
The argument \textit{main} must be the filename of the main file.

There are a couple of
considerations in setting up the main and child documents:

%%%%%%%%%%%%%%%%%%%%%%%%%%%%%%%%%%%%%%%%
\paragraph{Restrictions.}

Please note the following restrictions:
\begin{itemize}
\item
|\childdocmain| must be called with one argument \textit{main}
to ensure compatibility with earlier version of the package.
It must either be empty (|\childdocmain{}|)
or precisely match the filename of the main file in which it is specified.
See \secref{sec:detection} for further information.
\item
The filename \textit{main} must be specified without the |.tex| extension.
\item
The filename \textit{main} is case sensitive
(even in case-insensitive file systems)
due to internal string comparison.
\item
The argument \textit{main} should be fully expanded, it cannot be a macro.
\item
Subdirectories and special characters should be avoided in filenames.
\item
The command |\childdocmain{|\textit{main}|}| must be followed by a whitespace.
It should not be followed immediately by another command
or by a comment mark `|%|'.
This is because the \TeX{} parser reads the token immediately following
the argument of |\childdocmain| and puts it
at the beginning of every child section;
however, a white\-space is ignored.
\end{itemize}

%%%%%%%%%%%%%%%%%%%%%%%%%%%%%%%%%%%%%%%%
\paragraph{Content of Main File.}

It is advisable to place all content in the child files included by |\include|.
Any output contained in the main file will appear in all child documents
unless suppressed manually;
it cannot be suppressed automatically by the |\includeonly| directive
and thus should normally be avoided.
A method to include some content in the main file
by means of conditional processing is described in \secref{sec:conditional}.

%%%%%%%%%%%%%%%%%%%%%%%%%%%%%%%%%%%%%%%%
\paragraph{Page Numbering.}

When only a part of the document is compiled,
the appropriate numbering of pages
(as well as other status parameters)
is determined from the |.aux| files.
The latter contain information from previous passes.
However this information needs to propagate through
all intermediate child documents.
Therefore the page numbering in child documents may well
be inconsistent until the complete document is compiled at least once.

A useful (if unconventional) way to always ensure a consistent
page numbering is to restart the numbering in each child document
and denote the pages by `\textit{child}|.|\textit{page}'
where \textit{child} represents the chapter/section number of the child file.
This can be achieved by the command
|\numberwithin{page}{|\textit{child}|}|
of the \textsf{amsmath} package
where \textit{child} can be |chapter| or |section|
depending on the chosen structuring.
Alternatively, one can modify the macro |\thepage| appropriately
and reset the counter |page| at the start of each child file.

%%%%%%%%%%%%%%%%%%%%%%%%%%%%%%%%%%%%%%%%%%%%%%%%%%%%%%%%%%%%%%%%%%%%%%%%%%%%%%%%
\subsection{Conditional Processing}
\label{sec:conditional}

The package provides a mechanism to compile different versions
of a document. To customise the versions further some conditional processing
can come in handy to distinguish which version is being compiled.
The package provides two macros to describe the compilation context:

%%%%%%%%%%%%%%%%%%%%%%%%%%%%%%%%%%%%%%%%
\DescribeMacro{\ifchilddoc}
The conditional |\ifchilddoc| distinguishes between the compilation of
child documents and the main document:
%
\begin{center}
|\ifchilddoc |\textit{child-code}| |[|\||else |\textit{main-code}]| \||fi|
\end{center}

%%%%%%%%%%%%%%%%%%%%%%%%%%%%%%%%%%%%%%%%
\DescribeMacro{\childdocname}
\DescribeMacro{\childdocjob}
The macro |\childdocname| contains the filename (without extension)
of the main or child file being processed.
Note that |\childdocjob| will always contain the name of the main file.

%%%%%%%%%%%%%%%%%%%%%%%%%%%%%%%%%%%%%%%%
\paragraph{Title Page.}

Conditional processing can be used to include a title or banner page
in the main document when proper precautions are taken.
Importantly, the code in the main file should ensure that the page counter
(as well as other status parameters which are stored in the |.aux| files)
takes the same value after the conditional processing.
Otherwise the page numbers may take divergent values
depending on which part is compiled.

For example, a title page could be declared by:
%
\begin{center}
\begin{tabular}{l}
|\ifchilddoc\||else|\\
|\addtocounter{page}{-1}|\\
\textit{code for title page}\\
|\newpage|\\
|\||fi|
\end{tabular}
\end{center}
%
A banner page for the child documents can be generated by:
%
\begin{center}
\begin{tabular}{l}
|\ifchilddoc|\\
|\addtocounter{page}{-1}|\\
\textit{code for banner page}\\
|\newpage|\\
|\||fi|
\end{tabular}
\end{center}
%
Here one could write a message such as:
\begin{center}
|This is the part \childdocname{} of \childdocjob{}.|
\end{center}

%%%%%%%%%%%%%%%%%%%%%%%%%%%%%%%%%%%%%%%%%%%%%%%%%%%%%%%%%%%%%%%%%%%%%%%%%%%%%%%%
\subsection{Flags}
\label{sec:flags}

The package makes it easy to generate different versions
of the main or child documents.
To this end compilation flags can be defined
and assigned different default values.
They will be particularly useful in conjunction
with the forwarding mechanism described in \secref{sec:forward}.

For example, it may be useful to have a flag |\version|
which can be set to |draft| or |final|.
The document source will contain some conditional code
depending on the value of |\version|.
Suppose further, the flag should default to |final| for the main file
and to |draft| for child files
which is a natural assignment for editing the document.
This is achieved by placing the following code
in the preamble of the main document
(below the |\childdocmain| directive):
%
\begin{center}
\begin{tabular}{l}
|\ifchilddoc|\\
|\providecommand{\version}{draft}|\\
|\||else|\\
|\providecommand{\version}{final}|\\
|\||fi|
\end{tabular}
\end{center}
%
The definition by |\providecommand| makes sure
that previous definitions are not overwritten.
Further statements |\providecommand{\version}{...}|
can thus be added before the above code to override it.

For the main file, one might add a line
(between |\childdocmain| and the above block)
%
\begin{center}
|%\ifchilddoc\||else\providecommand{\version}{draft}\||fi|
\end{center}
%
which can be uncommented to produce a draft version.
Likewise one can add a line to the very top of a child file
(above the |\childdocof{|\textit{main}|}| directive)
%
\begin{center}
|%\providecommand{\version}{final}|
\end{center}
%
which can be uncommented to produce the final version of this child document.

%%%%%%%%%%%%%%%%%%%%%%%%%%%%%%%%%%%%%%%%%%%%%%%%%%%%%%%%%%%%%%%%%%%%%%%%%%%%%%%%
\subsection{Forwarding}
\label{sec:forward}

Different versions of the main or child documents
using compilation flags as described in \secref{sec:flags}
can be (permanently) stored in different files
for convenient compilation, viewing and distribution.
To this end, the package defines a command
to pass on compilation to a different file:

%%%%%%%%%%%%%%%%%%%%%%%%%%%%%%%%%%%%%%%%
\DescribeMacro{\childdocforward}
The command |\childdocforward| redirects processing to
another source file:
%
\begin{center}
\begin{tabular}{l}
|\input{childdoc.def}|\\
|\childdocforward[|\textit{main}|]{|\textit{dest}|}|\\
\end{tabular}
\end{center}
%
The argument \textit{dest} is the destination file
(without extension).
It should be the main file or one of the child files.
Note that further \textsf{childdoc} directives
such as |\childdocof| and |\childdocforward|
in the indicated file will be processed in this form.
The optional argument \textit{main}
passes on directly to the main file \textit{main}
while pretending to compile the child \textit{dest}.
This form behaves as if \textit{dest}
issues |\childdocof{|\textit{main}|}| right away,
and no further \textsf{childdoc} directives will be processed.

%%%%%%%%%%%%%%%%%%%%%%%%%%%%%%%%%%%%%%%%
\DescribeMacro{\...prefix}
In the alternative form |\childdocforwardprefix|,
%
\begin{center}
\begin{tabular}{l}
|\input{childdoc.def}|\\
|\childdocforwardprefix[|\textit{main}|]{|\textit{prefix}|}{|\textit{dest}|}|
\end{tabular}
\end{center}
%
the destination file is determined by a pattern
depending on the current file:
To make this work, the current file must be called
`{\textit{prefix}\hspace{0.2em}\textit{suffix}}'
with \textit{prefix} matching precisely the argument.
Processing is then passed on to the file
`{\textit{dest}\hspace{0.2em}\textit{suffix}}'.
Surely, the same effect is achieved by
directly specifying the
argument `{\textit{dest}\hspace{0.2em}\textit{suffix}}'
in the first form.
However, that requires to set up a different file
for each child. With the alternative form of the command
all these files can have exactly the same content
which simplifies setting them up and maintaining them.

For example, the following file |draft.tex|
with a compilation flag |\version| as described in \secref{sec:flags}
compiles the main document as a draft:
%
\begin{center}
\begin{tabular}{l}
|\def\version{draft}|\\
|\input{childdoc.def}|\\
|\childdocforward{|\textit{main}|}|
\end{tabular}
\end{center}
%
Likewise, the following files |final|\textit{nn}|.tex|
compile the final version of the child document
|child|\textit{nn}|.tex|:
%
\begin{center}
\begin{tabular}{l}
|\def\version{final}|\\
|\input{childdoc.def}|\\
|\childdocforwardprefix{final}{child}|
\end{tabular}
\end{center}
%

Note that when several versions of a main file and/or of each child file
are to be generated, it may be convenient to set up a |Makefile| or
shell script to automatise the process.

%%%%%%%%%%%%%%%%%%%%%%%%%%%%%%%%%%%%%%%%%%%%%%%%%%%%%%%%%%%%%%%%%%%%%%%%%%%%%%%%
\subsection{Command Line Processing}
\label{sec:commandline}

The effect of redirection files can also be achieved by invoking
the \LaTeX{} compiler with a more elaborate command line.
Most conveniently this should be done as part
of a shell script or a |Makefile|.

When using \textsf{childdoc} in the main file, the following
command lines effectively perform a redirection
(note that depending on the shell being used,
backslashes may have to be doubled: `|\|' $\to$ `|\\|'):
%
\begin{center}
|... -jobname "|\textit{target}|" |\\|"|[\textit{flags}]%
|\input{childdoc.def}\childdocforward[|\textit{main}|]{|\textit{dest}|}"|
\end{center}
%
Here \textit{target} is the name of the output file,
\textit{main} is the name of the main file
and \textit{dest} is the name of the main or child file to be processed
(all filenames without extensions).
The optional argument \textit{main} can be omitted
if \textit{main} matches \textit{dest}.
Optionally, compilation \textit{flags} can be defined via |\def| commands.
This command line makes the \TeX{} engine believe
it is compiling the file \textit{target}
whose content is specified as the latter parameter.
The provided code then forwards the processing to
\textit{main} or \textit{dest} as described in \secref{sec:forward}.

%%%%%%%%%%%%%%%%%%%%%%%%%%%%%%%%%%%%%%%%%%%%%%%%%%%%%%%%%%%%%%%%%%%%%%%%%%%%%%%%
\subsection{Include by Input}
\label{sec:input}

Including child documents by |\include| has some restrictions by design.
Most notably, the content of a child document always occupies
its own set of pages; pages cannot be shared between child documents.
Usually, this behaviour makes perfect sense
because each child document contain an essential part of the document.
However, in some situations it may be desirable to compose
a document from a collection of parts
without having mandatory page breaks between then.
For this case, the package
provides a mechanism to include parts
by |\input| which can also be processed individually.
However, by construction this mechanism
requires manual handling of the content to be output.

%%%%%%%%%%%%%%%%%%%%%%%%%%%%%%%%%%%%%%%%
\DescribeMacro{\ifchilddocmanual}
The main file should be prepared as usual, see \secref{sec:include}.
However, the document body must make a distinction
between processing of an individual part and of the main document, e.g.:
%
\begin{center}
\begin{tabular}{l}
|\ifchilddocmanual|\\
|\input{\childdocname}|\\
|\||else|\\
\textit{document body with }|\input{|\textit{part}|}|\\
|\||fi|
\end{tabular}
\end{center}
%
The conditional |\ifchilddocmanual| is true whenever
a part to be included by |\input| is being compiled,
and the name of the part is stored in |\childdocname|.

%%%%%%%%%%%%%%%%%%%%%%%%%%%%%%%%%%%%%%%%
\DescribeMacro{\childdocby}
Each part to be included by |\input| should start with:
%
\begin{center}
\begin{tabular}{l}
|\input{childdoc.def}|\\
|\childdocby{|\textit{main}|}|\\
\end{tabular}
\end{center}
%
The directive |\childdocby| is similar to |\childdocof|
described in \secref{sec:include},
but the subsequent selection of content must be done manually.
To that end, both |\ifchilddoc| and |\ifchilddocmanual|
will be true upon processing of a part,
and the name of the part is stored in |\childdocname|.
Note that |\jobname| will be set to the filename of the current part
so that each part receives an individual |.aux| file
that does not interfere with the |.aux| file(s) of the main document.
This behaviour can be altered by the alternative form
|\childdocby[*]{|\textit{main}|}| (with a non-empty optional argument)
which uses the |.aux| file of the main document
by setting |\jobname| to \textit{main}.

%%%%%%%%%%%%%%%%%%%%%%%%%%%%%%%%%%%%%%%%%%%%%%%%%%%%%%%%%%%%%%%%%%%%%%%%%%%%%%%%
\subsection{Driver Development}
\label{sec:driver}

The \textsf{childdoc} mechanism can also be use for the development
of definition files such as \LaTeX{} styles or classes.
This case differs from the above setup with multiple parts
included by |\include| in that no |\includeonly| should be invoked.
This can be achieved by starting the include file
(before |\ProvidesPackage|) with:
%
\begin{center}
\begin{tabular}{l}
|\input{childdoc.def}|\\
|\childdocforward{|\textit{main}|}|\\
\end{tabular}
\end{center}
%
or alternatively with:
%
\begin{center}
\begin{tabular}{l}
|\input{childdoc.def}|\\
|\childdocby{|\textit{main}|}|\\
\end{tabular}
\end{center}
%
Both forms have slightly different effects as described above.
The main file is prepared as usual, see \secref{sec:include}.

%%%%%%%%%%%%%%%%%%%%%%%%%%%%%%%%%%%%%%%%%%%%%%%%%%%%%%%%%%%%%%%%%%%%%%%%%%%%%%%%
\subsection{Legacy Detection}
\label{sec:detection}

The directive |\childdocmain| in the main file can detect
whether the complete document or merely a child is to be compiled
even without using the directive |\childdocof|.
This method is deprecated because it is less robust
and there is no compelling reason to use it;
it is merely provided for backward compatibility
and it may be removed in future versions.

If the detection mechanism is to be used,
it is mandatory to correctly specify
the filename of the main file as the argument of |\childdocmain|:
%
\begin{center}
\begin{tabular}{l}
|\input{childdoc.def}|\\
|\childdocmain{|\textit{main}|}|\\
\end{tabular}
\end{center}
%
If |\jobname| does not match the argument \textit{main} of |\childdocmain|,
it is assumed that |\jobname| points to the child file to be compiled.
When using |\childdocmain| with the main file specified as argument,
it suffices to start a child file
with just |\input{|\textit{main}|}|
without loading of the package and using |\childdocof|.
If instead all processing is done
with the appropriate \textsf{childdoc} directives,
the argument of \textit{main} of |\childdocmain| can be empty.

An alternative version of the command line processing described
in \secref{sec:commandline} using the detection mechanism reads:
%
\begin{center}
|... -jobname "|\textit{target}|" "|[\textit{flags}]%
[|\def\jobname{|\textit{dest}|}|]|\input{|\textit{main}|}"|
\end{center}

%%%%%%%%%%%%%%%%%%%%%%%%%%%%%%%%%%%%%%%%%%%%%%%%%%%%%%%%%%%%%%%%%%%%%%%%%%%%%%%%
\subsection{Manual Code}
\label{sec:manual}

In case one cannot be certain whether the definitions file |childdoc.def|
is installed on the target \TeX{} distribution
and one prefers not to ship it,
it is conceivable to paste a few relevant commands into the sources.

To that end, drop all statements |\input{childdoc.def}|
and perform the replacements as outlined below.
Instead of |\childdocmain{|\textit{main}|}| add the following code
to the top of the main file:
%
\begin{center}
\begin{tabular}{l}
|\||ifdefined\childdocname\endinput\||fi\newif\ifchilddoc|\\
|\edef\childdocname{\scantokens\expandafter{\jobname\noexpand}}|\\
|\def\childdocmain{|\textit{main}|}\||ifx\childdocmain\childdocname\||else|\\
|\childdoctrue\includeonly{\childdocname}\let\jobname\childdocmain\||fi|\\
\end{tabular}
\end{center}
%
Instead of |\childdocof{|\textit{main}|}| just include the main file
at the top of each child file:
%
\begin{center}
|\input{|\textit{main}|}|
\end{center}
%
A simple redirection |\childdocforward{|\textit{dest}|}| is achieved by:
%
\begin{center}
|\def\jobname{|\textit{dest}|}\input{\jobname}|
\end{center}
%
The redirection with prefix
|\childdocforwardprefix[|\textit{prefix}|]{|\textit{dest}|}|
is accomplished by:
%
\begin{center}
\begin{tabular}{l}
|{\edef\jobname{\scantokens\expandafter{\jobname\noexpand}}|\\
|\def\redirectjob |\textit{prefix}|#1~~~{\gdef\jobname{|\textit{dest}|#1}}|\\
|\expandafter\redirectjob\jobname~~~}\input{\jobname}|
\end{tabular}
\end{center}

In an alternative approach,
child documents can be compiled by a specific command line
without additional code or specific definitions:
%
\begin{center}
|... -jobname "|\textit{target}|" "|[\textit{flags}]%
|\includeonly{|\textit{dest}|}\input{|\textit{main}|}"|
\end{center}
%

%%%%%%%%%%%%%%%%%%%%%%%%%%%%%%%%%%%%%%%%%%%%%%%%%%%%%%%%%%%%%%%%%%%%%%%%%%%%%%%%
%%%%%%%%%%%%%%%%%%%%%%%%%%%%%%%%%%%%%%%%%%%%%%%%%%%%%%%%%%%%%%%%%%%%%%%%%%%%%%%%
\section{Information}

%%%%%%%%%%%%%%%%%%%%%%%%%%%%%%%%%%%%%%%%%%%%%%%%%%%%%%%%%%%%%%%%%%%%%%%%%%%%%%%%
\subsection{Copyright}

Copyright \copyright{} 2017--2018 Niklas Beisert

This work may be distributed and/or modified under the
conditions of the \LaTeX{} Project Public License, either version 1.3
of this license or (at your option) any later version.
The latest version of this license is in
  \url{http://www.latex-project.org/lppl.txt}
and version 1.3 or later is part of all distributions of \LaTeX{}
version 2005/12/01 or later.

This work has the LPPL maintenance status `maintained'.

The Current Maintainer of this work is Niklas Beisert.

This work consists of the files |README.txt|, |childdoc.ins| and |childdoc.dtx|
as well as the derived files |childdoc.def|, |cdocsamp.tex|
with |cdocsch1.tex|, |cdocsch2.tex|, |cdocspt3.tex|, |cdocspt4.tex|,
|cdocsdrf.tex|, |cdocsfn1.tex|, |cdocsfn2.tex|
as well as |childdoc.pdf|.

%%%%%%%%%%%%%%%%%%%%%%%%%%%%%%%%%%%%%%%%%%%%%%%%%%%%%%%%%%%%%%%%%%%%%%%%%%%%%%%%
\subsection{Files and Installation}

The package consists of the files:
%
\begin{center}
\begin{tabular}{ll}
    |README.txt|   & readme file \\
    |childdoc.ins| & installation file \\
    |childdoc.dtx| & source file \\
    |childdoc.def| & definition file \\
    |cdocsamp.tex| & sample main file \\
    |cdocsch1.tex| & sample include file \\
    |cdocsch2.tex| & sample include file \\
    |cdocspt3.tex| & sample part file \\
    |cdocspt4.tex| & sample part file \\
    |cdocsdrf.tex| & sample redirection file \\
    |cdocsfn1.tex| & sample redirection file \\
    |cdocsfn2.tex| & sample redirection file \\
    |childdoc.pdf| & manual
\end{tabular}
\end{center}
%
The distribution consists of the files
|README.txt|, |childdoc.ins| and |childdoc.dtx|.
%
\begin{itemize}
\item
Run (pdf)\LaTeX{} on |childdoc.dtx|
to compile the manual |childdoc.pdf| (this file).
\item
Run \LaTeX{} on |childdoc.ins| to create the definitions file |childdoc.def|
and the sample |cdocsamp.tex| with include files
|cdocsch1.tex|, |cdocsch2.tex|, |cdocspt3.tex|, |cdocspt4.tex|,
|cdocsdrf.tex|, |cdocsfn1.tex|, |cdocsfn2.tex|.
Then copy the file |childdoc.def| to an appropriate directory of your \LaTeX{}
distribution, e.g.\ \textit{texmf-root}|/tex/latex/childdoc|.
\end{itemize}

%%%%%%%%%%%%%%%%%%%%%%%%%%%%%%%%%%%%%%%%%%%%%%%%%%%%%%%%%%%%%%%%%%%%%%%%%%%%%%%%
\subsection{Related CTAN Packages}

There are several other packages which offer a similar functionality:
%
\begin{itemize}
\item
The packages
\href{http://ctan.org/pkg/docmute}{\textsf{docmute}},
\href{http://ctan.org/pkg/includex}{\textsf{includex}} and
\href{http://ctan.org/pkg/standalone}{\textsf{standalone}}
provide commands to include only the document body of
a child file thus allowing both files to be compiled individually.
\item
The packages \href{http://ctan.org/pkg/subdocs}{\textsf{subdocs}}
and \href{http://ctan.org/pkg/subfiles}{\textsf{subfiles}}
provide structures in which the main and child documents can be
encapsulated and allowing them to be compiled individually.
The inclusion mechanism is different from the conventional |\include|.
\item
The package \href{http://ctan.org/pkg/combine}{\textsf{combine}}
is an elaborate solution to combine several documents into one.
\end{itemize}
%
See also the CTAN topic \href{http://ctan.org/topic/subdocs}{\textsf{subdocs}}
for further related packages.
The present package differs from the above solutions in that
a document structure constructed with the conventional |\include| mechanism
just needs two extra commands at the top of every file
such that all constituent files can be compiled individually.

%%%%%%%%%%%%%%%%%%%%%%%%%%%%%%%%%%%%%%%%%%%%%%%%%%%%%%%%%%%%%%%%%%%%%%%%%%%%%%%%
%\subsection{Feature Suggestions}
%
%The following is a list of features which may be useful for future
%versions of this package:
%%
%\begin{itemize}
%\item
%\ldots
%\end{itemize}

%%%%%%%%%%%%%%%%%%%%%%%%%%%%%%%%%%%%%%%%%%%%%%%%%%%%%%%%%%%%%%%%%%%%%%%%%%%%%%%%
\subsection{Revision History}

%%%%%%%%%%%%%%%%%%%%%%%%%%%%%%%%%%%%%%%%
\paragraph{v2.0:} 2018/12/30

\begin{itemize}
\item
immediate forward processing
\item
added |\childdocby| mechanism
\item
manual restructured
\end{itemize}

%%%%%%%%%%%%%%%%%%%%%%%%%%%%%%%%%%%%%%%%
\paragraph{v1.6:} 2018/01/17

\begin{itemize}
\item
application for development of include files
\item
corrections to manual
\end{itemize}

%%%%%%%%%%%%%%%%%%%%%%%%%%%%%%%%%%%%%%%%
\paragraph{v1.5:} 2017/05/21

\begin{itemize}
\item
more complete structuring introduced
\item
|\childdocof| introduced
\item
|\childdoc| renamed to |\childdocmain|
\item
|\childredirect| renamed to |\childdocforward| and |\childdocforwardprefix|
and functionality expanded
\end{itemize}

%%%%%%%%%%%%%%%%%%%%%%%%%%%%%%%%%%%%%%%%
\paragraph{v1.0:} 2017/04/27

\begin{itemize}
\item
manual and install package
\item
first version published on CTAN
\end{itemize}

%%%%%%%%%%%%%%%%%%%%%%%%%%%%%%%%%%%%%%%%
\paragraph{v0.6:} 2017/04/26

\begin{itemize}
\item
redirection mechanism added
\end{itemize}

%%%%%%%%%%%%%%%%%%%%%%%%%%%%%%%%%%%%%%%%
\paragraph{v0.5:} 2017/04/26

\begin{itemize}
\item
functionality in definition file
\end{itemize}


%%%%%%%%%%%%%%%%%%%%%%%%%%%%%%%%%%%%%%%%%%%%%%%%%%%%%%%%%%%%%%%%%%%%%%%%%%%%%%%%
%%%%%%%%%%%%%%%%%%%%%%%%%%%%%%%%%%%%%%%%%%%%%%%%%%%%%%%%%%%%%%%%%%%%%%%%%%%%%%%%
%%%%%%%%%%%%%%%%%%%%%%%%%%%%%%%%%%%%%%%%%%%%%%%%%%%%%%%%%%%%%%%%%%%%%%%%%%%%%%%%
\appendix

\settowidth\MacroIndent{\rmfamily\scriptsize 000\ }

 \DocInput{childdoc.dtx}

\end{document}
%</driver>
% \fi
%
% %%%%%%%%%%%%%%%%%%%%%%%%%%%%%%%%%%%%%%%%%%%%%%%%%%%%%%%%%%%%%%%%%%%%%%%%%%%%%%
% %%%%%%%%%%%%%%%%%%%%%%%%%%%%%%%%%%%%%%%%%%%%%%%%%%%%%%%%%%%%%%%%%%%%%%%%%%%%%%
% \section{Sample}
%\iffalse
%<*samplemain>
%\fi
%
% The following presents a sample document
% with two chapters, two parts, a title page,
% a compile flag as well as three forwarding files to set the flag.
% It consists of eight |.tex| files:
% \begin{center}
% \begin{tabular}{ll}
% |cdocsamp.tex|&main file\\
% |cdocsch1.tex|&include file for chapter 1\\
% |cdocsch2.tex|&include file for chapter 2\\
% |cdocspt3.tex|&include file for part 3\\
% |cdocspt4.tex|&include file for part 4\\
% |cdocsdrf.tex|&forwarding file for main file in draft mode\\
% |cdocsfi1.tex|&forwarding file for final version of chapter 1\\
% |cdocsfi2.tex|&forwarding file for final version of chapter 2\\
% \end{tabular}
% \end{center}
% Each of the eight files can be compiled directly by the \LaTeX{} compiler.
%
% %%%%%%%%%%%%%%%%%%%%%%%%%%%%%%%%%%%%%%
% \paragraph{Main File.}
%
% The main file is called |cdocsamp.tex|.
%
% Load the \textsf{childdoc} definitions and
% declare the filename for the main document:
%    \begin{macrocode}
\input{childdoc.def}
\childdocmain{}
%    \end{macrocode}

% Optional override for |\version| flag:
%    \begin{macrocode}
%%\ifchilddoc\else\providecommand{\version}{draft}\fi
%    \end{macrocode}

% Define the default values for the |\version| flag
% (|final| for the main file and |draft| for childs):
%    \begin{macrocode}
\ifchilddoc
\providecommand{\version}{draft}
\else
\providecommand{\version}{final}
\fi
%    \end{macrocode}

% Load the standard document class:
%    \begin{macrocode}
\documentclass[12pt]{article}
%    \end{macrocode}

% Start the document body:
%    \begin{macrocode}
\begin{document}
%    \end{macrocode}

% Declare a title page.
% Print title, part of document being processed and version flag:
%    \begin{macrocode}
\addtocounter{page}{-1}
\begin{center}
{\LARGE\bfseries{}childdoc example\par}
\vspace{1cm}
\ifchilddoc
\ifchilddocmanual part\else chapter\fi:
`\childdocname' of `\childdocjob'\par
\else
main document: `\childdocjob'\par
\fi
version: \version\par
\end{center}
\newpage
%    \end{macrocode}

% Manually include selected file,
% otherwise process as usual:
%    \begin{macrocode}
\ifchilddocmanual
\section*{part `\childdocname'}
\input{\childdocname}
\else
%    \end{macrocode}

% Include the two chapters:
%    \begin{macrocode}
\include{cdocsch1}
\include{cdocsch2}
%    \end{macrocode}

% Include the two parts unless only chapters should be displayed:
%    \begin{macrocode}
\ifchilddoc\else
\section{part three}
\input{cdocspt3}
\section{part four}
\input{cdocspt4}
\fi
%    \end{macrocode}

% Process as usual until here:
%    \begin{macrocode}
\fi
%    \end{macrocode}

% End of document body:
%    \begin{macrocode}
\end{document}
%    \end{macrocode}
%\iffalse
%</samplemain>
%\fi
%
% %%%%%%%%%%%%%%%%%%%%%%%%%%%%%%%%%%%%%%
% \paragraph{Chapter Include Files.}
%
% The include files are called |cdocsch1.tex| and |cdocsch2.tex|.
%
%\iffalse
%<*samplechap1|samplechap2>
%\fi

% Optional override for |\version| flag:
%    \begin{macrocode}
%%\providecommand{\version}{final}
%    \end{macrocode}

% Include the main document:
%    \begin{macrocode}
\input{childdoc.def}
\childdocof{cdocsamp}
%    \end{macrocode}

%\iffalse
%</samplechap1|samplechap2>
%\fi
%
%\iffalse
%<*samplechap1>
%\fi
% Some text for chapter 1:
%    \begin{macrocode}
\section{one}
some text in chapter one
%    \end{macrocode}

%\iffalse
%</samplechap1>
%\fi
% Some text for chapter 2:
%\iffalse
%<*samplechap2>
%\fi
%    \begin{macrocode}
\section{two}
more text in chapter two
%    \end{macrocode}

%\iffalse
%</samplechap2>
%\fi
%
% %%%%%%%%%%%%%%%%%%%%%%%%%%%%%%%%%%%%%%
% \paragraph{Part Include Files.}
%
% The include files are called |cdocspt3.tex| and |cdocspt4.tex|.
%
%\iffalse
%<*samplepart3|samplepart4>
%\fi

% Optional override for |\version| flag:
%    \begin{macrocode}
%%\providecommand{\version}{final}
%    \end{macrocode}

% Include the main document:
%    \begin{macrocode}
\input{childdoc.def}
\childdocby{cdocsamp}
%    \end{macrocode}

%\iffalse
%</samplepart3|samplepart4>
%\fi
%
%\iffalse
%<*samplepart3>
%\fi
% Some text for part 3:
%    \begin{macrocode}
some text in part three
%    \end{macrocode}

%\iffalse
%</samplepart3>
%\fi
% Some text for part 4:
%\iffalse
%<*samplepart4>
%\fi
%    \begin{macrocode}
more text in part four
%    \end{macrocode}

%\iffalse
%</samplepart4>
%\fi
%
% %%%%%%%%%%%%%%%%%%%%%%%%%%%%%%%%%%%%%%
% \paragraph{Forwarding for a Complete Draft.}
%
% The following forwarding file |cdocsdrf.tex|
% compiles the main document in draft mode:
%\iffalse
%<*sampledraft>
%\fi
%    \begin{macrocode}
\def\version{draft}
\input{childdoc.def}
\childdocforward{cdocsamp}
%    \end{macrocode}

%\iffalse
%</sampledraft>
%\fi
%
% %%%%%%%%%%%%%%%%%%%%%%%%%%%%%%%%%%%%%%
% \paragraph{Forwarding for Final Version of the Chapters.}
%
% The following forwarding files |cdocsfn1.tex| and |cdocsfn2.tex|
% (with identical content)
% compile the final versions of the child documents
% |cdocsch1.tex| and |cdocsch2.tex|, respectively:
%\iffalse
%<*samplefinal>
%\fi
%    \begin{macrocode}
\def\version{final}
\input{childdoc.def}
\childdocforwardprefix[cdocsamp]{cdocsfn}{cdocsch}
%    \end{macrocode}

%\iffalse
%</samplefinal>
%\fi
%
% %%%%%%%%%%%%%%%%%%%%%%%%%%%%%%%%%%%%%%
% \paragraph{Command Line Processing.}
%
% The following three command lines generate the output files
% |cdocscld|, |cdocscl1| and |cdocscl2|
% which should be identical to
% |cdocsdrf|, |cdocsch1| and |cdocsfn2|, respectively:
% \begin{center}
% \begin{tabular}{l}
% |latex -jobname cdocscld \|\\
% |  "\def\version{draft}\input{childdoc.def}\childdocforward{cdocsamp}"|\\
% |latex -jobname cdocscl1 \|\\
% |  "\input{childdoc.def}\childdocforward[cdocsamp]{cdocsch1}"|\\
% |latex -jobname cdocscl2 \|\\
% |  "\def\version{final}\input{childdoc.def}\childdocforward{cdocsch2}"|
% \end{tabular}
% \end{center}
% Note that the trailing backslash on each first line
% merely continues the input to the second line
% (for convenient cut ant paste).
% Furthermore, the command |latex| can be replaced by any
% of its alternative versions such as |pdflatex|.
%
% %%%%%%%%%%%%%%%%%%%%%%%%%%%%%%%%%%%%%%%%%%%%%%%%%%%%%%%%%%%%%%%%%%%%%%%%%%%%%%
% %%%%%%%%%%%%%%%%%%%%%%%%%%%%%%%%%%%%%%%%%%%%%%%%%%%%%%%%%%%%%%%%%%%%%%%%%%%%%%
% \section{Implementation}
%\iffalse
%<*package>
%\fi
%
% This section describes the definitions file |childdoc.def|.

% The definitions cannot be loaded using |\usepackage| or |\RequirePackage|
% which has a mechanism to prevent loading a style file more than once.
% When loading the definitions by means of |\input|
% multiple instances have to be prevented manually:
%\iffalse
%This code needs to be before the `\ProvidesFile' directive
%which is defined at the beginning of this file.
%Therefore it is also placed there and commented out here.
%</package>
%<*discard>
%\fi
%    \begin{macrocode}
\ifdefined\childdocmain\endinput\fi
%    \end{macrocode}
%\iffalse
%</discard>
%<*package>
%\fi
%
% \macro{\ifchilddoc}
% \macro{\ifchilddocmanual}
% The conditional |\ifchilddoc| tells whether a
% child (true) or main (false) document is being compiled.
% The conditional |\ifchilddocmanual| tells whether
% the |\includeonly| mechanism is used (false) or
% the selection of child files must be performed manually (true).
% The definitions initialise to false:
%    \begin{macrocode}
\newif\ifchilddoc
\newif\ifchilddocmanual
%    \end{macrocode}

% \macro{\childdocname}
% \macro{\childdocjob}
% The macro |\childdocname| stores the name of the main document
% to be compiled. The macro |\childdocjob| stores the name of
% the document on which the \LaTeX{} compiler was originally invoked.
% The content of |\jobname| cannot be compared
% to filenames specified in the source due to different catcodes.
% The following code rescans |\jobname|, stores the result
% in |\childdocname| and saves a copy in |\childdocjob|:
%    \begin{macrocode}
\edef\childdocname{\scantokens\expandafter{\jobname\noexpand}}
\let\childdocjob\childdocname
%    \end{macrocode}

% \macro{\childdocdisable}
% The macro |\childdocdisable| prevents the main file
% from being processed more than once.
% At this stage, the main document command |\childdocmain|
% is assumed to be called once again where it should do nothing.
% Any subsequent call to it should prevent
% a secondary processing of the main document
% It overwrites the forwarding commands
% |\childdocof| and |\childdocforward|
% with empty macros to prevent further inclusions of the main document:
%    \begin{macrocode}
\newcommand{\childdocdisable}
{
  \renewcommand{\childdocmain}[1]{\renewcommand{\childdocmain}[1]{\endinput}}
  \renewcommand{\childdocof}[1]{}
  \renewcommand{\childdocby}[2][]{}
  \renewcommand{\childdocforward}[2][]{}
  \renewcommand{\childdocdisable}{}
}
%    \end{macrocode}

% \macro{\childdocmain}
% The macro |\childdocmain| is to be called at the top of the main file
% with nothing or the main filename (without extension) as argument.
% First, it breaks loops.
% If the argument is not empty and does not match |\childdocname|
% (which is set by the first inclusion of |childdoc.def|),
% |\ifchilddoc| is set to true, |\includeonly| is applied to the child file
% and |\jobname| is set to the main file
% (for proper handling of |.aux| files):
%    \begin{macrocode}
\newcommand{\childdocmain}[1]
{
  \childdocdisable\childdocmain{}
  \if?#1?\else
    \begingroup
      \def\childdoctmp{#1}
      \ifx\childdoctmp\childdocname
        \def\childdoctmp{}
      \else
        \def\childdoctmp
        {
          \childdoctrue
          \includeonly{\childdocname}
          \def\childdocjob{#1}
          \def\jobname{#1}
        }
      \fi
      \expandafter
    \endgroup
    \childdoctmp
  \fi
}
%    \end{macrocode}

% \macro{\childdocof}
% The command |\childdocof| redirects
% compilation to the main file |#1|.
%    \begin{macrocode}
\newcommand{\childdocof}[1]
{
  \childdocdisable
  \childdoctrue
  \includeonly{\childdocname}
  \def\jobname{#1}
  \def\childdocjob{#1}
  \input{#1}
}
%    \end{macrocode}

% \macro{\childdocby}
% The command |\childdocby| ....
%    \begin{macrocode}
\newcommand{\childdocby}[2][]
{
  \childdocdisable
  \childdoctrue
  \childdocmanualtrue
  \if?#1?\else
    \def\jobname{#2}
  \fi
  \def\childdocjob{#2}
  \input{#2}
  \endinput
}
%    \end{macrocode}

% \macro{\childdocforward}
% The command |\childdocforward| redirects
% compilation to the main file or
% (if the optional argument is given) a child file.
% Parameters are set as if the main file
% or a child file starting with |\childdocof| was compiled.
% Then compilation is handed over to the main file:
%    \begin{macrocode}
\newcommand{\childdocforward}[2][]
{
  \begingroup
    \if?#1?
      \def\childdoctmp
      {
        \def\childdocname{#2}
        \def\childdocjob{#2}
        \def\jobname{#2}
        \input{#2}
        \endinput
      }
    \else
      \def\childdoctmp
      {
        \childdocdisable
        \def\childdocname{#2}
        \childdoctrue
        \includeonly{#2}
        \def\childdocjob{#1}
        \def\jobname{#1}
        \input{#1}
        \endinput
      }
    \fi
    \expandafter
  \endgroup
  \childdoctmp
}
%    \end{macrocode}

% \macro{\childdocforwardprefix}
% The command |\childdocforwardprefix| redirects
% compilation to the main or a child file by means of a pattern.
% The prefix |#1| in the current filename is replaced by |#2|
% and the suffix of the current filename is kept
% (it is assumed that the filename does not contain the substring `|~~~|'
% which is used as a delimiter).
% Compilation is handed over to the new file by |\childdocforward|:
%    \begin{macrocode}
\newcommand{\childdocforwardprefix}[3][]
{
  \begingroup
    \def\childdocextract #2##1~~~{\def\childdoctmp{\childdocforward[#1]{#3##1}}}
    \expandafter\childdocextract\childdocname~~~
    \expandafter
  \endgroup
  \childdoctmp
}
%    \end{macrocode}

% \macro{\childdoc}
% The deprecated macro |\childdoc| is a legacy version of |\childdocmain|:
%    \begin{macrocode}
\newcommand{\childdoc}{\childdocmain}
%    \end{macrocode}

% \macro{\childdocredirect}
% The deprecated macro |\childdocredirect| is a legacy version
% of |\childdocforward| and |\childdocforwardprefix|:
%    \begin{macrocode}
\newcommand{\childdocredirect}[2][]
{
  \begingroup
    \if?#1?
      \def\childdoctmp{\childdocforward{#2}}
    \else
      \def\childdoctmp{\childdocforwardprefix{#1}{#2}}
    \fi
    \expandafter
  \endgroup
  \childdoctmp
}
%    \end{macrocode}

%\iffalse
%</package>
%\fi
%
\endinput
|\\
|\childdocmain{}|\\
\end{tabular}
\end{center}
at the very top of the main \LaTeX{} file,
in particular \emph{before} the |\documentclass| statement!
The argument of |\childdocmain| should be left empty
(but it must be present).

%%%%%%%%%%%%%%%%%%%%%%%%%%%%%%%%%%%%%%%%
\DescribeMacro{\childdocof}
Furthermore, add the commands
\begin{center}
\begin{tabular}{l}
|% \iffalse
%
% childdoc.dtx Copyright (C) 2017-2018 Niklas Beisert
%
% This work may be distributed and/or modified under the
% conditions of the LaTeX Project Public License, either version 1.3
% of this license or (at your option) any later version.
% The latest version of this license is in
%   http://www.latex-project.org/lppl.txt
% and version 1.3 or later is part of all distributions of LaTeX
% version 2005/12/01 or later.
%
% This work has the LPPL maintenance status `maintained'.
%
% The Current Maintainer of this work is Niklas Beisert.
%
% This work consists of the files childdoc.dtx and childdoc.ins
% and the derived files childdoc.def and cdocsamp.tex with
% cdocsch1.tex, cdocsch2.tex, cdocsdrf.tex, cdocsfn1.tex, cdocsfn2.tex.
%
%<package>\ifdefined\childdocmain\endinput\fi
%<package>\ProvidesFile{childdoc.def}[2018/12/30 v2.0 child document driver]
%<samplemain>\ProvidesFile{cdocsamp.tex}[2018/12/30 v2.0 sample for childdoc]
%<*driver>
%\ProvidesFile{childdoc.drv}[2018/12/30 v2.0 childdoc reference manual file]
\PassOptionsToClass{10pt,a4paper}{article}
\documentclass{ltxdoc}

\usepackage[margin=35mm]{geometry}
\usepackage{hyperref}
\usepackage{hyperxmp}
\usepackage[usenames]{color}

\hypersetup{colorlinks=true}
\hypersetup{pdfstartview=FitH}
\hypersetup{pdfpagemode=UseNone}
\hypersetup{pdfsource={}}
\hypersetup{pdflang={en-UK}}
\hypersetup{pdfcopyright={Copyright 2017-2018 Niklas Beisert.
  This work may be distributed and/or modified under the
  conditions of the LaTeX Project Public License, either version 1.3
  of this license or (at your option) any later version.}}
\hypersetup{pdflicenseurl={http://www.latex-project.org/lppl.txt}}
\hypersetup{pdfcontactaddress={ETH Zurich, ITP, HIT K,
  Wolfgang-Pauli-Strasse 27}}
\hypersetup{pdfcontactpostcode={8093}}
\hypersetup{pdfcontactcity={Zurich}}
\hypersetup{pdfcontactcountry={Switzerland}}
\hypersetup{pdfcontactemail={nbeisert@itp.phys.ethz.ch}}
\hypersetup{pdfcontacturl={http://people.phys.ethz.ch/\xmptilde nbeisert/}}

\newcommand{\secref}[1]{\hyperref[#1]{section \ref*{#1}}}

\parskip1ex
\parindent0pt
\let\olditemize\itemize
\def\itemize{\olditemize\parskip0pt}

\begin{document}

\title{The \textsf{childdoc} Package}
\hypersetup{pdftitle={The childdoc Package}}
\author{Niklas Beisert\\[2ex]
  Institut f\"ur Theoretische Physik\\
  Eidgen\"ossische Technische Hochschule Z\"urich\\
  Wolfgang-Pauli-Strasse 27, 8093 Z\"urich, Switzerland\\[1ex]
  \href{mailto:nbeisert@itp.phys.ethz.ch}
  {\texttt{nbeisert@itp.phys.ethz.ch}}}
\hypersetup{pdfauthor={Niklas Beisert}}
\hypersetup{pdfsubject={Manual for the LaTeX2e Package childdoc}}
\date{30 December 2018, \textsf{v2.0}}
\maketitle

\begin{abstract}\noindent
\textsf{childdoc} is a \LaTeXe{} package
that enables the direct compilation
of document sections included by |\include|
to individual files.
\end{abstract}

\begingroup
\parskip0ex
\tableofcontents
\endgroup

%%%%%%%%%%%%%%%%%%%%%%%%%%%%%%%%%%%%%%%%%%%%%%%%%%%%%%%%%%%%%%%%%%%%%%%%%%%%%%%%
%%%%%%%%%%%%%%%%%%%%%%%%%%%%%%%%%%%%%%%%%%%%%%%%%%%%%%%%%%%%%%%%%%%%%%%%%%%%%%%%
\section{Introduction}

\LaTeX{} provides a mechanism to structure a large document (such as a book)
into a main file and several child files (containing the chapters)
using the |\include| command.
This mechanism is beneficial for documents
which span hundreds of pages in order to
make the source file(s) more manageable.
Moreover, compilation can be restricted to
selected child files by means of the |\includeonly| command.
The latter feature can be used to reduce the compilation time while editing
(this was significantly more useful in the earlier days of \LaTeX{})
or to generate a smaller document which is easier to navigate.
Another application of |\includeonly| is to generate
documents consisting of selected parts of the complete document.

However, there are a few drawbacks of the plain |\include| mechanism:
\begin{itemize}
\item
The child files cannot be compiled on their own,
they can only be compiled via the main file.
A naive editing environment
(such as a text editor with an option
to have the current file processed by \LaTeX)
may require one to switch to the main file before compiling;
attempting to compile the child file produces errors.
\item
The main file must be modified (each time)
to adjust the |\includeonly| command
to the present needs. This easily leaves the main file in a messy state.
\item
The generated document will always carry the filename
of the main document. This is inconvenient if
several child files are to be compiled and
to be kept for distribution.
\end{itemize}

The present package provides a simple interface
to make child files individually compilable by \LaTeX{}.
Compiling a child file then has the same effect as compiling
the main file with an |\includeonly| command
to select the appropriate child.
Moreover the generated document will carry the name of the child
rather than the main file.
This resolves all three above issues.

This feature is meant to make the editing of books,
thesis documents and lecture notes somewhat more convenient.
However, the package can also be used efficiently for
composing a series of documents (such as exercise sheets)
which are typically distributed individually.
It then assists the author in generating the individual documents
(potentially in different versions)
as well as a document containing the collected series.
Another application is in developing style files
or other kinds of included material
where compilation of the style file could redirect
to a sample or test file.

%%%%%%%%%%%%%%%%%%%%%%%%%%%%%%%%%%%%%%%%%%%%%%%%%%%%%%%%%%%%%%%%%%%%%%%%%%%%%%%%
%%%%%%%%%%%%%%%%%%%%%%%%%%%%%%%%%%%%%%%%%%%%%%%%%%%%%%%%%%%%%%%%%%%%%%%%%%%%%%%%
\section{Usage}

First of all, the package \textsf{childdoc} is \emph{not} a standard
\LaTeXe{} |.sty| style file! Therefore it needs to be invoked in
a non-standard way.

%%%%%%%%%%%%%%%%%%%%%%%%%%%%%%%%%%%%%%%%%%%%%%%%%%%%%%%%%%%%%%%%%%%%%%%%%%%%%%%%
\subsection{Included Files}
\label{sec:include}

%%%%%%%%%%%%%%%%%%%%%%%%%%%%%%%%%%%%%%%%
\DescribeMacro{\childdocmain}
To use the package, add the commands
\begin{center}
\begin{tabular}{l}
|\input{childdoc.def}|\\
|\childdocmain{}|\\
\end{tabular}
\end{center}
at the very top of the main \LaTeX{} file,
in particular \emph{before} the |\documentclass| statement!
The argument of |\childdocmain| should be left empty
(but it must be present).

%%%%%%%%%%%%%%%%%%%%%%%%%%%%%%%%%%%%%%%%
\DescribeMacro{\childdocof}
Furthermore, add the commands
\begin{center}
\begin{tabular}{l}
|\input{childdoc.def}|\\
|\childdocof{|\textit{main}|}|\\
\end{tabular}
\end{center}
at the top of every child file \textit{child}
which is included by |\include{|\textit{child}|}|
from within the main file
(or at least for those files to be compiled individually).
The argument \textit{main} must be the filename of the main file.

There are a couple of
considerations in setting up the main and child documents:

%%%%%%%%%%%%%%%%%%%%%%%%%%%%%%%%%%%%%%%%
\paragraph{Restrictions.}

Please note the following restrictions:
\begin{itemize}
\item
|\childdocmain| must be called with one argument \textit{main}
to ensure compatibility with earlier version of the package.
It must either be empty (|\childdocmain{}|)
or precisely match the filename of the main file in which it is specified.
See \secref{sec:detection} for further information.
\item
The filename \textit{main} must be specified without the |.tex| extension.
\item
The filename \textit{main} is case sensitive
(even in case-insensitive file systems)
due to internal string comparison.
\item
The argument \textit{main} should be fully expanded, it cannot be a macro.
\item
Subdirectories and special characters should be avoided in filenames.
\item
The command |\childdocmain{|\textit{main}|}| must be followed by a whitespace.
It should not be followed immediately by another command
or by a comment mark `|%|'.
This is because the \TeX{} parser reads the token immediately following
the argument of |\childdocmain| and puts it
at the beginning of every child section;
however, a white\-space is ignored.
\end{itemize}

%%%%%%%%%%%%%%%%%%%%%%%%%%%%%%%%%%%%%%%%
\paragraph{Content of Main File.}

It is advisable to place all content in the child files included by |\include|.
Any output contained in the main file will appear in all child documents
unless suppressed manually;
it cannot be suppressed automatically by the |\includeonly| directive
and thus should normally be avoided.
A method to include some content in the main file
by means of conditional processing is described in \secref{sec:conditional}.

%%%%%%%%%%%%%%%%%%%%%%%%%%%%%%%%%%%%%%%%
\paragraph{Page Numbering.}

When only a part of the document is compiled,
the appropriate numbering of pages
(as well as other status parameters)
is determined from the |.aux| files.
The latter contain information from previous passes.
However this information needs to propagate through
all intermediate child documents.
Therefore the page numbering in child documents may well
be inconsistent until the complete document is compiled at least once.

A useful (if unconventional) way to always ensure a consistent
page numbering is to restart the numbering in each child document
and denote the pages by `\textit{child}|.|\textit{page}'
where \textit{child} represents the chapter/section number of the child file.
This can be achieved by the command
|\numberwithin{page}{|\textit{child}|}|
of the \textsf{amsmath} package
where \textit{child} can be |chapter| or |section|
depending on the chosen structuring.
Alternatively, one can modify the macro |\thepage| appropriately
and reset the counter |page| at the start of each child file.

%%%%%%%%%%%%%%%%%%%%%%%%%%%%%%%%%%%%%%%%%%%%%%%%%%%%%%%%%%%%%%%%%%%%%%%%%%%%%%%%
\subsection{Conditional Processing}
\label{sec:conditional}

The package provides a mechanism to compile different versions
of a document. To customise the versions further some conditional processing
can come in handy to distinguish which version is being compiled.
The package provides two macros to describe the compilation context:

%%%%%%%%%%%%%%%%%%%%%%%%%%%%%%%%%%%%%%%%
\DescribeMacro{\ifchilddoc}
The conditional |\ifchilddoc| distinguishes between the compilation of
child documents and the main document:
%
\begin{center}
|\ifchilddoc |\textit{child-code}| |[|\||else |\textit{main-code}]| \||fi|
\end{center}

%%%%%%%%%%%%%%%%%%%%%%%%%%%%%%%%%%%%%%%%
\DescribeMacro{\childdocname}
\DescribeMacro{\childdocjob}
The macro |\childdocname| contains the filename (without extension)
of the main or child file being processed.
Note that |\childdocjob| will always contain the name of the main file.

%%%%%%%%%%%%%%%%%%%%%%%%%%%%%%%%%%%%%%%%
\paragraph{Title Page.}

Conditional processing can be used to include a title or banner page
in the main document when proper precautions are taken.
Importantly, the code in the main file should ensure that the page counter
(as well as other status parameters which are stored in the |.aux| files)
takes the same value after the conditional processing.
Otherwise the page numbers may take divergent values
depending on which part is compiled.

For example, a title page could be declared by:
%
\begin{center}
\begin{tabular}{l}
|\ifchilddoc\||else|\\
|\addtocounter{page}{-1}|\\
\textit{code for title page}\\
|\newpage|\\
|\||fi|
\end{tabular}
\end{center}
%
A banner page for the child documents can be generated by:
%
\begin{center}
\begin{tabular}{l}
|\ifchilddoc|\\
|\addtocounter{page}{-1}|\\
\textit{code for banner page}\\
|\newpage|\\
|\||fi|
\end{tabular}
\end{center}
%
Here one could write a message such as:
\begin{center}
|This is the part \childdocname{} of \childdocjob{}.|
\end{center}

%%%%%%%%%%%%%%%%%%%%%%%%%%%%%%%%%%%%%%%%%%%%%%%%%%%%%%%%%%%%%%%%%%%%%%%%%%%%%%%%
\subsection{Flags}
\label{sec:flags}

The package makes it easy to generate different versions
of the main or child documents.
To this end compilation flags can be defined
and assigned different default values.
They will be particularly useful in conjunction
with the forwarding mechanism described in \secref{sec:forward}.

For example, it may be useful to have a flag |\version|
which can be set to |draft| or |final|.
The document source will contain some conditional code
depending on the value of |\version|.
Suppose further, the flag should default to |final| for the main file
and to |draft| for child files
which is a natural assignment for editing the document.
This is achieved by placing the following code
in the preamble of the main document
(below the |\childdocmain| directive):
%
\begin{center}
\begin{tabular}{l}
|\ifchilddoc|\\
|\providecommand{\version}{draft}|\\
|\||else|\\
|\providecommand{\version}{final}|\\
|\||fi|
\end{tabular}
\end{center}
%
The definition by |\providecommand| makes sure
that previous definitions are not overwritten.
Further statements |\providecommand{\version}{...}|
can thus be added before the above code to override it.

For the main file, one might add a line
(between |\childdocmain| and the above block)
%
\begin{center}
|%\ifchilddoc\||else\providecommand{\version}{draft}\||fi|
\end{center}
%
which can be uncommented to produce a draft version.
Likewise one can add a line to the very top of a child file
(above the |\childdocof{|\textit{main}|}| directive)
%
\begin{center}
|%\providecommand{\version}{final}|
\end{center}
%
which can be uncommented to produce the final version of this child document.

%%%%%%%%%%%%%%%%%%%%%%%%%%%%%%%%%%%%%%%%%%%%%%%%%%%%%%%%%%%%%%%%%%%%%%%%%%%%%%%%
\subsection{Forwarding}
\label{sec:forward}

Different versions of the main or child documents
using compilation flags as described in \secref{sec:flags}
can be (permanently) stored in different files
for convenient compilation, viewing and distribution.
To this end, the package defines a command
to pass on compilation to a different file:

%%%%%%%%%%%%%%%%%%%%%%%%%%%%%%%%%%%%%%%%
\DescribeMacro{\childdocforward}
The command |\childdocforward| redirects processing to
another source file:
%
\begin{center}
\begin{tabular}{l}
|\input{childdoc.def}|\\
|\childdocforward[|\textit{main}|]{|\textit{dest}|}|\\
\end{tabular}
\end{center}
%
The argument \textit{dest} is the destination file
(without extension).
It should be the main file or one of the child files.
Note that further \textsf{childdoc} directives
such as |\childdocof| and |\childdocforward|
in the indicated file will be processed in this form.
The optional argument \textit{main}
passes on directly to the main file \textit{main}
while pretending to compile the child \textit{dest}.
This form behaves as if \textit{dest}
issues |\childdocof{|\textit{main}|}| right away,
and no further \textsf{childdoc} directives will be processed.

%%%%%%%%%%%%%%%%%%%%%%%%%%%%%%%%%%%%%%%%
\DescribeMacro{\...prefix}
In the alternative form |\childdocforwardprefix|,
%
\begin{center}
\begin{tabular}{l}
|\input{childdoc.def}|\\
|\childdocforwardprefix[|\textit{main}|]{|\textit{prefix}|}{|\textit{dest}|}|
\end{tabular}
\end{center}
%
the destination file is determined by a pattern
depending on the current file:
To make this work, the current file must be called
`{\textit{prefix}\hspace{0.2em}\textit{suffix}}'
with \textit{prefix} matching precisely the argument.
Processing is then passed on to the file
`{\textit{dest}\hspace{0.2em}\textit{suffix}}'.
Surely, the same effect is achieved by
directly specifying the
argument `{\textit{dest}\hspace{0.2em}\textit{suffix}}'
in the first form.
However, that requires to set up a different file
for each child. With the alternative form of the command
all these files can have exactly the same content
which simplifies setting them up and maintaining them.

For example, the following file |draft.tex|
with a compilation flag |\version| as described in \secref{sec:flags}
compiles the main document as a draft:
%
\begin{center}
\begin{tabular}{l}
|\def\version{draft}|\\
|\input{childdoc.def}|\\
|\childdocforward{|\textit{main}|}|
\end{tabular}
\end{center}
%
Likewise, the following files |final|\textit{nn}|.tex|
compile the final version of the child document
|child|\textit{nn}|.tex|:
%
\begin{center}
\begin{tabular}{l}
|\def\version{final}|\\
|\input{childdoc.def}|\\
|\childdocforwardprefix{final}{child}|
\end{tabular}
\end{center}
%

Note that when several versions of a main file and/or of each child file
are to be generated, it may be convenient to set up a |Makefile| or
shell script to automatise the process.

%%%%%%%%%%%%%%%%%%%%%%%%%%%%%%%%%%%%%%%%%%%%%%%%%%%%%%%%%%%%%%%%%%%%%%%%%%%%%%%%
\subsection{Command Line Processing}
\label{sec:commandline}

The effect of redirection files can also be achieved by invoking
the \LaTeX{} compiler with a more elaborate command line.
Most conveniently this should be done as part
of a shell script or a |Makefile|.

When using \textsf{childdoc} in the main file, the following
command lines effectively perform a redirection
(note that depending on the shell being used,
backslashes may have to be doubled: `|\|' $\to$ `|\\|'):
%
\begin{center}
|... -jobname "|\textit{target}|" |\\|"|[\textit{flags}]%
|\input{childdoc.def}\childdocforward[|\textit{main}|]{|\textit{dest}|}"|
\end{center}
%
Here \textit{target} is the name of the output file,
\textit{main} is the name of the main file
and \textit{dest} is the name of the main or child file to be processed
(all filenames without extensions).
The optional argument \textit{main} can be omitted
if \textit{main} matches \textit{dest}.
Optionally, compilation \textit{flags} can be defined via |\def| commands.
This command line makes the \TeX{} engine believe
it is compiling the file \textit{target}
whose content is specified as the latter parameter.
The provided code then forwards the processing to
\textit{main} or \textit{dest} as described in \secref{sec:forward}.

%%%%%%%%%%%%%%%%%%%%%%%%%%%%%%%%%%%%%%%%%%%%%%%%%%%%%%%%%%%%%%%%%%%%%%%%%%%%%%%%
\subsection{Include by Input}
\label{sec:input}

Including child documents by |\include| has some restrictions by design.
Most notably, the content of a child document always occupies
its own set of pages; pages cannot be shared between child documents.
Usually, this behaviour makes perfect sense
because each child document contain an essential part of the document.
However, in some situations it may be desirable to compose
a document from a collection of parts
without having mandatory page breaks between then.
For this case, the package
provides a mechanism to include parts
by |\input| which can also be processed individually.
However, by construction this mechanism
requires manual handling of the content to be output.

%%%%%%%%%%%%%%%%%%%%%%%%%%%%%%%%%%%%%%%%
\DescribeMacro{\ifchilddocmanual}
The main file should be prepared as usual, see \secref{sec:include}.
However, the document body must make a distinction
between processing of an individual part and of the main document, e.g.:
%
\begin{center}
\begin{tabular}{l}
|\ifchilddocmanual|\\
|\input{\childdocname}|\\
|\||else|\\
\textit{document body with }|\input{|\textit{part}|}|\\
|\||fi|
\end{tabular}
\end{center}
%
The conditional |\ifchilddocmanual| is true whenever
a part to be included by |\input| is being compiled,
and the name of the part is stored in |\childdocname|.

%%%%%%%%%%%%%%%%%%%%%%%%%%%%%%%%%%%%%%%%
\DescribeMacro{\childdocby}
Each part to be included by |\input| should start with:
%
\begin{center}
\begin{tabular}{l}
|\input{childdoc.def}|\\
|\childdocby{|\textit{main}|}|\\
\end{tabular}
\end{center}
%
The directive |\childdocby| is similar to |\childdocof|
described in \secref{sec:include},
but the subsequent selection of content must be done manually.
To that end, both |\ifchilddoc| and |\ifchilddocmanual|
will be true upon processing of a part,
and the name of the part is stored in |\childdocname|.
Note that |\jobname| will be set to the filename of the current part
so that each part receives an individual |.aux| file
that does not interfere with the |.aux| file(s) of the main document.
This behaviour can be altered by the alternative form
|\childdocby[*]{|\textit{main}|}| (with a non-empty optional argument)
which uses the |.aux| file of the main document
by setting |\jobname| to \textit{main}.

%%%%%%%%%%%%%%%%%%%%%%%%%%%%%%%%%%%%%%%%%%%%%%%%%%%%%%%%%%%%%%%%%%%%%%%%%%%%%%%%
\subsection{Driver Development}
\label{sec:driver}

The \textsf{childdoc} mechanism can also be use for the development
of definition files such as \LaTeX{} styles or classes.
This case differs from the above setup with multiple parts
included by |\include| in that no |\includeonly| should be invoked.
This can be achieved by starting the include file
(before |\ProvidesPackage|) with:
%
\begin{center}
\begin{tabular}{l}
|\input{childdoc.def}|\\
|\childdocforward{|\textit{main}|}|\\
\end{tabular}
\end{center}
%
or alternatively with:
%
\begin{center}
\begin{tabular}{l}
|\input{childdoc.def}|\\
|\childdocby{|\textit{main}|}|\\
\end{tabular}
\end{center}
%
Both forms have slightly different effects as described above.
The main file is prepared as usual, see \secref{sec:include}.

%%%%%%%%%%%%%%%%%%%%%%%%%%%%%%%%%%%%%%%%%%%%%%%%%%%%%%%%%%%%%%%%%%%%%%%%%%%%%%%%
\subsection{Legacy Detection}
\label{sec:detection}

The directive |\childdocmain| in the main file can detect
whether the complete document or merely a child is to be compiled
even without using the directive |\childdocof|.
This method is deprecated because it is less robust
and there is no compelling reason to use it;
it is merely provided for backward compatibility
and it may be removed in future versions.

If the detection mechanism is to be used,
it is mandatory to correctly specify
the filename of the main file as the argument of |\childdocmain|:
%
\begin{center}
\begin{tabular}{l}
|\input{childdoc.def}|\\
|\childdocmain{|\textit{main}|}|\\
\end{tabular}
\end{center}
%
If |\jobname| does not match the argument \textit{main} of |\childdocmain|,
it is assumed that |\jobname| points to the child file to be compiled.
When using |\childdocmain| with the main file specified as argument,
it suffices to start a child file
with just |\input{|\textit{main}|}|
without loading of the package and using |\childdocof|.
If instead all processing is done
with the appropriate \textsf{childdoc} directives,
the argument of \textit{main} of |\childdocmain| can be empty.

An alternative version of the command line processing described
in \secref{sec:commandline} using the detection mechanism reads:
%
\begin{center}
|... -jobname "|\textit{target}|" "|[\textit{flags}]%
[|\def\jobname{|\textit{dest}|}|]|\input{|\textit{main}|}"|
\end{center}

%%%%%%%%%%%%%%%%%%%%%%%%%%%%%%%%%%%%%%%%%%%%%%%%%%%%%%%%%%%%%%%%%%%%%%%%%%%%%%%%
\subsection{Manual Code}
\label{sec:manual}

In case one cannot be certain whether the definitions file |childdoc.def|
is installed on the target \TeX{} distribution
and one prefers not to ship it,
it is conceivable to paste a few relevant commands into the sources.

To that end, drop all statements |\input{childdoc.def}|
and perform the replacements as outlined below.
Instead of |\childdocmain{|\textit{main}|}| add the following code
to the top of the main file:
%
\begin{center}
\begin{tabular}{l}
|\||ifdefined\childdocname\endinput\||fi\newif\ifchilddoc|\\
|\edef\childdocname{\scantokens\expandafter{\jobname\noexpand}}|\\
|\def\childdocmain{|\textit{main}|}\||ifx\childdocmain\childdocname\||else|\\
|\childdoctrue\includeonly{\childdocname}\let\jobname\childdocmain\||fi|\\
\end{tabular}
\end{center}
%
Instead of |\childdocof{|\textit{main}|}| just include the main file
at the top of each child file:
%
\begin{center}
|\input{|\textit{main}|}|
\end{center}
%
A simple redirection |\childdocforward{|\textit{dest}|}| is achieved by:
%
\begin{center}
|\def\jobname{|\textit{dest}|}\input{\jobname}|
\end{center}
%
The redirection with prefix
|\childdocforwardprefix[|\textit{prefix}|]{|\textit{dest}|}|
is accomplished by:
%
\begin{center}
\begin{tabular}{l}
|{\edef\jobname{\scantokens\expandafter{\jobname\noexpand}}|\\
|\def\redirectjob |\textit{prefix}|#1~~~{\gdef\jobname{|\textit{dest}|#1}}|\\
|\expandafter\redirectjob\jobname~~~}\input{\jobname}|
\end{tabular}
\end{center}

In an alternative approach,
child documents can be compiled by a specific command line
without additional code or specific definitions:
%
\begin{center}
|... -jobname "|\textit{target}|" "|[\textit{flags}]%
|\includeonly{|\textit{dest}|}\input{|\textit{main}|}"|
\end{center}
%

%%%%%%%%%%%%%%%%%%%%%%%%%%%%%%%%%%%%%%%%%%%%%%%%%%%%%%%%%%%%%%%%%%%%%%%%%%%%%%%%
%%%%%%%%%%%%%%%%%%%%%%%%%%%%%%%%%%%%%%%%%%%%%%%%%%%%%%%%%%%%%%%%%%%%%%%%%%%%%%%%
\section{Information}

%%%%%%%%%%%%%%%%%%%%%%%%%%%%%%%%%%%%%%%%%%%%%%%%%%%%%%%%%%%%%%%%%%%%%%%%%%%%%%%%
\subsection{Copyright}

Copyright \copyright{} 2017--2018 Niklas Beisert

This work may be distributed and/or modified under the
conditions of the \LaTeX{} Project Public License, either version 1.3
of this license or (at your option) any later version.
The latest version of this license is in
  \url{http://www.latex-project.org/lppl.txt}
and version 1.3 or later is part of all distributions of \LaTeX{}
version 2005/12/01 or later.

This work has the LPPL maintenance status `maintained'.

The Current Maintainer of this work is Niklas Beisert.

This work consists of the files |README.txt|, |childdoc.ins| and |childdoc.dtx|
as well as the derived files |childdoc.def|, |cdocsamp.tex|
with |cdocsch1.tex|, |cdocsch2.tex|, |cdocspt3.tex|, |cdocspt4.tex|,
|cdocsdrf.tex|, |cdocsfn1.tex|, |cdocsfn2.tex|
as well as |childdoc.pdf|.

%%%%%%%%%%%%%%%%%%%%%%%%%%%%%%%%%%%%%%%%%%%%%%%%%%%%%%%%%%%%%%%%%%%%%%%%%%%%%%%%
\subsection{Files and Installation}

The package consists of the files:
%
\begin{center}
\begin{tabular}{ll}
    |README.txt|   & readme file \\
    |childdoc.ins| & installation file \\
    |childdoc.dtx| & source file \\
    |childdoc.def| & definition file \\
    |cdocsamp.tex| & sample main file \\
    |cdocsch1.tex| & sample include file \\
    |cdocsch2.tex| & sample include file \\
    |cdocspt3.tex| & sample part file \\
    |cdocspt4.tex| & sample part file \\
    |cdocsdrf.tex| & sample redirection file \\
    |cdocsfn1.tex| & sample redirection file \\
    |cdocsfn2.tex| & sample redirection file \\
    |childdoc.pdf| & manual
\end{tabular}
\end{center}
%
The distribution consists of the files
|README.txt|, |childdoc.ins| and |childdoc.dtx|.
%
\begin{itemize}
\item
Run (pdf)\LaTeX{} on |childdoc.dtx|
to compile the manual |childdoc.pdf| (this file).
\item
Run \LaTeX{} on |childdoc.ins| to create the definitions file |childdoc.def|
and the sample |cdocsamp.tex| with include files
|cdocsch1.tex|, |cdocsch2.tex|, |cdocspt3.tex|, |cdocspt4.tex|,
|cdocsdrf.tex|, |cdocsfn1.tex|, |cdocsfn2.tex|.
Then copy the file |childdoc.def| to an appropriate directory of your \LaTeX{}
distribution, e.g.\ \textit{texmf-root}|/tex/latex/childdoc|.
\end{itemize}

%%%%%%%%%%%%%%%%%%%%%%%%%%%%%%%%%%%%%%%%%%%%%%%%%%%%%%%%%%%%%%%%%%%%%%%%%%%%%%%%
\subsection{Related CTAN Packages}

There are several other packages which offer a similar functionality:
%
\begin{itemize}
\item
The packages
\href{http://ctan.org/pkg/docmute}{\textsf{docmute}},
\href{http://ctan.org/pkg/includex}{\textsf{includex}} and
\href{http://ctan.org/pkg/standalone}{\textsf{standalone}}
provide commands to include only the document body of
a child file thus allowing both files to be compiled individually.
\item
The packages \href{http://ctan.org/pkg/subdocs}{\textsf{subdocs}}
and \href{http://ctan.org/pkg/subfiles}{\textsf{subfiles}}
provide structures in which the main and child documents can be
encapsulated and allowing them to be compiled individually.
The inclusion mechanism is different from the conventional |\include|.
\item
The package \href{http://ctan.org/pkg/combine}{\textsf{combine}}
is an elaborate solution to combine several documents into one.
\end{itemize}
%
See also the CTAN topic \href{http://ctan.org/topic/subdocs}{\textsf{subdocs}}
for further related packages.
The present package differs from the above solutions in that
a document structure constructed with the conventional |\include| mechanism
just needs two extra commands at the top of every file
such that all constituent files can be compiled individually.

%%%%%%%%%%%%%%%%%%%%%%%%%%%%%%%%%%%%%%%%%%%%%%%%%%%%%%%%%%%%%%%%%%%%%%%%%%%%%%%%
%\subsection{Feature Suggestions}
%
%The following is a list of features which may be useful for future
%versions of this package:
%%
%\begin{itemize}
%\item
%\ldots
%\end{itemize}

%%%%%%%%%%%%%%%%%%%%%%%%%%%%%%%%%%%%%%%%%%%%%%%%%%%%%%%%%%%%%%%%%%%%%%%%%%%%%%%%
\subsection{Revision History}

%%%%%%%%%%%%%%%%%%%%%%%%%%%%%%%%%%%%%%%%
\paragraph{v2.0:} 2018/12/30

\begin{itemize}
\item
immediate forward processing
\item
added |\childdocby| mechanism
\item
manual restructured
\end{itemize}

%%%%%%%%%%%%%%%%%%%%%%%%%%%%%%%%%%%%%%%%
\paragraph{v1.6:} 2018/01/17

\begin{itemize}
\item
application for development of include files
\item
corrections to manual
\end{itemize}

%%%%%%%%%%%%%%%%%%%%%%%%%%%%%%%%%%%%%%%%
\paragraph{v1.5:} 2017/05/21

\begin{itemize}
\item
more complete structuring introduced
\item
|\childdocof| introduced
\item
|\childdoc| renamed to |\childdocmain|
\item
|\childredirect| renamed to |\childdocforward| and |\childdocforwardprefix|
and functionality expanded
\end{itemize}

%%%%%%%%%%%%%%%%%%%%%%%%%%%%%%%%%%%%%%%%
\paragraph{v1.0:} 2017/04/27

\begin{itemize}
\item
manual and install package
\item
first version published on CTAN
\end{itemize}

%%%%%%%%%%%%%%%%%%%%%%%%%%%%%%%%%%%%%%%%
\paragraph{v0.6:} 2017/04/26

\begin{itemize}
\item
redirection mechanism added
\end{itemize}

%%%%%%%%%%%%%%%%%%%%%%%%%%%%%%%%%%%%%%%%
\paragraph{v0.5:} 2017/04/26

\begin{itemize}
\item
functionality in definition file
\end{itemize}


%%%%%%%%%%%%%%%%%%%%%%%%%%%%%%%%%%%%%%%%%%%%%%%%%%%%%%%%%%%%%%%%%%%%%%%%%%%%%%%%
%%%%%%%%%%%%%%%%%%%%%%%%%%%%%%%%%%%%%%%%%%%%%%%%%%%%%%%%%%%%%%%%%%%%%%%%%%%%%%%%
%%%%%%%%%%%%%%%%%%%%%%%%%%%%%%%%%%%%%%%%%%%%%%%%%%%%%%%%%%%%%%%%%%%%%%%%%%%%%%%%
\appendix

\settowidth\MacroIndent{\rmfamily\scriptsize 000\ }

 \DocInput{childdoc.dtx}

\end{document}
%</driver>
% \fi
%
% %%%%%%%%%%%%%%%%%%%%%%%%%%%%%%%%%%%%%%%%%%%%%%%%%%%%%%%%%%%%%%%%%%%%%%%%%%%%%%
% %%%%%%%%%%%%%%%%%%%%%%%%%%%%%%%%%%%%%%%%%%%%%%%%%%%%%%%%%%%%%%%%%%%%%%%%%%%%%%
% \section{Sample}
%\iffalse
%<*samplemain>
%\fi
%
% The following presents a sample document
% with two chapters, two parts, a title page,
% a compile flag as well as three forwarding files to set the flag.
% It consists of eight |.tex| files:
% \begin{center}
% \begin{tabular}{ll}
% |cdocsamp.tex|&main file\\
% |cdocsch1.tex|&include file for chapter 1\\
% |cdocsch2.tex|&include file for chapter 2\\
% |cdocspt3.tex|&include file for part 3\\
% |cdocspt4.tex|&include file for part 4\\
% |cdocsdrf.tex|&forwarding file for main file in draft mode\\
% |cdocsfi1.tex|&forwarding file for final version of chapter 1\\
% |cdocsfi2.tex|&forwarding file for final version of chapter 2\\
% \end{tabular}
% \end{center}
% Each of the eight files can be compiled directly by the \LaTeX{} compiler.
%
% %%%%%%%%%%%%%%%%%%%%%%%%%%%%%%%%%%%%%%
% \paragraph{Main File.}
%
% The main file is called |cdocsamp.tex|.
%
% Load the \textsf{childdoc} definitions and
% declare the filename for the main document:
%    \begin{macrocode}
\input{childdoc.def}
\childdocmain{}
%    \end{macrocode}

% Optional override for |\version| flag:
%    \begin{macrocode}
%%\ifchilddoc\else\providecommand{\version}{draft}\fi
%    \end{macrocode}

% Define the default values for the |\version| flag
% (|final| for the main file and |draft| for childs):
%    \begin{macrocode}
\ifchilddoc
\providecommand{\version}{draft}
\else
\providecommand{\version}{final}
\fi
%    \end{macrocode}

% Load the standard document class:
%    \begin{macrocode}
\documentclass[12pt]{article}
%    \end{macrocode}

% Start the document body:
%    \begin{macrocode}
\begin{document}
%    \end{macrocode}

% Declare a title page.
% Print title, part of document being processed and version flag:
%    \begin{macrocode}
\addtocounter{page}{-1}
\begin{center}
{\LARGE\bfseries{}childdoc example\par}
\vspace{1cm}
\ifchilddoc
\ifchilddocmanual part\else chapter\fi:
`\childdocname' of `\childdocjob'\par
\else
main document: `\childdocjob'\par
\fi
version: \version\par
\end{center}
\newpage
%    \end{macrocode}

% Manually include selected file,
% otherwise process as usual:
%    \begin{macrocode}
\ifchilddocmanual
\section*{part `\childdocname'}
\input{\childdocname}
\else
%    \end{macrocode}

% Include the two chapters:
%    \begin{macrocode}
\include{cdocsch1}
\include{cdocsch2}
%    \end{macrocode}

% Include the two parts unless only chapters should be displayed:
%    \begin{macrocode}
\ifchilddoc\else
\section{part three}
\input{cdocspt3}
\section{part four}
\input{cdocspt4}
\fi
%    \end{macrocode}

% Process as usual until here:
%    \begin{macrocode}
\fi
%    \end{macrocode}

% End of document body:
%    \begin{macrocode}
\end{document}
%    \end{macrocode}
%\iffalse
%</samplemain>
%\fi
%
% %%%%%%%%%%%%%%%%%%%%%%%%%%%%%%%%%%%%%%
% \paragraph{Chapter Include Files.}
%
% The include files are called |cdocsch1.tex| and |cdocsch2.tex|.
%
%\iffalse
%<*samplechap1|samplechap2>
%\fi

% Optional override for |\version| flag:
%    \begin{macrocode}
%%\providecommand{\version}{final}
%    \end{macrocode}

% Include the main document:
%    \begin{macrocode}
\input{childdoc.def}
\childdocof{cdocsamp}
%    \end{macrocode}

%\iffalse
%</samplechap1|samplechap2>
%\fi
%
%\iffalse
%<*samplechap1>
%\fi
% Some text for chapter 1:
%    \begin{macrocode}
\section{one}
some text in chapter one
%    \end{macrocode}

%\iffalse
%</samplechap1>
%\fi
% Some text for chapter 2:
%\iffalse
%<*samplechap2>
%\fi
%    \begin{macrocode}
\section{two}
more text in chapter two
%    \end{macrocode}

%\iffalse
%</samplechap2>
%\fi
%
% %%%%%%%%%%%%%%%%%%%%%%%%%%%%%%%%%%%%%%
% \paragraph{Part Include Files.}
%
% The include files are called |cdocspt3.tex| and |cdocspt4.tex|.
%
%\iffalse
%<*samplepart3|samplepart4>
%\fi

% Optional override for |\version| flag:
%    \begin{macrocode}
%%\providecommand{\version}{final}
%    \end{macrocode}

% Include the main document:
%    \begin{macrocode}
\input{childdoc.def}
\childdocby{cdocsamp}
%    \end{macrocode}

%\iffalse
%</samplepart3|samplepart4>
%\fi
%
%\iffalse
%<*samplepart3>
%\fi
% Some text for part 3:
%    \begin{macrocode}
some text in part three
%    \end{macrocode}

%\iffalse
%</samplepart3>
%\fi
% Some text for part 4:
%\iffalse
%<*samplepart4>
%\fi
%    \begin{macrocode}
more text in part four
%    \end{macrocode}

%\iffalse
%</samplepart4>
%\fi
%
% %%%%%%%%%%%%%%%%%%%%%%%%%%%%%%%%%%%%%%
% \paragraph{Forwarding for a Complete Draft.}
%
% The following forwarding file |cdocsdrf.tex|
% compiles the main document in draft mode:
%\iffalse
%<*sampledraft>
%\fi
%    \begin{macrocode}
\def\version{draft}
\input{childdoc.def}
\childdocforward{cdocsamp}
%    \end{macrocode}

%\iffalse
%</sampledraft>
%\fi
%
% %%%%%%%%%%%%%%%%%%%%%%%%%%%%%%%%%%%%%%
% \paragraph{Forwarding for Final Version of the Chapters.}
%
% The following forwarding files |cdocsfn1.tex| and |cdocsfn2.tex|
% (with identical content)
% compile the final versions of the child documents
% |cdocsch1.tex| and |cdocsch2.tex|, respectively:
%\iffalse
%<*samplefinal>
%\fi
%    \begin{macrocode}
\def\version{final}
\input{childdoc.def}
\childdocforwardprefix[cdocsamp]{cdocsfn}{cdocsch}
%    \end{macrocode}

%\iffalse
%</samplefinal>
%\fi
%
% %%%%%%%%%%%%%%%%%%%%%%%%%%%%%%%%%%%%%%
% \paragraph{Command Line Processing.}
%
% The following three command lines generate the output files
% |cdocscld|, |cdocscl1| and |cdocscl2|
% which should be identical to
% |cdocsdrf|, |cdocsch1| and |cdocsfn2|, respectively:
% \begin{center}
% \begin{tabular}{l}
% |latex -jobname cdocscld \|\\
% |  "\def\version{draft}\input{childdoc.def}\childdocforward{cdocsamp}"|\\
% |latex -jobname cdocscl1 \|\\
% |  "\input{childdoc.def}\childdocforward[cdocsamp]{cdocsch1}"|\\
% |latex -jobname cdocscl2 \|\\
% |  "\def\version{final}\input{childdoc.def}\childdocforward{cdocsch2}"|
% \end{tabular}
% \end{center}
% Note that the trailing backslash on each first line
% merely continues the input to the second line
% (for convenient cut ant paste).
% Furthermore, the command |latex| can be replaced by any
% of its alternative versions such as |pdflatex|.
%
% %%%%%%%%%%%%%%%%%%%%%%%%%%%%%%%%%%%%%%%%%%%%%%%%%%%%%%%%%%%%%%%%%%%%%%%%%%%%%%
% %%%%%%%%%%%%%%%%%%%%%%%%%%%%%%%%%%%%%%%%%%%%%%%%%%%%%%%%%%%%%%%%%%%%%%%%%%%%%%
% \section{Implementation}
%\iffalse
%<*package>
%\fi
%
% This section describes the definitions file |childdoc.def|.

% The definitions cannot be loaded using |\usepackage| or |\RequirePackage|
% which has a mechanism to prevent loading a style file more than once.
% When loading the definitions by means of |\input|
% multiple instances have to be prevented manually:
%\iffalse
%This code needs to be before the `\ProvidesFile' directive
%which is defined at the beginning of this file.
%Therefore it is also placed there and commented out here.
%</package>
%<*discard>
%\fi
%    \begin{macrocode}
\ifdefined\childdocmain\endinput\fi
%    \end{macrocode}
%\iffalse
%</discard>
%<*package>
%\fi
%
% \macro{\ifchilddoc}
% \macro{\ifchilddocmanual}
% The conditional |\ifchilddoc| tells whether a
% child (true) or main (false) document is being compiled.
% The conditional |\ifchilddocmanual| tells whether
% the |\includeonly| mechanism is used (false) or
% the selection of child files must be performed manually (true).
% The definitions initialise to false:
%    \begin{macrocode}
\newif\ifchilddoc
\newif\ifchilddocmanual
%    \end{macrocode}

% \macro{\childdocname}
% \macro{\childdocjob}
% The macro |\childdocname| stores the name of the main document
% to be compiled. The macro |\childdocjob| stores the name of
% the document on which the \LaTeX{} compiler was originally invoked.
% The content of |\jobname| cannot be compared
% to filenames specified in the source due to different catcodes.
% The following code rescans |\jobname|, stores the result
% in |\childdocname| and saves a copy in |\childdocjob|:
%    \begin{macrocode}
\edef\childdocname{\scantokens\expandafter{\jobname\noexpand}}
\let\childdocjob\childdocname
%    \end{macrocode}

% \macro{\childdocdisable}
% The macro |\childdocdisable| prevents the main file
% from being processed more than once.
% At this stage, the main document command |\childdocmain|
% is assumed to be called once again where it should do nothing.
% Any subsequent call to it should prevent
% a secondary processing of the main document
% It overwrites the forwarding commands
% |\childdocof| and |\childdocforward|
% with empty macros to prevent further inclusions of the main document:
%    \begin{macrocode}
\newcommand{\childdocdisable}
{
  \renewcommand{\childdocmain}[1]{\renewcommand{\childdocmain}[1]{\endinput}}
  \renewcommand{\childdocof}[1]{}
  \renewcommand{\childdocby}[2][]{}
  \renewcommand{\childdocforward}[2][]{}
  \renewcommand{\childdocdisable}{}
}
%    \end{macrocode}

% \macro{\childdocmain}
% The macro |\childdocmain| is to be called at the top of the main file
% with nothing or the main filename (without extension) as argument.
% First, it breaks loops.
% If the argument is not empty and does not match |\childdocname|
% (which is set by the first inclusion of |childdoc.def|),
% |\ifchilddoc| is set to true, |\includeonly| is applied to the child file
% and |\jobname| is set to the main file
% (for proper handling of |.aux| files):
%    \begin{macrocode}
\newcommand{\childdocmain}[1]
{
  \childdocdisable\childdocmain{}
  \if?#1?\else
    \begingroup
      \def\childdoctmp{#1}
      \ifx\childdoctmp\childdocname
        \def\childdoctmp{}
      \else
        \def\childdoctmp
        {
          \childdoctrue
          \includeonly{\childdocname}
          \def\childdocjob{#1}
          \def\jobname{#1}
        }
      \fi
      \expandafter
    \endgroup
    \childdoctmp
  \fi
}
%    \end{macrocode}

% \macro{\childdocof}
% The command |\childdocof| redirects
% compilation to the main file |#1|.
%    \begin{macrocode}
\newcommand{\childdocof}[1]
{
  \childdocdisable
  \childdoctrue
  \includeonly{\childdocname}
  \def\jobname{#1}
  \def\childdocjob{#1}
  \input{#1}
}
%    \end{macrocode}

% \macro{\childdocby}
% The command |\childdocby| ....
%    \begin{macrocode}
\newcommand{\childdocby}[2][]
{
  \childdocdisable
  \childdoctrue
  \childdocmanualtrue
  \if?#1?\else
    \def\jobname{#2}
  \fi
  \def\childdocjob{#2}
  \input{#2}
  \endinput
}
%    \end{macrocode}

% \macro{\childdocforward}
% The command |\childdocforward| redirects
% compilation to the main file or
% (if the optional argument is given) a child file.
% Parameters are set as if the main file
% or a child file starting with |\childdocof| was compiled.
% Then compilation is handed over to the main file:
%    \begin{macrocode}
\newcommand{\childdocforward}[2][]
{
  \begingroup
    \if?#1?
      \def\childdoctmp
      {
        \def\childdocname{#2}
        \def\childdocjob{#2}
        \def\jobname{#2}
        \input{#2}
        \endinput
      }
    \else
      \def\childdoctmp
      {
        \childdocdisable
        \def\childdocname{#2}
        \childdoctrue
        \includeonly{#2}
        \def\childdocjob{#1}
        \def\jobname{#1}
        \input{#1}
        \endinput
      }
    \fi
    \expandafter
  \endgroup
  \childdoctmp
}
%    \end{macrocode}

% \macro{\childdocforwardprefix}
% The command |\childdocforwardprefix| redirects
% compilation to the main or a child file by means of a pattern.
% The prefix |#1| in the current filename is replaced by |#2|
% and the suffix of the current filename is kept
% (it is assumed that the filename does not contain the substring `|~~~|'
% which is used as a delimiter).
% Compilation is handed over to the new file by |\childdocforward|:
%    \begin{macrocode}
\newcommand{\childdocforwardprefix}[3][]
{
  \begingroup
    \def\childdocextract #2##1~~~{\def\childdoctmp{\childdocforward[#1]{#3##1}}}
    \expandafter\childdocextract\childdocname~~~
    \expandafter
  \endgroup
  \childdoctmp
}
%    \end{macrocode}

% \macro{\childdoc}
% The deprecated macro |\childdoc| is a legacy version of |\childdocmain|:
%    \begin{macrocode}
\newcommand{\childdoc}{\childdocmain}
%    \end{macrocode}

% \macro{\childdocredirect}
% The deprecated macro |\childdocredirect| is a legacy version
% of |\childdocforward| and |\childdocforwardprefix|:
%    \begin{macrocode}
\newcommand{\childdocredirect}[2][]
{
  \begingroup
    \if?#1?
      \def\childdoctmp{\childdocforward{#2}}
    \else
      \def\childdoctmp{\childdocforwardprefix{#1}{#2}}
    \fi
    \expandafter
  \endgroup
  \childdoctmp
}
%    \end{macrocode}

%\iffalse
%</package>
%\fi
%
\endinput
|\\
|\childdocof{|\textit{main}|}|\\
\end{tabular}
\end{center}
at the top of every child file \textit{child}
which is included by |\include{|\textit{child}|}|
from within the main file
(or at least for those files to be compiled individually).
The argument \textit{main} must be the filename of the main file.

There are a couple of
considerations in setting up the main and child documents:

%%%%%%%%%%%%%%%%%%%%%%%%%%%%%%%%%%%%%%%%
\paragraph{Restrictions.}

Please note the following restrictions:
\begin{itemize}
\item
|\childdocmain| must be called with one argument \textit{main}
to ensure compatibility with earlier version of the package.
It must either be empty (|\childdocmain{}|)
or precisely match the filename of the main file in which it is specified.
See \secref{sec:detection} for further information.
\item
The filename \textit{main} must be specified without the |.tex| extension.
\item
The filename \textit{main} is case sensitive
(even in case-insensitive file systems)
due to internal string comparison.
\item
The argument \textit{main} should be fully expanded, it cannot be a macro.
\item
Subdirectories and special characters should be avoided in filenames.
\item
The command |\childdocmain{|\textit{main}|}| must be followed by a whitespace.
It should not be followed immediately by another command
or by a comment mark `|%|'.
This is because the \TeX{} parser reads the token immediately following
the argument of |\childdocmain| and puts it
at the beginning of every child section;
however, a white\-space is ignored.
\end{itemize}

%%%%%%%%%%%%%%%%%%%%%%%%%%%%%%%%%%%%%%%%
\paragraph{Content of Main File.}

It is advisable to place all content in the child files included by |\include|.
Any output contained in the main file will appear in all child documents
unless suppressed manually;
it cannot be suppressed automatically by the |\includeonly| directive
and thus should normally be avoided.
A method to include some content in the main file
by means of conditional processing is described in \secref{sec:conditional}.

%%%%%%%%%%%%%%%%%%%%%%%%%%%%%%%%%%%%%%%%
\paragraph{Page Numbering.}

When only a part of the document is compiled,
the appropriate numbering of pages
(as well as other status parameters)
is determined from the |.aux| files.
The latter contain information from previous passes.
However this information needs to propagate through
all intermediate child documents.
Therefore the page numbering in child documents may well
be inconsistent until the complete document is compiled at least once.

A useful (if unconventional) way to always ensure a consistent
page numbering is to restart the numbering in each child document
and denote the pages by `\textit{child}|.|\textit{page}'
where \textit{child} represents the chapter/section number of the child file.
This can be achieved by the command
|\numberwithin{page}{|\textit{child}|}|
of the \textsf{amsmath} package
where \textit{child} can be |chapter| or |section|
depending on the chosen structuring.
Alternatively, one can modify the macro |\thepage| appropriately
and reset the counter |page| at the start of each child file.

%%%%%%%%%%%%%%%%%%%%%%%%%%%%%%%%%%%%%%%%%%%%%%%%%%%%%%%%%%%%%%%%%%%%%%%%%%%%%%%%
\subsection{Conditional Processing}
\label{sec:conditional}

The package provides a mechanism to compile different versions
of a document. To customise the versions further some conditional processing
can come in handy to distinguish which version is being compiled.
The package provides two macros to describe the compilation context:

%%%%%%%%%%%%%%%%%%%%%%%%%%%%%%%%%%%%%%%%
\DescribeMacro{\ifchilddoc}
The conditional |\ifchilddoc| distinguishes between the compilation of
child documents and the main document:
%
\begin{center}
|\ifchilddoc |\textit{child-code}| |[|\||else |\textit{main-code}]| \||fi|
\end{center}

%%%%%%%%%%%%%%%%%%%%%%%%%%%%%%%%%%%%%%%%
\DescribeMacro{\childdocname}
\DescribeMacro{\childdocjob}
The macro |\childdocname| contains the filename (without extension)
of the main or child file being processed.
Note that |\childdocjob| will always contain the name of the main file.

%%%%%%%%%%%%%%%%%%%%%%%%%%%%%%%%%%%%%%%%
\paragraph{Title Page.}

Conditional processing can be used to include a title or banner page
in the main document when proper precautions are taken.
Importantly, the code in the main file should ensure that the page counter
(as well as other status parameters which are stored in the |.aux| files)
takes the same value after the conditional processing.
Otherwise the page numbers may take divergent values
depending on which part is compiled.

For example, a title page could be declared by:
%
\begin{center}
\begin{tabular}{l}
|\ifchilddoc\||else|\\
|\addtocounter{page}{-1}|\\
\textit{code for title page}\\
|\newpage|\\
|\||fi|
\end{tabular}
\end{center}
%
A banner page for the child documents can be generated by:
%
\begin{center}
\begin{tabular}{l}
|\ifchilddoc|\\
|\addtocounter{page}{-1}|\\
\textit{code for banner page}\\
|\newpage|\\
|\||fi|
\end{tabular}
\end{center}
%
Here one could write a message such as:
\begin{center}
|This is the part \childdocname{} of \childdocjob{}.|
\end{center}

%%%%%%%%%%%%%%%%%%%%%%%%%%%%%%%%%%%%%%%%%%%%%%%%%%%%%%%%%%%%%%%%%%%%%%%%%%%%%%%%
\subsection{Flags}
\label{sec:flags}

The package makes it easy to generate different versions
of the main or child documents.
To this end compilation flags can be defined
and assigned different default values.
They will be particularly useful in conjunction
with the forwarding mechanism described in \secref{sec:forward}.

For example, it may be useful to have a flag |\version|
which can be set to |draft| or |final|.
The document source will contain some conditional code
depending on the value of |\version|.
Suppose further, the flag should default to |final| for the main file
and to |draft| for child files
which is a natural assignment for editing the document.
This is achieved by placing the following code
in the preamble of the main document
(below the |\childdocmain| directive):
%
\begin{center}
\begin{tabular}{l}
|\ifchilddoc|\\
|\providecommand{\version}{draft}|\\
|\||else|\\
|\providecommand{\version}{final}|\\
|\||fi|
\end{tabular}
\end{center}
%
The definition by |\providecommand| makes sure
that previous definitions are not overwritten.
Further statements |\providecommand{\version}{...}|
can thus be added before the above code to override it.

For the main file, one might add a line
(between |\childdocmain| and the above block)
%
\begin{center}
|%\ifchilddoc\||else\providecommand{\version}{draft}\||fi|
\end{center}
%
which can be uncommented to produce a draft version.
Likewise one can add a line to the very top of a child file
(above the |\childdocof{|\textit{main}|}| directive)
%
\begin{center}
|%\providecommand{\version}{final}|
\end{center}
%
which can be uncommented to produce the final version of this child document.

%%%%%%%%%%%%%%%%%%%%%%%%%%%%%%%%%%%%%%%%%%%%%%%%%%%%%%%%%%%%%%%%%%%%%%%%%%%%%%%%
\subsection{Forwarding}
\label{sec:forward}

Different versions of the main or child documents
using compilation flags as described in \secref{sec:flags}
can be (permanently) stored in different files
for convenient compilation, viewing and distribution.
To this end, the package defines a command
to pass on compilation to a different file:

%%%%%%%%%%%%%%%%%%%%%%%%%%%%%%%%%%%%%%%%
\DescribeMacro{\childdocforward}
The command |\childdocforward| redirects processing to
another source file:
%
\begin{center}
\begin{tabular}{l}
|% \iffalse
%
% childdoc.dtx Copyright (C) 2017-2018 Niklas Beisert
%
% This work may be distributed and/or modified under the
% conditions of the LaTeX Project Public License, either version 1.3
% of this license or (at your option) any later version.
% The latest version of this license is in
%   http://www.latex-project.org/lppl.txt
% and version 1.3 or later is part of all distributions of LaTeX
% version 2005/12/01 or later.
%
% This work has the LPPL maintenance status `maintained'.
%
% The Current Maintainer of this work is Niklas Beisert.
%
% This work consists of the files childdoc.dtx and childdoc.ins
% and the derived files childdoc.def and cdocsamp.tex with
% cdocsch1.tex, cdocsch2.tex, cdocsdrf.tex, cdocsfn1.tex, cdocsfn2.tex.
%
%<package>\ifdefined\childdocmain\endinput\fi
%<package>\ProvidesFile{childdoc.def}[2018/12/30 v2.0 child document driver]
%<samplemain>\ProvidesFile{cdocsamp.tex}[2018/12/30 v2.0 sample for childdoc]
%<*driver>
%\ProvidesFile{childdoc.drv}[2018/12/30 v2.0 childdoc reference manual file]
\PassOptionsToClass{10pt,a4paper}{article}
\documentclass{ltxdoc}

\usepackage[margin=35mm]{geometry}
\usepackage{hyperref}
\usepackage{hyperxmp}
\usepackage[usenames]{color}

\hypersetup{colorlinks=true}
\hypersetup{pdfstartview=FitH}
\hypersetup{pdfpagemode=UseNone}
\hypersetup{pdfsource={}}
\hypersetup{pdflang={en-UK}}
\hypersetup{pdfcopyright={Copyright 2017-2018 Niklas Beisert.
  This work may be distributed and/or modified under the
  conditions of the LaTeX Project Public License, either version 1.3
  of this license or (at your option) any later version.}}
\hypersetup{pdflicenseurl={http://www.latex-project.org/lppl.txt}}
\hypersetup{pdfcontactaddress={ETH Zurich, ITP, HIT K,
  Wolfgang-Pauli-Strasse 27}}
\hypersetup{pdfcontactpostcode={8093}}
\hypersetup{pdfcontactcity={Zurich}}
\hypersetup{pdfcontactcountry={Switzerland}}
\hypersetup{pdfcontactemail={nbeisert@itp.phys.ethz.ch}}
\hypersetup{pdfcontacturl={http://people.phys.ethz.ch/\xmptilde nbeisert/}}

\newcommand{\secref}[1]{\hyperref[#1]{section \ref*{#1}}}

\parskip1ex
\parindent0pt
\let\olditemize\itemize
\def\itemize{\olditemize\parskip0pt}

\begin{document}

\title{The \textsf{childdoc} Package}
\hypersetup{pdftitle={The childdoc Package}}
\author{Niklas Beisert\\[2ex]
  Institut f\"ur Theoretische Physik\\
  Eidgen\"ossische Technische Hochschule Z\"urich\\
  Wolfgang-Pauli-Strasse 27, 8093 Z\"urich, Switzerland\\[1ex]
  \href{mailto:nbeisert@itp.phys.ethz.ch}
  {\texttt{nbeisert@itp.phys.ethz.ch}}}
\hypersetup{pdfauthor={Niklas Beisert}}
\hypersetup{pdfsubject={Manual for the LaTeX2e Package childdoc}}
\date{30 December 2018, \textsf{v2.0}}
\maketitle

\begin{abstract}\noindent
\textsf{childdoc} is a \LaTeXe{} package
that enables the direct compilation
of document sections included by |\include|
to individual files.
\end{abstract}

\begingroup
\parskip0ex
\tableofcontents
\endgroup

%%%%%%%%%%%%%%%%%%%%%%%%%%%%%%%%%%%%%%%%%%%%%%%%%%%%%%%%%%%%%%%%%%%%%%%%%%%%%%%%
%%%%%%%%%%%%%%%%%%%%%%%%%%%%%%%%%%%%%%%%%%%%%%%%%%%%%%%%%%%%%%%%%%%%%%%%%%%%%%%%
\section{Introduction}

\LaTeX{} provides a mechanism to structure a large document (such as a book)
into a main file and several child files (containing the chapters)
using the |\include| command.
This mechanism is beneficial for documents
which span hundreds of pages in order to
make the source file(s) more manageable.
Moreover, compilation can be restricted to
selected child files by means of the |\includeonly| command.
The latter feature can be used to reduce the compilation time while editing
(this was significantly more useful in the earlier days of \LaTeX{})
or to generate a smaller document which is easier to navigate.
Another application of |\includeonly| is to generate
documents consisting of selected parts of the complete document.

However, there are a few drawbacks of the plain |\include| mechanism:
\begin{itemize}
\item
The child files cannot be compiled on their own,
they can only be compiled via the main file.
A naive editing environment
(such as a text editor with an option
to have the current file processed by \LaTeX)
may require one to switch to the main file before compiling;
attempting to compile the child file produces errors.
\item
The main file must be modified (each time)
to adjust the |\includeonly| command
to the present needs. This easily leaves the main file in a messy state.
\item
The generated document will always carry the filename
of the main document. This is inconvenient if
several child files are to be compiled and
to be kept for distribution.
\end{itemize}

The present package provides a simple interface
to make child files individually compilable by \LaTeX{}.
Compiling a child file then has the same effect as compiling
the main file with an |\includeonly| command
to select the appropriate child.
Moreover the generated document will carry the name of the child
rather than the main file.
This resolves all three above issues.

This feature is meant to make the editing of books,
thesis documents and lecture notes somewhat more convenient.
However, the package can also be used efficiently for
composing a series of documents (such as exercise sheets)
which are typically distributed individually.
It then assists the author in generating the individual documents
(potentially in different versions)
as well as a document containing the collected series.
Another application is in developing style files
or other kinds of included material
where compilation of the style file could redirect
to a sample or test file.

%%%%%%%%%%%%%%%%%%%%%%%%%%%%%%%%%%%%%%%%%%%%%%%%%%%%%%%%%%%%%%%%%%%%%%%%%%%%%%%%
%%%%%%%%%%%%%%%%%%%%%%%%%%%%%%%%%%%%%%%%%%%%%%%%%%%%%%%%%%%%%%%%%%%%%%%%%%%%%%%%
\section{Usage}

First of all, the package \textsf{childdoc} is \emph{not} a standard
\LaTeXe{} |.sty| style file! Therefore it needs to be invoked in
a non-standard way.

%%%%%%%%%%%%%%%%%%%%%%%%%%%%%%%%%%%%%%%%%%%%%%%%%%%%%%%%%%%%%%%%%%%%%%%%%%%%%%%%
\subsection{Included Files}
\label{sec:include}

%%%%%%%%%%%%%%%%%%%%%%%%%%%%%%%%%%%%%%%%
\DescribeMacro{\childdocmain}
To use the package, add the commands
\begin{center}
\begin{tabular}{l}
|\input{childdoc.def}|\\
|\childdocmain{}|\\
\end{tabular}
\end{center}
at the very top of the main \LaTeX{} file,
in particular \emph{before} the |\documentclass| statement!
The argument of |\childdocmain| should be left empty
(but it must be present).

%%%%%%%%%%%%%%%%%%%%%%%%%%%%%%%%%%%%%%%%
\DescribeMacro{\childdocof}
Furthermore, add the commands
\begin{center}
\begin{tabular}{l}
|\input{childdoc.def}|\\
|\childdocof{|\textit{main}|}|\\
\end{tabular}
\end{center}
at the top of every child file \textit{child}
which is included by |\include{|\textit{child}|}|
from within the main file
(or at least for those files to be compiled individually).
The argument \textit{main} must be the filename of the main file.

There are a couple of
considerations in setting up the main and child documents:

%%%%%%%%%%%%%%%%%%%%%%%%%%%%%%%%%%%%%%%%
\paragraph{Restrictions.}

Please note the following restrictions:
\begin{itemize}
\item
|\childdocmain| must be called with one argument \textit{main}
to ensure compatibility with earlier version of the package.
It must either be empty (|\childdocmain{}|)
or precisely match the filename of the main file in which it is specified.
See \secref{sec:detection} for further information.
\item
The filename \textit{main} must be specified without the |.tex| extension.
\item
The filename \textit{main} is case sensitive
(even in case-insensitive file systems)
due to internal string comparison.
\item
The argument \textit{main} should be fully expanded, it cannot be a macro.
\item
Subdirectories and special characters should be avoided in filenames.
\item
The command |\childdocmain{|\textit{main}|}| must be followed by a whitespace.
It should not be followed immediately by another command
or by a comment mark `|%|'.
This is because the \TeX{} parser reads the token immediately following
the argument of |\childdocmain| and puts it
at the beginning of every child section;
however, a white\-space is ignored.
\end{itemize}

%%%%%%%%%%%%%%%%%%%%%%%%%%%%%%%%%%%%%%%%
\paragraph{Content of Main File.}

It is advisable to place all content in the child files included by |\include|.
Any output contained in the main file will appear in all child documents
unless suppressed manually;
it cannot be suppressed automatically by the |\includeonly| directive
and thus should normally be avoided.
A method to include some content in the main file
by means of conditional processing is described in \secref{sec:conditional}.

%%%%%%%%%%%%%%%%%%%%%%%%%%%%%%%%%%%%%%%%
\paragraph{Page Numbering.}

When only a part of the document is compiled,
the appropriate numbering of pages
(as well as other status parameters)
is determined from the |.aux| files.
The latter contain information from previous passes.
However this information needs to propagate through
all intermediate child documents.
Therefore the page numbering in child documents may well
be inconsistent until the complete document is compiled at least once.

A useful (if unconventional) way to always ensure a consistent
page numbering is to restart the numbering in each child document
and denote the pages by `\textit{child}|.|\textit{page}'
where \textit{child} represents the chapter/section number of the child file.
This can be achieved by the command
|\numberwithin{page}{|\textit{child}|}|
of the \textsf{amsmath} package
where \textit{child} can be |chapter| or |section|
depending on the chosen structuring.
Alternatively, one can modify the macro |\thepage| appropriately
and reset the counter |page| at the start of each child file.

%%%%%%%%%%%%%%%%%%%%%%%%%%%%%%%%%%%%%%%%%%%%%%%%%%%%%%%%%%%%%%%%%%%%%%%%%%%%%%%%
\subsection{Conditional Processing}
\label{sec:conditional}

The package provides a mechanism to compile different versions
of a document. To customise the versions further some conditional processing
can come in handy to distinguish which version is being compiled.
The package provides two macros to describe the compilation context:

%%%%%%%%%%%%%%%%%%%%%%%%%%%%%%%%%%%%%%%%
\DescribeMacro{\ifchilddoc}
The conditional |\ifchilddoc| distinguishes between the compilation of
child documents and the main document:
%
\begin{center}
|\ifchilddoc |\textit{child-code}| |[|\||else |\textit{main-code}]| \||fi|
\end{center}

%%%%%%%%%%%%%%%%%%%%%%%%%%%%%%%%%%%%%%%%
\DescribeMacro{\childdocname}
\DescribeMacro{\childdocjob}
The macro |\childdocname| contains the filename (without extension)
of the main or child file being processed.
Note that |\childdocjob| will always contain the name of the main file.

%%%%%%%%%%%%%%%%%%%%%%%%%%%%%%%%%%%%%%%%
\paragraph{Title Page.}

Conditional processing can be used to include a title or banner page
in the main document when proper precautions are taken.
Importantly, the code in the main file should ensure that the page counter
(as well as other status parameters which are stored in the |.aux| files)
takes the same value after the conditional processing.
Otherwise the page numbers may take divergent values
depending on which part is compiled.

For example, a title page could be declared by:
%
\begin{center}
\begin{tabular}{l}
|\ifchilddoc\||else|\\
|\addtocounter{page}{-1}|\\
\textit{code for title page}\\
|\newpage|\\
|\||fi|
\end{tabular}
\end{center}
%
A banner page for the child documents can be generated by:
%
\begin{center}
\begin{tabular}{l}
|\ifchilddoc|\\
|\addtocounter{page}{-1}|\\
\textit{code for banner page}\\
|\newpage|\\
|\||fi|
\end{tabular}
\end{center}
%
Here one could write a message such as:
\begin{center}
|This is the part \childdocname{} of \childdocjob{}.|
\end{center}

%%%%%%%%%%%%%%%%%%%%%%%%%%%%%%%%%%%%%%%%%%%%%%%%%%%%%%%%%%%%%%%%%%%%%%%%%%%%%%%%
\subsection{Flags}
\label{sec:flags}

The package makes it easy to generate different versions
of the main or child documents.
To this end compilation flags can be defined
and assigned different default values.
They will be particularly useful in conjunction
with the forwarding mechanism described in \secref{sec:forward}.

For example, it may be useful to have a flag |\version|
which can be set to |draft| or |final|.
The document source will contain some conditional code
depending on the value of |\version|.
Suppose further, the flag should default to |final| for the main file
and to |draft| for child files
which is a natural assignment for editing the document.
This is achieved by placing the following code
in the preamble of the main document
(below the |\childdocmain| directive):
%
\begin{center}
\begin{tabular}{l}
|\ifchilddoc|\\
|\providecommand{\version}{draft}|\\
|\||else|\\
|\providecommand{\version}{final}|\\
|\||fi|
\end{tabular}
\end{center}
%
The definition by |\providecommand| makes sure
that previous definitions are not overwritten.
Further statements |\providecommand{\version}{...}|
can thus be added before the above code to override it.

For the main file, one might add a line
(between |\childdocmain| and the above block)
%
\begin{center}
|%\ifchilddoc\||else\providecommand{\version}{draft}\||fi|
\end{center}
%
which can be uncommented to produce a draft version.
Likewise one can add a line to the very top of a child file
(above the |\childdocof{|\textit{main}|}| directive)
%
\begin{center}
|%\providecommand{\version}{final}|
\end{center}
%
which can be uncommented to produce the final version of this child document.

%%%%%%%%%%%%%%%%%%%%%%%%%%%%%%%%%%%%%%%%%%%%%%%%%%%%%%%%%%%%%%%%%%%%%%%%%%%%%%%%
\subsection{Forwarding}
\label{sec:forward}

Different versions of the main or child documents
using compilation flags as described in \secref{sec:flags}
can be (permanently) stored in different files
for convenient compilation, viewing and distribution.
To this end, the package defines a command
to pass on compilation to a different file:

%%%%%%%%%%%%%%%%%%%%%%%%%%%%%%%%%%%%%%%%
\DescribeMacro{\childdocforward}
The command |\childdocforward| redirects processing to
another source file:
%
\begin{center}
\begin{tabular}{l}
|\input{childdoc.def}|\\
|\childdocforward[|\textit{main}|]{|\textit{dest}|}|\\
\end{tabular}
\end{center}
%
The argument \textit{dest} is the destination file
(without extension).
It should be the main file or one of the child files.
Note that further \textsf{childdoc} directives
such as |\childdocof| and |\childdocforward|
in the indicated file will be processed in this form.
The optional argument \textit{main}
passes on directly to the main file \textit{main}
while pretending to compile the child \textit{dest}.
This form behaves as if \textit{dest}
issues |\childdocof{|\textit{main}|}| right away,
and no further \textsf{childdoc} directives will be processed.

%%%%%%%%%%%%%%%%%%%%%%%%%%%%%%%%%%%%%%%%
\DescribeMacro{\...prefix}
In the alternative form |\childdocforwardprefix|,
%
\begin{center}
\begin{tabular}{l}
|\input{childdoc.def}|\\
|\childdocforwardprefix[|\textit{main}|]{|\textit{prefix}|}{|\textit{dest}|}|
\end{tabular}
\end{center}
%
the destination file is determined by a pattern
depending on the current file:
To make this work, the current file must be called
`{\textit{prefix}\hspace{0.2em}\textit{suffix}}'
with \textit{prefix} matching precisely the argument.
Processing is then passed on to the file
`{\textit{dest}\hspace{0.2em}\textit{suffix}}'.
Surely, the same effect is achieved by
directly specifying the
argument `{\textit{dest}\hspace{0.2em}\textit{suffix}}'
in the first form.
However, that requires to set up a different file
for each child. With the alternative form of the command
all these files can have exactly the same content
which simplifies setting them up and maintaining them.

For example, the following file |draft.tex|
with a compilation flag |\version| as described in \secref{sec:flags}
compiles the main document as a draft:
%
\begin{center}
\begin{tabular}{l}
|\def\version{draft}|\\
|\input{childdoc.def}|\\
|\childdocforward{|\textit{main}|}|
\end{tabular}
\end{center}
%
Likewise, the following files |final|\textit{nn}|.tex|
compile the final version of the child document
|child|\textit{nn}|.tex|:
%
\begin{center}
\begin{tabular}{l}
|\def\version{final}|\\
|\input{childdoc.def}|\\
|\childdocforwardprefix{final}{child}|
\end{tabular}
\end{center}
%

Note that when several versions of a main file and/or of each child file
are to be generated, it may be convenient to set up a |Makefile| or
shell script to automatise the process.

%%%%%%%%%%%%%%%%%%%%%%%%%%%%%%%%%%%%%%%%%%%%%%%%%%%%%%%%%%%%%%%%%%%%%%%%%%%%%%%%
\subsection{Command Line Processing}
\label{sec:commandline}

The effect of redirection files can also be achieved by invoking
the \LaTeX{} compiler with a more elaborate command line.
Most conveniently this should be done as part
of a shell script or a |Makefile|.

When using \textsf{childdoc} in the main file, the following
command lines effectively perform a redirection
(note that depending on the shell being used,
backslashes may have to be doubled: `|\|' $\to$ `|\\|'):
%
\begin{center}
|... -jobname "|\textit{target}|" |\\|"|[\textit{flags}]%
|\input{childdoc.def}\childdocforward[|\textit{main}|]{|\textit{dest}|}"|
\end{center}
%
Here \textit{target} is the name of the output file,
\textit{main} is the name of the main file
and \textit{dest} is the name of the main or child file to be processed
(all filenames without extensions).
The optional argument \textit{main} can be omitted
if \textit{main} matches \textit{dest}.
Optionally, compilation \textit{flags} can be defined via |\def| commands.
This command line makes the \TeX{} engine believe
it is compiling the file \textit{target}
whose content is specified as the latter parameter.
The provided code then forwards the processing to
\textit{main} or \textit{dest} as described in \secref{sec:forward}.

%%%%%%%%%%%%%%%%%%%%%%%%%%%%%%%%%%%%%%%%%%%%%%%%%%%%%%%%%%%%%%%%%%%%%%%%%%%%%%%%
\subsection{Include by Input}
\label{sec:input}

Including child documents by |\include| has some restrictions by design.
Most notably, the content of a child document always occupies
its own set of pages; pages cannot be shared between child documents.
Usually, this behaviour makes perfect sense
because each child document contain an essential part of the document.
However, in some situations it may be desirable to compose
a document from a collection of parts
without having mandatory page breaks between then.
For this case, the package
provides a mechanism to include parts
by |\input| which can also be processed individually.
However, by construction this mechanism
requires manual handling of the content to be output.

%%%%%%%%%%%%%%%%%%%%%%%%%%%%%%%%%%%%%%%%
\DescribeMacro{\ifchilddocmanual}
The main file should be prepared as usual, see \secref{sec:include}.
However, the document body must make a distinction
between processing of an individual part and of the main document, e.g.:
%
\begin{center}
\begin{tabular}{l}
|\ifchilddocmanual|\\
|\input{\childdocname}|\\
|\||else|\\
\textit{document body with }|\input{|\textit{part}|}|\\
|\||fi|
\end{tabular}
\end{center}
%
The conditional |\ifchilddocmanual| is true whenever
a part to be included by |\input| is being compiled,
and the name of the part is stored in |\childdocname|.

%%%%%%%%%%%%%%%%%%%%%%%%%%%%%%%%%%%%%%%%
\DescribeMacro{\childdocby}
Each part to be included by |\input| should start with:
%
\begin{center}
\begin{tabular}{l}
|\input{childdoc.def}|\\
|\childdocby{|\textit{main}|}|\\
\end{tabular}
\end{center}
%
The directive |\childdocby| is similar to |\childdocof|
described in \secref{sec:include},
but the subsequent selection of content must be done manually.
To that end, both |\ifchilddoc| and |\ifchilddocmanual|
will be true upon processing of a part,
and the name of the part is stored in |\childdocname|.
Note that |\jobname| will be set to the filename of the current part
so that each part receives an individual |.aux| file
that does not interfere with the |.aux| file(s) of the main document.
This behaviour can be altered by the alternative form
|\childdocby[*]{|\textit{main}|}| (with a non-empty optional argument)
which uses the |.aux| file of the main document
by setting |\jobname| to \textit{main}.

%%%%%%%%%%%%%%%%%%%%%%%%%%%%%%%%%%%%%%%%%%%%%%%%%%%%%%%%%%%%%%%%%%%%%%%%%%%%%%%%
\subsection{Driver Development}
\label{sec:driver}

The \textsf{childdoc} mechanism can also be use for the development
of definition files such as \LaTeX{} styles or classes.
This case differs from the above setup with multiple parts
included by |\include| in that no |\includeonly| should be invoked.
This can be achieved by starting the include file
(before |\ProvidesPackage|) with:
%
\begin{center}
\begin{tabular}{l}
|\input{childdoc.def}|\\
|\childdocforward{|\textit{main}|}|\\
\end{tabular}
\end{center}
%
or alternatively with:
%
\begin{center}
\begin{tabular}{l}
|\input{childdoc.def}|\\
|\childdocby{|\textit{main}|}|\\
\end{tabular}
\end{center}
%
Both forms have slightly different effects as described above.
The main file is prepared as usual, see \secref{sec:include}.

%%%%%%%%%%%%%%%%%%%%%%%%%%%%%%%%%%%%%%%%%%%%%%%%%%%%%%%%%%%%%%%%%%%%%%%%%%%%%%%%
\subsection{Legacy Detection}
\label{sec:detection}

The directive |\childdocmain| in the main file can detect
whether the complete document or merely a child is to be compiled
even without using the directive |\childdocof|.
This method is deprecated because it is less robust
and there is no compelling reason to use it;
it is merely provided for backward compatibility
and it may be removed in future versions.

If the detection mechanism is to be used,
it is mandatory to correctly specify
the filename of the main file as the argument of |\childdocmain|:
%
\begin{center}
\begin{tabular}{l}
|\input{childdoc.def}|\\
|\childdocmain{|\textit{main}|}|\\
\end{tabular}
\end{center}
%
If |\jobname| does not match the argument \textit{main} of |\childdocmain|,
it is assumed that |\jobname| points to the child file to be compiled.
When using |\childdocmain| with the main file specified as argument,
it suffices to start a child file
with just |\input{|\textit{main}|}|
without loading of the package and using |\childdocof|.
If instead all processing is done
with the appropriate \textsf{childdoc} directives,
the argument of \textit{main} of |\childdocmain| can be empty.

An alternative version of the command line processing described
in \secref{sec:commandline} using the detection mechanism reads:
%
\begin{center}
|... -jobname "|\textit{target}|" "|[\textit{flags}]%
[|\def\jobname{|\textit{dest}|}|]|\input{|\textit{main}|}"|
\end{center}

%%%%%%%%%%%%%%%%%%%%%%%%%%%%%%%%%%%%%%%%%%%%%%%%%%%%%%%%%%%%%%%%%%%%%%%%%%%%%%%%
\subsection{Manual Code}
\label{sec:manual}

In case one cannot be certain whether the definitions file |childdoc.def|
is installed on the target \TeX{} distribution
and one prefers not to ship it,
it is conceivable to paste a few relevant commands into the sources.

To that end, drop all statements |\input{childdoc.def}|
and perform the replacements as outlined below.
Instead of |\childdocmain{|\textit{main}|}| add the following code
to the top of the main file:
%
\begin{center}
\begin{tabular}{l}
|\||ifdefined\childdocname\endinput\||fi\newif\ifchilddoc|\\
|\edef\childdocname{\scantokens\expandafter{\jobname\noexpand}}|\\
|\def\childdocmain{|\textit{main}|}\||ifx\childdocmain\childdocname\||else|\\
|\childdoctrue\includeonly{\childdocname}\let\jobname\childdocmain\||fi|\\
\end{tabular}
\end{center}
%
Instead of |\childdocof{|\textit{main}|}| just include the main file
at the top of each child file:
%
\begin{center}
|\input{|\textit{main}|}|
\end{center}
%
A simple redirection |\childdocforward{|\textit{dest}|}| is achieved by:
%
\begin{center}
|\def\jobname{|\textit{dest}|}\input{\jobname}|
\end{center}
%
The redirection with prefix
|\childdocforwardprefix[|\textit{prefix}|]{|\textit{dest}|}|
is accomplished by:
%
\begin{center}
\begin{tabular}{l}
|{\edef\jobname{\scantokens\expandafter{\jobname\noexpand}}|\\
|\def\redirectjob |\textit{prefix}|#1~~~{\gdef\jobname{|\textit{dest}|#1}}|\\
|\expandafter\redirectjob\jobname~~~}\input{\jobname}|
\end{tabular}
\end{center}

In an alternative approach,
child documents can be compiled by a specific command line
without additional code or specific definitions:
%
\begin{center}
|... -jobname "|\textit{target}|" "|[\textit{flags}]%
|\includeonly{|\textit{dest}|}\input{|\textit{main}|}"|
\end{center}
%

%%%%%%%%%%%%%%%%%%%%%%%%%%%%%%%%%%%%%%%%%%%%%%%%%%%%%%%%%%%%%%%%%%%%%%%%%%%%%%%%
%%%%%%%%%%%%%%%%%%%%%%%%%%%%%%%%%%%%%%%%%%%%%%%%%%%%%%%%%%%%%%%%%%%%%%%%%%%%%%%%
\section{Information}

%%%%%%%%%%%%%%%%%%%%%%%%%%%%%%%%%%%%%%%%%%%%%%%%%%%%%%%%%%%%%%%%%%%%%%%%%%%%%%%%
\subsection{Copyright}

Copyright \copyright{} 2017--2018 Niklas Beisert

This work may be distributed and/or modified under the
conditions of the \LaTeX{} Project Public License, either version 1.3
of this license or (at your option) any later version.
The latest version of this license is in
  \url{http://www.latex-project.org/lppl.txt}
and version 1.3 or later is part of all distributions of \LaTeX{}
version 2005/12/01 or later.

This work has the LPPL maintenance status `maintained'.

The Current Maintainer of this work is Niklas Beisert.

This work consists of the files |README.txt|, |childdoc.ins| and |childdoc.dtx|
as well as the derived files |childdoc.def|, |cdocsamp.tex|
with |cdocsch1.tex|, |cdocsch2.tex|, |cdocspt3.tex|, |cdocspt4.tex|,
|cdocsdrf.tex|, |cdocsfn1.tex|, |cdocsfn2.tex|
as well as |childdoc.pdf|.

%%%%%%%%%%%%%%%%%%%%%%%%%%%%%%%%%%%%%%%%%%%%%%%%%%%%%%%%%%%%%%%%%%%%%%%%%%%%%%%%
\subsection{Files and Installation}

The package consists of the files:
%
\begin{center}
\begin{tabular}{ll}
    |README.txt|   & readme file \\
    |childdoc.ins| & installation file \\
    |childdoc.dtx| & source file \\
    |childdoc.def| & definition file \\
    |cdocsamp.tex| & sample main file \\
    |cdocsch1.tex| & sample include file \\
    |cdocsch2.tex| & sample include file \\
    |cdocspt3.tex| & sample part file \\
    |cdocspt4.tex| & sample part file \\
    |cdocsdrf.tex| & sample redirection file \\
    |cdocsfn1.tex| & sample redirection file \\
    |cdocsfn2.tex| & sample redirection file \\
    |childdoc.pdf| & manual
\end{tabular}
\end{center}
%
The distribution consists of the files
|README.txt|, |childdoc.ins| and |childdoc.dtx|.
%
\begin{itemize}
\item
Run (pdf)\LaTeX{} on |childdoc.dtx|
to compile the manual |childdoc.pdf| (this file).
\item
Run \LaTeX{} on |childdoc.ins| to create the definitions file |childdoc.def|
and the sample |cdocsamp.tex| with include files
|cdocsch1.tex|, |cdocsch2.tex|, |cdocspt3.tex|, |cdocspt4.tex|,
|cdocsdrf.tex|, |cdocsfn1.tex|, |cdocsfn2.tex|.
Then copy the file |childdoc.def| to an appropriate directory of your \LaTeX{}
distribution, e.g.\ \textit{texmf-root}|/tex/latex/childdoc|.
\end{itemize}

%%%%%%%%%%%%%%%%%%%%%%%%%%%%%%%%%%%%%%%%%%%%%%%%%%%%%%%%%%%%%%%%%%%%%%%%%%%%%%%%
\subsection{Related CTAN Packages}

There are several other packages which offer a similar functionality:
%
\begin{itemize}
\item
The packages
\href{http://ctan.org/pkg/docmute}{\textsf{docmute}},
\href{http://ctan.org/pkg/includex}{\textsf{includex}} and
\href{http://ctan.org/pkg/standalone}{\textsf{standalone}}
provide commands to include only the document body of
a child file thus allowing both files to be compiled individually.
\item
The packages \href{http://ctan.org/pkg/subdocs}{\textsf{subdocs}}
and \href{http://ctan.org/pkg/subfiles}{\textsf{subfiles}}
provide structures in which the main and child documents can be
encapsulated and allowing them to be compiled individually.
The inclusion mechanism is different from the conventional |\include|.
\item
The package \href{http://ctan.org/pkg/combine}{\textsf{combine}}
is an elaborate solution to combine several documents into one.
\end{itemize}
%
See also the CTAN topic \href{http://ctan.org/topic/subdocs}{\textsf{subdocs}}
for further related packages.
The present package differs from the above solutions in that
a document structure constructed with the conventional |\include| mechanism
just needs two extra commands at the top of every file
such that all constituent files can be compiled individually.

%%%%%%%%%%%%%%%%%%%%%%%%%%%%%%%%%%%%%%%%%%%%%%%%%%%%%%%%%%%%%%%%%%%%%%%%%%%%%%%%
%\subsection{Feature Suggestions}
%
%The following is a list of features which may be useful for future
%versions of this package:
%%
%\begin{itemize}
%\item
%\ldots
%\end{itemize}

%%%%%%%%%%%%%%%%%%%%%%%%%%%%%%%%%%%%%%%%%%%%%%%%%%%%%%%%%%%%%%%%%%%%%%%%%%%%%%%%
\subsection{Revision History}

%%%%%%%%%%%%%%%%%%%%%%%%%%%%%%%%%%%%%%%%
\paragraph{v2.0:} 2018/12/30

\begin{itemize}
\item
immediate forward processing
\item
added |\childdocby| mechanism
\item
manual restructured
\end{itemize}

%%%%%%%%%%%%%%%%%%%%%%%%%%%%%%%%%%%%%%%%
\paragraph{v1.6:} 2018/01/17

\begin{itemize}
\item
application for development of include files
\item
corrections to manual
\end{itemize}

%%%%%%%%%%%%%%%%%%%%%%%%%%%%%%%%%%%%%%%%
\paragraph{v1.5:} 2017/05/21

\begin{itemize}
\item
more complete structuring introduced
\item
|\childdocof| introduced
\item
|\childdoc| renamed to |\childdocmain|
\item
|\childredirect| renamed to |\childdocforward| and |\childdocforwardprefix|
and functionality expanded
\end{itemize}

%%%%%%%%%%%%%%%%%%%%%%%%%%%%%%%%%%%%%%%%
\paragraph{v1.0:} 2017/04/27

\begin{itemize}
\item
manual and install package
\item
first version published on CTAN
\end{itemize}

%%%%%%%%%%%%%%%%%%%%%%%%%%%%%%%%%%%%%%%%
\paragraph{v0.6:} 2017/04/26

\begin{itemize}
\item
redirection mechanism added
\end{itemize}

%%%%%%%%%%%%%%%%%%%%%%%%%%%%%%%%%%%%%%%%
\paragraph{v0.5:} 2017/04/26

\begin{itemize}
\item
functionality in definition file
\end{itemize}


%%%%%%%%%%%%%%%%%%%%%%%%%%%%%%%%%%%%%%%%%%%%%%%%%%%%%%%%%%%%%%%%%%%%%%%%%%%%%%%%
%%%%%%%%%%%%%%%%%%%%%%%%%%%%%%%%%%%%%%%%%%%%%%%%%%%%%%%%%%%%%%%%%%%%%%%%%%%%%%%%
%%%%%%%%%%%%%%%%%%%%%%%%%%%%%%%%%%%%%%%%%%%%%%%%%%%%%%%%%%%%%%%%%%%%%%%%%%%%%%%%
\appendix

\settowidth\MacroIndent{\rmfamily\scriptsize 000\ }

 \DocInput{childdoc.dtx}

\end{document}
%</driver>
% \fi
%
% %%%%%%%%%%%%%%%%%%%%%%%%%%%%%%%%%%%%%%%%%%%%%%%%%%%%%%%%%%%%%%%%%%%%%%%%%%%%%%
% %%%%%%%%%%%%%%%%%%%%%%%%%%%%%%%%%%%%%%%%%%%%%%%%%%%%%%%%%%%%%%%%%%%%%%%%%%%%%%
% \section{Sample}
%\iffalse
%<*samplemain>
%\fi
%
% The following presents a sample document
% with two chapters, two parts, a title page,
% a compile flag as well as three forwarding files to set the flag.
% It consists of eight |.tex| files:
% \begin{center}
% \begin{tabular}{ll}
% |cdocsamp.tex|&main file\\
% |cdocsch1.tex|&include file for chapter 1\\
% |cdocsch2.tex|&include file for chapter 2\\
% |cdocspt3.tex|&include file for part 3\\
% |cdocspt4.tex|&include file for part 4\\
% |cdocsdrf.tex|&forwarding file for main file in draft mode\\
% |cdocsfi1.tex|&forwarding file for final version of chapter 1\\
% |cdocsfi2.tex|&forwarding file for final version of chapter 2\\
% \end{tabular}
% \end{center}
% Each of the eight files can be compiled directly by the \LaTeX{} compiler.
%
% %%%%%%%%%%%%%%%%%%%%%%%%%%%%%%%%%%%%%%
% \paragraph{Main File.}
%
% The main file is called |cdocsamp.tex|.
%
% Load the \textsf{childdoc} definitions and
% declare the filename for the main document:
%    \begin{macrocode}
\input{childdoc.def}
\childdocmain{}
%    \end{macrocode}

% Optional override for |\version| flag:
%    \begin{macrocode}
%%\ifchilddoc\else\providecommand{\version}{draft}\fi
%    \end{macrocode}

% Define the default values for the |\version| flag
% (|final| for the main file and |draft| for childs):
%    \begin{macrocode}
\ifchilddoc
\providecommand{\version}{draft}
\else
\providecommand{\version}{final}
\fi
%    \end{macrocode}

% Load the standard document class:
%    \begin{macrocode}
\documentclass[12pt]{article}
%    \end{macrocode}

% Start the document body:
%    \begin{macrocode}
\begin{document}
%    \end{macrocode}

% Declare a title page.
% Print title, part of document being processed and version flag:
%    \begin{macrocode}
\addtocounter{page}{-1}
\begin{center}
{\LARGE\bfseries{}childdoc example\par}
\vspace{1cm}
\ifchilddoc
\ifchilddocmanual part\else chapter\fi:
`\childdocname' of `\childdocjob'\par
\else
main document: `\childdocjob'\par
\fi
version: \version\par
\end{center}
\newpage
%    \end{macrocode}

% Manually include selected file,
% otherwise process as usual:
%    \begin{macrocode}
\ifchilddocmanual
\section*{part `\childdocname'}
\input{\childdocname}
\else
%    \end{macrocode}

% Include the two chapters:
%    \begin{macrocode}
\include{cdocsch1}
\include{cdocsch2}
%    \end{macrocode}

% Include the two parts unless only chapters should be displayed:
%    \begin{macrocode}
\ifchilddoc\else
\section{part three}
\input{cdocspt3}
\section{part four}
\input{cdocspt4}
\fi
%    \end{macrocode}

% Process as usual until here:
%    \begin{macrocode}
\fi
%    \end{macrocode}

% End of document body:
%    \begin{macrocode}
\end{document}
%    \end{macrocode}
%\iffalse
%</samplemain>
%\fi
%
% %%%%%%%%%%%%%%%%%%%%%%%%%%%%%%%%%%%%%%
% \paragraph{Chapter Include Files.}
%
% The include files are called |cdocsch1.tex| and |cdocsch2.tex|.
%
%\iffalse
%<*samplechap1|samplechap2>
%\fi

% Optional override for |\version| flag:
%    \begin{macrocode}
%%\providecommand{\version}{final}
%    \end{macrocode}

% Include the main document:
%    \begin{macrocode}
\input{childdoc.def}
\childdocof{cdocsamp}
%    \end{macrocode}

%\iffalse
%</samplechap1|samplechap2>
%\fi
%
%\iffalse
%<*samplechap1>
%\fi
% Some text for chapter 1:
%    \begin{macrocode}
\section{one}
some text in chapter one
%    \end{macrocode}

%\iffalse
%</samplechap1>
%\fi
% Some text for chapter 2:
%\iffalse
%<*samplechap2>
%\fi
%    \begin{macrocode}
\section{two}
more text in chapter two
%    \end{macrocode}

%\iffalse
%</samplechap2>
%\fi
%
% %%%%%%%%%%%%%%%%%%%%%%%%%%%%%%%%%%%%%%
% \paragraph{Part Include Files.}
%
% The include files are called |cdocspt3.tex| and |cdocspt4.tex|.
%
%\iffalse
%<*samplepart3|samplepart4>
%\fi

% Optional override for |\version| flag:
%    \begin{macrocode}
%%\providecommand{\version}{final}
%    \end{macrocode}

% Include the main document:
%    \begin{macrocode}
\input{childdoc.def}
\childdocby{cdocsamp}
%    \end{macrocode}

%\iffalse
%</samplepart3|samplepart4>
%\fi
%
%\iffalse
%<*samplepart3>
%\fi
% Some text for part 3:
%    \begin{macrocode}
some text in part three
%    \end{macrocode}

%\iffalse
%</samplepart3>
%\fi
% Some text for part 4:
%\iffalse
%<*samplepart4>
%\fi
%    \begin{macrocode}
more text in part four
%    \end{macrocode}

%\iffalse
%</samplepart4>
%\fi
%
% %%%%%%%%%%%%%%%%%%%%%%%%%%%%%%%%%%%%%%
% \paragraph{Forwarding for a Complete Draft.}
%
% The following forwarding file |cdocsdrf.tex|
% compiles the main document in draft mode:
%\iffalse
%<*sampledraft>
%\fi
%    \begin{macrocode}
\def\version{draft}
\input{childdoc.def}
\childdocforward{cdocsamp}
%    \end{macrocode}

%\iffalse
%</sampledraft>
%\fi
%
% %%%%%%%%%%%%%%%%%%%%%%%%%%%%%%%%%%%%%%
% \paragraph{Forwarding for Final Version of the Chapters.}
%
% The following forwarding files |cdocsfn1.tex| and |cdocsfn2.tex|
% (with identical content)
% compile the final versions of the child documents
% |cdocsch1.tex| and |cdocsch2.tex|, respectively:
%\iffalse
%<*samplefinal>
%\fi
%    \begin{macrocode}
\def\version{final}
\input{childdoc.def}
\childdocforwardprefix[cdocsamp]{cdocsfn}{cdocsch}
%    \end{macrocode}

%\iffalse
%</samplefinal>
%\fi
%
% %%%%%%%%%%%%%%%%%%%%%%%%%%%%%%%%%%%%%%
% \paragraph{Command Line Processing.}
%
% The following three command lines generate the output files
% |cdocscld|, |cdocscl1| and |cdocscl2|
% which should be identical to
% |cdocsdrf|, |cdocsch1| and |cdocsfn2|, respectively:
% \begin{center}
% \begin{tabular}{l}
% |latex -jobname cdocscld \|\\
% |  "\def\version{draft}\input{childdoc.def}\childdocforward{cdocsamp}"|\\
% |latex -jobname cdocscl1 \|\\
% |  "\input{childdoc.def}\childdocforward[cdocsamp]{cdocsch1}"|\\
% |latex -jobname cdocscl2 \|\\
% |  "\def\version{final}\input{childdoc.def}\childdocforward{cdocsch2}"|
% \end{tabular}
% \end{center}
% Note that the trailing backslash on each first line
% merely continues the input to the second line
% (for convenient cut ant paste).
% Furthermore, the command |latex| can be replaced by any
% of its alternative versions such as |pdflatex|.
%
% %%%%%%%%%%%%%%%%%%%%%%%%%%%%%%%%%%%%%%%%%%%%%%%%%%%%%%%%%%%%%%%%%%%%%%%%%%%%%%
% %%%%%%%%%%%%%%%%%%%%%%%%%%%%%%%%%%%%%%%%%%%%%%%%%%%%%%%%%%%%%%%%%%%%%%%%%%%%%%
% \section{Implementation}
%\iffalse
%<*package>
%\fi
%
% This section describes the definitions file |childdoc.def|.

% The definitions cannot be loaded using |\usepackage| or |\RequirePackage|
% which has a mechanism to prevent loading a style file more than once.
% When loading the definitions by means of |\input|
% multiple instances have to be prevented manually:
%\iffalse
%This code needs to be before the `\ProvidesFile' directive
%which is defined at the beginning of this file.
%Therefore it is also placed there and commented out here.
%</package>
%<*discard>
%\fi
%    \begin{macrocode}
\ifdefined\childdocmain\endinput\fi
%    \end{macrocode}
%\iffalse
%</discard>
%<*package>
%\fi
%
% \macro{\ifchilddoc}
% \macro{\ifchilddocmanual}
% The conditional |\ifchilddoc| tells whether a
% child (true) or main (false) document is being compiled.
% The conditional |\ifchilddocmanual| tells whether
% the |\includeonly| mechanism is used (false) or
% the selection of child files must be performed manually (true).
% The definitions initialise to false:
%    \begin{macrocode}
\newif\ifchilddoc
\newif\ifchilddocmanual
%    \end{macrocode}

% \macro{\childdocname}
% \macro{\childdocjob}
% The macro |\childdocname| stores the name of the main document
% to be compiled. The macro |\childdocjob| stores the name of
% the document on which the \LaTeX{} compiler was originally invoked.
% The content of |\jobname| cannot be compared
% to filenames specified in the source due to different catcodes.
% The following code rescans |\jobname|, stores the result
% in |\childdocname| and saves a copy in |\childdocjob|:
%    \begin{macrocode}
\edef\childdocname{\scantokens\expandafter{\jobname\noexpand}}
\let\childdocjob\childdocname
%    \end{macrocode}

% \macro{\childdocdisable}
% The macro |\childdocdisable| prevents the main file
% from being processed more than once.
% At this stage, the main document command |\childdocmain|
% is assumed to be called once again where it should do nothing.
% Any subsequent call to it should prevent
% a secondary processing of the main document
% It overwrites the forwarding commands
% |\childdocof| and |\childdocforward|
% with empty macros to prevent further inclusions of the main document:
%    \begin{macrocode}
\newcommand{\childdocdisable}
{
  \renewcommand{\childdocmain}[1]{\renewcommand{\childdocmain}[1]{\endinput}}
  \renewcommand{\childdocof}[1]{}
  \renewcommand{\childdocby}[2][]{}
  \renewcommand{\childdocforward}[2][]{}
  \renewcommand{\childdocdisable}{}
}
%    \end{macrocode}

% \macro{\childdocmain}
% The macro |\childdocmain| is to be called at the top of the main file
% with nothing or the main filename (without extension) as argument.
% First, it breaks loops.
% If the argument is not empty and does not match |\childdocname|
% (which is set by the first inclusion of |childdoc.def|),
% |\ifchilddoc| is set to true, |\includeonly| is applied to the child file
% and |\jobname| is set to the main file
% (for proper handling of |.aux| files):
%    \begin{macrocode}
\newcommand{\childdocmain}[1]
{
  \childdocdisable\childdocmain{}
  \if?#1?\else
    \begingroup
      \def\childdoctmp{#1}
      \ifx\childdoctmp\childdocname
        \def\childdoctmp{}
      \else
        \def\childdoctmp
        {
          \childdoctrue
          \includeonly{\childdocname}
          \def\childdocjob{#1}
          \def\jobname{#1}
        }
      \fi
      \expandafter
    \endgroup
    \childdoctmp
  \fi
}
%    \end{macrocode}

% \macro{\childdocof}
% The command |\childdocof| redirects
% compilation to the main file |#1|.
%    \begin{macrocode}
\newcommand{\childdocof}[1]
{
  \childdocdisable
  \childdoctrue
  \includeonly{\childdocname}
  \def\jobname{#1}
  \def\childdocjob{#1}
  \input{#1}
}
%    \end{macrocode}

% \macro{\childdocby}
% The command |\childdocby| ....
%    \begin{macrocode}
\newcommand{\childdocby}[2][]
{
  \childdocdisable
  \childdoctrue
  \childdocmanualtrue
  \if?#1?\else
    \def\jobname{#2}
  \fi
  \def\childdocjob{#2}
  \input{#2}
  \endinput
}
%    \end{macrocode}

% \macro{\childdocforward}
% The command |\childdocforward| redirects
% compilation to the main file or
% (if the optional argument is given) a child file.
% Parameters are set as if the main file
% or a child file starting with |\childdocof| was compiled.
% Then compilation is handed over to the main file:
%    \begin{macrocode}
\newcommand{\childdocforward}[2][]
{
  \begingroup
    \if?#1?
      \def\childdoctmp
      {
        \def\childdocname{#2}
        \def\childdocjob{#2}
        \def\jobname{#2}
        \input{#2}
        \endinput
      }
    \else
      \def\childdoctmp
      {
        \childdocdisable
        \def\childdocname{#2}
        \childdoctrue
        \includeonly{#2}
        \def\childdocjob{#1}
        \def\jobname{#1}
        \input{#1}
        \endinput
      }
    \fi
    \expandafter
  \endgroup
  \childdoctmp
}
%    \end{macrocode}

% \macro{\childdocforwardprefix}
% The command |\childdocforwardprefix| redirects
% compilation to the main or a child file by means of a pattern.
% The prefix |#1| in the current filename is replaced by |#2|
% and the suffix of the current filename is kept
% (it is assumed that the filename does not contain the substring `|~~~|'
% which is used as a delimiter).
% Compilation is handed over to the new file by |\childdocforward|:
%    \begin{macrocode}
\newcommand{\childdocforwardprefix}[3][]
{
  \begingroup
    \def\childdocextract #2##1~~~{\def\childdoctmp{\childdocforward[#1]{#3##1}}}
    \expandafter\childdocextract\childdocname~~~
    \expandafter
  \endgroup
  \childdoctmp
}
%    \end{macrocode}

% \macro{\childdoc}
% The deprecated macro |\childdoc| is a legacy version of |\childdocmain|:
%    \begin{macrocode}
\newcommand{\childdoc}{\childdocmain}
%    \end{macrocode}

% \macro{\childdocredirect}
% The deprecated macro |\childdocredirect| is a legacy version
% of |\childdocforward| and |\childdocforwardprefix|:
%    \begin{macrocode}
\newcommand{\childdocredirect}[2][]
{
  \begingroup
    \if?#1?
      \def\childdoctmp{\childdocforward{#2}}
    \else
      \def\childdoctmp{\childdocforwardprefix{#1}{#2}}
    \fi
    \expandafter
  \endgroup
  \childdoctmp
}
%    \end{macrocode}

%\iffalse
%</package>
%\fi
%
\endinput
|\\
|\childdocforward[|\textit{main}|]{|\textit{dest}|}|\\
\end{tabular}
\end{center}
%
The argument \textit{dest} is the destination file
(without extension).
It should be the main file or one of the child files.
Note that further \textsf{childdoc} directives
such as |\childdocof| and |\childdocforward|
in the indicated file will be processed in this form.
The optional argument \textit{main}
passes on directly to the main file \textit{main}
while pretending to compile the child \textit{dest}.
This form behaves as if \textit{dest}
issues |\childdocof{|\textit{main}|}| right away,
and no further \textsf{childdoc} directives will be processed.

%%%%%%%%%%%%%%%%%%%%%%%%%%%%%%%%%%%%%%%%
\DescribeMacro{\...prefix}
In the alternative form |\childdocforwardprefix|,
%
\begin{center}
\begin{tabular}{l}
|% \iffalse
%
% childdoc.dtx Copyright (C) 2017-2018 Niklas Beisert
%
% This work may be distributed and/or modified under the
% conditions of the LaTeX Project Public License, either version 1.3
% of this license or (at your option) any later version.
% The latest version of this license is in
%   http://www.latex-project.org/lppl.txt
% and version 1.3 or later is part of all distributions of LaTeX
% version 2005/12/01 or later.
%
% This work has the LPPL maintenance status `maintained'.
%
% The Current Maintainer of this work is Niklas Beisert.
%
% This work consists of the files childdoc.dtx and childdoc.ins
% and the derived files childdoc.def and cdocsamp.tex with
% cdocsch1.tex, cdocsch2.tex, cdocsdrf.tex, cdocsfn1.tex, cdocsfn2.tex.
%
%<package>\ifdefined\childdocmain\endinput\fi
%<package>\ProvidesFile{childdoc.def}[2018/12/30 v2.0 child document driver]
%<samplemain>\ProvidesFile{cdocsamp.tex}[2018/12/30 v2.0 sample for childdoc]
%<*driver>
%\ProvidesFile{childdoc.drv}[2018/12/30 v2.0 childdoc reference manual file]
\PassOptionsToClass{10pt,a4paper}{article}
\documentclass{ltxdoc}

\usepackage[margin=35mm]{geometry}
\usepackage{hyperref}
\usepackage{hyperxmp}
\usepackage[usenames]{color}

\hypersetup{colorlinks=true}
\hypersetup{pdfstartview=FitH}
\hypersetup{pdfpagemode=UseNone}
\hypersetup{pdfsource={}}
\hypersetup{pdflang={en-UK}}
\hypersetup{pdfcopyright={Copyright 2017-2018 Niklas Beisert.
  This work may be distributed and/or modified under the
  conditions of the LaTeX Project Public License, either version 1.3
  of this license or (at your option) any later version.}}
\hypersetup{pdflicenseurl={http://www.latex-project.org/lppl.txt}}
\hypersetup{pdfcontactaddress={ETH Zurich, ITP, HIT K,
  Wolfgang-Pauli-Strasse 27}}
\hypersetup{pdfcontactpostcode={8093}}
\hypersetup{pdfcontactcity={Zurich}}
\hypersetup{pdfcontactcountry={Switzerland}}
\hypersetup{pdfcontactemail={nbeisert@itp.phys.ethz.ch}}
\hypersetup{pdfcontacturl={http://people.phys.ethz.ch/\xmptilde nbeisert/}}

\newcommand{\secref}[1]{\hyperref[#1]{section \ref*{#1}}}

\parskip1ex
\parindent0pt
\let\olditemize\itemize
\def\itemize{\olditemize\parskip0pt}

\begin{document}

\title{The \textsf{childdoc} Package}
\hypersetup{pdftitle={The childdoc Package}}
\author{Niklas Beisert\\[2ex]
  Institut f\"ur Theoretische Physik\\
  Eidgen\"ossische Technische Hochschule Z\"urich\\
  Wolfgang-Pauli-Strasse 27, 8093 Z\"urich, Switzerland\\[1ex]
  \href{mailto:nbeisert@itp.phys.ethz.ch}
  {\texttt{nbeisert@itp.phys.ethz.ch}}}
\hypersetup{pdfauthor={Niklas Beisert}}
\hypersetup{pdfsubject={Manual for the LaTeX2e Package childdoc}}
\date{30 December 2018, \textsf{v2.0}}
\maketitle

\begin{abstract}\noindent
\textsf{childdoc} is a \LaTeXe{} package
that enables the direct compilation
of document sections included by |\include|
to individual files.
\end{abstract}

\begingroup
\parskip0ex
\tableofcontents
\endgroup

%%%%%%%%%%%%%%%%%%%%%%%%%%%%%%%%%%%%%%%%%%%%%%%%%%%%%%%%%%%%%%%%%%%%%%%%%%%%%%%%
%%%%%%%%%%%%%%%%%%%%%%%%%%%%%%%%%%%%%%%%%%%%%%%%%%%%%%%%%%%%%%%%%%%%%%%%%%%%%%%%
\section{Introduction}

\LaTeX{} provides a mechanism to structure a large document (such as a book)
into a main file and several child files (containing the chapters)
using the |\include| command.
This mechanism is beneficial for documents
which span hundreds of pages in order to
make the source file(s) more manageable.
Moreover, compilation can be restricted to
selected child files by means of the |\includeonly| command.
The latter feature can be used to reduce the compilation time while editing
(this was significantly more useful in the earlier days of \LaTeX{})
or to generate a smaller document which is easier to navigate.
Another application of |\includeonly| is to generate
documents consisting of selected parts of the complete document.

However, there are a few drawbacks of the plain |\include| mechanism:
\begin{itemize}
\item
The child files cannot be compiled on their own,
they can only be compiled via the main file.
A naive editing environment
(such as a text editor with an option
to have the current file processed by \LaTeX)
may require one to switch to the main file before compiling;
attempting to compile the child file produces errors.
\item
The main file must be modified (each time)
to adjust the |\includeonly| command
to the present needs. This easily leaves the main file in a messy state.
\item
The generated document will always carry the filename
of the main document. This is inconvenient if
several child files are to be compiled and
to be kept for distribution.
\end{itemize}

The present package provides a simple interface
to make child files individually compilable by \LaTeX{}.
Compiling a child file then has the same effect as compiling
the main file with an |\includeonly| command
to select the appropriate child.
Moreover the generated document will carry the name of the child
rather than the main file.
This resolves all three above issues.

This feature is meant to make the editing of books,
thesis documents and lecture notes somewhat more convenient.
However, the package can also be used efficiently for
composing a series of documents (such as exercise sheets)
which are typically distributed individually.
It then assists the author in generating the individual documents
(potentially in different versions)
as well as a document containing the collected series.
Another application is in developing style files
or other kinds of included material
where compilation of the style file could redirect
to a sample or test file.

%%%%%%%%%%%%%%%%%%%%%%%%%%%%%%%%%%%%%%%%%%%%%%%%%%%%%%%%%%%%%%%%%%%%%%%%%%%%%%%%
%%%%%%%%%%%%%%%%%%%%%%%%%%%%%%%%%%%%%%%%%%%%%%%%%%%%%%%%%%%%%%%%%%%%%%%%%%%%%%%%
\section{Usage}

First of all, the package \textsf{childdoc} is \emph{not} a standard
\LaTeXe{} |.sty| style file! Therefore it needs to be invoked in
a non-standard way.

%%%%%%%%%%%%%%%%%%%%%%%%%%%%%%%%%%%%%%%%%%%%%%%%%%%%%%%%%%%%%%%%%%%%%%%%%%%%%%%%
\subsection{Included Files}
\label{sec:include}

%%%%%%%%%%%%%%%%%%%%%%%%%%%%%%%%%%%%%%%%
\DescribeMacro{\childdocmain}
To use the package, add the commands
\begin{center}
\begin{tabular}{l}
|\input{childdoc.def}|\\
|\childdocmain{}|\\
\end{tabular}
\end{center}
at the very top of the main \LaTeX{} file,
in particular \emph{before} the |\documentclass| statement!
The argument of |\childdocmain| should be left empty
(but it must be present).

%%%%%%%%%%%%%%%%%%%%%%%%%%%%%%%%%%%%%%%%
\DescribeMacro{\childdocof}
Furthermore, add the commands
\begin{center}
\begin{tabular}{l}
|\input{childdoc.def}|\\
|\childdocof{|\textit{main}|}|\\
\end{tabular}
\end{center}
at the top of every child file \textit{child}
which is included by |\include{|\textit{child}|}|
from within the main file
(or at least for those files to be compiled individually).
The argument \textit{main} must be the filename of the main file.

There are a couple of
considerations in setting up the main and child documents:

%%%%%%%%%%%%%%%%%%%%%%%%%%%%%%%%%%%%%%%%
\paragraph{Restrictions.}

Please note the following restrictions:
\begin{itemize}
\item
|\childdocmain| must be called with one argument \textit{main}
to ensure compatibility with earlier version of the package.
It must either be empty (|\childdocmain{}|)
or precisely match the filename of the main file in which it is specified.
See \secref{sec:detection} for further information.
\item
The filename \textit{main} must be specified without the |.tex| extension.
\item
The filename \textit{main} is case sensitive
(even in case-insensitive file systems)
due to internal string comparison.
\item
The argument \textit{main} should be fully expanded, it cannot be a macro.
\item
Subdirectories and special characters should be avoided in filenames.
\item
The command |\childdocmain{|\textit{main}|}| must be followed by a whitespace.
It should not be followed immediately by another command
or by a comment mark `|%|'.
This is because the \TeX{} parser reads the token immediately following
the argument of |\childdocmain| and puts it
at the beginning of every child section;
however, a white\-space is ignored.
\end{itemize}

%%%%%%%%%%%%%%%%%%%%%%%%%%%%%%%%%%%%%%%%
\paragraph{Content of Main File.}

It is advisable to place all content in the child files included by |\include|.
Any output contained in the main file will appear in all child documents
unless suppressed manually;
it cannot be suppressed automatically by the |\includeonly| directive
and thus should normally be avoided.
A method to include some content in the main file
by means of conditional processing is described in \secref{sec:conditional}.

%%%%%%%%%%%%%%%%%%%%%%%%%%%%%%%%%%%%%%%%
\paragraph{Page Numbering.}

When only a part of the document is compiled,
the appropriate numbering of pages
(as well as other status parameters)
is determined from the |.aux| files.
The latter contain information from previous passes.
However this information needs to propagate through
all intermediate child documents.
Therefore the page numbering in child documents may well
be inconsistent until the complete document is compiled at least once.

A useful (if unconventional) way to always ensure a consistent
page numbering is to restart the numbering in each child document
and denote the pages by `\textit{child}|.|\textit{page}'
where \textit{child} represents the chapter/section number of the child file.
This can be achieved by the command
|\numberwithin{page}{|\textit{child}|}|
of the \textsf{amsmath} package
where \textit{child} can be |chapter| or |section|
depending on the chosen structuring.
Alternatively, one can modify the macro |\thepage| appropriately
and reset the counter |page| at the start of each child file.

%%%%%%%%%%%%%%%%%%%%%%%%%%%%%%%%%%%%%%%%%%%%%%%%%%%%%%%%%%%%%%%%%%%%%%%%%%%%%%%%
\subsection{Conditional Processing}
\label{sec:conditional}

The package provides a mechanism to compile different versions
of a document. To customise the versions further some conditional processing
can come in handy to distinguish which version is being compiled.
The package provides two macros to describe the compilation context:

%%%%%%%%%%%%%%%%%%%%%%%%%%%%%%%%%%%%%%%%
\DescribeMacro{\ifchilddoc}
The conditional |\ifchilddoc| distinguishes between the compilation of
child documents and the main document:
%
\begin{center}
|\ifchilddoc |\textit{child-code}| |[|\||else |\textit{main-code}]| \||fi|
\end{center}

%%%%%%%%%%%%%%%%%%%%%%%%%%%%%%%%%%%%%%%%
\DescribeMacro{\childdocname}
\DescribeMacro{\childdocjob}
The macro |\childdocname| contains the filename (without extension)
of the main or child file being processed.
Note that |\childdocjob| will always contain the name of the main file.

%%%%%%%%%%%%%%%%%%%%%%%%%%%%%%%%%%%%%%%%
\paragraph{Title Page.}

Conditional processing can be used to include a title or banner page
in the main document when proper precautions are taken.
Importantly, the code in the main file should ensure that the page counter
(as well as other status parameters which are stored in the |.aux| files)
takes the same value after the conditional processing.
Otherwise the page numbers may take divergent values
depending on which part is compiled.

For example, a title page could be declared by:
%
\begin{center}
\begin{tabular}{l}
|\ifchilddoc\||else|\\
|\addtocounter{page}{-1}|\\
\textit{code for title page}\\
|\newpage|\\
|\||fi|
\end{tabular}
\end{center}
%
A banner page for the child documents can be generated by:
%
\begin{center}
\begin{tabular}{l}
|\ifchilddoc|\\
|\addtocounter{page}{-1}|\\
\textit{code for banner page}\\
|\newpage|\\
|\||fi|
\end{tabular}
\end{center}
%
Here one could write a message such as:
\begin{center}
|This is the part \childdocname{} of \childdocjob{}.|
\end{center}

%%%%%%%%%%%%%%%%%%%%%%%%%%%%%%%%%%%%%%%%%%%%%%%%%%%%%%%%%%%%%%%%%%%%%%%%%%%%%%%%
\subsection{Flags}
\label{sec:flags}

The package makes it easy to generate different versions
of the main or child documents.
To this end compilation flags can be defined
and assigned different default values.
They will be particularly useful in conjunction
with the forwarding mechanism described in \secref{sec:forward}.

For example, it may be useful to have a flag |\version|
which can be set to |draft| or |final|.
The document source will contain some conditional code
depending on the value of |\version|.
Suppose further, the flag should default to |final| for the main file
and to |draft| for child files
which is a natural assignment for editing the document.
This is achieved by placing the following code
in the preamble of the main document
(below the |\childdocmain| directive):
%
\begin{center}
\begin{tabular}{l}
|\ifchilddoc|\\
|\providecommand{\version}{draft}|\\
|\||else|\\
|\providecommand{\version}{final}|\\
|\||fi|
\end{tabular}
\end{center}
%
The definition by |\providecommand| makes sure
that previous definitions are not overwritten.
Further statements |\providecommand{\version}{...}|
can thus be added before the above code to override it.

For the main file, one might add a line
(between |\childdocmain| and the above block)
%
\begin{center}
|%\ifchilddoc\||else\providecommand{\version}{draft}\||fi|
\end{center}
%
which can be uncommented to produce a draft version.
Likewise one can add a line to the very top of a child file
(above the |\childdocof{|\textit{main}|}| directive)
%
\begin{center}
|%\providecommand{\version}{final}|
\end{center}
%
which can be uncommented to produce the final version of this child document.

%%%%%%%%%%%%%%%%%%%%%%%%%%%%%%%%%%%%%%%%%%%%%%%%%%%%%%%%%%%%%%%%%%%%%%%%%%%%%%%%
\subsection{Forwarding}
\label{sec:forward}

Different versions of the main or child documents
using compilation flags as described in \secref{sec:flags}
can be (permanently) stored in different files
for convenient compilation, viewing and distribution.
To this end, the package defines a command
to pass on compilation to a different file:

%%%%%%%%%%%%%%%%%%%%%%%%%%%%%%%%%%%%%%%%
\DescribeMacro{\childdocforward}
The command |\childdocforward| redirects processing to
another source file:
%
\begin{center}
\begin{tabular}{l}
|\input{childdoc.def}|\\
|\childdocforward[|\textit{main}|]{|\textit{dest}|}|\\
\end{tabular}
\end{center}
%
The argument \textit{dest} is the destination file
(without extension).
It should be the main file or one of the child files.
Note that further \textsf{childdoc} directives
such as |\childdocof| and |\childdocforward|
in the indicated file will be processed in this form.
The optional argument \textit{main}
passes on directly to the main file \textit{main}
while pretending to compile the child \textit{dest}.
This form behaves as if \textit{dest}
issues |\childdocof{|\textit{main}|}| right away,
and no further \textsf{childdoc} directives will be processed.

%%%%%%%%%%%%%%%%%%%%%%%%%%%%%%%%%%%%%%%%
\DescribeMacro{\...prefix}
In the alternative form |\childdocforwardprefix|,
%
\begin{center}
\begin{tabular}{l}
|\input{childdoc.def}|\\
|\childdocforwardprefix[|\textit{main}|]{|\textit{prefix}|}{|\textit{dest}|}|
\end{tabular}
\end{center}
%
the destination file is determined by a pattern
depending on the current file:
To make this work, the current file must be called
`{\textit{prefix}\hspace{0.2em}\textit{suffix}}'
with \textit{prefix} matching precisely the argument.
Processing is then passed on to the file
`{\textit{dest}\hspace{0.2em}\textit{suffix}}'.
Surely, the same effect is achieved by
directly specifying the
argument `{\textit{dest}\hspace{0.2em}\textit{suffix}}'
in the first form.
However, that requires to set up a different file
for each child. With the alternative form of the command
all these files can have exactly the same content
which simplifies setting them up and maintaining them.

For example, the following file |draft.tex|
with a compilation flag |\version| as described in \secref{sec:flags}
compiles the main document as a draft:
%
\begin{center}
\begin{tabular}{l}
|\def\version{draft}|\\
|\input{childdoc.def}|\\
|\childdocforward{|\textit{main}|}|
\end{tabular}
\end{center}
%
Likewise, the following files |final|\textit{nn}|.tex|
compile the final version of the child document
|child|\textit{nn}|.tex|:
%
\begin{center}
\begin{tabular}{l}
|\def\version{final}|\\
|\input{childdoc.def}|\\
|\childdocforwardprefix{final}{child}|
\end{tabular}
\end{center}
%

Note that when several versions of a main file and/or of each child file
are to be generated, it may be convenient to set up a |Makefile| or
shell script to automatise the process.

%%%%%%%%%%%%%%%%%%%%%%%%%%%%%%%%%%%%%%%%%%%%%%%%%%%%%%%%%%%%%%%%%%%%%%%%%%%%%%%%
\subsection{Command Line Processing}
\label{sec:commandline}

The effect of redirection files can also be achieved by invoking
the \LaTeX{} compiler with a more elaborate command line.
Most conveniently this should be done as part
of a shell script or a |Makefile|.

When using \textsf{childdoc} in the main file, the following
command lines effectively perform a redirection
(note that depending on the shell being used,
backslashes may have to be doubled: `|\|' $\to$ `|\\|'):
%
\begin{center}
|... -jobname "|\textit{target}|" |\\|"|[\textit{flags}]%
|\input{childdoc.def}\childdocforward[|\textit{main}|]{|\textit{dest}|}"|
\end{center}
%
Here \textit{target} is the name of the output file,
\textit{main} is the name of the main file
and \textit{dest} is the name of the main or child file to be processed
(all filenames without extensions).
The optional argument \textit{main} can be omitted
if \textit{main} matches \textit{dest}.
Optionally, compilation \textit{flags} can be defined via |\def| commands.
This command line makes the \TeX{} engine believe
it is compiling the file \textit{target}
whose content is specified as the latter parameter.
The provided code then forwards the processing to
\textit{main} or \textit{dest} as described in \secref{sec:forward}.

%%%%%%%%%%%%%%%%%%%%%%%%%%%%%%%%%%%%%%%%%%%%%%%%%%%%%%%%%%%%%%%%%%%%%%%%%%%%%%%%
\subsection{Include by Input}
\label{sec:input}

Including child documents by |\include| has some restrictions by design.
Most notably, the content of a child document always occupies
its own set of pages; pages cannot be shared between child documents.
Usually, this behaviour makes perfect sense
because each child document contain an essential part of the document.
However, in some situations it may be desirable to compose
a document from a collection of parts
without having mandatory page breaks between then.
For this case, the package
provides a mechanism to include parts
by |\input| which can also be processed individually.
However, by construction this mechanism
requires manual handling of the content to be output.

%%%%%%%%%%%%%%%%%%%%%%%%%%%%%%%%%%%%%%%%
\DescribeMacro{\ifchilddocmanual}
The main file should be prepared as usual, see \secref{sec:include}.
However, the document body must make a distinction
between processing of an individual part and of the main document, e.g.:
%
\begin{center}
\begin{tabular}{l}
|\ifchilddocmanual|\\
|\input{\childdocname}|\\
|\||else|\\
\textit{document body with }|\input{|\textit{part}|}|\\
|\||fi|
\end{tabular}
\end{center}
%
The conditional |\ifchilddocmanual| is true whenever
a part to be included by |\input| is being compiled,
and the name of the part is stored in |\childdocname|.

%%%%%%%%%%%%%%%%%%%%%%%%%%%%%%%%%%%%%%%%
\DescribeMacro{\childdocby}
Each part to be included by |\input| should start with:
%
\begin{center}
\begin{tabular}{l}
|\input{childdoc.def}|\\
|\childdocby{|\textit{main}|}|\\
\end{tabular}
\end{center}
%
The directive |\childdocby| is similar to |\childdocof|
described in \secref{sec:include},
but the subsequent selection of content must be done manually.
To that end, both |\ifchilddoc| and |\ifchilddocmanual|
will be true upon processing of a part,
and the name of the part is stored in |\childdocname|.
Note that |\jobname| will be set to the filename of the current part
so that each part receives an individual |.aux| file
that does not interfere with the |.aux| file(s) of the main document.
This behaviour can be altered by the alternative form
|\childdocby[*]{|\textit{main}|}| (with a non-empty optional argument)
which uses the |.aux| file of the main document
by setting |\jobname| to \textit{main}.

%%%%%%%%%%%%%%%%%%%%%%%%%%%%%%%%%%%%%%%%%%%%%%%%%%%%%%%%%%%%%%%%%%%%%%%%%%%%%%%%
\subsection{Driver Development}
\label{sec:driver}

The \textsf{childdoc} mechanism can also be use for the development
of definition files such as \LaTeX{} styles or classes.
This case differs from the above setup with multiple parts
included by |\include| in that no |\includeonly| should be invoked.
This can be achieved by starting the include file
(before |\ProvidesPackage|) with:
%
\begin{center}
\begin{tabular}{l}
|\input{childdoc.def}|\\
|\childdocforward{|\textit{main}|}|\\
\end{tabular}
\end{center}
%
or alternatively with:
%
\begin{center}
\begin{tabular}{l}
|\input{childdoc.def}|\\
|\childdocby{|\textit{main}|}|\\
\end{tabular}
\end{center}
%
Both forms have slightly different effects as described above.
The main file is prepared as usual, see \secref{sec:include}.

%%%%%%%%%%%%%%%%%%%%%%%%%%%%%%%%%%%%%%%%%%%%%%%%%%%%%%%%%%%%%%%%%%%%%%%%%%%%%%%%
\subsection{Legacy Detection}
\label{sec:detection}

The directive |\childdocmain| in the main file can detect
whether the complete document or merely a child is to be compiled
even without using the directive |\childdocof|.
This method is deprecated because it is less robust
and there is no compelling reason to use it;
it is merely provided for backward compatibility
and it may be removed in future versions.

If the detection mechanism is to be used,
it is mandatory to correctly specify
the filename of the main file as the argument of |\childdocmain|:
%
\begin{center}
\begin{tabular}{l}
|\input{childdoc.def}|\\
|\childdocmain{|\textit{main}|}|\\
\end{tabular}
\end{center}
%
If |\jobname| does not match the argument \textit{main} of |\childdocmain|,
it is assumed that |\jobname| points to the child file to be compiled.
When using |\childdocmain| with the main file specified as argument,
it suffices to start a child file
with just |\input{|\textit{main}|}|
without loading of the package and using |\childdocof|.
If instead all processing is done
with the appropriate \textsf{childdoc} directives,
the argument of \textit{main} of |\childdocmain| can be empty.

An alternative version of the command line processing described
in \secref{sec:commandline} using the detection mechanism reads:
%
\begin{center}
|... -jobname "|\textit{target}|" "|[\textit{flags}]%
[|\def\jobname{|\textit{dest}|}|]|\input{|\textit{main}|}"|
\end{center}

%%%%%%%%%%%%%%%%%%%%%%%%%%%%%%%%%%%%%%%%%%%%%%%%%%%%%%%%%%%%%%%%%%%%%%%%%%%%%%%%
\subsection{Manual Code}
\label{sec:manual}

In case one cannot be certain whether the definitions file |childdoc.def|
is installed on the target \TeX{} distribution
and one prefers not to ship it,
it is conceivable to paste a few relevant commands into the sources.

To that end, drop all statements |\input{childdoc.def}|
and perform the replacements as outlined below.
Instead of |\childdocmain{|\textit{main}|}| add the following code
to the top of the main file:
%
\begin{center}
\begin{tabular}{l}
|\||ifdefined\childdocname\endinput\||fi\newif\ifchilddoc|\\
|\edef\childdocname{\scantokens\expandafter{\jobname\noexpand}}|\\
|\def\childdocmain{|\textit{main}|}\||ifx\childdocmain\childdocname\||else|\\
|\childdoctrue\includeonly{\childdocname}\let\jobname\childdocmain\||fi|\\
\end{tabular}
\end{center}
%
Instead of |\childdocof{|\textit{main}|}| just include the main file
at the top of each child file:
%
\begin{center}
|\input{|\textit{main}|}|
\end{center}
%
A simple redirection |\childdocforward{|\textit{dest}|}| is achieved by:
%
\begin{center}
|\def\jobname{|\textit{dest}|}\input{\jobname}|
\end{center}
%
The redirection with prefix
|\childdocforwardprefix[|\textit{prefix}|]{|\textit{dest}|}|
is accomplished by:
%
\begin{center}
\begin{tabular}{l}
|{\edef\jobname{\scantokens\expandafter{\jobname\noexpand}}|\\
|\def\redirectjob |\textit{prefix}|#1~~~{\gdef\jobname{|\textit{dest}|#1}}|\\
|\expandafter\redirectjob\jobname~~~}\input{\jobname}|
\end{tabular}
\end{center}

In an alternative approach,
child documents can be compiled by a specific command line
without additional code or specific definitions:
%
\begin{center}
|... -jobname "|\textit{target}|" "|[\textit{flags}]%
|\includeonly{|\textit{dest}|}\input{|\textit{main}|}"|
\end{center}
%

%%%%%%%%%%%%%%%%%%%%%%%%%%%%%%%%%%%%%%%%%%%%%%%%%%%%%%%%%%%%%%%%%%%%%%%%%%%%%%%%
%%%%%%%%%%%%%%%%%%%%%%%%%%%%%%%%%%%%%%%%%%%%%%%%%%%%%%%%%%%%%%%%%%%%%%%%%%%%%%%%
\section{Information}

%%%%%%%%%%%%%%%%%%%%%%%%%%%%%%%%%%%%%%%%%%%%%%%%%%%%%%%%%%%%%%%%%%%%%%%%%%%%%%%%
\subsection{Copyright}

Copyright \copyright{} 2017--2018 Niklas Beisert

This work may be distributed and/or modified under the
conditions of the \LaTeX{} Project Public License, either version 1.3
of this license or (at your option) any later version.
The latest version of this license is in
  \url{http://www.latex-project.org/lppl.txt}
and version 1.3 or later is part of all distributions of \LaTeX{}
version 2005/12/01 or later.

This work has the LPPL maintenance status `maintained'.

The Current Maintainer of this work is Niklas Beisert.

This work consists of the files |README.txt|, |childdoc.ins| and |childdoc.dtx|
as well as the derived files |childdoc.def|, |cdocsamp.tex|
with |cdocsch1.tex|, |cdocsch2.tex|, |cdocspt3.tex|, |cdocspt4.tex|,
|cdocsdrf.tex|, |cdocsfn1.tex|, |cdocsfn2.tex|
as well as |childdoc.pdf|.

%%%%%%%%%%%%%%%%%%%%%%%%%%%%%%%%%%%%%%%%%%%%%%%%%%%%%%%%%%%%%%%%%%%%%%%%%%%%%%%%
\subsection{Files and Installation}

The package consists of the files:
%
\begin{center}
\begin{tabular}{ll}
    |README.txt|   & readme file \\
    |childdoc.ins| & installation file \\
    |childdoc.dtx| & source file \\
    |childdoc.def| & definition file \\
    |cdocsamp.tex| & sample main file \\
    |cdocsch1.tex| & sample include file \\
    |cdocsch2.tex| & sample include file \\
    |cdocspt3.tex| & sample part file \\
    |cdocspt4.tex| & sample part file \\
    |cdocsdrf.tex| & sample redirection file \\
    |cdocsfn1.tex| & sample redirection file \\
    |cdocsfn2.tex| & sample redirection file \\
    |childdoc.pdf| & manual
\end{tabular}
\end{center}
%
The distribution consists of the files
|README.txt|, |childdoc.ins| and |childdoc.dtx|.
%
\begin{itemize}
\item
Run (pdf)\LaTeX{} on |childdoc.dtx|
to compile the manual |childdoc.pdf| (this file).
\item
Run \LaTeX{} on |childdoc.ins| to create the definitions file |childdoc.def|
and the sample |cdocsamp.tex| with include files
|cdocsch1.tex|, |cdocsch2.tex|, |cdocspt3.tex|, |cdocspt4.tex|,
|cdocsdrf.tex|, |cdocsfn1.tex|, |cdocsfn2.tex|.
Then copy the file |childdoc.def| to an appropriate directory of your \LaTeX{}
distribution, e.g.\ \textit{texmf-root}|/tex/latex/childdoc|.
\end{itemize}

%%%%%%%%%%%%%%%%%%%%%%%%%%%%%%%%%%%%%%%%%%%%%%%%%%%%%%%%%%%%%%%%%%%%%%%%%%%%%%%%
\subsection{Related CTAN Packages}

There are several other packages which offer a similar functionality:
%
\begin{itemize}
\item
The packages
\href{http://ctan.org/pkg/docmute}{\textsf{docmute}},
\href{http://ctan.org/pkg/includex}{\textsf{includex}} and
\href{http://ctan.org/pkg/standalone}{\textsf{standalone}}
provide commands to include only the document body of
a child file thus allowing both files to be compiled individually.
\item
The packages \href{http://ctan.org/pkg/subdocs}{\textsf{subdocs}}
and \href{http://ctan.org/pkg/subfiles}{\textsf{subfiles}}
provide structures in which the main and child documents can be
encapsulated and allowing them to be compiled individually.
The inclusion mechanism is different from the conventional |\include|.
\item
The package \href{http://ctan.org/pkg/combine}{\textsf{combine}}
is an elaborate solution to combine several documents into one.
\end{itemize}
%
See also the CTAN topic \href{http://ctan.org/topic/subdocs}{\textsf{subdocs}}
for further related packages.
The present package differs from the above solutions in that
a document structure constructed with the conventional |\include| mechanism
just needs two extra commands at the top of every file
such that all constituent files can be compiled individually.

%%%%%%%%%%%%%%%%%%%%%%%%%%%%%%%%%%%%%%%%%%%%%%%%%%%%%%%%%%%%%%%%%%%%%%%%%%%%%%%%
%\subsection{Feature Suggestions}
%
%The following is a list of features which may be useful for future
%versions of this package:
%%
%\begin{itemize}
%\item
%\ldots
%\end{itemize}

%%%%%%%%%%%%%%%%%%%%%%%%%%%%%%%%%%%%%%%%%%%%%%%%%%%%%%%%%%%%%%%%%%%%%%%%%%%%%%%%
\subsection{Revision History}

%%%%%%%%%%%%%%%%%%%%%%%%%%%%%%%%%%%%%%%%
\paragraph{v2.0:} 2018/12/30

\begin{itemize}
\item
immediate forward processing
\item
added |\childdocby| mechanism
\item
manual restructured
\end{itemize}

%%%%%%%%%%%%%%%%%%%%%%%%%%%%%%%%%%%%%%%%
\paragraph{v1.6:} 2018/01/17

\begin{itemize}
\item
application for development of include files
\item
corrections to manual
\end{itemize}

%%%%%%%%%%%%%%%%%%%%%%%%%%%%%%%%%%%%%%%%
\paragraph{v1.5:} 2017/05/21

\begin{itemize}
\item
more complete structuring introduced
\item
|\childdocof| introduced
\item
|\childdoc| renamed to |\childdocmain|
\item
|\childredirect| renamed to |\childdocforward| and |\childdocforwardprefix|
and functionality expanded
\end{itemize}

%%%%%%%%%%%%%%%%%%%%%%%%%%%%%%%%%%%%%%%%
\paragraph{v1.0:} 2017/04/27

\begin{itemize}
\item
manual and install package
\item
first version published on CTAN
\end{itemize}

%%%%%%%%%%%%%%%%%%%%%%%%%%%%%%%%%%%%%%%%
\paragraph{v0.6:} 2017/04/26

\begin{itemize}
\item
redirection mechanism added
\end{itemize}

%%%%%%%%%%%%%%%%%%%%%%%%%%%%%%%%%%%%%%%%
\paragraph{v0.5:} 2017/04/26

\begin{itemize}
\item
functionality in definition file
\end{itemize}


%%%%%%%%%%%%%%%%%%%%%%%%%%%%%%%%%%%%%%%%%%%%%%%%%%%%%%%%%%%%%%%%%%%%%%%%%%%%%%%%
%%%%%%%%%%%%%%%%%%%%%%%%%%%%%%%%%%%%%%%%%%%%%%%%%%%%%%%%%%%%%%%%%%%%%%%%%%%%%%%%
%%%%%%%%%%%%%%%%%%%%%%%%%%%%%%%%%%%%%%%%%%%%%%%%%%%%%%%%%%%%%%%%%%%%%%%%%%%%%%%%
\appendix

\settowidth\MacroIndent{\rmfamily\scriptsize 000\ }

 \DocInput{childdoc.dtx}

\end{document}
%</driver>
% \fi
%
% %%%%%%%%%%%%%%%%%%%%%%%%%%%%%%%%%%%%%%%%%%%%%%%%%%%%%%%%%%%%%%%%%%%%%%%%%%%%%%
% %%%%%%%%%%%%%%%%%%%%%%%%%%%%%%%%%%%%%%%%%%%%%%%%%%%%%%%%%%%%%%%%%%%%%%%%%%%%%%
% \section{Sample}
%\iffalse
%<*samplemain>
%\fi
%
% The following presents a sample document
% with two chapters, two parts, a title page,
% a compile flag as well as three forwarding files to set the flag.
% It consists of eight |.tex| files:
% \begin{center}
% \begin{tabular}{ll}
% |cdocsamp.tex|&main file\\
% |cdocsch1.tex|&include file for chapter 1\\
% |cdocsch2.tex|&include file for chapter 2\\
% |cdocspt3.tex|&include file for part 3\\
% |cdocspt4.tex|&include file for part 4\\
% |cdocsdrf.tex|&forwarding file for main file in draft mode\\
% |cdocsfi1.tex|&forwarding file for final version of chapter 1\\
% |cdocsfi2.tex|&forwarding file for final version of chapter 2\\
% \end{tabular}
% \end{center}
% Each of the eight files can be compiled directly by the \LaTeX{} compiler.
%
% %%%%%%%%%%%%%%%%%%%%%%%%%%%%%%%%%%%%%%
% \paragraph{Main File.}
%
% The main file is called |cdocsamp.tex|.
%
% Load the \textsf{childdoc} definitions and
% declare the filename for the main document:
%    \begin{macrocode}
\input{childdoc.def}
\childdocmain{}
%    \end{macrocode}

% Optional override for |\version| flag:
%    \begin{macrocode}
%%\ifchilddoc\else\providecommand{\version}{draft}\fi
%    \end{macrocode}

% Define the default values for the |\version| flag
% (|final| for the main file and |draft| for childs):
%    \begin{macrocode}
\ifchilddoc
\providecommand{\version}{draft}
\else
\providecommand{\version}{final}
\fi
%    \end{macrocode}

% Load the standard document class:
%    \begin{macrocode}
\documentclass[12pt]{article}
%    \end{macrocode}

% Start the document body:
%    \begin{macrocode}
\begin{document}
%    \end{macrocode}

% Declare a title page.
% Print title, part of document being processed and version flag:
%    \begin{macrocode}
\addtocounter{page}{-1}
\begin{center}
{\LARGE\bfseries{}childdoc example\par}
\vspace{1cm}
\ifchilddoc
\ifchilddocmanual part\else chapter\fi:
`\childdocname' of `\childdocjob'\par
\else
main document: `\childdocjob'\par
\fi
version: \version\par
\end{center}
\newpage
%    \end{macrocode}

% Manually include selected file,
% otherwise process as usual:
%    \begin{macrocode}
\ifchilddocmanual
\section*{part `\childdocname'}
\input{\childdocname}
\else
%    \end{macrocode}

% Include the two chapters:
%    \begin{macrocode}
\include{cdocsch1}
\include{cdocsch2}
%    \end{macrocode}

% Include the two parts unless only chapters should be displayed:
%    \begin{macrocode}
\ifchilddoc\else
\section{part three}
\input{cdocspt3}
\section{part four}
\input{cdocspt4}
\fi
%    \end{macrocode}

% Process as usual until here:
%    \begin{macrocode}
\fi
%    \end{macrocode}

% End of document body:
%    \begin{macrocode}
\end{document}
%    \end{macrocode}
%\iffalse
%</samplemain>
%\fi
%
% %%%%%%%%%%%%%%%%%%%%%%%%%%%%%%%%%%%%%%
% \paragraph{Chapter Include Files.}
%
% The include files are called |cdocsch1.tex| and |cdocsch2.tex|.
%
%\iffalse
%<*samplechap1|samplechap2>
%\fi

% Optional override for |\version| flag:
%    \begin{macrocode}
%%\providecommand{\version}{final}
%    \end{macrocode}

% Include the main document:
%    \begin{macrocode}
\input{childdoc.def}
\childdocof{cdocsamp}
%    \end{macrocode}

%\iffalse
%</samplechap1|samplechap2>
%\fi
%
%\iffalse
%<*samplechap1>
%\fi
% Some text for chapter 1:
%    \begin{macrocode}
\section{one}
some text in chapter one
%    \end{macrocode}

%\iffalse
%</samplechap1>
%\fi
% Some text for chapter 2:
%\iffalse
%<*samplechap2>
%\fi
%    \begin{macrocode}
\section{two}
more text in chapter two
%    \end{macrocode}

%\iffalse
%</samplechap2>
%\fi
%
% %%%%%%%%%%%%%%%%%%%%%%%%%%%%%%%%%%%%%%
% \paragraph{Part Include Files.}
%
% The include files are called |cdocspt3.tex| and |cdocspt4.tex|.
%
%\iffalse
%<*samplepart3|samplepart4>
%\fi

% Optional override for |\version| flag:
%    \begin{macrocode}
%%\providecommand{\version}{final}
%    \end{macrocode}

% Include the main document:
%    \begin{macrocode}
\input{childdoc.def}
\childdocby{cdocsamp}
%    \end{macrocode}

%\iffalse
%</samplepart3|samplepart4>
%\fi
%
%\iffalse
%<*samplepart3>
%\fi
% Some text for part 3:
%    \begin{macrocode}
some text in part three
%    \end{macrocode}

%\iffalse
%</samplepart3>
%\fi
% Some text for part 4:
%\iffalse
%<*samplepart4>
%\fi
%    \begin{macrocode}
more text in part four
%    \end{macrocode}

%\iffalse
%</samplepart4>
%\fi
%
% %%%%%%%%%%%%%%%%%%%%%%%%%%%%%%%%%%%%%%
% \paragraph{Forwarding for a Complete Draft.}
%
% The following forwarding file |cdocsdrf.tex|
% compiles the main document in draft mode:
%\iffalse
%<*sampledraft>
%\fi
%    \begin{macrocode}
\def\version{draft}
\input{childdoc.def}
\childdocforward{cdocsamp}
%    \end{macrocode}

%\iffalse
%</sampledraft>
%\fi
%
% %%%%%%%%%%%%%%%%%%%%%%%%%%%%%%%%%%%%%%
% \paragraph{Forwarding for Final Version of the Chapters.}
%
% The following forwarding files |cdocsfn1.tex| and |cdocsfn2.tex|
% (with identical content)
% compile the final versions of the child documents
% |cdocsch1.tex| and |cdocsch2.tex|, respectively:
%\iffalse
%<*samplefinal>
%\fi
%    \begin{macrocode}
\def\version{final}
\input{childdoc.def}
\childdocforwardprefix[cdocsamp]{cdocsfn}{cdocsch}
%    \end{macrocode}

%\iffalse
%</samplefinal>
%\fi
%
% %%%%%%%%%%%%%%%%%%%%%%%%%%%%%%%%%%%%%%
% \paragraph{Command Line Processing.}
%
% The following three command lines generate the output files
% |cdocscld|, |cdocscl1| and |cdocscl2|
% which should be identical to
% |cdocsdrf|, |cdocsch1| and |cdocsfn2|, respectively:
% \begin{center}
% \begin{tabular}{l}
% |latex -jobname cdocscld \|\\
% |  "\def\version{draft}\input{childdoc.def}\childdocforward{cdocsamp}"|\\
% |latex -jobname cdocscl1 \|\\
% |  "\input{childdoc.def}\childdocforward[cdocsamp]{cdocsch1}"|\\
% |latex -jobname cdocscl2 \|\\
% |  "\def\version{final}\input{childdoc.def}\childdocforward{cdocsch2}"|
% \end{tabular}
% \end{center}
% Note that the trailing backslash on each first line
% merely continues the input to the second line
% (for convenient cut ant paste).
% Furthermore, the command |latex| can be replaced by any
% of its alternative versions such as |pdflatex|.
%
% %%%%%%%%%%%%%%%%%%%%%%%%%%%%%%%%%%%%%%%%%%%%%%%%%%%%%%%%%%%%%%%%%%%%%%%%%%%%%%
% %%%%%%%%%%%%%%%%%%%%%%%%%%%%%%%%%%%%%%%%%%%%%%%%%%%%%%%%%%%%%%%%%%%%%%%%%%%%%%
% \section{Implementation}
%\iffalse
%<*package>
%\fi
%
% This section describes the definitions file |childdoc.def|.

% The definitions cannot be loaded using |\usepackage| or |\RequirePackage|
% which has a mechanism to prevent loading a style file more than once.
% When loading the definitions by means of |\input|
% multiple instances have to be prevented manually:
%\iffalse
%This code needs to be before the `\ProvidesFile' directive
%which is defined at the beginning of this file.
%Therefore it is also placed there and commented out here.
%</package>
%<*discard>
%\fi
%    \begin{macrocode}
\ifdefined\childdocmain\endinput\fi
%    \end{macrocode}
%\iffalse
%</discard>
%<*package>
%\fi
%
% \macro{\ifchilddoc}
% \macro{\ifchilddocmanual}
% The conditional |\ifchilddoc| tells whether a
% child (true) or main (false) document is being compiled.
% The conditional |\ifchilddocmanual| tells whether
% the |\includeonly| mechanism is used (false) or
% the selection of child files must be performed manually (true).
% The definitions initialise to false:
%    \begin{macrocode}
\newif\ifchilddoc
\newif\ifchilddocmanual
%    \end{macrocode}

% \macro{\childdocname}
% \macro{\childdocjob}
% The macro |\childdocname| stores the name of the main document
% to be compiled. The macro |\childdocjob| stores the name of
% the document on which the \LaTeX{} compiler was originally invoked.
% The content of |\jobname| cannot be compared
% to filenames specified in the source due to different catcodes.
% The following code rescans |\jobname|, stores the result
% in |\childdocname| and saves a copy in |\childdocjob|:
%    \begin{macrocode}
\edef\childdocname{\scantokens\expandafter{\jobname\noexpand}}
\let\childdocjob\childdocname
%    \end{macrocode}

% \macro{\childdocdisable}
% The macro |\childdocdisable| prevents the main file
% from being processed more than once.
% At this stage, the main document command |\childdocmain|
% is assumed to be called once again where it should do nothing.
% Any subsequent call to it should prevent
% a secondary processing of the main document
% It overwrites the forwarding commands
% |\childdocof| and |\childdocforward|
% with empty macros to prevent further inclusions of the main document:
%    \begin{macrocode}
\newcommand{\childdocdisable}
{
  \renewcommand{\childdocmain}[1]{\renewcommand{\childdocmain}[1]{\endinput}}
  \renewcommand{\childdocof}[1]{}
  \renewcommand{\childdocby}[2][]{}
  \renewcommand{\childdocforward}[2][]{}
  \renewcommand{\childdocdisable}{}
}
%    \end{macrocode}

% \macro{\childdocmain}
% The macro |\childdocmain| is to be called at the top of the main file
% with nothing or the main filename (without extension) as argument.
% First, it breaks loops.
% If the argument is not empty and does not match |\childdocname|
% (which is set by the first inclusion of |childdoc.def|),
% |\ifchilddoc| is set to true, |\includeonly| is applied to the child file
% and |\jobname| is set to the main file
% (for proper handling of |.aux| files):
%    \begin{macrocode}
\newcommand{\childdocmain}[1]
{
  \childdocdisable\childdocmain{}
  \if?#1?\else
    \begingroup
      \def\childdoctmp{#1}
      \ifx\childdoctmp\childdocname
        \def\childdoctmp{}
      \else
        \def\childdoctmp
        {
          \childdoctrue
          \includeonly{\childdocname}
          \def\childdocjob{#1}
          \def\jobname{#1}
        }
      \fi
      \expandafter
    \endgroup
    \childdoctmp
  \fi
}
%    \end{macrocode}

% \macro{\childdocof}
% The command |\childdocof| redirects
% compilation to the main file |#1|.
%    \begin{macrocode}
\newcommand{\childdocof}[1]
{
  \childdocdisable
  \childdoctrue
  \includeonly{\childdocname}
  \def\jobname{#1}
  \def\childdocjob{#1}
  \input{#1}
}
%    \end{macrocode}

% \macro{\childdocby}
% The command |\childdocby| ....
%    \begin{macrocode}
\newcommand{\childdocby}[2][]
{
  \childdocdisable
  \childdoctrue
  \childdocmanualtrue
  \if?#1?\else
    \def\jobname{#2}
  \fi
  \def\childdocjob{#2}
  \input{#2}
  \endinput
}
%    \end{macrocode}

% \macro{\childdocforward}
% The command |\childdocforward| redirects
% compilation to the main file or
% (if the optional argument is given) a child file.
% Parameters are set as if the main file
% or a child file starting with |\childdocof| was compiled.
% Then compilation is handed over to the main file:
%    \begin{macrocode}
\newcommand{\childdocforward}[2][]
{
  \begingroup
    \if?#1?
      \def\childdoctmp
      {
        \def\childdocname{#2}
        \def\childdocjob{#2}
        \def\jobname{#2}
        \input{#2}
        \endinput
      }
    \else
      \def\childdoctmp
      {
        \childdocdisable
        \def\childdocname{#2}
        \childdoctrue
        \includeonly{#2}
        \def\childdocjob{#1}
        \def\jobname{#1}
        \input{#1}
        \endinput
      }
    \fi
    \expandafter
  \endgroup
  \childdoctmp
}
%    \end{macrocode}

% \macro{\childdocforwardprefix}
% The command |\childdocforwardprefix| redirects
% compilation to the main or a child file by means of a pattern.
% The prefix |#1| in the current filename is replaced by |#2|
% and the suffix of the current filename is kept
% (it is assumed that the filename does not contain the substring `|~~~|'
% which is used as a delimiter).
% Compilation is handed over to the new file by |\childdocforward|:
%    \begin{macrocode}
\newcommand{\childdocforwardprefix}[3][]
{
  \begingroup
    \def\childdocextract #2##1~~~{\def\childdoctmp{\childdocforward[#1]{#3##1}}}
    \expandafter\childdocextract\childdocname~~~
    \expandafter
  \endgroup
  \childdoctmp
}
%    \end{macrocode}

% \macro{\childdoc}
% The deprecated macro |\childdoc| is a legacy version of |\childdocmain|:
%    \begin{macrocode}
\newcommand{\childdoc}{\childdocmain}
%    \end{macrocode}

% \macro{\childdocredirect}
% The deprecated macro |\childdocredirect| is a legacy version
% of |\childdocforward| and |\childdocforwardprefix|:
%    \begin{macrocode}
\newcommand{\childdocredirect}[2][]
{
  \begingroup
    \if?#1?
      \def\childdoctmp{\childdocforward{#2}}
    \else
      \def\childdoctmp{\childdocforwardprefix{#1}{#2}}
    \fi
    \expandafter
  \endgroup
  \childdoctmp
}
%    \end{macrocode}

%\iffalse
%</package>
%\fi
%
\endinput
|\\
|\childdocforwardprefix[|\textit{main}|]{|\textit{prefix}|}{|\textit{dest}|}|
\end{tabular}
\end{center}
%
the destination file is determined by a pattern
depending on the current file:
To make this work, the current file must be called
`{\textit{prefix}\hspace{0.2em}\textit{suffix}}'
with \textit{prefix} matching precisely the argument.
Processing is then passed on to the file
`{\textit{dest}\hspace{0.2em}\textit{suffix}}'.
Surely, the same effect is achieved by
directly specifying the
argument `{\textit{dest}\hspace{0.2em}\textit{suffix}}'
in the first form.
However, that requires to set up a different file
for each child. With the alternative form of the command
all these files can have exactly the same content
which simplifies setting them up and maintaining them.

For example, the following file |draft.tex|
with a compilation flag |\version| as described in \secref{sec:flags}
compiles the main document as a draft:
%
\begin{center}
\begin{tabular}{l}
|\def\version{draft}|\\
|% \iffalse
%
% childdoc.dtx Copyright (C) 2017-2018 Niklas Beisert
%
% This work may be distributed and/or modified under the
% conditions of the LaTeX Project Public License, either version 1.3
% of this license or (at your option) any later version.
% The latest version of this license is in
%   http://www.latex-project.org/lppl.txt
% and version 1.3 or later is part of all distributions of LaTeX
% version 2005/12/01 or later.
%
% This work has the LPPL maintenance status `maintained'.
%
% The Current Maintainer of this work is Niklas Beisert.
%
% This work consists of the files childdoc.dtx and childdoc.ins
% and the derived files childdoc.def and cdocsamp.tex with
% cdocsch1.tex, cdocsch2.tex, cdocsdrf.tex, cdocsfn1.tex, cdocsfn2.tex.
%
%<package>\ifdefined\childdocmain\endinput\fi
%<package>\ProvidesFile{childdoc.def}[2018/12/30 v2.0 child document driver]
%<samplemain>\ProvidesFile{cdocsamp.tex}[2018/12/30 v2.0 sample for childdoc]
%<*driver>
%\ProvidesFile{childdoc.drv}[2018/12/30 v2.0 childdoc reference manual file]
\PassOptionsToClass{10pt,a4paper}{article}
\documentclass{ltxdoc}

\usepackage[margin=35mm]{geometry}
\usepackage{hyperref}
\usepackage{hyperxmp}
\usepackage[usenames]{color}

\hypersetup{colorlinks=true}
\hypersetup{pdfstartview=FitH}
\hypersetup{pdfpagemode=UseNone}
\hypersetup{pdfsource={}}
\hypersetup{pdflang={en-UK}}
\hypersetup{pdfcopyright={Copyright 2017-2018 Niklas Beisert.
  This work may be distributed and/or modified under the
  conditions of the LaTeX Project Public License, either version 1.3
  of this license or (at your option) any later version.}}
\hypersetup{pdflicenseurl={http://www.latex-project.org/lppl.txt}}
\hypersetup{pdfcontactaddress={ETH Zurich, ITP, HIT K,
  Wolfgang-Pauli-Strasse 27}}
\hypersetup{pdfcontactpostcode={8093}}
\hypersetup{pdfcontactcity={Zurich}}
\hypersetup{pdfcontactcountry={Switzerland}}
\hypersetup{pdfcontactemail={nbeisert@itp.phys.ethz.ch}}
\hypersetup{pdfcontacturl={http://people.phys.ethz.ch/\xmptilde nbeisert/}}

\newcommand{\secref}[1]{\hyperref[#1]{section \ref*{#1}}}

\parskip1ex
\parindent0pt
\let\olditemize\itemize
\def\itemize{\olditemize\parskip0pt}

\begin{document}

\title{The \textsf{childdoc} Package}
\hypersetup{pdftitle={The childdoc Package}}
\author{Niklas Beisert\\[2ex]
  Institut f\"ur Theoretische Physik\\
  Eidgen\"ossische Technische Hochschule Z\"urich\\
  Wolfgang-Pauli-Strasse 27, 8093 Z\"urich, Switzerland\\[1ex]
  \href{mailto:nbeisert@itp.phys.ethz.ch}
  {\texttt{nbeisert@itp.phys.ethz.ch}}}
\hypersetup{pdfauthor={Niklas Beisert}}
\hypersetup{pdfsubject={Manual for the LaTeX2e Package childdoc}}
\date{30 December 2018, \textsf{v2.0}}
\maketitle

\begin{abstract}\noindent
\textsf{childdoc} is a \LaTeXe{} package
that enables the direct compilation
of document sections included by |\include|
to individual files.
\end{abstract}

\begingroup
\parskip0ex
\tableofcontents
\endgroup

%%%%%%%%%%%%%%%%%%%%%%%%%%%%%%%%%%%%%%%%%%%%%%%%%%%%%%%%%%%%%%%%%%%%%%%%%%%%%%%%
%%%%%%%%%%%%%%%%%%%%%%%%%%%%%%%%%%%%%%%%%%%%%%%%%%%%%%%%%%%%%%%%%%%%%%%%%%%%%%%%
\section{Introduction}

\LaTeX{} provides a mechanism to structure a large document (such as a book)
into a main file and several child files (containing the chapters)
using the |\include| command.
This mechanism is beneficial for documents
which span hundreds of pages in order to
make the source file(s) more manageable.
Moreover, compilation can be restricted to
selected child files by means of the |\includeonly| command.
The latter feature can be used to reduce the compilation time while editing
(this was significantly more useful in the earlier days of \LaTeX{})
or to generate a smaller document which is easier to navigate.
Another application of |\includeonly| is to generate
documents consisting of selected parts of the complete document.

However, there are a few drawbacks of the plain |\include| mechanism:
\begin{itemize}
\item
The child files cannot be compiled on their own,
they can only be compiled via the main file.
A naive editing environment
(such as a text editor with an option
to have the current file processed by \LaTeX)
may require one to switch to the main file before compiling;
attempting to compile the child file produces errors.
\item
The main file must be modified (each time)
to adjust the |\includeonly| command
to the present needs. This easily leaves the main file in a messy state.
\item
The generated document will always carry the filename
of the main document. This is inconvenient if
several child files are to be compiled and
to be kept for distribution.
\end{itemize}

The present package provides a simple interface
to make child files individually compilable by \LaTeX{}.
Compiling a child file then has the same effect as compiling
the main file with an |\includeonly| command
to select the appropriate child.
Moreover the generated document will carry the name of the child
rather than the main file.
This resolves all three above issues.

This feature is meant to make the editing of books,
thesis documents and lecture notes somewhat more convenient.
However, the package can also be used efficiently for
composing a series of documents (such as exercise sheets)
which are typically distributed individually.
It then assists the author in generating the individual documents
(potentially in different versions)
as well as a document containing the collected series.
Another application is in developing style files
or other kinds of included material
where compilation of the style file could redirect
to a sample or test file.

%%%%%%%%%%%%%%%%%%%%%%%%%%%%%%%%%%%%%%%%%%%%%%%%%%%%%%%%%%%%%%%%%%%%%%%%%%%%%%%%
%%%%%%%%%%%%%%%%%%%%%%%%%%%%%%%%%%%%%%%%%%%%%%%%%%%%%%%%%%%%%%%%%%%%%%%%%%%%%%%%
\section{Usage}

First of all, the package \textsf{childdoc} is \emph{not} a standard
\LaTeXe{} |.sty| style file! Therefore it needs to be invoked in
a non-standard way.

%%%%%%%%%%%%%%%%%%%%%%%%%%%%%%%%%%%%%%%%%%%%%%%%%%%%%%%%%%%%%%%%%%%%%%%%%%%%%%%%
\subsection{Included Files}
\label{sec:include}

%%%%%%%%%%%%%%%%%%%%%%%%%%%%%%%%%%%%%%%%
\DescribeMacro{\childdocmain}
To use the package, add the commands
\begin{center}
\begin{tabular}{l}
|\input{childdoc.def}|\\
|\childdocmain{}|\\
\end{tabular}
\end{center}
at the very top of the main \LaTeX{} file,
in particular \emph{before} the |\documentclass| statement!
The argument of |\childdocmain| should be left empty
(but it must be present).

%%%%%%%%%%%%%%%%%%%%%%%%%%%%%%%%%%%%%%%%
\DescribeMacro{\childdocof}
Furthermore, add the commands
\begin{center}
\begin{tabular}{l}
|\input{childdoc.def}|\\
|\childdocof{|\textit{main}|}|\\
\end{tabular}
\end{center}
at the top of every child file \textit{child}
which is included by |\include{|\textit{child}|}|
from within the main file
(or at least for those files to be compiled individually).
The argument \textit{main} must be the filename of the main file.

There are a couple of
considerations in setting up the main and child documents:

%%%%%%%%%%%%%%%%%%%%%%%%%%%%%%%%%%%%%%%%
\paragraph{Restrictions.}

Please note the following restrictions:
\begin{itemize}
\item
|\childdocmain| must be called with one argument \textit{main}
to ensure compatibility with earlier version of the package.
It must either be empty (|\childdocmain{}|)
or precisely match the filename of the main file in which it is specified.
See \secref{sec:detection} for further information.
\item
The filename \textit{main} must be specified without the |.tex| extension.
\item
The filename \textit{main} is case sensitive
(even in case-insensitive file systems)
due to internal string comparison.
\item
The argument \textit{main} should be fully expanded, it cannot be a macro.
\item
Subdirectories and special characters should be avoided in filenames.
\item
The command |\childdocmain{|\textit{main}|}| must be followed by a whitespace.
It should not be followed immediately by another command
or by a comment mark `|%|'.
This is because the \TeX{} parser reads the token immediately following
the argument of |\childdocmain| and puts it
at the beginning of every child section;
however, a white\-space is ignored.
\end{itemize}

%%%%%%%%%%%%%%%%%%%%%%%%%%%%%%%%%%%%%%%%
\paragraph{Content of Main File.}

It is advisable to place all content in the child files included by |\include|.
Any output contained in the main file will appear in all child documents
unless suppressed manually;
it cannot be suppressed automatically by the |\includeonly| directive
and thus should normally be avoided.
A method to include some content in the main file
by means of conditional processing is described in \secref{sec:conditional}.

%%%%%%%%%%%%%%%%%%%%%%%%%%%%%%%%%%%%%%%%
\paragraph{Page Numbering.}

When only a part of the document is compiled,
the appropriate numbering of pages
(as well as other status parameters)
is determined from the |.aux| files.
The latter contain information from previous passes.
However this information needs to propagate through
all intermediate child documents.
Therefore the page numbering in child documents may well
be inconsistent until the complete document is compiled at least once.

A useful (if unconventional) way to always ensure a consistent
page numbering is to restart the numbering in each child document
and denote the pages by `\textit{child}|.|\textit{page}'
where \textit{child} represents the chapter/section number of the child file.
This can be achieved by the command
|\numberwithin{page}{|\textit{child}|}|
of the \textsf{amsmath} package
where \textit{child} can be |chapter| or |section|
depending on the chosen structuring.
Alternatively, one can modify the macro |\thepage| appropriately
and reset the counter |page| at the start of each child file.

%%%%%%%%%%%%%%%%%%%%%%%%%%%%%%%%%%%%%%%%%%%%%%%%%%%%%%%%%%%%%%%%%%%%%%%%%%%%%%%%
\subsection{Conditional Processing}
\label{sec:conditional}

The package provides a mechanism to compile different versions
of a document. To customise the versions further some conditional processing
can come in handy to distinguish which version is being compiled.
The package provides two macros to describe the compilation context:

%%%%%%%%%%%%%%%%%%%%%%%%%%%%%%%%%%%%%%%%
\DescribeMacro{\ifchilddoc}
The conditional |\ifchilddoc| distinguishes between the compilation of
child documents and the main document:
%
\begin{center}
|\ifchilddoc |\textit{child-code}| |[|\||else |\textit{main-code}]| \||fi|
\end{center}

%%%%%%%%%%%%%%%%%%%%%%%%%%%%%%%%%%%%%%%%
\DescribeMacro{\childdocname}
\DescribeMacro{\childdocjob}
The macro |\childdocname| contains the filename (without extension)
of the main or child file being processed.
Note that |\childdocjob| will always contain the name of the main file.

%%%%%%%%%%%%%%%%%%%%%%%%%%%%%%%%%%%%%%%%
\paragraph{Title Page.}

Conditional processing can be used to include a title or banner page
in the main document when proper precautions are taken.
Importantly, the code in the main file should ensure that the page counter
(as well as other status parameters which are stored in the |.aux| files)
takes the same value after the conditional processing.
Otherwise the page numbers may take divergent values
depending on which part is compiled.

For example, a title page could be declared by:
%
\begin{center}
\begin{tabular}{l}
|\ifchilddoc\||else|\\
|\addtocounter{page}{-1}|\\
\textit{code for title page}\\
|\newpage|\\
|\||fi|
\end{tabular}
\end{center}
%
A banner page for the child documents can be generated by:
%
\begin{center}
\begin{tabular}{l}
|\ifchilddoc|\\
|\addtocounter{page}{-1}|\\
\textit{code for banner page}\\
|\newpage|\\
|\||fi|
\end{tabular}
\end{center}
%
Here one could write a message such as:
\begin{center}
|This is the part \childdocname{} of \childdocjob{}.|
\end{center}

%%%%%%%%%%%%%%%%%%%%%%%%%%%%%%%%%%%%%%%%%%%%%%%%%%%%%%%%%%%%%%%%%%%%%%%%%%%%%%%%
\subsection{Flags}
\label{sec:flags}

The package makes it easy to generate different versions
of the main or child documents.
To this end compilation flags can be defined
and assigned different default values.
They will be particularly useful in conjunction
with the forwarding mechanism described in \secref{sec:forward}.

For example, it may be useful to have a flag |\version|
which can be set to |draft| or |final|.
The document source will contain some conditional code
depending on the value of |\version|.
Suppose further, the flag should default to |final| for the main file
and to |draft| for child files
which is a natural assignment for editing the document.
This is achieved by placing the following code
in the preamble of the main document
(below the |\childdocmain| directive):
%
\begin{center}
\begin{tabular}{l}
|\ifchilddoc|\\
|\providecommand{\version}{draft}|\\
|\||else|\\
|\providecommand{\version}{final}|\\
|\||fi|
\end{tabular}
\end{center}
%
The definition by |\providecommand| makes sure
that previous definitions are not overwritten.
Further statements |\providecommand{\version}{...}|
can thus be added before the above code to override it.

For the main file, one might add a line
(between |\childdocmain| and the above block)
%
\begin{center}
|%\ifchilddoc\||else\providecommand{\version}{draft}\||fi|
\end{center}
%
which can be uncommented to produce a draft version.
Likewise one can add a line to the very top of a child file
(above the |\childdocof{|\textit{main}|}| directive)
%
\begin{center}
|%\providecommand{\version}{final}|
\end{center}
%
which can be uncommented to produce the final version of this child document.

%%%%%%%%%%%%%%%%%%%%%%%%%%%%%%%%%%%%%%%%%%%%%%%%%%%%%%%%%%%%%%%%%%%%%%%%%%%%%%%%
\subsection{Forwarding}
\label{sec:forward}

Different versions of the main or child documents
using compilation flags as described in \secref{sec:flags}
can be (permanently) stored in different files
for convenient compilation, viewing and distribution.
To this end, the package defines a command
to pass on compilation to a different file:

%%%%%%%%%%%%%%%%%%%%%%%%%%%%%%%%%%%%%%%%
\DescribeMacro{\childdocforward}
The command |\childdocforward| redirects processing to
another source file:
%
\begin{center}
\begin{tabular}{l}
|\input{childdoc.def}|\\
|\childdocforward[|\textit{main}|]{|\textit{dest}|}|\\
\end{tabular}
\end{center}
%
The argument \textit{dest} is the destination file
(without extension).
It should be the main file or one of the child files.
Note that further \textsf{childdoc} directives
such as |\childdocof| and |\childdocforward|
in the indicated file will be processed in this form.
The optional argument \textit{main}
passes on directly to the main file \textit{main}
while pretending to compile the child \textit{dest}.
This form behaves as if \textit{dest}
issues |\childdocof{|\textit{main}|}| right away,
and no further \textsf{childdoc} directives will be processed.

%%%%%%%%%%%%%%%%%%%%%%%%%%%%%%%%%%%%%%%%
\DescribeMacro{\...prefix}
In the alternative form |\childdocforwardprefix|,
%
\begin{center}
\begin{tabular}{l}
|\input{childdoc.def}|\\
|\childdocforwardprefix[|\textit{main}|]{|\textit{prefix}|}{|\textit{dest}|}|
\end{tabular}
\end{center}
%
the destination file is determined by a pattern
depending on the current file:
To make this work, the current file must be called
`{\textit{prefix}\hspace{0.2em}\textit{suffix}}'
with \textit{prefix} matching precisely the argument.
Processing is then passed on to the file
`{\textit{dest}\hspace{0.2em}\textit{suffix}}'.
Surely, the same effect is achieved by
directly specifying the
argument `{\textit{dest}\hspace{0.2em}\textit{suffix}}'
in the first form.
However, that requires to set up a different file
for each child. With the alternative form of the command
all these files can have exactly the same content
which simplifies setting them up and maintaining them.

For example, the following file |draft.tex|
with a compilation flag |\version| as described in \secref{sec:flags}
compiles the main document as a draft:
%
\begin{center}
\begin{tabular}{l}
|\def\version{draft}|\\
|\input{childdoc.def}|\\
|\childdocforward{|\textit{main}|}|
\end{tabular}
\end{center}
%
Likewise, the following files |final|\textit{nn}|.tex|
compile the final version of the child document
|child|\textit{nn}|.tex|:
%
\begin{center}
\begin{tabular}{l}
|\def\version{final}|\\
|\input{childdoc.def}|\\
|\childdocforwardprefix{final}{child}|
\end{tabular}
\end{center}
%

Note that when several versions of a main file and/or of each child file
are to be generated, it may be convenient to set up a |Makefile| or
shell script to automatise the process.

%%%%%%%%%%%%%%%%%%%%%%%%%%%%%%%%%%%%%%%%%%%%%%%%%%%%%%%%%%%%%%%%%%%%%%%%%%%%%%%%
\subsection{Command Line Processing}
\label{sec:commandline}

The effect of redirection files can also be achieved by invoking
the \LaTeX{} compiler with a more elaborate command line.
Most conveniently this should be done as part
of a shell script or a |Makefile|.

When using \textsf{childdoc} in the main file, the following
command lines effectively perform a redirection
(note that depending on the shell being used,
backslashes may have to be doubled: `|\|' $\to$ `|\\|'):
%
\begin{center}
|... -jobname "|\textit{target}|" |\\|"|[\textit{flags}]%
|\input{childdoc.def}\childdocforward[|\textit{main}|]{|\textit{dest}|}"|
\end{center}
%
Here \textit{target} is the name of the output file,
\textit{main} is the name of the main file
and \textit{dest} is the name of the main or child file to be processed
(all filenames without extensions).
The optional argument \textit{main} can be omitted
if \textit{main} matches \textit{dest}.
Optionally, compilation \textit{flags} can be defined via |\def| commands.
This command line makes the \TeX{} engine believe
it is compiling the file \textit{target}
whose content is specified as the latter parameter.
The provided code then forwards the processing to
\textit{main} or \textit{dest} as described in \secref{sec:forward}.

%%%%%%%%%%%%%%%%%%%%%%%%%%%%%%%%%%%%%%%%%%%%%%%%%%%%%%%%%%%%%%%%%%%%%%%%%%%%%%%%
\subsection{Include by Input}
\label{sec:input}

Including child documents by |\include| has some restrictions by design.
Most notably, the content of a child document always occupies
its own set of pages; pages cannot be shared between child documents.
Usually, this behaviour makes perfect sense
because each child document contain an essential part of the document.
However, in some situations it may be desirable to compose
a document from a collection of parts
without having mandatory page breaks between then.
For this case, the package
provides a mechanism to include parts
by |\input| which can also be processed individually.
However, by construction this mechanism
requires manual handling of the content to be output.

%%%%%%%%%%%%%%%%%%%%%%%%%%%%%%%%%%%%%%%%
\DescribeMacro{\ifchilddocmanual}
The main file should be prepared as usual, see \secref{sec:include}.
However, the document body must make a distinction
between processing of an individual part and of the main document, e.g.:
%
\begin{center}
\begin{tabular}{l}
|\ifchilddocmanual|\\
|\input{\childdocname}|\\
|\||else|\\
\textit{document body with }|\input{|\textit{part}|}|\\
|\||fi|
\end{tabular}
\end{center}
%
The conditional |\ifchilddocmanual| is true whenever
a part to be included by |\input| is being compiled,
and the name of the part is stored in |\childdocname|.

%%%%%%%%%%%%%%%%%%%%%%%%%%%%%%%%%%%%%%%%
\DescribeMacro{\childdocby}
Each part to be included by |\input| should start with:
%
\begin{center}
\begin{tabular}{l}
|\input{childdoc.def}|\\
|\childdocby{|\textit{main}|}|\\
\end{tabular}
\end{center}
%
The directive |\childdocby| is similar to |\childdocof|
described in \secref{sec:include},
but the subsequent selection of content must be done manually.
To that end, both |\ifchilddoc| and |\ifchilddocmanual|
will be true upon processing of a part,
and the name of the part is stored in |\childdocname|.
Note that |\jobname| will be set to the filename of the current part
so that each part receives an individual |.aux| file
that does not interfere with the |.aux| file(s) of the main document.
This behaviour can be altered by the alternative form
|\childdocby[*]{|\textit{main}|}| (with a non-empty optional argument)
which uses the |.aux| file of the main document
by setting |\jobname| to \textit{main}.

%%%%%%%%%%%%%%%%%%%%%%%%%%%%%%%%%%%%%%%%%%%%%%%%%%%%%%%%%%%%%%%%%%%%%%%%%%%%%%%%
\subsection{Driver Development}
\label{sec:driver}

The \textsf{childdoc} mechanism can also be use for the development
of definition files such as \LaTeX{} styles or classes.
This case differs from the above setup with multiple parts
included by |\include| in that no |\includeonly| should be invoked.
This can be achieved by starting the include file
(before |\ProvidesPackage|) with:
%
\begin{center}
\begin{tabular}{l}
|\input{childdoc.def}|\\
|\childdocforward{|\textit{main}|}|\\
\end{tabular}
\end{center}
%
or alternatively with:
%
\begin{center}
\begin{tabular}{l}
|\input{childdoc.def}|\\
|\childdocby{|\textit{main}|}|\\
\end{tabular}
\end{center}
%
Both forms have slightly different effects as described above.
The main file is prepared as usual, see \secref{sec:include}.

%%%%%%%%%%%%%%%%%%%%%%%%%%%%%%%%%%%%%%%%%%%%%%%%%%%%%%%%%%%%%%%%%%%%%%%%%%%%%%%%
\subsection{Legacy Detection}
\label{sec:detection}

The directive |\childdocmain| in the main file can detect
whether the complete document or merely a child is to be compiled
even without using the directive |\childdocof|.
This method is deprecated because it is less robust
and there is no compelling reason to use it;
it is merely provided for backward compatibility
and it may be removed in future versions.

If the detection mechanism is to be used,
it is mandatory to correctly specify
the filename of the main file as the argument of |\childdocmain|:
%
\begin{center}
\begin{tabular}{l}
|\input{childdoc.def}|\\
|\childdocmain{|\textit{main}|}|\\
\end{tabular}
\end{center}
%
If |\jobname| does not match the argument \textit{main} of |\childdocmain|,
it is assumed that |\jobname| points to the child file to be compiled.
When using |\childdocmain| with the main file specified as argument,
it suffices to start a child file
with just |\input{|\textit{main}|}|
without loading of the package and using |\childdocof|.
If instead all processing is done
with the appropriate \textsf{childdoc} directives,
the argument of \textit{main} of |\childdocmain| can be empty.

An alternative version of the command line processing described
in \secref{sec:commandline} using the detection mechanism reads:
%
\begin{center}
|... -jobname "|\textit{target}|" "|[\textit{flags}]%
[|\def\jobname{|\textit{dest}|}|]|\input{|\textit{main}|}"|
\end{center}

%%%%%%%%%%%%%%%%%%%%%%%%%%%%%%%%%%%%%%%%%%%%%%%%%%%%%%%%%%%%%%%%%%%%%%%%%%%%%%%%
\subsection{Manual Code}
\label{sec:manual}

In case one cannot be certain whether the definitions file |childdoc.def|
is installed on the target \TeX{} distribution
and one prefers not to ship it,
it is conceivable to paste a few relevant commands into the sources.

To that end, drop all statements |\input{childdoc.def}|
and perform the replacements as outlined below.
Instead of |\childdocmain{|\textit{main}|}| add the following code
to the top of the main file:
%
\begin{center}
\begin{tabular}{l}
|\||ifdefined\childdocname\endinput\||fi\newif\ifchilddoc|\\
|\edef\childdocname{\scantokens\expandafter{\jobname\noexpand}}|\\
|\def\childdocmain{|\textit{main}|}\||ifx\childdocmain\childdocname\||else|\\
|\childdoctrue\includeonly{\childdocname}\let\jobname\childdocmain\||fi|\\
\end{tabular}
\end{center}
%
Instead of |\childdocof{|\textit{main}|}| just include the main file
at the top of each child file:
%
\begin{center}
|\input{|\textit{main}|}|
\end{center}
%
A simple redirection |\childdocforward{|\textit{dest}|}| is achieved by:
%
\begin{center}
|\def\jobname{|\textit{dest}|}\input{\jobname}|
\end{center}
%
The redirection with prefix
|\childdocforwardprefix[|\textit{prefix}|]{|\textit{dest}|}|
is accomplished by:
%
\begin{center}
\begin{tabular}{l}
|{\edef\jobname{\scantokens\expandafter{\jobname\noexpand}}|\\
|\def\redirectjob |\textit{prefix}|#1~~~{\gdef\jobname{|\textit{dest}|#1}}|\\
|\expandafter\redirectjob\jobname~~~}\input{\jobname}|
\end{tabular}
\end{center}

In an alternative approach,
child documents can be compiled by a specific command line
without additional code or specific definitions:
%
\begin{center}
|... -jobname "|\textit{target}|" "|[\textit{flags}]%
|\includeonly{|\textit{dest}|}\input{|\textit{main}|}"|
\end{center}
%

%%%%%%%%%%%%%%%%%%%%%%%%%%%%%%%%%%%%%%%%%%%%%%%%%%%%%%%%%%%%%%%%%%%%%%%%%%%%%%%%
%%%%%%%%%%%%%%%%%%%%%%%%%%%%%%%%%%%%%%%%%%%%%%%%%%%%%%%%%%%%%%%%%%%%%%%%%%%%%%%%
\section{Information}

%%%%%%%%%%%%%%%%%%%%%%%%%%%%%%%%%%%%%%%%%%%%%%%%%%%%%%%%%%%%%%%%%%%%%%%%%%%%%%%%
\subsection{Copyright}

Copyright \copyright{} 2017--2018 Niklas Beisert

This work may be distributed and/or modified under the
conditions of the \LaTeX{} Project Public License, either version 1.3
of this license or (at your option) any later version.
The latest version of this license is in
  \url{http://www.latex-project.org/lppl.txt}
and version 1.3 or later is part of all distributions of \LaTeX{}
version 2005/12/01 or later.

This work has the LPPL maintenance status `maintained'.

The Current Maintainer of this work is Niklas Beisert.

This work consists of the files |README.txt|, |childdoc.ins| and |childdoc.dtx|
as well as the derived files |childdoc.def|, |cdocsamp.tex|
with |cdocsch1.tex|, |cdocsch2.tex|, |cdocspt3.tex|, |cdocspt4.tex|,
|cdocsdrf.tex|, |cdocsfn1.tex|, |cdocsfn2.tex|
as well as |childdoc.pdf|.

%%%%%%%%%%%%%%%%%%%%%%%%%%%%%%%%%%%%%%%%%%%%%%%%%%%%%%%%%%%%%%%%%%%%%%%%%%%%%%%%
\subsection{Files and Installation}

The package consists of the files:
%
\begin{center}
\begin{tabular}{ll}
    |README.txt|   & readme file \\
    |childdoc.ins| & installation file \\
    |childdoc.dtx| & source file \\
    |childdoc.def| & definition file \\
    |cdocsamp.tex| & sample main file \\
    |cdocsch1.tex| & sample include file \\
    |cdocsch2.tex| & sample include file \\
    |cdocspt3.tex| & sample part file \\
    |cdocspt4.tex| & sample part file \\
    |cdocsdrf.tex| & sample redirection file \\
    |cdocsfn1.tex| & sample redirection file \\
    |cdocsfn2.tex| & sample redirection file \\
    |childdoc.pdf| & manual
\end{tabular}
\end{center}
%
The distribution consists of the files
|README.txt|, |childdoc.ins| and |childdoc.dtx|.
%
\begin{itemize}
\item
Run (pdf)\LaTeX{} on |childdoc.dtx|
to compile the manual |childdoc.pdf| (this file).
\item
Run \LaTeX{} on |childdoc.ins| to create the definitions file |childdoc.def|
and the sample |cdocsamp.tex| with include files
|cdocsch1.tex|, |cdocsch2.tex|, |cdocspt3.tex|, |cdocspt4.tex|,
|cdocsdrf.tex|, |cdocsfn1.tex|, |cdocsfn2.tex|.
Then copy the file |childdoc.def| to an appropriate directory of your \LaTeX{}
distribution, e.g.\ \textit{texmf-root}|/tex/latex/childdoc|.
\end{itemize}

%%%%%%%%%%%%%%%%%%%%%%%%%%%%%%%%%%%%%%%%%%%%%%%%%%%%%%%%%%%%%%%%%%%%%%%%%%%%%%%%
\subsection{Related CTAN Packages}

There are several other packages which offer a similar functionality:
%
\begin{itemize}
\item
The packages
\href{http://ctan.org/pkg/docmute}{\textsf{docmute}},
\href{http://ctan.org/pkg/includex}{\textsf{includex}} and
\href{http://ctan.org/pkg/standalone}{\textsf{standalone}}
provide commands to include only the document body of
a child file thus allowing both files to be compiled individually.
\item
The packages \href{http://ctan.org/pkg/subdocs}{\textsf{subdocs}}
and \href{http://ctan.org/pkg/subfiles}{\textsf{subfiles}}
provide structures in which the main and child documents can be
encapsulated and allowing them to be compiled individually.
The inclusion mechanism is different from the conventional |\include|.
\item
The package \href{http://ctan.org/pkg/combine}{\textsf{combine}}
is an elaborate solution to combine several documents into one.
\end{itemize}
%
See also the CTAN topic \href{http://ctan.org/topic/subdocs}{\textsf{subdocs}}
for further related packages.
The present package differs from the above solutions in that
a document structure constructed with the conventional |\include| mechanism
just needs two extra commands at the top of every file
such that all constituent files can be compiled individually.

%%%%%%%%%%%%%%%%%%%%%%%%%%%%%%%%%%%%%%%%%%%%%%%%%%%%%%%%%%%%%%%%%%%%%%%%%%%%%%%%
%\subsection{Feature Suggestions}
%
%The following is a list of features which may be useful for future
%versions of this package:
%%
%\begin{itemize}
%\item
%\ldots
%\end{itemize}

%%%%%%%%%%%%%%%%%%%%%%%%%%%%%%%%%%%%%%%%%%%%%%%%%%%%%%%%%%%%%%%%%%%%%%%%%%%%%%%%
\subsection{Revision History}

%%%%%%%%%%%%%%%%%%%%%%%%%%%%%%%%%%%%%%%%
\paragraph{v2.0:} 2018/12/30

\begin{itemize}
\item
immediate forward processing
\item
added |\childdocby| mechanism
\item
manual restructured
\end{itemize}

%%%%%%%%%%%%%%%%%%%%%%%%%%%%%%%%%%%%%%%%
\paragraph{v1.6:} 2018/01/17

\begin{itemize}
\item
application for development of include files
\item
corrections to manual
\end{itemize}

%%%%%%%%%%%%%%%%%%%%%%%%%%%%%%%%%%%%%%%%
\paragraph{v1.5:} 2017/05/21

\begin{itemize}
\item
more complete structuring introduced
\item
|\childdocof| introduced
\item
|\childdoc| renamed to |\childdocmain|
\item
|\childredirect| renamed to |\childdocforward| and |\childdocforwardprefix|
and functionality expanded
\end{itemize}

%%%%%%%%%%%%%%%%%%%%%%%%%%%%%%%%%%%%%%%%
\paragraph{v1.0:} 2017/04/27

\begin{itemize}
\item
manual and install package
\item
first version published on CTAN
\end{itemize}

%%%%%%%%%%%%%%%%%%%%%%%%%%%%%%%%%%%%%%%%
\paragraph{v0.6:} 2017/04/26

\begin{itemize}
\item
redirection mechanism added
\end{itemize}

%%%%%%%%%%%%%%%%%%%%%%%%%%%%%%%%%%%%%%%%
\paragraph{v0.5:} 2017/04/26

\begin{itemize}
\item
functionality in definition file
\end{itemize}


%%%%%%%%%%%%%%%%%%%%%%%%%%%%%%%%%%%%%%%%%%%%%%%%%%%%%%%%%%%%%%%%%%%%%%%%%%%%%%%%
%%%%%%%%%%%%%%%%%%%%%%%%%%%%%%%%%%%%%%%%%%%%%%%%%%%%%%%%%%%%%%%%%%%%%%%%%%%%%%%%
%%%%%%%%%%%%%%%%%%%%%%%%%%%%%%%%%%%%%%%%%%%%%%%%%%%%%%%%%%%%%%%%%%%%%%%%%%%%%%%%
\appendix

\settowidth\MacroIndent{\rmfamily\scriptsize 000\ }

 \DocInput{childdoc.dtx}

\end{document}
%</driver>
% \fi
%
% %%%%%%%%%%%%%%%%%%%%%%%%%%%%%%%%%%%%%%%%%%%%%%%%%%%%%%%%%%%%%%%%%%%%%%%%%%%%%%
% %%%%%%%%%%%%%%%%%%%%%%%%%%%%%%%%%%%%%%%%%%%%%%%%%%%%%%%%%%%%%%%%%%%%%%%%%%%%%%
% \section{Sample}
%\iffalse
%<*samplemain>
%\fi
%
% The following presents a sample document
% with two chapters, two parts, a title page,
% a compile flag as well as three forwarding files to set the flag.
% It consists of eight |.tex| files:
% \begin{center}
% \begin{tabular}{ll}
% |cdocsamp.tex|&main file\\
% |cdocsch1.tex|&include file for chapter 1\\
% |cdocsch2.tex|&include file for chapter 2\\
% |cdocspt3.tex|&include file for part 3\\
% |cdocspt4.tex|&include file for part 4\\
% |cdocsdrf.tex|&forwarding file for main file in draft mode\\
% |cdocsfi1.tex|&forwarding file for final version of chapter 1\\
% |cdocsfi2.tex|&forwarding file for final version of chapter 2\\
% \end{tabular}
% \end{center}
% Each of the eight files can be compiled directly by the \LaTeX{} compiler.
%
% %%%%%%%%%%%%%%%%%%%%%%%%%%%%%%%%%%%%%%
% \paragraph{Main File.}
%
% The main file is called |cdocsamp.tex|.
%
% Load the \textsf{childdoc} definitions and
% declare the filename for the main document:
%    \begin{macrocode}
\input{childdoc.def}
\childdocmain{}
%    \end{macrocode}

% Optional override for |\version| flag:
%    \begin{macrocode}
%%\ifchilddoc\else\providecommand{\version}{draft}\fi
%    \end{macrocode}

% Define the default values for the |\version| flag
% (|final| for the main file and |draft| for childs):
%    \begin{macrocode}
\ifchilddoc
\providecommand{\version}{draft}
\else
\providecommand{\version}{final}
\fi
%    \end{macrocode}

% Load the standard document class:
%    \begin{macrocode}
\documentclass[12pt]{article}
%    \end{macrocode}

% Start the document body:
%    \begin{macrocode}
\begin{document}
%    \end{macrocode}

% Declare a title page.
% Print title, part of document being processed and version flag:
%    \begin{macrocode}
\addtocounter{page}{-1}
\begin{center}
{\LARGE\bfseries{}childdoc example\par}
\vspace{1cm}
\ifchilddoc
\ifchilddocmanual part\else chapter\fi:
`\childdocname' of `\childdocjob'\par
\else
main document: `\childdocjob'\par
\fi
version: \version\par
\end{center}
\newpage
%    \end{macrocode}

% Manually include selected file,
% otherwise process as usual:
%    \begin{macrocode}
\ifchilddocmanual
\section*{part `\childdocname'}
\input{\childdocname}
\else
%    \end{macrocode}

% Include the two chapters:
%    \begin{macrocode}
\include{cdocsch1}
\include{cdocsch2}
%    \end{macrocode}

% Include the two parts unless only chapters should be displayed:
%    \begin{macrocode}
\ifchilddoc\else
\section{part three}
\input{cdocspt3}
\section{part four}
\input{cdocspt4}
\fi
%    \end{macrocode}

% Process as usual until here:
%    \begin{macrocode}
\fi
%    \end{macrocode}

% End of document body:
%    \begin{macrocode}
\end{document}
%    \end{macrocode}
%\iffalse
%</samplemain>
%\fi
%
% %%%%%%%%%%%%%%%%%%%%%%%%%%%%%%%%%%%%%%
% \paragraph{Chapter Include Files.}
%
% The include files are called |cdocsch1.tex| and |cdocsch2.tex|.
%
%\iffalse
%<*samplechap1|samplechap2>
%\fi

% Optional override for |\version| flag:
%    \begin{macrocode}
%%\providecommand{\version}{final}
%    \end{macrocode}

% Include the main document:
%    \begin{macrocode}
\input{childdoc.def}
\childdocof{cdocsamp}
%    \end{macrocode}

%\iffalse
%</samplechap1|samplechap2>
%\fi
%
%\iffalse
%<*samplechap1>
%\fi
% Some text for chapter 1:
%    \begin{macrocode}
\section{one}
some text in chapter one
%    \end{macrocode}

%\iffalse
%</samplechap1>
%\fi
% Some text for chapter 2:
%\iffalse
%<*samplechap2>
%\fi
%    \begin{macrocode}
\section{two}
more text in chapter two
%    \end{macrocode}

%\iffalse
%</samplechap2>
%\fi
%
% %%%%%%%%%%%%%%%%%%%%%%%%%%%%%%%%%%%%%%
% \paragraph{Part Include Files.}
%
% The include files are called |cdocspt3.tex| and |cdocspt4.tex|.
%
%\iffalse
%<*samplepart3|samplepart4>
%\fi

% Optional override for |\version| flag:
%    \begin{macrocode}
%%\providecommand{\version}{final}
%    \end{macrocode}

% Include the main document:
%    \begin{macrocode}
\input{childdoc.def}
\childdocby{cdocsamp}
%    \end{macrocode}

%\iffalse
%</samplepart3|samplepart4>
%\fi
%
%\iffalse
%<*samplepart3>
%\fi
% Some text for part 3:
%    \begin{macrocode}
some text in part three
%    \end{macrocode}

%\iffalse
%</samplepart3>
%\fi
% Some text for part 4:
%\iffalse
%<*samplepart4>
%\fi
%    \begin{macrocode}
more text in part four
%    \end{macrocode}

%\iffalse
%</samplepart4>
%\fi
%
% %%%%%%%%%%%%%%%%%%%%%%%%%%%%%%%%%%%%%%
% \paragraph{Forwarding for a Complete Draft.}
%
% The following forwarding file |cdocsdrf.tex|
% compiles the main document in draft mode:
%\iffalse
%<*sampledraft>
%\fi
%    \begin{macrocode}
\def\version{draft}
\input{childdoc.def}
\childdocforward{cdocsamp}
%    \end{macrocode}

%\iffalse
%</sampledraft>
%\fi
%
% %%%%%%%%%%%%%%%%%%%%%%%%%%%%%%%%%%%%%%
% \paragraph{Forwarding for Final Version of the Chapters.}
%
% The following forwarding files |cdocsfn1.tex| and |cdocsfn2.tex|
% (with identical content)
% compile the final versions of the child documents
% |cdocsch1.tex| and |cdocsch2.tex|, respectively:
%\iffalse
%<*samplefinal>
%\fi
%    \begin{macrocode}
\def\version{final}
\input{childdoc.def}
\childdocforwardprefix[cdocsamp]{cdocsfn}{cdocsch}
%    \end{macrocode}

%\iffalse
%</samplefinal>
%\fi
%
% %%%%%%%%%%%%%%%%%%%%%%%%%%%%%%%%%%%%%%
% \paragraph{Command Line Processing.}
%
% The following three command lines generate the output files
% |cdocscld|, |cdocscl1| and |cdocscl2|
% which should be identical to
% |cdocsdrf|, |cdocsch1| and |cdocsfn2|, respectively:
% \begin{center}
% \begin{tabular}{l}
% |latex -jobname cdocscld \|\\
% |  "\def\version{draft}\input{childdoc.def}\childdocforward{cdocsamp}"|\\
% |latex -jobname cdocscl1 \|\\
% |  "\input{childdoc.def}\childdocforward[cdocsamp]{cdocsch1}"|\\
% |latex -jobname cdocscl2 \|\\
% |  "\def\version{final}\input{childdoc.def}\childdocforward{cdocsch2}"|
% \end{tabular}
% \end{center}
% Note that the trailing backslash on each first line
% merely continues the input to the second line
% (for convenient cut ant paste).
% Furthermore, the command |latex| can be replaced by any
% of its alternative versions such as |pdflatex|.
%
% %%%%%%%%%%%%%%%%%%%%%%%%%%%%%%%%%%%%%%%%%%%%%%%%%%%%%%%%%%%%%%%%%%%%%%%%%%%%%%
% %%%%%%%%%%%%%%%%%%%%%%%%%%%%%%%%%%%%%%%%%%%%%%%%%%%%%%%%%%%%%%%%%%%%%%%%%%%%%%
% \section{Implementation}
%\iffalse
%<*package>
%\fi
%
% This section describes the definitions file |childdoc.def|.

% The definitions cannot be loaded using |\usepackage| or |\RequirePackage|
% which has a mechanism to prevent loading a style file more than once.
% When loading the definitions by means of |\input|
% multiple instances have to be prevented manually:
%\iffalse
%This code needs to be before the `\ProvidesFile' directive
%which is defined at the beginning of this file.
%Therefore it is also placed there and commented out here.
%</package>
%<*discard>
%\fi
%    \begin{macrocode}
\ifdefined\childdocmain\endinput\fi
%    \end{macrocode}
%\iffalse
%</discard>
%<*package>
%\fi
%
% \macro{\ifchilddoc}
% \macro{\ifchilddocmanual}
% The conditional |\ifchilddoc| tells whether a
% child (true) or main (false) document is being compiled.
% The conditional |\ifchilddocmanual| tells whether
% the |\includeonly| mechanism is used (false) or
% the selection of child files must be performed manually (true).
% The definitions initialise to false:
%    \begin{macrocode}
\newif\ifchilddoc
\newif\ifchilddocmanual
%    \end{macrocode}

% \macro{\childdocname}
% \macro{\childdocjob}
% The macro |\childdocname| stores the name of the main document
% to be compiled. The macro |\childdocjob| stores the name of
% the document on which the \LaTeX{} compiler was originally invoked.
% The content of |\jobname| cannot be compared
% to filenames specified in the source due to different catcodes.
% The following code rescans |\jobname|, stores the result
% in |\childdocname| and saves a copy in |\childdocjob|:
%    \begin{macrocode}
\edef\childdocname{\scantokens\expandafter{\jobname\noexpand}}
\let\childdocjob\childdocname
%    \end{macrocode}

% \macro{\childdocdisable}
% The macro |\childdocdisable| prevents the main file
% from being processed more than once.
% At this stage, the main document command |\childdocmain|
% is assumed to be called once again where it should do nothing.
% Any subsequent call to it should prevent
% a secondary processing of the main document
% It overwrites the forwarding commands
% |\childdocof| and |\childdocforward|
% with empty macros to prevent further inclusions of the main document:
%    \begin{macrocode}
\newcommand{\childdocdisable}
{
  \renewcommand{\childdocmain}[1]{\renewcommand{\childdocmain}[1]{\endinput}}
  \renewcommand{\childdocof}[1]{}
  \renewcommand{\childdocby}[2][]{}
  \renewcommand{\childdocforward}[2][]{}
  \renewcommand{\childdocdisable}{}
}
%    \end{macrocode}

% \macro{\childdocmain}
% The macro |\childdocmain| is to be called at the top of the main file
% with nothing or the main filename (without extension) as argument.
% First, it breaks loops.
% If the argument is not empty and does not match |\childdocname|
% (which is set by the first inclusion of |childdoc.def|),
% |\ifchilddoc| is set to true, |\includeonly| is applied to the child file
% and |\jobname| is set to the main file
% (for proper handling of |.aux| files):
%    \begin{macrocode}
\newcommand{\childdocmain}[1]
{
  \childdocdisable\childdocmain{}
  \if?#1?\else
    \begingroup
      \def\childdoctmp{#1}
      \ifx\childdoctmp\childdocname
        \def\childdoctmp{}
      \else
        \def\childdoctmp
        {
          \childdoctrue
          \includeonly{\childdocname}
          \def\childdocjob{#1}
          \def\jobname{#1}
        }
      \fi
      \expandafter
    \endgroup
    \childdoctmp
  \fi
}
%    \end{macrocode}

% \macro{\childdocof}
% The command |\childdocof| redirects
% compilation to the main file |#1|.
%    \begin{macrocode}
\newcommand{\childdocof}[1]
{
  \childdocdisable
  \childdoctrue
  \includeonly{\childdocname}
  \def\jobname{#1}
  \def\childdocjob{#1}
  \input{#1}
}
%    \end{macrocode}

% \macro{\childdocby}
% The command |\childdocby| ....
%    \begin{macrocode}
\newcommand{\childdocby}[2][]
{
  \childdocdisable
  \childdoctrue
  \childdocmanualtrue
  \if?#1?\else
    \def\jobname{#2}
  \fi
  \def\childdocjob{#2}
  \input{#2}
  \endinput
}
%    \end{macrocode}

% \macro{\childdocforward}
% The command |\childdocforward| redirects
% compilation to the main file or
% (if the optional argument is given) a child file.
% Parameters are set as if the main file
% or a child file starting with |\childdocof| was compiled.
% Then compilation is handed over to the main file:
%    \begin{macrocode}
\newcommand{\childdocforward}[2][]
{
  \begingroup
    \if?#1?
      \def\childdoctmp
      {
        \def\childdocname{#2}
        \def\childdocjob{#2}
        \def\jobname{#2}
        \input{#2}
        \endinput
      }
    \else
      \def\childdoctmp
      {
        \childdocdisable
        \def\childdocname{#2}
        \childdoctrue
        \includeonly{#2}
        \def\childdocjob{#1}
        \def\jobname{#1}
        \input{#1}
        \endinput
      }
    \fi
    \expandafter
  \endgroup
  \childdoctmp
}
%    \end{macrocode}

% \macro{\childdocforwardprefix}
% The command |\childdocforwardprefix| redirects
% compilation to the main or a child file by means of a pattern.
% The prefix |#1| in the current filename is replaced by |#2|
% and the suffix of the current filename is kept
% (it is assumed that the filename does not contain the substring `|~~~|'
% which is used as a delimiter).
% Compilation is handed over to the new file by |\childdocforward|:
%    \begin{macrocode}
\newcommand{\childdocforwardprefix}[3][]
{
  \begingroup
    \def\childdocextract #2##1~~~{\def\childdoctmp{\childdocforward[#1]{#3##1}}}
    \expandafter\childdocextract\childdocname~~~
    \expandafter
  \endgroup
  \childdoctmp
}
%    \end{macrocode}

% \macro{\childdoc}
% The deprecated macro |\childdoc| is a legacy version of |\childdocmain|:
%    \begin{macrocode}
\newcommand{\childdoc}{\childdocmain}
%    \end{macrocode}

% \macro{\childdocredirect}
% The deprecated macro |\childdocredirect| is a legacy version
% of |\childdocforward| and |\childdocforwardprefix|:
%    \begin{macrocode}
\newcommand{\childdocredirect}[2][]
{
  \begingroup
    \if?#1?
      \def\childdoctmp{\childdocforward{#2}}
    \else
      \def\childdoctmp{\childdocforwardprefix{#1}{#2}}
    \fi
    \expandafter
  \endgroup
  \childdoctmp
}
%    \end{macrocode}

%\iffalse
%</package>
%\fi
%
\endinput
|\\
|\childdocforward{|\textit{main}|}|
\end{tabular}
\end{center}
%
Likewise, the following files |final|\textit{nn}|.tex|
compile the final version of the child document
|child|\textit{nn}|.tex|:
%
\begin{center}
\begin{tabular}{l}
|\def\version{final}|\\
|% \iffalse
%
% childdoc.dtx Copyright (C) 2017-2018 Niklas Beisert
%
% This work may be distributed and/or modified under the
% conditions of the LaTeX Project Public License, either version 1.3
% of this license or (at your option) any later version.
% The latest version of this license is in
%   http://www.latex-project.org/lppl.txt
% and version 1.3 or later is part of all distributions of LaTeX
% version 2005/12/01 or later.
%
% This work has the LPPL maintenance status `maintained'.
%
% The Current Maintainer of this work is Niklas Beisert.
%
% This work consists of the files childdoc.dtx and childdoc.ins
% and the derived files childdoc.def and cdocsamp.tex with
% cdocsch1.tex, cdocsch2.tex, cdocsdrf.tex, cdocsfn1.tex, cdocsfn2.tex.
%
%<package>\ifdefined\childdocmain\endinput\fi
%<package>\ProvidesFile{childdoc.def}[2018/12/30 v2.0 child document driver]
%<samplemain>\ProvidesFile{cdocsamp.tex}[2018/12/30 v2.0 sample for childdoc]
%<*driver>
%\ProvidesFile{childdoc.drv}[2018/12/30 v2.0 childdoc reference manual file]
\PassOptionsToClass{10pt,a4paper}{article}
\documentclass{ltxdoc}

\usepackage[margin=35mm]{geometry}
\usepackage{hyperref}
\usepackage{hyperxmp}
\usepackage[usenames]{color}

\hypersetup{colorlinks=true}
\hypersetup{pdfstartview=FitH}
\hypersetup{pdfpagemode=UseNone}
\hypersetup{pdfsource={}}
\hypersetup{pdflang={en-UK}}
\hypersetup{pdfcopyright={Copyright 2017-2018 Niklas Beisert.
  This work may be distributed and/or modified under the
  conditions of the LaTeX Project Public License, either version 1.3
  of this license or (at your option) any later version.}}
\hypersetup{pdflicenseurl={http://www.latex-project.org/lppl.txt}}
\hypersetup{pdfcontactaddress={ETH Zurich, ITP, HIT K,
  Wolfgang-Pauli-Strasse 27}}
\hypersetup{pdfcontactpostcode={8093}}
\hypersetup{pdfcontactcity={Zurich}}
\hypersetup{pdfcontactcountry={Switzerland}}
\hypersetup{pdfcontactemail={nbeisert@itp.phys.ethz.ch}}
\hypersetup{pdfcontacturl={http://people.phys.ethz.ch/\xmptilde nbeisert/}}

\newcommand{\secref}[1]{\hyperref[#1]{section \ref*{#1}}}

\parskip1ex
\parindent0pt
\let\olditemize\itemize
\def\itemize{\olditemize\parskip0pt}

\begin{document}

\title{The \textsf{childdoc} Package}
\hypersetup{pdftitle={The childdoc Package}}
\author{Niklas Beisert\\[2ex]
  Institut f\"ur Theoretische Physik\\
  Eidgen\"ossische Technische Hochschule Z\"urich\\
  Wolfgang-Pauli-Strasse 27, 8093 Z\"urich, Switzerland\\[1ex]
  \href{mailto:nbeisert@itp.phys.ethz.ch}
  {\texttt{nbeisert@itp.phys.ethz.ch}}}
\hypersetup{pdfauthor={Niklas Beisert}}
\hypersetup{pdfsubject={Manual for the LaTeX2e Package childdoc}}
\date{30 December 2018, \textsf{v2.0}}
\maketitle

\begin{abstract}\noindent
\textsf{childdoc} is a \LaTeXe{} package
that enables the direct compilation
of document sections included by |\include|
to individual files.
\end{abstract}

\begingroup
\parskip0ex
\tableofcontents
\endgroup

%%%%%%%%%%%%%%%%%%%%%%%%%%%%%%%%%%%%%%%%%%%%%%%%%%%%%%%%%%%%%%%%%%%%%%%%%%%%%%%%
%%%%%%%%%%%%%%%%%%%%%%%%%%%%%%%%%%%%%%%%%%%%%%%%%%%%%%%%%%%%%%%%%%%%%%%%%%%%%%%%
\section{Introduction}

\LaTeX{} provides a mechanism to structure a large document (such as a book)
into a main file and several child files (containing the chapters)
using the |\include| command.
This mechanism is beneficial for documents
which span hundreds of pages in order to
make the source file(s) more manageable.
Moreover, compilation can be restricted to
selected child files by means of the |\includeonly| command.
The latter feature can be used to reduce the compilation time while editing
(this was significantly more useful in the earlier days of \LaTeX{})
or to generate a smaller document which is easier to navigate.
Another application of |\includeonly| is to generate
documents consisting of selected parts of the complete document.

However, there are a few drawbacks of the plain |\include| mechanism:
\begin{itemize}
\item
The child files cannot be compiled on their own,
they can only be compiled via the main file.
A naive editing environment
(such as a text editor with an option
to have the current file processed by \LaTeX)
may require one to switch to the main file before compiling;
attempting to compile the child file produces errors.
\item
The main file must be modified (each time)
to adjust the |\includeonly| command
to the present needs. This easily leaves the main file in a messy state.
\item
The generated document will always carry the filename
of the main document. This is inconvenient if
several child files are to be compiled and
to be kept for distribution.
\end{itemize}

The present package provides a simple interface
to make child files individually compilable by \LaTeX{}.
Compiling a child file then has the same effect as compiling
the main file with an |\includeonly| command
to select the appropriate child.
Moreover the generated document will carry the name of the child
rather than the main file.
This resolves all three above issues.

This feature is meant to make the editing of books,
thesis documents and lecture notes somewhat more convenient.
However, the package can also be used efficiently for
composing a series of documents (such as exercise sheets)
which are typically distributed individually.
It then assists the author in generating the individual documents
(potentially in different versions)
as well as a document containing the collected series.
Another application is in developing style files
or other kinds of included material
where compilation of the style file could redirect
to a sample or test file.

%%%%%%%%%%%%%%%%%%%%%%%%%%%%%%%%%%%%%%%%%%%%%%%%%%%%%%%%%%%%%%%%%%%%%%%%%%%%%%%%
%%%%%%%%%%%%%%%%%%%%%%%%%%%%%%%%%%%%%%%%%%%%%%%%%%%%%%%%%%%%%%%%%%%%%%%%%%%%%%%%
\section{Usage}

First of all, the package \textsf{childdoc} is \emph{not} a standard
\LaTeXe{} |.sty| style file! Therefore it needs to be invoked in
a non-standard way.

%%%%%%%%%%%%%%%%%%%%%%%%%%%%%%%%%%%%%%%%%%%%%%%%%%%%%%%%%%%%%%%%%%%%%%%%%%%%%%%%
\subsection{Included Files}
\label{sec:include}

%%%%%%%%%%%%%%%%%%%%%%%%%%%%%%%%%%%%%%%%
\DescribeMacro{\childdocmain}
To use the package, add the commands
\begin{center}
\begin{tabular}{l}
|\input{childdoc.def}|\\
|\childdocmain{}|\\
\end{tabular}
\end{center}
at the very top of the main \LaTeX{} file,
in particular \emph{before} the |\documentclass| statement!
The argument of |\childdocmain| should be left empty
(but it must be present).

%%%%%%%%%%%%%%%%%%%%%%%%%%%%%%%%%%%%%%%%
\DescribeMacro{\childdocof}
Furthermore, add the commands
\begin{center}
\begin{tabular}{l}
|\input{childdoc.def}|\\
|\childdocof{|\textit{main}|}|\\
\end{tabular}
\end{center}
at the top of every child file \textit{child}
which is included by |\include{|\textit{child}|}|
from within the main file
(or at least for those files to be compiled individually).
The argument \textit{main} must be the filename of the main file.

There are a couple of
considerations in setting up the main and child documents:

%%%%%%%%%%%%%%%%%%%%%%%%%%%%%%%%%%%%%%%%
\paragraph{Restrictions.}

Please note the following restrictions:
\begin{itemize}
\item
|\childdocmain| must be called with one argument \textit{main}
to ensure compatibility with earlier version of the package.
It must either be empty (|\childdocmain{}|)
or precisely match the filename of the main file in which it is specified.
See \secref{sec:detection} for further information.
\item
The filename \textit{main} must be specified without the |.tex| extension.
\item
The filename \textit{main} is case sensitive
(even in case-insensitive file systems)
due to internal string comparison.
\item
The argument \textit{main} should be fully expanded, it cannot be a macro.
\item
Subdirectories and special characters should be avoided in filenames.
\item
The command |\childdocmain{|\textit{main}|}| must be followed by a whitespace.
It should not be followed immediately by another command
or by a comment mark `|%|'.
This is because the \TeX{} parser reads the token immediately following
the argument of |\childdocmain| and puts it
at the beginning of every child section;
however, a white\-space is ignored.
\end{itemize}

%%%%%%%%%%%%%%%%%%%%%%%%%%%%%%%%%%%%%%%%
\paragraph{Content of Main File.}

It is advisable to place all content in the child files included by |\include|.
Any output contained in the main file will appear in all child documents
unless suppressed manually;
it cannot be suppressed automatically by the |\includeonly| directive
and thus should normally be avoided.
A method to include some content in the main file
by means of conditional processing is described in \secref{sec:conditional}.

%%%%%%%%%%%%%%%%%%%%%%%%%%%%%%%%%%%%%%%%
\paragraph{Page Numbering.}

When only a part of the document is compiled,
the appropriate numbering of pages
(as well as other status parameters)
is determined from the |.aux| files.
The latter contain information from previous passes.
However this information needs to propagate through
all intermediate child documents.
Therefore the page numbering in child documents may well
be inconsistent until the complete document is compiled at least once.

A useful (if unconventional) way to always ensure a consistent
page numbering is to restart the numbering in each child document
and denote the pages by `\textit{child}|.|\textit{page}'
where \textit{child} represents the chapter/section number of the child file.
This can be achieved by the command
|\numberwithin{page}{|\textit{child}|}|
of the \textsf{amsmath} package
where \textit{child} can be |chapter| or |section|
depending on the chosen structuring.
Alternatively, one can modify the macro |\thepage| appropriately
and reset the counter |page| at the start of each child file.

%%%%%%%%%%%%%%%%%%%%%%%%%%%%%%%%%%%%%%%%%%%%%%%%%%%%%%%%%%%%%%%%%%%%%%%%%%%%%%%%
\subsection{Conditional Processing}
\label{sec:conditional}

The package provides a mechanism to compile different versions
of a document. To customise the versions further some conditional processing
can come in handy to distinguish which version is being compiled.
The package provides two macros to describe the compilation context:

%%%%%%%%%%%%%%%%%%%%%%%%%%%%%%%%%%%%%%%%
\DescribeMacro{\ifchilddoc}
The conditional |\ifchilddoc| distinguishes between the compilation of
child documents and the main document:
%
\begin{center}
|\ifchilddoc |\textit{child-code}| |[|\||else |\textit{main-code}]| \||fi|
\end{center}

%%%%%%%%%%%%%%%%%%%%%%%%%%%%%%%%%%%%%%%%
\DescribeMacro{\childdocname}
\DescribeMacro{\childdocjob}
The macro |\childdocname| contains the filename (without extension)
of the main or child file being processed.
Note that |\childdocjob| will always contain the name of the main file.

%%%%%%%%%%%%%%%%%%%%%%%%%%%%%%%%%%%%%%%%
\paragraph{Title Page.}

Conditional processing can be used to include a title or banner page
in the main document when proper precautions are taken.
Importantly, the code in the main file should ensure that the page counter
(as well as other status parameters which are stored in the |.aux| files)
takes the same value after the conditional processing.
Otherwise the page numbers may take divergent values
depending on which part is compiled.

For example, a title page could be declared by:
%
\begin{center}
\begin{tabular}{l}
|\ifchilddoc\||else|\\
|\addtocounter{page}{-1}|\\
\textit{code for title page}\\
|\newpage|\\
|\||fi|
\end{tabular}
\end{center}
%
A banner page for the child documents can be generated by:
%
\begin{center}
\begin{tabular}{l}
|\ifchilddoc|\\
|\addtocounter{page}{-1}|\\
\textit{code for banner page}\\
|\newpage|\\
|\||fi|
\end{tabular}
\end{center}
%
Here one could write a message such as:
\begin{center}
|This is the part \childdocname{} of \childdocjob{}.|
\end{center}

%%%%%%%%%%%%%%%%%%%%%%%%%%%%%%%%%%%%%%%%%%%%%%%%%%%%%%%%%%%%%%%%%%%%%%%%%%%%%%%%
\subsection{Flags}
\label{sec:flags}

The package makes it easy to generate different versions
of the main or child documents.
To this end compilation flags can be defined
and assigned different default values.
They will be particularly useful in conjunction
with the forwarding mechanism described in \secref{sec:forward}.

For example, it may be useful to have a flag |\version|
which can be set to |draft| or |final|.
The document source will contain some conditional code
depending on the value of |\version|.
Suppose further, the flag should default to |final| for the main file
and to |draft| for child files
which is a natural assignment for editing the document.
This is achieved by placing the following code
in the preamble of the main document
(below the |\childdocmain| directive):
%
\begin{center}
\begin{tabular}{l}
|\ifchilddoc|\\
|\providecommand{\version}{draft}|\\
|\||else|\\
|\providecommand{\version}{final}|\\
|\||fi|
\end{tabular}
\end{center}
%
The definition by |\providecommand| makes sure
that previous definitions are not overwritten.
Further statements |\providecommand{\version}{...}|
can thus be added before the above code to override it.

For the main file, one might add a line
(between |\childdocmain| and the above block)
%
\begin{center}
|%\ifchilddoc\||else\providecommand{\version}{draft}\||fi|
\end{center}
%
which can be uncommented to produce a draft version.
Likewise one can add a line to the very top of a child file
(above the |\childdocof{|\textit{main}|}| directive)
%
\begin{center}
|%\providecommand{\version}{final}|
\end{center}
%
which can be uncommented to produce the final version of this child document.

%%%%%%%%%%%%%%%%%%%%%%%%%%%%%%%%%%%%%%%%%%%%%%%%%%%%%%%%%%%%%%%%%%%%%%%%%%%%%%%%
\subsection{Forwarding}
\label{sec:forward}

Different versions of the main or child documents
using compilation flags as described in \secref{sec:flags}
can be (permanently) stored in different files
for convenient compilation, viewing and distribution.
To this end, the package defines a command
to pass on compilation to a different file:

%%%%%%%%%%%%%%%%%%%%%%%%%%%%%%%%%%%%%%%%
\DescribeMacro{\childdocforward}
The command |\childdocforward| redirects processing to
another source file:
%
\begin{center}
\begin{tabular}{l}
|\input{childdoc.def}|\\
|\childdocforward[|\textit{main}|]{|\textit{dest}|}|\\
\end{tabular}
\end{center}
%
The argument \textit{dest} is the destination file
(without extension).
It should be the main file or one of the child files.
Note that further \textsf{childdoc} directives
such as |\childdocof| and |\childdocforward|
in the indicated file will be processed in this form.
The optional argument \textit{main}
passes on directly to the main file \textit{main}
while pretending to compile the child \textit{dest}.
This form behaves as if \textit{dest}
issues |\childdocof{|\textit{main}|}| right away,
and no further \textsf{childdoc} directives will be processed.

%%%%%%%%%%%%%%%%%%%%%%%%%%%%%%%%%%%%%%%%
\DescribeMacro{\...prefix}
In the alternative form |\childdocforwardprefix|,
%
\begin{center}
\begin{tabular}{l}
|\input{childdoc.def}|\\
|\childdocforwardprefix[|\textit{main}|]{|\textit{prefix}|}{|\textit{dest}|}|
\end{tabular}
\end{center}
%
the destination file is determined by a pattern
depending on the current file:
To make this work, the current file must be called
`{\textit{prefix}\hspace{0.2em}\textit{suffix}}'
with \textit{prefix} matching precisely the argument.
Processing is then passed on to the file
`{\textit{dest}\hspace{0.2em}\textit{suffix}}'.
Surely, the same effect is achieved by
directly specifying the
argument `{\textit{dest}\hspace{0.2em}\textit{suffix}}'
in the first form.
However, that requires to set up a different file
for each child. With the alternative form of the command
all these files can have exactly the same content
which simplifies setting them up and maintaining them.

For example, the following file |draft.tex|
with a compilation flag |\version| as described in \secref{sec:flags}
compiles the main document as a draft:
%
\begin{center}
\begin{tabular}{l}
|\def\version{draft}|\\
|\input{childdoc.def}|\\
|\childdocforward{|\textit{main}|}|
\end{tabular}
\end{center}
%
Likewise, the following files |final|\textit{nn}|.tex|
compile the final version of the child document
|child|\textit{nn}|.tex|:
%
\begin{center}
\begin{tabular}{l}
|\def\version{final}|\\
|\input{childdoc.def}|\\
|\childdocforwardprefix{final}{child}|
\end{tabular}
\end{center}
%

Note that when several versions of a main file and/or of each child file
are to be generated, it may be convenient to set up a |Makefile| or
shell script to automatise the process.

%%%%%%%%%%%%%%%%%%%%%%%%%%%%%%%%%%%%%%%%%%%%%%%%%%%%%%%%%%%%%%%%%%%%%%%%%%%%%%%%
\subsection{Command Line Processing}
\label{sec:commandline}

The effect of redirection files can also be achieved by invoking
the \LaTeX{} compiler with a more elaborate command line.
Most conveniently this should be done as part
of a shell script or a |Makefile|.

When using \textsf{childdoc} in the main file, the following
command lines effectively perform a redirection
(note that depending on the shell being used,
backslashes may have to be doubled: `|\|' $\to$ `|\\|'):
%
\begin{center}
|... -jobname "|\textit{target}|" |\\|"|[\textit{flags}]%
|\input{childdoc.def}\childdocforward[|\textit{main}|]{|\textit{dest}|}"|
\end{center}
%
Here \textit{target} is the name of the output file,
\textit{main} is the name of the main file
and \textit{dest} is the name of the main or child file to be processed
(all filenames without extensions).
The optional argument \textit{main} can be omitted
if \textit{main} matches \textit{dest}.
Optionally, compilation \textit{flags} can be defined via |\def| commands.
This command line makes the \TeX{} engine believe
it is compiling the file \textit{target}
whose content is specified as the latter parameter.
The provided code then forwards the processing to
\textit{main} or \textit{dest} as described in \secref{sec:forward}.

%%%%%%%%%%%%%%%%%%%%%%%%%%%%%%%%%%%%%%%%%%%%%%%%%%%%%%%%%%%%%%%%%%%%%%%%%%%%%%%%
\subsection{Include by Input}
\label{sec:input}

Including child documents by |\include| has some restrictions by design.
Most notably, the content of a child document always occupies
its own set of pages; pages cannot be shared between child documents.
Usually, this behaviour makes perfect sense
because each child document contain an essential part of the document.
However, in some situations it may be desirable to compose
a document from a collection of parts
without having mandatory page breaks between then.
For this case, the package
provides a mechanism to include parts
by |\input| which can also be processed individually.
However, by construction this mechanism
requires manual handling of the content to be output.

%%%%%%%%%%%%%%%%%%%%%%%%%%%%%%%%%%%%%%%%
\DescribeMacro{\ifchilddocmanual}
The main file should be prepared as usual, see \secref{sec:include}.
However, the document body must make a distinction
between processing of an individual part and of the main document, e.g.:
%
\begin{center}
\begin{tabular}{l}
|\ifchilddocmanual|\\
|\input{\childdocname}|\\
|\||else|\\
\textit{document body with }|\input{|\textit{part}|}|\\
|\||fi|
\end{tabular}
\end{center}
%
The conditional |\ifchilddocmanual| is true whenever
a part to be included by |\input| is being compiled,
and the name of the part is stored in |\childdocname|.

%%%%%%%%%%%%%%%%%%%%%%%%%%%%%%%%%%%%%%%%
\DescribeMacro{\childdocby}
Each part to be included by |\input| should start with:
%
\begin{center}
\begin{tabular}{l}
|\input{childdoc.def}|\\
|\childdocby{|\textit{main}|}|\\
\end{tabular}
\end{center}
%
The directive |\childdocby| is similar to |\childdocof|
described in \secref{sec:include},
but the subsequent selection of content must be done manually.
To that end, both |\ifchilddoc| and |\ifchilddocmanual|
will be true upon processing of a part,
and the name of the part is stored in |\childdocname|.
Note that |\jobname| will be set to the filename of the current part
so that each part receives an individual |.aux| file
that does not interfere with the |.aux| file(s) of the main document.
This behaviour can be altered by the alternative form
|\childdocby[*]{|\textit{main}|}| (with a non-empty optional argument)
which uses the |.aux| file of the main document
by setting |\jobname| to \textit{main}.

%%%%%%%%%%%%%%%%%%%%%%%%%%%%%%%%%%%%%%%%%%%%%%%%%%%%%%%%%%%%%%%%%%%%%%%%%%%%%%%%
\subsection{Driver Development}
\label{sec:driver}

The \textsf{childdoc} mechanism can also be use for the development
of definition files such as \LaTeX{} styles or classes.
This case differs from the above setup with multiple parts
included by |\include| in that no |\includeonly| should be invoked.
This can be achieved by starting the include file
(before |\ProvidesPackage|) with:
%
\begin{center}
\begin{tabular}{l}
|\input{childdoc.def}|\\
|\childdocforward{|\textit{main}|}|\\
\end{tabular}
\end{center}
%
or alternatively with:
%
\begin{center}
\begin{tabular}{l}
|\input{childdoc.def}|\\
|\childdocby{|\textit{main}|}|\\
\end{tabular}
\end{center}
%
Both forms have slightly different effects as described above.
The main file is prepared as usual, see \secref{sec:include}.

%%%%%%%%%%%%%%%%%%%%%%%%%%%%%%%%%%%%%%%%%%%%%%%%%%%%%%%%%%%%%%%%%%%%%%%%%%%%%%%%
\subsection{Legacy Detection}
\label{sec:detection}

The directive |\childdocmain| in the main file can detect
whether the complete document or merely a child is to be compiled
even without using the directive |\childdocof|.
This method is deprecated because it is less robust
and there is no compelling reason to use it;
it is merely provided for backward compatibility
and it may be removed in future versions.

If the detection mechanism is to be used,
it is mandatory to correctly specify
the filename of the main file as the argument of |\childdocmain|:
%
\begin{center}
\begin{tabular}{l}
|\input{childdoc.def}|\\
|\childdocmain{|\textit{main}|}|\\
\end{tabular}
\end{center}
%
If |\jobname| does not match the argument \textit{main} of |\childdocmain|,
it is assumed that |\jobname| points to the child file to be compiled.
When using |\childdocmain| with the main file specified as argument,
it suffices to start a child file
with just |\input{|\textit{main}|}|
without loading of the package and using |\childdocof|.
If instead all processing is done
with the appropriate \textsf{childdoc} directives,
the argument of \textit{main} of |\childdocmain| can be empty.

An alternative version of the command line processing described
in \secref{sec:commandline} using the detection mechanism reads:
%
\begin{center}
|... -jobname "|\textit{target}|" "|[\textit{flags}]%
[|\def\jobname{|\textit{dest}|}|]|\input{|\textit{main}|}"|
\end{center}

%%%%%%%%%%%%%%%%%%%%%%%%%%%%%%%%%%%%%%%%%%%%%%%%%%%%%%%%%%%%%%%%%%%%%%%%%%%%%%%%
\subsection{Manual Code}
\label{sec:manual}

In case one cannot be certain whether the definitions file |childdoc.def|
is installed on the target \TeX{} distribution
and one prefers not to ship it,
it is conceivable to paste a few relevant commands into the sources.

To that end, drop all statements |\input{childdoc.def}|
and perform the replacements as outlined below.
Instead of |\childdocmain{|\textit{main}|}| add the following code
to the top of the main file:
%
\begin{center}
\begin{tabular}{l}
|\||ifdefined\childdocname\endinput\||fi\newif\ifchilddoc|\\
|\edef\childdocname{\scantokens\expandafter{\jobname\noexpand}}|\\
|\def\childdocmain{|\textit{main}|}\||ifx\childdocmain\childdocname\||else|\\
|\childdoctrue\includeonly{\childdocname}\let\jobname\childdocmain\||fi|\\
\end{tabular}
\end{center}
%
Instead of |\childdocof{|\textit{main}|}| just include the main file
at the top of each child file:
%
\begin{center}
|\input{|\textit{main}|}|
\end{center}
%
A simple redirection |\childdocforward{|\textit{dest}|}| is achieved by:
%
\begin{center}
|\def\jobname{|\textit{dest}|}\input{\jobname}|
\end{center}
%
The redirection with prefix
|\childdocforwardprefix[|\textit{prefix}|]{|\textit{dest}|}|
is accomplished by:
%
\begin{center}
\begin{tabular}{l}
|{\edef\jobname{\scantokens\expandafter{\jobname\noexpand}}|\\
|\def\redirectjob |\textit{prefix}|#1~~~{\gdef\jobname{|\textit{dest}|#1}}|\\
|\expandafter\redirectjob\jobname~~~}\input{\jobname}|
\end{tabular}
\end{center}

In an alternative approach,
child documents can be compiled by a specific command line
without additional code or specific definitions:
%
\begin{center}
|... -jobname "|\textit{target}|" "|[\textit{flags}]%
|\includeonly{|\textit{dest}|}\input{|\textit{main}|}"|
\end{center}
%

%%%%%%%%%%%%%%%%%%%%%%%%%%%%%%%%%%%%%%%%%%%%%%%%%%%%%%%%%%%%%%%%%%%%%%%%%%%%%%%%
%%%%%%%%%%%%%%%%%%%%%%%%%%%%%%%%%%%%%%%%%%%%%%%%%%%%%%%%%%%%%%%%%%%%%%%%%%%%%%%%
\section{Information}

%%%%%%%%%%%%%%%%%%%%%%%%%%%%%%%%%%%%%%%%%%%%%%%%%%%%%%%%%%%%%%%%%%%%%%%%%%%%%%%%
\subsection{Copyright}

Copyright \copyright{} 2017--2018 Niklas Beisert

This work may be distributed and/or modified under the
conditions of the \LaTeX{} Project Public License, either version 1.3
of this license or (at your option) any later version.
The latest version of this license is in
  \url{http://www.latex-project.org/lppl.txt}
and version 1.3 or later is part of all distributions of \LaTeX{}
version 2005/12/01 or later.

This work has the LPPL maintenance status `maintained'.

The Current Maintainer of this work is Niklas Beisert.

This work consists of the files |README.txt|, |childdoc.ins| and |childdoc.dtx|
as well as the derived files |childdoc.def|, |cdocsamp.tex|
with |cdocsch1.tex|, |cdocsch2.tex|, |cdocspt3.tex|, |cdocspt4.tex|,
|cdocsdrf.tex|, |cdocsfn1.tex|, |cdocsfn2.tex|
as well as |childdoc.pdf|.

%%%%%%%%%%%%%%%%%%%%%%%%%%%%%%%%%%%%%%%%%%%%%%%%%%%%%%%%%%%%%%%%%%%%%%%%%%%%%%%%
\subsection{Files and Installation}

The package consists of the files:
%
\begin{center}
\begin{tabular}{ll}
    |README.txt|   & readme file \\
    |childdoc.ins| & installation file \\
    |childdoc.dtx| & source file \\
    |childdoc.def| & definition file \\
    |cdocsamp.tex| & sample main file \\
    |cdocsch1.tex| & sample include file \\
    |cdocsch2.tex| & sample include file \\
    |cdocspt3.tex| & sample part file \\
    |cdocspt4.tex| & sample part file \\
    |cdocsdrf.tex| & sample redirection file \\
    |cdocsfn1.tex| & sample redirection file \\
    |cdocsfn2.tex| & sample redirection file \\
    |childdoc.pdf| & manual
\end{tabular}
\end{center}
%
The distribution consists of the files
|README.txt|, |childdoc.ins| and |childdoc.dtx|.
%
\begin{itemize}
\item
Run (pdf)\LaTeX{} on |childdoc.dtx|
to compile the manual |childdoc.pdf| (this file).
\item
Run \LaTeX{} on |childdoc.ins| to create the definitions file |childdoc.def|
and the sample |cdocsamp.tex| with include files
|cdocsch1.tex|, |cdocsch2.tex|, |cdocspt3.tex|, |cdocspt4.tex|,
|cdocsdrf.tex|, |cdocsfn1.tex|, |cdocsfn2.tex|.
Then copy the file |childdoc.def| to an appropriate directory of your \LaTeX{}
distribution, e.g.\ \textit{texmf-root}|/tex/latex/childdoc|.
\end{itemize}

%%%%%%%%%%%%%%%%%%%%%%%%%%%%%%%%%%%%%%%%%%%%%%%%%%%%%%%%%%%%%%%%%%%%%%%%%%%%%%%%
\subsection{Related CTAN Packages}

There are several other packages which offer a similar functionality:
%
\begin{itemize}
\item
The packages
\href{http://ctan.org/pkg/docmute}{\textsf{docmute}},
\href{http://ctan.org/pkg/includex}{\textsf{includex}} and
\href{http://ctan.org/pkg/standalone}{\textsf{standalone}}
provide commands to include only the document body of
a child file thus allowing both files to be compiled individually.
\item
The packages \href{http://ctan.org/pkg/subdocs}{\textsf{subdocs}}
and \href{http://ctan.org/pkg/subfiles}{\textsf{subfiles}}
provide structures in which the main and child documents can be
encapsulated and allowing them to be compiled individually.
The inclusion mechanism is different from the conventional |\include|.
\item
The package \href{http://ctan.org/pkg/combine}{\textsf{combine}}
is an elaborate solution to combine several documents into one.
\end{itemize}
%
See also the CTAN topic \href{http://ctan.org/topic/subdocs}{\textsf{subdocs}}
for further related packages.
The present package differs from the above solutions in that
a document structure constructed with the conventional |\include| mechanism
just needs two extra commands at the top of every file
such that all constituent files can be compiled individually.

%%%%%%%%%%%%%%%%%%%%%%%%%%%%%%%%%%%%%%%%%%%%%%%%%%%%%%%%%%%%%%%%%%%%%%%%%%%%%%%%
%\subsection{Feature Suggestions}
%
%The following is a list of features which may be useful for future
%versions of this package:
%%
%\begin{itemize}
%\item
%\ldots
%\end{itemize}

%%%%%%%%%%%%%%%%%%%%%%%%%%%%%%%%%%%%%%%%%%%%%%%%%%%%%%%%%%%%%%%%%%%%%%%%%%%%%%%%
\subsection{Revision History}

%%%%%%%%%%%%%%%%%%%%%%%%%%%%%%%%%%%%%%%%
\paragraph{v2.0:} 2018/12/30

\begin{itemize}
\item
immediate forward processing
\item
added |\childdocby| mechanism
\item
manual restructured
\end{itemize}

%%%%%%%%%%%%%%%%%%%%%%%%%%%%%%%%%%%%%%%%
\paragraph{v1.6:} 2018/01/17

\begin{itemize}
\item
application for development of include files
\item
corrections to manual
\end{itemize}

%%%%%%%%%%%%%%%%%%%%%%%%%%%%%%%%%%%%%%%%
\paragraph{v1.5:} 2017/05/21

\begin{itemize}
\item
more complete structuring introduced
\item
|\childdocof| introduced
\item
|\childdoc| renamed to |\childdocmain|
\item
|\childredirect| renamed to |\childdocforward| and |\childdocforwardprefix|
and functionality expanded
\end{itemize}

%%%%%%%%%%%%%%%%%%%%%%%%%%%%%%%%%%%%%%%%
\paragraph{v1.0:} 2017/04/27

\begin{itemize}
\item
manual and install package
\item
first version published on CTAN
\end{itemize}

%%%%%%%%%%%%%%%%%%%%%%%%%%%%%%%%%%%%%%%%
\paragraph{v0.6:} 2017/04/26

\begin{itemize}
\item
redirection mechanism added
\end{itemize}

%%%%%%%%%%%%%%%%%%%%%%%%%%%%%%%%%%%%%%%%
\paragraph{v0.5:} 2017/04/26

\begin{itemize}
\item
functionality in definition file
\end{itemize}


%%%%%%%%%%%%%%%%%%%%%%%%%%%%%%%%%%%%%%%%%%%%%%%%%%%%%%%%%%%%%%%%%%%%%%%%%%%%%%%%
%%%%%%%%%%%%%%%%%%%%%%%%%%%%%%%%%%%%%%%%%%%%%%%%%%%%%%%%%%%%%%%%%%%%%%%%%%%%%%%%
%%%%%%%%%%%%%%%%%%%%%%%%%%%%%%%%%%%%%%%%%%%%%%%%%%%%%%%%%%%%%%%%%%%%%%%%%%%%%%%%
\appendix

\settowidth\MacroIndent{\rmfamily\scriptsize 000\ }

 \DocInput{childdoc.dtx}

\end{document}
%</driver>
% \fi
%
% %%%%%%%%%%%%%%%%%%%%%%%%%%%%%%%%%%%%%%%%%%%%%%%%%%%%%%%%%%%%%%%%%%%%%%%%%%%%%%
% %%%%%%%%%%%%%%%%%%%%%%%%%%%%%%%%%%%%%%%%%%%%%%%%%%%%%%%%%%%%%%%%%%%%%%%%%%%%%%
% \section{Sample}
%\iffalse
%<*samplemain>
%\fi
%
% The following presents a sample document
% with two chapters, two parts, a title page,
% a compile flag as well as three forwarding files to set the flag.
% It consists of eight |.tex| files:
% \begin{center}
% \begin{tabular}{ll}
% |cdocsamp.tex|&main file\\
% |cdocsch1.tex|&include file for chapter 1\\
% |cdocsch2.tex|&include file for chapter 2\\
% |cdocspt3.tex|&include file for part 3\\
% |cdocspt4.tex|&include file for part 4\\
% |cdocsdrf.tex|&forwarding file for main file in draft mode\\
% |cdocsfi1.tex|&forwarding file for final version of chapter 1\\
% |cdocsfi2.tex|&forwarding file for final version of chapter 2\\
% \end{tabular}
% \end{center}
% Each of the eight files can be compiled directly by the \LaTeX{} compiler.
%
% %%%%%%%%%%%%%%%%%%%%%%%%%%%%%%%%%%%%%%
% \paragraph{Main File.}
%
% The main file is called |cdocsamp.tex|.
%
% Load the \textsf{childdoc} definitions and
% declare the filename for the main document:
%    \begin{macrocode}
\input{childdoc.def}
\childdocmain{}
%    \end{macrocode}

% Optional override for |\version| flag:
%    \begin{macrocode}
%%\ifchilddoc\else\providecommand{\version}{draft}\fi
%    \end{macrocode}

% Define the default values for the |\version| flag
% (|final| for the main file and |draft| for childs):
%    \begin{macrocode}
\ifchilddoc
\providecommand{\version}{draft}
\else
\providecommand{\version}{final}
\fi
%    \end{macrocode}

% Load the standard document class:
%    \begin{macrocode}
\documentclass[12pt]{article}
%    \end{macrocode}

% Start the document body:
%    \begin{macrocode}
\begin{document}
%    \end{macrocode}

% Declare a title page.
% Print title, part of document being processed and version flag:
%    \begin{macrocode}
\addtocounter{page}{-1}
\begin{center}
{\LARGE\bfseries{}childdoc example\par}
\vspace{1cm}
\ifchilddoc
\ifchilddocmanual part\else chapter\fi:
`\childdocname' of `\childdocjob'\par
\else
main document: `\childdocjob'\par
\fi
version: \version\par
\end{center}
\newpage
%    \end{macrocode}

% Manually include selected file,
% otherwise process as usual:
%    \begin{macrocode}
\ifchilddocmanual
\section*{part `\childdocname'}
\input{\childdocname}
\else
%    \end{macrocode}

% Include the two chapters:
%    \begin{macrocode}
\include{cdocsch1}
\include{cdocsch2}
%    \end{macrocode}

% Include the two parts unless only chapters should be displayed:
%    \begin{macrocode}
\ifchilddoc\else
\section{part three}
\input{cdocspt3}
\section{part four}
\input{cdocspt4}
\fi
%    \end{macrocode}

% Process as usual until here:
%    \begin{macrocode}
\fi
%    \end{macrocode}

% End of document body:
%    \begin{macrocode}
\end{document}
%    \end{macrocode}
%\iffalse
%</samplemain>
%\fi
%
% %%%%%%%%%%%%%%%%%%%%%%%%%%%%%%%%%%%%%%
% \paragraph{Chapter Include Files.}
%
% The include files are called |cdocsch1.tex| and |cdocsch2.tex|.
%
%\iffalse
%<*samplechap1|samplechap2>
%\fi

% Optional override for |\version| flag:
%    \begin{macrocode}
%%\providecommand{\version}{final}
%    \end{macrocode}

% Include the main document:
%    \begin{macrocode}
\input{childdoc.def}
\childdocof{cdocsamp}
%    \end{macrocode}

%\iffalse
%</samplechap1|samplechap2>
%\fi
%
%\iffalse
%<*samplechap1>
%\fi
% Some text for chapter 1:
%    \begin{macrocode}
\section{one}
some text in chapter one
%    \end{macrocode}

%\iffalse
%</samplechap1>
%\fi
% Some text for chapter 2:
%\iffalse
%<*samplechap2>
%\fi
%    \begin{macrocode}
\section{two}
more text in chapter two
%    \end{macrocode}

%\iffalse
%</samplechap2>
%\fi
%
% %%%%%%%%%%%%%%%%%%%%%%%%%%%%%%%%%%%%%%
% \paragraph{Part Include Files.}
%
% The include files are called |cdocspt3.tex| and |cdocspt4.tex|.
%
%\iffalse
%<*samplepart3|samplepart4>
%\fi

% Optional override for |\version| flag:
%    \begin{macrocode}
%%\providecommand{\version}{final}
%    \end{macrocode}

% Include the main document:
%    \begin{macrocode}
\input{childdoc.def}
\childdocby{cdocsamp}
%    \end{macrocode}

%\iffalse
%</samplepart3|samplepart4>
%\fi
%
%\iffalse
%<*samplepart3>
%\fi
% Some text for part 3:
%    \begin{macrocode}
some text in part three
%    \end{macrocode}

%\iffalse
%</samplepart3>
%\fi
% Some text for part 4:
%\iffalse
%<*samplepart4>
%\fi
%    \begin{macrocode}
more text in part four
%    \end{macrocode}

%\iffalse
%</samplepart4>
%\fi
%
% %%%%%%%%%%%%%%%%%%%%%%%%%%%%%%%%%%%%%%
% \paragraph{Forwarding for a Complete Draft.}
%
% The following forwarding file |cdocsdrf.tex|
% compiles the main document in draft mode:
%\iffalse
%<*sampledraft>
%\fi
%    \begin{macrocode}
\def\version{draft}
\input{childdoc.def}
\childdocforward{cdocsamp}
%    \end{macrocode}

%\iffalse
%</sampledraft>
%\fi
%
% %%%%%%%%%%%%%%%%%%%%%%%%%%%%%%%%%%%%%%
% \paragraph{Forwarding for Final Version of the Chapters.}
%
% The following forwarding files |cdocsfn1.tex| and |cdocsfn2.tex|
% (with identical content)
% compile the final versions of the child documents
% |cdocsch1.tex| and |cdocsch2.tex|, respectively:
%\iffalse
%<*samplefinal>
%\fi
%    \begin{macrocode}
\def\version{final}
\input{childdoc.def}
\childdocforwardprefix[cdocsamp]{cdocsfn}{cdocsch}
%    \end{macrocode}

%\iffalse
%</samplefinal>
%\fi
%
% %%%%%%%%%%%%%%%%%%%%%%%%%%%%%%%%%%%%%%
% \paragraph{Command Line Processing.}
%
% The following three command lines generate the output files
% |cdocscld|, |cdocscl1| and |cdocscl2|
% which should be identical to
% |cdocsdrf|, |cdocsch1| and |cdocsfn2|, respectively:
% \begin{center}
% \begin{tabular}{l}
% |latex -jobname cdocscld \|\\
% |  "\def\version{draft}\input{childdoc.def}\childdocforward{cdocsamp}"|\\
% |latex -jobname cdocscl1 \|\\
% |  "\input{childdoc.def}\childdocforward[cdocsamp]{cdocsch1}"|\\
% |latex -jobname cdocscl2 \|\\
% |  "\def\version{final}\input{childdoc.def}\childdocforward{cdocsch2}"|
% \end{tabular}
% \end{center}
% Note that the trailing backslash on each first line
% merely continues the input to the second line
% (for convenient cut ant paste).
% Furthermore, the command |latex| can be replaced by any
% of its alternative versions such as |pdflatex|.
%
% %%%%%%%%%%%%%%%%%%%%%%%%%%%%%%%%%%%%%%%%%%%%%%%%%%%%%%%%%%%%%%%%%%%%%%%%%%%%%%
% %%%%%%%%%%%%%%%%%%%%%%%%%%%%%%%%%%%%%%%%%%%%%%%%%%%%%%%%%%%%%%%%%%%%%%%%%%%%%%
% \section{Implementation}
%\iffalse
%<*package>
%\fi
%
% This section describes the definitions file |childdoc.def|.

% The definitions cannot be loaded using |\usepackage| or |\RequirePackage|
% which has a mechanism to prevent loading a style file more than once.
% When loading the definitions by means of |\input|
% multiple instances have to be prevented manually:
%\iffalse
%This code needs to be before the `\ProvidesFile' directive
%which is defined at the beginning of this file.
%Therefore it is also placed there and commented out here.
%</package>
%<*discard>
%\fi
%    \begin{macrocode}
\ifdefined\childdocmain\endinput\fi
%    \end{macrocode}
%\iffalse
%</discard>
%<*package>
%\fi
%
% \macro{\ifchilddoc}
% \macro{\ifchilddocmanual}
% The conditional |\ifchilddoc| tells whether a
% child (true) or main (false) document is being compiled.
% The conditional |\ifchilddocmanual| tells whether
% the |\includeonly| mechanism is used (false) or
% the selection of child files must be performed manually (true).
% The definitions initialise to false:
%    \begin{macrocode}
\newif\ifchilddoc
\newif\ifchilddocmanual
%    \end{macrocode}

% \macro{\childdocname}
% \macro{\childdocjob}
% The macro |\childdocname| stores the name of the main document
% to be compiled. The macro |\childdocjob| stores the name of
% the document on which the \LaTeX{} compiler was originally invoked.
% The content of |\jobname| cannot be compared
% to filenames specified in the source due to different catcodes.
% The following code rescans |\jobname|, stores the result
% in |\childdocname| and saves a copy in |\childdocjob|:
%    \begin{macrocode}
\edef\childdocname{\scantokens\expandafter{\jobname\noexpand}}
\let\childdocjob\childdocname
%    \end{macrocode}

% \macro{\childdocdisable}
% The macro |\childdocdisable| prevents the main file
% from being processed more than once.
% At this stage, the main document command |\childdocmain|
% is assumed to be called once again where it should do nothing.
% Any subsequent call to it should prevent
% a secondary processing of the main document
% It overwrites the forwarding commands
% |\childdocof| and |\childdocforward|
% with empty macros to prevent further inclusions of the main document:
%    \begin{macrocode}
\newcommand{\childdocdisable}
{
  \renewcommand{\childdocmain}[1]{\renewcommand{\childdocmain}[1]{\endinput}}
  \renewcommand{\childdocof}[1]{}
  \renewcommand{\childdocby}[2][]{}
  \renewcommand{\childdocforward}[2][]{}
  \renewcommand{\childdocdisable}{}
}
%    \end{macrocode}

% \macro{\childdocmain}
% The macro |\childdocmain| is to be called at the top of the main file
% with nothing or the main filename (without extension) as argument.
% First, it breaks loops.
% If the argument is not empty and does not match |\childdocname|
% (which is set by the first inclusion of |childdoc.def|),
% |\ifchilddoc| is set to true, |\includeonly| is applied to the child file
% and |\jobname| is set to the main file
% (for proper handling of |.aux| files):
%    \begin{macrocode}
\newcommand{\childdocmain}[1]
{
  \childdocdisable\childdocmain{}
  \if?#1?\else
    \begingroup
      \def\childdoctmp{#1}
      \ifx\childdoctmp\childdocname
        \def\childdoctmp{}
      \else
        \def\childdoctmp
        {
          \childdoctrue
          \includeonly{\childdocname}
          \def\childdocjob{#1}
          \def\jobname{#1}
        }
      \fi
      \expandafter
    \endgroup
    \childdoctmp
  \fi
}
%    \end{macrocode}

% \macro{\childdocof}
% The command |\childdocof| redirects
% compilation to the main file |#1|.
%    \begin{macrocode}
\newcommand{\childdocof}[1]
{
  \childdocdisable
  \childdoctrue
  \includeonly{\childdocname}
  \def\jobname{#1}
  \def\childdocjob{#1}
  \input{#1}
}
%    \end{macrocode}

% \macro{\childdocby}
% The command |\childdocby| ....
%    \begin{macrocode}
\newcommand{\childdocby}[2][]
{
  \childdocdisable
  \childdoctrue
  \childdocmanualtrue
  \if?#1?\else
    \def\jobname{#2}
  \fi
  \def\childdocjob{#2}
  \input{#2}
  \endinput
}
%    \end{macrocode}

% \macro{\childdocforward}
% The command |\childdocforward| redirects
% compilation to the main file or
% (if the optional argument is given) a child file.
% Parameters are set as if the main file
% or a child file starting with |\childdocof| was compiled.
% Then compilation is handed over to the main file:
%    \begin{macrocode}
\newcommand{\childdocforward}[2][]
{
  \begingroup
    \if?#1?
      \def\childdoctmp
      {
        \def\childdocname{#2}
        \def\childdocjob{#2}
        \def\jobname{#2}
        \input{#2}
        \endinput
      }
    \else
      \def\childdoctmp
      {
        \childdocdisable
        \def\childdocname{#2}
        \childdoctrue
        \includeonly{#2}
        \def\childdocjob{#1}
        \def\jobname{#1}
        \input{#1}
        \endinput
      }
    \fi
    \expandafter
  \endgroup
  \childdoctmp
}
%    \end{macrocode}

% \macro{\childdocforwardprefix}
% The command |\childdocforwardprefix| redirects
% compilation to the main or a child file by means of a pattern.
% The prefix |#1| in the current filename is replaced by |#2|
% and the suffix of the current filename is kept
% (it is assumed that the filename does not contain the substring `|~~~|'
% which is used as a delimiter).
% Compilation is handed over to the new file by |\childdocforward|:
%    \begin{macrocode}
\newcommand{\childdocforwardprefix}[3][]
{
  \begingroup
    \def\childdocextract #2##1~~~{\def\childdoctmp{\childdocforward[#1]{#3##1}}}
    \expandafter\childdocextract\childdocname~~~
    \expandafter
  \endgroup
  \childdoctmp
}
%    \end{macrocode}

% \macro{\childdoc}
% The deprecated macro |\childdoc| is a legacy version of |\childdocmain|:
%    \begin{macrocode}
\newcommand{\childdoc}{\childdocmain}
%    \end{macrocode}

% \macro{\childdocredirect}
% The deprecated macro |\childdocredirect| is a legacy version
% of |\childdocforward| and |\childdocforwardprefix|:
%    \begin{macrocode}
\newcommand{\childdocredirect}[2][]
{
  \begingroup
    \if?#1?
      \def\childdoctmp{\childdocforward{#2}}
    \else
      \def\childdoctmp{\childdocforwardprefix{#1}{#2}}
    \fi
    \expandafter
  \endgroup
  \childdoctmp
}
%    \end{macrocode}

%\iffalse
%</package>
%\fi
%
\endinput
|\\
|\childdocforwardprefix{final}{child}|
\end{tabular}
\end{center}
%

Note that when several versions of a main file and/or of each child file
are to be generated, it may be convenient to set up a |Makefile| or
shell script to automatise the process.

%%%%%%%%%%%%%%%%%%%%%%%%%%%%%%%%%%%%%%%%%%%%%%%%%%%%%%%%%%%%%%%%%%%%%%%%%%%%%%%%
\subsection{Command Line Processing}
\label{sec:commandline}

The effect of redirection files can also be achieved by invoking
the \LaTeX{} compiler with a more elaborate command line.
Most conveniently this should be done as part
of a shell script or a |Makefile|.

When using \textsf{childdoc} in the main file, the following
command lines effectively perform a redirection
(note that depending on the shell being used,
backslashes may have to be doubled: `|\|' $\to$ `|\\|'):
%
\begin{center}
|... -jobname "|\textit{target}|" |\\|"|[\textit{flags}]%
|% \iffalse
%
% childdoc.dtx Copyright (C) 2017-2018 Niklas Beisert
%
% This work may be distributed and/or modified under the
% conditions of the LaTeX Project Public License, either version 1.3
% of this license or (at your option) any later version.
% The latest version of this license is in
%   http://www.latex-project.org/lppl.txt
% and version 1.3 or later is part of all distributions of LaTeX
% version 2005/12/01 or later.
%
% This work has the LPPL maintenance status `maintained'.
%
% The Current Maintainer of this work is Niklas Beisert.
%
% This work consists of the files childdoc.dtx and childdoc.ins
% and the derived files childdoc.def and cdocsamp.tex with
% cdocsch1.tex, cdocsch2.tex, cdocsdrf.tex, cdocsfn1.tex, cdocsfn2.tex.
%
%<package>\ifdefined\childdocmain\endinput\fi
%<package>\ProvidesFile{childdoc.def}[2018/12/30 v2.0 child document driver]
%<samplemain>\ProvidesFile{cdocsamp.tex}[2018/12/30 v2.0 sample for childdoc]
%<*driver>
%\ProvidesFile{childdoc.drv}[2018/12/30 v2.0 childdoc reference manual file]
\PassOptionsToClass{10pt,a4paper}{article}
\documentclass{ltxdoc}

\usepackage[margin=35mm]{geometry}
\usepackage{hyperref}
\usepackage{hyperxmp}
\usepackage[usenames]{color}

\hypersetup{colorlinks=true}
\hypersetup{pdfstartview=FitH}
\hypersetup{pdfpagemode=UseNone}
\hypersetup{pdfsource={}}
\hypersetup{pdflang={en-UK}}
\hypersetup{pdfcopyright={Copyright 2017-2018 Niklas Beisert.
  This work may be distributed and/or modified under the
  conditions of the LaTeX Project Public License, either version 1.3
  of this license or (at your option) any later version.}}
\hypersetup{pdflicenseurl={http://www.latex-project.org/lppl.txt}}
\hypersetup{pdfcontactaddress={ETH Zurich, ITP, HIT K,
  Wolfgang-Pauli-Strasse 27}}
\hypersetup{pdfcontactpostcode={8093}}
\hypersetup{pdfcontactcity={Zurich}}
\hypersetup{pdfcontactcountry={Switzerland}}
\hypersetup{pdfcontactemail={nbeisert@itp.phys.ethz.ch}}
\hypersetup{pdfcontacturl={http://people.phys.ethz.ch/\xmptilde nbeisert/}}

\newcommand{\secref}[1]{\hyperref[#1]{section \ref*{#1}}}

\parskip1ex
\parindent0pt
\let\olditemize\itemize
\def\itemize{\olditemize\parskip0pt}

\begin{document}

\title{The \textsf{childdoc} Package}
\hypersetup{pdftitle={The childdoc Package}}
\author{Niklas Beisert\\[2ex]
  Institut f\"ur Theoretische Physik\\
  Eidgen\"ossische Technische Hochschule Z\"urich\\
  Wolfgang-Pauli-Strasse 27, 8093 Z\"urich, Switzerland\\[1ex]
  \href{mailto:nbeisert@itp.phys.ethz.ch}
  {\texttt{nbeisert@itp.phys.ethz.ch}}}
\hypersetup{pdfauthor={Niklas Beisert}}
\hypersetup{pdfsubject={Manual for the LaTeX2e Package childdoc}}
\date{30 December 2018, \textsf{v2.0}}
\maketitle

\begin{abstract}\noindent
\textsf{childdoc} is a \LaTeXe{} package
that enables the direct compilation
of document sections included by |\include|
to individual files.
\end{abstract}

\begingroup
\parskip0ex
\tableofcontents
\endgroup

%%%%%%%%%%%%%%%%%%%%%%%%%%%%%%%%%%%%%%%%%%%%%%%%%%%%%%%%%%%%%%%%%%%%%%%%%%%%%%%%
%%%%%%%%%%%%%%%%%%%%%%%%%%%%%%%%%%%%%%%%%%%%%%%%%%%%%%%%%%%%%%%%%%%%%%%%%%%%%%%%
\section{Introduction}

\LaTeX{} provides a mechanism to structure a large document (such as a book)
into a main file and several child files (containing the chapters)
using the |\include| command.
This mechanism is beneficial for documents
which span hundreds of pages in order to
make the source file(s) more manageable.
Moreover, compilation can be restricted to
selected child files by means of the |\includeonly| command.
The latter feature can be used to reduce the compilation time while editing
(this was significantly more useful in the earlier days of \LaTeX{})
or to generate a smaller document which is easier to navigate.
Another application of |\includeonly| is to generate
documents consisting of selected parts of the complete document.

However, there are a few drawbacks of the plain |\include| mechanism:
\begin{itemize}
\item
The child files cannot be compiled on their own,
they can only be compiled via the main file.
A naive editing environment
(such as a text editor with an option
to have the current file processed by \LaTeX)
may require one to switch to the main file before compiling;
attempting to compile the child file produces errors.
\item
The main file must be modified (each time)
to adjust the |\includeonly| command
to the present needs. This easily leaves the main file in a messy state.
\item
The generated document will always carry the filename
of the main document. This is inconvenient if
several child files are to be compiled and
to be kept for distribution.
\end{itemize}

The present package provides a simple interface
to make child files individually compilable by \LaTeX{}.
Compiling a child file then has the same effect as compiling
the main file with an |\includeonly| command
to select the appropriate child.
Moreover the generated document will carry the name of the child
rather than the main file.
This resolves all three above issues.

This feature is meant to make the editing of books,
thesis documents and lecture notes somewhat more convenient.
However, the package can also be used efficiently for
composing a series of documents (such as exercise sheets)
which are typically distributed individually.
It then assists the author in generating the individual documents
(potentially in different versions)
as well as a document containing the collected series.
Another application is in developing style files
or other kinds of included material
where compilation of the style file could redirect
to a sample or test file.

%%%%%%%%%%%%%%%%%%%%%%%%%%%%%%%%%%%%%%%%%%%%%%%%%%%%%%%%%%%%%%%%%%%%%%%%%%%%%%%%
%%%%%%%%%%%%%%%%%%%%%%%%%%%%%%%%%%%%%%%%%%%%%%%%%%%%%%%%%%%%%%%%%%%%%%%%%%%%%%%%
\section{Usage}

First of all, the package \textsf{childdoc} is \emph{not} a standard
\LaTeXe{} |.sty| style file! Therefore it needs to be invoked in
a non-standard way.

%%%%%%%%%%%%%%%%%%%%%%%%%%%%%%%%%%%%%%%%%%%%%%%%%%%%%%%%%%%%%%%%%%%%%%%%%%%%%%%%
\subsection{Included Files}
\label{sec:include}

%%%%%%%%%%%%%%%%%%%%%%%%%%%%%%%%%%%%%%%%
\DescribeMacro{\childdocmain}
To use the package, add the commands
\begin{center}
\begin{tabular}{l}
|\input{childdoc.def}|\\
|\childdocmain{}|\\
\end{tabular}
\end{center}
at the very top of the main \LaTeX{} file,
in particular \emph{before} the |\documentclass| statement!
The argument of |\childdocmain| should be left empty
(but it must be present).

%%%%%%%%%%%%%%%%%%%%%%%%%%%%%%%%%%%%%%%%
\DescribeMacro{\childdocof}
Furthermore, add the commands
\begin{center}
\begin{tabular}{l}
|\input{childdoc.def}|\\
|\childdocof{|\textit{main}|}|\\
\end{tabular}
\end{center}
at the top of every child file \textit{child}
which is included by |\include{|\textit{child}|}|
from within the main file
(or at least for those files to be compiled individually).
The argument \textit{main} must be the filename of the main file.

There are a couple of
considerations in setting up the main and child documents:

%%%%%%%%%%%%%%%%%%%%%%%%%%%%%%%%%%%%%%%%
\paragraph{Restrictions.}

Please note the following restrictions:
\begin{itemize}
\item
|\childdocmain| must be called with one argument \textit{main}
to ensure compatibility with earlier version of the package.
It must either be empty (|\childdocmain{}|)
or precisely match the filename of the main file in which it is specified.
See \secref{sec:detection} for further information.
\item
The filename \textit{main} must be specified without the |.tex| extension.
\item
The filename \textit{main} is case sensitive
(even in case-insensitive file systems)
due to internal string comparison.
\item
The argument \textit{main} should be fully expanded, it cannot be a macro.
\item
Subdirectories and special characters should be avoided in filenames.
\item
The command |\childdocmain{|\textit{main}|}| must be followed by a whitespace.
It should not be followed immediately by another command
or by a comment mark `|%|'.
This is because the \TeX{} parser reads the token immediately following
the argument of |\childdocmain| and puts it
at the beginning of every child section;
however, a white\-space is ignored.
\end{itemize}

%%%%%%%%%%%%%%%%%%%%%%%%%%%%%%%%%%%%%%%%
\paragraph{Content of Main File.}

It is advisable to place all content in the child files included by |\include|.
Any output contained in the main file will appear in all child documents
unless suppressed manually;
it cannot be suppressed automatically by the |\includeonly| directive
and thus should normally be avoided.
A method to include some content in the main file
by means of conditional processing is described in \secref{sec:conditional}.

%%%%%%%%%%%%%%%%%%%%%%%%%%%%%%%%%%%%%%%%
\paragraph{Page Numbering.}

When only a part of the document is compiled,
the appropriate numbering of pages
(as well as other status parameters)
is determined from the |.aux| files.
The latter contain information from previous passes.
However this information needs to propagate through
all intermediate child documents.
Therefore the page numbering in child documents may well
be inconsistent until the complete document is compiled at least once.

A useful (if unconventional) way to always ensure a consistent
page numbering is to restart the numbering in each child document
and denote the pages by `\textit{child}|.|\textit{page}'
where \textit{child} represents the chapter/section number of the child file.
This can be achieved by the command
|\numberwithin{page}{|\textit{child}|}|
of the \textsf{amsmath} package
where \textit{child} can be |chapter| or |section|
depending on the chosen structuring.
Alternatively, one can modify the macro |\thepage| appropriately
and reset the counter |page| at the start of each child file.

%%%%%%%%%%%%%%%%%%%%%%%%%%%%%%%%%%%%%%%%%%%%%%%%%%%%%%%%%%%%%%%%%%%%%%%%%%%%%%%%
\subsection{Conditional Processing}
\label{sec:conditional}

The package provides a mechanism to compile different versions
of a document. To customise the versions further some conditional processing
can come in handy to distinguish which version is being compiled.
The package provides two macros to describe the compilation context:

%%%%%%%%%%%%%%%%%%%%%%%%%%%%%%%%%%%%%%%%
\DescribeMacro{\ifchilddoc}
The conditional |\ifchilddoc| distinguishes between the compilation of
child documents and the main document:
%
\begin{center}
|\ifchilddoc |\textit{child-code}| |[|\||else |\textit{main-code}]| \||fi|
\end{center}

%%%%%%%%%%%%%%%%%%%%%%%%%%%%%%%%%%%%%%%%
\DescribeMacro{\childdocname}
\DescribeMacro{\childdocjob}
The macro |\childdocname| contains the filename (without extension)
of the main or child file being processed.
Note that |\childdocjob| will always contain the name of the main file.

%%%%%%%%%%%%%%%%%%%%%%%%%%%%%%%%%%%%%%%%
\paragraph{Title Page.}

Conditional processing can be used to include a title or banner page
in the main document when proper precautions are taken.
Importantly, the code in the main file should ensure that the page counter
(as well as other status parameters which are stored in the |.aux| files)
takes the same value after the conditional processing.
Otherwise the page numbers may take divergent values
depending on which part is compiled.

For example, a title page could be declared by:
%
\begin{center}
\begin{tabular}{l}
|\ifchilddoc\||else|\\
|\addtocounter{page}{-1}|\\
\textit{code for title page}\\
|\newpage|\\
|\||fi|
\end{tabular}
\end{center}
%
A banner page for the child documents can be generated by:
%
\begin{center}
\begin{tabular}{l}
|\ifchilddoc|\\
|\addtocounter{page}{-1}|\\
\textit{code for banner page}\\
|\newpage|\\
|\||fi|
\end{tabular}
\end{center}
%
Here one could write a message such as:
\begin{center}
|This is the part \childdocname{} of \childdocjob{}.|
\end{center}

%%%%%%%%%%%%%%%%%%%%%%%%%%%%%%%%%%%%%%%%%%%%%%%%%%%%%%%%%%%%%%%%%%%%%%%%%%%%%%%%
\subsection{Flags}
\label{sec:flags}

The package makes it easy to generate different versions
of the main or child documents.
To this end compilation flags can be defined
and assigned different default values.
They will be particularly useful in conjunction
with the forwarding mechanism described in \secref{sec:forward}.

For example, it may be useful to have a flag |\version|
which can be set to |draft| or |final|.
The document source will contain some conditional code
depending on the value of |\version|.
Suppose further, the flag should default to |final| for the main file
and to |draft| for child files
which is a natural assignment for editing the document.
This is achieved by placing the following code
in the preamble of the main document
(below the |\childdocmain| directive):
%
\begin{center}
\begin{tabular}{l}
|\ifchilddoc|\\
|\providecommand{\version}{draft}|\\
|\||else|\\
|\providecommand{\version}{final}|\\
|\||fi|
\end{tabular}
\end{center}
%
The definition by |\providecommand| makes sure
that previous definitions are not overwritten.
Further statements |\providecommand{\version}{...}|
can thus be added before the above code to override it.

For the main file, one might add a line
(between |\childdocmain| and the above block)
%
\begin{center}
|%\ifchilddoc\||else\providecommand{\version}{draft}\||fi|
\end{center}
%
which can be uncommented to produce a draft version.
Likewise one can add a line to the very top of a child file
(above the |\childdocof{|\textit{main}|}| directive)
%
\begin{center}
|%\providecommand{\version}{final}|
\end{center}
%
which can be uncommented to produce the final version of this child document.

%%%%%%%%%%%%%%%%%%%%%%%%%%%%%%%%%%%%%%%%%%%%%%%%%%%%%%%%%%%%%%%%%%%%%%%%%%%%%%%%
\subsection{Forwarding}
\label{sec:forward}

Different versions of the main or child documents
using compilation flags as described in \secref{sec:flags}
can be (permanently) stored in different files
for convenient compilation, viewing and distribution.
To this end, the package defines a command
to pass on compilation to a different file:

%%%%%%%%%%%%%%%%%%%%%%%%%%%%%%%%%%%%%%%%
\DescribeMacro{\childdocforward}
The command |\childdocforward| redirects processing to
another source file:
%
\begin{center}
\begin{tabular}{l}
|\input{childdoc.def}|\\
|\childdocforward[|\textit{main}|]{|\textit{dest}|}|\\
\end{tabular}
\end{center}
%
The argument \textit{dest} is the destination file
(without extension).
It should be the main file or one of the child files.
Note that further \textsf{childdoc} directives
such as |\childdocof| and |\childdocforward|
in the indicated file will be processed in this form.
The optional argument \textit{main}
passes on directly to the main file \textit{main}
while pretending to compile the child \textit{dest}.
This form behaves as if \textit{dest}
issues |\childdocof{|\textit{main}|}| right away,
and no further \textsf{childdoc} directives will be processed.

%%%%%%%%%%%%%%%%%%%%%%%%%%%%%%%%%%%%%%%%
\DescribeMacro{\...prefix}
In the alternative form |\childdocforwardprefix|,
%
\begin{center}
\begin{tabular}{l}
|\input{childdoc.def}|\\
|\childdocforwardprefix[|\textit{main}|]{|\textit{prefix}|}{|\textit{dest}|}|
\end{tabular}
\end{center}
%
the destination file is determined by a pattern
depending on the current file:
To make this work, the current file must be called
`{\textit{prefix}\hspace{0.2em}\textit{suffix}}'
with \textit{prefix} matching precisely the argument.
Processing is then passed on to the file
`{\textit{dest}\hspace{0.2em}\textit{suffix}}'.
Surely, the same effect is achieved by
directly specifying the
argument `{\textit{dest}\hspace{0.2em}\textit{suffix}}'
in the first form.
However, that requires to set up a different file
for each child. With the alternative form of the command
all these files can have exactly the same content
which simplifies setting them up and maintaining them.

For example, the following file |draft.tex|
with a compilation flag |\version| as described in \secref{sec:flags}
compiles the main document as a draft:
%
\begin{center}
\begin{tabular}{l}
|\def\version{draft}|\\
|\input{childdoc.def}|\\
|\childdocforward{|\textit{main}|}|
\end{tabular}
\end{center}
%
Likewise, the following files |final|\textit{nn}|.tex|
compile the final version of the child document
|child|\textit{nn}|.tex|:
%
\begin{center}
\begin{tabular}{l}
|\def\version{final}|\\
|\input{childdoc.def}|\\
|\childdocforwardprefix{final}{child}|
\end{tabular}
\end{center}
%

Note that when several versions of a main file and/or of each child file
are to be generated, it may be convenient to set up a |Makefile| or
shell script to automatise the process.

%%%%%%%%%%%%%%%%%%%%%%%%%%%%%%%%%%%%%%%%%%%%%%%%%%%%%%%%%%%%%%%%%%%%%%%%%%%%%%%%
\subsection{Command Line Processing}
\label{sec:commandline}

The effect of redirection files can also be achieved by invoking
the \LaTeX{} compiler with a more elaborate command line.
Most conveniently this should be done as part
of a shell script or a |Makefile|.

When using \textsf{childdoc} in the main file, the following
command lines effectively perform a redirection
(note that depending on the shell being used,
backslashes may have to be doubled: `|\|' $\to$ `|\\|'):
%
\begin{center}
|... -jobname "|\textit{target}|" |\\|"|[\textit{flags}]%
|\input{childdoc.def}\childdocforward[|\textit{main}|]{|\textit{dest}|}"|
\end{center}
%
Here \textit{target} is the name of the output file,
\textit{main} is the name of the main file
and \textit{dest} is the name of the main or child file to be processed
(all filenames without extensions).
The optional argument \textit{main} can be omitted
if \textit{main} matches \textit{dest}.
Optionally, compilation \textit{flags} can be defined via |\def| commands.
This command line makes the \TeX{} engine believe
it is compiling the file \textit{target}
whose content is specified as the latter parameter.
The provided code then forwards the processing to
\textit{main} or \textit{dest} as described in \secref{sec:forward}.

%%%%%%%%%%%%%%%%%%%%%%%%%%%%%%%%%%%%%%%%%%%%%%%%%%%%%%%%%%%%%%%%%%%%%%%%%%%%%%%%
\subsection{Include by Input}
\label{sec:input}

Including child documents by |\include| has some restrictions by design.
Most notably, the content of a child document always occupies
its own set of pages; pages cannot be shared between child documents.
Usually, this behaviour makes perfect sense
because each child document contain an essential part of the document.
However, in some situations it may be desirable to compose
a document from a collection of parts
without having mandatory page breaks between then.
For this case, the package
provides a mechanism to include parts
by |\input| which can also be processed individually.
However, by construction this mechanism
requires manual handling of the content to be output.

%%%%%%%%%%%%%%%%%%%%%%%%%%%%%%%%%%%%%%%%
\DescribeMacro{\ifchilddocmanual}
The main file should be prepared as usual, see \secref{sec:include}.
However, the document body must make a distinction
between processing of an individual part and of the main document, e.g.:
%
\begin{center}
\begin{tabular}{l}
|\ifchilddocmanual|\\
|\input{\childdocname}|\\
|\||else|\\
\textit{document body with }|\input{|\textit{part}|}|\\
|\||fi|
\end{tabular}
\end{center}
%
The conditional |\ifchilddocmanual| is true whenever
a part to be included by |\input| is being compiled,
and the name of the part is stored in |\childdocname|.

%%%%%%%%%%%%%%%%%%%%%%%%%%%%%%%%%%%%%%%%
\DescribeMacro{\childdocby}
Each part to be included by |\input| should start with:
%
\begin{center}
\begin{tabular}{l}
|\input{childdoc.def}|\\
|\childdocby{|\textit{main}|}|\\
\end{tabular}
\end{center}
%
The directive |\childdocby| is similar to |\childdocof|
described in \secref{sec:include},
but the subsequent selection of content must be done manually.
To that end, both |\ifchilddoc| and |\ifchilddocmanual|
will be true upon processing of a part,
and the name of the part is stored in |\childdocname|.
Note that |\jobname| will be set to the filename of the current part
so that each part receives an individual |.aux| file
that does not interfere with the |.aux| file(s) of the main document.
This behaviour can be altered by the alternative form
|\childdocby[*]{|\textit{main}|}| (with a non-empty optional argument)
which uses the |.aux| file of the main document
by setting |\jobname| to \textit{main}.

%%%%%%%%%%%%%%%%%%%%%%%%%%%%%%%%%%%%%%%%%%%%%%%%%%%%%%%%%%%%%%%%%%%%%%%%%%%%%%%%
\subsection{Driver Development}
\label{sec:driver}

The \textsf{childdoc} mechanism can also be use for the development
of definition files such as \LaTeX{} styles or classes.
This case differs from the above setup with multiple parts
included by |\include| in that no |\includeonly| should be invoked.
This can be achieved by starting the include file
(before |\ProvidesPackage|) with:
%
\begin{center}
\begin{tabular}{l}
|\input{childdoc.def}|\\
|\childdocforward{|\textit{main}|}|\\
\end{tabular}
\end{center}
%
or alternatively with:
%
\begin{center}
\begin{tabular}{l}
|\input{childdoc.def}|\\
|\childdocby{|\textit{main}|}|\\
\end{tabular}
\end{center}
%
Both forms have slightly different effects as described above.
The main file is prepared as usual, see \secref{sec:include}.

%%%%%%%%%%%%%%%%%%%%%%%%%%%%%%%%%%%%%%%%%%%%%%%%%%%%%%%%%%%%%%%%%%%%%%%%%%%%%%%%
\subsection{Legacy Detection}
\label{sec:detection}

The directive |\childdocmain| in the main file can detect
whether the complete document or merely a child is to be compiled
even without using the directive |\childdocof|.
This method is deprecated because it is less robust
and there is no compelling reason to use it;
it is merely provided for backward compatibility
and it may be removed in future versions.

If the detection mechanism is to be used,
it is mandatory to correctly specify
the filename of the main file as the argument of |\childdocmain|:
%
\begin{center}
\begin{tabular}{l}
|\input{childdoc.def}|\\
|\childdocmain{|\textit{main}|}|\\
\end{tabular}
\end{center}
%
If |\jobname| does not match the argument \textit{main} of |\childdocmain|,
it is assumed that |\jobname| points to the child file to be compiled.
When using |\childdocmain| with the main file specified as argument,
it suffices to start a child file
with just |\input{|\textit{main}|}|
without loading of the package and using |\childdocof|.
If instead all processing is done
with the appropriate \textsf{childdoc} directives,
the argument of \textit{main} of |\childdocmain| can be empty.

An alternative version of the command line processing described
in \secref{sec:commandline} using the detection mechanism reads:
%
\begin{center}
|... -jobname "|\textit{target}|" "|[\textit{flags}]%
[|\def\jobname{|\textit{dest}|}|]|\input{|\textit{main}|}"|
\end{center}

%%%%%%%%%%%%%%%%%%%%%%%%%%%%%%%%%%%%%%%%%%%%%%%%%%%%%%%%%%%%%%%%%%%%%%%%%%%%%%%%
\subsection{Manual Code}
\label{sec:manual}

In case one cannot be certain whether the definitions file |childdoc.def|
is installed on the target \TeX{} distribution
and one prefers not to ship it,
it is conceivable to paste a few relevant commands into the sources.

To that end, drop all statements |\input{childdoc.def}|
and perform the replacements as outlined below.
Instead of |\childdocmain{|\textit{main}|}| add the following code
to the top of the main file:
%
\begin{center}
\begin{tabular}{l}
|\||ifdefined\childdocname\endinput\||fi\newif\ifchilddoc|\\
|\edef\childdocname{\scantokens\expandafter{\jobname\noexpand}}|\\
|\def\childdocmain{|\textit{main}|}\||ifx\childdocmain\childdocname\||else|\\
|\childdoctrue\includeonly{\childdocname}\let\jobname\childdocmain\||fi|\\
\end{tabular}
\end{center}
%
Instead of |\childdocof{|\textit{main}|}| just include the main file
at the top of each child file:
%
\begin{center}
|\input{|\textit{main}|}|
\end{center}
%
A simple redirection |\childdocforward{|\textit{dest}|}| is achieved by:
%
\begin{center}
|\def\jobname{|\textit{dest}|}\input{\jobname}|
\end{center}
%
The redirection with prefix
|\childdocforwardprefix[|\textit{prefix}|]{|\textit{dest}|}|
is accomplished by:
%
\begin{center}
\begin{tabular}{l}
|{\edef\jobname{\scantokens\expandafter{\jobname\noexpand}}|\\
|\def\redirectjob |\textit{prefix}|#1~~~{\gdef\jobname{|\textit{dest}|#1}}|\\
|\expandafter\redirectjob\jobname~~~}\input{\jobname}|
\end{tabular}
\end{center}

In an alternative approach,
child documents can be compiled by a specific command line
without additional code or specific definitions:
%
\begin{center}
|... -jobname "|\textit{target}|" "|[\textit{flags}]%
|\includeonly{|\textit{dest}|}\input{|\textit{main}|}"|
\end{center}
%

%%%%%%%%%%%%%%%%%%%%%%%%%%%%%%%%%%%%%%%%%%%%%%%%%%%%%%%%%%%%%%%%%%%%%%%%%%%%%%%%
%%%%%%%%%%%%%%%%%%%%%%%%%%%%%%%%%%%%%%%%%%%%%%%%%%%%%%%%%%%%%%%%%%%%%%%%%%%%%%%%
\section{Information}

%%%%%%%%%%%%%%%%%%%%%%%%%%%%%%%%%%%%%%%%%%%%%%%%%%%%%%%%%%%%%%%%%%%%%%%%%%%%%%%%
\subsection{Copyright}

Copyright \copyright{} 2017--2018 Niklas Beisert

This work may be distributed and/or modified under the
conditions of the \LaTeX{} Project Public License, either version 1.3
of this license or (at your option) any later version.
The latest version of this license is in
  \url{http://www.latex-project.org/lppl.txt}
and version 1.3 or later is part of all distributions of \LaTeX{}
version 2005/12/01 or later.

This work has the LPPL maintenance status `maintained'.

The Current Maintainer of this work is Niklas Beisert.

This work consists of the files |README.txt|, |childdoc.ins| and |childdoc.dtx|
as well as the derived files |childdoc.def|, |cdocsamp.tex|
with |cdocsch1.tex|, |cdocsch2.tex|, |cdocspt3.tex|, |cdocspt4.tex|,
|cdocsdrf.tex|, |cdocsfn1.tex|, |cdocsfn2.tex|
as well as |childdoc.pdf|.

%%%%%%%%%%%%%%%%%%%%%%%%%%%%%%%%%%%%%%%%%%%%%%%%%%%%%%%%%%%%%%%%%%%%%%%%%%%%%%%%
\subsection{Files and Installation}

The package consists of the files:
%
\begin{center}
\begin{tabular}{ll}
    |README.txt|   & readme file \\
    |childdoc.ins| & installation file \\
    |childdoc.dtx| & source file \\
    |childdoc.def| & definition file \\
    |cdocsamp.tex| & sample main file \\
    |cdocsch1.tex| & sample include file \\
    |cdocsch2.tex| & sample include file \\
    |cdocspt3.tex| & sample part file \\
    |cdocspt4.tex| & sample part file \\
    |cdocsdrf.tex| & sample redirection file \\
    |cdocsfn1.tex| & sample redirection file \\
    |cdocsfn2.tex| & sample redirection file \\
    |childdoc.pdf| & manual
\end{tabular}
\end{center}
%
The distribution consists of the files
|README.txt|, |childdoc.ins| and |childdoc.dtx|.
%
\begin{itemize}
\item
Run (pdf)\LaTeX{} on |childdoc.dtx|
to compile the manual |childdoc.pdf| (this file).
\item
Run \LaTeX{} on |childdoc.ins| to create the definitions file |childdoc.def|
and the sample |cdocsamp.tex| with include files
|cdocsch1.tex|, |cdocsch2.tex|, |cdocspt3.tex|, |cdocspt4.tex|,
|cdocsdrf.tex|, |cdocsfn1.tex|, |cdocsfn2.tex|.
Then copy the file |childdoc.def| to an appropriate directory of your \LaTeX{}
distribution, e.g.\ \textit{texmf-root}|/tex/latex/childdoc|.
\end{itemize}

%%%%%%%%%%%%%%%%%%%%%%%%%%%%%%%%%%%%%%%%%%%%%%%%%%%%%%%%%%%%%%%%%%%%%%%%%%%%%%%%
\subsection{Related CTAN Packages}

There are several other packages which offer a similar functionality:
%
\begin{itemize}
\item
The packages
\href{http://ctan.org/pkg/docmute}{\textsf{docmute}},
\href{http://ctan.org/pkg/includex}{\textsf{includex}} and
\href{http://ctan.org/pkg/standalone}{\textsf{standalone}}
provide commands to include only the document body of
a child file thus allowing both files to be compiled individually.
\item
The packages \href{http://ctan.org/pkg/subdocs}{\textsf{subdocs}}
and \href{http://ctan.org/pkg/subfiles}{\textsf{subfiles}}
provide structures in which the main and child documents can be
encapsulated and allowing them to be compiled individually.
The inclusion mechanism is different from the conventional |\include|.
\item
The package \href{http://ctan.org/pkg/combine}{\textsf{combine}}
is an elaborate solution to combine several documents into one.
\end{itemize}
%
See also the CTAN topic \href{http://ctan.org/topic/subdocs}{\textsf{subdocs}}
for further related packages.
The present package differs from the above solutions in that
a document structure constructed with the conventional |\include| mechanism
just needs two extra commands at the top of every file
such that all constituent files can be compiled individually.

%%%%%%%%%%%%%%%%%%%%%%%%%%%%%%%%%%%%%%%%%%%%%%%%%%%%%%%%%%%%%%%%%%%%%%%%%%%%%%%%
%\subsection{Feature Suggestions}
%
%The following is a list of features which may be useful for future
%versions of this package:
%%
%\begin{itemize}
%\item
%\ldots
%\end{itemize}

%%%%%%%%%%%%%%%%%%%%%%%%%%%%%%%%%%%%%%%%%%%%%%%%%%%%%%%%%%%%%%%%%%%%%%%%%%%%%%%%
\subsection{Revision History}

%%%%%%%%%%%%%%%%%%%%%%%%%%%%%%%%%%%%%%%%
\paragraph{v2.0:} 2018/12/30

\begin{itemize}
\item
immediate forward processing
\item
added |\childdocby| mechanism
\item
manual restructured
\end{itemize}

%%%%%%%%%%%%%%%%%%%%%%%%%%%%%%%%%%%%%%%%
\paragraph{v1.6:} 2018/01/17

\begin{itemize}
\item
application for development of include files
\item
corrections to manual
\end{itemize}

%%%%%%%%%%%%%%%%%%%%%%%%%%%%%%%%%%%%%%%%
\paragraph{v1.5:} 2017/05/21

\begin{itemize}
\item
more complete structuring introduced
\item
|\childdocof| introduced
\item
|\childdoc| renamed to |\childdocmain|
\item
|\childredirect| renamed to |\childdocforward| and |\childdocforwardprefix|
and functionality expanded
\end{itemize}

%%%%%%%%%%%%%%%%%%%%%%%%%%%%%%%%%%%%%%%%
\paragraph{v1.0:} 2017/04/27

\begin{itemize}
\item
manual and install package
\item
first version published on CTAN
\end{itemize}

%%%%%%%%%%%%%%%%%%%%%%%%%%%%%%%%%%%%%%%%
\paragraph{v0.6:} 2017/04/26

\begin{itemize}
\item
redirection mechanism added
\end{itemize}

%%%%%%%%%%%%%%%%%%%%%%%%%%%%%%%%%%%%%%%%
\paragraph{v0.5:} 2017/04/26

\begin{itemize}
\item
functionality in definition file
\end{itemize}


%%%%%%%%%%%%%%%%%%%%%%%%%%%%%%%%%%%%%%%%%%%%%%%%%%%%%%%%%%%%%%%%%%%%%%%%%%%%%%%%
%%%%%%%%%%%%%%%%%%%%%%%%%%%%%%%%%%%%%%%%%%%%%%%%%%%%%%%%%%%%%%%%%%%%%%%%%%%%%%%%
%%%%%%%%%%%%%%%%%%%%%%%%%%%%%%%%%%%%%%%%%%%%%%%%%%%%%%%%%%%%%%%%%%%%%%%%%%%%%%%%
\appendix

\settowidth\MacroIndent{\rmfamily\scriptsize 000\ }

 \DocInput{childdoc.dtx}

\end{document}
%</driver>
% \fi
%
% %%%%%%%%%%%%%%%%%%%%%%%%%%%%%%%%%%%%%%%%%%%%%%%%%%%%%%%%%%%%%%%%%%%%%%%%%%%%%%
% %%%%%%%%%%%%%%%%%%%%%%%%%%%%%%%%%%%%%%%%%%%%%%%%%%%%%%%%%%%%%%%%%%%%%%%%%%%%%%
% \section{Sample}
%\iffalse
%<*samplemain>
%\fi
%
% The following presents a sample document
% with two chapters, two parts, a title page,
% a compile flag as well as three forwarding files to set the flag.
% It consists of eight |.tex| files:
% \begin{center}
% \begin{tabular}{ll}
% |cdocsamp.tex|&main file\\
% |cdocsch1.tex|&include file for chapter 1\\
% |cdocsch2.tex|&include file for chapter 2\\
% |cdocspt3.tex|&include file for part 3\\
% |cdocspt4.tex|&include file for part 4\\
% |cdocsdrf.tex|&forwarding file for main file in draft mode\\
% |cdocsfi1.tex|&forwarding file for final version of chapter 1\\
% |cdocsfi2.tex|&forwarding file for final version of chapter 2\\
% \end{tabular}
% \end{center}
% Each of the eight files can be compiled directly by the \LaTeX{} compiler.
%
% %%%%%%%%%%%%%%%%%%%%%%%%%%%%%%%%%%%%%%
% \paragraph{Main File.}
%
% The main file is called |cdocsamp.tex|.
%
% Load the \textsf{childdoc} definitions and
% declare the filename for the main document:
%    \begin{macrocode}
\input{childdoc.def}
\childdocmain{}
%    \end{macrocode}

% Optional override for |\version| flag:
%    \begin{macrocode}
%%\ifchilddoc\else\providecommand{\version}{draft}\fi
%    \end{macrocode}

% Define the default values for the |\version| flag
% (|final| for the main file and |draft| for childs):
%    \begin{macrocode}
\ifchilddoc
\providecommand{\version}{draft}
\else
\providecommand{\version}{final}
\fi
%    \end{macrocode}

% Load the standard document class:
%    \begin{macrocode}
\documentclass[12pt]{article}
%    \end{macrocode}

% Start the document body:
%    \begin{macrocode}
\begin{document}
%    \end{macrocode}

% Declare a title page.
% Print title, part of document being processed and version flag:
%    \begin{macrocode}
\addtocounter{page}{-1}
\begin{center}
{\LARGE\bfseries{}childdoc example\par}
\vspace{1cm}
\ifchilddoc
\ifchilddocmanual part\else chapter\fi:
`\childdocname' of `\childdocjob'\par
\else
main document: `\childdocjob'\par
\fi
version: \version\par
\end{center}
\newpage
%    \end{macrocode}

% Manually include selected file,
% otherwise process as usual:
%    \begin{macrocode}
\ifchilddocmanual
\section*{part `\childdocname'}
\input{\childdocname}
\else
%    \end{macrocode}

% Include the two chapters:
%    \begin{macrocode}
\include{cdocsch1}
\include{cdocsch2}
%    \end{macrocode}

% Include the two parts unless only chapters should be displayed:
%    \begin{macrocode}
\ifchilddoc\else
\section{part three}
\input{cdocspt3}
\section{part four}
\input{cdocspt4}
\fi
%    \end{macrocode}

% Process as usual until here:
%    \begin{macrocode}
\fi
%    \end{macrocode}

% End of document body:
%    \begin{macrocode}
\end{document}
%    \end{macrocode}
%\iffalse
%</samplemain>
%\fi
%
% %%%%%%%%%%%%%%%%%%%%%%%%%%%%%%%%%%%%%%
% \paragraph{Chapter Include Files.}
%
% The include files are called |cdocsch1.tex| and |cdocsch2.tex|.
%
%\iffalse
%<*samplechap1|samplechap2>
%\fi

% Optional override for |\version| flag:
%    \begin{macrocode}
%%\providecommand{\version}{final}
%    \end{macrocode}

% Include the main document:
%    \begin{macrocode}
\input{childdoc.def}
\childdocof{cdocsamp}
%    \end{macrocode}

%\iffalse
%</samplechap1|samplechap2>
%\fi
%
%\iffalse
%<*samplechap1>
%\fi
% Some text for chapter 1:
%    \begin{macrocode}
\section{one}
some text in chapter one
%    \end{macrocode}

%\iffalse
%</samplechap1>
%\fi
% Some text for chapter 2:
%\iffalse
%<*samplechap2>
%\fi
%    \begin{macrocode}
\section{two}
more text in chapter two
%    \end{macrocode}

%\iffalse
%</samplechap2>
%\fi
%
% %%%%%%%%%%%%%%%%%%%%%%%%%%%%%%%%%%%%%%
% \paragraph{Part Include Files.}
%
% The include files are called |cdocspt3.tex| and |cdocspt4.tex|.
%
%\iffalse
%<*samplepart3|samplepart4>
%\fi

% Optional override for |\version| flag:
%    \begin{macrocode}
%%\providecommand{\version}{final}
%    \end{macrocode}

% Include the main document:
%    \begin{macrocode}
\input{childdoc.def}
\childdocby{cdocsamp}
%    \end{macrocode}

%\iffalse
%</samplepart3|samplepart4>
%\fi
%
%\iffalse
%<*samplepart3>
%\fi
% Some text for part 3:
%    \begin{macrocode}
some text in part three
%    \end{macrocode}

%\iffalse
%</samplepart3>
%\fi
% Some text for part 4:
%\iffalse
%<*samplepart4>
%\fi
%    \begin{macrocode}
more text in part four
%    \end{macrocode}

%\iffalse
%</samplepart4>
%\fi
%
% %%%%%%%%%%%%%%%%%%%%%%%%%%%%%%%%%%%%%%
% \paragraph{Forwarding for a Complete Draft.}
%
% The following forwarding file |cdocsdrf.tex|
% compiles the main document in draft mode:
%\iffalse
%<*sampledraft>
%\fi
%    \begin{macrocode}
\def\version{draft}
\input{childdoc.def}
\childdocforward{cdocsamp}
%    \end{macrocode}

%\iffalse
%</sampledraft>
%\fi
%
% %%%%%%%%%%%%%%%%%%%%%%%%%%%%%%%%%%%%%%
% \paragraph{Forwarding for Final Version of the Chapters.}
%
% The following forwarding files |cdocsfn1.tex| and |cdocsfn2.tex|
% (with identical content)
% compile the final versions of the child documents
% |cdocsch1.tex| and |cdocsch2.tex|, respectively:
%\iffalse
%<*samplefinal>
%\fi
%    \begin{macrocode}
\def\version{final}
\input{childdoc.def}
\childdocforwardprefix[cdocsamp]{cdocsfn}{cdocsch}
%    \end{macrocode}

%\iffalse
%</samplefinal>
%\fi
%
% %%%%%%%%%%%%%%%%%%%%%%%%%%%%%%%%%%%%%%
% \paragraph{Command Line Processing.}
%
% The following three command lines generate the output files
% |cdocscld|, |cdocscl1| and |cdocscl2|
% which should be identical to
% |cdocsdrf|, |cdocsch1| and |cdocsfn2|, respectively:
% \begin{center}
% \begin{tabular}{l}
% |latex -jobname cdocscld \|\\
% |  "\def\version{draft}\input{childdoc.def}\childdocforward{cdocsamp}"|\\
% |latex -jobname cdocscl1 \|\\
% |  "\input{childdoc.def}\childdocforward[cdocsamp]{cdocsch1}"|\\
% |latex -jobname cdocscl2 \|\\
% |  "\def\version{final}\input{childdoc.def}\childdocforward{cdocsch2}"|
% \end{tabular}
% \end{center}
% Note that the trailing backslash on each first line
% merely continues the input to the second line
% (for convenient cut ant paste).
% Furthermore, the command |latex| can be replaced by any
% of its alternative versions such as |pdflatex|.
%
% %%%%%%%%%%%%%%%%%%%%%%%%%%%%%%%%%%%%%%%%%%%%%%%%%%%%%%%%%%%%%%%%%%%%%%%%%%%%%%
% %%%%%%%%%%%%%%%%%%%%%%%%%%%%%%%%%%%%%%%%%%%%%%%%%%%%%%%%%%%%%%%%%%%%%%%%%%%%%%
% \section{Implementation}
%\iffalse
%<*package>
%\fi
%
% This section describes the definitions file |childdoc.def|.

% The definitions cannot be loaded using |\usepackage| or |\RequirePackage|
% which has a mechanism to prevent loading a style file more than once.
% When loading the definitions by means of |\input|
% multiple instances have to be prevented manually:
%\iffalse
%This code needs to be before the `\ProvidesFile' directive
%which is defined at the beginning of this file.
%Therefore it is also placed there and commented out here.
%</package>
%<*discard>
%\fi
%    \begin{macrocode}
\ifdefined\childdocmain\endinput\fi
%    \end{macrocode}
%\iffalse
%</discard>
%<*package>
%\fi
%
% \macro{\ifchilddoc}
% \macro{\ifchilddocmanual}
% The conditional |\ifchilddoc| tells whether a
% child (true) or main (false) document is being compiled.
% The conditional |\ifchilddocmanual| tells whether
% the |\includeonly| mechanism is used (false) or
% the selection of child files must be performed manually (true).
% The definitions initialise to false:
%    \begin{macrocode}
\newif\ifchilddoc
\newif\ifchilddocmanual
%    \end{macrocode}

% \macro{\childdocname}
% \macro{\childdocjob}
% The macro |\childdocname| stores the name of the main document
% to be compiled. The macro |\childdocjob| stores the name of
% the document on which the \LaTeX{} compiler was originally invoked.
% The content of |\jobname| cannot be compared
% to filenames specified in the source due to different catcodes.
% The following code rescans |\jobname|, stores the result
% in |\childdocname| and saves a copy in |\childdocjob|:
%    \begin{macrocode}
\edef\childdocname{\scantokens\expandafter{\jobname\noexpand}}
\let\childdocjob\childdocname
%    \end{macrocode}

% \macro{\childdocdisable}
% The macro |\childdocdisable| prevents the main file
% from being processed more than once.
% At this stage, the main document command |\childdocmain|
% is assumed to be called once again where it should do nothing.
% Any subsequent call to it should prevent
% a secondary processing of the main document
% It overwrites the forwarding commands
% |\childdocof| and |\childdocforward|
% with empty macros to prevent further inclusions of the main document:
%    \begin{macrocode}
\newcommand{\childdocdisable}
{
  \renewcommand{\childdocmain}[1]{\renewcommand{\childdocmain}[1]{\endinput}}
  \renewcommand{\childdocof}[1]{}
  \renewcommand{\childdocby}[2][]{}
  \renewcommand{\childdocforward}[2][]{}
  \renewcommand{\childdocdisable}{}
}
%    \end{macrocode}

% \macro{\childdocmain}
% The macro |\childdocmain| is to be called at the top of the main file
% with nothing or the main filename (without extension) as argument.
% First, it breaks loops.
% If the argument is not empty and does not match |\childdocname|
% (which is set by the first inclusion of |childdoc.def|),
% |\ifchilddoc| is set to true, |\includeonly| is applied to the child file
% and |\jobname| is set to the main file
% (for proper handling of |.aux| files):
%    \begin{macrocode}
\newcommand{\childdocmain}[1]
{
  \childdocdisable\childdocmain{}
  \if?#1?\else
    \begingroup
      \def\childdoctmp{#1}
      \ifx\childdoctmp\childdocname
        \def\childdoctmp{}
      \else
        \def\childdoctmp
        {
          \childdoctrue
          \includeonly{\childdocname}
          \def\childdocjob{#1}
          \def\jobname{#1}
        }
      \fi
      \expandafter
    \endgroup
    \childdoctmp
  \fi
}
%    \end{macrocode}

% \macro{\childdocof}
% The command |\childdocof| redirects
% compilation to the main file |#1|.
%    \begin{macrocode}
\newcommand{\childdocof}[1]
{
  \childdocdisable
  \childdoctrue
  \includeonly{\childdocname}
  \def\jobname{#1}
  \def\childdocjob{#1}
  \input{#1}
}
%    \end{macrocode}

% \macro{\childdocby}
% The command |\childdocby| ....
%    \begin{macrocode}
\newcommand{\childdocby}[2][]
{
  \childdocdisable
  \childdoctrue
  \childdocmanualtrue
  \if?#1?\else
    \def\jobname{#2}
  \fi
  \def\childdocjob{#2}
  \input{#2}
  \endinput
}
%    \end{macrocode}

% \macro{\childdocforward}
% The command |\childdocforward| redirects
% compilation to the main file or
% (if the optional argument is given) a child file.
% Parameters are set as if the main file
% or a child file starting with |\childdocof| was compiled.
% Then compilation is handed over to the main file:
%    \begin{macrocode}
\newcommand{\childdocforward}[2][]
{
  \begingroup
    \if?#1?
      \def\childdoctmp
      {
        \def\childdocname{#2}
        \def\childdocjob{#2}
        \def\jobname{#2}
        \input{#2}
        \endinput
      }
    \else
      \def\childdoctmp
      {
        \childdocdisable
        \def\childdocname{#2}
        \childdoctrue
        \includeonly{#2}
        \def\childdocjob{#1}
        \def\jobname{#1}
        \input{#1}
        \endinput
      }
    \fi
    \expandafter
  \endgroup
  \childdoctmp
}
%    \end{macrocode}

% \macro{\childdocforwardprefix}
% The command |\childdocforwardprefix| redirects
% compilation to the main or a child file by means of a pattern.
% The prefix |#1| in the current filename is replaced by |#2|
% and the suffix of the current filename is kept
% (it is assumed that the filename does not contain the substring `|~~~|'
% which is used as a delimiter).
% Compilation is handed over to the new file by |\childdocforward|:
%    \begin{macrocode}
\newcommand{\childdocforwardprefix}[3][]
{
  \begingroup
    \def\childdocextract #2##1~~~{\def\childdoctmp{\childdocforward[#1]{#3##1}}}
    \expandafter\childdocextract\childdocname~~~
    \expandafter
  \endgroup
  \childdoctmp
}
%    \end{macrocode}

% \macro{\childdoc}
% The deprecated macro |\childdoc| is a legacy version of |\childdocmain|:
%    \begin{macrocode}
\newcommand{\childdoc}{\childdocmain}
%    \end{macrocode}

% \macro{\childdocredirect}
% The deprecated macro |\childdocredirect| is a legacy version
% of |\childdocforward| and |\childdocforwardprefix|:
%    \begin{macrocode}
\newcommand{\childdocredirect}[2][]
{
  \begingroup
    \if?#1?
      \def\childdoctmp{\childdocforward{#2}}
    \else
      \def\childdoctmp{\childdocforwardprefix{#1}{#2}}
    \fi
    \expandafter
  \endgroup
  \childdoctmp
}
%    \end{macrocode}

%\iffalse
%</package>
%\fi
%
\endinput
\childdocforward[|\textit{main}|]{|\textit{dest}|}"|
\end{center}
%
Here \textit{target} is the name of the output file,
\textit{main} is the name of the main file
and \textit{dest} is the name of the main or child file to be processed
(all filenames without extensions).
The optional argument \textit{main} can be omitted
if \textit{main} matches \textit{dest}.
Optionally, compilation \textit{flags} can be defined via |\def| commands.
This command line makes the \TeX{} engine believe
it is compiling the file \textit{target}
whose content is specified as the latter parameter.
The provided code then forwards the processing to
\textit{main} or \textit{dest} as described in \secref{sec:forward}.

%%%%%%%%%%%%%%%%%%%%%%%%%%%%%%%%%%%%%%%%%%%%%%%%%%%%%%%%%%%%%%%%%%%%%%%%%%%%%%%%
\subsection{Include by Input}
\label{sec:input}

Including child documents by |\include| has some restrictions by design.
Most notably, the content of a child document always occupies
its own set of pages; pages cannot be shared between child documents.
Usually, this behaviour makes perfect sense
because each child document contain an essential part of the document.
However, in some situations it may be desirable to compose
a document from a collection of parts
without having mandatory page breaks between then.
For this case, the package
provides a mechanism to include parts
by |\input| which can also be processed individually.
However, by construction this mechanism
requires manual handling of the content to be output.

%%%%%%%%%%%%%%%%%%%%%%%%%%%%%%%%%%%%%%%%
\DescribeMacro{\ifchilddocmanual}
The main file should be prepared as usual, see \secref{sec:include}.
However, the document body must make a distinction
between processing of an individual part and of the main document, e.g.:
%
\begin{center}
\begin{tabular}{l}
|\ifchilddocmanual|\\
|\input{\childdocname}|\\
|\||else|\\
\textit{document body with }|\input{|\textit{part}|}|\\
|\||fi|
\end{tabular}
\end{center}
%
The conditional |\ifchilddocmanual| is true whenever
a part to be included by |\input| is being compiled,
and the name of the part is stored in |\childdocname|.

%%%%%%%%%%%%%%%%%%%%%%%%%%%%%%%%%%%%%%%%
\DescribeMacro{\childdocby}
Each part to be included by |\input| should start with:
%
\begin{center}
\begin{tabular}{l}
|% \iffalse
%
% childdoc.dtx Copyright (C) 2017-2018 Niklas Beisert
%
% This work may be distributed and/or modified under the
% conditions of the LaTeX Project Public License, either version 1.3
% of this license or (at your option) any later version.
% The latest version of this license is in
%   http://www.latex-project.org/lppl.txt
% and version 1.3 or later is part of all distributions of LaTeX
% version 2005/12/01 or later.
%
% This work has the LPPL maintenance status `maintained'.
%
% The Current Maintainer of this work is Niklas Beisert.
%
% This work consists of the files childdoc.dtx and childdoc.ins
% and the derived files childdoc.def and cdocsamp.tex with
% cdocsch1.tex, cdocsch2.tex, cdocsdrf.tex, cdocsfn1.tex, cdocsfn2.tex.
%
%<package>\ifdefined\childdocmain\endinput\fi
%<package>\ProvidesFile{childdoc.def}[2018/12/30 v2.0 child document driver]
%<samplemain>\ProvidesFile{cdocsamp.tex}[2018/12/30 v2.0 sample for childdoc]
%<*driver>
%\ProvidesFile{childdoc.drv}[2018/12/30 v2.0 childdoc reference manual file]
\PassOptionsToClass{10pt,a4paper}{article}
\documentclass{ltxdoc}

\usepackage[margin=35mm]{geometry}
\usepackage{hyperref}
\usepackage{hyperxmp}
\usepackage[usenames]{color}

\hypersetup{colorlinks=true}
\hypersetup{pdfstartview=FitH}
\hypersetup{pdfpagemode=UseNone}
\hypersetup{pdfsource={}}
\hypersetup{pdflang={en-UK}}
\hypersetup{pdfcopyright={Copyright 2017-2018 Niklas Beisert.
  This work may be distributed and/or modified under the
  conditions of the LaTeX Project Public License, either version 1.3
  of this license or (at your option) any later version.}}
\hypersetup{pdflicenseurl={http://www.latex-project.org/lppl.txt}}
\hypersetup{pdfcontactaddress={ETH Zurich, ITP, HIT K,
  Wolfgang-Pauli-Strasse 27}}
\hypersetup{pdfcontactpostcode={8093}}
\hypersetup{pdfcontactcity={Zurich}}
\hypersetup{pdfcontactcountry={Switzerland}}
\hypersetup{pdfcontactemail={nbeisert@itp.phys.ethz.ch}}
\hypersetup{pdfcontacturl={http://people.phys.ethz.ch/\xmptilde nbeisert/}}

\newcommand{\secref}[1]{\hyperref[#1]{section \ref*{#1}}}

\parskip1ex
\parindent0pt
\let\olditemize\itemize
\def\itemize{\olditemize\parskip0pt}

\begin{document}

\title{The \textsf{childdoc} Package}
\hypersetup{pdftitle={The childdoc Package}}
\author{Niklas Beisert\\[2ex]
  Institut f\"ur Theoretische Physik\\
  Eidgen\"ossische Technische Hochschule Z\"urich\\
  Wolfgang-Pauli-Strasse 27, 8093 Z\"urich, Switzerland\\[1ex]
  \href{mailto:nbeisert@itp.phys.ethz.ch}
  {\texttt{nbeisert@itp.phys.ethz.ch}}}
\hypersetup{pdfauthor={Niklas Beisert}}
\hypersetup{pdfsubject={Manual for the LaTeX2e Package childdoc}}
\date{30 December 2018, \textsf{v2.0}}
\maketitle

\begin{abstract}\noindent
\textsf{childdoc} is a \LaTeXe{} package
that enables the direct compilation
of document sections included by |\include|
to individual files.
\end{abstract}

\begingroup
\parskip0ex
\tableofcontents
\endgroup

%%%%%%%%%%%%%%%%%%%%%%%%%%%%%%%%%%%%%%%%%%%%%%%%%%%%%%%%%%%%%%%%%%%%%%%%%%%%%%%%
%%%%%%%%%%%%%%%%%%%%%%%%%%%%%%%%%%%%%%%%%%%%%%%%%%%%%%%%%%%%%%%%%%%%%%%%%%%%%%%%
\section{Introduction}

\LaTeX{} provides a mechanism to structure a large document (such as a book)
into a main file and several child files (containing the chapters)
using the |\include| command.
This mechanism is beneficial for documents
which span hundreds of pages in order to
make the source file(s) more manageable.
Moreover, compilation can be restricted to
selected child files by means of the |\includeonly| command.
The latter feature can be used to reduce the compilation time while editing
(this was significantly more useful in the earlier days of \LaTeX{})
or to generate a smaller document which is easier to navigate.
Another application of |\includeonly| is to generate
documents consisting of selected parts of the complete document.

However, there are a few drawbacks of the plain |\include| mechanism:
\begin{itemize}
\item
The child files cannot be compiled on their own,
they can only be compiled via the main file.
A naive editing environment
(such as a text editor with an option
to have the current file processed by \LaTeX)
may require one to switch to the main file before compiling;
attempting to compile the child file produces errors.
\item
The main file must be modified (each time)
to adjust the |\includeonly| command
to the present needs. This easily leaves the main file in a messy state.
\item
The generated document will always carry the filename
of the main document. This is inconvenient if
several child files are to be compiled and
to be kept for distribution.
\end{itemize}

The present package provides a simple interface
to make child files individually compilable by \LaTeX{}.
Compiling a child file then has the same effect as compiling
the main file with an |\includeonly| command
to select the appropriate child.
Moreover the generated document will carry the name of the child
rather than the main file.
This resolves all three above issues.

This feature is meant to make the editing of books,
thesis documents and lecture notes somewhat more convenient.
However, the package can also be used efficiently for
composing a series of documents (such as exercise sheets)
which are typically distributed individually.
It then assists the author in generating the individual documents
(potentially in different versions)
as well as a document containing the collected series.
Another application is in developing style files
or other kinds of included material
where compilation of the style file could redirect
to a sample or test file.

%%%%%%%%%%%%%%%%%%%%%%%%%%%%%%%%%%%%%%%%%%%%%%%%%%%%%%%%%%%%%%%%%%%%%%%%%%%%%%%%
%%%%%%%%%%%%%%%%%%%%%%%%%%%%%%%%%%%%%%%%%%%%%%%%%%%%%%%%%%%%%%%%%%%%%%%%%%%%%%%%
\section{Usage}

First of all, the package \textsf{childdoc} is \emph{not} a standard
\LaTeXe{} |.sty| style file! Therefore it needs to be invoked in
a non-standard way.

%%%%%%%%%%%%%%%%%%%%%%%%%%%%%%%%%%%%%%%%%%%%%%%%%%%%%%%%%%%%%%%%%%%%%%%%%%%%%%%%
\subsection{Included Files}
\label{sec:include}

%%%%%%%%%%%%%%%%%%%%%%%%%%%%%%%%%%%%%%%%
\DescribeMacro{\childdocmain}
To use the package, add the commands
\begin{center}
\begin{tabular}{l}
|\input{childdoc.def}|\\
|\childdocmain{}|\\
\end{tabular}
\end{center}
at the very top of the main \LaTeX{} file,
in particular \emph{before} the |\documentclass| statement!
The argument of |\childdocmain| should be left empty
(but it must be present).

%%%%%%%%%%%%%%%%%%%%%%%%%%%%%%%%%%%%%%%%
\DescribeMacro{\childdocof}
Furthermore, add the commands
\begin{center}
\begin{tabular}{l}
|\input{childdoc.def}|\\
|\childdocof{|\textit{main}|}|\\
\end{tabular}
\end{center}
at the top of every child file \textit{child}
which is included by |\include{|\textit{child}|}|
from within the main file
(or at least for those files to be compiled individually).
The argument \textit{main} must be the filename of the main file.

There are a couple of
considerations in setting up the main and child documents:

%%%%%%%%%%%%%%%%%%%%%%%%%%%%%%%%%%%%%%%%
\paragraph{Restrictions.}

Please note the following restrictions:
\begin{itemize}
\item
|\childdocmain| must be called with one argument \textit{main}
to ensure compatibility with earlier version of the package.
It must either be empty (|\childdocmain{}|)
or precisely match the filename of the main file in which it is specified.
See \secref{sec:detection} for further information.
\item
The filename \textit{main} must be specified without the |.tex| extension.
\item
The filename \textit{main} is case sensitive
(even in case-insensitive file systems)
due to internal string comparison.
\item
The argument \textit{main} should be fully expanded, it cannot be a macro.
\item
Subdirectories and special characters should be avoided in filenames.
\item
The command |\childdocmain{|\textit{main}|}| must be followed by a whitespace.
It should not be followed immediately by another command
or by a comment mark `|%|'.
This is because the \TeX{} parser reads the token immediately following
the argument of |\childdocmain| and puts it
at the beginning of every child section;
however, a white\-space is ignored.
\end{itemize}

%%%%%%%%%%%%%%%%%%%%%%%%%%%%%%%%%%%%%%%%
\paragraph{Content of Main File.}

It is advisable to place all content in the child files included by |\include|.
Any output contained in the main file will appear in all child documents
unless suppressed manually;
it cannot be suppressed automatically by the |\includeonly| directive
and thus should normally be avoided.
A method to include some content in the main file
by means of conditional processing is described in \secref{sec:conditional}.

%%%%%%%%%%%%%%%%%%%%%%%%%%%%%%%%%%%%%%%%
\paragraph{Page Numbering.}

When only a part of the document is compiled,
the appropriate numbering of pages
(as well as other status parameters)
is determined from the |.aux| files.
The latter contain information from previous passes.
However this information needs to propagate through
all intermediate child documents.
Therefore the page numbering in child documents may well
be inconsistent until the complete document is compiled at least once.

A useful (if unconventional) way to always ensure a consistent
page numbering is to restart the numbering in each child document
and denote the pages by `\textit{child}|.|\textit{page}'
where \textit{child} represents the chapter/section number of the child file.
This can be achieved by the command
|\numberwithin{page}{|\textit{child}|}|
of the \textsf{amsmath} package
where \textit{child} can be |chapter| or |section|
depending on the chosen structuring.
Alternatively, one can modify the macro |\thepage| appropriately
and reset the counter |page| at the start of each child file.

%%%%%%%%%%%%%%%%%%%%%%%%%%%%%%%%%%%%%%%%%%%%%%%%%%%%%%%%%%%%%%%%%%%%%%%%%%%%%%%%
\subsection{Conditional Processing}
\label{sec:conditional}

The package provides a mechanism to compile different versions
of a document. To customise the versions further some conditional processing
can come in handy to distinguish which version is being compiled.
The package provides two macros to describe the compilation context:

%%%%%%%%%%%%%%%%%%%%%%%%%%%%%%%%%%%%%%%%
\DescribeMacro{\ifchilddoc}
The conditional |\ifchilddoc| distinguishes between the compilation of
child documents and the main document:
%
\begin{center}
|\ifchilddoc |\textit{child-code}| |[|\||else |\textit{main-code}]| \||fi|
\end{center}

%%%%%%%%%%%%%%%%%%%%%%%%%%%%%%%%%%%%%%%%
\DescribeMacro{\childdocname}
\DescribeMacro{\childdocjob}
The macro |\childdocname| contains the filename (without extension)
of the main or child file being processed.
Note that |\childdocjob| will always contain the name of the main file.

%%%%%%%%%%%%%%%%%%%%%%%%%%%%%%%%%%%%%%%%
\paragraph{Title Page.}

Conditional processing can be used to include a title or banner page
in the main document when proper precautions are taken.
Importantly, the code in the main file should ensure that the page counter
(as well as other status parameters which are stored in the |.aux| files)
takes the same value after the conditional processing.
Otherwise the page numbers may take divergent values
depending on which part is compiled.

For example, a title page could be declared by:
%
\begin{center}
\begin{tabular}{l}
|\ifchilddoc\||else|\\
|\addtocounter{page}{-1}|\\
\textit{code for title page}\\
|\newpage|\\
|\||fi|
\end{tabular}
\end{center}
%
A banner page for the child documents can be generated by:
%
\begin{center}
\begin{tabular}{l}
|\ifchilddoc|\\
|\addtocounter{page}{-1}|\\
\textit{code for banner page}\\
|\newpage|\\
|\||fi|
\end{tabular}
\end{center}
%
Here one could write a message such as:
\begin{center}
|This is the part \childdocname{} of \childdocjob{}.|
\end{center}

%%%%%%%%%%%%%%%%%%%%%%%%%%%%%%%%%%%%%%%%%%%%%%%%%%%%%%%%%%%%%%%%%%%%%%%%%%%%%%%%
\subsection{Flags}
\label{sec:flags}

The package makes it easy to generate different versions
of the main or child documents.
To this end compilation flags can be defined
and assigned different default values.
They will be particularly useful in conjunction
with the forwarding mechanism described in \secref{sec:forward}.

For example, it may be useful to have a flag |\version|
which can be set to |draft| or |final|.
The document source will contain some conditional code
depending on the value of |\version|.
Suppose further, the flag should default to |final| for the main file
and to |draft| for child files
which is a natural assignment for editing the document.
This is achieved by placing the following code
in the preamble of the main document
(below the |\childdocmain| directive):
%
\begin{center}
\begin{tabular}{l}
|\ifchilddoc|\\
|\providecommand{\version}{draft}|\\
|\||else|\\
|\providecommand{\version}{final}|\\
|\||fi|
\end{tabular}
\end{center}
%
The definition by |\providecommand| makes sure
that previous definitions are not overwritten.
Further statements |\providecommand{\version}{...}|
can thus be added before the above code to override it.

For the main file, one might add a line
(between |\childdocmain| and the above block)
%
\begin{center}
|%\ifchilddoc\||else\providecommand{\version}{draft}\||fi|
\end{center}
%
which can be uncommented to produce a draft version.
Likewise one can add a line to the very top of a child file
(above the |\childdocof{|\textit{main}|}| directive)
%
\begin{center}
|%\providecommand{\version}{final}|
\end{center}
%
which can be uncommented to produce the final version of this child document.

%%%%%%%%%%%%%%%%%%%%%%%%%%%%%%%%%%%%%%%%%%%%%%%%%%%%%%%%%%%%%%%%%%%%%%%%%%%%%%%%
\subsection{Forwarding}
\label{sec:forward}

Different versions of the main or child documents
using compilation flags as described in \secref{sec:flags}
can be (permanently) stored in different files
for convenient compilation, viewing and distribution.
To this end, the package defines a command
to pass on compilation to a different file:

%%%%%%%%%%%%%%%%%%%%%%%%%%%%%%%%%%%%%%%%
\DescribeMacro{\childdocforward}
The command |\childdocforward| redirects processing to
another source file:
%
\begin{center}
\begin{tabular}{l}
|\input{childdoc.def}|\\
|\childdocforward[|\textit{main}|]{|\textit{dest}|}|\\
\end{tabular}
\end{center}
%
The argument \textit{dest} is the destination file
(without extension).
It should be the main file or one of the child files.
Note that further \textsf{childdoc} directives
such as |\childdocof| and |\childdocforward|
in the indicated file will be processed in this form.
The optional argument \textit{main}
passes on directly to the main file \textit{main}
while pretending to compile the child \textit{dest}.
This form behaves as if \textit{dest}
issues |\childdocof{|\textit{main}|}| right away,
and no further \textsf{childdoc} directives will be processed.

%%%%%%%%%%%%%%%%%%%%%%%%%%%%%%%%%%%%%%%%
\DescribeMacro{\...prefix}
In the alternative form |\childdocforwardprefix|,
%
\begin{center}
\begin{tabular}{l}
|\input{childdoc.def}|\\
|\childdocforwardprefix[|\textit{main}|]{|\textit{prefix}|}{|\textit{dest}|}|
\end{tabular}
\end{center}
%
the destination file is determined by a pattern
depending on the current file:
To make this work, the current file must be called
`{\textit{prefix}\hspace{0.2em}\textit{suffix}}'
with \textit{prefix} matching precisely the argument.
Processing is then passed on to the file
`{\textit{dest}\hspace{0.2em}\textit{suffix}}'.
Surely, the same effect is achieved by
directly specifying the
argument `{\textit{dest}\hspace{0.2em}\textit{suffix}}'
in the first form.
However, that requires to set up a different file
for each child. With the alternative form of the command
all these files can have exactly the same content
which simplifies setting them up and maintaining them.

For example, the following file |draft.tex|
with a compilation flag |\version| as described in \secref{sec:flags}
compiles the main document as a draft:
%
\begin{center}
\begin{tabular}{l}
|\def\version{draft}|\\
|\input{childdoc.def}|\\
|\childdocforward{|\textit{main}|}|
\end{tabular}
\end{center}
%
Likewise, the following files |final|\textit{nn}|.tex|
compile the final version of the child document
|child|\textit{nn}|.tex|:
%
\begin{center}
\begin{tabular}{l}
|\def\version{final}|\\
|\input{childdoc.def}|\\
|\childdocforwardprefix{final}{child}|
\end{tabular}
\end{center}
%

Note that when several versions of a main file and/or of each child file
are to be generated, it may be convenient to set up a |Makefile| or
shell script to automatise the process.

%%%%%%%%%%%%%%%%%%%%%%%%%%%%%%%%%%%%%%%%%%%%%%%%%%%%%%%%%%%%%%%%%%%%%%%%%%%%%%%%
\subsection{Command Line Processing}
\label{sec:commandline}

The effect of redirection files can also be achieved by invoking
the \LaTeX{} compiler with a more elaborate command line.
Most conveniently this should be done as part
of a shell script or a |Makefile|.

When using \textsf{childdoc} in the main file, the following
command lines effectively perform a redirection
(note that depending on the shell being used,
backslashes may have to be doubled: `|\|' $\to$ `|\\|'):
%
\begin{center}
|... -jobname "|\textit{target}|" |\\|"|[\textit{flags}]%
|\input{childdoc.def}\childdocforward[|\textit{main}|]{|\textit{dest}|}"|
\end{center}
%
Here \textit{target} is the name of the output file,
\textit{main} is the name of the main file
and \textit{dest} is the name of the main or child file to be processed
(all filenames without extensions).
The optional argument \textit{main} can be omitted
if \textit{main} matches \textit{dest}.
Optionally, compilation \textit{flags} can be defined via |\def| commands.
This command line makes the \TeX{} engine believe
it is compiling the file \textit{target}
whose content is specified as the latter parameter.
The provided code then forwards the processing to
\textit{main} or \textit{dest} as described in \secref{sec:forward}.

%%%%%%%%%%%%%%%%%%%%%%%%%%%%%%%%%%%%%%%%%%%%%%%%%%%%%%%%%%%%%%%%%%%%%%%%%%%%%%%%
\subsection{Include by Input}
\label{sec:input}

Including child documents by |\include| has some restrictions by design.
Most notably, the content of a child document always occupies
its own set of pages; pages cannot be shared between child documents.
Usually, this behaviour makes perfect sense
because each child document contain an essential part of the document.
However, in some situations it may be desirable to compose
a document from a collection of parts
without having mandatory page breaks between then.
For this case, the package
provides a mechanism to include parts
by |\input| which can also be processed individually.
However, by construction this mechanism
requires manual handling of the content to be output.

%%%%%%%%%%%%%%%%%%%%%%%%%%%%%%%%%%%%%%%%
\DescribeMacro{\ifchilddocmanual}
The main file should be prepared as usual, see \secref{sec:include}.
However, the document body must make a distinction
between processing of an individual part and of the main document, e.g.:
%
\begin{center}
\begin{tabular}{l}
|\ifchilddocmanual|\\
|\input{\childdocname}|\\
|\||else|\\
\textit{document body with }|\input{|\textit{part}|}|\\
|\||fi|
\end{tabular}
\end{center}
%
The conditional |\ifchilddocmanual| is true whenever
a part to be included by |\input| is being compiled,
and the name of the part is stored in |\childdocname|.

%%%%%%%%%%%%%%%%%%%%%%%%%%%%%%%%%%%%%%%%
\DescribeMacro{\childdocby}
Each part to be included by |\input| should start with:
%
\begin{center}
\begin{tabular}{l}
|\input{childdoc.def}|\\
|\childdocby{|\textit{main}|}|\\
\end{tabular}
\end{center}
%
The directive |\childdocby| is similar to |\childdocof|
described in \secref{sec:include},
but the subsequent selection of content must be done manually.
To that end, both |\ifchilddoc| and |\ifchilddocmanual|
will be true upon processing of a part,
and the name of the part is stored in |\childdocname|.
Note that |\jobname| will be set to the filename of the current part
so that each part receives an individual |.aux| file
that does not interfere with the |.aux| file(s) of the main document.
This behaviour can be altered by the alternative form
|\childdocby[*]{|\textit{main}|}| (with a non-empty optional argument)
which uses the |.aux| file of the main document
by setting |\jobname| to \textit{main}.

%%%%%%%%%%%%%%%%%%%%%%%%%%%%%%%%%%%%%%%%%%%%%%%%%%%%%%%%%%%%%%%%%%%%%%%%%%%%%%%%
\subsection{Driver Development}
\label{sec:driver}

The \textsf{childdoc} mechanism can also be use for the development
of definition files such as \LaTeX{} styles or classes.
This case differs from the above setup with multiple parts
included by |\include| in that no |\includeonly| should be invoked.
This can be achieved by starting the include file
(before |\ProvidesPackage|) with:
%
\begin{center}
\begin{tabular}{l}
|\input{childdoc.def}|\\
|\childdocforward{|\textit{main}|}|\\
\end{tabular}
\end{center}
%
or alternatively with:
%
\begin{center}
\begin{tabular}{l}
|\input{childdoc.def}|\\
|\childdocby{|\textit{main}|}|\\
\end{tabular}
\end{center}
%
Both forms have slightly different effects as described above.
The main file is prepared as usual, see \secref{sec:include}.

%%%%%%%%%%%%%%%%%%%%%%%%%%%%%%%%%%%%%%%%%%%%%%%%%%%%%%%%%%%%%%%%%%%%%%%%%%%%%%%%
\subsection{Legacy Detection}
\label{sec:detection}

The directive |\childdocmain| in the main file can detect
whether the complete document or merely a child is to be compiled
even without using the directive |\childdocof|.
This method is deprecated because it is less robust
and there is no compelling reason to use it;
it is merely provided for backward compatibility
and it may be removed in future versions.

If the detection mechanism is to be used,
it is mandatory to correctly specify
the filename of the main file as the argument of |\childdocmain|:
%
\begin{center}
\begin{tabular}{l}
|\input{childdoc.def}|\\
|\childdocmain{|\textit{main}|}|\\
\end{tabular}
\end{center}
%
If |\jobname| does not match the argument \textit{main} of |\childdocmain|,
it is assumed that |\jobname| points to the child file to be compiled.
When using |\childdocmain| with the main file specified as argument,
it suffices to start a child file
with just |\input{|\textit{main}|}|
without loading of the package and using |\childdocof|.
If instead all processing is done
with the appropriate \textsf{childdoc} directives,
the argument of \textit{main} of |\childdocmain| can be empty.

An alternative version of the command line processing described
in \secref{sec:commandline} using the detection mechanism reads:
%
\begin{center}
|... -jobname "|\textit{target}|" "|[\textit{flags}]%
[|\def\jobname{|\textit{dest}|}|]|\input{|\textit{main}|}"|
\end{center}

%%%%%%%%%%%%%%%%%%%%%%%%%%%%%%%%%%%%%%%%%%%%%%%%%%%%%%%%%%%%%%%%%%%%%%%%%%%%%%%%
\subsection{Manual Code}
\label{sec:manual}

In case one cannot be certain whether the definitions file |childdoc.def|
is installed on the target \TeX{} distribution
and one prefers not to ship it,
it is conceivable to paste a few relevant commands into the sources.

To that end, drop all statements |\input{childdoc.def}|
and perform the replacements as outlined below.
Instead of |\childdocmain{|\textit{main}|}| add the following code
to the top of the main file:
%
\begin{center}
\begin{tabular}{l}
|\||ifdefined\childdocname\endinput\||fi\newif\ifchilddoc|\\
|\edef\childdocname{\scantokens\expandafter{\jobname\noexpand}}|\\
|\def\childdocmain{|\textit{main}|}\||ifx\childdocmain\childdocname\||else|\\
|\childdoctrue\includeonly{\childdocname}\let\jobname\childdocmain\||fi|\\
\end{tabular}
\end{center}
%
Instead of |\childdocof{|\textit{main}|}| just include the main file
at the top of each child file:
%
\begin{center}
|\input{|\textit{main}|}|
\end{center}
%
A simple redirection |\childdocforward{|\textit{dest}|}| is achieved by:
%
\begin{center}
|\def\jobname{|\textit{dest}|}\input{\jobname}|
\end{center}
%
The redirection with prefix
|\childdocforwardprefix[|\textit{prefix}|]{|\textit{dest}|}|
is accomplished by:
%
\begin{center}
\begin{tabular}{l}
|{\edef\jobname{\scantokens\expandafter{\jobname\noexpand}}|\\
|\def\redirectjob |\textit{prefix}|#1~~~{\gdef\jobname{|\textit{dest}|#1}}|\\
|\expandafter\redirectjob\jobname~~~}\input{\jobname}|
\end{tabular}
\end{center}

In an alternative approach,
child documents can be compiled by a specific command line
without additional code or specific definitions:
%
\begin{center}
|... -jobname "|\textit{target}|" "|[\textit{flags}]%
|\includeonly{|\textit{dest}|}\input{|\textit{main}|}"|
\end{center}
%

%%%%%%%%%%%%%%%%%%%%%%%%%%%%%%%%%%%%%%%%%%%%%%%%%%%%%%%%%%%%%%%%%%%%%%%%%%%%%%%%
%%%%%%%%%%%%%%%%%%%%%%%%%%%%%%%%%%%%%%%%%%%%%%%%%%%%%%%%%%%%%%%%%%%%%%%%%%%%%%%%
\section{Information}

%%%%%%%%%%%%%%%%%%%%%%%%%%%%%%%%%%%%%%%%%%%%%%%%%%%%%%%%%%%%%%%%%%%%%%%%%%%%%%%%
\subsection{Copyright}

Copyright \copyright{} 2017--2018 Niklas Beisert

This work may be distributed and/or modified under the
conditions of the \LaTeX{} Project Public License, either version 1.3
of this license or (at your option) any later version.
The latest version of this license is in
  \url{http://www.latex-project.org/lppl.txt}
and version 1.3 or later is part of all distributions of \LaTeX{}
version 2005/12/01 or later.

This work has the LPPL maintenance status `maintained'.

The Current Maintainer of this work is Niklas Beisert.

This work consists of the files |README.txt|, |childdoc.ins| and |childdoc.dtx|
as well as the derived files |childdoc.def|, |cdocsamp.tex|
with |cdocsch1.tex|, |cdocsch2.tex|, |cdocspt3.tex|, |cdocspt4.tex|,
|cdocsdrf.tex|, |cdocsfn1.tex|, |cdocsfn2.tex|
as well as |childdoc.pdf|.

%%%%%%%%%%%%%%%%%%%%%%%%%%%%%%%%%%%%%%%%%%%%%%%%%%%%%%%%%%%%%%%%%%%%%%%%%%%%%%%%
\subsection{Files and Installation}

The package consists of the files:
%
\begin{center}
\begin{tabular}{ll}
    |README.txt|   & readme file \\
    |childdoc.ins| & installation file \\
    |childdoc.dtx| & source file \\
    |childdoc.def| & definition file \\
    |cdocsamp.tex| & sample main file \\
    |cdocsch1.tex| & sample include file \\
    |cdocsch2.tex| & sample include file \\
    |cdocspt3.tex| & sample part file \\
    |cdocspt4.tex| & sample part file \\
    |cdocsdrf.tex| & sample redirection file \\
    |cdocsfn1.tex| & sample redirection file \\
    |cdocsfn2.tex| & sample redirection file \\
    |childdoc.pdf| & manual
\end{tabular}
\end{center}
%
The distribution consists of the files
|README.txt|, |childdoc.ins| and |childdoc.dtx|.
%
\begin{itemize}
\item
Run (pdf)\LaTeX{} on |childdoc.dtx|
to compile the manual |childdoc.pdf| (this file).
\item
Run \LaTeX{} on |childdoc.ins| to create the definitions file |childdoc.def|
and the sample |cdocsamp.tex| with include files
|cdocsch1.tex|, |cdocsch2.tex|, |cdocspt3.tex|, |cdocspt4.tex|,
|cdocsdrf.tex|, |cdocsfn1.tex|, |cdocsfn2.tex|.
Then copy the file |childdoc.def| to an appropriate directory of your \LaTeX{}
distribution, e.g.\ \textit{texmf-root}|/tex/latex/childdoc|.
\end{itemize}

%%%%%%%%%%%%%%%%%%%%%%%%%%%%%%%%%%%%%%%%%%%%%%%%%%%%%%%%%%%%%%%%%%%%%%%%%%%%%%%%
\subsection{Related CTAN Packages}

There are several other packages which offer a similar functionality:
%
\begin{itemize}
\item
The packages
\href{http://ctan.org/pkg/docmute}{\textsf{docmute}},
\href{http://ctan.org/pkg/includex}{\textsf{includex}} and
\href{http://ctan.org/pkg/standalone}{\textsf{standalone}}
provide commands to include only the document body of
a child file thus allowing both files to be compiled individually.
\item
The packages \href{http://ctan.org/pkg/subdocs}{\textsf{subdocs}}
and \href{http://ctan.org/pkg/subfiles}{\textsf{subfiles}}
provide structures in which the main and child documents can be
encapsulated and allowing them to be compiled individually.
The inclusion mechanism is different from the conventional |\include|.
\item
The package \href{http://ctan.org/pkg/combine}{\textsf{combine}}
is an elaborate solution to combine several documents into one.
\end{itemize}
%
See also the CTAN topic \href{http://ctan.org/topic/subdocs}{\textsf{subdocs}}
for further related packages.
The present package differs from the above solutions in that
a document structure constructed with the conventional |\include| mechanism
just needs two extra commands at the top of every file
such that all constituent files can be compiled individually.

%%%%%%%%%%%%%%%%%%%%%%%%%%%%%%%%%%%%%%%%%%%%%%%%%%%%%%%%%%%%%%%%%%%%%%%%%%%%%%%%
%\subsection{Feature Suggestions}
%
%The following is a list of features which may be useful for future
%versions of this package:
%%
%\begin{itemize}
%\item
%\ldots
%\end{itemize}

%%%%%%%%%%%%%%%%%%%%%%%%%%%%%%%%%%%%%%%%%%%%%%%%%%%%%%%%%%%%%%%%%%%%%%%%%%%%%%%%
\subsection{Revision History}

%%%%%%%%%%%%%%%%%%%%%%%%%%%%%%%%%%%%%%%%
\paragraph{v2.0:} 2018/12/30

\begin{itemize}
\item
immediate forward processing
\item
added |\childdocby| mechanism
\item
manual restructured
\end{itemize}

%%%%%%%%%%%%%%%%%%%%%%%%%%%%%%%%%%%%%%%%
\paragraph{v1.6:} 2018/01/17

\begin{itemize}
\item
application for development of include files
\item
corrections to manual
\end{itemize}

%%%%%%%%%%%%%%%%%%%%%%%%%%%%%%%%%%%%%%%%
\paragraph{v1.5:} 2017/05/21

\begin{itemize}
\item
more complete structuring introduced
\item
|\childdocof| introduced
\item
|\childdoc| renamed to |\childdocmain|
\item
|\childredirect| renamed to |\childdocforward| and |\childdocforwardprefix|
and functionality expanded
\end{itemize}

%%%%%%%%%%%%%%%%%%%%%%%%%%%%%%%%%%%%%%%%
\paragraph{v1.0:} 2017/04/27

\begin{itemize}
\item
manual and install package
\item
first version published on CTAN
\end{itemize}

%%%%%%%%%%%%%%%%%%%%%%%%%%%%%%%%%%%%%%%%
\paragraph{v0.6:} 2017/04/26

\begin{itemize}
\item
redirection mechanism added
\end{itemize}

%%%%%%%%%%%%%%%%%%%%%%%%%%%%%%%%%%%%%%%%
\paragraph{v0.5:} 2017/04/26

\begin{itemize}
\item
functionality in definition file
\end{itemize}


%%%%%%%%%%%%%%%%%%%%%%%%%%%%%%%%%%%%%%%%%%%%%%%%%%%%%%%%%%%%%%%%%%%%%%%%%%%%%%%%
%%%%%%%%%%%%%%%%%%%%%%%%%%%%%%%%%%%%%%%%%%%%%%%%%%%%%%%%%%%%%%%%%%%%%%%%%%%%%%%%
%%%%%%%%%%%%%%%%%%%%%%%%%%%%%%%%%%%%%%%%%%%%%%%%%%%%%%%%%%%%%%%%%%%%%%%%%%%%%%%%
\appendix

\settowidth\MacroIndent{\rmfamily\scriptsize 000\ }

 \DocInput{childdoc.dtx}

\end{document}
%</driver>
% \fi
%
% %%%%%%%%%%%%%%%%%%%%%%%%%%%%%%%%%%%%%%%%%%%%%%%%%%%%%%%%%%%%%%%%%%%%%%%%%%%%%%
% %%%%%%%%%%%%%%%%%%%%%%%%%%%%%%%%%%%%%%%%%%%%%%%%%%%%%%%%%%%%%%%%%%%%%%%%%%%%%%
% \section{Sample}
%\iffalse
%<*samplemain>
%\fi
%
% The following presents a sample document
% with two chapters, two parts, a title page,
% a compile flag as well as three forwarding files to set the flag.
% It consists of eight |.tex| files:
% \begin{center}
% \begin{tabular}{ll}
% |cdocsamp.tex|&main file\\
% |cdocsch1.tex|&include file for chapter 1\\
% |cdocsch2.tex|&include file for chapter 2\\
% |cdocspt3.tex|&include file for part 3\\
% |cdocspt4.tex|&include file for part 4\\
% |cdocsdrf.tex|&forwarding file for main file in draft mode\\
% |cdocsfi1.tex|&forwarding file for final version of chapter 1\\
% |cdocsfi2.tex|&forwarding file for final version of chapter 2\\
% \end{tabular}
% \end{center}
% Each of the eight files can be compiled directly by the \LaTeX{} compiler.
%
% %%%%%%%%%%%%%%%%%%%%%%%%%%%%%%%%%%%%%%
% \paragraph{Main File.}
%
% The main file is called |cdocsamp.tex|.
%
% Load the \textsf{childdoc} definitions and
% declare the filename for the main document:
%    \begin{macrocode}
\input{childdoc.def}
\childdocmain{}
%    \end{macrocode}

% Optional override for |\version| flag:
%    \begin{macrocode}
%%\ifchilddoc\else\providecommand{\version}{draft}\fi
%    \end{macrocode}

% Define the default values for the |\version| flag
% (|final| for the main file and |draft| for childs):
%    \begin{macrocode}
\ifchilddoc
\providecommand{\version}{draft}
\else
\providecommand{\version}{final}
\fi
%    \end{macrocode}

% Load the standard document class:
%    \begin{macrocode}
\documentclass[12pt]{article}
%    \end{macrocode}

% Start the document body:
%    \begin{macrocode}
\begin{document}
%    \end{macrocode}

% Declare a title page.
% Print title, part of document being processed and version flag:
%    \begin{macrocode}
\addtocounter{page}{-1}
\begin{center}
{\LARGE\bfseries{}childdoc example\par}
\vspace{1cm}
\ifchilddoc
\ifchilddocmanual part\else chapter\fi:
`\childdocname' of `\childdocjob'\par
\else
main document: `\childdocjob'\par
\fi
version: \version\par
\end{center}
\newpage
%    \end{macrocode}

% Manually include selected file,
% otherwise process as usual:
%    \begin{macrocode}
\ifchilddocmanual
\section*{part `\childdocname'}
\input{\childdocname}
\else
%    \end{macrocode}

% Include the two chapters:
%    \begin{macrocode}
\include{cdocsch1}
\include{cdocsch2}
%    \end{macrocode}

% Include the two parts unless only chapters should be displayed:
%    \begin{macrocode}
\ifchilddoc\else
\section{part three}
\input{cdocspt3}
\section{part four}
\input{cdocspt4}
\fi
%    \end{macrocode}

% Process as usual until here:
%    \begin{macrocode}
\fi
%    \end{macrocode}

% End of document body:
%    \begin{macrocode}
\end{document}
%    \end{macrocode}
%\iffalse
%</samplemain>
%\fi
%
% %%%%%%%%%%%%%%%%%%%%%%%%%%%%%%%%%%%%%%
% \paragraph{Chapter Include Files.}
%
% The include files are called |cdocsch1.tex| and |cdocsch2.tex|.
%
%\iffalse
%<*samplechap1|samplechap2>
%\fi

% Optional override for |\version| flag:
%    \begin{macrocode}
%%\providecommand{\version}{final}
%    \end{macrocode}

% Include the main document:
%    \begin{macrocode}
\input{childdoc.def}
\childdocof{cdocsamp}
%    \end{macrocode}

%\iffalse
%</samplechap1|samplechap2>
%\fi
%
%\iffalse
%<*samplechap1>
%\fi
% Some text for chapter 1:
%    \begin{macrocode}
\section{one}
some text in chapter one
%    \end{macrocode}

%\iffalse
%</samplechap1>
%\fi
% Some text for chapter 2:
%\iffalse
%<*samplechap2>
%\fi
%    \begin{macrocode}
\section{two}
more text in chapter two
%    \end{macrocode}

%\iffalse
%</samplechap2>
%\fi
%
% %%%%%%%%%%%%%%%%%%%%%%%%%%%%%%%%%%%%%%
% \paragraph{Part Include Files.}
%
% The include files are called |cdocspt3.tex| and |cdocspt4.tex|.
%
%\iffalse
%<*samplepart3|samplepart4>
%\fi

% Optional override for |\version| flag:
%    \begin{macrocode}
%%\providecommand{\version}{final}
%    \end{macrocode}

% Include the main document:
%    \begin{macrocode}
\input{childdoc.def}
\childdocby{cdocsamp}
%    \end{macrocode}

%\iffalse
%</samplepart3|samplepart4>
%\fi
%
%\iffalse
%<*samplepart3>
%\fi
% Some text for part 3:
%    \begin{macrocode}
some text in part three
%    \end{macrocode}

%\iffalse
%</samplepart3>
%\fi
% Some text for part 4:
%\iffalse
%<*samplepart4>
%\fi
%    \begin{macrocode}
more text in part four
%    \end{macrocode}

%\iffalse
%</samplepart4>
%\fi
%
% %%%%%%%%%%%%%%%%%%%%%%%%%%%%%%%%%%%%%%
% \paragraph{Forwarding for a Complete Draft.}
%
% The following forwarding file |cdocsdrf.tex|
% compiles the main document in draft mode:
%\iffalse
%<*sampledraft>
%\fi
%    \begin{macrocode}
\def\version{draft}
\input{childdoc.def}
\childdocforward{cdocsamp}
%    \end{macrocode}

%\iffalse
%</sampledraft>
%\fi
%
% %%%%%%%%%%%%%%%%%%%%%%%%%%%%%%%%%%%%%%
% \paragraph{Forwarding for Final Version of the Chapters.}
%
% The following forwarding files |cdocsfn1.tex| and |cdocsfn2.tex|
% (with identical content)
% compile the final versions of the child documents
% |cdocsch1.tex| and |cdocsch2.tex|, respectively:
%\iffalse
%<*samplefinal>
%\fi
%    \begin{macrocode}
\def\version{final}
\input{childdoc.def}
\childdocforwardprefix[cdocsamp]{cdocsfn}{cdocsch}
%    \end{macrocode}

%\iffalse
%</samplefinal>
%\fi
%
% %%%%%%%%%%%%%%%%%%%%%%%%%%%%%%%%%%%%%%
% \paragraph{Command Line Processing.}
%
% The following three command lines generate the output files
% |cdocscld|, |cdocscl1| and |cdocscl2|
% which should be identical to
% |cdocsdrf|, |cdocsch1| and |cdocsfn2|, respectively:
% \begin{center}
% \begin{tabular}{l}
% |latex -jobname cdocscld \|\\
% |  "\def\version{draft}\input{childdoc.def}\childdocforward{cdocsamp}"|\\
% |latex -jobname cdocscl1 \|\\
% |  "\input{childdoc.def}\childdocforward[cdocsamp]{cdocsch1}"|\\
% |latex -jobname cdocscl2 \|\\
% |  "\def\version{final}\input{childdoc.def}\childdocforward{cdocsch2}"|
% \end{tabular}
% \end{center}
% Note that the trailing backslash on each first line
% merely continues the input to the second line
% (for convenient cut ant paste).
% Furthermore, the command |latex| can be replaced by any
% of its alternative versions such as |pdflatex|.
%
% %%%%%%%%%%%%%%%%%%%%%%%%%%%%%%%%%%%%%%%%%%%%%%%%%%%%%%%%%%%%%%%%%%%%%%%%%%%%%%
% %%%%%%%%%%%%%%%%%%%%%%%%%%%%%%%%%%%%%%%%%%%%%%%%%%%%%%%%%%%%%%%%%%%%%%%%%%%%%%
% \section{Implementation}
%\iffalse
%<*package>
%\fi
%
% This section describes the definitions file |childdoc.def|.

% The definitions cannot be loaded using |\usepackage| or |\RequirePackage|
% which has a mechanism to prevent loading a style file more than once.
% When loading the definitions by means of |\input|
% multiple instances have to be prevented manually:
%\iffalse
%This code needs to be before the `\ProvidesFile' directive
%which is defined at the beginning of this file.
%Therefore it is also placed there and commented out here.
%</package>
%<*discard>
%\fi
%    \begin{macrocode}
\ifdefined\childdocmain\endinput\fi
%    \end{macrocode}
%\iffalse
%</discard>
%<*package>
%\fi
%
% \macro{\ifchilddoc}
% \macro{\ifchilddocmanual}
% The conditional |\ifchilddoc| tells whether a
% child (true) or main (false) document is being compiled.
% The conditional |\ifchilddocmanual| tells whether
% the |\includeonly| mechanism is used (false) or
% the selection of child files must be performed manually (true).
% The definitions initialise to false:
%    \begin{macrocode}
\newif\ifchilddoc
\newif\ifchilddocmanual
%    \end{macrocode}

% \macro{\childdocname}
% \macro{\childdocjob}
% The macro |\childdocname| stores the name of the main document
% to be compiled. The macro |\childdocjob| stores the name of
% the document on which the \LaTeX{} compiler was originally invoked.
% The content of |\jobname| cannot be compared
% to filenames specified in the source due to different catcodes.
% The following code rescans |\jobname|, stores the result
% in |\childdocname| and saves a copy in |\childdocjob|:
%    \begin{macrocode}
\edef\childdocname{\scantokens\expandafter{\jobname\noexpand}}
\let\childdocjob\childdocname
%    \end{macrocode}

% \macro{\childdocdisable}
% The macro |\childdocdisable| prevents the main file
% from being processed more than once.
% At this stage, the main document command |\childdocmain|
% is assumed to be called once again where it should do nothing.
% Any subsequent call to it should prevent
% a secondary processing of the main document
% It overwrites the forwarding commands
% |\childdocof| and |\childdocforward|
% with empty macros to prevent further inclusions of the main document:
%    \begin{macrocode}
\newcommand{\childdocdisable}
{
  \renewcommand{\childdocmain}[1]{\renewcommand{\childdocmain}[1]{\endinput}}
  \renewcommand{\childdocof}[1]{}
  \renewcommand{\childdocby}[2][]{}
  \renewcommand{\childdocforward}[2][]{}
  \renewcommand{\childdocdisable}{}
}
%    \end{macrocode}

% \macro{\childdocmain}
% The macro |\childdocmain| is to be called at the top of the main file
% with nothing or the main filename (without extension) as argument.
% First, it breaks loops.
% If the argument is not empty and does not match |\childdocname|
% (which is set by the first inclusion of |childdoc.def|),
% |\ifchilddoc| is set to true, |\includeonly| is applied to the child file
% and |\jobname| is set to the main file
% (for proper handling of |.aux| files):
%    \begin{macrocode}
\newcommand{\childdocmain}[1]
{
  \childdocdisable\childdocmain{}
  \if?#1?\else
    \begingroup
      \def\childdoctmp{#1}
      \ifx\childdoctmp\childdocname
        \def\childdoctmp{}
      \else
        \def\childdoctmp
        {
          \childdoctrue
          \includeonly{\childdocname}
          \def\childdocjob{#1}
          \def\jobname{#1}
        }
      \fi
      \expandafter
    \endgroup
    \childdoctmp
  \fi
}
%    \end{macrocode}

% \macro{\childdocof}
% The command |\childdocof| redirects
% compilation to the main file |#1|.
%    \begin{macrocode}
\newcommand{\childdocof}[1]
{
  \childdocdisable
  \childdoctrue
  \includeonly{\childdocname}
  \def\jobname{#1}
  \def\childdocjob{#1}
  \input{#1}
}
%    \end{macrocode}

% \macro{\childdocby}
% The command |\childdocby| ....
%    \begin{macrocode}
\newcommand{\childdocby}[2][]
{
  \childdocdisable
  \childdoctrue
  \childdocmanualtrue
  \if?#1?\else
    \def\jobname{#2}
  \fi
  \def\childdocjob{#2}
  \input{#2}
  \endinput
}
%    \end{macrocode}

% \macro{\childdocforward}
% The command |\childdocforward| redirects
% compilation to the main file or
% (if the optional argument is given) a child file.
% Parameters are set as if the main file
% or a child file starting with |\childdocof| was compiled.
% Then compilation is handed over to the main file:
%    \begin{macrocode}
\newcommand{\childdocforward}[2][]
{
  \begingroup
    \if?#1?
      \def\childdoctmp
      {
        \def\childdocname{#2}
        \def\childdocjob{#2}
        \def\jobname{#2}
        \input{#2}
        \endinput
      }
    \else
      \def\childdoctmp
      {
        \childdocdisable
        \def\childdocname{#2}
        \childdoctrue
        \includeonly{#2}
        \def\childdocjob{#1}
        \def\jobname{#1}
        \input{#1}
        \endinput
      }
    \fi
    \expandafter
  \endgroup
  \childdoctmp
}
%    \end{macrocode}

% \macro{\childdocforwardprefix}
% The command |\childdocforwardprefix| redirects
% compilation to the main or a child file by means of a pattern.
% The prefix |#1| in the current filename is replaced by |#2|
% and the suffix of the current filename is kept
% (it is assumed that the filename does not contain the substring `|~~~|'
% which is used as a delimiter).
% Compilation is handed over to the new file by |\childdocforward|:
%    \begin{macrocode}
\newcommand{\childdocforwardprefix}[3][]
{
  \begingroup
    \def\childdocextract #2##1~~~{\def\childdoctmp{\childdocforward[#1]{#3##1}}}
    \expandafter\childdocextract\childdocname~~~
    \expandafter
  \endgroup
  \childdoctmp
}
%    \end{macrocode}

% \macro{\childdoc}
% The deprecated macro |\childdoc| is a legacy version of |\childdocmain|:
%    \begin{macrocode}
\newcommand{\childdoc}{\childdocmain}
%    \end{macrocode}

% \macro{\childdocredirect}
% The deprecated macro |\childdocredirect| is a legacy version
% of |\childdocforward| and |\childdocforwardprefix|:
%    \begin{macrocode}
\newcommand{\childdocredirect}[2][]
{
  \begingroup
    \if?#1?
      \def\childdoctmp{\childdocforward{#2}}
    \else
      \def\childdoctmp{\childdocforwardprefix{#1}{#2}}
    \fi
    \expandafter
  \endgroup
  \childdoctmp
}
%    \end{macrocode}

%\iffalse
%</package>
%\fi
%
\endinput
|\\
|\childdocby{|\textit{main}|}|\\
\end{tabular}
\end{center}
%
The directive |\childdocby| is similar to |\childdocof|
described in \secref{sec:include},
but the subsequent selection of content must be done manually.
To that end, both |\ifchilddoc| and |\ifchilddocmanual|
will be true upon processing of a part,
and the name of the part is stored in |\childdocname|.
Note that |\jobname| will be set to the filename of the current part
so that each part receives an individual |.aux| file
that does not interfere with the |.aux| file(s) of the main document.
This behaviour can be altered by the alternative form
|\childdocby[*]{|\textit{main}|}| (with a non-empty optional argument)
which uses the |.aux| file of the main document
by setting |\jobname| to \textit{main}.

%%%%%%%%%%%%%%%%%%%%%%%%%%%%%%%%%%%%%%%%%%%%%%%%%%%%%%%%%%%%%%%%%%%%%%%%%%%%%%%%
\subsection{Driver Development}
\label{sec:driver}

The \textsf{childdoc} mechanism can also be use for the development
of definition files such as \LaTeX{} styles or classes.
This case differs from the above setup with multiple parts
included by |\include| in that no |\includeonly| should be invoked.
This can be achieved by starting the include file
(before |\ProvidesPackage|) with:
%
\begin{center}
\begin{tabular}{l}
|% \iffalse
%
% childdoc.dtx Copyright (C) 2017-2018 Niklas Beisert
%
% This work may be distributed and/or modified under the
% conditions of the LaTeX Project Public License, either version 1.3
% of this license or (at your option) any later version.
% The latest version of this license is in
%   http://www.latex-project.org/lppl.txt
% and version 1.3 or later is part of all distributions of LaTeX
% version 2005/12/01 or later.
%
% This work has the LPPL maintenance status `maintained'.
%
% The Current Maintainer of this work is Niklas Beisert.
%
% This work consists of the files childdoc.dtx and childdoc.ins
% and the derived files childdoc.def and cdocsamp.tex with
% cdocsch1.tex, cdocsch2.tex, cdocsdrf.tex, cdocsfn1.tex, cdocsfn2.tex.
%
%<package>\ifdefined\childdocmain\endinput\fi
%<package>\ProvidesFile{childdoc.def}[2018/12/30 v2.0 child document driver]
%<samplemain>\ProvidesFile{cdocsamp.tex}[2018/12/30 v2.0 sample for childdoc]
%<*driver>
%\ProvidesFile{childdoc.drv}[2018/12/30 v2.0 childdoc reference manual file]
\PassOptionsToClass{10pt,a4paper}{article}
\documentclass{ltxdoc}

\usepackage[margin=35mm]{geometry}
\usepackage{hyperref}
\usepackage{hyperxmp}
\usepackage[usenames]{color}

\hypersetup{colorlinks=true}
\hypersetup{pdfstartview=FitH}
\hypersetup{pdfpagemode=UseNone}
\hypersetup{pdfsource={}}
\hypersetup{pdflang={en-UK}}
\hypersetup{pdfcopyright={Copyright 2017-2018 Niklas Beisert.
  This work may be distributed and/or modified under the
  conditions of the LaTeX Project Public License, either version 1.3
  of this license or (at your option) any later version.}}
\hypersetup{pdflicenseurl={http://www.latex-project.org/lppl.txt}}
\hypersetup{pdfcontactaddress={ETH Zurich, ITP, HIT K,
  Wolfgang-Pauli-Strasse 27}}
\hypersetup{pdfcontactpostcode={8093}}
\hypersetup{pdfcontactcity={Zurich}}
\hypersetup{pdfcontactcountry={Switzerland}}
\hypersetup{pdfcontactemail={nbeisert@itp.phys.ethz.ch}}
\hypersetup{pdfcontacturl={http://people.phys.ethz.ch/\xmptilde nbeisert/}}

\newcommand{\secref}[1]{\hyperref[#1]{section \ref*{#1}}}

\parskip1ex
\parindent0pt
\let\olditemize\itemize
\def\itemize{\olditemize\parskip0pt}

\begin{document}

\title{The \textsf{childdoc} Package}
\hypersetup{pdftitle={The childdoc Package}}
\author{Niklas Beisert\\[2ex]
  Institut f\"ur Theoretische Physik\\
  Eidgen\"ossische Technische Hochschule Z\"urich\\
  Wolfgang-Pauli-Strasse 27, 8093 Z\"urich, Switzerland\\[1ex]
  \href{mailto:nbeisert@itp.phys.ethz.ch}
  {\texttt{nbeisert@itp.phys.ethz.ch}}}
\hypersetup{pdfauthor={Niklas Beisert}}
\hypersetup{pdfsubject={Manual for the LaTeX2e Package childdoc}}
\date{30 December 2018, \textsf{v2.0}}
\maketitle

\begin{abstract}\noindent
\textsf{childdoc} is a \LaTeXe{} package
that enables the direct compilation
of document sections included by |\include|
to individual files.
\end{abstract}

\begingroup
\parskip0ex
\tableofcontents
\endgroup

%%%%%%%%%%%%%%%%%%%%%%%%%%%%%%%%%%%%%%%%%%%%%%%%%%%%%%%%%%%%%%%%%%%%%%%%%%%%%%%%
%%%%%%%%%%%%%%%%%%%%%%%%%%%%%%%%%%%%%%%%%%%%%%%%%%%%%%%%%%%%%%%%%%%%%%%%%%%%%%%%
\section{Introduction}

\LaTeX{} provides a mechanism to structure a large document (such as a book)
into a main file and several child files (containing the chapters)
using the |\include| command.
This mechanism is beneficial for documents
which span hundreds of pages in order to
make the source file(s) more manageable.
Moreover, compilation can be restricted to
selected child files by means of the |\includeonly| command.
The latter feature can be used to reduce the compilation time while editing
(this was significantly more useful in the earlier days of \LaTeX{})
or to generate a smaller document which is easier to navigate.
Another application of |\includeonly| is to generate
documents consisting of selected parts of the complete document.

However, there are a few drawbacks of the plain |\include| mechanism:
\begin{itemize}
\item
The child files cannot be compiled on their own,
they can only be compiled via the main file.
A naive editing environment
(such as a text editor with an option
to have the current file processed by \LaTeX)
may require one to switch to the main file before compiling;
attempting to compile the child file produces errors.
\item
The main file must be modified (each time)
to adjust the |\includeonly| command
to the present needs. This easily leaves the main file in a messy state.
\item
The generated document will always carry the filename
of the main document. This is inconvenient if
several child files are to be compiled and
to be kept for distribution.
\end{itemize}

The present package provides a simple interface
to make child files individually compilable by \LaTeX{}.
Compiling a child file then has the same effect as compiling
the main file with an |\includeonly| command
to select the appropriate child.
Moreover the generated document will carry the name of the child
rather than the main file.
This resolves all three above issues.

This feature is meant to make the editing of books,
thesis documents and lecture notes somewhat more convenient.
However, the package can also be used efficiently for
composing a series of documents (such as exercise sheets)
which are typically distributed individually.
It then assists the author in generating the individual documents
(potentially in different versions)
as well as a document containing the collected series.
Another application is in developing style files
or other kinds of included material
where compilation of the style file could redirect
to a sample or test file.

%%%%%%%%%%%%%%%%%%%%%%%%%%%%%%%%%%%%%%%%%%%%%%%%%%%%%%%%%%%%%%%%%%%%%%%%%%%%%%%%
%%%%%%%%%%%%%%%%%%%%%%%%%%%%%%%%%%%%%%%%%%%%%%%%%%%%%%%%%%%%%%%%%%%%%%%%%%%%%%%%
\section{Usage}

First of all, the package \textsf{childdoc} is \emph{not} a standard
\LaTeXe{} |.sty| style file! Therefore it needs to be invoked in
a non-standard way.

%%%%%%%%%%%%%%%%%%%%%%%%%%%%%%%%%%%%%%%%%%%%%%%%%%%%%%%%%%%%%%%%%%%%%%%%%%%%%%%%
\subsection{Included Files}
\label{sec:include}

%%%%%%%%%%%%%%%%%%%%%%%%%%%%%%%%%%%%%%%%
\DescribeMacro{\childdocmain}
To use the package, add the commands
\begin{center}
\begin{tabular}{l}
|\input{childdoc.def}|\\
|\childdocmain{}|\\
\end{tabular}
\end{center}
at the very top of the main \LaTeX{} file,
in particular \emph{before} the |\documentclass| statement!
The argument of |\childdocmain| should be left empty
(but it must be present).

%%%%%%%%%%%%%%%%%%%%%%%%%%%%%%%%%%%%%%%%
\DescribeMacro{\childdocof}
Furthermore, add the commands
\begin{center}
\begin{tabular}{l}
|\input{childdoc.def}|\\
|\childdocof{|\textit{main}|}|\\
\end{tabular}
\end{center}
at the top of every child file \textit{child}
which is included by |\include{|\textit{child}|}|
from within the main file
(or at least for those files to be compiled individually).
The argument \textit{main} must be the filename of the main file.

There are a couple of
considerations in setting up the main and child documents:

%%%%%%%%%%%%%%%%%%%%%%%%%%%%%%%%%%%%%%%%
\paragraph{Restrictions.}

Please note the following restrictions:
\begin{itemize}
\item
|\childdocmain| must be called with one argument \textit{main}
to ensure compatibility with earlier version of the package.
It must either be empty (|\childdocmain{}|)
or precisely match the filename of the main file in which it is specified.
See \secref{sec:detection} for further information.
\item
The filename \textit{main} must be specified without the |.tex| extension.
\item
The filename \textit{main} is case sensitive
(even in case-insensitive file systems)
due to internal string comparison.
\item
The argument \textit{main} should be fully expanded, it cannot be a macro.
\item
Subdirectories and special characters should be avoided in filenames.
\item
The command |\childdocmain{|\textit{main}|}| must be followed by a whitespace.
It should not be followed immediately by another command
or by a comment mark `|%|'.
This is because the \TeX{} parser reads the token immediately following
the argument of |\childdocmain| and puts it
at the beginning of every child section;
however, a white\-space is ignored.
\end{itemize}

%%%%%%%%%%%%%%%%%%%%%%%%%%%%%%%%%%%%%%%%
\paragraph{Content of Main File.}

It is advisable to place all content in the child files included by |\include|.
Any output contained in the main file will appear in all child documents
unless suppressed manually;
it cannot be suppressed automatically by the |\includeonly| directive
and thus should normally be avoided.
A method to include some content in the main file
by means of conditional processing is described in \secref{sec:conditional}.

%%%%%%%%%%%%%%%%%%%%%%%%%%%%%%%%%%%%%%%%
\paragraph{Page Numbering.}

When only a part of the document is compiled,
the appropriate numbering of pages
(as well as other status parameters)
is determined from the |.aux| files.
The latter contain information from previous passes.
However this information needs to propagate through
all intermediate child documents.
Therefore the page numbering in child documents may well
be inconsistent until the complete document is compiled at least once.

A useful (if unconventional) way to always ensure a consistent
page numbering is to restart the numbering in each child document
and denote the pages by `\textit{child}|.|\textit{page}'
where \textit{child} represents the chapter/section number of the child file.
This can be achieved by the command
|\numberwithin{page}{|\textit{child}|}|
of the \textsf{amsmath} package
where \textit{child} can be |chapter| or |section|
depending on the chosen structuring.
Alternatively, one can modify the macro |\thepage| appropriately
and reset the counter |page| at the start of each child file.

%%%%%%%%%%%%%%%%%%%%%%%%%%%%%%%%%%%%%%%%%%%%%%%%%%%%%%%%%%%%%%%%%%%%%%%%%%%%%%%%
\subsection{Conditional Processing}
\label{sec:conditional}

The package provides a mechanism to compile different versions
of a document. To customise the versions further some conditional processing
can come in handy to distinguish which version is being compiled.
The package provides two macros to describe the compilation context:

%%%%%%%%%%%%%%%%%%%%%%%%%%%%%%%%%%%%%%%%
\DescribeMacro{\ifchilddoc}
The conditional |\ifchilddoc| distinguishes between the compilation of
child documents and the main document:
%
\begin{center}
|\ifchilddoc |\textit{child-code}| |[|\||else |\textit{main-code}]| \||fi|
\end{center}

%%%%%%%%%%%%%%%%%%%%%%%%%%%%%%%%%%%%%%%%
\DescribeMacro{\childdocname}
\DescribeMacro{\childdocjob}
The macro |\childdocname| contains the filename (without extension)
of the main or child file being processed.
Note that |\childdocjob| will always contain the name of the main file.

%%%%%%%%%%%%%%%%%%%%%%%%%%%%%%%%%%%%%%%%
\paragraph{Title Page.}

Conditional processing can be used to include a title or banner page
in the main document when proper precautions are taken.
Importantly, the code in the main file should ensure that the page counter
(as well as other status parameters which are stored in the |.aux| files)
takes the same value after the conditional processing.
Otherwise the page numbers may take divergent values
depending on which part is compiled.

For example, a title page could be declared by:
%
\begin{center}
\begin{tabular}{l}
|\ifchilddoc\||else|\\
|\addtocounter{page}{-1}|\\
\textit{code for title page}\\
|\newpage|\\
|\||fi|
\end{tabular}
\end{center}
%
A banner page for the child documents can be generated by:
%
\begin{center}
\begin{tabular}{l}
|\ifchilddoc|\\
|\addtocounter{page}{-1}|\\
\textit{code for banner page}\\
|\newpage|\\
|\||fi|
\end{tabular}
\end{center}
%
Here one could write a message such as:
\begin{center}
|This is the part \childdocname{} of \childdocjob{}.|
\end{center}

%%%%%%%%%%%%%%%%%%%%%%%%%%%%%%%%%%%%%%%%%%%%%%%%%%%%%%%%%%%%%%%%%%%%%%%%%%%%%%%%
\subsection{Flags}
\label{sec:flags}

The package makes it easy to generate different versions
of the main or child documents.
To this end compilation flags can be defined
and assigned different default values.
They will be particularly useful in conjunction
with the forwarding mechanism described in \secref{sec:forward}.

For example, it may be useful to have a flag |\version|
which can be set to |draft| or |final|.
The document source will contain some conditional code
depending on the value of |\version|.
Suppose further, the flag should default to |final| for the main file
and to |draft| for child files
which is a natural assignment for editing the document.
This is achieved by placing the following code
in the preamble of the main document
(below the |\childdocmain| directive):
%
\begin{center}
\begin{tabular}{l}
|\ifchilddoc|\\
|\providecommand{\version}{draft}|\\
|\||else|\\
|\providecommand{\version}{final}|\\
|\||fi|
\end{tabular}
\end{center}
%
The definition by |\providecommand| makes sure
that previous definitions are not overwritten.
Further statements |\providecommand{\version}{...}|
can thus be added before the above code to override it.

For the main file, one might add a line
(between |\childdocmain| and the above block)
%
\begin{center}
|%\ifchilddoc\||else\providecommand{\version}{draft}\||fi|
\end{center}
%
which can be uncommented to produce a draft version.
Likewise one can add a line to the very top of a child file
(above the |\childdocof{|\textit{main}|}| directive)
%
\begin{center}
|%\providecommand{\version}{final}|
\end{center}
%
which can be uncommented to produce the final version of this child document.

%%%%%%%%%%%%%%%%%%%%%%%%%%%%%%%%%%%%%%%%%%%%%%%%%%%%%%%%%%%%%%%%%%%%%%%%%%%%%%%%
\subsection{Forwarding}
\label{sec:forward}

Different versions of the main or child documents
using compilation flags as described in \secref{sec:flags}
can be (permanently) stored in different files
for convenient compilation, viewing and distribution.
To this end, the package defines a command
to pass on compilation to a different file:

%%%%%%%%%%%%%%%%%%%%%%%%%%%%%%%%%%%%%%%%
\DescribeMacro{\childdocforward}
The command |\childdocforward| redirects processing to
another source file:
%
\begin{center}
\begin{tabular}{l}
|\input{childdoc.def}|\\
|\childdocforward[|\textit{main}|]{|\textit{dest}|}|\\
\end{tabular}
\end{center}
%
The argument \textit{dest} is the destination file
(without extension).
It should be the main file or one of the child files.
Note that further \textsf{childdoc} directives
such as |\childdocof| and |\childdocforward|
in the indicated file will be processed in this form.
The optional argument \textit{main}
passes on directly to the main file \textit{main}
while pretending to compile the child \textit{dest}.
This form behaves as if \textit{dest}
issues |\childdocof{|\textit{main}|}| right away,
and no further \textsf{childdoc} directives will be processed.

%%%%%%%%%%%%%%%%%%%%%%%%%%%%%%%%%%%%%%%%
\DescribeMacro{\...prefix}
In the alternative form |\childdocforwardprefix|,
%
\begin{center}
\begin{tabular}{l}
|\input{childdoc.def}|\\
|\childdocforwardprefix[|\textit{main}|]{|\textit{prefix}|}{|\textit{dest}|}|
\end{tabular}
\end{center}
%
the destination file is determined by a pattern
depending on the current file:
To make this work, the current file must be called
`{\textit{prefix}\hspace{0.2em}\textit{suffix}}'
with \textit{prefix} matching precisely the argument.
Processing is then passed on to the file
`{\textit{dest}\hspace{0.2em}\textit{suffix}}'.
Surely, the same effect is achieved by
directly specifying the
argument `{\textit{dest}\hspace{0.2em}\textit{suffix}}'
in the first form.
However, that requires to set up a different file
for each child. With the alternative form of the command
all these files can have exactly the same content
which simplifies setting them up and maintaining them.

For example, the following file |draft.tex|
with a compilation flag |\version| as described in \secref{sec:flags}
compiles the main document as a draft:
%
\begin{center}
\begin{tabular}{l}
|\def\version{draft}|\\
|\input{childdoc.def}|\\
|\childdocforward{|\textit{main}|}|
\end{tabular}
\end{center}
%
Likewise, the following files |final|\textit{nn}|.tex|
compile the final version of the child document
|child|\textit{nn}|.tex|:
%
\begin{center}
\begin{tabular}{l}
|\def\version{final}|\\
|\input{childdoc.def}|\\
|\childdocforwardprefix{final}{child}|
\end{tabular}
\end{center}
%

Note that when several versions of a main file and/or of each child file
are to be generated, it may be convenient to set up a |Makefile| or
shell script to automatise the process.

%%%%%%%%%%%%%%%%%%%%%%%%%%%%%%%%%%%%%%%%%%%%%%%%%%%%%%%%%%%%%%%%%%%%%%%%%%%%%%%%
\subsection{Command Line Processing}
\label{sec:commandline}

The effect of redirection files can also be achieved by invoking
the \LaTeX{} compiler with a more elaborate command line.
Most conveniently this should be done as part
of a shell script or a |Makefile|.

When using \textsf{childdoc} in the main file, the following
command lines effectively perform a redirection
(note that depending on the shell being used,
backslashes may have to be doubled: `|\|' $\to$ `|\\|'):
%
\begin{center}
|... -jobname "|\textit{target}|" |\\|"|[\textit{flags}]%
|\input{childdoc.def}\childdocforward[|\textit{main}|]{|\textit{dest}|}"|
\end{center}
%
Here \textit{target} is the name of the output file,
\textit{main} is the name of the main file
and \textit{dest} is the name of the main or child file to be processed
(all filenames without extensions).
The optional argument \textit{main} can be omitted
if \textit{main} matches \textit{dest}.
Optionally, compilation \textit{flags} can be defined via |\def| commands.
This command line makes the \TeX{} engine believe
it is compiling the file \textit{target}
whose content is specified as the latter parameter.
The provided code then forwards the processing to
\textit{main} or \textit{dest} as described in \secref{sec:forward}.

%%%%%%%%%%%%%%%%%%%%%%%%%%%%%%%%%%%%%%%%%%%%%%%%%%%%%%%%%%%%%%%%%%%%%%%%%%%%%%%%
\subsection{Include by Input}
\label{sec:input}

Including child documents by |\include| has some restrictions by design.
Most notably, the content of a child document always occupies
its own set of pages; pages cannot be shared between child documents.
Usually, this behaviour makes perfect sense
because each child document contain an essential part of the document.
However, in some situations it may be desirable to compose
a document from a collection of parts
without having mandatory page breaks between then.
For this case, the package
provides a mechanism to include parts
by |\input| which can also be processed individually.
However, by construction this mechanism
requires manual handling of the content to be output.

%%%%%%%%%%%%%%%%%%%%%%%%%%%%%%%%%%%%%%%%
\DescribeMacro{\ifchilddocmanual}
The main file should be prepared as usual, see \secref{sec:include}.
However, the document body must make a distinction
between processing of an individual part and of the main document, e.g.:
%
\begin{center}
\begin{tabular}{l}
|\ifchilddocmanual|\\
|\input{\childdocname}|\\
|\||else|\\
\textit{document body with }|\input{|\textit{part}|}|\\
|\||fi|
\end{tabular}
\end{center}
%
The conditional |\ifchilddocmanual| is true whenever
a part to be included by |\input| is being compiled,
and the name of the part is stored in |\childdocname|.

%%%%%%%%%%%%%%%%%%%%%%%%%%%%%%%%%%%%%%%%
\DescribeMacro{\childdocby}
Each part to be included by |\input| should start with:
%
\begin{center}
\begin{tabular}{l}
|\input{childdoc.def}|\\
|\childdocby{|\textit{main}|}|\\
\end{tabular}
\end{center}
%
The directive |\childdocby| is similar to |\childdocof|
described in \secref{sec:include},
but the subsequent selection of content must be done manually.
To that end, both |\ifchilddoc| and |\ifchilddocmanual|
will be true upon processing of a part,
and the name of the part is stored in |\childdocname|.
Note that |\jobname| will be set to the filename of the current part
so that each part receives an individual |.aux| file
that does not interfere with the |.aux| file(s) of the main document.
This behaviour can be altered by the alternative form
|\childdocby[*]{|\textit{main}|}| (with a non-empty optional argument)
which uses the |.aux| file of the main document
by setting |\jobname| to \textit{main}.

%%%%%%%%%%%%%%%%%%%%%%%%%%%%%%%%%%%%%%%%%%%%%%%%%%%%%%%%%%%%%%%%%%%%%%%%%%%%%%%%
\subsection{Driver Development}
\label{sec:driver}

The \textsf{childdoc} mechanism can also be use for the development
of definition files such as \LaTeX{} styles or classes.
This case differs from the above setup with multiple parts
included by |\include| in that no |\includeonly| should be invoked.
This can be achieved by starting the include file
(before |\ProvidesPackage|) with:
%
\begin{center}
\begin{tabular}{l}
|\input{childdoc.def}|\\
|\childdocforward{|\textit{main}|}|\\
\end{tabular}
\end{center}
%
or alternatively with:
%
\begin{center}
\begin{tabular}{l}
|\input{childdoc.def}|\\
|\childdocby{|\textit{main}|}|\\
\end{tabular}
\end{center}
%
Both forms have slightly different effects as described above.
The main file is prepared as usual, see \secref{sec:include}.

%%%%%%%%%%%%%%%%%%%%%%%%%%%%%%%%%%%%%%%%%%%%%%%%%%%%%%%%%%%%%%%%%%%%%%%%%%%%%%%%
\subsection{Legacy Detection}
\label{sec:detection}

The directive |\childdocmain| in the main file can detect
whether the complete document or merely a child is to be compiled
even without using the directive |\childdocof|.
This method is deprecated because it is less robust
and there is no compelling reason to use it;
it is merely provided for backward compatibility
and it may be removed in future versions.

If the detection mechanism is to be used,
it is mandatory to correctly specify
the filename of the main file as the argument of |\childdocmain|:
%
\begin{center}
\begin{tabular}{l}
|\input{childdoc.def}|\\
|\childdocmain{|\textit{main}|}|\\
\end{tabular}
\end{center}
%
If |\jobname| does not match the argument \textit{main} of |\childdocmain|,
it is assumed that |\jobname| points to the child file to be compiled.
When using |\childdocmain| with the main file specified as argument,
it suffices to start a child file
with just |\input{|\textit{main}|}|
without loading of the package and using |\childdocof|.
If instead all processing is done
with the appropriate \textsf{childdoc} directives,
the argument of \textit{main} of |\childdocmain| can be empty.

An alternative version of the command line processing described
in \secref{sec:commandline} using the detection mechanism reads:
%
\begin{center}
|... -jobname "|\textit{target}|" "|[\textit{flags}]%
[|\def\jobname{|\textit{dest}|}|]|\input{|\textit{main}|}"|
\end{center}

%%%%%%%%%%%%%%%%%%%%%%%%%%%%%%%%%%%%%%%%%%%%%%%%%%%%%%%%%%%%%%%%%%%%%%%%%%%%%%%%
\subsection{Manual Code}
\label{sec:manual}

In case one cannot be certain whether the definitions file |childdoc.def|
is installed on the target \TeX{} distribution
and one prefers not to ship it,
it is conceivable to paste a few relevant commands into the sources.

To that end, drop all statements |\input{childdoc.def}|
and perform the replacements as outlined below.
Instead of |\childdocmain{|\textit{main}|}| add the following code
to the top of the main file:
%
\begin{center}
\begin{tabular}{l}
|\||ifdefined\childdocname\endinput\||fi\newif\ifchilddoc|\\
|\edef\childdocname{\scantokens\expandafter{\jobname\noexpand}}|\\
|\def\childdocmain{|\textit{main}|}\||ifx\childdocmain\childdocname\||else|\\
|\childdoctrue\includeonly{\childdocname}\let\jobname\childdocmain\||fi|\\
\end{tabular}
\end{center}
%
Instead of |\childdocof{|\textit{main}|}| just include the main file
at the top of each child file:
%
\begin{center}
|\input{|\textit{main}|}|
\end{center}
%
A simple redirection |\childdocforward{|\textit{dest}|}| is achieved by:
%
\begin{center}
|\def\jobname{|\textit{dest}|}\input{\jobname}|
\end{center}
%
The redirection with prefix
|\childdocforwardprefix[|\textit{prefix}|]{|\textit{dest}|}|
is accomplished by:
%
\begin{center}
\begin{tabular}{l}
|{\edef\jobname{\scantokens\expandafter{\jobname\noexpand}}|\\
|\def\redirectjob |\textit{prefix}|#1~~~{\gdef\jobname{|\textit{dest}|#1}}|\\
|\expandafter\redirectjob\jobname~~~}\input{\jobname}|
\end{tabular}
\end{center}

In an alternative approach,
child documents can be compiled by a specific command line
without additional code or specific definitions:
%
\begin{center}
|... -jobname "|\textit{target}|" "|[\textit{flags}]%
|\includeonly{|\textit{dest}|}\input{|\textit{main}|}"|
\end{center}
%

%%%%%%%%%%%%%%%%%%%%%%%%%%%%%%%%%%%%%%%%%%%%%%%%%%%%%%%%%%%%%%%%%%%%%%%%%%%%%%%%
%%%%%%%%%%%%%%%%%%%%%%%%%%%%%%%%%%%%%%%%%%%%%%%%%%%%%%%%%%%%%%%%%%%%%%%%%%%%%%%%
\section{Information}

%%%%%%%%%%%%%%%%%%%%%%%%%%%%%%%%%%%%%%%%%%%%%%%%%%%%%%%%%%%%%%%%%%%%%%%%%%%%%%%%
\subsection{Copyright}

Copyright \copyright{} 2017--2018 Niklas Beisert

This work may be distributed and/or modified under the
conditions of the \LaTeX{} Project Public License, either version 1.3
of this license or (at your option) any later version.
The latest version of this license is in
  \url{http://www.latex-project.org/lppl.txt}
and version 1.3 or later is part of all distributions of \LaTeX{}
version 2005/12/01 or later.

This work has the LPPL maintenance status `maintained'.

The Current Maintainer of this work is Niklas Beisert.

This work consists of the files |README.txt|, |childdoc.ins| and |childdoc.dtx|
as well as the derived files |childdoc.def|, |cdocsamp.tex|
with |cdocsch1.tex|, |cdocsch2.tex|, |cdocspt3.tex|, |cdocspt4.tex|,
|cdocsdrf.tex|, |cdocsfn1.tex|, |cdocsfn2.tex|
as well as |childdoc.pdf|.

%%%%%%%%%%%%%%%%%%%%%%%%%%%%%%%%%%%%%%%%%%%%%%%%%%%%%%%%%%%%%%%%%%%%%%%%%%%%%%%%
\subsection{Files and Installation}

The package consists of the files:
%
\begin{center}
\begin{tabular}{ll}
    |README.txt|   & readme file \\
    |childdoc.ins| & installation file \\
    |childdoc.dtx| & source file \\
    |childdoc.def| & definition file \\
    |cdocsamp.tex| & sample main file \\
    |cdocsch1.tex| & sample include file \\
    |cdocsch2.tex| & sample include file \\
    |cdocspt3.tex| & sample part file \\
    |cdocspt4.tex| & sample part file \\
    |cdocsdrf.tex| & sample redirection file \\
    |cdocsfn1.tex| & sample redirection file \\
    |cdocsfn2.tex| & sample redirection file \\
    |childdoc.pdf| & manual
\end{tabular}
\end{center}
%
The distribution consists of the files
|README.txt|, |childdoc.ins| and |childdoc.dtx|.
%
\begin{itemize}
\item
Run (pdf)\LaTeX{} on |childdoc.dtx|
to compile the manual |childdoc.pdf| (this file).
\item
Run \LaTeX{} on |childdoc.ins| to create the definitions file |childdoc.def|
and the sample |cdocsamp.tex| with include files
|cdocsch1.tex|, |cdocsch2.tex|, |cdocspt3.tex|, |cdocspt4.tex|,
|cdocsdrf.tex|, |cdocsfn1.tex|, |cdocsfn2.tex|.
Then copy the file |childdoc.def| to an appropriate directory of your \LaTeX{}
distribution, e.g.\ \textit{texmf-root}|/tex/latex/childdoc|.
\end{itemize}

%%%%%%%%%%%%%%%%%%%%%%%%%%%%%%%%%%%%%%%%%%%%%%%%%%%%%%%%%%%%%%%%%%%%%%%%%%%%%%%%
\subsection{Related CTAN Packages}

There are several other packages which offer a similar functionality:
%
\begin{itemize}
\item
The packages
\href{http://ctan.org/pkg/docmute}{\textsf{docmute}},
\href{http://ctan.org/pkg/includex}{\textsf{includex}} and
\href{http://ctan.org/pkg/standalone}{\textsf{standalone}}
provide commands to include only the document body of
a child file thus allowing both files to be compiled individually.
\item
The packages \href{http://ctan.org/pkg/subdocs}{\textsf{subdocs}}
and \href{http://ctan.org/pkg/subfiles}{\textsf{subfiles}}
provide structures in which the main and child documents can be
encapsulated and allowing them to be compiled individually.
The inclusion mechanism is different from the conventional |\include|.
\item
The package \href{http://ctan.org/pkg/combine}{\textsf{combine}}
is an elaborate solution to combine several documents into one.
\end{itemize}
%
See also the CTAN topic \href{http://ctan.org/topic/subdocs}{\textsf{subdocs}}
for further related packages.
The present package differs from the above solutions in that
a document structure constructed with the conventional |\include| mechanism
just needs two extra commands at the top of every file
such that all constituent files can be compiled individually.

%%%%%%%%%%%%%%%%%%%%%%%%%%%%%%%%%%%%%%%%%%%%%%%%%%%%%%%%%%%%%%%%%%%%%%%%%%%%%%%%
%\subsection{Feature Suggestions}
%
%The following is a list of features which may be useful for future
%versions of this package:
%%
%\begin{itemize}
%\item
%\ldots
%\end{itemize}

%%%%%%%%%%%%%%%%%%%%%%%%%%%%%%%%%%%%%%%%%%%%%%%%%%%%%%%%%%%%%%%%%%%%%%%%%%%%%%%%
\subsection{Revision History}

%%%%%%%%%%%%%%%%%%%%%%%%%%%%%%%%%%%%%%%%
\paragraph{v2.0:} 2018/12/30

\begin{itemize}
\item
immediate forward processing
\item
added |\childdocby| mechanism
\item
manual restructured
\end{itemize}

%%%%%%%%%%%%%%%%%%%%%%%%%%%%%%%%%%%%%%%%
\paragraph{v1.6:} 2018/01/17

\begin{itemize}
\item
application for development of include files
\item
corrections to manual
\end{itemize}

%%%%%%%%%%%%%%%%%%%%%%%%%%%%%%%%%%%%%%%%
\paragraph{v1.5:} 2017/05/21

\begin{itemize}
\item
more complete structuring introduced
\item
|\childdocof| introduced
\item
|\childdoc| renamed to |\childdocmain|
\item
|\childredirect| renamed to |\childdocforward| and |\childdocforwardprefix|
and functionality expanded
\end{itemize}

%%%%%%%%%%%%%%%%%%%%%%%%%%%%%%%%%%%%%%%%
\paragraph{v1.0:} 2017/04/27

\begin{itemize}
\item
manual and install package
\item
first version published on CTAN
\end{itemize}

%%%%%%%%%%%%%%%%%%%%%%%%%%%%%%%%%%%%%%%%
\paragraph{v0.6:} 2017/04/26

\begin{itemize}
\item
redirection mechanism added
\end{itemize}

%%%%%%%%%%%%%%%%%%%%%%%%%%%%%%%%%%%%%%%%
\paragraph{v0.5:} 2017/04/26

\begin{itemize}
\item
functionality in definition file
\end{itemize}


%%%%%%%%%%%%%%%%%%%%%%%%%%%%%%%%%%%%%%%%%%%%%%%%%%%%%%%%%%%%%%%%%%%%%%%%%%%%%%%%
%%%%%%%%%%%%%%%%%%%%%%%%%%%%%%%%%%%%%%%%%%%%%%%%%%%%%%%%%%%%%%%%%%%%%%%%%%%%%%%%
%%%%%%%%%%%%%%%%%%%%%%%%%%%%%%%%%%%%%%%%%%%%%%%%%%%%%%%%%%%%%%%%%%%%%%%%%%%%%%%%
\appendix

\settowidth\MacroIndent{\rmfamily\scriptsize 000\ }

 \DocInput{childdoc.dtx}

\end{document}
%</driver>
% \fi
%
% %%%%%%%%%%%%%%%%%%%%%%%%%%%%%%%%%%%%%%%%%%%%%%%%%%%%%%%%%%%%%%%%%%%%%%%%%%%%%%
% %%%%%%%%%%%%%%%%%%%%%%%%%%%%%%%%%%%%%%%%%%%%%%%%%%%%%%%%%%%%%%%%%%%%%%%%%%%%%%
% \section{Sample}
%\iffalse
%<*samplemain>
%\fi
%
% The following presents a sample document
% with two chapters, two parts, a title page,
% a compile flag as well as three forwarding files to set the flag.
% It consists of eight |.tex| files:
% \begin{center}
% \begin{tabular}{ll}
% |cdocsamp.tex|&main file\\
% |cdocsch1.tex|&include file for chapter 1\\
% |cdocsch2.tex|&include file for chapter 2\\
% |cdocspt3.tex|&include file for part 3\\
% |cdocspt4.tex|&include file for part 4\\
% |cdocsdrf.tex|&forwarding file for main file in draft mode\\
% |cdocsfi1.tex|&forwarding file for final version of chapter 1\\
% |cdocsfi2.tex|&forwarding file for final version of chapter 2\\
% \end{tabular}
% \end{center}
% Each of the eight files can be compiled directly by the \LaTeX{} compiler.
%
% %%%%%%%%%%%%%%%%%%%%%%%%%%%%%%%%%%%%%%
% \paragraph{Main File.}
%
% The main file is called |cdocsamp.tex|.
%
% Load the \textsf{childdoc} definitions and
% declare the filename for the main document:
%    \begin{macrocode}
\input{childdoc.def}
\childdocmain{}
%    \end{macrocode}

% Optional override for |\version| flag:
%    \begin{macrocode}
%%\ifchilddoc\else\providecommand{\version}{draft}\fi
%    \end{macrocode}

% Define the default values for the |\version| flag
% (|final| for the main file and |draft| for childs):
%    \begin{macrocode}
\ifchilddoc
\providecommand{\version}{draft}
\else
\providecommand{\version}{final}
\fi
%    \end{macrocode}

% Load the standard document class:
%    \begin{macrocode}
\documentclass[12pt]{article}
%    \end{macrocode}

% Start the document body:
%    \begin{macrocode}
\begin{document}
%    \end{macrocode}

% Declare a title page.
% Print title, part of document being processed and version flag:
%    \begin{macrocode}
\addtocounter{page}{-1}
\begin{center}
{\LARGE\bfseries{}childdoc example\par}
\vspace{1cm}
\ifchilddoc
\ifchilddocmanual part\else chapter\fi:
`\childdocname' of `\childdocjob'\par
\else
main document: `\childdocjob'\par
\fi
version: \version\par
\end{center}
\newpage
%    \end{macrocode}

% Manually include selected file,
% otherwise process as usual:
%    \begin{macrocode}
\ifchilddocmanual
\section*{part `\childdocname'}
\input{\childdocname}
\else
%    \end{macrocode}

% Include the two chapters:
%    \begin{macrocode}
\include{cdocsch1}
\include{cdocsch2}
%    \end{macrocode}

% Include the two parts unless only chapters should be displayed:
%    \begin{macrocode}
\ifchilddoc\else
\section{part three}
\input{cdocspt3}
\section{part four}
\input{cdocspt4}
\fi
%    \end{macrocode}

% Process as usual until here:
%    \begin{macrocode}
\fi
%    \end{macrocode}

% End of document body:
%    \begin{macrocode}
\end{document}
%    \end{macrocode}
%\iffalse
%</samplemain>
%\fi
%
% %%%%%%%%%%%%%%%%%%%%%%%%%%%%%%%%%%%%%%
% \paragraph{Chapter Include Files.}
%
% The include files are called |cdocsch1.tex| and |cdocsch2.tex|.
%
%\iffalse
%<*samplechap1|samplechap2>
%\fi

% Optional override for |\version| flag:
%    \begin{macrocode}
%%\providecommand{\version}{final}
%    \end{macrocode}

% Include the main document:
%    \begin{macrocode}
\input{childdoc.def}
\childdocof{cdocsamp}
%    \end{macrocode}

%\iffalse
%</samplechap1|samplechap2>
%\fi
%
%\iffalse
%<*samplechap1>
%\fi
% Some text for chapter 1:
%    \begin{macrocode}
\section{one}
some text in chapter one
%    \end{macrocode}

%\iffalse
%</samplechap1>
%\fi
% Some text for chapter 2:
%\iffalse
%<*samplechap2>
%\fi
%    \begin{macrocode}
\section{two}
more text in chapter two
%    \end{macrocode}

%\iffalse
%</samplechap2>
%\fi
%
% %%%%%%%%%%%%%%%%%%%%%%%%%%%%%%%%%%%%%%
% \paragraph{Part Include Files.}
%
% The include files are called |cdocspt3.tex| and |cdocspt4.tex|.
%
%\iffalse
%<*samplepart3|samplepart4>
%\fi

% Optional override for |\version| flag:
%    \begin{macrocode}
%%\providecommand{\version}{final}
%    \end{macrocode}

% Include the main document:
%    \begin{macrocode}
\input{childdoc.def}
\childdocby{cdocsamp}
%    \end{macrocode}

%\iffalse
%</samplepart3|samplepart4>
%\fi
%
%\iffalse
%<*samplepart3>
%\fi
% Some text for part 3:
%    \begin{macrocode}
some text in part three
%    \end{macrocode}

%\iffalse
%</samplepart3>
%\fi
% Some text for part 4:
%\iffalse
%<*samplepart4>
%\fi
%    \begin{macrocode}
more text in part four
%    \end{macrocode}

%\iffalse
%</samplepart4>
%\fi
%
% %%%%%%%%%%%%%%%%%%%%%%%%%%%%%%%%%%%%%%
% \paragraph{Forwarding for a Complete Draft.}
%
% The following forwarding file |cdocsdrf.tex|
% compiles the main document in draft mode:
%\iffalse
%<*sampledraft>
%\fi
%    \begin{macrocode}
\def\version{draft}
\input{childdoc.def}
\childdocforward{cdocsamp}
%    \end{macrocode}

%\iffalse
%</sampledraft>
%\fi
%
% %%%%%%%%%%%%%%%%%%%%%%%%%%%%%%%%%%%%%%
% \paragraph{Forwarding for Final Version of the Chapters.}
%
% The following forwarding files |cdocsfn1.tex| and |cdocsfn2.tex|
% (with identical content)
% compile the final versions of the child documents
% |cdocsch1.tex| and |cdocsch2.tex|, respectively:
%\iffalse
%<*samplefinal>
%\fi
%    \begin{macrocode}
\def\version{final}
\input{childdoc.def}
\childdocforwardprefix[cdocsamp]{cdocsfn}{cdocsch}
%    \end{macrocode}

%\iffalse
%</samplefinal>
%\fi
%
% %%%%%%%%%%%%%%%%%%%%%%%%%%%%%%%%%%%%%%
% \paragraph{Command Line Processing.}
%
% The following three command lines generate the output files
% |cdocscld|, |cdocscl1| and |cdocscl2|
% which should be identical to
% |cdocsdrf|, |cdocsch1| and |cdocsfn2|, respectively:
% \begin{center}
% \begin{tabular}{l}
% |latex -jobname cdocscld \|\\
% |  "\def\version{draft}\input{childdoc.def}\childdocforward{cdocsamp}"|\\
% |latex -jobname cdocscl1 \|\\
% |  "\input{childdoc.def}\childdocforward[cdocsamp]{cdocsch1}"|\\
% |latex -jobname cdocscl2 \|\\
% |  "\def\version{final}\input{childdoc.def}\childdocforward{cdocsch2}"|
% \end{tabular}
% \end{center}
% Note that the trailing backslash on each first line
% merely continues the input to the second line
% (for convenient cut ant paste).
% Furthermore, the command |latex| can be replaced by any
% of its alternative versions such as |pdflatex|.
%
% %%%%%%%%%%%%%%%%%%%%%%%%%%%%%%%%%%%%%%%%%%%%%%%%%%%%%%%%%%%%%%%%%%%%%%%%%%%%%%
% %%%%%%%%%%%%%%%%%%%%%%%%%%%%%%%%%%%%%%%%%%%%%%%%%%%%%%%%%%%%%%%%%%%%%%%%%%%%%%
% \section{Implementation}
%\iffalse
%<*package>
%\fi
%
% This section describes the definitions file |childdoc.def|.

% The definitions cannot be loaded using |\usepackage| or |\RequirePackage|
% which has a mechanism to prevent loading a style file more than once.
% When loading the definitions by means of |\input|
% multiple instances have to be prevented manually:
%\iffalse
%This code needs to be before the `\ProvidesFile' directive
%which is defined at the beginning of this file.
%Therefore it is also placed there and commented out here.
%</package>
%<*discard>
%\fi
%    \begin{macrocode}
\ifdefined\childdocmain\endinput\fi
%    \end{macrocode}
%\iffalse
%</discard>
%<*package>
%\fi
%
% \macro{\ifchilddoc}
% \macro{\ifchilddocmanual}
% The conditional |\ifchilddoc| tells whether a
% child (true) or main (false) document is being compiled.
% The conditional |\ifchilddocmanual| tells whether
% the |\includeonly| mechanism is used (false) or
% the selection of child files must be performed manually (true).
% The definitions initialise to false:
%    \begin{macrocode}
\newif\ifchilddoc
\newif\ifchilddocmanual
%    \end{macrocode}

% \macro{\childdocname}
% \macro{\childdocjob}
% The macro |\childdocname| stores the name of the main document
% to be compiled. The macro |\childdocjob| stores the name of
% the document on which the \LaTeX{} compiler was originally invoked.
% The content of |\jobname| cannot be compared
% to filenames specified in the source due to different catcodes.
% The following code rescans |\jobname|, stores the result
% in |\childdocname| and saves a copy in |\childdocjob|:
%    \begin{macrocode}
\edef\childdocname{\scantokens\expandafter{\jobname\noexpand}}
\let\childdocjob\childdocname
%    \end{macrocode}

% \macro{\childdocdisable}
% The macro |\childdocdisable| prevents the main file
% from being processed more than once.
% At this stage, the main document command |\childdocmain|
% is assumed to be called once again where it should do nothing.
% Any subsequent call to it should prevent
% a secondary processing of the main document
% It overwrites the forwarding commands
% |\childdocof| and |\childdocforward|
% with empty macros to prevent further inclusions of the main document:
%    \begin{macrocode}
\newcommand{\childdocdisable}
{
  \renewcommand{\childdocmain}[1]{\renewcommand{\childdocmain}[1]{\endinput}}
  \renewcommand{\childdocof}[1]{}
  \renewcommand{\childdocby}[2][]{}
  \renewcommand{\childdocforward}[2][]{}
  \renewcommand{\childdocdisable}{}
}
%    \end{macrocode}

% \macro{\childdocmain}
% The macro |\childdocmain| is to be called at the top of the main file
% with nothing or the main filename (without extension) as argument.
% First, it breaks loops.
% If the argument is not empty and does not match |\childdocname|
% (which is set by the first inclusion of |childdoc.def|),
% |\ifchilddoc| is set to true, |\includeonly| is applied to the child file
% and |\jobname| is set to the main file
% (for proper handling of |.aux| files):
%    \begin{macrocode}
\newcommand{\childdocmain}[1]
{
  \childdocdisable\childdocmain{}
  \if?#1?\else
    \begingroup
      \def\childdoctmp{#1}
      \ifx\childdoctmp\childdocname
        \def\childdoctmp{}
      \else
        \def\childdoctmp
        {
          \childdoctrue
          \includeonly{\childdocname}
          \def\childdocjob{#1}
          \def\jobname{#1}
        }
      \fi
      \expandafter
    \endgroup
    \childdoctmp
  \fi
}
%    \end{macrocode}

% \macro{\childdocof}
% The command |\childdocof| redirects
% compilation to the main file |#1|.
%    \begin{macrocode}
\newcommand{\childdocof}[1]
{
  \childdocdisable
  \childdoctrue
  \includeonly{\childdocname}
  \def\jobname{#1}
  \def\childdocjob{#1}
  \input{#1}
}
%    \end{macrocode}

% \macro{\childdocby}
% The command |\childdocby| ....
%    \begin{macrocode}
\newcommand{\childdocby}[2][]
{
  \childdocdisable
  \childdoctrue
  \childdocmanualtrue
  \if?#1?\else
    \def\jobname{#2}
  \fi
  \def\childdocjob{#2}
  \input{#2}
  \endinput
}
%    \end{macrocode}

% \macro{\childdocforward}
% The command |\childdocforward| redirects
% compilation to the main file or
% (if the optional argument is given) a child file.
% Parameters are set as if the main file
% or a child file starting with |\childdocof| was compiled.
% Then compilation is handed over to the main file:
%    \begin{macrocode}
\newcommand{\childdocforward}[2][]
{
  \begingroup
    \if?#1?
      \def\childdoctmp
      {
        \def\childdocname{#2}
        \def\childdocjob{#2}
        \def\jobname{#2}
        \input{#2}
        \endinput
      }
    \else
      \def\childdoctmp
      {
        \childdocdisable
        \def\childdocname{#2}
        \childdoctrue
        \includeonly{#2}
        \def\childdocjob{#1}
        \def\jobname{#1}
        \input{#1}
        \endinput
      }
    \fi
    \expandafter
  \endgroup
  \childdoctmp
}
%    \end{macrocode}

% \macro{\childdocforwardprefix}
% The command |\childdocforwardprefix| redirects
% compilation to the main or a child file by means of a pattern.
% The prefix |#1| in the current filename is replaced by |#2|
% and the suffix of the current filename is kept
% (it is assumed that the filename does not contain the substring `|~~~|'
% which is used as a delimiter).
% Compilation is handed over to the new file by |\childdocforward|:
%    \begin{macrocode}
\newcommand{\childdocforwardprefix}[3][]
{
  \begingroup
    \def\childdocextract #2##1~~~{\def\childdoctmp{\childdocforward[#1]{#3##1}}}
    \expandafter\childdocextract\childdocname~~~
    \expandafter
  \endgroup
  \childdoctmp
}
%    \end{macrocode}

% \macro{\childdoc}
% The deprecated macro |\childdoc| is a legacy version of |\childdocmain|:
%    \begin{macrocode}
\newcommand{\childdoc}{\childdocmain}
%    \end{macrocode}

% \macro{\childdocredirect}
% The deprecated macro |\childdocredirect| is a legacy version
% of |\childdocforward| and |\childdocforwardprefix|:
%    \begin{macrocode}
\newcommand{\childdocredirect}[2][]
{
  \begingroup
    \if?#1?
      \def\childdoctmp{\childdocforward{#2}}
    \else
      \def\childdoctmp{\childdocforwardprefix{#1}{#2}}
    \fi
    \expandafter
  \endgroup
  \childdoctmp
}
%    \end{macrocode}

%\iffalse
%</package>
%\fi
%
\endinput
|\\
|\childdocforward{|\textit{main}|}|\\
\end{tabular}
\end{center}
%
or alternatively with:
%
\begin{center}
\begin{tabular}{l}
|% \iffalse
%
% childdoc.dtx Copyright (C) 2017-2018 Niklas Beisert
%
% This work may be distributed and/or modified under the
% conditions of the LaTeX Project Public License, either version 1.3
% of this license or (at your option) any later version.
% The latest version of this license is in
%   http://www.latex-project.org/lppl.txt
% and version 1.3 or later is part of all distributions of LaTeX
% version 2005/12/01 or later.
%
% This work has the LPPL maintenance status `maintained'.
%
% The Current Maintainer of this work is Niklas Beisert.
%
% This work consists of the files childdoc.dtx and childdoc.ins
% and the derived files childdoc.def and cdocsamp.tex with
% cdocsch1.tex, cdocsch2.tex, cdocsdrf.tex, cdocsfn1.tex, cdocsfn2.tex.
%
%<package>\ifdefined\childdocmain\endinput\fi
%<package>\ProvidesFile{childdoc.def}[2018/12/30 v2.0 child document driver]
%<samplemain>\ProvidesFile{cdocsamp.tex}[2018/12/30 v2.0 sample for childdoc]
%<*driver>
%\ProvidesFile{childdoc.drv}[2018/12/30 v2.0 childdoc reference manual file]
\PassOptionsToClass{10pt,a4paper}{article}
\documentclass{ltxdoc}

\usepackage[margin=35mm]{geometry}
\usepackage{hyperref}
\usepackage{hyperxmp}
\usepackage[usenames]{color}

\hypersetup{colorlinks=true}
\hypersetup{pdfstartview=FitH}
\hypersetup{pdfpagemode=UseNone}
\hypersetup{pdfsource={}}
\hypersetup{pdflang={en-UK}}
\hypersetup{pdfcopyright={Copyright 2017-2018 Niklas Beisert.
  This work may be distributed and/or modified under the
  conditions of the LaTeX Project Public License, either version 1.3
  of this license or (at your option) any later version.}}
\hypersetup{pdflicenseurl={http://www.latex-project.org/lppl.txt}}
\hypersetup{pdfcontactaddress={ETH Zurich, ITP, HIT K,
  Wolfgang-Pauli-Strasse 27}}
\hypersetup{pdfcontactpostcode={8093}}
\hypersetup{pdfcontactcity={Zurich}}
\hypersetup{pdfcontactcountry={Switzerland}}
\hypersetup{pdfcontactemail={nbeisert@itp.phys.ethz.ch}}
\hypersetup{pdfcontacturl={http://people.phys.ethz.ch/\xmptilde nbeisert/}}

\newcommand{\secref}[1]{\hyperref[#1]{section \ref*{#1}}}

\parskip1ex
\parindent0pt
\let\olditemize\itemize
\def\itemize{\olditemize\parskip0pt}

\begin{document}

\title{The \textsf{childdoc} Package}
\hypersetup{pdftitle={The childdoc Package}}
\author{Niklas Beisert\\[2ex]
  Institut f\"ur Theoretische Physik\\
  Eidgen\"ossische Technische Hochschule Z\"urich\\
  Wolfgang-Pauli-Strasse 27, 8093 Z\"urich, Switzerland\\[1ex]
  \href{mailto:nbeisert@itp.phys.ethz.ch}
  {\texttt{nbeisert@itp.phys.ethz.ch}}}
\hypersetup{pdfauthor={Niklas Beisert}}
\hypersetup{pdfsubject={Manual for the LaTeX2e Package childdoc}}
\date{30 December 2018, \textsf{v2.0}}
\maketitle

\begin{abstract}\noindent
\textsf{childdoc} is a \LaTeXe{} package
that enables the direct compilation
of document sections included by |\include|
to individual files.
\end{abstract}

\begingroup
\parskip0ex
\tableofcontents
\endgroup

%%%%%%%%%%%%%%%%%%%%%%%%%%%%%%%%%%%%%%%%%%%%%%%%%%%%%%%%%%%%%%%%%%%%%%%%%%%%%%%%
%%%%%%%%%%%%%%%%%%%%%%%%%%%%%%%%%%%%%%%%%%%%%%%%%%%%%%%%%%%%%%%%%%%%%%%%%%%%%%%%
\section{Introduction}

\LaTeX{} provides a mechanism to structure a large document (such as a book)
into a main file and several child files (containing the chapters)
using the |\include| command.
This mechanism is beneficial for documents
which span hundreds of pages in order to
make the source file(s) more manageable.
Moreover, compilation can be restricted to
selected child files by means of the |\includeonly| command.
The latter feature can be used to reduce the compilation time while editing
(this was significantly more useful in the earlier days of \LaTeX{})
or to generate a smaller document which is easier to navigate.
Another application of |\includeonly| is to generate
documents consisting of selected parts of the complete document.

However, there are a few drawbacks of the plain |\include| mechanism:
\begin{itemize}
\item
The child files cannot be compiled on their own,
they can only be compiled via the main file.
A naive editing environment
(such as a text editor with an option
to have the current file processed by \LaTeX)
may require one to switch to the main file before compiling;
attempting to compile the child file produces errors.
\item
The main file must be modified (each time)
to adjust the |\includeonly| command
to the present needs. This easily leaves the main file in a messy state.
\item
The generated document will always carry the filename
of the main document. This is inconvenient if
several child files are to be compiled and
to be kept for distribution.
\end{itemize}

The present package provides a simple interface
to make child files individually compilable by \LaTeX{}.
Compiling a child file then has the same effect as compiling
the main file with an |\includeonly| command
to select the appropriate child.
Moreover the generated document will carry the name of the child
rather than the main file.
This resolves all three above issues.

This feature is meant to make the editing of books,
thesis documents and lecture notes somewhat more convenient.
However, the package can also be used efficiently for
composing a series of documents (such as exercise sheets)
which are typically distributed individually.
It then assists the author in generating the individual documents
(potentially in different versions)
as well as a document containing the collected series.
Another application is in developing style files
or other kinds of included material
where compilation of the style file could redirect
to a sample or test file.

%%%%%%%%%%%%%%%%%%%%%%%%%%%%%%%%%%%%%%%%%%%%%%%%%%%%%%%%%%%%%%%%%%%%%%%%%%%%%%%%
%%%%%%%%%%%%%%%%%%%%%%%%%%%%%%%%%%%%%%%%%%%%%%%%%%%%%%%%%%%%%%%%%%%%%%%%%%%%%%%%
\section{Usage}

First of all, the package \textsf{childdoc} is \emph{not} a standard
\LaTeXe{} |.sty| style file! Therefore it needs to be invoked in
a non-standard way.

%%%%%%%%%%%%%%%%%%%%%%%%%%%%%%%%%%%%%%%%%%%%%%%%%%%%%%%%%%%%%%%%%%%%%%%%%%%%%%%%
\subsection{Included Files}
\label{sec:include}

%%%%%%%%%%%%%%%%%%%%%%%%%%%%%%%%%%%%%%%%
\DescribeMacro{\childdocmain}
To use the package, add the commands
\begin{center}
\begin{tabular}{l}
|\input{childdoc.def}|\\
|\childdocmain{}|\\
\end{tabular}
\end{center}
at the very top of the main \LaTeX{} file,
in particular \emph{before} the |\documentclass| statement!
The argument of |\childdocmain| should be left empty
(but it must be present).

%%%%%%%%%%%%%%%%%%%%%%%%%%%%%%%%%%%%%%%%
\DescribeMacro{\childdocof}
Furthermore, add the commands
\begin{center}
\begin{tabular}{l}
|\input{childdoc.def}|\\
|\childdocof{|\textit{main}|}|\\
\end{tabular}
\end{center}
at the top of every child file \textit{child}
which is included by |\include{|\textit{child}|}|
from within the main file
(or at least for those files to be compiled individually).
The argument \textit{main} must be the filename of the main file.

There are a couple of
considerations in setting up the main and child documents:

%%%%%%%%%%%%%%%%%%%%%%%%%%%%%%%%%%%%%%%%
\paragraph{Restrictions.}

Please note the following restrictions:
\begin{itemize}
\item
|\childdocmain| must be called with one argument \textit{main}
to ensure compatibility with earlier version of the package.
It must either be empty (|\childdocmain{}|)
or precisely match the filename of the main file in which it is specified.
See \secref{sec:detection} for further information.
\item
The filename \textit{main} must be specified without the |.tex| extension.
\item
The filename \textit{main} is case sensitive
(even in case-insensitive file systems)
due to internal string comparison.
\item
The argument \textit{main} should be fully expanded, it cannot be a macro.
\item
Subdirectories and special characters should be avoided in filenames.
\item
The command |\childdocmain{|\textit{main}|}| must be followed by a whitespace.
It should not be followed immediately by another command
or by a comment mark `|%|'.
This is because the \TeX{} parser reads the token immediately following
the argument of |\childdocmain| and puts it
at the beginning of every child section;
however, a white\-space is ignored.
\end{itemize}

%%%%%%%%%%%%%%%%%%%%%%%%%%%%%%%%%%%%%%%%
\paragraph{Content of Main File.}

It is advisable to place all content in the child files included by |\include|.
Any output contained in the main file will appear in all child documents
unless suppressed manually;
it cannot be suppressed automatically by the |\includeonly| directive
and thus should normally be avoided.
A method to include some content in the main file
by means of conditional processing is described in \secref{sec:conditional}.

%%%%%%%%%%%%%%%%%%%%%%%%%%%%%%%%%%%%%%%%
\paragraph{Page Numbering.}

When only a part of the document is compiled,
the appropriate numbering of pages
(as well as other status parameters)
is determined from the |.aux| files.
The latter contain information from previous passes.
However this information needs to propagate through
all intermediate child documents.
Therefore the page numbering in child documents may well
be inconsistent until the complete document is compiled at least once.

A useful (if unconventional) way to always ensure a consistent
page numbering is to restart the numbering in each child document
and denote the pages by `\textit{child}|.|\textit{page}'
where \textit{child} represents the chapter/section number of the child file.
This can be achieved by the command
|\numberwithin{page}{|\textit{child}|}|
of the \textsf{amsmath} package
where \textit{child} can be |chapter| or |section|
depending on the chosen structuring.
Alternatively, one can modify the macro |\thepage| appropriately
and reset the counter |page| at the start of each child file.

%%%%%%%%%%%%%%%%%%%%%%%%%%%%%%%%%%%%%%%%%%%%%%%%%%%%%%%%%%%%%%%%%%%%%%%%%%%%%%%%
\subsection{Conditional Processing}
\label{sec:conditional}

The package provides a mechanism to compile different versions
of a document. To customise the versions further some conditional processing
can come in handy to distinguish which version is being compiled.
The package provides two macros to describe the compilation context:

%%%%%%%%%%%%%%%%%%%%%%%%%%%%%%%%%%%%%%%%
\DescribeMacro{\ifchilddoc}
The conditional |\ifchilddoc| distinguishes between the compilation of
child documents and the main document:
%
\begin{center}
|\ifchilddoc |\textit{child-code}| |[|\||else |\textit{main-code}]| \||fi|
\end{center}

%%%%%%%%%%%%%%%%%%%%%%%%%%%%%%%%%%%%%%%%
\DescribeMacro{\childdocname}
\DescribeMacro{\childdocjob}
The macro |\childdocname| contains the filename (without extension)
of the main or child file being processed.
Note that |\childdocjob| will always contain the name of the main file.

%%%%%%%%%%%%%%%%%%%%%%%%%%%%%%%%%%%%%%%%
\paragraph{Title Page.}

Conditional processing can be used to include a title or banner page
in the main document when proper precautions are taken.
Importantly, the code in the main file should ensure that the page counter
(as well as other status parameters which are stored in the |.aux| files)
takes the same value after the conditional processing.
Otherwise the page numbers may take divergent values
depending on which part is compiled.

For example, a title page could be declared by:
%
\begin{center}
\begin{tabular}{l}
|\ifchilddoc\||else|\\
|\addtocounter{page}{-1}|\\
\textit{code for title page}\\
|\newpage|\\
|\||fi|
\end{tabular}
\end{center}
%
A banner page for the child documents can be generated by:
%
\begin{center}
\begin{tabular}{l}
|\ifchilddoc|\\
|\addtocounter{page}{-1}|\\
\textit{code for banner page}\\
|\newpage|\\
|\||fi|
\end{tabular}
\end{center}
%
Here one could write a message such as:
\begin{center}
|This is the part \childdocname{} of \childdocjob{}.|
\end{center}

%%%%%%%%%%%%%%%%%%%%%%%%%%%%%%%%%%%%%%%%%%%%%%%%%%%%%%%%%%%%%%%%%%%%%%%%%%%%%%%%
\subsection{Flags}
\label{sec:flags}

The package makes it easy to generate different versions
of the main or child documents.
To this end compilation flags can be defined
and assigned different default values.
They will be particularly useful in conjunction
with the forwarding mechanism described in \secref{sec:forward}.

For example, it may be useful to have a flag |\version|
which can be set to |draft| or |final|.
The document source will contain some conditional code
depending on the value of |\version|.
Suppose further, the flag should default to |final| for the main file
and to |draft| for child files
which is a natural assignment for editing the document.
This is achieved by placing the following code
in the preamble of the main document
(below the |\childdocmain| directive):
%
\begin{center}
\begin{tabular}{l}
|\ifchilddoc|\\
|\providecommand{\version}{draft}|\\
|\||else|\\
|\providecommand{\version}{final}|\\
|\||fi|
\end{tabular}
\end{center}
%
The definition by |\providecommand| makes sure
that previous definitions are not overwritten.
Further statements |\providecommand{\version}{...}|
can thus be added before the above code to override it.

For the main file, one might add a line
(between |\childdocmain| and the above block)
%
\begin{center}
|%\ifchilddoc\||else\providecommand{\version}{draft}\||fi|
\end{center}
%
which can be uncommented to produce a draft version.
Likewise one can add a line to the very top of a child file
(above the |\childdocof{|\textit{main}|}| directive)
%
\begin{center}
|%\providecommand{\version}{final}|
\end{center}
%
which can be uncommented to produce the final version of this child document.

%%%%%%%%%%%%%%%%%%%%%%%%%%%%%%%%%%%%%%%%%%%%%%%%%%%%%%%%%%%%%%%%%%%%%%%%%%%%%%%%
\subsection{Forwarding}
\label{sec:forward}

Different versions of the main or child documents
using compilation flags as described in \secref{sec:flags}
can be (permanently) stored in different files
for convenient compilation, viewing and distribution.
To this end, the package defines a command
to pass on compilation to a different file:

%%%%%%%%%%%%%%%%%%%%%%%%%%%%%%%%%%%%%%%%
\DescribeMacro{\childdocforward}
The command |\childdocforward| redirects processing to
another source file:
%
\begin{center}
\begin{tabular}{l}
|\input{childdoc.def}|\\
|\childdocforward[|\textit{main}|]{|\textit{dest}|}|\\
\end{tabular}
\end{center}
%
The argument \textit{dest} is the destination file
(without extension).
It should be the main file or one of the child files.
Note that further \textsf{childdoc} directives
such as |\childdocof| and |\childdocforward|
in the indicated file will be processed in this form.
The optional argument \textit{main}
passes on directly to the main file \textit{main}
while pretending to compile the child \textit{dest}.
This form behaves as if \textit{dest}
issues |\childdocof{|\textit{main}|}| right away,
and no further \textsf{childdoc} directives will be processed.

%%%%%%%%%%%%%%%%%%%%%%%%%%%%%%%%%%%%%%%%
\DescribeMacro{\...prefix}
In the alternative form |\childdocforwardprefix|,
%
\begin{center}
\begin{tabular}{l}
|\input{childdoc.def}|\\
|\childdocforwardprefix[|\textit{main}|]{|\textit{prefix}|}{|\textit{dest}|}|
\end{tabular}
\end{center}
%
the destination file is determined by a pattern
depending on the current file:
To make this work, the current file must be called
`{\textit{prefix}\hspace{0.2em}\textit{suffix}}'
with \textit{prefix} matching precisely the argument.
Processing is then passed on to the file
`{\textit{dest}\hspace{0.2em}\textit{suffix}}'.
Surely, the same effect is achieved by
directly specifying the
argument `{\textit{dest}\hspace{0.2em}\textit{suffix}}'
in the first form.
However, that requires to set up a different file
for each child. With the alternative form of the command
all these files can have exactly the same content
which simplifies setting them up and maintaining them.

For example, the following file |draft.tex|
with a compilation flag |\version| as described in \secref{sec:flags}
compiles the main document as a draft:
%
\begin{center}
\begin{tabular}{l}
|\def\version{draft}|\\
|\input{childdoc.def}|\\
|\childdocforward{|\textit{main}|}|
\end{tabular}
\end{center}
%
Likewise, the following files |final|\textit{nn}|.tex|
compile the final version of the child document
|child|\textit{nn}|.tex|:
%
\begin{center}
\begin{tabular}{l}
|\def\version{final}|\\
|\input{childdoc.def}|\\
|\childdocforwardprefix{final}{child}|
\end{tabular}
\end{center}
%

Note that when several versions of a main file and/or of each child file
are to be generated, it may be convenient to set up a |Makefile| or
shell script to automatise the process.

%%%%%%%%%%%%%%%%%%%%%%%%%%%%%%%%%%%%%%%%%%%%%%%%%%%%%%%%%%%%%%%%%%%%%%%%%%%%%%%%
\subsection{Command Line Processing}
\label{sec:commandline}

The effect of redirection files can also be achieved by invoking
the \LaTeX{} compiler with a more elaborate command line.
Most conveniently this should be done as part
of a shell script or a |Makefile|.

When using \textsf{childdoc} in the main file, the following
command lines effectively perform a redirection
(note that depending on the shell being used,
backslashes may have to be doubled: `|\|' $\to$ `|\\|'):
%
\begin{center}
|... -jobname "|\textit{target}|" |\\|"|[\textit{flags}]%
|\input{childdoc.def}\childdocforward[|\textit{main}|]{|\textit{dest}|}"|
\end{center}
%
Here \textit{target} is the name of the output file,
\textit{main} is the name of the main file
and \textit{dest} is the name of the main or child file to be processed
(all filenames without extensions).
The optional argument \textit{main} can be omitted
if \textit{main} matches \textit{dest}.
Optionally, compilation \textit{flags} can be defined via |\def| commands.
This command line makes the \TeX{} engine believe
it is compiling the file \textit{target}
whose content is specified as the latter parameter.
The provided code then forwards the processing to
\textit{main} or \textit{dest} as described in \secref{sec:forward}.

%%%%%%%%%%%%%%%%%%%%%%%%%%%%%%%%%%%%%%%%%%%%%%%%%%%%%%%%%%%%%%%%%%%%%%%%%%%%%%%%
\subsection{Include by Input}
\label{sec:input}

Including child documents by |\include| has some restrictions by design.
Most notably, the content of a child document always occupies
its own set of pages; pages cannot be shared between child documents.
Usually, this behaviour makes perfect sense
because each child document contain an essential part of the document.
However, in some situations it may be desirable to compose
a document from a collection of parts
without having mandatory page breaks between then.
For this case, the package
provides a mechanism to include parts
by |\input| which can also be processed individually.
However, by construction this mechanism
requires manual handling of the content to be output.

%%%%%%%%%%%%%%%%%%%%%%%%%%%%%%%%%%%%%%%%
\DescribeMacro{\ifchilddocmanual}
The main file should be prepared as usual, see \secref{sec:include}.
However, the document body must make a distinction
between processing of an individual part and of the main document, e.g.:
%
\begin{center}
\begin{tabular}{l}
|\ifchilddocmanual|\\
|\input{\childdocname}|\\
|\||else|\\
\textit{document body with }|\input{|\textit{part}|}|\\
|\||fi|
\end{tabular}
\end{center}
%
The conditional |\ifchilddocmanual| is true whenever
a part to be included by |\input| is being compiled,
and the name of the part is stored in |\childdocname|.

%%%%%%%%%%%%%%%%%%%%%%%%%%%%%%%%%%%%%%%%
\DescribeMacro{\childdocby}
Each part to be included by |\input| should start with:
%
\begin{center}
\begin{tabular}{l}
|\input{childdoc.def}|\\
|\childdocby{|\textit{main}|}|\\
\end{tabular}
\end{center}
%
The directive |\childdocby| is similar to |\childdocof|
described in \secref{sec:include},
but the subsequent selection of content must be done manually.
To that end, both |\ifchilddoc| and |\ifchilddocmanual|
will be true upon processing of a part,
and the name of the part is stored in |\childdocname|.
Note that |\jobname| will be set to the filename of the current part
so that each part receives an individual |.aux| file
that does not interfere with the |.aux| file(s) of the main document.
This behaviour can be altered by the alternative form
|\childdocby[*]{|\textit{main}|}| (with a non-empty optional argument)
which uses the |.aux| file of the main document
by setting |\jobname| to \textit{main}.

%%%%%%%%%%%%%%%%%%%%%%%%%%%%%%%%%%%%%%%%%%%%%%%%%%%%%%%%%%%%%%%%%%%%%%%%%%%%%%%%
\subsection{Driver Development}
\label{sec:driver}

The \textsf{childdoc} mechanism can also be use for the development
of definition files such as \LaTeX{} styles or classes.
This case differs from the above setup with multiple parts
included by |\include| in that no |\includeonly| should be invoked.
This can be achieved by starting the include file
(before |\ProvidesPackage|) with:
%
\begin{center}
\begin{tabular}{l}
|\input{childdoc.def}|\\
|\childdocforward{|\textit{main}|}|\\
\end{tabular}
\end{center}
%
or alternatively with:
%
\begin{center}
\begin{tabular}{l}
|\input{childdoc.def}|\\
|\childdocby{|\textit{main}|}|\\
\end{tabular}
\end{center}
%
Both forms have slightly different effects as described above.
The main file is prepared as usual, see \secref{sec:include}.

%%%%%%%%%%%%%%%%%%%%%%%%%%%%%%%%%%%%%%%%%%%%%%%%%%%%%%%%%%%%%%%%%%%%%%%%%%%%%%%%
\subsection{Legacy Detection}
\label{sec:detection}

The directive |\childdocmain| in the main file can detect
whether the complete document or merely a child is to be compiled
even without using the directive |\childdocof|.
This method is deprecated because it is less robust
and there is no compelling reason to use it;
it is merely provided for backward compatibility
and it may be removed in future versions.

If the detection mechanism is to be used,
it is mandatory to correctly specify
the filename of the main file as the argument of |\childdocmain|:
%
\begin{center}
\begin{tabular}{l}
|\input{childdoc.def}|\\
|\childdocmain{|\textit{main}|}|\\
\end{tabular}
\end{center}
%
If |\jobname| does not match the argument \textit{main} of |\childdocmain|,
it is assumed that |\jobname| points to the child file to be compiled.
When using |\childdocmain| with the main file specified as argument,
it suffices to start a child file
with just |\input{|\textit{main}|}|
without loading of the package and using |\childdocof|.
If instead all processing is done
with the appropriate \textsf{childdoc} directives,
the argument of \textit{main} of |\childdocmain| can be empty.

An alternative version of the command line processing described
in \secref{sec:commandline} using the detection mechanism reads:
%
\begin{center}
|... -jobname "|\textit{target}|" "|[\textit{flags}]%
[|\def\jobname{|\textit{dest}|}|]|\input{|\textit{main}|}"|
\end{center}

%%%%%%%%%%%%%%%%%%%%%%%%%%%%%%%%%%%%%%%%%%%%%%%%%%%%%%%%%%%%%%%%%%%%%%%%%%%%%%%%
\subsection{Manual Code}
\label{sec:manual}

In case one cannot be certain whether the definitions file |childdoc.def|
is installed on the target \TeX{} distribution
and one prefers not to ship it,
it is conceivable to paste a few relevant commands into the sources.

To that end, drop all statements |\input{childdoc.def}|
and perform the replacements as outlined below.
Instead of |\childdocmain{|\textit{main}|}| add the following code
to the top of the main file:
%
\begin{center}
\begin{tabular}{l}
|\||ifdefined\childdocname\endinput\||fi\newif\ifchilddoc|\\
|\edef\childdocname{\scantokens\expandafter{\jobname\noexpand}}|\\
|\def\childdocmain{|\textit{main}|}\||ifx\childdocmain\childdocname\||else|\\
|\childdoctrue\includeonly{\childdocname}\let\jobname\childdocmain\||fi|\\
\end{tabular}
\end{center}
%
Instead of |\childdocof{|\textit{main}|}| just include the main file
at the top of each child file:
%
\begin{center}
|\input{|\textit{main}|}|
\end{center}
%
A simple redirection |\childdocforward{|\textit{dest}|}| is achieved by:
%
\begin{center}
|\def\jobname{|\textit{dest}|}\input{\jobname}|
\end{center}
%
The redirection with prefix
|\childdocforwardprefix[|\textit{prefix}|]{|\textit{dest}|}|
is accomplished by:
%
\begin{center}
\begin{tabular}{l}
|{\edef\jobname{\scantokens\expandafter{\jobname\noexpand}}|\\
|\def\redirectjob |\textit{prefix}|#1~~~{\gdef\jobname{|\textit{dest}|#1}}|\\
|\expandafter\redirectjob\jobname~~~}\input{\jobname}|
\end{tabular}
\end{center}

In an alternative approach,
child documents can be compiled by a specific command line
without additional code or specific definitions:
%
\begin{center}
|... -jobname "|\textit{target}|" "|[\textit{flags}]%
|\includeonly{|\textit{dest}|}\input{|\textit{main}|}"|
\end{center}
%

%%%%%%%%%%%%%%%%%%%%%%%%%%%%%%%%%%%%%%%%%%%%%%%%%%%%%%%%%%%%%%%%%%%%%%%%%%%%%%%%
%%%%%%%%%%%%%%%%%%%%%%%%%%%%%%%%%%%%%%%%%%%%%%%%%%%%%%%%%%%%%%%%%%%%%%%%%%%%%%%%
\section{Information}

%%%%%%%%%%%%%%%%%%%%%%%%%%%%%%%%%%%%%%%%%%%%%%%%%%%%%%%%%%%%%%%%%%%%%%%%%%%%%%%%
\subsection{Copyright}

Copyright \copyright{} 2017--2018 Niklas Beisert

This work may be distributed and/or modified under the
conditions of the \LaTeX{} Project Public License, either version 1.3
of this license or (at your option) any later version.
The latest version of this license is in
  \url{http://www.latex-project.org/lppl.txt}
and version 1.3 or later is part of all distributions of \LaTeX{}
version 2005/12/01 or later.

This work has the LPPL maintenance status `maintained'.

The Current Maintainer of this work is Niklas Beisert.

This work consists of the files |README.txt|, |childdoc.ins| and |childdoc.dtx|
as well as the derived files |childdoc.def|, |cdocsamp.tex|
with |cdocsch1.tex|, |cdocsch2.tex|, |cdocspt3.tex|, |cdocspt4.tex|,
|cdocsdrf.tex|, |cdocsfn1.tex|, |cdocsfn2.tex|
as well as |childdoc.pdf|.

%%%%%%%%%%%%%%%%%%%%%%%%%%%%%%%%%%%%%%%%%%%%%%%%%%%%%%%%%%%%%%%%%%%%%%%%%%%%%%%%
\subsection{Files and Installation}

The package consists of the files:
%
\begin{center}
\begin{tabular}{ll}
    |README.txt|   & readme file \\
    |childdoc.ins| & installation file \\
    |childdoc.dtx| & source file \\
    |childdoc.def| & definition file \\
    |cdocsamp.tex| & sample main file \\
    |cdocsch1.tex| & sample include file \\
    |cdocsch2.tex| & sample include file \\
    |cdocspt3.tex| & sample part file \\
    |cdocspt4.tex| & sample part file \\
    |cdocsdrf.tex| & sample redirection file \\
    |cdocsfn1.tex| & sample redirection file \\
    |cdocsfn2.tex| & sample redirection file \\
    |childdoc.pdf| & manual
\end{tabular}
\end{center}
%
The distribution consists of the files
|README.txt|, |childdoc.ins| and |childdoc.dtx|.
%
\begin{itemize}
\item
Run (pdf)\LaTeX{} on |childdoc.dtx|
to compile the manual |childdoc.pdf| (this file).
\item
Run \LaTeX{} on |childdoc.ins| to create the definitions file |childdoc.def|
and the sample |cdocsamp.tex| with include files
|cdocsch1.tex|, |cdocsch2.tex|, |cdocspt3.tex|, |cdocspt4.tex|,
|cdocsdrf.tex|, |cdocsfn1.tex|, |cdocsfn2.tex|.
Then copy the file |childdoc.def| to an appropriate directory of your \LaTeX{}
distribution, e.g.\ \textit{texmf-root}|/tex/latex/childdoc|.
\end{itemize}

%%%%%%%%%%%%%%%%%%%%%%%%%%%%%%%%%%%%%%%%%%%%%%%%%%%%%%%%%%%%%%%%%%%%%%%%%%%%%%%%
\subsection{Related CTAN Packages}

There are several other packages which offer a similar functionality:
%
\begin{itemize}
\item
The packages
\href{http://ctan.org/pkg/docmute}{\textsf{docmute}},
\href{http://ctan.org/pkg/includex}{\textsf{includex}} and
\href{http://ctan.org/pkg/standalone}{\textsf{standalone}}
provide commands to include only the document body of
a child file thus allowing both files to be compiled individually.
\item
The packages \href{http://ctan.org/pkg/subdocs}{\textsf{subdocs}}
and \href{http://ctan.org/pkg/subfiles}{\textsf{subfiles}}
provide structures in which the main and child documents can be
encapsulated and allowing them to be compiled individually.
The inclusion mechanism is different from the conventional |\include|.
\item
The package \href{http://ctan.org/pkg/combine}{\textsf{combine}}
is an elaborate solution to combine several documents into one.
\end{itemize}
%
See also the CTAN topic \href{http://ctan.org/topic/subdocs}{\textsf{subdocs}}
for further related packages.
The present package differs from the above solutions in that
a document structure constructed with the conventional |\include| mechanism
just needs two extra commands at the top of every file
such that all constituent files can be compiled individually.

%%%%%%%%%%%%%%%%%%%%%%%%%%%%%%%%%%%%%%%%%%%%%%%%%%%%%%%%%%%%%%%%%%%%%%%%%%%%%%%%
%\subsection{Feature Suggestions}
%
%The following is a list of features which may be useful for future
%versions of this package:
%%
%\begin{itemize}
%\item
%\ldots
%\end{itemize}

%%%%%%%%%%%%%%%%%%%%%%%%%%%%%%%%%%%%%%%%%%%%%%%%%%%%%%%%%%%%%%%%%%%%%%%%%%%%%%%%
\subsection{Revision History}

%%%%%%%%%%%%%%%%%%%%%%%%%%%%%%%%%%%%%%%%
\paragraph{v2.0:} 2018/12/30

\begin{itemize}
\item
immediate forward processing
\item
added |\childdocby| mechanism
\item
manual restructured
\end{itemize}

%%%%%%%%%%%%%%%%%%%%%%%%%%%%%%%%%%%%%%%%
\paragraph{v1.6:} 2018/01/17

\begin{itemize}
\item
application for development of include files
\item
corrections to manual
\end{itemize}

%%%%%%%%%%%%%%%%%%%%%%%%%%%%%%%%%%%%%%%%
\paragraph{v1.5:} 2017/05/21

\begin{itemize}
\item
more complete structuring introduced
\item
|\childdocof| introduced
\item
|\childdoc| renamed to |\childdocmain|
\item
|\childredirect| renamed to |\childdocforward| and |\childdocforwardprefix|
and functionality expanded
\end{itemize}

%%%%%%%%%%%%%%%%%%%%%%%%%%%%%%%%%%%%%%%%
\paragraph{v1.0:} 2017/04/27

\begin{itemize}
\item
manual and install package
\item
first version published on CTAN
\end{itemize}

%%%%%%%%%%%%%%%%%%%%%%%%%%%%%%%%%%%%%%%%
\paragraph{v0.6:} 2017/04/26

\begin{itemize}
\item
redirection mechanism added
\end{itemize}

%%%%%%%%%%%%%%%%%%%%%%%%%%%%%%%%%%%%%%%%
\paragraph{v0.5:} 2017/04/26

\begin{itemize}
\item
functionality in definition file
\end{itemize}


%%%%%%%%%%%%%%%%%%%%%%%%%%%%%%%%%%%%%%%%%%%%%%%%%%%%%%%%%%%%%%%%%%%%%%%%%%%%%%%%
%%%%%%%%%%%%%%%%%%%%%%%%%%%%%%%%%%%%%%%%%%%%%%%%%%%%%%%%%%%%%%%%%%%%%%%%%%%%%%%%
%%%%%%%%%%%%%%%%%%%%%%%%%%%%%%%%%%%%%%%%%%%%%%%%%%%%%%%%%%%%%%%%%%%%%%%%%%%%%%%%
\appendix

\settowidth\MacroIndent{\rmfamily\scriptsize 000\ }

 \DocInput{childdoc.dtx}

\end{document}
%</driver>
% \fi
%
% %%%%%%%%%%%%%%%%%%%%%%%%%%%%%%%%%%%%%%%%%%%%%%%%%%%%%%%%%%%%%%%%%%%%%%%%%%%%%%
% %%%%%%%%%%%%%%%%%%%%%%%%%%%%%%%%%%%%%%%%%%%%%%%%%%%%%%%%%%%%%%%%%%%%%%%%%%%%%%
% \section{Sample}
%\iffalse
%<*samplemain>
%\fi
%
% The following presents a sample document
% with two chapters, two parts, a title page,
% a compile flag as well as three forwarding files to set the flag.
% It consists of eight |.tex| files:
% \begin{center}
% \begin{tabular}{ll}
% |cdocsamp.tex|&main file\\
% |cdocsch1.tex|&include file for chapter 1\\
% |cdocsch2.tex|&include file for chapter 2\\
% |cdocspt3.tex|&include file for part 3\\
% |cdocspt4.tex|&include file for part 4\\
% |cdocsdrf.tex|&forwarding file for main file in draft mode\\
% |cdocsfi1.tex|&forwarding file for final version of chapter 1\\
% |cdocsfi2.tex|&forwarding file for final version of chapter 2\\
% \end{tabular}
% \end{center}
% Each of the eight files can be compiled directly by the \LaTeX{} compiler.
%
% %%%%%%%%%%%%%%%%%%%%%%%%%%%%%%%%%%%%%%
% \paragraph{Main File.}
%
% The main file is called |cdocsamp.tex|.
%
% Load the \textsf{childdoc} definitions and
% declare the filename for the main document:
%    \begin{macrocode}
\input{childdoc.def}
\childdocmain{}
%    \end{macrocode}

% Optional override for |\version| flag:
%    \begin{macrocode}
%%\ifchilddoc\else\providecommand{\version}{draft}\fi
%    \end{macrocode}

% Define the default values for the |\version| flag
% (|final| for the main file and |draft| for childs):
%    \begin{macrocode}
\ifchilddoc
\providecommand{\version}{draft}
\else
\providecommand{\version}{final}
\fi
%    \end{macrocode}

% Load the standard document class:
%    \begin{macrocode}
\documentclass[12pt]{article}
%    \end{macrocode}

% Start the document body:
%    \begin{macrocode}
\begin{document}
%    \end{macrocode}

% Declare a title page.
% Print title, part of document being processed and version flag:
%    \begin{macrocode}
\addtocounter{page}{-1}
\begin{center}
{\LARGE\bfseries{}childdoc example\par}
\vspace{1cm}
\ifchilddoc
\ifchilddocmanual part\else chapter\fi:
`\childdocname' of `\childdocjob'\par
\else
main document: `\childdocjob'\par
\fi
version: \version\par
\end{center}
\newpage
%    \end{macrocode}

% Manually include selected file,
% otherwise process as usual:
%    \begin{macrocode}
\ifchilddocmanual
\section*{part `\childdocname'}
\input{\childdocname}
\else
%    \end{macrocode}

% Include the two chapters:
%    \begin{macrocode}
\include{cdocsch1}
\include{cdocsch2}
%    \end{macrocode}

% Include the two parts unless only chapters should be displayed:
%    \begin{macrocode}
\ifchilddoc\else
\section{part three}
\input{cdocspt3}
\section{part four}
\input{cdocspt4}
\fi
%    \end{macrocode}

% Process as usual until here:
%    \begin{macrocode}
\fi
%    \end{macrocode}

% End of document body:
%    \begin{macrocode}
\end{document}
%    \end{macrocode}
%\iffalse
%</samplemain>
%\fi
%
% %%%%%%%%%%%%%%%%%%%%%%%%%%%%%%%%%%%%%%
% \paragraph{Chapter Include Files.}
%
% The include files are called |cdocsch1.tex| and |cdocsch2.tex|.
%
%\iffalse
%<*samplechap1|samplechap2>
%\fi

% Optional override for |\version| flag:
%    \begin{macrocode}
%%\providecommand{\version}{final}
%    \end{macrocode}

% Include the main document:
%    \begin{macrocode}
\input{childdoc.def}
\childdocof{cdocsamp}
%    \end{macrocode}

%\iffalse
%</samplechap1|samplechap2>
%\fi
%
%\iffalse
%<*samplechap1>
%\fi
% Some text for chapter 1:
%    \begin{macrocode}
\section{one}
some text in chapter one
%    \end{macrocode}

%\iffalse
%</samplechap1>
%\fi
% Some text for chapter 2:
%\iffalse
%<*samplechap2>
%\fi
%    \begin{macrocode}
\section{two}
more text in chapter two
%    \end{macrocode}

%\iffalse
%</samplechap2>
%\fi
%
% %%%%%%%%%%%%%%%%%%%%%%%%%%%%%%%%%%%%%%
% \paragraph{Part Include Files.}
%
% The include files are called |cdocspt3.tex| and |cdocspt4.tex|.
%
%\iffalse
%<*samplepart3|samplepart4>
%\fi

% Optional override for |\version| flag:
%    \begin{macrocode}
%%\providecommand{\version}{final}
%    \end{macrocode}

% Include the main document:
%    \begin{macrocode}
\input{childdoc.def}
\childdocby{cdocsamp}
%    \end{macrocode}

%\iffalse
%</samplepart3|samplepart4>
%\fi
%
%\iffalse
%<*samplepart3>
%\fi
% Some text for part 3:
%    \begin{macrocode}
some text in part three
%    \end{macrocode}

%\iffalse
%</samplepart3>
%\fi
% Some text for part 4:
%\iffalse
%<*samplepart4>
%\fi
%    \begin{macrocode}
more text in part four
%    \end{macrocode}

%\iffalse
%</samplepart4>
%\fi
%
% %%%%%%%%%%%%%%%%%%%%%%%%%%%%%%%%%%%%%%
% \paragraph{Forwarding for a Complete Draft.}
%
% The following forwarding file |cdocsdrf.tex|
% compiles the main document in draft mode:
%\iffalse
%<*sampledraft>
%\fi
%    \begin{macrocode}
\def\version{draft}
\input{childdoc.def}
\childdocforward{cdocsamp}
%    \end{macrocode}

%\iffalse
%</sampledraft>
%\fi
%
% %%%%%%%%%%%%%%%%%%%%%%%%%%%%%%%%%%%%%%
% \paragraph{Forwarding for Final Version of the Chapters.}
%
% The following forwarding files |cdocsfn1.tex| and |cdocsfn2.tex|
% (with identical content)
% compile the final versions of the child documents
% |cdocsch1.tex| and |cdocsch2.tex|, respectively:
%\iffalse
%<*samplefinal>
%\fi
%    \begin{macrocode}
\def\version{final}
\input{childdoc.def}
\childdocforwardprefix[cdocsamp]{cdocsfn}{cdocsch}
%    \end{macrocode}

%\iffalse
%</samplefinal>
%\fi
%
% %%%%%%%%%%%%%%%%%%%%%%%%%%%%%%%%%%%%%%
% \paragraph{Command Line Processing.}
%
% The following three command lines generate the output files
% |cdocscld|, |cdocscl1| and |cdocscl2|
% which should be identical to
% |cdocsdrf|, |cdocsch1| and |cdocsfn2|, respectively:
% \begin{center}
% \begin{tabular}{l}
% |latex -jobname cdocscld \|\\
% |  "\def\version{draft}\input{childdoc.def}\childdocforward{cdocsamp}"|\\
% |latex -jobname cdocscl1 \|\\
% |  "\input{childdoc.def}\childdocforward[cdocsamp]{cdocsch1}"|\\
% |latex -jobname cdocscl2 \|\\
% |  "\def\version{final}\input{childdoc.def}\childdocforward{cdocsch2}"|
% \end{tabular}
% \end{center}
% Note that the trailing backslash on each first line
% merely continues the input to the second line
% (for convenient cut ant paste).
% Furthermore, the command |latex| can be replaced by any
% of its alternative versions such as |pdflatex|.
%
% %%%%%%%%%%%%%%%%%%%%%%%%%%%%%%%%%%%%%%%%%%%%%%%%%%%%%%%%%%%%%%%%%%%%%%%%%%%%%%
% %%%%%%%%%%%%%%%%%%%%%%%%%%%%%%%%%%%%%%%%%%%%%%%%%%%%%%%%%%%%%%%%%%%%%%%%%%%%%%
% \section{Implementation}
%\iffalse
%<*package>
%\fi
%
% This section describes the definitions file |childdoc.def|.

% The definitions cannot be loaded using |\usepackage| or |\RequirePackage|
% which has a mechanism to prevent loading a style file more than once.
% When loading the definitions by means of |\input|
% multiple instances have to be prevented manually:
%\iffalse
%This code needs to be before the `\ProvidesFile' directive
%which is defined at the beginning of this file.
%Therefore it is also placed there and commented out here.
%</package>
%<*discard>
%\fi
%    \begin{macrocode}
\ifdefined\childdocmain\endinput\fi
%    \end{macrocode}
%\iffalse
%</discard>
%<*package>
%\fi
%
% \macro{\ifchilddoc}
% \macro{\ifchilddocmanual}
% The conditional |\ifchilddoc| tells whether a
% child (true) or main (false) document is being compiled.
% The conditional |\ifchilddocmanual| tells whether
% the |\includeonly| mechanism is used (false) or
% the selection of child files must be performed manually (true).
% The definitions initialise to false:
%    \begin{macrocode}
\newif\ifchilddoc
\newif\ifchilddocmanual
%    \end{macrocode}

% \macro{\childdocname}
% \macro{\childdocjob}
% The macro |\childdocname| stores the name of the main document
% to be compiled. The macro |\childdocjob| stores the name of
% the document on which the \LaTeX{} compiler was originally invoked.
% The content of |\jobname| cannot be compared
% to filenames specified in the source due to different catcodes.
% The following code rescans |\jobname|, stores the result
% in |\childdocname| and saves a copy in |\childdocjob|:
%    \begin{macrocode}
\edef\childdocname{\scantokens\expandafter{\jobname\noexpand}}
\let\childdocjob\childdocname
%    \end{macrocode}

% \macro{\childdocdisable}
% The macro |\childdocdisable| prevents the main file
% from being processed more than once.
% At this stage, the main document command |\childdocmain|
% is assumed to be called once again where it should do nothing.
% Any subsequent call to it should prevent
% a secondary processing of the main document
% It overwrites the forwarding commands
% |\childdocof| and |\childdocforward|
% with empty macros to prevent further inclusions of the main document:
%    \begin{macrocode}
\newcommand{\childdocdisable}
{
  \renewcommand{\childdocmain}[1]{\renewcommand{\childdocmain}[1]{\endinput}}
  \renewcommand{\childdocof}[1]{}
  \renewcommand{\childdocby}[2][]{}
  \renewcommand{\childdocforward}[2][]{}
  \renewcommand{\childdocdisable}{}
}
%    \end{macrocode}

% \macro{\childdocmain}
% The macro |\childdocmain| is to be called at the top of the main file
% with nothing or the main filename (without extension) as argument.
% First, it breaks loops.
% If the argument is not empty and does not match |\childdocname|
% (which is set by the first inclusion of |childdoc.def|),
% |\ifchilddoc| is set to true, |\includeonly| is applied to the child file
% and |\jobname| is set to the main file
% (for proper handling of |.aux| files):
%    \begin{macrocode}
\newcommand{\childdocmain}[1]
{
  \childdocdisable\childdocmain{}
  \if?#1?\else
    \begingroup
      \def\childdoctmp{#1}
      \ifx\childdoctmp\childdocname
        \def\childdoctmp{}
      \else
        \def\childdoctmp
        {
          \childdoctrue
          \includeonly{\childdocname}
          \def\childdocjob{#1}
          \def\jobname{#1}
        }
      \fi
      \expandafter
    \endgroup
    \childdoctmp
  \fi
}
%    \end{macrocode}

% \macro{\childdocof}
% The command |\childdocof| redirects
% compilation to the main file |#1|.
%    \begin{macrocode}
\newcommand{\childdocof}[1]
{
  \childdocdisable
  \childdoctrue
  \includeonly{\childdocname}
  \def\jobname{#1}
  \def\childdocjob{#1}
  \input{#1}
}
%    \end{macrocode}

% \macro{\childdocby}
% The command |\childdocby| ....
%    \begin{macrocode}
\newcommand{\childdocby}[2][]
{
  \childdocdisable
  \childdoctrue
  \childdocmanualtrue
  \if?#1?\else
    \def\jobname{#2}
  \fi
  \def\childdocjob{#2}
  \input{#2}
  \endinput
}
%    \end{macrocode}

% \macro{\childdocforward}
% The command |\childdocforward| redirects
% compilation to the main file or
% (if the optional argument is given) a child file.
% Parameters are set as if the main file
% or a child file starting with |\childdocof| was compiled.
% Then compilation is handed over to the main file:
%    \begin{macrocode}
\newcommand{\childdocforward}[2][]
{
  \begingroup
    \if?#1?
      \def\childdoctmp
      {
        \def\childdocname{#2}
        \def\childdocjob{#2}
        \def\jobname{#2}
        \input{#2}
        \endinput
      }
    \else
      \def\childdoctmp
      {
        \childdocdisable
        \def\childdocname{#2}
        \childdoctrue
        \includeonly{#2}
        \def\childdocjob{#1}
        \def\jobname{#1}
        \input{#1}
        \endinput
      }
    \fi
    \expandafter
  \endgroup
  \childdoctmp
}
%    \end{macrocode}

% \macro{\childdocforwardprefix}
% The command |\childdocforwardprefix| redirects
% compilation to the main or a child file by means of a pattern.
% The prefix |#1| in the current filename is replaced by |#2|
% and the suffix of the current filename is kept
% (it is assumed that the filename does not contain the substring `|~~~|'
% which is used as a delimiter).
% Compilation is handed over to the new file by |\childdocforward|:
%    \begin{macrocode}
\newcommand{\childdocforwardprefix}[3][]
{
  \begingroup
    \def\childdocextract #2##1~~~{\def\childdoctmp{\childdocforward[#1]{#3##1}}}
    \expandafter\childdocextract\childdocname~~~
    \expandafter
  \endgroup
  \childdoctmp
}
%    \end{macrocode}

% \macro{\childdoc}
% The deprecated macro |\childdoc| is a legacy version of |\childdocmain|:
%    \begin{macrocode}
\newcommand{\childdoc}{\childdocmain}
%    \end{macrocode}

% \macro{\childdocredirect}
% The deprecated macro |\childdocredirect| is a legacy version
% of |\childdocforward| and |\childdocforwardprefix|:
%    \begin{macrocode}
\newcommand{\childdocredirect}[2][]
{
  \begingroup
    \if?#1?
      \def\childdoctmp{\childdocforward{#2}}
    \else
      \def\childdoctmp{\childdocforwardprefix{#1}{#2}}
    \fi
    \expandafter
  \endgroup
  \childdoctmp
}
%    \end{macrocode}

%\iffalse
%</package>
%\fi
%
\endinput
|\\
|\childdocby{|\textit{main}|}|\\
\end{tabular}
\end{center}
%
Both forms have slightly different effects as described above.
The main file is prepared as usual, see \secref{sec:include}.

%%%%%%%%%%%%%%%%%%%%%%%%%%%%%%%%%%%%%%%%%%%%%%%%%%%%%%%%%%%%%%%%%%%%%%%%%%%%%%%%
\subsection{Legacy Detection}
\label{sec:detection}

The directive |\childdocmain| in the main file can detect
whether the complete document or merely a child is to be compiled
even without using the directive |\childdocof|.
This method is deprecated because it is less robust
and there is no compelling reason to use it;
it is merely provided for backward compatibility
and it may be removed in future versions.

If the detection mechanism is to be used,
it is mandatory to correctly specify
the filename of the main file as the argument of |\childdocmain|:
%
\begin{center}
\begin{tabular}{l}
|% \iffalse
%
% childdoc.dtx Copyright (C) 2017-2018 Niklas Beisert
%
% This work may be distributed and/or modified under the
% conditions of the LaTeX Project Public License, either version 1.3
% of this license or (at your option) any later version.
% The latest version of this license is in
%   http://www.latex-project.org/lppl.txt
% and version 1.3 or later is part of all distributions of LaTeX
% version 2005/12/01 or later.
%
% This work has the LPPL maintenance status `maintained'.
%
% The Current Maintainer of this work is Niklas Beisert.
%
% This work consists of the files childdoc.dtx and childdoc.ins
% and the derived files childdoc.def and cdocsamp.tex with
% cdocsch1.tex, cdocsch2.tex, cdocsdrf.tex, cdocsfn1.tex, cdocsfn2.tex.
%
%<package>\ifdefined\childdocmain\endinput\fi
%<package>\ProvidesFile{childdoc.def}[2018/12/30 v2.0 child document driver]
%<samplemain>\ProvidesFile{cdocsamp.tex}[2018/12/30 v2.0 sample for childdoc]
%<*driver>
%\ProvidesFile{childdoc.drv}[2018/12/30 v2.0 childdoc reference manual file]
\PassOptionsToClass{10pt,a4paper}{article}
\documentclass{ltxdoc}

\usepackage[margin=35mm]{geometry}
\usepackage{hyperref}
\usepackage{hyperxmp}
\usepackage[usenames]{color}

\hypersetup{colorlinks=true}
\hypersetup{pdfstartview=FitH}
\hypersetup{pdfpagemode=UseNone}
\hypersetup{pdfsource={}}
\hypersetup{pdflang={en-UK}}
\hypersetup{pdfcopyright={Copyright 2017-2018 Niklas Beisert.
  This work may be distributed and/or modified under the
  conditions of the LaTeX Project Public License, either version 1.3
  of this license or (at your option) any later version.}}
\hypersetup{pdflicenseurl={http://www.latex-project.org/lppl.txt}}
\hypersetup{pdfcontactaddress={ETH Zurich, ITP, HIT K,
  Wolfgang-Pauli-Strasse 27}}
\hypersetup{pdfcontactpostcode={8093}}
\hypersetup{pdfcontactcity={Zurich}}
\hypersetup{pdfcontactcountry={Switzerland}}
\hypersetup{pdfcontactemail={nbeisert@itp.phys.ethz.ch}}
\hypersetup{pdfcontacturl={http://people.phys.ethz.ch/\xmptilde nbeisert/}}

\newcommand{\secref}[1]{\hyperref[#1]{section \ref*{#1}}}

\parskip1ex
\parindent0pt
\let\olditemize\itemize
\def\itemize{\olditemize\parskip0pt}

\begin{document}

\title{The \textsf{childdoc} Package}
\hypersetup{pdftitle={The childdoc Package}}
\author{Niklas Beisert\\[2ex]
  Institut f\"ur Theoretische Physik\\
  Eidgen\"ossische Technische Hochschule Z\"urich\\
  Wolfgang-Pauli-Strasse 27, 8093 Z\"urich, Switzerland\\[1ex]
  \href{mailto:nbeisert@itp.phys.ethz.ch}
  {\texttt{nbeisert@itp.phys.ethz.ch}}}
\hypersetup{pdfauthor={Niklas Beisert}}
\hypersetup{pdfsubject={Manual for the LaTeX2e Package childdoc}}
\date{30 December 2018, \textsf{v2.0}}
\maketitle

\begin{abstract}\noindent
\textsf{childdoc} is a \LaTeXe{} package
that enables the direct compilation
of document sections included by |\include|
to individual files.
\end{abstract}

\begingroup
\parskip0ex
\tableofcontents
\endgroup

%%%%%%%%%%%%%%%%%%%%%%%%%%%%%%%%%%%%%%%%%%%%%%%%%%%%%%%%%%%%%%%%%%%%%%%%%%%%%%%%
%%%%%%%%%%%%%%%%%%%%%%%%%%%%%%%%%%%%%%%%%%%%%%%%%%%%%%%%%%%%%%%%%%%%%%%%%%%%%%%%
\section{Introduction}

\LaTeX{} provides a mechanism to structure a large document (such as a book)
into a main file and several child files (containing the chapters)
using the |\include| command.
This mechanism is beneficial for documents
which span hundreds of pages in order to
make the source file(s) more manageable.
Moreover, compilation can be restricted to
selected child files by means of the |\includeonly| command.
The latter feature can be used to reduce the compilation time while editing
(this was significantly more useful in the earlier days of \LaTeX{})
or to generate a smaller document which is easier to navigate.
Another application of |\includeonly| is to generate
documents consisting of selected parts of the complete document.

However, there are a few drawbacks of the plain |\include| mechanism:
\begin{itemize}
\item
The child files cannot be compiled on their own,
they can only be compiled via the main file.
A naive editing environment
(such as a text editor with an option
to have the current file processed by \LaTeX)
may require one to switch to the main file before compiling;
attempting to compile the child file produces errors.
\item
The main file must be modified (each time)
to adjust the |\includeonly| command
to the present needs. This easily leaves the main file in a messy state.
\item
The generated document will always carry the filename
of the main document. This is inconvenient if
several child files are to be compiled and
to be kept for distribution.
\end{itemize}

The present package provides a simple interface
to make child files individually compilable by \LaTeX{}.
Compiling a child file then has the same effect as compiling
the main file with an |\includeonly| command
to select the appropriate child.
Moreover the generated document will carry the name of the child
rather than the main file.
This resolves all three above issues.

This feature is meant to make the editing of books,
thesis documents and lecture notes somewhat more convenient.
However, the package can also be used efficiently for
composing a series of documents (such as exercise sheets)
which are typically distributed individually.
It then assists the author in generating the individual documents
(potentially in different versions)
as well as a document containing the collected series.
Another application is in developing style files
or other kinds of included material
where compilation of the style file could redirect
to a sample or test file.

%%%%%%%%%%%%%%%%%%%%%%%%%%%%%%%%%%%%%%%%%%%%%%%%%%%%%%%%%%%%%%%%%%%%%%%%%%%%%%%%
%%%%%%%%%%%%%%%%%%%%%%%%%%%%%%%%%%%%%%%%%%%%%%%%%%%%%%%%%%%%%%%%%%%%%%%%%%%%%%%%
\section{Usage}

First of all, the package \textsf{childdoc} is \emph{not} a standard
\LaTeXe{} |.sty| style file! Therefore it needs to be invoked in
a non-standard way.

%%%%%%%%%%%%%%%%%%%%%%%%%%%%%%%%%%%%%%%%%%%%%%%%%%%%%%%%%%%%%%%%%%%%%%%%%%%%%%%%
\subsection{Included Files}
\label{sec:include}

%%%%%%%%%%%%%%%%%%%%%%%%%%%%%%%%%%%%%%%%
\DescribeMacro{\childdocmain}
To use the package, add the commands
\begin{center}
\begin{tabular}{l}
|\input{childdoc.def}|\\
|\childdocmain{}|\\
\end{tabular}
\end{center}
at the very top of the main \LaTeX{} file,
in particular \emph{before} the |\documentclass| statement!
The argument of |\childdocmain| should be left empty
(but it must be present).

%%%%%%%%%%%%%%%%%%%%%%%%%%%%%%%%%%%%%%%%
\DescribeMacro{\childdocof}
Furthermore, add the commands
\begin{center}
\begin{tabular}{l}
|\input{childdoc.def}|\\
|\childdocof{|\textit{main}|}|\\
\end{tabular}
\end{center}
at the top of every child file \textit{child}
which is included by |\include{|\textit{child}|}|
from within the main file
(or at least for those files to be compiled individually).
The argument \textit{main} must be the filename of the main file.

There are a couple of
considerations in setting up the main and child documents:

%%%%%%%%%%%%%%%%%%%%%%%%%%%%%%%%%%%%%%%%
\paragraph{Restrictions.}

Please note the following restrictions:
\begin{itemize}
\item
|\childdocmain| must be called with one argument \textit{main}
to ensure compatibility with earlier version of the package.
It must either be empty (|\childdocmain{}|)
or precisely match the filename of the main file in which it is specified.
See \secref{sec:detection} for further information.
\item
The filename \textit{main} must be specified without the |.tex| extension.
\item
The filename \textit{main} is case sensitive
(even in case-insensitive file systems)
due to internal string comparison.
\item
The argument \textit{main} should be fully expanded, it cannot be a macro.
\item
Subdirectories and special characters should be avoided in filenames.
\item
The command |\childdocmain{|\textit{main}|}| must be followed by a whitespace.
It should not be followed immediately by another command
or by a comment mark `|%|'.
This is because the \TeX{} parser reads the token immediately following
the argument of |\childdocmain| and puts it
at the beginning of every child section;
however, a white\-space is ignored.
\end{itemize}

%%%%%%%%%%%%%%%%%%%%%%%%%%%%%%%%%%%%%%%%
\paragraph{Content of Main File.}

It is advisable to place all content in the child files included by |\include|.
Any output contained in the main file will appear in all child documents
unless suppressed manually;
it cannot be suppressed automatically by the |\includeonly| directive
and thus should normally be avoided.
A method to include some content in the main file
by means of conditional processing is described in \secref{sec:conditional}.

%%%%%%%%%%%%%%%%%%%%%%%%%%%%%%%%%%%%%%%%
\paragraph{Page Numbering.}

When only a part of the document is compiled,
the appropriate numbering of pages
(as well as other status parameters)
is determined from the |.aux| files.
The latter contain information from previous passes.
However this information needs to propagate through
all intermediate child documents.
Therefore the page numbering in child documents may well
be inconsistent until the complete document is compiled at least once.

A useful (if unconventional) way to always ensure a consistent
page numbering is to restart the numbering in each child document
and denote the pages by `\textit{child}|.|\textit{page}'
where \textit{child} represents the chapter/section number of the child file.
This can be achieved by the command
|\numberwithin{page}{|\textit{child}|}|
of the \textsf{amsmath} package
where \textit{child} can be |chapter| or |section|
depending on the chosen structuring.
Alternatively, one can modify the macro |\thepage| appropriately
and reset the counter |page| at the start of each child file.

%%%%%%%%%%%%%%%%%%%%%%%%%%%%%%%%%%%%%%%%%%%%%%%%%%%%%%%%%%%%%%%%%%%%%%%%%%%%%%%%
\subsection{Conditional Processing}
\label{sec:conditional}

The package provides a mechanism to compile different versions
of a document. To customise the versions further some conditional processing
can come in handy to distinguish which version is being compiled.
The package provides two macros to describe the compilation context:

%%%%%%%%%%%%%%%%%%%%%%%%%%%%%%%%%%%%%%%%
\DescribeMacro{\ifchilddoc}
The conditional |\ifchilddoc| distinguishes between the compilation of
child documents and the main document:
%
\begin{center}
|\ifchilddoc |\textit{child-code}| |[|\||else |\textit{main-code}]| \||fi|
\end{center}

%%%%%%%%%%%%%%%%%%%%%%%%%%%%%%%%%%%%%%%%
\DescribeMacro{\childdocname}
\DescribeMacro{\childdocjob}
The macro |\childdocname| contains the filename (without extension)
of the main or child file being processed.
Note that |\childdocjob| will always contain the name of the main file.

%%%%%%%%%%%%%%%%%%%%%%%%%%%%%%%%%%%%%%%%
\paragraph{Title Page.}

Conditional processing can be used to include a title or banner page
in the main document when proper precautions are taken.
Importantly, the code in the main file should ensure that the page counter
(as well as other status parameters which are stored in the |.aux| files)
takes the same value after the conditional processing.
Otherwise the page numbers may take divergent values
depending on which part is compiled.

For example, a title page could be declared by:
%
\begin{center}
\begin{tabular}{l}
|\ifchilddoc\||else|\\
|\addtocounter{page}{-1}|\\
\textit{code for title page}\\
|\newpage|\\
|\||fi|
\end{tabular}
\end{center}
%
A banner page for the child documents can be generated by:
%
\begin{center}
\begin{tabular}{l}
|\ifchilddoc|\\
|\addtocounter{page}{-1}|\\
\textit{code for banner page}\\
|\newpage|\\
|\||fi|
\end{tabular}
\end{center}
%
Here one could write a message such as:
\begin{center}
|This is the part \childdocname{} of \childdocjob{}.|
\end{center}

%%%%%%%%%%%%%%%%%%%%%%%%%%%%%%%%%%%%%%%%%%%%%%%%%%%%%%%%%%%%%%%%%%%%%%%%%%%%%%%%
\subsection{Flags}
\label{sec:flags}

The package makes it easy to generate different versions
of the main or child documents.
To this end compilation flags can be defined
and assigned different default values.
They will be particularly useful in conjunction
with the forwarding mechanism described in \secref{sec:forward}.

For example, it may be useful to have a flag |\version|
which can be set to |draft| or |final|.
The document source will contain some conditional code
depending on the value of |\version|.
Suppose further, the flag should default to |final| for the main file
and to |draft| for child files
which is a natural assignment for editing the document.
This is achieved by placing the following code
in the preamble of the main document
(below the |\childdocmain| directive):
%
\begin{center}
\begin{tabular}{l}
|\ifchilddoc|\\
|\providecommand{\version}{draft}|\\
|\||else|\\
|\providecommand{\version}{final}|\\
|\||fi|
\end{tabular}
\end{center}
%
The definition by |\providecommand| makes sure
that previous definitions are not overwritten.
Further statements |\providecommand{\version}{...}|
can thus be added before the above code to override it.

For the main file, one might add a line
(between |\childdocmain| and the above block)
%
\begin{center}
|%\ifchilddoc\||else\providecommand{\version}{draft}\||fi|
\end{center}
%
which can be uncommented to produce a draft version.
Likewise one can add a line to the very top of a child file
(above the |\childdocof{|\textit{main}|}| directive)
%
\begin{center}
|%\providecommand{\version}{final}|
\end{center}
%
which can be uncommented to produce the final version of this child document.

%%%%%%%%%%%%%%%%%%%%%%%%%%%%%%%%%%%%%%%%%%%%%%%%%%%%%%%%%%%%%%%%%%%%%%%%%%%%%%%%
\subsection{Forwarding}
\label{sec:forward}

Different versions of the main or child documents
using compilation flags as described in \secref{sec:flags}
can be (permanently) stored in different files
for convenient compilation, viewing and distribution.
To this end, the package defines a command
to pass on compilation to a different file:

%%%%%%%%%%%%%%%%%%%%%%%%%%%%%%%%%%%%%%%%
\DescribeMacro{\childdocforward}
The command |\childdocforward| redirects processing to
another source file:
%
\begin{center}
\begin{tabular}{l}
|\input{childdoc.def}|\\
|\childdocforward[|\textit{main}|]{|\textit{dest}|}|\\
\end{tabular}
\end{center}
%
The argument \textit{dest} is the destination file
(without extension).
It should be the main file or one of the child files.
Note that further \textsf{childdoc} directives
such as |\childdocof| and |\childdocforward|
in the indicated file will be processed in this form.
The optional argument \textit{main}
passes on directly to the main file \textit{main}
while pretending to compile the child \textit{dest}.
This form behaves as if \textit{dest}
issues |\childdocof{|\textit{main}|}| right away,
and no further \textsf{childdoc} directives will be processed.

%%%%%%%%%%%%%%%%%%%%%%%%%%%%%%%%%%%%%%%%
\DescribeMacro{\...prefix}
In the alternative form |\childdocforwardprefix|,
%
\begin{center}
\begin{tabular}{l}
|\input{childdoc.def}|\\
|\childdocforwardprefix[|\textit{main}|]{|\textit{prefix}|}{|\textit{dest}|}|
\end{tabular}
\end{center}
%
the destination file is determined by a pattern
depending on the current file:
To make this work, the current file must be called
`{\textit{prefix}\hspace{0.2em}\textit{suffix}}'
with \textit{prefix} matching precisely the argument.
Processing is then passed on to the file
`{\textit{dest}\hspace{0.2em}\textit{suffix}}'.
Surely, the same effect is achieved by
directly specifying the
argument `{\textit{dest}\hspace{0.2em}\textit{suffix}}'
in the first form.
However, that requires to set up a different file
for each child. With the alternative form of the command
all these files can have exactly the same content
which simplifies setting them up and maintaining them.

For example, the following file |draft.tex|
with a compilation flag |\version| as described in \secref{sec:flags}
compiles the main document as a draft:
%
\begin{center}
\begin{tabular}{l}
|\def\version{draft}|\\
|\input{childdoc.def}|\\
|\childdocforward{|\textit{main}|}|
\end{tabular}
\end{center}
%
Likewise, the following files |final|\textit{nn}|.tex|
compile the final version of the child document
|child|\textit{nn}|.tex|:
%
\begin{center}
\begin{tabular}{l}
|\def\version{final}|\\
|\input{childdoc.def}|\\
|\childdocforwardprefix{final}{child}|
\end{tabular}
\end{center}
%

Note that when several versions of a main file and/or of each child file
are to be generated, it may be convenient to set up a |Makefile| or
shell script to automatise the process.

%%%%%%%%%%%%%%%%%%%%%%%%%%%%%%%%%%%%%%%%%%%%%%%%%%%%%%%%%%%%%%%%%%%%%%%%%%%%%%%%
\subsection{Command Line Processing}
\label{sec:commandline}

The effect of redirection files can also be achieved by invoking
the \LaTeX{} compiler with a more elaborate command line.
Most conveniently this should be done as part
of a shell script or a |Makefile|.

When using \textsf{childdoc} in the main file, the following
command lines effectively perform a redirection
(note that depending on the shell being used,
backslashes may have to be doubled: `|\|' $\to$ `|\\|'):
%
\begin{center}
|... -jobname "|\textit{target}|" |\\|"|[\textit{flags}]%
|\input{childdoc.def}\childdocforward[|\textit{main}|]{|\textit{dest}|}"|
\end{center}
%
Here \textit{target} is the name of the output file,
\textit{main} is the name of the main file
and \textit{dest} is the name of the main or child file to be processed
(all filenames without extensions).
The optional argument \textit{main} can be omitted
if \textit{main} matches \textit{dest}.
Optionally, compilation \textit{flags} can be defined via |\def| commands.
This command line makes the \TeX{} engine believe
it is compiling the file \textit{target}
whose content is specified as the latter parameter.
The provided code then forwards the processing to
\textit{main} or \textit{dest} as described in \secref{sec:forward}.

%%%%%%%%%%%%%%%%%%%%%%%%%%%%%%%%%%%%%%%%%%%%%%%%%%%%%%%%%%%%%%%%%%%%%%%%%%%%%%%%
\subsection{Include by Input}
\label{sec:input}

Including child documents by |\include| has some restrictions by design.
Most notably, the content of a child document always occupies
its own set of pages; pages cannot be shared between child documents.
Usually, this behaviour makes perfect sense
because each child document contain an essential part of the document.
However, in some situations it may be desirable to compose
a document from a collection of parts
without having mandatory page breaks between then.
For this case, the package
provides a mechanism to include parts
by |\input| which can also be processed individually.
However, by construction this mechanism
requires manual handling of the content to be output.

%%%%%%%%%%%%%%%%%%%%%%%%%%%%%%%%%%%%%%%%
\DescribeMacro{\ifchilddocmanual}
The main file should be prepared as usual, see \secref{sec:include}.
However, the document body must make a distinction
between processing of an individual part and of the main document, e.g.:
%
\begin{center}
\begin{tabular}{l}
|\ifchilddocmanual|\\
|\input{\childdocname}|\\
|\||else|\\
\textit{document body with }|\input{|\textit{part}|}|\\
|\||fi|
\end{tabular}
\end{center}
%
The conditional |\ifchilddocmanual| is true whenever
a part to be included by |\input| is being compiled,
and the name of the part is stored in |\childdocname|.

%%%%%%%%%%%%%%%%%%%%%%%%%%%%%%%%%%%%%%%%
\DescribeMacro{\childdocby}
Each part to be included by |\input| should start with:
%
\begin{center}
\begin{tabular}{l}
|\input{childdoc.def}|\\
|\childdocby{|\textit{main}|}|\\
\end{tabular}
\end{center}
%
The directive |\childdocby| is similar to |\childdocof|
described in \secref{sec:include},
but the subsequent selection of content must be done manually.
To that end, both |\ifchilddoc| and |\ifchilddocmanual|
will be true upon processing of a part,
and the name of the part is stored in |\childdocname|.
Note that |\jobname| will be set to the filename of the current part
so that each part receives an individual |.aux| file
that does not interfere with the |.aux| file(s) of the main document.
This behaviour can be altered by the alternative form
|\childdocby[*]{|\textit{main}|}| (with a non-empty optional argument)
which uses the |.aux| file of the main document
by setting |\jobname| to \textit{main}.

%%%%%%%%%%%%%%%%%%%%%%%%%%%%%%%%%%%%%%%%%%%%%%%%%%%%%%%%%%%%%%%%%%%%%%%%%%%%%%%%
\subsection{Driver Development}
\label{sec:driver}

The \textsf{childdoc} mechanism can also be use for the development
of definition files such as \LaTeX{} styles or classes.
This case differs from the above setup with multiple parts
included by |\include| in that no |\includeonly| should be invoked.
This can be achieved by starting the include file
(before |\ProvidesPackage|) with:
%
\begin{center}
\begin{tabular}{l}
|\input{childdoc.def}|\\
|\childdocforward{|\textit{main}|}|\\
\end{tabular}
\end{center}
%
or alternatively with:
%
\begin{center}
\begin{tabular}{l}
|\input{childdoc.def}|\\
|\childdocby{|\textit{main}|}|\\
\end{tabular}
\end{center}
%
Both forms have slightly different effects as described above.
The main file is prepared as usual, see \secref{sec:include}.

%%%%%%%%%%%%%%%%%%%%%%%%%%%%%%%%%%%%%%%%%%%%%%%%%%%%%%%%%%%%%%%%%%%%%%%%%%%%%%%%
\subsection{Legacy Detection}
\label{sec:detection}

The directive |\childdocmain| in the main file can detect
whether the complete document or merely a child is to be compiled
even without using the directive |\childdocof|.
This method is deprecated because it is less robust
and there is no compelling reason to use it;
it is merely provided for backward compatibility
and it may be removed in future versions.

If the detection mechanism is to be used,
it is mandatory to correctly specify
the filename of the main file as the argument of |\childdocmain|:
%
\begin{center}
\begin{tabular}{l}
|\input{childdoc.def}|\\
|\childdocmain{|\textit{main}|}|\\
\end{tabular}
\end{center}
%
If |\jobname| does not match the argument \textit{main} of |\childdocmain|,
it is assumed that |\jobname| points to the child file to be compiled.
When using |\childdocmain| with the main file specified as argument,
it suffices to start a child file
with just |\input{|\textit{main}|}|
without loading of the package and using |\childdocof|.
If instead all processing is done
with the appropriate \textsf{childdoc} directives,
the argument of \textit{main} of |\childdocmain| can be empty.

An alternative version of the command line processing described
in \secref{sec:commandline} using the detection mechanism reads:
%
\begin{center}
|... -jobname "|\textit{target}|" "|[\textit{flags}]%
[|\def\jobname{|\textit{dest}|}|]|\input{|\textit{main}|}"|
\end{center}

%%%%%%%%%%%%%%%%%%%%%%%%%%%%%%%%%%%%%%%%%%%%%%%%%%%%%%%%%%%%%%%%%%%%%%%%%%%%%%%%
\subsection{Manual Code}
\label{sec:manual}

In case one cannot be certain whether the definitions file |childdoc.def|
is installed on the target \TeX{} distribution
and one prefers not to ship it,
it is conceivable to paste a few relevant commands into the sources.

To that end, drop all statements |\input{childdoc.def}|
and perform the replacements as outlined below.
Instead of |\childdocmain{|\textit{main}|}| add the following code
to the top of the main file:
%
\begin{center}
\begin{tabular}{l}
|\||ifdefined\childdocname\endinput\||fi\newif\ifchilddoc|\\
|\edef\childdocname{\scantokens\expandafter{\jobname\noexpand}}|\\
|\def\childdocmain{|\textit{main}|}\||ifx\childdocmain\childdocname\||else|\\
|\childdoctrue\includeonly{\childdocname}\let\jobname\childdocmain\||fi|\\
\end{tabular}
\end{center}
%
Instead of |\childdocof{|\textit{main}|}| just include the main file
at the top of each child file:
%
\begin{center}
|\input{|\textit{main}|}|
\end{center}
%
A simple redirection |\childdocforward{|\textit{dest}|}| is achieved by:
%
\begin{center}
|\def\jobname{|\textit{dest}|}\input{\jobname}|
\end{center}
%
The redirection with prefix
|\childdocforwardprefix[|\textit{prefix}|]{|\textit{dest}|}|
is accomplished by:
%
\begin{center}
\begin{tabular}{l}
|{\edef\jobname{\scantokens\expandafter{\jobname\noexpand}}|\\
|\def\redirectjob |\textit{prefix}|#1~~~{\gdef\jobname{|\textit{dest}|#1}}|\\
|\expandafter\redirectjob\jobname~~~}\input{\jobname}|
\end{tabular}
\end{center}

In an alternative approach,
child documents can be compiled by a specific command line
without additional code or specific definitions:
%
\begin{center}
|... -jobname "|\textit{target}|" "|[\textit{flags}]%
|\includeonly{|\textit{dest}|}\input{|\textit{main}|}"|
\end{center}
%

%%%%%%%%%%%%%%%%%%%%%%%%%%%%%%%%%%%%%%%%%%%%%%%%%%%%%%%%%%%%%%%%%%%%%%%%%%%%%%%%
%%%%%%%%%%%%%%%%%%%%%%%%%%%%%%%%%%%%%%%%%%%%%%%%%%%%%%%%%%%%%%%%%%%%%%%%%%%%%%%%
\section{Information}

%%%%%%%%%%%%%%%%%%%%%%%%%%%%%%%%%%%%%%%%%%%%%%%%%%%%%%%%%%%%%%%%%%%%%%%%%%%%%%%%
\subsection{Copyright}

Copyright \copyright{} 2017--2018 Niklas Beisert

This work may be distributed and/or modified under the
conditions of the \LaTeX{} Project Public License, either version 1.3
of this license or (at your option) any later version.
The latest version of this license is in
  \url{http://www.latex-project.org/lppl.txt}
and version 1.3 or later is part of all distributions of \LaTeX{}
version 2005/12/01 or later.

This work has the LPPL maintenance status `maintained'.

The Current Maintainer of this work is Niklas Beisert.

This work consists of the files |README.txt|, |childdoc.ins| and |childdoc.dtx|
as well as the derived files |childdoc.def|, |cdocsamp.tex|
with |cdocsch1.tex|, |cdocsch2.tex|, |cdocspt3.tex|, |cdocspt4.tex|,
|cdocsdrf.tex|, |cdocsfn1.tex|, |cdocsfn2.tex|
as well as |childdoc.pdf|.

%%%%%%%%%%%%%%%%%%%%%%%%%%%%%%%%%%%%%%%%%%%%%%%%%%%%%%%%%%%%%%%%%%%%%%%%%%%%%%%%
\subsection{Files and Installation}

The package consists of the files:
%
\begin{center}
\begin{tabular}{ll}
    |README.txt|   & readme file \\
    |childdoc.ins| & installation file \\
    |childdoc.dtx| & source file \\
    |childdoc.def| & definition file \\
    |cdocsamp.tex| & sample main file \\
    |cdocsch1.tex| & sample include file \\
    |cdocsch2.tex| & sample include file \\
    |cdocspt3.tex| & sample part file \\
    |cdocspt4.tex| & sample part file \\
    |cdocsdrf.tex| & sample redirection file \\
    |cdocsfn1.tex| & sample redirection file \\
    |cdocsfn2.tex| & sample redirection file \\
    |childdoc.pdf| & manual
\end{tabular}
\end{center}
%
The distribution consists of the files
|README.txt|, |childdoc.ins| and |childdoc.dtx|.
%
\begin{itemize}
\item
Run (pdf)\LaTeX{} on |childdoc.dtx|
to compile the manual |childdoc.pdf| (this file).
\item
Run \LaTeX{} on |childdoc.ins| to create the definitions file |childdoc.def|
and the sample |cdocsamp.tex| with include files
|cdocsch1.tex|, |cdocsch2.tex|, |cdocspt3.tex|, |cdocspt4.tex|,
|cdocsdrf.tex|, |cdocsfn1.tex|, |cdocsfn2.tex|.
Then copy the file |childdoc.def| to an appropriate directory of your \LaTeX{}
distribution, e.g.\ \textit{texmf-root}|/tex/latex/childdoc|.
\end{itemize}

%%%%%%%%%%%%%%%%%%%%%%%%%%%%%%%%%%%%%%%%%%%%%%%%%%%%%%%%%%%%%%%%%%%%%%%%%%%%%%%%
\subsection{Related CTAN Packages}

There are several other packages which offer a similar functionality:
%
\begin{itemize}
\item
The packages
\href{http://ctan.org/pkg/docmute}{\textsf{docmute}},
\href{http://ctan.org/pkg/includex}{\textsf{includex}} and
\href{http://ctan.org/pkg/standalone}{\textsf{standalone}}
provide commands to include only the document body of
a child file thus allowing both files to be compiled individually.
\item
The packages \href{http://ctan.org/pkg/subdocs}{\textsf{subdocs}}
and \href{http://ctan.org/pkg/subfiles}{\textsf{subfiles}}
provide structures in which the main and child documents can be
encapsulated and allowing them to be compiled individually.
The inclusion mechanism is different from the conventional |\include|.
\item
The package \href{http://ctan.org/pkg/combine}{\textsf{combine}}
is an elaborate solution to combine several documents into one.
\end{itemize}
%
See also the CTAN topic \href{http://ctan.org/topic/subdocs}{\textsf{subdocs}}
for further related packages.
The present package differs from the above solutions in that
a document structure constructed with the conventional |\include| mechanism
just needs two extra commands at the top of every file
such that all constituent files can be compiled individually.

%%%%%%%%%%%%%%%%%%%%%%%%%%%%%%%%%%%%%%%%%%%%%%%%%%%%%%%%%%%%%%%%%%%%%%%%%%%%%%%%
%\subsection{Feature Suggestions}
%
%The following is a list of features which may be useful for future
%versions of this package:
%%
%\begin{itemize}
%\item
%\ldots
%\end{itemize}

%%%%%%%%%%%%%%%%%%%%%%%%%%%%%%%%%%%%%%%%%%%%%%%%%%%%%%%%%%%%%%%%%%%%%%%%%%%%%%%%
\subsection{Revision History}

%%%%%%%%%%%%%%%%%%%%%%%%%%%%%%%%%%%%%%%%
\paragraph{v2.0:} 2018/12/30

\begin{itemize}
\item
immediate forward processing
\item
added |\childdocby| mechanism
\item
manual restructured
\end{itemize}

%%%%%%%%%%%%%%%%%%%%%%%%%%%%%%%%%%%%%%%%
\paragraph{v1.6:} 2018/01/17

\begin{itemize}
\item
application for development of include files
\item
corrections to manual
\end{itemize}

%%%%%%%%%%%%%%%%%%%%%%%%%%%%%%%%%%%%%%%%
\paragraph{v1.5:} 2017/05/21

\begin{itemize}
\item
more complete structuring introduced
\item
|\childdocof| introduced
\item
|\childdoc| renamed to |\childdocmain|
\item
|\childredirect| renamed to |\childdocforward| and |\childdocforwardprefix|
and functionality expanded
\end{itemize}

%%%%%%%%%%%%%%%%%%%%%%%%%%%%%%%%%%%%%%%%
\paragraph{v1.0:} 2017/04/27

\begin{itemize}
\item
manual and install package
\item
first version published on CTAN
\end{itemize}

%%%%%%%%%%%%%%%%%%%%%%%%%%%%%%%%%%%%%%%%
\paragraph{v0.6:} 2017/04/26

\begin{itemize}
\item
redirection mechanism added
\end{itemize}

%%%%%%%%%%%%%%%%%%%%%%%%%%%%%%%%%%%%%%%%
\paragraph{v0.5:} 2017/04/26

\begin{itemize}
\item
functionality in definition file
\end{itemize}


%%%%%%%%%%%%%%%%%%%%%%%%%%%%%%%%%%%%%%%%%%%%%%%%%%%%%%%%%%%%%%%%%%%%%%%%%%%%%%%%
%%%%%%%%%%%%%%%%%%%%%%%%%%%%%%%%%%%%%%%%%%%%%%%%%%%%%%%%%%%%%%%%%%%%%%%%%%%%%%%%
%%%%%%%%%%%%%%%%%%%%%%%%%%%%%%%%%%%%%%%%%%%%%%%%%%%%%%%%%%%%%%%%%%%%%%%%%%%%%%%%
\appendix

\settowidth\MacroIndent{\rmfamily\scriptsize 000\ }

 \DocInput{childdoc.dtx}

\end{document}
%</driver>
% \fi
%
% %%%%%%%%%%%%%%%%%%%%%%%%%%%%%%%%%%%%%%%%%%%%%%%%%%%%%%%%%%%%%%%%%%%%%%%%%%%%%%
% %%%%%%%%%%%%%%%%%%%%%%%%%%%%%%%%%%%%%%%%%%%%%%%%%%%%%%%%%%%%%%%%%%%%%%%%%%%%%%
% \section{Sample}
%\iffalse
%<*samplemain>
%\fi
%
% The following presents a sample document
% with two chapters, two parts, a title page,
% a compile flag as well as three forwarding files to set the flag.
% It consists of eight |.tex| files:
% \begin{center}
% \begin{tabular}{ll}
% |cdocsamp.tex|&main file\\
% |cdocsch1.tex|&include file for chapter 1\\
% |cdocsch2.tex|&include file for chapter 2\\
% |cdocspt3.tex|&include file for part 3\\
% |cdocspt4.tex|&include file for part 4\\
% |cdocsdrf.tex|&forwarding file for main file in draft mode\\
% |cdocsfi1.tex|&forwarding file for final version of chapter 1\\
% |cdocsfi2.tex|&forwarding file for final version of chapter 2\\
% \end{tabular}
% \end{center}
% Each of the eight files can be compiled directly by the \LaTeX{} compiler.
%
% %%%%%%%%%%%%%%%%%%%%%%%%%%%%%%%%%%%%%%
% \paragraph{Main File.}
%
% The main file is called |cdocsamp.tex|.
%
% Load the \textsf{childdoc} definitions and
% declare the filename for the main document:
%    \begin{macrocode}
\input{childdoc.def}
\childdocmain{}
%    \end{macrocode}

% Optional override for |\version| flag:
%    \begin{macrocode}
%%\ifchilddoc\else\providecommand{\version}{draft}\fi
%    \end{macrocode}

% Define the default values for the |\version| flag
% (|final| for the main file and |draft| for childs):
%    \begin{macrocode}
\ifchilddoc
\providecommand{\version}{draft}
\else
\providecommand{\version}{final}
\fi
%    \end{macrocode}

% Load the standard document class:
%    \begin{macrocode}
\documentclass[12pt]{article}
%    \end{macrocode}

% Start the document body:
%    \begin{macrocode}
\begin{document}
%    \end{macrocode}

% Declare a title page.
% Print title, part of document being processed and version flag:
%    \begin{macrocode}
\addtocounter{page}{-1}
\begin{center}
{\LARGE\bfseries{}childdoc example\par}
\vspace{1cm}
\ifchilddoc
\ifchilddocmanual part\else chapter\fi:
`\childdocname' of `\childdocjob'\par
\else
main document: `\childdocjob'\par
\fi
version: \version\par
\end{center}
\newpage
%    \end{macrocode}

% Manually include selected file,
% otherwise process as usual:
%    \begin{macrocode}
\ifchilddocmanual
\section*{part `\childdocname'}
\input{\childdocname}
\else
%    \end{macrocode}

% Include the two chapters:
%    \begin{macrocode}
\include{cdocsch1}
\include{cdocsch2}
%    \end{macrocode}

% Include the two parts unless only chapters should be displayed:
%    \begin{macrocode}
\ifchilddoc\else
\section{part three}
\input{cdocspt3}
\section{part four}
\input{cdocspt4}
\fi
%    \end{macrocode}

% Process as usual until here:
%    \begin{macrocode}
\fi
%    \end{macrocode}

% End of document body:
%    \begin{macrocode}
\end{document}
%    \end{macrocode}
%\iffalse
%</samplemain>
%\fi
%
% %%%%%%%%%%%%%%%%%%%%%%%%%%%%%%%%%%%%%%
% \paragraph{Chapter Include Files.}
%
% The include files are called |cdocsch1.tex| and |cdocsch2.tex|.
%
%\iffalse
%<*samplechap1|samplechap2>
%\fi

% Optional override for |\version| flag:
%    \begin{macrocode}
%%\providecommand{\version}{final}
%    \end{macrocode}

% Include the main document:
%    \begin{macrocode}
\input{childdoc.def}
\childdocof{cdocsamp}
%    \end{macrocode}

%\iffalse
%</samplechap1|samplechap2>
%\fi
%
%\iffalse
%<*samplechap1>
%\fi
% Some text for chapter 1:
%    \begin{macrocode}
\section{one}
some text in chapter one
%    \end{macrocode}

%\iffalse
%</samplechap1>
%\fi
% Some text for chapter 2:
%\iffalse
%<*samplechap2>
%\fi
%    \begin{macrocode}
\section{two}
more text in chapter two
%    \end{macrocode}

%\iffalse
%</samplechap2>
%\fi
%
% %%%%%%%%%%%%%%%%%%%%%%%%%%%%%%%%%%%%%%
% \paragraph{Part Include Files.}
%
% The include files are called |cdocspt3.tex| and |cdocspt4.tex|.
%
%\iffalse
%<*samplepart3|samplepart4>
%\fi

% Optional override for |\version| flag:
%    \begin{macrocode}
%%\providecommand{\version}{final}
%    \end{macrocode}

% Include the main document:
%    \begin{macrocode}
\input{childdoc.def}
\childdocby{cdocsamp}
%    \end{macrocode}

%\iffalse
%</samplepart3|samplepart4>
%\fi
%
%\iffalse
%<*samplepart3>
%\fi
% Some text for part 3:
%    \begin{macrocode}
some text in part three
%    \end{macrocode}

%\iffalse
%</samplepart3>
%\fi
% Some text for part 4:
%\iffalse
%<*samplepart4>
%\fi
%    \begin{macrocode}
more text in part four
%    \end{macrocode}

%\iffalse
%</samplepart4>
%\fi
%
% %%%%%%%%%%%%%%%%%%%%%%%%%%%%%%%%%%%%%%
% \paragraph{Forwarding for a Complete Draft.}
%
% The following forwarding file |cdocsdrf.tex|
% compiles the main document in draft mode:
%\iffalse
%<*sampledraft>
%\fi
%    \begin{macrocode}
\def\version{draft}
\input{childdoc.def}
\childdocforward{cdocsamp}
%    \end{macrocode}

%\iffalse
%</sampledraft>
%\fi
%
% %%%%%%%%%%%%%%%%%%%%%%%%%%%%%%%%%%%%%%
% \paragraph{Forwarding for Final Version of the Chapters.}
%
% The following forwarding files |cdocsfn1.tex| and |cdocsfn2.tex|
% (with identical content)
% compile the final versions of the child documents
% |cdocsch1.tex| and |cdocsch2.tex|, respectively:
%\iffalse
%<*samplefinal>
%\fi
%    \begin{macrocode}
\def\version{final}
\input{childdoc.def}
\childdocforwardprefix[cdocsamp]{cdocsfn}{cdocsch}
%    \end{macrocode}

%\iffalse
%</samplefinal>
%\fi
%
% %%%%%%%%%%%%%%%%%%%%%%%%%%%%%%%%%%%%%%
% \paragraph{Command Line Processing.}
%
% The following three command lines generate the output files
% |cdocscld|, |cdocscl1| and |cdocscl2|
% which should be identical to
% |cdocsdrf|, |cdocsch1| and |cdocsfn2|, respectively:
% \begin{center}
% \begin{tabular}{l}
% |latex -jobname cdocscld \|\\
% |  "\def\version{draft}\input{childdoc.def}\childdocforward{cdocsamp}"|\\
% |latex -jobname cdocscl1 \|\\
% |  "\input{childdoc.def}\childdocforward[cdocsamp]{cdocsch1}"|\\
% |latex -jobname cdocscl2 \|\\
% |  "\def\version{final}\input{childdoc.def}\childdocforward{cdocsch2}"|
% \end{tabular}
% \end{center}
% Note that the trailing backslash on each first line
% merely continues the input to the second line
% (for convenient cut ant paste).
% Furthermore, the command |latex| can be replaced by any
% of its alternative versions such as |pdflatex|.
%
% %%%%%%%%%%%%%%%%%%%%%%%%%%%%%%%%%%%%%%%%%%%%%%%%%%%%%%%%%%%%%%%%%%%%%%%%%%%%%%
% %%%%%%%%%%%%%%%%%%%%%%%%%%%%%%%%%%%%%%%%%%%%%%%%%%%%%%%%%%%%%%%%%%%%%%%%%%%%%%
% \section{Implementation}
%\iffalse
%<*package>
%\fi
%
% This section describes the definitions file |childdoc.def|.

% The definitions cannot be loaded using |\usepackage| or |\RequirePackage|
% which has a mechanism to prevent loading a style file more than once.
% When loading the definitions by means of |\input|
% multiple instances have to be prevented manually:
%\iffalse
%This code needs to be before the `\ProvidesFile' directive
%which is defined at the beginning of this file.
%Therefore it is also placed there and commented out here.
%</package>
%<*discard>
%\fi
%    \begin{macrocode}
\ifdefined\childdocmain\endinput\fi
%    \end{macrocode}
%\iffalse
%</discard>
%<*package>
%\fi
%
% \macro{\ifchilddoc}
% \macro{\ifchilddocmanual}
% The conditional |\ifchilddoc| tells whether a
% child (true) or main (false) document is being compiled.
% The conditional |\ifchilddocmanual| tells whether
% the |\includeonly| mechanism is used (false) or
% the selection of child files must be performed manually (true).
% The definitions initialise to false:
%    \begin{macrocode}
\newif\ifchilddoc
\newif\ifchilddocmanual
%    \end{macrocode}

% \macro{\childdocname}
% \macro{\childdocjob}
% The macro |\childdocname| stores the name of the main document
% to be compiled. The macro |\childdocjob| stores the name of
% the document on which the \LaTeX{} compiler was originally invoked.
% The content of |\jobname| cannot be compared
% to filenames specified in the source due to different catcodes.
% The following code rescans |\jobname|, stores the result
% in |\childdocname| and saves a copy in |\childdocjob|:
%    \begin{macrocode}
\edef\childdocname{\scantokens\expandafter{\jobname\noexpand}}
\let\childdocjob\childdocname
%    \end{macrocode}

% \macro{\childdocdisable}
% The macro |\childdocdisable| prevents the main file
% from being processed more than once.
% At this stage, the main document command |\childdocmain|
% is assumed to be called once again where it should do nothing.
% Any subsequent call to it should prevent
% a secondary processing of the main document
% It overwrites the forwarding commands
% |\childdocof| and |\childdocforward|
% with empty macros to prevent further inclusions of the main document:
%    \begin{macrocode}
\newcommand{\childdocdisable}
{
  \renewcommand{\childdocmain}[1]{\renewcommand{\childdocmain}[1]{\endinput}}
  \renewcommand{\childdocof}[1]{}
  \renewcommand{\childdocby}[2][]{}
  \renewcommand{\childdocforward}[2][]{}
  \renewcommand{\childdocdisable}{}
}
%    \end{macrocode}

% \macro{\childdocmain}
% The macro |\childdocmain| is to be called at the top of the main file
% with nothing or the main filename (without extension) as argument.
% First, it breaks loops.
% If the argument is not empty and does not match |\childdocname|
% (which is set by the first inclusion of |childdoc.def|),
% |\ifchilddoc| is set to true, |\includeonly| is applied to the child file
% and |\jobname| is set to the main file
% (for proper handling of |.aux| files):
%    \begin{macrocode}
\newcommand{\childdocmain}[1]
{
  \childdocdisable\childdocmain{}
  \if?#1?\else
    \begingroup
      \def\childdoctmp{#1}
      \ifx\childdoctmp\childdocname
        \def\childdoctmp{}
      \else
        \def\childdoctmp
        {
          \childdoctrue
          \includeonly{\childdocname}
          \def\childdocjob{#1}
          \def\jobname{#1}
        }
      \fi
      \expandafter
    \endgroup
    \childdoctmp
  \fi
}
%    \end{macrocode}

% \macro{\childdocof}
% The command |\childdocof| redirects
% compilation to the main file |#1|.
%    \begin{macrocode}
\newcommand{\childdocof}[1]
{
  \childdocdisable
  \childdoctrue
  \includeonly{\childdocname}
  \def\jobname{#1}
  \def\childdocjob{#1}
  \input{#1}
}
%    \end{macrocode}

% \macro{\childdocby}
% The command |\childdocby| ....
%    \begin{macrocode}
\newcommand{\childdocby}[2][]
{
  \childdocdisable
  \childdoctrue
  \childdocmanualtrue
  \if?#1?\else
    \def\jobname{#2}
  \fi
  \def\childdocjob{#2}
  \input{#2}
  \endinput
}
%    \end{macrocode}

% \macro{\childdocforward}
% The command |\childdocforward| redirects
% compilation to the main file or
% (if the optional argument is given) a child file.
% Parameters are set as if the main file
% or a child file starting with |\childdocof| was compiled.
% Then compilation is handed over to the main file:
%    \begin{macrocode}
\newcommand{\childdocforward}[2][]
{
  \begingroup
    \if?#1?
      \def\childdoctmp
      {
        \def\childdocname{#2}
        \def\childdocjob{#2}
        \def\jobname{#2}
        \input{#2}
        \endinput
      }
    \else
      \def\childdoctmp
      {
        \childdocdisable
        \def\childdocname{#2}
        \childdoctrue
        \includeonly{#2}
        \def\childdocjob{#1}
        \def\jobname{#1}
        \input{#1}
        \endinput
      }
    \fi
    \expandafter
  \endgroup
  \childdoctmp
}
%    \end{macrocode}

% \macro{\childdocforwardprefix}
% The command |\childdocforwardprefix| redirects
% compilation to the main or a child file by means of a pattern.
% The prefix |#1| in the current filename is replaced by |#2|
% and the suffix of the current filename is kept
% (it is assumed that the filename does not contain the substring `|~~~|'
% which is used as a delimiter).
% Compilation is handed over to the new file by |\childdocforward|:
%    \begin{macrocode}
\newcommand{\childdocforwardprefix}[3][]
{
  \begingroup
    \def\childdocextract #2##1~~~{\def\childdoctmp{\childdocforward[#1]{#3##1}}}
    \expandafter\childdocextract\childdocname~~~
    \expandafter
  \endgroup
  \childdoctmp
}
%    \end{macrocode}

% \macro{\childdoc}
% The deprecated macro |\childdoc| is a legacy version of |\childdocmain|:
%    \begin{macrocode}
\newcommand{\childdoc}{\childdocmain}
%    \end{macrocode}

% \macro{\childdocredirect}
% The deprecated macro |\childdocredirect| is a legacy version
% of |\childdocforward| and |\childdocforwardprefix|:
%    \begin{macrocode}
\newcommand{\childdocredirect}[2][]
{
  \begingroup
    \if?#1?
      \def\childdoctmp{\childdocforward{#2}}
    \else
      \def\childdoctmp{\childdocforwardprefix{#1}{#2}}
    \fi
    \expandafter
  \endgroup
  \childdoctmp
}
%    \end{macrocode}

%\iffalse
%</package>
%\fi
%
\endinput
|\\
|\childdocmain{|\textit{main}|}|\\
\end{tabular}
\end{center}
%
If |\jobname| does not match the argument \textit{main} of |\childdocmain|,
it is assumed that |\jobname| points to the child file to be compiled.
When using |\childdocmain| with the main file specified as argument,
it suffices to start a child file
with just |\input{|\textit{main}|}|
without loading of the package and using |\childdocof|.
If instead all processing is done
with the appropriate \textsf{childdoc} directives,
the argument of \textit{main} of |\childdocmain| can be empty.

An alternative version of the command line processing described
in \secref{sec:commandline} using the detection mechanism reads:
%
\begin{center}
|... -jobname "|\textit{target}|" "|[\textit{flags}]%
[|\def\jobname{|\textit{dest}|}|]|\input{|\textit{main}|}"|
\end{center}

%%%%%%%%%%%%%%%%%%%%%%%%%%%%%%%%%%%%%%%%%%%%%%%%%%%%%%%%%%%%%%%%%%%%%%%%%%%%%%%%
\subsection{Manual Code}
\label{sec:manual}

In case one cannot be certain whether the definitions file |childdoc.def|
is installed on the target \TeX{} distribution
and one prefers not to ship it,
it is conceivable to paste a few relevant commands into the sources.

To that end, drop all statements |% \iffalse
%
% childdoc.dtx Copyright (C) 2017-2018 Niklas Beisert
%
% This work may be distributed and/or modified under the
% conditions of the LaTeX Project Public License, either version 1.3
% of this license or (at your option) any later version.
% The latest version of this license is in
%   http://www.latex-project.org/lppl.txt
% and version 1.3 or later is part of all distributions of LaTeX
% version 2005/12/01 or later.
%
% This work has the LPPL maintenance status `maintained'.
%
% The Current Maintainer of this work is Niklas Beisert.
%
% This work consists of the files childdoc.dtx and childdoc.ins
% and the derived files childdoc.def and cdocsamp.tex with
% cdocsch1.tex, cdocsch2.tex, cdocsdrf.tex, cdocsfn1.tex, cdocsfn2.tex.
%
%<package>\ifdefined\childdocmain\endinput\fi
%<package>\ProvidesFile{childdoc.def}[2018/12/30 v2.0 child document driver]
%<samplemain>\ProvidesFile{cdocsamp.tex}[2018/12/30 v2.0 sample for childdoc]
%<*driver>
%\ProvidesFile{childdoc.drv}[2018/12/30 v2.0 childdoc reference manual file]
\PassOptionsToClass{10pt,a4paper}{article}
\documentclass{ltxdoc}

\usepackage[margin=35mm]{geometry}
\usepackage{hyperref}
\usepackage{hyperxmp}
\usepackage[usenames]{color}

\hypersetup{colorlinks=true}
\hypersetup{pdfstartview=FitH}
\hypersetup{pdfpagemode=UseNone}
\hypersetup{pdfsource={}}
\hypersetup{pdflang={en-UK}}
\hypersetup{pdfcopyright={Copyright 2017-2018 Niklas Beisert.
  This work may be distributed and/or modified under the
  conditions of the LaTeX Project Public License, either version 1.3
  of this license or (at your option) any later version.}}
\hypersetup{pdflicenseurl={http://www.latex-project.org/lppl.txt}}
\hypersetup{pdfcontactaddress={ETH Zurich, ITP, HIT K,
  Wolfgang-Pauli-Strasse 27}}
\hypersetup{pdfcontactpostcode={8093}}
\hypersetup{pdfcontactcity={Zurich}}
\hypersetup{pdfcontactcountry={Switzerland}}
\hypersetup{pdfcontactemail={nbeisert@itp.phys.ethz.ch}}
\hypersetup{pdfcontacturl={http://people.phys.ethz.ch/\xmptilde nbeisert/}}

\newcommand{\secref}[1]{\hyperref[#1]{section \ref*{#1}}}

\parskip1ex
\parindent0pt
\let\olditemize\itemize
\def\itemize{\olditemize\parskip0pt}

\begin{document}

\title{The \textsf{childdoc} Package}
\hypersetup{pdftitle={The childdoc Package}}
\author{Niklas Beisert\\[2ex]
  Institut f\"ur Theoretische Physik\\
  Eidgen\"ossische Technische Hochschule Z\"urich\\
  Wolfgang-Pauli-Strasse 27, 8093 Z\"urich, Switzerland\\[1ex]
  \href{mailto:nbeisert@itp.phys.ethz.ch}
  {\texttt{nbeisert@itp.phys.ethz.ch}}}
\hypersetup{pdfauthor={Niklas Beisert}}
\hypersetup{pdfsubject={Manual for the LaTeX2e Package childdoc}}
\date{30 December 2018, \textsf{v2.0}}
\maketitle

\begin{abstract}\noindent
\textsf{childdoc} is a \LaTeXe{} package
that enables the direct compilation
of document sections included by |\include|
to individual files.
\end{abstract}

\begingroup
\parskip0ex
\tableofcontents
\endgroup

%%%%%%%%%%%%%%%%%%%%%%%%%%%%%%%%%%%%%%%%%%%%%%%%%%%%%%%%%%%%%%%%%%%%%%%%%%%%%%%%
%%%%%%%%%%%%%%%%%%%%%%%%%%%%%%%%%%%%%%%%%%%%%%%%%%%%%%%%%%%%%%%%%%%%%%%%%%%%%%%%
\section{Introduction}

\LaTeX{} provides a mechanism to structure a large document (such as a book)
into a main file and several child files (containing the chapters)
using the |\include| command.
This mechanism is beneficial for documents
which span hundreds of pages in order to
make the source file(s) more manageable.
Moreover, compilation can be restricted to
selected child files by means of the |\includeonly| command.
The latter feature can be used to reduce the compilation time while editing
(this was significantly more useful in the earlier days of \LaTeX{})
or to generate a smaller document which is easier to navigate.
Another application of |\includeonly| is to generate
documents consisting of selected parts of the complete document.

However, there are a few drawbacks of the plain |\include| mechanism:
\begin{itemize}
\item
The child files cannot be compiled on their own,
they can only be compiled via the main file.
A naive editing environment
(such as a text editor with an option
to have the current file processed by \LaTeX)
may require one to switch to the main file before compiling;
attempting to compile the child file produces errors.
\item
The main file must be modified (each time)
to adjust the |\includeonly| command
to the present needs. This easily leaves the main file in a messy state.
\item
The generated document will always carry the filename
of the main document. This is inconvenient if
several child files are to be compiled and
to be kept for distribution.
\end{itemize}

The present package provides a simple interface
to make child files individually compilable by \LaTeX{}.
Compiling a child file then has the same effect as compiling
the main file with an |\includeonly| command
to select the appropriate child.
Moreover the generated document will carry the name of the child
rather than the main file.
This resolves all three above issues.

This feature is meant to make the editing of books,
thesis documents and lecture notes somewhat more convenient.
However, the package can also be used efficiently for
composing a series of documents (such as exercise sheets)
which are typically distributed individually.
It then assists the author in generating the individual documents
(potentially in different versions)
as well as a document containing the collected series.
Another application is in developing style files
or other kinds of included material
where compilation of the style file could redirect
to a sample or test file.

%%%%%%%%%%%%%%%%%%%%%%%%%%%%%%%%%%%%%%%%%%%%%%%%%%%%%%%%%%%%%%%%%%%%%%%%%%%%%%%%
%%%%%%%%%%%%%%%%%%%%%%%%%%%%%%%%%%%%%%%%%%%%%%%%%%%%%%%%%%%%%%%%%%%%%%%%%%%%%%%%
\section{Usage}

First of all, the package \textsf{childdoc} is \emph{not} a standard
\LaTeXe{} |.sty| style file! Therefore it needs to be invoked in
a non-standard way.

%%%%%%%%%%%%%%%%%%%%%%%%%%%%%%%%%%%%%%%%%%%%%%%%%%%%%%%%%%%%%%%%%%%%%%%%%%%%%%%%
\subsection{Included Files}
\label{sec:include}

%%%%%%%%%%%%%%%%%%%%%%%%%%%%%%%%%%%%%%%%
\DescribeMacro{\childdocmain}
To use the package, add the commands
\begin{center}
\begin{tabular}{l}
|\input{childdoc.def}|\\
|\childdocmain{}|\\
\end{tabular}
\end{center}
at the very top of the main \LaTeX{} file,
in particular \emph{before} the |\documentclass| statement!
The argument of |\childdocmain| should be left empty
(but it must be present).

%%%%%%%%%%%%%%%%%%%%%%%%%%%%%%%%%%%%%%%%
\DescribeMacro{\childdocof}
Furthermore, add the commands
\begin{center}
\begin{tabular}{l}
|\input{childdoc.def}|\\
|\childdocof{|\textit{main}|}|\\
\end{tabular}
\end{center}
at the top of every child file \textit{child}
which is included by |\include{|\textit{child}|}|
from within the main file
(or at least for those files to be compiled individually).
The argument \textit{main} must be the filename of the main file.

There are a couple of
considerations in setting up the main and child documents:

%%%%%%%%%%%%%%%%%%%%%%%%%%%%%%%%%%%%%%%%
\paragraph{Restrictions.}

Please note the following restrictions:
\begin{itemize}
\item
|\childdocmain| must be called with one argument \textit{main}
to ensure compatibility with earlier version of the package.
It must either be empty (|\childdocmain{}|)
or precisely match the filename of the main file in which it is specified.
See \secref{sec:detection} for further information.
\item
The filename \textit{main} must be specified without the |.tex| extension.
\item
The filename \textit{main} is case sensitive
(even in case-insensitive file systems)
due to internal string comparison.
\item
The argument \textit{main} should be fully expanded, it cannot be a macro.
\item
Subdirectories and special characters should be avoided in filenames.
\item
The command |\childdocmain{|\textit{main}|}| must be followed by a whitespace.
It should not be followed immediately by another command
or by a comment mark `|%|'.
This is because the \TeX{} parser reads the token immediately following
the argument of |\childdocmain| and puts it
at the beginning of every child section;
however, a white\-space is ignored.
\end{itemize}

%%%%%%%%%%%%%%%%%%%%%%%%%%%%%%%%%%%%%%%%
\paragraph{Content of Main File.}

It is advisable to place all content in the child files included by |\include|.
Any output contained in the main file will appear in all child documents
unless suppressed manually;
it cannot be suppressed automatically by the |\includeonly| directive
and thus should normally be avoided.
A method to include some content in the main file
by means of conditional processing is described in \secref{sec:conditional}.

%%%%%%%%%%%%%%%%%%%%%%%%%%%%%%%%%%%%%%%%
\paragraph{Page Numbering.}

When only a part of the document is compiled,
the appropriate numbering of pages
(as well as other status parameters)
is determined from the |.aux| files.
The latter contain information from previous passes.
However this information needs to propagate through
all intermediate child documents.
Therefore the page numbering in child documents may well
be inconsistent until the complete document is compiled at least once.

A useful (if unconventional) way to always ensure a consistent
page numbering is to restart the numbering in each child document
and denote the pages by `\textit{child}|.|\textit{page}'
where \textit{child} represents the chapter/section number of the child file.
This can be achieved by the command
|\numberwithin{page}{|\textit{child}|}|
of the \textsf{amsmath} package
where \textit{child} can be |chapter| or |section|
depending on the chosen structuring.
Alternatively, one can modify the macro |\thepage| appropriately
and reset the counter |page| at the start of each child file.

%%%%%%%%%%%%%%%%%%%%%%%%%%%%%%%%%%%%%%%%%%%%%%%%%%%%%%%%%%%%%%%%%%%%%%%%%%%%%%%%
\subsection{Conditional Processing}
\label{sec:conditional}

The package provides a mechanism to compile different versions
of a document. To customise the versions further some conditional processing
can come in handy to distinguish which version is being compiled.
The package provides two macros to describe the compilation context:

%%%%%%%%%%%%%%%%%%%%%%%%%%%%%%%%%%%%%%%%
\DescribeMacro{\ifchilddoc}
The conditional |\ifchilddoc| distinguishes between the compilation of
child documents and the main document:
%
\begin{center}
|\ifchilddoc |\textit{child-code}| |[|\||else |\textit{main-code}]| \||fi|
\end{center}

%%%%%%%%%%%%%%%%%%%%%%%%%%%%%%%%%%%%%%%%
\DescribeMacro{\childdocname}
\DescribeMacro{\childdocjob}
The macro |\childdocname| contains the filename (without extension)
of the main or child file being processed.
Note that |\childdocjob| will always contain the name of the main file.

%%%%%%%%%%%%%%%%%%%%%%%%%%%%%%%%%%%%%%%%
\paragraph{Title Page.}

Conditional processing can be used to include a title or banner page
in the main document when proper precautions are taken.
Importantly, the code in the main file should ensure that the page counter
(as well as other status parameters which are stored in the |.aux| files)
takes the same value after the conditional processing.
Otherwise the page numbers may take divergent values
depending on which part is compiled.

For example, a title page could be declared by:
%
\begin{center}
\begin{tabular}{l}
|\ifchilddoc\||else|\\
|\addtocounter{page}{-1}|\\
\textit{code for title page}\\
|\newpage|\\
|\||fi|
\end{tabular}
\end{center}
%
A banner page for the child documents can be generated by:
%
\begin{center}
\begin{tabular}{l}
|\ifchilddoc|\\
|\addtocounter{page}{-1}|\\
\textit{code for banner page}\\
|\newpage|\\
|\||fi|
\end{tabular}
\end{center}
%
Here one could write a message such as:
\begin{center}
|This is the part \childdocname{} of \childdocjob{}.|
\end{center}

%%%%%%%%%%%%%%%%%%%%%%%%%%%%%%%%%%%%%%%%%%%%%%%%%%%%%%%%%%%%%%%%%%%%%%%%%%%%%%%%
\subsection{Flags}
\label{sec:flags}

The package makes it easy to generate different versions
of the main or child documents.
To this end compilation flags can be defined
and assigned different default values.
They will be particularly useful in conjunction
with the forwarding mechanism described in \secref{sec:forward}.

For example, it may be useful to have a flag |\version|
which can be set to |draft| or |final|.
The document source will contain some conditional code
depending on the value of |\version|.
Suppose further, the flag should default to |final| for the main file
and to |draft| for child files
which is a natural assignment for editing the document.
This is achieved by placing the following code
in the preamble of the main document
(below the |\childdocmain| directive):
%
\begin{center}
\begin{tabular}{l}
|\ifchilddoc|\\
|\providecommand{\version}{draft}|\\
|\||else|\\
|\providecommand{\version}{final}|\\
|\||fi|
\end{tabular}
\end{center}
%
The definition by |\providecommand| makes sure
that previous definitions are not overwritten.
Further statements |\providecommand{\version}{...}|
can thus be added before the above code to override it.

For the main file, one might add a line
(between |\childdocmain| and the above block)
%
\begin{center}
|%\ifchilddoc\||else\providecommand{\version}{draft}\||fi|
\end{center}
%
which can be uncommented to produce a draft version.
Likewise one can add a line to the very top of a child file
(above the |\childdocof{|\textit{main}|}| directive)
%
\begin{center}
|%\providecommand{\version}{final}|
\end{center}
%
which can be uncommented to produce the final version of this child document.

%%%%%%%%%%%%%%%%%%%%%%%%%%%%%%%%%%%%%%%%%%%%%%%%%%%%%%%%%%%%%%%%%%%%%%%%%%%%%%%%
\subsection{Forwarding}
\label{sec:forward}

Different versions of the main or child documents
using compilation flags as described in \secref{sec:flags}
can be (permanently) stored in different files
for convenient compilation, viewing and distribution.
To this end, the package defines a command
to pass on compilation to a different file:

%%%%%%%%%%%%%%%%%%%%%%%%%%%%%%%%%%%%%%%%
\DescribeMacro{\childdocforward}
The command |\childdocforward| redirects processing to
another source file:
%
\begin{center}
\begin{tabular}{l}
|\input{childdoc.def}|\\
|\childdocforward[|\textit{main}|]{|\textit{dest}|}|\\
\end{tabular}
\end{center}
%
The argument \textit{dest} is the destination file
(without extension).
It should be the main file or one of the child files.
Note that further \textsf{childdoc} directives
such as |\childdocof| and |\childdocforward|
in the indicated file will be processed in this form.
The optional argument \textit{main}
passes on directly to the main file \textit{main}
while pretending to compile the child \textit{dest}.
This form behaves as if \textit{dest}
issues |\childdocof{|\textit{main}|}| right away,
and no further \textsf{childdoc} directives will be processed.

%%%%%%%%%%%%%%%%%%%%%%%%%%%%%%%%%%%%%%%%
\DescribeMacro{\...prefix}
In the alternative form |\childdocforwardprefix|,
%
\begin{center}
\begin{tabular}{l}
|\input{childdoc.def}|\\
|\childdocforwardprefix[|\textit{main}|]{|\textit{prefix}|}{|\textit{dest}|}|
\end{tabular}
\end{center}
%
the destination file is determined by a pattern
depending on the current file:
To make this work, the current file must be called
`{\textit{prefix}\hspace{0.2em}\textit{suffix}}'
with \textit{prefix} matching precisely the argument.
Processing is then passed on to the file
`{\textit{dest}\hspace{0.2em}\textit{suffix}}'.
Surely, the same effect is achieved by
directly specifying the
argument `{\textit{dest}\hspace{0.2em}\textit{suffix}}'
in the first form.
However, that requires to set up a different file
for each child. With the alternative form of the command
all these files can have exactly the same content
which simplifies setting them up and maintaining them.

For example, the following file |draft.tex|
with a compilation flag |\version| as described in \secref{sec:flags}
compiles the main document as a draft:
%
\begin{center}
\begin{tabular}{l}
|\def\version{draft}|\\
|\input{childdoc.def}|\\
|\childdocforward{|\textit{main}|}|
\end{tabular}
\end{center}
%
Likewise, the following files |final|\textit{nn}|.tex|
compile the final version of the child document
|child|\textit{nn}|.tex|:
%
\begin{center}
\begin{tabular}{l}
|\def\version{final}|\\
|\input{childdoc.def}|\\
|\childdocforwardprefix{final}{child}|
\end{tabular}
\end{center}
%

Note that when several versions of a main file and/or of each child file
are to be generated, it may be convenient to set up a |Makefile| or
shell script to automatise the process.

%%%%%%%%%%%%%%%%%%%%%%%%%%%%%%%%%%%%%%%%%%%%%%%%%%%%%%%%%%%%%%%%%%%%%%%%%%%%%%%%
\subsection{Command Line Processing}
\label{sec:commandline}

The effect of redirection files can also be achieved by invoking
the \LaTeX{} compiler with a more elaborate command line.
Most conveniently this should be done as part
of a shell script or a |Makefile|.

When using \textsf{childdoc} in the main file, the following
command lines effectively perform a redirection
(note that depending on the shell being used,
backslashes may have to be doubled: `|\|' $\to$ `|\\|'):
%
\begin{center}
|... -jobname "|\textit{target}|" |\\|"|[\textit{flags}]%
|\input{childdoc.def}\childdocforward[|\textit{main}|]{|\textit{dest}|}"|
\end{center}
%
Here \textit{target} is the name of the output file,
\textit{main} is the name of the main file
and \textit{dest} is the name of the main or child file to be processed
(all filenames without extensions).
The optional argument \textit{main} can be omitted
if \textit{main} matches \textit{dest}.
Optionally, compilation \textit{flags} can be defined via |\def| commands.
This command line makes the \TeX{} engine believe
it is compiling the file \textit{target}
whose content is specified as the latter parameter.
The provided code then forwards the processing to
\textit{main} or \textit{dest} as described in \secref{sec:forward}.

%%%%%%%%%%%%%%%%%%%%%%%%%%%%%%%%%%%%%%%%%%%%%%%%%%%%%%%%%%%%%%%%%%%%%%%%%%%%%%%%
\subsection{Include by Input}
\label{sec:input}

Including child documents by |\include| has some restrictions by design.
Most notably, the content of a child document always occupies
its own set of pages; pages cannot be shared between child documents.
Usually, this behaviour makes perfect sense
because each child document contain an essential part of the document.
However, in some situations it may be desirable to compose
a document from a collection of parts
without having mandatory page breaks between then.
For this case, the package
provides a mechanism to include parts
by |\input| which can also be processed individually.
However, by construction this mechanism
requires manual handling of the content to be output.

%%%%%%%%%%%%%%%%%%%%%%%%%%%%%%%%%%%%%%%%
\DescribeMacro{\ifchilddocmanual}
The main file should be prepared as usual, see \secref{sec:include}.
However, the document body must make a distinction
between processing of an individual part and of the main document, e.g.:
%
\begin{center}
\begin{tabular}{l}
|\ifchilddocmanual|\\
|\input{\childdocname}|\\
|\||else|\\
\textit{document body with }|\input{|\textit{part}|}|\\
|\||fi|
\end{tabular}
\end{center}
%
The conditional |\ifchilddocmanual| is true whenever
a part to be included by |\input| is being compiled,
and the name of the part is stored in |\childdocname|.

%%%%%%%%%%%%%%%%%%%%%%%%%%%%%%%%%%%%%%%%
\DescribeMacro{\childdocby}
Each part to be included by |\input| should start with:
%
\begin{center}
\begin{tabular}{l}
|\input{childdoc.def}|\\
|\childdocby{|\textit{main}|}|\\
\end{tabular}
\end{center}
%
The directive |\childdocby| is similar to |\childdocof|
described in \secref{sec:include},
but the subsequent selection of content must be done manually.
To that end, both |\ifchilddoc| and |\ifchilddocmanual|
will be true upon processing of a part,
and the name of the part is stored in |\childdocname|.
Note that |\jobname| will be set to the filename of the current part
so that each part receives an individual |.aux| file
that does not interfere with the |.aux| file(s) of the main document.
This behaviour can be altered by the alternative form
|\childdocby[*]{|\textit{main}|}| (with a non-empty optional argument)
which uses the |.aux| file of the main document
by setting |\jobname| to \textit{main}.

%%%%%%%%%%%%%%%%%%%%%%%%%%%%%%%%%%%%%%%%%%%%%%%%%%%%%%%%%%%%%%%%%%%%%%%%%%%%%%%%
\subsection{Driver Development}
\label{sec:driver}

The \textsf{childdoc} mechanism can also be use for the development
of definition files such as \LaTeX{} styles or classes.
This case differs from the above setup with multiple parts
included by |\include| in that no |\includeonly| should be invoked.
This can be achieved by starting the include file
(before |\ProvidesPackage|) with:
%
\begin{center}
\begin{tabular}{l}
|\input{childdoc.def}|\\
|\childdocforward{|\textit{main}|}|\\
\end{tabular}
\end{center}
%
or alternatively with:
%
\begin{center}
\begin{tabular}{l}
|\input{childdoc.def}|\\
|\childdocby{|\textit{main}|}|\\
\end{tabular}
\end{center}
%
Both forms have slightly different effects as described above.
The main file is prepared as usual, see \secref{sec:include}.

%%%%%%%%%%%%%%%%%%%%%%%%%%%%%%%%%%%%%%%%%%%%%%%%%%%%%%%%%%%%%%%%%%%%%%%%%%%%%%%%
\subsection{Legacy Detection}
\label{sec:detection}

The directive |\childdocmain| in the main file can detect
whether the complete document or merely a child is to be compiled
even without using the directive |\childdocof|.
This method is deprecated because it is less robust
and there is no compelling reason to use it;
it is merely provided for backward compatibility
and it may be removed in future versions.

If the detection mechanism is to be used,
it is mandatory to correctly specify
the filename of the main file as the argument of |\childdocmain|:
%
\begin{center}
\begin{tabular}{l}
|\input{childdoc.def}|\\
|\childdocmain{|\textit{main}|}|\\
\end{tabular}
\end{center}
%
If |\jobname| does not match the argument \textit{main} of |\childdocmain|,
it is assumed that |\jobname| points to the child file to be compiled.
When using |\childdocmain| with the main file specified as argument,
it suffices to start a child file
with just |\input{|\textit{main}|}|
without loading of the package and using |\childdocof|.
If instead all processing is done
with the appropriate \textsf{childdoc} directives,
the argument of \textit{main} of |\childdocmain| can be empty.

An alternative version of the command line processing described
in \secref{sec:commandline} using the detection mechanism reads:
%
\begin{center}
|... -jobname "|\textit{target}|" "|[\textit{flags}]%
[|\def\jobname{|\textit{dest}|}|]|\input{|\textit{main}|}"|
\end{center}

%%%%%%%%%%%%%%%%%%%%%%%%%%%%%%%%%%%%%%%%%%%%%%%%%%%%%%%%%%%%%%%%%%%%%%%%%%%%%%%%
\subsection{Manual Code}
\label{sec:manual}

In case one cannot be certain whether the definitions file |childdoc.def|
is installed on the target \TeX{} distribution
and one prefers not to ship it,
it is conceivable to paste a few relevant commands into the sources.

To that end, drop all statements |\input{childdoc.def}|
and perform the replacements as outlined below.
Instead of |\childdocmain{|\textit{main}|}| add the following code
to the top of the main file:
%
\begin{center}
\begin{tabular}{l}
|\||ifdefined\childdocname\endinput\||fi\newif\ifchilddoc|\\
|\edef\childdocname{\scantokens\expandafter{\jobname\noexpand}}|\\
|\def\childdocmain{|\textit{main}|}\||ifx\childdocmain\childdocname\||else|\\
|\childdoctrue\includeonly{\childdocname}\let\jobname\childdocmain\||fi|\\
\end{tabular}
\end{center}
%
Instead of |\childdocof{|\textit{main}|}| just include the main file
at the top of each child file:
%
\begin{center}
|\input{|\textit{main}|}|
\end{center}
%
A simple redirection |\childdocforward{|\textit{dest}|}| is achieved by:
%
\begin{center}
|\def\jobname{|\textit{dest}|}\input{\jobname}|
\end{center}
%
The redirection with prefix
|\childdocforwardprefix[|\textit{prefix}|]{|\textit{dest}|}|
is accomplished by:
%
\begin{center}
\begin{tabular}{l}
|{\edef\jobname{\scantokens\expandafter{\jobname\noexpand}}|\\
|\def\redirectjob |\textit{prefix}|#1~~~{\gdef\jobname{|\textit{dest}|#1}}|\\
|\expandafter\redirectjob\jobname~~~}\input{\jobname}|
\end{tabular}
\end{center}

In an alternative approach,
child documents can be compiled by a specific command line
without additional code or specific definitions:
%
\begin{center}
|... -jobname "|\textit{target}|" "|[\textit{flags}]%
|\includeonly{|\textit{dest}|}\input{|\textit{main}|}"|
\end{center}
%

%%%%%%%%%%%%%%%%%%%%%%%%%%%%%%%%%%%%%%%%%%%%%%%%%%%%%%%%%%%%%%%%%%%%%%%%%%%%%%%%
%%%%%%%%%%%%%%%%%%%%%%%%%%%%%%%%%%%%%%%%%%%%%%%%%%%%%%%%%%%%%%%%%%%%%%%%%%%%%%%%
\section{Information}

%%%%%%%%%%%%%%%%%%%%%%%%%%%%%%%%%%%%%%%%%%%%%%%%%%%%%%%%%%%%%%%%%%%%%%%%%%%%%%%%
\subsection{Copyright}

Copyright \copyright{} 2017--2018 Niklas Beisert

This work may be distributed and/or modified under the
conditions of the \LaTeX{} Project Public License, either version 1.3
of this license or (at your option) any later version.
The latest version of this license is in
  \url{http://www.latex-project.org/lppl.txt}
and version 1.3 or later is part of all distributions of \LaTeX{}
version 2005/12/01 or later.

This work has the LPPL maintenance status `maintained'.

The Current Maintainer of this work is Niklas Beisert.

This work consists of the files |README.txt|, |childdoc.ins| and |childdoc.dtx|
as well as the derived files |childdoc.def|, |cdocsamp.tex|
with |cdocsch1.tex|, |cdocsch2.tex|, |cdocspt3.tex|, |cdocspt4.tex|,
|cdocsdrf.tex|, |cdocsfn1.tex|, |cdocsfn2.tex|
as well as |childdoc.pdf|.

%%%%%%%%%%%%%%%%%%%%%%%%%%%%%%%%%%%%%%%%%%%%%%%%%%%%%%%%%%%%%%%%%%%%%%%%%%%%%%%%
\subsection{Files and Installation}

The package consists of the files:
%
\begin{center}
\begin{tabular}{ll}
    |README.txt|   & readme file \\
    |childdoc.ins| & installation file \\
    |childdoc.dtx| & source file \\
    |childdoc.def| & definition file \\
    |cdocsamp.tex| & sample main file \\
    |cdocsch1.tex| & sample include file \\
    |cdocsch2.tex| & sample include file \\
    |cdocspt3.tex| & sample part file \\
    |cdocspt4.tex| & sample part file \\
    |cdocsdrf.tex| & sample redirection file \\
    |cdocsfn1.tex| & sample redirection file \\
    |cdocsfn2.tex| & sample redirection file \\
    |childdoc.pdf| & manual
\end{tabular}
\end{center}
%
The distribution consists of the files
|README.txt|, |childdoc.ins| and |childdoc.dtx|.
%
\begin{itemize}
\item
Run (pdf)\LaTeX{} on |childdoc.dtx|
to compile the manual |childdoc.pdf| (this file).
\item
Run \LaTeX{} on |childdoc.ins| to create the definitions file |childdoc.def|
and the sample |cdocsamp.tex| with include files
|cdocsch1.tex|, |cdocsch2.tex|, |cdocspt3.tex|, |cdocspt4.tex|,
|cdocsdrf.tex|, |cdocsfn1.tex|, |cdocsfn2.tex|.
Then copy the file |childdoc.def| to an appropriate directory of your \LaTeX{}
distribution, e.g.\ \textit{texmf-root}|/tex/latex/childdoc|.
\end{itemize}

%%%%%%%%%%%%%%%%%%%%%%%%%%%%%%%%%%%%%%%%%%%%%%%%%%%%%%%%%%%%%%%%%%%%%%%%%%%%%%%%
\subsection{Related CTAN Packages}

There are several other packages which offer a similar functionality:
%
\begin{itemize}
\item
The packages
\href{http://ctan.org/pkg/docmute}{\textsf{docmute}},
\href{http://ctan.org/pkg/includex}{\textsf{includex}} and
\href{http://ctan.org/pkg/standalone}{\textsf{standalone}}
provide commands to include only the document body of
a child file thus allowing both files to be compiled individually.
\item
The packages \href{http://ctan.org/pkg/subdocs}{\textsf{subdocs}}
and \href{http://ctan.org/pkg/subfiles}{\textsf{subfiles}}
provide structures in which the main and child documents can be
encapsulated and allowing them to be compiled individually.
The inclusion mechanism is different from the conventional |\include|.
\item
The package \href{http://ctan.org/pkg/combine}{\textsf{combine}}
is an elaborate solution to combine several documents into one.
\end{itemize}
%
See also the CTAN topic \href{http://ctan.org/topic/subdocs}{\textsf{subdocs}}
for further related packages.
The present package differs from the above solutions in that
a document structure constructed with the conventional |\include| mechanism
just needs two extra commands at the top of every file
such that all constituent files can be compiled individually.

%%%%%%%%%%%%%%%%%%%%%%%%%%%%%%%%%%%%%%%%%%%%%%%%%%%%%%%%%%%%%%%%%%%%%%%%%%%%%%%%
%\subsection{Feature Suggestions}
%
%The following is a list of features which may be useful for future
%versions of this package:
%%
%\begin{itemize}
%\item
%\ldots
%\end{itemize}

%%%%%%%%%%%%%%%%%%%%%%%%%%%%%%%%%%%%%%%%%%%%%%%%%%%%%%%%%%%%%%%%%%%%%%%%%%%%%%%%
\subsection{Revision History}

%%%%%%%%%%%%%%%%%%%%%%%%%%%%%%%%%%%%%%%%
\paragraph{v2.0:} 2018/12/30

\begin{itemize}
\item
immediate forward processing
\item
added |\childdocby| mechanism
\item
manual restructured
\end{itemize}

%%%%%%%%%%%%%%%%%%%%%%%%%%%%%%%%%%%%%%%%
\paragraph{v1.6:} 2018/01/17

\begin{itemize}
\item
application for development of include files
\item
corrections to manual
\end{itemize}

%%%%%%%%%%%%%%%%%%%%%%%%%%%%%%%%%%%%%%%%
\paragraph{v1.5:} 2017/05/21

\begin{itemize}
\item
more complete structuring introduced
\item
|\childdocof| introduced
\item
|\childdoc| renamed to |\childdocmain|
\item
|\childredirect| renamed to |\childdocforward| and |\childdocforwardprefix|
and functionality expanded
\end{itemize}

%%%%%%%%%%%%%%%%%%%%%%%%%%%%%%%%%%%%%%%%
\paragraph{v1.0:} 2017/04/27

\begin{itemize}
\item
manual and install package
\item
first version published on CTAN
\end{itemize}

%%%%%%%%%%%%%%%%%%%%%%%%%%%%%%%%%%%%%%%%
\paragraph{v0.6:} 2017/04/26

\begin{itemize}
\item
redirection mechanism added
\end{itemize}

%%%%%%%%%%%%%%%%%%%%%%%%%%%%%%%%%%%%%%%%
\paragraph{v0.5:} 2017/04/26

\begin{itemize}
\item
functionality in definition file
\end{itemize}


%%%%%%%%%%%%%%%%%%%%%%%%%%%%%%%%%%%%%%%%%%%%%%%%%%%%%%%%%%%%%%%%%%%%%%%%%%%%%%%%
%%%%%%%%%%%%%%%%%%%%%%%%%%%%%%%%%%%%%%%%%%%%%%%%%%%%%%%%%%%%%%%%%%%%%%%%%%%%%%%%
%%%%%%%%%%%%%%%%%%%%%%%%%%%%%%%%%%%%%%%%%%%%%%%%%%%%%%%%%%%%%%%%%%%%%%%%%%%%%%%%
\appendix

\settowidth\MacroIndent{\rmfamily\scriptsize 000\ }

 \DocInput{childdoc.dtx}

\end{document}
%</driver>
% \fi
%
% %%%%%%%%%%%%%%%%%%%%%%%%%%%%%%%%%%%%%%%%%%%%%%%%%%%%%%%%%%%%%%%%%%%%%%%%%%%%%%
% %%%%%%%%%%%%%%%%%%%%%%%%%%%%%%%%%%%%%%%%%%%%%%%%%%%%%%%%%%%%%%%%%%%%%%%%%%%%%%
% \section{Sample}
%\iffalse
%<*samplemain>
%\fi
%
% The following presents a sample document
% with two chapters, two parts, a title page,
% a compile flag as well as three forwarding files to set the flag.
% It consists of eight |.tex| files:
% \begin{center}
% \begin{tabular}{ll}
% |cdocsamp.tex|&main file\\
% |cdocsch1.tex|&include file for chapter 1\\
% |cdocsch2.tex|&include file for chapter 2\\
% |cdocspt3.tex|&include file for part 3\\
% |cdocspt4.tex|&include file for part 4\\
% |cdocsdrf.tex|&forwarding file for main file in draft mode\\
% |cdocsfi1.tex|&forwarding file for final version of chapter 1\\
% |cdocsfi2.tex|&forwarding file for final version of chapter 2\\
% \end{tabular}
% \end{center}
% Each of the eight files can be compiled directly by the \LaTeX{} compiler.
%
% %%%%%%%%%%%%%%%%%%%%%%%%%%%%%%%%%%%%%%
% \paragraph{Main File.}
%
% The main file is called |cdocsamp.tex|.
%
% Load the \textsf{childdoc} definitions and
% declare the filename for the main document:
%    \begin{macrocode}
\input{childdoc.def}
\childdocmain{}
%    \end{macrocode}

% Optional override for |\version| flag:
%    \begin{macrocode}
%%\ifchilddoc\else\providecommand{\version}{draft}\fi
%    \end{macrocode}

% Define the default values for the |\version| flag
% (|final| for the main file and |draft| for childs):
%    \begin{macrocode}
\ifchilddoc
\providecommand{\version}{draft}
\else
\providecommand{\version}{final}
\fi
%    \end{macrocode}

% Load the standard document class:
%    \begin{macrocode}
\documentclass[12pt]{article}
%    \end{macrocode}

% Start the document body:
%    \begin{macrocode}
\begin{document}
%    \end{macrocode}

% Declare a title page.
% Print title, part of document being processed and version flag:
%    \begin{macrocode}
\addtocounter{page}{-1}
\begin{center}
{\LARGE\bfseries{}childdoc example\par}
\vspace{1cm}
\ifchilddoc
\ifchilddocmanual part\else chapter\fi:
`\childdocname' of `\childdocjob'\par
\else
main document: `\childdocjob'\par
\fi
version: \version\par
\end{center}
\newpage
%    \end{macrocode}

% Manually include selected file,
% otherwise process as usual:
%    \begin{macrocode}
\ifchilddocmanual
\section*{part `\childdocname'}
\input{\childdocname}
\else
%    \end{macrocode}

% Include the two chapters:
%    \begin{macrocode}
\include{cdocsch1}
\include{cdocsch2}
%    \end{macrocode}

% Include the two parts unless only chapters should be displayed:
%    \begin{macrocode}
\ifchilddoc\else
\section{part three}
\input{cdocspt3}
\section{part four}
\input{cdocspt4}
\fi
%    \end{macrocode}

% Process as usual until here:
%    \begin{macrocode}
\fi
%    \end{macrocode}

% End of document body:
%    \begin{macrocode}
\end{document}
%    \end{macrocode}
%\iffalse
%</samplemain>
%\fi
%
% %%%%%%%%%%%%%%%%%%%%%%%%%%%%%%%%%%%%%%
% \paragraph{Chapter Include Files.}
%
% The include files are called |cdocsch1.tex| and |cdocsch2.tex|.
%
%\iffalse
%<*samplechap1|samplechap2>
%\fi

% Optional override for |\version| flag:
%    \begin{macrocode}
%%\providecommand{\version}{final}
%    \end{macrocode}

% Include the main document:
%    \begin{macrocode}
\input{childdoc.def}
\childdocof{cdocsamp}
%    \end{macrocode}

%\iffalse
%</samplechap1|samplechap2>
%\fi
%
%\iffalse
%<*samplechap1>
%\fi
% Some text for chapter 1:
%    \begin{macrocode}
\section{one}
some text in chapter one
%    \end{macrocode}

%\iffalse
%</samplechap1>
%\fi
% Some text for chapter 2:
%\iffalse
%<*samplechap2>
%\fi
%    \begin{macrocode}
\section{two}
more text in chapter two
%    \end{macrocode}

%\iffalse
%</samplechap2>
%\fi
%
% %%%%%%%%%%%%%%%%%%%%%%%%%%%%%%%%%%%%%%
% \paragraph{Part Include Files.}
%
% The include files are called |cdocspt3.tex| and |cdocspt4.tex|.
%
%\iffalse
%<*samplepart3|samplepart4>
%\fi

% Optional override for |\version| flag:
%    \begin{macrocode}
%%\providecommand{\version}{final}
%    \end{macrocode}

% Include the main document:
%    \begin{macrocode}
\input{childdoc.def}
\childdocby{cdocsamp}
%    \end{macrocode}

%\iffalse
%</samplepart3|samplepart4>
%\fi
%
%\iffalse
%<*samplepart3>
%\fi
% Some text for part 3:
%    \begin{macrocode}
some text in part three
%    \end{macrocode}

%\iffalse
%</samplepart3>
%\fi
% Some text for part 4:
%\iffalse
%<*samplepart4>
%\fi
%    \begin{macrocode}
more text in part four
%    \end{macrocode}

%\iffalse
%</samplepart4>
%\fi
%
% %%%%%%%%%%%%%%%%%%%%%%%%%%%%%%%%%%%%%%
% \paragraph{Forwarding for a Complete Draft.}
%
% The following forwarding file |cdocsdrf.tex|
% compiles the main document in draft mode:
%\iffalse
%<*sampledraft>
%\fi
%    \begin{macrocode}
\def\version{draft}
\input{childdoc.def}
\childdocforward{cdocsamp}
%    \end{macrocode}

%\iffalse
%</sampledraft>
%\fi
%
% %%%%%%%%%%%%%%%%%%%%%%%%%%%%%%%%%%%%%%
% \paragraph{Forwarding for Final Version of the Chapters.}
%
% The following forwarding files |cdocsfn1.tex| and |cdocsfn2.tex|
% (with identical content)
% compile the final versions of the child documents
% |cdocsch1.tex| and |cdocsch2.tex|, respectively:
%\iffalse
%<*samplefinal>
%\fi
%    \begin{macrocode}
\def\version{final}
\input{childdoc.def}
\childdocforwardprefix[cdocsamp]{cdocsfn}{cdocsch}
%    \end{macrocode}

%\iffalse
%</samplefinal>
%\fi
%
% %%%%%%%%%%%%%%%%%%%%%%%%%%%%%%%%%%%%%%
% \paragraph{Command Line Processing.}
%
% The following three command lines generate the output files
% |cdocscld|, |cdocscl1| and |cdocscl2|
% which should be identical to
% |cdocsdrf|, |cdocsch1| and |cdocsfn2|, respectively:
% \begin{center}
% \begin{tabular}{l}
% |latex -jobname cdocscld \|\\
% |  "\def\version{draft}\input{childdoc.def}\childdocforward{cdocsamp}"|\\
% |latex -jobname cdocscl1 \|\\
% |  "\input{childdoc.def}\childdocforward[cdocsamp]{cdocsch1}"|\\
% |latex -jobname cdocscl2 \|\\
% |  "\def\version{final}\input{childdoc.def}\childdocforward{cdocsch2}"|
% \end{tabular}
% \end{center}
% Note that the trailing backslash on each first line
% merely continues the input to the second line
% (for convenient cut ant paste).
% Furthermore, the command |latex| can be replaced by any
% of its alternative versions such as |pdflatex|.
%
% %%%%%%%%%%%%%%%%%%%%%%%%%%%%%%%%%%%%%%%%%%%%%%%%%%%%%%%%%%%%%%%%%%%%%%%%%%%%%%
% %%%%%%%%%%%%%%%%%%%%%%%%%%%%%%%%%%%%%%%%%%%%%%%%%%%%%%%%%%%%%%%%%%%%%%%%%%%%%%
% \section{Implementation}
%\iffalse
%<*package>
%\fi
%
% This section describes the definitions file |childdoc.def|.

% The definitions cannot be loaded using |\usepackage| or |\RequirePackage|
% which has a mechanism to prevent loading a style file more than once.
% When loading the definitions by means of |\input|
% multiple instances have to be prevented manually:
%\iffalse
%This code needs to be before the `\ProvidesFile' directive
%which is defined at the beginning of this file.
%Therefore it is also placed there and commented out here.
%</package>
%<*discard>
%\fi
%    \begin{macrocode}
\ifdefined\childdocmain\endinput\fi
%    \end{macrocode}
%\iffalse
%</discard>
%<*package>
%\fi
%
% \macro{\ifchilddoc}
% \macro{\ifchilddocmanual}
% The conditional |\ifchilddoc| tells whether a
% child (true) or main (false) document is being compiled.
% The conditional |\ifchilddocmanual| tells whether
% the |\includeonly| mechanism is used (false) or
% the selection of child files must be performed manually (true).
% The definitions initialise to false:
%    \begin{macrocode}
\newif\ifchilddoc
\newif\ifchilddocmanual
%    \end{macrocode}

% \macro{\childdocname}
% \macro{\childdocjob}
% The macro |\childdocname| stores the name of the main document
% to be compiled. The macro |\childdocjob| stores the name of
% the document on which the \LaTeX{} compiler was originally invoked.
% The content of |\jobname| cannot be compared
% to filenames specified in the source due to different catcodes.
% The following code rescans |\jobname|, stores the result
% in |\childdocname| and saves a copy in |\childdocjob|:
%    \begin{macrocode}
\edef\childdocname{\scantokens\expandafter{\jobname\noexpand}}
\let\childdocjob\childdocname
%    \end{macrocode}

% \macro{\childdocdisable}
% The macro |\childdocdisable| prevents the main file
% from being processed more than once.
% At this stage, the main document command |\childdocmain|
% is assumed to be called once again where it should do nothing.
% Any subsequent call to it should prevent
% a secondary processing of the main document
% It overwrites the forwarding commands
% |\childdocof| and |\childdocforward|
% with empty macros to prevent further inclusions of the main document:
%    \begin{macrocode}
\newcommand{\childdocdisable}
{
  \renewcommand{\childdocmain}[1]{\renewcommand{\childdocmain}[1]{\endinput}}
  \renewcommand{\childdocof}[1]{}
  \renewcommand{\childdocby}[2][]{}
  \renewcommand{\childdocforward}[2][]{}
  \renewcommand{\childdocdisable}{}
}
%    \end{macrocode}

% \macro{\childdocmain}
% The macro |\childdocmain| is to be called at the top of the main file
% with nothing or the main filename (without extension) as argument.
% First, it breaks loops.
% If the argument is not empty and does not match |\childdocname|
% (which is set by the first inclusion of |childdoc.def|),
% |\ifchilddoc| is set to true, |\includeonly| is applied to the child file
% and |\jobname| is set to the main file
% (for proper handling of |.aux| files):
%    \begin{macrocode}
\newcommand{\childdocmain}[1]
{
  \childdocdisable\childdocmain{}
  \if?#1?\else
    \begingroup
      \def\childdoctmp{#1}
      \ifx\childdoctmp\childdocname
        \def\childdoctmp{}
      \else
        \def\childdoctmp
        {
          \childdoctrue
          \includeonly{\childdocname}
          \def\childdocjob{#1}
          \def\jobname{#1}
        }
      \fi
      \expandafter
    \endgroup
    \childdoctmp
  \fi
}
%    \end{macrocode}

% \macro{\childdocof}
% The command |\childdocof| redirects
% compilation to the main file |#1|.
%    \begin{macrocode}
\newcommand{\childdocof}[1]
{
  \childdocdisable
  \childdoctrue
  \includeonly{\childdocname}
  \def\jobname{#1}
  \def\childdocjob{#1}
  \input{#1}
}
%    \end{macrocode}

% \macro{\childdocby}
% The command |\childdocby| ....
%    \begin{macrocode}
\newcommand{\childdocby}[2][]
{
  \childdocdisable
  \childdoctrue
  \childdocmanualtrue
  \if?#1?\else
    \def\jobname{#2}
  \fi
  \def\childdocjob{#2}
  \input{#2}
  \endinput
}
%    \end{macrocode}

% \macro{\childdocforward}
% The command |\childdocforward| redirects
% compilation to the main file or
% (if the optional argument is given) a child file.
% Parameters are set as if the main file
% or a child file starting with |\childdocof| was compiled.
% Then compilation is handed over to the main file:
%    \begin{macrocode}
\newcommand{\childdocforward}[2][]
{
  \begingroup
    \if?#1?
      \def\childdoctmp
      {
        \def\childdocname{#2}
        \def\childdocjob{#2}
        \def\jobname{#2}
        \input{#2}
        \endinput
      }
    \else
      \def\childdoctmp
      {
        \childdocdisable
        \def\childdocname{#2}
        \childdoctrue
        \includeonly{#2}
        \def\childdocjob{#1}
        \def\jobname{#1}
        \input{#1}
        \endinput
      }
    \fi
    \expandafter
  \endgroup
  \childdoctmp
}
%    \end{macrocode}

% \macro{\childdocforwardprefix}
% The command |\childdocforwardprefix| redirects
% compilation to the main or a child file by means of a pattern.
% The prefix |#1| in the current filename is replaced by |#2|
% and the suffix of the current filename is kept
% (it is assumed that the filename does not contain the substring `|~~~|'
% which is used as a delimiter).
% Compilation is handed over to the new file by |\childdocforward|:
%    \begin{macrocode}
\newcommand{\childdocforwardprefix}[3][]
{
  \begingroup
    \def\childdocextract #2##1~~~{\def\childdoctmp{\childdocforward[#1]{#3##1}}}
    \expandafter\childdocextract\childdocname~~~
    \expandafter
  \endgroup
  \childdoctmp
}
%    \end{macrocode}

% \macro{\childdoc}
% The deprecated macro |\childdoc| is a legacy version of |\childdocmain|:
%    \begin{macrocode}
\newcommand{\childdoc}{\childdocmain}
%    \end{macrocode}

% \macro{\childdocredirect}
% The deprecated macro |\childdocredirect| is a legacy version
% of |\childdocforward| and |\childdocforwardprefix|:
%    \begin{macrocode}
\newcommand{\childdocredirect}[2][]
{
  \begingroup
    \if?#1?
      \def\childdoctmp{\childdocforward{#2}}
    \else
      \def\childdoctmp{\childdocforwardprefix{#1}{#2}}
    \fi
    \expandafter
  \endgroup
  \childdoctmp
}
%    \end{macrocode}

%\iffalse
%</package>
%\fi
%
\endinput
|
and perform the replacements as outlined below.
Instead of |\childdocmain{|\textit{main}|}| add the following code
to the top of the main file:
%
\begin{center}
\begin{tabular}{l}
|\||ifdefined\childdocname\endinput\||fi\newif\ifchilddoc|\\
|\edef\childdocname{\scantokens\expandafter{\jobname\noexpand}}|\\
|\def\childdocmain{|\textit{main}|}\||ifx\childdocmain\childdocname\||else|\\
|\childdoctrue\includeonly{\childdocname}\let\jobname\childdocmain\||fi|\\
\end{tabular}
\end{center}
%
Instead of |\childdocof{|\textit{main}|}| just include the main file
at the top of each child file:
%
\begin{center}
|\input{|\textit{main}|}|
\end{center}
%
A simple redirection |\childdocforward{|\textit{dest}|}| is achieved by:
%
\begin{center}
|\def\jobname{|\textit{dest}|}\input{\jobname}|
\end{center}
%
The redirection with prefix
|\childdocforwardprefix[|\textit{prefix}|]{|\textit{dest}|}|
is accomplished by:
%
\begin{center}
\begin{tabular}{l}
|{\edef\jobname{\scantokens\expandafter{\jobname\noexpand}}|\\
|\def\redirectjob |\textit{prefix}|#1~~~{\gdef\jobname{|\textit{dest}|#1}}|\\
|\expandafter\redirectjob\jobname~~~}\input{\jobname}|
\end{tabular}
\end{center}

In an alternative approach,
child documents can be compiled by a specific command line
without additional code or specific definitions:
%
\begin{center}
|... -jobname "|\textit{target}|" "|[\textit{flags}]%
|\includeonly{|\textit{dest}|}\input{|\textit{main}|}"|
\end{center}
%

%%%%%%%%%%%%%%%%%%%%%%%%%%%%%%%%%%%%%%%%%%%%%%%%%%%%%%%%%%%%%%%%%%%%%%%%%%%%%%%%
%%%%%%%%%%%%%%%%%%%%%%%%%%%%%%%%%%%%%%%%%%%%%%%%%%%%%%%%%%%%%%%%%%%%%%%%%%%%%%%%
\section{Information}

%%%%%%%%%%%%%%%%%%%%%%%%%%%%%%%%%%%%%%%%%%%%%%%%%%%%%%%%%%%%%%%%%%%%%%%%%%%%%%%%
\subsection{Copyright}

Copyright \copyright{} 2017--2018 Niklas Beisert

This work may be distributed and/or modified under the
conditions of the \LaTeX{} Project Public License, either version 1.3
of this license or (at your option) any later version.
The latest version of this license is in
  \url{http://www.latex-project.org/lppl.txt}
and version 1.3 or later is part of all distributions of \LaTeX{}
version 2005/12/01 or later.

This work has the LPPL maintenance status `maintained'.

The Current Maintainer of this work is Niklas Beisert.

This work consists of the files |README.txt|, |childdoc.ins| and |childdoc.dtx|
as well as the derived files |childdoc.def|, |cdocsamp.tex|
with |cdocsch1.tex|, |cdocsch2.tex|, |cdocspt3.tex|, |cdocspt4.tex|,
|cdocsdrf.tex|, |cdocsfn1.tex|, |cdocsfn2.tex|
as well as |childdoc.pdf|.

%%%%%%%%%%%%%%%%%%%%%%%%%%%%%%%%%%%%%%%%%%%%%%%%%%%%%%%%%%%%%%%%%%%%%%%%%%%%%%%%
\subsection{Files and Installation}

The package consists of the files:
%
\begin{center}
\begin{tabular}{ll}
    |README.txt|   & readme file \\
    |childdoc.ins| & installation file \\
    |childdoc.dtx| & source file \\
    |childdoc.def| & definition file \\
    |cdocsamp.tex| & sample main file \\
    |cdocsch1.tex| & sample include file \\
    |cdocsch2.tex| & sample include file \\
    |cdocspt3.tex| & sample part file \\
    |cdocspt4.tex| & sample part file \\
    |cdocsdrf.tex| & sample redirection file \\
    |cdocsfn1.tex| & sample redirection file \\
    |cdocsfn2.tex| & sample redirection file \\
    |childdoc.pdf| & manual
\end{tabular}
\end{center}
%
The distribution consists of the files
|README.txt|, |childdoc.ins| and |childdoc.dtx|.
%
\begin{itemize}
\item
Run (pdf)\LaTeX{} on |childdoc.dtx|
to compile the manual |childdoc.pdf| (this file).
\item
Run \LaTeX{} on |childdoc.ins| to create the definitions file |childdoc.def|
and the sample |cdocsamp.tex| with include files
|cdocsch1.tex|, |cdocsch2.tex|, |cdocspt3.tex|, |cdocspt4.tex|,
|cdocsdrf.tex|, |cdocsfn1.tex|, |cdocsfn2.tex|.
Then copy the file |childdoc.def| to an appropriate directory of your \LaTeX{}
distribution, e.g.\ \textit{texmf-root}|/tex/latex/childdoc|.
\end{itemize}

%%%%%%%%%%%%%%%%%%%%%%%%%%%%%%%%%%%%%%%%%%%%%%%%%%%%%%%%%%%%%%%%%%%%%%%%%%%%%%%%
\subsection{Related CTAN Packages}

There are several other packages which offer a similar functionality:
%
\begin{itemize}
\item
The packages
\href{http://ctan.org/pkg/docmute}{\textsf{docmute}},
\href{http://ctan.org/pkg/includex}{\textsf{includex}} and
\href{http://ctan.org/pkg/standalone}{\textsf{standalone}}
provide commands to include only the document body of
a child file thus allowing both files to be compiled individually.
\item
The packages \href{http://ctan.org/pkg/subdocs}{\textsf{subdocs}}
and \href{http://ctan.org/pkg/subfiles}{\textsf{subfiles}}
provide structures in which the main and child documents can be
encapsulated and allowing them to be compiled individually.
The inclusion mechanism is different from the conventional |\include|.
\item
The package \href{http://ctan.org/pkg/combine}{\textsf{combine}}
is an elaborate solution to combine several documents into one.
\end{itemize}
%
See also the CTAN topic \href{http://ctan.org/topic/subdocs}{\textsf{subdocs}}
for further related packages.
The present package differs from the above solutions in that
a document structure constructed with the conventional |\include| mechanism
just needs two extra commands at the top of every file
such that all constituent files can be compiled individually.

%%%%%%%%%%%%%%%%%%%%%%%%%%%%%%%%%%%%%%%%%%%%%%%%%%%%%%%%%%%%%%%%%%%%%%%%%%%%%%%%
%\subsection{Feature Suggestions}
%
%The following is a list of features which may be useful for future
%versions of this package:
%%
%\begin{itemize}
%\item
%\ldots
%\end{itemize}

%%%%%%%%%%%%%%%%%%%%%%%%%%%%%%%%%%%%%%%%%%%%%%%%%%%%%%%%%%%%%%%%%%%%%%%%%%%%%%%%
\subsection{Revision History}

%%%%%%%%%%%%%%%%%%%%%%%%%%%%%%%%%%%%%%%%
\paragraph{v2.0:} 2018/12/30

\begin{itemize}
\item
immediate forward processing
\item
added |\childdocby| mechanism
\item
manual restructured
\end{itemize}

%%%%%%%%%%%%%%%%%%%%%%%%%%%%%%%%%%%%%%%%
\paragraph{v1.6:} 2018/01/17

\begin{itemize}
\item
application for development of include files
\item
corrections to manual
\end{itemize}

%%%%%%%%%%%%%%%%%%%%%%%%%%%%%%%%%%%%%%%%
\paragraph{v1.5:} 2017/05/21

\begin{itemize}
\item
more complete structuring introduced
\item
|\childdocof| introduced
\item
|\childdoc| renamed to |\childdocmain|
\item
|\childredirect| renamed to |\childdocforward| and |\childdocforwardprefix|
and functionality expanded
\end{itemize}

%%%%%%%%%%%%%%%%%%%%%%%%%%%%%%%%%%%%%%%%
\paragraph{v1.0:} 2017/04/27

\begin{itemize}
\item
manual and install package
\item
first version published on CTAN
\end{itemize}

%%%%%%%%%%%%%%%%%%%%%%%%%%%%%%%%%%%%%%%%
\paragraph{v0.6:} 2017/04/26

\begin{itemize}
\item
redirection mechanism added
\end{itemize}

%%%%%%%%%%%%%%%%%%%%%%%%%%%%%%%%%%%%%%%%
\paragraph{v0.5:} 2017/04/26

\begin{itemize}
\item
functionality in definition file
\end{itemize}


%%%%%%%%%%%%%%%%%%%%%%%%%%%%%%%%%%%%%%%%%%%%%%%%%%%%%%%%%%%%%%%%%%%%%%%%%%%%%%%%
%%%%%%%%%%%%%%%%%%%%%%%%%%%%%%%%%%%%%%%%%%%%%%%%%%%%%%%%%%%%%%%%%%%%%%%%%%%%%%%%
%%%%%%%%%%%%%%%%%%%%%%%%%%%%%%%%%%%%%%%%%%%%%%%%%%%%%%%%%%%%%%%%%%%%%%%%%%%%%%%%
\appendix

\settowidth\MacroIndent{\rmfamily\scriptsize 000\ }

 \DocInput{childdoc.dtx}

\end{document}
%</driver>
% \fi
%
% %%%%%%%%%%%%%%%%%%%%%%%%%%%%%%%%%%%%%%%%%%%%%%%%%%%%%%%%%%%%%%%%%%%%%%%%%%%%%%
% %%%%%%%%%%%%%%%%%%%%%%%%%%%%%%%%%%%%%%%%%%%%%%%%%%%%%%%%%%%%%%%%%%%%%%%%%%%%%%
% \section{Sample}
%\iffalse
%<*samplemain>
%\fi
%
% The following presents a sample document
% with two chapters, two parts, a title page,
% a compile flag as well as three forwarding files to set the flag.
% It consists of eight |.tex| files:
% \begin{center}
% \begin{tabular}{ll}
% |cdocsamp.tex|&main file\\
% |cdocsch1.tex|&include file for chapter 1\\
% |cdocsch2.tex|&include file for chapter 2\\
% |cdocspt3.tex|&include file for part 3\\
% |cdocspt4.tex|&include file for part 4\\
% |cdocsdrf.tex|&forwarding file for main file in draft mode\\
% |cdocsfi1.tex|&forwarding file for final version of chapter 1\\
% |cdocsfi2.tex|&forwarding file for final version of chapter 2\\
% \end{tabular}
% \end{center}
% Each of the eight files can be compiled directly by the \LaTeX{} compiler.
%
% %%%%%%%%%%%%%%%%%%%%%%%%%%%%%%%%%%%%%%
% \paragraph{Main File.}
%
% The main file is called |cdocsamp.tex|.
%
% Load the \textsf{childdoc} definitions and
% declare the filename for the main document:
%    \begin{macrocode}
% \iffalse
%
% childdoc.dtx Copyright (C) 2017-2018 Niklas Beisert
%
% This work may be distributed and/or modified under the
% conditions of the LaTeX Project Public License, either version 1.3
% of this license or (at your option) any later version.
% The latest version of this license is in
%   http://www.latex-project.org/lppl.txt
% and version 1.3 or later is part of all distributions of LaTeX
% version 2005/12/01 or later.
%
% This work has the LPPL maintenance status `maintained'.
%
% The Current Maintainer of this work is Niklas Beisert.
%
% This work consists of the files childdoc.dtx and childdoc.ins
% and the derived files childdoc.def and cdocsamp.tex with
% cdocsch1.tex, cdocsch2.tex, cdocsdrf.tex, cdocsfn1.tex, cdocsfn2.tex.
%
%<package>\ifdefined\childdocmain\endinput\fi
%<package>\ProvidesFile{childdoc.def}[2018/12/30 v2.0 child document driver]
%<samplemain>\ProvidesFile{cdocsamp.tex}[2018/12/30 v2.0 sample for childdoc]
%<*driver>
%\ProvidesFile{childdoc.drv}[2018/12/30 v2.0 childdoc reference manual file]
\PassOptionsToClass{10pt,a4paper}{article}
\documentclass{ltxdoc}

\usepackage[margin=35mm]{geometry}
\usepackage{hyperref}
\usepackage{hyperxmp}
\usepackage[usenames]{color}

\hypersetup{colorlinks=true}
\hypersetup{pdfstartview=FitH}
\hypersetup{pdfpagemode=UseNone}
\hypersetup{pdfsource={}}
\hypersetup{pdflang={en-UK}}
\hypersetup{pdfcopyright={Copyright 2017-2018 Niklas Beisert.
  This work may be distributed and/or modified under the
  conditions of the LaTeX Project Public License, either version 1.3
  of this license or (at your option) any later version.}}
\hypersetup{pdflicenseurl={http://www.latex-project.org/lppl.txt}}
\hypersetup{pdfcontactaddress={ETH Zurich, ITP, HIT K,
  Wolfgang-Pauli-Strasse 27}}
\hypersetup{pdfcontactpostcode={8093}}
\hypersetup{pdfcontactcity={Zurich}}
\hypersetup{pdfcontactcountry={Switzerland}}
\hypersetup{pdfcontactemail={nbeisert@itp.phys.ethz.ch}}
\hypersetup{pdfcontacturl={http://people.phys.ethz.ch/\xmptilde nbeisert/}}

\newcommand{\secref}[1]{\hyperref[#1]{section \ref*{#1}}}

\parskip1ex
\parindent0pt
\let\olditemize\itemize
\def\itemize{\olditemize\parskip0pt}

\begin{document}

\title{The \textsf{childdoc} Package}
\hypersetup{pdftitle={The childdoc Package}}
\author{Niklas Beisert\\[2ex]
  Institut f\"ur Theoretische Physik\\
  Eidgen\"ossische Technische Hochschule Z\"urich\\
  Wolfgang-Pauli-Strasse 27, 8093 Z\"urich, Switzerland\\[1ex]
  \href{mailto:nbeisert@itp.phys.ethz.ch}
  {\texttt{nbeisert@itp.phys.ethz.ch}}}
\hypersetup{pdfauthor={Niklas Beisert}}
\hypersetup{pdfsubject={Manual for the LaTeX2e Package childdoc}}
\date{30 December 2018, \textsf{v2.0}}
\maketitle

\begin{abstract}\noindent
\textsf{childdoc} is a \LaTeXe{} package
that enables the direct compilation
of document sections included by |\include|
to individual files.
\end{abstract}

\begingroup
\parskip0ex
\tableofcontents
\endgroup

%%%%%%%%%%%%%%%%%%%%%%%%%%%%%%%%%%%%%%%%%%%%%%%%%%%%%%%%%%%%%%%%%%%%%%%%%%%%%%%%
%%%%%%%%%%%%%%%%%%%%%%%%%%%%%%%%%%%%%%%%%%%%%%%%%%%%%%%%%%%%%%%%%%%%%%%%%%%%%%%%
\section{Introduction}

\LaTeX{} provides a mechanism to structure a large document (such as a book)
into a main file and several child files (containing the chapters)
using the |\include| command.
This mechanism is beneficial for documents
which span hundreds of pages in order to
make the source file(s) more manageable.
Moreover, compilation can be restricted to
selected child files by means of the |\includeonly| command.
The latter feature can be used to reduce the compilation time while editing
(this was significantly more useful in the earlier days of \LaTeX{})
or to generate a smaller document which is easier to navigate.
Another application of |\includeonly| is to generate
documents consisting of selected parts of the complete document.

However, there are a few drawbacks of the plain |\include| mechanism:
\begin{itemize}
\item
The child files cannot be compiled on their own,
they can only be compiled via the main file.
A naive editing environment
(such as a text editor with an option
to have the current file processed by \LaTeX)
may require one to switch to the main file before compiling;
attempting to compile the child file produces errors.
\item
The main file must be modified (each time)
to adjust the |\includeonly| command
to the present needs. This easily leaves the main file in a messy state.
\item
The generated document will always carry the filename
of the main document. This is inconvenient if
several child files are to be compiled and
to be kept for distribution.
\end{itemize}

The present package provides a simple interface
to make child files individually compilable by \LaTeX{}.
Compiling a child file then has the same effect as compiling
the main file with an |\includeonly| command
to select the appropriate child.
Moreover the generated document will carry the name of the child
rather than the main file.
This resolves all three above issues.

This feature is meant to make the editing of books,
thesis documents and lecture notes somewhat more convenient.
However, the package can also be used efficiently for
composing a series of documents (such as exercise sheets)
which are typically distributed individually.
It then assists the author in generating the individual documents
(potentially in different versions)
as well as a document containing the collected series.
Another application is in developing style files
or other kinds of included material
where compilation of the style file could redirect
to a sample or test file.

%%%%%%%%%%%%%%%%%%%%%%%%%%%%%%%%%%%%%%%%%%%%%%%%%%%%%%%%%%%%%%%%%%%%%%%%%%%%%%%%
%%%%%%%%%%%%%%%%%%%%%%%%%%%%%%%%%%%%%%%%%%%%%%%%%%%%%%%%%%%%%%%%%%%%%%%%%%%%%%%%
\section{Usage}

First of all, the package \textsf{childdoc} is \emph{not} a standard
\LaTeXe{} |.sty| style file! Therefore it needs to be invoked in
a non-standard way.

%%%%%%%%%%%%%%%%%%%%%%%%%%%%%%%%%%%%%%%%%%%%%%%%%%%%%%%%%%%%%%%%%%%%%%%%%%%%%%%%
\subsection{Included Files}
\label{sec:include}

%%%%%%%%%%%%%%%%%%%%%%%%%%%%%%%%%%%%%%%%
\DescribeMacro{\childdocmain}
To use the package, add the commands
\begin{center}
\begin{tabular}{l}
|\input{childdoc.def}|\\
|\childdocmain{}|\\
\end{tabular}
\end{center}
at the very top of the main \LaTeX{} file,
in particular \emph{before} the |\documentclass| statement!
The argument of |\childdocmain| should be left empty
(but it must be present).

%%%%%%%%%%%%%%%%%%%%%%%%%%%%%%%%%%%%%%%%
\DescribeMacro{\childdocof}
Furthermore, add the commands
\begin{center}
\begin{tabular}{l}
|\input{childdoc.def}|\\
|\childdocof{|\textit{main}|}|\\
\end{tabular}
\end{center}
at the top of every child file \textit{child}
which is included by |\include{|\textit{child}|}|
from within the main file
(or at least for those files to be compiled individually).
The argument \textit{main} must be the filename of the main file.

There are a couple of
considerations in setting up the main and child documents:

%%%%%%%%%%%%%%%%%%%%%%%%%%%%%%%%%%%%%%%%
\paragraph{Restrictions.}

Please note the following restrictions:
\begin{itemize}
\item
|\childdocmain| must be called with one argument \textit{main}
to ensure compatibility with earlier version of the package.
It must either be empty (|\childdocmain{}|)
or precisely match the filename of the main file in which it is specified.
See \secref{sec:detection} for further information.
\item
The filename \textit{main} must be specified without the |.tex| extension.
\item
The filename \textit{main} is case sensitive
(even in case-insensitive file systems)
due to internal string comparison.
\item
The argument \textit{main} should be fully expanded, it cannot be a macro.
\item
Subdirectories and special characters should be avoided in filenames.
\item
The command |\childdocmain{|\textit{main}|}| must be followed by a whitespace.
It should not be followed immediately by another command
or by a comment mark `|%|'.
This is because the \TeX{} parser reads the token immediately following
the argument of |\childdocmain| and puts it
at the beginning of every child section;
however, a white\-space is ignored.
\end{itemize}

%%%%%%%%%%%%%%%%%%%%%%%%%%%%%%%%%%%%%%%%
\paragraph{Content of Main File.}

It is advisable to place all content in the child files included by |\include|.
Any output contained in the main file will appear in all child documents
unless suppressed manually;
it cannot be suppressed automatically by the |\includeonly| directive
and thus should normally be avoided.
A method to include some content in the main file
by means of conditional processing is described in \secref{sec:conditional}.

%%%%%%%%%%%%%%%%%%%%%%%%%%%%%%%%%%%%%%%%
\paragraph{Page Numbering.}

When only a part of the document is compiled,
the appropriate numbering of pages
(as well as other status parameters)
is determined from the |.aux| files.
The latter contain information from previous passes.
However this information needs to propagate through
all intermediate child documents.
Therefore the page numbering in child documents may well
be inconsistent until the complete document is compiled at least once.

A useful (if unconventional) way to always ensure a consistent
page numbering is to restart the numbering in each child document
and denote the pages by `\textit{child}|.|\textit{page}'
where \textit{child} represents the chapter/section number of the child file.
This can be achieved by the command
|\numberwithin{page}{|\textit{child}|}|
of the \textsf{amsmath} package
where \textit{child} can be |chapter| or |section|
depending on the chosen structuring.
Alternatively, one can modify the macro |\thepage| appropriately
and reset the counter |page| at the start of each child file.

%%%%%%%%%%%%%%%%%%%%%%%%%%%%%%%%%%%%%%%%%%%%%%%%%%%%%%%%%%%%%%%%%%%%%%%%%%%%%%%%
\subsection{Conditional Processing}
\label{sec:conditional}

The package provides a mechanism to compile different versions
of a document. To customise the versions further some conditional processing
can come in handy to distinguish which version is being compiled.
The package provides two macros to describe the compilation context:

%%%%%%%%%%%%%%%%%%%%%%%%%%%%%%%%%%%%%%%%
\DescribeMacro{\ifchilddoc}
The conditional |\ifchilddoc| distinguishes between the compilation of
child documents and the main document:
%
\begin{center}
|\ifchilddoc |\textit{child-code}| |[|\||else |\textit{main-code}]| \||fi|
\end{center}

%%%%%%%%%%%%%%%%%%%%%%%%%%%%%%%%%%%%%%%%
\DescribeMacro{\childdocname}
\DescribeMacro{\childdocjob}
The macro |\childdocname| contains the filename (without extension)
of the main or child file being processed.
Note that |\childdocjob| will always contain the name of the main file.

%%%%%%%%%%%%%%%%%%%%%%%%%%%%%%%%%%%%%%%%
\paragraph{Title Page.}

Conditional processing can be used to include a title or banner page
in the main document when proper precautions are taken.
Importantly, the code in the main file should ensure that the page counter
(as well as other status parameters which are stored in the |.aux| files)
takes the same value after the conditional processing.
Otherwise the page numbers may take divergent values
depending on which part is compiled.

For example, a title page could be declared by:
%
\begin{center}
\begin{tabular}{l}
|\ifchilddoc\||else|\\
|\addtocounter{page}{-1}|\\
\textit{code for title page}\\
|\newpage|\\
|\||fi|
\end{tabular}
\end{center}
%
A banner page for the child documents can be generated by:
%
\begin{center}
\begin{tabular}{l}
|\ifchilddoc|\\
|\addtocounter{page}{-1}|\\
\textit{code for banner page}\\
|\newpage|\\
|\||fi|
\end{tabular}
\end{center}
%
Here one could write a message such as:
\begin{center}
|This is the part \childdocname{} of \childdocjob{}.|
\end{center}

%%%%%%%%%%%%%%%%%%%%%%%%%%%%%%%%%%%%%%%%%%%%%%%%%%%%%%%%%%%%%%%%%%%%%%%%%%%%%%%%
\subsection{Flags}
\label{sec:flags}

The package makes it easy to generate different versions
of the main or child documents.
To this end compilation flags can be defined
and assigned different default values.
They will be particularly useful in conjunction
with the forwarding mechanism described in \secref{sec:forward}.

For example, it may be useful to have a flag |\version|
which can be set to |draft| or |final|.
The document source will contain some conditional code
depending on the value of |\version|.
Suppose further, the flag should default to |final| for the main file
and to |draft| for child files
which is a natural assignment for editing the document.
This is achieved by placing the following code
in the preamble of the main document
(below the |\childdocmain| directive):
%
\begin{center}
\begin{tabular}{l}
|\ifchilddoc|\\
|\providecommand{\version}{draft}|\\
|\||else|\\
|\providecommand{\version}{final}|\\
|\||fi|
\end{tabular}
\end{center}
%
The definition by |\providecommand| makes sure
that previous definitions are not overwritten.
Further statements |\providecommand{\version}{...}|
can thus be added before the above code to override it.

For the main file, one might add a line
(between |\childdocmain| and the above block)
%
\begin{center}
|%\ifchilddoc\||else\providecommand{\version}{draft}\||fi|
\end{center}
%
which can be uncommented to produce a draft version.
Likewise one can add a line to the very top of a child file
(above the |\childdocof{|\textit{main}|}| directive)
%
\begin{center}
|%\providecommand{\version}{final}|
\end{center}
%
which can be uncommented to produce the final version of this child document.

%%%%%%%%%%%%%%%%%%%%%%%%%%%%%%%%%%%%%%%%%%%%%%%%%%%%%%%%%%%%%%%%%%%%%%%%%%%%%%%%
\subsection{Forwarding}
\label{sec:forward}

Different versions of the main or child documents
using compilation flags as described in \secref{sec:flags}
can be (permanently) stored in different files
for convenient compilation, viewing and distribution.
To this end, the package defines a command
to pass on compilation to a different file:

%%%%%%%%%%%%%%%%%%%%%%%%%%%%%%%%%%%%%%%%
\DescribeMacro{\childdocforward}
The command |\childdocforward| redirects processing to
another source file:
%
\begin{center}
\begin{tabular}{l}
|\input{childdoc.def}|\\
|\childdocforward[|\textit{main}|]{|\textit{dest}|}|\\
\end{tabular}
\end{center}
%
The argument \textit{dest} is the destination file
(without extension).
It should be the main file or one of the child files.
Note that further \textsf{childdoc} directives
such as |\childdocof| and |\childdocforward|
in the indicated file will be processed in this form.
The optional argument \textit{main}
passes on directly to the main file \textit{main}
while pretending to compile the child \textit{dest}.
This form behaves as if \textit{dest}
issues |\childdocof{|\textit{main}|}| right away,
and no further \textsf{childdoc} directives will be processed.

%%%%%%%%%%%%%%%%%%%%%%%%%%%%%%%%%%%%%%%%
\DescribeMacro{\...prefix}
In the alternative form |\childdocforwardprefix|,
%
\begin{center}
\begin{tabular}{l}
|\input{childdoc.def}|\\
|\childdocforwardprefix[|\textit{main}|]{|\textit{prefix}|}{|\textit{dest}|}|
\end{tabular}
\end{center}
%
the destination file is determined by a pattern
depending on the current file:
To make this work, the current file must be called
`{\textit{prefix}\hspace{0.2em}\textit{suffix}}'
with \textit{prefix} matching precisely the argument.
Processing is then passed on to the file
`{\textit{dest}\hspace{0.2em}\textit{suffix}}'.
Surely, the same effect is achieved by
directly specifying the
argument `{\textit{dest}\hspace{0.2em}\textit{suffix}}'
in the first form.
However, that requires to set up a different file
for each child. With the alternative form of the command
all these files can have exactly the same content
which simplifies setting them up and maintaining them.

For example, the following file |draft.tex|
with a compilation flag |\version| as described in \secref{sec:flags}
compiles the main document as a draft:
%
\begin{center}
\begin{tabular}{l}
|\def\version{draft}|\\
|\input{childdoc.def}|\\
|\childdocforward{|\textit{main}|}|
\end{tabular}
\end{center}
%
Likewise, the following files |final|\textit{nn}|.tex|
compile the final version of the child document
|child|\textit{nn}|.tex|:
%
\begin{center}
\begin{tabular}{l}
|\def\version{final}|\\
|\input{childdoc.def}|\\
|\childdocforwardprefix{final}{child}|
\end{tabular}
\end{center}
%

Note that when several versions of a main file and/or of each child file
are to be generated, it may be convenient to set up a |Makefile| or
shell script to automatise the process.

%%%%%%%%%%%%%%%%%%%%%%%%%%%%%%%%%%%%%%%%%%%%%%%%%%%%%%%%%%%%%%%%%%%%%%%%%%%%%%%%
\subsection{Command Line Processing}
\label{sec:commandline}

The effect of redirection files can also be achieved by invoking
the \LaTeX{} compiler with a more elaborate command line.
Most conveniently this should be done as part
of a shell script or a |Makefile|.

When using \textsf{childdoc} in the main file, the following
command lines effectively perform a redirection
(note that depending on the shell being used,
backslashes may have to be doubled: `|\|' $\to$ `|\\|'):
%
\begin{center}
|... -jobname "|\textit{target}|" |\\|"|[\textit{flags}]%
|\input{childdoc.def}\childdocforward[|\textit{main}|]{|\textit{dest}|}"|
\end{center}
%
Here \textit{target} is the name of the output file,
\textit{main} is the name of the main file
and \textit{dest} is the name of the main or child file to be processed
(all filenames without extensions).
The optional argument \textit{main} can be omitted
if \textit{main} matches \textit{dest}.
Optionally, compilation \textit{flags} can be defined via |\def| commands.
This command line makes the \TeX{} engine believe
it is compiling the file \textit{target}
whose content is specified as the latter parameter.
The provided code then forwards the processing to
\textit{main} or \textit{dest} as described in \secref{sec:forward}.

%%%%%%%%%%%%%%%%%%%%%%%%%%%%%%%%%%%%%%%%%%%%%%%%%%%%%%%%%%%%%%%%%%%%%%%%%%%%%%%%
\subsection{Include by Input}
\label{sec:input}

Including child documents by |\include| has some restrictions by design.
Most notably, the content of a child document always occupies
its own set of pages; pages cannot be shared between child documents.
Usually, this behaviour makes perfect sense
because each child document contain an essential part of the document.
However, in some situations it may be desirable to compose
a document from a collection of parts
without having mandatory page breaks between then.
For this case, the package
provides a mechanism to include parts
by |\input| which can also be processed individually.
However, by construction this mechanism
requires manual handling of the content to be output.

%%%%%%%%%%%%%%%%%%%%%%%%%%%%%%%%%%%%%%%%
\DescribeMacro{\ifchilddocmanual}
The main file should be prepared as usual, see \secref{sec:include}.
However, the document body must make a distinction
between processing of an individual part and of the main document, e.g.:
%
\begin{center}
\begin{tabular}{l}
|\ifchilddocmanual|\\
|\input{\childdocname}|\\
|\||else|\\
\textit{document body with }|\input{|\textit{part}|}|\\
|\||fi|
\end{tabular}
\end{center}
%
The conditional |\ifchilddocmanual| is true whenever
a part to be included by |\input| is being compiled,
and the name of the part is stored in |\childdocname|.

%%%%%%%%%%%%%%%%%%%%%%%%%%%%%%%%%%%%%%%%
\DescribeMacro{\childdocby}
Each part to be included by |\input| should start with:
%
\begin{center}
\begin{tabular}{l}
|\input{childdoc.def}|\\
|\childdocby{|\textit{main}|}|\\
\end{tabular}
\end{center}
%
The directive |\childdocby| is similar to |\childdocof|
described in \secref{sec:include},
but the subsequent selection of content must be done manually.
To that end, both |\ifchilddoc| and |\ifchilddocmanual|
will be true upon processing of a part,
and the name of the part is stored in |\childdocname|.
Note that |\jobname| will be set to the filename of the current part
so that each part receives an individual |.aux| file
that does not interfere with the |.aux| file(s) of the main document.
This behaviour can be altered by the alternative form
|\childdocby[*]{|\textit{main}|}| (with a non-empty optional argument)
which uses the |.aux| file of the main document
by setting |\jobname| to \textit{main}.

%%%%%%%%%%%%%%%%%%%%%%%%%%%%%%%%%%%%%%%%%%%%%%%%%%%%%%%%%%%%%%%%%%%%%%%%%%%%%%%%
\subsection{Driver Development}
\label{sec:driver}

The \textsf{childdoc} mechanism can also be use for the development
of definition files such as \LaTeX{} styles or classes.
This case differs from the above setup with multiple parts
included by |\include| in that no |\includeonly| should be invoked.
This can be achieved by starting the include file
(before |\ProvidesPackage|) with:
%
\begin{center}
\begin{tabular}{l}
|\input{childdoc.def}|\\
|\childdocforward{|\textit{main}|}|\\
\end{tabular}
\end{center}
%
or alternatively with:
%
\begin{center}
\begin{tabular}{l}
|\input{childdoc.def}|\\
|\childdocby{|\textit{main}|}|\\
\end{tabular}
\end{center}
%
Both forms have slightly different effects as described above.
The main file is prepared as usual, see \secref{sec:include}.

%%%%%%%%%%%%%%%%%%%%%%%%%%%%%%%%%%%%%%%%%%%%%%%%%%%%%%%%%%%%%%%%%%%%%%%%%%%%%%%%
\subsection{Legacy Detection}
\label{sec:detection}

The directive |\childdocmain| in the main file can detect
whether the complete document or merely a child is to be compiled
even without using the directive |\childdocof|.
This method is deprecated because it is less robust
and there is no compelling reason to use it;
it is merely provided for backward compatibility
and it may be removed in future versions.

If the detection mechanism is to be used,
it is mandatory to correctly specify
the filename of the main file as the argument of |\childdocmain|:
%
\begin{center}
\begin{tabular}{l}
|\input{childdoc.def}|\\
|\childdocmain{|\textit{main}|}|\\
\end{tabular}
\end{center}
%
If |\jobname| does not match the argument \textit{main} of |\childdocmain|,
it is assumed that |\jobname| points to the child file to be compiled.
When using |\childdocmain| with the main file specified as argument,
it suffices to start a child file
with just |\input{|\textit{main}|}|
without loading of the package and using |\childdocof|.
If instead all processing is done
with the appropriate \textsf{childdoc} directives,
the argument of \textit{main} of |\childdocmain| can be empty.

An alternative version of the command line processing described
in \secref{sec:commandline} using the detection mechanism reads:
%
\begin{center}
|... -jobname "|\textit{target}|" "|[\textit{flags}]%
[|\def\jobname{|\textit{dest}|}|]|\input{|\textit{main}|}"|
\end{center}

%%%%%%%%%%%%%%%%%%%%%%%%%%%%%%%%%%%%%%%%%%%%%%%%%%%%%%%%%%%%%%%%%%%%%%%%%%%%%%%%
\subsection{Manual Code}
\label{sec:manual}

In case one cannot be certain whether the definitions file |childdoc.def|
is installed on the target \TeX{} distribution
and one prefers not to ship it,
it is conceivable to paste a few relevant commands into the sources.

To that end, drop all statements |\input{childdoc.def}|
and perform the replacements as outlined below.
Instead of |\childdocmain{|\textit{main}|}| add the following code
to the top of the main file:
%
\begin{center}
\begin{tabular}{l}
|\||ifdefined\childdocname\endinput\||fi\newif\ifchilddoc|\\
|\edef\childdocname{\scantokens\expandafter{\jobname\noexpand}}|\\
|\def\childdocmain{|\textit{main}|}\||ifx\childdocmain\childdocname\||else|\\
|\childdoctrue\includeonly{\childdocname}\let\jobname\childdocmain\||fi|\\
\end{tabular}
\end{center}
%
Instead of |\childdocof{|\textit{main}|}| just include the main file
at the top of each child file:
%
\begin{center}
|\input{|\textit{main}|}|
\end{center}
%
A simple redirection |\childdocforward{|\textit{dest}|}| is achieved by:
%
\begin{center}
|\def\jobname{|\textit{dest}|}\input{\jobname}|
\end{center}
%
The redirection with prefix
|\childdocforwardprefix[|\textit{prefix}|]{|\textit{dest}|}|
is accomplished by:
%
\begin{center}
\begin{tabular}{l}
|{\edef\jobname{\scantokens\expandafter{\jobname\noexpand}}|\\
|\def\redirectjob |\textit{prefix}|#1~~~{\gdef\jobname{|\textit{dest}|#1}}|\\
|\expandafter\redirectjob\jobname~~~}\input{\jobname}|
\end{tabular}
\end{center}

In an alternative approach,
child documents can be compiled by a specific command line
without additional code or specific definitions:
%
\begin{center}
|... -jobname "|\textit{target}|" "|[\textit{flags}]%
|\includeonly{|\textit{dest}|}\input{|\textit{main}|}"|
\end{center}
%

%%%%%%%%%%%%%%%%%%%%%%%%%%%%%%%%%%%%%%%%%%%%%%%%%%%%%%%%%%%%%%%%%%%%%%%%%%%%%%%%
%%%%%%%%%%%%%%%%%%%%%%%%%%%%%%%%%%%%%%%%%%%%%%%%%%%%%%%%%%%%%%%%%%%%%%%%%%%%%%%%
\section{Information}

%%%%%%%%%%%%%%%%%%%%%%%%%%%%%%%%%%%%%%%%%%%%%%%%%%%%%%%%%%%%%%%%%%%%%%%%%%%%%%%%
\subsection{Copyright}

Copyright \copyright{} 2017--2018 Niklas Beisert

This work may be distributed and/or modified under the
conditions of the \LaTeX{} Project Public License, either version 1.3
of this license or (at your option) any later version.
The latest version of this license is in
  \url{http://www.latex-project.org/lppl.txt}
and version 1.3 or later is part of all distributions of \LaTeX{}
version 2005/12/01 or later.

This work has the LPPL maintenance status `maintained'.

The Current Maintainer of this work is Niklas Beisert.

This work consists of the files |README.txt|, |childdoc.ins| and |childdoc.dtx|
as well as the derived files |childdoc.def|, |cdocsamp.tex|
with |cdocsch1.tex|, |cdocsch2.tex|, |cdocspt3.tex|, |cdocspt4.tex|,
|cdocsdrf.tex|, |cdocsfn1.tex|, |cdocsfn2.tex|
as well as |childdoc.pdf|.

%%%%%%%%%%%%%%%%%%%%%%%%%%%%%%%%%%%%%%%%%%%%%%%%%%%%%%%%%%%%%%%%%%%%%%%%%%%%%%%%
\subsection{Files and Installation}

The package consists of the files:
%
\begin{center}
\begin{tabular}{ll}
    |README.txt|   & readme file \\
    |childdoc.ins| & installation file \\
    |childdoc.dtx| & source file \\
    |childdoc.def| & definition file \\
    |cdocsamp.tex| & sample main file \\
    |cdocsch1.tex| & sample include file \\
    |cdocsch2.tex| & sample include file \\
    |cdocspt3.tex| & sample part file \\
    |cdocspt4.tex| & sample part file \\
    |cdocsdrf.tex| & sample redirection file \\
    |cdocsfn1.tex| & sample redirection file \\
    |cdocsfn2.tex| & sample redirection file \\
    |childdoc.pdf| & manual
\end{tabular}
\end{center}
%
The distribution consists of the files
|README.txt|, |childdoc.ins| and |childdoc.dtx|.
%
\begin{itemize}
\item
Run (pdf)\LaTeX{} on |childdoc.dtx|
to compile the manual |childdoc.pdf| (this file).
\item
Run \LaTeX{} on |childdoc.ins| to create the definitions file |childdoc.def|
and the sample |cdocsamp.tex| with include files
|cdocsch1.tex|, |cdocsch2.tex|, |cdocspt3.tex|, |cdocspt4.tex|,
|cdocsdrf.tex|, |cdocsfn1.tex|, |cdocsfn2.tex|.
Then copy the file |childdoc.def| to an appropriate directory of your \LaTeX{}
distribution, e.g.\ \textit{texmf-root}|/tex/latex/childdoc|.
\end{itemize}

%%%%%%%%%%%%%%%%%%%%%%%%%%%%%%%%%%%%%%%%%%%%%%%%%%%%%%%%%%%%%%%%%%%%%%%%%%%%%%%%
\subsection{Related CTAN Packages}

There are several other packages which offer a similar functionality:
%
\begin{itemize}
\item
The packages
\href{http://ctan.org/pkg/docmute}{\textsf{docmute}},
\href{http://ctan.org/pkg/includex}{\textsf{includex}} and
\href{http://ctan.org/pkg/standalone}{\textsf{standalone}}
provide commands to include only the document body of
a child file thus allowing both files to be compiled individually.
\item
The packages \href{http://ctan.org/pkg/subdocs}{\textsf{subdocs}}
and \href{http://ctan.org/pkg/subfiles}{\textsf{subfiles}}
provide structures in which the main and child documents can be
encapsulated and allowing them to be compiled individually.
The inclusion mechanism is different from the conventional |\include|.
\item
The package \href{http://ctan.org/pkg/combine}{\textsf{combine}}
is an elaborate solution to combine several documents into one.
\end{itemize}
%
See also the CTAN topic \href{http://ctan.org/topic/subdocs}{\textsf{subdocs}}
for further related packages.
The present package differs from the above solutions in that
a document structure constructed with the conventional |\include| mechanism
just needs two extra commands at the top of every file
such that all constituent files can be compiled individually.

%%%%%%%%%%%%%%%%%%%%%%%%%%%%%%%%%%%%%%%%%%%%%%%%%%%%%%%%%%%%%%%%%%%%%%%%%%%%%%%%
%\subsection{Feature Suggestions}
%
%The following is a list of features which may be useful for future
%versions of this package:
%%
%\begin{itemize}
%\item
%\ldots
%\end{itemize}

%%%%%%%%%%%%%%%%%%%%%%%%%%%%%%%%%%%%%%%%%%%%%%%%%%%%%%%%%%%%%%%%%%%%%%%%%%%%%%%%
\subsection{Revision History}

%%%%%%%%%%%%%%%%%%%%%%%%%%%%%%%%%%%%%%%%
\paragraph{v2.0:} 2018/12/30

\begin{itemize}
\item
immediate forward processing
\item
added |\childdocby| mechanism
\item
manual restructured
\end{itemize}

%%%%%%%%%%%%%%%%%%%%%%%%%%%%%%%%%%%%%%%%
\paragraph{v1.6:} 2018/01/17

\begin{itemize}
\item
application for development of include files
\item
corrections to manual
\end{itemize}

%%%%%%%%%%%%%%%%%%%%%%%%%%%%%%%%%%%%%%%%
\paragraph{v1.5:} 2017/05/21

\begin{itemize}
\item
more complete structuring introduced
\item
|\childdocof| introduced
\item
|\childdoc| renamed to |\childdocmain|
\item
|\childredirect| renamed to |\childdocforward| and |\childdocforwardprefix|
and functionality expanded
\end{itemize}

%%%%%%%%%%%%%%%%%%%%%%%%%%%%%%%%%%%%%%%%
\paragraph{v1.0:} 2017/04/27

\begin{itemize}
\item
manual and install package
\item
first version published on CTAN
\end{itemize}

%%%%%%%%%%%%%%%%%%%%%%%%%%%%%%%%%%%%%%%%
\paragraph{v0.6:} 2017/04/26

\begin{itemize}
\item
redirection mechanism added
\end{itemize}

%%%%%%%%%%%%%%%%%%%%%%%%%%%%%%%%%%%%%%%%
\paragraph{v0.5:} 2017/04/26

\begin{itemize}
\item
functionality in definition file
\end{itemize}


%%%%%%%%%%%%%%%%%%%%%%%%%%%%%%%%%%%%%%%%%%%%%%%%%%%%%%%%%%%%%%%%%%%%%%%%%%%%%%%%
%%%%%%%%%%%%%%%%%%%%%%%%%%%%%%%%%%%%%%%%%%%%%%%%%%%%%%%%%%%%%%%%%%%%%%%%%%%%%%%%
%%%%%%%%%%%%%%%%%%%%%%%%%%%%%%%%%%%%%%%%%%%%%%%%%%%%%%%%%%%%%%%%%%%%%%%%%%%%%%%%
\appendix

\settowidth\MacroIndent{\rmfamily\scriptsize 000\ }

 \DocInput{childdoc.dtx}

\end{document}
%</driver>
% \fi
%
% %%%%%%%%%%%%%%%%%%%%%%%%%%%%%%%%%%%%%%%%%%%%%%%%%%%%%%%%%%%%%%%%%%%%%%%%%%%%%%
% %%%%%%%%%%%%%%%%%%%%%%%%%%%%%%%%%%%%%%%%%%%%%%%%%%%%%%%%%%%%%%%%%%%%%%%%%%%%%%
% \section{Sample}
%\iffalse
%<*samplemain>
%\fi
%
% The following presents a sample document
% with two chapters, two parts, a title page,
% a compile flag as well as three forwarding files to set the flag.
% It consists of eight |.tex| files:
% \begin{center}
% \begin{tabular}{ll}
% |cdocsamp.tex|&main file\\
% |cdocsch1.tex|&include file for chapter 1\\
% |cdocsch2.tex|&include file for chapter 2\\
% |cdocspt3.tex|&include file for part 3\\
% |cdocspt4.tex|&include file for part 4\\
% |cdocsdrf.tex|&forwarding file for main file in draft mode\\
% |cdocsfi1.tex|&forwarding file for final version of chapter 1\\
% |cdocsfi2.tex|&forwarding file for final version of chapter 2\\
% \end{tabular}
% \end{center}
% Each of the eight files can be compiled directly by the \LaTeX{} compiler.
%
% %%%%%%%%%%%%%%%%%%%%%%%%%%%%%%%%%%%%%%
% \paragraph{Main File.}
%
% The main file is called |cdocsamp.tex|.
%
% Load the \textsf{childdoc} definitions and
% declare the filename for the main document:
%    \begin{macrocode}
\input{childdoc.def}
\childdocmain{}
%    \end{macrocode}

% Optional override for |\version| flag:
%    \begin{macrocode}
%%\ifchilddoc\else\providecommand{\version}{draft}\fi
%    \end{macrocode}

% Define the default values for the |\version| flag
% (|final| for the main file and |draft| for childs):
%    \begin{macrocode}
\ifchilddoc
\providecommand{\version}{draft}
\else
\providecommand{\version}{final}
\fi
%    \end{macrocode}

% Load the standard document class:
%    \begin{macrocode}
\documentclass[12pt]{article}
%    \end{macrocode}

% Start the document body:
%    \begin{macrocode}
\begin{document}
%    \end{macrocode}

% Declare a title page.
% Print title, part of document being processed and version flag:
%    \begin{macrocode}
\addtocounter{page}{-1}
\begin{center}
{\LARGE\bfseries{}childdoc example\par}
\vspace{1cm}
\ifchilddoc
\ifchilddocmanual part\else chapter\fi:
`\childdocname' of `\childdocjob'\par
\else
main document: `\childdocjob'\par
\fi
version: \version\par
\end{center}
\newpage
%    \end{macrocode}

% Manually include selected file,
% otherwise process as usual:
%    \begin{macrocode}
\ifchilddocmanual
\section*{part `\childdocname'}
\input{\childdocname}
\else
%    \end{macrocode}

% Include the two chapters:
%    \begin{macrocode}
\include{cdocsch1}
\include{cdocsch2}
%    \end{macrocode}

% Include the two parts unless only chapters should be displayed:
%    \begin{macrocode}
\ifchilddoc\else
\section{part three}
\input{cdocspt3}
\section{part four}
\input{cdocspt4}
\fi
%    \end{macrocode}

% Process as usual until here:
%    \begin{macrocode}
\fi
%    \end{macrocode}

% End of document body:
%    \begin{macrocode}
\end{document}
%    \end{macrocode}
%\iffalse
%</samplemain>
%\fi
%
% %%%%%%%%%%%%%%%%%%%%%%%%%%%%%%%%%%%%%%
% \paragraph{Chapter Include Files.}
%
% The include files are called |cdocsch1.tex| and |cdocsch2.tex|.
%
%\iffalse
%<*samplechap1|samplechap2>
%\fi

% Optional override for |\version| flag:
%    \begin{macrocode}
%%\providecommand{\version}{final}
%    \end{macrocode}

% Include the main document:
%    \begin{macrocode}
\input{childdoc.def}
\childdocof{cdocsamp}
%    \end{macrocode}

%\iffalse
%</samplechap1|samplechap2>
%\fi
%
%\iffalse
%<*samplechap1>
%\fi
% Some text for chapter 1:
%    \begin{macrocode}
\section{one}
some text in chapter one
%    \end{macrocode}

%\iffalse
%</samplechap1>
%\fi
% Some text for chapter 2:
%\iffalse
%<*samplechap2>
%\fi
%    \begin{macrocode}
\section{two}
more text in chapter two
%    \end{macrocode}

%\iffalse
%</samplechap2>
%\fi
%
% %%%%%%%%%%%%%%%%%%%%%%%%%%%%%%%%%%%%%%
% \paragraph{Part Include Files.}
%
% The include files are called |cdocspt3.tex| and |cdocspt4.tex|.
%
%\iffalse
%<*samplepart3|samplepart4>
%\fi

% Optional override for |\version| flag:
%    \begin{macrocode}
%%\providecommand{\version}{final}
%    \end{macrocode}

% Include the main document:
%    \begin{macrocode}
\input{childdoc.def}
\childdocby{cdocsamp}
%    \end{macrocode}

%\iffalse
%</samplepart3|samplepart4>
%\fi
%
%\iffalse
%<*samplepart3>
%\fi
% Some text for part 3:
%    \begin{macrocode}
some text in part three
%    \end{macrocode}

%\iffalse
%</samplepart3>
%\fi
% Some text for part 4:
%\iffalse
%<*samplepart4>
%\fi
%    \begin{macrocode}
more text in part four
%    \end{macrocode}

%\iffalse
%</samplepart4>
%\fi
%
% %%%%%%%%%%%%%%%%%%%%%%%%%%%%%%%%%%%%%%
% \paragraph{Forwarding for a Complete Draft.}
%
% The following forwarding file |cdocsdrf.tex|
% compiles the main document in draft mode:
%\iffalse
%<*sampledraft>
%\fi
%    \begin{macrocode}
\def\version{draft}
\input{childdoc.def}
\childdocforward{cdocsamp}
%    \end{macrocode}

%\iffalse
%</sampledraft>
%\fi
%
% %%%%%%%%%%%%%%%%%%%%%%%%%%%%%%%%%%%%%%
% \paragraph{Forwarding for Final Version of the Chapters.}
%
% The following forwarding files |cdocsfn1.tex| and |cdocsfn2.tex|
% (with identical content)
% compile the final versions of the child documents
% |cdocsch1.tex| and |cdocsch2.tex|, respectively:
%\iffalse
%<*samplefinal>
%\fi
%    \begin{macrocode}
\def\version{final}
\input{childdoc.def}
\childdocforwardprefix[cdocsamp]{cdocsfn}{cdocsch}
%    \end{macrocode}

%\iffalse
%</samplefinal>
%\fi
%
% %%%%%%%%%%%%%%%%%%%%%%%%%%%%%%%%%%%%%%
% \paragraph{Command Line Processing.}
%
% The following three command lines generate the output files
% |cdocscld|, |cdocscl1| and |cdocscl2|
% which should be identical to
% |cdocsdrf|, |cdocsch1| and |cdocsfn2|, respectively:
% \begin{center}
% \begin{tabular}{l}
% |latex -jobname cdocscld \|\\
% |  "\def\version{draft}\input{childdoc.def}\childdocforward{cdocsamp}"|\\
% |latex -jobname cdocscl1 \|\\
% |  "\input{childdoc.def}\childdocforward[cdocsamp]{cdocsch1}"|\\
% |latex -jobname cdocscl2 \|\\
% |  "\def\version{final}\input{childdoc.def}\childdocforward{cdocsch2}"|
% \end{tabular}
% \end{center}
% Note that the trailing backslash on each first line
% merely continues the input to the second line
% (for convenient cut ant paste).
% Furthermore, the command |latex| can be replaced by any
% of its alternative versions such as |pdflatex|.
%
% %%%%%%%%%%%%%%%%%%%%%%%%%%%%%%%%%%%%%%%%%%%%%%%%%%%%%%%%%%%%%%%%%%%%%%%%%%%%%%
% %%%%%%%%%%%%%%%%%%%%%%%%%%%%%%%%%%%%%%%%%%%%%%%%%%%%%%%%%%%%%%%%%%%%%%%%%%%%%%
% \section{Implementation}
%\iffalse
%<*package>
%\fi
%
% This section describes the definitions file |childdoc.def|.

% The definitions cannot be loaded using |\usepackage| or |\RequirePackage|
% which has a mechanism to prevent loading a style file more than once.
% When loading the definitions by means of |\input|
% multiple instances have to be prevented manually:
%\iffalse
%This code needs to be before the `\ProvidesFile' directive
%which is defined at the beginning of this file.
%Therefore it is also placed there and commented out here.
%</package>
%<*discard>
%\fi
%    \begin{macrocode}
\ifdefined\childdocmain\endinput\fi
%    \end{macrocode}
%\iffalse
%</discard>
%<*package>
%\fi
%
% \macro{\ifchilddoc}
% \macro{\ifchilddocmanual}
% The conditional |\ifchilddoc| tells whether a
% child (true) or main (false) document is being compiled.
% The conditional |\ifchilddocmanual| tells whether
% the |\includeonly| mechanism is used (false) or
% the selection of child files must be performed manually (true).
% The definitions initialise to false:
%    \begin{macrocode}
\newif\ifchilddoc
\newif\ifchilddocmanual
%    \end{macrocode}

% \macro{\childdocname}
% \macro{\childdocjob}
% The macro |\childdocname| stores the name of the main document
% to be compiled. The macro |\childdocjob| stores the name of
% the document on which the \LaTeX{} compiler was originally invoked.
% The content of |\jobname| cannot be compared
% to filenames specified in the source due to different catcodes.
% The following code rescans |\jobname|, stores the result
% in |\childdocname| and saves a copy in |\childdocjob|:
%    \begin{macrocode}
\edef\childdocname{\scantokens\expandafter{\jobname\noexpand}}
\let\childdocjob\childdocname
%    \end{macrocode}

% \macro{\childdocdisable}
% The macro |\childdocdisable| prevents the main file
% from being processed more than once.
% At this stage, the main document command |\childdocmain|
% is assumed to be called once again where it should do nothing.
% Any subsequent call to it should prevent
% a secondary processing of the main document
% It overwrites the forwarding commands
% |\childdocof| and |\childdocforward|
% with empty macros to prevent further inclusions of the main document:
%    \begin{macrocode}
\newcommand{\childdocdisable}
{
  \renewcommand{\childdocmain}[1]{\renewcommand{\childdocmain}[1]{\endinput}}
  \renewcommand{\childdocof}[1]{}
  \renewcommand{\childdocby}[2][]{}
  \renewcommand{\childdocforward}[2][]{}
  \renewcommand{\childdocdisable}{}
}
%    \end{macrocode}

% \macro{\childdocmain}
% The macro |\childdocmain| is to be called at the top of the main file
% with nothing or the main filename (without extension) as argument.
% First, it breaks loops.
% If the argument is not empty and does not match |\childdocname|
% (which is set by the first inclusion of |childdoc.def|),
% |\ifchilddoc| is set to true, |\includeonly| is applied to the child file
% and |\jobname| is set to the main file
% (for proper handling of |.aux| files):
%    \begin{macrocode}
\newcommand{\childdocmain}[1]
{
  \childdocdisable\childdocmain{}
  \if?#1?\else
    \begingroup
      \def\childdoctmp{#1}
      \ifx\childdoctmp\childdocname
        \def\childdoctmp{}
      \else
        \def\childdoctmp
        {
          \childdoctrue
          \includeonly{\childdocname}
          \def\childdocjob{#1}
          \def\jobname{#1}
        }
      \fi
      \expandafter
    \endgroup
    \childdoctmp
  \fi
}
%    \end{macrocode}

% \macro{\childdocof}
% The command |\childdocof| redirects
% compilation to the main file |#1|.
%    \begin{macrocode}
\newcommand{\childdocof}[1]
{
  \childdocdisable
  \childdoctrue
  \includeonly{\childdocname}
  \def\jobname{#1}
  \def\childdocjob{#1}
  \input{#1}
}
%    \end{macrocode}

% \macro{\childdocby}
% The command |\childdocby| ....
%    \begin{macrocode}
\newcommand{\childdocby}[2][]
{
  \childdocdisable
  \childdoctrue
  \childdocmanualtrue
  \if?#1?\else
    \def\jobname{#2}
  \fi
  \def\childdocjob{#2}
  \input{#2}
  \endinput
}
%    \end{macrocode}

% \macro{\childdocforward}
% The command |\childdocforward| redirects
% compilation to the main file or
% (if the optional argument is given) a child file.
% Parameters are set as if the main file
% or a child file starting with |\childdocof| was compiled.
% Then compilation is handed over to the main file:
%    \begin{macrocode}
\newcommand{\childdocforward}[2][]
{
  \begingroup
    \if?#1?
      \def\childdoctmp
      {
        \def\childdocname{#2}
        \def\childdocjob{#2}
        \def\jobname{#2}
        \input{#2}
        \endinput
      }
    \else
      \def\childdoctmp
      {
        \childdocdisable
        \def\childdocname{#2}
        \childdoctrue
        \includeonly{#2}
        \def\childdocjob{#1}
        \def\jobname{#1}
        \input{#1}
        \endinput
      }
    \fi
    \expandafter
  \endgroup
  \childdoctmp
}
%    \end{macrocode}

% \macro{\childdocforwardprefix}
% The command |\childdocforwardprefix| redirects
% compilation to the main or a child file by means of a pattern.
% The prefix |#1| in the current filename is replaced by |#2|
% and the suffix of the current filename is kept
% (it is assumed that the filename does not contain the substring `|~~~|'
% which is used as a delimiter).
% Compilation is handed over to the new file by |\childdocforward|:
%    \begin{macrocode}
\newcommand{\childdocforwardprefix}[3][]
{
  \begingroup
    \def\childdocextract #2##1~~~{\def\childdoctmp{\childdocforward[#1]{#3##1}}}
    \expandafter\childdocextract\childdocname~~~
    \expandafter
  \endgroup
  \childdoctmp
}
%    \end{macrocode}

% \macro{\childdoc}
% The deprecated macro |\childdoc| is a legacy version of |\childdocmain|:
%    \begin{macrocode}
\newcommand{\childdoc}{\childdocmain}
%    \end{macrocode}

% \macro{\childdocredirect}
% The deprecated macro |\childdocredirect| is a legacy version
% of |\childdocforward| and |\childdocforwardprefix|:
%    \begin{macrocode}
\newcommand{\childdocredirect}[2][]
{
  \begingroup
    \if?#1?
      \def\childdoctmp{\childdocforward{#2}}
    \else
      \def\childdoctmp{\childdocforwardprefix{#1}{#2}}
    \fi
    \expandafter
  \endgroup
  \childdoctmp
}
%    \end{macrocode}

%\iffalse
%</package>
%\fi
%
\endinput

\childdocmain{}
%    \end{macrocode}

% Optional override for |\version| flag:
%    \begin{macrocode}
%%\ifchilddoc\else\providecommand{\version}{draft}\fi
%    \end{macrocode}

% Define the default values for the |\version| flag
% (|final| for the main file and |draft| for childs):
%    \begin{macrocode}
\ifchilddoc
\providecommand{\version}{draft}
\else
\providecommand{\version}{final}
\fi
%    \end{macrocode}

% Load the standard document class:
%    \begin{macrocode}
\documentclass[12pt]{article}
%    \end{macrocode}

% Start the document body:
%    \begin{macrocode}
\begin{document}
%    \end{macrocode}

% Declare a title page.
% Print title, part of document being processed and version flag:
%    \begin{macrocode}
\addtocounter{page}{-1}
\begin{center}
{\LARGE\bfseries{}childdoc example\par}
\vspace{1cm}
\ifchilddoc
\ifchilddocmanual part\else chapter\fi:
`\childdocname' of `\childdocjob'\par
\else
main document: `\childdocjob'\par
\fi
version: \version\par
\end{center}
\newpage
%    \end{macrocode}

% Manually include selected file,
% otherwise process as usual:
%    \begin{macrocode}
\ifchilddocmanual
\section*{part `\childdocname'}
\input{\childdocname}
\else
%    \end{macrocode}

% Include the two chapters:
%    \begin{macrocode}
\include{cdocsch1}
\include{cdocsch2}
%    \end{macrocode}

% Include the two parts unless only chapters should be displayed:
%    \begin{macrocode}
\ifchilddoc\else
\section{part three}
\input{cdocspt3}
\section{part four}
\input{cdocspt4}
\fi
%    \end{macrocode}

% Process as usual until here:
%    \begin{macrocode}
\fi
%    \end{macrocode}

% End of document body:
%    \begin{macrocode}
\end{document}
%    \end{macrocode}
%\iffalse
%</samplemain>
%\fi
%
% %%%%%%%%%%%%%%%%%%%%%%%%%%%%%%%%%%%%%%
% \paragraph{Chapter Include Files.}
%
% The include files are called |cdocsch1.tex| and |cdocsch2.tex|.
%
%\iffalse
%<*samplechap1|samplechap2>
%\fi

% Optional override for |\version| flag:
%    \begin{macrocode}
%%\providecommand{\version}{final}
%    \end{macrocode}

% Include the main document:
%    \begin{macrocode}
% \iffalse
%
% childdoc.dtx Copyright (C) 2017-2018 Niklas Beisert
%
% This work may be distributed and/or modified under the
% conditions of the LaTeX Project Public License, either version 1.3
% of this license or (at your option) any later version.
% The latest version of this license is in
%   http://www.latex-project.org/lppl.txt
% and version 1.3 or later is part of all distributions of LaTeX
% version 2005/12/01 or later.
%
% This work has the LPPL maintenance status `maintained'.
%
% The Current Maintainer of this work is Niklas Beisert.
%
% This work consists of the files childdoc.dtx and childdoc.ins
% and the derived files childdoc.def and cdocsamp.tex with
% cdocsch1.tex, cdocsch2.tex, cdocsdrf.tex, cdocsfn1.tex, cdocsfn2.tex.
%
%<package>\ifdefined\childdocmain\endinput\fi
%<package>\ProvidesFile{childdoc.def}[2018/12/30 v2.0 child document driver]
%<samplemain>\ProvidesFile{cdocsamp.tex}[2018/12/30 v2.0 sample for childdoc]
%<*driver>
%\ProvidesFile{childdoc.drv}[2018/12/30 v2.0 childdoc reference manual file]
\PassOptionsToClass{10pt,a4paper}{article}
\documentclass{ltxdoc}

\usepackage[margin=35mm]{geometry}
\usepackage{hyperref}
\usepackage{hyperxmp}
\usepackage[usenames]{color}

\hypersetup{colorlinks=true}
\hypersetup{pdfstartview=FitH}
\hypersetup{pdfpagemode=UseNone}
\hypersetup{pdfsource={}}
\hypersetup{pdflang={en-UK}}
\hypersetup{pdfcopyright={Copyright 2017-2018 Niklas Beisert.
  This work may be distributed and/or modified under the
  conditions of the LaTeX Project Public License, either version 1.3
  of this license or (at your option) any later version.}}
\hypersetup{pdflicenseurl={http://www.latex-project.org/lppl.txt}}
\hypersetup{pdfcontactaddress={ETH Zurich, ITP, HIT K,
  Wolfgang-Pauli-Strasse 27}}
\hypersetup{pdfcontactpostcode={8093}}
\hypersetup{pdfcontactcity={Zurich}}
\hypersetup{pdfcontactcountry={Switzerland}}
\hypersetup{pdfcontactemail={nbeisert@itp.phys.ethz.ch}}
\hypersetup{pdfcontacturl={http://people.phys.ethz.ch/\xmptilde nbeisert/}}

\newcommand{\secref}[1]{\hyperref[#1]{section \ref*{#1}}}

\parskip1ex
\parindent0pt
\let\olditemize\itemize
\def\itemize{\olditemize\parskip0pt}

\begin{document}

\title{The \textsf{childdoc} Package}
\hypersetup{pdftitle={The childdoc Package}}
\author{Niklas Beisert\\[2ex]
  Institut f\"ur Theoretische Physik\\
  Eidgen\"ossische Technische Hochschule Z\"urich\\
  Wolfgang-Pauli-Strasse 27, 8093 Z\"urich, Switzerland\\[1ex]
  \href{mailto:nbeisert@itp.phys.ethz.ch}
  {\texttt{nbeisert@itp.phys.ethz.ch}}}
\hypersetup{pdfauthor={Niklas Beisert}}
\hypersetup{pdfsubject={Manual for the LaTeX2e Package childdoc}}
\date{30 December 2018, \textsf{v2.0}}
\maketitle

\begin{abstract}\noindent
\textsf{childdoc} is a \LaTeXe{} package
that enables the direct compilation
of document sections included by |\include|
to individual files.
\end{abstract}

\begingroup
\parskip0ex
\tableofcontents
\endgroup

%%%%%%%%%%%%%%%%%%%%%%%%%%%%%%%%%%%%%%%%%%%%%%%%%%%%%%%%%%%%%%%%%%%%%%%%%%%%%%%%
%%%%%%%%%%%%%%%%%%%%%%%%%%%%%%%%%%%%%%%%%%%%%%%%%%%%%%%%%%%%%%%%%%%%%%%%%%%%%%%%
\section{Introduction}

\LaTeX{} provides a mechanism to structure a large document (such as a book)
into a main file and several child files (containing the chapters)
using the |\include| command.
This mechanism is beneficial for documents
which span hundreds of pages in order to
make the source file(s) more manageable.
Moreover, compilation can be restricted to
selected child files by means of the |\includeonly| command.
The latter feature can be used to reduce the compilation time while editing
(this was significantly more useful in the earlier days of \LaTeX{})
or to generate a smaller document which is easier to navigate.
Another application of |\includeonly| is to generate
documents consisting of selected parts of the complete document.

However, there are a few drawbacks of the plain |\include| mechanism:
\begin{itemize}
\item
The child files cannot be compiled on their own,
they can only be compiled via the main file.
A naive editing environment
(such as a text editor with an option
to have the current file processed by \LaTeX)
may require one to switch to the main file before compiling;
attempting to compile the child file produces errors.
\item
The main file must be modified (each time)
to adjust the |\includeonly| command
to the present needs. This easily leaves the main file in a messy state.
\item
The generated document will always carry the filename
of the main document. This is inconvenient if
several child files are to be compiled and
to be kept for distribution.
\end{itemize}

The present package provides a simple interface
to make child files individually compilable by \LaTeX{}.
Compiling a child file then has the same effect as compiling
the main file with an |\includeonly| command
to select the appropriate child.
Moreover the generated document will carry the name of the child
rather than the main file.
This resolves all three above issues.

This feature is meant to make the editing of books,
thesis documents and lecture notes somewhat more convenient.
However, the package can also be used efficiently for
composing a series of documents (such as exercise sheets)
which are typically distributed individually.
It then assists the author in generating the individual documents
(potentially in different versions)
as well as a document containing the collected series.
Another application is in developing style files
or other kinds of included material
where compilation of the style file could redirect
to a sample or test file.

%%%%%%%%%%%%%%%%%%%%%%%%%%%%%%%%%%%%%%%%%%%%%%%%%%%%%%%%%%%%%%%%%%%%%%%%%%%%%%%%
%%%%%%%%%%%%%%%%%%%%%%%%%%%%%%%%%%%%%%%%%%%%%%%%%%%%%%%%%%%%%%%%%%%%%%%%%%%%%%%%
\section{Usage}

First of all, the package \textsf{childdoc} is \emph{not} a standard
\LaTeXe{} |.sty| style file! Therefore it needs to be invoked in
a non-standard way.

%%%%%%%%%%%%%%%%%%%%%%%%%%%%%%%%%%%%%%%%%%%%%%%%%%%%%%%%%%%%%%%%%%%%%%%%%%%%%%%%
\subsection{Included Files}
\label{sec:include}

%%%%%%%%%%%%%%%%%%%%%%%%%%%%%%%%%%%%%%%%
\DescribeMacro{\childdocmain}
To use the package, add the commands
\begin{center}
\begin{tabular}{l}
|\input{childdoc.def}|\\
|\childdocmain{}|\\
\end{tabular}
\end{center}
at the very top of the main \LaTeX{} file,
in particular \emph{before} the |\documentclass| statement!
The argument of |\childdocmain| should be left empty
(but it must be present).

%%%%%%%%%%%%%%%%%%%%%%%%%%%%%%%%%%%%%%%%
\DescribeMacro{\childdocof}
Furthermore, add the commands
\begin{center}
\begin{tabular}{l}
|\input{childdoc.def}|\\
|\childdocof{|\textit{main}|}|\\
\end{tabular}
\end{center}
at the top of every child file \textit{child}
which is included by |\include{|\textit{child}|}|
from within the main file
(or at least for those files to be compiled individually).
The argument \textit{main} must be the filename of the main file.

There are a couple of
considerations in setting up the main and child documents:

%%%%%%%%%%%%%%%%%%%%%%%%%%%%%%%%%%%%%%%%
\paragraph{Restrictions.}

Please note the following restrictions:
\begin{itemize}
\item
|\childdocmain| must be called with one argument \textit{main}
to ensure compatibility with earlier version of the package.
It must either be empty (|\childdocmain{}|)
or precisely match the filename of the main file in which it is specified.
See \secref{sec:detection} for further information.
\item
The filename \textit{main} must be specified without the |.tex| extension.
\item
The filename \textit{main} is case sensitive
(even in case-insensitive file systems)
due to internal string comparison.
\item
The argument \textit{main} should be fully expanded, it cannot be a macro.
\item
Subdirectories and special characters should be avoided in filenames.
\item
The command |\childdocmain{|\textit{main}|}| must be followed by a whitespace.
It should not be followed immediately by another command
or by a comment mark `|%|'.
This is because the \TeX{} parser reads the token immediately following
the argument of |\childdocmain| and puts it
at the beginning of every child section;
however, a white\-space is ignored.
\end{itemize}

%%%%%%%%%%%%%%%%%%%%%%%%%%%%%%%%%%%%%%%%
\paragraph{Content of Main File.}

It is advisable to place all content in the child files included by |\include|.
Any output contained in the main file will appear in all child documents
unless suppressed manually;
it cannot be suppressed automatically by the |\includeonly| directive
and thus should normally be avoided.
A method to include some content in the main file
by means of conditional processing is described in \secref{sec:conditional}.

%%%%%%%%%%%%%%%%%%%%%%%%%%%%%%%%%%%%%%%%
\paragraph{Page Numbering.}

When only a part of the document is compiled,
the appropriate numbering of pages
(as well as other status parameters)
is determined from the |.aux| files.
The latter contain information from previous passes.
However this information needs to propagate through
all intermediate child documents.
Therefore the page numbering in child documents may well
be inconsistent until the complete document is compiled at least once.

A useful (if unconventional) way to always ensure a consistent
page numbering is to restart the numbering in each child document
and denote the pages by `\textit{child}|.|\textit{page}'
where \textit{child} represents the chapter/section number of the child file.
This can be achieved by the command
|\numberwithin{page}{|\textit{child}|}|
of the \textsf{amsmath} package
where \textit{child} can be |chapter| or |section|
depending on the chosen structuring.
Alternatively, one can modify the macro |\thepage| appropriately
and reset the counter |page| at the start of each child file.

%%%%%%%%%%%%%%%%%%%%%%%%%%%%%%%%%%%%%%%%%%%%%%%%%%%%%%%%%%%%%%%%%%%%%%%%%%%%%%%%
\subsection{Conditional Processing}
\label{sec:conditional}

The package provides a mechanism to compile different versions
of a document. To customise the versions further some conditional processing
can come in handy to distinguish which version is being compiled.
The package provides two macros to describe the compilation context:

%%%%%%%%%%%%%%%%%%%%%%%%%%%%%%%%%%%%%%%%
\DescribeMacro{\ifchilddoc}
The conditional |\ifchilddoc| distinguishes between the compilation of
child documents and the main document:
%
\begin{center}
|\ifchilddoc |\textit{child-code}| |[|\||else |\textit{main-code}]| \||fi|
\end{center}

%%%%%%%%%%%%%%%%%%%%%%%%%%%%%%%%%%%%%%%%
\DescribeMacro{\childdocname}
\DescribeMacro{\childdocjob}
The macro |\childdocname| contains the filename (without extension)
of the main or child file being processed.
Note that |\childdocjob| will always contain the name of the main file.

%%%%%%%%%%%%%%%%%%%%%%%%%%%%%%%%%%%%%%%%
\paragraph{Title Page.}

Conditional processing can be used to include a title or banner page
in the main document when proper precautions are taken.
Importantly, the code in the main file should ensure that the page counter
(as well as other status parameters which are stored in the |.aux| files)
takes the same value after the conditional processing.
Otherwise the page numbers may take divergent values
depending on which part is compiled.

For example, a title page could be declared by:
%
\begin{center}
\begin{tabular}{l}
|\ifchilddoc\||else|\\
|\addtocounter{page}{-1}|\\
\textit{code for title page}\\
|\newpage|\\
|\||fi|
\end{tabular}
\end{center}
%
A banner page for the child documents can be generated by:
%
\begin{center}
\begin{tabular}{l}
|\ifchilddoc|\\
|\addtocounter{page}{-1}|\\
\textit{code for banner page}\\
|\newpage|\\
|\||fi|
\end{tabular}
\end{center}
%
Here one could write a message such as:
\begin{center}
|This is the part \childdocname{} of \childdocjob{}.|
\end{center}

%%%%%%%%%%%%%%%%%%%%%%%%%%%%%%%%%%%%%%%%%%%%%%%%%%%%%%%%%%%%%%%%%%%%%%%%%%%%%%%%
\subsection{Flags}
\label{sec:flags}

The package makes it easy to generate different versions
of the main or child documents.
To this end compilation flags can be defined
and assigned different default values.
They will be particularly useful in conjunction
with the forwarding mechanism described in \secref{sec:forward}.

For example, it may be useful to have a flag |\version|
which can be set to |draft| or |final|.
The document source will contain some conditional code
depending on the value of |\version|.
Suppose further, the flag should default to |final| for the main file
and to |draft| for child files
which is a natural assignment for editing the document.
This is achieved by placing the following code
in the preamble of the main document
(below the |\childdocmain| directive):
%
\begin{center}
\begin{tabular}{l}
|\ifchilddoc|\\
|\providecommand{\version}{draft}|\\
|\||else|\\
|\providecommand{\version}{final}|\\
|\||fi|
\end{tabular}
\end{center}
%
The definition by |\providecommand| makes sure
that previous definitions are not overwritten.
Further statements |\providecommand{\version}{...}|
can thus be added before the above code to override it.

For the main file, one might add a line
(between |\childdocmain| and the above block)
%
\begin{center}
|%\ifchilddoc\||else\providecommand{\version}{draft}\||fi|
\end{center}
%
which can be uncommented to produce a draft version.
Likewise one can add a line to the very top of a child file
(above the |\childdocof{|\textit{main}|}| directive)
%
\begin{center}
|%\providecommand{\version}{final}|
\end{center}
%
which can be uncommented to produce the final version of this child document.

%%%%%%%%%%%%%%%%%%%%%%%%%%%%%%%%%%%%%%%%%%%%%%%%%%%%%%%%%%%%%%%%%%%%%%%%%%%%%%%%
\subsection{Forwarding}
\label{sec:forward}

Different versions of the main or child documents
using compilation flags as described in \secref{sec:flags}
can be (permanently) stored in different files
for convenient compilation, viewing and distribution.
To this end, the package defines a command
to pass on compilation to a different file:

%%%%%%%%%%%%%%%%%%%%%%%%%%%%%%%%%%%%%%%%
\DescribeMacro{\childdocforward}
The command |\childdocforward| redirects processing to
another source file:
%
\begin{center}
\begin{tabular}{l}
|\input{childdoc.def}|\\
|\childdocforward[|\textit{main}|]{|\textit{dest}|}|\\
\end{tabular}
\end{center}
%
The argument \textit{dest} is the destination file
(without extension).
It should be the main file or one of the child files.
Note that further \textsf{childdoc} directives
such as |\childdocof| and |\childdocforward|
in the indicated file will be processed in this form.
The optional argument \textit{main}
passes on directly to the main file \textit{main}
while pretending to compile the child \textit{dest}.
This form behaves as if \textit{dest}
issues |\childdocof{|\textit{main}|}| right away,
and no further \textsf{childdoc} directives will be processed.

%%%%%%%%%%%%%%%%%%%%%%%%%%%%%%%%%%%%%%%%
\DescribeMacro{\...prefix}
In the alternative form |\childdocforwardprefix|,
%
\begin{center}
\begin{tabular}{l}
|\input{childdoc.def}|\\
|\childdocforwardprefix[|\textit{main}|]{|\textit{prefix}|}{|\textit{dest}|}|
\end{tabular}
\end{center}
%
the destination file is determined by a pattern
depending on the current file:
To make this work, the current file must be called
`{\textit{prefix}\hspace{0.2em}\textit{suffix}}'
with \textit{prefix} matching precisely the argument.
Processing is then passed on to the file
`{\textit{dest}\hspace{0.2em}\textit{suffix}}'.
Surely, the same effect is achieved by
directly specifying the
argument `{\textit{dest}\hspace{0.2em}\textit{suffix}}'
in the first form.
However, that requires to set up a different file
for each child. With the alternative form of the command
all these files can have exactly the same content
which simplifies setting them up and maintaining them.

For example, the following file |draft.tex|
with a compilation flag |\version| as described in \secref{sec:flags}
compiles the main document as a draft:
%
\begin{center}
\begin{tabular}{l}
|\def\version{draft}|\\
|\input{childdoc.def}|\\
|\childdocforward{|\textit{main}|}|
\end{tabular}
\end{center}
%
Likewise, the following files |final|\textit{nn}|.tex|
compile the final version of the child document
|child|\textit{nn}|.tex|:
%
\begin{center}
\begin{tabular}{l}
|\def\version{final}|\\
|\input{childdoc.def}|\\
|\childdocforwardprefix{final}{child}|
\end{tabular}
\end{center}
%

Note that when several versions of a main file and/or of each child file
are to be generated, it may be convenient to set up a |Makefile| or
shell script to automatise the process.

%%%%%%%%%%%%%%%%%%%%%%%%%%%%%%%%%%%%%%%%%%%%%%%%%%%%%%%%%%%%%%%%%%%%%%%%%%%%%%%%
\subsection{Command Line Processing}
\label{sec:commandline}

The effect of redirection files can also be achieved by invoking
the \LaTeX{} compiler with a more elaborate command line.
Most conveniently this should be done as part
of a shell script or a |Makefile|.

When using \textsf{childdoc} in the main file, the following
command lines effectively perform a redirection
(note that depending on the shell being used,
backslashes may have to be doubled: `|\|' $\to$ `|\\|'):
%
\begin{center}
|... -jobname "|\textit{target}|" |\\|"|[\textit{flags}]%
|\input{childdoc.def}\childdocforward[|\textit{main}|]{|\textit{dest}|}"|
\end{center}
%
Here \textit{target} is the name of the output file,
\textit{main} is the name of the main file
and \textit{dest} is the name of the main or child file to be processed
(all filenames without extensions).
The optional argument \textit{main} can be omitted
if \textit{main} matches \textit{dest}.
Optionally, compilation \textit{flags} can be defined via |\def| commands.
This command line makes the \TeX{} engine believe
it is compiling the file \textit{target}
whose content is specified as the latter parameter.
The provided code then forwards the processing to
\textit{main} or \textit{dest} as described in \secref{sec:forward}.

%%%%%%%%%%%%%%%%%%%%%%%%%%%%%%%%%%%%%%%%%%%%%%%%%%%%%%%%%%%%%%%%%%%%%%%%%%%%%%%%
\subsection{Include by Input}
\label{sec:input}

Including child documents by |\include| has some restrictions by design.
Most notably, the content of a child document always occupies
its own set of pages; pages cannot be shared between child documents.
Usually, this behaviour makes perfect sense
because each child document contain an essential part of the document.
However, in some situations it may be desirable to compose
a document from a collection of parts
without having mandatory page breaks between then.
For this case, the package
provides a mechanism to include parts
by |\input| which can also be processed individually.
However, by construction this mechanism
requires manual handling of the content to be output.

%%%%%%%%%%%%%%%%%%%%%%%%%%%%%%%%%%%%%%%%
\DescribeMacro{\ifchilddocmanual}
The main file should be prepared as usual, see \secref{sec:include}.
However, the document body must make a distinction
between processing of an individual part and of the main document, e.g.:
%
\begin{center}
\begin{tabular}{l}
|\ifchilddocmanual|\\
|\input{\childdocname}|\\
|\||else|\\
\textit{document body with }|\input{|\textit{part}|}|\\
|\||fi|
\end{tabular}
\end{center}
%
The conditional |\ifchilddocmanual| is true whenever
a part to be included by |\input| is being compiled,
and the name of the part is stored in |\childdocname|.

%%%%%%%%%%%%%%%%%%%%%%%%%%%%%%%%%%%%%%%%
\DescribeMacro{\childdocby}
Each part to be included by |\input| should start with:
%
\begin{center}
\begin{tabular}{l}
|\input{childdoc.def}|\\
|\childdocby{|\textit{main}|}|\\
\end{tabular}
\end{center}
%
The directive |\childdocby| is similar to |\childdocof|
described in \secref{sec:include},
but the subsequent selection of content must be done manually.
To that end, both |\ifchilddoc| and |\ifchilddocmanual|
will be true upon processing of a part,
and the name of the part is stored in |\childdocname|.
Note that |\jobname| will be set to the filename of the current part
so that each part receives an individual |.aux| file
that does not interfere with the |.aux| file(s) of the main document.
This behaviour can be altered by the alternative form
|\childdocby[*]{|\textit{main}|}| (with a non-empty optional argument)
which uses the |.aux| file of the main document
by setting |\jobname| to \textit{main}.

%%%%%%%%%%%%%%%%%%%%%%%%%%%%%%%%%%%%%%%%%%%%%%%%%%%%%%%%%%%%%%%%%%%%%%%%%%%%%%%%
\subsection{Driver Development}
\label{sec:driver}

The \textsf{childdoc} mechanism can also be use for the development
of definition files such as \LaTeX{} styles or classes.
This case differs from the above setup with multiple parts
included by |\include| in that no |\includeonly| should be invoked.
This can be achieved by starting the include file
(before |\ProvidesPackage|) with:
%
\begin{center}
\begin{tabular}{l}
|\input{childdoc.def}|\\
|\childdocforward{|\textit{main}|}|\\
\end{tabular}
\end{center}
%
or alternatively with:
%
\begin{center}
\begin{tabular}{l}
|\input{childdoc.def}|\\
|\childdocby{|\textit{main}|}|\\
\end{tabular}
\end{center}
%
Both forms have slightly different effects as described above.
The main file is prepared as usual, see \secref{sec:include}.

%%%%%%%%%%%%%%%%%%%%%%%%%%%%%%%%%%%%%%%%%%%%%%%%%%%%%%%%%%%%%%%%%%%%%%%%%%%%%%%%
\subsection{Legacy Detection}
\label{sec:detection}

The directive |\childdocmain| in the main file can detect
whether the complete document or merely a child is to be compiled
even without using the directive |\childdocof|.
This method is deprecated because it is less robust
and there is no compelling reason to use it;
it is merely provided for backward compatibility
and it may be removed in future versions.

If the detection mechanism is to be used,
it is mandatory to correctly specify
the filename of the main file as the argument of |\childdocmain|:
%
\begin{center}
\begin{tabular}{l}
|\input{childdoc.def}|\\
|\childdocmain{|\textit{main}|}|\\
\end{tabular}
\end{center}
%
If |\jobname| does not match the argument \textit{main} of |\childdocmain|,
it is assumed that |\jobname| points to the child file to be compiled.
When using |\childdocmain| with the main file specified as argument,
it suffices to start a child file
with just |\input{|\textit{main}|}|
without loading of the package and using |\childdocof|.
If instead all processing is done
with the appropriate \textsf{childdoc} directives,
the argument of \textit{main} of |\childdocmain| can be empty.

An alternative version of the command line processing described
in \secref{sec:commandline} using the detection mechanism reads:
%
\begin{center}
|... -jobname "|\textit{target}|" "|[\textit{flags}]%
[|\def\jobname{|\textit{dest}|}|]|\input{|\textit{main}|}"|
\end{center}

%%%%%%%%%%%%%%%%%%%%%%%%%%%%%%%%%%%%%%%%%%%%%%%%%%%%%%%%%%%%%%%%%%%%%%%%%%%%%%%%
\subsection{Manual Code}
\label{sec:manual}

In case one cannot be certain whether the definitions file |childdoc.def|
is installed on the target \TeX{} distribution
and one prefers not to ship it,
it is conceivable to paste a few relevant commands into the sources.

To that end, drop all statements |\input{childdoc.def}|
and perform the replacements as outlined below.
Instead of |\childdocmain{|\textit{main}|}| add the following code
to the top of the main file:
%
\begin{center}
\begin{tabular}{l}
|\||ifdefined\childdocname\endinput\||fi\newif\ifchilddoc|\\
|\edef\childdocname{\scantokens\expandafter{\jobname\noexpand}}|\\
|\def\childdocmain{|\textit{main}|}\||ifx\childdocmain\childdocname\||else|\\
|\childdoctrue\includeonly{\childdocname}\let\jobname\childdocmain\||fi|\\
\end{tabular}
\end{center}
%
Instead of |\childdocof{|\textit{main}|}| just include the main file
at the top of each child file:
%
\begin{center}
|\input{|\textit{main}|}|
\end{center}
%
A simple redirection |\childdocforward{|\textit{dest}|}| is achieved by:
%
\begin{center}
|\def\jobname{|\textit{dest}|}\input{\jobname}|
\end{center}
%
The redirection with prefix
|\childdocforwardprefix[|\textit{prefix}|]{|\textit{dest}|}|
is accomplished by:
%
\begin{center}
\begin{tabular}{l}
|{\edef\jobname{\scantokens\expandafter{\jobname\noexpand}}|\\
|\def\redirectjob |\textit{prefix}|#1~~~{\gdef\jobname{|\textit{dest}|#1}}|\\
|\expandafter\redirectjob\jobname~~~}\input{\jobname}|
\end{tabular}
\end{center}

In an alternative approach,
child documents can be compiled by a specific command line
without additional code or specific definitions:
%
\begin{center}
|... -jobname "|\textit{target}|" "|[\textit{flags}]%
|\includeonly{|\textit{dest}|}\input{|\textit{main}|}"|
\end{center}
%

%%%%%%%%%%%%%%%%%%%%%%%%%%%%%%%%%%%%%%%%%%%%%%%%%%%%%%%%%%%%%%%%%%%%%%%%%%%%%%%%
%%%%%%%%%%%%%%%%%%%%%%%%%%%%%%%%%%%%%%%%%%%%%%%%%%%%%%%%%%%%%%%%%%%%%%%%%%%%%%%%
\section{Information}

%%%%%%%%%%%%%%%%%%%%%%%%%%%%%%%%%%%%%%%%%%%%%%%%%%%%%%%%%%%%%%%%%%%%%%%%%%%%%%%%
\subsection{Copyright}

Copyright \copyright{} 2017--2018 Niklas Beisert

This work may be distributed and/or modified under the
conditions of the \LaTeX{} Project Public License, either version 1.3
of this license or (at your option) any later version.
The latest version of this license is in
  \url{http://www.latex-project.org/lppl.txt}
and version 1.3 or later is part of all distributions of \LaTeX{}
version 2005/12/01 or later.

This work has the LPPL maintenance status `maintained'.

The Current Maintainer of this work is Niklas Beisert.

This work consists of the files |README.txt|, |childdoc.ins| and |childdoc.dtx|
as well as the derived files |childdoc.def|, |cdocsamp.tex|
with |cdocsch1.tex|, |cdocsch2.tex|, |cdocspt3.tex|, |cdocspt4.tex|,
|cdocsdrf.tex|, |cdocsfn1.tex|, |cdocsfn2.tex|
as well as |childdoc.pdf|.

%%%%%%%%%%%%%%%%%%%%%%%%%%%%%%%%%%%%%%%%%%%%%%%%%%%%%%%%%%%%%%%%%%%%%%%%%%%%%%%%
\subsection{Files and Installation}

The package consists of the files:
%
\begin{center}
\begin{tabular}{ll}
    |README.txt|   & readme file \\
    |childdoc.ins| & installation file \\
    |childdoc.dtx| & source file \\
    |childdoc.def| & definition file \\
    |cdocsamp.tex| & sample main file \\
    |cdocsch1.tex| & sample include file \\
    |cdocsch2.tex| & sample include file \\
    |cdocspt3.tex| & sample part file \\
    |cdocspt4.tex| & sample part file \\
    |cdocsdrf.tex| & sample redirection file \\
    |cdocsfn1.tex| & sample redirection file \\
    |cdocsfn2.tex| & sample redirection file \\
    |childdoc.pdf| & manual
\end{tabular}
\end{center}
%
The distribution consists of the files
|README.txt|, |childdoc.ins| and |childdoc.dtx|.
%
\begin{itemize}
\item
Run (pdf)\LaTeX{} on |childdoc.dtx|
to compile the manual |childdoc.pdf| (this file).
\item
Run \LaTeX{} on |childdoc.ins| to create the definitions file |childdoc.def|
and the sample |cdocsamp.tex| with include files
|cdocsch1.tex|, |cdocsch2.tex|, |cdocspt3.tex|, |cdocspt4.tex|,
|cdocsdrf.tex|, |cdocsfn1.tex|, |cdocsfn2.tex|.
Then copy the file |childdoc.def| to an appropriate directory of your \LaTeX{}
distribution, e.g.\ \textit{texmf-root}|/tex/latex/childdoc|.
\end{itemize}

%%%%%%%%%%%%%%%%%%%%%%%%%%%%%%%%%%%%%%%%%%%%%%%%%%%%%%%%%%%%%%%%%%%%%%%%%%%%%%%%
\subsection{Related CTAN Packages}

There are several other packages which offer a similar functionality:
%
\begin{itemize}
\item
The packages
\href{http://ctan.org/pkg/docmute}{\textsf{docmute}},
\href{http://ctan.org/pkg/includex}{\textsf{includex}} and
\href{http://ctan.org/pkg/standalone}{\textsf{standalone}}
provide commands to include only the document body of
a child file thus allowing both files to be compiled individually.
\item
The packages \href{http://ctan.org/pkg/subdocs}{\textsf{subdocs}}
and \href{http://ctan.org/pkg/subfiles}{\textsf{subfiles}}
provide structures in which the main and child documents can be
encapsulated and allowing them to be compiled individually.
The inclusion mechanism is different from the conventional |\include|.
\item
The package \href{http://ctan.org/pkg/combine}{\textsf{combine}}
is an elaborate solution to combine several documents into one.
\end{itemize}
%
See also the CTAN topic \href{http://ctan.org/topic/subdocs}{\textsf{subdocs}}
for further related packages.
The present package differs from the above solutions in that
a document structure constructed with the conventional |\include| mechanism
just needs two extra commands at the top of every file
such that all constituent files can be compiled individually.

%%%%%%%%%%%%%%%%%%%%%%%%%%%%%%%%%%%%%%%%%%%%%%%%%%%%%%%%%%%%%%%%%%%%%%%%%%%%%%%%
%\subsection{Feature Suggestions}
%
%The following is a list of features which may be useful for future
%versions of this package:
%%
%\begin{itemize}
%\item
%\ldots
%\end{itemize}

%%%%%%%%%%%%%%%%%%%%%%%%%%%%%%%%%%%%%%%%%%%%%%%%%%%%%%%%%%%%%%%%%%%%%%%%%%%%%%%%
\subsection{Revision History}

%%%%%%%%%%%%%%%%%%%%%%%%%%%%%%%%%%%%%%%%
\paragraph{v2.0:} 2018/12/30

\begin{itemize}
\item
immediate forward processing
\item
added |\childdocby| mechanism
\item
manual restructured
\end{itemize}

%%%%%%%%%%%%%%%%%%%%%%%%%%%%%%%%%%%%%%%%
\paragraph{v1.6:} 2018/01/17

\begin{itemize}
\item
application for development of include files
\item
corrections to manual
\end{itemize}

%%%%%%%%%%%%%%%%%%%%%%%%%%%%%%%%%%%%%%%%
\paragraph{v1.5:} 2017/05/21

\begin{itemize}
\item
more complete structuring introduced
\item
|\childdocof| introduced
\item
|\childdoc| renamed to |\childdocmain|
\item
|\childredirect| renamed to |\childdocforward| and |\childdocforwardprefix|
and functionality expanded
\end{itemize}

%%%%%%%%%%%%%%%%%%%%%%%%%%%%%%%%%%%%%%%%
\paragraph{v1.0:} 2017/04/27

\begin{itemize}
\item
manual and install package
\item
first version published on CTAN
\end{itemize}

%%%%%%%%%%%%%%%%%%%%%%%%%%%%%%%%%%%%%%%%
\paragraph{v0.6:} 2017/04/26

\begin{itemize}
\item
redirection mechanism added
\end{itemize}

%%%%%%%%%%%%%%%%%%%%%%%%%%%%%%%%%%%%%%%%
\paragraph{v0.5:} 2017/04/26

\begin{itemize}
\item
functionality in definition file
\end{itemize}


%%%%%%%%%%%%%%%%%%%%%%%%%%%%%%%%%%%%%%%%%%%%%%%%%%%%%%%%%%%%%%%%%%%%%%%%%%%%%%%%
%%%%%%%%%%%%%%%%%%%%%%%%%%%%%%%%%%%%%%%%%%%%%%%%%%%%%%%%%%%%%%%%%%%%%%%%%%%%%%%%
%%%%%%%%%%%%%%%%%%%%%%%%%%%%%%%%%%%%%%%%%%%%%%%%%%%%%%%%%%%%%%%%%%%%%%%%%%%%%%%%
\appendix

\settowidth\MacroIndent{\rmfamily\scriptsize 000\ }

 \DocInput{childdoc.dtx}

\end{document}
%</driver>
% \fi
%
% %%%%%%%%%%%%%%%%%%%%%%%%%%%%%%%%%%%%%%%%%%%%%%%%%%%%%%%%%%%%%%%%%%%%%%%%%%%%%%
% %%%%%%%%%%%%%%%%%%%%%%%%%%%%%%%%%%%%%%%%%%%%%%%%%%%%%%%%%%%%%%%%%%%%%%%%%%%%%%
% \section{Sample}
%\iffalse
%<*samplemain>
%\fi
%
% The following presents a sample document
% with two chapters, two parts, a title page,
% a compile flag as well as three forwarding files to set the flag.
% It consists of eight |.tex| files:
% \begin{center}
% \begin{tabular}{ll}
% |cdocsamp.tex|&main file\\
% |cdocsch1.tex|&include file for chapter 1\\
% |cdocsch2.tex|&include file for chapter 2\\
% |cdocspt3.tex|&include file for part 3\\
% |cdocspt4.tex|&include file for part 4\\
% |cdocsdrf.tex|&forwarding file for main file in draft mode\\
% |cdocsfi1.tex|&forwarding file for final version of chapter 1\\
% |cdocsfi2.tex|&forwarding file for final version of chapter 2\\
% \end{tabular}
% \end{center}
% Each of the eight files can be compiled directly by the \LaTeX{} compiler.
%
% %%%%%%%%%%%%%%%%%%%%%%%%%%%%%%%%%%%%%%
% \paragraph{Main File.}
%
% The main file is called |cdocsamp.tex|.
%
% Load the \textsf{childdoc} definitions and
% declare the filename for the main document:
%    \begin{macrocode}
\input{childdoc.def}
\childdocmain{}
%    \end{macrocode}

% Optional override for |\version| flag:
%    \begin{macrocode}
%%\ifchilddoc\else\providecommand{\version}{draft}\fi
%    \end{macrocode}

% Define the default values for the |\version| flag
% (|final| for the main file and |draft| for childs):
%    \begin{macrocode}
\ifchilddoc
\providecommand{\version}{draft}
\else
\providecommand{\version}{final}
\fi
%    \end{macrocode}

% Load the standard document class:
%    \begin{macrocode}
\documentclass[12pt]{article}
%    \end{macrocode}

% Start the document body:
%    \begin{macrocode}
\begin{document}
%    \end{macrocode}

% Declare a title page.
% Print title, part of document being processed and version flag:
%    \begin{macrocode}
\addtocounter{page}{-1}
\begin{center}
{\LARGE\bfseries{}childdoc example\par}
\vspace{1cm}
\ifchilddoc
\ifchilddocmanual part\else chapter\fi:
`\childdocname' of `\childdocjob'\par
\else
main document: `\childdocjob'\par
\fi
version: \version\par
\end{center}
\newpage
%    \end{macrocode}

% Manually include selected file,
% otherwise process as usual:
%    \begin{macrocode}
\ifchilddocmanual
\section*{part `\childdocname'}
\input{\childdocname}
\else
%    \end{macrocode}

% Include the two chapters:
%    \begin{macrocode}
\include{cdocsch1}
\include{cdocsch2}
%    \end{macrocode}

% Include the two parts unless only chapters should be displayed:
%    \begin{macrocode}
\ifchilddoc\else
\section{part three}
\input{cdocspt3}
\section{part four}
\input{cdocspt4}
\fi
%    \end{macrocode}

% Process as usual until here:
%    \begin{macrocode}
\fi
%    \end{macrocode}

% End of document body:
%    \begin{macrocode}
\end{document}
%    \end{macrocode}
%\iffalse
%</samplemain>
%\fi
%
% %%%%%%%%%%%%%%%%%%%%%%%%%%%%%%%%%%%%%%
% \paragraph{Chapter Include Files.}
%
% The include files are called |cdocsch1.tex| and |cdocsch2.tex|.
%
%\iffalse
%<*samplechap1|samplechap2>
%\fi

% Optional override for |\version| flag:
%    \begin{macrocode}
%%\providecommand{\version}{final}
%    \end{macrocode}

% Include the main document:
%    \begin{macrocode}
\input{childdoc.def}
\childdocof{cdocsamp}
%    \end{macrocode}

%\iffalse
%</samplechap1|samplechap2>
%\fi
%
%\iffalse
%<*samplechap1>
%\fi
% Some text for chapter 1:
%    \begin{macrocode}
\section{one}
some text in chapter one
%    \end{macrocode}

%\iffalse
%</samplechap1>
%\fi
% Some text for chapter 2:
%\iffalse
%<*samplechap2>
%\fi
%    \begin{macrocode}
\section{two}
more text in chapter two
%    \end{macrocode}

%\iffalse
%</samplechap2>
%\fi
%
% %%%%%%%%%%%%%%%%%%%%%%%%%%%%%%%%%%%%%%
% \paragraph{Part Include Files.}
%
% The include files are called |cdocspt3.tex| and |cdocspt4.tex|.
%
%\iffalse
%<*samplepart3|samplepart4>
%\fi

% Optional override for |\version| flag:
%    \begin{macrocode}
%%\providecommand{\version}{final}
%    \end{macrocode}

% Include the main document:
%    \begin{macrocode}
\input{childdoc.def}
\childdocby{cdocsamp}
%    \end{macrocode}

%\iffalse
%</samplepart3|samplepart4>
%\fi
%
%\iffalse
%<*samplepart3>
%\fi
% Some text for part 3:
%    \begin{macrocode}
some text in part three
%    \end{macrocode}

%\iffalse
%</samplepart3>
%\fi
% Some text for part 4:
%\iffalse
%<*samplepart4>
%\fi
%    \begin{macrocode}
more text in part four
%    \end{macrocode}

%\iffalse
%</samplepart4>
%\fi
%
% %%%%%%%%%%%%%%%%%%%%%%%%%%%%%%%%%%%%%%
% \paragraph{Forwarding for a Complete Draft.}
%
% The following forwarding file |cdocsdrf.tex|
% compiles the main document in draft mode:
%\iffalse
%<*sampledraft>
%\fi
%    \begin{macrocode}
\def\version{draft}
\input{childdoc.def}
\childdocforward{cdocsamp}
%    \end{macrocode}

%\iffalse
%</sampledraft>
%\fi
%
% %%%%%%%%%%%%%%%%%%%%%%%%%%%%%%%%%%%%%%
% \paragraph{Forwarding for Final Version of the Chapters.}
%
% The following forwarding files |cdocsfn1.tex| and |cdocsfn2.tex|
% (with identical content)
% compile the final versions of the child documents
% |cdocsch1.tex| and |cdocsch2.tex|, respectively:
%\iffalse
%<*samplefinal>
%\fi
%    \begin{macrocode}
\def\version{final}
\input{childdoc.def}
\childdocforwardprefix[cdocsamp]{cdocsfn}{cdocsch}
%    \end{macrocode}

%\iffalse
%</samplefinal>
%\fi
%
% %%%%%%%%%%%%%%%%%%%%%%%%%%%%%%%%%%%%%%
% \paragraph{Command Line Processing.}
%
% The following three command lines generate the output files
% |cdocscld|, |cdocscl1| and |cdocscl2|
% which should be identical to
% |cdocsdrf|, |cdocsch1| and |cdocsfn2|, respectively:
% \begin{center}
% \begin{tabular}{l}
% |latex -jobname cdocscld \|\\
% |  "\def\version{draft}\input{childdoc.def}\childdocforward{cdocsamp}"|\\
% |latex -jobname cdocscl1 \|\\
% |  "\input{childdoc.def}\childdocforward[cdocsamp]{cdocsch1}"|\\
% |latex -jobname cdocscl2 \|\\
% |  "\def\version{final}\input{childdoc.def}\childdocforward{cdocsch2}"|
% \end{tabular}
% \end{center}
% Note that the trailing backslash on each first line
% merely continues the input to the second line
% (for convenient cut ant paste).
% Furthermore, the command |latex| can be replaced by any
% of its alternative versions such as |pdflatex|.
%
% %%%%%%%%%%%%%%%%%%%%%%%%%%%%%%%%%%%%%%%%%%%%%%%%%%%%%%%%%%%%%%%%%%%%%%%%%%%%%%
% %%%%%%%%%%%%%%%%%%%%%%%%%%%%%%%%%%%%%%%%%%%%%%%%%%%%%%%%%%%%%%%%%%%%%%%%%%%%%%
% \section{Implementation}
%\iffalse
%<*package>
%\fi
%
% This section describes the definitions file |childdoc.def|.

% The definitions cannot be loaded using |\usepackage| or |\RequirePackage|
% which has a mechanism to prevent loading a style file more than once.
% When loading the definitions by means of |\input|
% multiple instances have to be prevented manually:
%\iffalse
%This code needs to be before the `\ProvidesFile' directive
%which is defined at the beginning of this file.
%Therefore it is also placed there and commented out here.
%</package>
%<*discard>
%\fi
%    \begin{macrocode}
\ifdefined\childdocmain\endinput\fi
%    \end{macrocode}
%\iffalse
%</discard>
%<*package>
%\fi
%
% \macro{\ifchilddoc}
% \macro{\ifchilddocmanual}
% The conditional |\ifchilddoc| tells whether a
% child (true) or main (false) document is being compiled.
% The conditional |\ifchilddocmanual| tells whether
% the |\includeonly| mechanism is used (false) or
% the selection of child files must be performed manually (true).
% The definitions initialise to false:
%    \begin{macrocode}
\newif\ifchilddoc
\newif\ifchilddocmanual
%    \end{macrocode}

% \macro{\childdocname}
% \macro{\childdocjob}
% The macro |\childdocname| stores the name of the main document
% to be compiled. The macro |\childdocjob| stores the name of
% the document on which the \LaTeX{} compiler was originally invoked.
% The content of |\jobname| cannot be compared
% to filenames specified in the source due to different catcodes.
% The following code rescans |\jobname|, stores the result
% in |\childdocname| and saves a copy in |\childdocjob|:
%    \begin{macrocode}
\edef\childdocname{\scantokens\expandafter{\jobname\noexpand}}
\let\childdocjob\childdocname
%    \end{macrocode}

% \macro{\childdocdisable}
% The macro |\childdocdisable| prevents the main file
% from being processed more than once.
% At this stage, the main document command |\childdocmain|
% is assumed to be called once again where it should do nothing.
% Any subsequent call to it should prevent
% a secondary processing of the main document
% It overwrites the forwarding commands
% |\childdocof| and |\childdocforward|
% with empty macros to prevent further inclusions of the main document:
%    \begin{macrocode}
\newcommand{\childdocdisable}
{
  \renewcommand{\childdocmain}[1]{\renewcommand{\childdocmain}[1]{\endinput}}
  \renewcommand{\childdocof}[1]{}
  \renewcommand{\childdocby}[2][]{}
  \renewcommand{\childdocforward}[2][]{}
  \renewcommand{\childdocdisable}{}
}
%    \end{macrocode}

% \macro{\childdocmain}
% The macro |\childdocmain| is to be called at the top of the main file
% with nothing or the main filename (without extension) as argument.
% First, it breaks loops.
% If the argument is not empty and does not match |\childdocname|
% (which is set by the first inclusion of |childdoc.def|),
% |\ifchilddoc| is set to true, |\includeonly| is applied to the child file
% and |\jobname| is set to the main file
% (for proper handling of |.aux| files):
%    \begin{macrocode}
\newcommand{\childdocmain}[1]
{
  \childdocdisable\childdocmain{}
  \if?#1?\else
    \begingroup
      \def\childdoctmp{#1}
      \ifx\childdoctmp\childdocname
        \def\childdoctmp{}
      \else
        \def\childdoctmp
        {
          \childdoctrue
          \includeonly{\childdocname}
          \def\childdocjob{#1}
          \def\jobname{#1}
        }
      \fi
      \expandafter
    \endgroup
    \childdoctmp
  \fi
}
%    \end{macrocode}

% \macro{\childdocof}
% The command |\childdocof| redirects
% compilation to the main file |#1|.
%    \begin{macrocode}
\newcommand{\childdocof}[1]
{
  \childdocdisable
  \childdoctrue
  \includeonly{\childdocname}
  \def\jobname{#1}
  \def\childdocjob{#1}
  \input{#1}
}
%    \end{macrocode}

% \macro{\childdocby}
% The command |\childdocby| ....
%    \begin{macrocode}
\newcommand{\childdocby}[2][]
{
  \childdocdisable
  \childdoctrue
  \childdocmanualtrue
  \if?#1?\else
    \def\jobname{#2}
  \fi
  \def\childdocjob{#2}
  \input{#2}
  \endinput
}
%    \end{macrocode}

% \macro{\childdocforward}
% The command |\childdocforward| redirects
% compilation to the main file or
% (if the optional argument is given) a child file.
% Parameters are set as if the main file
% or a child file starting with |\childdocof| was compiled.
% Then compilation is handed over to the main file:
%    \begin{macrocode}
\newcommand{\childdocforward}[2][]
{
  \begingroup
    \if?#1?
      \def\childdoctmp
      {
        \def\childdocname{#2}
        \def\childdocjob{#2}
        \def\jobname{#2}
        \input{#2}
        \endinput
      }
    \else
      \def\childdoctmp
      {
        \childdocdisable
        \def\childdocname{#2}
        \childdoctrue
        \includeonly{#2}
        \def\childdocjob{#1}
        \def\jobname{#1}
        \input{#1}
        \endinput
      }
    \fi
    \expandafter
  \endgroup
  \childdoctmp
}
%    \end{macrocode}

% \macro{\childdocforwardprefix}
% The command |\childdocforwardprefix| redirects
% compilation to the main or a child file by means of a pattern.
% The prefix |#1| in the current filename is replaced by |#2|
% and the suffix of the current filename is kept
% (it is assumed that the filename does not contain the substring `|~~~|'
% which is used as a delimiter).
% Compilation is handed over to the new file by |\childdocforward|:
%    \begin{macrocode}
\newcommand{\childdocforwardprefix}[3][]
{
  \begingroup
    \def\childdocextract #2##1~~~{\def\childdoctmp{\childdocforward[#1]{#3##1}}}
    \expandafter\childdocextract\childdocname~~~
    \expandafter
  \endgroup
  \childdoctmp
}
%    \end{macrocode}

% \macro{\childdoc}
% The deprecated macro |\childdoc| is a legacy version of |\childdocmain|:
%    \begin{macrocode}
\newcommand{\childdoc}{\childdocmain}
%    \end{macrocode}

% \macro{\childdocredirect}
% The deprecated macro |\childdocredirect| is a legacy version
% of |\childdocforward| and |\childdocforwardprefix|:
%    \begin{macrocode}
\newcommand{\childdocredirect}[2][]
{
  \begingroup
    \if?#1?
      \def\childdoctmp{\childdocforward{#2}}
    \else
      \def\childdoctmp{\childdocforwardprefix{#1}{#2}}
    \fi
    \expandafter
  \endgroup
  \childdoctmp
}
%    \end{macrocode}

%\iffalse
%</package>
%\fi
%
\endinput

\childdocof{cdocsamp}
%    \end{macrocode}

%\iffalse
%</samplechap1|samplechap2>
%\fi
%
%\iffalse
%<*samplechap1>
%\fi
% Some text for chapter 1:
%    \begin{macrocode}
\section{one}
some text in chapter one
%    \end{macrocode}

%\iffalse
%</samplechap1>
%\fi
% Some text for chapter 2:
%\iffalse
%<*samplechap2>
%\fi
%    \begin{macrocode}
\section{two}
more text in chapter two
%    \end{macrocode}

%\iffalse
%</samplechap2>
%\fi
%
% %%%%%%%%%%%%%%%%%%%%%%%%%%%%%%%%%%%%%%
% \paragraph{Part Include Files.}
%
% The include files are called |cdocspt3.tex| and |cdocspt4.tex|.
%
%\iffalse
%<*samplepart3|samplepart4>
%\fi

% Optional override for |\version| flag:
%    \begin{macrocode}
%%\providecommand{\version}{final}
%    \end{macrocode}

% Include the main document:
%    \begin{macrocode}
% \iffalse
%
% childdoc.dtx Copyright (C) 2017-2018 Niklas Beisert
%
% This work may be distributed and/or modified under the
% conditions of the LaTeX Project Public License, either version 1.3
% of this license or (at your option) any later version.
% The latest version of this license is in
%   http://www.latex-project.org/lppl.txt
% and version 1.3 or later is part of all distributions of LaTeX
% version 2005/12/01 or later.
%
% This work has the LPPL maintenance status `maintained'.
%
% The Current Maintainer of this work is Niklas Beisert.
%
% This work consists of the files childdoc.dtx and childdoc.ins
% and the derived files childdoc.def and cdocsamp.tex with
% cdocsch1.tex, cdocsch2.tex, cdocsdrf.tex, cdocsfn1.tex, cdocsfn2.tex.
%
%<package>\ifdefined\childdocmain\endinput\fi
%<package>\ProvidesFile{childdoc.def}[2018/12/30 v2.0 child document driver]
%<samplemain>\ProvidesFile{cdocsamp.tex}[2018/12/30 v2.0 sample for childdoc]
%<*driver>
%\ProvidesFile{childdoc.drv}[2018/12/30 v2.0 childdoc reference manual file]
\PassOptionsToClass{10pt,a4paper}{article}
\documentclass{ltxdoc}

\usepackage[margin=35mm]{geometry}
\usepackage{hyperref}
\usepackage{hyperxmp}
\usepackage[usenames]{color}

\hypersetup{colorlinks=true}
\hypersetup{pdfstartview=FitH}
\hypersetup{pdfpagemode=UseNone}
\hypersetup{pdfsource={}}
\hypersetup{pdflang={en-UK}}
\hypersetup{pdfcopyright={Copyright 2017-2018 Niklas Beisert.
  This work may be distributed and/or modified under the
  conditions of the LaTeX Project Public License, either version 1.3
  of this license or (at your option) any later version.}}
\hypersetup{pdflicenseurl={http://www.latex-project.org/lppl.txt}}
\hypersetup{pdfcontactaddress={ETH Zurich, ITP, HIT K,
  Wolfgang-Pauli-Strasse 27}}
\hypersetup{pdfcontactpostcode={8093}}
\hypersetup{pdfcontactcity={Zurich}}
\hypersetup{pdfcontactcountry={Switzerland}}
\hypersetup{pdfcontactemail={nbeisert@itp.phys.ethz.ch}}
\hypersetup{pdfcontacturl={http://people.phys.ethz.ch/\xmptilde nbeisert/}}

\newcommand{\secref}[1]{\hyperref[#1]{section \ref*{#1}}}

\parskip1ex
\parindent0pt
\let\olditemize\itemize
\def\itemize{\olditemize\parskip0pt}

\begin{document}

\title{The \textsf{childdoc} Package}
\hypersetup{pdftitle={The childdoc Package}}
\author{Niklas Beisert\\[2ex]
  Institut f\"ur Theoretische Physik\\
  Eidgen\"ossische Technische Hochschule Z\"urich\\
  Wolfgang-Pauli-Strasse 27, 8093 Z\"urich, Switzerland\\[1ex]
  \href{mailto:nbeisert@itp.phys.ethz.ch}
  {\texttt{nbeisert@itp.phys.ethz.ch}}}
\hypersetup{pdfauthor={Niklas Beisert}}
\hypersetup{pdfsubject={Manual for the LaTeX2e Package childdoc}}
\date{30 December 2018, \textsf{v2.0}}
\maketitle

\begin{abstract}\noindent
\textsf{childdoc} is a \LaTeXe{} package
that enables the direct compilation
of document sections included by |\include|
to individual files.
\end{abstract}

\begingroup
\parskip0ex
\tableofcontents
\endgroup

%%%%%%%%%%%%%%%%%%%%%%%%%%%%%%%%%%%%%%%%%%%%%%%%%%%%%%%%%%%%%%%%%%%%%%%%%%%%%%%%
%%%%%%%%%%%%%%%%%%%%%%%%%%%%%%%%%%%%%%%%%%%%%%%%%%%%%%%%%%%%%%%%%%%%%%%%%%%%%%%%
\section{Introduction}

\LaTeX{} provides a mechanism to structure a large document (such as a book)
into a main file and several child files (containing the chapters)
using the |\include| command.
This mechanism is beneficial for documents
which span hundreds of pages in order to
make the source file(s) more manageable.
Moreover, compilation can be restricted to
selected child files by means of the |\includeonly| command.
The latter feature can be used to reduce the compilation time while editing
(this was significantly more useful in the earlier days of \LaTeX{})
or to generate a smaller document which is easier to navigate.
Another application of |\includeonly| is to generate
documents consisting of selected parts of the complete document.

However, there are a few drawbacks of the plain |\include| mechanism:
\begin{itemize}
\item
The child files cannot be compiled on their own,
they can only be compiled via the main file.
A naive editing environment
(such as a text editor with an option
to have the current file processed by \LaTeX)
may require one to switch to the main file before compiling;
attempting to compile the child file produces errors.
\item
The main file must be modified (each time)
to adjust the |\includeonly| command
to the present needs. This easily leaves the main file in a messy state.
\item
The generated document will always carry the filename
of the main document. This is inconvenient if
several child files are to be compiled and
to be kept for distribution.
\end{itemize}

The present package provides a simple interface
to make child files individually compilable by \LaTeX{}.
Compiling a child file then has the same effect as compiling
the main file with an |\includeonly| command
to select the appropriate child.
Moreover the generated document will carry the name of the child
rather than the main file.
This resolves all three above issues.

This feature is meant to make the editing of books,
thesis documents and lecture notes somewhat more convenient.
However, the package can also be used efficiently for
composing a series of documents (such as exercise sheets)
which are typically distributed individually.
It then assists the author in generating the individual documents
(potentially in different versions)
as well as a document containing the collected series.
Another application is in developing style files
or other kinds of included material
where compilation of the style file could redirect
to a sample or test file.

%%%%%%%%%%%%%%%%%%%%%%%%%%%%%%%%%%%%%%%%%%%%%%%%%%%%%%%%%%%%%%%%%%%%%%%%%%%%%%%%
%%%%%%%%%%%%%%%%%%%%%%%%%%%%%%%%%%%%%%%%%%%%%%%%%%%%%%%%%%%%%%%%%%%%%%%%%%%%%%%%
\section{Usage}

First of all, the package \textsf{childdoc} is \emph{not} a standard
\LaTeXe{} |.sty| style file! Therefore it needs to be invoked in
a non-standard way.

%%%%%%%%%%%%%%%%%%%%%%%%%%%%%%%%%%%%%%%%%%%%%%%%%%%%%%%%%%%%%%%%%%%%%%%%%%%%%%%%
\subsection{Included Files}
\label{sec:include}

%%%%%%%%%%%%%%%%%%%%%%%%%%%%%%%%%%%%%%%%
\DescribeMacro{\childdocmain}
To use the package, add the commands
\begin{center}
\begin{tabular}{l}
|\input{childdoc.def}|\\
|\childdocmain{}|\\
\end{tabular}
\end{center}
at the very top of the main \LaTeX{} file,
in particular \emph{before} the |\documentclass| statement!
The argument of |\childdocmain| should be left empty
(but it must be present).

%%%%%%%%%%%%%%%%%%%%%%%%%%%%%%%%%%%%%%%%
\DescribeMacro{\childdocof}
Furthermore, add the commands
\begin{center}
\begin{tabular}{l}
|\input{childdoc.def}|\\
|\childdocof{|\textit{main}|}|\\
\end{tabular}
\end{center}
at the top of every child file \textit{child}
which is included by |\include{|\textit{child}|}|
from within the main file
(or at least for those files to be compiled individually).
The argument \textit{main} must be the filename of the main file.

There are a couple of
considerations in setting up the main and child documents:

%%%%%%%%%%%%%%%%%%%%%%%%%%%%%%%%%%%%%%%%
\paragraph{Restrictions.}

Please note the following restrictions:
\begin{itemize}
\item
|\childdocmain| must be called with one argument \textit{main}
to ensure compatibility with earlier version of the package.
It must either be empty (|\childdocmain{}|)
or precisely match the filename of the main file in which it is specified.
See \secref{sec:detection} for further information.
\item
The filename \textit{main} must be specified without the |.tex| extension.
\item
The filename \textit{main} is case sensitive
(even in case-insensitive file systems)
due to internal string comparison.
\item
The argument \textit{main} should be fully expanded, it cannot be a macro.
\item
Subdirectories and special characters should be avoided in filenames.
\item
The command |\childdocmain{|\textit{main}|}| must be followed by a whitespace.
It should not be followed immediately by another command
or by a comment mark `|%|'.
This is because the \TeX{} parser reads the token immediately following
the argument of |\childdocmain| and puts it
at the beginning of every child section;
however, a white\-space is ignored.
\end{itemize}

%%%%%%%%%%%%%%%%%%%%%%%%%%%%%%%%%%%%%%%%
\paragraph{Content of Main File.}

It is advisable to place all content in the child files included by |\include|.
Any output contained in the main file will appear in all child documents
unless suppressed manually;
it cannot be suppressed automatically by the |\includeonly| directive
and thus should normally be avoided.
A method to include some content in the main file
by means of conditional processing is described in \secref{sec:conditional}.

%%%%%%%%%%%%%%%%%%%%%%%%%%%%%%%%%%%%%%%%
\paragraph{Page Numbering.}

When only a part of the document is compiled,
the appropriate numbering of pages
(as well as other status parameters)
is determined from the |.aux| files.
The latter contain information from previous passes.
However this information needs to propagate through
all intermediate child documents.
Therefore the page numbering in child documents may well
be inconsistent until the complete document is compiled at least once.

A useful (if unconventional) way to always ensure a consistent
page numbering is to restart the numbering in each child document
and denote the pages by `\textit{child}|.|\textit{page}'
where \textit{child} represents the chapter/section number of the child file.
This can be achieved by the command
|\numberwithin{page}{|\textit{child}|}|
of the \textsf{amsmath} package
where \textit{child} can be |chapter| or |section|
depending on the chosen structuring.
Alternatively, one can modify the macro |\thepage| appropriately
and reset the counter |page| at the start of each child file.

%%%%%%%%%%%%%%%%%%%%%%%%%%%%%%%%%%%%%%%%%%%%%%%%%%%%%%%%%%%%%%%%%%%%%%%%%%%%%%%%
\subsection{Conditional Processing}
\label{sec:conditional}

The package provides a mechanism to compile different versions
of a document. To customise the versions further some conditional processing
can come in handy to distinguish which version is being compiled.
The package provides two macros to describe the compilation context:

%%%%%%%%%%%%%%%%%%%%%%%%%%%%%%%%%%%%%%%%
\DescribeMacro{\ifchilddoc}
The conditional |\ifchilddoc| distinguishes between the compilation of
child documents and the main document:
%
\begin{center}
|\ifchilddoc |\textit{child-code}| |[|\||else |\textit{main-code}]| \||fi|
\end{center}

%%%%%%%%%%%%%%%%%%%%%%%%%%%%%%%%%%%%%%%%
\DescribeMacro{\childdocname}
\DescribeMacro{\childdocjob}
The macro |\childdocname| contains the filename (without extension)
of the main or child file being processed.
Note that |\childdocjob| will always contain the name of the main file.

%%%%%%%%%%%%%%%%%%%%%%%%%%%%%%%%%%%%%%%%
\paragraph{Title Page.}

Conditional processing can be used to include a title or banner page
in the main document when proper precautions are taken.
Importantly, the code in the main file should ensure that the page counter
(as well as other status parameters which are stored in the |.aux| files)
takes the same value after the conditional processing.
Otherwise the page numbers may take divergent values
depending on which part is compiled.

For example, a title page could be declared by:
%
\begin{center}
\begin{tabular}{l}
|\ifchilddoc\||else|\\
|\addtocounter{page}{-1}|\\
\textit{code for title page}\\
|\newpage|\\
|\||fi|
\end{tabular}
\end{center}
%
A banner page for the child documents can be generated by:
%
\begin{center}
\begin{tabular}{l}
|\ifchilddoc|\\
|\addtocounter{page}{-1}|\\
\textit{code for banner page}\\
|\newpage|\\
|\||fi|
\end{tabular}
\end{center}
%
Here one could write a message such as:
\begin{center}
|This is the part \childdocname{} of \childdocjob{}.|
\end{center}

%%%%%%%%%%%%%%%%%%%%%%%%%%%%%%%%%%%%%%%%%%%%%%%%%%%%%%%%%%%%%%%%%%%%%%%%%%%%%%%%
\subsection{Flags}
\label{sec:flags}

The package makes it easy to generate different versions
of the main or child documents.
To this end compilation flags can be defined
and assigned different default values.
They will be particularly useful in conjunction
with the forwarding mechanism described in \secref{sec:forward}.

For example, it may be useful to have a flag |\version|
which can be set to |draft| or |final|.
The document source will contain some conditional code
depending on the value of |\version|.
Suppose further, the flag should default to |final| for the main file
and to |draft| for child files
which is a natural assignment for editing the document.
This is achieved by placing the following code
in the preamble of the main document
(below the |\childdocmain| directive):
%
\begin{center}
\begin{tabular}{l}
|\ifchilddoc|\\
|\providecommand{\version}{draft}|\\
|\||else|\\
|\providecommand{\version}{final}|\\
|\||fi|
\end{tabular}
\end{center}
%
The definition by |\providecommand| makes sure
that previous definitions are not overwritten.
Further statements |\providecommand{\version}{...}|
can thus be added before the above code to override it.

For the main file, one might add a line
(between |\childdocmain| and the above block)
%
\begin{center}
|%\ifchilddoc\||else\providecommand{\version}{draft}\||fi|
\end{center}
%
which can be uncommented to produce a draft version.
Likewise one can add a line to the very top of a child file
(above the |\childdocof{|\textit{main}|}| directive)
%
\begin{center}
|%\providecommand{\version}{final}|
\end{center}
%
which can be uncommented to produce the final version of this child document.

%%%%%%%%%%%%%%%%%%%%%%%%%%%%%%%%%%%%%%%%%%%%%%%%%%%%%%%%%%%%%%%%%%%%%%%%%%%%%%%%
\subsection{Forwarding}
\label{sec:forward}

Different versions of the main or child documents
using compilation flags as described in \secref{sec:flags}
can be (permanently) stored in different files
for convenient compilation, viewing and distribution.
To this end, the package defines a command
to pass on compilation to a different file:

%%%%%%%%%%%%%%%%%%%%%%%%%%%%%%%%%%%%%%%%
\DescribeMacro{\childdocforward}
The command |\childdocforward| redirects processing to
another source file:
%
\begin{center}
\begin{tabular}{l}
|\input{childdoc.def}|\\
|\childdocforward[|\textit{main}|]{|\textit{dest}|}|\\
\end{tabular}
\end{center}
%
The argument \textit{dest} is the destination file
(without extension).
It should be the main file or one of the child files.
Note that further \textsf{childdoc} directives
such as |\childdocof| and |\childdocforward|
in the indicated file will be processed in this form.
The optional argument \textit{main}
passes on directly to the main file \textit{main}
while pretending to compile the child \textit{dest}.
This form behaves as if \textit{dest}
issues |\childdocof{|\textit{main}|}| right away,
and no further \textsf{childdoc} directives will be processed.

%%%%%%%%%%%%%%%%%%%%%%%%%%%%%%%%%%%%%%%%
\DescribeMacro{\...prefix}
In the alternative form |\childdocforwardprefix|,
%
\begin{center}
\begin{tabular}{l}
|\input{childdoc.def}|\\
|\childdocforwardprefix[|\textit{main}|]{|\textit{prefix}|}{|\textit{dest}|}|
\end{tabular}
\end{center}
%
the destination file is determined by a pattern
depending on the current file:
To make this work, the current file must be called
`{\textit{prefix}\hspace{0.2em}\textit{suffix}}'
with \textit{prefix} matching precisely the argument.
Processing is then passed on to the file
`{\textit{dest}\hspace{0.2em}\textit{suffix}}'.
Surely, the same effect is achieved by
directly specifying the
argument `{\textit{dest}\hspace{0.2em}\textit{suffix}}'
in the first form.
However, that requires to set up a different file
for each child. With the alternative form of the command
all these files can have exactly the same content
which simplifies setting them up and maintaining them.

For example, the following file |draft.tex|
with a compilation flag |\version| as described in \secref{sec:flags}
compiles the main document as a draft:
%
\begin{center}
\begin{tabular}{l}
|\def\version{draft}|\\
|\input{childdoc.def}|\\
|\childdocforward{|\textit{main}|}|
\end{tabular}
\end{center}
%
Likewise, the following files |final|\textit{nn}|.tex|
compile the final version of the child document
|child|\textit{nn}|.tex|:
%
\begin{center}
\begin{tabular}{l}
|\def\version{final}|\\
|\input{childdoc.def}|\\
|\childdocforwardprefix{final}{child}|
\end{tabular}
\end{center}
%

Note that when several versions of a main file and/or of each child file
are to be generated, it may be convenient to set up a |Makefile| or
shell script to automatise the process.

%%%%%%%%%%%%%%%%%%%%%%%%%%%%%%%%%%%%%%%%%%%%%%%%%%%%%%%%%%%%%%%%%%%%%%%%%%%%%%%%
\subsection{Command Line Processing}
\label{sec:commandline}

The effect of redirection files can also be achieved by invoking
the \LaTeX{} compiler with a more elaborate command line.
Most conveniently this should be done as part
of a shell script or a |Makefile|.

When using \textsf{childdoc} in the main file, the following
command lines effectively perform a redirection
(note that depending on the shell being used,
backslashes may have to be doubled: `|\|' $\to$ `|\\|'):
%
\begin{center}
|... -jobname "|\textit{target}|" |\\|"|[\textit{flags}]%
|\input{childdoc.def}\childdocforward[|\textit{main}|]{|\textit{dest}|}"|
\end{center}
%
Here \textit{target} is the name of the output file,
\textit{main} is the name of the main file
and \textit{dest} is the name of the main or child file to be processed
(all filenames without extensions).
The optional argument \textit{main} can be omitted
if \textit{main} matches \textit{dest}.
Optionally, compilation \textit{flags} can be defined via |\def| commands.
This command line makes the \TeX{} engine believe
it is compiling the file \textit{target}
whose content is specified as the latter parameter.
The provided code then forwards the processing to
\textit{main} or \textit{dest} as described in \secref{sec:forward}.

%%%%%%%%%%%%%%%%%%%%%%%%%%%%%%%%%%%%%%%%%%%%%%%%%%%%%%%%%%%%%%%%%%%%%%%%%%%%%%%%
\subsection{Include by Input}
\label{sec:input}

Including child documents by |\include| has some restrictions by design.
Most notably, the content of a child document always occupies
its own set of pages; pages cannot be shared between child documents.
Usually, this behaviour makes perfect sense
because each child document contain an essential part of the document.
However, in some situations it may be desirable to compose
a document from a collection of parts
without having mandatory page breaks between then.
For this case, the package
provides a mechanism to include parts
by |\input| which can also be processed individually.
However, by construction this mechanism
requires manual handling of the content to be output.

%%%%%%%%%%%%%%%%%%%%%%%%%%%%%%%%%%%%%%%%
\DescribeMacro{\ifchilddocmanual}
The main file should be prepared as usual, see \secref{sec:include}.
However, the document body must make a distinction
between processing of an individual part and of the main document, e.g.:
%
\begin{center}
\begin{tabular}{l}
|\ifchilddocmanual|\\
|\input{\childdocname}|\\
|\||else|\\
\textit{document body with }|\input{|\textit{part}|}|\\
|\||fi|
\end{tabular}
\end{center}
%
The conditional |\ifchilddocmanual| is true whenever
a part to be included by |\input| is being compiled,
and the name of the part is stored in |\childdocname|.

%%%%%%%%%%%%%%%%%%%%%%%%%%%%%%%%%%%%%%%%
\DescribeMacro{\childdocby}
Each part to be included by |\input| should start with:
%
\begin{center}
\begin{tabular}{l}
|\input{childdoc.def}|\\
|\childdocby{|\textit{main}|}|\\
\end{tabular}
\end{center}
%
The directive |\childdocby| is similar to |\childdocof|
described in \secref{sec:include},
but the subsequent selection of content must be done manually.
To that end, both |\ifchilddoc| and |\ifchilddocmanual|
will be true upon processing of a part,
and the name of the part is stored in |\childdocname|.
Note that |\jobname| will be set to the filename of the current part
so that each part receives an individual |.aux| file
that does not interfere with the |.aux| file(s) of the main document.
This behaviour can be altered by the alternative form
|\childdocby[*]{|\textit{main}|}| (with a non-empty optional argument)
which uses the |.aux| file of the main document
by setting |\jobname| to \textit{main}.

%%%%%%%%%%%%%%%%%%%%%%%%%%%%%%%%%%%%%%%%%%%%%%%%%%%%%%%%%%%%%%%%%%%%%%%%%%%%%%%%
\subsection{Driver Development}
\label{sec:driver}

The \textsf{childdoc} mechanism can also be use for the development
of definition files such as \LaTeX{} styles or classes.
This case differs from the above setup with multiple parts
included by |\include| in that no |\includeonly| should be invoked.
This can be achieved by starting the include file
(before |\ProvidesPackage|) with:
%
\begin{center}
\begin{tabular}{l}
|\input{childdoc.def}|\\
|\childdocforward{|\textit{main}|}|\\
\end{tabular}
\end{center}
%
or alternatively with:
%
\begin{center}
\begin{tabular}{l}
|\input{childdoc.def}|\\
|\childdocby{|\textit{main}|}|\\
\end{tabular}
\end{center}
%
Both forms have slightly different effects as described above.
The main file is prepared as usual, see \secref{sec:include}.

%%%%%%%%%%%%%%%%%%%%%%%%%%%%%%%%%%%%%%%%%%%%%%%%%%%%%%%%%%%%%%%%%%%%%%%%%%%%%%%%
\subsection{Legacy Detection}
\label{sec:detection}

The directive |\childdocmain| in the main file can detect
whether the complete document or merely a child is to be compiled
even without using the directive |\childdocof|.
This method is deprecated because it is less robust
and there is no compelling reason to use it;
it is merely provided for backward compatibility
and it may be removed in future versions.

If the detection mechanism is to be used,
it is mandatory to correctly specify
the filename of the main file as the argument of |\childdocmain|:
%
\begin{center}
\begin{tabular}{l}
|\input{childdoc.def}|\\
|\childdocmain{|\textit{main}|}|\\
\end{tabular}
\end{center}
%
If |\jobname| does not match the argument \textit{main} of |\childdocmain|,
it is assumed that |\jobname| points to the child file to be compiled.
When using |\childdocmain| with the main file specified as argument,
it suffices to start a child file
with just |\input{|\textit{main}|}|
without loading of the package and using |\childdocof|.
If instead all processing is done
with the appropriate \textsf{childdoc} directives,
the argument of \textit{main} of |\childdocmain| can be empty.

An alternative version of the command line processing described
in \secref{sec:commandline} using the detection mechanism reads:
%
\begin{center}
|... -jobname "|\textit{target}|" "|[\textit{flags}]%
[|\def\jobname{|\textit{dest}|}|]|\input{|\textit{main}|}"|
\end{center}

%%%%%%%%%%%%%%%%%%%%%%%%%%%%%%%%%%%%%%%%%%%%%%%%%%%%%%%%%%%%%%%%%%%%%%%%%%%%%%%%
\subsection{Manual Code}
\label{sec:manual}

In case one cannot be certain whether the definitions file |childdoc.def|
is installed on the target \TeX{} distribution
and one prefers not to ship it,
it is conceivable to paste a few relevant commands into the sources.

To that end, drop all statements |\input{childdoc.def}|
and perform the replacements as outlined below.
Instead of |\childdocmain{|\textit{main}|}| add the following code
to the top of the main file:
%
\begin{center}
\begin{tabular}{l}
|\||ifdefined\childdocname\endinput\||fi\newif\ifchilddoc|\\
|\edef\childdocname{\scantokens\expandafter{\jobname\noexpand}}|\\
|\def\childdocmain{|\textit{main}|}\||ifx\childdocmain\childdocname\||else|\\
|\childdoctrue\includeonly{\childdocname}\let\jobname\childdocmain\||fi|\\
\end{tabular}
\end{center}
%
Instead of |\childdocof{|\textit{main}|}| just include the main file
at the top of each child file:
%
\begin{center}
|\input{|\textit{main}|}|
\end{center}
%
A simple redirection |\childdocforward{|\textit{dest}|}| is achieved by:
%
\begin{center}
|\def\jobname{|\textit{dest}|}\input{\jobname}|
\end{center}
%
The redirection with prefix
|\childdocforwardprefix[|\textit{prefix}|]{|\textit{dest}|}|
is accomplished by:
%
\begin{center}
\begin{tabular}{l}
|{\edef\jobname{\scantokens\expandafter{\jobname\noexpand}}|\\
|\def\redirectjob |\textit{prefix}|#1~~~{\gdef\jobname{|\textit{dest}|#1}}|\\
|\expandafter\redirectjob\jobname~~~}\input{\jobname}|
\end{tabular}
\end{center}

In an alternative approach,
child documents can be compiled by a specific command line
without additional code or specific definitions:
%
\begin{center}
|... -jobname "|\textit{target}|" "|[\textit{flags}]%
|\includeonly{|\textit{dest}|}\input{|\textit{main}|}"|
\end{center}
%

%%%%%%%%%%%%%%%%%%%%%%%%%%%%%%%%%%%%%%%%%%%%%%%%%%%%%%%%%%%%%%%%%%%%%%%%%%%%%%%%
%%%%%%%%%%%%%%%%%%%%%%%%%%%%%%%%%%%%%%%%%%%%%%%%%%%%%%%%%%%%%%%%%%%%%%%%%%%%%%%%
\section{Information}

%%%%%%%%%%%%%%%%%%%%%%%%%%%%%%%%%%%%%%%%%%%%%%%%%%%%%%%%%%%%%%%%%%%%%%%%%%%%%%%%
\subsection{Copyright}

Copyright \copyright{} 2017--2018 Niklas Beisert

This work may be distributed and/or modified under the
conditions of the \LaTeX{} Project Public License, either version 1.3
of this license or (at your option) any later version.
The latest version of this license is in
  \url{http://www.latex-project.org/lppl.txt}
and version 1.3 or later is part of all distributions of \LaTeX{}
version 2005/12/01 or later.

This work has the LPPL maintenance status `maintained'.

The Current Maintainer of this work is Niklas Beisert.

This work consists of the files |README.txt|, |childdoc.ins| and |childdoc.dtx|
as well as the derived files |childdoc.def|, |cdocsamp.tex|
with |cdocsch1.tex|, |cdocsch2.tex|, |cdocspt3.tex|, |cdocspt4.tex|,
|cdocsdrf.tex|, |cdocsfn1.tex|, |cdocsfn2.tex|
as well as |childdoc.pdf|.

%%%%%%%%%%%%%%%%%%%%%%%%%%%%%%%%%%%%%%%%%%%%%%%%%%%%%%%%%%%%%%%%%%%%%%%%%%%%%%%%
\subsection{Files and Installation}

The package consists of the files:
%
\begin{center}
\begin{tabular}{ll}
    |README.txt|   & readme file \\
    |childdoc.ins| & installation file \\
    |childdoc.dtx| & source file \\
    |childdoc.def| & definition file \\
    |cdocsamp.tex| & sample main file \\
    |cdocsch1.tex| & sample include file \\
    |cdocsch2.tex| & sample include file \\
    |cdocspt3.tex| & sample part file \\
    |cdocspt4.tex| & sample part file \\
    |cdocsdrf.tex| & sample redirection file \\
    |cdocsfn1.tex| & sample redirection file \\
    |cdocsfn2.tex| & sample redirection file \\
    |childdoc.pdf| & manual
\end{tabular}
\end{center}
%
The distribution consists of the files
|README.txt|, |childdoc.ins| and |childdoc.dtx|.
%
\begin{itemize}
\item
Run (pdf)\LaTeX{} on |childdoc.dtx|
to compile the manual |childdoc.pdf| (this file).
\item
Run \LaTeX{} on |childdoc.ins| to create the definitions file |childdoc.def|
and the sample |cdocsamp.tex| with include files
|cdocsch1.tex|, |cdocsch2.tex|, |cdocspt3.tex|, |cdocspt4.tex|,
|cdocsdrf.tex|, |cdocsfn1.tex|, |cdocsfn2.tex|.
Then copy the file |childdoc.def| to an appropriate directory of your \LaTeX{}
distribution, e.g.\ \textit{texmf-root}|/tex/latex/childdoc|.
\end{itemize}

%%%%%%%%%%%%%%%%%%%%%%%%%%%%%%%%%%%%%%%%%%%%%%%%%%%%%%%%%%%%%%%%%%%%%%%%%%%%%%%%
\subsection{Related CTAN Packages}

There are several other packages which offer a similar functionality:
%
\begin{itemize}
\item
The packages
\href{http://ctan.org/pkg/docmute}{\textsf{docmute}},
\href{http://ctan.org/pkg/includex}{\textsf{includex}} and
\href{http://ctan.org/pkg/standalone}{\textsf{standalone}}
provide commands to include only the document body of
a child file thus allowing both files to be compiled individually.
\item
The packages \href{http://ctan.org/pkg/subdocs}{\textsf{subdocs}}
and \href{http://ctan.org/pkg/subfiles}{\textsf{subfiles}}
provide structures in which the main and child documents can be
encapsulated and allowing them to be compiled individually.
The inclusion mechanism is different from the conventional |\include|.
\item
The package \href{http://ctan.org/pkg/combine}{\textsf{combine}}
is an elaborate solution to combine several documents into one.
\end{itemize}
%
See also the CTAN topic \href{http://ctan.org/topic/subdocs}{\textsf{subdocs}}
for further related packages.
The present package differs from the above solutions in that
a document structure constructed with the conventional |\include| mechanism
just needs two extra commands at the top of every file
such that all constituent files can be compiled individually.

%%%%%%%%%%%%%%%%%%%%%%%%%%%%%%%%%%%%%%%%%%%%%%%%%%%%%%%%%%%%%%%%%%%%%%%%%%%%%%%%
%\subsection{Feature Suggestions}
%
%The following is a list of features which may be useful for future
%versions of this package:
%%
%\begin{itemize}
%\item
%\ldots
%\end{itemize}

%%%%%%%%%%%%%%%%%%%%%%%%%%%%%%%%%%%%%%%%%%%%%%%%%%%%%%%%%%%%%%%%%%%%%%%%%%%%%%%%
\subsection{Revision History}

%%%%%%%%%%%%%%%%%%%%%%%%%%%%%%%%%%%%%%%%
\paragraph{v2.0:} 2018/12/30

\begin{itemize}
\item
immediate forward processing
\item
added |\childdocby| mechanism
\item
manual restructured
\end{itemize}

%%%%%%%%%%%%%%%%%%%%%%%%%%%%%%%%%%%%%%%%
\paragraph{v1.6:} 2018/01/17

\begin{itemize}
\item
application for development of include files
\item
corrections to manual
\end{itemize}

%%%%%%%%%%%%%%%%%%%%%%%%%%%%%%%%%%%%%%%%
\paragraph{v1.5:} 2017/05/21

\begin{itemize}
\item
more complete structuring introduced
\item
|\childdocof| introduced
\item
|\childdoc| renamed to |\childdocmain|
\item
|\childredirect| renamed to |\childdocforward| and |\childdocforwardprefix|
and functionality expanded
\end{itemize}

%%%%%%%%%%%%%%%%%%%%%%%%%%%%%%%%%%%%%%%%
\paragraph{v1.0:} 2017/04/27

\begin{itemize}
\item
manual and install package
\item
first version published on CTAN
\end{itemize}

%%%%%%%%%%%%%%%%%%%%%%%%%%%%%%%%%%%%%%%%
\paragraph{v0.6:} 2017/04/26

\begin{itemize}
\item
redirection mechanism added
\end{itemize}

%%%%%%%%%%%%%%%%%%%%%%%%%%%%%%%%%%%%%%%%
\paragraph{v0.5:} 2017/04/26

\begin{itemize}
\item
functionality in definition file
\end{itemize}


%%%%%%%%%%%%%%%%%%%%%%%%%%%%%%%%%%%%%%%%%%%%%%%%%%%%%%%%%%%%%%%%%%%%%%%%%%%%%%%%
%%%%%%%%%%%%%%%%%%%%%%%%%%%%%%%%%%%%%%%%%%%%%%%%%%%%%%%%%%%%%%%%%%%%%%%%%%%%%%%%
%%%%%%%%%%%%%%%%%%%%%%%%%%%%%%%%%%%%%%%%%%%%%%%%%%%%%%%%%%%%%%%%%%%%%%%%%%%%%%%%
\appendix

\settowidth\MacroIndent{\rmfamily\scriptsize 000\ }

 \DocInput{childdoc.dtx}

\end{document}
%</driver>
% \fi
%
% %%%%%%%%%%%%%%%%%%%%%%%%%%%%%%%%%%%%%%%%%%%%%%%%%%%%%%%%%%%%%%%%%%%%%%%%%%%%%%
% %%%%%%%%%%%%%%%%%%%%%%%%%%%%%%%%%%%%%%%%%%%%%%%%%%%%%%%%%%%%%%%%%%%%%%%%%%%%%%
% \section{Sample}
%\iffalse
%<*samplemain>
%\fi
%
% The following presents a sample document
% with two chapters, two parts, a title page,
% a compile flag as well as three forwarding files to set the flag.
% It consists of eight |.tex| files:
% \begin{center}
% \begin{tabular}{ll}
% |cdocsamp.tex|&main file\\
% |cdocsch1.tex|&include file for chapter 1\\
% |cdocsch2.tex|&include file for chapter 2\\
% |cdocspt3.tex|&include file for part 3\\
% |cdocspt4.tex|&include file for part 4\\
% |cdocsdrf.tex|&forwarding file for main file in draft mode\\
% |cdocsfi1.tex|&forwarding file for final version of chapter 1\\
% |cdocsfi2.tex|&forwarding file for final version of chapter 2\\
% \end{tabular}
% \end{center}
% Each of the eight files can be compiled directly by the \LaTeX{} compiler.
%
% %%%%%%%%%%%%%%%%%%%%%%%%%%%%%%%%%%%%%%
% \paragraph{Main File.}
%
% The main file is called |cdocsamp.tex|.
%
% Load the \textsf{childdoc} definitions and
% declare the filename for the main document:
%    \begin{macrocode}
\input{childdoc.def}
\childdocmain{}
%    \end{macrocode}

% Optional override for |\version| flag:
%    \begin{macrocode}
%%\ifchilddoc\else\providecommand{\version}{draft}\fi
%    \end{macrocode}

% Define the default values for the |\version| flag
% (|final| for the main file and |draft| for childs):
%    \begin{macrocode}
\ifchilddoc
\providecommand{\version}{draft}
\else
\providecommand{\version}{final}
\fi
%    \end{macrocode}

% Load the standard document class:
%    \begin{macrocode}
\documentclass[12pt]{article}
%    \end{macrocode}

% Start the document body:
%    \begin{macrocode}
\begin{document}
%    \end{macrocode}

% Declare a title page.
% Print title, part of document being processed and version flag:
%    \begin{macrocode}
\addtocounter{page}{-1}
\begin{center}
{\LARGE\bfseries{}childdoc example\par}
\vspace{1cm}
\ifchilddoc
\ifchilddocmanual part\else chapter\fi:
`\childdocname' of `\childdocjob'\par
\else
main document: `\childdocjob'\par
\fi
version: \version\par
\end{center}
\newpage
%    \end{macrocode}

% Manually include selected file,
% otherwise process as usual:
%    \begin{macrocode}
\ifchilddocmanual
\section*{part `\childdocname'}
\input{\childdocname}
\else
%    \end{macrocode}

% Include the two chapters:
%    \begin{macrocode}
\include{cdocsch1}
\include{cdocsch2}
%    \end{macrocode}

% Include the two parts unless only chapters should be displayed:
%    \begin{macrocode}
\ifchilddoc\else
\section{part three}
\input{cdocspt3}
\section{part four}
\input{cdocspt4}
\fi
%    \end{macrocode}

% Process as usual until here:
%    \begin{macrocode}
\fi
%    \end{macrocode}

% End of document body:
%    \begin{macrocode}
\end{document}
%    \end{macrocode}
%\iffalse
%</samplemain>
%\fi
%
% %%%%%%%%%%%%%%%%%%%%%%%%%%%%%%%%%%%%%%
% \paragraph{Chapter Include Files.}
%
% The include files are called |cdocsch1.tex| and |cdocsch2.tex|.
%
%\iffalse
%<*samplechap1|samplechap2>
%\fi

% Optional override for |\version| flag:
%    \begin{macrocode}
%%\providecommand{\version}{final}
%    \end{macrocode}

% Include the main document:
%    \begin{macrocode}
\input{childdoc.def}
\childdocof{cdocsamp}
%    \end{macrocode}

%\iffalse
%</samplechap1|samplechap2>
%\fi
%
%\iffalse
%<*samplechap1>
%\fi
% Some text for chapter 1:
%    \begin{macrocode}
\section{one}
some text in chapter one
%    \end{macrocode}

%\iffalse
%</samplechap1>
%\fi
% Some text for chapter 2:
%\iffalse
%<*samplechap2>
%\fi
%    \begin{macrocode}
\section{two}
more text in chapter two
%    \end{macrocode}

%\iffalse
%</samplechap2>
%\fi
%
% %%%%%%%%%%%%%%%%%%%%%%%%%%%%%%%%%%%%%%
% \paragraph{Part Include Files.}
%
% The include files are called |cdocspt3.tex| and |cdocspt4.tex|.
%
%\iffalse
%<*samplepart3|samplepart4>
%\fi

% Optional override for |\version| flag:
%    \begin{macrocode}
%%\providecommand{\version}{final}
%    \end{macrocode}

% Include the main document:
%    \begin{macrocode}
\input{childdoc.def}
\childdocby{cdocsamp}
%    \end{macrocode}

%\iffalse
%</samplepart3|samplepart4>
%\fi
%
%\iffalse
%<*samplepart3>
%\fi
% Some text for part 3:
%    \begin{macrocode}
some text in part three
%    \end{macrocode}

%\iffalse
%</samplepart3>
%\fi
% Some text for part 4:
%\iffalse
%<*samplepart4>
%\fi
%    \begin{macrocode}
more text in part four
%    \end{macrocode}

%\iffalse
%</samplepart4>
%\fi
%
% %%%%%%%%%%%%%%%%%%%%%%%%%%%%%%%%%%%%%%
% \paragraph{Forwarding for a Complete Draft.}
%
% The following forwarding file |cdocsdrf.tex|
% compiles the main document in draft mode:
%\iffalse
%<*sampledraft>
%\fi
%    \begin{macrocode}
\def\version{draft}
\input{childdoc.def}
\childdocforward{cdocsamp}
%    \end{macrocode}

%\iffalse
%</sampledraft>
%\fi
%
% %%%%%%%%%%%%%%%%%%%%%%%%%%%%%%%%%%%%%%
% \paragraph{Forwarding for Final Version of the Chapters.}
%
% The following forwarding files |cdocsfn1.tex| and |cdocsfn2.tex|
% (with identical content)
% compile the final versions of the child documents
% |cdocsch1.tex| and |cdocsch2.tex|, respectively:
%\iffalse
%<*samplefinal>
%\fi
%    \begin{macrocode}
\def\version{final}
\input{childdoc.def}
\childdocforwardprefix[cdocsamp]{cdocsfn}{cdocsch}
%    \end{macrocode}

%\iffalse
%</samplefinal>
%\fi
%
% %%%%%%%%%%%%%%%%%%%%%%%%%%%%%%%%%%%%%%
% \paragraph{Command Line Processing.}
%
% The following three command lines generate the output files
% |cdocscld|, |cdocscl1| and |cdocscl2|
% which should be identical to
% |cdocsdrf|, |cdocsch1| and |cdocsfn2|, respectively:
% \begin{center}
% \begin{tabular}{l}
% |latex -jobname cdocscld \|\\
% |  "\def\version{draft}\input{childdoc.def}\childdocforward{cdocsamp}"|\\
% |latex -jobname cdocscl1 \|\\
% |  "\input{childdoc.def}\childdocforward[cdocsamp]{cdocsch1}"|\\
% |latex -jobname cdocscl2 \|\\
% |  "\def\version{final}\input{childdoc.def}\childdocforward{cdocsch2}"|
% \end{tabular}
% \end{center}
% Note that the trailing backslash on each first line
% merely continues the input to the second line
% (for convenient cut ant paste).
% Furthermore, the command |latex| can be replaced by any
% of its alternative versions such as |pdflatex|.
%
% %%%%%%%%%%%%%%%%%%%%%%%%%%%%%%%%%%%%%%%%%%%%%%%%%%%%%%%%%%%%%%%%%%%%%%%%%%%%%%
% %%%%%%%%%%%%%%%%%%%%%%%%%%%%%%%%%%%%%%%%%%%%%%%%%%%%%%%%%%%%%%%%%%%%%%%%%%%%%%
% \section{Implementation}
%\iffalse
%<*package>
%\fi
%
% This section describes the definitions file |childdoc.def|.

% The definitions cannot be loaded using |\usepackage| or |\RequirePackage|
% which has a mechanism to prevent loading a style file more than once.
% When loading the definitions by means of |\input|
% multiple instances have to be prevented manually:
%\iffalse
%This code needs to be before the `\ProvidesFile' directive
%which is defined at the beginning of this file.
%Therefore it is also placed there and commented out here.
%</package>
%<*discard>
%\fi
%    \begin{macrocode}
\ifdefined\childdocmain\endinput\fi
%    \end{macrocode}
%\iffalse
%</discard>
%<*package>
%\fi
%
% \macro{\ifchilddoc}
% \macro{\ifchilddocmanual}
% The conditional |\ifchilddoc| tells whether a
% child (true) or main (false) document is being compiled.
% The conditional |\ifchilddocmanual| tells whether
% the |\includeonly| mechanism is used (false) or
% the selection of child files must be performed manually (true).
% The definitions initialise to false:
%    \begin{macrocode}
\newif\ifchilddoc
\newif\ifchilddocmanual
%    \end{macrocode}

% \macro{\childdocname}
% \macro{\childdocjob}
% The macro |\childdocname| stores the name of the main document
% to be compiled. The macro |\childdocjob| stores the name of
% the document on which the \LaTeX{} compiler was originally invoked.
% The content of |\jobname| cannot be compared
% to filenames specified in the source due to different catcodes.
% The following code rescans |\jobname|, stores the result
% in |\childdocname| and saves a copy in |\childdocjob|:
%    \begin{macrocode}
\edef\childdocname{\scantokens\expandafter{\jobname\noexpand}}
\let\childdocjob\childdocname
%    \end{macrocode}

% \macro{\childdocdisable}
% The macro |\childdocdisable| prevents the main file
% from being processed more than once.
% At this stage, the main document command |\childdocmain|
% is assumed to be called once again where it should do nothing.
% Any subsequent call to it should prevent
% a secondary processing of the main document
% It overwrites the forwarding commands
% |\childdocof| and |\childdocforward|
% with empty macros to prevent further inclusions of the main document:
%    \begin{macrocode}
\newcommand{\childdocdisable}
{
  \renewcommand{\childdocmain}[1]{\renewcommand{\childdocmain}[1]{\endinput}}
  \renewcommand{\childdocof}[1]{}
  \renewcommand{\childdocby}[2][]{}
  \renewcommand{\childdocforward}[2][]{}
  \renewcommand{\childdocdisable}{}
}
%    \end{macrocode}

% \macro{\childdocmain}
% The macro |\childdocmain| is to be called at the top of the main file
% with nothing or the main filename (without extension) as argument.
% First, it breaks loops.
% If the argument is not empty and does not match |\childdocname|
% (which is set by the first inclusion of |childdoc.def|),
% |\ifchilddoc| is set to true, |\includeonly| is applied to the child file
% and |\jobname| is set to the main file
% (for proper handling of |.aux| files):
%    \begin{macrocode}
\newcommand{\childdocmain}[1]
{
  \childdocdisable\childdocmain{}
  \if?#1?\else
    \begingroup
      \def\childdoctmp{#1}
      \ifx\childdoctmp\childdocname
        \def\childdoctmp{}
      \else
        \def\childdoctmp
        {
          \childdoctrue
          \includeonly{\childdocname}
          \def\childdocjob{#1}
          \def\jobname{#1}
        }
      \fi
      \expandafter
    \endgroup
    \childdoctmp
  \fi
}
%    \end{macrocode}

% \macro{\childdocof}
% The command |\childdocof| redirects
% compilation to the main file |#1|.
%    \begin{macrocode}
\newcommand{\childdocof}[1]
{
  \childdocdisable
  \childdoctrue
  \includeonly{\childdocname}
  \def\jobname{#1}
  \def\childdocjob{#1}
  \input{#1}
}
%    \end{macrocode}

% \macro{\childdocby}
% The command |\childdocby| ....
%    \begin{macrocode}
\newcommand{\childdocby}[2][]
{
  \childdocdisable
  \childdoctrue
  \childdocmanualtrue
  \if?#1?\else
    \def\jobname{#2}
  \fi
  \def\childdocjob{#2}
  \input{#2}
  \endinput
}
%    \end{macrocode}

% \macro{\childdocforward}
% The command |\childdocforward| redirects
% compilation to the main file or
% (if the optional argument is given) a child file.
% Parameters are set as if the main file
% or a child file starting with |\childdocof| was compiled.
% Then compilation is handed over to the main file:
%    \begin{macrocode}
\newcommand{\childdocforward}[2][]
{
  \begingroup
    \if?#1?
      \def\childdoctmp
      {
        \def\childdocname{#2}
        \def\childdocjob{#2}
        \def\jobname{#2}
        \input{#2}
        \endinput
      }
    \else
      \def\childdoctmp
      {
        \childdocdisable
        \def\childdocname{#2}
        \childdoctrue
        \includeonly{#2}
        \def\childdocjob{#1}
        \def\jobname{#1}
        \input{#1}
        \endinput
      }
    \fi
    \expandafter
  \endgroup
  \childdoctmp
}
%    \end{macrocode}

% \macro{\childdocforwardprefix}
% The command |\childdocforwardprefix| redirects
% compilation to the main or a child file by means of a pattern.
% The prefix |#1| in the current filename is replaced by |#2|
% and the suffix of the current filename is kept
% (it is assumed that the filename does not contain the substring `|~~~|'
% which is used as a delimiter).
% Compilation is handed over to the new file by |\childdocforward|:
%    \begin{macrocode}
\newcommand{\childdocforwardprefix}[3][]
{
  \begingroup
    \def\childdocextract #2##1~~~{\def\childdoctmp{\childdocforward[#1]{#3##1}}}
    \expandafter\childdocextract\childdocname~~~
    \expandafter
  \endgroup
  \childdoctmp
}
%    \end{macrocode}

% \macro{\childdoc}
% The deprecated macro |\childdoc| is a legacy version of |\childdocmain|:
%    \begin{macrocode}
\newcommand{\childdoc}{\childdocmain}
%    \end{macrocode}

% \macro{\childdocredirect}
% The deprecated macro |\childdocredirect| is a legacy version
% of |\childdocforward| and |\childdocforwardprefix|:
%    \begin{macrocode}
\newcommand{\childdocredirect}[2][]
{
  \begingroup
    \if?#1?
      \def\childdoctmp{\childdocforward{#2}}
    \else
      \def\childdoctmp{\childdocforwardprefix{#1}{#2}}
    \fi
    \expandafter
  \endgroup
  \childdoctmp
}
%    \end{macrocode}

%\iffalse
%</package>
%\fi
%
\endinput

\childdocby{cdocsamp}
%    \end{macrocode}

%\iffalse
%</samplepart3|samplepart4>
%\fi
%
%\iffalse
%<*samplepart3>
%\fi
% Some text for part 3:
%    \begin{macrocode}
some text in part three
%    \end{macrocode}

%\iffalse
%</samplepart3>
%\fi
% Some text for part 4:
%\iffalse
%<*samplepart4>
%\fi
%    \begin{macrocode}
more text in part four
%    \end{macrocode}

%\iffalse
%</samplepart4>
%\fi
%
% %%%%%%%%%%%%%%%%%%%%%%%%%%%%%%%%%%%%%%
% \paragraph{Forwarding for a Complete Draft.}
%
% The following forwarding file |cdocsdrf.tex|
% compiles the main document in draft mode:
%\iffalse
%<*sampledraft>
%\fi
%    \begin{macrocode}
\def\version{draft}
% \iffalse
%
% childdoc.dtx Copyright (C) 2017-2018 Niklas Beisert
%
% This work may be distributed and/or modified under the
% conditions of the LaTeX Project Public License, either version 1.3
% of this license or (at your option) any later version.
% The latest version of this license is in
%   http://www.latex-project.org/lppl.txt
% and version 1.3 or later is part of all distributions of LaTeX
% version 2005/12/01 or later.
%
% This work has the LPPL maintenance status `maintained'.
%
% The Current Maintainer of this work is Niklas Beisert.
%
% This work consists of the files childdoc.dtx and childdoc.ins
% and the derived files childdoc.def and cdocsamp.tex with
% cdocsch1.tex, cdocsch2.tex, cdocsdrf.tex, cdocsfn1.tex, cdocsfn2.tex.
%
%<package>\ifdefined\childdocmain\endinput\fi
%<package>\ProvidesFile{childdoc.def}[2018/12/30 v2.0 child document driver]
%<samplemain>\ProvidesFile{cdocsamp.tex}[2018/12/30 v2.0 sample for childdoc]
%<*driver>
%\ProvidesFile{childdoc.drv}[2018/12/30 v2.0 childdoc reference manual file]
\PassOptionsToClass{10pt,a4paper}{article}
\documentclass{ltxdoc}

\usepackage[margin=35mm]{geometry}
\usepackage{hyperref}
\usepackage{hyperxmp}
\usepackage[usenames]{color}

\hypersetup{colorlinks=true}
\hypersetup{pdfstartview=FitH}
\hypersetup{pdfpagemode=UseNone}
\hypersetup{pdfsource={}}
\hypersetup{pdflang={en-UK}}
\hypersetup{pdfcopyright={Copyright 2017-2018 Niklas Beisert.
  This work may be distributed and/or modified under the
  conditions of the LaTeX Project Public License, either version 1.3
  of this license or (at your option) any later version.}}
\hypersetup{pdflicenseurl={http://www.latex-project.org/lppl.txt}}
\hypersetup{pdfcontactaddress={ETH Zurich, ITP, HIT K,
  Wolfgang-Pauli-Strasse 27}}
\hypersetup{pdfcontactpostcode={8093}}
\hypersetup{pdfcontactcity={Zurich}}
\hypersetup{pdfcontactcountry={Switzerland}}
\hypersetup{pdfcontactemail={nbeisert@itp.phys.ethz.ch}}
\hypersetup{pdfcontacturl={http://people.phys.ethz.ch/\xmptilde nbeisert/}}

\newcommand{\secref}[1]{\hyperref[#1]{section \ref*{#1}}}

\parskip1ex
\parindent0pt
\let\olditemize\itemize
\def\itemize{\olditemize\parskip0pt}

\begin{document}

\title{The \textsf{childdoc} Package}
\hypersetup{pdftitle={The childdoc Package}}
\author{Niklas Beisert\\[2ex]
  Institut f\"ur Theoretische Physik\\
  Eidgen\"ossische Technische Hochschule Z\"urich\\
  Wolfgang-Pauli-Strasse 27, 8093 Z\"urich, Switzerland\\[1ex]
  \href{mailto:nbeisert@itp.phys.ethz.ch}
  {\texttt{nbeisert@itp.phys.ethz.ch}}}
\hypersetup{pdfauthor={Niklas Beisert}}
\hypersetup{pdfsubject={Manual for the LaTeX2e Package childdoc}}
\date{30 December 2018, \textsf{v2.0}}
\maketitle

\begin{abstract}\noindent
\textsf{childdoc} is a \LaTeXe{} package
that enables the direct compilation
of document sections included by |\include|
to individual files.
\end{abstract}

\begingroup
\parskip0ex
\tableofcontents
\endgroup

%%%%%%%%%%%%%%%%%%%%%%%%%%%%%%%%%%%%%%%%%%%%%%%%%%%%%%%%%%%%%%%%%%%%%%%%%%%%%%%%
%%%%%%%%%%%%%%%%%%%%%%%%%%%%%%%%%%%%%%%%%%%%%%%%%%%%%%%%%%%%%%%%%%%%%%%%%%%%%%%%
\section{Introduction}

\LaTeX{} provides a mechanism to structure a large document (such as a book)
into a main file and several child files (containing the chapters)
using the |\include| command.
This mechanism is beneficial for documents
which span hundreds of pages in order to
make the source file(s) more manageable.
Moreover, compilation can be restricted to
selected child files by means of the |\includeonly| command.
The latter feature can be used to reduce the compilation time while editing
(this was significantly more useful in the earlier days of \LaTeX{})
or to generate a smaller document which is easier to navigate.
Another application of |\includeonly| is to generate
documents consisting of selected parts of the complete document.

However, there are a few drawbacks of the plain |\include| mechanism:
\begin{itemize}
\item
The child files cannot be compiled on their own,
they can only be compiled via the main file.
A naive editing environment
(such as a text editor with an option
to have the current file processed by \LaTeX)
may require one to switch to the main file before compiling;
attempting to compile the child file produces errors.
\item
The main file must be modified (each time)
to adjust the |\includeonly| command
to the present needs. This easily leaves the main file in a messy state.
\item
The generated document will always carry the filename
of the main document. This is inconvenient if
several child files are to be compiled and
to be kept for distribution.
\end{itemize}

The present package provides a simple interface
to make child files individually compilable by \LaTeX{}.
Compiling a child file then has the same effect as compiling
the main file with an |\includeonly| command
to select the appropriate child.
Moreover the generated document will carry the name of the child
rather than the main file.
This resolves all three above issues.

This feature is meant to make the editing of books,
thesis documents and lecture notes somewhat more convenient.
However, the package can also be used efficiently for
composing a series of documents (such as exercise sheets)
which are typically distributed individually.
It then assists the author in generating the individual documents
(potentially in different versions)
as well as a document containing the collected series.
Another application is in developing style files
or other kinds of included material
where compilation of the style file could redirect
to a sample or test file.

%%%%%%%%%%%%%%%%%%%%%%%%%%%%%%%%%%%%%%%%%%%%%%%%%%%%%%%%%%%%%%%%%%%%%%%%%%%%%%%%
%%%%%%%%%%%%%%%%%%%%%%%%%%%%%%%%%%%%%%%%%%%%%%%%%%%%%%%%%%%%%%%%%%%%%%%%%%%%%%%%
\section{Usage}

First of all, the package \textsf{childdoc} is \emph{not} a standard
\LaTeXe{} |.sty| style file! Therefore it needs to be invoked in
a non-standard way.

%%%%%%%%%%%%%%%%%%%%%%%%%%%%%%%%%%%%%%%%%%%%%%%%%%%%%%%%%%%%%%%%%%%%%%%%%%%%%%%%
\subsection{Included Files}
\label{sec:include}

%%%%%%%%%%%%%%%%%%%%%%%%%%%%%%%%%%%%%%%%
\DescribeMacro{\childdocmain}
To use the package, add the commands
\begin{center}
\begin{tabular}{l}
|\input{childdoc.def}|\\
|\childdocmain{}|\\
\end{tabular}
\end{center}
at the very top of the main \LaTeX{} file,
in particular \emph{before} the |\documentclass| statement!
The argument of |\childdocmain| should be left empty
(but it must be present).

%%%%%%%%%%%%%%%%%%%%%%%%%%%%%%%%%%%%%%%%
\DescribeMacro{\childdocof}
Furthermore, add the commands
\begin{center}
\begin{tabular}{l}
|\input{childdoc.def}|\\
|\childdocof{|\textit{main}|}|\\
\end{tabular}
\end{center}
at the top of every child file \textit{child}
which is included by |\include{|\textit{child}|}|
from within the main file
(or at least for those files to be compiled individually).
The argument \textit{main} must be the filename of the main file.

There are a couple of
considerations in setting up the main and child documents:

%%%%%%%%%%%%%%%%%%%%%%%%%%%%%%%%%%%%%%%%
\paragraph{Restrictions.}

Please note the following restrictions:
\begin{itemize}
\item
|\childdocmain| must be called with one argument \textit{main}
to ensure compatibility with earlier version of the package.
It must either be empty (|\childdocmain{}|)
or precisely match the filename of the main file in which it is specified.
See \secref{sec:detection} for further information.
\item
The filename \textit{main} must be specified without the |.tex| extension.
\item
The filename \textit{main} is case sensitive
(even in case-insensitive file systems)
due to internal string comparison.
\item
The argument \textit{main} should be fully expanded, it cannot be a macro.
\item
Subdirectories and special characters should be avoided in filenames.
\item
The command |\childdocmain{|\textit{main}|}| must be followed by a whitespace.
It should not be followed immediately by another command
or by a comment mark `|%|'.
This is because the \TeX{} parser reads the token immediately following
the argument of |\childdocmain| and puts it
at the beginning of every child section;
however, a white\-space is ignored.
\end{itemize}

%%%%%%%%%%%%%%%%%%%%%%%%%%%%%%%%%%%%%%%%
\paragraph{Content of Main File.}

It is advisable to place all content in the child files included by |\include|.
Any output contained in the main file will appear in all child documents
unless suppressed manually;
it cannot be suppressed automatically by the |\includeonly| directive
and thus should normally be avoided.
A method to include some content in the main file
by means of conditional processing is described in \secref{sec:conditional}.

%%%%%%%%%%%%%%%%%%%%%%%%%%%%%%%%%%%%%%%%
\paragraph{Page Numbering.}

When only a part of the document is compiled,
the appropriate numbering of pages
(as well as other status parameters)
is determined from the |.aux| files.
The latter contain information from previous passes.
However this information needs to propagate through
all intermediate child documents.
Therefore the page numbering in child documents may well
be inconsistent until the complete document is compiled at least once.

A useful (if unconventional) way to always ensure a consistent
page numbering is to restart the numbering in each child document
and denote the pages by `\textit{child}|.|\textit{page}'
where \textit{child} represents the chapter/section number of the child file.
This can be achieved by the command
|\numberwithin{page}{|\textit{child}|}|
of the \textsf{amsmath} package
where \textit{child} can be |chapter| or |section|
depending on the chosen structuring.
Alternatively, one can modify the macro |\thepage| appropriately
and reset the counter |page| at the start of each child file.

%%%%%%%%%%%%%%%%%%%%%%%%%%%%%%%%%%%%%%%%%%%%%%%%%%%%%%%%%%%%%%%%%%%%%%%%%%%%%%%%
\subsection{Conditional Processing}
\label{sec:conditional}

The package provides a mechanism to compile different versions
of a document. To customise the versions further some conditional processing
can come in handy to distinguish which version is being compiled.
The package provides two macros to describe the compilation context:

%%%%%%%%%%%%%%%%%%%%%%%%%%%%%%%%%%%%%%%%
\DescribeMacro{\ifchilddoc}
The conditional |\ifchilddoc| distinguishes between the compilation of
child documents and the main document:
%
\begin{center}
|\ifchilddoc |\textit{child-code}| |[|\||else |\textit{main-code}]| \||fi|
\end{center}

%%%%%%%%%%%%%%%%%%%%%%%%%%%%%%%%%%%%%%%%
\DescribeMacro{\childdocname}
\DescribeMacro{\childdocjob}
The macro |\childdocname| contains the filename (without extension)
of the main or child file being processed.
Note that |\childdocjob| will always contain the name of the main file.

%%%%%%%%%%%%%%%%%%%%%%%%%%%%%%%%%%%%%%%%
\paragraph{Title Page.}

Conditional processing can be used to include a title or banner page
in the main document when proper precautions are taken.
Importantly, the code in the main file should ensure that the page counter
(as well as other status parameters which are stored in the |.aux| files)
takes the same value after the conditional processing.
Otherwise the page numbers may take divergent values
depending on which part is compiled.

For example, a title page could be declared by:
%
\begin{center}
\begin{tabular}{l}
|\ifchilddoc\||else|\\
|\addtocounter{page}{-1}|\\
\textit{code for title page}\\
|\newpage|\\
|\||fi|
\end{tabular}
\end{center}
%
A banner page for the child documents can be generated by:
%
\begin{center}
\begin{tabular}{l}
|\ifchilddoc|\\
|\addtocounter{page}{-1}|\\
\textit{code for banner page}\\
|\newpage|\\
|\||fi|
\end{tabular}
\end{center}
%
Here one could write a message such as:
\begin{center}
|This is the part \childdocname{} of \childdocjob{}.|
\end{center}

%%%%%%%%%%%%%%%%%%%%%%%%%%%%%%%%%%%%%%%%%%%%%%%%%%%%%%%%%%%%%%%%%%%%%%%%%%%%%%%%
\subsection{Flags}
\label{sec:flags}

The package makes it easy to generate different versions
of the main or child documents.
To this end compilation flags can be defined
and assigned different default values.
They will be particularly useful in conjunction
with the forwarding mechanism described in \secref{sec:forward}.

For example, it may be useful to have a flag |\version|
which can be set to |draft| or |final|.
The document source will contain some conditional code
depending on the value of |\version|.
Suppose further, the flag should default to |final| for the main file
and to |draft| for child files
which is a natural assignment for editing the document.
This is achieved by placing the following code
in the preamble of the main document
(below the |\childdocmain| directive):
%
\begin{center}
\begin{tabular}{l}
|\ifchilddoc|\\
|\providecommand{\version}{draft}|\\
|\||else|\\
|\providecommand{\version}{final}|\\
|\||fi|
\end{tabular}
\end{center}
%
The definition by |\providecommand| makes sure
that previous definitions are not overwritten.
Further statements |\providecommand{\version}{...}|
can thus be added before the above code to override it.

For the main file, one might add a line
(between |\childdocmain| and the above block)
%
\begin{center}
|%\ifchilddoc\||else\providecommand{\version}{draft}\||fi|
\end{center}
%
which can be uncommented to produce a draft version.
Likewise one can add a line to the very top of a child file
(above the |\childdocof{|\textit{main}|}| directive)
%
\begin{center}
|%\providecommand{\version}{final}|
\end{center}
%
which can be uncommented to produce the final version of this child document.

%%%%%%%%%%%%%%%%%%%%%%%%%%%%%%%%%%%%%%%%%%%%%%%%%%%%%%%%%%%%%%%%%%%%%%%%%%%%%%%%
\subsection{Forwarding}
\label{sec:forward}

Different versions of the main or child documents
using compilation flags as described in \secref{sec:flags}
can be (permanently) stored in different files
for convenient compilation, viewing and distribution.
To this end, the package defines a command
to pass on compilation to a different file:

%%%%%%%%%%%%%%%%%%%%%%%%%%%%%%%%%%%%%%%%
\DescribeMacro{\childdocforward}
The command |\childdocforward| redirects processing to
another source file:
%
\begin{center}
\begin{tabular}{l}
|\input{childdoc.def}|\\
|\childdocforward[|\textit{main}|]{|\textit{dest}|}|\\
\end{tabular}
\end{center}
%
The argument \textit{dest} is the destination file
(without extension).
It should be the main file or one of the child files.
Note that further \textsf{childdoc} directives
such as |\childdocof| and |\childdocforward|
in the indicated file will be processed in this form.
The optional argument \textit{main}
passes on directly to the main file \textit{main}
while pretending to compile the child \textit{dest}.
This form behaves as if \textit{dest}
issues |\childdocof{|\textit{main}|}| right away,
and no further \textsf{childdoc} directives will be processed.

%%%%%%%%%%%%%%%%%%%%%%%%%%%%%%%%%%%%%%%%
\DescribeMacro{\...prefix}
In the alternative form |\childdocforwardprefix|,
%
\begin{center}
\begin{tabular}{l}
|\input{childdoc.def}|\\
|\childdocforwardprefix[|\textit{main}|]{|\textit{prefix}|}{|\textit{dest}|}|
\end{tabular}
\end{center}
%
the destination file is determined by a pattern
depending on the current file:
To make this work, the current file must be called
`{\textit{prefix}\hspace{0.2em}\textit{suffix}}'
with \textit{prefix} matching precisely the argument.
Processing is then passed on to the file
`{\textit{dest}\hspace{0.2em}\textit{suffix}}'.
Surely, the same effect is achieved by
directly specifying the
argument `{\textit{dest}\hspace{0.2em}\textit{suffix}}'
in the first form.
However, that requires to set up a different file
for each child. With the alternative form of the command
all these files can have exactly the same content
which simplifies setting them up and maintaining them.

For example, the following file |draft.tex|
with a compilation flag |\version| as described in \secref{sec:flags}
compiles the main document as a draft:
%
\begin{center}
\begin{tabular}{l}
|\def\version{draft}|\\
|\input{childdoc.def}|\\
|\childdocforward{|\textit{main}|}|
\end{tabular}
\end{center}
%
Likewise, the following files |final|\textit{nn}|.tex|
compile the final version of the child document
|child|\textit{nn}|.tex|:
%
\begin{center}
\begin{tabular}{l}
|\def\version{final}|\\
|\input{childdoc.def}|\\
|\childdocforwardprefix{final}{child}|
\end{tabular}
\end{center}
%

Note that when several versions of a main file and/or of each child file
are to be generated, it may be convenient to set up a |Makefile| or
shell script to automatise the process.

%%%%%%%%%%%%%%%%%%%%%%%%%%%%%%%%%%%%%%%%%%%%%%%%%%%%%%%%%%%%%%%%%%%%%%%%%%%%%%%%
\subsection{Command Line Processing}
\label{sec:commandline}

The effect of redirection files can also be achieved by invoking
the \LaTeX{} compiler with a more elaborate command line.
Most conveniently this should be done as part
of a shell script or a |Makefile|.

When using \textsf{childdoc} in the main file, the following
command lines effectively perform a redirection
(note that depending on the shell being used,
backslashes may have to be doubled: `|\|' $\to$ `|\\|'):
%
\begin{center}
|... -jobname "|\textit{target}|" |\\|"|[\textit{flags}]%
|\input{childdoc.def}\childdocforward[|\textit{main}|]{|\textit{dest}|}"|
\end{center}
%
Here \textit{target} is the name of the output file,
\textit{main} is the name of the main file
and \textit{dest} is the name of the main or child file to be processed
(all filenames without extensions).
The optional argument \textit{main} can be omitted
if \textit{main} matches \textit{dest}.
Optionally, compilation \textit{flags} can be defined via |\def| commands.
This command line makes the \TeX{} engine believe
it is compiling the file \textit{target}
whose content is specified as the latter parameter.
The provided code then forwards the processing to
\textit{main} or \textit{dest} as described in \secref{sec:forward}.

%%%%%%%%%%%%%%%%%%%%%%%%%%%%%%%%%%%%%%%%%%%%%%%%%%%%%%%%%%%%%%%%%%%%%%%%%%%%%%%%
\subsection{Include by Input}
\label{sec:input}

Including child documents by |\include| has some restrictions by design.
Most notably, the content of a child document always occupies
its own set of pages; pages cannot be shared between child documents.
Usually, this behaviour makes perfect sense
because each child document contain an essential part of the document.
However, in some situations it may be desirable to compose
a document from a collection of parts
without having mandatory page breaks between then.
For this case, the package
provides a mechanism to include parts
by |\input| which can also be processed individually.
However, by construction this mechanism
requires manual handling of the content to be output.

%%%%%%%%%%%%%%%%%%%%%%%%%%%%%%%%%%%%%%%%
\DescribeMacro{\ifchilddocmanual}
The main file should be prepared as usual, see \secref{sec:include}.
However, the document body must make a distinction
between processing of an individual part and of the main document, e.g.:
%
\begin{center}
\begin{tabular}{l}
|\ifchilddocmanual|\\
|\input{\childdocname}|\\
|\||else|\\
\textit{document body with }|\input{|\textit{part}|}|\\
|\||fi|
\end{tabular}
\end{center}
%
The conditional |\ifchilddocmanual| is true whenever
a part to be included by |\input| is being compiled,
and the name of the part is stored in |\childdocname|.

%%%%%%%%%%%%%%%%%%%%%%%%%%%%%%%%%%%%%%%%
\DescribeMacro{\childdocby}
Each part to be included by |\input| should start with:
%
\begin{center}
\begin{tabular}{l}
|\input{childdoc.def}|\\
|\childdocby{|\textit{main}|}|\\
\end{tabular}
\end{center}
%
The directive |\childdocby| is similar to |\childdocof|
described in \secref{sec:include},
but the subsequent selection of content must be done manually.
To that end, both |\ifchilddoc| and |\ifchilddocmanual|
will be true upon processing of a part,
and the name of the part is stored in |\childdocname|.
Note that |\jobname| will be set to the filename of the current part
so that each part receives an individual |.aux| file
that does not interfere with the |.aux| file(s) of the main document.
This behaviour can be altered by the alternative form
|\childdocby[*]{|\textit{main}|}| (with a non-empty optional argument)
which uses the |.aux| file of the main document
by setting |\jobname| to \textit{main}.

%%%%%%%%%%%%%%%%%%%%%%%%%%%%%%%%%%%%%%%%%%%%%%%%%%%%%%%%%%%%%%%%%%%%%%%%%%%%%%%%
\subsection{Driver Development}
\label{sec:driver}

The \textsf{childdoc} mechanism can also be use for the development
of definition files such as \LaTeX{} styles or classes.
This case differs from the above setup with multiple parts
included by |\include| in that no |\includeonly| should be invoked.
This can be achieved by starting the include file
(before |\ProvidesPackage|) with:
%
\begin{center}
\begin{tabular}{l}
|\input{childdoc.def}|\\
|\childdocforward{|\textit{main}|}|\\
\end{tabular}
\end{center}
%
or alternatively with:
%
\begin{center}
\begin{tabular}{l}
|\input{childdoc.def}|\\
|\childdocby{|\textit{main}|}|\\
\end{tabular}
\end{center}
%
Both forms have slightly different effects as described above.
The main file is prepared as usual, see \secref{sec:include}.

%%%%%%%%%%%%%%%%%%%%%%%%%%%%%%%%%%%%%%%%%%%%%%%%%%%%%%%%%%%%%%%%%%%%%%%%%%%%%%%%
\subsection{Legacy Detection}
\label{sec:detection}

The directive |\childdocmain| in the main file can detect
whether the complete document or merely a child is to be compiled
even without using the directive |\childdocof|.
This method is deprecated because it is less robust
and there is no compelling reason to use it;
it is merely provided for backward compatibility
and it may be removed in future versions.

If the detection mechanism is to be used,
it is mandatory to correctly specify
the filename of the main file as the argument of |\childdocmain|:
%
\begin{center}
\begin{tabular}{l}
|\input{childdoc.def}|\\
|\childdocmain{|\textit{main}|}|\\
\end{tabular}
\end{center}
%
If |\jobname| does not match the argument \textit{main} of |\childdocmain|,
it is assumed that |\jobname| points to the child file to be compiled.
When using |\childdocmain| with the main file specified as argument,
it suffices to start a child file
with just |\input{|\textit{main}|}|
without loading of the package and using |\childdocof|.
If instead all processing is done
with the appropriate \textsf{childdoc} directives,
the argument of \textit{main} of |\childdocmain| can be empty.

An alternative version of the command line processing described
in \secref{sec:commandline} using the detection mechanism reads:
%
\begin{center}
|... -jobname "|\textit{target}|" "|[\textit{flags}]%
[|\def\jobname{|\textit{dest}|}|]|\input{|\textit{main}|}"|
\end{center}

%%%%%%%%%%%%%%%%%%%%%%%%%%%%%%%%%%%%%%%%%%%%%%%%%%%%%%%%%%%%%%%%%%%%%%%%%%%%%%%%
\subsection{Manual Code}
\label{sec:manual}

In case one cannot be certain whether the definitions file |childdoc.def|
is installed on the target \TeX{} distribution
and one prefers not to ship it,
it is conceivable to paste a few relevant commands into the sources.

To that end, drop all statements |\input{childdoc.def}|
and perform the replacements as outlined below.
Instead of |\childdocmain{|\textit{main}|}| add the following code
to the top of the main file:
%
\begin{center}
\begin{tabular}{l}
|\||ifdefined\childdocname\endinput\||fi\newif\ifchilddoc|\\
|\edef\childdocname{\scantokens\expandafter{\jobname\noexpand}}|\\
|\def\childdocmain{|\textit{main}|}\||ifx\childdocmain\childdocname\||else|\\
|\childdoctrue\includeonly{\childdocname}\let\jobname\childdocmain\||fi|\\
\end{tabular}
\end{center}
%
Instead of |\childdocof{|\textit{main}|}| just include the main file
at the top of each child file:
%
\begin{center}
|\input{|\textit{main}|}|
\end{center}
%
A simple redirection |\childdocforward{|\textit{dest}|}| is achieved by:
%
\begin{center}
|\def\jobname{|\textit{dest}|}\input{\jobname}|
\end{center}
%
The redirection with prefix
|\childdocforwardprefix[|\textit{prefix}|]{|\textit{dest}|}|
is accomplished by:
%
\begin{center}
\begin{tabular}{l}
|{\edef\jobname{\scantokens\expandafter{\jobname\noexpand}}|\\
|\def\redirectjob |\textit{prefix}|#1~~~{\gdef\jobname{|\textit{dest}|#1}}|\\
|\expandafter\redirectjob\jobname~~~}\input{\jobname}|
\end{tabular}
\end{center}

In an alternative approach,
child documents can be compiled by a specific command line
without additional code or specific definitions:
%
\begin{center}
|... -jobname "|\textit{target}|" "|[\textit{flags}]%
|\includeonly{|\textit{dest}|}\input{|\textit{main}|}"|
\end{center}
%

%%%%%%%%%%%%%%%%%%%%%%%%%%%%%%%%%%%%%%%%%%%%%%%%%%%%%%%%%%%%%%%%%%%%%%%%%%%%%%%%
%%%%%%%%%%%%%%%%%%%%%%%%%%%%%%%%%%%%%%%%%%%%%%%%%%%%%%%%%%%%%%%%%%%%%%%%%%%%%%%%
\section{Information}

%%%%%%%%%%%%%%%%%%%%%%%%%%%%%%%%%%%%%%%%%%%%%%%%%%%%%%%%%%%%%%%%%%%%%%%%%%%%%%%%
\subsection{Copyright}

Copyright \copyright{} 2017--2018 Niklas Beisert

This work may be distributed and/or modified under the
conditions of the \LaTeX{} Project Public License, either version 1.3
of this license or (at your option) any later version.
The latest version of this license is in
  \url{http://www.latex-project.org/lppl.txt}
and version 1.3 or later is part of all distributions of \LaTeX{}
version 2005/12/01 or later.

This work has the LPPL maintenance status `maintained'.

The Current Maintainer of this work is Niklas Beisert.

This work consists of the files |README.txt|, |childdoc.ins| and |childdoc.dtx|
as well as the derived files |childdoc.def|, |cdocsamp.tex|
with |cdocsch1.tex|, |cdocsch2.tex|, |cdocspt3.tex|, |cdocspt4.tex|,
|cdocsdrf.tex|, |cdocsfn1.tex|, |cdocsfn2.tex|
as well as |childdoc.pdf|.

%%%%%%%%%%%%%%%%%%%%%%%%%%%%%%%%%%%%%%%%%%%%%%%%%%%%%%%%%%%%%%%%%%%%%%%%%%%%%%%%
\subsection{Files and Installation}

The package consists of the files:
%
\begin{center}
\begin{tabular}{ll}
    |README.txt|   & readme file \\
    |childdoc.ins| & installation file \\
    |childdoc.dtx| & source file \\
    |childdoc.def| & definition file \\
    |cdocsamp.tex| & sample main file \\
    |cdocsch1.tex| & sample include file \\
    |cdocsch2.tex| & sample include file \\
    |cdocspt3.tex| & sample part file \\
    |cdocspt4.tex| & sample part file \\
    |cdocsdrf.tex| & sample redirection file \\
    |cdocsfn1.tex| & sample redirection file \\
    |cdocsfn2.tex| & sample redirection file \\
    |childdoc.pdf| & manual
\end{tabular}
\end{center}
%
The distribution consists of the files
|README.txt|, |childdoc.ins| and |childdoc.dtx|.
%
\begin{itemize}
\item
Run (pdf)\LaTeX{} on |childdoc.dtx|
to compile the manual |childdoc.pdf| (this file).
\item
Run \LaTeX{} on |childdoc.ins| to create the definitions file |childdoc.def|
and the sample |cdocsamp.tex| with include files
|cdocsch1.tex|, |cdocsch2.tex|, |cdocspt3.tex|, |cdocspt4.tex|,
|cdocsdrf.tex|, |cdocsfn1.tex|, |cdocsfn2.tex|.
Then copy the file |childdoc.def| to an appropriate directory of your \LaTeX{}
distribution, e.g.\ \textit{texmf-root}|/tex/latex/childdoc|.
\end{itemize}

%%%%%%%%%%%%%%%%%%%%%%%%%%%%%%%%%%%%%%%%%%%%%%%%%%%%%%%%%%%%%%%%%%%%%%%%%%%%%%%%
\subsection{Related CTAN Packages}

There are several other packages which offer a similar functionality:
%
\begin{itemize}
\item
The packages
\href{http://ctan.org/pkg/docmute}{\textsf{docmute}},
\href{http://ctan.org/pkg/includex}{\textsf{includex}} and
\href{http://ctan.org/pkg/standalone}{\textsf{standalone}}
provide commands to include only the document body of
a child file thus allowing both files to be compiled individually.
\item
The packages \href{http://ctan.org/pkg/subdocs}{\textsf{subdocs}}
and \href{http://ctan.org/pkg/subfiles}{\textsf{subfiles}}
provide structures in which the main and child documents can be
encapsulated and allowing them to be compiled individually.
The inclusion mechanism is different from the conventional |\include|.
\item
The package \href{http://ctan.org/pkg/combine}{\textsf{combine}}
is an elaborate solution to combine several documents into one.
\end{itemize}
%
See also the CTAN topic \href{http://ctan.org/topic/subdocs}{\textsf{subdocs}}
for further related packages.
The present package differs from the above solutions in that
a document structure constructed with the conventional |\include| mechanism
just needs two extra commands at the top of every file
such that all constituent files can be compiled individually.

%%%%%%%%%%%%%%%%%%%%%%%%%%%%%%%%%%%%%%%%%%%%%%%%%%%%%%%%%%%%%%%%%%%%%%%%%%%%%%%%
%\subsection{Feature Suggestions}
%
%The following is a list of features which may be useful for future
%versions of this package:
%%
%\begin{itemize}
%\item
%\ldots
%\end{itemize}

%%%%%%%%%%%%%%%%%%%%%%%%%%%%%%%%%%%%%%%%%%%%%%%%%%%%%%%%%%%%%%%%%%%%%%%%%%%%%%%%
\subsection{Revision History}

%%%%%%%%%%%%%%%%%%%%%%%%%%%%%%%%%%%%%%%%
\paragraph{v2.0:} 2018/12/30

\begin{itemize}
\item
immediate forward processing
\item
added |\childdocby| mechanism
\item
manual restructured
\end{itemize}

%%%%%%%%%%%%%%%%%%%%%%%%%%%%%%%%%%%%%%%%
\paragraph{v1.6:} 2018/01/17

\begin{itemize}
\item
application for development of include files
\item
corrections to manual
\end{itemize}

%%%%%%%%%%%%%%%%%%%%%%%%%%%%%%%%%%%%%%%%
\paragraph{v1.5:} 2017/05/21

\begin{itemize}
\item
more complete structuring introduced
\item
|\childdocof| introduced
\item
|\childdoc| renamed to |\childdocmain|
\item
|\childredirect| renamed to |\childdocforward| and |\childdocforwardprefix|
and functionality expanded
\end{itemize}

%%%%%%%%%%%%%%%%%%%%%%%%%%%%%%%%%%%%%%%%
\paragraph{v1.0:} 2017/04/27

\begin{itemize}
\item
manual and install package
\item
first version published on CTAN
\end{itemize}

%%%%%%%%%%%%%%%%%%%%%%%%%%%%%%%%%%%%%%%%
\paragraph{v0.6:} 2017/04/26

\begin{itemize}
\item
redirection mechanism added
\end{itemize}

%%%%%%%%%%%%%%%%%%%%%%%%%%%%%%%%%%%%%%%%
\paragraph{v0.5:} 2017/04/26

\begin{itemize}
\item
functionality in definition file
\end{itemize}


%%%%%%%%%%%%%%%%%%%%%%%%%%%%%%%%%%%%%%%%%%%%%%%%%%%%%%%%%%%%%%%%%%%%%%%%%%%%%%%%
%%%%%%%%%%%%%%%%%%%%%%%%%%%%%%%%%%%%%%%%%%%%%%%%%%%%%%%%%%%%%%%%%%%%%%%%%%%%%%%%
%%%%%%%%%%%%%%%%%%%%%%%%%%%%%%%%%%%%%%%%%%%%%%%%%%%%%%%%%%%%%%%%%%%%%%%%%%%%%%%%
\appendix

\settowidth\MacroIndent{\rmfamily\scriptsize 000\ }

 \DocInput{childdoc.dtx}

\end{document}
%</driver>
% \fi
%
% %%%%%%%%%%%%%%%%%%%%%%%%%%%%%%%%%%%%%%%%%%%%%%%%%%%%%%%%%%%%%%%%%%%%%%%%%%%%%%
% %%%%%%%%%%%%%%%%%%%%%%%%%%%%%%%%%%%%%%%%%%%%%%%%%%%%%%%%%%%%%%%%%%%%%%%%%%%%%%
% \section{Sample}
%\iffalse
%<*samplemain>
%\fi
%
% The following presents a sample document
% with two chapters, two parts, a title page,
% a compile flag as well as three forwarding files to set the flag.
% It consists of eight |.tex| files:
% \begin{center}
% \begin{tabular}{ll}
% |cdocsamp.tex|&main file\\
% |cdocsch1.tex|&include file for chapter 1\\
% |cdocsch2.tex|&include file for chapter 2\\
% |cdocspt3.tex|&include file for part 3\\
% |cdocspt4.tex|&include file for part 4\\
% |cdocsdrf.tex|&forwarding file for main file in draft mode\\
% |cdocsfi1.tex|&forwarding file for final version of chapter 1\\
% |cdocsfi2.tex|&forwarding file for final version of chapter 2\\
% \end{tabular}
% \end{center}
% Each of the eight files can be compiled directly by the \LaTeX{} compiler.
%
% %%%%%%%%%%%%%%%%%%%%%%%%%%%%%%%%%%%%%%
% \paragraph{Main File.}
%
% The main file is called |cdocsamp.tex|.
%
% Load the \textsf{childdoc} definitions and
% declare the filename for the main document:
%    \begin{macrocode}
\input{childdoc.def}
\childdocmain{}
%    \end{macrocode}

% Optional override for |\version| flag:
%    \begin{macrocode}
%%\ifchilddoc\else\providecommand{\version}{draft}\fi
%    \end{macrocode}

% Define the default values for the |\version| flag
% (|final| for the main file and |draft| for childs):
%    \begin{macrocode}
\ifchilddoc
\providecommand{\version}{draft}
\else
\providecommand{\version}{final}
\fi
%    \end{macrocode}

% Load the standard document class:
%    \begin{macrocode}
\documentclass[12pt]{article}
%    \end{macrocode}

% Start the document body:
%    \begin{macrocode}
\begin{document}
%    \end{macrocode}

% Declare a title page.
% Print title, part of document being processed and version flag:
%    \begin{macrocode}
\addtocounter{page}{-1}
\begin{center}
{\LARGE\bfseries{}childdoc example\par}
\vspace{1cm}
\ifchilddoc
\ifchilddocmanual part\else chapter\fi:
`\childdocname' of `\childdocjob'\par
\else
main document: `\childdocjob'\par
\fi
version: \version\par
\end{center}
\newpage
%    \end{macrocode}

% Manually include selected file,
% otherwise process as usual:
%    \begin{macrocode}
\ifchilddocmanual
\section*{part `\childdocname'}
\input{\childdocname}
\else
%    \end{macrocode}

% Include the two chapters:
%    \begin{macrocode}
\include{cdocsch1}
\include{cdocsch2}
%    \end{macrocode}

% Include the two parts unless only chapters should be displayed:
%    \begin{macrocode}
\ifchilddoc\else
\section{part three}
\input{cdocspt3}
\section{part four}
\input{cdocspt4}
\fi
%    \end{macrocode}

% Process as usual until here:
%    \begin{macrocode}
\fi
%    \end{macrocode}

% End of document body:
%    \begin{macrocode}
\end{document}
%    \end{macrocode}
%\iffalse
%</samplemain>
%\fi
%
% %%%%%%%%%%%%%%%%%%%%%%%%%%%%%%%%%%%%%%
% \paragraph{Chapter Include Files.}
%
% The include files are called |cdocsch1.tex| and |cdocsch2.tex|.
%
%\iffalse
%<*samplechap1|samplechap2>
%\fi

% Optional override for |\version| flag:
%    \begin{macrocode}
%%\providecommand{\version}{final}
%    \end{macrocode}

% Include the main document:
%    \begin{macrocode}
\input{childdoc.def}
\childdocof{cdocsamp}
%    \end{macrocode}

%\iffalse
%</samplechap1|samplechap2>
%\fi
%
%\iffalse
%<*samplechap1>
%\fi
% Some text for chapter 1:
%    \begin{macrocode}
\section{one}
some text in chapter one
%    \end{macrocode}

%\iffalse
%</samplechap1>
%\fi
% Some text for chapter 2:
%\iffalse
%<*samplechap2>
%\fi
%    \begin{macrocode}
\section{two}
more text in chapter two
%    \end{macrocode}

%\iffalse
%</samplechap2>
%\fi
%
% %%%%%%%%%%%%%%%%%%%%%%%%%%%%%%%%%%%%%%
% \paragraph{Part Include Files.}
%
% The include files are called |cdocspt3.tex| and |cdocspt4.tex|.
%
%\iffalse
%<*samplepart3|samplepart4>
%\fi

% Optional override for |\version| flag:
%    \begin{macrocode}
%%\providecommand{\version}{final}
%    \end{macrocode}

% Include the main document:
%    \begin{macrocode}
\input{childdoc.def}
\childdocby{cdocsamp}
%    \end{macrocode}

%\iffalse
%</samplepart3|samplepart4>
%\fi
%
%\iffalse
%<*samplepart3>
%\fi
% Some text for part 3:
%    \begin{macrocode}
some text in part three
%    \end{macrocode}

%\iffalse
%</samplepart3>
%\fi
% Some text for part 4:
%\iffalse
%<*samplepart4>
%\fi
%    \begin{macrocode}
more text in part four
%    \end{macrocode}

%\iffalse
%</samplepart4>
%\fi
%
% %%%%%%%%%%%%%%%%%%%%%%%%%%%%%%%%%%%%%%
% \paragraph{Forwarding for a Complete Draft.}
%
% The following forwarding file |cdocsdrf.tex|
% compiles the main document in draft mode:
%\iffalse
%<*sampledraft>
%\fi
%    \begin{macrocode}
\def\version{draft}
\input{childdoc.def}
\childdocforward{cdocsamp}
%    \end{macrocode}

%\iffalse
%</sampledraft>
%\fi
%
% %%%%%%%%%%%%%%%%%%%%%%%%%%%%%%%%%%%%%%
% \paragraph{Forwarding for Final Version of the Chapters.}
%
% The following forwarding files |cdocsfn1.tex| and |cdocsfn2.tex|
% (with identical content)
% compile the final versions of the child documents
% |cdocsch1.tex| and |cdocsch2.tex|, respectively:
%\iffalse
%<*samplefinal>
%\fi
%    \begin{macrocode}
\def\version{final}
\input{childdoc.def}
\childdocforwardprefix[cdocsamp]{cdocsfn}{cdocsch}
%    \end{macrocode}

%\iffalse
%</samplefinal>
%\fi
%
% %%%%%%%%%%%%%%%%%%%%%%%%%%%%%%%%%%%%%%
% \paragraph{Command Line Processing.}
%
% The following three command lines generate the output files
% |cdocscld|, |cdocscl1| and |cdocscl2|
% which should be identical to
% |cdocsdrf|, |cdocsch1| and |cdocsfn2|, respectively:
% \begin{center}
% \begin{tabular}{l}
% |latex -jobname cdocscld \|\\
% |  "\def\version{draft}\input{childdoc.def}\childdocforward{cdocsamp}"|\\
% |latex -jobname cdocscl1 \|\\
% |  "\input{childdoc.def}\childdocforward[cdocsamp]{cdocsch1}"|\\
% |latex -jobname cdocscl2 \|\\
% |  "\def\version{final}\input{childdoc.def}\childdocforward{cdocsch2}"|
% \end{tabular}
% \end{center}
% Note that the trailing backslash on each first line
% merely continues the input to the second line
% (for convenient cut ant paste).
% Furthermore, the command |latex| can be replaced by any
% of its alternative versions such as |pdflatex|.
%
% %%%%%%%%%%%%%%%%%%%%%%%%%%%%%%%%%%%%%%%%%%%%%%%%%%%%%%%%%%%%%%%%%%%%%%%%%%%%%%
% %%%%%%%%%%%%%%%%%%%%%%%%%%%%%%%%%%%%%%%%%%%%%%%%%%%%%%%%%%%%%%%%%%%%%%%%%%%%%%
% \section{Implementation}
%\iffalse
%<*package>
%\fi
%
% This section describes the definitions file |childdoc.def|.

% The definitions cannot be loaded using |\usepackage| or |\RequirePackage|
% which has a mechanism to prevent loading a style file more than once.
% When loading the definitions by means of |\input|
% multiple instances have to be prevented manually:
%\iffalse
%This code needs to be before the `\ProvidesFile' directive
%which is defined at the beginning of this file.
%Therefore it is also placed there and commented out here.
%</package>
%<*discard>
%\fi
%    \begin{macrocode}
\ifdefined\childdocmain\endinput\fi
%    \end{macrocode}
%\iffalse
%</discard>
%<*package>
%\fi
%
% \macro{\ifchilddoc}
% \macro{\ifchilddocmanual}
% The conditional |\ifchilddoc| tells whether a
% child (true) or main (false) document is being compiled.
% The conditional |\ifchilddocmanual| tells whether
% the |\includeonly| mechanism is used (false) or
% the selection of child files must be performed manually (true).
% The definitions initialise to false:
%    \begin{macrocode}
\newif\ifchilddoc
\newif\ifchilddocmanual
%    \end{macrocode}

% \macro{\childdocname}
% \macro{\childdocjob}
% The macro |\childdocname| stores the name of the main document
% to be compiled. The macro |\childdocjob| stores the name of
% the document on which the \LaTeX{} compiler was originally invoked.
% The content of |\jobname| cannot be compared
% to filenames specified in the source due to different catcodes.
% The following code rescans |\jobname|, stores the result
% in |\childdocname| and saves a copy in |\childdocjob|:
%    \begin{macrocode}
\edef\childdocname{\scantokens\expandafter{\jobname\noexpand}}
\let\childdocjob\childdocname
%    \end{macrocode}

% \macro{\childdocdisable}
% The macro |\childdocdisable| prevents the main file
% from being processed more than once.
% At this stage, the main document command |\childdocmain|
% is assumed to be called once again where it should do nothing.
% Any subsequent call to it should prevent
% a secondary processing of the main document
% It overwrites the forwarding commands
% |\childdocof| and |\childdocforward|
% with empty macros to prevent further inclusions of the main document:
%    \begin{macrocode}
\newcommand{\childdocdisable}
{
  \renewcommand{\childdocmain}[1]{\renewcommand{\childdocmain}[1]{\endinput}}
  \renewcommand{\childdocof}[1]{}
  \renewcommand{\childdocby}[2][]{}
  \renewcommand{\childdocforward}[2][]{}
  \renewcommand{\childdocdisable}{}
}
%    \end{macrocode}

% \macro{\childdocmain}
% The macro |\childdocmain| is to be called at the top of the main file
% with nothing or the main filename (without extension) as argument.
% First, it breaks loops.
% If the argument is not empty and does not match |\childdocname|
% (which is set by the first inclusion of |childdoc.def|),
% |\ifchilddoc| is set to true, |\includeonly| is applied to the child file
% and |\jobname| is set to the main file
% (for proper handling of |.aux| files):
%    \begin{macrocode}
\newcommand{\childdocmain}[1]
{
  \childdocdisable\childdocmain{}
  \if?#1?\else
    \begingroup
      \def\childdoctmp{#1}
      \ifx\childdoctmp\childdocname
        \def\childdoctmp{}
      \else
        \def\childdoctmp
        {
          \childdoctrue
          \includeonly{\childdocname}
          \def\childdocjob{#1}
          \def\jobname{#1}
        }
      \fi
      \expandafter
    \endgroup
    \childdoctmp
  \fi
}
%    \end{macrocode}

% \macro{\childdocof}
% The command |\childdocof| redirects
% compilation to the main file |#1|.
%    \begin{macrocode}
\newcommand{\childdocof}[1]
{
  \childdocdisable
  \childdoctrue
  \includeonly{\childdocname}
  \def\jobname{#1}
  \def\childdocjob{#1}
  \input{#1}
}
%    \end{macrocode}

% \macro{\childdocby}
% The command |\childdocby| ....
%    \begin{macrocode}
\newcommand{\childdocby}[2][]
{
  \childdocdisable
  \childdoctrue
  \childdocmanualtrue
  \if?#1?\else
    \def\jobname{#2}
  \fi
  \def\childdocjob{#2}
  \input{#2}
  \endinput
}
%    \end{macrocode}

% \macro{\childdocforward}
% The command |\childdocforward| redirects
% compilation to the main file or
% (if the optional argument is given) a child file.
% Parameters are set as if the main file
% or a child file starting with |\childdocof| was compiled.
% Then compilation is handed over to the main file:
%    \begin{macrocode}
\newcommand{\childdocforward}[2][]
{
  \begingroup
    \if?#1?
      \def\childdoctmp
      {
        \def\childdocname{#2}
        \def\childdocjob{#2}
        \def\jobname{#2}
        \input{#2}
        \endinput
      }
    \else
      \def\childdoctmp
      {
        \childdocdisable
        \def\childdocname{#2}
        \childdoctrue
        \includeonly{#2}
        \def\childdocjob{#1}
        \def\jobname{#1}
        \input{#1}
        \endinput
      }
    \fi
    \expandafter
  \endgroup
  \childdoctmp
}
%    \end{macrocode}

% \macro{\childdocforwardprefix}
% The command |\childdocforwardprefix| redirects
% compilation to the main or a child file by means of a pattern.
% The prefix |#1| in the current filename is replaced by |#2|
% and the suffix of the current filename is kept
% (it is assumed that the filename does not contain the substring `|~~~|'
% which is used as a delimiter).
% Compilation is handed over to the new file by |\childdocforward|:
%    \begin{macrocode}
\newcommand{\childdocforwardprefix}[3][]
{
  \begingroup
    \def\childdocextract #2##1~~~{\def\childdoctmp{\childdocforward[#1]{#3##1}}}
    \expandafter\childdocextract\childdocname~~~
    \expandafter
  \endgroup
  \childdoctmp
}
%    \end{macrocode}

% \macro{\childdoc}
% The deprecated macro |\childdoc| is a legacy version of |\childdocmain|:
%    \begin{macrocode}
\newcommand{\childdoc}{\childdocmain}
%    \end{macrocode}

% \macro{\childdocredirect}
% The deprecated macro |\childdocredirect| is a legacy version
% of |\childdocforward| and |\childdocforwardprefix|:
%    \begin{macrocode}
\newcommand{\childdocredirect}[2][]
{
  \begingroup
    \if?#1?
      \def\childdoctmp{\childdocforward{#2}}
    \else
      \def\childdoctmp{\childdocforwardprefix{#1}{#2}}
    \fi
    \expandafter
  \endgroup
  \childdoctmp
}
%    \end{macrocode}

%\iffalse
%</package>
%\fi
%
\endinput

\childdocforward{cdocsamp}
%    \end{macrocode}

%\iffalse
%</sampledraft>
%\fi
%
% %%%%%%%%%%%%%%%%%%%%%%%%%%%%%%%%%%%%%%
% \paragraph{Forwarding for Final Version of the Chapters.}
%
% The following forwarding files |cdocsfn1.tex| and |cdocsfn2.tex|
% (with identical content)
% compile the final versions of the child documents
% |cdocsch1.tex| and |cdocsch2.tex|, respectively:
%\iffalse
%<*samplefinal>
%\fi
%    \begin{macrocode}
\def\version{final}
% \iffalse
%
% childdoc.dtx Copyright (C) 2017-2018 Niklas Beisert
%
% This work may be distributed and/or modified under the
% conditions of the LaTeX Project Public License, either version 1.3
% of this license or (at your option) any later version.
% The latest version of this license is in
%   http://www.latex-project.org/lppl.txt
% and version 1.3 or later is part of all distributions of LaTeX
% version 2005/12/01 or later.
%
% This work has the LPPL maintenance status `maintained'.
%
% The Current Maintainer of this work is Niklas Beisert.
%
% This work consists of the files childdoc.dtx and childdoc.ins
% and the derived files childdoc.def and cdocsamp.tex with
% cdocsch1.tex, cdocsch2.tex, cdocsdrf.tex, cdocsfn1.tex, cdocsfn2.tex.
%
%<package>\ifdefined\childdocmain\endinput\fi
%<package>\ProvidesFile{childdoc.def}[2018/12/30 v2.0 child document driver]
%<samplemain>\ProvidesFile{cdocsamp.tex}[2018/12/30 v2.0 sample for childdoc]
%<*driver>
%\ProvidesFile{childdoc.drv}[2018/12/30 v2.0 childdoc reference manual file]
\PassOptionsToClass{10pt,a4paper}{article}
\documentclass{ltxdoc}

\usepackage[margin=35mm]{geometry}
\usepackage{hyperref}
\usepackage{hyperxmp}
\usepackage[usenames]{color}

\hypersetup{colorlinks=true}
\hypersetup{pdfstartview=FitH}
\hypersetup{pdfpagemode=UseNone}
\hypersetup{pdfsource={}}
\hypersetup{pdflang={en-UK}}
\hypersetup{pdfcopyright={Copyright 2017-2018 Niklas Beisert.
  This work may be distributed and/or modified under the
  conditions of the LaTeX Project Public License, either version 1.3
  of this license or (at your option) any later version.}}
\hypersetup{pdflicenseurl={http://www.latex-project.org/lppl.txt}}
\hypersetup{pdfcontactaddress={ETH Zurich, ITP, HIT K,
  Wolfgang-Pauli-Strasse 27}}
\hypersetup{pdfcontactpostcode={8093}}
\hypersetup{pdfcontactcity={Zurich}}
\hypersetup{pdfcontactcountry={Switzerland}}
\hypersetup{pdfcontactemail={nbeisert@itp.phys.ethz.ch}}
\hypersetup{pdfcontacturl={http://people.phys.ethz.ch/\xmptilde nbeisert/}}

\newcommand{\secref}[1]{\hyperref[#1]{section \ref*{#1}}}

\parskip1ex
\parindent0pt
\let\olditemize\itemize
\def\itemize{\olditemize\parskip0pt}

\begin{document}

\title{The \textsf{childdoc} Package}
\hypersetup{pdftitle={The childdoc Package}}
\author{Niklas Beisert\\[2ex]
  Institut f\"ur Theoretische Physik\\
  Eidgen\"ossische Technische Hochschule Z\"urich\\
  Wolfgang-Pauli-Strasse 27, 8093 Z\"urich, Switzerland\\[1ex]
  \href{mailto:nbeisert@itp.phys.ethz.ch}
  {\texttt{nbeisert@itp.phys.ethz.ch}}}
\hypersetup{pdfauthor={Niklas Beisert}}
\hypersetup{pdfsubject={Manual for the LaTeX2e Package childdoc}}
\date{30 December 2018, \textsf{v2.0}}
\maketitle

\begin{abstract}\noindent
\textsf{childdoc} is a \LaTeXe{} package
that enables the direct compilation
of document sections included by |\include|
to individual files.
\end{abstract}

\begingroup
\parskip0ex
\tableofcontents
\endgroup

%%%%%%%%%%%%%%%%%%%%%%%%%%%%%%%%%%%%%%%%%%%%%%%%%%%%%%%%%%%%%%%%%%%%%%%%%%%%%%%%
%%%%%%%%%%%%%%%%%%%%%%%%%%%%%%%%%%%%%%%%%%%%%%%%%%%%%%%%%%%%%%%%%%%%%%%%%%%%%%%%
\section{Introduction}

\LaTeX{} provides a mechanism to structure a large document (such as a book)
into a main file and several child files (containing the chapters)
using the |\include| command.
This mechanism is beneficial for documents
which span hundreds of pages in order to
make the source file(s) more manageable.
Moreover, compilation can be restricted to
selected child files by means of the |\includeonly| command.
The latter feature can be used to reduce the compilation time while editing
(this was significantly more useful in the earlier days of \LaTeX{})
or to generate a smaller document which is easier to navigate.
Another application of |\includeonly| is to generate
documents consisting of selected parts of the complete document.

However, there are a few drawbacks of the plain |\include| mechanism:
\begin{itemize}
\item
The child files cannot be compiled on their own,
they can only be compiled via the main file.
A naive editing environment
(such as a text editor with an option
to have the current file processed by \LaTeX)
may require one to switch to the main file before compiling;
attempting to compile the child file produces errors.
\item
The main file must be modified (each time)
to adjust the |\includeonly| command
to the present needs. This easily leaves the main file in a messy state.
\item
The generated document will always carry the filename
of the main document. This is inconvenient if
several child files are to be compiled and
to be kept for distribution.
\end{itemize}

The present package provides a simple interface
to make child files individually compilable by \LaTeX{}.
Compiling a child file then has the same effect as compiling
the main file with an |\includeonly| command
to select the appropriate child.
Moreover the generated document will carry the name of the child
rather than the main file.
This resolves all three above issues.

This feature is meant to make the editing of books,
thesis documents and lecture notes somewhat more convenient.
However, the package can also be used efficiently for
composing a series of documents (such as exercise sheets)
which are typically distributed individually.
It then assists the author in generating the individual documents
(potentially in different versions)
as well as a document containing the collected series.
Another application is in developing style files
or other kinds of included material
where compilation of the style file could redirect
to a sample or test file.

%%%%%%%%%%%%%%%%%%%%%%%%%%%%%%%%%%%%%%%%%%%%%%%%%%%%%%%%%%%%%%%%%%%%%%%%%%%%%%%%
%%%%%%%%%%%%%%%%%%%%%%%%%%%%%%%%%%%%%%%%%%%%%%%%%%%%%%%%%%%%%%%%%%%%%%%%%%%%%%%%
\section{Usage}

First of all, the package \textsf{childdoc} is \emph{not} a standard
\LaTeXe{} |.sty| style file! Therefore it needs to be invoked in
a non-standard way.

%%%%%%%%%%%%%%%%%%%%%%%%%%%%%%%%%%%%%%%%%%%%%%%%%%%%%%%%%%%%%%%%%%%%%%%%%%%%%%%%
\subsection{Included Files}
\label{sec:include}

%%%%%%%%%%%%%%%%%%%%%%%%%%%%%%%%%%%%%%%%
\DescribeMacro{\childdocmain}
To use the package, add the commands
\begin{center}
\begin{tabular}{l}
|\input{childdoc.def}|\\
|\childdocmain{}|\\
\end{tabular}
\end{center}
at the very top of the main \LaTeX{} file,
in particular \emph{before} the |\documentclass| statement!
The argument of |\childdocmain| should be left empty
(but it must be present).

%%%%%%%%%%%%%%%%%%%%%%%%%%%%%%%%%%%%%%%%
\DescribeMacro{\childdocof}
Furthermore, add the commands
\begin{center}
\begin{tabular}{l}
|\input{childdoc.def}|\\
|\childdocof{|\textit{main}|}|\\
\end{tabular}
\end{center}
at the top of every child file \textit{child}
which is included by |\include{|\textit{child}|}|
from within the main file
(or at least for those files to be compiled individually).
The argument \textit{main} must be the filename of the main file.

There are a couple of
considerations in setting up the main and child documents:

%%%%%%%%%%%%%%%%%%%%%%%%%%%%%%%%%%%%%%%%
\paragraph{Restrictions.}

Please note the following restrictions:
\begin{itemize}
\item
|\childdocmain| must be called with one argument \textit{main}
to ensure compatibility with earlier version of the package.
It must either be empty (|\childdocmain{}|)
or precisely match the filename of the main file in which it is specified.
See \secref{sec:detection} for further information.
\item
The filename \textit{main} must be specified without the |.tex| extension.
\item
The filename \textit{main} is case sensitive
(even in case-insensitive file systems)
due to internal string comparison.
\item
The argument \textit{main} should be fully expanded, it cannot be a macro.
\item
Subdirectories and special characters should be avoided in filenames.
\item
The command |\childdocmain{|\textit{main}|}| must be followed by a whitespace.
It should not be followed immediately by another command
or by a comment mark `|%|'.
This is because the \TeX{} parser reads the token immediately following
the argument of |\childdocmain| and puts it
at the beginning of every child section;
however, a white\-space is ignored.
\end{itemize}

%%%%%%%%%%%%%%%%%%%%%%%%%%%%%%%%%%%%%%%%
\paragraph{Content of Main File.}

It is advisable to place all content in the child files included by |\include|.
Any output contained in the main file will appear in all child documents
unless suppressed manually;
it cannot be suppressed automatically by the |\includeonly| directive
and thus should normally be avoided.
A method to include some content in the main file
by means of conditional processing is described in \secref{sec:conditional}.

%%%%%%%%%%%%%%%%%%%%%%%%%%%%%%%%%%%%%%%%
\paragraph{Page Numbering.}

When only a part of the document is compiled,
the appropriate numbering of pages
(as well as other status parameters)
is determined from the |.aux| files.
The latter contain information from previous passes.
However this information needs to propagate through
all intermediate child documents.
Therefore the page numbering in child documents may well
be inconsistent until the complete document is compiled at least once.

A useful (if unconventional) way to always ensure a consistent
page numbering is to restart the numbering in each child document
and denote the pages by `\textit{child}|.|\textit{page}'
where \textit{child} represents the chapter/section number of the child file.
This can be achieved by the command
|\numberwithin{page}{|\textit{child}|}|
of the \textsf{amsmath} package
where \textit{child} can be |chapter| or |section|
depending on the chosen structuring.
Alternatively, one can modify the macro |\thepage| appropriately
and reset the counter |page| at the start of each child file.

%%%%%%%%%%%%%%%%%%%%%%%%%%%%%%%%%%%%%%%%%%%%%%%%%%%%%%%%%%%%%%%%%%%%%%%%%%%%%%%%
\subsection{Conditional Processing}
\label{sec:conditional}

The package provides a mechanism to compile different versions
of a document. To customise the versions further some conditional processing
can come in handy to distinguish which version is being compiled.
The package provides two macros to describe the compilation context:

%%%%%%%%%%%%%%%%%%%%%%%%%%%%%%%%%%%%%%%%
\DescribeMacro{\ifchilddoc}
The conditional |\ifchilddoc| distinguishes between the compilation of
child documents and the main document:
%
\begin{center}
|\ifchilddoc |\textit{child-code}| |[|\||else |\textit{main-code}]| \||fi|
\end{center}

%%%%%%%%%%%%%%%%%%%%%%%%%%%%%%%%%%%%%%%%
\DescribeMacro{\childdocname}
\DescribeMacro{\childdocjob}
The macro |\childdocname| contains the filename (without extension)
of the main or child file being processed.
Note that |\childdocjob| will always contain the name of the main file.

%%%%%%%%%%%%%%%%%%%%%%%%%%%%%%%%%%%%%%%%
\paragraph{Title Page.}

Conditional processing can be used to include a title or banner page
in the main document when proper precautions are taken.
Importantly, the code in the main file should ensure that the page counter
(as well as other status parameters which are stored in the |.aux| files)
takes the same value after the conditional processing.
Otherwise the page numbers may take divergent values
depending on which part is compiled.

For example, a title page could be declared by:
%
\begin{center}
\begin{tabular}{l}
|\ifchilddoc\||else|\\
|\addtocounter{page}{-1}|\\
\textit{code for title page}\\
|\newpage|\\
|\||fi|
\end{tabular}
\end{center}
%
A banner page for the child documents can be generated by:
%
\begin{center}
\begin{tabular}{l}
|\ifchilddoc|\\
|\addtocounter{page}{-1}|\\
\textit{code for banner page}\\
|\newpage|\\
|\||fi|
\end{tabular}
\end{center}
%
Here one could write a message such as:
\begin{center}
|This is the part \childdocname{} of \childdocjob{}.|
\end{center}

%%%%%%%%%%%%%%%%%%%%%%%%%%%%%%%%%%%%%%%%%%%%%%%%%%%%%%%%%%%%%%%%%%%%%%%%%%%%%%%%
\subsection{Flags}
\label{sec:flags}

The package makes it easy to generate different versions
of the main or child documents.
To this end compilation flags can be defined
and assigned different default values.
They will be particularly useful in conjunction
with the forwarding mechanism described in \secref{sec:forward}.

For example, it may be useful to have a flag |\version|
which can be set to |draft| or |final|.
The document source will contain some conditional code
depending on the value of |\version|.
Suppose further, the flag should default to |final| for the main file
and to |draft| for child files
which is a natural assignment for editing the document.
This is achieved by placing the following code
in the preamble of the main document
(below the |\childdocmain| directive):
%
\begin{center}
\begin{tabular}{l}
|\ifchilddoc|\\
|\providecommand{\version}{draft}|\\
|\||else|\\
|\providecommand{\version}{final}|\\
|\||fi|
\end{tabular}
\end{center}
%
The definition by |\providecommand| makes sure
that previous definitions are not overwritten.
Further statements |\providecommand{\version}{...}|
can thus be added before the above code to override it.

For the main file, one might add a line
(between |\childdocmain| and the above block)
%
\begin{center}
|%\ifchilddoc\||else\providecommand{\version}{draft}\||fi|
\end{center}
%
which can be uncommented to produce a draft version.
Likewise one can add a line to the very top of a child file
(above the |\childdocof{|\textit{main}|}| directive)
%
\begin{center}
|%\providecommand{\version}{final}|
\end{center}
%
which can be uncommented to produce the final version of this child document.

%%%%%%%%%%%%%%%%%%%%%%%%%%%%%%%%%%%%%%%%%%%%%%%%%%%%%%%%%%%%%%%%%%%%%%%%%%%%%%%%
\subsection{Forwarding}
\label{sec:forward}

Different versions of the main or child documents
using compilation flags as described in \secref{sec:flags}
can be (permanently) stored in different files
for convenient compilation, viewing and distribution.
To this end, the package defines a command
to pass on compilation to a different file:

%%%%%%%%%%%%%%%%%%%%%%%%%%%%%%%%%%%%%%%%
\DescribeMacro{\childdocforward}
The command |\childdocforward| redirects processing to
another source file:
%
\begin{center}
\begin{tabular}{l}
|\input{childdoc.def}|\\
|\childdocforward[|\textit{main}|]{|\textit{dest}|}|\\
\end{tabular}
\end{center}
%
The argument \textit{dest} is the destination file
(without extension).
It should be the main file or one of the child files.
Note that further \textsf{childdoc} directives
such as |\childdocof| and |\childdocforward|
in the indicated file will be processed in this form.
The optional argument \textit{main}
passes on directly to the main file \textit{main}
while pretending to compile the child \textit{dest}.
This form behaves as if \textit{dest}
issues |\childdocof{|\textit{main}|}| right away,
and no further \textsf{childdoc} directives will be processed.

%%%%%%%%%%%%%%%%%%%%%%%%%%%%%%%%%%%%%%%%
\DescribeMacro{\...prefix}
In the alternative form |\childdocforwardprefix|,
%
\begin{center}
\begin{tabular}{l}
|\input{childdoc.def}|\\
|\childdocforwardprefix[|\textit{main}|]{|\textit{prefix}|}{|\textit{dest}|}|
\end{tabular}
\end{center}
%
the destination file is determined by a pattern
depending on the current file:
To make this work, the current file must be called
`{\textit{prefix}\hspace{0.2em}\textit{suffix}}'
with \textit{prefix} matching precisely the argument.
Processing is then passed on to the file
`{\textit{dest}\hspace{0.2em}\textit{suffix}}'.
Surely, the same effect is achieved by
directly specifying the
argument `{\textit{dest}\hspace{0.2em}\textit{suffix}}'
in the first form.
However, that requires to set up a different file
for each child. With the alternative form of the command
all these files can have exactly the same content
which simplifies setting them up and maintaining them.

For example, the following file |draft.tex|
with a compilation flag |\version| as described in \secref{sec:flags}
compiles the main document as a draft:
%
\begin{center}
\begin{tabular}{l}
|\def\version{draft}|\\
|\input{childdoc.def}|\\
|\childdocforward{|\textit{main}|}|
\end{tabular}
\end{center}
%
Likewise, the following files |final|\textit{nn}|.tex|
compile the final version of the child document
|child|\textit{nn}|.tex|:
%
\begin{center}
\begin{tabular}{l}
|\def\version{final}|\\
|\input{childdoc.def}|\\
|\childdocforwardprefix{final}{child}|
\end{tabular}
\end{center}
%

Note that when several versions of a main file and/or of each child file
are to be generated, it may be convenient to set up a |Makefile| or
shell script to automatise the process.

%%%%%%%%%%%%%%%%%%%%%%%%%%%%%%%%%%%%%%%%%%%%%%%%%%%%%%%%%%%%%%%%%%%%%%%%%%%%%%%%
\subsection{Command Line Processing}
\label{sec:commandline}

The effect of redirection files can also be achieved by invoking
the \LaTeX{} compiler with a more elaborate command line.
Most conveniently this should be done as part
of a shell script or a |Makefile|.

When using \textsf{childdoc} in the main file, the following
command lines effectively perform a redirection
(note that depending on the shell being used,
backslashes may have to be doubled: `|\|' $\to$ `|\\|'):
%
\begin{center}
|... -jobname "|\textit{target}|" |\\|"|[\textit{flags}]%
|\input{childdoc.def}\childdocforward[|\textit{main}|]{|\textit{dest}|}"|
\end{center}
%
Here \textit{target} is the name of the output file,
\textit{main} is the name of the main file
and \textit{dest} is the name of the main or child file to be processed
(all filenames without extensions).
The optional argument \textit{main} can be omitted
if \textit{main} matches \textit{dest}.
Optionally, compilation \textit{flags} can be defined via |\def| commands.
This command line makes the \TeX{} engine believe
it is compiling the file \textit{target}
whose content is specified as the latter parameter.
The provided code then forwards the processing to
\textit{main} or \textit{dest} as described in \secref{sec:forward}.

%%%%%%%%%%%%%%%%%%%%%%%%%%%%%%%%%%%%%%%%%%%%%%%%%%%%%%%%%%%%%%%%%%%%%%%%%%%%%%%%
\subsection{Include by Input}
\label{sec:input}

Including child documents by |\include| has some restrictions by design.
Most notably, the content of a child document always occupies
its own set of pages; pages cannot be shared between child documents.
Usually, this behaviour makes perfect sense
because each child document contain an essential part of the document.
However, in some situations it may be desirable to compose
a document from a collection of parts
without having mandatory page breaks between then.
For this case, the package
provides a mechanism to include parts
by |\input| which can also be processed individually.
However, by construction this mechanism
requires manual handling of the content to be output.

%%%%%%%%%%%%%%%%%%%%%%%%%%%%%%%%%%%%%%%%
\DescribeMacro{\ifchilddocmanual}
The main file should be prepared as usual, see \secref{sec:include}.
However, the document body must make a distinction
between processing of an individual part and of the main document, e.g.:
%
\begin{center}
\begin{tabular}{l}
|\ifchilddocmanual|\\
|\input{\childdocname}|\\
|\||else|\\
\textit{document body with }|\input{|\textit{part}|}|\\
|\||fi|
\end{tabular}
\end{center}
%
The conditional |\ifchilddocmanual| is true whenever
a part to be included by |\input| is being compiled,
and the name of the part is stored in |\childdocname|.

%%%%%%%%%%%%%%%%%%%%%%%%%%%%%%%%%%%%%%%%
\DescribeMacro{\childdocby}
Each part to be included by |\input| should start with:
%
\begin{center}
\begin{tabular}{l}
|\input{childdoc.def}|\\
|\childdocby{|\textit{main}|}|\\
\end{tabular}
\end{center}
%
The directive |\childdocby| is similar to |\childdocof|
described in \secref{sec:include},
but the subsequent selection of content must be done manually.
To that end, both |\ifchilddoc| and |\ifchilddocmanual|
will be true upon processing of a part,
and the name of the part is stored in |\childdocname|.
Note that |\jobname| will be set to the filename of the current part
so that each part receives an individual |.aux| file
that does not interfere with the |.aux| file(s) of the main document.
This behaviour can be altered by the alternative form
|\childdocby[*]{|\textit{main}|}| (with a non-empty optional argument)
which uses the |.aux| file of the main document
by setting |\jobname| to \textit{main}.

%%%%%%%%%%%%%%%%%%%%%%%%%%%%%%%%%%%%%%%%%%%%%%%%%%%%%%%%%%%%%%%%%%%%%%%%%%%%%%%%
\subsection{Driver Development}
\label{sec:driver}

The \textsf{childdoc} mechanism can also be use for the development
of definition files such as \LaTeX{} styles or classes.
This case differs from the above setup with multiple parts
included by |\include| in that no |\includeonly| should be invoked.
This can be achieved by starting the include file
(before |\ProvidesPackage|) with:
%
\begin{center}
\begin{tabular}{l}
|\input{childdoc.def}|\\
|\childdocforward{|\textit{main}|}|\\
\end{tabular}
\end{center}
%
or alternatively with:
%
\begin{center}
\begin{tabular}{l}
|\input{childdoc.def}|\\
|\childdocby{|\textit{main}|}|\\
\end{tabular}
\end{center}
%
Both forms have slightly different effects as described above.
The main file is prepared as usual, see \secref{sec:include}.

%%%%%%%%%%%%%%%%%%%%%%%%%%%%%%%%%%%%%%%%%%%%%%%%%%%%%%%%%%%%%%%%%%%%%%%%%%%%%%%%
\subsection{Legacy Detection}
\label{sec:detection}

The directive |\childdocmain| in the main file can detect
whether the complete document or merely a child is to be compiled
even without using the directive |\childdocof|.
This method is deprecated because it is less robust
and there is no compelling reason to use it;
it is merely provided for backward compatibility
and it may be removed in future versions.

If the detection mechanism is to be used,
it is mandatory to correctly specify
the filename of the main file as the argument of |\childdocmain|:
%
\begin{center}
\begin{tabular}{l}
|\input{childdoc.def}|\\
|\childdocmain{|\textit{main}|}|\\
\end{tabular}
\end{center}
%
If |\jobname| does not match the argument \textit{main} of |\childdocmain|,
it is assumed that |\jobname| points to the child file to be compiled.
When using |\childdocmain| with the main file specified as argument,
it suffices to start a child file
with just |\input{|\textit{main}|}|
without loading of the package and using |\childdocof|.
If instead all processing is done
with the appropriate \textsf{childdoc} directives,
the argument of \textit{main} of |\childdocmain| can be empty.

An alternative version of the command line processing described
in \secref{sec:commandline} using the detection mechanism reads:
%
\begin{center}
|... -jobname "|\textit{target}|" "|[\textit{flags}]%
[|\def\jobname{|\textit{dest}|}|]|\input{|\textit{main}|}"|
\end{center}

%%%%%%%%%%%%%%%%%%%%%%%%%%%%%%%%%%%%%%%%%%%%%%%%%%%%%%%%%%%%%%%%%%%%%%%%%%%%%%%%
\subsection{Manual Code}
\label{sec:manual}

In case one cannot be certain whether the definitions file |childdoc.def|
is installed on the target \TeX{} distribution
and one prefers not to ship it,
it is conceivable to paste a few relevant commands into the sources.

To that end, drop all statements |\input{childdoc.def}|
and perform the replacements as outlined below.
Instead of |\childdocmain{|\textit{main}|}| add the following code
to the top of the main file:
%
\begin{center}
\begin{tabular}{l}
|\||ifdefined\childdocname\endinput\||fi\newif\ifchilddoc|\\
|\edef\childdocname{\scantokens\expandafter{\jobname\noexpand}}|\\
|\def\childdocmain{|\textit{main}|}\||ifx\childdocmain\childdocname\||else|\\
|\childdoctrue\includeonly{\childdocname}\let\jobname\childdocmain\||fi|\\
\end{tabular}
\end{center}
%
Instead of |\childdocof{|\textit{main}|}| just include the main file
at the top of each child file:
%
\begin{center}
|\input{|\textit{main}|}|
\end{center}
%
A simple redirection |\childdocforward{|\textit{dest}|}| is achieved by:
%
\begin{center}
|\def\jobname{|\textit{dest}|}\input{\jobname}|
\end{center}
%
The redirection with prefix
|\childdocforwardprefix[|\textit{prefix}|]{|\textit{dest}|}|
is accomplished by:
%
\begin{center}
\begin{tabular}{l}
|{\edef\jobname{\scantokens\expandafter{\jobname\noexpand}}|\\
|\def\redirectjob |\textit{prefix}|#1~~~{\gdef\jobname{|\textit{dest}|#1}}|\\
|\expandafter\redirectjob\jobname~~~}\input{\jobname}|
\end{tabular}
\end{center}

In an alternative approach,
child documents can be compiled by a specific command line
without additional code or specific definitions:
%
\begin{center}
|... -jobname "|\textit{target}|" "|[\textit{flags}]%
|\includeonly{|\textit{dest}|}\input{|\textit{main}|}"|
\end{center}
%

%%%%%%%%%%%%%%%%%%%%%%%%%%%%%%%%%%%%%%%%%%%%%%%%%%%%%%%%%%%%%%%%%%%%%%%%%%%%%%%%
%%%%%%%%%%%%%%%%%%%%%%%%%%%%%%%%%%%%%%%%%%%%%%%%%%%%%%%%%%%%%%%%%%%%%%%%%%%%%%%%
\section{Information}

%%%%%%%%%%%%%%%%%%%%%%%%%%%%%%%%%%%%%%%%%%%%%%%%%%%%%%%%%%%%%%%%%%%%%%%%%%%%%%%%
\subsection{Copyright}

Copyright \copyright{} 2017--2018 Niklas Beisert

This work may be distributed and/or modified under the
conditions of the \LaTeX{} Project Public License, either version 1.3
of this license or (at your option) any later version.
The latest version of this license is in
  \url{http://www.latex-project.org/lppl.txt}
and version 1.3 or later is part of all distributions of \LaTeX{}
version 2005/12/01 or later.

This work has the LPPL maintenance status `maintained'.

The Current Maintainer of this work is Niklas Beisert.

This work consists of the files |README.txt|, |childdoc.ins| and |childdoc.dtx|
as well as the derived files |childdoc.def|, |cdocsamp.tex|
with |cdocsch1.tex|, |cdocsch2.tex|, |cdocspt3.tex|, |cdocspt4.tex|,
|cdocsdrf.tex|, |cdocsfn1.tex|, |cdocsfn2.tex|
as well as |childdoc.pdf|.

%%%%%%%%%%%%%%%%%%%%%%%%%%%%%%%%%%%%%%%%%%%%%%%%%%%%%%%%%%%%%%%%%%%%%%%%%%%%%%%%
\subsection{Files and Installation}

The package consists of the files:
%
\begin{center}
\begin{tabular}{ll}
    |README.txt|   & readme file \\
    |childdoc.ins| & installation file \\
    |childdoc.dtx| & source file \\
    |childdoc.def| & definition file \\
    |cdocsamp.tex| & sample main file \\
    |cdocsch1.tex| & sample include file \\
    |cdocsch2.tex| & sample include file \\
    |cdocspt3.tex| & sample part file \\
    |cdocspt4.tex| & sample part file \\
    |cdocsdrf.tex| & sample redirection file \\
    |cdocsfn1.tex| & sample redirection file \\
    |cdocsfn2.tex| & sample redirection file \\
    |childdoc.pdf| & manual
\end{tabular}
\end{center}
%
The distribution consists of the files
|README.txt|, |childdoc.ins| and |childdoc.dtx|.
%
\begin{itemize}
\item
Run (pdf)\LaTeX{} on |childdoc.dtx|
to compile the manual |childdoc.pdf| (this file).
\item
Run \LaTeX{} on |childdoc.ins| to create the definitions file |childdoc.def|
and the sample |cdocsamp.tex| with include files
|cdocsch1.tex|, |cdocsch2.tex|, |cdocspt3.tex|, |cdocspt4.tex|,
|cdocsdrf.tex|, |cdocsfn1.tex|, |cdocsfn2.tex|.
Then copy the file |childdoc.def| to an appropriate directory of your \LaTeX{}
distribution, e.g.\ \textit{texmf-root}|/tex/latex/childdoc|.
\end{itemize}

%%%%%%%%%%%%%%%%%%%%%%%%%%%%%%%%%%%%%%%%%%%%%%%%%%%%%%%%%%%%%%%%%%%%%%%%%%%%%%%%
\subsection{Related CTAN Packages}

There are several other packages which offer a similar functionality:
%
\begin{itemize}
\item
The packages
\href{http://ctan.org/pkg/docmute}{\textsf{docmute}},
\href{http://ctan.org/pkg/includex}{\textsf{includex}} and
\href{http://ctan.org/pkg/standalone}{\textsf{standalone}}
provide commands to include only the document body of
a child file thus allowing both files to be compiled individually.
\item
The packages \href{http://ctan.org/pkg/subdocs}{\textsf{subdocs}}
and \href{http://ctan.org/pkg/subfiles}{\textsf{subfiles}}
provide structures in which the main and child documents can be
encapsulated and allowing them to be compiled individually.
The inclusion mechanism is different from the conventional |\include|.
\item
The package \href{http://ctan.org/pkg/combine}{\textsf{combine}}
is an elaborate solution to combine several documents into one.
\end{itemize}
%
See also the CTAN topic \href{http://ctan.org/topic/subdocs}{\textsf{subdocs}}
for further related packages.
The present package differs from the above solutions in that
a document structure constructed with the conventional |\include| mechanism
just needs two extra commands at the top of every file
such that all constituent files can be compiled individually.

%%%%%%%%%%%%%%%%%%%%%%%%%%%%%%%%%%%%%%%%%%%%%%%%%%%%%%%%%%%%%%%%%%%%%%%%%%%%%%%%
%\subsection{Feature Suggestions}
%
%The following is a list of features which may be useful for future
%versions of this package:
%%
%\begin{itemize}
%\item
%\ldots
%\end{itemize}

%%%%%%%%%%%%%%%%%%%%%%%%%%%%%%%%%%%%%%%%%%%%%%%%%%%%%%%%%%%%%%%%%%%%%%%%%%%%%%%%
\subsection{Revision History}

%%%%%%%%%%%%%%%%%%%%%%%%%%%%%%%%%%%%%%%%
\paragraph{v2.0:} 2018/12/30

\begin{itemize}
\item
immediate forward processing
\item
added |\childdocby| mechanism
\item
manual restructured
\end{itemize}

%%%%%%%%%%%%%%%%%%%%%%%%%%%%%%%%%%%%%%%%
\paragraph{v1.6:} 2018/01/17

\begin{itemize}
\item
application for development of include files
\item
corrections to manual
\end{itemize}

%%%%%%%%%%%%%%%%%%%%%%%%%%%%%%%%%%%%%%%%
\paragraph{v1.5:} 2017/05/21

\begin{itemize}
\item
more complete structuring introduced
\item
|\childdocof| introduced
\item
|\childdoc| renamed to |\childdocmain|
\item
|\childredirect| renamed to |\childdocforward| and |\childdocforwardprefix|
and functionality expanded
\end{itemize}

%%%%%%%%%%%%%%%%%%%%%%%%%%%%%%%%%%%%%%%%
\paragraph{v1.0:} 2017/04/27

\begin{itemize}
\item
manual and install package
\item
first version published on CTAN
\end{itemize}

%%%%%%%%%%%%%%%%%%%%%%%%%%%%%%%%%%%%%%%%
\paragraph{v0.6:} 2017/04/26

\begin{itemize}
\item
redirection mechanism added
\end{itemize}

%%%%%%%%%%%%%%%%%%%%%%%%%%%%%%%%%%%%%%%%
\paragraph{v0.5:} 2017/04/26

\begin{itemize}
\item
functionality in definition file
\end{itemize}


%%%%%%%%%%%%%%%%%%%%%%%%%%%%%%%%%%%%%%%%%%%%%%%%%%%%%%%%%%%%%%%%%%%%%%%%%%%%%%%%
%%%%%%%%%%%%%%%%%%%%%%%%%%%%%%%%%%%%%%%%%%%%%%%%%%%%%%%%%%%%%%%%%%%%%%%%%%%%%%%%
%%%%%%%%%%%%%%%%%%%%%%%%%%%%%%%%%%%%%%%%%%%%%%%%%%%%%%%%%%%%%%%%%%%%%%%%%%%%%%%%
\appendix

\settowidth\MacroIndent{\rmfamily\scriptsize 000\ }

 \DocInput{childdoc.dtx}

\end{document}
%</driver>
% \fi
%
% %%%%%%%%%%%%%%%%%%%%%%%%%%%%%%%%%%%%%%%%%%%%%%%%%%%%%%%%%%%%%%%%%%%%%%%%%%%%%%
% %%%%%%%%%%%%%%%%%%%%%%%%%%%%%%%%%%%%%%%%%%%%%%%%%%%%%%%%%%%%%%%%%%%%%%%%%%%%%%
% \section{Sample}
%\iffalse
%<*samplemain>
%\fi
%
% The following presents a sample document
% with two chapters, two parts, a title page,
% a compile flag as well as three forwarding files to set the flag.
% It consists of eight |.tex| files:
% \begin{center}
% \begin{tabular}{ll}
% |cdocsamp.tex|&main file\\
% |cdocsch1.tex|&include file for chapter 1\\
% |cdocsch2.tex|&include file for chapter 2\\
% |cdocspt3.tex|&include file for part 3\\
% |cdocspt4.tex|&include file for part 4\\
% |cdocsdrf.tex|&forwarding file for main file in draft mode\\
% |cdocsfi1.tex|&forwarding file for final version of chapter 1\\
% |cdocsfi2.tex|&forwarding file for final version of chapter 2\\
% \end{tabular}
% \end{center}
% Each of the eight files can be compiled directly by the \LaTeX{} compiler.
%
% %%%%%%%%%%%%%%%%%%%%%%%%%%%%%%%%%%%%%%
% \paragraph{Main File.}
%
% The main file is called |cdocsamp.tex|.
%
% Load the \textsf{childdoc} definitions and
% declare the filename for the main document:
%    \begin{macrocode}
\input{childdoc.def}
\childdocmain{}
%    \end{macrocode}

% Optional override for |\version| flag:
%    \begin{macrocode}
%%\ifchilddoc\else\providecommand{\version}{draft}\fi
%    \end{macrocode}

% Define the default values for the |\version| flag
% (|final| for the main file and |draft| for childs):
%    \begin{macrocode}
\ifchilddoc
\providecommand{\version}{draft}
\else
\providecommand{\version}{final}
\fi
%    \end{macrocode}

% Load the standard document class:
%    \begin{macrocode}
\documentclass[12pt]{article}
%    \end{macrocode}

% Start the document body:
%    \begin{macrocode}
\begin{document}
%    \end{macrocode}

% Declare a title page.
% Print title, part of document being processed and version flag:
%    \begin{macrocode}
\addtocounter{page}{-1}
\begin{center}
{\LARGE\bfseries{}childdoc example\par}
\vspace{1cm}
\ifchilddoc
\ifchilddocmanual part\else chapter\fi:
`\childdocname' of `\childdocjob'\par
\else
main document: `\childdocjob'\par
\fi
version: \version\par
\end{center}
\newpage
%    \end{macrocode}

% Manually include selected file,
% otherwise process as usual:
%    \begin{macrocode}
\ifchilddocmanual
\section*{part `\childdocname'}
\input{\childdocname}
\else
%    \end{macrocode}

% Include the two chapters:
%    \begin{macrocode}
\include{cdocsch1}
\include{cdocsch2}
%    \end{macrocode}

% Include the two parts unless only chapters should be displayed:
%    \begin{macrocode}
\ifchilddoc\else
\section{part three}
\input{cdocspt3}
\section{part four}
\input{cdocspt4}
\fi
%    \end{macrocode}

% Process as usual until here:
%    \begin{macrocode}
\fi
%    \end{macrocode}

% End of document body:
%    \begin{macrocode}
\end{document}
%    \end{macrocode}
%\iffalse
%</samplemain>
%\fi
%
% %%%%%%%%%%%%%%%%%%%%%%%%%%%%%%%%%%%%%%
% \paragraph{Chapter Include Files.}
%
% The include files are called |cdocsch1.tex| and |cdocsch2.tex|.
%
%\iffalse
%<*samplechap1|samplechap2>
%\fi

% Optional override for |\version| flag:
%    \begin{macrocode}
%%\providecommand{\version}{final}
%    \end{macrocode}

% Include the main document:
%    \begin{macrocode}
\input{childdoc.def}
\childdocof{cdocsamp}
%    \end{macrocode}

%\iffalse
%</samplechap1|samplechap2>
%\fi
%
%\iffalse
%<*samplechap1>
%\fi
% Some text for chapter 1:
%    \begin{macrocode}
\section{one}
some text in chapter one
%    \end{macrocode}

%\iffalse
%</samplechap1>
%\fi
% Some text for chapter 2:
%\iffalse
%<*samplechap2>
%\fi
%    \begin{macrocode}
\section{two}
more text in chapter two
%    \end{macrocode}

%\iffalse
%</samplechap2>
%\fi
%
% %%%%%%%%%%%%%%%%%%%%%%%%%%%%%%%%%%%%%%
% \paragraph{Part Include Files.}
%
% The include files are called |cdocspt3.tex| and |cdocspt4.tex|.
%
%\iffalse
%<*samplepart3|samplepart4>
%\fi

% Optional override for |\version| flag:
%    \begin{macrocode}
%%\providecommand{\version}{final}
%    \end{macrocode}

% Include the main document:
%    \begin{macrocode}
\input{childdoc.def}
\childdocby{cdocsamp}
%    \end{macrocode}

%\iffalse
%</samplepart3|samplepart4>
%\fi
%
%\iffalse
%<*samplepart3>
%\fi
% Some text for part 3:
%    \begin{macrocode}
some text in part three
%    \end{macrocode}

%\iffalse
%</samplepart3>
%\fi
% Some text for part 4:
%\iffalse
%<*samplepart4>
%\fi
%    \begin{macrocode}
more text in part four
%    \end{macrocode}

%\iffalse
%</samplepart4>
%\fi
%
% %%%%%%%%%%%%%%%%%%%%%%%%%%%%%%%%%%%%%%
% \paragraph{Forwarding for a Complete Draft.}
%
% The following forwarding file |cdocsdrf.tex|
% compiles the main document in draft mode:
%\iffalse
%<*sampledraft>
%\fi
%    \begin{macrocode}
\def\version{draft}
\input{childdoc.def}
\childdocforward{cdocsamp}
%    \end{macrocode}

%\iffalse
%</sampledraft>
%\fi
%
% %%%%%%%%%%%%%%%%%%%%%%%%%%%%%%%%%%%%%%
% \paragraph{Forwarding for Final Version of the Chapters.}
%
% The following forwarding files |cdocsfn1.tex| and |cdocsfn2.tex|
% (with identical content)
% compile the final versions of the child documents
% |cdocsch1.tex| and |cdocsch2.tex|, respectively:
%\iffalse
%<*samplefinal>
%\fi
%    \begin{macrocode}
\def\version{final}
\input{childdoc.def}
\childdocforwardprefix[cdocsamp]{cdocsfn}{cdocsch}
%    \end{macrocode}

%\iffalse
%</samplefinal>
%\fi
%
% %%%%%%%%%%%%%%%%%%%%%%%%%%%%%%%%%%%%%%
% \paragraph{Command Line Processing.}
%
% The following three command lines generate the output files
% |cdocscld|, |cdocscl1| and |cdocscl2|
% which should be identical to
% |cdocsdrf|, |cdocsch1| and |cdocsfn2|, respectively:
% \begin{center}
% \begin{tabular}{l}
% |latex -jobname cdocscld \|\\
% |  "\def\version{draft}\input{childdoc.def}\childdocforward{cdocsamp}"|\\
% |latex -jobname cdocscl1 \|\\
% |  "\input{childdoc.def}\childdocforward[cdocsamp]{cdocsch1}"|\\
% |latex -jobname cdocscl2 \|\\
% |  "\def\version{final}\input{childdoc.def}\childdocforward{cdocsch2}"|
% \end{tabular}
% \end{center}
% Note that the trailing backslash on each first line
% merely continues the input to the second line
% (for convenient cut ant paste).
% Furthermore, the command |latex| can be replaced by any
% of its alternative versions such as |pdflatex|.
%
% %%%%%%%%%%%%%%%%%%%%%%%%%%%%%%%%%%%%%%%%%%%%%%%%%%%%%%%%%%%%%%%%%%%%%%%%%%%%%%
% %%%%%%%%%%%%%%%%%%%%%%%%%%%%%%%%%%%%%%%%%%%%%%%%%%%%%%%%%%%%%%%%%%%%%%%%%%%%%%
% \section{Implementation}
%\iffalse
%<*package>
%\fi
%
% This section describes the definitions file |childdoc.def|.

% The definitions cannot be loaded using |\usepackage| or |\RequirePackage|
% which has a mechanism to prevent loading a style file more than once.
% When loading the definitions by means of |\input|
% multiple instances have to be prevented manually:
%\iffalse
%This code needs to be before the `\ProvidesFile' directive
%which is defined at the beginning of this file.
%Therefore it is also placed there and commented out here.
%</package>
%<*discard>
%\fi
%    \begin{macrocode}
\ifdefined\childdocmain\endinput\fi
%    \end{macrocode}
%\iffalse
%</discard>
%<*package>
%\fi
%
% \macro{\ifchilddoc}
% \macro{\ifchilddocmanual}
% The conditional |\ifchilddoc| tells whether a
% child (true) or main (false) document is being compiled.
% The conditional |\ifchilddocmanual| tells whether
% the |\includeonly| mechanism is used (false) or
% the selection of child files must be performed manually (true).
% The definitions initialise to false:
%    \begin{macrocode}
\newif\ifchilddoc
\newif\ifchilddocmanual
%    \end{macrocode}

% \macro{\childdocname}
% \macro{\childdocjob}
% The macro |\childdocname| stores the name of the main document
% to be compiled. The macro |\childdocjob| stores the name of
% the document on which the \LaTeX{} compiler was originally invoked.
% The content of |\jobname| cannot be compared
% to filenames specified in the source due to different catcodes.
% The following code rescans |\jobname|, stores the result
% in |\childdocname| and saves a copy in |\childdocjob|:
%    \begin{macrocode}
\edef\childdocname{\scantokens\expandafter{\jobname\noexpand}}
\let\childdocjob\childdocname
%    \end{macrocode}

% \macro{\childdocdisable}
% The macro |\childdocdisable| prevents the main file
% from being processed more than once.
% At this stage, the main document command |\childdocmain|
% is assumed to be called once again where it should do nothing.
% Any subsequent call to it should prevent
% a secondary processing of the main document
% It overwrites the forwarding commands
% |\childdocof| and |\childdocforward|
% with empty macros to prevent further inclusions of the main document:
%    \begin{macrocode}
\newcommand{\childdocdisable}
{
  \renewcommand{\childdocmain}[1]{\renewcommand{\childdocmain}[1]{\endinput}}
  \renewcommand{\childdocof}[1]{}
  \renewcommand{\childdocby}[2][]{}
  \renewcommand{\childdocforward}[2][]{}
  \renewcommand{\childdocdisable}{}
}
%    \end{macrocode}

% \macro{\childdocmain}
% The macro |\childdocmain| is to be called at the top of the main file
% with nothing or the main filename (without extension) as argument.
% First, it breaks loops.
% If the argument is not empty and does not match |\childdocname|
% (which is set by the first inclusion of |childdoc.def|),
% |\ifchilddoc| is set to true, |\includeonly| is applied to the child file
% and |\jobname| is set to the main file
% (for proper handling of |.aux| files):
%    \begin{macrocode}
\newcommand{\childdocmain}[1]
{
  \childdocdisable\childdocmain{}
  \if?#1?\else
    \begingroup
      \def\childdoctmp{#1}
      \ifx\childdoctmp\childdocname
        \def\childdoctmp{}
      \else
        \def\childdoctmp
        {
          \childdoctrue
          \includeonly{\childdocname}
          \def\childdocjob{#1}
          \def\jobname{#1}
        }
      \fi
      \expandafter
    \endgroup
    \childdoctmp
  \fi
}
%    \end{macrocode}

% \macro{\childdocof}
% The command |\childdocof| redirects
% compilation to the main file |#1|.
%    \begin{macrocode}
\newcommand{\childdocof}[1]
{
  \childdocdisable
  \childdoctrue
  \includeonly{\childdocname}
  \def\jobname{#1}
  \def\childdocjob{#1}
  \input{#1}
}
%    \end{macrocode}

% \macro{\childdocby}
% The command |\childdocby| ....
%    \begin{macrocode}
\newcommand{\childdocby}[2][]
{
  \childdocdisable
  \childdoctrue
  \childdocmanualtrue
  \if?#1?\else
    \def\jobname{#2}
  \fi
  \def\childdocjob{#2}
  \input{#2}
  \endinput
}
%    \end{macrocode}

% \macro{\childdocforward}
% The command |\childdocforward| redirects
% compilation to the main file or
% (if the optional argument is given) a child file.
% Parameters are set as if the main file
% or a child file starting with |\childdocof| was compiled.
% Then compilation is handed over to the main file:
%    \begin{macrocode}
\newcommand{\childdocforward}[2][]
{
  \begingroup
    \if?#1?
      \def\childdoctmp
      {
        \def\childdocname{#2}
        \def\childdocjob{#2}
        \def\jobname{#2}
        \input{#2}
        \endinput
      }
    \else
      \def\childdoctmp
      {
        \childdocdisable
        \def\childdocname{#2}
        \childdoctrue
        \includeonly{#2}
        \def\childdocjob{#1}
        \def\jobname{#1}
        \input{#1}
        \endinput
      }
    \fi
    \expandafter
  \endgroup
  \childdoctmp
}
%    \end{macrocode}

% \macro{\childdocforwardprefix}
% The command |\childdocforwardprefix| redirects
% compilation to the main or a child file by means of a pattern.
% The prefix |#1| in the current filename is replaced by |#2|
% and the suffix of the current filename is kept
% (it is assumed that the filename does not contain the substring `|~~~|'
% which is used as a delimiter).
% Compilation is handed over to the new file by |\childdocforward|:
%    \begin{macrocode}
\newcommand{\childdocforwardprefix}[3][]
{
  \begingroup
    \def\childdocextract #2##1~~~{\def\childdoctmp{\childdocforward[#1]{#3##1}}}
    \expandafter\childdocextract\childdocname~~~
    \expandafter
  \endgroup
  \childdoctmp
}
%    \end{macrocode}

% \macro{\childdoc}
% The deprecated macro |\childdoc| is a legacy version of |\childdocmain|:
%    \begin{macrocode}
\newcommand{\childdoc}{\childdocmain}
%    \end{macrocode}

% \macro{\childdocredirect}
% The deprecated macro |\childdocredirect| is a legacy version
% of |\childdocforward| and |\childdocforwardprefix|:
%    \begin{macrocode}
\newcommand{\childdocredirect}[2][]
{
  \begingroup
    \if?#1?
      \def\childdoctmp{\childdocforward{#2}}
    \else
      \def\childdoctmp{\childdocforwardprefix{#1}{#2}}
    \fi
    \expandafter
  \endgroup
  \childdoctmp
}
%    \end{macrocode}

%\iffalse
%</package>
%\fi
%
\endinput

\childdocforwardprefix[cdocsamp]{cdocsfn}{cdocsch}
%    \end{macrocode}

%\iffalse
%</samplefinal>
%\fi
%
% %%%%%%%%%%%%%%%%%%%%%%%%%%%%%%%%%%%%%%
% \paragraph{Command Line Processing.}
%
% The following three command lines generate the output files
% |cdocscld|, |cdocscl1| and |cdocscl2|
% which should be identical to
% |cdocsdrf|, |cdocsch1| and |cdocsfn2|, respectively:
% \begin{center}
% \begin{tabular}{l}
% |latex -jobname cdocscld \|\\
% |  "\def\version{draft}% \iffalse
%
% childdoc.dtx Copyright (C) 2017-2018 Niklas Beisert
%
% This work may be distributed and/or modified under the
% conditions of the LaTeX Project Public License, either version 1.3
% of this license or (at your option) any later version.
% The latest version of this license is in
%   http://www.latex-project.org/lppl.txt
% and version 1.3 or later is part of all distributions of LaTeX
% version 2005/12/01 or later.
%
% This work has the LPPL maintenance status `maintained'.
%
% The Current Maintainer of this work is Niklas Beisert.
%
% This work consists of the files childdoc.dtx and childdoc.ins
% and the derived files childdoc.def and cdocsamp.tex with
% cdocsch1.tex, cdocsch2.tex, cdocsdrf.tex, cdocsfn1.tex, cdocsfn2.tex.
%
%<package>\ifdefined\childdocmain\endinput\fi
%<package>\ProvidesFile{childdoc.def}[2018/12/30 v2.0 child document driver]
%<samplemain>\ProvidesFile{cdocsamp.tex}[2018/12/30 v2.0 sample for childdoc]
%<*driver>
%\ProvidesFile{childdoc.drv}[2018/12/30 v2.0 childdoc reference manual file]
\PassOptionsToClass{10pt,a4paper}{article}
\documentclass{ltxdoc}

\usepackage[margin=35mm]{geometry}
\usepackage{hyperref}
\usepackage{hyperxmp}
\usepackage[usenames]{color}

\hypersetup{colorlinks=true}
\hypersetup{pdfstartview=FitH}
\hypersetup{pdfpagemode=UseNone}
\hypersetup{pdfsource={}}
\hypersetup{pdflang={en-UK}}
\hypersetup{pdfcopyright={Copyright 2017-2018 Niklas Beisert.
  This work may be distributed and/or modified under the
  conditions of the LaTeX Project Public License, either version 1.3
  of this license or (at your option) any later version.}}
\hypersetup{pdflicenseurl={http://www.latex-project.org/lppl.txt}}
\hypersetup{pdfcontactaddress={ETH Zurich, ITP, HIT K,
  Wolfgang-Pauli-Strasse 27}}
\hypersetup{pdfcontactpostcode={8093}}
\hypersetup{pdfcontactcity={Zurich}}
\hypersetup{pdfcontactcountry={Switzerland}}
\hypersetup{pdfcontactemail={nbeisert@itp.phys.ethz.ch}}
\hypersetup{pdfcontacturl={http://people.phys.ethz.ch/\xmptilde nbeisert/}}

\newcommand{\secref}[1]{\hyperref[#1]{section \ref*{#1}}}

\parskip1ex
\parindent0pt
\let\olditemize\itemize
\def\itemize{\olditemize\parskip0pt}

\begin{document}

\title{The \textsf{childdoc} Package}
\hypersetup{pdftitle={The childdoc Package}}
\author{Niklas Beisert\\[2ex]
  Institut f\"ur Theoretische Physik\\
  Eidgen\"ossische Technische Hochschule Z\"urich\\
  Wolfgang-Pauli-Strasse 27, 8093 Z\"urich, Switzerland\\[1ex]
  \href{mailto:nbeisert@itp.phys.ethz.ch}
  {\texttt{nbeisert@itp.phys.ethz.ch}}}
\hypersetup{pdfauthor={Niklas Beisert}}
\hypersetup{pdfsubject={Manual for the LaTeX2e Package childdoc}}
\date{30 December 2018, \textsf{v2.0}}
\maketitle

\begin{abstract}\noindent
\textsf{childdoc} is a \LaTeXe{} package
that enables the direct compilation
of document sections included by |\include|
to individual files.
\end{abstract}

\begingroup
\parskip0ex
\tableofcontents
\endgroup

%%%%%%%%%%%%%%%%%%%%%%%%%%%%%%%%%%%%%%%%%%%%%%%%%%%%%%%%%%%%%%%%%%%%%%%%%%%%%%%%
%%%%%%%%%%%%%%%%%%%%%%%%%%%%%%%%%%%%%%%%%%%%%%%%%%%%%%%%%%%%%%%%%%%%%%%%%%%%%%%%
\section{Introduction}

\LaTeX{} provides a mechanism to structure a large document (such as a book)
into a main file and several child files (containing the chapters)
using the |\include| command.
This mechanism is beneficial for documents
which span hundreds of pages in order to
make the source file(s) more manageable.
Moreover, compilation can be restricted to
selected child files by means of the |\includeonly| command.
The latter feature can be used to reduce the compilation time while editing
(this was significantly more useful in the earlier days of \LaTeX{})
or to generate a smaller document which is easier to navigate.
Another application of |\includeonly| is to generate
documents consisting of selected parts of the complete document.

However, there are a few drawbacks of the plain |\include| mechanism:
\begin{itemize}
\item
The child files cannot be compiled on their own,
they can only be compiled via the main file.
A naive editing environment
(such as a text editor with an option
to have the current file processed by \LaTeX)
may require one to switch to the main file before compiling;
attempting to compile the child file produces errors.
\item
The main file must be modified (each time)
to adjust the |\includeonly| command
to the present needs. This easily leaves the main file in a messy state.
\item
The generated document will always carry the filename
of the main document. This is inconvenient if
several child files are to be compiled and
to be kept for distribution.
\end{itemize}

The present package provides a simple interface
to make child files individually compilable by \LaTeX{}.
Compiling a child file then has the same effect as compiling
the main file with an |\includeonly| command
to select the appropriate child.
Moreover the generated document will carry the name of the child
rather than the main file.
This resolves all three above issues.

This feature is meant to make the editing of books,
thesis documents and lecture notes somewhat more convenient.
However, the package can also be used efficiently for
composing a series of documents (such as exercise sheets)
which are typically distributed individually.
It then assists the author in generating the individual documents
(potentially in different versions)
as well as a document containing the collected series.
Another application is in developing style files
or other kinds of included material
where compilation of the style file could redirect
to a sample or test file.

%%%%%%%%%%%%%%%%%%%%%%%%%%%%%%%%%%%%%%%%%%%%%%%%%%%%%%%%%%%%%%%%%%%%%%%%%%%%%%%%
%%%%%%%%%%%%%%%%%%%%%%%%%%%%%%%%%%%%%%%%%%%%%%%%%%%%%%%%%%%%%%%%%%%%%%%%%%%%%%%%
\section{Usage}

First of all, the package \textsf{childdoc} is \emph{not} a standard
\LaTeXe{} |.sty| style file! Therefore it needs to be invoked in
a non-standard way.

%%%%%%%%%%%%%%%%%%%%%%%%%%%%%%%%%%%%%%%%%%%%%%%%%%%%%%%%%%%%%%%%%%%%%%%%%%%%%%%%
\subsection{Included Files}
\label{sec:include}

%%%%%%%%%%%%%%%%%%%%%%%%%%%%%%%%%%%%%%%%
\DescribeMacro{\childdocmain}
To use the package, add the commands
\begin{center}
\begin{tabular}{l}
|\input{childdoc.def}|\\
|\childdocmain{}|\\
\end{tabular}
\end{center}
at the very top of the main \LaTeX{} file,
in particular \emph{before} the |\documentclass| statement!
The argument of |\childdocmain| should be left empty
(but it must be present).

%%%%%%%%%%%%%%%%%%%%%%%%%%%%%%%%%%%%%%%%
\DescribeMacro{\childdocof}
Furthermore, add the commands
\begin{center}
\begin{tabular}{l}
|\input{childdoc.def}|\\
|\childdocof{|\textit{main}|}|\\
\end{tabular}
\end{center}
at the top of every child file \textit{child}
which is included by |\include{|\textit{child}|}|
from within the main file
(or at least for those files to be compiled individually).
The argument \textit{main} must be the filename of the main file.

There are a couple of
considerations in setting up the main and child documents:

%%%%%%%%%%%%%%%%%%%%%%%%%%%%%%%%%%%%%%%%
\paragraph{Restrictions.}

Please note the following restrictions:
\begin{itemize}
\item
|\childdocmain| must be called with one argument \textit{main}
to ensure compatibility with earlier version of the package.
It must either be empty (|\childdocmain{}|)
or precisely match the filename of the main file in which it is specified.
See \secref{sec:detection} for further information.
\item
The filename \textit{main} must be specified without the |.tex| extension.
\item
The filename \textit{main} is case sensitive
(even in case-insensitive file systems)
due to internal string comparison.
\item
The argument \textit{main} should be fully expanded, it cannot be a macro.
\item
Subdirectories and special characters should be avoided in filenames.
\item
The command |\childdocmain{|\textit{main}|}| must be followed by a whitespace.
It should not be followed immediately by another command
or by a comment mark `|%|'.
This is because the \TeX{} parser reads the token immediately following
the argument of |\childdocmain| and puts it
at the beginning of every child section;
however, a white\-space is ignored.
\end{itemize}

%%%%%%%%%%%%%%%%%%%%%%%%%%%%%%%%%%%%%%%%
\paragraph{Content of Main File.}

It is advisable to place all content in the child files included by |\include|.
Any output contained in the main file will appear in all child documents
unless suppressed manually;
it cannot be suppressed automatically by the |\includeonly| directive
and thus should normally be avoided.
A method to include some content in the main file
by means of conditional processing is described in \secref{sec:conditional}.

%%%%%%%%%%%%%%%%%%%%%%%%%%%%%%%%%%%%%%%%
\paragraph{Page Numbering.}

When only a part of the document is compiled,
the appropriate numbering of pages
(as well as other status parameters)
is determined from the |.aux| files.
The latter contain information from previous passes.
However this information needs to propagate through
all intermediate child documents.
Therefore the page numbering in child documents may well
be inconsistent until the complete document is compiled at least once.

A useful (if unconventional) way to always ensure a consistent
page numbering is to restart the numbering in each child document
and denote the pages by `\textit{child}|.|\textit{page}'
where \textit{child} represents the chapter/section number of the child file.
This can be achieved by the command
|\numberwithin{page}{|\textit{child}|}|
of the \textsf{amsmath} package
where \textit{child} can be |chapter| or |section|
depending on the chosen structuring.
Alternatively, one can modify the macro |\thepage| appropriately
and reset the counter |page| at the start of each child file.

%%%%%%%%%%%%%%%%%%%%%%%%%%%%%%%%%%%%%%%%%%%%%%%%%%%%%%%%%%%%%%%%%%%%%%%%%%%%%%%%
\subsection{Conditional Processing}
\label{sec:conditional}

The package provides a mechanism to compile different versions
of a document. To customise the versions further some conditional processing
can come in handy to distinguish which version is being compiled.
The package provides two macros to describe the compilation context:

%%%%%%%%%%%%%%%%%%%%%%%%%%%%%%%%%%%%%%%%
\DescribeMacro{\ifchilddoc}
The conditional |\ifchilddoc| distinguishes between the compilation of
child documents and the main document:
%
\begin{center}
|\ifchilddoc |\textit{child-code}| |[|\||else |\textit{main-code}]| \||fi|
\end{center}

%%%%%%%%%%%%%%%%%%%%%%%%%%%%%%%%%%%%%%%%
\DescribeMacro{\childdocname}
\DescribeMacro{\childdocjob}
The macro |\childdocname| contains the filename (without extension)
of the main or child file being processed.
Note that |\childdocjob| will always contain the name of the main file.

%%%%%%%%%%%%%%%%%%%%%%%%%%%%%%%%%%%%%%%%
\paragraph{Title Page.}

Conditional processing can be used to include a title or banner page
in the main document when proper precautions are taken.
Importantly, the code in the main file should ensure that the page counter
(as well as other status parameters which are stored in the |.aux| files)
takes the same value after the conditional processing.
Otherwise the page numbers may take divergent values
depending on which part is compiled.

For example, a title page could be declared by:
%
\begin{center}
\begin{tabular}{l}
|\ifchilddoc\||else|\\
|\addtocounter{page}{-1}|\\
\textit{code for title page}\\
|\newpage|\\
|\||fi|
\end{tabular}
\end{center}
%
A banner page for the child documents can be generated by:
%
\begin{center}
\begin{tabular}{l}
|\ifchilddoc|\\
|\addtocounter{page}{-1}|\\
\textit{code for banner page}\\
|\newpage|\\
|\||fi|
\end{tabular}
\end{center}
%
Here one could write a message such as:
\begin{center}
|This is the part \childdocname{} of \childdocjob{}.|
\end{center}

%%%%%%%%%%%%%%%%%%%%%%%%%%%%%%%%%%%%%%%%%%%%%%%%%%%%%%%%%%%%%%%%%%%%%%%%%%%%%%%%
\subsection{Flags}
\label{sec:flags}

The package makes it easy to generate different versions
of the main or child documents.
To this end compilation flags can be defined
and assigned different default values.
They will be particularly useful in conjunction
with the forwarding mechanism described in \secref{sec:forward}.

For example, it may be useful to have a flag |\version|
which can be set to |draft| or |final|.
The document source will contain some conditional code
depending on the value of |\version|.
Suppose further, the flag should default to |final| for the main file
and to |draft| for child files
which is a natural assignment for editing the document.
This is achieved by placing the following code
in the preamble of the main document
(below the |\childdocmain| directive):
%
\begin{center}
\begin{tabular}{l}
|\ifchilddoc|\\
|\providecommand{\version}{draft}|\\
|\||else|\\
|\providecommand{\version}{final}|\\
|\||fi|
\end{tabular}
\end{center}
%
The definition by |\providecommand| makes sure
that previous definitions are not overwritten.
Further statements |\providecommand{\version}{...}|
can thus be added before the above code to override it.

For the main file, one might add a line
(between |\childdocmain| and the above block)
%
\begin{center}
|%\ifchilddoc\||else\providecommand{\version}{draft}\||fi|
\end{center}
%
which can be uncommented to produce a draft version.
Likewise one can add a line to the very top of a child file
(above the |\childdocof{|\textit{main}|}| directive)
%
\begin{center}
|%\providecommand{\version}{final}|
\end{center}
%
which can be uncommented to produce the final version of this child document.

%%%%%%%%%%%%%%%%%%%%%%%%%%%%%%%%%%%%%%%%%%%%%%%%%%%%%%%%%%%%%%%%%%%%%%%%%%%%%%%%
\subsection{Forwarding}
\label{sec:forward}

Different versions of the main or child documents
using compilation flags as described in \secref{sec:flags}
can be (permanently) stored in different files
for convenient compilation, viewing and distribution.
To this end, the package defines a command
to pass on compilation to a different file:

%%%%%%%%%%%%%%%%%%%%%%%%%%%%%%%%%%%%%%%%
\DescribeMacro{\childdocforward}
The command |\childdocforward| redirects processing to
another source file:
%
\begin{center}
\begin{tabular}{l}
|\input{childdoc.def}|\\
|\childdocforward[|\textit{main}|]{|\textit{dest}|}|\\
\end{tabular}
\end{center}
%
The argument \textit{dest} is the destination file
(without extension).
It should be the main file or one of the child files.
Note that further \textsf{childdoc} directives
such as |\childdocof| and |\childdocforward|
in the indicated file will be processed in this form.
The optional argument \textit{main}
passes on directly to the main file \textit{main}
while pretending to compile the child \textit{dest}.
This form behaves as if \textit{dest}
issues |\childdocof{|\textit{main}|}| right away,
and no further \textsf{childdoc} directives will be processed.

%%%%%%%%%%%%%%%%%%%%%%%%%%%%%%%%%%%%%%%%
\DescribeMacro{\...prefix}
In the alternative form |\childdocforwardprefix|,
%
\begin{center}
\begin{tabular}{l}
|\input{childdoc.def}|\\
|\childdocforwardprefix[|\textit{main}|]{|\textit{prefix}|}{|\textit{dest}|}|
\end{tabular}
\end{center}
%
the destination file is determined by a pattern
depending on the current file:
To make this work, the current file must be called
`{\textit{prefix}\hspace{0.2em}\textit{suffix}}'
with \textit{prefix} matching precisely the argument.
Processing is then passed on to the file
`{\textit{dest}\hspace{0.2em}\textit{suffix}}'.
Surely, the same effect is achieved by
directly specifying the
argument `{\textit{dest}\hspace{0.2em}\textit{suffix}}'
in the first form.
However, that requires to set up a different file
for each child. With the alternative form of the command
all these files can have exactly the same content
which simplifies setting them up and maintaining them.

For example, the following file |draft.tex|
with a compilation flag |\version| as described in \secref{sec:flags}
compiles the main document as a draft:
%
\begin{center}
\begin{tabular}{l}
|\def\version{draft}|\\
|\input{childdoc.def}|\\
|\childdocforward{|\textit{main}|}|
\end{tabular}
\end{center}
%
Likewise, the following files |final|\textit{nn}|.tex|
compile the final version of the child document
|child|\textit{nn}|.tex|:
%
\begin{center}
\begin{tabular}{l}
|\def\version{final}|\\
|\input{childdoc.def}|\\
|\childdocforwardprefix{final}{child}|
\end{tabular}
\end{center}
%

Note that when several versions of a main file and/or of each child file
are to be generated, it may be convenient to set up a |Makefile| or
shell script to automatise the process.

%%%%%%%%%%%%%%%%%%%%%%%%%%%%%%%%%%%%%%%%%%%%%%%%%%%%%%%%%%%%%%%%%%%%%%%%%%%%%%%%
\subsection{Command Line Processing}
\label{sec:commandline}

The effect of redirection files can also be achieved by invoking
the \LaTeX{} compiler with a more elaborate command line.
Most conveniently this should be done as part
of a shell script or a |Makefile|.

When using \textsf{childdoc} in the main file, the following
command lines effectively perform a redirection
(note that depending on the shell being used,
backslashes may have to be doubled: `|\|' $\to$ `|\\|'):
%
\begin{center}
|... -jobname "|\textit{target}|" |\\|"|[\textit{flags}]%
|\input{childdoc.def}\childdocforward[|\textit{main}|]{|\textit{dest}|}"|
\end{center}
%
Here \textit{target} is the name of the output file,
\textit{main} is the name of the main file
and \textit{dest} is the name of the main or child file to be processed
(all filenames without extensions).
The optional argument \textit{main} can be omitted
if \textit{main} matches \textit{dest}.
Optionally, compilation \textit{flags} can be defined via |\def| commands.
This command line makes the \TeX{} engine believe
it is compiling the file \textit{target}
whose content is specified as the latter parameter.
The provided code then forwards the processing to
\textit{main} or \textit{dest} as described in \secref{sec:forward}.

%%%%%%%%%%%%%%%%%%%%%%%%%%%%%%%%%%%%%%%%%%%%%%%%%%%%%%%%%%%%%%%%%%%%%%%%%%%%%%%%
\subsection{Include by Input}
\label{sec:input}

Including child documents by |\include| has some restrictions by design.
Most notably, the content of a child document always occupies
its own set of pages; pages cannot be shared between child documents.
Usually, this behaviour makes perfect sense
because each child document contain an essential part of the document.
However, in some situations it may be desirable to compose
a document from a collection of parts
without having mandatory page breaks between then.
For this case, the package
provides a mechanism to include parts
by |\input| which can also be processed individually.
However, by construction this mechanism
requires manual handling of the content to be output.

%%%%%%%%%%%%%%%%%%%%%%%%%%%%%%%%%%%%%%%%
\DescribeMacro{\ifchilddocmanual}
The main file should be prepared as usual, see \secref{sec:include}.
However, the document body must make a distinction
between processing of an individual part and of the main document, e.g.:
%
\begin{center}
\begin{tabular}{l}
|\ifchilddocmanual|\\
|\input{\childdocname}|\\
|\||else|\\
\textit{document body with }|\input{|\textit{part}|}|\\
|\||fi|
\end{tabular}
\end{center}
%
The conditional |\ifchilddocmanual| is true whenever
a part to be included by |\input| is being compiled,
and the name of the part is stored in |\childdocname|.

%%%%%%%%%%%%%%%%%%%%%%%%%%%%%%%%%%%%%%%%
\DescribeMacro{\childdocby}
Each part to be included by |\input| should start with:
%
\begin{center}
\begin{tabular}{l}
|\input{childdoc.def}|\\
|\childdocby{|\textit{main}|}|\\
\end{tabular}
\end{center}
%
The directive |\childdocby| is similar to |\childdocof|
described in \secref{sec:include},
but the subsequent selection of content must be done manually.
To that end, both |\ifchilddoc| and |\ifchilddocmanual|
will be true upon processing of a part,
and the name of the part is stored in |\childdocname|.
Note that |\jobname| will be set to the filename of the current part
so that each part receives an individual |.aux| file
that does not interfere with the |.aux| file(s) of the main document.
This behaviour can be altered by the alternative form
|\childdocby[*]{|\textit{main}|}| (with a non-empty optional argument)
which uses the |.aux| file of the main document
by setting |\jobname| to \textit{main}.

%%%%%%%%%%%%%%%%%%%%%%%%%%%%%%%%%%%%%%%%%%%%%%%%%%%%%%%%%%%%%%%%%%%%%%%%%%%%%%%%
\subsection{Driver Development}
\label{sec:driver}

The \textsf{childdoc} mechanism can also be use for the development
of definition files such as \LaTeX{} styles or classes.
This case differs from the above setup with multiple parts
included by |\include| in that no |\includeonly| should be invoked.
This can be achieved by starting the include file
(before |\ProvidesPackage|) with:
%
\begin{center}
\begin{tabular}{l}
|\input{childdoc.def}|\\
|\childdocforward{|\textit{main}|}|\\
\end{tabular}
\end{center}
%
or alternatively with:
%
\begin{center}
\begin{tabular}{l}
|\input{childdoc.def}|\\
|\childdocby{|\textit{main}|}|\\
\end{tabular}
\end{center}
%
Both forms have slightly different effects as described above.
The main file is prepared as usual, see \secref{sec:include}.

%%%%%%%%%%%%%%%%%%%%%%%%%%%%%%%%%%%%%%%%%%%%%%%%%%%%%%%%%%%%%%%%%%%%%%%%%%%%%%%%
\subsection{Legacy Detection}
\label{sec:detection}

The directive |\childdocmain| in the main file can detect
whether the complete document or merely a child is to be compiled
even without using the directive |\childdocof|.
This method is deprecated because it is less robust
and there is no compelling reason to use it;
it is merely provided for backward compatibility
and it may be removed in future versions.

If the detection mechanism is to be used,
it is mandatory to correctly specify
the filename of the main file as the argument of |\childdocmain|:
%
\begin{center}
\begin{tabular}{l}
|\input{childdoc.def}|\\
|\childdocmain{|\textit{main}|}|\\
\end{tabular}
\end{center}
%
If |\jobname| does not match the argument \textit{main} of |\childdocmain|,
it is assumed that |\jobname| points to the child file to be compiled.
When using |\childdocmain| with the main file specified as argument,
it suffices to start a child file
with just |\input{|\textit{main}|}|
without loading of the package and using |\childdocof|.
If instead all processing is done
with the appropriate \textsf{childdoc} directives,
the argument of \textit{main} of |\childdocmain| can be empty.

An alternative version of the command line processing described
in \secref{sec:commandline} using the detection mechanism reads:
%
\begin{center}
|... -jobname "|\textit{target}|" "|[\textit{flags}]%
[|\def\jobname{|\textit{dest}|}|]|\input{|\textit{main}|}"|
\end{center}

%%%%%%%%%%%%%%%%%%%%%%%%%%%%%%%%%%%%%%%%%%%%%%%%%%%%%%%%%%%%%%%%%%%%%%%%%%%%%%%%
\subsection{Manual Code}
\label{sec:manual}

In case one cannot be certain whether the definitions file |childdoc.def|
is installed on the target \TeX{} distribution
and one prefers not to ship it,
it is conceivable to paste a few relevant commands into the sources.

To that end, drop all statements |\input{childdoc.def}|
and perform the replacements as outlined below.
Instead of |\childdocmain{|\textit{main}|}| add the following code
to the top of the main file:
%
\begin{center}
\begin{tabular}{l}
|\||ifdefined\childdocname\endinput\||fi\newif\ifchilddoc|\\
|\edef\childdocname{\scantokens\expandafter{\jobname\noexpand}}|\\
|\def\childdocmain{|\textit{main}|}\||ifx\childdocmain\childdocname\||else|\\
|\childdoctrue\includeonly{\childdocname}\let\jobname\childdocmain\||fi|\\
\end{tabular}
\end{center}
%
Instead of |\childdocof{|\textit{main}|}| just include the main file
at the top of each child file:
%
\begin{center}
|\input{|\textit{main}|}|
\end{center}
%
A simple redirection |\childdocforward{|\textit{dest}|}| is achieved by:
%
\begin{center}
|\def\jobname{|\textit{dest}|}\input{\jobname}|
\end{center}
%
The redirection with prefix
|\childdocforwardprefix[|\textit{prefix}|]{|\textit{dest}|}|
is accomplished by:
%
\begin{center}
\begin{tabular}{l}
|{\edef\jobname{\scantokens\expandafter{\jobname\noexpand}}|\\
|\def\redirectjob |\textit{prefix}|#1~~~{\gdef\jobname{|\textit{dest}|#1}}|\\
|\expandafter\redirectjob\jobname~~~}\input{\jobname}|
\end{tabular}
\end{center}

In an alternative approach,
child documents can be compiled by a specific command line
without additional code or specific definitions:
%
\begin{center}
|... -jobname "|\textit{target}|" "|[\textit{flags}]%
|\includeonly{|\textit{dest}|}\input{|\textit{main}|}"|
\end{center}
%

%%%%%%%%%%%%%%%%%%%%%%%%%%%%%%%%%%%%%%%%%%%%%%%%%%%%%%%%%%%%%%%%%%%%%%%%%%%%%%%%
%%%%%%%%%%%%%%%%%%%%%%%%%%%%%%%%%%%%%%%%%%%%%%%%%%%%%%%%%%%%%%%%%%%%%%%%%%%%%%%%
\section{Information}

%%%%%%%%%%%%%%%%%%%%%%%%%%%%%%%%%%%%%%%%%%%%%%%%%%%%%%%%%%%%%%%%%%%%%%%%%%%%%%%%
\subsection{Copyright}

Copyright \copyright{} 2017--2018 Niklas Beisert

This work may be distributed and/or modified under the
conditions of the \LaTeX{} Project Public License, either version 1.3
of this license or (at your option) any later version.
The latest version of this license is in
  \url{http://www.latex-project.org/lppl.txt}
and version 1.3 or later is part of all distributions of \LaTeX{}
version 2005/12/01 or later.

This work has the LPPL maintenance status `maintained'.

The Current Maintainer of this work is Niklas Beisert.

This work consists of the files |README.txt|, |childdoc.ins| and |childdoc.dtx|
as well as the derived files |childdoc.def|, |cdocsamp.tex|
with |cdocsch1.tex|, |cdocsch2.tex|, |cdocspt3.tex|, |cdocspt4.tex|,
|cdocsdrf.tex|, |cdocsfn1.tex|, |cdocsfn2.tex|
as well as |childdoc.pdf|.

%%%%%%%%%%%%%%%%%%%%%%%%%%%%%%%%%%%%%%%%%%%%%%%%%%%%%%%%%%%%%%%%%%%%%%%%%%%%%%%%
\subsection{Files and Installation}

The package consists of the files:
%
\begin{center}
\begin{tabular}{ll}
    |README.txt|   & readme file \\
    |childdoc.ins| & installation file \\
    |childdoc.dtx| & source file \\
    |childdoc.def| & definition file \\
    |cdocsamp.tex| & sample main file \\
    |cdocsch1.tex| & sample include file \\
    |cdocsch2.tex| & sample include file \\
    |cdocspt3.tex| & sample part file \\
    |cdocspt4.tex| & sample part file \\
    |cdocsdrf.tex| & sample redirection file \\
    |cdocsfn1.tex| & sample redirection file \\
    |cdocsfn2.tex| & sample redirection file \\
    |childdoc.pdf| & manual
\end{tabular}
\end{center}
%
The distribution consists of the files
|README.txt|, |childdoc.ins| and |childdoc.dtx|.
%
\begin{itemize}
\item
Run (pdf)\LaTeX{} on |childdoc.dtx|
to compile the manual |childdoc.pdf| (this file).
\item
Run \LaTeX{} on |childdoc.ins| to create the definitions file |childdoc.def|
and the sample |cdocsamp.tex| with include files
|cdocsch1.tex|, |cdocsch2.tex|, |cdocspt3.tex|, |cdocspt4.tex|,
|cdocsdrf.tex|, |cdocsfn1.tex|, |cdocsfn2.tex|.
Then copy the file |childdoc.def| to an appropriate directory of your \LaTeX{}
distribution, e.g.\ \textit{texmf-root}|/tex/latex/childdoc|.
\end{itemize}

%%%%%%%%%%%%%%%%%%%%%%%%%%%%%%%%%%%%%%%%%%%%%%%%%%%%%%%%%%%%%%%%%%%%%%%%%%%%%%%%
\subsection{Related CTAN Packages}

There are several other packages which offer a similar functionality:
%
\begin{itemize}
\item
The packages
\href{http://ctan.org/pkg/docmute}{\textsf{docmute}},
\href{http://ctan.org/pkg/includex}{\textsf{includex}} and
\href{http://ctan.org/pkg/standalone}{\textsf{standalone}}
provide commands to include only the document body of
a child file thus allowing both files to be compiled individually.
\item
The packages \href{http://ctan.org/pkg/subdocs}{\textsf{subdocs}}
and \href{http://ctan.org/pkg/subfiles}{\textsf{subfiles}}
provide structures in which the main and child documents can be
encapsulated and allowing them to be compiled individually.
The inclusion mechanism is different from the conventional |\include|.
\item
The package \href{http://ctan.org/pkg/combine}{\textsf{combine}}
is an elaborate solution to combine several documents into one.
\end{itemize}
%
See also the CTAN topic \href{http://ctan.org/topic/subdocs}{\textsf{subdocs}}
for further related packages.
The present package differs from the above solutions in that
a document structure constructed with the conventional |\include| mechanism
just needs two extra commands at the top of every file
such that all constituent files can be compiled individually.

%%%%%%%%%%%%%%%%%%%%%%%%%%%%%%%%%%%%%%%%%%%%%%%%%%%%%%%%%%%%%%%%%%%%%%%%%%%%%%%%
%\subsection{Feature Suggestions}
%
%The following is a list of features which may be useful for future
%versions of this package:
%%
%\begin{itemize}
%\item
%\ldots
%\end{itemize}

%%%%%%%%%%%%%%%%%%%%%%%%%%%%%%%%%%%%%%%%%%%%%%%%%%%%%%%%%%%%%%%%%%%%%%%%%%%%%%%%
\subsection{Revision History}

%%%%%%%%%%%%%%%%%%%%%%%%%%%%%%%%%%%%%%%%
\paragraph{v2.0:} 2018/12/30

\begin{itemize}
\item
immediate forward processing
\item
added |\childdocby| mechanism
\item
manual restructured
\end{itemize}

%%%%%%%%%%%%%%%%%%%%%%%%%%%%%%%%%%%%%%%%
\paragraph{v1.6:} 2018/01/17

\begin{itemize}
\item
application for development of include files
\item
corrections to manual
\end{itemize}

%%%%%%%%%%%%%%%%%%%%%%%%%%%%%%%%%%%%%%%%
\paragraph{v1.5:} 2017/05/21

\begin{itemize}
\item
more complete structuring introduced
\item
|\childdocof| introduced
\item
|\childdoc| renamed to |\childdocmain|
\item
|\childredirect| renamed to |\childdocforward| and |\childdocforwardprefix|
and functionality expanded
\end{itemize}

%%%%%%%%%%%%%%%%%%%%%%%%%%%%%%%%%%%%%%%%
\paragraph{v1.0:} 2017/04/27

\begin{itemize}
\item
manual and install package
\item
first version published on CTAN
\end{itemize}

%%%%%%%%%%%%%%%%%%%%%%%%%%%%%%%%%%%%%%%%
\paragraph{v0.6:} 2017/04/26

\begin{itemize}
\item
redirection mechanism added
\end{itemize}

%%%%%%%%%%%%%%%%%%%%%%%%%%%%%%%%%%%%%%%%
\paragraph{v0.5:} 2017/04/26

\begin{itemize}
\item
functionality in definition file
\end{itemize}


%%%%%%%%%%%%%%%%%%%%%%%%%%%%%%%%%%%%%%%%%%%%%%%%%%%%%%%%%%%%%%%%%%%%%%%%%%%%%%%%
%%%%%%%%%%%%%%%%%%%%%%%%%%%%%%%%%%%%%%%%%%%%%%%%%%%%%%%%%%%%%%%%%%%%%%%%%%%%%%%%
%%%%%%%%%%%%%%%%%%%%%%%%%%%%%%%%%%%%%%%%%%%%%%%%%%%%%%%%%%%%%%%%%%%%%%%%%%%%%%%%
\appendix

\settowidth\MacroIndent{\rmfamily\scriptsize 000\ }

 \DocInput{childdoc.dtx}

\end{document}
%</driver>
% \fi
%
% %%%%%%%%%%%%%%%%%%%%%%%%%%%%%%%%%%%%%%%%%%%%%%%%%%%%%%%%%%%%%%%%%%%%%%%%%%%%%%
% %%%%%%%%%%%%%%%%%%%%%%%%%%%%%%%%%%%%%%%%%%%%%%%%%%%%%%%%%%%%%%%%%%%%%%%%%%%%%%
% \section{Sample}
%\iffalse
%<*samplemain>
%\fi
%
% The following presents a sample document
% with two chapters, two parts, a title page,
% a compile flag as well as three forwarding files to set the flag.
% It consists of eight |.tex| files:
% \begin{center}
% \begin{tabular}{ll}
% |cdocsamp.tex|&main file\\
% |cdocsch1.tex|&include file for chapter 1\\
% |cdocsch2.tex|&include file for chapter 2\\
% |cdocspt3.tex|&include file for part 3\\
% |cdocspt4.tex|&include file for part 4\\
% |cdocsdrf.tex|&forwarding file for main file in draft mode\\
% |cdocsfi1.tex|&forwarding file for final version of chapter 1\\
% |cdocsfi2.tex|&forwarding file for final version of chapter 2\\
% \end{tabular}
% \end{center}
% Each of the eight files can be compiled directly by the \LaTeX{} compiler.
%
% %%%%%%%%%%%%%%%%%%%%%%%%%%%%%%%%%%%%%%
% \paragraph{Main File.}
%
% The main file is called |cdocsamp.tex|.
%
% Load the \textsf{childdoc} definitions and
% declare the filename for the main document:
%    \begin{macrocode}
\input{childdoc.def}
\childdocmain{}
%    \end{macrocode}

% Optional override for |\version| flag:
%    \begin{macrocode}
%%\ifchilddoc\else\providecommand{\version}{draft}\fi
%    \end{macrocode}

% Define the default values for the |\version| flag
% (|final| for the main file and |draft| for childs):
%    \begin{macrocode}
\ifchilddoc
\providecommand{\version}{draft}
\else
\providecommand{\version}{final}
\fi
%    \end{macrocode}

% Load the standard document class:
%    \begin{macrocode}
\documentclass[12pt]{article}
%    \end{macrocode}

% Start the document body:
%    \begin{macrocode}
\begin{document}
%    \end{macrocode}

% Declare a title page.
% Print title, part of document being processed and version flag:
%    \begin{macrocode}
\addtocounter{page}{-1}
\begin{center}
{\LARGE\bfseries{}childdoc example\par}
\vspace{1cm}
\ifchilddoc
\ifchilddocmanual part\else chapter\fi:
`\childdocname' of `\childdocjob'\par
\else
main document: `\childdocjob'\par
\fi
version: \version\par
\end{center}
\newpage
%    \end{macrocode}

% Manually include selected file,
% otherwise process as usual:
%    \begin{macrocode}
\ifchilddocmanual
\section*{part `\childdocname'}
\input{\childdocname}
\else
%    \end{macrocode}

% Include the two chapters:
%    \begin{macrocode}
\include{cdocsch1}
\include{cdocsch2}
%    \end{macrocode}

% Include the two parts unless only chapters should be displayed:
%    \begin{macrocode}
\ifchilddoc\else
\section{part three}
\input{cdocspt3}
\section{part four}
\input{cdocspt4}
\fi
%    \end{macrocode}

% Process as usual until here:
%    \begin{macrocode}
\fi
%    \end{macrocode}

% End of document body:
%    \begin{macrocode}
\end{document}
%    \end{macrocode}
%\iffalse
%</samplemain>
%\fi
%
% %%%%%%%%%%%%%%%%%%%%%%%%%%%%%%%%%%%%%%
% \paragraph{Chapter Include Files.}
%
% The include files are called |cdocsch1.tex| and |cdocsch2.tex|.
%
%\iffalse
%<*samplechap1|samplechap2>
%\fi

% Optional override for |\version| flag:
%    \begin{macrocode}
%%\providecommand{\version}{final}
%    \end{macrocode}

% Include the main document:
%    \begin{macrocode}
\input{childdoc.def}
\childdocof{cdocsamp}
%    \end{macrocode}

%\iffalse
%</samplechap1|samplechap2>
%\fi
%
%\iffalse
%<*samplechap1>
%\fi
% Some text for chapter 1:
%    \begin{macrocode}
\section{one}
some text in chapter one
%    \end{macrocode}

%\iffalse
%</samplechap1>
%\fi
% Some text for chapter 2:
%\iffalse
%<*samplechap2>
%\fi
%    \begin{macrocode}
\section{two}
more text in chapter two
%    \end{macrocode}

%\iffalse
%</samplechap2>
%\fi
%
% %%%%%%%%%%%%%%%%%%%%%%%%%%%%%%%%%%%%%%
% \paragraph{Part Include Files.}
%
% The include files are called |cdocspt3.tex| and |cdocspt4.tex|.
%
%\iffalse
%<*samplepart3|samplepart4>
%\fi

% Optional override for |\version| flag:
%    \begin{macrocode}
%%\providecommand{\version}{final}
%    \end{macrocode}

% Include the main document:
%    \begin{macrocode}
\input{childdoc.def}
\childdocby{cdocsamp}
%    \end{macrocode}

%\iffalse
%</samplepart3|samplepart4>
%\fi
%
%\iffalse
%<*samplepart3>
%\fi
% Some text for part 3:
%    \begin{macrocode}
some text in part three
%    \end{macrocode}

%\iffalse
%</samplepart3>
%\fi
% Some text for part 4:
%\iffalse
%<*samplepart4>
%\fi
%    \begin{macrocode}
more text in part four
%    \end{macrocode}

%\iffalse
%</samplepart4>
%\fi
%
% %%%%%%%%%%%%%%%%%%%%%%%%%%%%%%%%%%%%%%
% \paragraph{Forwarding for a Complete Draft.}
%
% The following forwarding file |cdocsdrf.tex|
% compiles the main document in draft mode:
%\iffalse
%<*sampledraft>
%\fi
%    \begin{macrocode}
\def\version{draft}
\input{childdoc.def}
\childdocforward{cdocsamp}
%    \end{macrocode}

%\iffalse
%</sampledraft>
%\fi
%
% %%%%%%%%%%%%%%%%%%%%%%%%%%%%%%%%%%%%%%
% \paragraph{Forwarding for Final Version of the Chapters.}
%
% The following forwarding files |cdocsfn1.tex| and |cdocsfn2.tex|
% (with identical content)
% compile the final versions of the child documents
% |cdocsch1.tex| and |cdocsch2.tex|, respectively:
%\iffalse
%<*samplefinal>
%\fi
%    \begin{macrocode}
\def\version{final}
\input{childdoc.def}
\childdocforwardprefix[cdocsamp]{cdocsfn}{cdocsch}
%    \end{macrocode}

%\iffalse
%</samplefinal>
%\fi
%
% %%%%%%%%%%%%%%%%%%%%%%%%%%%%%%%%%%%%%%
% \paragraph{Command Line Processing.}
%
% The following three command lines generate the output files
% |cdocscld|, |cdocscl1| and |cdocscl2|
% which should be identical to
% |cdocsdrf|, |cdocsch1| and |cdocsfn2|, respectively:
% \begin{center}
% \begin{tabular}{l}
% |latex -jobname cdocscld \|\\
% |  "\def\version{draft}\input{childdoc.def}\childdocforward{cdocsamp}"|\\
% |latex -jobname cdocscl1 \|\\
% |  "\input{childdoc.def}\childdocforward[cdocsamp]{cdocsch1}"|\\
% |latex -jobname cdocscl2 \|\\
% |  "\def\version{final}\input{childdoc.def}\childdocforward{cdocsch2}"|
% \end{tabular}
% \end{center}
% Note that the trailing backslash on each first line
% merely continues the input to the second line
% (for convenient cut ant paste).
% Furthermore, the command |latex| can be replaced by any
% of its alternative versions such as |pdflatex|.
%
% %%%%%%%%%%%%%%%%%%%%%%%%%%%%%%%%%%%%%%%%%%%%%%%%%%%%%%%%%%%%%%%%%%%%%%%%%%%%%%
% %%%%%%%%%%%%%%%%%%%%%%%%%%%%%%%%%%%%%%%%%%%%%%%%%%%%%%%%%%%%%%%%%%%%%%%%%%%%%%
% \section{Implementation}
%\iffalse
%<*package>
%\fi
%
% This section describes the definitions file |childdoc.def|.

% The definitions cannot be loaded using |\usepackage| or |\RequirePackage|
% which has a mechanism to prevent loading a style file more than once.
% When loading the definitions by means of |\input|
% multiple instances have to be prevented manually:
%\iffalse
%This code needs to be before the `\ProvidesFile' directive
%which is defined at the beginning of this file.
%Therefore it is also placed there and commented out here.
%</package>
%<*discard>
%\fi
%    \begin{macrocode}
\ifdefined\childdocmain\endinput\fi
%    \end{macrocode}
%\iffalse
%</discard>
%<*package>
%\fi
%
% \macro{\ifchilddoc}
% \macro{\ifchilddocmanual}
% The conditional |\ifchilddoc| tells whether a
% child (true) or main (false) document is being compiled.
% The conditional |\ifchilddocmanual| tells whether
% the |\includeonly| mechanism is used (false) or
% the selection of child files must be performed manually (true).
% The definitions initialise to false:
%    \begin{macrocode}
\newif\ifchilddoc
\newif\ifchilddocmanual
%    \end{macrocode}

% \macro{\childdocname}
% \macro{\childdocjob}
% The macro |\childdocname| stores the name of the main document
% to be compiled. The macro |\childdocjob| stores the name of
% the document on which the \LaTeX{} compiler was originally invoked.
% The content of |\jobname| cannot be compared
% to filenames specified in the source due to different catcodes.
% The following code rescans |\jobname|, stores the result
% in |\childdocname| and saves a copy in |\childdocjob|:
%    \begin{macrocode}
\edef\childdocname{\scantokens\expandafter{\jobname\noexpand}}
\let\childdocjob\childdocname
%    \end{macrocode}

% \macro{\childdocdisable}
% The macro |\childdocdisable| prevents the main file
% from being processed more than once.
% At this stage, the main document command |\childdocmain|
% is assumed to be called once again where it should do nothing.
% Any subsequent call to it should prevent
% a secondary processing of the main document
% It overwrites the forwarding commands
% |\childdocof| and |\childdocforward|
% with empty macros to prevent further inclusions of the main document:
%    \begin{macrocode}
\newcommand{\childdocdisable}
{
  \renewcommand{\childdocmain}[1]{\renewcommand{\childdocmain}[1]{\endinput}}
  \renewcommand{\childdocof}[1]{}
  \renewcommand{\childdocby}[2][]{}
  \renewcommand{\childdocforward}[2][]{}
  \renewcommand{\childdocdisable}{}
}
%    \end{macrocode}

% \macro{\childdocmain}
% The macro |\childdocmain| is to be called at the top of the main file
% with nothing or the main filename (without extension) as argument.
% First, it breaks loops.
% If the argument is not empty and does not match |\childdocname|
% (which is set by the first inclusion of |childdoc.def|),
% |\ifchilddoc| is set to true, |\includeonly| is applied to the child file
% and |\jobname| is set to the main file
% (for proper handling of |.aux| files):
%    \begin{macrocode}
\newcommand{\childdocmain}[1]
{
  \childdocdisable\childdocmain{}
  \if?#1?\else
    \begingroup
      \def\childdoctmp{#1}
      \ifx\childdoctmp\childdocname
        \def\childdoctmp{}
      \else
        \def\childdoctmp
        {
          \childdoctrue
          \includeonly{\childdocname}
          \def\childdocjob{#1}
          \def\jobname{#1}
        }
      \fi
      \expandafter
    \endgroup
    \childdoctmp
  \fi
}
%    \end{macrocode}

% \macro{\childdocof}
% The command |\childdocof| redirects
% compilation to the main file |#1|.
%    \begin{macrocode}
\newcommand{\childdocof}[1]
{
  \childdocdisable
  \childdoctrue
  \includeonly{\childdocname}
  \def\jobname{#1}
  \def\childdocjob{#1}
  \input{#1}
}
%    \end{macrocode}

% \macro{\childdocby}
% The command |\childdocby| ....
%    \begin{macrocode}
\newcommand{\childdocby}[2][]
{
  \childdocdisable
  \childdoctrue
  \childdocmanualtrue
  \if?#1?\else
    \def\jobname{#2}
  \fi
  \def\childdocjob{#2}
  \input{#2}
  \endinput
}
%    \end{macrocode}

% \macro{\childdocforward}
% The command |\childdocforward| redirects
% compilation to the main file or
% (if the optional argument is given) a child file.
% Parameters are set as if the main file
% or a child file starting with |\childdocof| was compiled.
% Then compilation is handed over to the main file:
%    \begin{macrocode}
\newcommand{\childdocforward}[2][]
{
  \begingroup
    \if?#1?
      \def\childdoctmp
      {
        \def\childdocname{#2}
        \def\childdocjob{#2}
        \def\jobname{#2}
        \input{#2}
        \endinput
      }
    \else
      \def\childdoctmp
      {
        \childdocdisable
        \def\childdocname{#2}
        \childdoctrue
        \includeonly{#2}
        \def\childdocjob{#1}
        \def\jobname{#1}
        \input{#1}
        \endinput
      }
    \fi
    \expandafter
  \endgroup
  \childdoctmp
}
%    \end{macrocode}

% \macro{\childdocforwardprefix}
% The command |\childdocforwardprefix| redirects
% compilation to the main or a child file by means of a pattern.
% The prefix |#1| in the current filename is replaced by |#2|
% and the suffix of the current filename is kept
% (it is assumed that the filename does not contain the substring `|~~~|'
% which is used as a delimiter).
% Compilation is handed over to the new file by |\childdocforward|:
%    \begin{macrocode}
\newcommand{\childdocforwardprefix}[3][]
{
  \begingroup
    \def\childdocextract #2##1~~~{\def\childdoctmp{\childdocforward[#1]{#3##1}}}
    \expandafter\childdocextract\childdocname~~~
    \expandafter
  \endgroup
  \childdoctmp
}
%    \end{macrocode}

% \macro{\childdoc}
% The deprecated macro |\childdoc| is a legacy version of |\childdocmain|:
%    \begin{macrocode}
\newcommand{\childdoc}{\childdocmain}
%    \end{macrocode}

% \macro{\childdocredirect}
% The deprecated macro |\childdocredirect| is a legacy version
% of |\childdocforward| and |\childdocforwardprefix|:
%    \begin{macrocode}
\newcommand{\childdocredirect}[2][]
{
  \begingroup
    \if?#1?
      \def\childdoctmp{\childdocforward{#2}}
    \else
      \def\childdoctmp{\childdocforwardprefix{#1}{#2}}
    \fi
    \expandafter
  \endgroup
  \childdoctmp
}
%    \end{macrocode}

%\iffalse
%</package>
%\fi
%
\endinput
\childdocforward{cdocsamp}"|\\
% |latex -jobname cdocscl1 \|\\
% |  "% \iffalse
%
% childdoc.dtx Copyright (C) 2017-2018 Niklas Beisert
%
% This work may be distributed and/or modified under the
% conditions of the LaTeX Project Public License, either version 1.3
% of this license or (at your option) any later version.
% The latest version of this license is in
%   http://www.latex-project.org/lppl.txt
% and version 1.3 or later is part of all distributions of LaTeX
% version 2005/12/01 or later.
%
% This work has the LPPL maintenance status `maintained'.
%
% The Current Maintainer of this work is Niklas Beisert.
%
% This work consists of the files childdoc.dtx and childdoc.ins
% and the derived files childdoc.def and cdocsamp.tex with
% cdocsch1.tex, cdocsch2.tex, cdocsdrf.tex, cdocsfn1.tex, cdocsfn2.tex.
%
%<package>\ifdefined\childdocmain\endinput\fi
%<package>\ProvidesFile{childdoc.def}[2018/12/30 v2.0 child document driver]
%<samplemain>\ProvidesFile{cdocsamp.tex}[2018/12/30 v2.0 sample for childdoc]
%<*driver>
%\ProvidesFile{childdoc.drv}[2018/12/30 v2.0 childdoc reference manual file]
\PassOptionsToClass{10pt,a4paper}{article}
\documentclass{ltxdoc}

\usepackage[margin=35mm]{geometry}
\usepackage{hyperref}
\usepackage{hyperxmp}
\usepackage[usenames]{color}

\hypersetup{colorlinks=true}
\hypersetup{pdfstartview=FitH}
\hypersetup{pdfpagemode=UseNone}
\hypersetup{pdfsource={}}
\hypersetup{pdflang={en-UK}}
\hypersetup{pdfcopyright={Copyright 2017-2018 Niklas Beisert.
  This work may be distributed and/or modified under the
  conditions of the LaTeX Project Public License, either version 1.3
  of this license or (at your option) any later version.}}
\hypersetup{pdflicenseurl={http://www.latex-project.org/lppl.txt}}
\hypersetup{pdfcontactaddress={ETH Zurich, ITP, HIT K,
  Wolfgang-Pauli-Strasse 27}}
\hypersetup{pdfcontactpostcode={8093}}
\hypersetup{pdfcontactcity={Zurich}}
\hypersetup{pdfcontactcountry={Switzerland}}
\hypersetup{pdfcontactemail={nbeisert@itp.phys.ethz.ch}}
\hypersetup{pdfcontacturl={http://people.phys.ethz.ch/\xmptilde nbeisert/}}

\newcommand{\secref}[1]{\hyperref[#1]{section \ref*{#1}}}

\parskip1ex
\parindent0pt
\let\olditemize\itemize
\def\itemize{\olditemize\parskip0pt}

\begin{document}

\title{The \textsf{childdoc} Package}
\hypersetup{pdftitle={The childdoc Package}}
\author{Niklas Beisert\\[2ex]
  Institut f\"ur Theoretische Physik\\
  Eidgen\"ossische Technische Hochschule Z\"urich\\
  Wolfgang-Pauli-Strasse 27, 8093 Z\"urich, Switzerland\\[1ex]
  \href{mailto:nbeisert@itp.phys.ethz.ch}
  {\texttt{nbeisert@itp.phys.ethz.ch}}}
\hypersetup{pdfauthor={Niklas Beisert}}
\hypersetup{pdfsubject={Manual for the LaTeX2e Package childdoc}}
\date{30 December 2018, \textsf{v2.0}}
\maketitle

\begin{abstract}\noindent
\textsf{childdoc} is a \LaTeXe{} package
that enables the direct compilation
of document sections included by |\include|
to individual files.
\end{abstract}

\begingroup
\parskip0ex
\tableofcontents
\endgroup

%%%%%%%%%%%%%%%%%%%%%%%%%%%%%%%%%%%%%%%%%%%%%%%%%%%%%%%%%%%%%%%%%%%%%%%%%%%%%%%%
%%%%%%%%%%%%%%%%%%%%%%%%%%%%%%%%%%%%%%%%%%%%%%%%%%%%%%%%%%%%%%%%%%%%%%%%%%%%%%%%
\section{Introduction}

\LaTeX{} provides a mechanism to structure a large document (such as a book)
into a main file and several child files (containing the chapters)
using the |\include| command.
This mechanism is beneficial for documents
which span hundreds of pages in order to
make the source file(s) more manageable.
Moreover, compilation can be restricted to
selected child files by means of the |\includeonly| command.
The latter feature can be used to reduce the compilation time while editing
(this was significantly more useful in the earlier days of \LaTeX{})
or to generate a smaller document which is easier to navigate.
Another application of |\includeonly| is to generate
documents consisting of selected parts of the complete document.

However, there are a few drawbacks of the plain |\include| mechanism:
\begin{itemize}
\item
The child files cannot be compiled on their own,
they can only be compiled via the main file.
A naive editing environment
(such as a text editor with an option
to have the current file processed by \LaTeX)
may require one to switch to the main file before compiling;
attempting to compile the child file produces errors.
\item
The main file must be modified (each time)
to adjust the |\includeonly| command
to the present needs. This easily leaves the main file in a messy state.
\item
The generated document will always carry the filename
of the main document. This is inconvenient if
several child files are to be compiled and
to be kept for distribution.
\end{itemize}

The present package provides a simple interface
to make child files individually compilable by \LaTeX{}.
Compiling a child file then has the same effect as compiling
the main file with an |\includeonly| command
to select the appropriate child.
Moreover the generated document will carry the name of the child
rather than the main file.
This resolves all three above issues.

This feature is meant to make the editing of books,
thesis documents and lecture notes somewhat more convenient.
However, the package can also be used efficiently for
composing a series of documents (such as exercise sheets)
which are typically distributed individually.
It then assists the author in generating the individual documents
(potentially in different versions)
as well as a document containing the collected series.
Another application is in developing style files
or other kinds of included material
where compilation of the style file could redirect
to a sample or test file.

%%%%%%%%%%%%%%%%%%%%%%%%%%%%%%%%%%%%%%%%%%%%%%%%%%%%%%%%%%%%%%%%%%%%%%%%%%%%%%%%
%%%%%%%%%%%%%%%%%%%%%%%%%%%%%%%%%%%%%%%%%%%%%%%%%%%%%%%%%%%%%%%%%%%%%%%%%%%%%%%%
\section{Usage}

First of all, the package \textsf{childdoc} is \emph{not} a standard
\LaTeXe{} |.sty| style file! Therefore it needs to be invoked in
a non-standard way.

%%%%%%%%%%%%%%%%%%%%%%%%%%%%%%%%%%%%%%%%%%%%%%%%%%%%%%%%%%%%%%%%%%%%%%%%%%%%%%%%
\subsection{Included Files}
\label{sec:include}

%%%%%%%%%%%%%%%%%%%%%%%%%%%%%%%%%%%%%%%%
\DescribeMacro{\childdocmain}
To use the package, add the commands
\begin{center}
\begin{tabular}{l}
|\input{childdoc.def}|\\
|\childdocmain{}|\\
\end{tabular}
\end{center}
at the very top of the main \LaTeX{} file,
in particular \emph{before} the |\documentclass| statement!
The argument of |\childdocmain| should be left empty
(but it must be present).

%%%%%%%%%%%%%%%%%%%%%%%%%%%%%%%%%%%%%%%%
\DescribeMacro{\childdocof}
Furthermore, add the commands
\begin{center}
\begin{tabular}{l}
|\input{childdoc.def}|\\
|\childdocof{|\textit{main}|}|\\
\end{tabular}
\end{center}
at the top of every child file \textit{child}
which is included by |\include{|\textit{child}|}|
from within the main file
(or at least for those files to be compiled individually).
The argument \textit{main} must be the filename of the main file.

There are a couple of
considerations in setting up the main and child documents:

%%%%%%%%%%%%%%%%%%%%%%%%%%%%%%%%%%%%%%%%
\paragraph{Restrictions.}

Please note the following restrictions:
\begin{itemize}
\item
|\childdocmain| must be called with one argument \textit{main}
to ensure compatibility with earlier version of the package.
It must either be empty (|\childdocmain{}|)
or precisely match the filename of the main file in which it is specified.
See \secref{sec:detection} for further information.
\item
The filename \textit{main} must be specified without the |.tex| extension.
\item
The filename \textit{main} is case sensitive
(even in case-insensitive file systems)
due to internal string comparison.
\item
The argument \textit{main} should be fully expanded, it cannot be a macro.
\item
Subdirectories and special characters should be avoided in filenames.
\item
The command |\childdocmain{|\textit{main}|}| must be followed by a whitespace.
It should not be followed immediately by another command
or by a comment mark `|%|'.
This is because the \TeX{} parser reads the token immediately following
the argument of |\childdocmain| and puts it
at the beginning of every child section;
however, a white\-space is ignored.
\end{itemize}

%%%%%%%%%%%%%%%%%%%%%%%%%%%%%%%%%%%%%%%%
\paragraph{Content of Main File.}

It is advisable to place all content in the child files included by |\include|.
Any output contained in the main file will appear in all child documents
unless suppressed manually;
it cannot be suppressed automatically by the |\includeonly| directive
and thus should normally be avoided.
A method to include some content in the main file
by means of conditional processing is described in \secref{sec:conditional}.

%%%%%%%%%%%%%%%%%%%%%%%%%%%%%%%%%%%%%%%%
\paragraph{Page Numbering.}

When only a part of the document is compiled,
the appropriate numbering of pages
(as well as other status parameters)
is determined from the |.aux| files.
The latter contain information from previous passes.
However this information needs to propagate through
all intermediate child documents.
Therefore the page numbering in child documents may well
be inconsistent until the complete document is compiled at least once.

A useful (if unconventional) way to always ensure a consistent
page numbering is to restart the numbering in each child document
and denote the pages by `\textit{child}|.|\textit{page}'
where \textit{child} represents the chapter/section number of the child file.
This can be achieved by the command
|\numberwithin{page}{|\textit{child}|}|
of the \textsf{amsmath} package
where \textit{child} can be |chapter| or |section|
depending on the chosen structuring.
Alternatively, one can modify the macro |\thepage| appropriately
and reset the counter |page| at the start of each child file.

%%%%%%%%%%%%%%%%%%%%%%%%%%%%%%%%%%%%%%%%%%%%%%%%%%%%%%%%%%%%%%%%%%%%%%%%%%%%%%%%
\subsection{Conditional Processing}
\label{sec:conditional}

The package provides a mechanism to compile different versions
of a document. To customise the versions further some conditional processing
can come in handy to distinguish which version is being compiled.
The package provides two macros to describe the compilation context:

%%%%%%%%%%%%%%%%%%%%%%%%%%%%%%%%%%%%%%%%
\DescribeMacro{\ifchilddoc}
The conditional |\ifchilddoc| distinguishes between the compilation of
child documents and the main document:
%
\begin{center}
|\ifchilddoc |\textit{child-code}| |[|\||else |\textit{main-code}]| \||fi|
\end{center}

%%%%%%%%%%%%%%%%%%%%%%%%%%%%%%%%%%%%%%%%
\DescribeMacro{\childdocname}
\DescribeMacro{\childdocjob}
The macro |\childdocname| contains the filename (without extension)
of the main or child file being processed.
Note that |\childdocjob| will always contain the name of the main file.

%%%%%%%%%%%%%%%%%%%%%%%%%%%%%%%%%%%%%%%%
\paragraph{Title Page.}

Conditional processing can be used to include a title or banner page
in the main document when proper precautions are taken.
Importantly, the code in the main file should ensure that the page counter
(as well as other status parameters which are stored in the |.aux| files)
takes the same value after the conditional processing.
Otherwise the page numbers may take divergent values
depending on which part is compiled.

For example, a title page could be declared by:
%
\begin{center}
\begin{tabular}{l}
|\ifchilddoc\||else|\\
|\addtocounter{page}{-1}|\\
\textit{code for title page}\\
|\newpage|\\
|\||fi|
\end{tabular}
\end{center}
%
A banner page for the child documents can be generated by:
%
\begin{center}
\begin{tabular}{l}
|\ifchilddoc|\\
|\addtocounter{page}{-1}|\\
\textit{code for banner page}\\
|\newpage|\\
|\||fi|
\end{tabular}
\end{center}
%
Here one could write a message such as:
\begin{center}
|This is the part \childdocname{} of \childdocjob{}.|
\end{center}

%%%%%%%%%%%%%%%%%%%%%%%%%%%%%%%%%%%%%%%%%%%%%%%%%%%%%%%%%%%%%%%%%%%%%%%%%%%%%%%%
\subsection{Flags}
\label{sec:flags}

The package makes it easy to generate different versions
of the main or child documents.
To this end compilation flags can be defined
and assigned different default values.
They will be particularly useful in conjunction
with the forwarding mechanism described in \secref{sec:forward}.

For example, it may be useful to have a flag |\version|
which can be set to |draft| or |final|.
The document source will contain some conditional code
depending on the value of |\version|.
Suppose further, the flag should default to |final| for the main file
and to |draft| for child files
which is a natural assignment for editing the document.
This is achieved by placing the following code
in the preamble of the main document
(below the |\childdocmain| directive):
%
\begin{center}
\begin{tabular}{l}
|\ifchilddoc|\\
|\providecommand{\version}{draft}|\\
|\||else|\\
|\providecommand{\version}{final}|\\
|\||fi|
\end{tabular}
\end{center}
%
The definition by |\providecommand| makes sure
that previous definitions are not overwritten.
Further statements |\providecommand{\version}{...}|
can thus be added before the above code to override it.

For the main file, one might add a line
(between |\childdocmain| and the above block)
%
\begin{center}
|%\ifchilddoc\||else\providecommand{\version}{draft}\||fi|
\end{center}
%
which can be uncommented to produce a draft version.
Likewise one can add a line to the very top of a child file
(above the |\childdocof{|\textit{main}|}| directive)
%
\begin{center}
|%\providecommand{\version}{final}|
\end{center}
%
which can be uncommented to produce the final version of this child document.

%%%%%%%%%%%%%%%%%%%%%%%%%%%%%%%%%%%%%%%%%%%%%%%%%%%%%%%%%%%%%%%%%%%%%%%%%%%%%%%%
\subsection{Forwarding}
\label{sec:forward}

Different versions of the main or child documents
using compilation flags as described in \secref{sec:flags}
can be (permanently) stored in different files
for convenient compilation, viewing and distribution.
To this end, the package defines a command
to pass on compilation to a different file:

%%%%%%%%%%%%%%%%%%%%%%%%%%%%%%%%%%%%%%%%
\DescribeMacro{\childdocforward}
The command |\childdocforward| redirects processing to
another source file:
%
\begin{center}
\begin{tabular}{l}
|\input{childdoc.def}|\\
|\childdocforward[|\textit{main}|]{|\textit{dest}|}|\\
\end{tabular}
\end{center}
%
The argument \textit{dest} is the destination file
(without extension).
It should be the main file or one of the child files.
Note that further \textsf{childdoc} directives
such as |\childdocof| and |\childdocforward|
in the indicated file will be processed in this form.
The optional argument \textit{main}
passes on directly to the main file \textit{main}
while pretending to compile the child \textit{dest}.
This form behaves as if \textit{dest}
issues |\childdocof{|\textit{main}|}| right away,
and no further \textsf{childdoc} directives will be processed.

%%%%%%%%%%%%%%%%%%%%%%%%%%%%%%%%%%%%%%%%
\DescribeMacro{\...prefix}
In the alternative form |\childdocforwardprefix|,
%
\begin{center}
\begin{tabular}{l}
|\input{childdoc.def}|\\
|\childdocforwardprefix[|\textit{main}|]{|\textit{prefix}|}{|\textit{dest}|}|
\end{tabular}
\end{center}
%
the destination file is determined by a pattern
depending on the current file:
To make this work, the current file must be called
`{\textit{prefix}\hspace{0.2em}\textit{suffix}}'
with \textit{prefix} matching precisely the argument.
Processing is then passed on to the file
`{\textit{dest}\hspace{0.2em}\textit{suffix}}'.
Surely, the same effect is achieved by
directly specifying the
argument `{\textit{dest}\hspace{0.2em}\textit{suffix}}'
in the first form.
However, that requires to set up a different file
for each child. With the alternative form of the command
all these files can have exactly the same content
which simplifies setting them up and maintaining them.

For example, the following file |draft.tex|
with a compilation flag |\version| as described in \secref{sec:flags}
compiles the main document as a draft:
%
\begin{center}
\begin{tabular}{l}
|\def\version{draft}|\\
|\input{childdoc.def}|\\
|\childdocforward{|\textit{main}|}|
\end{tabular}
\end{center}
%
Likewise, the following files |final|\textit{nn}|.tex|
compile the final version of the child document
|child|\textit{nn}|.tex|:
%
\begin{center}
\begin{tabular}{l}
|\def\version{final}|\\
|\input{childdoc.def}|\\
|\childdocforwardprefix{final}{child}|
\end{tabular}
\end{center}
%

Note that when several versions of a main file and/or of each child file
are to be generated, it may be convenient to set up a |Makefile| or
shell script to automatise the process.

%%%%%%%%%%%%%%%%%%%%%%%%%%%%%%%%%%%%%%%%%%%%%%%%%%%%%%%%%%%%%%%%%%%%%%%%%%%%%%%%
\subsection{Command Line Processing}
\label{sec:commandline}

The effect of redirection files can also be achieved by invoking
the \LaTeX{} compiler with a more elaborate command line.
Most conveniently this should be done as part
of a shell script or a |Makefile|.

When using \textsf{childdoc} in the main file, the following
command lines effectively perform a redirection
(note that depending on the shell being used,
backslashes may have to be doubled: `|\|' $\to$ `|\\|'):
%
\begin{center}
|... -jobname "|\textit{target}|" |\\|"|[\textit{flags}]%
|\input{childdoc.def}\childdocforward[|\textit{main}|]{|\textit{dest}|}"|
\end{center}
%
Here \textit{target} is the name of the output file,
\textit{main} is the name of the main file
and \textit{dest} is the name of the main or child file to be processed
(all filenames without extensions).
The optional argument \textit{main} can be omitted
if \textit{main} matches \textit{dest}.
Optionally, compilation \textit{flags} can be defined via |\def| commands.
This command line makes the \TeX{} engine believe
it is compiling the file \textit{target}
whose content is specified as the latter parameter.
The provided code then forwards the processing to
\textit{main} or \textit{dest} as described in \secref{sec:forward}.

%%%%%%%%%%%%%%%%%%%%%%%%%%%%%%%%%%%%%%%%%%%%%%%%%%%%%%%%%%%%%%%%%%%%%%%%%%%%%%%%
\subsection{Include by Input}
\label{sec:input}

Including child documents by |\include| has some restrictions by design.
Most notably, the content of a child document always occupies
its own set of pages; pages cannot be shared between child documents.
Usually, this behaviour makes perfect sense
because each child document contain an essential part of the document.
However, in some situations it may be desirable to compose
a document from a collection of parts
without having mandatory page breaks between then.
For this case, the package
provides a mechanism to include parts
by |\input| which can also be processed individually.
However, by construction this mechanism
requires manual handling of the content to be output.

%%%%%%%%%%%%%%%%%%%%%%%%%%%%%%%%%%%%%%%%
\DescribeMacro{\ifchilddocmanual}
The main file should be prepared as usual, see \secref{sec:include}.
However, the document body must make a distinction
between processing of an individual part and of the main document, e.g.:
%
\begin{center}
\begin{tabular}{l}
|\ifchilddocmanual|\\
|\input{\childdocname}|\\
|\||else|\\
\textit{document body with }|\input{|\textit{part}|}|\\
|\||fi|
\end{tabular}
\end{center}
%
The conditional |\ifchilddocmanual| is true whenever
a part to be included by |\input| is being compiled,
and the name of the part is stored in |\childdocname|.

%%%%%%%%%%%%%%%%%%%%%%%%%%%%%%%%%%%%%%%%
\DescribeMacro{\childdocby}
Each part to be included by |\input| should start with:
%
\begin{center}
\begin{tabular}{l}
|\input{childdoc.def}|\\
|\childdocby{|\textit{main}|}|\\
\end{tabular}
\end{center}
%
The directive |\childdocby| is similar to |\childdocof|
described in \secref{sec:include},
but the subsequent selection of content must be done manually.
To that end, both |\ifchilddoc| and |\ifchilddocmanual|
will be true upon processing of a part,
and the name of the part is stored in |\childdocname|.
Note that |\jobname| will be set to the filename of the current part
so that each part receives an individual |.aux| file
that does not interfere with the |.aux| file(s) of the main document.
This behaviour can be altered by the alternative form
|\childdocby[*]{|\textit{main}|}| (with a non-empty optional argument)
which uses the |.aux| file of the main document
by setting |\jobname| to \textit{main}.

%%%%%%%%%%%%%%%%%%%%%%%%%%%%%%%%%%%%%%%%%%%%%%%%%%%%%%%%%%%%%%%%%%%%%%%%%%%%%%%%
\subsection{Driver Development}
\label{sec:driver}

The \textsf{childdoc} mechanism can also be use for the development
of definition files such as \LaTeX{} styles or classes.
This case differs from the above setup with multiple parts
included by |\include| in that no |\includeonly| should be invoked.
This can be achieved by starting the include file
(before |\ProvidesPackage|) with:
%
\begin{center}
\begin{tabular}{l}
|\input{childdoc.def}|\\
|\childdocforward{|\textit{main}|}|\\
\end{tabular}
\end{center}
%
or alternatively with:
%
\begin{center}
\begin{tabular}{l}
|\input{childdoc.def}|\\
|\childdocby{|\textit{main}|}|\\
\end{tabular}
\end{center}
%
Both forms have slightly different effects as described above.
The main file is prepared as usual, see \secref{sec:include}.

%%%%%%%%%%%%%%%%%%%%%%%%%%%%%%%%%%%%%%%%%%%%%%%%%%%%%%%%%%%%%%%%%%%%%%%%%%%%%%%%
\subsection{Legacy Detection}
\label{sec:detection}

The directive |\childdocmain| in the main file can detect
whether the complete document or merely a child is to be compiled
even without using the directive |\childdocof|.
This method is deprecated because it is less robust
and there is no compelling reason to use it;
it is merely provided for backward compatibility
and it may be removed in future versions.

If the detection mechanism is to be used,
it is mandatory to correctly specify
the filename of the main file as the argument of |\childdocmain|:
%
\begin{center}
\begin{tabular}{l}
|\input{childdoc.def}|\\
|\childdocmain{|\textit{main}|}|\\
\end{tabular}
\end{center}
%
If |\jobname| does not match the argument \textit{main} of |\childdocmain|,
it is assumed that |\jobname| points to the child file to be compiled.
When using |\childdocmain| with the main file specified as argument,
it suffices to start a child file
with just |\input{|\textit{main}|}|
without loading of the package and using |\childdocof|.
If instead all processing is done
with the appropriate \textsf{childdoc} directives,
the argument of \textit{main} of |\childdocmain| can be empty.

An alternative version of the command line processing described
in \secref{sec:commandline} using the detection mechanism reads:
%
\begin{center}
|... -jobname "|\textit{target}|" "|[\textit{flags}]%
[|\def\jobname{|\textit{dest}|}|]|\input{|\textit{main}|}"|
\end{center}

%%%%%%%%%%%%%%%%%%%%%%%%%%%%%%%%%%%%%%%%%%%%%%%%%%%%%%%%%%%%%%%%%%%%%%%%%%%%%%%%
\subsection{Manual Code}
\label{sec:manual}

In case one cannot be certain whether the definitions file |childdoc.def|
is installed on the target \TeX{} distribution
and one prefers not to ship it,
it is conceivable to paste a few relevant commands into the sources.

To that end, drop all statements |\input{childdoc.def}|
and perform the replacements as outlined below.
Instead of |\childdocmain{|\textit{main}|}| add the following code
to the top of the main file:
%
\begin{center}
\begin{tabular}{l}
|\||ifdefined\childdocname\endinput\||fi\newif\ifchilddoc|\\
|\edef\childdocname{\scantokens\expandafter{\jobname\noexpand}}|\\
|\def\childdocmain{|\textit{main}|}\||ifx\childdocmain\childdocname\||else|\\
|\childdoctrue\includeonly{\childdocname}\let\jobname\childdocmain\||fi|\\
\end{tabular}
\end{center}
%
Instead of |\childdocof{|\textit{main}|}| just include the main file
at the top of each child file:
%
\begin{center}
|\input{|\textit{main}|}|
\end{center}
%
A simple redirection |\childdocforward{|\textit{dest}|}| is achieved by:
%
\begin{center}
|\def\jobname{|\textit{dest}|}\input{\jobname}|
\end{center}
%
The redirection with prefix
|\childdocforwardprefix[|\textit{prefix}|]{|\textit{dest}|}|
is accomplished by:
%
\begin{center}
\begin{tabular}{l}
|{\edef\jobname{\scantokens\expandafter{\jobname\noexpand}}|\\
|\def\redirectjob |\textit{prefix}|#1~~~{\gdef\jobname{|\textit{dest}|#1}}|\\
|\expandafter\redirectjob\jobname~~~}\input{\jobname}|
\end{tabular}
\end{center}

In an alternative approach,
child documents can be compiled by a specific command line
without additional code or specific definitions:
%
\begin{center}
|... -jobname "|\textit{target}|" "|[\textit{flags}]%
|\includeonly{|\textit{dest}|}\input{|\textit{main}|}"|
\end{center}
%

%%%%%%%%%%%%%%%%%%%%%%%%%%%%%%%%%%%%%%%%%%%%%%%%%%%%%%%%%%%%%%%%%%%%%%%%%%%%%%%%
%%%%%%%%%%%%%%%%%%%%%%%%%%%%%%%%%%%%%%%%%%%%%%%%%%%%%%%%%%%%%%%%%%%%%%%%%%%%%%%%
\section{Information}

%%%%%%%%%%%%%%%%%%%%%%%%%%%%%%%%%%%%%%%%%%%%%%%%%%%%%%%%%%%%%%%%%%%%%%%%%%%%%%%%
\subsection{Copyright}

Copyright \copyright{} 2017--2018 Niklas Beisert

This work may be distributed and/or modified under the
conditions of the \LaTeX{} Project Public License, either version 1.3
of this license or (at your option) any later version.
The latest version of this license is in
  \url{http://www.latex-project.org/lppl.txt}
and version 1.3 or later is part of all distributions of \LaTeX{}
version 2005/12/01 or later.

This work has the LPPL maintenance status `maintained'.

The Current Maintainer of this work is Niklas Beisert.

This work consists of the files |README.txt|, |childdoc.ins| and |childdoc.dtx|
as well as the derived files |childdoc.def|, |cdocsamp.tex|
with |cdocsch1.tex|, |cdocsch2.tex|, |cdocspt3.tex|, |cdocspt4.tex|,
|cdocsdrf.tex|, |cdocsfn1.tex|, |cdocsfn2.tex|
as well as |childdoc.pdf|.

%%%%%%%%%%%%%%%%%%%%%%%%%%%%%%%%%%%%%%%%%%%%%%%%%%%%%%%%%%%%%%%%%%%%%%%%%%%%%%%%
\subsection{Files and Installation}

The package consists of the files:
%
\begin{center}
\begin{tabular}{ll}
    |README.txt|   & readme file \\
    |childdoc.ins| & installation file \\
    |childdoc.dtx| & source file \\
    |childdoc.def| & definition file \\
    |cdocsamp.tex| & sample main file \\
    |cdocsch1.tex| & sample include file \\
    |cdocsch2.tex| & sample include file \\
    |cdocspt3.tex| & sample part file \\
    |cdocspt4.tex| & sample part file \\
    |cdocsdrf.tex| & sample redirection file \\
    |cdocsfn1.tex| & sample redirection file \\
    |cdocsfn2.tex| & sample redirection file \\
    |childdoc.pdf| & manual
\end{tabular}
\end{center}
%
The distribution consists of the files
|README.txt|, |childdoc.ins| and |childdoc.dtx|.
%
\begin{itemize}
\item
Run (pdf)\LaTeX{} on |childdoc.dtx|
to compile the manual |childdoc.pdf| (this file).
\item
Run \LaTeX{} on |childdoc.ins| to create the definitions file |childdoc.def|
and the sample |cdocsamp.tex| with include files
|cdocsch1.tex|, |cdocsch2.tex|, |cdocspt3.tex|, |cdocspt4.tex|,
|cdocsdrf.tex|, |cdocsfn1.tex|, |cdocsfn2.tex|.
Then copy the file |childdoc.def| to an appropriate directory of your \LaTeX{}
distribution, e.g.\ \textit{texmf-root}|/tex/latex/childdoc|.
\end{itemize}

%%%%%%%%%%%%%%%%%%%%%%%%%%%%%%%%%%%%%%%%%%%%%%%%%%%%%%%%%%%%%%%%%%%%%%%%%%%%%%%%
\subsection{Related CTAN Packages}

There are several other packages which offer a similar functionality:
%
\begin{itemize}
\item
The packages
\href{http://ctan.org/pkg/docmute}{\textsf{docmute}},
\href{http://ctan.org/pkg/includex}{\textsf{includex}} and
\href{http://ctan.org/pkg/standalone}{\textsf{standalone}}
provide commands to include only the document body of
a child file thus allowing both files to be compiled individually.
\item
The packages \href{http://ctan.org/pkg/subdocs}{\textsf{subdocs}}
and \href{http://ctan.org/pkg/subfiles}{\textsf{subfiles}}
provide structures in which the main and child documents can be
encapsulated and allowing them to be compiled individually.
The inclusion mechanism is different from the conventional |\include|.
\item
The package \href{http://ctan.org/pkg/combine}{\textsf{combine}}
is an elaborate solution to combine several documents into one.
\end{itemize}
%
See also the CTAN topic \href{http://ctan.org/topic/subdocs}{\textsf{subdocs}}
for further related packages.
The present package differs from the above solutions in that
a document structure constructed with the conventional |\include| mechanism
just needs two extra commands at the top of every file
such that all constituent files can be compiled individually.

%%%%%%%%%%%%%%%%%%%%%%%%%%%%%%%%%%%%%%%%%%%%%%%%%%%%%%%%%%%%%%%%%%%%%%%%%%%%%%%%
%\subsection{Feature Suggestions}
%
%The following is a list of features which may be useful for future
%versions of this package:
%%
%\begin{itemize}
%\item
%\ldots
%\end{itemize}

%%%%%%%%%%%%%%%%%%%%%%%%%%%%%%%%%%%%%%%%%%%%%%%%%%%%%%%%%%%%%%%%%%%%%%%%%%%%%%%%
\subsection{Revision History}

%%%%%%%%%%%%%%%%%%%%%%%%%%%%%%%%%%%%%%%%
\paragraph{v2.0:} 2018/12/30

\begin{itemize}
\item
immediate forward processing
\item
added |\childdocby| mechanism
\item
manual restructured
\end{itemize}

%%%%%%%%%%%%%%%%%%%%%%%%%%%%%%%%%%%%%%%%
\paragraph{v1.6:} 2018/01/17

\begin{itemize}
\item
application for development of include files
\item
corrections to manual
\end{itemize}

%%%%%%%%%%%%%%%%%%%%%%%%%%%%%%%%%%%%%%%%
\paragraph{v1.5:} 2017/05/21

\begin{itemize}
\item
more complete structuring introduced
\item
|\childdocof| introduced
\item
|\childdoc| renamed to |\childdocmain|
\item
|\childredirect| renamed to |\childdocforward| and |\childdocforwardprefix|
and functionality expanded
\end{itemize}

%%%%%%%%%%%%%%%%%%%%%%%%%%%%%%%%%%%%%%%%
\paragraph{v1.0:} 2017/04/27

\begin{itemize}
\item
manual and install package
\item
first version published on CTAN
\end{itemize}

%%%%%%%%%%%%%%%%%%%%%%%%%%%%%%%%%%%%%%%%
\paragraph{v0.6:} 2017/04/26

\begin{itemize}
\item
redirection mechanism added
\end{itemize}

%%%%%%%%%%%%%%%%%%%%%%%%%%%%%%%%%%%%%%%%
\paragraph{v0.5:} 2017/04/26

\begin{itemize}
\item
functionality in definition file
\end{itemize}


%%%%%%%%%%%%%%%%%%%%%%%%%%%%%%%%%%%%%%%%%%%%%%%%%%%%%%%%%%%%%%%%%%%%%%%%%%%%%%%%
%%%%%%%%%%%%%%%%%%%%%%%%%%%%%%%%%%%%%%%%%%%%%%%%%%%%%%%%%%%%%%%%%%%%%%%%%%%%%%%%
%%%%%%%%%%%%%%%%%%%%%%%%%%%%%%%%%%%%%%%%%%%%%%%%%%%%%%%%%%%%%%%%%%%%%%%%%%%%%%%%
\appendix

\settowidth\MacroIndent{\rmfamily\scriptsize 000\ }

 \DocInput{childdoc.dtx}

\end{document}
%</driver>
% \fi
%
% %%%%%%%%%%%%%%%%%%%%%%%%%%%%%%%%%%%%%%%%%%%%%%%%%%%%%%%%%%%%%%%%%%%%%%%%%%%%%%
% %%%%%%%%%%%%%%%%%%%%%%%%%%%%%%%%%%%%%%%%%%%%%%%%%%%%%%%%%%%%%%%%%%%%%%%%%%%%%%
% \section{Sample}
%\iffalse
%<*samplemain>
%\fi
%
% The following presents a sample document
% with two chapters, two parts, a title page,
% a compile flag as well as three forwarding files to set the flag.
% It consists of eight |.tex| files:
% \begin{center}
% \begin{tabular}{ll}
% |cdocsamp.tex|&main file\\
% |cdocsch1.tex|&include file for chapter 1\\
% |cdocsch2.tex|&include file for chapter 2\\
% |cdocspt3.tex|&include file for part 3\\
% |cdocspt4.tex|&include file for part 4\\
% |cdocsdrf.tex|&forwarding file for main file in draft mode\\
% |cdocsfi1.tex|&forwarding file for final version of chapter 1\\
% |cdocsfi2.tex|&forwarding file for final version of chapter 2\\
% \end{tabular}
% \end{center}
% Each of the eight files can be compiled directly by the \LaTeX{} compiler.
%
% %%%%%%%%%%%%%%%%%%%%%%%%%%%%%%%%%%%%%%
% \paragraph{Main File.}
%
% The main file is called |cdocsamp.tex|.
%
% Load the \textsf{childdoc} definitions and
% declare the filename for the main document:
%    \begin{macrocode}
\input{childdoc.def}
\childdocmain{}
%    \end{macrocode}

% Optional override for |\version| flag:
%    \begin{macrocode}
%%\ifchilddoc\else\providecommand{\version}{draft}\fi
%    \end{macrocode}

% Define the default values for the |\version| flag
% (|final| for the main file and |draft| for childs):
%    \begin{macrocode}
\ifchilddoc
\providecommand{\version}{draft}
\else
\providecommand{\version}{final}
\fi
%    \end{macrocode}

% Load the standard document class:
%    \begin{macrocode}
\documentclass[12pt]{article}
%    \end{macrocode}

% Start the document body:
%    \begin{macrocode}
\begin{document}
%    \end{macrocode}

% Declare a title page.
% Print title, part of document being processed and version flag:
%    \begin{macrocode}
\addtocounter{page}{-1}
\begin{center}
{\LARGE\bfseries{}childdoc example\par}
\vspace{1cm}
\ifchilddoc
\ifchilddocmanual part\else chapter\fi:
`\childdocname' of `\childdocjob'\par
\else
main document: `\childdocjob'\par
\fi
version: \version\par
\end{center}
\newpage
%    \end{macrocode}

% Manually include selected file,
% otherwise process as usual:
%    \begin{macrocode}
\ifchilddocmanual
\section*{part `\childdocname'}
\input{\childdocname}
\else
%    \end{macrocode}

% Include the two chapters:
%    \begin{macrocode}
\include{cdocsch1}
\include{cdocsch2}
%    \end{macrocode}

% Include the two parts unless only chapters should be displayed:
%    \begin{macrocode}
\ifchilddoc\else
\section{part three}
\input{cdocspt3}
\section{part four}
\input{cdocspt4}
\fi
%    \end{macrocode}

% Process as usual until here:
%    \begin{macrocode}
\fi
%    \end{macrocode}

% End of document body:
%    \begin{macrocode}
\end{document}
%    \end{macrocode}
%\iffalse
%</samplemain>
%\fi
%
% %%%%%%%%%%%%%%%%%%%%%%%%%%%%%%%%%%%%%%
% \paragraph{Chapter Include Files.}
%
% The include files are called |cdocsch1.tex| and |cdocsch2.tex|.
%
%\iffalse
%<*samplechap1|samplechap2>
%\fi

% Optional override for |\version| flag:
%    \begin{macrocode}
%%\providecommand{\version}{final}
%    \end{macrocode}

% Include the main document:
%    \begin{macrocode}
\input{childdoc.def}
\childdocof{cdocsamp}
%    \end{macrocode}

%\iffalse
%</samplechap1|samplechap2>
%\fi
%
%\iffalse
%<*samplechap1>
%\fi
% Some text for chapter 1:
%    \begin{macrocode}
\section{one}
some text in chapter one
%    \end{macrocode}

%\iffalse
%</samplechap1>
%\fi
% Some text for chapter 2:
%\iffalse
%<*samplechap2>
%\fi
%    \begin{macrocode}
\section{two}
more text in chapter two
%    \end{macrocode}

%\iffalse
%</samplechap2>
%\fi
%
% %%%%%%%%%%%%%%%%%%%%%%%%%%%%%%%%%%%%%%
% \paragraph{Part Include Files.}
%
% The include files are called |cdocspt3.tex| and |cdocspt4.tex|.
%
%\iffalse
%<*samplepart3|samplepart4>
%\fi

% Optional override for |\version| flag:
%    \begin{macrocode}
%%\providecommand{\version}{final}
%    \end{macrocode}

% Include the main document:
%    \begin{macrocode}
\input{childdoc.def}
\childdocby{cdocsamp}
%    \end{macrocode}

%\iffalse
%</samplepart3|samplepart4>
%\fi
%
%\iffalse
%<*samplepart3>
%\fi
% Some text for part 3:
%    \begin{macrocode}
some text in part three
%    \end{macrocode}

%\iffalse
%</samplepart3>
%\fi
% Some text for part 4:
%\iffalse
%<*samplepart4>
%\fi
%    \begin{macrocode}
more text in part four
%    \end{macrocode}

%\iffalse
%</samplepart4>
%\fi
%
% %%%%%%%%%%%%%%%%%%%%%%%%%%%%%%%%%%%%%%
% \paragraph{Forwarding for a Complete Draft.}
%
% The following forwarding file |cdocsdrf.tex|
% compiles the main document in draft mode:
%\iffalse
%<*sampledraft>
%\fi
%    \begin{macrocode}
\def\version{draft}
\input{childdoc.def}
\childdocforward{cdocsamp}
%    \end{macrocode}

%\iffalse
%</sampledraft>
%\fi
%
% %%%%%%%%%%%%%%%%%%%%%%%%%%%%%%%%%%%%%%
% \paragraph{Forwarding for Final Version of the Chapters.}
%
% The following forwarding files |cdocsfn1.tex| and |cdocsfn2.tex|
% (with identical content)
% compile the final versions of the child documents
% |cdocsch1.tex| and |cdocsch2.tex|, respectively:
%\iffalse
%<*samplefinal>
%\fi
%    \begin{macrocode}
\def\version{final}
\input{childdoc.def}
\childdocforwardprefix[cdocsamp]{cdocsfn}{cdocsch}
%    \end{macrocode}

%\iffalse
%</samplefinal>
%\fi
%
% %%%%%%%%%%%%%%%%%%%%%%%%%%%%%%%%%%%%%%
% \paragraph{Command Line Processing.}
%
% The following three command lines generate the output files
% |cdocscld|, |cdocscl1| and |cdocscl2|
% which should be identical to
% |cdocsdrf|, |cdocsch1| and |cdocsfn2|, respectively:
% \begin{center}
% \begin{tabular}{l}
% |latex -jobname cdocscld \|\\
% |  "\def\version{draft}\input{childdoc.def}\childdocforward{cdocsamp}"|\\
% |latex -jobname cdocscl1 \|\\
% |  "\input{childdoc.def}\childdocforward[cdocsamp]{cdocsch1}"|\\
% |latex -jobname cdocscl2 \|\\
% |  "\def\version{final}\input{childdoc.def}\childdocforward{cdocsch2}"|
% \end{tabular}
% \end{center}
% Note that the trailing backslash on each first line
% merely continues the input to the second line
% (for convenient cut ant paste).
% Furthermore, the command |latex| can be replaced by any
% of its alternative versions such as |pdflatex|.
%
% %%%%%%%%%%%%%%%%%%%%%%%%%%%%%%%%%%%%%%%%%%%%%%%%%%%%%%%%%%%%%%%%%%%%%%%%%%%%%%
% %%%%%%%%%%%%%%%%%%%%%%%%%%%%%%%%%%%%%%%%%%%%%%%%%%%%%%%%%%%%%%%%%%%%%%%%%%%%%%
% \section{Implementation}
%\iffalse
%<*package>
%\fi
%
% This section describes the definitions file |childdoc.def|.

% The definitions cannot be loaded using |\usepackage| or |\RequirePackage|
% which has a mechanism to prevent loading a style file more than once.
% When loading the definitions by means of |\input|
% multiple instances have to be prevented manually:
%\iffalse
%This code needs to be before the `\ProvidesFile' directive
%which is defined at the beginning of this file.
%Therefore it is also placed there and commented out here.
%</package>
%<*discard>
%\fi
%    \begin{macrocode}
\ifdefined\childdocmain\endinput\fi
%    \end{macrocode}
%\iffalse
%</discard>
%<*package>
%\fi
%
% \macro{\ifchilddoc}
% \macro{\ifchilddocmanual}
% The conditional |\ifchilddoc| tells whether a
% child (true) or main (false) document is being compiled.
% The conditional |\ifchilddocmanual| tells whether
% the |\includeonly| mechanism is used (false) or
% the selection of child files must be performed manually (true).
% The definitions initialise to false:
%    \begin{macrocode}
\newif\ifchilddoc
\newif\ifchilddocmanual
%    \end{macrocode}

% \macro{\childdocname}
% \macro{\childdocjob}
% The macro |\childdocname| stores the name of the main document
% to be compiled. The macro |\childdocjob| stores the name of
% the document on which the \LaTeX{} compiler was originally invoked.
% The content of |\jobname| cannot be compared
% to filenames specified in the source due to different catcodes.
% The following code rescans |\jobname|, stores the result
% in |\childdocname| and saves a copy in |\childdocjob|:
%    \begin{macrocode}
\edef\childdocname{\scantokens\expandafter{\jobname\noexpand}}
\let\childdocjob\childdocname
%    \end{macrocode}

% \macro{\childdocdisable}
% The macro |\childdocdisable| prevents the main file
% from being processed more than once.
% At this stage, the main document command |\childdocmain|
% is assumed to be called once again where it should do nothing.
% Any subsequent call to it should prevent
% a secondary processing of the main document
% It overwrites the forwarding commands
% |\childdocof| and |\childdocforward|
% with empty macros to prevent further inclusions of the main document:
%    \begin{macrocode}
\newcommand{\childdocdisable}
{
  \renewcommand{\childdocmain}[1]{\renewcommand{\childdocmain}[1]{\endinput}}
  \renewcommand{\childdocof}[1]{}
  \renewcommand{\childdocby}[2][]{}
  \renewcommand{\childdocforward}[2][]{}
  \renewcommand{\childdocdisable}{}
}
%    \end{macrocode}

% \macro{\childdocmain}
% The macro |\childdocmain| is to be called at the top of the main file
% with nothing or the main filename (without extension) as argument.
% First, it breaks loops.
% If the argument is not empty and does not match |\childdocname|
% (which is set by the first inclusion of |childdoc.def|),
% |\ifchilddoc| is set to true, |\includeonly| is applied to the child file
% and |\jobname| is set to the main file
% (for proper handling of |.aux| files):
%    \begin{macrocode}
\newcommand{\childdocmain}[1]
{
  \childdocdisable\childdocmain{}
  \if?#1?\else
    \begingroup
      \def\childdoctmp{#1}
      \ifx\childdoctmp\childdocname
        \def\childdoctmp{}
      \else
        \def\childdoctmp
        {
          \childdoctrue
          \includeonly{\childdocname}
          \def\childdocjob{#1}
          \def\jobname{#1}
        }
      \fi
      \expandafter
    \endgroup
    \childdoctmp
  \fi
}
%    \end{macrocode}

% \macro{\childdocof}
% The command |\childdocof| redirects
% compilation to the main file |#1|.
%    \begin{macrocode}
\newcommand{\childdocof}[1]
{
  \childdocdisable
  \childdoctrue
  \includeonly{\childdocname}
  \def\jobname{#1}
  \def\childdocjob{#1}
  \input{#1}
}
%    \end{macrocode}

% \macro{\childdocby}
% The command |\childdocby| ....
%    \begin{macrocode}
\newcommand{\childdocby}[2][]
{
  \childdocdisable
  \childdoctrue
  \childdocmanualtrue
  \if?#1?\else
    \def\jobname{#2}
  \fi
  \def\childdocjob{#2}
  \input{#2}
  \endinput
}
%    \end{macrocode}

% \macro{\childdocforward}
% The command |\childdocforward| redirects
% compilation to the main file or
% (if the optional argument is given) a child file.
% Parameters are set as if the main file
% or a child file starting with |\childdocof| was compiled.
% Then compilation is handed over to the main file:
%    \begin{macrocode}
\newcommand{\childdocforward}[2][]
{
  \begingroup
    \if?#1?
      \def\childdoctmp
      {
        \def\childdocname{#2}
        \def\childdocjob{#2}
        \def\jobname{#2}
        \input{#2}
        \endinput
      }
    \else
      \def\childdoctmp
      {
        \childdocdisable
        \def\childdocname{#2}
        \childdoctrue
        \includeonly{#2}
        \def\childdocjob{#1}
        \def\jobname{#1}
        \input{#1}
        \endinput
      }
    \fi
    \expandafter
  \endgroup
  \childdoctmp
}
%    \end{macrocode}

% \macro{\childdocforwardprefix}
% The command |\childdocforwardprefix| redirects
% compilation to the main or a child file by means of a pattern.
% The prefix |#1| in the current filename is replaced by |#2|
% and the suffix of the current filename is kept
% (it is assumed that the filename does not contain the substring `|~~~|'
% which is used as a delimiter).
% Compilation is handed over to the new file by |\childdocforward|:
%    \begin{macrocode}
\newcommand{\childdocforwardprefix}[3][]
{
  \begingroup
    \def\childdocextract #2##1~~~{\def\childdoctmp{\childdocforward[#1]{#3##1}}}
    \expandafter\childdocextract\childdocname~~~
    \expandafter
  \endgroup
  \childdoctmp
}
%    \end{macrocode}

% \macro{\childdoc}
% The deprecated macro |\childdoc| is a legacy version of |\childdocmain|:
%    \begin{macrocode}
\newcommand{\childdoc}{\childdocmain}
%    \end{macrocode}

% \macro{\childdocredirect}
% The deprecated macro |\childdocredirect| is a legacy version
% of |\childdocforward| and |\childdocforwardprefix|:
%    \begin{macrocode}
\newcommand{\childdocredirect}[2][]
{
  \begingroup
    \if?#1?
      \def\childdoctmp{\childdocforward{#2}}
    \else
      \def\childdoctmp{\childdocforwardprefix{#1}{#2}}
    \fi
    \expandafter
  \endgroup
  \childdoctmp
}
%    \end{macrocode}

%\iffalse
%</package>
%\fi
%
\endinput
\childdocforward[cdocsamp]{cdocsch1}"|\\
% |latex -jobname cdocscl2 \|\\
% |  "\def\version{final}% \iffalse
%
% childdoc.dtx Copyright (C) 2017-2018 Niklas Beisert
%
% This work may be distributed and/or modified under the
% conditions of the LaTeX Project Public License, either version 1.3
% of this license or (at your option) any later version.
% The latest version of this license is in
%   http://www.latex-project.org/lppl.txt
% and version 1.3 or later is part of all distributions of LaTeX
% version 2005/12/01 or later.
%
% This work has the LPPL maintenance status `maintained'.
%
% The Current Maintainer of this work is Niklas Beisert.
%
% This work consists of the files childdoc.dtx and childdoc.ins
% and the derived files childdoc.def and cdocsamp.tex with
% cdocsch1.tex, cdocsch2.tex, cdocsdrf.tex, cdocsfn1.tex, cdocsfn2.tex.
%
%<package>\ifdefined\childdocmain\endinput\fi
%<package>\ProvidesFile{childdoc.def}[2018/12/30 v2.0 child document driver]
%<samplemain>\ProvidesFile{cdocsamp.tex}[2018/12/30 v2.0 sample for childdoc]
%<*driver>
%\ProvidesFile{childdoc.drv}[2018/12/30 v2.0 childdoc reference manual file]
\PassOptionsToClass{10pt,a4paper}{article}
\documentclass{ltxdoc}

\usepackage[margin=35mm]{geometry}
\usepackage{hyperref}
\usepackage{hyperxmp}
\usepackage[usenames]{color}

\hypersetup{colorlinks=true}
\hypersetup{pdfstartview=FitH}
\hypersetup{pdfpagemode=UseNone}
\hypersetup{pdfsource={}}
\hypersetup{pdflang={en-UK}}
\hypersetup{pdfcopyright={Copyright 2017-2018 Niklas Beisert.
  This work may be distributed and/or modified under the
  conditions of the LaTeX Project Public License, either version 1.3
  of this license or (at your option) any later version.}}
\hypersetup{pdflicenseurl={http://www.latex-project.org/lppl.txt}}
\hypersetup{pdfcontactaddress={ETH Zurich, ITP, HIT K,
  Wolfgang-Pauli-Strasse 27}}
\hypersetup{pdfcontactpostcode={8093}}
\hypersetup{pdfcontactcity={Zurich}}
\hypersetup{pdfcontactcountry={Switzerland}}
\hypersetup{pdfcontactemail={nbeisert@itp.phys.ethz.ch}}
\hypersetup{pdfcontacturl={http://people.phys.ethz.ch/\xmptilde nbeisert/}}

\newcommand{\secref}[1]{\hyperref[#1]{section \ref*{#1}}}

\parskip1ex
\parindent0pt
\let\olditemize\itemize
\def\itemize{\olditemize\parskip0pt}

\begin{document}

\title{The \textsf{childdoc} Package}
\hypersetup{pdftitle={The childdoc Package}}
\author{Niklas Beisert\\[2ex]
  Institut f\"ur Theoretische Physik\\
  Eidgen\"ossische Technische Hochschule Z\"urich\\
  Wolfgang-Pauli-Strasse 27, 8093 Z\"urich, Switzerland\\[1ex]
  \href{mailto:nbeisert@itp.phys.ethz.ch}
  {\texttt{nbeisert@itp.phys.ethz.ch}}}
\hypersetup{pdfauthor={Niklas Beisert}}
\hypersetup{pdfsubject={Manual for the LaTeX2e Package childdoc}}
\date{30 December 2018, \textsf{v2.0}}
\maketitle

\begin{abstract}\noindent
\textsf{childdoc} is a \LaTeXe{} package
that enables the direct compilation
of document sections included by |\include|
to individual files.
\end{abstract}

\begingroup
\parskip0ex
\tableofcontents
\endgroup

%%%%%%%%%%%%%%%%%%%%%%%%%%%%%%%%%%%%%%%%%%%%%%%%%%%%%%%%%%%%%%%%%%%%%%%%%%%%%%%%
%%%%%%%%%%%%%%%%%%%%%%%%%%%%%%%%%%%%%%%%%%%%%%%%%%%%%%%%%%%%%%%%%%%%%%%%%%%%%%%%
\section{Introduction}

\LaTeX{} provides a mechanism to structure a large document (such as a book)
into a main file and several child files (containing the chapters)
using the |\include| command.
This mechanism is beneficial for documents
which span hundreds of pages in order to
make the source file(s) more manageable.
Moreover, compilation can be restricted to
selected child files by means of the |\includeonly| command.
The latter feature can be used to reduce the compilation time while editing
(this was significantly more useful in the earlier days of \LaTeX{})
or to generate a smaller document which is easier to navigate.
Another application of |\includeonly| is to generate
documents consisting of selected parts of the complete document.

However, there are a few drawbacks of the plain |\include| mechanism:
\begin{itemize}
\item
The child files cannot be compiled on their own,
they can only be compiled via the main file.
A naive editing environment
(such as a text editor with an option
to have the current file processed by \LaTeX)
may require one to switch to the main file before compiling;
attempting to compile the child file produces errors.
\item
The main file must be modified (each time)
to adjust the |\includeonly| command
to the present needs. This easily leaves the main file in a messy state.
\item
The generated document will always carry the filename
of the main document. This is inconvenient if
several child files are to be compiled and
to be kept for distribution.
\end{itemize}

The present package provides a simple interface
to make child files individually compilable by \LaTeX{}.
Compiling a child file then has the same effect as compiling
the main file with an |\includeonly| command
to select the appropriate child.
Moreover the generated document will carry the name of the child
rather than the main file.
This resolves all three above issues.

This feature is meant to make the editing of books,
thesis documents and lecture notes somewhat more convenient.
However, the package can also be used efficiently for
composing a series of documents (such as exercise sheets)
which are typically distributed individually.
It then assists the author in generating the individual documents
(potentially in different versions)
as well as a document containing the collected series.
Another application is in developing style files
or other kinds of included material
where compilation of the style file could redirect
to a sample or test file.

%%%%%%%%%%%%%%%%%%%%%%%%%%%%%%%%%%%%%%%%%%%%%%%%%%%%%%%%%%%%%%%%%%%%%%%%%%%%%%%%
%%%%%%%%%%%%%%%%%%%%%%%%%%%%%%%%%%%%%%%%%%%%%%%%%%%%%%%%%%%%%%%%%%%%%%%%%%%%%%%%
\section{Usage}

First of all, the package \textsf{childdoc} is \emph{not} a standard
\LaTeXe{} |.sty| style file! Therefore it needs to be invoked in
a non-standard way.

%%%%%%%%%%%%%%%%%%%%%%%%%%%%%%%%%%%%%%%%%%%%%%%%%%%%%%%%%%%%%%%%%%%%%%%%%%%%%%%%
\subsection{Included Files}
\label{sec:include}

%%%%%%%%%%%%%%%%%%%%%%%%%%%%%%%%%%%%%%%%
\DescribeMacro{\childdocmain}
To use the package, add the commands
\begin{center}
\begin{tabular}{l}
|\input{childdoc.def}|\\
|\childdocmain{}|\\
\end{tabular}
\end{center}
at the very top of the main \LaTeX{} file,
in particular \emph{before} the |\documentclass| statement!
The argument of |\childdocmain| should be left empty
(but it must be present).

%%%%%%%%%%%%%%%%%%%%%%%%%%%%%%%%%%%%%%%%
\DescribeMacro{\childdocof}
Furthermore, add the commands
\begin{center}
\begin{tabular}{l}
|\input{childdoc.def}|\\
|\childdocof{|\textit{main}|}|\\
\end{tabular}
\end{center}
at the top of every child file \textit{child}
which is included by |\include{|\textit{child}|}|
from within the main file
(or at least for those files to be compiled individually).
The argument \textit{main} must be the filename of the main file.

There are a couple of
considerations in setting up the main and child documents:

%%%%%%%%%%%%%%%%%%%%%%%%%%%%%%%%%%%%%%%%
\paragraph{Restrictions.}

Please note the following restrictions:
\begin{itemize}
\item
|\childdocmain| must be called with one argument \textit{main}
to ensure compatibility with earlier version of the package.
It must either be empty (|\childdocmain{}|)
or precisely match the filename of the main file in which it is specified.
See \secref{sec:detection} for further information.
\item
The filename \textit{main} must be specified without the |.tex| extension.
\item
The filename \textit{main} is case sensitive
(even in case-insensitive file systems)
due to internal string comparison.
\item
The argument \textit{main} should be fully expanded, it cannot be a macro.
\item
Subdirectories and special characters should be avoided in filenames.
\item
The command |\childdocmain{|\textit{main}|}| must be followed by a whitespace.
It should not be followed immediately by another command
or by a comment mark `|%|'.
This is because the \TeX{} parser reads the token immediately following
the argument of |\childdocmain| and puts it
at the beginning of every child section;
however, a white\-space is ignored.
\end{itemize}

%%%%%%%%%%%%%%%%%%%%%%%%%%%%%%%%%%%%%%%%
\paragraph{Content of Main File.}

It is advisable to place all content in the child files included by |\include|.
Any output contained in the main file will appear in all child documents
unless suppressed manually;
it cannot be suppressed automatically by the |\includeonly| directive
and thus should normally be avoided.
A method to include some content in the main file
by means of conditional processing is described in \secref{sec:conditional}.

%%%%%%%%%%%%%%%%%%%%%%%%%%%%%%%%%%%%%%%%
\paragraph{Page Numbering.}

When only a part of the document is compiled,
the appropriate numbering of pages
(as well as other status parameters)
is determined from the |.aux| files.
The latter contain information from previous passes.
However this information needs to propagate through
all intermediate child documents.
Therefore the page numbering in child documents may well
be inconsistent until the complete document is compiled at least once.

A useful (if unconventional) way to always ensure a consistent
page numbering is to restart the numbering in each child document
and denote the pages by `\textit{child}|.|\textit{page}'
where \textit{child} represents the chapter/section number of the child file.
This can be achieved by the command
|\numberwithin{page}{|\textit{child}|}|
of the \textsf{amsmath} package
where \textit{child} can be |chapter| or |section|
depending on the chosen structuring.
Alternatively, one can modify the macro |\thepage| appropriately
and reset the counter |page| at the start of each child file.

%%%%%%%%%%%%%%%%%%%%%%%%%%%%%%%%%%%%%%%%%%%%%%%%%%%%%%%%%%%%%%%%%%%%%%%%%%%%%%%%
\subsection{Conditional Processing}
\label{sec:conditional}

The package provides a mechanism to compile different versions
of a document. To customise the versions further some conditional processing
can come in handy to distinguish which version is being compiled.
The package provides two macros to describe the compilation context:

%%%%%%%%%%%%%%%%%%%%%%%%%%%%%%%%%%%%%%%%
\DescribeMacro{\ifchilddoc}
The conditional |\ifchilddoc| distinguishes between the compilation of
child documents and the main document:
%
\begin{center}
|\ifchilddoc |\textit{child-code}| |[|\||else |\textit{main-code}]| \||fi|
\end{center}

%%%%%%%%%%%%%%%%%%%%%%%%%%%%%%%%%%%%%%%%
\DescribeMacro{\childdocname}
\DescribeMacro{\childdocjob}
The macro |\childdocname| contains the filename (without extension)
of the main or child file being processed.
Note that |\childdocjob| will always contain the name of the main file.

%%%%%%%%%%%%%%%%%%%%%%%%%%%%%%%%%%%%%%%%
\paragraph{Title Page.}

Conditional processing can be used to include a title or banner page
in the main document when proper precautions are taken.
Importantly, the code in the main file should ensure that the page counter
(as well as other status parameters which are stored in the |.aux| files)
takes the same value after the conditional processing.
Otherwise the page numbers may take divergent values
depending on which part is compiled.

For example, a title page could be declared by:
%
\begin{center}
\begin{tabular}{l}
|\ifchilddoc\||else|\\
|\addtocounter{page}{-1}|\\
\textit{code for title page}\\
|\newpage|\\
|\||fi|
\end{tabular}
\end{center}
%
A banner page for the child documents can be generated by:
%
\begin{center}
\begin{tabular}{l}
|\ifchilddoc|\\
|\addtocounter{page}{-1}|\\
\textit{code for banner page}\\
|\newpage|\\
|\||fi|
\end{tabular}
\end{center}
%
Here one could write a message such as:
\begin{center}
|This is the part \childdocname{} of \childdocjob{}.|
\end{center}

%%%%%%%%%%%%%%%%%%%%%%%%%%%%%%%%%%%%%%%%%%%%%%%%%%%%%%%%%%%%%%%%%%%%%%%%%%%%%%%%
\subsection{Flags}
\label{sec:flags}

The package makes it easy to generate different versions
of the main or child documents.
To this end compilation flags can be defined
and assigned different default values.
They will be particularly useful in conjunction
with the forwarding mechanism described in \secref{sec:forward}.

For example, it may be useful to have a flag |\version|
which can be set to |draft| or |final|.
The document source will contain some conditional code
depending on the value of |\version|.
Suppose further, the flag should default to |final| for the main file
and to |draft| for child files
which is a natural assignment for editing the document.
This is achieved by placing the following code
in the preamble of the main document
(below the |\childdocmain| directive):
%
\begin{center}
\begin{tabular}{l}
|\ifchilddoc|\\
|\providecommand{\version}{draft}|\\
|\||else|\\
|\providecommand{\version}{final}|\\
|\||fi|
\end{tabular}
\end{center}
%
The definition by |\providecommand| makes sure
that previous definitions are not overwritten.
Further statements |\providecommand{\version}{...}|
can thus be added before the above code to override it.

For the main file, one might add a line
(between |\childdocmain| and the above block)
%
\begin{center}
|%\ifchilddoc\||else\providecommand{\version}{draft}\||fi|
\end{center}
%
which can be uncommented to produce a draft version.
Likewise one can add a line to the very top of a child file
(above the |\childdocof{|\textit{main}|}| directive)
%
\begin{center}
|%\providecommand{\version}{final}|
\end{center}
%
which can be uncommented to produce the final version of this child document.

%%%%%%%%%%%%%%%%%%%%%%%%%%%%%%%%%%%%%%%%%%%%%%%%%%%%%%%%%%%%%%%%%%%%%%%%%%%%%%%%
\subsection{Forwarding}
\label{sec:forward}

Different versions of the main or child documents
using compilation flags as described in \secref{sec:flags}
can be (permanently) stored in different files
for convenient compilation, viewing and distribution.
To this end, the package defines a command
to pass on compilation to a different file:

%%%%%%%%%%%%%%%%%%%%%%%%%%%%%%%%%%%%%%%%
\DescribeMacro{\childdocforward}
The command |\childdocforward| redirects processing to
another source file:
%
\begin{center}
\begin{tabular}{l}
|\input{childdoc.def}|\\
|\childdocforward[|\textit{main}|]{|\textit{dest}|}|\\
\end{tabular}
\end{center}
%
The argument \textit{dest} is the destination file
(without extension).
It should be the main file or one of the child files.
Note that further \textsf{childdoc} directives
such as |\childdocof| and |\childdocforward|
in the indicated file will be processed in this form.
The optional argument \textit{main}
passes on directly to the main file \textit{main}
while pretending to compile the child \textit{dest}.
This form behaves as if \textit{dest}
issues |\childdocof{|\textit{main}|}| right away,
and no further \textsf{childdoc} directives will be processed.

%%%%%%%%%%%%%%%%%%%%%%%%%%%%%%%%%%%%%%%%
\DescribeMacro{\...prefix}
In the alternative form |\childdocforwardprefix|,
%
\begin{center}
\begin{tabular}{l}
|\input{childdoc.def}|\\
|\childdocforwardprefix[|\textit{main}|]{|\textit{prefix}|}{|\textit{dest}|}|
\end{tabular}
\end{center}
%
the destination file is determined by a pattern
depending on the current file:
To make this work, the current file must be called
`{\textit{prefix}\hspace{0.2em}\textit{suffix}}'
with \textit{prefix} matching precisely the argument.
Processing is then passed on to the file
`{\textit{dest}\hspace{0.2em}\textit{suffix}}'.
Surely, the same effect is achieved by
directly specifying the
argument `{\textit{dest}\hspace{0.2em}\textit{suffix}}'
in the first form.
However, that requires to set up a different file
for each child. With the alternative form of the command
all these files can have exactly the same content
which simplifies setting them up and maintaining them.

For example, the following file |draft.tex|
with a compilation flag |\version| as described in \secref{sec:flags}
compiles the main document as a draft:
%
\begin{center}
\begin{tabular}{l}
|\def\version{draft}|\\
|\input{childdoc.def}|\\
|\childdocforward{|\textit{main}|}|
\end{tabular}
\end{center}
%
Likewise, the following files |final|\textit{nn}|.tex|
compile the final version of the child document
|child|\textit{nn}|.tex|:
%
\begin{center}
\begin{tabular}{l}
|\def\version{final}|\\
|\input{childdoc.def}|\\
|\childdocforwardprefix{final}{child}|
\end{tabular}
\end{center}
%

Note that when several versions of a main file and/or of each child file
are to be generated, it may be convenient to set up a |Makefile| or
shell script to automatise the process.

%%%%%%%%%%%%%%%%%%%%%%%%%%%%%%%%%%%%%%%%%%%%%%%%%%%%%%%%%%%%%%%%%%%%%%%%%%%%%%%%
\subsection{Command Line Processing}
\label{sec:commandline}

The effect of redirection files can also be achieved by invoking
the \LaTeX{} compiler with a more elaborate command line.
Most conveniently this should be done as part
of a shell script or a |Makefile|.

When using \textsf{childdoc} in the main file, the following
command lines effectively perform a redirection
(note that depending on the shell being used,
backslashes may have to be doubled: `|\|' $\to$ `|\\|'):
%
\begin{center}
|... -jobname "|\textit{target}|" |\\|"|[\textit{flags}]%
|\input{childdoc.def}\childdocforward[|\textit{main}|]{|\textit{dest}|}"|
\end{center}
%
Here \textit{target} is the name of the output file,
\textit{main} is the name of the main file
and \textit{dest} is the name of the main or child file to be processed
(all filenames without extensions).
The optional argument \textit{main} can be omitted
if \textit{main} matches \textit{dest}.
Optionally, compilation \textit{flags} can be defined via |\def| commands.
This command line makes the \TeX{} engine believe
it is compiling the file \textit{target}
whose content is specified as the latter parameter.
The provided code then forwards the processing to
\textit{main} or \textit{dest} as described in \secref{sec:forward}.

%%%%%%%%%%%%%%%%%%%%%%%%%%%%%%%%%%%%%%%%%%%%%%%%%%%%%%%%%%%%%%%%%%%%%%%%%%%%%%%%
\subsection{Include by Input}
\label{sec:input}

Including child documents by |\include| has some restrictions by design.
Most notably, the content of a child document always occupies
its own set of pages; pages cannot be shared between child documents.
Usually, this behaviour makes perfect sense
because each child document contain an essential part of the document.
However, in some situations it may be desirable to compose
a document from a collection of parts
without having mandatory page breaks between then.
For this case, the package
provides a mechanism to include parts
by |\input| which can also be processed individually.
However, by construction this mechanism
requires manual handling of the content to be output.

%%%%%%%%%%%%%%%%%%%%%%%%%%%%%%%%%%%%%%%%
\DescribeMacro{\ifchilddocmanual}
The main file should be prepared as usual, see \secref{sec:include}.
However, the document body must make a distinction
between processing of an individual part and of the main document, e.g.:
%
\begin{center}
\begin{tabular}{l}
|\ifchilddocmanual|\\
|\input{\childdocname}|\\
|\||else|\\
\textit{document body with }|\input{|\textit{part}|}|\\
|\||fi|
\end{tabular}
\end{center}
%
The conditional |\ifchilddocmanual| is true whenever
a part to be included by |\input| is being compiled,
and the name of the part is stored in |\childdocname|.

%%%%%%%%%%%%%%%%%%%%%%%%%%%%%%%%%%%%%%%%
\DescribeMacro{\childdocby}
Each part to be included by |\input| should start with:
%
\begin{center}
\begin{tabular}{l}
|\input{childdoc.def}|\\
|\childdocby{|\textit{main}|}|\\
\end{tabular}
\end{center}
%
The directive |\childdocby| is similar to |\childdocof|
described in \secref{sec:include},
but the subsequent selection of content must be done manually.
To that end, both |\ifchilddoc| and |\ifchilddocmanual|
will be true upon processing of a part,
and the name of the part is stored in |\childdocname|.
Note that |\jobname| will be set to the filename of the current part
so that each part receives an individual |.aux| file
that does not interfere with the |.aux| file(s) of the main document.
This behaviour can be altered by the alternative form
|\childdocby[*]{|\textit{main}|}| (with a non-empty optional argument)
which uses the |.aux| file of the main document
by setting |\jobname| to \textit{main}.

%%%%%%%%%%%%%%%%%%%%%%%%%%%%%%%%%%%%%%%%%%%%%%%%%%%%%%%%%%%%%%%%%%%%%%%%%%%%%%%%
\subsection{Driver Development}
\label{sec:driver}

The \textsf{childdoc} mechanism can also be use for the development
of definition files such as \LaTeX{} styles or classes.
This case differs from the above setup with multiple parts
included by |\include| in that no |\includeonly| should be invoked.
This can be achieved by starting the include file
(before |\ProvidesPackage|) with:
%
\begin{center}
\begin{tabular}{l}
|\input{childdoc.def}|\\
|\childdocforward{|\textit{main}|}|\\
\end{tabular}
\end{center}
%
or alternatively with:
%
\begin{center}
\begin{tabular}{l}
|\input{childdoc.def}|\\
|\childdocby{|\textit{main}|}|\\
\end{tabular}
\end{center}
%
Both forms have slightly different effects as described above.
The main file is prepared as usual, see \secref{sec:include}.

%%%%%%%%%%%%%%%%%%%%%%%%%%%%%%%%%%%%%%%%%%%%%%%%%%%%%%%%%%%%%%%%%%%%%%%%%%%%%%%%
\subsection{Legacy Detection}
\label{sec:detection}

The directive |\childdocmain| in the main file can detect
whether the complete document or merely a child is to be compiled
even without using the directive |\childdocof|.
This method is deprecated because it is less robust
and there is no compelling reason to use it;
it is merely provided for backward compatibility
and it may be removed in future versions.

If the detection mechanism is to be used,
it is mandatory to correctly specify
the filename of the main file as the argument of |\childdocmain|:
%
\begin{center}
\begin{tabular}{l}
|\input{childdoc.def}|\\
|\childdocmain{|\textit{main}|}|\\
\end{tabular}
\end{center}
%
If |\jobname| does not match the argument \textit{main} of |\childdocmain|,
it is assumed that |\jobname| points to the child file to be compiled.
When using |\childdocmain| with the main file specified as argument,
it suffices to start a child file
with just |\input{|\textit{main}|}|
without loading of the package and using |\childdocof|.
If instead all processing is done
with the appropriate \textsf{childdoc} directives,
the argument of \textit{main} of |\childdocmain| can be empty.

An alternative version of the command line processing described
in \secref{sec:commandline} using the detection mechanism reads:
%
\begin{center}
|... -jobname "|\textit{target}|" "|[\textit{flags}]%
[|\def\jobname{|\textit{dest}|}|]|\input{|\textit{main}|}"|
\end{center}

%%%%%%%%%%%%%%%%%%%%%%%%%%%%%%%%%%%%%%%%%%%%%%%%%%%%%%%%%%%%%%%%%%%%%%%%%%%%%%%%
\subsection{Manual Code}
\label{sec:manual}

In case one cannot be certain whether the definitions file |childdoc.def|
is installed on the target \TeX{} distribution
and one prefers not to ship it,
it is conceivable to paste a few relevant commands into the sources.

To that end, drop all statements |\input{childdoc.def}|
and perform the replacements as outlined below.
Instead of |\childdocmain{|\textit{main}|}| add the following code
to the top of the main file:
%
\begin{center}
\begin{tabular}{l}
|\||ifdefined\childdocname\endinput\||fi\newif\ifchilddoc|\\
|\edef\childdocname{\scantokens\expandafter{\jobname\noexpand}}|\\
|\def\childdocmain{|\textit{main}|}\||ifx\childdocmain\childdocname\||else|\\
|\childdoctrue\includeonly{\childdocname}\let\jobname\childdocmain\||fi|\\
\end{tabular}
\end{center}
%
Instead of |\childdocof{|\textit{main}|}| just include the main file
at the top of each child file:
%
\begin{center}
|\input{|\textit{main}|}|
\end{center}
%
A simple redirection |\childdocforward{|\textit{dest}|}| is achieved by:
%
\begin{center}
|\def\jobname{|\textit{dest}|}\input{\jobname}|
\end{center}
%
The redirection with prefix
|\childdocforwardprefix[|\textit{prefix}|]{|\textit{dest}|}|
is accomplished by:
%
\begin{center}
\begin{tabular}{l}
|{\edef\jobname{\scantokens\expandafter{\jobname\noexpand}}|\\
|\def\redirectjob |\textit{prefix}|#1~~~{\gdef\jobname{|\textit{dest}|#1}}|\\
|\expandafter\redirectjob\jobname~~~}\input{\jobname}|
\end{tabular}
\end{center}

In an alternative approach,
child documents can be compiled by a specific command line
without additional code or specific definitions:
%
\begin{center}
|... -jobname "|\textit{target}|" "|[\textit{flags}]%
|\includeonly{|\textit{dest}|}\input{|\textit{main}|}"|
\end{center}
%

%%%%%%%%%%%%%%%%%%%%%%%%%%%%%%%%%%%%%%%%%%%%%%%%%%%%%%%%%%%%%%%%%%%%%%%%%%%%%%%%
%%%%%%%%%%%%%%%%%%%%%%%%%%%%%%%%%%%%%%%%%%%%%%%%%%%%%%%%%%%%%%%%%%%%%%%%%%%%%%%%
\section{Information}

%%%%%%%%%%%%%%%%%%%%%%%%%%%%%%%%%%%%%%%%%%%%%%%%%%%%%%%%%%%%%%%%%%%%%%%%%%%%%%%%
\subsection{Copyright}

Copyright \copyright{} 2017--2018 Niklas Beisert

This work may be distributed and/or modified under the
conditions of the \LaTeX{} Project Public License, either version 1.3
of this license or (at your option) any later version.
The latest version of this license is in
  \url{http://www.latex-project.org/lppl.txt}
and version 1.3 or later is part of all distributions of \LaTeX{}
version 2005/12/01 or later.

This work has the LPPL maintenance status `maintained'.

The Current Maintainer of this work is Niklas Beisert.

This work consists of the files |README.txt|, |childdoc.ins| and |childdoc.dtx|
as well as the derived files |childdoc.def|, |cdocsamp.tex|
with |cdocsch1.tex|, |cdocsch2.tex|, |cdocspt3.tex|, |cdocspt4.tex|,
|cdocsdrf.tex|, |cdocsfn1.tex|, |cdocsfn2.tex|
as well as |childdoc.pdf|.

%%%%%%%%%%%%%%%%%%%%%%%%%%%%%%%%%%%%%%%%%%%%%%%%%%%%%%%%%%%%%%%%%%%%%%%%%%%%%%%%
\subsection{Files and Installation}

The package consists of the files:
%
\begin{center}
\begin{tabular}{ll}
    |README.txt|   & readme file \\
    |childdoc.ins| & installation file \\
    |childdoc.dtx| & source file \\
    |childdoc.def| & definition file \\
    |cdocsamp.tex| & sample main file \\
    |cdocsch1.tex| & sample include file \\
    |cdocsch2.tex| & sample include file \\
    |cdocspt3.tex| & sample part file \\
    |cdocspt4.tex| & sample part file \\
    |cdocsdrf.tex| & sample redirection file \\
    |cdocsfn1.tex| & sample redirection file \\
    |cdocsfn2.tex| & sample redirection file \\
    |childdoc.pdf| & manual
\end{tabular}
\end{center}
%
The distribution consists of the files
|README.txt|, |childdoc.ins| and |childdoc.dtx|.
%
\begin{itemize}
\item
Run (pdf)\LaTeX{} on |childdoc.dtx|
to compile the manual |childdoc.pdf| (this file).
\item
Run \LaTeX{} on |childdoc.ins| to create the definitions file |childdoc.def|
and the sample |cdocsamp.tex| with include files
|cdocsch1.tex|, |cdocsch2.tex|, |cdocspt3.tex|, |cdocspt4.tex|,
|cdocsdrf.tex|, |cdocsfn1.tex|, |cdocsfn2.tex|.
Then copy the file |childdoc.def| to an appropriate directory of your \LaTeX{}
distribution, e.g.\ \textit{texmf-root}|/tex/latex/childdoc|.
\end{itemize}

%%%%%%%%%%%%%%%%%%%%%%%%%%%%%%%%%%%%%%%%%%%%%%%%%%%%%%%%%%%%%%%%%%%%%%%%%%%%%%%%
\subsection{Related CTAN Packages}

There are several other packages which offer a similar functionality:
%
\begin{itemize}
\item
The packages
\href{http://ctan.org/pkg/docmute}{\textsf{docmute}},
\href{http://ctan.org/pkg/includex}{\textsf{includex}} and
\href{http://ctan.org/pkg/standalone}{\textsf{standalone}}
provide commands to include only the document body of
a child file thus allowing both files to be compiled individually.
\item
The packages \href{http://ctan.org/pkg/subdocs}{\textsf{subdocs}}
and \href{http://ctan.org/pkg/subfiles}{\textsf{subfiles}}
provide structures in which the main and child documents can be
encapsulated and allowing them to be compiled individually.
The inclusion mechanism is different from the conventional |\include|.
\item
The package \href{http://ctan.org/pkg/combine}{\textsf{combine}}
is an elaborate solution to combine several documents into one.
\end{itemize}
%
See also the CTAN topic \href{http://ctan.org/topic/subdocs}{\textsf{subdocs}}
for further related packages.
The present package differs from the above solutions in that
a document structure constructed with the conventional |\include| mechanism
just needs two extra commands at the top of every file
such that all constituent files can be compiled individually.

%%%%%%%%%%%%%%%%%%%%%%%%%%%%%%%%%%%%%%%%%%%%%%%%%%%%%%%%%%%%%%%%%%%%%%%%%%%%%%%%
%\subsection{Feature Suggestions}
%
%The following is a list of features which may be useful for future
%versions of this package:
%%
%\begin{itemize}
%\item
%\ldots
%\end{itemize}

%%%%%%%%%%%%%%%%%%%%%%%%%%%%%%%%%%%%%%%%%%%%%%%%%%%%%%%%%%%%%%%%%%%%%%%%%%%%%%%%
\subsection{Revision History}

%%%%%%%%%%%%%%%%%%%%%%%%%%%%%%%%%%%%%%%%
\paragraph{v2.0:} 2018/12/30

\begin{itemize}
\item
immediate forward processing
\item
added |\childdocby| mechanism
\item
manual restructured
\end{itemize}

%%%%%%%%%%%%%%%%%%%%%%%%%%%%%%%%%%%%%%%%
\paragraph{v1.6:} 2018/01/17

\begin{itemize}
\item
application for development of include files
\item
corrections to manual
\end{itemize}

%%%%%%%%%%%%%%%%%%%%%%%%%%%%%%%%%%%%%%%%
\paragraph{v1.5:} 2017/05/21

\begin{itemize}
\item
more complete structuring introduced
\item
|\childdocof| introduced
\item
|\childdoc| renamed to |\childdocmain|
\item
|\childredirect| renamed to |\childdocforward| and |\childdocforwardprefix|
and functionality expanded
\end{itemize}

%%%%%%%%%%%%%%%%%%%%%%%%%%%%%%%%%%%%%%%%
\paragraph{v1.0:} 2017/04/27

\begin{itemize}
\item
manual and install package
\item
first version published on CTAN
\end{itemize}

%%%%%%%%%%%%%%%%%%%%%%%%%%%%%%%%%%%%%%%%
\paragraph{v0.6:} 2017/04/26

\begin{itemize}
\item
redirection mechanism added
\end{itemize}

%%%%%%%%%%%%%%%%%%%%%%%%%%%%%%%%%%%%%%%%
\paragraph{v0.5:} 2017/04/26

\begin{itemize}
\item
functionality in definition file
\end{itemize}


%%%%%%%%%%%%%%%%%%%%%%%%%%%%%%%%%%%%%%%%%%%%%%%%%%%%%%%%%%%%%%%%%%%%%%%%%%%%%%%%
%%%%%%%%%%%%%%%%%%%%%%%%%%%%%%%%%%%%%%%%%%%%%%%%%%%%%%%%%%%%%%%%%%%%%%%%%%%%%%%%
%%%%%%%%%%%%%%%%%%%%%%%%%%%%%%%%%%%%%%%%%%%%%%%%%%%%%%%%%%%%%%%%%%%%%%%%%%%%%%%%
\appendix

\settowidth\MacroIndent{\rmfamily\scriptsize 000\ }

 \DocInput{childdoc.dtx}

\end{document}
%</driver>
% \fi
%
% %%%%%%%%%%%%%%%%%%%%%%%%%%%%%%%%%%%%%%%%%%%%%%%%%%%%%%%%%%%%%%%%%%%%%%%%%%%%%%
% %%%%%%%%%%%%%%%%%%%%%%%%%%%%%%%%%%%%%%%%%%%%%%%%%%%%%%%%%%%%%%%%%%%%%%%%%%%%%%
% \section{Sample}
%\iffalse
%<*samplemain>
%\fi
%
% The following presents a sample document
% with two chapters, two parts, a title page,
% a compile flag as well as three forwarding files to set the flag.
% It consists of eight |.tex| files:
% \begin{center}
% \begin{tabular}{ll}
% |cdocsamp.tex|&main file\\
% |cdocsch1.tex|&include file for chapter 1\\
% |cdocsch2.tex|&include file for chapter 2\\
% |cdocspt3.tex|&include file for part 3\\
% |cdocspt4.tex|&include file for part 4\\
% |cdocsdrf.tex|&forwarding file for main file in draft mode\\
% |cdocsfi1.tex|&forwarding file for final version of chapter 1\\
% |cdocsfi2.tex|&forwarding file for final version of chapter 2\\
% \end{tabular}
% \end{center}
% Each of the eight files can be compiled directly by the \LaTeX{} compiler.
%
% %%%%%%%%%%%%%%%%%%%%%%%%%%%%%%%%%%%%%%
% \paragraph{Main File.}
%
% The main file is called |cdocsamp.tex|.
%
% Load the \textsf{childdoc} definitions and
% declare the filename for the main document:
%    \begin{macrocode}
\input{childdoc.def}
\childdocmain{}
%    \end{macrocode}

% Optional override for |\version| flag:
%    \begin{macrocode}
%%\ifchilddoc\else\providecommand{\version}{draft}\fi
%    \end{macrocode}

% Define the default values for the |\version| flag
% (|final| for the main file and |draft| for childs):
%    \begin{macrocode}
\ifchilddoc
\providecommand{\version}{draft}
\else
\providecommand{\version}{final}
\fi
%    \end{macrocode}

% Load the standard document class:
%    \begin{macrocode}
\documentclass[12pt]{article}
%    \end{macrocode}

% Start the document body:
%    \begin{macrocode}
\begin{document}
%    \end{macrocode}

% Declare a title page.
% Print title, part of document being processed and version flag:
%    \begin{macrocode}
\addtocounter{page}{-1}
\begin{center}
{\LARGE\bfseries{}childdoc example\par}
\vspace{1cm}
\ifchilddoc
\ifchilddocmanual part\else chapter\fi:
`\childdocname' of `\childdocjob'\par
\else
main document: `\childdocjob'\par
\fi
version: \version\par
\end{center}
\newpage
%    \end{macrocode}

% Manually include selected file,
% otherwise process as usual:
%    \begin{macrocode}
\ifchilddocmanual
\section*{part `\childdocname'}
\input{\childdocname}
\else
%    \end{macrocode}

% Include the two chapters:
%    \begin{macrocode}
\include{cdocsch1}
\include{cdocsch2}
%    \end{macrocode}

% Include the two parts unless only chapters should be displayed:
%    \begin{macrocode}
\ifchilddoc\else
\section{part three}
\input{cdocspt3}
\section{part four}
\input{cdocspt4}
\fi
%    \end{macrocode}

% Process as usual until here:
%    \begin{macrocode}
\fi
%    \end{macrocode}

% End of document body:
%    \begin{macrocode}
\end{document}
%    \end{macrocode}
%\iffalse
%</samplemain>
%\fi
%
% %%%%%%%%%%%%%%%%%%%%%%%%%%%%%%%%%%%%%%
% \paragraph{Chapter Include Files.}
%
% The include files are called |cdocsch1.tex| and |cdocsch2.tex|.
%
%\iffalse
%<*samplechap1|samplechap2>
%\fi

% Optional override for |\version| flag:
%    \begin{macrocode}
%%\providecommand{\version}{final}
%    \end{macrocode}

% Include the main document:
%    \begin{macrocode}
\input{childdoc.def}
\childdocof{cdocsamp}
%    \end{macrocode}

%\iffalse
%</samplechap1|samplechap2>
%\fi
%
%\iffalse
%<*samplechap1>
%\fi
% Some text for chapter 1:
%    \begin{macrocode}
\section{one}
some text in chapter one
%    \end{macrocode}

%\iffalse
%</samplechap1>
%\fi
% Some text for chapter 2:
%\iffalse
%<*samplechap2>
%\fi
%    \begin{macrocode}
\section{two}
more text in chapter two
%    \end{macrocode}

%\iffalse
%</samplechap2>
%\fi
%
% %%%%%%%%%%%%%%%%%%%%%%%%%%%%%%%%%%%%%%
% \paragraph{Part Include Files.}
%
% The include files are called |cdocspt3.tex| and |cdocspt4.tex|.
%
%\iffalse
%<*samplepart3|samplepart4>
%\fi

% Optional override for |\version| flag:
%    \begin{macrocode}
%%\providecommand{\version}{final}
%    \end{macrocode}

% Include the main document:
%    \begin{macrocode}
\input{childdoc.def}
\childdocby{cdocsamp}
%    \end{macrocode}

%\iffalse
%</samplepart3|samplepart4>
%\fi
%
%\iffalse
%<*samplepart3>
%\fi
% Some text for part 3:
%    \begin{macrocode}
some text in part three
%    \end{macrocode}

%\iffalse
%</samplepart3>
%\fi
% Some text for part 4:
%\iffalse
%<*samplepart4>
%\fi
%    \begin{macrocode}
more text in part four
%    \end{macrocode}

%\iffalse
%</samplepart4>
%\fi
%
% %%%%%%%%%%%%%%%%%%%%%%%%%%%%%%%%%%%%%%
% \paragraph{Forwarding for a Complete Draft.}
%
% The following forwarding file |cdocsdrf.tex|
% compiles the main document in draft mode:
%\iffalse
%<*sampledraft>
%\fi
%    \begin{macrocode}
\def\version{draft}
\input{childdoc.def}
\childdocforward{cdocsamp}
%    \end{macrocode}

%\iffalse
%</sampledraft>
%\fi
%
% %%%%%%%%%%%%%%%%%%%%%%%%%%%%%%%%%%%%%%
% \paragraph{Forwarding for Final Version of the Chapters.}
%
% The following forwarding files |cdocsfn1.tex| and |cdocsfn2.tex|
% (with identical content)
% compile the final versions of the child documents
% |cdocsch1.tex| and |cdocsch2.tex|, respectively:
%\iffalse
%<*samplefinal>
%\fi
%    \begin{macrocode}
\def\version{final}
\input{childdoc.def}
\childdocforwardprefix[cdocsamp]{cdocsfn}{cdocsch}
%    \end{macrocode}

%\iffalse
%</samplefinal>
%\fi
%
% %%%%%%%%%%%%%%%%%%%%%%%%%%%%%%%%%%%%%%
% \paragraph{Command Line Processing.}
%
% The following three command lines generate the output files
% |cdocscld|, |cdocscl1| and |cdocscl2|
% which should be identical to
% |cdocsdrf|, |cdocsch1| and |cdocsfn2|, respectively:
% \begin{center}
% \begin{tabular}{l}
% |latex -jobname cdocscld \|\\
% |  "\def\version{draft}\input{childdoc.def}\childdocforward{cdocsamp}"|\\
% |latex -jobname cdocscl1 \|\\
% |  "\input{childdoc.def}\childdocforward[cdocsamp]{cdocsch1}"|\\
% |latex -jobname cdocscl2 \|\\
% |  "\def\version{final}\input{childdoc.def}\childdocforward{cdocsch2}"|
% \end{tabular}
% \end{center}
% Note that the trailing backslash on each first line
% merely continues the input to the second line
% (for convenient cut ant paste).
% Furthermore, the command |latex| can be replaced by any
% of its alternative versions such as |pdflatex|.
%
% %%%%%%%%%%%%%%%%%%%%%%%%%%%%%%%%%%%%%%%%%%%%%%%%%%%%%%%%%%%%%%%%%%%%%%%%%%%%%%
% %%%%%%%%%%%%%%%%%%%%%%%%%%%%%%%%%%%%%%%%%%%%%%%%%%%%%%%%%%%%%%%%%%%%%%%%%%%%%%
% \section{Implementation}
%\iffalse
%<*package>
%\fi
%
% This section describes the definitions file |childdoc.def|.

% The definitions cannot be loaded using |\usepackage| or |\RequirePackage|
% which has a mechanism to prevent loading a style file more than once.
% When loading the definitions by means of |\input|
% multiple instances have to be prevented manually:
%\iffalse
%This code needs to be before the `\ProvidesFile' directive
%which is defined at the beginning of this file.
%Therefore it is also placed there and commented out here.
%</package>
%<*discard>
%\fi
%    \begin{macrocode}
\ifdefined\childdocmain\endinput\fi
%    \end{macrocode}
%\iffalse
%</discard>
%<*package>
%\fi
%
% \macro{\ifchilddoc}
% \macro{\ifchilddocmanual}
% The conditional |\ifchilddoc| tells whether a
% child (true) or main (false) document is being compiled.
% The conditional |\ifchilddocmanual| tells whether
% the |\includeonly| mechanism is used (false) or
% the selection of child files must be performed manually (true).
% The definitions initialise to false:
%    \begin{macrocode}
\newif\ifchilddoc
\newif\ifchilddocmanual
%    \end{macrocode}

% \macro{\childdocname}
% \macro{\childdocjob}
% The macro |\childdocname| stores the name of the main document
% to be compiled. The macro |\childdocjob| stores the name of
% the document on which the \LaTeX{} compiler was originally invoked.
% The content of |\jobname| cannot be compared
% to filenames specified in the source due to different catcodes.
% The following code rescans |\jobname|, stores the result
% in |\childdocname| and saves a copy in |\childdocjob|:
%    \begin{macrocode}
\edef\childdocname{\scantokens\expandafter{\jobname\noexpand}}
\let\childdocjob\childdocname
%    \end{macrocode}

% \macro{\childdocdisable}
% The macro |\childdocdisable| prevents the main file
% from being processed more than once.
% At this stage, the main document command |\childdocmain|
% is assumed to be called once again where it should do nothing.
% Any subsequent call to it should prevent
% a secondary processing of the main document
% It overwrites the forwarding commands
% |\childdocof| and |\childdocforward|
% with empty macros to prevent further inclusions of the main document:
%    \begin{macrocode}
\newcommand{\childdocdisable}
{
  \renewcommand{\childdocmain}[1]{\renewcommand{\childdocmain}[1]{\endinput}}
  \renewcommand{\childdocof}[1]{}
  \renewcommand{\childdocby}[2][]{}
  \renewcommand{\childdocforward}[2][]{}
  \renewcommand{\childdocdisable}{}
}
%    \end{macrocode}

% \macro{\childdocmain}
% The macro |\childdocmain| is to be called at the top of the main file
% with nothing or the main filename (without extension) as argument.
% First, it breaks loops.
% If the argument is not empty and does not match |\childdocname|
% (which is set by the first inclusion of |childdoc.def|),
% |\ifchilddoc| is set to true, |\includeonly| is applied to the child file
% and |\jobname| is set to the main file
% (for proper handling of |.aux| files):
%    \begin{macrocode}
\newcommand{\childdocmain}[1]
{
  \childdocdisable\childdocmain{}
  \if?#1?\else
    \begingroup
      \def\childdoctmp{#1}
      \ifx\childdoctmp\childdocname
        \def\childdoctmp{}
      \else
        \def\childdoctmp
        {
          \childdoctrue
          \includeonly{\childdocname}
          \def\childdocjob{#1}
          \def\jobname{#1}
        }
      \fi
      \expandafter
    \endgroup
    \childdoctmp
  \fi
}
%    \end{macrocode}

% \macro{\childdocof}
% The command |\childdocof| redirects
% compilation to the main file |#1|.
%    \begin{macrocode}
\newcommand{\childdocof}[1]
{
  \childdocdisable
  \childdoctrue
  \includeonly{\childdocname}
  \def\jobname{#1}
  \def\childdocjob{#1}
  \input{#1}
}
%    \end{macrocode}

% \macro{\childdocby}
% The command |\childdocby| ....
%    \begin{macrocode}
\newcommand{\childdocby}[2][]
{
  \childdocdisable
  \childdoctrue
  \childdocmanualtrue
  \if?#1?\else
    \def\jobname{#2}
  \fi
  \def\childdocjob{#2}
  \input{#2}
  \endinput
}
%    \end{macrocode}

% \macro{\childdocforward}
% The command |\childdocforward| redirects
% compilation to the main file or
% (if the optional argument is given) a child file.
% Parameters are set as if the main file
% or a child file starting with |\childdocof| was compiled.
% Then compilation is handed over to the main file:
%    \begin{macrocode}
\newcommand{\childdocforward}[2][]
{
  \begingroup
    \if?#1?
      \def\childdoctmp
      {
        \def\childdocname{#2}
        \def\childdocjob{#2}
        \def\jobname{#2}
        \input{#2}
        \endinput
      }
    \else
      \def\childdoctmp
      {
        \childdocdisable
        \def\childdocname{#2}
        \childdoctrue
        \includeonly{#2}
        \def\childdocjob{#1}
        \def\jobname{#1}
        \input{#1}
        \endinput
      }
    \fi
    \expandafter
  \endgroup
  \childdoctmp
}
%    \end{macrocode}

% \macro{\childdocforwardprefix}
% The command |\childdocforwardprefix| redirects
% compilation to the main or a child file by means of a pattern.
% The prefix |#1| in the current filename is replaced by |#2|
% and the suffix of the current filename is kept
% (it is assumed that the filename does not contain the substring `|~~~|'
% which is used as a delimiter).
% Compilation is handed over to the new file by |\childdocforward|:
%    \begin{macrocode}
\newcommand{\childdocforwardprefix}[3][]
{
  \begingroup
    \def\childdocextract #2##1~~~{\def\childdoctmp{\childdocforward[#1]{#3##1}}}
    \expandafter\childdocextract\childdocname~~~
    \expandafter
  \endgroup
  \childdoctmp
}
%    \end{macrocode}

% \macro{\childdoc}
% The deprecated macro |\childdoc| is a legacy version of |\childdocmain|:
%    \begin{macrocode}
\newcommand{\childdoc}{\childdocmain}
%    \end{macrocode}

% \macro{\childdocredirect}
% The deprecated macro |\childdocredirect| is a legacy version
% of |\childdocforward| and |\childdocforwardprefix|:
%    \begin{macrocode}
\newcommand{\childdocredirect}[2][]
{
  \begingroup
    \if?#1?
      \def\childdoctmp{\childdocforward{#2}}
    \else
      \def\childdoctmp{\childdocforwardprefix{#1}{#2}}
    \fi
    \expandafter
  \endgroup
  \childdoctmp
}
%    \end{macrocode}

%\iffalse
%</package>
%\fi
%
\endinput
\childdocforward{cdocsch2}"|
% \end{tabular}
% \end{center}
% Note that the trailing backslash on each first line
% merely continues the input to the second line
% (for convenient cut ant paste).
% Furthermore, the command |latex| can be replaced by any
% of its alternative versions such as |pdflatex|.
%
% %%%%%%%%%%%%%%%%%%%%%%%%%%%%%%%%%%%%%%%%%%%%%%%%%%%%%%%%%%%%%%%%%%%%%%%%%%%%%%
% %%%%%%%%%%%%%%%%%%%%%%%%%%%%%%%%%%%%%%%%%%%%%%%%%%%%%%%%%%%%%%%%%%%%%%%%%%%%%%
% \section{Implementation}
%\iffalse
%<*package>
%\fi
%
% This section describes the definitions file |childdoc.def|.

% The definitions cannot be loaded using |\usepackage| or |\RequirePackage|
% which has a mechanism to prevent loading a style file more than once.
% When loading the definitions by means of |\input|
% multiple instances have to be prevented manually:
%\iffalse
%This code needs to be before the `\ProvidesFile' directive
%which is defined at the beginning of this file.
%Therefore it is also placed there and commented out here.
%</package>
%<*discard>
%\fi
%    \begin{macrocode}
\ifdefined\childdocmain\endinput\fi
%    \end{macrocode}
%\iffalse
%</discard>
%<*package>
%\fi
%
% \macro{\ifchilddoc}
% \macro{\ifchilddocmanual}
% The conditional |\ifchilddoc| tells whether a
% child (true) or main (false) document is being compiled.
% The conditional |\ifchilddocmanual| tells whether
% the |\includeonly| mechanism is used (false) or
% the selection of child files must be performed manually (true).
% The definitions initialise to false:
%    \begin{macrocode}
\newif\ifchilddoc
\newif\ifchilddocmanual
%    \end{macrocode}

% \macro{\childdocname}
% \macro{\childdocjob}
% The macro |\childdocname| stores the name of the main document
% to be compiled. The macro |\childdocjob| stores the name of
% the document on which the \LaTeX{} compiler was originally invoked.
% The content of |\jobname| cannot be compared
% to filenames specified in the source due to different catcodes.
% The following code rescans |\jobname|, stores the result
% in |\childdocname| and saves a copy in |\childdocjob|:
%    \begin{macrocode}
\edef\childdocname{\scantokens\expandafter{\jobname\noexpand}}
\let\childdocjob\childdocname
%    \end{macrocode}

% \macro{\childdocdisable}
% The macro |\childdocdisable| prevents the main file
% from being processed more than once.
% At this stage, the main document command |\childdocmain|
% is assumed to be called once again where it should do nothing.
% Any subsequent call to it should prevent
% a secondary processing of the main document
% It overwrites the forwarding commands
% |\childdocof| and |\childdocforward|
% with empty macros to prevent further inclusions of the main document:
%    \begin{macrocode}
\newcommand{\childdocdisable}
{
  \renewcommand{\childdocmain}[1]{\renewcommand{\childdocmain}[1]{\endinput}}
  \renewcommand{\childdocof}[1]{}
  \renewcommand{\childdocby}[2][]{}
  \renewcommand{\childdocforward}[2][]{}
  \renewcommand{\childdocdisable}{}
}
%    \end{macrocode}

% \macro{\childdocmain}
% The macro |\childdocmain| is to be called at the top of the main file
% with nothing or the main filename (without extension) as argument.
% First, it breaks loops.
% If the argument is not empty and does not match |\childdocname|
% (which is set by the first inclusion of |childdoc.def|),
% |\ifchilddoc| is set to true, |\includeonly| is applied to the child file
% and |\jobname| is set to the main file
% (for proper handling of |.aux| files):
%    \begin{macrocode}
\newcommand{\childdocmain}[1]
{
  \childdocdisable\childdocmain{}
  \if?#1?\else
    \begingroup
      \def\childdoctmp{#1}
      \ifx\childdoctmp\childdocname
        \def\childdoctmp{}
      \else
        \def\childdoctmp
        {
          \childdoctrue
          \includeonly{\childdocname}
          \def\childdocjob{#1}
          \def\jobname{#1}
        }
      \fi
      \expandafter
    \endgroup
    \childdoctmp
  \fi
}
%    \end{macrocode}

% \macro{\childdocof}
% The command |\childdocof| redirects
% compilation to the main file |#1|.
%    \begin{macrocode}
\newcommand{\childdocof}[1]
{
  \childdocdisable
  \childdoctrue
  \includeonly{\childdocname}
  \def\jobname{#1}
  \def\childdocjob{#1}
  \input{#1}
}
%    \end{macrocode}

% \macro{\childdocby}
% The command |\childdocby| ....
%    \begin{macrocode}
\newcommand{\childdocby}[2][]
{
  \childdocdisable
  \childdoctrue
  \childdocmanualtrue
  \if?#1?\else
    \def\jobname{#2}
  \fi
  \def\childdocjob{#2}
  \input{#2}
  \endinput
}
%    \end{macrocode}

% \macro{\childdocforward}
% The command |\childdocforward| redirects
% compilation to the main file or
% (if the optional argument is given) a child file.
% Parameters are set as if the main file
% or a child file starting with |\childdocof| was compiled.
% Then compilation is handed over to the main file:
%    \begin{macrocode}
\newcommand{\childdocforward}[2][]
{
  \begingroup
    \if?#1?
      \def\childdoctmp
      {
        \def\childdocname{#2}
        \def\childdocjob{#2}
        \def\jobname{#2}
        \input{#2}
        \endinput
      }
    \else
      \def\childdoctmp
      {
        \childdocdisable
        \def\childdocname{#2}
        \childdoctrue
        \includeonly{#2}
        \def\childdocjob{#1}
        \def\jobname{#1}
        \input{#1}
        \endinput
      }
    \fi
    \expandafter
  \endgroup
  \childdoctmp
}
%    \end{macrocode}

% \macro{\childdocforwardprefix}
% The command |\childdocforwardprefix| redirects
% compilation to the main or a child file by means of a pattern.
% The prefix |#1| in the current filename is replaced by |#2|
% and the suffix of the current filename is kept
% (it is assumed that the filename does not contain the substring `|~~~|'
% which is used as a delimiter).
% Compilation is handed over to the new file by |\childdocforward|:
%    \begin{macrocode}
\newcommand{\childdocforwardprefix}[3][]
{
  \begingroup
    \def\childdocextract #2##1~~~{\def\childdoctmp{\childdocforward[#1]{#3##1}}}
    \expandafter\childdocextract\childdocname~~~
    \expandafter
  \endgroup
  \childdoctmp
}
%    \end{macrocode}

% \macro{\childdoc}
% The deprecated macro |\childdoc| is a legacy version of |\childdocmain|:
%    \begin{macrocode}
\newcommand{\childdoc}{\childdocmain}
%    \end{macrocode}

% \macro{\childdocredirect}
% The deprecated macro |\childdocredirect| is a legacy version
% of |\childdocforward| and |\childdocforwardprefix|:
%    \begin{macrocode}
\newcommand{\childdocredirect}[2][]
{
  \begingroup
    \if?#1?
      \def\childdoctmp{\childdocforward{#2}}
    \else
      \def\childdoctmp{\childdocforwardprefix{#1}{#2}}
    \fi
    \expandafter
  \endgroup
  \childdoctmp
}
%    \end{macrocode}

%\iffalse
%</package>
%\fi
%
\endinput
|\\
|\childdocforwardprefix[|\textit{main}|]{|\textit{prefix}|}{|\textit{dest}|}|
\end{tabular}
\end{center}
%
the destination file is determined by a pattern
depending on the current file:
To make this work, the current file must be called
`{\textit{prefix}\hspace{0.2em}\textit{suffix}}'
with \textit{prefix} matching precisely the argument.
Processing is then passed on to the file
`{\textit{dest}\hspace{0.2em}\textit{suffix}}'.
Surely, the same effect is achieved by
directly specifying the
argument `{\textit{dest}\hspace{0.2em}\textit{suffix}}'
in the first form.
However, that requires to set up a different file
for each child. With the alternative form of the command
all these files can have exactly the same content
which simplifies setting them up and maintaining them.

For example, the following file |draft.tex|
with a compilation flag |\version| as described in \secref{sec:flags}
compiles the main document as a draft:
%
\begin{center}
\begin{tabular}{l}
|\def\version{draft}|\\
|% \iffalse
%
% childdoc.dtx Copyright (C) 2017-2018 Niklas Beisert
%
% This work may be distributed and/or modified under the
% conditions of the LaTeX Project Public License, either version 1.3
% of this license or (at your option) any later version.
% The latest version of this license is in
%   http://www.latex-project.org/lppl.txt
% and version 1.3 or later is part of all distributions of LaTeX
% version 2005/12/01 or later.
%
% This work has the LPPL maintenance status `maintained'.
%
% The Current Maintainer of this work is Niklas Beisert.
%
% This work consists of the files childdoc.dtx and childdoc.ins
% and the derived files childdoc.def and cdocsamp.tex with
% cdocsch1.tex, cdocsch2.tex, cdocsdrf.tex, cdocsfn1.tex, cdocsfn2.tex.
%
%<package>\ifdefined\childdocmain\endinput\fi
%<package>\ProvidesFile{childdoc.def}[2018/12/30 v2.0 child document driver]
%<samplemain>\ProvidesFile{cdocsamp.tex}[2018/12/30 v2.0 sample for childdoc]
%<*driver>
%\ProvidesFile{childdoc.drv}[2018/12/30 v2.0 childdoc reference manual file]
\PassOptionsToClass{10pt,a4paper}{article}
\documentclass{ltxdoc}

\usepackage[margin=35mm]{geometry}
\usepackage{hyperref}
\usepackage{hyperxmp}
\usepackage[usenames]{color}

\hypersetup{colorlinks=true}
\hypersetup{pdfstartview=FitH}
\hypersetup{pdfpagemode=UseNone}
\hypersetup{pdfsource={}}
\hypersetup{pdflang={en-UK}}
\hypersetup{pdfcopyright={Copyright 2017-2018 Niklas Beisert.
  This work may be distributed and/or modified under the
  conditions of the LaTeX Project Public License, either version 1.3
  of this license or (at your option) any later version.}}
\hypersetup{pdflicenseurl={http://www.latex-project.org/lppl.txt}}
\hypersetup{pdfcontactaddress={ETH Zurich, ITP, HIT K,
  Wolfgang-Pauli-Strasse 27}}
\hypersetup{pdfcontactpostcode={8093}}
\hypersetup{pdfcontactcity={Zurich}}
\hypersetup{pdfcontactcountry={Switzerland}}
\hypersetup{pdfcontactemail={nbeisert@itp.phys.ethz.ch}}
\hypersetup{pdfcontacturl={http://people.phys.ethz.ch/\xmptilde nbeisert/}}

\newcommand{\secref}[1]{\hyperref[#1]{section \ref*{#1}}}

\parskip1ex
\parindent0pt
\let\olditemize\itemize
\def\itemize{\olditemize\parskip0pt}

\begin{document}

\title{The \textsf{childdoc} Package}
\hypersetup{pdftitle={The childdoc Package}}
\author{Niklas Beisert\\[2ex]
  Institut f\"ur Theoretische Physik\\
  Eidgen\"ossische Technische Hochschule Z\"urich\\
  Wolfgang-Pauli-Strasse 27, 8093 Z\"urich, Switzerland\\[1ex]
  \href{mailto:nbeisert@itp.phys.ethz.ch}
  {\texttt{nbeisert@itp.phys.ethz.ch}}}
\hypersetup{pdfauthor={Niklas Beisert}}
\hypersetup{pdfsubject={Manual for the LaTeX2e Package childdoc}}
\date{30 December 2018, \textsf{v2.0}}
\maketitle

\begin{abstract}\noindent
\textsf{childdoc} is a \LaTeXe{} package
that enables the direct compilation
of document sections included by |\include|
to individual files.
\end{abstract}

\begingroup
\parskip0ex
\tableofcontents
\endgroup

%%%%%%%%%%%%%%%%%%%%%%%%%%%%%%%%%%%%%%%%%%%%%%%%%%%%%%%%%%%%%%%%%%%%%%%%%%%%%%%%
%%%%%%%%%%%%%%%%%%%%%%%%%%%%%%%%%%%%%%%%%%%%%%%%%%%%%%%%%%%%%%%%%%%%%%%%%%%%%%%%
\section{Introduction}

\LaTeX{} provides a mechanism to structure a large document (such as a book)
into a main file and several child files (containing the chapters)
using the |\include| command.
This mechanism is beneficial for documents
which span hundreds of pages in order to
make the source file(s) more manageable.
Moreover, compilation can be restricted to
selected child files by means of the |\includeonly| command.
The latter feature can be used to reduce the compilation time while editing
(this was significantly more useful in the earlier days of \LaTeX{})
or to generate a smaller document which is easier to navigate.
Another application of |\includeonly| is to generate
documents consisting of selected parts of the complete document.

However, there are a few drawbacks of the plain |\include| mechanism:
\begin{itemize}
\item
The child files cannot be compiled on their own,
they can only be compiled via the main file.
A naive editing environment
(such as a text editor with an option
to have the current file processed by \LaTeX)
may require one to switch to the main file before compiling;
attempting to compile the child file produces errors.
\item
The main file must be modified (each time)
to adjust the |\includeonly| command
to the present needs. This easily leaves the main file in a messy state.
\item
The generated document will always carry the filename
of the main document. This is inconvenient if
several child files are to be compiled and
to be kept for distribution.
\end{itemize}

The present package provides a simple interface
to make child files individually compilable by \LaTeX{}.
Compiling a child file then has the same effect as compiling
the main file with an |\includeonly| command
to select the appropriate child.
Moreover the generated document will carry the name of the child
rather than the main file.
This resolves all three above issues.

This feature is meant to make the editing of books,
thesis documents and lecture notes somewhat more convenient.
However, the package can also be used efficiently for
composing a series of documents (such as exercise sheets)
which are typically distributed individually.
It then assists the author in generating the individual documents
(potentially in different versions)
as well as a document containing the collected series.
Another application is in developing style files
or other kinds of included material
where compilation of the style file could redirect
to a sample or test file.

%%%%%%%%%%%%%%%%%%%%%%%%%%%%%%%%%%%%%%%%%%%%%%%%%%%%%%%%%%%%%%%%%%%%%%%%%%%%%%%%
%%%%%%%%%%%%%%%%%%%%%%%%%%%%%%%%%%%%%%%%%%%%%%%%%%%%%%%%%%%%%%%%%%%%%%%%%%%%%%%%
\section{Usage}

First of all, the package \textsf{childdoc} is \emph{not} a standard
\LaTeXe{} |.sty| style file! Therefore it needs to be invoked in
a non-standard way.

%%%%%%%%%%%%%%%%%%%%%%%%%%%%%%%%%%%%%%%%%%%%%%%%%%%%%%%%%%%%%%%%%%%%%%%%%%%%%%%%
\subsection{Included Files}
\label{sec:include}

%%%%%%%%%%%%%%%%%%%%%%%%%%%%%%%%%%%%%%%%
\DescribeMacro{\childdocmain}
To use the package, add the commands
\begin{center}
\begin{tabular}{l}
|% \iffalse
%
% childdoc.dtx Copyright (C) 2017-2018 Niklas Beisert
%
% This work may be distributed and/or modified under the
% conditions of the LaTeX Project Public License, either version 1.3
% of this license or (at your option) any later version.
% The latest version of this license is in
%   http://www.latex-project.org/lppl.txt
% and version 1.3 or later is part of all distributions of LaTeX
% version 2005/12/01 or later.
%
% This work has the LPPL maintenance status `maintained'.
%
% The Current Maintainer of this work is Niklas Beisert.
%
% This work consists of the files childdoc.dtx and childdoc.ins
% and the derived files childdoc.def and cdocsamp.tex with
% cdocsch1.tex, cdocsch2.tex, cdocsdrf.tex, cdocsfn1.tex, cdocsfn2.tex.
%
%<package>\ifdefined\childdocmain\endinput\fi
%<package>\ProvidesFile{childdoc.def}[2018/12/30 v2.0 child document driver]
%<samplemain>\ProvidesFile{cdocsamp.tex}[2018/12/30 v2.0 sample for childdoc]
%<*driver>
%\ProvidesFile{childdoc.drv}[2018/12/30 v2.0 childdoc reference manual file]
\PassOptionsToClass{10pt,a4paper}{article}
\documentclass{ltxdoc}

\usepackage[margin=35mm]{geometry}
\usepackage{hyperref}
\usepackage{hyperxmp}
\usepackage[usenames]{color}

\hypersetup{colorlinks=true}
\hypersetup{pdfstartview=FitH}
\hypersetup{pdfpagemode=UseNone}
\hypersetup{pdfsource={}}
\hypersetup{pdflang={en-UK}}
\hypersetup{pdfcopyright={Copyright 2017-2018 Niklas Beisert.
  This work may be distributed and/or modified under the
  conditions of the LaTeX Project Public License, either version 1.3
  of this license or (at your option) any later version.}}
\hypersetup{pdflicenseurl={http://www.latex-project.org/lppl.txt}}
\hypersetup{pdfcontactaddress={ETH Zurich, ITP, HIT K,
  Wolfgang-Pauli-Strasse 27}}
\hypersetup{pdfcontactpostcode={8093}}
\hypersetup{pdfcontactcity={Zurich}}
\hypersetup{pdfcontactcountry={Switzerland}}
\hypersetup{pdfcontactemail={nbeisert@itp.phys.ethz.ch}}
\hypersetup{pdfcontacturl={http://people.phys.ethz.ch/\xmptilde nbeisert/}}

\newcommand{\secref}[1]{\hyperref[#1]{section \ref*{#1}}}

\parskip1ex
\parindent0pt
\let\olditemize\itemize
\def\itemize{\olditemize\parskip0pt}

\begin{document}

\title{The \textsf{childdoc} Package}
\hypersetup{pdftitle={The childdoc Package}}
\author{Niklas Beisert\\[2ex]
  Institut f\"ur Theoretische Physik\\
  Eidgen\"ossische Technische Hochschule Z\"urich\\
  Wolfgang-Pauli-Strasse 27, 8093 Z\"urich, Switzerland\\[1ex]
  \href{mailto:nbeisert@itp.phys.ethz.ch}
  {\texttt{nbeisert@itp.phys.ethz.ch}}}
\hypersetup{pdfauthor={Niklas Beisert}}
\hypersetup{pdfsubject={Manual for the LaTeX2e Package childdoc}}
\date{30 December 2018, \textsf{v2.0}}
\maketitle

\begin{abstract}\noindent
\textsf{childdoc} is a \LaTeXe{} package
that enables the direct compilation
of document sections included by |\include|
to individual files.
\end{abstract}

\begingroup
\parskip0ex
\tableofcontents
\endgroup

%%%%%%%%%%%%%%%%%%%%%%%%%%%%%%%%%%%%%%%%%%%%%%%%%%%%%%%%%%%%%%%%%%%%%%%%%%%%%%%%
%%%%%%%%%%%%%%%%%%%%%%%%%%%%%%%%%%%%%%%%%%%%%%%%%%%%%%%%%%%%%%%%%%%%%%%%%%%%%%%%
\section{Introduction}

\LaTeX{} provides a mechanism to structure a large document (such as a book)
into a main file and several child files (containing the chapters)
using the |\include| command.
This mechanism is beneficial for documents
which span hundreds of pages in order to
make the source file(s) more manageable.
Moreover, compilation can be restricted to
selected child files by means of the |\includeonly| command.
The latter feature can be used to reduce the compilation time while editing
(this was significantly more useful in the earlier days of \LaTeX{})
or to generate a smaller document which is easier to navigate.
Another application of |\includeonly| is to generate
documents consisting of selected parts of the complete document.

However, there are a few drawbacks of the plain |\include| mechanism:
\begin{itemize}
\item
The child files cannot be compiled on their own,
they can only be compiled via the main file.
A naive editing environment
(such as a text editor with an option
to have the current file processed by \LaTeX)
may require one to switch to the main file before compiling;
attempting to compile the child file produces errors.
\item
The main file must be modified (each time)
to adjust the |\includeonly| command
to the present needs. This easily leaves the main file in a messy state.
\item
The generated document will always carry the filename
of the main document. This is inconvenient if
several child files are to be compiled and
to be kept for distribution.
\end{itemize}

The present package provides a simple interface
to make child files individually compilable by \LaTeX{}.
Compiling a child file then has the same effect as compiling
the main file with an |\includeonly| command
to select the appropriate child.
Moreover the generated document will carry the name of the child
rather than the main file.
This resolves all three above issues.

This feature is meant to make the editing of books,
thesis documents and lecture notes somewhat more convenient.
However, the package can also be used efficiently for
composing a series of documents (such as exercise sheets)
which are typically distributed individually.
It then assists the author in generating the individual documents
(potentially in different versions)
as well as a document containing the collected series.
Another application is in developing style files
or other kinds of included material
where compilation of the style file could redirect
to a sample or test file.

%%%%%%%%%%%%%%%%%%%%%%%%%%%%%%%%%%%%%%%%%%%%%%%%%%%%%%%%%%%%%%%%%%%%%%%%%%%%%%%%
%%%%%%%%%%%%%%%%%%%%%%%%%%%%%%%%%%%%%%%%%%%%%%%%%%%%%%%%%%%%%%%%%%%%%%%%%%%%%%%%
\section{Usage}

First of all, the package \textsf{childdoc} is \emph{not} a standard
\LaTeXe{} |.sty| style file! Therefore it needs to be invoked in
a non-standard way.

%%%%%%%%%%%%%%%%%%%%%%%%%%%%%%%%%%%%%%%%%%%%%%%%%%%%%%%%%%%%%%%%%%%%%%%%%%%%%%%%
\subsection{Included Files}
\label{sec:include}

%%%%%%%%%%%%%%%%%%%%%%%%%%%%%%%%%%%%%%%%
\DescribeMacro{\childdocmain}
To use the package, add the commands
\begin{center}
\begin{tabular}{l}
|\input{childdoc.def}|\\
|\childdocmain{}|\\
\end{tabular}
\end{center}
at the very top of the main \LaTeX{} file,
in particular \emph{before} the |\documentclass| statement!
The argument of |\childdocmain| should be left empty
(but it must be present).

%%%%%%%%%%%%%%%%%%%%%%%%%%%%%%%%%%%%%%%%
\DescribeMacro{\childdocof}
Furthermore, add the commands
\begin{center}
\begin{tabular}{l}
|\input{childdoc.def}|\\
|\childdocof{|\textit{main}|}|\\
\end{tabular}
\end{center}
at the top of every child file \textit{child}
which is included by |\include{|\textit{child}|}|
from within the main file
(or at least for those files to be compiled individually).
The argument \textit{main} must be the filename of the main file.

There are a couple of
considerations in setting up the main and child documents:

%%%%%%%%%%%%%%%%%%%%%%%%%%%%%%%%%%%%%%%%
\paragraph{Restrictions.}

Please note the following restrictions:
\begin{itemize}
\item
|\childdocmain| must be called with one argument \textit{main}
to ensure compatibility with earlier version of the package.
It must either be empty (|\childdocmain{}|)
or precisely match the filename of the main file in which it is specified.
See \secref{sec:detection} for further information.
\item
The filename \textit{main} must be specified without the |.tex| extension.
\item
The filename \textit{main} is case sensitive
(even in case-insensitive file systems)
due to internal string comparison.
\item
The argument \textit{main} should be fully expanded, it cannot be a macro.
\item
Subdirectories and special characters should be avoided in filenames.
\item
The command |\childdocmain{|\textit{main}|}| must be followed by a whitespace.
It should not be followed immediately by another command
or by a comment mark `|%|'.
This is because the \TeX{} parser reads the token immediately following
the argument of |\childdocmain| and puts it
at the beginning of every child section;
however, a white\-space is ignored.
\end{itemize}

%%%%%%%%%%%%%%%%%%%%%%%%%%%%%%%%%%%%%%%%
\paragraph{Content of Main File.}

It is advisable to place all content in the child files included by |\include|.
Any output contained in the main file will appear in all child documents
unless suppressed manually;
it cannot be suppressed automatically by the |\includeonly| directive
and thus should normally be avoided.
A method to include some content in the main file
by means of conditional processing is described in \secref{sec:conditional}.

%%%%%%%%%%%%%%%%%%%%%%%%%%%%%%%%%%%%%%%%
\paragraph{Page Numbering.}

When only a part of the document is compiled,
the appropriate numbering of pages
(as well as other status parameters)
is determined from the |.aux| files.
The latter contain information from previous passes.
However this information needs to propagate through
all intermediate child documents.
Therefore the page numbering in child documents may well
be inconsistent until the complete document is compiled at least once.

A useful (if unconventional) way to always ensure a consistent
page numbering is to restart the numbering in each child document
and denote the pages by `\textit{child}|.|\textit{page}'
where \textit{child} represents the chapter/section number of the child file.
This can be achieved by the command
|\numberwithin{page}{|\textit{child}|}|
of the \textsf{amsmath} package
where \textit{child} can be |chapter| or |section|
depending on the chosen structuring.
Alternatively, one can modify the macro |\thepage| appropriately
and reset the counter |page| at the start of each child file.

%%%%%%%%%%%%%%%%%%%%%%%%%%%%%%%%%%%%%%%%%%%%%%%%%%%%%%%%%%%%%%%%%%%%%%%%%%%%%%%%
\subsection{Conditional Processing}
\label{sec:conditional}

The package provides a mechanism to compile different versions
of a document. To customise the versions further some conditional processing
can come in handy to distinguish which version is being compiled.
The package provides two macros to describe the compilation context:

%%%%%%%%%%%%%%%%%%%%%%%%%%%%%%%%%%%%%%%%
\DescribeMacro{\ifchilddoc}
The conditional |\ifchilddoc| distinguishes between the compilation of
child documents and the main document:
%
\begin{center}
|\ifchilddoc |\textit{child-code}| |[|\||else |\textit{main-code}]| \||fi|
\end{center}

%%%%%%%%%%%%%%%%%%%%%%%%%%%%%%%%%%%%%%%%
\DescribeMacro{\childdocname}
\DescribeMacro{\childdocjob}
The macro |\childdocname| contains the filename (without extension)
of the main or child file being processed.
Note that |\childdocjob| will always contain the name of the main file.

%%%%%%%%%%%%%%%%%%%%%%%%%%%%%%%%%%%%%%%%
\paragraph{Title Page.}

Conditional processing can be used to include a title or banner page
in the main document when proper precautions are taken.
Importantly, the code in the main file should ensure that the page counter
(as well as other status parameters which are stored in the |.aux| files)
takes the same value after the conditional processing.
Otherwise the page numbers may take divergent values
depending on which part is compiled.

For example, a title page could be declared by:
%
\begin{center}
\begin{tabular}{l}
|\ifchilddoc\||else|\\
|\addtocounter{page}{-1}|\\
\textit{code for title page}\\
|\newpage|\\
|\||fi|
\end{tabular}
\end{center}
%
A banner page for the child documents can be generated by:
%
\begin{center}
\begin{tabular}{l}
|\ifchilddoc|\\
|\addtocounter{page}{-1}|\\
\textit{code for banner page}\\
|\newpage|\\
|\||fi|
\end{tabular}
\end{center}
%
Here one could write a message such as:
\begin{center}
|This is the part \childdocname{} of \childdocjob{}.|
\end{center}

%%%%%%%%%%%%%%%%%%%%%%%%%%%%%%%%%%%%%%%%%%%%%%%%%%%%%%%%%%%%%%%%%%%%%%%%%%%%%%%%
\subsection{Flags}
\label{sec:flags}

The package makes it easy to generate different versions
of the main or child documents.
To this end compilation flags can be defined
and assigned different default values.
They will be particularly useful in conjunction
with the forwarding mechanism described in \secref{sec:forward}.

For example, it may be useful to have a flag |\version|
which can be set to |draft| or |final|.
The document source will contain some conditional code
depending on the value of |\version|.
Suppose further, the flag should default to |final| for the main file
and to |draft| for child files
which is a natural assignment for editing the document.
This is achieved by placing the following code
in the preamble of the main document
(below the |\childdocmain| directive):
%
\begin{center}
\begin{tabular}{l}
|\ifchilddoc|\\
|\providecommand{\version}{draft}|\\
|\||else|\\
|\providecommand{\version}{final}|\\
|\||fi|
\end{tabular}
\end{center}
%
The definition by |\providecommand| makes sure
that previous definitions are not overwritten.
Further statements |\providecommand{\version}{...}|
can thus be added before the above code to override it.

For the main file, one might add a line
(between |\childdocmain| and the above block)
%
\begin{center}
|%\ifchilddoc\||else\providecommand{\version}{draft}\||fi|
\end{center}
%
which can be uncommented to produce a draft version.
Likewise one can add a line to the very top of a child file
(above the |\childdocof{|\textit{main}|}| directive)
%
\begin{center}
|%\providecommand{\version}{final}|
\end{center}
%
which can be uncommented to produce the final version of this child document.

%%%%%%%%%%%%%%%%%%%%%%%%%%%%%%%%%%%%%%%%%%%%%%%%%%%%%%%%%%%%%%%%%%%%%%%%%%%%%%%%
\subsection{Forwarding}
\label{sec:forward}

Different versions of the main or child documents
using compilation flags as described in \secref{sec:flags}
can be (permanently) stored in different files
for convenient compilation, viewing and distribution.
To this end, the package defines a command
to pass on compilation to a different file:

%%%%%%%%%%%%%%%%%%%%%%%%%%%%%%%%%%%%%%%%
\DescribeMacro{\childdocforward}
The command |\childdocforward| redirects processing to
another source file:
%
\begin{center}
\begin{tabular}{l}
|\input{childdoc.def}|\\
|\childdocforward[|\textit{main}|]{|\textit{dest}|}|\\
\end{tabular}
\end{center}
%
The argument \textit{dest} is the destination file
(without extension).
It should be the main file or one of the child files.
Note that further \textsf{childdoc} directives
such as |\childdocof| and |\childdocforward|
in the indicated file will be processed in this form.
The optional argument \textit{main}
passes on directly to the main file \textit{main}
while pretending to compile the child \textit{dest}.
This form behaves as if \textit{dest}
issues |\childdocof{|\textit{main}|}| right away,
and no further \textsf{childdoc} directives will be processed.

%%%%%%%%%%%%%%%%%%%%%%%%%%%%%%%%%%%%%%%%
\DescribeMacro{\...prefix}
In the alternative form |\childdocforwardprefix|,
%
\begin{center}
\begin{tabular}{l}
|\input{childdoc.def}|\\
|\childdocforwardprefix[|\textit{main}|]{|\textit{prefix}|}{|\textit{dest}|}|
\end{tabular}
\end{center}
%
the destination file is determined by a pattern
depending on the current file:
To make this work, the current file must be called
`{\textit{prefix}\hspace{0.2em}\textit{suffix}}'
with \textit{prefix} matching precisely the argument.
Processing is then passed on to the file
`{\textit{dest}\hspace{0.2em}\textit{suffix}}'.
Surely, the same effect is achieved by
directly specifying the
argument `{\textit{dest}\hspace{0.2em}\textit{suffix}}'
in the first form.
However, that requires to set up a different file
for each child. With the alternative form of the command
all these files can have exactly the same content
which simplifies setting them up and maintaining them.

For example, the following file |draft.tex|
with a compilation flag |\version| as described in \secref{sec:flags}
compiles the main document as a draft:
%
\begin{center}
\begin{tabular}{l}
|\def\version{draft}|\\
|\input{childdoc.def}|\\
|\childdocforward{|\textit{main}|}|
\end{tabular}
\end{center}
%
Likewise, the following files |final|\textit{nn}|.tex|
compile the final version of the child document
|child|\textit{nn}|.tex|:
%
\begin{center}
\begin{tabular}{l}
|\def\version{final}|\\
|\input{childdoc.def}|\\
|\childdocforwardprefix{final}{child}|
\end{tabular}
\end{center}
%

Note that when several versions of a main file and/or of each child file
are to be generated, it may be convenient to set up a |Makefile| or
shell script to automatise the process.

%%%%%%%%%%%%%%%%%%%%%%%%%%%%%%%%%%%%%%%%%%%%%%%%%%%%%%%%%%%%%%%%%%%%%%%%%%%%%%%%
\subsection{Command Line Processing}
\label{sec:commandline}

The effect of redirection files can also be achieved by invoking
the \LaTeX{} compiler with a more elaborate command line.
Most conveniently this should be done as part
of a shell script or a |Makefile|.

When using \textsf{childdoc} in the main file, the following
command lines effectively perform a redirection
(note that depending on the shell being used,
backslashes may have to be doubled: `|\|' $\to$ `|\\|'):
%
\begin{center}
|... -jobname "|\textit{target}|" |\\|"|[\textit{flags}]%
|\input{childdoc.def}\childdocforward[|\textit{main}|]{|\textit{dest}|}"|
\end{center}
%
Here \textit{target} is the name of the output file,
\textit{main} is the name of the main file
and \textit{dest} is the name of the main or child file to be processed
(all filenames without extensions).
The optional argument \textit{main} can be omitted
if \textit{main} matches \textit{dest}.
Optionally, compilation \textit{flags} can be defined via |\def| commands.
This command line makes the \TeX{} engine believe
it is compiling the file \textit{target}
whose content is specified as the latter parameter.
The provided code then forwards the processing to
\textit{main} or \textit{dest} as described in \secref{sec:forward}.

%%%%%%%%%%%%%%%%%%%%%%%%%%%%%%%%%%%%%%%%%%%%%%%%%%%%%%%%%%%%%%%%%%%%%%%%%%%%%%%%
\subsection{Include by Input}
\label{sec:input}

Including child documents by |\include| has some restrictions by design.
Most notably, the content of a child document always occupies
its own set of pages; pages cannot be shared between child documents.
Usually, this behaviour makes perfect sense
because each child document contain an essential part of the document.
However, in some situations it may be desirable to compose
a document from a collection of parts
without having mandatory page breaks between then.
For this case, the package
provides a mechanism to include parts
by |\input| which can also be processed individually.
However, by construction this mechanism
requires manual handling of the content to be output.

%%%%%%%%%%%%%%%%%%%%%%%%%%%%%%%%%%%%%%%%
\DescribeMacro{\ifchilddocmanual}
The main file should be prepared as usual, see \secref{sec:include}.
However, the document body must make a distinction
between processing of an individual part and of the main document, e.g.:
%
\begin{center}
\begin{tabular}{l}
|\ifchilddocmanual|\\
|\input{\childdocname}|\\
|\||else|\\
\textit{document body with }|\input{|\textit{part}|}|\\
|\||fi|
\end{tabular}
\end{center}
%
The conditional |\ifchilddocmanual| is true whenever
a part to be included by |\input| is being compiled,
and the name of the part is stored in |\childdocname|.

%%%%%%%%%%%%%%%%%%%%%%%%%%%%%%%%%%%%%%%%
\DescribeMacro{\childdocby}
Each part to be included by |\input| should start with:
%
\begin{center}
\begin{tabular}{l}
|\input{childdoc.def}|\\
|\childdocby{|\textit{main}|}|\\
\end{tabular}
\end{center}
%
The directive |\childdocby| is similar to |\childdocof|
described in \secref{sec:include},
but the subsequent selection of content must be done manually.
To that end, both |\ifchilddoc| and |\ifchilddocmanual|
will be true upon processing of a part,
and the name of the part is stored in |\childdocname|.
Note that |\jobname| will be set to the filename of the current part
so that each part receives an individual |.aux| file
that does not interfere with the |.aux| file(s) of the main document.
This behaviour can be altered by the alternative form
|\childdocby[*]{|\textit{main}|}| (with a non-empty optional argument)
which uses the |.aux| file of the main document
by setting |\jobname| to \textit{main}.

%%%%%%%%%%%%%%%%%%%%%%%%%%%%%%%%%%%%%%%%%%%%%%%%%%%%%%%%%%%%%%%%%%%%%%%%%%%%%%%%
\subsection{Driver Development}
\label{sec:driver}

The \textsf{childdoc} mechanism can also be use for the development
of definition files such as \LaTeX{} styles or classes.
This case differs from the above setup with multiple parts
included by |\include| in that no |\includeonly| should be invoked.
This can be achieved by starting the include file
(before |\ProvidesPackage|) with:
%
\begin{center}
\begin{tabular}{l}
|\input{childdoc.def}|\\
|\childdocforward{|\textit{main}|}|\\
\end{tabular}
\end{center}
%
or alternatively with:
%
\begin{center}
\begin{tabular}{l}
|\input{childdoc.def}|\\
|\childdocby{|\textit{main}|}|\\
\end{tabular}
\end{center}
%
Both forms have slightly different effects as described above.
The main file is prepared as usual, see \secref{sec:include}.

%%%%%%%%%%%%%%%%%%%%%%%%%%%%%%%%%%%%%%%%%%%%%%%%%%%%%%%%%%%%%%%%%%%%%%%%%%%%%%%%
\subsection{Legacy Detection}
\label{sec:detection}

The directive |\childdocmain| in the main file can detect
whether the complete document or merely a child is to be compiled
even without using the directive |\childdocof|.
This method is deprecated because it is less robust
and there is no compelling reason to use it;
it is merely provided for backward compatibility
and it may be removed in future versions.

If the detection mechanism is to be used,
it is mandatory to correctly specify
the filename of the main file as the argument of |\childdocmain|:
%
\begin{center}
\begin{tabular}{l}
|\input{childdoc.def}|\\
|\childdocmain{|\textit{main}|}|\\
\end{tabular}
\end{center}
%
If |\jobname| does not match the argument \textit{main} of |\childdocmain|,
it is assumed that |\jobname| points to the child file to be compiled.
When using |\childdocmain| with the main file specified as argument,
it suffices to start a child file
with just |\input{|\textit{main}|}|
without loading of the package and using |\childdocof|.
If instead all processing is done
with the appropriate \textsf{childdoc} directives,
the argument of \textit{main} of |\childdocmain| can be empty.

An alternative version of the command line processing described
in \secref{sec:commandline} using the detection mechanism reads:
%
\begin{center}
|... -jobname "|\textit{target}|" "|[\textit{flags}]%
[|\def\jobname{|\textit{dest}|}|]|\input{|\textit{main}|}"|
\end{center}

%%%%%%%%%%%%%%%%%%%%%%%%%%%%%%%%%%%%%%%%%%%%%%%%%%%%%%%%%%%%%%%%%%%%%%%%%%%%%%%%
\subsection{Manual Code}
\label{sec:manual}

In case one cannot be certain whether the definitions file |childdoc.def|
is installed on the target \TeX{} distribution
and one prefers not to ship it,
it is conceivable to paste a few relevant commands into the sources.

To that end, drop all statements |\input{childdoc.def}|
and perform the replacements as outlined below.
Instead of |\childdocmain{|\textit{main}|}| add the following code
to the top of the main file:
%
\begin{center}
\begin{tabular}{l}
|\||ifdefined\childdocname\endinput\||fi\newif\ifchilddoc|\\
|\edef\childdocname{\scantokens\expandafter{\jobname\noexpand}}|\\
|\def\childdocmain{|\textit{main}|}\||ifx\childdocmain\childdocname\||else|\\
|\childdoctrue\includeonly{\childdocname}\let\jobname\childdocmain\||fi|\\
\end{tabular}
\end{center}
%
Instead of |\childdocof{|\textit{main}|}| just include the main file
at the top of each child file:
%
\begin{center}
|\input{|\textit{main}|}|
\end{center}
%
A simple redirection |\childdocforward{|\textit{dest}|}| is achieved by:
%
\begin{center}
|\def\jobname{|\textit{dest}|}\input{\jobname}|
\end{center}
%
The redirection with prefix
|\childdocforwardprefix[|\textit{prefix}|]{|\textit{dest}|}|
is accomplished by:
%
\begin{center}
\begin{tabular}{l}
|{\edef\jobname{\scantokens\expandafter{\jobname\noexpand}}|\\
|\def\redirectjob |\textit{prefix}|#1~~~{\gdef\jobname{|\textit{dest}|#1}}|\\
|\expandafter\redirectjob\jobname~~~}\input{\jobname}|
\end{tabular}
\end{center}

In an alternative approach,
child documents can be compiled by a specific command line
without additional code or specific definitions:
%
\begin{center}
|... -jobname "|\textit{target}|" "|[\textit{flags}]%
|\includeonly{|\textit{dest}|}\input{|\textit{main}|}"|
\end{center}
%

%%%%%%%%%%%%%%%%%%%%%%%%%%%%%%%%%%%%%%%%%%%%%%%%%%%%%%%%%%%%%%%%%%%%%%%%%%%%%%%%
%%%%%%%%%%%%%%%%%%%%%%%%%%%%%%%%%%%%%%%%%%%%%%%%%%%%%%%%%%%%%%%%%%%%%%%%%%%%%%%%
\section{Information}

%%%%%%%%%%%%%%%%%%%%%%%%%%%%%%%%%%%%%%%%%%%%%%%%%%%%%%%%%%%%%%%%%%%%%%%%%%%%%%%%
\subsection{Copyright}

Copyright \copyright{} 2017--2018 Niklas Beisert

This work may be distributed and/or modified under the
conditions of the \LaTeX{} Project Public License, either version 1.3
of this license or (at your option) any later version.
The latest version of this license is in
  \url{http://www.latex-project.org/lppl.txt}
and version 1.3 or later is part of all distributions of \LaTeX{}
version 2005/12/01 or later.

This work has the LPPL maintenance status `maintained'.

The Current Maintainer of this work is Niklas Beisert.

This work consists of the files |README.txt|, |childdoc.ins| and |childdoc.dtx|
as well as the derived files |childdoc.def|, |cdocsamp.tex|
with |cdocsch1.tex|, |cdocsch2.tex|, |cdocspt3.tex|, |cdocspt4.tex|,
|cdocsdrf.tex|, |cdocsfn1.tex|, |cdocsfn2.tex|
as well as |childdoc.pdf|.

%%%%%%%%%%%%%%%%%%%%%%%%%%%%%%%%%%%%%%%%%%%%%%%%%%%%%%%%%%%%%%%%%%%%%%%%%%%%%%%%
\subsection{Files and Installation}

The package consists of the files:
%
\begin{center}
\begin{tabular}{ll}
    |README.txt|   & readme file \\
    |childdoc.ins| & installation file \\
    |childdoc.dtx| & source file \\
    |childdoc.def| & definition file \\
    |cdocsamp.tex| & sample main file \\
    |cdocsch1.tex| & sample include file \\
    |cdocsch2.tex| & sample include file \\
    |cdocspt3.tex| & sample part file \\
    |cdocspt4.tex| & sample part file \\
    |cdocsdrf.tex| & sample redirection file \\
    |cdocsfn1.tex| & sample redirection file \\
    |cdocsfn2.tex| & sample redirection file \\
    |childdoc.pdf| & manual
\end{tabular}
\end{center}
%
The distribution consists of the files
|README.txt|, |childdoc.ins| and |childdoc.dtx|.
%
\begin{itemize}
\item
Run (pdf)\LaTeX{} on |childdoc.dtx|
to compile the manual |childdoc.pdf| (this file).
\item
Run \LaTeX{} on |childdoc.ins| to create the definitions file |childdoc.def|
and the sample |cdocsamp.tex| with include files
|cdocsch1.tex|, |cdocsch2.tex|, |cdocspt3.tex|, |cdocspt4.tex|,
|cdocsdrf.tex|, |cdocsfn1.tex|, |cdocsfn2.tex|.
Then copy the file |childdoc.def| to an appropriate directory of your \LaTeX{}
distribution, e.g.\ \textit{texmf-root}|/tex/latex/childdoc|.
\end{itemize}

%%%%%%%%%%%%%%%%%%%%%%%%%%%%%%%%%%%%%%%%%%%%%%%%%%%%%%%%%%%%%%%%%%%%%%%%%%%%%%%%
\subsection{Related CTAN Packages}

There are several other packages which offer a similar functionality:
%
\begin{itemize}
\item
The packages
\href{http://ctan.org/pkg/docmute}{\textsf{docmute}},
\href{http://ctan.org/pkg/includex}{\textsf{includex}} and
\href{http://ctan.org/pkg/standalone}{\textsf{standalone}}
provide commands to include only the document body of
a child file thus allowing both files to be compiled individually.
\item
The packages \href{http://ctan.org/pkg/subdocs}{\textsf{subdocs}}
and \href{http://ctan.org/pkg/subfiles}{\textsf{subfiles}}
provide structures in which the main and child documents can be
encapsulated and allowing them to be compiled individually.
The inclusion mechanism is different from the conventional |\include|.
\item
The package \href{http://ctan.org/pkg/combine}{\textsf{combine}}
is an elaborate solution to combine several documents into one.
\end{itemize}
%
See also the CTAN topic \href{http://ctan.org/topic/subdocs}{\textsf{subdocs}}
for further related packages.
The present package differs from the above solutions in that
a document structure constructed with the conventional |\include| mechanism
just needs two extra commands at the top of every file
such that all constituent files can be compiled individually.

%%%%%%%%%%%%%%%%%%%%%%%%%%%%%%%%%%%%%%%%%%%%%%%%%%%%%%%%%%%%%%%%%%%%%%%%%%%%%%%%
%\subsection{Feature Suggestions}
%
%The following is a list of features which may be useful for future
%versions of this package:
%%
%\begin{itemize}
%\item
%\ldots
%\end{itemize}

%%%%%%%%%%%%%%%%%%%%%%%%%%%%%%%%%%%%%%%%%%%%%%%%%%%%%%%%%%%%%%%%%%%%%%%%%%%%%%%%
\subsection{Revision History}

%%%%%%%%%%%%%%%%%%%%%%%%%%%%%%%%%%%%%%%%
\paragraph{v2.0:} 2018/12/30

\begin{itemize}
\item
immediate forward processing
\item
added |\childdocby| mechanism
\item
manual restructured
\end{itemize}

%%%%%%%%%%%%%%%%%%%%%%%%%%%%%%%%%%%%%%%%
\paragraph{v1.6:} 2018/01/17

\begin{itemize}
\item
application for development of include files
\item
corrections to manual
\end{itemize}

%%%%%%%%%%%%%%%%%%%%%%%%%%%%%%%%%%%%%%%%
\paragraph{v1.5:} 2017/05/21

\begin{itemize}
\item
more complete structuring introduced
\item
|\childdocof| introduced
\item
|\childdoc| renamed to |\childdocmain|
\item
|\childredirect| renamed to |\childdocforward| and |\childdocforwardprefix|
and functionality expanded
\end{itemize}

%%%%%%%%%%%%%%%%%%%%%%%%%%%%%%%%%%%%%%%%
\paragraph{v1.0:} 2017/04/27

\begin{itemize}
\item
manual and install package
\item
first version published on CTAN
\end{itemize}

%%%%%%%%%%%%%%%%%%%%%%%%%%%%%%%%%%%%%%%%
\paragraph{v0.6:} 2017/04/26

\begin{itemize}
\item
redirection mechanism added
\end{itemize}

%%%%%%%%%%%%%%%%%%%%%%%%%%%%%%%%%%%%%%%%
\paragraph{v0.5:} 2017/04/26

\begin{itemize}
\item
functionality in definition file
\end{itemize}


%%%%%%%%%%%%%%%%%%%%%%%%%%%%%%%%%%%%%%%%%%%%%%%%%%%%%%%%%%%%%%%%%%%%%%%%%%%%%%%%
%%%%%%%%%%%%%%%%%%%%%%%%%%%%%%%%%%%%%%%%%%%%%%%%%%%%%%%%%%%%%%%%%%%%%%%%%%%%%%%%
%%%%%%%%%%%%%%%%%%%%%%%%%%%%%%%%%%%%%%%%%%%%%%%%%%%%%%%%%%%%%%%%%%%%%%%%%%%%%%%%
\appendix

\settowidth\MacroIndent{\rmfamily\scriptsize 000\ }

 \DocInput{childdoc.dtx}

\end{document}
%</driver>
% \fi
%
% %%%%%%%%%%%%%%%%%%%%%%%%%%%%%%%%%%%%%%%%%%%%%%%%%%%%%%%%%%%%%%%%%%%%%%%%%%%%%%
% %%%%%%%%%%%%%%%%%%%%%%%%%%%%%%%%%%%%%%%%%%%%%%%%%%%%%%%%%%%%%%%%%%%%%%%%%%%%%%
% \section{Sample}
%\iffalse
%<*samplemain>
%\fi
%
% The following presents a sample document
% with two chapters, two parts, a title page,
% a compile flag as well as three forwarding files to set the flag.
% It consists of eight |.tex| files:
% \begin{center}
% \begin{tabular}{ll}
% |cdocsamp.tex|&main file\\
% |cdocsch1.tex|&include file for chapter 1\\
% |cdocsch2.tex|&include file for chapter 2\\
% |cdocspt3.tex|&include file for part 3\\
% |cdocspt4.tex|&include file for part 4\\
% |cdocsdrf.tex|&forwarding file for main file in draft mode\\
% |cdocsfi1.tex|&forwarding file for final version of chapter 1\\
% |cdocsfi2.tex|&forwarding file for final version of chapter 2\\
% \end{tabular}
% \end{center}
% Each of the eight files can be compiled directly by the \LaTeX{} compiler.
%
% %%%%%%%%%%%%%%%%%%%%%%%%%%%%%%%%%%%%%%
% \paragraph{Main File.}
%
% The main file is called |cdocsamp.tex|.
%
% Load the \textsf{childdoc} definitions and
% declare the filename for the main document:
%    \begin{macrocode}
\input{childdoc.def}
\childdocmain{}
%    \end{macrocode}

% Optional override for |\version| flag:
%    \begin{macrocode}
%%\ifchilddoc\else\providecommand{\version}{draft}\fi
%    \end{macrocode}

% Define the default values for the |\version| flag
% (|final| for the main file and |draft| for childs):
%    \begin{macrocode}
\ifchilddoc
\providecommand{\version}{draft}
\else
\providecommand{\version}{final}
\fi
%    \end{macrocode}

% Load the standard document class:
%    \begin{macrocode}
\documentclass[12pt]{article}
%    \end{macrocode}

% Start the document body:
%    \begin{macrocode}
\begin{document}
%    \end{macrocode}

% Declare a title page.
% Print title, part of document being processed and version flag:
%    \begin{macrocode}
\addtocounter{page}{-1}
\begin{center}
{\LARGE\bfseries{}childdoc example\par}
\vspace{1cm}
\ifchilddoc
\ifchilddocmanual part\else chapter\fi:
`\childdocname' of `\childdocjob'\par
\else
main document: `\childdocjob'\par
\fi
version: \version\par
\end{center}
\newpage
%    \end{macrocode}

% Manually include selected file,
% otherwise process as usual:
%    \begin{macrocode}
\ifchilddocmanual
\section*{part `\childdocname'}
\input{\childdocname}
\else
%    \end{macrocode}

% Include the two chapters:
%    \begin{macrocode}
\include{cdocsch1}
\include{cdocsch2}
%    \end{macrocode}

% Include the two parts unless only chapters should be displayed:
%    \begin{macrocode}
\ifchilddoc\else
\section{part three}
\input{cdocspt3}
\section{part four}
\input{cdocspt4}
\fi
%    \end{macrocode}

% Process as usual until here:
%    \begin{macrocode}
\fi
%    \end{macrocode}

% End of document body:
%    \begin{macrocode}
\end{document}
%    \end{macrocode}
%\iffalse
%</samplemain>
%\fi
%
% %%%%%%%%%%%%%%%%%%%%%%%%%%%%%%%%%%%%%%
% \paragraph{Chapter Include Files.}
%
% The include files are called |cdocsch1.tex| and |cdocsch2.tex|.
%
%\iffalse
%<*samplechap1|samplechap2>
%\fi

% Optional override for |\version| flag:
%    \begin{macrocode}
%%\providecommand{\version}{final}
%    \end{macrocode}

% Include the main document:
%    \begin{macrocode}
\input{childdoc.def}
\childdocof{cdocsamp}
%    \end{macrocode}

%\iffalse
%</samplechap1|samplechap2>
%\fi
%
%\iffalse
%<*samplechap1>
%\fi
% Some text for chapter 1:
%    \begin{macrocode}
\section{one}
some text in chapter one
%    \end{macrocode}

%\iffalse
%</samplechap1>
%\fi
% Some text for chapter 2:
%\iffalse
%<*samplechap2>
%\fi
%    \begin{macrocode}
\section{two}
more text in chapter two
%    \end{macrocode}

%\iffalse
%</samplechap2>
%\fi
%
% %%%%%%%%%%%%%%%%%%%%%%%%%%%%%%%%%%%%%%
% \paragraph{Part Include Files.}
%
% The include files are called |cdocspt3.tex| and |cdocspt4.tex|.
%
%\iffalse
%<*samplepart3|samplepart4>
%\fi

% Optional override for |\version| flag:
%    \begin{macrocode}
%%\providecommand{\version}{final}
%    \end{macrocode}

% Include the main document:
%    \begin{macrocode}
\input{childdoc.def}
\childdocby{cdocsamp}
%    \end{macrocode}

%\iffalse
%</samplepart3|samplepart4>
%\fi
%
%\iffalse
%<*samplepart3>
%\fi
% Some text for part 3:
%    \begin{macrocode}
some text in part three
%    \end{macrocode}

%\iffalse
%</samplepart3>
%\fi
% Some text for part 4:
%\iffalse
%<*samplepart4>
%\fi
%    \begin{macrocode}
more text in part four
%    \end{macrocode}

%\iffalse
%</samplepart4>
%\fi
%
% %%%%%%%%%%%%%%%%%%%%%%%%%%%%%%%%%%%%%%
% \paragraph{Forwarding for a Complete Draft.}
%
% The following forwarding file |cdocsdrf.tex|
% compiles the main document in draft mode:
%\iffalse
%<*sampledraft>
%\fi
%    \begin{macrocode}
\def\version{draft}
\input{childdoc.def}
\childdocforward{cdocsamp}
%    \end{macrocode}

%\iffalse
%</sampledraft>
%\fi
%
% %%%%%%%%%%%%%%%%%%%%%%%%%%%%%%%%%%%%%%
% \paragraph{Forwarding for Final Version of the Chapters.}
%
% The following forwarding files |cdocsfn1.tex| and |cdocsfn2.tex|
% (with identical content)
% compile the final versions of the child documents
% |cdocsch1.tex| and |cdocsch2.tex|, respectively:
%\iffalse
%<*samplefinal>
%\fi
%    \begin{macrocode}
\def\version{final}
\input{childdoc.def}
\childdocforwardprefix[cdocsamp]{cdocsfn}{cdocsch}
%    \end{macrocode}

%\iffalse
%</samplefinal>
%\fi
%
% %%%%%%%%%%%%%%%%%%%%%%%%%%%%%%%%%%%%%%
% \paragraph{Command Line Processing.}
%
% The following three command lines generate the output files
% |cdocscld|, |cdocscl1| and |cdocscl2|
% which should be identical to
% |cdocsdrf|, |cdocsch1| and |cdocsfn2|, respectively:
% \begin{center}
% \begin{tabular}{l}
% |latex -jobname cdocscld \|\\
% |  "\def\version{draft}\input{childdoc.def}\childdocforward{cdocsamp}"|\\
% |latex -jobname cdocscl1 \|\\
% |  "\input{childdoc.def}\childdocforward[cdocsamp]{cdocsch1}"|\\
% |latex -jobname cdocscl2 \|\\
% |  "\def\version{final}\input{childdoc.def}\childdocforward{cdocsch2}"|
% \end{tabular}
% \end{center}
% Note that the trailing backslash on each first line
% merely continues the input to the second line
% (for convenient cut ant paste).
% Furthermore, the command |latex| can be replaced by any
% of its alternative versions such as |pdflatex|.
%
% %%%%%%%%%%%%%%%%%%%%%%%%%%%%%%%%%%%%%%%%%%%%%%%%%%%%%%%%%%%%%%%%%%%%%%%%%%%%%%
% %%%%%%%%%%%%%%%%%%%%%%%%%%%%%%%%%%%%%%%%%%%%%%%%%%%%%%%%%%%%%%%%%%%%%%%%%%%%%%
% \section{Implementation}
%\iffalse
%<*package>
%\fi
%
% This section describes the definitions file |childdoc.def|.

% The definitions cannot be loaded using |\usepackage| or |\RequirePackage|
% which has a mechanism to prevent loading a style file more than once.
% When loading the definitions by means of |\input|
% multiple instances have to be prevented manually:
%\iffalse
%This code needs to be before the `\ProvidesFile' directive
%which is defined at the beginning of this file.
%Therefore it is also placed there and commented out here.
%</package>
%<*discard>
%\fi
%    \begin{macrocode}
\ifdefined\childdocmain\endinput\fi
%    \end{macrocode}
%\iffalse
%</discard>
%<*package>
%\fi
%
% \macro{\ifchilddoc}
% \macro{\ifchilddocmanual}
% The conditional |\ifchilddoc| tells whether a
% child (true) or main (false) document is being compiled.
% The conditional |\ifchilddocmanual| tells whether
% the |\includeonly| mechanism is used (false) or
% the selection of child files must be performed manually (true).
% The definitions initialise to false:
%    \begin{macrocode}
\newif\ifchilddoc
\newif\ifchilddocmanual
%    \end{macrocode}

% \macro{\childdocname}
% \macro{\childdocjob}
% The macro |\childdocname| stores the name of the main document
% to be compiled. The macro |\childdocjob| stores the name of
% the document on which the \LaTeX{} compiler was originally invoked.
% The content of |\jobname| cannot be compared
% to filenames specified in the source due to different catcodes.
% The following code rescans |\jobname|, stores the result
% in |\childdocname| and saves a copy in |\childdocjob|:
%    \begin{macrocode}
\edef\childdocname{\scantokens\expandafter{\jobname\noexpand}}
\let\childdocjob\childdocname
%    \end{macrocode}

% \macro{\childdocdisable}
% The macro |\childdocdisable| prevents the main file
% from being processed more than once.
% At this stage, the main document command |\childdocmain|
% is assumed to be called once again where it should do nothing.
% Any subsequent call to it should prevent
% a secondary processing of the main document
% It overwrites the forwarding commands
% |\childdocof| and |\childdocforward|
% with empty macros to prevent further inclusions of the main document:
%    \begin{macrocode}
\newcommand{\childdocdisable}
{
  \renewcommand{\childdocmain}[1]{\renewcommand{\childdocmain}[1]{\endinput}}
  \renewcommand{\childdocof}[1]{}
  \renewcommand{\childdocby}[2][]{}
  \renewcommand{\childdocforward}[2][]{}
  \renewcommand{\childdocdisable}{}
}
%    \end{macrocode}

% \macro{\childdocmain}
% The macro |\childdocmain| is to be called at the top of the main file
% with nothing or the main filename (without extension) as argument.
% First, it breaks loops.
% If the argument is not empty and does not match |\childdocname|
% (which is set by the first inclusion of |childdoc.def|),
% |\ifchilddoc| is set to true, |\includeonly| is applied to the child file
% and |\jobname| is set to the main file
% (for proper handling of |.aux| files):
%    \begin{macrocode}
\newcommand{\childdocmain}[1]
{
  \childdocdisable\childdocmain{}
  \if?#1?\else
    \begingroup
      \def\childdoctmp{#1}
      \ifx\childdoctmp\childdocname
        \def\childdoctmp{}
      \else
        \def\childdoctmp
        {
          \childdoctrue
          \includeonly{\childdocname}
          \def\childdocjob{#1}
          \def\jobname{#1}
        }
      \fi
      \expandafter
    \endgroup
    \childdoctmp
  \fi
}
%    \end{macrocode}

% \macro{\childdocof}
% The command |\childdocof| redirects
% compilation to the main file |#1|.
%    \begin{macrocode}
\newcommand{\childdocof}[1]
{
  \childdocdisable
  \childdoctrue
  \includeonly{\childdocname}
  \def\jobname{#1}
  \def\childdocjob{#1}
  \input{#1}
}
%    \end{macrocode}

% \macro{\childdocby}
% The command |\childdocby| ....
%    \begin{macrocode}
\newcommand{\childdocby}[2][]
{
  \childdocdisable
  \childdoctrue
  \childdocmanualtrue
  \if?#1?\else
    \def\jobname{#2}
  \fi
  \def\childdocjob{#2}
  \input{#2}
  \endinput
}
%    \end{macrocode}

% \macro{\childdocforward}
% The command |\childdocforward| redirects
% compilation to the main file or
% (if the optional argument is given) a child file.
% Parameters are set as if the main file
% or a child file starting with |\childdocof| was compiled.
% Then compilation is handed over to the main file:
%    \begin{macrocode}
\newcommand{\childdocforward}[2][]
{
  \begingroup
    \if?#1?
      \def\childdoctmp
      {
        \def\childdocname{#2}
        \def\childdocjob{#2}
        \def\jobname{#2}
        \input{#2}
        \endinput
      }
    \else
      \def\childdoctmp
      {
        \childdocdisable
        \def\childdocname{#2}
        \childdoctrue
        \includeonly{#2}
        \def\childdocjob{#1}
        \def\jobname{#1}
        \input{#1}
        \endinput
      }
    \fi
    \expandafter
  \endgroup
  \childdoctmp
}
%    \end{macrocode}

% \macro{\childdocforwardprefix}
% The command |\childdocforwardprefix| redirects
% compilation to the main or a child file by means of a pattern.
% The prefix |#1| in the current filename is replaced by |#2|
% and the suffix of the current filename is kept
% (it is assumed that the filename does not contain the substring `|~~~|'
% which is used as a delimiter).
% Compilation is handed over to the new file by |\childdocforward|:
%    \begin{macrocode}
\newcommand{\childdocforwardprefix}[3][]
{
  \begingroup
    \def\childdocextract #2##1~~~{\def\childdoctmp{\childdocforward[#1]{#3##1}}}
    \expandafter\childdocextract\childdocname~~~
    \expandafter
  \endgroup
  \childdoctmp
}
%    \end{macrocode}

% \macro{\childdoc}
% The deprecated macro |\childdoc| is a legacy version of |\childdocmain|:
%    \begin{macrocode}
\newcommand{\childdoc}{\childdocmain}
%    \end{macrocode}

% \macro{\childdocredirect}
% The deprecated macro |\childdocredirect| is a legacy version
% of |\childdocforward| and |\childdocforwardprefix|:
%    \begin{macrocode}
\newcommand{\childdocredirect}[2][]
{
  \begingroup
    \if?#1?
      \def\childdoctmp{\childdocforward{#2}}
    \else
      \def\childdoctmp{\childdocforwardprefix{#1}{#2}}
    \fi
    \expandafter
  \endgroup
  \childdoctmp
}
%    \end{macrocode}

%\iffalse
%</package>
%\fi
%
\endinput
|\\
|\childdocmain{}|\\
\end{tabular}
\end{center}
at the very top of the main \LaTeX{} file,
in particular \emph{before} the |\documentclass| statement!
The argument of |\childdocmain| should be left empty
(but it must be present).

%%%%%%%%%%%%%%%%%%%%%%%%%%%%%%%%%%%%%%%%
\DescribeMacro{\childdocof}
Furthermore, add the commands
\begin{center}
\begin{tabular}{l}
|% \iffalse
%
% childdoc.dtx Copyright (C) 2017-2018 Niklas Beisert
%
% This work may be distributed and/or modified under the
% conditions of the LaTeX Project Public License, either version 1.3
% of this license or (at your option) any later version.
% The latest version of this license is in
%   http://www.latex-project.org/lppl.txt
% and version 1.3 or later is part of all distributions of LaTeX
% version 2005/12/01 or later.
%
% This work has the LPPL maintenance status `maintained'.
%
% The Current Maintainer of this work is Niklas Beisert.
%
% This work consists of the files childdoc.dtx and childdoc.ins
% and the derived files childdoc.def and cdocsamp.tex with
% cdocsch1.tex, cdocsch2.tex, cdocsdrf.tex, cdocsfn1.tex, cdocsfn2.tex.
%
%<package>\ifdefined\childdocmain\endinput\fi
%<package>\ProvidesFile{childdoc.def}[2018/12/30 v2.0 child document driver]
%<samplemain>\ProvidesFile{cdocsamp.tex}[2018/12/30 v2.0 sample for childdoc]
%<*driver>
%\ProvidesFile{childdoc.drv}[2018/12/30 v2.0 childdoc reference manual file]
\PassOptionsToClass{10pt,a4paper}{article}
\documentclass{ltxdoc}

\usepackage[margin=35mm]{geometry}
\usepackage{hyperref}
\usepackage{hyperxmp}
\usepackage[usenames]{color}

\hypersetup{colorlinks=true}
\hypersetup{pdfstartview=FitH}
\hypersetup{pdfpagemode=UseNone}
\hypersetup{pdfsource={}}
\hypersetup{pdflang={en-UK}}
\hypersetup{pdfcopyright={Copyright 2017-2018 Niklas Beisert.
  This work may be distributed and/or modified under the
  conditions of the LaTeX Project Public License, either version 1.3
  of this license or (at your option) any later version.}}
\hypersetup{pdflicenseurl={http://www.latex-project.org/lppl.txt}}
\hypersetup{pdfcontactaddress={ETH Zurich, ITP, HIT K,
  Wolfgang-Pauli-Strasse 27}}
\hypersetup{pdfcontactpostcode={8093}}
\hypersetup{pdfcontactcity={Zurich}}
\hypersetup{pdfcontactcountry={Switzerland}}
\hypersetup{pdfcontactemail={nbeisert@itp.phys.ethz.ch}}
\hypersetup{pdfcontacturl={http://people.phys.ethz.ch/\xmptilde nbeisert/}}

\newcommand{\secref}[1]{\hyperref[#1]{section \ref*{#1}}}

\parskip1ex
\parindent0pt
\let\olditemize\itemize
\def\itemize{\olditemize\parskip0pt}

\begin{document}

\title{The \textsf{childdoc} Package}
\hypersetup{pdftitle={The childdoc Package}}
\author{Niklas Beisert\\[2ex]
  Institut f\"ur Theoretische Physik\\
  Eidgen\"ossische Technische Hochschule Z\"urich\\
  Wolfgang-Pauli-Strasse 27, 8093 Z\"urich, Switzerland\\[1ex]
  \href{mailto:nbeisert@itp.phys.ethz.ch}
  {\texttt{nbeisert@itp.phys.ethz.ch}}}
\hypersetup{pdfauthor={Niklas Beisert}}
\hypersetup{pdfsubject={Manual for the LaTeX2e Package childdoc}}
\date{30 December 2018, \textsf{v2.0}}
\maketitle

\begin{abstract}\noindent
\textsf{childdoc} is a \LaTeXe{} package
that enables the direct compilation
of document sections included by |\include|
to individual files.
\end{abstract}

\begingroup
\parskip0ex
\tableofcontents
\endgroup

%%%%%%%%%%%%%%%%%%%%%%%%%%%%%%%%%%%%%%%%%%%%%%%%%%%%%%%%%%%%%%%%%%%%%%%%%%%%%%%%
%%%%%%%%%%%%%%%%%%%%%%%%%%%%%%%%%%%%%%%%%%%%%%%%%%%%%%%%%%%%%%%%%%%%%%%%%%%%%%%%
\section{Introduction}

\LaTeX{} provides a mechanism to structure a large document (such as a book)
into a main file and several child files (containing the chapters)
using the |\include| command.
This mechanism is beneficial for documents
which span hundreds of pages in order to
make the source file(s) more manageable.
Moreover, compilation can be restricted to
selected child files by means of the |\includeonly| command.
The latter feature can be used to reduce the compilation time while editing
(this was significantly more useful in the earlier days of \LaTeX{})
or to generate a smaller document which is easier to navigate.
Another application of |\includeonly| is to generate
documents consisting of selected parts of the complete document.

However, there are a few drawbacks of the plain |\include| mechanism:
\begin{itemize}
\item
The child files cannot be compiled on their own,
they can only be compiled via the main file.
A naive editing environment
(such as a text editor with an option
to have the current file processed by \LaTeX)
may require one to switch to the main file before compiling;
attempting to compile the child file produces errors.
\item
The main file must be modified (each time)
to adjust the |\includeonly| command
to the present needs. This easily leaves the main file in a messy state.
\item
The generated document will always carry the filename
of the main document. This is inconvenient if
several child files are to be compiled and
to be kept for distribution.
\end{itemize}

The present package provides a simple interface
to make child files individually compilable by \LaTeX{}.
Compiling a child file then has the same effect as compiling
the main file with an |\includeonly| command
to select the appropriate child.
Moreover the generated document will carry the name of the child
rather than the main file.
This resolves all three above issues.

This feature is meant to make the editing of books,
thesis documents and lecture notes somewhat more convenient.
However, the package can also be used efficiently for
composing a series of documents (such as exercise sheets)
which are typically distributed individually.
It then assists the author in generating the individual documents
(potentially in different versions)
as well as a document containing the collected series.
Another application is in developing style files
or other kinds of included material
where compilation of the style file could redirect
to a sample or test file.

%%%%%%%%%%%%%%%%%%%%%%%%%%%%%%%%%%%%%%%%%%%%%%%%%%%%%%%%%%%%%%%%%%%%%%%%%%%%%%%%
%%%%%%%%%%%%%%%%%%%%%%%%%%%%%%%%%%%%%%%%%%%%%%%%%%%%%%%%%%%%%%%%%%%%%%%%%%%%%%%%
\section{Usage}

First of all, the package \textsf{childdoc} is \emph{not} a standard
\LaTeXe{} |.sty| style file! Therefore it needs to be invoked in
a non-standard way.

%%%%%%%%%%%%%%%%%%%%%%%%%%%%%%%%%%%%%%%%%%%%%%%%%%%%%%%%%%%%%%%%%%%%%%%%%%%%%%%%
\subsection{Included Files}
\label{sec:include}

%%%%%%%%%%%%%%%%%%%%%%%%%%%%%%%%%%%%%%%%
\DescribeMacro{\childdocmain}
To use the package, add the commands
\begin{center}
\begin{tabular}{l}
|\input{childdoc.def}|\\
|\childdocmain{}|\\
\end{tabular}
\end{center}
at the very top of the main \LaTeX{} file,
in particular \emph{before} the |\documentclass| statement!
The argument of |\childdocmain| should be left empty
(but it must be present).

%%%%%%%%%%%%%%%%%%%%%%%%%%%%%%%%%%%%%%%%
\DescribeMacro{\childdocof}
Furthermore, add the commands
\begin{center}
\begin{tabular}{l}
|\input{childdoc.def}|\\
|\childdocof{|\textit{main}|}|\\
\end{tabular}
\end{center}
at the top of every child file \textit{child}
which is included by |\include{|\textit{child}|}|
from within the main file
(or at least for those files to be compiled individually).
The argument \textit{main} must be the filename of the main file.

There are a couple of
considerations in setting up the main and child documents:

%%%%%%%%%%%%%%%%%%%%%%%%%%%%%%%%%%%%%%%%
\paragraph{Restrictions.}

Please note the following restrictions:
\begin{itemize}
\item
|\childdocmain| must be called with one argument \textit{main}
to ensure compatibility with earlier version of the package.
It must either be empty (|\childdocmain{}|)
or precisely match the filename of the main file in which it is specified.
See \secref{sec:detection} for further information.
\item
The filename \textit{main} must be specified without the |.tex| extension.
\item
The filename \textit{main} is case sensitive
(even in case-insensitive file systems)
due to internal string comparison.
\item
The argument \textit{main} should be fully expanded, it cannot be a macro.
\item
Subdirectories and special characters should be avoided in filenames.
\item
The command |\childdocmain{|\textit{main}|}| must be followed by a whitespace.
It should not be followed immediately by another command
or by a comment mark `|%|'.
This is because the \TeX{} parser reads the token immediately following
the argument of |\childdocmain| and puts it
at the beginning of every child section;
however, a white\-space is ignored.
\end{itemize}

%%%%%%%%%%%%%%%%%%%%%%%%%%%%%%%%%%%%%%%%
\paragraph{Content of Main File.}

It is advisable to place all content in the child files included by |\include|.
Any output contained in the main file will appear in all child documents
unless suppressed manually;
it cannot be suppressed automatically by the |\includeonly| directive
and thus should normally be avoided.
A method to include some content in the main file
by means of conditional processing is described in \secref{sec:conditional}.

%%%%%%%%%%%%%%%%%%%%%%%%%%%%%%%%%%%%%%%%
\paragraph{Page Numbering.}

When only a part of the document is compiled,
the appropriate numbering of pages
(as well as other status parameters)
is determined from the |.aux| files.
The latter contain information from previous passes.
However this information needs to propagate through
all intermediate child documents.
Therefore the page numbering in child documents may well
be inconsistent until the complete document is compiled at least once.

A useful (if unconventional) way to always ensure a consistent
page numbering is to restart the numbering in each child document
and denote the pages by `\textit{child}|.|\textit{page}'
where \textit{child} represents the chapter/section number of the child file.
This can be achieved by the command
|\numberwithin{page}{|\textit{child}|}|
of the \textsf{amsmath} package
where \textit{child} can be |chapter| or |section|
depending on the chosen structuring.
Alternatively, one can modify the macro |\thepage| appropriately
and reset the counter |page| at the start of each child file.

%%%%%%%%%%%%%%%%%%%%%%%%%%%%%%%%%%%%%%%%%%%%%%%%%%%%%%%%%%%%%%%%%%%%%%%%%%%%%%%%
\subsection{Conditional Processing}
\label{sec:conditional}

The package provides a mechanism to compile different versions
of a document. To customise the versions further some conditional processing
can come in handy to distinguish which version is being compiled.
The package provides two macros to describe the compilation context:

%%%%%%%%%%%%%%%%%%%%%%%%%%%%%%%%%%%%%%%%
\DescribeMacro{\ifchilddoc}
The conditional |\ifchilddoc| distinguishes between the compilation of
child documents and the main document:
%
\begin{center}
|\ifchilddoc |\textit{child-code}| |[|\||else |\textit{main-code}]| \||fi|
\end{center}

%%%%%%%%%%%%%%%%%%%%%%%%%%%%%%%%%%%%%%%%
\DescribeMacro{\childdocname}
\DescribeMacro{\childdocjob}
The macro |\childdocname| contains the filename (without extension)
of the main or child file being processed.
Note that |\childdocjob| will always contain the name of the main file.

%%%%%%%%%%%%%%%%%%%%%%%%%%%%%%%%%%%%%%%%
\paragraph{Title Page.}

Conditional processing can be used to include a title or banner page
in the main document when proper precautions are taken.
Importantly, the code in the main file should ensure that the page counter
(as well as other status parameters which are stored in the |.aux| files)
takes the same value after the conditional processing.
Otherwise the page numbers may take divergent values
depending on which part is compiled.

For example, a title page could be declared by:
%
\begin{center}
\begin{tabular}{l}
|\ifchilddoc\||else|\\
|\addtocounter{page}{-1}|\\
\textit{code for title page}\\
|\newpage|\\
|\||fi|
\end{tabular}
\end{center}
%
A banner page for the child documents can be generated by:
%
\begin{center}
\begin{tabular}{l}
|\ifchilddoc|\\
|\addtocounter{page}{-1}|\\
\textit{code for banner page}\\
|\newpage|\\
|\||fi|
\end{tabular}
\end{center}
%
Here one could write a message such as:
\begin{center}
|This is the part \childdocname{} of \childdocjob{}.|
\end{center}

%%%%%%%%%%%%%%%%%%%%%%%%%%%%%%%%%%%%%%%%%%%%%%%%%%%%%%%%%%%%%%%%%%%%%%%%%%%%%%%%
\subsection{Flags}
\label{sec:flags}

The package makes it easy to generate different versions
of the main or child documents.
To this end compilation flags can be defined
and assigned different default values.
They will be particularly useful in conjunction
with the forwarding mechanism described in \secref{sec:forward}.

For example, it may be useful to have a flag |\version|
which can be set to |draft| or |final|.
The document source will contain some conditional code
depending on the value of |\version|.
Suppose further, the flag should default to |final| for the main file
and to |draft| for child files
which is a natural assignment for editing the document.
This is achieved by placing the following code
in the preamble of the main document
(below the |\childdocmain| directive):
%
\begin{center}
\begin{tabular}{l}
|\ifchilddoc|\\
|\providecommand{\version}{draft}|\\
|\||else|\\
|\providecommand{\version}{final}|\\
|\||fi|
\end{tabular}
\end{center}
%
The definition by |\providecommand| makes sure
that previous definitions are not overwritten.
Further statements |\providecommand{\version}{...}|
can thus be added before the above code to override it.

For the main file, one might add a line
(between |\childdocmain| and the above block)
%
\begin{center}
|%\ifchilddoc\||else\providecommand{\version}{draft}\||fi|
\end{center}
%
which can be uncommented to produce a draft version.
Likewise one can add a line to the very top of a child file
(above the |\childdocof{|\textit{main}|}| directive)
%
\begin{center}
|%\providecommand{\version}{final}|
\end{center}
%
which can be uncommented to produce the final version of this child document.

%%%%%%%%%%%%%%%%%%%%%%%%%%%%%%%%%%%%%%%%%%%%%%%%%%%%%%%%%%%%%%%%%%%%%%%%%%%%%%%%
\subsection{Forwarding}
\label{sec:forward}

Different versions of the main or child documents
using compilation flags as described in \secref{sec:flags}
can be (permanently) stored in different files
for convenient compilation, viewing and distribution.
To this end, the package defines a command
to pass on compilation to a different file:

%%%%%%%%%%%%%%%%%%%%%%%%%%%%%%%%%%%%%%%%
\DescribeMacro{\childdocforward}
The command |\childdocforward| redirects processing to
another source file:
%
\begin{center}
\begin{tabular}{l}
|\input{childdoc.def}|\\
|\childdocforward[|\textit{main}|]{|\textit{dest}|}|\\
\end{tabular}
\end{center}
%
The argument \textit{dest} is the destination file
(without extension).
It should be the main file or one of the child files.
Note that further \textsf{childdoc} directives
such as |\childdocof| and |\childdocforward|
in the indicated file will be processed in this form.
The optional argument \textit{main}
passes on directly to the main file \textit{main}
while pretending to compile the child \textit{dest}.
This form behaves as if \textit{dest}
issues |\childdocof{|\textit{main}|}| right away,
and no further \textsf{childdoc} directives will be processed.

%%%%%%%%%%%%%%%%%%%%%%%%%%%%%%%%%%%%%%%%
\DescribeMacro{\...prefix}
In the alternative form |\childdocforwardprefix|,
%
\begin{center}
\begin{tabular}{l}
|\input{childdoc.def}|\\
|\childdocforwardprefix[|\textit{main}|]{|\textit{prefix}|}{|\textit{dest}|}|
\end{tabular}
\end{center}
%
the destination file is determined by a pattern
depending on the current file:
To make this work, the current file must be called
`{\textit{prefix}\hspace{0.2em}\textit{suffix}}'
with \textit{prefix} matching precisely the argument.
Processing is then passed on to the file
`{\textit{dest}\hspace{0.2em}\textit{suffix}}'.
Surely, the same effect is achieved by
directly specifying the
argument `{\textit{dest}\hspace{0.2em}\textit{suffix}}'
in the first form.
However, that requires to set up a different file
for each child. With the alternative form of the command
all these files can have exactly the same content
which simplifies setting them up and maintaining them.

For example, the following file |draft.tex|
with a compilation flag |\version| as described in \secref{sec:flags}
compiles the main document as a draft:
%
\begin{center}
\begin{tabular}{l}
|\def\version{draft}|\\
|\input{childdoc.def}|\\
|\childdocforward{|\textit{main}|}|
\end{tabular}
\end{center}
%
Likewise, the following files |final|\textit{nn}|.tex|
compile the final version of the child document
|child|\textit{nn}|.tex|:
%
\begin{center}
\begin{tabular}{l}
|\def\version{final}|\\
|\input{childdoc.def}|\\
|\childdocforwardprefix{final}{child}|
\end{tabular}
\end{center}
%

Note that when several versions of a main file and/or of each child file
are to be generated, it may be convenient to set up a |Makefile| or
shell script to automatise the process.

%%%%%%%%%%%%%%%%%%%%%%%%%%%%%%%%%%%%%%%%%%%%%%%%%%%%%%%%%%%%%%%%%%%%%%%%%%%%%%%%
\subsection{Command Line Processing}
\label{sec:commandline}

The effect of redirection files can also be achieved by invoking
the \LaTeX{} compiler with a more elaborate command line.
Most conveniently this should be done as part
of a shell script or a |Makefile|.

When using \textsf{childdoc} in the main file, the following
command lines effectively perform a redirection
(note that depending on the shell being used,
backslashes may have to be doubled: `|\|' $\to$ `|\\|'):
%
\begin{center}
|... -jobname "|\textit{target}|" |\\|"|[\textit{flags}]%
|\input{childdoc.def}\childdocforward[|\textit{main}|]{|\textit{dest}|}"|
\end{center}
%
Here \textit{target} is the name of the output file,
\textit{main} is the name of the main file
and \textit{dest} is the name of the main or child file to be processed
(all filenames without extensions).
The optional argument \textit{main} can be omitted
if \textit{main} matches \textit{dest}.
Optionally, compilation \textit{flags} can be defined via |\def| commands.
This command line makes the \TeX{} engine believe
it is compiling the file \textit{target}
whose content is specified as the latter parameter.
The provided code then forwards the processing to
\textit{main} or \textit{dest} as described in \secref{sec:forward}.

%%%%%%%%%%%%%%%%%%%%%%%%%%%%%%%%%%%%%%%%%%%%%%%%%%%%%%%%%%%%%%%%%%%%%%%%%%%%%%%%
\subsection{Include by Input}
\label{sec:input}

Including child documents by |\include| has some restrictions by design.
Most notably, the content of a child document always occupies
its own set of pages; pages cannot be shared between child documents.
Usually, this behaviour makes perfect sense
because each child document contain an essential part of the document.
However, in some situations it may be desirable to compose
a document from a collection of parts
without having mandatory page breaks between then.
For this case, the package
provides a mechanism to include parts
by |\input| which can also be processed individually.
However, by construction this mechanism
requires manual handling of the content to be output.

%%%%%%%%%%%%%%%%%%%%%%%%%%%%%%%%%%%%%%%%
\DescribeMacro{\ifchilddocmanual}
The main file should be prepared as usual, see \secref{sec:include}.
However, the document body must make a distinction
between processing of an individual part and of the main document, e.g.:
%
\begin{center}
\begin{tabular}{l}
|\ifchilddocmanual|\\
|\input{\childdocname}|\\
|\||else|\\
\textit{document body with }|\input{|\textit{part}|}|\\
|\||fi|
\end{tabular}
\end{center}
%
The conditional |\ifchilddocmanual| is true whenever
a part to be included by |\input| is being compiled,
and the name of the part is stored in |\childdocname|.

%%%%%%%%%%%%%%%%%%%%%%%%%%%%%%%%%%%%%%%%
\DescribeMacro{\childdocby}
Each part to be included by |\input| should start with:
%
\begin{center}
\begin{tabular}{l}
|\input{childdoc.def}|\\
|\childdocby{|\textit{main}|}|\\
\end{tabular}
\end{center}
%
The directive |\childdocby| is similar to |\childdocof|
described in \secref{sec:include},
but the subsequent selection of content must be done manually.
To that end, both |\ifchilddoc| and |\ifchilddocmanual|
will be true upon processing of a part,
and the name of the part is stored in |\childdocname|.
Note that |\jobname| will be set to the filename of the current part
so that each part receives an individual |.aux| file
that does not interfere with the |.aux| file(s) of the main document.
This behaviour can be altered by the alternative form
|\childdocby[*]{|\textit{main}|}| (with a non-empty optional argument)
which uses the |.aux| file of the main document
by setting |\jobname| to \textit{main}.

%%%%%%%%%%%%%%%%%%%%%%%%%%%%%%%%%%%%%%%%%%%%%%%%%%%%%%%%%%%%%%%%%%%%%%%%%%%%%%%%
\subsection{Driver Development}
\label{sec:driver}

The \textsf{childdoc} mechanism can also be use for the development
of definition files such as \LaTeX{} styles or classes.
This case differs from the above setup with multiple parts
included by |\include| in that no |\includeonly| should be invoked.
This can be achieved by starting the include file
(before |\ProvidesPackage|) with:
%
\begin{center}
\begin{tabular}{l}
|\input{childdoc.def}|\\
|\childdocforward{|\textit{main}|}|\\
\end{tabular}
\end{center}
%
or alternatively with:
%
\begin{center}
\begin{tabular}{l}
|\input{childdoc.def}|\\
|\childdocby{|\textit{main}|}|\\
\end{tabular}
\end{center}
%
Both forms have slightly different effects as described above.
The main file is prepared as usual, see \secref{sec:include}.

%%%%%%%%%%%%%%%%%%%%%%%%%%%%%%%%%%%%%%%%%%%%%%%%%%%%%%%%%%%%%%%%%%%%%%%%%%%%%%%%
\subsection{Legacy Detection}
\label{sec:detection}

The directive |\childdocmain| in the main file can detect
whether the complete document or merely a child is to be compiled
even without using the directive |\childdocof|.
This method is deprecated because it is less robust
and there is no compelling reason to use it;
it is merely provided for backward compatibility
and it may be removed in future versions.

If the detection mechanism is to be used,
it is mandatory to correctly specify
the filename of the main file as the argument of |\childdocmain|:
%
\begin{center}
\begin{tabular}{l}
|\input{childdoc.def}|\\
|\childdocmain{|\textit{main}|}|\\
\end{tabular}
\end{center}
%
If |\jobname| does not match the argument \textit{main} of |\childdocmain|,
it is assumed that |\jobname| points to the child file to be compiled.
When using |\childdocmain| with the main file specified as argument,
it suffices to start a child file
with just |\input{|\textit{main}|}|
without loading of the package and using |\childdocof|.
If instead all processing is done
with the appropriate \textsf{childdoc} directives,
the argument of \textit{main} of |\childdocmain| can be empty.

An alternative version of the command line processing described
in \secref{sec:commandline} using the detection mechanism reads:
%
\begin{center}
|... -jobname "|\textit{target}|" "|[\textit{flags}]%
[|\def\jobname{|\textit{dest}|}|]|\input{|\textit{main}|}"|
\end{center}

%%%%%%%%%%%%%%%%%%%%%%%%%%%%%%%%%%%%%%%%%%%%%%%%%%%%%%%%%%%%%%%%%%%%%%%%%%%%%%%%
\subsection{Manual Code}
\label{sec:manual}

In case one cannot be certain whether the definitions file |childdoc.def|
is installed on the target \TeX{} distribution
and one prefers not to ship it,
it is conceivable to paste a few relevant commands into the sources.

To that end, drop all statements |\input{childdoc.def}|
and perform the replacements as outlined below.
Instead of |\childdocmain{|\textit{main}|}| add the following code
to the top of the main file:
%
\begin{center}
\begin{tabular}{l}
|\||ifdefined\childdocname\endinput\||fi\newif\ifchilddoc|\\
|\edef\childdocname{\scantokens\expandafter{\jobname\noexpand}}|\\
|\def\childdocmain{|\textit{main}|}\||ifx\childdocmain\childdocname\||else|\\
|\childdoctrue\includeonly{\childdocname}\let\jobname\childdocmain\||fi|\\
\end{tabular}
\end{center}
%
Instead of |\childdocof{|\textit{main}|}| just include the main file
at the top of each child file:
%
\begin{center}
|\input{|\textit{main}|}|
\end{center}
%
A simple redirection |\childdocforward{|\textit{dest}|}| is achieved by:
%
\begin{center}
|\def\jobname{|\textit{dest}|}\input{\jobname}|
\end{center}
%
The redirection with prefix
|\childdocforwardprefix[|\textit{prefix}|]{|\textit{dest}|}|
is accomplished by:
%
\begin{center}
\begin{tabular}{l}
|{\edef\jobname{\scantokens\expandafter{\jobname\noexpand}}|\\
|\def\redirectjob |\textit{prefix}|#1~~~{\gdef\jobname{|\textit{dest}|#1}}|\\
|\expandafter\redirectjob\jobname~~~}\input{\jobname}|
\end{tabular}
\end{center}

In an alternative approach,
child documents can be compiled by a specific command line
without additional code or specific definitions:
%
\begin{center}
|... -jobname "|\textit{target}|" "|[\textit{flags}]%
|\includeonly{|\textit{dest}|}\input{|\textit{main}|}"|
\end{center}
%

%%%%%%%%%%%%%%%%%%%%%%%%%%%%%%%%%%%%%%%%%%%%%%%%%%%%%%%%%%%%%%%%%%%%%%%%%%%%%%%%
%%%%%%%%%%%%%%%%%%%%%%%%%%%%%%%%%%%%%%%%%%%%%%%%%%%%%%%%%%%%%%%%%%%%%%%%%%%%%%%%
\section{Information}

%%%%%%%%%%%%%%%%%%%%%%%%%%%%%%%%%%%%%%%%%%%%%%%%%%%%%%%%%%%%%%%%%%%%%%%%%%%%%%%%
\subsection{Copyright}

Copyright \copyright{} 2017--2018 Niklas Beisert

This work may be distributed and/or modified under the
conditions of the \LaTeX{} Project Public License, either version 1.3
of this license or (at your option) any later version.
The latest version of this license is in
  \url{http://www.latex-project.org/lppl.txt}
and version 1.3 or later is part of all distributions of \LaTeX{}
version 2005/12/01 or later.

This work has the LPPL maintenance status `maintained'.

The Current Maintainer of this work is Niklas Beisert.

This work consists of the files |README.txt|, |childdoc.ins| and |childdoc.dtx|
as well as the derived files |childdoc.def|, |cdocsamp.tex|
with |cdocsch1.tex|, |cdocsch2.tex|, |cdocspt3.tex|, |cdocspt4.tex|,
|cdocsdrf.tex|, |cdocsfn1.tex|, |cdocsfn2.tex|
as well as |childdoc.pdf|.

%%%%%%%%%%%%%%%%%%%%%%%%%%%%%%%%%%%%%%%%%%%%%%%%%%%%%%%%%%%%%%%%%%%%%%%%%%%%%%%%
\subsection{Files and Installation}

The package consists of the files:
%
\begin{center}
\begin{tabular}{ll}
    |README.txt|   & readme file \\
    |childdoc.ins| & installation file \\
    |childdoc.dtx| & source file \\
    |childdoc.def| & definition file \\
    |cdocsamp.tex| & sample main file \\
    |cdocsch1.tex| & sample include file \\
    |cdocsch2.tex| & sample include file \\
    |cdocspt3.tex| & sample part file \\
    |cdocspt4.tex| & sample part file \\
    |cdocsdrf.tex| & sample redirection file \\
    |cdocsfn1.tex| & sample redirection file \\
    |cdocsfn2.tex| & sample redirection file \\
    |childdoc.pdf| & manual
\end{tabular}
\end{center}
%
The distribution consists of the files
|README.txt|, |childdoc.ins| and |childdoc.dtx|.
%
\begin{itemize}
\item
Run (pdf)\LaTeX{} on |childdoc.dtx|
to compile the manual |childdoc.pdf| (this file).
\item
Run \LaTeX{} on |childdoc.ins| to create the definitions file |childdoc.def|
and the sample |cdocsamp.tex| with include files
|cdocsch1.tex|, |cdocsch2.tex|, |cdocspt3.tex|, |cdocspt4.tex|,
|cdocsdrf.tex|, |cdocsfn1.tex|, |cdocsfn2.tex|.
Then copy the file |childdoc.def| to an appropriate directory of your \LaTeX{}
distribution, e.g.\ \textit{texmf-root}|/tex/latex/childdoc|.
\end{itemize}

%%%%%%%%%%%%%%%%%%%%%%%%%%%%%%%%%%%%%%%%%%%%%%%%%%%%%%%%%%%%%%%%%%%%%%%%%%%%%%%%
\subsection{Related CTAN Packages}

There are several other packages which offer a similar functionality:
%
\begin{itemize}
\item
The packages
\href{http://ctan.org/pkg/docmute}{\textsf{docmute}},
\href{http://ctan.org/pkg/includex}{\textsf{includex}} and
\href{http://ctan.org/pkg/standalone}{\textsf{standalone}}
provide commands to include only the document body of
a child file thus allowing both files to be compiled individually.
\item
The packages \href{http://ctan.org/pkg/subdocs}{\textsf{subdocs}}
and \href{http://ctan.org/pkg/subfiles}{\textsf{subfiles}}
provide structures in which the main and child documents can be
encapsulated and allowing them to be compiled individually.
The inclusion mechanism is different from the conventional |\include|.
\item
The package \href{http://ctan.org/pkg/combine}{\textsf{combine}}
is an elaborate solution to combine several documents into one.
\end{itemize}
%
See also the CTAN topic \href{http://ctan.org/topic/subdocs}{\textsf{subdocs}}
for further related packages.
The present package differs from the above solutions in that
a document structure constructed with the conventional |\include| mechanism
just needs two extra commands at the top of every file
such that all constituent files can be compiled individually.

%%%%%%%%%%%%%%%%%%%%%%%%%%%%%%%%%%%%%%%%%%%%%%%%%%%%%%%%%%%%%%%%%%%%%%%%%%%%%%%%
%\subsection{Feature Suggestions}
%
%The following is a list of features which may be useful for future
%versions of this package:
%%
%\begin{itemize}
%\item
%\ldots
%\end{itemize}

%%%%%%%%%%%%%%%%%%%%%%%%%%%%%%%%%%%%%%%%%%%%%%%%%%%%%%%%%%%%%%%%%%%%%%%%%%%%%%%%
\subsection{Revision History}

%%%%%%%%%%%%%%%%%%%%%%%%%%%%%%%%%%%%%%%%
\paragraph{v2.0:} 2018/12/30

\begin{itemize}
\item
immediate forward processing
\item
added |\childdocby| mechanism
\item
manual restructured
\end{itemize}

%%%%%%%%%%%%%%%%%%%%%%%%%%%%%%%%%%%%%%%%
\paragraph{v1.6:} 2018/01/17

\begin{itemize}
\item
application for development of include files
\item
corrections to manual
\end{itemize}

%%%%%%%%%%%%%%%%%%%%%%%%%%%%%%%%%%%%%%%%
\paragraph{v1.5:} 2017/05/21

\begin{itemize}
\item
more complete structuring introduced
\item
|\childdocof| introduced
\item
|\childdoc| renamed to |\childdocmain|
\item
|\childredirect| renamed to |\childdocforward| and |\childdocforwardprefix|
and functionality expanded
\end{itemize}

%%%%%%%%%%%%%%%%%%%%%%%%%%%%%%%%%%%%%%%%
\paragraph{v1.0:} 2017/04/27

\begin{itemize}
\item
manual and install package
\item
first version published on CTAN
\end{itemize}

%%%%%%%%%%%%%%%%%%%%%%%%%%%%%%%%%%%%%%%%
\paragraph{v0.6:} 2017/04/26

\begin{itemize}
\item
redirection mechanism added
\end{itemize}

%%%%%%%%%%%%%%%%%%%%%%%%%%%%%%%%%%%%%%%%
\paragraph{v0.5:} 2017/04/26

\begin{itemize}
\item
functionality in definition file
\end{itemize}


%%%%%%%%%%%%%%%%%%%%%%%%%%%%%%%%%%%%%%%%%%%%%%%%%%%%%%%%%%%%%%%%%%%%%%%%%%%%%%%%
%%%%%%%%%%%%%%%%%%%%%%%%%%%%%%%%%%%%%%%%%%%%%%%%%%%%%%%%%%%%%%%%%%%%%%%%%%%%%%%%
%%%%%%%%%%%%%%%%%%%%%%%%%%%%%%%%%%%%%%%%%%%%%%%%%%%%%%%%%%%%%%%%%%%%%%%%%%%%%%%%
\appendix

\settowidth\MacroIndent{\rmfamily\scriptsize 000\ }

 \DocInput{childdoc.dtx}

\end{document}
%</driver>
% \fi
%
% %%%%%%%%%%%%%%%%%%%%%%%%%%%%%%%%%%%%%%%%%%%%%%%%%%%%%%%%%%%%%%%%%%%%%%%%%%%%%%
% %%%%%%%%%%%%%%%%%%%%%%%%%%%%%%%%%%%%%%%%%%%%%%%%%%%%%%%%%%%%%%%%%%%%%%%%%%%%%%
% \section{Sample}
%\iffalse
%<*samplemain>
%\fi
%
% The following presents a sample document
% with two chapters, two parts, a title page,
% a compile flag as well as three forwarding files to set the flag.
% It consists of eight |.tex| files:
% \begin{center}
% \begin{tabular}{ll}
% |cdocsamp.tex|&main file\\
% |cdocsch1.tex|&include file for chapter 1\\
% |cdocsch2.tex|&include file for chapter 2\\
% |cdocspt3.tex|&include file for part 3\\
% |cdocspt4.tex|&include file for part 4\\
% |cdocsdrf.tex|&forwarding file for main file in draft mode\\
% |cdocsfi1.tex|&forwarding file for final version of chapter 1\\
% |cdocsfi2.tex|&forwarding file for final version of chapter 2\\
% \end{tabular}
% \end{center}
% Each of the eight files can be compiled directly by the \LaTeX{} compiler.
%
% %%%%%%%%%%%%%%%%%%%%%%%%%%%%%%%%%%%%%%
% \paragraph{Main File.}
%
% The main file is called |cdocsamp.tex|.
%
% Load the \textsf{childdoc} definitions and
% declare the filename for the main document:
%    \begin{macrocode}
\input{childdoc.def}
\childdocmain{}
%    \end{macrocode}

% Optional override for |\version| flag:
%    \begin{macrocode}
%%\ifchilddoc\else\providecommand{\version}{draft}\fi
%    \end{macrocode}

% Define the default values for the |\version| flag
% (|final| for the main file and |draft| for childs):
%    \begin{macrocode}
\ifchilddoc
\providecommand{\version}{draft}
\else
\providecommand{\version}{final}
\fi
%    \end{macrocode}

% Load the standard document class:
%    \begin{macrocode}
\documentclass[12pt]{article}
%    \end{macrocode}

% Start the document body:
%    \begin{macrocode}
\begin{document}
%    \end{macrocode}

% Declare a title page.
% Print title, part of document being processed and version flag:
%    \begin{macrocode}
\addtocounter{page}{-1}
\begin{center}
{\LARGE\bfseries{}childdoc example\par}
\vspace{1cm}
\ifchilddoc
\ifchilddocmanual part\else chapter\fi:
`\childdocname' of `\childdocjob'\par
\else
main document: `\childdocjob'\par
\fi
version: \version\par
\end{center}
\newpage
%    \end{macrocode}

% Manually include selected file,
% otherwise process as usual:
%    \begin{macrocode}
\ifchilddocmanual
\section*{part `\childdocname'}
\input{\childdocname}
\else
%    \end{macrocode}

% Include the two chapters:
%    \begin{macrocode}
\include{cdocsch1}
\include{cdocsch2}
%    \end{macrocode}

% Include the two parts unless only chapters should be displayed:
%    \begin{macrocode}
\ifchilddoc\else
\section{part three}
\input{cdocspt3}
\section{part four}
\input{cdocspt4}
\fi
%    \end{macrocode}

% Process as usual until here:
%    \begin{macrocode}
\fi
%    \end{macrocode}

% End of document body:
%    \begin{macrocode}
\end{document}
%    \end{macrocode}
%\iffalse
%</samplemain>
%\fi
%
% %%%%%%%%%%%%%%%%%%%%%%%%%%%%%%%%%%%%%%
% \paragraph{Chapter Include Files.}
%
% The include files are called |cdocsch1.tex| and |cdocsch2.tex|.
%
%\iffalse
%<*samplechap1|samplechap2>
%\fi

% Optional override for |\version| flag:
%    \begin{macrocode}
%%\providecommand{\version}{final}
%    \end{macrocode}

% Include the main document:
%    \begin{macrocode}
\input{childdoc.def}
\childdocof{cdocsamp}
%    \end{macrocode}

%\iffalse
%</samplechap1|samplechap2>
%\fi
%
%\iffalse
%<*samplechap1>
%\fi
% Some text for chapter 1:
%    \begin{macrocode}
\section{one}
some text in chapter one
%    \end{macrocode}

%\iffalse
%</samplechap1>
%\fi
% Some text for chapter 2:
%\iffalse
%<*samplechap2>
%\fi
%    \begin{macrocode}
\section{two}
more text in chapter two
%    \end{macrocode}

%\iffalse
%</samplechap2>
%\fi
%
% %%%%%%%%%%%%%%%%%%%%%%%%%%%%%%%%%%%%%%
% \paragraph{Part Include Files.}
%
% The include files are called |cdocspt3.tex| and |cdocspt4.tex|.
%
%\iffalse
%<*samplepart3|samplepart4>
%\fi

% Optional override for |\version| flag:
%    \begin{macrocode}
%%\providecommand{\version}{final}
%    \end{macrocode}

% Include the main document:
%    \begin{macrocode}
\input{childdoc.def}
\childdocby{cdocsamp}
%    \end{macrocode}

%\iffalse
%</samplepart3|samplepart4>
%\fi
%
%\iffalse
%<*samplepart3>
%\fi
% Some text for part 3:
%    \begin{macrocode}
some text in part three
%    \end{macrocode}

%\iffalse
%</samplepart3>
%\fi
% Some text for part 4:
%\iffalse
%<*samplepart4>
%\fi
%    \begin{macrocode}
more text in part four
%    \end{macrocode}

%\iffalse
%</samplepart4>
%\fi
%
% %%%%%%%%%%%%%%%%%%%%%%%%%%%%%%%%%%%%%%
% \paragraph{Forwarding for a Complete Draft.}
%
% The following forwarding file |cdocsdrf.tex|
% compiles the main document in draft mode:
%\iffalse
%<*sampledraft>
%\fi
%    \begin{macrocode}
\def\version{draft}
\input{childdoc.def}
\childdocforward{cdocsamp}
%    \end{macrocode}

%\iffalse
%</sampledraft>
%\fi
%
% %%%%%%%%%%%%%%%%%%%%%%%%%%%%%%%%%%%%%%
% \paragraph{Forwarding for Final Version of the Chapters.}
%
% The following forwarding files |cdocsfn1.tex| and |cdocsfn2.tex|
% (with identical content)
% compile the final versions of the child documents
% |cdocsch1.tex| and |cdocsch2.tex|, respectively:
%\iffalse
%<*samplefinal>
%\fi
%    \begin{macrocode}
\def\version{final}
\input{childdoc.def}
\childdocforwardprefix[cdocsamp]{cdocsfn}{cdocsch}
%    \end{macrocode}

%\iffalse
%</samplefinal>
%\fi
%
% %%%%%%%%%%%%%%%%%%%%%%%%%%%%%%%%%%%%%%
% \paragraph{Command Line Processing.}
%
% The following three command lines generate the output files
% |cdocscld|, |cdocscl1| and |cdocscl2|
% which should be identical to
% |cdocsdrf|, |cdocsch1| and |cdocsfn2|, respectively:
% \begin{center}
% \begin{tabular}{l}
% |latex -jobname cdocscld \|\\
% |  "\def\version{draft}\input{childdoc.def}\childdocforward{cdocsamp}"|\\
% |latex -jobname cdocscl1 \|\\
% |  "\input{childdoc.def}\childdocforward[cdocsamp]{cdocsch1}"|\\
% |latex -jobname cdocscl2 \|\\
% |  "\def\version{final}\input{childdoc.def}\childdocforward{cdocsch2}"|
% \end{tabular}
% \end{center}
% Note that the trailing backslash on each first line
% merely continues the input to the second line
% (for convenient cut ant paste).
% Furthermore, the command |latex| can be replaced by any
% of its alternative versions such as |pdflatex|.
%
% %%%%%%%%%%%%%%%%%%%%%%%%%%%%%%%%%%%%%%%%%%%%%%%%%%%%%%%%%%%%%%%%%%%%%%%%%%%%%%
% %%%%%%%%%%%%%%%%%%%%%%%%%%%%%%%%%%%%%%%%%%%%%%%%%%%%%%%%%%%%%%%%%%%%%%%%%%%%%%
% \section{Implementation}
%\iffalse
%<*package>
%\fi
%
% This section describes the definitions file |childdoc.def|.

% The definitions cannot be loaded using |\usepackage| or |\RequirePackage|
% which has a mechanism to prevent loading a style file more than once.
% When loading the definitions by means of |\input|
% multiple instances have to be prevented manually:
%\iffalse
%This code needs to be before the `\ProvidesFile' directive
%which is defined at the beginning of this file.
%Therefore it is also placed there and commented out here.
%</package>
%<*discard>
%\fi
%    \begin{macrocode}
\ifdefined\childdocmain\endinput\fi
%    \end{macrocode}
%\iffalse
%</discard>
%<*package>
%\fi
%
% \macro{\ifchilddoc}
% \macro{\ifchilddocmanual}
% The conditional |\ifchilddoc| tells whether a
% child (true) or main (false) document is being compiled.
% The conditional |\ifchilddocmanual| tells whether
% the |\includeonly| mechanism is used (false) or
% the selection of child files must be performed manually (true).
% The definitions initialise to false:
%    \begin{macrocode}
\newif\ifchilddoc
\newif\ifchilddocmanual
%    \end{macrocode}

% \macro{\childdocname}
% \macro{\childdocjob}
% The macro |\childdocname| stores the name of the main document
% to be compiled. The macro |\childdocjob| stores the name of
% the document on which the \LaTeX{} compiler was originally invoked.
% The content of |\jobname| cannot be compared
% to filenames specified in the source due to different catcodes.
% The following code rescans |\jobname|, stores the result
% in |\childdocname| and saves a copy in |\childdocjob|:
%    \begin{macrocode}
\edef\childdocname{\scantokens\expandafter{\jobname\noexpand}}
\let\childdocjob\childdocname
%    \end{macrocode}

% \macro{\childdocdisable}
% The macro |\childdocdisable| prevents the main file
% from being processed more than once.
% At this stage, the main document command |\childdocmain|
% is assumed to be called once again where it should do nothing.
% Any subsequent call to it should prevent
% a secondary processing of the main document
% It overwrites the forwarding commands
% |\childdocof| and |\childdocforward|
% with empty macros to prevent further inclusions of the main document:
%    \begin{macrocode}
\newcommand{\childdocdisable}
{
  \renewcommand{\childdocmain}[1]{\renewcommand{\childdocmain}[1]{\endinput}}
  \renewcommand{\childdocof}[1]{}
  \renewcommand{\childdocby}[2][]{}
  \renewcommand{\childdocforward}[2][]{}
  \renewcommand{\childdocdisable}{}
}
%    \end{macrocode}

% \macro{\childdocmain}
% The macro |\childdocmain| is to be called at the top of the main file
% with nothing or the main filename (without extension) as argument.
% First, it breaks loops.
% If the argument is not empty and does not match |\childdocname|
% (which is set by the first inclusion of |childdoc.def|),
% |\ifchilddoc| is set to true, |\includeonly| is applied to the child file
% and |\jobname| is set to the main file
% (for proper handling of |.aux| files):
%    \begin{macrocode}
\newcommand{\childdocmain}[1]
{
  \childdocdisable\childdocmain{}
  \if?#1?\else
    \begingroup
      \def\childdoctmp{#1}
      \ifx\childdoctmp\childdocname
        \def\childdoctmp{}
      \else
        \def\childdoctmp
        {
          \childdoctrue
          \includeonly{\childdocname}
          \def\childdocjob{#1}
          \def\jobname{#1}
        }
      \fi
      \expandafter
    \endgroup
    \childdoctmp
  \fi
}
%    \end{macrocode}

% \macro{\childdocof}
% The command |\childdocof| redirects
% compilation to the main file |#1|.
%    \begin{macrocode}
\newcommand{\childdocof}[1]
{
  \childdocdisable
  \childdoctrue
  \includeonly{\childdocname}
  \def\jobname{#1}
  \def\childdocjob{#1}
  \input{#1}
}
%    \end{macrocode}

% \macro{\childdocby}
% The command |\childdocby| ....
%    \begin{macrocode}
\newcommand{\childdocby}[2][]
{
  \childdocdisable
  \childdoctrue
  \childdocmanualtrue
  \if?#1?\else
    \def\jobname{#2}
  \fi
  \def\childdocjob{#2}
  \input{#2}
  \endinput
}
%    \end{macrocode}

% \macro{\childdocforward}
% The command |\childdocforward| redirects
% compilation to the main file or
% (if the optional argument is given) a child file.
% Parameters are set as if the main file
% or a child file starting with |\childdocof| was compiled.
% Then compilation is handed over to the main file:
%    \begin{macrocode}
\newcommand{\childdocforward}[2][]
{
  \begingroup
    \if?#1?
      \def\childdoctmp
      {
        \def\childdocname{#2}
        \def\childdocjob{#2}
        \def\jobname{#2}
        \input{#2}
        \endinput
      }
    \else
      \def\childdoctmp
      {
        \childdocdisable
        \def\childdocname{#2}
        \childdoctrue
        \includeonly{#2}
        \def\childdocjob{#1}
        \def\jobname{#1}
        \input{#1}
        \endinput
      }
    \fi
    \expandafter
  \endgroup
  \childdoctmp
}
%    \end{macrocode}

% \macro{\childdocforwardprefix}
% The command |\childdocforwardprefix| redirects
% compilation to the main or a child file by means of a pattern.
% The prefix |#1| in the current filename is replaced by |#2|
% and the suffix of the current filename is kept
% (it is assumed that the filename does not contain the substring `|~~~|'
% which is used as a delimiter).
% Compilation is handed over to the new file by |\childdocforward|:
%    \begin{macrocode}
\newcommand{\childdocforwardprefix}[3][]
{
  \begingroup
    \def\childdocextract #2##1~~~{\def\childdoctmp{\childdocforward[#1]{#3##1}}}
    \expandafter\childdocextract\childdocname~~~
    \expandafter
  \endgroup
  \childdoctmp
}
%    \end{macrocode}

% \macro{\childdoc}
% The deprecated macro |\childdoc| is a legacy version of |\childdocmain|:
%    \begin{macrocode}
\newcommand{\childdoc}{\childdocmain}
%    \end{macrocode}

% \macro{\childdocredirect}
% The deprecated macro |\childdocredirect| is a legacy version
% of |\childdocforward| and |\childdocforwardprefix|:
%    \begin{macrocode}
\newcommand{\childdocredirect}[2][]
{
  \begingroup
    \if?#1?
      \def\childdoctmp{\childdocforward{#2}}
    \else
      \def\childdoctmp{\childdocforwardprefix{#1}{#2}}
    \fi
    \expandafter
  \endgroup
  \childdoctmp
}
%    \end{macrocode}

%\iffalse
%</package>
%\fi
%
\endinput
|\\
|\childdocof{|\textit{main}|}|\\
\end{tabular}
\end{center}
at the top of every child file \textit{child}
which is included by |\include{|\textit{child}|}|
from within the main file
(or at least for those files to be compiled individually).
The argument \textit{main} must be the filename of the main file.

There are a couple of
considerations in setting up the main and child documents:

%%%%%%%%%%%%%%%%%%%%%%%%%%%%%%%%%%%%%%%%
\paragraph{Restrictions.}

Please note the following restrictions:
\begin{itemize}
\item
|\childdocmain| must be called with one argument \textit{main}
to ensure compatibility with earlier version of the package.
It must either be empty (|\childdocmain{}|)
or precisely match the filename of the main file in which it is specified.
See \secref{sec:detection} for further information.
\item
The filename \textit{main} must be specified without the |.tex| extension.
\item
The filename \textit{main} is case sensitive
(even in case-insensitive file systems)
due to internal string comparison.
\item
The argument \textit{main} should be fully expanded, it cannot be a macro.
\item
Subdirectories and special characters should be avoided in filenames.
\item
The command |\childdocmain{|\textit{main}|}| must be followed by a whitespace.
It should not be followed immediately by another command
or by a comment mark `|%|'.
This is because the \TeX{} parser reads the token immediately following
the argument of |\childdocmain| and puts it
at the beginning of every child section;
however, a white\-space is ignored.
\end{itemize}

%%%%%%%%%%%%%%%%%%%%%%%%%%%%%%%%%%%%%%%%
\paragraph{Content of Main File.}

It is advisable to place all content in the child files included by |\include|.
Any output contained in the main file will appear in all child documents
unless suppressed manually;
it cannot be suppressed automatically by the |\includeonly| directive
and thus should normally be avoided.
A method to include some content in the main file
by means of conditional processing is described in \secref{sec:conditional}.

%%%%%%%%%%%%%%%%%%%%%%%%%%%%%%%%%%%%%%%%
\paragraph{Page Numbering.}

When only a part of the document is compiled,
the appropriate numbering of pages
(as well as other status parameters)
is determined from the |.aux| files.
The latter contain information from previous passes.
However this information needs to propagate through
all intermediate child documents.
Therefore the page numbering in child documents may well
be inconsistent until the complete document is compiled at least once.

A useful (if unconventional) way to always ensure a consistent
page numbering is to restart the numbering in each child document
and denote the pages by `\textit{child}|.|\textit{page}'
where \textit{child} represents the chapter/section number of the child file.
This can be achieved by the command
|\numberwithin{page}{|\textit{child}|}|
of the \textsf{amsmath} package
where \textit{child} can be |chapter| or |section|
depending on the chosen structuring.
Alternatively, one can modify the macro |\thepage| appropriately
and reset the counter |page| at the start of each child file.

%%%%%%%%%%%%%%%%%%%%%%%%%%%%%%%%%%%%%%%%%%%%%%%%%%%%%%%%%%%%%%%%%%%%%%%%%%%%%%%%
\subsection{Conditional Processing}
\label{sec:conditional}

The package provides a mechanism to compile different versions
of a document. To customise the versions further some conditional processing
can come in handy to distinguish which version is being compiled.
The package provides two macros to describe the compilation context:

%%%%%%%%%%%%%%%%%%%%%%%%%%%%%%%%%%%%%%%%
\DescribeMacro{\ifchilddoc}
The conditional |\ifchilddoc| distinguishes between the compilation of
child documents and the main document:
%
\begin{center}
|\ifchilddoc |\textit{child-code}| |[|\||else |\textit{main-code}]| \||fi|
\end{center}

%%%%%%%%%%%%%%%%%%%%%%%%%%%%%%%%%%%%%%%%
\DescribeMacro{\childdocname}
\DescribeMacro{\childdocjob}
The macro |\childdocname| contains the filename (without extension)
of the main or child file being processed.
Note that |\childdocjob| will always contain the name of the main file.

%%%%%%%%%%%%%%%%%%%%%%%%%%%%%%%%%%%%%%%%
\paragraph{Title Page.}

Conditional processing can be used to include a title or banner page
in the main document when proper precautions are taken.
Importantly, the code in the main file should ensure that the page counter
(as well as other status parameters which are stored in the |.aux| files)
takes the same value after the conditional processing.
Otherwise the page numbers may take divergent values
depending on which part is compiled.

For example, a title page could be declared by:
%
\begin{center}
\begin{tabular}{l}
|\ifchilddoc\||else|\\
|\addtocounter{page}{-1}|\\
\textit{code for title page}\\
|\newpage|\\
|\||fi|
\end{tabular}
\end{center}
%
A banner page for the child documents can be generated by:
%
\begin{center}
\begin{tabular}{l}
|\ifchilddoc|\\
|\addtocounter{page}{-1}|\\
\textit{code for banner page}\\
|\newpage|\\
|\||fi|
\end{tabular}
\end{center}
%
Here one could write a message such as:
\begin{center}
|This is the part \childdocname{} of \childdocjob{}.|
\end{center}

%%%%%%%%%%%%%%%%%%%%%%%%%%%%%%%%%%%%%%%%%%%%%%%%%%%%%%%%%%%%%%%%%%%%%%%%%%%%%%%%
\subsection{Flags}
\label{sec:flags}

The package makes it easy to generate different versions
of the main or child documents.
To this end compilation flags can be defined
and assigned different default values.
They will be particularly useful in conjunction
with the forwarding mechanism described in \secref{sec:forward}.

For example, it may be useful to have a flag |\version|
which can be set to |draft| or |final|.
The document source will contain some conditional code
depending on the value of |\version|.
Suppose further, the flag should default to |final| for the main file
and to |draft| for child files
which is a natural assignment for editing the document.
This is achieved by placing the following code
in the preamble of the main document
(below the |\childdocmain| directive):
%
\begin{center}
\begin{tabular}{l}
|\ifchilddoc|\\
|\providecommand{\version}{draft}|\\
|\||else|\\
|\providecommand{\version}{final}|\\
|\||fi|
\end{tabular}
\end{center}
%
The definition by |\providecommand| makes sure
that previous definitions are not overwritten.
Further statements |\providecommand{\version}{...}|
can thus be added before the above code to override it.

For the main file, one might add a line
(between |\childdocmain| and the above block)
%
\begin{center}
|%\ifchilddoc\||else\providecommand{\version}{draft}\||fi|
\end{center}
%
which can be uncommented to produce a draft version.
Likewise one can add a line to the very top of a child file
(above the |\childdocof{|\textit{main}|}| directive)
%
\begin{center}
|%\providecommand{\version}{final}|
\end{center}
%
which can be uncommented to produce the final version of this child document.

%%%%%%%%%%%%%%%%%%%%%%%%%%%%%%%%%%%%%%%%%%%%%%%%%%%%%%%%%%%%%%%%%%%%%%%%%%%%%%%%
\subsection{Forwarding}
\label{sec:forward}

Different versions of the main or child documents
using compilation flags as described in \secref{sec:flags}
can be (permanently) stored in different files
for convenient compilation, viewing and distribution.
To this end, the package defines a command
to pass on compilation to a different file:

%%%%%%%%%%%%%%%%%%%%%%%%%%%%%%%%%%%%%%%%
\DescribeMacro{\childdocforward}
The command |\childdocforward| redirects processing to
another source file:
%
\begin{center}
\begin{tabular}{l}
|% \iffalse
%
% childdoc.dtx Copyright (C) 2017-2018 Niklas Beisert
%
% This work may be distributed and/or modified under the
% conditions of the LaTeX Project Public License, either version 1.3
% of this license or (at your option) any later version.
% The latest version of this license is in
%   http://www.latex-project.org/lppl.txt
% and version 1.3 or later is part of all distributions of LaTeX
% version 2005/12/01 or later.
%
% This work has the LPPL maintenance status `maintained'.
%
% The Current Maintainer of this work is Niklas Beisert.
%
% This work consists of the files childdoc.dtx and childdoc.ins
% and the derived files childdoc.def and cdocsamp.tex with
% cdocsch1.tex, cdocsch2.tex, cdocsdrf.tex, cdocsfn1.tex, cdocsfn2.tex.
%
%<package>\ifdefined\childdocmain\endinput\fi
%<package>\ProvidesFile{childdoc.def}[2018/12/30 v2.0 child document driver]
%<samplemain>\ProvidesFile{cdocsamp.tex}[2018/12/30 v2.0 sample for childdoc]
%<*driver>
%\ProvidesFile{childdoc.drv}[2018/12/30 v2.0 childdoc reference manual file]
\PassOptionsToClass{10pt,a4paper}{article}
\documentclass{ltxdoc}

\usepackage[margin=35mm]{geometry}
\usepackage{hyperref}
\usepackage{hyperxmp}
\usepackage[usenames]{color}

\hypersetup{colorlinks=true}
\hypersetup{pdfstartview=FitH}
\hypersetup{pdfpagemode=UseNone}
\hypersetup{pdfsource={}}
\hypersetup{pdflang={en-UK}}
\hypersetup{pdfcopyright={Copyright 2017-2018 Niklas Beisert.
  This work may be distributed and/or modified under the
  conditions of the LaTeX Project Public License, either version 1.3
  of this license or (at your option) any later version.}}
\hypersetup{pdflicenseurl={http://www.latex-project.org/lppl.txt}}
\hypersetup{pdfcontactaddress={ETH Zurich, ITP, HIT K,
  Wolfgang-Pauli-Strasse 27}}
\hypersetup{pdfcontactpostcode={8093}}
\hypersetup{pdfcontactcity={Zurich}}
\hypersetup{pdfcontactcountry={Switzerland}}
\hypersetup{pdfcontactemail={nbeisert@itp.phys.ethz.ch}}
\hypersetup{pdfcontacturl={http://people.phys.ethz.ch/\xmptilde nbeisert/}}

\newcommand{\secref}[1]{\hyperref[#1]{section \ref*{#1}}}

\parskip1ex
\parindent0pt
\let\olditemize\itemize
\def\itemize{\olditemize\parskip0pt}

\begin{document}

\title{The \textsf{childdoc} Package}
\hypersetup{pdftitle={The childdoc Package}}
\author{Niklas Beisert\\[2ex]
  Institut f\"ur Theoretische Physik\\
  Eidgen\"ossische Technische Hochschule Z\"urich\\
  Wolfgang-Pauli-Strasse 27, 8093 Z\"urich, Switzerland\\[1ex]
  \href{mailto:nbeisert@itp.phys.ethz.ch}
  {\texttt{nbeisert@itp.phys.ethz.ch}}}
\hypersetup{pdfauthor={Niklas Beisert}}
\hypersetup{pdfsubject={Manual for the LaTeX2e Package childdoc}}
\date{30 December 2018, \textsf{v2.0}}
\maketitle

\begin{abstract}\noindent
\textsf{childdoc} is a \LaTeXe{} package
that enables the direct compilation
of document sections included by |\include|
to individual files.
\end{abstract}

\begingroup
\parskip0ex
\tableofcontents
\endgroup

%%%%%%%%%%%%%%%%%%%%%%%%%%%%%%%%%%%%%%%%%%%%%%%%%%%%%%%%%%%%%%%%%%%%%%%%%%%%%%%%
%%%%%%%%%%%%%%%%%%%%%%%%%%%%%%%%%%%%%%%%%%%%%%%%%%%%%%%%%%%%%%%%%%%%%%%%%%%%%%%%
\section{Introduction}

\LaTeX{} provides a mechanism to structure a large document (such as a book)
into a main file and several child files (containing the chapters)
using the |\include| command.
This mechanism is beneficial for documents
which span hundreds of pages in order to
make the source file(s) more manageable.
Moreover, compilation can be restricted to
selected child files by means of the |\includeonly| command.
The latter feature can be used to reduce the compilation time while editing
(this was significantly more useful in the earlier days of \LaTeX{})
or to generate a smaller document which is easier to navigate.
Another application of |\includeonly| is to generate
documents consisting of selected parts of the complete document.

However, there are a few drawbacks of the plain |\include| mechanism:
\begin{itemize}
\item
The child files cannot be compiled on their own,
they can only be compiled via the main file.
A naive editing environment
(such as a text editor with an option
to have the current file processed by \LaTeX)
may require one to switch to the main file before compiling;
attempting to compile the child file produces errors.
\item
The main file must be modified (each time)
to adjust the |\includeonly| command
to the present needs. This easily leaves the main file in a messy state.
\item
The generated document will always carry the filename
of the main document. This is inconvenient if
several child files are to be compiled and
to be kept for distribution.
\end{itemize}

The present package provides a simple interface
to make child files individually compilable by \LaTeX{}.
Compiling a child file then has the same effect as compiling
the main file with an |\includeonly| command
to select the appropriate child.
Moreover the generated document will carry the name of the child
rather than the main file.
This resolves all three above issues.

This feature is meant to make the editing of books,
thesis documents and lecture notes somewhat more convenient.
However, the package can also be used efficiently for
composing a series of documents (such as exercise sheets)
which are typically distributed individually.
It then assists the author in generating the individual documents
(potentially in different versions)
as well as a document containing the collected series.
Another application is in developing style files
or other kinds of included material
where compilation of the style file could redirect
to a sample or test file.

%%%%%%%%%%%%%%%%%%%%%%%%%%%%%%%%%%%%%%%%%%%%%%%%%%%%%%%%%%%%%%%%%%%%%%%%%%%%%%%%
%%%%%%%%%%%%%%%%%%%%%%%%%%%%%%%%%%%%%%%%%%%%%%%%%%%%%%%%%%%%%%%%%%%%%%%%%%%%%%%%
\section{Usage}

First of all, the package \textsf{childdoc} is \emph{not} a standard
\LaTeXe{} |.sty| style file! Therefore it needs to be invoked in
a non-standard way.

%%%%%%%%%%%%%%%%%%%%%%%%%%%%%%%%%%%%%%%%%%%%%%%%%%%%%%%%%%%%%%%%%%%%%%%%%%%%%%%%
\subsection{Included Files}
\label{sec:include}

%%%%%%%%%%%%%%%%%%%%%%%%%%%%%%%%%%%%%%%%
\DescribeMacro{\childdocmain}
To use the package, add the commands
\begin{center}
\begin{tabular}{l}
|\input{childdoc.def}|\\
|\childdocmain{}|\\
\end{tabular}
\end{center}
at the very top of the main \LaTeX{} file,
in particular \emph{before} the |\documentclass| statement!
The argument of |\childdocmain| should be left empty
(but it must be present).

%%%%%%%%%%%%%%%%%%%%%%%%%%%%%%%%%%%%%%%%
\DescribeMacro{\childdocof}
Furthermore, add the commands
\begin{center}
\begin{tabular}{l}
|\input{childdoc.def}|\\
|\childdocof{|\textit{main}|}|\\
\end{tabular}
\end{center}
at the top of every child file \textit{child}
which is included by |\include{|\textit{child}|}|
from within the main file
(or at least for those files to be compiled individually).
The argument \textit{main} must be the filename of the main file.

There are a couple of
considerations in setting up the main and child documents:

%%%%%%%%%%%%%%%%%%%%%%%%%%%%%%%%%%%%%%%%
\paragraph{Restrictions.}

Please note the following restrictions:
\begin{itemize}
\item
|\childdocmain| must be called with one argument \textit{main}
to ensure compatibility with earlier version of the package.
It must either be empty (|\childdocmain{}|)
or precisely match the filename of the main file in which it is specified.
See \secref{sec:detection} for further information.
\item
The filename \textit{main} must be specified without the |.tex| extension.
\item
The filename \textit{main} is case sensitive
(even in case-insensitive file systems)
due to internal string comparison.
\item
The argument \textit{main} should be fully expanded, it cannot be a macro.
\item
Subdirectories and special characters should be avoided in filenames.
\item
The command |\childdocmain{|\textit{main}|}| must be followed by a whitespace.
It should not be followed immediately by another command
or by a comment mark `|%|'.
This is because the \TeX{} parser reads the token immediately following
the argument of |\childdocmain| and puts it
at the beginning of every child section;
however, a white\-space is ignored.
\end{itemize}

%%%%%%%%%%%%%%%%%%%%%%%%%%%%%%%%%%%%%%%%
\paragraph{Content of Main File.}

It is advisable to place all content in the child files included by |\include|.
Any output contained in the main file will appear in all child documents
unless suppressed manually;
it cannot be suppressed automatically by the |\includeonly| directive
and thus should normally be avoided.
A method to include some content in the main file
by means of conditional processing is described in \secref{sec:conditional}.

%%%%%%%%%%%%%%%%%%%%%%%%%%%%%%%%%%%%%%%%
\paragraph{Page Numbering.}

When only a part of the document is compiled,
the appropriate numbering of pages
(as well as other status parameters)
is determined from the |.aux| files.
The latter contain information from previous passes.
However this information needs to propagate through
all intermediate child documents.
Therefore the page numbering in child documents may well
be inconsistent until the complete document is compiled at least once.

A useful (if unconventional) way to always ensure a consistent
page numbering is to restart the numbering in each child document
and denote the pages by `\textit{child}|.|\textit{page}'
where \textit{child} represents the chapter/section number of the child file.
This can be achieved by the command
|\numberwithin{page}{|\textit{child}|}|
of the \textsf{amsmath} package
where \textit{child} can be |chapter| or |section|
depending on the chosen structuring.
Alternatively, one can modify the macro |\thepage| appropriately
and reset the counter |page| at the start of each child file.

%%%%%%%%%%%%%%%%%%%%%%%%%%%%%%%%%%%%%%%%%%%%%%%%%%%%%%%%%%%%%%%%%%%%%%%%%%%%%%%%
\subsection{Conditional Processing}
\label{sec:conditional}

The package provides a mechanism to compile different versions
of a document. To customise the versions further some conditional processing
can come in handy to distinguish which version is being compiled.
The package provides two macros to describe the compilation context:

%%%%%%%%%%%%%%%%%%%%%%%%%%%%%%%%%%%%%%%%
\DescribeMacro{\ifchilddoc}
The conditional |\ifchilddoc| distinguishes between the compilation of
child documents and the main document:
%
\begin{center}
|\ifchilddoc |\textit{child-code}| |[|\||else |\textit{main-code}]| \||fi|
\end{center}

%%%%%%%%%%%%%%%%%%%%%%%%%%%%%%%%%%%%%%%%
\DescribeMacro{\childdocname}
\DescribeMacro{\childdocjob}
The macro |\childdocname| contains the filename (without extension)
of the main or child file being processed.
Note that |\childdocjob| will always contain the name of the main file.

%%%%%%%%%%%%%%%%%%%%%%%%%%%%%%%%%%%%%%%%
\paragraph{Title Page.}

Conditional processing can be used to include a title or banner page
in the main document when proper precautions are taken.
Importantly, the code in the main file should ensure that the page counter
(as well as other status parameters which are stored in the |.aux| files)
takes the same value after the conditional processing.
Otherwise the page numbers may take divergent values
depending on which part is compiled.

For example, a title page could be declared by:
%
\begin{center}
\begin{tabular}{l}
|\ifchilddoc\||else|\\
|\addtocounter{page}{-1}|\\
\textit{code for title page}\\
|\newpage|\\
|\||fi|
\end{tabular}
\end{center}
%
A banner page for the child documents can be generated by:
%
\begin{center}
\begin{tabular}{l}
|\ifchilddoc|\\
|\addtocounter{page}{-1}|\\
\textit{code for banner page}\\
|\newpage|\\
|\||fi|
\end{tabular}
\end{center}
%
Here one could write a message such as:
\begin{center}
|This is the part \childdocname{} of \childdocjob{}.|
\end{center}

%%%%%%%%%%%%%%%%%%%%%%%%%%%%%%%%%%%%%%%%%%%%%%%%%%%%%%%%%%%%%%%%%%%%%%%%%%%%%%%%
\subsection{Flags}
\label{sec:flags}

The package makes it easy to generate different versions
of the main or child documents.
To this end compilation flags can be defined
and assigned different default values.
They will be particularly useful in conjunction
with the forwarding mechanism described in \secref{sec:forward}.

For example, it may be useful to have a flag |\version|
which can be set to |draft| or |final|.
The document source will contain some conditional code
depending on the value of |\version|.
Suppose further, the flag should default to |final| for the main file
and to |draft| for child files
which is a natural assignment for editing the document.
This is achieved by placing the following code
in the preamble of the main document
(below the |\childdocmain| directive):
%
\begin{center}
\begin{tabular}{l}
|\ifchilddoc|\\
|\providecommand{\version}{draft}|\\
|\||else|\\
|\providecommand{\version}{final}|\\
|\||fi|
\end{tabular}
\end{center}
%
The definition by |\providecommand| makes sure
that previous definitions are not overwritten.
Further statements |\providecommand{\version}{...}|
can thus be added before the above code to override it.

For the main file, one might add a line
(between |\childdocmain| and the above block)
%
\begin{center}
|%\ifchilddoc\||else\providecommand{\version}{draft}\||fi|
\end{center}
%
which can be uncommented to produce a draft version.
Likewise one can add a line to the very top of a child file
(above the |\childdocof{|\textit{main}|}| directive)
%
\begin{center}
|%\providecommand{\version}{final}|
\end{center}
%
which can be uncommented to produce the final version of this child document.

%%%%%%%%%%%%%%%%%%%%%%%%%%%%%%%%%%%%%%%%%%%%%%%%%%%%%%%%%%%%%%%%%%%%%%%%%%%%%%%%
\subsection{Forwarding}
\label{sec:forward}

Different versions of the main or child documents
using compilation flags as described in \secref{sec:flags}
can be (permanently) stored in different files
for convenient compilation, viewing and distribution.
To this end, the package defines a command
to pass on compilation to a different file:

%%%%%%%%%%%%%%%%%%%%%%%%%%%%%%%%%%%%%%%%
\DescribeMacro{\childdocforward}
The command |\childdocforward| redirects processing to
another source file:
%
\begin{center}
\begin{tabular}{l}
|\input{childdoc.def}|\\
|\childdocforward[|\textit{main}|]{|\textit{dest}|}|\\
\end{tabular}
\end{center}
%
The argument \textit{dest} is the destination file
(without extension).
It should be the main file or one of the child files.
Note that further \textsf{childdoc} directives
such as |\childdocof| and |\childdocforward|
in the indicated file will be processed in this form.
The optional argument \textit{main}
passes on directly to the main file \textit{main}
while pretending to compile the child \textit{dest}.
This form behaves as if \textit{dest}
issues |\childdocof{|\textit{main}|}| right away,
and no further \textsf{childdoc} directives will be processed.

%%%%%%%%%%%%%%%%%%%%%%%%%%%%%%%%%%%%%%%%
\DescribeMacro{\...prefix}
In the alternative form |\childdocforwardprefix|,
%
\begin{center}
\begin{tabular}{l}
|\input{childdoc.def}|\\
|\childdocforwardprefix[|\textit{main}|]{|\textit{prefix}|}{|\textit{dest}|}|
\end{tabular}
\end{center}
%
the destination file is determined by a pattern
depending on the current file:
To make this work, the current file must be called
`{\textit{prefix}\hspace{0.2em}\textit{suffix}}'
with \textit{prefix} matching precisely the argument.
Processing is then passed on to the file
`{\textit{dest}\hspace{0.2em}\textit{suffix}}'.
Surely, the same effect is achieved by
directly specifying the
argument `{\textit{dest}\hspace{0.2em}\textit{suffix}}'
in the first form.
However, that requires to set up a different file
for each child. With the alternative form of the command
all these files can have exactly the same content
which simplifies setting them up and maintaining them.

For example, the following file |draft.tex|
with a compilation flag |\version| as described in \secref{sec:flags}
compiles the main document as a draft:
%
\begin{center}
\begin{tabular}{l}
|\def\version{draft}|\\
|\input{childdoc.def}|\\
|\childdocforward{|\textit{main}|}|
\end{tabular}
\end{center}
%
Likewise, the following files |final|\textit{nn}|.tex|
compile the final version of the child document
|child|\textit{nn}|.tex|:
%
\begin{center}
\begin{tabular}{l}
|\def\version{final}|\\
|\input{childdoc.def}|\\
|\childdocforwardprefix{final}{child}|
\end{tabular}
\end{center}
%

Note that when several versions of a main file and/or of each child file
are to be generated, it may be convenient to set up a |Makefile| or
shell script to automatise the process.

%%%%%%%%%%%%%%%%%%%%%%%%%%%%%%%%%%%%%%%%%%%%%%%%%%%%%%%%%%%%%%%%%%%%%%%%%%%%%%%%
\subsection{Command Line Processing}
\label{sec:commandline}

The effect of redirection files can also be achieved by invoking
the \LaTeX{} compiler with a more elaborate command line.
Most conveniently this should be done as part
of a shell script or a |Makefile|.

When using \textsf{childdoc} in the main file, the following
command lines effectively perform a redirection
(note that depending on the shell being used,
backslashes may have to be doubled: `|\|' $\to$ `|\\|'):
%
\begin{center}
|... -jobname "|\textit{target}|" |\\|"|[\textit{flags}]%
|\input{childdoc.def}\childdocforward[|\textit{main}|]{|\textit{dest}|}"|
\end{center}
%
Here \textit{target} is the name of the output file,
\textit{main} is the name of the main file
and \textit{dest} is the name of the main or child file to be processed
(all filenames without extensions).
The optional argument \textit{main} can be omitted
if \textit{main} matches \textit{dest}.
Optionally, compilation \textit{flags} can be defined via |\def| commands.
This command line makes the \TeX{} engine believe
it is compiling the file \textit{target}
whose content is specified as the latter parameter.
The provided code then forwards the processing to
\textit{main} or \textit{dest} as described in \secref{sec:forward}.

%%%%%%%%%%%%%%%%%%%%%%%%%%%%%%%%%%%%%%%%%%%%%%%%%%%%%%%%%%%%%%%%%%%%%%%%%%%%%%%%
\subsection{Include by Input}
\label{sec:input}

Including child documents by |\include| has some restrictions by design.
Most notably, the content of a child document always occupies
its own set of pages; pages cannot be shared between child documents.
Usually, this behaviour makes perfect sense
because each child document contain an essential part of the document.
However, in some situations it may be desirable to compose
a document from a collection of parts
without having mandatory page breaks between then.
For this case, the package
provides a mechanism to include parts
by |\input| which can also be processed individually.
However, by construction this mechanism
requires manual handling of the content to be output.

%%%%%%%%%%%%%%%%%%%%%%%%%%%%%%%%%%%%%%%%
\DescribeMacro{\ifchilddocmanual}
The main file should be prepared as usual, see \secref{sec:include}.
However, the document body must make a distinction
between processing of an individual part and of the main document, e.g.:
%
\begin{center}
\begin{tabular}{l}
|\ifchilddocmanual|\\
|\input{\childdocname}|\\
|\||else|\\
\textit{document body with }|\input{|\textit{part}|}|\\
|\||fi|
\end{tabular}
\end{center}
%
The conditional |\ifchilddocmanual| is true whenever
a part to be included by |\input| is being compiled,
and the name of the part is stored in |\childdocname|.

%%%%%%%%%%%%%%%%%%%%%%%%%%%%%%%%%%%%%%%%
\DescribeMacro{\childdocby}
Each part to be included by |\input| should start with:
%
\begin{center}
\begin{tabular}{l}
|\input{childdoc.def}|\\
|\childdocby{|\textit{main}|}|\\
\end{tabular}
\end{center}
%
The directive |\childdocby| is similar to |\childdocof|
described in \secref{sec:include},
but the subsequent selection of content must be done manually.
To that end, both |\ifchilddoc| and |\ifchilddocmanual|
will be true upon processing of a part,
and the name of the part is stored in |\childdocname|.
Note that |\jobname| will be set to the filename of the current part
so that each part receives an individual |.aux| file
that does not interfere with the |.aux| file(s) of the main document.
This behaviour can be altered by the alternative form
|\childdocby[*]{|\textit{main}|}| (with a non-empty optional argument)
which uses the |.aux| file of the main document
by setting |\jobname| to \textit{main}.

%%%%%%%%%%%%%%%%%%%%%%%%%%%%%%%%%%%%%%%%%%%%%%%%%%%%%%%%%%%%%%%%%%%%%%%%%%%%%%%%
\subsection{Driver Development}
\label{sec:driver}

The \textsf{childdoc} mechanism can also be use for the development
of definition files such as \LaTeX{} styles or classes.
This case differs from the above setup with multiple parts
included by |\include| in that no |\includeonly| should be invoked.
This can be achieved by starting the include file
(before |\ProvidesPackage|) with:
%
\begin{center}
\begin{tabular}{l}
|\input{childdoc.def}|\\
|\childdocforward{|\textit{main}|}|\\
\end{tabular}
\end{center}
%
or alternatively with:
%
\begin{center}
\begin{tabular}{l}
|\input{childdoc.def}|\\
|\childdocby{|\textit{main}|}|\\
\end{tabular}
\end{center}
%
Both forms have slightly different effects as described above.
The main file is prepared as usual, see \secref{sec:include}.

%%%%%%%%%%%%%%%%%%%%%%%%%%%%%%%%%%%%%%%%%%%%%%%%%%%%%%%%%%%%%%%%%%%%%%%%%%%%%%%%
\subsection{Legacy Detection}
\label{sec:detection}

The directive |\childdocmain| in the main file can detect
whether the complete document or merely a child is to be compiled
even without using the directive |\childdocof|.
This method is deprecated because it is less robust
and there is no compelling reason to use it;
it is merely provided for backward compatibility
and it may be removed in future versions.

If the detection mechanism is to be used,
it is mandatory to correctly specify
the filename of the main file as the argument of |\childdocmain|:
%
\begin{center}
\begin{tabular}{l}
|\input{childdoc.def}|\\
|\childdocmain{|\textit{main}|}|\\
\end{tabular}
\end{center}
%
If |\jobname| does not match the argument \textit{main} of |\childdocmain|,
it is assumed that |\jobname| points to the child file to be compiled.
When using |\childdocmain| with the main file specified as argument,
it suffices to start a child file
with just |\input{|\textit{main}|}|
without loading of the package and using |\childdocof|.
If instead all processing is done
with the appropriate \textsf{childdoc} directives,
the argument of \textit{main} of |\childdocmain| can be empty.

An alternative version of the command line processing described
in \secref{sec:commandline} using the detection mechanism reads:
%
\begin{center}
|... -jobname "|\textit{target}|" "|[\textit{flags}]%
[|\def\jobname{|\textit{dest}|}|]|\input{|\textit{main}|}"|
\end{center}

%%%%%%%%%%%%%%%%%%%%%%%%%%%%%%%%%%%%%%%%%%%%%%%%%%%%%%%%%%%%%%%%%%%%%%%%%%%%%%%%
\subsection{Manual Code}
\label{sec:manual}

In case one cannot be certain whether the definitions file |childdoc.def|
is installed on the target \TeX{} distribution
and one prefers not to ship it,
it is conceivable to paste a few relevant commands into the sources.

To that end, drop all statements |\input{childdoc.def}|
and perform the replacements as outlined below.
Instead of |\childdocmain{|\textit{main}|}| add the following code
to the top of the main file:
%
\begin{center}
\begin{tabular}{l}
|\||ifdefined\childdocname\endinput\||fi\newif\ifchilddoc|\\
|\edef\childdocname{\scantokens\expandafter{\jobname\noexpand}}|\\
|\def\childdocmain{|\textit{main}|}\||ifx\childdocmain\childdocname\||else|\\
|\childdoctrue\includeonly{\childdocname}\let\jobname\childdocmain\||fi|\\
\end{tabular}
\end{center}
%
Instead of |\childdocof{|\textit{main}|}| just include the main file
at the top of each child file:
%
\begin{center}
|\input{|\textit{main}|}|
\end{center}
%
A simple redirection |\childdocforward{|\textit{dest}|}| is achieved by:
%
\begin{center}
|\def\jobname{|\textit{dest}|}\input{\jobname}|
\end{center}
%
The redirection with prefix
|\childdocforwardprefix[|\textit{prefix}|]{|\textit{dest}|}|
is accomplished by:
%
\begin{center}
\begin{tabular}{l}
|{\edef\jobname{\scantokens\expandafter{\jobname\noexpand}}|\\
|\def\redirectjob |\textit{prefix}|#1~~~{\gdef\jobname{|\textit{dest}|#1}}|\\
|\expandafter\redirectjob\jobname~~~}\input{\jobname}|
\end{tabular}
\end{center}

In an alternative approach,
child documents can be compiled by a specific command line
without additional code or specific definitions:
%
\begin{center}
|... -jobname "|\textit{target}|" "|[\textit{flags}]%
|\includeonly{|\textit{dest}|}\input{|\textit{main}|}"|
\end{center}
%

%%%%%%%%%%%%%%%%%%%%%%%%%%%%%%%%%%%%%%%%%%%%%%%%%%%%%%%%%%%%%%%%%%%%%%%%%%%%%%%%
%%%%%%%%%%%%%%%%%%%%%%%%%%%%%%%%%%%%%%%%%%%%%%%%%%%%%%%%%%%%%%%%%%%%%%%%%%%%%%%%
\section{Information}

%%%%%%%%%%%%%%%%%%%%%%%%%%%%%%%%%%%%%%%%%%%%%%%%%%%%%%%%%%%%%%%%%%%%%%%%%%%%%%%%
\subsection{Copyright}

Copyright \copyright{} 2017--2018 Niklas Beisert

This work may be distributed and/or modified under the
conditions of the \LaTeX{} Project Public License, either version 1.3
of this license or (at your option) any later version.
The latest version of this license is in
  \url{http://www.latex-project.org/lppl.txt}
and version 1.3 or later is part of all distributions of \LaTeX{}
version 2005/12/01 or later.

This work has the LPPL maintenance status `maintained'.

The Current Maintainer of this work is Niklas Beisert.

This work consists of the files |README.txt|, |childdoc.ins| and |childdoc.dtx|
as well as the derived files |childdoc.def|, |cdocsamp.tex|
with |cdocsch1.tex|, |cdocsch2.tex|, |cdocspt3.tex|, |cdocspt4.tex|,
|cdocsdrf.tex|, |cdocsfn1.tex|, |cdocsfn2.tex|
as well as |childdoc.pdf|.

%%%%%%%%%%%%%%%%%%%%%%%%%%%%%%%%%%%%%%%%%%%%%%%%%%%%%%%%%%%%%%%%%%%%%%%%%%%%%%%%
\subsection{Files and Installation}

The package consists of the files:
%
\begin{center}
\begin{tabular}{ll}
    |README.txt|   & readme file \\
    |childdoc.ins| & installation file \\
    |childdoc.dtx| & source file \\
    |childdoc.def| & definition file \\
    |cdocsamp.tex| & sample main file \\
    |cdocsch1.tex| & sample include file \\
    |cdocsch2.tex| & sample include file \\
    |cdocspt3.tex| & sample part file \\
    |cdocspt4.tex| & sample part file \\
    |cdocsdrf.tex| & sample redirection file \\
    |cdocsfn1.tex| & sample redirection file \\
    |cdocsfn2.tex| & sample redirection file \\
    |childdoc.pdf| & manual
\end{tabular}
\end{center}
%
The distribution consists of the files
|README.txt|, |childdoc.ins| and |childdoc.dtx|.
%
\begin{itemize}
\item
Run (pdf)\LaTeX{} on |childdoc.dtx|
to compile the manual |childdoc.pdf| (this file).
\item
Run \LaTeX{} on |childdoc.ins| to create the definitions file |childdoc.def|
and the sample |cdocsamp.tex| with include files
|cdocsch1.tex|, |cdocsch2.tex|, |cdocspt3.tex|, |cdocspt4.tex|,
|cdocsdrf.tex|, |cdocsfn1.tex|, |cdocsfn2.tex|.
Then copy the file |childdoc.def| to an appropriate directory of your \LaTeX{}
distribution, e.g.\ \textit{texmf-root}|/tex/latex/childdoc|.
\end{itemize}

%%%%%%%%%%%%%%%%%%%%%%%%%%%%%%%%%%%%%%%%%%%%%%%%%%%%%%%%%%%%%%%%%%%%%%%%%%%%%%%%
\subsection{Related CTAN Packages}

There are several other packages which offer a similar functionality:
%
\begin{itemize}
\item
The packages
\href{http://ctan.org/pkg/docmute}{\textsf{docmute}},
\href{http://ctan.org/pkg/includex}{\textsf{includex}} and
\href{http://ctan.org/pkg/standalone}{\textsf{standalone}}
provide commands to include only the document body of
a child file thus allowing both files to be compiled individually.
\item
The packages \href{http://ctan.org/pkg/subdocs}{\textsf{subdocs}}
and \href{http://ctan.org/pkg/subfiles}{\textsf{subfiles}}
provide structures in which the main and child documents can be
encapsulated and allowing them to be compiled individually.
The inclusion mechanism is different from the conventional |\include|.
\item
The package \href{http://ctan.org/pkg/combine}{\textsf{combine}}
is an elaborate solution to combine several documents into one.
\end{itemize}
%
See also the CTAN topic \href{http://ctan.org/topic/subdocs}{\textsf{subdocs}}
for further related packages.
The present package differs from the above solutions in that
a document structure constructed with the conventional |\include| mechanism
just needs two extra commands at the top of every file
such that all constituent files can be compiled individually.

%%%%%%%%%%%%%%%%%%%%%%%%%%%%%%%%%%%%%%%%%%%%%%%%%%%%%%%%%%%%%%%%%%%%%%%%%%%%%%%%
%\subsection{Feature Suggestions}
%
%The following is a list of features which may be useful for future
%versions of this package:
%%
%\begin{itemize}
%\item
%\ldots
%\end{itemize}

%%%%%%%%%%%%%%%%%%%%%%%%%%%%%%%%%%%%%%%%%%%%%%%%%%%%%%%%%%%%%%%%%%%%%%%%%%%%%%%%
\subsection{Revision History}

%%%%%%%%%%%%%%%%%%%%%%%%%%%%%%%%%%%%%%%%
\paragraph{v2.0:} 2018/12/30

\begin{itemize}
\item
immediate forward processing
\item
added |\childdocby| mechanism
\item
manual restructured
\end{itemize}

%%%%%%%%%%%%%%%%%%%%%%%%%%%%%%%%%%%%%%%%
\paragraph{v1.6:} 2018/01/17

\begin{itemize}
\item
application for development of include files
\item
corrections to manual
\end{itemize}

%%%%%%%%%%%%%%%%%%%%%%%%%%%%%%%%%%%%%%%%
\paragraph{v1.5:} 2017/05/21

\begin{itemize}
\item
more complete structuring introduced
\item
|\childdocof| introduced
\item
|\childdoc| renamed to |\childdocmain|
\item
|\childredirect| renamed to |\childdocforward| and |\childdocforwardprefix|
and functionality expanded
\end{itemize}

%%%%%%%%%%%%%%%%%%%%%%%%%%%%%%%%%%%%%%%%
\paragraph{v1.0:} 2017/04/27

\begin{itemize}
\item
manual and install package
\item
first version published on CTAN
\end{itemize}

%%%%%%%%%%%%%%%%%%%%%%%%%%%%%%%%%%%%%%%%
\paragraph{v0.6:} 2017/04/26

\begin{itemize}
\item
redirection mechanism added
\end{itemize}

%%%%%%%%%%%%%%%%%%%%%%%%%%%%%%%%%%%%%%%%
\paragraph{v0.5:} 2017/04/26

\begin{itemize}
\item
functionality in definition file
\end{itemize}


%%%%%%%%%%%%%%%%%%%%%%%%%%%%%%%%%%%%%%%%%%%%%%%%%%%%%%%%%%%%%%%%%%%%%%%%%%%%%%%%
%%%%%%%%%%%%%%%%%%%%%%%%%%%%%%%%%%%%%%%%%%%%%%%%%%%%%%%%%%%%%%%%%%%%%%%%%%%%%%%%
%%%%%%%%%%%%%%%%%%%%%%%%%%%%%%%%%%%%%%%%%%%%%%%%%%%%%%%%%%%%%%%%%%%%%%%%%%%%%%%%
\appendix

\settowidth\MacroIndent{\rmfamily\scriptsize 000\ }

 \DocInput{childdoc.dtx}

\end{document}
%</driver>
% \fi
%
% %%%%%%%%%%%%%%%%%%%%%%%%%%%%%%%%%%%%%%%%%%%%%%%%%%%%%%%%%%%%%%%%%%%%%%%%%%%%%%
% %%%%%%%%%%%%%%%%%%%%%%%%%%%%%%%%%%%%%%%%%%%%%%%%%%%%%%%%%%%%%%%%%%%%%%%%%%%%%%
% \section{Sample}
%\iffalse
%<*samplemain>
%\fi
%
% The following presents a sample document
% with two chapters, two parts, a title page,
% a compile flag as well as three forwarding files to set the flag.
% It consists of eight |.tex| files:
% \begin{center}
% \begin{tabular}{ll}
% |cdocsamp.tex|&main file\\
% |cdocsch1.tex|&include file for chapter 1\\
% |cdocsch2.tex|&include file for chapter 2\\
% |cdocspt3.tex|&include file for part 3\\
% |cdocspt4.tex|&include file for part 4\\
% |cdocsdrf.tex|&forwarding file for main file in draft mode\\
% |cdocsfi1.tex|&forwarding file for final version of chapter 1\\
% |cdocsfi2.tex|&forwarding file for final version of chapter 2\\
% \end{tabular}
% \end{center}
% Each of the eight files can be compiled directly by the \LaTeX{} compiler.
%
% %%%%%%%%%%%%%%%%%%%%%%%%%%%%%%%%%%%%%%
% \paragraph{Main File.}
%
% The main file is called |cdocsamp.tex|.
%
% Load the \textsf{childdoc} definitions and
% declare the filename for the main document:
%    \begin{macrocode}
\input{childdoc.def}
\childdocmain{}
%    \end{macrocode}

% Optional override for |\version| flag:
%    \begin{macrocode}
%%\ifchilddoc\else\providecommand{\version}{draft}\fi
%    \end{macrocode}

% Define the default values for the |\version| flag
% (|final| for the main file and |draft| for childs):
%    \begin{macrocode}
\ifchilddoc
\providecommand{\version}{draft}
\else
\providecommand{\version}{final}
\fi
%    \end{macrocode}

% Load the standard document class:
%    \begin{macrocode}
\documentclass[12pt]{article}
%    \end{macrocode}

% Start the document body:
%    \begin{macrocode}
\begin{document}
%    \end{macrocode}

% Declare a title page.
% Print title, part of document being processed and version flag:
%    \begin{macrocode}
\addtocounter{page}{-1}
\begin{center}
{\LARGE\bfseries{}childdoc example\par}
\vspace{1cm}
\ifchilddoc
\ifchilddocmanual part\else chapter\fi:
`\childdocname' of `\childdocjob'\par
\else
main document: `\childdocjob'\par
\fi
version: \version\par
\end{center}
\newpage
%    \end{macrocode}

% Manually include selected file,
% otherwise process as usual:
%    \begin{macrocode}
\ifchilddocmanual
\section*{part `\childdocname'}
\input{\childdocname}
\else
%    \end{macrocode}

% Include the two chapters:
%    \begin{macrocode}
\include{cdocsch1}
\include{cdocsch2}
%    \end{macrocode}

% Include the two parts unless only chapters should be displayed:
%    \begin{macrocode}
\ifchilddoc\else
\section{part three}
\input{cdocspt3}
\section{part four}
\input{cdocspt4}
\fi
%    \end{macrocode}

% Process as usual until here:
%    \begin{macrocode}
\fi
%    \end{macrocode}

% End of document body:
%    \begin{macrocode}
\end{document}
%    \end{macrocode}
%\iffalse
%</samplemain>
%\fi
%
% %%%%%%%%%%%%%%%%%%%%%%%%%%%%%%%%%%%%%%
% \paragraph{Chapter Include Files.}
%
% The include files are called |cdocsch1.tex| and |cdocsch2.tex|.
%
%\iffalse
%<*samplechap1|samplechap2>
%\fi

% Optional override for |\version| flag:
%    \begin{macrocode}
%%\providecommand{\version}{final}
%    \end{macrocode}

% Include the main document:
%    \begin{macrocode}
\input{childdoc.def}
\childdocof{cdocsamp}
%    \end{macrocode}

%\iffalse
%</samplechap1|samplechap2>
%\fi
%
%\iffalse
%<*samplechap1>
%\fi
% Some text for chapter 1:
%    \begin{macrocode}
\section{one}
some text in chapter one
%    \end{macrocode}

%\iffalse
%</samplechap1>
%\fi
% Some text for chapter 2:
%\iffalse
%<*samplechap2>
%\fi
%    \begin{macrocode}
\section{two}
more text in chapter two
%    \end{macrocode}

%\iffalse
%</samplechap2>
%\fi
%
% %%%%%%%%%%%%%%%%%%%%%%%%%%%%%%%%%%%%%%
% \paragraph{Part Include Files.}
%
% The include files are called |cdocspt3.tex| and |cdocspt4.tex|.
%
%\iffalse
%<*samplepart3|samplepart4>
%\fi

% Optional override for |\version| flag:
%    \begin{macrocode}
%%\providecommand{\version}{final}
%    \end{macrocode}

% Include the main document:
%    \begin{macrocode}
\input{childdoc.def}
\childdocby{cdocsamp}
%    \end{macrocode}

%\iffalse
%</samplepart3|samplepart4>
%\fi
%
%\iffalse
%<*samplepart3>
%\fi
% Some text for part 3:
%    \begin{macrocode}
some text in part three
%    \end{macrocode}

%\iffalse
%</samplepart3>
%\fi
% Some text for part 4:
%\iffalse
%<*samplepart4>
%\fi
%    \begin{macrocode}
more text in part four
%    \end{macrocode}

%\iffalse
%</samplepart4>
%\fi
%
% %%%%%%%%%%%%%%%%%%%%%%%%%%%%%%%%%%%%%%
% \paragraph{Forwarding for a Complete Draft.}
%
% The following forwarding file |cdocsdrf.tex|
% compiles the main document in draft mode:
%\iffalse
%<*sampledraft>
%\fi
%    \begin{macrocode}
\def\version{draft}
\input{childdoc.def}
\childdocforward{cdocsamp}
%    \end{macrocode}

%\iffalse
%</sampledraft>
%\fi
%
% %%%%%%%%%%%%%%%%%%%%%%%%%%%%%%%%%%%%%%
% \paragraph{Forwarding for Final Version of the Chapters.}
%
% The following forwarding files |cdocsfn1.tex| and |cdocsfn2.tex|
% (with identical content)
% compile the final versions of the child documents
% |cdocsch1.tex| and |cdocsch2.tex|, respectively:
%\iffalse
%<*samplefinal>
%\fi
%    \begin{macrocode}
\def\version{final}
\input{childdoc.def}
\childdocforwardprefix[cdocsamp]{cdocsfn}{cdocsch}
%    \end{macrocode}

%\iffalse
%</samplefinal>
%\fi
%
% %%%%%%%%%%%%%%%%%%%%%%%%%%%%%%%%%%%%%%
% \paragraph{Command Line Processing.}
%
% The following three command lines generate the output files
% |cdocscld|, |cdocscl1| and |cdocscl2|
% which should be identical to
% |cdocsdrf|, |cdocsch1| and |cdocsfn2|, respectively:
% \begin{center}
% \begin{tabular}{l}
% |latex -jobname cdocscld \|\\
% |  "\def\version{draft}\input{childdoc.def}\childdocforward{cdocsamp}"|\\
% |latex -jobname cdocscl1 \|\\
% |  "\input{childdoc.def}\childdocforward[cdocsamp]{cdocsch1}"|\\
% |latex -jobname cdocscl2 \|\\
% |  "\def\version{final}\input{childdoc.def}\childdocforward{cdocsch2}"|
% \end{tabular}
% \end{center}
% Note that the trailing backslash on each first line
% merely continues the input to the second line
% (for convenient cut ant paste).
% Furthermore, the command |latex| can be replaced by any
% of its alternative versions such as |pdflatex|.
%
% %%%%%%%%%%%%%%%%%%%%%%%%%%%%%%%%%%%%%%%%%%%%%%%%%%%%%%%%%%%%%%%%%%%%%%%%%%%%%%
% %%%%%%%%%%%%%%%%%%%%%%%%%%%%%%%%%%%%%%%%%%%%%%%%%%%%%%%%%%%%%%%%%%%%%%%%%%%%%%
% \section{Implementation}
%\iffalse
%<*package>
%\fi
%
% This section describes the definitions file |childdoc.def|.

% The definitions cannot be loaded using |\usepackage| or |\RequirePackage|
% which has a mechanism to prevent loading a style file more than once.
% When loading the definitions by means of |\input|
% multiple instances have to be prevented manually:
%\iffalse
%This code needs to be before the `\ProvidesFile' directive
%which is defined at the beginning of this file.
%Therefore it is also placed there and commented out here.
%</package>
%<*discard>
%\fi
%    \begin{macrocode}
\ifdefined\childdocmain\endinput\fi
%    \end{macrocode}
%\iffalse
%</discard>
%<*package>
%\fi
%
% \macro{\ifchilddoc}
% \macro{\ifchilddocmanual}
% The conditional |\ifchilddoc| tells whether a
% child (true) or main (false) document is being compiled.
% The conditional |\ifchilddocmanual| tells whether
% the |\includeonly| mechanism is used (false) or
% the selection of child files must be performed manually (true).
% The definitions initialise to false:
%    \begin{macrocode}
\newif\ifchilddoc
\newif\ifchilddocmanual
%    \end{macrocode}

% \macro{\childdocname}
% \macro{\childdocjob}
% The macro |\childdocname| stores the name of the main document
% to be compiled. The macro |\childdocjob| stores the name of
% the document on which the \LaTeX{} compiler was originally invoked.
% The content of |\jobname| cannot be compared
% to filenames specified in the source due to different catcodes.
% The following code rescans |\jobname|, stores the result
% in |\childdocname| and saves a copy in |\childdocjob|:
%    \begin{macrocode}
\edef\childdocname{\scantokens\expandafter{\jobname\noexpand}}
\let\childdocjob\childdocname
%    \end{macrocode}

% \macro{\childdocdisable}
% The macro |\childdocdisable| prevents the main file
% from being processed more than once.
% At this stage, the main document command |\childdocmain|
% is assumed to be called once again where it should do nothing.
% Any subsequent call to it should prevent
% a secondary processing of the main document
% It overwrites the forwarding commands
% |\childdocof| and |\childdocforward|
% with empty macros to prevent further inclusions of the main document:
%    \begin{macrocode}
\newcommand{\childdocdisable}
{
  \renewcommand{\childdocmain}[1]{\renewcommand{\childdocmain}[1]{\endinput}}
  \renewcommand{\childdocof}[1]{}
  \renewcommand{\childdocby}[2][]{}
  \renewcommand{\childdocforward}[2][]{}
  \renewcommand{\childdocdisable}{}
}
%    \end{macrocode}

% \macro{\childdocmain}
% The macro |\childdocmain| is to be called at the top of the main file
% with nothing or the main filename (without extension) as argument.
% First, it breaks loops.
% If the argument is not empty and does not match |\childdocname|
% (which is set by the first inclusion of |childdoc.def|),
% |\ifchilddoc| is set to true, |\includeonly| is applied to the child file
% and |\jobname| is set to the main file
% (for proper handling of |.aux| files):
%    \begin{macrocode}
\newcommand{\childdocmain}[1]
{
  \childdocdisable\childdocmain{}
  \if?#1?\else
    \begingroup
      \def\childdoctmp{#1}
      \ifx\childdoctmp\childdocname
        \def\childdoctmp{}
      \else
        \def\childdoctmp
        {
          \childdoctrue
          \includeonly{\childdocname}
          \def\childdocjob{#1}
          \def\jobname{#1}
        }
      \fi
      \expandafter
    \endgroup
    \childdoctmp
  \fi
}
%    \end{macrocode}

% \macro{\childdocof}
% The command |\childdocof| redirects
% compilation to the main file |#1|.
%    \begin{macrocode}
\newcommand{\childdocof}[1]
{
  \childdocdisable
  \childdoctrue
  \includeonly{\childdocname}
  \def\jobname{#1}
  \def\childdocjob{#1}
  \input{#1}
}
%    \end{macrocode}

% \macro{\childdocby}
% The command |\childdocby| ....
%    \begin{macrocode}
\newcommand{\childdocby}[2][]
{
  \childdocdisable
  \childdoctrue
  \childdocmanualtrue
  \if?#1?\else
    \def\jobname{#2}
  \fi
  \def\childdocjob{#2}
  \input{#2}
  \endinput
}
%    \end{macrocode}

% \macro{\childdocforward}
% The command |\childdocforward| redirects
% compilation to the main file or
% (if the optional argument is given) a child file.
% Parameters are set as if the main file
% or a child file starting with |\childdocof| was compiled.
% Then compilation is handed over to the main file:
%    \begin{macrocode}
\newcommand{\childdocforward}[2][]
{
  \begingroup
    \if?#1?
      \def\childdoctmp
      {
        \def\childdocname{#2}
        \def\childdocjob{#2}
        \def\jobname{#2}
        \input{#2}
        \endinput
      }
    \else
      \def\childdoctmp
      {
        \childdocdisable
        \def\childdocname{#2}
        \childdoctrue
        \includeonly{#2}
        \def\childdocjob{#1}
        \def\jobname{#1}
        \input{#1}
        \endinput
      }
    \fi
    \expandafter
  \endgroup
  \childdoctmp
}
%    \end{macrocode}

% \macro{\childdocforwardprefix}
% The command |\childdocforwardprefix| redirects
% compilation to the main or a child file by means of a pattern.
% The prefix |#1| in the current filename is replaced by |#2|
% and the suffix of the current filename is kept
% (it is assumed that the filename does not contain the substring `|~~~|'
% which is used as a delimiter).
% Compilation is handed over to the new file by |\childdocforward|:
%    \begin{macrocode}
\newcommand{\childdocforwardprefix}[3][]
{
  \begingroup
    \def\childdocextract #2##1~~~{\def\childdoctmp{\childdocforward[#1]{#3##1}}}
    \expandafter\childdocextract\childdocname~~~
    \expandafter
  \endgroup
  \childdoctmp
}
%    \end{macrocode}

% \macro{\childdoc}
% The deprecated macro |\childdoc| is a legacy version of |\childdocmain|:
%    \begin{macrocode}
\newcommand{\childdoc}{\childdocmain}
%    \end{macrocode}

% \macro{\childdocredirect}
% The deprecated macro |\childdocredirect| is a legacy version
% of |\childdocforward| and |\childdocforwardprefix|:
%    \begin{macrocode}
\newcommand{\childdocredirect}[2][]
{
  \begingroup
    \if?#1?
      \def\childdoctmp{\childdocforward{#2}}
    \else
      \def\childdoctmp{\childdocforwardprefix{#1}{#2}}
    \fi
    \expandafter
  \endgroup
  \childdoctmp
}
%    \end{macrocode}

%\iffalse
%</package>
%\fi
%
\endinput
|\\
|\childdocforward[|\textit{main}|]{|\textit{dest}|}|\\
\end{tabular}
\end{center}
%
The argument \textit{dest} is the destination file
(without extension).
It should be the main file or one of the child files.
Note that further \textsf{childdoc} directives
such as |\childdocof| and |\childdocforward|
in the indicated file will be processed in this form.
The optional argument \textit{main}
passes on directly to the main file \textit{main}
while pretending to compile the child \textit{dest}.
This form behaves as if \textit{dest}
issues |\childdocof{|\textit{main}|}| right away,
and no further \textsf{childdoc} directives will be processed.

%%%%%%%%%%%%%%%%%%%%%%%%%%%%%%%%%%%%%%%%
\DescribeMacro{\...prefix}
In the alternative form |\childdocforwardprefix|,
%
\begin{center}
\begin{tabular}{l}
|% \iffalse
%
% childdoc.dtx Copyright (C) 2017-2018 Niklas Beisert
%
% This work may be distributed and/or modified under the
% conditions of the LaTeX Project Public License, either version 1.3
% of this license or (at your option) any later version.
% The latest version of this license is in
%   http://www.latex-project.org/lppl.txt
% and version 1.3 or later is part of all distributions of LaTeX
% version 2005/12/01 or later.
%
% This work has the LPPL maintenance status `maintained'.
%
% The Current Maintainer of this work is Niklas Beisert.
%
% This work consists of the files childdoc.dtx and childdoc.ins
% and the derived files childdoc.def and cdocsamp.tex with
% cdocsch1.tex, cdocsch2.tex, cdocsdrf.tex, cdocsfn1.tex, cdocsfn2.tex.
%
%<package>\ifdefined\childdocmain\endinput\fi
%<package>\ProvidesFile{childdoc.def}[2018/12/30 v2.0 child document driver]
%<samplemain>\ProvidesFile{cdocsamp.tex}[2018/12/30 v2.0 sample for childdoc]
%<*driver>
%\ProvidesFile{childdoc.drv}[2018/12/30 v2.0 childdoc reference manual file]
\PassOptionsToClass{10pt,a4paper}{article}
\documentclass{ltxdoc}

\usepackage[margin=35mm]{geometry}
\usepackage{hyperref}
\usepackage{hyperxmp}
\usepackage[usenames]{color}

\hypersetup{colorlinks=true}
\hypersetup{pdfstartview=FitH}
\hypersetup{pdfpagemode=UseNone}
\hypersetup{pdfsource={}}
\hypersetup{pdflang={en-UK}}
\hypersetup{pdfcopyright={Copyright 2017-2018 Niklas Beisert.
  This work may be distributed and/or modified under the
  conditions of the LaTeX Project Public License, either version 1.3
  of this license or (at your option) any later version.}}
\hypersetup{pdflicenseurl={http://www.latex-project.org/lppl.txt}}
\hypersetup{pdfcontactaddress={ETH Zurich, ITP, HIT K,
  Wolfgang-Pauli-Strasse 27}}
\hypersetup{pdfcontactpostcode={8093}}
\hypersetup{pdfcontactcity={Zurich}}
\hypersetup{pdfcontactcountry={Switzerland}}
\hypersetup{pdfcontactemail={nbeisert@itp.phys.ethz.ch}}
\hypersetup{pdfcontacturl={http://people.phys.ethz.ch/\xmptilde nbeisert/}}

\newcommand{\secref}[1]{\hyperref[#1]{section \ref*{#1}}}

\parskip1ex
\parindent0pt
\let\olditemize\itemize
\def\itemize{\olditemize\parskip0pt}

\begin{document}

\title{The \textsf{childdoc} Package}
\hypersetup{pdftitle={The childdoc Package}}
\author{Niklas Beisert\\[2ex]
  Institut f\"ur Theoretische Physik\\
  Eidgen\"ossische Technische Hochschule Z\"urich\\
  Wolfgang-Pauli-Strasse 27, 8093 Z\"urich, Switzerland\\[1ex]
  \href{mailto:nbeisert@itp.phys.ethz.ch}
  {\texttt{nbeisert@itp.phys.ethz.ch}}}
\hypersetup{pdfauthor={Niklas Beisert}}
\hypersetup{pdfsubject={Manual for the LaTeX2e Package childdoc}}
\date{30 December 2018, \textsf{v2.0}}
\maketitle

\begin{abstract}\noindent
\textsf{childdoc} is a \LaTeXe{} package
that enables the direct compilation
of document sections included by |\include|
to individual files.
\end{abstract}

\begingroup
\parskip0ex
\tableofcontents
\endgroup

%%%%%%%%%%%%%%%%%%%%%%%%%%%%%%%%%%%%%%%%%%%%%%%%%%%%%%%%%%%%%%%%%%%%%%%%%%%%%%%%
%%%%%%%%%%%%%%%%%%%%%%%%%%%%%%%%%%%%%%%%%%%%%%%%%%%%%%%%%%%%%%%%%%%%%%%%%%%%%%%%
\section{Introduction}

\LaTeX{} provides a mechanism to structure a large document (such as a book)
into a main file and several child files (containing the chapters)
using the |\include| command.
This mechanism is beneficial for documents
which span hundreds of pages in order to
make the source file(s) more manageable.
Moreover, compilation can be restricted to
selected child files by means of the |\includeonly| command.
The latter feature can be used to reduce the compilation time while editing
(this was significantly more useful in the earlier days of \LaTeX{})
or to generate a smaller document which is easier to navigate.
Another application of |\includeonly| is to generate
documents consisting of selected parts of the complete document.

However, there are a few drawbacks of the plain |\include| mechanism:
\begin{itemize}
\item
The child files cannot be compiled on their own,
they can only be compiled via the main file.
A naive editing environment
(such as a text editor with an option
to have the current file processed by \LaTeX)
may require one to switch to the main file before compiling;
attempting to compile the child file produces errors.
\item
The main file must be modified (each time)
to adjust the |\includeonly| command
to the present needs. This easily leaves the main file in a messy state.
\item
The generated document will always carry the filename
of the main document. This is inconvenient if
several child files are to be compiled and
to be kept for distribution.
\end{itemize}

The present package provides a simple interface
to make child files individually compilable by \LaTeX{}.
Compiling a child file then has the same effect as compiling
the main file with an |\includeonly| command
to select the appropriate child.
Moreover the generated document will carry the name of the child
rather than the main file.
This resolves all three above issues.

This feature is meant to make the editing of books,
thesis documents and lecture notes somewhat more convenient.
However, the package can also be used efficiently for
composing a series of documents (such as exercise sheets)
which are typically distributed individually.
It then assists the author in generating the individual documents
(potentially in different versions)
as well as a document containing the collected series.
Another application is in developing style files
or other kinds of included material
where compilation of the style file could redirect
to a sample or test file.

%%%%%%%%%%%%%%%%%%%%%%%%%%%%%%%%%%%%%%%%%%%%%%%%%%%%%%%%%%%%%%%%%%%%%%%%%%%%%%%%
%%%%%%%%%%%%%%%%%%%%%%%%%%%%%%%%%%%%%%%%%%%%%%%%%%%%%%%%%%%%%%%%%%%%%%%%%%%%%%%%
\section{Usage}

First of all, the package \textsf{childdoc} is \emph{not} a standard
\LaTeXe{} |.sty| style file! Therefore it needs to be invoked in
a non-standard way.

%%%%%%%%%%%%%%%%%%%%%%%%%%%%%%%%%%%%%%%%%%%%%%%%%%%%%%%%%%%%%%%%%%%%%%%%%%%%%%%%
\subsection{Included Files}
\label{sec:include}

%%%%%%%%%%%%%%%%%%%%%%%%%%%%%%%%%%%%%%%%
\DescribeMacro{\childdocmain}
To use the package, add the commands
\begin{center}
\begin{tabular}{l}
|\input{childdoc.def}|\\
|\childdocmain{}|\\
\end{tabular}
\end{center}
at the very top of the main \LaTeX{} file,
in particular \emph{before} the |\documentclass| statement!
The argument of |\childdocmain| should be left empty
(but it must be present).

%%%%%%%%%%%%%%%%%%%%%%%%%%%%%%%%%%%%%%%%
\DescribeMacro{\childdocof}
Furthermore, add the commands
\begin{center}
\begin{tabular}{l}
|\input{childdoc.def}|\\
|\childdocof{|\textit{main}|}|\\
\end{tabular}
\end{center}
at the top of every child file \textit{child}
which is included by |\include{|\textit{child}|}|
from within the main file
(or at least for those files to be compiled individually).
The argument \textit{main} must be the filename of the main file.

There are a couple of
considerations in setting up the main and child documents:

%%%%%%%%%%%%%%%%%%%%%%%%%%%%%%%%%%%%%%%%
\paragraph{Restrictions.}

Please note the following restrictions:
\begin{itemize}
\item
|\childdocmain| must be called with one argument \textit{main}
to ensure compatibility with earlier version of the package.
It must either be empty (|\childdocmain{}|)
or precisely match the filename of the main file in which it is specified.
See \secref{sec:detection} for further information.
\item
The filename \textit{main} must be specified without the |.tex| extension.
\item
The filename \textit{main} is case sensitive
(even in case-insensitive file systems)
due to internal string comparison.
\item
The argument \textit{main} should be fully expanded, it cannot be a macro.
\item
Subdirectories and special characters should be avoided in filenames.
\item
The command |\childdocmain{|\textit{main}|}| must be followed by a whitespace.
It should not be followed immediately by another command
or by a comment mark `|%|'.
This is because the \TeX{} parser reads the token immediately following
the argument of |\childdocmain| and puts it
at the beginning of every child section;
however, a white\-space is ignored.
\end{itemize}

%%%%%%%%%%%%%%%%%%%%%%%%%%%%%%%%%%%%%%%%
\paragraph{Content of Main File.}

It is advisable to place all content in the child files included by |\include|.
Any output contained in the main file will appear in all child documents
unless suppressed manually;
it cannot be suppressed automatically by the |\includeonly| directive
and thus should normally be avoided.
A method to include some content in the main file
by means of conditional processing is described in \secref{sec:conditional}.

%%%%%%%%%%%%%%%%%%%%%%%%%%%%%%%%%%%%%%%%
\paragraph{Page Numbering.}

When only a part of the document is compiled,
the appropriate numbering of pages
(as well as other status parameters)
is determined from the |.aux| files.
The latter contain information from previous passes.
However this information needs to propagate through
all intermediate child documents.
Therefore the page numbering in child documents may well
be inconsistent until the complete document is compiled at least once.

A useful (if unconventional) way to always ensure a consistent
page numbering is to restart the numbering in each child document
and denote the pages by `\textit{child}|.|\textit{page}'
where \textit{child} represents the chapter/section number of the child file.
This can be achieved by the command
|\numberwithin{page}{|\textit{child}|}|
of the \textsf{amsmath} package
where \textit{child} can be |chapter| or |section|
depending on the chosen structuring.
Alternatively, one can modify the macro |\thepage| appropriately
and reset the counter |page| at the start of each child file.

%%%%%%%%%%%%%%%%%%%%%%%%%%%%%%%%%%%%%%%%%%%%%%%%%%%%%%%%%%%%%%%%%%%%%%%%%%%%%%%%
\subsection{Conditional Processing}
\label{sec:conditional}

The package provides a mechanism to compile different versions
of a document. To customise the versions further some conditional processing
can come in handy to distinguish which version is being compiled.
The package provides two macros to describe the compilation context:

%%%%%%%%%%%%%%%%%%%%%%%%%%%%%%%%%%%%%%%%
\DescribeMacro{\ifchilddoc}
The conditional |\ifchilddoc| distinguishes between the compilation of
child documents and the main document:
%
\begin{center}
|\ifchilddoc |\textit{child-code}| |[|\||else |\textit{main-code}]| \||fi|
\end{center}

%%%%%%%%%%%%%%%%%%%%%%%%%%%%%%%%%%%%%%%%
\DescribeMacro{\childdocname}
\DescribeMacro{\childdocjob}
The macro |\childdocname| contains the filename (without extension)
of the main or child file being processed.
Note that |\childdocjob| will always contain the name of the main file.

%%%%%%%%%%%%%%%%%%%%%%%%%%%%%%%%%%%%%%%%
\paragraph{Title Page.}

Conditional processing can be used to include a title or banner page
in the main document when proper precautions are taken.
Importantly, the code in the main file should ensure that the page counter
(as well as other status parameters which are stored in the |.aux| files)
takes the same value after the conditional processing.
Otherwise the page numbers may take divergent values
depending on which part is compiled.

For example, a title page could be declared by:
%
\begin{center}
\begin{tabular}{l}
|\ifchilddoc\||else|\\
|\addtocounter{page}{-1}|\\
\textit{code for title page}\\
|\newpage|\\
|\||fi|
\end{tabular}
\end{center}
%
A banner page for the child documents can be generated by:
%
\begin{center}
\begin{tabular}{l}
|\ifchilddoc|\\
|\addtocounter{page}{-1}|\\
\textit{code for banner page}\\
|\newpage|\\
|\||fi|
\end{tabular}
\end{center}
%
Here one could write a message such as:
\begin{center}
|This is the part \childdocname{} of \childdocjob{}.|
\end{center}

%%%%%%%%%%%%%%%%%%%%%%%%%%%%%%%%%%%%%%%%%%%%%%%%%%%%%%%%%%%%%%%%%%%%%%%%%%%%%%%%
\subsection{Flags}
\label{sec:flags}

The package makes it easy to generate different versions
of the main or child documents.
To this end compilation flags can be defined
and assigned different default values.
They will be particularly useful in conjunction
with the forwarding mechanism described in \secref{sec:forward}.

For example, it may be useful to have a flag |\version|
which can be set to |draft| or |final|.
The document source will contain some conditional code
depending on the value of |\version|.
Suppose further, the flag should default to |final| for the main file
and to |draft| for child files
which is a natural assignment for editing the document.
This is achieved by placing the following code
in the preamble of the main document
(below the |\childdocmain| directive):
%
\begin{center}
\begin{tabular}{l}
|\ifchilddoc|\\
|\providecommand{\version}{draft}|\\
|\||else|\\
|\providecommand{\version}{final}|\\
|\||fi|
\end{tabular}
\end{center}
%
The definition by |\providecommand| makes sure
that previous definitions are not overwritten.
Further statements |\providecommand{\version}{...}|
can thus be added before the above code to override it.

For the main file, one might add a line
(between |\childdocmain| and the above block)
%
\begin{center}
|%\ifchilddoc\||else\providecommand{\version}{draft}\||fi|
\end{center}
%
which can be uncommented to produce a draft version.
Likewise one can add a line to the very top of a child file
(above the |\childdocof{|\textit{main}|}| directive)
%
\begin{center}
|%\providecommand{\version}{final}|
\end{center}
%
which can be uncommented to produce the final version of this child document.

%%%%%%%%%%%%%%%%%%%%%%%%%%%%%%%%%%%%%%%%%%%%%%%%%%%%%%%%%%%%%%%%%%%%%%%%%%%%%%%%
\subsection{Forwarding}
\label{sec:forward}

Different versions of the main or child documents
using compilation flags as described in \secref{sec:flags}
can be (permanently) stored in different files
for convenient compilation, viewing and distribution.
To this end, the package defines a command
to pass on compilation to a different file:

%%%%%%%%%%%%%%%%%%%%%%%%%%%%%%%%%%%%%%%%
\DescribeMacro{\childdocforward}
The command |\childdocforward| redirects processing to
another source file:
%
\begin{center}
\begin{tabular}{l}
|\input{childdoc.def}|\\
|\childdocforward[|\textit{main}|]{|\textit{dest}|}|\\
\end{tabular}
\end{center}
%
The argument \textit{dest} is the destination file
(without extension).
It should be the main file or one of the child files.
Note that further \textsf{childdoc} directives
such as |\childdocof| and |\childdocforward|
in the indicated file will be processed in this form.
The optional argument \textit{main}
passes on directly to the main file \textit{main}
while pretending to compile the child \textit{dest}.
This form behaves as if \textit{dest}
issues |\childdocof{|\textit{main}|}| right away,
and no further \textsf{childdoc} directives will be processed.

%%%%%%%%%%%%%%%%%%%%%%%%%%%%%%%%%%%%%%%%
\DescribeMacro{\...prefix}
In the alternative form |\childdocforwardprefix|,
%
\begin{center}
\begin{tabular}{l}
|\input{childdoc.def}|\\
|\childdocforwardprefix[|\textit{main}|]{|\textit{prefix}|}{|\textit{dest}|}|
\end{tabular}
\end{center}
%
the destination file is determined by a pattern
depending on the current file:
To make this work, the current file must be called
`{\textit{prefix}\hspace{0.2em}\textit{suffix}}'
with \textit{prefix} matching precisely the argument.
Processing is then passed on to the file
`{\textit{dest}\hspace{0.2em}\textit{suffix}}'.
Surely, the same effect is achieved by
directly specifying the
argument `{\textit{dest}\hspace{0.2em}\textit{suffix}}'
in the first form.
However, that requires to set up a different file
for each child. With the alternative form of the command
all these files can have exactly the same content
which simplifies setting them up and maintaining them.

For example, the following file |draft.tex|
with a compilation flag |\version| as described in \secref{sec:flags}
compiles the main document as a draft:
%
\begin{center}
\begin{tabular}{l}
|\def\version{draft}|\\
|\input{childdoc.def}|\\
|\childdocforward{|\textit{main}|}|
\end{tabular}
\end{center}
%
Likewise, the following files |final|\textit{nn}|.tex|
compile the final version of the child document
|child|\textit{nn}|.tex|:
%
\begin{center}
\begin{tabular}{l}
|\def\version{final}|\\
|\input{childdoc.def}|\\
|\childdocforwardprefix{final}{child}|
\end{tabular}
\end{center}
%

Note that when several versions of a main file and/or of each child file
are to be generated, it may be convenient to set up a |Makefile| or
shell script to automatise the process.

%%%%%%%%%%%%%%%%%%%%%%%%%%%%%%%%%%%%%%%%%%%%%%%%%%%%%%%%%%%%%%%%%%%%%%%%%%%%%%%%
\subsection{Command Line Processing}
\label{sec:commandline}

The effect of redirection files can also be achieved by invoking
the \LaTeX{} compiler with a more elaborate command line.
Most conveniently this should be done as part
of a shell script or a |Makefile|.

When using \textsf{childdoc} in the main file, the following
command lines effectively perform a redirection
(note that depending on the shell being used,
backslashes may have to be doubled: `|\|' $\to$ `|\\|'):
%
\begin{center}
|... -jobname "|\textit{target}|" |\\|"|[\textit{flags}]%
|\input{childdoc.def}\childdocforward[|\textit{main}|]{|\textit{dest}|}"|
\end{center}
%
Here \textit{target} is the name of the output file,
\textit{main} is the name of the main file
and \textit{dest} is the name of the main or child file to be processed
(all filenames without extensions).
The optional argument \textit{main} can be omitted
if \textit{main} matches \textit{dest}.
Optionally, compilation \textit{flags} can be defined via |\def| commands.
This command line makes the \TeX{} engine believe
it is compiling the file \textit{target}
whose content is specified as the latter parameter.
The provided code then forwards the processing to
\textit{main} or \textit{dest} as described in \secref{sec:forward}.

%%%%%%%%%%%%%%%%%%%%%%%%%%%%%%%%%%%%%%%%%%%%%%%%%%%%%%%%%%%%%%%%%%%%%%%%%%%%%%%%
\subsection{Include by Input}
\label{sec:input}

Including child documents by |\include| has some restrictions by design.
Most notably, the content of a child document always occupies
its own set of pages; pages cannot be shared between child documents.
Usually, this behaviour makes perfect sense
because each child document contain an essential part of the document.
However, in some situations it may be desirable to compose
a document from a collection of parts
without having mandatory page breaks between then.
For this case, the package
provides a mechanism to include parts
by |\input| which can also be processed individually.
However, by construction this mechanism
requires manual handling of the content to be output.

%%%%%%%%%%%%%%%%%%%%%%%%%%%%%%%%%%%%%%%%
\DescribeMacro{\ifchilddocmanual}
The main file should be prepared as usual, see \secref{sec:include}.
However, the document body must make a distinction
between processing of an individual part and of the main document, e.g.:
%
\begin{center}
\begin{tabular}{l}
|\ifchilddocmanual|\\
|\input{\childdocname}|\\
|\||else|\\
\textit{document body with }|\input{|\textit{part}|}|\\
|\||fi|
\end{tabular}
\end{center}
%
The conditional |\ifchilddocmanual| is true whenever
a part to be included by |\input| is being compiled,
and the name of the part is stored in |\childdocname|.

%%%%%%%%%%%%%%%%%%%%%%%%%%%%%%%%%%%%%%%%
\DescribeMacro{\childdocby}
Each part to be included by |\input| should start with:
%
\begin{center}
\begin{tabular}{l}
|\input{childdoc.def}|\\
|\childdocby{|\textit{main}|}|\\
\end{tabular}
\end{center}
%
The directive |\childdocby| is similar to |\childdocof|
described in \secref{sec:include},
but the subsequent selection of content must be done manually.
To that end, both |\ifchilddoc| and |\ifchilddocmanual|
will be true upon processing of a part,
and the name of the part is stored in |\childdocname|.
Note that |\jobname| will be set to the filename of the current part
so that each part receives an individual |.aux| file
that does not interfere with the |.aux| file(s) of the main document.
This behaviour can be altered by the alternative form
|\childdocby[*]{|\textit{main}|}| (with a non-empty optional argument)
which uses the |.aux| file of the main document
by setting |\jobname| to \textit{main}.

%%%%%%%%%%%%%%%%%%%%%%%%%%%%%%%%%%%%%%%%%%%%%%%%%%%%%%%%%%%%%%%%%%%%%%%%%%%%%%%%
\subsection{Driver Development}
\label{sec:driver}

The \textsf{childdoc} mechanism can also be use for the development
of definition files such as \LaTeX{} styles or classes.
This case differs from the above setup with multiple parts
included by |\include| in that no |\includeonly| should be invoked.
This can be achieved by starting the include file
(before |\ProvidesPackage|) with:
%
\begin{center}
\begin{tabular}{l}
|\input{childdoc.def}|\\
|\childdocforward{|\textit{main}|}|\\
\end{tabular}
\end{center}
%
or alternatively with:
%
\begin{center}
\begin{tabular}{l}
|\input{childdoc.def}|\\
|\childdocby{|\textit{main}|}|\\
\end{tabular}
\end{center}
%
Both forms have slightly different effects as described above.
The main file is prepared as usual, see \secref{sec:include}.

%%%%%%%%%%%%%%%%%%%%%%%%%%%%%%%%%%%%%%%%%%%%%%%%%%%%%%%%%%%%%%%%%%%%%%%%%%%%%%%%
\subsection{Legacy Detection}
\label{sec:detection}

The directive |\childdocmain| in the main file can detect
whether the complete document or merely a child is to be compiled
even without using the directive |\childdocof|.
This method is deprecated because it is less robust
and there is no compelling reason to use it;
it is merely provided for backward compatibility
and it may be removed in future versions.

If the detection mechanism is to be used,
it is mandatory to correctly specify
the filename of the main file as the argument of |\childdocmain|:
%
\begin{center}
\begin{tabular}{l}
|\input{childdoc.def}|\\
|\childdocmain{|\textit{main}|}|\\
\end{tabular}
\end{center}
%
If |\jobname| does not match the argument \textit{main} of |\childdocmain|,
it is assumed that |\jobname| points to the child file to be compiled.
When using |\childdocmain| with the main file specified as argument,
it suffices to start a child file
with just |\input{|\textit{main}|}|
without loading of the package and using |\childdocof|.
If instead all processing is done
with the appropriate \textsf{childdoc} directives,
the argument of \textit{main} of |\childdocmain| can be empty.

An alternative version of the command line processing described
in \secref{sec:commandline} using the detection mechanism reads:
%
\begin{center}
|... -jobname "|\textit{target}|" "|[\textit{flags}]%
[|\def\jobname{|\textit{dest}|}|]|\input{|\textit{main}|}"|
\end{center}

%%%%%%%%%%%%%%%%%%%%%%%%%%%%%%%%%%%%%%%%%%%%%%%%%%%%%%%%%%%%%%%%%%%%%%%%%%%%%%%%
\subsection{Manual Code}
\label{sec:manual}

In case one cannot be certain whether the definitions file |childdoc.def|
is installed on the target \TeX{} distribution
and one prefers not to ship it,
it is conceivable to paste a few relevant commands into the sources.

To that end, drop all statements |\input{childdoc.def}|
and perform the replacements as outlined below.
Instead of |\childdocmain{|\textit{main}|}| add the following code
to the top of the main file:
%
\begin{center}
\begin{tabular}{l}
|\||ifdefined\childdocname\endinput\||fi\newif\ifchilddoc|\\
|\edef\childdocname{\scantokens\expandafter{\jobname\noexpand}}|\\
|\def\childdocmain{|\textit{main}|}\||ifx\childdocmain\childdocname\||else|\\
|\childdoctrue\includeonly{\childdocname}\let\jobname\childdocmain\||fi|\\
\end{tabular}
\end{center}
%
Instead of |\childdocof{|\textit{main}|}| just include the main file
at the top of each child file:
%
\begin{center}
|\input{|\textit{main}|}|
\end{center}
%
A simple redirection |\childdocforward{|\textit{dest}|}| is achieved by:
%
\begin{center}
|\def\jobname{|\textit{dest}|}\input{\jobname}|
\end{center}
%
The redirection with prefix
|\childdocforwardprefix[|\textit{prefix}|]{|\textit{dest}|}|
is accomplished by:
%
\begin{center}
\begin{tabular}{l}
|{\edef\jobname{\scantokens\expandafter{\jobname\noexpand}}|\\
|\def\redirectjob |\textit{prefix}|#1~~~{\gdef\jobname{|\textit{dest}|#1}}|\\
|\expandafter\redirectjob\jobname~~~}\input{\jobname}|
\end{tabular}
\end{center}

In an alternative approach,
child documents can be compiled by a specific command line
without additional code or specific definitions:
%
\begin{center}
|... -jobname "|\textit{target}|" "|[\textit{flags}]%
|\includeonly{|\textit{dest}|}\input{|\textit{main}|}"|
\end{center}
%

%%%%%%%%%%%%%%%%%%%%%%%%%%%%%%%%%%%%%%%%%%%%%%%%%%%%%%%%%%%%%%%%%%%%%%%%%%%%%%%%
%%%%%%%%%%%%%%%%%%%%%%%%%%%%%%%%%%%%%%%%%%%%%%%%%%%%%%%%%%%%%%%%%%%%%%%%%%%%%%%%
\section{Information}

%%%%%%%%%%%%%%%%%%%%%%%%%%%%%%%%%%%%%%%%%%%%%%%%%%%%%%%%%%%%%%%%%%%%%%%%%%%%%%%%
\subsection{Copyright}

Copyright \copyright{} 2017--2018 Niklas Beisert

This work may be distributed and/or modified under the
conditions of the \LaTeX{} Project Public License, either version 1.3
of this license or (at your option) any later version.
The latest version of this license is in
  \url{http://www.latex-project.org/lppl.txt}
and version 1.3 or later is part of all distributions of \LaTeX{}
version 2005/12/01 or later.

This work has the LPPL maintenance status `maintained'.

The Current Maintainer of this work is Niklas Beisert.

This work consists of the files |README.txt|, |childdoc.ins| and |childdoc.dtx|
as well as the derived files |childdoc.def|, |cdocsamp.tex|
with |cdocsch1.tex|, |cdocsch2.tex|, |cdocspt3.tex|, |cdocspt4.tex|,
|cdocsdrf.tex|, |cdocsfn1.tex|, |cdocsfn2.tex|
as well as |childdoc.pdf|.

%%%%%%%%%%%%%%%%%%%%%%%%%%%%%%%%%%%%%%%%%%%%%%%%%%%%%%%%%%%%%%%%%%%%%%%%%%%%%%%%
\subsection{Files and Installation}

The package consists of the files:
%
\begin{center}
\begin{tabular}{ll}
    |README.txt|   & readme file \\
    |childdoc.ins| & installation file \\
    |childdoc.dtx| & source file \\
    |childdoc.def| & definition file \\
    |cdocsamp.tex| & sample main file \\
    |cdocsch1.tex| & sample include file \\
    |cdocsch2.tex| & sample include file \\
    |cdocspt3.tex| & sample part file \\
    |cdocspt4.tex| & sample part file \\
    |cdocsdrf.tex| & sample redirection file \\
    |cdocsfn1.tex| & sample redirection file \\
    |cdocsfn2.tex| & sample redirection file \\
    |childdoc.pdf| & manual
\end{tabular}
\end{center}
%
The distribution consists of the files
|README.txt|, |childdoc.ins| and |childdoc.dtx|.
%
\begin{itemize}
\item
Run (pdf)\LaTeX{} on |childdoc.dtx|
to compile the manual |childdoc.pdf| (this file).
\item
Run \LaTeX{} on |childdoc.ins| to create the definitions file |childdoc.def|
and the sample |cdocsamp.tex| with include files
|cdocsch1.tex|, |cdocsch2.tex|, |cdocspt3.tex|, |cdocspt4.tex|,
|cdocsdrf.tex|, |cdocsfn1.tex|, |cdocsfn2.tex|.
Then copy the file |childdoc.def| to an appropriate directory of your \LaTeX{}
distribution, e.g.\ \textit{texmf-root}|/tex/latex/childdoc|.
\end{itemize}

%%%%%%%%%%%%%%%%%%%%%%%%%%%%%%%%%%%%%%%%%%%%%%%%%%%%%%%%%%%%%%%%%%%%%%%%%%%%%%%%
\subsection{Related CTAN Packages}

There are several other packages which offer a similar functionality:
%
\begin{itemize}
\item
The packages
\href{http://ctan.org/pkg/docmute}{\textsf{docmute}},
\href{http://ctan.org/pkg/includex}{\textsf{includex}} and
\href{http://ctan.org/pkg/standalone}{\textsf{standalone}}
provide commands to include only the document body of
a child file thus allowing both files to be compiled individually.
\item
The packages \href{http://ctan.org/pkg/subdocs}{\textsf{subdocs}}
and \href{http://ctan.org/pkg/subfiles}{\textsf{subfiles}}
provide structures in which the main and child documents can be
encapsulated and allowing them to be compiled individually.
The inclusion mechanism is different from the conventional |\include|.
\item
The package \href{http://ctan.org/pkg/combine}{\textsf{combine}}
is an elaborate solution to combine several documents into one.
\end{itemize}
%
See also the CTAN topic \href{http://ctan.org/topic/subdocs}{\textsf{subdocs}}
for further related packages.
The present package differs from the above solutions in that
a document structure constructed with the conventional |\include| mechanism
just needs two extra commands at the top of every file
such that all constituent files can be compiled individually.

%%%%%%%%%%%%%%%%%%%%%%%%%%%%%%%%%%%%%%%%%%%%%%%%%%%%%%%%%%%%%%%%%%%%%%%%%%%%%%%%
%\subsection{Feature Suggestions}
%
%The following is a list of features which may be useful for future
%versions of this package:
%%
%\begin{itemize}
%\item
%\ldots
%\end{itemize}

%%%%%%%%%%%%%%%%%%%%%%%%%%%%%%%%%%%%%%%%%%%%%%%%%%%%%%%%%%%%%%%%%%%%%%%%%%%%%%%%
\subsection{Revision History}

%%%%%%%%%%%%%%%%%%%%%%%%%%%%%%%%%%%%%%%%
\paragraph{v2.0:} 2018/12/30

\begin{itemize}
\item
immediate forward processing
\item
added |\childdocby| mechanism
\item
manual restructured
\end{itemize}

%%%%%%%%%%%%%%%%%%%%%%%%%%%%%%%%%%%%%%%%
\paragraph{v1.6:} 2018/01/17

\begin{itemize}
\item
application for development of include files
\item
corrections to manual
\end{itemize}

%%%%%%%%%%%%%%%%%%%%%%%%%%%%%%%%%%%%%%%%
\paragraph{v1.5:} 2017/05/21

\begin{itemize}
\item
more complete structuring introduced
\item
|\childdocof| introduced
\item
|\childdoc| renamed to |\childdocmain|
\item
|\childredirect| renamed to |\childdocforward| and |\childdocforwardprefix|
and functionality expanded
\end{itemize}

%%%%%%%%%%%%%%%%%%%%%%%%%%%%%%%%%%%%%%%%
\paragraph{v1.0:} 2017/04/27

\begin{itemize}
\item
manual and install package
\item
first version published on CTAN
\end{itemize}

%%%%%%%%%%%%%%%%%%%%%%%%%%%%%%%%%%%%%%%%
\paragraph{v0.6:} 2017/04/26

\begin{itemize}
\item
redirection mechanism added
\end{itemize}

%%%%%%%%%%%%%%%%%%%%%%%%%%%%%%%%%%%%%%%%
\paragraph{v0.5:} 2017/04/26

\begin{itemize}
\item
functionality in definition file
\end{itemize}


%%%%%%%%%%%%%%%%%%%%%%%%%%%%%%%%%%%%%%%%%%%%%%%%%%%%%%%%%%%%%%%%%%%%%%%%%%%%%%%%
%%%%%%%%%%%%%%%%%%%%%%%%%%%%%%%%%%%%%%%%%%%%%%%%%%%%%%%%%%%%%%%%%%%%%%%%%%%%%%%%
%%%%%%%%%%%%%%%%%%%%%%%%%%%%%%%%%%%%%%%%%%%%%%%%%%%%%%%%%%%%%%%%%%%%%%%%%%%%%%%%
\appendix

\settowidth\MacroIndent{\rmfamily\scriptsize 000\ }

 \DocInput{childdoc.dtx}

\end{document}
%</driver>
% \fi
%
% %%%%%%%%%%%%%%%%%%%%%%%%%%%%%%%%%%%%%%%%%%%%%%%%%%%%%%%%%%%%%%%%%%%%%%%%%%%%%%
% %%%%%%%%%%%%%%%%%%%%%%%%%%%%%%%%%%%%%%%%%%%%%%%%%%%%%%%%%%%%%%%%%%%%%%%%%%%%%%
% \section{Sample}
%\iffalse
%<*samplemain>
%\fi
%
% The following presents a sample document
% with two chapters, two parts, a title page,
% a compile flag as well as three forwarding files to set the flag.
% It consists of eight |.tex| files:
% \begin{center}
% \begin{tabular}{ll}
% |cdocsamp.tex|&main file\\
% |cdocsch1.tex|&include file for chapter 1\\
% |cdocsch2.tex|&include file for chapter 2\\
% |cdocspt3.tex|&include file for part 3\\
% |cdocspt4.tex|&include file for part 4\\
% |cdocsdrf.tex|&forwarding file for main file in draft mode\\
% |cdocsfi1.tex|&forwarding file for final version of chapter 1\\
% |cdocsfi2.tex|&forwarding file for final version of chapter 2\\
% \end{tabular}
% \end{center}
% Each of the eight files can be compiled directly by the \LaTeX{} compiler.
%
% %%%%%%%%%%%%%%%%%%%%%%%%%%%%%%%%%%%%%%
% \paragraph{Main File.}
%
% The main file is called |cdocsamp.tex|.
%
% Load the \textsf{childdoc} definitions and
% declare the filename for the main document:
%    \begin{macrocode}
\input{childdoc.def}
\childdocmain{}
%    \end{macrocode}

% Optional override for |\version| flag:
%    \begin{macrocode}
%%\ifchilddoc\else\providecommand{\version}{draft}\fi
%    \end{macrocode}

% Define the default values for the |\version| flag
% (|final| for the main file and |draft| for childs):
%    \begin{macrocode}
\ifchilddoc
\providecommand{\version}{draft}
\else
\providecommand{\version}{final}
\fi
%    \end{macrocode}

% Load the standard document class:
%    \begin{macrocode}
\documentclass[12pt]{article}
%    \end{macrocode}

% Start the document body:
%    \begin{macrocode}
\begin{document}
%    \end{macrocode}

% Declare a title page.
% Print title, part of document being processed and version flag:
%    \begin{macrocode}
\addtocounter{page}{-1}
\begin{center}
{\LARGE\bfseries{}childdoc example\par}
\vspace{1cm}
\ifchilddoc
\ifchilddocmanual part\else chapter\fi:
`\childdocname' of `\childdocjob'\par
\else
main document: `\childdocjob'\par
\fi
version: \version\par
\end{center}
\newpage
%    \end{macrocode}

% Manually include selected file,
% otherwise process as usual:
%    \begin{macrocode}
\ifchilddocmanual
\section*{part `\childdocname'}
\input{\childdocname}
\else
%    \end{macrocode}

% Include the two chapters:
%    \begin{macrocode}
\include{cdocsch1}
\include{cdocsch2}
%    \end{macrocode}

% Include the two parts unless only chapters should be displayed:
%    \begin{macrocode}
\ifchilddoc\else
\section{part three}
\input{cdocspt3}
\section{part four}
\input{cdocspt4}
\fi
%    \end{macrocode}

% Process as usual until here:
%    \begin{macrocode}
\fi
%    \end{macrocode}

% End of document body:
%    \begin{macrocode}
\end{document}
%    \end{macrocode}
%\iffalse
%</samplemain>
%\fi
%
% %%%%%%%%%%%%%%%%%%%%%%%%%%%%%%%%%%%%%%
% \paragraph{Chapter Include Files.}
%
% The include files are called |cdocsch1.tex| and |cdocsch2.tex|.
%
%\iffalse
%<*samplechap1|samplechap2>
%\fi

% Optional override for |\version| flag:
%    \begin{macrocode}
%%\providecommand{\version}{final}
%    \end{macrocode}

% Include the main document:
%    \begin{macrocode}
\input{childdoc.def}
\childdocof{cdocsamp}
%    \end{macrocode}

%\iffalse
%</samplechap1|samplechap2>
%\fi
%
%\iffalse
%<*samplechap1>
%\fi
% Some text for chapter 1:
%    \begin{macrocode}
\section{one}
some text in chapter one
%    \end{macrocode}

%\iffalse
%</samplechap1>
%\fi
% Some text for chapter 2:
%\iffalse
%<*samplechap2>
%\fi
%    \begin{macrocode}
\section{two}
more text in chapter two
%    \end{macrocode}

%\iffalse
%</samplechap2>
%\fi
%
% %%%%%%%%%%%%%%%%%%%%%%%%%%%%%%%%%%%%%%
% \paragraph{Part Include Files.}
%
% The include files are called |cdocspt3.tex| and |cdocspt4.tex|.
%
%\iffalse
%<*samplepart3|samplepart4>
%\fi

% Optional override for |\version| flag:
%    \begin{macrocode}
%%\providecommand{\version}{final}
%    \end{macrocode}

% Include the main document:
%    \begin{macrocode}
\input{childdoc.def}
\childdocby{cdocsamp}
%    \end{macrocode}

%\iffalse
%</samplepart3|samplepart4>
%\fi
%
%\iffalse
%<*samplepart3>
%\fi
% Some text for part 3:
%    \begin{macrocode}
some text in part three
%    \end{macrocode}

%\iffalse
%</samplepart3>
%\fi
% Some text for part 4:
%\iffalse
%<*samplepart4>
%\fi
%    \begin{macrocode}
more text in part four
%    \end{macrocode}

%\iffalse
%</samplepart4>
%\fi
%
% %%%%%%%%%%%%%%%%%%%%%%%%%%%%%%%%%%%%%%
% \paragraph{Forwarding for a Complete Draft.}
%
% The following forwarding file |cdocsdrf.tex|
% compiles the main document in draft mode:
%\iffalse
%<*sampledraft>
%\fi
%    \begin{macrocode}
\def\version{draft}
\input{childdoc.def}
\childdocforward{cdocsamp}
%    \end{macrocode}

%\iffalse
%</sampledraft>
%\fi
%
% %%%%%%%%%%%%%%%%%%%%%%%%%%%%%%%%%%%%%%
% \paragraph{Forwarding for Final Version of the Chapters.}
%
% The following forwarding files |cdocsfn1.tex| and |cdocsfn2.tex|
% (with identical content)
% compile the final versions of the child documents
% |cdocsch1.tex| and |cdocsch2.tex|, respectively:
%\iffalse
%<*samplefinal>
%\fi
%    \begin{macrocode}
\def\version{final}
\input{childdoc.def}
\childdocforwardprefix[cdocsamp]{cdocsfn}{cdocsch}
%    \end{macrocode}

%\iffalse
%</samplefinal>
%\fi
%
% %%%%%%%%%%%%%%%%%%%%%%%%%%%%%%%%%%%%%%
% \paragraph{Command Line Processing.}
%
% The following three command lines generate the output files
% |cdocscld|, |cdocscl1| and |cdocscl2|
% which should be identical to
% |cdocsdrf|, |cdocsch1| and |cdocsfn2|, respectively:
% \begin{center}
% \begin{tabular}{l}
% |latex -jobname cdocscld \|\\
% |  "\def\version{draft}\input{childdoc.def}\childdocforward{cdocsamp}"|\\
% |latex -jobname cdocscl1 \|\\
% |  "\input{childdoc.def}\childdocforward[cdocsamp]{cdocsch1}"|\\
% |latex -jobname cdocscl2 \|\\
% |  "\def\version{final}\input{childdoc.def}\childdocforward{cdocsch2}"|
% \end{tabular}
% \end{center}
% Note that the trailing backslash on each first line
% merely continues the input to the second line
% (for convenient cut ant paste).
% Furthermore, the command |latex| can be replaced by any
% of its alternative versions such as |pdflatex|.
%
% %%%%%%%%%%%%%%%%%%%%%%%%%%%%%%%%%%%%%%%%%%%%%%%%%%%%%%%%%%%%%%%%%%%%%%%%%%%%%%
% %%%%%%%%%%%%%%%%%%%%%%%%%%%%%%%%%%%%%%%%%%%%%%%%%%%%%%%%%%%%%%%%%%%%%%%%%%%%%%
% \section{Implementation}
%\iffalse
%<*package>
%\fi
%
% This section describes the definitions file |childdoc.def|.

% The definitions cannot be loaded using |\usepackage| or |\RequirePackage|
% which has a mechanism to prevent loading a style file more than once.
% When loading the definitions by means of |\input|
% multiple instances have to be prevented manually:
%\iffalse
%This code needs to be before the `\ProvidesFile' directive
%which is defined at the beginning of this file.
%Therefore it is also placed there and commented out here.
%</package>
%<*discard>
%\fi
%    \begin{macrocode}
\ifdefined\childdocmain\endinput\fi
%    \end{macrocode}
%\iffalse
%</discard>
%<*package>
%\fi
%
% \macro{\ifchilddoc}
% \macro{\ifchilddocmanual}
% The conditional |\ifchilddoc| tells whether a
% child (true) or main (false) document is being compiled.
% The conditional |\ifchilddocmanual| tells whether
% the |\includeonly| mechanism is used (false) or
% the selection of child files must be performed manually (true).
% The definitions initialise to false:
%    \begin{macrocode}
\newif\ifchilddoc
\newif\ifchilddocmanual
%    \end{macrocode}

% \macro{\childdocname}
% \macro{\childdocjob}
% The macro |\childdocname| stores the name of the main document
% to be compiled. The macro |\childdocjob| stores the name of
% the document on which the \LaTeX{} compiler was originally invoked.
% The content of |\jobname| cannot be compared
% to filenames specified in the source due to different catcodes.
% The following code rescans |\jobname|, stores the result
% in |\childdocname| and saves a copy in |\childdocjob|:
%    \begin{macrocode}
\edef\childdocname{\scantokens\expandafter{\jobname\noexpand}}
\let\childdocjob\childdocname
%    \end{macrocode}

% \macro{\childdocdisable}
% The macro |\childdocdisable| prevents the main file
% from being processed more than once.
% At this stage, the main document command |\childdocmain|
% is assumed to be called once again where it should do nothing.
% Any subsequent call to it should prevent
% a secondary processing of the main document
% It overwrites the forwarding commands
% |\childdocof| and |\childdocforward|
% with empty macros to prevent further inclusions of the main document:
%    \begin{macrocode}
\newcommand{\childdocdisable}
{
  \renewcommand{\childdocmain}[1]{\renewcommand{\childdocmain}[1]{\endinput}}
  \renewcommand{\childdocof}[1]{}
  \renewcommand{\childdocby}[2][]{}
  \renewcommand{\childdocforward}[2][]{}
  \renewcommand{\childdocdisable}{}
}
%    \end{macrocode}

% \macro{\childdocmain}
% The macro |\childdocmain| is to be called at the top of the main file
% with nothing or the main filename (without extension) as argument.
% First, it breaks loops.
% If the argument is not empty and does not match |\childdocname|
% (which is set by the first inclusion of |childdoc.def|),
% |\ifchilddoc| is set to true, |\includeonly| is applied to the child file
% and |\jobname| is set to the main file
% (for proper handling of |.aux| files):
%    \begin{macrocode}
\newcommand{\childdocmain}[1]
{
  \childdocdisable\childdocmain{}
  \if?#1?\else
    \begingroup
      \def\childdoctmp{#1}
      \ifx\childdoctmp\childdocname
        \def\childdoctmp{}
      \else
        \def\childdoctmp
        {
          \childdoctrue
          \includeonly{\childdocname}
          \def\childdocjob{#1}
          \def\jobname{#1}
        }
      \fi
      \expandafter
    \endgroup
    \childdoctmp
  \fi
}
%    \end{macrocode}

% \macro{\childdocof}
% The command |\childdocof| redirects
% compilation to the main file |#1|.
%    \begin{macrocode}
\newcommand{\childdocof}[1]
{
  \childdocdisable
  \childdoctrue
  \includeonly{\childdocname}
  \def\jobname{#1}
  \def\childdocjob{#1}
  \input{#1}
}
%    \end{macrocode}

% \macro{\childdocby}
% The command |\childdocby| ....
%    \begin{macrocode}
\newcommand{\childdocby}[2][]
{
  \childdocdisable
  \childdoctrue
  \childdocmanualtrue
  \if?#1?\else
    \def\jobname{#2}
  \fi
  \def\childdocjob{#2}
  \input{#2}
  \endinput
}
%    \end{macrocode}

% \macro{\childdocforward}
% The command |\childdocforward| redirects
% compilation to the main file or
% (if the optional argument is given) a child file.
% Parameters are set as if the main file
% or a child file starting with |\childdocof| was compiled.
% Then compilation is handed over to the main file:
%    \begin{macrocode}
\newcommand{\childdocforward}[2][]
{
  \begingroup
    \if?#1?
      \def\childdoctmp
      {
        \def\childdocname{#2}
        \def\childdocjob{#2}
        \def\jobname{#2}
        \input{#2}
        \endinput
      }
    \else
      \def\childdoctmp
      {
        \childdocdisable
        \def\childdocname{#2}
        \childdoctrue
        \includeonly{#2}
        \def\childdocjob{#1}
        \def\jobname{#1}
        \input{#1}
        \endinput
      }
    \fi
    \expandafter
  \endgroup
  \childdoctmp
}
%    \end{macrocode}

% \macro{\childdocforwardprefix}
% The command |\childdocforwardprefix| redirects
% compilation to the main or a child file by means of a pattern.
% The prefix |#1| in the current filename is replaced by |#2|
% and the suffix of the current filename is kept
% (it is assumed that the filename does not contain the substring `|~~~|'
% which is used as a delimiter).
% Compilation is handed over to the new file by |\childdocforward|:
%    \begin{macrocode}
\newcommand{\childdocforwardprefix}[3][]
{
  \begingroup
    \def\childdocextract #2##1~~~{\def\childdoctmp{\childdocforward[#1]{#3##1}}}
    \expandafter\childdocextract\childdocname~~~
    \expandafter
  \endgroup
  \childdoctmp
}
%    \end{macrocode}

% \macro{\childdoc}
% The deprecated macro |\childdoc| is a legacy version of |\childdocmain|:
%    \begin{macrocode}
\newcommand{\childdoc}{\childdocmain}
%    \end{macrocode}

% \macro{\childdocredirect}
% The deprecated macro |\childdocredirect| is a legacy version
% of |\childdocforward| and |\childdocforwardprefix|:
%    \begin{macrocode}
\newcommand{\childdocredirect}[2][]
{
  \begingroup
    \if?#1?
      \def\childdoctmp{\childdocforward{#2}}
    \else
      \def\childdoctmp{\childdocforwardprefix{#1}{#2}}
    \fi
    \expandafter
  \endgroup
  \childdoctmp
}
%    \end{macrocode}

%\iffalse
%</package>
%\fi
%
\endinput
|\\
|\childdocforwardprefix[|\textit{main}|]{|\textit{prefix}|}{|\textit{dest}|}|
\end{tabular}
\end{center}
%
the destination file is determined by a pattern
depending on the current file:
To make this work, the current file must be called
`{\textit{prefix}\hspace{0.2em}\textit{suffix}}'
with \textit{prefix} matching precisely the argument.
Processing is then passed on to the file
`{\textit{dest}\hspace{0.2em}\textit{suffix}}'.
Surely, the same effect is achieved by
directly specifying the
argument `{\textit{dest}\hspace{0.2em}\textit{suffix}}'
in the first form.
However, that requires to set up a different file
for each child. With the alternative form of the command
all these files can have exactly the same content
which simplifies setting them up and maintaining them.

For example, the following file |draft.tex|
with a compilation flag |\version| as described in \secref{sec:flags}
compiles the main document as a draft:
%
\begin{center}
\begin{tabular}{l}
|\def\version{draft}|\\
|% \iffalse
%
% childdoc.dtx Copyright (C) 2017-2018 Niklas Beisert
%
% This work may be distributed and/or modified under the
% conditions of the LaTeX Project Public License, either version 1.3
% of this license or (at your option) any later version.
% The latest version of this license is in
%   http://www.latex-project.org/lppl.txt
% and version 1.3 or later is part of all distributions of LaTeX
% version 2005/12/01 or later.
%
% This work has the LPPL maintenance status `maintained'.
%
% The Current Maintainer of this work is Niklas Beisert.
%
% This work consists of the files childdoc.dtx and childdoc.ins
% and the derived files childdoc.def and cdocsamp.tex with
% cdocsch1.tex, cdocsch2.tex, cdocsdrf.tex, cdocsfn1.tex, cdocsfn2.tex.
%
%<package>\ifdefined\childdocmain\endinput\fi
%<package>\ProvidesFile{childdoc.def}[2018/12/30 v2.0 child document driver]
%<samplemain>\ProvidesFile{cdocsamp.tex}[2018/12/30 v2.0 sample for childdoc]
%<*driver>
%\ProvidesFile{childdoc.drv}[2018/12/30 v2.0 childdoc reference manual file]
\PassOptionsToClass{10pt,a4paper}{article}
\documentclass{ltxdoc}

\usepackage[margin=35mm]{geometry}
\usepackage{hyperref}
\usepackage{hyperxmp}
\usepackage[usenames]{color}

\hypersetup{colorlinks=true}
\hypersetup{pdfstartview=FitH}
\hypersetup{pdfpagemode=UseNone}
\hypersetup{pdfsource={}}
\hypersetup{pdflang={en-UK}}
\hypersetup{pdfcopyright={Copyright 2017-2018 Niklas Beisert.
  This work may be distributed and/or modified under the
  conditions of the LaTeX Project Public License, either version 1.3
  of this license or (at your option) any later version.}}
\hypersetup{pdflicenseurl={http://www.latex-project.org/lppl.txt}}
\hypersetup{pdfcontactaddress={ETH Zurich, ITP, HIT K,
  Wolfgang-Pauli-Strasse 27}}
\hypersetup{pdfcontactpostcode={8093}}
\hypersetup{pdfcontactcity={Zurich}}
\hypersetup{pdfcontactcountry={Switzerland}}
\hypersetup{pdfcontactemail={nbeisert@itp.phys.ethz.ch}}
\hypersetup{pdfcontacturl={http://people.phys.ethz.ch/\xmptilde nbeisert/}}

\newcommand{\secref}[1]{\hyperref[#1]{section \ref*{#1}}}

\parskip1ex
\parindent0pt
\let\olditemize\itemize
\def\itemize{\olditemize\parskip0pt}

\begin{document}

\title{The \textsf{childdoc} Package}
\hypersetup{pdftitle={The childdoc Package}}
\author{Niklas Beisert\\[2ex]
  Institut f\"ur Theoretische Physik\\
  Eidgen\"ossische Technische Hochschule Z\"urich\\
  Wolfgang-Pauli-Strasse 27, 8093 Z\"urich, Switzerland\\[1ex]
  \href{mailto:nbeisert@itp.phys.ethz.ch}
  {\texttt{nbeisert@itp.phys.ethz.ch}}}
\hypersetup{pdfauthor={Niklas Beisert}}
\hypersetup{pdfsubject={Manual for the LaTeX2e Package childdoc}}
\date{30 December 2018, \textsf{v2.0}}
\maketitle

\begin{abstract}\noindent
\textsf{childdoc} is a \LaTeXe{} package
that enables the direct compilation
of document sections included by |\include|
to individual files.
\end{abstract}

\begingroup
\parskip0ex
\tableofcontents
\endgroup

%%%%%%%%%%%%%%%%%%%%%%%%%%%%%%%%%%%%%%%%%%%%%%%%%%%%%%%%%%%%%%%%%%%%%%%%%%%%%%%%
%%%%%%%%%%%%%%%%%%%%%%%%%%%%%%%%%%%%%%%%%%%%%%%%%%%%%%%%%%%%%%%%%%%%%%%%%%%%%%%%
\section{Introduction}

\LaTeX{} provides a mechanism to structure a large document (such as a book)
into a main file and several child files (containing the chapters)
using the |\include| command.
This mechanism is beneficial for documents
which span hundreds of pages in order to
make the source file(s) more manageable.
Moreover, compilation can be restricted to
selected child files by means of the |\includeonly| command.
The latter feature can be used to reduce the compilation time while editing
(this was significantly more useful in the earlier days of \LaTeX{})
or to generate a smaller document which is easier to navigate.
Another application of |\includeonly| is to generate
documents consisting of selected parts of the complete document.

However, there are a few drawbacks of the plain |\include| mechanism:
\begin{itemize}
\item
The child files cannot be compiled on their own,
they can only be compiled via the main file.
A naive editing environment
(such as a text editor with an option
to have the current file processed by \LaTeX)
may require one to switch to the main file before compiling;
attempting to compile the child file produces errors.
\item
The main file must be modified (each time)
to adjust the |\includeonly| command
to the present needs. This easily leaves the main file in a messy state.
\item
The generated document will always carry the filename
of the main document. This is inconvenient if
several child files are to be compiled and
to be kept for distribution.
\end{itemize}

The present package provides a simple interface
to make child files individually compilable by \LaTeX{}.
Compiling a child file then has the same effect as compiling
the main file with an |\includeonly| command
to select the appropriate child.
Moreover the generated document will carry the name of the child
rather than the main file.
This resolves all three above issues.

This feature is meant to make the editing of books,
thesis documents and lecture notes somewhat more convenient.
However, the package can also be used efficiently for
composing a series of documents (such as exercise sheets)
which are typically distributed individually.
It then assists the author in generating the individual documents
(potentially in different versions)
as well as a document containing the collected series.
Another application is in developing style files
or other kinds of included material
where compilation of the style file could redirect
to a sample or test file.

%%%%%%%%%%%%%%%%%%%%%%%%%%%%%%%%%%%%%%%%%%%%%%%%%%%%%%%%%%%%%%%%%%%%%%%%%%%%%%%%
%%%%%%%%%%%%%%%%%%%%%%%%%%%%%%%%%%%%%%%%%%%%%%%%%%%%%%%%%%%%%%%%%%%%%%%%%%%%%%%%
\section{Usage}

First of all, the package \textsf{childdoc} is \emph{not} a standard
\LaTeXe{} |.sty| style file! Therefore it needs to be invoked in
a non-standard way.

%%%%%%%%%%%%%%%%%%%%%%%%%%%%%%%%%%%%%%%%%%%%%%%%%%%%%%%%%%%%%%%%%%%%%%%%%%%%%%%%
\subsection{Included Files}
\label{sec:include}

%%%%%%%%%%%%%%%%%%%%%%%%%%%%%%%%%%%%%%%%
\DescribeMacro{\childdocmain}
To use the package, add the commands
\begin{center}
\begin{tabular}{l}
|\input{childdoc.def}|\\
|\childdocmain{}|\\
\end{tabular}
\end{center}
at the very top of the main \LaTeX{} file,
in particular \emph{before} the |\documentclass| statement!
The argument of |\childdocmain| should be left empty
(but it must be present).

%%%%%%%%%%%%%%%%%%%%%%%%%%%%%%%%%%%%%%%%
\DescribeMacro{\childdocof}
Furthermore, add the commands
\begin{center}
\begin{tabular}{l}
|\input{childdoc.def}|\\
|\childdocof{|\textit{main}|}|\\
\end{tabular}
\end{center}
at the top of every child file \textit{child}
which is included by |\include{|\textit{child}|}|
from within the main file
(or at least for those files to be compiled individually).
The argument \textit{main} must be the filename of the main file.

There are a couple of
considerations in setting up the main and child documents:

%%%%%%%%%%%%%%%%%%%%%%%%%%%%%%%%%%%%%%%%
\paragraph{Restrictions.}

Please note the following restrictions:
\begin{itemize}
\item
|\childdocmain| must be called with one argument \textit{main}
to ensure compatibility with earlier version of the package.
It must either be empty (|\childdocmain{}|)
or precisely match the filename of the main file in which it is specified.
See \secref{sec:detection} for further information.
\item
The filename \textit{main} must be specified without the |.tex| extension.
\item
The filename \textit{main} is case sensitive
(even in case-insensitive file systems)
due to internal string comparison.
\item
The argument \textit{main} should be fully expanded, it cannot be a macro.
\item
Subdirectories and special characters should be avoided in filenames.
\item
The command |\childdocmain{|\textit{main}|}| must be followed by a whitespace.
It should not be followed immediately by another command
or by a comment mark `|%|'.
This is because the \TeX{} parser reads the token immediately following
the argument of |\childdocmain| and puts it
at the beginning of every child section;
however, a white\-space is ignored.
\end{itemize}

%%%%%%%%%%%%%%%%%%%%%%%%%%%%%%%%%%%%%%%%
\paragraph{Content of Main File.}

It is advisable to place all content in the child files included by |\include|.
Any output contained in the main file will appear in all child documents
unless suppressed manually;
it cannot be suppressed automatically by the |\includeonly| directive
and thus should normally be avoided.
A method to include some content in the main file
by means of conditional processing is described in \secref{sec:conditional}.

%%%%%%%%%%%%%%%%%%%%%%%%%%%%%%%%%%%%%%%%
\paragraph{Page Numbering.}

When only a part of the document is compiled,
the appropriate numbering of pages
(as well as other status parameters)
is determined from the |.aux| files.
The latter contain information from previous passes.
However this information needs to propagate through
all intermediate child documents.
Therefore the page numbering in child documents may well
be inconsistent until the complete document is compiled at least once.

A useful (if unconventional) way to always ensure a consistent
page numbering is to restart the numbering in each child document
and denote the pages by `\textit{child}|.|\textit{page}'
where \textit{child} represents the chapter/section number of the child file.
This can be achieved by the command
|\numberwithin{page}{|\textit{child}|}|
of the \textsf{amsmath} package
where \textit{child} can be |chapter| or |section|
depending on the chosen structuring.
Alternatively, one can modify the macro |\thepage| appropriately
and reset the counter |page| at the start of each child file.

%%%%%%%%%%%%%%%%%%%%%%%%%%%%%%%%%%%%%%%%%%%%%%%%%%%%%%%%%%%%%%%%%%%%%%%%%%%%%%%%
\subsection{Conditional Processing}
\label{sec:conditional}

The package provides a mechanism to compile different versions
of a document. To customise the versions further some conditional processing
can come in handy to distinguish which version is being compiled.
The package provides two macros to describe the compilation context:

%%%%%%%%%%%%%%%%%%%%%%%%%%%%%%%%%%%%%%%%
\DescribeMacro{\ifchilddoc}
The conditional |\ifchilddoc| distinguishes between the compilation of
child documents and the main document:
%
\begin{center}
|\ifchilddoc |\textit{child-code}| |[|\||else |\textit{main-code}]| \||fi|
\end{center}

%%%%%%%%%%%%%%%%%%%%%%%%%%%%%%%%%%%%%%%%
\DescribeMacro{\childdocname}
\DescribeMacro{\childdocjob}
The macro |\childdocname| contains the filename (without extension)
of the main or child file being processed.
Note that |\childdocjob| will always contain the name of the main file.

%%%%%%%%%%%%%%%%%%%%%%%%%%%%%%%%%%%%%%%%
\paragraph{Title Page.}

Conditional processing can be used to include a title or banner page
in the main document when proper precautions are taken.
Importantly, the code in the main file should ensure that the page counter
(as well as other status parameters which are stored in the |.aux| files)
takes the same value after the conditional processing.
Otherwise the page numbers may take divergent values
depending on which part is compiled.

For example, a title page could be declared by:
%
\begin{center}
\begin{tabular}{l}
|\ifchilddoc\||else|\\
|\addtocounter{page}{-1}|\\
\textit{code for title page}\\
|\newpage|\\
|\||fi|
\end{tabular}
\end{center}
%
A banner page for the child documents can be generated by:
%
\begin{center}
\begin{tabular}{l}
|\ifchilddoc|\\
|\addtocounter{page}{-1}|\\
\textit{code for banner page}\\
|\newpage|\\
|\||fi|
\end{tabular}
\end{center}
%
Here one could write a message such as:
\begin{center}
|This is the part \childdocname{} of \childdocjob{}.|
\end{center}

%%%%%%%%%%%%%%%%%%%%%%%%%%%%%%%%%%%%%%%%%%%%%%%%%%%%%%%%%%%%%%%%%%%%%%%%%%%%%%%%
\subsection{Flags}
\label{sec:flags}

The package makes it easy to generate different versions
of the main or child documents.
To this end compilation flags can be defined
and assigned different default values.
They will be particularly useful in conjunction
with the forwarding mechanism described in \secref{sec:forward}.

For example, it may be useful to have a flag |\version|
which can be set to |draft| or |final|.
The document source will contain some conditional code
depending on the value of |\version|.
Suppose further, the flag should default to |final| for the main file
and to |draft| for child files
which is a natural assignment for editing the document.
This is achieved by placing the following code
in the preamble of the main document
(below the |\childdocmain| directive):
%
\begin{center}
\begin{tabular}{l}
|\ifchilddoc|\\
|\providecommand{\version}{draft}|\\
|\||else|\\
|\providecommand{\version}{final}|\\
|\||fi|
\end{tabular}
\end{center}
%
The definition by |\providecommand| makes sure
that previous definitions are not overwritten.
Further statements |\providecommand{\version}{...}|
can thus be added before the above code to override it.

For the main file, one might add a line
(between |\childdocmain| and the above block)
%
\begin{center}
|%\ifchilddoc\||else\providecommand{\version}{draft}\||fi|
\end{center}
%
which can be uncommented to produce a draft version.
Likewise one can add a line to the very top of a child file
(above the |\childdocof{|\textit{main}|}| directive)
%
\begin{center}
|%\providecommand{\version}{final}|
\end{center}
%
which can be uncommented to produce the final version of this child document.

%%%%%%%%%%%%%%%%%%%%%%%%%%%%%%%%%%%%%%%%%%%%%%%%%%%%%%%%%%%%%%%%%%%%%%%%%%%%%%%%
\subsection{Forwarding}
\label{sec:forward}

Different versions of the main or child documents
using compilation flags as described in \secref{sec:flags}
can be (permanently) stored in different files
for convenient compilation, viewing and distribution.
To this end, the package defines a command
to pass on compilation to a different file:

%%%%%%%%%%%%%%%%%%%%%%%%%%%%%%%%%%%%%%%%
\DescribeMacro{\childdocforward}
The command |\childdocforward| redirects processing to
another source file:
%
\begin{center}
\begin{tabular}{l}
|\input{childdoc.def}|\\
|\childdocforward[|\textit{main}|]{|\textit{dest}|}|\\
\end{tabular}
\end{center}
%
The argument \textit{dest} is the destination file
(without extension).
It should be the main file or one of the child files.
Note that further \textsf{childdoc} directives
such as |\childdocof| and |\childdocforward|
in the indicated file will be processed in this form.
The optional argument \textit{main}
passes on directly to the main file \textit{main}
while pretending to compile the child \textit{dest}.
This form behaves as if \textit{dest}
issues |\childdocof{|\textit{main}|}| right away,
and no further \textsf{childdoc} directives will be processed.

%%%%%%%%%%%%%%%%%%%%%%%%%%%%%%%%%%%%%%%%
\DescribeMacro{\...prefix}
In the alternative form |\childdocforwardprefix|,
%
\begin{center}
\begin{tabular}{l}
|\input{childdoc.def}|\\
|\childdocforwardprefix[|\textit{main}|]{|\textit{prefix}|}{|\textit{dest}|}|
\end{tabular}
\end{center}
%
the destination file is determined by a pattern
depending on the current file:
To make this work, the current file must be called
`{\textit{prefix}\hspace{0.2em}\textit{suffix}}'
with \textit{prefix} matching precisely the argument.
Processing is then passed on to the file
`{\textit{dest}\hspace{0.2em}\textit{suffix}}'.
Surely, the same effect is achieved by
directly specifying the
argument `{\textit{dest}\hspace{0.2em}\textit{suffix}}'
in the first form.
However, that requires to set up a different file
for each child. With the alternative form of the command
all these files can have exactly the same content
which simplifies setting them up and maintaining them.

For example, the following file |draft.tex|
with a compilation flag |\version| as described in \secref{sec:flags}
compiles the main document as a draft:
%
\begin{center}
\begin{tabular}{l}
|\def\version{draft}|\\
|\input{childdoc.def}|\\
|\childdocforward{|\textit{main}|}|
\end{tabular}
\end{center}
%
Likewise, the following files |final|\textit{nn}|.tex|
compile the final version of the child document
|child|\textit{nn}|.tex|:
%
\begin{center}
\begin{tabular}{l}
|\def\version{final}|\\
|\input{childdoc.def}|\\
|\childdocforwardprefix{final}{child}|
\end{tabular}
\end{center}
%

Note that when several versions of a main file and/or of each child file
are to be generated, it may be convenient to set up a |Makefile| or
shell script to automatise the process.

%%%%%%%%%%%%%%%%%%%%%%%%%%%%%%%%%%%%%%%%%%%%%%%%%%%%%%%%%%%%%%%%%%%%%%%%%%%%%%%%
\subsection{Command Line Processing}
\label{sec:commandline}

The effect of redirection files can also be achieved by invoking
the \LaTeX{} compiler with a more elaborate command line.
Most conveniently this should be done as part
of a shell script or a |Makefile|.

When using \textsf{childdoc} in the main file, the following
command lines effectively perform a redirection
(note that depending on the shell being used,
backslashes may have to be doubled: `|\|' $\to$ `|\\|'):
%
\begin{center}
|... -jobname "|\textit{target}|" |\\|"|[\textit{flags}]%
|\input{childdoc.def}\childdocforward[|\textit{main}|]{|\textit{dest}|}"|
\end{center}
%
Here \textit{target} is the name of the output file,
\textit{main} is the name of the main file
and \textit{dest} is the name of the main or child file to be processed
(all filenames without extensions).
The optional argument \textit{main} can be omitted
if \textit{main} matches \textit{dest}.
Optionally, compilation \textit{flags} can be defined via |\def| commands.
This command line makes the \TeX{} engine believe
it is compiling the file \textit{target}
whose content is specified as the latter parameter.
The provided code then forwards the processing to
\textit{main} or \textit{dest} as described in \secref{sec:forward}.

%%%%%%%%%%%%%%%%%%%%%%%%%%%%%%%%%%%%%%%%%%%%%%%%%%%%%%%%%%%%%%%%%%%%%%%%%%%%%%%%
\subsection{Include by Input}
\label{sec:input}

Including child documents by |\include| has some restrictions by design.
Most notably, the content of a child document always occupies
its own set of pages; pages cannot be shared between child documents.
Usually, this behaviour makes perfect sense
because each child document contain an essential part of the document.
However, in some situations it may be desirable to compose
a document from a collection of parts
without having mandatory page breaks between then.
For this case, the package
provides a mechanism to include parts
by |\input| which can also be processed individually.
However, by construction this mechanism
requires manual handling of the content to be output.

%%%%%%%%%%%%%%%%%%%%%%%%%%%%%%%%%%%%%%%%
\DescribeMacro{\ifchilddocmanual}
The main file should be prepared as usual, see \secref{sec:include}.
However, the document body must make a distinction
between processing of an individual part and of the main document, e.g.:
%
\begin{center}
\begin{tabular}{l}
|\ifchilddocmanual|\\
|\input{\childdocname}|\\
|\||else|\\
\textit{document body with }|\input{|\textit{part}|}|\\
|\||fi|
\end{tabular}
\end{center}
%
The conditional |\ifchilddocmanual| is true whenever
a part to be included by |\input| is being compiled,
and the name of the part is stored in |\childdocname|.

%%%%%%%%%%%%%%%%%%%%%%%%%%%%%%%%%%%%%%%%
\DescribeMacro{\childdocby}
Each part to be included by |\input| should start with:
%
\begin{center}
\begin{tabular}{l}
|\input{childdoc.def}|\\
|\childdocby{|\textit{main}|}|\\
\end{tabular}
\end{center}
%
The directive |\childdocby| is similar to |\childdocof|
described in \secref{sec:include},
but the subsequent selection of content must be done manually.
To that end, both |\ifchilddoc| and |\ifchilddocmanual|
will be true upon processing of a part,
and the name of the part is stored in |\childdocname|.
Note that |\jobname| will be set to the filename of the current part
so that each part receives an individual |.aux| file
that does not interfere with the |.aux| file(s) of the main document.
This behaviour can be altered by the alternative form
|\childdocby[*]{|\textit{main}|}| (with a non-empty optional argument)
which uses the |.aux| file of the main document
by setting |\jobname| to \textit{main}.

%%%%%%%%%%%%%%%%%%%%%%%%%%%%%%%%%%%%%%%%%%%%%%%%%%%%%%%%%%%%%%%%%%%%%%%%%%%%%%%%
\subsection{Driver Development}
\label{sec:driver}

The \textsf{childdoc} mechanism can also be use for the development
of definition files such as \LaTeX{} styles or classes.
This case differs from the above setup with multiple parts
included by |\include| in that no |\includeonly| should be invoked.
This can be achieved by starting the include file
(before |\ProvidesPackage|) with:
%
\begin{center}
\begin{tabular}{l}
|\input{childdoc.def}|\\
|\childdocforward{|\textit{main}|}|\\
\end{tabular}
\end{center}
%
or alternatively with:
%
\begin{center}
\begin{tabular}{l}
|\input{childdoc.def}|\\
|\childdocby{|\textit{main}|}|\\
\end{tabular}
\end{center}
%
Both forms have slightly different effects as described above.
The main file is prepared as usual, see \secref{sec:include}.

%%%%%%%%%%%%%%%%%%%%%%%%%%%%%%%%%%%%%%%%%%%%%%%%%%%%%%%%%%%%%%%%%%%%%%%%%%%%%%%%
\subsection{Legacy Detection}
\label{sec:detection}

The directive |\childdocmain| in the main file can detect
whether the complete document or merely a child is to be compiled
even without using the directive |\childdocof|.
This method is deprecated because it is less robust
and there is no compelling reason to use it;
it is merely provided for backward compatibility
and it may be removed in future versions.

If the detection mechanism is to be used,
it is mandatory to correctly specify
the filename of the main file as the argument of |\childdocmain|:
%
\begin{center}
\begin{tabular}{l}
|\input{childdoc.def}|\\
|\childdocmain{|\textit{main}|}|\\
\end{tabular}
\end{center}
%
If |\jobname| does not match the argument \textit{main} of |\childdocmain|,
it is assumed that |\jobname| points to the child file to be compiled.
When using |\childdocmain| with the main file specified as argument,
it suffices to start a child file
with just |\input{|\textit{main}|}|
without loading of the package and using |\childdocof|.
If instead all processing is done
with the appropriate \textsf{childdoc} directives,
the argument of \textit{main} of |\childdocmain| can be empty.

An alternative version of the command line processing described
in \secref{sec:commandline} using the detection mechanism reads:
%
\begin{center}
|... -jobname "|\textit{target}|" "|[\textit{flags}]%
[|\def\jobname{|\textit{dest}|}|]|\input{|\textit{main}|}"|
\end{center}

%%%%%%%%%%%%%%%%%%%%%%%%%%%%%%%%%%%%%%%%%%%%%%%%%%%%%%%%%%%%%%%%%%%%%%%%%%%%%%%%
\subsection{Manual Code}
\label{sec:manual}

In case one cannot be certain whether the definitions file |childdoc.def|
is installed on the target \TeX{} distribution
and one prefers not to ship it,
it is conceivable to paste a few relevant commands into the sources.

To that end, drop all statements |\input{childdoc.def}|
and perform the replacements as outlined below.
Instead of |\childdocmain{|\textit{main}|}| add the following code
to the top of the main file:
%
\begin{center}
\begin{tabular}{l}
|\||ifdefined\childdocname\endinput\||fi\newif\ifchilddoc|\\
|\edef\childdocname{\scantokens\expandafter{\jobname\noexpand}}|\\
|\def\childdocmain{|\textit{main}|}\||ifx\childdocmain\childdocname\||else|\\
|\childdoctrue\includeonly{\childdocname}\let\jobname\childdocmain\||fi|\\
\end{tabular}
\end{center}
%
Instead of |\childdocof{|\textit{main}|}| just include the main file
at the top of each child file:
%
\begin{center}
|\input{|\textit{main}|}|
\end{center}
%
A simple redirection |\childdocforward{|\textit{dest}|}| is achieved by:
%
\begin{center}
|\def\jobname{|\textit{dest}|}\input{\jobname}|
\end{center}
%
The redirection with prefix
|\childdocforwardprefix[|\textit{prefix}|]{|\textit{dest}|}|
is accomplished by:
%
\begin{center}
\begin{tabular}{l}
|{\edef\jobname{\scantokens\expandafter{\jobname\noexpand}}|\\
|\def\redirectjob |\textit{prefix}|#1~~~{\gdef\jobname{|\textit{dest}|#1}}|\\
|\expandafter\redirectjob\jobname~~~}\input{\jobname}|
\end{tabular}
\end{center}

In an alternative approach,
child documents can be compiled by a specific command line
without additional code or specific definitions:
%
\begin{center}
|... -jobname "|\textit{target}|" "|[\textit{flags}]%
|\includeonly{|\textit{dest}|}\input{|\textit{main}|}"|
\end{center}
%

%%%%%%%%%%%%%%%%%%%%%%%%%%%%%%%%%%%%%%%%%%%%%%%%%%%%%%%%%%%%%%%%%%%%%%%%%%%%%%%%
%%%%%%%%%%%%%%%%%%%%%%%%%%%%%%%%%%%%%%%%%%%%%%%%%%%%%%%%%%%%%%%%%%%%%%%%%%%%%%%%
\section{Information}

%%%%%%%%%%%%%%%%%%%%%%%%%%%%%%%%%%%%%%%%%%%%%%%%%%%%%%%%%%%%%%%%%%%%%%%%%%%%%%%%
\subsection{Copyright}

Copyright \copyright{} 2017--2018 Niklas Beisert

This work may be distributed and/or modified under the
conditions of the \LaTeX{} Project Public License, either version 1.3
of this license or (at your option) any later version.
The latest version of this license is in
  \url{http://www.latex-project.org/lppl.txt}
and version 1.3 or later is part of all distributions of \LaTeX{}
version 2005/12/01 or later.

This work has the LPPL maintenance status `maintained'.

The Current Maintainer of this work is Niklas Beisert.

This work consists of the files |README.txt|, |childdoc.ins| and |childdoc.dtx|
as well as the derived files |childdoc.def|, |cdocsamp.tex|
with |cdocsch1.tex|, |cdocsch2.tex|, |cdocspt3.tex|, |cdocspt4.tex|,
|cdocsdrf.tex|, |cdocsfn1.tex|, |cdocsfn2.tex|
as well as |childdoc.pdf|.

%%%%%%%%%%%%%%%%%%%%%%%%%%%%%%%%%%%%%%%%%%%%%%%%%%%%%%%%%%%%%%%%%%%%%%%%%%%%%%%%
\subsection{Files and Installation}

The package consists of the files:
%
\begin{center}
\begin{tabular}{ll}
    |README.txt|   & readme file \\
    |childdoc.ins| & installation file \\
    |childdoc.dtx| & source file \\
    |childdoc.def| & definition file \\
    |cdocsamp.tex| & sample main file \\
    |cdocsch1.tex| & sample include file \\
    |cdocsch2.tex| & sample include file \\
    |cdocspt3.tex| & sample part file \\
    |cdocspt4.tex| & sample part file \\
    |cdocsdrf.tex| & sample redirection file \\
    |cdocsfn1.tex| & sample redirection file \\
    |cdocsfn2.tex| & sample redirection file \\
    |childdoc.pdf| & manual
\end{tabular}
\end{center}
%
The distribution consists of the files
|README.txt|, |childdoc.ins| and |childdoc.dtx|.
%
\begin{itemize}
\item
Run (pdf)\LaTeX{} on |childdoc.dtx|
to compile the manual |childdoc.pdf| (this file).
\item
Run \LaTeX{} on |childdoc.ins| to create the definitions file |childdoc.def|
and the sample |cdocsamp.tex| with include files
|cdocsch1.tex|, |cdocsch2.tex|, |cdocspt3.tex|, |cdocspt4.tex|,
|cdocsdrf.tex|, |cdocsfn1.tex|, |cdocsfn2.tex|.
Then copy the file |childdoc.def| to an appropriate directory of your \LaTeX{}
distribution, e.g.\ \textit{texmf-root}|/tex/latex/childdoc|.
\end{itemize}

%%%%%%%%%%%%%%%%%%%%%%%%%%%%%%%%%%%%%%%%%%%%%%%%%%%%%%%%%%%%%%%%%%%%%%%%%%%%%%%%
\subsection{Related CTAN Packages}

There are several other packages which offer a similar functionality:
%
\begin{itemize}
\item
The packages
\href{http://ctan.org/pkg/docmute}{\textsf{docmute}},
\href{http://ctan.org/pkg/includex}{\textsf{includex}} and
\href{http://ctan.org/pkg/standalone}{\textsf{standalone}}
provide commands to include only the document body of
a child file thus allowing both files to be compiled individually.
\item
The packages \href{http://ctan.org/pkg/subdocs}{\textsf{subdocs}}
and \href{http://ctan.org/pkg/subfiles}{\textsf{subfiles}}
provide structures in which the main and child documents can be
encapsulated and allowing them to be compiled individually.
The inclusion mechanism is different from the conventional |\include|.
\item
The package \href{http://ctan.org/pkg/combine}{\textsf{combine}}
is an elaborate solution to combine several documents into one.
\end{itemize}
%
See also the CTAN topic \href{http://ctan.org/topic/subdocs}{\textsf{subdocs}}
for further related packages.
The present package differs from the above solutions in that
a document structure constructed with the conventional |\include| mechanism
just needs two extra commands at the top of every file
such that all constituent files can be compiled individually.

%%%%%%%%%%%%%%%%%%%%%%%%%%%%%%%%%%%%%%%%%%%%%%%%%%%%%%%%%%%%%%%%%%%%%%%%%%%%%%%%
%\subsection{Feature Suggestions}
%
%The following is a list of features which may be useful for future
%versions of this package:
%%
%\begin{itemize}
%\item
%\ldots
%\end{itemize}

%%%%%%%%%%%%%%%%%%%%%%%%%%%%%%%%%%%%%%%%%%%%%%%%%%%%%%%%%%%%%%%%%%%%%%%%%%%%%%%%
\subsection{Revision History}

%%%%%%%%%%%%%%%%%%%%%%%%%%%%%%%%%%%%%%%%
\paragraph{v2.0:} 2018/12/30

\begin{itemize}
\item
immediate forward processing
\item
added |\childdocby| mechanism
\item
manual restructured
\end{itemize}

%%%%%%%%%%%%%%%%%%%%%%%%%%%%%%%%%%%%%%%%
\paragraph{v1.6:} 2018/01/17

\begin{itemize}
\item
application for development of include files
\item
corrections to manual
\end{itemize}

%%%%%%%%%%%%%%%%%%%%%%%%%%%%%%%%%%%%%%%%
\paragraph{v1.5:} 2017/05/21

\begin{itemize}
\item
more complete structuring introduced
\item
|\childdocof| introduced
\item
|\childdoc| renamed to |\childdocmain|
\item
|\childredirect| renamed to |\childdocforward| and |\childdocforwardprefix|
and functionality expanded
\end{itemize}

%%%%%%%%%%%%%%%%%%%%%%%%%%%%%%%%%%%%%%%%
\paragraph{v1.0:} 2017/04/27

\begin{itemize}
\item
manual and install package
\item
first version published on CTAN
\end{itemize}

%%%%%%%%%%%%%%%%%%%%%%%%%%%%%%%%%%%%%%%%
\paragraph{v0.6:} 2017/04/26

\begin{itemize}
\item
redirection mechanism added
\end{itemize}

%%%%%%%%%%%%%%%%%%%%%%%%%%%%%%%%%%%%%%%%
\paragraph{v0.5:} 2017/04/26

\begin{itemize}
\item
functionality in definition file
\end{itemize}


%%%%%%%%%%%%%%%%%%%%%%%%%%%%%%%%%%%%%%%%%%%%%%%%%%%%%%%%%%%%%%%%%%%%%%%%%%%%%%%%
%%%%%%%%%%%%%%%%%%%%%%%%%%%%%%%%%%%%%%%%%%%%%%%%%%%%%%%%%%%%%%%%%%%%%%%%%%%%%%%%
%%%%%%%%%%%%%%%%%%%%%%%%%%%%%%%%%%%%%%%%%%%%%%%%%%%%%%%%%%%%%%%%%%%%%%%%%%%%%%%%
\appendix

\settowidth\MacroIndent{\rmfamily\scriptsize 000\ }

 \DocInput{childdoc.dtx}

\end{document}
%</driver>
% \fi
%
% %%%%%%%%%%%%%%%%%%%%%%%%%%%%%%%%%%%%%%%%%%%%%%%%%%%%%%%%%%%%%%%%%%%%%%%%%%%%%%
% %%%%%%%%%%%%%%%%%%%%%%%%%%%%%%%%%%%%%%%%%%%%%%%%%%%%%%%%%%%%%%%%%%%%%%%%%%%%%%
% \section{Sample}
%\iffalse
%<*samplemain>
%\fi
%
% The following presents a sample document
% with two chapters, two parts, a title page,
% a compile flag as well as three forwarding files to set the flag.
% It consists of eight |.tex| files:
% \begin{center}
% \begin{tabular}{ll}
% |cdocsamp.tex|&main file\\
% |cdocsch1.tex|&include file for chapter 1\\
% |cdocsch2.tex|&include file for chapter 2\\
% |cdocspt3.tex|&include file for part 3\\
% |cdocspt4.tex|&include file for part 4\\
% |cdocsdrf.tex|&forwarding file for main file in draft mode\\
% |cdocsfi1.tex|&forwarding file for final version of chapter 1\\
% |cdocsfi2.tex|&forwarding file for final version of chapter 2\\
% \end{tabular}
% \end{center}
% Each of the eight files can be compiled directly by the \LaTeX{} compiler.
%
% %%%%%%%%%%%%%%%%%%%%%%%%%%%%%%%%%%%%%%
% \paragraph{Main File.}
%
% The main file is called |cdocsamp.tex|.
%
% Load the \textsf{childdoc} definitions and
% declare the filename for the main document:
%    \begin{macrocode}
\input{childdoc.def}
\childdocmain{}
%    \end{macrocode}

% Optional override for |\version| flag:
%    \begin{macrocode}
%%\ifchilddoc\else\providecommand{\version}{draft}\fi
%    \end{macrocode}

% Define the default values for the |\version| flag
% (|final| for the main file and |draft| for childs):
%    \begin{macrocode}
\ifchilddoc
\providecommand{\version}{draft}
\else
\providecommand{\version}{final}
\fi
%    \end{macrocode}

% Load the standard document class:
%    \begin{macrocode}
\documentclass[12pt]{article}
%    \end{macrocode}

% Start the document body:
%    \begin{macrocode}
\begin{document}
%    \end{macrocode}

% Declare a title page.
% Print title, part of document being processed and version flag:
%    \begin{macrocode}
\addtocounter{page}{-1}
\begin{center}
{\LARGE\bfseries{}childdoc example\par}
\vspace{1cm}
\ifchilddoc
\ifchilddocmanual part\else chapter\fi:
`\childdocname' of `\childdocjob'\par
\else
main document: `\childdocjob'\par
\fi
version: \version\par
\end{center}
\newpage
%    \end{macrocode}

% Manually include selected file,
% otherwise process as usual:
%    \begin{macrocode}
\ifchilddocmanual
\section*{part `\childdocname'}
\input{\childdocname}
\else
%    \end{macrocode}

% Include the two chapters:
%    \begin{macrocode}
\include{cdocsch1}
\include{cdocsch2}
%    \end{macrocode}

% Include the two parts unless only chapters should be displayed:
%    \begin{macrocode}
\ifchilddoc\else
\section{part three}
\input{cdocspt3}
\section{part four}
\input{cdocspt4}
\fi
%    \end{macrocode}

% Process as usual until here:
%    \begin{macrocode}
\fi
%    \end{macrocode}

% End of document body:
%    \begin{macrocode}
\end{document}
%    \end{macrocode}
%\iffalse
%</samplemain>
%\fi
%
% %%%%%%%%%%%%%%%%%%%%%%%%%%%%%%%%%%%%%%
% \paragraph{Chapter Include Files.}
%
% The include files are called |cdocsch1.tex| and |cdocsch2.tex|.
%
%\iffalse
%<*samplechap1|samplechap2>
%\fi

% Optional override for |\version| flag:
%    \begin{macrocode}
%%\providecommand{\version}{final}
%    \end{macrocode}

% Include the main document:
%    \begin{macrocode}
\input{childdoc.def}
\childdocof{cdocsamp}
%    \end{macrocode}

%\iffalse
%</samplechap1|samplechap2>
%\fi
%
%\iffalse
%<*samplechap1>
%\fi
% Some text for chapter 1:
%    \begin{macrocode}
\section{one}
some text in chapter one
%    \end{macrocode}

%\iffalse
%</samplechap1>
%\fi
% Some text for chapter 2:
%\iffalse
%<*samplechap2>
%\fi
%    \begin{macrocode}
\section{two}
more text in chapter two
%    \end{macrocode}

%\iffalse
%</samplechap2>
%\fi
%
% %%%%%%%%%%%%%%%%%%%%%%%%%%%%%%%%%%%%%%
% \paragraph{Part Include Files.}
%
% The include files are called |cdocspt3.tex| and |cdocspt4.tex|.
%
%\iffalse
%<*samplepart3|samplepart4>
%\fi

% Optional override for |\version| flag:
%    \begin{macrocode}
%%\providecommand{\version}{final}
%    \end{macrocode}

% Include the main document:
%    \begin{macrocode}
\input{childdoc.def}
\childdocby{cdocsamp}
%    \end{macrocode}

%\iffalse
%</samplepart3|samplepart4>
%\fi
%
%\iffalse
%<*samplepart3>
%\fi
% Some text for part 3:
%    \begin{macrocode}
some text in part three
%    \end{macrocode}

%\iffalse
%</samplepart3>
%\fi
% Some text for part 4:
%\iffalse
%<*samplepart4>
%\fi
%    \begin{macrocode}
more text in part four
%    \end{macrocode}

%\iffalse
%</samplepart4>
%\fi
%
% %%%%%%%%%%%%%%%%%%%%%%%%%%%%%%%%%%%%%%
% \paragraph{Forwarding for a Complete Draft.}
%
% The following forwarding file |cdocsdrf.tex|
% compiles the main document in draft mode:
%\iffalse
%<*sampledraft>
%\fi
%    \begin{macrocode}
\def\version{draft}
\input{childdoc.def}
\childdocforward{cdocsamp}
%    \end{macrocode}

%\iffalse
%</sampledraft>
%\fi
%
% %%%%%%%%%%%%%%%%%%%%%%%%%%%%%%%%%%%%%%
% \paragraph{Forwarding for Final Version of the Chapters.}
%
% The following forwarding files |cdocsfn1.tex| and |cdocsfn2.tex|
% (with identical content)
% compile the final versions of the child documents
% |cdocsch1.tex| and |cdocsch2.tex|, respectively:
%\iffalse
%<*samplefinal>
%\fi
%    \begin{macrocode}
\def\version{final}
\input{childdoc.def}
\childdocforwardprefix[cdocsamp]{cdocsfn}{cdocsch}
%    \end{macrocode}

%\iffalse
%</samplefinal>
%\fi
%
% %%%%%%%%%%%%%%%%%%%%%%%%%%%%%%%%%%%%%%
% \paragraph{Command Line Processing.}
%
% The following three command lines generate the output files
% |cdocscld|, |cdocscl1| and |cdocscl2|
% which should be identical to
% |cdocsdrf|, |cdocsch1| and |cdocsfn2|, respectively:
% \begin{center}
% \begin{tabular}{l}
% |latex -jobname cdocscld \|\\
% |  "\def\version{draft}\input{childdoc.def}\childdocforward{cdocsamp}"|\\
% |latex -jobname cdocscl1 \|\\
% |  "\input{childdoc.def}\childdocforward[cdocsamp]{cdocsch1}"|\\
% |latex -jobname cdocscl2 \|\\
% |  "\def\version{final}\input{childdoc.def}\childdocforward{cdocsch2}"|
% \end{tabular}
% \end{center}
% Note that the trailing backslash on each first line
% merely continues the input to the second line
% (for convenient cut ant paste).
% Furthermore, the command |latex| can be replaced by any
% of its alternative versions such as |pdflatex|.
%
% %%%%%%%%%%%%%%%%%%%%%%%%%%%%%%%%%%%%%%%%%%%%%%%%%%%%%%%%%%%%%%%%%%%%%%%%%%%%%%
% %%%%%%%%%%%%%%%%%%%%%%%%%%%%%%%%%%%%%%%%%%%%%%%%%%%%%%%%%%%%%%%%%%%%%%%%%%%%%%
% \section{Implementation}
%\iffalse
%<*package>
%\fi
%
% This section describes the definitions file |childdoc.def|.

% The definitions cannot be loaded using |\usepackage| or |\RequirePackage|
% which has a mechanism to prevent loading a style file more than once.
% When loading the definitions by means of |\input|
% multiple instances have to be prevented manually:
%\iffalse
%This code needs to be before the `\ProvidesFile' directive
%which is defined at the beginning of this file.
%Therefore it is also placed there and commented out here.
%</package>
%<*discard>
%\fi
%    \begin{macrocode}
\ifdefined\childdocmain\endinput\fi
%    \end{macrocode}
%\iffalse
%</discard>
%<*package>
%\fi
%
% \macro{\ifchilddoc}
% \macro{\ifchilddocmanual}
% The conditional |\ifchilddoc| tells whether a
% child (true) or main (false) document is being compiled.
% The conditional |\ifchilddocmanual| tells whether
% the |\includeonly| mechanism is used (false) or
% the selection of child files must be performed manually (true).
% The definitions initialise to false:
%    \begin{macrocode}
\newif\ifchilddoc
\newif\ifchilddocmanual
%    \end{macrocode}

% \macro{\childdocname}
% \macro{\childdocjob}
% The macro |\childdocname| stores the name of the main document
% to be compiled. The macro |\childdocjob| stores the name of
% the document on which the \LaTeX{} compiler was originally invoked.
% The content of |\jobname| cannot be compared
% to filenames specified in the source due to different catcodes.
% The following code rescans |\jobname|, stores the result
% in |\childdocname| and saves a copy in |\childdocjob|:
%    \begin{macrocode}
\edef\childdocname{\scantokens\expandafter{\jobname\noexpand}}
\let\childdocjob\childdocname
%    \end{macrocode}

% \macro{\childdocdisable}
% The macro |\childdocdisable| prevents the main file
% from being processed more than once.
% At this stage, the main document command |\childdocmain|
% is assumed to be called once again where it should do nothing.
% Any subsequent call to it should prevent
% a secondary processing of the main document
% It overwrites the forwarding commands
% |\childdocof| and |\childdocforward|
% with empty macros to prevent further inclusions of the main document:
%    \begin{macrocode}
\newcommand{\childdocdisable}
{
  \renewcommand{\childdocmain}[1]{\renewcommand{\childdocmain}[1]{\endinput}}
  \renewcommand{\childdocof}[1]{}
  \renewcommand{\childdocby}[2][]{}
  \renewcommand{\childdocforward}[2][]{}
  \renewcommand{\childdocdisable}{}
}
%    \end{macrocode}

% \macro{\childdocmain}
% The macro |\childdocmain| is to be called at the top of the main file
% with nothing or the main filename (without extension) as argument.
% First, it breaks loops.
% If the argument is not empty and does not match |\childdocname|
% (which is set by the first inclusion of |childdoc.def|),
% |\ifchilddoc| is set to true, |\includeonly| is applied to the child file
% and |\jobname| is set to the main file
% (for proper handling of |.aux| files):
%    \begin{macrocode}
\newcommand{\childdocmain}[1]
{
  \childdocdisable\childdocmain{}
  \if?#1?\else
    \begingroup
      \def\childdoctmp{#1}
      \ifx\childdoctmp\childdocname
        \def\childdoctmp{}
      \else
        \def\childdoctmp
        {
          \childdoctrue
          \includeonly{\childdocname}
          \def\childdocjob{#1}
          \def\jobname{#1}
        }
      \fi
      \expandafter
    \endgroup
    \childdoctmp
  \fi
}
%    \end{macrocode}

% \macro{\childdocof}
% The command |\childdocof| redirects
% compilation to the main file |#1|.
%    \begin{macrocode}
\newcommand{\childdocof}[1]
{
  \childdocdisable
  \childdoctrue
  \includeonly{\childdocname}
  \def\jobname{#1}
  \def\childdocjob{#1}
  \input{#1}
}
%    \end{macrocode}

% \macro{\childdocby}
% The command |\childdocby| ....
%    \begin{macrocode}
\newcommand{\childdocby}[2][]
{
  \childdocdisable
  \childdoctrue
  \childdocmanualtrue
  \if?#1?\else
    \def\jobname{#2}
  \fi
  \def\childdocjob{#2}
  \input{#2}
  \endinput
}
%    \end{macrocode}

% \macro{\childdocforward}
% The command |\childdocforward| redirects
% compilation to the main file or
% (if the optional argument is given) a child file.
% Parameters are set as if the main file
% or a child file starting with |\childdocof| was compiled.
% Then compilation is handed over to the main file:
%    \begin{macrocode}
\newcommand{\childdocforward}[2][]
{
  \begingroup
    \if?#1?
      \def\childdoctmp
      {
        \def\childdocname{#2}
        \def\childdocjob{#2}
        \def\jobname{#2}
        \input{#2}
        \endinput
      }
    \else
      \def\childdoctmp
      {
        \childdocdisable
        \def\childdocname{#2}
        \childdoctrue
        \includeonly{#2}
        \def\childdocjob{#1}
        \def\jobname{#1}
        \input{#1}
        \endinput
      }
    \fi
    \expandafter
  \endgroup
  \childdoctmp
}
%    \end{macrocode}

% \macro{\childdocforwardprefix}
% The command |\childdocforwardprefix| redirects
% compilation to the main or a child file by means of a pattern.
% The prefix |#1| in the current filename is replaced by |#2|
% and the suffix of the current filename is kept
% (it is assumed that the filename does not contain the substring `|~~~|'
% which is used as a delimiter).
% Compilation is handed over to the new file by |\childdocforward|:
%    \begin{macrocode}
\newcommand{\childdocforwardprefix}[3][]
{
  \begingroup
    \def\childdocextract #2##1~~~{\def\childdoctmp{\childdocforward[#1]{#3##1}}}
    \expandafter\childdocextract\childdocname~~~
    \expandafter
  \endgroup
  \childdoctmp
}
%    \end{macrocode}

% \macro{\childdoc}
% The deprecated macro |\childdoc| is a legacy version of |\childdocmain|:
%    \begin{macrocode}
\newcommand{\childdoc}{\childdocmain}
%    \end{macrocode}

% \macro{\childdocredirect}
% The deprecated macro |\childdocredirect| is a legacy version
% of |\childdocforward| and |\childdocforwardprefix|:
%    \begin{macrocode}
\newcommand{\childdocredirect}[2][]
{
  \begingroup
    \if?#1?
      \def\childdoctmp{\childdocforward{#2}}
    \else
      \def\childdoctmp{\childdocforwardprefix{#1}{#2}}
    \fi
    \expandafter
  \endgroup
  \childdoctmp
}
%    \end{macrocode}

%\iffalse
%</package>
%\fi
%
\endinput
|\\
|\childdocforward{|\textit{main}|}|
\end{tabular}
\end{center}
%
Likewise, the following files |final|\textit{nn}|.tex|
compile the final version of the child document
|child|\textit{nn}|.tex|:
%
\begin{center}
\begin{tabular}{l}
|\def\version{final}|\\
|% \iffalse
%
% childdoc.dtx Copyright (C) 2017-2018 Niklas Beisert
%
% This work may be distributed and/or modified under the
% conditions of the LaTeX Project Public License, either version 1.3
% of this license or (at your option) any later version.
% The latest version of this license is in
%   http://www.latex-project.org/lppl.txt
% and version 1.3 or later is part of all distributions of LaTeX
% version 2005/12/01 or later.
%
% This work has the LPPL maintenance status `maintained'.
%
% The Current Maintainer of this work is Niklas Beisert.
%
% This work consists of the files childdoc.dtx and childdoc.ins
% and the derived files childdoc.def and cdocsamp.tex with
% cdocsch1.tex, cdocsch2.tex, cdocsdrf.tex, cdocsfn1.tex, cdocsfn2.tex.
%
%<package>\ifdefined\childdocmain\endinput\fi
%<package>\ProvidesFile{childdoc.def}[2018/12/30 v2.0 child document driver]
%<samplemain>\ProvidesFile{cdocsamp.tex}[2018/12/30 v2.0 sample for childdoc]
%<*driver>
%\ProvidesFile{childdoc.drv}[2018/12/30 v2.0 childdoc reference manual file]
\PassOptionsToClass{10pt,a4paper}{article}
\documentclass{ltxdoc}

\usepackage[margin=35mm]{geometry}
\usepackage{hyperref}
\usepackage{hyperxmp}
\usepackage[usenames]{color}

\hypersetup{colorlinks=true}
\hypersetup{pdfstartview=FitH}
\hypersetup{pdfpagemode=UseNone}
\hypersetup{pdfsource={}}
\hypersetup{pdflang={en-UK}}
\hypersetup{pdfcopyright={Copyright 2017-2018 Niklas Beisert.
  This work may be distributed and/or modified under the
  conditions of the LaTeX Project Public License, either version 1.3
  of this license or (at your option) any later version.}}
\hypersetup{pdflicenseurl={http://www.latex-project.org/lppl.txt}}
\hypersetup{pdfcontactaddress={ETH Zurich, ITP, HIT K,
  Wolfgang-Pauli-Strasse 27}}
\hypersetup{pdfcontactpostcode={8093}}
\hypersetup{pdfcontactcity={Zurich}}
\hypersetup{pdfcontactcountry={Switzerland}}
\hypersetup{pdfcontactemail={nbeisert@itp.phys.ethz.ch}}
\hypersetup{pdfcontacturl={http://people.phys.ethz.ch/\xmptilde nbeisert/}}

\newcommand{\secref}[1]{\hyperref[#1]{section \ref*{#1}}}

\parskip1ex
\parindent0pt
\let\olditemize\itemize
\def\itemize{\olditemize\parskip0pt}

\begin{document}

\title{The \textsf{childdoc} Package}
\hypersetup{pdftitle={The childdoc Package}}
\author{Niklas Beisert\\[2ex]
  Institut f\"ur Theoretische Physik\\
  Eidgen\"ossische Technische Hochschule Z\"urich\\
  Wolfgang-Pauli-Strasse 27, 8093 Z\"urich, Switzerland\\[1ex]
  \href{mailto:nbeisert@itp.phys.ethz.ch}
  {\texttt{nbeisert@itp.phys.ethz.ch}}}
\hypersetup{pdfauthor={Niklas Beisert}}
\hypersetup{pdfsubject={Manual for the LaTeX2e Package childdoc}}
\date{30 December 2018, \textsf{v2.0}}
\maketitle

\begin{abstract}\noindent
\textsf{childdoc} is a \LaTeXe{} package
that enables the direct compilation
of document sections included by |\include|
to individual files.
\end{abstract}

\begingroup
\parskip0ex
\tableofcontents
\endgroup

%%%%%%%%%%%%%%%%%%%%%%%%%%%%%%%%%%%%%%%%%%%%%%%%%%%%%%%%%%%%%%%%%%%%%%%%%%%%%%%%
%%%%%%%%%%%%%%%%%%%%%%%%%%%%%%%%%%%%%%%%%%%%%%%%%%%%%%%%%%%%%%%%%%%%%%%%%%%%%%%%
\section{Introduction}

\LaTeX{} provides a mechanism to structure a large document (such as a book)
into a main file and several child files (containing the chapters)
using the |\include| command.
This mechanism is beneficial for documents
which span hundreds of pages in order to
make the source file(s) more manageable.
Moreover, compilation can be restricted to
selected child files by means of the |\includeonly| command.
The latter feature can be used to reduce the compilation time while editing
(this was significantly more useful in the earlier days of \LaTeX{})
or to generate a smaller document which is easier to navigate.
Another application of |\includeonly| is to generate
documents consisting of selected parts of the complete document.

However, there are a few drawbacks of the plain |\include| mechanism:
\begin{itemize}
\item
The child files cannot be compiled on their own,
they can only be compiled via the main file.
A naive editing environment
(such as a text editor with an option
to have the current file processed by \LaTeX)
may require one to switch to the main file before compiling;
attempting to compile the child file produces errors.
\item
The main file must be modified (each time)
to adjust the |\includeonly| command
to the present needs. This easily leaves the main file in a messy state.
\item
The generated document will always carry the filename
of the main document. This is inconvenient if
several child files are to be compiled and
to be kept for distribution.
\end{itemize}

The present package provides a simple interface
to make child files individually compilable by \LaTeX{}.
Compiling a child file then has the same effect as compiling
the main file with an |\includeonly| command
to select the appropriate child.
Moreover the generated document will carry the name of the child
rather than the main file.
This resolves all three above issues.

This feature is meant to make the editing of books,
thesis documents and lecture notes somewhat more convenient.
However, the package can also be used efficiently for
composing a series of documents (such as exercise sheets)
which are typically distributed individually.
It then assists the author in generating the individual documents
(potentially in different versions)
as well as a document containing the collected series.
Another application is in developing style files
or other kinds of included material
where compilation of the style file could redirect
to a sample or test file.

%%%%%%%%%%%%%%%%%%%%%%%%%%%%%%%%%%%%%%%%%%%%%%%%%%%%%%%%%%%%%%%%%%%%%%%%%%%%%%%%
%%%%%%%%%%%%%%%%%%%%%%%%%%%%%%%%%%%%%%%%%%%%%%%%%%%%%%%%%%%%%%%%%%%%%%%%%%%%%%%%
\section{Usage}

First of all, the package \textsf{childdoc} is \emph{not} a standard
\LaTeXe{} |.sty| style file! Therefore it needs to be invoked in
a non-standard way.

%%%%%%%%%%%%%%%%%%%%%%%%%%%%%%%%%%%%%%%%%%%%%%%%%%%%%%%%%%%%%%%%%%%%%%%%%%%%%%%%
\subsection{Included Files}
\label{sec:include}

%%%%%%%%%%%%%%%%%%%%%%%%%%%%%%%%%%%%%%%%
\DescribeMacro{\childdocmain}
To use the package, add the commands
\begin{center}
\begin{tabular}{l}
|\input{childdoc.def}|\\
|\childdocmain{}|\\
\end{tabular}
\end{center}
at the very top of the main \LaTeX{} file,
in particular \emph{before} the |\documentclass| statement!
The argument of |\childdocmain| should be left empty
(but it must be present).

%%%%%%%%%%%%%%%%%%%%%%%%%%%%%%%%%%%%%%%%
\DescribeMacro{\childdocof}
Furthermore, add the commands
\begin{center}
\begin{tabular}{l}
|\input{childdoc.def}|\\
|\childdocof{|\textit{main}|}|\\
\end{tabular}
\end{center}
at the top of every child file \textit{child}
which is included by |\include{|\textit{child}|}|
from within the main file
(or at least for those files to be compiled individually).
The argument \textit{main} must be the filename of the main file.

There are a couple of
considerations in setting up the main and child documents:

%%%%%%%%%%%%%%%%%%%%%%%%%%%%%%%%%%%%%%%%
\paragraph{Restrictions.}

Please note the following restrictions:
\begin{itemize}
\item
|\childdocmain| must be called with one argument \textit{main}
to ensure compatibility with earlier version of the package.
It must either be empty (|\childdocmain{}|)
or precisely match the filename of the main file in which it is specified.
See \secref{sec:detection} for further information.
\item
The filename \textit{main} must be specified without the |.tex| extension.
\item
The filename \textit{main} is case sensitive
(even in case-insensitive file systems)
due to internal string comparison.
\item
The argument \textit{main} should be fully expanded, it cannot be a macro.
\item
Subdirectories and special characters should be avoided in filenames.
\item
The command |\childdocmain{|\textit{main}|}| must be followed by a whitespace.
It should not be followed immediately by another command
or by a comment mark `|%|'.
This is because the \TeX{} parser reads the token immediately following
the argument of |\childdocmain| and puts it
at the beginning of every child section;
however, a white\-space is ignored.
\end{itemize}

%%%%%%%%%%%%%%%%%%%%%%%%%%%%%%%%%%%%%%%%
\paragraph{Content of Main File.}

It is advisable to place all content in the child files included by |\include|.
Any output contained in the main file will appear in all child documents
unless suppressed manually;
it cannot be suppressed automatically by the |\includeonly| directive
and thus should normally be avoided.
A method to include some content in the main file
by means of conditional processing is described in \secref{sec:conditional}.

%%%%%%%%%%%%%%%%%%%%%%%%%%%%%%%%%%%%%%%%
\paragraph{Page Numbering.}

When only a part of the document is compiled,
the appropriate numbering of pages
(as well as other status parameters)
is determined from the |.aux| files.
The latter contain information from previous passes.
However this information needs to propagate through
all intermediate child documents.
Therefore the page numbering in child documents may well
be inconsistent until the complete document is compiled at least once.

A useful (if unconventional) way to always ensure a consistent
page numbering is to restart the numbering in each child document
and denote the pages by `\textit{child}|.|\textit{page}'
where \textit{child} represents the chapter/section number of the child file.
This can be achieved by the command
|\numberwithin{page}{|\textit{child}|}|
of the \textsf{amsmath} package
where \textit{child} can be |chapter| or |section|
depending on the chosen structuring.
Alternatively, one can modify the macro |\thepage| appropriately
and reset the counter |page| at the start of each child file.

%%%%%%%%%%%%%%%%%%%%%%%%%%%%%%%%%%%%%%%%%%%%%%%%%%%%%%%%%%%%%%%%%%%%%%%%%%%%%%%%
\subsection{Conditional Processing}
\label{sec:conditional}

The package provides a mechanism to compile different versions
of a document. To customise the versions further some conditional processing
can come in handy to distinguish which version is being compiled.
The package provides two macros to describe the compilation context:

%%%%%%%%%%%%%%%%%%%%%%%%%%%%%%%%%%%%%%%%
\DescribeMacro{\ifchilddoc}
The conditional |\ifchilddoc| distinguishes between the compilation of
child documents and the main document:
%
\begin{center}
|\ifchilddoc |\textit{child-code}| |[|\||else |\textit{main-code}]| \||fi|
\end{center}

%%%%%%%%%%%%%%%%%%%%%%%%%%%%%%%%%%%%%%%%
\DescribeMacro{\childdocname}
\DescribeMacro{\childdocjob}
The macro |\childdocname| contains the filename (without extension)
of the main or child file being processed.
Note that |\childdocjob| will always contain the name of the main file.

%%%%%%%%%%%%%%%%%%%%%%%%%%%%%%%%%%%%%%%%
\paragraph{Title Page.}

Conditional processing can be used to include a title or banner page
in the main document when proper precautions are taken.
Importantly, the code in the main file should ensure that the page counter
(as well as other status parameters which are stored in the |.aux| files)
takes the same value after the conditional processing.
Otherwise the page numbers may take divergent values
depending on which part is compiled.

For example, a title page could be declared by:
%
\begin{center}
\begin{tabular}{l}
|\ifchilddoc\||else|\\
|\addtocounter{page}{-1}|\\
\textit{code for title page}\\
|\newpage|\\
|\||fi|
\end{tabular}
\end{center}
%
A banner page for the child documents can be generated by:
%
\begin{center}
\begin{tabular}{l}
|\ifchilddoc|\\
|\addtocounter{page}{-1}|\\
\textit{code for banner page}\\
|\newpage|\\
|\||fi|
\end{tabular}
\end{center}
%
Here one could write a message such as:
\begin{center}
|This is the part \childdocname{} of \childdocjob{}.|
\end{center}

%%%%%%%%%%%%%%%%%%%%%%%%%%%%%%%%%%%%%%%%%%%%%%%%%%%%%%%%%%%%%%%%%%%%%%%%%%%%%%%%
\subsection{Flags}
\label{sec:flags}

The package makes it easy to generate different versions
of the main or child documents.
To this end compilation flags can be defined
and assigned different default values.
They will be particularly useful in conjunction
with the forwarding mechanism described in \secref{sec:forward}.

For example, it may be useful to have a flag |\version|
which can be set to |draft| or |final|.
The document source will contain some conditional code
depending on the value of |\version|.
Suppose further, the flag should default to |final| for the main file
and to |draft| for child files
which is a natural assignment for editing the document.
This is achieved by placing the following code
in the preamble of the main document
(below the |\childdocmain| directive):
%
\begin{center}
\begin{tabular}{l}
|\ifchilddoc|\\
|\providecommand{\version}{draft}|\\
|\||else|\\
|\providecommand{\version}{final}|\\
|\||fi|
\end{tabular}
\end{center}
%
The definition by |\providecommand| makes sure
that previous definitions are not overwritten.
Further statements |\providecommand{\version}{...}|
can thus be added before the above code to override it.

For the main file, one might add a line
(between |\childdocmain| and the above block)
%
\begin{center}
|%\ifchilddoc\||else\providecommand{\version}{draft}\||fi|
\end{center}
%
which can be uncommented to produce a draft version.
Likewise one can add a line to the very top of a child file
(above the |\childdocof{|\textit{main}|}| directive)
%
\begin{center}
|%\providecommand{\version}{final}|
\end{center}
%
which can be uncommented to produce the final version of this child document.

%%%%%%%%%%%%%%%%%%%%%%%%%%%%%%%%%%%%%%%%%%%%%%%%%%%%%%%%%%%%%%%%%%%%%%%%%%%%%%%%
\subsection{Forwarding}
\label{sec:forward}

Different versions of the main or child documents
using compilation flags as described in \secref{sec:flags}
can be (permanently) stored in different files
for convenient compilation, viewing and distribution.
To this end, the package defines a command
to pass on compilation to a different file:

%%%%%%%%%%%%%%%%%%%%%%%%%%%%%%%%%%%%%%%%
\DescribeMacro{\childdocforward}
The command |\childdocforward| redirects processing to
another source file:
%
\begin{center}
\begin{tabular}{l}
|\input{childdoc.def}|\\
|\childdocforward[|\textit{main}|]{|\textit{dest}|}|\\
\end{tabular}
\end{center}
%
The argument \textit{dest} is the destination file
(without extension).
It should be the main file or one of the child files.
Note that further \textsf{childdoc} directives
such as |\childdocof| and |\childdocforward|
in the indicated file will be processed in this form.
The optional argument \textit{main}
passes on directly to the main file \textit{main}
while pretending to compile the child \textit{dest}.
This form behaves as if \textit{dest}
issues |\childdocof{|\textit{main}|}| right away,
and no further \textsf{childdoc} directives will be processed.

%%%%%%%%%%%%%%%%%%%%%%%%%%%%%%%%%%%%%%%%
\DescribeMacro{\...prefix}
In the alternative form |\childdocforwardprefix|,
%
\begin{center}
\begin{tabular}{l}
|\input{childdoc.def}|\\
|\childdocforwardprefix[|\textit{main}|]{|\textit{prefix}|}{|\textit{dest}|}|
\end{tabular}
\end{center}
%
the destination file is determined by a pattern
depending on the current file:
To make this work, the current file must be called
`{\textit{prefix}\hspace{0.2em}\textit{suffix}}'
with \textit{prefix} matching precisely the argument.
Processing is then passed on to the file
`{\textit{dest}\hspace{0.2em}\textit{suffix}}'.
Surely, the same effect is achieved by
directly specifying the
argument `{\textit{dest}\hspace{0.2em}\textit{suffix}}'
in the first form.
However, that requires to set up a different file
for each child. With the alternative form of the command
all these files can have exactly the same content
which simplifies setting them up and maintaining them.

For example, the following file |draft.tex|
with a compilation flag |\version| as described in \secref{sec:flags}
compiles the main document as a draft:
%
\begin{center}
\begin{tabular}{l}
|\def\version{draft}|\\
|\input{childdoc.def}|\\
|\childdocforward{|\textit{main}|}|
\end{tabular}
\end{center}
%
Likewise, the following files |final|\textit{nn}|.tex|
compile the final version of the child document
|child|\textit{nn}|.tex|:
%
\begin{center}
\begin{tabular}{l}
|\def\version{final}|\\
|\input{childdoc.def}|\\
|\childdocforwardprefix{final}{child}|
\end{tabular}
\end{center}
%

Note that when several versions of a main file and/or of each child file
are to be generated, it may be convenient to set up a |Makefile| or
shell script to automatise the process.

%%%%%%%%%%%%%%%%%%%%%%%%%%%%%%%%%%%%%%%%%%%%%%%%%%%%%%%%%%%%%%%%%%%%%%%%%%%%%%%%
\subsection{Command Line Processing}
\label{sec:commandline}

The effect of redirection files can also be achieved by invoking
the \LaTeX{} compiler with a more elaborate command line.
Most conveniently this should be done as part
of a shell script or a |Makefile|.

When using \textsf{childdoc} in the main file, the following
command lines effectively perform a redirection
(note that depending on the shell being used,
backslashes may have to be doubled: `|\|' $\to$ `|\\|'):
%
\begin{center}
|... -jobname "|\textit{target}|" |\\|"|[\textit{flags}]%
|\input{childdoc.def}\childdocforward[|\textit{main}|]{|\textit{dest}|}"|
\end{center}
%
Here \textit{target} is the name of the output file,
\textit{main} is the name of the main file
and \textit{dest} is the name of the main or child file to be processed
(all filenames without extensions).
The optional argument \textit{main} can be omitted
if \textit{main} matches \textit{dest}.
Optionally, compilation \textit{flags} can be defined via |\def| commands.
This command line makes the \TeX{} engine believe
it is compiling the file \textit{target}
whose content is specified as the latter parameter.
The provided code then forwards the processing to
\textit{main} or \textit{dest} as described in \secref{sec:forward}.

%%%%%%%%%%%%%%%%%%%%%%%%%%%%%%%%%%%%%%%%%%%%%%%%%%%%%%%%%%%%%%%%%%%%%%%%%%%%%%%%
\subsection{Include by Input}
\label{sec:input}

Including child documents by |\include| has some restrictions by design.
Most notably, the content of a child document always occupies
its own set of pages; pages cannot be shared between child documents.
Usually, this behaviour makes perfect sense
because each child document contain an essential part of the document.
However, in some situations it may be desirable to compose
a document from a collection of parts
without having mandatory page breaks between then.
For this case, the package
provides a mechanism to include parts
by |\input| which can also be processed individually.
However, by construction this mechanism
requires manual handling of the content to be output.

%%%%%%%%%%%%%%%%%%%%%%%%%%%%%%%%%%%%%%%%
\DescribeMacro{\ifchilddocmanual}
The main file should be prepared as usual, see \secref{sec:include}.
However, the document body must make a distinction
between processing of an individual part and of the main document, e.g.:
%
\begin{center}
\begin{tabular}{l}
|\ifchilddocmanual|\\
|\input{\childdocname}|\\
|\||else|\\
\textit{document body with }|\input{|\textit{part}|}|\\
|\||fi|
\end{tabular}
\end{center}
%
The conditional |\ifchilddocmanual| is true whenever
a part to be included by |\input| is being compiled,
and the name of the part is stored in |\childdocname|.

%%%%%%%%%%%%%%%%%%%%%%%%%%%%%%%%%%%%%%%%
\DescribeMacro{\childdocby}
Each part to be included by |\input| should start with:
%
\begin{center}
\begin{tabular}{l}
|\input{childdoc.def}|\\
|\childdocby{|\textit{main}|}|\\
\end{tabular}
\end{center}
%
The directive |\childdocby| is similar to |\childdocof|
described in \secref{sec:include},
but the subsequent selection of content must be done manually.
To that end, both |\ifchilddoc| and |\ifchilddocmanual|
will be true upon processing of a part,
and the name of the part is stored in |\childdocname|.
Note that |\jobname| will be set to the filename of the current part
so that each part receives an individual |.aux| file
that does not interfere with the |.aux| file(s) of the main document.
This behaviour can be altered by the alternative form
|\childdocby[*]{|\textit{main}|}| (with a non-empty optional argument)
which uses the |.aux| file of the main document
by setting |\jobname| to \textit{main}.

%%%%%%%%%%%%%%%%%%%%%%%%%%%%%%%%%%%%%%%%%%%%%%%%%%%%%%%%%%%%%%%%%%%%%%%%%%%%%%%%
\subsection{Driver Development}
\label{sec:driver}

The \textsf{childdoc} mechanism can also be use for the development
of definition files such as \LaTeX{} styles or classes.
This case differs from the above setup with multiple parts
included by |\include| in that no |\includeonly| should be invoked.
This can be achieved by starting the include file
(before |\ProvidesPackage|) with:
%
\begin{center}
\begin{tabular}{l}
|\input{childdoc.def}|\\
|\childdocforward{|\textit{main}|}|\\
\end{tabular}
\end{center}
%
or alternatively with:
%
\begin{center}
\begin{tabular}{l}
|\input{childdoc.def}|\\
|\childdocby{|\textit{main}|}|\\
\end{tabular}
\end{center}
%
Both forms have slightly different effects as described above.
The main file is prepared as usual, see \secref{sec:include}.

%%%%%%%%%%%%%%%%%%%%%%%%%%%%%%%%%%%%%%%%%%%%%%%%%%%%%%%%%%%%%%%%%%%%%%%%%%%%%%%%
\subsection{Legacy Detection}
\label{sec:detection}

The directive |\childdocmain| in the main file can detect
whether the complete document or merely a child is to be compiled
even without using the directive |\childdocof|.
This method is deprecated because it is less robust
and there is no compelling reason to use it;
it is merely provided for backward compatibility
and it may be removed in future versions.

If the detection mechanism is to be used,
it is mandatory to correctly specify
the filename of the main file as the argument of |\childdocmain|:
%
\begin{center}
\begin{tabular}{l}
|\input{childdoc.def}|\\
|\childdocmain{|\textit{main}|}|\\
\end{tabular}
\end{center}
%
If |\jobname| does not match the argument \textit{main} of |\childdocmain|,
it is assumed that |\jobname| points to the child file to be compiled.
When using |\childdocmain| with the main file specified as argument,
it suffices to start a child file
with just |\input{|\textit{main}|}|
without loading of the package and using |\childdocof|.
If instead all processing is done
with the appropriate \textsf{childdoc} directives,
the argument of \textit{main} of |\childdocmain| can be empty.

An alternative version of the command line processing described
in \secref{sec:commandline} using the detection mechanism reads:
%
\begin{center}
|... -jobname "|\textit{target}|" "|[\textit{flags}]%
[|\def\jobname{|\textit{dest}|}|]|\input{|\textit{main}|}"|
\end{center}

%%%%%%%%%%%%%%%%%%%%%%%%%%%%%%%%%%%%%%%%%%%%%%%%%%%%%%%%%%%%%%%%%%%%%%%%%%%%%%%%
\subsection{Manual Code}
\label{sec:manual}

In case one cannot be certain whether the definitions file |childdoc.def|
is installed on the target \TeX{} distribution
and one prefers not to ship it,
it is conceivable to paste a few relevant commands into the sources.

To that end, drop all statements |\input{childdoc.def}|
and perform the replacements as outlined below.
Instead of |\childdocmain{|\textit{main}|}| add the following code
to the top of the main file:
%
\begin{center}
\begin{tabular}{l}
|\||ifdefined\childdocname\endinput\||fi\newif\ifchilddoc|\\
|\edef\childdocname{\scantokens\expandafter{\jobname\noexpand}}|\\
|\def\childdocmain{|\textit{main}|}\||ifx\childdocmain\childdocname\||else|\\
|\childdoctrue\includeonly{\childdocname}\let\jobname\childdocmain\||fi|\\
\end{tabular}
\end{center}
%
Instead of |\childdocof{|\textit{main}|}| just include the main file
at the top of each child file:
%
\begin{center}
|\input{|\textit{main}|}|
\end{center}
%
A simple redirection |\childdocforward{|\textit{dest}|}| is achieved by:
%
\begin{center}
|\def\jobname{|\textit{dest}|}\input{\jobname}|
\end{center}
%
The redirection with prefix
|\childdocforwardprefix[|\textit{prefix}|]{|\textit{dest}|}|
is accomplished by:
%
\begin{center}
\begin{tabular}{l}
|{\edef\jobname{\scantokens\expandafter{\jobname\noexpand}}|\\
|\def\redirectjob |\textit{prefix}|#1~~~{\gdef\jobname{|\textit{dest}|#1}}|\\
|\expandafter\redirectjob\jobname~~~}\input{\jobname}|
\end{tabular}
\end{center}

In an alternative approach,
child documents can be compiled by a specific command line
without additional code or specific definitions:
%
\begin{center}
|... -jobname "|\textit{target}|" "|[\textit{flags}]%
|\includeonly{|\textit{dest}|}\input{|\textit{main}|}"|
\end{center}
%

%%%%%%%%%%%%%%%%%%%%%%%%%%%%%%%%%%%%%%%%%%%%%%%%%%%%%%%%%%%%%%%%%%%%%%%%%%%%%%%%
%%%%%%%%%%%%%%%%%%%%%%%%%%%%%%%%%%%%%%%%%%%%%%%%%%%%%%%%%%%%%%%%%%%%%%%%%%%%%%%%
\section{Information}

%%%%%%%%%%%%%%%%%%%%%%%%%%%%%%%%%%%%%%%%%%%%%%%%%%%%%%%%%%%%%%%%%%%%%%%%%%%%%%%%
\subsection{Copyright}

Copyright \copyright{} 2017--2018 Niklas Beisert

This work may be distributed and/or modified under the
conditions of the \LaTeX{} Project Public License, either version 1.3
of this license or (at your option) any later version.
The latest version of this license is in
  \url{http://www.latex-project.org/lppl.txt}
and version 1.3 or later is part of all distributions of \LaTeX{}
version 2005/12/01 or later.

This work has the LPPL maintenance status `maintained'.

The Current Maintainer of this work is Niklas Beisert.

This work consists of the files |README.txt|, |childdoc.ins| and |childdoc.dtx|
as well as the derived files |childdoc.def|, |cdocsamp.tex|
with |cdocsch1.tex|, |cdocsch2.tex|, |cdocspt3.tex|, |cdocspt4.tex|,
|cdocsdrf.tex|, |cdocsfn1.tex|, |cdocsfn2.tex|
as well as |childdoc.pdf|.

%%%%%%%%%%%%%%%%%%%%%%%%%%%%%%%%%%%%%%%%%%%%%%%%%%%%%%%%%%%%%%%%%%%%%%%%%%%%%%%%
\subsection{Files and Installation}

The package consists of the files:
%
\begin{center}
\begin{tabular}{ll}
    |README.txt|   & readme file \\
    |childdoc.ins| & installation file \\
    |childdoc.dtx| & source file \\
    |childdoc.def| & definition file \\
    |cdocsamp.tex| & sample main file \\
    |cdocsch1.tex| & sample include file \\
    |cdocsch2.tex| & sample include file \\
    |cdocspt3.tex| & sample part file \\
    |cdocspt4.tex| & sample part file \\
    |cdocsdrf.tex| & sample redirection file \\
    |cdocsfn1.tex| & sample redirection file \\
    |cdocsfn2.tex| & sample redirection file \\
    |childdoc.pdf| & manual
\end{tabular}
\end{center}
%
The distribution consists of the files
|README.txt|, |childdoc.ins| and |childdoc.dtx|.
%
\begin{itemize}
\item
Run (pdf)\LaTeX{} on |childdoc.dtx|
to compile the manual |childdoc.pdf| (this file).
\item
Run \LaTeX{} on |childdoc.ins| to create the definitions file |childdoc.def|
and the sample |cdocsamp.tex| with include files
|cdocsch1.tex|, |cdocsch2.tex|, |cdocspt3.tex|, |cdocspt4.tex|,
|cdocsdrf.tex|, |cdocsfn1.tex|, |cdocsfn2.tex|.
Then copy the file |childdoc.def| to an appropriate directory of your \LaTeX{}
distribution, e.g.\ \textit{texmf-root}|/tex/latex/childdoc|.
\end{itemize}

%%%%%%%%%%%%%%%%%%%%%%%%%%%%%%%%%%%%%%%%%%%%%%%%%%%%%%%%%%%%%%%%%%%%%%%%%%%%%%%%
\subsection{Related CTAN Packages}

There are several other packages which offer a similar functionality:
%
\begin{itemize}
\item
The packages
\href{http://ctan.org/pkg/docmute}{\textsf{docmute}},
\href{http://ctan.org/pkg/includex}{\textsf{includex}} and
\href{http://ctan.org/pkg/standalone}{\textsf{standalone}}
provide commands to include only the document body of
a child file thus allowing both files to be compiled individually.
\item
The packages \href{http://ctan.org/pkg/subdocs}{\textsf{subdocs}}
and \href{http://ctan.org/pkg/subfiles}{\textsf{subfiles}}
provide structures in which the main and child documents can be
encapsulated and allowing them to be compiled individually.
The inclusion mechanism is different from the conventional |\include|.
\item
The package \href{http://ctan.org/pkg/combine}{\textsf{combine}}
is an elaborate solution to combine several documents into one.
\end{itemize}
%
See also the CTAN topic \href{http://ctan.org/topic/subdocs}{\textsf{subdocs}}
for further related packages.
The present package differs from the above solutions in that
a document structure constructed with the conventional |\include| mechanism
just needs two extra commands at the top of every file
such that all constituent files can be compiled individually.

%%%%%%%%%%%%%%%%%%%%%%%%%%%%%%%%%%%%%%%%%%%%%%%%%%%%%%%%%%%%%%%%%%%%%%%%%%%%%%%%
%\subsection{Feature Suggestions}
%
%The following is a list of features which may be useful for future
%versions of this package:
%%
%\begin{itemize}
%\item
%\ldots
%\end{itemize}

%%%%%%%%%%%%%%%%%%%%%%%%%%%%%%%%%%%%%%%%%%%%%%%%%%%%%%%%%%%%%%%%%%%%%%%%%%%%%%%%
\subsection{Revision History}

%%%%%%%%%%%%%%%%%%%%%%%%%%%%%%%%%%%%%%%%
\paragraph{v2.0:} 2018/12/30

\begin{itemize}
\item
immediate forward processing
\item
added |\childdocby| mechanism
\item
manual restructured
\end{itemize}

%%%%%%%%%%%%%%%%%%%%%%%%%%%%%%%%%%%%%%%%
\paragraph{v1.6:} 2018/01/17

\begin{itemize}
\item
application for development of include files
\item
corrections to manual
\end{itemize}

%%%%%%%%%%%%%%%%%%%%%%%%%%%%%%%%%%%%%%%%
\paragraph{v1.5:} 2017/05/21

\begin{itemize}
\item
more complete structuring introduced
\item
|\childdocof| introduced
\item
|\childdoc| renamed to |\childdocmain|
\item
|\childredirect| renamed to |\childdocforward| and |\childdocforwardprefix|
and functionality expanded
\end{itemize}

%%%%%%%%%%%%%%%%%%%%%%%%%%%%%%%%%%%%%%%%
\paragraph{v1.0:} 2017/04/27

\begin{itemize}
\item
manual and install package
\item
first version published on CTAN
\end{itemize}

%%%%%%%%%%%%%%%%%%%%%%%%%%%%%%%%%%%%%%%%
\paragraph{v0.6:} 2017/04/26

\begin{itemize}
\item
redirection mechanism added
\end{itemize}

%%%%%%%%%%%%%%%%%%%%%%%%%%%%%%%%%%%%%%%%
\paragraph{v0.5:} 2017/04/26

\begin{itemize}
\item
functionality in definition file
\end{itemize}


%%%%%%%%%%%%%%%%%%%%%%%%%%%%%%%%%%%%%%%%%%%%%%%%%%%%%%%%%%%%%%%%%%%%%%%%%%%%%%%%
%%%%%%%%%%%%%%%%%%%%%%%%%%%%%%%%%%%%%%%%%%%%%%%%%%%%%%%%%%%%%%%%%%%%%%%%%%%%%%%%
%%%%%%%%%%%%%%%%%%%%%%%%%%%%%%%%%%%%%%%%%%%%%%%%%%%%%%%%%%%%%%%%%%%%%%%%%%%%%%%%
\appendix

\settowidth\MacroIndent{\rmfamily\scriptsize 000\ }

 \DocInput{childdoc.dtx}

\end{document}
%</driver>
% \fi
%
% %%%%%%%%%%%%%%%%%%%%%%%%%%%%%%%%%%%%%%%%%%%%%%%%%%%%%%%%%%%%%%%%%%%%%%%%%%%%%%
% %%%%%%%%%%%%%%%%%%%%%%%%%%%%%%%%%%%%%%%%%%%%%%%%%%%%%%%%%%%%%%%%%%%%%%%%%%%%%%
% \section{Sample}
%\iffalse
%<*samplemain>
%\fi
%
% The following presents a sample document
% with two chapters, two parts, a title page,
% a compile flag as well as three forwarding files to set the flag.
% It consists of eight |.tex| files:
% \begin{center}
% \begin{tabular}{ll}
% |cdocsamp.tex|&main file\\
% |cdocsch1.tex|&include file for chapter 1\\
% |cdocsch2.tex|&include file for chapter 2\\
% |cdocspt3.tex|&include file for part 3\\
% |cdocspt4.tex|&include file for part 4\\
% |cdocsdrf.tex|&forwarding file for main file in draft mode\\
% |cdocsfi1.tex|&forwarding file for final version of chapter 1\\
% |cdocsfi2.tex|&forwarding file for final version of chapter 2\\
% \end{tabular}
% \end{center}
% Each of the eight files can be compiled directly by the \LaTeX{} compiler.
%
% %%%%%%%%%%%%%%%%%%%%%%%%%%%%%%%%%%%%%%
% \paragraph{Main File.}
%
% The main file is called |cdocsamp.tex|.
%
% Load the \textsf{childdoc} definitions and
% declare the filename for the main document:
%    \begin{macrocode}
\input{childdoc.def}
\childdocmain{}
%    \end{macrocode}

% Optional override for |\version| flag:
%    \begin{macrocode}
%%\ifchilddoc\else\providecommand{\version}{draft}\fi
%    \end{macrocode}

% Define the default values for the |\version| flag
% (|final| for the main file and |draft| for childs):
%    \begin{macrocode}
\ifchilddoc
\providecommand{\version}{draft}
\else
\providecommand{\version}{final}
\fi
%    \end{macrocode}

% Load the standard document class:
%    \begin{macrocode}
\documentclass[12pt]{article}
%    \end{macrocode}

% Start the document body:
%    \begin{macrocode}
\begin{document}
%    \end{macrocode}

% Declare a title page.
% Print title, part of document being processed and version flag:
%    \begin{macrocode}
\addtocounter{page}{-1}
\begin{center}
{\LARGE\bfseries{}childdoc example\par}
\vspace{1cm}
\ifchilddoc
\ifchilddocmanual part\else chapter\fi:
`\childdocname' of `\childdocjob'\par
\else
main document: `\childdocjob'\par
\fi
version: \version\par
\end{center}
\newpage
%    \end{macrocode}

% Manually include selected file,
% otherwise process as usual:
%    \begin{macrocode}
\ifchilddocmanual
\section*{part `\childdocname'}
\input{\childdocname}
\else
%    \end{macrocode}

% Include the two chapters:
%    \begin{macrocode}
\include{cdocsch1}
\include{cdocsch2}
%    \end{macrocode}

% Include the two parts unless only chapters should be displayed:
%    \begin{macrocode}
\ifchilddoc\else
\section{part three}
\input{cdocspt3}
\section{part four}
\input{cdocspt4}
\fi
%    \end{macrocode}

% Process as usual until here:
%    \begin{macrocode}
\fi
%    \end{macrocode}

% End of document body:
%    \begin{macrocode}
\end{document}
%    \end{macrocode}
%\iffalse
%</samplemain>
%\fi
%
% %%%%%%%%%%%%%%%%%%%%%%%%%%%%%%%%%%%%%%
% \paragraph{Chapter Include Files.}
%
% The include files are called |cdocsch1.tex| and |cdocsch2.tex|.
%
%\iffalse
%<*samplechap1|samplechap2>
%\fi

% Optional override for |\version| flag:
%    \begin{macrocode}
%%\providecommand{\version}{final}
%    \end{macrocode}

% Include the main document:
%    \begin{macrocode}
\input{childdoc.def}
\childdocof{cdocsamp}
%    \end{macrocode}

%\iffalse
%</samplechap1|samplechap2>
%\fi
%
%\iffalse
%<*samplechap1>
%\fi
% Some text for chapter 1:
%    \begin{macrocode}
\section{one}
some text in chapter one
%    \end{macrocode}

%\iffalse
%</samplechap1>
%\fi
% Some text for chapter 2:
%\iffalse
%<*samplechap2>
%\fi
%    \begin{macrocode}
\section{two}
more text in chapter two
%    \end{macrocode}

%\iffalse
%</samplechap2>
%\fi
%
% %%%%%%%%%%%%%%%%%%%%%%%%%%%%%%%%%%%%%%
% \paragraph{Part Include Files.}
%
% The include files are called |cdocspt3.tex| and |cdocspt4.tex|.
%
%\iffalse
%<*samplepart3|samplepart4>
%\fi

% Optional override for |\version| flag:
%    \begin{macrocode}
%%\providecommand{\version}{final}
%    \end{macrocode}

% Include the main document:
%    \begin{macrocode}
\input{childdoc.def}
\childdocby{cdocsamp}
%    \end{macrocode}

%\iffalse
%</samplepart3|samplepart4>
%\fi
%
%\iffalse
%<*samplepart3>
%\fi
% Some text for part 3:
%    \begin{macrocode}
some text in part three
%    \end{macrocode}

%\iffalse
%</samplepart3>
%\fi
% Some text for part 4:
%\iffalse
%<*samplepart4>
%\fi
%    \begin{macrocode}
more text in part four
%    \end{macrocode}

%\iffalse
%</samplepart4>
%\fi
%
% %%%%%%%%%%%%%%%%%%%%%%%%%%%%%%%%%%%%%%
% \paragraph{Forwarding for a Complete Draft.}
%
% The following forwarding file |cdocsdrf.tex|
% compiles the main document in draft mode:
%\iffalse
%<*sampledraft>
%\fi
%    \begin{macrocode}
\def\version{draft}
\input{childdoc.def}
\childdocforward{cdocsamp}
%    \end{macrocode}

%\iffalse
%</sampledraft>
%\fi
%
% %%%%%%%%%%%%%%%%%%%%%%%%%%%%%%%%%%%%%%
% \paragraph{Forwarding for Final Version of the Chapters.}
%
% The following forwarding files |cdocsfn1.tex| and |cdocsfn2.tex|
% (with identical content)
% compile the final versions of the child documents
% |cdocsch1.tex| and |cdocsch2.tex|, respectively:
%\iffalse
%<*samplefinal>
%\fi
%    \begin{macrocode}
\def\version{final}
\input{childdoc.def}
\childdocforwardprefix[cdocsamp]{cdocsfn}{cdocsch}
%    \end{macrocode}

%\iffalse
%</samplefinal>
%\fi
%
% %%%%%%%%%%%%%%%%%%%%%%%%%%%%%%%%%%%%%%
% \paragraph{Command Line Processing.}
%
% The following three command lines generate the output files
% |cdocscld|, |cdocscl1| and |cdocscl2|
% which should be identical to
% |cdocsdrf|, |cdocsch1| and |cdocsfn2|, respectively:
% \begin{center}
% \begin{tabular}{l}
% |latex -jobname cdocscld \|\\
% |  "\def\version{draft}\input{childdoc.def}\childdocforward{cdocsamp}"|\\
% |latex -jobname cdocscl1 \|\\
% |  "\input{childdoc.def}\childdocforward[cdocsamp]{cdocsch1}"|\\
% |latex -jobname cdocscl2 \|\\
% |  "\def\version{final}\input{childdoc.def}\childdocforward{cdocsch2}"|
% \end{tabular}
% \end{center}
% Note that the trailing backslash on each first line
% merely continues the input to the second line
% (for convenient cut ant paste).
% Furthermore, the command |latex| can be replaced by any
% of its alternative versions such as |pdflatex|.
%
% %%%%%%%%%%%%%%%%%%%%%%%%%%%%%%%%%%%%%%%%%%%%%%%%%%%%%%%%%%%%%%%%%%%%%%%%%%%%%%
% %%%%%%%%%%%%%%%%%%%%%%%%%%%%%%%%%%%%%%%%%%%%%%%%%%%%%%%%%%%%%%%%%%%%%%%%%%%%%%
% \section{Implementation}
%\iffalse
%<*package>
%\fi
%
% This section describes the definitions file |childdoc.def|.

% The definitions cannot be loaded using |\usepackage| or |\RequirePackage|
% which has a mechanism to prevent loading a style file more than once.
% When loading the definitions by means of |\input|
% multiple instances have to be prevented manually:
%\iffalse
%This code needs to be before the `\ProvidesFile' directive
%which is defined at the beginning of this file.
%Therefore it is also placed there and commented out here.
%</package>
%<*discard>
%\fi
%    \begin{macrocode}
\ifdefined\childdocmain\endinput\fi
%    \end{macrocode}
%\iffalse
%</discard>
%<*package>
%\fi
%
% \macro{\ifchilddoc}
% \macro{\ifchilddocmanual}
% The conditional |\ifchilddoc| tells whether a
% child (true) or main (false) document is being compiled.
% The conditional |\ifchilddocmanual| tells whether
% the |\includeonly| mechanism is used (false) or
% the selection of child files must be performed manually (true).
% The definitions initialise to false:
%    \begin{macrocode}
\newif\ifchilddoc
\newif\ifchilddocmanual
%    \end{macrocode}

% \macro{\childdocname}
% \macro{\childdocjob}
% The macro |\childdocname| stores the name of the main document
% to be compiled. The macro |\childdocjob| stores the name of
% the document on which the \LaTeX{} compiler was originally invoked.
% The content of |\jobname| cannot be compared
% to filenames specified in the source due to different catcodes.
% The following code rescans |\jobname|, stores the result
% in |\childdocname| and saves a copy in |\childdocjob|:
%    \begin{macrocode}
\edef\childdocname{\scantokens\expandafter{\jobname\noexpand}}
\let\childdocjob\childdocname
%    \end{macrocode}

% \macro{\childdocdisable}
% The macro |\childdocdisable| prevents the main file
% from being processed more than once.
% At this stage, the main document command |\childdocmain|
% is assumed to be called once again where it should do nothing.
% Any subsequent call to it should prevent
% a secondary processing of the main document
% It overwrites the forwarding commands
% |\childdocof| and |\childdocforward|
% with empty macros to prevent further inclusions of the main document:
%    \begin{macrocode}
\newcommand{\childdocdisable}
{
  \renewcommand{\childdocmain}[1]{\renewcommand{\childdocmain}[1]{\endinput}}
  \renewcommand{\childdocof}[1]{}
  \renewcommand{\childdocby}[2][]{}
  \renewcommand{\childdocforward}[2][]{}
  \renewcommand{\childdocdisable}{}
}
%    \end{macrocode}

% \macro{\childdocmain}
% The macro |\childdocmain| is to be called at the top of the main file
% with nothing or the main filename (without extension) as argument.
% First, it breaks loops.
% If the argument is not empty and does not match |\childdocname|
% (which is set by the first inclusion of |childdoc.def|),
% |\ifchilddoc| is set to true, |\includeonly| is applied to the child file
% and |\jobname| is set to the main file
% (for proper handling of |.aux| files):
%    \begin{macrocode}
\newcommand{\childdocmain}[1]
{
  \childdocdisable\childdocmain{}
  \if?#1?\else
    \begingroup
      \def\childdoctmp{#1}
      \ifx\childdoctmp\childdocname
        \def\childdoctmp{}
      \else
        \def\childdoctmp
        {
          \childdoctrue
          \includeonly{\childdocname}
          \def\childdocjob{#1}
          \def\jobname{#1}
        }
      \fi
      \expandafter
    \endgroup
    \childdoctmp
  \fi
}
%    \end{macrocode}

% \macro{\childdocof}
% The command |\childdocof| redirects
% compilation to the main file |#1|.
%    \begin{macrocode}
\newcommand{\childdocof}[1]
{
  \childdocdisable
  \childdoctrue
  \includeonly{\childdocname}
  \def\jobname{#1}
  \def\childdocjob{#1}
  \input{#1}
}
%    \end{macrocode}

% \macro{\childdocby}
% The command |\childdocby| ....
%    \begin{macrocode}
\newcommand{\childdocby}[2][]
{
  \childdocdisable
  \childdoctrue
  \childdocmanualtrue
  \if?#1?\else
    \def\jobname{#2}
  \fi
  \def\childdocjob{#2}
  \input{#2}
  \endinput
}
%    \end{macrocode}

% \macro{\childdocforward}
% The command |\childdocforward| redirects
% compilation to the main file or
% (if the optional argument is given) a child file.
% Parameters are set as if the main file
% or a child file starting with |\childdocof| was compiled.
% Then compilation is handed over to the main file:
%    \begin{macrocode}
\newcommand{\childdocforward}[2][]
{
  \begingroup
    \if?#1?
      \def\childdoctmp
      {
        \def\childdocname{#2}
        \def\childdocjob{#2}
        \def\jobname{#2}
        \input{#2}
        \endinput
      }
    \else
      \def\childdoctmp
      {
        \childdocdisable
        \def\childdocname{#2}
        \childdoctrue
        \includeonly{#2}
        \def\childdocjob{#1}
        \def\jobname{#1}
        \input{#1}
        \endinput
      }
    \fi
    \expandafter
  \endgroup
  \childdoctmp
}
%    \end{macrocode}

% \macro{\childdocforwardprefix}
% The command |\childdocforwardprefix| redirects
% compilation to the main or a child file by means of a pattern.
% The prefix |#1| in the current filename is replaced by |#2|
% and the suffix of the current filename is kept
% (it is assumed that the filename does not contain the substring `|~~~|'
% which is used as a delimiter).
% Compilation is handed over to the new file by |\childdocforward|:
%    \begin{macrocode}
\newcommand{\childdocforwardprefix}[3][]
{
  \begingroup
    \def\childdocextract #2##1~~~{\def\childdoctmp{\childdocforward[#1]{#3##1}}}
    \expandafter\childdocextract\childdocname~~~
    \expandafter
  \endgroup
  \childdoctmp
}
%    \end{macrocode}

% \macro{\childdoc}
% The deprecated macro |\childdoc| is a legacy version of |\childdocmain|:
%    \begin{macrocode}
\newcommand{\childdoc}{\childdocmain}
%    \end{macrocode}

% \macro{\childdocredirect}
% The deprecated macro |\childdocredirect| is a legacy version
% of |\childdocforward| and |\childdocforwardprefix|:
%    \begin{macrocode}
\newcommand{\childdocredirect}[2][]
{
  \begingroup
    \if?#1?
      \def\childdoctmp{\childdocforward{#2}}
    \else
      \def\childdoctmp{\childdocforwardprefix{#1}{#2}}
    \fi
    \expandafter
  \endgroup
  \childdoctmp
}
%    \end{macrocode}

%\iffalse
%</package>
%\fi
%
\endinput
|\\
|\childdocforwardprefix{final}{child}|
\end{tabular}
\end{center}
%

Note that when several versions of a main file and/or of each child file
are to be generated, it may be convenient to set up a |Makefile| or
shell script to automatise the process.

%%%%%%%%%%%%%%%%%%%%%%%%%%%%%%%%%%%%%%%%%%%%%%%%%%%%%%%%%%%%%%%%%%%%%%%%%%%%%%%%
\subsection{Command Line Processing}
\label{sec:commandline}

The effect of redirection files can also be achieved by invoking
the \LaTeX{} compiler with a more elaborate command line.
Most conveniently this should be done as part
of a shell script or a |Makefile|.

When using \textsf{childdoc} in the main file, the following
command lines effectively perform a redirection
(note that depending on the shell being used,
backslashes may have to be doubled: `|\|' $\to$ `|\\|'):
%
\begin{center}
|... -jobname "|\textit{target}|" |\\|"|[\textit{flags}]%
|% \iffalse
%
% childdoc.dtx Copyright (C) 2017-2018 Niklas Beisert
%
% This work may be distributed and/or modified under the
% conditions of the LaTeX Project Public License, either version 1.3
% of this license or (at your option) any later version.
% The latest version of this license is in
%   http://www.latex-project.org/lppl.txt
% and version 1.3 or later is part of all distributions of LaTeX
% version 2005/12/01 or later.
%
% This work has the LPPL maintenance status `maintained'.
%
% The Current Maintainer of this work is Niklas Beisert.
%
% This work consists of the files childdoc.dtx and childdoc.ins
% and the derived files childdoc.def and cdocsamp.tex with
% cdocsch1.tex, cdocsch2.tex, cdocsdrf.tex, cdocsfn1.tex, cdocsfn2.tex.
%
%<package>\ifdefined\childdocmain\endinput\fi
%<package>\ProvidesFile{childdoc.def}[2018/12/30 v2.0 child document driver]
%<samplemain>\ProvidesFile{cdocsamp.tex}[2018/12/30 v2.0 sample for childdoc]
%<*driver>
%\ProvidesFile{childdoc.drv}[2018/12/30 v2.0 childdoc reference manual file]
\PassOptionsToClass{10pt,a4paper}{article}
\documentclass{ltxdoc}

\usepackage[margin=35mm]{geometry}
\usepackage{hyperref}
\usepackage{hyperxmp}
\usepackage[usenames]{color}

\hypersetup{colorlinks=true}
\hypersetup{pdfstartview=FitH}
\hypersetup{pdfpagemode=UseNone}
\hypersetup{pdfsource={}}
\hypersetup{pdflang={en-UK}}
\hypersetup{pdfcopyright={Copyright 2017-2018 Niklas Beisert.
  This work may be distributed and/or modified under the
  conditions of the LaTeX Project Public License, either version 1.3
  of this license or (at your option) any later version.}}
\hypersetup{pdflicenseurl={http://www.latex-project.org/lppl.txt}}
\hypersetup{pdfcontactaddress={ETH Zurich, ITP, HIT K,
  Wolfgang-Pauli-Strasse 27}}
\hypersetup{pdfcontactpostcode={8093}}
\hypersetup{pdfcontactcity={Zurich}}
\hypersetup{pdfcontactcountry={Switzerland}}
\hypersetup{pdfcontactemail={nbeisert@itp.phys.ethz.ch}}
\hypersetup{pdfcontacturl={http://people.phys.ethz.ch/\xmptilde nbeisert/}}

\newcommand{\secref}[1]{\hyperref[#1]{section \ref*{#1}}}

\parskip1ex
\parindent0pt
\let\olditemize\itemize
\def\itemize{\olditemize\parskip0pt}

\begin{document}

\title{The \textsf{childdoc} Package}
\hypersetup{pdftitle={The childdoc Package}}
\author{Niklas Beisert\\[2ex]
  Institut f\"ur Theoretische Physik\\
  Eidgen\"ossische Technische Hochschule Z\"urich\\
  Wolfgang-Pauli-Strasse 27, 8093 Z\"urich, Switzerland\\[1ex]
  \href{mailto:nbeisert@itp.phys.ethz.ch}
  {\texttt{nbeisert@itp.phys.ethz.ch}}}
\hypersetup{pdfauthor={Niklas Beisert}}
\hypersetup{pdfsubject={Manual for the LaTeX2e Package childdoc}}
\date{30 December 2018, \textsf{v2.0}}
\maketitle

\begin{abstract}\noindent
\textsf{childdoc} is a \LaTeXe{} package
that enables the direct compilation
of document sections included by |\include|
to individual files.
\end{abstract}

\begingroup
\parskip0ex
\tableofcontents
\endgroup

%%%%%%%%%%%%%%%%%%%%%%%%%%%%%%%%%%%%%%%%%%%%%%%%%%%%%%%%%%%%%%%%%%%%%%%%%%%%%%%%
%%%%%%%%%%%%%%%%%%%%%%%%%%%%%%%%%%%%%%%%%%%%%%%%%%%%%%%%%%%%%%%%%%%%%%%%%%%%%%%%
\section{Introduction}

\LaTeX{} provides a mechanism to structure a large document (such as a book)
into a main file and several child files (containing the chapters)
using the |\include| command.
This mechanism is beneficial for documents
which span hundreds of pages in order to
make the source file(s) more manageable.
Moreover, compilation can be restricted to
selected child files by means of the |\includeonly| command.
The latter feature can be used to reduce the compilation time while editing
(this was significantly more useful in the earlier days of \LaTeX{})
or to generate a smaller document which is easier to navigate.
Another application of |\includeonly| is to generate
documents consisting of selected parts of the complete document.

However, there are a few drawbacks of the plain |\include| mechanism:
\begin{itemize}
\item
The child files cannot be compiled on their own,
they can only be compiled via the main file.
A naive editing environment
(such as a text editor with an option
to have the current file processed by \LaTeX)
may require one to switch to the main file before compiling;
attempting to compile the child file produces errors.
\item
The main file must be modified (each time)
to adjust the |\includeonly| command
to the present needs. This easily leaves the main file in a messy state.
\item
The generated document will always carry the filename
of the main document. This is inconvenient if
several child files are to be compiled and
to be kept for distribution.
\end{itemize}

The present package provides a simple interface
to make child files individually compilable by \LaTeX{}.
Compiling a child file then has the same effect as compiling
the main file with an |\includeonly| command
to select the appropriate child.
Moreover the generated document will carry the name of the child
rather than the main file.
This resolves all three above issues.

This feature is meant to make the editing of books,
thesis documents and lecture notes somewhat more convenient.
However, the package can also be used efficiently for
composing a series of documents (such as exercise sheets)
which are typically distributed individually.
It then assists the author in generating the individual documents
(potentially in different versions)
as well as a document containing the collected series.
Another application is in developing style files
or other kinds of included material
where compilation of the style file could redirect
to a sample or test file.

%%%%%%%%%%%%%%%%%%%%%%%%%%%%%%%%%%%%%%%%%%%%%%%%%%%%%%%%%%%%%%%%%%%%%%%%%%%%%%%%
%%%%%%%%%%%%%%%%%%%%%%%%%%%%%%%%%%%%%%%%%%%%%%%%%%%%%%%%%%%%%%%%%%%%%%%%%%%%%%%%
\section{Usage}

First of all, the package \textsf{childdoc} is \emph{not} a standard
\LaTeXe{} |.sty| style file! Therefore it needs to be invoked in
a non-standard way.

%%%%%%%%%%%%%%%%%%%%%%%%%%%%%%%%%%%%%%%%%%%%%%%%%%%%%%%%%%%%%%%%%%%%%%%%%%%%%%%%
\subsection{Included Files}
\label{sec:include}

%%%%%%%%%%%%%%%%%%%%%%%%%%%%%%%%%%%%%%%%
\DescribeMacro{\childdocmain}
To use the package, add the commands
\begin{center}
\begin{tabular}{l}
|\input{childdoc.def}|\\
|\childdocmain{}|\\
\end{tabular}
\end{center}
at the very top of the main \LaTeX{} file,
in particular \emph{before} the |\documentclass| statement!
The argument of |\childdocmain| should be left empty
(but it must be present).

%%%%%%%%%%%%%%%%%%%%%%%%%%%%%%%%%%%%%%%%
\DescribeMacro{\childdocof}
Furthermore, add the commands
\begin{center}
\begin{tabular}{l}
|\input{childdoc.def}|\\
|\childdocof{|\textit{main}|}|\\
\end{tabular}
\end{center}
at the top of every child file \textit{child}
which is included by |\include{|\textit{child}|}|
from within the main file
(or at least for those files to be compiled individually).
The argument \textit{main} must be the filename of the main file.

There are a couple of
considerations in setting up the main and child documents:

%%%%%%%%%%%%%%%%%%%%%%%%%%%%%%%%%%%%%%%%
\paragraph{Restrictions.}

Please note the following restrictions:
\begin{itemize}
\item
|\childdocmain| must be called with one argument \textit{main}
to ensure compatibility with earlier version of the package.
It must either be empty (|\childdocmain{}|)
or precisely match the filename of the main file in which it is specified.
See \secref{sec:detection} for further information.
\item
The filename \textit{main} must be specified without the |.tex| extension.
\item
The filename \textit{main} is case sensitive
(even in case-insensitive file systems)
due to internal string comparison.
\item
The argument \textit{main} should be fully expanded, it cannot be a macro.
\item
Subdirectories and special characters should be avoided in filenames.
\item
The command |\childdocmain{|\textit{main}|}| must be followed by a whitespace.
It should not be followed immediately by another command
or by a comment mark `|%|'.
This is because the \TeX{} parser reads the token immediately following
the argument of |\childdocmain| and puts it
at the beginning of every child section;
however, a white\-space is ignored.
\end{itemize}

%%%%%%%%%%%%%%%%%%%%%%%%%%%%%%%%%%%%%%%%
\paragraph{Content of Main File.}

It is advisable to place all content in the child files included by |\include|.
Any output contained in the main file will appear in all child documents
unless suppressed manually;
it cannot be suppressed automatically by the |\includeonly| directive
and thus should normally be avoided.
A method to include some content in the main file
by means of conditional processing is described in \secref{sec:conditional}.

%%%%%%%%%%%%%%%%%%%%%%%%%%%%%%%%%%%%%%%%
\paragraph{Page Numbering.}

When only a part of the document is compiled,
the appropriate numbering of pages
(as well as other status parameters)
is determined from the |.aux| files.
The latter contain information from previous passes.
However this information needs to propagate through
all intermediate child documents.
Therefore the page numbering in child documents may well
be inconsistent until the complete document is compiled at least once.

A useful (if unconventional) way to always ensure a consistent
page numbering is to restart the numbering in each child document
and denote the pages by `\textit{child}|.|\textit{page}'
where \textit{child} represents the chapter/section number of the child file.
This can be achieved by the command
|\numberwithin{page}{|\textit{child}|}|
of the \textsf{amsmath} package
where \textit{child} can be |chapter| or |section|
depending on the chosen structuring.
Alternatively, one can modify the macro |\thepage| appropriately
and reset the counter |page| at the start of each child file.

%%%%%%%%%%%%%%%%%%%%%%%%%%%%%%%%%%%%%%%%%%%%%%%%%%%%%%%%%%%%%%%%%%%%%%%%%%%%%%%%
\subsection{Conditional Processing}
\label{sec:conditional}

The package provides a mechanism to compile different versions
of a document. To customise the versions further some conditional processing
can come in handy to distinguish which version is being compiled.
The package provides two macros to describe the compilation context:

%%%%%%%%%%%%%%%%%%%%%%%%%%%%%%%%%%%%%%%%
\DescribeMacro{\ifchilddoc}
The conditional |\ifchilddoc| distinguishes between the compilation of
child documents and the main document:
%
\begin{center}
|\ifchilddoc |\textit{child-code}| |[|\||else |\textit{main-code}]| \||fi|
\end{center}

%%%%%%%%%%%%%%%%%%%%%%%%%%%%%%%%%%%%%%%%
\DescribeMacro{\childdocname}
\DescribeMacro{\childdocjob}
The macro |\childdocname| contains the filename (without extension)
of the main or child file being processed.
Note that |\childdocjob| will always contain the name of the main file.

%%%%%%%%%%%%%%%%%%%%%%%%%%%%%%%%%%%%%%%%
\paragraph{Title Page.}

Conditional processing can be used to include a title or banner page
in the main document when proper precautions are taken.
Importantly, the code in the main file should ensure that the page counter
(as well as other status parameters which are stored in the |.aux| files)
takes the same value after the conditional processing.
Otherwise the page numbers may take divergent values
depending on which part is compiled.

For example, a title page could be declared by:
%
\begin{center}
\begin{tabular}{l}
|\ifchilddoc\||else|\\
|\addtocounter{page}{-1}|\\
\textit{code for title page}\\
|\newpage|\\
|\||fi|
\end{tabular}
\end{center}
%
A banner page for the child documents can be generated by:
%
\begin{center}
\begin{tabular}{l}
|\ifchilddoc|\\
|\addtocounter{page}{-1}|\\
\textit{code for banner page}\\
|\newpage|\\
|\||fi|
\end{tabular}
\end{center}
%
Here one could write a message such as:
\begin{center}
|This is the part \childdocname{} of \childdocjob{}.|
\end{center}

%%%%%%%%%%%%%%%%%%%%%%%%%%%%%%%%%%%%%%%%%%%%%%%%%%%%%%%%%%%%%%%%%%%%%%%%%%%%%%%%
\subsection{Flags}
\label{sec:flags}

The package makes it easy to generate different versions
of the main or child documents.
To this end compilation flags can be defined
and assigned different default values.
They will be particularly useful in conjunction
with the forwarding mechanism described in \secref{sec:forward}.

For example, it may be useful to have a flag |\version|
which can be set to |draft| or |final|.
The document source will contain some conditional code
depending on the value of |\version|.
Suppose further, the flag should default to |final| for the main file
and to |draft| for child files
which is a natural assignment for editing the document.
This is achieved by placing the following code
in the preamble of the main document
(below the |\childdocmain| directive):
%
\begin{center}
\begin{tabular}{l}
|\ifchilddoc|\\
|\providecommand{\version}{draft}|\\
|\||else|\\
|\providecommand{\version}{final}|\\
|\||fi|
\end{tabular}
\end{center}
%
The definition by |\providecommand| makes sure
that previous definitions are not overwritten.
Further statements |\providecommand{\version}{...}|
can thus be added before the above code to override it.

For the main file, one might add a line
(between |\childdocmain| and the above block)
%
\begin{center}
|%\ifchilddoc\||else\providecommand{\version}{draft}\||fi|
\end{center}
%
which can be uncommented to produce a draft version.
Likewise one can add a line to the very top of a child file
(above the |\childdocof{|\textit{main}|}| directive)
%
\begin{center}
|%\providecommand{\version}{final}|
\end{center}
%
which can be uncommented to produce the final version of this child document.

%%%%%%%%%%%%%%%%%%%%%%%%%%%%%%%%%%%%%%%%%%%%%%%%%%%%%%%%%%%%%%%%%%%%%%%%%%%%%%%%
\subsection{Forwarding}
\label{sec:forward}

Different versions of the main or child documents
using compilation flags as described in \secref{sec:flags}
can be (permanently) stored in different files
for convenient compilation, viewing and distribution.
To this end, the package defines a command
to pass on compilation to a different file:

%%%%%%%%%%%%%%%%%%%%%%%%%%%%%%%%%%%%%%%%
\DescribeMacro{\childdocforward}
The command |\childdocforward| redirects processing to
another source file:
%
\begin{center}
\begin{tabular}{l}
|\input{childdoc.def}|\\
|\childdocforward[|\textit{main}|]{|\textit{dest}|}|\\
\end{tabular}
\end{center}
%
The argument \textit{dest} is the destination file
(without extension).
It should be the main file or one of the child files.
Note that further \textsf{childdoc} directives
such as |\childdocof| and |\childdocforward|
in the indicated file will be processed in this form.
The optional argument \textit{main}
passes on directly to the main file \textit{main}
while pretending to compile the child \textit{dest}.
This form behaves as if \textit{dest}
issues |\childdocof{|\textit{main}|}| right away,
and no further \textsf{childdoc} directives will be processed.

%%%%%%%%%%%%%%%%%%%%%%%%%%%%%%%%%%%%%%%%
\DescribeMacro{\...prefix}
In the alternative form |\childdocforwardprefix|,
%
\begin{center}
\begin{tabular}{l}
|\input{childdoc.def}|\\
|\childdocforwardprefix[|\textit{main}|]{|\textit{prefix}|}{|\textit{dest}|}|
\end{tabular}
\end{center}
%
the destination file is determined by a pattern
depending on the current file:
To make this work, the current file must be called
`{\textit{prefix}\hspace{0.2em}\textit{suffix}}'
with \textit{prefix} matching precisely the argument.
Processing is then passed on to the file
`{\textit{dest}\hspace{0.2em}\textit{suffix}}'.
Surely, the same effect is achieved by
directly specifying the
argument `{\textit{dest}\hspace{0.2em}\textit{suffix}}'
in the first form.
However, that requires to set up a different file
for each child. With the alternative form of the command
all these files can have exactly the same content
which simplifies setting them up and maintaining them.

For example, the following file |draft.tex|
with a compilation flag |\version| as described in \secref{sec:flags}
compiles the main document as a draft:
%
\begin{center}
\begin{tabular}{l}
|\def\version{draft}|\\
|\input{childdoc.def}|\\
|\childdocforward{|\textit{main}|}|
\end{tabular}
\end{center}
%
Likewise, the following files |final|\textit{nn}|.tex|
compile the final version of the child document
|child|\textit{nn}|.tex|:
%
\begin{center}
\begin{tabular}{l}
|\def\version{final}|\\
|\input{childdoc.def}|\\
|\childdocforwardprefix{final}{child}|
\end{tabular}
\end{center}
%

Note that when several versions of a main file and/or of each child file
are to be generated, it may be convenient to set up a |Makefile| or
shell script to automatise the process.

%%%%%%%%%%%%%%%%%%%%%%%%%%%%%%%%%%%%%%%%%%%%%%%%%%%%%%%%%%%%%%%%%%%%%%%%%%%%%%%%
\subsection{Command Line Processing}
\label{sec:commandline}

The effect of redirection files can also be achieved by invoking
the \LaTeX{} compiler with a more elaborate command line.
Most conveniently this should be done as part
of a shell script or a |Makefile|.

When using \textsf{childdoc} in the main file, the following
command lines effectively perform a redirection
(note that depending on the shell being used,
backslashes may have to be doubled: `|\|' $\to$ `|\\|'):
%
\begin{center}
|... -jobname "|\textit{target}|" |\\|"|[\textit{flags}]%
|\input{childdoc.def}\childdocforward[|\textit{main}|]{|\textit{dest}|}"|
\end{center}
%
Here \textit{target} is the name of the output file,
\textit{main} is the name of the main file
and \textit{dest} is the name of the main or child file to be processed
(all filenames without extensions).
The optional argument \textit{main} can be omitted
if \textit{main} matches \textit{dest}.
Optionally, compilation \textit{flags} can be defined via |\def| commands.
This command line makes the \TeX{} engine believe
it is compiling the file \textit{target}
whose content is specified as the latter parameter.
The provided code then forwards the processing to
\textit{main} or \textit{dest} as described in \secref{sec:forward}.

%%%%%%%%%%%%%%%%%%%%%%%%%%%%%%%%%%%%%%%%%%%%%%%%%%%%%%%%%%%%%%%%%%%%%%%%%%%%%%%%
\subsection{Include by Input}
\label{sec:input}

Including child documents by |\include| has some restrictions by design.
Most notably, the content of a child document always occupies
its own set of pages; pages cannot be shared between child documents.
Usually, this behaviour makes perfect sense
because each child document contain an essential part of the document.
However, in some situations it may be desirable to compose
a document from a collection of parts
without having mandatory page breaks between then.
For this case, the package
provides a mechanism to include parts
by |\input| which can also be processed individually.
However, by construction this mechanism
requires manual handling of the content to be output.

%%%%%%%%%%%%%%%%%%%%%%%%%%%%%%%%%%%%%%%%
\DescribeMacro{\ifchilddocmanual}
The main file should be prepared as usual, see \secref{sec:include}.
However, the document body must make a distinction
between processing of an individual part and of the main document, e.g.:
%
\begin{center}
\begin{tabular}{l}
|\ifchilddocmanual|\\
|\input{\childdocname}|\\
|\||else|\\
\textit{document body with }|\input{|\textit{part}|}|\\
|\||fi|
\end{tabular}
\end{center}
%
The conditional |\ifchilddocmanual| is true whenever
a part to be included by |\input| is being compiled,
and the name of the part is stored in |\childdocname|.

%%%%%%%%%%%%%%%%%%%%%%%%%%%%%%%%%%%%%%%%
\DescribeMacro{\childdocby}
Each part to be included by |\input| should start with:
%
\begin{center}
\begin{tabular}{l}
|\input{childdoc.def}|\\
|\childdocby{|\textit{main}|}|\\
\end{tabular}
\end{center}
%
The directive |\childdocby| is similar to |\childdocof|
described in \secref{sec:include},
but the subsequent selection of content must be done manually.
To that end, both |\ifchilddoc| and |\ifchilddocmanual|
will be true upon processing of a part,
and the name of the part is stored in |\childdocname|.
Note that |\jobname| will be set to the filename of the current part
so that each part receives an individual |.aux| file
that does not interfere with the |.aux| file(s) of the main document.
This behaviour can be altered by the alternative form
|\childdocby[*]{|\textit{main}|}| (with a non-empty optional argument)
which uses the |.aux| file of the main document
by setting |\jobname| to \textit{main}.

%%%%%%%%%%%%%%%%%%%%%%%%%%%%%%%%%%%%%%%%%%%%%%%%%%%%%%%%%%%%%%%%%%%%%%%%%%%%%%%%
\subsection{Driver Development}
\label{sec:driver}

The \textsf{childdoc} mechanism can also be use for the development
of definition files such as \LaTeX{} styles or classes.
This case differs from the above setup with multiple parts
included by |\include| in that no |\includeonly| should be invoked.
This can be achieved by starting the include file
(before |\ProvidesPackage|) with:
%
\begin{center}
\begin{tabular}{l}
|\input{childdoc.def}|\\
|\childdocforward{|\textit{main}|}|\\
\end{tabular}
\end{center}
%
or alternatively with:
%
\begin{center}
\begin{tabular}{l}
|\input{childdoc.def}|\\
|\childdocby{|\textit{main}|}|\\
\end{tabular}
\end{center}
%
Both forms have slightly different effects as described above.
The main file is prepared as usual, see \secref{sec:include}.

%%%%%%%%%%%%%%%%%%%%%%%%%%%%%%%%%%%%%%%%%%%%%%%%%%%%%%%%%%%%%%%%%%%%%%%%%%%%%%%%
\subsection{Legacy Detection}
\label{sec:detection}

The directive |\childdocmain| in the main file can detect
whether the complete document or merely a child is to be compiled
even without using the directive |\childdocof|.
This method is deprecated because it is less robust
and there is no compelling reason to use it;
it is merely provided for backward compatibility
and it may be removed in future versions.

If the detection mechanism is to be used,
it is mandatory to correctly specify
the filename of the main file as the argument of |\childdocmain|:
%
\begin{center}
\begin{tabular}{l}
|\input{childdoc.def}|\\
|\childdocmain{|\textit{main}|}|\\
\end{tabular}
\end{center}
%
If |\jobname| does not match the argument \textit{main} of |\childdocmain|,
it is assumed that |\jobname| points to the child file to be compiled.
When using |\childdocmain| with the main file specified as argument,
it suffices to start a child file
with just |\input{|\textit{main}|}|
without loading of the package and using |\childdocof|.
If instead all processing is done
with the appropriate \textsf{childdoc} directives,
the argument of \textit{main} of |\childdocmain| can be empty.

An alternative version of the command line processing described
in \secref{sec:commandline} using the detection mechanism reads:
%
\begin{center}
|... -jobname "|\textit{target}|" "|[\textit{flags}]%
[|\def\jobname{|\textit{dest}|}|]|\input{|\textit{main}|}"|
\end{center}

%%%%%%%%%%%%%%%%%%%%%%%%%%%%%%%%%%%%%%%%%%%%%%%%%%%%%%%%%%%%%%%%%%%%%%%%%%%%%%%%
\subsection{Manual Code}
\label{sec:manual}

In case one cannot be certain whether the definitions file |childdoc.def|
is installed on the target \TeX{} distribution
and one prefers not to ship it,
it is conceivable to paste a few relevant commands into the sources.

To that end, drop all statements |\input{childdoc.def}|
and perform the replacements as outlined below.
Instead of |\childdocmain{|\textit{main}|}| add the following code
to the top of the main file:
%
\begin{center}
\begin{tabular}{l}
|\||ifdefined\childdocname\endinput\||fi\newif\ifchilddoc|\\
|\edef\childdocname{\scantokens\expandafter{\jobname\noexpand}}|\\
|\def\childdocmain{|\textit{main}|}\||ifx\childdocmain\childdocname\||else|\\
|\childdoctrue\includeonly{\childdocname}\let\jobname\childdocmain\||fi|\\
\end{tabular}
\end{center}
%
Instead of |\childdocof{|\textit{main}|}| just include the main file
at the top of each child file:
%
\begin{center}
|\input{|\textit{main}|}|
\end{center}
%
A simple redirection |\childdocforward{|\textit{dest}|}| is achieved by:
%
\begin{center}
|\def\jobname{|\textit{dest}|}\input{\jobname}|
\end{center}
%
The redirection with prefix
|\childdocforwardprefix[|\textit{prefix}|]{|\textit{dest}|}|
is accomplished by:
%
\begin{center}
\begin{tabular}{l}
|{\edef\jobname{\scantokens\expandafter{\jobname\noexpand}}|\\
|\def\redirectjob |\textit{prefix}|#1~~~{\gdef\jobname{|\textit{dest}|#1}}|\\
|\expandafter\redirectjob\jobname~~~}\input{\jobname}|
\end{tabular}
\end{center}

In an alternative approach,
child documents can be compiled by a specific command line
without additional code or specific definitions:
%
\begin{center}
|... -jobname "|\textit{target}|" "|[\textit{flags}]%
|\includeonly{|\textit{dest}|}\input{|\textit{main}|}"|
\end{center}
%

%%%%%%%%%%%%%%%%%%%%%%%%%%%%%%%%%%%%%%%%%%%%%%%%%%%%%%%%%%%%%%%%%%%%%%%%%%%%%%%%
%%%%%%%%%%%%%%%%%%%%%%%%%%%%%%%%%%%%%%%%%%%%%%%%%%%%%%%%%%%%%%%%%%%%%%%%%%%%%%%%
\section{Information}

%%%%%%%%%%%%%%%%%%%%%%%%%%%%%%%%%%%%%%%%%%%%%%%%%%%%%%%%%%%%%%%%%%%%%%%%%%%%%%%%
\subsection{Copyright}

Copyright \copyright{} 2017--2018 Niklas Beisert

This work may be distributed and/or modified under the
conditions of the \LaTeX{} Project Public License, either version 1.3
of this license or (at your option) any later version.
The latest version of this license is in
  \url{http://www.latex-project.org/lppl.txt}
and version 1.3 or later is part of all distributions of \LaTeX{}
version 2005/12/01 or later.

This work has the LPPL maintenance status `maintained'.

The Current Maintainer of this work is Niklas Beisert.

This work consists of the files |README.txt|, |childdoc.ins| and |childdoc.dtx|
as well as the derived files |childdoc.def|, |cdocsamp.tex|
with |cdocsch1.tex|, |cdocsch2.tex|, |cdocspt3.tex|, |cdocspt4.tex|,
|cdocsdrf.tex|, |cdocsfn1.tex|, |cdocsfn2.tex|
as well as |childdoc.pdf|.

%%%%%%%%%%%%%%%%%%%%%%%%%%%%%%%%%%%%%%%%%%%%%%%%%%%%%%%%%%%%%%%%%%%%%%%%%%%%%%%%
\subsection{Files and Installation}

The package consists of the files:
%
\begin{center}
\begin{tabular}{ll}
    |README.txt|   & readme file \\
    |childdoc.ins| & installation file \\
    |childdoc.dtx| & source file \\
    |childdoc.def| & definition file \\
    |cdocsamp.tex| & sample main file \\
    |cdocsch1.tex| & sample include file \\
    |cdocsch2.tex| & sample include file \\
    |cdocspt3.tex| & sample part file \\
    |cdocspt4.tex| & sample part file \\
    |cdocsdrf.tex| & sample redirection file \\
    |cdocsfn1.tex| & sample redirection file \\
    |cdocsfn2.tex| & sample redirection file \\
    |childdoc.pdf| & manual
\end{tabular}
\end{center}
%
The distribution consists of the files
|README.txt|, |childdoc.ins| and |childdoc.dtx|.
%
\begin{itemize}
\item
Run (pdf)\LaTeX{} on |childdoc.dtx|
to compile the manual |childdoc.pdf| (this file).
\item
Run \LaTeX{} on |childdoc.ins| to create the definitions file |childdoc.def|
and the sample |cdocsamp.tex| with include files
|cdocsch1.tex|, |cdocsch2.tex|, |cdocspt3.tex|, |cdocspt4.tex|,
|cdocsdrf.tex|, |cdocsfn1.tex|, |cdocsfn2.tex|.
Then copy the file |childdoc.def| to an appropriate directory of your \LaTeX{}
distribution, e.g.\ \textit{texmf-root}|/tex/latex/childdoc|.
\end{itemize}

%%%%%%%%%%%%%%%%%%%%%%%%%%%%%%%%%%%%%%%%%%%%%%%%%%%%%%%%%%%%%%%%%%%%%%%%%%%%%%%%
\subsection{Related CTAN Packages}

There are several other packages which offer a similar functionality:
%
\begin{itemize}
\item
The packages
\href{http://ctan.org/pkg/docmute}{\textsf{docmute}},
\href{http://ctan.org/pkg/includex}{\textsf{includex}} and
\href{http://ctan.org/pkg/standalone}{\textsf{standalone}}
provide commands to include only the document body of
a child file thus allowing both files to be compiled individually.
\item
The packages \href{http://ctan.org/pkg/subdocs}{\textsf{subdocs}}
and \href{http://ctan.org/pkg/subfiles}{\textsf{subfiles}}
provide structures in which the main and child documents can be
encapsulated and allowing them to be compiled individually.
The inclusion mechanism is different from the conventional |\include|.
\item
The package \href{http://ctan.org/pkg/combine}{\textsf{combine}}
is an elaborate solution to combine several documents into one.
\end{itemize}
%
See also the CTAN topic \href{http://ctan.org/topic/subdocs}{\textsf{subdocs}}
for further related packages.
The present package differs from the above solutions in that
a document structure constructed with the conventional |\include| mechanism
just needs two extra commands at the top of every file
such that all constituent files can be compiled individually.

%%%%%%%%%%%%%%%%%%%%%%%%%%%%%%%%%%%%%%%%%%%%%%%%%%%%%%%%%%%%%%%%%%%%%%%%%%%%%%%%
%\subsection{Feature Suggestions}
%
%The following is a list of features which may be useful for future
%versions of this package:
%%
%\begin{itemize}
%\item
%\ldots
%\end{itemize}

%%%%%%%%%%%%%%%%%%%%%%%%%%%%%%%%%%%%%%%%%%%%%%%%%%%%%%%%%%%%%%%%%%%%%%%%%%%%%%%%
\subsection{Revision History}

%%%%%%%%%%%%%%%%%%%%%%%%%%%%%%%%%%%%%%%%
\paragraph{v2.0:} 2018/12/30

\begin{itemize}
\item
immediate forward processing
\item
added |\childdocby| mechanism
\item
manual restructured
\end{itemize}

%%%%%%%%%%%%%%%%%%%%%%%%%%%%%%%%%%%%%%%%
\paragraph{v1.6:} 2018/01/17

\begin{itemize}
\item
application for development of include files
\item
corrections to manual
\end{itemize}

%%%%%%%%%%%%%%%%%%%%%%%%%%%%%%%%%%%%%%%%
\paragraph{v1.5:} 2017/05/21

\begin{itemize}
\item
more complete structuring introduced
\item
|\childdocof| introduced
\item
|\childdoc| renamed to |\childdocmain|
\item
|\childredirect| renamed to |\childdocforward| and |\childdocforwardprefix|
and functionality expanded
\end{itemize}

%%%%%%%%%%%%%%%%%%%%%%%%%%%%%%%%%%%%%%%%
\paragraph{v1.0:} 2017/04/27

\begin{itemize}
\item
manual and install package
\item
first version published on CTAN
\end{itemize}

%%%%%%%%%%%%%%%%%%%%%%%%%%%%%%%%%%%%%%%%
\paragraph{v0.6:} 2017/04/26

\begin{itemize}
\item
redirection mechanism added
\end{itemize}

%%%%%%%%%%%%%%%%%%%%%%%%%%%%%%%%%%%%%%%%
\paragraph{v0.5:} 2017/04/26

\begin{itemize}
\item
functionality in definition file
\end{itemize}


%%%%%%%%%%%%%%%%%%%%%%%%%%%%%%%%%%%%%%%%%%%%%%%%%%%%%%%%%%%%%%%%%%%%%%%%%%%%%%%%
%%%%%%%%%%%%%%%%%%%%%%%%%%%%%%%%%%%%%%%%%%%%%%%%%%%%%%%%%%%%%%%%%%%%%%%%%%%%%%%%
%%%%%%%%%%%%%%%%%%%%%%%%%%%%%%%%%%%%%%%%%%%%%%%%%%%%%%%%%%%%%%%%%%%%%%%%%%%%%%%%
\appendix

\settowidth\MacroIndent{\rmfamily\scriptsize 000\ }

 \DocInput{childdoc.dtx}

\end{document}
%</driver>
% \fi
%
% %%%%%%%%%%%%%%%%%%%%%%%%%%%%%%%%%%%%%%%%%%%%%%%%%%%%%%%%%%%%%%%%%%%%%%%%%%%%%%
% %%%%%%%%%%%%%%%%%%%%%%%%%%%%%%%%%%%%%%%%%%%%%%%%%%%%%%%%%%%%%%%%%%%%%%%%%%%%%%
% \section{Sample}
%\iffalse
%<*samplemain>
%\fi
%
% The following presents a sample document
% with two chapters, two parts, a title page,
% a compile flag as well as three forwarding files to set the flag.
% It consists of eight |.tex| files:
% \begin{center}
% \begin{tabular}{ll}
% |cdocsamp.tex|&main file\\
% |cdocsch1.tex|&include file for chapter 1\\
% |cdocsch2.tex|&include file for chapter 2\\
% |cdocspt3.tex|&include file for part 3\\
% |cdocspt4.tex|&include file for part 4\\
% |cdocsdrf.tex|&forwarding file for main file in draft mode\\
% |cdocsfi1.tex|&forwarding file for final version of chapter 1\\
% |cdocsfi2.tex|&forwarding file for final version of chapter 2\\
% \end{tabular}
% \end{center}
% Each of the eight files can be compiled directly by the \LaTeX{} compiler.
%
% %%%%%%%%%%%%%%%%%%%%%%%%%%%%%%%%%%%%%%
% \paragraph{Main File.}
%
% The main file is called |cdocsamp.tex|.
%
% Load the \textsf{childdoc} definitions and
% declare the filename for the main document:
%    \begin{macrocode}
\input{childdoc.def}
\childdocmain{}
%    \end{macrocode}

% Optional override for |\version| flag:
%    \begin{macrocode}
%%\ifchilddoc\else\providecommand{\version}{draft}\fi
%    \end{macrocode}

% Define the default values for the |\version| flag
% (|final| for the main file and |draft| for childs):
%    \begin{macrocode}
\ifchilddoc
\providecommand{\version}{draft}
\else
\providecommand{\version}{final}
\fi
%    \end{macrocode}

% Load the standard document class:
%    \begin{macrocode}
\documentclass[12pt]{article}
%    \end{macrocode}

% Start the document body:
%    \begin{macrocode}
\begin{document}
%    \end{macrocode}

% Declare a title page.
% Print title, part of document being processed and version flag:
%    \begin{macrocode}
\addtocounter{page}{-1}
\begin{center}
{\LARGE\bfseries{}childdoc example\par}
\vspace{1cm}
\ifchilddoc
\ifchilddocmanual part\else chapter\fi:
`\childdocname' of `\childdocjob'\par
\else
main document: `\childdocjob'\par
\fi
version: \version\par
\end{center}
\newpage
%    \end{macrocode}

% Manually include selected file,
% otherwise process as usual:
%    \begin{macrocode}
\ifchilddocmanual
\section*{part `\childdocname'}
\input{\childdocname}
\else
%    \end{macrocode}

% Include the two chapters:
%    \begin{macrocode}
\include{cdocsch1}
\include{cdocsch2}
%    \end{macrocode}

% Include the two parts unless only chapters should be displayed:
%    \begin{macrocode}
\ifchilddoc\else
\section{part three}
\input{cdocspt3}
\section{part four}
\input{cdocspt4}
\fi
%    \end{macrocode}

% Process as usual until here:
%    \begin{macrocode}
\fi
%    \end{macrocode}

% End of document body:
%    \begin{macrocode}
\end{document}
%    \end{macrocode}
%\iffalse
%</samplemain>
%\fi
%
% %%%%%%%%%%%%%%%%%%%%%%%%%%%%%%%%%%%%%%
% \paragraph{Chapter Include Files.}
%
% The include files are called |cdocsch1.tex| and |cdocsch2.tex|.
%
%\iffalse
%<*samplechap1|samplechap2>
%\fi

% Optional override for |\version| flag:
%    \begin{macrocode}
%%\providecommand{\version}{final}
%    \end{macrocode}

% Include the main document:
%    \begin{macrocode}
\input{childdoc.def}
\childdocof{cdocsamp}
%    \end{macrocode}

%\iffalse
%</samplechap1|samplechap2>
%\fi
%
%\iffalse
%<*samplechap1>
%\fi
% Some text for chapter 1:
%    \begin{macrocode}
\section{one}
some text in chapter one
%    \end{macrocode}

%\iffalse
%</samplechap1>
%\fi
% Some text for chapter 2:
%\iffalse
%<*samplechap2>
%\fi
%    \begin{macrocode}
\section{two}
more text in chapter two
%    \end{macrocode}

%\iffalse
%</samplechap2>
%\fi
%
% %%%%%%%%%%%%%%%%%%%%%%%%%%%%%%%%%%%%%%
% \paragraph{Part Include Files.}
%
% The include files are called |cdocspt3.tex| and |cdocspt4.tex|.
%
%\iffalse
%<*samplepart3|samplepart4>
%\fi

% Optional override for |\version| flag:
%    \begin{macrocode}
%%\providecommand{\version}{final}
%    \end{macrocode}

% Include the main document:
%    \begin{macrocode}
\input{childdoc.def}
\childdocby{cdocsamp}
%    \end{macrocode}

%\iffalse
%</samplepart3|samplepart4>
%\fi
%
%\iffalse
%<*samplepart3>
%\fi
% Some text for part 3:
%    \begin{macrocode}
some text in part three
%    \end{macrocode}

%\iffalse
%</samplepart3>
%\fi
% Some text for part 4:
%\iffalse
%<*samplepart4>
%\fi
%    \begin{macrocode}
more text in part four
%    \end{macrocode}

%\iffalse
%</samplepart4>
%\fi
%
% %%%%%%%%%%%%%%%%%%%%%%%%%%%%%%%%%%%%%%
% \paragraph{Forwarding for a Complete Draft.}
%
% The following forwarding file |cdocsdrf.tex|
% compiles the main document in draft mode:
%\iffalse
%<*sampledraft>
%\fi
%    \begin{macrocode}
\def\version{draft}
\input{childdoc.def}
\childdocforward{cdocsamp}
%    \end{macrocode}

%\iffalse
%</sampledraft>
%\fi
%
% %%%%%%%%%%%%%%%%%%%%%%%%%%%%%%%%%%%%%%
% \paragraph{Forwarding for Final Version of the Chapters.}
%
% The following forwarding files |cdocsfn1.tex| and |cdocsfn2.tex|
% (with identical content)
% compile the final versions of the child documents
% |cdocsch1.tex| and |cdocsch2.tex|, respectively:
%\iffalse
%<*samplefinal>
%\fi
%    \begin{macrocode}
\def\version{final}
\input{childdoc.def}
\childdocforwardprefix[cdocsamp]{cdocsfn}{cdocsch}
%    \end{macrocode}

%\iffalse
%</samplefinal>
%\fi
%
% %%%%%%%%%%%%%%%%%%%%%%%%%%%%%%%%%%%%%%
% \paragraph{Command Line Processing.}
%
% The following three command lines generate the output files
% |cdocscld|, |cdocscl1| and |cdocscl2|
% which should be identical to
% |cdocsdrf|, |cdocsch1| and |cdocsfn2|, respectively:
% \begin{center}
% \begin{tabular}{l}
% |latex -jobname cdocscld \|\\
% |  "\def\version{draft}\input{childdoc.def}\childdocforward{cdocsamp}"|\\
% |latex -jobname cdocscl1 \|\\
% |  "\input{childdoc.def}\childdocforward[cdocsamp]{cdocsch1}"|\\
% |latex -jobname cdocscl2 \|\\
% |  "\def\version{final}\input{childdoc.def}\childdocforward{cdocsch2}"|
% \end{tabular}
% \end{center}
% Note that the trailing backslash on each first line
% merely continues the input to the second line
% (for convenient cut ant paste).
% Furthermore, the command |latex| can be replaced by any
% of its alternative versions such as |pdflatex|.
%
% %%%%%%%%%%%%%%%%%%%%%%%%%%%%%%%%%%%%%%%%%%%%%%%%%%%%%%%%%%%%%%%%%%%%%%%%%%%%%%
% %%%%%%%%%%%%%%%%%%%%%%%%%%%%%%%%%%%%%%%%%%%%%%%%%%%%%%%%%%%%%%%%%%%%%%%%%%%%%%
% \section{Implementation}
%\iffalse
%<*package>
%\fi
%
% This section describes the definitions file |childdoc.def|.

% The definitions cannot be loaded using |\usepackage| or |\RequirePackage|
% which has a mechanism to prevent loading a style file more than once.
% When loading the definitions by means of |\input|
% multiple instances have to be prevented manually:
%\iffalse
%This code needs to be before the `\ProvidesFile' directive
%which is defined at the beginning of this file.
%Therefore it is also placed there and commented out here.
%</package>
%<*discard>
%\fi
%    \begin{macrocode}
\ifdefined\childdocmain\endinput\fi
%    \end{macrocode}
%\iffalse
%</discard>
%<*package>
%\fi
%
% \macro{\ifchilddoc}
% \macro{\ifchilddocmanual}
% The conditional |\ifchilddoc| tells whether a
% child (true) or main (false) document is being compiled.
% The conditional |\ifchilddocmanual| tells whether
% the |\includeonly| mechanism is used (false) or
% the selection of child files must be performed manually (true).
% The definitions initialise to false:
%    \begin{macrocode}
\newif\ifchilddoc
\newif\ifchilddocmanual
%    \end{macrocode}

% \macro{\childdocname}
% \macro{\childdocjob}
% The macro |\childdocname| stores the name of the main document
% to be compiled. The macro |\childdocjob| stores the name of
% the document on which the \LaTeX{} compiler was originally invoked.
% The content of |\jobname| cannot be compared
% to filenames specified in the source due to different catcodes.
% The following code rescans |\jobname|, stores the result
% in |\childdocname| and saves a copy in |\childdocjob|:
%    \begin{macrocode}
\edef\childdocname{\scantokens\expandafter{\jobname\noexpand}}
\let\childdocjob\childdocname
%    \end{macrocode}

% \macro{\childdocdisable}
% The macro |\childdocdisable| prevents the main file
% from being processed more than once.
% At this stage, the main document command |\childdocmain|
% is assumed to be called once again where it should do nothing.
% Any subsequent call to it should prevent
% a secondary processing of the main document
% It overwrites the forwarding commands
% |\childdocof| and |\childdocforward|
% with empty macros to prevent further inclusions of the main document:
%    \begin{macrocode}
\newcommand{\childdocdisable}
{
  \renewcommand{\childdocmain}[1]{\renewcommand{\childdocmain}[1]{\endinput}}
  \renewcommand{\childdocof}[1]{}
  \renewcommand{\childdocby}[2][]{}
  \renewcommand{\childdocforward}[2][]{}
  \renewcommand{\childdocdisable}{}
}
%    \end{macrocode}

% \macro{\childdocmain}
% The macro |\childdocmain| is to be called at the top of the main file
% with nothing or the main filename (without extension) as argument.
% First, it breaks loops.
% If the argument is not empty and does not match |\childdocname|
% (which is set by the first inclusion of |childdoc.def|),
% |\ifchilddoc| is set to true, |\includeonly| is applied to the child file
% and |\jobname| is set to the main file
% (for proper handling of |.aux| files):
%    \begin{macrocode}
\newcommand{\childdocmain}[1]
{
  \childdocdisable\childdocmain{}
  \if?#1?\else
    \begingroup
      \def\childdoctmp{#1}
      \ifx\childdoctmp\childdocname
        \def\childdoctmp{}
      \else
        \def\childdoctmp
        {
          \childdoctrue
          \includeonly{\childdocname}
          \def\childdocjob{#1}
          \def\jobname{#1}
        }
      \fi
      \expandafter
    \endgroup
    \childdoctmp
  \fi
}
%    \end{macrocode}

% \macro{\childdocof}
% The command |\childdocof| redirects
% compilation to the main file |#1|.
%    \begin{macrocode}
\newcommand{\childdocof}[1]
{
  \childdocdisable
  \childdoctrue
  \includeonly{\childdocname}
  \def\jobname{#1}
  \def\childdocjob{#1}
  \input{#1}
}
%    \end{macrocode}

% \macro{\childdocby}
% The command |\childdocby| ....
%    \begin{macrocode}
\newcommand{\childdocby}[2][]
{
  \childdocdisable
  \childdoctrue
  \childdocmanualtrue
  \if?#1?\else
    \def\jobname{#2}
  \fi
  \def\childdocjob{#2}
  \input{#2}
  \endinput
}
%    \end{macrocode}

% \macro{\childdocforward}
% The command |\childdocforward| redirects
% compilation to the main file or
% (if the optional argument is given) a child file.
% Parameters are set as if the main file
% or a child file starting with |\childdocof| was compiled.
% Then compilation is handed over to the main file:
%    \begin{macrocode}
\newcommand{\childdocforward}[2][]
{
  \begingroup
    \if?#1?
      \def\childdoctmp
      {
        \def\childdocname{#2}
        \def\childdocjob{#2}
        \def\jobname{#2}
        \input{#2}
        \endinput
      }
    \else
      \def\childdoctmp
      {
        \childdocdisable
        \def\childdocname{#2}
        \childdoctrue
        \includeonly{#2}
        \def\childdocjob{#1}
        \def\jobname{#1}
        \input{#1}
        \endinput
      }
    \fi
    \expandafter
  \endgroup
  \childdoctmp
}
%    \end{macrocode}

% \macro{\childdocforwardprefix}
% The command |\childdocforwardprefix| redirects
% compilation to the main or a child file by means of a pattern.
% The prefix |#1| in the current filename is replaced by |#2|
% and the suffix of the current filename is kept
% (it is assumed that the filename does not contain the substring `|~~~|'
% which is used as a delimiter).
% Compilation is handed over to the new file by |\childdocforward|:
%    \begin{macrocode}
\newcommand{\childdocforwardprefix}[3][]
{
  \begingroup
    \def\childdocextract #2##1~~~{\def\childdoctmp{\childdocforward[#1]{#3##1}}}
    \expandafter\childdocextract\childdocname~~~
    \expandafter
  \endgroup
  \childdoctmp
}
%    \end{macrocode}

% \macro{\childdoc}
% The deprecated macro |\childdoc| is a legacy version of |\childdocmain|:
%    \begin{macrocode}
\newcommand{\childdoc}{\childdocmain}
%    \end{macrocode}

% \macro{\childdocredirect}
% The deprecated macro |\childdocredirect| is a legacy version
% of |\childdocforward| and |\childdocforwardprefix|:
%    \begin{macrocode}
\newcommand{\childdocredirect}[2][]
{
  \begingroup
    \if?#1?
      \def\childdoctmp{\childdocforward{#2}}
    \else
      \def\childdoctmp{\childdocforwardprefix{#1}{#2}}
    \fi
    \expandafter
  \endgroup
  \childdoctmp
}
%    \end{macrocode}

%\iffalse
%</package>
%\fi
%
\endinput
\childdocforward[|\textit{main}|]{|\textit{dest}|}"|
\end{center}
%
Here \textit{target} is the name of the output file,
\textit{main} is the name of the main file
and \textit{dest} is the name of the main or child file to be processed
(all filenames without extensions).
The optional argument \textit{main} can be omitted
if \textit{main} matches \textit{dest}.
Optionally, compilation \textit{flags} can be defined via |\def| commands.
This command line makes the \TeX{} engine believe
it is compiling the file \textit{target}
whose content is specified as the latter parameter.
The provided code then forwards the processing to
\textit{main} or \textit{dest} as described in \secref{sec:forward}.

%%%%%%%%%%%%%%%%%%%%%%%%%%%%%%%%%%%%%%%%%%%%%%%%%%%%%%%%%%%%%%%%%%%%%%%%%%%%%%%%
\subsection{Include by Input}
\label{sec:input}

Including child documents by |\include| has some restrictions by design.
Most notably, the content of a child document always occupies
its own set of pages; pages cannot be shared between child documents.
Usually, this behaviour makes perfect sense
because each child document contain an essential part of the document.
However, in some situations it may be desirable to compose
a document from a collection of parts
without having mandatory page breaks between then.
For this case, the package
provides a mechanism to include parts
by |\input| which can also be processed individually.
However, by construction this mechanism
requires manual handling of the content to be output.

%%%%%%%%%%%%%%%%%%%%%%%%%%%%%%%%%%%%%%%%
\DescribeMacro{\ifchilddocmanual}
The main file should be prepared as usual, see \secref{sec:include}.
However, the document body must make a distinction
between processing of an individual part and of the main document, e.g.:
%
\begin{center}
\begin{tabular}{l}
|\ifchilddocmanual|\\
|\input{\childdocname}|\\
|\||else|\\
\textit{document body with }|\input{|\textit{part}|}|\\
|\||fi|
\end{tabular}
\end{center}
%
The conditional |\ifchilddocmanual| is true whenever
a part to be included by |\input| is being compiled,
and the name of the part is stored in |\childdocname|.

%%%%%%%%%%%%%%%%%%%%%%%%%%%%%%%%%%%%%%%%
\DescribeMacro{\childdocby}
Each part to be included by |\input| should start with:
%
\begin{center}
\begin{tabular}{l}
|% \iffalse
%
% childdoc.dtx Copyright (C) 2017-2018 Niklas Beisert
%
% This work may be distributed and/or modified under the
% conditions of the LaTeX Project Public License, either version 1.3
% of this license or (at your option) any later version.
% The latest version of this license is in
%   http://www.latex-project.org/lppl.txt
% and version 1.3 or later is part of all distributions of LaTeX
% version 2005/12/01 or later.
%
% This work has the LPPL maintenance status `maintained'.
%
% The Current Maintainer of this work is Niklas Beisert.
%
% This work consists of the files childdoc.dtx and childdoc.ins
% and the derived files childdoc.def and cdocsamp.tex with
% cdocsch1.tex, cdocsch2.tex, cdocsdrf.tex, cdocsfn1.tex, cdocsfn2.tex.
%
%<package>\ifdefined\childdocmain\endinput\fi
%<package>\ProvidesFile{childdoc.def}[2018/12/30 v2.0 child document driver]
%<samplemain>\ProvidesFile{cdocsamp.tex}[2018/12/30 v2.0 sample for childdoc]
%<*driver>
%\ProvidesFile{childdoc.drv}[2018/12/30 v2.0 childdoc reference manual file]
\PassOptionsToClass{10pt,a4paper}{article}
\documentclass{ltxdoc}

\usepackage[margin=35mm]{geometry}
\usepackage{hyperref}
\usepackage{hyperxmp}
\usepackage[usenames]{color}

\hypersetup{colorlinks=true}
\hypersetup{pdfstartview=FitH}
\hypersetup{pdfpagemode=UseNone}
\hypersetup{pdfsource={}}
\hypersetup{pdflang={en-UK}}
\hypersetup{pdfcopyright={Copyright 2017-2018 Niklas Beisert.
  This work may be distributed and/or modified under the
  conditions of the LaTeX Project Public License, either version 1.3
  of this license or (at your option) any later version.}}
\hypersetup{pdflicenseurl={http://www.latex-project.org/lppl.txt}}
\hypersetup{pdfcontactaddress={ETH Zurich, ITP, HIT K,
  Wolfgang-Pauli-Strasse 27}}
\hypersetup{pdfcontactpostcode={8093}}
\hypersetup{pdfcontactcity={Zurich}}
\hypersetup{pdfcontactcountry={Switzerland}}
\hypersetup{pdfcontactemail={nbeisert@itp.phys.ethz.ch}}
\hypersetup{pdfcontacturl={http://people.phys.ethz.ch/\xmptilde nbeisert/}}

\newcommand{\secref}[1]{\hyperref[#1]{section \ref*{#1}}}

\parskip1ex
\parindent0pt
\let\olditemize\itemize
\def\itemize{\olditemize\parskip0pt}

\begin{document}

\title{The \textsf{childdoc} Package}
\hypersetup{pdftitle={The childdoc Package}}
\author{Niklas Beisert\\[2ex]
  Institut f\"ur Theoretische Physik\\
  Eidgen\"ossische Technische Hochschule Z\"urich\\
  Wolfgang-Pauli-Strasse 27, 8093 Z\"urich, Switzerland\\[1ex]
  \href{mailto:nbeisert@itp.phys.ethz.ch}
  {\texttt{nbeisert@itp.phys.ethz.ch}}}
\hypersetup{pdfauthor={Niklas Beisert}}
\hypersetup{pdfsubject={Manual for the LaTeX2e Package childdoc}}
\date{30 December 2018, \textsf{v2.0}}
\maketitle

\begin{abstract}\noindent
\textsf{childdoc} is a \LaTeXe{} package
that enables the direct compilation
of document sections included by |\include|
to individual files.
\end{abstract}

\begingroup
\parskip0ex
\tableofcontents
\endgroup

%%%%%%%%%%%%%%%%%%%%%%%%%%%%%%%%%%%%%%%%%%%%%%%%%%%%%%%%%%%%%%%%%%%%%%%%%%%%%%%%
%%%%%%%%%%%%%%%%%%%%%%%%%%%%%%%%%%%%%%%%%%%%%%%%%%%%%%%%%%%%%%%%%%%%%%%%%%%%%%%%
\section{Introduction}

\LaTeX{} provides a mechanism to structure a large document (such as a book)
into a main file and several child files (containing the chapters)
using the |\include| command.
This mechanism is beneficial for documents
which span hundreds of pages in order to
make the source file(s) more manageable.
Moreover, compilation can be restricted to
selected child files by means of the |\includeonly| command.
The latter feature can be used to reduce the compilation time while editing
(this was significantly more useful in the earlier days of \LaTeX{})
or to generate a smaller document which is easier to navigate.
Another application of |\includeonly| is to generate
documents consisting of selected parts of the complete document.

However, there are a few drawbacks of the plain |\include| mechanism:
\begin{itemize}
\item
The child files cannot be compiled on their own,
they can only be compiled via the main file.
A naive editing environment
(such as a text editor with an option
to have the current file processed by \LaTeX)
may require one to switch to the main file before compiling;
attempting to compile the child file produces errors.
\item
The main file must be modified (each time)
to adjust the |\includeonly| command
to the present needs. This easily leaves the main file in a messy state.
\item
The generated document will always carry the filename
of the main document. This is inconvenient if
several child files are to be compiled and
to be kept for distribution.
\end{itemize}

The present package provides a simple interface
to make child files individually compilable by \LaTeX{}.
Compiling a child file then has the same effect as compiling
the main file with an |\includeonly| command
to select the appropriate child.
Moreover the generated document will carry the name of the child
rather than the main file.
This resolves all three above issues.

This feature is meant to make the editing of books,
thesis documents and lecture notes somewhat more convenient.
However, the package can also be used efficiently for
composing a series of documents (such as exercise sheets)
which are typically distributed individually.
It then assists the author in generating the individual documents
(potentially in different versions)
as well as a document containing the collected series.
Another application is in developing style files
or other kinds of included material
where compilation of the style file could redirect
to a sample or test file.

%%%%%%%%%%%%%%%%%%%%%%%%%%%%%%%%%%%%%%%%%%%%%%%%%%%%%%%%%%%%%%%%%%%%%%%%%%%%%%%%
%%%%%%%%%%%%%%%%%%%%%%%%%%%%%%%%%%%%%%%%%%%%%%%%%%%%%%%%%%%%%%%%%%%%%%%%%%%%%%%%
\section{Usage}

First of all, the package \textsf{childdoc} is \emph{not} a standard
\LaTeXe{} |.sty| style file! Therefore it needs to be invoked in
a non-standard way.

%%%%%%%%%%%%%%%%%%%%%%%%%%%%%%%%%%%%%%%%%%%%%%%%%%%%%%%%%%%%%%%%%%%%%%%%%%%%%%%%
\subsection{Included Files}
\label{sec:include}

%%%%%%%%%%%%%%%%%%%%%%%%%%%%%%%%%%%%%%%%
\DescribeMacro{\childdocmain}
To use the package, add the commands
\begin{center}
\begin{tabular}{l}
|\input{childdoc.def}|\\
|\childdocmain{}|\\
\end{tabular}
\end{center}
at the very top of the main \LaTeX{} file,
in particular \emph{before} the |\documentclass| statement!
The argument of |\childdocmain| should be left empty
(but it must be present).

%%%%%%%%%%%%%%%%%%%%%%%%%%%%%%%%%%%%%%%%
\DescribeMacro{\childdocof}
Furthermore, add the commands
\begin{center}
\begin{tabular}{l}
|\input{childdoc.def}|\\
|\childdocof{|\textit{main}|}|\\
\end{tabular}
\end{center}
at the top of every child file \textit{child}
which is included by |\include{|\textit{child}|}|
from within the main file
(or at least for those files to be compiled individually).
The argument \textit{main} must be the filename of the main file.

There are a couple of
considerations in setting up the main and child documents:

%%%%%%%%%%%%%%%%%%%%%%%%%%%%%%%%%%%%%%%%
\paragraph{Restrictions.}

Please note the following restrictions:
\begin{itemize}
\item
|\childdocmain| must be called with one argument \textit{main}
to ensure compatibility with earlier version of the package.
It must either be empty (|\childdocmain{}|)
or precisely match the filename of the main file in which it is specified.
See \secref{sec:detection} for further information.
\item
The filename \textit{main} must be specified without the |.tex| extension.
\item
The filename \textit{main} is case sensitive
(even in case-insensitive file systems)
due to internal string comparison.
\item
The argument \textit{main} should be fully expanded, it cannot be a macro.
\item
Subdirectories and special characters should be avoided in filenames.
\item
The command |\childdocmain{|\textit{main}|}| must be followed by a whitespace.
It should not be followed immediately by another command
or by a comment mark `|%|'.
This is because the \TeX{} parser reads the token immediately following
the argument of |\childdocmain| and puts it
at the beginning of every child section;
however, a white\-space is ignored.
\end{itemize}

%%%%%%%%%%%%%%%%%%%%%%%%%%%%%%%%%%%%%%%%
\paragraph{Content of Main File.}

It is advisable to place all content in the child files included by |\include|.
Any output contained in the main file will appear in all child documents
unless suppressed manually;
it cannot be suppressed automatically by the |\includeonly| directive
and thus should normally be avoided.
A method to include some content in the main file
by means of conditional processing is described in \secref{sec:conditional}.

%%%%%%%%%%%%%%%%%%%%%%%%%%%%%%%%%%%%%%%%
\paragraph{Page Numbering.}

When only a part of the document is compiled,
the appropriate numbering of pages
(as well as other status parameters)
is determined from the |.aux| files.
The latter contain information from previous passes.
However this information needs to propagate through
all intermediate child documents.
Therefore the page numbering in child documents may well
be inconsistent until the complete document is compiled at least once.

A useful (if unconventional) way to always ensure a consistent
page numbering is to restart the numbering in each child document
and denote the pages by `\textit{child}|.|\textit{page}'
where \textit{child} represents the chapter/section number of the child file.
This can be achieved by the command
|\numberwithin{page}{|\textit{child}|}|
of the \textsf{amsmath} package
where \textit{child} can be |chapter| or |section|
depending on the chosen structuring.
Alternatively, one can modify the macro |\thepage| appropriately
and reset the counter |page| at the start of each child file.

%%%%%%%%%%%%%%%%%%%%%%%%%%%%%%%%%%%%%%%%%%%%%%%%%%%%%%%%%%%%%%%%%%%%%%%%%%%%%%%%
\subsection{Conditional Processing}
\label{sec:conditional}

The package provides a mechanism to compile different versions
of a document. To customise the versions further some conditional processing
can come in handy to distinguish which version is being compiled.
The package provides two macros to describe the compilation context:

%%%%%%%%%%%%%%%%%%%%%%%%%%%%%%%%%%%%%%%%
\DescribeMacro{\ifchilddoc}
The conditional |\ifchilddoc| distinguishes between the compilation of
child documents and the main document:
%
\begin{center}
|\ifchilddoc |\textit{child-code}| |[|\||else |\textit{main-code}]| \||fi|
\end{center}

%%%%%%%%%%%%%%%%%%%%%%%%%%%%%%%%%%%%%%%%
\DescribeMacro{\childdocname}
\DescribeMacro{\childdocjob}
The macro |\childdocname| contains the filename (without extension)
of the main or child file being processed.
Note that |\childdocjob| will always contain the name of the main file.

%%%%%%%%%%%%%%%%%%%%%%%%%%%%%%%%%%%%%%%%
\paragraph{Title Page.}

Conditional processing can be used to include a title or banner page
in the main document when proper precautions are taken.
Importantly, the code in the main file should ensure that the page counter
(as well as other status parameters which are stored in the |.aux| files)
takes the same value after the conditional processing.
Otherwise the page numbers may take divergent values
depending on which part is compiled.

For example, a title page could be declared by:
%
\begin{center}
\begin{tabular}{l}
|\ifchilddoc\||else|\\
|\addtocounter{page}{-1}|\\
\textit{code for title page}\\
|\newpage|\\
|\||fi|
\end{tabular}
\end{center}
%
A banner page for the child documents can be generated by:
%
\begin{center}
\begin{tabular}{l}
|\ifchilddoc|\\
|\addtocounter{page}{-1}|\\
\textit{code for banner page}\\
|\newpage|\\
|\||fi|
\end{tabular}
\end{center}
%
Here one could write a message such as:
\begin{center}
|This is the part \childdocname{} of \childdocjob{}.|
\end{center}

%%%%%%%%%%%%%%%%%%%%%%%%%%%%%%%%%%%%%%%%%%%%%%%%%%%%%%%%%%%%%%%%%%%%%%%%%%%%%%%%
\subsection{Flags}
\label{sec:flags}

The package makes it easy to generate different versions
of the main or child documents.
To this end compilation flags can be defined
and assigned different default values.
They will be particularly useful in conjunction
with the forwarding mechanism described in \secref{sec:forward}.

For example, it may be useful to have a flag |\version|
which can be set to |draft| or |final|.
The document source will contain some conditional code
depending on the value of |\version|.
Suppose further, the flag should default to |final| for the main file
and to |draft| for child files
which is a natural assignment for editing the document.
This is achieved by placing the following code
in the preamble of the main document
(below the |\childdocmain| directive):
%
\begin{center}
\begin{tabular}{l}
|\ifchilddoc|\\
|\providecommand{\version}{draft}|\\
|\||else|\\
|\providecommand{\version}{final}|\\
|\||fi|
\end{tabular}
\end{center}
%
The definition by |\providecommand| makes sure
that previous definitions are not overwritten.
Further statements |\providecommand{\version}{...}|
can thus be added before the above code to override it.

For the main file, one might add a line
(between |\childdocmain| and the above block)
%
\begin{center}
|%\ifchilddoc\||else\providecommand{\version}{draft}\||fi|
\end{center}
%
which can be uncommented to produce a draft version.
Likewise one can add a line to the very top of a child file
(above the |\childdocof{|\textit{main}|}| directive)
%
\begin{center}
|%\providecommand{\version}{final}|
\end{center}
%
which can be uncommented to produce the final version of this child document.

%%%%%%%%%%%%%%%%%%%%%%%%%%%%%%%%%%%%%%%%%%%%%%%%%%%%%%%%%%%%%%%%%%%%%%%%%%%%%%%%
\subsection{Forwarding}
\label{sec:forward}

Different versions of the main or child documents
using compilation flags as described in \secref{sec:flags}
can be (permanently) stored in different files
for convenient compilation, viewing and distribution.
To this end, the package defines a command
to pass on compilation to a different file:

%%%%%%%%%%%%%%%%%%%%%%%%%%%%%%%%%%%%%%%%
\DescribeMacro{\childdocforward}
The command |\childdocforward| redirects processing to
another source file:
%
\begin{center}
\begin{tabular}{l}
|\input{childdoc.def}|\\
|\childdocforward[|\textit{main}|]{|\textit{dest}|}|\\
\end{tabular}
\end{center}
%
The argument \textit{dest} is the destination file
(without extension).
It should be the main file or one of the child files.
Note that further \textsf{childdoc} directives
such as |\childdocof| and |\childdocforward|
in the indicated file will be processed in this form.
The optional argument \textit{main}
passes on directly to the main file \textit{main}
while pretending to compile the child \textit{dest}.
This form behaves as if \textit{dest}
issues |\childdocof{|\textit{main}|}| right away,
and no further \textsf{childdoc} directives will be processed.

%%%%%%%%%%%%%%%%%%%%%%%%%%%%%%%%%%%%%%%%
\DescribeMacro{\...prefix}
In the alternative form |\childdocforwardprefix|,
%
\begin{center}
\begin{tabular}{l}
|\input{childdoc.def}|\\
|\childdocforwardprefix[|\textit{main}|]{|\textit{prefix}|}{|\textit{dest}|}|
\end{tabular}
\end{center}
%
the destination file is determined by a pattern
depending on the current file:
To make this work, the current file must be called
`{\textit{prefix}\hspace{0.2em}\textit{suffix}}'
with \textit{prefix} matching precisely the argument.
Processing is then passed on to the file
`{\textit{dest}\hspace{0.2em}\textit{suffix}}'.
Surely, the same effect is achieved by
directly specifying the
argument `{\textit{dest}\hspace{0.2em}\textit{suffix}}'
in the first form.
However, that requires to set up a different file
for each child. With the alternative form of the command
all these files can have exactly the same content
which simplifies setting them up and maintaining them.

For example, the following file |draft.tex|
with a compilation flag |\version| as described in \secref{sec:flags}
compiles the main document as a draft:
%
\begin{center}
\begin{tabular}{l}
|\def\version{draft}|\\
|\input{childdoc.def}|\\
|\childdocforward{|\textit{main}|}|
\end{tabular}
\end{center}
%
Likewise, the following files |final|\textit{nn}|.tex|
compile the final version of the child document
|child|\textit{nn}|.tex|:
%
\begin{center}
\begin{tabular}{l}
|\def\version{final}|\\
|\input{childdoc.def}|\\
|\childdocforwardprefix{final}{child}|
\end{tabular}
\end{center}
%

Note that when several versions of a main file and/or of each child file
are to be generated, it may be convenient to set up a |Makefile| or
shell script to automatise the process.

%%%%%%%%%%%%%%%%%%%%%%%%%%%%%%%%%%%%%%%%%%%%%%%%%%%%%%%%%%%%%%%%%%%%%%%%%%%%%%%%
\subsection{Command Line Processing}
\label{sec:commandline}

The effect of redirection files can also be achieved by invoking
the \LaTeX{} compiler with a more elaborate command line.
Most conveniently this should be done as part
of a shell script or a |Makefile|.

When using \textsf{childdoc} in the main file, the following
command lines effectively perform a redirection
(note that depending on the shell being used,
backslashes may have to be doubled: `|\|' $\to$ `|\\|'):
%
\begin{center}
|... -jobname "|\textit{target}|" |\\|"|[\textit{flags}]%
|\input{childdoc.def}\childdocforward[|\textit{main}|]{|\textit{dest}|}"|
\end{center}
%
Here \textit{target} is the name of the output file,
\textit{main} is the name of the main file
and \textit{dest} is the name of the main or child file to be processed
(all filenames without extensions).
The optional argument \textit{main} can be omitted
if \textit{main} matches \textit{dest}.
Optionally, compilation \textit{flags} can be defined via |\def| commands.
This command line makes the \TeX{} engine believe
it is compiling the file \textit{target}
whose content is specified as the latter parameter.
The provided code then forwards the processing to
\textit{main} or \textit{dest} as described in \secref{sec:forward}.

%%%%%%%%%%%%%%%%%%%%%%%%%%%%%%%%%%%%%%%%%%%%%%%%%%%%%%%%%%%%%%%%%%%%%%%%%%%%%%%%
\subsection{Include by Input}
\label{sec:input}

Including child documents by |\include| has some restrictions by design.
Most notably, the content of a child document always occupies
its own set of pages; pages cannot be shared between child documents.
Usually, this behaviour makes perfect sense
because each child document contain an essential part of the document.
However, in some situations it may be desirable to compose
a document from a collection of parts
without having mandatory page breaks between then.
For this case, the package
provides a mechanism to include parts
by |\input| which can also be processed individually.
However, by construction this mechanism
requires manual handling of the content to be output.

%%%%%%%%%%%%%%%%%%%%%%%%%%%%%%%%%%%%%%%%
\DescribeMacro{\ifchilddocmanual}
The main file should be prepared as usual, see \secref{sec:include}.
However, the document body must make a distinction
between processing of an individual part and of the main document, e.g.:
%
\begin{center}
\begin{tabular}{l}
|\ifchilddocmanual|\\
|\input{\childdocname}|\\
|\||else|\\
\textit{document body with }|\input{|\textit{part}|}|\\
|\||fi|
\end{tabular}
\end{center}
%
The conditional |\ifchilddocmanual| is true whenever
a part to be included by |\input| is being compiled,
and the name of the part is stored in |\childdocname|.

%%%%%%%%%%%%%%%%%%%%%%%%%%%%%%%%%%%%%%%%
\DescribeMacro{\childdocby}
Each part to be included by |\input| should start with:
%
\begin{center}
\begin{tabular}{l}
|\input{childdoc.def}|\\
|\childdocby{|\textit{main}|}|\\
\end{tabular}
\end{center}
%
The directive |\childdocby| is similar to |\childdocof|
described in \secref{sec:include},
but the subsequent selection of content must be done manually.
To that end, both |\ifchilddoc| and |\ifchilddocmanual|
will be true upon processing of a part,
and the name of the part is stored in |\childdocname|.
Note that |\jobname| will be set to the filename of the current part
so that each part receives an individual |.aux| file
that does not interfere with the |.aux| file(s) of the main document.
This behaviour can be altered by the alternative form
|\childdocby[*]{|\textit{main}|}| (with a non-empty optional argument)
which uses the |.aux| file of the main document
by setting |\jobname| to \textit{main}.

%%%%%%%%%%%%%%%%%%%%%%%%%%%%%%%%%%%%%%%%%%%%%%%%%%%%%%%%%%%%%%%%%%%%%%%%%%%%%%%%
\subsection{Driver Development}
\label{sec:driver}

The \textsf{childdoc} mechanism can also be use for the development
of definition files such as \LaTeX{} styles or classes.
This case differs from the above setup with multiple parts
included by |\include| in that no |\includeonly| should be invoked.
This can be achieved by starting the include file
(before |\ProvidesPackage|) with:
%
\begin{center}
\begin{tabular}{l}
|\input{childdoc.def}|\\
|\childdocforward{|\textit{main}|}|\\
\end{tabular}
\end{center}
%
or alternatively with:
%
\begin{center}
\begin{tabular}{l}
|\input{childdoc.def}|\\
|\childdocby{|\textit{main}|}|\\
\end{tabular}
\end{center}
%
Both forms have slightly different effects as described above.
The main file is prepared as usual, see \secref{sec:include}.

%%%%%%%%%%%%%%%%%%%%%%%%%%%%%%%%%%%%%%%%%%%%%%%%%%%%%%%%%%%%%%%%%%%%%%%%%%%%%%%%
\subsection{Legacy Detection}
\label{sec:detection}

The directive |\childdocmain| in the main file can detect
whether the complete document or merely a child is to be compiled
even without using the directive |\childdocof|.
This method is deprecated because it is less robust
and there is no compelling reason to use it;
it is merely provided for backward compatibility
and it may be removed in future versions.

If the detection mechanism is to be used,
it is mandatory to correctly specify
the filename of the main file as the argument of |\childdocmain|:
%
\begin{center}
\begin{tabular}{l}
|\input{childdoc.def}|\\
|\childdocmain{|\textit{main}|}|\\
\end{tabular}
\end{center}
%
If |\jobname| does not match the argument \textit{main} of |\childdocmain|,
it is assumed that |\jobname| points to the child file to be compiled.
When using |\childdocmain| with the main file specified as argument,
it suffices to start a child file
with just |\input{|\textit{main}|}|
without loading of the package and using |\childdocof|.
If instead all processing is done
with the appropriate \textsf{childdoc} directives,
the argument of \textit{main} of |\childdocmain| can be empty.

An alternative version of the command line processing described
in \secref{sec:commandline} using the detection mechanism reads:
%
\begin{center}
|... -jobname "|\textit{target}|" "|[\textit{flags}]%
[|\def\jobname{|\textit{dest}|}|]|\input{|\textit{main}|}"|
\end{center}

%%%%%%%%%%%%%%%%%%%%%%%%%%%%%%%%%%%%%%%%%%%%%%%%%%%%%%%%%%%%%%%%%%%%%%%%%%%%%%%%
\subsection{Manual Code}
\label{sec:manual}

In case one cannot be certain whether the definitions file |childdoc.def|
is installed on the target \TeX{} distribution
and one prefers not to ship it,
it is conceivable to paste a few relevant commands into the sources.

To that end, drop all statements |\input{childdoc.def}|
and perform the replacements as outlined below.
Instead of |\childdocmain{|\textit{main}|}| add the following code
to the top of the main file:
%
\begin{center}
\begin{tabular}{l}
|\||ifdefined\childdocname\endinput\||fi\newif\ifchilddoc|\\
|\edef\childdocname{\scantokens\expandafter{\jobname\noexpand}}|\\
|\def\childdocmain{|\textit{main}|}\||ifx\childdocmain\childdocname\||else|\\
|\childdoctrue\includeonly{\childdocname}\let\jobname\childdocmain\||fi|\\
\end{tabular}
\end{center}
%
Instead of |\childdocof{|\textit{main}|}| just include the main file
at the top of each child file:
%
\begin{center}
|\input{|\textit{main}|}|
\end{center}
%
A simple redirection |\childdocforward{|\textit{dest}|}| is achieved by:
%
\begin{center}
|\def\jobname{|\textit{dest}|}\input{\jobname}|
\end{center}
%
The redirection with prefix
|\childdocforwardprefix[|\textit{prefix}|]{|\textit{dest}|}|
is accomplished by:
%
\begin{center}
\begin{tabular}{l}
|{\edef\jobname{\scantokens\expandafter{\jobname\noexpand}}|\\
|\def\redirectjob |\textit{prefix}|#1~~~{\gdef\jobname{|\textit{dest}|#1}}|\\
|\expandafter\redirectjob\jobname~~~}\input{\jobname}|
\end{tabular}
\end{center}

In an alternative approach,
child documents can be compiled by a specific command line
without additional code or specific definitions:
%
\begin{center}
|... -jobname "|\textit{target}|" "|[\textit{flags}]%
|\includeonly{|\textit{dest}|}\input{|\textit{main}|}"|
\end{center}
%

%%%%%%%%%%%%%%%%%%%%%%%%%%%%%%%%%%%%%%%%%%%%%%%%%%%%%%%%%%%%%%%%%%%%%%%%%%%%%%%%
%%%%%%%%%%%%%%%%%%%%%%%%%%%%%%%%%%%%%%%%%%%%%%%%%%%%%%%%%%%%%%%%%%%%%%%%%%%%%%%%
\section{Information}

%%%%%%%%%%%%%%%%%%%%%%%%%%%%%%%%%%%%%%%%%%%%%%%%%%%%%%%%%%%%%%%%%%%%%%%%%%%%%%%%
\subsection{Copyright}

Copyright \copyright{} 2017--2018 Niklas Beisert

This work may be distributed and/or modified under the
conditions of the \LaTeX{} Project Public License, either version 1.3
of this license or (at your option) any later version.
The latest version of this license is in
  \url{http://www.latex-project.org/lppl.txt}
and version 1.3 or later is part of all distributions of \LaTeX{}
version 2005/12/01 or later.

This work has the LPPL maintenance status `maintained'.

The Current Maintainer of this work is Niklas Beisert.

This work consists of the files |README.txt|, |childdoc.ins| and |childdoc.dtx|
as well as the derived files |childdoc.def|, |cdocsamp.tex|
with |cdocsch1.tex|, |cdocsch2.tex|, |cdocspt3.tex|, |cdocspt4.tex|,
|cdocsdrf.tex|, |cdocsfn1.tex|, |cdocsfn2.tex|
as well as |childdoc.pdf|.

%%%%%%%%%%%%%%%%%%%%%%%%%%%%%%%%%%%%%%%%%%%%%%%%%%%%%%%%%%%%%%%%%%%%%%%%%%%%%%%%
\subsection{Files and Installation}

The package consists of the files:
%
\begin{center}
\begin{tabular}{ll}
    |README.txt|   & readme file \\
    |childdoc.ins| & installation file \\
    |childdoc.dtx| & source file \\
    |childdoc.def| & definition file \\
    |cdocsamp.tex| & sample main file \\
    |cdocsch1.tex| & sample include file \\
    |cdocsch2.tex| & sample include file \\
    |cdocspt3.tex| & sample part file \\
    |cdocspt4.tex| & sample part file \\
    |cdocsdrf.tex| & sample redirection file \\
    |cdocsfn1.tex| & sample redirection file \\
    |cdocsfn2.tex| & sample redirection file \\
    |childdoc.pdf| & manual
\end{tabular}
\end{center}
%
The distribution consists of the files
|README.txt|, |childdoc.ins| and |childdoc.dtx|.
%
\begin{itemize}
\item
Run (pdf)\LaTeX{} on |childdoc.dtx|
to compile the manual |childdoc.pdf| (this file).
\item
Run \LaTeX{} on |childdoc.ins| to create the definitions file |childdoc.def|
and the sample |cdocsamp.tex| with include files
|cdocsch1.tex|, |cdocsch2.tex|, |cdocspt3.tex|, |cdocspt4.tex|,
|cdocsdrf.tex|, |cdocsfn1.tex|, |cdocsfn2.tex|.
Then copy the file |childdoc.def| to an appropriate directory of your \LaTeX{}
distribution, e.g.\ \textit{texmf-root}|/tex/latex/childdoc|.
\end{itemize}

%%%%%%%%%%%%%%%%%%%%%%%%%%%%%%%%%%%%%%%%%%%%%%%%%%%%%%%%%%%%%%%%%%%%%%%%%%%%%%%%
\subsection{Related CTAN Packages}

There are several other packages which offer a similar functionality:
%
\begin{itemize}
\item
The packages
\href{http://ctan.org/pkg/docmute}{\textsf{docmute}},
\href{http://ctan.org/pkg/includex}{\textsf{includex}} and
\href{http://ctan.org/pkg/standalone}{\textsf{standalone}}
provide commands to include only the document body of
a child file thus allowing both files to be compiled individually.
\item
The packages \href{http://ctan.org/pkg/subdocs}{\textsf{subdocs}}
and \href{http://ctan.org/pkg/subfiles}{\textsf{subfiles}}
provide structures in which the main and child documents can be
encapsulated and allowing them to be compiled individually.
The inclusion mechanism is different from the conventional |\include|.
\item
The package \href{http://ctan.org/pkg/combine}{\textsf{combine}}
is an elaborate solution to combine several documents into one.
\end{itemize}
%
See also the CTAN topic \href{http://ctan.org/topic/subdocs}{\textsf{subdocs}}
for further related packages.
The present package differs from the above solutions in that
a document structure constructed with the conventional |\include| mechanism
just needs two extra commands at the top of every file
such that all constituent files can be compiled individually.

%%%%%%%%%%%%%%%%%%%%%%%%%%%%%%%%%%%%%%%%%%%%%%%%%%%%%%%%%%%%%%%%%%%%%%%%%%%%%%%%
%\subsection{Feature Suggestions}
%
%The following is a list of features which may be useful for future
%versions of this package:
%%
%\begin{itemize}
%\item
%\ldots
%\end{itemize}

%%%%%%%%%%%%%%%%%%%%%%%%%%%%%%%%%%%%%%%%%%%%%%%%%%%%%%%%%%%%%%%%%%%%%%%%%%%%%%%%
\subsection{Revision History}

%%%%%%%%%%%%%%%%%%%%%%%%%%%%%%%%%%%%%%%%
\paragraph{v2.0:} 2018/12/30

\begin{itemize}
\item
immediate forward processing
\item
added |\childdocby| mechanism
\item
manual restructured
\end{itemize}

%%%%%%%%%%%%%%%%%%%%%%%%%%%%%%%%%%%%%%%%
\paragraph{v1.6:} 2018/01/17

\begin{itemize}
\item
application for development of include files
\item
corrections to manual
\end{itemize}

%%%%%%%%%%%%%%%%%%%%%%%%%%%%%%%%%%%%%%%%
\paragraph{v1.5:} 2017/05/21

\begin{itemize}
\item
more complete structuring introduced
\item
|\childdocof| introduced
\item
|\childdoc| renamed to |\childdocmain|
\item
|\childredirect| renamed to |\childdocforward| and |\childdocforwardprefix|
and functionality expanded
\end{itemize}

%%%%%%%%%%%%%%%%%%%%%%%%%%%%%%%%%%%%%%%%
\paragraph{v1.0:} 2017/04/27

\begin{itemize}
\item
manual and install package
\item
first version published on CTAN
\end{itemize}

%%%%%%%%%%%%%%%%%%%%%%%%%%%%%%%%%%%%%%%%
\paragraph{v0.6:} 2017/04/26

\begin{itemize}
\item
redirection mechanism added
\end{itemize}

%%%%%%%%%%%%%%%%%%%%%%%%%%%%%%%%%%%%%%%%
\paragraph{v0.5:} 2017/04/26

\begin{itemize}
\item
functionality in definition file
\end{itemize}


%%%%%%%%%%%%%%%%%%%%%%%%%%%%%%%%%%%%%%%%%%%%%%%%%%%%%%%%%%%%%%%%%%%%%%%%%%%%%%%%
%%%%%%%%%%%%%%%%%%%%%%%%%%%%%%%%%%%%%%%%%%%%%%%%%%%%%%%%%%%%%%%%%%%%%%%%%%%%%%%%
%%%%%%%%%%%%%%%%%%%%%%%%%%%%%%%%%%%%%%%%%%%%%%%%%%%%%%%%%%%%%%%%%%%%%%%%%%%%%%%%
\appendix

\settowidth\MacroIndent{\rmfamily\scriptsize 000\ }

 \DocInput{childdoc.dtx}

\end{document}
%</driver>
% \fi
%
% %%%%%%%%%%%%%%%%%%%%%%%%%%%%%%%%%%%%%%%%%%%%%%%%%%%%%%%%%%%%%%%%%%%%%%%%%%%%%%
% %%%%%%%%%%%%%%%%%%%%%%%%%%%%%%%%%%%%%%%%%%%%%%%%%%%%%%%%%%%%%%%%%%%%%%%%%%%%%%
% \section{Sample}
%\iffalse
%<*samplemain>
%\fi
%
% The following presents a sample document
% with two chapters, two parts, a title page,
% a compile flag as well as three forwarding files to set the flag.
% It consists of eight |.tex| files:
% \begin{center}
% \begin{tabular}{ll}
% |cdocsamp.tex|&main file\\
% |cdocsch1.tex|&include file for chapter 1\\
% |cdocsch2.tex|&include file for chapter 2\\
% |cdocspt3.tex|&include file for part 3\\
% |cdocspt4.tex|&include file for part 4\\
% |cdocsdrf.tex|&forwarding file for main file in draft mode\\
% |cdocsfi1.tex|&forwarding file for final version of chapter 1\\
% |cdocsfi2.tex|&forwarding file for final version of chapter 2\\
% \end{tabular}
% \end{center}
% Each of the eight files can be compiled directly by the \LaTeX{} compiler.
%
% %%%%%%%%%%%%%%%%%%%%%%%%%%%%%%%%%%%%%%
% \paragraph{Main File.}
%
% The main file is called |cdocsamp.tex|.
%
% Load the \textsf{childdoc} definitions and
% declare the filename for the main document:
%    \begin{macrocode}
\input{childdoc.def}
\childdocmain{}
%    \end{macrocode}

% Optional override for |\version| flag:
%    \begin{macrocode}
%%\ifchilddoc\else\providecommand{\version}{draft}\fi
%    \end{macrocode}

% Define the default values for the |\version| flag
% (|final| for the main file and |draft| for childs):
%    \begin{macrocode}
\ifchilddoc
\providecommand{\version}{draft}
\else
\providecommand{\version}{final}
\fi
%    \end{macrocode}

% Load the standard document class:
%    \begin{macrocode}
\documentclass[12pt]{article}
%    \end{macrocode}

% Start the document body:
%    \begin{macrocode}
\begin{document}
%    \end{macrocode}

% Declare a title page.
% Print title, part of document being processed and version flag:
%    \begin{macrocode}
\addtocounter{page}{-1}
\begin{center}
{\LARGE\bfseries{}childdoc example\par}
\vspace{1cm}
\ifchilddoc
\ifchilddocmanual part\else chapter\fi:
`\childdocname' of `\childdocjob'\par
\else
main document: `\childdocjob'\par
\fi
version: \version\par
\end{center}
\newpage
%    \end{macrocode}

% Manually include selected file,
% otherwise process as usual:
%    \begin{macrocode}
\ifchilddocmanual
\section*{part `\childdocname'}
\input{\childdocname}
\else
%    \end{macrocode}

% Include the two chapters:
%    \begin{macrocode}
\include{cdocsch1}
\include{cdocsch2}
%    \end{macrocode}

% Include the two parts unless only chapters should be displayed:
%    \begin{macrocode}
\ifchilddoc\else
\section{part three}
\input{cdocspt3}
\section{part four}
\input{cdocspt4}
\fi
%    \end{macrocode}

% Process as usual until here:
%    \begin{macrocode}
\fi
%    \end{macrocode}

% End of document body:
%    \begin{macrocode}
\end{document}
%    \end{macrocode}
%\iffalse
%</samplemain>
%\fi
%
% %%%%%%%%%%%%%%%%%%%%%%%%%%%%%%%%%%%%%%
% \paragraph{Chapter Include Files.}
%
% The include files are called |cdocsch1.tex| and |cdocsch2.tex|.
%
%\iffalse
%<*samplechap1|samplechap2>
%\fi

% Optional override for |\version| flag:
%    \begin{macrocode}
%%\providecommand{\version}{final}
%    \end{macrocode}

% Include the main document:
%    \begin{macrocode}
\input{childdoc.def}
\childdocof{cdocsamp}
%    \end{macrocode}

%\iffalse
%</samplechap1|samplechap2>
%\fi
%
%\iffalse
%<*samplechap1>
%\fi
% Some text for chapter 1:
%    \begin{macrocode}
\section{one}
some text in chapter one
%    \end{macrocode}

%\iffalse
%</samplechap1>
%\fi
% Some text for chapter 2:
%\iffalse
%<*samplechap2>
%\fi
%    \begin{macrocode}
\section{two}
more text in chapter two
%    \end{macrocode}

%\iffalse
%</samplechap2>
%\fi
%
% %%%%%%%%%%%%%%%%%%%%%%%%%%%%%%%%%%%%%%
% \paragraph{Part Include Files.}
%
% The include files are called |cdocspt3.tex| and |cdocspt4.tex|.
%
%\iffalse
%<*samplepart3|samplepart4>
%\fi

% Optional override for |\version| flag:
%    \begin{macrocode}
%%\providecommand{\version}{final}
%    \end{macrocode}

% Include the main document:
%    \begin{macrocode}
\input{childdoc.def}
\childdocby{cdocsamp}
%    \end{macrocode}

%\iffalse
%</samplepart3|samplepart4>
%\fi
%
%\iffalse
%<*samplepart3>
%\fi
% Some text for part 3:
%    \begin{macrocode}
some text in part three
%    \end{macrocode}

%\iffalse
%</samplepart3>
%\fi
% Some text for part 4:
%\iffalse
%<*samplepart4>
%\fi
%    \begin{macrocode}
more text in part four
%    \end{macrocode}

%\iffalse
%</samplepart4>
%\fi
%
% %%%%%%%%%%%%%%%%%%%%%%%%%%%%%%%%%%%%%%
% \paragraph{Forwarding for a Complete Draft.}
%
% The following forwarding file |cdocsdrf.tex|
% compiles the main document in draft mode:
%\iffalse
%<*sampledraft>
%\fi
%    \begin{macrocode}
\def\version{draft}
\input{childdoc.def}
\childdocforward{cdocsamp}
%    \end{macrocode}

%\iffalse
%</sampledraft>
%\fi
%
% %%%%%%%%%%%%%%%%%%%%%%%%%%%%%%%%%%%%%%
% \paragraph{Forwarding for Final Version of the Chapters.}
%
% The following forwarding files |cdocsfn1.tex| and |cdocsfn2.tex|
% (with identical content)
% compile the final versions of the child documents
% |cdocsch1.tex| and |cdocsch2.tex|, respectively:
%\iffalse
%<*samplefinal>
%\fi
%    \begin{macrocode}
\def\version{final}
\input{childdoc.def}
\childdocforwardprefix[cdocsamp]{cdocsfn}{cdocsch}
%    \end{macrocode}

%\iffalse
%</samplefinal>
%\fi
%
% %%%%%%%%%%%%%%%%%%%%%%%%%%%%%%%%%%%%%%
% \paragraph{Command Line Processing.}
%
% The following three command lines generate the output files
% |cdocscld|, |cdocscl1| and |cdocscl2|
% which should be identical to
% |cdocsdrf|, |cdocsch1| and |cdocsfn2|, respectively:
% \begin{center}
% \begin{tabular}{l}
% |latex -jobname cdocscld \|\\
% |  "\def\version{draft}\input{childdoc.def}\childdocforward{cdocsamp}"|\\
% |latex -jobname cdocscl1 \|\\
% |  "\input{childdoc.def}\childdocforward[cdocsamp]{cdocsch1}"|\\
% |latex -jobname cdocscl2 \|\\
% |  "\def\version{final}\input{childdoc.def}\childdocforward{cdocsch2}"|
% \end{tabular}
% \end{center}
% Note that the trailing backslash on each first line
% merely continues the input to the second line
% (for convenient cut ant paste).
% Furthermore, the command |latex| can be replaced by any
% of its alternative versions such as |pdflatex|.
%
% %%%%%%%%%%%%%%%%%%%%%%%%%%%%%%%%%%%%%%%%%%%%%%%%%%%%%%%%%%%%%%%%%%%%%%%%%%%%%%
% %%%%%%%%%%%%%%%%%%%%%%%%%%%%%%%%%%%%%%%%%%%%%%%%%%%%%%%%%%%%%%%%%%%%%%%%%%%%%%
% \section{Implementation}
%\iffalse
%<*package>
%\fi
%
% This section describes the definitions file |childdoc.def|.

% The definitions cannot be loaded using |\usepackage| or |\RequirePackage|
% which has a mechanism to prevent loading a style file more than once.
% When loading the definitions by means of |\input|
% multiple instances have to be prevented manually:
%\iffalse
%This code needs to be before the `\ProvidesFile' directive
%which is defined at the beginning of this file.
%Therefore it is also placed there and commented out here.
%</package>
%<*discard>
%\fi
%    \begin{macrocode}
\ifdefined\childdocmain\endinput\fi
%    \end{macrocode}
%\iffalse
%</discard>
%<*package>
%\fi
%
% \macro{\ifchilddoc}
% \macro{\ifchilddocmanual}
% The conditional |\ifchilddoc| tells whether a
% child (true) or main (false) document is being compiled.
% The conditional |\ifchilddocmanual| tells whether
% the |\includeonly| mechanism is used (false) or
% the selection of child files must be performed manually (true).
% The definitions initialise to false:
%    \begin{macrocode}
\newif\ifchilddoc
\newif\ifchilddocmanual
%    \end{macrocode}

% \macro{\childdocname}
% \macro{\childdocjob}
% The macro |\childdocname| stores the name of the main document
% to be compiled. The macro |\childdocjob| stores the name of
% the document on which the \LaTeX{} compiler was originally invoked.
% The content of |\jobname| cannot be compared
% to filenames specified in the source due to different catcodes.
% The following code rescans |\jobname|, stores the result
% in |\childdocname| and saves a copy in |\childdocjob|:
%    \begin{macrocode}
\edef\childdocname{\scantokens\expandafter{\jobname\noexpand}}
\let\childdocjob\childdocname
%    \end{macrocode}

% \macro{\childdocdisable}
% The macro |\childdocdisable| prevents the main file
% from being processed more than once.
% At this stage, the main document command |\childdocmain|
% is assumed to be called once again where it should do nothing.
% Any subsequent call to it should prevent
% a secondary processing of the main document
% It overwrites the forwarding commands
% |\childdocof| and |\childdocforward|
% with empty macros to prevent further inclusions of the main document:
%    \begin{macrocode}
\newcommand{\childdocdisable}
{
  \renewcommand{\childdocmain}[1]{\renewcommand{\childdocmain}[1]{\endinput}}
  \renewcommand{\childdocof}[1]{}
  \renewcommand{\childdocby}[2][]{}
  \renewcommand{\childdocforward}[2][]{}
  \renewcommand{\childdocdisable}{}
}
%    \end{macrocode}

% \macro{\childdocmain}
% The macro |\childdocmain| is to be called at the top of the main file
% with nothing or the main filename (without extension) as argument.
% First, it breaks loops.
% If the argument is not empty and does not match |\childdocname|
% (which is set by the first inclusion of |childdoc.def|),
% |\ifchilddoc| is set to true, |\includeonly| is applied to the child file
% and |\jobname| is set to the main file
% (for proper handling of |.aux| files):
%    \begin{macrocode}
\newcommand{\childdocmain}[1]
{
  \childdocdisable\childdocmain{}
  \if?#1?\else
    \begingroup
      \def\childdoctmp{#1}
      \ifx\childdoctmp\childdocname
        \def\childdoctmp{}
      \else
        \def\childdoctmp
        {
          \childdoctrue
          \includeonly{\childdocname}
          \def\childdocjob{#1}
          \def\jobname{#1}
        }
      \fi
      \expandafter
    \endgroup
    \childdoctmp
  \fi
}
%    \end{macrocode}

% \macro{\childdocof}
% The command |\childdocof| redirects
% compilation to the main file |#1|.
%    \begin{macrocode}
\newcommand{\childdocof}[1]
{
  \childdocdisable
  \childdoctrue
  \includeonly{\childdocname}
  \def\jobname{#1}
  \def\childdocjob{#1}
  \input{#1}
}
%    \end{macrocode}

% \macro{\childdocby}
% The command |\childdocby| ....
%    \begin{macrocode}
\newcommand{\childdocby}[2][]
{
  \childdocdisable
  \childdoctrue
  \childdocmanualtrue
  \if?#1?\else
    \def\jobname{#2}
  \fi
  \def\childdocjob{#2}
  \input{#2}
  \endinput
}
%    \end{macrocode}

% \macro{\childdocforward}
% The command |\childdocforward| redirects
% compilation to the main file or
% (if the optional argument is given) a child file.
% Parameters are set as if the main file
% or a child file starting with |\childdocof| was compiled.
% Then compilation is handed over to the main file:
%    \begin{macrocode}
\newcommand{\childdocforward}[2][]
{
  \begingroup
    \if?#1?
      \def\childdoctmp
      {
        \def\childdocname{#2}
        \def\childdocjob{#2}
        \def\jobname{#2}
        \input{#2}
        \endinput
      }
    \else
      \def\childdoctmp
      {
        \childdocdisable
        \def\childdocname{#2}
        \childdoctrue
        \includeonly{#2}
        \def\childdocjob{#1}
        \def\jobname{#1}
        \input{#1}
        \endinput
      }
    \fi
    \expandafter
  \endgroup
  \childdoctmp
}
%    \end{macrocode}

% \macro{\childdocforwardprefix}
% The command |\childdocforwardprefix| redirects
% compilation to the main or a child file by means of a pattern.
% The prefix |#1| in the current filename is replaced by |#2|
% and the suffix of the current filename is kept
% (it is assumed that the filename does not contain the substring `|~~~|'
% which is used as a delimiter).
% Compilation is handed over to the new file by |\childdocforward|:
%    \begin{macrocode}
\newcommand{\childdocforwardprefix}[3][]
{
  \begingroup
    \def\childdocextract #2##1~~~{\def\childdoctmp{\childdocforward[#1]{#3##1}}}
    \expandafter\childdocextract\childdocname~~~
    \expandafter
  \endgroup
  \childdoctmp
}
%    \end{macrocode}

% \macro{\childdoc}
% The deprecated macro |\childdoc| is a legacy version of |\childdocmain|:
%    \begin{macrocode}
\newcommand{\childdoc}{\childdocmain}
%    \end{macrocode}

% \macro{\childdocredirect}
% The deprecated macro |\childdocredirect| is a legacy version
% of |\childdocforward| and |\childdocforwardprefix|:
%    \begin{macrocode}
\newcommand{\childdocredirect}[2][]
{
  \begingroup
    \if?#1?
      \def\childdoctmp{\childdocforward{#2}}
    \else
      \def\childdoctmp{\childdocforwardprefix{#1}{#2}}
    \fi
    \expandafter
  \endgroup
  \childdoctmp
}
%    \end{macrocode}

%\iffalse
%</package>
%\fi
%
\endinput
|\\
|\childdocby{|\textit{main}|}|\\
\end{tabular}
\end{center}
%
The directive |\childdocby| is similar to |\childdocof|
described in \secref{sec:include},
but the subsequent selection of content must be done manually.
To that end, both |\ifchilddoc| and |\ifchilddocmanual|
will be true upon processing of a part,
and the name of the part is stored in |\childdocname|.
Note that |\jobname| will be set to the filename of the current part
so that each part receives an individual |.aux| file
that does not interfere with the |.aux| file(s) of the main document.
This behaviour can be altered by the alternative form
|\childdocby[*]{|\textit{main}|}| (with a non-empty optional argument)
which uses the |.aux| file of the main document
by setting |\jobname| to \textit{main}.

%%%%%%%%%%%%%%%%%%%%%%%%%%%%%%%%%%%%%%%%%%%%%%%%%%%%%%%%%%%%%%%%%%%%%%%%%%%%%%%%
\subsection{Driver Development}
\label{sec:driver}

The \textsf{childdoc} mechanism can also be use for the development
of definition files such as \LaTeX{} styles or classes.
This case differs from the above setup with multiple parts
included by |\include| in that no |\includeonly| should be invoked.
This can be achieved by starting the include file
(before |\ProvidesPackage|) with:
%
\begin{center}
\begin{tabular}{l}
|% \iffalse
%
% childdoc.dtx Copyright (C) 2017-2018 Niklas Beisert
%
% This work may be distributed and/or modified under the
% conditions of the LaTeX Project Public License, either version 1.3
% of this license or (at your option) any later version.
% The latest version of this license is in
%   http://www.latex-project.org/lppl.txt
% and version 1.3 or later is part of all distributions of LaTeX
% version 2005/12/01 or later.
%
% This work has the LPPL maintenance status `maintained'.
%
% The Current Maintainer of this work is Niklas Beisert.
%
% This work consists of the files childdoc.dtx and childdoc.ins
% and the derived files childdoc.def and cdocsamp.tex with
% cdocsch1.tex, cdocsch2.tex, cdocsdrf.tex, cdocsfn1.tex, cdocsfn2.tex.
%
%<package>\ifdefined\childdocmain\endinput\fi
%<package>\ProvidesFile{childdoc.def}[2018/12/30 v2.0 child document driver]
%<samplemain>\ProvidesFile{cdocsamp.tex}[2018/12/30 v2.0 sample for childdoc]
%<*driver>
%\ProvidesFile{childdoc.drv}[2018/12/30 v2.0 childdoc reference manual file]
\PassOptionsToClass{10pt,a4paper}{article}
\documentclass{ltxdoc}

\usepackage[margin=35mm]{geometry}
\usepackage{hyperref}
\usepackage{hyperxmp}
\usepackage[usenames]{color}

\hypersetup{colorlinks=true}
\hypersetup{pdfstartview=FitH}
\hypersetup{pdfpagemode=UseNone}
\hypersetup{pdfsource={}}
\hypersetup{pdflang={en-UK}}
\hypersetup{pdfcopyright={Copyright 2017-2018 Niklas Beisert.
  This work may be distributed and/or modified under the
  conditions of the LaTeX Project Public License, either version 1.3
  of this license or (at your option) any later version.}}
\hypersetup{pdflicenseurl={http://www.latex-project.org/lppl.txt}}
\hypersetup{pdfcontactaddress={ETH Zurich, ITP, HIT K,
  Wolfgang-Pauli-Strasse 27}}
\hypersetup{pdfcontactpostcode={8093}}
\hypersetup{pdfcontactcity={Zurich}}
\hypersetup{pdfcontactcountry={Switzerland}}
\hypersetup{pdfcontactemail={nbeisert@itp.phys.ethz.ch}}
\hypersetup{pdfcontacturl={http://people.phys.ethz.ch/\xmptilde nbeisert/}}

\newcommand{\secref}[1]{\hyperref[#1]{section \ref*{#1}}}

\parskip1ex
\parindent0pt
\let\olditemize\itemize
\def\itemize{\olditemize\parskip0pt}

\begin{document}

\title{The \textsf{childdoc} Package}
\hypersetup{pdftitle={The childdoc Package}}
\author{Niklas Beisert\\[2ex]
  Institut f\"ur Theoretische Physik\\
  Eidgen\"ossische Technische Hochschule Z\"urich\\
  Wolfgang-Pauli-Strasse 27, 8093 Z\"urich, Switzerland\\[1ex]
  \href{mailto:nbeisert@itp.phys.ethz.ch}
  {\texttt{nbeisert@itp.phys.ethz.ch}}}
\hypersetup{pdfauthor={Niklas Beisert}}
\hypersetup{pdfsubject={Manual for the LaTeX2e Package childdoc}}
\date{30 December 2018, \textsf{v2.0}}
\maketitle

\begin{abstract}\noindent
\textsf{childdoc} is a \LaTeXe{} package
that enables the direct compilation
of document sections included by |\include|
to individual files.
\end{abstract}

\begingroup
\parskip0ex
\tableofcontents
\endgroup

%%%%%%%%%%%%%%%%%%%%%%%%%%%%%%%%%%%%%%%%%%%%%%%%%%%%%%%%%%%%%%%%%%%%%%%%%%%%%%%%
%%%%%%%%%%%%%%%%%%%%%%%%%%%%%%%%%%%%%%%%%%%%%%%%%%%%%%%%%%%%%%%%%%%%%%%%%%%%%%%%
\section{Introduction}

\LaTeX{} provides a mechanism to structure a large document (such as a book)
into a main file and several child files (containing the chapters)
using the |\include| command.
This mechanism is beneficial for documents
which span hundreds of pages in order to
make the source file(s) more manageable.
Moreover, compilation can be restricted to
selected child files by means of the |\includeonly| command.
The latter feature can be used to reduce the compilation time while editing
(this was significantly more useful in the earlier days of \LaTeX{})
or to generate a smaller document which is easier to navigate.
Another application of |\includeonly| is to generate
documents consisting of selected parts of the complete document.

However, there are a few drawbacks of the plain |\include| mechanism:
\begin{itemize}
\item
The child files cannot be compiled on their own,
they can only be compiled via the main file.
A naive editing environment
(such as a text editor with an option
to have the current file processed by \LaTeX)
may require one to switch to the main file before compiling;
attempting to compile the child file produces errors.
\item
The main file must be modified (each time)
to adjust the |\includeonly| command
to the present needs. This easily leaves the main file in a messy state.
\item
The generated document will always carry the filename
of the main document. This is inconvenient if
several child files are to be compiled and
to be kept for distribution.
\end{itemize}

The present package provides a simple interface
to make child files individually compilable by \LaTeX{}.
Compiling a child file then has the same effect as compiling
the main file with an |\includeonly| command
to select the appropriate child.
Moreover the generated document will carry the name of the child
rather than the main file.
This resolves all three above issues.

This feature is meant to make the editing of books,
thesis documents and lecture notes somewhat more convenient.
However, the package can also be used efficiently for
composing a series of documents (such as exercise sheets)
which are typically distributed individually.
It then assists the author in generating the individual documents
(potentially in different versions)
as well as a document containing the collected series.
Another application is in developing style files
or other kinds of included material
where compilation of the style file could redirect
to a sample or test file.

%%%%%%%%%%%%%%%%%%%%%%%%%%%%%%%%%%%%%%%%%%%%%%%%%%%%%%%%%%%%%%%%%%%%%%%%%%%%%%%%
%%%%%%%%%%%%%%%%%%%%%%%%%%%%%%%%%%%%%%%%%%%%%%%%%%%%%%%%%%%%%%%%%%%%%%%%%%%%%%%%
\section{Usage}

First of all, the package \textsf{childdoc} is \emph{not} a standard
\LaTeXe{} |.sty| style file! Therefore it needs to be invoked in
a non-standard way.

%%%%%%%%%%%%%%%%%%%%%%%%%%%%%%%%%%%%%%%%%%%%%%%%%%%%%%%%%%%%%%%%%%%%%%%%%%%%%%%%
\subsection{Included Files}
\label{sec:include}

%%%%%%%%%%%%%%%%%%%%%%%%%%%%%%%%%%%%%%%%
\DescribeMacro{\childdocmain}
To use the package, add the commands
\begin{center}
\begin{tabular}{l}
|\input{childdoc.def}|\\
|\childdocmain{}|\\
\end{tabular}
\end{center}
at the very top of the main \LaTeX{} file,
in particular \emph{before} the |\documentclass| statement!
The argument of |\childdocmain| should be left empty
(but it must be present).

%%%%%%%%%%%%%%%%%%%%%%%%%%%%%%%%%%%%%%%%
\DescribeMacro{\childdocof}
Furthermore, add the commands
\begin{center}
\begin{tabular}{l}
|\input{childdoc.def}|\\
|\childdocof{|\textit{main}|}|\\
\end{tabular}
\end{center}
at the top of every child file \textit{child}
which is included by |\include{|\textit{child}|}|
from within the main file
(or at least for those files to be compiled individually).
The argument \textit{main} must be the filename of the main file.

There are a couple of
considerations in setting up the main and child documents:

%%%%%%%%%%%%%%%%%%%%%%%%%%%%%%%%%%%%%%%%
\paragraph{Restrictions.}

Please note the following restrictions:
\begin{itemize}
\item
|\childdocmain| must be called with one argument \textit{main}
to ensure compatibility with earlier version of the package.
It must either be empty (|\childdocmain{}|)
or precisely match the filename of the main file in which it is specified.
See \secref{sec:detection} for further information.
\item
The filename \textit{main} must be specified without the |.tex| extension.
\item
The filename \textit{main} is case sensitive
(even in case-insensitive file systems)
due to internal string comparison.
\item
The argument \textit{main} should be fully expanded, it cannot be a macro.
\item
Subdirectories and special characters should be avoided in filenames.
\item
The command |\childdocmain{|\textit{main}|}| must be followed by a whitespace.
It should not be followed immediately by another command
or by a comment mark `|%|'.
This is because the \TeX{} parser reads the token immediately following
the argument of |\childdocmain| and puts it
at the beginning of every child section;
however, a white\-space is ignored.
\end{itemize}

%%%%%%%%%%%%%%%%%%%%%%%%%%%%%%%%%%%%%%%%
\paragraph{Content of Main File.}

It is advisable to place all content in the child files included by |\include|.
Any output contained in the main file will appear in all child documents
unless suppressed manually;
it cannot be suppressed automatically by the |\includeonly| directive
and thus should normally be avoided.
A method to include some content in the main file
by means of conditional processing is described in \secref{sec:conditional}.

%%%%%%%%%%%%%%%%%%%%%%%%%%%%%%%%%%%%%%%%
\paragraph{Page Numbering.}

When only a part of the document is compiled,
the appropriate numbering of pages
(as well as other status parameters)
is determined from the |.aux| files.
The latter contain information from previous passes.
However this information needs to propagate through
all intermediate child documents.
Therefore the page numbering in child documents may well
be inconsistent until the complete document is compiled at least once.

A useful (if unconventional) way to always ensure a consistent
page numbering is to restart the numbering in each child document
and denote the pages by `\textit{child}|.|\textit{page}'
where \textit{child} represents the chapter/section number of the child file.
This can be achieved by the command
|\numberwithin{page}{|\textit{child}|}|
of the \textsf{amsmath} package
where \textit{child} can be |chapter| or |section|
depending on the chosen structuring.
Alternatively, one can modify the macro |\thepage| appropriately
and reset the counter |page| at the start of each child file.

%%%%%%%%%%%%%%%%%%%%%%%%%%%%%%%%%%%%%%%%%%%%%%%%%%%%%%%%%%%%%%%%%%%%%%%%%%%%%%%%
\subsection{Conditional Processing}
\label{sec:conditional}

The package provides a mechanism to compile different versions
of a document. To customise the versions further some conditional processing
can come in handy to distinguish which version is being compiled.
The package provides two macros to describe the compilation context:

%%%%%%%%%%%%%%%%%%%%%%%%%%%%%%%%%%%%%%%%
\DescribeMacro{\ifchilddoc}
The conditional |\ifchilddoc| distinguishes between the compilation of
child documents and the main document:
%
\begin{center}
|\ifchilddoc |\textit{child-code}| |[|\||else |\textit{main-code}]| \||fi|
\end{center}

%%%%%%%%%%%%%%%%%%%%%%%%%%%%%%%%%%%%%%%%
\DescribeMacro{\childdocname}
\DescribeMacro{\childdocjob}
The macro |\childdocname| contains the filename (without extension)
of the main or child file being processed.
Note that |\childdocjob| will always contain the name of the main file.

%%%%%%%%%%%%%%%%%%%%%%%%%%%%%%%%%%%%%%%%
\paragraph{Title Page.}

Conditional processing can be used to include a title or banner page
in the main document when proper precautions are taken.
Importantly, the code in the main file should ensure that the page counter
(as well as other status parameters which are stored in the |.aux| files)
takes the same value after the conditional processing.
Otherwise the page numbers may take divergent values
depending on which part is compiled.

For example, a title page could be declared by:
%
\begin{center}
\begin{tabular}{l}
|\ifchilddoc\||else|\\
|\addtocounter{page}{-1}|\\
\textit{code for title page}\\
|\newpage|\\
|\||fi|
\end{tabular}
\end{center}
%
A banner page for the child documents can be generated by:
%
\begin{center}
\begin{tabular}{l}
|\ifchilddoc|\\
|\addtocounter{page}{-1}|\\
\textit{code for banner page}\\
|\newpage|\\
|\||fi|
\end{tabular}
\end{center}
%
Here one could write a message such as:
\begin{center}
|This is the part \childdocname{} of \childdocjob{}.|
\end{center}

%%%%%%%%%%%%%%%%%%%%%%%%%%%%%%%%%%%%%%%%%%%%%%%%%%%%%%%%%%%%%%%%%%%%%%%%%%%%%%%%
\subsection{Flags}
\label{sec:flags}

The package makes it easy to generate different versions
of the main or child documents.
To this end compilation flags can be defined
and assigned different default values.
They will be particularly useful in conjunction
with the forwarding mechanism described in \secref{sec:forward}.

For example, it may be useful to have a flag |\version|
which can be set to |draft| or |final|.
The document source will contain some conditional code
depending on the value of |\version|.
Suppose further, the flag should default to |final| for the main file
and to |draft| for child files
which is a natural assignment for editing the document.
This is achieved by placing the following code
in the preamble of the main document
(below the |\childdocmain| directive):
%
\begin{center}
\begin{tabular}{l}
|\ifchilddoc|\\
|\providecommand{\version}{draft}|\\
|\||else|\\
|\providecommand{\version}{final}|\\
|\||fi|
\end{tabular}
\end{center}
%
The definition by |\providecommand| makes sure
that previous definitions are not overwritten.
Further statements |\providecommand{\version}{...}|
can thus be added before the above code to override it.

For the main file, one might add a line
(between |\childdocmain| and the above block)
%
\begin{center}
|%\ifchilddoc\||else\providecommand{\version}{draft}\||fi|
\end{center}
%
which can be uncommented to produce a draft version.
Likewise one can add a line to the very top of a child file
(above the |\childdocof{|\textit{main}|}| directive)
%
\begin{center}
|%\providecommand{\version}{final}|
\end{center}
%
which can be uncommented to produce the final version of this child document.

%%%%%%%%%%%%%%%%%%%%%%%%%%%%%%%%%%%%%%%%%%%%%%%%%%%%%%%%%%%%%%%%%%%%%%%%%%%%%%%%
\subsection{Forwarding}
\label{sec:forward}

Different versions of the main or child documents
using compilation flags as described in \secref{sec:flags}
can be (permanently) stored in different files
for convenient compilation, viewing and distribution.
To this end, the package defines a command
to pass on compilation to a different file:

%%%%%%%%%%%%%%%%%%%%%%%%%%%%%%%%%%%%%%%%
\DescribeMacro{\childdocforward}
The command |\childdocforward| redirects processing to
another source file:
%
\begin{center}
\begin{tabular}{l}
|\input{childdoc.def}|\\
|\childdocforward[|\textit{main}|]{|\textit{dest}|}|\\
\end{tabular}
\end{center}
%
The argument \textit{dest} is the destination file
(without extension).
It should be the main file or one of the child files.
Note that further \textsf{childdoc} directives
such as |\childdocof| and |\childdocforward|
in the indicated file will be processed in this form.
The optional argument \textit{main}
passes on directly to the main file \textit{main}
while pretending to compile the child \textit{dest}.
This form behaves as if \textit{dest}
issues |\childdocof{|\textit{main}|}| right away,
and no further \textsf{childdoc} directives will be processed.

%%%%%%%%%%%%%%%%%%%%%%%%%%%%%%%%%%%%%%%%
\DescribeMacro{\...prefix}
In the alternative form |\childdocforwardprefix|,
%
\begin{center}
\begin{tabular}{l}
|\input{childdoc.def}|\\
|\childdocforwardprefix[|\textit{main}|]{|\textit{prefix}|}{|\textit{dest}|}|
\end{tabular}
\end{center}
%
the destination file is determined by a pattern
depending on the current file:
To make this work, the current file must be called
`{\textit{prefix}\hspace{0.2em}\textit{suffix}}'
with \textit{prefix} matching precisely the argument.
Processing is then passed on to the file
`{\textit{dest}\hspace{0.2em}\textit{suffix}}'.
Surely, the same effect is achieved by
directly specifying the
argument `{\textit{dest}\hspace{0.2em}\textit{suffix}}'
in the first form.
However, that requires to set up a different file
for each child. With the alternative form of the command
all these files can have exactly the same content
which simplifies setting them up and maintaining them.

For example, the following file |draft.tex|
with a compilation flag |\version| as described in \secref{sec:flags}
compiles the main document as a draft:
%
\begin{center}
\begin{tabular}{l}
|\def\version{draft}|\\
|\input{childdoc.def}|\\
|\childdocforward{|\textit{main}|}|
\end{tabular}
\end{center}
%
Likewise, the following files |final|\textit{nn}|.tex|
compile the final version of the child document
|child|\textit{nn}|.tex|:
%
\begin{center}
\begin{tabular}{l}
|\def\version{final}|\\
|\input{childdoc.def}|\\
|\childdocforwardprefix{final}{child}|
\end{tabular}
\end{center}
%

Note that when several versions of a main file and/or of each child file
are to be generated, it may be convenient to set up a |Makefile| or
shell script to automatise the process.

%%%%%%%%%%%%%%%%%%%%%%%%%%%%%%%%%%%%%%%%%%%%%%%%%%%%%%%%%%%%%%%%%%%%%%%%%%%%%%%%
\subsection{Command Line Processing}
\label{sec:commandline}

The effect of redirection files can also be achieved by invoking
the \LaTeX{} compiler with a more elaborate command line.
Most conveniently this should be done as part
of a shell script or a |Makefile|.

When using \textsf{childdoc} in the main file, the following
command lines effectively perform a redirection
(note that depending on the shell being used,
backslashes may have to be doubled: `|\|' $\to$ `|\\|'):
%
\begin{center}
|... -jobname "|\textit{target}|" |\\|"|[\textit{flags}]%
|\input{childdoc.def}\childdocforward[|\textit{main}|]{|\textit{dest}|}"|
\end{center}
%
Here \textit{target} is the name of the output file,
\textit{main} is the name of the main file
and \textit{dest} is the name of the main or child file to be processed
(all filenames without extensions).
The optional argument \textit{main} can be omitted
if \textit{main} matches \textit{dest}.
Optionally, compilation \textit{flags} can be defined via |\def| commands.
This command line makes the \TeX{} engine believe
it is compiling the file \textit{target}
whose content is specified as the latter parameter.
The provided code then forwards the processing to
\textit{main} or \textit{dest} as described in \secref{sec:forward}.

%%%%%%%%%%%%%%%%%%%%%%%%%%%%%%%%%%%%%%%%%%%%%%%%%%%%%%%%%%%%%%%%%%%%%%%%%%%%%%%%
\subsection{Include by Input}
\label{sec:input}

Including child documents by |\include| has some restrictions by design.
Most notably, the content of a child document always occupies
its own set of pages; pages cannot be shared between child documents.
Usually, this behaviour makes perfect sense
because each child document contain an essential part of the document.
However, in some situations it may be desirable to compose
a document from a collection of parts
without having mandatory page breaks between then.
For this case, the package
provides a mechanism to include parts
by |\input| which can also be processed individually.
However, by construction this mechanism
requires manual handling of the content to be output.

%%%%%%%%%%%%%%%%%%%%%%%%%%%%%%%%%%%%%%%%
\DescribeMacro{\ifchilddocmanual}
The main file should be prepared as usual, see \secref{sec:include}.
However, the document body must make a distinction
between processing of an individual part and of the main document, e.g.:
%
\begin{center}
\begin{tabular}{l}
|\ifchilddocmanual|\\
|\input{\childdocname}|\\
|\||else|\\
\textit{document body with }|\input{|\textit{part}|}|\\
|\||fi|
\end{tabular}
\end{center}
%
The conditional |\ifchilddocmanual| is true whenever
a part to be included by |\input| is being compiled,
and the name of the part is stored in |\childdocname|.

%%%%%%%%%%%%%%%%%%%%%%%%%%%%%%%%%%%%%%%%
\DescribeMacro{\childdocby}
Each part to be included by |\input| should start with:
%
\begin{center}
\begin{tabular}{l}
|\input{childdoc.def}|\\
|\childdocby{|\textit{main}|}|\\
\end{tabular}
\end{center}
%
The directive |\childdocby| is similar to |\childdocof|
described in \secref{sec:include},
but the subsequent selection of content must be done manually.
To that end, both |\ifchilddoc| and |\ifchilddocmanual|
will be true upon processing of a part,
and the name of the part is stored in |\childdocname|.
Note that |\jobname| will be set to the filename of the current part
so that each part receives an individual |.aux| file
that does not interfere with the |.aux| file(s) of the main document.
This behaviour can be altered by the alternative form
|\childdocby[*]{|\textit{main}|}| (with a non-empty optional argument)
which uses the |.aux| file of the main document
by setting |\jobname| to \textit{main}.

%%%%%%%%%%%%%%%%%%%%%%%%%%%%%%%%%%%%%%%%%%%%%%%%%%%%%%%%%%%%%%%%%%%%%%%%%%%%%%%%
\subsection{Driver Development}
\label{sec:driver}

The \textsf{childdoc} mechanism can also be use for the development
of definition files such as \LaTeX{} styles or classes.
This case differs from the above setup with multiple parts
included by |\include| in that no |\includeonly| should be invoked.
This can be achieved by starting the include file
(before |\ProvidesPackage|) with:
%
\begin{center}
\begin{tabular}{l}
|\input{childdoc.def}|\\
|\childdocforward{|\textit{main}|}|\\
\end{tabular}
\end{center}
%
or alternatively with:
%
\begin{center}
\begin{tabular}{l}
|\input{childdoc.def}|\\
|\childdocby{|\textit{main}|}|\\
\end{tabular}
\end{center}
%
Both forms have slightly different effects as described above.
The main file is prepared as usual, see \secref{sec:include}.

%%%%%%%%%%%%%%%%%%%%%%%%%%%%%%%%%%%%%%%%%%%%%%%%%%%%%%%%%%%%%%%%%%%%%%%%%%%%%%%%
\subsection{Legacy Detection}
\label{sec:detection}

The directive |\childdocmain| in the main file can detect
whether the complete document or merely a child is to be compiled
even without using the directive |\childdocof|.
This method is deprecated because it is less robust
and there is no compelling reason to use it;
it is merely provided for backward compatibility
and it may be removed in future versions.

If the detection mechanism is to be used,
it is mandatory to correctly specify
the filename of the main file as the argument of |\childdocmain|:
%
\begin{center}
\begin{tabular}{l}
|\input{childdoc.def}|\\
|\childdocmain{|\textit{main}|}|\\
\end{tabular}
\end{center}
%
If |\jobname| does not match the argument \textit{main} of |\childdocmain|,
it is assumed that |\jobname| points to the child file to be compiled.
When using |\childdocmain| with the main file specified as argument,
it suffices to start a child file
with just |\input{|\textit{main}|}|
without loading of the package and using |\childdocof|.
If instead all processing is done
with the appropriate \textsf{childdoc} directives,
the argument of \textit{main} of |\childdocmain| can be empty.

An alternative version of the command line processing described
in \secref{sec:commandline} using the detection mechanism reads:
%
\begin{center}
|... -jobname "|\textit{target}|" "|[\textit{flags}]%
[|\def\jobname{|\textit{dest}|}|]|\input{|\textit{main}|}"|
\end{center}

%%%%%%%%%%%%%%%%%%%%%%%%%%%%%%%%%%%%%%%%%%%%%%%%%%%%%%%%%%%%%%%%%%%%%%%%%%%%%%%%
\subsection{Manual Code}
\label{sec:manual}

In case one cannot be certain whether the definitions file |childdoc.def|
is installed on the target \TeX{} distribution
and one prefers not to ship it,
it is conceivable to paste a few relevant commands into the sources.

To that end, drop all statements |\input{childdoc.def}|
and perform the replacements as outlined below.
Instead of |\childdocmain{|\textit{main}|}| add the following code
to the top of the main file:
%
\begin{center}
\begin{tabular}{l}
|\||ifdefined\childdocname\endinput\||fi\newif\ifchilddoc|\\
|\edef\childdocname{\scantokens\expandafter{\jobname\noexpand}}|\\
|\def\childdocmain{|\textit{main}|}\||ifx\childdocmain\childdocname\||else|\\
|\childdoctrue\includeonly{\childdocname}\let\jobname\childdocmain\||fi|\\
\end{tabular}
\end{center}
%
Instead of |\childdocof{|\textit{main}|}| just include the main file
at the top of each child file:
%
\begin{center}
|\input{|\textit{main}|}|
\end{center}
%
A simple redirection |\childdocforward{|\textit{dest}|}| is achieved by:
%
\begin{center}
|\def\jobname{|\textit{dest}|}\input{\jobname}|
\end{center}
%
The redirection with prefix
|\childdocforwardprefix[|\textit{prefix}|]{|\textit{dest}|}|
is accomplished by:
%
\begin{center}
\begin{tabular}{l}
|{\edef\jobname{\scantokens\expandafter{\jobname\noexpand}}|\\
|\def\redirectjob |\textit{prefix}|#1~~~{\gdef\jobname{|\textit{dest}|#1}}|\\
|\expandafter\redirectjob\jobname~~~}\input{\jobname}|
\end{tabular}
\end{center}

In an alternative approach,
child documents can be compiled by a specific command line
without additional code or specific definitions:
%
\begin{center}
|... -jobname "|\textit{target}|" "|[\textit{flags}]%
|\includeonly{|\textit{dest}|}\input{|\textit{main}|}"|
\end{center}
%

%%%%%%%%%%%%%%%%%%%%%%%%%%%%%%%%%%%%%%%%%%%%%%%%%%%%%%%%%%%%%%%%%%%%%%%%%%%%%%%%
%%%%%%%%%%%%%%%%%%%%%%%%%%%%%%%%%%%%%%%%%%%%%%%%%%%%%%%%%%%%%%%%%%%%%%%%%%%%%%%%
\section{Information}

%%%%%%%%%%%%%%%%%%%%%%%%%%%%%%%%%%%%%%%%%%%%%%%%%%%%%%%%%%%%%%%%%%%%%%%%%%%%%%%%
\subsection{Copyright}

Copyright \copyright{} 2017--2018 Niklas Beisert

This work may be distributed and/or modified under the
conditions of the \LaTeX{} Project Public License, either version 1.3
of this license or (at your option) any later version.
The latest version of this license is in
  \url{http://www.latex-project.org/lppl.txt}
and version 1.3 or later is part of all distributions of \LaTeX{}
version 2005/12/01 or later.

This work has the LPPL maintenance status `maintained'.

The Current Maintainer of this work is Niklas Beisert.

This work consists of the files |README.txt|, |childdoc.ins| and |childdoc.dtx|
as well as the derived files |childdoc.def|, |cdocsamp.tex|
with |cdocsch1.tex|, |cdocsch2.tex|, |cdocspt3.tex|, |cdocspt4.tex|,
|cdocsdrf.tex|, |cdocsfn1.tex|, |cdocsfn2.tex|
as well as |childdoc.pdf|.

%%%%%%%%%%%%%%%%%%%%%%%%%%%%%%%%%%%%%%%%%%%%%%%%%%%%%%%%%%%%%%%%%%%%%%%%%%%%%%%%
\subsection{Files and Installation}

The package consists of the files:
%
\begin{center}
\begin{tabular}{ll}
    |README.txt|   & readme file \\
    |childdoc.ins| & installation file \\
    |childdoc.dtx| & source file \\
    |childdoc.def| & definition file \\
    |cdocsamp.tex| & sample main file \\
    |cdocsch1.tex| & sample include file \\
    |cdocsch2.tex| & sample include file \\
    |cdocspt3.tex| & sample part file \\
    |cdocspt4.tex| & sample part file \\
    |cdocsdrf.tex| & sample redirection file \\
    |cdocsfn1.tex| & sample redirection file \\
    |cdocsfn2.tex| & sample redirection file \\
    |childdoc.pdf| & manual
\end{tabular}
\end{center}
%
The distribution consists of the files
|README.txt|, |childdoc.ins| and |childdoc.dtx|.
%
\begin{itemize}
\item
Run (pdf)\LaTeX{} on |childdoc.dtx|
to compile the manual |childdoc.pdf| (this file).
\item
Run \LaTeX{} on |childdoc.ins| to create the definitions file |childdoc.def|
and the sample |cdocsamp.tex| with include files
|cdocsch1.tex|, |cdocsch2.tex|, |cdocspt3.tex|, |cdocspt4.tex|,
|cdocsdrf.tex|, |cdocsfn1.tex|, |cdocsfn2.tex|.
Then copy the file |childdoc.def| to an appropriate directory of your \LaTeX{}
distribution, e.g.\ \textit{texmf-root}|/tex/latex/childdoc|.
\end{itemize}

%%%%%%%%%%%%%%%%%%%%%%%%%%%%%%%%%%%%%%%%%%%%%%%%%%%%%%%%%%%%%%%%%%%%%%%%%%%%%%%%
\subsection{Related CTAN Packages}

There are several other packages which offer a similar functionality:
%
\begin{itemize}
\item
The packages
\href{http://ctan.org/pkg/docmute}{\textsf{docmute}},
\href{http://ctan.org/pkg/includex}{\textsf{includex}} and
\href{http://ctan.org/pkg/standalone}{\textsf{standalone}}
provide commands to include only the document body of
a child file thus allowing both files to be compiled individually.
\item
The packages \href{http://ctan.org/pkg/subdocs}{\textsf{subdocs}}
and \href{http://ctan.org/pkg/subfiles}{\textsf{subfiles}}
provide structures in which the main and child documents can be
encapsulated and allowing them to be compiled individually.
The inclusion mechanism is different from the conventional |\include|.
\item
The package \href{http://ctan.org/pkg/combine}{\textsf{combine}}
is an elaborate solution to combine several documents into one.
\end{itemize}
%
See also the CTAN topic \href{http://ctan.org/topic/subdocs}{\textsf{subdocs}}
for further related packages.
The present package differs from the above solutions in that
a document structure constructed with the conventional |\include| mechanism
just needs two extra commands at the top of every file
such that all constituent files can be compiled individually.

%%%%%%%%%%%%%%%%%%%%%%%%%%%%%%%%%%%%%%%%%%%%%%%%%%%%%%%%%%%%%%%%%%%%%%%%%%%%%%%%
%\subsection{Feature Suggestions}
%
%The following is a list of features which may be useful for future
%versions of this package:
%%
%\begin{itemize}
%\item
%\ldots
%\end{itemize}

%%%%%%%%%%%%%%%%%%%%%%%%%%%%%%%%%%%%%%%%%%%%%%%%%%%%%%%%%%%%%%%%%%%%%%%%%%%%%%%%
\subsection{Revision History}

%%%%%%%%%%%%%%%%%%%%%%%%%%%%%%%%%%%%%%%%
\paragraph{v2.0:} 2018/12/30

\begin{itemize}
\item
immediate forward processing
\item
added |\childdocby| mechanism
\item
manual restructured
\end{itemize}

%%%%%%%%%%%%%%%%%%%%%%%%%%%%%%%%%%%%%%%%
\paragraph{v1.6:} 2018/01/17

\begin{itemize}
\item
application for development of include files
\item
corrections to manual
\end{itemize}

%%%%%%%%%%%%%%%%%%%%%%%%%%%%%%%%%%%%%%%%
\paragraph{v1.5:} 2017/05/21

\begin{itemize}
\item
more complete structuring introduced
\item
|\childdocof| introduced
\item
|\childdoc| renamed to |\childdocmain|
\item
|\childredirect| renamed to |\childdocforward| and |\childdocforwardprefix|
and functionality expanded
\end{itemize}

%%%%%%%%%%%%%%%%%%%%%%%%%%%%%%%%%%%%%%%%
\paragraph{v1.0:} 2017/04/27

\begin{itemize}
\item
manual and install package
\item
first version published on CTAN
\end{itemize}

%%%%%%%%%%%%%%%%%%%%%%%%%%%%%%%%%%%%%%%%
\paragraph{v0.6:} 2017/04/26

\begin{itemize}
\item
redirection mechanism added
\end{itemize}

%%%%%%%%%%%%%%%%%%%%%%%%%%%%%%%%%%%%%%%%
\paragraph{v0.5:} 2017/04/26

\begin{itemize}
\item
functionality in definition file
\end{itemize}


%%%%%%%%%%%%%%%%%%%%%%%%%%%%%%%%%%%%%%%%%%%%%%%%%%%%%%%%%%%%%%%%%%%%%%%%%%%%%%%%
%%%%%%%%%%%%%%%%%%%%%%%%%%%%%%%%%%%%%%%%%%%%%%%%%%%%%%%%%%%%%%%%%%%%%%%%%%%%%%%%
%%%%%%%%%%%%%%%%%%%%%%%%%%%%%%%%%%%%%%%%%%%%%%%%%%%%%%%%%%%%%%%%%%%%%%%%%%%%%%%%
\appendix

\settowidth\MacroIndent{\rmfamily\scriptsize 000\ }

 \DocInput{childdoc.dtx}

\end{document}
%</driver>
% \fi
%
% %%%%%%%%%%%%%%%%%%%%%%%%%%%%%%%%%%%%%%%%%%%%%%%%%%%%%%%%%%%%%%%%%%%%%%%%%%%%%%
% %%%%%%%%%%%%%%%%%%%%%%%%%%%%%%%%%%%%%%%%%%%%%%%%%%%%%%%%%%%%%%%%%%%%%%%%%%%%%%
% \section{Sample}
%\iffalse
%<*samplemain>
%\fi
%
% The following presents a sample document
% with two chapters, two parts, a title page,
% a compile flag as well as three forwarding files to set the flag.
% It consists of eight |.tex| files:
% \begin{center}
% \begin{tabular}{ll}
% |cdocsamp.tex|&main file\\
% |cdocsch1.tex|&include file for chapter 1\\
% |cdocsch2.tex|&include file for chapter 2\\
% |cdocspt3.tex|&include file for part 3\\
% |cdocspt4.tex|&include file for part 4\\
% |cdocsdrf.tex|&forwarding file for main file in draft mode\\
% |cdocsfi1.tex|&forwarding file for final version of chapter 1\\
% |cdocsfi2.tex|&forwarding file for final version of chapter 2\\
% \end{tabular}
% \end{center}
% Each of the eight files can be compiled directly by the \LaTeX{} compiler.
%
% %%%%%%%%%%%%%%%%%%%%%%%%%%%%%%%%%%%%%%
% \paragraph{Main File.}
%
% The main file is called |cdocsamp.tex|.
%
% Load the \textsf{childdoc} definitions and
% declare the filename for the main document:
%    \begin{macrocode}
\input{childdoc.def}
\childdocmain{}
%    \end{macrocode}

% Optional override for |\version| flag:
%    \begin{macrocode}
%%\ifchilddoc\else\providecommand{\version}{draft}\fi
%    \end{macrocode}

% Define the default values for the |\version| flag
% (|final| for the main file and |draft| for childs):
%    \begin{macrocode}
\ifchilddoc
\providecommand{\version}{draft}
\else
\providecommand{\version}{final}
\fi
%    \end{macrocode}

% Load the standard document class:
%    \begin{macrocode}
\documentclass[12pt]{article}
%    \end{macrocode}

% Start the document body:
%    \begin{macrocode}
\begin{document}
%    \end{macrocode}

% Declare a title page.
% Print title, part of document being processed and version flag:
%    \begin{macrocode}
\addtocounter{page}{-1}
\begin{center}
{\LARGE\bfseries{}childdoc example\par}
\vspace{1cm}
\ifchilddoc
\ifchilddocmanual part\else chapter\fi:
`\childdocname' of `\childdocjob'\par
\else
main document: `\childdocjob'\par
\fi
version: \version\par
\end{center}
\newpage
%    \end{macrocode}

% Manually include selected file,
% otherwise process as usual:
%    \begin{macrocode}
\ifchilddocmanual
\section*{part `\childdocname'}
\input{\childdocname}
\else
%    \end{macrocode}

% Include the two chapters:
%    \begin{macrocode}
\include{cdocsch1}
\include{cdocsch2}
%    \end{macrocode}

% Include the two parts unless only chapters should be displayed:
%    \begin{macrocode}
\ifchilddoc\else
\section{part three}
\input{cdocspt3}
\section{part four}
\input{cdocspt4}
\fi
%    \end{macrocode}

% Process as usual until here:
%    \begin{macrocode}
\fi
%    \end{macrocode}

% End of document body:
%    \begin{macrocode}
\end{document}
%    \end{macrocode}
%\iffalse
%</samplemain>
%\fi
%
% %%%%%%%%%%%%%%%%%%%%%%%%%%%%%%%%%%%%%%
% \paragraph{Chapter Include Files.}
%
% The include files are called |cdocsch1.tex| and |cdocsch2.tex|.
%
%\iffalse
%<*samplechap1|samplechap2>
%\fi

% Optional override for |\version| flag:
%    \begin{macrocode}
%%\providecommand{\version}{final}
%    \end{macrocode}

% Include the main document:
%    \begin{macrocode}
\input{childdoc.def}
\childdocof{cdocsamp}
%    \end{macrocode}

%\iffalse
%</samplechap1|samplechap2>
%\fi
%
%\iffalse
%<*samplechap1>
%\fi
% Some text for chapter 1:
%    \begin{macrocode}
\section{one}
some text in chapter one
%    \end{macrocode}

%\iffalse
%</samplechap1>
%\fi
% Some text for chapter 2:
%\iffalse
%<*samplechap2>
%\fi
%    \begin{macrocode}
\section{two}
more text in chapter two
%    \end{macrocode}

%\iffalse
%</samplechap2>
%\fi
%
% %%%%%%%%%%%%%%%%%%%%%%%%%%%%%%%%%%%%%%
% \paragraph{Part Include Files.}
%
% The include files are called |cdocspt3.tex| and |cdocspt4.tex|.
%
%\iffalse
%<*samplepart3|samplepart4>
%\fi

% Optional override for |\version| flag:
%    \begin{macrocode}
%%\providecommand{\version}{final}
%    \end{macrocode}

% Include the main document:
%    \begin{macrocode}
\input{childdoc.def}
\childdocby{cdocsamp}
%    \end{macrocode}

%\iffalse
%</samplepart3|samplepart4>
%\fi
%
%\iffalse
%<*samplepart3>
%\fi
% Some text for part 3:
%    \begin{macrocode}
some text in part three
%    \end{macrocode}

%\iffalse
%</samplepart3>
%\fi
% Some text for part 4:
%\iffalse
%<*samplepart4>
%\fi
%    \begin{macrocode}
more text in part four
%    \end{macrocode}

%\iffalse
%</samplepart4>
%\fi
%
% %%%%%%%%%%%%%%%%%%%%%%%%%%%%%%%%%%%%%%
% \paragraph{Forwarding for a Complete Draft.}
%
% The following forwarding file |cdocsdrf.tex|
% compiles the main document in draft mode:
%\iffalse
%<*sampledraft>
%\fi
%    \begin{macrocode}
\def\version{draft}
\input{childdoc.def}
\childdocforward{cdocsamp}
%    \end{macrocode}

%\iffalse
%</sampledraft>
%\fi
%
% %%%%%%%%%%%%%%%%%%%%%%%%%%%%%%%%%%%%%%
% \paragraph{Forwarding for Final Version of the Chapters.}
%
% The following forwarding files |cdocsfn1.tex| and |cdocsfn2.tex|
% (with identical content)
% compile the final versions of the child documents
% |cdocsch1.tex| and |cdocsch2.tex|, respectively:
%\iffalse
%<*samplefinal>
%\fi
%    \begin{macrocode}
\def\version{final}
\input{childdoc.def}
\childdocforwardprefix[cdocsamp]{cdocsfn}{cdocsch}
%    \end{macrocode}

%\iffalse
%</samplefinal>
%\fi
%
% %%%%%%%%%%%%%%%%%%%%%%%%%%%%%%%%%%%%%%
% \paragraph{Command Line Processing.}
%
% The following three command lines generate the output files
% |cdocscld|, |cdocscl1| and |cdocscl2|
% which should be identical to
% |cdocsdrf|, |cdocsch1| and |cdocsfn2|, respectively:
% \begin{center}
% \begin{tabular}{l}
% |latex -jobname cdocscld \|\\
% |  "\def\version{draft}\input{childdoc.def}\childdocforward{cdocsamp}"|\\
% |latex -jobname cdocscl1 \|\\
% |  "\input{childdoc.def}\childdocforward[cdocsamp]{cdocsch1}"|\\
% |latex -jobname cdocscl2 \|\\
% |  "\def\version{final}\input{childdoc.def}\childdocforward{cdocsch2}"|
% \end{tabular}
% \end{center}
% Note that the trailing backslash on each first line
% merely continues the input to the second line
% (for convenient cut ant paste).
% Furthermore, the command |latex| can be replaced by any
% of its alternative versions such as |pdflatex|.
%
% %%%%%%%%%%%%%%%%%%%%%%%%%%%%%%%%%%%%%%%%%%%%%%%%%%%%%%%%%%%%%%%%%%%%%%%%%%%%%%
% %%%%%%%%%%%%%%%%%%%%%%%%%%%%%%%%%%%%%%%%%%%%%%%%%%%%%%%%%%%%%%%%%%%%%%%%%%%%%%
% \section{Implementation}
%\iffalse
%<*package>
%\fi
%
% This section describes the definitions file |childdoc.def|.

% The definitions cannot be loaded using |\usepackage| or |\RequirePackage|
% which has a mechanism to prevent loading a style file more than once.
% When loading the definitions by means of |\input|
% multiple instances have to be prevented manually:
%\iffalse
%This code needs to be before the `\ProvidesFile' directive
%which is defined at the beginning of this file.
%Therefore it is also placed there and commented out here.
%</package>
%<*discard>
%\fi
%    \begin{macrocode}
\ifdefined\childdocmain\endinput\fi
%    \end{macrocode}
%\iffalse
%</discard>
%<*package>
%\fi
%
% \macro{\ifchilddoc}
% \macro{\ifchilddocmanual}
% The conditional |\ifchilddoc| tells whether a
% child (true) or main (false) document is being compiled.
% The conditional |\ifchilddocmanual| tells whether
% the |\includeonly| mechanism is used (false) or
% the selection of child files must be performed manually (true).
% The definitions initialise to false:
%    \begin{macrocode}
\newif\ifchilddoc
\newif\ifchilddocmanual
%    \end{macrocode}

% \macro{\childdocname}
% \macro{\childdocjob}
% The macro |\childdocname| stores the name of the main document
% to be compiled. The macro |\childdocjob| stores the name of
% the document on which the \LaTeX{} compiler was originally invoked.
% The content of |\jobname| cannot be compared
% to filenames specified in the source due to different catcodes.
% The following code rescans |\jobname|, stores the result
% in |\childdocname| and saves a copy in |\childdocjob|:
%    \begin{macrocode}
\edef\childdocname{\scantokens\expandafter{\jobname\noexpand}}
\let\childdocjob\childdocname
%    \end{macrocode}

% \macro{\childdocdisable}
% The macro |\childdocdisable| prevents the main file
% from being processed more than once.
% At this stage, the main document command |\childdocmain|
% is assumed to be called once again where it should do nothing.
% Any subsequent call to it should prevent
% a secondary processing of the main document
% It overwrites the forwarding commands
% |\childdocof| and |\childdocforward|
% with empty macros to prevent further inclusions of the main document:
%    \begin{macrocode}
\newcommand{\childdocdisable}
{
  \renewcommand{\childdocmain}[1]{\renewcommand{\childdocmain}[1]{\endinput}}
  \renewcommand{\childdocof}[1]{}
  \renewcommand{\childdocby}[2][]{}
  \renewcommand{\childdocforward}[2][]{}
  \renewcommand{\childdocdisable}{}
}
%    \end{macrocode}

% \macro{\childdocmain}
% The macro |\childdocmain| is to be called at the top of the main file
% with nothing or the main filename (without extension) as argument.
% First, it breaks loops.
% If the argument is not empty and does not match |\childdocname|
% (which is set by the first inclusion of |childdoc.def|),
% |\ifchilddoc| is set to true, |\includeonly| is applied to the child file
% and |\jobname| is set to the main file
% (for proper handling of |.aux| files):
%    \begin{macrocode}
\newcommand{\childdocmain}[1]
{
  \childdocdisable\childdocmain{}
  \if?#1?\else
    \begingroup
      \def\childdoctmp{#1}
      \ifx\childdoctmp\childdocname
        \def\childdoctmp{}
      \else
        \def\childdoctmp
        {
          \childdoctrue
          \includeonly{\childdocname}
          \def\childdocjob{#1}
          \def\jobname{#1}
        }
      \fi
      \expandafter
    \endgroup
    \childdoctmp
  \fi
}
%    \end{macrocode}

% \macro{\childdocof}
% The command |\childdocof| redirects
% compilation to the main file |#1|.
%    \begin{macrocode}
\newcommand{\childdocof}[1]
{
  \childdocdisable
  \childdoctrue
  \includeonly{\childdocname}
  \def\jobname{#1}
  \def\childdocjob{#1}
  \input{#1}
}
%    \end{macrocode}

% \macro{\childdocby}
% The command |\childdocby| ....
%    \begin{macrocode}
\newcommand{\childdocby}[2][]
{
  \childdocdisable
  \childdoctrue
  \childdocmanualtrue
  \if?#1?\else
    \def\jobname{#2}
  \fi
  \def\childdocjob{#2}
  \input{#2}
  \endinput
}
%    \end{macrocode}

% \macro{\childdocforward}
% The command |\childdocforward| redirects
% compilation to the main file or
% (if the optional argument is given) a child file.
% Parameters are set as if the main file
% or a child file starting with |\childdocof| was compiled.
% Then compilation is handed over to the main file:
%    \begin{macrocode}
\newcommand{\childdocforward}[2][]
{
  \begingroup
    \if?#1?
      \def\childdoctmp
      {
        \def\childdocname{#2}
        \def\childdocjob{#2}
        \def\jobname{#2}
        \input{#2}
        \endinput
      }
    \else
      \def\childdoctmp
      {
        \childdocdisable
        \def\childdocname{#2}
        \childdoctrue
        \includeonly{#2}
        \def\childdocjob{#1}
        \def\jobname{#1}
        \input{#1}
        \endinput
      }
    \fi
    \expandafter
  \endgroup
  \childdoctmp
}
%    \end{macrocode}

% \macro{\childdocforwardprefix}
% The command |\childdocforwardprefix| redirects
% compilation to the main or a child file by means of a pattern.
% The prefix |#1| in the current filename is replaced by |#2|
% and the suffix of the current filename is kept
% (it is assumed that the filename does not contain the substring `|~~~|'
% which is used as a delimiter).
% Compilation is handed over to the new file by |\childdocforward|:
%    \begin{macrocode}
\newcommand{\childdocforwardprefix}[3][]
{
  \begingroup
    \def\childdocextract #2##1~~~{\def\childdoctmp{\childdocforward[#1]{#3##1}}}
    \expandafter\childdocextract\childdocname~~~
    \expandafter
  \endgroup
  \childdoctmp
}
%    \end{macrocode}

% \macro{\childdoc}
% The deprecated macro |\childdoc| is a legacy version of |\childdocmain|:
%    \begin{macrocode}
\newcommand{\childdoc}{\childdocmain}
%    \end{macrocode}

% \macro{\childdocredirect}
% The deprecated macro |\childdocredirect| is a legacy version
% of |\childdocforward| and |\childdocforwardprefix|:
%    \begin{macrocode}
\newcommand{\childdocredirect}[2][]
{
  \begingroup
    \if?#1?
      \def\childdoctmp{\childdocforward{#2}}
    \else
      \def\childdoctmp{\childdocforwardprefix{#1}{#2}}
    \fi
    \expandafter
  \endgroup
  \childdoctmp
}
%    \end{macrocode}

%\iffalse
%</package>
%\fi
%
\endinput
|\\
|\childdocforward{|\textit{main}|}|\\
\end{tabular}
\end{center}
%
or alternatively with:
%
\begin{center}
\begin{tabular}{l}
|% \iffalse
%
% childdoc.dtx Copyright (C) 2017-2018 Niklas Beisert
%
% This work may be distributed and/or modified under the
% conditions of the LaTeX Project Public License, either version 1.3
% of this license or (at your option) any later version.
% The latest version of this license is in
%   http://www.latex-project.org/lppl.txt
% and version 1.3 or later is part of all distributions of LaTeX
% version 2005/12/01 or later.
%
% This work has the LPPL maintenance status `maintained'.
%
% The Current Maintainer of this work is Niklas Beisert.
%
% This work consists of the files childdoc.dtx and childdoc.ins
% and the derived files childdoc.def and cdocsamp.tex with
% cdocsch1.tex, cdocsch2.tex, cdocsdrf.tex, cdocsfn1.tex, cdocsfn2.tex.
%
%<package>\ifdefined\childdocmain\endinput\fi
%<package>\ProvidesFile{childdoc.def}[2018/12/30 v2.0 child document driver]
%<samplemain>\ProvidesFile{cdocsamp.tex}[2018/12/30 v2.0 sample for childdoc]
%<*driver>
%\ProvidesFile{childdoc.drv}[2018/12/30 v2.0 childdoc reference manual file]
\PassOptionsToClass{10pt,a4paper}{article}
\documentclass{ltxdoc}

\usepackage[margin=35mm]{geometry}
\usepackage{hyperref}
\usepackage{hyperxmp}
\usepackage[usenames]{color}

\hypersetup{colorlinks=true}
\hypersetup{pdfstartview=FitH}
\hypersetup{pdfpagemode=UseNone}
\hypersetup{pdfsource={}}
\hypersetup{pdflang={en-UK}}
\hypersetup{pdfcopyright={Copyright 2017-2018 Niklas Beisert.
  This work may be distributed and/or modified under the
  conditions of the LaTeX Project Public License, either version 1.3
  of this license or (at your option) any later version.}}
\hypersetup{pdflicenseurl={http://www.latex-project.org/lppl.txt}}
\hypersetup{pdfcontactaddress={ETH Zurich, ITP, HIT K,
  Wolfgang-Pauli-Strasse 27}}
\hypersetup{pdfcontactpostcode={8093}}
\hypersetup{pdfcontactcity={Zurich}}
\hypersetup{pdfcontactcountry={Switzerland}}
\hypersetup{pdfcontactemail={nbeisert@itp.phys.ethz.ch}}
\hypersetup{pdfcontacturl={http://people.phys.ethz.ch/\xmptilde nbeisert/}}

\newcommand{\secref}[1]{\hyperref[#1]{section \ref*{#1}}}

\parskip1ex
\parindent0pt
\let\olditemize\itemize
\def\itemize{\olditemize\parskip0pt}

\begin{document}

\title{The \textsf{childdoc} Package}
\hypersetup{pdftitle={The childdoc Package}}
\author{Niklas Beisert\\[2ex]
  Institut f\"ur Theoretische Physik\\
  Eidgen\"ossische Technische Hochschule Z\"urich\\
  Wolfgang-Pauli-Strasse 27, 8093 Z\"urich, Switzerland\\[1ex]
  \href{mailto:nbeisert@itp.phys.ethz.ch}
  {\texttt{nbeisert@itp.phys.ethz.ch}}}
\hypersetup{pdfauthor={Niklas Beisert}}
\hypersetup{pdfsubject={Manual for the LaTeX2e Package childdoc}}
\date{30 December 2018, \textsf{v2.0}}
\maketitle

\begin{abstract}\noindent
\textsf{childdoc} is a \LaTeXe{} package
that enables the direct compilation
of document sections included by |\include|
to individual files.
\end{abstract}

\begingroup
\parskip0ex
\tableofcontents
\endgroup

%%%%%%%%%%%%%%%%%%%%%%%%%%%%%%%%%%%%%%%%%%%%%%%%%%%%%%%%%%%%%%%%%%%%%%%%%%%%%%%%
%%%%%%%%%%%%%%%%%%%%%%%%%%%%%%%%%%%%%%%%%%%%%%%%%%%%%%%%%%%%%%%%%%%%%%%%%%%%%%%%
\section{Introduction}

\LaTeX{} provides a mechanism to structure a large document (such as a book)
into a main file and several child files (containing the chapters)
using the |\include| command.
This mechanism is beneficial for documents
which span hundreds of pages in order to
make the source file(s) more manageable.
Moreover, compilation can be restricted to
selected child files by means of the |\includeonly| command.
The latter feature can be used to reduce the compilation time while editing
(this was significantly more useful in the earlier days of \LaTeX{})
or to generate a smaller document which is easier to navigate.
Another application of |\includeonly| is to generate
documents consisting of selected parts of the complete document.

However, there are a few drawbacks of the plain |\include| mechanism:
\begin{itemize}
\item
The child files cannot be compiled on their own,
they can only be compiled via the main file.
A naive editing environment
(such as a text editor with an option
to have the current file processed by \LaTeX)
may require one to switch to the main file before compiling;
attempting to compile the child file produces errors.
\item
The main file must be modified (each time)
to adjust the |\includeonly| command
to the present needs. This easily leaves the main file in a messy state.
\item
The generated document will always carry the filename
of the main document. This is inconvenient if
several child files are to be compiled and
to be kept for distribution.
\end{itemize}

The present package provides a simple interface
to make child files individually compilable by \LaTeX{}.
Compiling a child file then has the same effect as compiling
the main file with an |\includeonly| command
to select the appropriate child.
Moreover the generated document will carry the name of the child
rather than the main file.
This resolves all three above issues.

This feature is meant to make the editing of books,
thesis documents and lecture notes somewhat more convenient.
However, the package can also be used efficiently for
composing a series of documents (such as exercise sheets)
which are typically distributed individually.
It then assists the author in generating the individual documents
(potentially in different versions)
as well as a document containing the collected series.
Another application is in developing style files
or other kinds of included material
where compilation of the style file could redirect
to a sample or test file.

%%%%%%%%%%%%%%%%%%%%%%%%%%%%%%%%%%%%%%%%%%%%%%%%%%%%%%%%%%%%%%%%%%%%%%%%%%%%%%%%
%%%%%%%%%%%%%%%%%%%%%%%%%%%%%%%%%%%%%%%%%%%%%%%%%%%%%%%%%%%%%%%%%%%%%%%%%%%%%%%%
\section{Usage}

First of all, the package \textsf{childdoc} is \emph{not} a standard
\LaTeXe{} |.sty| style file! Therefore it needs to be invoked in
a non-standard way.

%%%%%%%%%%%%%%%%%%%%%%%%%%%%%%%%%%%%%%%%%%%%%%%%%%%%%%%%%%%%%%%%%%%%%%%%%%%%%%%%
\subsection{Included Files}
\label{sec:include}

%%%%%%%%%%%%%%%%%%%%%%%%%%%%%%%%%%%%%%%%
\DescribeMacro{\childdocmain}
To use the package, add the commands
\begin{center}
\begin{tabular}{l}
|\input{childdoc.def}|\\
|\childdocmain{}|\\
\end{tabular}
\end{center}
at the very top of the main \LaTeX{} file,
in particular \emph{before} the |\documentclass| statement!
The argument of |\childdocmain| should be left empty
(but it must be present).

%%%%%%%%%%%%%%%%%%%%%%%%%%%%%%%%%%%%%%%%
\DescribeMacro{\childdocof}
Furthermore, add the commands
\begin{center}
\begin{tabular}{l}
|\input{childdoc.def}|\\
|\childdocof{|\textit{main}|}|\\
\end{tabular}
\end{center}
at the top of every child file \textit{child}
which is included by |\include{|\textit{child}|}|
from within the main file
(or at least for those files to be compiled individually).
The argument \textit{main} must be the filename of the main file.

There are a couple of
considerations in setting up the main and child documents:

%%%%%%%%%%%%%%%%%%%%%%%%%%%%%%%%%%%%%%%%
\paragraph{Restrictions.}

Please note the following restrictions:
\begin{itemize}
\item
|\childdocmain| must be called with one argument \textit{main}
to ensure compatibility with earlier version of the package.
It must either be empty (|\childdocmain{}|)
or precisely match the filename of the main file in which it is specified.
See \secref{sec:detection} for further information.
\item
The filename \textit{main} must be specified without the |.tex| extension.
\item
The filename \textit{main} is case sensitive
(even in case-insensitive file systems)
due to internal string comparison.
\item
The argument \textit{main} should be fully expanded, it cannot be a macro.
\item
Subdirectories and special characters should be avoided in filenames.
\item
The command |\childdocmain{|\textit{main}|}| must be followed by a whitespace.
It should not be followed immediately by another command
or by a comment mark `|%|'.
This is because the \TeX{} parser reads the token immediately following
the argument of |\childdocmain| and puts it
at the beginning of every child section;
however, a white\-space is ignored.
\end{itemize}

%%%%%%%%%%%%%%%%%%%%%%%%%%%%%%%%%%%%%%%%
\paragraph{Content of Main File.}

It is advisable to place all content in the child files included by |\include|.
Any output contained in the main file will appear in all child documents
unless suppressed manually;
it cannot be suppressed automatically by the |\includeonly| directive
and thus should normally be avoided.
A method to include some content in the main file
by means of conditional processing is described in \secref{sec:conditional}.

%%%%%%%%%%%%%%%%%%%%%%%%%%%%%%%%%%%%%%%%
\paragraph{Page Numbering.}

When only a part of the document is compiled,
the appropriate numbering of pages
(as well as other status parameters)
is determined from the |.aux| files.
The latter contain information from previous passes.
However this information needs to propagate through
all intermediate child documents.
Therefore the page numbering in child documents may well
be inconsistent until the complete document is compiled at least once.

A useful (if unconventional) way to always ensure a consistent
page numbering is to restart the numbering in each child document
and denote the pages by `\textit{child}|.|\textit{page}'
where \textit{child} represents the chapter/section number of the child file.
This can be achieved by the command
|\numberwithin{page}{|\textit{child}|}|
of the \textsf{amsmath} package
where \textit{child} can be |chapter| or |section|
depending on the chosen structuring.
Alternatively, one can modify the macro |\thepage| appropriately
and reset the counter |page| at the start of each child file.

%%%%%%%%%%%%%%%%%%%%%%%%%%%%%%%%%%%%%%%%%%%%%%%%%%%%%%%%%%%%%%%%%%%%%%%%%%%%%%%%
\subsection{Conditional Processing}
\label{sec:conditional}

The package provides a mechanism to compile different versions
of a document. To customise the versions further some conditional processing
can come in handy to distinguish which version is being compiled.
The package provides two macros to describe the compilation context:

%%%%%%%%%%%%%%%%%%%%%%%%%%%%%%%%%%%%%%%%
\DescribeMacro{\ifchilddoc}
The conditional |\ifchilddoc| distinguishes between the compilation of
child documents and the main document:
%
\begin{center}
|\ifchilddoc |\textit{child-code}| |[|\||else |\textit{main-code}]| \||fi|
\end{center}

%%%%%%%%%%%%%%%%%%%%%%%%%%%%%%%%%%%%%%%%
\DescribeMacro{\childdocname}
\DescribeMacro{\childdocjob}
The macro |\childdocname| contains the filename (without extension)
of the main or child file being processed.
Note that |\childdocjob| will always contain the name of the main file.

%%%%%%%%%%%%%%%%%%%%%%%%%%%%%%%%%%%%%%%%
\paragraph{Title Page.}

Conditional processing can be used to include a title or banner page
in the main document when proper precautions are taken.
Importantly, the code in the main file should ensure that the page counter
(as well as other status parameters which are stored in the |.aux| files)
takes the same value after the conditional processing.
Otherwise the page numbers may take divergent values
depending on which part is compiled.

For example, a title page could be declared by:
%
\begin{center}
\begin{tabular}{l}
|\ifchilddoc\||else|\\
|\addtocounter{page}{-1}|\\
\textit{code for title page}\\
|\newpage|\\
|\||fi|
\end{tabular}
\end{center}
%
A banner page for the child documents can be generated by:
%
\begin{center}
\begin{tabular}{l}
|\ifchilddoc|\\
|\addtocounter{page}{-1}|\\
\textit{code for banner page}\\
|\newpage|\\
|\||fi|
\end{tabular}
\end{center}
%
Here one could write a message such as:
\begin{center}
|This is the part \childdocname{} of \childdocjob{}.|
\end{center}

%%%%%%%%%%%%%%%%%%%%%%%%%%%%%%%%%%%%%%%%%%%%%%%%%%%%%%%%%%%%%%%%%%%%%%%%%%%%%%%%
\subsection{Flags}
\label{sec:flags}

The package makes it easy to generate different versions
of the main or child documents.
To this end compilation flags can be defined
and assigned different default values.
They will be particularly useful in conjunction
with the forwarding mechanism described in \secref{sec:forward}.

For example, it may be useful to have a flag |\version|
which can be set to |draft| or |final|.
The document source will contain some conditional code
depending on the value of |\version|.
Suppose further, the flag should default to |final| for the main file
and to |draft| for child files
which is a natural assignment for editing the document.
This is achieved by placing the following code
in the preamble of the main document
(below the |\childdocmain| directive):
%
\begin{center}
\begin{tabular}{l}
|\ifchilddoc|\\
|\providecommand{\version}{draft}|\\
|\||else|\\
|\providecommand{\version}{final}|\\
|\||fi|
\end{tabular}
\end{center}
%
The definition by |\providecommand| makes sure
that previous definitions are not overwritten.
Further statements |\providecommand{\version}{...}|
can thus be added before the above code to override it.

For the main file, one might add a line
(between |\childdocmain| and the above block)
%
\begin{center}
|%\ifchilddoc\||else\providecommand{\version}{draft}\||fi|
\end{center}
%
which can be uncommented to produce a draft version.
Likewise one can add a line to the very top of a child file
(above the |\childdocof{|\textit{main}|}| directive)
%
\begin{center}
|%\providecommand{\version}{final}|
\end{center}
%
which can be uncommented to produce the final version of this child document.

%%%%%%%%%%%%%%%%%%%%%%%%%%%%%%%%%%%%%%%%%%%%%%%%%%%%%%%%%%%%%%%%%%%%%%%%%%%%%%%%
\subsection{Forwarding}
\label{sec:forward}

Different versions of the main or child documents
using compilation flags as described in \secref{sec:flags}
can be (permanently) stored in different files
for convenient compilation, viewing and distribution.
To this end, the package defines a command
to pass on compilation to a different file:

%%%%%%%%%%%%%%%%%%%%%%%%%%%%%%%%%%%%%%%%
\DescribeMacro{\childdocforward}
The command |\childdocforward| redirects processing to
another source file:
%
\begin{center}
\begin{tabular}{l}
|\input{childdoc.def}|\\
|\childdocforward[|\textit{main}|]{|\textit{dest}|}|\\
\end{tabular}
\end{center}
%
The argument \textit{dest} is the destination file
(without extension).
It should be the main file or one of the child files.
Note that further \textsf{childdoc} directives
such as |\childdocof| and |\childdocforward|
in the indicated file will be processed in this form.
The optional argument \textit{main}
passes on directly to the main file \textit{main}
while pretending to compile the child \textit{dest}.
This form behaves as if \textit{dest}
issues |\childdocof{|\textit{main}|}| right away,
and no further \textsf{childdoc} directives will be processed.

%%%%%%%%%%%%%%%%%%%%%%%%%%%%%%%%%%%%%%%%
\DescribeMacro{\...prefix}
In the alternative form |\childdocforwardprefix|,
%
\begin{center}
\begin{tabular}{l}
|\input{childdoc.def}|\\
|\childdocforwardprefix[|\textit{main}|]{|\textit{prefix}|}{|\textit{dest}|}|
\end{tabular}
\end{center}
%
the destination file is determined by a pattern
depending on the current file:
To make this work, the current file must be called
`{\textit{prefix}\hspace{0.2em}\textit{suffix}}'
with \textit{prefix} matching precisely the argument.
Processing is then passed on to the file
`{\textit{dest}\hspace{0.2em}\textit{suffix}}'.
Surely, the same effect is achieved by
directly specifying the
argument `{\textit{dest}\hspace{0.2em}\textit{suffix}}'
in the first form.
However, that requires to set up a different file
for each child. With the alternative form of the command
all these files can have exactly the same content
which simplifies setting them up and maintaining them.

For example, the following file |draft.tex|
with a compilation flag |\version| as described in \secref{sec:flags}
compiles the main document as a draft:
%
\begin{center}
\begin{tabular}{l}
|\def\version{draft}|\\
|\input{childdoc.def}|\\
|\childdocforward{|\textit{main}|}|
\end{tabular}
\end{center}
%
Likewise, the following files |final|\textit{nn}|.tex|
compile the final version of the child document
|child|\textit{nn}|.tex|:
%
\begin{center}
\begin{tabular}{l}
|\def\version{final}|\\
|\input{childdoc.def}|\\
|\childdocforwardprefix{final}{child}|
\end{tabular}
\end{center}
%

Note that when several versions of a main file and/or of each child file
are to be generated, it may be convenient to set up a |Makefile| or
shell script to automatise the process.

%%%%%%%%%%%%%%%%%%%%%%%%%%%%%%%%%%%%%%%%%%%%%%%%%%%%%%%%%%%%%%%%%%%%%%%%%%%%%%%%
\subsection{Command Line Processing}
\label{sec:commandline}

The effect of redirection files can also be achieved by invoking
the \LaTeX{} compiler with a more elaborate command line.
Most conveniently this should be done as part
of a shell script or a |Makefile|.

When using \textsf{childdoc} in the main file, the following
command lines effectively perform a redirection
(note that depending on the shell being used,
backslashes may have to be doubled: `|\|' $\to$ `|\\|'):
%
\begin{center}
|... -jobname "|\textit{target}|" |\\|"|[\textit{flags}]%
|\input{childdoc.def}\childdocforward[|\textit{main}|]{|\textit{dest}|}"|
\end{center}
%
Here \textit{target} is the name of the output file,
\textit{main} is the name of the main file
and \textit{dest} is the name of the main or child file to be processed
(all filenames without extensions).
The optional argument \textit{main} can be omitted
if \textit{main} matches \textit{dest}.
Optionally, compilation \textit{flags} can be defined via |\def| commands.
This command line makes the \TeX{} engine believe
it is compiling the file \textit{target}
whose content is specified as the latter parameter.
The provided code then forwards the processing to
\textit{main} or \textit{dest} as described in \secref{sec:forward}.

%%%%%%%%%%%%%%%%%%%%%%%%%%%%%%%%%%%%%%%%%%%%%%%%%%%%%%%%%%%%%%%%%%%%%%%%%%%%%%%%
\subsection{Include by Input}
\label{sec:input}

Including child documents by |\include| has some restrictions by design.
Most notably, the content of a child document always occupies
its own set of pages; pages cannot be shared between child documents.
Usually, this behaviour makes perfect sense
because each child document contain an essential part of the document.
However, in some situations it may be desirable to compose
a document from a collection of parts
without having mandatory page breaks between then.
For this case, the package
provides a mechanism to include parts
by |\input| which can also be processed individually.
However, by construction this mechanism
requires manual handling of the content to be output.

%%%%%%%%%%%%%%%%%%%%%%%%%%%%%%%%%%%%%%%%
\DescribeMacro{\ifchilddocmanual}
The main file should be prepared as usual, see \secref{sec:include}.
However, the document body must make a distinction
between processing of an individual part and of the main document, e.g.:
%
\begin{center}
\begin{tabular}{l}
|\ifchilddocmanual|\\
|\input{\childdocname}|\\
|\||else|\\
\textit{document body with }|\input{|\textit{part}|}|\\
|\||fi|
\end{tabular}
\end{center}
%
The conditional |\ifchilddocmanual| is true whenever
a part to be included by |\input| is being compiled,
and the name of the part is stored in |\childdocname|.

%%%%%%%%%%%%%%%%%%%%%%%%%%%%%%%%%%%%%%%%
\DescribeMacro{\childdocby}
Each part to be included by |\input| should start with:
%
\begin{center}
\begin{tabular}{l}
|\input{childdoc.def}|\\
|\childdocby{|\textit{main}|}|\\
\end{tabular}
\end{center}
%
The directive |\childdocby| is similar to |\childdocof|
described in \secref{sec:include},
but the subsequent selection of content must be done manually.
To that end, both |\ifchilddoc| and |\ifchilddocmanual|
will be true upon processing of a part,
and the name of the part is stored in |\childdocname|.
Note that |\jobname| will be set to the filename of the current part
so that each part receives an individual |.aux| file
that does not interfere with the |.aux| file(s) of the main document.
This behaviour can be altered by the alternative form
|\childdocby[*]{|\textit{main}|}| (with a non-empty optional argument)
which uses the |.aux| file of the main document
by setting |\jobname| to \textit{main}.

%%%%%%%%%%%%%%%%%%%%%%%%%%%%%%%%%%%%%%%%%%%%%%%%%%%%%%%%%%%%%%%%%%%%%%%%%%%%%%%%
\subsection{Driver Development}
\label{sec:driver}

The \textsf{childdoc} mechanism can also be use for the development
of definition files such as \LaTeX{} styles or classes.
This case differs from the above setup with multiple parts
included by |\include| in that no |\includeonly| should be invoked.
This can be achieved by starting the include file
(before |\ProvidesPackage|) with:
%
\begin{center}
\begin{tabular}{l}
|\input{childdoc.def}|\\
|\childdocforward{|\textit{main}|}|\\
\end{tabular}
\end{center}
%
or alternatively with:
%
\begin{center}
\begin{tabular}{l}
|\input{childdoc.def}|\\
|\childdocby{|\textit{main}|}|\\
\end{tabular}
\end{center}
%
Both forms have slightly different effects as described above.
The main file is prepared as usual, see \secref{sec:include}.

%%%%%%%%%%%%%%%%%%%%%%%%%%%%%%%%%%%%%%%%%%%%%%%%%%%%%%%%%%%%%%%%%%%%%%%%%%%%%%%%
\subsection{Legacy Detection}
\label{sec:detection}

The directive |\childdocmain| in the main file can detect
whether the complete document or merely a child is to be compiled
even without using the directive |\childdocof|.
This method is deprecated because it is less robust
and there is no compelling reason to use it;
it is merely provided for backward compatibility
and it may be removed in future versions.

If the detection mechanism is to be used,
it is mandatory to correctly specify
the filename of the main file as the argument of |\childdocmain|:
%
\begin{center}
\begin{tabular}{l}
|\input{childdoc.def}|\\
|\childdocmain{|\textit{main}|}|\\
\end{tabular}
\end{center}
%
If |\jobname| does not match the argument \textit{main} of |\childdocmain|,
it is assumed that |\jobname| points to the child file to be compiled.
When using |\childdocmain| with the main file specified as argument,
it suffices to start a child file
with just |\input{|\textit{main}|}|
without loading of the package and using |\childdocof|.
If instead all processing is done
with the appropriate \textsf{childdoc} directives,
the argument of \textit{main} of |\childdocmain| can be empty.

An alternative version of the command line processing described
in \secref{sec:commandline} using the detection mechanism reads:
%
\begin{center}
|... -jobname "|\textit{target}|" "|[\textit{flags}]%
[|\def\jobname{|\textit{dest}|}|]|\input{|\textit{main}|}"|
\end{center}

%%%%%%%%%%%%%%%%%%%%%%%%%%%%%%%%%%%%%%%%%%%%%%%%%%%%%%%%%%%%%%%%%%%%%%%%%%%%%%%%
\subsection{Manual Code}
\label{sec:manual}

In case one cannot be certain whether the definitions file |childdoc.def|
is installed on the target \TeX{} distribution
and one prefers not to ship it,
it is conceivable to paste a few relevant commands into the sources.

To that end, drop all statements |\input{childdoc.def}|
and perform the replacements as outlined below.
Instead of |\childdocmain{|\textit{main}|}| add the following code
to the top of the main file:
%
\begin{center}
\begin{tabular}{l}
|\||ifdefined\childdocname\endinput\||fi\newif\ifchilddoc|\\
|\edef\childdocname{\scantokens\expandafter{\jobname\noexpand}}|\\
|\def\childdocmain{|\textit{main}|}\||ifx\childdocmain\childdocname\||else|\\
|\childdoctrue\includeonly{\childdocname}\let\jobname\childdocmain\||fi|\\
\end{tabular}
\end{center}
%
Instead of |\childdocof{|\textit{main}|}| just include the main file
at the top of each child file:
%
\begin{center}
|\input{|\textit{main}|}|
\end{center}
%
A simple redirection |\childdocforward{|\textit{dest}|}| is achieved by:
%
\begin{center}
|\def\jobname{|\textit{dest}|}\input{\jobname}|
\end{center}
%
The redirection with prefix
|\childdocforwardprefix[|\textit{prefix}|]{|\textit{dest}|}|
is accomplished by:
%
\begin{center}
\begin{tabular}{l}
|{\edef\jobname{\scantokens\expandafter{\jobname\noexpand}}|\\
|\def\redirectjob |\textit{prefix}|#1~~~{\gdef\jobname{|\textit{dest}|#1}}|\\
|\expandafter\redirectjob\jobname~~~}\input{\jobname}|
\end{tabular}
\end{center}

In an alternative approach,
child documents can be compiled by a specific command line
without additional code or specific definitions:
%
\begin{center}
|... -jobname "|\textit{target}|" "|[\textit{flags}]%
|\includeonly{|\textit{dest}|}\input{|\textit{main}|}"|
\end{center}
%

%%%%%%%%%%%%%%%%%%%%%%%%%%%%%%%%%%%%%%%%%%%%%%%%%%%%%%%%%%%%%%%%%%%%%%%%%%%%%%%%
%%%%%%%%%%%%%%%%%%%%%%%%%%%%%%%%%%%%%%%%%%%%%%%%%%%%%%%%%%%%%%%%%%%%%%%%%%%%%%%%
\section{Information}

%%%%%%%%%%%%%%%%%%%%%%%%%%%%%%%%%%%%%%%%%%%%%%%%%%%%%%%%%%%%%%%%%%%%%%%%%%%%%%%%
\subsection{Copyright}

Copyright \copyright{} 2017--2018 Niklas Beisert

This work may be distributed and/or modified under the
conditions of the \LaTeX{} Project Public License, either version 1.3
of this license or (at your option) any later version.
The latest version of this license is in
  \url{http://www.latex-project.org/lppl.txt}
and version 1.3 or later is part of all distributions of \LaTeX{}
version 2005/12/01 or later.

This work has the LPPL maintenance status `maintained'.

The Current Maintainer of this work is Niklas Beisert.

This work consists of the files |README.txt|, |childdoc.ins| and |childdoc.dtx|
as well as the derived files |childdoc.def|, |cdocsamp.tex|
with |cdocsch1.tex|, |cdocsch2.tex|, |cdocspt3.tex|, |cdocspt4.tex|,
|cdocsdrf.tex|, |cdocsfn1.tex|, |cdocsfn2.tex|
as well as |childdoc.pdf|.

%%%%%%%%%%%%%%%%%%%%%%%%%%%%%%%%%%%%%%%%%%%%%%%%%%%%%%%%%%%%%%%%%%%%%%%%%%%%%%%%
\subsection{Files and Installation}

The package consists of the files:
%
\begin{center}
\begin{tabular}{ll}
    |README.txt|   & readme file \\
    |childdoc.ins| & installation file \\
    |childdoc.dtx| & source file \\
    |childdoc.def| & definition file \\
    |cdocsamp.tex| & sample main file \\
    |cdocsch1.tex| & sample include file \\
    |cdocsch2.tex| & sample include file \\
    |cdocspt3.tex| & sample part file \\
    |cdocspt4.tex| & sample part file \\
    |cdocsdrf.tex| & sample redirection file \\
    |cdocsfn1.tex| & sample redirection file \\
    |cdocsfn2.tex| & sample redirection file \\
    |childdoc.pdf| & manual
\end{tabular}
\end{center}
%
The distribution consists of the files
|README.txt|, |childdoc.ins| and |childdoc.dtx|.
%
\begin{itemize}
\item
Run (pdf)\LaTeX{} on |childdoc.dtx|
to compile the manual |childdoc.pdf| (this file).
\item
Run \LaTeX{} on |childdoc.ins| to create the definitions file |childdoc.def|
and the sample |cdocsamp.tex| with include files
|cdocsch1.tex|, |cdocsch2.tex|, |cdocspt3.tex|, |cdocspt4.tex|,
|cdocsdrf.tex|, |cdocsfn1.tex|, |cdocsfn2.tex|.
Then copy the file |childdoc.def| to an appropriate directory of your \LaTeX{}
distribution, e.g.\ \textit{texmf-root}|/tex/latex/childdoc|.
\end{itemize}

%%%%%%%%%%%%%%%%%%%%%%%%%%%%%%%%%%%%%%%%%%%%%%%%%%%%%%%%%%%%%%%%%%%%%%%%%%%%%%%%
\subsection{Related CTAN Packages}

There are several other packages which offer a similar functionality:
%
\begin{itemize}
\item
The packages
\href{http://ctan.org/pkg/docmute}{\textsf{docmute}},
\href{http://ctan.org/pkg/includex}{\textsf{includex}} and
\href{http://ctan.org/pkg/standalone}{\textsf{standalone}}
provide commands to include only the document body of
a child file thus allowing both files to be compiled individually.
\item
The packages \href{http://ctan.org/pkg/subdocs}{\textsf{subdocs}}
and \href{http://ctan.org/pkg/subfiles}{\textsf{subfiles}}
provide structures in which the main and child documents can be
encapsulated and allowing them to be compiled individually.
The inclusion mechanism is different from the conventional |\include|.
\item
The package \href{http://ctan.org/pkg/combine}{\textsf{combine}}
is an elaborate solution to combine several documents into one.
\end{itemize}
%
See also the CTAN topic \href{http://ctan.org/topic/subdocs}{\textsf{subdocs}}
for further related packages.
The present package differs from the above solutions in that
a document structure constructed with the conventional |\include| mechanism
just needs two extra commands at the top of every file
such that all constituent files can be compiled individually.

%%%%%%%%%%%%%%%%%%%%%%%%%%%%%%%%%%%%%%%%%%%%%%%%%%%%%%%%%%%%%%%%%%%%%%%%%%%%%%%%
%\subsection{Feature Suggestions}
%
%The following is a list of features which may be useful for future
%versions of this package:
%%
%\begin{itemize}
%\item
%\ldots
%\end{itemize}

%%%%%%%%%%%%%%%%%%%%%%%%%%%%%%%%%%%%%%%%%%%%%%%%%%%%%%%%%%%%%%%%%%%%%%%%%%%%%%%%
\subsection{Revision History}

%%%%%%%%%%%%%%%%%%%%%%%%%%%%%%%%%%%%%%%%
\paragraph{v2.0:} 2018/12/30

\begin{itemize}
\item
immediate forward processing
\item
added |\childdocby| mechanism
\item
manual restructured
\end{itemize}

%%%%%%%%%%%%%%%%%%%%%%%%%%%%%%%%%%%%%%%%
\paragraph{v1.6:} 2018/01/17

\begin{itemize}
\item
application for development of include files
\item
corrections to manual
\end{itemize}

%%%%%%%%%%%%%%%%%%%%%%%%%%%%%%%%%%%%%%%%
\paragraph{v1.5:} 2017/05/21

\begin{itemize}
\item
more complete structuring introduced
\item
|\childdocof| introduced
\item
|\childdoc| renamed to |\childdocmain|
\item
|\childredirect| renamed to |\childdocforward| and |\childdocforwardprefix|
and functionality expanded
\end{itemize}

%%%%%%%%%%%%%%%%%%%%%%%%%%%%%%%%%%%%%%%%
\paragraph{v1.0:} 2017/04/27

\begin{itemize}
\item
manual and install package
\item
first version published on CTAN
\end{itemize}

%%%%%%%%%%%%%%%%%%%%%%%%%%%%%%%%%%%%%%%%
\paragraph{v0.6:} 2017/04/26

\begin{itemize}
\item
redirection mechanism added
\end{itemize}

%%%%%%%%%%%%%%%%%%%%%%%%%%%%%%%%%%%%%%%%
\paragraph{v0.5:} 2017/04/26

\begin{itemize}
\item
functionality in definition file
\end{itemize}


%%%%%%%%%%%%%%%%%%%%%%%%%%%%%%%%%%%%%%%%%%%%%%%%%%%%%%%%%%%%%%%%%%%%%%%%%%%%%%%%
%%%%%%%%%%%%%%%%%%%%%%%%%%%%%%%%%%%%%%%%%%%%%%%%%%%%%%%%%%%%%%%%%%%%%%%%%%%%%%%%
%%%%%%%%%%%%%%%%%%%%%%%%%%%%%%%%%%%%%%%%%%%%%%%%%%%%%%%%%%%%%%%%%%%%%%%%%%%%%%%%
\appendix

\settowidth\MacroIndent{\rmfamily\scriptsize 000\ }

 \DocInput{childdoc.dtx}

\end{document}
%</driver>
% \fi
%
% %%%%%%%%%%%%%%%%%%%%%%%%%%%%%%%%%%%%%%%%%%%%%%%%%%%%%%%%%%%%%%%%%%%%%%%%%%%%%%
% %%%%%%%%%%%%%%%%%%%%%%%%%%%%%%%%%%%%%%%%%%%%%%%%%%%%%%%%%%%%%%%%%%%%%%%%%%%%%%
% \section{Sample}
%\iffalse
%<*samplemain>
%\fi
%
% The following presents a sample document
% with two chapters, two parts, a title page,
% a compile flag as well as three forwarding files to set the flag.
% It consists of eight |.tex| files:
% \begin{center}
% \begin{tabular}{ll}
% |cdocsamp.tex|&main file\\
% |cdocsch1.tex|&include file for chapter 1\\
% |cdocsch2.tex|&include file for chapter 2\\
% |cdocspt3.tex|&include file for part 3\\
% |cdocspt4.tex|&include file for part 4\\
% |cdocsdrf.tex|&forwarding file for main file in draft mode\\
% |cdocsfi1.tex|&forwarding file for final version of chapter 1\\
% |cdocsfi2.tex|&forwarding file for final version of chapter 2\\
% \end{tabular}
% \end{center}
% Each of the eight files can be compiled directly by the \LaTeX{} compiler.
%
% %%%%%%%%%%%%%%%%%%%%%%%%%%%%%%%%%%%%%%
% \paragraph{Main File.}
%
% The main file is called |cdocsamp.tex|.
%
% Load the \textsf{childdoc} definitions and
% declare the filename for the main document:
%    \begin{macrocode}
\input{childdoc.def}
\childdocmain{}
%    \end{macrocode}

% Optional override for |\version| flag:
%    \begin{macrocode}
%%\ifchilddoc\else\providecommand{\version}{draft}\fi
%    \end{macrocode}

% Define the default values for the |\version| flag
% (|final| for the main file and |draft| for childs):
%    \begin{macrocode}
\ifchilddoc
\providecommand{\version}{draft}
\else
\providecommand{\version}{final}
\fi
%    \end{macrocode}

% Load the standard document class:
%    \begin{macrocode}
\documentclass[12pt]{article}
%    \end{macrocode}

% Start the document body:
%    \begin{macrocode}
\begin{document}
%    \end{macrocode}

% Declare a title page.
% Print title, part of document being processed and version flag:
%    \begin{macrocode}
\addtocounter{page}{-1}
\begin{center}
{\LARGE\bfseries{}childdoc example\par}
\vspace{1cm}
\ifchilddoc
\ifchilddocmanual part\else chapter\fi:
`\childdocname' of `\childdocjob'\par
\else
main document: `\childdocjob'\par
\fi
version: \version\par
\end{center}
\newpage
%    \end{macrocode}

% Manually include selected file,
% otherwise process as usual:
%    \begin{macrocode}
\ifchilddocmanual
\section*{part `\childdocname'}
\input{\childdocname}
\else
%    \end{macrocode}

% Include the two chapters:
%    \begin{macrocode}
\include{cdocsch1}
\include{cdocsch2}
%    \end{macrocode}

% Include the two parts unless only chapters should be displayed:
%    \begin{macrocode}
\ifchilddoc\else
\section{part three}
\input{cdocspt3}
\section{part four}
\input{cdocspt4}
\fi
%    \end{macrocode}

% Process as usual until here:
%    \begin{macrocode}
\fi
%    \end{macrocode}

% End of document body:
%    \begin{macrocode}
\end{document}
%    \end{macrocode}
%\iffalse
%</samplemain>
%\fi
%
% %%%%%%%%%%%%%%%%%%%%%%%%%%%%%%%%%%%%%%
% \paragraph{Chapter Include Files.}
%
% The include files are called |cdocsch1.tex| and |cdocsch2.tex|.
%
%\iffalse
%<*samplechap1|samplechap2>
%\fi

% Optional override for |\version| flag:
%    \begin{macrocode}
%%\providecommand{\version}{final}
%    \end{macrocode}

% Include the main document:
%    \begin{macrocode}
\input{childdoc.def}
\childdocof{cdocsamp}
%    \end{macrocode}

%\iffalse
%</samplechap1|samplechap2>
%\fi
%
%\iffalse
%<*samplechap1>
%\fi
% Some text for chapter 1:
%    \begin{macrocode}
\section{one}
some text in chapter one
%    \end{macrocode}

%\iffalse
%</samplechap1>
%\fi
% Some text for chapter 2:
%\iffalse
%<*samplechap2>
%\fi
%    \begin{macrocode}
\section{two}
more text in chapter two
%    \end{macrocode}

%\iffalse
%</samplechap2>
%\fi
%
% %%%%%%%%%%%%%%%%%%%%%%%%%%%%%%%%%%%%%%
% \paragraph{Part Include Files.}
%
% The include files are called |cdocspt3.tex| and |cdocspt4.tex|.
%
%\iffalse
%<*samplepart3|samplepart4>
%\fi

% Optional override for |\version| flag:
%    \begin{macrocode}
%%\providecommand{\version}{final}
%    \end{macrocode}

% Include the main document:
%    \begin{macrocode}
\input{childdoc.def}
\childdocby{cdocsamp}
%    \end{macrocode}

%\iffalse
%</samplepart3|samplepart4>
%\fi
%
%\iffalse
%<*samplepart3>
%\fi
% Some text for part 3:
%    \begin{macrocode}
some text in part three
%    \end{macrocode}

%\iffalse
%</samplepart3>
%\fi
% Some text for part 4:
%\iffalse
%<*samplepart4>
%\fi
%    \begin{macrocode}
more text in part four
%    \end{macrocode}

%\iffalse
%</samplepart4>
%\fi
%
% %%%%%%%%%%%%%%%%%%%%%%%%%%%%%%%%%%%%%%
% \paragraph{Forwarding for a Complete Draft.}
%
% The following forwarding file |cdocsdrf.tex|
% compiles the main document in draft mode:
%\iffalse
%<*sampledraft>
%\fi
%    \begin{macrocode}
\def\version{draft}
\input{childdoc.def}
\childdocforward{cdocsamp}
%    \end{macrocode}

%\iffalse
%</sampledraft>
%\fi
%
% %%%%%%%%%%%%%%%%%%%%%%%%%%%%%%%%%%%%%%
% \paragraph{Forwarding for Final Version of the Chapters.}
%
% The following forwarding files |cdocsfn1.tex| and |cdocsfn2.tex|
% (with identical content)
% compile the final versions of the child documents
% |cdocsch1.tex| and |cdocsch2.tex|, respectively:
%\iffalse
%<*samplefinal>
%\fi
%    \begin{macrocode}
\def\version{final}
\input{childdoc.def}
\childdocforwardprefix[cdocsamp]{cdocsfn}{cdocsch}
%    \end{macrocode}

%\iffalse
%</samplefinal>
%\fi
%
% %%%%%%%%%%%%%%%%%%%%%%%%%%%%%%%%%%%%%%
% \paragraph{Command Line Processing.}
%
% The following three command lines generate the output files
% |cdocscld|, |cdocscl1| and |cdocscl2|
% which should be identical to
% |cdocsdrf|, |cdocsch1| and |cdocsfn2|, respectively:
% \begin{center}
% \begin{tabular}{l}
% |latex -jobname cdocscld \|\\
% |  "\def\version{draft}\input{childdoc.def}\childdocforward{cdocsamp}"|\\
% |latex -jobname cdocscl1 \|\\
% |  "\input{childdoc.def}\childdocforward[cdocsamp]{cdocsch1}"|\\
% |latex -jobname cdocscl2 \|\\
% |  "\def\version{final}\input{childdoc.def}\childdocforward{cdocsch2}"|
% \end{tabular}
% \end{center}
% Note that the trailing backslash on each first line
% merely continues the input to the second line
% (for convenient cut ant paste).
% Furthermore, the command |latex| can be replaced by any
% of its alternative versions such as |pdflatex|.
%
% %%%%%%%%%%%%%%%%%%%%%%%%%%%%%%%%%%%%%%%%%%%%%%%%%%%%%%%%%%%%%%%%%%%%%%%%%%%%%%
% %%%%%%%%%%%%%%%%%%%%%%%%%%%%%%%%%%%%%%%%%%%%%%%%%%%%%%%%%%%%%%%%%%%%%%%%%%%%%%
% \section{Implementation}
%\iffalse
%<*package>
%\fi
%
% This section describes the definitions file |childdoc.def|.

% The definitions cannot be loaded using |\usepackage| or |\RequirePackage|
% which has a mechanism to prevent loading a style file more than once.
% When loading the definitions by means of |\input|
% multiple instances have to be prevented manually:
%\iffalse
%This code needs to be before the `\ProvidesFile' directive
%which is defined at the beginning of this file.
%Therefore it is also placed there and commented out here.
%</package>
%<*discard>
%\fi
%    \begin{macrocode}
\ifdefined\childdocmain\endinput\fi
%    \end{macrocode}
%\iffalse
%</discard>
%<*package>
%\fi
%
% \macro{\ifchilddoc}
% \macro{\ifchilddocmanual}
% The conditional |\ifchilddoc| tells whether a
% child (true) or main (false) document is being compiled.
% The conditional |\ifchilddocmanual| tells whether
% the |\includeonly| mechanism is used (false) or
% the selection of child files must be performed manually (true).
% The definitions initialise to false:
%    \begin{macrocode}
\newif\ifchilddoc
\newif\ifchilddocmanual
%    \end{macrocode}

% \macro{\childdocname}
% \macro{\childdocjob}
% The macro |\childdocname| stores the name of the main document
% to be compiled. The macro |\childdocjob| stores the name of
% the document on which the \LaTeX{} compiler was originally invoked.
% The content of |\jobname| cannot be compared
% to filenames specified in the source due to different catcodes.
% The following code rescans |\jobname|, stores the result
% in |\childdocname| and saves a copy in |\childdocjob|:
%    \begin{macrocode}
\edef\childdocname{\scantokens\expandafter{\jobname\noexpand}}
\let\childdocjob\childdocname
%    \end{macrocode}

% \macro{\childdocdisable}
% The macro |\childdocdisable| prevents the main file
% from being processed more than once.
% At this stage, the main document command |\childdocmain|
% is assumed to be called once again where it should do nothing.
% Any subsequent call to it should prevent
% a secondary processing of the main document
% It overwrites the forwarding commands
% |\childdocof| and |\childdocforward|
% with empty macros to prevent further inclusions of the main document:
%    \begin{macrocode}
\newcommand{\childdocdisable}
{
  \renewcommand{\childdocmain}[1]{\renewcommand{\childdocmain}[1]{\endinput}}
  \renewcommand{\childdocof}[1]{}
  \renewcommand{\childdocby}[2][]{}
  \renewcommand{\childdocforward}[2][]{}
  \renewcommand{\childdocdisable}{}
}
%    \end{macrocode}

% \macro{\childdocmain}
% The macro |\childdocmain| is to be called at the top of the main file
% with nothing or the main filename (without extension) as argument.
% First, it breaks loops.
% If the argument is not empty and does not match |\childdocname|
% (which is set by the first inclusion of |childdoc.def|),
% |\ifchilddoc| is set to true, |\includeonly| is applied to the child file
% and |\jobname| is set to the main file
% (for proper handling of |.aux| files):
%    \begin{macrocode}
\newcommand{\childdocmain}[1]
{
  \childdocdisable\childdocmain{}
  \if?#1?\else
    \begingroup
      \def\childdoctmp{#1}
      \ifx\childdoctmp\childdocname
        \def\childdoctmp{}
      \else
        \def\childdoctmp
        {
          \childdoctrue
          \includeonly{\childdocname}
          \def\childdocjob{#1}
          \def\jobname{#1}
        }
      \fi
      \expandafter
    \endgroup
    \childdoctmp
  \fi
}
%    \end{macrocode}

% \macro{\childdocof}
% The command |\childdocof| redirects
% compilation to the main file |#1|.
%    \begin{macrocode}
\newcommand{\childdocof}[1]
{
  \childdocdisable
  \childdoctrue
  \includeonly{\childdocname}
  \def\jobname{#1}
  \def\childdocjob{#1}
  \input{#1}
}
%    \end{macrocode}

% \macro{\childdocby}
% The command |\childdocby| ....
%    \begin{macrocode}
\newcommand{\childdocby}[2][]
{
  \childdocdisable
  \childdoctrue
  \childdocmanualtrue
  \if?#1?\else
    \def\jobname{#2}
  \fi
  \def\childdocjob{#2}
  \input{#2}
  \endinput
}
%    \end{macrocode}

% \macro{\childdocforward}
% The command |\childdocforward| redirects
% compilation to the main file or
% (if the optional argument is given) a child file.
% Parameters are set as if the main file
% or a child file starting with |\childdocof| was compiled.
% Then compilation is handed over to the main file:
%    \begin{macrocode}
\newcommand{\childdocforward}[2][]
{
  \begingroup
    \if?#1?
      \def\childdoctmp
      {
        \def\childdocname{#2}
        \def\childdocjob{#2}
        \def\jobname{#2}
        \input{#2}
        \endinput
      }
    \else
      \def\childdoctmp
      {
        \childdocdisable
        \def\childdocname{#2}
        \childdoctrue
        \includeonly{#2}
        \def\childdocjob{#1}
        \def\jobname{#1}
        \input{#1}
        \endinput
      }
    \fi
    \expandafter
  \endgroup
  \childdoctmp
}
%    \end{macrocode}

% \macro{\childdocforwardprefix}
% The command |\childdocforwardprefix| redirects
% compilation to the main or a child file by means of a pattern.
% The prefix |#1| in the current filename is replaced by |#2|
% and the suffix of the current filename is kept
% (it is assumed that the filename does not contain the substring `|~~~|'
% which is used as a delimiter).
% Compilation is handed over to the new file by |\childdocforward|:
%    \begin{macrocode}
\newcommand{\childdocforwardprefix}[3][]
{
  \begingroup
    \def\childdocextract #2##1~~~{\def\childdoctmp{\childdocforward[#1]{#3##1}}}
    \expandafter\childdocextract\childdocname~~~
    \expandafter
  \endgroup
  \childdoctmp
}
%    \end{macrocode}

% \macro{\childdoc}
% The deprecated macro |\childdoc| is a legacy version of |\childdocmain|:
%    \begin{macrocode}
\newcommand{\childdoc}{\childdocmain}
%    \end{macrocode}

% \macro{\childdocredirect}
% The deprecated macro |\childdocredirect| is a legacy version
% of |\childdocforward| and |\childdocforwardprefix|:
%    \begin{macrocode}
\newcommand{\childdocredirect}[2][]
{
  \begingroup
    \if?#1?
      \def\childdoctmp{\childdocforward{#2}}
    \else
      \def\childdoctmp{\childdocforwardprefix{#1}{#2}}
    \fi
    \expandafter
  \endgroup
  \childdoctmp
}
%    \end{macrocode}

%\iffalse
%</package>
%\fi
%
\endinput
|\\
|\childdocby{|\textit{main}|}|\\
\end{tabular}
\end{center}
%
Both forms have slightly different effects as described above.
The main file is prepared as usual, see \secref{sec:include}.

%%%%%%%%%%%%%%%%%%%%%%%%%%%%%%%%%%%%%%%%%%%%%%%%%%%%%%%%%%%%%%%%%%%%%%%%%%%%%%%%
\subsection{Legacy Detection}
\label{sec:detection}

The directive |\childdocmain| in the main file can detect
whether the complete document or merely a child is to be compiled
even without using the directive |\childdocof|.
This method is deprecated because it is less robust
and there is no compelling reason to use it;
it is merely provided for backward compatibility
and it may be removed in future versions.

If the detection mechanism is to be used,
it is mandatory to correctly specify
the filename of the main file as the argument of |\childdocmain|:
%
\begin{center}
\begin{tabular}{l}
|% \iffalse
%
% childdoc.dtx Copyright (C) 2017-2018 Niklas Beisert
%
% This work may be distributed and/or modified under the
% conditions of the LaTeX Project Public License, either version 1.3
% of this license or (at your option) any later version.
% The latest version of this license is in
%   http://www.latex-project.org/lppl.txt
% and version 1.3 or later is part of all distributions of LaTeX
% version 2005/12/01 or later.
%
% This work has the LPPL maintenance status `maintained'.
%
% The Current Maintainer of this work is Niklas Beisert.
%
% This work consists of the files childdoc.dtx and childdoc.ins
% and the derived files childdoc.def and cdocsamp.tex with
% cdocsch1.tex, cdocsch2.tex, cdocsdrf.tex, cdocsfn1.tex, cdocsfn2.tex.
%
%<package>\ifdefined\childdocmain\endinput\fi
%<package>\ProvidesFile{childdoc.def}[2018/12/30 v2.0 child document driver]
%<samplemain>\ProvidesFile{cdocsamp.tex}[2018/12/30 v2.0 sample for childdoc]
%<*driver>
%\ProvidesFile{childdoc.drv}[2018/12/30 v2.0 childdoc reference manual file]
\PassOptionsToClass{10pt,a4paper}{article}
\documentclass{ltxdoc}

\usepackage[margin=35mm]{geometry}
\usepackage{hyperref}
\usepackage{hyperxmp}
\usepackage[usenames]{color}

\hypersetup{colorlinks=true}
\hypersetup{pdfstartview=FitH}
\hypersetup{pdfpagemode=UseNone}
\hypersetup{pdfsource={}}
\hypersetup{pdflang={en-UK}}
\hypersetup{pdfcopyright={Copyright 2017-2018 Niklas Beisert.
  This work may be distributed and/or modified under the
  conditions of the LaTeX Project Public License, either version 1.3
  of this license or (at your option) any later version.}}
\hypersetup{pdflicenseurl={http://www.latex-project.org/lppl.txt}}
\hypersetup{pdfcontactaddress={ETH Zurich, ITP, HIT K,
  Wolfgang-Pauli-Strasse 27}}
\hypersetup{pdfcontactpostcode={8093}}
\hypersetup{pdfcontactcity={Zurich}}
\hypersetup{pdfcontactcountry={Switzerland}}
\hypersetup{pdfcontactemail={nbeisert@itp.phys.ethz.ch}}
\hypersetup{pdfcontacturl={http://people.phys.ethz.ch/\xmptilde nbeisert/}}

\newcommand{\secref}[1]{\hyperref[#1]{section \ref*{#1}}}

\parskip1ex
\parindent0pt
\let\olditemize\itemize
\def\itemize{\olditemize\parskip0pt}

\begin{document}

\title{The \textsf{childdoc} Package}
\hypersetup{pdftitle={The childdoc Package}}
\author{Niklas Beisert\\[2ex]
  Institut f\"ur Theoretische Physik\\
  Eidgen\"ossische Technische Hochschule Z\"urich\\
  Wolfgang-Pauli-Strasse 27, 8093 Z\"urich, Switzerland\\[1ex]
  \href{mailto:nbeisert@itp.phys.ethz.ch}
  {\texttt{nbeisert@itp.phys.ethz.ch}}}
\hypersetup{pdfauthor={Niklas Beisert}}
\hypersetup{pdfsubject={Manual for the LaTeX2e Package childdoc}}
\date{30 December 2018, \textsf{v2.0}}
\maketitle

\begin{abstract}\noindent
\textsf{childdoc} is a \LaTeXe{} package
that enables the direct compilation
of document sections included by |\include|
to individual files.
\end{abstract}

\begingroup
\parskip0ex
\tableofcontents
\endgroup

%%%%%%%%%%%%%%%%%%%%%%%%%%%%%%%%%%%%%%%%%%%%%%%%%%%%%%%%%%%%%%%%%%%%%%%%%%%%%%%%
%%%%%%%%%%%%%%%%%%%%%%%%%%%%%%%%%%%%%%%%%%%%%%%%%%%%%%%%%%%%%%%%%%%%%%%%%%%%%%%%
\section{Introduction}

\LaTeX{} provides a mechanism to structure a large document (such as a book)
into a main file and several child files (containing the chapters)
using the |\include| command.
This mechanism is beneficial for documents
which span hundreds of pages in order to
make the source file(s) more manageable.
Moreover, compilation can be restricted to
selected child files by means of the |\includeonly| command.
The latter feature can be used to reduce the compilation time while editing
(this was significantly more useful in the earlier days of \LaTeX{})
or to generate a smaller document which is easier to navigate.
Another application of |\includeonly| is to generate
documents consisting of selected parts of the complete document.

However, there are a few drawbacks of the plain |\include| mechanism:
\begin{itemize}
\item
The child files cannot be compiled on their own,
they can only be compiled via the main file.
A naive editing environment
(such as a text editor with an option
to have the current file processed by \LaTeX)
may require one to switch to the main file before compiling;
attempting to compile the child file produces errors.
\item
The main file must be modified (each time)
to adjust the |\includeonly| command
to the present needs. This easily leaves the main file in a messy state.
\item
The generated document will always carry the filename
of the main document. This is inconvenient if
several child files are to be compiled and
to be kept for distribution.
\end{itemize}

The present package provides a simple interface
to make child files individually compilable by \LaTeX{}.
Compiling a child file then has the same effect as compiling
the main file with an |\includeonly| command
to select the appropriate child.
Moreover the generated document will carry the name of the child
rather than the main file.
This resolves all three above issues.

This feature is meant to make the editing of books,
thesis documents and lecture notes somewhat more convenient.
However, the package can also be used efficiently for
composing a series of documents (such as exercise sheets)
which are typically distributed individually.
It then assists the author in generating the individual documents
(potentially in different versions)
as well as a document containing the collected series.
Another application is in developing style files
or other kinds of included material
where compilation of the style file could redirect
to a sample or test file.

%%%%%%%%%%%%%%%%%%%%%%%%%%%%%%%%%%%%%%%%%%%%%%%%%%%%%%%%%%%%%%%%%%%%%%%%%%%%%%%%
%%%%%%%%%%%%%%%%%%%%%%%%%%%%%%%%%%%%%%%%%%%%%%%%%%%%%%%%%%%%%%%%%%%%%%%%%%%%%%%%
\section{Usage}

First of all, the package \textsf{childdoc} is \emph{not} a standard
\LaTeXe{} |.sty| style file! Therefore it needs to be invoked in
a non-standard way.

%%%%%%%%%%%%%%%%%%%%%%%%%%%%%%%%%%%%%%%%%%%%%%%%%%%%%%%%%%%%%%%%%%%%%%%%%%%%%%%%
\subsection{Included Files}
\label{sec:include}

%%%%%%%%%%%%%%%%%%%%%%%%%%%%%%%%%%%%%%%%
\DescribeMacro{\childdocmain}
To use the package, add the commands
\begin{center}
\begin{tabular}{l}
|\input{childdoc.def}|\\
|\childdocmain{}|\\
\end{tabular}
\end{center}
at the very top of the main \LaTeX{} file,
in particular \emph{before} the |\documentclass| statement!
The argument of |\childdocmain| should be left empty
(but it must be present).

%%%%%%%%%%%%%%%%%%%%%%%%%%%%%%%%%%%%%%%%
\DescribeMacro{\childdocof}
Furthermore, add the commands
\begin{center}
\begin{tabular}{l}
|\input{childdoc.def}|\\
|\childdocof{|\textit{main}|}|\\
\end{tabular}
\end{center}
at the top of every child file \textit{child}
which is included by |\include{|\textit{child}|}|
from within the main file
(or at least for those files to be compiled individually).
The argument \textit{main} must be the filename of the main file.

There are a couple of
considerations in setting up the main and child documents:

%%%%%%%%%%%%%%%%%%%%%%%%%%%%%%%%%%%%%%%%
\paragraph{Restrictions.}

Please note the following restrictions:
\begin{itemize}
\item
|\childdocmain| must be called with one argument \textit{main}
to ensure compatibility with earlier version of the package.
It must either be empty (|\childdocmain{}|)
or precisely match the filename of the main file in which it is specified.
See \secref{sec:detection} for further information.
\item
The filename \textit{main} must be specified without the |.tex| extension.
\item
The filename \textit{main} is case sensitive
(even in case-insensitive file systems)
due to internal string comparison.
\item
The argument \textit{main} should be fully expanded, it cannot be a macro.
\item
Subdirectories and special characters should be avoided in filenames.
\item
The command |\childdocmain{|\textit{main}|}| must be followed by a whitespace.
It should not be followed immediately by another command
or by a comment mark `|%|'.
This is because the \TeX{} parser reads the token immediately following
the argument of |\childdocmain| and puts it
at the beginning of every child section;
however, a white\-space is ignored.
\end{itemize}

%%%%%%%%%%%%%%%%%%%%%%%%%%%%%%%%%%%%%%%%
\paragraph{Content of Main File.}

It is advisable to place all content in the child files included by |\include|.
Any output contained in the main file will appear in all child documents
unless suppressed manually;
it cannot be suppressed automatically by the |\includeonly| directive
and thus should normally be avoided.
A method to include some content in the main file
by means of conditional processing is described in \secref{sec:conditional}.

%%%%%%%%%%%%%%%%%%%%%%%%%%%%%%%%%%%%%%%%
\paragraph{Page Numbering.}

When only a part of the document is compiled,
the appropriate numbering of pages
(as well as other status parameters)
is determined from the |.aux| files.
The latter contain information from previous passes.
However this information needs to propagate through
all intermediate child documents.
Therefore the page numbering in child documents may well
be inconsistent until the complete document is compiled at least once.

A useful (if unconventional) way to always ensure a consistent
page numbering is to restart the numbering in each child document
and denote the pages by `\textit{child}|.|\textit{page}'
where \textit{child} represents the chapter/section number of the child file.
This can be achieved by the command
|\numberwithin{page}{|\textit{child}|}|
of the \textsf{amsmath} package
where \textit{child} can be |chapter| or |section|
depending on the chosen structuring.
Alternatively, one can modify the macro |\thepage| appropriately
and reset the counter |page| at the start of each child file.

%%%%%%%%%%%%%%%%%%%%%%%%%%%%%%%%%%%%%%%%%%%%%%%%%%%%%%%%%%%%%%%%%%%%%%%%%%%%%%%%
\subsection{Conditional Processing}
\label{sec:conditional}

The package provides a mechanism to compile different versions
of a document. To customise the versions further some conditional processing
can come in handy to distinguish which version is being compiled.
The package provides two macros to describe the compilation context:

%%%%%%%%%%%%%%%%%%%%%%%%%%%%%%%%%%%%%%%%
\DescribeMacro{\ifchilddoc}
The conditional |\ifchilddoc| distinguishes between the compilation of
child documents and the main document:
%
\begin{center}
|\ifchilddoc |\textit{child-code}| |[|\||else |\textit{main-code}]| \||fi|
\end{center}

%%%%%%%%%%%%%%%%%%%%%%%%%%%%%%%%%%%%%%%%
\DescribeMacro{\childdocname}
\DescribeMacro{\childdocjob}
The macro |\childdocname| contains the filename (without extension)
of the main or child file being processed.
Note that |\childdocjob| will always contain the name of the main file.

%%%%%%%%%%%%%%%%%%%%%%%%%%%%%%%%%%%%%%%%
\paragraph{Title Page.}

Conditional processing can be used to include a title or banner page
in the main document when proper precautions are taken.
Importantly, the code in the main file should ensure that the page counter
(as well as other status parameters which are stored in the |.aux| files)
takes the same value after the conditional processing.
Otherwise the page numbers may take divergent values
depending on which part is compiled.

For example, a title page could be declared by:
%
\begin{center}
\begin{tabular}{l}
|\ifchilddoc\||else|\\
|\addtocounter{page}{-1}|\\
\textit{code for title page}\\
|\newpage|\\
|\||fi|
\end{tabular}
\end{center}
%
A banner page for the child documents can be generated by:
%
\begin{center}
\begin{tabular}{l}
|\ifchilddoc|\\
|\addtocounter{page}{-1}|\\
\textit{code for banner page}\\
|\newpage|\\
|\||fi|
\end{tabular}
\end{center}
%
Here one could write a message such as:
\begin{center}
|This is the part \childdocname{} of \childdocjob{}.|
\end{center}

%%%%%%%%%%%%%%%%%%%%%%%%%%%%%%%%%%%%%%%%%%%%%%%%%%%%%%%%%%%%%%%%%%%%%%%%%%%%%%%%
\subsection{Flags}
\label{sec:flags}

The package makes it easy to generate different versions
of the main or child documents.
To this end compilation flags can be defined
and assigned different default values.
They will be particularly useful in conjunction
with the forwarding mechanism described in \secref{sec:forward}.

For example, it may be useful to have a flag |\version|
which can be set to |draft| or |final|.
The document source will contain some conditional code
depending on the value of |\version|.
Suppose further, the flag should default to |final| for the main file
and to |draft| for child files
which is a natural assignment for editing the document.
This is achieved by placing the following code
in the preamble of the main document
(below the |\childdocmain| directive):
%
\begin{center}
\begin{tabular}{l}
|\ifchilddoc|\\
|\providecommand{\version}{draft}|\\
|\||else|\\
|\providecommand{\version}{final}|\\
|\||fi|
\end{tabular}
\end{center}
%
The definition by |\providecommand| makes sure
that previous definitions are not overwritten.
Further statements |\providecommand{\version}{...}|
can thus be added before the above code to override it.

For the main file, one might add a line
(between |\childdocmain| and the above block)
%
\begin{center}
|%\ifchilddoc\||else\providecommand{\version}{draft}\||fi|
\end{center}
%
which can be uncommented to produce a draft version.
Likewise one can add a line to the very top of a child file
(above the |\childdocof{|\textit{main}|}| directive)
%
\begin{center}
|%\providecommand{\version}{final}|
\end{center}
%
which can be uncommented to produce the final version of this child document.

%%%%%%%%%%%%%%%%%%%%%%%%%%%%%%%%%%%%%%%%%%%%%%%%%%%%%%%%%%%%%%%%%%%%%%%%%%%%%%%%
\subsection{Forwarding}
\label{sec:forward}

Different versions of the main or child documents
using compilation flags as described in \secref{sec:flags}
can be (permanently) stored in different files
for convenient compilation, viewing and distribution.
To this end, the package defines a command
to pass on compilation to a different file:

%%%%%%%%%%%%%%%%%%%%%%%%%%%%%%%%%%%%%%%%
\DescribeMacro{\childdocforward}
The command |\childdocforward| redirects processing to
another source file:
%
\begin{center}
\begin{tabular}{l}
|\input{childdoc.def}|\\
|\childdocforward[|\textit{main}|]{|\textit{dest}|}|\\
\end{tabular}
\end{center}
%
The argument \textit{dest} is the destination file
(without extension).
It should be the main file or one of the child files.
Note that further \textsf{childdoc} directives
such as |\childdocof| and |\childdocforward|
in the indicated file will be processed in this form.
The optional argument \textit{main}
passes on directly to the main file \textit{main}
while pretending to compile the child \textit{dest}.
This form behaves as if \textit{dest}
issues |\childdocof{|\textit{main}|}| right away,
and no further \textsf{childdoc} directives will be processed.

%%%%%%%%%%%%%%%%%%%%%%%%%%%%%%%%%%%%%%%%
\DescribeMacro{\...prefix}
In the alternative form |\childdocforwardprefix|,
%
\begin{center}
\begin{tabular}{l}
|\input{childdoc.def}|\\
|\childdocforwardprefix[|\textit{main}|]{|\textit{prefix}|}{|\textit{dest}|}|
\end{tabular}
\end{center}
%
the destination file is determined by a pattern
depending on the current file:
To make this work, the current file must be called
`{\textit{prefix}\hspace{0.2em}\textit{suffix}}'
with \textit{prefix} matching precisely the argument.
Processing is then passed on to the file
`{\textit{dest}\hspace{0.2em}\textit{suffix}}'.
Surely, the same effect is achieved by
directly specifying the
argument `{\textit{dest}\hspace{0.2em}\textit{suffix}}'
in the first form.
However, that requires to set up a different file
for each child. With the alternative form of the command
all these files can have exactly the same content
which simplifies setting them up and maintaining them.

For example, the following file |draft.tex|
with a compilation flag |\version| as described in \secref{sec:flags}
compiles the main document as a draft:
%
\begin{center}
\begin{tabular}{l}
|\def\version{draft}|\\
|\input{childdoc.def}|\\
|\childdocforward{|\textit{main}|}|
\end{tabular}
\end{center}
%
Likewise, the following files |final|\textit{nn}|.tex|
compile the final version of the child document
|child|\textit{nn}|.tex|:
%
\begin{center}
\begin{tabular}{l}
|\def\version{final}|\\
|\input{childdoc.def}|\\
|\childdocforwardprefix{final}{child}|
\end{tabular}
\end{center}
%

Note that when several versions of a main file and/or of each child file
are to be generated, it may be convenient to set up a |Makefile| or
shell script to automatise the process.

%%%%%%%%%%%%%%%%%%%%%%%%%%%%%%%%%%%%%%%%%%%%%%%%%%%%%%%%%%%%%%%%%%%%%%%%%%%%%%%%
\subsection{Command Line Processing}
\label{sec:commandline}

The effect of redirection files can also be achieved by invoking
the \LaTeX{} compiler with a more elaborate command line.
Most conveniently this should be done as part
of a shell script or a |Makefile|.

When using \textsf{childdoc} in the main file, the following
command lines effectively perform a redirection
(note that depending on the shell being used,
backslashes may have to be doubled: `|\|' $\to$ `|\\|'):
%
\begin{center}
|... -jobname "|\textit{target}|" |\\|"|[\textit{flags}]%
|\input{childdoc.def}\childdocforward[|\textit{main}|]{|\textit{dest}|}"|
\end{center}
%
Here \textit{target} is the name of the output file,
\textit{main} is the name of the main file
and \textit{dest} is the name of the main or child file to be processed
(all filenames without extensions).
The optional argument \textit{main} can be omitted
if \textit{main} matches \textit{dest}.
Optionally, compilation \textit{flags} can be defined via |\def| commands.
This command line makes the \TeX{} engine believe
it is compiling the file \textit{target}
whose content is specified as the latter parameter.
The provided code then forwards the processing to
\textit{main} or \textit{dest} as described in \secref{sec:forward}.

%%%%%%%%%%%%%%%%%%%%%%%%%%%%%%%%%%%%%%%%%%%%%%%%%%%%%%%%%%%%%%%%%%%%%%%%%%%%%%%%
\subsection{Include by Input}
\label{sec:input}

Including child documents by |\include| has some restrictions by design.
Most notably, the content of a child document always occupies
its own set of pages; pages cannot be shared between child documents.
Usually, this behaviour makes perfect sense
because each child document contain an essential part of the document.
However, in some situations it may be desirable to compose
a document from a collection of parts
without having mandatory page breaks between then.
For this case, the package
provides a mechanism to include parts
by |\input| which can also be processed individually.
However, by construction this mechanism
requires manual handling of the content to be output.

%%%%%%%%%%%%%%%%%%%%%%%%%%%%%%%%%%%%%%%%
\DescribeMacro{\ifchilddocmanual}
The main file should be prepared as usual, see \secref{sec:include}.
However, the document body must make a distinction
between processing of an individual part and of the main document, e.g.:
%
\begin{center}
\begin{tabular}{l}
|\ifchilddocmanual|\\
|\input{\childdocname}|\\
|\||else|\\
\textit{document body with }|\input{|\textit{part}|}|\\
|\||fi|
\end{tabular}
\end{center}
%
The conditional |\ifchilddocmanual| is true whenever
a part to be included by |\input| is being compiled,
and the name of the part is stored in |\childdocname|.

%%%%%%%%%%%%%%%%%%%%%%%%%%%%%%%%%%%%%%%%
\DescribeMacro{\childdocby}
Each part to be included by |\input| should start with:
%
\begin{center}
\begin{tabular}{l}
|\input{childdoc.def}|\\
|\childdocby{|\textit{main}|}|\\
\end{tabular}
\end{center}
%
The directive |\childdocby| is similar to |\childdocof|
described in \secref{sec:include},
but the subsequent selection of content must be done manually.
To that end, both |\ifchilddoc| and |\ifchilddocmanual|
will be true upon processing of a part,
and the name of the part is stored in |\childdocname|.
Note that |\jobname| will be set to the filename of the current part
so that each part receives an individual |.aux| file
that does not interfere with the |.aux| file(s) of the main document.
This behaviour can be altered by the alternative form
|\childdocby[*]{|\textit{main}|}| (with a non-empty optional argument)
which uses the |.aux| file of the main document
by setting |\jobname| to \textit{main}.

%%%%%%%%%%%%%%%%%%%%%%%%%%%%%%%%%%%%%%%%%%%%%%%%%%%%%%%%%%%%%%%%%%%%%%%%%%%%%%%%
\subsection{Driver Development}
\label{sec:driver}

The \textsf{childdoc} mechanism can also be use for the development
of definition files such as \LaTeX{} styles or classes.
This case differs from the above setup with multiple parts
included by |\include| in that no |\includeonly| should be invoked.
This can be achieved by starting the include file
(before |\ProvidesPackage|) with:
%
\begin{center}
\begin{tabular}{l}
|\input{childdoc.def}|\\
|\childdocforward{|\textit{main}|}|\\
\end{tabular}
\end{center}
%
or alternatively with:
%
\begin{center}
\begin{tabular}{l}
|\input{childdoc.def}|\\
|\childdocby{|\textit{main}|}|\\
\end{tabular}
\end{center}
%
Both forms have slightly different effects as described above.
The main file is prepared as usual, see \secref{sec:include}.

%%%%%%%%%%%%%%%%%%%%%%%%%%%%%%%%%%%%%%%%%%%%%%%%%%%%%%%%%%%%%%%%%%%%%%%%%%%%%%%%
\subsection{Legacy Detection}
\label{sec:detection}

The directive |\childdocmain| in the main file can detect
whether the complete document or merely a child is to be compiled
even without using the directive |\childdocof|.
This method is deprecated because it is less robust
and there is no compelling reason to use it;
it is merely provided for backward compatibility
and it may be removed in future versions.

If the detection mechanism is to be used,
it is mandatory to correctly specify
the filename of the main file as the argument of |\childdocmain|:
%
\begin{center}
\begin{tabular}{l}
|\input{childdoc.def}|\\
|\childdocmain{|\textit{main}|}|\\
\end{tabular}
\end{center}
%
If |\jobname| does not match the argument \textit{main} of |\childdocmain|,
it is assumed that |\jobname| points to the child file to be compiled.
When using |\childdocmain| with the main file specified as argument,
it suffices to start a child file
with just |\input{|\textit{main}|}|
without loading of the package and using |\childdocof|.
If instead all processing is done
with the appropriate \textsf{childdoc} directives,
the argument of \textit{main} of |\childdocmain| can be empty.

An alternative version of the command line processing described
in \secref{sec:commandline} using the detection mechanism reads:
%
\begin{center}
|... -jobname "|\textit{target}|" "|[\textit{flags}]%
[|\def\jobname{|\textit{dest}|}|]|\input{|\textit{main}|}"|
\end{center}

%%%%%%%%%%%%%%%%%%%%%%%%%%%%%%%%%%%%%%%%%%%%%%%%%%%%%%%%%%%%%%%%%%%%%%%%%%%%%%%%
\subsection{Manual Code}
\label{sec:manual}

In case one cannot be certain whether the definitions file |childdoc.def|
is installed on the target \TeX{} distribution
and one prefers not to ship it,
it is conceivable to paste a few relevant commands into the sources.

To that end, drop all statements |\input{childdoc.def}|
and perform the replacements as outlined below.
Instead of |\childdocmain{|\textit{main}|}| add the following code
to the top of the main file:
%
\begin{center}
\begin{tabular}{l}
|\||ifdefined\childdocname\endinput\||fi\newif\ifchilddoc|\\
|\edef\childdocname{\scantokens\expandafter{\jobname\noexpand}}|\\
|\def\childdocmain{|\textit{main}|}\||ifx\childdocmain\childdocname\||else|\\
|\childdoctrue\includeonly{\childdocname}\let\jobname\childdocmain\||fi|\\
\end{tabular}
\end{center}
%
Instead of |\childdocof{|\textit{main}|}| just include the main file
at the top of each child file:
%
\begin{center}
|\input{|\textit{main}|}|
\end{center}
%
A simple redirection |\childdocforward{|\textit{dest}|}| is achieved by:
%
\begin{center}
|\def\jobname{|\textit{dest}|}\input{\jobname}|
\end{center}
%
The redirection with prefix
|\childdocforwardprefix[|\textit{prefix}|]{|\textit{dest}|}|
is accomplished by:
%
\begin{center}
\begin{tabular}{l}
|{\edef\jobname{\scantokens\expandafter{\jobname\noexpand}}|\\
|\def\redirectjob |\textit{prefix}|#1~~~{\gdef\jobname{|\textit{dest}|#1}}|\\
|\expandafter\redirectjob\jobname~~~}\input{\jobname}|
\end{tabular}
\end{center}

In an alternative approach,
child documents can be compiled by a specific command line
without additional code or specific definitions:
%
\begin{center}
|... -jobname "|\textit{target}|" "|[\textit{flags}]%
|\includeonly{|\textit{dest}|}\input{|\textit{main}|}"|
\end{center}
%

%%%%%%%%%%%%%%%%%%%%%%%%%%%%%%%%%%%%%%%%%%%%%%%%%%%%%%%%%%%%%%%%%%%%%%%%%%%%%%%%
%%%%%%%%%%%%%%%%%%%%%%%%%%%%%%%%%%%%%%%%%%%%%%%%%%%%%%%%%%%%%%%%%%%%%%%%%%%%%%%%
\section{Information}

%%%%%%%%%%%%%%%%%%%%%%%%%%%%%%%%%%%%%%%%%%%%%%%%%%%%%%%%%%%%%%%%%%%%%%%%%%%%%%%%
\subsection{Copyright}

Copyright \copyright{} 2017--2018 Niklas Beisert

This work may be distributed and/or modified under the
conditions of the \LaTeX{} Project Public License, either version 1.3
of this license or (at your option) any later version.
The latest version of this license is in
  \url{http://www.latex-project.org/lppl.txt}
and version 1.3 or later is part of all distributions of \LaTeX{}
version 2005/12/01 or later.

This work has the LPPL maintenance status `maintained'.

The Current Maintainer of this work is Niklas Beisert.

This work consists of the files |README.txt|, |childdoc.ins| and |childdoc.dtx|
as well as the derived files |childdoc.def|, |cdocsamp.tex|
with |cdocsch1.tex|, |cdocsch2.tex|, |cdocspt3.tex|, |cdocspt4.tex|,
|cdocsdrf.tex|, |cdocsfn1.tex|, |cdocsfn2.tex|
as well as |childdoc.pdf|.

%%%%%%%%%%%%%%%%%%%%%%%%%%%%%%%%%%%%%%%%%%%%%%%%%%%%%%%%%%%%%%%%%%%%%%%%%%%%%%%%
\subsection{Files and Installation}

The package consists of the files:
%
\begin{center}
\begin{tabular}{ll}
    |README.txt|   & readme file \\
    |childdoc.ins| & installation file \\
    |childdoc.dtx| & source file \\
    |childdoc.def| & definition file \\
    |cdocsamp.tex| & sample main file \\
    |cdocsch1.tex| & sample include file \\
    |cdocsch2.tex| & sample include file \\
    |cdocspt3.tex| & sample part file \\
    |cdocspt4.tex| & sample part file \\
    |cdocsdrf.tex| & sample redirection file \\
    |cdocsfn1.tex| & sample redirection file \\
    |cdocsfn2.tex| & sample redirection file \\
    |childdoc.pdf| & manual
\end{tabular}
\end{center}
%
The distribution consists of the files
|README.txt|, |childdoc.ins| and |childdoc.dtx|.
%
\begin{itemize}
\item
Run (pdf)\LaTeX{} on |childdoc.dtx|
to compile the manual |childdoc.pdf| (this file).
\item
Run \LaTeX{} on |childdoc.ins| to create the definitions file |childdoc.def|
and the sample |cdocsamp.tex| with include files
|cdocsch1.tex|, |cdocsch2.tex|, |cdocspt3.tex|, |cdocspt4.tex|,
|cdocsdrf.tex|, |cdocsfn1.tex|, |cdocsfn2.tex|.
Then copy the file |childdoc.def| to an appropriate directory of your \LaTeX{}
distribution, e.g.\ \textit{texmf-root}|/tex/latex/childdoc|.
\end{itemize}

%%%%%%%%%%%%%%%%%%%%%%%%%%%%%%%%%%%%%%%%%%%%%%%%%%%%%%%%%%%%%%%%%%%%%%%%%%%%%%%%
\subsection{Related CTAN Packages}

There are several other packages which offer a similar functionality:
%
\begin{itemize}
\item
The packages
\href{http://ctan.org/pkg/docmute}{\textsf{docmute}},
\href{http://ctan.org/pkg/includex}{\textsf{includex}} and
\href{http://ctan.org/pkg/standalone}{\textsf{standalone}}
provide commands to include only the document body of
a child file thus allowing both files to be compiled individually.
\item
The packages \href{http://ctan.org/pkg/subdocs}{\textsf{subdocs}}
and \href{http://ctan.org/pkg/subfiles}{\textsf{subfiles}}
provide structures in which the main and child documents can be
encapsulated and allowing them to be compiled individually.
The inclusion mechanism is different from the conventional |\include|.
\item
The package \href{http://ctan.org/pkg/combine}{\textsf{combine}}
is an elaborate solution to combine several documents into one.
\end{itemize}
%
See also the CTAN topic \href{http://ctan.org/topic/subdocs}{\textsf{subdocs}}
for further related packages.
The present package differs from the above solutions in that
a document structure constructed with the conventional |\include| mechanism
just needs two extra commands at the top of every file
such that all constituent files can be compiled individually.

%%%%%%%%%%%%%%%%%%%%%%%%%%%%%%%%%%%%%%%%%%%%%%%%%%%%%%%%%%%%%%%%%%%%%%%%%%%%%%%%
%\subsection{Feature Suggestions}
%
%The following is a list of features which may be useful for future
%versions of this package:
%%
%\begin{itemize}
%\item
%\ldots
%\end{itemize}

%%%%%%%%%%%%%%%%%%%%%%%%%%%%%%%%%%%%%%%%%%%%%%%%%%%%%%%%%%%%%%%%%%%%%%%%%%%%%%%%
\subsection{Revision History}

%%%%%%%%%%%%%%%%%%%%%%%%%%%%%%%%%%%%%%%%
\paragraph{v2.0:} 2018/12/30

\begin{itemize}
\item
immediate forward processing
\item
added |\childdocby| mechanism
\item
manual restructured
\end{itemize}

%%%%%%%%%%%%%%%%%%%%%%%%%%%%%%%%%%%%%%%%
\paragraph{v1.6:} 2018/01/17

\begin{itemize}
\item
application for development of include files
\item
corrections to manual
\end{itemize}

%%%%%%%%%%%%%%%%%%%%%%%%%%%%%%%%%%%%%%%%
\paragraph{v1.5:} 2017/05/21

\begin{itemize}
\item
more complete structuring introduced
\item
|\childdocof| introduced
\item
|\childdoc| renamed to |\childdocmain|
\item
|\childredirect| renamed to |\childdocforward| and |\childdocforwardprefix|
and functionality expanded
\end{itemize}

%%%%%%%%%%%%%%%%%%%%%%%%%%%%%%%%%%%%%%%%
\paragraph{v1.0:} 2017/04/27

\begin{itemize}
\item
manual and install package
\item
first version published on CTAN
\end{itemize}

%%%%%%%%%%%%%%%%%%%%%%%%%%%%%%%%%%%%%%%%
\paragraph{v0.6:} 2017/04/26

\begin{itemize}
\item
redirection mechanism added
\end{itemize}

%%%%%%%%%%%%%%%%%%%%%%%%%%%%%%%%%%%%%%%%
\paragraph{v0.5:} 2017/04/26

\begin{itemize}
\item
functionality in definition file
\end{itemize}


%%%%%%%%%%%%%%%%%%%%%%%%%%%%%%%%%%%%%%%%%%%%%%%%%%%%%%%%%%%%%%%%%%%%%%%%%%%%%%%%
%%%%%%%%%%%%%%%%%%%%%%%%%%%%%%%%%%%%%%%%%%%%%%%%%%%%%%%%%%%%%%%%%%%%%%%%%%%%%%%%
%%%%%%%%%%%%%%%%%%%%%%%%%%%%%%%%%%%%%%%%%%%%%%%%%%%%%%%%%%%%%%%%%%%%%%%%%%%%%%%%
\appendix

\settowidth\MacroIndent{\rmfamily\scriptsize 000\ }

 \DocInput{childdoc.dtx}

\end{document}
%</driver>
% \fi
%
% %%%%%%%%%%%%%%%%%%%%%%%%%%%%%%%%%%%%%%%%%%%%%%%%%%%%%%%%%%%%%%%%%%%%%%%%%%%%%%
% %%%%%%%%%%%%%%%%%%%%%%%%%%%%%%%%%%%%%%%%%%%%%%%%%%%%%%%%%%%%%%%%%%%%%%%%%%%%%%
% \section{Sample}
%\iffalse
%<*samplemain>
%\fi
%
% The following presents a sample document
% with two chapters, two parts, a title page,
% a compile flag as well as three forwarding files to set the flag.
% It consists of eight |.tex| files:
% \begin{center}
% \begin{tabular}{ll}
% |cdocsamp.tex|&main file\\
% |cdocsch1.tex|&include file for chapter 1\\
% |cdocsch2.tex|&include file for chapter 2\\
% |cdocspt3.tex|&include file for part 3\\
% |cdocspt4.tex|&include file for part 4\\
% |cdocsdrf.tex|&forwarding file for main file in draft mode\\
% |cdocsfi1.tex|&forwarding file for final version of chapter 1\\
% |cdocsfi2.tex|&forwarding file for final version of chapter 2\\
% \end{tabular}
% \end{center}
% Each of the eight files can be compiled directly by the \LaTeX{} compiler.
%
% %%%%%%%%%%%%%%%%%%%%%%%%%%%%%%%%%%%%%%
% \paragraph{Main File.}
%
% The main file is called |cdocsamp.tex|.
%
% Load the \textsf{childdoc} definitions and
% declare the filename for the main document:
%    \begin{macrocode}
\input{childdoc.def}
\childdocmain{}
%    \end{macrocode}

% Optional override for |\version| flag:
%    \begin{macrocode}
%%\ifchilddoc\else\providecommand{\version}{draft}\fi
%    \end{macrocode}

% Define the default values for the |\version| flag
% (|final| for the main file and |draft| for childs):
%    \begin{macrocode}
\ifchilddoc
\providecommand{\version}{draft}
\else
\providecommand{\version}{final}
\fi
%    \end{macrocode}

% Load the standard document class:
%    \begin{macrocode}
\documentclass[12pt]{article}
%    \end{macrocode}

% Start the document body:
%    \begin{macrocode}
\begin{document}
%    \end{macrocode}

% Declare a title page.
% Print title, part of document being processed and version flag:
%    \begin{macrocode}
\addtocounter{page}{-1}
\begin{center}
{\LARGE\bfseries{}childdoc example\par}
\vspace{1cm}
\ifchilddoc
\ifchilddocmanual part\else chapter\fi:
`\childdocname' of `\childdocjob'\par
\else
main document: `\childdocjob'\par
\fi
version: \version\par
\end{center}
\newpage
%    \end{macrocode}

% Manually include selected file,
% otherwise process as usual:
%    \begin{macrocode}
\ifchilddocmanual
\section*{part `\childdocname'}
\input{\childdocname}
\else
%    \end{macrocode}

% Include the two chapters:
%    \begin{macrocode}
\include{cdocsch1}
\include{cdocsch2}
%    \end{macrocode}

% Include the two parts unless only chapters should be displayed:
%    \begin{macrocode}
\ifchilddoc\else
\section{part three}
\input{cdocspt3}
\section{part four}
\input{cdocspt4}
\fi
%    \end{macrocode}

% Process as usual until here:
%    \begin{macrocode}
\fi
%    \end{macrocode}

% End of document body:
%    \begin{macrocode}
\end{document}
%    \end{macrocode}
%\iffalse
%</samplemain>
%\fi
%
% %%%%%%%%%%%%%%%%%%%%%%%%%%%%%%%%%%%%%%
% \paragraph{Chapter Include Files.}
%
% The include files are called |cdocsch1.tex| and |cdocsch2.tex|.
%
%\iffalse
%<*samplechap1|samplechap2>
%\fi

% Optional override for |\version| flag:
%    \begin{macrocode}
%%\providecommand{\version}{final}
%    \end{macrocode}

% Include the main document:
%    \begin{macrocode}
\input{childdoc.def}
\childdocof{cdocsamp}
%    \end{macrocode}

%\iffalse
%</samplechap1|samplechap2>
%\fi
%
%\iffalse
%<*samplechap1>
%\fi
% Some text for chapter 1:
%    \begin{macrocode}
\section{one}
some text in chapter one
%    \end{macrocode}

%\iffalse
%</samplechap1>
%\fi
% Some text for chapter 2:
%\iffalse
%<*samplechap2>
%\fi
%    \begin{macrocode}
\section{two}
more text in chapter two
%    \end{macrocode}

%\iffalse
%</samplechap2>
%\fi
%
% %%%%%%%%%%%%%%%%%%%%%%%%%%%%%%%%%%%%%%
% \paragraph{Part Include Files.}
%
% The include files are called |cdocspt3.tex| and |cdocspt4.tex|.
%
%\iffalse
%<*samplepart3|samplepart4>
%\fi

% Optional override for |\version| flag:
%    \begin{macrocode}
%%\providecommand{\version}{final}
%    \end{macrocode}

% Include the main document:
%    \begin{macrocode}
\input{childdoc.def}
\childdocby{cdocsamp}
%    \end{macrocode}

%\iffalse
%</samplepart3|samplepart4>
%\fi
%
%\iffalse
%<*samplepart3>
%\fi
% Some text for part 3:
%    \begin{macrocode}
some text in part three
%    \end{macrocode}

%\iffalse
%</samplepart3>
%\fi
% Some text for part 4:
%\iffalse
%<*samplepart4>
%\fi
%    \begin{macrocode}
more text in part four
%    \end{macrocode}

%\iffalse
%</samplepart4>
%\fi
%
% %%%%%%%%%%%%%%%%%%%%%%%%%%%%%%%%%%%%%%
% \paragraph{Forwarding for a Complete Draft.}
%
% The following forwarding file |cdocsdrf.tex|
% compiles the main document in draft mode:
%\iffalse
%<*sampledraft>
%\fi
%    \begin{macrocode}
\def\version{draft}
\input{childdoc.def}
\childdocforward{cdocsamp}
%    \end{macrocode}

%\iffalse
%</sampledraft>
%\fi
%
% %%%%%%%%%%%%%%%%%%%%%%%%%%%%%%%%%%%%%%
% \paragraph{Forwarding for Final Version of the Chapters.}
%
% The following forwarding files |cdocsfn1.tex| and |cdocsfn2.tex|
% (with identical content)
% compile the final versions of the child documents
% |cdocsch1.tex| and |cdocsch2.tex|, respectively:
%\iffalse
%<*samplefinal>
%\fi
%    \begin{macrocode}
\def\version{final}
\input{childdoc.def}
\childdocforwardprefix[cdocsamp]{cdocsfn}{cdocsch}
%    \end{macrocode}

%\iffalse
%</samplefinal>
%\fi
%
% %%%%%%%%%%%%%%%%%%%%%%%%%%%%%%%%%%%%%%
% \paragraph{Command Line Processing.}
%
% The following three command lines generate the output files
% |cdocscld|, |cdocscl1| and |cdocscl2|
% which should be identical to
% |cdocsdrf|, |cdocsch1| and |cdocsfn2|, respectively:
% \begin{center}
% \begin{tabular}{l}
% |latex -jobname cdocscld \|\\
% |  "\def\version{draft}\input{childdoc.def}\childdocforward{cdocsamp}"|\\
% |latex -jobname cdocscl1 \|\\
% |  "\input{childdoc.def}\childdocforward[cdocsamp]{cdocsch1}"|\\
% |latex -jobname cdocscl2 \|\\
% |  "\def\version{final}\input{childdoc.def}\childdocforward{cdocsch2}"|
% \end{tabular}
% \end{center}
% Note that the trailing backslash on each first line
% merely continues the input to the second line
% (for convenient cut ant paste).
% Furthermore, the command |latex| can be replaced by any
% of its alternative versions such as |pdflatex|.
%
% %%%%%%%%%%%%%%%%%%%%%%%%%%%%%%%%%%%%%%%%%%%%%%%%%%%%%%%%%%%%%%%%%%%%%%%%%%%%%%
% %%%%%%%%%%%%%%%%%%%%%%%%%%%%%%%%%%%%%%%%%%%%%%%%%%%%%%%%%%%%%%%%%%%%%%%%%%%%%%
% \section{Implementation}
%\iffalse
%<*package>
%\fi
%
% This section describes the definitions file |childdoc.def|.

% The definitions cannot be loaded using |\usepackage| or |\RequirePackage|
% which has a mechanism to prevent loading a style file more than once.
% When loading the definitions by means of |\input|
% multiple instances have to be prevented manually:
%\iffalse
%This code needs to be before the `\ProvidesFile' directive
%which is defined at the beginning of this file.
%Therefore it is also placed there and commented out here.
%</package>
%<*discard>
%\fi
%    \begin{macrocode}
\ifdefined\childdocmain\endinput\fi
%    \end{macrocode}
%\iffalse
%</discard>
%<*package>
%\fi
%
% \macro{\ifchilddoc}
% \macro{\ifchilddocmanual}
% The conditional |\ifchilddoc| tells whether a
% child (true) or main (false) document is being compiled.
% The conditional |\ifchilddocmanual| tells whether
% the |\includeonly| mechanism is used (false) or
% the selection of child files must be performed manually (true).
% The definitions initialise to false:
%    \begin{macrocode}
\newif\ifchilddoc
\newif\ifchilddocmanual
%    \end{macrocode}

% \macro{\childdocname}
% \macro{\childdocjob}
% The macro |\childdocname| stores the name of the main document
% to be compiled. The macro |\childdocjob| stores the name of
% the document on which the \LaTeX{} compiler was originally invoked.
% The content of |\jobname| cannot be compared
% to filenames specified in the source due to different catcodes.
% The following code rescans |\jobname|, stores the result
% in |\childdocname| and saves a copy in |\childdocjob|:
%    \begin{macrocode}
\edef\childdocname{\scantokens\expandafter{\jobname\noexpand}}
\let\childdocjob\childdocname
%    \end{macrocode}

% \macro{\childdocdisable}
% The macro |\childdocdisable| prevents the main file
% from being processed more than once.
% At this stage, the main document command |\childdocmain|
% is assumed to be called once again where it should do nothing.
% Any subsequent call to it should prevent
% a secondary processing of the main document
% It overwrites the forwarding commands
% |\childdocof| and |\childdocforward|
% with empty macros to prevent further inclusions of the main document:
%    \begin{macrocode}
\newcommand{\childdocdisable}
{
  \renewcommand{\childdocmain}[1]{\renewcommand{\childdocmain}[1]{\endinput}}
  \renewcommand{\childdocof}[1]{}
  \renewcommand{\childdocby}[2][]{}
  \renewcommand{\childdocforward}[2][]{}
  \renewcommand{\childdocdisable}{}
}
%    \end{macrocode}

% \macro{\childdocmain}
% The macro |\childdocmain| is to be called at the top of the main file
% with nothing or the main filename (without extension) as argument.
% First, it breaks loops.
% If the argument is not empty and does not match |\childdocname|
% (which is set by the first inclusion of |childdoc.def|),
% |\ifchilddoc| is set to true, |\includeonly| is applied to the child file
% and |\jobname| is set to the main file
% (for proper handling of |.aux| files):
%    \begin{macrocode}
\newcommand{\childdocmain}[1]
{
  \childdocdisable\childdocmain{}
  \if?#1?\else
    \begingroup
      \def\childdoctmp{#1}
      \ifx\childdoctmp\childdocname
        \def\childdoctmp{}
      \else
        \def\childdoctmp
        {
          \childdoctrue
          \includeonly{\childdocname}
          \def\childdocjob{#1}
          \def\jobname{#1}
        }
      \fi
      \expandafter
    \endgroup
    \childdoctmp
  \fi
}
%    \end{macrocode}

% \macro{\childdocof}
% The command |\childdocof| redirects
% compilation to the main file |#1|.
%    \begin{macrocode}
\newcommand{\childdocof}[1]
{
  \childdocdisable
  \childdoctrue
  \includeonly{\childdocname}
  \def\jobname{#1}
  \def\childdocjob{#1}
  \input{#1}
}
%    \end{macrocode}

% \macro{\childdocby}
% The command |\childdocby| ....
%    \begin{macrocode}
\newcommand{\childdocby}[2][]
{
  \childdocdisable
  \childdoctrue
  \childdocmanualtrue
  \if?#1?\else
    \def\jobname{#2}
  \fi
  \def\childdocjob{#2}
  \input{#2}
  \endinput
}
%    \end{macrocode}

% \macro{\childdocforward}
% The command |\childdocforward| redirects
% compilation to the main file or
% (if the optional argument is given) a child file.
% Parameters are set as if the main file
% or a child file starting with |\childdocof| was compiled.
% Then compilation is handed over to the main file:
%    \begin{macrocode}
\newcommand{\childdocforward}[2][]
{
  \begingroup
    \if?#1?
      \def\childdoctmp
      {
        \def\childdocname{#2}
        \def\childdocjob{#2}
        \def\jobname{#2}
        \input{#2}
        \endinput
      }
    \else
      \def\childdoctmp
      {
        \childdocdisable
        \def\childdocname{#2}
        \childdoctrue
        \includeonly{#2}
        \def\childdocjob{#1}
        \def\jobname{#1}
        \input{#1}
        \endinput
      }
    \fi
    \expandafter
  \endgroup
  \childdoctmp
}
%    \end{macrocode}

% \macro{\childdocforwardprefix}
% The command |\childdocforwardprefix| redirects
% compilation to the main or a child file by means of a pattern.
% The prefix |#1| in the current filename is replaced by |#2|
% and the suffix of the current filename is kept
% (it is assumed that the filename does not contain the substring `|~~~|'
% which is used as a delimiter).
% Compilation is handed over to the new file by |\childdocforward|:
%    \begin{macrocode}
\newcommand{\childdocforwardprefix}[3][]
{
  \begingroup
    \def\childdocextract #2##1~~~{\def\childdoctmp{\childdocforward[#1]{#3##1}}}
    \expandafter\childdocextract\childdocname~~~
    \expandafter
  \endgroup
  \childdoctmp
}
%    \end{macrocode}

% \macro{\childdoc}
% The deprecated macro |\childdoc| is a legacy version of |\childdocmain|:
%    \begin{macrocode}
\newcommand{\childdoc}{\childdocmain}
%    \end{macrocode}

% \macro{\childdocredirect}
% The deprecated macro |\childdocredirect| is a legacy version
% of |\childdocforward| and |\childdocforwardprefix|:
%    \begin{macrocode}
\newcommand{\childdocredirect}[2][]
{
  \begingroup
    \if?#1?
      \def\childdoctmp{\childdocforward{#2}}
    \else
      \def\childdoctmp{\childdocforwardprefix{#1}{#2}}
    \fi
    \expandafter
  \endgroup
  \childdoctmp
}
%    \end{macrocode}

%\iffalse
%</package>
%\fi
%
\endinput
|\\
|\childdocmain{|\textit{main}|}|\\
\end{tabular}
\end{center}
%
If |\jobname| does not match the argument \textit{main} of |\childdocmain|,
it is assumed that |\jobname| points to the child file to be compiled.
When using |\childdocmain| with the main file specified as argument,
it suffices to start a child file
with just |\input{|\textit{main}|}|
without loading of the package and using |\childdocof|.
If instead all processing is done
with the appropriate \textsf{childdoc} directives,
the argument of \textit{main} of |\childdocmain| can be empty.

An alternative version of the command line processing described
in \secref{sec:commandline} using the detection mechanism reads:
%
\begin{center}
|... -jobname "|\textit{target}|" "|[\textit{flags}]%
[|\def\jobname{|\textit{dest}|}|]|\input{|\textit{main}|}"|
\end{center}

%%%%%%%%%%%%%%%%%%%%%%%%%%%%%%%%%%%%%%%%%%%%%%%%%%%%%%%%%%%%%%%%%%%%%%%%%%%%%%%%
\subsection{Manual Code}
\label{sec:manual}

In case one cannot be certain whether the definitions file |childdoc.def|
is installed on the target \TeX{} distribution
and one prefers not to ship it,
it is conceivable to paste a few relevant commands into the sources.

To that end, drop all statements |% \iffalse
%
% childdoc.dtx Copyright (C) 2017-2018 Niklas Beisert
%
% This work may be distributed and/or modified under the
% conditions of the LaTeX Project Public License, either version 1.3
% of this license or (at your option) any later version.
% The latest version of this license is in
%   http://www.latex-project.org/lppl.txt
% and version 1.3 or later is part of all distributions of LaTeX
% version 2005/12/01 or later.
%
% This work has the LPPL maintenance status `maintained'.
%
% The Current Maintainer of this work is Niklas Beisert.
%
% This work consists of the files childdoc.dtx and childdoc.ins
% and the derived files childdoc.def and cdocsamp.tex with
% cdocsch1.tex, cdocsch2.tex, cdocsdrf.tex, cdocsfn1.tex, cdocsfn2.tex.
%
%<package>\ifdefined\childdocmain\endinput\fi
%<package>\ProvidesFile{childdoc.def}[2018/12/30 v2.0 child document driver]
%<samplemain>\ProvidesFile{cdocsamp.tex}[2018/12/30 v2.0 sample for childdoc]
%<*driver>
%\ProvidesFile{childdoc.drv}[2018/12/30 v2.0 childdoc reference manual file]
\PassOptionsToClass{10pt,a4paper}{article}
\documentclass{ltxdoc}

\usepackage[margin=35mm]{geometry}
\usepackage{hyperref}
\usepackage{hyperxmp}
\usepackage[usenames]{color}

\hypersetup{colorlinks=true}
\hypersetup{pdfstartview=FitH}
\hypersetup{pdfpagemode=UseNone}
\hypersetup{pdfsource={}}
\hypersetup{pdflang={en-UK}}
\hypersetup{pdfcopyright={Copyright 2017-2018 Niklas Beisert.
  This work may be distributed and/or modified under the
  conditions of the LaTeX Project Public License, either version 1.3
  of this license or (at your option) any later version.}}
\hypersetup{pdflicenseurl={http://www.latex-project.org/lppl.txt}}
\hypersetup{pdfcontactaddress={ETH Zurich, ITP, HIT K,
  Wolfgang-Pauli-Strasse 27}}
\hypersetup{pdfcontactpostcode={8093}}
\hypersetup{pdfcontactcity={Zurich}}
\hypersetup{pdfcontactcountry={Switzerland}}
\hypersetup{pdfcontactemail={nbeisert@itp.phys.ethz.ch}}
\hypersetup{pdfcontacturl={http://people.phys.ethz.ch/\xmptilde nbeisert/}}

\newcommand{\secref}[1]{\hyperref[#1]{section \ref*{#1}}}

\parskip1ex
\parindent0pt
\let\olditemize\itemize
\def\itemize{\olditemize\parskip0pt}

\begin{document}

\title{The \textsf{childdoc} Package}
\hypersetup{pdftitle={The childdoc Package}}
\author{Niklas Beisert\\[2ex]
  Institut f\"ur Theoretische Physik\\
  Eidgen\"ossische Technische Hochschule Z\"urich\\
  Wolfgang-Pauli-Strasse 27, 8093 Z\"urich, Switzerland\\[1ex]
  \href{mailto:nbeisert@itp.phys.ethz.ch}
  {\texttt{nbeisert@itp.phys.ethz.ch}}}
\hypersetup{pdfauthor={Niklas Beisert}}
\hypersetup{pdfsubject={Manual for the LaTeX2e Package childdoc}}
\date{30 December 2018, \textsf{v2.0}}
\maketitle

\begin{abstract}\noindent
\textsf{childdoc} is a \LaTeXe{} package
that enables the direct compilation
of document sections included by |\include|
to individual files.
\end{abstract}

\begingroup
\parskip0ex
\tableofcontents
\endgroup

%%%%%%%%%%%%%%%%%%%%%%%%%%%%%%%%%%%%%%%%%%%%%%%%%%%%%%%%%%%%%%%%%%%%%%%%%%%%%%%%
%%%%%%%%%%%%%%%%%%%%%%%%%%%%%%%%%%%%%%%%%%%%%%%%%%%%%%%%%%%%%%%%%%%%%%%%%%%%%%%%
\section{Introduction}

\LaTeX{} provides a mechanism to structure a large document (such as a book)
into a main file and several child files (containing the chapters)
using the |\include| command.
This mechanism is beneficial for documents
which span hundreds of pages in order to
make the source file(s) more manageable.
Moreover, compilation can be restricted to
selected child files by means of the |\includeonly| command.
The latter feature can be used to reduce the compilation time while editing
(this was significantly more useful in the earlier days of \LaTeX{})
or to generate a smaller document which is easier to navigate.
Another application of |\includeonly| is to generate
documents consisting of selected parts of the complete document.

However, there are a few drawbacks of the plain |\include| mechanism:
\begin{itemize}
\item
The child files cannot be compiled on their own,
they can only be compiled via the main file.
A naive editing environment
(such as a text editor with an option
to have the current file processed by \LaTeX)
may require one to switch to the main file before compiling;
attempting to compile the child file produces errors.
\item
The main file must be modified (each time)
to adjust the |\includeonly| command
to the present needs. This easily leaves the main file in a messy state.
\item
The generated document will always carry the filename
of the main document. This is inconvenient if
several child files are to be compiled and
to be kept for distribution.
\end{itemize}

The present package provides a simple interface
to make child files individually compilable by \LaTeX{}.
Compiling a child file then has the same effect as compiling
the main file with an |\includeonly| command
to select the appropriate child.
Moreover the generated document will carry the name of the child
rather than the main file.
This resolves all three above issues.

This feature is meant to make the editing of books,
thesis documents and lecture notes somewhat more convenient.
However, the package can also be used efficiently for
composing a series of documents (such as exercise sheets)
which are typically distributed individually.
It then assists the author in generating the individual documents
(potentially in different versions)
as well as a document containing the collected series.
Another application is in developing style files
or other kinds of included material
where compilation of the style file could redirect
to a sample or test file.

%%%%%%%%%%%%%%%%%%%%%%%%%%%%%%%%%%%%%%%%%%%%%%%%%%%%%%%%%%%%%%%%%%%%%%%%%%%%%%%%
%%%%%%%%%%%%%%%%%%%%%%%%%%%%%%%%%%%%%%%%%%%%%%%%%%%%%%%%%%%%%%%%%%%%%%%%%%%%%%%%
\section{Usage}

First of all, the package \textsf{childdoc} is \emph{not} a standard
\LaTeXe{} |.sty| style file! Therefore it needs to be invoked in
a non-standard way.

%%%%%%%%%%%%%%%%%%%%%%%%%%%%%%%%%%%%%%%%%%%%%%%%%%%%%%%%%%%%%%%%%%%%%%%%%%%%%%%%
\subsection{Included Files}
\label{sec:include}

%%%%%%%%%%%%%%%%%%%%%%%%%%%%%%%%%%%%%%%%
\DescribeMacro{\childdocmain}
To use the package, add the commands
\begin{center}
\begin{tabular}{l}
|\input{childdoc.def}|\\
|\childdocmain{}|\\
\end{tabular}
\end{center}
at the very top of the main \LaTeX{} file,
in particular \emph{before} the |\documentclass| statement!
The argument of |\childdocmain| should be left empty
(but it must be present).

%%%%%%%%%%%%%%%%%%%%%%%%%%%%%%%%%%%%%%%%
\DescribeMacro{\childdocof}
Furthermore, add the commands
\begin{center}
\begin{tabular}{l}
|\input{childdoc.def}|\\
|\childdocof{|\textit{main}|}|\\
\end{tabular}
\end{center}
at the top of every child file \textit{child}
which is included by |\include{|\textit{child}|}|
from within the main file
(or at least for those files to be compiled individually).
The argument \textit{main} must be the filename of the main file.

There are a couple of
considerations in setting up the main and child documents:

%%%%%%%%%%%%%%%%%%%%%%%%%%%%%%%%%%%%%%%%
\paragraph{Restrictions.}

Please note the following restrictions:
\begin{itemize}
\item
|\childdocmain| must be called with one argument \textit{main}
to ensure compatibility with earlier version of the package.
It must either be empty (|\childdocmain{}|)
or precisely match the filename of the main file in which it is specified.
See \secref{sec:detection} for further information.
\item
The filename \textit{main} must be specified without the |.tex| extension.
\item
The filename \textit{main} is case sensitive
(even in case-insensitive file systems)
due to internal string comparison.
\item
The argument \textit{main} should be fully expanded, it cannot be a macro.
\item
Subdirectories and special characters should be avoided in filenames.
\item
The command |\childdocmain{|\textit{main}|}| must be followed by a whitespace.
It should not be followed immediately by another command
or by a comment mark `|%|'.
This is because the \TeX{} parser reads the token immediately following
the argument of |\childdocmain| and puts it
at the beginning of every child section;
however, a white\-space is ignored.
\end{itemize}

%%%%%%%%%%%%%%%%%%%%%%%%%%%%%%%%%%%%%%%%
\paragraph{Content of Main File.}

It is advisable to place all content in the child files included by |\include|.
Any output contained in the main file will appear in all child documents
unless suppressed manually;
it cannot be suppressed automatically by the |\includeonly| directive
and thus should normally be avoided.
A method to include some content in the main file
by means of conditional processing is described in \secref{sec:conditional}.

%%%%%%%%%%%%%%%%%%%%%%%%%%%%%%%%%%%%%%%%
\paragraph{Page Numbering.}

When only a part of the document is compiled,
the appropriate numbering of pages
(as well as other status parameters)
is determined from the |.aux| files.
The latter contain information from previous passes.
However this information needs to propagate through
all intermediate child documents.
Therefore the page numbering in child documents may well
be inconsistent until the complete document is compiled at least once.

A useful (if unconventional) way to always ensure a consistent
page numbering is to restart the numbering in each child document
and denote the pages by `\textit{child}|.|\textit{page}'
where \textit{child} represents the chapter/section number of the child file.
This can be achieved by the command
|\numberwithin{page}{|\textit{child}|}|
of the \textsf{amsmath} package
where \textit{child} can be |chapter| or |section|
depending on the chosen structuring.
Alternatively, one can modify the macro |\thepage| appropriately
and reset the counter |page| at the start of each child file.

%%%%%%%%%%%%%%%%%%%%%%%%%%%%%%%%%%%%%%%%%%%%%%%%%%%%%%%%%%%%%%%%%%%%%%%%%%%%%%%%
\subsection{Conditional Processing}
\label{sec:conditional}

The package provides a mechanism to compile different versions
of a document. To customise the versions further some conditional processing
can come in handy to distinguish which version is being compiled.
The package provides two macros to describe the compilation context:

%%%%%%%%%%%%%%%%%%%%%%%%%%%%%%%%%%%%%%%%
\DescribeMacro{\ifchilddoc}
The conditional |\ifchilddoc| distinguishes between the compilation of
child documents and the main document:
%
\begin{center}
|\ifchilddoc |\textit{child-code}| |[|\||else |\textit{main-code}]| \||fi|
\end{center}

%%%%%%%%%%%%%%%%%%%%%%%%%%%%%%%%%%%%%%%%
\DescribeMacro{\childdocname}
\DescribeMacro{\childdocjob}
The macro |\childdocname| contains the filename (without extension)
of the main or child file being processed.
Note that |\childdocjob| will always contain the name of the main file.

%%%%%%%%%%%%%%%%%%%%%%%%%%%%%%%%%%%%%%%%
\paragraph{Title Page.}

Conditional processing can be used to include a title or banner page
in the main document when proper precautions are taken.
Importantly, the code in the main file should ensure that the page counter
(as well as other status parameters which are stored in the |.aux| files)
takes the same value after the conditional processing.
Otherwise the page numbers may take divergent values
depending on which part is compiled.

For example, a title page could be declared by:
%
\begin{center}
\begin{tabular}{l}
|\ifchilddoc\||else|\\
|\addtocounter{page}{-1}|\\
\textit{code for title page}\\
|\newpage|\\
|\||fi|
\end{tabular}
\end{center}
%
A banner page for the child documents can be generated by:
%
\begin{center}
\begin{tabular}{l}
|\ifchilddoc|\\
|\addtocounter{page}{-1}|\\
\textit{code for banner page}\\
|\newpage|\\
|\||fi|
\end{tabular}
\end{center}
%
Here one could write a message such as:
\begin{center}
|This is the part \childdocname{} of \childdocjob{}.|
\end{center}

%%%%%%%%%%%%%%%%%%%%%%%%%%%%%%%%%%%%%%%%%%%%%%%%%%%%%%%%%%%%%%%%%%%%%%%%%%%%%%%%
\subsection{Flags}
\label{sec:flags}

The package makes it easy to generate different versions
of the main or child documents.
To this end compilation flags can be defined
and assigned different default values.
They will be particularly useful in conjunction
with the forwarding mechanism described in \secref{sec:forward}.

For example, it may be useful to have a flag |\version|
which can be set to |draft| or |final|.
The document source will contain some conditional code
depending on the value of |\version|.
Suppose further, the flag should default to |final| for the main file
and to |draft| for child files
which is a natural assignment for editing the document.
This is achieved by placing the following code
in the preamble of the main document
(below the |\childdocmain| directive):
%
\begin{center}
\begin{tabular}{l}
|\ifchilddoc|\\
|\providecommand{\version}{draft}|\\
|\||else|\\
|\providecommand{\version}{final}|\\
|\||fi|
\end{tabular}
\end{center}
%
The definition by |\providecommand| makes sure
that previous definitions are not overwritten.
Further statements |\providecommand{\version}{...}|
can thus be added before the above code to override it.

For the main file, one might add a line
(between |\childdocmain| and the above block)
%
\begin{center}
|%\ifchilddoc\||else\providecommand{\version}{draft}\||fi|
\end{center}
%
which can be uncommented to produce a draft version.
Likewise one can add a line to the very top of a child file
(above the |\childdocof{|\textit{main}|}| directive)
%
\begin{center}
|%\providecommand{\version}{final}|
\end{center}
%
which can be uncommented to produce the final version of this child document.

%%%%%%%%%%%%%%%%%%%%%%%%%%%%%%%%%%%%%%%%%%%%%%%%%%%%%%%%%%%%%%%%%%%%%%%%%%%%%%%%
\subsection{Forwarding}
\label{sec:forward}

Different versions of the main or child documents
using compilation flags as described in \secref{sec:flags}
can be (permanently) stored in different files
for convenient compilation, viewing and distribution.
To this end, the package defines a command
to pass on compilation to a different file:

%%%%%%%%%%%%%%%%%%%%%%%%%%%%%%%%%%%%%%%%
\DescribeMacro{\childdocforward}
The command |\childdocforward| redirects processing to
another source file:
%
\begin{center}
\begin{tabular}{l}
|\input{childdoc.def}|\\
|\childdocforward[|\textit{main}|]{|\textit{dest}|}|\\
\end{tabular}
\end{center}
%
The argument \textit{dest} is the destination file
(without extension).
It should be the main file or one of the child files.
Note that further \textsf{childdoc} directives
such as |\childdocof| and |\childdocforward|
in the indicated file will be processed in this form.
The optional argument \textit{main}
passes on directly to the main file \textit{main}
while pretending to compile the child \textit{dest}.
This form behaves as if \textit{dest}
issues |\childdocof{|\textit{main}|}| right away,
and no further \textsf{childdoc} directives will be processed.

%%%%%%%%%%%%%%%%%%%%%%%%%%%%%%%%%%%%%%%%
\DescribeMacro{\...prefix}
In the alternative form |\childdocforwardprefix|,
%
\begin{center}
\begin{tabular}{l}
|\input{childdoc.def}|\\
|\childdocforwardprefix[|\textit{main}|]{|\textit{prefix}|}{|\textit{dest}|}|
\end{tabular}
\end{center}
%
the destination file is determined by a pattern
depending on the current file:
To make this work, the current file must be called
`{\textit{prefix}\hspace{0.2em}\textit{suffix}}'
with \textit{prefix} matching precisely the argument.
Processing is then passed on to the file
`{\textit{dest}\hspace{0.2em}\textit{suffix}}'.
Surely, the same effect is achieved by
directly specifying the
argument `{\textit{dest}\hspace{0.2em}\textit{suffix}}'
in the first form.
However, that requires to set up a different file
for each child. With the alternative form of the command
all these files can have exactly the same content
which simplifies setting them up and maintaining them.

For example, the following file |draft.tex|
with a compilation flag |\version| as described in \secref{sec:flags}
compiles the main document as a draft:
%
\begin{center}
\begin{tabular}{l}
|\def\version{draft}|\\
|\input{childdoc.def}|\\
|\childdocforward{|\textit{main}|}|
\end{tabular}
\end{center}
%
Likewise, the following files |final|\textit{nn}|.tex|
compile the final version of the child document
|child|\textit{nn}|.tex|:
%
\begin{center}
\begin{tabular}{l}
|\def\version{final}|\\
|\input{childdoc.def}|\\
|\childdocforwardprefix{final}{child}|
\end{tabular}
\end{center}
%

Note that when several versions of a main file and/or of each child file
are to be generated, it may be convenient to set up a |Makefile| or
shell script to automatise the process.

%%%%%%%%%%%%%%%%%%%%%%%%%%%%%%%%%%%%%%%%%%%%%%%%%%%%%%%%%%%%%%%%%%%%%%%%%%%%%%%%
\subsection{Command Line Processing}
\label{sec:commandline}

The effect of redirection files can also be achieved by invoking
the \LaTeX{} compiler with a more elaborate command line.
Most conveniently this should be done as part
of a shell script or a |Makefile|.

When using \textsf{childdoc} in the main file, the following
command lines effectively perform a redirection
(note that depending on the shell being used,
backslashes may have to be doubled: `|\|' $\to$ `|\\|'):
%
\begin{center}
|... -jobname "|\textit{target}|" |\\|"|[\textit{flags}]%
|\input{childdoc.def}\childdocforward[|\textit{main}|]{|\textit{dest}|}"|
\end{center}
%
Here \textit{target} is the name of the output file,
\textit{main} is the name of the main file
and \textit{dest} is the name of the main or child file to be processed
(all filenames without extensions).
The optional argument \textit{main} can be omitted
if \textit{main} matches \textit{dest}.
Optionally, compilation \textit{flags} can be defined via |\def| commands.
This command line makes the \TeX{} engine believe
it is compiling the file \textit{target}
whose content is specified as the latter parameter.
The provided code then forwards the processing to
\textit{main} or \textit{dest} as described in \secref{sec:forward}.

%%%%%%%%%%%%%%%%%%%%%%%%%%%%%%%%%%%%%%%%%%%%%%%%%%%%%%%%%%%%%%%%%%%%%%%%%%%%%%%%
\subsection{Include by Input}
\label{sec:input}

Including child documents by |\include| has some restrictions by design.
Most notably, the content of a child document always occupies
its own set of pages; pages cannot be shared between child documents.
Usually, this behaviour makes perfect sense
because each child document contain an essential part of the document.
However, in some situations it may be desirable to compose
a document from a collection of parts
without having mandatory page breaks between then.
For this case, the package
provides a mechanism to include parts
by |\input| which can also be processed individually.
However, by construction this mechanism
requires manual handling of the content to be output.

%%%%%%%%%%%%%%%%%%%%%%%%%%%%%%%%%%%%%%%%
\DescribeMacro{\ifchilddocmanual}
The main file should be prepared as usual, see \secref{sec:include}.
However, the document body must make a distinction
between processing of an individual part and of the main document, e.g.:
%
\begin{center}
\begin{tabular}{l}
|\ifchilddocmanual|\\
|\input{\childdocname}|\\
|\||else|\\
\textit{document body with }|\input{|\textit{part}|}|\\
|\||fi|
\end{tabular}
\end{center}
%
The conditional |\ifchilddocmanual| is true whenever
a part to be included by |\input| is being compiled,
and the name of the part is stored in |\childdocname|.

%%%%%%%%%%%%%%%%%%%%%%%%%%%%%%%%%%%%%%%%
\DescribeMacro{\childdocby}
Each part to be included by |\input| should start with:
%
\begin{center}
\begin{tabular}{l}
|\input{childdoc.def}|\\
|\childdocby{|\textit{main}|}|\\
\end{tabular}
\end{center}
%
The directive |\childdocby| is similar to |\childdocof|
described in \secref{sec:include},
but the subsequent selection of content must be done manually.
To that end, both |\ifchilddoc| and |\ifchilddocmanual|
will be true upon processing of a part,
and the name of the part is stored in |\childdocname|.
Note that |\jobname| will be set to the filename of the current part
so that each part receives an individual |.aux| file
that does not interfere with the |.aux| file(s) of the main document.
This behaviour can be altered by the alternative form
|\childdocby[*]{|\textit{main}|}| (with a non-empty optional argument)
which uses the |.aux| file of the main document
by setting |\jobname| to \textit{main}.

%%%%%%%%%%%%%%%%%%%%%%%%%%%%%%%%%%%%%%%%%%%%%%%%%%%%%%%%%%%%%%%%%%%%%%%%%%%%%%%%
\subsection{Driver Development}
\label{sec:driver}

The \textsf{childdoc} mechanism can also be use for the development
of definition files such as \LaTeX{} styles or classes.
This case differs from the above setup with multiple parts
included by |\include| in that no |\includeonly| should be invoked.
This can be achieved by starting the include file
(before |\ProvidesPackage|) with:
%
\begin{center}
\begin{tabular}{l}
|\input{childdoc.def}|\\
|\childdocforward{|\textit{main}|}|\\
\end{tabular}
\end{center}
%
or alternatively with:
%
\begin{center}
\begin{tabular}{l}
|\input{childdoc.def}|\\
|\childdocby{|\textit{main}|}|\\
\end{tabular}
\end{center}
%
Both forms have slightly different effects as described above.
The main file is prepared as usual, see \secref{sec:include}.

%%%%%%%%%%%%%%%%%%%%%%%%%%%%%%%%%%%%%%%%%%%%%%%%%%%%%%%%%%%%%%%%%%%%%%%%%%%%%%%%
\subsection{Legacy Detection}
\label{sec:detection}

The directive |\childdocmain| in the main file can detect
whether the complete document or merely a child is to be compiled
even without using the directive |\childdocof|.
This method is deprecated because it is less robust
and there is no compelling reason to use it;
it is merely provided for backward compatibility
and it may be removed in future versions.

If the detection mechanism is to be used,
it is mandatory to correctly specify
the filename of the main file as the argument of |\childdocmain|:
%
\begin{center}
\begin{tabular}{l}
|\input{childdoc.def}|\\
|\childdocmain{|\textit{main}|}|\\
\end{tabular}
\end{center}
%
If |\jobname| does not match the argument \textit{main} of |\childdocmain|,
it is assumed that |\jobname| points to the child file to be compiled.
When using |\childdocmain| with the main file specified as argument,
it suffices to start a child file
with just |\input{|\textit{main}|}|
without loading of the package and using |\childdocof|.
If instead all processing is done
with the appropriate \textsf{childdoc} directives,
the argument of \textit{main} of |\childdocmain| can be empty.

An alternative version of the command line processing described
in \secref{sec:commandline} using the detection mechanism reads:
%
\begin{center}
|... -jobname "|\textit{target}|" "|[\textit{flags}]%
[|\def\jobname{|\textit{dest}|}|]|\input{|\textit{main}|}"|
\end{center}

%%%%%%%%%%%%%%%%%%%%%%%%%%%%%%%%%%%%%%%%%%%%%%%%%%%%%%%%%%%%%%%%%%%%%%%%%%%%%%%%
\subsection{Manual Code}
\label{sec:manual}

In case one cannot be certain whether the definitions file |childdoc.def|
is installed on the target \TeX{} distribution
and one prefers not to ship it,
it is conceivable to paste a few relevant commands into the sources.

To that end, drop all statements |\input{childdoc.def}|
and perform the replacements as outlined below.
Instead of |\childdocmain{|\textit{main}|}| add the following code
to the top of the main file:
%
\begin{center}
\begin{tabular}{l}
|\||ifdefined\childdocname\endinput\||fi\newif\ifchilddoc|\\
|\edef\childdocname{\scantokens\expandafter{\jobname\noexpand}}|\\
|\def\childdocmain{|\textit{main}|}\||ifx\childdocmain\childdocname\||else|\\
|\childdoctrue\includeonly{\childdocname}\let\jobname\childdocmain\||fi|\\
\end{tabular}
\end{center}
%
Instead of |\childdocof{|\textit{main}|}| just include the main file
at the top of each child file:
%
\begin{center}
|\input{|\textit{main}|}|
\end{center}
%
A simple redirection |\childdocforward{|\textit{dest}|}| is achieved by:
%
\begin{center}
|\def\jobname{|\textit{dest}|}\input{\jobname}|
\end{center}
%
The redirection with prefix
|\childdocforwardprefix[|\textit{prefix}|]{|\textit{dest}|}|
is accomplished by:
%
\begin{center}
\begin{tabular}{l}
|{\edef\jobname{\scantokens\expandafter{\jobname\noexpand}}|\\
|\def\redirectjob |\textit{prefix}|#1~~~{\gdef\jobname{|\textit{dest}|#1}}|\\
|\expandafter\redirectjob\jobname~~~}\input{\jobname}|
\end{tabular}
\end{center}

In an alternative approach,
child documents can be compiled by a specific command line
without additional code or specific definitions:
%
\begin{center}
|... -jobname "|\textit{target}|" "|[\textit{flags}]%
|\includeonly{|\textit{dest}|}\input{|\textit{main}|}"|
\end{center}
%

%%%%%%%%%%%%%%%%%%%%%%%%%%%%%%%%%%%%%%%%%%%%%%%%%%%%%%%%%%%%%%%%%%%%%%%%%%%%%%%%
%%%%%%%%%%%%%%%%%%%%%%%%%%%%%%%%%%%%%%%%%%%%%%%%%%%%%%%%%%%%%%%%%%%%%%%%%%%%%%%%
\section{Information}

%%%%%%%%%%%%%%%%%%%%%%%%%%%%%%%%%%%%%%%%%%%%%%%%%%%%%%%%%%%%%%%%%%%%%%%%%%%%%%%%
\subsection{Copyright}

Copyright \copyright{} 2017--2018 Niklas Beisert

This work may be distributed and/or modified under the
conditions of the \LaTeX{} Project Public License, either version 1.3
of this license or (at your option) any later version.
The latest version of this license is in
  \url{http://www.latex-project.org/lppl.txt}
and version 1.3 or later is part of all distributions of \LaTeX{}
version 2005/12/01 or later.

This work has the LPPL maintenance status `maintained'.

The Current Maintainer of this work is Niklas Beisert.

This work consists of the files |README.txt|, |childdoc.ins| and |childdoc.dtx|
as well as the derived files |childdoc.def|, |cdocsamp.tex|
with |cdocsch1.tex|, |cdocsch2.tex|, |cdocspt3.tex|, |cdocspt4.tex|,
|cdocsdrf.tex|, |cdocsfn1.tex|, |cdocsfn2.tex|
as well as |childdoc.pdf|.

%%%%%%%%%%%%%%%%%%%%%%%%%%%%%%%%%%%%%%%%%%%%%%%%%%%%%%%%%%%%%%%%%%%%%%%%%%%%%%%%
\subsection{Files and Installation}

The package consists of the files:
%
\begin{center}
\begin{tabular}{ll}
    |README.txt|   & readme file \\
    |childdoc.ins| & installation file \\
    |childdoc.dtx| & source file \\
    |childdoc.def| & definition file \\
    |cdocsamp.tex| & sample main file \\
    |cdocsch1.tex| & sample include file \\
    |cdocsch2.tex| & sample include file \\
    |cdocspt3.tex| & sample part file \\
    |cdocspt4.tex| & sample part file \\
    |cdocsdrf.tex| & sample redirection file \\
    |cdocsfn1.tex| & sample redirection file \\
    |cdocsfn2.tex| & sample redirection file \\
    |childdoc.pdf| & manual
\end{tabular}
\end{center}
%
The distribution consists of the files
|README.txt|, |childdoc.ins| and |childdoc.dtx|.
%
\begin{itemize}
\item
Run (pdf)\LaTeX{} on |childdoc.dtx|
to compile the manual |childdoc.pdf| (this file).
\item
Run \LaTeX{} on |childdoc.ins| to create the definitions file |childdoc.def|
and the sample |cdocsamp.tex| with include files
|cdocsch1.tex|, |cdocsch2.tex|, |cdocspt3.tex|, |cdocspt4.tex|,
|cdocsdrf.tex|, |cdocsfn1.tex|, |cdocsfn2.tex|.
Then copy the file |childdoc.def| to an appropriate directory of your \LaTeX{}
distribution, e.g.\ \textit{texmf-root}|/tex/latex/childdoc|.
\end{itemize}

%%%%%%%%%%%%%%%%%%%%%%%%%%%%%%%%%%%%%%%%%%%%%%%%%%%%%%%%%%%%%%%%%%%%%%%%%%%%%%%%
\subsection{Related CTAN Packages}

There are several other packages which offer a similar functionality:
%
\begin{itemize}
\item
The packages
\href{http://ctan.org/pkg/docmute}{\textsf{docmute}},
\href{http://ctan.org/pkg/includex}{\textsf{includex}} and
\href{http://ctan.org/pkg/standalone}{\textsf{standalone}}
provide commands to include only the document body of
a child file thus allowing both files to be compiled individually.
\item
The packages \href{http://ctan.org/pkg/subdocs}{\textsf{subdocs}}
and \href{http://ctan.org/pkg/subfiles}{\textsf{subfiles}}
provide structures in which the main and child documents can be
encapsulated and allowing them to be compiled individually.
The inclusion mechanism is different from the conventional |\include|.
\item
The package \href{http://ctan.org/pkg/combine}{\textsf{combine}}
is an elaborate solution to combine several documents into one.
\end{itemize}
%
See also the CTAN topic \href{http://ctan.org/topic/subdocs}{\textsf{subdocs}}
for further related packages.
The present package differs from the above solutions in that
a document structure constructed with the conventional |\include| mechanism
just needs two extra commands at the top of every file
such that all constituent files can be compiled individually.

%%%%%%%%%%%%%%%%%%%%%%%%%%%%%%%%%%%%%%%%%%%%%%%%%%%%%%%%%%%%%%%%%%%%%%%%%%%%%%%%
%\subsection{Feature Suggestions}
%
%The following is a list of features which may be useful for future
%versions of this package:
%%
%\begin{itemize}
%\item
%\ldots
%\end{itemize}

%%%%%%%%%%%%%%%%%%%%%%%%%%%%%%%%%%%%%%%%%%%%%%%%%%%%%%%%%%%%%%%%%%%%%%%%%%%%%%%%
\subsection{Revision History}

%%%%%%%%%%%%%%%%%%%%%%%%%%%%%%%%%%%%%%%%
\paragraph{v2.0:} 2018/12/30

\begin{itemize}
\item
immediate forward processing
\item
added |\childdocby| mechanism
\item
manual restructured
\end{itemize}

%%%%%%%%%%%%%%%%%%%%%%%%%%%%%%%%%%%%%%%%
\paragraph{v1.6:} 2018/01/17

\begin{itemize}
\item
application for development of include files
\item
corrections to manual
\end{itemize}

%%%%%%%%%%%%%%%%%%%%%%%%%%%%%%%%%%%%%%%%
\paragraph{v1.5:} 2017/05/21

\begin{itemize}
\item
more complete structuring introduced
\item
|\childdocof| introduced
\item
|\childdoc| renamed to |\childdocmain|
\item
|\childredirect| renamed to |\childdocforward| and |\childdocforwardprefix|
and functionality expanded
\end{itemize}

%%%%%%%%%%%%%%%%%%%%%%%%%%%%%%%%%%%%%%%%
\paragraph{v1.0:} 2017/04/27

\begin{itemize}
\item
manual and install package
\item
first version published on CTAN
\end{itemize}

%%%%%%%%%%%%%%%%%%%%%%%%%%%%%%%%%%%%%%%%
\paragraph{v0.6:} 2017/04/26

\begin{itemize}
\item
redirection mechanism added
\end{itemize}

%%%%%%%%%%%%%%%%%%%%%%%%%%%%%%%%%%%%%%%%
\paragraph{v0.5:} 2017/04/26

\begin{itemize}
\item
functionality in definition file
\end{itemize}


%%%%%%%%%%%%%%%%%%%%%%%%%%%%%%%%%%%%%%%%%%%%%%%%%%%%%%%%%%%%%%%%%%%%%%%%%%%%%%%%
%%%%%%%%%%%%%%%%%%%%%%%%%%%%%%%%%%%%%%%%%%%%%%%%%%%%%%%%%%%%%%%%%%%%%%%%%%%%%%%%
%%%%%%%%%%%%%%%%%%%%%%%%%%%%%%%%%%%%%%%%%%%%%%%%%%%%%%%%%%%%%%%%%%%%%%%%%%%%%%%%
\appendix

\settowidth\MacroIndent{\rmfamily\scriptsize 000\ }

 \DocInput{childdoc.dtx}

\end{document}
%</driver>
% \fi
%
% %%%%%%%%%%%%%%%%%%%%%%%%%%%%%%%%%%%%%%%%%%%%%%%%%%%%%%%%%%%%%%%%%%%%%%%%%%%%%%
% %%%%%%%%%%%%%%%%%%%%%%%%%%%%%%%%%%%%%%%%%%%%%%%%%%%%%%%%%%%%%%%%%%%%%%%%%%%%%%
% \section{Sample}
%\iffalse
%<*samplemain>
%\fi
%
% The following presents a sample document
% with two chapters, two parts, a title page,
% a compile flag as well as three forwarding files to set the flag.
% It consists of eight |.tex| files:
% \begin{center}
% \begin{tabular}{ll}
% |cdocsamp.tex|&main file\\
% |cdocsch1.tex|&include file for chapter 1\\
% |cdocsch2.tex|&include file for chapter 2\\
% |cdocspt3.tex|&include file for part 3\\
% |cdocspt4.tex|&include file for part 4\\
% |cdocsdrf.tex|&forwarding file for main file in draft mode\\
% |cdocsfi1.tex|&forwarding file for final version of chapter 1\\
% |cdocsfi2.tex|&forwarding file for final version of chapter 2\\
% \end{tabular}
% \end{center}
% Each of the eight files can be compiled directly by the \LaTeX{} compiler.
%
% %%%%%%%%%%%%%%%%%%%%%%%%%%%%%%%%%%%%%%
% \paragraph{Main File.}
%
% The main file is called |cdocsamp.tex|.
%
% Load the \textsf{childdoc} definitions and
% declare the filename for the main document:
%    \begin{macrocode}
\input{childdoc.def}
\childdocmain{}
%    \end{macrocode}

% Optional override for |\version| flag:
%    \begin{macrocode}
%%\ifchilddoc\else\providecommand{\version}{draft}\fi
%    \end{macrocode}

% Define the default values for the |\version| flag
% (|final| for the main file and |draft| for childs):
%    \begin{macrocode}
\ifchilddoc
\providecommand{\version}{draft}
\else
\providecommand{\version}{final}
\fi
%    \end{macrocode}

% Load the standard document class:
%    \begin{macrocode}
\documentclass[12pt]{article}
%    \end{macrocode}

% Start the document body:
%    \begin{macrocode}
\begin{document}
%    \end{macrocode}

% Declare a title page.
% Print title, part of document being processed and version flag:
%    \begin{macrocode}
\addtocounter{page}{-1}
\begin{center}
{\LARGE\bfseries{}childdoc example\par}
\vspace{1cm}
\ifchilddoc
\ifchilddocmanual part\else chapter\fi:
`\childdocname' of `\childdocjob'\par
\else
main document: `\childdocjob'\par
\fi
version: \version\par
\end{center}
\newpage
%    \end{macrocode}

% Manually include selected file,
% otherwise process as usual:
%    \begin{macrocode}
\ifchilddocmanual
\section*{part `\childdocname'}
\input{\childdocname}
\else
%    \end{macrocode}

% Include the two chapters:
%    \begin{macrocode}
\include{cdocsch1}
\include{cdocsch2}
%    \end{macrocode}

% Include the two parts unless only chapters should be displayed:
%    \begin{macrocode}
\ifchilddoc\else
\section{part three}
\input{cdocspt3}
\section{part four}
\input{cdocspt4}
\fi
%    \end{macrocode}

% Process as usual until here:
%    \begin{macrocode}
\fi
%    \end{macrocode}

% End of document body:
%    \begin{macrocode}
\end{document}
%    \end{macrocode}
%\iffalse
%</samplemain>
%\fi
%
% %%%%%%%%%%%%%%%%%%%%%%%%%%%%%%%%%%%%%%
% \paragraph{Chapter Include Files.}
%
% The include files are called |cdocsch1.tex| and |cdocsch2.tex|.
%
%\iffalse
%<*samplechap1|samplechap2>
%\fi

% Optional override for |\version| flag:
%    \begin{macrocode}
%%\providecommand{\version}{final}
%    \end{macrocode}

% Include the main document:
%    \begin{macrocode}
\input{childdoc.def}
\childdocof{cdocsamp}
%    \end{macrocode}

%\iffalse
%</samplechap1|samplechap2>
%\fi
%
%\iffalse
%<*samplechap1>
%\fi
% Some text for chapter 1:
%    \begin{macrocode}
\section{one}
some text in chapter one
%    \end{macrocode}

%\iffalse
%</samplechap1>
%\fi
% Some text for chapter 2:
%\iffalse
%<*samplechap2>
%\fi
%    \begin{macrocode}
\section{two}
more text in chapter two
%    \end{macrocode}

%\iffalse
%</samplechap2>
%\fi
%
% %%%%%%%%%%%%%%%%%%%%%%%%%%%%%%%%%%%%%%
% \paragraph{Part Include Files.}
%
% The include files are called |cdocspt3.tex| and |cdocspt4.tex|.
%
%\iffalse
%<*samplepart3|samplepart4>
%\fi

% Optional override for |\version| flag:
%    \begin{macrocode}
%%\providecommand{\version}{final}
%    \end{macrocode}

% Include the main document:
%    \begin{macrocode}
\input{childdoc.def}
\childdocby{cdocsamp}
%    \end{macrocode}

%\iffalse
%</samplepart3|samplepart4>
%\fi
%
%\iffalse
%<*samplepart3>
%\fi
% Some text for part 3:
%    \begin{macrocode}
some text in part three
%    \end{macrocode}

%\iffalse
%</samplepart3>
%\fi
% Some text for part 4:
%\iffalse
%<*samplepart4>
%\fi
%    \begin{macrocode}
more text in part four
%    \end{macrocode}

%\iffalse
%</samplepart4>
%\fi
%
% %%%%%%%%%%%%%%%%%%%%%%%%%%%%%%%%%%%%%%
% \paragraph{Forwarding for a Complete Draft.}
%
% The following forwarding file |cdocsdrf.tex|
% compiles the main document in draft mode:
%\iffalse
%<*sampledraft>
%\fi
%    \begin{macrocode}
\def\version{draft}
\input{childdoc.def}
\childdocforward{cdocsamp}
%    \end{macrocode}

%\iffalse
%</sampledraft>
%\fi
%
% %%%%%%%%%%%%%%%%%%%%%%%%%%%%%%%%%%%%%%
% \paragraph{Forwarding for Final Version of the Chapters.}
%
% The following forwarding files |cdocsfn1.tex| and |cdocsfn2.tex|
% (with identical content)
% compile the final versions of the child documents
% |cdocsch1.tex| and |cdocsch2.tex|, respectively:
%\iffalse
%<*samplefinal>
%\fi
%    \begin{macrocode}
\def\version{final}
\input{childdoc.def}
\childdocforwardprefix[cdocsamp]{cdocsfn}{cdocsch}
%    \end{macrocode}

%\iffalse
%</samplefinal>
%\fi
%
% %%%%%%%%%%%%%%%%%%%%%%%%%%%%%%%%%%%%%%
% \paragraph{Command Line Processing.}
%
% The following three command lines generate the output files
% |cdocscld|, |cdocscl1| and |cdocscl2|
% which should be identical to
% |cdocsdrf|, |cdocsch1| and |cdocsfn2|, respectively:
% \begin{center}
% \begin{tabular}{l}
% |latex -jobname cdocscld \|\\
% |  "\def\version{draft}\input{childdoc.def}\childdocforward{cdocsamp}"|\\
% |latex -jobname cdocscl1 \|\\
% |  "\input{childdoc.def}\childdocforward[cdocsamp]{cdocsch1}"|\\
% |latex -jobname cdocscl2 \|\\
% |  "\def\version{final}\input{childdoc.def}\childdocforward{cdocsch2}"|
% \end{tabular}
% \end{center}
% Note that the trailing backslash on each first line
% merely continues the input to the second line
% (for convenient cut ant paste).
% Furthermore, the command |latex| can be replaced by any
% of its alternative versions such as |pdflatex|.
%
% %%%%%%%%%%%%%%%%%%%%%%%%%%%%%%%%%%%%%%%%%%%%%%%%%%%%%%%%%%%%%%%%%%%%%%%%%%%%%%
% %%%%%%%%%%%%%%%%%%%%%%%%%%%%%%%%%%%%%%%%%%%%%%%%%%%%%%%%%%%%%%%%%%%%%%%%%%%%%%
% \section{Implementation}
%\iffalse
%<*package>
%\fi
%
% This section describes the definitions file |childdoc.def|.

% The definitions cannot be loaded using |\usepackage| or |\RequirePackage|
% which has a mechanism to prevent loading a style file more than once.
% When loading the definitions by means of |\input|
% multiple instances have to be prevented manually:
%\iffalse
%This code needs to be before the `\ProvidesFile' directive
%which is defined at the beginning of this file.
%Therefore it is also placed there and commented out here.
%</package>
%<*discard>
%\fi
%    \begin{macrocode}
\ifdefined\childdocmain\endinput\fi
%    \end{macrocode}
%\iffalse
%</discard>
%<*package>
%\fi
%
% \macro{\ifchilddoc}
% \macro{\ifchilddocmanual}
% The conditional |\ifchilddoc| tells whether a
% child (true) or main (false) document is being compiled.
% The conditional |\ifchilddocmanual| tells whether
% the |\includeonly| mechanism is used (false) or
% the selection of child files must be performed manually (true).
% The definitions initialise to false:
%    \begin{macrocode}
\newif\ifchilddoc
\newif\ifchilddocmanual
%    \end{macrocode}

% \macro{\childdocname}
% \macro{\childdocjob}
% The macro |\childdocname| stores the name of the main document
% to be compiled. The macro |\childdocjob| stores the name of
% the document on which the \LaTeX{} compiler was originally invoked.
% The content of |\jobname| cannot be compared
% to filenames specified in the source due to different catcodes.
% The following code rescans |\jobname|, stores the result
% in |\childdocname| and saves a copy in |\childdocjob|:
%    \begin{macrocode}
\edef\childdocname{\scantokens\expandafter{\jobname\noexpand}}
\let\childdocjob\childdocname
%    \end{macrocode}

% \macro{\childdocdisable}
% The macro |\childdocdisable| prevents the main file
% from being processed more than once.
% At this stage, the main document command |\childdocmain|
% is assumed to be called once again where it should do nothing.
% Any subsequent call to it should prevent
% a secondary processing of the main document
% It overwrites the forwarding commands
% |\childdocof| and |\childdocforward|
% with empty macros to prevent further inclusions of the main document:
%    \begin{macrocode}
\newcommand{\childdocdisable}
{
  \renewcommand{\childdocmain}[1]{\renewcommand{\childdocmain}[1]{\endinput}}
  \renewcommand{\childdocof}[1]{}
  \renewcommand{\childdocby}[2][]{}
  \renewcommand{\childdocforward}[2][]{}
  \renewcommand{\childdocdisable}{}
}
%    \end{macrocode}

% \macro{\childdocmain}
% The macro |\childdocmain| is to be called at the top of the main file
% with nothing or the main filename (without extension) as argument.
% First, it breaks loops.
% If the argument is not empty and does not match |\childdocname|
% (which is set by the first inclusion of |childdoc.def|),
% |\ifchilddoc| is set to true, |\includeonly| is applied to the child file
% and |\jobname| is set to the main file
% (for proper handling of |.aux| files):
%    \begin{macrocode}
\newcommand{\childdocmain}[1]
{
  \childdocdisable\childdocmain{}
  \if?#1?\else
    \begingroup
      \def\childdoctmp{#1}
      \ifx\childdoctmp\childdocname
        \def\childdoctmp{}
      \else
        \def\childdoctmp
        {
          \childdoctrue
          \includeonly{\childdocname}
          \def\childdocjob{#1}
          \def\jobname{#1}
        }
      \fi
      \expandafter
    \endgroup
    \childdoctmp
  \fi
}
%    \end{macrocode}

% \macro{\childdocof}
% The command |\childdocof| redirects
% compilation to the main file |#1|.
%    \begin{macrocode}
\newcommand{\childdocof}[1]
{
  \childdocdisable
  \childdoctrue
  \includeonly{\childdocname}
  \def\jobname{#1}
  \def\childdocjob{#1}
  \input{#1}
}
%    \end{macrocode}

% \macro{\childdocby}
% The command |\childdocby| ....
%    \begin{macrocode}
\newcommand{\childdocby}[2][]
{
  \childdocdisable
  \childdoctrue
  \childdocmanualtrue
  \if?#1?\else
    \def\jobname{#2}
  \fi
  \def\childdocjob{#2}
  \input{#2}
  \endinput
}
%    \end{macrocode}

% \macro{\childdocforward}
% The command |\childdocforward| redirects
% compilation to the main file or
% (if the optional argument is given) a child file.
% Parameters are set as if the main file
% or a child file starting with |\childdocof| was compiled.
% Then compilation is handed over to the main file:
%    \begin{macrocode}
\newcommand{\childdocforward}[2][]
{
  \begingroup
    \if?#1?
      \def\childdoctmp
      {
        \def\childdocname{#2}
        \def\childdocjob{#2}
        \def\jobname{#2}
        \input{#2}
        \endinput
      }
    \else
      \def\childdoctmp
      {
        \childdocdisable
        \def\childdocname{#2}
        \childdoctrue
        \includeonly{#2}
        \def\childdocjob{#1}
        \def\jobname{#1}
        \input{#1}
        \endinput
      }
    \fi
    \expandafter
  \endgroup
  \childdoctmp
}
%    \end{macrocode}

% \macro{\childdocforwardprefix}
% The command |\childdocforwardprefix| redirects
% compilation to the main or a child file by means of a pattern.
% The prefix |#1| in the current filename is replaced by |#2|
% and the suffix of the current filename is kept
% (it is assumed that the filename does not contain the substring `|~~~|'
% which is used as a delimiter).
% Compilation is handed over to the new file by |\childdocforward|:
%    \begin{macrocode}
\newcommand{\childdocforwardprefix}[3][]
{
  \begingroup
    \def\childdocextract #2##1~~~{\def\childdoctmp{\childdocforward[#1]{#3##1}}}
    \expandafter\childdocextract\childdocname~~~
    \expandafter
  \endgroup
  \childdoctmp
}
%    \end{macrocode}

% \macro{\childdoc}
% The deprecated macro |\childdoc| is a legacy version of |\childdocmain|:
%    \begin{macrocode}
\newcommand{\childdoc}{\childdocmain}
%    \end{macrocode}

% \macro{\childdocredirect}
% The deprecated macro |\childdocredirect| is a legacy version
% of |\childdocforward| and |\childdocforwardprefix|:
%    \begin{macrocode}
\newcommand{\childdocredirect}[2][]
{
  \begingroup
    \if?#1?
      \def\childdoctmp{\childdocforward{#2}}
    \else
      \def\childdoctmp{\childdocforwardprefix{#1}{#2}}
    \fi
    \expandafter
  \endgroup
  \childdoctmp
}
%    \end{macrocode}

%\iffalse
%</package>
%\fi
%
\endinput
|
and perform the replacements as outlined below.
Instead of |\childdocmain{|\textit{main}|}| add the following code
to the top of the main file:
%
\begin{center}
\begin{tabular}{l}
|\||ifdefined\childdocname\endinput\||fi\newif\ifchilddoc|\\
|\edef\childdocname{\scantokens\expandafter{\jobname\noexpand}}|\\
|\def\childdocmain{|\textit{main}|}\||ifx\childdocmain\childdocname\||else|\\
|\childdoctrue\includeonly{\childdocname}\let\jobname\childdocmain\||fi|\\
\end{tabular}
\end{center}
%
Instead of |\childdocof{|\textit{main}|}| just include the main file
at the top of each child file:
%
\begin{center}
|\input{|\textit{main}|}|
\end{center}
%
A simple redirection |\childdocforward{|\textit{dest}|}| is achieved by:
%
\begin{center}
|\def\jobname{|\textit{dest}|}\input{\jobname}|
\end{center}
%
The redirection with prefix
|\childdocforwardprefix[|\textit{prefix}|]{|\textit{dest}|}|
is accomplished by:
%
\begin{center}
\begin{tabular}{l}
|{\edef\jobname{\scantokens\expandafter{\jobname\noexpand}}|\\
|\def\redirectjob |\textit{prefix}|#1~~~{\gdef\jobname{|\textit{dest}|#1}}|\\
|\expandafter\redirectjob\jobname~~~}\input{\jobname}|
\end{tabular}
\end{center}

In an alternative approach,
child documents can be compiled by a specific command line
without additional code or specific definitions:
%
\begin{center}
|... -jobname "|\textit{target}|" "|[\textit{flags}]%
|\includeonly{|\textit{dest}|}\input{|\textit{main}|}"|
\end{center}
%

%%%%%%%%%%%%%%%%%%%%%%%%%%%%%%%%%%%%%%%%%%%%%%%%%%%%%%%%%%%%%%%%%%%%%%%%%%%%%%%%
%%%%%%%%%%%%%%%%%%%%%%%%%%%%%%%%%%%%%%%%%%%%%%%%%%%%%%%%%%%%%%%%%%%%%%%%%%%%%%%%
\section{Information}

%%%%%%%%%%%%%%%%%%%%%%%%%%%%%%%%%%%%%%%%%%%%%%%%%%%%%%%%%%%%%%%%%%%%%%%%%%%%%%%%
\subsection{Copyright}

Copyright \copyright{} 2017--2018 Niklas Beisert

This work may be distributed and/or modified under the
conditions of the \LaTeX{} Project Public License, either version 1.3
of this license or (at your option) any later version.
The latest version of this license is in
  \url{http://www.latex-project.org/lppl.txt}
and version 1.3 or later is part of all distributions of \LaTeX{}
version 2005/12/01 or later.

This work has the LPPL maintenance status `maintained'.

The Current Maintainer of this work is Niklas Beisert.

This work consists of the files |README.txt|, |childdoc.ins| and |childdoc.dtx|
as well as the derived files |childdoc.def|, |cdocsamp.tex|
with |cdocsch1.tex|, |cdocsch2.tex|, |cdocspt3.tex|, |cdocspt4.tex|,
|cdocsdrf.tex|, |cdocsfn1.tex|, |cdocsfn2.tex|
as well as |childdoc.pdf|.

%%%%%%%%%%%%%%%%%%%%%%%%%%%%%%%%%%%%%%%%%%%%%%%%%%%%%%%%%%%%%%%%%%%%%%%%%%%%%%%%
\subsection{Files and Installation}

The package consists of the files:
%
\begin{center}
\begin{tabular}{ll}
    |README.txt|   & readme file \\
    |childdoc.ins| & installation file \\
    |childdoc.dtx| & source file \\
    |childdoc.def| & definition file \\
    |cdocsamp.tex| & sample main file \\
    |cdocsch1.tex| & sample include file \\
    |cdocsch2.tex| & sample include file \\
    |cdocspt3.tex| & sample part file \\
    |cdocspt4.tex| & sample part file \\
    |cdocsdrf.tex| & sample redirection file \\
    |cdocsfn1.tex| & sample redirection file \\
    |cdocsfn2.tex| & sample redirection file \\
    |childdoc.pdf| & manual
\end{tabular}
\end{center}
%
The distribution consists of the files
|README.txt|, |childdoc.ins| and |childdoc.dtx|.
%
\begin{itemize}
\item
Run (pdf)\LaTeX{} on |childdoc.dtx|
to compile the manual |childdoc.pdf| (this file).
\item
Run \LaTeX{} on |childdoc.ins| to create the definitions file |childdoc.def|
and the sample |cdocsamp.tex| with include files
|cdocsch1.tex|, |cdocsch2.tex|, |cdocspt3.tex|, |cdocspt4.tex|,
|cdocsdrf.tex|, |cdocsfn1.tex|, |cdocsfn2.tex|.
Then copy the file |childdoc.def| to an appropriate directory of your \LaTeX{}
distribution, e.g.\ \textit{texmf-root}|/tex/latex/childdoc|.
\end{itemize}

%%%%%%%%%%%%%%%%%%%%%%%%%%%%%%%%%%%%%%%%%%%%%%%%%%%%%%%%%%%%%%%%%%%%%%%%%%%%%%%%
\subsection{Related CTAN Packages}

There are several other packages which offer a similar functionality:
%
\begin{itemize}
\item
The packages
\href{http://ctan.org/pkg/docmute}{\textsf{docmute}},
\href{http://ctan.org/pkg/includex}{\textsf{includex}} and
\href{http://ctan.org/pkg/standalone}{\textsf{standalone}}
provide commands to include only the document body of
a child file thus allowing both files to be compiled individually.
\item
The packages \href{http://ctan.org/pkg/subdocs}{\textsf{subdocs}}
and \href{http://ctan.org/pkg/subfiles}{\textsf{subfiles}}
provide structures in which the main and child documents can be
encapsulated and allowing them to be compiled individually.
The inclusion mechanism is different from the conventional |\include|.
\item
The package \href{http://ctan.org/pkg/combine}{\textsf{combine}}
is an elaborate solution to combine several documents into one.
\end{itemize}
%
See also the CTAN topic \href{http://ctan.org/topic/subdocs}{\textsf{subdocs}}
for further related packages.
The present package differs from the above solutions in that
a document structure constructed with the conventional |\include| mechanism
just needs two extra commands at the top of every file
such that all constituent files can be compiled individually.

%%%%%%%%%%%%%%%%%%%%%%%%%%%%%%%%%%%%%%%%%%%%%%%%%%%%%%%%%%%%%%%%%%%%%%%%%%%%%%%%
%\subsection{Feature Suggestions}
%
%The following is a list of features which may be useful for future
%versions of this package:
%%
%\begin{itemize}
%\item
%\ldots
%\end{itemize}

%%%%%%%%%%%%%%%%%%%%%%%%%%%%%%%%%%%%%%%%%%%%%%%%%%%%%%%%%%%%%%%%%%%%%%%%%%%%%%%%
\subsection{Revision History}

%%%%%%%%%%%%%%%%%%%%%%%%%%%%%%%%%%%%%%%%
\paragraph{v2.0:} 2018/12/30

\begin{itemize}
\item
immediate forward processing
\item
added |\childdocby| mechanism
\item
manual restructured
\end{itemize}

%%%%%%%%%%%%%%%%%%%%%%%%%%%%%%%%%%%%%%%%
\paragraph{v1.6:} 2018/01/17

\begin{itemize}
\item
application for development of include files
\item
corrections to manual
\end{itemize}

%%%%%%%%%%%%%%%%%%%%%%%%%%%%%%%%%%%%%%%%
\paragraph{v1.5:} 2017/05/21

\begin{itemize}
\item
more complete structuring introduced
\item
|\childdocof| introduced
\item
|\childdoc| renamed to |\childdocmain|
\item
|\childredirect| renamed to |\childdocforward| and |\childdocforwardprefix|
and functionality expanded
\end{itemize}

%%%%%%%%%%%%%%%%%%%%%%%%%%%%%%%%%%%%%%%%
\paragraph{v1.0:} 2017/04/27

\begin{itemize}
\item
manual and install package
\item
first version published on CTAN
\end{itemize}

%%%%%%%%%%%%%%%%%%%%%%%%%%%%%%%%%%%%%%%%
\paragraph{v0.6:} 2017/04/26

\begin{itemize}
\item
redirection mechanism added
\end{itemize}

%%%%%%%%%%%%%%%%%%%%%%%%%%%%%%%%%%%%%%%%
\paragraph{v0.5:} 2017/04/26

\begin{itemize}
\item
functionality in definition file
\end{itemize}


%%%%%%%%%%%%%%%%%%%%%%%%%%%%%%%%%%%%%%%%%%%%%%%%%%%%%%%%%%%%%%%%%%%%%%%%%%%%%%%%
%%%%%%%%%%%%%%%%%%%%%%%%%%%%%%%%%%%%%%%%%%%%%%%%%%%%%%%%%%%%%%%%%%%%%%%%%%%%%%%%
%%%%%%%%%%%%%%%%%%%%%%%%%%%%%%%%%%%%%%%%%%%%%%%%%%%%%%%%%%%%%%%%%%%%%%%%%%%%%%%%
\appendix

\settowidth\MacroIndent{\rmfamily\scriptsize 000\ }

 \DocInput{childdoc.dtx}

\end{document}
%</driver>
% \fi
%
% %%%%%%%%%%%%%%%%%%%%%%%%%%%%%%%%%%%%%%%%%%%%%%%%%%%%%%%%%%%%%%%%%%%%%%%%%%%%%%
% %%%%%%%%%%%%%%%%%%%%%%%%%%%%%%%%%%%%%%%%%%%%%%%%%%%%%%%%%%%%%%%%%%%%%%%%%%%%%%
% \section{Sample}
%\iffalse
%<*samplemain>
%\fi
%
% The following presents a sample document
% with two chapters, two parts, a title page,
% a compile flag as well as three forwarding files to set the flag.
% It consists of eight |.tex| files:
% \begin{center}
% \begin{tabular}{ll}
% |cdocsamp.tex|&main file\\
% |cdocsch1.tex|&include file for chapter 1\\
% |cdocsch2.tex|&include file for chapter 2\\
% |cdocspt3.tex|&include file for part 3\\
% |cdocspt4.tex|&include file for part 4\\
% |cdocsdrf.tex|&forwarding file for main file in draft mode\\
% |cdocsfi1.tex|&forwarding file for final version of chapter 1\\
% |cdocsfi2.tex|&forwarding file for final version of chapter 2\\
% \end{tabular}
% \end{center}
% Each of the eight files can be compiled directly by the \LaTeX{} compiler.
%
% %%%%%%%%%%%%%%%%%%%%%%%%%%%%%%%%%%%%%%
% \paragraph{Main File.}
%
% The main file is called |cdocsamp.tex|.
%
% Load the \textsf{childdoc} definitions and
% declare the filename for the main document:
%    \begin{macrocode}
% \iffalse
%
% childdoc.dtx Copyright (C) 2017-2018 Niklas Beisert
%
% This work may be distributed and/or modified under the
% conditions of the LaTeX Project Public License, either version 1.3
% of this license or (at your option) any later version.
% The latest version of this license is in
%   http://www.latex-project.org/lppl.txt
% and version 1.3 or later is part of all distributions of LaTeX
% version 2005/12/01 or later.
%
% This work has the LPPL maintenance status `maintained'.
%
% The Current Maintainer of this work is Niklas Beisert.
%
% This work consists of the files childdoc.dtx and childdoc.ins
% and the derived files childdoc.def and cdocsamp.tex with
% cdocsch1.tex, cdocsch2.tex, cdocsdrf.tex, cdocsfn1.tex, cdocsfn2.tex.
%
%<package>\ifdefined\childdocmain\endinput\fi
%<package>\ProvidesFile{childdoc.def}[2018/12/30 v2.0 child document driver]
%<samplemain>\ProvidesFile{cdocsamp.tex}[2018/12/30 v2.0 sample for childdoc]
%<*driver>
%\ProvidesFile{childdoc.drv}[2018/12/30 v2.0 childdoc reference manual file]
\PassOptionsToClass{10pt,a4paper}{article}
\documentclass{ltxdoc}

\usepackage[margin=35mm]{geometry}
\usepackage{hyperref}
\usepackage{hyperxmp}
\usepackage[usenames]{color}

\hypersetup{colorlinks=true}
\hypersetup{pdfstartview=FitH}
\hypersetup{pdfpagemode=UseNone}
\hypersetup{pdfsource={}}
\hypersetup{pdflang={en-UK}}
\hypersetup{pdfcopyright={Copyright 2017-2018 Niklas Beisert.
  This work may be distributed and/or modified under the
  conditions of the LaTeX Project Public License, either version 1.3
  of this license or (at your option) any later version.}}
\hypersetup{pdflicenseurl={http://www.latex-project.org/lppl.txt}}
\hypersetup{pdfcontactaddress={ETH Zurich, ITP, HIT K,
  Wolfgang-Pauli-Strasse 27}}
\hypersetup{pdfcontactpostcode={8093}}
\hypersetup{pdfcontactcity={Zurich}}
\hypersetup{pdfcontactcountry={Switzerland}}
\hypersetup{pdfcontactemail={nbeisert@itp.phys.ethz.ch}}
\hypersetup{pdfcontacturl={http://people.phys.ethz.ch/\xmptilde nbeisert/}}

\newcommand{\secref}[1]{\hyperref[#1]{section \ref*{#1}}}

\parskip1ex
\parindent0pt
\let\olditemize\itemize
\def\itemize{\olditemize\parskip0pt}

\begin{document}

\title{The \textsf{childdoc} Package}
\hypersetup{pdftitle={The childdoc Package}}
\author{Niklas Beisert\\[2ex]
  Institut f\"ur Theoretische Physik\\
  Eidgen\"ossische Technische Hochschule Z\"urich\\
  Wolfgang-Pauli-Strasse 27, 8093 Z\"urich, Switzerland\\[1ex]
  \href{mailto:nbeisert@itp.phys.ethz.ch}
  {\texttt{nbeisert@itp.phys.ethz.ch}}}
\hypersetup{pdfauthor={Niklas Beisert}}
\hypersetup{pdfsubject={Manual for the LaTeX2e Package childdoc}}
\date{30 December 2018, \textsf{v2.0}}
\maketitle

\begin{abstract}\noindent
\textsf{childdoc} is a \LaTeXe{} package
that enables the direct compilation
of document sections included by |\include|
to individual files.
\end{abstract}

\begingroup
\parskip0ex
\tableofcontents
\endgroup

%%%%%%%%%%%%%%%%%%%%%%%%%%%%%%%%%%%%%%%%%%%%%%%%%%%%%%%%%%%%%%%%%%%%%%%%%%%%%%%%
%%%%%%%%%%%%%%%%%%%%%%%%%%%%%%%%%%%%%%%%%%%%%%%%%%%%%%%%%%%%%%%%%%%%%%%%%%%%%%%%
\section{Introduction}

\LaTeX{} provides a mechanism to structure a large document (such as a book)
into a main file and several child files (containing the chapters)
using the |\include| command.
This mechanism is beneficial for documents
which span hundreds of pages in order to
make the source file(s) more manageable.
Moreover, compilation can be restricted to
selected child files by means of the |\includeonly| command.
The latter feature can be used to reduce the compilation time while editing
(this was significantly more useful in the earlier days of \LaTeX{})
or to generate a smaller document which is easier to navigate.
Another application of |\includeonly| is to generate
documents consisting of selected parts of the complete document.

However, there are a few drawbacks of the plain |\include| mechanism:
\begin{itemize}
\item
The child files cannot be compiled on their own,
they can only be compiled via the main file.
A naive editing environment
(such as a text editor with an option
to have the current file processed by \LaTeX)
may require one to switch to the main file before compiling;
attempting to compile the child file produces errors.
\item
The main file must be modified (each time)
to adjust the |\includeonly| command
to the present needs. This easily leaves the main file in a messy state.
\item
The generated document will always carry the filename
of the main document. This is inconvenient if
several child files are to be compiled and
to be kept for distribution.
\end{itemize}

The present package provides a simple interface
to make child files individually compilable by \LaTeX{}.
Compiling a child file then has the same effect as compiling
the main file with an |\includeonly| command
to select the appropriate child.
Moreover the generated document will carry the name of the child
rather than the main file.
This resolves all three above issues.

This feature is meant to make the editing of books,
thesis documents and lecture notes somewhat more convenient.
However, the package can also be used efficiently for
composing a series of documents (such as exercise sheets)
which are typically distributed individually.
It then assists the author in generating the individual documents
(potentially in different versions)
as well as a document containing the collected series.
Another application is in developing style files
or other kinds of included material
where compilation of the style file could redirect
to a sample or test file.

%%%%%%%%%%%%%%%%%%%%%%%%%%%%%%%%%%%%%%%%%%%%%%%%%%%%%%%%%%%%%%%%%%%%%%%%%%%%%%%%
%%%%%%%%%%%%%%%%%%%%%%%%%%%%%%%%%%%%%%%%%%%%%%%%%%%%%%%%%%%%%%%%%%%%%%%%%%%%%%%%
\section{Usage}

First of all, the package \textsf{childdoc} is \emph{not} a standard
\LaTeXe{} |.sty| style file! Therefore it needs to be invoked in
a non-standard way.

%%%%%%%%%%%%%%%%%%%%%%%%%%%%%%%%%%%%%%%%%%%%%%%%%%%%%%%%%%%%%%%%%%%%%%%%%%%%%%%%
\subsection{Included Files}
\label{sec:include}

%%%%%%%%%%%%%%%%%%%%%%%%%%%%%%%%%%%%%%%%
\DescribeMacro{\childdocmain}
To use the package, add the commands
\begin{center}
\begin{tabular}{l}
|\input{childdoc.def}|\\
|\childdocmain{}|\\
\end{tabular}
\end{center}
at the very top of the main \LaTeX{} file,
in particular \emph{before} the |\documentclass| statement!
The argument of |\childdocmain| should be left empty
(but it must be present).

%%%%%%%%%%%%%%%%%%%%%%%%%%%%%%%%%%%%%%%%
\DescribeMacro{\childdocof}
Furthermore, add the commands
\begin{center}
\begin{tabular}{l}
|\input{childdoc.def}|\\
|\childdocof{|\textit{main}|}|\\
\end{tabular}
\end{center}
at the top of every child file \textit{child}
which is included by |\include{|\textit{child}|}|
from within the main file
(or at least for those files to be compiled individually).
The argument \textit{main} must be the filename of the main file.

There are a couple of
considerations in setting up the main and child documents:

%%%%%%%%%%%%%%%%%%%%%%%%%%%%%%%%%%%%%%%%
\paragraph{Restrictions.}

Please note the following restrictions:
\begin{itemize}
\item
|\childdocmain| must be called with one argument \textit{main}
to ensure compatibility with earlier version of the package.
It must either be empty (|\childdocmain{}|)
or precisely match the filename of the main file in which it is specified.
See \secref{sec:detection} for further information.
\item
The filename \textit{main} must be specified without the |.tex| extension.
\item
The filename \textit{main} is case sensitive
(even in case-insensitive file systems)
due to internal string comparison.
\item
The argument \textit{main} should be fully expanded, it cannot be a macro.
\item
Subdirectories and special characters should be avoided in filenames.
\item
The command |\childdocmain{|\textit{main}|}| must be followed by a whitespace.
It should not be followed immediately by another command
or by a comment mark `|%|'.
This is because the \TeX{} parser reads the token immediately following
the argument of |\childdocmain| and puts it
at the beginning of every child section;
however, a white\-space is ignored.
\end{itemize}

%%%%%%%%%%%%%%%%%%%%%%%%%%%%%%%%%%%%%%%%
\paragraph{Content of Main File.}

It is advisable to place all content in the child files included by |\include|.
Any output contained in the main file will appear in all child documents
unless suppressed manually;
it cannot be suppressed automatically by the |\includeonly| directive
and thus should normally be avoided.
A method to include some content in the main file
by means of conditional processing is described in \secref{sec:conditional}.

%%%%%%%%%%%%%%%%%%%%%%%%%%%%%%%%%%%%%%%%
\paragraph{Page Numbering.}

When only a part of the document is compiled,
the appropriate numbering of pages
(as well as other status parameters)
is determined from the |.aux| files.
The latter contain information from previous passes.
However this information needs to propagate through
all intermediate child documents.
Therefore the page numbering in child documents may well
be inconsistent until the complete document is compiled at least once.

A useful (if unconventional) way to always ensure a consistent
page numbering is to restart the numbering in each child document
and denote the pages by `\textit{child}|.|\textit{page}'
where \textit{child} represents the chapter/section number of the child file.
This can be achieved by the command
|\numberwithin{page}{|\textit{child}|}|
of the \textsf{amsmath} package
where \textit{child} can be |chapter| or |section|
depending on the chosen structuring.
Alternatively, one can modify the macro |\thepage| appropriately
and reset the counter |page| at the start of each child file.

%%%%%%%%%%%%%%%%%%%%%%%%%%%%%%%%%%%%%%%%%%%%%%%%%%%%%%%%%%%%%%%%%%%%%%%%%%%%%%%%
\subsection{Conditional Processing}
\label{sec:conditional}

The package provides a mechanism to compile different versions
of a document. To customise the versions further some conditional processing
can come in handy to distinguish which version is being compiled.
The package provides two macros to describe the compilation context:

%%%%%%%%%%%%%%%%%%%%%%%%%%%%%%%%%%%%%%%%
\DescribeMacro{\ifchilddoc}
The conditional |\ifchilddoc| distinguishes between the compilation of
child documents and the main document:
%
\begin{center}
|\ifchilddoc |\textit{child-code}| |[|\||else |\textit{main-code}]| \||fi|
\end{center}

%%%%%%%%%%%%%%%%%%%%%%%%%%%%%%%%%%%%%%%%
\DescribeMacro{\childdocname}
\DescribeMacro{\childdocjob}
The macro |\childdocname| contains the filename (without extension)
of the main or child file being processed.
Note that |\childdocjob| will always contain the name of the main file.

%%%%%%%%%%%%%%%%%%%%%%%%%%%%%%%%%%%%%%%%
\paragraph{Title Page.}

Conditional processing can be used to include a title or banner page
in the main document when proper precautions are taken.
Importantly, the code in the main file should ensure that the page counter
(as well as other status parameters which are stored in the |.aux| files)
takes the same value after the conditional processing.
Otherwise the page numbers may take divergent values
depending on which part is compiled.

For example, a title page could be declared by:
%
\begin{center}
\begin{tabular}{l}
|\ifchilddoc\||else|\\
|\addtocounter{page}{-1}|\\
\textit{code for title page}\\
|\newpage|\\
|\||fi|
\end{tabular}
\end{center}
%
A banner page for the child documents can be generated by:
%
\begin{center}
\begin{tabular}{l}
|\ifchilddoc|\\
|\addtocounter{page}{-1}|\\
\textit{code for banner page}\\
|\newpage|\\
|\||fi|
\end{tabular}
\end{center}
%
Here one could write a message such as:
\begin{center}
|This is the part \childdocname{} of \childdocjob{}.|
\end{center}

%%%%%%%%%%%%%%%%%%%%%%%%%%%%%%%%%%%%%%%%%%%%%%%%%%%%%%%%%%%%%%%%%%%%%%%%%%%%%%%%
\subsection{Flags}
\label{sec:flags}

The package makes it easy to generate different versions
of the main or child documents.
To this end compilation flags can be defined
and assigned different default values.
They will be particularly useful in conjunction
with the forwarding mechanism described in \secref{sec:forward}.

For example, it may be useful to have a flag |\version|
which can be set to |draft| or |final|.
The document source will contain some conditional code
depending on the value of |\version|.
Suppose further, the flag should default to |final| for the main file
and to |draft| for child files
which is a natural assignment for editing the document.
This is achieved by placing the following code
in the preamble of the main document
(below the |\childdocmain| directive):
%
\begin{center}
\begin{tabular}{l}
|\ifchilddoc|\\
|\providecommand{\version}{draft}|\\
|\||else|\\
|\providecommand{\version}{final}|\\
|\||fi|
\end{tabular}
\end{center}
%
The definition by |\providecommand| makes sure
that previous definitions are not overwritten.
Further statements |\providecommand{\version}{...}|
can thus be added before the above code to override it.

For the main file, one might add a line
(between |\childdocmain| and the above block)
%
\begin{center}
|%\ifchilddoc\||else\providecommand{\version}{draft}\||fi|
\end{center}
%
which can be uncommented to produce a draft version.
Likewise one can add a line to the very top of a child file
(above the |\childdocof{|\textit{main}|}| directive)
%
\begin{center}
|%\providecommand{\version}{final}|
\end{center}
%
which can be uncommented to produce the final version of this child document.

%%%%%%%%%%%%%%%%%%%%%%%%%%%%%%%%%%%%%%%%%%%%%%%%%%%%%%%%%%%%%%%%%%%%%%%%%%%%%%%%
\subsection{Forwarding}
\label{sec:forward}

Different versions of the main or child documents
using compilation flags as described in \secref{sec:flags}
can be (permanently) stored in different files
for convenient compilation, viewing and distribution.
To this end, the package defines a command
to pass on compilation to a different file:

%%%%%%%%%%%%%%%%%%%%%%%%%%%%%%%%%%%%%%%%
\DescribeMacro{\childdocforward}
The command |\childdocforward| redirects processing to
another source file:
%
\begin{center}
\begin{tabular}{l}
|\input{childdoc.def}|\\
|\childdocforward[|\textit{main}|]{|\textit{dest}|}|\\
\end{tabular}
\end{center}
%
The argument \textit{dest} is the destination file
(without extension).
It should be the main file or one of the child files.
Note that further \textsf{childdoc} directives
such as |\childdocof| and |\childdocforward|
in the indicated file will be processed in this form.
The optional argument \textit{main}
passes on directly to the main file \textit{main}
while pretending to compile the child \textit{dest}.
This form behaves as if \textit{dest}
issues |\childdocof{|\textit{main}|}| right away,
and no further \textsf{childdoc} directives will be processed.

%%%%%%%%%%%%%%%%%%%%%%%%%%%%%%%%%%%%%%%%
\DescribeMacro{\...prefix}
In the alternative form |\childdocforwardprefix|,
%
\begin{center}
\begin{tabular}{l}
|\input{childdoc.def}|\\
|\childdocforwardprefix[|\textit{main}|]{|\textit{prefix}|}{|\textit{dest}|}|
\end{tabular}
\end{center}
%
the destination file is determined by a pattern
depending on the current file:
To make this work, the current file must be called
`{\textit{prefix}\hspace{0.2em}\textit{suffix}}'
with \textit{prefix} matching precisely the argument.
Processing is then passed on to the file
`{\textit{dest}\hspace{0.2em}\textit{suffix}}'.
Surely, the same effect is achieved by
directly specifying the
argument `{\textit{dest}\hspace{0.2em}\textit{suffix}}'
in the first form.
However, that requires to set up a different file
for each child. With the alternative form of the command
all these files can have exactly the same content
which simplifies setting them up and maintaining them.

For example, the following file |draft.tex|
with a compilation flag |\version| as described in \secref{sec:flags}
compiles the main document as a draft:
%
\begin{center}
\begin{tabular}{l}
|\def\version{draft}|\\
|\input{childdoc.def}|\\
|\childdocforward{|\textit{main}|}|
\end{tabular}
\end{center}
%
Likewise, the following files |final|\textit{nn}|.tex|
compile the final version of the child document
|child|\textit{nn}|.tex|:
%
\begin{center}
\begin{tabular}{l}
|\def\version{final}|\\
|\input{childdoc.def}|\\
|\childdocforwardprefix{final}{child}|
\end{tabular}
\end{center}
%

Note that when several versions of a main file and/or of each child file
are to be generated, it may be convenient to set up a |Makefile| or
shell script to automatise the process.

%%%%%%%%%%%%%%%%%%%%%%%%%%%%%%%%%%%%%%%%%%%%%%%%%%%%%%%%%%%%%%%%%%%%%%%%%%%%%%%%
\subsection{Command Line Processing}
\label{sec:commandline}

The effect of redirection files can also be achieved by invoking
the \LaTeX{} compiler with a more elaborate command line.
Most conveniently this should be done as part
of a shell script or a |Makefile|.

When using \textsf{childdoc} in the main file, the following
command lines effectively perform a redirection
(note that depending on the shell being used,
backslashes may have to be doubled: `|\|' $\to$ `|\\|'):
%
\begin{center}
|... -jobname "|\textit{target}|" |\\|"|[\textit{flags}]%
|\input{childdoc.def}\childdocforward[|\textit{main}|]{|\textit{dest}|}"|
\end{center}
%
Here \textit{target} is the name of the output file,
\textit{main} is the name of the main file
and \textit{dest} is the name of the main or child file to be processed
(all filenames without extensions).
The optional argument \textit{main} can be omitted
if \textit{main} matches \textit{dest}.
Optionally, compilation \textit{flags} can be defined via |\def| commands.
This command line makes the \TeX{} engine believe
it is compiling the file \textit{target}
whose content is specified as the latter parameter.
The provided code then forwards the processing to
\textit{main} or \textit{dest} as described in \secref{sec:forward}.

%%%%%%%%%%%%%%%%%%%%%%%%%%%%%%%%%%%%%%%%%%%%%%%%%%%%%%%%%%%%%%%%%%%%%%%%%%%%%%%%
\subsection{Include by Input}
\label{sec:input}

Including child documents by |\include| has some restrictions by design.
Most notably, the content of a child document always occupies
its own set of pages; pages cannot be shared between child documents.
Usually, this behaviour makes perfect sense
because each child document contain an essential part of the document.
However, in some situations it may be desirable to compose
a document from a collection of parts
without having mandatory page breaks between then.
For this case, the package
provides a mechanism to include parts
by |\input| which can also be processed individually.
However, by construction this mechanism
requires manual handling of the content to be output.

%%%%%%%%%%%%%%%%%%%%%%%%%%%%%%%%%%%%%%%%
\DescribeMacro{\ifchilddocmanual}
The main file should be prepared as usual, see \secref{sec:include}.
However, the document body must make a distinction
between processing of an individual part and of the main document, e.g.:
%
\begin{center}
\begin{tabular}{l}
|\ifchilddocmanual|\\
|\input{\childdocname}|\\
|\||else|\\
\textit{document body with }|\input{|\textit{part}|}|\\
|\||fi|
\end{tabular}
\end{center}
%
The conditional |\ifchilddocmanual| is true whenever
a part to be included by |\input| is being compiled,
and the name of the part is stored in |\childdocname|.

%%%%%%%%%%%%%%%%%%%%%%%%%%%%%%%%%%%%%%%%
\DescribeMacro{\childdocby}
Each part to be included by |\input| should start with:
%
\begin{center}
\begin{tabular}{l}
|\input{childdoc.def}|\\
|\childdocby{|\textit{main}|}|\\
\end{tabular}
\end{center}
%
The directive |\childdocby| is similar to |\childdocof|
described in \secref{sec:include},
but the subsequent selection of content must be done manually.
To that end, both |\ifchilddoc| and |\ifchilddocmanual|
will be true upon processing of a part,
and the name of the part is stored in |\childdocname|.
Note that |\jobname| will be set to the filename of the current part
so that each part receives an individual |.aux| file
that does not interfere with the |.aux| file(s) of the main document.
This behaviour can be altered by the alternative form
|\childdocby[*]{|\textit{main}|}| (with a non-empty optional argument)
which uses the |.aux| file of the main document
by setting |\jobname| to \textit{main}.

%%%%%%%%%%%%%%%%%%%%%%%%%%%%%%%%%%%%%%%%%%%%%%%%%%%%%%%%%%%%%%%%%%%%%%%%%%%%%%%%
\subsection{Driver Development}
\label{sec:driver}

The \textsf{childdoc} mechanism can also be use for the development
of definition files such as \LaTeX{} styles or classes.
This case differs from the above setup with multiple parts
included by |\include| in that no |\includeonly| should be invoked.
This can be achieved by starting the include file
(before |\ProvidesPackage|) with:
%
\begin{center}
\begin{tabular}{l}
|\input{childdoc.def}|\\
|\childdocforward{|\textit{main}|}|\\
\end{tabular}
\end{center}
%
or alternatively with:
%
\begin{center}
\begin{tabular}{l}
|\input{childdoc.def}|\\
|\childdocby{|\textit{main}|}|\\
\end{tabular}
\end{center}
%
Both forms have slightly different effects as described above.
The main file is prepared as usual, see \secref{sec:include}.

%%%%%%%%%%%%%%%%%%%%%%%%%%%%%%%%%%%%%%%%%%%%%%%%%%%%%%%%%%%%%%%%%%%%%%%%%%%%%%%%
\subsection{Legacy Detection}
\label{sec:detection}

The directive |\childdocmain| in the main file can detect
whether the complete document or merely a child is to be compiled
even without using the directive |\childdocof|.
This method is deprecated because it is less robust
and there is no compelling reason to use it;
it is merely provided for backward compatibility
and it may be removed in future versions.

If the detection mechanism is to be used,
it is mandatory to correctly specify
the filename of the main file as the argument of |\childdocmain|:
%
\begin{center}
\begin{tabular}{l}
|\input{childdoc.def}|\\
|\childdocmain{|\textit{main}|}|\\
\end{tabular}
\end{center}
%
If |\jobname| does not match the argument \textit{main} of |\childdocmain|,
it is assumed that |\jobname| points to the child file to be compiled.
When using |\childdocmain| with the main file specified as argument,
it suffices to start a child file
with just |\input{|\textit{main}|}|
without loading of the package and using |\childdocof|.
If instead all processing is done
with the appropriate \textsf{childdoc} directives,
the argument of \textit{main} of |\childdocmain| can be empty.

An alternative version of the command line processing described
in \secref{sec:commandline} using the detection mechanism reads:
%
\begin{center}
|... -jobname "|\textit{target}|" "|[\textit{flags}]%
[|\def\jobname{|\textit{dest}|}|]|\input{|\textit{main}|}"|
\end{center}

%%%%%%%%%%%%%%%%%%%%%%%%%%%%%%%%%%%%%%%%%%%%%%%%%%%%%%%%%%%%%%%%%%%%%%%%%%%%%%%%
\subsection{Manual Code}
\label{sec:manual}

In case one cannot be certain whether the definitions file |childdoc.def|
is installed on the target \TeX{} distribution
and one prefers not to ship it,
it is conceivable to paste a few relevant commands into the sources.

To that end, drop all statements |\input{childdoc.def}|
and perform the replacements as outlined below.
Instead of |\childdocmain{|\textit{main}|}| add the following code
to the top of the main file:
%
\begin{center}
\begin{tabular}{l}
|\||ifdefined\childdocname\endinput\||fi\newif\ifchilddoc|\\
|\edef\childdocname{\scantokens\expandafter{\jobname\noexpand}}|\\
|\def\childdocmain{|\textit{main}|}\||ifx\childdocmain\childdocname\||else|\\
|\childdoctrue\includeonly{\childdocname}\let\jobname\childdocmain\||fi|\\
\end{tabular}
\end{center}
%
Instead of |\childdocof{|\textit{main}|}| just include the main file
at the top of each child file:
%
\begin{center}
|\input{|\textit{main}|}|
\end{center}
%
A simple redirection |\childdocforward{|\textit{dest}|}| is achieved by:
%
\begin{center}
|\def\jobname{|\textit{dest}|}\input{\jobname}|
\end{center}
%
The redirection with prefix
|\childdocforwardprefix[|\textit{prefix}|]{|\textit{dest}|}|
is accomplished by:
%
\begin{center}
\begin{tabular}{l}
|{\edef\jobname{\scantokens\expandafter{\jobname\noexpand}}|\\
|\def\redirectjob |\textit{prefix}|#1~~~{\gdef\jobname{|\textit{dest}|#1}}|\\
|\expandafter\redirectjob\jobname~~~}\input{\jobname}|
\end{tabular}
\end{center}

In an alternative approach,
child documents can be compiled by a specific command line
without additional code or specific definitions:
%
\begin{center}
|... -jobname "|\textit{target}|" "|[\textit{flags}]%
|\includeonly{|\textit{dest}|}\input{|\textit{main}|}"|
\end{center}
%

%%%%%%%%%%%%%%%%%%%%%%%%%%%%%%%%%%%%%%%%%%%%%%%%%%%%%%%%%%%%%%%%%%%%%%%%%%%%%%%%
%%%%%%%%%%%%%%%%%%%%%%%%%%%%%%%%%%%%%%%%%%%%%%%%%%%%%%%%%%%%%%%%%%%%%%%%%%%%%%%%
\section{Information}

%%%%%%%%%%%%%%%%%%%%%%%%%%%%%%%%%%%%%%%%%%%%%%%%%%%%%%%%%%%%%%%%%%%%%%%%%%%%%%%%
\subsection{Copyright}

Copyright \copyright{} 2017--2018 Niklas Beisert

This work may be distributed and/or modified under the
conditions of the \LaTeX{} Project Public License, either version 1.3
of this license or (at your option) any later version.
The latest version of this license is in
  \url{http://www.latex-project.org/lppl.txt}
and version 1.3 or later is part of all distributions of \LaTeX{}
version 2005/12/01 or later.

This work has the LPPL maintenance status `maintained'.

The Current Maintainer of this work is Niklas Beisert.

This work consists of the files |README.txt|, |childdoc.ins| and |childdoc.dtx|
as well as the derived files |childdoc.def|, |cdocsamp.tex|
with |cdocsch1.tex|, |cdocsch2.tex|, |cdocspt3.tex|, |cdocspt4.tex|,
|cdocsdrf.tex|, |cdocsfn1.tex|, |cdocsfn2.tex|
as well as |childdoc.pdf|.

%%%%%%%%%%%%%%%%%%%%%%%%%%%%%%%%%%%%%%%%%%%%%%%%%%%%%%%%%%%%%%%%%%%%%%%%%%%%%%%%
\subsection{Files and Installation}

The package consists of the files:
%
\begin{center}
\begin{tabular}{ll}
    |README.txt|   & readme file \\
    |childdoc.ins| & installation file \\
    |childdoc.dtx| & source file \\
    |childdoc.def| & definition file \\
    |cdocsamp.tex| & sample main file \\
    |cdocsch1.tex| & sample include file \\
    |cdocsch2.tex| & sample include file \\
    |cdocspt3.tex| & sample part file \\
    |cdocspt4.tex| & sample part file \\
    |cdocsdrf.tex| & sample redirection file \\
    |cdocsfn1.tex| & sample redirection file \\
    |cdocsfn2.tex| & sample redirection file \\
    |childdoc.pdf| & manual
\end{tabular}
\end{center}
%
The distribution consists of the files
|README.txt|, |childdoc.ins| and |childdoc.dtx|.
%
\begin{itemize}
\item
Run (pdf)\LaTeX{} on |childdoc.dtx|
to compile the manual |childdoc.pdf| (this file).
\item
Run \LaTeX{} on |childdoc.ins| to create the definitions file |childdoc.def|
and the sample |cdocsamp.tex| with include files
|cdocsch1.tex|, |cdocsch2.tex|, |cdocspt3.tex|, |cdocspt4.tex|,
|cdocsdrf.tex|, |cdocsfn1.tex|, |cdocsfn2.tex|.
Then copy the file |childdoc.def| to an appropriate directory of your \LaTeX{}
distribution, e.g.\ \textit{texmf-root}|/tex/latex/childdoc|.
\end{itemize}

%%%%%%%%%%%%%%%%%%%%%%%%%%%%%%%%%%%%%%%%%%%%%%%%%%%%%%%%%%%%%%%%%%%%%%%%%%%%%%%%
\subsection{Related CTAN Packages}

There are several other packages which offer a similar functionality:
%
\begin{itemize}
\item
The packages
\href{http://ctan.org/pkg/docmute}{\textsf{docmute}},
\href{http://ctan.org/pkg/includex}{\textsf{includex}} and
\href{http://ctan.org/pkg/standalone}{\textsf{standalone}}
provide commands to include only the document body of
a child file thus allowing both files to be compiled individually.
\item
The packages \href{http://ctan.org/pkg/subdocs}{\textsf{subdocs}}
and \href{http://ctan.org/pkg/subfiles}{\textsf{subfiles}}
provide structures in which the main and child documents can be
encapsulated and allowing them to be compiled individually.
The inclusion mechanism is different from the conventional |\include|.
\item
The package \href{http://ctan.org/pkg/combine}{\textsf{combine}}
is an elaborate solution to combine several documents into one.
\end{itemize}
%
See also the CTAN topic \href{http://ctan.org/topic/subdocs}{\textsf{subdocs}}
for further related packages.
The present package differs from the above solutions in that
a document structure constructed with the conventional |\include| mechanism
just needs two extra commands at the top of every file
such that all constituent files can be compiled individually.

%%%%%%%%%%%%%%%%%%%%%%%%%%%%%%%%%%%%%%%%%%%%%%%%%%%%%%%%%%%%%%%%%%%%%%%%%%%%%%%%
%\subsection{Feature Suggestions}
%
%The following is a list of features which may be useful for future
%versions of this package:
%%
%\begin{itemize}
%\item
%\ldots
%\end{itemize}

%%%%%%%%%%%%%%%%%%%%%%%%%%%%%%%%%%%%%%%%%%%%%%%%%%%%%%%%%%%%%%%%%%%%%%%%%%%%%%%%
\subsection{Revision History}

%%%%%%%%%%%%%%%%%%%%%%%%%%%%%%%%%%%%%%%%
\paragraph{v2.0:} 2018/12/30

\begin{itemize}
\item
immediate forward processing
\item
added |\childdocby| mechanism
\item
manual restructured
\end{itemize}

%%%%%%%%%%%%%%%%%%%%%%%%%%%%%%%%%%%%%%%%
\paragraph{v1.6:} 2018/01/17

\begin{itemize}
\item
application for development of include files
\item
corrections to manual
\end{itemize}

%%%%%%%%%%%%%%%%%%%%%%%%%%%%%%%%%%%%%%%%
\paragraph{v1.5:} 2017/05/21

\begin{itemize}
\item
more complete structuring introduced
\item
|\childdocof| introduced
\item
|\childdoc| renamed to |\childdocmain|
\item
|\childredirect| renamed to |\childdocforward| and |\childdocforwardprefix|
and functionality expanded
\end{itemize}

%%%%%%%%%%%%%%%%%%%%%%%%%%%%%%%%%%%%%%%%
\paragraph{v1.0:} 2017/04/27

\begin{itemize}
\item
manual and install package
\item
first version published on CTAN
\end{itemize}

%%%%%%%%%%%%%%%%%%%%%%%%%%%%%%%%%%%%%%%%
\paragraph{v0.6:} 2017/04/26

\begin{itemize}
\item
redirection mechanism added
\end{itemize}

%%%%%%%%%%%%%%%%%%%%%%%%%%%%%%%%%%%%%%%%
\paragraph{v0.5:} 2017/04/26

\begin{itemize}
\item
functionality in definition file
\end{itemize}


%%%%%%%%%%%%%%%%%%%%%%%%%%%%%%%%%%%%%%%%%%%%%%%%%%%%%%%%%%%%%%%%%%%%%%%%%%%%%%%%
%%%%%%%%%%%%%%%%%%%%%%%%%%%%%%%%%%%%%%%%%%%%%%%%%%%%%%%%%%%%%%%%%%%%%%%%%%%%%%%%
%%%%%%%%%%%%%%%%%%%%%%%%%%%%%%%%%%%%%%%%%%%%%%%%%%%%%%%%%%%%%%%%%%%%%%%%%%%%%%%%
\appendix

\settowidth\MacroIndent{\rmfamily\scriptsize 000\ }

 \DocInput{childdoc.dtx}

\end{document}
%</driver>
% \fi
%
% %%%%%%%%%%%%%%%%%%%%%%%%%%%%%%%%%%%%%%%%%%%%%%%%%%%%%%%%%%%%%%%%%%%%%%%%%%%%%%
% %%%%%%%%%%%%%%%%%%%%%%%%%%%%%%%%%%%%%%%%%%%%%%%%%%%%%%%%%%%%%%%%%%%%%%%%%%%%%%
% \section{Sample}
%\iffalse
%<*samplemain>
%\fi
%
% The following presents a sample document
% with two chapters, two parts, a title page,
% a compile flag as well as three forwarding files to set the flag.
% It consists of eight |.tex| files:
% \begin{center}
% \begin{tabular}{ll}
% |cdocsamp.tex|&main file\\
% |cdocsch1.tex|&include file for chapter 1\\
% |cdocsch2.tex|&include file for chapter 2\\
% |cdocspt3.tex|&include file for part 3\\
% |cdocspt4.tex|&include file for part 4\\
% |cdocsdrf.tex|&forwarding file for main file in draft mode\\
% |cdocsfi1.tex|&forwarding file for final version of chapter 1\\
% |cdocsfi2.tex|&forwarding file for final version of chapter 2\\
% \end{tabular}
% \end{center}
% Each of the eight files can be compiled directly by the \LaTeX{} compiler.
%
% %%%%%%%%%%%%%%%%%%%%%%%%%%%%%%%%%%%%%%
% \paragraph{Main File.}
%
% The main file is called |cdocsamp.tex|.
%
% Load the \textsf{childdoc} definitions and
% declare the filename for the main document:
%    \begin{macrocode}
\input{childdoc.def}
\childdocmain{}
%    \end{macrocode}

% Optional override for |\version| flag:
%    \begin{macrocode}
%%\ifchilddoc\else\providecommand{\version}{draft}\fi
%    \end{macrocode}

% Define the default values for the |\version| flag
% (|final| for the main file and |draft| for childs):
%    \begin{macrocode}
\ifchilddoc
\providecommand{\version}{draft}
\else
\providecommand{\version}{final}
\fi
%    \end{macrocode}

% Load the standard document class:
%    \begin{macrocode}
\documentclass[12pt]{article}
%    \end{macrocode}

% Start the document body:
%    \begin{macrocode}
\begin{document}
%    \end{macrocode}

% Declare a title page.
% Print title, part of document being processed and version flag:
%    \begin{macrocode}
\addtocounter{page}{-1}
\begin{center}
{\LARGE\bfseries{}childdoc example\par}
\vspace{1cm}
\ifchilddoc
\ifchilddocmanual part\else chapter\fi:
`\childdocname' of `\childdocjob'\par
\else
main document: `\childdocjob'\par
\fi
version: \version\par
\end{center}
\newpage
%    \end{macrocode}

% Manually include selected file,
% otherwise process as usual:
%    \begin{macrocode}
\ifchilddocmanual
\section*{part `\childdocname'}
\input{\childdocname}
\else
%    \end{macrocode}

% Include the two chapters:
%    \begin{macrocode}
\include{cdocsch1}
\include{cdocsch2}
%    \end{macrocode}

% Include the two parts unless only chapters should be displayed:
%    \begin{macrocode}
\ifchilddoc\else
\section{part three}
\input{cdocspt3}
\section{part four}
\input{cdocspt4}
\fi
%    \end{macrocode}

% Process as usual until here:
%    \begin{macrocode}
\fi
%    \end{macrocode}

% End of document body:
%    \begin{macrocode}
\end{document}
%    \end{macrocode}
%\iffalse
%</samplemain>
%\fi
%
% %%%%%%%%%%%%%%%%%%%%%%%%%%%%%%%%%%%%%%
% \paragraph{Chapter Include Files.}
%
% The include files are called |cdocsch1.tex| and |cdocsch2.tex|.
%
%\iffalse
%<*samplechap1|samplechap2>
%\fi

% Optional override for |\version| flag:
%    \begin{macrocode}
%%\providecommand{\version}{final}
%    \end{macrocode}

% Include the main document:
%    \begin{macrocode}
\input{childdoc.def}
\childdocof{cdocsamp}
%    \end{macrocode}

%\iffalse
%</samplechap1|samplechap2>
%\fi
%
%\iffalse
%<*samplechap1>
%\fi
% Some text for chapter 1:
%    \begin{macrocode}
\section{one}
some text in chapter one
%    \end{macrocode}

%\iffalse
%</samplechap1>
%\fi
% Some text for chapter 2:
%\iffalse
%<*samplechap2>
%\fi
%    \begin{macrocode}
\section{two}
more text in chapter two
%    \end{macrocode}

%\iffalse
%</samplechap2>
%\fi
%
% %%%%%%%%%%%%%%%%%%%%%%%%%%%%%%%%%%%%%%
% \paragraph{Part Include Files.}
%
% The include files are called |cdocspt3.tex| and |cdocspt4.tex|.
%
%\iffalse
%<*samplepart3|samplepart4>
%\fi

% Optional override for |\version| flag:
%    \begin{macrocode}
%%\providecommand{\version}{final}
%    \end{macrocode}

% Include the main document:
%    \begin{macrocode}
\input{childdoc.def}
\childdocby{cdocsamp}
%    \end{macrocode}

%\iffalse
%</samplepart3|samplepart4>
%\fi
%
%\iffalse
%<*samplepart3>
%\fi
% Some text for part 3:
%    \begin{macrocode}
some text in part three
%    \end{macrocode}

%\iffalse
%</samplepart3>
%\fi
% Some text for part 4:
%\iffalse
%<*samplepart4>
%\fi
%    \begin{macrocode}
more text in part four
%    \end{macrocode}

%\iffalse
%</samplepart4>
%\fi
%
% %%%%%%%%%%%%%%%%%%%%%%%%%%%%%%%%%%%%%%
% \paragraph{Forwarding for a Complete Draft.}
%
% The following forwarding file |cdocsdrf.tex|
% compiles the main document in draft mode:
%\iffalse
%<*sampledraft>
%\fi
%    \begin{macrocode}
\def\version{draft}
\input{childdoc.def}
\childdocforward{cdocsamp}
%    \end{macrocode}

%\iffalse
%</sampledraft>
%\fi
%
% %%%%%%%%%%%%%%%%%%%%%%%%%%%%%%%%%%%%%%
% \paragraph{Forwarding for Final Version of the Chapters.}
%
% The following forwarding files |cdocsfn1.tex| and |cdocsfn2.tex|
% (with identical content)
% compile the final versions of the child documents
% |cdocsch1.tex| and |cdocsch2.tex|, respectively:
%\iffalse
%<*samplefinal>
%\fi
%    \begin{macrocode}
\def\version{final}
\input{childdoc.def}
\childdocforwardprefix[cdocsamp]{cdocsfn}{cdocsch}
%    \end{macrocode}

%\iffalse
%</samplefinal>
%\fi
%
% %%%%%%%%%%%%%%%%%%%%%%%%%%%%%%%%%%%%%%
% \paragraph{Command Line Processing.}
%
% The following three command lines generate the output files
% |cdocscld|, |cdocscl1| and |cdocscl2|
% which should be identical to
% |cdocsdrf|, |cdocsch1| and |cdocsfn2|, respectively:
% \begin{center}
% \begin{tabular}{l}
% |latex -jobname cdocscld \|\\
% |  "\def\version{draft}\input{childdoc.def}\childdocforward{cdocsamp}"|\\
% |latex -jobname cdocscl1 \|\\
% |  "\input{childdoc.def}\childdocforward[cdocsamp]{cdocsch1}"|\\
% |latex -jobname cdocscl2 \|\\
% |  "\def\version{final}\input{childdoc.def}\childdocforward{cdocsch2}"|
% \end{tabular}
% \end{center}
% Note that the trailing backslash on each first line
% merely continues the input to the second line
% (for convenient cut ant paste).
% Furthermore, the command |latex| can be replaced by any
% of its alternative versions such as |pdflatex|.
%
% %%%%%%%%%%%%%%%%%%%%%%%%%%%%%%%%%%%%%%%%%%%%%%%%%%%%%%%%%%%%%%%%%%%%%%%%%%%%%%
% %%%%%%%%%%%%%%%%%%%%%%%%%%%%%%%%%%%%%%%%%%%%%%%%%%%%%%%%%%%%%%%%%%%%%%%%%%%%%%
% \section{Implementation}
%\iffalse
%<*package>
%\fi
%
% This section describes the definitions file |childdoc.def|.

% The definitions cannot be loaded using |\usepackage| or |\RequirePackage|
% which has a mechanism to prevent loading a style file more than once.
% When loading the definitions by means of |\input|
% multiple instances have to be prevented manually:
%\iffalse
%This code needs to be before the `\ProvidesFile' directive
%which is defined at the beginning of this file.
%Therefore it is also placed there and commented out here.
%</package>
%<*discard>
%\fi
%    \begin{macrocode}
\ifdefined\childdocmain\endinput\fi
%    \end{macrocode}
%\iffalse
%</discard>
%<*package>
%\fi
%
% \macro{\ifchilddoc}
% \macro{\ifchilddocmanual}
% The conditional |\ifchilddoc| tells whether a
% child (true) or main (false) document is being compiled.
% The conditional |\ifchilddocmanual| tells whether
% the |\includeonly| mechanism is used (false) or
% the selection of child files must be performed manually (true).
% The definitions initialise to false:
%    \begin{macrocode}
\newif\ifchilddoc
\newif\ifchilddocmanual
%    \end{macrocode}

% \macro{\childdocname}
% \macro{\childdocjob}
% The macro |\childdocname| stores the name of the main document
% to be compiled. The macro |\childdocjob| stores the name of
% the document on which the \LaTeX{} compiler was originally invoked.
% The content of |\jobname| cannot be compared
% to filenames specified in the source due to different catcodes.
% The following code rescans |\jobname|, stores the result
% in |\childdocname| and saves a copy in |\childdocjob|:
%    \begin{macrocode}
\edef\childdocname{\scantokens\expandafter{\jobname\noexpand}}
\let\childdocjob\childdocname
%    \end{macrocode}

% \macro{\childdocdisable}
% The macro |\childdocdisable| prevents the main file
% from being processed more than once.
% At this stage, the main document command |\childdocmain|
% is assumed to be called once again where it should do nothing.
% Any subsequent call to it should prevent
% a secondary processing of the main document
% It overwrites the forwarding commands
% |\childdocof| and |\childdocforward|
% with empty macros to prevent further inclusions of the main document:
%    \begin{macrocode}
\newcommand{\childdocdisable}
{
  \renewcommand{\childdocmain}[1]{\renewcommand{\childdocmain}[1]{\endinput}}
  \renewcommand{\childdocof}[1]{}
  \renewcommand{\childdocby}[2][]{}
  \renewcommand{\childdocforward}[2][]{}
  \renewcommand{\childdocdisable}{}
}
%    \end{macrocode}

% \macro{\childdocmain}
% The macro |\childdocmain| is to be called at the top of the main file
% with nothing or the main filename (without extension) as argument.
% First, it breaks loops.
% If the argument is not empty and does not match |\childdocname|
% (which is set by the first inclusion of |childdoc.def|),
% |\ifchilddoc| is set to true, |\includeonly| is applied to the child file
% and |\jobname| is set to the main file
% (for proper handling of |.aux| files):
%    \begin{macrocode}
\newcommand{\childdocmain}[1]
{
  \childdocdisable\childdocmain{}
  \if?#1?\else
    \begingroup
      \def\childdoctmp{#1}
      \ifx\childdoctmp\childdocname
        \def\childdoctmp{}
      \else
        \def\childdoctmp
        {
          \childdoctrue
          \includeonly{\childdocname}
          \def\childdocjob{#1}
          \def\jobname{#1}
        }
      \fi
      \expandafter
    \endgroup
    \childdoctmp
  \fi
}
%    \end{macrocode}

% \macro{\childdocof}
% The command |\childdocof| redirects
% compilation to the main file |#1|.
%    \begin{macrocode}
\newcommand{\childdocof}[1]
{
  \childdocdisable
  \childdoctrue
  \includeonly{\childdocname}
  \def\jobname{#1}
  \def\childdocjob{#1}
  \input{#1}
}
%    \end{macrocode}

% \macro{\childdocby}
% The command |\childdocby| ....
%    \begin{macrocode}
\newcommand{\childdocby}[2][]
{
  \childdocdisable
  \childdoctrue
  \childdocmanualtrue
  \if?#1?\else
    \def\jobname{#2}
  \fi
  \def\childdocjob{#2}
  \input{#2}
  \endinput
}
%    \end{macrocode}

% \macro{\childdocforward}
% The command |\childdocforward| redirects
% compilation to the main file or
% (if the optional argument is given) a child file.
% Parameters are set as if the main file
% or a child file starting with |\childdocof| was compiled.
% Then compilation is handed over to the main file:
%    \begin{macrocode}
\newcommand{\childdocforward}[2][]
{
  \begingroup
    \if?#1?
      \def\childdoctmp
      {
        \def\childdocname{#2}
        \def\childdocjob{#2}
        \def\jobname{#2}
        \input{#2}
        \endinput
      }
    \else
      \def\childdoctmp
      {
        \childdocdisable
        \def\childdocname{#2}
        \childdoctrue
        \includeonly{#2}
        \def\childdocjob{#1}
        \def\jobname{#1}
        \input{#1}
        \endinput
      }
    \fi
    \expandafter
  \endgroup
  \childdoctmp
}
%    \end{macrocode}

% \macro{\childdocforwardprefix}
% The command |\childdocforwardprefix| redirects
% compilation to the main or a child file by means of a pattern.
% The prefix |#1| in the current filename is replaced by |#2|
% and the suffix of the current filename is kept
% (it is assumed that the filename does not contain the substring `|~~~|'
% which is used as a delimiter).
% Compilation is handed over to the new file by |\childdocforward|:
%    \begin{macrocode}
\newcommand{\childdocforwardprefix}[3][]
{
  \begingroup
    \def\childdocextract #2##1~~~{\def\childdoctmp{\childdocforward[#1]{#3##1}}}
    \expandafter\childdocextract\childdocname~~~
    \expandafter
  \endgroup
  \childdoctmp
}
%    \end{macrocode}

% \macro{\childdoc}
% The deprecated macro |\childdoc| is a legacy version of |\childdocmain|:
%    \begin{macrocode}
\newcommand{\childdoc}{\childdocmain}
%    \end{macrocode}

% \macro{\childdocredirect}
% The deprecated macro |\childdocredirect| is a legacy version
% of |\childdocforward| and |\childdocforwardprefix|:
%    \begin{macrocode}
\newcommand{\childdocredirect}[2][]
{
  \begingroup
    \if?#1?
      \def\childdoctmp{\childdocforward{#2}}
    \else
      \def\childdoctmp{\childdocforwardprefix{#1}{#2}}
    \fi
    \expandafter
  \endgroup
  \childdoctmp
}
%    \end{macrocode}

%\iffalse
%</package>
%\fi
%
\endinput

\childdocmain{}
%    \end{macrocode}

% Optional override for |\version| flag:
%    \begin{macrocode}
%%\ifchilddoc\else\providecommand{\version}{draft}\fi
%    \end{macrocode}

% Define the default values for the |\version| flag
% (|final| for the main file and |draft| for childs):
%    \begin{macrocode}
\ifchilddoc
\providecommand{\version}{draft}
\else
\providecommand{\version}{final}
\fi
%    \end{macrocode}

% Load the standard document class:
%    \begin{macrocode}
\documentclass[12pt]{article}
%    \end{macrocode}

% Start the document body:
%    \begin{macrocode}
\begin{document}
%    \end{macrocode}

% Declare a title page.
% Print title, part of document being processed and version flag:
%    \begin{macrocode}
\addtocounter{page}{-1}
\begin{center}
{\LARGE\bfseries{}childdoc example\par}
\vspace{1cm}
\ifchilddoc
\ifchilddocmanual part\else chapter\fi:
`\childdocname' of `\childdocjob'\par
\else
main document: `\childdocjob'\par
\fi
version: \version\par
\end{center}
\newpage
%    \end{macrocode}

% Manually include selected file,
% otherwise process as usual:
%    \begin{macrocode}
\ifchilddocmanual
\section*{part `\childdocname'}
\input{\childdocname}
\else
%    \end{macrocode}

% Include the two chapters:
%    \begin{macrocode}
\include{cdocsch1}
\include{cdocsch2}
%    \end{macrocode}

% Include the two parts unless only chapters should be displayed:
%    \begin{macrocode}
\ifchilddoc\else
\section{part three}
\input{cdocspt3}
\section{part four}
\input{cdocspt4}
\fi
%    \end{macrocode}

% Process as usual until here:
%    \begin{macrocode}
\fi
%    \end{macrocode}

% End of document body:
%    \begin{macrocode}
\end{document}
%    \end{macrocode}
%\iffalse
%</samplemain>
%\fi
%
% %%%%%%%%%%%%%%%%%%%%%%%%%%%%%%%%%%%%%%
% \paragraph{Chapter Include Files.}
%
% The include files are called |cdocsch1.tex| and |cdocsch2.tex|.
%
%\iffalse
%<*samplechap1|samplechap2>
%\fi

% Optional override for |\version| flag:
%    \begin{macrocode}
%%\providecommand{\version}{final}
%    \end{macrocode}

% Include the main document:
%    \begin{macrocode}
% \iffalse
%
% childdoc.dtx Copyright (C) 2017-2018 Niklas Beisert
%
% This work may be distributed and/or modified under the
% conditions of the LaTeX Project Public License, either version 1.3
% of this license or (at your option) any later version.
% The latest version of this license is in
%   http://www.latex-project.org/lppl.txt
% and version 1.3 or later is part of all distributions of LaTeX
% version 2005/12/01 or later.
%
% This work has the LPPL maintenance status `maintained'.
%
% The Current Maintainer of this work is Niklas Beisert.
%
% This work consists of the files childdoc.dtx and childdoc.ins
% and the derived files childdoc.def and cdocsamp.tex with
% cdocsch1.tex, cdocsch2.tex, cdocsdrf.tex, cdocsfn1.tex, cdocsfn2.tex.
%
%<package>\ifdefined\childdocmain\endinput\fi
%<package>\ProvidesFile{childdoc.def}[2018/12/30 v2.0 child document driver]
%<samplemain>\ProvidesFile{cdocsamp.tex}[2018/12/30 v2.0 sample for childdoc]
%<*driver>
%\ProvidesFile{childdoc.drv}[2018/12/30 v2.0 childdoc reference manual file]
\PassOptionsToClass{10pt,a4paper}{article}
\documentclass{ltxdoc}

\usepackage[margin=35mm]{geometry}
\usepackage{hyperref}
\usepackage{hyperxmp}
\usepackage[usenames]{color}

\hypersetup{colorlinks=true}
\hypersetup{pdfstartview=FitH}
\hypersetup{pdfpagemode=UseNone}
\hypersetup{pdfsource={}}
\hypersetup{pdflang={en-UK}}
\hypersetup{pdfcopyright={Copyright 2017-2018 Niklas Beisert.
  This work may be distributed and/or modified under the
  conditions of the LaTeX Project Public License, either version 1.3
  of this license or (at your option) any later version.}}
\hypersetup{pdflicenseurl={http://www.latex-project.org/lppl.txt}}
\hypersetup{pdfcontactaddress={ETH Zurich, ITP, HIT K,
  Wolfgang-Pauli-Strasse 27}}
\hypersetup{pdfcontactpostcode={8093}}
\hypersetup{pdfcontactcity={Zurich}}
\hypersetup{pdfcontactcountry={Switzerland}}
\hypersetup{pdfcontactemail={nbeisert@itp.phys.ethz.ch}}
\hypersetup{pdfcontacturl={http://people.phys.ethz.ch/\xmptilde nbeisert/}}

\newcommand{\secref}[1]{\hyperref[#1]{section \ref*{#1}}}

\parskip1ex
\parindent0pt
\let\olditemize\itemize
\def\itemize{\olditemize\parskip0pt}

\begin{document}

\title{The \textsf{childdoc} Package}
\hypersetup{pdftitle={The childdoc Package}}
\author{Niklas Beisert\\[2ex]
  Institut f\"ur Theoretische Physik\\
  Eidgen\"ossische Technische Hochschule Z\"urich\\
  Wolfgang-Pauli-Strasse 27, 8093 Z\"urich, Switzerland\\[1ex]
  \href{mailto:nbeisert@itp.phys.ethz.ch}
  {\texttt{nbeisert@itp.phys.ethz.ch}}}
\hypersetup{pdfauthor={Niklas Beisert}}
\hypersetup{pdfsubject={Manual for the LaTeX2e Package childdoc}}
\date{30 December 2018, \textsf{v2.0}}
\maketitle

\begin{abstract}\noindent
\textsf{childdoc} is a \LaTeXe{} package
that enables the direct compilation
of document sections included by |\include|
to individual files.
\end{abstract}

\begingroup
\parskip0ex
\tableofcontents
\endgroup

%%%%%%%%%%%%%%%%%%%%%%%%%%%%%%%%%%%%%%%%%%%%%%%%%%%%%%%%%%%%%%%%%%%%%%%%%%%%%%%%
%%%%%%%%%%%%%%%%%%%%%%%%%%%%%%%%%%%%%%%%%%%%%%%%%%%%%%%%%%%%%%%%%%%%%%%%%%%%%%%%
\section{Introduction}

\LaTeX{} provides a mechanism to structure a large document (such as a book)
into a main file and several child files (containing the chapters)
using the |\include| command.
This mechanism is beneficial for documents
which span hundreds of pages in order to
make the source file(s) more manageable.
Moreover, compilation can be restricted to
selected child files by means of the |\includeonly| command.
The latter feature can be used to reduce the compilation time while editing
(this was significantly more useful in the earlier days of \LaTeX{})
or to generate a smaller document which is easier to navigate.
Another application of |\includeonly| is to generate
documents consisting of selected parts of the complete document.

However, there are a few drawbacks of the plain |\include| mechanism:
\begin{itemize}
\item
The child files cannot be compiled on their own,
they can only be compiled via the main file.
A naive editing environment
(such as a text editor with an option
to have the current file processed by \LaTeX)
may require one to switch to the main file before compiling;
attempting to compile the child file produces errors.
\item
The main file must be modified (each time)
to adjust the |\includeonly| command
to the present needs. This easily leaves the main file in a messy state.
\item
The generated document will always carry the filename
of the main document. This is inconvenient if
several child files are to be compiled and
to be kept for distribution.
\end{itemize}

The present package provides a simple interface
to make child files individually compilable by \LaTeX{}.
Compiling a child file then has the same effect as compiling
the main file with an |\includeonly| command
to select the appropriate child.
Moreover the generated document will carry the name of the child
rather than the main file.
This resolves all three above issues.

This feature is meant to make the editing of books,
thesis documents and lecture notes somewhat more convenient.
However, the package can also be used efficiently for
composing a series of documents (such as exercise sheets)
which are typically distributed individually.
It then assists the author in generating the individual documents
(potentially in different versions)
as well as a document containing the collected series.
Another application is in developing style files
or other kinds of included material
where compilation of the style file could redirect
to a sample or test file.

%%%%%%%%%%%%%%%%%%%%%%%%%%%%%%%%%%%%%%%%%%%%%%%%%%%%%%%%%%%%%%%%%%%%%%%%%%%%%%%%
%%%%%%%%%%%%%%%%%%%%%%%%%%%%%%%%%%%%%%%%%%%%%%%%%%%%%%%%%%%%%%%%%%%%%%%%%%%%%%%%
\section{Usage}

First of all, the package \textsf{childdoc} is \emph{not} a standard
\LaTeXe{} |.sty| style file! Therefore it needs to be invoked in
a non-standard way.

%%%%%%%%%%%%%%%%%%%%%%%%%%%%%%%%%%%%%%%%%%%%%%%%%%%%%%%%%%%%%%%%%%%%%%%%%%%%%%%%
\subsection{Included Files}
\label{sec:include}

%%%%%%%%%%%%%%%%%%%%%%%%%%%%%%%%%%%%%%%%
\DescribeMacro{\childdocmain}
To use the package, add the commands
\begin{center}
\begin{tabular}{l}
|\input{childdoc.def}|\\
|\childdocmain{}|\\
\end{tabular}
\end{center}
at the very top of the main \LaTeX{} file,
in particular \emph{before} the |\documentclass| statement!
The argument of |\childdocmain| should be left empty
(but it must be present).

%%%%%%%%%%%%%%%%%%%%%%%%%%%%%%%%%%%%%%%%
\DescribeMacro{\childdocof}
Furthermore, add the commands
\begin{center}
\begin{tabular}{l}
|\input{childdoc.def}|\\
|\childdocof{|\textit{main}|}|\\
\end{tabular}
\end{center}
at the top of every child file \textit{child}
which is included by |\include{|\textit{child}|}|
from within the main file
(or at least for those files to be compiled individually).
The argument \textit{main} must be the filename of the main file.

There are a couple of
considerations in setting up the main and child documents:

%%%%%%%%%%%%%%%%%%%%%%%%%%%%%%%%%%%%%%%%
\paragraph{Restrictions.}

Please note the following restrictions:
\begin{itemize}
\item
|\childdocmain| must be called with one argument \textit{main}
to ensure compatibility with earlier version of the package.
It must either be empty (|\childdocmain{}|)
or precisely match the filename of the main file in which it is specified.
See \secref{sec:detection} for further information.
\item
The filename \textit{main} must be specified without the |.tex| extension.
\item
The filename \textit{main} is case sensitive
(even in case-insensitive file systems)
due to internal string comparison.
\item
The argument \textit{main} should be fully expanded, it cannot be a macro.
\item
Subdirectories and special characters should be avoided in filenames.
\item
The command |\childdocmain{|\textit{main}|}| must be followed by a whitespace.
It should not be followed immediately by another command
or by a comment mark `|%|'.
This is because the \TeX{} parser reads the token immediately following
the argument of |\childdocmain| and puts it
at the beginning of every child section;
however, a white\-space is ignored.
\end{itemize}

%%%%%%%%%%%%%%%%%%%%%%%%%%%%%%%%%%%%%%%%
\paragraph{Content of Main File.}

It is advisable to place all content in the child files included by |\include|.
Any output contained in the main file will appear in all child documents
unless suppressed manually;
it cannot be suppressed automatically by the |\includeonly| directive
and thus should normally be avoided.
A method to include some content in the main file
by means of conditional processing is described in \secref{sec:conditional}.

%%%%%%%%%%%%%%%%%%%%%%%%%%%%%%%%%%%%%%%%
\paragraph{Page Numbering.}

When only a part of the document is compiled,
the appropriate numbering of pages
(as well as other status parameters)
is determined from the |.aux| files.
The latter contain information from previous passes.
However this information needs to propagate through
all intermediate child documents.
Therefore the page numbering in child documents may well
be inconsistent until the complete document is compiled at least once.

A useful (if unconventional) way to always ensure a consistent
page numbering is to restart the numbering in each child document
and denote the pages by `\textit{child}|.|\textit{page}'
where \textit{child} represents the chapter/section number of the child file.
This can be achieved by the command
|\numberwithin{page}{|\textit{child}|}|
of the \textsf{amsmath} package
where \textit{child} can be |chapter| or |section|
depending on the chosen structuring.
Alternatively, one can modify the macro |\thepage| appropriately
and reset the counter |page| at the start of each child file.

%%%%%%%%%%%%%%%%%%%%%%%%%%%%%%%%%%%%%%%%%%%%%%%%%%%%%%%%%%%%%%%%%%%%%%%%%%%%%%%%
\subsection{Conditional Processing}
\label{sec:conditional}

The package provides a mechanism to compile different versions
of a document. To customise the versions further some conditional processing
can come in handy to distinguish which version is being compiled.
The package provides two macros to describe the compilation context:

%%%%%%%%%%%%%%%%%%%%%%%%%%%%%%%%%%%%%%%%
\DescribeMacro{\ifchilddoc}
The conditional |\ifchilddoc| distinguishes between the compilation of
child documents and the main document:
%
\begin{center}
|\ifchilddoc |\textit{child-code}| |[|\||else |\textit{main-code}]| \||fi|
\end{center}

%%%%%%%%%%%%%%%%%%%%%%%%%%%%%%%%%%%%%%%%
\DescribeMacro{\childdocname}
\DescribeMacro{\childdocjob}
The macro |\childdocname| contains the filename (without extension)
of the main or child file being processed.
Note that |\childdocjob| will always contain the name of the main file.

%%%%%%%%%%%%%%%%%%%%%%%%%%%%%%%%%%%%%%%%
\paragraph{Title Page.}

Conditional processing can be used to include a title or banner page
in the main document when proper precautions are taken.
Importantly, the code in the main file should ensure that the page counter
(as well as other status parameters which are stored in the |.aux| files)
takes the same value after the conditional processing.
Otherwise the page numbers may take divergent values
depending on which part is compiled.

For example, a title page could be declared by:
%
\begin{center}
\begin{tabular}{l}
|\ifchilddoc\||else|\\
|\addtocounter{page}{-1}|\\
\textit{code for title page}\\
|\newpage|\\
|\||fi|
\end{tabular}
\end{center}
%
A banner page for the child documents can be generated by:
%
\begin{center}
\begin{tabular}{l}
|\ifchilddoc|\\
|\addtocounter{page}{-1}|\\
\textit{code for banner page}\\
|\newpage|\\
|\||fi|
\end{tabular}
\end{center}
%
Here one could write a message such as:
\begin{center}
|This is the part \childdocname{} of \childdocjob{}.|
\end{center}

%%%%%%%%%%%%%%%%%%%%%%%%%%%%%%%%%%%%%%%%%%%%%%%%%%%%%%%%%%%%%%%%%%%%%%%%%%%%%%%%
\subsection{Flags}
\label{sec:flags}

The package makes it easy to generate different versions
of the main or child documents.
To this end compilation flags can be defined
and assigned different default values.
They will be particularly useful in conjunction
with the forwarding mechanism described in \secref{sec:forward}.

For example, it may be useful to have a flag |\version|
which can be set to |draft| or |final|.
The document source will contain some conditional code
depending on the value of |\version|.
Suppose further, the flag should default to |final| for the main file
and to |draft| for child files
which is a natural assignment for editing the document.
This is achieved by placing the following code
in the preamble of the main document
(below the |\childdocmain| directive):
%
\begin{center}
\begin{tabular}{l}
|\ifchilddoc|\\
|\providecommand{\version}{draft}|\\
|\||else|\\
|\providecommand{\version}{final}|\\
|\||fi|
\end{tabular}
\end{center}
%
The definition by |\providecommand| makes sure
that previous definitions are not overwritten.
Further statements |\providecommand{\version}{...}|
can thus be added before the above code to override it.

For the main file, one might add a line
(between |\childdocmain| and the above block)
%
\begin{center}
|%\ifchilddoc\||else\providecommand{\version}{draft}\||fi|
\end{center}
%
which can be uncommented to produce a draft version.
Likewise one can add a line to the very top of a child file
(above the |\childdocof{|\textit{main}|}| directive)
%
\begin{center}
|%\providecommand{\version}{final}|
\end{center}
%
which can be uncommented to produce the final version of this child document.

%%%%%%%%%%%%%%%%%%%%%%%%%%%%%%%%%%%%%%%%%%%%%%%%%%%%%%%%%%%%%%%%%%%%%%%%%%%%%%%%
\subsection{Forwarding}
\label{sec:forward}

Different versions of the main or child documents
using compilation flags as described in \secref{sec:flags}
can be (permanently) stored in different files
for convenient compilation, viewing and distribution.
To this end, the package defines a command
to pass on compilation to a different file:

%%%%%%%%%%%%%%%%%%%%%%%%%%%%%%%%%%%%%%%%
\DescribeMacro{\childdocforward}
The command |\childdocforward| redirects processing to
another source file:
%
\begin{center}
\begin{tabular}{l}
|\input{childdoc.def}|\\
|\childdocforward[|\textit{main}|]{|\textit{dest}|}|\\
\end{tabular}
\end{center}
%
The argument \textit{dest} is the destination file
(without extension).
It should be the main file or one of the child files.
Note that further \textsf{childdoc} directives
such as |\childdocof| and |\childdocforward|
in the indicated file will be processed in this form.
The optional argument \textit{main}
passes on directly to the main file \textit{main}
while pretending to compile the child \textit{dest}.
This form behaves as if \textit{dest}
issues |\childdocof{|\textit{main}|}| right away,
and no further \textsf{childdoc} directives will be processed.

%%%%%%%%%%%%%%%%%%%%%%%%%%%%%%%%%%%%%%%%
\DescribeMacro{\...prefix}
In the alternative form |\childdocforwardprefix|,
%
\begin{center}
\begin{tabular}{l}
|\input{childdoc.def}|\\
|\childdocforwardprefix[|\textit{main}|]{|\textit{prefix}|}{|\textit{dest}|}|
\end{tabular}
\end{center}
%
the destination file is determined by a pattern
depending on the current file:
To make this work, the current file must be called
`{\textit{prefix}\hspace{0.2em}\textit{suffix}}'
with \textit{prefix} matching precisely the argument.
Processing is then passed on to the file
`{\textit{dest}\hspace{0.2em}\textit{suffix}}'.
Surely, the same effect is achieved by
directly specifying the
argument `{\textit{dest}\hspace{0.2em}\textit{suffix}}'
in the first form.
However, that requires to set up a different file
for each child. With the alternative form of the command
all these files can have exactly the same content
which simplifies setting them up and maintaining them.

For example, the following file |draft.tex|
with a compilation flag |\version| as described in \secref{sec:flags}
compiles the main document as a draft:
%
\begin{center}
\begin{tabular}{l}
|\def\version{draft}|\\
|\input{childdoc.def}|\\
|\childdocforward{|\textit{main}|}|
\end{tabular}
\end{center}
%
Likewise, the following files |final|\textit{nn}|.tex|
compile the final version of the child document
|child|\textit{nn}|.tex|:
%
\begin{center}
\begin{tabular}{l}
|\def\version{final}|\\
|\input{childdoc.def}|\\
|\childdocforwardprefix{final}{child}|
\end{tabular}
\end{center}
%

Note that when several versions of a main file and/or of each child file
are to be generated, it may be convenient to set up a |Makefile| or
shell script to automatise the process.

%%%%%%%%%%%%%%%%%%%%%%%%%%%%%%%%%%%%%%%%%%%%%%%%%%%%%%%%%%%%%%%%%%%%%%%%%%%%%%%%
\subsection{Command Line Processing}
\label{sec:commandline}

The effect of redirection files can also be achieved by invoking
the \LaTeX{} compiler with a more elaborate command line.
Most conveniently this should be done as part
of a shell script or a |Makefile|.

When using \textsf{childdoc} in the main file, the following
command lines effectively perform a redirection
(note that depending on the shell being used,
backslashes may have to be doubled: `|\|' $\to$ `|\\|'):
%
\begin{center}
|... -jobname "|\textit{target}|" |\\|"|[\textit{flags}]%
|\input{childdoc.def}\childdocforward[|\textit{main}|]{|\textit{dest}|}"|
\end{center}
%
Here \textit{target} is the name of the output file,
\textit{main} is the name of the main file
and \textit{dest} is the name of the main or child file to be processed
(all filenames without extensions).
The optional argument \textit{main} can be omitted
if \textit{main} matches \textit{dest}.
Optionally, compilation \textit{flags} can be defined via |\def| commands.
This command line makes the \TeX{} engine believe
it is compiling the file \textit{target}
whose content is specified as the latter parameter.
The provided code then forwards the processing to
\textit{main} or \textit{dest} as described in \secref{sec:forward}.

%%%%%%%%%%%%%%%%%%%%%%%%%%%%%%%%%%%%%%%%%%%%%%%%%%%%%%%%%%%%%%%%%%%%%%%%%%%%%%%%
\subsection{Include by Input}
\label{sec:input}

Including child documents by |\include| has some restrictions by design.
Most notably, the content of a child document always occupies
its own set of pages; pages cannot be shared between child documents.
Usually, this behaviour makes perfect sense
because each child document contain an essential part of the document.
However, in some situations it may be desirable to compose
a document from a collection of parts
without having mandatory page breaks between then.
For this case, the package
provides a mechanism to include parts
by |\input| which can also be processed individually.
However, by construction this mechanism
requires manual handling of the content to be output.

%%%%%%%%%%%%%%%%%%%%%%%%%%%%%%%%%%%%%%%%
\DescribeMacro{\ifchilddocmanual}
The main file should be prepared as usual, see \secref{sec:include}.
However, the document body must make a distinction
between processing of an individual part and of the main document, e.g.:
%
\begin{center}
\begin{tabular}{l}
|\ifchilddocmanual|\\
|\input{\childdocname}|\\
|\||else|\\
\textit{document body with }|\input{|\textit{part}|}|\\
|\||fi|
\end{tabular}
\end{center}
%
The conditional |\ifchilddocmanual| is true whenever
a part to be included by |\input| is being compiled,
and the name of the part is stored in |\childdocname|.

%%%%%%%%%%%%%%%%%%%%%%%%%%%%%%%%%%%%%%%%
\DescribeMacro{\childdocby}
Each part to be included by |\input| should start with:
%
\begin{center}
\begin{tabular}{l}
|\input{childdoc.def}|\\
|\childdocby{|\textit{main}|}|\\
\end{tabular}
\end{center}
%
The directive |\childdocby| is similar to |\childdocof|
described in \secref{sec:include},
but the subsequent selection of content must be done manually.
To that end, both |\ifchilddoc| and |\ifchilddocmanual|
will be true upon processing of a part,
and the name of the part is stored in |\childdocname|.
Note that |\jobname| will be set to the filename of the current part
so that each part receives an individual |.aux| file
that does not interfere with the |.aux| file(s) of the main document.
This behaviour can be altered by the alternative form
|\childdocby[*]{|\textit{main}|}| (with a non-empty optional argument)
which uses the |.aux| file of the main document
by setting |\jobname| to \textit{main}.

%%%%%%%%%%%%%%%%%%%%%%%%%%%%%%%%%%%%%%%%%%%%%%%%%%%%%%%%%%%%%%%%%%%%%%%%%%%%%%%%
\subsection{Driver Development}
\label{sec:driver}

The \textsf{childdoc} mechanism can also be use for the development
of definition files such as \LaTeX{} styles or classes.
This case differs from the above setup with multiple parts
included by |\include| in that no |\includeonly| should be invoked.
This can be achieved by starting the include file
(before |\ProvidesPackage|) with:
%
\begin{center}
\begin{tabular}{l}
|\input{childdoc.def}|\\
|\childdocforward{|\textit{main}|}|\\
\end{tabular}
\end{center}
%
or alternatively with:
%
\begin{center}
\begin{tabular}{l}
|\input{childdoc.def}|\\
|\childdocby{|\textit{main}|}|\\
\end{tabular}
\end{center}
%
Both forms have slightly different effects as described above.
The main file is prepared as usual, see \secref{sec:include}.

%%%%%%%%%%%%%%%%%%%%%%%%%%%%%%%%%%%%%%%%%%%%%%%%%%%%%%%%%%%%%%%%%%%%%%%%%%%%%%%%
\subsection{Legacy Detection}
\label{sec:detection}

The directive |\childdocmain| in the main file can detect
whether the complete document or merely a child is to be compiled
even without using the directive |\childdocof|.
This method is deprecated because it is less robust
and there is no compelling reason to use it;
it is merely provided for backward compatibility
and it may be removed in future versions.

If the detection mechanism is to be used,
it is mandatory to correctly specify
the filename of the main file as the argument of |\childdocmain|:
%
\begin{center}
\begin{tabular}{l}
|\input{childdoc.def}|\\
|\childdocmain{|\textit{main}|}|\\
\end{tabular}
\end{center}
%
If |\jobname| does not match the argument \textit{main} of |\childdocmain|,
it is assumed that |\jobname| points to the child file to be compiled.
When using |\childdocmain| with the main file specified as argument,
it suffices to start a child file
with just |\input{|\textit{main}|}|
without loading of the package and using |\childdocof|.
If instead all processing is done
with the appropriate \textsf{childdoc} directives,
the argument of \textit{main} of |\childdocmain| can be empty.

An alternative version of the command line processing described
in \secref{sec:commandline} using the detection mechanism reads:
%
\begin{center}
|... -jobname "|\textit{target}|" "|[\textit{flags}]%
[|\def\jobname{|\textit{dest}|}|]|\input{|\textit{main}|}"|
\end{center}

%%%%%%%%%%%%%%%%%%%%%%%%%%%%%%%%%%%%%%%%%%%%%%%%%%%%%%%%%%%%%%%%%%%%%%%%%%%%%%%%
\subsection{Manual Code}
\label{sec:manual}

In case one cannot be certain whether the definitions file |childdoc.def|
is installed on the target \TeX{} distribution
and one prefers not to ship it,
it is conceivable to paste a few relevant commands into the sources.

To that end, drop all statements |\input{childdoc.def}|
and perform the replacements as outlined below.
Instead of |\childdocmain{|\textit{main}|}| add the following code
to the top of the main file:
%
\begin{center}
\begin{tabular}{l}
|\||ifdefined\childdocname\endinput\||fi\newif\ifchilddoc|\\
|\edef\childdocname{\scantokens\expandafter{\jobname\noexpand}}|\\
|\def\childdocmain{|\textit{main}|}\||ifx\childdocmain\childdocname\||else|\\
|\childdoctrue\includeonly{\childdocname}\let\jobname\childdocmain\||fi|\\
\end{tabular}
\end{center}
%
Instead of |\childdocof{|\textit{main}|}| just include the main file
at the top of each child file:
%
\begin{center}
|\input{|\textit{main}|}|
\end{center}
%
A simple redirection |\childdocforward{|\textit{dest}|}| is achieved by:
%
\begin{center}
|\def\jobname{|\textit{dest}|}\input{\jobname}|
\end{center}
%
The redirection with prefix
|\childdocforwardprefix[|\textit{prefix}|]{|\textit{dest}|}|
is accomplished by:
%
\begin{center}
\begin{tabular}{l}
|{\edef\jobname{\scantokens\expandafter{\jobname\noexpand}}|\\
|\def\redirectjob |\textit{prefix}|#1~~~{\gdef\jobname{|\textit{dest}|#1}}|\\
|\expandafter\redirectjob\jobname~~~}\input{\jobname}|
\end{tabular}
\end{center}

In an alternative approach,
child documents can be compiled by a specific command line
without additional code or specific definitions:
%
\begin{center}
|... -jobname "|\textit{target}|" "|[\textit{flags}]%
|\includeonly{|\textit{dest}|}\input{|\textit{main}|}"|
\end{center}
%

%%%%%%%%%%%%%%%%%%%%%%%%%%%%%%%%%%%%%%%%%%%%%%%%%%%%%%%%%%%%%%%%%%%%%%%%%%%%%%%%
%%%%%%%%%%%%%%%%%%%%%%%%%%%%%%%%%%%%%%%%%%%%%%%%%%%%%%%%%%%%%%%%%%%%%%%%%%%%%%%%
\section{Information}

%%%%%%%%%%%%%%%%%%%%%%%%%%%%%%%%%%%%%%%%%%%%%%%%%%%%%%%%%%%%%%%%%%%%%%%%%%%%%%%%
\subsection{Copyright}

Copyright \copyright{} 2017--2018 Niklas Beisert

This work may be distributed and/or modified under the
conditions of the \LaTeX{} Project Public License, either version 1.3
of this license or (at your option) any later version.
The latest version of this license is in
  \url{http://www.latex-project.org/lppl.txt}
and version 1.3 or later is part of all distributions of \LaTeX{}
version 2005/12/01 or later.

This work has the LPPL maintenance status `maintained'.

The Current Maintainer of this work is Niklas Beisert.

This work consists of the files |README.txt|, |childdoc.ins| and |childdoc.dtx|
as well as the derived files |childdoc.def|, |cdocsamp.tex|
with |cdocsch1.tex|, |cdocsch2.tex|, |cdocspt3.tex|, |cdocspt4.tex|,
|cdocsdrf.tex|, |cdocsfn1.tex|, |cdocsfn2.tex|
as well as |childdoc.pdf|.

%%%%%%%%%%%%%%%%%%%%%%%%%%%%%%%%%%%%%%%%%%%%%%%%%%%%%%%%%%%%%%%%%%%%%%%%%%%%%%%%
\subsection{Files and Installation}

The package consists of the files:
%
\begin{center}
\begin{tabular}{ll}
    |README.txt|   & readme file \\
    |childdoc.ins| & installation file \\
    |childdoc.dtx| & source file \\
    |childdoc.def| & definition file \\
    |cdocsamp.tex| & sample main file \\
    |cdocsch1.tex| & sample include file \\
    |cdocsch2.tex| & sample include file \\
    |cdocspt3.tex| & sample part file \\
    |cdocspt4.tex| & sample part file \\
    |cdocsdrf.tex| & sample redirection file \\
    |cdocsfn1.tex| & sample redirection file \\
    |cdocsfn2.tex| & sample redirection file \\
    |childdoc.pdf| & manual
\end{tabular}
\end{center}
%
The distribution consists of the files
|README.txt|, |childdoc.ins| and |childdoc.dtx|.
%
\begin{itemize}
\item
Run (pdf)\LaTeX{} on |childdoc.dtx|
to compile the manual |childdoc.pdf| (this file).
\item
Run \LaTeX{} on |childdoc.ins| to create the definitions file |childdoc.def|
and the sample |cdocsamp.tex| with include files
|cdocsch1.tex|, |cdocsch2.tex|, |cdocspt3.tex|, |cdocspt4.tex|,
|cdocsdrf.tex|, |cdocsfn1.tex|, |cdocsfn2.tex|.
Then copy the file |childdoc.def| to an appropriate directory of your \LaTeX{}
distribution, e.g.\ \textit{texmf-root}|/tex/latex/childdoc|.
\end{itemize}

%%%%%%%%%%%%%%%%%%%%%%%%%%%%%%%%%%%%%%%%%%%%%%%%%%%%%%%%%%%%%%%%%%%%%%%%%%%%%%%%
\subsection{Related CTAN Packages}

There are several other packages which offer a similar functionality:
%
\begin{itemize}
\item
The packages
\href{http://ctan.org/pkg/docmute}{\textsf{docmute}},
\href{http://ctan.org/pkg/includex}{\textsf{includex}} and
\href{http://ctan.org/pkg/standalone}{\textsf{standalone}}
provide commands to include only the document body of
a child file thus allowing both files to be compiled individually.
\item
The packages \href{http://ctan.org/pkg/subdocs}{\textsf{subdocs}}
and \href{http://ctan.org/pkg/subfiles}{\textsf{subfiles}}
provide structures in which the main and child documents can be
encapsulated and allowing them to be compiled individually.
The inclusion mechanism is different from the conventional |\include|.
\item
The package \href{http://ctan.org/pkg/combine}{\textsf{combine}}
is an elaborate solution to combine several documents into one.
\end{itemize}
%
See also the CTAN topic \href{http://ctan.org/topic/subdocs}{\textsf{subdocs}}
for further related packages.
The present package differs from the above solutions in that
a document structure constructed with the conventional |\include| mechanism
just needs two extra commands at the top of every file
such that all constituent files can be compiled individually.

%%%%%%%%%%%%%%%%%%%%%%%%%%%%%%%%%%%%%%%%%%%%%%%%%%%%%%%%%%%%%%%%%%%%%%%%%%%%%%%%
%\subsection{Feature Suggestions}
%
%The following is a list of features which may be useful for future
%versions of this package:
%%
%\begin{itemize}
%\item
%\ldots
%\end{itemize}

%%%%%%%%%%%%%%%%%%%%%%%%%%%%%%%%%%%%%%%%%%%%%%%%%%%%%%%%%%%%%%%%%%%%%%%%%%%%%%%%
\subsection{Revision History}

%%%%%%%%%%%%%%%%%%%%%%%%%%%%%%%%%%%%%%%%
\paragraph{v2.0:} 2018/12/30

\begin{itemize}
\item
immediate forward processing
\item
added |\childdocby| mechanism
\item
manual restructured
\end{itemize}

%%%%%%%%%%%%%%%%%%%%%%%%%%%%%%%%%%%%%%%%
\paragraph{v1.6:} 2018/01/17

\begin{itemize}
\item
application for development of include files
\item
corrections to manual
\end{itemize}

%%%%%%%%%%%%%%%%%%%%%%%%%%%%%%%%%%%%%%%%
\paragraph{v1.5:} 2017/05/21

\begin{itemize}
\item
more complete structuring introduced
\item
|\childdocof| introduced
\item
|\childdoc| renamed to |\childdocmain|
\item
|\childredirect| renamed to |\childdocforward| and |\childdocforwardprefix|
and functionality expanded
\end{itemize}

%%%%%%%%%%%%%%%%%%%%%%%%%%%%%%%%%%%%%%%%
\paragraph{v1.0:} 2017/04/27

\begin{itemize}
\item
manual and install package
\item
first version published on CTAN
\end{itemize}

%%%%%%%%%%%%%%%%%%%%%%%%%%%%%%%%%%%%%%%%
\paragraph{v0.6:} 2017/04/26

\begin{itemize}
\item
redirection mechanism added
\end{itemize}

%%%%%%%%%%%%%%%%%%%%%%%%%%%%%%%%%%%%%%%%
\paragraph{v0.5:} 2017/04/26

\begin{itemize}
\item
functionality in definition file
\end{itemize}


%%%%%%%%%%%%%%%%%%%%%%%%%%%%%%%%%%%%%%%%%%%%%%%%%%%%%%%%%%%%%%%%%%%%%%%%%%%%%%%%
%%%%%%%%%%%%%%%%%%%%%%%%%%%%%%%%%%%%%%%%%%%%%%%%%%%%%%%%%%%%%%%%%%%%%%%%%%%%%%%%
%%%%%%%%%%%%%%%%%%%%%%%%%%%%%%%%%%%%%%%%%%%%%%%%%%%%%%%%%%%%%%%%%%%%%%%%%%%%%%%%
\appendix

\settowidth\MacroIndent{\rmfamily\scriptsize 000\ }

 \DocInput{childdoc.dtx}

\end{document}
%</driver>
% \fi
%
% %%%%%%%%%%%%%%%%%%%%%%%%%%%%%%%%%%%%%%%%%%%%%%%%%%%%%%%%%%%%%%%%%%%%%%%%%%%%%%
% %%%%%%%%%%%%%%%%%%%%%%%%%%%%%%%%%%%%%%%%%%%%%%%%%%%%%%%%%%%%%%%%%%%%%%%%%%%%%%
% \section{Sample}
%\iffalse
%<*samplemain>
%\fi
%
% The following presents a sample document
% with two chapters, two parts, a title page,
% a compile flag as well as three forwarding files to set the flag.
% It consists of eight |.tex| files:
% \begin{center}
% \begin{tabular}{ll}
% |cdocsamp.tex|&main file\\
% |cdocsch1.tex|&include file for chapter 1\\
% |cdocsch2.tex|&include file for chapter 2\\
% |cdocspt3.tex|&include file for part 3\\
% |cdocspt4.tex|&include file for part 4\\
% |cdocsdrf.tex|&forwarding file for main file in draft mode\\
% |cdocsfi1.tex|&forwarding file for final version of chapter 1\\
% |cdocsfi2.tex|&forwarding file for final version of chapter 2\\
% \end{tabular}
% \end{center}
% Each of the eight files can be compiled directly by the \LaTeX{} compiler.
%
% %%%%%%%%%%%%%%%%%%%%%%%%%%%%%%%%%%%%%%
% \paragraph{Main File.}
%
% The main file is called |cdocsamp.tex|.
%
% Load the \textsf{childdoc} definitions and
% declare the filename for the main document:
%    \begin{macrocode}
\input{childdoc.def}
\childdocmain{}
%    \end{macrocode}

% Optional override for |\version| flag:
%    \begin{macrocode}
%%\ifchilddoc\else\providecommand{\version}{draft}\fi
%    \end{macrocode}

% Define the default values for the |\version| flag
% (|final| for the main file and |draft| for childs):
%    \begin{macrocode}
\ifchilddoc
\providecommand{\version}{draft}
\else
\providecommand{\version}{final}
\fi
%    \end{macrocode}

% Load the standard document class:
%    \begin{macrocode}
\documentclass[12pt]{article}
%    \end{macrocode}

% Start the document body:
%    \begin{macrocode}
\begin{document}
%    \end{macrocode}

% Declare a title page.
% Print title, part of document being processed and version flag:
%    \begin{macrocode}
\addtocounter{page}{-1}
\begin{center}
{\LARGE\bfseries{}childdoc example\par}
\vspace{1cm}
\ifchilddoc
\ifchilddocmanual part\else chapter\fi:
`\childdocname' of `\childdocjob'\par
\else
main document: `\childdocjob'\par
\fi
version: \version\par
\end{center}
\newpage
%    \end{macrocode}

% Manually include selected file,
% otherwise process as usual:
%    \begin{macrocode}
\ifchilddocmanual
\section*{part `\childdocname'}
\input{\childdocname}
\else
%    \end{macrocode}

% Include the two chapters:
%    \begin{macrocode}
\include{cdocsch1}
\include{cdocsch2}
%    \end{macrocode}

% Include the two parts unless only chapters should be displayed:
%    \begin{macrocode}
\ifchilddoc\else
\section{part three}
\input{cdocspt3}
\section{part four}
\input{cdocspt4}
\fi
%    \end{macrocode}

% Process as usual until here:
%    \begin{macrocode}
\fi
%    \end{macrocode}

% End of document body:
%    \begin{macrocode}
\end{document}
%    \end{macrocode}
%\iffalse
%</samplemain>
%\fi
%
% %%%%%%%%%%%%%%%%%%%%%%%%%%%%%%%%%%%%%%
% \paragraph{Chapter Include Files.}
%
% The include files are called |cdocsch1.tex| and |cdocsch2.tex|.
%
%\iffalse
%<*samplechap1|samplechap2>
%\fi

% Optional override for |\version| flag:
%    \begin{macrocode}
%%\providecommand{\version}{final}
%    \end{macrocode}

% Include the main document:
%    \begin{macrocode}
\input{childdoc.def}
\childdocof{cdocsamp}
%    \end{macrocode}

%\iffalse
%</samplechap1|samplechap2>
%\fi
%
%\iffalse
%<*samplechap1>
%\fi
% Some text for chapter 1:
%    \begin{macrocode}
\section{one}
some text in chapter one
%    \end{macrocode}

%\iffalse
%</samplechap1>
%\fi
% Some text for chapter 2:
%\iffalse
%<*samplechap2>
%\fi
%    \begin{macrocode}
\section{two}
more text in chapter two
%    \end{macrocode}

%\iffalse
%</samplechap2>
%\fi
%
% %%%%%%%%%%%%%%%%%%%%%%%%%%%%%%%%%%%%%%
% \paragraph{Part Include Files.}
%
% The include files are called |cdocspt3.tex| and |cdocspt4.tex|.
%
%\iffalse
%<*samplepart3|samplepart4>
%\fi

% Optional override for |\version| flag:
%    \begin{macrocode}
%%\providecommand{\version}{final}
%    \end{macrocode}

% Include the main document:
%    \begin{macrocode}
\input{childdoc.def}
\childdocby{cdocsamp}
%    \end{macrocode}

%\iffalse
%</samplepart3|samplepart4>
%\fi
%
%\iffalse
%<*samplepart3>
%\fi
% Some text for part 3:
%    \begin{macrocode}
some text in part three
%    \end{macrocode}

%\iffalse
%</samplepart3>
%\fi
% Some text for part 4:
%\iffalse
%<*samplepart4>
%\fi
%    \begin{macrocode}
more text in part four
%    \end{macrocode}

%\iffalse
%</samplepart4>
%\fi
%
% %%%%%%%%%%%%%%%%%%%%%%%%%%%%%%%%%%%%%%
% \paragraph{Forwarding for a Complete Draft.}
%
% The following forwarding file |cdocsdrf.tex|
% compiles the main document in draft mode:
%\iffalse
%<*sampledraft>
%\fi
%    \begin{macrocode}
\def\version{draft}
\input{childdoc.def}
\childdocforward{cdocsamp}
%    \end{macrocode}

%\iffalse
%</sampledraft>
%\fi
%
% %%%%%%%%%%%%%%%%%%%%%%%%%%%%%%%%%%%%%%
% \paragraph{Forwarding for Final Version of the Chapters.}
%
% The following forwarding files |cdocsfn1.tex| and |cdocsfn2.tex|
% (with identical content)
% compile the final versions of the child documents
% |cdocsch1.tex| and |cdocsch2.tex|, respectively:
%\iffalse
%<*samplefinal>
%\fi
%    \begin{macrocode}
\def\version{final}
\input{childdoc.def}
\childdocforwardprefix[cdocsamp]{cdocsfn}{cdocsch}
%    \end{macrocode}

%\iffalse
%</samplefinal>
%\fi
%
% %%%%%%%%%%%%%%%%%%%%%%%%%%%%%%%%%%%%%%
% \paragraph{Command Line Processing.}
%
% The following three command lines generate the output files
% |cdocscld|, |cdocscl1| and |cdocscl2|
% which should be identical to
% |cdocsdrf|, |cdocsch1| and |cdocsfn2|, respectively:
% \begin{center}
% \begin{tabular}{l}
% |latex -jobname cdocscld \|\\
% |  "\def\version{draft}\input{childdoc.def}\childdocforward{cdocsamp}"|\\
% |latex -jobname cdocscl1 \|\\
% |  "\input{childdoc.def}\childdocforward[cdocsamp]{cdocsch1}"|\\
% |latex -jobname cdocscl2 \|\\
% |  "\def\version{final}\input{childdoc.def}\childdocforward{cdocsch2}"|
% \end{tabular}
% \end{center}
% Note that the trailing backslash on each first line
% merely continues the input to the second line
% (for convenient cut ant paste).
% Furthermore, the command |latex| can be replaced by any
% of its alternative versions such as |pdflatex|.
%
% %%%%%%%%%%%%%%%%%%%%%%%%%%%%%%%%%%%%%%%%%%%%%%%%%%%%%%%%%%%%%%%%%%%%%%%%%%%%%%
% %%%%%%%%%%%%%%%%%%%%%%%%%%%%%%%%%%%%%%%%%%%%%%%%%%%%%%%%%%%%%%%%%%%%%%%%%%%%%%
% \section{Implementation}
%\iffalse
%<*package>
%\fi
%
% This section describes the definitions file |childdoc.def|.

% The definitions cannot be loaded using |\usepackage| or |\RequirePackage|
% which has a mechanism to prevent loading a style file more than once.
% When loading the definitions by means of |\input|
% multiple instances have to be prevented manually:
%\iffalse
%This code needs to be before the `\ProvidesFile' directive
%which is defined at the beginning of this file.
%Therefore it is also placed there and commented out here.
%</package>
%<*discard>
%\fi
%    \begin{macrocode}
\ifdefined\childdocmain\endinput\fi
%    \end{macrocode}
%\iffalse
%</discard>
%<*package>
%\fi
%
% \macro{\ifchilddoc}
% \macro{\ifchilddocmanual}
% The conditional |\ifchilddoc| tells whether a
% child (true) or main (false) document is being compiled.
% The conditional |\ifchilddocmanual| tells whether
% the |\includeonly| mechanism is used (false) or
% the selection of child files must be performed manually (true).
% The definitions initialise to false:
%    \begin{macrocode}
\newif\ifchilddoc
\newif\ifchilddocmanual
%    \end{macrocode}

% \macro{\childdocname}
% \macro{\childdocjob}
% The macro |\childdocname| stores the name of the main document
% to be compiled. The macro |\childdocjob| stores the name of
% the document on which the \LaTeX{} compiler was originally invoked.
% The content of |\jobname| cannot be compared
% to filenames specified in the source due to different catcodes.
% The following code rescans |\jobname|, stores the result
% in |\childdocname| and saves a copy in |\childdocjob|:
%    \begin{macrocode}
\edef\childdocname{\scantokens\expandafter{\jobname\noexpand}}
\let\childdocjob\childdocname
%    \end{macrocode}

% \macro{\childdocdisable}
% The macro |\childdocdisable| prevents the main file
% from being processed more than once.
% At this stage, the main document command |\childdocmain|
% is assumed to be called once again where it should do nothing.
% Any subsequent call to it should prevent
% a secondary processing of the main document
% It overwrites the forwarding commands
% |\childdocof| and |\childdocforward|
% with empty macros to prevent further inclusions of the main document:
%    \begin{macrocode}
\newcommand{\childdocdisable}
{
  \renewcommand{\childdocmain}[1]{\renewcommand{\childdocmain}[1]{\endinput}}
  \renewcommand{\childdocof}[1]{}
  \renewcommand{\childdocby}[2][]{}
  \renewcommand{\childdocforward}[2][]{}
  \renewcommand{\childdocdisable}{}
}
%    \end{macrocode}

% \macro{\childdocmain}
% The macro |\childdocmain| is to be called at the top of the main file
% with nothing or the main filename (without extension) as argument.
% First, it breaks loops.
% If the argument is not empty and does not match |\childdocname|
% (which is set by the first inclusion of |childdoc.def|),
% |\ifchilddoc| is set to true, |\includeonly| is applied to the child file
% and |\jobname| is set to the main file
% (for proper handling of |.aux| files):
%    \begin{macrocode}
\newcommand{\childdocmain}[1]
{
  \childdocdisable\childdocmain{}
  \if?#1?\else
    \begingroup
      \def\childdoctmp{#1}
      \ifx\childdoctmp\childdocname
        \def\childdoctmp{}
      \else
        \def\childdoctmp
        {
          \childdoctrue
          \includeonly{\childdocname}
          \def\childdocjob{#1}
          \def\jobname{#1}
        }
      \fi
      \expandafter
    \endgroup
    \childdoctmp
  \fi
}
%    \end{macrocode}

% \macro{\childdocof}
% The command |\childdocof| redirects
% compilation to the main file |#1|.
%    \begin{macrocode}
\newcommand{\childdocof}[1]
{
  \childdocdisable
  \childdoctrue
  \includeonly{\childdocname}
  \def\jobname{#1}
  \def\childdocjob{#1}
  \input{#1}
}
%    \end{macrocode}

% \macro{\childdocby}
% The command |\childdocby| ....
%    \begin{macrocode}
\newcommand{\childdocby}[2][]
{
  \childdocdisable
  \childdoctrue
  \childdocmanualtrue
  \if?#1?\else
    \def\jobname{#2}
  \fi
  \def\childdocjob{#2}
  \input{#2}
  \endinput
}
%    \end{macrocode}

% \macro{\childdocforward}
% The command |\childdocforward| redirects
% compilation to the main file or
% (if the optional argument is given) a child file.
% Parameters are set as if the main file
% or a child file starting with |\childdocof| was compiled.
% Then compilation is handed over to the main file:
%    \begin{macrocode}
\newcommand{\childdocforward}[2][]
{
  \begingroup
    \if?#1?
      \def\childdoctmp
      {
        \def\childdocname{#2}
        \def\childdocjob{#2}
        \def\jobname{#2}
        \input{#2}
        \endinput
      }
    \else
      \def\childdoctmp
      {
        \childdocdisable
        \def\childdocname{#2}
        \childdoctrue
        \includeonly{#2}
        \def\childdocjob{#1}
        \def\jobname{#1}
        \input{#1}
        \endinput
      }
    \fi
    \expandafter
  \endgroup
  \childdoctmp
}
%    \end{macrocode}

% \macro{\childdocforwardprefix}
% The command |\childdocforwardprefix| redirects
% compilation to the main or a child file by means of a pattern.
% The prefix |#1| in the current filename is replaced by |#2|
% and the suffix of the current filename is kept
% (it is assumed that the filename does not contain the substring `|~~~|'
% which is used as a delimiter).
% Compilation is handed over to the new file by |\childdocforward|:
%    \begin{macrocode}
\newcommand{\childdocforwardprefix}[3][]
{
  \begingroup
    \def\childdocextract #2##1~~~{\def\childdoctmp{\childdocforward[#1]{#3##1}}}
    \expandafter\childdocextract\childdocname~~~
    \expandafter
  \endgroup
  \childdoctmp
}
%    \end{macrocode}

% \macro{\childdoc}
% The deprecated macro |\childdoc| is a legacy version of |\childdocmain|:
%    \begin{macrocode}
\newcommand{\childdoc}{\childdocmain}
%    \end{macrocode}

% \macro{\childdocredirect}
% The deprecated macro |\childdocredirect| is a legacy version
% of |\childdocforward| and |\childdocforwardprefix|:
%    \begin{macrocode}
\newcommand{\childdocredirect}[2][]
{
  \begingroup
    \if?#1?
      \def\childdoctmp{\childdocforward{#2}}
    \else
      \def\childdoctmp{\childdocforwardprefix{#1}{#2}}
    \fi
    \expandafter
  \endgroup
  \childdoctmp
}
%    \end{macrocode}

%\iffalse
%</package>
%\fi
%
\endinput

\childdocof{cdocsamp}
%    \end{macrocode}

%\iffalse
%</samplechap1|samplechap2>
%\fi
%
%\iffalse
%<*samplechap1>
%\fi
% Some text for chapter 1:
%    \begin{macrocode}
\section{one}
some text in chapter one
%    \end{macrocode}

%\iffalse
%</samplechap1>
%\fi
% Some text for chapter 2:
%\iffalse
%<*samplechap2>
%\fi
%    \begin{macrocode}
\section{two}
more text in chapter two
%    \end{macrocode}

%\iffalse
%</samplechap2>
%\fi
%
% %%%%%%%%%%%%%%%%%%%%%%%%%%%%%%%%%%%%%%
% \paragraph{Part Include Files.}
%
% The include files are called |cdocspt3.tex| and |cdocspt4.tex|.
%
%\iffalse
%<*samplepart3|samplepart4>
%\fi

% Optional override for |\version| flag:
%    \begin{macrocode}
%%\providecommand{\version}{final}
%    \end{macrocode}

% Include the main document:
%    \begin{macrocode}
% \iffalse
%
% childdoc.dtx Copyright (C) 2017-2018 Niklas Beisert
%
% This work may be distributed and/or modified under the
% conditions of the LaTeX Project Public License, either version 1.3
% of this license or (at your option) any later version.
% The latest version of this license is in
%   http://www.latex-project.org/lppl.txt
% and version 1.3 or later is part of all distributions of LaTeX
% version 2005/12/01 or later.
%
% This work has the LPPL maintenance status `maintained'.
%
% The Current Maintainer of this work is Niklas Beisert.
%
% This work consists of the files childdoc.dtx and childdoc.ins
% and the derived files childdoc.def and cdocsamp.tex with
% cdocsch1.tex, cdocsch2.tex, cdocsdrf.tex, cdocsfn1.tex, cdocsfn2.tex.
%
%<package>\ifdefined\childdocmain\endinput\fi
%<package>\ProvidesFile{childdoc.def}[2018/12/30 v2.0 child document driver]
%<samplemain>\ProvidesFile{cdocsamp.tex}[2018/12/30 v2.0 sample for childdoc]
%<*driver>
%\ProvidesFile{childdoc.drv}[2018/12/30 v2.0 childdoc reference manual file]
\PassOptionsToClass{10pt,a4paper}{article}
\documentclass{ltxdoc}

\usepackage[margin=35mm]{geometry}
\usepackage{hyperref}
\usepackage{hyperxmp}
\usepackage[usenames]{color}

\hypersetup{colorlinks=true}
\hypersetup{pdfstartview=FitH}
\hypersetup{pdfpagemode=UseNone}
\hypersetup{pdfsource={}}
\hypersetup{pdflang={en-UK}}
\hypersetup{pdfcopyright={Copyright 2017-2018 Niklas Beisert.
  This work may be distributed and/or modified under the
  conditions of the LaTeX Project Public License, either version 1.3
  of this license or (at your option) any later version.}}
\hypersetup{pdflicenseurl={http://www.latex-project.org/lppl.txt}}
\hypersetup{pdfcontactaddress={ETH Zurich, ITP, HIT K,
  Wolfgang-Pauli-Strasse 27}}
\hypersetup{pdfcontactpostcode={8093}}
\hypersetup{pdfcontactcity={Zurich}}
\hypersetup{pdfcontactcountry={Switzerland}}
\hypersetup{pdfcontactemail={nbeisert@itp.phys.ethz.ch}}
\hypersetup{pdfcontacturl={http://people.phys.ethz.ch/\xmptilde nbeisert/}}

\newcommand{\secref}[1]{\hyperref[#1]{section \ref*{#1}}}

\parskip1ex
\parindent0pt
\let\olditemize\itemize
\def\itemize{\olditemize\parskip0pt}

\begin{document}

\title{The \textsf{childdoc} Package}
\hypersetup{pdftitle={The childdoc Package}}
\author{Niklas Beisert\\[2ex]
  Institut f\"ur Theoretische Physik\\
  Eidgen\"ossische Technische Hochschule Z\"urich\\
  Wolfgang-Pauli-Strasse 27, 8093 Z\"urich, Switzerland\\[1ex]
  \href{mailto:nbeisert@itp.phys.ethz.ch}
  {\texttt{nbeisert@itp.phys.ethz.ch}}}
\hypersetup{pdfauthor={Niklas Beisert}}
\hypersetup{pdfsubject={Manual for the LaTeX2e Package childdoc}}
\date{30 December 2018, \textsf{v2.0}}
\maketitle

\begin{abstract}\noindent
\textsf{childdoc} is a \LaTeXe{} package
that enables the direct compilation
of document sections included by |\include|
to individual files.
\end{abstract}

\begingroup
\parskip0ex
\tableofcontents
\endgroup

%%%%%%%%%%%%%%%%%%%%%%%%%%%%%%%%%%%%%%%%%%%%%%%%%%%%%%%%%%%%%%%%%%%%%%%%%%%%%%%%
%%%%%%%%%%%%%%%%%%%%%%%%%%%%%%%%%%%%%%%%%%%%%%%%%%%%%%%%%%%%%%%%%%%%%%%%%%%%%%%%
\section{Introduction}

\LaTeX{} provides a mechanism to structure a large document (such as a book)
into a main file and several child files (containing the chapters)
using the |\include| command.
This mechanism is beneficial for documents
which span hundreds of pages in order to
make the source file(s) more manageable.
Moreover, compilation can be restricted to
selected child files by means of the |\includeonly| command.
The latter feature can be used to reduce the compilation time while editing
(this was significantly more useful in the earlier days of \LaTeX{})
or to generate a smaller document which is easier to navigate.
Another application of |\includeonly| is to generate
documents consisting of selected parts of the complete document.

However, there are a few drawbacks of the plain |\include| mechanism:
\begin{itemize}
\item
The child files cannot be compiled on their own,
they can only be compiled via the main file.
A naive editing environment
(such as a text editor with an option
to have the current file processed by \LaTeX)
may require one to switch to the main file before compiling;
attempting to compile the child file produces errors.
\item
The main file must be modified (each time)
to adjust the |\includeonly| command
to the present needs. This easily leaves the main file in a messy state.
\item
The generated document will always carry the filename
of the main document. This is inconvenient if
several child files are to be compiled and
to be kept for distribution.
\end{itemize}

The present package provides a simple interface
to make child files individually compilable by \LaTeX{}.
Compiling a child file then has the same effect as compiling
the main file with an |\includeonly| command
to select the appropriate child.
Moreover the generated document will carry the name of the child
rather than the main file.
This resolves all three above issues.

This feature is meant to make the editing of books,
thesis documents and lecture notes somewhat more convenient.
However, the package can also be used efficiently for
composing a series of documents (such as exercise sheets)
which are typically distributed individually.
It then assists the author in generating the individual documents
(potentially in different versions)
as well as a document containing the collected series.
Another application is in developing style files
or other kinds of included material
where compilation of the style file could redirect
to a sample or test file.

%%%%%%%%%%%%%%%%%%%%%%%%%%%%%%%%%%%%%%%%%%%%%%%%%%%%%%%%%%%%%%%%%%%%%%%%%%%%%%%%
%%%%%%%%%%%%%%%%%%%%%%%%%%%%%%%%%%%%%%%%%%%%%%%%%%%%%%%%%%%%%%%%%%%%%%%%%%%%%%%%
\section{Usage}

First of all, the package \textsf{childdoc} is \emph{not} a standard
\LaTeXe{} |.sty| style file! Therefore it needs to be invoked in
a non-standard way.

%%%%%%%%%%%%%%%%%%%%%%%%%%%%%%%%%%%%%%%%%%%%%%%%%%%%%%%%%%%%%%%%%%%%%%%%%%%%%%%%
\subsection{Included Files}
\label{sec:include}

%%%%%%%%%%%%%%%%%%%%%%%%%%%%%%%%%%%%%%%%
\DescribeMacro{\childdocmain}
To use the package, add the commands
\begin{center}
\begin{tabular}{l}
|\input{childdoc.def}|\\
|\childdocmain{}|\\
\end{tabular}
\end{center}
at the very top of the main \LaTeX{} file,
in particular \emph{before} the |\documentclass| statement!
The argument of |\childdocmain| should be left empty
(but it must be present).

%%%%%%%%%%%%%%%%%%%%%%%%%%%%%%%%%%%%%%%%
\DescribeMacro{\childdocof}
Furthermore, add the commands
\begin{center}
\begin{tabular}{l}
|\input{childdoc.def}|\\
|\childdocof{|\textit{main}|}|\\
\end{tabular}
\end{center}
at the top of every child file \textit{child}
which is included by |\include{|\textit{child}|}|
from within the main file
(or at least for those files to be compiled individually).
The argument \textit{main} must be the filename of the main file.

There are a couple of
considerations in setting up the main and child documents:

%%%%%%%%%%%%%%%%%%%%%%%%%%%%%%%%%%%%%%%%
\paragraph{Restrictions.}

Please note the following restrictions:
\begin{itemize}
\item
|\childdocmain| must be called with one argument \textit{main}
to ensure compatibility with earlier version of the package.
It must either be empty (|\childdocmain{}|)
or precisely match the filename of the main file in which it is specified.
See \secref{sec:detection} for further information.
\item
The filename \textit{main} must be specified without the |.tex| extension.
\item
The filename \textit{main} is case sensitive
(even in case-insensitive file systems)
due to internal string comparison.
\item
The argument \textit{main} should be fully expanded, it cannot be a macro.
\item
Subdirectories and special characters should be avoided in filenames.
\item
The command |\childdocmain{|\textit{main}|}| must be followed by a whitespace.
It should not be followed immediately by another command
or by a comment mark `|%|'.
This is because the \TeX{} parser reads the token immediately following
the argument of |\childdocmain| and puts it
at the beginning of every child section;
however, a white\-space is ignored.
\end{itemize}

%%%%%%%%%%%%%%%%%%%%%%%%%%%%%%%%%%%%%%%%
\paragraph{Content of Main File.}

It is advisable to place all content in the child files included by |\include|.
Any output contained in the main file will appear in all child documents
unless suppressed manually;
it cannot be suppressed automatically by the |\includeonly| directive
and thus should normally be avoided.
A method to include some content in the main file
by means of conditional processing is described in \secref{sec:conditional}.

%%%%%%%%%%%%%%%%%%%%%%%%%%%%%%%%%%%%%%%%
\paragraph{Page Numbering.}

When only a part of the document is compiled,
the appropriate numbering of pages
(as well as other status parameters)
is determined from the |.aux| files.
The latter contain information from previous passes.
However this information needs to propagate through
all intermediate child documents.
Therefore the page numbering in child documents may well
be inconsistent until the complete document is compiled at least once.

A useful (if unconventional) way to always ensure a consistent
page numbering is to restart the numbering in each child document
and denote the pages by `\textit{child}|.|\textit{page}'
where \textit{child} represents the chapter/section number of the child file.
This can be achieved by the command
|\numberwithin{page}{|\textit{child}|}|
of the \textsf{amsmath} package
where \textit{child} can be |chapter| or |section|
depending on the chosen structuring.
Alternatively, one can modify the macro |\thepage| appropriately
and reset the counter |page| at the start of each child file.

%%%%%%%%%%%%%%%%%%%%%%%%%%%%%%%%%%%%%%%%%%%%%%%%%%%%%%%%%%%%%%%%%%%%%%%%%%%%%%%%
\subsection{Conditional Processing}
\label{sec:conditional}

The package provides a mechanism to compile different versions
of a document. To customise the versions further some conditional processing
can come in handy to distinguish which version is being compiled.
The package provides two macros to describe the compilation context:

%%%%%%%%%%%%%%%%%%%%%%%%%%%%%%%%%%%%%%%%
\DescribeMacro{\ifchilddoc}
The conditional |\ifchilddoc| distinguishes between the compilation of
child documents and the main document:
%
\begin{center}
|\ifchilddoc |\textit{child-code}| |[|\||else |\textit{main-code}]| \||fi|
\end{center}

%%%%%%%%%%%%%%%%%%%%%%%%%%%%%%%%%%%%%%%%
\DescribeMacro{\childdocname}
\DescribeMacro{\childdocjob}
The macro |\childdocname| contains the filename (without extension)
of the main or child file being processed.
Note that |\childdocjob| will always contain the name of the main file.

%%%%%%%%%%%%%%%%%%%%%%%%%%%%%%%%%%%%%%%%
\paragraph{Title Page.}

Conditional processing can be used to include a title or banner page
in the main document when proper precautions are taken.
Importantly, the code in the main file should ensure that the page counter
(as well as other status parameters which are stored in the |.aux| files)
takes the same value after the conditional processing.
Otherwise the page numbers may take divergent values
depending on which part is compiled.

For example, a title page could be declared by:
%
\begin{center}
\begin{tabular}{l}
|\ifchilddoc\||else|\\
|\addtocounter{page}{-1}|\\
\textit{code for title page}\\
|\newpage|\\
|\||fi|
\end{tabular}
\end{center}
%
A banner page for the child documents can be generated by:
%
\begin{center}
\begin{tabular}{l}
|\ifchilddoc|\\
|\addtocounter{page}{-1}|\\
\textit{code for banner page}\\
|\newpage|\\
|\||fi|
\end{tabular}
\end{center}
%
Here one could write a message such as:
\begin{center}
|This is the part \childdocname{} of \childdocjob{}.|
\end{center}

%%%%%%%%%%%%%%%%%%%%%%%%%%%%%%%%%%%%%%%%%%%%%%%%%%%%%%%%%%%%%%%%%%%%%%%%%%%%%%%%
\subsection{Flags}
\label{sec:flags}

The package makes it easy to generate different versions
of the main or child documents.
To this end compilation flags can be defined
and assigned different default values.
They will be particularly useful in conjunction
with the forwarding mechanism described in \secref{sec:forward}.

For example, it may be useful to have a flag |\version|
which can be set to |draft| or |final|.
The document source will contain some conditional code
depending on the value of |\version|.
Suppose further, the flag should default to |final| for the main file
and to |draft| for child files
which is a natural assignment for editing the document.
This is achieved by placing the following code
in the preamble of the main document
(below the |\childdocmain| directive):
%
\begin{center}
\begin{tabular}{l}
|\ifchilddoc|\\
|\providecommand{\version}{draft}|\\
|\||else|\\
|\providecommand{\version}{final}|\\
|\||fi|
\end{tabular}
\end{center}
%
The definition by |\providecommand| makes sure
that previous definitions are not overwritten.
Further statements |\providecommand{\version}{...}|
can thus be added before the above code to override it.

For the main file, one might add a line
(between |\childdocmain| and the above block)
%
\begin{center}
|%\ifchilddoc\||else\providecommand{\version}{draft}\||fi|
\end{center}
%
which can be uncommented to produce a draft version.
Likewise one can add a line to the very top of a child file
(above the |\childdocof{|\textit{main}|}| directive)
%
\begin{center}
|%\providecommand{\version}{final}|
\end{center}
%
which can be uncommented to produce the final version of this child document.

%%%%%%%%%%%%%%%%%%%%%%%%%%%%%%%%%%%%%%%%%%%%%%%%%%%%%%%%%%%%%%%%%%%%%%%%%%%%%%%%
\subsection{Forwarding}
\label{sec:forward}

Different versions of the main or child documents
using compilation flags as described in \secref{sec:flags}
can be (permanently) stored in different files
for convenient compilation, viewing and distribution.
To this end, the package defines a command
to pass on compilation to a different file:

%%%%%%%%%%%%%%%%%%%%%%%%%%%%%%%%%%%%%%%%
\DescribeMacro{\childdocforward}
The command |\childdocforward| redirects processing to
another source file:
%
\begin{center}
\begin{tabular}{l}
|\input{childdoc.def}|\\
|\childdocforward[|\textit{main}|]{|\textit{dest}|}|\\
\end{tabular}
\end{center}
%
The argument \textit{dest} is the destination file
(without extension).
It should be the main file or one of the child files.
Note that further \textsf{childdoc} directives
such as |\childdocof| and |\childdocforward|
in the indicated file will be processed in this form.
The optional argument \textit{main}
passes on directly to the main file \textit{main}
while pretending to compile the child \textit{dest}.
This form behaves as if \textit{dest}
issues |\childdocof{|\textit{main}|}| right away,
and no further \textsf{childdoc} directives will be processed.

%%%%%%%%%%%%%%%%%%%%%%%%%%%%%%%%%%%%%%%%
\DescribeMacro{\...prefix}
In the alternative form |\childdocforwardprefix|,
%
\begin{center}
\begin{tabular}{l}
|\input{childdoc.def}|\\
|\childdocforwardprefix[|\textit{main}|]{|\textit{prefix}|}{|\textit{dest}|}|
\end{tabular}
\end{center}
%
the destination file is determined by a pattern
depending on the current file:
To make this work, the current file must be called
`{\textit{prefix}\hspace{0.2em}\textit{suffix}}'
with \textit{prefix} matching precisely the argument.
Processing is then passed on to the file
`{\textit{dest}\hspace{0.2em}\textit{suffix}}'.
Surely, the same effect is achieved by
directly specifying the
argument `{\textit{dest}\hspace{0.2em}\textit{suffix}}'
in the first form.
However, that requires to set up a different file
for each child. With the alternative form of the command
all these files can have exactly the same content
which simplifies setting them up and maintaining them.

For example, the following file |draft.tex|
with a compilation flag |\version| as described in \secref{sec:flags}
compiles the main document as a draft:
%
\begin{center}
\begin{tabular}{l}
|\def\version{draft}|\\
|\input{childdoc.def}|\\
|\childdocforward{|\textit{main}|}|
\end{tabular}
\end{center}
%
Likewise, the following files |final|\textit{nn}|.tex|
compile the final version of the child document
|child|\textit{nn}|.tex|:
%
\begin{center}
\begin{tabular}{l}
|\def\version{final}|\\
|\input{childdoc.def}|\\
|\childdocforwardprefix{final}{child}|
\end{tabular}
\end{center}
%

Note that when several versions of a main file and/or of each child file
are to be generated, it may be convenient to set up a |Makefile| or
shell script to automatise the process.

%%%%%%%%%%%%%%%%%%%%%%%%%%%%%%%%%%%%%%%%%%%%%%%%%%%%%%%%%%%%%%%%%%%%%%%%%%%%%%%%
\subsection{Command Line Processing}
\label{sec:commandline}

The effect of redirection files can also be achieved by invoking
the \LaTeX{} compiler with a more elaborate command line.
Most conveniently this should be done as part
of a shell script or a |Makefile|.

When using \textsf{childdoc} in the main file, the following
command lines effectively perform a redirection
(note that depending on the shell being used,
backslashes may have to be doubled: `|\|' $\to$ `|\\|'):
%
\begin{center}
|... -jobname "|\textit{target}|" |\\|"|[\textit{flags}]%
|\input{childdoc.def}\childdocforward[|\textit{main}|]{|\textit{dest}|}"|
\end{center}
%
Here \textit{target} is the name of the output file,
\textit{main} is the name of the main file
and \textit{dest} is the name of the main or child file to be processed
(all filenames without extensions).
The optional argument \textit{main} can be omitted
if \textit{main} matches \textit{dest}.
Optionally, compilation \textit{flags} can be defined via |\def| commands.
This command line makes the \TeX{} engine believe
it is compiling the file \textit{target}
whose content is specified as the latter parameter.
The provided code then forwards the processing to
\textit{main} or \textit{dest} as described in \secref{sec:forward}.

%%%%%%%%%%%%%%%%%%%%%%%%%%%%%%%%%%%%%%%%%%%%%%%%%%%%%%%%%%%%%%%%%%%%%%%%%%%%%%%%
\subsection{Include by Input}
\label{sec:input}

Including child documents by |\include| has some restrictions by design.
Most notably, the content of a child document always occupies
its own set of pages; pages cannot be shared between child documents.
Usually, this behaviour makes perfect sense
because each child document contain an essential part of the document.
However, in some situations it may be desirable to compose
a document from a collection of parts
without having mandatory page breaks between then.
For this case, the package
provides a mechanism to include parts
by |\input| which can also be processed individually.
However, by construction this mechanism
requires manual handling of the content to be output.

%%%%%%%%%%%%%%%%%%%%%%%%%%%%%%%%%%%%%%%%
\DescribeMacro{\ifchilddocmanual}
The main file should be prepared as usual, see \secref{sec:include}.
However, the document body must make a distinction
between processing of an individual part and of the main document, e.g.:
%
\begin{center}
\begin{tabular}{l}
|\ifchilddocmanual|\\
|\input{\childdocname}|\\
|\||else|\\
\textit{document body with }|\input{|\textit{part}|}|\\
|\||fi|
\end{tabular}
\end{center}
%
The conditional |\ifchilddocmanual| is true whenever
a part to be included by |\input| is being compiled,
and the name of the part is stored in |\childdocname|.

%%%%%%%%%%%%%%%%%%%%%%%%%%%%%%%%%%%%%%%%
\DescribeMacro{\childdocby}
Each part to be included by |\input| should start with:
%
\begin{center}
\begin{tabular}{l}
|\input{childdoc.def}|\\
|\childdocby{|\textit{main}|}|\\
\end{tabular}
\end{center}
%
The directive |\childdocby| is similar to |\childdocof|
described in \secref{sec:include},
but the subsequent selection of content must be done manually.
To that end, both |\ifchilddoc| and |\ifchilddocmanual|
will be true upon processing of a part,
and the name of the part is stored in |\childdocname|.
Note that |\jobname| will be set to the filename of the current part
so that each part receives an individual |.aux| file
that does not interfere with the |.aux| file(s) of the main document.
This behaviour can be altered by the alternative form
|\childdocby[*]{|\textit{main}|}| (with a non-empty optional argument)
which uses the |.aux| file of the main document
by setting |\jobname| to \textit{main}.

%%%%%%%%%%%%%%%%%%%%%%%%%%%%%%%%%%%%%%%%%%%%%%%%%%%%%%%%%%%%%%%%%%%%%%%%%%%%%%%%
\subsection{Driver Development}
\label{sec:driver}

The \textsf{childdoc} mechanism can also be use for the development
of definition files such as \LaTeX{} styles or classes.
This case differs from the above setup with multiple parts
included by |\include| in that no |\includeonly| should be invoked.
This can be achieved by starting the include file
(before |\ProvidesPackage|) with:
%
\begin{center}
\begin{tabular}{l}
|\input{childdoc.def}|\\
|\childdocforward{|\textit{main}|}|\\
\end{tabular}
\end{center}
%
or alternatively with:
%
\begin{center}
\begin{tabular}{l}
|\input{childdoc.def}|\\
|\childdocby{|\textit{main}|}|\\
\end{tabular}
\end{center}
%
Both forms have slightly different effects as described above.
The main file is prepared as usual, see \secref{sec:include}.

%%%%%%%%%%%%%%%%%%%%%%%%%%%%%%%%%%%%%%%%%%%%%%%%%%%%%%%%%%%%%%%%%%%%%%%%%%%%%%%%
\subsection{Legacy Detection}
\label{sec:detection}

The directive |\childdocmain| in the main file can detect
whether the complete document or merely a child is to be compiled
even without using the directive |\childdocof|.
This method is deprecated because it is less robust
and there is no compelling reason to use it;
it is merely provided for backward compatibility
and it may be removed in future versions.

If the detection mechanism is to be used,
it is mandatory to correctly specify
the filename of the main file as the argument of |\childdocmain|:
%
\begin{center}
\begin{tabular}{l}
|\input{childdoc.def}|\\
|\childdocmain{|\textit{main}|}|\\
\end{tabular}
\end{center}
%
If |\jobname| does not match the argument \textit{main} of |\childdocmain|,
it is assumed that |\jobname| points to the child file to be compiled.
When using |\childdocmain| with the main file specified as argument,
it suffices to start a child file
with just |\input{|\textit{main}|}|
without loading of the package and using |\childdocof|.
If instead all processing is done
with the appropriate \textsf{childdoc} directives,
the argument of \textit{main} of |\childdocmain| can be empty.

An alternative version of the command line processing described
in \secref{sec:commandline} using the detection mechanism reads:
%
\begin{center}
|... -jobname "|\textit{target}|" "|[\textit{flags}]%
[|\def\jobname{|\textit{dest}|}|]|\input{|\textit{main}|}"|
\end{center}

%%%%%%%%%%%%%%%%%%%%%%%%%%%%%%%%%%%%%%%%%%%%%%%%%%%%%%%%%%%%%%%%%%%%%%%%%%%%%%%%
\subsection{Manual Code}
\label{sec:manual}

In case one cannot be certain whether the definitions file |childdoc.def|
is installed on the target \TeX{} distribution
and one prefers not to ship it,
it is conceivable to paste a few relevant commands into the sources.

To that end, drop all statements |\input{childdoc.def}|
and perform the replacements as outlined below.
Instead of |\childdocmain{|\textit{main}|}| add the following code
to the top of the main file:
%
\begin{center}
\begin{tabular}{l}
|\||ifdefined\childdocname\endinput\||fi\newif\ifchilddoc|\\
|\edef\childdocname{\scantokens\expandafter{\jobname\noexpand}}|\\
|\def\childdocmain{|\textit{main}|}\||ifx\childdocmain\childdocname\||else|\\
|\childdoctrue\includeonly{\childdocname}\let\jobname\childdocmain\||fi|\\
\end{tabular}
\end{center}
%
Instead of |\childdocof{|\textit{main}|}| just include the main file
at the top of each child file:
%
\begin{center}
|\input{|\textit{main}|}|
\end{center}
%
A simple redirection |\childdocforward{|\textit{dest}|}| is achieved by:
%
\begin{center}
|\def\jobname{|\textit{dest}|}\input{\jobname}|
\end{center}
%
The redirection with prefix
|\childdocforwardprefix[|\textit{prefix}|]{|\textit{dest}|}|
is accomplished by:
%
\begin{center}
\begin{tabular}{l}
|{\edef\jobname{\scantokens\expandafter{\jobname\noexpand}}|\\
|\def\redirectjob |\textit{prefix}|#1~~~{\gdef\jobname{|\textit{dest}|#1}}|\\
|\expandafter\redirectjob\jobname~~~}\input{\jobname}|
\end{tabular}
\end{center}

In an alternative approach,
child documents can be compiled by a specific command line
without additional code or specific definitions:
%
\begin{center}
|... -jobname "|\textit{target}|" "|[\textit{flags}]%
|\includeonly{|\textit{dest}|}\input{|\textit{main}|}"|
\end{center}
%

%%%%%%%%%%%%%%%%%%%%%%%%%%%%%%%%%%%%%%%%%%%%%%%%%%%%%%%%%%%%%%%%%%%%%%%%%%%%%%%%
%%%%%%%%%%%%%%%%%%%%%%%%%%%%%%%%%%%%%%%%%%%%%%%%%%%%%%%%%%%%%%%%%%%%%%%%%%%%%%%%
\section{Information}

%%%%%%%%%%%%%%%%%%%%%%%%%%%%%%%%%%%%%%%%%%%%%%%%%%%%%%%%%%%%%%%%%%%%%%%%%%%%%%%%
\subsection{Copyright}

Copyright \copyright{} 2017--2018 Niklas Beisert

This work may be distributed and/or modified under the
conditions of the \LaTeX{} Project Public License, either version 1.3
of this license or (at your option) any later version.
The latest version of this license is in
  \url{http://www.latex-project.org/lppl.txt}
and version 1.3 or later is part of all distributions of \LaTeX{}
version 2005/12/01 or later.

This work has the LPPL maintenance status `maintained'.

The Current Maintainer of this work is Niklas Beisert.

This work consists of the files |README.txt|, |childdoc.ins| and |childdoc.dtx|
as well as the derived files |childdoc.def|, |cdocsamp.tex|
with |cdocsch1.tex|, |cdocsch2.tex|, |cdocspt3.tex|, |cdocspt4.tex|,
|cdocsdrf.tex|, |cdocsfn1.tex|, |cdocsfn2.tex|
as well as |childdoc.pdf|.

%%%%%%%%%%%%%%%%%%%%%%%%%%%%%%%%%%%%%%%%%%%%%%%%%%%%%%%%%%%%%%%%%%%%%%%%%%%%%%%%
\subsection{Files and Installation}

The package consists of the files:
%
\begin{center}
\begin{tabular}{ll}
    |README.txt|   & readme file \\
    |childdoc.ins| & installation file \\
    |childdoc.dtx| & source file \\
    |childdoc.def| & definition file \\
    |cdocsamp.tex| & sample main file \\
    |cdocsch1.tex| & sample include file \\
    |cdocsch2.tex| & sample include file \\
    |cdocspt3.tex| & sample part file \\
    |cdocspt4.tex| & sample part file \\
    |cdocsdrf.tex| & sample redirection file \\
    |cdocsfn1.tex| & sample redirection file \\
    |cdocsfn2.tex| & sample redirection file \\
    |childdoc.pdf| & manual
\end{tabular}
\end{center}
%
The distribution consists of the files
|README.txt|, |childdoc.ins| and |childdoc.dtx|.
%
\begin{itemize}
\item
Run (pdf)\LaTeX{} on |childdoc.dtx|
to compile the manual |childdoc.pdf| (this file).
\item
Run \LaTeX{} on |childdoc.ins| to create the definitions file |childdoc.def|
and the sample |cdocsamp.tex| with include files
|cdocsch1.tex|, |cdocsch2.tex|, |cdocspt3.tex|, |cdocspt4.tex|,
|cdocsdrf.tex|, |cdocsfn1.tex|, |cdocsfn2.tex|.
Then copy the file |childdoc.def| to an appropriate directory of your \LaTeX{}
distribution, e.g.\ \textit{texmf-root}|/tex/latex/childdoc|.
\end{itemize}

%%%%%%%%%%%%%%%%%%%%%%%%%%%%%%%%%%%%%%%%%%%%%%%%%%%%%%%%%%%%%%%%%%%%%%%%%%%%%%%%
\subsection{Related CTAN Packages}

There are several other packages which offer a similar functionality:
%
\begin{itemize}
\item
The packages
\href{http://ctan.org/pkg/docmute}{\textsf{docmute}},
\href{http://ctan.org/pkg/includex}{\textsf{includex}} and
\href{http://ctan.org/pkg/standalone}{\textsf{standalone}}
provide commands to include only the document body of
a child file thus allowing both files to be compiled individually.
\item
The packages \href{http://ctan.org/pkg/subdocs}{\textsf{subdocs}}
and \href{http://ctan.org/pkg/subfiles}{\textsf{subfiles}}
provide structures in which the main and child documents can be
encapsulated and allowing them to be compiled individually.
The inclusion mechanism is different from the conventional |\include|.
\item
The package \href{http://ctan.org/pkg/combine}{\textsf{combine}}
is an elaborate solution to combine several documents into one.
\end{itemize}
%
See also the CTAN topic \href{http://ctan.org/topic/subdocs}{\textsf{subdocs}}
for further related packages.
The present package differs from the above solutions in that
a document structure constructed with the conventional |\include| mechanism
just needs two extra commands at the top of every file
such that all constituent files can be compiled individually.

%%%%%%%%%%%%%%%%%%%%%%%%%%%%%%%%%%%%%%%%%%%%%%%%%%%%%%%%%%%%%%%%%%%%%%%%%%%%%%%%
%\subsection{Feature Suggestions}
%
%The following is a list of features which may be useful for future
%versions of this package:
%%
%\begin{itemize}
%\item
%\ldots
%\end{itemize}

%%%%%%%%%%%%%%%%%%%%%%%%%%%%%%%%%%%%%%%%%%%%%%%%%%%%%%%%%%%%%%%%%%%%%%%%%%%%%%%%
\subsection{Revision History}

%%%%%%%%%%%%%%%%%%%%%%%%%%%%%%%%%%%%%%%%
\paragraph{v2.0:} 2018/12/30

\begin{itemize}
\item
immediate forward processing
\item
added |\childdocby| mechanism
\item
manual restructured
\end{itemize}

%%%%%%%%%%%%%%%%%%%%%%%%%%%%%%%%%%%%%%%%
\paragraph{v1.6:} 2018/01/17

\begin{itemize}
\item
application for development of include files
\item
corrections to manual
\end{itemize}

%%%%%%%%%%%%%%%%%%%%%%%%%%%%%%%%%%%%%%%%
\paragraph{v1.5:} 2017/05/21

\begin{itemize}
\item
more complete structuring introduced
\item
|\childdocof| introduced
\item
|\childdoc| renamed to |\childdocmain|
\item
|\childredirect| renamed to |\childdocforward| and |\childdocforwardprefix|
and functionality expanded
\end{itemize}

%%%%%%%%%%%%%%%%%%%%%%%%%%%%%%%%%%%%%%%%
\paragraph{v1.0:} 2017/04/27

\begin{itemize}
\item
manual and install package
\item
first version published on CTAN
\end{itemize}

%%%%%%%%%%%%%%%%%%%%%%%%%%%%%%%%%%%%%%%%
\paragraph{v0.6:} 2017/04/26

\begin{itemize}
\item
redirection mechanism added
\end{itemize}

%%%%%%%%%%%%%%%%%%%%%%%%%%%%%%%%%%%%%%%%
\paragraph{v0.5:} 2017/04/26

\begin{itemize}
\item
functionality in definition file
\end{itemize}


%%%%%%%%%%%%%%%%%%%%%%%%%%%%%%%%%%%%%%%%%%%%%%%%%%%%%%%%%%%%%%%%%%%%%%%%%%%%%%%%
%%%%%%%%%%%%%%%%%%%%%%%%%%%%%%%%%%%%%%%%%%%%%%%%%%%%%%%%%%%%%%%%%%%%%%%%%%%%%%%%
%%%%%%%%%%%%%%%%%%%%%%%%%%%%%%%%%%%%%%%%%%%%%%%%%%%%%%%%%%%%%%%%%%%%%%%%%%%%%%%%
\appendix

\settowidth\MacroIndent{\rmfamily\scriptsize 000\ }

 \DocInput{childdoc.dtx}

\end{document}
%</driver>
% \fi
%
% %%%%%%%%%%%%%%%%%%%%%%%%%%%%%%%%%%%%%%%%%%%%%%%%%%%%%%%%%%%%%%%%%%%%%%%%%%%%%%
% %%%%%%%%%%%%%%%%%%%%%%%%%%%%%%%%%%%%%%%%%%%%%%%%%%%%%%%%%%%%%%%%%%%%%%%%%%%%%%
% \section{Sample}
%\iffalse
%<*samplemain>
%\fi
%
% The following presents a sample document
% with two chapters, two parts, a title page,
% a compile flag as well as three forwarding files to set the flag.
% It consists of eight |.tex| files:
% \begin{center}
% \begin{tabular}{ll}
% |cdocsamp.tex|&main file\\
% |cdocsch1.tex|&include file for chapter 1\\
% |cdocsch2.tex|&include file for chapter 2\\
% |cdocspt3.tex|&include file for part 3\\
% |cdocspt4.tex|&include file for part 4\\
% |cdocsdrf.tex|&forwarding file for main file in draft mode\\
% |cdocsfi1.tex|&forwarding file for final version of chapter 1\\
% |cdocsfi2.tex|&forwarding file for final version of chapter 2\\
% \end{tabular}
% \end{center}
% Each of the eight files can be compiled directly by the \LaTeX{} compiler.
%
% %%%%%%%%%%%%%%%%%%%%%%%%%%%%%%%%%%%%%%
% \paragraph{Main File.}
%
% The main file is called |cdocsamp.tex|.
%
% Load the \textsf{childdoc} definitions and
% declare the filename for the main document:
%    \begin{macrocode}
\input{childdoc.def}
\childdocmain{}
%    \end{macrocode}

% Optional override for |\version| flag:
%    \begin{macrocode}
%%\ifchilddoc\else\providecommand{\version}{draft}\fi
%    \end{macrocode}

% Define the default values for the |\version| flag
% (|final| for the main file and |draft| for childs):
%    \begin{macrocode}
\ifchilddoc
\providecommand{\version}{draft}
\else
\providecommand{\version}{final}
\fi
%    \end{macrocode}

% Load the standard document class:
%    \begin{macrocode}
\documentclass[12pt]{article}
%    \end{macrocode}

% Start the document body:
%    \begin{macrocode}
\begin{document}
%    \end{macrocode}

% Declare a title page.
% Print title, part of document being processed and version flag:
%    \begin{macrocode}
\addtocounter{page}{-1}
\begin{center}
{\LARGE\bfseries{}childdoc example\par}
\vspace{1cm}
\ifchilddoc
\ifchilddocmanual part\else chapter\fi:
`\childdocname' of `\childdocjob'\par
\else
main document: `\childdocjob'\par
\fi
version: \version\par
\end{center}
\newpage
%    \end{macrocode}

% Manually include selected file,
% otherwise process as usual:
%    \begin{macrocode}
\ifchilddocmanual
\section*{part `\childdocname'}
\input{\childdocname}
\else
%    \end{macrocode}

% Include the two chapters:
%    \begin{macrocode}
\include{cdocsch1}
\include{cdocsch2}
%    \end{macrocode}

% Include the two parts unless only chapters should be displayed:
%    \begin{macrocode}
\ifchilddoc\else
\section{part three}
\input{cdocspt3}
\section{part four}
\input{cdocspt4}
\fi
%    \end{macrocode}

% Process as usual until here:
%    \begin{macrocode}
\fi
%    \end{macrocode}

% End of document body:
%    \begin{macrocode}
\end{document}
%    \end{macrocode}
%\iffalse
%</samplemain>
%\fi
%
% %%%%%%%%%%%%%%%%%%%%%%%%%%%%%%%%%%%%%%
% \paragraph{Chapter Include Files.}
%
% The include files are called |cdocsch1.tex| and |cdocsch2.tex|.
%
%\iffalse
%<*samplechap1|samplechap2>
%\fi

% Optional override for |\version| flag:
%    \begin{macrocode}
%%\providecommand{\version}{final}
%    \end{macrocode}

% Include the main document:
%    \begin{macrocode}
\input{childdoc.def}
\childdocof{cdocsamp}
%    \end{macrocode}

%\iffalse
%</samplechap1|samplechap2>
%\fi
%
%\iffalse
%<*samplechap1>
%\fi
% Some text for chapter 1:
%    \begin{macrocode}
\section{one}
some text in chapter one
%    \end{macrocode}

%\iffalse
%</samplechap1>
%\fi
% Some text for chapter 2:
%\iffalse
%<*samplechap2>
%\fi
%    \begin{macrocode}
\section{two}
more text in chapter two
%    \end{macrocode}

%\iffalse
%</samplechap2>
%\fi
%
% %%%%%%%%%%%%%%%%%%%%%%%%%%%%%%%%%%%%%%
% \paragraph{Part Include Files.}
%
% The include files are called |cdocspt3.tex| and |cdocspt4.tex|.
%
%\iffalse
%<*samplepart3|samplepart4>
%\fi

% Optional override for |\version| flag:
%    \begin{macrocode}
%%\providecommand{\version}{final}
%    \end{macrocode}

% Include the main document:
%    \begin{macrocode}
\input{childdoc.def}
\childdocby{cdocsamp}
%    \end{macrocode}

%\iffalse
%</samplepart3|samplepart4>
%\fi
%
%\iffalse
%<*samplepart3>
%\fi
% Some text for part 3:
%    \begin{macrocode}
some text in part three
%    \end{macrocode}

%\iffalse
%</samplepart3>
%\fi
% Some text for part 4:
%\iffalse
%<*samplepart4>
%\fi
%    \begin{macrocode}
more text in part four
%    \end{macrocode}

%\iffalse
%</samplepart4>
%\fi
%
% %%%%%%%%%%%%%%%%%%%%%%%%%%%%%%%%%%%%%%
% \paragraph{Forwarding for a Complete Draft.}
%
% The following forwarding file |cdocsdrf.tex|
% compiles the main document in draft mode:
%\iffalse
%<*sampledraft>
%\fi
%    \begin{macrocode}
\def\version{draft}
\input{childdoc.def}
\childdocforward{cdocsamp}
%    \end{macrocode}

%\iffalse
%</sampledraft>
%\fi
%
% %%%%%%%%%%%%%%%%%%%%%%%%%%%%%%%%%%%%%%
% \paragraph{Forwarding for Final Version of the Chapters.}
%
% The following forwarding files |cdocsfn1.tex| and |cdocsfn2.tex|
% (with identical content)
% compile the final versions of the child documents
% |cdocsch1.tex| and |cdocsch2.tex|, respectively:
%\iffalse
%<*samplefinal>
%\fi
%    \begin{macrocode}
\def\version{final}
\input{childdoc.def}
\childdocforwardprefix[cdocsamp]{cdocsfn}{cdocsch}
%    \end{macrocode}

%\iffalse
%</samplefinal>
%\fi
%
% %%%%%%%%%%%%%%%%%%%%%%%%%%%%%%%%%%%%%%
% \paragraph{Command Line Processing.}
%
% The following three command lines generate the output files
% |cdocscld|, |cdocscl1| and |cdocscl2|
% which should be identical to
% |cdocsdrf|, |cdocsch1| and |cdocsfn2|, respectively:
% \begin{center}
% \begin{tabular}{l}
% |latex -jobname cdocscld \|\\
% |  "\def\version{draft}\input{childdoc.def}\childdocforward{cdocsamp}"|\\
% |latex -jobname cdocscl1 \|\\
% |  "\input{childdoc.def}\childdocforward[cdocsamp]{cdocsch1}"|\\
% |latex -jobname cdocscl2 \|\\
% |  "\def\version{final}\input{childdoc.def}\childdocforward{cdocsch2}"|
% \end{tabular}
% \end{center}
% Note that the trailing backslash on each first line
% merely continues the input to the second line
% (for convenient cut ant paste).
% Furthermore, the command |latex| can be replaced by any
% of its alternative versions such as |pdflatex|.
%
% %%%%%%%%%%%%%%%%%%%%%%%%%%%%%%%%%%%%%%%%%%%%%%%%%%%%%%%%%%%%%%%%%%%%%%%%%%%%%%
% %%%%%%%%%%%%%%%%%%%%%%%%%%%%%%%%%%%%%%%%%%%%%%%%%%%%%%%%%%%%%%%%%%%%%%%%%%%%%%
% \section{Implementation}
%\iffalse
%<*package>
%\fi
%
% This section describes the definitions file |childdoc.def|.

% The definitions cannot be loaded using |\usepackage| or |\RequirePackage|
% which has a mechanism to prevent loading a style file more than once.
% When loading the definitions by means of |\input|
% multiple instances have to be prevented manually:
%\iffalse
%This code needs to be before the `\ProvidesFile' directive
%which is defined at the beginning of this file.
%Therefore it is also placed there and commented out here.
%</package>
%<*discard>
%\fi
%    \begin{macrocode}
\ifdefined\childdocmain\endinput\fi
%    \end{macrocode}
%\iffalse
%</discard>
%<*package>
%\fi
%
% \macro{\ifchilddoc}
% \macro{\ifchilddocmanual}
% The conditional |\ifchilddoc| tells whether a
% child (true) or main (false) document is being compiled.
% The conditional |\ifchilddocmanual| tells whether
% the |\includeonly| mechanism is used (false) or
% the selection of child files must be performed manually (true).
% The definitions initialise to false:
%    \begin{macrocode}
\newif\ifchilddoc
\newif\ifchilddocmanual
%    \end{macrocode}

% \macro{\childdocname}
% \macro{\childdocjob}
% The macro |\childdocname| stores the name of the main document
% to be compiled. The macro |\childdocjob| stores the name of
% the document on which the \LaTeX{} compiler was originally invoked.
% The content of |\jobname| cannot be compared
% to filenames specified in the source due to different catcodes.
% The following code rescans |\jobname|, stores the result
% in |\childdocname| and saves a copy in |\childdocjob|:
%    \begin{macrocode}
\edef\childdocname{\scantokens\expandafter{\jobname\noexpand}}
\let\childdocjob\childdocname
%    \end{macrocode}

% \macro{\childdocdisable}
% The macro |\childdocdisable| prevents the main file
% from being processed more than once.
% At this stage, the main document command |\childdocmain|
% is assumed to be called once again where it should do nothing.
% Any subsequent call to it should prevent
% a secondary processing of the main document
% It overwrites the forwarding commands
% |\childdocof| and |\childdocforward|
% with empty macros to prevent further inclusions of the main document:
%    \begin{macrocode}
\newcommand{\childdocdisable}
{
  \renewcommand{\childdocmain}[1]{\renewcommand{\childdocmain}[1]{\endinput}}
  \renewcommand{\childdocof}[1]{}
  \renewcommand{\childdocby}[2][]{}
  \renewcommand{\childdocforward}[2][]{}
  \renewcommand{\childdocdisable}{}
}
%    \end{macrocode}

% \macro{\childdocmain}
% The macro |\childdocmain| is to be called at the top of the main file
% with nothing or the main filename (without extension) as argument.
% First, it breaks loops.
% If the argument is not empty and does not match |\childdocname|
% (which is set by the first inclusion of |childdoc.def|),
% |\ifchilddoc| is set to true, |\includeonly| is applied to the child file
% and |\jobname| is set to the main file
% (for proper handling of |.aux| files):
%    \begin{macrocode}
\newcommand{\childdocmain}[1]
{
  \childdocdisable\childdocmain{}
  \if?#1?\else
    \begingroup
      \def\childdoctmp{#1}
      \ifx\childdoctmp\childdocname
        \def\childdoctmp{}
      \else
        \def\childdoctmp
        {
          \childdoctrue
          \includeonly{\childdocname}
          \def\childdocjob{#1}
          \def\jobname{#1}
        }
      \fi
      \expandafter
    \endgroup
    \childdoctmp
  \fi
}
%    \end{macrocode}

% \macro{\childdocof}
% The command |\childdocof| redirects
% compilation to the main file |#1|.
%    \begin{macrocode}
\newcommand{\childdocof}[1]
{
  \childdocdisable
  \childdoctrue
  \includeonly{\childdocname}
  \def\jobname{#1}
  \def\childdocjob{#1}
  \input{#1}
}
%    \end{macrocode}

% \macro{\childdocby}
% The command |\childdocby| ....
%    \begin{macrocode}
\newcommand{\childdocby}[2][]
{
  \childdocdisable
  \childdoctrue
  \childdocmanualtrue
  \if?#1?\else
    \def\jobname{#2}
  \fi
  \def\childdocjob{#2}
  \input{#2}
  \endinput
}
%    \end{macrocode}

% \macro{\childdocforward}
% The command |\childdocforward| redirects
% compilation to the main file or
% (if the optional argument is given) a child file.
% Parameters are set as if the main file
% or a child file starting with |\childdocof| was compiled.
% Then compilation is handed over to the main file:
%    \begin{macrocode}
\newcommand{\childdocforward}[2][]
{
  \begingroup
    \if?#1?
      \def\childdoctmp
      {
        \def\childdocname{#2}
        \def\childdocjob{#2}
        \def\jobname{#2}
        \input{#2}
        \endinput
      }
    \else
      \def\childdoctmp
      {
        \childdocdisable
        \def\childdocname{#2}
        \childdoctrue
        \includeonly{#2}
        \def\childdocjob{#1}
        \def\jobname{#1}
        \input{#1}
        \endinput
      }
    \fi
    \expandafter
  \endgroup
  \childdoctmp
}
%    \end{macrocode}

% \macro{\childdocforwardprefix}
% The command |\childdocforwardprefix| redirects
% compilation to the main or a child file by means of a pattern.
% The prefix |#1| in the current filename is replaced by |#2|
% and the suffix of the current filename is kept
% (it is assumed that the filename does not contain the substring `|~~~|'
% which is used as a delimiter).
% Compilation is handed over to the new file by |\childdocforward|:
%    \begin{macrocode}
\newcommand{\childdocforwardprefix}[3][]
{
  \begingroup
    \def\childdocextract #2##1~~~{\def\childdoctmp{\childdocforward[#1]{#3##1}}}
    \expandafter\childdocextract\childdocname~~~
    \expandafter
  \endgroup
  \childdoctmp
}
%    \end{macrocode}

% \macro{\childdoc}
% The deprecated macro |\childdoc| is a legacy version of |\childdocmain|:
%    \begin{macrocode}
\newcommand{\childdoc}{\childdocmain}
%    \end{macrocode}

% \macro{\childdocredirect}
% The deprecated macro |\childdocredirect| is a legacy version
% of |\childdocforward| and |\childdocforwardprefix|:
%    \begin{macrocode}
\newcommand{\childdocredirect}[2][]
{
  \begingroup
    \if?#1?
      \def\childdoctmp{\childdocforward{#2}}
    \else
      \def\childdoctmp{\childdocforwardprefix{#1}{#2}}
    \fi
    \expandafter
  \endgroup
  \childdoctmp
}
%    \end{macrocode}

%\iffalse
%</package>
%\fi
%
\endinput

\childdocby{cdocsamp}
%    \end{macrocode}

%\iffalse
%</samplepart3|samplepart4>
%\fi
%
%\iffalse
%<*samplepart3>
%\fi
% Some text for part 3:
%    \begin{macrocode}
some text in part three
%    \end{macrocode}

%\iffalse
%</samplepart3>
%\fi
% Some text for part 4:
%\iffalse
%<*samplepart4>
%\fi
%    \begin{macrocode}
more text in part four
%    \end{macrocode}

%\iffalse
%</samplepart4>
%\fi
%
% %%%%%%%%%%%%%%%%%%%%%%%%%%%%%%%%%%%%%%
% \paragraph{Forwarding for a Complete Draft.}
%
% The following forwarding file |cdocsdrf.tex|
% compiles the main document in draft mode:
%\iffalse
%<*sampledraft>
%\fi
%    \begin{macrocode}
\def\version{draft}
% \iffalse
%
% childdoc.dtx Copyright (C) 2017-2018 Niklas Beisert
%
% This work may be distributed and/or modified under the
% conditions of the LaTeX Project Public License, either version 1.3
% of this license or (at your option) any later version.
% The latest version of this license is in
%   http://www.latex-project.org/lppl.txt
% and version 1.3 or later is part of all distributions of LaTeX
% version 2005/12/01 or later.
%
% This work has the LPPL maintenance status `maintained'.
%
% The Current Maintainer of this work is Niklas Beisert.
%
% This work consists of the files childdoc.dtx and childdoc.ins
% and the derived files childdoc.def and cdocsamp.tex with
% cdocsch1.tex, cdocsch2.tex, cdocsdrf.tex, cdocsfn1.tex, cdocsfn2.tex.
%
%<package>\ifdefined\childdocmain\endinput\fi
%<package>\ProvidesFile{childdoc.def}[2018/12/30 v2.0 child document driver]
%<samplemain>\ProvidesFile{cdocsamp.tex}[2018/12/30 v2.0 sample for childdoc]
%<*driver>
%\ProvidesFile{childdoc.drv}[2018/12/30 v2.0 childdoc reference manual file]
\PassOptionsToClass{10pt,a4paper}{article}
\documentclass{ltxdoc}

\usepackage[margin=35mm]{geometry}
\usepackage{hyperref}
\usepackage{hyperxmp}
\usepackage[usenames]{color}

\hypersetup{colorlinks=true}
\hypersetup{pdfstartview=FitH}
\hypersetup{pdfpagemode=UseNone}
\hypersetup{pdfsource={}}
\hypersetup{pdflang={en-UK}}
\hypersetup{pdfcopyright={Copyright 2017-2018 Niklas Beisert.
  This work may be distributed and/or modified under the
  conditions of the LaTeX Project Public License, either version 1.3
  of this license or (at your option) any later version.}}
\hypersetup{pdflicenseurl={http://www.latex-project.org/lppl.txt}}
\hypersetup{pdfcontactaddress={ETH Zurich, ITP, HIT K,
  Wolfgang-Pauli-Strasse 27}}
\hypersetup{pdfcontactpostcode={8093}}
\hypersetup{pdfcontactcity={Zurich}}
\hypersetup{pdfcontactcountry={Switzerland}}
\hypersetup{pdfcontactemail={nbeisert@itp.phys.ethz.ch}}
\hypersetup{pdfcontacturl={http://people.phys.ethz.ch/\xmptilde nbeisert/}}

\newcommand{\secref}[1]{\hyperref[#1]{section \ref*{#1}}}

\parskip1ex
\parindent0pt
\let\olditemize\itemize
\def\itemize{\olditemize\parskip0pt}

\begin{document}

\title{The \textsf{childdoc} Package}
\hypersetup{pdftitle={The childdoc Package}}
\author{Niklas Beisert\\[2ex]
  Institut f\"ur Theoretische Physik\\
  Eidgen\"ossische Technische Hochschule Z\"urich\\
  Wolfgang-Pauli-Strasse 27, 8093 Z\"urich, Switzerland\\[1ex]
  \href{mailto:nbeisert@itp.phys.ethz.ch}
  {\texttt{nbeisert@itp.phys.ethz.ch}}}
\hypersetup{pdfauthor={Niklas Beisert}}
\hypersetup{pdfsubject={Manual for the LaTeX2e Package childdoc}}
\date{30 December 2018, \textsf{v2.0}}
\maketitle

\begin{abstract}\noindent
\textsf{childdoc} is a \LaTeXe{} package
that enables the direct compilation
of document sections included by |\include|
to individual files.
\end{abstract}

\begingroup
\parskip0ex
\tableofcontents
\endgroup

%%%%%%%%%%%%%%%%%%%%%%%%%%%%%%%%%%%%%%%%%%%%%%%%%%%%%%%%%%%%%%%%%%%%%%%%%%%%%%%%
%%%%%%%%%%%%%%%%%%%%%%%%%%%%%%%%%%%%%%%%%%%%%%%%%%%%%%%%%%%%%%%%%%%%%%%%%%%%%%%%
\section{Introduction}

\LaTeX{} provides a mechanism to structure a large document (such as a book)
into a main file and several child files (containing the chapters)
using the |\include| command.
This mechanism is beneficial for documents
which span hundreds of pages in order to
make the source file(s) more manageable.
Moreover, compilation can be restricted to
selected child files by means of the |\includeonly| command.
The latter feature can be used to reduce the compilation time while editing
(this was significantly more useful in the earlier days of \LaTeX{})
or to generate a smaller document which is easier to navigate.
Another application of |\includeonly| is to generate
documents consisting of selected parts of the complete document.

However, there are a few drawbacks of the plain |\include| mechanism:
\begin{itemize}
\item
The child files cannot be compiled on their own,
they can only be compiled via the main file.
A naive editing environment
(such as a text editor with an option
to have the current file processed by \LaTeX)
may require one to switch to the main file before compiling;
attempting to compile the child file produces errors.
\item
The main file must be modified (each time)
to adjust the |\includeonly| command
to the present needs. This easily leaves the main file in a messy state.
\item
The generated document will always carry the filename
of the main document. This is inconvenient if
several child files are to be compiled and
to be kept for distribution.
\end{itemize}

The present package provides a simple interface
to make child files individually compilable by \LaTeX{}.
Compiling a child file then has the same effect as compiling
the main file with an |\includeonly| command
to select the appropriate child.
Moreover the generated document will carry the name of the child
rather than the main file.
This resolves all three above issues.

This feature is meant to make the editing of books,
thesis documents and lecture notes somewhat more convenient.
However, the package can also be used efficiently for
composing a series of documents (such as exercise sheets)
which are typically distributed individually.
It then assists the author in generating the individual documents
(potentially in different versions)
as well as a document containing the collected series.
Another application is in developing style files
or other kinds of included material
where compilation of the style file could redirect
to a sample or test file.

%%%%%%%%%%%%%%%%%%%%%%%%%%%%%%%%%%%%%%%%%%%%%%%%%%%%%%%%%%%%%%%%%%%%%%%%%%%%%%%%
%%%%%%%%%%%%%%%%%%%%%%%%%%%%%%%%%%%%%%%%%%%%%%%%%%%%%%%%%%%%%%%%%%%%%%%%%%%%%%%%
\section{Usage}

First of all, the package \textsf{childdoc} is \emph{not} a standard
\LaTeXe{} |.sty| style file! Therefore it needs to be invoked in
a non-standard way.

%%%%%%%%%%%%%%%%%%%%%%%%%%%%%%%%%%%%%%%%%%%%%%%%%%%%%%%%%%%%%%%%%%%%%%%%%%%%%%%%
\subsection{Included Files}
\label{sec:include}

%%%%%%%%%%%%%%%%%%%%%%%%%%%%%%%%%%%%%%%%
\DescribeMacro{\childdocmain}
To use the package, add the commands
\begin{center}
\begin{tabular}{l}
|\input{childdoc.def}|\\
|\childdocmain{}|\\
\end{tabular}
\end{center}
at the very top of the main \LaTeX{} file,
in particular \emph{before} the |\documentclass| statement!
The argument of |\childdocmain| should be left empty
(but it must be present).

%%%%%%%%%%%%%%%%%%%%%%%%%%%%%%%%%%%%%%%%
\DescribeMacro{\childdocof}
Furthermore, add the commands
\begin{center}
\begin{tabular}{l}
|\input{childdoc.def}|\\
|\childdocof{|\textit{main}|}|\\
\end{tabular}
\end{center}
at the top of every child file \textit{child}
which is included by |\include{|\textit{child}|}|
from within the main file
(or at least for those files to be compiled individually).
The argument \textit{main} must be the filename of the main file.

There are a couple of
considerations in setting up the main and child documents:

%%%%%%%%%%%%%%%%%%%%%%%%%%%%%%%%%%%%%%%%
\paragraph{Restrictions.}

Please note the following restrictions:
\begin{itemize}
\item
|\childdocmain| must be called with one argument \textit{main}
to ensure compatibility with earlier version of the package.
It must either be empty (|\childdocmain{}|)
or precisely match the filename of the main file in which it is specified.
See \secref{sec:detection} for further information.
\item
The filename \textit{main} must be specified without the |.tex| extension.
\item
The filename \textit{main} is case sensitive
(even in case-insensitive file systems)
due to internal string comparison.
\item
The argument \textit{main} should be fully expanded, it cannot be a macro.
\item
Subdirectories and special characters should be avoided in filenames.
\item
The command |\childdocmain{|\textit{main}|}| must be followed by a whitespace.
It should not be followed immediately by another command
or by a comment mark `|%|'.
This is because the \TeX{} parser reads the token immediately following
the argument of |\childdocmain| and puts it
at the beginning of every child section;
however, a white\-space is ignored.
\end{itemize}

%%%%%%%%%%%%%%%%%%%%%%%%%%%%%%%%%%%%%%%%
\paragraph{Content of Main File.}

It is advisable to place all content in the child files included by |\include|.
Any output contained in the main file will appear in all child documents
unless suppressed manually;
it cannot be suppressed automatically by the |\includeonly| directive
and thus should normally be avoided.
A method to include some content in the main file
by means of conditional processing is described in \secref{sec:conditional}.

%%%%%%%%%%%%%%%%%%%%%%%%%%%%%%%%%%%%%%%%
\paragraph{Page Numbering.}

When only a part of the document is compiled,
the appropriate numbering of pages
(as well as other status parameters)
is determined from the |.aux| files.
The latter contain information from previous passes.
However this information needs to propagate through
all intermediate child documents.
Therefore the page numbering in child documents may well
be inconsistent until the complete document is compiled at least once.

A useful (if unconventional) way to always ensure a consistent
page numbering is to restart the numbering in each child document
and denote the pages by `\textit{child}|.|\textit{page}'
where \textit{child} represents the chapter/section number of the child file.
This can be achieved by the command
|\numberwithin{page}{|\textit{child}|}|
of the \textsf{amsmath} package
where \textit{child} can be |chapter| or |section|
depending on the chosen structuring.
Alternatively, one can modify the macro |\thepage| appropriately
and reset the counter |page| at the start of each child file.

%%%%%%%%%%%%%%%%%%%%%%%%%%%%%%%%%%%%%%%%%%%%%%%%%%%%%%%%%%%%%%%%%%%%%%%%%%%%%%%%
\subsection{Conditional Processing}
\label{sec:conditional}

The package provides a mechanism to compile different versions
of a document. To customise the versions further some conditional processing
can come in handy to distinguish which version is being compiled.
The package provides two macros to describe the compilation context:

%%%%%%%%%%%%%%%%%%%%%%%%%%%%%%%%%%%%%%%%
\DescribeMacro{\ifchilddoc}
The conditional |\ifchilddoc| distinguishes between the compilation of
child documents and the main document:
%
\begin{center}
|\ifchilddoc |\textit{child-code}| |[|\||else |\textit{main-code}]| \||fi|
\end{center}

%%%%%%%%%%%%%%%%%%%%%%%%%%%%%%%%%%%%%%%%
\DescribeMacro{\childdocname}
\DescribeMacro{\childdocjob}
The macro |\childdocname| contains the filename (without extension)
of the main or child file being processed.
Note that |\childdocjob| will always contain the name of the main file.

%%%%%%%%%%%%%%%%%%%%%%%%%%%%%%%%%%%%%%%%
\paragraph{Title Page.}

Conditional processing can be used to include a title or banner page
in the main document when proper precautions are taken.
Importantly, the code in the main file should ensure that the page counter
(as well as other status parameters which are stored in the |.aux| files)
takes the same value after the conditional processing.
Otherwise the page numbers may take divergent values
depending on which part is compiled.

For example, a title page could be declared by:
%
\begin{center}
\begin{tabular}{l}
|\ifchilddoc\||else|\\
|\addtocounter{page}{-1}|\\
\textit{code for title page}\\
|\newpage|\\
|\||fi|
\end{tabular}
\end{center}
%
A banner page for the child documents can be generated by:
%
\begin{center}
\begin{tabular}{l}
|\ifchilddoc|\\
|\addtocounter{page}{-1}|\\
\textit{code for banner page}\\
|\newpage|\\
|\||fi|
\end{tabular}
\end{center}
%
Here one could write a message such as:
\begin{center}
|This is the part \childdocname{} of \childdocjob{}.|
\end{center}

%%%%%%%%%%%%%%%%%%%%%%%%%%%%%%%%%%%%%%%%%%%%%%%%%%%%%%%%%%%%%%%%%%%%%%%%%%%%%%%%
\subsection{Flags}
\label{sec:flags}

The package makes it easy to generate different versions
of the main or child documents.
To this end compilation flags can be defined
and assigned different default values.
They will be particularly useful in conjunction
with the forwarding mechanism described in \secref{sec:forward}.

For example, it may be useful to have a flag |\version|
which can be set to |draft| or |final|.
The document source will contain some conditional code
depending on the value of |\version|.
Suppose further, the flag should default to |final| for the main file
and to |draft| for child files
which is a natural assignment for editing the document.
This is achieved by placing the following code
in the preamble of the main document
(below the |\childdocmain| directive):
%
\begin{center}
\begin{tabular}{l}
|\ifchilddoc|\\
|\providecommand{\version}{draft}|\\
|\||else|\\
|\providecommand{\version}{final}|\\
|\||fi|
\end{tabular}
\end{center}
%
The definition by |\providecommand| makes sure
that previous definitions are not overwritten.
Further statements |\providecommand{\version}{...}|
can thus be added before the above code to override it.

For the main file, one might add a line
(between |\childdocmain| and the above block)
%
\begin{center}
|%\ifchilddoc\||else\providecommand{\version}{draft}\||fi|
\end{center}
%
which can be uncommented to produce a draft version.
Likewise one can add a line to the very top of a child file
(above the |\childdocof{|\textit{main}|}| directive)
%
\begin{center}
|%\providecommand{\version}{final}|
\end{center}
%
which can be uncommented to produce the final version of this child document.

%%%%%%%%%%%%%%%%%%%%%%%%%%%%%%%%%%%%%%%%%%%%%%%%%%%%%%%%%%%%%%%%%%%%%%%%%%%%%%%%
\subsection{Forwarding}
\label{sec:forward}

Different versions of the main or child documents
using compilation flags as described in \secref{sec:flags}
can be (permanently) stored in different files
for convenient compilation, viewing and distribution.
To this end, the package defines a command
to pass on compilation to a different file:

%%%%%%%%%%%%%%%%%%%%%%%%%%%%%%%%%%%%%%%%
\DescribeMacro{\childdocforward}
The command |\childdocforward| redirects processing to
another source file:
%
\begin{center}
\begin{tabular}{l}
|\input{childdoc.def}|\\
|\childdocforward[|\textit{main}|]{|\textit{dest}|}|\\
\end{tabular}
\end{center}
%
The argument \textit{dest} is the destination file
(without extension).
It should be the main file or one of the child files.
Note that further \textsf{childdoc} directives
such as |\childdocof| and |\childdocforward|
in the indicated file will be processed in this form.
The optional argument \textit{main}
passes on directly to the main file \textit{main}
while pretending to compile the child \textit{dest}.
This form behaves as if \textit{dest}
issues |\childdocof{|\textit{main}|}| right away,
and no further \textsf{childdoc} directives will be processed.

%%%%%%%%%%%%%%%%%%%%%%%%%%%%%%%%%%%%%%%%
\DescribeMacro{\...prefix}
In the alternative form |\childdocforwardprefix|,
%
\begin{center}
\begin{tabular}{l}
|\input{childdoc.def}|\\
|\childdocforwardprefix[|\textit{main}|]{|\textit{prefix}|}{|\textit{dest}|}|
\end{tabular}
\end{center}
%
the destination file is determined by a pattern
depending on the current file:
To make this work, the current file must be called
`{\textit{prefix}\hspace{0.2em}\textit{suffix}}'
with \textit{prefix} matching precisely the argument.
Processing is then passed on to the file
`{\textit{dest}\hspace{0.2em}\textit{suffix}}'.
Surely, the same effect is achieved by
directly specifying the
argument `{\textit{dest}\hspace{0.2em}\textit{suffix}}'
in the first form.
However, that requires to set up a different file
for each child. With the alternative form of the command
all these files can have exactly the same content
which simplifies setting them up and maintaining them.

For example, the following file |draft.tex|
with a compilation flag |\version| as described in \secref{sec:flags}
compiles the main document as a draft:
%
\begin{center}
\begin{tabular}{l}
|\def\version{draft}|\\
|\input{childdoc.def}|\\
|\childdocforward{|\textit{main}|}|
\end{tabular}
\end{center}
%
Likewise, the following files |final|\textit{nn}|.tex|
compile the final version of the child document
|child|\textit{nn}|.tex|:
%
\begin{center}
\begin{tabular}{l}
|\def\version{final}|\\
|\input{childdoc.def}|\\
|\childdocforwardprefix{final}{child}|
\end{tabular}
\end{center}
%

Note that when several versions of a main file and/or of each child file
are to be generated, it may be convenient to set up a |Makefile| or
shell script to automatise the process.

%%%%%%%%%%%%%%%%%%%%%%%%%%%%%%%%%%%%%%%%%%%%%%%%%%%%%%%%%%%%%%%%%%%%%%%%%%%%%%%%
\subsection{Command Line Processing}
\label{sec:commandline}

The effect of redirection files can also be achieved by invoking
the \LaTeX{} compiler with a more elaborate command line.
Most conveniently this should be done as part
of a shell script or a |Makefile|.

When using \textsf{childdoc} in the main file, the following
command lines effectively perform a redirection
(note that depending on the shell being used,
backslashes may have to be doubled: `|\|' $\to$ `|\\|'):
%
\begin{center}
|... -jobname "|\textit{target}|" |\\|"|[\textit{flags}]%
|\input{childdoc.def}\childdocforward[|\textit{main}|]{|\textit{dest}|}"|
\end{center}
%
Here \textit{target} is the name of the output file,
\textit{main} is the name of the main file
and \textit{dest} is the name of the main or child file to be processed
(all filenames without extensions).
The optional argument \textit{main} can be omitted
if \textit{main} matches \textit{dest}.
Optionally, compilation \textit{flags} can be defined via |\def| commands.
This command line makes the \TeX{} engine believe
it is compiling the file \textit{target}
whose content is specified as the latter parameter.
The provided code then forwards the processing to
\textit{main} or \textit{dest} as described in \secref{sec:forward}.

%%%%%%%%%%%%%%%%%%%%%%%%%%%%%%%%%%%%%%%%%%%%%%%%%%%%%%%%%%%%%%%%%%%%%%%%%%%%%%%%
\subsection{Include by Input}
\label{sec:input}

Including child documents by |\include| has some restrictions by design.
Most notably, the content of a child document always occupies
its own set of pages; pages cannot be shared between child documents.
Usually, this behaviour makes perfect sense
because each child document contain an essential part of the document.
However, in some situations it may be desirable to compose
a document from a collection of parts
without having mandatory page breaks between then.
For this case, the package
provides a mechanism to include parts
by |\input| which can also be processed individually.
However, by construction this mechanism
requires manual handling of the content to be output.

%%%%%%%%%%%%%%%%%%%%%%%%%%%%%%%%%%%%%%%%
\DescribeMacro{\ifchilddocmanual}
The main file should be prepared as usual, see \secref{sec:include}.
However, the document body must make a distinction
between processing of an individual part and of the main document, e.g.:
%
\begin{center}
\begin{tabular}{l}
|\ifchilddocmanual|\\
|\input{\childdocname}|\\
|\||else|\\
\textit{document body with }|\input{|\textit{part}|}|\\
|\||fi|
\end{tabular}
\end{center}
%
The conditional |\ifchilddocmanual| is true whenever
a part to be included by |\input| is being compiled,
and the name of the part is stored in |\childdocname|.

%%%%%%%%%%%%%%%%%%%%%%%%%%%%%%%%%%%%%%%%
\DescribeMacro{\childdocby}
Each part to be included by |\input| should start with:
%
\begin{center}
\begin{tabular}{l}
|\input{childdoc.def}|\\
|\childdocby{|\textit{main}|}|\\
\end{tabular}
\end{center}
%
The directive |\childdocby| is similar to |\childdocof|
described in \secref{sec:include},
but the subsequent selection of content must be done manually.
To that end, both |\ifchilddoc| and |\ifchilddocmanual|
will be true upon processing of a part,
and the name of the part is stored in |\childdocname|.
Note that |\jobname| will be set to the filename of the current part
so that each part receives an individual |.aux| file
that does not interfere with the |.aux| file(s) of the main document.
This behaviour can be altered by the alternative form
|\childdocby[*]{|\textit{main}|}| (with a non-empty optional argument)
which uses the |.aux| file of the main document
by setting |\jobname| to \textit{main}.

%%%%%%%%%%%%%%%%%%%%%%%%%%%%%%%%%%%%%%%%%%%%%%%%%%%%%%%%%%%%%%%%%%%%%%%%%%%%%%%%
\subsection{Driver Development}
\label{sec:driver}

The \textsf{childdoc} mechanism can also be use for the development
of definition files such as \LaTeX{} styles or classes.
This case differs from the above setup with multiple parts
included by |\include| in that no |\includeonly| should be invoked.
This can be achieved by starting the include file
(before |\ProvidesPackage|) with:
%
\begin{center}
\begin{tabular}{l}
|\input{childdoc.def}|\\
|\childdocforward{|\textit{main}|}|\\
\end{tabular}
\end{center}
%
or alternatively with:
%
\begin{center}
\begin{tabular}{l}
|\input{childdoc.def}|\\
|\childdocby{|\textit{main}|}|\\
\end{tabular}
\end{center}
%
Both forms have slightly different effects as described above.
The main file is prepared as usual, see \secref{sec:include}.

%%%%%%%%%%%%%%%%%%%%%%%%%%%%%%%%%%%%%%%%%%%%%%%%%%%%%%%%%%%%%%%%%%%%%%%%%%%%%%%%
\subsection{Legacy Detection}
\label{sec:detection}

The directive |\childdocmain| in the main file can detect
whether the complete document or merely a child is to be compiled
even without using the directive |\childdocof|.
This method is deprecated because it is less robust
and there is no compelling reason to use it;
it is merely provided for backward compatibility
and it may be removed in future versions.

If the detection mechanism is to be used,
it is mandatory to correctly specify
the filename of the main file as the argument of |\childdocmain|:
%
\begin{center}
\begin{tabular}{l}
|\input{childdoc.def}|\\
|\childdocmain{|\textit{main}|}|\\
\end{tabular}
\end{center}
%
If |\jobname| does not match the argument \textit{main} of |\childdocmain|,
it is assumed that |\jobname| points to the child file to be compiled.
When using |\childdocmain| with the main file specified as argument,
it suffices to start a child file
with just |\input{|\textit{main}|}|
without loading of the package and using |\childdocof|.
If instead all processing is done
with the appropriate \textsf{childdoc} directives,
the argument of \textit{main} of |\childdocmain| can be empty.

An alternative version of the command line processing described
in \secref{sec:commandline} using the detection mechanism reads:
%
\begin{center}
|... -jobname "|\textit{target}|" "|[\textit{flags}]%
[|\def\jobname{|\textit{dest}|}|]|\input{|\textit{main}|}"|
\end{center}

%%%%%%%%%%%%%%%%%%%%%%%%%%%%%%%%%%%%%%%%%%%%%%%%%%%%%%%%%%%%%%%%%%%%%%%%%%%%%%%%
\subsection{Manual Code}
\label{sec:manual}

In case one cannot be certain whether the definitions file |childdoc.def|
is installed on the target \TeX{} distribution
and one prefers not to ship it,
it is conceivable to paste a few relevant commands into the sources.

To that end, drop all statements |\input{childdoc.def}|
and perform the replacements as outlined below.
Instead of |\childdocmain{|\textit{main}|}| add the following code
to the top of the main file:
%
\begin{center}
\begin{tabular}{l}
|\||ifdefined\childdocname\endinput\||fi\newif\ifchilddoc|\\
|\edef\childdocname{\scantokens\expandafter{\jobname\noexpand}}|\\
|\def\childdocmain{|\textit{main}|}\||ifx\childdocmain\childdocname\||else|\\
|\childdoctrue\includeonly{\childdocname}\let\jobname\childdocmain\||fi|\\
\end{tabular}
\end{center}
%
Instead of |\childdocof{|\textit{main}|}| just include the main file
at the top of each child file:
%
\begin{center}
|\input{|\textit{main}|}|
\end{center}
%
A simple redirection |\childdocforward{|\textit{dest}|}| is achieved by:
%
\begin{center}
|\def\jobname{|\textit{dest}|}\input{\jobname}|
\end{center}
%
The redirection with prefix
|\childdocforwardprefix[|\textit{prefix}|]{|\textit{dest}|}|
is accomplished by:
%
\begin{center}
\begin{tabular}{l}
|{\edef\jobname{\scantokens\expandafter{\jobname\noexpand}}|\\
|\def\redirectjob |\textit{prefix}|#1~~~{\gdef\jobname{|\textit{dest}|#1}}|\\
|\expandafter\redirectjob\jobname~~~}\input{\jobname}|
\end{tabular}
\end{center}

In an alternative approach,
child documents can be compiled by a specific command line
without additional code or specific definitions:
%
\begin{center}
|... -jobname "|\textit{target}|" "|[\textit{flags}]%
|\includeonly{|\textit{dest}|}\input{|\textit{main}|}"|
\end{center}
%

%%%%%%%%%%%%%%%%%%%%%%%%%%%%%%%%%%%%%%%%%%%%%%%%%%%%%%%%%%%%%%%%%%%%%%%%%%%%%%%%
%%%%%%%%%%%%%%%%%%%%%%%%%%%%%%%%%%%%%%%%%%%%%%%%%%%%%%%%%%%%%%%%%%%%%%%%%%%%%%%%
\section{Information}

%%%%%%%%%%%%%%%%%%%%%%%%%%%%%%%%%%%%%%%%%%%%%%%%%%%%%%%%%%%%%%%%%%%%%%%%%%%%%%%%
\subsection{Copyright}

Copyright \copyright{} 2017--2018 Niklas Beisert

This work may be distributed and/or modified under the
conditions of the \LaTeX{} Project Public License, either version 1.3
of this license or (at your option) any later version.
The latest version of this license is in
  \url{http://www.latex-project.org/lppl.txt}
and version 1.3 or later is part of all distributions of \LaTeX{}
version 2005/12/01 or later.

This work has the LPPL maintenance status `maintained'.

The Current Maintainer of this work is Niklas Beisert.

This work consists of the files |README.txt|, |childdoc.ins| and |childdoc.dtx|
as well as the derived files |childdoc.def|, |cdocsamp.tex|
with |cdocsch1.tex|, |cdocsch2.tex|, |cdocspt3.tex|, |cdocspt4.tex|,
|cdocsdrf.tex|, |cdocsfn1.tex|, |cdocsfn2.tex|
as well as |childdoc.pdf|.

%%%%%%%%%%%%%%%%%%%%%%%%%%%%%%%%%%%%%%%%%%%%%%%%%%%%%%%%%%%%%%%%%%%%%%%%%%%%%%%%
\subsection{Files and Installation}

The package consists of the files:
%
\begin{center}
\begin{tabular}{ll}
    |README.txt|   & readme file \\
    |childdoc.ins| & installation file \\
    |childdoc.dtx| & source file \\
    |childdoc.def| & definition file \\
    |cdocsamp.tex| & sample main file \\
    |cdocsch1.tex| & sample include file \\
    |cdocsch2.tex| & sample include file \\
    |cdocspt3.tex| & sample part file \\
    |cdocspt4.tex| & sample part file \\
    |cdocsdrf.tex| & sample redirection file \\
    |cdocsfn1.tex| & sample redirection file \\
    |cdocsfn2.tex| & sample redirection file \\
    |childdoc.pdf| & manual
\end{tabular}
\end{center}
%
The distribution consists of the files
|README.txt|, |childdoc.ins| and |childdoc.dtx|.
%
\begin{itemize}
\item
Run (pdf)\LaTeX{} on |childdoc.dtx|
to compile the manual |childdoc.pdf| (this file).
\item
Run \LaTeX{} on |childdoc.ins| to create the definitions file |childdoc.def|
and the sample |cdocsamp.tex| with include files
|cdocsch1.tex|, |cdocsch2.tex|, |cdocspt3.tex|, |cdocspt4.tex|,
|cdocsdrf.tex|, |cdocsfn1.tex|, |cdocsfn2.tex|.
Then copy the file |childdoc.def| to an appropriate directory of your \LaTeX{}
distribution, e.g.\ \textit{texmf-root}|/tex/latex/childdoc|.
\end{itemize}

%%%%%%%%%%%%%%%%%%%%%%%%%%%%%%%%%%%%%%%%%%%%%%%%%%%%%%%%%%%%%%%%%%%%%%%%%%%%%%%%
\subsection{Related CTAN Packages}

There are several other packages which offer a similar functionality:
%
\begin{itemize}
\item
The packages
\href{http://ctan.org/pkg/docmute}{\textsf{docmute}},
\href{http://ctan.org/pkg/includex}{\textsf{includex}} and
\href{http://ctan.org/pkg/standalone}{\textsf{standalone}}
provide commands to include only the document body of
a child file thus allowing both files to be compiled individually.
\item
The packages \href{http://ctan.org/pkg/subdocs}{\textsf{subdocs}}
and \href{http://ctan.org/pkg/subfiles}{\textsf{subfiles}}
provide structures in which the main and child documents can be
encapsulated and allowing them to be compiled individually.
The inclusion mechanism is different from the conventional |\include|.
\item
The package \href{http://ctan.org/pkg/combine}{\textsf{combine}}
is an elaborate solution to combine several documents into one.
\end{itemize}
%
See also the CTAN topic \href{http://ctan.org/topic/subdocs}{\textsf{subdocs}}
for further related packages.
The present package differs from the above solutions in that
a document structure constructed with the conventional |\include| mechanism
just needs two extra commands at the top of every file
such that all constituent files can be compiled individually.

%%%%%%%%%%%%%%%%%%%%%%%%%%%%%%%%%%%%%%%%%%%%%%%%%%%%%%%%%%%%%%%%%%%%%%%%%%%%%%%%
%\subsection{Feature Suggestions}
%
%The following is a list of features which may be useful for future
%versions of this package:
%%
%\begin{itemize}
%\item
%\ldots
%\end{itemize}

%%%%%%%%%%%%%%%%%%%%%%%%%%%%%%%%%%%%%%%%%%%%%%%%%%%%%%%%%%%%%%%%%%%%%%%%%%%%%%%%
\subsection{Revision History}

%%%%%%%%%%%%%%%%%%%%%%%%%%%%%%%%%%%%%%%%
\paragraph{v2.0:} 2018/12/30

\begin{itemize}
\item
immediate forward processing
\item
added |\childdocby| mechanism
\item
manual restructured
\end{itemize}

%%%%%%%%%%%%%%%%%%%%%%%%%%%%%%%%%%%%%%%%
\paragraph{v1.6:} 2018/01/17

\begin{itemize}
\item
application for development of include files
\item
corrections to manual
\end{itemize}

%%%%%%%%%%%%%%%%%%%%%%%%%%%%%%%%%%%%%%%%
\paragraph{v1.5:} 2017/05/21

\begin{itemize}
\item
more complete structuring introduced
\item
|\childdocof| introduced
\item
|\childdoc| renamed to |\childdocmain|
\item
|\childredirect| renamed to |\childdocforward| and |\childdocforwardprefix|
and functionality expanded
\end{itemize}

%%%%%%%%%%%%%%%%%%%%%%%%%%%%%%%%%%%%%%%%
\paragraph{v1.0:} 2017/04/27

\begin{itemize}
\item
manual and install package
\item
first version published on CTAN
\end{itemize}

%%%%%%%%%%%%%%%%%%%%%%%%%%%%%%%%%%%%%%%%
\paragraph{v0.6:} 2017/04/26

\begin{itemize}
\item
redirection mechanism added
\end{itemize}

%%%%%%%%%%%%%%%%%%%%%%%%%%%%%%%%%%%%%%%%
\paragraph{v0.5:} 2017/04/26

\begin{itemize}
\item
functionality in definition file
\end{itemize}


%%%%%%%%%%%%%%%%%%%%%%%%%%%%%%%%%%%%%%%%%%%%%%%%%%%%%%%%%%%%%%%%%%%%%%%%%%%%%%%%
%%%%%%%%%%%%%%%%%%%%%%%%%%%%%%%%%%%%%%%%%%%%%%%%%%%%%%%%%%%%%%%%%%%%%%%%%%%%%%%%
%%%%%%%%%%%%%%%%%%%%%%%%%%%%%%%%%%%%%%%%%%%%%%%%%%%%%%%%%%%%%%%%%%%%%%%%%%%%%%%%
\appendix

\settowidth\MacroIndent{\rmfamily\scriptsize 000\ }

 \DocInput{childdoc.dtx}

\end{document}
%</driver>
% \fi
%
% %%%%%%%%%%%%%%%%%%%%%%%%%%%%%%%%%%%%%%%%%%%%%%%%%%%%%%%%%%%%%%%%%%%%%%%%%%%%%%
% %%%%%%%%%%%%%%%%%%%%%%%%%%%%%%%%%%%%%%%%%%%%%%%%%%%%%%%%%%%%%%%%%%%%%%%%%%%%%%
% \section{Sample}
%\iffalse
%<*samplemain>
%\fi
%
% The following presents a sample document
% with two chapters, two parts, a title page,
% a compile flag as well as three forwarding files to set the flag.
% It consists of eight |.tex| files:
% \begin{center}
% \begin{tabular}{ll}
% |cdocsamp.tex|&main file\\
% |cdocsch1.tex|&include file for chapter 1\\
% |cdocsch2.tex|&include file for chapter 2\\
% |cdocspt3.tex|&include file for part 3\\
% |cdocspt4.tex|&include file for part 4\\
% |cdocsdrf.tex|&forwarding file for main file in draft mode\\
% |cdocsfi1.tex|&forwarding file for final version of chapter 1\\
% |cdocsfi2.tex|&forwarding file for final version of chapter 2\\
% \end{tabular}
% \end{center}
% Each of the eight files can be compiled directly by the \LaTeX{} compiler.
%
% %%%%%%%%%%%%%%%%%%%%%%%%%%%%%%%%%%%%%%
% \paragraph{Main File.}
%
% The main file is called |cdocsamp.tex|.
%
% Load the \textsf{childdoc} definitions and
% declare the filename for the main document:
%    \begin{macrocode}
\input{childdoc.def}
\childdocmain{}
%    \end{macrocode}

% Optional override for |\version| flag:
%    \begin{macrocode}
%%\ifchilddoc\else\providecommand{\version}{draft}\fi
%    \end{macrocode}

% Define the default values for the |\version| flag
% (|final| for the main file and |draft| for childs):
%    \begin{macrocode}
\ifchilddoc
\providecommand{\version}{draft}
\else
\providecommand{\version}{final}
\fi
%    \end{macrocode}

% Load the standard document class:
%    \begin{macrocode}
\documentclass[12pt]{article}
%    \end{macrocode}

% Start the document body:
%    \begin{macrocode}
\begin{document}
%    \end{macrocode}

% Declare a title page.
% Print title, part of document being processed and version flag:
%    \begin{macrocode}
\addtocounter{page}{-1}
\begin{center}
{\LARGE\bfseries{}childdoc example\par}
\vspace{1cm}
\ifchilddoc
\ifchilddocmanual part\else chapter\fi:
`\childdocname' of `\childdocjob'\par
\else
main document: `\childdocjob'\par
\fi
version: \version\par
\end{center}
\newpage
%    \end{macrocode}

% Manually include selected file,
% otherwise process as usual:
%    \begin{macrocode}
\ifchilddocmanual
\section*{part `\childdocname'}
\input{\childdocname}
\else
%    \end{macrocode}

% Include the two chapters:
%    \begin{macrocode}
\include{cdocsch1}
\include{cdocsch2}
%    \end{macrocode}

% Include the two parts unless only chapters should be displayed:
%    \begin{macrocode}
\ifchilddoc\else
\section{part three}
\input{cdocspt3}
\section{part four}
\input{cdocspt4}
\fi
%    \end{macrocode}

% Process as usual until here:
%    \begin{macrocode}
\fi
%    \end{macrocode}

% End of document body:
%    \begin{macrocode}
\end{document}
%    \end{macrocode}
%\iffalse
%</samplemain>
%\fi
%
% %%%%%%%%%%%%%%%%%%%%%%%%%%%%%%%%%%%%%%
% \paragraph{Chapter Include Files.}
%
% The include files are called |cdocsch1.tex| and |cdocsch2.tex|.
%
%\iffalse
%<*samplechap1|samplechap2>
%\fi

% Optional override for |\version| flag:
%    \begin{macrocode}
%%\providecommand{\version}{final}
%    \end{macrocode}

% Include the main document:
%    \begin{macrocode}
\input{childdoc.def}
\childdocof{cdocsamp}
%    \end{macrocode}

%\iffalse
%</samplechap1|samplechap2>
%\fi
%
%\iffalse
%<*samplechap1>
%\fi
% Some text for chapter 1:
%    \begin{macrocode}
\section{one}
some text in chapter one
%    \end{macrocode}

%\iffalse
%</samplechap1>
%\fi
% Some text for chapter 2:
%\iffalse
%<*samplechap2>
%\fi
%    \begin{macrocode}
\section{two}
more text in chapter two
%    \end{macrocode}

%\iffalse
%</samplechap2>
%\fi
%
% %%%%%%%%%%%%%%%%%%%%%%%%%%%%%%%%%%%%%%
% \paragraph{Part Include Files.}
%
% The include files are called |cdocspt3.tex| and |cdocspt4.tex|.
%
%\iffalse
%<*samplepart3|samplepart4>
%\fi

% Optional override for |\version| flag:
%    \begin{macrocode}
%%\providecommand{\version}{final}
%    \end{macrocode}

% Include the main document:
%    \begin{macrocode}
\input{childdoc.def}
\childdocby{cdocsamp}
%    \end{macrocode}

%\iffalse
%</samplepart3|samplepart4>
%\fi
%
%\iffalse
%<*samplepart3>
%\fi
% Some text for part 3:
%    \begin{macrocode}
some text in part three
%    \end{macrocode}

%\iffalse
%</samplepart3>
%\fi
% Some text for part 4:
%\iffalse
%<*samplepart4>
%\fi
%    \begin{macrocode}
more text in part four
%    \end{macrocode}

%\iffalse
%</samplepart4>
%\fi
%
% %%%%%%%%%%%%%%%%%%%%%%%%%%%%%%%%%%%%%%
% \paragraph{Forwarding for a Complete Draft.}
%
% The following forwarding file |cdocsdrf.tex|
% compiles the main document in draft mode:
%\iffalse
%<*sampledraft>
%\fi
%    \begin{macrocode}
\def\version{draft}
\input{childdoc.def}
\childdocforward{cdocsamp}
%    \end{macrocode}

%\iffalse
%</sampledraft>
%\fi
%
% %%%%%%%%%%%%%%%%%%%%%%%%%%%%%%%%%%%%%%
% \paragraph{Forwarding for Final Version of the Chapters.}
%
% The following forwarding files |cdocsfn1.tex| and |cdocsfn2.tex|
% (with identical content)
% compile the final versions of the child documents
% |cdocsch1.tex| and |cdocsch2.tex|, respectively:
%\iffalse
%<*samplefinal>
%\fi
%    \begin{macrocode}
\def\version{final}
\input{childdoc.def}
\childdocforwardprefix[cdocsamp]{cdocsfn}{cdocsch}
%    \end{macrocode}

%\iffalse
%</samplefinal>
%\fi
%
% %%%%%%%%%%%%%%%%%%%%%%%%%%%%%%%%%%%%%%
% \paragraph{Command Line Processing.}
%
% The following three command lines generate the output files
% |cdocscld|, |cdocscl1| and |cdocscl2|
% which should be identical to
% |cdocsdrf|, |cdocsch1| and |cdocsfn2|, respectively:
% \begin{center}
% \begin{tabular}{l}
% |latex -jobname cdocscld \|\\
% |  "\def\version{draft}\input{childdoc.def}\childdocforward{cdocsamp}"|\\
% |latex -jobname cdocscl1 \|\\
% |  "\input{childdoc.def}\childdocforward[cdocsamp]{cdocsch1}"|\\
% |latex -jobname cdocscl2 \|\\
% |  "\def\version{final}\input{childdoc.def}\childdocforward{cdocsch2}"|
% \end{tabular}
% \end{center}
% Note that the trailing backslash on each first line
% merely continues the input to the second line
% (for convenient cut ant paste).
% Furthermore, the command |latex| can be replaced by any
% of its alternative versions such as |pdflatex|.
%
% %%%%%%%%%%%%%%%%%%%%%%%%%%%%%%%%%%%%%%%%%%%%%%%%%%%%%%%%%%%%%%%%%%%%%%%%%%%%%%
% %%%%%%%%%%%%%%%%%%%%%%%%%%%%%%%%%%%%%%%%%%%%%%%%%%%%%%%%%%%%%%%%%%%%%%%%%%%%%%
% \section{Implementation}
%\iffalse
%<*package>
%\fi
%
% This section describes the definitions file |childdoc.def|.

% The definitions cannot be loaded using |\usepackage| or |\RequirePackage|
% which has a mechanism to prevent loading a style file more than once.
% When loading the definitions by means of |\input|
% multiple instances have to be prevented manually:
%\iffalse
%This code needs to be before the `\ProvidesFile' directive
%which is defined at the beginning of this file.
%Therefore it is also placed there and commented out here.
%</package>
%<*discard>
%\fi
%    \begin{macrocode}
\ifdefined\childdocmain\endinput\fi
%    \end{macrocode}
%\iffalse
%</discard>
%<*package>
%\fi
%
% \macro{\ifchilddoc}
% \macro{\ifchilddocmanual}
% The conditional |\ifchilddoc| tells whether a
% child (true) or main (false) document is being compiled.
% The conditional |\ifchilddocmanual| tells whether
% the |\includeonly| mechanism is used (false) or
% the selection of child files must be performed manually (true).
% The definitions initialise to false:
%    \begin{macrocode}
\newif\ifchilddoc
\newif\ifchilddocmanual
%    \end{macrocode}

% \macro{\childdocname}
% \macro{\childdocjob}
% The macro |\childdocname| stores the name of the main document
% to be compiled. The macro |\childdocjob| stores the name of
% the document on which the \LaTeX{} compiler was originally invoked.
% The content of |\jobname| cannot be compared
% to filenames specified in the source due to different catcodes.
% The following code rescans |\jobname|, stores the result
% in |\childdocname| and saves a copy in |\childdocjob|:
%    \begin{macrocode}
\edef\childdocname{\scantokens\expandafter{\jobname\noexpand}}
\let\childdocjob\childdocname
%    \end{macrocode}

% \macro{\childdocdisable}
% The macro |\childdocdisable| prevents the main file
% from being processed more than once.
% At this stage, the main document command |\childdocmain|
% is assumed to be called once again where it should do nothing.
% Any subsequent call to it should prevent
% a secondary processing of the main document
% It overwrites the forwarding commands
% |\childdocof| and |\childdocforward|
% with empty macros to prevent further inclusions of the main document:
%    \begin{macrocode}
\newcommand{\childdocdisable}
{
  \renewcommand{\childdocmain}[1]{\renewcommand{\childdocmain}[1]{\endinput}}
  \renewcommand{\childdocof}[1]{}
  \renewcommand{\childdocby}[2][]{}
  \renewcommand{\childdocforward}[2][]{}
  \renewcommand{\childdocdisable}{}
}
%    \end{macrocode}

% \macro{\childdocmain}
% The macro |\childdocmain| is to be called at the top of the main file
% with nothing or the main filename (without extension) as argument.
% First, it breaks loops.
% If the argument is not empty and does not match |\childdocname|
% (which is set by the first inclusion of |childdoc.def|),
% |\ifchilddoc| is set to true, |\includeonly| is applied to the child file
% and |\jobname| is set to the main file
% (for proper handling of |.aux| files):
%    \begin{macrocode}
\newcommand{\childdocmain}[1]
{
  \childdocdisable\childdocmain{}
  \if?#1?\else
    \begingroup
      \def\childdoctmp{#1}
      \ifx\childdoctmp\childdocname
        \def\childdoctmp{}
      \else
        \def\childdoctmp
        {
          \childdoctrue
          \includeonly{\childdocname}
          \def\childdocjob{#1}
          \def\jobname{#1}
        }
      \fi
      \expandafter
    \endgroup
    \childdoctmp
  \fi
}
%    \end{macrocode}

% \macro{\childdocof}
% The command |\childdocof| redirects
% compilation to the main file |#1|.
%    \begin{macrocode}
\newcommand{\childdocof}[1]
{
  \childdocdisable
  \childdoctrue
  \includeonly{\childdocname}
  \def\jobname{#1}
  \def\childdocjob{#1}
  \input{#1}
}
%    \end{macrocode}

% \macro{\childdocby}
% The command |\childdocby| ....
%    \begin{macrocode}
\newcommand{\childdocby}[2][]
{
  \childdocdisable
  \childdoctrue
  \childdocmanualtrue
  \if?#1?\else
    \def\jobname{#2}
  \fi
  \def\childdocjob{#2}
  \input{#2}
  \endinput
}
%    \end{macrocode}

% \macro{\childdocforward}
% The command |\childdocforward| redirects
% compilation to the main file or
% (if the optional argument is given) a child file.
% Parameters are set as if the main file
% or a child file starting with |\childdocof| was compiled.
% Then compilation is handed over to the main file:
%    \begin{macrocode}
\newcommand{\childdocforward}[2][]
{
  \begingroup
    \if?#1?
      \def\childdoctmp
      {
        \def\childdocname{#2}
        \def\childdocjob{#2}
        \def\jobname{#2}
        \input{#2}
        \endinput
      }
    \else
      \def\childdoctmp
      {
        \childdocdisable
        \def\childdocname{#2}
        \childdoctrue
        \includeonly{#2}
        \def\childdocjob{#1}
        \def\jobname{#1}
        \input{#1}
        \endinput
      }
    \fi
    \expandafter
  \endgroup
  \childdoctmp
}
%    \end{macrocode}

% \macro{\childdocforwardprefix}
% The command |\childdocforwardprefix| redirects
% compilation to the main or a child file by means of a pattern.
% The prefix |#1| in the current filename is replaced by |#2|
% and the suffix of the current filename is kept
% (it is assumed that the filename does not contain the substring `|~~~|'
% which is used as a delimiter).
% Compilation is handed over to the new file by |\childdocforward|:
%    \begin{macrocode}
\newcommand{\childdocforwardprefix}[3][]
{
  \begingroup
    \def\childdocextract #2##1~~~{\def\childdoctmp{\childdocforward[#1]{#3##1}}}
    \expandafter\childdocextract\childdocname~~~
    \expandafter
  \endgroup
  \childdoctmp
}
%    \end{macrocode}

% \macro{\childdoc}
% The deprecated macro |\childdoc| is a legacy version of |\childdocmain|:
%    \begin{macrocode}
\newcommand{\childdoc}{\childdocmain}
%    \end{macrocode}

% \macro{\childdocredirect}
% The deprecated macro |\childdocredirect| is a legacy version
% of |\childdocforward| and |\childdocforwardprefix|:
%    \begin{macrocode}
\newcommand{\childdocredirect}[2][]
{
  \begingroup
    \if?#1?
      \def\childdoctmp{\childdocforward{#2}}
    \else
      \def\childdoctmp{\childdocforwardprefix{#1}{#2}}
    \fi
    \expandafter
  \endgroup
  \childdoctmp
}
%    \end{macrocode}

%\iffalse
%</package>
%\fi
%
\endinput

\childdocforward{cdocsamp}
%    \end{macrocode}

%\iffalse
%</sampledraft>
%\fi
%
% %%%%%%%%%%%%%%%%%%%%%%%%%%%%%%%%%%%%%%
% \paragraph{Forwarding for Final Version of the Chapters.}
%
% The following forwarding files |cdocsfn1.tex| and |cdocsfn2.tex|
% (with identical content)
% compile the final versions of the child documents
% |cdocsch1.tex| and |cdocsch2.tex|, respectively:
%\iffalse
%<*samplefinal>
%\fi
%    \begin{macrocode}
\def\version{final}
% \iffalse
%
% childdoc.dtx Copyright (C) 2017-2018 Niklas Beisert
%
% This work may be distributed and/or modified under the
% conditions of the LaTeX Project Public License, either version 1.3
% of this license or (at your option) any later version.
% The latest version of this license is in
%   http://www.latex-project.org/lppl.txt
% and version 1.3 or later is part of all distributions of LaTeX
% version 2005/12/01 or later.
%
% This work has the LPPL maintenance status `maintained'.
%
% The Current Maintainer of this work is Niklas Beisert.
%
% This work consists of the files childdoc.dtx and childdoc.ins
% and the derived files childdoc.def and cdocsamp.tex with
% cdocsch1.tex, cdocsch2.tex, cdocsdrf.tex, cdocsfn1.tex, cdocsfn2.tex.
%
%<package>\ifdefined\childdocmain\endinput\fi
%<package>\ProvidesFile{childdoc.def}[2018/12/30 v2.0 child document driver]
%<samplemain>\ProvidesFile{cdocsamp.tex}[2018/12/30 v2.0 sample for childdoc]
%<*driver>
%\ProvidesFile{childdoc.drv}[2018/12/30 v2.0 childdoc reference manual file]
\PassOptionsToClass{10pt,a4paper}{article}
\documentclass{ltxdoc}

\usepackage[margin=35mm]{geometry}
\usepackage{hyperref}
\usepackage{hyperxmp}
\usepackage[usenames]{color}

\hypersetup{colorlinks=true}
\hypersetup{pdfstartview=FitH}
\hypersetup{pdfpagemode=UseNone}
\hypersetup{pdfsource={}}
\hypersetup{pdflang={en-UK}}
\hypersetup{pdfcopyright={Copyright 2017-2018 Niklas Beisert.
  This work may be distributed and/or modified under the
  conditions of the LaTeX Project Public License, either version 1.3
  of this license or (at your option) any later version.}}
\hypersetup{pdflicenseurl={http://www.latex-project.org/lppl.txt}}
\hypersetup{pdfcontactaddress={ETH Zurich, ITP, HIT K,
  Wolfgang-Pauli-Strasse 27}}
\hypersetup{pdfcontactpostcode={8093}}
\hypersetup{pdfcontactcity={Zurich}}
\hypersetup{pdfcontactcountry={Switzerland}}
\hypersetup{pdfcontactemail={nbeisert@itp.phys.ethz.ch}}
\hypersetup{pdfcontacturl={http://people.phys.ethz.ch/\xmptilde nbeisert/}}

\newcommand{\secref}[1]{\hyperref[#1]{section \ref*{#1}}}

\parskip1ex
\parindent0pt
\let\olditemize\itemize
\def\itemize{\olditemize\parskip0pt}

\begin{document}

\title{The \textsf{childdoc} Package}
\hypersetup{pdftitle={The childdoc Package}}
\author{Niklas Beisert\\[2ex]
  Institut f\"ur Theoretische Physik\\
  Eidgen\"ossische Technische Hochschule Z\"urich\\
  Wolfgang-Pauli-Strasse 27, 8093 Z\"urich, Switzerland\\[1ex]
  \href{mailto:nbeisert@itp.phys.ethz.ch}
  {\texttt{nbeisert@itp.phys.ethz.ch}}}
\hypersetup{pdfauthor={Niklas Beisert}}
\hypersetup{pdfsubject={Manual for the LaTeX2e Package childdoc}}
\date{30 December 2018, \textsf{v2.0}}
\maketitle

\begin{abstract}\noindent
\textsf{childdoc} is a \LaTeXe{} package
that enables the direct compilation
of document sections included by |\include|
to individual files.
\end{abstract}

\begingroup
\parskip0ex
\tableofcontents
\endgroup

%%%%%%%%%%%%%%%%%%%%%%%%%%%%%%%%%%%%%%%%%%%%%%%%%%%%%%%%%%%%%%%%%%%%%%%%%%%%%%%%
%%%%%%%%%%%%%%%%%%%%%%%%%%%%%%%%%%%%%%%%%%%%%%%%%%%%%%%%%%%%%%%%%%%%%%%%%%%%%%%%
\section{Introduction}

\LaTeX{} provides a mechanism to structure a large document (such as a book)
into a main file and several child files (containing the chapters)
using the |\include| command.
This mechanism is beneficial for documents
which span hundreds of pages in order to
make the source file(s) more manageable.
Moreover, compilation can be restricted to
selected child files by means of the |\includeonly| command.
The latter feature can be used to reduce the compilation time while editing
(this was significantly more useful in the earlier days of \LaTeX{})
or to generate a smaller document which is easier to navigate.
Another application of |\includeonly| is to generate
documents consisting of selected parts of the complete document.

However, there are a few drawbacks of the plain |\include| mechanism:
\begin{itemize}
\item
The child files cannot be compiled on their own,
they can only be compiled via the main file.
A naive editing environment
(such as a text editor with an option
to have the current file processed by \LaTeX)
may require one to switch to the main file before compiling;
attempting to compile the child file produces errors.
\item
The main file must be modified (each time)
to adjust the |\includeonly| command
to the present needs. This easily leaves the main file in a messy state.
\item
The generated document will always carry the filename
of the main document. This is inconvenient if
several child files are to be compiled and
to be kept for distribution.
\end{itemize}

The present package provides a simple interface
to make child files individually compilable by \LaTeX{}.
Compiling a child file then has the same effect as compiling
the main file with an |\includeonly| command
to select the appropriate child.
Moreover the generated document will carry the name of the child
rather than the main file.
This resolves all three above issues.

This feature is meant to make the editing of books,
thesis documents and lecture notes somewhat more convenient.
However, the package can also be used efficiently for
composing a series of documents (such as exercise sheets)
which are typically distributed individually.
It then assists the author in generating the individual documents
(potentially in different versions)
as well as a document containing the collected series.
Another application is in developing style files
or other kinds of included material
where compilation of the style file could redirect
to a sample or test file.

%%%%%%%%%%%%%%%%%%%%%%%%%%%%%%%%%%%%%%%%%%%%%%%%%%%%%%%%%%%%%%%%%%%%%%%%%%%%%%%%
%%%%%%%%%%%%%%%%%%%%%%%%%%%%%%%%%%%%%%%%%%%%%%%%%%%%%%%%%%%%%%%%%%%%%%%%%%%%%%%%
\section{Usage}

First of all, the package \textsf{childdoc} is \emph{not} a standard
\LaTeXe{} |.sty| style file! Therefore it needs to be invoked in
a non-standard way.

%%%%%%%%%%%%%%%%%%%%%%%%%%%%%%%%%%%%%%%%%%%%%%%%%%%%%%%%%%%%%%%%%%%%%%%%%%%%%%%%
\subsection{Included Files}
\label{sec:include}

%%%%%%%%%%%%%%%%%%%%%%%%%%%%%%%%%%%%%%%%
\DescribeMacro{\childdocmain}
To use the package, add the commands
\begin{center}
\begin{tabular}{l}
|\input{childdoc.def}|\\
|\childdocmain{}|\\
\end{tabular}
\end{center}
at the very top of the main \LaTeX{} file,
in particular \emph{before} the |\documentclass| statement!
The argument of |\childdocmain| should be left empty
(but it must be present).

%%%%%%%%%%%%%%%%%%%%%%%%%%%%%%%%%%%%%%%%
\DescribeMacro{\childdocof}
Furthermore, add the commands
\begin{center}
\begin{tabular}{l}
|\input{childdoc.def}|\\
|\childdocof{|\textit{main}|}|\\
\end{tabular}
\end{center}
at the top of every child file \textit{child}
which is included by |\include{|\textit{child}|}|
from within the main file
(or at least for those files to be compiled individually).
The argument \textit{main} must be the filename of the main file.

There are a couple of
considerations in setting up the main and child documents:

%%%%%%%%%%%%%%%%%%%%%%%%%%%%%%%%%%%%%%%%
\paragraph{Restrictions.}

Please note the following restrictions:
\begin{itemize}
\item
|\childdocmain| must be called with one argument \textit{main}
to ensure compatibility with earlier version of the package.
It must either be empty (|\childdocmain{}|)
or precisely match the filename of the main file in which it is specified.
See \secref{sec:detection} for further information.
\item
The filename \textit{main} must be specified without the |.tex| extension.
\item
The filename \textit{main} is case sensitive
(even in case-insensitive file systems)
due to internal string comparison.
\item
The argument \textit{main} should be fully expanded, it cannot be a macro.
\item
Subdirectories and special characters should be avoided in filenames.
\item
The command |\childdocmain{|\textit{main}|}| must be followed by a whitespace.
It should not be followed immediately by another command
or by a comment mark `|%|'.
This is because the \TeX{} parser reads the token immediately following
the argument of |\childdocmain| and puts it
at the beginning of every child section;
however, a white\-space is ignored.
\end{itemize}

%%%%%%%%%%%%%%%%%%%%%%%%%%%%%%%%%%%%%%%%
\paragraph{Content of Main File.}

It is advisable to place all content in the child files included by |\include|.
Any output contained in the main file will appear in all child documents
unless suppressed manually;
it cannot be suppressed automatically by the |\includeonly| directive
and thus should normally be avoided.
A method to include some content in the main file
by means of conditional processing is described in \secref{sec:conditional}.

%%%%%%%%%%%%%%%%%%%%%%%%%%%%%%%%%%%%%%%%
\paragraph{Page Numbering.}

When only a part of the document is compiled,
the appropriate numbering of pages
(as well as other status parameters)
is determined from the |.aux| files.
The latter contain information from previous passes.
However this information needs to propagate through
all intermediate child documents.
Therefore the page numbering in child documents may well
be inconsistent until the complete document is compiled at least once.

A useful (if unconventional) way to always ensure a consistent
page numbering is to restart the numbering in each child document
and denote the pages by `\textit{child}|.|\textit{page}'
where \textit{child} represents the chapter/section number of the child file.
This can be achieved by the command
|\numberwithin{page}{|\textit{child}|}|
of the \textsf{amsmath} package
where \textit{child} can be |chapter| or |section|
depending on the chosen structuring.
Alternatively, one can modify the macro |\thepage| appropriately
and reset the counter |page| at the start of each child file.

%%%%%%%%%%%%%%%%%%%%%%%%%%%%%%%%%%%%%%%%%%%%%%%%%%%%%%%%%%%%%%%%%%%%%%%%%%%%%%%%
\subsection{Conditional Processing}
\label{sec:conditional}

The package provides a mechanism to compile different versions
of a document. To customise the versions further some conditional processing
can come in handy to distinguish which version is being compiled.
The package provides two macros to describe the compilation context:

%%%%%%%%%%%%%%%%%%%%%%%%%%%%%%%%%%%%%%%%
\DescribeMacro{\ifchilddoc}
The conditional |\ifchilddoc| distinguishes between the compilation of
child documents and the main document:
%
\begin{center}
|\ifchilddoc |\textit{child-code}| |[|\||else |\textit{main-code}]| \||fi|
\end{center}

%%%%%%%%%%%%%%%%%%%%%%%%%%%%%%%%%%%%%%%%
\DescribeMacro{\childdocname}
\DescribeMacro{\childdocjob}
The macro |\childdocname| contains the filename (without extension)
of the main or child file being processed.
Note that |\childdocjob| will always contain the name of the main file.

%%%%%%%%%%%%%%%%%%%%%%%%%%%%%%%%%%%%%%%%
\paragraph{Title Page.}

Conditional processing can be used to include a title or banner page
in the main document when proper precautions are taken.
Importantly, the code in the main file should ensure that the page counter
(as well as other status parameters which are stored in the |.aux| files)
takes the same value after the conditional processing.
Otherwise the page numbers may take divergent values
depending on which part is compiled.

For example, a title page could be declared by:
%
\begin{center}
\begin{tabular}{l}
|\ifchilddoc\||else|\\
|\addtocounter{page}{-1}|\\
\textit{code for title page}\\
|\newpage|\\
|\||fi|
\end{tabular}
\end{center}
%
A banner page for the child documents can be generated by:
%
\begin{center}
\begin{tabular}{l}
|\ifchilddoc|\\
|\addtocounter{page}{-1}|\\
\textit{code for banner page}\\
|\newpage|\\
|\||fi|
\end{tabular}
\end{center}
%
Here one could write a message such as:
\begin{center}
|This is the part \childdocname{} of \childdocjob{}.|
\end{center}

%%%%%%%%%%%%%%%%%%%%%%%%%%%%%%%%%%%%%%%%%%%%%%%%%%%%%%%%%%%%%%%%%%%%%%%%%%%%%%%%
\subsection{Flags}
\label{sec:flags}

The package makes it easy to generate different versions
of the main or child documents.
To this end compilation flags can be defined
and assigned different default values.
They will be particularly useful in conjunction
with the forwarding mechanism described in \secref{sec:forward}.

For example, it may be useful to have a flag |\version|
which can be set to |draft| or |final|.
The document source will contain some conditional code
depending on the value of |\version|.
Suppose further, the flag should default to |final| for the main file
and to |draft| for child files
which is a natural assignment for editing the document.
This is achieved by placing the following code
in the preamble of the main document
(below the |\childdocmain| directive):
%
\begin{center}
\begin{tabular}{l}
|\ifchilddoc|\\
|\providecommand{\version}{draft}|\\
|\||else|\\
|\providecommand{\version}{final}|\\
|\||fi|
\end{tabular}
\end{center}
%
The definition by |\providecommand| makes sure
that previous definitions are not overwritten.
Further statements |\providecommand{\version}{...}|
can thus be added before the above code to override it.

For the main file, one might add a line
(between |\childdocmain| and the above block)
%
\begin{center}
|%\ifchilddoc\||else\providecommand{\version}{draft}\||fi|
\end{center}
%
which can be uncommented to produce a draft version.
Likewise one can add a line to the very top of a child file
(above the |\childdocof{|\textit{main}|}| directive)
%
\begin{center}
|%\providecommand{\version}{final}|
\end{center}
%
which can be uncommented to produce the final version of this child document.

%%%%%%%%%%%%%%%%%%%%%%%%%%%%%%%%%%%%%%%%%%%%%%%%%%%%%%%%%%%%%%%%%%%%%%%%%%%%%%%%
\subsection{Forwarding}
\label{sec:forward}

Different versions of the main or child documents
using compilation flags as described in \secref{sec:flags}
can be (permanently) stored in different files
for convenient compilation, viewing and distribution.
To this end, the package defines a command
to pass on compilation to a different file:

%%%%%%%%%%%%%%%%%%%%%%%%%%%%%%%%%%%%%%%%
\DescribeMacro{\childdocforward}
The command |\childdocforward| redirects processing to
another source file:
%
\begin{center}
\begin{tabular}{l}
|\input{childdoc.def}|\\
|\childdocforward[|\textit{main}|]{|\textit{dest}|}|\\
\end{tabular}
\end{center}
%
The argument \textit{dest} is the destination file
(without extension).
It should be the main file or one of the child files.
Note that further \textsf{childdoc} directives
such as |\childdocof| and |\childdocforward|
in the indicated file will be processed in this form.
The optional argument \textit{main}
passes on directly to the main file \textit{main}
while pretending to compile the child \textit{dest}.
This form behaves as if \textit{dest}
issues |\childdocof{|\textit{main}|}| right away,
and no further \textsf{childdoc} directives will be processed.

%%%%%%%%%%%%%%%%%%%%%%%%%%%%%%%%%%%%%%%%
\DescribeMacro{\...prefix}
In the alternative form |\childdocforwardprefix|,
%
\begin{center}
\begin{tabular}{l}
|\input{childdoc.def}|\\
|\childdocforwardprefix[|\textit{main}|]{|\textit{prefix}|}{|\textit{dest}|}|
\end{tabular}
\end{center}
%
the destination file is determined by a pattern
depending on the current file:
To make this work, the current file must be called
`{\textit{prefix}\hspace{0.2em}\textit{suffix}}'
with \textit{prefix} matching precisely the argument.
Processing is then passed on to the file
`{\textit{dest}\hspace{0.2em}\textit{suffix}}'.
Surely, the same effect is achieved by
directly specifying the
argument `{\textit{dest}\hspace{0.2em}\textit{suffix}}'
in the first form.
However, that requires to set up a different file
for each child. With the alternative form of the command
all these files can have exactly the same content
which simplifies setting them up and maintaining them.

For example, the following file |draft.tex|
with a compilation flag |\version| as described in \secref{sec:flags}
compiles the main document as a draft:
%
\begin{center}
\begin{tabular}{l}
|\def\version{draft}|\\
|\input{childdoc.def}|\\
|\childdocforward{|\textit{main}|}|
\end{tabular}
\end{center}
%
Likewise, the following files |final|\textit{nn}|.tex|
compile the final version of the child document
|child|\textit{nn}|.tex|:
%
\begin{center}
\begin{tabular}{l}
|\def\version{final}|\\
|\input{childdoc.def}|\\
|\childdocforwardprefix{final}{child}|
\end{tabular}
\end{center}
%

Note that when several versions of a main file and/or of each child file
are to be generated, it may be convenient to set up a |Makefile| or
shell script to automatise the process.

%%%%%%%%%%%%%%%%%%%%%%%%%%%%%%%%%%%%%%%%%%%%%%%%%%%%%%%%%%%%%%%%%%%%%%%%%%%%%%%%
\subsection{Command Line Processing}
\label{sec:commandline}

The effect of redirection files can also be achieved by invoking
the \LaTeX{} compiler with a more elaborate command line.
Most conveniently this should be done as part
of a shell script or a |Makefile|.

When using \textsf{childdoc} in the main file, the following
command lines effectively perform a redirection
(note that depending on the shell being used,
backslashes may have to be doubled: `|\|' $\to$ `|\\|'):
%
\begin{center}
|... -jobname "|\textit{target}|" |\\|"|[\textit{flags}]%
|\input{childdoc.def}\childdocforward[|\textit{main}|]{|\textit{dest}|}"|
\end{center}
%
Here \textit{target} is the name of the output file,
\textit{main} is the name of the main file
and \textit{dest} is the name of the main or child file to be processed
(all filenames without extensions).
The optional argument \textit{main} can be omitted
if \textit{main} matches \textit{dest}.
Optionally, compilation \textit{flags} can be defined via |\def| commands.
This command line makes the \TeX{} engine believe
it is compiling the file \textit{target}
whose content is specified as the latter parameter.
The provided code then forwards the processing to
\textit{main} or \textit{dest} as described in \secref{sec:forward}.

%%%%%%%%%%%%%%%%%%%%%%%%%%%%%%%%%%%%%%%%%%%%%%%%%%%%%%%%%%%%%%%%%%%%%%%%%%%%%%%%
\subsection{Include by Input}
\label{sec:input}

Including child documents by |\include| has some restrictions by design.
Most notably, the content of a child document always occupies
its own set of pages; pages cannot be shared between child documents.
Usually, this behaviour makes perfect sense
because each child document contain an essential part of the document.
However, in some situations it may be desirable to compose
a document from a collection of parts
without having mandatory page breaks between then.
For this case, the package
provides a mechanism to include parts
by |\input| which can also be processed individually.
However, by construction this mechanism
requires manual handling of the content to be output.

%%%%%%%%%%%%%%%%%%%%%%%%%%%%%%%%%%%%%%%%
\DescribeMacro{\ifchilddocmanual}
The main file should be prepared as usual, see \secref{sec:include}.
However, the document body must make a distinction
between processing of an individual part and of the main document, e.g.:
%
\begin{center}
\begin{tabular}{l}
|\ifchilddocmanual|\\
|\input{\childdocname}|\\
|\||else|\\
\textit{document body with }|\input{|\textit{part}|}|\\
|\||fi|
\end{tabular}
\end{center}
%
The conditional |\ifchilddocmanual| is true whenever
a part to be included by |\input| is being compiled,
and the name of the part is stored in |\childdocname|.

%%%%%%%%%%%%%%%%%%%%%%%%%%%%%%%%%%%%%%%%
\DescribeMacro{\childdocby}
Each part to be included by |\input| should start with:
%
\begin{center}
\begin{tabular}{l}
|\input{childdoc.def}|\\
|\childdocby{|\textit{main}|}|\\
\end{tabular}
\end{center}
%
The directive |\childdocby| is similar to |\childdocof|
described in \secref{sec:include},
but the subsequent selection of content must be done manually.
To that end, both |\ifchilddoc| and |\ifchilddocmanual|
will be true upon processing of a part,
and the name of the part is stored in |\childdocname|.
Note that |\jobname| will be set to the filename of the current part
so that each part receives an individual |.aux| file
that does not interfere with the |.aux| file(s) of the main document.
This behaviour can be altered by the alternative form
|\childdocby[*]{|\textit{main}|}| (with a non-empty optional argument)
which uses the |.aux| file of the main document
by setting |\jobname| to \textit{main}.

%%%%%%%%%%%%%%%%%%%%%%%%%%%%%%%%%%%%%%%%%%%%%%%%%%%%%%%%%%%%%%%%%%%%%%%%%%%%%%%%
\subsection{Driver Development}
\label{sec:driver}

The \textsf{childdoc} mechanism can also be use for the development
of definition files such as \LaTeX{} styles or classes.
This case differs from the above setup with multiple parts
included by |\include| in that no |\includeonly| should be invoked.
This can be achieved by starting the include file
(before |\ProvidesPackage|) with:
%
\begin{center}
\begin{tabular}{l}
|\input{childdoc.def}|\\
|\childdocforward{|\textit{main}|}|\\
\end{tabular}
\end{center}
%
or alternatively with:
%
\begin{center}
\begin{tabular}{l}
|\input{childdoc.def}|\\
|\childdocby{|\textit{main}|}|\\
\end{tabular}
\end{center}
%
Both forms have slightly different effects as described above.
The main file is prepared as usual, see \secref{sec:include}.

%%%%%%%%%%%%%%%%%%%%%%%%%%%%%%%%%%%%%%%%%%%%%%%%%%%%%%%%%%%%%%%%%%%%%%%%%%%%%%%%
\subsection{Legacy Detection}
\label{sec:detection}

The directive |\childdocmain| in the main file can detect
whether the complete document or merely a child is to be compiled
even without using the directive |\childdocof|.
This method is deprecated because it is less robust
and there is no compelling reason to use it;
it is merely provided for backward compatibility
and it may be removed in future versions.

If the detection mechanism is to be used,
it is mandatory to correctly specify
the filename of the main file as the argument of |\childdocmain|:
%
\begin{center}
\begin{tabular}{l}
|\input{childdoc.def}|\\
|\childdocmain{|\textit{main}|}|\\
\end{tabular}
\end{center}
%
If |\jobname| does not match the argument \textit{main} of |\childdocmain|,
it is assumed that |\jobname| points to the child file to be compiled.
When using |\childdocmain| with the main file specified as argument,
it suffices to start a child file
with just |\input{|\textit{main}|}|
without loading of the package and using |\childdocof|.
If instead all processing is done
with the appropriate \textsf{childdoc} directives,
the argument of \textit{main} of |\childdocmain| can be empty.

An alternative version of the command line processing described
in \secref{sec:commandline} using the detection mechanism reads:
%
\begin{center}
|... -jobname "|\textit{target}|" "|[\textit{flags}]%
[|\def\jobname{|\textit{dest}|}|]|\input{|\textit{main}|}"|
\end{center}

%%%%%%%%%%%%%%%%%%%%%%%%%%%%%%%%%%%%%%%%%%%%%%%%%%%%%%%%%%%%%%%%%%%%%%%%%%%%%%%%
\subsection{Manual Code}
\label{sec:manual}

In case one cannot be certain whether the definitions file |childdoc.def|
is installed on the target \TeX{} distribution
and one prefers not to ship it,
it is conceivable to paste a few relevant commands into the sources.

To that end, drop all statements |\input{childdoc.def}|
and perform the replacements as outlined below.
Instead of |\childdocmain{|\textit{main}|}| add the following code
to the top of the main file:
%
\begin{center}
\begin{tabular}{l}
|\||ifdefined\childdocname\endinput\||fi\newif\ifchilddoc|\\
|\edef\childdocname{\scantokens\expandafter{\jobname\noexpand}}|\\
|\def\childdocmain{|\textit{main}|}\||ifx\childdocmain\childdocname\||else|\\
|\childdoctrue\includeonly{\childdocname}\let\jobname\childdocmain\||fi|\\
\end{tabular}
\end{center}
%
Instead of |\childdocof{|\textit{main}|}| just include the main file
at the top of each child file:
%
\begin{center}
|\input{|\textit{main}|}|
\end{center}
%
A simple redirection |\childdocforward{|\textit{dest}|}| is achieved by:
%
\begin{center}
|\def\jobname{|\textit{dest}|}\input{\jobname}|
\end{center}
%
The redirection with prefix
|\childdocforwardprefix[|\textit{prefix}|]{|\textit{dest}|}|
is accomplished by:
%
\begin{center}
\begin{tabular}{l}
|{\edef\jobname{\scantokens\expandafter{\jobname\noexpand}}|\\
|\def\redirectjob |\textit{prefix}|#1~~~{\gdef\jobname{|\textit{dest}|#1}}|\\
|\expandafter\redirectjob\jobname~~~}\input{\jobname}|
\end{tabular}
\end{center}

In an alternative approach,
child documents can be compiled by a specific command line
without additional code or specific definitions:
%
\begin{center}
|... -jobname "|\textit{target}|" "|[\textit{flags}]%
|\includeonly{|\textit{dest}|}\input{|\textit{main}|}"|
\end{center}
%

%%%%%%%%%%%%%%%%%%%%%%%%%%%%%%%%%%%%%%%%%%%%%%%%%%%%%%%%%%%%%%%%%%%%%%%%%%%%%%%%
%%%%%%%%%%%%%%%%%%%%%%%%%%%%%%%%%%%%%%%%%%%%%%%%%%%%%%%%%%%%%%%%%%%%%%%%%%%%%%%%
\section{Information}

%%%%%%%%%%%%%%%%%%%%%%%%%%%%%%%%%%%%%%%%%%%%%%%%%%%%%%%%%%%%%%%%%%%%%%%%%%%%%%%%
\subsection{Copyright}

Copyright \copyright{} 2017--2018 Niklas Beisert

This work may be distributed and/or modified under the
conditions of the \LaTeX{} Project Public License, either version 1.3
of this license or (at your option) any later version.
The latest version of this license is in
  \url{http://www.latex-project.org/lppl.txt}
and version 1.3 or later is part of all distributions of \LaTeX{}
version 2005/12/01 or later.

This work has the LPPL maintenance status `maintained'.

The Current Maintainer of this work is Niklas Beisert.

This work consists of the files |README.txt|, |childdoc.ins| and |childdoc.dtx|
as well as the derived files |childdoc.def|, |cdocsamp.tex|
with |cdocsch1.tex|, |cdocsch2.tex|, |cdocspt3.tex|, |cdocspt4.tex|,
|cdocsdrf.tex|, |cdocsfn1.tex|, |cdocsfn2.tex|
as well as |childdoc.pdf|.

%%%%%%%%%%%%%%%%%%%%%%%%%%%%%%%%%%%%%%%%%%%%%%%%%%%%%%%%%%%%%%%%%%%%%%%%%%%%%%%%
\subsection{Files and Installation}

The package consists of the files:
%
\begin{center}
\begin{tabular}{ll}
    |README.txt|   & readme file \\
    |childdoc.ins| & installation file \\
    |childdoc.dtx| & source file \\
    |childdoc.def| & definition file \\
    |cdocsamp.tex| & sample main file \\
    |cdocsch1.tex| & sample include file \\
    |cdocsch2.tex| & sample include file \\
    |cdocspt3.tex| & sample part file \\
    |cdocspt4.tex| & sample part file \\
    |cdocsdrf.tex| & sample redirection file \\
    |cdocsfn1.tex| & sample redirection file \\
    |cdocsfn2.tex| & sample redirection file \\
    |childdoc.pdf| & manual
\end{tabular}
\end{center}
%
The distribution consists of the files
|README.txt|, |childdoc.ins| and |childdoc.dtx|.
%
\begin{itemize}
\item
Run (pdf)\LaTeX{} on |childdoc.dtx|
to compile the manual |childdoc.pdf| (this file).
\item
Run \LaTeX{} on |childdoc.ins| to create the definitions file |childdoc.def|
and the sample |cdocsamp.tex| with include files
|cdocsch1.tex|, |cdocsch2.tex|, |cdocspt3.tex|, |cdocspt4.tex|,
|cdocsdrf.tex|, |cdocsfn1.tex|, |cdocsfn2.tex|.
Then copy the file |childdoc.def| to an appropriate directory of your \LaTeX{}
distribution, e.g.\ \textit{texmf-root}|/tex/latex/childdoc|.
\end{itemize}

%%%%%%%%%%%%%%%%%%%%%%%%%%%%%%%%%%%%%%%%%%%%%%%%%%%%%%%%%%%%%%%%%%%%%%%%%%%%%%%%
\subsection{Related CTAN Packages}

There are several other packages which offer a similar functionality:
%
\begin{itemize}
\item
The packages
\href{http://ctan.org/pkg/docmute}{\textsf{docmute}},
\href{http://ctan.org/pkg/includex}{\textsf{includex}} and
\href{http://ctan.org/pkg/standalone}{\textsf{standalone}}
provide commands to include only the document body of
a child file thus allowing both files to be compiled individually.
\item
The packages \href{http://ctan.org/pkg/subdocs}{\textsf{subdocs}}
and \href{http://ctan.org/pkg/subfiles}{\textsf{subfiles}}
provide structures in which the main and child documents can be
encapsulated and allowing them to be compiled individually.
The inclusion mechanism is different from the conventional |\include|.
\item
The package \href{http://ctan.org/pkg/combine}{\textsf{combine}}
is an elaborate solution to combine several documents into one.
\end{itemize}
%
See also the CTAN topic \href{http://ctan.org/topic/subdocs}{\textsf{subdocs}}
for further related packages.
The present package differs from the above solutions in that
a document structure constructed with the conventional |\include| mechanism
just needs two extra commands at the top of every file
such that all constituent files can be compiled individually.

%%%%%%%%%%%%%%%%%%%%%%%%%%%%%%%%%%%%%%%%%%%%%%%%%%%%%%%%%%%%%%%%%%%%%%%%%%%%%%%%
%\subsection{Feature Suggestions}
%
%The following is a list of features which may be useful for future
%versions of this package:
%%
%\begin{itemize}
%\item
%\ldots
%\end{itemize}

%%%%%%%%%%%%%%%%%%%%%%%%%%%%%%%%%%%%%%%%%%%%%%%%%%%%%%%%%%%%%%%%%%%%%%%%%%%%%%%%
\subsection{Revision History}

%%%%%%%%%%%%%%%%%%%%%%%%%%%%%%%%%%%%%%%%
\paragraph{v2.0:} 2018/12/30

\begin{itemize}
\item
immediate forward processing
\item
added |\childdocby| mechanism
\item
manual restructured
\end{itemize}

%%%%%%%%%%%%%%%%%%%%%%%%%%%%%%%%%%%%%%%%
\paragraph{v1.6:} 2018/01/17

\begin{itemize}
\item
application for development of include files
\item
corrections to manual
\end{itemize}

%%%%%%%%%%%%%%%%%%%%%%%%%%%%%%%%%%%%%%%%
\paragraph{v1.5:} 2017/05/21

\begin{itemize}
\item
more complete structuring introduced
\item
|\childdocof| introduced
\item
|\childdoc| renamed to |\childdocmain|
\item
|\childredirect| renamed to |\childdocforward| and |\childdocforwardprefix|
and functionality expanded
\end{itemize}

%%%%%%%%%%%%%%%%%%%%%%%%%%%%%%%%%%%%%%%%
\paragraph{v1.0:} 2017/04/27

\begin{itemize}
\item
manual and install package
\item
first version published on CTAN
\end{itemize}

%%%%%%%%%%%%%%%%%%%%%%%%%%%%%%%%%%%%%%%%
\paragraph{v0.6:} 2017/04/26

\begin{itemize}
\item
redirection mechanism added
\end{itemize}

%%%%%%%%%%%%%%%%%%%%%%%%%%%%%%%%%%%%%%%%
\paragraph{v0.5:} 2017/04/26

\begin{itemize}
\item
functionality in definition file
\end{itemize}


%%%%%%%%%%%%%%%%%%%%%%%%%%%%%%%%%%%%%%%%%%%%%%%%%%%%%%%%%%%%%%%%%%%%%%%%%%%%%%%%
%%%%%%%%%%%%%%%%%%%%%%%%%%%%%%%%%%%%%%%%%%%%%%%%%%%%%%%%%%%%%%%%%%%%%%%%%%%%%%%%
%%%%%%%%%%%%%%%%%%%%%%%%%%%%%%%%%%%%%%%%%%%%%%%%%%%%%%%%%%%%%%%%%%%%%%%%%%%%%%%%
\appendix

\settowidth\MacroIndent{\rmfamily\scriptsize 000\ }

 \DocInput{childdoc.dtx}

\end{document}
%</driver>
% \fi
%
% %%%%%%%%%%%%%%%%%%%%%%%%%%%%%%%%%%%%%%%%%%%%%%%%%%%%%%%%%%%%%%%%%%%%%%%%%%%%%%
% %%%%%%%%%%%%%%%%%%%%%%%%%%%%%%%%%%%%%%%%%%%%%%%%%%%%%%%%%%%%%%%%%%%%%%%%%%%%%%
% \section{Sample}
%\iffalse
%<*samplemain>
%\fi
%
% The following presents a sample document
% with two chapters, two parts, a title page,
% a compile flag as well as three forwarding files to set the flag.
% It consists of eight |.tex| files:
% \begin{center}
% \begin{tabular}{ll}
% |cdocsamp.tex|&main file\\
% |cdocsch1.tex|&include file for chapter 1\\
% |cdocsch2.tex|&include file for chapter 2\\
% |cdocspt3.tex|&include file for part 3\\
% |cdocspt4.tex|&include file for part 4\\
% |cdocsdrf.tex|&forwarding file for main file in draft mode\\
% |cdocsfi1.tex|&forwarding file for final version of chapter 1\\
% |cdocsfi2.tex|&forwarding file for final version of chapter 2\\
% \end{tabular}
% \end{center}
% Each of the eight files can be compiled directly by the \LaTeX{} compiler.
%
% %%%%%%%%%%%%%%%%%%%%%%%%%%%%%%%%%%%%%%
% \paragraph{Main File.}
%
% The main file is called |cdocsamp.tex|.
%
% Load the \textsf{childdoc} definitions and
% declare the filename for the main document:
%    \begin{macrocode}
\input{childdoc.def}
\childdocmain{}
%    \end{macrocode}

% Optional override for |\version| flag:
%    \begin{macrocode}
%%\ifchilddoc\else\providecommand{\version}{draft}\fi
%    \end{macrocode}

% Define the default values for the |\version| flag
% (|final| for the main file and |draft| for childs):
%    \begin{macrocode}
\ifchilddoc
\providecommand{\version}{draft}
\else
\providecommand{\version}{final}
\fi
%    \end{macrocode}

% Load the standard document class:
%    \begin{macrocode}
\documentclass[12pt]{article}
%    \end{macrocode}

% Start the document body:
%    \begin{macrocode}
\begin{document}
%    \end{macrocode}

% Declare a title page.
% Print title, part of document being processed and version flag:
%    \begin{macrocode}
\addtocounter{page}{-1}
\begin{center}
{\LARGE\bfseries{}childdoc example\par}
\vspace{1cm}
\ifchilddoc
\ifchilddocmanual part\else chapter\fi:
`\childdocname' of `\childdocjob'\par
\else
main document: `\childdocjob'\par
\fi
version: \version\par
\end{center}
\newpage
%    \end{macrocode}

% Manually include selected file,
% otherwise process as usual:
%    \begin{macrocode}
\ifchilddocmanual
\section*{part `\childdocname'}
\input{\childdocname}
\else
%    \end{macrocode}

% Include the two chapters:
%    \begin{macrocode}
\include{cdocsch1}
\include{cdocsch2}
%    \end{macrocode}

% Include the two parts unless only chapters should be displayed:
%    \begin{macrocode}
\ifchilddoc\else
\section{part three}
\input{cdocspt3}
\section{part four}
\input{cdocspt4}
\fi
%    \end{macrocode}

% Process as usual until here:
%    \begin{macrocode}
\fi
%    \end{macrocode}

% End of document body:
%    \begin{macrocode}
\end{document}
%    \end{macrocode}
%\iffalse
%</samplemain>
%\fi
%
% %%%%%%%%%%%%%%%%%%%%%%%%%%%%%%%%%%%%%%
% \paragraph{Chapter Include Files.}
%
% The include files are called |cdocsch1.tex| and |cdocsch2.tex|.
%
%\iffalse
%<*samplechap1|samplechap2>
%\fi

% Optional override for |\version| flag:
%    \begin{macrocode}
%%\providecommand{\version}{final}
%    \end{macrocode}

% Include the main document:
%    \begin{macrocode}
\input{childdoc.def}
\childdocof{cdocsamp}
%    \end{macrocode}

%\iffalse
%</samplechap1|samplechap2>
%\fi
%
%\iffalse
%<*samplechap1>
%\fi
% Some text for chapter 1:
%    \begin{macrocode}
\section{one}
some text in chapter one
%    \end{macrocode}

%\iffalse
%</samplechap1>
%\fi
% Some text for chapter 2:
%\iffalse
%<*samplechap2>
%\fi
%    \begin{macrocode}
\section{two}
more text in chapter two
%    \end{macrocode}

%\iffalse
%</samplechap2>
%\fi
%
% %%%%%%%%%%%%%%%%%%%%%%%%%%%%%%%%%%%%%%
% \paragraph{Part Include Files.}
%
% The include files are called |cdocspt3.tex| and |cdocspt4.tex|.
%
%\iffalse
%<*samplepart3|samplepart4>
%\fi

% Optional override for |\version| flag:
%    \begin{macrocode}
%%\providecommand{\version}{final}
%    \end{macrocode}

% Include the main document:
%    \begin{macrocode}
\input{childdoc.def}
\childdocby{cdocsamp}
%    \end{macrocode}

%\iffalse
%</samplepart3|samplepart4>
%\fi
%
%\iffalse
%<*samplepart3>
%\fi
% Some text for part 3:
%    \begin{macrocode}
some text in part three
%    \end{macrocode}

%\iffalse
%</samplepart3>
%\fi
% Some text for part 4:
%\iffalse
%<*samplepart4>
%\fi
%    \begin{macrocode}
more text in part four
%    \end{macrocode}

%\iffalse
%</samplepart4>
%\fi
%
% %%%%%%%%%%%%%%%%%%%%%%%%%%%%%%%%%%%%%%
% \paragraph{Forwarding for a Complete Draft.}
%
% The following forwarding file |cdocsdrf.tex|
% compiles the main document in draft mode:
%\iffalse
%<*sampledraft>
%\fi
%    \begin{macrocode}
\def\version{draft}
\input{childdoc.def}
\childdocforward{cdocsamp}
%    \end{macrocode}

%\iffalse
%</sampledraft>
%\fi
%
% %%%%%%%%%%%%%%%%%%%%%%%%%%%%%%%%%%%%%%
% \paragraph{Forwarding for Final Version of the Chapters.}
%
% The following forwarding files |cdocsfn1.tex| and |cdocsfn2.tex|
% (with identical content)
% compile the final versions of the child documents
% |cdocsch1.tex| and |cdocsch2.tex|, respectively:
%\iffalse
%<*samplefinal>
%\fi
%    \begin{macrocode}
\def\version{final}
\input{childdoc.def}
\childdocforwardprefix[cdocsamp]{cdocsfn}{cdocsch}
%    \end{macrocode}

%\iffalse
%</samplefinal>
%\fi
%
% %%%%%%%%%%%%%%%%%%%%%%%%%%%%%%%%%%%%%%
% \paragraph{Command Line Processing.}
%
% The following three command lines generate the output files
% |cdocscld|, |cdocscl1| and |cdocscl2|
% which should be identical to
% |cdocsdrf|, |cdocsch1| and |cdocsfn2|, respectively:
% \begin{center}
% \begin{tabular}{l}
% |latex -jobname cdocscld \|\\
% |  "\def\version{draft}\input{childdoc.def}\childdocforward{cdocsamp}"|\\
% |latex -jobname cdocscl1 \|\\
% |  "\input{childdoc.def}\childdocforward[cdocsamp]{cdocsch1}"|\\
% |latex -jobname cdocscl2 \|\\
% |  "\def\version{final}\input{childdoc.def}\childdocforward{cdocsch2}"|
% \end{tabular}
% \end{center}
% Note that the trailing backslash on each first line
% merely continues the input to the second line
% (for convenient cut ant paste).
% Furthermore, the command |latex| can be replaced by any
% of its alternative versions such as |pdflatex|.
%
% %%%%%%%%%%%%%%%%%%%%%%%%%%%%%%%%%%%%%%%%%%%%%%%%%%%%%%%%%%%%%%%%%%%%%%%%%%%%%%
% %%%%%%%%%%%%%%%%%%%%%%%%%%%%%%%%%%%%%%%%%%%%%%%%%%%%%%%%%%%%%%%%%%%%%%%%%%%%%%
% \section{Implementation}
%\iffalse
%<*package>
%\fi
%
% This section describes the definitions file |childdoc.def|.

% The definitions cannot be loaded using |\usepackage| or |\RequirePackage|
% which has a mechanism to prevent loading a style file more than once.
% When loading the definitions by means of |\input|
% multiple instances have to be prevented manually:
%\iffalse
%This code needs to be before the `\ProvidesFile' directive
%which is defined at the beginning of this file.
%Therefore it is also placed there and commented out here.
%</package>
%<*discard>
%\fi
%    \begin{macrocode}
\ifdefined\childdocmain\endinput\fi
%    \end{macrocode}
%\iffalse
%</discard>
%<*package>
%\fi
%
% \macro{\ifchilddoc}
% \macro{\ifchilddocmanual}
% The conditional |\ifchilddoc| tells whether a
% child (true) or main (false) document is being compiled.
% The conditional |\ifchilddocmanual| tells whether
% the |\includeonly| mechanism is used (false) or
% the selection of child files must be performed manually (true).
% The definitions initialise to false:
%    \begin{macrocode}
\newif\ifchilddoc
\newif\ifchilddocmanual
%    \end{macrocode}

% \macro{\childdocname}
% \macro{\childdocjob}
% The macro |\childdocname| stores the name of the main document
% to be compiled. The macro |\childdocjob| stores the name of
% the document on which the \LaTeX{} compiler was originally invoked.
% The content of |\jobname| cannot be compared
% to filenames specified in the source due to different catcodes.
% The following code rescans |\jobname|, stores the result
% in |\childdocname| and saves a copy in |\childdocjob|:
%    \begin{macrocode}
\edef\childdocname{\scantokens\expandafter{\jobname\noexpand}}
\let\childdocjob\childdocname
%    \end{macrocode}

% \macro{\childdocdisable}
% The macro |\childdocdisable| prevents the main file
% from being processed more than once.
% At this stage, the main document command |\childdocmain|
% is assumed to be called once again where it should do nothing.
% Any subsequent call to it should prevent
% a secondary processing of the main document
% It overwrites the forwarding commands
% |\childdocof| and |\childdocforward|
% with empty macros to prevent further inclusions of the main document:
%    \begin{macrocode}
\newcommand{\childdocdisable}
{
  \renewcommand{\childdocmain}[1]{\renewcommand{\childdocmain}[1]{\endinput}}
  \renewcommand{\childdocof}[1]{}
  \renewcommand{\childdocby}[2][]{}
  \renewcommand{\childdocforward}[2][]{}
  \renewcommand{\childdocdisable}{}
}
%    \end{macrocode}

% \macro{\childdocmain}
% The macro |\childdocmain| is to be called at the top of the main file
% with nothing or the main filename (without extension) as argument.
% First, it breaks loops.
% If the argument is not empty and does not match |\childdocname|
% (which is set by the first inclusion of |childdoc.def|),
% |\ifchilddoc| is set to true, |\includeonly| is applied to the child file
% and |\jobname| is set to the main file
% (for proper handling of |.aux| files):
%    \begin{macrocode}
\newcommand{\childdocmain}[1]
{
  \childdocdisable\childdocmain{}
  \if?#1?\else
    \begingroup
      \def\childdoctmp{#1}
      \ifx\childdoctmp\childdocname
        \def\childdoctmp{}
      \else
        \def\childdoctmp
        {
          \childdoctrue
          \includeonly{\childdocname}
          \def\childdocjob{#1}
          \def\jobname{#1}
        }
      \fi
      \expandafter
    \endgroup
    \childdoctmp
  \fi
}
%    \end{macrocode}

% \macro{\childdocof}
% The command |\childdocof| redirects
% compilation to the main file |#1|.
%    \begin{macrocode}
\newcommand{\childdocof}[1]
{
  \childdocdisable
  \childdoctrue
  \includeonly{\childdocname}
  \def\jobname{#1}
  \def\childdocjob{#1}
  \input{#1}
}
%    \end{macrocode}

% \macro{\childdocby}
% The command |\childdocby| ....
%    \begin{macrocode}
\newcommand{\childdocby}[2][]
{
  \childdocdisable
  \childdoctrue
  \childdocmanualtrue
  \if?#1?\else
    \def\jobname{#2}
  \fi
  \def\childdocjob{#2}
  \input{#2}
  \endinput
}
%    \end{macrocode}

% \macro{\childdocforward}
% The command |\childdocforward| redirects
% compilation to the main file or
% (if the optional argument is given) a child file.
% Parameters are set as if the main file
% or a child file starting with |\childdocof| was compiled.
% Then compilation is handed over to the main file:
%    \begin{macrocode}
\newcommand{\childdocforward}[2][]
{
  \begingroup
    \if?#1?
      \def\childdoctmp
      {
        \def\childdocname{#2}
        \def\childdocjob{#2}
        \def\jobname{#2}
        \input{#2}
        \endinput
      }
    \else
      \def\childdoctmp
      {
        \childdocdisable
        \def\childdocname{#2}
        \childdoctrue
        \includeonly{#2}
        \def\childdocjob{#1}
        \def\jobname{#1}
        \input{#1}
        \endinput
      }
    \fi
    \expandafter
  \endgroup
  \childdoctmp
}
%    \end{macrocode}

% \macro{\childdocforwardprefix}
% The command |\childdocforwardprefix| redirects
% compilation to the main or a child file by means of a pattern.
% The prefix |#1| in the current filename is replaced by |#2|
% and the suffix of the current filename is kept
% (it is assumed that the filename does not contain the substring `|~~~|'
% which is used as a delimiter).
% Compilation is handed over to the new file by |\childdocforward|:
%    \begin{macrocode}
\newcommand{\childdocforwardprefix}[3][]
{
  \begingroup
    \def\childdocextract #2##1~~~{\def\childdoctmp{\childdocforward[#1]{#3##1}}}
    \expandafter\childdocextract\childdocname~~~
    \expandafter
  \endgroup
  \childdoctmp
}
%    \end{macrocode}

% \macro{\childdoc}
% The deprecated macro |\childdoc| is a legacy version of |\childdocmain|:
%    \begin{macrocode}
\newcommand{\childdoc}{\childdocmain}
%    \end{macrocode}

% \macro{\childdocredirect}
% The deprecated macro |\childdocredirect| is a legacy version
% of |\childdocforward| and |\childdocforwardprefix|:
%    \begin{macrocode}
\newcommand{\childdocredirect}[2][]
{
  \begingroup
    \if?#1?
      \def\childdoctmp{\childdocforward{#2}}
    \else
      \def\childdoctmp{\childdocforwardprefix{#1}{#2}}
    \fi
    \expandafter
  \endgroup
  \childdoctmp
}
%    \end{macrocode}

%\iffalse
%</package>
%\fi
%
\endinput

\childdocforwardprefix[cdocsamp]{cdocsfn}{cdocsch}
%    \end{macrocode}

%\iffalse
%</samplefinal>
%\fi
%
% %%%%%%%%%%%%%%%%%%%%%%%%%%%%%%%%%%%%%%
% \paragraph{Command Line Processing.}
%
% The following three command lines generate the output files
% |cdocscld|, |cdocscl1| and |cdocscl2|
% which should be identical to
% |cdocsdrf|, |cdocsch1| and |cdocsfn2|, respectively:
% \begin{center}
% \begin{tabular}{l}
% |latex -jobname cdocscld \|\\
% |  "\def\version{draft}% \iffalse
%
% childdoc.dtx Copyright (C) 2017-2018 Niklas Beisert
%
% This work may be distributed and/or modified under the
% conditions of the LaTeX Project Public License, either version 1.3
% of this license or (at your option) any later version.
% The latest version of this license is in
%   http://www.latex-project.org/lppl.txt
% and version 1.3 or later is part of all distributions of LaTeX
% version 2005/12/01 or later.
%
% This work has the LPPL maintenance status `maintained'.
%
% The Current Maintainer of this work is Niklas Beisert.
%
% This work consists of the files childdoc.dtx and childdoc.ins
% and the derived files childdoc.def and cdocsamp.tex with
% cdocsch1.tex, cdocsch2.tex, cdocsdrf.tex, cdocsfn1.tex, cdocsfn2.tex.
%
%<package>\ifdefined\childdocmain\endinput\fi
%<package>\ProvidesFile{childdoc.def}[2018/12/30 v2.0 child document driver]
%<samplemain>\ProvidesFile{cdocsamp.tex}[2018/12/30 v2.0 sample for childdoc]
%<*driver>
%\ProvidesFile{childdoc.drv}[2018/12/30 v2.0 childdoc reference manual file]
\PassOptionsToClass{10pt,a4paper}{article}
\documentclass{ltxdoc}

\usepackage[margin=35mm]{geometry}
\usepackage{hyperref}
\usepackage{hyperxmp}
\usepackage[usenames]{color}

\hypersetup{colorlinks=true}
\hypersetup{pdfstartview=FitH}
\hypersetup{pdfpagemode=UseNone}
\hypersetup{pdfsource={}}
\hypersetup{pdflang={en-UK}}
\hypersetup{pdfcopyright={Copyright 2017-2018 Niklas Beisert.
  This work may be distributed and/or modified under the
  conditions of the LaTeX Project Public License, either version 1.3
  of this license or (at your option) any later version.}}
\hypersetup{pdflicenseurl={http://www.latex-project.org/lppl.txt}}
\hypersetup{pdfcontactaddress={ETH Zurich, ITP, HIT K,
  Wolfgang-Pauli-Strasse 27}}
\hypersetup{pdfcontactpostcode={8093}}
\hypersetup{pdfcontactcity={Zurich}}
\hypersetup{pdfcontactcountry={Switzerland}}
\hypersetup{pdfcontactemail={nbeisert@itp.phys.ethz.ch}}
\hypersetup{pdfcontacturl={http://people.phys.ethz.ch/\xmptilde nbeisert/}}

\newcommand{\secref}[1]{\hyperref[#1]{section \ref*{#1}}}

\parskip1ex
\parindent0pt
\let\olditemize\itemize
\def\itemize{\olditemize\parskip0pt}

\begin{document}

\title{The \textsf{childdoc} Package}
\hypersetup{pdftitle={The childdoc Package}}
\author{Niklas Beisert\\[2ex]
  Institut f\"ur Theoretische Physik\\
  Eidgen\"ossische Technische Hochschule Z\"urich\\
  Wolfgang-Pauli-Strasse 27, 8093 Z\"urich, Switzerland\\[1ex]
  \href{mailto:nbeisert@itp.phys.ethz.ch}
  {\texttt{nbeisert@itp.phys.ethz.ch}}}
\hypersetup{pdfauthor={Niklas Beisert}}
\hypersetup{pdfsubject={Manual for the LaTeX2e Package childdoc}}
\date{30 December 2018, \textsf{v2.0}}
\maketitle

\begin{abstract}\noindent
\textsf{childdoc} is a \LaTeXe{} package
that enables the direct compilation
of document sections included by |\include|
to individual files.
\end{abstract}

\begingroup
\parskip0ex
\tableofcontents
\endgroup

%%%%%%%%%%%%%%%%%%%%%%%%%%%%%%%%%%%%%%%%%%%%%%%%%%%%%%%%%%%%%%%%%%%%%%%%%%%%%%%%
%%%%%%%%%%%%%%%%%%%%%%%%%%%%%%%%%%%%%%%%%%%%%%%%%%%%%%%%%%%%%%%%%%%%%%%%%%%%%%%%
\section{Introduction}

\LaTeX{} provides a mechanism to structure a large document (such as a book)
into a main file and several child files (containing the chapters)
using the |\include| command.
This mechanism is beneficial for documents
which span hundreds of pages in order to
make the source file(s) more manageable.
Moreover, compilation can be restricted to
selected child files by means of the |\includeonly| command.
The latter feature can be used to reduce the compilation time while editing
(this was significantly more useful in the earlier days of \LaTeX{})
or to generate a smaller document which is easier to navigate.
Another application of |\includeonly| is to generate
documents consisting of selected parts of the complete document.

However, there are a few drawbacks of the plain |\include| mechanism:
\begin{itemize}
\item
The child files cannot be compiled on their own,
they can only be compiled via the main file.
A naive editing environment
(such as a text editor with an option
to have the current file processed by \LaTeX)
may require one to switch to the main file before compiling;
attempting to compile the child file produces errors.
\item
The main file must be modified (each time)
to adjust the |\includeonly| command
to the present needs. This easily leaves the main file in a messy state.
\item
The generated document will always carry the filename
of the main document. This is inconvenient if
several child files are to be compiled and
to be kept for distribution.
\end{itemize}

The present package provides a simple interface
to make child files individually compilable by \LaTeX{}.
Compiling a child file then has the same effect as compiling
the main file with an |\includeonly| command
to select the appropriate child.
Moreover the generated document will carry the name of the child
rather than the main file.
This resolves all three above issues.

This feature is meant to make the editing of books,
thesis documents and lecture notes somewhat more convenient.
However, the package can also be used efficiently for
composing a series of documents (such as exercise sheets)
which are typically distributed individually.
It then assists the author in generating the individual documents
(potentially in different versions)
as well as a document containing the collected series.
Another application is in developing style files
or other kinds of included material
where compilation of the style file could redirect
to a sample or test file.

%%%%%%%%%%%%%%%%%%%%%%%%%%%%%%%%%%%%%%%%%%%%%%%%%%%%%%%%%%%%%%%%%%%%%%%%%%%%%%%%
%%%%%%%%%%%%%%%%%%%%%%%%%%%%%%%%%%%%%%%%%%%%%%%%%%%%%%%%%%%%%%%%%%%%%%%%%%%%%%%%
\section{Usage}

First of all, the package \textsf{childdoc} is \emph{not} a standard
\LaTeXe{} |.sty| style file! Therefore it needs to be invoked in
a non-standard way.

%%%%%%%%%%%%%%%%%%%%%%%%%%%%%%%%%%%%%%%%%%%%%%%%%%%%%%%%%%%%%%%%%%%%%%%%%%%%%%%%
\subsection{Included Files}
\label{sec:include}

%%%%%%%%%%%%%%%%%%%%%%%%%%%%%%%%%%%%%%%%
\DescribeMacro{\childdocmain}
To use the package, add the commands
\begin{center}
\begin{tabular}{l}
|\input{childdoc.def}|\\
|\childdocmain{}|\\
\end{tabular}
\end{center}
at the very top of the main \LaTeX{} file,
in particular \emph{before} the |\documentclass| statement!
The argument of |\childdocmain| should be left empty
(but it must be present).

%%%%%%%%%%%%%%%%%%%%%%%%%%%%%%%%%%%%%%%%
\DescribeMacro{\childdocof}
Furthermore, add the commands
\begin{center}
\begin{tabular}{l}
|\input{childdoc.def}|\\
|\childdocof{|\textit{main}|}|\\
\end{tabular}
\end{center}
at the top of every child file \textit{child}
which is included by |\include{|\textit{child}|}|
from within the main file
(or at least for those files to be compiled individually).
The argument \textit{main} must be the filename of the main file.

There are a couple of
considerations in setting up the main and child documents:

%%%%%%%%%%%%%%%%%%%%%%%%%%%%%%%%%%%%%%%%
\paragraph{Restrictions.}

Please note the following restrictions:
\begin{itemize}
\item
|\childdocmain| must be called with one argument \textit{main}
to ensure compatibility with earlier version of the package.
It must either be empty (|\childdocmain{}|)
or precisely match the filename of the main file in which it is specified.
See \secref{sec:detection} for further information.
\item
The filename \textit{main} must be specified without the |.tex| extension.
\item
The filename \textit{main} is case sensitive
(even in case-insensitive file systems)
due to internal string comparison.
\item
The argument \textit{main} should be fully expanded, it cannot be a macro.
\item
Subdirectories and special characters should be avoided in filenames.
\item
The command |\childdocmain{|\textit{main}|}| must be followed by a whitespace.
It should not be followed immediately by another command
or by a comment mark `|%|'.
This is because the \TeX{} parser reads the token immediately following
the argument of |\childdocmain| and puts it
at the beginning of every child section;
however, a white\-space is ignored.
\end{itemize}

%%%%%%%%%%%%%%%%%%%%%%%%%%%%%%%%%%%%%%%%
\paragraph{Content of Main File.}

It is advisable to place all content in the child files included by |\include|.
Any output contained in the main file will appear in all child documents
unless suppressed manually;
it cannot be suppressed automatically by the |\includeonly| directive
and thus should normally be avoided.
A method to include some content in the main file
by means of conditional processing is described in \secref{sec:conditional}.

%%%%%%%%%%%%%%%%%%%%%%%%%%%%%%%%%%%%%%%%
\paragraph{Page Numbering.}

When only a part of the document is compiled,
the appropriate numbering of pages
(as well as other status parameters)
is determined from the |.aux| files.
The latter contain information from previous passes.
However this information needs to propagate through
all intermediate child documents.
Therefore the page numbering in child documents may well
be inconsistent until the complete document is compiled at least once.

A useful (if unconventional) way to always ensure a consistent
page numbering is to restart the numbering in each child document
and denote the pages by `\textit{child}|.|\textit{page}'
where \textit{child} represents the chapter/section number of the child file.
This can be achieved by the command
|\numberwithin{page}{|\textit{child}|}|
of the \textsf{amsmath} package
where \textit{child} can be |chapter| or |section|
depending on the chosen structuring.
Alternatively, one can modify the macro |\thepage| appropriately
and reset the counter |page| at the start of each child file.

%%%%%%%%%%%%%%%%%%%%%%%%%%%%%%%%%%%%%%%%%%%%%%%%%%%%%%%%%%%%%%%%%%%%%%%%%%%%%%%%
\subsection{Conditional Processing}
\label{sec:conditional}

The package provides a mechanism to compile different versions
of a document. To customise the versions further some conditional processing
can come in handy to distinguish which version is being compiled.
The package provides two macros to describe the compilation context:

%%%%%%%%%%%%%%%%%%%%%%%%%%%%%%%%%%%%%%%%
\DescribeMacro{\ifchilddoc}
The conditional |\ifchilddoc| distinguishes between the compilation of
child documents and the main document:
%
\begin{center}
|\ifchilddoc |\textit{child-code}| |[|\||else |\textit{main-code}]| \||fi|
\end{center}

%%%%%%%%%%%%%%%%%%%%%%%%%%%%%%%%%%%%%%%%
\DescribeMacro{\childdocname}
\DescribeMacro{\childdocjob}
The macro |\childdocname| contains the filename (without extension)
of the main or child file being processed.
Note that |\childdocjob| will always contain the name of the main file.

%%%%%%%%%%%%%%%%%%%%%%%%%%%%%%%%%%%%%%%%
\paragraph{Title Page.}

Conditional processing can be used to include a title or banner page
in the main document when proper precautions are taken.
Importantly, the code in the main file should ensure that the page counter
(as well as other status parameters which are stored in the |.aux| files)
takes the same value after the conditional processing.
Otherwise the page numbers may take divergent values
depending on which part is compiled.

For example, a title page could be declared by:
%
\begin{center}
\begin{tabular}{l}
|\ifchilddoc\||else|\\
|\addtocounter{page}{-1}|\\
\textit{code for title page}\\
|\newpage|\\
|\||fi|
\end{tabular}
\end{center}
%
A banner page for the child documents can be generated by:
%
\begin{center}
\begin{tabular}{l}
|\ifchilddoc|\\
|\addtocounter{page}{-1}|\\
\textit{code for banner page}\\
|\newpage|\\
|\||fi|
\end{tabular}
\end{center}
%
Here one could write a message such as:
\begin{center}
|This is the part \childdocname{} of \childdocjob{}.|
\end{center}

%%%%%%%%%%%%%%%%%%%%%%%%%%%%%%%%%%%%%%%%%%%%%%%%%%%%%%%%%%%%%%%%%%%%%%%%%%%%%%%%
\subsection{Flags}
\label{sec:flags}

The package makes it easy to generate different versions
of the main or child documents.
To this end compilation flags can be defined
and assigned different default values.
They will be particularly useful in conjunction
with the forwarding mechanism described in \secref{sec:forward}.

For example, it may be useful to have a flag |\version|
which can be set to |draft| or |final|.
The document source will contain some conditional code
depending on the value of |\version|.
Suppose further, the flag should default to |final| for the main file
and to |draft| for child files
which is a natural assignment for editing the document.
This is achieved by placing the following code
in the preamble of the main document
(below the |\childdocmain| directive):
%
\begin{center}
\begin{tabular}{l}
|\ifchilddoc|\\
|\providecommand{\version}{draft}|\\
|\||else|\\
|\providecommand{\version}{final}|\\
|\||fi|
\end{tabular}
\end{center}
%
The definition by |\providecommand| makes sure
that previous definitions are not overwritten.
Further statements |\providecommand{\version}{...}|
can thus be added before the above code to override it.

For the main file, one might add a line
(between |\childdocmain| and the above block)
%
\begin{center}
|%\ifchilddoc\||else\providecommand{\version}{draft}\||fi|
\end{center}
%
which can be uncommented to produce a draft version.
Likewise one can add a line to the very top of a child file
(above the |\childdocof{|\textit{main}|}| directive)
%
\begin{center}
|%\providecommand{\version}{final}|
\end{center}
%
which can be uncommented to produce the final version of this child document.

%%%%%%%%%%%%%%%%%%%%%%%%%%%%%%%%%%%%%%%%%%%%%%%%%%%%%%%%%%%%%%%%%%%%%%%%%%%%%%%%
\subsection{Forwarding}
\label{sec:forward}

Different versions of the main or child documents
using compilation flags as described in \secref{sec:flags}
can be (permanently) stored in different files
for convenient compilation, viewing and distribution.
To this end, the package defines a command
to pass on compilation to a different file:

%%%%%%%%%%%%%%%%%%%%%%%%%%%%%%%%%%%%%%%%
\DescribeMacro{\childdocforward}
The command |\childdocforward| redirects processing to
another source file:
%
\begin{center}
\begin{tabular}{l}
|\input{childdoc.def}|\\
|\childdocforward[|\textit{main}|]{|\textit{dest}|}|\\
\end{tabular}
\end{center}
%
The argument \textit{dest} is the destination file
(without extension).
It should be the main file or one of the child files.
Note that further \textsf{childdoc} directives
such as |\childdocof| and |\childdocforward|
in the indicated file will be processed in this form.
The optional argument \textit{main}
passes on directly to the main file \textit{main}
while pretending to compile the child \textit{dest}.
This form behaves as if \textit{dest}
issues |\childdocof{|\textit{main}|}| right away,
and no further \textsf{childdoc} directives will be processed.

%%%%%%%%%%%%%%%%%%%%%%%%%%%%%%%%%%%%%%%%
\DescribeMacro{\...prefix}
In the alternative form |\childdocforwardprefix|,
%
\begin{center}
\begin{tabular}{l}
|\input{childdoc.def}|\\
|\childdocforwardprefix[|\textit{main}|]{|\textit{prefix}|}{|\textit{dest}|}|
\end{tabular}
\end{center}
%
the destination file is determined by a pattern
depending on the current file:
To make this work, the current file must be called
`{\textit{prefix}\hspace{0.2em}\textit{suffix}}'
with \textit{prefix} matching precisely the argument.
Processing is then passed on to the file
`{\textit{dest}\hspace{0.2em}\textit{suffix}}'.
Surely, the same effect is achieved by
directly specifying the
argument `{\textit{dest}\hspace{0.2em}\textit{suffix}}'
in the first form.
However, that requires to set up a different file
for each child. With the alternative form of the command
all these files can have exactly the same content
which simplifies setting them up and maintaining them.

For example, the following file |draft.tex|
with a compilation flag |\version| as described in \secref{sec:flags}
compiles the main document as a draft:
%
\begin{center}
\begin{tabular}{l}
|\def\version{draft}|\\
|\input{childdoc.def}|\\
|\childdocforward{|\textit{main}|}|
\end{tabular}
\end{center}
%
Likewise, the following files |final|\textit{nn}|.tex|
compile the final version of the child document
|child|\textit{nn}|.tex|:
%
\begin{center}
\begin{tabular}{l}
|\def\version{final}|\\
|\input{childdoc.def}|\\
|\childdocforwardprefix{final}{child}|
\end{tabular}
\end{center}
%

Note that when several versions of a main file and/or of each child file
are to be generated, it may be convenient to set up a |Makefile| or
shell script to automatise the process.

%%%%%%%%%%%%%%%%%%%%%%%%%%%%%%%%%%%%%%%%%%%%%%%%%%%%%%%%%%%%%%%%%%%%%%%%%%%%%%%%
\subsection{Command Line Processing}
\label{sec:commandline}

The effect of redirection files can also be achieved by invoking
the \LaTeX{} compiler with a more elaborate command line.
Most conveniently this should be done as part
of a shell script or a |Makefile|.

When using \textsf{childdoc} in the main file, the following
command lines effectively perform a redirection
(note that depending on the shell being used,
backslashes may have to be doubled: `|\|' $\to$ `|\\|'):
%
\begin{center}
|... -jobname "|\textit{target}|" |\\|"|[\textit{flags}]%
|\input{childdoc.def}\childdocforward[|\textit{main}|]{|\textit{dest}|}"|
\end{center}
%
Here \textit{target} is the name of the output file,
\textit{main} is the name of the main file
and \textit{dest} is the name of the main or child file to be processed
(all filenames without extensions).
The optional argument \textit{main} can be omitted
if \textit{main} matches \textit{dest}.
Optionally, compilation \textit{flags} can be defined via |\def| commands.
This command line makes the \TeX{} engine believe
it is compiling the file \textit{target}
whose content is specified as the latter parameter.
The provided code then forwards the processing to
\textit{main} or \textit{dest} as described in \secref{sec:forward}.

%%%%%%%%%%%%%%%%%%%%%%%%%%%%%%%%%%%%%%%%%%%%%%%%%%%%%%%%%%%%%%%%%%%%%%%%%%%%%%%%
\subsection{Include by Input}
\label{sec:input}

Including child documents by |\include| has some restrictions by design.
Most notably, the content of a child document always occupies
its own set of pages; pages cannot be shared between child documents.
Usually, this behaviour makes perfect sense
because each child document contain an essential part of the document.
However, in some situations it may be desirable to compose
a document from a collection of parts
without having mandatory page breaks between then.
For this case, the package
provides a mechanism to include parts
by |\input| which can also be processed individually.
However, by construction this mechanism
requires manual handling of the content to be output.

%%%%%%%%%%%%%%%%%%%%%%%%%%%%%%%%%%%%%%%%
\DescribeMacro{\ifchilddocmanual}
The main file should be prepared as usual, see \secref{sec:include}.
However, the document body must make a distinction
between processing of an individual part and of the main document, e.g.:
%
\begin{center}
\begin{tabular}{l}
|\ifchilddocmanual|\\
|\input{\childdocname}|\\
|\||else|\\
\textit{document body with }|\input{|\textit{part}|}|\\
|\||fi|
\end{tabular}
\end{center}
%
The conditional |\ifchilddocmanual| is true whenever
a part to be included by |\input| is being compiled,
and the name of the part is stored in |\childdocname|.

%%%%%%%%%%%%%%%%%%%%%%%%%%%%%%%%%%%%%%%%
\DescribeMacro{\childdocby}
Each part to be included by |\input| should start with:
%
\begin{center}
\begin{tabular}{l}
|\input{childdoc.def}|\\
|\childdocby{|\textit{main}|}|\\
\end{tabular}
\end{center}
%
The directive |\childdocby| is similar to |\childdocof|
described in \secref{sec:include},
but the subsequent selection of content must be done manually.
To that end, both |\ifchilddoc| and |\ifchilddocmanual|
will be true upon processing of a part,
and the name of the part is stored in |\childdocname|.
Note that |\jobname| will be set to the filename of the current part
so that each part receives an individual |.aux| file
that does not interfere with the |.aux| file(s) of the main document.
This behaviour can be altered by the alternative form
|\childdocby[*]{|\textit{main}|}| (with a non-empty optional argument)
which uses the |.aux| file of the main document
by setting |\jobname| to \textit{main}.

%%%%%%%%%%%%%%%%%%%%%%%%%%%%%%%%%%%%%%%%%%%%%%%%%%%%%%%%%%%%%%%%%%%%%%%%%%%%%%%%
\subsection{Driver Development}
\label{sec:driver}

The \textsf{childdoc} mechanism can also be use for the development
of definition files such as \LaTeX{} styles or classes.
This case differs from the above setup with multiple parts
included by |\include| in that no |\includeonly| should be invoked.
This can be achieved by starting the include file
(before |\ProvidesPackage|) with:
%
\begin{center}
\begin{tabular}{l}
|\input{childdoc.def}|\\
|\childdocforward{|\textit{main}|}|\\
\end{tabular}
\end{center}
%
or alternatively with:
%
\begin{center}
\begin{tabular}{l}
|\input{childdoc.def}|\\
|\childdocby{|\textit{main}|}|\\
\end{tabular}
\end{center}
%
Both forms have slightly different effects as described above.
The main file is prepared as usual, see \secref{sec:include}.

%%%%%%%%%%%%%%%%%%%%%%%%%%%%%%%%%%%%%%%%%%%%%%%%%%%%%%%%%%%%%%%%%%%%%%%%%%%%%%%%
\subsection{Legacy Detection}
\label{sec:detection}

The directive |\childdocmain| in the main file can detect
whether the complete document or merely a child is to be compiled
even without using the directive |\childdocof|.
This method is deprecated because it is less robust
and there is no compelling reason to use it;
it is merely provided for backward compatibility
and it may be removed in future versions.

If the detection mechanism is to be used,
it is mandatory to correctly specify
the filename of the main file as the argument of |\childdocmain|:
%
\begin{center}
\begin{tabular}{l}
|\input{childdoc.def}|\\
|\childdocmain{|\textit{main}|}|\\
\end{tabular}
\end{center}
%
If |\jobname| does not match the argument \textit{main} of |\childdocmain|,
it is assumed that |\jobname| points to the child file to be compiled.
When using |\childdocmain| with the main file specified as argument,
it suffices to start a child file
with just |\input{|\textit{main}|}|
without loading of the package and using |\childdocof|.
If instead all processing is done
with the appropriate \textsf{childdoc} directives,
the argument of \textit{main} of |\childdocmain| can be empty.

An alternative version of the command line processing described
in \secref{sec:commandline} using the detection mechanism reads:
%
\begin{center}
|... -jobname "|\textit{target}|" "|[\textit{flags}]%
[|\def\jobname{|\textit{dest}|}|]|\input{|\textit{main}|}"|
\end{center}

%%%%%%%%%%%%%%%%%%%%%%%%%%%%%%%%%%%%%%%%%%%%%%%%%%%%%%%%%%%%%%%%%%%%%%%%%%%%%%%%
\subsection{Manual Code}
\label{sec:manual}

In case one cannot be certain whether the definitions file |childdoc.def|
is installed on the target \TeX{} distribution
and one prefers not to ship it,
it is conceivable to paste a few relevant commands into the sources.

To that end, drop all statements |\input{childdoc.def}|
and perform the replacements as outlined below.
Instead of |\childdocmain{|\textit{main}|}| add the following code
to the top of the main file:
%
\begin{center}
\begin{tabular}{l}
|\||ifdefined\childdocname\endinput\||fi\newif\ifchilddoc|\\
|\edef\childdocname{\scantokens\expandafter{\jobname\noexpand}}|\\
|\def\childdocmain{|\textit{main}|}\||ifx\childdocmain\childdocname\||else|\\
|\childdoctrue\includeonly{\childdocname}\let\jobname\childdocmain\||fi|\\
\end{tabular}
\end{center}
%
Instead of |\childdocof{|\textit{main}|}| just include the main file
at the top of each child file:
%
\begin{center}
|\input{|\textit{main}|}|
\end{center}
%
A simple redirection |\childdocforward{|\textit{dest}|}| is achieved by:
%
\begin{center}
|\def\jobname{|\textit{dest}|}\input{\jobname}|
\end{center}
%
The redirection with prefix
|\childdocforwardprefix[|\textit{prefix}|]{|\textit{dest}|}|
is accomplished by:
%
\begin{center}
\begin{tabular}{l}
|{\edef\jobname{\scantokens\expandafter{\jobname\noexpand}}|\\
|\def\redirectjob |\textit{prefix}|#1~~~{\gdef\jobname{|\textit{dest}|#1}}|\\
|\expandafter\redirectjob\jobname~~~}\input{\jobname}|
\end{tabular}
\end{center}

In an alternative approach,
child documents can be compiled by a specific command line
without additional code or specific definitions:
%
\begin{center}
|... -jobname "|\textit{target}|" "|[\textit{flags}]%
|\includeonly{|\textit{dest}|}\input{|\textit{main}|}"|
\end{center}
%

%%%%%%%%%%%%%%%%%%%%%%%%%%%%%%%%%%%%%%%%%%%%%%%%%%%%%%%%%%%%%%%%%%%%%%%%%%%%%%%%
%%%%%%%%%%%%%%%%%%%%%%%%%%%%%%%%%%%%%%%%%%%%%%%%%%%%%%%%%%%%%%%%%%%%%%%%%%%%%%%%
\section{Information}

%%%%%%%%%%%%%%%%%%%%%%%%%%%%%%%%%%%%%%%%%%%%%%%%%%%%%%%%%%%%%%%%%%%%%%%%%%%%%%%%
\subsection{Copyright}

Copyright \copyright{} 2017--2018 Niklas Beisert

This work may be distributed and/or modified under the
conditions of the \LaTeX{} Project Public License, either version 1.3
of this license or (at your option) any later version.
The latest version of this license is in
  \url{http://www.latex-project.org/lppl.txt}
and version 1.3 or later is part of all distributions of \LaTeX{}
version 2005/12/01 or later.

This work has the LPPL maintenance status `maintained'.

The Current Maintainer of this work is Niklas Beisert.

This work consists of the files |README.txt|, |childdoc.ins| and |childdoc.dtx|
as well as the derived files |childdoc.def|, |cdocsamp.tex|
with |cdocsch1.tex|, |cdocsch2.tex|, |cdocspt3.tex|, |cdocspt4.tex|,
|cdocsdrf.tex|, |cdocsfn1.tex|, |cdocsfn2.tex|
as well as |childdoc.pdf|.

%%%%%%%%%%%%%%%%%%%%%%%%%%%%%%%%%%%%%%%%%%%%%%%%%%%%%%%%%%%%%%%%%%%%%%%%%%%%%%%%
\subsection{Files and Installation}

The package consists of the files:
%
\begin{center}
\begin{tabular}{ll}
    |README.txt|   & readme file \\
    |childdoc.ins| & installation file \\
    |childdoc.dtx| & source file \\
    |childdoc.def| & definition file \\
    |cdocsamp.tex| & sample main file \\
    |cdocsch1.tex| & sample include file \\
    |cdocsch2.tex| & sample include file \\
    |cdocspt3.tex| & sample part file \\
    |cdocspt4.tex| & sample part file \\
    |cdocsdrf.tex| & sample redirection file \\
    |cdocsfn1.tex| & sample redirection file \\
    |cdocsfn2.tex| & sample redirection file \\
    |childdoc.pdf| & manual
\end{tabular}
\end{center}
%
The distribution consists of the files
|README.txt|, |childdoc.ins| and |childdoc.dtx|.
%
\begin{itemize}
\item
Run (pdf)\LaTeX{} on |childdoc.dtx|
to compile the manual |childdoc.pdf| (this file).
\item
Run \LaTeX{} on |childdoc.ins| to create the definitions file |childdoc.def|
and the sample |cdocsamp.tex| with include files
|cdocsch1.tex|, |cdocsch2.tex|, |cdocspt3.tex|, |cdocspt4.tex|,
|cdocsdrf.tex|, |cdocsfn1.tex|, |cdocsfn2.tex|.
Then copy the file |childdoc.def| to an appropriate directory of your \LaTeX{}
distribution, e.g.\ \textit{texmf-root}|/tex/latex/childdoc|.
\end{itemize}

%%%%%%%%%%%%%%%%%%%%%%%%%%%%%%%%%%%%%%%%%%%%%%%%%%%%%%%%%%%%%%%%%%%%%%%%%%%%%%%%
\subsection{Related CTAN Packages}

There are several other packages which offer a similar functionality:
%
\begin{itemize}
\item
The packages
\href{http://ctan.org/pkg/docmute}{\textsf{docmute}},
\href{http://ctan.org/pkg/includex}{\textsf{includex}} and
\href{http://ctan.org/pkg/standalone}{\textsf{standalone}}
provide commands to include only the document body of
a child file thus allowing both files to be compiled individually.
\item
The packages \href{http://ctan.org/pkg/subdocs}{\textsf{subdocs}}
and \href{http://ctan.org/pkg/subfiles}{\textsf{subfiles}}
provide structures in which the main and child documents can be
encapsulated and allowing them to be compiled individually.
The inclusion mechanism is different from the conventional |\include|.
\item
The package \href{http://ctan.org/pkg/combine}{\textsf{combine}}
is an elaborate solution to combine several documents into one.
\end{itemize}
%
See also the CTAN topic \href{http://ctan.org/topic/subdocs}{\textsf{subdocs}}
for further related packages.
The present package differs from the above solutions in that
a document structure constructed with the conventional |\include| mechanism
just needs two extra commands at the top of every file
such that all constituent files can be compiled individually.

%%%%%%%%%%%%%%%%%%%%%%%%%%%%%%%%%%%%%%%%%%%%%%%%%%%%%%%%%%%%%%%%%%%%%%%%%%%%%%%%
%\subsection{Feature Suggestions}
%
%The following is a list of features which may be useful for future
%versions of this package:
%%
%\begin{itemize}
%\item
%\ldots
%\end{itemize}

%%%%%%%%%%%%%%%%%%%%%%%%%%%%%%%%%%%%%%%%%%%%%%%%%%%%%%%%%%%%%%%%%%%%%%%%%%%%%%%%
\subsection{Revision History}

%%%%%%%%%%%%%%%%%%%%%%%%%%%%%%%%%%%%%%%%
\paragraph{v2.0:} 2018/12/30

\begin{itemize}
\item
immediate forward processing
\item
added |\childdocby| mechanism
\item
manual restructured
\end{itemize}

%%%%%%%%%%%%%%%%%%%%%%%%%%%%%%%%%%%%%%%%
\paragraph{v1.6:} 2018/01/17

\begin{itemize}
\item
application for development of include files
\item
corrections to manual
\end{itemize}

%%%%%%%%%%%%%%%%%%%%%%%%%%%%%%%%%%%%%%%%
\paragraph{v1.5:} 2017/05/21

\begin{itemize}
\item
more complete structuring introduced
\item
|\childdocof| introduced
\item
|\childdoc| renamed to |\childdocmain|
\item
|\childredirect| renamed to |\childdocforward| and |\childdocforwardprefix|
and functionality expanded
\end{itemize}

%%%%%%%%%%%%%%%%%%%%%%%%%%%%%%%%%%%%%%%%
\paragraph{v1.0:} 2017/04/27

\begin{itemize}
\item
manual and install package
\item
first version published on CTAN
\end{itemize}

%%%%%%%%%%%%%%%%%%%%%%%%%%%%%%%%%%%%%%%%
\paragraph{v0.6:} 2017/04/26

\begin{itemize}
\item
redirection mechanism added
\end{itemize}

%%%%%%%%%%%%%%%%%%%%%%%%%%%%%%%%%%%%%%%%
\paragraph{v0.5:} 2017/04/26

\begin{itemize}
\item
functionality in definition file
\end{itemize}


%%%%%%%%%%%%%%%%%%%%%%%%%%%%%%%%%%%%%%%%%%%%%%%%%%%%%%%%%%%%%%%%%%%%%%%%%%%%%%%%
%%%%%%%%%%%%%%%%%%%%%%%%%%%%%%%%%%%%%%%%%%%%%%%%%%%%%%%%%%%%%%%%%%%%%%%%%%%%%%%%
%%%%%%%%%%%%%%%%%%%%%%%%%%%%%%%%%%%%%%%%%%%%%%%%%%%%%%%%%%%%%%%%%%%%%%%%%%%%%%%%
\appendix

\settowidth\MacroIndent{\rmfamily\scriptsize 000\ }

 \DocInput{childdoc.dtx}

\end{document}
%</driver>
% \fi
%
% %%%%%%%%%%%%%%%%%%%%%%%%%%%%%%%%%%%%%%%%%%%%%%%%%%%%%%%%%%%%%%%%%%%%%%%%%%%%%%
% %%%%%%%%%%%%%%%%%%%%%%%%%%%%%%%%%%%%%%%%%%%%%%%%%%%%%%%%%%%%%%%%%%%%%%%%%%%%%%
% \section{Sample}
%\iffalse
%<*samplemain>
%\fi
%
% The following presents a sample document
% with two chapters, two parts, a title page,
% a compile flag as well as three forwarding files to set the flag.
% It consists of eight |.tex| files:
% \begin{center}
% \begin{tabular}{ll}
% |cdocsamp.tex|&main file\\
% |cdocsch1.tex|&include file for chapter 1\\
% |cdocsch2.tex|&include file for chapter 2\\
% |cdocspt3.tex|&include file for part 3\\
% |cdocspt4.tex|&include file for part 4\\
% |cdocsdrf.tex|&forwarding file for main file in draft mode\\
% |cdocsfi1.tex|&forwarding file for final version of chapter 1\\
% |cdocsfi2.tex|&forwarding file for final version of chapter 2\\
% \end{tabular}
% \end{center}
% Each of the eight files can be compiled directly by the \LaTeX{} compiler.
%
% %%%%%%%%%%%%%%%%%%%%%%%%%%%%%%%%%%%%%%
% \paragraph{Main File.}
%
% The main file is called |cdocsamp.tex|.
%
% Load the \textsf{childdoc} definitions and
% declare the filename for the main document:
%    \begin{macrocode}
\input{childdoc.def}
\childdocmain{}
%    \end{macrocode}

% Optional override for |\version| flag:
%    \begin{macrocode}
%%\ifchilddoc\else\providecommand{\version}{draft}\fi
%    \end{macrocode}

% Define the default values for the |\version| flag
% (|final| for the main file and |draft| for childs):
%    \begin{macrocode}
\ifchilddoc
\providecommand{\version}{draft}
\else
\providecommand{\version}{final}
\fi
%    \end{macrocode}

% Load the standard document class:
%    \begin{macrocode}
\documentclass[12pt]{article}
%    \end{macrocode}

% Start the document body:
%    \begin{macrocode}
\begin{document}
%    \end{macrocode}

% Declare a title page.
% Print title, part of document being processed and version flag:
%    \begin{macrocode}
\addtocounter{page}{-1}
\begin{center}
{\LARGE\bfseries{}childdoc example\par}
\vspace{1cm}
\ifchilddoc
\ifchilddocmanual part\else chapter\fi:
`\childdocname' of `\childdocjob'\par
\else
main document: `\childdocjob'\par
\fi
version: \version\par
\end{center}
\newpage
%    \end{macrocode}

% Manually include selected file,
% otherwise process as usual:
%    \begin{macrocode}
\ifchilddocmanual
\section*{part `\childdocname'}
\input{\childdocname}
\else
%    \end{macrocode}

% Include the two chapters:
%    \begin{macrocode}
\include{cdocsch1}
\include{cdocsch2}
%    \end{macrocode}

% Include the two parts unless only chapters should be displayed:
%    \begin{macrocode}
\ifchilddoc\else
\section{part three}
\input{cdocspt3}
\section{part four}
\input{cdocspt4}
\fi
%    \end{macrocode}

% Process as usual until here:
%    \begin{macrocode}
\fi
%    \end{macrocode}

% End of document body:
%    \begin{macrocode}
\end{document}
%    \end{macrocode}
%\iffalse
%</samplemain>
%\fi
%
% %%%%%%%%%%%%%%%%%%%%%%%%%%%%%%%%%%%%%%
% \paragraph{Chapter Include Files.}
%
% The include files are called |cdocsch1.tex| and |cdocsch2.tex|.
%
%\iffalse
%<*samplechap1|samplechap2>
%\fi

% Optional override for |\version| flag:
%    \begin{macrocode}
%%\providecommand{\version}{final}
%    \end{macrocode}

% Include the main document:
%    \begin{macrocode}
\input{childdoc.def}
\childdocof{cdocsamp}
%    \end{macrocode}

%\iffalse
%</samplechap1|samplechap2>
%\fi
%
%\iffalse
%<*samplechap1>
%\fi
% Some text for chapter 1:
%    \begin{macrocode}
\section{one}
some text in chapter one
%    \end{macrocode}

%\iffalse
%</samplechap1>
%\fi
% Some text for chapter 2:
%\iffalse
%<*samplechap2>
%\fi
%    \begin{macrocode}
\section{two}
more text in chapter two
%    \end{macrocode}

%\iffalse
%</samplechap2>
%\fi
%
% %%%%%%%%%%%%%%%%%%%%%%%%%%%%%%%%%%%%%%
% \paragraph{Part Include Files.}
%
% The include files are called |cdocspt3.tex| and |cdocspt4.tex|.
%
%\iffalse
%<*samplepart3|samplepart4>
%\fi

% Optional override for |\version| flag:
%    \begin{macrocode}
%%\providecommand{\version}{final}
%    \end{macrocode}

% Include the main document:
%    \begin{macrocode}
\input{childdoc.def}
\childdocby{cdocsamp}
%    \end{macrocode}

%\iffalse
%</samplepart3|samplepart4>
%\fi
%
%\iffalse
%<*samplepart3>
%\fi
% Some text for part 3:
%    \begin{macrocode}
some text in part three
%    \end{macrocode}

%\iffalse
%</samplepart3>
%\fi
% Some text for part 4:
%\iffalse
%<*samplepart4>
%\fi
%    \begin{macrocode}
more text in part four
%    \end{macrocode}

%\iffalse
%</samplepart4>
%\fi
%
% %%%%%%%%%%%%%%%%%%%%%%%%%%%%%%%%%%%%%%
% \paragraph{Forwarding for a Complete Draft.}
%
% The following forwarding file |cdocsdrf.tex|
% compiles the main document in draft mode:
%\iffalse
%<*sampledraft>
%\fi
%    \begin{macrocode}
\def\version{draft}
\input{childdoc.def}
\childdocforward{cdocsamp}
%    \end{macrocode}

%\iffalse
%</sampledraft>
%\fi
%
% %%%%%%%%%%%%%%%%%%%%%%%%%%%%%%%%%%%%%%
% \paragraph{Forwarding for Final Version of the Chapters.}
%
% The following forwarding files |cdocsfn1.tex| and |cdocsfn2.tex|
% (with identical content)
% compile the final versions of the child documents
% |cdocsch1.tex| and |cdocsch2.tex|, respectively:
%\iffalse
%<*samplefinal>
%\fi
%    \begin{macrocode}
\def\version{final}
\input{childdoc.def}
\childdocforwardprefix[cdocsamp]{cdocsfn}{cdocsch}
%    \end{macrocode}

%\iffalse
%</samplefinal>
%\fi
%
% %%%%%%%%%%%%%%%%%%%%%%%%%%%%%%%%%%%%%%
% \paragraph{Command Line Processing.}
%
% The following three command lines generate the output files
% |cdocscld|, |cdocscl1| and |cdocscl2|
% which should be identical to
% |cdocsdrf|, |cdocsch1| and |cdocsfn2|, respectively:
% \begin{center}
% \begin{tabular}{l}
% |latex -jobname cdocscld \|\\
% |  "\def\version{draft}\input{childdoc.def}\childdocforward{cdocsamp}"|\\
% |latex -jobname cdocscl1 \|\\
% |  "\input{childdoc.def}\childdocforward[cdocsamp]{cdocsch1}"|\\
% |latex -jobname cdocscl2 \|\\
% |  "\def\version{final}\input{childdoc.def}\childdocforward{cdocsch2}"|
% \end{tabular}
% \end{center}
% Note that the trailing backslash on each first line
% merely continues the input to the second line
% (for convenient cut ant paste).
% Furthermore, the command |latex| can be replaced by any
% of its alternative versions such as |pdflatex|.
%
% %%%%%%%%%%%%%%%%%%%%%%%%%%%%%%%%%%%%%%%%%%%%%%%%%%%%%%%%%%%%%%%%%%%%%%%%%%%%%%
% %%%%%%%%%%%%%%%%%%%%%%%%%%%%%%%%%%%%%%%%%%%%%%%%%%%%%%%%%%%%%%%%%%%%%%%%%%%%%%
% \section{Implementation}
%\iffalse
%<*package>
%\fi
%
% This section describes the definitions file |childdoc.def|.

% The definitions cannot be loaded using |\usepackage| or |\RequirePackage|
% which has a mechanism to prevent loading a style file more than once.
% When loading the definitions by means of |\input|
% multiple instances have to be prevented manually:
%\iffalse
%This code needs to be before the `\ProvidesFile' directive
%which is defined at the beginning of this file.
%Therefore it is also placed there and commented out here.
%</package>
%<*discard>
%\fi
%    \begin{macrocode}
\ifdefined\childdocmain\endinput\fi
%    \end{macrocode}
%\iffalse
%</discard>
%<*package>
%\fi
%
% \macro{\ifchilddoc}
% \macro{\ifchilddocmanual}
% The conditional |\ifchilddoc| tells whether a
% child (true) or main (false) document is being compiled.
% The conditional |\ifchilddocmanual| tells whether
% the |\includeonly| mechanism is used (false) or
% the selection of child files must be performed manually (true).
% The definitions initialise to false:
%    \begin{macrocode}
\newif\ifchilddoc
\newif\ifchilddocmanual
%    \end{macrocode}

% \macro{\childdocname}
% \macro{\childdocjob}
% The macro |\childdocname| stores the name of the main document
% to be compiled. The macro |\childdocjob| stores the name of
% the document on which the \LaTeX{} compiler was originally invoked.
% The content of |\jobname| cannot be compared
% to filenames specified in the source due to different catcodes.
% The following code rescans |\jobname|, stores the result
% in |\childdocname| and saves a copy in |\childdocjob|:
%    \begin{macrocode}
\edef\childdocname{\scantokens\expandafter{\jobname\noexpand}}
\let\childdocjob\childdocname
%    \end{macrocode}

% \macro{\childdocdisable}
% The macro |\childdocdisable| prevents the main file
% from being processed more than once.
% At this stage, the main document command |\childdocmain|
% is assumed to be called once again where it should do nothing.
% Any subsequent call to it should prevent
% a secondary processing of the main document
% It overwrites the forwarding commands
% |\childdocof| and |\childdocforward|
% with empty macros to prevent further inclusions of the main document:
%    \begin{macrocode}
\newcommand{\childdocdisable}
{
  \renewcommand{\childdocmain}[1]{\renewcommand{\childdocmain}[1]{\endinput}}
  \renewcommand{\childdocof}[1]{}
  \renewcommand{\childdocby}[2][]{}
  \renewcommand{\childdocforward}[2][]{}
  \renewcommand{\childdocdisable}{}
}
%    \end{macrocode}

% \macro{\childdocmain}
% The macro |\childdocmain| is to be called at the top of the main file
% with nothing or the main filename (without extension) as argument.
% First, it breaks loops.
% If the argument is not empty and does not match |\childdocname|
% (which is set by the first inclusion of |childdoc.def|),
% |\ifchilddoc| is set to true, |\includeonly| is applied to the child file
% and |\jobname| is set to the main file
% (for proper handling of |.aux| files):
%    \begin{macrocode}
\newcommand{\childdocmain}[1]
{
  \childdocdisable\childdocmain{}
  \if?#1?\else
    \begingroup
      \def\childdoctmp{#1}
      \ifx\childdoctmp\childdocname
        \def\childdoctmp{}
      \else
        \def\childdoctmp
        {
          \childdoctrue
          \includeonly{\childdocname}
          \def\childdocjob{#1}
          \def\jobname{#1}
        }
      \fi
      \expandafter
    \endgroup
    \childdoctmp
  \fi
}
%    \end{macrocode}

% \macro{\childdocof}
% The command |\childdocof| redirects
% compilation to the main file |#1|.
%    \begin{macrocode}
\newcommand{\childdocof}[1]
{
  \childdocdisable
  \childdoctrue
  \includeonly{\childdocname}
  \def\jobname{#1}
  \def\childdocjob{#1}
  \input{#1}
}
%    \end{macrocode}

% \macro{\childdocby}
% The command |\childdocby| ....
%    \begin{macrocode}
\newcommand{\childdocby}[2][]
{
  \childdocdisable
  \childdoctrue
  \childdocmanualtrue
  \if?#1?\else
    \def\jobname{#2}
  \fi
  \def\childdocjob{#2}
  \input{#2}
  \endinput
}
%    \end{macrocode}

% \macro{\childdocforward}
% The command |\childdocforward| redirects
% compilation to the main file or
% (if the optional argument is given) a child file.
% Parameters are set as if the main file
% or a child file starting with |\childdocof| was compiled.
% Then compilation is handed over to the main file:
%    \begin{macrocode}
\newcommand{\childdocforward}[2][]
{
  \begingroup
    \if?#1?
      \def\childdoctmp
      {
        \def\childdocname{#2}
        \def\childdocjob{#2}
        \def\jobname{#2}
        \input{#2}
        \endinput
      }
    \else
      \def\childdoctmp
      {
        \childdocdisable
        \def\childdocname{#2}
        \childdoctrue
        \includeonly{#2}
        \def\childdocjob{#1}
        \def\jobname{#1}
        \input{#1}
        \endinput
      }
    \fi
    \expandafter
  \endgroup
  \childdoctmp
}
%    \end{macrocode}

% \macro{\childdocforwardprefix}
% The command |\childdocforwardprefix| redirects
% compilation to the main or a child file by means of a pattern.
% The prefix |#1| in the current filename is replaced by |#2|
% and the suffix of the current filename is kept
% (it is assumed that the filename does not contain the substring `|~~~|'
% which is used as a delimiter).
% Compilation is handed over to the new file by |\childdocforward|:
%    \begin{macrocode}
\newcommand{\childdocforwardprefix}[3][]
{
  \begingroup
    \def\childdocextract #2##1~~~{\def\childdoctmp{\childdocforward[#1]{#3##1}}}
    \expandafter\childdocextract\childdocname~~~
    \expandafter
  \endgroup
  \childdoctmp
}
%    \end{macrocode}

% \macro{\childdoc}
% The deprecated macro |\childdoc| is a legacy version of |\childdocmain|:
%    \begin{macrocode}
\newcommand{\childdoc}{\childdocmain}
%    \end{macrocode}

% \macro{\childdocredirect}
% The deprecated macro |\childdocredirect| is a legacy version
% of |\childdocforward| and |\childdocforwardprefix|:
%    \begin{macrocode}
\newcommand{\childdocredirect}[2][]
{
  \begingroup
    \if?#1?
      \def\childdoctmp{\childdocforward{#2}}
    \else
      \def\childdoctmp{\childdocforwardprefix{#1}{#2}}
    \fi
    \expandafter
  \endgroup
  \childdoctmp
}
%    \end{macrocode}

%\iffalse
%</package>
%\fi
%
\endinput
\childdocforward{cdocsamp}"|\\
% |latex -jobname cdocscl1 \|\\
% |  "% \iffalse
%
% childdoc.dtx Copyright (C) 2017-2018 Niklas Beisert
%
% This work may be distributed and/or modified under the
% conditions of the LaTeX Project Public License, either version 1.3
% of this license or (at your option) any later version.
% The latest version of this license is in
%   http://www.latex-project.org/lppl.txt
% and version 1.3 or later is part of all distributions of LaTeX
% version 2005/12/01 or later.
%
% This work has the LPPL maintenance status `maintained'.
%
% The Current Maintainer of this work is Niklas Beisert.
%
% This work consists of the files childdoc.dtx and childdoc.ins
% and the derived files childdoc.def and cdocsamp.tex with
% cdocsch1.tex, cdocsch2.tex, cdocsdrf.tex, cdocsfn1.tex, cdocsfn2.tex.
%
%<package>\ifdefined\childdocmain\endinput\fi
%<package>\ProvidesFile{childdoc.def}[2018/12/30 v2.0 child document driver]
%<samplemain>\ProvidesFile{cdocsamp.tex}[2018/12/30 v2.0 sample for childdoc]
%<*driver>
%\ProvidesFile{childdoc.drv}[2018/12/30 v2.0 childdoc reference manual file]
\PassOptionsToClass{10pt,a4paper}{article}
\documentclass{ltxdoc}

\usepackage[margin=35mm]{geometry}
\usepackage{hyperref}
\usepackage{hyperxmp}
\usepackage[usenames]{color}

\hypersetup{colorlinks=true}
\hypersetup{pdfstartview=FitH}
\hypersetup{pdfpagemode=UseNone}
\hypersetup{pdfsource={}}
\hypersetup{pdflang={en-UK}}
\hypersetup{pdfcopyright={Copyright 2017-2018 Niklas Beisert.
  This work may be distributed and/or modified under the
  conditions of the LaTeX Project Public License, either version 1.3
  of this license or (at your option) any later version.}}
\hypersetup{pdflicenseurl={http://www.latex-project.org/lppl.txt}}
\hypersetup{pdfcontactaddress={ETH Zurich, ITP, HIT K,
  Wolfgang-Pauli-Strasse 27}}
\hypersetup{pdfcontactpostcode={8093}}
\hypersetup{pdfcontactcity={Zurich}}
\hypersetup{pdfcontactcountry={Switzerland}}
\hypersetup{pdfcontactemail={nbeisert@itp.phys.ethz.ch}}
\hypersetup{pdfcontacturl={http://people.phys.ethz.ch/\xmptilde nbeisert/}}

\newcommand{\secref}[1]{\hyperref[#1]{section \ref*{#1}}}

\parskip1ex
\parindent0pt
\let\olditemize\itemize
\def\itemize{\olditemize\parskip0pt}

\begin{document}

\title{The \textsf{childdoc} Package}
\hypersetup{pdftitle={The childdoc Package}}
\author{Niklas Beisert\\[2ex]
  Institut f\"ur Theoretische Physik\\
  Eidgen\"ossische Technische Hochschule Z\"urich\\
  Wolfgang-Pauli-Strasse 27, 8093 Z\"urich, Switzerland\\[1ex]
  \href{mailto:nbeisert@itp.phys.ethz.ch}
  {\texttt{nbeisert@itp.phys.ethz.ch}}}
\hypersetup{pdfauthor={Niklas Beisert}}
\hypersetup{pdfsubject={Manual for the LaTeX2e Package childdoc}}
\date{30 December 2018, \textsf{v2.0}}
\maketitle

\begin{abstract}\noindent
\textsf{childdoc} is a \LaTeXe{} package
that enables the direct compilation
of document sections included by |\include|
to individual files.
\end{abstract}

\begingroup
\parskip0ex
\tableofcontents
\endgroup

%%%%%%%%%%%%%%%%%%%%%%%%%%%%%%%%%%%%%%%%%%%%%%%%%%%%%%%%%%%%%%%%%%%%%%%%%%%%%%%%
%%%%%%%%%%%%%%%%%%%%%%%%%%%%%%%%%%%%%%%%%%%%%%%%%%%%%%%%%%%%%%%%%%%%%%%%%%%%%%%%
\section{Introduction}

\LaTeX{} provides a mechanism to structure a large document (such as a book)
into a main file and several child files (containing the chapters)
using the |\include| command.
This mechanism is beneficial for documents
which span hundreds of pages in order to
make the source file(s) more manageable.
Moreover, compilation can be restricted to
selected child files by means of the |\includeonly| command.
The latter feature can be used to reduce the compilation time while editing
(this was significantly more useful in the earlier days of \LaTeX{})
or to generate a smaller document which is easier to navigate.
Another application of |\includeonly| is to generate
documents consisting of selected parts of the complete document.

However, there are a few drawbacks of the plain |\include| mechanism:
\begin{itemize}
\item
The child files cannot be compiled on their own,
they can only be compiled via the main file.
A naive editing environment
(such as a text editor with an option
to have the current file processed by \LaTeX)
may require one to switch to the main file before compiling;
attempting to compile the child file produces errors.
\item
The main file must be modified (each time)
to adjust the |\includeonly| command
to the present needs. This easily leaves the main file in a messy state.
\item
The generated document will always carry the filename
of the main document. This is inconvenient if
several child files are to be compiled and
to be kept for distribution.
\end{itemize}

The present package provides a simple interface
to make child files individually compilable by \LaTeX{}.
Compiling a child file then has the same effect as compiling
the main file with an |\includeonly| command
to select the appropriate child.
Moreover the generated document will carry the name of the child
rather than the main file.
This resolves all three above issues.

This feature is meant to make the editing of books,
thesis documents and lecture notes somewhat more convenient.
However, the package can also be used efficiently for
composing a series of documents (such as exercise sheets)
which are typically distributed individually.
It then assists the author in generating the individual documents
(potentially in different versions)
as well as a document containing the collected series.
Another application is in developing style files
or other kinds of included material
where compilation of the style file could redirect
to a sample or test file.

%%%%%%%%%%%%%%%%%%%%%%%%%%%%%%%%%%%%%%%%%%%%%%%%%%%%%%%%%%%%%%%%%%%%%%%%%%%%%%%%
%%%%%%%%%%%%%%%%%%%%%%%%%%%%%%%%%%%%%%%%%%%%%%%%%%%%%%%%%%%%%%%%%%%%%%%%%%%%%%%%
\section{Usage}

First of all, the package \textsf{childdoc} is \emph{not} a standard
\LaTeXe{} |.sty| style file! Therefore it needs to be invoked in
a non-standard way.

%%%%%%%%%%%%%%%%%%%%%%%%%%%%%%%%%%%%%%%%%%%%%%%%%%%%%%%%%%%%%%%%%%%%%%%%%%%%%%%%
\subsection{Included Files}
\label{sec:include}

%%%%%%%%%%%%%%%%%%%%%%%%%%%%%%%%%%%%%%%%
\DescribeMacro{\childdocmain}
To use the package, add the commands
\begin{center}
\begin{tabular}{l}
|\input{childdoc.def}|\\
|\childdocmain{}|\\
\end{tabular}
\end{center}
at the very top of the main \LaTeX{} file,
in particular \emph{before} the |\documentclass| statement!
The argument of |\childdocmain| should be left empty
(but it must be present).

%%%%%%%%%%%%%%%%%%%%%%%%%%%%%%%%%%%%%%%%
\DescribeMacro{\childdocof}
Furthermore, add the commands
\begin{center}
\begin{tabular}{l}
|\input{childdoc.def}|\\
|\childdocof{|\textit{main}|}|\\
\end{tabular}
\end{center}
at the top of every child file \textit{child}
which is included by |\include{|\textit{child}|}|
from within the main file
(or at least for those files to be compiled individually).
The argument \textit{main} must be the filename of the main file.

There are a couple of
considerations in setting up the main and child documents:

%%%%%%%%%%%%%%%%%%%%%%%%%%%%%%%%%%%%%%%%
\paragraph{Restrictions.}

Please note the following restrictions:
\begin{itemize}
\item
|\childdocmain| must be called with one argument \textit{main}
to ensure compatibility with earlier version of the package.
It must either be empty (|\childdocmain{}|)
or precisely match the filename of the main file in which it is specified.
See \secref{sec:detection} for further information.
\item
The filename \textit{main} must be specified without the |.tex| extension.
\item
The filename \textit{main} is case sensitive
(even in case-insensitive file systems)
due to internal string comparison.
\item
The argument \textit{main} should be fully expanded, it cannot be a macro.
\item
Subdirectories and special characters should be avoided in filenames.
\item
The command |\childdocmain{|\textit{main}|}| must be followed by a whitespace.
It should not be followed immediately by another command
or by a comment mark `|%|'.
This is because the \TeX{} parser reads the token immediately following
the argument of |\childdocmain| and puts it
at the beginning of every child section;
however, a white\-space is ignored.
\end{itemize}

%%%%%%%%%%%%%%%%%%%%%%%%%%%%%%%%%%%%%%%%
\paragraph{Content of Main File.}

It is advisable to place all content in the child files included by |\include|.
Any output contained in the main file will appear in all child documents
unless suppressed manually;
it cannot be suppressed automatically by the |\includeonly| directive
and thus should normally be avoided.
A method to include some content in the main file
by means of conditional processing is described in \secref{sec:conditional}.

%%%%%%%%%%%%%%%%%%%%%%%%%%%%%%%%%%%%%%%%
\paragraph{Page Numbering.}

When only a part of the document is compiled,
the appropriate numbering of pages
(as well as other status parameters)
is determined from the |.aux| files.
The latter contain information from previous passes.
However this information needs to propagate through
all intermediate child documents.
Therefore the page numbering in child documents may well
be inconsistent until the complete document is compiled at least once.

A useful (if unconventional) way to always ensure a consistent
page numbering is to restart the numbering in each child document
and denote the pages by `\textit{child}|.|\textit{page}'
where \textit{child} represents the chapter/section number of the child file.
This can be achieved by the command
|\numberwithin{page}{|\textit{child}|}|
of the \textsf{amsmath} package
where \textit{child} can be |chapter| or |section|
depending on the chosen structuring.
Alternatively, one can modify the macro |\thepage| appropriately
and reset the counter |page| at the start of each child file.

%%%%%%%%%%%%%%%%%%%%%%%%%%%%%%%%%%%%%%%%%%%%%%%%%%%%%%%%%%%%%%%%%%%%%%%%%%%%%%%%
\subsection{Conditional Processing}
\label{sec:conditional}

The package provides a mechanism to compile different versions
of a document. To customise the versions further some conditional processing
can come in handy to distinguish which version is being compiled.
The package provides two macros to describe the compilation context:

%%%%%%%%%%%%%%%%%%%%%%%%%%%%%%%%%%%%%%%%
\DescribeMacro{\ifchilddoc}
The conditional |\ifchilddoc| distinguishes between the compilation of
child documents and the main document:
%
\begin{center}
|\ifchilddoc |\textit{child-code}| |[|\||else |\textit{main-code}]| \||fi|
\end{center}

%%%%%%%%%%%%%%%%%%%%%%%%%%%%%%%%%%%%%%%%
\DescribeMacro{\childdocname}
\DescribeMacro{\childdocjob}
The macro |\childdocname| contains the filename (without extension)
of the main or child file being processed.
Note that |\childdocjob| will always contain the name of the main file.

%%%%%%%%%%%%%%%%%%%%%%%%%%%%%%%%%%%%%%%%
\paragraph{Title Page.}

Conditional processing can be used to include a title or banner page
in the main document when proper precautions are taken.
Importantly, the code in the main file should ensure that the page counter
(as well as other status parameters which are stored in the |.aux| files)
takes the same value after the conditional processing.
Otherwise the page numbers may take divergent values
depending on which part is compiled.

For example, a title page could be declared by:
%
\begin{center}
\begin{tabular}{l}
|\ifchilddoc\||else|\\
|\addtocounter{page}{-1}|\\
\textit{code for title page}\\
|\newpage|\\
|\||fi|
\end{tabular}
\end{center}
%
A banner page for the child documents can be generated by:
%
\begin{center}
\begin{tabular}{l}
|\ifchilddoc|\\
|\addtocounter{page}{-1}|\\
\textit{code for banner page}\\
|\newpage|\\
|\||fi|
\end{tabular}
\end{center}
%
Here one could write a message such as:
\begin{center}
|This is the part \childdocname{} of \childdocjob{}.|
\end{center}

%%%%%%%%%%%%%%%%%%%%%%%%%%%%%%%%%%%%%%%%%%%%%%%%%%%%%%%%%%%%%%%%%%%%%%%%%%%%%%%%
\subsection{Flags}
\label{sec:flags}

The package makes it easy to generate different versions
of the main or child documents.
To this end compilation flags can be defined
and assigned different default values.
They will be particularly useful in conjunction
with the forwarding mechanism described in \secref{sec:forward}.

For example, it may be useful to have a flag |\version|
which can be set to |draft| or |final|.
The document source will contain some conditional code
depending on the value of |\version|.
Suppose further, the flag should default to |final| for the main file
and to |draft| for child files
which is a natural assignment for editing the document.
This is achieved by placing the following code
in the preamble of the main document
(below the |\childdocmain| directive):
%
\begin{center}
\begin{tabular}{l}
|\ifchilddoc|\\
|\providecommand{\version}{draft}|\\
|\||else|\\
|\providecommand{\version}{final}|\\
|\||fi|
\end{tabular}
\end{center}
%
The definition by |\providecommand| makes sure
that previous definitions are not overwritten.
Further statements |\providecommand{\version}{...}|
can thus be added before the above code to override it.

For the main file, one might add a line
(between |\childdocmain| and the above block)
%
\begin{center}
|%\ifchilddoc\||else\providecommand{\version}{draft}\||fi|
\end{center}
%
which can be uncommented to produce a draft version.
Likewise one can add a line to the very top of a child file
(above the |\childdocof{|\textit{main}|}| directive)
%
\begin{center}
|%\providecommand{\version}{final}|
\end{center}
%
which can be uncommented to produce the final version of this child document.

%%%%%%%%%%%%%%%%%%%%%%%%%%%%%%%%%%%%%%%%%%%%%%%%%%%%%%%%%%%%%%%%%%%%%%%%%%%%%%%%
\subsection{Forwarding}
\label{sec:forward}

Different versions of the main or child documents
using compilation flags as described in \secref{sec:flags}
can be (permanently) stored in different files
for convenient compilation, viewing and distribution.
To this end, the package defines a command
to pass on compilation to a different file:

%%%%%%%%%%%%%%%%%%%%%%%%%%%%%%%%%%%%%%%%
\DescribeMacro{\childdocforward}
The command |\childdocforward| redirects processing to
another source file:
%
\begin{center}
\begin{tabular}{l}
|\input{childdoc.def}|\\
|\childdocforward[|\textit{main}|]{|\textit{dest}|}|\\
\end{tabular}
\end{center}
%
The argument \textit{dest} is the destination file
(without extension).
It should be the main file or one of the child files.
Note that further \textsf{childdoc} directives
such as |\childdocof| and |\childdocforward|
in the indicated file will be processed in this form.
The optional argument \textit{main}
passes on directly to the main file \textit{main}
while pretending to compile the child \textit{dest}.
This form behaves as if \textit{dest}
issues |\childdocof{|\textit{main}|}| right away,
and no further \textsf{childdoc} directives will be processed.

%%%%%%%%%%%%%%%%%%%%%%%%%%%%%%%%%%%%%%%%
\DescribeMacro{\...prefix}
In the alternative form |\childdocforwardprefix|,
%
\begin{center}
\begin{tabular}{l}
|\input{childdoc.def}|\\
|\childdocforwardprefix[|\textit{main}|]{|\textit{prefix}|}{|\textit{dest}|}|
\end{tabular}
\end{center}
%
the destination file is determined by a pattern
depending on the current file:
To make this work, the current file must be called
`{\textit{prefix}\hspace{0.2em}\textit{suffix}}'
with \textit{prefix} matching precisely the argument.
Processing is then passed on to the file
`{\textit{dest}\hspace{0.2em}\textit{suffix}}'.
Surely, the same effect is achieved by
directly specifying the
argument `{\textit{dest}\hspace{0.2em}\textit{suffix}}'
in the first form.
However, that requires to set up a different file
for each child. With the alternative form of the command
all these files can have exactly the same content
which simplifies setting them up and maintaining them.

For example, the following file |draft.tex|
with a compilation flag |\version| as described in \secref{sec:flags}
compiles the main document as a draft:
%
\begin{center}
\begin{tabular}{l}
|\def\version{draft}|\\
|\input{childdoc.def}|\\
|\childdocforward{|\textit{main}|}|
\end{tabular}
\end{center}
%
Likewise, the following files |final|\textit{nn}|.tex|
compile the final version of the child document
|child|\textit{nn}|.tex|:
%
\begin{center}
\begin{tabular}{l}
|\def\version{final}|\\
|\input{childdoc.def}|\\
|\childdocforwardprefix{final}{child}|
\end{tabular}
\end{center}
%

Note that when several versions of a main file and/or of each child file
are to be generated, it may be convenient to set up a |Makefile| or
shell script to automatise the process.

%%%%%%%%%%%%%%%%%%%%%%%%%%%%%%%%%%%%%%%%%%%%%%%%%%%%%%%%%%%%%%%%%%%%%%%%%%%%%%%%
\subsection{Command Line Processing}
\label{sec:commandline}

The effect of redirection files can also be achieved by invoking
the \LaTeX{} compiler with a more elaborate command line.
Most conveniently this should be done as part
of a shell script or a |Makefile|.

When using \textsf{childdoc} in the main file, the following
command lines effectively perform a redirection
(note that depending on the shell being used,
backslashes may have to be doubled: `|\|' $\to$ `|\\|'):
%
\begin{center}
|... -jobname "|\textit{target}|" |\\|"|[\textit{flags}]%
|\input{childdoc.def}\childdocforward[|\textit{main}|]{|\textit{dest}|}"|
\end{center}
%
Here \textit{target} is the name of the output file,
\textit{main} is the name of the main file
and \textit{dest} is the name of the main or child file to be processed
(all filenames without extensions).
The optional argument \textit{main} can be omitted
if \textit{main} matches \textit{dest}.
Optionally, compilation \textit{flags} can be defined via |\def| commands.
This command line makes the \TeX{} engine believe
it is compiling the file \textit{target}
whose content is specified as the latter parameter.
The provided code then forwards the processing to
\textit{main} or \textit{dest} as described in \secref{sec:forward}.

%%%%%%%%%%%%%%%%%%%%%%%%%%%%%%%%%%%%%%%%%%%%%%%%%%%%%%%%%%%%%%%%%%%%%%%%%%%%%%%%
\subsection{Include by Input}
\label{sec:input}

Including child documents by |\include| has some restrictions by design.
Most notably, the content of a child document always occupies
its own set of pages; pages cannot be shared between child documents.
Usually, this behaviour makes perfect sense
because each child document contain an essential part of the document.
However, in some situations it may be desirable to compose
a document from a collection of parts
without having mandatory page breaks between then.
For this case, the package
provides a mechanism to include parts
by |\input| which can also be processed individually.
However, by construction this mechanism
requires manual handling of the content to be output.

%%%%%%%%%%%%%%%%%%%%%%%%%%%%%%%%%%%%%%%%
\DescribeMacro{\ifchilddocmanual}
The main file should be prepared as usual, see \secref{sec:include}.
However, the document body must make a distinction
between processing of an individual part and of the main document, e.g.:
%
\begin{center}
\begin{tabular}{l}
|\ifchilddocmanual|\\
|\input{\childdocname}|\\
|\||else|\\
\textit{document body with }|\input{|\textit{part}|}|\\
|\||fi|
\end{tabular}
\end{center}
%
The conditional |\ifchilddocmanual| is true whenever
a part to be included by |\input| is being compiled,
and the name of the part is stored in |\childdocname|.

%%%%%%%%%%%%%%%%%%%%%%%%%%%%%%%%%%%%%%%%
\DescribeMacro{\childdocby}
Each part to be included by |\input| should start with:
%
\begin{center}
\begin{tabular}{l}
|\input{childdoc.def}|\\
|\childdocby{|\textit{main}|}|\\
\end{tabular}
\end{center}
%
The directive |\childdocby| is similar to |\childdocof|
described in \secref{sec:include},
but the subsequent selection of content must be done manually.
To that end, both |\ifchilddoc| and |\ifchilddocmanual|
will be true upon processing of a part,
and the name of the part is stored in |\childdocname|.
Note that |\jobname| will be set to the filename of the current part
so that each part receives an individual |.aux| file
that does not interfere with the |.aux| file(s) of the main document.
This behaviour can be altered by the alternative form
|\childdocby[*]{|\textit{main}|}| (with a non-empty optional argument)
which uses the |.aux| file of the main document
by setting |\jobname| to \textit{main}.

%%%%%%%%%%%%%%%%%%%%%%%%%%%%%%%%%%%%%%%%%%%%%%%%%%%%%%%%%%%%%%%%%%%%%%%%%%%%%%%%
\subsection{Driver Development}
\label{sec:driver}

The \textsf{childdoc} mechanism can also be use for the development
of definition files such as \LaTeX{} styles or classes.
This case differs from the above setup with multiple parts
included by |\include| in that no |\includeonly| should be invoked.
This can be achieved by starting the include file
(before |\ProvidesPackage|) with:
%
\begin{center}
\begin{tabular}{l}
|\input{childdoc.def}|\\
|\childdocforward{|\textit{main}|}|\\
\end{tabular}
\end{center}
%
or alternatively with:
%
\begin{center}
\begin{tabular}{l}
|\input{childdoc.def}|\\
|\childdocby{|\textit{main}|}|\\
\end{tabular}
\end{center}
%
Both forms have slightly different effects as described above.
The main file is prepared as usual, see \secref{sec:include}.

%%%%%%%%%%%%%%%%%%%%%%%%%%%%%%%%%%%%%%%%%%%%%%%%%%%%%%%%%%%%%%%%%%%%%%%%%%%%%%%%
\subsection{Legacy Detection}
\label{sec:detection}

The directive |\childdocmain| in the main file can detect
whether the complete document or merely a child is to be compiled
even without using the directive |\childdocof|.
This method is deprecated because it is less robust
and there is no compelling reason to use it;
it is merely provided for backward compatibility
and it may be removed in future versions.

If the detection mechanism is to be used,
it is mandatory to correctly specify
the filename of the main file as the argument of |\childdocmain|:
%
\begin{center}
\begin{tabular}{l}
|\input{childdoc.def}|\\
|\childdocmain{|\textit{main}|}|\\
\end{tabular}
\end{center}
%
If |\jobname| does not match the argument \textit{main} of |\childdocmain|,
it is assumed that |\jobname| points to the child file to be compiled.
When using |\childdocmain| with the main file specified as argument,
it suffices to start a child file
with just |\input{|\textit{main}|}|
without loading of the package and using |\childdocof|.
If instead all processing is done
with the appropriate \textsf{childdoc} directives,
the argument of \textit{main} of |\childdocmain| can be empty.

An alternative version of the command line processing described
in \secref{sec:commandline} using the detection mechanism reads:
%
\begin{center}
|... -jobname "|\textit{target}|" "|[\textit{flags}]%
[|\def\jobname{|\textit{dest}|}|]|\input{|\textit{main}|}"|
\end{center}

%%%%%%%%%%%%%%%%%%%%%%%%%%%%%%%%%%%%%%%%%%%%%%%%%%%%%%%%%%%%%%%%%%%%%%%%%%%%%%%%
\subsection{Manual Code}
\label{sec:manual}

In case one cannot be certain whether the definitions file |childdoc.def|
is installed on the target \TeX{} distribution
and one prefers not to ship it,
it is conceivable to paste a few relevant commands into the sources.

To that end, drop all statements |\input{childdoc.def}|
and perform the replacements as outlined below.
Instead of |\childdocmain{|\textit{main}|}| add the following code
to the top of the main file:
%
\begin{center}
\begin{tabular}{l}
|\||ifdefined\childdocname\endinput\||fi\newif\ifchilddoc|\\
|\edef\childdocname{\scantokens\expandafter{\jobname\noexpand}}|\\
|\def\childdocmain{|\textit{main}|}\||ifx\childdocmain\childdocname\||else|\\
|\childdoctrue\includeonly{\childdocname}\let\jobname\childdocmain\||fi|\\
\end{tabular}
\end{center}
%
Instead of |\childdocof{|\textit{main}|}| just include the main file
at the top of each child file:
%
\begin{center}
|\input{|\textit{main}|}|
\end{center}
%
A simple redirection |\childdocforward{|\textit{dest}|}| is achieved by:
%
\begin{center}
|\def\jobname{|\textit{dest}|}\input{\jobname}|
\end{center}
%
The redirection with prefix
|\childdocforwardprefix[|\textit{prefix}|]{|\textit{dest}|}|
is accomplished by:
%
\begin{center}
\begin{tabular}{l}
|{\edef\jobname{\scantokens\expandafter{\jobname\noexpand}}|\\
|\def\redirectjob |\textit{prefix}|#1~~~{\gdef\jobname{|\textit{dest}|#1}}|\\
|\expandafter\redirectjob\jobname~~~}\input{\jobname}|
\end{tabular}
\end{center}

In an alternative approach,
child documents can be compiled by a specific command line
without additional code or specific definitions:
%
\begin{center}
|... -jobname "|\textit{target}|" "|[\textit{flags}]%
|\includeonly{|\textit{dest}|}\input{|\textit{main}|}"|
\end{center}
%

%%%%%%%%%%%%%%%%%%%%%%%%%%%%%%%%%%%%%%%%%%%%%%%%%%%%%%%%%%%%%%%%%%%%%%%%%%%%%%%%
%%%%%%%%%%%%%%%%%%%%%%%%%%%%%%%%%%%%%%%%%%%%%%%%%%%%%%%%%%%%%%%%%%%%%%%%%%%%%%%%
\section{Information}

%%%%%%%%%%%%%%%%%%%%%%%%%%%%%%%%%%%%%%%%%%%%%%%%%%%%%%%%%%%%%%%%%%%%%%%%%%%%%%%%
\subsection{Copyright}

Copyright \copyright{} 2017--2018 Niklas Beisert

This work may be distributed and/or modified under the
conditions of the \LaTeX{} Project Public License, either version 1.3
of this license or (at your option) any later version.
The latest version of this license is in
  \url{http://www.latex-project.org/lppl.txt}
and version 1.3 or later is part of all distributions of \LaTeX{}
version 2005/12/01 or later.

This work has the LPPL maintenance status `maintained'.

The Current Maintainer of this work is Niklas Beisert.

This work consists of the files |README.txt|, |childdoc.ins| and |childdoc.dtx|
as well as the derived files |childdoc.def|, |cdocsamp.tex|
with |cdocsch1.tex|, |cdocsch2.tex|, |cdocspt3.tex|, |cdocspt4.tex|,
|cdocsdrf.tex|, |cdocsfn1.tex|, |cdocsfn2.tex|
as well as |childdoc.pdf|.

%%%%%%%%%%%%%%%%%%%%%%%%%%%%%%%%%%%%%%%%%%%%%%%%%%%%%%%%%%%%%%%%%%%%%%%%%%%%%%%%
\subsection{Files and Installation}

The package consists of the files:
%
\begin{center}
\begin{tabular}{ll}
    |README.txt|   & readme file \\
    |childdoc.ins| & installation file \\
    |childdoc.dtx| & source file \\
    |childdoc.def| & definition file \\
    |cdocsamp.tex| & sample main file \\
    |cdocsch1.tex| & sample include file \\
    |cdocsch2.tex| & sample include file \\
    |cdocspt3.tex| & sample part file \\
    |cdocspt4.tex| & sample part file \\
    |cdocsdrf.tex| & sample redirection file \\
    |cdocsfn1.tex| & sample redirection file \\
    |cdocsfn2.tex| & sample redirection file \\
    |childdoc.pdf| & manual
\end{tabular}
\end{center}
%
The distribution consists of the files
|README.txt|, |childdoc.ins| and |childdoc.dtx|.
%
\begin{itemize}
\item
Run (pdf)\LaTeX{} on |childdoc.dtx|
to compile the manual |childdoc.pdf| (this file).
\item
Run \LaTeX{} on |childdoc.ins| to create the definitions file |childdoc.def|
and the sample |cdocsamp.tex| with include files
|cdocsch1.tex|, |cdocsch2.tex|, |cdocspt3.tex|, |cdocspt4.tex|,
|cdocsdrf.tex|, |cdocsfn1.tex|, |cdocsfn2.tex|.
Then copy the file |childdoc.def| to an appropriate directory of your \LaTeX{}
distribution, e.g.\ \textit{texmf-root}|/tex/latex/childdoc|.
\end{itemize}

%%%%%%%%%%%%%%%%%%%%%%%%%%%%%%%%%%%%%%%%%%%%%%%%%%%%%%%%%%%%%%%%%%%%%%%%%%%%%%%%
\subsection{Related CTAN Packages}

There are several other packages which offer a similar functionality:
%
\begin{itemize}
\item
The packages
\href{http://ctan.org/pkg/docmute}{\textsf{docmute}},
\href{http://ctan.org/pkg/includex}{\textsf{includex}} and
\href{http://ctan.org/pkg/standalone}{\textsf{standalone}}
provide commands to include only the document body of
a child file thus allowing both files to be compiled individually.
\item
The packages \href{http://ctan.org/pkg/subdocs}{\textsf{subdocs}}
and \href{http://ctan.org/pkg/subfiles}{\textsf{subfiles}}
provide structures in which the main and child documents can be
encapsulated and allowing them to be compiled individually.
The inclusion mechanism is different from the conventional |\include|.
\item
The package \href{http://ctan.org/pkg/combine}{\textsf{combine}}
is an elaborate solution to combine several documents into one.
\end{itemize}
%
See also the CTAN topic \href{http://ctan.org/topic/subdocs}{\textsf{subdocs}}
for further related packages.
The present package differs from the above solutions in that
a document structure constructed with the conventional |\include| mechanism
just needs two extra commands at the top of every file
such that all constituent files can be compiled individually.

%%%%%%%%%%%%%%%%%%%%%%%%%%%%%%%%%%%%%%%%%%%%%%%%%%%%%%%%%%%%%%%%%%%%%%%%%%%%%%%%
%\subsection{Feature Suggestions}
%
%The following is a list of features which may be useful for future
%versions of this package:
%%
%\begin{itemize}
%\item
%\ldots
%\end{itemize}

%%%%%%%%%%%%%%%%%%%%%%%%%%%%%%%%%%%%%%%%%%%%%%%%%%%%%%%%%%%%%%%%%%%%%%%%%%%%%%%%
\subsection{Revision History}

%%%%%%%%%%%%%%%%%%%%%%%%%%%%%%%%%%%%%%%%
\paragraph{v2.0:} 2018/12/30

\begin{itemize}
\item
immediate forward processing
\item
added |\childdocby| mechanism
\item
manual restructured
\end{itemize}

%%%%%%%%%%%%%%%%%%%%%%%%%%%%%%%%%%%%%%%%
\paragraph{v1.6:} 2018/01/17

\begin{itemize}
\item
application for development of include files
\item
corrections to manual
\end{itemize}

%%%%%%%%%%%%%%%%%%%%%%%%%%%%%%%%%%%%%%%%
\paragraph{v1.5:} 2017/05/21

\begin{itemize}
\item
more complete structuring introduced
\item
|\childdocof| introduced
\item
|\childdoc| renamed to |\childdocmain|
\item
|\childredirect| renamed to |\childdocforward| and |\childdocforwardprefix|
and functionality expanded
\end{itemize}

%%%%%%%%%%%%%%%%%%%%%%%%%%%%%%%%%%%%%%%%
\paragraph{v1.0:} 2017/04/27

\begin{itemize}
\item
manual and install package
\item
first version published on CTAN
\end{itemize}

%%%%%%%%%%%%%%%%%%%%%%%%%%%%%%%%%%%%%%%%
\paragraph{v0.6:} 2017/04/26

\begin{itemize}
\item
redirection mechanism added
\end{itemize}

%%%%%%%%%%%%%%%%%%%%%%%%%%%%%%%%%%%%%%%%
\paragraph{v0.5:} 2017/04/26

\begin{itemize}
\item
functionality in definition file
\end{itemize}


%%%%%%%%%%%%%%%%%%%%%%%%%%%%%%%%%%%%%%%%%%%%%%%%%%%%%%%%%%%%%%%%%%%%%%%%%%%%%%%%
%%%%%%%%%%%%%%%%%%%%%%%%%%%%%%%%%%%%%%%%%%%%%%%%%%%%%%%%%%%%%%%%%%%%%%%%%%%%%%%%
%%%%%%%%%%%%%%%%%%%%%%%%%%%%%%%%%%%%%%%%%%%%%%%%%%%%%%%%%%%%%%%%%%%%%%%%%%%%%%%%
\appendix

\settowidth\MacroIndent{\rmfamily\scriptsize 000\ }

 \DocInput{childdoc.dtx}

\end{document}
%</driver>
% \fi
%
% %%%%%%%%%%%%%%%%%%%%%%%%%%%%%%%%%%%%%%%%%%%%%%%%%%%%%%%%%%%%%%%%%%%%%%%%%%%%%%
% %%%%%%%%%%%%%%%%%%%%%%%%%%%%%%%%%%%%%%%%%%%%%%%%%%%%%%%%%%%%%%%%%%%%%%%%%%%%%%
% \section{Sample}
%\iffalse
%<*samplemain>
%\fi
%
% The following presents a sample document
% with two chapters, two parts, a title page,
% a compile flag as well as three forwarding files to set the flag.
% It consists of eight |.tex| files:
% \begin{center}
% \begin{tabular}{ll}
% |cdocsamp.tex|&main file\\
% |cdocsch1.tex|&include file for chapter 1\\
% |cdocsch2.tex|&include file for chapter 2\\
% |cdocspt3.tex|&include file for part 3\\
% |cdocspt4.tex|&include file for part 4\\
% |cdocsdrf.tex|&forwarding file for main file in draft mode\\
% |cdocsfi1.tex|&forwarding file for final version of chapter 1\\
% |cdocsfi2.tex|&forwarding file for final version of chapter 2\\
% \end{tabular}
% \end{center}
% Each of the eight files can be compiled directly by the \LaTeX{} compiler.
%
% %%%%%%%%%%%%%%%%%%%%%%%%%%%%%%%%%%%%%%
% \paragraph{Main File.}
%
% The main file is called |cdocsamp.tex|.
%
% Load the \textsf{childdoc} definitions and
% declare the filename for the main document:
%    \begin{macrocode}
\input{childdoc.def}
\childdocmain{}
%    \end{macrocode}

% Optional override for |\version| flag:
%    \begin{macrocode}
%%\ifchilddoc\else\providecommand{\version}{draft}\fi
%    \end{macrocode}

% Define the default values for the |\version| flag
% (|final| for the main file and |draft| for childs):
%    \begin{macrocode}
\ifchilddoc
\providecommand{\version}{draft}
\else
\providecommand{\version}{final}
\fi
%    \end{macrocode}

% Load the standard document class:
%    \begin{macrocode}
\documentclass[12pt]{article}
%    \end{macrocode}

% Start the document body:
%    \begin{macrocode}
\begin{document}
%    \end{macrocode}

% Declare a title page.
% Print title, part of document being processed and version flag:
%    \begin{macrocode}
\addtocounter{page}{-1}
\begin{center}
{\LARGE\bfseries{}childdoc example\par}
\vspace{1cm}
\ifchilddoc
\ifchilddocmanual part\else chapter\fi:
`\childdocname' of `\childdocjob'\par
\else
main document: `\childdocjob'\par
\fi
version: \version\par
\end{center}
\newpage
%    \end{macrocode}

% Manually include selected file,
% otherwise process as usual:
%    \begin{macrocode}
\ifchilddocmanual
\section*{part `\childdocname'}
\input{\childdocname}
\else
%    \end{macrocode}

% Include the two chapters:
%    \begin{macrocode}
\include{cdocsch1}
\include{cdocsch2}
%    \end{macrocode}

% Include the two parts unless only chapters should be displayed:
%    \begin{macrocode}
\ifchilddoc\else
\section{part three}
\input{cdocspt3}
\section{part four}
\input{cdocspt4}
\fi
%    \end{macrocode}

% Process as usual until here:
%    \begin{macrocode}
\fi
%    \end{macrocode}

% End of document body:
%    \begin{macrocode}
\end{document}
%    \end{macrocode}
%\iffalse
%</samplemain>
%\fi
%
% %%%%%%%%%%%%%%%%%%%%%%%%%%%%%%%%%%%%%%
% \paragraph{Chapter Include Files.}
%
% The include files are called |cdocsch1.tex| and |cdocsch2.tex|.
%
%\iffalse
%<*samplechap1|samplechap2>
%\fi

% Optional override for |\version| flag:
%    \begin{macrocode}
%%\providecommand{\version}{final}
%    \end{macrocode}

% Include the main document:
%    \begin{macrocode}
\input{childdoc.def}
\childdocof{cdocsamp}
%    \end{macrocode}

%\iffalse
%</samplechap1|samplechap2>
%\fi
%
%\iffalse
%<*samplechap1>
%\fi
% Some text for chapter 1:
%    \begin{macrocode}
\section{one}
some text in chapter one
%    \end{macrocode}

%\iffalse
%</samplechap1>
%\fi
% Some text for chapter 2:
%\iffalse
%<*samplechap2>
%\fi
%    \begin{macrocode}
\section{two}
more text in chapter two
%    \end{macrocode}

%\iffalse
%</samplechap2>
%\fi
%
% %%%%%%%%%%%%%%%%%%%%%%%%%%%%%%%%%%%%%%
% \paragraph{Part Include Files.}
%
% The include files are called |cdocspt3.tex| and |cdocspt4.tex|.
%
%\iffalse
%<*samplepart3|samplepart4>
%\fi

% Optional override for |\version| flag:
%    \begin{macrocode}
%%\providecommand{\version}{final}
%    \end{macrocode}

% Include the main document:
%    \begin{macrocode}
\input{childdoc.def}
\childdocby{cdocsamp}
%    \end{macrocode}

%\iffalse
%</samplepart3|samplepart4>
%\fi
%
%\iffalse
%<*samplepart3>
%\fi
% Some text for part 3:
%    \begin{macrocode}
some text in part three
%    \end{macrocode}

%\iffalse
%</samplepart3>
%\fi
% Some text for part 4:
%\iffalse
%<*samplepart4>
%\fi
%    \begin{macrocode}
more text in part four
%    \end{macrocode}

%\iffalse
%</samplepart4>
%\fi
%
% %%%%%%%%%%%%%%%%%%%%%%%%%%%%%%%%%%%%%%
% \paragraph{Forwarding for a Complete Draft.}
%
% The following forwarding file |cdocsdrf.tex|
% compiles the main document in draft mode:
%\iffalse
%<*sampledraft>
%\fi
%    \begin{macrocode}
\def\version{draft}
\input{childdoc.def}
\childdocforward{cdocsamp}
%    \end{macrocode}

%\iffalse
%</sampledraft>
%\fi
%
% %%%%%%%%%%%%%%%%%%%%%%%%%%%%%%%%%%%%%%
% \paragraph{Forwarding for Final Version of the Chapters.}
%
% The following forwarding files |cdocsfn1.tex| and |cdocsfn2.tex|
% (with identical content)
% compile the final versions of the child documents
% |cdocsch1.tex| and |cdocsch2.tex|, respectively:
%\iffalse
%<*samplefinal>
%\fi
%    \begin{macrocode}
\def\version{final}
\input{childdoc.def}
\childdocforwardprefix[cdocsamp]{cdocsfn}{cdocsch}
%    \end{macrocode}

%\iffalse
%</samplefinal>
%\fi
%
% %%%%%%%%%%%%%%%%%%%%%%%%%%%%%%%%%%%%%%
% \paragraph{Command Line Processing.}
%
% The following three command lines generate the output files
% |cdocscld|, |cdocscl1| and |cdocscl2|
% which should be identical to
% |cdocsdrf|, |cdocsch1| and |cdocsfn2|, respectively:
% \begin{center}
% \begin{tabular}{l}
% |latex -jobname cdocscld \|\\
% |  "\def\version{draft}\input{childdoc.def}\childdocforward{cdocsamp}"|\\
% |latex -jobname cdocscl1 \|\\
% |  "\input{childdoc.def}\childdocforward[cdocsamp]{cdocsch1}"|\\
% |latex -jobname cdocscl2 \|\\
% |  "\def\version{final}\input{childdoc.def}\childdocforward{cdocsch2}"|
% \end{tabular}
% \end{center}
% Note that the trailing backslash on each first line
% merely continues the input to the second line
% (for convenient cut ant paste).
% Furthermore, the command |latex| can be replaced by any
% of its alternative versions such as |pdflatex|.
%
% %%%%%%%%%%%%%%%%%%%%%%%%%%%%%%%%%%%%%%%%%%%%%%%%%%%%%%%%%%%%%%%%%%%%%%%%%%%%%%
% %%%%%%%%%%%%%%%%%%%%%%%%%%%%%%%%%%%%%%%%%%%%%%%%%%%%%%%%%%%%%%%%%%%%%%%%%%%%%%
% \section{Implementation}
%\iffalse
%<*package>
%\fi
%
% This section describes the definitions file |childdoc.def|.

% The definitions cannot be loaded using |\usepackage| or |\RequirePackage|
% which has a mechanism to prevent loading a style file more than once.
% When loading the definitions by means of |\input|
% multiple instances have to be prevented manually:
%\iffalse
%This code needs to be before the `\ProvidesFile' directive
%which is defined at the beginning of this file.
%Therefore it is also placed there and commented out here.
%</package>
%<*discard>
%\fi
%    \begin{macrocode}
\ifdefined\childdocmain\endinput\fi
%    \end{macrocode}
%\iffalse
%</discard>
%<*package>
%\fi
%
% \macro{\ifchilddoc}
% \macro{\ifchilddocmanual}
% The conditional |\ifchilddoc| tells whether a
% child (true) or main (false) document is being compiled.
% The conditional |\ifchilddocmanual| tells whether
% the |\includeonly| mechanism is used (false) or
% the selection of child files must be performed manually (true).
% The definitions initialise to false:
%    \begin{macrocode}
\newif\ifchilddoc
\newif\ifchilddocmanual
%    \end{macrocode}

% \macro{\childdocname}
% \macro{\childdocjob}
% The macro |\childdocname| stores the name of the main document
% to be compiled. The macro |\childdocjob| stores the name of
% the document on which the \LaTeX{} compiler was originally invoked.
% The content of |\jobname| cannot be compared
% to filenames specified in the source due to different catcodes.
% The following code rescans |\jobname|, stores the result
% in |\childdocname| and saves a copy in |\childdocjob|:
%    \begin{macrocode}
\edef\childdocname{\scantokens\expandafter{\jobname\noexpand}}
\let\childdocjob\childdocname
%    \end{macrocode}

% \macro{\childdocdisable}
% The macro |\childdocdisable| prevents the main file
% from being processed more than once.
% At this stage, the main document command |\childdocmain|
% is assumed to be called once again where it should do nothing.
% Any subsequent call to it should prevent
% a secondary processing of the main document
% It overwrites the forwarding commands
% |\childdocof| and |\childdocforward|
% with empty macros to prevent further inclusions of the main document:
%    \begin{macrocode}
\newcommand{\childdocdisable}
{
  \renewcommand{\childdocmain}[1]{\renewcommand{\childdocmain}[1]{\endinput}}
  \renewcommand{\childdocof}[1]{}
  \renewcommand{\childdocby}[2][]{}
  \renewcommand{\childdocforward}[2][]{}
  \renewcommand{\childdocdisable}{}
}
%    \end{macrocode}

% \macro{\childdocmain}
% The macro |\childdocmain| is to be called at the top of the main file
% with nothing or the main filename (without extension) as argument.
% First, it breaks loops.
% If the argument is not empty and does not match |\childdocname|
% (which is set by the first inclusion of |childdoc.def|),
% |\ifchilddoc| is set to true, |\includeonly| is applied to the child file
% and |\jobname| is set to the main file
% (for proper handling of |.aux| files):
%    \begin{macrocode}
\newcommand{\childdocmain}[1]
{
  \childdocdisable\childdocmain{}
  \if?#1?\else
    \begingroup
      \def\childdoctmp{#1}
      \ifx\childdoctmp\childdocname
        \def\childdoctmp{}
      \else
        \def\childdoctmp
        {
          \childdoctrue
          \includeonly{\childdocname}
          \def\childdocjob{#1}
          \def\jobname{#1}
        }
      \fi
      \expandafter
    \endgroup
    \childdoctmp
  \fi
}
%    \end{macrocode}

% \macro{\childdocof}
% The command |\childdocof| redirects
% compilation to the main file |#1|.
%    \begin{macrocode}
\newcommand{\childdocof}[1]
{
  \childdocdisable
  \childdoctrue
  \includeonly{\childdocname}
  \def\jobname{#1}
  \def\childdocjob{#1}
  \input{#1}
}
%    \end{macrocode}

% \macro{\childdocby}
% The command |\childdocby| ....
%    \begin{macrocode}
\newcommand{\childdocby}[2][]
{
  \childdocdisable
  \childdoctrue
  \childdocmanualtrue
  \if?#1?\else
    \def\jobname{#2}
  \fi
  \def\childdocjob{#2}
  \input{#2}
  \endinput
}
%    \end{macrocode}

% \macro{\childdocforward}
% The command |\childdocforward| redirects
% compilation to the main file or
% (if the optional argument is given) a child file.
% Parameters are set as if the main file
% or a child file starting with |\childdocof| was compiled.
% Then compilation is handed over to the main file:
%    \begin{macrocode}
\newcommand{\childdocforward}[2][]
{
  \begingroup
    \if?#1?
      \def\childdoctmp
      {
        \def\childdocname{#2}
        \def\childdocjob{#2}
        \def\jobname{#2}
        \input{#2}
        \endinput
      }
    \else
      \def\childdoctmp
      {
        \childdocdisable
        \def\childdocname{#2}
        \childdoctrue
        \includeonly{#2}
        \def\childdocjob{#1}
        \def\jobname{#1}
        \input{#1}
        \endinput
      }
    \fi
    \expandafter
  \endgroup
  \childdoctmp
}
%    \end{macrocode}

% \macro{\childdocforwardprefix}
% The command |\childdocforwardprefix| redirects
% compilation to the main or a child file by means of a pattern.
% The prefix |#1| in the current filename is replaced by |#2|
% and the suffix of the current filename is kept
% (it is assumed that the filename does not contain the substring `|~~~|'
% which is used as a delimiter).
% Compilation is handed over to the new file by |\childdocforward|:
%    \begin{macrocode}
\newcommand{\childdocforwardprefix}[3][]
{
  \begingroup
    \def\childdocextract #2##1~~~{\def\childdoctmp{\childdocforward[#1]{#3##1}}}
    \expandafter\childdocextract\childdocname~~~
    \expandafter
  \endgroup
  \childdoctmp
}
%    \end{macrocode}

% \macro{\childdoc}
% The deprecated macro |\childdoc| is a legacy version of |\childdocmain|:
%    \begin{macrocode}
\newcommand{\childdoc}{\childdocmain}
%    \end{macrocode}

% \macro{\childdocredirect}
% The deprecated macro |\childdocredirect| is a legacy version
% of |\childdocforward| and |\childdocforwardprefix|:
%    \begin{macrocode}
\newcommand{\childdocredirect}[2][]
{
  \begingroup
    \if?#1?
      \def\childdoctmp{\childdocforward{#2}}
    \else
      \def\childdoctmp{\childdocforwardprefix{#1}{#2}}
    \fi
    \expandafter
  \endgroup
  \childdoctmp
}
%    \end{macrocode}

%\iffalse
%</package>
%\fi
%
\endinput
\childdocforward[cdocsamp]{cdocsch1}"|\\
% |latex -jobname cdocscl2 \|\\
% |  "\def\version{final}% \iffalse
%
% childdoc.dtx Copyright (C) 2017-2018 Niklas Beisert
%
% This work may be distributed and/or modified under the
% conditions of the LaTeX Project Public License, either version 1.3
% of this license or (at your option) any later version.
% The latest version of this license is in
%   http://www.latex-project.org/lppl.txt
% and version 1.3 or later is part of all distributions of LaTeX
% version 2005/12/01 or later.
%
% This work has the LPPL maintenance status `maintained'.
%
% The Current Maintainer of this work is Niklas Beisert.
%
% This work consists of the files childdoc.dtx and childdoc.ins
% and the derived files childdoc.def and cdocsamp.tex with
% cdocsch1.tex, cdocsch2.tex, cdocsdrf.tex, cdocsfn1.tex, cdocsfn2.tex.
%
%<package>\ifdefined\childdocmain\endinput\fi
%<package>\ProvidesFile{childdoc.def}[2018/12/30 v2.0 child document driver]
%<samplemain>\ProvidesFile{cdocsamp.tex}[2018/12/30 v2.0 sample for childdoc]
%<*driver>
%\ProvidesFile{childdoc.drv}[2018/12/30 v2.0 childdoc reference manual file]
\PassOptionsToClass{10pt,a4paper}{article}
\documentclass{ltxdoc}

\usepackage[margin=35mm]{geometry}
\usepackage{hyperref}
\usepackage{hyperxmp}
\usepackage[usenames]{color}

\hypersetup{colorlinks=true}
\hypersetup{pdfstartview=FitH}
\hypersetup{pdfpagemode=UseNone}
\hypersetup{pdfsource={}}
\hypersetup{pdflang={en-UK}}
\hypersetup{pdfcopyright={Copyright 2017-2018 Niklas Beisert.
  This work may be distributed and/or modified under the
  conditions of the LaTeX Project Public License, either version 1.3
  of this license or (at your option) any later version.}}
\hypersetup{pdflicenseurl={http://www.latex-project.org/lppl.txt}}
\hypersetup{pdfcontactaddress={ETH Zurich, ITP, HIT K,
  Wolfgang-Pauli-Strasse 27}}
\hypersetup{pdfcontactpostcode={8093}}
\hypersetup{pdfcontactcity={Zurich}}
\hypersetup{pdfcontactcountry={Switzerland}}
\hypersetup{pdfcontactemail={nbeisert@itp.phys.ethz.ch}}
\hypersetup{pdfcontacturl={http://people.phys.ethz.ch/\xmptilde nbeisert/}}

\newcommand{\secref}[1]{\hyperref[#1]{section \ref*{#1}}}

\parskip1ex
\parindent0pt
\let\olditemize\itemize
\def\itemize{\olditemize\parskip0pt}

\begin{document}

\title{The \textsf{childdoc} Package}
\hypersetup{pdftitle={The childdoc Package}}
\author{Niklas Beisert\\[2ex]
  Institut f\"ur Theoretische Physik\\
  Eidgen\"ossische Technische Hochschule Z\"urich\\
  Wolfgang-Pauli-Strasse 27, 8093 Z\"urich, Switzerland\\[1ex]
  \href{mailto:nbeisert@itp.phys.ethz.ch}
  {\texttt{nbeisert@itp.phys.ethz.ch}}}
\hypersetup{pdfauthor={Niklas Beisert}}
\hypersetup{pdfsubject={Manual for the LaTeX2e Package childdoc}}
\date{30 December 2018, \textsf{v2.0}}
\maketitle

\begin{abstract}\noindent
\textsf{childdoc} is a \LaTeXe{} package
that enables the direct compilation
of document sections included by |\include|
to individual files.
\end{abstract}

\begingroup
\parskip0ex
\tableofcontents
\endgroup

%%%%%%%%%%%%%%%%%%%%%%%%%%%%%%%%%%%%%%%%%%%%%%%%%%%%%%%%%%%%%%%%%%%%%%%%%%%%%%%%
%%%%%%%%%%%%%%%%%%%%%%%%%%%%%%%%%%%%%%%%%%%%%%%%%%%%%%%%%%%%%%%%%%%%%%%%%%%%%%%%
\section{Introduction}

\LaTeX{} provides a mechanism to structure a large document (such as a book)
into a main file and several child files (containing the chapters)
using the |\include| command.
This mechanism is beneficial for documents
which span hundreds of pages in order to
make the source file(s) more manageable.
Moreover, compilation can be restricted to
selected child files by means of the |\includeonly| command.
The latter feature can be used to reduce the compilation time while editing
(this was significantly more useful in the earlier days of \LaTeX{})
or to generate a smaller document which is easier to navigate.
Another application of |\includeonly| is to generate
documents consisting of selected parts of the complete document.

However, there are a few drawbacks of the plain |\include| mechanism:
\begin{itemize}
\item
The child files cannot be compiled on their own,
they can only be compiled via the main file.
A naive editing environment
(such as a text editor with an option
to have the current file processed by \LaTeX)
may require one to switch to the main file before compiling;
attempting to compile the child file produces errors.
\item
The main file must be modified (each time)
to adjust the |\includeonly| command
to the present needs. This easily leaves the main file in a messy state.
\item
The generated document will always carry the filename
of the main document. This is inconvenient if
several child files are to be compiled and
to be kept for distribution.
\end{itemize}

The present package provides a simple interface
to make child files individually compilable by \LaTeX{}.
Compiling a child file then has the same effect as compiling
the main file with an |\includeonly| command
to select the appropriate child.
Moreover the generated document will carry the name of the child
rather than the main file.
This resolves all three above issues.

This feature is meant to make the editing of books,
thesis documents and lecture notes somewhat more convenient.
However, the package can also be used efficiently for
composing a series of documents (such as exercise sheets)
which are typically distributed individually.
It then assists the author in generating the individual documents
(potentially in different versions)
as well as a document containing the collected series.
Another application is in developing style files
or other kinds of included material
where compilation of the style file could redirect
to a sample or test file.

%%%%%%%%%%%%%%%%%%%%%%%%%%%%%%%%%%%%%%%%%%%%%%%%%%%%%%%%%%%%%%%%%%%%%%%%%%%%%%%%
%%%%%%%%%%%%%%%%%%%%%%%%%%%%%%%%%%%%%%%%%%%%%%%%%%%%%%%%%%%%%%%%%%%%%%%%%%%%%%%%
\section{Usage}

First of all, the package \textsf{childdoc} is \emph{not} a standard
\LaTeXe{} |.sty| style file! Therefore it needs to be invoked in
a non-standard way.

%%%%%%%%%%%%%%%%%%%%%%%%%%%%%%%%%%%%%%%%%%%%%%%%%%%%%%%%%%%%%%%%%%%%%%%%%%%%%%%%
\subsection{Included Files}
\label{sec:include}

%%%%%%%%%%%%%%%%%%%%%%%%%%%%%%%%%%%%%%%%
\DescribeMacro{\childdocmain}
To use the package, add the commands
\begin{center}
\begin{tabular}{l}
|\input{childdoc.def}|\\
|\childdocmain{}|\\
\end{tabular}
\end{center}
at the very top of the main \LaTeX{} file,
in particular \emph{before} the |\documentclass| statement!
The argument of |\childdocmain| should be left empty
(but it must be present).

%%%%%%%%%%%%%%%%%%%%%%%%%%%%%%%%%%%%%%%%
\DescribeMacro{\childdocof}
Furthermore, add the commands
\begin{center}
\begin{tabular}{l}
|\input{childdoc.def}|\\
|\childdocof{|\textit{main}|}|\\
\end{tabular}
\end{center}
at the top of every child file \textit{child}
which is included by |\include{|\textit{child}|}|
from within the main file
(or at least for those files to be compiled individually).
The argument \textit{main} must be the filename of the main file.

There are a couple of
considerations in setting up the main and child documents:

%%%%%%%%%%%%%%%%%%%%%%%%%%%%%%%%%%%%%%%%
\paragraph{Restrictions.}

Please note the following restrictions:
\begin{itemize}
\item
|\childdocmain| must be called with one argument \textit{main}
to ensure compatibility with earlier version of the package.
It must either be empty (|\childdocmain{}|)
or precisely match the filename of the main file in which it is specified.
See \secref{sec:detection} for further information.
\item
The filename \textit{main} must be specified without the |.tex| extension.
\item
The filename \textit{main} is case sensitive
(even in case-insensitive file systems)
due to internal string comparison.
\item
The argument \textit{main} should be fully expanded, it cannot be a macro.
\item
Subdirectories and special characters should be avoided in filenames.
\item
The command |\childdocmain{|\textit{main}|}| must be followed by a whitespace.
It should not be followed immediately by another command
or by a comment mark `|%|'.
This is because the \TeX{} parser reads the token immediately following
the argument of |\childdocmain| and puts it
at the beginning of every child section;
however, a white\-space is ignored.
\end{itemize}

%%%%%%%%%%%%%%%%%%%%%%%%%%%%%%%%%%%%%%%%
\paragraph{Content of Main File.}

It is advisable to place all content in the child files included by |\include|.
Any output contained in the main file will appear in all child documents
unless suppressed manually;
it cannot be suppressed automatically by the |\includeonly| directive
and thus should normally be avoided.
A method to include some content in the main file
by means of conditional processing is described in \secref{sec:conditional}.

%%%%%%%%%%%%%%%%%%%%%%%%%%%%%%%%%%%%%%%%
\paragraph{Page Numbering.}

When only a part of the document is compiled,
the appropriate numbering of pages
(as well as other status parameters)
is determined from the |.aux| files.
The latter contain information from previous passes.
However this information needs to propagate through
all intermediate child documents.
Therefore the page numbering in child documents may well
be inconsistent until the complete document is compiled at least once.

A useful (if unconventional) way to always ensure a consistent
page numbering is to restart the numbering in each child document
and denote the pages by `\textit{child}|.|\textit{page}'
where \textit{child} represents the chapter/section number of the child file.
This can be achieved by the command
|\numberwithin{page}{|\textit{child}|}|
of the \textsf{amsmath} package
where \textit{child} can be |chapter| or |section|
depending on the chosen structuring.
Alternatively, one can modify the macro |\thepage| appropriately
and reset the counter |page| at the start of each child file.

%%%%%%%%%%%%%%%%%%%%%%%%%%%%%%%%%%%%%%%%%%%%%%%%%%%%%%%%%%%%%%%%%%%%%%%%%%%%%%%%
\subsection{Conditional Processing}
\label{sec:conditional}

The package provides a mechanism to compile different versions
of a document. To customise the versions further some conditional processing
can come in handy to distinguish which version is being compiled.
The package provides two macros to describe the compilation context:

%%%%%%%%%%%%%%%%%%%%%%%%%%%%%%%%%%%%%%%%
\DescribeMacro{\ifchilddoc}
The conditional |\ifchilddoc| distinguishes between the compilation of
child documents and the main document:
%
\begin{center}
|\ifchilddoc |\textit{child-code}| |[|\||else |\textit{main-code}]| \||fi|
\end{center}

%%%%%%%%%%%%%%%%%%%%%%%%%%%%%%%%%%%%%%%%
\DescribeMacro{\childdocname}
\DescribeMacro{\childdocjob}
The macro |\childdocname| contains the filename (without extension)
of the main or child file being processed.
Note that |\childdocjob| will always contain the name of the main file.

%%%%%%%%%%%%%%%%%%%%%%%%%%%%%%%%%%%%%%%%
\paragraph{Title Page.}

Conditional processing can be used to include a title or banner page
in the main document when proper precautions are taken.
Importantly, the code in the main file should ensure that the page counter
(as well as other status parameters which are stored in the |.aux| files)
takes the same value after the conditional processing.
Otherwise the page numbers may take divergent values
depending on which part is compiled.

For example, a title page could be declared by:
%
\begin{center}
\begin{tabular}{l}
|\ifchilddoc\||else|\\
|\addtocounter{page}{-1}|\\
\textit{code for title page}\\
|\newpage|\\
|\||fi|
\end{tabular}
\end{center}
%
A banner page for the child documents can be generated by:
%
\begin{center}
\begin{tabular}{l}
|\ifchilddoc|\\
|\addtocounter{page}{-1}|\\
\textit{code for banner page}\\
|\newpage|\\
|\||fi|
\end{tabular}
\end{center}
%
Here one could write a message such as:
\begin{center}
|This is the part \childdocname{} of \childdocjob{}.|
\end{center}

%%%%%%%%%%%%%%%%%%%%%%%%%%%%%%%%%%%%%%%%%%%%%%%%%%%%%%%%%%%%%%%%%%%%%%%%%%%%%%%%
\subsection{Flags}
\label{sec:flags}

The package makes it easy to generate different versions
of the main or child documents.
To this end compilation flags can be defined
and assigned different default values.
They will be particularly useful in conjunction
with the forwarding mechanism described in \secref{sec:forward}.

For example, it may be useful to have a flag |\version|
which can be set to |draft| or |final|.
The document source will contain some conditional code
depending on the value of |\version|.
Suppose further, the flag should default to |final| for the main file
and to |draft| for child files
which is a natural assignment for editing the document.
This is achieved by placing the following code
in the preamble of the main document
(below the |\childdocmain| directive):
%
\begin{center}
\begin{tabular}{l}
|\ifchilddoc|\\
|\providecommand{\version}{draft}|\\
|\||else|\\
|\providecommand{\version}{final}|\\
|\||fi|
\end{tabular}
\end{center}
%
The definition by |\providecommand| makes sure
that previous definitions are not overwritten.
Further statements |\providecommand{\version}{...}|
can thus be added before the above code to override it.

For the main file, one might add a line
(between |\childdocmain| and the above block)
%
\begin{center}
|%\ifchilddoc\||else\providecommand{\version}{draft}\||fi|
\end{center}
%
which can be uncommented to produce a draft version.
Likewise one can add a line to the very top of a child file
(above the |\childdocof{|\textit{main}|}| directive)
%
\begin{center}
|%\providecommand{\version}{final}|
\end{center}
%
which can be uncommented to produce the final version of this child document.

%%%%%%%%%%%%%%%%%%%%%%%%%%%%%%%%%%%%%%%%%%%%%%%%%%%%%%%%%%%%%%%%%%%%%%%%%%%%%%%%
\subsection{Forwarding}
\label{sec:forward}

Different versions of the main or child documents
using compilation flags as described in \secref{sec:flags}
can be (permanently) stored in different files
for convenient compilation, viewing and distribution.
To this end, the package defines a command
to pass on compilation to a different file:

%%%%%%%%%%%%%%%%%%%%%%%%%%%%%%%%%%%%%%%%
\DescribeMacro{\childdocforward}
The command |\childdocforward| redirects processing to
another source file:
%
\begin{center}
\begin{tabular}{l}
|\input{childdoc.def}|\\
|\childdocforward[|\textit{main}|]{|\textit{dest}|}|\\
\end{tabular}
\end{center}
%
The argument \textit{dest} is the destination file
(without extension).
It should be the main file or one of the child files.
Note that further \textsf{childdoc} directives
such as |\childdocof| and |\childdocforward|
in the indicated file will be processed in this form.
The optional argument \textit{main}
passes on directly to the main file \textit{main}
while pretending to compile the child \textit{dest}.
This form behaves as if \textit{dest}
issues |\childdocof{|\textit{main}|}| right away,
and no further \textsf{childdoc} directives will be processed.

%%%%%%%%%%%%%%%%%%%%%%%%%%%%%%%%%%%%%%%%
\DescribeMacro{\...prefix}
In the alternative form |\childdocforwardprefix|,
%
\begin{center}
\begin{tabular}{l}
|\input{childdoc.def}|\\
|\childdocforwardprefix[|\textit{main}|]{|\textit{prefix}|}{|\textit{dest}|}|
\end{tabular}
\end{center}
%
the destination file is determined by a pattern
depending on the current file:
To make this work, the current file must be called
`{\textit{prefix}\hspace{0.2em}\textit{suffix}}'
with \textit{prefix} matching precisely the argument.
Processing is then passed on to the file
`{\textit{dest}\hspace{0.2em}\textit{suffix}}'.
Surely, the same effect is achieved by
directly specifying the
argument `{\textit{dest}\hspace{0.2em}\textit{suffix}}'
in the first form.
However, that requires to set up a different file
for each child. With the alternative form of the command
all these files can have exactly the same content
which simplifies setting them up and maintaining them.

For example, the following file |draft.tex|
with a compilation flag |\version| as described in \secref{sec:flags}
compiles the main document as a draft:
%
\begin{center}
\begin{tabular}{l}
|\def\version{draft}|\\
|\input{childdoc.def}|\\
|\childdocforward{|\textit{main}|}|
\end{tabular}
\end{center}
%
Likewise, the following files |final|\textit{nn}|.tex|
compile the final version of the child document
|child|\textit{nn}|.tex|:
%
\begin{center}
\begin{tabular}{l}
|\def\version{final}|\\
|\input{childdoc.def}|\\
|\childdocforwardprefix{final}{child}|
\end{tabular}
\end{center}
%

Note that when several versions of a main file and/or of each child file
are to be generated, it may be convenient to set up a |Makefile| or
shell script to automatise the process.

%%%%%%%%%%%%%%%%%%%%%%%%%%%%%%%%%%%%%%%%%%%%%%%%%%%%%%%%%%%%%%%%%%%%%%%%%%%%%%%%
\subsection{Command Line Processing}
\label{sec:commandline}

The effect of redirection files can also be achieved by invoking
the \LaTeX{} compiler with a more elaborate command line.
Most conveniently this should be done as part
of a shell script or a |Makefile|.

When using \textsf{childdoc} in the main file, the following
command lines effectively perform a redirection
(note that depending on the shell being used,
backslashes may have to be doubled: `|\|' $\to$ `|\\|'):
%
\begin{center}
|... -jobname "|\textit{target}|" |\\|"|[\textit{flags}]%
|\input{childdoc.def}\childdocforward[|\textit{main}|]{|\textit{dest}|}"|
\end{center}
%
Here \textit{target} is the name of the output file,
\textit{main} is the name of the main file
and \textit{dest} is the name of the main or child file to be processed
(all filenames without extensions).
The optional argument \textit{main} can be omitted
if \textit{main} matches \textit{dest}.
Optionally, compilation \textit{flags} can be defined via |\def| commands.
This command line makes the \TeX{} engine believe
it is compiling the file \textit{target}
whose content is specified as the latter parameter.
The provided code then forwards the processing to
\textit{main} or \textit{dest} as described in \secref{sec:forward}.

%%%%%%%%%%%%%%%%%%%%%%%%%%%%%%%%%%%%%%%%%%%%%%%%%%%%%%%%%%%%%%%%%%%%%%%%%%%%%%%%
\subsection{Include by Input}
\label{sec:input}

Including child documents by |\include| has some restrictions by design.
Most notably, the content of a child document always occupies
its own set of pages; pages cannot be shared between child documents.
Usually, this behaviour makes perfect sense
because each child document contain an essential part of the document.
However, in some situations it may be desirable to compose
a document from a collection of parts
without having mandatory page breaks between then.
For this case, the package
provides a mechanism to include parts
by |\input| which can also be processed individually.
However, by construction this mechanism
requires manual handling of the content to be output.

%%%%%%%%%%%%%%%%%%%%%%%%%%%%%%%%%%%%%%%%
\DescribeMacro{\ifchilddocmanual}
The main file should be prepared as usual, see \secref{sec:include}.
However, the document body must make a distinction
between processing of an individual part and of the main document, e.g.:
%
\begin{center}
\begin{tabular}{l}
|\ifchilddocmanual|\\
|\input{\childdocname}|\\
|\||else|\\
\textit{document body with }|\input{|\textit{part}|}|\\
|\||fi|
\end{tabular}
\end{center}
%
The conditional |\ifchilddocmanual| is true whenever
a part to be included by |\input| is being compiled,
and the name of the part is stored in |\childdocname|.

%%%%%%%%%%%%%%%%%%%%%%%%%%%%%%%%%%%%%%%%
\DescribeMacro{\childdocby}
Each part to be included by |\input| should start with:
%
\begin{center}
\begin{tabular}{l}
|\input{childdoc.def}|\\
|\childdocby{|\textit{main}|}|\\
\end{tabular}
\end{center}
%
The directive |\childdocby| is similar to |\childdocof|
described in \secref{sec:include},
but the subsequent selection of content must be done manually.
To that end, both |\ifchilddoc| and |\ifchilddocmanual|
will be true upon processing of a part,
and the name of the part is stored in |\childdocname|.
Note that |\jobname| will be set to the filename of the current part
so that each part receives an individual |.aux| file
that does not interfere with the |.aux| file(s) of the main document.
This behaviour can be altered by the alternative form
|\childdocby[*]{|\textit{main}|}| (with a non-empty optional argument)
which uses the |.aux| file of the main document
by setting |\jobname| to \textit{main}.

%%%%%%%%%%%%%%%%%%%%%%%%%%%%%%%%%%%%%%%%%%%%%%%%%%%%%%%%%%%%%%%%%%%%%%%%%%%%%%%%
\subsection{Driver Development}
\label{sec:driver}

The \textsf{childdoc} mechanism can also be use for the development
of definition files such as \LaTeX{} styles or classes.
This case differs from the above setup with multiple parts
included by |\include| in that no |\includeonly| should be invoked.
This can be achieved by starting the include file
(before |\ProvidesPackage|) with:
%
\begin{center}
\begin{tabular}{l}
|\input{childdoc.def}|\\
|\childdocforward{|\textit{main}|}|\\
\end{tabular}
\end{center}
%
or alternatively with:
%
\begin{center}
\begin{tabular}{l}
|\input{childdoc.def}|\\
|\childdocby{|\textit{main}|}|\\
\end{tabular}
\end{center}
%
Both forms have slightly different effects as described above.
The main file is prepared as usual, see \secref{sec:include}.

%%%%%%%%%%%%%%%%%%%%%%%%%%%%%%%%%%%%%%%%%%%%%%%%%%%%%%%%%%%%%%%%%%%%%%%%%%%%%%%%
\subsection{Legacy Detection}
\label{sec:detection}

The directive |\childdocmain| in the main file can detect
whether the complete document or merely a child is to be compiled
even without using the directive |\childdocof|.
This method is deprecated because it is less robust
and there is no compelling reason to use it;
it is merely provided for backward compatibility
and it may be removed in future versions.

If the detection mechanism is to be used,
it is mandatory to correctly specify
the filename of the main file as the argument of |\childdocmain|:
%
\begin{center}
\begin{tabular}{l}
|\input{childdoc.def}|\\
|\childdocmain{|\textit{main}|}|\\
\end{tabular}
\end{center}
%
If |\jobname| does not match the argument \textit{main} of |\childdocmain|,
it is assumed that |\jobname| points to the child file to be compiled.
When using |\childdocmain| with the main file specified as argument,
it suffices to start a child file
with just |\input{|\textit{main}|}|
without loading of the package and using |\childdocof|.
If instead all processing is done
with the appropriate \textsf{childdoc} directives,
the argument of \textit{main} of |\childdocmain| can be empty.

An alternative version of the command line processing described
in \secref{sec:commandline} using the detection mechanism reads:
%
\begin{center}
|... -jobname "|\textit{target}|" "|[\textit{flags}]%
[|\def\jobname{|\textit{dest}|}|]|\input{|\textit{main}|}"|
\end{center}

%%%%%%%%%%%%%%%%%%%%%%%%%%%%%%%%%%%%%%%%%%%%%%%%%%%%%%%%%%%%%%%%%%%%%%%%%%%%%%%%
\subsection{Manual Code}
\label{sec:manual}

In case one cannot be certain whether the definitions file |childdoc.def|
is installed on the target \TeX{} distribution
and one prefers not to ship it,
it is conceivable to paste a few relevant commands into the sources.

To that end, drop all statements |\input{childdoc.def}|
and perform the replacements as outlined below.
Instead of |\childdocmain{|\textit{main}|}| add the following code
to the top of the main file:
%
\begin{center}
\begin{tabular}{l}
|\||ifdefined\childdocname\endinput\||fi\newif\ifchilddoc|\\
|\edef\childdocname{\scantokens\expandafter{\jobname\noexpand}}|\\
|\def\childdocmain{|\textit{main}|}\||ifx\childdocmain\childdocname\||else|\\
|\childdoctrue\includeonly{\childdocname}\let\jobname\childdocmain\||fi|\\
\end{tabular}
\end{center}
%
Instead of |\childdocof{|\textit{main}|}| just include the main file
at the top of each child file:
%
\begin{center}
|\input{|\textit{main}|}|
\end{center}
%
A simple redirection |\childdocforward{|\textit{dest}|}| is achieved by:
%
\begin{center}
|\def\jobname{|\textit{dest}|}\input{\jobname}|
\end{center}
%
The redirection with prefix
|\childdocforwardprefix[|\textit{prefix}|]{|\textit{dest}|}|
is accomplished by:
%
\begin{center}
\begin{tabular}{l}
|{\edef\jobname{\scantokens\expandafter{\jobname\noexpand}}|\\
|\def\redirectjob |\textit{prefix}|#1~~~{\gdef\jobname{|\textit{dest}|#1}}|\\
|\expandafter\redirectjob\jobname~~~}\input{\jobname}|
\end{tabular}
\end{center}

In an alternative approach,
child documents can be compiled by a specific command line
without additional code or specific definitions:
%
\begin{center}
|... -jobname "|\textit{target}|" "|[\textit{flags}]%
|\includeonly{|\textit{dest}|}\input{|\textit{main}|}"|
\end{center}
%

%%%%%%%%%%%%%%%%%%%%%%%%%%%%%%%%%%%%%%%%%%%%%%%%%%%%%%%%%%%%%%%%%%%%%%%%%%%%%%%%
%%%%%%%%%%%%%%%%%%%%%%%%%%%%%%%%%%%%%%%%%%%%%%%%%%%%%%%%%%%%%%%%%%%%%%%%%%%%%%%%
\section{Information}

%%%%%%%%%%%%%%%%%%%%%%%%%%%%%%%%%%%%%%%%%%%%%%%%%%%%%%%%%%%%%%%%%%%%%%%%%%%%%%%%
\subsection{Copyright}

Copyright \copyright{} 2017--2018 Niklas Beisert

This work may be distributed and/or modified under the
conditions of the \LaTeX{} Project Public License, either version 1.3
of this license or (at your option) any later version.
The latest version of this license is in
  \url{http://www.latex-project.org/lppl.txt}
and version 1.3 or later is part of all distributions of \LaTeX{}
version 2005/12/01 or later.

This work has the LPPL maintenance status `maintained'.

The Current Maintainer of this work is Niklas Beisert.

This work consists of the files |README.txt|, |childdoc.ins| and |childdoc.dtx|
as well as the derived files |childdoc.def|, |cdocsamp.tex|
with |cdocsch1.tex|, |cdocsch2.tex|, |cdocspt3.tex|, |cdocspt4.tex|,
|cdocsdrf.tex|, |cdocsfn1.tex|, |cdocsfn2.tex|
as well as |childdoc.pdf|.

%%%%%%%%%%%%%%%%%%%%%%%%%%%%%%%%%%%%%%%%%%%%%%%%%%%%%%%%%%%%%%%%%%%%%%%%%%%%%%%%
\subsection{Files and Installation}

The package consists of the files:
%
\begin{center}
\begin{tabular}{ll}
    |README.txt|   & readme file \\
    |childdoc.ins| & installation file \\
    |childdoc.dtx| & source file \\
    |childdoc.def| & definition file \\
    |cdocsamp.tex| & sample main file \\
    |cdocsch1.tex| & sample include file \\
    |cdocsch2.tex| & sample include file \\
    |cdocspt3.tex| & sample part file \\
    |cdocspt4.tex| & sample part file \\
    |cdocsdrf.tex| & sample redirection file \\
    |cdocsfn1.tex| & sample redirection file \\
    |cdocsfn2.tex| & sample redirection file \\
    |childdoc.pdf| & manual
\end{tabular}
\end{center}
%
The distribution consists of the files
|README.txt|, |childdoc.ins| and |childdoc.dtx|.
%
\begin{itemize}
\item
Run (pdf)\LaTeX{} on |childdoc.dtx|
to compile the manual |childdoc.pdf| (this file).
\item
Run \LaTeX{} on |childdoc.ins| to create the definitions file |childdoc.def|
and the sample |cdocsamp.tex| with include files
|cdocsch1.tex|, |cdocsch2.tex|, |cdocspt3.tex|, |cdocspt4.tex|,
|cdocsdrf.tex|, |cdocsfn1.tex|, |cdocsfn2.tex|.
Then copy the file |childdoc.def| to an appropriate directory of your \LaTeX{}
distribution, e.g.\ \textit{texmf-root}|/tex/latex/childdoc|.
\end{itemize}

%%%%%%%%%%%%%%%%%%%%%%%%%%%%%%%%%%%%%%%%%%%%%%%%%%%%%%%%%%%%%%%%%%%%%%%%%%%%%%%%
\subsection{Related CTAN Packages}

There are several other packages which offer a similar functionality:
%
\begin{itemize}
\item
The packages
\href{http://ctan.org/pkg/docmute}{\textsf{docmute}},
\href{http://ctan.org/pkg/includex}{\textsf{includex}} and
\href{http://ctan.org/pkg/standalone}{\textsf{standalone}}
provide commands to include only the document body of
a child file thus allowing both files to be compiled individually.
\item
The packages \href{http://ctan.org/pkg/subdocs}{\textsf{subdocs}}
and \href{http://ctan.org/pkg/subfiles}{\textsf{subfiles}}
provide structures in which the main and child documents can be
encapsulated and allowing them to be compiled individually.
The inclusion mechanism is different from the conventional |\include|.
\item
The package \href{http://ctan.org/pkg/combine}{\textsf{combine}}
is an elaborate solution to combine several documents into one.
\end{itemize}
%
See also the CTAN topic \href{http://ctan.org/topic/subdocs}{\textsf{subdocs}}
for further related packages.
The present package differs from the above solutions in that
a document structure constructed with the conventional |\include| mechanism
just needs two extra commands at the top of every file
such that all constituent files can be compiled individually.

%%%%%%%%%%%%%%%%%%%%%%%%%%%%%%%%%%%%%%%%%%%%%%%%%%%%%%%%%%%%%%%%%%%%%%%%%%%%%%%%
%\subsection{Feature Suggestions}
%
%The following is a list of features which may be useful for future
%versions of this package:
%%
%\begin{itemize}
%\item
%\ldots
%\end{itemize}

%%%%%%%%%%%%%%%%%%%%%%%%%%%%%%%%%%%%%%%%%%%%%%%%%%%%%%%%%%%%%%%%%%%%%%%%%%%%%%%%
\subsection{Revision History}

%%%%%%%%%%%%%%%%%%%%%%%%%%%%%%%%%%%%%%%%
\paragraph{v2.0:} 2018/12/30

\begin{itemize}
\item
immediate forward processing
\item
added |\childdocby| mechanism
\item
manual restructured
\end{itemize}

%%%%%%%%%%%%%%%%%%%%%%%%%%%%%%%%%%%%%%%%
\paragraph{v1.6:} 2018/01/17

\begin{itemize}
\item
application for development of include files
\item
corrections to manual
\end{itemize}

%%%%%%%%%%%%%%%%%%%%%%%%%%%%%%%%%%%%%%%%
\paragraph{v1.5:} 2017/05/21

\begin{itemize}
\item
more complete structuring introduced
\item
|\childdocof| introduced
\item
|\childdoc| renamed to |\childdocmain|
\item
|\childredirect| renamed to |\childdocforward| and |\childdocforwardprefix|
and functionality expanded
\end{itemize}

%%%%%%%%%%%%%%%%%%%%%%%%%%%%%%%%%%%%%%%%
\paragraph{v1.0:} 2017/04/27

\begin{itemize}
\item
manual and install package
\item
first version published on CTAN
\end{itemize}

%%%%%%%%%%%%%%%%%%%%%%%%%%%%%%%%%%%%%%%%
\paragraph{v0.6:} 2017/04/26

\begin{itemize}
\item
redirection mechanism added
\end{itemize}

%%%%%%%%%%%%%%%%%%%%%%%%%%%%%%%%%%%%%%%%
\paragraph{v0.5:} 2017/04/26

\begin{itemize}
\item
functionality in definition file
\end{itemize}


%%%%%%%%%%%%%%%%%%%%%%%%%%%%%%%%%%%%%%%%%%%%%%%%%%%%%%%%%%%%%%%%%%%%%%%%%%%%%%%%
%%%%%%%%%%%%%%%%%%%%%%%%%%%%%%%%%%%%%%%%%%%%%%%%%%%%%%%%%%%%%%%%%%%%%%%%%%%%%%%%
%%%%%%%%%%%%%%%%%%%%%%%%%%%%%%%%%%%%%%%%%%%%%%%%%%%%%%%%%%%%%%%%%%%%%%%%%%%%%%%%
\appendix

\settowidth\MacroIndent{\rmfamily\scriptsize 000\ }

 \DocInput{childdoc.dtx}

\end{document}
%</driver>
% \fi
%
% %%%%%%%%%%%%%%%%%%%%%%%%%%%%%%%%%%%%%%%%%%%%%%%%%%%%%%%%%%%%%%%%%%%%%%%%%%%%%%
% %%%%%%%%%%%%%%%%%%%%%%%%%%%%%%%%%%%%%%%%%%%%%%%%%%%%%%%%%%%%%%%%%%%%%%%%%%%%%%
% \section{Sample}
%\iffalse
%<*samplemain>
%\fi
%
% The following presents a sample document
% with two chapters, two parts, a title page,
% a compile flag as well as three forwarding files to set the flag.
% It consists of eight |.tex| files:
% \begin{center}
% \begin{tabular}{ll}
% |cdocsamp.tex|&main file\\
% |cdocsch1.tex|&include file for chapter 1\\
% |cdocsch2.tex|&include file for chapter 2\\
% |cdocspt3.tex|&include file for part 3\\
% |cdocspt4.tex|&include file for part 4\\
% |cdocsdrf.tex|&forwarding file for main file in draft mode\\
% |cdocsfi1.tex|&forwarding file for final version of chapter 1\\
% |cdocsfi2.tex|&forwarding file for final version of chapter 2\\
% \end{tabular}
% \end{center}
% Each of the eight files can be compiled directly by the \LaTeX{} compiler.
%
% %%%%%%%%%%%%%%%%%%%%%%%%%%%%%%%%%%%%%%
% \paragraph{Main File.}
%
% The main file is called |cdocsamp.tex|.
%
% Load the \textsf{childdoc} definitions and
% declare the filename for the main document:
%    \begin{macrocode}
\input{childdoc.def}
\childdocmain{}
%    \end{macrocode}

% Optional override for |\version| flag:
%    \begin{macrocode}
%%\ifchilddoc\else\providecommand{\version}{draft}\fi
%    \end{macrocode}

% Define the default values for the |\version| flag
% (|final| for the main file and |draft| for childs):
%    \begin{macrocode}
\ifchilddoc
\providecommand{\version}{draft}
\else
\providecommand{\version}{final}
\fi
%    \end{macrocode}

% Load the standard document class:
%    \begin{macrocode}
\documentclass[12pt]{article}
%    \end{macrocode}

% Start the document body:
%    \begin{macrocode}
\begin{document}
%    \end{macrocode}

% Declare a title page.
% Print title, part of document being processed and version flag:
%    \begin{macrocode}
\addtocounter{page}{-1}
\begin{center}
{\LARGE\bfseries{}childdoc example\par}
\vspace{1cm}
\ifchilddoc
\ifchilddocmanual part\else chapter\fi:
`\childdocname' of `\childdocjob'\par
\else
main document: `\childdocjob'\par
\fi
version: \version\par
\end{center}
\newpage
%    \end{macrocode}

% Manually include selected file,
% otherwise process as usual:
%    \begin{macrocode}
\ifchilddocmanual
\section*{part `\childdocname'}
\input{\childdocname}
\else
%    \end{macrocode}

% Include the two chapters:
%    \begin{macrocode}
\include{cdocsch1}
\include{cdocsch2}
%    \end{macrocode}

% Include the two parts unless only chapters should be displayed:
%    \begin{macrocode}
\ifchilddoc\else
\section{part three}
\input{cdocspt3}
\section{part four}
\input{cdocspt4}
\fi
%    \end{macrocode}

% Process as usual until here:
%    \begin{macrocode}
\fi
%    \end{macrocode}

% End of document body:
%    \begin{macrocode}
\end{document}
%    \end{macrocode}
%\iffalse
%</samplemain>
%\fi
%
% %%%%%%%%%%%%%%%%%%%%%%%%%%%%%%%%%%%%%%
% \paragraph{Chapter Include Files.}
%
% The include files are called |cdocsch1.tex| and |cdocsch2.tex|.
%
%\iffalse
%<*samplechap1|samplechap2>
%\fi

% Optional override for |\version| flag:
%    \begin{macrocode}
%%\providecommand{\version}{final}
%    \end{macrocode}

% Include the main document:
%    \begin{macrocode}
\input{childdoc.def}
\childdocof{cdocsamp}
%    \end{macrocode}

%\iffalse
%</samplechap1|samplechap2>
%\fi
%
%\iffalse
%<*samplechap1>
%\fi
% Some text for chapter 1:
%    \begin{macrocode}
\section{one}
some text in chapter one
%    \end{macrocode}

%\iffalse
%</samplechap1>
%\fi
% Some text for chapter 2:
%\iffalse
%<*samplechap2>
%\fi
%    \begin{macrocode}
\section{two}
more text in chapter two
%    \end{macrocode}

%\iffalse
%</samplechap2>
%\fi
%
% %%%%%%%%%%%%%%%%%%%%%%%%%%%%%%%%%%%%%%
% \paragraph{Part Include Files.}
%
% The include files are called |cdocspt3.tex| and |cdocspt4.tex|.
%
%\iffalse
%<*samplepart3|samplepart4>
%\fi

% Optional override for |\version| flag:
%    \begin{macrocode}
%%\providecommand{\version}{final}
%    \end{macrocode}

% Include the main document:
%    \begin{macrocode}
\input{childdoc.def}
\childdocby{cdocsamp}
%    \end{macrocode}

%\iffalse
%</samplepart3|samplepart4>
%\fi
%
%\iffalse
%<*samplepart3>
%\fi
% Some text for part 3:
%    \begin{macrocode}
some text in part three
%    \end{macrocode}

%\iffalse
%</samplepart3>
%\fi
% Some text for part 4:
%\iffalse
%<*samplepart4>
%\fi
%    \begin{macrocode}
more text in part four
%    \end{macrocode}

%\iffalse
%</samplepart4>
%\fi
%
% %%%%%%%%%%%%%%%%%%%%%%%%%%%%%%%%%%%%%%
% \paragraph{Forwarding for a Complete Draft.}
%
% The following forwarding file |cdocsdrf.tex|
% compiles the main document in draft mode:
%\iffalse
%<*sampledraft>
%\fi
%    \begin{macrocode}
\def\version{draft}
\input{childdoc.def}
\childdocforward{cdocsamp}
%    \end{macrocode}

%\iffalse
%</sampledraft>
%\fi
%
% %%%%%%%%%%%%%%%%%%%%%%%%%%%%%%%%%%%%%%
% \paragraph{Forwarding for Final Version of the Chapters.}
%
% The following forwarding files |cdocsfn1.tex| and |cdocsfn2.tex|
% (with identical content)
% compile the final versions of the child documents
% |cdocsch1.tex| and |cdocsch2.tex|, respectively:
%\iffalse
%<*samplefinal>
%\fi
%    \begin{macrocode}
\def\version{final}
\input{childdoc.def}
\childdocforwardprefix[cdocsamp]{cdocsfn}{cdocsch}
%    \end{macrocode}

%\iffalse
%</samplefinal>
%\fi
%
% %%%%%%%%%%%%%%%%%%%%%%%%%%%%%%%%%%%%%%
% \paragraph{Command Line Processing.}
%
% The following three command lines generate the output files
% |cdocscld|, |cdocscl1| and |cdocscl2|
% which should be identical to
% |cdocsdrf|, |cdocsch1| and |cdocsfn2|, respectively:
% \begin{center}
% \begin{tabular}{l}
% |latex -jobname cdocscld \|\\
% |  "\def\version{draft}\input{childdoc.def}\childdocforward{cdocsamp}"|\\
% |latex -jobname cdocscl1 \|\\
% |  "\input{childdoc.def}\childdocforward[cdocsamp]{cdocsch1}"|\\
% |latex -jobname cdocscl2 \|\\
% |  "\def\version{final}\input{childdoc.def}\childdocforward{cdocsch2}"|
% \end{tabular}
% \end{center}
% Note that the trailing backslash on each first line
% merely continues the input to the second line
% (for convenient cut ant paste).
% Furthermore, the command |latex| can be replaced by any
% of its alternative versions such as |pdflatex|.
%
% %%%%%%%%%%%%%%%%%%%%%%%%%%%%%%%%%%%%%%%%%%%%%%%%%%%%%%%%%%%%%%%%%%%%%%%%%%%%%%
% %%%%%%%%%%%%%%%%%%%%%%%%%%%%%%%%%%%%%%%%%%%%%%%%%%%%%%%%%%%%%%%%%%%%%%%%%%%%%%
% \section{Implementation}
%\iffalse
%<*package>
%\fi
%
% This section describes the definitions file |childdoc.def|.

% The definitions cannot be loaded using |\usepackage| or |\RequirePackage|
% which has a mechanism to prevent loading a style file more than once.
% When loading the definitions by means of |\input|
% multiple instances have to be prevented manually:
%\iffalse
%This code needs to be before the `\ProvidesFile' directive
%which is defined at the beginning of this file.
%Therefore it is also placed there and commented out here.
%</package>
%<*discard>
%\fi
%    \begin{macrocode}
\ifdefined\childdocmain\endinput\fi
%    \end{macrocode}
%\iffalse
%</discard>
%<*package>
%\fi
%
% \macro{\ifchilddoc}
% \macro{\ifchilddocmanual}
% The conditional |\ifchilddoc| tells whether a
% child (true) or main (false) document is being compiled.
% The conditional |\ifchilddocmanual| tells whether
% the |\includeonly| mechanism is used (false) or
% the selection of child files must be performed manually (true).
% The definitions initialise to false:
%    \begin{macrocode}
\newif\ifchilddoc
\newif\ifchilddocmanual
%    \end{macrocode}

% \macro{\childdocname}
% \macro{\childdocjob}
% The macro |\childdocname| stores the name of the main document
% to be compiled. The macro |\childdocjob| stores the name of
% the document on which the \LaTeX{} compiler was originally invoked.
% The content of |\jobname| cannot be compared
% to filenames specified in the source due to different catcodes.
% The following code rescans |\jobname|, stores the result
% in |\childdocname| and saves a copy in |\childdocjob|:
%    \begin{macrocode}
\edef\childdocname{\scantokens\expandafter{\jobname\noexpand}}
\let\childdocjob\childdocname
%    \end{macrocode}

% \macro{\childdocdisable}
% The macro |\childdocdisable| prevents the main file
% from being processed more than once.
% At this stage, the main document command |\childdocmain|
% is assumed to be called once again where it should do nothing.
% Any subsequent call to it should prevent
% a secondary processing of the main document
% It overwrites the forwarding commands
% |\childdocof| and |\childdocforward|
% with empty macros to prevent further inclusions of the main document:
%    \begin{macrocode}
\newcommand{\childdocdisable}
{
  \renewcommand{\childdocmain}[1]{\renewcommand{\childdocmain}[1]{\endinput}}
  \renewcommand{\childdocof}[1]{}
  \renewcommand{\childdocby}[2][]{}
  \renewcommand{\childdocforward}[2][]{}
  \renewcommand{\childdocdisable}{}
}
%    \end{macrocode}

% \macro{\childdocmain}
% The macro |\childdocmain| is to be called at the top of the main file
% with nothing or the main filename (without extension) as argument.
% First, it breaks loops.
% If the argument is not empty and does not match |\childdocname|
% (which is set by the first inclusion of |childdoc.def|),
% |\ifchilddoc| is set to true, |\includeonly| is applied to the child file
% and |\jobname| is set to the main file
% (for proper handling of |.aux| files):
%    \begin{macrocode}
\newcommand{\childdocmain}[1]
{
  \childdocdisable\childdocmain{}
  \if?#1?\else
    \begingroup
      \def\childdoctmp{#1}
      \ifx\childdoctmp\childdocname
        \def\childdoctmp{}
      \else
        \def\childdoctmp
        {
          \childdoctrue
          \includeonly{\childdocname}
          \def\childdocjob{#1}
          \def\jobname{#1}
        }
      \fi
      \expandafter
    \endgroup
    \childdoctmp
  \fi
}
%    \end{macrocode}

% \macro{\childdocof}
% The command |\childdocof| redirects
% compilation to the main file |#1|.
%    \begin{macrocode}
\newcommand{\childdocof}[1]
{
  \childdocdisable
  \childdoctrue
  \includeonly{\childdocname}
  \def\jobname{#1}
  \def\childdocjob{#1}
  \input{#1}
}
%    \end{macrocode}

% \macro{\childdocby}
% The command |\childdocby| ....
%    \begin{macrocode}
\newcommand{\childdocby}[2][]
{
  \childdocdisable
  \childdoctrue
  \childdocmanualtrue
  \if?#1?\else
    \def\jobname{#2}
  \fi
  \def\childdocjob{#2}
  \input{#2}
  \endinput
}
%    \end{macrocode}

% \macro{\childdocforward}
% The command |\childdocforward| redirects
% compilation to the main file or
% (if the optional argument is given) a child file.
% Parameters are set as if the main file
% or a child file starting with |\childdocof| was compiled.
% Then compilation is handed over to the main file:
%    \begin{macrocode}
\newcommand{\childdocforward}[2][]
{
  \begingroup
    \if?#1?
      \def\childdoctmp
      {
        \def\childdocname{#2}
        \def\childdocjob{#2}
        \def\jobname{#2}
        \input{#2}
        \endinput
      }
    \else
      \def\childdoctmp
      {
        \childdocdisable
        \def\childdocname{#2}
        \childdoctrue
        \includeonly{#2}
        \def\childdocjob{#1}
        \def\jobname{#1}
        \input{#1}
        \endinput
      }
    \fi
    \expandafter
  \endgroup
  \childdoctmp
}
%    \end{macrocode}

% \macro{\childdocforwardprefix}
% The command |\childdocforwardprefix| redirects
% compilation to the main or a child file by means of a pattern.
% The prefix |#1| in the current filename is replaced by |#2|
% and the suffix of the current filename is kept
% (it is assumed that the filename does not contain the substring `|~~~|'
% which is used as a delimiter).
% Compilation is handed over to the new file by |\childdocforward|:
%    \begin{macrocode}
\newcommand{\childdocforwardprefix}[3][]
{
  \begingroup
    \def\childdocextract #2##1~~~{\def\childdoctmp{\childdocforward[#1]{#3##1}}}
    \expandafter\childdocextract\childdocname~~~
    \expandafter
  \endgroup
  \childdoctmp
}
%    \end{macrocode}

% \macro{\childdoc}
% The deprecated macro |\childdoc| is a legacy version of |\childdocmain|:
%    \begin{macrocode}
\newcommand{\childdoc}{\childdocmain}
%    \end{macrocode}

% \macro{\childdocredirect}
% The deprecated macro |\childdocredirect| is a legacy version
% of |\childdocforward| and |\childdocforwardprefix|:
%    \begin{macrocode}
\newcommand{\childdocredirect}[2][]
{
  \begingroup
    \if?#1?
      \def\childdoctmp{\childdocforward{#2}}
    \else
      \def\childdoctmp{\childdocforwardprefix{#1}{#2}}
    \fi
    \expandafter
  \endgroup
  \childdoctmp
}
%    \end{macrocode}

%\iffalse
%</package>
%\fi
%
\endinput
\childdocforward{cdocsch2}"|
% \end{tabular}
% \end{center}
% Note that the trailing backslash on each first line
% merely continues the input to the second line
% (for convenient cut ant paste).
% Furthermore, the command |latex| can be replaced by any
% of its alternative versions such as |pdflatex|.
%
% %%%%%%%%%%%%%%%%%%%%%%%%%%%%%%%%%%%%%%%%%%%%%%%%%%%%%%%%%%%%%%%%%%%%%%%%%%%%%%
% %%%%%%%%%%%%%%%%%%%%%%%%%%%%%%%%%%%%%%%%%%%%%%%%%%%%%%%%%%%%%%%%%%%%%%%%%%%%%%
% \section{Implementation}
%\iffalse
%<*package>
%\fi
%
% This section describes the definitions file |childdoc.def|.

% The definitions cannot be loaded using |\usepackage| or |\RequirePackage|
% which has a mechanism to prevent loading a style file more than once.
% When loading the definitions by means of |\input|
% multiple instances have to be prevented manually:
%\iffalse
%This code needs to be before the `\ProvidesFile' directive
%which is defined at the beginning of this file.
%Therefore it is also placed there and commented out here.
%</package>
%<*discard>
%\fi
%    \begin{macrocode}
\ifdefined\childdocmain\endinput\fi
%    \end{macrocode}
%\iffalse
%</discard>
%<*package>
%\fi
%
% \macro{\ifchilddoc}
% \macro{\ifchilddocmanual}
% The conditional |\ifchilddoc| tells whether a
% child (true) or main (false) document is being compiled.
% The conditional |\ifchilddocmanual| tells whether
% the |\includeonly| mechanism is used (false) or
% the selection of child files must be performed manually (true).
% The definitions initialise to false:
%    \begin{macrocode}
\newif\ifchilddoc
\newif\ifchilddocmanual
%    \end{macrocode}

% \macro{\childdocname}
% \macro{\childdocjob}
% The macro |\childdocname| stores the name of the main document
% to be compiled. The macro |\childdocjob| stores the name of
% the document on which the \LaTeX{} compiler was originally invoked.
% The content of |\jobname| cannot be compared
% to filenames specified in the source due to different catcodes.
% The following code rescans |\jobname|, stores the result
% in |\childdocname| and saves a copy in |\childdocjob|:
%    \begin{macrocode}
\edef\childdocname{\scantokens\expandafter{\jobname\noexpand}}
\let\childdocjob\childdocname
%    \end{macrocode}

% \macro{\childdocdisable}
% The macro |\childdocdisable| prevents the main file
% from being processed more than once.
% At this stage, the main document command |\childdocmain|
% is assumed to be called once again where it should do nothing.
% Any subsequent call to it should prevent
% a secondary processing of the main document
% It overwrites the forwarding commands
% |\childdocof| and |\childdocforward|
% with empty macros to prevent further inclusions of the main document:
%    \begin{macrocode}
\newcommand{\childdocdisable}
{
  \renewcommand{\childdocmain}[1]{\renewcommand{\childdocmain}[1]{\endinput}}
  \renewcommand{\childdocof}[1]{}
  \renewcommand{\childdocby}[2][]{}
  \renewcommand{\childdocforward}[2][]{}
  \renewcommand{\childdocdisable}{}
}
%    \end{macrocode}

% \macro{\childdocmain}
% The macro |\childdocmain| is to be called at the top of the main file
% with nothing or the main filename (without extension) as argument.
% First, it breaks loops.
% If the argument is not empty and does not match |\childdocname|
% (which is set by the first inclusion of |childdoc.def|),
% |\ifchilddoc| is set to true, |\includeonly| is applied to the child file
% and |\jobname| is set to the main file
% (for proper handling of |.aux| files):
%    \begin{macrocode}
\newcommand{\childdocmain}[1]
{
  \childdocdisable\childdocmain{}
  \if?#1?\else
    \begingroup
      \def\childdoctmp{#1}
      \ifx\childdoctmp\childdocname
        \def\childdoctmp{}
      \else
        \def\childdoctmp
        {
          \childdoctrue
          \includeonly{\childdocname}
          \def\childdocjob{#1}
          \def\jobname{#1}
        }
      \fi
      \expandafter
    \endgroup
    \childdoctmp
  \fi
}
%    \end{macrocode}

% \macro{\childdocof}
% The command |\childdocof| redirects
% compilation to the main file |#1|.
%    \begin{macrocode}
\newcommand{\childdocof}[1]
{
  \childdocdisable
  \childdoctrue
  \includeonly{\childdocname}
  \def\jobname{#1}
  \def\childdocjob{#1}
  \input{#1}
}
%    \end{macrocode}

% \macro{\childdocby}
% The command |\childdocby| ....
%    \begin{macrocode}
\newcommand{\childdocby}[2][]
{
  \childdocdisable
  \childdoctrue
  \childdocmanualtrue
  \if?#1?\else
    \def\jobname{#2}
  \fi
  \def\childdocjob{#2}
  \input{#2}
  \endinput
}
%    \end{macrocode}

% \macro{\childdocforward}
% The command |\childdocforward| redirects
% compilation to the main file or
% (if the optional argument is given) a child file.
% Parameters are set as if the main file
% or a child file starting with |\childdocof| was compiled.
% Then compilation is handed over to the main file:
%    \begin{macrocode}
\newcommand{\childdocforward}[2][]
{
  \begingroup
    \if?#1?
      \def\childdoctmp
      {
        \def\childdocname{#2}
        \def\childdocjob{#2}
        \def\jobname{#2}
        \input{#2}
        \endinput
      }
    \else
      \def\childdoctmp
      {
        \childdocdisable
        \def\childdocname{#2}
        \childdoctrue
        \includeonly{#2}
        \def\childdocjob{#1}
        \def\jobname{#1}
        \input{#1}
        \endinput
      }
    \fi
    \expandafter
  \endgroup
  \childdoctmp
}
%    \end{macrocode}

% \macro{\childdocforwardprefix}
% The command |\childdocforwardprefix| redirects
% compilation to the main or a child file by means of a pattern.
% The prefix |#1| in the current filename is replaced by |#2|
% and the suffix of the current filename is kept
% (it is assumed that the filename does not contain the substring `|~~~|'
% which is used as a delimiter).
% Compilation is handed over to the new file by |\childdocforward|:
%    \begin{macrocode}
\newcommand{\childdocforwardprefix}[3][]
{
  \begingroup
    \def\childdocextract #2##1~~~{\def\childdoctmp{\childdocforward[#1]{#3##1}}}
    \expandafter\childdocextract\childdocname~~~
    \expandafter
  \endgroup
  \childdoctmp
}
%    \end{macrocode}

% \macro{\childdoc}
% The deprecated macro |\childdoc| is a legacy version of |\childdocmain|:
%    \begin{macrocode}
\newcommand{\childdoc}{\childdocmain}
%    \end{macrocode}

% \macro{\childdocredirect}
% The deprecated macro |\childdocredirect| is a legacy version
% of |\childdocforward| and |\childdocforwardprefix|:
%    \begin{macrocode}
\newcommand{\childdocredirect}[2][]
{
  \begingroup
    \if?#1?
      \def\childdoctmp{\childdocforward{#2}}
    \else
      \def\childdoctmp{\childdocforwardprefix{#1}{#2}}
    \fi
    \expandafter
  \endgroup
  \childdoctmp
}
%    \end{macrocode}

%\iffalse
%</package>
%\fi
%
\endinput
|\\
|\childdocforward{|\textit{main}|}|
\end{tabular}
\end{center}
%
Likewise, the following files |final|\textit{nn}|.tex|
compile the final version of the child document
|child|\textit{nn}|.tex|:
%
\begin{center}
\begin{tabular}{l}
|\def\version{final}|\\
|% \iffalse
%
% childdoc.dtx Copyright (C) 2017-2018 Niklas Beisert
%
% This work may be distributed and/or modified under the
% conditions of the LaTeX Project Public License, either version 1.3
% of this license or (at your option) any later version.
% The latest version of this license is in
%   http://www.latex-project.org/lppl.txt
% and version 1.3 or later is part of all distributions of LaTeX
% version 2005/12/01 or later.
%
% This work has the LPPL maintenance status `maintained'.
%
% The Current Maintainer of this work is Niklas Beisert.
%
% This work consists of the files childdoc.dtx and childdoc.ins
% and the derived files childdoc.def and cdocsamp.tex with
% cdocsch1.tex, cdocsch2.tex, cdocsdrf.tex, cdocsfn1.tex, cdocsfn2.tex.
%
%<package>\ifdefined\childdocmain\endinput\fi
%<package>\ProvidesFile{childdoc.def}[2018/12/30 v2.0 child document driver]
%<samplemain>\ProvidesFile{cdocsamp.tex}[2018/12/30 v2.0 sample for childdoc]
%<*driver>
%\ProvidesFile{childdoc.drv}[2018/12/30 v2.0 childdoc reference manual file]
\PassOptionsToClass{10pt,a4paper}{article}
\documentclass{ltxdoc}

\usepackage[margin=35mm]{geometry}
\usepackage{hyperref}
\usepackage{hyperxmp}
\usepackage[usenames]{color}

\hypersetup{colorlinks=true}
\hypersetup{pdfstartview=FitH}
\hypersetup{pdfpagemode=UseNone}
\hypersetup{pdfsource={}}
\hypersetup{pdflang={en-UK}}
\hypersetup{pdfcopyright={Copyright 2017-2018 Niklas Beisert.
  This work may be distributed and/or modified under the
  conditions of the LaTeX Project Public License, either version 1.3
  of this license or (at your option) any later version.}}
\hypersetup{pdflicenseurl={http://www.latex-project.org/lppl.txt}}
\hypersetup{pdfcontactaddress={ETH Zurich, ITP, HIT K,
  Wolfgang-Pauli-Strasse 27}}
\hypersetup{pdfcontactpostcode={8093}}
\hypersetup{pdfcontactcity={Zurich}}
\hypersetup{pdfcontactcountry={Switzerland}}
\hypersetup{pdfcontactemail={nbeisert@itp.phys.ethz.ch}}
\hypersetup{pdfcontacturl={http://people.phys.ethz.ch/\xmptilde nbeisert/}}

\newcommand{\secref}[1]{\hyperref[#1]{section \ref*{#1}}}

\parskip1ex
\parindent0pt
\let\olditemize\itemize
\def\itemize{\olditemize\parskip0pt}

\begin{document}

\title{The \textsf{childdoc} Package}
\hypersetup{pdftitle={The childdoc Package}}
\author{Niklas Beisert\\[2ex]
  Institut f\"ur Theoretische Physik\\
  Eidgen\"ossische Technische Hochschule Z\"urich\\
  Wolfgang-Pauli-Strasse 27, 8093 Z\"urich, Switzerland\\[1ex]
  \href{mailto:nbeisert@itp.phys.ethz.ch}
  {\texttt{nbeisert@itp.phys.ethz.ch}}}
\hypersetup{pdfauthor={Niklas Beisert}}
\hypersetup{pdfsubject={Manual for the LaTeX2e Package childdoc}}
\date{30 December 2018, \textsf{v2.0}}
\maketitle

\begin{abstract}\noindent
\textsf{childdoc} is a \LaTeXe{} package
that enables the direct compilation
of document sections included by |\include|
to individual files.
\end{abstract}

\begingroup
\parskip0ex
\tableofcontents
\endgroup

%%%%%%%%%%%%%%%%%%%%%%%%%%%%%%%%%%%%%%%%%%%%%%%%%%%%%%%%%%%%%%%%%%%%%%%%%%%%%%%%
%%%%%%%%%%%%%%%%%%%%%%%%%%%%%%%%%%%%%%%%%%%%%%%%%%%%%%%%%%%%%%%%%%%%%%%%%%%%%%%%
\section{Introduction}

\LaTeX{} provides a mechanism to structure a large document (such as a book)
into a main file and several child files (containing the chapters)
using the |\include| command.
This mechanism is beneficial for documents
which span hundreds of pages in order to
make the source file(s) more manageable.
Moreover, compilation can be restricted to
selected child files by means of the |\includeonly| command.
The latter feature can be used to reduce the compilation time while editing
(this was significantly more useful in the earlier days of \LaTeX{})
or to generate a smaller document which is easier to navigate.
Another application of |\includeonly| is to generate
documents consisting of selected parts of the complete document.

However, there are a few drawbacks of the plain |\include| mechanism:
\begin{itemize}
\item
The child files cannot be compiled on their own,
they can only be compiled via the main file.
A naive editing environment
(such as a text editor with an option
to have the current file processed by \LaTeX)
may require one to switch to the main file before compiling;
attempting to compile the child file produces errors.
\item
The main file must be modified (each time)
to adjust the |\includeonly| command
to the present needs. This easily leaves the main file in a messy state.
\item
The generated document will always carry the filename
of the main document. This is inconvenient if
several child files are to be compiled and
to be kept for distribution.
\end{itemize}

The present package provides a simple interface
to make child files individually compilable by \LaTeX{}.
Compiling a child file then has the same effect as compiling
the main file with an |\includeonly| command
to select the appropriate child.
Moreover the generated document will carry the name of the child
rather than the main file.
This resolves all three above issues.

This feature is meant to make the editing of books,
thesis documents and lecture notes somewhat more convenient.
However, the package can also be used efficiently for
composing a series of documents (such as exercise sheets)
which are typically distributed individually.
It then assists the author in generating the individual documents
(potentially in different versions)
as well as a document containing the collected series.
Another application is in developing style files
or other kinds of included material
where compilation of the style file could redirect
to a sample or test file.

%%%%%%%%%%%%%%%%%%%%%%%%%%%%%%%%%%%%%%%%%%%%%%%%%%%%%%%%%%%%%%%%%%%%%%%%%%%%%%%%
%%%%%%%%%%%%%%%%%%%%%%%%%%%%%%%%%%%%%%%%%%%%%%%%%%%%%%%%%%%%%%%%%%%%%%%%%%%%%%%%
\section{Usage}

First of all, the package \textsf{childdoc} is \emph{not} a standard
\LaTeXe{} |.sty| style file! Therefore it needs to be invoked in
a non-standard way.

%%%%%%%%%%%%%%%%%%%%%%%%%%%%%%%%%%%%%%%%%%%%%%%%%%%%%%%%%%%%%%%%%%%%%%%%%%%%%%%%
\subsection{Included Files}
\label{sec:include}

%%%%%%%%%%%%%%%%%%%%%%%%%%%%%%%%%%%%%%%%
\DescribeMacro{\childdocmain}
To use the package, add the commands
\begin{center}
\begin{tabular}{l}
|% \iffalse
%
% childdoc.dtx Copyright (C) 2017-2018 Niklas Beisert
%
% This work may be distributed and/or modified under the
% conditions of the LaTeX Project Public License, either version 1.3
% of this license or (at your option) any later version.
% The latest version of this license is in
%   http://www.latex-project.org/lppl.txt
% and version 1.3 or later is part of all distributions of LaTeX
% version 2005/12/01 or later.
%
% This work has the LPPL maintenance status `maintained'.
%
% The Current Maintainer of this work is Niklas Beisert.
%
% This work consists of the files childdoc.dtx and childdoc.ins
% and the derived files childdoc.def and cdocsamp.tex with
% cdocsch1.tex, cdocsch2.tex, cdocsdrf.tex, cdocsfn1.tex, cdocsfn2.tex.
%
%<package>\ifdefined\childdocmain\endinput\fi
%<package>\ProvidesFile{childdoc.def}[2018/12/30 v2.0 child document driver]
%<samplemain>\ProvidesFile{cdocsamp.tex}[2018/12/30 v2.0 sample for childdoc]
%<*driver>
%\ProvidesFile{childdoc.drv}[2018/12/30 v2.0 childdoc reference manual file]
\PassOptionsToClass{10pt,a4paper}{article}
\documentclass{ltxdoc}

\usepackage[margin=35mm]{geometry}
\usepackage{hyperref}
\usepackage{hyperxmp}
\usepackage[usenames]{color}

\hypersetup{colorlinks=true}
\hypersetup{pdfstartview=FitH}
\hypersetup{pdfpagemode=UseNone}
\hypersetup{pdfsource={}}
\hypersetup{pdflang={en-UK}}
\hypersetup{pdfcopyright={Copyright 2017-2018 Niklas Beisert.
  This work may be distributed and/or modified under the
  conditions of the LaTeX Project Public License, either version 1.3
  of this license or (at your option) any later version.}}
\hypersetup{pdflicenseurl={http://www.latex-project.org/lppl.txt}}
\hypersetup{pdfcontactaddress={ETH Zurich, ITP, HIT K,
  Wolfgang-Pauli-Strasse 27}}
\hypersetup{pdfcontactpostcode={8093}}
\hypersetup{pdfcontactcity={Zurich}}
\hypersetup{pdfcontactcountry={Switzerland}}
\hypersetup{pdfcontactemail={nbeisert@itp.phys.ethz.ch}}
\hypersetup{pdfcontacturl={http://people.phys.ethz.ch/\xmptilde nbeisert/}}

\newcommand{\secref}[1]{\hyperref[#1]{section \ref*{#1}}}

\parskip1ex
\parindent0pt
\let\olditemize\itemize
\def\itemize{\olditemize\parskip0pt}

\begin{document}

\title{The \textsf{childdoc} Package}
\hypersetup{pdftitle={The childdoc Package}}
\author{Niklas Beisert\\[2ex]
  Institut f\"ur Theoretische Physik\\
  Eidgen\"ossische Technische Hochschule Z\"urich\\
  Wolfgang-Pauli-Strasse 27, 8093 Z\"urich, Switzerland\\[1ex]
  \href{mailto:nbeisert@itp.phys.ethz.ch}
  {\texttt{nbeisert@itp.phys.ethz.ch}}}
\hypersetup{pdfauthor={Niklas Beisert}}
\hypersetup{pdfsubject={Manual for the LaTeX2e Package childdoc}}
\date{30 December 2018, \textsf{v2.0}}
\maketitle

\begin{abstract}\noindent
\textsf{childdoc} is a \LaTeXe{} package
that enables the direct compilation
of document sections included by |\include|
to individual files.
\end{abstract}

\begingroup
\parskip0ex
\tableofcontents
\endgroup

%%%%%%%%%%%%%%%%%%%%%%%%%%%%%%%%%%%%%%%%%%%%%%%%%%%%%%%%%%%%%%%%%%%%%%%%%%%%%%%%
%%%%%%%%%%%%%%%%%%%%%%%%%%%%%%%%%%%%%%%%%%%%%%%%%%%%%%%%%%%%%%%%%%%%%%%%%%%%%%%%
\section{Introduction}

\LaTeX{} provides a mechanism to structure a large document (such as a book)
into a main file and several child files (containing the chapters)
using the |\include| command.
This mechanism is beneficial for documents
which span hundreds of pages in order to
make the source file(s) more manageable.
Moreover, compilation can be restricted to
selected child files by means of the |\includeonly| command.
The latter feature can be used to reduce the compilation time while editing
(this was significantly more useful in the earlier days of \LaTeX{})
or to generate a smaller document which is easier to navigate.
Another application of |\includeonly| is to generate
documents consisting of selected parts of the complete document.

However, there are a few drawbacks of the plain |\include| mechanism:
\begin{itemize}
\item
The child files cannot be compiled on their own,
they can only be compiled via the main file.
A naive editing environment
(such as a text editor with an option
to have the current file processed by \LaTeX)
may require one to switch to the main file before compiling;
attempting to compile the child file produces errors.
\item
The main file must be modified (each time)
to adjust the |\includeonly| command
to the present needs. This easily leaves the main file in a messy state.
\item
The generated document will always carry the filename
of the main document. This is inconvenient if
several child files are to be compiled and
to be kept for distribution.
\end{itemize}

The present package provides a simple interface
to make child files individually compilable by \LaTeX{}.
Compiling a child file then has the same effect as compiling
the main file with an |\includeonly| command
to select the appropriate child.
Moreover the generated document will carry the name of the child
rather than the main file.
This resolves all three above issues.

This feature is meant to make the editing of books,
thesis documents and lecture notes somewhat more convenient.
However, the package can also be used efficiently for
composing a series of documents (such as exercise sheets)
which are typically distributed individually.
It then assists the author in generating the individual documents
(potentially in different versions)
as well as a document containing the collected series.
Another application is in developing style files
or other kinds of included material
where compilation of the style file could redirect
to a sample or test file.

%%%%%%%%%%%%%%%%%%%%%%%%%%%%%%%%%%%%%%%%%%%%%%%%%%%%%%%%%%%%%%%%%%%%%%%%%%%%%%%%
%%%%%%%%%%%%%%%%%%%%%%%%%%%%%%%%%%%%%%%%%%%%%%%%%%%%%%%%%%%%%%%%%%%%%%%%%%%%%%%%
\section{Usage}

First of all, the package \textsf{childdoc} is \emph{not} a standard
\LaTeXe{} |.sty| style file! Therefore it needs to be invoked in
a non-standard way.

%%%%%%%%%%%%%%%%%%%%%%%%%%%%%%%%%%%%%%%%%%%%%%%%%%%%%%%%%%%%%%%%%%%%%%%%%%%%%%%%
\subsection{Included Files}
\label{sec:include}

%%%%%%%%%%%%%%%%%%%%%%%%%%%%%%%%%%%%%%%%
\DescribeMacro{\childdocmain}
To use the package, add the commands
\begin{center}
\begin{tabular}{l}
|\input{childdoc.def}|\\
|\childdocmain{}|\\
\end{tabular}
\end{center}
at the very top of the main \LaTeX{} file,
in particular \emph{before} the |\documentclass| statement!
The argument of |\childdocmain| should be left empty
(but it must be present).

%%%%%%%%%%%%%%%%%%%%%%%%%%%%%%%%%%%%%%%%
\DescribeMacro{\childdocof}
Furthermore, add the commands
\begin{center}
\begin{tabular}{l}
|\input{childdoc.def}|\\
|\childdocof{|\textit{main}|}|\\
\end{tabular}
\end{center}
at the top of every child file \textit{child}
which is included by |\include{|\textit{child}|}|
from within the main file
(or at least for those files to be compiled individually).
The argument \textit{main} must be the filename of the main file.

There are a couple of
considerations in setting up the main and child documents:

%%%%%%%%%%%%%%%%%%%%%%%%%%%%%%%%%%%%%%%%
\paragraph{Restrictions.}

Please note the following restrictions:
\begin{itemize}
\item
|\childdocmain| must be called with one argument \textit{main}
to ensure compatibility with earlier version of the package.
It must either be empty (|\childdocmain{}|)
or precisely match the filename of the main file in which it is specified.
See \secref{sec:detection} for further information.
\item
The filename \textit{main} must be specified without the |.tex| extension.
\item
The filename \textit{main} is case sensitive
(even in case-insensitive file systems)
due to internal string comparison.
\item
The argument \textit{main} should be fully expanded, it cannot be a macro.
\item
Subdirectories and special characters should be avoided in filenames.
\item
The command |\childdocmain{|\textit{main}|}| must be followed by a whitespace.
It should not be followed immediately by another command
or by a comment mark `|%|'.
This is because the \TeX{} parser reads the token immediately following
the argument of |\childdocmain| and puts it
at the beginning of every child section;
however, a white\-space is ignored.
\end{itemize}

%%%%%%%%%%%%%%%%%%%%%%%%%%%%%%%%%%%%%%%%
\paragraph{Content of Main File.}

It is advisable to place all content in the child files included by |\include|.
Any output contained in the main file will appear in all child documents
unless suppressed manually;
it cannot be suppressed automatically by the |\includeonly| directive
and thus should normally be avoided.
A method to include some content in the main file
by means of conditional processing is described in \secref{sec:conditional}.

%%%%%%%%%%%%%%%%%%%%%%%%%%%%%%%%%%%%%%%%
\paragraph{Page Numbering.}

When only a part of the document is compiled,
the appropriate numbering of pages
(as well as other status parameters)
is determined from the |.aux| files.
The latter contain information from previous passes.
However this information needs to propagate through
all intermediate child documents.
Therefore the page numbering in child documents may well
be inconsistent until the complete document is compiled at least once.

A useful (if unconventional) way to always ensure a consistent
page numbering is to restart the numbering in each child document
and denote the pages by `\textit{child}|.|\textit{page}'
where \textit{child} represents the chapter/section number of the child file.
This can be achieved by the command
|\numberwithin{page}{|\textit{child}|}|
of the \textsf{amsmath} package
where \textit{child} can be |chapter| or |section|
depending on the chosen structuring.
Alternatively, one can modify the macro |\thepage| appropriately
and reset the counter |page| at the start of each child file.

%%%%%%%%%%%%%%%%%%%%%%%%%%%%%%%%%%%%%%%%%%%%%%%%%%%%%%%%%%%%%%%%%%%%%%%%%%%%%%%%
\subsection{Conditional Processing}
\label{sec:conditional}

The package provides a mechanism to compile different versions
of a document. To customise the versions further some conditional processing
can come in handy to distinguish which version is being compiled.
The package provides two macros to describe the compilation context:

%%%%%%%%%%%%%%%%%%%%%%%%%%%%%%%%%%%%%%%%
\DescribeMacro{\ifchilddoc}
The conditional |\ifchilddoc| distinguishes between the compilation of
child documents and the main document:
%
\begin{center}
|\ifchilddoc |\textit{child-code}| |[|\||else |\textit{main-code}]| \||fi|
\end{center}

%%%%%%%%%%%%%%%%%%%%%%%%%%%%%%%%%%%%%%%%
\DescribeMacro{\childdocname}
\DescribeMacro{\childdocjob}
The macro |\childdocname| contains the filename (without extension)
of the main or child file being processed.
Note that |\childdocjob| will always contain the name of the main file.

%%%%%%%%%%%%%%%%%%%%%%%%%%%%%%%%%%%%%%%%
\paragraph{Title Page.}

Conditional processing can be used to include a title or banner page
in the main document when proper precautions are taken.
Importantly, the code in the main file should ensure that the page counter
(as well as other status parameters which are stored in the |.aux| files)
takes the same value after the conditional processing.
Otherwise the page numbers may take divergent values
depending on which part is compiled.

For example, a title page could be declared by:
%
\begin{center}
\begin{tabular}{l}
|\ifchilddoc\||else|\\
|\addtocounter{page}{-1}|\\
\textit{code for title page}\\
|\newpage|\\
|\||fi|
\end{tabular}
\end{center}
%
A banner page for the child documents can be generated by:
%
\begin{center}
\begin{tabular}{l}
|\ifchilddoc|\\
|\addtocounter{page}{-1}|\\
\textit{code for banner page}\\
|\newpage|\\
|\||fi|
\end{tabular}
\end{center}
%
Here one could write a message such as:
\begin{center}
|This is the part \childdocname{} of \childdocjob{}.|
\end{center}

%%%%%%%%%%%%%%%%%%%%%%%%%%%%%%%%%%%%%%%%%%%%%%%%%%%%%%%%%%%%%%%%%%%%%%%%%%%%%%%%
\subsection{Flags}
\label{sec:flags}

The package makes it easy to generate different versions
of the main or child documents.
To this end compilation flags can be defined
and assigned different default values.
They will be particularly useful in conjunction
with the forwarding mechanism described in \secref{sec:forward}.

For example, it may be useful to have a flag |\version|
which can be set to |draft| or |final|.
The document source will contain some conditional code
depending on the value of |\version|.
Suppose further, the flag should default to |final| for the main file
and to |draft| for child files
which is a natural assignment for editing the document.
This is achieved by placing the following code
in the preamble of the main document
(below the |\childdocmain| directive):
%
\begin{center}
\begin{tabular}{l}
|\ifchilddoc|\\
|\providecommand{\version}{draft}|\\
|\||else|\\
|\providecommand{\version}{final}|\\
|\||fi|
\end{tabular}
\end{center}
%
The definition by |\providecommand| makes sure
that previous definitions are not overwritten.
Further statements |\providecommand{\version}{...}|
can thus be added before the above code to override it.

For the main file, one might add a line
(between |\childdocmain| and the above block)
%
\begin{center}
|%\ifchilddoc\||else\providecommand{\version}{draft}\||fi|
\end{center}
%
which can be uncommented to produce a draft version.
Likewise one can add a line to the very top of a child file
(above the |\childdocof{|\textit{main}|}| directive)
%
\begin{center}
|%\providecommand{\version}{final}|
\end{center}
%
which can be uncommented to produce the final version of this child document.

%%%%%%%%%%%%%%%%%%%%%%%%%%%%%%%%%%%%%%%%%%%%%%%%%%%%%%%%%%%%%%%%%%%%%%%%%%%%%%%%
\subsection{Forwarding}
\label{sec:forward}

Different versions of the main or child documents
using compilation flags as described in \secref{sec:flags}
can be (permanently) stored in different files
for convenient compilation, viewing and distribution.
To this end, the package defines a command
to pass on compilation to a different file:

%%%%%%%%%%%%%%%%%%%%%%%%%%%%%%%%%%%%%%%%
\DescribeMacro{\childdocforward}
The command |\childdocforward| redirects processing to
another source file:
%
\begin{center}
\begin{tabular}{l}
|\input{childdoc.def}|\\
|\childdocforward[|\textit{main}|]{|\textit{dest}|}|\\
\end{tabular}
\end{center}
%
The argument \textit{dest} is the destination file
(without extension).
It should be the main file or one of the child files.
Note that further \textsf{childdoc} directives
such as |\childdocof| and |\childdocforward|
in the indicated file will be processed in this form.
The optional argument \textit{main}
passes on directly to the main file \textit{main}
while pretending to compile the child \textit{dest}.
This form behaves as if \textit{dest}
issues |\childdocof{|\textit{main}|}| right away,
and no further \textsf{childdoc} directives will be processed.

%%%%%%%%%%%%%%%%%%%%%%%%%%%%%%%%%%%%%%%%
\DescribeMacro{\...prefix}
In the alternative form |\childdocforwardprefix|,
%
\begin{center}
\begin{tabular}{l}
|\input{childdoc.def}|\\
|\childdocforwardprefix[|\textit{main}|]{|\textit{prefix}|}{|\textit{dest}|}|
\end{tabular}
\end{center}
%
the destination file is determined by a pattern
depending on the current file:
To make this work, the current file must be called
`{\textit{prefix}\hspace{0.2em}\textit{suffix}}'
with \textit{prefix} matching precisely the argument.
Processing is then passed on to the file
`{\textit{dest}\hspace{0.2em}\textit{suffix}}'.
Surely, the same effect is achieved by
directly specifying the
argument `{\textit{dest}\hspace{0.2em}\textit{suffix}}'
in the first form.
However, that requires to set up a different file
for each child. With the alternative form of the command
all these files can have exactly the same content
which simplifies setting them up and maintaining them.

For example, the following file |draft.tex|
with a compilation flag |\version| as described in \secref{sec:flags}
compiles the main document as a draft:
%
\begin{center}
\begin{tabular}{l}
|\def\version{draft}|\\
|\input{childdoc.def}|\\
|\childdocforward{|\textit{main}|}|
\end{tabular}
\end{center}
%
Likewise, the following files |final|\textit{nn}|.tex|
compile the final version of the child document
|child|\textit{nn}|.tex|:
%
\begin{center}
\begin{tabular}{l}
|\def\version{final}|\\
|\input{childdoc.def}|\\
|\childdocforwardprefix{final}{child}|
\end{tabular}
\end{center}
%

Note that when several versions of a main file and/or of each child file
are to be generated, it may be convenient to set up a |Makefile| or
shell script to automatise the process.

%%%%%%%%%%%%%%%%%%%%%%%%%%%%%%%%%%%%%%%%%%%%%%%%%%%%%%%%%%%%%%%%%%%%%%%%%%%%%%%%
\subsection{Command Line Processing}
\label{sec:commandline}

The effect of redirection files can also be achieved by invoking
the \LaTeX{} compiler with a more elaborate command line.
Most conveniently this should be done as part
of a shell script or a |Makefile|.

When using \textsf{childdoc} in the main file, the following
command lines effectively perform a redirection
(note that depending on the shell being used,
backslashes may have to be doubled: `|\|' $\to$ `|\\|'):
%
\begin{center}
|... -jobname "|\textit{target}|" |\\|"|[\textit{flags}]%
|\input{childdoc.def}\childdocforward[|\textit{main}|]{|\textit{dest}|}"|
\end{center}
%
Here \textit{target} is the name of the output file,
\textit{main} is the name of the main file
and \textit{dest} is the name of the main or child file to be processed
(all filenames without extensions).
The optional argument \textit{main} can be omitted
if \textit{main} matches \textit{dest}.
Optionally, compilation \textit{flags} can be defined via |\def| commands.
This command line makes the \TeX{} engine believe
it is compiling the file \textit{target}
whose content is specified as the latter parameter.
The provided code then forwards the processing to
\textit{main} or \textit{dest} as described in \secref{sec:forward}.

%%%%%%%%%%%%%%%%%%%%%%%%%%%%%%%%%%%%%%%%%%%%%%%%%%%%%%%%%%%%%%%%%%%%%%%%%%%%%%%%
\subsection{Include by Input}
\label{sec:input}

Including child documents by |\include| has some restrictions by design.
Most notably, the content of a child document always occupies
its own set of pages; pages cannot be shared between child documents.
Usually, this behaviour makes perfect sense
because each child document contain an essential part of the document.
However, in some situations it may be desirable to compose
a document from a collection of parts
without having mandatory page breaks between then.
For this case, the package
provides a mechanism to include parts
by |\input| which can also be processed individually.
However, by construction this mechanism
requires manual handling of the content to be output.

%%%%%%%%%%%%%%%%%%%%%%%%%%%%%%%%%%%%%%%%
\DescribeMacro{\ifchilddocmanual}
The main file should be prepared as usual, see \secref{sec:include}.
However, the document body must make a distinction
between processing of an individual part and of the main document, e.g.:
%
\begin{center}
\begin{tabular}{l}
|\ifchilddocmanual|\\
|\input{\childdocname}|\\
|\||else|\\
\textit{document body with }|\input{|\textit{part}|}|\\
|\||fi|
\end{tabular}
\end{center}
%
The conditional |\ifchilddocmanual| is true whenever
a part to be included by |\input| is being compiled,
and the name of the part is stored in |\childdocname|.

%%%%%%%%%%%%%%%%%%%%%%%%%%%%%%%%%%%%%%%%
\DescribeMacro{\childdocby}
Each part to be included by |\input| should start with:
%
\begin{center}
\begin{tabular}{l}
|\input{childdoc.def}|\\
|\childdocby{|\textit{main}|}|\\
\end{tabular}
\end{center}
%
The directive |\childdocby| is similar to |\childdocof|
described in \secref{sec:include},
but the subsequent selection of content must be done manually.
To that end, both |\ifchilddoc| and |\ifchilddocmanual|
will be true upon processing of a part,
and the name of the part is stored in |\childdocname|.
Note that |\jobname| will be set to the filename of the current part
so that each part receives an individual |.aux| file
that does not interfere with the |.aux| file(s) of the main document.
This behaviour can be altered by the alternative form
|\childdocby[*]{|\textit{main}|}| (with a non-empty optional argument)
which uses the |.aux| file of the main document
by setting |\jobname| to \textit{main}.

%%%%%%%%%%%%%%%%%%%%%%%%%%%%%%%%%%%%%%%%%%%%%%%%%%%%%%%%%%%%%%%%%%%%%%%%%%%%%%%%
\subsection{Driver Development}
\label{sec:driver}

The \textsf{childdoc} mechanism can also be use for the development
of definition files such as \LaTeX{} styles or classes.
This case differs from the above setup with multiple parts
included by |\include| in that no |\includeonly| should be invoked.
This can be achieved by starting the include file
(before |\ProvidesPackage|) with:
%
\begin{center}
\begin{tabular}{l}
|\input{childdoc.def}|\\
|\childdocforward{|\textit{main}|}|\\
\end{tabular}
\end{center}
%
or alternatively with:
%
\begin{center}
\begin{tabular}{l}
|\input{childdoc.def}|\\
|\childdocby{|\textit{main}|}|\\
\end{tabular}
\end{center}
%
Both forms have slightly different effects as described above.
The main file is prepared as usual, see \secref{sec:include}.

%%%%%%%%%%%%%%%%%%%%%%%%%%%%%%%%%%%%%%%%%%%%%%%%%%%%%%%%%%%%%%%%%%%%%%%%%%%%%%%%
\subsection{Legacy Detection}
\label{sec:detection}

The directive |\childdocmain| in the main file can detect
whether the complete document or merely a child is to be compiled
even without using the directive |\childdocof|.
This method is deprecated because it is less robust
and there is no compelling reason to use it;
it is merely provided for backward compatibility
and it may be removed in future versions.

If the detection mechanism is to be used,
it is mandatory to correctly specify
the filename of the main file as the argument of |\childdocmain|:
%
\begin{center}
\begin{tabular}{l}
|\input{childdoc.def}|\\
|\childdocmain{|\textit{main}|}|\\
\end{tabular}
\end{center}
%
If |\jobname| does not match the argument \textit{main} of |\childdocmain|,
it is assumed that |\jobname| points to the child file to be compiled.
When using |\childdocmain| with the main file specified as argument,
it suffices to start a child file
with just |\input{|\textit{main}|}|
without loading of the package and using |\childdocof|.
If instead all processing is done
with the appropriate \textsf{childdoc} directives,
the argument of \textit{main} of |\childdocmain| can be empty.

An alternative version of the command line processing described
in \secref{sec:commandline} using the detection mechanism reads:
%
\begin{center}
|... -jobname "|\textit{target}|" "|[\textit{flags}]%
[|\def\jobname{|\textit{dest}|}|]|\input{|\textit{main}|}"|
\end{center}

%%%%%%%%%%%%%%%%%%%%%%%%%%%%%%%%%%%%%%%%%%%%%%%%%%%%%%%%%%%%%%%%%%%%%%%%%%%%%%%%
\subsection{Manual Code}
\label{sec:manual}

In case one cannot be certain whether the definitions file |childdoc.def|
is installed on the target \TeX{} distribution
and one prefers not to ship it,
it is conceivable to paste a few relevant commands into the sources.

To that end, drop all statements |\input{childdoc.def}|
and perform the replacements as outlined below.
Instead of |\childdocmain{|\textit{main}|}| add the following code
to the top of the main file:
%
\begin{center}
\begin{tabular}{l}
|\||ifdefined\childdocname\endinput\||fi\newif\ifchilddoc|\\
|\edef\childdocname{\scantokens\expandafter{\jobname\noexpand}}|\\
|\def\childdocmain{|\textit{main}|}\||ifx\childdocmain\childdocname\||else|\\
|\childdoctrue\includeonly{\childdocname}\let\jobname\childdocmain\||fi|\\
\end{tabular}
\end{center}
%
Instead of |\childdocof{|\textit{main}|}| just include the main file
at the top of each child file:
%
\begin{center}
|\input{|\textit{main}|}|
\end{center}
%
A simple redirection |\childdocforward{|\textit{dest}|}| is achieved by:
%
\begin{center}
|\def\jobname{|\textit{dest}|}\input{\jobname}|
\end{center}
%
The redirection with prefix
|\childdocforwardprefix[|\textit{prefix}|]{|\textit{dest}|}|
is accomplished by:
%
\begin{center}
\begin{tabular}{l}
|{\edef\jobname{\scantokens\expandafter{\jobname\noexpand}}|\\
|\def\redirectjob |\textit{prefix}|#1~~~{\gdef\jobname{|\textit{dest}|#1}}|\\
|\expandafter\redirectjob\jobname~~~}\input{\jobname}|
\end{tabular}
\end{center}

In an alternative approach,
child documents can be compiled by a specific command line
without additional code or specific definitions:
%
\begin{center}
|... -jobname "|\textit{target}|" "|[\textit{flags}]%
|\includeonly{|\textit{dest}|}\input{|\textit{main}|}"|
\end{center}
%

%%%%%%%%%%%%%%%%%%%%%%%%%%%%%%%%%%%%%%%%%%%%%%%%%%%%%%%%%%%%%%%%%%%%%%%%%%%%%%%%
%%%%%%%%%%%%%%%%%%%%%%%%%%%%%%%%%%%%%%%%%%%%%%%%%%%%%%%%%%%%%%%%%%%%%%%%%%%%%%%%
\section{Information}

%%%%%%%%%%%%%%%%%%%%%%%%%%%%%%%%%%%%%%%%%%%%%%%%%%%%%%%%%%%%%%%%%%%%%%%%%%%%%%%%
\subsection{Copyright}

Copyright \copyright{} 2017--2018 Niklas Beisert

This work may be distributed and/or modified under the
conditions of the \LaTeX{} Project Public License, either version 1.3
of this license or (at your option) any later version.
The latest version of this license is in
  \url{http://www.latex-project.org/lppl.txt}
and version 1.3 or later is part of all distributions of \LaTeX{}
version 2005/12/01 or later.

This work has the LPPL maintenance status `maintained'.

The Current Maintainer of this work is Niklas Beisert.

This work consists of the files |README.txt|, |childdoc.ins| and |childdoc.dtx|
as well as the derived files |childdoc.def|, |cdocsamp.tex|
with |cdocsch1.tex|, |cdocsch2.tex|, |cdocspt3.tex|, |cdocspt4.tex|,
|cdocsdrf.tex|, |cdocsfn1.tex|, |cdocsfn2.tex|
as well as |childdoc.pdf|.

%%%%%%%%%%%%%%%%%%%%%%%%%%%%%%%%%%%%%%%%%%%%%%%%%%%%%%%%%%%%%%%%%%%%%%%%%%%%%%%%
\subsection{Files and Installation}

The package consists of the files:
%
\begin{center}
\begin{tabular}{ll}
    |README.txt|   & readme file \\
    |childdoc.ins| & installation file \\
    |childdoc.dtx| & source file \\
    |childdoc.def| & definition file \\
    |cdocsamp.tex| & sample main file \\
    |cdocsch1.tex| & sample include file \\
    |cdocsch2.tex| & sample include file \\
    |cdocspt3.tex| & sample part file \\
    |cdocspt4.tex| & sample part file \\
    |cdocsdrf.tex| & sample redirection file \\
    |cdocsfn1.tex| & sample redirection file \\
    |cdocsfn2.tex| & sample redirection file \\
    |childdoc.pdf| & manual
\end{tabular}
\end{center}
%
The distribution consists of the files
|README.txt|, |childdoc.ins| and |childdoc.dtx|.
%
\begin{itemize}
\item
Run (pdf)\LaTeX{} on |childdoc.dtx|
to compile the manual |childdoc.pdf| (this file).
\item
Run \LaTeX{} on |childdoc.ins| to create the definitions file |childdoc.def|
and the sample |cdocsamp.tex| with include files
|cdocsch1.tex|, |cdocsch2.tex|, |cdocspt3.tex|, |cdocspt4.tex|,
|cdocsdrf.tex|, |cdocsfn1.tex|, |cdocsfn2.tex|.
Then copy the file |childdoc.def| to an appropriate directory of your \LaTeX{}
distribution, e.g.\ \textit{texmf-root}|/tex/latex/childdoc|.
\end{itemize}

%%%%%%%%%%%%%%%%%%%%%%%%%%%%%%%%%%%%%%%%%%%%%%%%%%%%%%%%%%%%%%%%%%%%%%%%%%%%%%%%
\subsection{Related CTAN Packages}

There are several other packages which offer a similar functionality:
%
\begin{itemize}
\item
The packages
\href{http://ctan.org/pkg/docmute}{\textsf{docmute}},
\href{http://ctan.org/pkg/includex}{\textsf{includex}} and
\href{http://ctan.org/pkg/standalone}{\textsf{standalone}}
provide commands to include only the document body of
a child file thus allowing both files to be compiled individually.
\item
The packages \href{http://ctan.org/pkg/subdocs}{\textsf{subdocs}}
and \href{http://ctan.org/pkg/subfiles}{\textsf{subfiles}}
provide structures in which the main and child documents can be
encapsulated and allowing them to be compiled individually.
The inclusion mechanism is different from the conventional |\include|.
\item
The package \href{http://ctan.org/pkg/combine}{\textsf{combine}}
is an elaborate solution to combine several documents into one.
\end{itemize}
%
See also the CTAN topic \href{http://ctan.org/topic/subdocs}{\textsf{subdocs}}
for further related packages.
The present package differs from the above solutions in that
a document structure constructed with the conventional |\include| mechanism
just needs two extra commands at the top of every file
such that all constituent files can be compiled individually.

%%%%%%%%%%%%%%%%%%%%%%%%%%%%%%%%%%%%%%%%%%%%%%%%%%%%%%%%%%%%%%%%%%%%%%%%%%%%%%%%
%\subsection{Feature Suggestions}
%
%The following is a list of features which may be useful for future
%versions of this package:
%%
%\begin{itemize}
%\item
%\ldots
%\end{itemize}

%%%%%%%%%%%%%%%%%%%%%%%%%%%%%%%%%%%%%%%%%%%%%%%%%%%%%%%%%%%%%%%%%%%%%%%%%%%%%%%%
\subsection{Revision History}

%%%%%%%%%%%%%%%%%%%%%%%%%%%%%%%%%%%%%%%%
\paragraph{v2.0:} 2018/12/30

\begin{itemize}
\item
immediate forward processing
\item
added |\childdocby| mechanism
\item
manual restructured
\end{itemize}

%%%%%%%%%%%%%%%%%%%%%%%%%%%%%%%%%%%%%%%%
\paragraph{v1.6:} 2018/01/17

\begin{itemize}
\item
application for development of include files
\item
corrections to manual
\end{itemize}

%%%%%%%%%%%%%%%%%%%%%%%%%%%%%%%%%%%%%%%%
\paragraph{v1.5:} 2017/05/21

\begin{itemize}
\item
more complete structuring introduced
\item
|\childdocof| introduced
\item
|\childdoc| renamed to |\childdocmain|
\item
|\childredirect| renamed to |\childdocforward| and |\childdocforwardprefix|
and functionality expanded
\end{itemize}

%%%%%%%%%%%%%%%%%%%%%%%%%%%%%%%%%%%%%%%%
\paragraph{v1.0:} 2017/04/27

\begin{itemize}
\item
manual and install package
\item
first version published on CTAN
\end{itemize}

%%%%%%%%%%%%%%%%%%%%%%%%%%%%%%%%%%%%%%%%
\paragraph{v0.6:} 2017/04/26

\begin{itemize}
\item
redirection mechanism added
\end{itemize}

%%%%%%%%%%%%%%%%%%%%%%%%%%%%%%%%%%%%%%%%
\paragraph{v0.5:} 2017/04/26

\begin{itemize}
\item
functionality in definition file
\end{itemize}


%%%%%%%%%%%%%%%%%%%%%%%%%%%%%%%%%%%%%%%%%%%%%%%%%%%%%%%%%%%%%%%%%%%%%%%%%%%%%%%%
%%%%%%%%%%%%%%%%%%%%%%%%%%%%%%%%%%%%%%%%%%%%%%%%%%%%%%%%%%%%%%%%%%%%%%%%%%%%%%%%
%%%%%%%%%%%%%%%%%%%%%%%%%%%%%%%%%%%%%%%%%%%%%%%%%%%%%%%%%%%%%%%%%%%%%%%%%%%%%%%%
\appendix

\settowidth\MacroIndent{\rmfamily\scriptsize 000\ }

 \DocInput{childdoc.dtx}

\end{document}
%</driver>
% \fi
%
% %%%%%%%%%%%%%%%%%%%%%%%%%%%%%%%%%%%%%%%%%%%%%%%%%%%%%%%%%%%%%%%%%%%%%%%%%%%%%%
% %%%%%%%%%%%%%%%%%%%%%%%%%%%%%%%%%%%%%%%%%%%%%%%%%%%%%%%%%%%%%%%%%%%%%%%%%%%%%%
% \section{Sample}
%\iffalse
%<*samplemain>
%\fi
%
% The following presents a sample document
% with two chapters, two parts, a title page,
% a compile flag as well as three forwarding files to set the flag.
% It consists of eight |.tex| files:
% \begin{center}
% \begin{tabular}{ll}
% |cdocsamp.tex|&main file\\
% |cdocsch1.tex|&include file for chapter 1\\
% |cdocsch2.tex|&include file for chapter 2\\
% |cdocspt3.tex|&include file for part 3\\
% |cdocspt4.tex|&include file for part 4\\
% |cdocsdrf.tex|&forwarding file for main file in draft mode\\
% |cdocsfi1.tex|&forwarding file for final version of chapter 1\\
% |cdocsfi2.tex|&forwarding file for final version of chapter 2\\
% \end{tabular}
% \end{center}
% Each of the eight files can be compiled directly by the \LaTeX{} compiler.
%
% %%%%%%%%%%%%%%%%%%%%%%%%%%%%%%%%%%%%%%
% \paragraph{Main File.}
%
% The main file is called |cdocsamp.tex|.
%
% Load the \textsf{childdoc} definitions and
% declare the filename for the main document:
%    \begin{macrocode}
\input{childdoc.def}
\childdocmain{}
%    \end{macrocode}

% Optional override for |\version| flag:
%    \begin{macrocode}
%%\ifchilddoc\else\providecommand{\version}{draft}\fi
%    \end{macrocode}

% Define the default values for the |\version| flag
% (|final| for the main file and |draft| for childs):
%    \begin{macrocode}
\ifchilddoc
\providecommand{\version}{draft}
\else
\providecommand{\version}{final}
\fi
%    \end{macrocode}

% Load the standard document class:
%    \begin{macrocode}
\documentclass[12pt]{article}
%    \end{macrocode}

% Start the document body:
%    \begin{macrocode}
\begin{document}
%    \end{macrocode}

% Declare a title page.
% Print title, part of document being processed and version flag:
%    \begin{macrocode}
\addtocounter{page}{-1}
\begin{center}
{\LARGE\bfseries{}childdoc example\par}
\vspace{1cm}
\ifchilddoc
\ifchilddocmanual part\else chapter\fi:
`\childdocname' of `\childdocjob'\par
\else
main document: `\childdocjob'\par
\fi
version: \version\par
\end{center}
\newpage
%    \end{macrocode}

% Manually include selected file,
% otherwise process as usual:
%    \begin{macrocode}
\ifchilddocmanual
\section*{part `\childdocname'}
\input{\childdocname}
\else
%    \end{macrocode}

% Include the two chapters:
%    \begin{macrocode}
\include{cdocsch1}
\include{cdocsch2}
%    \end{macrocode}

% Include the two parts unless only chapters should be displayed:
%    \begin{macrocode}
\ifchilddoc\else
\section{part three}
\input{cdocspt3}
\section{part four}
\input{cdocspt4}
\fi
%    \end{macrocode}

% Process as usual until here:
%    \begin{macrocode}
\fi
%    \end{macrocode}

% End of document body:
%    \begin{macrocode}
\end{document}
%    \end{macrocode}
%\iffalse
%</samplemain>
%\fi
%
% %%%%%%%%%%%%%%%%%%%%%%%%%%%%%%%%%%%%%%
% \paragraph{Chapter Include Files.}
%
% The include files are called |cdocsch1.tex| and |cdocsch2.tex|.
%
%\iffalse
%<*samplechap1|samplechap2>
%\fi

% Optional override for |\version| flag:
%    \begin{macrocode}
%%\providecommand{\version}{final}
%    \end{macrocode}

% Include the main document:
%    \begin{macrocode}
\input{childdoc.def}
\childdocof{cdocsamp}
%    \end{macrocode}

%\iffalse
%</samplechap1|samplechap2>
%\fi
%
%\iffalse
%<*samplechap1>
%\fi
% Some text for chapter 1:
%    \begin{macrocode}
\section{one}
some text in chapter one
%    \end{macrocode}

%\iffalse
%</samplechap1>
%\fi
% Some text for chapter 2:
%\iffalse
%<*samplechap2>
%\fi
%    \begin{macrocode}
\section{two}
more text in chapter two
%    \end{macrocode}

%\iffalse
%</samplechap2>
%\fi
%
% %%%%%%%%%%%%%%%%%%%%%%%%%%%%%%%%%%%%%%
% \paragraph{Part Include Files.}
%
% The include files are called |cdocspt3.tex| and |cdocspt4.tex|.
%
%\iffalse
%<*samplepart3|samplepart4>
%\fi

% Optional override for |\version| flag:
%    \begin{macrocode}
%%\providecommand{\version}{final}
%    \end{macrocode}

% Include the main document:
%    \begin{macrocode}
\input{childdoc.def}
\childdocby{cdocsamp}
%    \end{macrocode}

%\iffalse
%</samplepart3|samplepart4>
%\fi
%
%\iffalse
%<*samplepart3>
%\fi
% Some text for part 3:
%    \begin{macrocode}
some text in part three
%    \end{macrocode}

%\iffalse
%</samplepart3>
%\fi
% Some text for part 4:
%\iffalse
%<*samplepart4>
%\fi
%    \begin{macrocode}
more text in part four
%    \end{macrocode}

%\iffalse
%</samplepart4>
%\fi
%
% %%%%%%%%%%%%%%%%%%%%%%%%%%%%%%%%%%%%%%
% \paragraph{Forwarding for a Complete Draft.}
%
% The following forwarding file |cdocsdrf.tex|
% compiles the main document in draft mode:
%\iffalse
%<*sampledraft>
%\fi
%    \begin{macrocode}
\def\version{draft}
\input{childdoc.def}
\childdocforward{cdocsamp}
%    \end{macrocode}

%\iffalse
%</sampledraft>
%\fi
%
% %%%%%%%%%%%%%%%%%%%%%%%%%%%%%%%%%%%%%%
% \paragraph{Forwarding for Final Version of the Chapters.}
%
% The following forwarding files |cdocsfn1.tex| and |cdocsfn2.tex|
% (with identical content)
% compile the final versions of the child documents
% |cdocsch1.tex| and |cdocsch2.tex|, respectively:
%\iffalse
%<*samplefinal>
%\fi
%    \begin{macrocode}
\def\version{final}
\input{childdoc.def}
\childdocforwardprefix[cdocsamp]{cdocsfn}{cdocsch}
%    \end{macrocode}

%\iffalse
%</samplefinal>
%\fi
%
% %%%%%%%%%%%%%%%%%%%%%%%%%%%%%%%%%%%%%%
% \paragraph{Command Line Processing.}
%
% The following three command lines generate the output files
% |cdocscld|, |cdocscl1| and |cdocscl2|
% which should be identical to
% |cdocsdrf|, |cdocsch1| and |cdocsfn2|, respectively:
% \begin{center}
% \begin{tabular}{l}
% |latex -jobname cdocscld \|\\
% |  "\def\version{draft}\input{childdoc.def}\childdocforward{cdocsamp}"|\\
% |latex -jobname cdocscl1 \|\\
% |  "\input{childdoc.def}\childdocforward[cdocsamp]{cdocsch1}"|\\
% |latex -jobname cdocscl2 \|\\
% |  "\def\version{final}\input{childdoc.def}\childdocforward{cdocsch2}"|
% \end{tabular}
% \end{center}
% Note that the trailing backslash on each first line
% merely continues the input to the second line
% (for convenient cut ant paste).
% Furthermore, the command |latex| can be replaced by any
% of its alternative versions such as |pdflatex|.
%
% %%%%%%%%%%%%%%%%%%%%%%%%%%%%%%%%%%%%%%%%%%%%%%%%%%%%%%%%%%%%%%%%%%%%%%%%%%%%%%
% %%%%%%%%%%%%%%%%%%%%%%%%%%%%%%%%%%%%%%%%%%%%%%%%%%%%%%%%%%%%%%%%%%%%%%%%%%%%%%
% \section{Implementation}
%\iffalse
%<*package>
%\fi
%
% This section describes the definitions file |childdoc.def|.

% The definitions cannot be loaded using |\usepackage| or |\RequirePackage|
% which has a mechanism to prevent loading a style file more than once.
% When loading the definitions by means of |\input|
% multiple instances have to be prevented manually:
%\iffalse
%This code needs to be before the `\ProvidesFile' directive
%which is defined at the beginning of this file.
%Therefore it is also placed there and commented out here.
%</package>
%<*discard>
%\fi
%    \begin{macrocode}
\ifdefined\childdocmain\endinput\fi
%    \end{macrocode}
%\iffalse
%</discard>
%<*package>
%\fi
%
% \macro{\ifchilddoc}
% \macro{\ifchilddocmanual}
% The conditional |\ifchilddoc| tells whether a
% child (true) or main (false) document is being compiled.
% The conditional |\ifchilddocmanual| tells whether
% the |\includeonly| mechanism is used (false) or
% the selection of child files must be performed manually (true).
% The definitions initialise to false:
%    \begin{macrocode}
\newif\ifchilddoc
\newif\ifchilddocmanual
%    \end{macrocode}

% \macro{\childdocname}
% \macro{\childdocjob}
% The macro |\childdocname| stores the name of the main document
% to be compiled. The macro |\childdocjob| stores the name of
% the document on which the \LaTeX{} compiler was originally invoked.
% The content of |\jobname| cannot be compared
% to filenames specified in the source due to different catcodes.
% The following code rescans |\jobname|, stores the result
% in |\childdocname| and saves a copy in |\childdocjob|:
%    \begin{macrocode}
\edef\childdocname{\scantokens\expandafter{\jobname\noexpand}}
\let\childdocjob\childdocname
%    \end{macrocode}

% \macro{\childdocdisable}
% The macro |\childdocdisable| prevents the main file
% from being processed more than once.
% At this stage, the main document command |\childdocmain|
% is assumed to be called once again where it should do nothing.
% Any subsequent call to it should prevent
% a secondary processing of the main document
% It overwrites the forwarding commands
% |\childdocof| and |\childdocforward|
% with empty macros to prevent further inclusions of the main document:
%    \begin{macrocode}
\newcommand{\childdocdisable}
{
  \renewcommand{\childdocmain}[1]{\renewcommand{\childdocmain}[1]{\endinput}}
  \renewcommand{\childdocof}[1]{}
  \renewcommand{\childdocby}[2][]{}
  \renewcommand{\childdocforward}[2][]{}
  \renewcommand{\childdocdisable}{}
}
%    \end{macrocode}

% \macro{\childdocmain}
% The macro |\childdocmain| is to be called at the top of the main file
% with nothing or the main filename (without extension) as argument.
% First, it breaks loops.
% If the argument is not empty and does not match |\childdocname|
% (which is set by the first inclusion of |childdoc.def|),
% |\ifchilddoc| is set to true, |\includeonly| is applied to the child file
% and |\jobname| is set to the main file
% (for proper handling of |.aux| files):
%    \begin{macrocode}
\newcommand{\childdocmain}[1]
{
  \childdocdisable\childdocmain{}
  \if?#1?\else
    \begingroup
      \def\childdoctmp{#1}
      \ifx\childdoctmp\childdocname
        \def\childdoctmp{}
      \else
        \def\childdoctmp
        {
          \childdoctrue
          \includeonly{\childdocname}
          \def\childdocjob{#1}
          \def\jobname{#1}
        }
      \fi
      \expandafter
    \endgroup
    \childdoctmp
  \fi
}
%    \end{macrocode}

% \macro{\childdocof}
% The command |\childdocof| redirects
% compilation to the main file |#1|.
%    \begin{macrocode}
\newcommand{\childdocof}[1]
{
  \childdocdisable
  \childdoctrue
  \includeonly{\childdocname}
  \def\jobname{#1}
  \def\childdocjob{#1}
  \input{#1}
}
%    \end{macrocode}

% \macro{\childdocby}
% The command |\childdocby| ....
%    \begin{macrocode}
\newcommand{\childdocby}[2][]
{
  \childdocdisable
  \childdoctrue
  \childdocmanualtrue
  \if?#1?\else
    \def\jobname{#2}
  \fi
  \def\childdocjob{#2}
  \input{#2}
  \endinput
}
%    \end{macrocode}

% \macro{\childdocforward}
% The command |\childdocforward| redirects
% compilation to the main file or
% (if the optional argument is given) a child file.
% Parameters are set as if the main file
% or a child file starting with |\childdocof| was compiled.
% Then compilation is handed over to the main file:
%    \begin{macrocode}
\newcommand{\childdocforward}[2][]
{
  \begingroup
    \if?#1?
      \def\childdoctmp
      {
        \def\childdocname{#2}
        \def\childdocjob{#2}
        \def\jobname{#2}
        \input{#2}
        \endinput
      }
    \else
      \def\childdoctmp
      {
        \childdocdisable
        \def\childdocname{#2}
        \childdoctrue
        \includeonly{#2}
        \def\childdocjob{#1}
        \def\jobname{#1}
        \input{#1}
        \endinput
      }
    \fi
    \expandafter
  \endgroup
  \childdoctmp
}
%    \end{macrocode}

% \macro{\childdocforwardprefix}
% The command |\childdocforwardprefix| redirects
% compilation to the main or a child file by means of a pattern.
% The prefix |#1| in the current filename is replaced by |#2|
% and the suffix of the current filename is kept
% (it is assumed that the filename does not contain the substring `|~~~|'
% which is used as a delimiter).
% Compilation is handed over to the new file by |\childdocforward|:
%    \begin{macrocode}
\newcommand{\childdocforwardprefix}[3][]
{
  \begingroup
    \def\childdocextract #2##1~~~{\def\childdoctmp{\childdocforward[#1]{#3##1}}}
    \expandafter\childdocextract\childdocname~~~
    \expandafter
  \endgroup
  \childdoctmp
}
%    \end{macrocode}

% \macro{\childdoc}
% The deprecated macro |\childdoc| is a legacy version of |\childdocmain|:
%    \begin{macrocode}
\newcommand{\childdoc}{\childdocmain}
%    \end{macrocode}

% \macro{\childdocredirect}
% The deprecated macro |\childdocredirect| is a legacy version
% of |\childdocforward| and |\childdocforwardprefix|:
%    \begin{macrocode}
\newcommand{\childdocredirect}[2][]
{
  \begingroup
    \if?#1?
      \def\childdoctmp{\childdocforward{#2}}
    \else
      \def\childdoctmp{\childdocforwardprefix{#1}{#2}}
    \fi
    \expandafter
  \endgroup
  \childdoctmp
}
%    \end{macrocode}

%\iffalse
%</package>
%\fi
%
\endinput
|\\
|\childdocmain{}|\\
\end{tabular}
\end{center}
at the very top of the main \LaTeX{} file,
in particular \emph{before} the |\documentclass| statement!
The argument of |\childdocmain| should be left empty
(but it must be present).

%%%%%%%%%%%%%%%%%%%%%%%%%%%%%%%%%%%%%%%%
\DescribeMacro{\childdocof}
Furthermore, add the commands
\begin{center}
\begin{tabular}{l}
|% \iffalse
%
% childdoc.dtx Copyright (C) 2017-2018 Niklas Beisert
%
% This work may be distributed and/or modified under the
% conditions of the LaTeX Project Public License, either version 1.3
% of this license or (at your option) any later version.
% The latest version of this license is in
%   http://www.latex-project.org/lppl.txt
% and version 1.3 or later is part of all distributions of LaTeX
% version 2005/12/01 or later.
%
% This work has the LPPL maintenance status `maintained'.
%
% The Current Maintainer of this work is Niklas Beisert.
%
% This work consists of the files childdoc.dtx and childdoc.ins
% and the derived files childdoc.def and cdocsamp.tex with
% cdocsch1.tex, cdocsch2.tex, cdocsdrf.tex, cdocsfn1.tex, cdocsfn2.tex.
%
%<package>\ifdefined\childdocmain\endinput\fi
%<package>\ProvidesFile{childdoc.def}[2018/12/30 v2.0 child document driver]
%<samplemain>\ProvidesFile{cdocsamp.tex}[2018/12/30 v2.0 sample for childdoc]
%<*driver>
%\ProvidesFile{childdoc.drv}[2018/12/30 v2.0 childdoc reference manual file]
\PassOptionsToClass{10pt,a4paper}{article}
\documentclass{ltxdoc}

\usepackage[margin=35mm]{geometry}
\usepackage{hyperref}
\usepackage{hyperxmp}
\usepackage[usenames]{color}

\hypersetup{colorlinks=true}
\hypersetup{pdfstartview=FitH}
\hypersetup{pdfpagemode=UseNone}
\hypersetup{pdfsource={}}
\hypersetup{pdflang={en-UK}}
\hypersetup{pdfcopyright={Copyright 2017-2018 Niklas Beisert.
  This work may be distributed and/or modified under the
  conditions of the LaTeX Project Public License, either version 1.3
  of this license or (at your option) any later version.}}
\hypersetup{pdflicenseurl={http://www.latex-project.org/lppl.txt}}
\hypersetup{pdfcontactaddress={ETH Zurich, ITP, HIT K,
  Wolfgang-Pauli-Strasse 27}}
\hypersetup{pdfcontactpostcode={8093}}
\hypersetup{pdfcontactcity={Zurich}}
\hypersetup{pdfcontactcountry={Switzerland}}
\hypersetup{pdfcontactemail={nbeisert@itp.phys.ethz.ch}}
\hypersetup{pdfcontacturl={http://people.phys.ethz.ch/\xmptilde nbeisert/}}

\newcommand{\secref}[1]{\hyperref[#1]{section \ref*{#1}}}

\parskip1ex
\parindent0pt
\let\olditemize\itemize
\def\itemize{\olditemize\parskip0pt}

\begin{document}

\title{The \textsf{childdoc} Package}
\hypersetup{pdftitle={The childdoc Package}}
\author{Niklas Beisert\\[2ex]
  Institut f\"ur Theoretische Physik\\
  Eidgen\"ossische Technische Hochschule Z\"urich\\
  Wolfgang-Pauli-Strasse 27, 8093 Z\"urich, Switzerland\\[1ex]
  \href{mailto:nbeisert@itp.phys.ethz.ch}
  {\texttt{nbeisert@itp.phys.ethz.ch}}}
\hypersetup{pdfauthor={Niklas Beisert}}
\hypersetup{pdfsubject={Manual for the LaTeX2e Package childdoc}}
\date{30 December 2018, \textsf{v2.0}}
\maketitle

\begin{abstract}\noindent
\textsf{childdoc} is a \LaTeXe{} package
that enables the direct compilation
of document sections included by |\include|
to individual files.
\end{abstract}

\begingroup
\parskip0ex
\tableofcontents
\endgroup

%%%%%%%%%%%%%%%%%%%%%%%%%%%%%%%%%%%%%%%%%%%%%%%%%%%%%%%%%%%%%%%%%%%%%%%%%%%%%%%%
%%%%%%%%%%%%%%%%%%%%%%%%%%%%%%%%%%%%%%%%%%%%%%%%%%%%%%%%%%%%%%%%%%%%%%%%%%%%%%%%
\section{Introduction}

\LaTeX{} provides a mechanism to structure a large document (such as a book)
into a main file and several child files (containing the chapters)
using the |\include| command.
This mechanism is beneficial for documents
which span hundreds of pages in order to
make the source file(s) more manageable.
Moreover, compilation can be restricted to
selected child files by means of the |\includeonly| command.
The latter feature can be used to reduce the compilation time while editing
(this was significantly more useful in the earlier days of \LaTeX{})
or to generate a smaller document which is easier to navigate.
Another application of |\includeonly| is to generate
documents consisting of selected parts of the complete document.

However, there are a few drawbacks of the plain |\include| mechanism:
\begin{itemize}
\item
The child files cannot be compiled on their own,
they can only be compiled via the main file.
A naive editing environment
(such as a text editor with an option
to have the current file processed by \LaTeX)
may require one to switch to the main file before compiling;
attempting to compile the child file produces errors.
\item
The main file must be modified (each time)
to adjust the |\includeonly| command
to the present needs. This easily leaves the main file in a messy state.
\item
The generated document will always carry the filename
of the main document. This is inconvenient if
several child files are to be compiled and
to be kept for distribution.
\end{itemize}

The present package provides a simple interface
to make child files individually compilable by \LaTeX{}.
Compiling a child file then has the same effect as compiling
the main file with an |\includeonly| command
to select the appropriate child.
Moreover the generated document will carry the name of the child
rather than the main file.
This resolves all three above issues.

This feature is meant to make the editing of books,
thesis documents and lecture notes somewhat more convenient.
However, the package can also be used efficiently for
composing a series of documents (such as exercise sheets)
which are typically distributed individually.
It then assists the author in generating the individual documents
(potentially in different versions)
as well as a document containing the collected series.
Another application is in developing style files
or other kinds of included material
where compilation of the style file could redirect
to a sample or test file.

%%%%%%%%%%%%%%%%%%%%%%%%%%%%%%%%%%%%%%%%%%%%%%%%%%%%%%%%%%%%%%%%%%%%%%%%%%%%%%%%
%%%%%%%%%%%%%%%%%%%%%%%%%%%%%%%%%%%%%%%%%%%%%%%%%%%%%%%%%%%%%%%%%%%%%%%%%%%%%%%%
\section{Usage}

First of all, the package \textsf{childdoc} is \emph{not} a standard
\LaTeXe{} |.sty| style file! Therefore it needs to be invoked in
a non-standard way.

%%%%%%%%%%%%%%%%%%%%%%%%%%%%%%%%%%%%%%%%%%%%%%%%%%%%%%%%%%%%%%%%%%%%%%%%%%%%%%%%
\subsection{Included Files}
\label{sec:include}

%%%%%%%%%%%%%%%%%%%%%%%%%%%%%%%%%%%%%%%%
\DescribeMacro{\childdocmain}
To use the package, add the commands
\begin{center}
\begin{tabular}{l}
|\input{childdoc.def}|\\
|\childdocmain{}|\\
\end{tabular}
\end{center}
at the very top of the main \LaTeX{} file,
in particular \emph{before} the |\documentclass| statement!
The argument of |\childdocmain| should be left empty
(but it must be present).

%%%%%%%%%%%%%%%%%%%%%%%%%%%%%%%%%%%%%%%%
\DescribeMacro{\childdocof}
Furthermore, add the commands
\begin{center}
\begin{tabular}{l}
|\input{childdoc.def}|\\
|\childdocof{|\textit{main}|}|\\
\end{tabular}
\end{center}
at the top of every child file \textit{child}
which is included by |\include{|\textit{child}|}|
from within the main file
(or at least for those files to be compiled individually).
The argument \textit{main} must be the filename of the main file.

There are a couple of
considerations in setting up the main and child documents:

%%%%%%%%%%%%%%%%%%%%%%%%%%%%%%%%%%%%%%%%
\paragraph{Restrictions.}

Please note the following restrictions:
\begin{itemize}
\item
|\childdocmain| must be called with one argument \textit{main}
to ensure compatibility with earlier version of the package.
It must either be empty (|\childdocmain{}|)
or precisely match the filename of the main file in which it is specified.
See \secref{sec:detection} for further information.
\item
The filename \textit{main} must be specified without the |.tex| extension.
\item
The filename \textit{main} is case sensitive
(even in case-insensitive file systems)
due to internal string comparison.
\item
The argument \textit{main} should be fully expanded, it cannot be a macro.
\item
Subdirectories and special characters should be avoided in filenames.
\item
The command |\childdocmain{|\textit{main}|}| must be followed by a whitespace.
It should not be followed immediately by another command
or by a comment mark `|%|'.
This is because the \TeX{} parser reads the token immediately following
the argument of |\childdocmain| and puts it
at the beginning of every child section;
however, a white\-space is ignored.
\end{itemize}

%%%%%%%%%%%%%%%%%%%%%%%%%%%%%%%%%%%%%%%%
\paragraph{Content of Main File.}

It is advisable to place all content in the child files included by |\include|.
Any output contained in the main file will appear in all child documents
unless suppressed manually;
it cannot be suppressed automatically by the |\includeonly| directive
and thus should normally be avoided.
A method to include some content in the main file
by means of conditional processing is described in \secref{sec:conditional}.

%%%%%%%%%%%%%%%%%%%%%%%%%%%%%%%%%%%%%%%%
\paragraph{Page Numbering.}

When only a part of the document is compiled,
the appropriate numbering of pages
(as well as other status parameters)
is determined from the |.aux| files.
The latter contain information from previous passes.
However this information needs to propagate through
all intermediate child documents.
Therefore the page numbering in child documents may well
be inconsistent until the complete document is compiled at least once.

A useful (if unconventional) way to always ensure a consistent
page numbering is to restart the numbering in each child document
and denote the pages by `\textit{child}|.|\textit{page}'
where \textit{child} represents the chapter/section number of the child file.
This can be achieved by the command
|\numberwithin{page}{|\textit{child}|}|
of the \textsf{amsmath} package
where \textit{child} can be |chapter| or |section|
depending on the chosen structuring.
Alternatively, one can modify the macro |\thepage| appropriately
and reset the counter |page| at the start of each child file.

%%%%%%%%%%%%%%%%%%%%%%%%%%%%%%%%%%%%%%%%%%%%%%%%%%%%%%%%%%%%%%%%%%%%%%%%%%%%%%%%
\subsection{Conditional Processing}
\label{sec:conditional}

The package provides a mechanism to compile different versions
of a document. To customise the versions further some conditional processing
can come in handy to distinguish which version is being compiled.
The package provides two macros to describe the compilation context:

%%%%%%%%%%%%%%%%%%%%%%%%%%%%%%%%%%%%%%%%
\DescribeMacro{\ifchilddoc}
The conditional |\ifchilddoc| distinguishes between the compilation of
child documents and the main document:
%
\begin{center}
|\ifchilddoc |\textit{child-code}| |[|\||else |\textit{main-code}]| \||fi|
\end{center}

%%%%%%%%%%%%%%%%%%%%%%%%%%%%%%%%%%%%%%%%
\DescribeMacro{\childdocname}
\DescribeMacro{\childdocjob}
The macro |\childdocname| contains the filename (without extension)
of the main or child file being processed.
Note that |\childdocjob| will always contain the name of the main file.

%%%%%%%%%%%%%%%%%%%%%%%%%%%%%%%%%%%%%%%%
\paragraph{Title Page.}

Conditional processing can be used to include a title or banner page
in the main document when proper precautions are taken.
Importantly, the code in the main file should ensure that the page counter
(as well as other status parameters which are stored in the |.aux| files)
takes the same value after the conditional processing.
Otherwise the page numbers may take divergent values
depending on which part is compiled.

For example, a title page could be declared by:
%
\begin{center}
\begin{tabular}{l}
|\ifchilddoc\||else|\\
|\addtocounter{page}{-1}|\\
\textit{code for title page}\\
|\newpage|\\
|\||fi|
\end{tabular}
\end{center}
%
A banner page for the child documents can be generated by:
%
\begin{center}
\begin{tabular}{l}
|\ifchilddoc|\\
|\addtocounter{page}{-1}|\\
\textit{code for banner page}\\
|\newpage|\\
|\||fi|
\end{tabular}
\end{center}
%
Here one could write a message such as:
\begin{center}
|This is the part \childdocname{} of \childdocjob{}.|
\end{center}

%%%%%%%%%%%%%%%%%%%%%%%%%%%%%%%%%%%%%%%%%%%%%%%%%%%%%%%%%%%%%%%%%%%%%%%%%%%%%%%%
\subsection{Flags}
\label{sec:flags}

The package makes it easy to generate different versions
of the main or child documents.
To this end compilation flags can be defined
and assigned different default values.
They will be particularly useful in conjunction
with the forwarding mechanism described in \secref{sec:forward}.

For example, it may be useful to have a flag |\version|
which can be set to |draft| or |final|.
The document source will contain some conditional code
depending on the value of |\version|.
Suppose further, the flag should default to |final| for the main file
and to |draft| for child files
which is a natural assignment for editing the document.
This is achieved by placing the following code
in the preamble of the main document
(below the |\childdocmain| directive):
%
\begin{center}
\begin{tabular}{l}
|\ifchilddoc|\\
|\providecommand{\version}{draft}|\\
|\||else|\\
|\providecommand{\version}{final}|\\
|\||fi|
\end{tabular}
\end{center}
%
The definition by |\providecommand| makes sure
that previous definitions are not overwritten.
Further statements |\providecommand{\version}{...}|
can thus be added before the above code to override it.

For the main file, one might add a line
(between |\childdocmain| and the above block)
%
\begin{center}
|%\ifchilddoc\||else\providecommand{\version}{draft}\||fi|
\end{center}
%
which can be uncommented to produce a draft version.
Likewise one can add a line to the very top of a child file
(above the |\childdocof{|\textit{main}|}| directive)
%
\begin{center}
|%\providecommand{\version}{final}|
\end{center}
%
which can be uncommented to produce the final version of this child document.

%%%%%%%%%%%%%%%%%%%%%%%%%%%%%%%%%%%%%%%%%%%%%%%%%%%%%%%%%%%%%%%%%%%%%%%%%%%%%%%%
\subsection{Forwarding}
\label{sec:forward}

Different versions of the main or child documents
using compilation flags as described in \secref{sec:flags}
can be (permanently) stored in different files
for convenient compilation, viewing and distribution.
To this end, the package defines a command
to pass on compilation to a different file:

%%%%%%%%%%%%%%%%%%%%%%%%%%%%%%%%%%%%%%%%
\DescribeMacro{\childdocforward}
The command |\childdocforward| redirects processing to
another source file:
%
\begin{center}
\begin{tabular}{l}
|\input{childdoc.def}|\\
|\childdocforward[|\textit{main}|]{|\textit{dest}|}|\\
\end{tabular}
\end{center}
%
The argument \textit{dest} is the destination file
(without extension).
It should be the main file or one of the child files.
Note that further \textsf{childdoc} directives
such as |\childdocof| and |\childdocforward|
in the indicated file will be processed in this form.
The optional argument \textit{main}
passes on directly to the main file \textit{main}
while pretending to compile the child \textit{dest}.
This form behaves as if \textit{dest}
issues |\childdocof{|\textit{main}|}| right away,
and no further \textsf{childdoc} directives will be processed.

%%%%%%%%%%%%%%%%%%%%%%%%%%%%%%%%%%%%%%%%
\DescribeMacro{\...prefix}
In the alternative form |\childdocforwardprefix|,
%
\begin{center}
\begin{tabular}{l}
|\input{childdoc.def}|\\
|\childdocforwardprefix[|\textit{main}|]{|\textit{prefix}|}{|\textit{dest}|}|
\end{tabular}
\end{center}
%
the destination file is determined by a pattern
depending on the current file:
To make this work, the current file must be called
`{\textit{prefix}\hspace{0.2em}\textit{suffix}}'
with \textit{prefix} matching precisely the argument.
Processing is then passed on to the file
`{\textit{dest}\hspace{0.2em}\textit{suffix}}'.
Surely, the same effect is achieved by
directly specifying the
argument `{\textit{dest}\hspace{0.2em}\textit{suffix}}'
in the first form.
However, that requires to set up a different file
for each child. With the alternative form of the command
all these files can have exactly the same content
which simplifies setting them up and maintaining them.

For example, the following file |draft.tex|
with a compilation flag |\version| as described in \secref{sec:flags}
compiles the main document as a draft:
%
\begin{center}
\begin{tabular}{l}
|\def\version{draft}|\\
|\input{childdoc.def}|\\
|\childdocforward{|\textit{main}|}|
\end{tabular}
\end{center}
%
Likewise, the following files |final|\textit{nn}|.tex|
compile the final version of the child document
|child|\textit{nn}|.tex|:
%
\begin{center}
\begin{tabular}{l}
|\def\version{final}|\\
|\input{childdoc.def}|\\
|\childdocforwardprefix{final}{child}|
\end{tabular}
\end{center}
%

Note that when several versions of a main file and/or of each child file
are to be generated, it may be convenient to set up a |Makefile| or
shell script to automatise the process.

%%%%%%%%%%%%%%%%%%%%%%%%%%%%%%%%%%%%%%%%%%%%%%%%%%%%%%%%%%%%%%%%%%%%%%%%%%%%%%%%
\subsection{Command Line Processing}
\label{sec:commandline}

The effect of redirection files can also be achieved by invoking
the \LaTeX{} compiler with a more elaborate command line.
Most conveniently this should be done as part
of a shell script or a |Makefile|.

When using \textsf{childdoc} in the main file, the following
command lines effectively perform a redirection
(note that depending on the shell being used,
backslashes may have to be doubled: `|\|' $\to$ `|\\|'):
%
\begin{center}
|... -jobname "|\textit{target}|" |\\|"|[\textit{flags}]%
|\input{childdoc.def}\childdocforward[|\textit{main}|]{|\textit{dest}|}"|
\end{center}
%
Here \textit{target} is the name of the output file,
\textit{main} is the name of the main file
and \textit{dest} is the name of the main or child file to be processed
(all filenames without extensions).
The optional argument \textit{main} can be omitted
if \textit{main} matches \textit{dest}.
Optionally, compilation \textit{flags} can be defined via |\def| commands.
This command line makes the \TeX{} engine believe
it is compiling the file \textit{target}
whose content is specified as the latter parameter.
The provided code then forwards the processing to
\textit{main} or \textit{dest} as described in \secref{sec:forward}.

%%%%%%%%%%%%%%%%%%%%%%%%%%%%%%%%%%%%%%%%%%%%%%%%%%%%%%%%%%%%%%%%%%%%%%%%%%%%%%%%
\subsection{Include by Input}
\label{sec:input}

Including child documents by |\include| has some restrictions by design.
Most notably, the content of a child document always occupies
its own set of pages; pages cannot be shared between child documents.
Usually, this behaviour makes perfect sense
because each child document contain an essential part of the document.
However, in some situations it may be desirable to compose
a document from a collection of parts
without having mandatory page breaks between then.
For this case, the package
provides a mechanism to include parts
by |\input| which can also be processed individually.
However, by construction this mechanism
requires manual handling of the content to be output.

%%%%%%%%%%%%%%%%%%%%%%%%%%%%%%%%%%%%%%%%
\DescribeMacro{\ifchilddocmanual}
The main file should be prepared as usual, see \secref{sec:include}.
However, the document body must make a distinction
between processing of an individual part and of the main document, e.g.:
%
\begin{center}
\begin{tabular}{l}
|\ifchilddocmanual|\\
|\input{\childdocname}|\\
|\||else|\\
\textit{document body with }|\input{|\textit{part}|}|\\
|\||fi|
\end{tabular}
\end{center}
%
The conditional |\ifchilddocmanual| is true whenever
a part to be included by |\input| is being compiled,
and the name of the part is stored in |\childdocname|.

%%%%%%%%%%%%%%%%%%%%%%%%%%%%%%%%%%%%%%%%
\DescribeMacro{\childdocby}
Each part to be included by |\input| should start with:
%
\begin{center}
\begin{tabular}{l}
|\input{childdoc.def}|\\
|\childdocby{|\textit{main}|}|\\
\end{tabular}
\end{center}
%
The directive |\childdocby| is similar to |\childdocof|
described in \secref{sec:include},
but the subsequent selection of content must be done manually.
To that end, both |\ifchilddoc| and |\ifchilddocmanual|
will be true upon processing of a part,
and the name of the part is stored in |\childdocname|.
Note that |\jobname| will be set to the filename of the current part
so that each part receives an individual |.aux| file
that does not interfere with the |.aux| file(s) of the main document.
This behaviour can be altered by the alternative form
|\childdocby[*]{|\textit{main}|}| (with a non-empty optional argument)
which uses the |.aux| file of the main document
by setting |\jobname| to \textit{main}.

%%%%%%%%%%%%%%%%%%%%%%%%%%%%%%%%%%%%%%%%%%%%%%%%%%%%%%%%%%%%%%%%%%%%%%%%%%%%%%%%
\subsection{Driver Development}
\label{sec:driver}

The \textsf{childdoc} mechanism can also be use for the development
of definition files such as \LaTeX{} styles or classes.
This case differs from the above setup with multiple parts
included by |\include| in that no |\includeonly| should be invoked.
This can be achieved by starting the include file
(before |\ProvidesPackage|) with:
%
\begin{center}
\begin{tabular}{l}
|\input{childdoc.def}|\\
|\childdocforward{|\textit{main}|}|\\
\end{tabular}
\end{center}
%
or alternatively with:
%
\begin{center}
\begin{tabular}{l}
|\input{childdoc.def}|\\
|\childdocby{|\textit{main}|}|\\
\end{tabular}
\end{center}
%
Both forms have slightly different effects as described above.
The main file is prepared as usual, see \secref{sec:include}.

%%%%%%%%%%%%%%%%%%%%%%%%%%%%%%%%%%%%%%%%%%%%%%%%%%%%%%%%%%%%%%%%%%%%%%%%%%%%%%%%
\subsection{Legacy Detection}
\label{sec:detection}

The directive |\childdocmain| in the main file can detect
whether the complete document or merely a child is to be compiled
even without using the directive |\childdocof|.
This method is deprecated because it is less robust
and there is no compelling reason to use it;
it is merely provided for backward compatibility
and it may be removed in future versions.

If the detection mechanism is to be used,
it is mandatory to correctly specify
the filename of the main file as the argument of |\childdocmain|:
%
\begin{center}
\begin{tabular}{l}
|\input{childdoc.def}|\\
|\childdocmain{|\textit{main}|}|\\
\end{tabular}
\end{center}
%
If |\jobname| does not match the argument \textit{main} of |\childdocmain|,
it is assumed that |\jobname| points to the child file to be compiled.
When using |\childdocmain| with the main file specified as argument,
it suffices to start a child file
with just |\input{|\textit{main}|}|
without loading of the package and using |\childdocof|.
If instead all processing is done
with the appropriate \textsf{childdoc} directives,
the argument of \textit{main} of |\childdocmain| can be empty.

An alternative version of the command line processing described
in \secref{sec:commandline} using the detection mechanism reads:
%
\begin{center}
|... -jobname "|\textit{target}|" "|[\textit{flags}]%
[|\def\jobname{|\textit{dest}|}|]|\input{|\textit{main}|}"|
\end{center}

%%%%%%%%%%%%%%%%%%%%%%%%%%%%%%%%%%%%%%%%%%%%%%%%%%%%%%%%%%%%%%%%%%%%%%%%%%%%%%%%
\subsection{Manual Code}
\label{sec:manual}

In case one cannot be certain whether the definitions file |childdoc.def|
is installed on the target \TeX{} distribution
and one prefers not to ship it,
it is conceivable to paste a few relevant commands into the sources.

To that end, drop all statements |\input{childdoc.def}|
and perform the replacements as outlined below.
Instead of |\childdocmain{|\textit{main}|}| add the following code
to the top of the main file:
%
\begin{center}
\begin{tabular}{l}
|\||ifdefined\childdocname\endinput\||fi\newif\ifchilddoc|\\
|\edef\childdocname{\scantokens\expandafter{\jobname\noexpand}}|\\
|\def\childdocmain{|\textit{main}|}\||ifx\childdocmain\childdocname\||else|\\
|\childdoctrue\includeonly{\childdocname}\let\jobname\childdocmain\||fi|\\
\end{tabular}
\end{center}
%
Instead of |\childdocof{|\textit{main}|}| just include the main file
at the top of each child file:
%
\begin{center}
|\input{|\textit{main}|}|
\end{center}
%
A simple redirection |\childdocforward{|\textit{dest}|}| is achieved by:
%
\begin{center}
|\def\jobname{|\textit{dest}|}\input{\jobname}|
\end{center}
%
The redirection with prefix
|\childdocforwardprefix[|\textit{prefix}|]{|\textit{dest}|}|
is accomplished by:
%
\begin{center}
\begin{tabular}{l}
|{\edef\jobname{\scantokens\expandafter{\jobname\noexpand}}|\\
|\def\redirectjob |\textit{prefix}|#1~~~{\gdef\jobname{|\textit{dest}|#1}}|\\
|\expandafter\redirectjob\jobname~~~}\input{\jobname}|
\end{tabular}
\end{center}

In an alternative approach,
child documents can be compiled by a specific command line
without additional code or specific definitions:
%
\begin{center}
|... -jobname "|\textit{target}|" "|[\textit{flags}]%
|\includeonly{|\textit{dest}|}\input{|\textit{main}|}"|
\end{center}
%

%%%%%%%%%%%%%%%%%%%%%%%%%%%%%%%%%%%%%%%%%%%%%%%%%%%%%%%%%%%%%%%%%%%%%%%%%%%%%%%%
%%%%%%%%%%%%%%%%%%%%%%%%%%%%%%%%%%%%%%%%%%%%%%%%%%%%%%%%%%%%%%%%%%%%%%%%%%%%%%%%
\section{Information}

%%%%%%%%%%%%%%%%%%%%%%%%%%%%%%%%%%%%%%%%%%%%%%%%%%%%%%%%%%%%%%%%%%%%%%%%%%%%%%%%
\subsection{Copyright}

Copyright \copyright{} 2017--2018 Niklas Beisert

This work may be distributed and/or modified under the
conditions of the \LaTeX{} Project Public License, either version 1.3
of this license or (at your option) any later version.
The latest version of this license is in
  \url{http://www.latex-project.org/lppl.txt}
and version 1.3 or later is part of all distributions of \LaTeX{}
version 2005/12/01 or later.

This work has the LPPL maintenance status `maintained'.

The Current Maintainer of this work is Niklas Beisert.

This work consists of the files |README.txt|, |childdoc.ins| and |childdoc.dtx|
as well as the derived files |childdoc.def|, |cdocsamp.tex|
with |cdocsch1.tex|, |cdocsch2.tex|, |cdocspt3.tex|, |cdocspt4.tex|,
|cdocsdrf.tex|, |cdocsfn1.tex|, |cdocsfn2.tex|
as well as |childdoc.pdf|.

%%%%%%%%%%%%%%%%%%%%%%%%%%%%%%%%%%%%%%%%%%%%%%%%%%%%%%%%%%%%%%%%%%%%%%%%%%%%%%%%
\subsection{Files and Installation}

The package consists of the files:
%
\begin{center}
\begin{tabular}{ll}
    |README.txt|   & readme file \\
    |childdoc.ins| & installation file \\
    |childdoc.dtx| & source file \\
    |childdoc.def| & definition file \\
    |cdocsamp.tex| & sample main file \\
    |cdocsch1.tex| & sample include file \\
    |cdocsch2.tex| & sample include file \\
    |cdocspt3.tex| & sample part file \\
    |cdocspt4.tex| & sample part file \\
    |cdocsdrf.tex| & sample redirection file \\
    |cdocsfn1.tex| & sample redirection file \\
    |cdocsfn2.tex| & sample redirection file \\
    |childdoc.pdf| & manual
\end{tabular}
\end{center}
%
The distribution consists of the files
|README.txt|, |childdoc.ins| and |childdoc.dtx|.
%
\begin{itemize}
\item
Run (pdf)\LaTeX{} on |childdoc.dtx|
to compile the manual |childdoc.pdf| (this file).
\item
Run \LaTeX{} on |childdoc.ins| to create the definitions file |childdoc.def|
and the sample |cdocsamp.tex| with include files
|cdocsch1.tex|, |cdocsch2.tex|, |cdocspt3.tex|, |cdocspt4.tex|,
|cdocsdrf.tex|, |cdocsfn1.tex|, |cdocsfn2.tex|.
Then copy the file |childdoc.def| to an appropriate directory of your \LaTeX{}
distribution, e.g.\ \textit{texmf-root}|/tex/latex/childdoc|.
\end{itemize}

%%%%%%%%%%%%%%%%%%%%%%%%%%%%%%%%%%%%%%%%%%%%%%%%%%%%%%%%%%%%%%%%%%%%%%%%%%%%%%%%
\subsection{Related CTAN Packages}

There are several other packages which offer a similar functionality:
%
\begin{itemize}
\item
The packages
\href{http://ctan.org/pkg/docmute}{\textsf{docmute}},
\href{http://ctan.org/pkg/includex}{\textsf{includex}} and
\href{http://ctan.org/pkg/standalone}{\textsf{standalone}}
provide commands to include only the document body of
a child file thus allowing both files to be compiled individually.
\item
The packages \href{http://ctan.org/pkg/subdocs}{\textsf{subdocs}}
and \href{http://ctan.org/pkg/subfiles}{\textsf{subfiles}}
provide structures in which the main and child documents can be
encapsulated and allowing them to be compiled individually.
The inclusion mechanism is different from the conventional |\include|.
\item
The package \href{http://ctan.org/pkg/combine}{\textsf{combine}}
is an elaborate solution to combine several documents into one.
\end{itemize}
%
See also the CTAN topic \href{http://ctan.org/topic/subdocs}{\textsf{subdocs}}
for further related packages.
The present package differs from the above solutions in that
a document structure constructed with the conventional |\include| mechanism
just needs two extra commands at the top of every file
such that all constituent files can be compiled individually.

%%%%%%%%%%%%%%%%%%%%%%%%%%%%%%%%%%%%%%%%%%%%%%%%%%%%%%%%%%%%%%%%%%%%%%%%%%%%%%%%
%\subsection{Feature Suggestions}
%
%The following is a list of features which may be useful for future
%versions of this package:
%%
%\begin{itemize}
%\item
%\ldots
%\end{itemize}

%%%%%%%%%%%%%%%%%%%%%%%%%%%%%%%%%%%%%%%%%%%%%%%%%%%%%%%%%%%%%%%%%%%%%%%%%%%%%%%%
\subsection{Revision History}

%%%%%%%%%%%%%%%%%%%%%%%%%%%%%%%%%%%%%%%%
\paragraph{v2.0:} 2018/12/30

\begin{itemize}
\item
immediate forward processing
\item
added |\childdocby| mechanism
\item
manual restructured
\end{itemize}

%%%%%%%%%%%%%%%%%%%%%%%%%%%%%%%%%%%%%%%%
\paragraph{v1.6:} 2018/01/17

\begin{itemize}
\item
application for development of include files
\item
corrections to manual
\end{itemize}

%%%%%%%%%%%%%%%%%%%%%%%%%%%%%%%%%%%%%%%%
\paragraph{v1.5:} 2017/05/21

\begin{itemize}
\item
more complete structuring introduced
\item
|\childdocof| introduced
\item
|\childdoc| renamed to |\childdocmain|
\item
|\childredirect| renamed to |\childdocforward| and |\childdocforwardprefix|
and functionality expanded
\end{itemize}

%%%%%%%%%%%%%%%%%%%%%%%%%%%%%%%%%%%%%%%%
\paragraph{v1.0:} 2017/04/27

\begin{itemize}
\item
manual and install package
\item
first version published on CTAN
\end{itemize}

%%%%%%%%%%%%%%%%%%%%%%%%%%%%%%%%%%%%%%%%
\paragraph{v0.6:} 2017/04/26

\begin{itemize}
\item
redirection mechanism added
\end{itemize}

%%%%%%%%%%%%%%%%%%%%%%%%%%%%%%%%%%%%%%%%
\paragraph{v0.5:} 2017/04/26

\begin{itemize}
\item
functionality in definition file
\end{itemize}


%%%%%%%%%%%%%%%%%%%%%%%%%%%%%%%%%%%%%%%%%%%%%%%%%%%%%%%%%%%%%%%%%%%%%%%%%%%%%%%%
%%%%%%%%%%%%%%%%%%%%%%%%%%%%%%%%%%%%%%%%%%%%%%%%%%%%%%%%%%%%%%%%%%%%%%%%%%%%%%%%
%%%%%%%%%%%%%%%%%%%%%%%%%%%%%%%%%%%%%%%%%%%%%%%%%%%%%%%%%%%%%%%%%%%%%%%%%%%%%%%%
\appendix

\settowidth\MacroIndent{\rmfamily\scriptsize 000\ }

 \DocInput{childdoc.dtx}

\end{document}
%</driver>
% \fi
%
% %%%%%%%%%%%%%%%%%%%%%%%%%%%%%%%%%%%%%%%%%%%%%%%%%%%%%%%%%%%%%%%%%%%%%%%%%%%%%%
% %%%%%%%%%%%%%%%%%%%%%%%%%%%%%%%%%%%%%%%%%%%%%%%%%%%%%%%%%%%%%%%%%%%%%%%%%%%%%%
% \section{Sample}
%\iffalse
%<*samplemain>
%\fi
%
% The following presents a sample document
% with two chapters, two parts, a title page,
% a compile flag as well as three forwarding files to set the flag.
% It consists of eight |.tex| files:
% \begin{center}
% \begin{tabular}{ll}
% |cdocsamp.tex|&main file\\
% |cdocsch1.tex|&include file for chapter 1\\
% |cdocsch2.tex|&include file for chapter 2\\
% |cdocspt3.tex|&include file for part 3\\
% |cdocspt4.tex|&include file for part 4\\
% |cdocsdrf.tex|&forwarding file for main file in draft mode\\
% |cdocsfi1.tex|&forwarding file for final version of chapter 1\\
% |cdocsfi2.tex|&forwarding file for final version of chapter 2\\
% \end{tabular}
% \end{center}
% Each of the eight files can be compiled directly by the \LaTeX{} compiler.
%
% %%%%%%%%%%%%%%%%%%%%%%%%%%%%%%%%%%%%%%
% \paragraph{Main File.}
%
% The main file is called |cdocsamp.tex|.
%
% Load the \textsf{childdoc} definitions and
% declare the filename for the main document:
%    \begin{macrocode}
\input{childdoc.def}
\childdocmain{}
%    \end{macrocode}

% Optional override for |\version| flag:
%    \begin{macrocode}
%%\ifchilddoc\else\providecommand{\version}{draft}\fi
%    \end{macrocode}

% Define the default values for the |\version| flag
% (|final| for the main file and |draft| for childs):
%    \begin{macrocode}
\ifchilddoc
\providecommand{\version}{draft}
\else
\providecommand{\version}{final}
\fi
%    \end{macrocode}

% Load the standard document class:
%    \begin{macrocode}
\documentclass[12pt]{article}
%    \end{macrocode}

% Start the document body:
%    \begin{macrocode}
\begin{document}
%    \end{macrocode}

% Declare a title page.
% Print title, part of document being processed and version flag:
%    \begin{macrocode}
\addtocounter{page}{-1}
\begin{center}
{\LARGE\bfseries{}childdoc example\par}
\vspace{1cm}
\ifchilddoc
\ifchilddocmanual part\else chapter\fi:
`\childdocname' of `\childdocjob'\par
\else
main document: `\childdocjob'\par
\fi
version: \version\par
\end{center}
\newpage
%    \end{macrocode}

% Manually include selected file,
% otherwise process as usual:
%    \begin{macrocode}
\ifchilddocmanual
\section*{part `\childdocname'}
\input{\childdocname}
\else
%    \end{macrocode}

% Include the two chapters:
%    \begin{macrocode}
\include{cdocsch1}
\include{cdocsch2}
%    \end{macrocode}

% Include the two parts unless only chapters should be displayed:
%    \begin{macrocode}
\ifchilddoc\else
\section{part three}
\input{cdocspt3}
\section{part four}
\input{cdocspt4}
\fi
%    \end{macrocode}

% Process as usual until here:
%    \begin{macrocode}
\fi
%    \end{macrocode}

% End of document body:
%    \begin{macrocode}
\end{document}
%    \end{macrocode}
%\iffalse
%</samplemain>
%\fi
%
% %%%%%%%%%%%%%%%%%%%%%%%%%%%%%%%%%%%%%%
% \paragraph{Chapter Include Files.}
%
% The include files are called |cdocsch1.tex| and |cdocsch2.tex|.
%
%\iffalse
%<*samplechap1|samplechap2>
%\fi

% Optional override for |\version| flag:
%    \begin{macrocode}
%%\providecommand{\version}{final}
%    \end{macrocode}

% Include the main document:
%    \begin{macrocode}
\input{childdoc.def}
\childdocof{cdocsamp}
%    \end{macrocode}

%\iffalse
%</samplechap1|samplechap2>
%\fi
%
%\iffalse
%<*samplechap1>
%\fi
% Some text for chapter 1:
%    \begin{macrocode}
\section{one}
some text in chapter one
%    \end{macrocode}

%\iffalse
%</samplechap1>
%\fi
% Some text for chapter 2:
%\iffalse
%<*samplechap2>
%\fi
%    \begin{macrocode}
\section{two}
more text in chapter two
%    \end{macrocode}

%\iffalse
%</samplechap2>
%\fi
%
% %%%%%%%%%%%%%%%%%%%%%%%%%%%%%%%%%%%%%%
% \paragraph{Part Include Files.}
%
% The include files are called |cdocspt3.tex| and |cdocspt4.tex|.
%
%\iffalse
%<*samplepart3|samplepart4>
%\fi

% Optional override for |\version| flag:
%    \begin{macrocode}
%%\providecommand{\version}{final}
%    \end{macrocode}

% Include the main document:
%    \begin{macrocode}
\input{childdoc.def}
\childdocby{cdocsamp}
%    \end{macrocode}

%\iffalse
%</samplepart3|samplepart4>
%\fi
%
%\iffalse
%<*samplepart3>
%\fi
% Some text for part 3:
%    \begin{macrocode}
some text in part three
%    \end{macrocode}

%\iffalse
%</samplepart3>
%\fi
% Some text for part 4:
%\iffalse
%<*samplepart4>
%\fi
%    \begin{macrocode}
more text in part four
%    \end{macrocode}

%\iffalse
%</samplepart4>
%\fi
%
% %%%%%%%%%%%%%%%%%%%%%%%%%%%%%%%%%%%%%%
% \paragraph{Forwarding for a Complete Draft.}
%
% The following forwarding file |cdocsdrf.tex|
% compiles the main document in draft mode:
%\iffalse
%<*sampledraft>
%\fi
%    \begin{macrocode}
\def\version{draft}
\input{childdoc.def}
\childdocforward{cdocsamp}
%    \end{macrocode}

%\iffalse
%</sampledraft>
%\fi
%
% %%%%%%%%%%%%%%%%%%%%%%%%%%%%%%%%%%%%%%
% \paragraph{Forwarding for Final Version of the Chapters.}
%
% The following forwarding files |cdocsfn1.tex| and |cdocsfn2.tex|
% (with identical content)
% compile the final versions of the child documents
% |cdocsch1.tex| and |cdocsch2.tex|, respectively:
%\iffalse
%<*samplefinal>
%\fi
%    \begin{macrocode}
\def\version{final}
\input{childdoc.def}
\childdocforwardprefix[cdocsamp]{cdocsfn}{cdocsch}
%    \end{macrocode}

%\iffalse
%</samplefinal>
%\fi
%
% %%%%%%%%%%%%%%%%%%%%%%%%%%%%%%%%%%%%%%
% \paragraph{Command Line Processing.}
%
% The following three command lines generate the output files
% |cdocscld|, |cdocscl1| and |cdocscl2|
% which should be identical to
% |cdocsdrf|, |cdocsch1| and |cdocsfn2|, respectively:
% \begin{center}
% \begin{tabular}{l}
% |latex -jobname cdocscld \|\\
% |  "\def\version{draft}\input{childdoc.def}\childdocforward{cdocsamp}"|\\
% |latex -jobname cdocscl1 \|\\
% |  "\input{childdoc.def}\childdocforward[cdocsamp]{cdocsch1}"|\\
% |latex -jobname cdocscl2 \|\\
% |  "\def\version{final}\input{childdoc.def}\childdocforward{cdocsch2}"|
% \end{tabular}
% \end{center}
% Note that the trailing backslash on each first line
% merely continues the input to the second line
% (for convenient cut ant paste).
% Furthermore, the command |latex| can be replaced by any
% of its alternative versions such as |pdflatex|.
%
% %%%%%%%%%%%%%%%%%%%%%%%%%%%%%%%%%%%%%%%%%%%%%%%%%%%%%%%%%%%%%%%%%%%%%%%%%%%%%%
% %%%%%%%%%%%%%%%%%%%%%%%%%%%%%%%%%%%%%%%%%%%%%%%%%%%%%%%%%%%%%%%%%%%%%%%%%%%%%%
% \section{Implementation}
%\iffalse
%<*package>
%\fi
%
% This section describes the definitions file |childdoc.def|.

% The definitions cannot be loaded using |\usepackage| or |\RequirePackage|
% which has a mechanism to prevent loading a style file more than once.
% When loading the definitions by means of |\input|
% multiple instances have to be prevented manually:
%\iffalse
%This code needs to be before the `\ProvidesFile' directive
%which is defined at the beginning of this file.
%Therefore it is also placed there and commented out here.
%</package>
%<*discard>
%\fi
%    \begin{macrocode}
\ifdefined\childdocmain\endinput\fi
%    \end{macrocode}
%\iffalse
%</discard>
%<*package>
%\fi
%
% \macro{\ifchilddoc}
% \macro{\ifchilddocmanual}
% The conditional |\ifchilddoc| tells whether a
% child (true) or main (false) document is being compiled.
% The conditional |\ifchilddocmanual| tells whether
% the |\includeonly| mechanism is used (false) or
% the selection of child files must be performed manually (true).
% The definitions initialise to false:
%    \begin{macrocode}
\newif\ifchilddoc
\newif\ifchilddocmanual
%    \end{macrocode}

% \macro{\childdocname}
% \macro{\childdocjob}
% The macro |\childdocname| stores the name of the main document
% to be compiled. The macro |\childdocjob| stores the name of
% the document on which the \LaTeX{} compiler was originally invoked.
% The content of |\jobname| cannot be compared
% to filenames specified in the source due to different catcodes.
% The following code rescans |\jobname|, stores the result
% in |\childdocname| and saves a copy in |\childdocjob|:
%    \begin{macrocode}
\edef\childdocname{\scantokens\expandafter{\jobname\noexpand}}
\let\childdocjob\childdocname
%    \end{macrocode}

% \macro{\childdocdisable}
% The macro |\childdocdisable| prevents the main file
% from being processed more than once.
% At this stage, the main document command |\childdocmain|
% is assumed to be called once again where it should do nothing.
% Any subsequent call to it should prevent
% a secondary processing of the main document
% It overwrites the forwarding commands
% |\childdocof| and |\childdocforward|
% with empty macros to prevent further inclusions of the main document:
%    \begin{macrocode}
\newcommand{\childdocdisable}
{
  \renewcommand{\childdocmain}[1]{\renewcommand{\childdocmain}[1]{\endinput}}
  \renewcommand{\childdocof}[1]{}
  \renewcommand{\childdocby}[2][]{}
  \renewcommand{\childdocforward}[2][]{}
  \renewcommand{\childdocdisable}{}
}
%    \end{macrocode}

% \macro{\childdocmain}
% The macro |\childdocmain| is to be called at the top of the main file
% with nothing or the main filename (without extension) as argument.
% First, it breaks loops.
% If the argument is not empty and does not match |\childdocname|
% (which is set by the first inclusion of |childdoc.def|),
% |\ifchilddoc| is set to true, |\includeonly| is applied to the child file
% and |\jobname| is set to the main file
% (for proper handling of |.aux| files):
%    \begin{macrocode}
\newcommand{\childdocmain}[1]
{
  \childdocdisable\childdocmain{}
  \if?#1?\else
    \begingroup
      \def\childdoctmp{#1}
      \ifx\childdoctmp\childdocname
        \def\childdoctmp{}
      \else
        \def\childdoctmp
        {
          \childdoctrue
          \includeonly{\childdocname}
          \def\childdocjob{#1}
          \def\jobname{#1}
        }
      \fi
      \expandafter
    \endgroup
    \childdoctmp
  \fi
}
%    \end{macrocode}

% \macro{\childdocof}
% The command |\childdocof| redirects
% compilation to the main file |#1|.
%    \begin{macrocode}
\newcommand{\childdocof}[1]
{
  \childdocdisable
  \childdoctrue
  \includeonly{\childdocname}
  \def\jobname{#1}
  \def\childdocjob{#1}
  \input{#1}
}
%    \end{macrocode}

% \macro{\childdocby}
% The command |\childdocby| ....
%    \begin{macrocode}
\newcommand{\childdocby}[2][]
{
  \childdocdisable
  \childdoctrue
  \childdocmanualtrue
  \if?#1?\else
    \def\jobname{#2}
  \fi
  \def\childdocjob{#2}
  \input{#2}
  \endinput
}
%    \end{macrocode}

% \macro{\childdocforward}
% The command |\childdocforward| redirects
% compilation to the main file or
% (if the optional argument is given) a child file.
% Parameters are set as if the main file
% or a child file starting with |\childdocof| was compiled.
% Then compilation is handed over to the main file:
%    \begin{macrocode}
\newcommand{\childdocforward}[2][]
{
  \begingroup
    \if?#1?
      \def\childdoctmp
      {
        \def\childdocname{#2}
        \def\childdocjob{#2}
        \def\jobname{#2}
        \input{#2}
        \endinput
      }
    \else
      \def\childdoctmp
      {
        \childdocdisable
        \def\childdocname{#2}
        \childdoctrue
        \includeonly{#2}
        \def\childdocjob{#1}
        \def\jobname{#1}
        \input{#1}
        \endinput
      }
    \fi
    \expandafter
  \endgroup
  \childdoctmp
}
%    \end{macrocode}

% \macro{\childdocforwardprefix}
% The command |\childdocforwardprefix| redirects
% compilation to the main or a child file by means of a pattern.
% The prefix |#1| in the current filename is replaced by |#2|
% and the suffix of the current filename is kept
% (it is assumed that the filename does not contain the substring `|~~~|'
% which is used as a delimiter).
% Compilation is handed over to the new file by |\childdocforward|:
%    \begin{macrocode}
\newcommand{\childdocforwardprefix}[3][]
{
  \begingroup
    \def\childdocextract #2##1~~~{\def\childdoctmp{\childdocforward[#1]{#3##1}}}
    \expandafter\childdocextract\childdocname~~~
    \expandafter
  \endgroup
  \childdoctmp
}
%    \end{macrocode}

% \macro{\childdoc}
% The deprecated macro |\childdoc| is a legacy version of |\childdocmain|:
%    \begin{macrocode}
\newcommand{\childdoc}{\childdocmain}
%    \end{macrocode}

% \macro{\childdocredirect}
% The deprecated macro |\childdocredirect| is a legacy version
% of |\childdocforward| and |\childdocforwardprefix|:
%    \begin{macrocode}
\newcommand{\childdocredirect}[2][]
{
  \begingroup
    \if?#1?
      \def\childdoctmp{\childdocforward{#2}}
    \else
      \def\childdoctmp{\childdocforwardprefix{#1}{#2}}
    \fi
    \expandafter
  \endgroup
  \childdoctmp
}
%    \end{macrocode}

%\iffalse
%</package>
%\fi
%
\endinput
|\\
|\childdocof{|\textit{main}|}|\\
\end{tabular}
\end{center}
at the top of every child file \textit{child}
which is included by |\include{|\textit{child}|}|
from within the main file
(or at least for those files to be compiled individually).
The argument \textit{main} must be the filename of the main file.

There are a couple of
considerations in setting up the main and child documents:

%%%%%%%%%%%%%%%%%%%%%%%%%%%%%%%%%%%%%%%%
\paragraph{Restrictions.}

Please note the following restrictions:
\begin{itemize}
\item
|\childdocmain| must be called with one argument \textit{main}
to ensure compatibility with earlier version of the package.
It must either be empty (|\childdocmain{}|)
or precisely match the filename of the main file in which it is specified.
See \secref{sec:detection} for further information.
\item
The filename \textit{main} must be specified without the |.tex| extension.
\item
The filename \textit{main} is case sensitive
(even in case-insensitive file systems)
due to internal string comparison.
\item
The argument \textit{main} should be fully expanded, it cannot be a macro.
\item
Subdirectories and special characters should be avoided in filenames.
\item
The command |\childdocmain{|\textit{main}|}| must be followed by a whitespace.
It should not be followed immediately by another command
or by a comment mark `|%|'.
This is because the \TeX{} parser reads the token immediately following
the argument of |\childdocmain| and puts it
at the beginning of every child section;
however, a white\-space is ignored.
\end{itemize}

%%%%%%%%%%%%%%%%%%%%%%%%%%%%%%%%%%%%%%%%
\paragraph{Content of Main File.}

It is advisable to place all content in the child files included by |\include|.
Any output contained in the main file will appear in all child documents
unless suppressed manually;
it cannot be suppressed automatically by the |\includeonly| directive
and thus should normally be avoided.
A method to include some content in the main file
by means of conditional processing is described in \secref{sec:conditional}.

%%%%%%%%%%%%%%%%%%%%%%%%%%%%%%%%%%%%%%%%
\paragraph{Page Numbering.}

When only a part of the document is compiled,
the appropriate numbering of pages
(as well as other status parameters)
is determined from the |.aux| files.
The latter contain information from previous passes.
However this information needs to propagate through
all intermediate child documents.
Therefore the page numbering in child documents may well
be inconsistent until the complete document is compiled at least once.

A useful (if unconventional) way to always ensure a consistent
page numbering is to restart the numbering in each child document
and denote the pages by `\textit{child}|.|\textit{page}'
where \textit{child} represents the chapter/section number of the child file.
This can be achieved by the command
|\numberwithin{page}{|\textit{child}|}|
of the \textsf{amsmath} package
where \textit{child} can be |chapter| or |section|
depending on the chosen structuring.
Alternatively, one can modify the macro |\thepage| appropriately
and reset the counter |page| at the start of each child file.

%%%%%%%%%%%%%%%%%%%%%%%%%%%%%%%%%%%%%%%%%%%%%%%%%%%%%%%%%%%%%%%%%%%%%%%%%%%%%%%%
\subsection{Conditional Processing}
\label{sec:conditional}

The package provides a mechanism to compile different versions
of a document. To customise the versions further some conditional processing
can come in handy to distinguish which version is being compiled.
The package provides two macros to describe the compilation context:

%%%%%%%%%%%%%%%%%%%%%%%%%%%%%%%%%%%%%%%%
\DescribeMacro{\ifchilddoc}
The conditional |\ifchilddoc| distinguishes between the compilation of
child documents and the main document:
%
\begin{center}
|\ifchilddoc |\textit{child-code}| |[|\||else |\textit{main-code}]| \||fi|
\end{center}

%%%%%%%%%%%%%%%%%%%%%%%%%%%%%%%%%%%%%%%%
\DescribeMacro{\childdocname}
\DescribeMacro{\childdocjob}
The macro |\childdocname| contains the filename (without extension)
of the main or child file being processed.
Note that |\childdocjob| will always contain the name of the main file.

%%%%%%%%%%%%%%%%%%%%%%%%%%%%%%%%%%%%%%%%
\paragraph{Title Page.}

Conditional processing can be used to include a title or banner page
in the main document when proper precautions are taken.
Importantly, the code in the main file should ensure that the page counter
(as well as other status parameters which are stored in the |.aux| files)
takes the same value after the conditional processing.
Otherwise the page numbers may take divergent values
depending on which part is compiled.

For example, a title page could be declared by:
%
\begin{center}
\begin{tabular}{l}
|\ifchilddoc\||else|\\
|\addtocounter{page}{-1}|\\
\textit{code for title page}\\
|\newpage|\\
|\||fi|
\end{tabular}
\end{center}
%
A banner page for the child documents can be generated by:
%
\begin{center}
\begin{tabular}{l}
|\ifchilddoc|\\
|\addtocounter{page}{-1}|\\
\textit{code for banner page}\\
|\newpage|\\
|\||fi|
\end{tabular}
\end{center}
%
Here one could write a message such as:
\begin{center}
|This is the part \childdocname{} of \childdocjob{}.|
\end{center}

%%%%%%%%%%%%%%%%%%%%%%%%%%%%%%%%%%%%%%%%%%%%%%%%%%%%%%%%%%%%%%%%%%%%%%%%%%%%%%%%
\subsection{Flags}
\label{sec:flags}

The package makes it easy to generate different versions
of the main or child documents.
To this end compilation flags can be defined
and assigned different default values.
They will be particularly useful in conjunction
with the forwarding mechanism described in \secref{sec:forward}.

For example, it may be useful to have a flag |\version|
which can be set to |draft| or |final|.
The document source will contain some conditional code
depending on the value of |\version|.
Suppose further, the flag should default to |final| for the main file
and to |draft| for child files
which is a natural assignment for editing the document.
This is achieved by placing the following code
in the preamble of the main document
(below the |\childdocmain| directive):
%
\begin{center}
\begin{tabular}{l}
|\ifchilddoc|\\
|\providecommand{\version}{draft}|\\
|\||else|\\
|\providecommand{\version}{final}|\\
|\||fi|
\end{tabular}
\end{center}
%
The definition by |\providecommand| makes sure
that previous definitions are not overwritten.
Further statements |\providecommand{\version}{...}|
can thus be added before the above code to override it.

For the main file, one might add a line
(between |\childdocmain| and the above block)
%
\begin{center}
|%\ifchilddoc\||else\providecommand{\version}{draft}\||fi|
\end{center}
%
which can be uncommented to produce a draft version.
Likewise one can add a line to the very top of a child file
(above the |\childdocof{|\textit{main}|}| directive)
%
\begin{center}
|%\providecommand{\version}{final}|
\end{center}
%
which can be uncommented to produce the final version of this child document.

%%%%%%%%%%%%%%%%%%%%%%%%%%%%%%%%%%%%%%%%%%%%%%%%%%%%%%%%%%%%%%%%%%%%%%%%%%%%%%%%
\subsection{Forwarding}
\label{sec:forward}

Different versions of the main or child documents
using compilation flags as described in \secref{sec:flags}
can be (permanently) stored in different files
for convenient compilation, viewing and distribution.
To this end, the package defines a command
to pass on compilation to a different file:

%%%%%%%%%%%%%%%%%%%%%%%%%%%%%%%%%%%%%%%%
\DescribeMacro{\childdocforward}
The command |\childdocforward| redirects processing to
another source file:
%
\begin{center}
\begin{tabular}{l}
|% \iffalse
%
% childdoc.dtx Copyright (C) 2017-2018 Niklas Beisert
%
% This work may be distributed and/or modified under the
% conditions of the LaTeX Project Public License, either version 1.3
% of this license or (at your option) any later version.
% The latest version of this license is in
%   http://www.latex-project.org/lppl.txt
% and version 1.3 or later is part of all distributions of LaTeX
% version 2005/12/01 or later.
%
% This work has the LPPL maintenance status `maintained'.
%
% The Current Maintainer of this work is Niklas Beisert.
%
% This work consists of the files childdoc.dtx and childdoc.ins
% and the derived files childdoc.def and cdocsamp.tex with
% cdocsch1.tex, cdocsch2.tex, cdocsdrf.tex, cdocsfn1.tex, cdocsfn2.tex.
%
%<package>\ifdefined\childdocmain\endinput\fi
%<package>\ProvidesFile{childdoc.def}[2018/12/30 v2.0 child document driver]
%<samplemain>\ProvidesFile{cdocsamp.tex}[2018/12/30 v2.0 sample for childdoc]
%<*driver>
%\ProvidesFile{childdoc.drv}[2018/12/30 v2.0 childdoc reference manual file]
\PassOptionsToClass{10pt,a4paper}{article}
\documentclass{ltxdoc}

\usepackage[margin=35mm]{geometry}
\usepackage{hyperref}
\usepackage{hyperxmp}
\usepackage[usenames]{color}

\hypersetup{colorlinks=true}
\hypersetup{pdfstartview=FitH}
\hypersetup{pdfpagemode=UseNone}
\hypersetup{pdfsource={}}
\hypersetup{pdflang={en-UK}}
\hypersetup{pdfcopyright={Copyright 2017-2018 Niklas Beisert.
  This work may be distributed and/or modified under the
  conditions of the LaTeX Project Public License, either version 1.3
  of this license or (at your option) any later version.}}
\hypersetup{pdflicenseurl={http://www.latex-project.org/lppl.txt}}
\hypersetup{pdfcontactaddress={ETH Zurich, ITP, HIT K,
  Wolfgang-Pauli-Strasse 27}}
\hypersetup{pdfcontactpostcode={8093}}
\hypersetup{pdfcontactcity={Zurich}}
\hypersetup{pdfcontactcountry={Switzerland}}
\hypersetup{pdfcontactemail={nbeisert@itp.phys.ethz.ch}}
\hypersetup{pdfcontacturl={http://people.phys.ethz.ch/\xmptilde nbeisert/}}

\newcommand{\secref}[1]{\hyperref[#1]{section \ref*{#1}}}

\parskip1ex
\parindent0pt
\let\olditemize\itemize
\def\itemize{\olditemize\parskip0pt}

\begin{document}

\title{The \textsf{childdoc} Package}
\hypersetup{pdftitle={The childdoc Package}}
\author{Niklas Beisert\\[2ex]
  Institut f\"ur Theoretische Physik\\
  Eidgen\"ossische Technische Hochschule Z\"urich\\
  Wolfgang-Pauli-Strasse 27, 8093 Z\"urich, Switzerland\\[1ex]
  \href{mailto:nbeisert@itp.phys.ethz.ch}
  {\texttt{nbeisert@itp.phys.ethz.ch}}}
\hypersetup{pdfauthor={Niklas Beisert}}
\hypersetup{pdfsubject={Manual for the LaTeX2e Package childdoc}}
\date{30 December 2018, \textsf{v2.0}}
\maketitle

\begin{abstract}\noindent
\textsf{childdoc} is a \LaTeXe{} package
that enables the direct compilation
of document sections included by |\include|
to individual files.
\end{abstract}

\begingroup
\parskip0ex
\tableofcontents
\endgroup

%%%%%%%%%%%%%%%%%%%%%%%%%%%%%%%%%%%%%%%%%%%%%%%%%%%%%%%%%%%%%%%%%%%%%%%%%%%%%%%%
%%%%%%%%%%%%%%%%%%%%%%%%%%%%%%%%%%%%%%%%%%%%%%%%%%%%%%%%%%%%%%%%%%%%%%%%%%%%%%%%
\section{Introduction}

\LaTeX{} provides a mechanism to structure a large document (such as a book)
into a main file and several child files (containing the chapters)
using the |\include| command.
This mechanism is beneficial for documents
which span hundreds of pages in order to
make the source file(s) more manageable.
Moreover, compilation can be restricted to
selected child files by means of the |\includeonly| command.
The latter feature can be used to reduce the compilation time while editing
(this was significantly more useful in the earlier days of \LaTeX{})
or to generate a smaller document which is easier to navigate.
Another application of |\includeonly| is to generate
documents consisting of selected parts of the complete document.

However, there are a few drawbacks of the plain |\include| mechanism:
\begin{itemize}
\item
The child files cannot be compiled on their own,
they can only be compiled via the main file.
A naive editing environment
(such as a text editor with an option
to have the current file processed by \LaTeX)
may require one to switch to the main file before compiling;
attempting to compile the child file produces errors.
\item
The main file must be modified (each time)
to adjust the |\includeonly| command
to the present needs. This easily leaves the main file in a messy state.
\item
The generated document will always carry the filename
of the main document. This is inconvenient if
several child files are to be compiled and
to be kept for distribution.
\end{itemize}

The present package provides a simple interface
to make child files individually compilable by \LaTeX{}.
Compiling a child file then has the same effect as compiling
the main file with an |\includeonly| command
to select the appropriate child.
Moreover the generated document will carry the name of the child
rather than the main file.
This resolves all three above issues.

This feature is meant to make the editing of books,
thesis documents and lecture notes somewhat more convenient.
However, the package can also be used efficiently for
composing a series of documents (such as exercise sheets)
which are typically distributed individually.
It then assists the author in generating the individual documents
(potentially in different versions)
as well as a document containing the collected series.
Another application is in developing style files
or other kinds of included material
where compilation of the style file could redirect
to a sample or test file.

%%%%%%%%%%%%%%%%%%%%%%%%%%%%%%%%%%%%%%%%%%%%%%%%%%%%%%%%%%%%%%%%%%%%%%%%%%%%%%%%
%%%%%%%%%%%%%%%%%%%%%%%%%%%%%%%%%%%%%%%%%%%%%%%%%%%%%%%%%%%%%%%%%%%%%%%%%%%%%%%%
\section{Usage}

First of all, the package \textsf{childdoc} is \emph{not} a standard
\LaTeXe{} |.sty| style file! Therefore it needs to be invoked in
a non-standard way.

%%%%%%%%%%%%%%%%%%%%%%%%%%%%%%%%%%%%%%%%%%%%%%%%%%%%%%%%%%%%%%%%%%%%%%%%%%%%%%%%
\subsection{Included Files}
\label{sec:include}

%%%%%%%%%%%%%%%%%%%%%%%%%%%%%%%%%%%%%%%%
\DescribeMacro{\childdocmain}
To use the package, add the commands
\begin{center}
\begin{tabular}{l}
|\input{childdoc.def}|\\
|\childdocmain{}|\\
\end{tabular}
\end{center}
at the very top of the main \LaTeX{} file,
in particular \emph{before} the |\documentclass| statement!
The argument of |\childdocmain| should be left empty
(but it must be present).

%%%%%%%%%%%%%%%%%%%%%%%%%%%%%%%%%%%%%%%%
\DescribeMacro{\childdocof}
Furthermore, add the commands
\begin{center}
\begin{tabular}{l}
|\input{childdoc.def}|\\
|\childdocof{|\textit{main}|}|\\
\end{tabular}
\end{center}
at the top of every child file \textit{child}
which is included by |\include{|\textit{child}|}|
from within the main file
(or at least for those files to be compiled individually).
The argument \textit{main} must be the filename of the main file.

There are a couple of
considerations in setting up the main and child documents:

%%%%%%%%%%%%%%%%%%%%%%%%%%%%%%%%%%%%%%%%
\paragraph{Restrictions.}

Please note the following restrictions:
\begin{itemize}
\item
|\childdocmain| must be called with one argument \textit{main}
to ensure compatibility with earlier version of the package.
It must either be empty (|\childdocmain{}|)
or precisely match the filename of the main file in which it is specified.
See \secref{sec:detection} for further information.
\item
The filename \textit{main} must be specified without the |.tex| extension.
\item
The filename \textit{main} is case sensitive
(even in case-insensitive file systems)
due to internal string comparison.
\item
The argument \textit{main} should be fully expanded, it cannot be a macro.
\item
Subdirectories and special characters should be avoided in filenames.
\item
The command |\childdocmain{|\textit{main}|}| must be followed by a whitespace.
It should not be followed immediately by another command
or by a comment mark `|%|'.
This is because the \TeX{} parser reads the token immediately following
the argument of |\childdocmain| and puts it
at the beginning of every child section;
however, a white\-space is ignored.
\end{itemize}

%%%%%%%%%%%%%%%%%%%%%%%%%%%%%%%%%%%%%%%%
\paragraph{Content of Main File.}

It is advisable to place all content in the child files included by |\include|.
Any output contained in the main file will appear in all child documents
unless suppressed manually;
it cannot be suppressed automatically by the |\includeonly| directive
and thus should normally be avoided.
A method to include some content in the main file
by means of conditional processing is described in \secref{sec:conditional}.

%%%%%%%%%%%%%%%%%%%%%%%%%%%%%%%%%%%%%%%%
\paragraph{Page Numbering.}

When only a part of the document is compiled,
the appropriate numbering of pages
(as well as other status parameters)
is determined from the |.aux| files.
The latter contain information from previous passes.
However this information needs to propagate through
all intermediate child documents.
Therefore the page numbering in child documents may well
be inconsistent until the complete document is compiled at least once.

A useful (if unconventional) way to always ensure a consistent
page numbering is to restart the numbering in each child document
and denote the pages by `\textit{child}|.|\textit{page}'
where \textit{child} represents the chapter/section number of the child file.
This can be achieved by the command
|\numberwithin{page}{|\textit{child}|}|
of the \textsf{amsmath} package
where \textit{child} can be |chapter| or |section|
depending on the chosen structuring.
Alternatively, one can modify the macro |\thepage| appropriately
and reset the counter |page| at the start of each child file.

%%%%%%%%%%%%%%%%%%%%%%%%%%%%%%%%%%%%%%%%%%%%%%%%%%%%%%%%%%%%%%%%%%%%%%%%%%%%%%%%
\subsection{Conditional Processing}
\label{sec:conditional}

The package provides a mechanism to compile different versions
of a document. To customise the versions further some conditional processing
can come in handy to distinguish which version is being compiled.
The package provides two macros to describe the compilation context:

%%%%%%%%%%%%%%%%%%%%%%%%%%%%%%%%%%%%%%%%
\DescribeMacro{\ifchilddoc}
The conditional |\ifchilddoc| distinguishes between the compilation of
child documents and the main document:
%
\begin{center}
|\ifchilddoc |\textit{child-code}| |[|\||else |\textit{main-code}]| \||fi|
\end{center}

%%%%%%%%%%%%%%%%%%%%%%%%%%%%%%%%%%%%%%%%
\DescribeMacro{\childdocname}
\DescribeMacro{\childdocjob}
The macro |\childdocname| contains the filename (without extension)
of the main or child file being processed.
Note that |\childdocjob| will always contain the name of the main file.

%%%%%%%%%%%%%%%%%%%%%%%%%%%%%%%%%%%%%%%%
\paragraph{Title Page.}

Conditional processing can be used to include a title or banner page
in the main document when proper precautions are taken.
Importantly, the code in the main file should ensure that the page counter
(as well as other status parameters which are stored in the |.aux| files)
takes the same value after the conditional processing.
Otherwise the page numbers may take divergent values
depending on which part is compiled.

For example, a title page could be declared by:
%
\begin{center}
\begin{tabular}{l}
|\ifchilddoc\||else|\\
|\addtocounter{page}{-1}|\\
\textit{code for title page}\\
|\newpage|\\
|\||fi|
\end{tabular}
\end{center}
%
A banner page for the child documents can be generated by:
%
\begin{center}
\begin{tabular}{l}
|\ifchilddoc|\\
|\addtocounter{page}{-1}|\\
\textit{code for banner page}\\
|\newpage|\\
|\||fi|
\end{tabular}
\end{center}
%
Here one could write a message such as:
\begin{center}
|This is the part \childdocname{} of \childdocjob{}.|
\end{center}

%%%%%%%%%%%%%%%%%%%%%%%%%%%%%%%%%%%%%%%%%%%%%%%%%%%%%%%%%%%%%%%%%%%%%%%%%%%%%%%%
\subsection{Flags}
\label{sec:flags}

The package makes it easy to generate different versions
of the main or child documents.
To this end compilation flags can be defined
and assigned different default values.
They will be particularly useful in conjunction
with the forwarding mechanism described in \secref{sec:forward}.

For example, it may be useful to have a flag |\version|
which can be set to |draft| or |final|.
The document source will contain some conditional code
depending on the value of |\version|.
Suppose further, the flag should default to |final| for the main file
and to |draft| for child files
which is a natural assignment for editing the document.
This is achieved by placing the following code
in the preamble of the main document
(below the |\childdocmain| directive):
%
\begin{center}
\begin{tabular}{l}
|\ifchilddoc|\\
|\providecommand{\version}{draft}|\\
|\||else|\\
|\providecommand{\version}{final}|\\
|\||fi|
\end{tabular}
\end{center}
%
The definition by |\providecommand| makes sure
that previous definitions are not overwritten.
Further statements |\providecommand{\version}{...}|
can thus be added before the above code to override it.

For the main file, one might add a line
(between |\childdocmain| and the above block)
%
\begin{center}
|%\ifchilddoc\||else\providecommand{\version}{draft}\||fi|
\end{center}
%
which can be uncommented to produce a draft version.
Likewise one can add a line to the very top of a child file
(above the |\childdocof{|\textit{main}|}| directive)
%
\begin{center}
|%\providecommand{\version}{final}|
\end{center}
%
which can be uncommented to produce the final version of this child document.

%%%%%%%%%%%%%%%%%%%%%%%%%%%%%%%%%%%%%%%%%%%%%%%%%%%%%%%%%%%%%%%%%%%%%%%%%%%%%%%%
\subsection{Forwarding}
\label{sec:forward}

Different versions of the main or child documents
using compilation flags as described in \secref{sec:flags}
can be (permanently) stored in different files
for convenient compilation, viewing and distribution.
To this end, the package defines a command
to pass on compilation to a different file:

%%%%%%%%%%%%%%%%%%%%%%%%%%%%%%%%%%%%%%%%
\DescribeMacro{\childdocforward}
The command |\childdocforward| redirects processing to
another source file:
%
\begin{center}
\begin{tabular}{l}
|\input{childdoc.def}|\\
|\childdocforward[|\textit{main}|]{|\textit{dest}|}|\\
\end{tabular}
\end{center}
%
The argument \textit{dest} is the destination file
(without extension).
It should be the main file or one of the child files.
Note that further \textsf{childdoc} directives
such as |\childdocof| and |\childdocforward|
in the indicated file will be processed in this form.
The optional argument \textit{main}
passes on directly to the main file \textit{main}
while pretending to compile the child \textit{dest}.
This form behaves as if \textit{dest}
issues |\childdocof{|\textit{main}|}| right away,
and no further \textsf{childdoc} directives will be processed.

%%%%%%%%%%%%%%%%%%%%%%%%%%%%%%%%%%%%%%%%
\DescribeMacro{\...prefix}
In the alternative form |\childdocforwardprefix|,
%
\begin{center}
\begin{tabular}{l}
|\input{childdoc.def}|\\
|\childdocforwardprefix[|\textit{main}|]{|\textit{prefix}|}{|\textit{dest}|}|
\end{tabular}
\end{center}
%
the destination file is determined by a pattern
depending on the current file:
To make this work, the current file must be called
`{\textit{prefix}\hspace{0.2em}\textit{suffix}}'
with \textit{prefix} matching precisely the argument.
Processing is then passed on to the file
`{\textit{dest}\hspace{0.2em}\textit{suffix}}'.
Surely, the same effect is achieved by
directly specifying the
argument `{\textit{dest}\hspace{0.2em}\textit{suffix}}'
in the first form.
However, that requires to set up a different file
for each child. With the alternative form of the command
all these files can have exactly the same content
which simplifies setting them up and maintaining them.

For example, the following file |draft.tex|
with a compilation flag |\version| as described in \secref{sec:flags}
compiles the main document as a draft:
%
\begin{center}
\begin{tabular}{l}
|\def\version{draft}|\\
|\input{childdoc.def}|\\
|\childdocforward{|\textit{main}|}|
\end{tabular}
\end{center}
%
Likewise, the following files |final|\textit{nn}|.tex|
compile the final version of the child document
|child|\textit{nn}|.tex|:
%
\begin{center}
\begin{tabular}{l}
|\def\version{final}|\\
|\input{childdoc.def}|\\
|\childdocforwardprefix{final}{child}|
\end{tabular}
\end{center}
%

Note that when several versions of a main file and/or of each child file
are to be generated, it may be convenient to set up a |Makefile| or
shell script to automatise the process.

%%%%%%%%%%%%%%%%%%%%%%%%%%%%%%%%%%%%%%%%%%%%%%%%%%%%%%%%%%%%%%%%%%%%%%%%%%%%%%%%
\subsection{Command Line Processing}
\label{sec:commandline}

The effect of redirection files can also be achieved by invoking
the \LaTeX{} compiler with a more elaborate command line.
Most conveniently this should be done as part
of a shell script or a |Makefile|.

When using \textsf{childdoc} in the main file, the following
command lines effectively perform a redirection
(note that depending on the shell being used,
backslashes may have to be doubled: `|\|' $\to$ `|\\|'):
%
\begin{center}
|... -jobname "|\textit{target}|" |\\|"|[\textit{flags}]%
|\input{childdoc.def}\childdocforward[|\textit{main}|]{|\textit{dest}|}"|
\end{center}
%
Here \textit{target} is the name of the output file,
\textit{main} is the name of the main file
and \textit{dest} is the name of the main or child file to be processed
(all filenames without extensions).
The optional argument \textit{main} can be omitted
if \textit{main} matches \textit{dest}.
Optionally, compilation \textit{flags} can be defined via |\def| commands.
This command line makes the \TeX{} engine believe
it is compiling the file \textit{target}
whose content is specified as the latter parameter.
The provided code then forwards the processing to
\textit{main} or \textit{dest} as described in \secref{sec:forward}.

%%%%%%%%%%%%%%%%%%%%%%%%%%%%%%%%%%%%%%%%%%%%%%%%%%%%%%%%%%%%%%%%%%%%%%%%%%%%%%%%
\subsection{Include by Input}
\label{sec:input}

Including child documents by |\include| has some restrictions by design.
Most notably, the content of a child document always occupies
its own set of pages; pages cannot be shared between child documents.
Usually, this behaviour makes perfect sense
because each child document contain an essential part of the document.
However, in some situations it may be desirable to compose
a document from a collection of parts
without having mandatory page breaks between then.
For this case, the package
provides a mechanism to include parts
by |\input| which can also be processed individually.
However, by construction this mechanism
requires manual handling of the content to be output.

%%%%%%%%%%%%%%%%%%%%%%%%%%%%%%%%%%%%%%%%
\DescribeMacro{\ifchilddocmanual}
The main file should be prepared as usual, see \secref{sec:include}.
However, the document body must make a distinction
between processing of an individual part and of the main document, e.g.:
%
\begin{center}
\begin{tabular}{l}
|\ifchilddocmanual|\\
|\input{\childdocname}|\\
|\||else|\\
\textit{document body with }|\input{|\textit{part}|}|\\
|\||fi|
\end{tabular}
\end{center}
%
The conditional |\ifchilddocmanual| is true whenever
a part to be included by |\input| is being compiled,
and the name of the part is stored in |\childdocname|.

%%%%%%%%%%%%%%%%%%%%%%%%%%%%%%%%%%%%%%%%
\DescribeMacro{\childdocby}
Each part to be included by |\input| should start with:
%
\begin{center}
\begin{tabular}{l}
|\input{childdoc.def}|\\
|\childdocby{|\textit{main}|}|\\
\end{tabular}
\end{center}
%
The directive |\childdocby| is similar to |\childdocof|
described in \secref{sec:include},
but the subsequent selection of content must be done manually.
To that end, both |\ifchilddoc| and |\ifchilddocmanual|
will be true upon processing of a part,
and the name of the part is stored in |\childdocname|.
Note that |\jobname| will be set to the filename of the current part
so that each part receives an individual |.aux| file
that does not interfere with the |.aux| file(s) of the main document.
This behaviour can be altered by the alternative form
|\childdocby[*]{|\textit{main}|}| (with a non-empty optional argument)
which uses the |.aux| file of the main document
by setting |\jobname| to \textit{main}.

%%%%%%%%%%%%%%%%%%%%%%%%%%%%%%%%%%%%%%%%%%%%%%%%%%%%%%%%%%%%%%%%%%%%%%%%%%%%%%%%
\subsection{Driver Development}
\label{sec:driver}

The \textsf{childdoc} mechanism can also be use for the development
of definition files such as \LaTeX{} styles or classes.
This case differs from the above setup with multiple parts
included by |\include| in that no |\includeonly| should be invoked.
This can be achieved by starting the include file
(before |\ProvidesPackage|) with:
%
\begin{center}
\begin{tabular}{l}
|\input{childdoc.def}|\\
|\childdocforward{|\textit{main}|}|\\
\end{tabular}
\end{center}
%
or alternatively with:
%
\begin{center}
\begin{tabular}{l}
|\input{childdoc.def}|\\
|\childdocby{|\textit{main}|}|\\
\end{tabular}
\end{center}
%
Both forms have slightly different effects as described above.
The main file is prepared as usual, see \secref{sec:include}.

%%%%%%%%%%%%%%%%%%%%%%%%%%%%%%%%%%%%%%%%%%%%%%%%%%%%%%%%%%%%%%%%%%%%%%%%%%%%%%%%
\subsection{Legacy Detection}
\label{sec:detection}

The directive |\childdocmain| in the main file can detect
whether the complete document or merely a child is to be compiled
even without using the directive |\childdocof|.
This method is deprecated because it is less robust
and there is no compelling reason to use it;
it is merely provided for backward compatibility
and it may be removed in future versions.

If the detection mechanism is to be used,
it is mandatory to correctly specify
the filename of the main file as the argument of |\childdocmain|:
%
\begin{center}
\begin{tabular}{l}
|\input{childdoc.def}|\\
|\childdocmain{|\textit{main}|}|\\
\end{tabular}
\end{center}
%
If |\jobname| does not match the argument \textit{main} of |\childdocmain|,
it is assumed that |\jobname| points to the child file to be compiled.
When using |\childdocmain| with the main file specified as argument,
it suffices to start a child file
with just |\input{|\textit{main}|}|
without loading of the package and using |\childdocof|.
If instead all processing is done
with the appropriate \textsf{childdoc} directives,
the argument of \textit{main} of |\childdocmain| can be empty.

An alternative version of the command line processing described
in \secref{sec:commandline} using the detection mechanism reads:
%
\begin{center}
|... -jobname "|\textit{target}|" "|[\textit{flags}]%
[|\def\jobname{|\textit{dest}|}|]|\input{|\textit{main}|}"|
\end{center}

%%%%%%%%%%%%%%%%%%%%%%%%%%%%%%%%%%%%%%%%%%%%%%%%%%%%%%%%%%%%%%%%%%%%%%%%%%%%%%%%
\subsection{Manual Code}
\label{sec:manual}

In case one cannot be certain whether the definitions file |childdoc.def|
is installed on the target \TeX{} distribution
and one prefers not to ship it,
it is conceivable to paste a few relevant commands into the sources.

To that end, drop all statements |\input{childdoc.def}|
and perform the replacements as outlined below.
Instead of |\childdocmain{|\textit{main}|}| add the following code
to the top of the main file:
%
\begin{center}
\begin{tabular}{l}
|\||ifdefined\childdocname\endinput\||fi\newif\ifchilddoc|\\
|\edef\childdocname{\scantokens\expandafter{\jobname\noexpand}}|\\
|\def\childdocmain{|\textit{main}|}\||ifx\childdocmain\childdocname\||else|\\
|\childdoctrue\includeonly{\childdocname}\let\jobname\childdocmain\||fi|\\
\end{tabular}
\end{center}
%
Instead of |\childdocof{|\textit{main}|}| just include the main file
at the top of each child file:
%
\begin{center}
|\input{|\textit{main}|}|
\end{center}
%
A simple redirection |\childdocforward{|\textit{dest}|}| is achieved by:
%
\begin{center}
|\def\jobname{|\textit{dest}|}\input{\jobname}|
\end{center}
%
The redirection with prefix
|\childdocforwardprefix[|\textit{prefix}|]{|\textit{dest}|}|
is accomplished by:
%
\begin{center}
\begin{tabular}{l}
|{\edef\jobname{\scantokens\expandafter{\jobname\noexpand}}|\\
|\def\redirectjob |\textit{prefix}|#1~~~{\gdef\jobname{|\textit{dest}|#1}}|\\
|\expandafter\redirectjob\jobname~~~}\input{\jobname}|
\end{tabular}
\end{center}

In an alternative approach,
child documents can be compiled by a specific command line
without additional code or specific definitions:
%
\begin{center}
|... -jobname "|\textit{target}|" "|[\textit{flags}]%
|\includeonly{|\textit{dest}|}\input{|\textit{main}|}"|
\end{center}
%

%%%%%%%%%%%%%%%%%%%%%%%%%%%%%%%%%%%%%%%%%%%%%%%%%%%%%%%%%%%%%%%%%%%%%%%%%%%%%%%%
%%%%%%%%%%%%%%%%%%%%%%%%%%%%%%%%%%%%%%%%%%%%%%%%%%%%%%%%%%%%%%%%%%%%%%%%%%%%%%%%
\section{Information}

%%%%%%%%%%%%%%%%%%%%%%%%%%%%%%%%%%%%%%%%%%%%%%%%%%%%%%%%%%%%%%%%%%%%%%%%%%%%%%%%
\subsection{Copyright}

Copyright \copyright{} 2017--2018 Niklas Beisert

This work may be distributed and/or modified under the
conditions of the \LaTeX{} Project Public License, either version 1.3
of this license or (at your option) any later version.
The latest version of this license is in
  \url{http://www.latex-project.org/lppl.txt}
and version 1.3 or later is part of all distributions of \LaTeX{}
version 2005/12/01 or later.

This work has the LPPL maintenance status `maintained'.

The Current Maintainer of this work is Niklas Beisert.

This work consists of the files |README.txt|, |childdoc.ins| and |childdoc.dtx|
as well as the derived files |childdoc.def|, |cdocsamp.tex|
with |cdocsch1.tex|, |cdocsch2.tex|, |cdocspt3.tex|, |cdocspt4.tex|,
|cdocsdrf.tex|, |cdocsfn1.tex|, |cdocsfn2.tex|
as well as |childdoc.pdf|.

%%%%%%%%%%%%%%%%%%%%%%%%%%%%%%%%%%%%%%%%%%%%%%%%%%%%%%%%%%%%%%%%%%%%%%%%%%%%%%%%
\subsection{Files and Installation}

The package consists of the files:
%
\begin{center}
\begin{tabular}{ll}
    |README.txt|   & readme file \\
    |childdoc.ins| & installation file \\
    |childdoc.dtx| & source file \\
    |childdoc.def| & definition file \\
    |cdocsamp.tex| & sample main file \\
    |cdocsch1.tex| & sample include file \\
    |cdocsch2.tex| & sample include file \\
    |cdocspt3.tex| & sample part file \\
    |cdocspt4.tex| & sample part file \\
    |cdocsdrf.tex| & sample redirection file \\
    |cdocsfn1.tex| & sample redirection file \\
    |cdocsfn2.tex| & sample redirection file \\
    |childdoc.pdf| & manual
\end{tabular}
\end{center}
%
The distribution consists of the files
|README.txt|, |childdoc.ins| and |childdoc.dtx|.
%
\begin{itemize}
\item
Run (pdf)\LaTeX{} on |childdoc.dtx|
to compile the manual |childdoc.pdf| (this file).
\item
Run \LaTeX{} on |childdoc.ins| to create the definitions file |childdoc.def|
and the sample |cdocsamp.tex| with include files
|cdocsch1.tex|, |cdocsch2.tex|, |cdocspt3.tex|, |cdocspt4.tex|,
|cdocsdrf.tex|, |cdocsfn1.tex|, |cdocsfn2.tex|.
Then copy the file |childdoc.def| to an appropriate directory of your \LaTeX{}
distribution, e.g.\ \textit{texmf-root}|/tex/latex/childdoc|.
\end{itemize}

%%%%%%%%%%%%%%%%%%%%%%%%%%%%%%%%%%%%%%%%%%%%%%%%%%%%%%%%%%%%%%%%%%%%%%%%%%%%%%%%
\subsection{Related CTAN Packages}

There are several other packages which offer a similar functionality:
%
\begin{itemize}
\item
The packages
\href{http://ctan.org/pkg/docmute}{\textsf{docmute}},
\href{http://ctan.org/pkg/includex}{\textsf{includex}} and
\href{http://ctan.org/pkg/standalone}{\textsf{standalone}}
provide commands to include only the document body of
a child file thus allowing both files to be compiled individually.
\item
The packages \href{http://ctan.org/pkg/subdocs}{\textsf{subdocs}}
and \href{http://ctan.org/pkg/subfiles}{\textsf{subfiles}}
provide structures in which the main and child documents can be
encapsulated and allowing them to be compiled individually.
The inclusion mechanism is different from the conventional |\include|.
\item
The package \href{http://ctan.org/pkg/combine}{\textsf{combine}}
is an elaborate solution to combine several documents into one.
\end{itemize}
%
See also the CTAN topic \href{http://ctan.org/topic/subdocs}{\textsf{subdocs}}
for further related packages.
The present package differs from the above solutions in that
a document structure constructed with the conventional |\include| mechanism
just needs two extra commands at the top of every file
such that all constituent files can be compiled individually.

%%%%%%%%%%%%%%%%%%%%%%%%%%%%%%%%%%%%%%%%%%%%%%%%%%%%%%%%%%%%%%%%%%%%%%%%%%%%%%%%
%\subsection{Feature Suggestions}
%
%The following is a list of features which may be useful for future
%versions of this package:
%%
%\begin{itemize}
%\item
%\ldots
%\end{itemize}

%%%%%%%%%%%%%%%%%%%%%%%%%%%%%%%%%%%%%%%%%%%%%%%%%%%%%%%%%%%%%%%%%%%%%%%%%%%%%%%%
\subsection{Revision History}

%%%%%%%%%%%%%%%%%%%%%%%%%%%%%%%%%%%%%%%%
\paragraph{v2.0:} 2018/12/30

\begin{itemize}
\item
immediate forward processing
\item
added |\childdocby| mechanism
\item
manual restructured
\end{itemize}

%%%%%%%%%%%%%%%%%%%%%%%%%%%%%%%%%%%%%%%%
\paragraph{v1.6:} 2018/01/17

\begin{itemize}
\item
application for development of include files
\item
corrections to manual
\end{itemize}

%%%%%%%%%%%%%%%%%%%%%%%%%%%%%%%%%%%%%%%%
\paragraph{v1.5:} 2017/05/21

\begin{itemize}
\item
more complete structuring introduced
\item
|\childdocof| introduced
\item
|\childdoc| renamed to |\childdocmain|
\item
|\childredirect| renamed to |\childdocforward| and |\childdocforwardprefix|
and functionality expanded
\end{itemize}

%%%%%%%%%%%%%%%%%%%%%%%%%%%%%%%%%%%%%%%%
\paragraph{v1.0:} 2017/04/27

\begin{itemize}
\item
manual and install package
\item
first version published on CTAN
\end{itemize}

%%%%%%%%%%%%%%%%%%%%%%%%%%%%%%%%%%%%%%%%
\paragraph{v0.6:} 2017/04/26

\begin{itemize}
\item
redirection mechanism added
\end{itemize}

%%%%%%%%%%%%%%%%%%%%%%%%%%%%%%%%%%%%%%%%
\paragraph{v0.5:} 2017/04/26

\begin{itemize}
\item
functionality in definition file
\end{itemize}


%%%%%%%%%%%%%%%%%%%%%%%%%%%%%%%%%%%%%%%%%%%%%%%%%%%%%%%%%%%%%%%%%%%%%%%%%%%%%%%%
%%%%%%%%%%%%%%%%%%%%%%%%%%%%%%%%%%%%%%%%%%%%%%%%%%%%%%%%%%%%%%%%%%%%%%%%%%%%%%%%
%%%%%%%%%%%%%%%%%%%%%%%%%%%%%%%%%%%%%%%%%%%%%%%%%%%%%%%%%%%%%%%%%%%%%%%%%%%%%%%%
\appendix

\settowidth\MacroIndent{\rmfamily\scriptsize 000\ }

 \DocInput{childdoc.dtx}

\end{document}
%</driver>
% \fi
%
% %%%%%%%%%%%%%%%%%%%%%%%%%%%%%%%%%%%%%%%%%%%%%%%%%%%%%%%%%%%%%%%%%%%%%%%%%%%%%%
% %%%%%%%%%%%%%%%%%%%%%%%%%%%%%%%%%%%%%%%%%%%%%%%%%%%%%%%%%%%%%%%%%%%%%%%%%%%%%%
% \section{Sample}
%\iffalse
%<*samplemain>
%\fi
%
% The following presents a sample document
% with two chapters, two parts, a title page,
% a compile flag as well as three forwarding files to set the flag.
% It consists of eight |.tex| files:
% \begin{center}
% \begin{tabular}{ll}
% |cdocsamp.tex|&main file\\
% |cdocsch1.tex|&include file for chapter 1\\
% |cdocsch2.tex|&include file for chapter 2\\
% |cdocspt3.tex|&include file for part 3\\
% |cdocspt4.tex|&include file for part 4\\
% |cdocsdrf.tex|&forwarding file for main file in draft mode\\
% |cdocsfi1.tex|&forwarding file for final version of chapter 1\\
% |cdocsfi2.tex|&forwarding file for final version of chapter 2\\
% \end{tabular}
% \end{center}
% Each of the eight files can be compiled directly by the \LaTeX{} compiler.
%
% %%%%%%%%%%%%%%%%%%%%%%%%%%%%%%%%%%%%%%
% \paragraph{Main File.}
%
% The main file is called |cdocsamp.tex|.
%
% Load the \textsf{childdoc} definitions and
% declare the filename for the main document:
%    \begin{macrocode}
\input{childdoc.def}
\childdocmain{}
%    \end{macrocode}

% Optional override for |\version| flag:
%    \begin{macrocode}
%%\ifchilddoc\else\providecommand{\version}{draft}\fi
%    \end{macrocode}

% Define the default values for the |\version| flag
% (|final| for the main file and |draft| for childs):
%    \begin{macrocode}
\ifchilddoc
\providecommand{\version}{draft}
\else
\providecommand{\version}{final}
\fi
%    \end{macrocode}

% Load the standard document class:
%    \begin{macrocode}
\documentclass[12pt]{article}
%    \end{macrocode}

% Start the document body:
%    \begin{macrocode}
\begin{document}
%    \end{macrocode}

% Declare a title page.
% Print title, part of document being processed and version flag:
%    \begin{macrocode}
\addtocounter{page}{-1}
\begin{center}
{\LARGE\bfseries{}childdoc example\par}
\vspace{1cm}
\ifchilddoc
\ifchilddocmanual part\else chapter\fi:
`\childdocname' of `\childdocjob'\par
\else
main document: `\childdocjob'\par
\fi
version: \version\par
\end{center}
\newpage
%    \end{macrocode}

% Manually include selected file,
% otherwise process as usual:
%    \begin{macrocode}
\ifchilddocmanual
\section*{part `\childdocname'}
\input{\childdocname}
\else
%    \end{macrocode}

% Include the two chapters:
%    \begin{macrocode}
\include{cdocsch1}
\include{cdocsch2}
%    \end{macrocode}

% Include the two parts unless only chapters should be displayed:
%    \begin{macrocode}
\ifchilddoc\else
\section{part three}
\input{cdocspt3}
\section{part four}
\input{cdocspt4}
\fi
%    \end{macrocode}

% Process as usual until here:
%    \begin{macrocode}
\fi
%    \end{macrocode}

% End of document body:
%    \begin{macrocode}
\end{document}
%    \end{macrocode}
%\iffalse
%</samplemain>
%\fi
%
% %%%%%%%%%%%%%%%%%%%%%%%%%%%%%%%%%%%%%%
% \paragraph{Chapter Include Files.}
%
% The include files are called |cdocsch1.tex| and |cdocsch2.tex|.
%
%\iffalse
%<*samplechap1|samplechap2>
%\fi

% Optional override for |\version| flag:
%    \begin{macrocode}
%%\providecommand{\version}{final}
%    \end{macrocode}

% Include the main document:
%    \begin{macrocode}
\input{childdoc.def}
\childdocof{cdocsamp}
%    \end{macrocode}

%\iffalse
%</samplechap1|samplechap2>
%\fi
%
%\iffalse
%<*samplechap1>
%\fi
% Some text for chapter 1:
%    \begin{macrocode}
\section{one}
some text in chapter one
%    \end{macrocode}

%\iffalse
%</samplechap1>
%\fi
% Some text for chapter 2:
%\iffalse
%<*samplechap2>
%\fi
%    \begin{macrocode}
\section{two}
more text in chapter two
%    \end{macrocode}

%\iffalse
%</samplechap2>
%\fi
%
% %%%%%%%%%%%%%%%%%%%%%%%%%%%%%%%%%%%%%%
% \paragraph{Part Include Files.}
%
% The include files are called |cdocspt3.tex| and |cdocspt4.tex|.
%
%\iffalse
%<*samplepart3|samplepart4>
%\fi

% Optional override for |\version| flag:
%    \begin{macrocode}
%%\providecommand{\version}{final}
%    \end{macrocode}

% Include the main document:
%    \begin{macrocode}
\input{childdoc.def}
\childdocby{cdocsamp}
%    \end{macrocode}

%\iffalse
%</samplepart3|samplepart4>
%\fi
%
%\iffalse
%<*samplepart3>
%\fi
% Some text for part 3:
%    \begin{macrocode}
some text in part three
%    \end{macrocode}

%\iffalse
%</samplepart3>
%\fi
% Some text for part 4:
%\iffalse
%<*samplepart4>
%\fi
%    \begin{macrocode}
more text in part four
%    \end{macrocode}

%\iffalse
%</samplepart4>
%\fi
%
% %%%%%%%%%%%%%%%%%%%%%%%%%%%%%%%%%%%%%%
% \paragraph{Forwarding for a Complete Draft.}
%
% The following forwarding file |cdocsdrf.tex|
% compiles the main document in draft mode:
%\iffalse
%<*sampledraft>
%\fi
%    \begin{macrocode}
\def\version{draft}
\input{childdoc.def}
\childdocforward{cdocsamp}
%    \end{macrocode}

%\iffalse
%</sampledraft>
%\fi
%
% %%%%%%%%%%%%%%%%%%%%%%%%%%%%%%%%%%%%%%
% \paragraph{Forwarding for Final Version of the Chapters.}
%
% The following forwarding files |cdocsfn1.tex| and |cdocsfn2.tex|
% (with identical content)
% compile the final versions of the child documents
% |cdocsch1.tex| and |cdocsch2.tex|, respectively:
%\iffalse
%<*samplefinal>
%\fi
%    \begin{macrocode}
\def\version{final}
\input{childdoc.def}
\childdocforwardprefix[cdocsamp]{cdocsfn}{cdocsch}
%    \end{macrocode}

%\iffalse
%</samplefinal>
%\fi
%
% %%%%%%%%%%%%%%%%%%%%%%%%%%%%%%%%%%%%%%
% \paragraph{Command Line Processing.}
%
% The following three command lines generate the output files
% |cdocscld|, |cdocscl1| and |cdocscl2|
% which should be identical to
% |cdocsdrf|, |cdocsch1| and |cdocsfn2|, respectively:
% \begin{center}
% \begin{tabular}{l}
% |latex -jobname cdocscld \|\\
% |  "\def\version{draft}\input{childdoc.def}\childdocforward{cdocsamp}"|\\
% |latex -jobname cdocscl1 \|\\
% |  "\input{childdoc.def}\childdocforward[cdocsamp]{cdocsch1}"|\\
% |latex -jobname cdocscl2 \|\\
% |  "\def\version{final}\input{childdoc.def}\childdocforward{cdocsch2}"|
% \end{tabular}
% \end{center}
% Note that the trailing backslash on each first line
% merely continues the input to the second line
% (for convenient cut ant paste).
% Furthermore, the command |latex| can be replaced by any
% of its alternative versions such as |pdflatex|.
%
% %%%%%%%%%%%%%%%%%%%%%%%%%%%%%%%%%%%%%%%%%%%%%%%%%%%%%%%%%%%%%%%%%%%%%%%%%%%%%%
% %%%%%%%%%%%%%%%%%%%%%%%%%%%%%%%%%%%%%%%%%%%%%%%%%%%%%%%%%%%%%%%%%%%%%%%%%%%%%%
% \section{Implementation}
%\iffalse
%<*package>
%\fi
%
% This section describes the definitions file |childdoc.def|.

% The definitions cannot be loaded using |\usepackage| or |\RequirePackage|
% which has a mechanism to prevent loading a style file more than once.
% When loading the definitions by means of |\input|
% multiple instances have to be prevented manually:
%\iffalse
%This code needs to be before the `\ProvidesFile' directive
%which is defined at the beginning of this file.
%Therefore it is also placed there and commented out here.
%</package>
%<*discard>
%\fi
%    \begin{macrocode}
\ifdefined\childdocmain\endinput\fi
%    \end{macrocode}
%\iffalse
%</discard>
%<*package>
%\fi
%
% \macro{\ifchilddoc}
% \macro{\ifchilddocmanual}
% The conditional |\ifchilddoc| tells whether a
% child (true) or main (false) document is being compiled.
% The conditional |\ifchilddocmanual| tells whether
% the |\includeonly| mechanism is used (false) or
% the selection of child files must be performed manually (true).
% The definitions initialise to false:
%    \begin{macrocode}
\newif\ifchilddoc
\newif\ifchilddocmanual
%    \end{macrocode}

% \macro{\childdocname}
% \macro{\childdocjob}
% The macro |\childdocname| stores the name of the main document
% to be compiled. The macro |\childdocjob| stores the name of
% the document on which the \LaTeX{} compiler was originally invoked.
% The content of |\jobname| cannot be compared
% to filenames specified in the source due to different catcodes.
% The following code rescans |\jobname|, stores the result
% in |\childdocname| and saves a copy in |\childdocjob|:
%    \begin{macrocode}
\edef\childdocname{\scantokens\expandafter{\jobname\noexpand}}
\let\childdocjob\childdocname
%    \end{macrocode}

% \macro{\childdocdisable}
% The macro |\childdocdisable| prevents the main file
% from being processed more than once.
% At this stage, the main document command |\childdocmain|
% is assumed to be called once again where it should do nothing.
% Any subsequent call to it should prevent
% a secondary processing of the main document
% It overwrites the forwarding commands
% |\childdocof| and |\childdocforward|
% with empty macros to prevent further inclusions of the main document:
%    \begin{macrocode}
\newcommand{\childdocdisable}
{
  \renewcommand{\childdocmain}[1]{\renewcommand{\childdocmain}[1]{\endinput}}
  \renewcommand{\childdocof}[1]{}
  \renewcommand{\childdocby}[2][]{}
  \renewcommand{\childdocforward}[2][]{}
  \renewcommand{\childdocdisable}{}
}
%    \end{macrocode}

% \macro{\childdocmain}
% The macro |\childdocmain| is to be called at the top of the main file
% with nothing or the main filename (without extension) as argument.
% First, it breaks loops.
% If the argument is not empty and does not match |\childdocname|
% (which is set by the first inclusion of |childdoc.def|),
% |\ifchilddoc| is set to true, |\includeonly| is applied to the child file
% and |\jobname| is set to the main file
% (for proper handling of |.aux| files):
%    \begin{macrocode}
\newcommand{\childdocmain}[1]
{
  \childdocdisable\childdocmain{}
  \if?#1?\else
    \begingroup
      \def\childdoctmp{#1}
      \ifx\childdoctmp\childdocname
        \def\childdoctmp{}
      \else
        \def\childdoctmp
        {
          \childdoctrue
          \includeonly{\childdocname}
          \def\childdocjob{#1}
          \def\jobname{#1}
        }
      \fi
      \expandafter
    \endgroup
    \childdoctmp
  \fi
}
%    \end{macrocode}

% \macro{\childdocof}
% The command |\childdocof| redirects
% compilation to the main file |#1|.
%    \begin{macrocode}
\newcommand{\childdocof}[1]
{
  \childdocdisable
  \childdoctrue
  \includeonly{\childdocname}
  \def\jobname{#1}
  \def\childdocjob{#1}
  \input{#1}
}
%    \end{macrocode}

% \macro{\childdocby}
% The command |\childdocby| ....
%    \begin{macrocode}
\newcommand{\childdocby}[2][]
{
  \childdocdisable
  \childdoctrue
  \childdocmanualtrue
  \if?#1?\else
    \def\jobname{#2}
  \fi
  \def\childdocjob{#2}
  \input{#2}
  \endinput
}
%    \end{macrocode}

% \macro{\childdocforward}
% The command |\childdocforward| redirects
% compilation to the main file or
% (if the optional argument is given) a child file.
% Parameters are set as if the main file
% or a child file starting with |\childdocof| was compiled.
% Then compilation is handed over to the main file:
%    \begin{macrocode}
\newcommand{\childdocforward}[2][]
{
  \begingroup
    \if?#1?
      \def\childdoctmp
      {
        \def\childdocname{#2}
        \def\childdocjob{#2}
        \def\jobname{#2}
        \input{#2}
        \endinput
      }
    \else
      \def\childdoctmp
      {
        \childdocdisable
        \def\childdocname{#2}
        \childdoctrue
        \includeonly{#2}
        \def\childdocjob{#1}
        \def\jobname{#1}
        \input{#1}
        \endinput
      }
    \fi
    \expandafter
  \endgroup
  \childdoctmp
}
%    \end{macrocode}

% \macro{\childdocforwardprefix}
% The command |\childdocforwardprefix| redirects
% compilation to the main or a child file by means of a pattern.
% The prefix |#1| in the current filename is replaced by |#2|
% and the suffix of the current filename is kept
% (it is assumed that the filename does not contain the substring `|~~~|'
% which is used as a delimiter).
% Compilation is handed over to the new file by |\childdocforward|:
%    \begin{macrocode}
\newcommand{\childdocforwardprefix}[3][]
{
  \begingroup
    \def\childdocextract #2##1~~~{\def\childdoctmp{\childdocforward[#1]{#3##1}}}
    \expandafter\childdocextract\childdocname~~~
    \expandafter
  \endgroup
  \childdoctmp
}
%    \end{macrocode}

% \macro{\childdoc}
% The deprecated macro |\childdoc| is a legacy version of |\childdocmain|:
%    \begin{macrocode}
\newcommand{\childdoc}{\childdocmain}
%    \end{macrocode}

% \macro{\childdocredirect}
% The deprecated macro |\childdocredirect| is a legacy version
% of |\childdocforward| and |\childdocforwardprefix|:
%    \begin{macrocode}
\newcommand{\childdocredirect}[2][]
{
  \begingroup
    \if?#1?
      \def\childdoctmp{\childdocforward{#2}}
    \else
      \def\childdoctmp{\childdocforwardprefix{#1}{#2}}
    \fi
    \expandafter
  \endgroup
  \childdoctmp
}
%    \end{macrocode}

%\iffalse
%</package>
%\fi
%
\endinput
|\\
|\childdocforward[|\textit{main}|]{|\textit{dest}|}|\\
\end{tabular}
\end{center}
%
The argument \textit{dest} is the destination file
(without extension).
It should be the main file or one of the child files.
Note that further \textsf{childdoc} directives
such as |\childdocof| and |\childdocforward|
in the indicated file will be processed in this form.
The optional argument \textit{main}
passes on directly to the main file \textit{main}
while pretending to compile the child \textit{dest}.
This form behaves as if \textit{dest}
issues |\childdocof{|\textit{main}|}| right away,
and no further \textsf{childdoc} directives will be processed.

%%%%%%%%%%%%%%%%%%%%%%%%%%%%%%%%%%%%%%%%
\DescribeMacro{\...prefix}
In the alternative form |\childdocforwardprefix|,
%
\begin{center}
\begin{tabular}{l}
|% \iffalse
%
% childdoc.dtx Copyright (C) 2017-2018 Niklas Beisert
%
% This work may be distributed and/or modified under the
% conditions of the LaTeX Project Public License, either version 1.3
% of this license or (at your option) any later version.
% The latest version of this license is in
%   http://www.latex-project.org/lppl.txt
% and version 1.3 or later is part of all distributions of LaTeX
% version 2005/12/01 or later.
%
% This work has the LPPL maintenance status `maintained'.
%
% The Current Maintainer of this work is Niklas Beisert.
%
% This work consists of the files childdoc.dtx and childdoc.ins
% and the derived files childdoc.def and cdocsamp.tex with
% cdocsch1.tex, cdocsch2.tex, cdocsdrf.tex, cdocsfn1.tex, cdocsfn2.tex.
%
%<package>\ifdefined\childdocmain\endinput\fi
%<package>\ProvidesFile{childdoc.def}[2018/12/30 v2.0 child document driver]
%<samplemain>\ProvidesFile{cdocsamp.tex}[2018/12/30 v2.0 sample for childdoc]
%<*driver>
%\ProvidesFile{childdoc.drv}[2018/12/30 v2.0 childdoc reference manual file]
\PassOptionsToClass{10pt,a4paper}{article}
\documentclass{ltxdoc}

\usepackage[margin=35mm]{geometry}
\usepackage{hyperref}
\usepackage{hyperxmp}
\usepackage[usenames]{color}

\hypersetup{colorlinks=true}
\hypersetup{pdfstartview=FitH}
\hypersetup{pdfpagemode=UseNone}
\hypersetup{pdfsource={}}
\hypersetup{pdflang={en-UK}}
\hypersetup{pdfcopyright={Copyright 2017-2018 Niklas Beisert.
  This work may be distributed and/or modified under the
  conditions of the LaTeX Project Public License, either version 1.3
  of this license or (at your option) any later version.}}
\hypersetup{pdflicenseurl={http://www.latex-project.org/lppl.txt}}
\hypersetup{pdfcontactaddress={ETH Zurich, ITP, HIT K,
  Wolfgang-Pauli-Strasse 27}}
\hypersetup{pdfcontactpostcode={8093}}
\hypersetup{pdfcontactcity={Zurich}}
\hypersetup{pdfcontactcountry={Switzerland}}
\hypersetup{pdfcontactemail={nbeisert@itp.phys.ethz.ch}}
\hypersetup{pdfcontacturl={http://people.phys.ethz.ch/\xmptilde nbeisert/}}

\newcommand{\secref}[1]{\hyperref[#1]{section \ref*{#1}}}

\parskip1ex
\parindent0pt
\let\olditemize\itemize
\def\itemize{\olditemize\parskip0pt}

\begin{document}

\title{The \textsf{childdoc} Package}
\hypersetup{pdftitle={The childdoc Package}}
\author{Niklas Beisert\\[2ex]
  Institut f\"ur Theoretische Physik\\
  Eidgen\"ossische Technische Hochschule Z\"urich\\
  Wolfgang-Pauli-Strasse 27, 8093 Z\"urich, Switzerland\\[1ex]
  \href{mailto:nbeisert@itp.phys.ethz.ch}
  {\texttt{nbeisert@itp.phys.ethz.ch}}}
\hypersetup{pdfauthor={Niklas Beisert}}
\hypersetup{pdfsubject={Manual for the LaTeX2e Package childdoc}}
\date{30 December 2018, \textsf{v2.0}}
\maketitle

\begin{abstract}\noindent
\textsf{childdoc} is a \LaTeXe{} package
that enables the direct compilation
of document sections included by |\include|
to individual files.
\end{abstract}

\begingroup
\parskip0ex
\tableofcontents
\endgroup

%%%%%%%%%%%%%%%%%%%%%%%%%%%%%%%%%%%%%%%%%%%%%%%%%%%%%%%%%%%%%%%%%%%%%%%%%%%%%%%%
%%%%%%%%%%%%%%%%%%%%%%%%%%%%%%%%%%%%%%%%%%%%%%%%%%%%%%%%%%%%%%%%%%%%%%%%%%%%%%%%
\section{Introduction}

\LaTeX{} provides a mechanism to structure a large document (such as a book)
into a main file and several child files (containing the chapters)
using the |\include| command.
This mechanism is beneficial for documents
which span hundreds of pages in order to
make the source file(s) more manageable.
Moreover, compilation can be restricted to
selected child files by means of the |\includeonly| command.
The latter feature can be used to reduce the compilation time while editing
(this was significantly more useful in the earlier days of \LaTeX{})
or to generate a smaller document which is easier to navigate.
Another application of |\includeonly| is to generate
documents consisting of selected parts of the complete document.

However, there are a few drawbacks of the plain |\include| mechanism:
\begin{itemize}
\item
The child files cannot be compiled on their own,
they can only be compiled via the main file.
A naive editing environment
(such as a text editor with an option
to have the current file processed by \LaTeX)
may require one to switch to the main file before compiling;
attempting to compile the child file produces errors.
\item
The main file must be modified (each time)
to adjust the |\includeonly| command
to the present needs. This easily leaves the main file in a messy state.
\item
The generated document will always carry the filename
of the main document. This is inconvenient if
several child files are to be compiled and
to be kept for distribution.
\end{itemize}

The present package provides a simple interface
to make child files individually compilable by \LaTeX{}.
Compiling a child file then has the same effect as compiling
the main file with an |\includeonly| command
to select the appropriate child.
Moreover the generated document will carry the name of the child
rather than the main file.
This resolves all three above issues.

This feature is meant to make the editing of books,
thesis documents and lecture notes somewhat more convenient.
However, the package can also be used efficiently for
composing a series of documents (such as exercise sheets)
which are typically distributed individually.
It then assists the author in generating the individual documents
(potentially in different versions)
as well as a document containing the collected series.
Another application is in developing style files
or other kinds of included material
where compilation of the style file could redirect
to a sample or test file.

%%%%%%%%%%%%%%%%%%%%%%%%%%%%%%%%%%%%%%%%%%%%%%%%%%%%%%%%%%%%%%%%%%%%%%%%%%%%%%%%
%%%%%%%%%%%%%%%%%%%%%%%%%%%%%%%%%%%%%%%%%%%%%%%%%%%%%%%%%%%%%%%%%%%%%%%%%%%%%%%%
\section{Usage}

First of all, the package \textsf{childdoc} is \emph{not} a standard
\LaTeXe{} |.sty| style file! Therefore it needs to be invoked in
a non-standard way.

%%%%%%%%%%%%%%%%%%%%%%%%%%%%%%%%%%%%%%%%%%%%%%%%%%%%%%%%%%%%%%%%%%%%%%%%%%%%%%%%
\subsection{Included Files}
\label{sec:include}

%%%%%%%%%%%%%%%%%%%%%%%%%%%%%%%%%%%%%%%%
\DescribeMacro{\childdocmain}
To use the package, add the commands
\begin{center}
\begin{tabular}{l}
|\input{childdoc.def}|\\
|\childdocmain{}|\\
\end{tabular}
\end{center}
at the very top of the main \LaTeX{} file,
in particular \emph{before} the |\documentclass| statement!
The argument of |\childdocmain| should be left empty
(but it must be present).

%%%%%%%%%%%%%%%%%%%%%%%%%%%%%%%%%%%%%%%%
\DescribeMacro{\childdocof}
Furthermore, add the commands
\begin{center}
\begin{tabular}{l}
|\input{childdoc.def}|\\
|\childdocof{|\textit{main}|}|\\
\end{tabular}
\end{center}
at the top of every child file \textit{child}
which is included by |\include{|\textit{child}|}|
from within the main file
(or at least for those files to be compiled individually).
The argument \textit{main} must be the filename of the main file.

There are a couple of
considerations in setting up the main and child documents:

%%%%%%%%%%%%%%%%%%%%%%%%%%%%%%%%%%%%%%%%
\paragraph{Restrictions.}

Please note the following restrictions:
\begin{itemize}
\item
|\childdocmain| must be called with one argument \textit{main}
to ensure compatibility with earlier version of the package.
It must either be empty (|\childdocmain{}|)
or precisely match the filename of the main file in which it is specified.
See \secref{sec:detection} for further information.
\item
The filename \textit{main} must be specified without the |.tex| extension.
\item
The filename \textit{main} is case sensitive
(even in case-insensitive file systems)
due to internal string comparison.
\item
The argument \textit{main} should be fully expanded, it cannot be a macro.
\item
Subdirectories and special characters should be avoided in filenames.
\item
The command |\childdocmain{|\textit{main}|}| must be followed by a whitespace.
It should not be followed immediately by another command
or by a comment mark `|%|'.
This is because the \TeX{} parser reads the token immediately following
the argument of |\childdocmain| and puts it
at the beginning of every child section;
however, a white\-space is ignored.
\end{itemize}

%%%%%%%%%%%%%%%%%%%%%%%%%%%%%%%%%%%%%%%%
\paragraph{Content of Main File.}

It is advisable to place all content in the child files included by |\include|.
Any output contained in the main file will appear in all child documents
unless suppressed manually;
it cannot be suppressed automatically by the |\includeonly| directive
and thus should normally be avoided.
A method to include some content in the main file
by means of conditional processing is described in \secref{sec:conditional}.

%%%%%%%%%%%%%%%%%%%%%%%%%%%%%%%%%%%%%%%%
\paragraph{Page Numbering.}

When only a part of the document is compiled,
the appropriate numbering of pages
(as well as other status parameters)
is determined from the |.aux| files.
The latter contain information from previous passes.
However this information needs to propagate through
all intermediate child documents.
Therefore the page numbering in child documents may well
be inconsistent until the complete document is compiled at least once.

A useful (if unconventional) way to always ensure a consistent
page numbering is to restart the numbering in each child document
and denote the pages by `\textit{child}|.|\textit{page}'
where \textit{child} represents the chapter/section number of the child file.
This can be achieved by the command
|\numberwithin{page}{|\textit{child}|}|
of the \textsf{amsmath} package
where \textit{child} can be |chapter| or |section|
depending on the chosen structuring.
Alternatively, one can modify the macro |\thepage| appropriately
and reset the counter |page| at the start of each child file.

%%%%%%%%%%%%%%%%%%%%%%%%%%%%%%%%%%%%%%%%%%%%%%%%%%%%%%%%%%%%%%%%%%%%%%%%%%%%%%%%
\subsection{Conditional Processing}
\label{sec:conditional}

The package provides a mechanism to compile different versions
of a document. To customise the versions further some conditional processing
can come in handy to distinguish which version is being compiled.
The package provides two macros to describe the compilation context:

%%%%%%%%%%%%%%%%%%%%%%%%%%%%%%%%%%%%%%%%
\DescribeMacro{\ifchilddoc}
The conditional |\ifchilddoc| distinguishes between the compilation of
child documents and the main document:
%
\begin{center}
|\ifchilddoc |\textit{child-code}| |[|\||else |\textit{main-code}]| \||fi|
\end{center}

%%%%%%%%%%%%%%%%%%%%%%%%%%%%%%%%%%%%%%%%
\DescribeMacro{\childdocname}
\DescribeMacro{\childdocjob}
The macro |\childdocname| contains the filename (without extension)
of the main or child file being processed.
Note that |\childdocjob| will always contain the name of the main file.

%%%%%%%%%%%%%%%%%%%%%%%%%%%%%%%%%%%%%%%%
\paragraph{Title Page.}

Conditional processing can be used to include a title or banner page
in the main document when proper precautions are taken.
Importantly, the code in the main file should ensure that the page counter
(as well as other status parameters which are stored in the |.aux| files)
takes the same value after the conditional processing.
Otherwise the page numbers may take divergent values
depending on which part is compiled.

For example, a title page could be declared by:
%
\begin{center}
\begin{tabular}{l}
|\ifchilddoc\||else|\\
|\addtocounter{page}{-1}|\\
\textit{code for title page}\\
|\newpage|\\
|\||fi|
\end{tabular}
\end{center}
%
A banner page for the child documents can be generated by:
%
\begin{center}
\begin{tabular}{l}
|\ifchilddoc|\\
|\addtocounter{page}{-1}|\\
\textit{code for banner page}\\
|\newpage|\\
|\||fi|
\end{tabular}
\end{center}
%
Here one could write a message such as:
\begin{center}
|This is the part \childdocname{} of \childdocjob{}.|
\end{center}

%%%%%%%%%%%%%%%%%%%%%%%%%%%%%%%%%%%%%%%%%%%%%%%%%%%%%%%%%%%%%%%%%%%%%%%%%%%%%%%%
\subsection{Flags}
\label{sec:flags}

The package makes it easy to generate different versions
of the main or child documents.
To this end compilation flags can be defined
and assigned different default values.
They will be particularly useful in conjunction
with the forwarding mechanism described in \secref{sec:forward}.

For example, it may be useful to have a flag |\version|
which can be set to |draft| or |final|.
The document source will contain some conditional code
depending on the value of |\version|.
Suppose further, the flag should default to |final| for the main file
and to |draft| for child files
which is a natural assignment for editing the document.
This is achieved by placing the following code
in the preamble of the main document
(below the |\childdocmain| directive):
%
\begin{center}
\begin{tabular}{l}
|\ifchilddoc|\\
|\providecommand{\version}{draft}|\\
|\||else|\\
|\providecommand{\version}{final}|\\
|\||fi|
\end{tabular}
\end{center}
%
The definition by |\providecommand| makes sure
that previous definitions are not overwritten.
Further statements |\providecommand{\version}{...}|
can thus be added before the above code to override it.

For the main file, one might add a line
(between |\childdocmain| and the above block)
%
\begin{center}
|%\ifchilddoc\||else\providecommand{\version}{draft}\||fi|
\end{center}
%
which can be uncommented to produce a draft version.
Likewise one can add a line to the very top of a child file
(above the |\childdocof{|\textit{main}|}| directive)
%
\begin{center}
|%\providecommand{\version}{final}|
\end{center}
%
which can be uncommented to produce the final version of this child document.

%%%%%%%%%%%%%%%%%%%%%%%%%%%%%%%%%%%%%%%%%%%%%%%%%%%%%%%%%%%%%%%%%%%%%%%%%%%%%%%%
\subsection{Forwarding}
\label{sec:forward}

Different versions of the main or child documents
using compilation flags as described in \secref{sec:flags}
can be (permanently) stored in different files
for convenient compilation, viewing and distribution.
To this end, the package defines a command
to pass on compilation to a different file:

%%%%%%%%%%%%%%%%%%%%%%%%%%%%%%%%%%%%%%%%
\DescribeMacro{\childdocforward}
The command |\childdocforward| redirects processing to
another source file:
%
\begin{center}
\begin{tabular}{l}
|\input{childdoc.def}|\\
|\childdocforward[|\textit{main}|]{|\textit{dest}|}|\\
\end{tabular}
\end{center}
%
The argument \textit{dest} is the destination file
(without extension).
It should be the main file or one of the child files.
Note that further \textsf{childdoc} directives
such as |\childdocof| and |\childdocforward|
in the indicated file will be processed in this form.
The optional argument \textit{main}
passes on directly to the main file \textit{main}
while pretending to compile the child \textit{dest}.
This form behaves as if \textit{dest}
issues |\childdocof{|\textit{main}|}| right away,
and no further \textsf{childdoc} directives will be processed.

%%%%%%%%%%%%%%%%%%%%%%%%%%%%%%%%%%%%%%%%
\DescribeMacro{\...prefix}
In the alternative form |\childdocforwardprefix|,
%
\begin{center}
\begin{tabular}{l}
|\input{childdoc.def}|\\
|\childdocforwardprefix[|\textit{main}|]{|\textit{prefix}|}{|\textit{dest}|}|
\end{tabular}
\end{center}
%
the destination file is determined by a pattern
depending on the current file:
To make this work, the current file must be called
`{\textit{prefix}\hspace{0.2em}\textit{suffix}}'
with \textit{prefix} matching precisely the argument.
Processing is then passed on to the file
`{\textit{dest}\hspace{0.2em}\textit{suffix}}'.
Surely, the same effect is achieved by
directly specifying the
argument `{\textit{dest}\hspace{0.2em}\textit{suffix}}'
in the first form.
However, that requires to set up a different file
for each child. With the alternative form of the command
all these files can have exactly the same content
which simplifies setting them up and maintaining them.

For example, the following file |draft.tex|
with a compilation flag |\version| as described in \secref{sec:flags}
compiles the main document as a draft:
%
\begin{center}
\begin{tabular}{l}
|\def\version{draft}|\\
|\input{childdoc.def}|\\
|\childdocforward{|\textit{main}|}|
\end{tabular}
\end{center}
%
Likewise, the following files |final|\textit{nn}|.tex|
compile the final version of the child document
|child|\textit{nn}|.tex|:
%
\begin{center}
\begin{tabular}{l}
|\def\version{final}|\\
|\input{childdoc.def}|\\
|\childdocforwardprefix{final}{child}|
\end{tabular}
\end{center}
%

Note that when several versions of a main file and/or of each child file
are to be generated, it may be convenient to set up a |Makefile| or
shell script to automatise the process.

%%%%%%%%%%%%%%%%%%%%%%%%%%%%%%%%%%%%%%%%%%%%%%%%%%%%%%%%%%%%%%%%%%%%%%%%%%%%%%%%
\subsection{Command Line Processing}
\label{sec:commandline}

The effect of redirection files can also be achieved by invoking
the \LaTeX{} compiler with a more elaborate command line.
Most conveniently this should be done as part
of a shell script or a |Makefile|.

When using \textsf{childdoc} in the main file, the following
command lines effectively perform a redirection
(note that depending on the shell being used,
backslashes may have to be doubled: `|\|' $\to$ `|\\|'):
%
\begin{center}
|... -jobname "|\textit{target}|" |\\|"|[\textit{flags}]%
|\input{childdoc.def}\childdocforward[|\textit{main}|]{|\textit{dest}|}"|
\end{center}
%
Here \textit{target} is the name of the output file,
\textit{main} is the name of the main file
and \textit{dest} is the name of the main or child file to be processed
(all filenames without extensions).
The optional argument \textit{main} can be omitted
if \textit{main} matches \textit{dest}.
Optionally, compilation \textit{flags} can be defined via |\def| commands.
This command line makes the \TeX{} engine believe
it is compiling the file \textit{target}
whose content is specified as the latter parameter.
The provided code then forwards the processing to
\textit{main} or \textit{dest} as described in \secref{sec:forward}.

%%%%%%%%%%%%%%%%%%%%%%%%%%%%%%%%%%%%%%%%%%%%%%%%%%%%%%%%%%%%%%%%%%%%%%%%%%%%%%%%
\subsection{Include by Input}
\label{sec:input}

Including child documents by |\include| has some restrictions by design.
Most notably, the content of a child document always occupies
its own set of pages; pages cannot be shared between child documents.
Usually, this behaviour makes perfect sense
because each child document contain an essential part of the document.
However, in some situations it may be desirable to compose
a document from a collection of parts
without having mandatory page breaks between then.
For this case, the package
provides a mechanism to include parts
by |\input| which can also be processed individually.
However, by construction this mechanism
requires manual handling of the content to be output.

%%%%%%%%%%%%%%%%%%%%%%%%%%%%%%%%%%%%%%%%
\DescribeMacro{\ifchilddocmanual}
The main file should be prepared as usual, see \secref{sec:include}.
However, the document body must make a distinction
between processing of an individual part and of the main document, e.g.:
%
\begin{center}
\begin{tabular}{l}
|\ifchilddocmanual|\\
|\input{\childdocname}|\\
|\||else|\\
\textit{document body with }|\input{|\textit{part}|}|\\
|\||fi|
\end{tabular}
\end{center}
%
The conditional |\ifchilddocmanual| is true whenever
a part to be included by |\input| is being compiled,
and the name of the part is stored in |\childdocname|.

%%%%%%%%%%%%%%%%%%%%%%%%%%%%%%%%%%%%%%%%
\DescribeMacro{\childdocby}
Each part to be included by |\input| should start with:
%
\begin{center}
\begin{tabular}{l}
|\input{childdoc.def}|\\
|\childdocby{|\textit{main}|}|\\
\end{tabular}
\end{center}
%
The directive |\childdocby| is similar to |\childdocof|
described in \secref{sec:include},
but the subsequent selection of content must be done manually.
To that end, both |\ifchilddoc| and |\ifchilddocmanual|
will be true upon processing of a part,
and the name of the part is stored in |\childdocname|.
Note that |\jobname| will be set to the filename of the current part
so that each part receives an individual |.aux| file
that does not interfere with the |.aux| file(s) of the main document.
This behaviour can be altered by the alternative form
|\childdocby[*]{|\textit{main}|}| (with a non-empty optional argument)
which uses the |.aux| file of the main document
by setting |\jobname| to \textit{main}.

%%%%%%%%%%%%%%%%%%%%%%%%%%%%%%%%%%%%%%%%%%%%%%%%%%%%%%%%%%%%%%%%%%%%%%%%%%%%%%%%
\subsection{Driver Development}
\label{sec:driver}

The \textsf{childdoc} mechanism can also be use for the development
of definition files such as \LaTeX{} styles or classes.
This case differs from the above setup with multiple parts
included by |\include| in that no |\includeonly| should be invoked.
This can be achieved by starting the include file
(before |\ProvidesPackage|) with:
%
\begin{center}
\begin{tabular}{l}
|\input{childdoc.def}|\\
|\childdocforward{|\textit{main}|}|\\
\end{tabular}
\end{center}
%
or alternatively with:
%
\begin{center}
\begin{tabular}{l}
|\input{childdoc.def}|\\
|\childdocby{|\textit{main}|}|\\
\end{tabular}
\end{center}
%
Both forms have slightly different effects as described above.
The main file is prepared as usual, see \secref{sec:include}.

%%%%%%%%%%%%%%%%%%%%%%%%%%%%%%%%%%%%%%%%%%%%%%%%%%%%%%%%%%%%%%%%%%%%%%%%%%%%%%%%
\subsection{Legacy Detection}
\label{sec:detection}

The directive |\childdocmain| in the main file can detect
whether the complete document or merely a child is to be compiled
even without using the directive |\childdocof|.
This method is deprecated because it is less robust
and there is no compelling reason to use it;
it is merely provided for backward compatibility
and it may be removed in future versions.

If the detection mechanism is to be used,
it is mandatory to correctly specify
the filename of the main file as the argument of |\childdocmain|:
%
\begin{center}
\begin{tabular}{l}
|\input{childdoc.def}|\\
|\childdocmain{|\textit{main}|}|\\
\end{tabular}
\end{center}
%
If |\jobname| does not match the argument \textit{main} of |\childdocmain|,
it is assumed that |\jobname| points to the child file to be compiled.
When using |\childdocmain| with the main file specified as argument,
it suffices to start a child file
with just |\input{|\textit{main}|}|
without loading of the package and using |\childdocof|.
If instead all processing is done
with the appropriate \textsf{childdoc} directives,
the argument of \textit{main} of |\childdocmain| can be empty.

An alternative version of the command line processing described
in \secref{sec:commandline} using the detection mechanism reads:
%
\begin{center}
|... -jobname "|\textit{target}|" "|[\textit{flags}]%
[|\def\jobname{|\textit{dest}|}|]|\input{|\textit{main}|}"|
\end{center}

%%%%%%%%%%%%%%%%%%%%%%%%%%%%%%%%%%%%%%%%%%%%%%%%%%%%%%%%%%%%%%%%%%%%%%%%%%%%%%%%
\subsection{Manual Code}
\label{sec:manual}

In case one cannot be certain whether the definitions file |childdoc.def|
is installed on the target \TeX{} distribution
and one prefers not to ship it,
it is conceivable to paste a few relevant commands into the sources.

To that end, drop all statements |\input{childdoc.def}|
and perform the replacements as outlined below.
Instead of |\childdocmain{|\textit{main}|}| add the following code
to the top of the main file:
%
\begin{center}
\begin{tabular}{l}
|\||ifdefined\childdocname\endinput\||fi\newif\ifchilddoc|\\
|\edef\childdocname{\scantokens\expandafter{\jobname\noexpand}}|\\
|\def\childdocmain{|\textit{main}|}\||ifx\childdocmain\childdocname\||else|\\
|\childdoctrue\includeonly{\childdocname}\let\jobname\childdocmain\||fi|\\
\end{tabular}
\end{center}
%
Instead of |\childdocof{|\textit{main}|}| just include the main file
at the top of each child file:
%
\begin{center}
|\input{|\textit{main}|}|
\end{center}
%
A simple redirection |\childdocforward{|\textit{dest}|}| is achieved by:
%
\begin{center}
|\def\jobname{|\textit{dest}|}\input{\jobname}|
\end{center}
%
The redirection with prefix
|\childdocforwardprefix[|\textit{prefix}|]{|\textit{dest}|}|
is accomplished by:
%
\begin{center}
\begin{tabular}{l}
|{\edef\jobname{\scantokens\expandafter{\jobname\noexpand}}|\\
|\def\redirectjob |\textit{prefix}|#1~~~{\gdef\jobname{|\textit{dest}|#1}}|\\
|\expandafter\redirectjob\jobname~~~}\input{\jobname}|
\end{tabular}
\end{center}

In an alternative approach,
child documents can be compiled by a specific command line
without additional code or specific definitions:
%
\begin{center}
|... -jobname "|\textit{target}|" "|[\textit{flags}]%
|\includeonly{|\textit{dest}|}\input{|\textit{main}|}"|
\end{center}
%

%%%%%%%%%%%%%%%%%%%%%%%%%%%%%%%%%%%%%%%%%%%%%%%%%%%%%%%%%%%%%%%%%%%%%%%%%%%%%%%%
%%%%%%%%%%%%%%%%%%%%%%%%%%%%%%%%%%%%%%%%%%%%%%%%%%%%%%%%%%%%%%%%%%%%%%%%%%%%%%%%
\section{Information}

%%%%%%%%%%%%%%%%%%%%%%%%%%%%%%%%%%%%%%%%%%%%%%%%%%%%%%%%%%%%%%%%%%%%%%%%%%%%%%%%
\subsection{Copyright}

Copyright \copyright{} 2017--2018 Niklas Beisert

This work may be distributed and/or modified under the
conditions of the \LaTeX{} Project Public License, either version 1.3
of this license or (at your option) any later version.
The latest version of this license is in
  \url{http://www.latex-project.org/lppl.txt}
and version 1.3 or later is part of all distributions of \LaTeX{}
version 2005/12/01 or later.

This work has the LPPL maintenance status `maintained'.

The Current Maintainer of this work is Niklas Beisert.

This work consists of the files |README.txt|, |childdoc.ins| and |childdoc.dtx|
as well as the derived files |childdoc.def|, |cdocsamp.tex|
with |cdocsch1.tex|, |cdocsch2.tex|, |cdocspt3.tex|, |cdocspt4.tex|,
|cdocsdrf.tex|, |cdocsfn1.tex|, |cdocsfn2.tex|
as well as |childdoc.pdf|.

%%%%%%%%%%%%%%%%%%%%%%%%%%%%%%%%%%%%%%%%%%%%%%%%%%%%%%%%%%%%%%%%%%%%%%%%%%%%%%%%
\subsection{Files and Installation}

The package consists of the files:
%
\begin{center}
\begin{tabular}{ll}
    |README.txt|   & readme file \\
    |childdoc.ins| & installation file \\
    |childdoc.dtx| & source file \\
    |childdoc.def| & definition file \\
    |cdocsamp.tex| & sample main file \\
    |cdocsch1.tex| & sample include file \\
    |cdocsch2.tex| & sample include file \\
    |cdocspt3.tex| & sample part file \\
    |cdocspt4.tex| & sample part file \\
    |cdocsdrf.tex| & sample redirection file \\
    |cdocsfn1.tex| & sample redirection file \\
    |cdocsfn2.tex| & sample redirection file \\
    |childdoc.pdf| & manual
\end{tabular}
\end{center}
%
The distribution consists of the files
|README.txt|, |childdoc.ins| and |childdoc.dtx|.
%
\begin{itemize}
\item
Run (pdf)\LaTeX{} on |childdoc.dtx|
to compile the manual |childdoc.pdf| (this file).
\item
Run \LaTeX{} on |childdoc.ins| to create the definitions file |childdoc.def|
and the sample |cdocsamp.tex| with include files
|cdocsch1.tex|, |cdocsch2.tex|, |cdocspt3.tex|, |cdocspt4.tex|,
|cdocsdrf.tex|, |cdocsfn1.tex|, |cdocsfn2.tex|.
Then copy the file |childdoc.def| to an appropriate directory of your \LaTeX{}
distribution, e.g.\ \textit{texmf-root}|/tex/latex/childdoc|.
\end{itemize}

%%%%%%%%%%%%%%%%%%%%%%%%%%%%%%%%%%%%%%%%%%%%%%%%%%%%%%%%%%%%%%%%%%%%%%%%%%%%%%%%
\subsection{Related CTAN Packages}

There are several other packages which offer a similar functionality:
%
\begin{itemize}
\item
The packages
\href{http://ctan.org/pkg/docmute}{\textsf{docmute}},
\href{http://ctan.org/pkg/includex}{\textsf{includex}} and
\href{http://ctan.org/pkg/standalone}{\textsf{standalone}}
provide commands to include only the document body of
a child file thus allowing both files to be compiled individually.
\item
The packages \href{http://ctan.org/pkg/subdocs}{\textsf{subdocs}}
and \href{http://ctan.org/pkg/subfiles}{\textsf{subfiles}}
provide structures in which the main and child documents can be
encapsulated and allowing them to be compiled individually.
The inclusion mechanism is different from the conventional |\include|.
\item
The package \href{http://ctan.org/pkg/combine}{\textsf{combine}}
is an elaborate solution to combine several documents into one.
\end{itemize}
%
See also the CTAN topic \href{http://ctan.org/topic/subdocs}{\textsf{subdocs}}
for further related packages.
The present package differs from the above solutions in that
a document structure constructed with the conventional |\include| mechanism
just needs two extra commands at the top of every file
such that all constituent files can be compiled individually.

%%%%%%%%%%%%%%%%%%%%%%%%%%%%%%%%%%%%%%%%%%%%%%%%%%%%%%%%%%%%%%%%%%%%%%%%%%%%%%%%
%\subsection{Feature Suggestions}
%
%The following is a list of features which may be useful for future
%versions of this package:
%%
%\begin{itemize}
%\item
%\ldots
%\end{itemize}

%%%%%%%%%%%%%%%%%%%%%%%%%%%%%%%%%%%%%%%%%%%%%%%%%%%%%%%%%%%%%%%%%%%%%%%%%%%%%%%%
\subsection{Revision History}

%%%%%%%%%%%%%%%%%%%%%%%%%%%%%%%%%%%%%%%%
\paragraph{v2.0:} 2018/12/30

\begin{itemize}
\item
immediate forward processing
\item
added |\childdocby| mechanism
\item
manual restructured
\end{itemize}

%%%%%%%%%%%%%%%%%%%%%%%%%%%%%%%%%%%%%%%%
\paragraph{v1.6:} 2018/01/17

\begin{itemize}
\item
application for development of include files
\item
corrections to manual
\end{itemize}

%%%%%%%%%%%%%%%%%%%%%%%%%%%%%%%%%%%%%%%%
\paragraph{v1.5:} 2017/05/21

\begin{itemize}
\item
more complete structuring introduced
\item
|\childdocof| introduced
\item
|\childdoc| renamed to |\childdocmain|
\item
|\childredirect| renamed to |\childdocforward| and |\childdocforwardprefix|
and functionality expanded
\end{itemize}

%%%%%%%%%%%%%%%%%%%%%%%%%%%%%%%%%%%%%%%%
\paragraph{v1.0:} 2017/04/27

\begin{itemize}
\item
manual and install package
\item
first version published on CTAN
\end{itemize}

%%%%%%%%%%%%%%%%%%%%%%%%%%%%%%%%%%%%%%%%
\paragraph{v0.6:} 2017/04/26

\begin{itemize}
\item
redirection mechanism added
\end{itemize}

%%%%%%%%%%%%%%%%%%%%%%%%%%%%%%%%%%%%%%%%
\paragraph{v0.5:} 2017/04/26

\begin{itemize}
\item
functionality in definition file
\end{itemize}


%%%%%%%%%%%%%%%%%%%%%%%%%%%%%%%%%%%%%%%%%%%%%%%%%%%%%%%%%%%%%%%%%%%%%%%%%%%%%%%%
%%%%%%%%%%%%%%%%%%%%%%%%%%%%%%%%%%%%%%%%%%%%%%%%%%%%%%%%%%%%%%%%%%%%%%%%%%%%%%%%
%%%%%%%%%%%%%%%%%%%%%%%%%%%%%%%%%%%%%%%%%%%%%%%%%%%%%%%%%%%%%%%%%%%%%%%%%%%%%%%%
\appendix

\settowidth\MacroIndent{\rmfamily\scriptsize 000\ }

 \DocInput{childdoc.dtx}

\end{document}
%</driver>
% \fi
%
% %%%%%%%%%%%%%%%%%%%%%%%%%%%%%%%%%%%%%%%%%%%%%%%%%%%%%%%%%%%%%%%%%%%%%%%%%%%%%%
% %%%%%%%%%%%%%%%%%%%%%%%%%%%%%%%%%%%%%%%%%%%%%%%%%%%%%%%%%%%%%%%%%%%%%%%%%%%%%%
% \section{Sample}
%\iffalse
%<*samplemain>
%\fi
%
% The following presents a sample document
% with two chapters, two parts, a title page,
% a compile flag as well as three forwarding files to set the flag.
% It consists of eight |.tex| files:
% \begin{center}
% \begin{tabular}{ll}
% |cdocsamp.tex|&main file\\
% |cdocsch1.tex|&include file for chapter 1\\
% |cdocsch2.tex|&include file for chapter 2\\
% |cdocspt3.tex|&include file for part 3\\
% |cdocspt4.tex|&include file for part 4\\
% |cdocsdrf.tex|&forwarding file for main file in draft mode\\
% |cdocsfi1.tex|&forwarding file for final version of chapter 1\\
% |cdocsfi2.tex|&forwarding file for final version of chapter 2\\
% \end{tabular}
% \end{center}
% Each of the eight files can be compiled directly by the \LaTeX{} compiler.
%
% %%%%%%%%%%%%%%%%%%%%%%%%%%%%%%%%%%%%%%
% \paragraph{Main File.}
%
% The main file is called |cdocsamp.tex|.
%
% Load the \textsf{childdoc} definitions and
% declare the filename for the main document:
%    \begin{macrocode}
\input{childdoc.def}
\childdocmain{}
%    \end{macrocode}

% Optional override for |\version| flag:
%    \begin{macrocode}
%%\ifchilddoc\else\providecommand{\version}{draft}\fi
%    \end{macrocode}

% Define the default values for the |\version| flag
% (|final| for the main file and |draft| for childs):
%    \begin{macrocode}
\ifchilddoc
\providecommand{\version}{draft}
\else
\providecommand{\version}{final}
\fi
%    \end{macrocode}

% Load the standard document class:
%    \begin{macrocode}
\documentclass[12pt]{article}
%    \end{macrocode}

% Start the document body:
%    \begin{macrocode}
\begin{document}
%    \end{macrocode}

% Declare a title page.
% Print title, part of document being processed and version flag:
%    \begin{macrocode}
\addtocounter{page}{-1}
\begin{center}
{\LARGE\bfseries{}childdoc example\par}
\vspace{1cm}
\ifchilddoc
\ifchilddocmanual part\else chapter\fi:
`\childdocname' of `\childdocjob'\par
\else
main document: `\childdocjob'\par
\fi
version: \version\par
\end{center}
\newpage
%    \end{macrocode}

% Manually include selected file,
% otherwise process as usual:
%    \begin{macrocode}
\ifchilddocmanual
\section*{part `\childdocname'}
\input{\childdocname}
\else
%    \end{macrocode}

% Include the two chapters:
%    \begin{macrocode}
\include{cdocsch1}
\include{cdocsch2}
%    \end{macrocode}

% Include the two parts unless only chapters should be displayed:
%    \begin{macrocode}
\ifchilddoc\else
\section{part three}
\input{cdocspt3}
\section{part four}
\input{cdocspt4}
\fi
%    \end{macrocode}

% Process as usual until here:
%    \begin{macrocode}
\fi
%    \end{macrocode}

% End of document body:
%    \begin{macrocode}
\end{document}
%    \end{macrocode}
%\iffalse
%</samplemain>
%\fi
%
% %%%%%%%%%%%%%%%%%%%%%%%%%%%%%%%%%%%%%%
% \paragraph{Chapter Include Files.}
%
% The include files are called |cdocsch1.tex| and |cdocsch2.tex|.
%
%\iffalse
%<*samplechap1|samplechap2>
%\fi

% Optional override for |\version| flag:
%    \begin{macrocode}
%%\providecommand{\version}{final}
%    \end{macrocode}

% Include the main document:
%    \begin{macrocode}
\input{childdoc.def}
\childdocof{cdocsamp}
%    \end{macrocode}

%\iffalse
%</samplechap1|samplechap2>
%\fi
%
%\iffalse
%<*samplechap1>
%\fi
% Some text for chapter 1:
%    \begin{macrocode}
\section{one}
some text in chapter one
%    \end{macrocode}

%\iffalse
%</samplechap1>
%\fi
% Some text for chapter 2:
%\iffalse
%<*samplechap2>
%\fi
%    \begin{macrocode}
\section{two}
more text in chapter two
%    \end{macrocode}

%\iffalse
%</samplechap2>
%\fi
%
% %%%%%%%%%%%%%%%%%%%%%%%%%%%%%%%%%%%%%%
% \paragraph{Part Include Files.}
%
% The include files are called |cdocspt3.tex| and |cdocspt4.tex|.
%
%\iffalse
%<*samplepart3|samplepart4>
%\fi

% Optional override for |\version| flag:
%    \begin{macrocode}
%%\providecommand{\version}{final}
%    \end{macrocode}

% Include the main document:
%    \begin{macrocode}
\input{childdoc.def}
\childdocby{cdocsamp}
%    \end{macrocode}

%\iffalse
%</samplepart3|samplepart4>
%\fi
%
%\iffalse
%<*samplepart3>
%\fi
% Some text for part 3:
%    \begin{macrocode}
some text in part three
%    \end{macrocode}

%\iffalse
%</samplepart3>
%\fi
% Some text for part 4:
%\iffalse
%<*samplepart4>
%\fi
%    \begin{macrocode}
more text in part four
%    \end{macrocode}

%\iffalse
%</samplepart4>
%\fi
%
% %%%%%%%%%%%%%%%%%%%%%%%%%%%%%%%%%%%%%%
% \paragraph{Forwarding for a Complete Draft.}
%
% The following forwarding file |cdocsdrf.tex|
% compiles the main document in draft mode:
%\iffalse
%<*sampledraft>
%\fi
%    \begin{macrocode}
\def\version{draft}
\input{childdoc.def}
\childdocforward{cdocsamp}
%    \end{macrocode}

%\iffalse
%</sampledraft>
%\fi
%
% %%%%%%%%%%%%%%%%%%%%%%%%%%%%%%%%%%%%%%
% \paragraph{Forwarding for Final Version of the Chapters.}
%
% The following forwarding files |cdocsfn1.tex| and |cdocsfn2.tex|
% (with identical content)
% compile the final versions of the child documents
% |cdocsch1.tex| and |cdocsch2.tex|, respectively:
%\iffalse
%<*samplefinal>
%\fi
%    \begin{macrocode}
\def\version{final}
\input{childdoc.def}
\childdocforwardprefix[cdocsamp]{cdocsfn}{cdocsch}
%    \end{macrocode}

%\iffalse
%</samplefinal>
%\fi
%
% %%%%%%%%%%%%%%%%%%%%%%%%%%%%%%%%%%%%%%
% \paragraph{Command Line Processing.}
%
% The following three command lines generate the output files
% |cdocscld|, |cdocscl1| and |cdocscl2|
% which should be identical to
% |cdocsdrf|, |cdocsch1| and |cdocsfn2|, respectively:
% \begin{center}
% \begin{tabular}{l}
% |latex -jobname cdocscld \|\\
% |  "\def\version{draft}\input{childdoc.def}\childdocforward{cdocsamp}"|\\
% |latex -jobname cdocscl1 \|\\
% |  "\input{childdoc.def}\childdocforward[cdocsamp]{cdocsch1}"|\\
% |latex -jobname cdocscl2 \|\\
% |  "\def\version{final}\input{childdoc.def}\childdocforward{cdocsch2}"|
% \end{tabular}
% \end{center}
% Note that the trailing backslash on each first line
% merely continues the input to the second line
% (for convenient cut ant paste).
% Furthermore, the command |latex| can be replaced by any
% of its alternative versions such as |pdflatex|.
%
% %%%%%%%%%%%%%%%%%%%%%%%%%%%%%%%%%%%%%%%%%%%%%%%%%%%%%%%%%%%%%%%%%%%%%%%%%%%%%%
% %%%%%%%%%%%%%%%%%%%%%%%%%%%%%%%%%%%%%%%%%%%%%%%%%%%%%%%%%%%%%%%%%%%%%%%%%%%%%%
% \section{Implementation}
%\iffalse
%<*package>
%\fi
%
% This section describes the definitions file |childdoc.def|.

% The definitions cannot be loaded using |\usepackage| or |\RequirePackage|
% which has a mechanism to prevent loading a style file more than once.
% When loading the definitions by means of |\input|
% multiple instances have to be prevented manually:
%\iffalse
%This code needs to be before the `\ProvidesFile' directive
%which is defined at the beginning of this file.
%Therefore it is also placed there and commented out here.
%</package>
%<*discard>
%\fi
%    \begin{macrocode}
\ifdefined\childdocmain\endinput\fi
%    \end{macrocode}
%\iffalse
%</discard>
%<*package>
%\fi
%
% \macro{\ifchilddoc}
% \macro{\ifchilddocmanual}
% The conditional |\ifchilddoc| tells whether a
% child (true) or main (false) document is being compiled.
% The conditional |\ifchilddocmanual| tells whether
% the |\includeonly| mechanism is used (false) or
% the selection of child files must be performed manually (true).
% The definitions initialise to false:
%    \begin{macrocode}
\newif\ifchilddoc
\newif\ifchilddocmanual
%    \end{macrocode}

% \macro{\childdocname}
% \macro{\childdocjob}
% The macro |\childdocname| stores the name of the main document
% to be compiled. The macro |\childdocjob| stores the name of
% the document on which the \LaTeX{} compiler was originally invoked.
% The content of |\jobname| cannot be compared
% to filenames specified in the source due to different catcodes.
% The following code rescans |\jobname|, stores the result
% in |\childdocname| and saves a copy in |\childdocjob|:
%    \begin{macrocode}
\edef\childdocname{\scantokens\expandafter{\jobname\noexpand}}
\let\childdocjob\childdocname
%    \end{macrocode}

% \macro{\childdocdisable}
% The macro |\childdocdisable| prevents the main file
% from being processed more than once.
% At this stage, the main document command |\childdocmain|
% is assumed to be called once again where it should do nothing.
% Any subsequent call to it should prevent
% a secondary processing of the main document
% It overwrites the forwarding commands
% |\childdocof| and |\childdocforward|
% with empty macros to prevent further inclusions of the main document:
%    \begin{macrocode}
\newcommand{\childdocdisable}
{
  \renewcommand{\childdocmain}[1]{\renewcommand{\childdocmain}[1]{\endinput}}
  \renewcommand{\childdocof}[1]{}
  \renewcommand{\childdocby}[2][]{}
  \renewcommand{\childdocforward}[2][]{}
  \renewcommand{\childdocdisable}{}
}
%    \end{macrocode}

% \macro{\childdocmain}
% The macro |\childdocmain| is to be called at the top of the main file
% with nothing or the main filename (without extension) as argument.
% First, it breaks loops.
% If the argument is not empty and does not match |\childdocname|
% (which is set by the first inclusion of |childdoc.def|),
% |\ifchilddoc| is set to true, |\includeonly| is applied to the child file
% and |\jobname| is set to the main file
% (for proper handling of |.aux| files):
%    \begin{macrocode}
\newcommand{\childdocmain}[1]
{
  \childdocdisable\childdocmain{}
  \if?#1?\else
    \begingroup
      \def\childdoctmp{#1}
      \ifx\childdoctmp\childdocname
        \def\childdoctmp{}
      \else
        \def\childdoctmp
        {
          \childdoctrue
          \includeonly{\childdocname}
          \def\childdocjob{#1}
          \def\jobname{#1}
        }
      \fi
      \expandafter
    \endgroup
    \childdoctmp
  \fi
}
%    \end{macrocode}

% \macro{\childdocof}
% The command |\childdocof| redirects
% compilation to the main file |#1|.
%    \begin{macrocode}
\newcommand{\childdocof}[1]
{
  \childdocdisable
  \childdoctrue
  \includeonly{\childdocname}
  \def\jobname{#1}
  \def\childdocjob{#1}
  \input{#1}
}
%    \end{macrocode}

% \macro{\childdocby}
% The command |\childdocby| ....
%    \begin{macrocode}
\newcommand{\childdocby}[2][]
{
  \childdocdisable
  \childdoctrue
  \childdocmanualtrue
  \if?#1?\else
    \def\jobname{#2}
  \fi
  \def\childdocjob{#2}
  \input{#2}
  \endinput
}
%    \end{macrocode}

% \macro{\childdocforward}
% The command |\childdocforward| redirects
% compilation to the main file or
% (if the optional argument is given) a child file.
% Parameters are set as if the main file
% or a child file starting with |\childdocof| was compiled.
% Then compilation is handed over to the main file:
%    \begin{macrocode}
\newcommand{\childdocforward}[2][]
{
  \begingroup
    \if?#1?
      \def\childdoctmp
      {
        \def\childdocname{#2}
        \def\childdocjob{#2}
        \def\jobname{#2}
        \input{#2}
        \endinput
      }
    \else
      \def\childdoctmp
      {
        \childdocdisable
        \def\childdocname{#2}
        \childdoctrue
        \includeonly{#2}
        \def\childdocjob{#1}
        \def\jobname{#1}
        \input{#1}
        \endinput
      }
    \fi
    \expandafter
  \endgroup
  \childdoctmp
}
%    \end{macrocode}

% \macro{\childdocforwardprefix}
% The command |\childdocforwardprefix| redirects
% compilation to the main or a child file by means of a pattern.
% The prefix |#1| in the current filename is replaced by |#2|
% and the suffix of the current filename is kept
% (it is assumed that the filename does not contain the substring `|~~~|'
% which is used as a delimiter).
% Compilation is handed over to the new file by |\childdocforward|:
%    \begin{macrocode}
\newcommand{\childdocforwardprefix}[3][]
{
  \begingroup
    \def\childdocextract #2##1~~~{\def\childdoctmp{\childdocforward[#1]{#3##1}}}
    \expandafter\childdocextract\childdocname~~~
    \expandafter
  \endgroup
  \childdoctmp
}
%    \end{macrocode}

% \macro{\childdoc}
% The deprecated macro |\childdoc| is a legacy version of |\childdocmain|:
%    \begin{macrocode}
\newcommand{\childdoc}{\childdocmain}
%    \end{macrocode}

% \macro{\childdocredirect}
% The deprecated macro |\childdocredirect| is a legacy version
% of |\childdocforward| and |\childdocforwardprefix|:
%    \begin{macrocode}
\newcommand{\childdocredirect}[2][]
{
  \begingroup
    \if?#1?
      \def\childdoctmp{\childdocforward{#2}}
    \else
      \def\childdoctmp{\childdocforwardprefix{#1}{#2}}
    \fi
    \expandafter
  \endgroup
  \childdoctmp
}
%    \end{macrocode}

%\iffalse
%</package>
%\fi
%
\endinput
|\\
|\childdocforwardprefix[|\textit{main}|]{|\textit{prefix}|}{|\textit{dest}|}|
\end{tabular}
\end{center}
%
the destination file is determined by a pattern
depending on the current file:
To make this work, the current file must be called
`{\textit{prefix}\hspace{0.2em}\textit{suffix}}'
with \textit{prefix} matching precisely the argument.
Processing is then passed on to the file
`{\textit{dest}\hspace{0.2em}\textit{suffix}}'.
Surely, the same effect is achieved by
directly specifying the
argument `{\textit{dest}\hspace{0.2em}\textit{suffix}}'
in the first form.
However, that requires to set up a different file
for each child. With the alternative form of the command
all these files can have exactly the same content
which simplifies setting them up and maintaining them.

For example, the following file |draft.tex|
with a compilation flag |\version| as described in \secref{sec:flags}
compiles the main document as a draft:
%
\begin{center}
\begin{tabular}{l}
|\def\version{draft}|\\
|% \iffalse
%
% childdoc.dtx Copyright (C) 2017-2018 Niklas Beisert
%
% This work may be distributed and/or modified under the
% conditions of the LaTeX Project Public License, either version 1.3
% of this license or (at your option) any later version.
% The latest version of this license is in
%   http://www.latex-project.org/lppl.txt
% and version 1.3 or later is part of all distributions of LaTeX
% version 2005/12/01 or later.
%
% This work has the LPPL maintenance status `maintained'.
%
% The Current Maintainer of this work is Niklas Beisert.
%
% This work consists of the files childdoc.dtx and childdoc.ins
% and the derived files childdoc.def and cdocsamp.tex with
% cdocsch1.tex, cdocsch2.tex, cdocsdrf.tex, cdocsfn1.tex, cdocsfn2.tex.
%
%<package>\ifdefined\childdocmain\endinput\fi
%<package>\ProvidesFile{childdoc.def}[2018/12/30 v2.0 child document driver]
%<samplemain>\ProvidesFile{cdocsamp.tex}[2018/12/30 v2.0 sample for childdoc]
%<*driver>
%\ProvidesFile{childdoc.drv}[2018/12/30 v2.0 childdoc reference manual file]
\PassOptionsToClass{10pt,a4paper}{article}
\documentclass{ltxdoc}

\usepackage[margin=35mm]{geometry}
\usepackage{hyperref}
\usepackage{hyperxmp}
\usepackage[usenames]{color}

\hypersetup{colorlinks=true}
\hypersetup{pdfstartview=FitH}
\hypersetup{pdfpagemode=UseNone}
\hypersetup{pdfsource={}}
\hypersetup{pdflang={en-UK}}
\hypersetup{pdfcopyright={Copyright 2017-2018 Niklas Beisert.
  This work may be distributed and/or modified under the
  conditions of the LaTeX Project Public License, either version 1.3
  of this license or (at your option) any later version.}}
\hypersetup{pdflicenseurl={http://www.latex-project.org/lppl.txt}}
\hypersetup{pdfcontactaddress={ETH Zurich, ITP, HIT K,
  Wolfgang-Pauli-Strasse 27}}
\hypersetup{pdfcontactpostcode={8093}}
\hypersetup{pdfcontactcity={Zurich}}
\hypersetup{pdfcontactcountry={Switzerland}}
\hypersetup{pdfcontactemail={nbeisert@itp.phys.ethz.ch}}
\hypersetup{pdfcontacturl={http://people.phys.ethz.ch/\xmptilde nbeisert/}}

\newcommand{\secref}[1]{\hyperref[#1]{section \ref*{#1}}}

\parskip1ex
\parindent0pt
\let\olditemize\itemize
\def\itemize{\olditemize\parskip0pt}

\begin{document}

\title{The \textsf{childdoc} Package}
\hypersetup{pdftitle={The childdoc Package}}
\author{Niklas Beisert\\[2ex]
  Institut f\"ur Theoretische Physik\\
  Eidgen\"ossische Technische Hochschule Z\"urich\\
  Wolfgang-Pauli-Strasse 27, 8093 Z\"urich, Switzerland\\[1ex]
  \href{mailto:nbeisert@itp.phys.ethz.ch}
  {\texttt{nbeisert@itp.phys.ethz.ch}}}
\hypersetup{pdfauthor={Niklas Beisert}}
\hypersetup{pdfsubject={Manual for the LaTeX2e Package childdoc}}
\date{30 December 2018, \textsf{v2.0}}
\maketitle

\begin{abstract}\noindent
\textsf{childdoc} is a \LaTeXe{} package
that enables the direct compilation
of document sections included by |\include|
to individual files.
\end{abstract}

\begingroup
\parskip0ex
\tableofcontents
\endgroup

%%%%%%%%%%%%%%%%%%%%%%%%%%%%%%%%%%%%%%%%%%%%%%%%%%%%%%%%%%%%%%%%%%%%%%%%%%%%%%%%
%%%%%%%%%%%%%%%%%%%%%%%%%%%%%%%%%%%%%%%%%%%%%%%%%%%%%%%%%%%%%%%%%%%%%%%%%%%%%%%%
\section{Introduction}

\LaTeX{} provides a mechanism to structure a large document (such as a book)
into a main file and several child files (containing the chapters)
using the |\include| command.
This mechanism is beneficial for documents
which span hundreds of pages in order to
make the source file(s) more manageable.
Moreover, compilation can be restricted to
selected child files by means of the |\includeonly| command.
The latter feature can be used to reduce the compilation time while editing
(this was significantly more useful in the earlier days of \LaTeX{})
or to generate a smaller document which is easier to navigate.
Another application of |\includeonly| is to generate
documents consisting of selected parts of the complete document.

However, there are a few drawbacks of the plain |\include| mechanism:
\begin{itemize}
\item
The child files cannot be compiled on their own,
they can only be compiled via the main file.
A naive editing environment
(such as a text editor with an option
to have the current file processed by \LaTeX)
may require one to switch to the main file before compiling;
attempting to compile the child file produces errors.
\item
The main file must be modified (each time)
to adjust the |\includeonly| command
to the present needs. This easily leaves the main file in a messy state.
\item
The generated document will always carry the filename
of the main document. This is inconvenient if
several child files are to be compiled and
to be kept for distribution.
\end{itemize}

The present package provides a simple interface
to make child files individually compilable by \LaTeX{}.
Compiling a child file then has the same effect as compiling
the main file with an |\includeonly| command
to select the appropriate child.
Moreover the generated document will carry the name of the child
rather than the main file.
This resolves all three above issues.

This feature is meant to make the editing of books,
thesis documents and lecture notes somewhat more convenient.
However, the package can also be used efficiently for
composing a series of documents (such as exercise sheets)
which are typically distributed individually.
It then assists the author in generating the individual documents
(potentially in different versions)
as well as a document containing the collected series.
Another application is in developing style files
or other kinds of included material
where compilation of the style file could redirect
to a sample or test file.

%%%%%%%%%%%%%%%%%%%%%%%%%%%%%%%%%%%%%%%%%%%%%%%%%%%%%%%%%%%%%%%%%%%%%%%%%%%%%%%%
%%%%%%%%%%%%%%%%%%%%%%%%%%%%%%%%%%%%%%%%%%%%%%%%%%%%%%%%%%%%%%%%%%%%%%%%%%%%%%%%
\section{Usage}

First of all, the package \textsf{childdoc} is \emph{not} a standard
\LaTeXe{} |.sty| style file! Therefore it needs to be invoked in
a non-standard way.

%%%%%%%%%%%%%%%%%%%%%%%%%%%%%%%%%%%%%%%%%%%%%%%%%%%%%%%%%%%%%%%%%%%%%%%%%%%%%%%%
\subsection{Included Files}
\label{sec:include}

%%%%%%%%%%%%%%%%%%%%%%%%%%%%%%%%%%%%%%%%
\DescribeMacro{\childdocmain}
To use the package, add the commands
\begin{center}
\begin{tabular}{l}
|\input{childdoc.def}|\\
|\childdocmain{}|\\
\end{tabular}
\end{center}
at the very top of the main \LaTeX{} file,
in particular \emph{before} the |\documentclass| statement!
The argument of |\childdocmain| should be left empty
(but it must be present).

%%%%%%%%%%%%%%%%%%%%%%%%%%%%%%%%%%%%%%%%
\DescribeMacro{\childdocof}
Furthermore, add the commands
\begin{center}
\begin{tabular}{l}
|\input{childdoc.def}|\\
|\childdocof{|\textit{main}|}|\\
\end{tabular}
\end{center}
at the top of every child file \textit{child}
which is included by |\include{|\textit{child}|}|
from within the main file
(or at least for those files to be compiled individually).
The argument \textit{main} must be the filename of the main file.

There are a couple of
considerations in setting up the main and child documents:

%%%%%%%%%%%%%%%%%%%%%%%%%%%%%%%%%%%%%%%%
\paragraph{Restrictions.}

Please note the following restrictions:
\begin{itemize}
\item
|\childdocmain| must be called with one argument \textit{main}
to ensure compatibility with earlier version of the package.
It must either be empty (|\childdocmain{}|)
or precisely match the filename of the main file in which it is specified.
See \secref{sec:detection} for further information.
\item
The filename \textit{main} must be specified without the |.tex| extension.
\item
The filename \textit{main} is case sensitive
(even in case-insensitive file systems)
due to internal string comparison.
\item
The argument \textit{main} should be fully expanded, it cannot be a macro.
\item
Subdirectories and special characters should be avoided in filenames.
\item
The command |\childdocmain{|\textit{main}|}| must be followed by a whitespace.
It should not be followed immediately by another command
or by a comment mark `|%|'.
This is because the \TeX{} parser reads the token immediately following
the argument of |\childdocmain| and puts it
at the beginning of every child section;
however, a white\-space is ignored.
\end{itemize}

%%%%%%%%%%%%%%%%%%%%%%%%%%%%%%%%%%%%%%%%
\paragraph{Content of Main File.}

It is advisable to place all content in the child files included by |\include|.
Any output contained in the main file will appear in all child documents
unless suppressed manually;
it cannot be suppressed automatically by the |\includeonly| directive
and thus should normally be avoided.
A method to include some content in the main file
by means of conditional processing is described in \secref{sec:conditional}.

%%%%%%%%%%%%%%%%%%%%%%%%%%%%%%%%%%%%%%%%
\paragraph{Page Numbering.}

When only a part of the document is compiled,
the appropriate numbering of pages
(as well as other status parameters)
is determined from the |.aux| files.
The latter contain information from previous passes.
However this information needs to propagate through
all intermediate child documents.
Therefore the page numbering in child documents may well
be inconsistent until the complete document is compiled at least once.

A useful (if unconventional) way to always ensure a consistent
page numbering is to restart the numbering in each child document
and denote the pages by `\textit{child}|.|\textit{page}'
where \textit{child} represents the chapter/section number of the child file.
This can be achieved by the command
|\numberwithin{page}{|\textit{child}|}|
of the \textsf{amsmath} package
where \textit{child} can be |chapter| or |section|
depending on the chosen structuring.
Alternatively, one can modify the macro |\thepage| appropriately
and reset the counter |page| at the start of each child file.

%%%%%%%%%%%%%%%%%%%%%%%%%%%%%%%%%%%%%%%%%%%%%%%%%%%%%%%%%%%%%%%%%%%%%%%%%%%%%%%%
\subsection{Conditional Processing}
\label{sec:conditional}

The package provides a mechanism to compile different versions
of a document. To customise the versions further some conditional processing
can come in handy to distinguish which version is being compiled.
The package provides two macros to describe the compilation context:

%%%%%%%%%%%%%%%%%%%%%%%%%%%%%%%%%%%%%%%%
\DescribeMacro{\ifchilddoc}
The conditional |\ifchilddoc| distinguishes between the compilation of
child documents and the main document:
%
\begin{center}
|\ifchilddoc |\textit{child-code}| |[|\||else |\textit{main-code}]| \||fi|
\end{center}

%%%%%%%%%%%%%%%%%%%%%%%%%%%%%%%%%%%%%%%%
\DescribeMacro{\childdocname}
\DescribeMacro{\childdocjob}
The macro |\childdocname| contains the filename (without extension)
of the main or child file being processed.
Note that |\childdocjob| will always contain the name of the main file.

%%%%%%%%%%%%%%%%%%%%%%%%%%%%%%%%%%%%%%%%
\paragraph{Title Page.}

Conditional processing can be used to include a title or banner page
in the main document when proper precautions are taken.
Importantly, the code in the main file should ensure that the page counter
(as well as other status parameters which are stored in the |.aux| files)
takes the same value after the conditional processing.
Otherwise the page numbers may take divergent values
depending on which part is compiled.

For example, a title page could be declared by:
%
\begin{center}
\begin{tabular}{l}
|\ifchilddoc\||else|\\
|\addtocounter{page}{-1}|\\
\textit{code for title page}\\
|\newpage|\\
|\||fi|
\end{tabular}
\end{center}
%
A banner page for the child documents can be generated by:
%
\begin{center}
\begin{tabular}{l}
|\ifchilddoc|\\
|\addtocounter{page}{-1}|\\
\textit{code for banner page}\\
|\newpage|\\
|\||fi|
\end{tabular}
\end{center}
%
Here one could write a message such as:
\begin{center}
|This is the part \childdocname{} of \childdocjob{}.|
\end{center}

%%%%%%%%%%%%%%%%%%%%%%%%%%%%%%%%%%%%%%%%%%%%%%%%%%%%%%%%%%%%%%%%%%%%%%%%%%%%%%%%
\subsection{Flags}
\label{sec:flags}

The package makes it easy to generate different versions
of the main or child documents.
To this end compilation flags can be defined
and assigned different default values.
They will be particularly useful in conjunction
with the forwarding mechanism described in \secref{sec:forward}.

For example, it may be useful to have a flag |\version|
which can be set to |draft| or |final|.
The document source will contain some conditional code
depending on the value of |\version|.
Suppose further, the flag should default to |final| for the main file
and to |draft| for child files
which is a natural assignment for editing the document.
This is achieved by placing the following code
in the preamble of the main document
(below the |\childdocmain| directive):
%
\begin{center}
\begin{tabular}{l}
|\ifchilddoc|\\
|\providecommand{\version}{draft}|\\
|\||else|\\
|\providecommand{\version}{final}|\\
|\||fi|
\end{tabular}
\end{center}
%
The definition by |\providecommand| makes sure
that previous definitions are not overwritten.
Further statements |\providecommand{\version}{...}|
can thus be added before the above code to override it.

For the main file, one might add a line
(between |\childdocmain| and the above block)
%
\begin{center}
|%\ifchilddoc\||else\providecommand{\version}{draft}\||fi|
\end{center}
%
which can be uncommented to produce a draft version.
Likewise one can add a line to the very top of a child file
(above the |\childdocof{|\textit{main}|}| directive)
%
\begin{center}
|%\providecommand{\version}{final}|
\end{center}
%
which can be uncommented to produce the final version of this child document.

%%%%%%%%%%%%%%%%%%%%%%%%%%%%%%%%%%%%%%%%%%%%%%%%%%%%%%%%%%%%%%%%%%%%%%%%%%%%%%%%
\subsection{Forwarding}
\label{sec:forward}

Different versions of the main or child documents
using compilation flags as described in \secref{sec:flags}
can be (permanently) stored in different files
for convenient compilation, viewing and distribution.
To this end, the package defines a command
to pass on compilation to a different file:

%%%%%%%%%%%%%%%%%%%%%%%%%%%%%%%%%%%%%%%%
\DescribeMacro{\childdocforward}
The command |\childdocforward| redirects processing to
another source file:
%
\begin{center}
\begin{tabular}{l}
|\input{childdoc.def}|\\
|\childdocforward[|\textit{main}|]{|\textit{dest}|}|\\
\end{tabular}
\end{center}
%
The argument \textit{dest} is the destination file
(without extension).
It should be the main file or one of the child files.
Note that further \textsf{childdoc} directives
such as |\childdocof| and |\childdocforward|
in the indicated file will be processed in this form.
The optional argument \textit{main}
passes on directly to the main file \textit{main}
while pretending to compile the child \textit{dest}.
This form behaves as if \textit{dest}
issues |\childdocof{|\textit{main}|}| right away,
and no further \textsf{childdoc} directives will be processed.

%%%%%%%%%%%%%%%%%%%%%%%%%%%%%%%%%%%%%%%%
\DescribeMacro{\...prefix}
In the alternative form |\childdocforwardprefix|,
%
\begin{center}
\begin{tabular}{l}
|\input{childdoc.def}|\\
|\childdocforwardprefix[|\textit{main}|]{|\textit{prefix}|}{|\textit{dest}|}|
\end{tabular}
\end{center}
%
the destination file is determined by a pattern
depending on the current file:
To make this work, the current file must be called
`{\textit{prefix}\hspace{0.2em}\textit{suffix}}'
with \textit{prefix} matching precisely the argument.
Processing is then passed on to the file
`{\textit{dest}\hspace{0.2em}\textit{suffix}}'.
Surely, the same effect is achieved by
directly specifying the
argument `{\textit{dest}\hspace{0.2em}\textit{suffix}}'
in the first form.
However, that requires to set up a different file
for each child. With the alternative form of the command
all these files can have exactly the same content
which simplifies setting them up and maintaining them.

For example, the following file |draft.tex|
with a compilation flag |\version| as described in \secref{sec:flags}
compiles the main document as a draft:
%
\begin{center}
\begin{tabular}{l}
|\def\version{draft}|\\
|\input{childdoc.def}|\\
|\childdocforward{|\textit{main}|}|
\end{tabular}
\end{center}
%
Likewise, the following files |final|\textit{nn}|.tex|
compile the final version of the child document
|child|\textit{nn}|.tex|:
%
\begin{center}
\begin{tabular}{l}
|\def\version{final}|\\
|\input{childdoc.def}|\\
|\childdocforwardprefix{final}{child}|
\end{tabular}
\end{center}
%

Note that when several versions of a main file and/or of each child file
are to be generated, it may be convenient to set up a |Makefile| or
shell script to automatise the process.

%%%%%%%%%%%%%%%%%%%%%%%%%%%%%%%%%%%%%%%%%%%%%%%%%%%%%%%%%%%%%%%%%%%%%%%%%%%%%%%%
\subsection{Command Line Processing}
\label{sec:commandline}

The effect of redirection files can also be achieved by invoking
the \LaTeX{} compiler with a more elaborate command line.
Most conveniently this should be done as part
of a shell script or a |Makefile|.

When using \textsf{childdoc} in the main file, the following
command lines effectively perform a redirection
(note that depending on the shell being used,
backslashes may have to be doubled: `|\|' $\to$ `|\\|'):
%
\begin{center}
|... -jobname "|\textit{target}|" |\\|"|[\textit{flags}]%
|\input{childdoc.def}\childdocforward[|\textit{main}|]{|\textit{dest}|}"|
\end{center}
%
Here \textit{target} is the name of the output file,
\textit{main} is the name of the main file
and \textit{dest} is the name of the main or child file to be processed
(all filenames without extensions).
The optional argument \textit{main} can be omitted
if \textit{main} matches \textit{dest}.
Optionally, compilation \textit{flags} can be defined via |\def| commands.
This command line makes the \TeX{} engine believe
it is compiling the file \textit{target}
whose content is specified as the latter parameter.
The provided code then forwards the processing to
\textit{main} or \textit{dest} as described in \secref{sec:forward}.

%%%%%%%%%%%%%%%%%%%%%%%%%%%%%%%%%%%%%%%%%%%%%%%%%%%%%%%%%%%%%%%%%%%%%%%%%%%%%%%%
\subsection{Include by Input}
\label{sec:input}

Including child documents by |\include| has some restrictions by design.
Most notably, the content of a child document always occupies
its own set of pages; pages cannot be shared between child documents.
Usually, this behaviour makes perfect sense
because each child document contain an essential part of the document.
However, in some situations it may be desirable to compose
a document from a collection of parts
without having mandatory page breaks between then.
For this case, the package
provides a mechanism to include parts
by |\input| which can also be processed individually.
However, by construction this mechanism
requires manual handling of the content to be output.

%%%%%%%%%%%%%%%%%%%%%%%%%%%%%%%%%%%%%%%%
\DescribeMacro{\ifchilddocmanual}
The main file should be prepared as usual, see \secref{sec:include}.
However, the document body must make a distinction
between processing of an individual part and of the main document, e.g.:
%
\begin{center}
\begin{tabular}{l}
|\ifchilddocmanual|\\
|\input{\childdocname}|\\
|\||else|\\
\textit{document body with }|\input{|\textit{part}|}|\\
|\||fi|
\end{tabular}
\end{center}
%
The conditional |\ifchilddocmanual| is true whenever
a part to be included by |\input| is being compiled,
and the name of the part is stored in |\childdocname|.

%%%%%%%%%%%%%%%%%%%%%%%%%%%%%%%%%%%%%%%%
\DescribeMacro{\childdocby}
Each part to be included by |\input| should start with:
%
\begin{center}
\begin{tabular}{l}
|\input{childdoc.def}|\\
|\childdocby{|\textit{main}|}|\\
\end{tabular}
\end{center}
%
The directive |\childdocby| is similar to |\childdocof|
described in \secref{sec:include},
but the subsequent selection of content must be done manually.
To that end, both |\ifchilddoc| and |\ifchilddocmanual|
will be true upon processing of a part,
and the name of the part is stored in |\childdocname|.
Note that |\jobname| will be set to the filename of the current part
so that each part receives an individual |.aux| file
that does not interfere with the |.aux| file(s) of the main document.
This behaviour can be altered by the alternative form
|\childdocby[*]{|\textit{main}|}| (with a non-empty optional argument)
which uses the |.aux| file of the main document
by setting |\jobname| to \textit{main}.

%%%%%%%%%%%%%%%%%%%%%%%%%%%%%%%%%%%%%%%%%%%%%%%%%%%%%%%%%%%%%%%%%%%%%%%%%%%%%%%%
\subsection{Driver Development}
\label{sec:driver}

The \textsf{childdoc} mechanism can also be use for the development
of definition files such as \LaTeX{} styles or classes.
This case differs from the above setup with multiple parts
included by |\include| in that no |\includeonly| should be invoked.
This can be achieved by starting the include file
(before |\ProvidesPackage|) with:
%
\begin{center}
\begin{tabular}{l}
|\input{childdoc.def}|\\
|\childdocforward{|\textit{main}|}|\\
\end{tabular}
\end{center}
%
or alternatively with:
%
\begin{center}
\begin{tabular}{l}
|\input{childdoc.def}|\\
|\childdocby{|\textit{main}|}|\\
\end{tabular}
\end{center}
%
Both forms have slightly different effects as described above.
The main file is prepared as usual, see \secref{sec:include}.

%%%%%%%%%%%%%%%%%%%%%%%%%%%%%%%%%%%%%%%%%%%%%%%%%%%%%%%%%%%%%%%%%%%%%%%%%%%%%%%%
\subsection{Legacy Detection}
\label{sec:detection}

The directive |\childdocmain| in the main file can detect
whether the complete document or merely a child is to be compiled
even without using the directive |\childdocof|.
This method is deprecated because it is less robust
and there is no compelling reason to use it;
it is merely provided for backward compatibility
and it may be removed in future versions.

If the detection mechanism is to be used,
it is mandatory to correctly specify
the filename of the main file as the argument of |\childdocmain|:
%
\begin{center}
\begin{tabular}{l}
|\input{childdoc.def}|\\
|\childdocmain{|\textit{main}|}|\\
\end{tabular}
\end{center}
%
If |\jobname| does not match the argument \textit{main} of |\childdocmain|,
it is assumed that |\jobname| points to the child file to be compiled.
When using |\childdocmain| with the main file specified as argument,
it suffices to start a child file
with just |\input{|\textit{main}|}|
without loading of the package and using |\childdocof|.
If instead all processing is done
with the appropriate \textsf{childdoc} directives,
the argument of \textit{main} of |\childdocmain| can be empty.

An alternative version of the command line processing described
in \secref{sec:commandline} using the detection mechanism reads:
%
\begin{center}
|... -jobname "|\textit{target}|" "|[\textit{flags}]%
[|\def\jobname{|\textit{dest}|}|]|\input{|\textit{main}|}"|
\end{center}

%%%%%%%%%%%%%%%%%%%%%%%%%%%%%%%%%%%%%%%%%%%%%%%%%%%%%%%%%%%%%%%%%%%%%%%%%%%%%%%%
\subsection{Manual Code}
\label{sec:manual}

In case one cannot be certain whether the definitions file |childdoc.def|
is installed on the target \TeX{} distribution
and one prefers not to ship it,
it is conceivable to paste a few relevant commands into the sources.

To that end, drop all statements |\input{childdoc.def}|
and perform the replacements as outlined below.
Instead of |\childdocmain{|\textit{main}|}| add the following code
to the top of the main file:
%
\begin{center}
\begin{tabular}{l}
|\||ifdefined\childdocname\endinput\||fi\newif\ifchilddoc|\\
|\edef\childdocname{\scantokens\expandafter{\jobname\noexpand}}|\\
|\def\childdocmain{|\textit{main}|}\||ifx\childdocmain\childdocname\||else|\\
|\childdoctrue\includeonly{\childdocname}\let\jobname\childdocmain\||fi|\\
\end{tabular}
\end{center}
%
Instead of |\childdocof{|\textit{main}|}| just include the main file
at the top of each child file:
%
\begin{center}
|\input{|\textit{main}|}|
\end{center}
%
A simple redirection |\childdocforward{|\textit{dest}|}| is achieved by:
%
\begin{center}
|\def\jobname{|\textit{dest}|}\input{\jobname}|
\end{center}
%
The redirection with prefix
|\childdocforwardprefix[|\textit{prefix}|]{|\textit{dest}|}|
is accomplished by:
%
\begin{center}
\begin{tabular}{l}
|{\edef\jobname{\scantokens\expandafter{\jobname\noexpand}}|\\
|\def\redirectjob |\textit{prefix}|#1~~~{\gdef\jobname{|\textit{dest}|#1}}|\\
|\expandafter\redirectjob\jobname~~~}\input{\jobname}|
\end{tabular}
\end{center}

In an alternative approach,
child documents can be compiled by a specific command line
without additional code or specific definitions:
%
\begin{center}
|... -jobname "|\textit{target}|" "|[\textit{flags}]%
|\includeonly{|\textit{dest}|}\input{|\textit{main}|}"|
\end{center}
%

%%%%%%%%%%%%%%%%%%%%%%%%%%%%%%%%%%%%%%%%%%%%%%%%%%%%%%%%%%%%%%%%%%%%%%%%%%%%%%%%
%%%%%%%%%%%%%%%%%%%%%%%%%%%%%%%%%%%%%%%%%%%%%%%%%%%%%%%%%%%%%%%%%%%%%%%%%%%%%%%%
\section{Information}

%%%%%%%%%%%%%%%%%%%%%%%%%%%%%%%%%%%%%%%%%%%%%%%%%%%%%%%%%%%%%%%%%%%%%%%%%%%%%%%%
\subsection{Copyright}

Copyright \copyright{} 2017--2018 Niklas Beisert

This work may be distributed and/or modified under the
conditions of the \LaTeX{} Project Public License, either version 1.3
of this license or (at your option) any later version.
The latest version of this license is in
  \url{http://www.latex-project.org/lppl.txt}
and version 1.3 or later is part of all distributions of \LaTeX{}
version 2005/12/01 or later.

This work has the LPPL maintenance status `maintained'.

The Current Maintainer of this work is Niklas Beisert.

This work consists of the files |README.txt|, |childdoc.ins| and |childdoc.dtx|
as well as the derived files |childdoc.def|, |cdocsamp.tex|
with |cdocsch1.tex|, |cdocsch2.tex|, |cdocspt3.tex|, |cdocspt4.tex|,
|cdocsdrf.tex|, |cdocsfn1.tex|, |cdocsfn2.tex|
as well as |childdoc.pdf|.

%%%%%%%%%%%%%%%%%%%%%%%%%%%%%%%%%%%%%%%%%%%%%%%%%%%%%%%%%%%%%%%%%%%%%%%%%%%%%%%%
\subsection{Files and Installation}

The package consists of the files:
%
\begin{center}
\begin{tabular}{ll}
    |README.txt|   & readme file \\
    |childdoc.ins| & installation file \\
    |childdoc.dtx| & source file \\
    |childdoc.def| & definition file \\
    |cdocsamp.tex| & sample main file \\
    |cdocsch1.tex| & sample include file \\
    |cdocsch2.tex| & sample include file \\
    |cdocspt3.tex| & sample part file \\
    |cdocspt4.tex| & sample part file \\
    |cdocsdrf.tex| & sample redirection file \\
    |cdocsfn1.tex| & sample redirection file \\
    |cdocsfn2.tex| & sample redirection file \\
    |childdoc.pdf| & manual
\end{tabular}
\end{center}
%
The distribution consists of the files
|README.txt|, |childdoc.ins| and |childdoc.dtx|.
%
\begin{itemize}
\item
Run (pdf)\LaTeX{} on |childdoc.dtx|
to compile the manual |childdoc.pdf| (this file).
\item
Run \LaTeX{} on |childdoc.ins| to create the definitions file |childdoc.def|
and the sample |cdocsamp.tex| with include files
|cdocsch1.tex|, |cdocsch2.tex|, |cdocspt3.tex|, |cdocspt4.tex|,
|cdocsdrf.tex|, |cdocsfn1.tex|, |cdocsfn2.tex|.
Then copy the file |childdoc.def| to an appropriate directory of your \LaTeX{}
distribution, e.g.\ \textit{texmf-root}|/tex/latex/childdoc|.
\end{itemize}

%%%%%%%%%%%%%%%%%%%%%%%%%%%%%%%%%%%%%%%%%%%%%%%%%%%%%%%%%%%%%%%%%%%%%%%%%%%%%%%%
\subsection{Related CTAN Packages}

There are several other packages which offer a similar functionality:
%
\begin{itemize}
\item
The packages
\href{http://ctan.org/pkg/docmute}{\textsf{docmute}},
\href{http://ctan.org/pkg/includex}{\textsf{includex}} and
\href{http://ctan.org/pkg/standalone}{\textsf{standalone}}
provide commands to include only the document body of
a child file thus allowing both files to be compiled individually.
\item
The packages \href{http://ctan.org/pkg/subdocs}{\textsf{subdocs}}
and \href{http://ctan.org/pkg/subfiles}{\textsf{subfiles}}
provide structures in which the main and child documents can be
encapsulated and allowing them to be compiled individually.
The inclusion mechanism is different from the conventional |\include|.
\item
The package \href{http://ctan.org/pkg/combine}{\textsf{combine}}
is an elaborate solution to combine several documents into one.
\end{itemize}
%
See also the CTAN topic \href{http://ctan.org/topic/subdocs}{\textsf{subdocs}}
for further related packages.
The present package differs from the above solutions in that
a document structure constructed with the conventional |\include| mechanism
just needs two extra commands at the top of every file
such that all constituent files can be compiled individually.

%%%%%%%%%%%%%%%%%%%%%%%%%%%%%%%%%%%%%%%%%%%%%%%%%%%%%%%%%%%%%%%%%%%%%%%%%%%%%%%%
%\subsection{Feature Suggestions}
%
%The following is a list of features which may be useful for future
%versions of this package:
%%
%\begin{itemize}
%\item
%\ldots
%\end{itemize}

%%%%%%%%%%%%%%%%%%%%%%%%%%%%%%%%%%%%%%%%%%%%%%%%%%%%%%%%%%%%%%%%%%%%%%%%%%%%%%%%
\subsection{Revision History}

%%%%%%%%%%%%%%%%%%%%%%%%%%%%%%%%%%%%%%%%
\paragraph{v2.0:} 2018/12/30

\begin{itemize}
\item
immediate forward processing
\item
added |\childdocby| mechanism
\item
manual restructured
\end{itemize}

%%%%%%%%%%%%%%%%%%%%%%%%%%%%%%%%%%%%%%%%
\paragraph{v1.6:} 2018/01/17

\begin{itemize}
\item
application for development of include files
\item
corrections to manual
\end{itemize}

%%%%%%%%%%%%%%%%%%%%%%%%%%%%%%%%%%%%%%%%
\paragraph{v1.5:} 2017/05/21

\begin{itemize}
\item
more complete structuring introduced
\item
|\childdocof| introduced
\item
|\childdoc| renamed to |\childdocmain|
\item
|\childredirect| renamed to |\childdocforward| and |\childdocforwardprefix|
and functionality expanded
\end{itemize}

%%%%%%%%%%%%%%%%%%%%%%%%%%%%%%%%%%%%%%%%
\paragraph{v1.0:} 2017/04/27

\begin{itemize}
\item
manual and install package
\item
first version published on CTAN
\end{itemize}

%%%%%%%%%%%%%%%%%%%%%%%%%%%%%%%%%%%%%%%%
\paragraph{v0.6:} 2017/04/26

\begin{itemize}
\item
redirection mechanism added
\end{itemize}

%%%%%%%%%%%%%%%%%%%%%%%%%%%%%%%%%%%%%%%%
\paragraph{v0.5:} 2017/04/26

\begin{itemize}
\item
functionality in definition file
\end{itemize}


%%%%%%%%%%%%%%%%%%%%%%%%%%%%%%%%%%%%%%%%%%%%%%%%%%%%%%%%%%%%%%%%%%%%%%%%%%%%%%%%
%%%%%%%%%%%%%%%%%%%%%%%%%%%%%%%%%%%%%%%%%%%%%%%%%%%%%%%%%%%%%%%%%%%%%%%%%%%%%%%%
%%%%%%%%%%%%%%%%%%%%%%%%%%%%%%%%%%%%%%%%%%%%%%%%%%%%%%%%%%%%%%%%%%%%%%%%%%%%%%%%
\appendix

\settowidth\MacroIndent{\rmfamily\scriptsize 000\ }

 \DocInput{childdoc.dtx}

\end{document}
%</driver>
% \fi
%
% %%%%%%%%%%%%%%%%%%%%%%%%%%%%%%%%%%%%%%%%%%%%%%%%%%%%%%%%%%%%%%%%%%%%%%%%%%%%%%
% %%%%%%%%%%%%%%%%%%%%%%%%%%%%%%%%%%%%%%%%%%%%%%%%%%%%%%%%%%%%%%%%%%%%%%%%%%%%%%
% \section{Sample}
%\iffalse
%<*samplemain>
%\fi
%
% The following presents a sample document
% with two chapters, two parts, a title page,
% a compile flag as well as three forwarding files to set the flag.
% It consists of eight |.tex| files:
% \begin{center}
% \begin{tabular}{ll}
% |cdocsamp.tex|&main file\\
% |cdocsch1.tex|&include file for chapter 1\\
% |cdocsch2.tex|&include file for chapter 2\\
% |cdocspt3.tex|&include file for part 3\\
% |cdocspt4.tex|&include file for part 4\\
% |cdocsdrf.tex|&forwarding file for main file in draft mode\\
% |cdocsfi1.tex|&forwarding file for final version of chapter 1\\
% |cdocsfi2.tex|&forwarding file for final version of chapter 2\\
% \end{tabular}
% \end{center}
% Each of the eight files can be compiled directly by the \LaTeX{} compiler.
%
% %%%%%%%%%%%%%%%%%%%%%%%%%%%%%%%%%%%%%%
% \paragraph{Main File.}
%
% The main file is called |cdocsamp.tex|.
%
% Load the \textsf{childdoc} definitions and
% declare the filename for the main document:
%    \begin{macrocode}
\input{childdoc.def}
\childdocmain{}
%    \end{macrocode}

% Optional override for |\version| flag:
%    \begin{macrocode}
%%\ifchilddoc\else\providecommand{\version}{draft}\fi
%    \end{macrocode}

% Define the default values for the |\version| flag
% (|final| for the main file and |draft| for childs):
%    \begin{macrocode}
\ifchilddoc
\providecommand{\version}{draft}
\else
\providecommand{\version}{final}
\fi
%    \end{macrocode}

% Load the standard document class:
%    \begin{macrocode}
\documentclass[12pt]{article}
%    \end{macrocode}

% Start the document body:
%    \begin{macrocode}
\begin{document}
%    \end{macrocode}

% Declare a title page.
% Print title, part of document being processed and version flag:
%    \begin{macrocode}
\addtocounter{page}{-1}
\begin{center}
{\LARGE\bfseries{}childdoc example\par}
\vspace{1cm}
\ifchilddoc
\ifchilddocmanual part\else chapter\fi:
`\childdocname' of `\childdocjob'\par
\else
main document: `\childdocjob'\par
\fi
version: \version\par
\end{center}
\newpage
%    \end{macrocode}

% Manually include selected file,
% otherwise process as usual:
%    \begin{macrocode}
\ifchilddocmanual
\section*{part `\childdocname'}
\input{\childdocname}
\else
%    \end{macrocode}

% Include the two chapters:
%    \begin{macrocode}
\include{cdocsch1}
\include{cdocsch2}
%    \end{macrocode}

% Include the two parts unless only chapters should be displayed:
%    \begin{macrocode}
\ifchilddoc\else
\section{part three}
\input{cdocspt3}
\section{part four}
\input{cdocspt4}
\fi
%    \end{macrocode}

% Process as usual until here:
%    \begin{macrocode}
\fi
%    \end{macrocode}

% End of document body:
%    \begin{macrocode}
\end{document}
%    \end{macrocode}
%\iffalse
%</samplemain>
%\fi
%
% %%%%%%%%%%%%%%%%%%%%%%%%%%%%%%%%%%%%%%
% \paragraph{Chapter Include Files.}
%
% The include files are called |cdocsch1.tex| and |cdocsch2.tex|.
%
%\iffalse
%<*samplechap1|samplechap2>
%\fi

% Optional override for |\version| flag:
%    \begin{macrocode}
%%\providecommand{\version}{final}
%    \end{macrocode}

% Include the main document:
%    \begin{macrocode}
\input{childdoc.def}
\childdocof{cdocsamp}
%    \end{macrocode}

%\iffalse
%</samplechap1|samplechap2>
%\fi
%
%\iffalse
%<*samplechap1>
%\fi
% Some text for chapter 1:
%    \begin{macrocode}
\section{one}
some text in chapter one
%    \end{macrocode}

%\iffalse
%</samplechap1>
%\fi
% Some text for chapter 2:
%\iffalse
%<*samplechap2>
%\fi
%    \begin{macrocode}
\section{two}
more text in chapter two
%    \end{macrocode}

%\iffalse
%</samplechap2>
%\fi
%
% %%%%%%%%%%%%%%%%%%%%%%%%%%%%%%%%%%%%%%
% \paragraph{Part Include Files.}
%
% The include files are called |cdocspt3.tex| and |cdocspt4.tex|.
%
%\iffalse
%<*samplepart3|samplepart4>
%\fi

% Optional override for |\version| flag:
%    \begin{macrocode}
%%\providecommand{\version}{final}
%    \end{macrocode}

% Include the main document:
%    \begin{macrocode}
\input{childdoc.def}
\childdocby{cdocsamp}
%    \end{macrocode}

%\iffalse
%</samplepart3|samplepart4>
%\fi
%
%\iffalse
%<*samplepart3>
%\fi
% Some text for part 3:
%    \begin{macrocode}
some text in part three
%    \end{macrocode}

%\iffalse
%</samplepart3>
%\fi
% Some text for part 4:
%\iffalse
%<*samplepart4>
%\fi
%    \begin{macrocode}
more text in part four
%    \end{macrocode}

%\iffalse
%</samplepart4>
%\fi
%
% %%%%%%%%%%%%%%%%%%%%%%%%%%%%%%%%%%%%%%
% \paragraph{Forwarding for a Complete Draft.}
%
% The following forwarding file |cdocsdrf.tex|
% compiles the main document in draft mode:
%\iffalse
%<*sampledraft>
%\fi
%    \begin{macrocode}
\def\version{draft}
\input{childdoc.def}
\childdocforward{cdocsamp}
%    \end{macrocode}

%\iffalse
%</sampledraft>
%\fi
%
% %%%%%%%%%%%%%%%%%%%%%%%%%%%%%%%%%%%%%%
% \paragraph{Forwarding for Final Version of the Chapters.}
%
% The following forwarding files |cdocsfn1.tex| and |cdocsfn2.tex|
% (with identical content)
% compile the final versions of the child documents
% |cdocsch1.tex| and |cdocsch2.tex|, respectively:
%\iffalse
%<*samplefinal>
%\fi
%    \begin{macrocode}
\def\version{final}
\input{childdoc.def}
\childdocforwardprefix[cdocsamp]{cdocsfn}{cdocsch}
%    \end{macrocode}

%\iffalse
%</samplefinal>
%\fi
%
% %%%%%%%%%%%%%%%%%%%%%%%%%%%%%%%%%%%%%%
% \paragraph{Command Line Processing.}
%
% The following three command lines generate the output files
% |cdocscld|, |cdocscl1| and |cdocscl2|
% which should be identical to
% |cdocsdrf|, |cdocsch1| and |cdocsfn2|, respectively:
% \begin{center}
% \begin{tabular}{l}
% |latex -jobname cdocscld \|\\
% |  "\def\version{draft}\input{childdoc.def}\childdocforward{cdocsamp}"|\\
% |latex -jobname cdocscl1 \|\\
% |  "\input{childdoc.def}\childdocforward[cdocsamp]{cdocsch1}"|\\
% |latex -jobname cdocscl2 \|\\
% |  "\def\version{final}\input{childdoc.def}\childdocforward{cdocsch2}"|
% \end{tabular}
% \end{center}
% Note that the trailing backslash on each first line
% merely continues the input to the second line
% (for convenient cut ant paste).
% Furthermore, the command |latex| can be replaced by any
% of its alternative versions such as |pdflatex|.
%
% %%%%%%%%%%%%%%%%%%%%%%%%%%%%%%%%%%%%%%%%%%%%%%%%%%%%%%%%%%%%%%%%%%%%%%%%%%%%%%
% %%%%%%%%%%%%%%%%%%%%%%%%%%%%%%%%%%%%%%%%%%%%%%%%%%%%%%%%%%%%%%%%%%%%%%%%%%%%%%
% \section{Implementation}
%\iffalse
%<*package>
%\fi
%
% This section describes the definitions file |childdoc.def|.

% The definitions cannot be loaded using |\usepackage| or |\RequirePackage|
% which has a mechanism to prevent loading a style file more than once.
% When loading the definitions by means of |\input|
% multiple instances have to be prevented manually:
%\iffalse
%This code needs to be before the `\ProvidesFile' directive
%which is defined at the beginning of this file.
%Therefore it is also placed there and commented out here.
%</package>
%<*discard>
%\fi
%    \begin{macrocode}
\ifdefined\childdocmain\endinput\fi
%    \end{macrocode}
%\iffalse
%</discard>
%<*package>
%\fi
%
% \macro{\ifchilddoc}
% \macro{\ifchilddocmanual}
% The conditional |\ifchilddoc| tells whether a
% child (true) or main (false) document is being compiled.
% The conditional |\ifchilddocmanual| tells whether
% the |\includeonly| mechanism is used (false) or
% the selection of child files must be performed manually (true).
% The definitions initialise to false:
%    \begin{macrocode}
\newif\ifchilddoc
\newif\ifchilddocmanual
%    \end{macrocode}

% \macro{\childdocname}
% \macro{\childdocjob}
% The macro |\childdocname| stores the name of the main document
% to be compiled. The macro |\childdocjob| stores the name of
% the document on which the \LaTeX{} compiler was originally invoked.
% The content of |\jobname| cannot be compared
% to filenames specified in the source due to different catcodes.
% The following code rescans |\jobname|, stores the result
% in |\childdocname| and saves a copy in |\childdocjob|:
%    \begin{macrocode}
\edef\childdocname{\scantokens\expandafter{\jobname\noexpand}}
\let\childdocjob\childdocname
%    \end{macrocode}

% \macro{\childdocdisable}
% The macro |\childdocdisable| prevents the main file
% from being processed more than once.
% At this stage, the main document command |\childdocmain|
% is assumed to be called once again where it should do nothing.
% Any subsequent call to it should prevent
% a secondary processing of the main document
% It overwrites the forwarding commands
% |\childdocof| and |\childdocforward|
% with empty macros to prevent further inclusions of the main document:
%    \begin{macrocode}
\newcommand{\childdocdisable}
{
  \renewcommand{\childdocmain}[1]{\renewcommand{\childdocmain}[1]{\endinput}}
  \renewcommand{\childdocof}[1]{}
  \renewcommand{\childdocby}[2][]{}
  \renewcommand{\childdocforward}[2][]{}
  \renewcommand{\childdocdisable}{}
}
%    \end{macrocode}

% \macro{\childdocmain}
% The macro |\childdocmain| is to be called at the top of the main file
% with nothing or the main filename (without extension) as argument.
% First, it breaks loops.
% If the argument is not empty and does not match |\childdocname|
% (which is set by the first inclusion of |childdoc.def|),
% |\ifchilddoc| is set to true, |\includeonly| is applied to the child file
% and |\jobname| is set to the main file
% (for proper handling of |.aux| files):
%    \begin{macrocode}
\newcommand{\childdocmain}[1]
{
  \childdocdisable\childdocmain{}
  \if?#1?\else
    \begingroup
      \def\childdoctmp{#1}
      \ifx\childdoctmp\childdocname
        \def\childdoctmp{}
      \else
        \def\childdoctmp
        {
          \childdoctrue
          \includeonly{\childdocname}
          \def\childdocjob{#1}
          \def\jobname{#1}
        }
      \fi
      \expandafter
    \endgroup
    \childdoctmp
  \fi
}
%    \end{macrocode}

% \macro{\childdocof}
% The command |\childdocof| redirects
% compilation to the main file |#1|.
%    \begin{macrocode}
\newcommand{\childdocof}[1]
{
  \childdocdisable
  \childdoctrue
  \includeonly{\childdocname}
  \def\jobname{#1}
  \def\childdocjob{#1}
  \input{#1}
}
%    \end{macrocode}

% \macro{\childdocby}
% The command |\childdocby| ....
%    \begin{macrocode}
\newcommand{\childdocby}[2][]
{
  \childdocdisable
  \childdoctrue
  \childdocmanualtrue
  \if?#1?\else
    \def\jobname{#2}
  \fi
  \def\childdocjob{#2}
  \input{#2}
  \endinput
}
%    \end{macrocode}

% \macro{\childdocforward}
% The command |\childdocforward| redirects
% compilation to the main file or
% (if the optional argument is given) a child file.
% Parameters are set as if the main file
% or a child file starting with |\childdocof| was compiled.
% Then compilation is handed over to the main file:
%    \begin{macrocode}
\newcommand{\childdocforward}[2][]
{
  \begingroup
    \if?#1?
      \def\childdoctmp
      {
        \def\childdocname{#2}
        \def\childdocjob{#2}
        \def\jobname{#2}
        \input{#2}
        \endinput
      }
    \else
      \def\childdoctmp
      {
        \childdocdisable
        \def\childdocname{#2}
        \childdoctrue
        \includeonly{#2}
        \def\childdocjob{#1}
        \def\jobname{#1}
        \input{#1}
        \endinput
      }
    \fi
    \expandafter
  \endgroup
  \childdoctmp
}
%    \end{macrocode}

% \macro{\childdocforwardprefix}
% The command |\childdocforwardprefix| redirects
% compilation to the main or a child file by means of a pattern.
% The prefix |#1| in the current filename is replaced by |#2|
% and the suffix of the current filename is kept
% (it is assumed that the filename does not contain the substring `|~~~|'
% which is used as a delimiter).
% Compilation is handed over to the new file by |\childdocforward|:
%    \begin{macrocode}
\newcommand{\childdocforwardprefix}[3][]
{
  \begingroup
    \def\childdocextract #2##1~~~{\def\childdoctmp{\childdocforward[#1]{#3##1}}}
    \expandafter\childdocextract\childdocname~~~
    \expandafter
  \endgroup
  \childdoctmp
}
%    \end{macrocode}

% \macro{\childdoc}
% The deprecated macro |\childdoc| is a legacy version of |\childdocmain|:
%    \begin{macrocode}
\newcommand{\childdoc}{\childdocmain}
%    \end{macrocode}

% \macro{\childdocredirect}
% The deprecated macro |\childdocredirect| is a legacy version
% of |\childdocforward| and |\childdocforwardprefix|:
%    \begin{macrocode}
\newcommand{\childdocredirect}[2][]
{
  \begingroup
    \if?#1?
      \def\childdoctmp{\childdocforward{#2}}
    \else
      \def\childdoctmp{\childdocforwardprefix{#1}{#2}}
    \fi
    \expandafter
  \endgroup
  \childdoctmp
}
%    \end{macrocode}

%\iffalse
%</package>
%\fi
%
\endinput
|\\
|\childdocforward{|\textit{main}|}|
\end{tabular}
\end{center}
%
Likewise, the following files |final|\textit{nn}|.tex|
compile the final version of the child document
|child|\textit{nn}|.tex|:
%
\begin{center}
\begin{tabular}{l}
|\def\version{final}|\\
|% \iffalse
%
% childdoc.dtx Copyright (C) 2017-2018 Niklas Beisert
%
% This work may be distributed and/or modified under the
% conditions of the LaTeX Project Public License, either version 1.3
% of this license or (at your option) any later version.
% The latest version of this license is in
%   http://www.latex-project.org/lppl.txt
% and version 1.3 or later is part of all distributions of LaTeX
% version 2005/12/01 or later.
%
% This work has the LPPL maintenance status `maintained'.
%
% The Current Maintainer of this work is Niklas Beisert.
%
% This work consists of the files childdoc.dtx and childdoc.ins
% and the derived files childdoc.def and cdocsamp.tex with
% cdocsch1.tex, cdocsch2.tex, cdocsdrf.tex, cdocsfn1.tex, cdocsfn2.tex.
%
%<package>\ifdefined\childdocmain\endinput\fi
%<package>\ProvidesFile{childdoc.def}[2018/12/30 v2.0 child document driver]
%<samplemain>\ProvidesFile{cdocsamp.tex}[2018/12/30 v2.0 sample for childdoc]
%<*driver>
%\ProvidesFile{childdoc.drv}[2018/12/30 v2.0 childdoc reference manual file]
\PassOptionsToClass{10pt,a4paper}{article}
\documentclass{ltxdoc}

\usepackage[margin=35mm]{geometry}
\usepackage{hyperref}
\usepackage{hyperxmp}
\usepackage[usenames]{color}

\hypersetup{colorlinks=true}
\hypersetup{pdfstartview=FitH}
\hypersetup{pdfpagemode=UseNone}
\hypersetup{pdfsource={}}
\hypersetup{pdflang={en-UK}}
\hypersetup{pdfcopyright={Copyright 2017-2018 Niklas Beisert.
  This work may be distributed and/or modified under the
  conditions of the LaTeX Project Public License, either version 1.3
  of this license or (at your option) any later version.}}
\hypersetup{pdflicenseurl={http://www.latex-project.org/lppl.txt}}
\hypersetup{pdfcontactaddress={ETH Zurich, ITP, HIT K,
  Wolfgang-Pauli-Strasse 27}}
\hypersetup{pdfcontactpostcode={8093}}
\hypersetup{pdfcontactcity={Zurich}}
\hypersetup{pdfcontactcountry={Switzerland}}
\hypersetup{pdfcontactemail={nbeisert@itp.phys.ethz.ch}}
\hypersetup{pdfcontacturl={http://people.phys.ethz.ch/\xmptilde nbeisert/}}

\newcommand{\secref}[1]{\hyperref[#1]{section \ref*{#1}}}

\parskip1ex
\parindent0pt
\let\olditemize\itemize
\def\itemize{\olditemize\parskip0pt}

\begin{document}

\title{The \textsf{childdoc} Package}
\hypersetup{pdftitle={The childdoc Package}}
\author{Niklas Beisert\\[2ex]
  Institut f\"ur Theoretische Physik\\
  Eidgen\"ossische Technische Hochschule Z\"urich\\
  Wolfgang-Pauli-Strasse 27, 8093 Z\"urich, Switzerland\\[1ex]
  \href{mailto:nbeisert@itp.phys.ethz.ch}
  {\texttt{nbeisert@itp.phys.ethz.ch}}}
\hypersetup{pdfauthor={Niklas Beisert}}
\hypersetup{pdfsubject={Manual for the LaTeX2e Package childdoc}}
\date{30 December 2018, \textsf{v2.0}}
\maketitle

\begin{abstract}\noindent
\textsf{childdoc} is a \LaTeXe{} package
that enables the direct compilation
of document sections included by |\include|
to individual files.
\end{abstract}

\begingroup
\parskip0ex
\tableofcontents
\endgroup

%%%%%%%%%%%%%%%%%%%%%%%%%%%%%%%%%%%%%%%%%%%%%%%%%%%%%%%%%%%%%%%%%%%%%%%%%%%%%%%%
%%%%%%%%%%%%%%%%%%%%%%%%%%%%%%%%%%%%%%%%%%%%%%%%%%%%%%%%%%%%%%%%%%%%%%%%%%%%%%%%
\section{Introduction}

\LaTeX{} provides a mechanism to structure a large document (such as a book)
into a main file and several child files (containing the chapters)
using the |\include| command.
This mechanism is beneficial for documents
which span hundreds of pages in order to
make the source file(s) more manageable.
Moreover, compilation can be restricted to
selected child files by means of the |\includeonly| command.
The latter feature can be used to reduce the compilation time while editing
(this was significantly more useful in the earlier days of \LaTeX{})
or to generate a smaller document which is easier to navigate.
Another application of |\includeonly| is to generate
documents consisting of selected parts of the complete document.

However, there are a few drawbacks of the plain |\include| mechanism:
\begin{itemize}
\item
The child files cannot be compiled on their own,
they can only be compiled via the main file.
A naive editing environment
(such as a text editor with an option
to have the current file processed by \LaTeX)
may require one to switch to the main file before compiling;
attempting to compile the child file produces errors.
\item
The main file must be modified (each time)
to adjust the |\includeonly| command
to the present needs. This easily leaves the main file in a messy state.
\item
The generated document will always carry the filename
of the main document. This is inconvenient if
several child files are to be compiled and
to be kept for distribution.
\end{itemize}

The present package provides a simple interface
to make child files individually compilable by \LaTeX{}.
Compiling a child file then has the same effect as compiling
the main file with an |\includeonly| command
to select the appropriate child.
Moreover the generated document will carry the name of the child
rather than the main file.
This resolves all three above issues.

This feature is meant to make the editing of books,
thesis documents and lecture notes somewhat more convenient.
However, the package can also be used efficiently for
composing a series of documents (such as exercise sheets)
which are typically distributed individually.
It then assists the author in generating the individual documents
(potentially in different versions)
as well as a document containing the collected series.
Another application is in developing style files
or other kinds of included material
where compilation of the style file could redirect
to a sample or test file.

%%%%%%%%%%%%%%%%%%%%%%%%%%%%%%%%%%%%%%%%%%%%%%%%%%%%%%%%%%%%%%%%%%%%%%%%%%%%%%%%
%%%%%%%%%%%%%%%%%%%%%%%%%%%%%%%%%%%%%%%%%%%%%%%%%%%%%%%%%%%%%%%%%%%%%%%%%%%%%%%%
\section{Usage}

First of all, the package \textsf{childdoc} is \emph{not} a standard
\LaTeXe{} |.sty| style file! Therefore it needs to be invoked in
a non-standard way.

%%%%%%%%%%%%%%%%%%%%%%%%%%%%%%%%%%%%%%%%%%%%%%%%%%%%%%%%%%%%%%%%%%%%%%%%%%%%%%%%
\subsection{Included Files}
\label{sec:include}

%%%%%%%%%%%%%%%%%%%%%%%%%%%%%%%%%%%%%%%%
\DescribeMacro{\childdocmain}
To use the package, add the commands
\begin{center}
\begin{tabular}{l}
|\input{childdoc.def}|\\
|\childdocmain{}|\\
\end{tabular}
\end{center}
at the very top of the main \LaTeX{} file,
in particular \emph{before} the |\documentclass| statement!
The argument of |\childdocmain| should be left empty
(but it must be present).

%%%%%%%%%%%%%%%%%%%%%%%%%%%%%%%%%%%%%%%%
\DescribeMacro{\childdocof}
Furthermore, add the commands
\begin{center}
\begin{tabular}{l}
|\input{childdoc.def}|\\
|\childdocof{|\textit{main}|}|\\
\end{tabular}
\end{center}
at the top of every child file \textit{child}
which is included by |\include{|\textit{child}|}|
from within the main file
(or at least for those files to be compiled individually).
The argument \textit{main} must be the filename of the main file.

There are a couple of
considerations in setting up the main and child documents:

%%%%%%%%%%%%%%%%%%%%%%%%%%%%%%%%%%%%%%%%
\paragraph{Restrictions.}

Please note the following restrictions:
\begin{itemize}
\item
|\childdocmain| must be called with one argument \textit{main}
to ensure compatibility with earlier version of the package.
It must either be empty (|\childdocmain{}|)
or precisely match the filename of the main file in which it is specified.
See \secref{sec:detection} for further information.
\item
The filename \textit{main} must be specified without the |.tex| extension.
\item
The filename \textit{main} is case sensitive
(even in case-insensitive file systems)
due to internal string comparison.
\item
The argument \textit{main} should be fully expanded, it cannot be a macro.
\item
Subdirectories and special characters should be avoided in filenames.
\item
The command |\childdocmain{|\textit{main}|}| must be followed by a whitespace.
It should not be followed immediately by another command
or by a comment mark `|%|'.
This is because the \TeX{} parser reads the token immediately following
the argument of |\childdocmain| and puts it
at the beginning of every child section;
however, a white\-space is ignored.
\end{itemize}

%%%%%%%%%%%%%%%%%%%%%%%%%%%%%%%%%%%%%%%%
\paragraph{Content of Main File.}

It is advisable to place all content in the child files included by |\include|.
Any output contained in the main file will appear in all child documents
unless suppressed manually;
it cannot be suppressed automatically by the |\includeonly| directive
and thus should normally be avoided.
A method to include some content in the main file
by means of conditional processing is described in \secref{sec:conditional}.

%%%%%%%%%%%%%%%%%%%%%%%%%%%%%%%%%%%%%%%%
\paragraph{Page Numbering.}

When only a part of the document is compiled,
the appropriate numbering of pages
(as well as other status parameters)
is determined from the |.aux| files.
The latter contain information from previous passes.
However this information needs to propagate through
all intermediate child documents.
Therefore the page numbering in child documents may well
be inconsistent until the complete document is compiled at least once.

A useful (if unconventional) way to always ensure a consistent
page numbering is to restart the numbering in each child document
and denote the pages by `\textit{child}|.|\textit{page}'
where \textit{child} represents the chapter/section number of the child file.
This can be achieved by the command
|\numberwithin{page}{|\textit{child}|}|
of the \textsf{amsmath} package
where \textit{child} can be |chapter| or |section|
depending on the chosen structuring.
Alternatively, one can modify the macro |\thepage| appropriately
and reset the counter |page| at the start of each child file.

%%%%%%%%%%%%%%%%%%%%%%%%%%%%%%%%%%%%%%%%%%%%%%%%%%%%%%%%%%%%%%%%%%%%%%%%%%%%%%%%
\subsection{Conditional Processing}
\label{sec:conditional}

The package provides a mechanism to compile different versions
of a document. To customise the versions further some conditional processing
can come in handy to distinguish which version is being compiled.
The package provides two macros to describe the compilation context:

%%%%%%%%%%%%%%%%%%%%%%%%%%%%%%%%%%%%%%%%
\DescribeMacro{\ifchilddoc}
The conditional |\ifchilddoc| distinguishes between the compilation of
child documents and the main document:
%
\begin{center}
|\ifchilddoc |\textit{child-code}| |[|\||else |\textit{main-code}]| \||fi|
\end{center}

%%%%%%%%%%%%%%%%%%%%%%%%%%%%%%%%%%%%%%%%
\DescribeMacro{\childdocname}
\DescribeMacro{\childdocjob}
The macro |\childdocname| contains the filename (without extension)
of the main or child file being processed.
Note that |\childdocjob| will always contain the name of the main file.

%%%%%%%%%%%%%%%%%%%%%%%%%%%%%%%%%%%%%%%%
\paragraph{Title Page.}

Conditional processing can be used to include a title or banner page
in the main document when proper precautions are taken.
Importantly, the code in the main file should ensure that the page counter
(as well as other status parameters which are stored in the |.aux| files)
takes the same value after the conditional processing.
Otherwise the page numbers may take divergent values
depending on which part is compiled.

For example, a title page could be declared by:
%
\begin{center}
\begin{tabular}{l}
|\ifchilddoc\||else|\\
|\addtocounter{page}{-1}|\\
\textit{code for title page}\\
|\newpage|\\
|\||fi|
\end{tabular}
\end{center}
%
A banner page for the child documents can be generated by:
%
\begin{center}
\begin{tabular}{l}
|\ifchilddoc|\\
|\addtocounter{page}{-1}|\\
\textit{code for banner page}\\
|\newpage|\\
|\||fi|
\end{tabular}
\end{center}
%
Here one could write a message such as:
\begin{center}
|This is the part \childdocname{} of \childdocjob{}.|
\end{center}

%%%%%%%%%%%%%%%%%%%%%%%%%%%%%%%%%%%%%%%%%%%%%%%%%%%%%%%%%%%%%%%%%%%%%%%%%%%%%%%%
\subsection{Flags}
\label{sec:flags}

The package makes it easy to generate different versions
of the main or child documents.
To this end compilation flags can be defined
and assigned different default values.
They will be particularly useful in conjunction
with the forwarding mechanism described in \secref{sec:forward}.

For example, it may be useful to have a flag |\version|
which can be set to |draft| or |final|.
The document source will contain some conditional code
depending on the value of |\version|.
Suppose further, the flag should default to |final| for the main file
and to |draft| for child files
which is a natural assignment for editing the document.
This is achieved by placing the following code
in the preamble of the main document
(below the |\childdocmain| directive):
%
\begin{center}
\begin{tabular}{l}
|\ifchilddoc|\\
|\providecommand{\version}{draft}|\\
|\||else|\\
|\providecommand{\version}{final}|\\
|\||fi|
\end{tabular}
\end{center}
%
The definition by |\providecommand| makes sure
that previous definitions are not overwritten.
Further statements |\providecommand{\version}{...}|
can thus be added before the above code to override it.

For the main file, one might add a line
(between |\childdocmain| and the above block)
%
\begin{center}
|%\ifchilddoc\||else\providecommand{\version}{draft}\||fi|
\end{center}
%
which can be uncommented to produce a draft version.
Likewise one can add a line to the very top of a child file
(above the |\childdocof{|\textit{main}|}| directive)
%
\begin{center}
|%\providecommand{\version}{final}|
\end{center}
%
which can be uncommented to produce the final version of this child document.

%%%%%%%%%%%%%%%%%%%%%%%%%%%%%%%%%%%%%%%%%%%%%%%%%%%%%%%%%%%%%%%%%%%%%%%%%%%%%%%%
\subsection{Forwarding}
\label{sec:forward}

Different versions of the main or child documents
using compilation flags as described in \secref{sec:flags}
can be (permanently) stored in different files
for convenient compilation, viewing and distribution.
To this end, the package defines a command
to pass on compilation to a different file:

%%%%%%%%%%%%%%%%%%%%%%%%%%%%%%%%%%%%%%%%
\DescribeMacro{\childdocforward}
The command |\childdocforward| redirects processing to
another source file:
%
\begin{center}
\begin{tabular}{l}
|\input{childdoc.def}|\\
|\childdocforward[|\textit{main}|]{|\textit{dest}|}|\\
\end{tabular}
\end{center}
%
The argument \textit{dest} is the destination file
(without extension).
It should be the main file or one of the child files.
Note that further \textsf{childdoc} directives
such as |\childdocof| and |\childdocforward|
in the indicated file will be processed in this form.
The optional argument \textit{main}
passes on directly to the main file \textit{main}
while pretending to compile the child \textit{dest}.
This form behaves as if \textit{dest}
issues |\childdocof{|\textit{main}|}| right away,
and no further \textsf{childdoc} directives will be processed.

%%%%%%%%%%%%%%%%%%%%%%%%%%%%%%%%%%%%%%%%
\DescribeMacro{\...prefix}
In the alternative form |\childdocforwardprefix|,
%
\begin{center}
\begin{tabular}{l}
|\input{childdoc.def}|\\
|\childdocforwardprefix[|\textit{main}|]{|\textit{prefix}|}{|\textit{dest}|}|
\end{tabular}
\end{center}
%
the destination file is determined by a pattern
depending on the current file:
To make this work, the current file must be called
`{\textit{prefix}\hspace{0.2em}\textit{suffix}}'
with \textit{prefix} matching precisely the argument.
Processing is then passed on to the file
`{\textit{dest}\hspace{0.2em}\textit{suffix}}'.
Surely, the same effect is achieved by
directly specifying the
argument `{\textit{dest}\hspace{0.2em}\textit{suffix}}'
in the first form.
However, that requires to set up a different file
for each child. With the alternative form of the command
all these files can have exactly the same content
which simplifies setting them up and maintaining them.

For example, the following file |draft.tex|
with a compilation flag |\version| as described in \secref{sec:flags}
compiles the main document as a draft:
%
\begin{center}
\begin{tabular}{l}
|\def\version{draft}|\\
|\input{childdoc.def}|\\
|\childdocforward{|\textit{main}|}|
\end{tabular}
\end{center}
%
Likewise, the following files |final|\textit{nn}|.tex|
compile the final version of the child document
|child|\textit{nn}|.tex|:
%
\begin{center}
\begin{tabular}{l}
|\def\version{final}|\\
|\input{childdoc.def}|\\
|\childdocforwardprefix{final}{child}|
\end{tabular}
\end{center}
%

Note that when several versions of a main file and/or of each child file
are to be generated, it may be convenient to set up a |Makefile| or
shell script to automatise the process.

%%%%%%%%%%%%%%%%%%%%%%%%%%%%%%%%%%%%%%%%%%%%%%%%%%%%%%%%%%%%%%%%%%%%%%%%%%%%%%%%
\subsection{Command Line Processing}
\label{sec:commandline}

The effect of redirection files can also be achieved by invoking
the \LaTeX{} compiler with a more elaborate command line.
Most conveniently this should be done as part
of a shell script or a |Makefile|.

When using \textsf{childdoc} in the main file, the following
command lines effectively perform a redirection
(note that depending on the shell being used,
backslashes may have to be doubled: `|\|' $\to$ `|\\|'):
%
\begin{center}
|... -jobname "|\textit{target}|" |\\|"|[\textit{flags}]%
|\input{childdoc.def}\childdocforward[|\textit{main}|]{|\textit{dest}|}"|
\end{center}
%
Here \textit{target} is the name of the output file,
\textit{main} is the name of the main file
and \textit{dest} is the name of the main or child file to be processed
(all filenames without extensions).
The optional argument \textit{main} can be omitted
if \textit{main} matches \textit{dest}.
Optionally, compilation \textit{flags} can be defined via |\def| commands.
This command line makes the \TeX{} engine believe
it is compiling the file \textit{target}
whose content is specified as the latter parameter.
The provided code then forwards the processing to
\textit{main} or \textit{dest} as described in \secref{sec:forward}.

%%%%%%%%%%%%%%%%%%%%%%%%%%%%%%%%%%%%%%%%%%%%%%%%%%%%%%%%%%%%%%%%%%%%%%%%%%%%%%%%
\subsection{Include by Input}
\label{sec:input}

Including child documents by |\include| has some restrictions by design.
Most notably, the content of a child document always occupies
its own set of pages; pages cannot be shared between child documents.
Usually, this behaviour makes perfect sense
because each child document contain an essential part of the document.
However, in some situations it may be desirable to compose
a document from a collection of parts
without having mandatory page breaks between then.
For this case, the package
provides a mechanism to include parts
by |\input| which can also be processed individually.
However, by construction this mechanism
requires manual handling of the content to be output.

%%%%%%%%%%%%%%%%%%%%%%%%%%%%%%%%%%%%%%%%
\DescribeMacro{\ifchilddocmanual}
The main file should be prepared as usual, see \secref{sec:include}.
However, the document body must make a distinction
between processing of an individual part and of the main document, e.g.:
%
\begin{center}
\begin{tabular}{l}
|\ifchilddocmanual|\\
|\input{\childdocname}|\\
|\||else|\\
\textit{document body with }|\input{|\textit{part}|}|\\
|\||fi|
\end{tabular}
\end{center}
%
The conditional |\ifchilddocmanual| is true whenever
a part to be included by |\input| is being compiled,
and the name of the part is stored in |\childdocname|.

%%%%%%%%%%%%%%%%%%%%%%%%%%%%%%%%%%%%%%%%
\DescribeMacro{\childdocby}
Each part to be included by |\input| should start with:
%
\begin{center}
\begin{tabular}{l}
|\input{childdoc.def}|\\
|\childdocby{|\textit{main}|}|\\
\end{tabular}
\end{center}
%
The directive |\childdocby| is similar to |\childdocof|
described in \secref{sec:include},
but the subsequent selection of content must be done manually.
To that end, both |\ifchilddoc| and |\ifchilddocmanual|
will be true upon processing of a part,
and the name of the part is stored in |\childdocname|.
Note that |\jobname| will be set to the filename of the current part
so that each part receives an individual |.aux| file
that does not interfere with the |.aux| file(s) of the main document.
This behaviour can be altered by the alternative form
|\childdocby[*]{|\textit{main}|}| (with a non-empty optional argument)
which uses the |.aux| file of the main document
by setting |\jobname| to \textit{main}.

%%%%%%%%%%%%%%%%%%%%%%%%%%%%%%%%%%%%%%%%%%%%%%%%%%%%%%%%%%%%%%%%%%%%%%%%%%%%%%%%
\subsection{Driver Development}
\label{sec:driver}

The \textsf{childdoc} mechanism can also be use for the development
of definition files such as \LaTeX{} styles or classes.
This case differs from the above setup with multiple parts
included by |\include| in that no |\includeonly| should be invoked.
This can be achieved by starting the include file
(before |\ProvidesPackage|) with:
%
\begin{center}
\begin{tabular}{l}
|\input{childdoc.def}|\\
|\childdocforward{|\textit{main}|}|\\
\end{tabular}
\end{center}
%
or alternatively with:
%
\begin{center}
\begin{tabular}{l}
|\input{childdoc.def}|\\
|\childdocby{|\textit{main}|}|\\
\end{tabular}
\end{center}
%
Both forms have slightly different effects as described above.
The main file is prepared as usual, see \secref{sec:include}.

%%%%%%%%%%%%%%%%%%%%%%%%%%%%%%%%%%%%%%%%%%%%%%%%%%%%%%%%%%%%%%%%%%%%%%%%%%%%%%%%
\subsection{Legacy Detection}
\label{sec:detection}

The directive |\childdocmain| in the main file can detect
whether the complete document or merely a child is to be compiled
even without using the directive |\childdocof|.
This method is deprecated because it is less robust
and there is no compelling reason to use it;
it is merely provided for backward compatibility
and it may be removed in future versions.

If the detection mechanism is to be used,
it is mandatory to correctly specify
the filename of the main file as the argument of |\childdocmain|:
%
\begin{center}
\begin{tabular}{l}
|\input{childdoc.def}|\\
|\childdocmain{|\textit{main}|}|\\
\end{tabular}
\end{center}
%
If |\jobname| does not match the argument \textit{main} of |\childdocmain|,
it is assumed that |\jobname| points to the child file to be compiled.
When using |\childdocmain| with the main file specified as argument,
it suffices to start a child file
with just |\input{|\textit{main}|}|
without loading of the package and using |\childdocof|.
If instead all processing is done
with the appropriate \textsf{childdoc} directives,
the argument of \textit{main} of |\childdocmain| can be empty.

An alternative version of the command line processing described
in \secref{sec:commandline} using the detection mechanism reads:
%
\begin{center}
|... -jobname "|\textit{target}|" "|[\textit{flags}]%
[|\def\jobname{|\textit{dest}|}|]|\input{|\textit{main}|}"|
\end{center}

%%%%%%%%%%%%%%%%%%%%%%%%%%%%%%%%%%%%%%%%%%%%%%%%%%%%%%%%%%%%%%%%%%%%%%%%%%%%%%%%
\subsection{Manual Code}
\label{sec:manual}

In case one cannot be certain whether the definitions file |childdoc.def|
is installed on the target \TeX{} distribution
and one prefers not to ship it,
it is conceivable to paste a few relevant commands into the sources.

To that end, drop all statements |\input{childdoc.def}|
and perform the replacements as outlined below.
Instead of |\childdocmain{|\textit{main}|}| add the following code
to the top of the main file:
%
\begin{center}
\begin{tabular}{l}
|\||ifdefined\childdocname\endinput\||fi\newif\ifchilddoc|\\
|\edef\childdocname{\scantokens\expandafter{\jobname\noexpand}}|\\
|\def\childdocmain{|\textit{main}|}\||ifx\childdocmain\childdocname\||else|\\
|\childdoctrue\includeonly{\childdocname}\let\jobname\childdocmain\||fi|\\
\end{tabular}
\end{center}
%
Instead of |\childdocof{|\textit{main}|}| just include the main file
at the top of each child file:
%
\begin{center}
|\input{|\textit{main}|}|
\end{center}
%
A simple redirection |\childdocforward{|\textit{dest}|}| is achieved by:
%
\begin{center}
|\def\jobname{|\textit{dest}|}\input{\jobname}|
\end{center}
%
The redirection with prefix
|\childdocforwardprefix[|\textit{prefix}|]{|\textit{dest}|}|
is accomplished by:
%
\begin{center}
\begin{tabular}{l}
|{\edef\jobname{\scantokens\expandafter{\jobname\noexpand}}|\\
|\def\redirectjob |\textit{prefix}|#1~~~{\gdef\jobname{|\textit{dest}|#1}}|\\
|\expandafter\redirectjob\jobname~~~}\input{\jobname}|
\end{tabular}
\end{center}

In an alternative approach,
child documents can be compiled by a specific command line
without additional code or specific definitions:
%
\begin{center}
|... -jobname "|\textit{target}|" "|[\textit{flags}]%
|\includeonly{|\textit{dest}|}\input{|\textit{main}|}"|
\end{center}
%

%%%%%%%%%%%%%%%%%%%%%%%%%%%%%%%%%%%%%%%%%%%%%%%%%%%%%%%%%%%%%%%%%%%%%%%%%%%%%%%%
%%%%%%%%%%%%%%%%%%%%%%%%%%%%%%%%%%%%%%%%%%%%%%%%%%%%%%%%%%%%%%%%%%%%%%%%%%%%%%%%
\section{Information}

%%%%%%%%%%%%%%%%%%%%%%%%%%%%%%%%%%%%%%%%%%%%%%%%%%%%%%%%%%%%%%%%%%%%%%%%%%%%%%%%
\subsection{Copyright}

Copyright \copyright{} 2017--2018 Niklas Beisert

This work may be distributed and/or modified under the
conditions of the \LaTeX{} Project Public License, either version 1.3
of this license or (at your option) any later version.
The latest version of this license is in
  \url{http://www.latex-project.org/lppl.txt}
and version 1.3 or later is part of all distributions of \LaTeX{}
version 2005/12/01 or later.

This work has the LPPL maintenance status `maintained'.

The Current Maintainer of this work is Niklas Beisert.

This work consists of the files |README.txt|, |childdoc.ins| and |childdoc.dtx|
as well as the derived files |childdoc.def|, |cdocsamp.tex|
with |cdocsch1.tex|, |cdocsch2.tex|, |cdocspt3.tex|, |cdocspt4.tex|,
|cdocsdrf.tex|, |cdocsfn1.tex|, |cdocsfn2.tex|
as well as |childdoc.pdf|.

%%%%%%%%%%%%%%%%%%%%%%%%%%%%%%%%%%%%%%%%%%%%%%%%%%%%%%%%%%%%%%%%%%%%%%%%%%%%%%%%
\subsection{Files and Installation}

The package consists of the files:
%
\begin{center}
\begin{tabular}{ll}
    |README.txt|   & readme file \\
    |childdoc.ins| & installation file \\
    |childdoc.dtx| & source file \\
    |childdoc.def| & definition file \\
    |cdocsamp.tex| & sample main file \\
    |cdocsch1.tex| & sample include file \\
    |cdocsch2.tex| & sample include file \\
    |cdocspt3.tex| & sample part file \\
    |cdocspt4.tex| & sample part file \\
    |cdocsdrf.tex| & sample redirection file \\
    |cdocsfn1.tex| & sample redirection file \\
    |cdocsfn2.tex| & sample redirection file \\
    |childdoc.pdf| & manual
\end{tabular}
\end{center}
%
The distribution consists of the files
|README.txt|, |childdoc.ins| and |childdoc.dtx|.
%
\begin{itemize}
\item
Run (pdf)\LaTeX{} on |childdoc.dtx|
to compile the manual |childdoc.pdf| (this file).
\item
Run \LaTeX{} on |childdoc.ins| to create the definitions file |childdoc.def|
and the sample |cdocsamp.tex| with include files
|cdocsch1.tex|, |cdocsch2.tex|, |cdocspt3.tex|, |cdocspt4.tex|,
|cdocsdrf.tex|, |cdocsfn1.tex|, |cdocsfn2.tex|.
Then copy the file |childdoc.def| to an appropriate directory of your \LaTeX{}
distribution, e.g.\ \textit{texmf-root}|/tex/latex/childdoc|.
\end{itemize}

%%%%%%%%%%%%%%%%%%%%%%%%%%%%%%%%%%%%%%%%%%%%%%%%%%%%%%%%%%%%%%%%%%%%%%%%%%%%%%%%
\subsection{Related CTAN Packages}

There are several other packages which offer a similar functionality:
%
\begin{itemize}
\item
The packages
\href{http://ctan.org/pkg/docmute}{\textsf{docmute}},
\href{http://ctan.org/pkg/includex}{\textsf{includex}} and
\href{http://ctan.org/pkg/standalone}{\textsf{standalone}}
provide commands to include only the document body of
a child file thus allowing both files to be compiled individually.
\item
The packages \href{http://ctan.org/pkg/subdocs}{\textsf{subdocs}}
and \href{http://ctan.org/pkg/subfiles}{\textsf{subfiles}}
provide structures in which the main and child documents can be
encapsulated and allowing them to be compiled individually.
The inclusion mechanism is different from the conventional |\include|.
\item
The package \href{http://ctan.org/pkg/combine}{\textsf{combine}}
is an elaborate solution to combine several documents into one.
\end{itemize}
%
See also the CTAN topic \href{http://ctan.org/topic/subdocs}{\textsf{subdocs}}
for further related packages.
The present package differs from the above solutions in that
a document structure constructed with the conventional |\include| mechanism
just needs two extra commands at the top of every file
such that all constituent files can be compiled individually.

%%%%%%%%%%%%%%%%%%%%%%%%%%%%%%%%%%%%%%%%%%%%%%%%%%%%%%%%%%%%%%%%%%%%%%%%%%%%%%%%
%\subsection{Feature Suggestions}
%
%The following is a list of features which may be useful for future
%versions of this package:
%%
%\begin{itemize}
%\item
%\ldots
%\end{itemize}

%%%%%%%%%%%%%%%%%%%%%%%%%%%%%%%%%%%%%%%%%%%%%%%%%%%%%%%%%%%%%%%%%%%%%%%%%%%%%%%%
\subsection{Revision History}

%%%%%%%%%%%%%%%%%%%%%%%%%%%%%%%%%%%%%%%%
\paragraph{v2.0:} 2018/12/30

\begin{itemize}
\item
immediate forward processing
\item
added |\childdocby| mechanism
\item
manual restructured
\end{itemize}

%%%%%%%%%%%%%%%%%%%%%%%%%%%%%%%%%%%%%%%%
\paragraph{v1.6:} 2018/01/17

\begin{itemize}
\item
application for development of include files
\item
corrections to manual
\end{itemize}

%%%%%%%%%%%%%%%%%%%%%%%%%%%%%%%%%%%%%%%%
\paragraph{v1.5:} 2017/05/21

\begin{itemize}
\item
more complete structuring introduced
\item
|\childdocof| introduced
\item
|\childdoc| renamed to |\childdocmain|
\item
|\childredirect| renamed to |\childdocforward| and |\childdocforwardprefix|
and functionality expanded
\end{itemize}

%%%%%%%%%%%%%%%%%%%%%%%%%%%%%%%%%%%%%%%%
\paragraph{v1.0:} 2017/04/27

\begin{itemize}
\item
manual and install package
\item
first version published on CTAN
\end{itemize}

%%%%%%%%%%%%%%%%%%%%%%%%%%%%%%%%%%%%%%%%
\paragraph{v0.6:} 2017/04/26

\begin{itemize}
\item
redirection mechanism added
\end{itemize}

%%%%%%%%%%%%%%%%%%%%%%%%%%%%%%%%%%%%%%%%
\paragraph{v0.5:} 2017/04/26

\begin{itemize}
\item
functionality in definition file
\end{itemize}


%%%%%%%%%%%%%%%%%%%%%%%%%%%%%%%%%%%%%%%%%%%%%%%%%%%%%%%%%%%%%%%%%%%%%%%%%%%%%%%%
%%%%%%%%%%%%%%%%%%%%%%%%%%%%%%%%%%%%%%%%%%%%%%%%%%%%%%%%%%%%%%%%%%%%%%%%%%%%%%%%
%%%%%%%%%%%%%%%%%%%%%%%%%%%%%%%%%%%%%%%%%%%%%%%%%%%%%%%%%%%%%%%%%%%%%%%%%%%%%%%%
\appendix

\settowidth\MacroIndent{\rmfamily\scriptsize 000\ }

 \DocInput{childdoc.dtx}

\end{document}
%</driver>
% \fi
%
% %%%%%%%%%%%%%%%%%%%%%%%%%%%%%%%%%%%%%%%%%%%%%%%%%%%%%%%%%%%%%%%%%%%%%%%%%%%%%%
% %%%%%%%%%%%%%%%%%%%%%%%%%%%%%%%%%%%%%%%%%%%%%%%%%%%%%%%%%%%%%%%%%%%%%%%%%%%%%%
% \section{Sample}
%\iffalse
%<*samplemain>
%\fi
%
% The following presents a sample document
% with two chapters, two parts, a title page,
% a compile flag as well as three forwarding files to set the flag.
% It consists of eight |.tex| files:
% \begin{center}
% \begin{tabular}{ll}
% |cdocsamp.tex|&main file\\
% |cdocsch1.tex|&include file for chapter 1\\
% |cdocsch2.tex|&include file for chapter 2\\
% |cdocspt3.tex|&include file for part 3\\
% |cdocspt4.tex|&include file for part 4\\
% |cdocsdrf.tex|&forwarding file for main file in draft mode\\
% |cdocsfi1.tex|&forwarding file for final version of chapter 1\\
% |cdocsfi2.tex|&forwarding file for final version of chapter 2\\
% \end{tabular}
% \end{center}
% Each of the eight files can be compiled directly by the \LaTeX{} compiler.
%
% %%%%%%%%%%%%%%%%%%%%%%%%%%%%%%%%%%%%%%
% \paragraph{Main File.}
%
% The main file is called |cdocsamp.tex|.
%
% Load the \textsf{childdoc} definitions and
% declare the filename for the main document:
%    \begin{macrocode}
\input{childdoc.def}
\childdocmain{}
%    \end{macrocode}

% Optional override for |\version| flag:
%    \begin{macrocode}
%%\ifchilddoc\else\providecommand{\version}{draft}\fi
%    \end{macrocode}

% Define the default values for the |\version| flag
% (|final| for the main file and |draft| for childs):
%    \begin{macrocode}
\ifchilddoc
\providecommand{\version}{draft}
\else
\providecommand{\version}{final}
\fi
%    \end{macrocode}

% Load the standard document class:
%    \begin{macrocode}
\documentclass[12pt]{article}
%    \end{macrocode}

% Start the document body:
%    \begin{macrocode}
\begin{document}
%    \end{macrocode}

% Declare a title page.
% Print title, part of document being processed and version flag:
%    \begin{macrocode}
\addtocounter{page}{-1}
\begin{center}
{\LARGE\bfseries{}childdoc example\par}
\vspace{1cm}
\ifchilddoc
\ifchilddocmanual part\else chapter\fi:
`\childdocname' of `\childdocjob'\par
\else
main document: `\childdocjob'\par
\fi
version: \version\par
\end{center}
\newpage
%    \end{macrocode}

% Manually include selected file,
% otherwise process as usual:
%    \begin{macrocode}
\ifchilddocmanual
\section*{part `\childdocname'}
\input{\childdocname}
\else
%    \end{macrocode}

% Include the two chapters:
%    \begin{macrocode}
\include{cdocsch1}
\include{cdocsch2}
%    \end{macrocode}

% Include the two parts unless only chapters should be displayed:
%    \begin{macrocode}
\ifchilddoc\else
\section{part three}
\input{cdocspt3}
\section{part four}
\input{cdocspt4}
\fi
%    \end{macrocode}

% Process as usual until here:
%    \begin{macrocode}
\fi
%    \end{macrocode}

% End of document body:
%    \begin{macrocode}
\end{document}
%    \end{macrocode}
%\iffalse
%</samplemain>
%\fi
%
% %%%%%%%%%%%%%%%%%%%%%%%%%%%%%%%%%%%%%%
% \paragraph{Chapter Include Files.}
%
% The include files are called |cdocsch1.tex| and |cdocsch2.tex|.
%
%\iffalse
%<*samplechap1|samplechap2>
%\fi

% Optional override for |\version| flag:
%    \begin{macrocode}
%%\providecommand{\version}{final}
%    \end{macrocode}

% Include the main document:
%    \begin{macrocode}
\input{childdoc.def}
\childdocof{cdocsamp}
%    \end{macrocode}

%\iffalse
%</samplechap1|samplechap2>
%\fi
%
%\iffalse
%<*samplechap1>
%\fi
% Some text for chapter 1:
%    \begin{macrocode}
\section{one}
some text in chapter one
%    \end{macrocode}

%\iffalse
%</samplechap1>
%\fi
% Some text for chapter 2:
%\iffalse
%<*samplechap2>
%\fi
%    \begin{macrocode}
\section{two}
more text in chapter two
%    \end{macrocode}

%\iffalse
%</samplechap2>
%\fi
%
% %%%%%%%%%%%%%%%%%%%%%%%%%%%%%%%%%%%%%%
% \paragraph{Part Include Files.}
%
% The include files are called |cdocspt3.tex| and |cdocspt4.tex|.
%
%\iffalse
%<*samplepart3|samplepart4>
%\fi

% Optional override for |\version| flag:
%    \begin{macrocode}
%%\providecommand{\version}{final}
%    \end{macrocode}

% Include the main document:
%    \begin{macrocode}
\input{childdoc.def}
\childdocby{cdocsamp}
%    \end{macrocode}

%\iffalse
%</samplepart3|samplepart4>
%\fi
%
%\iffalse
%<*samplepart3>
%\fi
% Some text for part 3:
%    \begin{macrocode}
some text in part three
%    \end{macrocode}

%\iffalse
%</samplepart3>
%\fi
% Some text for part 4:
%\iffalse
%<*samplepart4>
%\fi
%    \begin{macrocode}
more text in part four
%    \end{macrocode}

%\iffalse
%</samplepart4>
%\fi
%
% %%%%%%%%%%%%%%%%%%%%%%%%%%%%%%%%%%%%%%
% \paragraph{Forwarding for a Complete Draft.}
%
% The following forwarding file |cdocsdrf.tex|
% compiles the main document in draft mode:
%\iffalse
%<*sampledraft>
%\fi
%    \begin{macrocode}
\def\version{draft}
\input{childdoc.def}
\childdocforward{cdocsamp}
%    \end{macrocode}

%\iffalse
%</sampledraft>
%\fi
%
% %%%%%%%%%%%%%%%%%%%%%%%%%%%%%%%%%%%%%%
% \paragraph{Forwarding for Final Version of the Chapters.}
%
% The following forwarding files |cdocsfn1.tex| and |cdocsfn2.tex|
% (with identical content)
% compile the final versions of the child documents
% |cdocsch1.tex| and |cdocsch2.tex|, respectively:
%\iffalse
%<*samplefinal>
%\fi
%    \begin{macrocode}
\def\version{final}
\input{childdoc.def}
\childdocforwardprefix[cdocsamp]{cdocsfn}{cdocsch}
%    \end{macrocode}

%\iffalse
%</samplefinal>
%\fi
%
% %%%%%%%%%%%%%%%%%%%%%%%%%%%%%%%%%%%%%%
% \paragraph{Command Line Processing.}
%
% The following three command lines generate the output files
% |cdocscld|, |cdocscl1| and |cdocscl2|
% which should be identical to
% |cdocsdrf|, |cdocsch1| and |cdocsfn2|, respectively:
% \begin{center}
% \begin{tabular}{l}
% |latex -jobname cdocscld \|\\
% |  "\def\version{draft}\input{childdoc.def}\childdocforward{cdocsamp}"|\\
% |latex -jobname cdocscl1 \|\\
% |  "\input{childdoc.def}\childdocforward[cdocsamp]{cdocsch1}"|\\
% |latex -jobname cdocscl2 \|\\
% |  "\def\version{final}\input{childdoc.def}\childdocforward{cdocsch2}"|
% \end{tabular}
% \end{center}
% Note that the trailing backslash on each first line
% merely continues the input to the second line
% (for convenient cut ant paste).
% Furthermore, the command |latex| can be replaced by any
% of its alternative versions such as |pdflatex|.
%
% %%%%%%%%%%%%%%%%%%%%%%%%%%%%%%%%%%%%%%%%%%%%%%%%%%%%%%%%%%%%%%%%%%%%%%%%%%%%%%
% %%%%%%%%%%%%%%%%%%%%%%%%%%%%%%%%%%%%%%%%%%%%%%%%%%%%%%%%%%%%%%%%%%%%%%%%%%%%%%
% \section{Implementation}
%\iffalse
%<*package>
%\fi
%
% This section describes the definitions file |childdoc.def|.

% The definitions cannot be loaded using |\usepackage| or |\RequirePackage|
% which has a mechanism to prevent loading a style file more than once.
% When loading the definitions by means of |\input|
% multiple instances have to be prevented manually:
%\iffalse
%This code needs to be before the `\ProvidesFile' directive
%which is defined at the beginning of this file.
%Therefore it is also placed there and commented out here.
%</package>
%<*discard>
%\fi
%    \begin{macrocode}
\ifdefined\childdocmain\endinput\fi
%    \end{macrocode}
%\iffalse
%</discard>
%<*package>
%\fi
%
% \macro{\ifchilddoc}
% \macro{\ifchilddocmanual}
% The conditional |\ifchilddoc| tells whether a
% child (true) or main (false) document is being compiled.
% The conditional |\ifchilddocmanual| tells whether
% the |\includeonly| mechanism is used (false) or
% the selection of child files must be performed manually (true).
% The definitions initialise to false:
%    \begin{macrocode}
\newif\ifchilddoc
\newif\ifchilddocmanual
%    \end{macrocode}

% \macro{\childdocname}
% \macro{\childdocjob}
% The macro |\childdocname| stores the name of the main document
% to be compiled. The macro |\childdocjob| stores the name of
% the document on which the \LaTeX{} compiler was originally invoked.
% The content of |\jobname| cannot be compared
% to filenames specified in the source due to different catcodes.
% The following code rescans |\jobname|, stores the result
% in |\childdocname| and saves a copy in |\childdocjob|:
%    \begin{macrocode}
\edef\childdocname{\scantokens\expandafter{\jobname\noexpand}}
\let\childdocjob\childdocname
%    \end{macrocode}

% \macro{\childdocdisable}
% The macro |\childdocdisable| prevents the main file
% from being processed more than once.
% At this stage, the main document command |\childdocmain|
% is assumed to be called once again where it should do nothing.
% Any subsequent call to it should prevent
% a secondary processing of the main document
% It overwrites the forwarding commands
% |\childdocof| and |\childdocforward|
% with empty macros to prevent further inclusions of the main document:
%    \begin{macrocode}
\newcommand{\childdocdisable}
{
  \renewcommand{\childdocmain}[1]{\renewcommand{\childdocmain}[1]{\endinput}}
  \renewcommand{\childdocof}[1]{}
  \renewcommand{\childdocby}[2][]{}
  \renewcommand{\childdocforward}[2][]{}
  \renewcommand{\childdocdisable}{}
}
%    \end{macrocode}

% \macro{\childdocmain}
% The macro |\childdocmain| is to be called at the top of the main file
% with nothing or the main filename (without extension) as argument.
% First, it breaks loops.
% If the argument is not empty and does not match |\childdocname|
% (which is set by the first inclusion of |childdoc.def|),
% |\ifchilddoc| is set to true, |\includeonly| is applied to the child file
% and |\jobname| is set to the main file
% (for proper handling of |.aux| files):
%    \begin{macrocode}
\newcommand{\childdocmain}[1]
{
  \childdocdisable\childdocmain{}
  \if?#1?\else
    \begingroup
      \def\childdoctmp{#1}
      \ifx\childdoctmp\childdocname
        \def\childdoctmp{}
      \else
        \def\childdoctmp
        {
          \childdoctrue
          \includeonly{\childdocname}
          \def\childdocjob{#1}
          \def\jobname{#1}
        }
      \fi
      \expandafter
    \endgroup
    \childdoctmp
  \fi
}
%    \end{macrocode}

% \macro{\childdocof}
% The command |\childdocof| redirects
% compilation to the main file |#1|.
%    \begin{macrocode}
\newcommand{\childdocof}[1]
{
  \childdocdisable
  \childdoctrue
  \includeonly{\childdocname}
  \def\jobname{#1}
  \def\childdocjob{#1}
  \input{#1}
}
%    \end{macrocode}

% \macro{\childdocby}
% The command |\childdocby| ....
%    \begin{macrocode}
\newcommand{\childdocby}[2][]
{
  \childdocdisable
  \childdoctrue
  \childdocmanualtrue
  \if?#1?\else
    \def\jobname{#2}
  \fi
  \def\childdocjob{#2}
  \input{#2}
  \endinput
}
%    \end{macrocode}

% \macro{\childdocforward}
% The command |\childdocforward| redirects
% compilation to the main file or
% (if the optional argument is given) a child file.
% Parameters are set as if the main file
% or a child file starting with |\childdocof| was compiled.
% Then compilation is handed over to the main file:
%    \begin{macrocode}
\newcommand{\childdocforward}[2][]
{
  \begingroup
    \if?#1?
      \def\childdoctmp
      {
        \def\childdocname{#2}
        \def\childdocjob{#2}
        \def\jobname{#2}
        \input{#2}
        \endinput
      }
    \else
      \def\childdoctmp
      {
        \childdocdisable
        \def\childdocname{#2}
        \childdoctrue
        \includeonly{#2}
        \def\childdocjob{#1}
        \def\jobname{#1}
        \input{#1}
        \endinput
      }
    \fi
    \expandafter
  \endgroup
  \childdoctmp
}
%    \end{macrocode}

% \macro{\childdocforwardprefix}
% The command |\childdocforwardprefix| redirects
% compilation to the main or a child file by means of a pattern.
% The prefix |#1| in the current filename is replaced by |#2|
% and the suffix of the current filename is kept
% (it is assumed that the filename does not contain the substring `|~~~|'
% which is used as a delimiter).
% Compilation is handed over to the new file by |\childdocforward|:
%    \begin{macrocode}
\newcommand{\childdocforwardprefix}[3][]
{
  \begingroup
    \def\childdocextract #2##1~~~{\def\childdoctmp{\childdocforward[#1]{#3##1}}}
    \expandafter\childdocextract\childdocname~~~
    \expandafter
  \endgroup
  \childdoctmp
}
%    \end{macrocode}

% \macro{\childdoc}
% The deprecated macro |\childdoc| is a legacy version of |\childdocmain|:
%    \begin{macrocode}
\newcommand{\childdoc}{\childdocmain}
%    \end{macrocode}

% \macro{\childdocredirect}
% The deprecated macro |\childdocredirect| is a legacy version
% of |\childdocforward| and |\childdocforwardprefix|:
%    \begin{macrocode}
\newcommand{\childdocredirect}[2][]
{
  \begingroup
    \if?#1?
      \def\childdoctmp{\childdocforward{#2}}
    \else
      \def\childdoctmp{\childdocforwardprefix{#1}{#2}}
    \fi
    \expandafter
  \endgroup
  \childdoctmp
}
%    \end{macrocode}

%\iffalse
%</package>
%\fi
%
\endinput
|\\
|\childdocforwardprefix{final}{child}|
\end{tabular}
\end{center}
%

Note that when several versions of a main file and/or of each child file
are to be generated, it may be convenient to set up a |Makefile| or
shell script to automatise the process.

%%%%%%%%%%%%%%%%%%%%%%%%%%%%%%%%%%%%%%%%%%%%%%%%%%%%%%%%%%%%%%%%%%%%%%%%%%%%%%%%
\subsection{Command Line Processing}
\label{sec:commandline}

The effect of redirection files can also be achieved by invoking
the \LaTeX{} compiler with a more elaborate command line.
Most conveniently this should be done as part
of a shell script or a |Makefile|.

When using \textsf{childdoc} in the main file, the following
command lines effectively perform a redirection
(note that depending on the shell being used,
backslashes may have to be doubled: `|\|' $\to$ `|\\|'):
%
\begin{center}
|... -jobname "|\textit{target}|" |\\|"|[\textit{flags}]%
|% \iffalse
%
% childdoc.dtx Copyright (C) 2017-2018 Niklas Beisert
%
% This work may be distributed and/or modified under the
% conditions of the LaTeX Project Public License, either version 1.3
% of this license or (at your option) any later version.
% The latest version of this license is in
%   http://www.latex-project.org/lppl.txt
% and version 1.3 or later is part of all distributions of LaTeX
% version 2005/12/01 or later.
%
% This work has the LPPL maintenance status `maintained'.
%
% The Current Maintainer of this work is Niklas Beisert.
%
% This work consists of the files childdoc.dtx and childdoc.ins
% and the derived files childdoc.def and cdocsamp.tex with
% cdocsch1.tex, cdocsch2.tex, cdocsdrf.tex, cdocsfn1.tex, cdocsfn2.tex.
%
%<package>\ifdefined\childdocmain\endinput\fi
%<package>\ProvidesFile{childdoc.def}[2018/12/30 v2.0 child document driver]
%<samplemain>\ProvidesFile{cdocsamp.tex}[2018/12/30 v2.0 sample for childdoc]
%<*driver>
%\ProvidesFile{childdoc.drv}[2018/12/30 v2.0 childdoc reference manual file]
\PassOptionsToClass{10pt,a4paper}{article}
\documentclass{ltxdoc}

\usepackage[margin=35mm]{geometry}
\usepackage{hyperref}
\usepackage{hyperxmp}
\usepackage[usenames]{color}

\hypersetup{colorlinks=true}
\hypersetup{pdfstartview=FitH}
\hypersetup{pdfpagemode=UseNone}
\hypersetup{pdfsource={}}
\hypersetup{pdflang={en-UK}}
\hypersetup{pdfcopyright={Copyright 2017-2018 Niklas Beisert.
  This work may be distributed and/or modified under the
  conditions of the LaTeX Project Public License, either version 1.3
  of this license or (at your option) any later version.}}
\hypersetup{pdflicenseurl={http://www.latex-project.org/lppl.txt}}
\hypersetup{pdfcontactaddress={ETH Zurich, ITP, HIT K,
  Wolfgang-Pauli-Strasse 27}}
\hypersetup{pdfcontactpostcode={8093}}
\hypersetup{pdfcontactcity={Zurich}}
\hypersetup{pdfcontactcountry={Switzerland}}
\hypersetup{pdfcontactemail={nbeisert@itp.phys.ethz.ch}}
\hypersetup{pdfcontacturl={http://people.phys.ethz.ch/\xmptilde nbeisert/}}

\newcommand{\secref}[1]{\hyperref[#1]{section \ref*{#1}}}

\parskip1ex
\parindent0pt
\let\olditemize\itemize
\def\itemize{\olditemize\parskip0pt}

\begin{document}

\title{The \textsf{childdoc} Package}
\hypersetup{pdftitle={The childdoc Package}}
\author{Niklas Beisert\\[2ex]
  Institut f\"ur Theoretische Physik\\
  Eidgen\"ossische Technische Hochschule Z\"urich\\
  Wolfgang-Pauli-Strasse 27, 8093 Z\"urich, Switzerland\\[1ex]
  \href{mailto:nbeisert@itp.phys.ethz.ch}
  {\texttt{nbeisert@itp.phys.ethz.ch}}}
\hypersetup{pdfauthor={Niklas Beisert}}
\hypersetup{pdfsubject={Manual for the LaTeX2e Package childdoc}}
\date{30 December 2018, \textsf{v2.0}}
\maketitle

\begin{abstract}\noindent
\textsf{childdoc} is a \LaTeXe{} package
that enables the direct compilation
of document sections included by |\include|
to individual files.
\end{abstract}

\begingroup
\parskip0ex
\tableofcontents
\endgroup

%%%%%%%%%%%%%%%%%%%%%%%%%%%%%%%%%%%%%%%%%%%%%%%%%%%%%%%%%%%%%%%%%%%%%%%%%%%%%%%%
%%%%%%%%%%%%%%%%%%%%%%%%%%%%%%%%%%%%%%%%%%%%%%%%%%%%%%%%%%%%%%%%%%%%%%%%%%%%%%%%
\section{Introduction}

\LaTeX{} provides a mechanism to structure a large document (such as a book)
into a main file and several child files (containing the chapters)
using the |\include| command.
This mechanism is beneficial for documents
which span hundreds of pages in order to
make the source file(s) more manageable.
Moreover, compilation can be restricted to
selected child files by means of the |\includeonly| command.
The latter feature can be used to reduce the compilation time while editing
(this was significantly more useful in the earlier days of \LaTeX{})
or to generate a smaller document which is easier to navigate.
Another application of |\includeonly| is to generate
documents consisting of selected parts of the complete document.

However, there are a few drawbacks of the plain |\include| mechanism:
\begin{itemize}
\item
The child files cannot be compiled on their own,
they can only be compiled via the main file.
A naive editing environment
(such as a text editor with an option
to have the current file processed by \LaTeX)
may require one to switch to the main file before compiling;
attempting to compile the child file produces errors.
\item
The main file must be modified (each time)
to adjust the |\includeonly| command
to the present needs. This easily leaves the main file in a messy state.
\item
The generated document will always carry the filename
of the main document. This is inconvenient if
several child files are to be compiled and
to be kept for distribution.
\end{itemize}

The present package provides a simple interface
to make child files individually compilable by \LaTeX{}.
Compiling a child file then has the same effect as compiling
the main file with an |\includeonly| command
to select the appropriate child.
Moreover the generated document will carry the name of the child
rather than the main file.
This resolves all three above issues.

This feature is meant to make the editing of books,
thesis documents and lecture notes somewhat more convenient.
However, the package can also be used efficiently for
composing a series of documents (such as exercise sheets)
which are typically distributed individually.
It then assists the author in generating the individual documents
(potentially in different versions)
as well as a document containing the collected series.
Another application is in developing style files
or other kinds of included material
where compilation of the style file could redirect
to a sample or test file.

%%%%%%%%%%%%%%%%%%%%%%%%%%%%%%%%%%%%%%%%%%%%%%%%%%%%%%%%%%%%%%%%%%%%%%%%%%%%%%%%
%%%%%%%%%%%%%%%%%%%%%%%%%%%%%%%%%%%%%%%%%%%%%%%%%%%%%%%%%%%%%%%%%%%%%%%%%%%%%%%%
\section{Usage}

First of all, the package \textsf{childdoc} is \emph{not} a standard
\LaTeXe{} |.sty| style file! Therefore it needs to be invoked in
a non-standard way.

%%%%%%%%%%%%%%%%%%%%%%%%%%%%%%%%%%%%%%%%%%%%%%%%%%%%%%%%%%%%%%%%%%%%%%%%%%%%%%%%
\subsection{Included Files}
\label{sec:include}

%%%%%%%%%%%%%%%%%%%%%%%%%%%%%%%%%%%%%%%%
\DescribeMacro{\childdocmain}
To use the package, add the commands
\begin{center}
\begin{tabular}{l}
|\input{childdoc.def}|\\
|\childdocmain{}|\\
\end{tabular}
\end{center}
at the very top of the main \LaTeX{} file,
in particular \emph{before} the |\documentclass| statement!
The argument of |\childdocmain| should be left empty
(but it must be present).

%%%%%%%%%%%%%%%%%%%%%%%%%%%%%%%%%%%%%%%%
\DescribeMacro{\childdocof}
Furthermore, add the commands
\begin{center}
\begin{tabular}{l}
|\input{childdoc.def}|\\
|\childdocof{|\textit{main}|}|\\
\end{tabular}
\end{center}
at the top of every child file \textit{child}
which is included by |\include{|\textit{child}|}|
from within the main file
(or at least for those files to be compiled individually).
The argument \textit{main} must be the filename of the main file.

There are a couple of
considerations in setting up the main and child documents:

%%%%%%%%%%%%%%%%%%%%%%%%%%%%%%%%%%%%%%%%
\paragraph{Restrictions.}

Please note the following restrictions:
\begin{itemize}
\item
|\childdocmain| must be called with one argument \textit{main}
to ensure compatibility with earlier version of the package.
It must either be empty (|\childdocmain{}|)
or precisely match the filename of the main file in which it is specified.
See \secref{sec:detection} for further information.
\item
The filename \textit{main} must be specified without the |.tex| extension.
\item
The filename \textit{main} is case sensitive
(even in case-insensitive file systems)
due to internal string comparison.
\item
The argument \textit{main} should be fully expanded, it cannot be a macro.
\item
Subdirectories and special characters should be avoided in filenames.
\item
The command |\childdocmain{|\textit{main}|}| must be followed by a whitespace.
It should not be followed immediately by another command
or by a comment mark `|%|'.
This is because the \TeX{} parser reads the token immediately following
the argument of |\childdocmain| and puts it
at the beginning of every child section;
however, a white\-space is ignored.
\end{itemize}

%%%%%%%%%%%%%%%%%%%%%%%%%%%%%%%%%%%%%%%%
\paragraph{Content of Main File.}

It is advisable to place all content in the child files included by |\include|.
Any output contained in the main file will appear in all child documents
unless suppressed manually;
it cannot be suppressed automatically by the |\includeonly| directive
and thus should normally be avoided.
A method to include some content in the main file
by means of conditional processing is described in \secref{sec:conditional}.

%%%%%%%%%%%%%%%%%%%%%%%%%%%%%%%%%%%%%%%%
\paragraph{Page Numbering.}

When only a part of the document is compiled,
the appropriate numbering of pages
(as well as other status parameters)
is determined from the |.aux| files.
The latter contain information from previous passes.
However this information needs to propagate through
all intermediate child documents.
Therefore the page numbering in child documents may well
be inconsistent until the complete document is compiled at least once.

A useful (if unconventional) way to always ensure a consistent
page numbering is to restart the numbering in each child document
and denote the pages by `\textit{child}|.|\textit{page}'
where \textit{child} represents the chapter/section number of the child file.
This can be achieved by the command
|\numberwithin{page}{|\textit{child}|}|
of the \textsf{amsmath} package
where \textit{child} can be |chapter| or |section|
depending on the chosen structuring.
Alternatively, one can modify the macro |\thepage| appropriately
and reset the counter |page| at the start of each child file.

%%%%%%%%%%%%%%%%%%%%%%%%%%%%%%%%%%%%%%%%%%%%%%%%%%%%%%%%%%%%%%%%%%%%%%%%%%%%%%%%
\subsection{Conditional Processing}
\label{sec:conditional}

The package provides a mechanism to compile different versions
of a document. To customise the versions further some conditional processing
can come in handy to distinguish which version is being compiled.
The package provides two macros to describe the compilation context:

%%%%%%%%%%%%%%%%%%%%%%%%%%%%%%%%%%%%%%%%
\DescribeMacro{\ifchilddoc}
The conditional |\ifchilddoc| distinguishes between the compilation of
child documents and the main document:
%
\begin{center}
|\ifchilddoc |\textit{child-code}| |[|\||else |\textit{main-code}]| \||fi|
\end{center}

%%%%%%%%%%%%%%%%%%%%%%%%%%%%%%%%%%%%%%%%
\DescribeMacro{\childdocname}
\DescribeMacro{\childdocjob}
The macro |\childdocname| contains the filename (without extension)
of the main or child file being processed.
Note that |\childdocjob| will always contain the name of the main file.

%%%%%%%%%%%%%%%%%%%%%%%%%%%%%%%%%%%%%%%%
\paragraph{Title Page.}

Conditional processing can be used to include a title or banner page
in the main document when proper precautions are taken.
Importantly, the code in the main file should ensure that the page counter
(as well as other status parameters which are stored in the |.aux| files)
takes the same value after the conditional processing.
Otherwise the page numbers may take divergent values
depending on which part is compiled.

For example, a title page could be declared by:
%
\begin{center}
\begin{tabular}{l}
|\ifchilddoc\||else|\\
|\addtocounter{page}{-1}|\\
\textit{code for title page}\\
|\newpage|\\
|\||fi|
\end{tabular}
\end{center}
%
A banner page for the child documents can be generated by:
%
\begin{center}
\begin{tabular}{l}
|\ifchilddoc|\\
|\addtocounter{page}{-1}|\\
\textit{code for banner page}\\
|\newpage|\\
|\||fi|
\end{tabular}
\end{center}
%
Here one could write a message such as:
\begin{center}
|This is the part \childdocname{} of \childdocjob{}.|
\end{center}

%%%%%%%%%%%%%%%%%%%%%%%%%%%%%%%%%%%%%%%%%%%%%%%%%%%%%%%%%%%%%%%%%%%%%%%%%%%%%%%%
\subsection{Flags}
\label{sec:flags}

The package makes it easy to generate different versions
of the main or child documents.
To this end compilation flags can be defined
and assigned different default values.
They will be particularly useful in conjunction
with the forwarding mechanism described in \secref{sec:forward}.

For example, it may be useful to have a flag |\version|
which can be set to |draft| or |final|.
The document source will contain some conditional code
depending on the value of |\version|.
Suppose further, the flag should default to |final| for the main file
and to |draft| for child files
which is a natural assignment for editing the document.
This is achieved by placing the following code
in the preamble of the main document
(below the |\childdocmain| directive):
%
\begin{center}
\begin{tabular}{l}
|\ifchilddoc|\\
|\providecommand{\version}{draft}|\\
|\||else|\\
|\providecommand{\version}{final}|\\
|\||fi|
\end{tabular}
\end{center}
%
The definition by |\providecommand| makes sure
that previous definitions are not overwritten.
Further statements |\providecommand{\version}{...}|
can thus be added before the above code to override it.

For the main file, one might add a line
(between |\childdocmain| and the above block)
%
\begin{center}
|%\ifchilddoc\||else\providecommand{\version}{draft}\||fi|
\end{center}
%
which can be uncommented to produce a draft version.
Likewise one can add a line to the very top of a child file
(above the |\childdocof{|\textit{main}|}| directive)
%
\begin{center}
|%\providecommand{\version}{final}|
\end{center}
%
which can be uncommented to produce the final version of this child document.

%%%%%%%%%%%%%%%%%%%%%%%%%%%%%%%%%%%%%%%%%%%%%%%%%%%%%%%%%%%%%%%%%%%%%%%%%%%%%%%%
\subsection{Forwarding}
\label{sec:forward}

Different versions of the main or child documents
using compilation flags as described in \secref{sec:flags}
can be (permanently) stored in different files
for convenient compilation, viewing and distribution.
To this end, the package defines a command
to pass on compilation to a different file:

%%%%%%%%%%%%%%%%%%%%%%%%%%%%%%%%%%%%%%%%
\DescribeMacro{\childdocforward}
The command |\childdocforward| redirects processing to
another source file:
%
\begin{center}
\begin{tabular}{l}
|\input{childdoc.def}|\\
|\childdocforward[|\textit{main}|]{|\textit{dest}|}|\\
\end{tabular}
\end{center}
%
The argument \textit{dest} is the destination file
(without extension).
It should be the main file or one of the child files.
Note that further \textsf{childdoc} directives
such as |\childdocof| and |\childdocforward|
in the indicated file will be processed in this form.
The optional argument \textit{main}
passes on directly to the main file \textit{main}
while pretending to compile the child \textit{dest}.
This form behaves as if \textit{dest}
issues |\childdocof{|\textit{main}|}| right away,
and no further \textsf{childdoc} directives will be processed.

%%%%%%%%%%%%%%%%%%%%%%%%%%%%%%%%%%%%%%%%
\DescribeMacro{\...prefix}
In the alternative form |\childdocforwardprefix|,
%
\begin{center}
\begin{tabular}{l}
|\input{childdoc.def}|\\
|\childdocforwardprefix[|\textit{main}|]{|\textit{prefix}|}{|\textit{dest}|}|
\end{tabular}
\end{center}
%
the destination file is determined by a pattern
depending on the current file:
To make this work, the current file must be called
`{\textit{prefix}\hspace{0.2em}\textit{suffix}}'
with \textit{prefix} matching precisely the argument.
Processing is then passed on to the file
`{\textit{dest}\hspace{0.2em}\textit{suffix}}'.
Surely, the same effect is achieved by
directly specifying the
argument `{\textit{dest}\hspace{0.2em}\textit{suffix}}'
in the first form.
However, that requires to set up a different file
for each child. With the alternative form of the command
all these files can have exactly the same content
which simplifies setting them up and maintaining them.

For example, the following file |draft.tex|
with a compilation flag |\version| as described in \secref{sec:flags}
compiles the main document as a draft:
%
\begin{center}
\begin{tabular}{l}
|\def\version{draft}|\\
|\input{childdoc.def}|\\
|\childdocforward{|\textit{main}|}|
\end{tabular}
\end{center}
%
Likewise, the following files |final|\textit{nn}|.tex|
compile the final version of the child document
|child|\textit{nn}|.tex|:
%
\begin{center}
\begin{tabular}{l}
|\def\version{final}|\\
|\input{childdoc.def}|\\
|\childdocforwardprefix{final}{child}|
\end{tabular}
\end{center}
%

Note that when several versions of a main file and/or of each child file
are to be generated, it may be convenient to set up a |Makefile| or
shell script to automatise the process.

%%%%%%%%%%%%%%%%%%%%%%%%%%%%%%%%%%%%%%%%%%%%%%%%%%%%%%%%%%%%%%%%%%%%%%%%%%%%%%%%
\subsection{Command Line Processing}
\label{sec:commandline}

The effect of redirection files can also be achieved by invoking
the \LaTeX{} compiler with a more elaborate command line.
Most conveniently this should be done as part
of a shell script or a |Makefile|.

When using \textsf{childdoc} in the main file, the following
command lines effectively perform a redirection
(note that depending on the shell being used,
backslashes may have to be doubled: `|\|' $\to$ `|\\|'):
%
\begin{center}
|... -jobname "|\textit{target}|" |\\|"|[\textit{flags}]%
|\input{childdoc.def}\childdocforward[|\textit{main}|]{|\textit{dest}|}"|
\end{center}
%
Here \textit{target} is the name of the output file,
\textit{main} is the name of the main file
and \textit{dest} is the name of the main or child file to be processed
(all filenames without extensions).
The optional argument \textit{main} can be omitted
if \textit{main} matches \textit{dest}.
Optionally, compilation \textit{flags} can be defined via |\def| commands.
This command line makes the \TeX{} engine believe
it is compiling the file \textit{target}
whose content is specified as the latter parameter.
The provided code then forwards the processing to
\textit{main} or \textit{dest} as described in \secref{sec:forward}.

%%%%%%%%%%%%%%%%%%%%%%%%%%%%%%%%%%%%%%%%%%%%%%%%%%%%%%%%%%%%%%%%%%%%%%%%%%%%%%%%
\subsection{Include by Input}
\label{sec:input}

Including child documents by |\include| has some restrictions by design.
Most notably, the content of a child document always occupies
its own set of pages; pages cannot be shared between child documents.
Usually, this behaviour makes perfect sense
because each child document contain an essential part of the document.
However, in some situations it may be desirable to compose
a document from a collection of parts
without having mandatory page breaks between then.
For this case, the package
provides a mechanism to include parts
by |\input| which can also be processed individually.
However, by construction this mechanism
requires manual handling of the content to be output.

%%%%%%%%%%%%%%%%%%%%%%%%%%%%%%%%%%%%%%%%
\DescribeMacro{\ifchilddocmanual}
The main file should be prepared as usual, see \secref{sec:include}.
However, the document body must make a distinction
between processing of an individual part and of the main document, e.g.:
%
\begin{center}
\begin{tabular}{l}
|\ifchilddocmanual|\\
|\input{\childdocname}|\\
|\||else|\\
\textit{document body with }|\input{|\textit{part}|}|\\
|\||fi|
\end{tabular}
\end{center}
%
The conditional |\ifchilddocmanual| is true whenever
a part to be included by |\input| is being compiled,
and the name of the part is stored in |\childdocname|.

%%%%%%%%%%%%%%%%%%%%%%%%%%%%%%%%%%%%%%%%
\DescribeMacro{\childdocby}
Each part to be included by |\input| should start with:
%
\begin{center}
\begin{tabular}{l}
|\input{childdoc.def}|\\
|\childdocby{|\textit{main}|}|\\
\end{tabular}
\end{center}
%
The directive |\childdocby| is similar to |\childdocof|
described in \secref{sec:include},
but the subsequent selection of content must be done manually.
To that end, both |\ifchilddoc| and |\ifchilddocmanual|
will be true upon processing of a part,
and the name of the part is stored in |\childdocname|.
Note that |\jobname| will be set to the filename of the current part
so that each part receives an individual |.aux| file
that does not interfere with the |.aux| file(s) of the main document.
This behaviour can be altered by the alternative form
|\childdocby[*]{|\textit{main}|}| (with a non-empty optional argument)
which uses the |.aux| file of the main document
by setting |\jobname| to \textit{main}.

%%%%%%%%%%%%%%%%%%%%%%%%%%%%%%%%%%%%%%%%%%%%%%%%%%%%%%%%%%%%%%%%%%%%%%%%%%%%%%%%
\subsection{Driver Development}
\label{sec:driver}

The \textsf{childdoc} mechanism can also be use for the development
of definition files such as \LaTeX{} styles or classes.
This case differs from the above setup with multiple parts
included by |\include| in that no |\includeonly| should be invoked.
This can be achieved by starting the include file
(before |\ProvidesPackage|) with:
%
\begin{center}
\begin{tabular}{l}
|\input{childdoc.def}|\\
|\childdocforward{|\textit{main}|}|\\
\end{tabular}
\end{center}
%
or alternatively with:
%
\begin{center}
\begin{tabular}{l}
|\input{childdoc.def}|\\
|\childdocby{|\textit{main}|}|\\
\end{tabular}
\end{center}
%
Both forms have slightly different effects as described above.
The main file is prepared as usual, see \secref{sec:include}.

%%%%%%%%%%%%%%%%%%%%%%%%%%%%%%%%%%%%%%%%%%%%%%%%%%%%%%%%%%%%%%%%%%%%%%%%%%%%%%%%
\subsection{Legacy Detection}
\label{sec:detection}

The directive |\childdocmain| in the main file can detect
whether the complete document or merely a child is to be compiled
even without using the directive |\childdocof|.
This method is deprecated because it is less robust
and there is no compelling reason to use it;
it is merely provided for backward compatibility
and it may be removed in future versions.

If the detection mechanism is to be used,
it is mandatory to correctly specify
the filename of the main file as the argument of |\childdocmain|:
%
\begin{center}
\begin{tabular}{l}
|\input{childdoc.def}|\\
|\childdocmain{|\textit{main}|}|\\
\end{tabular}
\end{center}
%
If |\jobname| does not match the argument \textit{main} of |\childdocmain|,
it is assumed that |\jobname| points to the child file to be compiled.
When using |\childdocmain| with the main file specified as argument,
it suffices to start a child file
with just |\input{|\textit{main}|}|
without loading of the package and using |\childdocof|.
If instead all processing is done
with the appropriate \textsf{childdoc} directives,
the argument of \textit{main} of |\childdocmain| can be empty.

An alternative version of the command line processing described
in \secref{sec:commandline} using the detection mechanism reads:
%
\begin{center}
|... -jobname "|\textit{target}|" "|[\textit{flags}]%
[|\def\jobname{|\textit{dest}|}|]|\input{|\textit{main}|}"|
\end{center}

%%%%%%%%%%%%%%%%%%%%%%%%%%%%%%%%%%%%%%%%%%%%%%%%%%%%%%%%%%%%%%%%%%%%%%%%%%%%%%%%
\subsection{Manual Code}
\label{sec:manual}

In case one cannot be certain whether the definitions file |childdoc.def|
is installed on the target \TeX{} distribution
and one prefers not to ship it,
it is conceivable to paste a few relevant commands into the sources.

To that end, drop all statements |\input{childdoc.def}|
and perform the replacements as outlined below.
Instead of |\childdocmain{|\textit{main}|}| add the following code
to the top of the main file:
%
\begin{center}
\begin{tabular}{l}
|\||ifdefined\childdocname\endinput\||fi\newif\ifchilddoc|\\
|\edef\childdocname{\scantokens\expandafter{\jobname\noexpand}}|\\
|\def\childdocmain{|\textit{main}|}\||ifx\childdocmain\childdocname\||else|\\
|\childdoctrue\includeonly{\childdocname}\let\jobname\childdocmain\||fi|\\
\end{tabular}
\end{center}
%
Instead of |\childdocof{|\textit{main}|}| just include the main file
at the top of each child file:
%
\begin{center}
|\input{|\textit{main}|}|
\end{center}
%
A simple redirection |\childdocforward{|\textit{dest}|}| is achieved by:
%
\begin{center}
|\def\jobname{|\textit{dest}|}\input{\jobname}|
\end{center}
%
The redirection with prefix
|\childdocforwardprefix[|\textit{prefix}|]{|\textit{dest}|}|
is accomplished by:
%
\begin{center}
\begin{tabular}{l}
|{\edef\jobname{\scantokens\expandafter{\jobname\noexpand}}|\\
|\def\redirectjob |\textit{prefix}|#1~~~{\gdef\jobname{|\textit{dest}|#1}}|\\
|\expandafter\redirectjob\jobname~~~}\input{\jobname}|
\end{tabular}
\end{center}

In an alternative approach,
child documents can be compiled by a specific command line
without additional code or specific definitions:
%
\begin{center}
|... -jobname "|\textit{target}|" "|[\textit{flags}]%
|\includeonly{|\textit{dest}|}\input{|\textit{main}|}"|
\end{center}
%

%%%%%%%%%%%%%%%%%%%%%%%%%%%%%%%%%%%%%%%%%%%%%%%%%%%%%%%%%%%%%%%%%%%%%%%%%%%%%%%%
%%%%%%%%%%%%%%%%%%%%%%%%%%%%%%%%%%%%%%%%%%%%%%%%%%%%%%%%%%%%%%%%%%%%%%%%%%%%%%%%
\section{Information}

%%%%%%%%%%%%%%%%%%%%%%%%%%%%%%%%%%%%%%%%%%%%%%%%%%%%%%%%%%%%%%%%%%%%%%%%%%%%%%%%
\subsection{Copyright}

Copyright \copyright{} 2017--2018 Niklas Beisert

This work may be distributed and/or modified under the
conditions of the \LaTeX{} Project Public License, either version 1.3
of this license or (at your option) any later version.
The latest version of this license is in
  \url{http://www.latex-project.org/lppl.txt}
and version 1.3 or later is part of all distributions of \LaTeX{}
version 2005/12/01 or later.

This work has the LPPL maintenance status `maintained'.

The Current Maintainer of this work is Niklas Beisert.

This work consists of the files |README.txt|, |childdoc.ins| and |childdoc.dtx|
as well as the derived files |childdoc.def|, |cdocsamp.tex|
with |cdocsch1.tex|, |cdocsch2.tex|, |cdocspt3.tex|, |cdocspt4.tex|,
|cdocsdrf.tex|, |cdocsfn1.tex|, |cdocsfn2.tex|
as well as |childdoc.pdf|.

%%%%%%%%%%%%%%%%%%%%%%%%%%%%%%%%%%%%%%%%%%%%%%%%%%%%%%%%%%%%%%%%%%%%%%%%%%%%%%%%
\subsection{Files and Installation}

The package consists of the files:
%
\begin{center}
\begin{tabular}{ll}
    |README.txt|   & readme file \\
    |childdoc.ins| & installation file \\
    |childdoc.dtx| & source file \\
    |childdoc.def| & definition file \\
    |cdocsamp.tex| & sample main file \\
    |cdocsch1.tex| & sample include file \\
    |cdocsch2.tex| & sample include file \\
    |cdocspt3.tex| & sample part file \\
    |cdocspt4.tex| & sample part file \\
    |cdocsdrf.tex| & sample redirection file \\
    |cdocsfn1.tex| & sample redirection file \\
    |cdocsfn2.tex| & sample redirection file \\
    |childdoc.pdf| & manual
\end{tabular}
\end{center}
%
The distribution consists of the files
|README.txt|, |childdoc.ins| and |childdoc.dtx|.
%
\begin{itemize}
\item
Run (pdf)\LaTeX{} on |childdoc.dtx|
to compile the manual |childdoc.pdf| (this file).
\item
Run \LaTeX{} on |childdoc.ins| to create the definitions file |childdoc.def|
and the sample |cdocsamp.tex| with include files
|cdocsch1.tex|, |cdocsch2.tex|, |cdocspt3.tex|, |cdocspt4.tex|,
|cdocsdrf.tex|, |cdocsfn1.tex|, |cdocsfn2.tex|.
Then copy the file |childdoc.def| to an appropriate directory of your \LaTeX{}
distribution, e.g.\ \textit{texmf-root}|/tex/latex/childdoc|.
\end{itemize}

%%%%%%%%%%%%%%%%%%%%%%%%%%%%%%%%%%%%%%%%%%%%%%%%%%%%%%%%%%%%%%%%%%%%%%%%%%%%%%%%
\subsection{Related CTAN Packages}

There are several other packages which offer a similar functionality:
%
\begin{itemize}
\item
The packages
\href{http://ctan.org/pkg/docmute}{\textsf{docmute}},
\href{http://ctan.org/pkg/includex}{\textsf{includex}} and
\href{http://ctan.org/pkg/standalone}{\textsf{standalone}}
provide commands to include only the document body of
a child file thus allowing both files to be compiled individually.
\item
The packages \href{http://ctan.org/pkg/subdocs}{\textsf{subdocs}}
and \href{http://ctan.org/pkg/subfiles}{\textsf{subfiles}}
provide structures in which the main and child documents can be
encapsulated and allowing them to be compiled individually.
The inclusion mechanism is different from the conventional |\include|.
\item
The package \href{http://ctan.org/pkg/combine}{\textsf{combine}}
is an elaborate solution to combine several documents into one.
\end{itemize}
%
See also the CTAN topic \href{http://ctan.org/topic/subdocs}{\textsf{subdocs}}
for further related packages.
The present package differs from the above solutions in that
a document structure constructed with the conventional |\include| mechanism
just needs two extra commands at the top of every file
such that all constituent files can be compiled individually.

%%%%%%%%%%%%%%%%%%%%%%%%%%%%%%%%%%%%%%%%%%%%%%%%%%%%%%%%%%%%%%%%%%%%%%%%%%%%%%%%
%\subsection{Feature Suggestions}
%
%The following is a list of features which may be useful for future
%versions of this package:
%%
%\begin{itemize}
%\item
%\ldots
%\end{itemize}

%%%%%%%%%%%%%%%%%%%%%%%%%%%%%%%%%%%%%%%%%%%%%%%%%%%%%%%%%%%%%%%%%%%%%%%%%%%%%%%%
\subsection{Revision History}

%%%%%%%%%%%%%%%%%%%%%%%%%%%%%%%%%%%%%%%%
\paragraph{v2.0:} 2018/12/30

\begin{itemize}
\item
immediate forward processing
\item
added |\childdocby| mechanism
\item
manual restructured
\end{itemize}

%%%%%%%%%%%%%%%%%%%%%%%%%%%%%%%%%%%%%%%%
\paragraph{v1.6:} 2018/01/17

\begin{itemize}
\item
application for development of include files
\item
corrections to manual
\end{itemize}

%%%%%%%%%%%%%%%%%%%%%%%%%%%%%%%%%%%%%%%%
\paragraph{v1.5:} 2017/05/21

\begin{itemize}
\item
more complete structuring introduced
\item
|\childdocof| introduced
\item
|\childdoc| renamed to |\childdocmain|
\item
|\childredirect| renamed to |\childdocforward| and |\childdocforwardprefix|
and functionality expanded
\end{itemize}

%%%%%%%%%%%%%%%%%%%%%%%%%%%%%%%%%%%%%%%%
\paragraph{v1.0:} 2017/04/27

\begin{itemize}
\item
manual and install package
\item
first version published on CTAN
\end{itemize}

%%%%%%%%%%%%%%%%%%%%%%%%%%%%%%%%%%%%%%%%
\paragraph{v0.6:} 2017/04/26

\begin{itemize}
\item
redirection mechanism added
\end{itemize}

%%%%%%%%%%%%%%%%%%%%%%%%%%%%%%%%%%%%%%%%
\paragraph{v0.5:} 2017/04/26

\begin{itemize}
\item
functionality in definition file
\end{itemize}


%%%%%%%%%%%%%%%%%%%%%%%%%%%%%%%%%%%%%%%%%%%%%%%%%%%%%%%%%%%%%%%%%%%%%%%%%%%%%%%%
%%%%%%%%%%%%%%%%%%%%%%%%%%%%%%%%%%%%%%%%%%%%%%%%%%%%%%%%%%%%%%%%%%%%%%%%%%%%%%%%
%%%%%%%%%%%%%%%%%%%%%%%%%%%%%%%%%%%%%%%%%%%%%%%%%%%%%%%%%%%%%%%%%%%%%%%%%%%%%%%%
\appendix

\settowidth\MacroIndent{\rmfamily\scriptsize 000\ }

 \DocInput{childdoc.dtx}

\end{document}
%</driver>
% \fi
%
% %%%%%%%%%%%%%%%%%%%%%%%%%%%%%%%%%%%%%%%%%%%%%%%%%%%%%%%%%%%%%%%%%%%%%%%%%%%%%%
% %%%%%%%%%%%%%%%%%%%%%%%%%%%%%%%%%%%%%%%%%%%%%%%%%%%%%%%%%%%%%%%%%%%%%%%%%%%%%%
% \section{Sample}
%\iffalse
%<*samplemain>
%\fi
%
% The following presents a sample document
% with two chapters, two parts, a title page,
% a compile flag as well as three forwarding files to set the flag.
% It consists of eight |.tex| files:
% \begin{center}
% \begin{tabular}{ll}
% |cdocsamp.tex|&main file\\
% |cdocsch1.tex|&include file for chapter 1\\
% |cdocsch2.tex|&include file for chapter 2\\
% |cdocspt3.tex|&include file for part 3\\
% |cdocspt4.tex|&include file for part 4\\
% |cdocsdrf.tex|&forwarding file for main file in draft mode\\
% |cdocsfi1.tex|&forwarding file for final version of chapter 1\\
% |cdocsfi2.tex|&forwarding file for final version of chapter 2\\
% \end{tabular}
% \end{center}
% Each of the eight files can be compiled directly by the \LaTeX{} compiler.
%
% %%%%%%%%%%%%%%%%%%%%%%%%%%%%%%%%%%%%%%
% \paragraph{Main File.}
%
% The main file is called |cdocsamp.tex|.
%
% Load the \textsf{childdoc} definitions and
% declare the filename for the main document:
%    \begin{macrocode}
\input{childdoc.def}
\childdocmain{}
%    \end{macrocode}

% Optional override for |\version| flag:
%    \begin{macrocode}
%%\ifchilddoc\else\providecommand{\version}{draft}\fi
%    \end{macrocode}

% Define the default values for the |\version| flag
% (|final| for the main file and |draft| for childs):
%    \begin{macrocode}
\ifchilddoc
\providecommand{\version}{draft}
\else
\providecommand{\version}{final}
\fi
%    \end{macrocode}

% Load the standard document class:
%    \begin{macrocode}
\documentclass[12pt]{article}
%    \end{macrocode}

% Start the document body:
%    \begin{macrocode}
\begin{document}
%    \end{macrocode}

% Declare a title page.
% Print title, part of document being processed and version flag:
%    \begin{macrocode}
\addtocounter{page}{-1}
\begin{center}
{\LARGE\bfseries{}childdoc example\par}
\vspace{1cm}
\ifchilddoc
\ifchilddocmanual part\else chapter\fi:
`\childdocname' of `\childdocjob'\par
\else
main document: `\childdocjob'\par
\fi
version: \version\par
\end{center}
\newpage
%    \end{macrocode}

% Manually include selected file,
% otherwise process as usual:
%    \begin{macrocode}
\ifchilddocmanual
\section*{part `\childdocname'}
\input{\childdocname}
\else
%    \end{macrocode}

% Include the two chapters:
%    \begin{macrocode}
\include{cdocsch1}
\include{cdocsch2}
%    \end{macrocode}

% Include the two parts unless only chapters should be displayed:
%    \begin{macrocode}
\ifchilddoc\else
\section{part three}
\input{cdocspt3}
\section{part four}
\input{cdocspt4}
\fi
%    \end{macrocode}

% Process as usual until here:
%    \begin{macrocode}
\fi
%    \end{macrocode}

% End of document body:
%    \begin{macrocode}
\end{document}
%    \end{macrocode}
%\iffalse
%</samplemain>
%\fi
%
% %%%%%%%%%%%%%%%%%%%%%%%%%%%%%%%%%%%%%%
% \paragraph{Chapter Include Files.}
%
% The include files are called |cdocsch1.tex| and |cdocsch2.tex|.
%
%\iffalse
%<*samplechap1|samplechap2>
%\fi

% Optional override for |\version| flag:
%    \begin{macrocode}
%%\providecommand{\version}{final}
%    \end{macrocode}

% Include the main document:
%    \begin{macrocode}
\input{childdoc.def}
\childdocof{cdocsamp}
%    \end{macrocode}

%\iffalse
%</samplechap1|samplechap2>
%\fi
%
%\iffalse
%<*samplechap1>
%\fi
% Some text for chapter 1:
%    \begin{macrocode}
\section{one}
some text in chapter one
%    \end{macrocode}

%\iffalse
%</samplechap1>
%\fi
% Some text for chapter 2:
%\iffalse
%<*samplechap2>
%\fi
%    \begin{macrocode}
\section{two}
more text in chapter two
%    \end{macrocode}

%\iffalse
%</samplechap2>
%\fi
%
% %%%%%%%%%%%%%%%%%%%%%%%%%%%%%%%%%%%%%%
% \paragraph{Part Include Files.}
%
% The include files are called |cdocspt3.tex| and |cdocspt4.tex|.
%
%\iffalse
%<*samplepart3|samplepart4>
%\fi

% Optional override for |\version| flag:
%    \begin{macrocode}
%%\providecommand{\version}{final}
%    \end{macrocode}

% Include the main document:
%    \begin{macrocode}
\input{childdoc.def}
\childdocby{cdocsamp}
%    \end{macrocode}

%\iffalse
%</samplepart3|samplepart4>
%\fi
%
%\iffalse
%<*samplepart3>
%\fi
% Some text for part 3:
%    \begin{macrocode}
some text in part three
%    \end{macrocode}

%\iffalse
%</samplepart3>
%\fi
% Some text for part 4:
%\iffalse
%<*samplepart4>
%\fi
%    \begin{macrocode}
more text in part four
%    \end{macrocode}

%\iffalse
%</samplepart4>
%\fi
%
% %%%%%%%%%%%%%%%%%%%%%%%%%%%%%%%%%%%%%%
% \paragraph{Forwarding for a Complete Draft.}
%
% The following forwarding file |cdocsdrf.tex|
% compiles the main document in draft mode:
%\iffalse
%<*sampledraft>
%\fi
%    \begin{macrocode}
\def\version{draft}
\input{childdoc.def}
\childdocforward{cdocsamp}
%    \end{macrocode}

%\iffalse
%</sampledraft>
%\fi
%
% %%%%%%%%%%%%%%%%%%%%%%%%%%%%%%%%%%%%%%
% \paragraph{Forwarding for Final Version of the Chapters.}
%
% The following forwarding files |cdocsfn1.tex| and |cdocsfn2.tex|
% (with identical content)
% compile the final versions of the child documents
% |cdocsch1.tex| and |cdocsch2.tex|, respectively:
%\iffalse
%<*samplefinal>
%\fi
%    \begin{macrocode}
\def\version{final}
\input{childdoc.def}
\childdocforwardprefix[cdocsamp]{cdocsfn}{cdocsch}
%    \end{macrocode}

%\iffalse
%</samplefinal>
%\fi
%
% %%%%%%%%%%%%%%%%%%%%%%%%%%%%%%%%%%%%%%
% \paragraph{Command Line Processing.}
%
% The following three command lines generate the output files
% |cdocscld|, |cdocscl1| and |cdocscl2|
% which should be identical to
% |cdocsdrf|, |cdocsch1| and |cdocsfn2|, respectively:
% \begin{center}
% \begin{tabular}{l}
% |latex -jobname cdocscld \|\\
% |  "\def\version{draft}\input{childdoc.def}\childdocforward{cdocsamp}"|\\
% |latex -jobname cdocscl1 \|\\
% |  "\input{childdoc.def}\childdocforward[cdocsamp]{cdocsch1}"|\\
% |latex -jobname cdocscl2 \|\\
% |  "\def\version{final}\input{childdoc.def}\childdocforward{cdocsch2}"|
% \end{tabular}
% \end{center}
% Note that the trailing backslash on each first line
% merely continues the input to the second line
% (for convenient cut ant paste).
% Furthermore, the command |latex| can be replaced by any
% of its alternative versions such as |pdflatex|.
%
% %%%%%%%%%%%%%%%%%%%%%%%%%%%%%%%%%%%%%%%%%%%%%%%%%%%%%%%%%%%%%%%%%%%%%%%%%%%%%%
% %%%%%%%%%%%%%%%%%%%%%%%%%%%%%%%%%%%%%%%%%%%%%%%%%%%%%%%%%%%%%%%%%%%%%%%%%%%%%%
% \section{Implementation}
%\iffalse
%<*package>
%\fi
%
% This section describes the definitions file |childdoc.def|.

% The definitions cannot be loaded using |\usepackage| or |\RequirePackage|
% which has a mechanism to prevent loading a style file more than once.
% When loading the definitions by means of |\input|
% multiple instances have to be prevented manually:
%\iffalse
%This code needs to be before the `\ProvidesFile' directive
%which is defined at the beginning of this file.
%Therefore it is also placed there and commented out here.
%</package>
%<*discard>
%\fi
%    \begin{macrocode}
\ifdefined\childdocmain\endinput\fi
%    \end{macrocode}
%\iffalse
%</discard>
%<*package>
%\fi
%
% \macro{\ifchilddoc}
% \macro{\ifchilddocmanual}
% The conditional |\ifchilddoc| tells whether a
% child (true) or main (false) document is being compiled.
% The conditional |\ifchilddocmanual| tells whether
% the |\includeonly| mechanism is used (false) or
% the selection of child files must be performed manually (true).
% The definitions initialise to false:
%    \begin{macrocode}
\newif\ifchilddoc
\newif\ifchilddocmanual
%    \end{macrocode}

% \macro{\childdocname}
% \macro{\childdocjob}
% The macro |\childdocname| stores the name of the main document
% to be compiled. The macro |\childdocjob| stores the name of
% the document on which the \LaTeX{} compiler was originally invoked.
% The content of |\jobname| cannot be compared
% to filenames specified in the source due to different catcodes.
% The following code rescans |\jobname|, stores the result
% in |\childdocname| and saves a copy in |\childdocjob|:
%    \begin{macrocode}
\edef\childdocname{\scantokens\expandafter{\jobname\noexpand}}
\let\childdocjob\childdocname
%    \end{macrocode}

% \macro{\childdocdisable}
% The macro |\childdocdisable| prevents the main file
% from being processed more than once.
% At this stage, the main document command |\childdocmain|
% is assumed to be called once again where it should do nothing.
% Any subsequent call to it should prevent
% a secondary processing of the main document
% It overwrites the forwarding commands
% |\childdocof| and |\childdocforward|
% with empty macros to prevent further inclusions of the main document:
%    \begin{macrocode}
\newcommand{\childdocdisable}
{
  \renewcommand{\childdocmain}[1]{\renewcommand{\childdocmain}[1]{\endinput}}
  \renewcommand{\childdocof}[1]{}
  \renewcommand{\childdocby}[2][]{}
  \renewcommand{\childdocforward}[2][]{}
  \renewcommand{\childdocdisable}{}
}
%    \end{macrocode}

% \macro{\childdocmain}
% The macro |\childdocmain| is to be called at the top of the main file
% with nothing or the main filename (without extension) as argument.
% First, it breaks loops.
% If the argument is not empty and does not match |\childdocname|
% (which is set by the first inclusion of |childdoc.def|),
% |\ifchilddoc| is set to true, |\includeonly| is applied to the child file
% and |\jobname| is set to the main file
% (for proper handling of |.aux| files):
%    \begin{macrocode}
\newcommand{\childdocmain}[1]
{
  \childdocdisable\childdocmain{}
  \if?#1?\else
    \begingroup
      \def\childdoctmp{#1}
      \ifx\childdoctmp\childdocname
        \def\childdoctmp{}
      \else
        \def\childdoctmp
        {
          \childdoctrue
          \includeonly{\childdocname}
          \def\childdocjob{#1}
          \def\jobname{#1}
        }
      \fi
      \expandafter
    \endgroup
    \childdoctmp
  \fi
}
%    \end{macrocode}

% \macro{\childdocof}
% The command |\childdocof| redirects
% compilation to the main file |#1|.
%    \begin{macrocode}
\newcommand{\childdocof}[1]
{
  \childdocdisable
  \childdoctrue
  \includeonly{\childdocname}
  \def\jobname{#1}
  \def\childdocjob{#1}
  \input{#1}
}
%    \end{macrocode}

% \macro{\childdocby}
% The command |\childdocby| ....
%    \begin{macrocode}
\newcommand{\childdocby}[2][]
{
  \childdocdisable
  \childdoctrue
  \childdocmanualtrue
  \if?#1?\else
    \def\jobname{#2}
  \fi
  \def\childdocjob{#2}
  \input{#2}
  \endinput
}
%    \end{macrocode}

% \macro{\childdocforward}
% The command |\childdocforward| redirects
% compilation to the main file or
% (if the optional argument is given) a child file.
% Parameters are set as if the main file
% or a child file starting with |\childdocof| was compiled.
% Then compilation is handed over to the main file:
%    \begin{macrocode}
\newcommand{\childdocforward}[2][]
{
  \begingroup
    \if?#1?
      \def\childdoctmp
      {
        \def\childdocname{#2}
        \def\childdocjob{#2}
        \def\jobname{#2}
        \input{#2}
        \endinput
      }
    \else
      \def\childdoctmp
      {
        \childdocdisable
        \def\childdocname{#2}
        \childdoctrue
        \includeonly{#2}
        \def\childdocjob{#1}
        \def\jobname{#1}
        \input{#1}
        \endinput
      }
    \fi
    \expandafter
  \endgroup
  \childdoctmp
}
%    \end{macrocode}

% \macro{\childdocforwardprefix}
% The command |\childdocforwardprefix| redirects
% compilation to the main or a child file by means of a pattern.
% The prefix |#1| in the current filename is replaced by |#2|
% and the suffix of the current filename is kept
% (it is assumed that the filename does not contain the substring `|~~~|'
% which is used as a delimiter).
% Compilation is handed over to the new file by |\childdocforward|:
%    \begin{macrocode}
\newcommand{\childdocforwardprefix}[3][]
{
  \begingroup
    \def\childdocextract #2##1~~~{\def\childdoctmp{\childdocforward[#1]{#3##1}}}
    \expandafter\childdocextract\childdocname~~~
    \expandafter
  \endgroup
  \childdoctmp
}
%    \end{macrocode}

% \macro{\childdoc}
% The deprecated macro |\childdoc| is a legacy version of |\childdocmain|:
%    \begin{macrocode}
\newcommand{\childdoc}{\childdocmain}
%    \end{macrocode}

% \macro{\childdocredirect}
% The deprecated macro |\childdocredirect| is a legacy version
% of |\childdocforward| and |\childdocforwardprefix|:
%    \begin{macrocode}
\newcommand{\childdocredirect}[2][]
{
  \begingroup
    \if?#1?
      \def\childdoctmp{\childdocforward{#2}}
    \else
      \def\childdoctmp{\childdocforwardprefix{#1}{#2}}
    \fi
    \expandafter
  \endgroup
  \childdoctmp
}
%    \end{macrocode}

%\iffalse
%</package>
%\fi
%
\endinput
\childdocforward[|\textit{main}|]{|\textit{dest}|}"|
\end{center}
%
Here \textit{target} is the name of the output file,
\textit{main} is the name of the main file
and \textit{dest} is the name of the main or child file to be processed
(all filenames without extensions).
The optional argument \textit{main} can be omitted
if \textit{main} matches \textit{dest}.
Optionally, compilation \textit{flags} can be defined via |\def| commands.
This command line makes the \TeX{} engine believe
it is compiling the file \textit{target}
whose content is specified as the latter parameter.
The provided code then forwards the processing to
\textit{main} or \textit{dest} as described in \secref{sec:forward}.

%%%%%%%%%%%%%%%%%%%%%%%%%%%%%%%%%%%%%%%%%%%%%%%%%%%%%%%%%%%%%%%%%%%%%%%%%%%%%%%%
\subsection{Include by Input}
\label{sec:input}

Including child documents by |\include| has some restrictions by design.
Most notably, the content of a child document always occupies
its own set of pages; pages cannot be shared between child documents.
Usually, this behaviour makes perfect sense
because each child document contain an essential part of the document.
However, in some situations it may be desirable to compose
a document from a collection of parts
without having mandatory page breaks between then.
For this case, the package
provides a mechanism to include parts
by |\input| which can also be processed individually.
However, by construction this mechanism
requires manual handling of the content to be output.

%%%%%%%%%%%%%%%%%%%%%%%%%%%%%%%%%%%%%%%%
\DescribeMacro{\ifchilddocmanual}
The main file should be prepared as usual, see \secref{sec:include}.
However, the document body must make a distinction
between processing of an individual part and of the main document, e.g.:
%
\begin{center}
\begin{tabular}{l}
|\ifchilddocmanual|\\
|\input{\childdocname}|\\
|\||else|\\
\textit{document body with }|\input{|\textit{part}|}|\\
|\||fi|
\end{tabular}
\end{center}
%
The conditional |\ifchilddocmanual| is true whenever
a part to be included by |\input| is being compiled,
and the name of the part is stored in |\childdocname|.

%%%%%%%%%%%%%%%%%%%%%%%%%%%%%%%%%%%%%%%%
\DescribeMacro{\childdocby}
Each part to be included by |\input| should start with:
%
\begin{center}
\begin{tabular}{l}
|% \iffalse
%
% childdoc.dtx Copyright (C) 2017-2018 Niklas Beisert
%
% This work may be distributed and/or modified under the
% conditions of the LaTeX Project Public License, either version 1.3
% of this license or (at your option) any later version.
% The latest version of this license is in
%   http://www.latex-project.org/lppl.txt
% and version 1.3 or later is part of all distributions of LaTeX
% version 2005/12/01 or later.
%
% This work has the LPPL maintenance status `maintained'.
%
% The Current Maintainer of this work is Niklas Beisert.
%
% This work consists of the files childdoc.dtx and childdoc.ins
% and the derived files childdoc.def and cdocsamp.tex with
% cdocsch1.tex, cdocsch2.tex, cdocsdrf.tex, cdocsfn1.tex, cdocsfn2.tex.
%
%<package>\ifdefined\childdocmain\endinput\fi
%<package>\ProvidesFile{childdoc.def}[2018/12/30 v2.0 child document driver]
%<samplemain>\ProvidesFile{cdocsamp.tex}[2018/12/30 v2.0 sample for childdoc]
%<*driver>
%\ProvidesFile{childdoc.drv}[2018/12/30 v2.0 childdoc reference manual file]
\PassOptionsToClass{10pt,a4paper}{article}
\documentclass{ltxdoc}

\usepackage[margin=35mm]{geometry}
\usepackage{hyperref}
\usepackage{hyperxmp}
\usepackage[usenames]{color}

\hypersetup{colorlinks=true}
\hypersetup{pdfstartview=FitH}
\hypersetup{pdfpagemode=UseNone}
\hypersetup{pdfsource={}}
\hypersetup{pdflang={en-UK}}
\hypersetup{pdfcopyright={Copyright 2017-2018 Niklas Beisert.
  This work may be distributed and/or modified under the
  conditions of the LaTeX Project Public License, either version 1.3
  of this license or (at your option) any later version.}}
\hypersetup{pdflicenseurl={http://www.latex-project.org/lppl.txt}}
\hypersetup{pdfcontactaddress={ETH Zurich, ITP, HIT K,
  Wolfgang-Pauli-Strasse 27}}
\hypersetup{pdfcontactpostcode={8093}}
\hypersetup{pdfcontactcity={Zurich}}
\hypersetup{pdfcontactcountry={Switzerland}}
\hypersetup{pdfcontactemail={nbeisert@itp.phys.ethz.ch}}
\hypersetup{pdfcontacturl={http://people.phys.ethz.ch/\xmptilde nbeisert/}}

\newcommand{\secref}[1]{\hyperref[#1]{section \ref*{#1}}}

\parskip1ex
\parindent0pt
\let\olditemize\itemize
\def\itemize{\olditemize\parskip0pt}

\begin{document}

\title{The \textsf{childdoc} Package}
\hypersetup{pdftitle={The childdoc Package}}
\author{Niklas Beisert\\[2ex]
  Institut f\"ur Theoretische Physik\\
  Eidgen\"ossische Technische Hochschule Z\"urich\\
  Wolfgang-Pauli-Strasse 27, 8093 Z\"urich, Switzerland\\[1ex]
  \href{mailto:nbeisert@itp.phys.ethz.ch}
  {\texttt{nbeisert@itp.phys.ethz.ch}}}
\hypersetup{pdfauthor={Niklas Beisert}}
\hypersetup{pdfsubject={Manual for the LaTeX2e Package childdoc}}
\date{30 December 2018, \textsf{v2.0}}
\maketitle

\begin{abstract}\noindent
\textsf{childdoc} is a \LaTeXe{} package
that enables the direct compilation
of document sections included by |\include|
to individual files.
\end{abstract}

\begingroup
\parskip0ex
\tableofcontents
\endgroup

%%%%%%%%%%%%%%%%%%%%%%%%%%%%%%%%%%%%%%%%%%%%%%%%%%%%%%%%%%%%%%%%%%%%%%%%%%%%%%%%
%%%%%%%%%%%%%%%%%%%%%%%%%%%%%%%%%%%%%%%%%%%%%%%%%%%%%%%%%%%%%%%%%%%%%%%%%%%%%%%%
\section{Introduction}

\LaTeX{} provides a mechanism to structure a large document (such as a book)
into a main file and several child files (containing the chapters)
using the |\include| command.
This mechanism is beneficial for documents
which span hundreds of pages in order to
make the source file(s) more manageable.
Moreover, compilation can be restricted to
selected child files by means of the |\includeonly| command.
The latter feature can be used to reduce the compilation time while editing
(this was significantly more useful in the earlier days of \LaTeX{})
or to generate a smaller document which is easier to navigate.
Another application of |\includeonly| is to generate
documents consisting of selected parts of the complete document.

However, there are a few drawbacks of the plain |\include| mechanism:
\begin{itemize}
\item
The child files cannot be compiled on their own,
they can only be compiled via the main file.
A naive editing environment
(such as a text editor with an option
to have the current file processed by \LaTeX)
may require one to switch to the main file before compiling;
attempting to compile the child file produces errors.
\item
The main file must be modified (each time)
to adjust the |\includeonly| command
to the present needs. This easily leaves the main file in a messy state.
\item
The generated document will always carry the filename
of the main document. This is inconvenient if
several child files are to be compiled and
to be kept for distribution.
\end{itemize}

The present package provides a simple interface
to make child files individually compilable by \LaTeX{}.
Compiling a child file then has the same effect as compiling
the main file with an |\includeonly| command
to select the appropriate child.
Moreover the generated document will carry the name of the child
rather than the main file.
This resolves all three above issues.

This feature is meant to make the editing of books,
thesis documents and lecture notes somewhat more convenient.
However, the package can also be used efficiently for
composing a series of documents (such as exercise sheets)
which are typically distributed individually.
It then assists the author in generating the individual documents
(potentially in different versions)
as well as a document containing the collected series.
Another application is in developing style files
or other kinds of included material
where compilation of the style file could redirect
to a sample or test file.

%%%%%%%%%%%%%%%%%%%%%%%%%%%%%%%%%%%%%%%%%%%%%%%%%%%%%%%%%%%%%%%%%%%%%%%%%%%%%%%%
%%%%%%%%%%%%%%%%%%%%%%%%%%%%%%%%%%%%%%%%%%%%%%%%%%%%%%%%%%%%%%%%%%%%%%%%%%%%%%%%
\section{Usage}

First of all, the package \textsf{childdoc} is \emph{not} a standard
\LaTeXe{} |.sty| style file! Therefore it needs to be invoked in
a non-standard way.

%%%%%%%%%%%%%%%%%%%%%%%%%%%%%%%%%%%%%%%%%%%%%%%%%%%%%%%%%%%%%%%%%%%%%%%%%%%%%%%%
\subsection{Included Files}
\label{sec:include}

%%%%%%%%%%%%%%%%%%%%%%%%%%%%%%%%%%%%%%%%
\DescribeMacro{\childdocmain}
To use the package, add the commands
\begin{center}
\begin{tabular}{l}
|\input{childdoc.def}|\\
|\childdocmain{}|\\
\end{tabular}
\end{center}
at the very top of the main \LaTeX{} file,
in particular \emph{before} the |\documentclass| statement!
The argument of |\childdocmain| should be left empty
(but it must be present).

%%%%%%%%%%%%%%%%%%%%%%%%%%%%%%%%%%%%%%%%
\DescribeMacro{\childdocof}
Furthermore, add the commands
\begin{center}
\begin{tabular}{l}
|\input{childdoc.def}|\\
|\childdocof{|\textit{main}|}|\\
\end{tabular}
\end{center}
at the top of every child file \textit{child}
which is included by |\include{|\textit{child}|}|
from within the main file
(or at least for those files to be compiled individually).
The argument \textit{main} must be the filename of the main file.

There are a couple of
considerations in setting up the main and child documents:

%%%%%%%%%%%%%%%%%%%%%%%%%%%%%%%%%%%%%%%%
\paragraph{Restrictions.}

Please note the following restrictions:
\begin{itemize}
\item
|\childdocmain| must be called with one argument \textit{main}
to ensure compatibility with earlier version of the package.
It must either be empty (|\childdocmain{}|)
or precisely match the filename of the main file in which it is specified.
See \secref{sec:detection} for further information.
\item
The filename \textit{main} must be specified without the |.tex| extension.
\item
The filename \textit{main} is case sensitive
(even in case-insensitive file systems)
due to internal string comparison.
\item
The argument \textit{main} should be fully expanded, it cannot be a macro.
\item
Subdirectories and special characters should be avoided in filenames.
\item
The command |\childdocmain{|\textit{main}|}| must be followed by a whitespace.
It should not be followed immediately by another command
or by a comment mark `|%|'.
This is because the \TeX{} parser reads the token immediately following
the argument of |\childdocmain| and puts it
at the beginning of every child section;
however, a white\-space is ignored.
\end{itemize}

%%%%%%%%%%%%%%%%%%%%%%%%%%%%%%%%%%%%%%%%
\paragraph{Content of Main File.}

It is advisable to place all content in the child files included by |\include|.
Any output contained in the main file will appear in all child documents
unless suppressed manually;
it cannot be suppressed automatically by the |\includeonly| directive
and thus should normally be avoided.
A method to include some content in the main file
by means of conditional processing is described in \secref{sec:conditional}.

%%%%%%%%%%%%%%%%%%%%%%%%%%%%%%%%%%%%%%%%
\paragraph{Page Numbering.}

When only a part of the document is compiled,
the appropriate numbering of pages
(as well as other status parameters)
is determined from the |.aux| files.
The latter contain information from previous passes.
However this information needs to propagate through
all intermediate child documents.
Therefore the page numbering in child documents may well
be inconsistent until the complete document is compiled at least once.

A useful (if unconventional) way to always ensure a consistent
page numbering is to restart the numbering in each child document
and denote the pages by `\textit{child}|.|\textit{page}'
where \textit{child} represents the chapter/section number of the child file.
This can be achieved by the command
|\numberwithin{page}{|\textit{child}|}|
of the \textsf{amsmath} package
where \textit{child} can be |chapter| or |section|
depending on the chosen structuring.
Alternatively, one can modify the macro |\thepage| appropriately
and reset the counter |page| at the start of each child file.

%%%%%%%%%%%%%%%%%%%%%%%%%%%%%%%%%%%%%%%%%%%%%%%%%%%%%%%%%%%%%%%%%%%%%%%%%%%%%%%%
\subsection{Conditional Processing}
\label{sec:conditional}

The package provides a mechanism to compile different versions
of a document. To customise the versions further some conditional processing
can come in handy to distinguish which version is being compiled.
The package provides two macros to describe the compilation context:

%%%%%%%%%%%%%%%%%%%%%%%%%%%%%%%%%%%%%%%%
\DescribeMacro{\ifchilddoc}
The conditional |\ifchilddoc| distinguishes between the compilation of
child documents and the main document:
%
\begin{center}
|\ifchilddoc |\textit{child-code}| |[|\||else |\textit{main-code}]| \||fi|
\end{center}

%%%%%%%%%%%%%%%%%%%%%%%%%%%%%%%%%%%%%%%%
\DescribeMacro{\childdocname}
\DescribeMacro{\childdocjob}
The macro |\childdocname| contains the filename (without extension)
of the main or child file being processed.
Note that |\childdocjob| will always contain the name of the main file.

%%%%%%%%%%%%%%%%%%%%%%%%%%%%%%%%%%%%%%%%
\paragraph{Title Page.}

Conditional processing can be used to include a title or banner page
in the main document when proper precautions are taken.
Importantly, the code in the main file should ensure that the page counter
(as well as other status parameters which are stored in the |.aux| files)
takes the same value after the conditional processing.
Otherwise the page numbers may take divergent values
depending on which part is compiled.

For example, a title page could be declared by:
%
\begin{center}
\begin{tabular}{l}
|\ifchilddoc\||else|\\
|\addtocounter{page}{-1}|\\
\textit{code for title page}\\
|\newpage|\\
|\||fi|
\end{tabular}
\end{center}
%
A banner page for the child documents can be generated by:
%
\begin{center}
\begin{tabular}{l}
|\ifchilddoc|\\
|\addtocounter{page}{-1}|\\
\textit{code for banner page}\\
|\newpage|\\
|\||fi|
\end{tabular}
\end{center}
%
Here one could write a message such as:
\begin{center}
|This is the part \childdocname{} of \childdocjob{}.|
\end{center}

%%%%%%%%%%%%%%%%%%%%%%%%%%%%%%%%%%%%%%%%%%%%%%%%%%%%%%%%%%%%%%%%%%%%%%%%%%%%%%%%
\subsection{Flags}
\label{sec:flags}

The package makes it easy to generate different versions
of the main or child documents.
To this end compilation flags can be defined
and assigned different default values.
They will be particularly useful in conjunction
with the forwarding mechanism described in \secref{sec:forward}.

For example, it may be useful to have a flag |\version|
which can be set to |draft| or |final|.
The document source will contain some conditional code
depending on the value of |\version|.
Suppose further, the flag should default to |final| for the main file
and to |draft| for child files
which is a natural assignment for editing the document.
This is achieved by placing the following code
in the preamble of the main document
(below the |\childdocmain| directive):
%
\begin{center}
\begin{tabular}{l}
|\ifchilddoc|\\
|\providecommand{\version}{draft}|\\
|\||else|\\
|\providecommand{\version}{final}|\\
|\||fi|
\end{tabular}
\end{center}
%
The definition by |\providecommand| makes sure
that previous definitions are not overwritten.
Further statements |\providecommand{\version}{...}|
can thus be added before the above code to override it.

For the main file, one might add a line
(between |\childdocmain| and the above block)
%
\begin{center}
|%\ifchilddoc\||else\providecommand{\version}{draft}\||fi|
\end{center}
%
which can be uncommented to produce a draft version.
Likewise one can add a line to the very top of a child file
(above the |\childdocof{|\textit{main}|}| directive)
%
\begin{center}
|%\providecommand{\version}{final}|
\end{center}
%
which can be uncommented to produce the final version of this child document.

%%%%%%%%%%%%%%%%%%%%%%%%%%%%%%%%%%%%%%%%%%%%%%%%%%%%%%%%%%%%%%%%%%%%%%%%%%%%%%%%
\subsection{Forwarding}
\label{sec:forward}

Different versions of the main or child documents
using compilation flags as described in \secref{sec:flags}
can be (permanently) stored in different files
for convenient compilation, viewing and distribution.
To this end, the package defines a command
to pass on compilation to a different file:

%%%%%%%%%%%%%%%%%%%%%%%%%%%%%%%%%%%%%%%%
\DescribeMacro{\childdocforward}
The command |\childdocforward| redirects processing to
another source file:
%
\begin{center}
\begin{tabular}{l}
|\input{childdoc.def}|\\
|\childdocforward[|\textit{main}|]{|\textit{dest}|}|\\
\end{tabular}
\end{center}
%
The argument \textit{dest} is the destination file
(without extension).
It should be the main file or one of the child files.
Note that further \textsf{childdoc} directives
such as |\childdocof| and |\childdocforward|
in the indicated file will be processed in this form.
The optional argument \textit{main}
passes on directly to the main file \textit{main}
while pretending to compile the child \textit{dest}.
This form behaves as if \textit{dest}
issues |\childdocof{|\textit{main}|}| right away,
and no further \textsf{childdoc} directives will be processed.

%%%%%%%%%%%%%%%%%%%%%%%%%%%%%%%%%%%%%%%%
\DescribeMacro{\...prefix}
In the alternative form |\childdocforwardprefix|,
%
\begin{center}
\begin{tabular}{l}
|\input{childdoc.def}|\\
|\childdocforwardprefix[|\textit{main}|]{|\textit{prefix}|}{|\textit{dest}|}|
\end{tabular}
\end{center}
%
the destination file is determined by a pattern
depending on the current file:
To make this work, the current file must be called
`{\textit{prefix}\hspace{0.2em}\textit{suffix}}'
with \textit{prefix} matching precisely the argument.
Processing is then passed on to the file
`{\textit{dest}\hspace{0.2em}\textit{suffix}}'.
Surely, the same effect is achieved by
directly specifying the
argument `{\textit{dest}\hspace{0.2em}\textit{suffix}}'
in the first form.
However, that requires to set up a different file
for each child. With the alternative form of the command
all these files can have exactly the same content
which simplifies setting them up and maintaining them.

For example, the following file |draft.tex|
with a compilation flag |\version| as described in \secref{sec:flags}
compiles the main document as a draft:
%
\begin{center}
\begin{tabular}{l}
|\def\version{draft}|\\
|\input{childdoc.def}|\\
|\childdocforward{|\textit{main}|}|
\end{tabular}
\end{center}
%
Likewise, the following files |final|\textit{nn}|.tex|
compile the final version of the child document
|child|\textit{nn}|.tex|:
%
\begin{center}
\begin{tabular}{l}
|\def\version{final}|\\
|\input{childdoc.def}|\\
|\childdocforwardprefix{final}{child}|
\end{tabular}
\end{center}
%

Note that when several versions of a main file and/or of each child file
are to be generated, it may be convenient to set up a |Makefile| or
shell script to automatise the process.

%%%%%%%%%%%%%%%%%%%%%%%%%%%%%%%%%%%%%%%%%%%%%%%%%%%%%%%%%%%%%%%%%%%%%%%%%%%%%%%%
\subsection{Command Line Processing}
\label{sec:commandline}

The effect of redirection files can also be achieved by invoking
the \LaTeX{} compiler with a more elaborate command line.
Most conveniently this should be done as part
of a shell script or a |Makefile|.

When using \textsf{childdoc} in the main file, the following
command lines effectively perform a redirection
(note that depending on the shell being used,
backslashes may have to be doubled: `|\|' $\to$ `|\\|'):
%
\begin{center}
|... -jobname "|\textit{target}|" |\\|"|[\textit{flags}]%
|\input{childdoc.def}\childdocforward[|\textit{main}|]{|\textit{dest}|}"|
\end{center}
%
Here \textit{target} is the name of the output file,
\textit{main} is the name of the main file
and \textit{dest} is the name of the main or child file to be processed
(all filenames without extensions).
The optional argument \textit{main} can be omitted
if \textit{main} matches \textit{dest}.
Optionally, compilation \textit{flags} can be defined via |\def| commands.
This command line makes the \TeX{} engine believe
it is compiling the file \textit{target}
whose content is specified as the latter parameter.
The provided code then forwards the processing to
\textit{main} or \textit{dest} as described in \secref{sec:forward}.

%%%%%%%%%%%%%%%%%%%%%%%%%%%%%%%%%%%%%%%%%%%%%%%%%%%%%%%%%%%%%%%%%%%%%%%%%%%%%%%%
\subsection{Include by Input}
\label{sec:input}

Including child documents by |\include| has some restrictions by design.
Most notably, the content of a child document always occupies
its own set of pages; pages cannot be shared between child documents.
Usually, this behaviour makes perfect sense
because each child document contain an essential part of the document.
However, in some situations it may be desirable to compose
a document from a collection of parts
without having mandatory page breaks between then.
For this case, the package
provides a mechanism to include parts
by |\input| which can also be processed individually.
However, by construction this mechanism
requires manual handling of the content to be output.

%%%%%%%%%%%%%%%%%%%%%%%%%%%%%%%%%%%%%%%%
\DescribeMacro{\ifchilddocmanual}
The main file should be prepared as usual, see \secref{sec:include}.
However, the document body must make a distinction
between processing of an individual part and of the main document, e.g.:
%
\begin{center}
\begin{tabular}{l}
|\ifchilddocmanual|\\
|\input{\childdocname}|\\
|\||else|\\
\textit{document body with }|\input{|\textit{part}|}|\\
|\||fi|
\end{tabular}
\end{center}
%
The conditional |\ifchilddocmanual| is true whenever
a part to be included by |\input| is being compiled,
and the name of the part is stored in |\childdocname|.

%%%%%%%%%%%%%%%%%%%%%%%%%%%%%%%%%%%%%%%%
\DescribeMacro{\childdocby}
Each part to be included by |\input| should start with:
%
\begin{center}
\begin{tabular}{l}
|\input{childdoc.def}|\\
|\childdocby{|\textit{main}|}|\\
\end{tabular}
\end{center}
%
The directive |\childdocby| is similar to |\childdocof|
described in \secref{sec:include},
but the subsequent selection of content must be done manually.
To that end, both |\ifchilddoc| and |\ifchilddocmanual|
will be true upon processing of a part,
and the name of the part is stored in |\childdocname|.
Note that |\jobname| will be set to the filename of the current part
so that each part receives an individual |.aux| file
that does not interfere with the |.aux| file(s) of the main document.
This behaviour can be altered by the alternative form
|\childdocby[*]{|\textit{main}|}| (with a non-empty optional argument)
which uses the |.aux| file of the main document
by setting |\jobname| to \textit{main}.

%%%%%%%%%%%%%%%%%%%%%%%%%%%%%%%%%%%%%%%%%%%%%%%%%%%%%%%%%%%%%%%%%%%%%%%%%%%%%%%%
\subsection{Driver Development}
\label{sec:driver}

The \textsf{childdoc} mechanism can also be use for the development
of definition files such as \LaTeX{} styles or classes.
This case differs from the above setup with multiple parts
included by |\include| in that no |\includeonly| should be invoked.
This can be achieved by starting the include file
(before |\ProvidesPackage|) with:
%
\begin{center}
\begin{tabular}{l}
|\input{childdoc.def}|\\
|\childdocforward{|\textit{main}|}|\\
\end{tabular}
\end{center}
%
or alternatively with:
%
\begin{center}
\begin{tabular}{l}
|\input{childdoc.def}|\\
|\childdocby{|\textit{main}|}|\\
\end{tabular}
\end{center}
%
Both forms have slightly different effects as described above.
The main file is prepared as usual, see \secref{sec:include}.

%%%%%%%%%%%%%%%%%%%%%%%%%%%%%%%%%%%%%%%%%%%%%%%%%%%%%%%%%%%%%%%%%%%%%%%%%%%%%%%%
\subsection{Legacy Detection}
\label{sec:detection}

The directive |\childdocmain| in the main file can detect
whether the complete document or merely a child is to be compiled
even without using the directive |\childdocof|.
This method is deprecated because it is less robust
and there is no compelling reason to use it;
it is merely provided for backward compatibility
and it may be removed in future versions.

If the detection mechanism is to be used,
it is mandatory to correctly specify
the filename of the main file as the argument of |\childdocmain|:
%
\begin{center}
\begin{tabular}{l}
|\input{childdoc.def}|\\
|\childdocmain{|\textit{main}|}|\\
\end{tabular}
\end{center}
%
If |\jobname| does not match the argument \textit{main} of |\childdocmain|,
it is assumed that |\jobname| points to the child file to be compiled.
When using |\childdocmain| with the main file specified as argument,
it suffices to start a child file
with just |\input{|\textit{main}|}|
without loading of the package and using |\childdocof|.
If instead all processing is done
with the appropriate \textsf{childdoc} directives,
the argument of \textit{main} of |\childdocmain| can be empty.

An alternative version of the command line processing described
in \secref{sec:commandline} using the detection mechanism reads:
%
\begin{center}
|... -jobname "|\textit{target}|" "|[\textit{flags}]%
[|\def\jobname{|\textit{dest}|}|]|\input{|\textit{main}|}"|
\end{center}

%%%%%%%%%%%%%%%%%%%%%%%%%%%%%%%%%%%%%%%%%%%%%%%%%%%%%%%%%%%%%%%%%%%%%%%%%%%%%%%%
\subsection{Manual Code}
\label{sec:manual}

In case one cannot be certain whether the definitions file |childdoc.def|
is installed on the target \TeX{} distribution
and one prefers not to ship it,
it is conceivable to paste a few relevant commands into the sources.

To that end, drop all statements |\input{childdoc.def}|
and perform the replacements as outlined below.
Instead of |\childdocmain{|\textit{main}|}| add the following code
to the top of the main file:
%
\begin{center}
\begin{tabular}{l}
|\||ifdefined\childdocname\endinput\||fi\newif\ifchilddoc|\\
|\edef\childdocname{\scantokens\expandafter{\jobname\noexpand}}|\\
|\def\childdocmain{|\textit{main}|}\||ifx\childdocmain\childdocname\||else|\\
|\childdoctrue\includeonly{\childdocname}\let\jobname\childdocmain\||fi|\\
\end{tabular}
\end{center}
%
Instead of |\childdocof{|\textit{main}|}| just include the main file
at the top of each child file:
%
\begin{center}
|\input{|\textit{main}|}|
\end{center}
%
A simple redirection |\childdocforward{|\textit{dest}|}| is achieved by:
%
\begin{center}
|\def\jobname{|\textit{dest}|}\input{\jobname}|
\end{center}
%
The redirection with prefix
|\childdocforwardprefix[|\textit{prefix}|]{|\textit{dest}|}|
is accomplished by:
%
\begin{center}
\begin{tabular}{l}
|{\edef\jobname{\scantokens\expandafter{\jobname\noexpand}}|\\
|\def\redirectjob |\textit{prefix}|#1~~~{\gdef\jobname{|\textit{dest}|#1}}|\\
|\expandafter\redirectjob\jobname~~~}\input{\jobname}|
\end{tabular}
\end{center}

In an alternative approach,
child documents can be compiled by a specific command line
without additional code or specific definitions:
%
\begin{center}
|... -jobname "|\textit{target}|" "|[\textit{flags}]%
|\includeonly{|\textit{dest}|}\input{|\textit{main}|}"|
\end{center}
%

%%%%%%%%%%%%%%%%%%%%%%%%%%%%%%%%%%%%%%%%%%%%%%%%%%%%%%%%%%%%%%%%%%%%%%%%%%%%%%%%
%%%%%%%%%%%%%%%%%%%%%%%%%%%%%%%%%%%%%%%%%%%%%%%%%%%%%%%%%%%%%%%%%%%%%%%%%%%%%%%%
\section{Information}

%%%%%%%%%%%%%%%%%%%%%%%%%%%%%%%%%%%%%%%%%%%%%%%%%%%%%%%%%%%%%%%%%%%%%%%%%%%%%%%%
\subsection{Copyright}

Copyright \copyright{} 2017--2018 Niklas Beisert

This work may be distributed and/or modified under the
conditions of the \LaTeX{} Project Public License, either version 1.3
of this license or (at your option) any later version.
The latest version of this license is in
  \url{http://www.latex-project.org/lppl.txt}
and version 1.3 or later is part of all distributions of \LaTeX{}
version 2005/12/01 or later.

This work has the LPPL maintenance status `maintained'.

The Current Maintainer of this work is Niklas Beisert.

This work consists of the files |README.txt|, |childdoc.ins| and |childdoc.dtx|
as well as the derived files |childdoc.def|, |cdocsamp.tex|
with |cdocsch1.tex|, |cdocsch2.tex|, |cdocspt3.tex|, |cdocspt4.tex|,
|cdocsdrf.tex|, |cdocsfn1.tex|, |cdocsfn2.tex|
as well as |childdoc.pdf|.

%%%%%%%%%%%%%%%%%%%%%%%%%%%%%%%%%%%%%%%%%%%%%%%%%%%%%%%%%%%%%%%%%%%%%%%%%%%%%%%%
\subsection{Files and Installation}

The package consists of the files:
%
\begin{center}
\begin{tabular}{ll}
    |README.txt|   & readme file \\
    |childdoc.ins| & installation file \\
    |childdoc.dtx| & source file \\
    |childdoc.def| & definition file \\
    |cdocsamp.tex| & sample main file \\
    |cdocsch1.tex| & sample include file \\
    |cdocsch2.tex| & sample include file \\
    |cdocspt3.tex| & sample part file \\
    |cdocspt4.tex| & sample part file \\
    |cdocsdrf.tex| & sample redirection file \\
    |cdocsfn1.tex| & sample redirection file \\
    |cdocsfn2.tex| & sample redirection file \\
    |childdoc.pdf| & manual
\end{tabular}
\end{center}
%
The distribution consists of the files
|README.txt|, |childdoc.ins| and |childdoc.dtx|.
%
\begin{itemize}
\item
Run (pdf)\LaTeX{} on |childdoc.dtx|
to compile the manual |childdoc.pdf| (this file).
\item
Run \LaTeX{} on |childdoc.ins| to create the definitions file |childdoc.def|
and the sample |cdocsamp.tex| with include files
|cdocsch1.tex|, |cdocsch2.tex|, |cdocspt3.tex|, |cdocspt4.tex|,
|cdocsdrf.tex|, |cdocsfn1.tex|, |cdocsfn2.tex|.
Then copy the file |childdoc.def| to an appropriate directory of your \LaTeX{}
distribution, e.g.\ \textit{texmf-root}|/tex/latex/childdoc|.
\end{itemize}

%%%%%%%%%%%%%%%%%%%%%%%%%%%%%%%%%%%%%%%%%%%%%%%%%%%%%%%%%%%%%%%%%%%%%%%%%%%%%%%%
\subsection{Related CTAN Packages}

There are several other packages which offer a similar functionality:
%
\begin{itemize}
\item
The packages
\href{http://ctan.org/pkg/docmute}{\textsf{docmute}},
\href{http://ctan.org/pkg/includex}{\textsf{includex}} and
\href{http://ctan.org/pkg/standalone}{\textsf{standalone}}
provide commands to include only the document body of
a child file thus allowing both files to be compiled individually.
\item
The packages \href{http://ctan.org/pkg/subdocs}{\textsf{subdocs}}
and \href{http://ctan.org/pkg/subfiles}{\textsf{subfiles}}
provide structures in which the main and child documents can be
encapsulated and allowing them to be compiled individually.
The inclusion mechanism is different from the conventional |\include|.
\item
The package \href{http://ctan.org/pkg/combine}{\textsf{combine}}
is an elaborate solution to combine several documents into one.
\end{itemize}
%
See also the CTAN topic \href{http://ctan.org/topic/subdocs}{\textsf{subdocs}}
for further related packages.
The present package differs from the above solutions in that
a document structure constructed with the conventional |\include| mechanism
just needs two extra commands at the top of every file
such that all constituent files can be compiled individually.

%%%%%%%%%%%%%%%%%%%%%%%%%%%%%%%%%%%%%%%%%%%%%%%%%%%%%%%%%%%%%%%%%%%%%%%%%%%%%%%%
%\subsection{Feature Suggestions}
%
%The following is a list of features which may be useful for future
%versions of this package:
%%
%\begin{itemize}
%\item
%\ldots
%\end{itemize}

%%%%%%%%%%%%%%%%%%%%%%%%%%%%%%%%%%%%%%%%%%%%%%%%%%%%%%%%%%%%%%%%%%%%%%%%%%%%%%%%
\subsection{Revision History}

%%%%%%%%%%%%%%%%%%%%%%%%%%%%%%%%%%%%%%%%
\paragraph{v2.0:} 2018/12/30

\begin{itemize}
\item
immediate forward processing
\item
added |\childdocby| mechanism
\item
manual restructured
\end{itemize}

%%%%%%%%%%%%%%%%%%%%%%%%%%%%%%%%%%%%%%%%
\paragraph{v1.6:} 2018/01/17

\begin{itemize}
\item
application for development of include files
\item
corrections to manual
\end{itemize}

%%%%%%%%%%%%%%%%%%%%%%%%%%%%%%%%%%%%%%%%
\paragraph{v1.5:} 2017/05/21

\begin{itemize}
\item
more complete structuring introduced
\item
|\childdocof| introduced
\item
|\childdoc| renamed to |\childdocmain|
\item
|\childredirect| renamed to |\childdocforward| and |\childdocforwardprefix|
and functionality expanded
\end{itemize}

%%%%%%%%%%%%%%%%%%%%%%%%%%%%%%%%%%%%%%%%
\paragraph{v1.0:} 2017/04/27

\begin{itemize}
\item
manual and install package
\item
first version published on CTAN
\end{itemize}

%%%%%%%%%%%%%%%%%%%%%%%%%%%%%%%%%%%%%%%%
\paragraph{v0.6:} 2017/04/26

\begin{itemize}
\item
redirection mechanism added
\end{itemize}

%%%%%%%%%%%%%%%%%%%%%%%%%%%%%%%%%%%%%%%%
\paragraph{v0.5:} 2017/04/26

\begin{itemize}
\item
functionality in definition file
\end{itemize}


%%%%%%%%%%%%%%%%%%%%%%%%%%%%%%%%%%%%%%%%%%%%%%%%%%%%%%%%%%%%%%%%%%%%%%%%%%%%%%%%
%%%%%%%%%%%%%%%%%%%%%%%%%%%%%%%%%%%%%%%%%%%%%%%%%%%%%%%%%%%%%%%%%%%%%%%%%%%%%%%%
%%%%%%%%%%%%%%%%%%%%%%%%%%%%%%%%%%%%%%%%%%%%%%%%%%%%%%%%%%%%%%%%%%%%%%%%%%%%%%%%
\appendix

\settowidth\MacroIndent{\rmfamily\scriptsize 000\ }

 \DocInput{childdoc.dtx}

\end{document}
%</driver>
% \fi
%
% %%%%%%%%%%%%%%%%%%%%%%%%%%%%%%%%%%%%%%%%%%%%%%%%%%%%%%%%%%%%%%%%%%%%%%%%%%%%%%
% %%%%%%%%%%%%%%%%%%%%%%%%%%%%%%%%%%%%%%%%%%%%%%%%%%%%%%%%%%%%%%%%%%%%%%%%%%%%%%
% \section{Sample}
%\iffalse
%<*samplemain>
%\fi
%
% The following presents a sample document
% with two chapters, two parts, a title page,
% a compile flag as well as three forwarding files to set the flag.
% It consists of eight |.tex| files:
% \begin{center}
% \begin{tabular}{ll}
% |cdocsamp.tex|&main file\\
% |cdocsch1.tex|&include file for chapter 1\\
% |cdocsch2.tex|&include file for chapter 2\\
% |cdocspt3.tex|&include file for part 3\\
% |cdocspt4.tex|&include file for part 4\\
% |cdocsdrf.tex|&forwarding file for main file in draft mode\\
% |cdocsfi1.tex|&forwarding file for final version of chapter 1\\
% |cdocsfi2.tex|&forwarding file for final version of chapter 2\\
% \end{tabular}
% \end{center}
% Each of the eight files can be compiled directly by the \LaTeX{} compiler.
%
% %%%%%%%%%%%%%%%%%%%%%%%%%%%%%%%%%%%%%%
% \paragraph{Main File.}
%
% The main file is called |cdocsamp.tex|.
%
% Load the \textsf{childdoc} definitions and
% declare the filename for the main document:
%    \begin{macrocode}
\input{childdoc.def}
\childdocmain{}
%    \end{macrocode}

% Optional override for |\version| flag:
%    \begin{macrocode}
%%\ifchilddoc\else\providecommand{\version}{draft}\fi
%    \end{macrocode}

% Define the default values for the |\version| flag
% (|final| for the main file and |draft| for childs):
%    \begin{macrocode}
\ifchilddoc
\providecommand{\version}{draft}
\else
\providecommand{\version}{final}
\fi
%    \end{macrocode}

% Load the standard document class:
%    \begin{macrocode}
\documentclass[12pt]{article}
%    \end{macrocode}

% Start the document body:
%    \begin{macrocode}
\begin{document}
%    \end{macrocode}

% Declare a title page.
% Print title, part of document being processed and version flag:
%    \begin{macrocode}
\addtocounter{page}{-1}
\begin{center}
{\LARGE\bfseries{}childdoc example\par}
\vspace{1cm}
\ifchilddoc
\ifchilddocmanual part\else chapter\fi:
`\childdocname' of `\childdocjob'\par
\else
main document: `\childdocjob'\par
\fi
version: \version\par
\end{center}
\newpage
%    \end{macrocode}

% Manually include selected file,
% otherwise process as usual:
%    \begin{macrocode}
\ifchilddocmanual
\section*{part `\childdocname'}
\input{\childdocname}
\else
%    \end{macrocode}

% Include the two chapters:
%    \begin{macrocode}
\include{cdocsch1}
\include{cdocsch2}
%    \end{macrocode}

% Include the two parts unless only chapters should be displayed:
%    \begin{macrocode}
\ifchilddoc\else
\section{part three}
\input{cdocspt3}
\section{part four}
\input{cdocspt4}
\fi
%    \end{macrocode}

% Process as usual until here:
%    \begin{macrocode}
\fi
%    \end{macrocode}

% End of document body:
%    \begin{macrocode}
\end{document}
%    \end{macrocode}
%\iffalse
%</samplemain>
%\fi
%
% %%%%%%%%%%%%%%%%%%%%%%%%%%%%%%%%%%%%%%
% \paragraph{Chapter Include Files.}
%
% The include files are called |cdocsch1.tex| and |cdocsch2.tex|.
%
%\iffalse
%<*samplechap1|samplechap2>
%\fi

% Optional override for |\version| flag:
%    \begin{macrocode}
%%\providecommand{\version}{final}
%    \end{macrocode}

% Include the main document:
%    \begin{macrocode}
\input{childdoc.def}
\childdocof{cdocsamp}
%    \end{macrocode}

%\iffalse
%</samplechap1|samplechap2>
%\fi
%
%\iffalse
%<*samplechap1>
%\fi
% Some text for chapter 1:
%    \begin{macrocode}
\section{one}
some text in chapter one
%    \end{macrocode}

%\iffalse
%</samplechap1>
%\fi
% Some text for chapter 2:
%\iffalse
%<*samplechap2>
%\fi
%    \begin{macrocode}
\section{two}
more text in chapter two
%    \end{macrocode}

%\iffalse
%</samplechap2>
%\fi
%
% %%%%%%%%%%%%%%%%%%%%%%%%%%%%%%%%%%%%%%
% \paragraph{Part Include Files.}
%
% The include files are called |cdocspt3.tex| and |cdocspt4.tex|.
%
%\iffalse
%<*samplepart3|samplepart4>
%\fi

% Optional override for |\version| flag:
%    \begin{macrocode}
%%\providecommand{\version}{final}
%    \end{macrocode}

% Include the main document:
%    \begin{macrocode}
\input{childdoc.def}
\childdocby{cdocsamp}
%    \end{macrocode}

%\iffalse
%</samplepart3|samplepart4>
%\fi
%
%\iffalse
%<*samplepart3>
%\fi
% Some text for part 3:
%    \begin{macrocode}
some text in part three
%    \end{macrocode}

%\iffalse
%</samplepart3>
%\fi
% Some text for part 4:
%\iffalse
%<*samplepart4>
%\fi
%    \begin{macrocode}
more text in part four
%    \end{macrocode}

%\iffalse
%</samplepart4>
%\fi
%
% %%%%%%%%%%%%%%%%%%%%%%%%%%%%%%%%%%%%%%
% \paragraph{Forwarding for a Complete Draft.}
%
% The following forwarding file |cdocsdrf.tex|
% compiles the main document in draft mode:
%\iffalse
%<*sampledraft>
%\fi
%    \begin{macrocode}
\def\version{draft}
\input{childdoc.def}
\childdocforward{cdocsamp}
%    \end{macrocode}

%\iffalse
%</sampledraft>
%\fi
%
% %%%%%%%%%%%%%%%%%%%%%%%%%%%%%%%%%%%%%%
% \paragraph{Forwarding for Final Version of the Chapters.}
%
% The following forwarding files |cdocsfn1.tex| and |cdocsfn2.tex|
% (with identical content)
% compile the final versions of the child documents
% |cdocsch1.tex| and |cdocsch2.tex|, respectively:
%\iffalse
%<*samplefinal>
%\fi
%    \begin{macrocode}
\def\version{final}
\input{childdoc.def}
\childdocforwardprefix[cdocsamp]{cdocsfn}{cdocsch}
%    \end{macrocode}

%\iffalse
%</samplefinal>
%\fi
%
% %%%%%%%%%%%%%%%%%%%%%%%%%%%%%%%%%%%%%%
% \paragraph{Command Line Processing.}
%
% The following three command lines generate the output files
% |cdocscld|, |cdocscl1| and |cdocscl2|
% which should be identical to
% |cdocsdrf|, |cdocsch1| and |cdocsfn2|, respectively:
% \begin{center}
% \begin{tabular}{l}
% |latex -jobname cdocscld \|\\
% |  "\def\version{draft}\input{childdoc.def}\childdocforward{cdocsamp}"|\\
% |latex -jobname cdocscl1 \|\\
% |  "\input{childdoc.def}\childdocforward[cdocsamp]{cdocsch1}"|\\
% |latex -jobname cdocscl2 \|\\
% |  "\def\version{final}\input{childdoc.def}\childdocforward{cdocsch2}"|
% \end{tabular}
% \end{center}
% Note that the trailing backslash on each first line
% merely continues the input to the second line
% (for convenient cut ant paste).
% Furthermore, the command |latex| can be replaced by any
% of its alternative versions such as |pdflatex|.
%
% %%%%%%%%%%%%%%%%%%%%%%%%%%%%%%%%%%%%%%%%%%%%%%%%%%%%%%%%%%%%%%%%%%%%%%%%%%%%%%
% %%%%%%%%%%%%%%%%%%%%%%%%%%%%%%%%%%%%%%%%%%%%%%%%%%%%%%%%%%%%%%%%%%%%%%%%%%%%%%
% \section{Implementation}
%\iffalse
%<*package>
%\fi
%
% This section describes the definitions file |childdoc.def|.

% The definitions cannot be loaded using |\usepackage| or |\RequirePackage|
% which has a mechanism to prevent loading a style file more than once.
% When loading the definitions by means of |\input|
% multiple instances have to be prevented manually:
%\iffalse
%This code needs to be before the `\ProvidesFile' directive
%which is defined at the beginning of this file.
%Therefore it is also placed there and commented out here.
%</package>
%<*discard>
%\fi
%    \begin{macrocode}
\ifdefined\childdocmain\endinput\fi
%    \end{macrocode}
%\iffalse
%</discard>
%<*package>
%\fi
%
% \macro{\ifchilddoc}
% \macro{\ifchilddocmanual}
% The conditional |\ifchilddoc| tells whether a
% child (true) or main (false) document is being compiled.
% The conditional |\ifchilddocmanual| tells whether
% the |\includeonly| mechanism is used (false) or
% the selection of child files must be performed manually (true).
% The definitions initialise to false:
%    \begin{macrocode}
\newif\ifchilddoc
\newif\ifchilddocmanual
%    \end{macrocode}

% \macro{\childdocname}
% \macro{\childdocjob}
% The macro |\childdocname| stores the name of the main document
% to be compiled. The macro |\childdocjob| stores the name of
% the document on which the \LaTeX{} compiler was originally invoked.
% The content of |\jobname| cannot be compared
% to filenames specified in the source due to different catcodes.
% The following code rescans |\jobname|, stores the result
% in |\childdocname| and saves a copy in |\childdocjob|:
%    \begin{macrocode}
\edef\childdocname{\scantokens\expandafter{\jobname\noexpand}}
\let\childdocjob\childdocname
%    \end{macrocode}

% \macro{\childdocdisable}
% The macro |\childdocdisable| prevents the main file
% from being processed more than once.
% At this stage, the main document command |\childdocmain|
% is assumed to be called once again where it should do nothing.
% Any subsequent call to it should prevent
% a secondary processing of the main document
% It overwrites the forwarding commands
% |\childdocof| and |\childdocforward|
% with empty macros to prevent further inclusions of the main document:
%    \begin{macrocode}
\newcommand{\childdocdisable}
{
  \renewcommand{\childdocmain}[1]{\renewcommand{\childdocmain}[1]{\endinput}}
  \renewcommand{\childdocof}[1]{}
  \renewcommand{\childdocby}[2][]{}
  \renewcommand{\childdocforward}[2][]{}
  \renewcommand{\childdocdisable}{}
}
%    \end{macrocode}

% \macro{\childdocmain}
% The macro |\childdocmain| is to be called at the top of the main file
% with nothing or the main filename (without extension) as argument.
% First, it breaks loops.
% If the argument is not empty and does not match |\childdocname|
% (which is set by the first inclusion of |childdoc.def|),
% |\ifchilddoc| is set to true, |\includeonly| is applied to the child file
% and |\jobname| is set to the main file
% (for proper handling of |.aux| files):
%    \begin{macrocode}
\newcommand{\childdocmain}[1]
{
  \childdocdisable\childdocmain{}
  \if?#1?\else
    \begingroup
      \def\childdoctmp{#1}
      \ifx\childdoctmp\childdocname
        \def\childdoctmp{}
      \else
        \def\childdoctmp
        {
          \childdoctrue
          \includeonly{\childdocname}
          \def\childdocjob{#1}
          \def\jobname{#1}
        }
      \fi
      \expandafter
    \endgroup
    \childdoctmp
  \fi
}
%    \end{macrocode}

% \macro{\childdocof}
% The command |\childdocof| redirects
% compilation to the main file |#1|.
%    \begin{macrocode}
\newcommand{\childdocof}[1]
{
  \childdocdisable
  \childdoctrue
  \includeonly{\childdocname}
  \def\jobname{#1}
  \def\childdocjob{#1}
  \input{#1}
}
%    \end{macrocode}

% \macro{\childdocby}
% The command |\childdocby| ....
%    \begin{macrocode}
\newcommand{\childdocby}[2][]
{
  \childdocdisable
  \childdoctrue
  \childdocmanualtrue
  \if?#1?\else
    \def\jobname{#2}
  \fi
  \def\childdocjob{#2}
  \input{#2}
  \endinput
}
%    \end{macrocode}

% \macro{\childdocforward}
% The command |\childdocforward| redirects
% compilation to the main file or
% (if the optional argument is given) a child file.
% Parameters are set as if the main file
% or a child file starting with |\childdocof| was compiled.
% Then compilation is handed over to the main file:
%    \begin{macrocode}
\newcommand{\childdocforward}[2][]
{
  \begingroup
    \if?#1?
      \def\childdoctmp
      {
        \def\childdocname{#2}
        \def\childdocjob{#2}
        \def\jobname{#2}
        \input{#2}
        \endinput
      }
    \else
      \def\childdoctmp
      {
        \childdocdisable
        \def\childdocname{#2}
        \childdoctrue
        \includeonly{#2}
        \def\childdocjob{#1}
        \def\jobname{#1}
        \input{#1}
        \endinput
      }
    \fi
    \expandafter
  \endgroup
  \childdoctmp
}
%    \end{macrocode}

% \macro{\childdocforwardprefix}
% The command |\childdocforwardprefix| redirects
% compilation to the main or a child file by means of a pattern.
% The prefix |#1| in the current filename is replaced by |#2|
% and the suffix of the current filename is kept
% (it is assumed that the filename does not contain the substring `|~~~|'
% which is used as a delimiter).
% Compilation is handed over to the new file by |\childdocforward|:
%    \begin{macrocode}
\newcommand{\childdocforwardprefix}[3][]
{
  \begingroup
    \def\childdocextract #2##1~~~{\def\childdoctmp{\childdocforward[#1]{#3##1}}}
    \expandafter\childdocextract\childdocname~~~
    \expandafter
  \endgroup
  \childdoctmp
}
%    \end{macrocode}

% \macro{\childdoc}
% The deprecated macro |\childdoc| is a legacy version of |\childdocmain|:
%    \begin{macrocode}
\newcommand{\childdoc}{\childdocmain}
%    \end{macrocode}

% \macro{\childdocredirect}
% The deprecated macro |\childdocredirect| is a legacy version
% of |\childdocforward| and |\childdocforwardprefix|:
%    \begin{macrocode}
\newcommand{\childdocredirect}[2][]
{
  \begingroup
    \if?#1?
      \def\childdoctmp{\childdocforward{#2}}
    \else
      \def\childdoctmp{\childdocforwardprefix{#1}{#2}}
    \fi
    \expandafter
  \endgroup
  \childdoctmp
}
%    \end{macrocode}

%\iffalse
%</package>
%\fi
%
\endinput
|\\
|\childdocby{|\textit{main}|}|\\
\end{tabular}
\end{center}
%
The directive |\childdocby| is similar to |\childdocof|
described in \secref{sec:include},
but the subsequent selection of content must be done manually.
To that end, both |\ifchilddoc| and |\ifchilddocmanual|
will be true upon processing of a part,
and the name of the part is stored in |\childdocname|.
Note that |\jobname| will be set to the filename of the current part
so that each part receives an individual |.aux| file
that does not interfere with the |.aux| file(s) of the main document.
This behaviour can be altered by the alternative form
|\childdocby[*]{|\textit{main}|}| (with a non-empty optional argument)
which uses the |.aux| file of the main document
by setting |\jobname| to \textit{main}.

%%%%%%%%%%%%%%%%%%%%%%%%%%%%%%%%%%%%%%%%%%%%%%%%%%%%%%%%%%%%%%%%%%%%%%%%%%%%%%%%
\subsection{Driver Development}
\label{sec:driver}

The \textsf{childdoc} mechanism can also be use for the development
of definition files such as \LaTeX{} styles or classes.
This case differs from the above setup with multiple parts
included by |\include| in that no |\includeonly| should be invoked.
This can be achieved by starting the include file
(before |\ProvidesPackage|) with:
%
\begin{center}
\begin{tabular}{l}
|% \iffalse
%
% childdoc.dtx Copyright (C) 2017-2018 Niklas Beisert
%
% This work may be distributed and/or modified under the
% conditions of the LaTeX Project Public License, either version 1.3
% of this license or (at your option) any later version.
% The latest version of this license is in
%   http://www.latex-project.org/lppl.txt
% and version 1.3 or later is part of all distributions of LaTeX
% version 2005/12/01 or later.
%
% This work has the LPPL maintenance status `maintained'.
%
% The Current Maintainer of this work is Niklas Beisert.
%
% This work consists of the files childdoc.dtx and childdoc.ins
% and the derived files childdoc.def and cdocsamp.tex with
% cdocsch1.tex, cdocsch2.tex, cdocsdrf.tex, cdocsfn1.tex, cdocsfn2.tex.
%
%<package>\ifdefined\childdocmain\endinput\fi
%<package>\ProvidesFile{childdoc.def}[2018/12/30 v2.0 child document driver]
%<samplemain>\ProvidesFile{cdocsamp.tex}[2018/12/30 v2.0 sample for childdoc]
%<*driver>
%\ProvidesFile{childdoc.drv}[2018/12/30 v2.0 childdoc reference manual file]
\PassOptionsToClass{10pt,a4paper}{article}
\documentclass{ltxdoc}

\usepackage[margin=35mm]{geometry}
\usepackage{hyperref}
\usepackage{hyperxmp}
\usepackage[usenames]{color}

\hypersetup{colorlinks=true}
\hypersetup{pdfstartview=FitH}
\hypersetup{pdfpagemode=UseNone}
\hypersetup{pdfsource={}}
\hypersetup{pdflang={en-UK}}
\hypersetup{pdfcopyright={Copyright 2017-2018 Niklas Beisert.
  This work may be distributed and/or modified under the
  conditions of the LaTeX Project Public License, either version 1.3
  of this license or (at your option) any later version.}}
\hypersetup{pdflicenseurl={http://www.latex-project.org/lppl.txt}}
\hypersetup{pdfcontactaddress={ETH Zurich, ITP, HIT K,
  Wolfgang-Pauli-Strasse 27}}
\hypersetup{pdfcontactpostcode={8093}}
\hypersetup{pdfcontactcity={Zurich}}
\hypersetup{pdfcontactcountry={Switzerland}}
\hypersetup{pdfcontactemail={nbeisert@itp.phys.ethz.ch}}
\hypersetup{pdfcontacturl={http://people.phys.ethz.ch/\xmptilde nbeisert/}}

\newcommand{\secref}[1]{\hyperref[#1]{section \ref*{#1}}}

\parskip1ex
\parindent0pt
\let\olditemize\itemize
\def\itemize{\olditemize\parskip0pt}

\begin{document}

\title{The \textsf{childdoc} Package}
\hypersetup{pdftitle={The childdoc Package}}
\author{Niklas Beisert\\[2ex]
  Institut f\"ur Theoretische Physik\\
  Eidgen\"ossische Technische Hochschule Z\"urich\\
  Wolfgang-Pauli-Strasse 27, 8093 Z\"urich, Switzerland\\[1ex]
  \href{mailto:nbeisert@itp.phys.ethz.ch}
  {\texttt{nbeisert@itp.phys.ethz.ch}}}
\hypersetup{pdfauthor={Niklas Beisert}}
\hypersetup{pdfsubject={Manual for the LaTeX2e Package childdoc}}
\date{30 December 2018, \textsf{v2.0}}
\maketitle

\begin{abstract}\noindent
\textsf{childdoc} is a \LaTeXe{} package
that enables the direct compilation
of document sections included by |\include|
to individual files.
\end{abstract}

\begingroup
\parskip0ex
\tableofcontents
\endgroup

%%%%%%%%%%%%%%%%%%%%%%%%%%%%%%%%%%%%%%%%%%%%%%%%%%%%%%%%%%%%%%%%%%%%%%%%%%%%%%%%
%%%%%%%%%%%%%%%%%%%%%%%%%%%%%%%%%%%%%%%%%%%%%%%%%%%%%%%%%%%%%%%%%%%%%%%%%%%%%%%%
\section{Introduction}

\LaTeX{} provides a mechanism to structure a large document (such as a book)
into a main file and several child files (containing the chapters)
using the |\include| command.
This mechanism is beneficial for documents
which span hundreds of pages in order to
make the source file(s) more manageable.
Moreover, compilation can be restricted to
selected child files by means of the |\includeonly| command.
The latter feature can be used to reduce the compilation time while editing
(this was significantly more useful in the earlier days of \LaTeX{})
or to generate a smaller document which is easier to navigate.
Another application of |\includeonly| is to generate
documents consisting of selected parts of the complete document.

However, there are a few drawbacks of the plain |\include| mechanism:
\begin{itemize}
\item
The child files cannot be compiled on their own,
they can only be compiled via the main file.
A naive editing environment
(such as a text editor with an option
to have the current file processed by \LaTeX)
may require one to switch to the main file before compiling;
attempting to compile the child file produces errors.
\item
The main file must be modified (each time)
to adjust the |\includeonly| command
to the present needs. This easily leaves the main file in a messy state.
\item
The generated document will always carry the filename
of the main document. This is inconvenient if
several child files are to be compiled and
to be kept for distribution.
\end{itemize}

The present package provides a simple interface
to make child files individually compilable by \LaTeX{}.
Compiling a child file then has the same effect as compiling
the main file with an |\includeonly| command
to select the appropriate child.
Moreover the generated document will carry the name of the child
rather than the main file.
This resolves all three above issues.

This feature is meant to make the editing of books,
thesis documents and lecture notes somewhat more convenient.
However, the package can also be used efficiently for
composing a series of documents (such as exercise sheets)
which are typically distributed individually.
It then assists the author in generating the individual documents
(potentially in different versions)
as well as a document containing the collected series.
Another application is in developing style files
or other kinds of included material
where compilation of the style file could redirect
to a sample or test file.

%%%%%%%%%%%%%%%%%%%%%%%%%%%%%%%%%%%%%%%%%%%%%%%%%%%%%%%%%%%%%%%%%%%%%%%%%%%%%%%%
%%%%%%%%%%%%%%%%%%%%%%%%%%%%%%%%%%%%%%%%%%%%%%%%%%%%%%%%%%%%%%%%%%%%%%%%%%%%%%%%
\section{Usage}

First of all, the package \textsf{childdoc} is \emph{not} a standard
\LaTeXe{} |.sty| style file! Therefore it needs to be invoked in
a non-standard way.

%%%%%%%%%%%%%%%%%%%%%%%%%%%%%%%%%%%%%%%%%%%%%%%%%%%%%%%%%%%%%%%%%%%%%%%%%%%%%%%%
\subsection{Included Files}
\label{sec:include}

%%%%%%%%%%%%%%%%%%%%%%%%%%%%%%%%%%%%%%%%
\DescribeMacro{\childdocmain}
To use the package, add the commands
\begin{center}
\begin{tabular}{l}
|\input{childdoc.def}|\\
|\childdocmain{}|\\
\end{tabular}
\end{center}
at the very top of the main \LaTeX{} file,
in particular \emph{before} the |\documentclass| statement!
The argument of |\childdocmain| should be left empty
(but it must be present).

%%%%%%%%%%%%%%%%%%%%%%%%%%%%%%%%%%%%%%%%
\DescribeMacro{\childdocof}
Furthermore, add the commands
\begin{center}
\begin{tabular}{l}
|\input{childdoc.def}|\\
|\childdocof{|\textit{main}|}|\\
\end{tabular}
\end{center}
at the top of every child file \textit{child}
which is included by |\include{|\textit{child}|}|
from within the main file
(or at least for those files to be compiled individually).
The argument \textit{main} must be the filename of the main file.

There are a couple of
considerations in setting up the main and child documents:

%%%%%%%%%%%%%%%%%%%%%%%%%%%%%%%%%%%%%%%%
\paragraph{Restrictions.}

Please note the following restrictions:
\begin{itemize}
\item
|\childdocmain| must be called with one argument \textit{main}
to ensure compatibility with earlier version of the package.
It must either be empty (|\childdocmain{}|)
or precisely match the filename of the main file in which it is specified.
See \secref{sec:detection} for further information.
\item
The filename \textit{main} must be specified without the |.tex| extension.
\item
The filename \textit{main} is case sensitive
(even in case-insensitive file systems)
due to internal string comparison.
\item
The argument \textit{main} should be fully expanded, it cannot be a macro.
\item
Subdirectories and special characters should be avoided in filenames.
\item
The command |\childdocmain{|\textit{main}|}| must be followed by a whitespace.
It should not be followed immediately by another command
or by a comment mark `|%|'.
This is because the \TeX{} parser reads the token immediately following
the argument of |\childdocmain| and puts it
at the beginning of every child section;
however, a white\-space is ignored.
\end{itemize}

%%%%%%%%%%%%%%%%%%%%%%%%%%%%%%%%%%%%%%%%
\paragraph{Content of Main File.}

It is advisable to place all content in the child files included by |\include|.
Any output contained in the main file will appear in all child documents
unless suppressed manually;
it cannot be suppressed automatically by the |\includeonly| directive
and thus should normally be avoided.
A method to include some content in the main file
by means of conditional processing is described in \secref{sec:conditional}.

%%%%%%%%%%%%%%%%%%%%%%%%%%%%%%%%%%%%%%%%
\paragraph{Page Numbering.}

When only a part of the document is compiled,
the appropriate numbering of pages
(as well as other status parameters)
is determined from the |.aux| files.
The latter contain information from previous passes.
However this information needs to propagate through
all intermediate child documents.
Therefore the page numbering in child documents may well
be inconsistent until the complete document is compiled at least once.

A useful (if unconventional) way to always ensure a consistent
page numbering is to restart the numbering in each child document
and denote the pages by `\textit{child}|.|\textit{page}'
where \textit{child} represents the chapter/section number of the child file.
This can be achieved by the command
|\numberwithin{page}{|\textit{child}|}|
of the \textsf{amsmath} package
where \textit{child} can be |chapter| or |section|
depending on the chosen structuring.
Alternatively, one can modify the macro |\thepage| appropriately
and reset the counter |page| at the start of each child file.

%%%%%%%%%%%%%%%%%%%%%%%%%%%%%%%%%%%%%%%%%%%%%%%%%%%%%%%%%%%%%%%%%%%%%%%%%%%%%%%%
\subsection{Conditional Processing}
\label{sec:conditional}

The package provides a mechanism to compile different versions
of a document. To customise the versions further some conditional processing
can come in handy to distinguish which version is being compiled.
The package provides two macros to describe the compilation context:

%%%%%%%%%%%%%%%%%%%%%%%%%%%%%%%%%%%%%%%%
\DescribeMacro{\ifchilddoc}
The conditional |\ifchilddoc| distinguishes between the compilation of
child documents and the main document:
%
\begin{center}
|\ifchilddoc |\textit{child-code}| |[|\||else |\textit{main-code}]| \||fi|
\end{center}

%%%%%%%%%%%%%%%%%%%%%%%%%%%%%%%%%%%%%%%%
\DescribeMacro{\childdocname}
\DescribeMacro{\childdocjob}
The macro |\childdocname| contains the filename (without extension)
of the main or child file being processed.
Note that |\childdocjob| will always contain the name of the main file.

%%%%%%%%%%%%%%%%%%%%%%%%%%%%%%%%%%%%%%%%
\paragraph{Title Page.}

Conditional processing can be used to include a title or banner page
in the main document when proper precautions are taken.
Importantly, the code in the main file should ensure that the page counter
(as well as other status parameters which are stored in the |.aux| files)
takes the same value after the conditional processing.
Otherwise the page numbers may take divergent values
depending on which part is compiled.

For example, a title page could be declared by:
%
\begin{center}
\begin{tabular}{l}
|\ifchilddoc\||else|\\
|\addtocounter{page}{-1}|\\
\textit{code for title page}\\
|\newpage|\\
|\||fi|
\end{tabular}
\end{center}
%
A banner page for the child documents can be generated by:
%
\begin{center}
\begin{tabular}{l}
|\ifchilddoc|\\
|\addtocounter{page}{-1}|\\
\textit{code for banner page}\\
|\newpage|\\
|\||fi|
\end{tabular}
\end{center}
%
Here one could write a message such as:
\begin{center}
|This is the part \childdocname{} of \childdocjob{}.|
\end{center}

%%%%%%%%%%%%%%%%%%%%%%%%%%%%%%%%%%%%%%%%%%%%%%%%%%%%%%%%%%%%%%%%%%%%%%%%%%%%%%%%
\subsection{Flags}
\label{sec:flags}

The package makes it easy to generate different versions
of the main or child documents.
To this end compilation flags can be defined
and assigned different default values.
They will be particularly useful in conjunction
with the forwarding mechanism described in \secref{sec:forward}.

For example, it may be useful to have a flag |\version|
which can be set to |draft| or |final|.
The document source will contain some conditional code
depending on the value of |\version|.
Suppose further, the flag should default to |final| for the main file
and to |draft| for child files
which is a natural assignment for editing the document.
This is achieved by placing the following code
in the preamble of the main document
(below the |\childdocmain| directive):
%
\begin{center}
\begin{tabular}{l}
|\ifchilddoc|\\
|\providecommand{\version}{draft}|\\
|\||else|\\
|\providecommand{\version}{final}|\\
|\||fi|
\end{tabular}
\end{center}
%
The definition by |\providecommand| makes sure
that previous definitions are not overwritten.
Further statements |\providecommand{\version}{...}|
can thus be added before the above code to override it.

For the main file, one might add a line
(between |\childdocmain| and the above block)
%
\begin{center}
|%\ifchilddoc\||else\providecommand{\version}{draft}\||fi|
\end{center}
%
which can be uncommented to produce a draft version.
Likewise one can add a line to the very top of a child file
(above the |\childdocof{|\textit{main}|}| directive)
%
\begin{center}
|%\providecommand{\version}{final}|
\end{center}
%
which can be uncommented to produce the final version of this child document.

%%%%%%%%%%%%%%%%%%%%%%%%%%%%%%%%%%%%%%%%%%%%%%%%%%%%%%%%%%%%%%%%%%%%%%%%%%%%%%%%
\subsection{Forwarding}
\label{sec:forward}

Different versions of the main or child documents
using compilation flags as described in \secref{sec:flags}
can be (permanently) stored in different files
for convenient compilation, viewing and distribution.
To this end, the package defines a command
to pass on compilation to a different file:

%%%%%%%%%%%%%%%%%%%%%%%%%%%%%%%%%%%%%%%%
\DescribeMacro{\childdocforward}
The command |\childdocforward| redirects processing to
another source file:
%
\begin{center}
\begin{tabular}{l}
|\input{childdoc.def}|\\
|\childdocforward[|\textit{main}|]{|\textit{dest}|}|\\
\end{tabular}
\end{center}
%
The argument \textit{dest} is the destination file
(without extension).
It should be the main file or one of the child files.
Note that further \textsf{childdoc} directives
such as |\childdocof| and |\childdocforward|
in the indicated file will be processed in this form.
The optional argument \textit{main}
passes on directly to the main file \textit{main}
while pretending to compile the child \textit{dest}.
This form behaves as if \textit{dest}
issues |\childdocof{|\textit{main}|}| right away,
and no further \textsf{childdoc} directives will be processed.

%%%%%%%%%%%%%%%%%%%%%%%%%%%%%%%%%%%%%%%%
\DescribeMacro{\...prefix}
In the alternative form |\childdocforwardprefix|,
%
\begin{center}
\begin{tabular}{l}
|\input{childdoc.def}|\\
|\childdocforwardprefix[|\textit{main}|]{|\textit{prefix}|}{|\textit{dest}|}|
\end{tabular}
\end{center}
%
the destination file is determined by a pattern
depending on the current file:
To make this work, the current file must be called
`{\textit{prefix}\hspace{0.2em}\textit{suffix}}'
with \textit{prefix} matching precisely the argument.
Processing is then passed on to the file
`{\textit{dest}\hspace{0.2em}\textit{suffix}}'.
Surely, the same effect is achieved by
directly specifying the
argument `{\textit{dest}\hspace{0.2em}\textit{suffix}}'
in the first form.
However, that requires to set up a different file
for each child. With the alternative form of the command
all these files can have exactly the same content
which simplifies setting them up and maintaining them.

For example, the following file |draft.tex|
with a compilation flag |\version| as described in \secref{sec:flags}
compiles the main document as a draft:
%
\begin{center}
\begin{tabular}{l}
|\def\version{draft}|\\
|\input{childdoc.def}|\\
|\childdocforward{|\textit{main}|}|
\end{tabular}
\end{center}
%
Likewise, the following files |final|\textit{nn}|.tex|
compile the final version of the child document
|child|\textit{nn}|.tex|:
%
\begin{center}
\begin{tabular}{l}
|\def\version{final}|\\
|\input{childdoc.def}|\\
|\childdocforwardprefix{final}{child}|
\end{tabular}
\end{center}
%

Note that when several versions of a main file and/or of each child file
are to be generated, it may be convenient to set up a |Makefile| or
shell script to automatise the process.

%%%%%%%%%%%%%%%%%%%%%%%%%%%%%%%%%%%%%%%%%%%%%%%%%%%%%%%%%%%%%%%%%%%%%%%%%%%%%%%%
\subsection{Command Line Processing}
\label{sec:commandline}

The effect of redirection files can also be achieved by invoking
the \LaTeX{} compiler with a more elaborate command line.
Most conveniently this should be done as part
of a shell script or a |Makefile|.

When using \textsf{childdoc} in the main file, the following
command lines effectively perform a redirection
(note that depending on the shell being used,
backslashes may have to be doubled: `|\|' $\to$ `|\\|'):
%
\begin{center}
|... -jobname "|\textit{target}|" |\\|"|[\textit{flags}]%
|\input{childdoc.def}\childdocforward[|\textit{main}|]{|\textit{dest}|}"|
\end{center}
%
Here \textit{target} is the name of the output file,
\textit{main} is the name of the main file
and \textit{dest} is the name of the main or child file to be processed
(all filenames without extensions).
The optional argument \textit{main} can be omitted
if \textit{main} matches \textit{dest}.
Optionally, compilation \textit{flags} can be defined via |\def| commands.
This command line makes the \TeX{} engine believe
it is compiling the file \textit{target}
whose content is specified as the latter parameter.
The provided code then forwards the processing to
\textit{main} or \textit{dest} as described in \secref{sec:forward}.

%%%%%%%%%%%%%%%%%%%%%%%%%%%%%%%%%%%%%%%%%%%%%%%%%%%%%%%%%%%%%%%%%%%%%%%%%%%%%%%%
\subsection{Include by Input}
\label{sec:input}

Including child documents by |\include| has some restrictions by design.
Most notably, the content of a child document always occupies
its own set of pages; pages cannot be shared between child documents.
Usually, this behaviour makes perfect sense
because each child document contain an essential part of the document.
However, in some situations it may be desirable to compose
a document from a collection of parts
without having mandatory page breaks between then.
For this case, the package
provides a mechanism to include parts
by |\input| which can also be processed individually.
However, by construction this mechanism
requires manual handling of the content to be output.

%%%%%%%%%%%%%%%%%%%%%%%%%%%%%%%%%%%%%%%%
\DescribeMacro{\ifchilddocmanual}
The main file should be prepared as usual, see \secref{sec:include}.
However, the document body must make a distinction
between processing of an individual part and of the main document, e.g.:
%
\begin{center}
\begin{tabular}{l}
|\ifchilddocmanual|\\
|\input{\childdocname}|\\
|\||else|\\
\textit{document body with }|\input{|\textit{part}|}|\\
|\||fi|
\end{tabular}
\end{center}
%
The conditional |\ifchilddocmanual| is true whenever
a part to be included by |\input| is being compiled,
and the name of the part is stored in |\childdocname|.

%%%%%%%%%%%%%%%%%%%%%%%%%%%%%%%%%%%%%%%%
\DescribeMacro{\childdocby}
Each part to be included by |\input| should start with:
%
\begin{center}
\begin{tabular}{l}
|\input{childdoc.def}|\\
|\childdocby{|\textit{main}|}|\\
\end{tabular}
\end{center}
%
The directive |\childdocby| is similar to |\childdocof|
described in \secref{sec:include},
but the subsequent selection of content must be done manually.
To that end, both |\ifchilddoc| and |\ifchilddocmanual|
will be true upon processing of a part,
and the name of the part is stored in |\childdocname|.
Note that |\jobname| will be set to the filename of the current part
so that each part receives an individual |.aux| file
that does not interfere with the |.aux| file(s) of the main document.
This behaviour can be altered by the alternative form
|\childdocby[*]{|\textit{main}|}| (with a non-empty optional argument)
which uses the |.aux| file of the main document
by setting |\jobname| to \textit{main}.

%%%%%%%%%%%%%%%%%%%%%%%%%%%%%%%%%%%%%%%%%%%%%%%%%%%%%%%%%%%%%%%%%%%%%%%%%%%%%%%%
\subsection{Driver Development}
\label{sec:driver}

The \textsf{childdoc} mechanism can also be use for the development
of definition files such as \LaTeX{} styles or classes.
This case differs from the above setup with multiple parts
included by |\include| in that no |\includeonly| should be invoked.
This can be achieved by starting the include file
(before |\ProvidesPackage|) with:
%
\begin{center}
\begin{tabular}{l}
|\input{childdoc.def}|\\
|\childdocforward{|\textit{main}|}|\\
\end{tabular}
\end{center}
%
or alternatively with:
%
\begin{center}
\begin{tabular}{l}
|\input{childdoc.def}|\\
|\childdocby{|\textit{main}|}|\\
\end{tabular}
\end{center}
%
Both forms have slightly different effects as described above.
The main file is prepared as usual, see \secref{sec:include}.

%%%%%%%%%%%%%%%%%%%%%%%%%%%%%%%%%%%%%%%%%%%%%%%%%%%%%%%%%%%%%%%%%%%%%%%%%%%%%%%%
\subsection{Legacy Detection}
\label{sec:detection}

The directive |\childdocmain| in the main file can detect
whether the complete document or merely a child is to be compiled
even without using the directive |\childdocof|.
This method is deprecated because it is less robust
and there is no compelling reason to use it;
it is merely provided for backward compatibility
and it may be removed in future versions.

If the detection mechanism is to be used,
it is mandatory to correctly specify
the filename of the main file as the argument of |\childdocmain|:
%
\begin{center}
\begin{tabular}{l}
|\input{childdoc.def}|\\
|\childdocmain{|\textit{main}|}|\\
\end{tabular}
\end{center}
%
If |\jobname| does not match the argument \textit{main} of |\childdocmain|,
it is assumed that |\jobname| points to the child file to be compiled.
When using |\childdocmain| with the main file specified as argument,
it suffices to start a child file
with just |\input{|\textit{main}|}|
without loading of the package and using |\childdocof|.
If instead all processing is done
with the appropriate \textsf{childdoc} directives,
the argument of \textit{main} of |\childdocmain| can be empty.

An alternative version of the command line processing described
in \secref{sec:commandline} using the detection mechanism reads:
%
\begin{center}
|... -jobname "|\textit{target}|" "|[\textit{flags}]%
[|\def\jobname{|\textit{dest}|}|]|\input{|\textit{main}|}"|
\end{center}

%%%%%%%%%%%%%%%%%%%%%%%%%%%%%%%%%%%%%%%%%%%%%%%%%%%%%%%%%%%%%%%%%%%%%%%%%%%%%%%%
\subsection{Manual Code}
\label{sec:manual}

In case one cannot be certain whether the definitions file |childdoc.def|
is installed on the target \TeX{} distribution
and one prefers not to ship it,
it is conceivable to paste a few relevant commands into the sources.

To that end, drop all statements |\input{childdoc.def}|
and perform the replacements as outlined below.
Instead of |\childdocmain{|\textit{main}|}| add the following code
to the top of the main file:
%
\begin{center}
\begin{tabular}{l}
|\||ifdefined\childdocname\endinput\||fi\newif\ifchilddoc|\\
|\edef\childdocname{\scantokens\expandafter{\jobname\noexpand}}|\\
|\def\childdocmain{|\textit{main}|}\||ifx\childdocmain\childdocname\||else|\\
|\childdoctrue\includeonly{\childdocname}\let\jobname\childdocmain\||fi|\\
\end{tabular}
\end{center}
%
Instead of |\childdocof{|\textit{main}|}| just include the main file
at the top of each child file:
%
\begin{center}
|\input{|\textit{main}|}|
\end{center}
%
A simple redirection |\childdocforward{|\textit{dest}|}| is achieved by:
%
\begin{center}
|\def\jobname{|\textit{dest}|}\input{\jobname}|
\end{center}
%
The redirection with prefix
|\childdocforwardprefix[|\textit{prefix}|]{|\textit{dest}|}|
is accomplished by:
%
\begin{center}
\begin{tabular}{l}
|{\edef\jobname{\scantokens\expandafter{\jobname\noexpand}}|\\
|\def\redirectjob |\textit{prefix}|#1~~~{\gdef\jobname{|\textit{dest}|#1}}|\\
|\expandafter\redirectjob\jobname~~~}\input{\jobname}|
\end{tabular}
\end{center}

In an alternative approach,
child documents can be compiled by a specific command line
without additional code or specific definitions:
%
\begin{center}
|... -jobname "|\textit{target}|" "|[\textit{flags}]%
|\includeonly{|\textit{dest}|}\input{|\textit{main}|}"|
\end{center}
%

%%%%%%%%%%%%%%%%%%%%%%%%%%%%%%%%%%%%%%%%%%%%%%%%%%%%%%%%%%%%%%%%%%%%%%%%%%%%%%%%
%%%%%%%%%%%%%%%%%%%%%%%%%%%%%%%%%%%%%%%%%%%%%%%%%%%%%%%%%%%%%%%%%%%%%%%%%%%%%%%%
\section{Information}

%%%%%%%%%%%%%%%%%%%%%%%%%%%%%%%%%%%%%%%%%%%%%%%%%%%%%%%%%%%%%%%%%%%%%%%%%%%%%%%%
\subsection{Copyright}

Copyright \copyright{} 2017--2018 Niklas Beisert

This work may be distributed and/or modified under the
conditions of the \LaTeX{} Project Public License, either version 1.3
of this license or (at your option) any later version.
The latest version of this license is in
  \url{http://www.latex-project.org/lppl.txt}
and version 1.3 or later is part of all distributions of \LaTeX{}
version 2005/12/01 or later.

This work has the LPPL maintenance status `maintained'.

The Current Maintainer of this work is Niklas Beisert.

This work consists of the files |README.txt|, |childdoc.ins| and |childdoc.dtx|
as well as the derived files |childdoc.def|, |cdocsamp.tex|
with |cdocsch1.tex|, |cdocsch2.tex|, |cdocspt3.tex|, |cdocspt4.tex|,
|cdocsdrf.tex|, |cdocsfn1.tex|, |cdocsfn2.tex|
as well as |childdoc.pdf|.

%%%%%%%%%%%%%%%%%%%%%%%%%%%%%%%%%%%%%%%%%%%%%%%%%%%%%%%%%%%%%%%%%%%%%%%%%%%%%%%%
\subsection{Files and Installation}

The package consists of the files:
%
\begin{center}
\begin{tabular}{ll}
    |README.txt|   & readme file \\
    |childdoc.ins| & installation file \\
    |childdoc.dtx| & source file \\
    |childdoc.def| & definition file \\
    |cdocsamp.tex| & sample main file \\
    |cdocsch1.tex| & sample include file \\
    |cdocsch2.tex| & sample include file \\
    |cdocspt3.tex| & sample part file \\
    |cdocspt4.tex| & sample part file \\
    |cdocsdrf.tex| & sample redirection file \\
    |cdocsfn1.tex| & sample redirection file \\
    |cdocsfn2.tex| & sample redirection file \\
    |childdoc.pdf| & manual
\end{tabular}
\end{center}
%
The distribution consists of the files
|README.txt|, |childdoc.ins| and |childdoc.dtx|.
%
\begin{itemize}
\item
Run (pdf)\LaTeX{} on |childdoc.dtx|
to compile the manual |childdoc.pdf| (this file).
\item
Run \LaTeX{} on |childdoc.ins| to create the definitions file |childdoc.def|
and the sample |cdocsamp.tex| with include files
|cdocsch1.tex|, |cdocsch2.tex|, |cdocspt3.tex|, |cdocspt4.tex|,
|cdocsdrf.tex|, |cdocsfn1.tex|, |cdocsfn2.tex|.
Then copy the file |childdoc.def| to an appropriate directory of your \LaTeX{}
distribution, e.g.\ \textit{texmf-root}|/tex/latex/childdoc|.
\end{itemize}

%%%%%%%%%%%%%%%%%%%%%%%%%%%%%%%%%%%%%%%%%%%%%%%%%%%%%%%%%%%%%%%%%%%%%%%%%%%%%%%%
\subsection{Related CTAN Packages}

There are several other packages which offer a similar functionality:
%
\begin{itemize}
\item
The packages
\href{http://ctan.org/pkg/docmute}{\textsf{docmute}},
\href{http://ctan.org/pkg/includex}{\textsf{includex}} and
\href{http://ctan.org/pkg/standalone}{\textsf{standalone}}
provide commands to include only the document body of
a child file thus allowing both files to be compiled individually.
\item
The packages \href{http://ctan.org/pkg/subdocs}{\textsf{subdocs}}
and \href{http://ctan.org/pkg/subfiles}{\textsf{subfiles}}
provide structures in which the main and child documents can be
encapsulated and allowing them to be compiled individually.
The inclusion mechanism is different from the conventional |\include|.
\item
The package \href{http://ctan.org/pkg/combine}{\textsf{combine}}
is an elaborate solution to combine several documents into one.
\end{itemize}
%
See also the CTAN topic \href{http://ctan.org/topic/subdocs}{\textsf{subdocs}}
for further related packages.
The present package differs from the above solutions in that
a document structure constructed with the conventional |\include| mechanism
just needs two extra commands at the top of every file
such that all constituent files can be compiled individually.

%%%%%%%%%%%%%%%%%%%%%%%%%%%%%%%%%%%%%%%%%%%%%%%%%%%%%%%%%%%%%%%%%%%%%%%%%%%%%%%%
%\subsection{Feature Suggestions}
%
%The following is a list of features which may be useful for future
%versions of this package:
%%
%\begin{itemize}
%\item
%\ldots
%\end{itemize}

%%%%%%%%%%%%%%%%%%%%%%%%%%%%%%%%%%%%%%%%%%%%%%%%%%%%%%%%%%%%%%%%%%%%%%%%%%%%%%%%
\subsection{Revision History}

%%%%%%%%%%%%%%%%%%%%%%%%%%%%%%%%%%%%%%%%
\paragraph{v2.0:} 2018/12/30

\begin{itemize}
\item
immediate forward processing
\item
added |\childdocby| mechanism
\item
manual restructured
\end{itemize}

%%%%%%%%%%%%%%%%%%%%%%%%%%%%%%%%%%%%%%%%
\paragraph{v1.6:} 2018/01/17

\begin{itemize}
\item
application for development of include files
\item
corrections to manual
\end{itemize}

%%%%%%%%%%%%%%%%%%%%%%%%%%%%%%%%%%%%%%%%
\paragraph{v1.5:} 2017/05/21

\begin{itemize}
\item
more complete structuring introduced
\item
|\childdocof| introduced
\item
|\childdoc| renamed to |\childdocmain|
\item
|\childredirect| renamed to |\childdocforward| and |\childdocforwardprefix|
and functionality expanded
\end{itemize}

%%%%%%%%%%%%%%%%%%%%%%%%%%%%%%%%%%%%%%%%
\paragraph{v1.0:} 2017/04/27

\begin{itemize}
\item
manual and install package
\item
first version published on CTAN
\end{itemize}

%%%%%%%%%%%%%%%%%%%%%%%%%%%%%%%%%%%%%%%%
\paragraph{v0.6:} 2017/04/26

\begin{itemize}
\item
redirection mechanism added
\end{itemize}

%%%%%%%%%%%%%%%%%%%%%%%%%%%%%%%%%%%%%%%%
\paragraph{v0.5:} 2017/04/26

\begin{itemize}
\item
functionality in definition file
\end{itemize}


%%%%%%%%%%%%%%%%%%%%%%%%%%%%%%%%%%%%%%%%%%%%%%%%%%%%%%%%%%%%%%%%%%%%%%%%%%%%%%%%
%%%%%%%%%%%%%%%%%%%%%%%%%%%%%%%%%%%%%%%%%%%%%%%%%%%%%%%%%%%%%%%%%%%%%%%%%%%%%%%%
%%%%%%%%%%%%%%%%%%%%%%%%%%%%%%%%%%%%%%%%%%%%%%%%%%%%%%%%%%%%%%%%%%%%%%%%%%%%%%%%
\appendix

\settowidth\MacroIndent{\rmfamily\scriptsize 000\ }

 \DocInput{childdoc.dtx}

\end{document}
%</driver>
% \fi
%
% %%%%%%%%%%%%%%%%%%%%%%%%%%%%%%%%%%%%%%%%%%%%%%%%%%%%%%%%%%%%%%%%%%%%%%%%%%%%%%
% %%%%%%%%%%%%%%%%%%%%%%%%%%%%%%%%%%%%%%%%%%%%%%%%%%%%%%%%%%%%%%%%%%%%%%%%%%%%%%
% \section{Sample}
%\iffalse
%<*samplemain>
%\fi
%
% The following presents a sample document
% with two chapters, two parts, a title page,
% a compile flag as well as three forwarding files to set the flag.
% It consists of eight |.tex| files:
% \begin{center}
% \begin{tabular}{ll}
% |cdocsamp.tex|&main file\\
% |cdocsch1.tex|&include file for chapter 1\\
% |cdocsch2.tex|&include file for chapter 2\\
% |cdocspt3.tex|&include file for part 3\\
% |cdocspt4.tex|&include file for part 4\\
% |cdocsdrf.tex|&forwarding file for main file in draft mode\\
% |cdocsfi1.tex|&forwarding file for final version of chapter 1\\
% |cdocsfi2.tex|&forwarding file for final version of chapter 2\\
% \end{tabular}
% \end{center}
% Each of the eight files can be compiled directly by the \LaTeX{} compiler.
%
% %%%%%%%%%%%%%%%%%%%%%%%%%%%%%%%%%%%%%%
% \paragraph{Main File.}
%
% The main file is called |cdocsamp.tex|.
%
% Load the \textsf{childdoc} definitions and
% declare the filename for the main document:
%    \begin{macrocode}
\input{childdoc.def}
\childdocmain{}
%    \end{macrocode}

% Optional override for |\version| flag:
%    \begin{macrocode}
%%\ifchilddoc\else\providecommand{\version}{draft}\fi
%    \end{macrocode}

% Define the default values for the |\version| flag
% (|final| for the main file and |draft| for childs):
%    \begin{macrocode}
\ifchilddoc
\providecommand{\version}{draft}
\else
\providecommand{\version}{final}
\fi
%    \end{macrocode}

% Load the standard document class:
%    \begin{macrocode}
\documentclass[12pt]{article}
%    \end{macrocode}

% Start the document body:
%    \begin{macrocode}
\begin{document}
%    \end{macrocode}

% Declare a title page.
% Print title, part of document being processed and version flag:
%    \begin{macrocode}
\addtocounter{page}{-1}
\begin{center}
{\LARGE\bfseries{}childdoc example\par}
\vspace{1cm}
\ifchilddoc
\ifchilddocmanual part\else chapter\fi:
`\childdocname' of `\childdocjob'\par
\else
main document: `\childdocjob'\par
\fi
version: \version\par
\end{center}
\newpage
%    \end{macrocode}

% Manually include selected file,
% otherwise process as usual:
%    \begin{macrocode}
\ifchilddocmanual
\section*{part `\childdocname'}
\input{\childdocname}
\else
%    \end{macrocode}

% Include the two chapters:
%    \begin{macrocode}
\include{cdocsch1}
\include{cdocsch2}
%    \end{macrocode}

% Include the two parts unless only chapters should be displayed:
%    \begin{macrocode}
\ifchilddoc\else
\section{part three}
\input{cdocspt3}
\section{part four}
\input{cdocspt4}
\fi
%    \end{macrocode}

% Process as usual until here:
%    \begin{macrocode}
\fi
%    \end{macrocode}

% End of document body:
%    \begin{macrocode}
\end{document}
%    \end{macrocode}
%\iffalse
%</samplemain>
%\fi
%
% %%%%%%%%%%%%%%%%%%%%%%%%%%%%%%%%%%%%%%
% \paragraph{Chapter Include Files.}
%
% The include files are called |cdocsch1.tex| and |cdocsch2.tex|.
%
%\iffalse
%<*samplechap1|samplechap2>
%\fi

% Optional override for |\version| flag:
%    \begin{macrocode}
%%\providecommand{\version}{final}
%    \end{macrocode}

% Include the main document:
%    \begin{macrocode}
\input{childdoc.def}
\childdocof{cdocsamp}
%    \end{macrocode}

%\iffalse
%</samplechap1|samplechap2>
%\fi
%
%\iffalse
%<*samplechap1>
%\fi
% Some text for chapter 1:
%    \begin{macrocode}
\section{one}
some text in chapter one
%    \end{macrocode}

%\iffalse
%</samplechap1>
%\fi
% Some text for chapter 2:
%\iffalse
%<*samplechap2>
%\fi
%    \begin{macrocode}
\section{two}
more text in chapter two
%    \end{macrocode}

%\iffalse
%</samplechap2>
%\fi
%
% %%%%%%%%%%%%%%%%%%%%%%%%%%%%%%%%%%%%%%
% \paragraph{Part Include Files.}
%
% The include files are called |cdocspt3.tex| and |cdocspt4.tex|.
%
%\iffalse
%<*samplepart3|samplepart4>
%\fi

% Optional override for |\version| flag:
%    \begin{macrocode}
%%\providecommand{\version}{final}
%    \end{macrocode}

% Include the main document:
%    \begin{macrocode}
\input{childdoc.def}
\childdocby{cdocsamp}
%    \end{macrocode}

%\iffalse
%</samplepart3|samplepart4>
%\fi
%
%\iffalse
%<*samplepart3>
%\fi
% Some text for part 3:
%    \begin{macrocode}
some text in part three
%    \end{macrocode}

%\iffalse
%</samplepart3>
%\fi
% Some text for part 4:
%\iffalse
%<*samplepart4>
%\fi
%    \begin{macrocode}
more text in part four
%    \end{macrocode}

%\iffalse
%</samplepart4>
%\fi
%
% %%%%%%%%%%%%%%%%%%%%%%%%%%%%%%%%%%%%%%
% \paragraph{Forwarding for a Complete Draft.}
%
% The following forwarding file |cdocsdrf.tex|
% compiles the main document in draft mode:
%\iffalse
%<*sampledraft>
%\fi
%    \begin{macrocode}
\def\version{draft}
\input{childdoc.def}
\childdocforward{cdocsamp}
%    \end{macrocode}

%\iffalse
%</sampledraft>
%\fi
%
% %%%%%%%%%%%%%%%%%%%%%%%%%%%%%%%%%%%%%%
% \paragraph{Forwarding for Final Version of the Chapters.}
%
% The following forwarding files |cdocsfn1.tex| and |cdocsfn2.tex|
% (with identical content)
% compile the final versions of the child documents
% |cdocsch1.tex| and |cdocsch2.tex|, respectively:
%\iffalse
%<*samplefinal>
%\fi
%    \begin{macrocode}
\def\version{final}
\input{childdoc.def}
\childdocforwardprefix[cdocsamp]{cdocsfn}{cdocsch}
%    \end{macrocode}

%\iffalse
%</samplefinal>
%\fi
%
% %%%%%%%%%%%%%%%%%%%%%%%%%%%%%%%%%%%%%%
% \paragraph{Command Line Processing.}
%
% The following three command lines generate the output files
% |cdocscld|, |cdocscl1| and |cdocscl2|
% which should be identical to
% |cdocsdrf|, |cdocsch1| and |cdocsfn2|, respectively:
% \begin{center}
% \begin{tabular}{l}
% |latex -jobname cdocscld \|\\
% |  "\def\version{draft}\input{childdoc.def}\childdocforward{cdocsamp}"|\\
% |latex -jobname cdocscl1 \|\\
% |  "\input{childdoc.def}\childdocforward[cdocsamp]{cdocsch1}"|\\
% |latex -jobname cdocscl2 \|\\
% |  "\def\version{final}\input{childdoc.def}\childdocforward{cdocsch2}"|
% \end{tabular}
% \end{center}
% Note that the trailing backslash on each first line
% merely continues the input to the second line
% (for convenient cut ant paste).
% Furthermore, the command |latex| can be replaced by any
% of its alternative versions such as |pdflatex|.
%
% %%%%%%%%%%%%%%%%%%%%%%%%%%%%%%%%%%%%%%%%%%%%%%%%%%%%%%%%%%%%%%%%%%%%%%%%%%%%%%
% %%%%%%%%%%%%%%%%%%%%%%%%%%%%%%%%%%%%%%%%%%%%%%%%%%%%%%%%%%%%%%%%%%%%%%%%%%%%%%
% \section{Implementation}
%\iffalse
%<*package>
%\fi
%
% This section describes the definitions file |childdoc.def|.

% The definitions cannot be loaded using |\usepackage| or |\RequirePackage|
% which has a mechanism to prevent loading a style file more than once.
% When loading the definitions by means of |\input|
% multiple instances have to be prevented manually:
%\iffalse
%This code needs to be before the `\ProvidesFile' directive
%which is defined at the beginning of this file.
%Therefore it is also placed there and commented out here.
%</package>
%<*discard>
%\fi
%    \begin{macrocode}
\ifdefined\childdocmain\endinput\fi
%    \end{macrocode}
%\iffalse
%</discard>
%<*package>
%\fi
%
% \macro{\ifchilddoc}
% \macro{\ifchilddocmanual}
% The conditional |\ifchilddoc| tells whether a
% child (true) or main (false) document is being compiled.
% The conditional |\ifchilddocmanual| tells whether
% the |\includeonly| mechanism is used (false) or
% the selection of child files must be performed manually (true).
% The definitions initialise to false:
%    \begin{macrocode}
\newif\ifchilddoc
\newif\ifchilddocmanual
%    \end{macrocode}

% \macro{\childdocname}
% \macro{\childdocjob}
% The macro |\childdocname| stores the name of the main document
% to be compiled. The macro |\childdocjob| stores the name of
% the document on which the \LaTeX{} compiler was originally invoked.
% The content of |\jobname| cannot be compared
% to filenames specified in the source due to different catcodes.
% The following code rescans |\jobname|, stores the result
% in |\childdocname| and saves a copy in |\childdocjob|:
%    \begin{macrocode}
\edef\childdocname{\scantokens\expandafter{\jobname\noexpand}}
\let\childdocjob\childdocname
%    \end{macrocode}

% \macro{\childdocdisable}
% The macro |\childdocdisable| prevents the main file
% from being processed more than once.
% At this stage, the main document command |\childdocmain|
% is assumed to be called once again where it should do nothing.
% Any subsequent call to it should prevent
% a secondary processing of the main document
% It overwrites the forwarding commands
% |\childdocof| and |\childdocforward|
% with empty macros to prevent further inclusions of the main document:
%    \begin{macrocode}
\newcommand{\childdocdisable}
{
  \renewcommand{\childdocmain}[1]{\renewcommand{\childdocmain}[1]{\endinput}}
  \renewcommand{\childdocof}[1]{}
  \renewcommand{\childdocby}[2][]{}
  \renewcommand{\childdocforward}[2][]{}
  \renewcommand{\childdocdisable}{}
}
%    \end{macrocode}

% \macro{\childdocmain}
% The macro |\childdocmain| is to be called at the top of the main file
% with nothing or the main filename (without extension) as argument.
% First, it breaks loops.
% If the argument is not empty and does not match |\childdocname|
% (which is set by the first inclusion of |childdoc.def|),
% |\ifchilddoc| is set to true, |\includeonly| is applied to the child file
% and |\jobname| is set to the main file
% (for proper handling of |.aux| files):
%    \begin{macrocode}
\newcommand{\childdocmain}[1]
{
  \childdocdisable\childdocmain{}
  \if?#1?\else
    \begingroup
      \def\childdoctmp{#1}
      \ifx\childdoctmp\childdocname
        \def\childdoctmp{}
      \else
        \def\childdoctmp
        {
          \childdoctrue
          \includeonly{\childdocname}
          \def\childdocjob{#1}
          \def\jobname{#1}
        }
      \fi
      \expandafter
    \endgroup
    \childdoctmp
  \fi
}
%    \end{macrocode}

% \macro{\childdocof}
% The command |\childdocof| redirects
% compilation to the main file |#1|.
%    \begin{macrocode}
\newcommand{\childdocof}[1]
{
  \childdocdisable
  \childdoctrue
  \includeonly{\childdocname}
  \def\jobname{#1}
  \def\childdocjob{#1}
  \input{#1}
}
%    \end{macrocode}

% \macro{\childdocby}
% The command |\childdocby| ....
%    \begin{macrocode}
\newcommand{\childdocby}[2][]
{
  \childdocdisable
  \childdoctrue
  \childdocmanualtrue
  \if?#1?\else
    \def\jobname{#2}
  \fi
  \def\childdocjob{#2}
  \input{#2}
  \endinput
}
%    \end{macrocode}

% \macro{\childdocforward}
% The command |\childdocforward| redirects
% compilation to the main file or
% (if the optional argument is given) a child file.
% Parameters are set as if the main file
% or a child file starting with |\childdocof| was compiled.
% Then compilation is handed over to the main file:
%    \begin{macrocode}
\newcommand{\childdocforward}[2][]
{
  \begingroup
    \if?#1?
      \def\childdoctmp
      {
        \def\childdocname{#2}
        \def\childdocjob{#2}
        \def\jobname{#2}
        \input{#2}
        \endinput
      }
    \else
      \def\childdoctmp
      {
        \childdocdisable
        \def\childdocname{#2}
        \childdoctrue
        \includeonly{#2}
        \def\childdocjob{#1}
        \def\jobname{#1}
        \input{#1}
        \endinput
      }
    \fi
    \expandafter
  \endgroup
  \childdoctmp
}
%    \end{macrocode}

% \macro{\childdocforwardprefix}
% The command |\childdocforwardprefix| redirects
% compilation to the main or a child file by means of a pattern.
% The prefix |#1| in the current filename is replaced by |#2|
% and the suffix of the current filename is kept
% (it is assumed that the filename does not contain the substring `|~~~|'
% which is used as a delimiter).
% Compilation is handed over to the new file by |\childdocforward|:
%    \begin{macrocode}
\newcommand{\childdocforwardprefix}[3][]
{
  \begingroup
    \def\childdocextract #2##1~~~{\def\childdoctmp{\childdocforward[#1]{#3##1}}}
    \expandafter\childdocextract\childdocname~~~
    \expandafter
  \endgroup
  \childdoctmp
}
%    \end{macrocode}

% \macro{\childdoc}
% The deprecated macro |\childdoc| is a legacy version of |\childdocmain|:
%    \begin{macrocode}
\newcommand{\childdoc}{\childdocmain}
%    \end{macrocode}

% \macro{\childdocredirect}
% The deprecated macro |\childdocredirect| is a legacy version
% of |\childdocforward| and |\childdocforwardprefix|:
%    \begin{macrocode}
\newcommand{\childdocredirect}[2][]
{
  \begingroup
    \if?#1?
      \def\childdoctmp{\childdocforward{#2}}
    \else
      \def\childdoctmp{\childdocforwardprefix{#1}{#2}}
    \fi
    \expandafter
  \endgroup
  \childdoctmp
}
%    \end{macrocode}

%\iffalse
%</package>
%\fi
%
\endinput
|\\
|\childdocforward{|\textit{main}|}|\\
\end{tabular}
\end{center}
%
or alternatively with:
%
\begin{center}
\begin{tabular}{l}
|% \iffalse
%
% childdoc.dtx Copyright (C) 2017-2018 Niklas Beisert
%
% This work may be distributed and/or modified under the
% conditions of the LaTeX Project Public License, either version 1.3
% of this license or (at your option) any later version.
% The latest version of this license is in
%   http://www.latex-project.org/lppl.txt
% and version 1.3 or later is part of all distributions of LaTeX
% version 2005/12/01 or later.
%
% This work has the LPPL maintenance status `maintained'.
%
% The Current Maintainer of this work is Niklas Beisert.
%
% This work consists of the files childdoc.dtx and childdoc.ins
% and the derived files childdoc.def and cdocsamp.tex with
% cdocsch1.tex, cdocsch2.tex, cdocsdrf.tex, cdocsfn1.tex, cdocsfn2.tex.
%
%<package>\ifdefined\childdocmain\endinput\fi
%<package>\ProvidesFile{childdoc.def}[2018/12/30 v2.0 child document driver]
%<samplemain>\ProvidesFile{cdocsamp.tex}[2018/12/30 v2.0 sample for childdoc]
%<*driver>
%\ProvidesFile{childdoc.drv}[2018/12/30 v2.0 childdoc reference manual file]
\PassOptionsToClass{10pt,a4paper}{article}
\documentclass{ltxdoc}

\usepackage[margin=35mm]{geometry}
\usepackage{hyperref}
\usepackage{hyperxmp}
\usepackage[usenames]{color}

\hypersetup{colorlinks=true}
\hypersetup{pdfstartview=FitH}
\hypersetup{pdfpagemode=UseNone}
\hypersetup{pdfsource={}}
\hypersetup{pdflang={en-UK}}
\hypersetup{pdfcopyright={Copyright 2017-2018 Niklas Beisert.
  This work may be distributed and/or modified under the
  conditions of the LaTeX Project Public License, either version 1.3
  of this license or (at your option) any later version.}}
\hypersetup{pdflicenseurl={http://www.latex-project.org/lppl.txt}}
\hypersetup{pdfcontactaddress={ETH Zurich, ITP, HIT K,
  Wolfgang-Pauli-Strasse 27}}
\hypersetup{pdfcontactpostcode={8093}}
\hypersetup{pdfcontactcity={Zurich}}
\hypersetup{pdfcontactcountry={Switzerland}}
\hypersetup{pdfcontactemail={nbeisert@itp.phys.ethz.ch}}
\hypersetup{pdfcontacturl={http://people.phys.ethz.ch/\xmptilde nbeisert/}}

\newcommand{\secref}[1]{\hyperref[#1]{section \ref*{#1}}}

\parskip1ex
\parindent0pt
\let\olditemize\itemize
\def\itemize{\olditemize\parskip0pt}

\begin{document}

\title{The \textsf{childdoc} Package}
\hypersetup{pdftitle={The childdoc Package}}
\author{Niklas Beisert\\[2ex]
  Institut f\"ur Theoretische Physik\\
  Eidgen\"ossische Technische Hochschule Z\"urich\\
  Wolfgang-Pauli-Strasse 27, 8093 Z\"urich, Switzerland\\[1ex]
  \href{mailto:nbeisert@itp.phys.ethz.ch}
  {\texttt{nbeisert@itp.phys.ethz.ch}}}
\hypersetup{pdfauthor={Niklas Beisert}}
\hypersetup{pdfsubject={Manual for the LaTeX2e Package childdoc}}
\date{30 December 2018, \textsf{v2.0}}
\maketitle

\begin{abstract}\noindent
\textsf{childdoc} is a \LaTeXe{} package
that enables the direct compilation
of document sections included by |\include|
to individual files.
\end{abstract}

\begingroup
\parskip0ex
\tableofcontents
\endgroup

%%%%%%%%%%%%%%%%%%%%%%%%%%%%%%%%%%%%%%%%%%%%%%%%%%%%%%%%%%%%%%%%%%%%%%%%%%%%%%%%
%%%%%%%%%%%%%%%%%%%%%%%%%%%%%%%%%%%%%%%%%%%%%%%%%%%%%%%%%%%%%%%%%%%%%%%%%%%%%%%%
\section{Introduction}

\LaTeX{} provides a mechanism to structure a large document (such as a book)
into a main file and several child files (containing the chapters)
using the |\include| command.
This mechanism is beneficial for documents
which span hundreds of pages in order to
make the source file(s) more manageable.
Moreover, compilation can be restricted to
selected child files by means of the |\includeonly| command.
The latter feature can be used to reduce the compilation time while editing
(this was significantly more useful in the earlier days of \LaTeX{})
or to generate a smaller document which is easier to navigate.
Another application of |\includeonly| is to generate
documents consisting of selected parts of the complete document.

However, there are a few drawbacks of the plain |\include| mechanism:
\begin{itemize}
\item
The child files cannot be compiled on their own,
they can only be compiled via the main file.
A naive editing environment
(such as a text editor with an option
to have the current file processed by \LaTeX)
may require one to switch to the main file before compiling;
attempting to compile the child file produces errors.
\item
The main file must be modified (each time)
to adjust the |\includeonly| command
to the present needs. This easily leaves the main file in a messy state.
\item
The generated document will always carry the filename
of the main document. This is inconvenient if
several child files are to be compiled and
to be kept for distribution.
\end{itemize}

The present package provides a simple interface
to make child files individually compilable by \LaTeX{}.
Compiling a child file then has the same effect as compiling
the main file with an |\includeonly| command
to select the appropriate child.
Moreover the generated document will carry the name of the child
rather than the main file.
This resolves all three above issues.

This feature is meant to make the editing of books,
thesis documents and lecture notes somewhat more convenient.
However, the package can also be used efficiently for
composing a series of documents (such as exercise sheets)
which are typically distributed individually.
It then assists the author in generating the individual documents
(potentially in different versions)
as well as a document containing the collected series.
Another application is in developing style files
or other kinds of included material
where compilation of the style file could redirect
to a sample or test file.

%%%%%%%%%%%%%%%%%%%%%%%%%%%%%%%%%%%%%%%%%%%%%%%%%%%%%%%%%%%%%%%%%%%%%%%%%%%%%%%%
%%%%%%%%%%%%%%%%%%%%%%%%%%%%%%%%%%%%%%%%%%%%%%%%%%%%%%%%%%%%%%%%%%%%%%%%%%%%%%%%
\section{Usage}

First of all, the package \textsf{childdoc} is \emph{not} a standard
\LaTeXe{} |.sty| style file! Therefore it needs to be invoked in
a non-standard way.

%%%%%%%%%%%%%%%%%%%%%%%%%%%%%%%%%%%%%%%%%%%%%%%%%%%%%%%%%%%%%%%%%%%%%%%%%%%%%%%%
\subsection{Included Files}
\label{sec:include}

%%%%%%%%%%%%%%%%%%%%%%%%%%%%%%%%%%%%%%%%
\DescribeMacro{\childdocmain}
To use the package, add the commands
\begin{center}
\begin{tabular}{l}
|\input{childdoc.def}|\\
|\childdocmain{}|\\
\end{tabular}
\end{center}
at the very top of the main \LaTeX{} file,
in particular \emph{before} the |\documentclass| statement!
The argument of |\childdocmain| should be left empty
(but it must be present).

%%%%%%%%%%%%%%%%%%%%%%%%%%%%%%%%%%%%%%%%
\DescribeMacro{\childdocof}
Furthermore, add the commands
\begin{center}
\begin{tabular}{l}
|\input{childdoc.def}|\\
|\childdocof{|\textit{main}|}|\\
\end{tabular}
\end{center}
at the top of every child file \textit{child}
which is included by |\include{|\textit{child}|}|
from within the main file
(or at least for those files to be compiled individually).
The argument \textit{main} must be the filename of the main file.

There are a couple of
considerations in setting up the main and child documents:

%%%%%%%%%%%%%%%%%%%%%%%%%%%%%%%%%%%%%%%%
\paragraph{Restrictions.}

Please note the following restrictions:
\begin{itemize}
\item
|\childdocmain| must be called with one argument \textit{main}
to ensure compatibility with earlier version of the package.
It must either be empty (|\childdocmain{}|)
or precisely match the filename of the main file in which it is specified.
See \secref{sec:detection} for further information.
\item
The filename \textit{main} must be specified without the |.tex| extension.
\item
The filename \textit{main} is case sensitive
(even in case-insensitive file systems)
due to internal string comparison.
\item
The argument \textit{main} should be fully expanded, it cannot be a macro.
\item
Subdirectories and special characters should be avoided in filenames.
\item
The command |\childdocmain{|\textit{main}|}| must be followed by a whitespace.
It should not be followed immediately by another command
or by a comment mark `|%|'.
This is because the \TeX{} parser reads the token immediately following
the argument of |\childdocmain| and puts it
at the beginning of every child section;
however, a white\-space is ignored.
\end{itemize}

%%%%%%%%%%%%%%%%%%%%%%%%%%%%%%%%%%%%%%%%
\paragraph{Content of Main File.}

It is advisable to place all content in the child files included by |\include|.
Any output contained in the main file will appear in all child documents
unless suppressed manually;
it cannot be suppressed automatically by the |\includeonly| directive
and thus should normally be avoided.
A method to include some content in the main file
by means of conditional processing is described in \secref{sec:conditional}.

%%%%%%%%%%%%%%%%%%%%%%%%%%%%%%%%%%%%%%%%
\paragraph{Page Numbering.}

When only a part of the document is compiled,
the appropriate numbering of pages
(as well as other status parameters)
is determined from the |.aux| files.
The latter contain information from previous passes.
However this information needs to propagate through
all intermediate child documents.
Therefore the page numbering in child documents may well
be inconsistent until the complete document is compiled at least once.

A useful (if unconventional) way to always ensure a consistent
page numbering is to restart the numbering in each child document
and denote the pages by `\textit{child}|.|\textit{page}'
where \textit{child} represents the chapter/section number of the child file.
This can be achieved by the command
|\numberwithin{page}{|\textit{child}|}|
of the \textsf{amsmath} package
where \textit{child} can be |chapter| or |section|
depending on the chosen structuring.
Alternatively, one can modify the macro |\thepage| appropriately
and reset the counter |page| at the start of each child file.

%%%%%%%%%%%%%%%%%%%%%%%%%%%%%%%%%%%%%%%%%%%%%%%%%%%%%%%%%%%%%%%%%%%%%%%%%%%%%%%%
\subsection{Conditional Processing}
\label{sec:conditional}

The package provides a mechanism to compile different versions
of a document. To customise the versions further some conditional processing
can come in handy to distinguish which version is being compiled.
The package provides two macros to describe the compilation context:

%%%%%%%%%%%%%%%%%%%%%%%%%%%%%%%%%%%%%%%%
\DescribeMacro{\ifchilddoc}
The conditional |\ifchilddoc| distinguishes between the compilation of
child documents and the main document:
%
\begin{center}
|\ifchilddoc |\textit{child-code}| |[|\||else |\textit{main-code}]| \||fi|
\end{center}

%%%%%%%%%%%%%%%%%%%%%%%%%%%%%%%%%%%%%%%%
\DescribeMacro{\childdocname}
\DescribeMacro{\childdocjob}
The macro |\childdocname| contains the filename (without extension)
of the main or child file being processed.
Note that |\childdocjob| will always contain the name of the main file.

%%%%%%%%%%%%%%%%%%%%%%%%%%%%%%%%%%%%%%%%
\paragraph{Title Page.}

Conditional processing can be used to include a title or banner page
in the main document when proper precautions are taken.
Importantly, the code in the main file should ensure that the page counter
(as well as other status parameters which are stored in the |.aux| files)
takes the same value after the conditional processing.
Otherwise the page numbers may take divergent values
depending on which part is compiled.

For example, a title page could be declared by:
%
\begin{center}
\begin{tabular}{l}
|\ifchilddoc\||else|\\
|\addtocounter{page}{-1}|\\
\textit{code for title page}\\
|\newpage|\\
|\||fi|
\end{tabular}
\end{center}
%
A banner page for the child documents can be generated by:
%
\begin{center}
\begin{tabular}{l}
|\ifchilddoc|\\
|\addtocounter{page}{-1}|\\
\textit{code for banner page}\\
|\newpage|\\
|\||fi|
\end{tabular}
\end{center}
%
Here one could write a message such as:
\begin{center}
|This is the part \childdocname{} of \childdocjob{}.|
\end{center}

%%%%%%%%%%%%%%%%%%%%%%%%%%%%%%%%%%%%%%%%%%%%%%%%%%%%%%%%%%%%%%%%%%%%%%%%%%%%%%%%
\subsection{Flags}
\label{sec:flags}

The package makes it easy to generate different versions
of the main or child documents.
To this end compilation flags can be defined
and assigned different default values.
They will be particularly useful in conjunction
with the forwarding mechanism described in \secref{sec:forward}.

For example, it may be useful to have a flag |\version|
which can be set to |draft| or |final|.
The document source will contain some conditional code
depending on the value of |\version|.
Suppose further, the flag should default to |final| for the main file
and to |draft| for child files
which is a natural assignment for editing the document.
This is achieved by placing the following code
in the preamble of the main document
(below the |\childdocmain| directive):
%
\begin{center}
\begin{tabular}{l}
|\ifchilddoc|\\
|\providecommand{\version}{draft}|\\
|\||else|\\
|\providecommand{\version}{final}|\\
|\||fi|
\end{tabular}
\end{center}
%
The definition by |\providecommand| makes sure
that previous definitions are not overwritten.
Further statements |\providecommand{\version}{...}|
can thus be added before the above code to override it.

For the main file, one might add a line
(between |\childdocmain| and the above block)
%
\begin{center}
|%\ifchilddoc\||else\providecommand{\version}{draft}\||fi|
\end{center}
%
which can be uncommented to produce a draft version.
Likewise one can add a line to the very top of a child file
(above the |\childdocof{|\textit{main}|}| directive)
%
\begin{center}
|%\providecommand{\version}{final}|
\end{center}
%
which can be uncommented to produce the final version of this child document.

%%%%%%%%%%%%%%%%%%%%%%%%%%%%%%%%%%%%%%%%%%%%%%%%%%%%%%%%%%%%%%%%%%%%%%%%%%%%%%%%
\subsection{Forwarding}
\label{sec:forward}

Different versions of the main or child documents
using compilation flags as described in \secref{sec:flags}
can be (permanently) stored in different files
for convenient compilation, viewing and distribution.
To this end, the package defines a command
to pass on compilation to a different file:

%%%%%%%%%%%%%%%%%%%%%%%%%%%%%%%%%%%%%%%%
\DescribeMacro{\childdocforward}
The command |\childdocforward| redirects processing to
another source file:
%
\begin{center}
\begin{tabular}{l}
|\input{childdoc.def}|\\
|\childdocforward[|\textit{main}|]{|\textit{dest}|}|\\
\end{tabular}
\end{center}
%
The argument \textit{dest} is the destination file
(without extension).
It should be the main file or one of the child files.
Note that further \textsf{childdoc} directives
such as |\childdocof| and |\childdocforward|
in the indicated file will be processed in this form.
The optional argument \textit{main}
passes on directly to the main file \textit{main}
while pretending to compile the child \textit{dest}.
This form behaves as if \textit{dest}
issues |\childdocof{|\textit{main}|}| right away,
and no further \textsf{childdoc} directives will be processed.

%%%%%%%%%%%%%%%%%%%%%%%%%%%%%%%%%%%%%%%%
\DescribeMacro{\...prefix}
In the alternative form |\childdocforwardprefix|,
%
\begin{center}
\begin{tabular}{l}
|\input{childdoc.def}|\\
|\childdocforwardprefix[|\textit{main}|]{|\textit{prefix}|}{|\textit{dest}|}|
\end{tabular}
\end{center}
%
the destination file is determined by a pattern
depending on the current file:
To make this work, the current file must be called
`{\textit{prefix}\hspace{0.2em}\textit{suffix}}'
with \textit{prefix} matching precisely the argument.
Processing is then passed on to the file
`{\textit{dest}\hspace{0.2em}\textit{suffix}}'.
Surely, the same effect is achieved by
directly specifying the
argument `{\textit{dest}\hspace{0.2em}\textit{suffix}}'
in the first form.
However, that requires to set up a different file
for each child. With the alternative form of the command
all these files can have exactly the same content
which simplifies setting them up and maintaining them.

For example, the following file |draft.tex|
with a compilation flag |\version| as described in \secref{sec:flags}
compiles the main document as a draft:
%
\begin{center}
\begin{tabular}{l}
|\def\version{draft}|\\
|\input{childdoc.def}|\\
|\childdocforward{|\textit{main}|}|
\end{tabular}
\end{center}
%
Likewise, the following files |final|\textit{nn}|.tex|
compile the final version of the child document
|child|\textit{nn}|.tex|:
%
\begin{center}
\begin{tabular}{l}
|\def\version{final}|\\
|\input{childdoc.def}|\\
|\childdocforwardprefix{final}{child}|
\end{tabular}
\end{center}
%

Note that when several versions of a main file and/or of each child file
are to be generated, it may be convenient to set up a |Makefile| or
shell script to automatise the process.

%%%%%%%%%%%%%%%%%%%%%%%%%%%%%%%%%%%%%%%%%%%%%%%%%%%%%%%%%%%%%%%%%%%%%%%%%%%%%%%%
\subsection{Command Line Processing}
\label{sec:commandline}

The effect of redirection files can also be achieved by invoking
the \LaTeX{} compiler with a more elaborate command line.
Most conveniently this should be done as part
of a shell script or a |Makefile|.

When using \textsf{childdoc} in the main file, the following
command lines effectively perform a redirection
(note that depending on the shell being used,
backslashes may have to be doubled: `|\|' $\to$ `|\\|'):
%
\begin{center}
|... -jobname "|\textit{target}|" |\\|"|[\textit{flags}]%
|\input{childdoc.def}\childdocforward[|\textit{main}|]{|\textit{dest}|}"|
\end{center}
%
Here \textit{target} is the name of the output file,
\textit{main} is the name of the main file
and \textit{dest} is the name of the main or child file to be processed
(all filenames without extensions).
The optional argument \textit{main} can be omitted
if \textit{main} matches \textit{dest}.
Optionally, compilation \textit{flags} can be defined via |\def| commands.
This command line makes the \TeX{} engine believe
it is compiling the file \textit{target}
whose content is specified as the latter parameter.
The provided code then forwards the processing to
\textit{main} or \textit{dest} as described in \secref{sec:forward}.

%%%%%%%%%%%%%%%%%%%%%%%%%%%%%%%%%%%%%%%%%%%%%%%%%%%%%%%%%%%%%%%%%%%%%%%%%%%%%%%%
\subsection{Include by Input}
\label{sec:input}

Including child documents by |\include| has some restrictions by design.
Most notably, the content of a child document always occupies
its own set of pages; pages cannot be shared between child documents.
Usually, this behaviour makes perfect sense
because each child document contain an essential part of the document.
However, in some situations it may be desirable to compose
a document from a collection of parts
without having mandatory page breaks between then.
For this case, the package
provides a mechanism to include parts
by |\input| which can also be processed individually.
However, by construction this mechanism
requires manual handling of the content to be output.

%%%%%%%%%%%%%%%%%%%%%%%%%%%%%%%%%%%%%%%%
\DescribeMacro{\ifchilddocmanual}
The main file should be prepared as usual, see \secref{sec:include}.
However, the document body must make a distinction
between processing of an individual part and of the main document, e.g.:
%
\begin{center}
\begin{tabular}{l}
|\ifchilddocmanual|\\
|\input{\childdocname}|\\
|\||else|\\
\textit{document body with }|\input{|\textit{part}|}|\\
|\||fi|
\end{tabular}
\end{center}
%
The conditional |\ifchilddocmanual| is true whenever
a part to be included by |\input| is being compiled,
and the name of the part is stored in |\childdocname|.

%%%%%%%%%%%%%%%%%%%%%%%%%%%%%%%%%%%%%%%%
\DescribeMacro{\childdocby}
Each part to be included by |\input| should start with:
%
\begin{center}
\begin{tabular}{l}
|\input{childdoc.def}|\\
|\childdocby{|\textit{main}|}|\\
\end{tabular}
\end{center}
%
The directive |\childdocby| is similar to |\childdocof|
described in \secref{sec:include},
but the subsequent selection of content must be done manually.
To that end, both |\ifchilddoc| and |\ifchilddocmanual|
will be true upon processing of a part,
and the name of the part is stored in |\childdocname|.
Note that |\jobname| will be set to the filename of the current part
so that each part receives an individual |.aux| file
that does not interfere with the |.aux| file(s) of the main document.
This behaviour can be altered by the alternative form
|\childdocby[*]{|\textit{main}|}| (with a non-empty optional argument)
which uses the |.aux| file of the main document
by setting |\jobname| to \textit{main}.

%%%%%%%%%%%%%%%%%%%%%%%%%%%%%%%%%%%%%%%%%%%%%%%%%%%%%%%%%%%%%%%%%%%%%%%%%%%%%%%%
\subsection{Driver Development}
\label{sec:driver}

The \textsf{childdoc} mechanism can also be use for the development
of definition files such as \LaTeX{} styles or classes.
This case differs from the above setup with multiple parts
included by |\include| in that no |\includeonly| should be invoked.
This can be achieved by starting the include file
(before |\ProvidesPackage|) with:
%
\begin{center}
\begin{tabular}{l}
|\input{childdoc.def}|\\
|\childdocforward{|\textit{main}|}|\\
\end{tabular}
\end{center}
%
or alternatively with:
%
\begin{center}
\begin{tabular}{l}
|\input{childdoc.def}|\\
|\childdocby{|\textit{main}|}|\\
\end{tabular}
\end{center}
%
Both forms have slightly different effects as described above.
The main file is prepared as usual, see \secref{sec:include}.

%%%%%%%%%%%%%%%%%%%%%%%%%%%%%%%%%%%%%%%%%%%%%%%%%%%%%%%%%%%%%%%%%%%%%%%%%%%%%%%%
\subsection{Legacy Detection}
\label{sec:detection}

The directive |\childdocmain| in the main file can detect
whether the complete document or merely a child is to be compiled
even without using the directive |\childdocof|.
This method is deprecated because it is less robust
and there is no compelling reason to use it;
it is merely provided for backward compatibility
and it may be removed in future versions.

If the detection mechanism is to be used,
it is mandatory to correctly specify
the filename of the main file as the argument of |\childdocmain|:
%
\begin{center}
\begin{tabular}{l}
|\input{childdoc.def}|\\
|\childdocmain{|\textit{main}|}|\\
\end{tabular}
\end{center}
%
If |\jobname| does not match the argument \textit{main} of |\childdocmain|,
it is assumed that |\jobname| points to the child file to be compiled.
When using |\childdocmain| with the main file specified as argument,
it suffices to start a child file
with just |\input{|\textit{main}|}|
without loading of the package and using |\childdocof|.
If instead all processing is done
with the appropriate \textsf{childdoc} directives,
the argument of \textit{main} of |\childdocmain| can be empty.

An alternative version of the command line processing described
in \secref{sec:commandline} using the detection mechanism reads:
%
\begin{center}
|... -jobname "|\textit{target}|" "|[\textit{flags}]%
[|\def\jobname{|\textit{dest}|}|]|\input{|\textit{main}|}"|
\end{center}

%%%%%%%%%%%%%%%%%%%%%%%%%%%%%%%%%%%%%%%%%%%%%%%%%%%%%%%%%%%%%%%%%%%%%%%%%%%%%%%%
\subsection{Manual Code}
\label{sec:manual}

In case one cannot be certain whether the definitions file |childdoc.def|
is installed on the target \TeX{} distribution
and one prefers not to ship it,
it is conceivable to paste a few relevant commands into the sources.

To that end, drop all statements |\input{childdoc.def}|
and perform the replacements as outlined below.
Instead of |\childdocmain{|\textit{main}|}| add the following code
to the top of the main file:
%
\begin{center}
\begin{tabular}{l}
|\||ifdefined\childdocname\endinput\||fi\newif\ifchilddoc|\\
|\edef\childdocname{\scantokens\expandafter{\jobname\noexpand}}|\\
|\def\childdocmain{|\textit{main}|}\||ifx\childdocmain\childdocname\||else|\\
|\childdoctrue\includeonly{\childdocname}\let\jobname\childdocmain\||fi|\\
\end{tabular}
\end{center}
%
Instead of |\childdocof{|\textit{main}|}| just include the main file
at the top of each child file:
%
\begin{center}
|\input{|\textit{main}|}|
\end{center}
%
A simple redirection |\childdocforward{|\textit{dest}|}| is achieved by:
%
\begin{center}
|\def\jobname{|\textit{dest}|}\input{\jobname}|
\end{center}
%
The redirection with prefix
|\childdocforwardprefix[|\textit{prefix}|]{|\textit{dest}|}|
is accomplished by:
%
\begin{center}
\begin{tabular}{l}
|{\edef\jobname{\scantokens\expandafter{\jobname\noexpand}}|\\
|\def\redirectjob |\textit{prefix}|#1~~~{\gdef\jobname{|\textit{dest}|#1}}|\\
|\expandafter\redirectjob\jobname~~~}\input{\jobname}|
\end{tabular}
\end{center}

In an alternative approach,
child documents can be compiled by a specific command line
without additional code or specific definitions:
%
\begin{center}
|... -jobname "|\textit{target}|" "|[\textit{flags}]%
|\includeonly{|\textit{dest}|}\input{|\textit{main}|}"|
\end{center}
%

%%%%%%%%%%%%%%%%%%%%%%%%%%%%%%%%%%%%%%%%%%%%%%%%%%%%%%%%%%%%%%%%%%%%%%%%%%%%%%%%
%%%%%%%%%%%%%%%%%%%%%%%%%%%%%%%%%%%%%%%%%%%%%%%%%%%%%%%%%%%%%%%%%%%%%%%%%%%%%%%%
\section{Information}

%%%%%%%%%%%%%%%%%%%%%%%%%%%%%%%%%%%%%%%%%%%%%%%%%%%%%%%%%%%%%%%%%%%%%%%%%%%%%%%%
\subsection{Copyright}

Copyright \copyright{} 2017--2018 Niklas Beisert

This work may be distributed and/or modified under the
conditions of the \LaTeX{} Project Public License, either version 1.3
of this license or (at your option) any later version.
The latest version of this license is in
  \url{http://www.latex-project.org/lppl.txt}
and version 1.3 or later is part of all distributions of \LaTeX{}
version 2005/12/01 or later.

This work has the LPPL maintenance status `maintained'.

The Current Maintainer of this work is Niklas Beisert.

This work consists of the files |README.txt|, |childdoc.ins| and |childdoc.dtx|
as well as the derived files |childdoc.def|, |cdocsamp.tex|
with |cdocsch1.tex|, |cdocsch2.tex|, |cdocspt3.tex|, |cdocspt4.tex|,
|cdocsdrf.tex|, |cdocsfn1.tex|, |cdocsfn2.tex|
as well as |childdoc.pdf|.

%%%%%%%%%%%%%%%%%%%%%%%%%%%%%%%%%%%%%%%%%%%%%%%%%%%%%%%%%%%%%%%%%%%%%%%%%%%%%%%%
\subsection{Files and Installation}

The package consists of the files:
%
\begin{center}
\begin{tabular}{ll}
    |README.txt|   & readme file \\
    |childdoc.ins| & installation file \\
    |childdoc.dtx| & source file \\
    |childdoc.def| & definition file \\
    |cdocsamp.tex| & sample main file \\
    |cdocsch1.tex| & sample include file \\
    |cdocsch2.tex| & sample include file \\
    |cdocspt3.tex| & sample part file \\
    |cdocspt4.tex| & sample part file \\
    |cdocsdrf.tex| & sample redirection file \\
    |cdocsfn1.tex| & sample redirection file \\
    |cdocsfn2.tex| & sample redirection file \\
    |childdoc.pdf| & manual
\end{tabular}
\end{center}
%
The distribution consists of the files
|README.txt|, |childdoc.ins| and |childdoc.dtx|.
%
\begin{itemize}
\item
Run (pdf)\LaTeX{} on |childdoc.dtx|
to compile the manual |childdoc.pdf| (this file).
\item
Run \LaTeX{} on |childdoc.ins| to create the definitions file |childdoc.def|
and the sample |cdocsamp.tex| with include files
|cdocsch1.tex|, |cdocsch2.tex|, |cdocspt3.tex|, |cdocspt4.tex|,
|cdocsdrf.tex|, |cdocsfn1.tex|, |cdocsfn2.tex|.
Then copy the file |childdoc.def| to an appropriate directory of your \LaTeX{}
distribution, e.g.\ \textit{texmf-root}|/tex/latex/childdoc|.
\end{itemize}

%%%%%%%%%%%%%%%%%%%%%%%%%%%%%%%%%%%%%%%%%%%%%%%%%%%%%%%%%%%%%%%%%%%%%%%%%%%%%%%%
\subsection{Related CTAN Packages}

There are several other packages which offer a similar functionality:
%
\begin{itemize}
\item
The packages
\href{http://ctan.org/pkg/docmute}{\textsf{docmute}},
\href{http://ctan.org/pkg/includex}{\textsf{includex}} and
\href{http://ctan.org/pkg/standalone}{\textsf{standalone}}
provide commands to include only the document body of
a child file thus allowing both files to be compiled individually.
\item
The packages \href{http://ctan.org/pkg/subdocs}{\textsf{subdocs}}
and \href{http://ctan.org/pkg/subfiles}{\textsf{subfiles}}
provide structures in which the main and child documents can be
encapsulated and allowing them to be compiled individually.
The inclusion mechanism is different from the conventional |\include|.
\item
The package \href{http://ctan.org/pkg/combine}{\textsf{combine}}
is an elaborate solution to combine several documents into one.
\end{itemize}
%
See also the CTAN topic \href{http://ctan.org/topic/subdocs}{\textsf{subdocs}}
for further related packages.
The present package differs from the above solutions in that
a document structure constructed with the conventional |\include| mechanism
just needs two extra commands at the top of every file
such that all constituent files can be compiled individually.

%%%%%%%%%%%%%%%%%%%%%%%%%%%%%%%%%%%%%%%%%%%%%%%%%%%%%%%%%%%%%%%%%%%%%%%%%%%%%%%%
%\subsection{Feature Suggestions}
%
%The following is a list of features which may be useful for future
%versions of this package:
%%
%\begin{itemize}
%\item
%\ldots
%\end{itemize}

%%%%%%%%%%%%%%%%%%%%%%%%%%%%%%%%%%%%%%%%%%%%%%%%%%%%%%%%%%%%%%%%%%%%%%%%%%%%%%%%
\subsection{Revision History}

%%%%%%%%%%%%%%%%%%%%%%%%%%%%%%%%%%%%%%%%
\paragraph{v2.0:} 2018/12/30

\begin{itemize}
\item
immediate forward processing
\item
added |\childdocby| mechanism
\item
manual restructured
\end{itemize}

%%%%%%%%%%%%%%%%%%%%%%%%%%%%%%%%%%%%%%%%
\paragraph{v1.6:} 2018/01/17

\begin{itemize}
\item
application for development of include files
\item
corrections to manual
\end{itemize}

%%%%%%%%%%%%%%%%%%%%%%%%%%%%%%%%%%%%%%%%
\paragraph{v1.5:} 2017/05/21

\begin{itemize}
\item
more complete structuring introduced
\item
|\childdocof| introduced
\item
|\childdoc| renamed to |\childdocmain|
\item
|\childredirect| renamed to |\childdocforward| and |\childdocforwardprefix|
and functionality expanded
\end{itemize}

%%%%%%%%%%%%%%%%%%%%%%%%%%%%%%%%%%%%%%%%
\paragraph{v1.0:} 2017/04/27

\begin{itemize}
\item
manual and install package
\item
first version published on CTAN
\end{itemize}

%%%%%%%%%%%%%%%%%%%%%%%%%%%%%%%%%%%%%%%%
\paragraph{v0.6:} 2017/04/26

\begin{itemize}
\item
redirection mechanism added
\end{itemize}

%%%%%%%%%%%%%%%%%%%%%%%%%%%%%%%%%%%%%%%%
\paragraph{v0.5:} 2017/04/26

\begin{itemize}
\item
functionality in definition file
\end{itemize}


%%%%%%%%%%%%%%%%%%%%%%%%%%%%%%%%%%%%%%%%%%%%%%%%%%%%%%%%%%%%%%%%%%%%%%%%%%%%%%%%
%%%%%%%%%%%%%%%%%%%%%%%%%%%%%%%%%%%%%%%%%%%%%%%%%%%%%%%%%%%%%%%%%%%%%%%%%%%%%%%%
%%%%%%%%%%%%%%%%%%%%%%%%%%%%%%%%%%%%%%%%%%%%%%%%%%%%%%%%%%%%%%%%%%%%%%%%%%%%%%%%
\appendix

\settowidth\MacroIndent{\rmfamily\scriptsize 000\ }

 \DocInput{childdoc.dtx}

\end{document}
%</driver>
% \fi
%
% %%%%%%%%%%%%%%%%%%%%%%%%%%%%%%%%%%%%%%%%%%%%%%%%%%%%%%%%%%%%%%%%%%%%%%%%%%%%%%
% %%%%%%%%%%%%%%%%%%%%%%%%%%%%%%%%%%%%%%%%%%%%%%%%%%%%%%%%%%%%%%%%%%%%%%%%%%%%%%
% \section{Sample}
%\iffalse
%<*samplemain>
%\fi
%
% The following presents a sample document
% with two chapters, two parts, a title page,
% a compile flag as well as three forwarding files to set the flag.
% It consists of eight |.tex| files:
% \begin{center}
% \begin{tabular}{ll}
% |cdocsamp.tex|&main file\\
% |cdocsch1.tex|&include file for chapter 1\\
% |cdocsch2.tex|&include file for chapter 2\\
% |cdocspt3.tex|&include file for part 3\\
% |cdocspt4.tex|&include file for part 4\\
% |cdocsdrf.tex|&forwarding file for main file in draft mode\\
% |cdocsfi1.tex|&forwarding file for final version of chapter 1\\
% |cdocsfi2.tex|&forwarding file for final version of chapter 2\\
% \end{tabular}
% \end{center}
% Each of the eight files can be compiled directly by the \LaTeX{} compiler.
%
% %%%%%%%%%%%%%%%%%%%%%%%%%%%%%%%%%%%%%%
% \paragraph{Main File.}
%
% The main file is called |cdocsamp.tex|.
%
% Load the \textsf{childdoc} definitions and
% declare the filename for the main document:
%    \begin{macrocode}
\input{childdoc.def}
\childdocmain{}
%    \end{macrocode}

% Optional override for |\version| flag:
%    \begin{macrocode}
%%\ifchilddoc\else\providecommand{\version}{draft}\fi
%    \end{macrocode}

% Define the default values for the |\version| flag
% (|final| for the main file and |draft| for childs):
%    \begin{macrocode}
\ifchilddoc
\providecommand{\version}{draft}
\else
\providecommand{\version}{final}
\fi
%    \end{macrocode}

% Load the standard document class:
%    \begin{macrocode}
\documentclass[12pt]{article}
%    \end{macrocode}

% Start the document body:
%    \begin{macrocode}
\begin{document}
%    \end{macrocode}

% Declare a title page.
% Print title, part of document being processed and version flag:
%    \begin{macrocode}
\addtocounter{page}{-1}
\begin{center}
{\LARGE\bfseries{}childdoc example\par}
\vspace{1cm}
\ifchilddoc
\ifchilddocmanual part\else chapter\fi:
`\childdocname' of `\childdocjob'\par
\else
main document: `\childdocjob'\par
\fi
version: \version\par
\end{center}
\newpage
%    \end{macrocode}

% Manually include selected file,
% otherwise process as usual:
%    \begin{macrocode}
\ifchilddocmanual
\section*{part `\childdocname'}
\input{\childdocname}
\else
%    \end{macrocode}

% Include the two chapters:
%    \begin{macrocode}
\include{cdocsch1}
\include{cdocsch2}
%    \end{macrocode}

% Include the two parts unless only chapters should be displayed:
%    \begin{macrocode}
\ifchilddoc\else
\section{part three}
\input{cdocspt3}
\section{part four}
\input{cdocspt4}
\fi
%    \end{macrocode}

% Process as usual until here:
%    \begin{macrocode}
\fi
%    \end{macrocode}

% End of document body:
%    \begin{macrocode}
\end{document}
%    \end{macrocode}
%\iffalse
%</samplemain>
%\fi
%
% %%%%%%%%%%%%%%%%%%%%%%%%%%%%%%%%%%%%%%
% \paragraph{Chapter Include Files.}
%
% The include files are called |cdocsch1.tex| and |cdocsch2.tex|.
%
%\iffalse
%<*samplechap1|samplechap2>
%\fi

% Optional override for |\version| flag:
%    \begin{macrocode}
%%\providecommand{\version}{final}
%    \end{macrocode}

% Include the main document:
%    \begin{macrocode}
\input{childdoc.def}
\childdocof{cdocsamp}
%    \end{macrocode}

%\iffalse
%</samplechap1|samplechap2>
%\fi
%
%\iffalse
%<*samplechap1>
%\fi
% Some text for chapter 1:
%    \begin{macrocode}
\section{one}
some text in chapter one
%    \end{macrocode}

%\iffalse
%</samplechap1>
%\fi
% Some text for chapter 2:
%\iffalse
%<*samplechap2>
%\fi
%    \begin{macrocode}
\section{two}
more text in chapter two
%    \end{macrocode}

%\iffalse
%</samplechap2>
%\fi
%
% %%%%%%%%%%%%%%%%%%%%%%%%%%%%%%%%%%%%%%
% \paragraph{Part Include Files.}
%
% The include files are called |cdocspt3.tex| and |cdocspt4.tex|.
%
%\iffalse
%<*samplepart3|samplepart4>
%\fi

% Optional override for |\version| flag:
%    \begin{macrocode}
%%\providecommand{\version}{final}
%    \end{macrocode}

% Include the main document:
%    \begin{macrocode}
\input{childdoc.def}
\childdocby{cdocsamp}
%    \end{macrocode}

%\iffalse
%</samplepart3|samplepart4>
%\fi
%
%\iffalse
%<*samplepart3>
%\fi
% Some text for part 3:
%    \begin{macrocode}
some text in part three
%    \end{macrocode}

%\iffalse
%</samplepart3>
%\fi
% Some text for part 4:
%\iffalse
%<*samplepart4>
%\fi
%    \begin{macrocode}
more text in part four
%    \end{macrocode}

%\iffalse
%</samplepart4>
%\fi
%
% %%%%%%%%%%%%%%%%%%%%%%%%%%%%%%%%%%%%%%
% \paragraph{Forwarding for a Complete Draft.}
%
% The following forwarding file |cdocsdrf.tex|
% compiles the main document in draft mode:
%\iffalse
%<*sampledraft>
%\fi
%    \begin{macrocode}
\def\version{draft}
\input{childdoc.def}
\childdocforward{cdocsamp}
%    \end{macrocode}

%\iffalse
%</sampledraft>
%\fi
%
% %%%%%%%%%%%%%%%%%%%%%%%%%%%%%%%%%%%%%%
% \paragraph{Forwarding for Final Version of the Chapters.}
%
% The following forwarding files |cdocsfn1.tex| and |cdocsfn2.tex|
% (with identical content)
% compile the final versions of the child documents
% |cdocsch1.tex| and |cdocsch2.tex|, respectively:
%\iffalse
%<*samplefinal>
%\fi
%    \begin{macrocode}
\def\version{final}
\input{childdoc.def}
\childdocforwardprefix[cdocsamp]{cdocsfn}{cdocsch}
%    \end{macrocode}

%\iffalse
%</samplefinal>
%\fi
%
% %%%%%%%%%%%%%%%%%%%%%%%%%%%%%%%%%%%%%%
% \paragraph{Command Line Processing.}
%
% The following three command lines generate the output files
% |cdocscld|, |cdocscl1| and |cdocscl2|
% which should be identical to
% |cdocsdrf|, |cdocsch1| and |cdocsfn2|, respectively:
% \begin{center}
% \begin{tabular}{l}
% |latex -jobname cdocscld \|\\
% |  "\def\version{draft}\input{childdoc.def}\childdocforward{cdocsamp}"|\\
% |latex -jobname cdocscl1 \|\\
% |  "\input{childdoc.def}\childdocforward[cdocsamp]{cdocsch1}"|\\
% |latex -jobname cdocscl2 \|\\
% |  "\def\version{final}\input{childdoc.def}\childdocforward{cdocsch2}"|
% \end{tabular}
% \end{center}
% Note that the trailing backslash on each first line
% merely continues the input to the second line
% (for convenient cut ant paste).
% Furthermore, the command |latex| can be replaced by any
% of its alternative versions such as |pdflatex|.
%
% %%%%%%%%%%%%%%%%%%%%%%%%%%%%%%%%%%%%%%%%%%%%%%%%%%%%%%%%%%%%%%%%%%%%%%%%%%%%%%
% %%%%%%%%%%%%%%%%%%%%%%%%%%%%%%%%%%%%%%%%%%%%%%%%%%%%%%%%%%%%%%%%%%%%%%%%%%%%%%
% \section{Implementation}
%\iffalse
%<*package>
%\fi
%
% This section describes the definitions file |childdoc.def|.

% The definitions cannot be loaded using |\usepackage| or |\RequirePackage|
% which has a mechanism to prevent loading a style file more than once.
% When loading the definitions by means of |\input|
% multiple instances have to be prevented manually:
%\iffalse
%This code needs to be before the `\ProvidesFile' directive
%which is defined at the beginning of this file.
%Therefore it is also placed there and commented out here.
%</package>
%<*discard>
%\fi
%    \begin{macrocode}
\ifdefined\childdocmain\endinput\fi
%    \end{macrocode}
%\iffalse
%</discard>
%<*package>
%\fi
%
% \macro{\ifchilddoc}
% \macro{\ifchilddocmanual}
% The conditional |\ifchilddoc| tells whether a
% child (true) or main (false) document is being compiled.
% The conditional |\ifchilddocmanual| tells whether
% the |\includeonly| mechanism is used (false) or
% the selection of child files must be performed manually (true).
% The definitions initialise to false:
%    \begin{macrocode}
\newif\ifchilddoc
\newif\ifchilddocmanual
%    \end{macrocode}

% \macro{\childdocname}
% \macro{\childdocjob}
% The macro |\childdocname| stores the name of the main document
% to be compiled. The macro |\childdocjob| stores the name of
% the document on which the \LaTeX{} compiler was originally invoked.
% The content of |\jobname| cannot be compared
% to filenames specified in the source due to different catcodes.
% The following code rescans |\jobname|, stores the result
% in |\childdocname| and saves a copy in |\childdocjob|:
%    \begin{macrocode}
\edef\childdocname{\scantokens\expandafter{\jobname\noexpand}}
\let\childdocjob\childdocname
%    \end{macrocode}

% \macro{\childdocdisable}
% The macro |\childdocdisable| prevents the main file
% from being processed more than once.
% At this stage, the main document command |\childdocmain|
% is assumed to be called once again where it should do nothing.
% Any subsequent call to it should prevent
% a secondary processing of the main document
% It overwrites the forwarding commands
% |\childdocof| and |\childdocforward|
% with empty macros to prevent further inclusions of the main document:
%    \begin{macrocode}
\newcommand{\childdocdisable}
{
  \renewcommand{\childdocmain}[1]{\renewcommand{\childdocmain}[1]{\endinput}}
  \renewcommand{\childdocof}[1]{}
  \renewcommand{\childdocby}[2][]{}
  \renewcommand{\childdocforward}[2][]{}
  \renewcommand{\childdocdisable}{}
}
%    \end{macrocode}

% \macro{\childdocmain}
% The macro |\childdocmain| is to be called at the top of the main file
% with nothing or the main filename (without extension) as argument.
% First, it breaks loops.
% If the argument is not empty and does not match |\childdocname|
% (which is set by the first inclusion of |childdoc.def|),
% |\ifchilddoc| is set to true, |\includeonly| is applied to the child file
% and |\jobname| is set to the main file
% (for proper handling of |.aux| files):
%    \begin{macrocode}
\newcommand{\childdocmain}[1]
{
  \childdocdisable\childdocmain{}
  \if?#1?\else
    \begingroup
      \def\childdoctmp{#1}
      \ifx\childdoctmp\childdocname
        \def\childdoctmp{}
      \else
        \def\childdoctmp
        {
          \childdoctrue
          \includeonly{\childdocname}
          \def\childdocjob{#1}
          \def\jobname{#1}
        }
      \fi
      \expandafter
    \endgroup
    \childdoctmp
  \fi
}
%    \end{macrocode}

% \macro{\childdocof}
% The command |\childdocof| redirects
% compilation to the main file |#1|.
%    \begin{macrocode}
\newcommand{\childdocof}[1]
{
  \childdocdisable
  \childdoctrue
  \includeonly{\childdocname}
  \def\jobname{#1}
  \def\childdocjob{#1}
  \input{#1}
}
%    \end{macrocode}

% \macro{\childdocby}
% The command |\childdocby| ....
%    \begin{macrocode}
\newcommand{\childdocby}[2][]
{
  \childdocdisable
  \childdoctrue
  \childdocmanualtrue
  \if?#1?\else
    \def\jobname{#2}
  \fi
  \def\childdocjob{#2}
  \input{#2}
  \endinput
}
%    \end{macrocode}

% \macro{\childdocforward}
% The command |\childdocforward| redirects
% compilation to the main file or
% (if the optional argument is given) a child file.
% Parameters are set as if the main file
% or a child file starting with |\childdocof| was compiled.
% Then compilation is handed over to the main file:
%    \begin{macrocode}
\newcommand{\childdocforward}[2][]
{
  \begingroup
    \if?#1?
      \def\childdoctmp
      {
        \def\childdocname{#2}
        \def\childdocjob{#2}
        \def\jobname{#2}
        \input{#2}
        \endinput
      }
    \else
      \def\childdoctmp
      {
        \childdocdisable
        \def\childdocname{#2}
        \childdoctrue
        \includeonly{#2}
        \def\childdocjob{#1}
        \def\jobname{#1}
        \input{#1}
        \endinput
      }
    \fi
    \expandafter
  \endgroup
  \childdoctmp
}
%    \end{macrocode}

% \macro{\childdocforwardprefix}
% The command |\childdocforwardprefix| redirects
% compilation to the main or a child file by means of a pattern.
% The prefix |#1| in the current filename is replaced by |#2|
% and the suffix of the current filename is kept
% (it is assumed that the filename does not contain the substring `|~~~|'
% which is used as a delimiter).
% Compilation is handed over to the new file by |\childdocforward|:
%    \begin{macrocode}
\newcommand{\childdocforwardprefix}[3][]
{
  \begingroup
    \def\childdocextract #2##1~~~{\def\childdoctmp{\childdocforward[#1]{#3##1}}}
    \expandafter\childdocextract\childdocname~~~
    \expandafter
  \endgroup
  \childdoctmp
}
%    \end{macrocode}

% \macro{\childdoc}
% The deprecated macro |\childdoc| is a legacy version of |\childdocmain|:
%    \begin{macrocode}
\newcommand{\childdoc}{\childdocmain}
%    \end{macrocode}

% \macro{\childdocredirect}
% The deprecated macro |\childdocredirect| is a legacy version
% of |\childdocforward| and |\childdocforwardprefix|:
%    \begin{macrocode}
\newcommand{\childdocredirect}[2][]
{
  \begingroup
    \if?#1?
      \def\childdoctmp{\childdocforward{#2}}
    \else
      \def\childdoctmp{\childdocforwardprefix{#1}{#2}}
    \fi
    \expandafter
  \endgroup
  \childdoctmp
}
%    \end{macrocode}

%\iffalse
%</package>
%\fi
%
\endinput
|\\
|\childdocby{|\textit{main}|}|\\
\end{tabular}
\end{center}
%
Both forms have slightly different effects as described above.
The main file is prepared as usual, see \secref{sec:include}.

%%%%%%%%%%%%%%%%%%%%%%%%%%%%%%%%%%%%%%%%%%%%%%%%%%%%%%%%%%%%%%%%%%%%%%%%%%%%%%%%
\subsection{Legacy Detection}
\label{sec:detection}

The directive |\childdocmain| in the main file can detect
whether the complete document or merely a child is to be compiled
even without using the directive |\childdocof|.
This method is deprecated because it is less robust
and there is no compelling reason to use it;
it is merely provided for backward compatibility
and it may be removed in future versions.

If the detection mechanism is to be used,
it is mandatory to correctly specify
the filename of the main file as the argument of |\childdocmain|:
%
\begin{center}
\begin{tabular}{l}
|% \iffalse
%
% childdoc.dtx Copyright (C) 2017-2018 Niklas Beisert
%
% This work may be distributed and/or modified under the
% conditions of the LaTeX Project Public License, either version 1.3
% of this license or (at your option) any later version.
% The latest version of this license is in
%   http://www.latex-project.org/lppl.txt
% and version 1.3 or later is part of all distributions of LaTeX
% version 2005/12/01 or later.
%
% This work has the LPPL maintenance status `maintained'.
%
% The Current Maintainer of this work is Niklas Beisert.
%
% This work consists of the files childdoc.dtx and childdoc.ins
% and the derived files childdoc.def and cdocsamp.tex with
% cdocsch1.tex, cdocsch2.tex, cdocsdrf.tex, cdocsfn1.tex, cdocsfn2.tex.
%
%<package>\ifdefined\childdocmain\endinput\fi
%<package>\ProvidesFile{childdoc.def}[2018/12/30 v2.0 child document driver]
%<samplemain>\ProvidesFile{cdocsamp.tex}[2018/12/30 v2.0 sample for childdoc]
%<*driver>
%\ProvidesFile{childdoc.drv}[2018/12/30 v2.0 childdoc reference manual file]
\PassOptionsToClass{10pt,a4paper}{article}
\documentclass{ltxdoc}

\usepackage[margin=35mm]{geometry}
\usepackage{hyperref}
\usepackage{hyperxmp}
\usepackage[usenames]{color}

\hypersetup{colorlinks=true}
\hypersetup{pdfstartview=FitH}
\hypersetup{pdfpagemode=UseNone}
\hypersetup{pdfsource={}}
\hypersetup{pdflang={en-UK}}
\hypersetup{pdfcopyright={Copyright 2017-2018 Niklas Beisert.
  This work may be distributed and/or modified under the
  conditions of the LaTeX Project Public License, either version 1.3
  of this license or (at your option) any later version.}}
\hypersetup{pdflicenseurl={http://www.latex-project.org/lppl.txt}}
\hypersetup{pdfcontactaddress={ETH Zurich, ITP, HIT K,
  Wolfgang-Pauli-Strasse 27}}
\hypersetup{pdfcontactpostcode={8093}}
\hypersetup{pdfcontactcity={Zurich}}
\hypersetup{pdfcontactcountry={Switzerland}}
\hypersetup{pdfcontactemail={nbeisert@itp.phys.ethz.ch}}
\hypersetup{pdfcontacturl={http://people.phys.ethz.ch/\xmptilde nbeisert/}}

\newcommand{\secref}[1]{\hyperref[#1]{section \ref*{#1}}}

\parskip1ex
\parindent0pt
\let\olditemize\itemize
\def\itemize{\olditemize\parskip0pt}

\begin{document}

\title{The \textsf{childdoc} Package}
\hypersetup{pdftitle={The childdoc Package}}
\author{Niklas Beisert\\[2ex]
  Institut f\"ur Theoretische Physik\\
  Eidgen\"ossische Technische Hochschule Z\"urich\\
  Wolfgang-Pauli-Strasse 27, 8093 Z\"urich, Switzerland\\[1ex]
  \href{mailto:nbeisert@itp.phys.ethz.ch}
  {\texttt{nbeisert@itp.phys.ethz.ch}}}
\hypersetup{pdfauthor={Niklas Beisert}}
\hypersetup{pdfsubject={Manual for the LaTeX2e Package childdoc}}
\date{30 December 2018, \textsf{v2.0}}
\maketitle

\begin{abstract}\noindent
\textsf{childdoc} is a \LaTeXe{} package
that enables the direct compilation
of document sections included by |\include|
to individual files.
\end{abstract}

\begingroup
\parskip0ex
\tableofcontents
\endgroup

%%%%%%%%%%%%%%%%%%%%%%%%%%%%%%%%%%%%%%%%%%%%%%%%%%%%%%%%%%%%%%%%%%%%%%%%%%%%%%%%
%%%%%%%%%%%%%%%%%%%%%%%%%%%%%%%%%%%%%%%%%%%%%%%%%%%%%%%%%%%%%%%%%%%%%%%%%%%%%%%%
\section{Introduction}

\LaTeX{} provides a mechanism to structure a large document (such as a book)
into a main file and several child files (containing the chapters)
using the |\include| command.
This mechanism is beneficial for documents
which span hundreds of pages in order to
make the source file(s) more manageable.
Moreover, compilation can be restricted to
selected child files by means of the |\includeonly| command.
The latter feature can be used to reduce the compilation time while editing
(this was significantly more useful in the earlier days of \LaTeX{})
or to generate a smaller document which is easier to navigate.
Another application of |\includeonly| is to generate
documents consisting of selected parts of the complete document.

However, there are a few drawbacks of the plain |\include| mechanism:
\begin{itemize}
\item
The child files cannot be compiled on their own,
they can only be compiled via the main file.
A naive editing environment
(such as a text editor with an option
to have the current file processed by \LaTeX)
may require one to switch to the main file before compiling;
attempting to compile the child file produces errors.
\item
The main file must be modified (each time)
to adjust the |\includeonly| command
to the present needs. This easily leaves the main file in a messy state.
\item
The generated document will always carry the filename
of the main document. This is inconvenient if
several child files are to be compiled and
to be kept for distribution.
\end{itemize}

The present package provides a simple interface
to make child files individually compilable by \LaTeX{}.
Compiling a child file then has the same effect as compiling
the main file with an |\includeonly| command
to select the appropriate child.
Moreover the generated document will carry the name of the child
rather than the main file.
This resolves all three above issues.

This feature is meant to make the editing of books,
thesis documents and lecture notes somewhat more convenient.
However, the package can also be used efficiently for
composing a series of documents (such as exercise sheets)
which are typically distributed individually.
It then assists the author in generating the individual documents
(potentially in different versions)
as well as a document containing the collected series.
Another application is in developing style files
or other kinds of included material
where compilation of the style file could redirect
to a sample or test file.

%%%%%%%%%%%%%%%%%%%%%%%%%%%%%%%%%%%%%%%%%%%%%%%%%%%%%%%%%%%%%%%%%%%%%%%%%%%%%%%%
%%%%%%%%%%%%%%%%%%%%%%%%%%%%%%%%%%%%%%%%%%%%%%%%%%%%%%%%%%%%%%%%%%%%%%%%%%%%%%%%
\section{Usage}

First of all, the package \textsf{childdoc} is \emph{not} a standard
\LaTeXe{} |.sty| style file! Therefore it needs to be invoked in
a non-standard way.

%%%%%%%%%%%%%%%%%%%%%%%%%%%%%%%%%%%%%%%%%%%%%%%%%%%%%%%%%%%%%%%%%%%%%%%%%%%%%%%%
\subsection{Included Files}
\label{sec:include}

%%%%%%%%%%%%%%%%%%%%%%%%%%%%%%%%%%%%%%%%
\DescribeMacro{\childdocmain}
To use the package, add the commands
\begin{center}
\begin{tabular}{l}
|\input{childdoc.def}|\\
|\childdocmain{}|\\
\end{tabular}
\end{center}
at the very top of the main \LaTeX{} file,
in particular \emph{before} the |\documentclass| statement!
The argument of |\childdocmain| should be left empty
(but it must be present).

%%%%%%%%%%%%%%%%%%%%%%%%%%%%%%%%%%%%%%%%
\DescribeMacro{\childdocof}
Furthermore, add the commands
\begin{center}
\begin{tabular}{l}
|\input{childdoc.def}|\\
|\childdocof{|\textit{main}|}|\\
\end{tabular}
\end{center}
at the top of every child file \textit{child}
which is included by |\include{|\textit{child}|}|
from within the main file
(or at least for those files to be compiled individually).
The argument \textit{main} must be the filename of the main file.

There are a couple of
considerations in setting up the main and child documents:

%%%%%%%%%%%%%%%%%%%%%%%%%%%%%%%%%%%%%%%%
\paragraph{Restrictions.}

Please note the following restrictions:
\begin{itemize}
\item
|\childdocmain| must be called with one argument \textit{main}
to ensure compatibility with earlier version of the package.
It must either be empty (|\childdocmain{}|)
or precisely match the filename of the main file in which it is specified.
See \secref{sec:detection} for further information.
\item
The filename \textit{main} must be specified without the |.tex| extension.
\item
The filename \textit{main} is case sensitive
(even in case-insensitive file systems)
due to internal string comparison.
\item
The argument \textit{main} should be fully expanded, it cannot be a macro.
\item
Subdirectories and special characters should be avoided in filenames.
\item
The command |\childdocmain{|\textit{main}|}| must be followed by a whitespace.
It should not be followed immediately by another command
or by a comment mark `|%|'.
This is because the \TeX{} parser reads the token immediately following
the argument of |\childdocmain| and puts it
at the beginning of every child section;
however, a white\-space is ignored.
\end{itemize}

%%%%%%%%%%%%%%%%%%%%%%%%%%%%%%%%%%%%%%%%
\paragraph{Content of Main File.}

It is advisable to place all content in the child files included by |\include|.
Any output contained in the main file will appear in all child documents
unless suppressed manually;
it cannot be suppressed automatically by the |\includeonly| directive
and thus should normally be avoided.
A method to include some content in the main file
by means of conditional processing is described in \secref{sec:conditional}.

%%%%%%%%%%%%%%%%%%%%%%%%%%%%%%%%%%%%%%%%
\paragraph{Page Numbering.}

When only a part of the document is compiled,
the appropriate numbering of pages
(as well as other status parameters)
is determined from the |.aux| files.
The latter contain information from previous passes.
However this information needs to propagate through
all intermediate child documents.
Therefore the page numbering in child documents may well
be inconsistent until the complete document is compiled at least once.

A useful (if unconventional) way to always ensure a consistent
page numbering is to restart the numbering in each child document
and denote the pages by `\textit{child}|.|\textit{page}'
where \textit{child} represents the chapter/section number of the child file.
This can be achieved by the command
|\numberwithin{page}{|\textit{child}|}|
of the \textsf{amsmath} package
where \textit{child} can be |chapter| or |section|
depending on the chosen structuring.
Alternatively, one can modify the macro |\thepage| appropriately
and reset the counter |page| at the start of each child file.

%%%%%%%%%%%%%%%%%%%%%%%%%%%%%%%%%%%%%%%%%%%%%%%%%%%%%%%%%%%%%%%%%%%%%%%%%%%%%%%%
\subsection{Conditional Processing}
\label{sec:conditional}

The package provides a mechanism to compile different versions
of a document. To customise the versions further some conditional processing
can come in handy to distinguish which version is being compiled.
The package provides two macros to describe the compilation context:

%%%%%%%%%%%%%%%%%%%%%%%%%%%%%%%%%%%%%%%%
\DescribeMacro{\ifchilddoc}
The conditional |\ifchilddoc| distinguishes between the compilation of
child documents and the main document:
%
\begin{center}
|\ifchilddoc |\textit{child-code}| |[|\||else |\textit{main-code}]| \||fi|
\end{center}

%%%%%%%%%%%%%%%%%%%%%%%%%%%%%%%%%%%%%%%%
\DescribeMacro{\childdocname}
\DescribeMacro{\childdocjob}
The macro |\childdocname| contains the filename (without extension)
of the main or child file being processed.
Note that |\childdocjob| will always contain the name of the main file.

%%%%%%%%%%%%%%%%%%%%%%%%%%%%%%%%%%%%%%%%
\paragraph{Title Page.}

Conditional processing can be used to include a title or banner page
in the main document when proper precautions are taken.
Importantly, the code in the main file should ensure that the page counter
(as well as other status parameters which are stored in the |.aux| files)
takes the same value after the conditional processing.
Otherwise the page numbers may take divergent values
depending on which part is compiled.

For example, a title page could be declared by:
%
\begin{center}
\begin{tabular}{l}
|\ifchilddoc\||else|\\
|\addtocounter{page}{-1}|\\
\textit{code for title page}\\
|\newpage|\\
|\||fi|
\end{tabular}
\end{center}
%
A banner page for the child documents can be generated by:
%
\begin{center}
\begin{tabular}{l}
|\ifchilddoc|\\
|\addtocounter{page}{-1}|\\
\textit{code for banner page}\\
|\newpage|\\
|\||fi|
\end{tabular}
\end{center}
%
Here one could write a message such as:
\begin{center}
|This is the part \childdocname{} of \childdocjob{}.|
\end{center}

%%%%%%%%%%%%%%%%%%%%%%%%%%%%%%%%%%%%%%%%%%%%%%%%%%%%%%%%%%%%%%%%%%%%%%%%%%%%%%%%
\subsection{Flags}
\label{sec:flags}

The package makes it easy to generate different versions
of the main or child documents.
To this end compilation flags can be defined
and assigned different default values.
They will be particularly useful in conjunction
with the forwarding mechanism described in \secref{sec:forward}.

For example, it may be useful to have a flag |\version|
which can be set to |draft| or |final|.
The document source will contain some conditional code
depending on the value of |\version|.
Suppose further, the flag should default to |final| for the main file
and to |draft| for child files
which is a natural assignment for editing the document.
This is achieved by placing the following code
in the preamble of the main document
(below the |\childdocmain| directive):
%
\begin{center}
\begin{tabular}{l}
|\ifchilddoc|\\
|\providecommand{\version}{draft}|\\
|\||else|\\
|\providecommand{\version}{final}|\\
|\||fi|
\end{tabular}
\end{center}
%
The definition by |\providecommand| makes sure
that previous definitions are not overwritten.
Further statements |\providecommand{\version}{...}|
can thus be added before the above code to override it.

For the main file, one might add a line
(between |\childdocmain| and the above block)
%
\begin{center}
|%\ifchilddoc\||else\providecommand{\version}{draft}\||fi|
\end{center}
%
which can be uncommented to produce a draft version.
Likewise one can add a line to the very top of a child file
(above the |\childdocof{|\textit{main}|}| directive)
%
\begin{center}
|%\providecommand{\version}{final}|
\end{center}
%
which can be uncommented to produce the final version of this child document.

%%%%%%%%%%%%%%%%%%%%%%%%%%%%%%%%%%%%%%%%%%%%%%%%%%%%%%%%%%%%%%%%%%%%%%%%%%%%%%%%
\subsection{Forwarding}
\label{sec:forward}

Different versions of the main or child documents
using compilation flags as described in \secref{sec:flags}
can be (permanently) stored in different files
for convenient compilation, viewing and distribution.
To this end, the package defines a command
to pass on compilation to a different file:

%%%%%%%%%%%%%%%%%%%%%%%%%%%%%%%%%%%%%%%%
\DescribeMacro{\childdocforward}
The command |\childdocforward| redirects processing to
another source file:
%
\begin{center}
\begin{tabular}{l}
|\input{childdoc.def}|\\
|\childdocforward[|\textit{main}|]{|\textit{dest}|}|\\
\end{tabular}
\end{center}
%
The argument \textit{dest} is the destination file
(without extension).
It should be the main file or one of the child files.
Note that further \textsf{childdoc} directives
such as |\childdocof| and |\childdocforward|
in the indicated file will be processed in this form.
The optional argument \textit{main}
passes on directly to the main file \textit{main}
while pretending to compile the child \textit{dest}.
This form behaves as if \textit{dest}
issues |\childdocof{|\textit{main}|}| right away,
and no further \textsf{childdoc} directives will be processed.

%%%%%%%%%%%%%%%%%%%%%%%%%%%%%%%%%%%%%%%%
\DescribeMacro{\...prefix}
In the alternative form |\childdocforwardprefix|,
%
\begin{center}
\begin{tabular}{l}
|\input{childdoc.def}|\\
|\childdocforwardprefix[|\textit{main}|]{|\textit{prefix}|}{|\textit{dest}|}|
\end{tabular}
\end{center}
%
the destination file is determined by a pattern
depending on the current file:
To make this work, the current file must be called
`{\textit{prefix}\hspace{0.2em}\textit{suffix}}'
with \textit{prefix} matching precisely the argument.
Processing is then passed on to the file
`{\textit{dest}\hspace{0.2em}\textit{suffix}}'.
Surely, the same effect is achieved by
directly specifying the
argument `{\textit{dest}\hspace{0.2em}\textit{suffix}}'
in the first form.
However, that requires to set up a different file
for each child. With the alternative form of the command
all these files can have exactly the same content
which simplifies setting them up and maintaining them.

For example, the following file |draft.tex|
with a compilation flag |\version| as described in \secref{sec:flags}
compiles the main document as a draft:
%
\begin{center}
\begin{tabular}{l}
|\def\version{draft}|\\
|\input{childdoc.def}|\\
|\childdocforward{|\textit{main}|}|
\end{tabular}
\end{center}
%
Likewise, the following files |final|\textit{nn}|.tex|
compile the final version of the child document
|child|\textit{nn}|.tex|:
%
\begin{center}
\begin{tabular}{l}
|\def\version{final}|\\
|\input{childdoc.def}|\\
|\childdocforwardprefix{final}{child}|
\end{tabular}
\end{center}
%

Note that when several versions of a main file and/or of each child file
are to be generated, it may be convenient to set up a |Makefile| or
shell script to automatise the process.

%%%%%%%%%%%%%%%%%%%%%%%%%%%%%%%%%%%%%%%%%%%%%%%%%%%%%%%%%%%%%%%%%%%%%%%%%%%%%%%%
\subsection{Command Line Processing}
\label{sec:commandline}

The effect of redirection files can also be achieved by invoking
the \LaTeX{} compiler with a more elaborate command line.
Most conveniently this should be done as part
of a shell script or a |Makefile|.

When using \textsf{childdoc} in the main file, the following
command lines effectively perform a redirection
(note that depending on the shell being used,
backslashes may have to be doubled: `|\|' $\to$ `|\\|'):
%
\begin{center}
|... -jobname "|\textit{target}|" |\\|"|[\textit{flags}]%
|\input{childdoc.def}\childdocforward[|\textit{main}|]{|\textit{dest}|}"|
\end{center}
%
Here \textit{target} is the name of the output file,
\textit{main} is the name of the main file
and \textit{dest} is the name of the main or child file to be processed
(all filenames without extensions).
The optional argument \textit{main} can be omitted
if \textit{main} matches \textit{dest}.
Optionally, compilation \textit{flags} can be defined via |\def| commands.
This command line makes the \TeX{} engine believe
it is compiling the file \textit{target}
whose content is specified as the latter parameter.
The provided code then forwards the processing to
\textit{main} or \textit{dest} as described in \secref{sec:forward}.

%%%%%%%%%%%%%%%%%%%%%%%%%%%%%%%%%%%%%%%%%%%%%%%%%%%%%%%%%%%%%%%%%%%%%%%%%%%%%%%%
\subsection{Include by Input}
\label{sec:input}

Including child documents by |\include| has some restrictions by design.
Most notably, the content of a child document always occupies
its own set of pages; pages cannot be shared between child documents.
Usually, this behaviour makes perfect sense
because each child document contain an essential part of the document.
However, in some situations it may be desirable to compose
a document from a collection of parts
without having mandatory page breaks between then.
For this case, the package
provides a mechanism to include parts
by |\input| which can also be processed individually.
However, by construction this mechanism
requires manual handling of the content to be output.

%%%%%%%%%%%%%%%%%%%%%%%%%%%%%%%%%%%%%%%%
\DescribeMacro{\ifchilddocmanual}
The main file should be prepared as usual, see \secref{sec:include}.
However, the document body must make a distinction
between processing of an individual part and of the main document, e.g.:
%
\begin{center}
\begin{tabular}{l}
|\ifchilddocmanual|\\
|\input{\childdocname}|\\
|\||else|\\
\textit{document body with }|\input{|\textit{part}|}|\\
|\||fi|
\end{tabular}
\end{center}
%
The conditional |\ifchilddocmanual| is true whenever
a part to be included by |\input| is being compiled,
and the name of the part is stored in |\childdocname|.

%%%%%%%%%%%%%%%%%%%%%%%%%%%%%%%%%%%%%%%%
\DescribeMacro{\childdocby}
Each part to be included by |\input| should start with:
%
\begin{center}
\begin{tabular}{l}
|\input{childdoc.def}|\\
|\childdocby{|\textit{main}|}|\\
\end{tabular}
\end{center}
%
The directive |\childdocby| is similar to |\childdocof|
described in \secref{sec:include},
but the subsequent selection of content must be done manually.
To that end, both |\ifchilddoc| and |\ifchilddocmanual|
will be true upon processing of a part,
and the name of the part is stored in |\childdocname|.
Note that |\jobname| will be set to the filename of the current part
so that each part receives an individual |.aux| file
that does not interfere with the |.aux| file(s) of the main document.
This behaviour can be altered by the alternative form
|\childdocby[*]{|\textit{main}|}| (with a non-empty optional argument)
which uses the |.aux| file of the main document
by setting |\jobname| to \textit{main}.

%%%%%%%%%%%%%%%%%%%%%%%%%%%%%%%%%%%%%%%%%%%%%%%%%%%%%%%%%%%%%%%%%%%%%%%%%%%%%%%%
\subsection{Driver Development}
\label{sec:driver}

The \textsf{childdoc} mechanism can also be use for the development
of definition files such as \LaTeX{} styles or classes.
This case differs from the above setup with multiple parts
included by |\include| in that no |\includeonly| should be invoked.
This can be achieved by starting the include file
(before |\ProvidesPackage|) with:
%
\begin{center}
\begin{tabular}{l}
|\input{childdoc.def}|\\
|\childdocforward{|\textit{main}|}|\\
\end{tabular}
\end{center}
%
or alternatively with:
%
\begin{center}
\begin{tabular}{l}
|\input{childdoc.def}|\\
|\childdocby{|\textit{main}|}|\\
\end{tabular}
\end{center}
%
Both forms have slightly different effects as described above.
The main file is prepared as usual, see \secref{sec:include}.

%%%%%%%%%%%%%%%%%%%%%%%%%%%%%%%%%%%%%%%%%%%%%%%%%%%%%%%%%%%%%%%%%%%%%%%%%%%%%%%%
\subsection{Legacy Detection}
\label{sec:detection}

The directive |\childdocmain| in the main file can detect
whether the complete document or merely a child is to be compiled
even without using the directive |\childdocof|.
This method is deprecated because it is less robust
and there is no compelling reason to use it;
it is merely provided for backward compatibility
and it may be removed in future versions.

If the detection mechanism is to be used,
it is mandatory to correctly specify
the filename of the main file as the argument of |\childdocmain|:
%
\begin{center}
\begin{tabular}{l}
|\input{childdoc.def}|\\
|\childdocmain{|\textit{main}|}|\\
\end{tabular}
\end{center}
%
If |\jobname| does not match the argument \textit{main} of |\childdocmain|,
it is assumed that |\jobname| points to the child file to be compiled.
When using |\childdocmain| with the main file specified as argument,
it suffices to start a child file
with just |\input{|\textit{main}|}|
without loading of the package and using |\childdocof|.
If instead all processing is done
with the appropriate \textsf{childdoc} directives,
the argument of \textit{main} of |\childdocmain| can be empty.

An alternative version of the command line processing described
in \secref{sec:commandline} using the detection mechanism reads:
%
\begin{center}
|... -jobname "|\textit{target}|" "|[\textit{flags}]%
[|\def\jobname{|\textit{dest}|}|]|\input{|\textit{main}|}"|
\end{center}

%%%%%%%%%%%%%%%%%%%%%%%%%%%%%%%%%%%%%%%%%%%%%%%%%%%%%%%%%%%%%%%%%%%%%%%%%%%%%%%%
\subsection{Manual Code}
\label{sec:manual}

In case one cannot be certain whether the definitions file |childdoc.def|
is installed on the target \TeX{} distribution
and one prefers not to ship it,
it is conceivable to paste a few relevant commands into the sources.

To that end, drop all statements |\input{childdoc.def}|
and perform the replacements as outlined below.
Instead of |\childdocmain{|\textit{main}|}| add the following code
to the top of the main file:
%
\begin{center}
\begin{tabular}{l}
|\||ifdefined\childdocname\endinput\||fi\newif\ifchilddoc|\\
|\edef\childdocname{\scantokens\expandafter{\jobname\noexpand}}|\\
|\def\childdocmain{|\textit{main}|}\||ifx\childdocmain\childdocname\||else|\\
|\childdoctrue\includeonly{\childdocname}\let\jobname\childdocmain\||fi|\\
\end{tabular}
\end{center}
%
Instead of |\childdocof{|\textit{main}|}| just include the main file
at the top of each child file:
%
\begin{center}
|\input{|\textit{main}|}|
\end{center}
%
A simple redirection |\childdocforward{|\textit{dest}|}| is achieved by:
%
\begin{center}
|\def\jobname{|\textit{dest}|}\input{\jobname}|
\end{center}
%
The redirection with prefix
|\childdocforwardprefix[|\textit{prefix}|]{|\textit{dest}|}|
is accomplished by:
%
\begin{center}
\begin{tabular}{l}
|{\edef\jobname{\scantokens\expandafter{\jobname\noexpand}}|\\
|\def\redirectjob |\textit{prefix}|#1~~~{\gdef\jobname{|\textit{dest}|#1}}|\\
|\expandafter\redirectjob\jobname~~~}\input{\jobname}|
\end{tabular}
\end{center}

In an alternative approach,
child documents can be compiled by a specific command line
without additional code or specific definitions:
%
\begin{center}
|... -jobname "|\textit{target}|" "|[\textit{flags}]%
|\includeonly{|\textit{dest}|}\input{|\textit{main}|}"|
\end{center}
%

%%%%%%%%%%%%%%%%%%%%%%%%%%%%%%%%%%%%%%%%%%%%%%%%%%%%%%%%%%%%%%%%%%%%%%%%%%%%%%%%
%%%%%%%%%%%%%%%%%%%%%%%%%%%%%%%%%%%%%%%%%%%%%%%%%%%%%%%%%%%%%%%%%%%%%%%%%%%%%%%%
\section{Information}

%%%%%%%%%%%%%%%%%%%%%%%%%%%%%%%%%%%%%%%%%%%%%%%%%%%%%%%%%%%%%%%%%%%%%%%%%%%%%%%%
\subsection{Copyright}

Copyright \copyright{} 2017--2018 Niklas Beisert

This work may be distributed and/or modified under the
conditions of the \LaTeX{} Project Public License, either version 1.3
of this license or (at your option) any later version.
The latest version of this license is in
  \url{http://www.latex-project.org/lppl.txt}
and version 1.3 or later is part of all distributions of \LaTeX{}
version 2005/12/01 or later.

This work has the LPPL maintenance status `maintained'.

The Current Maintainer of this work is Niklas Beisert.

This work consists of the files |README.txt|, |childdoc.ins| and |childdoc.dtx|
as well as the derived files |childdoc.def|, |cdocsamp.tex|
with |cdocsch1.tex|, |cdocsch2.tex|, |cdocspt3.tex|, |cdocspt4.tex|,
|cdocsdrf.tex|, |cdocsfn1.tex|, |cdocsfn2.tex|
as well as |childdoc.pdf|.

%%%%%%%%%%%%%%%%%%%%%%%%%%%%%%%%%%%%%%%%%%%%%%%%%%%%%%%%%%%%%%%%%%%%%%%%%%%%%%%%
\subsection{Files and Installation}

The package consists of the files:
%
\begin{center}
\begin{tabular}{ll}
    |README.txt|   & readme file \\
    |childdoc.ins| & installation file \\
    |childdoc.dtx| & source file \\
    |childdoc.def| & definition file \\
    |cdocsamp.tex| & sample main file \\
    |cdocsch1.tex| & sample include file \\
    |cdocsch2.tex| & sample include file \\
    |cdocspt3.tex| & sample part file \\
    |cdocspt4.tex| & sample part file \\
    |cdocsdrf.tex| & sample redirection file \\
    |cdocsfn1.tex| & sample redirection file \\
    |cdocsfn2.tex| & sample redirection file \\
    |childdoc.pdf| & manual
\end{tabular}
\end{center}
%
The distribution consists of the files
|README.txt|, |childdoc.ins| and |childdoc.dtx|.
%
\begin{itemize}
\item
Run (pdf)\LaTeX{} on |childdoc.dtx|
to compile the manual |childdoc.pdf| (this file).
\item
Run \LaTeX{} on |childdoc.ins| to create the definitions file |childdoc.def|
and the sample |cdocsamp.tex| with include files
|cdocsch1.tex|, |cdocsch2.tex|, |cdocspt3.tex|, |cdocspt4.tex|,
|cdocsdrf.tex|, |cdocsfn1.tex|, |cdocsfn2.tex|.
Then copy the file |childdoc.def| to an appropriate directory of your \LaTeX{}
distribution, e.g.\ \textit{texmf-root}|/tex/latex/childdoc|.
\end{itemize}

%%%%%%%%%%%%%%%%%%%%%%%%%%%%%%%%%%%%%%%%%%%%%%%%%%%%%%%%%%%%%%%%%%%%%%%%%%%%%%%%
\subsection{Related CTAN Packages}

There are several other packages which offer a similar functionality:
%
\begin{itemize}
\item
The packages
\href{http://ctan.org/pkg/docmute}{\textsf{docmute}},
\href{http://ctan.org/pkg/includex}{\textsf{includex}} and
\href{http://ctan.org/pkg/standalone}{\textsf{standalone}}
provide commands to include only the document body of
a child file thus allowing both files to be compiled individually.
\item
The packages \href{http://ctan.org/pkg/subdocs}{\textsf{subdocs}}
and \href{http://ctan.org/pkg/subfiles}{\textsf{subfiles}}
provide structures in which the main and child documents can be
encapsulated and allowing them to be compiled individually.
The inclusion mechanism is different from the conventional |\include|.
\item
The package \href{http://ctan.org/pkg/combine}{\textsf{combine}}
is an elaborate solution to combine several documents into one.
\end{itemize}
%
See also the CTAN topic \href{http://ctan.org/topic/subdocs}{\textsf{subdocs}}
for further related packages.
The present package differs from the above solutions in that
a document structure constructed with the conventional |\include| mechanism
just needs two extra commands at the top of every file
such that all constituent files can be compiled individually.

%%%%%%%%%%%%%%%%%%%%%%%%%%%%%%%%%%%%%%%%%%%%%%%%%%%%%%%%%%%%%%%%%%%%%%%%%%%%%%%%
%\subsection{Feature Suggestions}
%
%The following is a list of features which may be useful for future
%versions of this package:
%%
%\begin{itemize}
%\item
%\ldots
%\end{itemize}

%%%%%%%%%%%%%%%%%%%%%%%%%%%%%%%%%%%%%%%%%%%%%%%%%%%%%%%%%%%%%%%%%%%%%%%%%%%%%%%%
\subsection{Revision History}

%%%%%%%%%%%%%%%%%%%%%%%%%%%%%%%%%%%%%%%%
\paragraph{v2.0:} 2018/12/30

\begin{itemize}
\item
immediate forward processing
\item
added |\childdocby| mechanism
\item
manual restructured
\end{itemize}

%%%%%%%%%%%%%%%%%%%%%%%%%%%%%%%%%%%%%%%%
\paragraph{v1.6:} 2018/01/17

\begin{itemize}
\item
application for development of include files
\item
corrections to manual
\end{itemize}

%%%%%%%%%%%%%%%%%%%%%%%%%%%%%%%%%%%%%%%%
\paragraph{v1.5:} 2017/05/21

\begin{itemize}
\item
more complete structuring introduced
\item
|\childdocof| introduced
\item
|\childdoc| renamed to |\childdocmain|
\item
|\childredirect| renamed to |\childdocforward| and |\childdocforwardprefix|
and functionality expanded
\end{itemize}

%%%%%%%%%%%%%%%%%%%%%%%%%%%%%%%%%%%%%%%%
\paragraph{v1.0:} 2017/04/27

\begin{itemize}
\item
manual and install package
\item
first version published on CTAN
\end{itemize}

%%%%%%%%%%%%%%%%%%%%%%%%%%%%%%%%%%%%%%%%
\paragraph{v0.6:} 2017/04/26

\begin{itemize}
\item
redirection mechanism added
\end{itemize}

%%%%%%%%%%%%%%%%%%%%%%%%%%%%%%%%%%%%%%%%
\paragraph{v0.5:} 2017/04/26

\begin{itemize}
\item
functionality in definition file
\end{itemize}


%%%%%%%%%%%%%%%%%%%%%%%%%%%%%%%%%%%%%%%%%%%%%%%%%%%%%%%%%%%%%%%%%%%%%%%%%%%%%%%%
%%%%%%%%%%%%%%%%%%%%%%%%%%%%%%%%%%%%%%%%%%%%%%%%%%%%%%%%%%%%%%%%%%%%%%%%%%%%%%%%
%%%%%%%%%%%%%%%%%%%%%%%%%%%%%%%%%%%%%%%%%%%%%%%%%%%%%%%%%%%%%%%%%%%%%%%%%%%%%%%%
\appendix

\settowidth\MacroIndent{\rmfamily\scriptsize 000\ }

 \DocInput{childdoc.dtx}

\end{document}
%</driver>
% \fi
%
% %%%%%%%%%%%%%%%%%%%%%%%%%%%%%%%%%%%%%%%%%%%%%%%%%%%%%%%%%%%%%%%%%%%%%%%%%%%%%%
% %%%%%%%%%%%%%%%%%%%%%%%%%%%%%%%%%%%%%%%%%%%%%%%%%%%%%%%%%%%%%%%%%%%%%%%%%%%%%%
% \section{Sample}
%\iffalse
%<*samplemain>
%\fi
%
% The following presents a sample document
% with two chapters, two parts, a title page,
% a compile flag as well as three forwarding files to set the flag.
% It consists of eight |.tex| files:
% \begin{center}
% \begin{tabular}{ll}
% |cdocsamp.tex|&main file\\
% |cdocsch1.tex|&include file for chapter 1\\
% |cdocsch2.tex|&include file for chapter 2\\
% |cdocspt3.tex|&include file for part 3\\
% |cdocspt4.tex|&include file for part 4\\
% |cdocsdrf.tex|&forwarding file for main file in draft mode\\
% |cdocsfi1.tex|&forwarding file for final version of chapter 1\\
% |cdocsfi2.tex|&forwarding file for final version of chapter 2\\
% \end{tabular}
% \end{center}
% Each of the eight files can be compiled directly by the \LaTeX{} compiler.
%
% %%%%%%%%%%%%%%%%%%%%%%%%%%%%%%%%%%%%%%
% \paragraph{Main File.}
%
% The main file is called |cdocsamp.tex|.
%
% Load the \textsf{childdoc} definitions and
% declare the filename for the main document:
%    \begin{macrocode}
\input{childdoc.def}
\childdocmain{}
%    \end{macrocode}

% Optional override for |\version| flag:
%    \begin{macrocode}
%%\ifchilddoc\else\providecommand{\version}{draft}\fi
%    \end{macrocode}

% Define the default values for the |\version| flag
% (|final| for the main file and |draft| for childs):
%    \begin{macrocode}
\ifchilddoc
\providecommand{\version}{draft}
\else
\providecommand{\version}{final}
\fi
%    \end{macrocode}

% Load the standard document class:
%    \begin{macrocode}
\documentclass[12pt]{article}
%    \end{macrocode}

% Start the document body:
%    \begin{macrocode}
\begin{document}
%    \end{macrocode}

% Declare a title page.
% Print title, part of document being processed and version flag:
%    \begin{macrocode}
\addtocounter{page}{-1}
\begin{center}
{\LARGE\bfseries{}childdoc example\par}
\vspace{1cm}
\ifchilddoc
\ifchilddocmanual part\else chapter\fi:
`\childdocname' of `\childdocjob'\par
\else
main document: `\childdocjob'\par
\fi
version: \version\par
\end{center}
\newpage
%    \end{macrocode}

% Manually include selected file,
% otherwise process as usual:
%    \begin{macrocode}
\ifchilddocmanual
\section*{part `\childdocname'}
\input{\childdocname}
\else
%    \end{macrocode}

% Include the two chapters:
%    \begin{macrocode}
\include{cdocsch1}
\include{cdocsch2}
%    \end{macrocode}

% Include the two parts unless only chapters should be displayed:
%    \begin{macrocode}
\ifchilddoc\else
\section{part three}
\input{cdocspt3}
\section{part four}
\input{cdocspt4}
\fi
%    \end{macrocode}

% Process as usual until here:
%    \begin{macrocode}
\fi
%    \end{macrocode}

% End of document body:
%    \begin{macrocode}
\end{document}
%    \end{macrocode}
%\iffalse
%</samplemain>
%\fi
%
% %%%%%%%%%%%%%%%%%%%%%%%%%%%%%%%%%%%%%%
% \paragraph{Chapter Include Files.}
%
% The include files are called |cdocsch1.tex| and |cdocsch2.tex|.
%
%\iffalse
%<*samplechap1|samplechap2>
%\fi

% Optional override for |\version| flag:
%    \begin{macrocode}
%%\providecommand{\version}{final}
%    \end{macrocode}

% Include the main document:
%    \begin{macrocode}
\input{childdoc.def}
\childdocof{cdocsamp}
%    \end{macrocode}

%\iffalse
%</samplechap1|samplechap2>
%\fi
%
%\iffalse
%<*samplechap1>
%\fi
% Some text for chapter 1:
%    \begin{macrocode}
\section{one}
some text in chapter one
%    \end{macrocode}

%\iffalse
%</samplechap1>
%\fi
% Some text for chapter 2:
%\iffalse
%<*samplechap2>
%\fi
%    \begin{macrocode}
\section{two}
more text in chapter two
%    \end{macrocode}

%\iffalse
%</samplechap2>
%\fi
%
% %%%%%%%%%%%%%%%%%%%%%%%%%%%%%%%%%%%%%%
% \paragraph{Part Include Files.}
%
% The include files are called |cdocspt3.tex| and |cdocspt4.tex|.
%
%\iffalse
%<*samplepart3|samplepart4>
%\fi

% Optional override for |\version| flag:
%    \begin{macrocode}
%%\providecommand{\version}{final}
%    \end{macrocode}

% Include the main document:
%    \begin{macrocode}
\input{childdoc.def}
\childdocby{cdocsamp}
%    \end{macrocode}

%\iffalse
%</samplepart3|samplepart4>
%\fi
%
%\iffalse
%<*samplepart3>
%\fi
% Some text for part 3:
%    \begin{macrocode}
some text in part three
%    \end{macrocode}

%\iffalse
%</samplepart3>
%\fi
% Some text for part 4:
%\iffalse
%<*samplepart4>
%\fi
%    \begin{macrocode}
more text in part four
%    \end{macrocode}

%\iffalse
%</samplepart4>
%\fi
%
% %%%%%%%%%%%%%%%%%%%%%%%%%%%%%%%%%%%%%%
% \paragraph{Forwarding for a Complete Draft.}
%
% The following forwarding file |cdocsdrf.tex|
% compiles the main document in draft mode:
%\iffalse
%<*sampledraft>
%\fi
%    \begin{macrocode}
\def\version{draft}
\input{childdoc.def}
\childdocforward{cdocsamp}
%    \end{macrocode}

%\iffalse
%</sampledraft>
%\fi
%
% %%%%%%%%%%%%%%%%%%%%%%%%%%%%%%%%%%%%%%
% \paragraph{Forwarding for Final Version of the Chapters.}
%
% The following forwarding files |cdocsfn1.tex| and |cdocsfn2.tex|
% (with identical content)
% compile the final versions of the child documents
% |cdocsch1.tex| and |cdocsch2.tex|, respectively:
%\iffalse
%<*samplefinal>
%\fi
%    \begin{macrocode}
\def\version{final}
\input{childdoc.def}
\childdocforwardprefix[cdocsamp]{cdocsfn}{cdocsch}
%    \end{macrocode}

%\iffalse
%</samplefinal>
%\fi
%
% %%%%%%%%%%%%%%%%%%%%%%%%%%%%%%%%%%%%%%
% \paragraph{Command Line Processing.}
%
% The following three command lines generate the output files
% |cdocscld|, |cdocscl1| and |cdocscl2|
% which should be identical to
% |cdocsdrf|, |cdocsch1| and |cdocsfn2|, respectively:
% \begin{center}
% \begin{tabular}{l}
% |latex -jobname cdocscld \|\\
% |  "\def\version{draft}\input{childdoc.def}\childdocforward{cdocsamp}"|\\
% |latex -jobname cdocscl1 \|\\
% |  "\input{childdoc.def}\childdocforward[cdocsamp]{cdocsch1}"|\\
% |latex -jobname cdocscl2 \|\\
% |  "\def\version{final}\input{childdoc.def}\childdocforward{cdocsch2}"|
% \end{tabular}
% \end{center}
% Note that the trailing backslash on each first line
% merely continues the input to the second line
% (for convenient cut ant paste).
% Furthermore, the command |latex| can be replaced by any
% of its alternative versions such as |pdflatex|.
%
% %%%%%%%%%%%%%%%%%%%%%%%%%%%%%%%%%%%%%%%%%%%%%%%%%%%%%%%%%%%%%%%%%%%%%%%%%%%%%%
% %%%%%%%%%%%%%%%%%%%%%%%%%%%%%%%%%%%%%%%%%%%%%%%%%%%%%%%%%%%%%%%%%%%%%%%%%%%%%%
% \section{Implementation}
%\iffalse
%<*package>
%\fi
%
% This section describes the definitions file |childdoc.def|.

% The definitions cannot be loaded using |\usepackage| or |\RequirePackage|
% which has a mechanism to prevent loading a style file more than once.
% When loading the definitions by means of |\input|
% multiple instances have to be prevented manually:
%\iffalse
%This code needs to be before the `\ProvidesFile' directive
%which is defined at the beginning of this file.
%Therefore it is also placed there and commented out here.
%</package>
%<*discard>
%\fi
%    \begin{macrocode}
\ifdefined\childdocmain\endinput\fi
%    \end{macrocode}
%\iffalse
%</discard>
%<*package>
%\fi
%
% \macro{\ifchilddoc}
% \macro{\ifchilddocmanual}
% The conditional |\ifchilddoc| tells whether a
% child (true) or main (false) document is being compiled.
% The conditional |\ifchilddocmanual| tells whether
% the |\includeonly| mechanism is used (false) or
% the selection of child files must be performed manually (true).
% The definitions initialise to false:
%    \begin{macrocode}
\newif\ifchilddoc
\newif\ifchilddocmanual
%    \end{macrocode}

% \macro{\childdocname}
% \macro{\childdocjob}
% The macro |\childdocname| stores the name of the main document
% to be compiled. The macro |\childdocjob| stores the name of
% the document on which the \LaTeX{} compiler was originally invoked.
% The content of |\jobname| cannot be compared
% to filenames specified in the source due to different catcodes.
% The following code rescans |\jobname|, stores the result
% in |\childdocname| and saves a copy in |\childdocjob|:
%    \begin{macrocode}
\edef\childdocname{\scantokens\expandafter{\jobname\noexpand}}
\let\childdocjob\childdocname
%    \end{macrocode}

% \macro{\childdocdisable}
% The macro |\childdocdisable| prevents the main file
% from being processed more than once.
% At this stage, the main document command |\childdocmain|
% is assumed to be called once again where it should do nothing.
% Any subsequent call to it should prevent
% a secondary processing of the main document
% It overwrites the forwarding commands
% |\childdocof| and |\childdocforward|
% with empty macros to prevent further inclusions of the main document:
%    \begin{macrocode}
\newcommand{\childdocdisable}
{
  \renewcommand{\childdocmain}[1]{\renewcommand{\childdocmain}[1]{\endinput}}
  \renewcommand{\childdocof}[1]{}
  \renewcommand{\childdocby}[2][]{}
  \renewcommand{\childdocforward}[2][]{}
  \renewcommand{\childdocdisable}{}
}
%    \end{macrocode}

% \macro{\childdocmain}
% The macro |\childdocmain| is to be called at the top of the main file
% with nothing or the main filename (without extension) as argument.
% First, it breaks loops.
% If the argument is not empty and does not match |\childdocname|
% (which is set by the first inclusion of |childdoc.def|),
% |\ifchilddoc| is set to true, |\includeonly| is applied to the child file
% and |\jobname| is set to the main file
% (for proper handling of |.aux| files):
%    \begin{macrocode}
\newcommand{\childdocmain}[1]
{
  \childdocdisable\childdocmain{}
  \if?#1?\else
    \begingroup
      \def\childdoctmp{#1}
      \ifx\childdoctmp\childdocname
        \def\childdoctmp{}
      \else
        \def\childdoctmp
        {
          \childdoctrue
          \includeonly{\childdocname}
          \def\childdocjob{#1}
          \def\jobname{#1}
        }
      \fi
      \expandafter
    \endgroup
    \childdoctmp
  \fi
}
%    \end{macrocode}

% \macro{\childdocof}
% The command |\childdocof| redirects
% compilation to the main file |#1|.
%    \begin{macrocode}
\newcommand{\childdocof}[1]
{
  \childdocdisable
  \childdoctrue
  \includeonly{\childdocname}
  \def\jobname{#1}
  \def\childdocjob{#1}
  \input{#1}
}
%    \end{macrocode}

% \macro{\childdocby}
% The command |\childdocby| ....
%    \begin{macrocode}
\newcommand{\childdocby}[2][]
{
  \childdocdisable
  \childdoctrue
  \childdocmanualtrue
  \if?#1?\else
    \def\jobname{#2}
  \fi
  \def\childdocjob{#2}
  \input{#2}
  \endinput
}
%    \end{macrocode}

% \macro{\childdocforward}
% The command |\childdocforward| redirects
% compilation to the main file or
% (if the optional argument is given) a child file.
% Parameters are set as if the main file
% or a child file starting with |\childdocof| was compiled.
% Then compilation is handed over to the main file:
%    \begin{macrocode}
\newcommand{\childdocforward}[2][]
{
  \begingroup
    \if?#1?
      \def\childdoctmp
      {
        \def\childdocname{#2}
        \def\childdocjob{#2}
        \def\jobname{#2}
        \input{#2}
        \endinput
      }
    \else
      \def\childdoctmp
      {
        \childdocdisable
        \def\childdocname{#2}
        \childdoctrue
        \includeonly{#2}
        \def\childdocjob{#1}
        \def\jobname{#1}
        \input{#1}
        \endinput
      }
    \fi
    \expandafter
  \endgroup
  \childdoctmp
}
%    \end{macrocode}

% \macro{\childdocforwardprefix}
% The command |\childdocforwardprefix| redirects
% compilation to the main or a child file by means of a pattern.
% The prefix |#1| in the current filename is replaced by |#2|
% and the suffix of the current filename is kept
% (it is assumed that the filename does not contain the substring `|~~~|'
% which is used as a delimiter).
% Compilation is handed over to the new file by |\childdocforward|:
%    \begin{macrocode}
\newcommand{\childdocforwardprefix}[3][]
{
  \begingroup
    \def\childdocextract #2##1~~~{\def\childdoctmp{\childdocforward[#1]{#3##1}}}
    \expandafter\childdocextract\childdocname~~~
    \expandafter
  \endgroup
  \childdoctmp
}
%    \end{macrocode}

% \macro{\childdoc}
% The deprecated macro |\childdoc| is a legacy version of |\childdocmain|:
%    \begin{macrocode}
\newcommand{\childdoc}{\childdocmain}
%    \end{macrocode}

% \macro{\childdocredirect}
% The deprecated macro |\childdocredirect| is a legacy version
% of |\childdocforward| and |\childdocforwardprefix|:
%    \begin{macrocode}
\newcommand{\childdocredirect}[2][]
{
  \begingroup
    \if?#1?
      \def\childdoctmp{\childdocforward{#2}}
    \else
      \def\childdoctmp{\childdocforwardprefix{#1}{#2}}
    \fi
    \expandafter
  \endgroup
  \childdoctmp
}
%    \end{macrocode}

%\iffalse
%</package>
%\fi
%
\endinput
|\\
|\childdocmain{|\textit{main}|}|\\
\end{tabular}
\end{center}
%
If |\jobname| does not match the argument \textit{main} of |\childdocmain|,
it is assumed that |\jobname| points to the child file to be compiled.
When using |\childdocmain| with the main file specified as argument,
it suffices to start a child file
with just |\input{|\textit{main}|}|
without loading of the package and using |\childdocof|.
If instead all processing is done
with the appropriate \textsf{childdoc} directives,
the argument of \textit{main} of |\childdocmain| can be empty.

An alternative version of the command line processing described
in \secref{sec:commandline} using the detection mechanism reads:
%
\begin{center}
|... -jobname "|\textit{target}|" "|[\textit{flags}]%
[|\def\jobname{|\textit{dest}|}|]|\input{|\textit{main}|}"|
\end{center}

%%%%%%%%%%%%%%%%%%%%%%%%%%%%%%%%%%%%%%%%%%%%%%%%%%%%%%%%%%%%%%%%%%%%%%%%%%%%%%%%
\subsection{Manual Code}
\label{sec:manual}

In case one cannot be certain whether the definitions file |childdoc.def|
is installed on the target \TeX{} distribution
and one prefers not to ship it,
it is conceivable to paste a few relevant commands into the sources.

To that end, drop all statements |% \iffalse
%
% childdoc.dtx Copyright (C) 2017-2018 Niklas Beisert
%
% This work may be distributed and/or modified under the
% conditions of the LaTeX Project Public License, either version 1.3
% of this license or (at your option) any later version.
% The latest version of this license is in
%   http://www.latex-project.org/lppl.txt
% and version 1.3 or later is part of all distributions of LaTeX
% version 2005/12/01 or later.
%
% This work has the LPPL maintenance status `maintained'.
%
% The Current Maintainer of this work is Niklas Beisert.
%
% This work consists of the files childdoc.dtx and childdoc.ins
% and the derived files childdoc.def and cdocsamp.tex with
% cdocsch1.tex, cdocsch2.tex, cdocsdrf.tex, cdocsfn1.tex, cdocsfn2.tex.
%
%<package>\ifdefined\childdocmain\endinput\fi
%<package>\ProvidesFile{childdoc.def}[2018/12/30 v2.0 child document driver]
%<samplemain>\ProvidesFile{cdocsamp.tex}[2018/12/30 v2.0 sample for childdoc]
%<*driver>
%\ProvidesFile{childdoc.drv}[2018/12/30 v2.0 childdoc reference manual file]
\PassOptionsToClass{10pt,a4paper}{article}
\documentclass{ltxdoc}

\usepackage[margin=35mm]{geometry}
\usepackage{hyperref}
\usepackage{hyperxmp}
\usepackage[usenames]{color}

\hypersetup{colorlinks=true}
\hypersetup{pdfstartview=FitH}
\hypersetup{pdfpagemode=UseNone}
\hypersetup{pdfsource={}}
\hypersetup{pdflang={en-UK}}
\hypersetup{pdfcopyright={Copyright 2017-2018 Niklas Beisert.
  This work may be distributed and/or modified under the
  conditions of the LaTeX Project Public License, either version 1.3
  of this license or (at your option) any later version.}}
\hypersetup{pdflicenseurl={http://www.latex-project.org/lppl.txt}}
\hypersetup{pdfcontactaddress={ETH Zurich, ITP, HIT K,
  Wolfgang-Pauli-Strasse 27}}
\hypersetup{pdfcontactpostcode={8093}}
\hypersetup{pdfcontactcity={Zurich}}
\hypersetup{pdfcontactcountry={Switzerland}}
\hypersetup{pdfcontactemail={nbeisert@itp.phys.ethz.ch}}
\hypersetup{pdfcontacturl={http://people.phys.ethz.ch/\xmptilde nbeisert/}}

\newcommand{\secref}[1]{\hyperref[#1]{section \ref*{#1}}}

\parskip1ex
\parindent0pt
\let\olditemize\itemize
\def\itemize{\olditemize\parskip0pt}

\begin{document}

\title{The \textsf{childdoc} Package}
\hypersetup{pdftitle={The childdoc Package}}
\author{Niklas Beisert\\[2ex]
  Institut f\"ur Theoretische Physik\\
  Eidgen\"ossische Technische Hochschule Z\"urich\\
  Wolfgang-Pauli-Strasse 27, 8093 Z\"urich, Switzerland\\[1ex]
  \href{mailto:nbeisert@itp.phys.ethz.ch}
  {\texttt{nbeisert@itp.phys.ethz.ch}}}
\hypersetup{pdfauthor={Niklas Beisert}}
\hypersetup{pdfsubject={Manual for the LaTeX2e Package childdoc}}
\date{30 December 2018, \textsf{v2.0}}
\maketitle

\begin{abstract}\noindent
\textsf{childdoc} is a \LaTeXe{} package
that enables the direct compilation
of document sections included by |\include|
to individual files.
\end{abstract}

\begingroup
\parskip0ex
\tableofcontents
\endgroup

%%%%%%%%%%%%%%%%%%%%%%%%%%%%%%%%%%%%%%%%%%%%%%%%%%%%%%%%%%%%%%%%%%%%%%%%%%%%%%%%
%%%%%%%%%%%%%%%%%%%%%%%%%%%%%%%%%%%%%%%%%%%%%%%%%%%%%%%%%%%%%%%%%%%%%%%%%%%%%%%%
\section{Introduction}

\LaTeX{} provides a mechanism to structure a large document (such as a book)
into a main file and several child files (containing the chapters)
using the |\include| command.
This mechanism is beneficial for documents
which span hundreds of pages in order to
make the source file(s) more manageable.
Moreover, compilation can be restricted to
selected child files by means of the |\includeonly| command.
The latter feature can be used to reduce the compilation time while editing
(this was significantly more useful in the earlier days of \LaTeX{})
or to generate a smaller document which is easier to navigate.
Another application of |\includeonly| is to generate
documents consisting of selected parts of the complete document.

However, there are a few drawbacks of the plain |\include| mechanism:
\begin{itemize}
\item
The child files cannot be compiled on their own,
they can only be compiled via the main file.
A naive editing environment
(such as a text editor with an option
to have the current file processed by \LaTeX)
may require one to switch to the main file before compiling;
attempting to compile the child file produces errors.
\item
The main file must be modified (each time)
to adjust the |\includeonly| command
to the present needs. This easily leaves the main file in a messy state.
\item
The generated document will always carry the filename
of the main document. This is inconvenient if
several child files are to be compiled and
to be kept for distribution.
\end{itemize}

The present package provides a simple interface
to make child files individually compilable by \LaTeX{}.
Compiling a child file then has the same effect as compiling
the main file with an |\includeonly| command
to select the appropriate child.
Moreover the generated document will carry the name of the child
rather than the main file.
This resolves all three above issues.

This feature is meant to make the editing of books,
thesis documents and lecture notes somewhat more convenient.
However, the package can also be used efficiently for
composing a series of documents (such as exercise sheets)
which are typically distributed individually.
It then assists the author in generating the individual documents
(potentially in different versions)
as well as a document containing the collected series.
Another application is in developing style files
or other kinds of included material
where compilation of the style file could redirect
to a sample or test file.

%%%%%%%%%%%%%%%%%%%%%%%%%%%%%%%%%%%%%%%%%%%%%%%%%%%%%%%%%%%%%%%%%%%%%%%%%%%%%%%%
%%%%%%%%%%%%%%%%%%%%%%%%%%%%%%%%%%%%%%%%%%%%%%%%%%%%%%%%%%%%%%%%%%%%%%%%%%%%%%%%
\section{Usage}

First of all, the package \textsf{childdoc} is \emph{not} a standard
\LaTeXe{} |.sty| style file! Therefore it needs to be invoked in
a non-standard way.

%%%%%%%%%%%%%%%%%%%%%%%%%%%%%%%%%%%%%%%%%%%%%%%%%%%%%%%%%%%%%%%%%%%%%%%%%%%%%%%%
\subsection{Included Files}
\label{sec:include}

%%%%%%%%%%%%%%%%%%%%%%%%%%%%%%%%%%%%%%%%
\DescribeMacro{\childdocmain}
To use the package, add the commands
\begin{center}
\begin{tabular}{l}
|\input{childdoc.def}|\\
|\childdocmain{}|\\
\end{tabular}
\end{center}
at the very top of the main \LaTeX{} file,
in particular \emph{before} the |\documentclass| statement!
The argument of |\childdocmain| should be left empty
(but it must be present).

%%%%%%%%%%%%%%%%%%%%%%%%%%%%%%%%%%%%%%%%
\DescribeMacro{\childdocof}
Furthermore, add the commands
\begin{center}
\begin{tabular}{l}
|\input{childdoc.def}|\\
|\childdocof{|\textit{main}|}|\\
\end{tabular}
\end{center}
at the top of every child file \textit{child}
which is included by |\include{|\textit{child}|}|
from within the main file
(or at least for those files to be compiled individually).
The argument \textit{main} must be the filename of the main file.

There are a couple of
considerations in setting up the main and child documents:

%%%%%%%%%%%%%%%%%%%%%%%%%%%%%%%%%%%%%%%%
\paragraph{Restrictions.}

Please note the following restrictions:
\begin{itemize}
\item
|\childdocmain| must be called with one argument \textit{main}
to ensure compatibility with earlier version of the package.
It must either be empty (|\childdocmain{}|)
or precisely match the filename of the main file in which it is specified.
See \secref{sec:detection} for further information.
\item
The filename \textit{main} must be specified without the |.tex| extension.
\item
The filename \textit{main} is case sensitive
(even in case-insensitive file systems)
due to internal string comparison.
\item
The argument \textit{main} should be fully expanded, it cannot be a macro.
\item
Subdirectories and special characters should be avoided in filenames.
\item
The command |\childdocmain{|\textit{main}|}| must be followed by a whitespace.
It should not be followed immediately by another command
or by a comment mark `|%|'.
This is because the \TeX{} parser reads the token immediately following
the argument of |\childdocmain| and puts it
at the beginning of every child section;
however, a white\-space is ignored.
\end{itemize}

%%%%%%%%%%%%%%%%%%%%%%%%%%%%%%%%%%%%%%%%
\paragraph{Content of Main File.}

It is advisable to place all content in the child files included by |\include|.
Any output contained in the main file will appear in all child documents
unless suppressed manually;
it cannot be suppressed automatically by the |\includeonly| directive
and thus should normally be avoided.
A method to include some content in the main file
by means of conditional processing is described in \secref{sec:conditional}.

%%%%%%%%%%%%%%%%%%%%%%%%%%%%%%%%%%%%%%%%
\paragraph{Page Numbering.}

When only a part of the document is compiled,
the appropriate numbering of pages
(as well as other status parameters)
is determined from the |.aux| files.
The latter contain information from previous passes.
However this information needs to propagate through
all intermediate child documents.
Therefore the page numbering in child documents may well
be inconsistent until the complete document is compiled at least once.

A useful (if unconventional) way to always ensure a consistent
page numbering is to restart the numbering in each child document
and denote the pages by `\textit{child}|.|\textit{page}'
where \textit{child} represents the chapter/section number of the child file.
This can be achieved by the command
|\numberwithin{page}{|\textit{child}|}|
of the \textsf{amsmath} package
where \textit{child} can be |chapter| or |section|
depending on the chosen structuring.
Alternatively, one can modify the macro |\thepage| appropriately
and reset the counter |page| at the start of each child file.

%%%%%%%%%%%%%%%%%%%%%%%%%%%%%%%%%%%%%%%%%%%%%%%%%%%%%%%%%%%%%%%%%%%%%%%%%%%%%%%%
\subsection{Conditional Processing}
\label{sec:conditional}

The package provides a mechanism to compile different versions
of a document. To customise the versions further some conditional processing
can come in handy to distinguish which version is being compiled.
The package provides two macros to describe the compilation context:

%%%%%%%%%%%%%%%%%%%%%%%%%%%%%%%%%%%%%%%%
\DescribeMacro{\ifchilddoc}
The conditional |\ifchilddoc| distinguishes between the compilation of
child documents and the main document:
%
\begin{center}
|\ifchilddoc |\textit{child-code}| |[|\||else |\textit{main-code}]| \||fi|
\end{center}

%%%%%%%%%%%%%%%%%%%%%%%%%%%%%%%%%%%%%%%%
\DescribeMacro{\childdocname}
\DescribeMacro{\childdocjob}
The macro |\childdocname| contains the filename (without extension)
of the main or child file being processed.
Note that |\childdocjob| will always contain the name of the main file.

%%%%%%%%%%%%%%%%%%%%%%%%%%%%%%%%%%%%%%%%
\paragraph{Title Page.}

Conditional processing can be used to include a title or banner page
in the main document when proper precautions are taken.
Importantly, the code in the main file should ensure that the page counter
(as well as other status parameters which are stored in the |.aux| files)
takes the same value after the conditional processing.
Otherwise the page numbers may take divergent values
depending on which part is compiled.

For example, a title page could be declared by:
%
\begin{center}
\begin{tabular}{l}
|\ifchilddoc\||else|\\
|\addtocounter{page}{-1}|\\
\textit{code for title page}\\
|\newpage|\\
|\||fi|
\end{tabular}
\end{center}
%
A banner page for the child documents can be generated by:
%
\begin{center}
\begin{tabular}{l}
|\ifchilddoc|\\
|\addtocounter{page}{-1}|\\
\textit{code for banner page}\\
|\newpage|\\
|\||fi|
\end{tabular}
\end{center}
%
Here one could write a message such as:
\begin{center}
|This is the part \childdocname{} of \childdocjob{}.|
\end{center}

%%%%%%%%%%%%%%%%%%%%%%%%%%%%%%%%%%%%%%%%%%%%%%%%%%%%%%%%%%%%%%%%%%%%%%%%%%%%%%%%
\subsection{Flags}
\label{sec:flags}

The package makes it easy to generate different versions
of the main or child documents.
To this end compilation flags can be defined
and assigned different default values.
They will be particularly useful in conjunction
with the forwarding mechanism described in \secref{sec:forward}.

For example, it may be useful to have a flag |\version|
which can be set to |draft| or |final|.
The document source will contain some conditional code
depending on the value of |\version|.
Suppose further, the flag should default to |final| for the main file
and to |draft| for child files
which is a natural assignment for editing the document.
This is achieved by placing the following code
in the preamble of the main document
(below the |\childdocmain| directive):
%
\begin{center}
\begin{tabular}{l}
|\ifchilddoc|\\
|\providecommand{\version}{draft}|\\
|\||else|\\
|\providecommand{\version}{final}|\\
|\||fi|
\end{tabular}
\end{center}
%
The definition by |\providecommand| makes sure
that previous definitions are not overwritten.
Further statements |\providecommand{\version}{...}|
can thus be added before the above code to override it.

For the main file, one might add a line
(between |\childdocmain| and the above block)
%
\begin{center}
|%\ifchilddoc\||else\providecommand{\version}{draft}\||fi|
\end{center}
%
which can be uncommented to produce a draft version.
Likewise one can add a line to the very top of a child file
(above the |\childdocof{|\textit{main}|}| directive)
%
\begin{center}
|%\providecommand{\version}{final}|
\end{center}
%
which can be uncommented to produce the final version of this child document.

%%%%%%%%%%%%%%%%%%%%%%%%%%%%%%%%%%%%%%%%%%%%%%%%%%%%%%%%%%%%%%%%%%%%%%%%%%%%%%%%
\subsection{Forwarding}
\label{sec:forward}

Different versions of the main or child documents
using compilation flags as described in \secref{sec:flags}
can be (permanently) stored in different files
for convenient compilation, viewing and distribution.
To this end, the package defines a command
to pass on compilation to a different file:

%%%%%%%%%%%%%%%%%%%%%%%%%%%%%%%%%%%%%%%%
\DescribeMacro{\childdocforward}
The command |\childdocforward| redirects processing to
another source file:
%
\begin{center}
\begin{tabular}{l}
|\input{childdoc.def}|\\
|\childdocforward[|\textit{main}|]{|\textit{dest}|}|\\
\end{tabular}
\end{center}
%
The argument \textit{dest} is the destination file
(without extension).
It should be the main file or one of the child files.
Note that further \textsf{childdoc} directives
such as |\childdocof| and |\childdocforward|
in the indicated file will be processed in this form.
The optional argument \textit{main}
passes on directly to the main file \textit{main}
while pretending to compile the child \textit{dest}.
This form behaves as if \textit{dest}
issues |\childdocof{|\textit{main}|}| right away,
and no further \textsf{childdoc} directives will be processed.

%%%%%%%%%%%%%%%%%%%%%%%%%%%%%%%%%%%%%%%%
\DescribeMacro{\...prefix}
In the alternative form |\childdocforwardprefix|,
%
\begin{center}
\begin{tabular}{l}
|\input{childdoc.def}|\\
|\childdocforwardprefix[|\textit{main}|]{|\textit{prefix}|}{|\textit{dest}|}|
\end{tabular}
\end{center}
%
the destination file is determined by a pattern
depending on the current file:
To make this work, the current file must be called
`{\textit{prefix}\hspace{0.2em}\textit{suffix}}'
with \textit{prefix} matching precisely the argument.
Processing is then passed on to the file
`{\textit{dest}\hspace{0.2em}\textit{suffix}}'.
Surely, the same effect is achieved by
directly specifying the
argument `{\textit{dest}\hspace{0.2em}\textit{suffix}}'
in the first form.
However, that requires to set up a different file
for each child. With the alternative form of the command
all these files can have exactly the same content
which simplifies setting them up and maintaining them.

For example, the following file |draft.tex|
with a compilation flag |\version| as described in \secref{sec:flags}
compiles the main document as a draft:
%
\begin{center}
\begin{tabular}{l}
|\def\version{draft}|\\
|\input{childdoc.def}|\\
|\childdocforward{|\textit{main}|}|
\end{tabular}
\end{center}
%
Likewise, the following files |final|\textit{nn}|.tex|
compile the final version of the child document
|child|\textit{nn}|.tex|:
%
\begin{center}
\begin{tabular}{l}
|\def\version{final}|\\
|\input{childdoc.def}|\\
|\childdocforwardprefix{final}{child}|
\end{tabular}
\end{center}
%

Note that when several versions of a main file and/or of each child file
are to be generated, it may be convenient to set up a |Makefile| or
shell script to automatise the process.

%%%%%%%%%%%%%%%%%%%%%%%%%%%%%%%%%%%%%%%%%%%%%%%%%%%%%%%%%%%%%%%%%%%%%%%%%%%%%%%%
\subsection{Command Line Processing}
\label{sec:commandline}

The effect of redirection files can also be achieved by invoking
the \LaTeX{} compiler with a more elaborate command line.
Most conveniently this should be done as part
of a shell script or a |Makefile|.

When using \textsf{childdoc} in the main file, the following
command lines effectively perform a redirection
(note that depending on the shell being used,
backslashes may have to be doubled: `|\|' $\to$ `|\\|'):
%
\begin{center}
|... -jobname "|\textit{target}|" |\\|"|[\textit{flags}]%
|\input{childdoc.def}\childdocforward[|\textit{main}|]{|\textit{dest}|}"|
\end{center}
%
Here \textit{target} is the name of the output file,
\textit{main} is the name of the main file
and \textit{dest} is the name of the main or child file to be processed
(all filenames without extensions).
The optional argument \textit{main} can be omitted
if \textit{main} matches \textit{dest}.
Optionally, compilation \textit{flags} can be defined via |\def| commands.
This command line makes the \TeX{} engine believe
it is compiling the file \textit{target}
whose content is specified as the latter parameter.
The provided code then forwards the processing to
\textit{main} or \textit{dest} as described in \secref{sec:forward}.

%%%%%%%%%%%%%%%%%%%%%%%%%%%%%%%%%%%%%%%%%%%%%%%%%%%%%%%%%%%%%%%%%%%%%%%%%%%%%%%%
\subsection{Include by Input}
\label{sec:input}

Including child documents by |\include| has some restrictions by design.
Most notably, the content of a child document always occupies
its own set of pages; pages cannot be shared between child documents.
Usually, this behaviour makes perfect sense
because each child document contain an essential part of the document.
However, in some situations it may be desirable to compose
a document from a collection of parts
without having mandatory page breaks between then.
For this case, the package
provides a mechanism to include parts
by |\input| which can also be processed individually.
However, by construction this mechanism
requires manual handling of the content to be output.

%%%%%%%%%%%%%%%%%%%%%%%%%%%%%%%%%%%%%%%%
\DescribeMacro{\ifchilddocmanual}
The main file should be prepared as usual, see \secref{sec:include}.
However, the document body must make a distinction
between processing of an individual part and of the main document, e.g.:
%
\begin{center}
\begin{tabular}{l}
|\ifchilddocmanual|\\
|\input{\childdocname}|\\
|\||else|\\
\textit{document body with }|\input{|\textit{part}|}|\\
|\||fi|
\end{tabular}
\end{center}
%
The conditional |\ifchilddocmanual| is true whenever
a part to be included by |\input| is being compiled,
and the name of the part is stored in |\childdocname|.

%%%%%%%%%%%%%%%%%%%%%%%%%%%%%%%%%%%%%%%%
\DescribeMacro{\childdocby}
Each part to be included by |\input| should start with:
%
\begin{center}
\begin{tabular}{l}
|\input{childdoc.def}|\\
|\childdocby{|\textit{main}|}|\\
\end{tabular}
\end{center}
%
The directive |\childdocby| is similar to |\childdocof|
described in \secref{sec:include},
but the subsequent selection of content must be done manually.
To that end, both |\ifchilddoc| and |\ifchilddocmanual|
will be true upon processing of a part,
and the name of the part is stored in |\childdocname|.
Note that |\jobname| will be set to the filename of the current part
so that each part receives an individual |.aux| file
that does not interfere with the |.aux| file(s) of the main document.
This behaviour can be altered by the alternative form
|\childdocby[*]{|\textit{main}|}| (with a non-empty optional argument)
which uses the |.aux| file of the main document
by setting |\jobname| to \textit{main}.

%%%%%%%%%%%%%%%%%%%%%%%%%%%%%%%%%%%%%%%%%%%%%%%%%%%%%%%%%%%%%%%%%%%%%%%%%%%%%%%%
\subsection{Driver Development}
\label{sec:driver}

The \textsf{childdoc} mechanism can also be use for the development
of definition files such as \LaTeX{} styles or classes.
This case differs from the above setup with multiple parts
included by |\include| in that no |\includeonly| should be invoked.
This can be achieved by starting the include file
(before |\ProvidesPackage|) with:
%
\begin{center}
\begin{tabular}{l}
|\input{childdoc.def}|\\
|\childdocforward{|\textit{main}|}|\\
\end{tabular}
\end{center}
%
or alternatively with:
%
\begin{center}
\begin{tabular}{l}
|\input{childdoc.def}|\\
|\childdocby{|\textit{main}|}|\\
\end{tabular}
\end{center}
%
Both forms have slightly different effects as described above.
The main file is prepared as usual, see \secref{sec:include}.

%%%%%%%%%%%%%%%%%%%%%%%%%%%%%%%%%%%%%%%%%%%%%%%%%%%%%%%%%%%%%%%%%%%%%%%%%%%%%%%%
\subsection{Legacy Detection}
\label{sec:detection}

The directive |\childdocmain| in the main file can detect
whether the complete document or merely a child is to be compiled
even without using the directive |\childdocof|.
This method is deprecated because it is less robust
and there is no compelling reason to use it;
it is merely provided for backward compatibility
and it may be removed in future versions.

If the detection mechanism is to be used,
it is mandatory to correctly specify
the filename of the main file as the argument of |\childdocmain|:
%
\begin{center}
\begin{tabular}{l}
|\input{childdoc.def}|\\
|\childdocmain{|\textit{main}|}|\\
\end{tabular}
\end{center}
%
If |\jobname| does not match the argument \textit{main} of |\childdocmain|,
it is assumed that |\jobname| points to the child file to be compiled.
When using |\childdocmain| with the main file specified as argument,
it suffices to start a child file
with just |\input{|\textit{main}|}|
without loading of the package and using |\childdocof|.
If instead all processing is done
with the appropriate \textsf{childdoc} directives,
the argument of \textit{main} of |\childdocmain| can be empty.

An alternative version of the command line processing described
in \secref{sec:commandline} using the detection mechanism reads:
%
\begin{center}
|... -jobname "|\textit{target}|" "|[\textit{flags}]%
[|\def\jobname{|\textit{dest}|}|]|\input{|\textit{main}|}"|
\end{center}

%%%%%%%%%%%%%%%%%%%%%%%%%%%%%%%%%%%%%%%%%%%%%%%%%%%%%%%%%%%%%%%%%%%%%%%%%%%%%%%%
\subsection{Manual Code}
\label{sec:manual}

In case one cannot be certain whether the definitions file |childdoc.def|
is installed on the target \TeX{} distribution
and one prefers not to ship it,
it is conceivable to paste a few relevant commands into the sources.

To that end, drop all statements |\input{childdoc.def}|
and perform the replacements as outlined below.
Instead of |\childdocmain{|\textit{main}|}| add the following code
to the top of the main file:
%
\begin{center}
\begin{tabular}{l}
|\||ifdefined\childdocname\endinput\||fi\newif\ifchilddoc|\\
|\edef\childdocname{\scantokens\expandafter{\jobname\noexpand}}|\\
|\def\childdocmain{|\textit{main}|}\||ifx\childdocmain\childdocname\||else|\\
|\childdoctrue\includeonly{\childdocname}\let\jobname\childdocmain\||fi|\\
\end{tabular}
\end{center}
%
Instead of |\childdocof{|\textit{main}|}| just include the main file
at the top of each child file:
%
\begin{center}
|\input{|\textit{main}|}|
\end{center}
%
A simple redirection |\childdocforward{|\textit{dest}|}| is achieved by:
%
\begin{center}
|\def\jobname{|\textit{dest}|}\input{\jobname}|
\end{center}
%
The redirection with prefix
|\childdocforwardprefix[|\textit{prefix}|]{|\textit{dest}|}|
is accomplished by:
%
\begin{center}
\begin{tabular}{l}
|{\edef\jobname{\scantokens\expandafter{\jobname\noexpand}}|\\
|\def\redirectjob |\textit{prefix}|#1~~~{\gdef\jobname{|\textit{dest}|#1}}|\\
|\expandafter\redirectjob\jobname~~~}\input{\jobname}|
\end{tabular}
\end{center}

In an alternative approach,
child documents can be compiled by a specific command line
without additional code or specific definitions:
%
\begin{center}
|... -jobname "|\textit{target}|" "|[\textit{flags}]%
|\includeonly{|\textit{dest}|}\input{|\textit{main}|}"|
\end{center}
%

%%%%%%%%%%%%%%%%%%%%%%%%%%%%%%%%%%%%%%%%%%%%%%%%%%%%%%%%%%%%%%%%%%%%%%%%%%%%%%%%
%%%%%%%%%%%%%%%%%%%%%%%%%%%%%%%%%%%%%%%%%%%%%%%%%%%%%%%%%%%%%%%%%%%%%%%%%%%%%%%%
\section{Information}

%%%%%%%%%%%%%%%%%%%%%%%%%%%%%%%%%%%%%%%%%%%%%%%%%%%%%%%%%%%%%%%%%%%%%%%%%%%%%%%%
\subsection{Copyright}

Copyright \copyright{} 2017--2018 Niklas Beisert

This work may be distributed and/or modified under the
conditions of the \LaTeX{} Project Public License, either version 1.3
of this license or (at your option) any later version.
The latest version of this license is in
  \url{http://www.latex-project.org/lppl.txt}
and version 1.3 or later is part of all distributions of \LaTeX{}
version 2005/12/01 or later.

This work has the LPPL maintenance status `maintained'.

The Current Maintainer of this work is Niklas Beisert.

This work consists of the files |README.txt|, |childdoc.ins| and |childdoc.dtx|
as well as the derived files |childdoc.def|, |cdocsamp.tex|
with |cdocsch1.tex|, |cdocsch2.tex|, |cdocspt3.tex|, |cdocspt4.tex|,
|cdocsdrf.tex|, |cdocsfn1.tex|, |cdocsfn2.tex|
as well as |childdoc.pdf|.

%%%%%%%%%%%%%%%%%%%%%%%%%%%%%%%%%%%%%%%%%%%%%%%%%%%%%%%%%%%%%%%%%%%%%%%%%%%%%%%%
\subsection{Files and Installation}

The package consists of the files:
%
\begin{center}
\begin{tabular}{ll}
    |README.txt|   & readme file \\
    |childdoc.ins| & installation file \\
    |childdoc.dtx| & source file \\
    |childdoc.def| & definition file \\
    |cdocsamp.tex| & sample main file \\
    |cdocsch1.tex| & sample include file \\
    |cdocsch2.tex| & sample include file \\
    |cdocspt3.tex| & sample part file \\
    |cdocspt4.tex| & sample part file \\
    |cdocsdrf.tex| & sample redirection file \\
    |cdocsfn1.tex| & sample redirection file \\
    |cdocsfn2.tex| & sample redirection file \\
    |childdoc.pdf| & manual
\end{tabular}
\end{center}
%
The distribution consists of the files
|README.txt|, |childdoc.ins| and |childdoc.dtx|.
%
\begin{itemize}
\item
Run (pdf)\LaTeX{} on |childdoc.dtx|
to compile the manual |childdoc.pdf| (this file).
\item
Run \LaTeX{} on |childdoc.ins| to create the definitions file |childdoc.def|
and the sample |cdocsamp.tex| with include files
|cdocsch1.tex|, |cdocsch2.tex|, |cdocspt3.tex|, |cdocspt4.tex|,
|cdocsdrf.tex|, |cdocsfn1.tex|, |cdocsfn2.tex|.
Then copy the file |childdoc.def| to an appropriate directory of your \LaTeX{}
distribution, e.g.\ \textit{texmf-root}|/tex/latex/childdoc|.
\end{itemize}

%%%%%%%%%%%%%%%%%%%%%%%%%%%%%%%%%%%%%%%%%%%%%%%%%%%%%%%%%%%%%%%%%%%%%%%%%%%%%%%%
\subsection{Related CTAN Packages}

There are several other packages which offer a similar functionality:
%
\begin{itemize}
\item
The packages
\href{http://ctan.org/pkg/docmute}{\textsf{docmute}},
\href{http://ctan.org/pkg/includex}{\textsf{includex}} and
\href{http://ctan.org/pkg/standalone}{\textsf{standalone}}
provide commands to include only the document body of
a child file thus allowing both files to be compiled individually.
\item
The packages \href{http://ctan.org/pkg/subdocs}{\textsf{subdocs}}
and \href{http://ctan.org/pkg/subfiles}{\textsf{subfiles}}
provide structures in which the main and child documents can be
encapsulated and allowing them to be compiled individually.
The inclusion mechanism is different from the conventional |\include|.
\item
The package \href{http://ctan.org/pkg/combine}{\textsf{combine}}
is an elaborate solution to combine several documents into one.
\end{itemize}
%
See also the CTAN topic \href{http://ctan.org/topic/subdocs}{\textsf{subdocs}}
for further related packages.
The present package differs from the above solutions in that
a document structure constructed with the conventional |\include| mechanism
just needs two extra commands at the top of every file
such that all constituent files can be compiled individually.

%%%%%%%%%%%%%%%%%%%%%%%%%%%%%%%%%%%%%%%%%%%%%%%%%%%%%%%%%%%%%%%%%%%%%%%%%%%%%%%%
%\subsection{Feature Suggestions}
%
%The following is a list of features which may be useful for future
%versions of this package:
%%
%\begin{itemize}
%\item
%\ldots
%\end{itemize}

%%%%%%%%%%%%%%%%%%%%%%%%%%%%%%%%%%%%%%%%%%%%%%%%%%%%%%%%%%%%%%%%%%%%%%%%%%%%%%%%
\subsection{Revision History}

%%%%%%%%%%%%%%%%%%%%%%%%%%%%%%%%%%%%%%%%
\paragraph{v2.0:} 2018/12/30

\begin{itemize}
\item
immediate forward processing
\item
added |\childdocby| mechanism
\item
manual restructured
\end{itemize}

%%%%%%%%%%%%%%%%%%%%%%%%%%%%%%%%%%%%%%%%
\paragraph{v1.6:} 2018/01/17

\begin{itemize}
\item
application for development of include files
\item
corrections to manual
\end{itemize}

%%%%%%%%%%%%%%%%%%%%%%%%%%%%%%%%%%%%%%%%
\paragraph{v1.5:} 2017/05/21

\begin{itemize}
\item
more complete structuring introduced
\item
|\childdocof| introduced
\item
|\childdoc| renamed to |\childdocmain|
\item
|\childredirect| renamed to |\childdocforward| and |\childdocforwardprefix|
and functionality expanded
\end{itemize}

%%%%%%%%%%%%%%%%%%%%%%%%%%%%%%%%%%%%%%%%
\paragraph{v1.0:} 2017/04/27

\begin{itemize}
\item
manual and install package
\item
first version published on CTAN
\end{itemize}

%%%%%%%%%%%%%%%%%%%%%%%%%%%%%%%%%%%%%%%%
\paragraph{v0.6:} 2017/04/26

\begin{itemize}
\item
redirection mechanism added
\end{itemize}

%%%%%%%%%%%%%%%%%%%%%%%%%%%%%%%%%%%%%%%%
\paragraph{v0.5:} 2017/04/26

\begin{itemize}
\item
functionality in definition file
\end{itemize}


%%%%%%%%%%%%%%%%%%%%%%%%%%%%%%%%%%%%%%%%%%%%%%%%%%%%%%%%%%%%%%%%%%%%%%%%%%%%%%%%
%%%%%%%%%%%%%%%%%%%%%%%%%%%%%%%%%%%%%%%%%%%%%%%%%%%%%%%%%%%%%%%%%%%%%%%%%%%%%%%%
%%%%%%%%%%%%%%%%%%%%%%%%%%%%%%%%%%%%%%%%%%%%%%%%%%%%%%%%%%%%%%%%%%%%%%%%%%%%%%%%
\appendix

\settowidth\MacroIndent{\rmfamily\scriptsize 000\ }

 \DocInput{childdoc.dtx}

\end{document}
%</driver>
% \fi
%
% %%%%%%%%%%%%%%%%%%%%%%%%%%%%%%%%%%%%%%%%%%%%%%%%%%%%%%%%%%%%%%%%%%%%%%%%%%%%%%
% %%%%%%%%%%%%%%%%%%%%%%%%%%%%%%%%%%%%%%%%%%%%%%%%%%%%%%%%%%%%%%%%%%%%%%%%%%%%%%
% \section{Sample}
%\iffalse
%<*samplemain>
%\fi
%
% The following presents a sample document
% with two chapters, two parts, a title page,
% a compile flag as well as three forwarding files to set the flag.
% It consists of eight |.tex| files:
% \begin{center}
% \begin{tabular}{ll}
% |cdocsamp.tex|&main file\\
% |cdocsch1.tex|&include file for chapter 1\\
% |cdocsch2.tex|&include file for chapter 2\\
% |cdocspt3.tex|&include file for part 3\\
% |cdocspt4.tex|&include file for part 4\\
% |cdocsdrf.tex|&forwarding file for main file in draft mode\\
% |cdocsfi1.tex|&forwarding file for final version of chapter 1\\
% |cdocsfi2.tex|&forwarding file for final version of chapter 2\\
% \end{tabular}
% \end{center}
% Each of the eight files can be compiled directly by the \LaTeX{} compiler.
%
% %%%%%%%%%%%%%%%%%%%%%%%%%%%%%%%%%%%%%%
% \paragraph{Main File.}
%
% The main file is called |cdocsamp.tex|.
%
% Load the \textsf{childdoc} definitions and
% declare the filename for the main document:
%    \begin{macrocode}
\input{childdoc.def}
\childdocmain{}
%    \end{macrocode}

% Optional override for |\version| flag:
%    \begin{macrocode}
%%\ifchilddoc\else\providecommand{\version}{draft}\fi
%    \end{macrocode}

% Define the default values for the |\version| flag
% (|final| for the main file and |draft| for childs):
%    \begin{macrocode}
\ifchilddoc
\providecommand{\version}{draft}
\else
\providecommand{\version}{final}
\fi
%    \end{macrocode}

% Load the standard document class:
%    \begin{macrocode}
\documentclass[12pt]{article}
%    \end{macrocode}

% Start the document body:
%    \begin{macrocode}
\begin{document}
%    \end{macrocode}

% Declare a title page.
% Print title, part of document being processed and version flag:
%    \begin{macrocode}
\addtocounter{page}{-1}
\begin{center}
{\LARGE\bfseries{}childdoc example\par}
\vspace{1cm}
\ifchilddoc
\ifchilddocmanual part\else chapter\fi:
`\childdocname' of `\childdocjob'\par
\else
main document: `\childdocjob'\par
\fi
version: \version\par
\end{center}
\newpage
%    \end{macrocode}

% Manually include selected file,
% otherwise process as usual:
%    \begin{macrocode}
\ifchilddocmanual
\section*{part `\childdocname'}
\input{\childdocname}
\else
%    \end{macrocode}

% Include the two chapters:
%    \begin{macrocode}
\include{cdocsch1}
\include{cdocsch2}
%    \end{macrocode}

% Include the two parts unless only chapters should be displayed:
%    \begin{macrocode}
\ifchilddoc\else
\section{part three}
\input{cdocspt3}
\section{part four}
\input{cdocspt4}
\fi
%    \end{macrocode}

% Process as usual until here:
%    \begin{macrocode}
\fi
%    \end{macrocode}

% End of document body:
%    \begin{macrocode}
\end{document}
%    \end{macrocode}
%\iffalse
%</samplemain>
%\fi
%
% %%%%%%%%%%%%%%%%%%%%%%%%%%%%%%%%%%%%%%
% \paragraph{Chapter Include Files.}
%
% The include files are called |cdocsch1.tex| and |cdocsch2.tex|.
%
%\iffalse
%<*samplechap1|samplechap2>
%\fi

% Optional override for |\version| flag:
%    \begin{macrocode}
%%\providecommand{\version}{final}
%    \end{macrocode}

% Include the main document:
%    \begin{macrocode}
\input{childdoc.def}
\childdocof{cdocsamp}
%    \end{macrocode}

%\iffalse
%</samplechap1|samplechap2>
%\fi
%
%\iffalse
%<*samplechap1>
%\fi
% Some text for chapter 1:
%    \begin{macrocode}
\section{one}
some text in chapter one
%    \end{macrocode}

%\iffalse
%</samplechap1>
%\fi
% Some text for chapter 2:
%\iffalse
%<*samplechap2>
%\fi
%    \begin{macrocode}
\section{two}
more text in chapter two
%    \end{macrocode}

%\iffalse
%</samplechap2>
%\fi
%
% %%%%%%%%%%%%%%%%%%%%%%%%%%%%%%%%%%%%%%
% \paragraph{Part Include Files.}
%
% The include files are called |cdocspt3.tex| and |cdocspt4.tex|.
%
%\iffalse
%<*samplepart3|samplepart4>
%\fi

% Optional override for |\version| flag:
%    \begin{macrocode}
%%\providecommand{\version}{final}
%    \end{macrocode}

% Include the main document:
%    \begin{macrocode}
\input{childdoc.def}
\childdocby{cdocsamp}
%    \end{macrocode}

%\iffalse
%</samplepart3|samplepart4>
%\fi
%
%\iffalse
%<*samplepart3>
%\fi
% Some text for part 3:
%    \begin{macrocode}
some text in part three
%    \end{macrocode}

%\iffalse
%</samplepart3>
%\fi
% Some text for part 4:
%\iffalse
%<*samplepart4>
%\fi
%    \begin{macrocode}
more text in part four
%    \end{macrocode}

%\iffalse
%</samplepart4>
%\fi
%
% %%%%%%%%%%%%%%%%%%%%%%%%%%%%%%%%%%%%%%
% \paragraph{Forwarding for a Complete Draft.}
%
% The following forwarding file |cdocsdrf.tex|
% compiles the main document in draft mode:
%\iffalse
%<*sampledraft>
%\fi
%    \begin{macrocode}
\def\version{draft}
\input{childdoc.def}
\childdocforward{cdocsamp}
%    \end{macrocode}

%\iffalse
%</sampledraft>
%\fi
%
% %%%%%%%%%%%%%%%%%%%%%%%%%%%%%%%%%%%%%%
% \paragraph{Forwarding for Final Version of the Chapters.}
%
% The following forwarding files |cdocsfn1.tex| and |cdocsfn2.tex|
% (with identical content)
% compile the final versions of the child documents
% |cdocsch1.tex| and |cdocsch2.tex|, respectively:
%\iffalse
%<*samplefinal>
%\fi
%    \begin{macrocode}
\def\version{final}
\input{childdoc.def}
\childdocforwardprefix[cdocsamp]{cdocsfn}{cdocsch}
%    \end{macrocode}

%\iffalse
%</samplefinal>
%\fi
%
% %%%%%%%%%%%%%%%%%%%%%%%%%%%%%%%%%%%%%%
% \paragraph{Command Line Processing.}
%
% The following three command lines generate the output files
% |cdocscld|, |cdocscl1| and |cdocscl2|
% which should be identical to
% |cdocsdrf|, |cdocsch1| and |cdocsfn2|, respectively:
% \begin{center}
% \begin{tabular}{l}
% |latex -jobname cdocscld \|\\
% |  "\def\version{draft}\input{childdoc.def}\childdocforward{cdocsamp}"|\\
% |latex -jobname cdocscl1 \|\\
% |  "\input{childdoc.def}\childdocforward[cdocsamp]{cdocsch1}"|\\
% |latex -jobname cdocscl2 \|\\
% |  "\def\version{final}\input{childdoc.def}\childdocforward{cdocsch2}"|
% \end{tabular}
% \end{center}
% Note that the trailing backslash on each first line
% merely continues the input to the second line
% (for convenient cut ant paste).
% Furthermore, the command |latex| can be replaced by any
% of its alternative versions such as |pdflatex|.
%
% %%%%%%%%%%%%%%%%%%%%%%%%%%%%%%%%%%%%%%%%%%%%%%%%%%%%%%%%%%%%%%%%%%%%%%%%%%%%%%
% %%%%%%%%%%%%%%%%%%%%%%%%%%%%%%%%%%%%%%%%%%%%%%%%%%%%%%%%%%%%%%%%%%%%%%%%%%%%%%
% \section{Implementation}
%\iffalse
%<*package>
%\fi
%
% This section describes the definitions file |childdoc.def|.

% The definitions cannot be loaded using |\usepackage| or |\RequirePackage|
% which has a mechanism to prevent loading a style file more than once.
% When loading the definitions by means of |\input|
% multiple instances have to be prevented manually:
%\iffalse
%This code needs to be before the `\ProvidesFile' directive
%which is defined at the beginning of this file.
%Therefore it is also placed there and commented out here.
%</package>
%<*discard>
%\fi
%    \begin{macrocode}
\ifdefined\childdocmain\endinput\fi
%    \end{macrocode}
%\iffalse
%</discard>
%<*package>
%\fi
%
% \macro{\ifchilddoc}
% \macro{\ifchilddocmanual}
% The conditional |\ifchilddoc| tells whether a
% child (true) or main (false) document is being compiled.
% The conditional |\ifchilddocmanual| tells whether
% the |\includeonly| mechanism is used (false) or
% the selection of child files must be performed manually (true).
% The definitions initialise to false:
%    \begin{macrocode}
\newif\ifchilddoc
\newif\ifchilddocmanual
%    \end{macrocode}

% \macro{\childdocname}
% \macro{\childdocjob}
% The macro |\childdocname| stores the name of the main document
% to be compiled. The macro |\childdocjob| stores the name of
% the document on which the \LaTeX{} compiler was originally invoked.
% The content of |\jobname| cannot be compared
% to filenames specified in the source due to different catcodes.
% The following code rescans |\jobname|, stores the result
% in |\childdocname| and saves a copy in |\childdocjob|:
%    \begin{macrocode}
\edef\childdocname{\scantokens\expandafter{\jobname\noexpand}}
\let\childdocjob\childdocname
%    \end{macrocode}

% \macro{\childdocdisable}
% The macro |\childdocdisable| prevents the main file
% from being processed more than once.
% At this stage, the main document command |\childdocmain|
% is assumed to be called once again where it should do nothing.
% Any subsequent call to it should prevent
% a secondary processing of the main document
% It overwrites the forwarding commands
% |\childdocof| and |\childdocforward|
% with empty macros to prevent further inclusions of the main document:
%    \begin{macrocode}
\newcommand{\childdocdisable}
{
  \renewcommand{\childdocmain}[1]{\renewcommand{\childdocmain}[1]{\endinput}}
  \renewcommand{\childdocof}[1]{}
  \renewcommand{\childdocby}[2][]{}
  \renewcommand{\childdocforward}[2][]{}
  \renewcommand{\childdocdisable}{}
}
%    \end{macrocode}

% \macro{\childdocmain}
% The macro |\childdocmain| is to be called at the top of the main file
% with nothing or the main filename (without extension) as argument.
% First, it breaks loops.
% If the argument is not empty and does not match |\childdocname|
% (which is set by the first inclusion of |childdoc.def|),
% |\ifchilddoc| is set to true, |\includeonly| is applied to the child file
% and |\jobname| is set to the main file
% (for proper handling of |.aux| files):
%    \begin{macrocode}
\newcommand{\childdocmain}[1]
{
  \childdocdisable\childdocmain{}
  \if?#1?\else
    \begingroup
      \def\childdoctmp{#1}
      \ifx\childdoctmp\childdocname
        \def\childdoctmp{}
      \else
        \def\childdoctmp
        {
          \childdoctrue
          \includeonly{\childdocname}
          \def\childdocjob{#1}
          \def\jobname{#1}
        }
      \fi
      \expandafter
    \endgroup
    \childdoctmp
  \fi
}
%    \end{macrocode}

% \macro{\childdocof}
% The command |\childdocof| redirects
% compilation to the main file |#1|.
%    \begin{macrocode}
\newcommand{\childdocof}[1]
{
  \childdocdisable
  \childdoctrue
  \includeonly{\childdocname}
  \def\jobname{#1}
  \def\childdocjob{#1}
  \input{#1}
}
%    \end{macrocode}

% \macro{\childdocby}
% The command |\childdocby| ....
%    \begin{macrocode}
\newcommand{\childdocby}[2][]
{
  \childdocdisable
  \childdoctrue
  \childdocmanualtrue
  \if?#1?\else
    \def\jobname{#2}
  \fi
  \def\childdocjob{#2}
  \input{#2}
  \endinput
}
%    \end{macrocode}

% \macro{\childdocforward}
% The command |\childdocforward| redirects
% compilation to the main file or
% (if the optional argument is given) a child file.
% Parameters are set as if the main file
% or a child file starting with |\childdocof| was compiled.
% Then compilation is handed over to the main file:
%    \begin{macrocode}
\newcommand{\childdocforward}[2][]
{
  \begingroup
    \if?#1?
      \def\childdoctmp
      {
        \def\childdocname{#2}
        \def\childdocjob{#2}
        \def\jobname{#2}
        \input{#2}
        \endinput
      }
    \else
      \def\childdoctmp
      {
        \childdocdisable
        \def\childdocname{#2}
        \childdoctrue
        \includeonly{#2}
        \def\childdocjob{#1}
        \def\jobname{#1}
        \input{#1}
        \endinput
      }
    \fi
    \expandafter
  \endgroup
  \childdoctmp
}
%    \end{macrocode}

% \macro{\childdocforwardprefix}
% The command |\childdocforwardprefix| redirects
% compilation to the main or a child file by means of a pattern.
% The prefix |#1| in the current filename is replaced by |#2|
% and the suffix of the current filename is kept
% (it is assumed that the filename does not contain the substring `|~~~|'
% which is used as a delimiter).
% Compilation is handed over to the new file by |\childdocforward|:
%    \begin{macrocode}
\newcommand{\childdocforwardprefix}[3][]
{
  \begingroup
    \def\childdocextract #2##1~~~{\def\childdoctmp{\childdocforward[#1]{#3##1}}}
    \expandafter\childdocextract\childdocname~~~
    \expandafter
  \endgroup
  \childdoctmp
}
%    \end{macrocode}

% \macro{\childdoc}
% The deprecated macro |\childdoc| is a legacy version of |\childdocmain|:
%    \begin{macrocode}
\newcommand{\childdoc}{\childdocmain}
%    \end{macrocode}

% \macro{\childdocredirect}
% The deprecated macro |\childdocredirect| is a legacy version
% of |\childdocforward| and |\childdocforwardprefix|:
%    \begin{macrocode}
\newcommand{\childdocredirect}[2][]
{
  \begingroup
    \if?#1?
      \def\childdoctmp{\childdocforward{#2}}
    \else
      \def\childdoctmp{\childdocforwardprefix{#1}{#2}}
    \fi
    \expandafter
  \endgroup
  \childdoctmp
}
%    \end{macrocode}

%\iffalse
%</package>
%\fi
%
\endinput
|
and perform the replacements as outlined below.
Instead of |\childdocmain{|\textit{main}|}| add the following code
to the top of the main file:
%
\begin{center}
\begin{tabular}{l}
|\||ifdefined\childdocname\endinput\||fi\newif\ifchilddoc|\\
|\edef\childdocname{\scantokens\expandafter{\jobname\noexpand}}|\\
|\def\childdocmain{|\textit{main}|}\||ifx\childdocmain\childdocname\||else|\\
|\childdoctrue\includeonly{\childdocname}\let\jobname\childdocmain\||fi|\\
\end{tabular}
\end{center}
%
Instead of |\childdocof{|\textit{main}|}| just include the main file
at the top of each child file:
%
\begin{center}
|\input{|\textit{main}|}|
\end{center}
%
A simple redirection |\childdocforward{|\textit{dest}|}| is achieved by:
%
\begin{center}
|\def\jobname{|\textit{dest}|}\input{\jobname}|
\end{center}
%
The redirection with prefix
|\childdocforwardprefix[|\textit{prefix}|]{|\textit{dest}|}|
is accomplished by:
%
\begin{center}
\begin{tabular}{l}
|{\edef\jobname{\scantokens\expandafter{\jobname\noexpand}}|\\
|\def\redirectjob |\textit{prefix}|#1~~~{\gdef\jobname{|\textit{dest}|#1}}|\\
|\expandafter\redirectjob\jobname~~~}\input{\jobname}|
\end{tabular}
\end{center}

In an alternative approach,
child documents can be compiled by a specific command line
without additional code or specific definitions:
%
\begin{center}
|... -jobname "|\textit{target}|" "|[\textit{flags}]%
|\includeonly{|\textit{dest}|}\input{|\textit{main}|}"|
\end{center}
%

%%%%%%%%%%%%%%%%%%%%%%%%%%%%%%%%%%%%%%%%%%%%%%%%%%%%%%%%%%%%%%%%%%%%%%%%%%%%%%%%
%%%%%%%%%%%%%%%%%%%%%%%%%%%%%%%%%%%%%%%%%%%%%%%%%%%%%%%%%%%%%%%%%%%%%%%%%%%%%%%%
\section{Information}

%%%%%%%%%%%%%%%%%%%%%%%%%%%%%%%%%%%%%%%%%%%%%%%%%%%%%%%%%%%%%%%%%%%%%%%%%%%%%%%%
\subsection{Copyright}

Copyright \copyright{} 2017--2018 Niklas Beisert

This work may be distributed and/or modified under the
conditions of the \LaTeX{} Project Public License, either version 1.3
of this license or (at your option) any later version.
The latest version of this license is in
  \url{http://www.latex-project.org/lppl.txt}
and version 1.3 or later is part of all distributions of \LaTeX{}
version 2005/12/01 or later.

This work has the LPPL maintenance status `maintained'.

The Current Maintainer of this work is Niklas Beisert.

This work consists of the files |README.txt|, |childdoc.ins| and |childdoc.dtx|
as well as the derived files |childdoc.def|, |cdocsamp.tex|
with |cdocsch1.tex|, |cdocsch2.tex|, |cdocspt3.tex|, |cdocspt4.tex|,
|cdocsdrf.tex|, |cdocsfn1.tex|, |cdocsfn2.tex|
as well as |childdoc.pdf|.

%%%%%%%%%%%%%%%%%%%%%%%%%%%%%%%%%%%%%%%%%%%%%%%%%%%%%%%%%%%%%%%%%%%%%%%%%%%%%%%%
\subsection{Files and Installation}

The package consists of the files:
%
\begin{center}
\begin{tabular}{ll}
    |README.txt|   & readme file \\
    |childdoc.ins| & installation file \\
    |childdoc.dtx| & source file \\
    |childdoc.def| & definition file \\
    |cdocsamp.tex| & sample main file \\
    |cdocsch1.tex| & sample include file \\
    |cdocsch2.tex| & sample include file \\
    |cdocspt3.tex| & sample part file \\
    |cdocspt4.tex| & sample part file \\
    |cdocsdrf.tex| & sample redirection file \\
    |cdocsfn1.tex| & sample redirection file \\
    |cdocsfn2.tex| & sample redirection file \\
    |childdoc.pdf| & manual
\end{tabular}
\end{center}
%
The distribution consists of the files
|README.txt|, |childdoc.ins| and |childdoc.dtx|.
%
\begin{itemize}
\item
Run (pdf)\LaTeX{} on |childdoc.dtx|
to compile the manual |childdoc.pdf| (this file).
\item
Run \LaTeX{} on |childdoc.ins| to create the definitions file |childdoc.def|
and the sample |cdocsamp.tex| with include files
|cdocsch1.tex|, |cdocsch2.tex|, |cdocspt3.tex|, |cdocspt4.tex|,
|cdocsdrf.tex|, |cdocsfn1.tex|, |cdocsfn2.tex|.
Then copy the file |childdoc.def| to an appropriate directory of your \LaTeX{}
distribution, e.g.\ \textit{texmf-root}|/tex/latex/childdoc|.
\end{itemize}

%%%%%%%%%%%%%%%%%%%%%%%%%%%%%%%%%%%%%%%%%%%%%%%%%%%%%%%%%%%%%%%%%%%%%%%%%%%%%%%%
\subsection{Related CTAN Packages}

There are several other packages which offer a similar functionality:
%
\begin{itemize}
\item
The packages
\href{http://ctan.org/pkg/docmute}{\textsf{docmute}},
\href{http://ctan.org/pkg/includex}{\textsf{includex}} and
\href{http://ctan.org/pkg/standalone}{\textsf{standalone}}
provide commands to include only the document body of
a child file thus allowing both files to be compiled individually.
\item
The packages \href{http://ctan.org/pkg/subdocs}{\textsf{subdocs}}
and \href{http://ctan.org/pkg/subfiles}{\textsf{subfiles}}
provide structures in which the main and child documents can be
encapsulated and allowing them to be compiled individually.
The inclusion mechanism is different from the conventional |\include|.
\item
The package \href{http://ctan.org/pkg/combine}{\textsf{combine}}
is an elaborate solution to combine several documents into one.
\end{itemize}
%
See also the CTAN topic \href{http://ctan.org/topic/subdocs}{\textsf{subdocs}}
for further related packages.
The present package differs from the above solutions in that
a document structure constructed with the conventional |\include| mechanism
just needs two extra commands at the top of every file
such that all constituent files can be compiled individually.

%%%%%%%%%%%%%%%%%%%%%%%%%%%%%%%%%%%%%%%%%%%%%%%%%%%%%%%%%%%%%%%%%%%%%%%%%%%%%%%%
%\subsection{Feature Suggestions}
%
%The following is a list of features which may be useful for future
%versions of this package:
%%
%\begin{itemize}
%\item
%\ldots
%\end{itemize}

%%%%%%%%%%%%%%%%%%%%%%%%%%%%%%%%%%%%%%%%%%%%%%%%%%%%%%%%%%%%%%%%%%%%%%%%%%%%%%%%
\subsection{Revision History}

%%%%%%%%%%%%%%%%%%%%%%%%%%%%%%%%%%%%%%%%
\paragraph{v2.0:} 2018/12/30

\begin{itemize}
\item
immediate forward processing
\item
added |\childdocby| mechanism
\item
manual restructured
\end{itemize}

%%%%%%%%%%%%%%%%%%%%%%%%%%%%%%%%%%%%%%%%
\paragraph{v1.6:} 2018/01/17

\begin{itemize}
\item
application for development of include files
\item
corrections to manual
\end{itemize}

%%%%%%%%%%%%%%%%%%%%%%%%%%%%%%%%%%%%%%%%
\paragraph{v1.5:} 2017/05/21

\begin{itemize}
\item
more complete structuring introduced
\item
|\childdocof| introduced
\item
|\childdoc| renamed to |\childdocmain|
\item
|\childredirect| renamed to |\childdocforward| and |\childdocforwardprefix|
and functionality expanded
\end{itemize}

%%%%%%%%%%%%%%%%%%%%%%%%%%%%%%%%%%%%%%%%
\paragraph{v1.0:} 2017/04/27

\begin{itemize}
\item
manual and install package
\item
first version published on CTAN
\end{itemize}

%%%%%%%%%%%%%%%%%%%%%%%%%%%%%%%%%%%%%%%%
\paragraph{v0.6:} 2017/04/26

\begin{itemize}
\item
redirection mechanism added
\end{itemize}

%%%%%%%%%%%%%%%%%%%%%%%%%%%%%%%%%%%%%%%%
\paragraph{v0.5:} 2017/04/26

\begin{itemize}
\item
functionality in definition file
\end{itemize}


%%%%%%%%%%%%%%%%%%%%%%%%%%%%%%%%%%%%%%%%%%%%%%%%%%%%%%%%%%%%%%%%%%%%%%%%%%%%%%%%
%%%%%%%%%%%%%%%%%%%%%%%%%%%%%%%%%%%%%%%%%%%%%%%%%%%%%%%%%%%%%%%%%%%%%%%%%%%%%%%%
%%%%%%%%%%%%%%%%%%%%%%%%%%%%%%%%%%%%%%%%%%%%%%%%%%%%%%%%%%%%%%%%%%%%%%%%%%%%%%%%
\appendix

\settowidth\MacroIndent{\rmfamily\scriptsize 000\ }

 \DocInput{childdoc.dtx}

\end{document}
%</driver>
% \fi
%
% %%%%%%%%%%%%%%%%%%%%%%%%%%%%%%%%%%%%%%%%%%%%%%%%%%%%%%%%%%%%%%%%%%%%%%%%%%%%%%
% %%%%%%%%%%%%%%%%%%%%%%%%%%%%%%%%%%%%%%%%%%%%%%%%%%%%%%%%%%%%%%%%%%%%%%%%%%%%%%
% \section{Sample}
%\iffalse
%<*samplemain>
%\fi
%
% The following presents a sample document
% with two chapters, two parts, a title page,
% a compile flag as well as three forwarding files to set the flag.
% It consists of eight |.tex| files:
% \begin{center}
% \begin{tabular}{ll}
% |cdocsamp.tex|&main file\\
% |cdocsch1.tex|&include file for chapter 1\\
% |cdocsch2.tex|&include file for chapter 2\\
% |cdocspt3.tex|&include file for part 3\\
% |cdocspt4.tex|&include file for part 4\\
% |cdocsdrf.tex|&forwarding file for main file in draft mode\\
% |cdocsfi1.tex|&forwarding file for final version of chapter 1\\
% |cdocsfi2.tex|&forwarding file for final version of chapter 2\\
% \end{tabular}
% \end{center}
% Each of the eight files can be compiled directly by the \LaTeX{} compiler.
%
% %%%%%%%%%%%%%%%%%%%%%%%%%%%%%%%%%%%%%%
% \paragraph{Main File.}
%
% The main file is called |cdocsamp.tex|.
%
% Load the \textsf{childdoc} definitions and
% declare the filename for the main document:
%    \begin{macrocode}
% \iffalse
%
% childdoc.dtx Copyright (C) 2017-2018 Niklas Beisert
%
% This work may be distributed and/or modified under the
% conditions of the LaTeX Project Public License, either version 1.3
% of this license or (at your option) any later version.
% The latest version of this license is in
%   http://www.latex-project.org/lppl.txt
% and version 1.3 or later is part of all distributions of LaTeX
% version 2005/12/01 or later.
%
% This work has the LPPL maintenance status `maintained'.
%
% The Current Maintainer of this work is Niklas Beisert.
%
% This work consists of the files childdoc.dtx and childdoc.ins
% and the derived files childdoc.def and cdocsamp.tex with
% cdocsch1.tex, cdocsch2.tex, cdocsdrf.tex, cdocsfn1.tex, cdocsfn2.tex.
%
%<package>\ifdefined\childdocmain\endinput\fi
%<package>\ProvidesFile{childdoc.def}[2018/12/30 v2.0 child document driver]
%<samplemain>\ProvidesFile{cdocsamp.tex}[2018/12/30 v2.0 sample for childdoc]
%<*driver>
%\ProvidesFile{childdoc.drv}[2018/12/30 v2.0 childdoc reference manual file]
\PassOptionsToClass{10pt,a4paper}{article}
\documentclass{ltxdoc}

\usepackage[margin=35mm]{geometry}
\usepackage{hyperref}
\usepackage{hyperxmp}
\usepackage[usenames]{color}

\hypersetup{colorlinks=true}
\hypersetup{pdfstartview=FitH}
\hypersetup{pdfpagemode=UseNone}
\hypersetup{pdfsource={}}
\hypersetup{pdflang={en-UK}}
\hypersetup{pdfcopyright={Copyright 2017-2018 Niklas Beisert.
  This work may be distributed and/or modified under the
  conditions of the LaTeX Project Public License, either version 1.3
  of this license or (at your option) any later version.}}
\hypersetup{pdflicenseurl={http://www.latex-project.org/lppl.txt}}
\hypersetup{pdfcontactaddress={ETH Zurich, ITP, HIT K,
  Wolfgang-Pauli-Strasse 27}}
\hypersetup{pdfcontactpostcode={8093}}
\hypersetup{pdfcontactcity={Zurich}}
\hypersetup{pdfcontactcountry={Switzerland}}
\hypersetup{pdfcontactemail={nbeisert@itp.phys.ethz.ch}}
\hypersetup{pdfcontacturl={http://people.phys.ethz.ch/\xmptilde nbeisert/}}

\newcommand{\secref}[1]{\hyperref[#1]{section \ref*{#1}}}

\parskip1ex
\parindent0pt
\let\olditemize\itemize
\def\itemize{\olditemize\parskip0pt}

\begin{document}

\title{The \textsf{childdoc} Package}
\hypersetup{pdftitle={The childdoc Package}}
\author{Niklas Beisert\\[2ex]
  Institut f\"ur Theoretische Physik\\
  Eidgen\"ossische Technische Hochschule Z\"urich\\
  Wolfgang-Pauli-Strasse 27, 8093 Z\"urich, Switzerland\\[1ex]
  \href{mailto:nbeisert@itp.phys.ethz.ch}
  {\texttt{nbeisert@itp.phys.ethz.ch}}}
\hypersetup{pdfauthor={Niklas Beisert}}
\hypersetup{pdfsubject={Manual for the LaTeX2e Package childdoc}}
\date{30 December 2018, \textsf{v2.0}}
\maketitle

\begin{abstract}\noindent
\textsf{childdoc} is a \LaTeXe{} package
that enables the direct compilation
of document sections included by |\include|
to individual files.
\end{abstract}

\begingroup
\parskip0ex
\tableofcontents
\endgroup

%%%%%%%%%%%%%%%%%%%%%%%%%%%%%%%%%%%%%%%%%%%%%%%%%%%%%%%%%%%%%%%%%%%%%%%%%%%%%%%%
%%%%%%%%%%%%%%%%%%%%%%%%%%%%%%%%%%%%%%%%%%%%%%%%%%%%%%%%%%%%%%%%%%%%%%%%%%%%%%%%
\section{Introduction}

\LaTeX{} provides a mechanism to structure a large document (such as a book)
into a main file and several child files (containing the chapters)
using the |\include| command.
This mechanism is beneficial for documents
which span hundreds of pages in order to
make the source file(s) more manageable.
Moreover, compilation can be restricted to
selected child files by means of the |\includeonly| command.
The latter feature can be used to reduce the compilation time while editing
(this was significantly more useful in the earlier days of \LaTeX{})
or to generate a smaller document which is easier to navigate.
Another application of |\includeonly| is to generate
documents consisting of selected parts of the complete document.

However, there are a few drawbacks of the plain |\include| mechanism:
\begin{itemize}
\item
The child files cannot be compiled on their own,
they can only be compiled via the main file.
A naive editing environment
(such as a text editor with an option
to have the current file processed by \LaTeX)
may require one to switch to the main file before compiling;
attempting to compile the child file produces errors.
\item
The main file must be modified (each time)
to adjust the |\includeonly| command
to the present needs. This easily leaves the main file in a messy state.
\item
The generated document will always carry the filename
of the main document. This is inconvenient if
several child files are to be compiled and
to be kept for distribution.
\end{itemize}

The present package provides a simple interface
to make child files individually compilable by \LaTeX{}.
Compiling a child file then has the same effect as compiling
the main file with an |\includeonly| command
to select the appropriate child.
Moreover the generated document will carry the name of the child
rather than the main file.
This resolves all three above issues.

This feature is meant to make the editing of books,
thesis documents and lecture notes somewhat more convenient.
However, the package can also be used efficiently for
composing a series of documents (such as exercise sheets)
which are typically distributed individually.
It then assists the author in generating the individual documents
(potentially in different versions)
as well as a document containing the collected series.
Another application is in developing style files
or other kinds of included material
where compilation of the style file could redirect
to a sample or test file.

%%%%%%%%%%%%%%%%%%%%%%%%%%%%%%%%%%%%%%%%%%%%%%%%%%%%%%%%%%%%%%%%%%%%%%%%%%%%%%%%
%%%%%%%%%%%%%%%%%%%%%%%%%%%%%%%%%%%%%%%%%%%%%%%%%%%%%%%%%%%%%%%%%%%%%%%%%%%%%%%%
\section{Usage}

First of all, the package \textsf{childdoc} is \emph{not} a standard
\LaTeXe{} |.sty| style file! Therefore it needs to be invoked in
a non-standard way.

%%%%%%%%%%%%%%%%%%%%%%%%%%%%%%%%%%%%%%%%%%%%%%%%%%%%%%%%%%%%%%%%%%%%%%%%%%%%%%%%
\subsection{Included Files}
\label{sec:include}

%%%%%%%%%%%%%%%%%%%%%%%%%%%%%%%%%%%%%%%%
\DescribeMacro{\childdocmain}
To use the package, add the commands
\begin{center}
\begin{tabular}{l}
|\input{childdoc.def}|\\
|\childdocmain{}|\\
\end{tabular}
\end{center}
at the very top of the main \LaTeX{} file,
in particular \emph{before} the |\documentclass| statement!
The argument of |\childdocmain| should be left empty
(but it must be present).

%%%%%%%%%%%%%%%%%%%%%%%%%%%%%%%%%%%%%%%%
\DescribeMacro{\childdocof}
Furthermore, add the commands
\begin{center}
\begin{tabular}{l}
|\input{childdoc.def}|\\
|\childdocof{|\textit{main}|}|\\
\end{tabular}
\end{center}
at the top of every child file \textit{child}
which is included by |\include{|\textit{child}|}|
from within the main file
(or at least for those files to be compiled individually).
The argument \textit{main} must be the filename of the main file.

There are a couple of
considerations in setting up the main and child documents:

%%%%%%%%%%%%%%%%%%%%%%%%%%%%%%%%%%%%%%%%
\paragraph{Restrictions.}

Please note the following restrictions:
\begin{itemize}
\item
|\childdocmain| must be called with one argument \textit{main}
to ensure compatibility with earlier version of the package.
It must either be empty (|\childdocmain{}|)
or precisely match the filename of the main file in which it is specified.
See \secref{sec:detection} for further information.
\item
The filename \textit{main} must be specified without the |.tex| extension.
\item
The filename \textit{main} is case sensitive
(even in case-insensitive file systems)
due to internal string comparison.
\item
The argument \textit{main} should be fully expanded, it cannot be a macro.
\item
Subdirectories and special characters should be avoided in filenames.
\item
The command |\childdocmain{|\textit{main}|}| must be followed by a whitespace.
It should not be followed immediately by another command
or by a comment mark `|%|'.
This is because the \TeX{} parser reads the token immediately following
the argument of |\childdocmain| and puts it
at the beginning of every child section;
however, a white\-space is ignored.
\end{itemize}

%%%%%%%%%%%%%%%%%%%%%%%%%%%%%%%%%%%%%%%%
\paragraph{Content of Main File.}

It is advisable to place all content in the child files included by |\include|.
Any output contained in the main file will appear in all child documents
unless suppressed manually;
it cannot be suppressed automatically by the |\includeonly| directive
and thus should normally be avoided.
A method to include some content in the main file
by means of conditional processing is described in \secref{sec:conditional}.

%%%%%%%%%%%%%%%%%%%%%%%%%%%%%%%%%%%%%%%%
\paragraph{Page Numbering.}

When only a part of the document is compiled,
the appropriate numbering of pages
(as well as other status parameters)
is determined from the |.aux| files.
The latter contain information from previous passes.
However this information needs to propagate through
all intermediate child documents.
Therefore the page numbering in child documents may well
be inconsistent until the complete document is compiled at least once.

A useful (if unconventional) way to always ensure a consistent
page numbering is to restart the numbering in each child document
and denote the pages by `\textit{child}|.|\textit{page}'
where \textit{child} represents the chapter/section number of the child file.
This can be achieved by the command
|\numberwithin{page}{|\textit{child}|}|
of the \textsf{amsmath} package
where \textit{child} can be |chapter| or |section|
depending on the chosen structuring.
Alternatively, one can modify the macro |\thepage| appropriately
and reset the counter |page| at the start of each child file.

%%%%%%%%%%%%%%%%%%%%%%%%%%%%%%%%%%%%%%%%%%%%%%%%%%%%%%%%%%%%%%%%%%%%%%%%%%%%%%%%
\subsection{Conditional Processing}
\label{sec:conditional}

The package provides a mechanism to compile different versions
of a document. To customise the versions further some conditional processing
can come in handy to distinguish which version is being compiled.
The package provides two macros to describe the compilation context:

%%%%%%%%%%%%%%%%%%%%%%%%%%%%%%%%%%%%%%%%
\DescribeMacro{\ifchilddoc}
The conditional |\ifchilddoc| distinguishes between the compilation of
child documents and the main document:
%
\begin{center}
|\ifchilddoc |\textit{child-code}| |[|\||else |\textit{main-code}]| \||fi|
\end{center}

%%%%%%%%%%%%%%%%%%%%%%%%%%%%%%%%%%%%%%%%
\DescribeMacro{\childdocname}
\DescribeMacro{\childdocjob}
The macro |\childdocname| contains the filename (without extension)
of the main or child file being processed.
Note that |\childdocjob| will always contain the name of the main file.

%%%%%%%%%%%%%%%%%%%%%%%%%%%%%%%%%%%%%%%%
\paragraph{Title Page.}

Conditional processing can be used to include a title or banner page
in the main document when proper precautions are taken.
Importantly, the code in the main file should ensure that the page counter
(as well as other status parameters which are stored in the |.aux| files)
takes the same value after the conditional processing.
Otherwise the page numbers may take divergent values
depending on which part is compiled.

For example, a title page could be declared by:
%
\begin{center}
\begin{tabular}{l}
|\ifchilddoc\||else|\\
|\addtocounter{page}{-1}|\\
\textit{code for title page}\\
|\newpage|\\
|\||fi|
\end{tabular}
\end{center}
%
A banner page for the child documents can be generated by:
%
\begin{center}
\begin{tabular}{l}
|\ifchilddoc|\\
|\addtocounter{page}{-1}|\\
\textit{code for banner page}\\
|\newpage|\\
|\||fi|
\end{tabular}
\end{center}
%
Here one could write a message such as:
\begin{center}
|This is the part \childdocname{} of \childdocjob{}.|
\end{center}

%%%%%%%%%%%%%%%%%%%%%%%%%%%%%%%%%%%%%%%%%%%%%%%%%%%%%%%%%%%%%%%%%%%%%%%%%%%%%%%%
\subsection{Flags}
\label{sec:flags}

The package makes it easy to generate different versions
of the main or child documents.
To this end compilation flags can be defined
and assigned different default values.
They will be particularly useful in conjunction
with the forwarding mechanism described in \secref{sec:forward}.

For example, it may be useful to have a flag |\version|
which can be set to |draft| or |final|.
The document source will contain some conditional code
depending on the value of |\version|.
Suppose further, the flag should default to |final| for the main file
and to |draft| for child files
which is a natural assignment for editing the document.
This is achieved by placing the following code
in the preamble of the main document
(below the |\childdocmain| directive):
%
\begin{center}
\begin{tabular}{l}
|\ifchilddoc|\\
|\providecommand{\version}{draft}|\\
|\||else|\\
|\providecommand{\version}{final}|\\
|\||fi|
\end{tabular}
\end{center}
%
The definition by |\providecommand| makes sure
that previous definitions are not overwritten.
Further statements |\providecommand{\version}{...}|
can thus be added before the above code to override it.

For the main file, one might add a line
(between |\childdocmain| and the above block)
%
\begin{center}
|%\ifchilddoc\||else\providecommand{\version}{draft}\||fi|
\end{center}
%
which can be uncommented to produce a draft version.
Likewise one can add a line to the very top of a child file
(above the |\childdocof{|\textit{main}|}| directive)
%
\begin{center}
|%\providecommand{\version}{final}|
\end{center}
%
which can be uncommented to produce the final version of this child document.

%%%%%%%%%%%%%%%%%%%%%%%%%%%%%%%%%%%%%%%%%%%%%%%%%%%%%%%%%%%%%%%%%%%%%%%%%%%%%%%%
\subsection{Forwarding}
\label{sec:forward}

Different versions of the main or child documents
using compilation flags as described in \secref{sec:flags}
can be (permanently) stored in different files
for convenient compilation, viewing and distribution.
To this end, the package defines a command
to pass on compilation to a different file:

%%%%%%%%%%%%%%%%%%%%%%%%%%%%%%%%%%%%%%%%
\DescribeMacro{\childdocforward}
The command |\childdocforward| redirects processing to
another source file:
%
\begin{center}
\begin{tabular}{l}
|\input{childdoc.def}|\\
|\childdocforward[|\textit{main}|]{|\textit{dest}|}|\\
\end{tabular}
\end{center}
%
The argument \textit{dest} is the destination file
(without extension).
It should be the main file or one of the child files.
Note that further \textsf{childdoc} directives
such as |\childdocof| and |\childdocforward|
in the indicated file will be processed in this form.
The optional argument \textit{main}
passes on directly to the main file \textit{main}
while pretending to compile the child \textit{dest}.
This form behaves as if \textit{dest}
issues |\childdocof{|\textit{main}|}| right away,
and no further \textsf{childdoc} directives will be processed.

%%%%%%%%%%%%%%%%%%%%%%%%%%%%%%%%%%%%%%%%
\DescribeMacro{\...prefix}
In the alternative form |\childdocforwardprefix|,
%
\begin{center}
\begin{tabular}{l}
|\input{childdoc.def}|\\
|\childdocforwardprefix[|\textit{main}|]{|\textit{prefix}|}{|\textit{dest}|}|
\end{tabular}
\end{center}
%
the destination file is determined by a pattern
depending on the current file:
To make this work, the current file must be called
`{\textit{prefix}\hspace{0.2em}\textit{suffix}}'
with \textit{prefix} matching precisely the argument.
Processing is then passed on to the file
`{\textit{dest}\hspace{0.2em}\textit{suffix}}'.
Surely, the same effect is achieved by
directly specifying the
argument `{\textit{dest}\hspace{0.2em}\textit{suffix}}'
in the first form.
However, that requires to set up a different file
for each child. With the alternative form of the command
all these files can have exactly the same content
which simplifies setting them up and maintaining them.

For example, the following file |draft.tex|
with a compilation flag |\version| as described in \secref{sec:flags}
compiles the main document as a draft:
%
\begin{center}
\begin{tabular}{l}
|\def\version{draft}|\\
|\input{childdoc.def}|\\
|\childdocforward{|\textit{main}|}|
\end{tabular}
\end{center}
%
Likewise, the following files |final|\textit{nn}|.tex|
compile the final version of the child document
|child|\textit{nn}|.tex|:
%
\begin{center}
\begin{tabular}{l}
|\def\version{final}|\\
|\input{childdoc.def}|\\
|\childdocforwardprefix{final}{child}|
\end{tabular}
\end{center}
%

Note that when several versions of a main file and/or of each child file
are to be generated, it may be convenient to set up a |Makefile| or
shell script to automatise the process.

%%%%%%%%%%%%%%%%%%%%%%%%%%%%%%%%%%%%%%%%%%%%%%%%%%%%%%%%%%%%%%%%%%%%%%%%%%%%%%%%
\subsection{Command Line Processing}
\label{sec:commandline}

The effect of redirection files can also be achieved by invoking
the \LaTeX{} compiler with a more elaborate command line.
Most conveniently this should be done as part
of a shell script or a |Makefile|.

When using \textsf{childdoc} in the main file, the following
command lines effectively perform a redirection
(note that depending on the shell being used,
backslashes may have to be doubled: `|\|' $\to$ `|\\|'):
%
\begin{center}
|... -jobname "|\textit{target}|" |\\|"|[\textit{flags}]%
|\input{childdoc.def}\childdocforward[|\textit{main}|]{|\textit{dest}|}"|
\end{center}
%
Here \textit{target} is the name of the output file,
\textit{main} is the name of the main file
and \textit{dest} is the name of the main or child file to be processed
(all filenames without extensions).
The optional argument \textit{main} can be omitted
if \textit{main} matches \textit{dest}.
Optionally, compilation \textit{flags} can be defined via |\def| commands.
This command line makes the \TeX{} engine believe
it is compiling the file \textit{target}
whose content is specified as the latter parameter.
The provided code then forwards the processing to
\textit{main} or \textit{dest} as described in \secref{sec:forward}.

%%%%%%%%%%%%%%%%%%%%%%%%%%%%%%%%%%%%%%%%%%%%%%%%%%%%%%%%%%%%%%%%%%%%%%%%%%%%%%%%
\subsection{Include by Input}
\label{sec:input}

Including child documents by |\include| has some restrictions by design.
Most notably, the content of a child document always occupies
its own set of pages; pages cannot be shared between child documents.
Usually, this behaviour makes perfect sense
because each child document contain an essential part of the document.
However, in some situations it may be desirable to compose
a document from a collection of parts
without having mandatory page breaks between then.
For this case, the package
provides a mechanism to include parts
by |\input| which can also be processed individually.
However, by construction this mechanism
requires manual handling of the content to be output.

%%%%%%%%%%%%%%%%%%%%%%%%%%%%%%%%%%%%%%%%
\DescribeMacro{\ifchilddocmanual}
The main file should be prepared as usual, see \secref{sec:include}.
However, the document body must make a distinction
between processing of an individual part and of the main document, e.g.:
%
\begin{center}
\begin{tabular}{l}
|\ifchilddocmanual|\\
|\input{\childdocname}|\\
|\||else|\\
\textit{document body with }|\input{|\textit{part}|}|\\
|\||fi|
\end{tabular}
\end{center}
%
The conditional |\ifchilddocmanual| is true whenever
a part to be included by |\input| is being compiled,
and the name of the part is stored in |\childdocname|.

%%%%%%%%%%%%%%%%%%%%%%%%%%%%%%%%%%%%%%%%
\DescribeMacro{\childdocby}
Each part to be included by |\input| should start with:
%
\begin{center}
\begin{tabular}{l}
|\input{childdoc.def}|\\
|\childdocby{|\textit{main}|}|\\
\end{tabular}
\end{center}
%
The directive |\childdocby| is similar to |\childdocof|
described in \secref{sec:include},
but the subsequent selection of content must be done manually.
To that end, both |\ifchilddoc| and |\ifchilddocmanual|
will be true upon processing of a part,
and the name of the part is stored in |\childdocname|.
Note that |\jobname| will be set to the filename of the current part
so that each part receives an individual |.aux| file
that does not interfere with the |.aux| file(s) of the main document.
This behaviour can be altered by the alternative form
|\childdocby[*]{|\textit{main}|}| (with a non-empty optional argument)
which uses the |.aux| file of the main document
by setting |\jobname| to \textit{main}.

%%%%%%%%%%%%%%%%%%%%%%%%%%%%%%%%%%%%%%%%%%%%%%%%%%%%%%%%%%%%%%%%%%%%%%%%%%%%%%%%
\subsection{Driver Development}
\label{sec:driver}

The \textsf{childdoc} mechanism can also be use for the development
of definition files such as \LaTeX{} styles or classes.
This case differs from the above setup with multiple parts
included by |\include| in that no |\includeonly| should be invoked.
This can be achieved by starting the include file
(before |\ProvidesPackage|) with:
%
\begin{center}
\begin{tabular}{l}
|\input{childdoc.def}|\\
|\childdocforward{|\textit{main}|}|\\
\end{tabular}
\end{center}
%
or alternatively with:
%
\begin{center}
\begin{tabular}{l}
|\input{childdoc.def}|\\
|\childdocby{|\textit{main}|}|\\
\end{tabular}
\end{center}
%
Both forms have slightly different effects as described above.
The main file is prepared as usual, see \secref{sec:include}.

%%%%%%%%%%%%%%%%%%%%%%%%%%%%%%%%%%%%%%%%%%%%%%%%%%%%%%%%%%%%%%%%%%%%%%%%%%%%%%%%
\subsection{Legacy Detection}
\label{sec:detection}

The directive |\childdocmain| in the main file can detect
whether the complete document or merely a child is to be compiled
even without using the directive |\childdocof|.
This method is deprecated because it is less robust
and there is no compelling reason to use it;
it is merely provided for backward compatibility
and it may be removed in future versions.

If the detection mechanism is to be used,
it is mandatory to correctly specify
the filename of the main file as the argument of |\childdocmain|:
%
\begin{center}
\begin{tabular}{l}
|\input{childdoc.def}|\\
|\childdocmain{|\textit{main}|}|\\
\end{tabular}
\end{center}
%
If |\jobname| does not match the argument \textit{main} of |\childdocmain|,
it is assumed that |\jobname| points to the child file to be compiled.
When using |\childdocmain| with the main file specified as argument,
it suffices to start a child file
with just |\input{|\textit{main}|}|
without loading of the package and using |\childdocof|.
If instead all processing is done
with the appropriate \textsf{childdoc} directives,
the argument of \textit{main} of |\childdocmain| can be empty.

An alternative version of the command line processing described
in \secref{sec:commandline} using the detection mechanism reads:
%
\begin{center}
|... -jobname "|\textit{target}|" "|[\textit{flags}]%
[|\def\jobname{|\textit{dest}|}|]|\input{|\textit{main}|}"|
\end{center}

%%%%%%%%%%%%%%%%%%%%%%%%%%%%%%%%%%%%%%%%%%%%%%%%%%%%%%%%%%%%%%%%%%%%%%%%%%%%%%%%
\subsection{Manual Code}
\label{sec:manual}

In case one cannot be certain whether the definitions file |childdoc.def|
is installed on the target \TeX{} distribution
and one prefers not to ship it,
it is conceivable to paste a few relevant commands into the sources.

To that end, drop all statements |\input{childdoc.def}|
and perform the replacements as outlined below.
Instead of |\childdocmain{|\textit{main}|}| add the following code
to the top of the main file:
%
\begin{center}
\begin{tabular}{l}
|\||ifdefined\childdocname\endinput\||fi\newif\ifchilddoc|\\
|\edef\childdocname{\scantokens\expandafter{\jobname\noexpand}}|\\
|\def\childdocmain{|\textit{main}|}\||ifx\childdocmain\childdocname\||else|\\
|\childdoctrue\includeonly{\childdocname}\let\jobname\childdocmain\||fi|\\
\end{tabular}
\end{center}
%
Instead of |\childdocof{|\textit{main}|}| just include the main file
at the top of each child file:
%
\begin{center}
|\input{|\textit{main}|}|
\end{center}
%
A simple redirection |\childdocforward{|\textit{dest}|}| is achieved by:
%
\begin{center}
|\def\jobname{|\textit{dest}|}\input{\jobname}|
\end{center}
%
The redirection with prefix
|\childdocforwardprefix[|\textit{prefix}|]{|\textit{dest}|}|
is accomplished by:
%
\begin{center}
\begin{tabular}{l}
|{\edef\jobname{\scantokens\expandafter{\jobname\noexpand}}|\\
|\def\redirectjob |\textit{prefix}|#1~~~{\gdef\jobname{|\textit{dest}|#1}}|\\
|\expandafter\redirectjob\jobname~~~}\input{\jobname}|
\end{tabular}
\end{center}

In an alternative approach,
child documents can be compiled by a specific command line
without additional code or specific definitions:
%
\begin{center}
|... -jobname "|\textit{target}|" "|[\textit{flags}]%
|\includeonly{|\textit{dest}|}\input{|\textit{main}|}"|
\end{center}
%

%%%%%%%%%%%%%%%%%%%%%%%%%%%%%%%%%%%%%%%%%%%%%%%%%%%%%%%%%%%%%%%%%%%%%%%%%%%%%%%%
%%%%%%%%%%%%%%%%%%%%%%%%%%%%%%%%%%%%%%%%%%%%%%%%%%%%%%%%%%%%%%%%%%%%%%%%%%%%%%%%
\section{Information}

%%%%%%%%%%%%%%%%%%%%%%%%%%%%%%%%%%%%%%%%%%%%%%%%%%%%%%%%%%%%%%%%%%%%%%%%%%%%%%%%
\subsection{Copyright}

Copyright \copyright{} 2017--2018 Niklas Beisert

This work may be distributed and/or modified under the
conditions of the \LaTeX{} Project Public License, either version 1.3
of this license or (at your option) any later version.
The latest version of this license is in
  \url{http://www.latex-project.org/lppl.txt}
and version 1.3 or later is part of all distributions of \LaTeX{}
version 2005/12/01 or later.

This work has the LPPL maintenance status `maintained'.

The Current Maintainer of this work is Niklas Beisert.

This work consists of the files |README.txt|, |childdoc.ins| and |childdoc.dtx|
as well as the derived files |childdoc.def|, |cdocsamp.tex|
with |cdocsch1.tex|, |cdocsch2.tex|, |cdocspt3.tex|, |cdocspt4.tex|,
|cdocsdrf.tex|, |cdocsfn1.tex|, |cdocsfn2.tex|
as well as |childdoc.pdf|.

%%%%%%%%%%%%%%%%%%%%%%%%%%%%%%%%%%%%%%%%%%%%%%%%%%%%%%%%%%%%%%%%%%%%%%%%%%%%%%%%
\subsection{Files and Installation}

The package consists of the files:
%
\begin{center}
\begin{tabular}{ll}
    |README.txt|   & readme file \\
    |childdoc.ins| & installation file \\
    |childdoc.dtx| & source file \\
    |childdoc.def| & definition file \\
    |cdocsamp.tex| & sample main file \\
    |cdocsch1.tex| & sample include file \\
    |cdocsch2.tex| & sample include file \\
    |cdocspt3.tex| & sample part file \\
    |cdocspt4.tex| & sample part file \\
    |cdocsdrf.tex| & sample redirection file \\
    |cdocsfn1.tex| & sample redirection file \\
    |cdocsfn2.tex| & sample redirection file \\
    |childdoc.pdf| & manual
\end{tabular}
\end{center}
%
The distribution consists of the files
|README.txt|, |childdoc.ins| and |childdoc.dtx|.
%
\begin{itemize}
\item
Run (pdf)\LaTeX{} on |childdoc.dtx|
to compile the manual |childdoc.pdf| (this file).
\item
Run \LaTeX{} on |childdoc.ins| to create the definitions file |childdoc.def|
and the sample |cdocsamp.tex| with include files
|cdocsch1.tex|, |cdocsch2.tex|, |cdocspt3.tex|, |cdocspt4.tex|,
|cdocsdrf.tex|, |cdocsfn1.tex|, |cdocsfn2.tex|.
Then copy the file |childdoc.def| to an appropriate directory of your \LaTeX{}
distribution, e.g.\ \textit{texmf-root}|/tex/latex/childdoc|.
\end{itemize}

%%%%%%%%%%%%%%%%%%%%%%%%%%%%%%%%%%%%%%%%%%%%%%%%%%%%%%%%%%%%%%%%%%%%%%%%%%%%%%%%
\subsection{Related CTAN Packages}

There are several other packages which offer a similar functionality:
%
\begin{itemize}
\item
The packages
\href{http://ctan.org/pkg/docmute}{\textsf{docmute}},
\href{http://ctan.org/pkg/includex}{\textsf{includex}} and
\href{http://ctan.org/pkg/standalone}{\textsf{standalone}}
provide commands to include only the document body of
a child file thus allowing both files to be compiled individually.
\item
The packages \href{http://ctan.org/pkg/subdocs}{\textsf{subdocs}}
and \href{http://ctan.org/pkg/subfiles}{\textsf{subfiles}}
provide structures in which the main and child documents can be
encapsulated and allowing them to be compiled individually.
The inclusion mechanism is different from the conventional |\include|.
\item
The package \href{http://ctan.org/pkg/combine}{\textsf{combine}}
is an elaborate solution to combine several documents into one.
\end{itemize}
%
See also the CTAN topic \href{http://ctan.org/topic/subdocs}{\textsf{subdocs}}
for further related packages.
The present package differs from the above solutions in that
a document structure constructed with the conventional |\include| mechanism
just needs two extra commands at the top of every file
such that all constituent files can be compiled individually.

%%%%%%%%%%%%%%%%%%%%%%%%%%%%%%%%%%%%%%%%%%%%%%%%%%%%%%%%%%%%%%%%%%%%%%%%%%%%%%%%
%\subsection{Feature Suggestions}
%
%The following is a list of features which may be useful for future
%versions of this package:
%%
%\begin{itemize}
%\item
%\ldots
%\end{itemize}

%%%%%%%%%%%%%%%%%%%%%%%%%%%%%%%%%%%%%%%%%%%%%%%%%%%%%%%%%%%%%%%%%%%%%%%%%%%%%%%%
\subsection{Revision History}

%%%%%%%%%%%%%%%%%%%%%%%%%%%%%%%%%%%%%%%%
\paragraph{v2.0:} 2018/12/30

\begin{itemize}
\item
immediate forward processing
\item
added |\childdocby| mechanism
\item
manual restructured
\end{itemize}

%%%%%%%%%%%%%%%%%%%%%%%%%%%%%%%%%%%%%%%%
\paragraph{v1.6:} 2018/01/17

\begin{itemize}
\item
application for development of include files
\item
corrections to manual
\end{itemize}

%%%%%%%%%%%%%%%%%%%%%%%%%%%%%%%%%%%%%%%%
\paragraph{v1.5:} 2017/05/21

\begin{itemize}
\item
more complete structuring introduced
\item
|\childdocof| introduced
\item
|\childdoc| renamed to |\childdocmain|
\item
|\childredirect| renamed to |\childdocforward| and |\childdocforwardprefix|
and functionality expanded
\end{itemize}

%%%%%%%%%%%%%%%%%%%%%%%%%%%%%%%%%%%%%%%%
\paragraph{v1.0:} 2017/04/27

\begin{itemize}
\item
manual and install package
\item
first version published on CTAN
\end{itemize}

%%%%%%%%%%%%%%%%%%%%%%%%%%%%%%%%%%%%%%%%
\paragraph{v0.6:} 2017/04/26

\begin{itemize}
\item
redirection mechanism added
\end{itemize}

%%%%%%%%%%%%%%%%%%%%%%%%%%%%%%%%%%%%%%%%
\paragraph{v0.5:} 2017/04/26

\begin{itemize}
\item
functionality in definition file
\end{itemize}


%%%%%%%%%%%%%%%%%%%%%%%%%%%%%%%%%%%%%%%%%%%%%%%%%%%%%%%%%%%%%%%%%%%%%%%%%%%%%%%%
%%%%%%%%%%%%%%%%%%%%%%%%%%%%%%%%%%%%%%%%%%%%%%%%%%%%%%%%%%%%%%%%%%%%%%%%%%%%%%%%
%%%%%%%%%%%%%%%%%%%%%%%%%%%%%%%%%%%%%%%%%%%%%%%%%%%%%%%%%%%%%%%%%%%%%%%%%%%%%%%%
\appendix

\settowidth\MacroIndent{\rmfamily\scriptsize 000\ }

 \DocInput{childdoc.dtx}

\end{document}
%</driver>
% \fi
%
% %%%%%%%%%%%%%%%%%%%%%%%%%%%%%%%%%%%%%%%%%%%%%%%%%%%%%%%%%%%%%%%%%%%%%%%%%%%%%%
% %%%%%%%%%%%%%%%%%%%%%%%%%%%%%%%%%%%%%%%%%%%%%%%%%%%%%%%%%%%%%%%%%%%%%%%%%%%%%%
% \section{Sample}
%\iffalse
%<*samplemain>
%\fi
%
% The following presents a sample document
% with two chapters, two parts, a title page,
% a compile flag as well as three forwarding files to set the flag.
% It consists of eight |.tex| files:
% \begin{center}
% \begin{tabular}{ll}
% |cdocsamp.tex|&main file\\
% |cdocsch1.tex|&include file for chapter 1\\
% |cdocsch2.tex|&include file for chapter 2\\
% |cdocspt3.tex|&include file for part 3\\
% |cdocspt4.tex|&include file for part 4\\
% |cdocsdrf.tex|&forwarding file for main file in draft mode\\
% |cdocsfi1.tex|&forwarding file for final version of chapter 1\\
% |cdocsfi2.tex|&forwarding file for final version of chapter 2\\
% \end{tabular}
% \end{center}
% Each of the eight files can be compiled directly by the \LaTeX{} compiler.
%
% %%%%%%%%%%%%%%%%%%%%%%%%%%%%%%%%%%%%%%
% \paragraph{Main File.}
%
% The main file is called |cdocsamp.tex|.
%
% Load the \textsf{childdoc} definitions and
% declare the filename for the main document:
%    \begin{macrocode}
\input{childdoc.def}
\childdocmain{}
%    \end{macrocode}

% Optional override for |\version| flag:
%    \begin{macrocode}
%%\ifchilddoc\else\providecommand{\version}{draft}\fi
%    \end{macrocode}

% Define the default values for the |\version| flag
% (|final| for the main file and |draft| for childs):
%    \begin{macrocode}
\ifchilddoc
\providecommand{\version}{draft}
\else
\providecommand{\version}{final}
\fi
%    \end{macrocode}

% Load the standard document class:
%    \begin{macrocode}
\documentclass[12pt]{article}
%    \end{macrocode}

% Start the document body:
%    \begin{macrocode}
\begin{document}
%    \end{macrocode}

% Declare a title page.
% Print title, part of document being processed and version flag:
%    \begin{macrocode}
\addtocounter{page}{-1}
\begin{center}
{\LARGE\bfseries{}childdoc example\par}
\vspace{1cm}
\ifchilddoc
\ifchilddocmanual part\else chapter\fi:
`\childdocname' of `\childdocjob'\par
\else
main document: `\childdocjob'\par
\fi
version: \version\par
\end{center}
\newpage
%    \end{macrocode}

% Manually include selected file,
% otherwise process as usual:
%    \begin{macrocode}
\ifchilddocmanual
\section*{part `\childdocname'}
\input{\childdocname}
\else
%    \end{macrocode}

% Include the two chapters:
%    \begin{macrocode}
\include{cdocsch1}
\include{cdocsch2}
%    \end{macrocode}

% Include the two parts unless only chapters should be displayed:
%    \begin{macrocode}
\ifchilddoc\else
\section{part three}
\input{cdocspt3}
\section{part four}
\input{cdocspt4}
\fi
%    \end{macrocode}

% Process as usual until here:
%    \begin{macrocode}
\fi
%    \end{macrocode}

% End of document body:
%    \begin{macrocode}
\end{document}
%    \end{macrocode}
%\iffalse
%</samplemain>
%\fi
%
% %%%%%%%%%%%%%%%%%%%%%%%%%%%%%%%%%%%%%%
% \paragraph{Chapter Include Files.}
%
% The include files are called |cdocsch1.tex| and |cdocsch2.tex|.
%
%\iffalse
%<*samplechap1|samplechap2>
%\fi

% Optional override for |\version| flag:
%    \begin{macrocode}
%%\providecommand{\version}{final}
%    \end{macrocode}

% Include the main document:
%    \begin{macrocode}
\input{childdoc.def}
\childdocof{cdocsamp}
%    \end{macrocode}

%\iffalse
%</samplechap1|samplechap2>
%\fi
%
%\iffalse
%<*samplechap1>
%\fi
% Some text for chapter 1:
%    \begin{macrocode}
\section{one}
some text in chapter one
%    \end{macrocode}

%\iffalse
%</samplechap1>
%\fi
% Some text for chapter 2:
%\iffalse
%<*samplechap2>
%\fi
%    \begin{macrocode}
\section{two}
more text in chapter two
%    \end{macrocode}

%\iffalse
%</samplechap2>
%\fi
%
% %%%%%%%%%%%%%%%%%%%%%%%%%%%%%%%%%%%%%%
% \paragraph{Part Include Files.}
%
% The include files are called |cdocspt3.tex| and |cdocspt4.tex|.
%
%\iffalse
%<*samplepart3|samplepart4>
%\fi

% Optional override for |\version| flag:
%    \begin{macrocode}
%%\providecommand{\version}{final}
%    \end{macrocode}

% Include the main document:
%    \begin{macrocode}
\input{childdoc.def}
\childdocby{cdocsamp}
%    \end{macrocode}

%\iffalse
%</samplepart3|samplepart4>
%\fi
%
%\iffalse
%<*samplepart3>
%\fi
% Some text for part 3:
%    \begin{macrocode}
some text in part three
%    \end{macrocode}

%\iffalse
%</samplepart3>
%\fi
% Some text for part 4:
%\iffalse
%<*samplepart4>
%\fi
%    \begin{macrocode}
more text in part four
%    \end{macrocode}

%\iffalse
%</samplepart4>
%\fi
%
% %%%%%%%%%%%%%%%%%%%%%%%%%%%%%%%%%%%%%%
% \paragraph{Forwarding for a Complete Draft.}
%
% The following forwarding file |cdocsdrf.tex|
% compiles the main document in draft mode:
%\iffalse
%<*sampledraft>
%\fi
%    \begin{macrocode}
\def\version{draft}
\input{childdoc.def}
\childdocforward{cdocsamp}
%    \end{macrocode}

%\iffalse
%</sampledraft>
%\fi
%
% %%%%%%%%%%%%%%%%%%%%%%%%%%%%%%%%%%%%%%
% \paragraph{Forwarding for Final Version of the Chapters.}
%
% The following forwarding files |cdocsfn1.tex| and |cdocsfn2.tex|
% (with identical content)
% compile the final versions of the child documents
% |cdocsch1.tex| and |cdocsch2.tex|, respectively:
%\iffalse
%<*samplefinal>
%\fi
%    \begin{macrocode}
\def\version{final}
\input{childdoc.def}
\childdocforwardprefix[cdocsamp]{cdocsfn}{cdocsch}
%    \end{macrocode}

%\iffalse
%</samplefinal>
%\fi
%
% %%%%%%%%%%%%%%%%%%%%%%%%%%%%%%%%%%%%%%
% \paragraph{Command Line Processing.}
%
% The following three command lines generate the output files
% |cdocscld|, |cdocscl1| and |cdocscl2|
% which should be identical to
% |cdocsdrf|, |cdocsch1| and |cdocsfn2|, respectively:
% \begin{center}
% \begin{tabular}{l}
% |latex -jobname cdocscld \|\\
% |  "\def\version{draft}\input{childdoc.def}\childdocforward{cdocsamp}"|\\
% |latex -jobname cdocscl1 \|\\
% |  "\input{childdoc.def}\childdocforward[cdocsamp]{cdocsch1}"|\\
% |latex -jobname cdocscl2 \|\\
% |  "\def\version{final}\input{childdoc.def}\childdocforward{cdocsch2}"|
% \end{tabular}
% \end{center}
% Note that the trailing backslash on each first line
% merely continues the input to the second line
% (for convenient cut ant paste).
% Furthermore, the command |latex| can be replaced by any
% of its alternative versions such as |pdflatex|.
%
% %%%%%%%%%%%%%%%%%%%%%%%%%%%%%%%%%%%%%%%%%%%%%%%%%%%%%%%%%%%%%%%%%%%%%%%%%%%%%%
% %%%%%%%%%%%%%%%%%%%%%%%%%%%%%%%%%%%%%%%%%%%%%%%%%%%%%%%%%%%%%%%%%%%%%%%%%%%%%%
% \section{Implementation}
%\iffalse
%<*package>
%\fi
%
% This section describes the definitions file |childdoc.def|.

% The definitions cannot be loaded using |\usepackage| or |\RequirePackage|
% which has a mechanism to prevent loading a style file more than once.
% When loading the definitions by means of |\input|
% multiple instances have to be prevented manually:
%\iffalse
%This code needs to be before the `\ProvidesFile' directive
%which is defined at the beginning of this file.
%Therefore it is also placed there and commented out here.
%</package>
%<*discard>
%\fi
%    \begin{macrocode}
\ifdefined\childdocmain\endinput\fi
%    \end{macrocode}
%\iffalse
%</discard>
%<*package>
%\fi
%
% \macro{\ifchilddoc}
% \macro{\ifchilddocmanual}
% The conditional |\ifchilddoc| tells whether a
% child (true) or main (false) document is being compiled.
% The conditional |\ifchilddocmanual| tells whether
% the |\includeonly| mechanism is used (false) or
% the selection of child files must be performed manually (true).
% The definitions initialise to false:
%    \begin{macrocode}
\newif\ifchilddoc
\newif\ifchilddocmanual
%    \end{macrocode}

% \macro{\childdocname}
% \macro{\childdocjob}
% The macro |\childdocname| stores the name of the main document
% to be compiled. The macro |\childdocjob| stores the name of
% the document on which the \LaTeX{} compiler was originally invoked.
% The content of |\jobname| cannot be compared
% to filenames specified in the source due to different catcodes.
% The following code rescans |\jobname|, stores the result
% in |\childdocname| and saves a copy in |\childdocjob|:
%    \begin{macrocode}
\edef\childdocname{\scantokens\expandafter{\jobname\noexpand}}
\let\childdocjob\childdocname
%    \end{macrocode}

% \macro{\childdocdisable}
% The macro |\childdocdisable| prevents the main file
% from being processed more than once.
% At this stage, the main document command |\childdocmain|
% is assumed to be called once again where it should do nothing.
% Any subsequent call to it should prevent
% a secondary processing of the main document
% It overwrites the forwarding commands
% |\childdocof| and |\childdocforward|
% with empty macros to prevent further inclusions of the main document:
%    \begin{macrocode}
\newcommand{\childdocdisable}
{
  \renewcommand{\childdocmain}[1]{\renewcommand{\childdocmain}[1]{\endinput}}
  \renewcommand{\childdocof}[1]{}
  \renewcommand{\childdocby}[2][]{}
  \renewcommand{\childdocforward}[2][]{}
  \renewcommand{\childdocdisable}{}
}
%    \end{macrocode}

% \macro{\childdocmain}
% The macro |\childdocmain| is to be called at the top of the main file
% with nothing or the main filename (without extension) as argument.
% First, it breaks loops.
% If the argument is not empty and does not match |\childdocname|
% (which is set by the first inclusion of |childdoc.def|),
% |\ifchilddoc| is set to true, |\includeonly| is applied to the child file
% and |\jobname| is set to the main file
% (for proper handling of |.aux| files):
%    \begin{macrocode}
\newcommand{\childdocmain}[1]
{
  \childdocdisable\childdocmain{}
  \if?#1?\else
    \begingroup
      \def\childdoctmp{#1}
      \ifx\childdoctmp\childdocname
        \def\childdoctmp{}
      \else
        \def\childdoctmp
        {
          \childdoctrue
          \includeonly{\childdocname}
          \def\childdocjob{#1}
          \def\jobname{#1}
        }
      \fi
      \expandafter
    \endgroup
    \childdoctmp
  \fi
}
%    \end{macrocode}

% \macro{\childdocof}
% The command |\childdocof| redirects
% compilation to the main file |#1|.
%    \begin{macrocode}
\newcommand{\childdocof}[1]
{
  \childdocdisable
  \childdoctrue
  \includeonly{\childdocname}
  \def\jobname{#1}
  \def\childdocjob{#1}
  \input{#1}
}
%    \end{macrocode}

% \macro{\childdocby}
% The command |\childdocby| ....
%    \begin{macrocode}
\newcommand{\childdocby}[2][]
{
  \childdocdisable
  \childdoctrue
  \childdocmanualtrue
  \if?#1?\else
    \def\jobname{#2}
  \fi
  \def\childdocjob{#2}
  \input{#2}
  \endinput
}
%    \end{macrocode}

% \macro{\childdocforward}
% The command |\childdocforward| redirects
% compilation to the main file or
% (if the optional argument is given) a child file.
% Parameters are set as if the main file
% or a child file starting with |\childdocof| was compiled.
% Then compilation is handed over to the main file:
%    \begin{macrocode}
\newcommand{\childdocforward}[2][]
{
  \begingroup
    \if?#1?
      \def\childdoctmp
      {
        \def\childdocname{#2}
        \def\childdocjob{#2}
        \def\jobname{#2}
        \input{#2}
        \endinput
      }
    \else
      \def\childdoctmp
      {
        \childdocdisable
        \def\childdocname{#2}
        \childdoctrue
        \includeonly{#2}
        \def\childdocjob{#1}
        \def\jobname{#1}
        \input{#1}
        \endinput
      }
    \fi
    \expandafter
  \endgroup
  \childdoctmp
}
%    \end{macrocode}

% \macro{\childdocforwardprefix}
% The command |\childdocforwardprefix| redirects
% compilation to the main or a child file by means of a pattern.
% The prefix |#1| in the current filename is replaced by |#2|
% and the suffix of the current filename is kept
% (it is assumed that the filename does not contain the substring `|~~~|'
% which is used as a delimiter).
% Compilation is handed over to the new file by |\childdocforward|:
%    \begin{macrocode}
\newcommand{\childdocforwardprefix}[3][]
{
  \begingroup
    \def\childdocextract #2##1~~~{\def\childdoctmp{\childdocforward[#1]{#3##1}}}
    \expandafter\childdocextract\childdocname~~~
    \expandafter
  \endgroup
  \childdoctmp
}
%    \end{macrocode}

% \macro{\childdoc}
% The deprecated macro |\childdoc| is a legacy version of |\childdocmain|:
%    \begin{macrocode}
\newcommand{\childdoc}{\childdocmain}
%    \end{macrocode}

% \macro{\childdocredirect}
% The deprecated macro |\childdocredirect| is a legacy version
% of |\childdocforward| and |\childdocforwardprefix|:
%    \begin{macrocode}
\newcommand{\childdocredirect}[2][]
{
  \begingroup
    \if?#1?
      \def\childdoctmp{\childdocforward{#2}}
    \else
      \def\childdoctmp{\childdocforwardprefix{#1}{#2}}
    \fi
    \expandafter
  \endgroup
  \childdoctmp
}
%    \end{macrocode}

%\iffalse
%</package>
%\fi
%
\endinput

\childdocmain{}
%    \end{macrocode}

% Optional override for |\version| flag:
%    \begin{macrocode}
%%\ifchilddoc\else\providecommand{\version}{draft}\fi
%    \end{macrocode}

% Define the default values for the |\version| flag
% (|final| for the main file and |draft| for childs):
%    \begin{macrocode}
\ifchilddoc
\providecommand{\version}{draft}
\else
\providecommand{\version}{final}
\fi
%    \end{macrocode}

% Load the standard document class:
%    \begin{macrocode}
\documentclass[12pt]{article}
%    \end{macrocode}

% Start the document body:
%    \begin{macrocode}
\begin{document}
%    \end{macrocode}

% Declare a title page.
% Print title, part of document being processed and version flag:
%    \begin{macrocode}
\addtocounter{page}{-1}
\begin{center}
{\LARGE\bfseries{}childdoc example\par}
\vspace{1cm}
\ifchilddoc
\ifchilddocmanual part\else chapter\fi:
`\childdocname' of `\childdocjob'\par
\else
main document: `\childdocjob'\par
\fi
version: \version\par
\end{center}
\newpage
%    \end{macrocode}

% Manually include selected file,
% otherwise process as usual:
%    \begin{macrocode}
\ifchilddocmanual
\section*{part `\childdocname'}
\input{\childdocname}
\else
%    \end{macrocode}

% Include the two chapters:
%    \begin{macrocode}
\include{cdocsch1}
\include{cdocsch2}
%    \end{macrocode}

% Include the two parts unless only chapters should be displayed:
%    \begin{macrocode}
\ifchilddoc\else
\section{part three}
\input{cdocspt3}
\section{part four}
\input{cdocspt4}
\fi
%    \end{macrocode}

% Process as usual until here:
%    \begin{macrocode}
\fi
%    \end{macrocode}

% End of document body:
%    \begin{macrocode}
\end{document}
%    \end{macrocode}
%\iffalse
%</samplemain>
%\fi
%
% %%%%%%%%%%%%%%%%%%%%%%%%%%%%%%%%%%%%%%
% \paragraph{Chapter Include Files.}
%
% The include files are called |cdocsch1.tex| and |cdocsch2.tex|.
%
%\iffalse
%<*samplechap1|samplechap2>
%\fi

% Optional override for |\version| flag:
%    \begin{macrocode}
%%\providecommand{\version}{final}
%    \end{macrocode}

% Include the main document:
%    \begin{macrocode}
% \iffalse
%
% childdoc.dtx Copyright (C) 2017-2018 Niklas Beisert
%
% This work may be distributed and/or modified under the
% conditions of the LaTeX Project Public License, either version 1.3
% of this license or (at your option) any later version.
% The latest version of this license is in
%   http://www.latex-project.org/lppl.txt
% and version 1.3 or later is part of all distributions of LaTeX
% version 2005/12/01 or later.
%
% This work has the LPPL maintenance status `maintained'.
%
% The Current Maintainer of this work is Niklas Beisert.
%
% This work consists of the files childdoc.dtx and childdoc.ins
% and the derived files childdoc.def and cdocsamp.tex with
% cdocsch1.tex, cdocsch2.tex, cdocsdrf.tex, cdocsfn1.tex, cdocsfn2.tex.
%
%<package>\ifdefined\childdocmain\endinput\fi
%<package>\ProvidesFile{childdoc.def}[2018/12/30 v2.0 child document driver]
%<samplemain>\ProvidesFile{cdocsamp.tex}[2018/12/30 v2.0 sample for childdoc]
%<*driver>
%\ProvidesFile{childdoc.drv}[2018/12/30 v2.0 childdoc reference manual file]
\PassOptionsToClass{10pt,a4paper}{article}
\documentclass{ltxdoc}

\usepackage[margin=35mm]{geometry}
\usepackage{hyperref}
\usepackage{hyperxmp}
\usepackage[usenames]{color}

\hypersetup{colorlinks=true}
\hypersetup{pdfstartview=FitH}
\hypersetup{pdfpagemode=UseNone}
\hypersetup{pdfsource={}}
\hypersetup{pdflang={en-UK}}
\hypersetup{pdfcopyright={Copyright 2017-2018 Niklas Beisert.
  This work may be distributed and/or modified under the
  conditions of the LaTeX Project Public License, either version 1.3
  of this license or (at your option) any later version.}}
\hypersetup{pdflicenseurl={http://www.latex-project.org/lppl.txt}}
\hypersetup{pdfcontactaddress={ETH Zurich, ITP, HIT K,
  Wolfgang-Pauli-Strasse 27}}
\hypersetup{pdfcontactpostcode={8093}}
\hypersetup{pdfcontactcity={Zurich}}
\hypersetup{pdfcontactcountry={Switzerland}}
\hypersetup{pdfcontactemail={nbeisert@itp.phys.ethz.ch}}
\hypersetup{pdfcontacturl={http://people.phys.ethz.ch/\xmptilde nbeisert/}}

\newcommand{\secref}[1]{\hyperref[#1]{section \ref*{#1}}}

\parskip1ex
\parindent0pt
\let\olditemize\itemize
\def\itemize{\olditemize\parskip0pt}

\begin{document}

\title{The \textsf{childdoc} Package}
\hypersetup{pdftitle={The childdoc Package}}
\author{Niklas Beisert\\[2ex]
  Institut f\"ur Theoretische Physik\\
  Eidgen\"ossische Technische Hochschule Z\"urich\\
  Wolfgang-Pauli-Strasse 27, 8093 Z\"urich, Switzerland\\[1ex]
  \href{mailto:nbeisert@itp.phys.ethz.ch}
  {\texttt{nbeisert@itp.phys.ethz.ch}}}
\hypersetup{pdfauthor={Niklas Beisert}}
\hypersetup{pdfsubject={Manual for the LaTeX2e Package childdoc}}
\date{30 December 2018, \textsf{v2.0}}
\maketitle

\begin{abstract}\noindent
\textsf{childdoc} is a \LaTeXe{} package
that enables the direct compilation
of document sections included by |\include|
to individual files.
\end{abstract}

\begingroup
\parskip0ex
\tableofcontents
\endgroup

%%%%%%%%%%%%%%%%%%%%%%%%%%%%%%%%%%%%%%%%%%%%%%%%%%%%%%%%%%%%%%%%%%%%%%%%%%%%%%%%
%%%%%%%%%%%%%%%%%%%%%%%%%%%%%%%%%%%%%%%%%%%%%%%%%%%%%%%%%%%%%%%%%%%%%%%%%%%%%%%%
\section{Introduction}

\LaTeX{} provides a mechanism to structure a large document (such as a book)
into a main file and several child files (containing the chapters)
using the |\include| command.
This mechanism is beneficial for documents
which span hundreds of pages in order to
make the source file(s) more manageable.
Moreover, compilation can be restricted to
selected child files by means of the |\includeonly| command.
The latter feature can be used to reduce the compilation time while editing
(this was significantly more useful in the earlier days of \LaTeX{})
or to generate a smaller document which is easier to navigate.
Another application of |\includeonly| is to generate
documents consisting of selected parts of the complete document.

However, there are a few drawbacks of the plain |\include| mechanism:
\begin{itemize}
\item
The child files cannot be compiled on their own,
they can only be compiled via the main file.
A naive editing environment
(such as a text editor with an option
to have the current file processed by \LaTeX)
may require one to switch to the main file before compiling;
attempting to compile the child file produces errors.
\item
The main file must be modified (each time)
to adjust the |\includeonly| command
to the present needs. This easily leaves the main file in a messy state.
\item
The generated document will always carry the filename
of the main document. This is inconvenient if
several child files are to be compiled and
to be kept for distribution.
\end{itemize}

The present package provides a simple interface
to make child files individually compilable by \LaTeX{}.
Compiling a child file then has the same effect as compiling
the main file with an |\includeonly| command
to select the appropriate child.
Moreover the generated document will carry the name of the child
rather than the main file.
This resolves all three above issues.

This feature is meant to make the editing of books,
thesis documents and lecture notes somewhat more convenient.
However, the package can also be used efficiently for
composing a series of documents (such as exercise sheets)
which are typically distributed individually.
It then assists the author in generating the individual documents
(potentially in different versions)
as well as a document containing the collected series.
Another application is in developing style files
or other kinds of included material
where compilation of the style file could redirect
to a sample or test file.

%%%%%%%%%%%%%%%%%%%%%%%%%%%%%%%%%%%%%%%%%%%%%%%%%%%%%%%%%%%%%%%%%%%%%%%%%%%%%%%%
%%%%%%%%%%%%%%%%%%%%%%%%%%%%%%%%%%%%%%%%%%%%%%%%%%%%%%%%%%%%%%%%%%%%%%%%%%%%%%%%
\section{Usage}

First of all, the package \textsf{childdoc} is \emph{not} a standard
\LaTeXe{} |.sty| style file! Therefore it needs to be invoked in
a non-standard way.

%%%%%%%%%%%%%%%%%%%%%%%%%%%%%%%%%%%%%%%%%%%%%%%%%%%%%%%%%%%%%%%%%%%%%%%%%%%%%%%%
\subsection{Included Files}
\label{sec:include}

%%%%%%%%%%%%%%%%%%%%%%%%%%%%%%%%%%%%%%%%
\DescribeMacro{\childdocmain}
To use the package, add the commands
\begin{center}
\begin{tabular}{l}
|\input{childdoc.def}|\\
|\childdocmain{}|\\
\end{tabular}
\end{center}
at the very top of the main \LaTeX{} file,
in particular \emph{before} the |\documentclass| statement!
The argument of |\childdocmain| should be left empty
(but it must be present).

%%%%%%%%%%%%%%%%%%%%%%%%%%%%%%%%%%%%%%%%
\DescribeMacro{\childdocof}
Furthermore, add the commands
\begin{center}
\begin{tabular}{l}
|\input{childdoc.def}|\\
|\childdocof{|\textit{main}|}|\\
\end{tabular}
\end{center}
at the top of every child file \textit{child}
which is included by |\include{|\textit{child}|}|
from within the main file
(or at least for those files to be compiled individually).
The argument \textit{main} must be the filename of the main file.

There are a couple of
considerations in setting up the main and child documents:

%%%%%%%%%%%%%%%%%%%%%%%%%%%%%%%%%%%%%%%%
\paragraph{Restrictions.}

Please note the following restrictions:
\begin{itemize}
\item
|\childdocmain| must be called with one argument \textit{main}
to ensure compatibility with earlier version of the package.
It must either be empty (|\childdocmain{}|)
or precisely match the filename of the main file in which it is specified.
See \secref{sec:detection} for further information.
\item
The filename \textit{main} must be specified without the |.tex| extension.
\item
The filename \textit{main} is case sensitive
(even in case-insensitive file systems)
due to internal string comparison.
\item
The argument \textit{main} should be fully expanded, it cannot be a macro.
\item
Subdirectories and special characters should be avoided in filenames.
\item
The command |\childdocmain{|\textit{main}|}| must be followed by a whitespace.
It should not be followed immediately by another command
or by a comment mark `|%|'.
This is because the \TeX{} parser reads the token immediately following
the argument of |\childdocmain| and puts it
at the beginning of every child section;
however, a white\-space is ignored.
\end{itemize}

%%%%%%%%%%%%%%%%%%%%%%%%%%%%%%%%%%%%%%%%
\paragraph{Content of Main File.}

It is advisable to place all content in the child files included by |\include|.
Any output contained in the main file will appear in all child documents
unless suppressed manually;
it cannot be suppressed automatically by the |\includeonly| directive
and thus should normally be avoided.
A method to include some content in the main file
by means of conditional processing is described in \secref{sec:conditional}.

%%%%%%%%%%%%%%%%%%%%%%%%%%%%%%%%%%%%%%%%
\paragraph{Page Numbering.}

When only a part of the document is compiled,
the appropriate numbering of pages
(as well as other status parameters)
is determined from the |.aux| files.
The latter contain information from previous passes.
However this information needs to propagate through
all intermediate child documents.
Therefore the page numbering in child documents may well
be inconsistent until the complete document is compiled at least once.

A useful (if unconventional) way to always ensure a consistent
page numbering is to restart the numbering in each child document
and denote the pages by `\textit{child}|.|\textit{page}'
where \textit{child} represents the chapter/section number of the child file.
This can be achieved by the command
|\numberwithin{page}{|\textit{child}|}|
of the \textsf{amsmath} package
where \textit{child} can be |chapter| or |section|
depending on the chosen structuring.
Alternatively, one can modify the macro |\thepage| appropriately
and reset the counter |page| at the start of each child file.

%%%%%%%%%%%%%%%%%%%%%%%%%%%%%%%%%%%%%%%%%%%%%%%%%%%%%%%%%%%%%%%%%%%%%%%%%%%%%%%%
\subsection{Conditional Processing}
\label{sec:conditional}

The package provides a mechanism to compile different versions
of a document. To customise the versions further some conditional processing
can come in handy to distinguish which version is being compiled.
The package provides two macros to describe the compilation context:

%%%%%%%%%%%%%%%%%%%%%%%%%%%%%%%%%%%%%%%%
\DescribeMacro{\ifchilddoc}
The conditional |\ifchilddoc| distinguishes between the compilation of
child documents and the main document:
%
\begin{center}
|\ifchilddoc |\textit{child-code}| |[|\||else |\textit{main-code}]| \||fi|
\end{center}

%%%%%%%%%%%%%%%%%%%%%%%%%%%%%%%%%%%%%%%%
\DescribeMacro{\childdocname}
\DescribeMacro{\childdocjob}
The macro |\childdocname| contains the filename (without extension)
of the main or child file being processed.
Note that |\childdocjob| will always contain the name of the main file.

%%%%%%%%%%%%%%%%%%%%%%%%%%%%%%%%%%%%%%%%
\paragraph{Title Page.}

Conditional processing can be used to include a title or banner page
in the main document when proper precautions are taken.
Importantly, the code in the main file should ensure that the page counter
(as well as other status parameters which are stored in the |.aux| files)
takes the same value after the conditional processing.
Otherwise the page numbers may take divergent values
depending on which part is compiled.

For example, a title page could be declared by:
%
\begin{center}
\begin{tabular}{l}
|\ifchilddoc\||else|\\
|\addtocounter{page}{-1}|\\
\textit{code for title page}\\
|\newpage|\\
|\||fi|
\end{tabular}
\end{center}
%
A banner page for the child documents can be generated by:
%
\begin{center}
\begin{tabular}{l}
|\ifchilddoc|\\
|\addtocounter{page}{-1}|\\
\textit{code for banner page}\\
|\newpage|\\
|\||fi|
\end{tabular}
\end{center}
%
Here one could write a message such as:
\begin{center}
|This is the part \childdocname{} of \childdocjob{}.|
\end{center}

%%%%%%%%%%%%%%%%%%%%%%%%%%%%%%%%%%%%%%%%%%%%%%%%%%%%%%%%%%%%%%%%%%%%%%%%%%%%%%%%
\subsection{Flags}
\label{sec:flags}

The package makes it easy to generate different versions
of the main or child documents.
To this end compilation flags can be defined
and assigned different default values.
They will be particularly useful in conjunction
with the forwarding mechanism described in \secref{sec:forward}.

For example, it may be useful to have a flag |\version|
which can be set to |draft| or |final|.
The document source will contain some conditional code
depending on the value of |\version|.
Suppose further, the flag should default to |final| for the main file
and to |draft| for child files
which is a natural assignment for editing the document.
This is achieved by placing the following code
in the preamble of the main document
(below the |\childdocmain| directive):
%
\begin{center}
\begin{tabular}{l}
|\ifchilddoc|\\
|\providecommand{\version}{draft}|\\
|\||else|\\
|\providecommand{\version}{final}|\\
|\||fi|
\end{tabular}
\end{center}
%
The definition by |\providecommand| makes sure
that previous definitions are not overwritten.
Further statements |\providecommand{\version}{...}|
can thus be added before the above code to override it.

For the main file, one might add a line
(between |\childdocmain| and the above block)
%
\begin{center}
|%\ifchilddoc\||else\providecommand{\version}{draft}\||fi|
\end{center}
%
which can be uncommented to produce a draft version.
Likewise one can add a line to the very top of a child file
(above the |\childdocof{|\textit{main}|}| directive)
%
\begin{center}
|%\providecommand{\version}{final}|
\end{center}
%
which can be uncommented to produce the final version of this child document.

%%%%%%%%%%%%%%%%%%%%%%%%%%%%%%%%%%%%%%%%%%%%%%%%%%%%%%%%%%%%%%%%%%%%%%%%%%%%%%%%
\subsection{Forwarding}
\label{sec:forward}

Different versions of the main or child documents
using compilation flags as described in \secref{sec:flags}
can be (permanently) stored in different files
for convenient compilation, viewing and distribution.
To this end, the package defines a command
to pass on compilation to a different file:

%%%%%%%%%%%%%%%%%%%%%%%%%%%%%%%%%%%%%%%%
\DescribeMacro{\childdocforward}
The command |\childdocforward| redirects processing to
another source file:
%
\begin{center}
\begin{tabular}{l}
|\input{childdoc.def}|\\
|\childdocforward[|\textit{main}|]{|\textit{dest}|}|\\
\end{tabular}
\end{center}
%
The argument \textit{dest} is the destination file
(without extension).
It should be the main file or one of the child files.
Note that further \textsf{childdoc} directives
such as |\childdocof| and |\childdocforward|
in the indicated file will be processed in this form.
The optional argument \textit{main}
passes on directly to the main file \textit{main}
while pretending to compile the child \textit{dest}.
This form behaves as if \textit{dest}
issues |\childdocof{|\textit{main}|}| right away,
and no further \textsf{childdoc} directives will be processed.

%%%%%%%%%%%%%%%%%%%%%%%%%%%%%%%%%%%%%%%%
\DescribeMacro{\...prefix}
In the alternative form |\childdocforwardprefix|,
%
\begin{center}
\begin{tabular}{l}
|\input{childdoc.def}|\\
|\childdocforwardprefix[|\textit{main}|]{|\textit{prefix}|}{|\textit{dest}|}|
\end{tabular}
\end{center}
%
the destination file is determined by a pattern
depending on the current file:
To make this work, the current file must be called
`{\textit{prefix}\hspace{0.2em}\textit{suffix}}'
with \textit{prefix} matching precisely the argument.
Processing is then passed on to the file
`{\textit{dest}\hspace{0.2em}\textit{suffix}}'.
Surely, the same effect is achieved by
directly specifying the
argument `{\textit{dest}\hspace{0.2em}\textit{suffix}}'
in the first form.
However, that requires to set up a different file
for each child. With the alternative form of the command
all these files can have exactly the same content
which simplifies setting them up and maintaining them.

For example, the following file |draft.tex|
with a compilation flag |\version| as described in \secref{sec:flags}
compiles the main document as a draft:
%
\begin{center}
\begin{tabular}{l}
|\def\version{draft}|\\
|\input{childdoc.def}|\\
|\childdocforward{|\textit{main}|}|
\end{tabular}
\end{center}
%
Likewise, the following files |final|\textit{nn}|.tex|
compile the final version of the child document
|child|\textit{nn}|.tex|:
%
\begin{center}
\begin{tabular}{l}
|\def\version{final}|\\
|\input{childdoc.def}|\\
|\childdocforwardprefix{final}{child}|
\end{tabular}
\end{center}
%

Note that when several versions of a main file and/or of each child file
are to be generated, it may be convenient to set up a |Makefile| or
shell script to automatise the process.

%%%%%%%%%%%%%%%%%%%%%%%%%%%%%%%%%%%%%%%%%%%%%%%%%%%%%%%%%%%%%%%%%%%%%%%%%%%%%%%%
\subsection{Command Line Processing}
\label{sec:commandline}

The effect of redirection files can also be achieved by invoking
the \LaTeX{} compiler with a more elaborate command line.
Most conveniently this should be done as part
of a shell script or a |Makefile|.

When using \textsf{childdoc} in the main file, the following
command lines effectively perform a redirection
(note that depending on the shell being used,
backslashes may have to be doubled: `|\|' $\to$ `|\\|'):
%
\begin{center}
|... -jobname "|\textit{target}|" |\\|"|[\textit{flags}]%
|\input{childdoc.def}\childdocforward[|\textit{main}|]{|\textit{dest}|}"|
\end{center}
%
Here \textit{target} is the name of the output file,
\textit{main} is the name of the main file
and \textit{dest} is the name of the main or child file to be processed
(all filenames without extensions).
The optional argument \textit{main} can be omitted
if \textit{main} matches \textit{dest}.
Optionally, compilation \textit{flags} can be defined via |\def| commands.
This command line makes the \TeX{} engine believe
it is compiling the file \textit{target}
whose content is specified as the latter parameter.
The provided code then forwards the processing to
\textit{main} or \textit{dest} as described in \secref{sec:forward}.

%%%%%%%%%%%%%%%%%%%%%%%%%%%%%%%%%%%%%%%%%%%%%%%%%%%%%%%%%%%%%%%%%%%%%%%%%%%%%%%%
\subsection{Include by Input}
\label{sec:input}

Including child documents by |\include| has some restrictions by design.
Most notably, the content of a child document always occupies
its own set of pages; pages cannot be shared between child documents.
Usually, this behaviour makes perfect sense
because each child document contain an essential part of the document.
However, in some situations it may be desirable to compose
a document from a collection of parts
without having mandatory page breaks between then.
For this case, the package
provides a mechanism to include parts
by |\input| which can also be processed individually.
However, by construction this mechanism
requires manual handling of the content to be output.

%%%%%%%%%%%%%%%%%%%%%%%%%%%%%%%%%%%%%%%%
\DescribeMacro{\ifchilddocmanual}
The main file should be prepared as usual, see \secref{sec:include}.
However, the document body must make a distinction
between processing of an individual part and of the main document, e.g.:
%
\begin{center}
\begin{tabular}{l}
|\ifchilddocmanual|\\
|\input{\childdocname}|\\
|\||else|\\
\textit{document body with }|\input{|\textit{part}|}|\\
|\||fi|
\end{tabular}
\end{center}
%
The conditional |\ifchilddocmanual| is true whenever
a part to be included by |\input| is being compiled,
and the name of the part is stored in |\childdocname|.

%%%%%%%%%%%%%%%%%%%%%%%%%%%%%%%%%%%%%%%%
\DescribeMacro{\childdocby}
Each part to be included by |\input| should start with:
%
\begin{center}
\begin{tabular}{l}
|\input{childdoc.def}|\\
|\childdocby{|\textit{main}|}|\\
\end{tabular}
\end{center}
%
The directive |\childdocby| is similar to |\childdocof|
described in \secref{sec:include},
but the subsequent selection of content must be done manually.
To that end, both |\ifchilddoc| and |\ifchilddocmanual|
will be true upon processing of a part,
and the name of the part is stored in |\childdocname|.
Note that |\jobname| will be set to the filename of the current part
so that each part receives an individual |.aux| file
that does not interfere with the |.aux| file(s) of the main document.
This behaviour can be altered by the alternative form
|\childdocby[*]{|\textit{main}|}| (with a non-empty optional argument)
which uses the |.aux| file of the main document
by setting |\jobname| to \textit{main}.

%%%%%%%%%%%%%%%%%%%%%%%%%%%%%%%%%%%%%%%%%%%%%%%%%%%%%%%%%%%%%%%%%%%%%%%%%%%%%%%%
\subsection{Driver Development}
\label{sec:driver}

The \textsf{childdoc} mechanism can also be use for the development
of definition files such as \LaTeX{} styles or classes.
This case differs from the above setup with multiple parts
included by |\include| in that no |\includeonly| should be invoked.
This can be achieved by starting the include file
(before |\ProvidesPackage|) with:
%
\begin{center}
\begin{tabular}{l}
|\input{childdoc.def}|\\
|\childdocforward{|\textit{main}|}|\\
\end{tabular}
\end{center}
%
or alternatively with:
%
\begin{center}
\begin{tabular}{l}
|\input{childdoc.def}|\\
|\childdocby{|\textit{main}|}|\\
\end{tabular}
\end{center}
%
Both forms have slightly different effects as described above.
The main file is prepared as usual, see \secref{sec:include}.

%%%%%%%%%%%%%%%%%%%%%%%%%%%%%%%%%%%%%%%%%%%%%%%%%%%%%%%%%%%%%%%%%%%%%%%%%%%%%%%%
\subsection{Legacy Detection}
\label{sec:detection}

The directive |\childdocmain| in the main file can detect
whether the complete document or merely a child is to be compiled
even without using the directive |\childdocof|.
This method is deprecated because it is less robust
and there is no compelling reason to use it;
it is merely provided for backward compatibility
and it may be removed in future versions.

If the detection mechanism is to be used,
it is mandatory to correctly specify
the filename of the main file as the argument of |\childdocmain|:
%
\begin{center}
\begin{tabular}{l}
|\input{childdoc.def}|\\
|\childdocmain{|\textit{main}|}|\\
\end{tabular}
\end{center}
%
If |\jobname| does not match the argument \textit{main} of |\childdocmain|,
it is assumed that |\jobname| points to the child file to be compiled.
When using |\childdocmain| with the main file specified as argument,
it suffices to start a child file
with just |\input{|\textit{main}|}|
without loading of the package and using |\childdocof|.
If instead all processing is done
with the appropriate \textsf{childdoc} directives,
the argument of \textit{main} of |\childdocmain| can be empty.

An alternative version of the command line processing described
in \secref{sec:commandline} using the detection mechanism reads:
%
\begin{center}
|... -jobname "|\textit{target}|" "|[\textit{flags}]%
[|\def\jobname{|\textit{dest}|}|]|\input{|\textit{main}|}"|
\end{center}

%%%%%%%%%%%%%%%%%%%%%%%%%%%%%%%%%%%%%%%%%%%%%%%%%%%%%%%%%%%%%%%%%%%%%%%%%%%%%%%%
\subsection{Manual Code}
\label{sec:manual}

In case one cannot be certain whether the definitions file |childdoc.def|
is installed on the target \TeX{} distribution
and one prefers not to ship it,
it is conceivable to paste a few relevant commands into the sources.

To that end, drop all statements |\input{childdoc.def}|
and perform the replacements as outlined below.
Instead of |\childdocmain{|\textit{main}|}| add the following code
to the top of the main file:
%
\begin{center}
\begin{tabular}{l}
|\||ifdefined\childdocname\endinput\||fi\newif\ifchilddoc|\\
|\edef\childdocname{\scantokens\expandafter{\jobname\noexpand}}|\\
|\def\childdocmain{|\textit{main}|}\||ifx\childdocmain\childdocname\||else|\\
|\childdoctrue\includeonly{\childdocname}\let\jobname\childdocmain\||fi|\\
\end{tabular}
\end{center}
%
Instead of |\childdocof{|\textit{main}|}| just include the main file
at the top of each child file:
%
\begin{center}
|\input{|\textit{main}|}|
\end{center}
%
A simple redirection |\childdocforward{|\textit{dest}|}| is achieved by:
%
\begin{center}
|\def\jobname{|\textit{dest}|}\input{\jobname}|
\end{center}
%
The redirection with prefix
|\childdocforwardprefix[|\textit{prefix}|]{|\textit{dest}|}|
is accomplished by:
%
\begin{center}
\begin{tabular}{l}
|{\edef\jobname{\scantokens\expandafter{\jobname\noexpand}}|\\
|\def\redirectjob |\textit{prefix}|#1~~~{\gdef\jobname{|\textit{dest}|#1}}|\\
|\expandafter\redirectjob\jobname~~~}\input{\jobname}|
\end{tabular}
\end{center}

In an alternative approach,
child documents can be compiled by a specific command line
without additional code or specific definitions:
%
\begin{center}
|... -jobname "|\textit{target}|" "|[\textit{flags}]%
|\includeonly{|\textit{dest}|}\input{|\textit{main}|}"|
\end{center}
%

%%%%%%%%%%%%%%%%%%%%%%%%%%%%%%%%%%%%%%%%%%%%%%%%%%%%%%%%%%%%%%%%%%%%%%%%%%%%%%%%
%%%%%%%%%%%%%%%%%%%%%%%%%%%%%%%%%%%%%%%%%%%%%%%%%%%%%%%%%%%%%%%%%%%%%%%%%%%%%%%%
\section{Information}

%%%%%%%%%%%%%%%%%%%%%%%%%%%%%%%%%%%%%%%%%%%%%%%%%%%%%%%%%%%%%%%%%%%%%%%%%%%%%%%%
\subsection{Copyright}

Copyright \copyright{} 2017--2018 Niklas Beisert

This work may be distributed and/or modified under the
conditions of the \LaTeX{} Project Public License, either version 1.3
of this license or (at your option) any later version.
The latest version of this license is in
  \url{http://www.latex-project.org/lppl.txt}
and version 1.3 or later is part of all distributions of \LaTeX{}
version 2005/12/01 or later.

This work has the LPPL maintenance status `maintained'.

The Current Maintainer of this work is Niklas Beisert.

This work consists of the files |README.txt|, |childdoc.ins| and |childdoc.dtx|
as well as the derived files |childdoc.def|, |cdocsamp.tex|
with |cdocsch1.tex|, |cdocsch2.tex|, |cdocspt3.tex|, |cdocspt4.tex|,
|cdocsdrf.tex|, |cdocsfn1.tex|, |cdocsfn2.tex|
as well as |childdoc.pdf|.

%%%%%%%%%%%%%%%%%%%%%%%%%%%%%%%%%%%%%%%%%%%%%%%%%%%%%%%%%%%%%%%%%%%%%%%%%%%%%%%%
\subsection{Files and Installation}

The package consists of the files:
%
\begin{center}
\begin{tabular}{ll}
    |README.txt|   & readme file \\
    |childdoc.ins| & installation file \\
    |childdoc.dtx| & source file \\
    |childdoc.def| & definition file \\
    |cdocsamp.tex| & sample main file \\
    |cdocsch1.tex| & sample include file \\
    |cdocsch2.tex| & sample include file \\
    |cdocspt3.tex| & sample part file \\
    |cdocspt4.tex| & sample part file \\
    |cdocsdrf.tex| & sample redirection file \\
    |cdocsfn1.tex| & sample redirection file \\
    |cdocsfn2.tex| & sample redirection file \\
    |childdoc.pdf| & manual
\end{tabular}
\end{center}
%
The distribution consists of the files
|README.txt|, |childdoc.ins| and |childdoc.dtx|.
%
\begin{itemize}
\item
Run (pdf)\LaTeX{} on |childdoc.dtx|
to compile the manual |childdoc.pdf| (this file).
\item
Run \LaTeX{} on |childdoc.ins| to create the definitions file |childdoc.def|
and the sample |cdocsamp.tex| with include files
|cdocsch1.tex|, |cdocsch2.tex|, |cdocspt3.tex|, |cdocspt4.tex|,
|cdocsdrf.tex|, |cdocsfn1.tex|, |cdocsfn2.tex|.
Then copy the file |childdoc.def| to an appropriate directory of your \LaTeX{}
distribution, e.g.\ \textit{texmf-root}|/tex/latex/childdoc|.
\end{itemize}

%%%%%%%%%%%%%%%%%%%%%%%%%%%%%%%%%%%%%%%%%%%%%%%%%%%%%%%%%%%%%%%%%%%%%%%%%%%%%%%%
\subsection{Related CTAN Packages}

There are several other packages which offer a similar functionality:
%
\begin{itemize}
\item
The packages
\href{http://ctan.org/pkg/docmute}{\textsf{docmute}},
\href{http://ctan.org/pkg/includex}{\textsf{includex}} and
\href{http://ctan.org/pkg/standalone}{\textsf{standalone}}
provide commands to include only the document body of
a child file thus allowing both files to be compiled individually.
\item
The packages \href{http://ctan.org/pkg/subdocs}{\textsf{subdocs}}
and \href{http://ctan.org/pkg/subfiles}{\textsf{subfiles}}
provide structures in which the main and child documents can be
encapsulated and allowing them to be compiled individually.
The inclusion mechanism is different from the conventional |\include|.
\item
The package \href{http://ctan.org/pkg/combine}{\textsf{combine}}
is an elaborate solution to combine several documents into one.
\end{itemize}
%
See also the CTAN topic \href{http://ctan.org/topic/subdocs}{\textsf{subdocs}}
for further related packages.
The present package differs from the above solutions in that
a document structure constructed with the conventional |\include| mechanism
just needs two extra commands at the top of every file
such that all constituent files can be compiled individually.

%%%%%%%%%%%%%%%%%%%%%%%%%%%%%%%%%%%%%%%%%%%%%%%%%%%%%%%%%%%%%%%%%%%%%%%%%%%%%%%%
%\subsection{Feature Suggestions}
%
%The following is a list of features which may be useful for future
%versions of this package:
%%
%\begin{itemize}
%\item
%\ldots
%\end{itemize}

%%%%%%%%%%%%%%%%%%%%%%%%%%%%%%%%%%%%%%%%%%%%%%%%%%%%%%%%%%%%%%%%%%%%%%%%%%%%%%%%
\subsection{Revision History}

%%%%%%%%%%%%%%%%%%%%%%%%%%%%%%%%%%%%%%%%
\paragraph{v2.0:} 2018/12/30

\begin{itemize}
\item
immediate forward processing
\item
added |\childdocby| mechanism
\item
manual restructured
\end{itemize}

%%%%%%%%%%%%%%%%%%%%%%%%%%%%%%%%%%%%%%%%
\paragraph{v1.6:} 2018/01/17

\begin{itemize}
\item
application for development of include files
\item
corrections to manual
\end{itemize}

%%%%%%%%%%%%%%%%%%%%%%%%%%%%%%%%%%%%%%%%
\paragraph{v1.5:} 2017/05/21

\begin{itemize}
\item
more complete structuring introduced
\item
|\childdocof| introduced
\item
|\childdoc| renamed to |\childdocmain|
\item
|\childredirect| renamed to |\childdocforward| and |\childdocforwardprefix|
and functionality expanded
\end{itemize}

%%%%%%%%%%%%%%%%%%%%%%%%%%%%%%%%%%%%%%%%
\paragraph{v1.0:} 2017/04/27

\begin{itemize}
\item
manual and install package
\item
first version published on CTAN
\end{itemize}

%%%%%%%%%%%%%%%%%%%%%%%%%%%%%%%%%%%%%%%%
\paragraph{v0.6:} 2017/04/26

\begin{itemize}
\item
redirection mechanism added
\end{itemize}

%%%%%%%%%%%%%%%%%%%%%%%%%%%%%%%%%%%%%%%%
\paragraph{v0.5:} 2017/04/26

\begin{itemize}
\item
functionality in definition file
\end{itemize}


%%%%%%%%%%%%%%%%%%%%%%%%%%%%%%%%%%%%%%%%%%%%%%%%%%%%%%%%%%%%%%%%%%%%%%%%%%%%%%%%
%%%%%%%%%%%%%%%%%%%%%%%%%%%%%%%%%%%%%%%%%%%%%%%%%%%%%%%%%%%%%%%%%%%%%%%%%%%%%%%%
%%%%%%%%%%%%%%%%%%%%%%%%%%%%%%%%%%%%%%%%%%%%%%%%%%%%%%%%%%%%%%%%%%%%%%%%%%%%%%%%
\appendix

\settowidth\MacroIndent{\rmfamily\scriptsize 000\ }

 \DocInput{childdoc.dtx}

\end{document}
%</driver>
% \fi
%
% %%%%%%%%%%%%%%%%%%%%%%%%%%%%%%%%%%%%%%%%%%%%%%%%%%%%%%%%%%%%%%%%%%%%%%%%%%%%%%
% %%%%%%%%%%%%%%%%%%%%%%%%%%%%%%%%%%%%%%%%%%%%%%%%%%%%%%%%%%%%%%%%%%%%%%%%%%%%%%
% \section{Sample}
%\iffalse
%<*samplemain>
%\fi
%
% The following presents a sample document
% with two chapters, two parts, a title page,
% a compile flag as well as three forwarding files to set the flag.
% It consists of eight |.tex| files:
% \begin{center}
% \begin{tabular}{ll}
% |cdocsamp.tex|&main file\\
% |cdocsch1.tex|&include file for chapter 1\\
% |cdocsch2.tex|&include file for chapter 2\\
% |cdocspt3.tex|&include file for part 3\\
% |cdocspt4.tex|&include file for part 4\\
% |cdocsdrf.tex|&forwarding file for main file in draft mode\\
% |cdocsfi1.tex|&forwarding file for final version of chapter 1\\
% |cdocsfi2.tex|&forwarding file for final version of chapter 2\\
% \end{tabular}
% \end{center}
% Each of the eight files can be compiled directly by the \LaTeX{} compiler.
%
% %%%%%%%%%%%%%%%%%%%%%%%%%%%%%%%%%%%%%%
% \paragraph{Main File.}
%
% The main file is called |cdocsamp.tex|.
%
% Load the \textsf{childdoc} definitions and
% declare the filename for the main document:
%    \begin{macrocode}
\input{childdoc.def}
\childdocmain{}
%    \end{macrocode}

% Optional override for |\version| flag:
%    \begin{macrocode}
%%\ifchilddoc\else\providecommand{\version}{draft}\fi
%    \end{macrocode}

% Define the default values for the |\version| flag
% (|final| for the main file and |draft| for childs):
%    \begin{macrocode}
\ifchilddoc
\providecommand{\version}{draft}
\else
\providecommand{\version}{final}
\fi
%    \end{macrocode}

% Load the standard document class:
%    \begin{macrocode}
\documentclass[12pt]{article}
%    \end{macrocode}

% Start the document body:
%    \begin{macrocode}
\begin{document}
%    \end{macrocode}

% Declare a title page.
% Print title, part of document being processed and version flag:
%    \begin{macrocode}
\addtocounter{page}{-1}
\begin{center}
{\LARGE\bfseries{}childdoc example\par}
\vspace{1cm}
\ifchilddoc
\ifchilddocmanual part\else chapter\fi:
`\childdocname' of `\childdocjob'\par
\else
main document: `\childdocjob'\par
\fi
version: \version\par
\end{center}
\newpage
%    \end{macrocode}

% Manually include selected file,
% otherwise process as usual:
%    \begin{macrocode}
\ifchilddocmanual
\section*{part `\childdocname'}
\input{\childdocname}
\else
%    \end{macrocode}

% Include the two chapters:
%    \begin{macrocode}
\include{cdocsch1}
\include{cdocsch2}
%    \end{macrocode}

% Include the two parts unless only chapters should be displayed:
%    \begin{macrocode}
\ifchilddoc\else
\section{part three}
\input{cdocspt3}
\section{part four}
\input{cdocspt4}
\fi
%    \end{macrocode}

% Process as usual until here:
%    \begin{macrocode}
\fi
%    \end{macrocode}

% End of document body:
%    \begin{macrocode}
\end{document}
%    \end{macrocode}
%\iffalse
%</samplemain>
%\fi
%
% %%%%%%%%%%%%%%%%%%%%%%%%%%%%%%%%%%%%%%
% \paragraph{Chapter Include Files.}
%
% The include files are called |cdocsch1.tex| and |cdocsch2.tex|.
%
%\iffalse
%<*samplechap1|samplechap2>
%\fi

% Optional override for |\version| flag:
%    \begin{macrocode}
%%\providecommand{\version}{final}
%    \end{macrocode}

% Include the main document:
%    \begin{macrocode}
\input{childdoc.def}
\childdocof{cdocsamp}
%    \end{macrocode}

%\iffalse
%</samplechap1|samplechap2>
%\fi
%
%\iffalse
%<*samplechap1>
%\fi
% Some text for chapter 1:
%    \begin{macrocode}
\section{one}
some text in chapter one
%    \end{macrocode}

%\iffalse
%</samplechap1>
%\fi
% Some text for chapter 2:
%\iffalse
%<*samplechap2>
%\fi
%    \begin{macrocode}
\section{two}
more text in chapter two
%    \end{macrocode}

%\iffalse
%</samplechap2>
%\fi
%
% %%%%%%%%%%%%%%%%%%%%%%%%%%%%%%%%%%%%%%
% \paragraph{Part Include Files.}
%
% The include files are called |cdocspt3.tex| and |cdocspt4.tex|.
%
%\iffalse
%<*samplepart3|samplepart4>
%\fi

% Optional override for |\version| flag:
%    \begin{macrocode}
%%\providecommand{\version}{final}
%    \end{macrocode}

% Include the main document:
%    \begin{macrocode}
\input{childdoc.def}
\childdocby{cdocsamp}
%    \end{macrocode}

%\iffalse
%</samplepart3|samplepart4>
%\fi
%
%\iffalse
%<*samplepart3>
%\fi
% Some text for part 3:
%    \begin{macrocode}
some text in part three
%    \end{macrocode}

%\iffalse
%</samplepart3>
%\fi
% Some text for part 4:
%\iffalse
%<*samplepart4>
%\fi
%    \begin{macrocode}
more text in part four
%    \end{macrocode}

%\iffalse
%</samplepart4>
%\fi
%
% %%%%%%%%%%%%%%%%%%%%%%%%%%%%%%%%%%%%%%
% \paragraph{Forwarding for a Complete Draft.}
%
% The following forwarding file |cdocsdrf.tex|
% compiles the main document in draft mode:
%\iffalse
%<*sampledraft>
%\fi
%    \begin{macrocode}
\def\version{draft}
\input{childdoc.def}
\childdocforward{cdocsamp}
%    \end{macrocode}

%\iffalse
%</sampledraft>
%\fi
%
% %%%%%%%%%%%%%%%%%%%%%%%%%%%%%%%%%%%%%%
% \paragraph{Forwarding for Final Version of the Chapters.}
%
% The following forwarding files |cdocsfn1.tex| and |cdocsfn2.tex|
% (with identical content)
% compile the final versions of the child documents
% |cdocsch1.tex| and |cdocsch2.tex|, respectively:
%\iffalse
%<*samplefinal>
%\fi
%    \begin{macrocode}
\def\version{final}
\input{childdoc.def}
\childdocforwardprefix[cdocsamp]{cdocsfn}{cdocsch}
%    \end{macrocode}

%\iffalse
%</samplefinal>
%\fi
%
% %%%%%%%%%%%%%%%%%%%%%%%%%%%%%%%%%%%%%%
% \paragraph{Command Line Processing.}
%
% The following three command lines generate the output files
% |cdocscld|, |cdocscl1| and |cdocscl2|
% which should be identical to
% |cdocsdrf|, |cdocsch1| and |cdocsfn2|, respectively:
% \begin{center}
% \begin{tabular}{l}
% |latex -jobname cdocscld \|\\
% |  "\def\version{draft}\input{childdoc.def}\childdocforward{cdocsamp}"|\\
% |latex -jobname cdocscl1 \|\\
% |  "\input{childdoc.def}\childdocforward[cdocsamp]{cdocsch1}"|\\
% |latex -jobname cdocscl2 \|\\
% |  "\def\version{final}\input{childdoc.def}\childdocforward{cdocsch2}"|
% \end{tabular}
% \end{center}
% Note that the trailing backslash on each first line
% merely continues the input to the second line
% (for convenient cut ant paste).
% Furthermore, the command |latex| can be replaced by any
% of its alternative versions such as |pdflatex|.
%
% %%%%%%%%%%%%%%%%%%%%%%%%%%%%%%%%%%%%%%%%%%%%%%%%%%%%%%%%%%%%%%%%%%%%%%%%%%%%%%
% %%%%%%%%%%%%%%%%%%%%%%%%%%%%%%%%%%%%%%%%%%%%%%%%%%%%%%%%%%%%%%%%%%%%%%%%%%%%%%
% \section{Implementation}
%\iffalse
%<*package>
%\fi
%
% This section describes the definitions file |childdoc.def|.

% The definitions cannot be loaded using |\usepackage| or |\RequirePackage|
% which has a mechanism to prevent loading a style file more than once.
% When loading the definitions by means of |\input|
% multiple instances have to be prevented manually:
%\iffalse
%This code needs to be before the `\ProvidesFile' directive
%which is defined at the beginning of this file.
%Therefore it is also placed there and commented out here.
%</package>
%<*discard>
%\fi
%    \begin{macrocode}
\ifdefined\childdocmain\endinput\fi
%    \end{macrocode}
%\iffalse
%</discard>
%<*package>
%\fi
%
% \macro{\ifchilddoc}
% \macro{\ifchilddocmanual}
% The conditional |\ifchilddoc| tells whether a
% child (true) or main (false) document is being compiled.
% The conditional |\ifchilddocmanual| tells whether
% the |\includeonly| mechanism is used (false) or
% the selection of child files must be performed manually (true).
% The definitions initialise to false:
%    \begin{macrocode}
\newif\ifchilddoc
\newif\ifchilddocmanual
%    \end{macrocode}

% \macro{\childdocname}
% \macro{\childdocjob}
% The macro |\childdocname| stores the name of the main document
% to be compiled. The macro |\childdocjob| stores the name of
% the document on which the \LaTeX{} compiler was originally invoked.
% The content of |\jobname| cannot be compared
% to filenames specified in the source due to different catcodes.
% The following code rescans |\jobname|, stores the result
% in |\childdocname| and saves a copy in |\childdocjob|:
%    \begin{macrocode}
\edef\childdocname{\scantokens\expandafter{\jobname\noexpand}}
\let\childdocjob\childdocname
%    \end{macrocode}

% \macro{\childdocdisable}
% The macro |\childdocdisable| prevents the main file
% from being processed more than once.
% At this stage, the main document command |\childdocmain|
% is assumed to be called once again where it should do nothing.
% Any subsequent call to it should prevent
% a secondary processing of the main document
% It overwrites the forwarding commands
% |\childdocof| and |\childdocforward|
% with empty macros to prevent further inclusions of the main document:
%    \begin{macrocode}
\newcommand{\childdocdisable}
{
  \renewcommand{\childdocmain}[1]{\renewcommand{\childdocmain}[1]{\endinput}}
  \renewcommand{\childdocof}[1]{}
  \renewcommand{\childdocby}[2][]{}
  \renewcommand{\childdocforward}[2][]{}
  \renewcommand{\childdocdisable}{}
}
%    \end{macrocode}

% \macro{\childdocmain}
% The macro |\childdocmain| is to be called at the top of the main file
% with nothing or the main filename (without extension) as argument.
% First, it breaks loops.
% If the argument is not empty and does not match |\childdocname|
% (which is set by the first inclusion of |childdoc.def|),
% |\ifchilddoc| is set to true, |\includeonly| is applied to the child file
% and |\jobname| is set to the main file
% (for proper handling of |.aux| files):
%    \begin{macrocode}
\newcommand{\childdocmain}[1]
{
  \childdocdisable\childdocmain{}
  \if?#1?\else
    \begingroup
      \def\childdoctmp{#1}
      \ifx\childdoctmp\childdocname
        \def\childdoctmp{}
      \else
        \def\childdoctmp
        {
          \childdoctrue
          \includeonly{\childdocname}
          \def\childdocjob{#1}
          \def\jobname{#1}
        }
      \fi
      \expandafter
    \endgroup
    \childdoctmp
  \fi
}
%    \end{macrocode}

% \macro{\childdocof}
% The command |\childdocof| redirects
% compilation to the main file |#1|.
%    \begin{macrocode}
\newcommand{\childdocof}[1]
{
  \childdocdisable
  \childdoctrue
  \includeonly{\childdocname}
  \def\jobname{#1}
  \def\childdocjob{#1}
  \input{#1}
}
%    \end{macrocode}

% \macro{\childdocby}
% The command |\childdocby| ....
%    \begin{macrocode}
\newcommand{\childdocby}[2][]
{
  \childdocdisable
  \childdoctrue
  \childdocmanualtrue
  \if?#1?\else
    \def\jobname{#2}
  \fi
  \def\childdocjob{#2}
  \input{#2}
  \endinput
}
%    \end{macrocode}

% \macro{\childdocforward}
% The command |\childdocforward| redirects
% compilation to the main file or
% (if the optional argument is given) a child file.
% Parameters are set as if the main file
% or a child file starting with |\childdocof| was compiled.
% Then compilation is handed over to the main file:
%    \begin{macrocode}
\newcommand{\childdocforward}[2][]
{
  \begingroup
    \if?#1?
      \def\childdoctmp
      {
        \def\childdocname{#2}
        \def\childdocjob{#2}
        \def\jobname{#2}
        \input{#2}
        \endinput
      }
    \else
      \def\childdoctmp
      {
        \childdocdisable
        \def\childdocname{#2}
        \childdoctrue
        \includeonly{#2}
        \def\childdocjob{#1}
        \def\jobname{#1}
        \input{#1}
        \endinput
      }
    \fi
    \expandafter
  \endgroup
  \childdoctmp
}
%    \end{macrocode}

% \macro{\childdocforwardprefix}
% The command |\childdocforwardprefix| redirects
% compilation to the main or a child file by means of a pattern.
% The prefix |#1| in the current filename is replaced by |#2|
% and the suffix of the current filename is kept
% (it is assumed that the filename does not contain the substring `|~~~|'
% which is used as a delimiter).
% Compilation is handed over to the new file by |\childdocforward|:
%    \begin{macrocode}
\newcommand{\childdocforwardprefix}[3][]
{
  \begingroup
    \def\childdocextract #2##1~~~{\def\childdoctmp{\childdocforward[#1]{#3##1}}}
    \expandafter\childdocextract\childdocname~~~
    \expandafter
  \endgroup
  \childdoctmp
}
%    \end{macrocode}

% \macro{\childdoc}
% The deprecated macro |\childdoc| is a legacy version of |\childdocmain|:
%    \begin{macrocode}
\newcommand{\childdoc}{\childdocmain}
%    \end{macrocode}

% \macro{\childdocredirect}
% The deprecated macro |\childdocredirect| is a legacy version
% of |\childdocforward| and |\childdocforwardprefix|:
%    \begin{macrocode}
\newcommand{\childdocredirect}[2][]
{
  \begingroup
    \if?#1?
      \def\childdoctmp{\childdocforward{#2}}
    \else
      \def\childdoctmp{\childdocforwardprefix{#1}{#2}}
    \fi
    \expandafter
  \endgroup
  \childdoctmp
}
%    \end{macrocode}

%\iffalse
%</package>
%\fi
%
\endinput

\childdocof{cdocsamp}
%    \end{macrocode}

%\iffalse
%</samplechap1|samplechap2>
%\fi
%
%\iffalse
%<*samplechap1>
%\fi
% Some text for chapter 1:
%    \begin{macrocode}
\section{one}
some text in chapter one
%    \end{macrocode}

%\iffalse
%</samplechap1>
%\fi
% Some text for chapter 2:
%\iffalse
%<*samplechap2>
%\fi
%    \begin{macrocode}
\section{two}
more text in chapter two
%    \end{macrocode}

%\iffalse
%</samplechap2>
%\fi
%
% %%%%%%%%%%%%%%%%%%%%%%%%%%%%%%%%%%%%%%
% \paragraph{Part Include Files.}
%
% The include files are called |cdocspt3.tex| and |cdocspt4.tex|.
%
%\iffalse
%<*samplepart3|samplepart4>
%\fi

% Optional override for |\version| flag:
%    \begin{macrocode}
%%\providecommand{\version}{final}
%    \end{macrocode}

% Include the main document:
%    \begin{macrocode}
% \iffalse
%
% childdoc.dtx Copyright (C) 2017-2018 Niklas Beisert
%
% This work may be distributed and/or modified under the
% conditions of the LaTeX Project Public License, either version 1.3
% of this license or (at your option) any later version.
% The latest version of this license is in
%   http://www.latex-project.org/lppl.txt
% and version 1.3 or later is part of all distributions of LaTeX
% version 2005/12/01 or later.
%
% This work has the LPPL maintenance status `maintained'.
%
% The Current Maintainer of this work is Niklas Beisert.
%
% This work consists of the files childdoc.dtx and childdoc.ins
% and the derived files childdoc.def and cdocsamp.tex with
% cdocsch1.tex, cdocsch2.tex, cdocsdrf.tex, cdocsfn1.tex, cdocsfn2.tex.
%
%<package>\ifdefined\childdocmain\endinput\fi
%<package>\ProvidesFile{childdoc.def}[2018/12/30 v2.0 child document driver]
%<samplemain>\ProvidesFile{cdocsamp.tex}[2018/12/30 v2.0 sample for childdoc]
%<*driver>
%\ProvidesFile{childdoc.drv}[2018/12/30 v2.0 childdoc reference manual file]
\PassOptionsToClass{10pt,a4paper}{article}
\documentclass{ltxdoc}

\usepackage[margin=35mm]{geometry}
\usepackage{hyperref}
\usepackage{hyperxmp}
\usepackage[usenames]{color}

\hypersetup{colorlinks=true}
\hypersetup{pdfstartview=FitH}
\hypersetup{pdfpagemode=UseNone}
\hypersetup{pdfsource={}}
\hypersetup{pdflang={en-UK}}
\hypersetup{pdfcopyright={Copyright 2017-2018 Niklas Beisert.
  This work may be distributed and/or modified under the
  conditions of the LaTeX Project Public License, either version 1.3
  of this license or (at your option) any later version.}}
\hypersetup{pdflicenseurl={http://www.latex-project.org/lppl.txt}}
\hypersetup{pdfcontactaddress={ETH Zurich, ITP, HIT K,
  Wolfgang-Pauli-Strasse 27}}
\hypersetup{pdfcontactpostcode={8093}}
\hypersetup{pdfcontactcity={Zurich}}
\hypersetup{pdfcontactcountry={Switzerland}}
\hypersetup{pdfcontactemail={nbeisert@itp.phys.ethz.ch}}
\hypersetup{pdfcontacturl={http://people.phys.ethz.ch/\xmptilde nbeisert/}}

\newcommand{\secref}[1]{\hyperref[#1]{section \ref*{#1}}}

\parskip1ex
\parindent0pt
\let\olditemize\itemize
\def\itemize{\olditemize\parskip0pt}

\begin{document}

\title{The \textsf{childdoc} Package}
\hypersetup{pdftitle={The childdoc Package}}
\author{Niklas Beisert\\[2ex]
  Institut f\"ur Theoretische Physik\\
  Eidgen\"ossische Technische Hochschule Z\"urich\\
  Wolfgang-Pauli-Strasse 27, 8093 Z\"urich, Switzerland\\[1ex]
  \href{mailto:nbeisert@itp.phys.ethz.ch}
  {\texttt{nbeisert@itp.phys.ethz.ch}}}
\hypersetup{pdfauthor={Niklas Beisert}}
\hypersetup{pdfsubject={Manual for the LaTeX2e Package childdoc}}
\date{30 December 2018, \textsf{v2.0}}
\maketitle

\begin{abstract}\noindent
\textsf{childdoc} is a \LaTeXe{} package
that enables the direct compilation
of document sections included by |\include|
to individual files.
\end{abstract}

\begingroup
\parskip0ex
\tableofcontents
\endgroup

%%%%%%%%%%%%%%%%%%%%%%%%%%%%%%%%%%%%%%%%%%%%%%%%%%%%%%%%%%%%%%%%%%%%%%%%%%%%%%%%
%%%%%%%%%%%%%%%%%%%%%%%%%%%%%%%%%%%%%%%%%%%%%%%%%%%%%%%%%%%%%%%%%%%%%%%%%%%%%%%%
\section{Introduction}

\LaTeX{} provides a mechanism to structure a large document (such as a book)
into a main file and several child files (containing the chapters)
using the |\include| command.
This mechanism is beneficial for documents
which span hundreds of pages in order to
make the source file(s) more manageable.
Moreover, compilation can be restricted to
selected child files by means of the |\includeonly| command.
The latter feature can be used to reduce the compilation time while editing
(this was significantly more useful in the earlier days of \LaTeX{})
or to generate a smaller document which is easier to navigate.
Another application of |\includeonly| is to generate
documents consisting of selected parts of the complete document.

However, there are a few drawbacks of the plain |\include| mechanism:
\begin{itemize}
\item
The child files cannot be compiled on their own,
they can only be compiled via the main file.
A naive editing environment
(such as a text editor with an option
to have the current file processed by \LaTeX)
may require one to switch to the main file before compiling;
attempting to compile the child file produces errors.
\item
The main file must be modified (each time)
to adjust the |\includeonly| command
to the present needs. This easily leaves the main file in a messy state.
\item
The generated document will always carry the filename
of the main document. This is inconvenient if
several child files are to be compiled and
to be kept for distribution.
\end{itemize}

The present package provides a simple interface
to make child files individually compilable by \LaTeX{}.
Compiling a child file then has the same effect as compiling
the main file with an |\includeonly| command
to select the appropriate child.
Moreover the generated document will carry the name of the child
rather than the main file.
This resolves all three above issues.

This feature is meant to make the editing of books,
thesis documents and lecture notes somewhat more convenient.
However, the package can also be used efficiently for
composing a series of documents (such as exercise sheets)
which are typically distributed individually.
It then assists the author in generating the individual documents
(potentially in different versions)
as well as a document containing the collected series.
Another application is in developing style files
or other kinds of included material
where compilation of the style file could redirect
to a sample or test file.

%%%%%%%%%%%%%%%%%%%%%%%%%%%%%%%%%%%%%%%%%%%%%%%%%%%%%%%%%%%%%%%%%%%%%%%%%%%%%%%%
%%%%%%%%%%%%%%%%%%%%%%%%%%%%%%%%%%%%%%%%%%%%%%%%%%%%%%%%%%%%%%%%%%%%%%%%%%%%%%%%
\section{Usage}

First of all, the package \textsf{childdoc} is \emph{not} a standard
\LaTeXe{} |.sty| style file! Therefore it needs to be invoked in
a non-standard way.

%%%%%%%%%%%%%%%%%%%%%%%%%%%%%%%%%%%%%%%%%%%%%%%%%%%%%%%%%%%%%%%%%%%%%%%%%%%%%%%%
\subsection{Included Files}
\label{sec:include}

%%%%%%%%%%%%%%%%%%%%%%%%%%%%%%%%%%%%%%%%
\DescribeMacro{\childdocmain}
To use the package, add the commands
\begin{center}
\begin{tabular}{l}
|\input{childdoc.def}|\\
|\childdocmain{}|\\
\end{tabular}
\end{center}
at the very top of the main \LaTeX{} file,
in particular \emph{before} the |\documentclass| statement!
The argument of |\childdocmain| should be left empty
(but it must be present).

%%%%%%%%%%%%%%%%%%%%%%%%%%%%%%%%%%%%%%%%
\DescribeMacro{\childdocof}
Furthermore, add the commands
\begin{center}
\begin{tabular}{l}
|\input{childdoc.def}|\\
|\childdocof{|\textit{main}|}|\\
\end{tabular}
\end{center}
at the top of every child file \textit{child}
which is included by |\include{|\textit{child}|}|
from within the main file
(or at least for those files to be compiled individually).
The argument \textit{main} must be the filename of the main file.

There are a couple of
considerations in setting up the main and child documents:

%%%%%%%%%%%%%%%%%%%%%%%%%%%%%%%%%%%%%%%%
\paragraph{Restrictions.}

Please note the following restrictions:
\begin{itemize}
\item
|\childdocmain| must be called with one argument \textit{main}
to ensure compatibility with earlier version of the package.
It must either be empty (|\childdocmain{}|)
or precisely match the filename of the main file in which it is specified.
See \secref{sec:detection} for further information.
\item
The filename \textit{main} must be specified without the |.tex| extension.
\item
The filename \textit{main} is case sensitive
(even in case-insensitive file systems)
due to internal string comparison.
\item
The argument \textit{main} should be fully expanded, it cannot be a macro.
\item
Subdirectories and special characters should be avoided in filenames.
\item
The command |\childdocmain{|\textit{main}|}| must be followed by a whitespace.
It should not be followed immediately by another command
or by a comment mark `|%|'.
This is because the \TeX{} parser reads the token immediately following
the argument of |\childdocmain| and puts it
at the beginning of every child section;
however, a white\-space is ignored.
\end{itemize}

%%%%%%%%%%%%%%%%%%%%%%%%%%%%%%%%%%%%%%%%
\paragraph{Content of Main File.}

It is advisable to place all content in the child files included by |\include|.
Any output contained in the main file will appear in all child documents
unless suppressed manually;
it cannot be suppressed automatically by the |\includeonly| directive
and thus should normally be avoided.
A method to include some content in the main file
by means of conditional processing is described in \secref{sec:conditional}.

%%%%%%%%%%%%%%%%%%%%%%%%%%%%%%%%%%%%%%%%
\paragraph{Page Numbering.}

When only a part of the document is compiled,
the appropriate numbering of pages
(as well as other status parameters)
is determined from the |.aux| files.
The latter contain information from previous passes.
However this information needs to propagate through
all intermediate child documents.
Therefore the page numbering in child documents may well
be inconsistent until the complete document is compiled at least once.

A useful (if unconventional) way to always ensure a consistent
page numbering is to restart the numbering in each child document
and denote the pages by `\textit{child}|.|\textit{page}'
where \textit{child} represents the chapter/section number of the child file.
This can be achieved by the command
|\numberwithin{page}{|\textit{child}|}|
of the \textsf{amsmath} package
where \textit{child} can be |chapter| or |section|
depending on the chosen structuring.
Alternatively, one can modify the macro |\thepage| appropriately
and reset the counter |page| at the start of each child file.

%%%%%%%%%%%%%%%%%%%%%%%%%%%%%%%%%%%%%%%%%%%%%%%%%%%%%%%%%%%%%%%%%%%%%%%%%%%%%%%%
\subsection{Conditional Processing}
\label{sec:conditional}

The package provides a mechanism to compile different versions
of a document. To customise the versions further some conditional processing
can come in handy to distinguish which version is being compiled.
The package provides two macros to describe the compilation context:

%%%%%%%%%%%%%%%%%%%%%%%%%%%%%%%%%%%%%%%%
\DescribeMacro{\ifchilddoc}
The conditional |\ifchilddoc| distinguishes between the compilation of
child documents and the main document:
%
\begin{center}
|\ifchilddoc |\textit{child-code}| |[|\||else |\textit{main-code}]| \||fi|
\end{center}

%%%%%%%%%%%%%%%%%%%%%%%%%%%%%%%%%%%%%%%%
\DescribeMacro{\childdocname}
\DescribeMacro{\childdocjob}
The macro |\childdocname| contains the filename (without extension)
of the main or child file being processed.
Note that |\childdocjob| will always contain the name of the main file.

%%%%%%%%%%%%%%%%%%%%%%%%%%%%%%%%%%%%%%%%
\paragraph{Title Page.}

Conditional processing can be used to include a title or banner page
in the main document when proper precautions are taken.
Importantly, the code in the main file should ensure that the page counter
(as well as other status parameters which are stored in the |.aux| files)
takes the same value after the conditional processing.
Otherwise the page numbers may take divergent values
depending on which part is compiled.

For example, a title page could be declared by:
%
\begin{center}
\begin{tabular}{l}
|\ifchilddoc\||else|\\
|\addtocounter{page}{-1}|\\
\textit{code for title page}\\
|\newpage|\\
|\||fi|
\end{tabular}
\end{center}
%
A banner page for the child documents can be generated by:
%
\begin{center}
\begin{tabular}{l}
|\ifchilddoc|\\
|\addtocounter{page}{-1}|\\
\textit{code for banner page}\\
|\newpage|\\
|\||fi|
\end{tabular}
\end{center}
%
Here one could write a message such as:
\begin{center}
|This is the part \childdocname{} of \childdocjob{}.|
\end{center}

%%%%%%%%%%%%%%%%%%%%%%%%%%%%%%%%%%%%%%%%%%%%%%%%%%%%%%%%%%%%%%%%%%%%%%%%%%%%%%%%
\subsection{Flags}
\label{sec:flags}

The package makes it easy to generate different versions
of the main or child documents.
To this end compilation flags can be defined
and assigned different default values.
They will be particularly useful in conjunction
with the forwarding mechanism described in \secref{sec:forward}.

For example, it may be useful to have a flag |\version|
which can be set to |draft| or |final|.
The document source will contain some conditional code
depending on the value of |\version|.
Suppose further, the flag should default to |final| for the main file
and to |draft| for child files
which is a natural assignment for editing the document.
This is achieved by placing the following code
in the preamble of the main document
(below the |\childdocmain| directive):
%
\begin{center}
\begin{tabular}{l}
|\ifchilddoc|\\
|\providecommand{\version}{draft}|\\
|\||else|\\
|\providecommand{\version}{final}|\\
|\||fi|
\end{tabular}
\end{center}
%
The definition by |\providecommand| makes sure
that previous definitions are not overwritten.
Further statements |\providecommand{\version}{...}|
can thus be added before the above code to override it.

For the main file, one might add a line
(between |\childdocmain| and the above block)
%
\begin{center}
|%\ifchilddoc\||else\providecommand{\version}{draft}\||fi|
\end{center}
%
which can be uncommented to produce a draft version.
Likewise one can add a line to the very top of a child file
(above the |\childdocof{|\textit{main}|}| directive)
%
\begin{center}
|%\providecommand{\version}{final}|
\end{center}
%
which can be uncommented to produce the final version of this child document.

%%%%%%%%%%%%%%%%%%%%%%%%%%%%%%%%%%%%%%%%%%%%%%%%%%%%%%%%%%%%%%%%%%%%%%%%%%%%%%%%
\subsection{Forwarding}
\label{sec:forward}

Different versions of the main or child documents
using compilation flags as described in \secref{sec:flags}
can be (permanently) stored in different files
for convenient compilation, viewing and distribution.
To this end, the package defines a command
to pass on compilation to a different file:

%%%%%%%%%%%%%%%%%%%%%%%%%%%%%%%%%%%%%%%%
\DescribeMacro{\childdocforward}
The command |\childdocforward| redirects processing to
another source file:
%
\begin{center}
\begin{tabular}{l}
|\input{childdoc.def}|\\
|\childdocforward[|\textit{main}|]{|\textit{dest}|}|\\
\end{tabular}
\end{center}
%
The argument \textit{dest} is the destination file
(without extension).
It should be the main file or one of the child files.
Note that further \textsf{childdoc} directives
such as |\childdocof| and |\childdocforward|
in the indicated file will be processed in this form.
The optional argument \textit{main}
passes on directly to the main file \textit{main}
while pretending to compile the child \textit{dest}.
This form behaves as if \textit{dest}
issues |\childdocof{|\textit{main}|}| right away,
and no further \textsf{childdoc} directives will be processed.

%%%%%%%%%%%%%%%%%%%%%%%%%%%%%%%%%%%%%%%%
\DescribeMacro{\...prefix}
In the alternative form |\childdocforwardprefix|,
%
\begin{center}
\begin{tabular}{l}
|\input{childdoc.def}|\\
|\childdocforwardprefix[|\textit{main}|]{|\textit{prefix}|}{|\textit{dest}|}|
\end{tabular}
\end{center}
%
the destination file is determined by a pattern
depending on the current file:
To make this work, the current file must be called
`{\textit{prefix}\hspace{0.2em}\textit{suffix}}'
with \textit{prefix} matching precisely the argument.
Processing is then passed on to the file
`{\textit{dest}\hspace{0.2em}\textit{suffix}}'.
Surely, the same effect is achieved by
directly specifying the
argument `{\textit{dest}\hspace{0.2em}\textit{suffix}}'
in the first form.
However, that requires to set up a different file
for each child. With the alternative form of the command
all these files can have exactly the same content
which simplifies setting them up and maintaining them.

For example, the following file |draft.tex|
with a compilation flag |\version| as described in \secref{sec:flags}
compiles the main document as a draft:
%
\begin{center}
\begin{tabular}{l}
|\def\version{draft}|\\
|\input{childdoc.def}|\\
|\childdocforward{|\textit{main}|}|
\end{tabular}
\end{center}
%
Likewise, the following files |final|\textit{nn}|.tex|
compile the final version of the child document
|child|\textit{nn}|.tex|:
%
\begin{center}
\begin{tabular}{l}
|\def\version{final}|\\
|\input{childdoc.def}|\\
|\childdocforwardprefix{final}{child}|
\end{tabular}
\end{center}
%

Note that when several versions of a main file and/or of each child file
are to be generated, it may be convenient to set up a |Makefile| or
shell script to automatise the process.

%%%%%%%%%%%%%%%%%%%%%%%%%%%%%%%%%%%%%%%%%%%%%%%%%%%%%%%%%%%%%%%%%%%%%%%%%%%%%%%%
\subsection{Command Line Processing}
\label{sec:commandline}

The effect of redirection files can also be achieved by invoking
the \LaTeX{} compiler with a more elaborate command line.
Most conveniently this should be done as part
of a shell script or a |Makefile|.

When using \textsf{childdoc} in the main file, the following
command lines effectively perform a redirection
(note that depending on the shell being used,
backslashes may have to be doubled: `|\|' $\to$ `|\\|'):
%
\begin{center}
|... -jobname "|\textit{target}|" |\\|"|[\textit{flags}]%
|\input{childdoc.def}\childdocforward[|\textit{main}|]{|\textit{dest}|}"|
\end{center}
%
Here \textit{target} is the name of the output file,
\textit{main} is the name of the main file
and \textit{dest} is the name of the main or child file to be processed
(all filenames without extensions).
The optional argument \textit{main} can be omitted
if \textit{main} matches \textit{dest}.
Optionally, compilation \textit{flags} can be defined via |\def| commands.
This command line makes the \TeX{} engine believe
it is compiling the file \textit{target}
whose content is specified as the latter parameter.
The provided code then forwards the processing to
\textit{main} or \textit{dest} as described in \secref{sec:forward}.

%%%%%%%%%%%%%%%%%%%%%%%%%%%%%%%%%%%%%%%%%%%%%%%%%%%%%%%%%%%%%%%%%%%%%%%%%%%%%%%%
\subsection{Include by Input}
\label{sec:input}

Including child documents by |\include| has some restrictions by design.
Most notably, the content of a child document always occupies
its own set of pages; pages cannot be shared between child documents.
Usually, this behaviour makes perfect sense
because each child document contain an essential part of the document.
However, in some situations it may be desirable to compose
a document from a collection of parts
without having mandatory page breaks between then.
For this case, the package
provides a mechanism to include parts
by |\input| which can also be processed individually.
However, by construction this mechanism
requires manual handling of the content to be output.

%%%%%%%%%%%%%%%%%%%%%%%%%%%%%%%%%%%%%%%%
\DescribeMacro{\ifchilddocmanual}
The main file should be prepared as usual, see \secref{sec:include}.
However, the document body must make a distinction
between processing of an individual part and of the main document, e.g.:
%
\begin{center}
\begin{tabular}{l}
|\ifchilddocmanual|\\
|\input{\childdocname}|\\
|\||else|\\
\textit{document body with }|\input{|\textit{part}|}|\\
|\||fi|
\end{tabular}
\end{center}
%
The conditional |\ifchilddocmanual| is true whenever
a part to be included by |\input| is being compiled,
and the name of the part is stored in |\childdocname|.

%%%%%%%%%%%%%%%%%%%%%%%%%%%%%%%%%%%%%%%%
\DescribeMacro{\childdocby}
Each part to be included by |\input| should start with:
%
\begin{center}
\begin{tabular}{l}
|\input{childdoc.def}|\\
|\childdocby{|\textit{main}|}|\\
\end{tabular}
\end{center}
%
The directive |\childdocby| is similar to |\childdocof|
described in \secref{sec:include},
but the subsequent selection of content must be done manually.
To that end, both |\ifchilddoc| and |\ifchilddocmanual|
will be true upon processing of a part,
and the name of the part is stored in |\childdocname|.
Note that |\jobname| will be set to the filename of the current part
so that each part receives an individual |.aux| file
that does not interfere with the |.aux| file(s) of the main document.
This behaviour can be altered by the alternative form
|\childdocby[*]{|\textit{main}|}| (with a non-empty optional argument)
which uses the |.aux| file of the main document
by setting |\jobname| to \textit{main}.

%%%%%%%%%%%%%%%%%%%%%%%%%%%%%%%%%%%%%%%%%%%%%%%%%%%%%%%%%%%%%%%%%%%%%%%%%%%%%%%%
\subsection{Driver Development}
\label{sec:driver}

The \textsf{childdoc} mechanism can also be use for the development
of definition files such as \LaTeX{} styles or classes.
This case differs from the above setup with multiple parts
included by |\include| in that no |\includeonly| should be invoked.
This can be achieved by starting the include file
(before |\ProvidesPackage|) with:
%
\begin{center}
\begin{tabular}{l}
|\input{childdoc.def}|\\
|\childdocforward{|\textit{main}|}|\\
\end{tabular}
\end{center}
%
or alternatively with:
%
\begin{center}
\begin{tabular}{l}
|\input{childdoc.def}|\\
|\childdocby{|\textit{main}|}|\\
\end{tabular}
\end{center}
%
Both forms have slightly different effects as described above.
The main file is prepared as usual, see \secref{sec:include}.

%%%%%%%%%%%%%%%%%%%%%%%%%%%%%%%%%%%%%%%%%%%%%%%%%%%%%%%%%%%%%%%%%%%%%%%%%%%%%%%%
\subsection{Legacy Detection}
\label{sec:detection}

The directive |\childdocmain| in the main file can detect
whether the complete document or merely a child is to be compiled
even without using the directive |\childdocof|.
This method is deprecated because it is less robust
and there is no compelling reason to use it;
it is merely provided for backward compatibility
and it may be removed in future versions.

If the detection mechanism is to be used,
it is mandatory to correctly specify
the filename of the main file as the argument of |\childdocmain|:
%
\begin{center}
\begin{tabular}{l}
|\input{childdoc.def}|\\
|\childdocmain{|\textit{main}|}|\\
\end{tabular}
\end{center}
%
If |\jobname| does not match the argument \textit{main} of |\childdocmain|,
it is assumed that |\jobname| points to the child file to be compiled.
When using |\childdocmain| with the main file specified as argument,
it suffices to start a child file
with just |\input{|\textit{main}|}|
without loading of the package and using |\childdocof|.
If instead all processing is done
with the appropriate \textsf{childdoc} directives,
the argument of \textit{main} of |\childdocmain| can be empty.

An alternative version of the command line processing described
in \secref{sec:commandline} using the detection mechanism reads:
%
\begin{center}
|... -jobname "|\textit{target}|" "|[\textit{flags}]%
[|\def\jobname{|\textit{dest}|}|]|\input{|\textit{main}|}"|
\end{center}

%%%%%%%%%%%%%%%%%%%%%%%%%%%%%%%%%%%%%%%%%%%%%%%%%%%%%%%%%%%%%%%%%%%%%%%%%%%%%%%%
\subsection{Manual Code}
\label{sec:manual}

In case one cannot be certain whether the definitions file |childdoc.def|
is installed on the target \TeX{} distribution
and one prefers not to ship it,
it is conceivable to paste a few relevant commands into the sources.

To that end, drop all statements |\input{childdoc.def}|
and perform the replacements as outlined below.
Instead of |\childdocmain{|\textit{main}|}| add the following code
to the top of the main file:
%
\begin{center}
\begin{tabular}{l}
|\||ifdefined\childdocname\endinput\||fi\newif\ifchilddoc|\\
|\edef\childdocname{\scantokens\expandafter{\jobname\noexpand}}|\\
|\def\childdocmain{|\textit{main}|}\||ifx\childdocmain\childdocname\||else|\\
|\childdoctrue\includeonly{\childdocname}\let\jobname\childdocmain\||fi|\\
\end{tabular}
\end{center}
%
Instead of |\childdocof{|\textit{main}|}| just include the main file
at the top of each child file:
%
\begin{center}
|\input{|\textit{main}|}|
\end{center}
%
A simple redirection |\childdocforward{|\textit{dest}|}| is achieved by:
%
\begin{center}
|\def\jobname{|\textit{dest}|}\input{\jobname}|
\end{center}
%
The redirection with prefix
|\childdocforwardprefix[|\textit{prefix}|]{|\textit{dest}|}|
is accomplished by:
%
\begin{center}
\begin{tabular}{l}
|{\edef\jobname{\scantokens\expandafter{\jobname\noexpand}}|\\
|\def\redirectjob |\textit{prefix}|#1~~~{\gdef\jobname{|\textit{dest}|#1}}|\\
|\expandafter\redirectjob\jobname~~~}\input{\jobname}|
\end{tabular}
\end{center}

In an alternative approach,
child documents can be compiled by a specific command line
without additional code or specific definitions:
%
\begin{center}
|... -jobname "|\textit{target}|" "|[\textit{flags}]%
|\includeonly{|\textit{dest}|}\input{|\textit{main}|}"|
\end{center}
%

%%%%%%%%%%%%%%%%%%%%%%%%%%%%%%%%%%%%%%%%%%%%%%%%%%%%%%%%%%%%%%%%%%%%%%%%%%%%%%%%
%%%%%%%%%%%%%%%%%%%%%%%%%%%%%%%%%%%%%%%%%%%%%%%%%%%%%%%%%%%%%%%%%%%%%%%%%%%%%%%%
\section{Information}

%%%%%%%%%%%%%%%%%%%%%%%%%%%%%%%%%%%%%%%%%%%%%%%%%%%%%%%%%%%%%%%%%%%%%%%%%%%%%%%%
\subsection{Copyright}

Copyright \copyright{} 2017--2018 Niklas Beisert

This work may be distributed and/or modified under the
conditions of the \LaTeX{} Project Public License, either version 1.3
of this license or (at your option) any later version.
The latest version of this license is in
  \url{http://www.latex-project.org/lppl.txt}
and version 1.3 or later is part of all distributions of \LaTeX{}
version 2005/12/01 or later.

This work has the LPPL maintenance status `maintained'.

The Current Maintainer of this work is Niklas Beisert.

This work consists of the files |README.txt|, |childdoc.ins| and |childdoc.dtx|
as well as the derived files |childdoc.def|, |cdocsamp.tex|
with |cdocsch1.tex|, |cdocsch2.tex|, |cdocspt3.tex|, |cdocspt4.tex|,
|cdocsdrf.tex|, |cdocsfn1.tex|, |cdocsfn2.tex|
as well as |childdoc.pdf|.

%%%%%%%%%%%%%%%%%%%%%%%%%%%%%%%%%%%%%%%%%%%%%%%%%%%%%%%%%%%%%%%%%%%%%%%%%%%%%%%%
\subsection{Files and Installation}

The package consists of the files:
%
\begin{center}
\begin{tabular}{ll}
    |README.txt|   & readme file \\
    |childdoc.ins| & installation file \\
    |childdoc.dtx| & source file \\
    |childdoc.def| & definition file \\
    |cdocsamp.tex| & sample main file \\
    |cdocsch1.tex| & sample include file \\
    |cdocsch2.tex| & sample include file \\
    |cdocspt3.tex| & sample part file \\
    |cdocspt4.tex| & sample part file \\
    |cdocsdrf.tex| & sample redirection file \\
    |cdocsfn1.tex| & sample redirection file \\
    |cdocsfn2.tex| & sample redirection file \\
    |childdoc.pdf| & manual
\end{tabular}
\end{center}
%
The distribution consists of the files
|README.txt|, |childdoc.ins| and |childdoc.dtx|.
%
\begin{itemize}
\item
Run (pdf)\LaTeX{} on |childdoc.dtx|
to compile the manual |childdoc.pdf| (this file).
\item
Run \LaTeX{} on |childdoc.ins| to create the definitions file |childdoc.def|
and the sample |cdocsamp.tex| with include files
|cdocsch1.tex|, |cdocsch2.tex|, |cdocspt3.tex|, |cdocspt4.tex|,
|cdocsdrf.tex|, |cdocsfn1.tex|, |cdocsfn2.tex|.
Then copy the file |childdoc.def| to an appropriate directory of your \LaTeX{}
distribution, e.g.\ \textit{texmf-root}|/tex/latex/childdoc|.
\end{itemize}

%%%%%%%%%%%%%%%%%%%%%%%%%%%%%%%%%%%%%%%%%%%%%%%%%%%%%%%%%%%%%%%%%%%%%%%%%%%%%%%%
\subsection{Related CTAN Packages}

There are several other packages which offer a similar functionality:
%
\begin{itemize}
\item
The packages
\href{http://ctan.org/pkg/docmute}{\textsf{docmute}},
\href{http://ctan.org/pkg/includex}{\textsf{includex}} and
\href{http://ctan.org/pkg/standalone}{\textsf{standalone}}
provide commands to include only the document body of
a child file thus allowing both files to be compiled individually.
\item
The packages \href{http://ctan.org/pkg/subdocs}{\textsf{subdocs}}
and \href{http://ctan.org/pkg/subfiles}{\textsf{subfiles}}
provide structures in which the main and child documents can be
encapsulated and allowing them to be compiled individually.
The inclusion mechanism is different from the conventional |\include|.
\item
The package \href{http://ctan.org/pkg/combine}{\textsf{combine}}
is an elaborate solution to combine several documents into one.
\end{itemize}
%
See also the CTAN topic \href{http://ctan.org/topic/subdocs}{\textsf{subdocs}}
for further related packages.
The present package differs from the above solutions in that
a document structure constructed with the conventional |\include| mechanism
just needs two extra commands at the top of every file
such that all constituent files can be compiled individually.

%%%%%%%%%%%%%%%%%%%%%%%%%%%%%%%%%%%%%%%%%%%%%%%%%%%%%%%%%%%%%%%%%%%%%%%%%%%%%%%%
%\subsection{Feature Suggestions}
%
%The following is a list of features which may be useful for future
%versions of this package:
%%
%\begin{itemize}
%\item
%\ldots
%\end{itemize}

%%%%%%%%%%%%%%%%%%%%%%%%%%%%%%%%%%%%%%%%%%%%%%%%%%%%%%%%%%%%%%%%%%%%%%%%%%%%%%%%
\subsection{Revision History}

%%%%%%%%%%%%%%%%%%%%%%%%%%%%%%%%%%%%%%%%
\paragraph{v2.0:} 2018/12/30

\begin{itemize}
\item
immediate forward processing
\item
added |\childdocby| mechanism
\item
manual restructured
\end{itemize}

%%%%%%%%%%%%%%%%%%%%%%%%%%%%%%%%%%%%%%%%
\paragraph{v1.6:} 2018/01/17

\begin{itemize}
\item
application for development of include files
\item
corrections to manual
\end{itemize}

%%%%%%%%%%%%%%%%%%%%%%%%%%%%%%%%%%%%%%%%
\paragraph{v1.5:} 2017/05/21

\begin{itemize}
\item
more complete structuring introduced
\item
|\childdocof| introduced
\item
|\childdoc| renamed to |\childdocmain|
\item
|\childredirect| renamed to |\childdocforward| and |\childdocforwardprefix|
and functionality expanded
\end{itemize}

%%%%%%%%%%%%%%%%%%%%%%%%%%%%%%%%%%%%%%%%
\paragraph{v1.0:} 2017/04/27

\begin{itemize}
\item
manual and install package
\item
first version published on CTAN
\end{itemize}

%%%%%%%%%%%%%%%%%%%%%%%%%%%%%%%%%%%%%%%%
\paragraph{v0.6:} 2017/04/26

\begin{itemize}
\item
redirection mechanism added
\end{itemize}

%%%%%%%%%%%%%%%%%%%%%%%%%%%%%%%%%%%%%%%%
\paragraph{v0.5:} 2017/04/26

\begin{itemize}
\item
functionality in definition file
\end{itemize}


%%%%%%%%%%%%%%%%%%%%%%%%%%%%%%%%%%%%%%%%%%%%%%%%%%%%%%%%%%%%%%%%%%%%%%%%%%%%%%%%
%%%%%%%%%%%%%%%%%%%%%%%%%%%%%%%%%%%%%%%%%%%%%%%%%%%%%%%%%%%%%%%%%%%%%%%%%%%%%%%%
%%%%%%%%%%%%%%%%%%%%%%%%%%%%%%%%%%%%%%%%%%%%%%%%%%%%%%%%%%%%%%%%%%%%%%%%%%%%%%%%
\appendix

\settowidth\MacroIndent{\rmfamily\scriptsize 000\ }

 \DocInput{childdoc.dtx}

\end{document}
%</driver>
% \fi
%
% %%%%%%%%%%%%%%%%%%%%%%%%%%%%%%%%%%%%%%%%%%%%%%%%%%%%%%%%%%%%%%%%%%%%%%%%%%%%%%
% %%%%%%%%%%%%%%%%%%%%%%%%%%%%%%%%%%%%%%%%%%%%%%%%%%%%%%%%%%%%%%%%%%%%%%%%%%%%%%
% \section{Sample}
%\iffalse
%<*samplemain>
%\fi
%
% The following presents a sample document
% with two chapters, two parts, a title page,
% a compile flag as well as three forwarding files to set the flag.
% It consists of eight |.tex| files:
% \begin{center}
% \begin{tabular}{ll}
% |cdocsamp.tex|&main file\\
% |cdocsch1.tex|&include file for chapter 1\\
% |cdocsch2.tex|&include file for chapter 2\\
% |cdocspt3.tex|&include file for part 3\\
% |cdocspt4.tex|&include file for part 4\\
% |cdocsdrf.tex|&forwarding file for main file in draft mode\\
% |cdocsfi1.tex|&forwarding file for final version of chapter 1\\
% |cdocsfi2.tex|&forwarding file for final version of chapter 2\\
% \end{tabular}
% \end{center}
% Each of the eight files can be compiled directly by the \LaTeX{} compiler.
%
% %%%%%%%%%%%%%%%%%%%%%%%%%%%%%%%%%%%%%%
% \paragraph{Main File.}
%
% The main file is called |cdocsamp.tex|.
%
% Load the \textsf{childdoc} definitions and
% declare the filename for the main document:
%    \begin{macrocode}
\input{childdoc.def}
\childdocmain{}
%    \end{macrocode}

% Optional override for |\version| flag:
%    \begin{macrocode}
%%\ifchilddoc\else\providecommand{\version}{draft}\fi
%    \end{macrocode}

% Define the default values for the |\version| flag
% (|final| for the main file and |draft| for childs):
%    \begin{macrocode}
\ifchilddoc
\providecommand{\version}{draft}
\else
\providecommand{\version}{final}
\fi
%    \end{macrocode}

% Load the standard document class:
%    \begin{macrocode}
\documentclass[12pt]{article}
%    \end{macrocode}

% Start the document body:
%    \begin{macrocode}
\begin{document}
%    \end{macrocode}

% Declare a title page.
% Print title, part of document being processed and version flag:
%    \begin{macrocode}
\addtocounter{page}{-1}
\begin{center}
{\LARGE\bfseries{}childdoc example\par}
\vspace{1cm}
\ifchilddoc
\ifchilddocmanual part\else chapter\fi:
`\childdocname' of `\childdocjob'\par
\else
main document: `\childdocjob'\par
\fi
version: \version\par
\end{center}
\newpage
%    \end{macrocode}

% Manually include selected file,
% otherwise process as usual:
%    \begin{macrocode}
\ifchilddocmanual
\section*{part `\childdocname'}
\input{\childdocname}
\else
%    \end{macrocode}

% Include the two chapters:
%    \begin{macrocode}
\include{cdocsch1}
\include{cdocsch2}
%    \end{macrocode}

% Include the two parts unless only chapters should be displayed:
%    \begin{macrocode}
\ifchilddoc\else
\section{part three}
\input{cdocspt3}
\section{part four}
\input{cdocspt4}
\fi
%    \end{macrocode}

% Process as usual until here:
%    \begin{macrocode}
\fi
%    \end{macrocode}

% End of document body:
%    \begin{macrocode}
\end{document}
%    \end{macrocode}
%\iffalse
%</samplemain>
%\fi
%
% %%%%%%%%%%%%%%%%%%%%%%%%%%%%%%%%%%%%%%
% \paragraph{Chapter Include Files.}
%
% The include files are called |cdocsch1.tex| and |cdocsch2.tex|.
%
%\iffalse
%<*samplechap1|samplechap2>
%\fi

% Optional override for |\version| flag:
%    \begin{macrocode}
%%\providecommand{\version}{final}
%    \end{macrocode}

% Include the main document:
%    \begin{macrocode}
\input{childdoc.def}
\childdocof{cdocsamp}
%    \end{macrocode}

%\iffalse
%</samplechap1|samplechap2>
%\fi
%
%\iffalse
%<*samplechap1>
%\fi
% Some text for chapter 1:
%    \begin{macrocode}
\section{one}
some text in chapter one
%    \end{macrocode}

%\iffalse
%</samplechap1>
%\fi
% Some text for chapter 2:
%\iffalse
%<*samplechap2>
%\fi
%    \begin{macrocode}
\section{two}
more text in chapter two
%    \end{macrocode}

%\iffalse
%</samplechap2>
%\fi
%
% %%%%%%%%%%%%%%%%%%%%%%%%%%%%%%%%%%%%%%
% \paragraph{Part Include Files.}
%
% The include files are called |cdocspt3.tex| and |cdocspt4.tex|.
%
%\iffalse
%<*samplepart3|samplepart4>
%\fi

% Optional override for |\version| flag:
%    \begin{macrocode}
%%\providecommand{\version}{final}
%    \end{macrocode}

% Include the main document:
%    \begin{macrocode}
\input{childdoc.def}
\childdocby{cdocsamp}
%    \end{macrocode}

%\iffalse
%</samplepart3|samplepart4>
%\fi
%
%\iffalse
%<*samplepart3>
%\fi
% Some text for part 3:
%    \begin{macrocode}
some text in part three
%    \end{macrocode}

%\iffalse
%</samplepart3>
%\fi
% Some text for part 4:
%\iffalse
%<*samplepart4>
%\fi
%    \begin{macrocode}
more text in part four
%    \end{macrocode}

%\iffalse
%</samplepart4>
%\fi
%
% %%%%%%%%%%%%%%%%%%%%%%%%%%%%%%%%%%%%%%
% \paragraph{Forwarding for a Complete Draft.}
%
% The following forwarding file |cdocsdrf.tex|
% compiles the main document in draft mode:
%\iffalse
%<*sampledraft>
%\fi
%    \begin{macrocode}
\def\version{draft}
\input{childdoc.def}
\childdocforward{cdocsamp}
%    \end{macrocode}

%\iffalse
%</sampledraft>
%\fi
%
% %%%%%%%%%%%%%%%%%%%%%%%%%%%%%%%%%%%%%%
% \paragraph{Forwarding for Final Version of the Chapters.}
%
% The following forwarding files |cdocsfn1.tex| and |cdocsfn2.tex|
% (with identical content)
% compile the final versions of the child documents
% |cdocsch1.tex| and |cdocsch2.tex|, respectively:
%\iffalse
%<*samplefinal>
%\fi
%    \begin{macrocode}
\def\version{final}
\input{childdoc.def}
\childdocforwardprefix[cdocsamp]{cdocsfn}{cdocsch}
%    \end{macrocode}

%\iffalse
%</samplefinal>
%\fi
%
% %%%%%%%%%%%%%%%%%%%%%%%%%%%%%%%%%%%%%%
% \paragraph{Command Line Processing.}
%
% The following three command lines generate the output files
% |cdocscld|, |cdocscl1| and |cdocscl2|
% which should be identical to
% |cdocsdrf|, |cdocsch1| and |cdocsfn2|, respectively:
% \begin{center}
% \begin{tabular}{l}
% |latex -jobname cdocscld \|\\
% |  "\def\version{draft}\input{childdoc.def}\childdocforward{cdocsamp}"|\\
% |latex -jobname cdocscl1 \|\\
% |  "\input{childdoc.def}\childdocforward[cdocsamp]{cdocsch1}"|\\
% |latex -jobname cdocscl2 \|\\
% |  "\def\version{final}\input{childdoc.def}\childdocforward{cdocsch2}"|
% \end{tabular}
% \end{center}
% Note that the trailing backslash on each first line
% merely continues the input to the second line
% (for convenient cut ant paste).
% Furthermore, the command |latex| can be replaced by any
% of its alternative versions such as |pdflatex|.
%
% %%%%%%%%%%%%%%%%%%%%%%%%%%%%%%%%%%%%%%%%%%%%%%%%%%%%%%%%%%%%%%%%%%%%%%%%%%%%%%
% %%%%%%%%%%%%%%%%%%%%%%%%%%%%%%%%%%%%%%%%%%%%%%%%%%%%%%%%%%%%%%%%%%%%%%%%%%%%%%
% \section{Implementation}
%\iffalse
%<*package>
%\fi
%
% This section describes the definitions file |childdoc.def|.

% The definitions cannot be loaded using |\usepackage| or |\RequirePackage|
% which has a mechanism to prevent loading a style file more than once.
% When loading the definitions by means of |\input|
% multiple instances have to be prevented manually:
%\iffalse
%This code needs to be before the `\ProvidesFile' directive
%which is defined at the beginning of this file.
%Therefore it is also placed there and commented out here.
%</package>
%<*discard>
%\fi
%    \begin{macrocode}
\ifdefined\childdocmain\endinput\fi
%    \end{macrocode}
%\iffalse
%</discard>
%<*package>
%\fi
%
% \macro{\ifchilddoc}
% \macro{\ifchilddocmanual}
% The conditional |\ifchilddoc| tells whether a
% child (true) or main (false) document is being compiled.
% The conditional |\ifchilddocmanual| tells whether
% the |\includeonly| mechanism is used (false) or
% the selection of child files must be performed manually (true).
% The definitions initialise to false:
%    \begin{macrocode}
\newif\ifchilddoc
\newif\ifchilddocmanual
%    \end{macrocode}

% \macro{\childdocname}
% \macro{\childdocjob}
% The macro |\childdocname| stores the name of the main document
% to be compiled. The macro |\childdocjob| stores the name of
% the document on which the \LaTeX{} compiler was originally invoked.
% The content of |\jobname| cannot be compared
% to filenames specified in the source due to different catcodes.
% The following code rescans |\jobname|, stores the result
% in |\childdocname| and saves a copy in |\childdocjob|:
%    \begin{macrocode}
\edef\childdocname{\scantokens\expandafter{\jobname\noexpand}}
\let\childdocjob\childdocname
%    \end{macrocode}

% \macro{\childdocdisable}
% The macro |\childdocdisable| prevents the main file
% from being processed more than once.
% At this stage, the main document command |\childdocmain|
% is assumed to be called once again where it should do nothing.
% Any subsequent call to it should prevent
% a secondary processing of the main document
% It overwrites the forwarding commands
% |\childdocof| and |\childdocforward|
% with empty macros to prevent further inclusions of the main document:
%    \begin{macrocode}
\newcommand{\childdocdisable}
{
  \renewcommand{\childdocmain}[1]{\renewcommand{\childdocmain}[1]{\endinput}}
  \renewcommand{\childdocof}[1]{}
  \renewcommand{\childdocby}[2][]{}
  \renewcommand{\childdocforward}[2][]{}
  \renewcommand{\childdocdisable}{}
}
%    \end{macrocode}

% \macro{\childdocmain}
% The macro |\childdocmain| is to be called at the top of the main file
% with nothing or the main filename (without extension) as argument.
% First, it breaks loops.
% If the argument is not empty and does not match |\childdocname|
% (which is set by the first inclusion of |childdoc.def|),
% |\ifchilddoc| is set to true, |\includeonly| is applied to the child file
% and |\jobname| is set to the main file
% (for proper handling of |.aux| files):
%    \begin{macrocode}
\newcommand{\childdocmain}[1]
{
  \childdocdisable\childdocmain{}
  \if?#1?\else
    \begingroup
      \def\childdoctmp{#1}
      \ifx\childdoctmp\childdocname
        \def\childdoctmp{}
      \else
        \def\childdoctmp
        {
          \childdoctrue
          \includeonly{\childdocname}
          \def\childdocjob{#1}
          \def\jobname{#1}
        }
      \fi
      \expandafter
    \endgroup
    \childdoctmp
  \fi
}
%    \end{macrocode}

% \macro{\childdocof}
% The command |\childdocof| redirects
% compilation to the main file |#1|.
%    \begin{macrocode}
\newcommand{\childdocof}[1]
{
  \childdocdisable
  \childdoctrue
  \includeonly{\childdocname}
  \def\jobname{#1}
  \def\childdocjob{#1}
  \input{#1}
}
%    \end{macrocode}

% \macro{\childdocby}
% The command |\childdocby| ....
%    \begin{macrocode}
\newcommand{\childdocby}[2][]
{
  \childdocdisable
  \childdoctrue
  \childdocmanualtrue
  \if?#1?\else
    \def\jobname{#2}
  \fi
  \def\childdocjob{#2}
  \input{#2}
  \endinput
}
%    \end{macrocode}

% \macro{\childdocforward}
% The command |\childdocforward| redirects
% compilation to the main file or
% (if the optional argument is given) a child file.
% Parameters are set as if the main file
% or a child file starting with |\childdocof| was compiled.
% Then compilation is handed over to the main file:
%    \begin{macrocode}
\newcommand{\childdocforward}[2][]
{
  \begingroup
    \if?#1?
      \def\childdoctmp
      {
        \def\childdocname{#2}
        \def\childdocjob{#2}
        \def\jobname{#2}
        \input{#2}
        \endinput
      }
    \else
      \def\childdoctmp
      {
        \childdocdisable
        \def\childdocname{#2}
        \childdoctrue
        \includeonly{#2}
        \def\childdocjob{#1}
        \def\jobname{#1}
        \input{#1}
        \endinput
      }
    \fi
    \expandafter
  \endgroup
  \childdoctmp
}
%    \end{macrocode}

% \macro{\childdocforwardprefix}
% The command |\childdocforwardprefix| redirects
% compilation to the main or a child file by means of a pattern.
% The prefix |#1| in the current filename is replaced by |#2|
% and the suffix of the current filename is kept
% (it is assumed that the filename does not contain the substring `|~~~|'
% which is used as a delimiter).
% Compilation is handed over to the new file by |\childdocforward|:
%    \begin{macrocode}
\newcommand{\childdocforwardprefix}[3][]
{
  \begingroup
    \def\childdocextract #2##1~~~{\def\childdoctmp{\childdocforward[#1]{#3##1}}}
    \expandafter\childdocextract\childdocname~~~
    \expandafter
  \endgroup
  \childdoctmp
}
%    \end{macrocode}

% \macro{\childdoc}
% The deprecated macro |\childdoc| is a legacy version of |\childdocmain|:
%    \begin{macrocode}
\newcommand{\childdoc}{\childdocmain}
%    \end{macrocode}

% \macro{\childdocredirect}
% The deprecated macro |\childdocredirect| is a legacy version
% of |\childdocforward| and |\childdocforwardprefix|:
%    \begin{macrocode}
\newcommand{\childdocredirect}[2][]
{
  \begingroup
    \if?#1?
      \def\childdoctmp{\childdocforward{#2}}
    \else
      \def\childdoctmp{\childdocforwardprefix{#1}{#2}}
    \fi
    \expandafter
  \endgroup
  \childdoctmp
}
%    \end{macrocode}

%\iffalse
%</package>
%\fi
%
\endinput

\childdocby{cdocsamp}
%    \end{macrocode}

%\iffalse
%</samplepart3|samplepart4>
%\fi
%
%\iffalse
%<*samplepart3>
%\fi
% Some text for part 3:
%    \begin{macrocode}
some text in part three
%    \end{macrocode}

%\iffalse
%</samplepart3>
%\fi
% Some text for part 4:
%\iffalse
%<*samplepart4>
%\fi
%    \begin{macrocode}
more text in part four
%    \end{macrocode}

%\iffalse
%</samplepart4>
%\fi
%
% %%%%%%%%%%%%%%%%%%%%%%%%%%%%%%%%%%%%%%
% \paragraph{Forwarding for a Complete Draft.}
%
% The following forwarding file |cdocsdrf.tex|
% compiles the main document in draft mode:
%\iffalse
%<*sampledraft>
%\fi
%    \begin{macrocode}
\def\version{draft}
% \iffalse
%
% childdoc.dtx Copyright (C) 2017-2018 Niklas Beisert
%
% This work may be distributed and/or modified under the
% conditions of the LaTeX Project Public License, either version 1.3
% of this license or (at your option) any later version.
% The latest version of this license is in
%   http://www.latex-project.org/lppl.txt
% and version 1.3 or later is part of all distributions of LaTeX
% version 2005/12/01 or later.
%
% This work has the LPPL maintenance status `maintained'.
%
% The Current Maintainer of this work is Niklas Beisert.
%
% This work consists of the files childdoc.dtx and childdoc.ins
% and the derived files childdoc.def and cdocsamp.tex with
% cdocsch1.tex, cdocsch2.tex, cdocsdrf.tex, cdocsfn1.tex, cdocsfn2.tex.
%
%<package>\ifdefined\childdocmain\endinput\fi
%<package>\ProvidesFile{childdoc.def}[2018/12/30 v2.0 child document driver]
%<samplemain>\ProvidesFile{cdocsamp.tex}[2018/12/30 v2.0 sample for childdoc]
%<*driver>
%\ProvidesFile{childdoc.drv}[2018/12/30 v2.0 childdoc reference manual file]
\PassOptionsToClass{10pt,a4paper}{article}
\documentclass{ltxdoc}

\usepackage[margin=35mm]{geometry}
\usepackage{hyperref}
\usepackage{hyperxmp}
\usepackage[usenames]{color}

\hypersetup{colorlinks=true}
\hypersetup{pdfstartview=FitH}
\hypersetup{pdfpagemode=UseNone}
\hypersetup{pdfsource={}}
\hypersetup{pdflang={en-UK}}
\hypersetup{pdfcopyright={Copyright 2017-2018 Niklas Beisert.
  This work may be distributed and/or modified under the
  conditions of the LaTeX Project Public License, either version 1.3
  of this license or (at your option) any later version.}}
\hypersetup{pdflicenseurl={http://www.latex-project.org/lppl.txt}}
\hypersetup{pdfcontactaddress={ETH Zurich, ITP, HIT K,
  Wolfgang-Pauli-Strasse 27}}
\hypersetup{pdfcontactpostcode={8093}}
\hypersetup{pdfcontactcity={Zurich}}
\hypersetup{pdfcontactcountry={Switzerland}}
\hypersetup{pdfcontactemail={nbeisert@itp.phys.ethz.ch}}
\hypersetup{pdfcontacturl={http://people.phys.ethz.ch/\xmptilde nbeisert/}}

\newcommand{\secref}[1]{\hyperref[#1]{section \ref*{#1}}}

\parskip1ex
\parindent0pt
\let\olditemize\itemize
\def\itemize{\olditemize\parskip0pt}

\begin{document}

\title{The \textsf{childdoc} Package}
\hypersetup{pdftitle={The childdoc Package}}
\author{Niklas Beisert\\[2ex]
  Institut f\"ur Theoretische Physik\\
  Eidgen\"ossische Technische Hochschule Z\"urich\\
  Wolfgang-Pauli-Strasse 27, 8093 Z\"urich, Switzerland\\[1ex]
  \href{mailto:nbeisert@itp.phys.ethz.ch}
  {\texttt{nbeisert@itp.phys.ethz.ch}}}
\hypersetup{pdfauthor={Niklas Beisert}}
\hypersetup{pdfsubject={Manual for the LaTeX2e Package childdoc}}
\date{30 December 2018, \textsf{v2.0}}
\maketitle

\begin{abstract}\noindent
\textsf{childdoc} is a \LaTeXe{} package
that enables the direct compilation
of document sections included by |\include|
to individual files.
\end{abstract}

\begingroup
\parskip0ex
\tableofcontents
\endgroup

%%%%%%%%%%%%%%%%%%%%%%%%%%%%%%%%%%%%%%%%%%%%%%%%%%%%%%%%%%%%%%%%%%%%%%%%%%%%%%%%
%%%%%%%%%%%%%%%%%%%%%%%%%%%%%%%%%%%%%%%%%%%%%%%%%%%%%%%%%%%%%%%%%%%%%%%%%%%%%%%%
\section{Introduction}

\LaTeX{} provides a mechanism to structure a large document (such as a book)
into a main file and several child files (containing the chapters)
using the |\include| command.
This mechanism is beneficial for documents
which span hundreds of pages in order to
make the source file(s) more manageable.
Moreover, compilation can be restricted to
selected child files by means of the |\includeonly| command.
The latter feature can be used to reduce the compilation time while editing
(this was significantly more useful in the earlier days of \LaTeX{})
or to generate a smaller document which is easier to navigate.
Another application of |\includeonly| is to generate
documents consisting of selected parts of the complete document.

However, there are a few drawbacks of the plain |\include| mechanism:
\begin{itemize}
\item
The child files cannot be compiled on their own,
they can only be compiled via the main file.
A naive editing environment
(such as a text editor with an option
to have the current file processed by \LaTeX)
may require one to switch to the main file before compiling;
attempting to compile the child file produces errors.
\item
The main file must be modified (each time)
to adjust the |\includeonly| command
to the present needs. This easily leaves the main file in a messy state.
\item
The generated document will always carry the filename
of the main document. This is inconvenient if
several child files are to be compiled and
to be kept for distribution.
\end{itemize}

The present package provides a simple interface
to make child files individually compilable by \LaTeX{}.
Compiling a child file then has the same effect as compiling
the main file with an |\includeonly| command
to select the appropriate child.
Moreover the generated document will carry the name of the child
rather than the main file.
This resolves all three above issues.

This feature is meant to make the editing of books,
thesis documents and lecture notes somewhat more convenient.
However, the package can also be used efficiently for
composing a series of documents (such as exercise sheets)
which are typically distributed individually.
It then assists the author in generating the individual documents
(potentially in different versions)
as well as a document containing the collected series.
Another application is in developing style files
or other kinds of included material
where compilation of the style file could redirect
to a sample or test file.

%%%%%%%%%%%%%%%%%%%%%%%%%%%%%%%%%%%%%%%%%%%%%%%%%%%%%%%%%%%%%%%%%%%%%%%%%%%%%%%%
%%%%%%%%%%%%%%%%%%%%%%%%%%%%%%%%%%%%%%%%%%%%%%%%%%%%%%%%%%%%%%%%%%%%%%%%%%%%%%%%
\section{Usage}

First of all, the package \textsf{childdoc} is \emph{not} a standard
\LaTeXe{} |.sty| style file! Therefore it needs to be invoked in
a non-standard way.

%%%%%%%%%%%%%%%%%%%%%%%%%%%%%%%%%%%%%%%%%%%%%%%%%%%%%%%%%%%%%%%%%%%%%%%%%%%%%%%%
\subsection{Included Files}
\label{sec:include}

%%%%%%%%%%%%%%%%%%%%%%%%%%%%%%%%%%%%%%%%
\DescribeMacro{\childdocmain}
To use the package, add the commands
\begin{center}
\begin{tabular}{l}
|\input{childdoc.def}|\\
|\childdocmain{}|\\
\end{tabular}
\end{center}
at the very top of the main \LaTeX{} file,
in particular \emph{before} the |\documentclass| statement!
The argument of |\childdocmain| should be left empty
(but it must be present).

%%%%%%%%%%%%%%%%%%%%%%%%%%%%%%%%%%%%%%%%
\DescribeMacro{\childdocof}
Furthermore, add the commands
\begin{center}
\begin{tabular}{l}
|\input{childdoc.def}|\\
|\childdocof{|\textit{main}|}|\\
\end{tabular}
\end{center}
at the top of every child file \textit{child}
which is included by |\include{|\textit{child}|}|
from within the main file
(or at least for those files to be compiled individually).
The argument \textit{main} must be the filename of the main file.

There are a couple of
considerations in setting up the main and child documents:

%%%%%%%%%%%%%%%%%%%%%%%%%%%%%%%%%%%%%%%%
\paragraph{Restrictions.}

Please note the following restrictions:
\begin{itemize}
\item
|\childdocmain| must be called with one argument \textit{main}
to ensure compatibility with earlier version of the package.
It must either be empty (|\childdocmain{}|)
or precisely match the filename of the main file in which it is specified.
See \secref{sec:detection} for further information.
\item
The filename \textit{main} must be specified without the |.tex| extension.
\item
The filename \textit{main} is case sensitive
(even in case-insensitive file systems)
due to internal string comparison.
\item
The argument \textit{main} should be fully expanded, it cannot be a macro.
\item
Subdirectories and special characters should be avoided in filenames.
\item
The command |\childdocmain{|\textit{main}|}| must be followed by a whitespace.
It should not be followed immediately by another command
or by a comment mark `|%|'.
This is because the \TeX{} parser reads the token immediately following
the argument of |\childdocmain| and puts it
at the beginning of every child section;
however, a white\-space is ignored.
\end{itemize}

%%%%%%%%%%%%%%%%%%%%%%%%%%%%%%%%%%%%%%%%
\paragraph{Content of Main File.}

It is advisable to place all content in the child files included by |\include|.
Any output contained in the main file will appear in all child documents
unless suppressed manually;
it cannot be suppressed automatically by the |\includeonly| directive
and thus should normally be avoided.
A method to include some content in the main file
by means of conditional processing is described in \secref{sec:conditional}.

%%%%%%%%%%%%%%%%%%%%%%%%%%%%%%%%%%%%%%%%
\paragraph{Page Numbering.}

When only a part of the document is compiled,
the appropriate numbering of pages
(as well as other status parameters)
is determined from the |.aux| files.
The latter contain information from previous passes.
However this information needs to propagate through
all intermediate child documents.
Therefore the page numbering in child documents may well
be inconsistent until the complete document is compiled at least once.

A useful (if unconventional) way to always ensure a consistent
page numbering is to restart the numbering in each child document
and denote the pages by `\textit{child}|.|\textit{page}'
where \textit{child} represents the chapter/section number of the child file.
This can be achieved by the command
|\numberwithin{page}{|\textit{child}|}|
of the \textsf{amsmath} package
where \textit{child} can be |chapter| or |section|
depending on the chosen structuring.
Alternatively, one can modify the macro |\thepage| appropriately
and reset the counter |page| at the start of each child file.

%%%%%%%%%%%%%%%%%%%%%%%%%%%%%%%%%%%%%%%%%%%%%%%%%%%%%%%%%%%%%%%%%%%%%%%%%%%%%%%%
\subsection{Conditional Processing}
\label{sec:conditional}

The package provides a mechanism to compile different versions
of a document. To customise the versions further some conditional processing
can come in handy to distinguish which version is being compiled.
The package provides two macros to describe the compilation context:

%%%%%%%%%%%%%%%%%%%%%%%%%%%%%%%%%%%%%%%%
\DescribeMacro{\ifchilddoc}
The conditional |\ifchilddoc| distinguishes between the compilation of
child documents and the main document:
%
\begin{center}
|\ifchilddoc |\textit{child-code}| |[|\||else |\textit{main-code}]| \||fi|
\end{center}

%%%%%%%%%%%%%%%%%%%%%%%%%%%%%%%%%%%%%%%%
\DescribeMacro{\childdocname}
\DescribeMacro{\childdocjob}
The macro |\childdocname| contains the filename (without extension)
of the main or child file being processed.
Note that |\childdocjob| will always contain the name of the main file.

%%%%%%%%%%%%%%%%%%%%%%%%%%%%%%%%%%%%%%%%
\paragraph{Title Page.}

Conditional processing can be used to include a title or banner page
in the main document when proper precautions are taken.
Importantly, the code in the main file should ensure that the page counter
(as well as other status parameters which are stored in the |.aux| files)
takes the same value after the conditional processing.
Otherwise the page numbers may take divergent values
depending on which part is compiled.

For example, a title page could be declared by:
%
\begin{center}
\begin{tabular}{l}
|\ifchilddoc\||else|\\
|\addtocounter{page}{-1}|\\
\textit{code for title page}\\
|\newpage|\\
|\||fi|
\end{tabular}
\end{center}
%
A banner page for the child documents can be generated by:
%
\begin{center}
\begin{tabular}{l}
|\ifchilddoc|\\
|\addtocounter{page}{-1}|\\
\textit{code for banner page}\\
|\newpage|\\
|\||fi|
\end{tabular}
\end{center}
%
Here one could write a message such as:
\begin{center}
|This is the part \childdocname{} of \childdocjob{}.|
\end{center}

%%%%%%%%%%%%%%%%%%%%%%%%%%%%%%%%%%%%%%%%%%%%%%%%%%%%%%%%%%%%%%%%%%%%%%%%%%%%%%%%
\subsection{Flags}
\label{sec:flags}

The package makes it easy to generate different versions
of the main or child documents.
To this end compilation flags can be defined
and assigned different default values.
They will be particularly useful in conjunction
with the forwarding mechanism described in \secref{sec:forward}.

For example, it may be useful to have a flag |\version|
which can be set to |draft| or |final|.
The document source will contain some conditional code
depending on the value of |\version|.
Suppose further, the flag should default to |final| for the main file
and to |draft| for child files
which is a natural assignment for editing the document.
This is achieved by placing the following code
in the preamble of the main document
(below the |\childdocmain| directive):
%
\begin{center}
\begin{tabular}{l}
|\ifchilddoc|\\
|\providecommand{\version}{draft}|\\
|\||else|\\
|\providecommand{\version}{final}|\\
|\||fi|
\end{tabular}
\end{center}
%
The definition by |\providecommand| makes sure
that previous definitions are not overwritten.
Further statements |\providecommand{\version}{...}|
can thus be added before the above code to override it.

For the main file, one might add a line
(between |\childdocmain| and the above block)
%
\begin{center}
|%\ifchilddoc\||else\providecommand{\version}{draft}\||fi|
\end{center}
%
which can be uncommented to produce a draft version.
Likewise one can add a line to the very top of a child file
(above the |\childdocof{|\textit{main}|}| directive)
%
\begin{center}
|%\providecommand{\version}{final}|
\end{center}
%
which can be uncommented to produce the final version of this child document.

%%%%%%%%%%%%%%%%%%%%%%%%%%%%%%%%%%%%%%%%%%%%%%%%%%%%%%%%%%%%%%%%%%%%%%%%%%%%%%%%
\subsection{Forwarding}
\label{sec:forward}

Different versions of the main or child documents
using compilation flags as described in \secref{sec:flags}
can be (permanently) stored in different files
for convenient compilation, viewing and distribution.
To this end, the package defines a command
to pass on compilation to a different file:

%%%%%%%%%%%%%%%%%%%%%%%%%%%%%%%%%%%%%%%%
\DescribeMacro{\childdocforward}
The command |\childdocforward| redirects processing to
another source file:
%
\begin{center}
\begin{tabular}{l}
|\input{childdoc.def}|\\
|\childdocforward[|\textit{main}|]{|\textit{dest}|}|\\
\end{tabular}
\end{center}
%
The argument \textit{dest} is the destination file
(without extension).
It should be the main file or one of the child files.
Note that further \textsf{childdoc} directives
such as |\childdocof| and |\childdocforward|
in the indicated file will be processed in this form.
The optional argument \textit{main}
passes on directly to the main file \textit{main}
while pretending to compile the child \textit{dest}.
This form behaves as if \textit{dest}
issues |\childdocof{|\textit{main}|}| right away,
and no further \textsf{childdoc} directives will be processed.

%%%%%%%%%%%%%%%%%%%%%%%%%%%%%%%%%%%%%%%%
\DescribeMacro{\...prefix}
In the alternative form |\childdocforwardprefix|,
%
\begin{center}
\begin{tabular}{l}
|\input{childdoc.def}|\\
|\childdocforwardprefix[|\textit{main}|]{|\textit{prefix}|}{|\textit{dest}|}|
\end{tabular}
\end{center}
%
the destination file is determined by a pattern
depending on the current file:
To make this work, the current file must be called
`{\textit{prefix}\hspace{0.2em}\textit{suffix}}'
with \textit{prefix} matching precisely the argument.
Processing is then passed on to the file
`{\textit{dest}\hspace{0.2em}\textit{suffix}}'.
Surely, the same effect is achieved by
directly specifying the
argument `{\textit{dest}\hspace{0.2em}\textit{suffix}}'
in the first form.
However, that requires to set up a different file
for each child. With the alternative form of the command
all these files can have exactly the same content
which simplifies setting them up and maintaining them.

For example, the following file |draft.tex|
with a compilation flag |\version| as described in \secref{sec:flags}
compiles the main document as a draft:
%
\begin{center}
\begin{tabular}{l}
|\def\version{draft}|\\
|\input{childdoc.def}|\\
|\childdocforward{|\textit{main}|}|
\end{tabular}
\end{center}
%
Likewise, the following files |final|\textit{nn}|.tex|
compile the final version of the child document
|child|\textit{nn}|.tex|:
%
\begin{center}
\begin{tabular}{l}
|\def\version{final}|\\
|\input{childdoc.def}|\\
|\childdocforwardprefix{final}{child}|
\end{tabular}
\end{center}
%

Note that when several versions of a main file and/or of each child file
are to be generated, it may be convenient to set up a |Makefile| or
shell script to automatise the process.

%%%%%%%%%%%%%%%%%%%%%%%%%%%%%%%%%%%%%%%%%%%%%%%%%%%%%%%%%%%%%%%%%%%%%%%%%%%%%%%%
\subsection{Command Line Processing}
\label{sec:commandline}

The effect of redirection files can also be achieved by invoking
the \LaTeX{} compiler with a more elaborate command line.
Most conveniently this should be done as part
of a shell script or a |Makefile|.

When using \textsf{childdoc} in the main file, the following
command lines effectively perform a redirection
(note that depending on the shell being used,
backslashes may have to be doubled: `|\|' $\to$ `|\\|'):
%
\begin{center}
|... -jobname "|\textit{target}|" |\\|"|[\textit{flags}]%
|\input{childdoc.def}\childdocforward[|\textit{main}|]{|\textit{dest}|}"|
\end{center}
%
Here \textit{target} is the name of the output file,
\textit{main} is the name of the main file
and \textit{dest} is the name of the main or child file to be processed
(all filenames without extensions).
The optional argument \textit{main} can be omitted
if \textit{main} matches \textit{dest}.
Optionally, compilation \textit{flags} can be defined via |\def| commands.
This command line makes the \TeX{} engine believe
it is compiling the file \textit{target}
whose content is specified as the latter parameter.
The provided code then forwards the processing to
\textit{main} or \textit{dest} as described in \secref{sec:forward}.

%%%%%%%%%%%%%%%%%%%%%%%%%%%%%%%%%%%%%%%%%%%%%%%%%%%%%%%%%%%%%%%%%%%%%%%%%%%%%%%%
\subsection{Include by Input}
\label{sec:input}

Including child documents by |\include| has some restrictions by design.
Most notably, the content of a child document always occupies
its own set of pages; pages cannot be shared between child documents.
Usually, this behaviour makes perfect sense
because each child document contain an essential part of the document.
However, in some situations it may be desirable to compose
a document from a collection of parts
without having mandatory page breaks between then.
For this case, the package
provides a mechanism to include parts
by |\input| which can also be processed individually.
However, by construction this mechanism
requires manual handling of the content to be output.

%%%%%%%%%%%%%%%%%%%%%%%%%%%%%%%%%%%%%%%%
\DescribeMacro{\ifchilddocmanual}
The main file should be prepared as usual, see \secref{sec:include}.
However, the document body must make a distinction
between processing of an individual part and of the main document, e.g.:
%
\begin{center}
\begin{tabular}{l}
|\ifchilddocmanual|\\
|\input{\childdocname}|\\
|\||else|\\
\textit{document body with }|\input{|\textit{part}|}|\\
|\||fi|
\end{tabular}
\end{center}
%
The conditional |\ifchilddocmanual| is true whenever
a part to be included by |\input| is being compiled,
and the name of the part is stored in |\childdocname|.

%%%%%%%%%%%%%%%%%%%%%%%%%%%%%%%%%%%%%%%%
\DescribeMacro{\childdocby}
Each part to be included by |\input| should start with:
%
\begin{center}
\begin{tabular}{l}
|\input{childdoc.def}|\\
|\childdocby{|\textit{main}|}|\\
\end{tabular}
\end{center}
%
The directive |\childdocby| is similar to |\childdocof|
described in \secref{sec:include},
but the subsequent selection of content must be done manually.
To that end, both |\ifchilddoc| and |\ifchilddocmanual|
will be true upon processing of a part,
and the name of the part is stored in |\childdocname|.
Note that |\jobname| will be set to the filename of the current part
so that each part receives an individual |.aux| file
that does not interfere with the |.aux| file(s) of the main document.
This behaviour can be altered by the alternative form
|\childdocby[*]{|\textit{main}|}| (with a non-empty optional argument)
which uses the |.aux| file of the main document
by setting |\jobname| to \textit{main}.

%%%%%%%%%%%%%%%%%%%%%%%%%%%%%%%%%%%%%%%%%%%%%%%%%%%%%%%%%%%%%%%%%%%%%%%%%%%%%%%%
\subsection{Driver Development}
\label{sec:driver}

The \textsf{childdoc} mechanism can also be use for the development
of definition files such as \LaTeX{} styles or classes.
This case differs from the above setup with multiple parts
included by |\include| in that no |\includeonly| should be invoked.
This can be achieved by starting the include file
(before |\ProvidesPackage|) with:
%
\begin{center}
\begin{tabular}{l}
|\input{childdoc.def}|\\
|\childdocforward{|\textit{main}|}|\\
\end{tabular}
\end{center}
%
or alternatively with:
%
\begin{center}
\begin{tabular}{l}
|\input{childdoc.def}|\\
|\childdocby{|\textit{main}|}|\\
\end{tabular}
\end{center}
%
Both forms have slightly different effects as described above.
The main file is prepared as usual, see \secref{sec:include}.

%%%%%%%%%%%%%%%%%%%%%%%%%%%%%%%%%%%%%%%%%%%%%%%%%%%%%%%%%%%%%%%%%%%%%%%%%%%%%%%%
\subsection{Legacy Detection}
\label{sec:detection}

The directive |\childdocmain| in the main file can detect
whether the complete document or merely a child is to be compiled
even without using the directive |\childdocof|.
This method is deprecated because it is less robust
and there is no compelling reason to use it;
it is merely provided for backward compatibility
and it may be removed in future versions.

If the detection mechanism is to be used,
it is mandatory to correctly specify
the filename of the main file as the argument of |\childdocmain|:
%
\begin{center}
\begin{tabular}{l}
|\input{childdoc.def}|\\
|\childdocmain{|\textit{main}|}|\\
\end{tabular}
\end{center}
%
If |\jobname| does not match the argument \textit{main} of |\childdocmain|,
it is assumed that |\jobname| points to the child file to be compiled.
When using |\childdocmain| with the main file specified as argument,
it suffices to start a child file
with just |\input{|\textit{main}|}|
without loading of the package and using |\childdocof|.
If instead all processing is done
with the appropriate \textsf{childdoc} directives,
the argument of \textit{main} of |\childdocmain| can be empty.

An alternative version of the command line processing described
in \secref{sec:commandline} using the detection mechanism reads:
%
\begin{center}
|... -jobname "|\textit{target}|" "|[\textit{flags}]%
[|\def\jobname{|\textit{dest}|}|]|\input{|\textit{main}|}"|
\end{center}

%%%%%%%%%%%%%%%%%%%%%%%%%%%%%%%%%%%%%%%%%%%%%%%%%%%%%%%%%%%%%%%%%%%%%%%%%%%%%%%%
\subsection{Manual Code}
\label{sec:manual}

In case one cannot be certain whether the definitions file |childdoc.def|
is installed on the target \TeX{} distribution
and one prefers not to ship it,
it is conceivable to paste a few relevant commands into the sources.

To that end, drop all statements |\input{childdoc.def}|
and perform the replacements as outlined below.
Instead of |\childdocmain{|\textit{main}|}| add the following code
to the top of the main file:
%
\begin{center}
\begin{tabular}{l}
|\||ifdefined\childdocname\endinput\||fi\newif\ifchilddoc|\\
|\edef\childdocname{\scantokens\expandafter{\jobname\noexpand}}|\\
|\def\childdocmain{|\textit{main}|}\||ifx\childdocmain\childdocname\||else|\\
|\childdoctrue\includeonly{\childdocname}\let\jobname\childdocmain\||fi|\\
\end{tabular}
\end{center}
%
Instead of |\childdocof{|\textit{main}|}| just include the main file
at the top of each child file:
%
\begin{center}
|\input{|\textit{main}|}|
\end{center}
%
A simple redirection |\childdocforward{|\textit{dest}|}| is achieved by:
%
\begin{center}
|\def\jobname{|\textit{dest}|}\input{\jobname}|
\end{center}
%
The redirection with prefix
|\childdocforwardprefix[|\textit{prefix}|]{|\textit{dest}|}|
is accomplished by:
%
\begin{center}
\begin{tabular}{l}
|{\edef\jobname{\scantokens\expandafter{\jobname\noexpand}}|\\
|\def\redirectjob |\textit{prefix}|#1~~~{\gdef\jobname{|\textit{dest}|#1}}|\\
|\expandafter\redirectjob\jobname~~~}\input{\jobname}|
\end{tabular}
\end{center}

In an alternative approach,
child documents can be compiled by a specific command line
without additional code or specific definitions:
%
\begin{center}
|... -jobname "|\textit{target}|" "|[\textit{flags}]%
|\includeonly{|\textit{dest}|}\input{|\textit{main}|}"|
\end{center}
%

%%%%%%%%%%%%%%%%%%%%%%%%%%%%%%%%%%%%%%%%%%%%%%%%%%%%%%%%%%%%%%%%%%%%%%%%%%%%%%%%
%%%%%%%%%%%%%%%%%%%%%%%%%%%%%%%%%%%%%%%%%%%%%%%%%%%%%%%%%%%%%%%%%%%%%%%%%%%%%%%%
\section{Information}

%%%%%%%%%%%%%%%%%%%%%%%%%%%%%%%%%%%%%%%%%%%%%%%%%%%%%%%%%%%%%%%%%%%%%%%%%%%%%%%%
\subsection{Copyright}

Copyright \copyright{} 2017--2018 Niklas Beisert

This work may be distributed and/or modified under the
conditions of the \LaTeX{} Project Public License, either version 1.3
of this license or (at your option) any later version.
The latest version of this license is in
  \url{http://www.latex-project.org/lppl.txt}
and version 1.3 or later is part of all distributions of \LaTeX{}
version 2005/12/01 or later.

This work has the LPPL maintenance status `maintained'.

The Current Maintainer of this work is Niklas Beisert.

This work consists of the files |README.txt|, |childdoc.ins| and |childdoc.dtx|
as well as the derived files |childdoc.def|, |cdocsamp.tex|
with |cdocsch1.tex|, |cdocsch2.tex|, |cdocspt3.tex|, |cdocspt4.tex|,
|cdocsdrf.tex|, |cdocsfn1.tex|, |cdocsfn2.tex|
as well as |childdoc.pdf|.

%%%%%%%%%%%%%%%%%%%%%%%%%%%%%%%%%%%%%%%%%%%%%%%%%%%%%%%%%%%%%%%%%%%%%%%%%%%%%%%%
\subsection{Files and Installation}

The package consists of the files:
%
\begin{center}
\begin{tabular}{ll}
    |README.txt|   & readme file \\
    |childdoc.ins| & installation file \\
    |childdoc.dtx| & source file \\
    |childdoc.def| & definition file \\
    |cdocsamp.tex| & sample main file \\
    |cdocsch1.tex| & sample include file \\
    |cdocsch2.tex| & sample include file \\
    |cdocspt3.tex| & sample part file \\
    |cdocspt4.tex| & sample part file \\
    |cdocsdrf.tex| & sample redirection file \\
    |cdocsfn1.tex| & sample redirection file \\
    |cdocsfn2.tex| & sample redirection file \\
    |childdoc.pdf| & manual
\end{tabular}
\end{center}
%
The distribution consists of the files
|README.txt|, |childdoc.ins| and |childdoc.dtx|.
%
\begin{itemize}
\item
Run (pdf)\LaTeX{} on |childdoc.dtx|
to compile the manual |childdoc.pdf| (this file).
\item
Run \LaTeX{} on |childdoc.ins| to create the definitions file |childdoc.def|
and the sample |cdocsamp.tex| with include files
|cdocsch1.tex|, |cdocsch2.tex|, |cdocspt3.tex|, |cdocspt4.tex|,
|cdocsdrf.tex|, |cdocsfn1.tex|, |cdocsfn2.tex|.
Then copy the file |childdoc.def| to an appropriate directory of your \LaTeX{}
distribution, e.g.\ \textit{texmf-root}|/tex/latex/childdoc|.
\end{itemize}

%%%%%%%%%%%%%%%%%%%%%%%%%%%%%%%%%%%%%%%%%%%%%%%%%%%%%%%%%%%%%%%%%%%%%%%%%%%%%%%%
\subsection{Related CTAN Packages}

There are several other packages which offer a similar functionality:
%
\begin{itemize}
\item
The packages
\href{http://ctan.org/pkg/docmute}{\textsf{docmute}},
\href{http://ctan.org/pkg/includex}{\textsf{includex}} and
\href{http://ctan.org/pkg/standalone}{\textsf{standalone}}
provide commands to include only the document body of
a child file thus allowing both files to be compiled individually.
\item
The packages \href{http://ctan.org/pkg/subdocs}{\textsf{subdocs}}
and \href{http://ctan.org/pkg/subfiles}{\textsf{subfiles}}
provide structures in which the main and child documents can be
encapsulated and allowing them to be compiled individually.
The inclusion mechanism is different from the conventional |\include|.
\item
The package \href{http://ctan.org/pkg/combine}{\textsf{combine}}
is an elaborate solution to combine several documents into one.
\end{itemize}
%
See also the CTAN topic \href{http://ctan.org/topic/subdocs}{\textsf{subdocs}}
for further related packages.
The present package differs from the above solutions in that
a document structure constructed with the conventional |\include| mechanism
just needs two extra commands at the top of every file
such that all constituent files can be compiled individually.

%%%%%%%%%%%%%%%%%%%%%%%%%%%%%%%%%%%%%%%%%%%%%%%%%%%%%%%%%%%%%%%%%%%%%%%%%%%%%%%%
%\subsection{Feature Suggestions}
%
%The following is a list of features which may be useful for future
%versions of this package:
%%
%\begin{itemize}
%\item
%\ldots
%\end{itemize}

%%%%%%%%%%%%%%%%%%%%%%%%%%%%%%%%%%%%%%%%%%%%%%%%%%%%%%%%%%%%%%%%%%%%%%%%%%%%%%%%
\subsection{Revision History}

%%%%%%%%%%%%%%%%%%%%%%%%%%%%%%%%%%%%%%%%
\paragraph{v2.0:} 2018/12/30

\begin{itemize}
\item
immediate forward processing
\item
added |\childdocby| mechanism
\item
manual restructured
\end{itemize}

%%%%%%%%%%%%%%%%%%%%%%%%%%%%%%%%%%%%%%%%
\paragraph{v1.6:} 2018/01/17

\begin{itemize}
\item
application for development of include files
\item
corrections to manual
\end{itemize}

%%%%%%%%%%%%%%%%%%%%%%%%%%%%%%%%%%%%%%%%
\paragraph{v1.5:} 2017/05/21

\begin{itemize}
\item
more complete structuring introduced
\item
|\childdocof| introduced
\item
|\childdoc| renamed to |\childdocmain|
\item
|\childredirect| renamed to |\childdocforward| and |\childdocforwardprefix|
and functionality expanded
\end{itemize}

%%%%%%%%%%%%%%%%%%%%%%%%%%%%%%%%%%%%%%%%
\paragraph{v1.0:} 2017/04/27

\begin{itemize}
\item
manual and install package
\item
first version published on CTAN
\end{itemize}

%%%%%%%%%%%%%%%%%%%%%%%%%%%%%%%%%%%%%%%%
\paragraph{v0.6:} 2017/04/26

\begin{itemize}
\item
redirection mechanism added
\end{itemize}

%%%%%%%%%%%%%%%%%%%%%%%%%%%%%%%%%%%%%%%%
\paragraph{v0.5:} 2017/04/26

\begin{itemize}
\item
functionality in definition file
\end{itemize}


%%%%%%%%%%%%%%%%%%%%%%%%%%%%%%%%%%%%%%%%%%%%%%%%%%%%%%%%%%%%%%%%%%%%%%%%%%%%%%%%
%%%%%%%%%%%%%%%%%%%%%%%%%%%%%%%%%%%%%%%%%%%%%%%%%%%%%%%%%%%%%%%%%%%%%%%%%%%%%%%%
%%%%%%%%%%%%%%%%%%%%%%%%%%%%%%%%%%%%%%%%%%%%%%%%%%%%%%%%%%%%%%%%%%%%%%%%%%%%%%%%
\appendix

\settowidth\MacroIndent{\rmfamily\scriptsize 000\ }

 \DocInput{childdoc.dtx}

\end{document}
%</driver>
% \fi
%
% %%%%%%%%%%%%%%%%%%%%%%%%%%%%%%%%%%%%%%%%%%%%%%%%%%%%%%%%%%%%%%%%%%%%%%%%%%%%%%
% %%%%%%%%%%%%%%%%%%%%%%%%%%%%%%%%%%%%%%%%%%%%%%%%%%%%%%%%%%%%%%%%%%%%%%%%%%%%%%
% \section{Sample}
%\iffalse
%<*samplemain>
%\fi
%
% The following presents a sample document
% with two chapters, two parts, a title page,
% a compile flag as well as three forwarding files to set the flag.
% It consists of eight |.tex| files:
% \begin{center}
% \begin{tabular}{ll}
% |cdocsamp.tex|&main file\\
% |cdocsch1.tex|&include file for chapter 1\\
% |cdocsch2.tex|&include file for chapter 2\\
% |cdocspt3.tex|&include file for part 3\\
% |cdocspt4.tex|&include file for part 4\\
% |cdocsdrf.tex|&forwarding file for main file in draft mode\\
% |cdocsfi1.tex|&forwarding file for final version of chapter 1\\
% |cdocsfi2.tex|&forwarding file for final version of chapter 2\\
% \end{tabular}
% \end{center}
% Each of the eight files can be compiled directly by the \LaTeX{} compiler.
%
% %%%%%%%%%%%%%%%%%%%%%%%%%%%%%%%%%%%%%%
% \paragraph{Main File.}
%
% The main file is called |cdocsamp.tex|.
%
% Load the \textsf{childdoc} definitions and
% declare the filename for the main document:
%    \begin{macrocode}
\input{childdoc.def}
\childdocmain{}
%    \end{macrocode}

% Optional override for |\version| flag:
%    \begin{macrocode}
%%\ifchilddoc\else\providecommand{\version}{draft}\fi
%    \end{macrocode}

% Define the default values for the |\version| flag
% (|final| for the main file and |draft| for childs):
%    \begin{macrocode}
\ifchilddoc
\providecommand{\version}{draft}
\else
\providecommand{\version}{final}
\fi
%    \end{macrocode}

% Load the standard document class:
%    \begin{macrocode}
\documentclass[12pt]{article}
%    \end{macrocode}

% Start the document body:
%    \begin{macrocode}
\begin{document}
%    \end{macrocode}

% Declare a title page.
% Print title, part of document being processed and version flag:
%    \begin{macrocode}
\addtocounter{page}{-1}
\begin{center}
{\LARGE\bfseries{}childdoc example\par}
\vspace{1cm}
\ifchilddoc
\ifchilddocmanual part\else chapter\fi:
`\childdocname' of `\childdocjob'\par
\else
main document: `\childdocjob'\par
\fi
version: \version\par
\end{center}
\newpage
%    \end{macrocode}

% Manually include selected file,
% otherwise process as usual:
%    \begin{macrocode}
\ifchilddocmanual
\section*{part `\childdocname'}
\input{\childdocname}
\else
%    \end{macrocode}

% Include the two chapters:
%    \begin{macrocode}
\include{cdocsch1}
\include{cdocsch2}
%    \end{macrocode}

% Include the two parts unless only chapters should be displayed:
%    \begin{macrocode}
\ifchilddoc\else
\section{part three}
\input{cdocspt3}
\section{part four}
\input{cdocspt4}
\fi
%    \end{macrocode}

% Process as usual until here:
%    \begin{macrocode}
\fi
%    \end{macrocode}

% End of document body:
%    \begin{macrocode}
\end{document}
%    \end{macrocode}
%\iffalse
%</samplemain>
%\fi
%
% %%%%%%%%%%%%%%%%%%%%%%%%%%%%%%%%%%%%%%
% \paragraph{Chapter Include Files.}
%
% The include files are called |cdocsch1.tex| and |cdocsch2.tex|.
%
%\iffalse
%<*samplechap1|samplechap2>
%\fi

% Optional override for |\version| flag:
%    \begin{macrocode}
%%\providecommand{\version}{final}
%    \end{macrocode}

% Include the main document:
%    \begin{macrocode}
\input{childdoc.def}
\childdocof{cdocsamp}
%    \end{macrocode}

%\iffalse
%</samplechap1|samplechap2>
%\fi
%
%\iffalse
%<*samplechap1>
%\fi
% Some text for chapter 1:
%    \begin{macrocode}
\section{one}
some text in chapter one
%    \end{macrocode}

%\iffalse
%</samplechap1>
%\fi
% Some text for chapter 2:
%\iffalse
%<*samplechap2>
%\fi
%    \begin{macrocode}
\section{two}
more text in chapter two
%    \end{macrocode}

%\iffalse
%</samplechap2>
%\fi
%
% %%%%%%%%%%%%%%%%%%%%%%%%%%%%%%%%%%%%%%
% \paragraph{Part Include Files.}
%
% The include files are called |cdocspt3.tex| and |cdocspt4.tex|.
%
%\iffalse
%<*samplepart3|samplepart4>
%\fi

% Optional override for |\version| flag:
%    \begin{macrocode}
%%\providecommand{\version}{final}
%    \end{macrocode}

% Include the main document:
%    \begin{macrocode}
\input{childdoc.def}
\childdocby{cdocsamp}
%    \end{macrocode}

%\iffalse
%</samplepart3|samplepart4>
%\fi
%
%\iffalse
%<*samplepart3>
%\fi
% Some text for part 3:
%    \begin{macrocode}
some text in part three
%    \end{macrocode}

%\iffalse
%</samplepart3>
%\fi
% Some text for part 4:
%\iffalse
%<*samplepart4>
%\fi
%    \begin{macrocode}
more text in part four
%    \end{macrocode}

%\iffalse
%</samplepart4>
%\fi
%
% %%%%%%%%%%%%%%%%%%%%%%%%%%%%%%%%%%%%%%
% \paragraph{Forwarding for a Complete Draft.}
%
% The following forwarding file |cdocsdrf.tex|
% compiles the main document in draft mode:
%\iffalse
%<*sampledraft>
%\fi
%    \begin{macrocode}
\def\version{draft}
\input{childdoc.def}
\childdocforward{cdocsamp}
%    \end{macrocode}

%\iffalse
%</sampledraft>
%\fi
%
% %%%%%%%%%%%%%%%%%%%%%%%%%%%%%%%%%%%%%%
% \paragraph{Forwarding for Final Version of the Chapters.}
%
% The following forwarding files |cdocsfn1.tex| and |cdocsfn2.tex|
% (with identical content)
% compile the final versions of the child documents
% |cdocsch1.tex| and |cdocsch2.tex|, respectively:
%\iffalse
%<*samplefinal>
%\fi
%    \begin{macrocode}
\def\version{final}
\input{childdoc.def}
\childdocforwardprefix[cdocsamp]{cdocsfn}{cdocsch}
%    \end{macrocode}

%\iffalse
%</samplefinal>
%\fi
%
% %%%%%%%%%%%%%%%%%%%%%%%%%%%%%%%%%%%%%%
% \paragraph{Command Line Processing.}
%
% The following three command lines generate the output files
% |cdocscld|, |cdocscl1| and |cdocscl2|
% which should be identical to
% |cdocsdrf|, |cdocsch1| and |cdocsfn2|, respectively:
% \begin{center}
% \begin{tabular}{l}
% |latex -jobname cdocscld \|\\
% |  "\def\version{draft}\input{childdoc.def}\childdocforward{cdocsamp}"|\\
% |latex -jobname cdocscl1 \|\\
% |  "\input{childdoc.def}\childdocforward[cdocsamp]{cdocsch1}"|\\
% |latex -jobname cdocscl2 \|\\
% |  "\def\version{final}\input{childdoc.def}\childdocforward{cdocsch2}"|
% \end{tabular}
% \end{center}
% Note that the trailing backslash on each first line
% merely continues the input to the second line
% (for convenient cut ant paste).
% Furthermore, the command |latex| can be replaced by any
% of its alternative versions such as |pdflatex|.
%
% %%%%%%%%%%%%%%%%%%%%%%%%%%%%%%%%%%%%%%%%%%%%%%%%%%%%%%%%%%%%%%%%%%%%%%%%%%%%%%
% %%%%%%%%%%%%%%%%%%%%%%%%%%%%%%%%%%%%%%%%%%%%%%%%%%%%%%%%%%%%%%%%%%%%%%%%%%%%%%
% \section{Implementation}
%\iffalse
%<*package>
%\fi
%
% This section describes the definitions file |childdoc.def|.

% The definitions cannot be loaded using |\usepackage| or |\RequirePackage|
% which has a mechanism to prevent loading a style file more than once.
% When loading the definitions by means of |\input|
% multiple instances have to be prevented manually:
%\iffalse
%This code needs to be before the `\ProvidesFile' directive
%which is defined at the beginning of this file.
%Therefore it is also placed there and commented out here.
%</package>
%<*discard>
%\fi
%    \begin{macrocode}
\ifdefined\childdocmain\endinput\fi
%    \end{macrocode}
%\iffalse
%</discard>
%<*package>
%\fi
%
% \macro{\ifchilddoc}
% \macro{\ifchilddocmanual}
% The conditional |\ifchilddoc| tells whether a
% child (true) or main (false) document is being compiled.
% The conditional |\ifchilddocmanual| tells whether
% the |\includeonly| mechanism is used (false) or
% the selection of child files must be performed manually (true).
% The definitions initialise to false:
%    \begin{macrocode}
\newif\ifchilddoc
\newif\ifchilddocmanual
%    \end{macrocode}

% \macro{\childdocname}
% \macro{\childdocjob}
% The macro |\childdocname| stores the name of the main document
% to be compiled. The macro |\childdocjob| stores the name of
% the document on which the \LaTeX{} compiler was originally invoked.
% The content of |\jobname| cannot be compared
% to filenames specified in the source due to different catcodes.
% The following code rescans |\jobname|, stores the result
% in |\childdocname| and saves a copy in |\childdocjob|:
%    \begin{macrocode}
\edef\childdocname{\scantokens\expandafter{\jobname\noexpand}}
\let\childdocjob\childdocname
%    \end{macrocode}

% \macro{\childdocdisable}
% The macro |\childdocdisable| prevents the main file
% from being processed more than once.
% At this stage, the main document command |\childdocmain|
% is assumed to be called once again where it should do nothing.
% Any subsequent call to it should prevent
% a secondary processing of the main document
% It overwrites the forwarding commands
% |\childdocof| and |\childdocforward|
% with empty macros to prevent further inclusions of the main document:
%    \begin{macrocode}
\newcommand{\childdocdisable}
{
  \renewcommand{\childdocmain}[1]{\renewcommand{\childdocmain}[1]{\endinput}}
  \renewcommand{\childdocof}[1]{}
  \renewcommand{\childdocby}[2][]{}
  \renewcommand{\childdocforward}[2][]{}
  \renewcommand{\childdocdisable}{}
}
%    \end{macrocode}

% \macro{\childdocmain}
% The macro |\childdocmain| is to be called at the top of the main file
% with nothing or the main filename (without extension) as argument.
% First, it breaks loops.
% If the argument is not empty and does not match |\childdocname|
% (which is set by the first inclusion of |childdoc.def|),
% |\ifchilddoc| is set to true, |\includeonly| is applied to the child file
% and |\jobname| is set to the main file
% (for proper handling of |.aux| files):
%    \begin{macrocode}
\newcommand{\childdocmain}[1]
{
  \childdocdisable\childdocmain{}
  \if?#1?\else
    \begingroup
      \def\childdoctmp{#1}
      \ifx\childdoctmp\childdocname
        \def\childdoctmp{}
      \else
        \def\childdoctmp
        {
          \childdoctrue
          \includeonly{\childdocname}
          \def\childdocjob{#1}
          \def\jobname{#1}
        }
      \fi
      \expandafter
    \endgroup
    \childdoctmp
  \fi
}
%    \end{macrocode}

% \macro{\childdocof}
% The command |\childdocof| redirects
% compilation to the main file |#1|.
%    \begin{macrocode}
\newcommand{\childdocof}[1]
{
  \childdocdisable
  \childdoctrue
  \includeonly{\childdocname}
  \def\jobname{#1}
  \def\childdocjob{#1}
  \input{#1}
}
%    \end{macrocode}

% \macro{\childdocby}
% The command |\childdocby| ....
%    \begin{macrocode}
\newcommand{\childdocby}[2][]
{
  \childdocdisable
  \childdoctrue
  \childdocmanualtrue
  \if?#1?\else
    \def\jobname{#2}
  \fi
  \def\childdocjob{#2}
  \input{#2}
  \endinput
}
%    \end{macrocode}

% \macro{\childdocforward}
% The command |\childdocforward| redirects
% compilation to the main file or
% (if the optional argument is given) a child file.
% Parameters are set as if the main file
% or a child file starting with |\childdocof| was compiled.
% Then compilation is handed over to the main file:
%    \begin{macrocode}
\newcommand{\childdocforward}[2][]
{
  \begingroup
    \if?#1?
      \def\childdoctmp
      {
        \def\childdocname{#2}
        \def\childdocjob{#2}
        \def\jobname{#2}
        \input{#2}
        \endinput
      }
    \else
      \def\childdoctmp
      {
        \childdocdisable
        \def\childdocname{#2}
        \childdoctrue
        \includeonly{#2}
        \def\childdocjob{#1}
        \def\jobname{#1}
        \input{#1}
        \endinput
      }
    \fi
    \expandafter
  \endgroup
  \childdoctmp
}
%    \end{macrocode}

% \macro{\childdocforwardprefix}
% The command |\childdocforwardprefix| redirects
% compilation to the main or a child file by means of a pattern.
% The prefix |#1| in the current filename is replaced by |#2|
% and the suffix of the current filename is kept
% (it is assumed that the filename does not contain the substring `|~~~|'
% which is used as a delimiter).
% Compilation is handed over to the new file by |\childdocforward|:
%    \begin{macrocode}
\newcommand{\childdocforwardprefix}[3][]
{
  \begingroup
    \def\childdocextract #2##1~~~{\def\childdoctmp{\childdocforward[#1]{#3##1}}}
    \expandafter\childdocextract\childdocname~~~
    \expandafter
  \endgroup
  \childdoctmp
}
%    \end{macrocode}

% \macro{\childdoc}
% The deprecated macro |\childdoc| is a legacy version of |\childdocmain|:
%    \begin{macrocode}
\newcommand{\childdoc}{\childdocmain}
%    \end{macrocode}

% \macro{\childdocredirect}
% The deprecated macro |\childdocredirect| is a legacy version
% of |\childdocforward| and |\childdocforwardprefix|:
%    \begin{macrocode}
\newcommand{\childdocredirect}[2][]
{
  \begingroup
    \if?#1?
      \def\childdoctmp{\childdocforward{#2}}
    \else
      \def\childdoctmp{\childdocforwardprefix{#1}{#2}}
    \fi
    \expandafter
  \endgroup
  \childdoctmp
}
%    \end{macrocode}

%\iffalse
%</package>
%\fi
%
\endinput

\childdocforward{cdocsamp}
%    \end{macrocode}

%\iffalse
%</sampledraft>
%\fi
%
% %%%%%%%%%%%%%%%%%%%%%%%%%%%%%%%%%%%%%%
% \paragraph{Forwarding for Final Version of the Chapters.}
%
% The following forwarding files |cdocsfn1.tex| and |cdocsfn2.tex|
% (with identical content)
% compile the final versions of the child documents
% |cdocsch1.tex| and |cdocsch2.tex|, respectively:
%\iffalse
%<*samplefinal>
%\fi
%    \begin{macrocode}
\def\version{final}
% \iffalse
%
% childdoc.dtx Copyright (C) 2017-2018 Niklas Beisert
%
% This work may be distributed and/or modified under the
% conditions of the LaTeX Project Public License, either version 1.3
% of this license or (at your option) any later version.
% The latest version of this license is in
%   http://www.latex-project.org/lppl.txt
% and version 1.3 or later is part of all distributions of LaTeX
% version 2005/12/01 or later.
%
% This work has the LPPL maintenance status `maintained'.
%
% The Current Maintainer of this work is Niklas Beisert.
%
% This work consists of the files childdoc.dtx and childdoc.ins
% and the derived files childdoc.def and cdocsamp.tex with
% cdocsch1.tex, cdocsch2.tex, cdocsdrf.tex, cdocsfn1.tex, cdocsfn2.tex.
%
%<package>\ifdefined\childdocmain\endinput\fi
%<package>\ProvidesFile{childdoc.def}[2018/12/30 v2.0 child document driver]
%<samplemain>\ProvidesFile{cdocsamp.tex}[2018/12/30 v2.0 sample for childdoc]
%<*driver>
%\ProvidesFile{childdoc.drv}[2018/12/30 v2.0 childdoc reference manual file]
\PassOptionsToClass{10pt,a4paper}{article}
\documentclass{ltxdoc}

\usepackage[margin=35mm]{geometry}
\usepackage{hyperref}
\usepackage{hyperxmp}
\usepackage[usenames]{color}

\hypersetup{colorlinks=true}
\hypersetup{pdfstartview=FitH}
\hypersetup{pdfpagemode=UseNone}
\hypersetup{pdfsource={}}
\hypersetup{pdflang={en-UK}}
\hypersetup{pdfcopyright={Copyright 2017-2018 Niklas Beisert.
  This work may be distributed and/or modified under the
  conditions of the LaTeX Project Public License, either version 1.3
  of this license or (at your option) any later version.}}
\hypersetup{pdflicenseurl={http://www.latex-project.org/lppl.txt}}
\hypersetup{pdfcontactaddress={ETH Zurich, ITP, HIT K,
  Wolfgang-Pauli-Strasse 27}}
\hypersetup{pdfcontactpostcode={8093}}
\hypersetup{pdfcontactcity={Zurich}}
\hypersetup{pdfcontactcountry={Switzerland}}
\hypersetup{pdfcontactemail={nbeisert@itp.phys.ethz.ch}}
\hypersetup{pdfcontacturl={http://people.phys.ethz.ch/\xmptilde nbeisert/}}

\newcommand{\secref}[1]{\hyperref[#1]{section \ref*{#1}}}

\parskip1ex
\parindent0pt
\let\olditemize\itemize
\def\itemize{\olditemize\parskip0pt}

\begin{document}

\title{The \textsf{childdoc} Package}
\hypersetup{pdftitle={The childdoc Package}}
\author{Niklas Beisert\\[2ex]
  Institut f\"ur Theoretische Physik\\
  Eidgen\"ossische Technische Hochschule Z\"urich\\
  Wolfgang-Pauli-Strasse 27, 8093 Z\"urich, Switzerland\\[1ex]
  \href{mailto:nbeisert@itp.phys.ethz.ch}
  {\texttt{nbeisert@itp.phys.ethz.ch}}}
\hypersetup{pdfauthor={Niklas Beisert}}
\hypersetup{pdfsubject={Manual for the LaTeX2e Package childdoc}}
\date{30 December 2018, \textsf{v2.0}}
\maketitle

\begin{abstract}\noindent
\textsf{childdoc} is a \LaTeXe{} package
that enables the direct compilation
of document sections included by |\include|
to individual files.
\end{abstract}

\begingroup
\parskip0ex
\tableofcontents
\endgroup

%%%%%%%%%%%%%%%%%%%%%%%%%%%%%%%%%%%%%%%%%%%%%%%%%%%%%%%%%%%%%%%%%%%%%%%%%%%%%%%%
%%%%%%%%%%%%%%%%%%%%%%%%%%%%%%%%%%%%%%%%%%%%%%%%%%%%%%%%%%%%%%%%%%%%%%%%%%%%%%%%
\section{Introduction}

\LaTeX{} provides a mechanism to structure a large document (such as a book)
into a main file and several child files (containing the chapters)
using the |\include| command.
This mechanism is beneficial for documents
which span hundreds of pages in order to
make the source file(s) more manageable.
Moreover, compilation can be restricted to
selected child files by means of the |\includeonly| command.
The latter feature can be used to reduce the compilation time while editing
(this was significantly more useful in the earlier days of \LaTeX{})
or to generate a smaller document which is easier to navigate.
Another application of |\includeonly| is to generate
documents consisting of selected parts of the complete document.

However, there are a few drawbacks of the plain |\include| mechanism:
\begin{itemize}
\item
The child files cannot be compiled on their own,
they can only be compiled via the main file.
A naive editing environment
(such as a text editor with an option
to have the current file processed by \LaTeX)
may require one to switch to the main file before compiling;
attempting to compile the child file produces errors.
\item
The main file must be modified (each time)
to adjust the |\includeonly| command
to the present needs. This easily leaves the main file in a messy state.
\item
The generated document will always carry the filename
of the main document. This is inconvenient if
several child files are to be compiled and
to be kept for distribution.
\end{itemize}

The present package provides a simple interface
to make child files individually compilable by \LaTeX{}.
Compiling a child file then has the same effect as compiling
the main file with an |\includeonly| command
to select the appropriate child.
Moreover the generated document will carry the name of the child
rather than the main file.
This resolves all three above issues.

This feature is meant to make the editing of books,
thesis documents and lecture notes somewhat more convenient.
However, the package can also be used efficiently for
composing a series of documents (such as exercise sheets)
which are typically distributed individually.
It then assists the author in generating the individual documents
(potentially in different versions)
as well as a document containing the collected series.
Another application is in developing style files
or other kinds of included material
where compilation of the style file could redirect
to a sample or test file.

%%%%%%%%%%%%%%%%%%%%%%%%%%%%%%%%%%%%%%%%%%%%%%%%%%%%%%%%%%%%%%%%%%%%%%%%%%%%%%%%
%%%%%%%%%%%%%%%%%%%%%%%%%%%%%%%%%%%%%%%%%%%%%%%%%%%%%%%%%%%%%%%%%%%%%%%%%%%%%%%%
\section{Usage}

First of all, the package \textsf{childdoc} is \emph{not} a standard
\LaTeXe{} |.sty| style file! Therefore it needs to be invoked in
a non-standard way.

%%%%%%%%%%%%%%%%%%%%%%%%%%%%%%%%%%%%%%%%%%%%%%%%%%%%%%%%%%%%%%%%%%%%%%%%%%%%%%%%
\subsection{Included Files}
\label{sec:include}

%%%%%%%%%%%%%%%%%%%%%%%%%%%%%%%%%%%%%%%%
\DescribeMacro{\childdocmain}
To use the package, add the commands
\begin{center}
\begin{tabular}{l}
|\input{childdoc.def}|\\
|\childdocmain{}|\\
\end{tabular}
\end{center}
at the very top of the main \LaTeX{} file,
in particular \emph{before} the |\documentclass| statement!
The argument of |\childdocmain| should be left empty
(but it must be present).

%%%%%%%%%%%%%%%%%%%%%%%%%%%%%%%%%%%%%%%%
\DescribeMacro{\childdocof}
Furthermore, add the commands
\begin{center}
\begin{tabular}{l}
|\input{childdoc.def}|\\
|\childdocof{|\textit{main}|}|\\
\end{tabular}
\end{center}
at the top of every child file \textit{child}
which is included by |\include{|\textit{child}|}|
from within the main file
(or at least for those files to be compiled individually).
The argument \textit{main} must be the filename of the main file.

There are a couple of
considerations in setting up the main and child documents:

%%%%%%%%%%%%%%%%%%%%%%%%%%%%%%%%%%%%%%%%
\paragraph{Restrictions.}

Please note the following restrictions:
\begin{itemize}
\item
|\childdocmain| must be called with one argument \textit{main}
to ensure compatibility with earlier version of the package.
It must either be empty (|\childdocmain{}|)
or precisely match the filename of the main file in which it is specified.
See \secref{sec:detection} for further information.
\item
The filename \textit{main} must be specified without the |.tex| extension.
\item
The filename \textit{main} is case sensitive
(even in case-insensitive file systems)
due to internal string comparison.
\item
The argument \textit{main} should be fully expanded, it cannot be a macro.
\item
Subdirectories and special characters should be avoided in filenames.
\item
The command |\childdocmain{|\textit{main}|}| must be followed by a whitespace.
It should not be followed immediately by another command
or by a comment mark `|%|'.
This is because the \TeX{} parser reads the token immediately following
the argument of |\childdocmain| and puts it
at the beginning of every child section;
however, a white\-space is ignored.
\end{itemize}

%%%%%%%%%%%%%%%%%%%%%%%%%%%%%%%%%%%%%%%%
\paragraph{Content of Main File.}

It is advisable to place all content in the child files included by |\include|.
Any output contained in the main file will appear in all child documents
unless suppressed manually;
it cannot be suppressed automatically by the |\includeonly| directive
and thus should normally be avoided.
A method to include some content in the main file
by means of conditional processing is described in \secref{sec:conditional}.

%%%%%%%%%%%%%%%%%%%%%%%%%%%%%%%%%%%%%%%%
\paragraph{Page Numbering.}

When only a part of the document is compiled,
the appropriate numbering of pages
(as well as other status parameters)
is determined from the |.aux| files.
The latter contain information from previous passes.
However this information needs to propagate through
all intermediate child documents.
Therefore the page numbering in child documents may well
be inconsistent until the complete document is compiled at least once.

A useful (if unconventional) way to always ensure a consistent
page numbering is to restart the numbering in each child document
and denote the pages by `\textit{child}|.|\textit{page}'
where \textit{child} represents the chapter/section number of the child file.
This can be achieved by the command
|\numberwithin{page}{|\textit{child}|}|
of the \textsf{amsmath} package
where \textit{child} can be |chapter| or |section|
depending on the chosen structuring.
Alternatively, one can modify the macro |\thepage| appropriately
and reset the counter |page| at the start of each child file.

%%%%%%%%%%%%%%%%%%%%%%%%%%%%%%%%%%%%%%%%%%%%%%%%%%%%%%%%%%%%%%%%%%%%%%%%%%%%%%%%
\subsection{Conditional Processing}
\label{sec:conditional}

The package provides a mechanism to compile different versions
of a document. To customise the versions further some conditional processing
can come in handy to distinguish which version is being compiled.
The package provides two macros to describe the compilation context:

%%%%%%%%%%%%%%%%%%%%%%%%%%%%%%%%%%%%%%%%
\DescribeMacro{\ifchilddoc}
The conditional |\ifchilddoc| distinguishes between the compilation of
child documents and the main document:
%
\begin{center}
|\ifchilddoc |\textit{child-code}| |[|\||else |\textit{main-code}]| \||fi|
\end{center}

%%%%%%%%%%%%%%%%%%%%%%%%%%%%%%%%%%%%%%%%
\DescribeMacro{\childdocname}
\DescribeMacro{\childdocjob}
The macro |\childdocname| contains the filename (without extension)
of the main or child file being processed.
Note that |\childdocjob| will always contain the name of the main file.

%%%%%%%%%%%%%%%%%%%%%%%%%%%%%%%%%%%%%%%%
\paragraph{Title Page.}

Conditional processing can be used to include a title or banner page
in the main document when proper precautions are taken.
Importantly, the code in the main file should ensure that the page counter
(as well as other status parameters which are stored in the |.aux| files)
takes the same value after the conditional processing.
Otherwise the page numbers may take divergent values
depending on which part is compiled.

For example, a title page could be declared by:
%
\begin{center}
\begin{tabular}{l}
|\ifchilddoc\||else|\\
|\addtocounter{page}{-1}|\\
\textit{code for title page}\\
|\newpage|\\
|\||fi|
\end{tabular}
\end{center}
%
A banner page for the child documents can be generated by:
%
\begin{center}
\begin{tabular}{l}
|\ifchilddoc|\\
|\addtocounter{page}{-1}|\\
\textit{code for banner page}\\
|\newpage|\\
|\||fi|
\end{tabular}
\end{center}
%
Here one could write a message such as:
\begin{center}
|This is the part \childdocname{} of \childdocjob{}.|
\end{center}

%%%%%%%%%%%%%%%%%%%%%%%%%%%%%%%%%%%%%%%%%%%%%%%%%%%%%%%%%%%%%%%%%%%%%%%%%%%%%%%%
\subsection{Flags}
\label{sec:flags}

The package makes it easy to generate different versions
of the main or child documents.
To this end compilation flags can be defined
and assigned different default values.
They will be particularly useful in conjunction
with the forwarding mechanism described in \secref{sec:forward}.

For example, it may be useful to have a flag |\version|
which can be set to |draft| or |final|.
The document source will contain some conditional code
depending on the value of |\version|.
Suppose further, the flag should default to |final| for the main file
and to |draft| for child files
which is a natural assignment for editing the document.
This is achieved by placing the following code
in the preamble of the main document
(below the |\childdocmain| directive):
%
\begin{center}
\begin{tabular}{l}
|\ifchilddoc|\\
|\providecommand{\version}{draft}|\\
|\||else|\\
|\providecommand{\version}{final}|\\
|\||fi|
\end{tabular}
\end{center}
%
The definition by |\providecommand| makes sure
that previous definitions are not overwritten.
Further statements |\providecommand{\version}{...}|
can thus be added before the above code to override it.

For the main file, one might add a line
(between |\childdocmain| and the above block)
%
\begin{center}
|%\ifchilddoc\||else\providecommand{\version}{draft}\||fi|
\end{center}
%
which can be uncommented to produce a draft version.
Likewise one can add a line to the very top of a child file
(above the |\childdocof{|\textit{main}|}| directive)
%
\begin{center}
|%\providecommand{\version}{final}|
\end{center}
%
which can be uncommented to produce the final version of this child document.

%%%%%%%%%%%%%%%%%%%%%%%%%%%%%%%%%%%%%%%%%%%%%%%%%%%%%%%%%%%%%%%%%%%%%%%%%%%%%%%%
\subsection{Forwarding}
\label{sec:forward}

Different versions of the main or child documents
using compilation flags as described in \secref{sec:flags}
can be (permanently) stored in different files
for convenient compilation, viewing and distribution.
To this end, the package defines a command
to pass on compilation to a different file:

%%%%%%%%%%%%%%%%%%%%%%%%%%%%%%%%%%%%%%%%
\DescribeMacro{\childdocforward}
The command |\childdocforward| redirects processing to
another source file:
%
\begin{center}
\begin{tabular}{l}
|\input{childdoc.def}|\\
|\childdocforward[|\textit{main}|]{|\textit{dest}|}|\\
\end{tabular}
\end{center}
%
The argument \textit{dest} is the destination file
(without extension).
It should be the main file or one of the child files.
Note that further \textsf{childdoc} directives
such as |\childdocof| and |\childdocforward|
in the indicated file will be processed in this form.
The optional argument \textit{main}
passes on directly to the main file \textit{main}
while pretending to compile the child \textit{dest}.
This form behaves as if \textit{dest}
issues |\childdocof{|\textit{main}|}| right away,
and no further \textsf{childdoc} directives will be processed.

%%%%%%%%%%%%%%%%%%%%%%%%%%%%%%%%%%%%%%%%
\DescribeMacro{\...prefix}
In the alternative form |\childdocforwardprefix|,
%
\begin{center}
\begin{tabular}{l}
|\input{childdoc.def}|\\
|\childdocforwardprefix[|\textit{main}|]{|\textit{prefix}|}{|\textit{dest}|}|
\end{tabular}
\end{center}
%
the destination file is determined by a pattern
depending on the current file:
To make this work, the current file must be called
`{\textit{prefix}\hspace{0.2em}\textit{suffix}}'
with \textit{prefix} matching precisely the argument.
Processing is then passed on to the file
`{\textit{dest}\hspace{0.2em}\textit{suffix}}'.
Surely, the same effect is achieved by
directly specifying the
argument `{\textit{dest}\hspace{0.2em}\textit{suffix}}'
in the first form.
However, that requires to set up a different file
for each child. With the alternative form of the command
all these files can have exactly the same content
which simplifies setting them up and maintaining them.

For example, the following file |draft.tex|
with a compilation flag |\version| as described in \secref{sec:flags}
compiles the main document as a draft:
%
\begin{center}
\begin{tabular}{l}
|\def\version{draft}|\\
|\input{childdoc.def}|\\
|\childdocforward{|\textit{main}|}|
\end{tabular}
\end{center}
%
Likewise, the following files |final|\textit{nn}|.tex|
compile the final version of the child document
|child|\textit{nn}|.tex|:
%
\begin{center}
\begin{tabular}{l}
|\def\version{final}|\\
|\input{childdoc.def}|\\
|\childdocforwardprefix{final}{child}|
\end{tabular}
\end{center}
%

Note that when several versions of a main file and/or of each child file
are to be generated, it may be convenient to set up a |Makefile| or
shell script to automatise the process.

%%%%%%%%%%%%%%%%%%%%%%%%%%%%%%%%%%%%%%%%%%%%%%%%%%%%%%%%%%%%%%%%%%%%%%%%%%%%%%%%
\subsection{Command Line Processing}
\label{sec:commandline}

The effect of redirection files can also be achieved by invoking
the \LaTeX{} compiler with a more elaborate command line.
Most conveniently this should be done as part
of a shell script or a |Makefile|.

When using \textsf{childdoc} in the main file, the following
command lines effectively perform a redirection
(note that depending on the shell being used,
backslashes may have to be doubled: `|\|' $\to$ `|\\|'):
%
\begin{center}
|... -jobname "|\textit{target}|" |\\|"|[\textit{flags}]%
|\input{childdoc.def}\childdocforward[|\textit{main}|]{|\textit{dest}|}"|
\end{center}
%
Here \textit{target} is the name of the output file,
\textit{main} is the name of the main file
and \textit{dest} is the name of the main or child file to be processed
(all filenames without extensions).
The optional argument \textit{main} can be omitted
if \textit{main} matches \textit{dest}.
Optionally, compilation \textit{flags} can be defined via |\def| commands.
This command line makes the \TeX{} engine believe
it is compiling the file \textit{target}
whose content is specified as the latter parameter.
The provided code then forwards the processing to
\textit{main} or \textit{dest} as described in \secref{sec:forward}.

%%%%%%%%%%%%%%%%%%%%%%%%%%%%%%%%%%%%%%%%%%%%%%%%%%%%%%%%%%%%%%%%%%%%%%%%%%%%%%%%
\subsection{Include by Input}
\label{sec:input}

Including child documents by |\include| has some restrictions by design.
Most notably, the content of a child document always occupies
its own set of pages; pages cannot be shared between child documents.
Usually, this behaviour makes perfect sense
because each child document contain an essential part of the document.
However, in some situations it may be desirable to compose
a document from a collection of parts
without having mandatory page breaks between then.
For this case, the package
provides a mechanism to include parts
by |\input| which can also be processed individually.
However, by construction this mechanism
requires manual handling of the content to be output.

%%%%%%%%%%%%%%%%%%%%%%%%%%%%%%%%%%%%%%%%
\DescribeMacro{\ifchilddocmanual}
The main file should be prepared as usual, see \secref{sec:include}.
However, the document body must make a distinction
between processing of an individual part and of the main document, e.g.:
%
\begin{center}
\begin{tabular}{l}
|\ifchilddocmanual|\\
|\input{\childdocname}|\\
|\||else|\\
\textit{document body with }|\input{|\textit{part}|}|\\
|\||fi|
\end{tabular}
\end{center}
%
The conditional |\ifchilddocmanual| is true whenever
a part to be included by |\input| is being compiled,
and the name of the part is stored in |\childdocname|.

%%%%%%%%%%%%%%%%%%%%%%%%%%%%%%%%%%%%%%%%
\DescribeMacro{\childdocby}
Each part to be included by |\input| should start with:
%
\begin{center}
\begin{tabular}{l}
|\input{childdoc.def}|\\
|\childdocby{|\textit{main}|}|\\
\end{tabular}
\end{center}
%
The directive |\childdocby| is similar to |\childdocof|
described in \secref{sec:include},
but the subsequent selection of content must be done manually.
To that end, both |\ifchilddoc| and |\ifchilddocmanual|
will be true upon processing of a part,
and the name of the part is stored in |\childdocname|.
Note that |\jobname| will be set to the filename of the current part
so that each part receives an individual |.aux| file
that does not interfere with the |.aux| file(s) of the main document.
This behaviour can be altered by the alternative form
|\childdocby[*]{|\textit{main}|}| (with a non-empty optional argument)
which uses the |.aux| file of the main document
by setting |\jobname| to \textit{main}.

%%%%%%%%%%%%%%%%%%%%%%%%%%%%%%%%%%%%%%%%%%%%%%%%%%%%%%%%%%%%%%%%%%%%%%%%%%%%%%%%
\subsection{Driver Development}
\label{sec:driver}

The \textsf{childdoc} mechanism can also be use for the development
of definition files such as \LaTeX{} styles or classes.
This case differs from the above setup with multiple parts
included by |\include| in that no |\includeonly| should be invoked.
This can be achieved by starting the include file
(before |\ProvidesPackage|) with:
%
\begin{center}
\begin{tabular}{l}
|\input{childdoc.def}|\\
|\childdocforward{|\textit{main}|}|\\
\end{tabular}
\end{center}
%
or alternatively with:
%
\begin{center}
\begin{tabular}{l}
|\input{childdoc.def}|\\
|\childdocby{|\textit{main}|}|\\
\end{tabular}
\end{center}
%
Both forms have slightly different effects as described above.
The main file is prepared as usual, see \secref{sec:include}.

%%%%%%%%%%%%%%%%%%%%%%%%%%%%%%%%%%%%%%%%%%%%%%%%%%%%%%%%%%%%%%%%%%%%%%%%%%%%%%%%
\subsection{Legacy Detection}
\label{sec:detection}

The directive |\childdocmain| in the main file can detect
whether the complete document or merely a child is to be compiled
even without using the directive |\childdocof|.
This method is deprecated because it is less robust
and there is no compelling reason to use it;
it is merely provided for backward compatibility
and it may be removed in future versions.

If the detection mechanism is to be used,
it is mandatory to correctly specify
the filename of the main file as the argument of |\childdocmain|:
%
\begin{center}
\begin{tabular}{l}
|\input{childdoc.def}|\\
|\childdocmain{|\textit{main}|}|\\
\end{tabular}
\end{center}
%
If |\jobname| does not match the argument \textit{main} of |\childdocmain|,
it is assumed that |\jobname| points to the child file to be compiled.
When using |\childdocmain| with the main file specified as argument,
it suffices to start a child file
with just |\input{|\textit{main}|}|
without loading of the package and using |\childdocof|.
If instead all processing is done
with the appropriate \textsf{childdoc} directives,
the argument of \textit{main} of |\childdocmain| can be empty.

An alternative version of the command line processing described
in \secref{sec:commandline} using the detection mechanism reads:
%
\begin{center}
|... -jobname "|\textit{target}|" "|[\textit{flags}]%
[|\def\jobname{|\textit{dest}|}|]|\input{|\textit{main}|}"|
\end{center}

%%%%%%%%%%%%%%%%%%%%%%%%%%%%%%%%%%%%%%%%%%%%%%%%%%%%%%%%%%%%%%%%%%%%%%%%%%%%%%%%
\subsection{Manual Code}
\label{sec:manual}

In case one cannot be certain whether the definitions file |childdoc.def|
is installed on the target \TeX{} distribution
and one prefers not to ship it,
it is conceivable to paste a few relevant commands into the sources.

To that end, drop all statements |\input{childdoc.def}|
and perform the replacements as outlined below.
Instead of |\childdocmain{|\textit{main}|}| add the following code
to the top of the main file:
%
\begin{center}
\begin{tabular}{l}
|\||ifdefined\childdocname\endinput\||fi\newif\ifchilddoc|\\
|\edef\childdocname{\scantokens\expandafter{\jobname\noexpand}}|\\
|\def\childdocmain{|\textit{main}|}\||ifx\childdocmain\childdocname\||else|\\
|\childdoctrue\includeonly{\childdocname}\let\jobname\childdocmain\||fi|\\
\end{tabular}
\end{center}
%
Instead of |\childdocof{|\textit{main}|}| just include the main file
at the top of each child file:
%
\begin{center}
|\input{|\textit{main}|}|
\end{center}
%
A simple redirection |\childdocforward{|\textit{dest}|}| is achieved by:
%
\begin{center}
|\def\jobname{|\textit{dest}|}\input{\jobname}|
\end{center}
%
The redirection with prefix
|\childdocforwardprefix[|\textit{prefix}|]{|\textit{dest}|}|
is accomplished by:
%
\begin{center}
\begin{tabular}{l}
|{\edef\jobname{\scantokens\expandafter{\jobname\noexpand}}|\\
|\def\redirectjob |\textit{prefix}|#1~~~{\gdef\jobname{|\textit{dest}|#1}}|\\
|\expandafter\redirectjob\jobname~~~}\input{\jobname}|
\end{tabular}
\end{center}

In an alternative approach,
child documents can be compiled by a specific command line
without additional code or specific definitions:
%
\begin{center}
|... -jobname "|\textit{target}|" "|[\textit{flags}]%
|\includeonly{|\textit{dest}|}\input{|\textit{main}|}"|
\end{center}
%

%%%%%%%%%%%%%%%%%%%%%%%%%%%%%%%%%%%%%%%%%%%%%%%%%%%%%%%%%%%%%%%%%%%%%%%%%%%%%%%%
%%%%%%%%%%%%%%%%%%%%%%%%%%%%%%%%%%%%%%%%%%%%%%%%%%%%%%%%%%%%%%%%%%%%%%%%%%%%%%%%
\section{Information}

%%%%%%%%%%%%%%%%%%%%%%%%%%%%%%%%%%%%%%%%%%%%%%%%%%%%%%%%%%%%%%%%%%%%%%%%%%%%%%%%
\subsection{Copyright}

Copyright \copyright{} 2017--2018 Niklas Beisert

This work may be distributed and/or modified under the
conditions of the \LaTeX{} Project Public License, either version 1.3
of this license or (at your option) any later version.
The latest version of this license is in
  \url{http://www.latex-project.org/lppl.txt}
and version 1.3 or later is part of all distributions of \LaTeX{}
version 2005/12/01 or later.

This work has the LPPL maintenance status `maintained'.

The Current Maintainer of this work is Niklas Beisert.

This work consists of the files |README.txt|, |childdoc.ins| and |childdoc.dtx|
as well as the derived files |childdoc.def|, |cdocsamp.tex|
with |cdocsch1.tex|, |cdocsch2.tex|, |cdocspt3.tex|, |cdocspt4.tex|,
|cdocsdrf.tex|, |cdocsfn1.tex|, |cdocsfn2.tex|
as well as |childdoc.pdf|.

%%%%%%%%%%%%%%%%%%%%%%%%%%%%%%%%%%%%%%%%%%%%%%%%%%%%%%%%%%%%%%%%%%%%%%%%%%%%%%%%
\subsection{Files and Installation}

The package consists of the files:
%
\begin{center}
\begin{tabular}{ll}
    |README.txt|   & readme file \\
    |childdoc.ins| & installation file \\
    |childdoc.dtx| & source file \\
    |childdoc.def| & definition file \\
    |cdocsamp.tex| & sample main file \\
    |cdocsch1.tex| & sample include file \\
    |cdocsch2.tex| & sample include file \\
    |cdocspt3.tex| & sample part file \\
    |cdocspt4.tex| & sample part file \\
    |cdocsdrf.tex| & sample redirection file \\
    |cdocsfn1.tex| & sample redirection file \\
    |cdocsfn2.tex| & sample redirection file \\
    |childdoc.pdf| & manual
\end{tabular}
\end{center}
%
The distribution consists of the files
|README.txt|, |childdoc.ins| and |childdoc.dtx|.
%
\begin{itemize}
\item
Run (pdf)\LaTeX{} on |childdoc.dtx|
to compile the manual |childdoc.pdf| (this file).
\item
Run \LaTeX{} on |childdoc.ins| to create the definitions file |childdoc.def|
and the sample |cdocsamp.tex| with include files
|cdocsch1.tex|, |cdocsch2.tex|, |cdocspt3.tex|, |cdocspt4.tex|,
|cdocsdrf.tex|, |cdocsfn1.tex|, |cdocsfn2.tex|.
Then copy the file |childdoc.def| to an appropriate directory of your \LaTeX{}
distribution, e.g.\ \textit{texmf-root}|/tex/latex/childdoc|.
\end{itemize}

%%%%%%%%%%%%%%%%%%%%%%%%%%%%%%%%%%%%%%%%%%%%%%%%%%%%%%%%%%%%%%%%%%%%%%%%%%%%%%%%
\subsection{Related CTAN Packages}

There are several other packages which offer a similar functionality:
%
\begin{itemize}
\item
The packages
\href{http://ctan.org/pkg/docmute}{\textsf{docmute}},
\href{http://ctan.org/pkg/includex}{\textsf{includex}} and
\href{http://ctan.org/pkg/standalone}{\textsf{standalone}}
provide commands to include only the document body of
a child file thus allowing both files to be compiled individually.
\item
The packages \href{http://ctan.org/pkg/subdocs}{\textsf{subdocs}}
and \href{http://ctan.org/pkg/subfiles}{\textsf{subfiles}}
provide structures in which the main and child documents can be
encapsulated and allowing them to be compiled individually.
The inclusion mechanism is different from the conventional |\include|.
\item
The package \href{http://ctan.org/pkg/combine}{\textsf{combine}}
is an elaborate solution to combine several documents into one.
\end{itemize}
%
See also the CTAN topic \href{http://ctan.org/topic/subdocs}{\textsf{subdocs}}
for further related packages.
The present package differs from the above solutions in that
a document structure constructed with the conventional |\include| mechanism
just needs two extra commands at the top of every file
such that all constituent files can be compiled individually.

%%%%%%%%%%%%%%%%%%%%%%%%%%%%%%%%%%%%%%%%%%%%%%%%%%%%%%%%%%%%%%%%%%%%%%%%%%%%%%%%
%\subsection{Feature Suggestions}
%
%The following is a list of features which may be useful for future
%versions of this package:
%%
%\begin{itemize}
%\item
%\ldots
%\end{itemize}

%%%%%%%%%%%%%%%%%%%%%%%%%%%%%%%%%%%%%%%%%%%%%%%%%%%%%%%%%%%%%%%%%%%%%%%%%%%%%%%%
\subsection{Revision History}

%%%%%%%%%%%%%%%%%%%%%%%%%%%%%%%%%%%%%%%%
\paragraph{v2.0:} 2018/12/30

\begin{itemize}
\item
immediate forward processing
\item
added |\childdocby| mechanism
\item
manual restructured
\end{itemize}

%%%%%%%%%%%%%%%%%%%%%%%%%%%%%%%%%%%%%%%%
\paragraph{v1.6:} 2018/01/17

\begin{itemize}
\item
application for development of include files
\item
corrections to manual
\end{itemize}

%%%%%%%%%%%%%%%%%%%%%%%%%%%%%%%%%%%%%%%%
\paragraph{v1.5:} 2017/05/21

\begin{itemize}
\item
more complete structuring introduced
\item
|\childdocof| introduced
\item
|\childdoc| renamed to |\childdocmain|
\item
|\childredirect| renamed to |\childdocforward| and |\childdocforwardprefix|
and functionality expanded
\end{itemize}

%%%%%%%%%%%%%%%%%%%%%%%%%%%%%%%%%%%%%%%%
\paragraph{v1.0:} 2017/04/27

\begin{itemize}
\item
manual and install package
\item
first version published on CTAN
\end{itemize}

%%%%%%%%%%%%%%%%%%%%%%%%%%%%%%%%%%%%%%%%
\paragraph{v0.6:} 2017/04/26

\begin{itemize}
\item
redirection mechanism added
\end{itemize}

%%%%%%%%%%%%%%%%%%%%%%%%%%%%%%%%%%%%%%%%
\paragraph{v0.5:} 2017/04/26

\begin{itemize}
\item
functionality in definition file
\end{itemize}


%%%%%%%%%%%%%%%%%%%%%%%%%%%%%%%%%%%%%%%%%%%%%%%%%%%%%%%%%%%%%%%%%%%%%%%%%%%%%%%%
%%%%%%%%%%%%%%%%%%%%%%%%%%%%%%%%%%%%%%%%%%%%%%%%%%%%%%%%%%%%%%%%%%%%%%%%%%%%%%%%
%%%%%%%%%%%%%%%%%%%%%%%%%%%%%%%%%%%%%%%%%%%%%%%%%%%%%%%%%%%%%%%%%%%%%%%%%%%%%%%%
\appendix

\settowidth\MacroIndent{\rmfamily\scriptsize 000\ }

 \DocInput{childdoc.dtx}

\end{document}
%</driver>
% \fi
%
% %%%%%%%%%%%%%%%%%%%%%%%%%%%%%%%%%%%%%%%%%%%%%%%%%%%%%%%%%%%%%%%%%%%%%%%%%%%%%%
% %%%%%%%%%%%%%%%%%%%%%%%%%%%%%%%%%%%%%%%%%%%%%%%%%%%%%%%%%%%%%%%%%%%%%%%%%%%%%%
% \section{Sample}
%\iffalse
%<*samplemain>
%\fi
%
% The following presents a sample document
% with two chapters, two parts, a title page,
% a compile flag as well as three forwarding files to set the flag.
% It consists of eight |.tex| files:
% \begin{center}
% \begin{tabular}{ll}
% |cdocsamp.tex|&main file\\
% |cdocsch1.tex|&include file for chapter 1\\
% |cdocsch2.tex|&include file for chapter 2\\
% |cdocspt3.tex|&include file for part 3\\
% |cdocspt4.tex|&include file for part 4\\
% |cdocsdrf.tex|&forwarding file for main file in draft mode\\
% |cdocsfi1.tex|&forwarding file for final version of chapter 1\\
% |cdocsfi2.tex|&forwarding file for final version of chapter 2\\
% \end{tabular}
% \end{center}
% Each of the eight files can be compiled directly by the \LaTeX{} compiler.
%
% %%%%%%%%%%%%%%%%%%%%%%%%%%%%%%%%%%%%%%
% \paragraph{Main File.}
%
% The main file is called |cdocsamp.tex|.
%
% Load the \textsf{childdoc} definitions and
% declare the filename for the main document:
%    \begin{macrocode}
\input{childdoc.def}
\childdocmain{}
%    \end{macrocode}

% Optional override for |\version| flag:
%    \begin{macrocode}
%%\ifchilddoc\else\providecommand{\version}{draft}\fi
%    \end{macrocode}

% Define the default values for the |\version| flag
% (|final| for the main file and |draft| for childs):
%    \begin{macrocode}
\ifchilddoc
\providecommand{\version}{draft}
\else
\providecommand{\version}{final}
\fi
%    \end{macrocode}

% Load the standard document class:
%    \begin{macrocode}
\documentclass[12pt]{article}
%    \end{macrocode}

% Start the document body:
%    \begin{macrocode}
\begin{document}
%    \end{macrocode}

% Declare a title page.
% Print title, part of document being processed and version flag:
%    \begin{macrocode}
\addtocounter{page}{-1}
\begin{center}
{\LARGE\bfseries{}childdoc example\par}
\vspace{1cm}
\ifchilddoc
\ifchilddocmanual part\else chapter\fi:
`\childdocname' of `\childdocjob'\par
\else
main document: `\childdocjob'\par
\fi
version: \version\par
\end{center}
\newpage
%    \end{macrocode}

% Manually include selected file,
% otherwise process as usual:
%    \begin{macrocode}
\ifchilddocmanual
\section*{part `\childdocname'}
\input{\childdocname}
\else
%    \end{macrocode}

% Include the two chapters:
%    \begin{macrocode}
\include{cdocsch1}
\include{cdocsch2}
%    \end{macrocode}

% Include the two parts unless only chapters should be displayed:
%    \begin{macrocode}
\ifchilddoc\else
\section{part three}
\input{cdocspt3}
\section{part four}
\input{cdocspt4}
\fi
%    \end{macrocode}

% Process as usual until here:
%    \begin{macrocode}
\fi
%    \end{macrocode}

% End of document body:
%    \begin{macrocode}
\end{document}
%    \end{macrocode}
%\iffalse
%</samplemain>
%\fi
%
% %%%%%%%%%%%%%%%%%%%%%%%%%%%%%%%%%%%%%%
% \paragraph{Chapter Include Files.}
%
% The include files are called |cdocsch1.tex| and |cdocsch2.tex|.
%
%\iffalse
%<*samplechap1|samplechap2>
%\fi

% Optional override for |\version| flag:
%    \begin{macrocode}
%%\providecommand{\version}{final}
%    \end{macrocode}

% Include the main document:
%    \begin{macrocode}
\input{childdoc.def}
\childdocof{cdocsamp}
%    \end{macrocode}

%\iffalse
%</samplechap1|samplechap2>
%\fi
%
%\iffalse
%<*samplechap1>
%\fi
% Some text for chapter 1:
%    \begin{macrocode}
\section{one}
some text in chapter one
%    \end{macrocode}

%\iffalse
%</samplechap1>
%\fi
% Some text for chapter 2:
%\iffalse
%<*samplechap2>
%\fi
%    \begin{macrocode}
\section{two}
more text in chapter two
%    \end{macrocode}

%\iffalse
%</samplechap2>
%\fi
%
% %%%%%%%%%%%%%%%%%%%%%%%%%%%%%%%%%%%%%%
% \paragraph{Part Include Files.}
%
% The include files are called |cdocspt3.tex| and |cdocspt4.tex|.
%
%\iffalse
%<*samplepart3|samplepart4>
%\fi

% Optional override for |\version| flag:
%    \begin{macrocode}
%%\providecommand{\version}{final}
%    \end{macrocode}

% Include the main document:
%    \begin{macrocode}
\input{childdoc.def}
\childdocby{cdocsamp}
%    \end{macrocode}

%\iffalse
%</samplepart3|samplepart4>
%\fi
%
%\iffalse
%<*samplepart3>
%\fi
% Some text for part 3:
%    \begin{macrocode}
some text in part three
%    \end{macrocode}

%\iffalse
%</samplepart3>
%\fi
% Some text for part 4:
%\iffalse
%<*samplepart4>
%\fi
%    \begin{macrocode}
more text in part four
%    \end{macrocode}

%\iffalse
%</samplepart4>
%\fi
%
% %%%%%%%%%%%%%%%%%%%%%%%%%%%%%%%%%%%%%%
% \paragraph{Forwarding for a Complete Draft.}
%
% The following forwarding file |cdocsdrf.tex|
% compiles the main document in draft mode:
%\iffalse
%<*sampledraft>
%\fi
%    \begin{macrocode}
\def\version{draft}
\input{childdoc.def}
\childdocforward{cdocsamp}
%    \end{macrocode}

%\iffalse
%</sampledraft>
%\fi
%
% %%%%%%%%%%%%%%%%%%%%%%%%%%%%%%%%%%%%%%
% \paragraph{Forwarding for Final Version of the Chapters.}
%
% The following forwarding files |cdocsfn1.tex| and |cdocsfn2.tex|
% (with identical content)
% compile the final versions of the child documents
% |cdocsch1.tex| and |cdocsch2.tex|, respectively:
%\iffalse
%<*samplefinal>
%\fi
%    \begin{macrocode}
\def\version{final}
\input{childdoc.def}
\childdocforwardprefix[cdocsamp]{cdocsfn}{cdocsch}
%    \end{macrocode}

%\iffalse
%</samplefinal>
%\fi
%
% %%%%%%%%%%%%%%%%%%%%%%%%%%%%%%%%%%%%%%
% \paragraph{Command Line Processing.}
%
% The following three command lines generate the output files
% |cdocscld|, |cdocscl1| and |cdocscl2|
% which should be identical to
% |cdocsdrf|, |cdocsch1| and |cdocsfn2|, respectively:
% \begin{center}
% \begin{tabular}{l}
% |latex -jobname cdocscld \|\\
% |  "\def\version{draft}\input{childdoc.def}\childdocforward{cdocsamp}"|\\
% |latex -jobname cdocscl1 \|\\
% |  "\input{childdoc.def}\childdocforward[cdocsamp]{cdocsch1}"|\\
% |latex -jobname cdocscl2 \|\\
% |  "\def\version{final}\input{childdoc.def}\childdocforward{cdocsch2}"|
% \end{tabular}
% \end{center}
% Note that the trailing backslash on each first line
% merely continues the input to the second line
% (for convenient cut ant paste).
% Furthermore, the command |latex| can be replaced by any
% of its alternative versions such as |pdflatex|.
%
% %%%%%%%%%%%%%%%%%%%%%%%%%%%%%%%%%%%%%%%%%%%%%%%%%%%%%%%%%%%%%%%%%%%%%%%%%%%%%%
% %%%%%%%%%%%%%%%%%%%%%%%%%%%%%%%%%%%%%%%%%%%%%%%%%%%%%%%%%%%%%%%%%%%%%%%%%%%%%%
% \section{Implementation}
%\iffalse
%<*package>
%\fi
%
% This section describes the definitions file |childdoc.def|.

% The definitions cannot be loaded using |\usepackage| or |\RequirePackage|
% which has a mechanism to prevent loading a style file more than once.
% When loading the definitions by means of |\input|
% multiple instances have to be prevented manually:
%\iffalse
%This code needs to be before the `\ProvidesFile' directive
%which is defined at the beginning of this file.
%Therefore it is also placed there and commented out here.
%</package>
%<*discard>
%\fi
%    \begin{macrocode}
\ifdefined\childdocmain\endinput\fi
%    \end{macrocode}
%\iffalse
%</discard>
%<*package>
%\fi
%
% \macro{\ifchilddoc}
% \macro{\ifchilddocmanual}
% The conditional |\ifchilddoc| tells whether a
% child (true) or main (false) document is being compiled.
% The conditional |\ifchilddocmanual| tells whether
% the |\includeonly| mechanism is used (false) or
% the selection of child files must be performed manually (true).
% The definitions initialise to false:
%    \begin{macrocode}
\newif\ifchilddoc
\newif\ifchilddocmanual
%    \end{macrocode}

% \macro{\childdocname}
% \macro{\childdocjob}
% The macro |\childdocname| stores the name of the main document
% to be compiled. The macro |\childdocjob| stores the name of
% the document on which the \LaTeX{} compiler was originally invoked.
% The content of |\jobname| cannot be compared
% to filenames specified in the source due to different catcodes.
% The following code rescans |\jobname|, stores the result
% in |\childdocname| and saves a copy in |\childdocjob|:
%    \begin{macrocode}
\edef\childdocname{\scantokens\expandafter{\jobname\noexpand}}
\let\childdocjob\childdocname
%    \end{macrocode}

% \macro{\childdocdisable}
% The macro |\childdocdisable| prevents the main file
% from being processed more than once.
% At this stage, the main document command |\childdocmain|
% is assumed to be called once again where it should do nothing.
% Any subsequent call to it should prevent
% a secondary processing of the main document
% It overwrites the forwarding commands
% |\childdocof| and |\childdocforward|
% with empty macros to prevent further inclusions of the main document:
%    \begin{macrocode}
\newcommand{\childdocdisable}
{
  \renewcommand{\childdocmain}[1]{\renewcommand{\childdocmain}[1]{\endinput}}
  \renewcommand{\childdocof}[1]{}
  \renewcommand{\childdocby}[2][]{}
  \renewcommand{\childdocforward}[2][]{}
  \renewcommand{\childdocdisable}{}
}
%    \end{macrocode}

% \macro{\childdocmain}
% The macro |\childdocmain| is to be called at the top of the main file
% with nothing or the main filename (without extension) as argument.
% First, it breaks loops.
% If the argument is not empty and does not match |\childdocname|
% (which is set by the first inclusion of |childdoc.def|),
% |\ifchilddoc| is set to true, |\includeonly| is applied to the child file
% and |\jobname| is set to the main file
% (for proper handling of |.aux| files):
%    \begin{macrocode}
\newcommand{\childdocmain}[1]
{
  \childdocdisable\childdocmain{}
  \if?#1?\else
    \begingroup
      \def\childdoctmp{#1}
      \ifx\childdoctmp\childdocname
        \def\childdoctmp{}
      \else
        \def\childdoctmp
        {
          \childdoctrue
          \includeonly{\childdocname}
          \def\childdocjob{#1}
          \def\jobname{#1}
        }
      \fi
      \expandafter
    \endgroup
    \childdoctmp
  \fi
}
%    \end{macrocode}

% \macro{\childdocof}
% The command |\childdocof| redirects
% compilation to the main file |#1|.
%    \begin{macrocode}
\newcommand{\childdocof}[1]
{
  \childdocdisable
  \childdoctrue
  \includeonly{\childdocname}
  \def\jobname{#1}
  \def\childdocjob{#1}
  \input{#1}
}
%    \end{macrocode}

% \macro{\childdocby}
% The command |\childdocby| ....
%    \begin{macrocode}
\newcommand{\childdocby}[2][]
{
  \childdocdisable
  \childdoctrue
  \childdocmanualtrue
  \if?#1?\else
    \def\jobname{#2}
  \fi
  \def\childdocjob{#2}
  \input{#2}
  \endinput
}
%    \end{macrocode}

% \macro{\childdocforward}
% The command |\childdocforward| redirects
% compilation to the main file or
% (if the optional argument is given) a child file.
% Parameters are set as if the main file
% or a child file starting with |\childdocof| was compiled.
% Then compilation is handed over to the main file:
%    \begin{macrocode}
\newcommand{\childdocforward}[2][]
{
  \begingroup
    \if?#1?
      \def\childdoctmp
      {
        \def\childdocname{#2}
        \def\childdocjob{#2}
        \def\jobname{#2}
        \input{#2}
        \endinput
      }
    \else
      \def\childdoctmp
      {
        \childdocdisable
        \def\childdocname{#2}
        \childdoctrue
        \includeonly{#2}
        \def\childdocjob{#1}
        \def\jobname{#1}
        \input{#1}
        \endinput
      }
    \fi
    \expandafter
  \endgroup
  \childdoctmp
}
%    \end{macrocode}

% \macro{\childdocforwardprefix}
% The command |\childdocforwardprefix| redirects
% compilation to the main or a child file by means of a pattern.
% The prefix |#1| in the current filename is replaced by |#2|
% and the suffix of the current filename is kept
% (it is assumed that the filename does not contain the substring `|~~~|'
% which is used as a delimiter).
% Compilation is handed over to the new file by |\childdocforward|:
%    \begin{macrocode}
\newcommand{\childdocforwardprefix}[3][]
{
  \begingroup
    \def\childdocextract #2##1~~~{\def\childdoctmp{\childdocforward[#1]{#3##1}}}
    \expandafter\childdocextract\childdocname~~~
    \expandafter
  \endgroup
  \childdoctmp
}
%    \end{macrocode}

% \macro{\childdoc}
% The deprecated macro |\childdoc| is a legacy version of |\childdocmain|:
%    \begin{macrocode}
\newcommand{\childdoc}{\childdocmain}
%    \end{macrocode}

% \macro{\childdocredirect}
% The deprecated macro |\childdocredirect| is a legacy version
% of |\childdocforward| and |\childdocforwardprefix|:
%    \begin{macrocode}
\newcommand{\childdocredirect}[2][]
{
  \begingroup
    \if?#1?
      \def\childdoctmp{\childdocforward{#2}}
    \else
      \def\childdoctmp{\childdocforwardprefix{#1}{#2}}
    \fi
    \expandafter
  \endgroup
  \childdoctmp
}
%    \end{macrocode}

%\iffalse
%</package>
%\fi
%
\endinput

\childdocforwardprefix[cdocsamp]{cdocsfn}{cdocsch}
%    \end{macrocode}

%\iffalse
%</samplefinal>
%\fi
%
% %%%%%%%%%%%%%%%%%%%%%%%%%%%%%%%%%%%%%%
% \paragraph{Command Line Processing.}
%
% The following three command lines generate the output files
% |cdocscld|, |cdocscl1| and |cdocscl2|
% which should be identical to
% |cdocsdrf|, |cdocsch1| and |cdocsfn2|, respectively:
% \begin{center}
% \begin{tabular}{l}
% |latex -jobname cdocscld \|\\
% |  "\def\version{draft}% \iffalse
%
% childdoc.dtx Copyright (C) 2017-2018 Niklas Beisert
%
% This work may be distributed and/or modified under the
% conditions of the LaTeX Project Public License, either version 1.3
% of this license or (at your option) any later version.
% The latest version of this license is in
%   http://www.latex-project.org/lppl.txt
% and version 1.3 or later is part of all distributions of LaTeX
% version 2005/12/01 or later.
%
% This work has the LPPL maintenance status `maintained'.
%
% The Current Maintainer of this work is Niklas Beisert.
%
% This work consists of the files childdoc.dtx and childdoc.ins
% and the derived files childdoc.def and cdocsamp.tex with
% cdocsch1.tex, cdocsch2.tex, cdocsdrf.tex, cdocsfn1.tex, cdocsfn2.tex.
%
%<package>\ifdefined\childdocmain\endinput\fi
%<package>\ProvidesFile{childdoc.def}[2018/12/30 v2.0 child document driver]
%<samplemain>\ProvidesFile{cdocsamp.tex}[2018/12/30 v2.0 sample for childdoc]
%<*driver>
%\ProvidesFile{childdoc.drv}[2018/12/30 v2.0 childdoc reference manual file]
\PassOptionsToClass{10pt,a4paper}{article}
\documentclass{ltxdoc}

\usepackage[margin=35mm]{geometry}
\usepackage{hyperref}
\usepackage{hyperxmp}
\usepackage[usenames]{color}

\hypersetup{colorlinks=true}
\hypersetup{pdfstartview=FitH}
\hypersetup{pdfpagemode=UseNone}
\hypersetup{pdfsource={}}
\hypersetup{pdflang={en-UK}}
\hypersetup{pdfcopyright={Copyright 2017-2018 Niklas Beisert.
  This work may be distributed and/or modified under the
  conditions of the LaTeX Project Public License, either version 1.3
  of this license or (at your option) any later version.}}
\hypersetup{pdflicenseurl={http://www.latex-project.org/lppl.txt}}
\hypersetup{pdfcontactaddress={ETH Zurich, ITP, HIT K,
  Wolfgang-Pauli-Strasse 27}}
\hypersetup{pdfcontactpostcode={8093}}
\hypersetup{pdfcontactcity={Zurich}}
\hypersetup{pdfcontactcountry={Switzerland}}
\hypersetup{pdfcontactemail={nbeisert@itp.phys.ethz.ch}}
\hypersetup{pdfcontacturl={http://people.phys.ethz.ch/\xmptilde nbeisert/}}

\newcommand{\secref}[1]{\hyperref[#1]{section \ref*{#1}}}

\parskip1ex
\parindent0pt
\let\olditemize\itemize
\def\itemize{\olditemize\parskip0pt}

\begin{document}

\title{The \textsf{childdoc} Package}
\hypersetup{pdftitle={The childdoc Package}}
\author{Niklas Beisert\\[2ex]
  Institut f\"ur Theoretische Physik\\
  Eidgen\"ossische Technische Hochschule Z\"urich\\
  Wolfgang-Pauli-Strasse 27, 8093 Z\"urich, Switzerland\\[1ex]
  \href{mailto:nbeisert@itp.phys.ethz.ch}
  {\texttt{nbeisert@itp.phys.ethz.ch}}}
\hypersetup{pdfauthor={Niklas Beisert}}
\hypersetup{pdfsubject={Manual for the LaTeX2e Package childdoc}}
\date{30 December 2018, \textsf{v2.0}}
\maketitle

\begin{abstract}\noindent
\textsf{childdoc} is a \LaTeXe{} package
that enables the direct compilation
of document sections included by |\include|
to individual files.
\end{abstract}

\begingroup
\parskip0ex
\tableofcontents
\endgroup

%%%%%%%%%%%%%%%%%%%%%%%%%%%%%%%%%%%%%%%%%%%%%%%%%%%%%%%%%%%%%%%%%%%%%%%%%%%%%%%%
%%%%%%%%%%%%%%%%%%%%%%%%%%%%%%%%%%%%%%%%%%%%%%%%%%%%%%%%%%%%%%%%%%%%%%%%%%%%%%%%
\section{Introduction}

\LaTeX{} provides a mechanism to structure a large document (such as a book)
into a main file and several child files (containing the chapters)
using the |\include| command.
This mechanism is beneficial for documents
which span hundreds of pages in order to
make the source file(s) more manageable.
Moreover, compilation can be restricted to
selected child files by means of the |\includeonly| command.
The latter feature can be used to reduce the compilation time while editing
(this was significantly more useful in the earlier days of \LaTeX{})
or to generate a smaller document which is easier to navigate.
Another application of |\includeonly| is to generate
documents consisting of selected parts of the complete document.

However, there are a few drawbacks of the plain |\include| mechanism:
\begin{itemize}
\item
The child files cannot be compiled on their own,
they can only be compiled via the main file.
A naive editing environment
(such as a text editor with an option
to have the current file processed by \LaTeX)
may require one to switch to the main file before compiling;
attempting to compile the child file produces errors.
\item
The main file must be modified (each time)
to adjust the |\includeonly| command
to the present needs. This easily leaves the main file in a messy state.
\item
The generated document will always carry the filename
of the main document. This is inconvenient if
several child files are to be compiled and
to be kept for distribution.
\end{itemize}

The present package provides a simple interface
to make child files individually compilable by \LaTeX{}.
Compiling a child file then has the same effect as compiling
the main file with an |\includeonly| command
to select the appropriate child.
Moreover the generated document will carry the name of the child
rather than the main file.
This resolves all three above issues.

This feature is meant to make the editing of books,
thesis documents and lecture notes somewhat more convenient.
However, the package can also be used efficiently for
composing a series of documents (such as exercise sheets)
which are typically distributed individually.
It then assists the author in generating the individual documents
(potentially in different versions)
as well as a document containing the collected series.
Another application is in developing style files
or other kinds of included material
where compilation of the style file could redirect
to a sample or test file.

%%%%%%%%%%%%%%%%%%%%%%%%%%%%%%%%%%%%%%%%%%%%%%%%%%%%%%%%%%%%%%%%%%%%%%%%%%%%%%%%
%%%%%%%%%%%%%%%%%%%%%%%%%%%%%%%%%%%%%%%%%%%%%%%%%%%%%%%%%%%%%%%%%%%%%%%%%%%%%%%%
\section{Usage}

First of all, the package \textsf{childdoc} is \emph{not} a standard
\LaTeXe{} |.sty| style file! Therefore it needs to be invoked in
a non-standard way.

%%%%%%%%%%%%%%%%%%%%%%%%%%%%%%%%%%%%%%%%%%%%%%%%%%%%%%%%%%%%%%%%%%%%%%%%%%%%%%%%
\subsection{Included Files}
\label{sec:include}

%%%%%%%%%%%%%%%%%%%%%%%%%%%%%%%%%%%%%%%%
\DescribeMacro{\childdocmain}
To use the package, add the commands
\begin{center}
\begin{tabular}{l}
|\input{childdoc.def}|\\
|\childdocmain{}|\\
\end{tabular}
\end{center}
at the very top of the main \LaTeX{} file,
in particular \emph{before} the |\documentclass| statement!
The argument of |\childdocmain| should be left empty
(but it must be present).

%%%%%%%%%%%%%%%%%%%%%%%%%%%%%%%%%%%%%%%%
\DescribeMacro{\childdocof}
Furthermore, add the commands
\begin{center}
\begin{tabular}{l}
|\input{childdoc.def}|\\
|\childdocof{|\textit{main}|}|\\
\end{tabular}
\end{center}
at the top of every child file \textit{child}
which is included by |\include{|\textit{child}|}|
from within the main file
(or at least for those files to be compiled individually).
The argument \textit{main} must be the filename of the main file.

There are a couple of
considerations in setting up the main and child documents:

%%%%%%%%%%%%%%%%%%%%%%%%%%%%%%%%%%%%%%%%
\paragraph{Restrictions.}

Please note the following restrictions:
\begin{itemize}
\item
|\childdocmain| must be called with one argument \textit{main}
to ensure compatibility with earlier version of the package.
It must either be empty (|\childdocmain{}|)
or precisely match the filename of the main file in which it is specified.
See \secref{sec:detection} for further information.
\item
The filename \textit{main} must be specified without the |.tex| extension.
\item
The filename \textit{main} is case sensitive
(even in case-insensitive file systems)
due to internal string comparison.
\item
The argument \textit{main} should be fully expanded, it cannot be a macro.
\item
Subdirectories and special characters should be avoided in filenames.
\item
The command |\childdocmain{|\textit{main}|}| must be followed by a whitespace.
It should not be followed immediately by another command
or by a comment mark `|%|'.
This is because the \TeX{} parser reads the token immediately following
the argument of |\childdocmain| and puts it
at the beginning of every child section;
however, a white\-space is ignored.
\end{itemize}

%%%%%%%%%%%%%%%%%%%%%%%%%%%%%%%%%%%%%%%%
\paragraph{Content of Main File.}

It is advisable to place all content in the child files included by |\include|.
Any output contained in the main file will appear in all child documents
unless suppressed manually;
it cannot be suppressed automatically by the |\includeonly| directive
and thus should normally be avoided.
A method to include some content in the main file
by means of conditional processing is described in \secref{sec:conditional}.

%%%%%%%%%%%%%%%%%%%%%%%%%%%%%%%%%%%%%%%%
\paragraph{Page Numbering.}

When only a part of the document is compiled,
the appropriate numbering of pages
(as well as other status parameters)
is determined from the |.aux| files.
The latter contain information from previous passes.
However this information needs to propagate through
all intermediate child documents.
Therefore the page numbering in child documents may well
be inconsistent until the complete document is compiled at least once.

A useful (if unconventional) way to always ensure a consistent
page numbering is to restart the numbering in each child document
and denote the pages by `\textit{child}|.|\textit{page}'
where \textit{child} represents the chapter/section number of the child file.
This can be achieved by the command
|\numberwithin{page}{|\textit{child}|}|
of the \textsf{amsmath} package
where \textit{child} can be |chapter| or |section|
depending on the chosen structuring.
Alternatively, one can modify the macro |\thepage| appropriately
and reset the counter |page| at the start of each child file.

%%%%%%%%%%%%%%%%%%%%%%%%%%%%%%%%%%%%%%%%%%%%%%%%%%%%%%%%%%%%%%%%%%%%%%%%%%%%%%%%
\subsection{Conditional Processing}
\label{sec:conditional}

The package provides a mechanism to compile different versions
of a document. To customise the versions further some conditional processing
can come in handy to distinguish which version is being compiled.
The package provides two macros to describe the compilation context:

%%%%%%%%%%%%%%%%%%%%%%%%%%%%%%%%%%%%%%%%
\DescribeMacro{\ifchilddoc}
The conditional |\ifchilddoc| distinguishes between the compilation of
child documents and the main document:
%
\begin{center}
|\ifchilddoc |\textit{child-code}| |[|\||else |\textit{main-code}]| \||fi|
\end{center}

%%%%%%%%%%%%%%%%%%%%%%%%%%%%%%%%%%%%%%%%
\DescribeMacro{\childdocname}
\DescribeMacro{\childdocjob}
The macro |\childdocname| contains the filename (without extension)
of the main or child file being processed.
Note that |\childdocjob| will always contain the name of the main file.

%%%%%%%%%%%%%%%%%%%%%%%%%%%%%%%%%%%%%%%%
\paragraph{Title Page.}

Conditional processing can be used to include a title or banner page
in the main document when proper precautions are taken.
Importantly, the code in the main file should ensure that the page counter
(as well as other status parameters which are stored in the |.aux| files)
takes the same value after the conditional processing.
Otherwise the page numbers may take divergent values
depending on which part is compiled.

For example, a title page could be declared by:
%
\begin{center}
\begin{tabular}{l}
|\ifchilddoc\||else|\\
|\addtocounter{page}{-1}|\\
\textit{code for title page}\\
|\newpage|\\
|\||fi|
\end{tabular}
\end{center}
%
A banner page for the child documents can be generated by:
%
\begin{center}
\begin{tabular}{l}
|\ifchilddoc|\\
|\addtocounter{page}{-1}|\\
\textit{code for banner page}\\
|\newpage|\\
|\||fi|
\end{tabular}
\end{center}
%
Here one could write a message such as:
\begin{center}
|This is the part \childdocname{} of \childdocjob{}.|
\end{center}

%%%%%%%%%%%%%%%%%%%%%%%%%%%%%%%%%%%%%%%%%%%%%%%%%%%%%%%%%%%%%%%%%%%%%%%%%%%%%%%%
\subsection{Flags}
\label{sec:flags}

The package makes it easy to generate different versions
of the main or child documents.
To this end compilation flags can be defined
and assigned different default values.
They will be particularly useful in conjunction
with the forwarding mechanism described in \secref{sec:forward}.

For example, it may be useful to have a flag |\version|
which can be set to |draft| or |final|.
The document source will contain some conditional code
depending on the value of |\version|.
Suppose further, the flag should default to |final| for the main file
and to |draft| for child files
which is a natural assignment for editing the document.
This is achieved by placing the following code
in the preamble of the main document
(below the |\childdocmain| directive):
%
\begin{center}
\begin{tabular}{l}
|\ifchilddoc|\\
|\providecommand{\version}{draft}|\\
|\||else|\\
|\providecommand{\version}{final}|\\
|\||fi|
\end{tabular}
\end{center}
%
The definition by |\providecommand| makes sure
that previous definitions are not overwritten.
Further statements |\providecommand{\version}{...}|
can thus be added before the above code to override it.

For the main file, one might add a line
(between |\childdocmain| and the above block)
%
\begin{center}
|%\ifchilddoc\||else\providecommand{\version}{draft}\||fi|
\end{center}
%
which can be uncommented to produce a draft version.
Likewise one can add a line to the very top of a child file
(above the |\childdocof{|\textit{main}|}| directive)
%
\begin{center}
|%\providecommand{\version}{final}|
\end{center}
%
which can be uncommented to produce the final version of this child document.

%%%%%%%%%%%%%%%%%%%%%%%%%%%%%%%%%%%%%%%%%%%%%%%%%%%%%%%%%%%%%%%%%%%%%%%%%%%%%%%%
\subsection{Forwarding}
\label{sec:forward}

Different versions of the main or child documents
using compilation flags as described in \secref{sec:flags}
can be (permanently) stored in different files
for convenient compilation, viewing and distribution.
To this end, the package defines a command
to pass on compilation to a different file:

%%%%%%%%%%%%%%%%%%%%%%%%%%%%%%%%%%%%%%%%
\DescribeMacro{\childdocforward}
The command |\childdocforward| redirects processing to
another source file:
%
\begin{center}
\begin{tabular}{l}
|\input{childdoc.def}|\\
|\childdocforward[|\textit{main}|]{|\textit{dest}|}|\\
\end{tabular}
\end{center}
%
The argument \textit{dest} is the destination file
(without extension).
It should be the main file or one of the child files.
Note that further \textsf{childdoc} directives
such as |\childdocof| and |\childdocforward|
in the indicated file will be processed in this form.
The optional argument \textit{main}
passes on directly to the main file \textit{main}
while pretending to compile the child \textit{dest}.
This form behaves as if \textit{dest}
issues |\childdocof{|\textit{main}|}| right away,
and no further \textsf{childdoc} directives will be processed.

%%%%%%%%%%%%%%%%%%%%%%%%%%%%%%%%%%%%%%%%
\DescribeMacro{\...prefix}
In the alternative form |\childdocforwardprefix|,
%
\begin{center}
\begin{tabular}{l}
|\input{childdoc.def}|\\
|\childdocforwardprefix[|\textit{main}|]{|\textit{prefix}|}{|\textit{dest}|}|
\end{tabular}
\end{center}
%
the destination file is determined by a pattern
depending on the current file:
To make this work, the current file must be called
`{\textit{prefix}\hspace{0.2em}\textit{suffix}}'
with \textit{prefix} matching precisely the argument.
Processing is then passed on to the file
`{\textit{dest}\hspace{0.2em}\textit{suffix}}'.
Surely, the same effect is achieved by
directly specifying the
argument `{\textit{dest}\hspace{0.2em}\textit{suffix}}'
in the first form.
However, that requires to set up a different file
for each child. With the alternative form of the command
all these files can have exactly the same content
which simplifies setting them up and maintaining them.

For example, the following file |draft.tex|
with a compilation flag |\version| as described in \secref{sec:flags}
compiles the main document as a draft:
%
\begin{center}
\begin{tabular}{l}
|\def\version{draft}|\\
|\input{childdoc.def}|\\
|\childdocforward{|\textit{main}|}|
\end{tabular}
\end{center}
%
Likewise, the following files |final|\textit{nn}|.tex|
compile the final version of the child document
|child|\textit{nn}|.tex|:
%
\begin{center}
\begin{tabular}{l}
|\def\version{final}|\\
|\input{childdoc.def}|\\
|\childdocforwardprefix{final}{child}|
\end{tabular}
\end{center}
%

Note that when several versions of a main file and/or of each child file
are to be generated, it may be convenient to set up a |Makefile| or
shell script to automatise the process.

%%%%%%%%%%%%%%%%%%%%%%%%%%%%%%%%%%%%%%%%%%%%%%%%%%%%%%%%%%%%%%%%%%%%%%%%%%%%%%%%
\subsection{Command Line Processing}
\label{sec:commandline}

The effect of redirection files can also be achieved by invoking
the \LaTeX{} compiler with a more elaborate command line.
Most conveniently this should be done as part
of a shell script or a |Makefile|.

When using \textsf{childdoc} in the main file, the following
command lines effectively perform a redirection
(note that depending on the shell being used,
backslashes may have to be doubled: `|\|' $\to$ `|\\|'):
%
\begin{center}
|... -jobname "|\textit{target}|" |\\|"|[\textit{flags}]%
|\input{childdoc.def}\childdocforward[|\textit{main}|]{|\textit{dest}|}"|
\end{center}
%
Here \textit{target} is the name of the output file,
\textit{main} is the name of the main file
and \textit{dest} is the name of the main or child file to be processed
(all filenames without extensions).
The optional argument \textit{main} can be omitted
if \textit{main} matches \textit{dest}.
Optionally, compilation \textit{flags} can be defined via |\def| commands.
This command line makes the \TeX{} engine believe
it is compiling the file \textit{target}
whose content is specified as the latter parameter.
The provided code then forwards the processing to
\textit{main} or \textit{dest} as described in \secref{sec:forward}.

%%%%%%%%%%%%%%%%%%%%%%%%%%%%%%%%%%%%%%%%%%%%%%%%%%%%%%%%%%%%%%%%%%%%%%%%%%%%%%%%
\subsection{Include by Input}
\label{sec:input}

Including child documents by |\include| has some restrictions by design.
Most notably, the content of a child document always occupies
its own set of pages; pages cannot be shared between child documents.
Usually, this behaviour makes perfect sense
because each child document contain an essential part of the document.
However, in some situations it may be desirable to compose
a document from a collection of parts
without having mandatory page breaks between then.
For this case, the package
provides a mechanism to include parts
by |\input| which can also be processed individually.
However, by construction this mechanism
requires manual handling of the content to be output.

%%%%%%%%%%%%%%%%%%%%%%%%%%%%%%%%%%%%%%%%
\DescribeMacro{\ifchilddocmanual}
The main file should be prepared as usual, see \secref{sec:include}.
However, the document body must make a distinction
between processing of an individual part and of the main document, e.g.:
%
\begin{center}
\begin{tabular}{l}
|\ifchilddocmanual|\\
|\input{\childdocname}|\\
|\||else|\\
\textit{document body with }|\input{|\textit{part}|}|\\
|\||fi|
\end{tabular}
\end{center}
%
The conditional |\ifchilddocmanual| is true whenever
a part to be included by |\input| is being compiled,
and the name of the part is stored in |\childdocname|.

%%%%%%%%%%%%%%%%%%%%%%%%%%%%%%%%%%%%%%%%
\DescribeMacro{\childdocby}
Each part to be included by |\input| should start with:
%
\begin{center}
\begin{tabular}{l}
|\input{childdoc.def}|\\
|\childdocby{|\textit{main}|}|\\
\end{tabular}
\end{center}
%
The directive |\childdocby| is similar to |\childdocof|
described in \secref{sec:include},
but the subsequent selection of content must be done manually.
To that end, both |\ifchilddoc| and |\ifchilddocmanual|
will be true upon processing of a part,
and the name of the part is stored in |\childdocname|.
Note that |\jobname| will be set to the filename of the current part
so that each part receives an individual |.aux| file
that does not interfere with the |.aux| file(s) of the main document.
This behaviour can be altered by the alternative form
|\childdocby[*]{|\textit{main}|}| (with a non-empty optional argument)
which uses the |.aux| file of the main document
by setting |\jobname| to \textit{main}.

%%%%%%%%%%%%%%%%%%%%%%%%%%%%%%%%%%%%%%%%%%%%%%%%%%%%%%%%%%%%%%%%%%%%%%%%%%%%%%%%
\subsection{Driver Development}
\label{sec:driver}

The \textsf{childdoc} mechanism can also be use for the development
of definition files such as \LaTeX{} styles or classes.
This case differs from the above setup with multiple parts
included by |\include| in that no |\includeonly| should be invoked.
This can be achieved by starting the include file
(before |\ProvidesPackage|) with:
%
\begin{center}
\begin{tabular}{l}
|\input{childdoc.def}|\\
|\childdocforward{|\textit{main}|}|\\
\end{tabular}
\end{center}
%
or alternatively with:
%
\begin{center}
\begin{tabular}{l}
|\input{childdoc.def}|\\
|\childdocby{|\textit{main}|}|\\
\end{tabular}
\end{center}
%
Both forms have slightly different effects as described above.
The main file is prepared as usual, see \secref{sec:include}.

%%%%%%%%%%%%%%%%%%%%%%%%%%%%%%%%%%%%%%%%%%%%%%%%%%%%%%%%%%%%%%%%%%%%%%%%%%%%%%%%
\subsection{Legacy Detection}
\label{sec:detection}

The directive |\childdocmain| in the main file can detect
whether the complete document or merely a child is to be compiled
even without using the directive |\childdocof|.
This method is deprecated because it is less robust
and there is no compelling reason to use it;
it is merely provided for backward compatibility
and it may be removed in future versions.

If the detection mechanism is to be used,
it is mandatory to correctly specify
the filename of the main file as the argument of |\childdocmain|:
%
\begin{center}
\begin{tabular}{l}
|\input{childdoc.def}|\\
|\childdocmain{|\textit{main}|}|\\
\end{tabular}
\end{center}
%
If |\jobname| does not match the argument \textit{main} of |\childdocmain|,
it is assumed that |\jobname| points to the child file to be compiled.
When using |\childdocmain| with the main file specified as argument,
it suffices to start a child file
with just |\input{|\textit{main}|}|
without loading of the package and using |\childdocof|.
If instead all processing is done
with the appropriate \textsf{childdoc} directives,
the argument of \textit{main} of |\childdocmain| can be empty.

An alternative version of the command line processing described
in \secref{sec:commandline} using the detection mechanism reads:
%
\begin{center}
|... -jobname "|\textit{target}|" "|[\textit{flags}]%
[|\def\jobname{|\textit{dest}|}|]|\input{|\textit{main}|}"|
\end{center}

%%%%%%%%%%%%%%%%%%%%%%%%%%%%%%%%%%%%%%%%%%%%%%%%%%%%%%%%%%%%%%%%%%%%%%%%%%%%%%%%
\subsection{Manual Code}
\label{sec:manual}

In case one cannot be certain whether the definitions file |childdoc.def|
is installed on the target \TeX{} distribution
and one prefers not to ship it,
it is conceivable to paste a few relevant commands into the sources.

To that end, drop all statements |\input{childdoc.def}|
and perform the replacements as outlined below.
Instead of |\childdocmain{|\textit{main}|}| add the following code
to the top of the main file:
%
\begin{center}
\begin{tabular}{l}
|\||ifdefined\childdocname\endinput\||fi\newif\ifchilddoc|\\
|\edef\childdocname{\scantokens\expandafter{\jobname\noexpand}}|\\
|\def\childdocmain{|\textit{main}|}\||ifx\childdocmain\childdocname\||else|\\
|\childdoctrue\includeonly{\childdocname}\let\jobname\childdocmain\||fi|\\
\end{tabular}
\end{center}
%
Instead of |\childdocof{|\textit{main}|}| just include the main file
at the top of each child file:
%
\begin{center}
|\input{|\textit{main}|}|
\end{center}
%
A simple redirection |\childdocforward{|\textit{dest}|}| is achieved by:
%
\begin{center}
|\def\jobname{|\textit{dest}|}\input{\jobname}|
\end{center}
%
The redirection with prefix
|\childdocforwardprefix[|\textit{prefix}|]{|\textit{dest}|}|
is accomplished by:
%
\begin{center}
\begin{tabular}{l}
|{\edef\jobname{\scantokens\expandafter{\jobname\noexpand}}|\\
|\def\redirectjob |\textit{prefix}|#1~~~{\gdef\jobname{|\textit{dest}|#1}}|\\
|\expandafter\redirectjob\jobname~~~}\input{\jobname}|
\end{tabular}
\end{center}

In an alternative approach,
child documents can be compiled by a specific command line
without additional code or specific definitions:
%
\begin{center}
|... -jobname "|\textit{target}|" "|[\textit{flags}]%
|\includeonly{|\textit{dest}|}\input{|\textit{main}|}"|
\end{center}
%

%%%%%%%%%%%%%%%%%%%%%%%%%%%%%%%%%%%%%%%%%%%%%%%%%%%%%%%%%%%%%%%%%%%%%%%%%%%%%%%%
%%%%%%%%%%%%%%%%%%%%%%%%%%%%%%%%%%%%%%%%%%%%%%%%%%%%%%%%%%%%%%%%%%%%%%%%%%%%%%%%
\section{Information}

%%%%%%%%%%%%%%%%%%%%%%%%%%%%%%%%%%%%%%%%%%%%%%%%%%%%%%%%%%%%%%%%%%%%%%%%%%%%%%%%
\subsection{Copyright}

Copyright \copyright{} 2017--2018 Niklas Beisert

This work may be distributed and/or modified under the
conditions of the \LaTeX{} Project Public License, either version 1.3
of this license or (at your option) any later version.
The latest version of this license is in
  \url{http://www.latex-project.org/lppl.txt}
and version 1.3 or later is part of all distributions of \LaTeX{}
version 2005/12/01 or later.

This work has the LPPL maintenance status `maintained'.

The Current Maintainer of this work is Niklas Beisert.

This work consists of the files |README.txt|, |childdoc.ins| and |childdoc.dtx|
as well as the derived files |childdoc.def|, |cdocsamp.tex|
with |cdocsch1.tex|, |cdocsch2.tex|, |cdocspt3.tex|, |cdocspt4.tex|,
|cdocsdrf.tex|, |cdocsfn1.tex|, |cdocsfn2.tex|
as well as |childdoc.pdf|.

%%%%%%%%%%%%%%%%%%%%%%%%%%%%%%%%%%%%%%%%%%%%%%%%%%%%%%%%%%%%%%%%%%%%%%%%%%%%%%%%
\subsection{Files and Installation}

The package consists of the files:
%
\begin{center}
\begin{tabular}{ll}
    |README.txt|   & readme file \\
    |childdoc.ins| & installation file \\
    |childdoc.dtx| & source file \\
    |childdoc.def| & definition file \\
    |cdocsamp.tex| & sample main file \\
    |cdocsch1.tex| & sample include file \\
    |cdocsch2.tex| & sample include file \\
    |cdocspt3.tex| & sample part file \\
    |cdocspt4.tex| & sample part file \\
    |cdocsdrf.tex| & sample redirection file \\
    |cdocsfn1.tex| & sample redirection file \\
    |cdocsfn2.tex| & sample redirection file \\
    |childdoc.pdf| & manual
\end{tabular}
\end{center}
%
The distribution consists of the files
|README.txt|, |childdoc.ins| and |childdoc.dtx|.
%
\begin{itemize}
\item
Run (pdf)\LaTeX{} on |childdoc.dtx|
to compile the manual |childdoc.pdf| (this file).
\item
Run \LaTeX{} on |childdoc.ins| to create the definitions file |childdoc.def|
and the sample |cdocsamp.tex| with include files
|cdocsch1.tex|, |cdocsch2.tex|, |cdocspt3.tex|, |cdocspt4.tex|,
|cdocsdrf.tex|, |cdocsfn1.tex|, |cdocsfn2.tex|.
Then copy the file |childdoc.def| to an appropriate directory of your \LaTeX{}
distribution, e.g.\ \textit{texmf-root}|/tex/latex/childdoc|.
\end{itemize}

%%%%%%%%%%%%%%%%%%%%%%%%%%%%%%%%%%%%%%%%%%%%%%%%%%%%%%%%%%%%%%%%%%%%%%%%%%%%%%%%
\subsection{Related CTAN Packages}

There are several other packages which offer a similar functionality:
%
\begin{itemize}
\item
The packages
\href{http://ctan.org/pkg/docmute}{\textsf{docmute}},
\href{http://ctan.org/pkg/includex}{\textsf{includex}} and
\href{http://ctan.org/pkg/standalone}{\textsf{standalone}}
provide commands to include only the document body of
a child file thus allowing both files to be compiled individually.
\item
The packages \href{http://ctan.org/pkg/subdocs}{\textsf{subdocs}}
and \href{http://ctan.org/pkg/subfiles}{\textsf{subfiles}}
provide structures in which the main and child documents can be
encapsulated and allowing them to be compiled individually.
The inclusion mechanism is different from the conventional |\include|.
\item
The package \href{http://ctan.org/pkg/combine}{\textsf{combine}}
is an elaborate solution to combine several documents into one.
\end{itemize}
%
See also the CTAN topic \href{http://ctan.org/topic/subdocs}{\textsf{subdocs}}
for further related packages.
The present package differs from the above solutions in that
a document structure constructed with the conventional |\include| mechanism
just needs two extra commands at the top of every file
such that all constituent files can be compiled individually.

%%%%%%%%%%%%%%%%%%%%%%%%%%%%%%%%%%%%%%%%%%%%%%%%%%%%%%%%%%%%%%%%%%%%%%%%%%%%%%%%
%\subsection{Feature Suggestions}
%
%The following is a list of features which may be useful for future
%versions of this package:
%%
%\begin{itemize}
%\item
%\ldots
%\end{itemize}

%%%%%%%%%%%%%%%%%%%%%%%%%%%%%%%%%%%%%%%%%%%%%%%%%%%%%%%%%%%%%%%%%%%%%%%%%%%%%%%%
\subsection{Revision History}

%%%%%%%%%%%%%%%%%%%%%%%%%%%%%%%%%%%%%%%%
\paragraph{v2.0:} 2018/12/30

\begin{itemize}
\item
immediate forward processing
\item
added |\childdocby| mechanism
\item
manual restructured
\end{itemize}

%%%%%%%%%%%%%%%%%%%%%%%%%%%%%%%%%%%%%%%%
\paragraph{v1.6:} 2018/01/17

\begin{itemize}
\item
application for development of include files
\item
corrections to manual
\end{itemize}

%%%%%%%%%%%%%%%%%%%%%%%%%%%%%%%%%%%%%%%%
\paragraph{v1.5:} 2017/05/21

\begin{itemize}
\item
more complete structuring introduced
\item
|\childdocof| introduced
\item
|\childdoc| renamed to |\childdocmain|
\item
|\childredirect| renamed to |\childdocforward| and |\childdocforwardprefix|
and functionality expanded
\end{itemize}

%%%%%%%%%%%%%%%%%%%%%%%%%%%%%%%%%%%%%%%%
\paragraph{v1.0:} 2017/04/27

\begin{itemize}
\item
manual and install package
\item
first version published on CTAN
\end{itemize}

%%%%%%%%%%%%%%%%%%%%%%%%%%%%%%%%%%%%%%%%
\paragraph{v0.6:} 2017/04/26

\begin{itemize}
\item
redirection mechanism added
\end{itemize}

%%%%%%%%%%%%%%%%%%%%%%%%%%%%%%%%%%%%%%%%
\paragraph{v0.5:} 2017/04/26

\begin{itemize}
\item
functionality in definition file
\end{itemize}


%%%%%%%%%%%%%%%%%%%%%%%%%%%%%%%%%%%%%%%%%%%%%%%%%%%%%%%%%%%%%%%%%%%%%%%%%%%%%%%%
%%%%%%%%%%%%%%%%%%%%%%%%%%%%%%%%%%%%%%%%%%%%%%%%%%%%%%%%%%%%%%%%%%%%%%%%%%%%%%%%
%%%%%%%%%%%%%%%%%%%%%%%%%%%%%%%%%%%%%%%%%%%%%%%%%%%%%%%%%%%%%%%%%%%%%%%%%%%%%%%%
\appendix

\settowidth\MacroIndent{\rmfamily\scriptsize 000\ }

 \DocInput{childdoc.dtx}

\end{document}
%</driver>
% \fi
%
% %%%%%%%%%%%%%%%%%%%%%%%%%%%%%%%%%%%%%%%%%%%%%%%%%%%%%%%%%%%%%%%%%%%%%%%%%%%%%%
% %%%%%%%%%%%%%%%%%%%%%%%%%%%%%%%%%%%%%%%%%%%%%%%%%%%%%%%%%%%%%%%%%%%%%%%%%%%%%%
% \section{Sample}
%\iffalse
%<*samplemain>
%\fi
%
% The following presents a sample document
% with two chapters, two parts, a title page,
% a compile flag as well as three forwarding files to set the flag.
% It consists of eight |.tex| files:
% \begin{center}
% \begin{tabular}{ll}
% |cdocsamp.tex|&main file\\
% |cdocsch1.tex|&include file for chapter 1\\
% |cdocsch2.tex|&include file for chapter 2\\
% |cdocspt3.tex|&include file for part 3\\
% |cdocspt4.tex|&include file for part 4\\
% |cdocsdrf.tex|&forwarding file for main file in draft mode\\
% |cdocsfi1.tex|&forwarding file for final version of chapter 1\\
% |cdocsfi2.tex|&forwarding file for final version of chapter 2\\
% \end{tabular}
% \end{center}
% Each of the eight files can be compiled directly by the \LaTeX{} compiler.
%
% %%%%%%%%%%%%%%%%%%%%%%%%%%%%%%%%%%%%%%
% \paragraph{Main File.}
%
% The main file is called |cdocsamp.tex|.
%
% Load the \textsf{childdoc} definitions and
% declare the filename for the main document:
%    \begin{macrocode}
\input{childdoc.def}
\childdocmain{}
%    \end{macrocode}

% Optional override for |\version| flag:
%    \begin{macrocode}
%%\ifchilddoc\else\providecommand{\version}{draft}\fi
%    \end{macrocode}

% Define the default values for the |\version| flag
% (|final| for the main file and |draft| for childs):
%    \begin{macrocode}
\ifchilddoc
\providecommand{\version}{draft}
\else
\providecommand{\version}{final}
\fi
%    \end{macrocode}

% Load the standard document class:
%    \begin{macrocode}
\documentclass[12pt]{article}
%    \end{macrocode}

% Start the document body:
%    \begin{macrocode}
\begin{document}
%    \end{macrocode}

% Declare a title page.
% Print title, part of document being processed and version flag:
%    \begin{macrocode}
\addtocounter{page}{-1}
\begin{center}
{\LARGE\bfseries{}childdoc example\par}
\vspace{1cm}
\ifchilddoc
\ifchilddocmanual part\else chapter\fi:
`\childdocname' of `\childdocjob'\par
\else
main document: `\childdocjob'\par
\fi
version: \version\par
\end{center}
\newpage
%    \end{macrocode}

% Manually include selected file,
% otherwise process as usual:
%    \begin{macrocode}
\ifchilddocmanual
\section*{part `\childdocname'}
\input{\childdocname}
\else
%    \end{macrocode}

% Include the two chapters:
%    \begin{macrocode}
\include{cdocsch1}
\include{cdocsch2}
%    \end{macrocode}

% Include the two parts unless only chapters should be displayed:
%    \begin{macrocode}
\ifchilddoc\else
\section{part three}
\input{cdocspt3}
\section{part four}
\input{cdocspt4}
\fi
%    \end{macrocode}

% Process as usual until here:
%    \begin{macrocode}
\fi
%    \end{macrocode}

% End of document body:
%    \begin{macrocode}
\end{document}
%    \end{macrocode}
%\iffalse
%</samplemain>
%\fi
%
% %%%%%%%%%%%%%%%%%%%%%%%%%%%%%%%%%%%%%%
% \paragraph{Chapter Include Files.}
%
% The include files are called |cdocsch1.tex| and |cdocsch2.tex|.
%
%\iffalse
%<*samplechap1|samplechap2>
%\fi

% Optional override for |\version| flag:
%    \begin{macrocode}
%%\providecommand{\version}{final}
%    \end{macrocode}

% Include the main document:
%    \begin{macrocode}
\input{childdoc.def}
\childdocof{cdocsamp}
%    \end{macrocode}

%\iffalse
%</samplechap1|samplechap2>
%\fi
%
%\iffalse
%<*samplechap1>
%\fi
% Some text for chapter 1:
%    \begin{macrocode}
\section{one}
some text in chapter one
%    \end{macrocode}

%\iffalse
%</samplechap1>
%\fi
% Some text for chapter 2:
%\iffalse
%<*samplechap2>
%\fi
%    \begin{macrocode}
\section{two}
more text in chapter two
%    \end{macrocode}

%\iffalse
%</samplechap2>
%\fi
%
% %%%%%%%%%%%%%%%%%%%%%%%%%%%%%%%%%%%%%%
% \paragraph{Part Include Files.}
%
% The include files are called |cdocspt3.tex| and |cdocspt4.tex|.
%
%\iffalse
%<*samplepart3|samplepart4>
%\fi

% Optional override for |\version| flag:
%    \begin{macrocode}
%%\providecommand{\version}{final}
%    \end{macrocode}

% Include the main document:
%    \begin{macrocode}
\input{childdoc.def}
\childdocby{cdocsamp}
%    \end{macrocode}

%\iffalse
%</samplepart3|samplepart4>
%\fi
%
%\iffalse
%<*samplepart3>
%\fi
% Some text for part 3:
%    \begin{macrocode}
some text in part three
%    \end{macrocode}

%\iffalse
%</samplepart3>
%\fi
% Some text for part 4:
%\iffalse
%<*samplepart4>
%\fi
%    \begin{macrocode}
more text in part four
%    \end{macrocode}

%\iffalse
%</samplepart4>
%\fi
%
% %%%%%%%%%%%%%%%%%%%%%%%%%%%%%%%%%%%%%%
% \paragraph{Forwarding for a Complete Draft.}
%
% The following forwarding file |cdocsdrf.tex|
% compiles the main document in draft mode:
%\iffalse
%<*sampledraft>
%\fi
%    \begin{macrocode}
\def\version{draft}
\input{childdoc.def}
\childdocforward{cdocsamp}
%    \end{macrocode}

%\iffalse
%</sampledraft>
%\fi
%
% %%%%%%%%%%%%%%%%%%%%%%%%%%%%%%%%%%%%%%
% \paragraph{Forwarding for Final Version of the Chapters.}
%
% The following forwarding files |cdocsfn1.tex| and |cdocsfn2.tex|
% (with identical content)
% compile the final versions of the child documents
% |cdocsch1.tex| and |cdocsch2.tex|, respectively:
%\iffalse
%<*samplefinal>
%\fi
%    \begin{macrocode}
\def\version{final}
\input{childdoc.def}
\childdocforwardprefix[cdocsamp]{cdocsfn}{cdocsch}
%    \end{macrocode}

%\iffalse
%</samplefinal>
%\fi
%
% %%%%%%%%%%%%%%%%%%%%%%%%%%%%%%%%%%%%%%
% \paragraph{Command Line Processing.}
%
% The following three command lines generate the output files
% |cdocscld|, |cdocscl1| and |cdocscl2|
% which should be identical to
% |cdocsdrf|, |cdocsch1| and |cdocsfn2|, respectively:
% \begin{center}
% \begin{tabular}{l}
% |latex -jobname cdocscld \|\\
% |  "\def\version{draft}\input{childdoc.def}\childdocforward{cdocsamp}"|\\
% |latex -jobname cdocscl1 \|\\
% |  "\input{childdoc.def}\childdocforward[cdocsamp]{cdocsch1}"|\\
% |latex -jobname cdocscl2 \|\\
% |  "\def\version{final}\input{childdoc.def}\childdocforward{cdocsch2}"|
% \end{tabular}
% \end{center}
% Note that the trailing backslash on each first line
% merely continues the input to the second line
% (for convenient cut ant paste).
% Furthermore, the command |latex| can be replaced by any
% of its alternative versions such as |pdflatex|.
%
% %%%%%%%%%%%%%%%%%%%%%%%%%%%%%%%%%%%%%%%%%%%%%%%%%%%%%%%%%%%%%%%%%%%%%%%%%%%%%%
% %%%%%%%%%%%%%%%%%%%%%%%%%%%%%%%%%%%%%%%%%%%%%%%%%%%%%%%%%%%%%%%%%%%%%%%%%%%%%%
% \section{Implementation}
%\iffalse
%<*package>
%\fi
%
% This section describes the definitions file |childdoc.def|.

% The definitions cannot be loaded using |\usepackage| or |\RequirePackage|
% which has a mechanism to prevent loading a style file more than once.
% When loading the definitions by means of |\input|
% multiple instances have to be prevented manually:
%\iffalse
%This code needs to be before the `\ProvidesFile' directive
%which is defined at the beginning of this file.
%Therefore it is also placed there and commented out here.
%</package>
%<*discard>
%\fi
%    \begin{macrocode}
\ifdefined\childdocmain\endinput\fi
%    \end{macrocode}
%\iffalse
%</discard>
%<*package>
%\fi
%
% \macro{\ifchilddoc}
% \macro{\ifchilddocmanual}
% The conditional |\ifchilddoc| tells whether a
% child (true) or main (false) document is being compiled.
% The conditional |\ifchilddocmanual| tells whether
% the |\includeonly| mechanism is used (false) or
% the selection of child files must be performed manually (true).
% The definitions initialise to false:
%    \begin{macrocode}
\newif\ifchilddoc
\newif\ifchilddocmanual
%    \end{macrocode}

% \macro{\childdocname}
% \macro{\childdocjob}
% The macro |\childdocname| stores the name of the main document
% to be compiled. The macro |\childdocjob| stores the name of
% the document on which the \LaTeX{} compiler was originally invoked.
% The content of |\jobname| cannot be compared
% to filenames specified in the source due to different catcodes.
% The following code rescans |\jobname|, stores the result
% in |\childdocname| and saves a copy in |\childdocjob|:
%    \begin{macrocode}
\edef\childdocname{\scantokens\expandafter{\jobname\noexpand}}
\let\childdocjob\childdocname
%    \end{macrocode}

% \macro{\childdocdisable}
% The macro |\childdocdisable| prevents the main file
% from being processed more than once.
% At this stage, the main document command |\childdocmain|
% is assumed to be called once again where it should do nothing.
% Any subsequent call to it should prevent
% a secondary processing of the main document
% It overwrites the forwarding commands
% |\childdocof| and |\childdocforward|
% with empty macros to prevent further inclusions of the main document:
%    \begin{macrocode}
\newcommand{\childdocdisable}
{
  \renewcommand{\childdocmain}[1]{\renewcommand{\childdocmain}[1]{\endinput}}
  \renewcommand{\childdocof}[1]{}
  \renewcommand{\childdocby}[2][]{}
  \renewcommand{\childdocforward}[2][]{}
  \renewcommand{\childdocdisable}{}
}
%    \end{macrocode}

% \macro{\childdocmain}
% The macro |\childdocmain| is to be called at the top of the main file
% with nothing or the main filename (without extension) as argument.
% First, it breaks loops.
% If the argument is not empty and does not match |\childdocname|
% (which is set by the first inclusion of |childdoc.def|),
% |\ifchilddoc| is set to true, |\includeonly| is applied to the child file
% and |\jobname| is set to the main file
% (for proper handling of |.aux| files):
%    \begin{macrocode}
\newcommand{\childdocmain}[1]
{
  \childdocdisable\childdocmain{}
  \if?#1?\else
    \begingroup
      \def\childdoctmp{#1}
      \ifx\childdoctmp\childdocname
        \def\childdoctmp{}
      \else
        \def\childdoctmp
        {
          \childdoctrue
          \includeonly{\childdocname}
          \def\childdocjob{#1}
          \def\jobname{#1}
        }
      \fi
      \expandafter
    \endgroup
    \childdoctmp
  \fi
}
%    \end{macrocode}

% \macro{\childdocof}
% The command |\childdocof| redirects
% compilation to the main file |#1|.
%    \begin{macrocode}
\newcommand{\childdocof}[1]
{
  \childdocdisable
  \childdoctrue
  \includeonly{\childdocname}
  \def\jobname{#1}
  \def\childdocjob{#1}
  \input{#1}
}
%    \end{macrocode}

% \macro{\childdocby}
% The command |\childdocby| ....
%    \begin{macrocode}
\newcommand{\childdocby}[2][]
{
  \childdocdisable
  \childdoctrue
  \childdocmanualtrue
  \if?#1?\else
    \def\jobname{#2}
  \fi
  \def\childdocjob{#2}
  \input{#2}
  \endinput
}
%    \end{macrocode}

% \macro{\childdocforward}
% The command |\childdocforward| redirects
% compilation to the main file or
% (if the optional argument is given) a child file.
% Parameters are set as if the main file
% or a child file starting with |\childdocof| was compiled.
% Then compilation is handed over to the main file:
%    \begin{macrocode}
\newcommand{\childdocforward}[2][]
{
  \begingroup
    \if?#1?
      \def\childdoctmp
      {
        \def\childdocname{#2}
        \def\childdocjob{#2}
        \def\jobname{#2}
        \input{#2}
        \endinput
      }
    \else
      \def\childdoctmp
      {
        \childdocdisable
        \def\childdocname{#2}
        \childdoctrue
        \includeonly{#2}
        \def\childdocjob{#1}
        \def\jobname{#1}
        \input{#1}
        \endinput
      }
    \fi
    \expandafter
  \endgroup
  \childdoctmp
}
%    \end{macrocode}

% \macro{\childdocforwardprefix}
% The command |\childdocforwardprefix| redirects
% compilation to the main or a child file by means of a pattern.
% The prefix |#1| in the current filename is replaced by |#2|
% and the suffix of the current filename is kept
% (it is assumed that the filename does not contain the substring `|~~~|'
% which is used as a delimiter).
% Compilation is handed over to the new file by |\childdocforward|:
%    \begin{macrocode}
\newcommand{\childdocforwardprefix}[3][]
{
  \begingroup
    \def\childdocextract #2##1~~~{\def\childdoctmp{\childdocforward[#1]{#3##1}}}
    \expandafter\childdocextract\childdocname~~~
    \expandafter
  \endgroup
  \childdoctmp
}
%    \end{macrocode}

% \macro{\childdoc}
% The deprecated macro |\childdoc| is a legacy version of |\childdocmain|:
%    \begin{macrocode}
\newcommand{\childdoc}{\childdocmain}
%    \end{macrocode}

% \macro{\childdocredirect}
% The deprecated macro |\childdocredirect| is a legacy version
% of |\childdocforward| and |\childdocforwardprefix|:
%    \begin{macrocode}
\newcommand{\childdocredirect}[2][]
{
  \begingroup
    \if?#1?
      \def\childdoctmp{\childdocforward{#2}}
    \else
      \def\childdoctmp{\childdocforwardprefix{#1}{#2}}
    \fi
    \expandafter
  \endgroup
  \childdoctmp
}
%    \end{macrocode}

%\iffalse
%</package>
%\fi
%
\endinput
\childdocforward{cdocsamp}"|\\
% |latex -jobname cdocscl1 \|\\
% |  "% \iffalse
%
% childdoc.dtx Copyright (C) 2017-2018 Niklas Beisert
%
% This work may be distributed and/or modified under the
% conditions of the LaTeX Project Public License, either version 1.3
% of this license or (at your option) any later version.
% The latest version of this license is in
%   http://www.latex-project.org/lppl.txt
% and version 1.3 or later is part of all distributions of LaTeX
% version 2005/12/01 or later.
%
% This work has the LPPL maintenance status `maintained'.
%
% The Current Maintainer of this work is Niklas Beisert.
%
% This work consists of the files childdoc.dtx and childdoc.ins
% and the derived files childdoc.def and cdocsamp.tex with
% cdocsch1.tex, cdocsch2.tex, cdocsdrf.tex, cdocsfn1.tex, cdocsfn2.tex.
%
%<package>\ifdefined\childdocmain\endinput\fi
%<package>\ProvidesFile{childdoc.def}[2018/12/30 v2.0 child document driver]
%<samplemain>\ProvidesFile{cdocsamp.tex}[2018/12/30 v2.0 sample for childdoc]
%<*driver>
%\ProvidesFile{childdoc.drv}[2018/12/30 v2.0 childdoc reference manual file]
\PassOptionsToClass{10pt,a4paper}{article}
\documentclass{ltxdoc}

\usepackage[margin=35mm]{geometry}
\usepackage{hyperref}
\usepackage{hyperxmp}
\usepackage[usenames]{color}

\hypersetup{colorlinks=true}
\hypersetup{pdfstartview=FitH}
\hypersetup{pdfpagemode=UseNone}
\hypersetup{pdfsource={}}
\hypersetup{pdflang={en-UK}}
\hypersetup{pdfcopyright={Copyright 2017-2018 Niklas Beisert.
  This work may be distributed and/or modified under the
  conditions of the LaTeX Project Public License, either version 1.3
  of this license or (at your option) any later version.}}
\hypersetup{pdflicenseurl={http://www.latex-project.org/lppl.txt}}
\hypersetup{pdfcontactaddress={ETH Zurich, ITP, HIT K,
  Wolfgang-Pauli-Strasse 27}}
\hypersetup{pdfcontactpostcode={8093}}
\hypersetup{pdfcontactcity={Zurich}}
\hypersetup{pdfcontactcountry={Switzerland}}
\hypersetup{pdfcontactemail={nbeisert@itp.phys.ethz.ch}}
\hypersetup{pdfcontacturl={http://people.phys.ethz.ch/\xmptilde nbeisert/}}

\newcommand{\secref}[1]{\hyperref[#1]{section \ref*{#1}}}

\parskip1ex
\parindent0pt
\let\olditemize\itemize
\def\itemize{\olditemize\parskip0pt}

\begin{document}

\title{The \textsf{childdoc} Package}
\hypersetup{pdftitle={The childdoc Package}}
\author{Niklas Beisert\\[2ex]
  Institut f\"ur Theoretische Physik\\
  Eidgen\"ossische Technische Hochschule Z\"urich\\
  Wolfgang-Pauli-Strasse 27, 8093 Z\"urich, Switzerland\\[1ex]
  \href{mailto:nbeisert@itp.phys.ethz.ch}
  {\texttt{nbeisert@itp.phys.ethz.ch}}}
\hypersetup{pdfauthor={Niklas Beisert}}
\hypersetup{pdfsubject={Manual for the LaTeX2e Package childdoc}}
\date{30 December 2018, \textsf{v2.0}}
\maketitle

\begin{abstract}\noindent
\textsf{childdoc} is a \LaTeXe{} package
that enables the direct compilation
of document sections included by |\include|
to individual files.
\end{abstract}

\begingroup
\parskip0ex
\tableofcontents
\endgroup

%%%%%%%%%%%%%%%%%%%%%%%%%%%%%%%%%%%%%%%%%%%%%%%%%%%%%%%%%%%%%%%%%%%%%%%%%%%%%%%%
%%%%%%%%%%%%%%%%%%%%%%%%%%%%%%%%%%%%%%%%%%%%%%%%%%%%%%%%%%%%%%%%%%%%%%%%%%%%%%%%
\section{Introduction}

\LaTeX{} provides a mechanism to structure a large document (such as a book)
into a main file and several child files (containing the chapters)
using the |\include| command.
This mechanism is beneficial for documents
which span hundreds of pages in order to
make the source file(s) more manageable.
Moreover, compilation can be restricted to
selected child files by means of the |\includeonly| command.
The latter feature can be used to reduce the compilation time while editing
(this was significantly more useful in the earlier days of \LaTeX{})
or to generate a smaller document which is easier to navigate.
Another application of |\includeonly| is to generate
documents consisting of selected parts of the complete document.

However, there are a few drawbacks of the plain |\include| mechanism:
\begin{itemize}
\item
The child files cannot be compiled on their own,
they can only be compiled via the main file.
A naive editing environment
(such as a text editor with an option
to have the current file processed by \LaTeX)
may require one to switch to the main file before compiling;
attempting to compile the child file produces errors.
\item
The main file must be modified (each time)
to adjust the |\includeonly| command
to the present needs. This easily leaves the main file in a messy state.
\item
The generated document will always carry the filename
of the main document. This is inconvenient if
several child files are to be compiled and
to be kept for distribution.
\end{itemize}

The present package provides a simple interface
to make child files individually compilable by \LaTeX{}.
Compiling a child file then has the same effect as compiling
the main file with an |\includeonly| command
to select the appropriate child.
Moreover the generated document will carry the name of the child
rather than the main file.
This resolves all three above issues.

This feature is meant to make the editing of books,
thesis documents and lecture notes somewhat more convenient.
However, the package can also be used efficiently for
composing a series of documents (such as exercise sheets)
which are typically distributed individually.
It then assists the author in generating the individual documents
(potentially in different versions)
as well as a document containing the collected series.
Another application is in developing style files
or other kinds of included material
where compilation of the style file could redirect
to a sample or test file.

%%%%%%%%%%%%%%%%%%%%%%%%%%%%%%%%%%%%%%%%%%%%%%%%%%%%%%%%%%%%%%%%%%%%%%%%%%%%%%%%
%%%%%%%%%%%%%%%%%%%%%%%%%%%%%%%%%%%%%%%%%%%%%%%%%%%%%%%%%%%%%%%%%%%%%%%%%%%%%%%%
\section{Usage}

First of all, the package \textsf{childdoc} is \emph{not} a standard
\LaTeXe{} |.sty| style file! Therefore it needs to be invoked in
a non-standard way.

%%%%%%%%%%%%%%%%%%%%%%%%%%%%%%%%%%%%%%%%%%%%%%%%%%%%%%%%%%%%%%%%%%%%%%%%%%%%%%%%
\subsection{Included Files}
\label{sec:include}

%%%%%%%%%%%%%%%%%%%%%%%%%%%%%%%%%%%%%%%%
\DescribeMacro{\childdocmain}
To use the package, add the commands
\begin{center}
\begin{tabular}{l}
|\input{childdoc.def}|\\
|\childdocmain{}|\\
\end{tabular}
\end{center}
at the very top of the main \LaTeX{} file,
in particular \emph{before} the |\documentclass| statement!
The argument of |\childdocmain| should be left empty
(but it must be present).

%%%%%%%%%%%%%%%%%%%%%%%%%%%%%%%%%%%%%%%%
\DescribeMacro{\childdocof}
Furthermore, add the commands
\begin{center}
\begin{tabular}{l}
|\input{childdoc.def}|\\
|\childdocof{|\textit{main}|}|\\
\end{tabular}
\end{center}
at the top of every child file \textit{child}
which is included by |\include{|\textit{child}|}|
from within the main file
(or at least for those files to be compiled individually).
The argument \textit{main} must be the filename of the main file.

There are a couple of
considerations in setting up the main and child documents:

%%%%%%%%%%%%%%%%%%%%%%%%%%%%%%%%%%%%%%%%
\paragraph{Restrictions.}

Please note the following restrictions:
\begin{itemize}
\item
|\childdocmain| must be called with one argument \textit{main}
to ensure compatibility with earlier version of the package.
It must either be empty (|\childdocmain{}|)
or precisely match the filename of the main file in which it is specified.
See \secref{sec:detection} for further information.
\item
The filename \textit{main} must be specified without the |.tex| extension.
\item
The filename \textit{main} is case sensitive
(even in case-insensitive file systems)
due to internal string comparison.
\item
The argument \textit{main} should be fully expanded, it cannot be a macro.
\item
Subdirectories and special characters should be avoided in filenames.
\item
The command |\childdocmain{|\textit{main}|}| must be followed by a whitespace.
It should not be followed immediately by another command
or by a comment mark `|%|'.
This is because the \TeX{} parser reads the token immediately following
the argument of |\childdocmain| and puts it
at the beginning of every child section;
however, a white\-space is ignored.
\end{itemize}

%%%%%%%%%%%%%%%%%%%%%%%%%%%%%%%%%%%%%%%%
\paragraph{Content of Main File.}

It is advisable to place all content in the child files included by |\include|.
Any output contained in the main file will appear in all child documents
unless suppressed manually;
it cannot be suppressed automatically by the |\includeonly| directive
and thus should normally be avoided.
A method to include some content in the main file
by means of conditional processing is described in \secref{sec:conditional}.

%%%%%%%%%%%%%%%%%%%%%%%%%%%%%%%%%%%%%%%%
\paragraph{Page Numbering.}

When only a part of the document is compiled,
the appropriate numbering of pages
(as well as other status parameters)
is determined from the |.aux| files.
The latter contain information from previous passes.
However this information needs to propagate through
all intermediate child documents.
Therefore the page numbering in child documents may well
be inconsistent until the complete document is compiled at least once.

A useful (if unconventional) way to always ensure a consistent
page numbering is to restart the numbering in each child document
and denote the pages by `\textit{child}|.|\textit{page}'
where \textit{child} represents the chapter/section number of the child file.
This can be achieved by the command
|\numberwithin{page}{|\textit{child}|}|
of the \textsf{amsmath} package
where \textit{child} can be |chapter| or |section|
depending on the chosen structuring.
Alternatively, one can modify the macro |\thepage| appropriately
and reset the counter |page| at the start of each child file.

%%%%%%%%%%%%%%%%%%%%%%%%%%%%%%%%%%%%%%%%%%%%%%%%%%%%%%%%%%%%%%%%%%%%%%%%%%%%%%%%
\subsection{Conditional Processing}
\label{sec:conditional}

The package provides a mechanism to compile different versions
of a document. To customise the versions further some conditional processing
can come in handy to distinguish which version is being compiled.
The package provides two macros to describe the compilation context:

%%%%%%%%%%%%%%%%%%%%%%%%%%%%%%%%%%%%%%%%
\DescribeMacro{\ifchilddoc}
The conditional |\ifchilddoc| distinguishes between the compilation of
child documents and the main document:
%
\begin{center}
|\ifchilddoc |\textit{child-code}| |[|\||else |\textit{main-code}]| \||fi|
\end{center}

%%%%%%%%%%%%%%%%%%%%%%%%%%%%%%%%%%%%%%%%
\DescribeMacro{\childdocname}
\DescribeMacro{\childdocjob}
The macro |\childdocname| contains the filename (without extension)
of the main or child file being processed.
Note that |\childdocjob| will always contain the name of the main file.

%%%%%%%%%%%%%%%%%%%%%%%%%%%%%%%%%%%%%%%%
\paragraph{Title Page.}

Conditional processing can be used to include a title or banner page
in the main document when proper precautions are taken.
Importantly, the code in the main file should ensure that the page counter
(as well as other status parameters which are stored in the |.aux| files)
takes the same value after the conditional processing.
Otherwise the page numbers may take divergent values
depending on which part is compiled.

For example, a title page could be declared by:
%
\begin{center}
\begin{tabular}{l}
|\ifchilddoc\||else|\\
|\addtocounter{page}{-1}|\\
\textit{code for title page}\\
|\newpage|\\
|\||fi|
\end{tabular}
\end{center}
%
A banner page for the child documents can be generated by:
%
\begin{center}
\begin{tabular}{l}
|\ifchilddoc|\\
|\addtocounter{page}{-1}|\\
\textit{code for banner page}\\
|\newpage|\\
|\||fi|
\end{tabular}
\end{center}
%
Here one could write a message such as:
\begin{center}
|This is the part \childdocname{} of \childdocjob{}.|
\end{center}

%%%%%%%%%%%%%%%%%%%%%%%%%%%%%%%%%%%%%%%%%%%%%%%%%%%%%%%%%%%%%%%%%%%%%%%%%%%%%%%%
\subsection{Flags}
\label{sec:flags}

The package makes it easy to generate different versions
of the main or child documents.
To this end compilation flags can be defined
and assigned different default values.
They will be particularly useful in conjunction
with the forwarding mechanism described in \secref{sec:forward}.

For example, it may be useful to have a flag |\version|
which can be set to |draft| or |final|.
The document source will contain some conditional code
depending on the value of |\version|.
Suppose further, the flag should default to |final| for the main file
and to |draft| for child files
which is a natural assignment for editing the document.
This is achieved by placing the following code
in the preamble of the main document
(below the |\childdocmain| directive):
%
\begin{center}
\begin{tabular}{l}
|\ifchilddoc|\\
|\providecommand{\version}{draft}|\\
|\||else|\\
|\providecommand{\version}{final}|\\
|\||fi|
\end{tabular}
\end{center}
%
The definition by |\providecommand| makes sure
that previous definitions are not overwritten.
Further statements |\providecommand{\version}{...}|
can thus be added before the above code to override it.

For the main file, one might add a line
(between |\childdocmain| and the above block)
%
\begin{center}
|%\ifchilddoc\||else\providecommand{\version}{draft}\||fi|
\end{center}
%
which can be uncommented to produce a draft version.
Likewise one can add a line to the very top of a child file
(above the |\childdocof{|\textit{main}|}| directive)
%
\begin{center}
|%\providecommand{\version}{final}|
\end{center}
%
which can be uncommented to produce the final version of this child document.

%%%%%%%%%%%%%%%%%%%%%%%%%%%%%%%%%%%%%%%%%%%%%%%%%%%%%%%%%%%%%%%%%%%%%%%%%%%%%%%%
\subsection{Forwarding}
\label{sec:forward}

Different versions of the main or child documents
using compilation flags as described in \secref{sec:flags}
can be (permanently) stored in different files
for convenient compilation, viewing and distribution.
To this end, the package defines a command
to pass on compilation to a different file:

%%%%%%%%%%%%%%%%%%%%%%%%%%%%%%%%%%%%%%%%
\DescribeMacro{\childdocforward}
The command |\childdocforward| redirects processing to
another source file:
%
\begin{center}
\begin{tabular}{l}
|\input{childdoc.def}|\\
|\childdocforward[|\textit{main}|]{|\textit{dest}|}|\\
\end{tabular}
\end{center}
%
The argument \textit{dest} is the destination file
(without extension).
It should be the main file or one of the child files.
Note that further \textsf{childdoc} directives
such as |\childdocof| and |\childdocforward|
in the indicated file will be processed in this form.
The optional argument \textit{main}
passes on directly to the main file \textit{main}
while pretending to compile the child \textit{dest}.
This form behaves as if \textit{dest}
issues |\childdocof{|\textit{main}|}| right away,
and no further \textsf{childdoc} directives will be processed.

%%%%%%%%%%%%%%%%%%%%%%%%%%%%%%%%%%%%%%%%
\DescribeMacro{\...prefix}
In the alternative form |\childdocforwardprefix|,
%
\begin{center}
\begin{tabular}{l}
|\input{childdoc.def}|\\
|\childdocforwardprefix[|\textit{main}|]{|\textit{prefix}|}{|\textit{dest}|}|
\end{tabular}
\end{center}
%
the destination file is determined by a pattern
depending on the current file:
To make this work, the current file must be called
`{\textit{prefix}\hspace{0.2em}\textit{suffix}}'
with \textit{prefix} matching precisely the argument.
Processing is then passed on to the file
`{\textit{dest}\hspace{0.2em}\textit{suffix}}'.
Surely, the same effect is achieved by
directly specifying the
argument `{\textit{dest}\hspace{0.2em}\textit{suffix}}'
in the first form.
However, that requires to set up a different file
for each child. With the alternative form of the command
all these files can have exactly the same content
which simplifies setting them up and maintaining them.

For example, the following file |draft.tex|
with a compilation flag |\version| as described in \secref{sec:flags}
compiles the main document as a draft:
%
\begin{center}
\begin{tabular}{l}
|\def\version{draft}|\\
|\input{childdoc.def}|\\
|\childdocforward{|\textit{main}|}|
\end{tabular}
\end{center}
%
Likewise, the following files |final|\textit{nn}|.tex|
compile the final version of the child document
|child|\textit{nn}|.tex|:
%
\begin{center}
\begin{tabular}{l}
|\def\version{final}|\\
|\input{childdoc.def}|\\
|\childdocforwardprefix{final}{child}|
\end{tabular}
\end{center}
%

Note that when several versions of a main file and/or of each child file
are to be generated, it may be convenient to set up a |Makefile| or
shell script to automatise the process.

%%%%%%%%%%%%%%%%%%%%%%%%%%%%%%%%%%%%%%%%%%%%%%%%%%%%%%%%%%%%%%%%%%%%%%%%%%%%%%%%
\subsection{Command Line Processing}
\label{sec:commandline}

The effect of redirection files can also be achieved by invoking
the \LaTeX{} compiler with a more elaborate command line.
Most conveniently this should be done as part
of a shell script or a |Makefile|.

When using \textsf{childdoc} in the main file, the following
command lines effectively perform a redirection
(note that depending on the shell being used,
backslashes may have to be doubled: `|\|' $\to$ `|\\|'):
%
\begin{center}
|... -jobname "|\textit{target}|" |\\|"|[\textit{flags}]%
|\input{childdoc.def}\childdocforward[|\textit{main}|]{|\textit{dest}|}"|
\end{center}
%
Here \textit{target} is the name of the output file,
\textit{main} is the name of the main file
and \textit{dest} is the name of the main or child file to be processed
(all filenames without extensions).
The optional argument \textit{main} can be omitted
if \textit{main} matches \textit{dest}.
Optionally, compilation \textit{flags} can be defined via |\def| commands.
This command line makes the \TeX{} engine believe
it is compiling the file \textit{target}
whose content is specified as the latter parameter.
The provided code then forwards the processing to
\textit{main} or \textit{dest} as described in \secref{sec:forward}.

%%%%%%%%%%%%%%%%%%%%%%%%%%%%%%%%%%%%%%%%%%%%%%%%%%%%%%%%%%%%%%%%%%%%%%%%%%%%%%%%
\subsection{Include by Input}
\label{sec:input}

Including child documents by |\include| has some restrictions by design.
Most notably, the content of a child document always occupies
its own set of pages; pages cannot be shared between child documents.
Usually, this behaviour makes perfect sense
because each child document contain an essential part of the document.
However, in some situations it may be desirable to compose
a document from a collection of parts
without having mandatory page breaks between then.
For this case, the package
provides a mechanism to include parts
by |\input| which can also be processed individually.
However, by construction this mechanism
requires manual handling of the content to be output.

%%%%%%%%%%%%%%%%%%%%%%%%%%%%%%%%%%%%%%%%
\DescribeMacro{\ifchilddocmanual}
The main file should be prepared as usual, see \secref{sec:include}.
However, the document body must make a distinction
between processing of an individual part and of the main document, e.g.:
%
\begin{center}
\begin{tabular}{l}
|\ifchilddocmanual|\\
|\input{\childdocname}|\\
|\||else|\\
\textit{document body with }|\input{|\textit{part}|}|\\
|\||fi|
\end{tabular}
\end{center}
%
The conditional |\ifchilddocmanual| is true whenever
a part to be included by |\input| is being compiled,
and the name of the part is stored in |\childdocname|.

%%%%%%%%%%%%%%%%%%%%%%%%%%%%%%%%%%%%%%%%
\DescribeMacro{\childdocby}
Each part to be included by |\input| should start with:
%
\begin{center}
\begin{tabular}{l}
|\input{childdoc.def}|\\
|\childdocby{|\textit{main}|}|\\
\end{tabular}
\end{center}
%
The directive |\childdocby| is similar to |\childdocof|
described in \secref{sec:include},
but the subsequent selection of content must be done manually.
To that end, both |\ifchilddoc| and |\ifchilddocmanual|
will be true upon processing of a part,
and the name of the part is stored in |\childdocname|.
Note that |\jobname| will be set to the filename of the current part
so that each part receives an individual |.aux| file
that does not interfere with the |.aux| file(s) of the main document.
This behaviour can be altered by the alternative form
|\childdocby[*]{|\textit{main}|}| (with a non-empty optional argument)
which uses the |.aux| file of the main document
by setting |\jobname| to \textit{main}.

%%%%%%%%%%%%%%%%%%%%%%%%%%%%%%%%%%%%%%%%%%%%%%%%%%%%%%%%%%%%%%%%%%%%%%%%%%%%%%%%
\subsection{Driver Development}
\label{sec:driver}

The \textsf{childdoc} mechanism can also be use for the development
of definition files such as \LaTeX{} styles or classes.
This case differs from the above setup with multiple parts
included by |\include| in that no |\includeonly| should be invoked.
This can be achieved by starting the include file
(before |\ProvidesPackage|) with:
%
\begin{center}
\begin{tabular}{l}
|\input{childdoc.def}|\\
|\childdocforward{|\textit{main}|}|\\
\end{tabular}
\end{center}
%
or alternatively with:
%
\begin{center}
\begin{tabular}{l}
|\input{childdoc.def}|\\
|\childdocby{|\textit{main}|}|\\
\end{tabular}
\end{center}
%
Both forms have slightly different effects as described above.
The main file is prepared as usual, see \secref{sec:include}.

%%%%%%%%%%%%%%%%%%%%%%%%%%%%%%%%%%%%%%%%%%%%%%%%%%%%%%%%%%%%%%%%%%%%%%%%%%%%%%%%
\subsection{Legacy Detection}
\label{sec:detection}

The directive |\childdocmain| in the main file can detect
whether the complete document or merely a child is to be compiled
even without using the directive |\childdocof|.
This method is deprecated because it is less robust
and there is no compelling reason to use it;
it is merely provided for backward compatibility
and it may be removed in future versions.

If the detection mechanism is to be used,
it is mandatory to correctly specify
the filename of the main file as the argument of |\childdocmain|:
%
\begin{center}
\begin{tabular}{l}
|\input{childdoc.def}|\\
|\childdocmain{|\textit{main}|}|\\
\end{tabular}
\end{center}
%
If |\jobname| does not match the argument \textit{main} of |\childdocmain|,
it is assumed that |\jobname| points to the child file to be compiled.
When using |\childdocmain| with the main file specified as argument,
it suffices to start a child file
with just |\input{|\textit{main}|}|
without loading of the package and using |\childdocof|.
If instead all processing is done
with the appropriate \textsf{childdoc} directives,
the argument of \textit{main} of |\childdocmain| can be empty.

An alternative version of the command line processing described
in \secref{sec:commandline} using the detection mechanism reads:
%
\begin{center}
|... -jobname "|\textit{target}|" "|[\textit{flags}]%
[|\def\jobname{|\textit{dest}|}|]|\input{|\textit{main}|}"|
\end{center}

%%%%%%%%%%%%%%%%%%%%%%%%%%%%%%%%%%%%%%%%%%%%%%%%%%%%%%%%%%%%%%%%%%%%%%%%%%%%%%%%
\subsection{Manual Code}
\label{sec:manual}

In case one cannot be certain whether the definitions file |childdoc.def|
is installed on the target \TeX{} distribution
and one prefers not to ship it,
it is conceivable to paste a few relevant commands into the sources.

To that end, drop all statements |\input{childdoc.def}|
and perform the replacements as outlined below.
Instead of |\childdocmain{|\textit{main}|}| add the following code
to the top of the main file:
%
\begin{center}
\begin{tabular}{l}
|\||ifdefined\childdocname\endinput\||fi\newif\ifchilddoc|\\
|\edef\childdocname{\scantokens\expandafter{\jobname\noexpand}}|\\
|\def\childdocmain{|\textit{main}|}\||ifx\childdocmain\childdocname\||else|\\
|\childdoctrue\includeonly{\childdocname}\let\jobname\childdocmain\||fi|\\
\end{tabular}
\end{center}
%
Instead of |\childdocof{|\textit{main}|}| just include the main file
at the top of each child file:
%
\begin{center}
|\input{|\textit{main}|}|
\end{center}
%
A simple redirection |\childdocforward{|\textit{dest}|}| is achieved by:
%
\begin{center}
|\def\jobname{|\textit{dest}|}\input{\jobname}|
\end{center}
%
The redirection with prefix
|\childdocforwardprefix[|\textit{prefix}|]{|\textit{dest}|}|
is accomplished by:
%
\begin{center}
\begin{tabular}{l}
|{\edef\jobname{\scantokens\expandafter{\jobname\noexpand}}|\\
|\def\redirectjob |\textit{prefix}|#1~~~{\gdef\jobname{|\textit{dest}|#1}}|\\
|\expandafter\redirectjob\jobname~~~}\input{\jobname}|
\end{tabular}
\end{center}

In an alternative approach,
child documents can be compiled by a specific command line
without additional code or specific definitions:
%
\begin{center}
|... -jobname "|\textit{target}|" "|[\textit{flags}]%
|\includeonly{|\textit{dest}|}\input{|\textit{main}|}"|
\end{center}
%

%%%%%%%%%%%%%%%%%%%%%%%%%%%%%%%%%%%%%%%%%%%%%%%%%%%%%%%%%%%%%%%%%%%%%%%%%%%%%%%%
%%%%%%%%%%%%%%%%%%%%%%%%%%%%%%%%%%%%%%%%%%%%%%%%%%%%%%%%%%%%%%%%%%%%%%%%%%%%%%%%
\section{Information}

%%%%%%%%%%%%%%%%%%%%%%%%%%%%%%%%%%%%%%%%%%%%%%%%%%%%%%%%%%%%%%%%%%%%%%%%%%%%%%%%
\subsection{Copyright}

Copyright \copyright{} 2017--2018 Niklas Beisert

This work may be distributed and/or modified under the
conditions of the \LaTeX{} Project Public License, either version 1.3
of this license or (at your option) any later version.
The latest version of this license is in
  \url{http://www.latex-project.org/lppl.txt}
and version 1.3 or later is part of all distributions of \LaTeX{}
version 2005/12/01 or later.

This work has the LPPL maintenance status `maintained'.

The Current Maintainer of this work is Niklas Beisert.

This work consists of the files |README.txt|, |childdoc.ins| and |childdoc.dtx|
as well as the derived files |childdoc.def|, |cdocsamp.tex|
with |cdocsch1.tex|, |cdocsch2.tex|, |cdocspt3.tex|, |cdocspt4.tex|,
|cdocsdrf.tex|, |cdocsfn1.tex|, |cdocsfn2.tex|
as well as |childdoc.pdf|.

%%%%%%%%%%%%%%%%%%%%%%%%%%%%%%%%%%%%%%%%%%%%%%%%%%%%%%%%%%%%%%%%%%%%%%%%%%%%%%%%
\subsection{Files and Installation}

The package consists of the files:
%
\begin{center}
\begin{tabular}{ll}
    |README.txt|   & readme file \\
    |childdoc.ins| & installation file \\
    |childdoc.dtx| & source file \\
    |childdoc.def| & definition file \\
    |cdocsamp.tex| & sample main file \\
    |cdocsch1.tex| & sample include file \\
    |cdocsch2.tex| & sample include file \\
    |cdocspt3.tex| & sample part file \\
    |cdocspt4.tex| & sample part file \\
    |cdocsdrf.tex| & sample redirection file \\
    |cdocsfn1.tex| & sample redirection file \\
    |cdocsfn2.tex| & sample redirection file \\
    |childdoc.pdf| & manual
\end{tabular}
\end{center}
%
The distribution consists of the files
|README.txt|, |childdoc.ins| and |childdoc.dtx|.
%
\begin{itemize}
\item
Run (pdf)\LaTeX{} on |childdoc.dtx|
to compile the manual |childdoc.pdf| (this file).
\item
Run \LaTeX{} on |childdoc.ins| to create the definitions file |childdoc.def|
and the sample |cdocsamp.tex| with include files
|cdocsch1.tex|, |cdocsch2.tex|, |cdocspt3.tex|, |cdocspt4.tex|,
|cdocsdrf.tex|, |cdocsfn1.tex|, |cdocsfn2.tex|.
Then copy the file |childdoc.def| to an appropriate directory of your \LaTeX{}
distribution, e.g.\ \textit{texmf-root}|/tex/latex/childdoc|.
\end{itemize}

%%%%%%%%%%%%%%%%%%%%%%%%%%%%%%%%%%%%%%%%%%%%%%%%%%%%%%%%%%%%%%%%%%%%%%%%%%%%%%%%
\subsection{Related CTAN Packages}

There are several other packages which offer a similar functionality:
%
\begin{itemize}
\item
The packages
\href{http://ctan.org/pkg/docmute}{\textsf{docmute}},
\href{http://ctan.org/pkg/includex}{\textsf{includex}} and
\href{http://ctan.org/pkg/standalone}{\textsf{standalone}}
provide commands to include only the document body of
a child file thus allowing both files to be compiled individually.
\item
The packages \href{http://ctan.org/pkg/subdocs}{\textsf{subdocs}}
and \href{http://ctan.org/pkg/subfiles}{\textsf{subfiles}}
provide structures in which the main and child documents can be
encapsulated and allowing them to be compiled individually.
The inclusion mechanism is different from the conventional |\include|.
\item
The package \href{http://ctan.org/pkg/combine}{\textsf{combine}}
is an elaborate solution to combine several documents into one.
\end{itemize}
%
See also the CTAN topic \href{http://ctan.org/topic/subdocs}{\textsf{subdocs}}
for further related packages.
The present package differs from the above solutions in that
a document structure constructed with the conventional |\include| mechanism
just needs two extra commands at the top of every file
such that all constituent files can be compiled individually.

%%%%%%%%%%%%%%%%%%%%%%%%%%%%%%%%%%%%%%%%%%%%%%%%%%%%%%%%%%%%%%%%%%%%%%%%%%%%%%%%
%\subsection{Feature Suggestions}
%
%The following is a list of features which may be useful for future
%versions of this package:
%%
%\begin{itemize}
%\item
%\ldots
%\end{itemize}

%%%%%%%%%%%%%%%%%%%%%%%%%%%%%%%%%%%%%%%%%%%%%%%%%%%%%%%%%%%%%%%%%%%%%%%%%%%%%%%%
\subsection{Revision History}

%%%%%%%%%%%%%%%%%%%%%%%%%%%%%%%%%%%%%%%%
\paragraph{v2.0:} 2018/12/30

\begin{itemize}
\item
immediate forward processing
\item
added |\childdocby| mechanism
\item
manual restructured
\end{itemize}

%%%%%%%%%%%%%%%%%%%%%%%%%%%%%%%%%%%%%%%%
\paragraph{v1.6:} 2018/01/17

\begin{itemize}
\item
application for development of include files
\item
corrections to manual
\end{itemize}

%%%%%%%%%%%%%%%%%%%%%%%%%%%%%%%%%%%%%%%%
\paragraph{v1.5:} 2017/05/21

\begin{itemize}
\item
more complete structuring introduced
\item
|\childdocof| introduced
\item
|\childdoc| renamed to |\childdocmain|
\item
|\childredirect| renamed to |\childdocforward| and |\childdocforwardprefix|
and functionality expanded
\end{itemize}

%%%%%%%%%%%%%%%%%%%%%%%%%%%%%%%%%%%%%%%%
\paragraph{v1.0:} 2017/04/27

\begin{itemize}
\item
manual and install package
\item
first version published on CTAN
\end{itemize}

%%%%%%%%%%%%%%%%%%%%%%%%%%%%%%%%%%%%%%%%
\paragraph{v0.6:} 2017/04/26

\begin{itemize}
\item
redirection mechanism added
\end{itemize}

%%%%%%%%%%%%%%%%%%%%%%%%%%%%%%%%%%%%%%%%
\paragraph{v0.5:} 2017/04/26

\begin{itemize}
\item
functionality in definition file
\end{itemize}


%%%%%%%%%%%%%%%%%%%%%%%%%%%%%%%%%%%%%%%%%%%%%%%%%%%%%%%%%%%%%%%%%%%%%%%%%%%%%%%%
%%%%%%%%%%%%%%%%%%%%%%%%%%%%%%%%%%%%%%%%%%%%%%%%%%%%%%%%%%%%%%%%%%%%%%%%%%%%%%%%
%%%%%%%%%%%%%%%%%%%%%%%%%%%%%%%%%%%%%%%%%%%%%%%%%%%%%%%%%%%%%%%%%%%%%%%%%%%%%%%%
\appendix

\settowidth\MacroIndent{\rmfamily\scriptsize 000\ }

 \DocInput{childdoc.dtx}

\end{document}
%</driver>
% \fi
%
% %%%%%%%%%%%%%%%%%%%%%%%%%%%%%%%%%%%%%%%%%%%%%%%%%%%%%%%%%%%%%%%%%%%%%%%%%%%%%%
% %%%%%%%%%%%%%%%%%%%%%%%%%%%%%%%%%%%%%%%%%%%%%%%%%%%%%%%%%%%%%%%%%%%%%%%%%%%%%%
% \section{Sample}
%\iffalse
%<*samplemain>
%\fi
%
% The following presents a sample document
% with two chapters, two parts, a title page,
% a compile flag as well as three forwarding files to set the flag.
% It consists of eight |.tex| files:
% \begin{center}
% \begin{tabular}{ll}
% |cdocsamp.tex|&main file\\
% |cdocsch1.tex|&include file for chapter 1\\
% |cdocsch2.tex|&include file for chapter 2\\
% |cdocspt3.tex|&include file for part 3\\
% |cdocspt4.tex|&include file for part 4\\
% |cdocsdrf.tex|&forwarding file for main file in draft mode\\
% |cdocsfi1.tex|&forwarding file for final version of chapter 1\\
% |cdocsfi2.tex|&forwarding file for final version of chapter 2\\
% \end{tabular}
% \end{center}
% Each of the eight files can be compiled directly by the \LaTeX{} compiler.
%
% %%%%%%%%%%%%%%%%%%%%%%%%%%%%%%%%%%%%%%
% \paragraph{Main File.}
%
% The main file is called |cdocsamp.tex|.
%
% Load the \textsf{childdoc} definitions and
% declare the filename for the main document:
%    \begin{macrocode}
\input{childdoc.def}
\childdocmain{}
%    \end{macrocode}

% Optional override for |\version| flag:
%    \begin{macrocode}
%%\ifchilddoc\else\providecommand{\version}{draft}\fi
%    \end{macrocode}

% Define the default values for the |\version| flag
% (|final| for the main file and |draft| for childs):
%    \begin{macrocode}
\ifchilddoc
\providecommand{\version}{draft}
\else
\providecommand{\version}{final}
\fi
%    \end{macrocode}

% Load the standard document class:
%    \begin{macrocode}
\documentclass[12pt]{article}
%    \end{macrocode}

% Start the document body:
%    \begin{macrocode}
\begin{document}
%    \end{macrocode}

% Declare a title page.
% Print title, part of document being processed and version flag:
%    \begin{macrocode}
\addtocounter{page}{-1}
\begin{center}
{\LARGE\bfseries{}childdoc example\par}
\vspace{1cm}
\ifchilddoc
\ifchilddocmanual part\else chapter\fi:
`\childdocname' of `\childdocjob'\par
\else
main document: `\childdocjob'\par
\fi
version: \version\par
\end{center}
\newpage
%    \end{macrocode}

% Manually include selected file,
% otherwise process as usual:
%    \begin{macrocode}
\ifchilddocmanual
\section*{part `\childdocname'}
\input{\childdocname}
\else
%    \end{macrocode}

% Include the two chapters:
%    \begin{macrocode}
\include{cdocsch1}
\include{cdocsch2}
%    \end{macrocode}

% Include the two parts unless only chapters should be displayed:
%    \begin{macrocode}
\ifchilddoc\else
\section{part three}
\input{cdocspt3}
\section{part four}
\input{cdocspt4}
\fi
%    \end{macrocode}

% Process as usual until here:
%    \begin{macrocode}
\fi
%    \end{macrocode}

% End of document body:
%    \begin{macrocode}
\end{document}
%    \end{macrocode}
%\iffalse
%</samplemain>
%\fi
%
% %%%%%%%%%%%%%%%%%%%%%%%%%%%%%%%%%%%%%%
% \paragraph{Chapter Include Files.}
%
% The include files are called |cdocsch1.tex| and |cdocsch2.tex|.
%
%\iffalse
%<*samplechap1|samplechap2>
%\fi

% Optional override for |\version| flag:
%    \begin{macrocode}
%%\providecommand{\version}{final}
%    \end{macrocode}

% Include the main document:
%    \begin{macrocode}
\input{childdoc.def}
\childdocof{cdocsamp}
%    \end{macrocode}

%\iffalse
%</samplechap1|samplechap2>
%\fi
%
%\iffalse
%<*samplechap1>
%\fi
% Some text for chapter 1:
%    \begin{macrocode}
\section{one}
some text in chapter one
%    \end{macrocode}

%\iffalse
%</samplechap1>
%\fi
% Some text for chapter 2:
%\iffalse
%<*samplechap2>
%\fi
%    \begin{macrocode}
\section{two}
more text in chapter two
%    \end{macrocode}

%\iffalse
%</samplechap2>
%\fi
%
% %%%%%%%%%%%%%%%%%%%%%%%%%%%%%%%%%%%%%%
% \paragraph{Part Include Files.}
%
% The include files are called |cdocspt3.tex| and |cdocspt4.tex|.
%
%\iffalse
%<*samplepart3|samplepart4>
%\fi

% Optional override for |\version| flag:
%    \begin{macrocode}
%%\providecommand{\version}{final}
%    \end{macrocode}

% Include the main document:
%    \begin{macrocode}
\input{childdoc.def}
\childdocby{cdocsamp}
%    \end{macrocode}

%\iffalse
%</samplepart3|samplepart4>
%\fi
%
%\iffalse
%<*samplepart3>
%\fi
% Some text for part 3:
%    \begin{macrocode}
some text in part three
%    \end{macrocode}

%\iffalse
%</samplepart3>
%\fi
% Some text for part 4:
%\iffalse
%<*samplepart4>
%\fi
%    \begin{macrocode}
more text in part four
%    \end{macrocode}

%\iffalse
%</samplepart4>
%\fi
%
% %%%%%%%%%%%%%%%%%%%%%%%%%%%%%%%%%%%%%%
% \paragraph{Forwarding for a Complete Draft.}
%
% The following forwarding file |cdocsdrf.tex|
% compiles the main document in draft mode:
%\iffalse
%<*sampledraft>
%\fi
%    \begin{macrocode}
\def\version{draft}
\input{childdoc.def}
\childdocforward{cdocsamp}
%    \end{macrocode}

%\iffalse
%</sampledraft>
%\fi
%
% %%%%%%%%%%%%%%%%%%%%%%%%%%%%%%%%%%%%%%
% \paragraph{Forwarding for Final Version of the Chapters.}
%
% The following forwarding files |cdocsfn1.tex| and |cdocsfn2.tex|
% (with identical content)
% compile the final versions of the child documents
% |cdocsch1.tex| and |cdocsch2.tex|, respectively:
%\iffalse
%<*samplefinal>
%\fi
%    \begin{macrocode}
\def\version{final}
\input{childdoc.def}
\childdocforwardprefix[cdocsamp]{cdocsfn}{cdocsch}
%    \end{macrocode}

%\iffalse
%</samplefinal>
%\fi
%
% %%%%%%%%%%%%%%%%%%%%%%%%%%%%%%%%%%%%%%
% \paragraph{Command Line Processing.}
%
% The following three command lines generate the output files
% |cdocscld|, |cdocscl1| and |cdocscl2|
% which should be identical to
% |cdocsdrf|, |cdocsch1| and |cdocsfn2|, respectively:
% \begin{center}
% \begin{tabular}{l}
% |latex -jobname cdocscld \|\\
% |  "\def\version{draft}\input{childdoc.def}\childdocforward{cdocsamp}"|\\
% |latex -jobname cdocscl1 \|\\
% |  "\input{childdoc.def}\childdocforward[cdocsamp]{cdocsch1}"|\\
% |latex -jobname cdocscl2 \|\\
% |  "\def\version{final}\input{childdoc.def}\childdocforward{cdocsch2}"|
% \end{tabular}
% \end{center}
% Note that the trailing backslash on each first line
% merely continues the input to the second line
% (for convenient cut ant paste).
% Furthermore, the command |latex| can be replaced by any
% of its alternative versions such as |pdflatex|.
%
% %%%%%%%%%%%%%%%%%%%%%%%%%%%%%%%%%%%%%%%%%%%%%%%%%%%%%%%%%%%%%%%%%%%%%%%%%%%%%%
% %%%%%%%%%%%%%%%%%%%%%%%%%%%%%%%%%%%%%%%%%%%%%%%%%%%%%%%%%%%%%%%%%%%%%%%%%%%%%%
% \section{Implementation}
%\iffalse
%<*package>
%\fi
%
% This section describes the definitions file |childdoc.def|.

% The definitions cannot be loaded using |\usepackage| or |\RequirePackage|
% which has a mechanism to prevent loading a style file more than once.
% When loading the definitions by means of |\input|
% multiple instances have to be prevented manually:
%\iffalse
%This code needs to be before the `\ProvidesFile' directive
%which is defined at the beginning of this file.
%Therefore it is also placed there and commented out here.
%</package>
%<*discard>
%\fi
%    \begin{macrocode}
\ifdefined\childdocmain\endinput\fi
%    \end{macrocode}
%\iffalse
%</discard>
%<*package>
%\fi
%
% \macro{\ifchilddoc}
% \macro{\ifchilddocmanual}
% The conditional |\ifchilddoc| tells whether a
% child (true) or main (false) document is being compiled.
% The conditional |\ifchilddocmanual| tells whether
% the |\includeonly| mechanism is used (false) or
% the selection of child files must be performed manually (true).
% The definitions initialise to false:
%    \begin{macrocode}
\newif\ifchilddoc
\newif\ifchilddocmanual
%    \end{macrocode}

% \macro{\childdocname}
% \macro{\childdocjob}
% The macro |\childdocname| stores the name of the main document
% to be compiled. The macro |\childdocjob| stores the name of
% the document on which the \LaTeX{} compiler was originally invoked.
% The content of |\jobname| cannot be compared
% to filenames specified in the source due to different catcodes.
% The following code rescans |\jobname|, stores the result
% in |\childdocname| and saves a copy in |\childdocjob|:
%    \begin{macrocode}
\edef\childdocname{\scantokens\expandafter{\jobname\noexpand}}
\let\childdocjob\childdocname
%    \end{macrocode}

% \macro{\childdocdisable}
% The macro |\childdocdisable| prevents the main file
% from being processed more than once.
% At this stage, the main document command |\childdocmain|
% is assumed to be called once again where it should do nothing.
% Any subsequent call to it should prevent
% a secondary processing of the main document
% It overwrites the forwarding commands
% |\childdocof| and |\childdocforward|
% with empty macros to prevent further inclusions of the main document:
%    \begin{macrocode}
\newcommand{\childdocdisable}
{
  \renewcommand{\childdocmain}[1]{\renewcommand{\childdocmain}[1]{\endinput}}
  \renewcommand{\childdocof}[1]{}
  \renewcommand{\childdocby}[2][]{}
  \renewcommand{\childdocforward}[2][]{}
  \renewcommand{\childdocdisable}{}
}
%    \end{macrocode}

% \macro{\childdocmain}
% The macro |\childdocmain| is to be called at the top of the main file
% with nothing or the main filename (without extension) as argument.
% First, it breaks loops.
% If the argument is not empty and does not match |\childdocname|
% (which is set by the first inclusion of |childdoc.def|),
% |\ifchilddoc| is set to true, |\includeonly| is applied to the child file
% and |\jobname| is set to the main file
% (for proper handling of |.aux| files):
%    \begin{macrocode}
\newcommand{\childdocmain}[1]
{
  \childdocdisable\childdocmain{}
  \if?#1?\else
    \begingroup
      \def\childdoctmp{#1}
      \ifx\childdoctmp\childdocname
        \def\childdoctmp{}
      \else
        \def\childdoctmp
        {
          \childdoctrue
          \includeonly{\childdocname}
          \def\childdocjob{#1}
          \def\jobname{#1}
        }
      \fi
      \expandafter
    \endgroup
    \childdoctmp
  \fi
}
%    \end{macrocode}

% \macro{\childdocof}
% The command |\childdocof| redirects
% compilation to the main file |#1|.
%    \begin{macrocode}
\newcommand{\childdocof}[1]
{
  \childdocdisable
  \childdoctrue
  \includeonly{\childdocname}
  \def\jobname{#1}
  \def\childdocjob{#1}
  \input{#1}
}
%    \end{macrocode}

% \macro{\childdocby}
% The command |\childdocby| ....
%    \begin{macrocode}
\newcommand{\childdocby}[2][]
{
  \childdocdisable
  \childdoctrue
  \childdocmanualtrue
  \if?#1?\else
    \def\jobname{#2}
  \fi
  \def\childdocjob{#2}
  \input{#2}
  \endinput
}
%    \end{macrocode}

% \macro{\childdocforward}
% The command |\childdocforward| redirects
% compilation to the main file or
% (if the optional argument is given) a child file.
% Parameters are set as if the main file
% or a child file starting with |\childdocof| was compiled.
% Then compilation is handed over to the main file:
%    \begin{macrocode}
\newcommand{\childdocforward}[2][]
{
  \begingroup
    \if?#1?
      \def\childdoctmp
      {
        \def\childdocname{#2}
        \def\childdocjob{#2}
        \def\jobname{#2}
        \input{#2}
        \endinput
      }
    \else
      \def\childdoctmp
      {
        \childdocdisable
        \def\childdocname{#2}
        \childdoctrue
        \includeonly{#2}
        \def\childdocjob{#1}
        \def\jobname{#1}
        \input{#1}
        \endinput
      }
    \fi
    \expandafter
  \endgroup
  \childdoctmp
}
%    \end{macrocode}

% \macro{\childdocforwardprefix}
% The command |\childdocforwardprefix| redirects
% compilation to the main or a child file by means of a pattern.
% The prefix |#1| in the current filename is replaced by |#2|
% and the suffix of the current filename is kept
% (it is assumed that the filename does not contain the substring `|~~~|'
% which is used as a delimiter).
% Compilation is handed over to the new file by |\childdocforward|:
%    \begin{macrocode}
\newcommand{\childdocforwardprefix}[3][]
{
  \begingroup
    \def\childdocextract #2##1~~~{\def\childdoctmp{\childdocforward[#1]{#3##1}}}
    \expandafter\childdocextract\childdocname~~~
    \expandafter
  \endgroup
  \childdoctmp
}
%    \end{macrocode}

% \macro{\childdoc}
% The deprecated macro |\childdoc| is a legacy version of |\childdocmain|:
%    \begin{macrocode}
\newcommand{\childdoc}{\childdocmain}
%    \end{macrocode}

% \macro{\childdocredirect}
% The deprecated macro |\childdocredirect| is a legacy version
% of |\childdocforward| and |\childdocforwardprefix|:
%    \begin{macrocode}
\newcommand{\childdocredirect}[2][]
{
  \begingroup
    \if?#1?
      \def\childdoctmp{\childdocforward{#2}}
    \else
      \def\childdoctmp{\childdocforwardprefix{#1}{#2}}
    \fi
    \expandafter
  \endgroup
  \childdoctmp
}
%    \end{macrocode}

%\iffalse
%</package>
%\fi
%
\endinput
\childdocforward[cdocsamp]{cdocsch1}"|\\
% |latex -jobname cdocscl2 \|\\
% |  "\def\version{final}% \iffalse
%
% childdoc.dtx Copyright (C) 2017-2018 Niklas Beisert
%
% This work may be distributed and/or modified under the
% conditions of the LaTeX Project Public License, either version 1.3
% of this license or (at your option) any later version.
% The latest version of this license is in
%   http://www.latex-project.org/lppl.txt
% and version 1.3 or later is part of all distributions of LaTeX
% version 2005/12/01 or later.
%
% This work has the LPPL maintenance status `maintained'.
%
% The Current Maintainer of this work is Niklas Beisert.
%
% This work consists of the files childdoc.dtx and childdoc.ins
% and the derived files childdoc.def and cdocsamp.tex with
% cdocsch1.tex, cdocsch2.tex, cdocsdrf.tex, cdocsfn1.tex, cdocsfn2.tex.
%
%<package>\ifdefined\childdocmain\endinput\fi
%<package>\ProvidesFile{childdoc.def}[2018/12/30 v2.0 child document driver]
%<samplemain>\ProvidesFile{cdocsamp.tex}[2018/12/30 v2.0 sample for childdoc]
%<*driver>
%\ProvidesFile{childdoc.drv}[2018/12/30 v2.0 childdoc reference manual file]
\PassOptionsToClass{10pt,a4paper}{article}
\documentclass{ltxdoc}

\usepackage[margin=35mm]{geometry}
\usepackage{hyperref}
\usepackage{hyperxmp}
\usepackage[usenames]{color}

\hypersetup{colorlinks=true}
\hypersetup{pdfstartview=FitH}
\hypersetup{pdfpagemode=UseNone}
\hypersetup{pdfsource={}}
\hypersetup{pdflang={en-UK}}
\hypersetup{pdfcopyright={Copyright 2017-2018 Niklas Beisert.
  This work may be distributed and/or modified under the
  conditions of the LaTeX Project Public License, either version 1.3
  of this license or (at your option) any later version.}}
\hypersetup{pdflicenseurl={http://www.latex-project.org/lppl.txt}}
\hypersetup{pdfcontactaddress={ETH Zurich, ITP, HIT K,
  Wolfgang-Pauli-Strasse 27}}
\hypersetup{pdfcontactpostcode={8093}}
\hypersetup{pdfcontactcity={Zurich}}
\hypersetup{pdfcontactcountry={Switzerland}}
\hypersetup{pdfcontactemail={nbeisert@itp.phys.ethz.ch}}
\hypersetup{pdfcontacturl={http://people.phys.ethz.ch/\xmptilde nbeisert/}}

\newcommand{\secref}[1]{\hyperref[#1]{section \ref*{#1}}}

\parskip1ex
\parindent0pt
\let\olditemize\itemize
\def\itemize{\olditemize\parskip0pt}

\begin{document}

\title{The \textsf{childdoc} Package}
\hypersetup{pdftitle={The childdoc Package}}
\author{Niklas Beisert\\[2ex]
  Institut f\"ur Theoretische Physik\\
  Eidgen\"ossische Technische Hochschule Z\"urich\\
  Wolfgang-Pauli-Strasse 27, 8093 Z\"urich, Switzerland\\[1ex]
  \href{mailto:nbeisert@itp.phys.ethz.ch}
  {\texttt{nbeisert@itp.phys.ethz.ch}}}
\hypersetup{pdfauthor={Niklas Beisert}}
\hypersetup{pdfsubject={Manual for the LaTeX2e Package childdoc}}
\date{30 December 2018, \textsf{v2.0}}
\maketitle

\begin{abstract}\noindent
\textsf{childdoc} is a \LaTeXe{} package
that enables the direct compilation
of document sections included by |\include|
to individual files.
\end{abstract}

\begingroup
\parskip0ex
\tableofcontents
\endgroup

%%%%%%%%%%%%%%%%%%%%%%%%%%%%%%%%%%%%%%%%%%%%%%%%%%%%%%%%%%%%%%%%%%%%%%%%%%%%%%%%
%%%%%%%%%%%%%%%%%%%%%%%%%%%%%%%%%%%%%%%%%%%%%%%%%%%%%%%%%%%%%%%%%%%%%%%%%%%%%%%%
\section{Introduction}

\LaTeX{} provides a mechanism to structure a large document (such as a book)
into a main file and several child files (containing the chapters)
using the |\include| command.
This mechanism is beneficial for documents
which span hundreds of pages in order to
make the source file(s) more manageable.
Moreover, compilation can be restricted to
selected child files by means of the |\includeonly| command.
The latter feature can be used to reduce the compilation time while editing
(this was significantly more useful in the earlier days of \LaTeX{})
or to generate a smaller document which is easier to navigate.
Another application of |\includeonly| is to generate
documents consisting of selected parts of the complete document.

However, there are a few drawbacks of the plain |\include| mechanism:
\begin{itemize}
\item
The child files cannot be compiled on their own,
they can only be compiled via the main file.
A naive editing environment
(such as a text editor with an option
to have the current file processed by \LaTeX)
may require one to switch to the main file before compiling;
attempting to compile the child file produces errors.
\item
The main file must be modified (each time)
to adjust the |\includeonly| command
to the present needs. This easily leaves the main file in a messy state.
\item
The generated document will always carry the filename
of the main document. This is inconvenient if
several child files are to be compiled and
to be kept for distribution.
\end{itemize}

The present package provides a simple interface
to make child files individually compilable by \LaTeX{}.
Compiling a child file then has the same effect as compiling
the main file with an |\includeonly| command
to select the appropriate child.
Moreover the generated document will carry the name of the child
rather than the main file.
This resolves all three above issues.

This feature is meant to make the editing of books,
thesis documents and lecture notes somewhat more convenient.
However, the package can also be used efficiently for
composing a series of documents (such as exercise sheets)
which are typically distributed individually.
It then assists the author in generating the individual documents
(potentially in different versions)
as well as a document containing the collected series.
Another application is in developing style files
or other kinds of included material
where compilation of the style file could redirect
to a sample or test file.

%%%%%%%%%%%%%%%%%%%%%%%%%%%%%%%%%%%%%%%%%%%%%%%%%%%%%%%%%%%%%%%%%%%%%%%%%%%%%%%%
%%%%%%%%%%%%%%%%%%%%%%%%%%%%%%%%%%%%%%%%%%%%%%%%%%%%%%%%%%%%%%%%%%%%%%%%%%%%%%%%
\section{Usage}

First of all, the package \textsf{childdoc} is \emph{not} a standard
\LaTeXe{} |.sty| style file! Therefore it needs to be invoked in
a non-standard way.

%%%%%%%%%%%%%%%%%%%%%%%%%%%%%%%%%%%%%%%%%%%%%%%%%%%%%%%%%%%%%%%%%%%%%%%%%%%%%%%%
\subsection{Included Files}
\label{sec:include}

%%%%%%%%%%%%%%%%%%%%%%%%%%%%%%%%%%%%%%%%
\DescribeMacro{\childdocmain}
To use the package, add the commands
\begin{center}
\begin{tabular}{l}
|\input{childdoc.def}|\\
|\childdocmain{}|\\
\end{tabular}
\end{center}
at the very top of the main \LaTeX{} file,
in particular \emph{before} the |\documentclass| statement!
The argument of |\childdocmain| should be left empty
(but it must be present).

%%%%%%%%%%%%%%%%%%%%%%%%%%%%%%%%%%%%%%%%
\DescribeMacro{\childdocof}
Furthermore, add the commands
\begin{center}
\begin{tabular}{l}
|\input{childdoc.def}|\\
|\childdocof{|\textit{main}|}|\\
\end{tabular}
\end{center}
at the top of every child file \textit{child}
which is included by |\include{|\textit{child}|}|
from within the main file
(or at least for those files to be compiled individually).
The argument \textit{main} must be the filename of the main file.

There are a couple of
considerations in setting up the main and child documents:

%%%%%%%%%%%%%%%%%%%%%%%%%%%%%%%%%%%%%%%%
\paragraph{Restrictions.}

Please note the following restrictions:
\begin{itemize}
\item
|\childdocmain| must be called with one argument \textit{main}
to ensure compatibility with earlier version of the package.
It must either be empty (|\childdocmain{}|)
or precisely match the filename of the main file in which it is specified.
See \secref{sec:detection} for further information.
\item
The filename \textit{main} must be specified without the |.tex| extension.
\item
The filename \textit{main} is case sensitive
(even in case-insensitive file systems)
due to internal string comparison.
\item
The argument \textit{main} should be fully expanded, it cannot be a macro.
\item
Subdirectories and special characters should be avoided in filenames.
\item
The command |\childdocmain{|\textit{main}|}| must be followed by a whitespace.
It should not be followed immediately by another command
or by a comment mark `|%|'.
This is because the \TeX{} parser reads the token immediately following
the argument of |\childdocmain| and puts it
at the beginning of every child section;
however, a white\-space is ignored.
\end{itemize}

%%%%%%%%%%%%%%%%%%%%%%%%%%%%%%%%%%%%%%%%
\paragraph{Content of Main File.}

It is advisable to place all content in the child files included by |\include|.
Any output contained in the main file will appear in all child documents
unless suppressed manually;
it cannot be suppressed automatically by the |\includeonly| directive
and thus should normally be avoided.
A method to include some content in the main file
by means of conditional processing is described in \secref{sec:conditional}.

%%%%%%%%%%%%%%%%%%%%%%%%%%%%%%%%%%%%%%%%
\paragraph{Page Numbering.}

When only a part of the document is compiled,
the appropriate numbering of pages
(as well as other status parameters)
is determined from the |.aux| files.
The latter contain information from previous passes.
However this information needs to propagate through
all intermediate child documents.
Therefore the page numbering in child documents may well
be inconsistent until the complete document is compiled at least once.

A useful (if unconventional) way to always ensure a consistent
page numbering is to restart the numbering in each child document
and denote the pages by `\textit{child}|.|\textit{page}'
where \textit{child} represents the chapter/section number of the child file.
This can be achieved by the command
|\numberwithin{page}{|\textit{child}|}|
of the \textsf{amsmath} package
where \textit{child} can be |chapter| or |section|
depending on the chosen structuring.
Alternatively, one can modify the macro |\thepage| appropriately
and reset the counter |page| at the start of each child file.

%%%%%%%%%%%%%%%%%%%%%%%%%%%%%%%%%%%%%%%%%%%%%%%%%%%%%%%%%%%%%%%%%%%%%%%%%%%%%%%%
\subsection{Conditional Processing}
\label{sec:conditional}

The package provides a mechanism to compile different versions
of a document. To customise the versions further some conditional processing
can come in handy to distinguish which version is being compiled.
The package provides two macros to describe the compilation context:

%%%%%%%%%%%%%%%%%%%%%%%%%%%%%%%%%%%%%%%%
\DescribeMacro{\ifchilddoc}
The conditional |\ifchilddoc| distinguishes between the compilation of
child documents and the main document:
%
\begin{center}
|\ifchilddoc |\textit{child-code}| |[|\||else |\textit{main-code}]| \||fi|
\end{center}

%%%%%%%%%%%%%%%%%%%%%%%%%%%%%%%%%%%%%%%%
\DescribeMacro{\childdocname}
\DescribeMacro{\childdocjob}
The macro |\childdocname| contains the filename (without extension)
of the main or child file being processed.
Note that |\childdocjob| will always contain the name of the main file.

%%%%%%%%%%%%%%%%%%%%%%%%%%%%%%%%%%%%%%%%
\paragraph{Title Page.}

Conditional processing can be used to include a title or banner page
in the main document when proper precautions are taken.
Importantly, the code in the main file should ensure that the page counter
(as well as other status parameters which are stored in the |.aux| files)
takes the same value after the conditional processing.
Otherwise the page numbers may take divergent values
depending on which part is compiled.

For example, a title page could be declared by:
%
\begin{center}
\begin{tabular}{l}
|\ifchilddoc\||else|\\
|\addtocounter{page}{-1}|\\
\textit{code for title page}\\
|\newpage|\\
|\||fi|
\end{tabular}
\end{center}
%
A banner page for the child documents can be generated by:
%
\begin{center}
\begin{tabular}{l}
|\ifchilddoc|\\
|\addtocounter{page}{-1}|\\
\textit{code for banner page}\\
|\newpage|\\
|\||fi|
\end{tabular}
\end{center}
%
Here one could write a message such as:
\begin{center}
|This is the part \childdocname{} of \childdocjob{}.|
\end{center}

%%%%%%%%%%%%%%%%%%%%%%%%%%%%%%%%%%%%%%%%%%%%%%%%%%%%%%%%%%%%%%%%%%%%%%%%%%%%%%%%
\subsection{Flags}
\label{sec:flags}

The package makes it easy to generate different versions
of the main or child documents.
To this end compilation flags can be defined
and assigned different default values.
They will be particularly useful in conjunction
with the forwarding mechanism described in \secref{sec:forward}.

For example, it may be useful to have a flag |\version|
which can be set to |draft| or |final|.
The document source will contain some conditional code
depending on the value of |\version|.
Suppose further, the flag should default to |final| for the main file
and to |draft| for child files
which is a natural assignment for editing the document.
This is achieved by placing the following code
in the preamble of the main document
(below the |\childdocmain| directive):
%
\begin{center}
\begin{tabular}{l}
|\ifchilddoc|\\
|\providecommand{\version}{draft}|\\
|\||else|\\
|\providecommand{\version}{final}|\\
|\||fi|
\end{tabular}
\end{center}
%
The definition by |\providecommand| makes sure
that previous definitions are not overwritten.
Further statements |\providecommand{\version}{...}|
can thus be added before the above code to override it.

For the main file, one might add a line
(between |\childdocmain| and the above block)
%
\begin{center}
|%\ifchilddoc\||else\providecommand{\version}{draft}\||fi|
\end{center}
%
which can be uncommented to produce a draft version.
Likewise one can add a line to the very top of a child file
(above the |\childdocof{|\textit{main}|}| directive)
%
\begin{center}
|%\providecommand{\version}{final}|
\end{center}
%
which can be uncommented to produce the final version of this child document.

%%%%%%%%%%%%%%%%%%%%%%%%%%%%%%%%%%%%%%%%%%%%%%%%%%%%%%%%%%%%%%%%%%%%%%%%%%%%%%%%
\subsection{Forwarding}
\label{sec:forward}

Different versions of the main or child documents
using compilation flags as described in \secref{sec:flags}
can be (permanently) stored in different files
for convenient compilation, viewing and distribution.
To this end, the package defines a command
to pass on compilation to a different file:

%%%%%%%%%%%%%%%%%%%%%%%%%%%%%%%%%%%%%%%%
\DescribeMacro{\childdocforward}
The command |\childdocforward| redirects processing to
another source file:
%
\begin{center}
\begin{tabular}{l}
|\input{childdoc.def}|\\
|\childdocforward[|\textit{main}|]{|\textit{dest}|}|\\
\end{tabular}
\end{center}
%
The argument \textit{dest} is the destination file
(without extension).
It should be the main file or one of the child files.
Note that further \textsf{childdoc} directives
such as |\childdocof| and |\childdocforward|
in the indicated file will be processed in this form.
The optional argument \textit{main}
passes on directly to the main file \textit{main}
while pretending to compile the child \textit{dest}.
This form behaves as if \textit{dest}
issues |\childdocof{|\textit{main}|}| right away,
and no further \textsf{childdoc} directives will be processed.

%%%%%%%%%%%%%%%%%%%%%%%%%%%%%%%%%%%%%%%%
\DescribeMacro{\...prefix}
In the alternative form |\childdocforwardprefix|,
%
\begin{center}
\begin{tabular}{l}
|\input{childdoc.def}|\\
|\childdocforwardprefix[|\textit{main}|]{|\textit{prefix}|}{|\textit{dest}|}|
\end{tabular}
\end{center}
%
the destination file is determined by a pattern
depending on the current file:
To make this work, the current file must be called
`{\textit{prefix}\hspace{0.2em}\textit{suffix}}'
with \textit{prefix} matching precisely the argument.
Processing is then passed on to the file
`{\textit{dest}\hspace{0.2em}\textit{suffix}}'.
Surely, the same effect is achieved by
directly specifying the
argument `{\textit{dest}\hspace{0.2em}\textit{suffix}}'
in the first form.
However, that requires to set up a different file
for each child. With the alternative form of the command
all these files can have exactly the same content
which simplifies setting them up and maintaining them.

For example, the following file |draft.tex|
with a compilation flag |\version| as described in \secref{sec:flags}
compiles the main document as a draft:
%
\begin{center}
\begin{tabular}{l}
|\def\version{draft}|\\
|\input{childdoc.def}|\\
|\childdocforward{|\textit{main}|}|
\end{tabular}
\end{center}
%
Likewise, the following files |final|\textit{nn}|.tex|
compile the final version of the child document
|child|\textit{nn}|.tex|:
%
\begin{center}
\begin{tabular}{l}
|\def\version{final}|\\
|\input{childdoc.def}|\\
|\childdocforwardprefix{final}{child}|
\end{tabular}
\end{center}
%

Note that when several versions of a main file and/or of each child file
are to be generated, it may be convenient to set up a |Makefile| or
shell script to automatise the process.

%%%%%%%%%%%%%%%%%%%%%%%%%%%%%%%%%%%%%%%%%%%%%%%%%%%%%%%%%%%%%%%%%%%%%%%%%%%%%%%%
\subsection{Command Line Processing}
\label{sec:commandline}

The effect of redirection files can also be achieved by invoking
the \LaTeX{} compiler with a more elaborate command line.
Most conveniently this should be done as part
of a shell script or a |Makefile|.

When using \textsf{childdoc} in the main file, the following
command lines effectively perform a redirection
(note that depending on the shell being used,
backslashes may have to be doubled: `|\|' $\to$ `|\\|'):
%
\begin{center}
|... -jobname "|\textit{target}|" |\\|"|[\textit{flags}]%
|\input{childdoc.def}\childdocforward[|\textit{main}|]{|\textit{dest}|}"|
\end{center}
%
Here \textit{target} is the name of the output file,
\textit{main} is the name of the main file
and \textit{dest} is the name of the main or child file to be processed
(all filenames without extensions).
The optional argument \textit{main} can be omitted
if \textit{main} matches \textit{dest}.
Optionally, compilation \textit{flags} can be defined via |\def| commands.
This command line makes the \TeX{} engine believe
it is compiling the file \textit{target}
whose content is specified as the latter parameter.
The provided code then forwards the processing to
\textit{main} or \textit{dest} as described in \secref{sec:forward}.

%%%%%%%%%%%%%%%%%%%%%%%%%%%%%%%%%%%%%%%%%%%%%%%%%%%%%%%%%%%%%%%%%%%%%%%%%%%%%%%%
\subsection{Include by Input}
\label{sec:input}

Including child documents by |\include| has some restrictions by design.
Most notably, the content of a child document always occupies
its own set of pages; pages cannot be shared between child documents.
Usually, this behaviour makes perfect sense
because each child document contain an essential part of the document.
However, in some situations it may be desirable to compose
a document from a collection of parts
without having mandatory page breaks between then.
For this case, the package
provides a mechanism to include parts
by |\input| which can also be processed individually.
However, by construction this mechanism
requires manual handling of the content to be output.

%%%%%%%%%%%%%%%%%%%%%%%%%%%%%%%%%%%%%%%%
\DescribeMacro{\ifchilddocmanual}
The main file should be prepared as usual, see \secref{sec:include}.
However, the document body must make a distinction
between processing of an individual part and of the main document, e.g.:
%
\begin{center}
\begin{tabular}{l}
|\ifchilddocmanual|\\
|\input{\childdocname}|\\
|\||else|\\
\textit{document body with }|\input{|\textit{part}|}|\\
|\||fi|
\end{tabular}
\end{center}
%
The conditional |\ifchilddocmanual| is true whenever
a part to be included by |\input| is being compiled,
and the name of the part is stored in |\childdocname|.

%%%%%%%%%%%%%%%%%%%%%%%%%%%%%%%%%%%%%%%%
\DescribeMacro{\childdocby}
Each part to be included by |\input| should start with:
%
\begin{center}
\begin{tabular}{l}
|\input{childdoc.def}|\\
|\childdocby{|\textit{main}|}|\\
\end{tabular}
\end{center}
%
The directive |\childdocby| is similar to |\childdocof|
described in \secref{sec:include},
but the subsequent selection of content must be done manually.
To that end, both |\ifchilddoc| and |\ifchilddocmanual|
will be true upon processing of a part,
and the name of the part is stored in |\childdocname|.
Note that |\jobname| will be set to the filename of the current part
so that each part receives an individual |.aux| file
that does not interfere with the |.aux| file(s) of the main document.
This behaviour can be altered by the alternative form
|\childdocby[*]{|\textit{main}|}| (with a non-empty optional argument)
which uses the |.aux| file of the main document
by setting |\jobname| to \textit{main}.

%%%%%%%%%%%%%%%%%%%%%%%%%%%%%%%%%%%%%%%%%%%%%%%%%%%%%%%%%%%%%%%%%%%%%%%%%%%%%%%%
\subsection{Driver Development}
\label{sec:driver}

The \textsf{childdoc} mechanism can also be use for the development
of definition files such as \LaTeX{} styles or classes.
This case differs from the above setup with multiple parts
included by |\include| in that no |\includeonly| should be invoked.
This can be achieved by starting the include file
(before |\ProvidesPackage|) with:
%
\begin{center}
\begin{tabular}{l}
|\input{childdoc.def}|\\
|\childdocforward{|\textit{main}|}|\\
\end{tabular}
\end{center}
%
or alternatively with:
%
\begin{center}
\begin{tabular}{l}
|\input{childdoc.def}|\\
|\childdocby{|\textit{main}|}|\\
\end{tabular}
\end{center}
%
Both forms have slightly different effects as described above.
The main file is prepared as usual, see \secref{sec:include}.

%%%%%%%%%%%%%%%%%%%%%%%%%%%%%%%%%%%%%%%%%%%%%%%%%%%%%%%%%%%%%%%%%%%%%%%%%%%%%%%%
\subsection{Legacy Detection}
\label{sec:detection}

The directive |\childdocmain| in the main file can detect
whether the complete document or merely a child is to be compiled
even without using the directive |\childdocof|.
This method is deprecated because it is less robust
and there is no compelling reason to use it;
it is merely provided for backward compatibility
and it may be removed in future versions.

If the detection mechanism is to be used,
it is mandatory to correctly specify
the filename of the main file as the argument of |\childdocmain|:
%
\begin{center}
\begin{tabular}{l}
|\input{childdoc.def}|\\
|\childdocmain{|\textit{main}|}|\\
\end{tabular}
\end{center}
%
If |\jobname| does not match the argument \textit{main} of |\childdocmain|,
it is assumed that |\jobname| points to the child file to be compiled.
When using |\childdocmain| with the main file specified as argument,
it suffices to start a child file
with just |\input{|\textit{main}|}|
without loading of the package and using |\childdocof|.
If instead all processing is done
with the appropriate \textsf{childdoc} directives,
the argument of \textit{main} of |\childdocmain| can be empty.

An alternative version of the command line processing described
in \secref{sec:commandline} using the detection mechanism reads:
%
\begin{center}
|... -jobname "|\textit{target}|" "|[\textit{flags}]%
[|\def\jobname{|\textit{dest}|}|]|\input{|\textit{main}|}"|
\end{center}

%%%%%%%%%%%%%%%%%%%%%%%%%%%%%%%%%%%%%%%%%%%%%%%%%%%%%%%%%%%%%%%%%%%%%%%%%%%%%%%%
\subsection{Manual Code}
\label{sec:manual}

In case one cannot be certain whether the definitions file |childdoc.def|
is installed on the target \TeX{} distribution
and one prefers not to ship it,
it is conceivable to paste a few relevant commands into the sources.

To that end, drop all statements |\input{childdoc.def}|
and perform the replacements as outlined below.
Instead of |\childdocmain{|\textit{main}|}| add the following code
to the top of the main file:
%
\begin{center}
\begin{tabular}{l}
|\||ifdefined\childdocname\endinput\||fi\newif\ifchilddoc|\\
|\edef\childdocname{\scantokens\expandafter{\jobname\noexpand}}|\\
|\def\childdocmain{|\textit{main}|}\||ifx\childdocmain\childdocname\||else|\\
|\childdoctrue\includeonly{\childdocname}\let\jobname\childdocmain\||fi|\\
\end{tabular}
\end{center}
%
Instead of |\childdocof{|\textit{main}|}| just include the main file
at the top of each child file:
%
\begin{center}
|\input{|\textit{main}|}|
\end{center}
%
A simple redirection |\childdocforward{|\textit{dest}|}| is achieved by:
%
\begin{center}
|\def\jobname{|\textit{dest}|}\input{\jobname}|
\end{center}
%
The redirection with prefix
|\childdocforwardprefix[|\textit{prefix}|]{|\textit{dest}|}|
is accomplished by:
%
\begin{center}
\begin{tabular}{l}
|{\edef\jobname{\scantokens\expandafter{\jobname\noexpand}}|\\
|\def\redirectjob |\textit{prefix}|#1~~~{\gdef\jobname{|\textit{dest}|#1}}|\\
|\expandafter\redirectjob\jobname~~~}\input{\jobname}|
\end{tabular}
\end{center}

In an alternative approach,
child documents can be compiled by a specific command line
without additional code or specific definitions:
%
\begin{center}
|... -jobname "|\textit{target}|" "|[\textit{flags}]%
|\includeonly{|\textit{dest}|}\input{|\textit{main}|}"|
\end{center}
%

%%%%%%%%%%%%%%%%%%%%%%%%%%%%%%%%%%%%%%%%%%%%%%%%%%%%%%%%%%%%%%%%%%%%%%%%%%%%%%%%
%%%%%%%%%%%%%%%%%%%%%%%%%%%%%%%%%%%%%%%%%%%%%%%%%%%%%%%%%%%%%%%%%%%%%%%%%%%%%%%%
\section{Information}

%%%%%%%%%%%%%%%%%%%%%%%%%%%%%%%%%%%%%%%%%%%%%%%%%%%%%%%%%%%%%%%%%%%%%%%%%%%%%%%%
\subsection{Copyright}

Copyright \copyright{} 2017--2018 Niklas Beisert

This work may be distributed and/or modified under the
conditions of the \LaTeX{} Project Public License, either version 1.3
of this license or (at your option) any later version.
The latest version of this license is in
  \url{http://www.latex-project.org/lppl.txt}
and version 1.3 or later is part of all distributions of \LaTeX{}
version 2005/12/01 or later.

This work has the LPPL maintenance status `maintained'.

The Current Maintainer of this work is Niklas Beisert.

This work consists of the files |README.txt|, |childdoc.ins| and |childdoc.dtx|
as well as the derived files |childdoc.def|, |cdocsamp.tex|
with |cdocsch1.tex|, |cdocsch2.tex|, |cdocspt3.tex|, |cdocspt4.tex|,
|cdocsdrf.tex|, |cdocsfn1.tex|, |cdocsfn2.tex|
as well as |childdoc.pdf|.

%%%%%%%%%%%%%%%%%%%%%%%%%%%%%%%%%%%%%%%%%%%%%%%%%%%%%%%%%%%%%%%%%%%%%%%%%%%%%%%%
\subsection{Files and Installation}

The package consists of the files:
%
\begin{center}
\begin{tabular}{ll}
    |README.txt|   & readme file \\
    |childdoc.ins| & installation file \\
    |childdoc.dtx| & source file \\
    |childdoc.def| & definition file \\
    |cdocsamp.tex| & sample main file \\
    |cdocsch1.tex| & sample include file \\
    |cdocsch2.tex| & sample include file \\
    |cdocspt3.tex| & sample part file \\
    |cdocspt4.tex| & sample part file \\
    |cdocsdrf.tex| & sample redirection file \\
    |cdocsfn1.tex| & sample redirection file \\
    |cdocsfn2.tex| & sample redirection file \\
    |childdoc.pdf| & manual
\end{tabular}
\end{center}
%
The distribution consists of the files
|README.txt|, |childdoc.ins| and |childdoc.dtx|.
%
\begin{itemize}
\item
Run (pdf)\LaTeX{} on |childdoc.dtx|
to compile the manual |childdoc.pdf| (this file).
\item
Run \LaTeX{} on |childdoc.ins| to create the definitions file |childdoc.def|
and the sample |cdocsamp.tex| with include files
|cdocsch1.tex|, |cdocsch2.tex|, |cdocspt3.tex|, |cdocspt4.tex|,
|cdocsdrf.tex|, |cdocsfn1.tex|, |cdocsfn2.tex|.
Then copy the file |childdoc.def| to an appropriate directory of your \LaTeX{}
distribution, e.g.\ \textit{texmf-root}|/tex/latex/childdoc|.
\end{itemize}

%%%%%%%%%%%%%%%%%%%%%%%%%%%%%%%%%%%%%%%%%%%%%%%%%%%%%%%%%%%%%%%%%%%%%%%%%%%%%%%%
\subsection{Related CTAN Packages}

There are several other packages which offer a similar functionality:
%
\begin{itemize}
\item
The packages
\href{http://ctan.org/pkg/docmute}{\textsf{docmute}},
\href{http://ctan.org/pkg/includex}{\textsf{includex}} and
\href{http://ctan.org/pkg/standalone}{\textsf{standalone}}
provide commands to include only the document body of
a child file thus allowing both files to be compiled individually.
\item
The packages \href{http://ctan.org/pkg/subdocs}{\textsf{subdocs}}
and \href{http://ctan.org/pkg/subfiles}{\textsf{subfiles}}
provide structures in which the main and child documents can be
encapsulated and allowing them to be compiled individually.
The inclusion mechanism is different from the conventional |\include|.
\item
The package \href{http://ctan.org/pkg/combine}{\textsf{combine}}
is an elaborate solution to combine several documents into one.
\end{itemize}
%
See also the CTAN topic \href{http://ctan.org/topic/subdocs}{\textsf{subdocs}}
for further related packages.
The present package differs from the above solutions in that
a document structure constructed with the conventional |\include| mechanism
just needs two extra commands at the top of every file
such that all constituent files can be compiled individually.

%%%%%%%%%%%%%%%%%%%%%%%%%%%%%%%%%%%%%%%%%%%%%%%%%%%%%%%%%%%%%%%%%%%%%%%%%%%%%%%%
%\subsection{Feature Suggestions}
%
%The following is a list of features which may be useful for future
%versions of this package:
%%
%\begin{itemize}
%\item
%\ldots
%\end{itemize}

%%%%%%%%%%%%%%%%%%%%%%%%%%%%%%%%%%%%%%%%%%%%%%%%%%%%%%%%%%%%%%%%%%%%%%%%%%%%%%%%
\subsection{Revision History}

%%%%%%%%%%%%%%%%%%%%%%%%%%%%%%%%%%%%%%%%
\paragraph{v2.0:} 2018/12/30

\begin{itemize}
\item
immediate forward processing
\item
added |\childdocby| mechanism
\item
manual restructured
\end{itemize}

%%%%%%%%%%%%%%%%%%%%%%%%%%%%%%%%%%%%%%%%
\paragraph{v1.6:} 2018/01/17

\begin{itemize}
\item
application for development of include files
\item
corrections to manual
\end{itemize}

%%%%%%%%%%%%%%%%%%%%%%%%%%%%%%%%%%%%%%%%
\paragraph{v1.5:} 2017/05/21

\begin{itemize}
\item
more complete structuring introduced
\item
|\childdocof| introduced
\item
|\childdoc| renamed to |\childdocmain|
\item
|\childredirect| renamed to |\childdocforward| and |\childdocforwardprefix|
and functionality expanded
\end{itemize}

%%%%%%%%%%%%%%%%%%%%%%%%%%%%%%%%%%%%%%%%
\paragraph{v1.0:} 2017/04/27

\begin{itemize}
\item
manual and install package
\item
first version published on CTAN
\end{itemize}

%%%%%%%%%%%%%%%%%%%%%%%%%%%%%%%%%%%%%%%%
\paragraph{v0.6:} 2017/04/26

\begin{itemize}
\item
redirection mechanism added
\end{itemize}

%%%%%%%%%%%%%%%%%%%%%%%%%%%%%%%%%%%%%%%%
\paragraph{v0.5:} 2017/04/26

\begin{itemize}
\item
functionality in definition file
\end{itemize}


%%%%%%%%%%%%%%%%%%%%%%%%%%%%%%%%%%%%%%%%%%%%%%%%%%%%%%%%%%%%%%%%%%%%%%%%%%%%%%%%
%%%%%%%%%%%%%%%%%%%%%%%%%%%%%%%%%%%%%%%%%%%%%%%%%%%%%%%%%%%%%%%%%%%%%%%%%%%%%%%%
%%%%%%%%%%%%%%%%%%%%%%%%%%%%%%%%%%%%%%%%%%%%%%%%%%%%%%%%%%%%%%%%%%%%%%%%%%%%%%%%
\appendix

\settowidth\MacroIndent{\rmfamily\scriptsize 000\ }

 \DocInput{childdoc.dtx}

\end{document}
%</driver>
% \fi
%
% %%%%%%%%%%%%%%%%%%%%%%%%%%%%%%%%%%%%%%%%%%%%%%%%%%%%%%%%%%%%%%%%%%%%%%%%%%%%%%
% %%%%%%%%%%%%%%%%%%%%%%%%%%%%%%%%%%%%%%%%%%%%%%%%%%%%%%%%%%%%%%%%%%%%%%%%%%%%%%
% \section{Sample}
%\iffalse
%<*samplemain>
%\fi
%
% The following presents a sample document
% with two chapters, two parts, a title page,
% a compile flag as well as three forwarding files to set the flag.
% It consists of eight |.tex| files:
% \begin{center}
% \begin{tabular}{ll}
% |cdocsamp.tex|&main file\\
% |cdocsch1.tex|&include file for chapter 1\\
% |cdocsch2.tex|&include file for chapter 2\\
% |cdocspt3.tex|&include file for part 3\\
% |cdocspt4.tex|&include file for part 4\\
% |cdocsdrf.tex|&forwarding file for main file in draft mode\\
% |cdocsfi1.tex|&forwarding file for final version of chapter 1\\
% |cdocsfi2.tex|&forwarding file for final version of chapter 2\\
% \end{tabular}
% \end{center}
% Each of the eight files can be compiled directly by the \LaTeX{} compiler.
%
% %%%%%%%%%%%%%%%%%%%%%%%%%%%%%%%%%%%%%%
% \paragraph{Main File.}
%
% The main file is called |cdocsamp.tex|.
%
% Load the \textsf{childdoc} definitions and
% declare the filename for the main document:
%    \begin{macrocode}
\input{childdoc.def}
\childdocmain{}
%    \end{macrocode}

% Optional override for |\version| flag:
%    \begin{macrocode}
%%\ifchilddoc\else\providecommand{\version}{draft}\fi
%    \end{macrocode}

% Define the default values for the |\version| flag
% (|final| for the main file and |draft| for childs):
%    \begin{macrocode}
\ifchilddoc
\providecommand{\version}{draft}
\else
\providecommand{\version}{final}
\fi
%    \end{macrocode}

% Load the standard document class:
%    \begin{macrocode}
\documentclass[12pt]{article}
%    \end{macrocode}

% Start the document body:
%    \begin{macrocode}
\begin{document}
%    \end{macrocode}

% Declare a title page.
% Print title, part of document being processed and version flag:
%    \begin{macrocode}
\addtocounter{page}{-1}
\begin{center}
{\LARGE\bfseries{}childdoc example\par}
\vspace{1cm}
\ifchilddoc
\ifchilddocmanual part\else chapter\fi:
`\childdocname' of `\childdocjob'\par
\else
main document: `\childdocjob'\par
\fi
version: \version\par
\end{center}
\newpage
%    \end{macrocode}

% Manually include selected file,
% otherwise process as usual:
%    \begin{macrocode}
\ifchilddocmanual
\section*{part `\childdocname'}
\input{\childdocname}
\else
%    \end{macrocode}

% Include the two chapters:
%    \begin{macrocode}
\include{cdocsch1}
\include{cdocsch2}
%    \end{macrocode}

% Include the two parts unless only chapters should be displayed:
%    \begin{macrocode}
\ifchilddoc\else
\section{part three}
\input{cdocspt3}
\section{part four}
\input{cdocspt4}
\fi
%    \end{macrocode}

% Process as usual until here:
%    \begin{macrocode}
\fi
%    \end{macrocode}

% End of document body:
%    \begin{macrocode}
\end{document}
%    \end{macrocode}
%\iffalse
%</samplemain>
%\fi
%
% %%%%%%%%%%%%%%%%%%%%%%%%%%%%%%%%%%%%%%
% \paragraph{Chapter Include Files.}
%
% The include files are called |cdocsch1.tex| and |cdocsch2.tex|.
%
%\iffalse
%<*samplechap1|samplechap2>
%\fi

% Optional override for |\version| flag:
%    \begin{macrocode}
%%\providecommand{\version}{final}
%    \end{macrocode}

% Include the main document:
%    \begin{macrocode}
\input{childdoc.def}
\childdocof{cdocsamp}
%    \end{macrocode}

%\iffalse
%</samplechap1|samplechap2>
%\fi
%
%\iffalse
%<*samplechap1>
%\fi
% Some text for chapter 1:
%    \begin{macrocode}
\section{one}
some text in chapter one
%    \end{macrocode}

%\iffalse
%</samplechap1>
%\fi
% Some text for chapter 2:
%\iffalse
%<*samplechap2>
%\fi
%    \begin{macrocode}
\section{two}
more text in chapter two
%    \end{macrocode}

%\iffalse
%</samplechap2>
%\fi
%
% %%%%%%%%%%%%%%%%%%%%%%%%%%%%%%%%%%%%%%
% \paragraph{Part Include Files.}
%
% The include files are called |cdocspt3.tex| and |cdocspt4.tex|.
%
%\iffalse
%<*samplepart3|samplepart4>
%\fi

% Optional override for |\version| flag:
%    \begin{macrocode}
%%\providecommand{\version}{final}
%    \end{macrocode}

% Include the main document:
%    \begin{macrocode}
\input{childdoc.def}
\childdocby{cdocsamp}
%    \end{macrocode}

%\iffalse
%</samplepart3|samplepart4>
%\fi
%
%\iffalse
%<*samplepart3>
%\fi
% Some text for part 3:
%    \begin{macrocode}
some text in part three
%    \end{macrocode}

%\iffalse
%</samplepart3>
%\fi
% Some text for part 4:
%\iffalse
%<*samplepart4>
%\fi
%    \begin{macrocode}
more text in part four
%    \end{macrocode}

%\iffalse
%</samplepart4>
%\fi
%
% %%%%%%%%%%%%%%%%%%%%%%%%%%%%%%%%%%%%%%
% \paragraph{Forwarding for a Complete Draft.}
%
% The following forwarding file |cdocsdrf.tex|
% compiles the main document in draft mode:
%\iffalse
%<*sampledraft>
%\fi
%    \begin{macrocode}
\def\version{draft}
\input{childdoc.def}
\childdocforward{cdocsamp}
%    \end{macrocode}

%\iffalse
%</sampledraft>
%\fi
%
% %%%%%%%%%%%%%%%%%%%%%%%%%%%%%%%%%%%%%%
% \paragraph{Forwarding for Final Version of the Chapters.}
%
% The following forwarding files |cdocsfn1.tex| and |cdocsfn2.tex|
% (with identical content)
% compile the final versions of the child documents
% |cdocsch1.tex| and |cdocsch2.tex|, respectively:
%\iffalse
%<*samplefinal>
%\fi
%    \begin{macrocode}
\def\version{final}
\input{childdoc.def}
\childdocforwardprefix[cdocsamp]{cdocsfn}{cdocsch}
%    \end{macrocode}

%\iffalse
%</samplefinal>
%\fi
%
% %%%%%%%%%%%%%%%%%%%%%%%%%%%%%%%%%%%%%%
% \paragraph{Command Line Processing.}
%
% The following three command lines generate the output files
% |cdocscld|, |cdocscl1| and |cdocscl2|
% which should be identical to
% |cdocsdrf|, |cdocsch1| and |cdocsfn2|, respectively:
% \begin{center}
% \begin{tabular}{l}
% |latex -jobname cdocscld \|\\
% |  "\def\version{draft}\input{childdoc.def}\childdocforward{cdocsamp}"|\\
% |latex -jobname cdocscl1 \|\\
% |  "\input{childdoc.def}\childdocforward[cdocsamp]{cdocsch1}"|\\
% |latex -jobname cdocscl2 \|\\
% |  "\def\version{final}\input{childdoc.def}\childdocforward{cdocsch2}"|
% \end{tabular}
% \end{center}
% Note that the trailing backslash on each first line
% merely continues the input to the second line
% (for convenient cut ant paste).
% Furthermore, the command |latex| can be replaced by any
% of its alternative versions such as |pdflatex|.
%
% %%%%%%%%%%%%%%%%%%%%%%%%%%%%%%%%%%%%%%%%%%%%%%%%%%%%%%%%%%%%%%%%%%%%%%%%%%%%%%
% %%%%%%%%%%%%%%%%%%%%%%%%%%%%%%%%%%%%%%%%%%%%%%%%%%%%%%%%%%%%%%%%%%%%%%%%%%%%%%
% \section{Implementation}
%\iffalse
%<*package>
%\fi
%
% This section describes the definitions file |childdoc.def|.

% The definitions cannot be loaded using |\usepackage| or |\RequirePackage|
% which has a mechanism to prevent loading a style file more than once.
% When loading the definitions by means of |\input|
% multiple instances have to be prevented manually:
%\iffalse
%This code needs to be before the `\ProvidesFile' directive
%which is defined at the beginning of this file.
%Therefore it is also placed there and commented out here.
%</package>
%<*discard>
%\fi
%    \begin{macrocode}
\ifdefined\childdocmain\endinput\fi
%    \end{macrocode}
%\iffalse
%</discard>
%<*package>
%\fi
%
% \macro{\ifchilddoc}
% \macro{\ifchilddocmanual}
% The conditional |\ifchilddoc| tells whether a
% child (true) or main (false) document is being compiled.
% The conditional |\ifchilddocmanual| tells whether
% the |\includeonly| mechanism is used (false) or
% the selection of child files must be performed manually (true).
% The definitions initialise to false:
%    \begin{macrocode}
\newif\ifchilddoc
\newif\ifchilddocmanual
%    \end{macrocode}

% \macro{\childdocname}
% \macro{\childdocjob}
% The macro |\childdocname| stores the name of the main document
% to be compiled. The macro |\childdocjob| stores the name of
% the document on which the \LaTeX{} compiler was originally invoked.
% The content of |\jobname| cannot be compared
% to filenames specified in the source due to different catcodes.
% The following code rescans |\jobname|, stores the result
% in |\childdocname| and saves a copy in |\childdocjob|:
%    \begin{macrocode}
\edef\childdocname{\scantokens\expandafter{\jobname\noexpand}}
\let\childdocjob\childdocname
%    \end{macrocode}

% \macro{\childdocdisable}
% The macro |\childdocdisable| prevents the main file
% from being processed more than once.
% At this stage, the main document command |\childdocmain|
% is assumed to be called once again where it should do nothing.
% Any subsequent call to it should prevent
% a secondary processing of the main document
% It overwrites the forwarding commands
% |\childdocof| and |\childdocforward|
% with empty macros to prevent further inclusions of the main document:
%    \begin{macrocode}
\newcommand{\childdocdisable}
{
  \renewcommand{\childdocmain}[1]{\renewcommand{\childdocmain}[1]{\endinput}}
  \renewcommand{\childdocof}[1]{}
  \renewcommand{\childdocby}[2][]{}
  \renewcommand{\childdocforward}[2][]{}
  \renewcommand{\childdocdisable}{}
}
%    \end{macrocode}

% \macro{\childdocmain}
% The macro |\childdocmain| is to be called at the top of the main file
% with nothing or the main filename (without extension) as argument.
% First, it breaks loops.
% If the argument is not empty and does not match |\childdocname|
% (which is set by the first inclusion of |childdoc.def|),
% |\ifchilddoc| is set to true, |\includeonly| is applied to the child file
% and |\jobname| is set to the main file
% (for proper handling of |.aux| files):
%    \begin{macrocode}
\newcommand{\childdocmain}[1]
{
  \childdocdisable\childdocmain{}
  \if?#1?\else
    \begingroup
      \def\childdoctmp{#1}
      \ifx\childdoctmp\childdocname
        \def\childdoctmp{}
      \else
        \def\childdoctmp
        {
          \childdoctrue
          \includeonly{\childdocname}
          \def\childdocjob{#1}
          \def\jobname{#1}
        }
      \fi
      \expandafter
    \endgroup
    \childdoctmp
  \fi
}
%    \end{macrocode}

% \macro{\childdocof}
% The command |\childdocof| redirects
% compilation to the main file |#1|.
%    \begin{macrocode}
\newcommand{\childdocof}[1]
{
  \childdocdisable
  \childdoctrue
  \includeonly{\childdocname}
  \def\jobname{#1}
  \def\childdocjob{#1}
  \input{#1}
}
%    \end{macrocode}

% \macro{\childdocby}
% The command |\childdocby| ....
%    \begin{macrocode}
\newcommand{\childdocby}[2][]
{
  \childdocdisable
  \childdoctrue
  \childdocmanualtrue
  \if?#1?\else
    \def\jobname{#2}
  \fi
  \def\childdocjob{#2}
  \input{#2}
  \endinput
}
%    \end{macrocode}

% \macro{\childdocforward}
% The command |\childdocforward| redirects
% compilation to the main file or
% (if the optional argument is given) a child file.
% Parameters are set as if the main file
% or a child file starting with |\childdocof| was compiled.
% Then compilation is handed over to the main file:
%    \begin{macrocode}
\newcommand{\childdocforward}[2][]
{
  \begingroup
    \if?#1?
      \def\childdoctmp
      {
        \def\childdocname{#2}
        \def\childdocjob{#2}
        \def\jobname{#2}
        \input{#2}
        \endinput
      }
    \else
      \def\childdoctmp
      {
        \childdocdisable
        \def\childdocname{#2}
        \childdoctrue
        \includeonly{#2}
        \def\childdocjob{#1}
        \def\jobname{#1}
        \input{#1}
        \endinput
      }
    \fi
    \expandafter
  \endgroup
  \childdoctmp
}
%    \end{macrocode}

% \macro{\childdocforwardprefix}
% The command |\childdocforwardprefix| redirects
% compilation to the main or a child file by means of a pattern.
% The prefix |#1| in the current filename is replaced by |#2|
% and the suffix of the current filename is kept
% (it is assumed that the filename does not contain the substring `|~~~|'
% which is used as a delimiter).
% Compilation is handed over to the new file by |\childdocforward|:
%    \begin{macrocode}
\newcommand{\childdocforwardprefix}[3][]
{
  \begingroup
    \def\childdocextract #2##1~~~{\def\childdoctmp{\childdocforward[#1]{#3##1}}}
    \expandafter\childdocextract\childdocname~~~
    \expandafter
  \endgroup
  \childdoctmp
}
%    \end{macrocode}

% \macro{\childdoc}
% The deprecated macro |\childdoc| is a legacy version of |\childdocmain|:
%    \begin{macrocode}
\newcommand{\childdoc}{\childdocmain}
%    \end{macrocode}

% \macro{\childdocredirect}
% The deprecated macro |\childdocredirect| is a legacy version
% of |\childdocforward| and |\childdocforwardprefix|:
%    \begin{macrocode}
\newcommand{\childdocredirect}[2][]
{
  \begingroup
    \if?#1?
      \def\childdoctmp{\childdocforward{#2}}
    \else
      \def\childdoctmp{\childdocforwardprefix{#1}{#2}}
    \fi
    \expandafter
  \endgroup
  \childdoctmp
}
%    \end{macrocode}

%\iffalse
%</package>
%\fi
%
\endinput
\childdocforward{cdocsch2}"|
% \end{tabular}
% \end{center}
% Note that the trailing backslash on each first line
% merely continues the input to the second line
% (for convenient cut ant paste).
% Furthermore, the command |latex| can be replaced by any
% of its alternative versions such as |pdflatex|.
%
% %%%%%%%%%%%%%%%%%%%%%%%%%%%%%%%%%%%%%%%%%%%%%%%%%%%%%%%%%%%%%%%%%%%%%%%%%%%%%%
% %%%%%%%%%%%%%%%%%%%%%%%%%%%%%%%%%%%%%%%%%%%%%%%%%%%%%%%%%%%%%%%%%%%%%%%%%%%%%%
% \section{Implementation}
%\iffalse
%<*package>
%\fi
%
% This section describes the definitions file |childdoc.def|.

% The definitions cannot be loaded using |\usepackage| or |\RequirePackage|
% which has a mechanism to prevent loading a style file more than once.
% When loading the definitions by means of |\input|
% multiple instances have to be prevented manually:
%\iffalse
%This code needs to be before the `\ProvidesFile' directive
%which is defined at the beginning of this file.
%Therefore it is also placed there and commented out here.
%</package>
%<*discard>
%\fi
%    \begin{macrocode}
\ifdefined\childdocmain\endinput\fi
%    \end{macrocode}
%\iffalse
%</discard>
%<*package>
%\fi
%
% \macro{\ifchilddoc}
% \macro{\ifchilddocmanual}
% The conditional |\ifchilddoc| tells whether a
% child (true) or main (false) document is being compiled.
% The conditional |\ifchilddocmanual| tells whether
% the |\includeonly| mechanism is used (false) or
% the selection of child files must be performed manually (true).
% The definitions initialise to false:
%    \begin{macrocode}
\newif\ifchilddoc
\newif\ifchilddocmanual
%    \end{macrocode}

% \macro{\childdocname}
% \macro{\childdocjob}
% The macro |\childdocname| stores the name of the main document
% to be compiled. The macro |\childdocjob| stores the name of
% the document on which the \LaTeX{} compiler was originally invoked.
% The content of |\jobname| cannot be compared
% to filenames specified in the source due to different catcodes.
% The following code rescans |\jobname|, stores the result
% in |\childdocname| and saves a copy in |\childdocjob|:
%    \begin{macrocode}
\edef\childdocname{\scantokens\expandafter{\jobname\noexpand}}
\let\childdocjob\childdocname
%    \end{macrocode}

% \macro{\childdocdisable}
% The macro |\childdocdisable| prevents the main file
% from being processed more than once.
% At this stage, the main document command |\childdocmain|
% is assumed to be called once again where it should do nothing.
% Any subsequent call to it should prevent
% a secondary processing of the main document
% It overwrites the forwarding commands
% |\childdocof| and |\childdocforward|
% with empty macros to prevent further inclusions of the main document:
%    \begin{macrocode}
\newcommand{\childdocdisable}
{
  \renewcommand{\childdocmain}[1]{\renewcommand{\childdocmain}[1]{\endinput}}
  \renewcommand{\childdocof}[1]{}
  \renewcommand{\childdocby}[2][]{}
  \renewcommand{\childdocforward}[2][]{}
  \renewcommand{\childdocdisable}{}
}
%    \end{macrocode}

% \macro{\childdocmain}
% The macro |\childdocmain| is to be called at the top of the main file
% with nothing or the main filename (without extension) as argument.
% First, it breaks loops.
% If the argument is not empty and does not match |\childdocname|
% (which is set by the first inclusion of |childdoc.def|),
% |\ifchilddoc| is set to true, |\includeonly| is applied to the child file
% and |\jobname| is set to the main file
% (for proper handling of |.aux| files):
%    \begin{macrocode}
\newcommand{\childdocmain}[1]
{
  \childdocdisable\childdocmain{}
  \if?#1?\else
    \begingroup
      \def\childdoctmp{#1}
      \ifx\childdoctmp\childdocname
        \def\childdoctmp{}
      \else
        \def\childdoctmp
        {
          \childdoctrue
          \includeonly{\childdocname}
          \def\childdocjob{#1}
          \def\jobname{#1}
        }
      \fi
      \expandafter
    \endgroup
    \childdoctmp
  \fi
}
%    \end{macrocode}

% \macro{\childdocof}
% The command |\childdocof| redirects
% compilation to the main file |#1|.
%    \begin{macrocode}
\newcommand{\childdocof}[1]
{
  \childdocdisable
  \childdoctrue
  \includeonly{\childdocname}
  \def\jobname{#1}
  \def\childdocjob{#1}
  \input{#1}
}
%    \end{macrocode}

% \macro{\childdocby}
% The command |\childdocby| ....
%    \begin{macrocode}
\newcommand{\childdocby}[2][]
{
  \childdocdisable
  \childdoctrue
  \childdocmanualtrue
  \if?#1?\else
    \def\jobname{#2}
  \fi
  \def\childdocjob{#2}
  \input{#2}
  \endinput
}
%    \end{macrocode}

% \macro{\childdocforward}
% The command |\childdocforward| redirects
% compilation to the main file or
% (if the optional argument is given) a child file.
% Parameters are set as if the main file
% or a child file starting with |\childdocof| was compiled.
% Then compilation is handed over to the main file:
%    \begin{macrocode}
\newcommand{\childdocforward}[2][]
{
  \begingroup
    \if?#1?
      \def\childdoctmp
      {
        \def\childdocname{#2}
        \def\childdocjob{#2}
        \def\jobname{#2}
        \input{#2}
        \endinput
      }
    \else
      \def\childdoctmp
      {
        \childdocdisable
        \def\childdocname{#2}
        \childdoctrue
        \includeonly{#2}
        \def\childdocjob{#1}
        \def\jobname{#1}
        \input{#1}
        \endinput
      }
    \fi
    \expandafter
  \endgroup
  \childdoctmp
}
%    \end{macrocode}

% \macro{\childdocforwardprefix}
% The command |\childdocforwardprefix| redirects
% compilation to the main or a child file by means of a pattern.
% The prefix |#1| in the current filename is replaced by |#2|
% and the suffix of the current filename is kept
% (it is assumed that the filename does not contain the substring `|~~~|'
% which is used as a delimiter).
% Compilation is handed over to the new file by |\childdocforward|:
%    \begin{macrocode}
\newcommand{\childdocforwardprefix}[3][]
{
  \begingroup
    \def\childdocextract #2##1~~~{\def\childdoctmp{\childdocforward[#1]{#3##1}}}
    \expandafter\childdocextract\childdocname~~~
    \expandafter
  \endgroup
  \childdoctmp
}
%    \end{macrocode}

% \macro{\childdoc}
% The deprecated macro |\childdoc| is a legacy version of |\childdocmain|:
%    \begin{macrocode}
\newcommand{\childdoc}{\childdocmain}
%    \end{macrocode}

% \macro{\childdocredirect}
% The deprecated macro |\childdocredirect| is a legacy version
% of |\childdocforward| and |\childdocforwardprefix|:
%    \begin{macrocode}
\newcommand{\childdocredirect}[2][]
{
  \begingroup
    \if?#1?
      \def\childdoctmp{\childdocforward{#2}}
    \else
      \def\childdoctmp{\childdocforwardprefix{#1}{#2}}
    \fi
    \expandafter
  \endgroup
  \childdoctmp
}
%    \end{macrocode}

%\iffalse
%</package>
%\fi
%
\endinput
|\\
|\childdocforwardprefix{final}{child}|
\end{tabular}
\end{center}
%

Note that when several versions of a main file and/or of each child file
are to be generated, it may be convenient to set up a |Makefile| or
shell script to automatise the process.

%%%%%%%%%%%%%%%%%%%%%%%%%%%%%%%%%%%%%%%%%%%%%%%%%%%%%%%%%%%%%%%%%%%%%%%%%%%%%%%%
\subsection{Command Line Processing}
\label{sec:commandline}

The effect of redirection files can also be achieved by invoking
the \LaTeX{} compiler with a more elaborate command line.
Most conveniently this should be done as part
of a shell script or a |Makefile|.

When using \textsf{childdoc} in the main file, the following
command lines effectively perform a redirection
(note that depending on the shell being used,
backslashes may have to be doubled: `|\|' $\to$ `|\\|'):
%
\begin{center}
|... -jobname "|\textit{target}|" |\\|"|[\textit{flags}]%
|% \iffalse
%
% childdoc.dtx Copyright (C) 2017-2018 Niklas Beisert
%
% This work may be distributed and/or modified under the
% conditions of the LaTeX Project Public License, either version 1.3
% of this license or (at your option) any later version.
% The latest version of this license is in
%   http://www.latex-project.org/lppl.txt
% and version 1.3 or later is part of all distributions of LaTeX
% version 2005/12/01 or later.
%
% This work has the LPPL maintenance status `maintained'.
%
% The Current Maintainer of this work is Niklas Beisert.
%
% This work consists of the files childdoc.dtx and childdoc.ins
% and the derived files childdoc.def and cdocsamp.tex with
% cdocsch1.tex, cdocsch2.tex, cdocsdrf.tex, cdocsfn1.tex, cdocsfn2.tex.
%
%<package>\ifdefined\childdocmain\endinput\fi
%<package>\ProvidesFile{childdoc.def}[2018/12/30 v2.0 child document driver]
%<samplemain>\ProvidesFile{cdocsamp.tex}[2018/12/30 v2.0 sample for childdoc]
%<*driver>
%\ProvidesFile{childdoc.drv}[2018/12/30 v2.0 childdoc reference manual file]
\PassOptionsToClass{10pt,a4paper}{article}
\documentclass{ltxdoc}

\usepackage[margin=35mm]{geometry}
\usepackage{hyperref}
\usepackage{hyperxmp}
\usepackage[usenames]{color}

\hypersetup{colorlinks=true}
\hypersetup{pdfstartview=FitH}
\hypersetup{pdfpagemode=UseNone}
\hypersetup{pdfsource={}}
\hypersetup{pdflang={en-UK}}
\hypersetup{pdfcopyright={Copyright 2017-2018 Niklas Beisert.
  This work may be distributed and/or modified under the
  conditions of the LaTeX Project Public License, either version 1.3
  of this license or (at your option) any later version.}}
\hypersetup{pdflicenseurl={http://www.latex-project.org/lppl.txt}}
\hypersetup{pdfcontactaddress={ETH Zurich, ITP, HIT K,
  Wolfgang-Pauli-Strasse 27}}
\hypersetup{pdfcontactpostcode={8093}}
\hypersetup{pdfcontactcity={Zurich}}
\hypersetup{pdfcontactcountry={Switzerland}}
\hypersetup{pdfcontactemail={nbeisert@itp.phys.ethz.ch}}
\hypersetup{pdfcontacturl={http://people.phys.ethz.ch/\xmptilde nbeisert/}}

\newcommand{\secref}[1]{\hyperref[#1]{section \ref*{#1}}}

\parskip1ex
\parindent0pt
\let\olditemize\itemize
\def\itemize{\olditemize\parskip0pt}

\begin{document}

\title{The \textsf{childdoc} Package}
\hypersetup{pdftitle={The childdoc Package}}
\author{Niklas Beisert\\[2ex]
  Institut f\"ur Theoretische Physik\\
  Eidgen\"ossische Technische Hochschule Z\"urich\\
  Wolfgang-Pauli-Strasse 27, 8093 Z\"urich, Switzerland\\[1ex]
  \href{mailto:nbeisert@itp.phys.ethz.ch}
  {\texttt{nbeisert@itp.phys.ethz.ch}}}
\hypersetup{pdfauthor={Niklas Beisert}}
\hypersetup{pdfsubject={Manual for the LaTeX2e Package childdoc}}
\date{30 December 2018, \textsf{v2.0}}
\maketitle

\begin{abstract}\noindent
\textsf{childdoc} is a \LaTeXe{} package
that enables the direct compilation
of document sections included by |\include|
to individual files.
\end{abstract}

\begingroup
\parskip0ex
\tableofcontents
\endgroup

%%%%%%%%%%%%%%%%%%%%%%%%%%%%%%%%%%%%%%%%%%%%%%%%%%%%%%%%%%%%%%%%%%%%%%%%%%%%%%%%
%%%%%%%%%%%%%%%%%%%%%%%%%%%%%%%%%%%%%%%%%%%%%%%%%%%%%%%%%%%%%%%%%%%%%%%%%%%%%%%%
\section{Introduction}

\LaTeX{} provides a mechanism to structure a large document (such as a book)
into a main file and several child files (containing the chapters)
using the |\include| command.
This mechanism is beneficial for documents
which span hundreds of pages in order to
make the source file(s) more manageable.
Moreover, compilation can be restricted to
selected child files by means of the |\includeonly| command.
The latter feature can be used to reduce the compilation time while editing
(this was significantly more useful in the earlier days of \LaTeX{})
or to generate a smaller document which is easier to navigate.
Another application of |\includeonly| is to generate
documents consisting of selected parts of the complete document.

However, there are a few drawbacks of the plain |\include| mechanism:
\begin{itemize}
\item
The child files cannot be compiled on their own,
they can only be compiled via the main file.
A naive editing environment
(such as a text editor with an option
to have the current file processed by \LaTeX)
may require one to switch to the main file before compiling;
attempting to compile the child file produces errors.
\item
The main file must be modified (each time)
to adjust the |\includeonly| command
to the present needs. This easily leaves the main file in a messy state.
\item
The generated document will always carry the filename
of the main document. This is inconvenient if
several child files are to be compiled and
to be kept for distribution.
\end{itemize}

The present package provides a simple interface
to make child files individually compilable by \LaTeX{}.
Compiling a child file then has the same effect as compiling
the main file with an |\includeonly| command
to select the appropriate child.
Moreover the generated document will carry the name of the child
rather than the main file.
This resolves all three above issues.

This feature is meant to make the editing of books,
thesis documents and lecture notes somewhat more convenient.
However, the package can also be used efficiently for
composing a series of documents (such as exercise sheets)
which are typically distributed individually.
It then assists the author in generating the individual documents
(potentially in different versions)
as well as a document containing the collected series.
Another application is in developing style files
or other kinds of included material
where compilation of the style file could redirect
to a sample or test file.

%%%%%%%%%%%%%%%%%%%%%%%%%%%%%%%%%%%%%%%%%%%%%%%%%%%%%%%%%%%%%%%%%%%%%%%%%%%%%%%%
%%%%%%%%%%%%%%%%%%%%%%%%%%%%%%%%%%%%%%%%%%%%%%%%%%%%%%%%%%%%%%%%%%%%%%%%%%%%%%%%
\section{Usage}

First of all, the package \textsf{childdoc} is \emph{not} a standard
\LaTeXe{} |.sty| style file! Therefore it needs to be invoked in
a non-standard way.

%%%%%%%%%%%%%%%%%%%%%%%%%%%%%%%%%%%%%%%%%%%%%%%%%%%%%%%%%%%%%%%%%%%%%%%%%%%%%%%%
\subsection{Included Files}
\label{sec:include}

%%%%%%%%%%%%%%%%%%%%%%%%%%%%%%%%%%%%%%%%
\DescribeMacro{\childdocmain}
To use the package, add the commands
\begin{center}
\begin{tabular}{l}
|% \iffalse
%
% childdoc.dtx Copyright (C) 2017-2018 Niklas Beisert
%
% This work may be distributed and/or modified under the
% conditions of the LaTeX Project Public License, either version 1.3
% of this license or (at your option) any later version.
% The latest version of this license is in
%   http://www.latex-project.org/lppl.txt
% and version 1.3 or later is part of all distributions of LaTeX
% version 2005/12/01 or later.
%
% This work has the LPPL maintenance status `maintained'.
%
% The Current Maintainer of this work is Niklas Beisert.
%
% This work consists of the files childdoc.dtx and childdoc.ins
% and the derived files childdoc.def and cdocsamp.tex with
% cdocsch1.tex, cdocsch2.tex, cdocsdrf.tex, cdocsfn1.tex, cdocsfn2.tex.
%
%<package>\ifdefined\childdocmain\endinput\fi
%<package>\ProvidesFile{childdoc.def}[2018/12/30 v2.0 child document driver]
%<samplemain>\ProvidesFile{cdocsamp.tex}[2018/12/30 v2.0 sample for childdoc]
%<*driver>
%\ProvidesFile{childdoc.drv}[2018/12/30 v2.0 childdoc reference manual file]
\PassOptionsToClass{10pt,a4paper}{article}
\documentclass{ltxdoc}

\usepackage[margin=35mm]{geometry}
\usepackage{hyperref}
\usepackage{hyperxmp}
\usepackage[usenames]{color}

\hypersetup{colorlinks=true}
\hypersetup{pdfstartview=FitH}
\hypersetup{pdfpagemode=UseNone}
\hypersetup{pdfsource={}}
\hypersetup{pdflang={en-UK}}
\hypersetup{pdfcopyright={Copyright 2017-2018 Niklas Beisert.
  This work may be distributed and/or modified under the
  conditions of the LaTeX Project Public License, either version 1.3
  of this license or (at your option) any later version.}}
\hypersetup{pdflicenseurl={http://www.latex-project.org/lppl.txt}}
\hypersetup{pdfcontactaddress={ETH Zurich, ITP, HIT K,
  Wolfgang-Pauli-Strasse 27}}
\hypersetup{pdfcontactpostcode={8093}}
\hypersetup{pdfcontactcity={Zurich}}
\hypersetup{pdfcontactcountry={Switzerland}}
\hypersetup{pdfcontactemail={nbeisert@itp.phys.ethz.ch}}
\hypersetup{pdfcontacturl={http://people.phys.ethz.ch/\xmptilde nbeisert/}}

\newcommand{\secref}[1]{\hyperref[#1]{section \ref*{#1}}}

\parskip1ex
\parindent0pt
\let\olditemize\itemize
\def\itemize{\olditemize\parskip0pt}

\begin{document}

\title{The \textsf{childdoc} Package}
\hypersetup{pdftitle={The childdoc Package}}
\author{Niklas Beisert\\[2ex]
  Institut f\"ur Theoretische Physik\\
  Eidgen\"ossische Technische Hochschule Z\"urich\\
  Wolfgang-Pauli-Strasse 27, 8093 Z\"urich, Switzerland\\[1ex]
  \href{mailto:nbeisert@itp.phys.ethz.ch}
  {\texttt{nbeisert@itp.phys.ethz.ch}}}
\hypersetup{pdfauthor={Niklas Beisert}}
\hypersetup{pdfsubject={Manual for the LaTeX2e Package childdoc}}
\date{30 December 2018, \textsf{v2.0}}
\maketitle

\begin{abstract}\noindent
\textsf{childdoc} is a \LaTeXe{} package
that enables the direct compilation
of document sections included by |\include|
to individual files.
\end{abstract}

\begingroup
\parskip0ex
\tableofcontents
\endgroup

%%%%%%%%%%%%%%%%%%%%%%%%%%%%%%%%%%%%%%%%%%%%%%%%%%%%%%%%%%%%%%%%%%%%%%%%%%%%%%%%
%%%%%%%%%%%%%%%%%%%%%%%%%%%%%%%%%%%%%%%%%%%%%%%%%%%%%%%%%%%%%%%%%%%%%%%%%%%%%%%%
\section{Introduction}

\LaTeX{} provides a mechanism to structure a large document (such as a book)
into a main file and several child files (containing the chapters)
using the |\include| command.
This mechanism is beneficial for documents
which span hundreds of pages in order to
make the source file(s) more manageable.
Moreover, compilation can be restricted to
selected child files by means of the |\includeonly| command.
The latter feature can be used to reduce the compilation time while editing
(this was significantly more useful in the earlier days of \LaTeX{})
or to generate a smaller document which is easier to navigate.
Another application of |\includeonly| is to generate
documents consisting of selected parts of the complete document.

However, there are a few drawbacks of the plain |\include| mechanism:
\begin{itemize}
\item
The child files cannot be compiled on their own,
they can only be compiled via the main file.
A naive editing environment
(such as a text editor with an option
to have the current file processed by \LaTeX)
may require one to switch to the main file before compiling;
attempting to compile the child file produces errors.
\item
The main file must be modified (each time)
to adjust the |\includeonly| command
to the present needs. This easily leaves the main file in a messy state.
\item
The generated document will always carry the filename
of the main document. This is inconvenient if
several child files are to be compiled and
to be kept for distribution.
\end{itemize}

The present package provides a simple interface
to make child files individually compilable by \LaTeX{}.
Compiling a child file then has the same effect as compiling
the main file with an |\includeonly| command
to select the appropriate child.
Moreover the generated document will carry the name of the child
rather than the main file.
This resolves all three above issues.

This feature is meant to make the editing of books,
thesis documents and lecture notes somewhat more convenient.
However, the package can also be used efficiently for
composing a series of documents (such as exercise sheets)
which are typically distributed individually.
It then assists the author in generating the individual documents
(potentially in different versions)
as well as a document containing the collected series.
Another application is in developing style files
or other kinds of included material
where compilation of the style file could redirect
to a sample or test file.

%%%%%%%%%%%%%%%%%%%%%%%%%%%%%%%%%%%%%%%%%%%%%%%%%%%%%%%%%%%%%%%%%%%%%%%%%%%%%%%%
%%%%%%%%%%%%%%%%%%%%%%%%%%%%%%%%%%%%%%%%%%%%%%%%%%%%%%%%%%%%%%%%%%%%%%%%%%%%%%%%
\section{Usage}

First of all, the package \textsf{childdoc} is \emph{not} a standard
\LaTeXe{} |.sty| style file! Therefore it needs to be invoked in
a non-standard way.

%%%%%%%%%%%%%%%%%%%%%%%%%%%%%%%%%%%%%%%%%%%%%%%%%%%%%%%%%%%%%%%%%%%%%%%%%%%%%%%%
\subsection{Included Files}
\label{sec:include}

%%%%%%%%%%%%%%%%%%%%%%%%%%%%%%%%%%%%%%%%
\DescribeMacro{\childdocmain}
To use the package, add the commands
\begin{center}
\begin{tabular}{l}
|\input{childdoc.def}|\\
|\childdocmain{}|\\
\end{tabular}
\end{center}
at the very top of the main \LaTeX{} file,
in particular \emph{before} the |\documentclass| statement!
The argument of |\childdocmain| should be left empty
(but it must be present).

%%%%%%%%%%%%%%%%%%%%%%%%%%%%%%%%%%%%%%%%
\DescribeMacro{\childdocof}
Furthermore, add the commands
\begin{center}
\begin{tabular}{l}
|\input{childdoc.def}|\\
|\childdocof{|\textit{main}|}|\\
\end{tabular}
\end{center}
at the top of every child file \textit{child}
which is included by |\include{|\textit{child}|}|
from within the main file
(or at least for those files to be compiled individually).
The argument \textit{main} must be the filename of the main file.

There are a couple of
considerations in setting up the main and child documents:

%%%%%%%%%%%%%%%%%%%%%%%%%%%%%%%%%%%%%%%%
\paragraph{Restrictions.}

Please note the following restrictions:
\begin{itemize}
\item
|\childdocmain| must be called with one argument \textit{main}
to ensure compatibility with earlier version of the package.
It must either be empty (|\childdocmain{}|)
or precisely match the filename of the main file in which it is specified.
See \secref{sec:detection} for further information.
\item
The filename \textit{main} must be specified without the |.tex| extension.
\item
The filename \textit{main} is case sensitive
(even in case-insensitive file systems)
due to internal string comparison.
\item
The argument \textit{main} should be fully expanded, it cannot be a macro.
\item
Subdirectories and special characters should be avoided in filenames.
\item
The command |\childdocmain{|\textit{main}|}| must be followed by a whitespace.
It should not be followed immediately by another command
or by a comment mark `|%|'.
This is because the \TeX{} parser reads the token immediately following
the argument of |\childdocmain| and puts it
at the beginning of every child section;
however, a white\-space is ignored.
\end{itemize}

%%%%%%%%%%%%%%%%%%%%%%%%%%%%%%%%%%%%%%%%
\paragraph{Content of Main File.}

It is advisable to place all content in the child files included by |\include|.
Any output contained in the main file will appear in all child documents
unless suppressed manually;
it cannot be suppressed automatically by the |\includeonly| directive
and thus should normally be avoided.
A method to include some content in the main file
by means of conditional processing is described in \secref{sec:conditional}.

%%%%%%%%%%%%%%%%%%%%%%%%%%%%%%%%%%%%%%%%
\paragraph{Page Numbering.}

When only a part of the document is compiled,
the appropriate numbering of pages
(as well as other status parameters)
is determined from the |.aux| files.
The latter contain information from previous passes.
However this information needs to propagate through
all intermediate child documents.
Therefore the page numbering in child documents may well
be inconsistent until the complete document is compiled at least once.

A useful (if unconventional) way to always ensure a consistent
page numbering is to restart the numbering in each child document
and denote the pages by `\textit{child}|.|\textit{page}'
where \textit{child} represents the chapter/section number of the child file.
This can be achieved by the command
|\numberwithin{page}{|\textit{child}|}|
of the \textsf{amsmath} package
where \textit{child} can be |chapter| or |section|
depending on the chosen structuring.
Alternatively, one can modify the macro |\thepage| appropriately
and reset the counter |page| at the start of each child file.

%%%%%%%%%%%%%%%%%%%%%%%%%%%%%%%%%%%%%%%%%%%%%%%%%%%%%%%%%%%%%%%%%%%%%%%%%%%%%%%%
\subsection{Conditional Processing}
\label{sec:conditional}

The package provides a mechanism to compile different versions
of a document. To customise the versions further some conditional processing
can come in handy to distinguish which version is being compiled.
The package provides two macros to describe the compilation context:

%%%%%%%%%%%%%%%%%%%%%%%%%%%%%%%%%%%%%%%%
\DescribeMacro{\ifchilddoc}
The conditional |\ifchilddoc| distinguishes between the compilation of
child documents and the main document:
%
\begin{center}
|\ifchilddoc |\textit{child-code}| |[|\||else |\textit{main-code}]| \||fi|
\end{center}

%%%%%%%%%%%%%%%%%%%%%%%%%%%%%%%%%%%%%%%%
\DescribeMacro{\childdocname}
\DescribeMacro{\childdocjob}
The macro |\childdocname| contains the filename (without extension)
of the main or child file being processed.
Note that |\childdocjob| will always contain the name of the main file.

%%%%%%%%%%%%%%%%%%%%%%%%%%%%%%%%%%%%%%%%
\paragraph{Title Page.}

Conditional processing can be used to include a title or banner page
in the main document when proper precautions are taken.
Importantly, the code in the main file should ensure that the page counter
(as well as other status parameters which are stored in the |.aux| files)
takes the same value after the conditional processing.
Otherwise the page numbers may take divergent values
depending on which part is compiled.

For example, a title page could be declared by:
%
\begin{center}
\begin{tabular}{l}
|\ifchilddoc\||else|\\
|\addtocounter{page}{-1}|\\
\textit{code for title page}\\
|\newpage|\\
|\||fi|
\end{tabular}
\end{center}
%
A banner page for the child documents can be generated by:
%
\begin{center}
\begin{tabular}{l}
|\ifchilddoc|\\
|\addtocounter{page}{-1}|\\
\textit{code for banner page}\\
|\newpage|\\
|\||fi|
\end{tabular}
\end{center}
%
Here one could write a message such as:
\begin{center}
|This is the part \childdocname{} of \childdocjob{}.|
\end{center}

%%%%%%%%%%%%%%%%%%%%%%%%%%%%%%%%%%%%%%%%%%%%%%%%%%%%%%%%%%%%%%%%%%%%%%%%%%%%%%%%
\subsection{Flags}
\label{sec:flags}

The package makes it easy to generate different versions
of the main or child documents.
To this end compilation flags can be defined
and assigned different default values.
They will be particularly useful in conjunction
with the forwarding mechanism described in \secref{sec:forward}.

For example, it may be useful to have a flag |\version|
which can be set to |draft| or |final|.
The document source will contain some conditional code
depending on the value of |\version|.
Suppose further, the flag should default to |final| for the main file
and to |draft| for child files
which is a natural assignment for editing the document.
This is achieved by placing the following code
in the preamble of the main document
(below the |\childdocmain| directive):
%
\begin{center}
\begin{tabular}{l}
|\ifchilddoc|\\
|\providecommand{\version}{draft}|\\
|\||else|\\
|\providecommand{\version}{final}|\\
|\||fi|
\end{tabular}
\end{center}
%
The definition by |\providecommand| makes sure
that previous definitions are not overwritten.
Further statements |\providecommand{\version}{...}|
can thus be added before the above code to override it.

For the main file, one might add a line
(between |\childdocmain| and the above block)
%
\begin{center}
|%\ifchilddoc\||else\providecommand{\version}{draft}\||fi|
\end{center}
%
which can be uncommented to produce a draft version.
Likewise one can add a line to the very top of a child file
(above the |\childdocof{|\textit{main}|}| directive)
%
\begin{center}
|%\providecommand{\version}{final}|
\end{center}
%
which can be uncommented to produce the final version of this child document.

%%%%%%%%%%%%%%%%%%%%%%%%%%%%%%%%%%%%%%%%%%%%%%%%%%%%%%%%%%%%%%%%%%%%%%%%%%%%%%%%
\subsection{Forwarding}
\label{sec:forward}

Different versions of the main or child documents
using compilation flags as described in \secref{sec:flags}
can be (permanently) stored in different files
for convenient compilation, viewing and distribution.
To this end, the package defines a command
to pass on compilation to a different file:

%%%%%%%%%%%%%%%%%%%%%%%%%%%%%%%%%%%%%%%%
\DescribeMacro{\childdocforward}
The command |\childdocforward| redirects processing to
another source file:
%
\begin{center}
\begin{tabular}{l}
|\input{childdoc.def}|\\
|\childdocforward[|\textit{main}|]{|\textit{dest}|}|\\
\end{tabular}
\end{center}
%
The argument \textit{dest} is the destination file
(without extension).
It should be the main file or one of the child files.
Note that further \textsf{childdoc} directives
such as |\childdocof| and |\childdocforward|
in the indicated file will be processed in this form.
The optional argument \textit{main}
passes on directly to the main file \textit{main}
while pretending to compile the child \textit{dest}.
This form behaves as if \textit{dest}
issues |\childdocof{|\textit{main}|}| right away,
and no further \textsf{childdoc} directives will be processed.

%%%%%%%%%%%%%%%%%%%%%%%%%%%%%%%%%%%%%%%%
\DescribeMacro{\...prefix}
In the alternative form |\childdocforwardprefix|,
%
\begin{center}
\begin{tabular}{l}
|\input{childdoc.def}|\\
|\childdocforwardprefix[|\textit{main}|]{|\textit{prefix}|}{|\textit{dest}|}|
\end{tabular}
\end{center}
%
the destination file is determined by a pattern
depending on the current file:
To make this work, the current file must be called
`{\textit{prefix}\hspace{0.2em}\textit{suffix}}'
with \textit{prefix} matching precisely the argument.
Processing is then passed on to the file
`{\textit{dest}\hspace{0.2em}\textit{suffix}}'.
Surely, the same effect is achieved by
directly specifying the
argument `{\textit{dest}\hspace{0.2em}\textit{suffix}}'
in the first form.
However, that requires to set up a different file
for each child. With the alternative form of the command
all these files can have exactly the same content
which simplifies setting them up and maintaining them.

For example, the following file |draft.tex|
with a compilation flag |\version| as described in \secref{sec:flags}
compiles the main document as a draft:
%
\begin{center}
\begin{tabular}{l}
|\def\version{draft}|\\
|\input{childdoc.def}|\\
|\childdocforward{|\textit{main}|}|
\end{tabular}
\end{center}
%
Likewise, the following files |final|\textit{nn}|.tex|
compile the final version of the child document
|child|\textit{nn}|.tex|:
%
\begin{center}
\begin{tabular}{l}
|\def\version{final}|\\
|\input{childdoc.def}|\\
|\childdocforwardprefix{final}{child}|
\end{tabular}
\end{center}
%

Note that when several versions of a main file and/or of each child file
are to be generated, it may be convenient to set up a |Makefile| or
shell script to automatise the process.

%%%%%%%%%%%%%%%%%%%%%%%%%%%%%%%%%%%%%%%%%%%%%%%%%%%%%%%%%%%%%%%%%%%%%%%%%%%%%%%%
\subsection{Command Line Processing}
\label{sec:commandline}

The effect of redirection files can also be achieved by invoking
the \LaTeX{} compiler with a more elaborate command line.
Most conveniently this should be done as part
of a shell script or a |Makefile|.

When using \textsf{childdoc} in the main file, the following
command lines effectively perform a redirection
(note that depending on the shell being used,
backslashes may have to be doubled: `|\|' $\to$ `|\\|'):
%
\begin{center}
|... -jobname "|\textit{target}|" |\\|"|[\textit{flags}]%
|\input{childdoc.def}\childdocforward[|\textit{main}|]{|\textit{dest}|}"|
\end{center}
%
Here \textit{target} is the name of the output file,
\textit{main} is the name of the main file
and \textit{dest} is the name of the main or child file to be processed
(all filenames without extensions).
The optional argument \textit{main} can be omitted
if \textit{main} matches \textit{dest}.
Optionally, compilation \textit{flags} can be defined via |\def| commands.
This command line makes the \TeX{} engine believe
it is compiling the file \textit{target}
whose content is specified as the latter parameter.
The provided code then forwards the processing to
\textit{main} or \textit{dest} as described in \secref{sec:forward}.

%%%%%%%%%%%%%%%%%%%%%%%%%%%%%%%%%%%%%%%%%%%%%%%%%%%%%%%%%%%%%%%%%%%%%%%%%%%%%%%%
\subsection{Include by Input}
\label{sec:input}

Including child documents by |\include| has some restrictions by design.
Most notably, the content of a child document always occupies
its own set of pages; pages cannot be shared between child documents.
Usually, this behaviour makes perfect sense
because each child document contain an essential part of the document.
However, in some situations it may be desirable to compose
a document from a collection of parts
without having mandatory page breaks between then.
For this case, the package
provides a mechanism to include parts
by |\input| which can also be processed individually.
However, by construction this mechanism
requires manual handling of the content to be output.

%%%%%%%%%%%%%%%%%%%%%%%%%%%%%%%%%%%%%%%%
\DescribeMacro{\ifchilddocmanual}
The main file should be prepared as usual, see \secref{sec:include}.
However, the document body must make a distinction
between processing of an individual part and of the main document, e.g.:
%
\begin{center}
\begin{tabular}{l}
|\ifchilddocmanual|\\
|\input{\childdocname}|\\
|\||else|\\
\textit{document body with }|\input{|\textit{part}|}|\\
|\||fi|
\end{tabular}
\end{center}
%
The conditional |\ifchilddocmanual| is true whenever
a part to be included by |\input| is being compiled,
and the name of the part is stored in |\childdocname|.

%%%%%%%%%%%%%%%%%%%%%%%%%%%%%%%%%%%%%%%%
\DescribeMacro{\childdocby}
Each part to be included by |\input| should start with:
%
\begin{center}
\begin{tabular}{l}
|\input{childdoc.def}|\\
|\childdocby{|\textit{main}|}|\\
\end{tabular}
\end{center}
%
The directive |\childdocby| is similar to |\childdocof|
described in \secref{sec:include},
but the subsequent selection of content must be done manually.
To that end, both |\ifchilddoc| and |\ifchilddocmanual|
will be true upon processing of a part,
and the name of the part is stored in |\childdocname|.
Note that |\jobname| will be set to the filename of the current part
so that each part receives an individual |.aux| file
that does not interfere with the |.aux| file(s) of the main document.
This behaviour can be altered by the alternative form
|\childdocby[*]{|\textit{main}|}| (with a non-empty optional argument)
which uses the |.aux| file of the main document
by setting |\jobname| to \textit{main}.

%%%%%%%%%%%%%%%%%%%%%%%%%%%%%%%%%%%%%%%%%%%%%%%%%%%%%%%%%%%%%%%%%%%%%%%%%%%%%%%%
\subsection{Driver Development}
\label{sec:driver}

The \textsf{childdoc} mechanism can also be use for the development
of definition files such as \LaTeX{} styles or classes.
This case differs from the above setup with multiple parts
included by |\include| in that no |\includeonly| should be invoked.
This can be achieved by starting the include file
(before |\ProvidesPackage|) with:
%
\begin{center}
\begin{tabular}{l}
|\input{childdoc.def}|\\
|\childdocforward{|\textit{main}|}|\\
\end{tabular}
\end{center}
%
or alternatively with:
%
\begin{center}
\begin{tabular}{l}
|\input{childdoc.def}|\\
|\childdocby{|\textit{main}|}|\\
\end{tabular}
\end{center}
%
Both forms have slightly different effects as described above.
The main file is prepared as usual, see \secref{sec:include}.

%%%%%%%%%%%%%%%%%%%%%%%%%%%%%%%%%%%%%%%%%%%%%%%%%%%%%%%%%%%%%%%%%%%%%%%%%%%%%%%%
\subsection{Legacy Detection}
\label{sec:detection}

The directive |\childdocmain| in the main file can detect
whether the complete document or merely a child is to be compiled
even without using the directive |\childdocof|.
This method is deprecated because it is less robust
and there is no compelling reason to use it;
it is merely provided for backward compatibility
and it may be removed in future versions.

If the detection mechanism is to be used,
it is mandatory to correctly specify
the filename of the main file as the argument of |\childdocmain|:
%
\begin{center}
\begin{tabular}{l}
|\input{childdoc.def}|\\
|\childdocmain{|\textit{main}|}|\\
\end{tabular}
\end{center}
%
If |\jobname| does not match the argument \textit{main} of |\childdocmain|,
it is assumed that |\jobname| points to the child file to be compiled.
When using |\childdocmain| with the main file specified as argument,
it suffices to start a child file
with just |\input{|\textit{main}|}|
without loading of the package and using |\childdocof|.
If instead all processing is done
with the appropriate \textsf{childdoc} directives,
the argument of \textit{main} of |\childdocmain| can be empty.

An alternative version of the command line processing described
in \secref{sec:commandline} using the detection mechanism reads:
%
\begin{center}
|... -jobname "|\textit{target}|" "|[\textit{flags}]%
[|\def\jobname{|\textit{dest}|}|]|\input{|\textit{main}|}"|
\end{center}

%%%%%%%%%%%%%%%%%%%%%%%%%%%%%%%%%%%%%%%%%%%%%%%%%%%%%%%%%%%%%%%%%%%%%%%%%%%%%%%%
\subsection{Manual Code}
\label{sec:manual}

In case one cannot be certain whether the definitions file |childdoc.def|
is installed on the target \TeX{} distribution
and one prefers not to ship it,
it is conceivable to paste a few relevant commands into the sources.

To that end, drop all statements |\input{childdoc.def}|
and perform the replacements as outlined below.
Instead of |\childdocmain{|\textit{main}|}| add the following code
to the top of the main file:
%
\begin{center}
\begin{tabular}{l}
|\||ifdefined\childdocname\endinput\||fi\newif\ifchilddoc|\\
|\edef\childdocname{\scantokens\expandafter{\jobname\noexpand}}|\\
|\def\childdocmain{|\textit{main}|}\||ifx\childdocmain\childdocname\||else|\\
|\childdoctrue\includeonly{\childdocname}\let\jobname\childdocmain\||fi|\\
\end{tabular}
\end{center}
%
Instead of |\childdocof{|\textit{main}|}| just include the main file
at the top of each child file:
%
\begin{center}
|\input{|\textit{main}|}|
\end{center}
%
A simple redirection |\childdocforward{|\textit{dest}|}| is achieved by:
%
\begin{center}
|\def\jobname{|\textit{dest}|}\input{\jobname}|
\end{center}
%
The redirection with prefix
|\childdocforwardprefix[|\textit{prefix}|]{|\textit{dest}|}|
is accomplished by:
%
\begin{center}
\begin{tabular}{l}
|{\edef\jobname{\scantokens\expandafter{\jobname\noexpand}}|\\
|\def\redirectjob |\textit{prefix}|#1~~~{\gdef\jobname{|\textit{dest}|#1}}|\\
|\expandafter\redirectjob\jobname~~~}\input{\jobname}|
\end{tabular}
\end{center}

In an alternative approach,
child documents can be compiled by a specific command line
without additional code or specific definitions:
%
\begin{center}
|... -jobname "|\textit{target}|" "|[\textit{flags}]%
|\includeonly{|\textit{dest}|}\input{|\textit{main}|}"|
\end{center}
%

%%%%%%%%%%%%%%%%%%%%%%%%%%%%%%%%%%%%%%%%%%%%%%%%%%%%%%%%%%%%%%%%%%%%%%%%%%%%%%%%
%%%%%%%%%%%%%%%%%%%%%%%%%%%%%%%%%%%%%%%%%%%%%%%%%%%%%%%%%%%%%%%%%%%%%%%%%%%%%%%%
\section{Information}

%%%%%%%%%%%%%%%%%%%%%%%%%%%%%%%%%%%%%%%%%%%%%%%%%%%%%%%%%%%%%%%%%%%%%%%%%%%%%%%%
\subsection{Copyright}

Copyright \copyright{} 2017--2018 Niklas Beisert

This work may be distributed and/or modified under the
conditions of the \LaTeX{} Project Public License, either version 1.3
of this license or (at your option) any later version.
The latest version of this license is in
  \url{http://www.latex-project.org/lppl.txt}
and version 1.3 or later is part of all distributions of \LaTeX{}
version 2005/12/01 or later.

This work has the LPPL maintenance status `maintained'.

The Current Maintainer of this work is Niklas Beisert.

This work consists of the files |README.txt|, |childdoc.ins| and |childdoc.dtx|
as well as the derived files |childdoc.def|, |cdocsamp.tex|
with |cdocsch1.tex|, |cdocsch2.tex|, |cdocspt3.tex|, |cdocspt4.tex|,
|cdocsdrf.tex|, |cdocsfn1.tex|, |cdocsfn2.tex|
as well as |childdoc.pdf|.

%%%%%%%%%%%%%%%%%%%%%%%%%%%%%%%%%%%%%%%%%%%%%%%%%%%%%%%%%%%%%%%%%%%%%%%%%%%%%%%%
\subsection{Files and Installation}

The package consists of the files:
%
\begin{center}
\begin{tabular}{ll}
    |README.txt|   & readme file \\
    |childdoc.ins| & installation file \\
    |childdoc.dtx| & source file \\
    |childdoc.def| & definition file \\
    |cdocsamp.tex| & sample main file \\
    |cdocsch1.tex| & sample include file \\
    |cdocsch2.tex| & sample include file \\
    |cdocspt3.tex| & sample part file \\
    |cdocspt4.tex| & sample part file \\
    |cdocsdrf.tex| & sample redirection file \\
    |cdocsfn1.tex| & sample redirection file \\
    |cdocsfn2.tex| & sample redirection file \\
    |childdoc.pdf| & manual
\end{tabular}
\end{center}
%
The distribution consists of the files
|README.txt|, |childdoc.ins| and |childdoc.dtx|.
%
\begin{itemize}
\item
Run (pdf)\LaTeX{} on |childdoc.dtx|
to compile the manual |childdoc.pdf| (this file).
\item
Run \LaTeX{} on |childdoc.ins| to create the definitions file |childdoc.def|
and the sample |cdocsamp.tex| with include files
|cdocsch1.tex|, |cdocsch2.tex|, |cdocspt3.tex|, |cdocspt4.tex|,
|cdocsdrf.tex|, |cdocsfn1.tex|, |cdocsfn2.tex|.
Then copy the file |childdoc.def| to an appropriate directory of your \LaTeX{}
distribution, e.g.\ \textit{texmf-root}|/tex/latex/childdoc|.
\end{itemize}

%%%%%%%%%%%%%%%%%%%%%%%%%%%%%%%%%%%%%%%%%%%%%%%%%%%%%%%%%%%%%%%%%%%%%%%%%%%%%%%%
\subsection{Related CTAN Packages}

There are several other packages which offer a similar functionality:
%
\begin{itemize}
\item
The packages
\href{http://ctan.org/pkg/docmute}{\textsf{docmute}},
\href{http://ctan.org/pkg/includex}{\textsf{includex}} and
\href{http://ctan.org/pkg/standalone}{\textsf{standalone}}
provide commands to include only the document body of
a child file thus allowing both files to be compiled individually.
\item
The packages \href{http://ctan.org/pkg/subdocs}{\textsf{subdocs}}
and \href{http://ctan.org/pkg/subfiles}{\textsf{subfiles}}
provide structures in which the main and child documents can be
encapsulated and allowing them to be compiled individually.
The inclusion mechanism is different from the conventional |\include|.
\item
The package \href{http://ctan.org/pkg/combine}{\textsf{combine}}
is an elaborate solution to combine several documents into one.
\end{itemize}
%
See also the CTAN topic \href{http://ctan.org/topic/subdocs}{\textsf{subdocs}}
for further related packages.
The present package differs from the above solutions in that
a document structure constructed with the conventional |\include| mechanism
just needs two extra commands at the top of every file
such that all constituent files can be compiled individually.

%%%%%%%%%%%%%%%%%%%%%%%%%%%%%%%%%%%%%%%%%%%%%%%%%%%%%%%%%%%%%%%%%%%%%%%%%%%%%%%%
%\subsection{Feature Suggestions}
%
%The following is a list of features which may be useful for future
%versions of this package:
%%
%\begin{itemize}
%\item
%\ldots
%\end{itemize}

%%%%%%%%%%%%%%%%%%%%%%%%%%%%%%%%%%%%%%%%%%%%%%%%%%%%%%%%%%%%%%%%%%%%%%%%%%%%%%%%
\subsection{Revision History}

%%%%%%%%%%%%%%%%%%%%%%%%%%%%%%%%%%%%%%%%
\paragraph{v2.0:} 2018/12/30

\begin{itemize}
\item
immediate forward processing
\item
added |\childdocby| mechanism
\item
manual restructured
\end{itemize}

%%%%%%%%%%%%%%%%%%%%%%%%%%%%%%%%%%%%%%%%
\paragraph{v1.6:} 2018/01/17

\begin{itemize}
\item
application for development of include files
\item
corrections to manual
\end{itemize}

%%%%%%%%%%%%%%%%%%%%%%%%%%%%%%%%%%%%%%%%
\paragraph{v1.5:} 2017/05/21

\begin{itemize}
\item
more complete structuring introduced
\item
|\childdocof| introduced
\item
|\childdoc| renamed to |\childdocmain|
\item
|\childredirect| renamed to |\childdocforward| and |\childdocforwardprefix|
and functionality expanded
\end{itemize}

%%%%%%%%%%%%%%%%%%%%%%%%%%%%%%%%%%%%%%%%
\paragraph{v1.0:} 2017/04/27

\begin{itemize}
\item
manual and install package
\item
first version published on CTAN
\end{itemize}

%%%%%%%%%%%%%%%%%%%%%%%%%%%%%%%%%%%%%%%%
\paragraph{v0.6:} 2017/04/26

\begin{itemize}
\item
redirection mechanism added
\end{itemize}

%%%%%%%%%%%%%%%%%%%%%%%%%%%%%%%%%%%%%%%%
\paragraph{v0.5:} 2017/04/26

\begin{itemize}
\item
functionality in definition file
\end{itemize}


%%%%%%%%%%%%%%%%%%%%%%%%%%%%%%%%%%%%%%%%%%%%%%%%%%%%%%%%%%%%%%%%%%%%%%%%%%%%%%%%
%%%%%%%%%%%%%%%%%%%%%%%%%%%%%%%%%%%%%%%%%%%%%%%%%%%%%%%%%%%%%%%%%%%%%%%%%%%%%%%%
%%%%%%%%%%%%%%%%%%%%%%%%%%%%%%%%%%%%%%%%%%%%%%%%%%%%%%%%%%%%%%%%%%%%%%%%%%%%%%%%
\appendix

\settowidth\MacroIndent{\rmfamily\scriptsize 000\ }

 \DocInput{childdoc.dtx}

\end{document}
%</driver>
% \fi
%
% %%%%%%%%%%%%%%%%%%%%%%%%%%%%%%%%%%%%%%%%%%%%%%%%%%%%%%%%%%%%%%%%%%%%%%%%%%%%%%
% %%%%%%%%%%%%%%%%%%%%%%%%%%%%%%%%%%%%%%%%%%%%%%%%%%%%%%%%%%%%%%%%%%%%%%%%%%%%%%
% \section{Sample}
%\iffalse
%<*samplemain>
%\fi
%
% The following presents a sample document
% with two chapters, two parts, a title page,
% a compile flag as well as three forwarding files to set the flag.
% It consists of eight |.tex| files:
% \begin{center}
% \begin{tabular}{ll}
% |cdocsamp.tex|&main file\\
% |cdocsch1.tex|&include file for chapter 1\\
% |cdocsch2.tex|&include file for chapter 2\\
% |cdocspt3.tex|&include file for part 3\\
% |cdocspt4.tex|&include file for part 4\\
% |cdocsdrf.tex|&forwarding file for main file in draft mode\\
% |cdocsfi1.tex|&forwarding file for final version of chapter 1\\
% |cdocsfi2.tex|&forwarding file for final version of chapter 2\\
% \end{tabular}
% \end{center}
% Each of the eight files can be compiled directly by the \LaTeX{} compiler.
%
% %%%%%%%%%%%%%%%%%%%%%%%%%%%%%%%%%%%%%%
% \paragraph{Main File.}
%
% The main file is called |cdocsamp.tex|.
%
% Load the \textsf{childdoc} definitions and
% declare the filename for the main document:
%    \begin{macrocode}
\input{childdoc.def}
\childdocmain{}
%    \end{macrocode}

% Optional override for |\version| flag:
%    \begin{macrocode}
%%\ifchilddoc\else\providecommand{\version}{draft}\fi
%    \end{macrocode}

% Define the default values for the |\version| flag
% (|final| for the main file and |draft| for childs):
%    \begin{macrocode}
\ifchilddoc
\providecommand{\version}{draft}
\else
\providecommand{\version}{final}
\fi
%    \end{macrocode}

% Load the standard document class:
%    \begin{macrocode}
\documentclass[12pt]{article}
%    \end{macrocode}

% Start the document body:
%    \begin{macrocode}
\begin{document}
%    \end{macrocode}

% Declare a title page.
% Print title, part of document being processed and version flag:
%    \begin{macrocode}
\addtocounter{page}{-1}
\begin{center}
{\LARGE\bfseries{}childdoc example\par}
\vspace{1cm}
\ifchilddoc
\ifchilddocmanual part\else chapter\fi:
`\childdocname' of `\childdocjob'\par
\else
main document: `\childdocjob'\par
\fi
version: \version\par
\end{center}
\newpage
%    \end{macrocode}

% Manually include selected file,
% otherwise process as usual:
%    \begin{macrocode}
\ifchilddocmanual
\section*{part `\childdocname'}
\input{\childdocname}
\else
%    \end{macrocode}

% Include the two chapters:
%    \begin{macrocode}
\include{cdocsch1}
\include{cdocsch2}
%    \end{macrocode}

% Include the two parts unless only chapters should be displayed:
%    \begin{macrocode}
\ifchilddoc\else
\section{part three}
\input{cdocspt3}
\section{part four}
\input{cdocspt4}
\fi
%    \end{macrocode}

% Process as usual until here:
%    \begin{macrocode}
\fi
%    \end{macrocode}

% End of document body:
%    \begin{macrocode}
\end{document}
%    \end{macrocode}
%\iffalse
%</samplemain>
%\fi
%
% %%%%%%%%%%%%%%%%%%%%%%%%%%%%%%%%%%%%%%
% \paragraph{Chapter Include Files.}
%
% The include files are called |cdocsch1.tex| and |cdocsch2.tex|.
%
%\iffalse
%<*samplechap1|samplechap2>
%\fi

% Optional override for |\version| flag:
%    \begin{macrocode}
%%\providecommand{\version}{final}
%    \end{macrocode}

% Include the main document:
%    \begin{macrocode}
\input{childdoc.def}
\childdocof{cdocsamp}
%    \end{macrocode}

%\iffalse
%</samplechap1|samplechap2>
%\fi
%
%\iffalse
%<*samplechap1>
%\fi
% Some text for chapter 1:
%    \begin{macrocode}
\section{one}
some text in chapter one
%    \end{macrocode}

%\iffalse
%</samplechap1>
%\fi
% Some text for chapter 2:
%\iffalse
%<*samplechap2>
%\fi
%    \begin{macrocode}
\section{two}
more text in chapter two
%    \end{macrocode}

%\iffalse
%</samplechap2>
%\fi
%
% %%%%%%%%%%%%%%%%%%%%%%%%%%%%%%%%%%%%%%
% \paragraph{Part Include Files.}
%
% The include files are called |cdocspt3.tex| and |cdocspt4.tex|.
%
%\iffalse
%<*samplepart3|samplepart4>
%\fi

% Optional override for |\version| flag:
%    \begin{macrocode}
%%\providecommand{\version}{final}
%    \end{macrocode}

% Include the main document:
%    \begin{macrocode}
\input{childdoc.def}
\childdocby{cdocsamp}
%    \end{macrocode}

%\iffalse
%</samplepart3|samplepart4>
%\fi
%
%\iffalse
%<*samplepart3>
%\fi
% Some text for part 3:
%    \begin{macrocode}
some text in part three
%    \end{macrocode}

%\iffalse
%</samplepart3>
%\fi
% Some text for part 4:
%\iffalse
%<*samplepart4>
%\fi
%    \begin{macrocode}
more text in part four
%    \end{macrocode}

%\iffalse
%</samplepart4>
%\fi
%
% %%%%%%%%%%%%%%%%%%%%%%%%%%%%%%%%%%%%%%
% \paragraph{Forwarding for a Complete Draft.}
%
% The following forwarding file |cdocsdrf.tex|
% compiles the main document in draft mode:
%\iffalse
%<*sampledraft>
%\fi
%    \begin{macrocode}
\def\version{draft}
\input{childdoc.def}
\childdocforward{cdocsamp}
%    \end{macrocode}

%\iffalse
%</sampledraft>
%\fi
%
% %%%%%%%%%%%%%%%%%%%%%%%%%%%%%%%%%%%%%%
% \paragraph{Forwarding for Final Version of the Chapters.}
%
% The following forwarding files |cdocsfn1.tex| and |cdocsfn2.tex|
% (with identical content)
% compile the final versions of the child documents
% |cdocsch1.tex| and |cdocsch2.tex|, respectively:
%\iffalse
%<*samplefinal>
%\fi
%    \begin{macrocode}
\def\version{final}
\input{childdoc.def}
\childdocforwardprefix[cdocsamp]{cdocsfn}{cdocsch}
%    \end{macrocode}

%\iffalse
%</samplefinal>
%\fi
%
% %%%%%%%%%%%%%%%%%%%%%%%%%%%%%%%%%%%%%%
% \paragraph{Command Line Processing.}
%
% The following three command lines generate the output files
% |cdocscld|, |cdocscl1| and |cdocscl2|
% which should be identical to
% |cdocsdrf|, |cdocsch1| and |cdocsfn2|, respectively:
% \begin{center}
% \begin{tabular}{l}
% |latex -jobname cdocscld \|\\
% |  "\def\version{draft}\input{childdoc.def}\childdocforward{cdocsamp}"|\\
% |latex -jobname cdocscl1 \|\\
% |  "\input{childdoc.def}\childdocforward[cdocsamp]{cdocsch1}"|\\
% |latex -jobname cdocscl2 \|\\
% |  "\def\version{final}\input{childdoc.def}\childdocforward{cdocsch2}"|
% \end{tabular}
% \end{center}
% Note that the trailing backslash on each first line
% merely continues the input to the second line
% (for convenient cut ant paste).
% Furthermore, the command |latex| can be replaced by any
% of its alternative versions such as |pdflatex|.
%
% %%%%%%%%%%%%%%%%%%%%%%%%%%%%%%%%%%%%%%%%%%%%%%%%%%%%%%%%%%%%%%%%%%%%%%%%%%%%%%
% %%%%%%%%%%%%%%%%%%%%%%%%%%%%%%%%%%%%%%%%%%%%%%%%%%%%%%%%%%%%%%%%%%%%%%%%%%%%%%
% \section{Implementation}
%\iffalse
%<*package>
%\fi
%
% This section describes the definitions file |childdoc.def|.

% The definitions cannot be loaded using |\usepackage| or |\RequirePackage|
% which has a mechanism to prevent loading a style file more than once.
% When loading the definitions by means of |\input|
% multiple instances have to be prevented manually:
%\iffalse
%This code needs to be before the `\ProvidesFile' directive
%which is defined at the beginning of this file.
%Therefore it is also placed there and commented out here.
%</package>
%<*discard>
%\fi
%    \begin{macrocode}
\ifdefined\childdocmain\endinput\fi
%    \end{macrocode}
%\iffalse
%</discard>
%<*package>
%\fi
%
% \macro{\ifchilddoc}
% \macro{\ifchilddocmanual}
% The conditional |\ifchilddoc| tells whether a
% child (true) or main (false) document is being compiled.
% The conditional |\ifchilddocmanual| tells whether
% the |\includeonly| mechanism is used (false) or
% the selection of child files must be performed manually (true).
% The definitions initialise to false:
%    \begin{macrocode}
\newif\ifchilddoc
\newif\ifchilddocmanual
%    \end{macrocode}

% \macro{\childdocname}
% \macro{\childdocjob}
% The macro |\childdocname| stores the name of the main document
% to be compiled. The macro |\childdocjob| stores the name of
% the document on which the \LaTeX{} compiler was originally invoked.
% The content of |\jobname| cannot be compared
% to filenames specified in the source due to different catcodes.
% The following code rescans |\jobname|, stores the result
% in |\childdocname| and saves a copy in |\childdocjob|:
%    \begin{macrocode}
\edef\childdocname{\scantokens\expandafter{\jobname\noexpand}}
\let\childdocjob\childdocname
%    \end{macrocode}

% \macro{\childdocdisable}
% The macro |\childdocdisable| prevents the main file
% from being processed more than once.
% At this stage, the main document command |\childdocmain|
% is assumed to be called once again where it should do nothing.
% Any subsequent call to it should prevent
% a secondary processing of the main document
% It overwrites the forwarding commands
% |\childdocof| and |\childdocforward|
% with empty macros to prevent further inclusions of the main document:
%    \begin{macrocode}
\newcommand{\childdocdisable}
{
  \renewcommand{\childdocmain}[1]{\renewcommand{\childdocmain}[1]{\endinput}}
  \renewcommand{\childdocof}[1]{}
  \renewcommand{\childdocby}[2][]{}
  \renewcommand{\childdocforward}[2][]{}
  \renewcommand{\childdocdisable}{}
}
%    \end{macrocode}

% \macro{\childdocmain}
% The macro |\childdocmain| is to be called at the top of the main file
% with nothing or the main filename (without extension) as argument.
% First, it breaks loops.
% If the argument is not empty and does not match |\childdocname|
% (which is set by the first inclusion of |childdoc.def|),
% |\ifchilddoc| is set to true, |\includeonly| is applied to the child file
% and |\jobname| is set to the main file
% (for proper handling of |.aux| files):
%    \begin{macrocode}
\newcommand{\childdocmain}[1]
{
  \childdocdisable\childdocmain{}
  \if?#1?\else
    \begingroup
      \def\childdoctmp{#1}
      \ifx\childdoctmp\childdocname
        \def\childdoctmp{}
      \else
        \def\childdoctmp
        {
          \childdoctrue
          \includeonly{\childdocname}
          \def\childdocjob{#1}
          \def\jobname{#1}
        }
      \fi
      \expandafter
    \endgroup
    \childdoctmp
  \fi
}
%    \end{macrocode}

% \macro{\childdocof}
% The command |\childdocof| redirects
% compilation to the main file |#1|.
%    \begin{macrocode}
\newcommand{\childdocof}[1]
{
  \childdocdisable
  \childdoctrue
  \includeonly{\childdocname}
  \def\jobname{#1}
  \def\childdocjob{#1}
  \input{#1}
}
%    \end{macrocode}

% \macro{\childdocby}
% The command |\childdocby| ....
%    \begin{macrocode}
\newcommand{\childdocby}[2][]
{
  \childdocdisable
  \childdoctrue
  \childdocmanualtrue
  \if?#1?\else
    \def\jobname{#2}
  \fi
  \def\childdocjob{#2}
  \input{#2}
  \endinput
}
%    \end{macrocode}

% \macro{\childdocforward}
% The command |\childdocforward| redirects
% compilation to the main file or
% (if the optional argument is given) a child file.
% Parameters are set as if the main file
% or a child file starting with |\childdocof| was compiled.
% Then compilation is handed over to the main file:
%    \begin{macrocode}
\newcommand{\childdocforward}[2][]
{
  \begingroup
    \if?#1?
      \def\childdoctmp
      {
        \def\childdocname{#2}
        \def\childdocjob{#2}
        \def\jobname{#2}
        \input{#2}
        \endinput
      }
    \else
      \def\childdoctmp
      {
        \childdocdisable
        \def\childdocname{#2}
        \childdoctrue
        \includeonly{#2}
        \def\childdocjob{#1}
        \def\jobname{#1}
        \input{#1}
        \endinput
      }
    \fi
    \expandafter
  \endgroup
  \childdoctmp
}
%    \end{macrocode}

% \macro{\childdocforwardprefix}
% The command |\childdocforwardprefix| redirects
% compilation to the main or a child file by means of a pattern.
% The prefix |#1| in the current filename is replaced by |#2|
% and the suffix of the current filename is kept
% (it is assumed that the filename does not contain the substring `|~~~|'
% which is used as a delimiter).
% Compilation is handed over to the new file by |\childdocforward|:
%    \begin{macrocode}
\newcommand{\childdocforwardprefix}[3][]
{
  \begingroup
    \def\childdocextract #2##1~~~{\def\childdoctmp{\childdocforward[#1]{#3##1}}}
    \expandafter\childdocextract\childdocname~~~
    \expandafter
  \endgroup
  \childdoctmp
}
%    \end{macrocode}

% \macro{\childdoc}
% The deprecated macro |\childdoc| is a legacy version of |\childdocmain|:
%    \begin{macrocode}
\newcommand{\childdoc}{\childdocmain}
%    \end{macrocode}

% \macro{\childdocredirect}
% The deprecated macro |\childdocredirect| is a legacy version
% of |\childdocforward| and |\childdocforwardprefix|:
%    \begin{macrocode}
\newcommand{\childdocredirect}[2][]
{
  \begingroup
    \if?#1?
      \def\childdoctmp{\childdocforward{#2}}
    \else
      \def\childdoctmp{\childdocforwardprefix{#1}{#2}}
    \fi
    \expandafter
  \endgroup
  \childdoctmp
}
%    \end{macrocode}

%\iffalse
%</package>
%\fi
%
\endinput
|\\
|\childdocmain{}|\\
\end{tabular}
\end{center}
at the very top of the main \LaTeX{} file,
in particular \emph{before} the |\documentclass| statement!
The argument of |\childdocmain| should be left empty
(but it must be present).

%%%%%%%%%%%%%%%%%%%%%%%%%%%%%%%%%%%%%%%%
\DescribeMacro{\childdocof}
Furthermore, add the commands
\begin{center}
\begin{tabular}{l}
|% \iffalse
%
% childdoc.dtx Copyright (C) 2017-2018 Niklas Beisert
%
% This work may be distributed and/or modified under the
% conditions of the LaTeX Project Public License, either version 1.3
% of this license or (at your option) any later version.
% The latest version of this license is in
%   http://www.latex-project.org/lppl.txt
% and version 1.3 or later is part of all distributions of LaTeX
% version 2005/12/01 or later.
%
% This work has the LPPL maintenance status `maintained'.
%
% The Current Maintainer of this work is Niklas Beisert.
%
% This work consists of the files childdoc.dtx and childdoc.ins
% and the derived files childdoc.def and cdocsamp.tex with
% cdocsch1.tex, cdocsch2.tex, cdocsdrf.tex, cdocsfn1.tex, cdocsfn2.tex.
%
%<package>\ifdefined\childdocmain\endinput\fi
%<package>\ProvidesFile{childdoc.def}[2018/12/30 v2.0 child document driver]
%<samplemain>\ProvidesFile{cdocsamp.tex}[2018/12/30 v2.0 sample for childdoc]
%<*driver>
%\ProvidesFile{childdoc.drv}[2018/12/30 v2.0 childdoc reference manual file]
\PassOptionsToClass{10pt,a4paper}{article}
\documentclass{ltxdoc}

\usepackage[margin=35mm]{geometry}
\usepackage{hyperref}
\usepackage{hyperxmp}
\usepackage[usenames]{color}

\hypersetup{colorlinks=true}
\hypersetup{pdfstartview=FitH}
\hypersetup{pdfpagemode=UseNone}
\hypersetup{pdfsource={}}
\hypersetup{pdflang={en-UK}}
\hypersetup{pdfcopyright={Copyright 2017-2018 Niklas Beisert.
  This work may be distributed and/or modified under the
  conditions of the LaTeX Project Public License, either version 1.3
  of this license or (at your option) any later version.}}
\hypersetup{pdflicenseurl={http://www.latex-project.org/lppl.txt}}
\hypersetup{pdfcontactaddress={ETH Zurich, ITP, HIT K,
  Wolfgang-Pauli-Strasse 27}}
\hypersetup{pdfcontactpostcode={8093}}
\hypersetup{pdfcontactcity={Zurich}}
\hypersetup{pdfcontactcountry={Switzerland}}
\hypersetup{pdfcontactemail={nbeisert@itp.phys.ethz.ch}}
\hypersetup{pdfcontacturl={http://people.phys.ethz.ch/\xmptilde nbeisert/}}

\newcommand{\secref}[1]{\hyperref[#1]{section \ref*{#1}}}

\parskip1ex
\parindent0pt
\let\olditemize\itemize
\def\itemize{\olditemize\parskip0pt}

\begin{document}

\title{The \textsf{childdoc} Package}
\hypersetup{pdftitle={The childdoc Package}}
\author{Niklas Beisert\\[2ex]
  Institut f\"ur Theoretische Physik\\
  Eidgen\"ossische Technische Hochschule Z\"urich\\
  Wolfgang-Pauli-Strasse 27, 8093 Z\"urich, Switzerland\\[1ex]
  \href{mailto:nbeisert@itp.phys.ethz.ch}
  {\texttt{nbeisert@itp.phys.ethz.ch}}}
\hypersetup{pdfauthor={Niklas Beisert}}
\hypersetup{pdfsubject={Manual for the LaTeX2e Package childdoc}}
\date{30 December 2018, \textsf{v2.0}}
\maketitle

\begin{abstract}\noindent
\textsf{childdoc} is a \LaTeXe{} package
that enables the direct compilation
of document sections included by |\include|
to individual files.
\end{abstract}

\begingroup
\parskip0ex
\tableofcontents
\endgroup

%%%%%%%%%%%%%%%%%%%%%%%%%%%%%%%%%%%%%%%%%%%%%%%%%%%%%%%%%%%%%%%%%%%%%%%%%%%%%%%%
%%%%%%%%%%%%%%%%%%%%%%%%%%%%%%%%%%%%%%%%%%%%%%%%%%%%%%%%%%%%%%%%%%%%%%%%%%%%%%%%
\section{Introduction}

\LaTeX{} provides a mechanism to structure a large document (such as a book)
into a main file and several child files (containing the chapters)
using the |\include| command.
This mechanism is beneficial for documents
which span hundreds of pages in order to
make the source file(s) more manageable.
Moreover, compilation can be restricted to
selected child files by means of the |\includeonly| command.
The latter feature can be used to reduce the compilation time while editing
(this was significantly more useful in the earlier days of \LaTeX{})
or to generate a smaller document which is easier to navigate.
Another application of |\includeonly| is to generate
documents consisting of selected parts of the complete document.

However, there are a few drawbacks of the plain |\include| mechanism:
\begin{itemize}
\item
The child files cannot be compiled on their own,
they can only be compiled via the main file.
A naive editing environment
(such as a text editor with an option
to have the current file processed by \LaTeX)
may require one to switch to the main file before compiling;
attempting to compile the child file produces errors.
\item
The main file must be modified (each time)
to adjust the |\includeonly| command
to the present needs. This easily leaves the main file in a messy state.
\item
The generated document will always carry the filename
of the main document. This is inconvenient if
several child files are to be compiled and
to be kept for distribution.
\end{itemize}

The present package provides a simple interface
to make child files individually compilable by \LaTeX{}.
Compiling a child file then has the same effect as compiling
the main file with an |\includeonly| command
to select the appropriate child.
Moreover the generated document will carry the name of the child
rather than the main file.
This resolves all three above issues.

This feature is meant to make the editing of books,
thesis documents and lecture notes somewhat more convenient.
However, the package can also be used efficiently for
composing a series of documents (such as exercise sheets)
which are typically distributed individually.
It then assists the author in generating the individual documents
(potentially in different versions)
as well as a document containing the collected series.
Another application is in developing style files
or other kinds of included material
where compilation of the style file could redirect
to a sample or test file.

%%%%%%%%%%%%%%%%%%%%%%%%%%%%%%%%%%%%%%%%%%%%%%%%%%%%%%%%%%%%%%%%%%%%%%%%%%%%%%%%
%%%%%%%%%%%%%%%%%%%%%%%%%%%%%%%%%%%%%%%%%%%%%%%%%%%%%%%%%%%%%%%%%%%%%%%%%%%%%%%%
\section{Usage}

First of all, the package \textsf{childdoc} is \emph{not} a standard
\LaTeXe{} |.sty| style file! Therefore it needs to be invoked in
a non-standard way.

%%%%%%%%%%%%%%%%%%%%%%%%%%%%%%%%%%%%%%%%%%%%%%%%%%%%%%%%%%%%%%%%%%%%%%%%%%%%%%%%
\subsection{Included Files}
\label{sec:include}

%%%%%%%%%%%%%%%%%%%%%%%%%%%%%%%%%%%%%%%%
\DescribeMacro{\childdocmain}
To use the package, add the commands
\begin{center}
\begin{tabular}{l}
|\input{childdoc.def}|\\
|\childdocmain{}|\\
\end{tabular}
\end{center}
at the very top of the main \LaTeX{} file,
in particular \emph{before} the |\documentclass| statement!
The argument of |\childdocmain| should be left empty
(but it must be present).

%%%%%%%%%%%%%%%%%%%%%%%%%%%%%%%%%%%%%%%%
\DescribeMacro{\childdocof}
Furthermore, add the commands
\begin{center}
\begin{tabular}{l}
|\input{childdoc.def}|\\
|\childdocof{|\textit{main}|}|\\
\end{tabular}
\end{center}
at the top of every child file \textit{child}
which is included by |\include{|\textit{child}|}|
from within the main file
(or at least for those files to be compiled individually).
The argument \textit{main} must be the filename of the main file.

There are a couple of
considerations in setting up the main and child documents:

%%%%%%%%%%%%%%%%%%%%%%%%%%%%%%%%%%%%%%%%
\paragraph{Restrictions.}

Please note the following restrictions:
\begin{itemize}
\item
|\childdocmain| must be called with one argument \textit{main}
to ensure compatibility with earlier version of the package.
It must either be empty (|\childdocmain{}|)
or precisely match the filename of the main file in which it is specified.
See \secref{sec:detection} for further information.
\item
The filename \textit{main} must be specified without the |.tex| extension.
\item
The filename \textit{main} is case sensitive
(even in case-insensitive file systems)
due to internal string comparison.
\item
The argument \textit{main} should be fully expanded, it cannot be a macro.
\item
Subdirectories and special characters should be avoided in filenames.
\item
The command |\childdocmain{|\textit{main}|}| must be followed by a whitespace.
It should not be followed immediately by another command
or by a comment mark `|%|'.
This is because the \TeX{} parser reads the token immediately following
the argument of |\childdocmain| and puts it
at the beginning of every child section;
however, a white\-space is ignored.
\end{itemize}

%%%%%%%%%%%%%%%%%%%%%%%%%%%%%%%%%%%%%%%%
\paragraph{Content of Main File.}

It is advisable to place all content in the child files included by |\include|.
Any output contained in the main file will appear in all child documents
unless suppressed manually;
it cannot be suppressed automatically by the |\includeonly| directive
and thus should normally be avoided.
A method to include some content in the main file
by means of conditional processing is described in \secref{sec:conditional}.

%%%%%%%%%%%%%%%%%%%%%%%%%%%%%%%%%%%%%%%%
\paragraph{Page Numbering.}

When only a part of the document is compiled,
the appropriate numbering of pages
(as well as other status parameters)
is determined from the |.aux| files.
The latter contain information from previous passes.
However this information needs to propagate through
all intermediate child documents.
Therefore the page numbering in child documents may well
be inconsistent until the complete document is compiled at least once.

A useful (if unconventional) way to always ensure a consistent
page numbering is to restart the numbering in each child document
and denote the pages by `\textit{child}|.|\textit{page}'
where \textit{child} represents the chapter/section number of the child file.
This can be achieved by the command
|\numberwithin{page}{|\textit{child}|}|
of the \textsf{amsmath} package
where \textit{child} can be |chapter| or |section|
depending on the chosen structuring.
Alternatively, one can modify the macro |\thepage| appropriately
and reset the counter |page| at the start of each child file.

%%%%%%%%%%%%%%%%%%%%%%%%%%%%%%%%%%%%%%%%%%%%%%%%%%%%%%%%%%%%%%%%%%%%%%%%%%%%%%%%
\subsection{Conditional Processing}
\label{sec:conditional}

The package provides a mechanism to compile different versions
of a document. To customise the versions further some conditional processing
can come in handy to distinguish which version is being compiled.
The package provides two macros to describe the compilation context:

%%%%%%%%%%%%%%%%%%%%%%%%%%%%%%%%%%%%%%%%
\DescribeMacro{\ifchilddoc}
The conditional |\ifchilddoc| distinguishes between the compilation of
child documents and the main document:
%
\begin{center}
|\ifchilddoc |\textit{child-code}| |[|\||else |\textit{main-code}]| \||fi|
\end{center}

%%%%%%%%%%%%%%%%%%%%%%%%%%%%%%%%%%%%%%%%
\DescribeMacro{\childdocname}
\DescribeMacro{\childdocjob}
The macro |\childdocname| contains the filename (without extension)
of the main or child file being processed.
Note that |\childdocjob| will always contain the name of the main file.

%%%%%%%%%%%%%%%%%%%%%%%%%%%%%%%%%%%%%%%%
\paragraph{Title Page.}

Conditional processing can be used to include a title or banner page
in the main document when proper precautions are taken.
Importantly, the code in the main file should ensure that the page counter
(as well as other status parameters which are stored in the |.aux| files)
takes the same value after the conditional processing.
Otherwise the page numbers may take divergent values
depending on which part is compiled.

For example, a title page could be declared by:
%
\begin{center}
\begin{tabular}{l}
|\ifchilddoc\||else|\\
|\addtocounter{page}{-1}|\\
\textit{code for title page}\\
|\newpage|\\
|\||fi|
\end{tabular}
\end{center}
%
A banner page for the child documents can be generated by:
%
\begin{center}
\begin{tabular}{l}
|\ifchilddoc|\\
|\addtocounter{page}{-1}|\\
\textit{code for banner page}\\
|\newpage|\\
|\||fi|
\end{tabular}
\end{center}
%
Here one could write a message such as:
\begin{center}
|This is the part \childdocname{} of \childdocjob{}.|
\end{center}

%%%%%%%%%%%%%%%%%%%%%%%%%%%%%%%%%%%%%%%%%%%%%%%%%%%%%%%%%%%%%%%%%%%%%%%%%%%%%%%%
\subsection{Flags}
\label{sec:flags}

The package makes it easy to generate different versions
of the main or child documents.
To this end compilation flags can be defined
and assigned different default values.
They will be particularly useful in conjunction
with the forwarding mechanism described in \secref{sec:forward}.

For example, it may be useful to have a flag |\version|
which can be set to |draft| or |final|.
The document source will contain some conditional code
depending on the value of |\version|.
Suppose further, the flag should default to |final| for the main file
and to |draft| for child files
which is a natural assignment for editing the document.
This is achieved by placing the following code
in the preamble of the main document
(below the |\childdocmain| directive):
%
\begin{center}
\begin{tabular}{l}
|\ifchilddoc|\\
|\providecommand{\version}{draft}|\\
|\||else|\\
|\providecommand{\version}{final}|\\
|\||fi|
\end{tabular}
\end{center}
%
The definition by |\providecommand| makes sure
that previous definitions are not overwritten.
Further statements |\providecommand{\version}{...}|
can thus be added before the above code to override it.

For the main file, one might add a line
(between |\childdocmain| and the above block)
%
\begin{center}
|%\ifchilddoc\||else\providecommand{\version}{draft}\||fi|
\end{center}
%
which can be uncommented to produce a draft version.
Likewise one can add a line to the very top of a child file
(above the |\childdocof{|\textit{main}|}| directive)
%
\begin{center}
|%\providecommand{\version}{final}|
\end{center}
%
which can be uncommented to produce the final version of this child document.

%%%%%%%%%%%%%%%%%%%%%%%%%%%%%%%%%%%%%%%%%%%%%%%%%%%%%%%%%%%%%%%%%%%%%%%%%%%%%%%%
\subsection{Forwarding}
\label{sec:forward}

Different versions of the main or child documents
using compilation flags as described in \secref{sec:flags}
can be (permanently) stored in different files
for convenient compilation, viewing and distribution.
To this end, the package defines a command
to pass on compilation to a different file:

%%%%%%%%%%%%%%%%%%%%%%%%%%%%%%%%%%%%%%%%
\DescribeMacro{\childdocforward}
The command |\childdocforward| redirects processing to
another source file:
%
\begin{center}
\begin{tabular}{l}
|\input{childdoc.def}|\\
|\childdocforward[|\textit{main}|]{|\textit{dest}|}|\\
\end{tabular}
\end{center}
%
The argument \textit{dest} is the destination file
(without extension).
It should be the main file or one of the child files.
Note that further \textsf{childdoc} directives
such as |\childdocof| and |\childdocforward|
in the indicated file will be processed in this form.
The optional argument \textit{main}
passes on directly to the main file \textit{main}
while pretending to compile the child \textit{dest}.
This form behaves as if \textit{dest}
issues |\childdocof{|\textit{main}|}| right away,
and no further \textsf{childdoc} directives will be processed.

%%%%%%%%%%%%%%%%%%%%%%%%%%%%%%%%%%%%%%%%
\DescribeMacro{\...prefix}
In the alternative form |\childdocforwardprefix|,
%
\begin{center}
\begin{tabular}{l}
|\input{childdoc.def}|\\
|\childdocforwardprefix[|\textit{main}|]{|\textit{prefix}|}{|\textit{dest}|}|
\end{tabular}
\end{center}
%
the destination file is determined by a pattern
depending on the current file:
To make this work, the current file must be called
`{\textit{prefix}\hspace{0.2em}\textit{suffix}}'
with \textit{prefix} matching precisely the argument.
Processing is then passed on to the file
`{\textit{dest}\hspace{0.2em}\textit{suffix}}'.
Surely, the same effect is achieved by
directly specifying the
argument `{\textit{dest}\hspace{0.2em}\textit{suffix}}'
in the first form.
However, that requires to set up a different file
for each child. With the alternative form of the command
all these files can have exactly the same content
which simplifies setting them up and maintaining them.

For example, the following file |draft.tex|
with a compilation flag |\version| as described in \secref{sec:flags}
compiles the main document as a draft:
%
\begin{center}
\begin{tabular}{l}
|\def\version{draft}|\\
|\input{childdoc.def}|\\
|\childdocforward{|\textit{main}|}|
\end{tabular}
\end{center}
%
Likewise, the following files |final|\textit{nn}|.tex|
compile the final version of the child document
|child|\textit{nn}|.tex|:
%
\begin{center}
\begin{tabular}{l}
|\def\version{final}|\\
|\input{childdoc.def}|\\
|\childdocforwardprefix{final}{child}|
\end{tabular}
\end{center}
%

Note that when several versions of a main file and/or of each child file
are to be generated, it may be convenient to set up a |Makefile| or
shell script to automatise the process.

%%%%%%%%%%%%%%%%%%%%%%%%%%%%%%%%%%%%%%%%%%%%%%%%%%%%%%%%%%%%%%%%%%%%%%%%%%%%%%%%
\subsection{Command Line Processing}
\label{sec:commandline}

The effect of redirection files can also be achieved by invoking
the \LaTeX{} compiler with a more elaborate command line.
Most conveniently this should be done as part
of a shell script or a |Makefile|.

When using \textsf{childdoc} in the main file, the following
command lines effectively perform a redirection
(note that depending on the shell being used,
backslashes may have to be doubled: `|\|' $\to$ `|\\|'):
%
\begin{center}
|... -jobname "|\textit{target}|" |\\|"|[\textit{flags}]%
|\input{childdoc.def}\childdocforward[|\textit{main}|]{|\textit{dest}|}"|
\end{center}
%
Here \textit{target} is the name of the output file,
\textit{main} is the name of the main file
and \textit{dest} is the name of the main or child file to be processed
(all filenames without extensions).
The optional argument \textit{main} can be omitted
if \textit{main} matches \textit{dest}.
Optionally, compilation \textit{flags} can be defined via |\def| commands.
This command line makes the \TeX{} engine believe
it is compiling the file \textit{target}
whose content is specified as the latter parameter.
The provided code then forwards the processing to
\textit{main} or \textit{dest} as described in \secref{sec:forward}.

%%%%%%%%%%%%%%%%%%%%%%%%%%%%%%%%%%%%%%%%%%%%%%%%%%%%%%%%%%%%%%%%%%%%%%%%%%%%%%%%
\subsection{Include by Input}
\label{sec:input}

Including child documents by |\include| has some restrictions by design.
Most notably, the content of a child document always occupies
its own set of pages; pages cannot be shared between child documents.
Usually, this behaviour makes perfect sense
because each child document contain an essential part of the document.
However, in some situations it may be desirable to compose
a document from a collection of parts
without having mandatory page breaks between then.
For this case, the package
provides a mechanism to include parts
by |\input| which can also be processed individually.
However, by construction this mechanism
requires manual handling of the content to be output.

%%%%%%%%%%%%%%%%%%%%%%%%%%%%%%%%%%%%%%%%
\DescribeMacro{\ifchilddocmanual}
The main file should be prepared as usual, see \secref{sec:include}.
However, the document body must make a distinction
between processing of an individual part and of the main document, e.g.:
%
\begin{center}
\begin{tabular}{l}
|\ifchilddocmanual|\\
|\input{\childdocname}|\\
|\||else|\\
\textit{document body with }|\input{|\textit{part}|}|\\
|\||fi|
\end{tabular}
\end{center}
%
The conditional |\ifchilddocmanual| is true whenever
a part to be included by |\input| is being compiled,
and the name of the part is stored in |\childdocname|.

%%%%%%%%%%%%%%%%%%%%%%%%%%%%%%%%%%%%%%%%
\DescribeMacro{\childdocby}
Each part to be included by |\input| should start with:
%
\begin{center}
\begin{tabular}{l}
|\input{childdoc.def}|\\
|\childdocby{|\textit{main}|}|\\
\end{tabular}
\end{center}
%
The directive |\childdocby| is similar to |\childdocof|
described in \secref{sec:include},
but the subsequent selection of content must be done manually.
To that end, both |\ifchilddoc| and |\ifchilddocmanual|
will be true upon processing of a part,
and the name of the part is stored in |\childdocname|.
Note that |\jobname| will be set to the filename of the current part
so that each part receives an individual |.aux| file
that does not interfere with the |.aux| file(s) of the main document.
This behaviour can be altered by the alternative form
|\childdocby[*]{|\textit{main}|}| (with a non-empty optional argument)
which uses the |.aux| file of the main document
by setting |\jobname| to \textit{main}.

%%%%%%%%%%%%%%%%%%%%%%%%%%%%%%%%%%%%%%%%%%%%%%%%%%%%%%%%%%%%%%%%%%%%%%%%%%%%%%%%
\subsection{Driver Development}
\label{sec:driver}

The \textsf{childdoc} mechanism can also be use for the development
of definition files such as \LaTeX{} styles or classes.
This case differs from the above setup with multiple parts
included by |\include| in that no |\includeonly| should be invoked.
This can be achieved by starting the include file
(before |\ProvidesPackage|) with:
%
\begin{center}
\begin{tabular}{l}
|\input{childdoc.def}|\\
|\childdocforward{|\textit{main}|}|\\
\end{tabular}
\end{center}
%
or alternatively with:
%
\begin{center}
\begin{tabular}{l}
|\input{childdoc.def}|\\
|\childdocby{|\textit{main}|}|\\
\end{tabular}
\end{center}
%
Both forms have slightly different effects as described above.
The main file is prepared as usual, see \secref{sec:include}.

%%%%%%%%%%%%%%%%%%%%%%%%%%%%%%%%%%%%%%%%%%%%%%%%%%%%%%%%%%%%%%%%%%%%%%%%%%%%%%%%
\subsection{Legacy Detection}
\label{sec:detection}

The directive |\childdocmain| in the main file can detect
whether the complete document or merely a child is to be compiled
even without using the directive |\childdocof|.
This method is deprecated because it is less robust
and there is no compelling reason to use it;
it is merely provided for backward compatibility
and it may be removed in future versions.

If the detection mechanism is to be used,
it is mandatory to correctly specify
the filename of the main file as the argument of |\childdocmain|:
%
\begin{center}
\begin{tabular}{l}
|\input{childdoc.def}|\\
|\childdocmain{|\textit{main}|}|\\
\end{tabular}
\end{center}
%
If |\jobname| does not match the argument \textit{main} of |\childdocmain|,
it is assumed that |\jobname| points to the child file to be compiled.
When using |\childdocmain| with the main file specified as argument,
it suffices to start a child file
with just |\input{|\textit{main}|}|
without loading of the package and using |\childdocof|.
If instead all processing is done
with the appropriate \textsf{childdoc} directives,
the argument of \textit{main} of |\childdocmain| can be empty.

An alternative version of the command line processing described
in \secref{sec:commandline} using the detection mechanism reads:
%
\begin{center}
|... -jobname "|\textit{target}|" "|[\textit{flags}]%
[|\def\jobname{|\textit{dest}|}|]|\input{|\textit{main}|}"|
\end{center}

%%%%%%%%%%%%%%%%%%%%%%%%%%%%%%%%%%%%%%%%%%%%%%%%%%%%%%%%%%%%%%%%%%%%%%%%%%%%%%%%
\subsection{Manual Code}
\label{sec:manual}

In case one cannot be certain whether the definitions file |childdoc.def|
is installed on the target \TeX{} distribution
and one prefers not to ship it,
it is conceivable to paste a few relevant commands into the sources.

To that end, drop all statements |\input{childdoc.def}|
and perform the replacements as outlined below.
Instead of |\childdocmain{|\textit{main}|}| add the following code
to the top of the main file:
%
\begin{center}
\begin{tabular}{l}
|\||ifdefined\childdocname\endinput\||fi\newif\ifchilddoc|\\
|\edef\childdocname{\scantokens\expandafter{\jobname\noexpand}}|\\
|\def\childdocmain{|\textit{main}|}\||ifx\childdocmain\childdocname\||else|\\
|\childdoctrue\includeonly{\childdocname}\let\jobname\childdocmain\||fi|\\
\end{tabular}
\end{center}
%
Instead of |\childdocof{|\textit{main}|}| just include the main file
at the top of each child file:
%
\begin{center}
|\input{|\textit{main}|}|
\end{center}
%
A simple redirection |\childdocforward{|\textit{dest}|}| is achieved by:
%
\begin{center}
|\def\jobname{|\textit{dest}|}\input{\jobname}|
\end{center}
%
The redirection with prefix
|\childdocforwardprefix[|\textit{prefix}|]{|\textit{dest}|}|
is accomplished by:
%
\begin{center}
\begin{tabular}{l}
|{\edef\jobname{\scantokens\expandafter{\jobname\noexpand}}|\\
|\def\redirectjob |\textit{prefix}|#1~~~{\gdef\jobname{|\textit{dest}|#1}}|\\
|\expandafter\redirectjob\jobname~~~}\input{\jobname}|
\end{tabular}
\end{center}

In an alternative approach,
child documents can be compiled by a specific command line
without additional code or specific definitions:
%
\begin{center}
|... -jobname "|\textit{target}|" "|[\textit{flags}]%
|\includeonly{|\textit{dest}|}\input{|\textit{main}|}"|
\end{center}
%

%%%%%%%%%%%%%%%%%%%%%%%%%%%%%%%%%%%%%%%%%%%%%%%%%%%%%%%%%%%%%%%%%%%%%%%%%%%%%%%%
%%%%%%%%%%%%%%%%%%%%%%%%%%%%%%%%%%%%%%%%%%%%%%%%%%%%%%%%%%%%%%%%%%%%%%%%%%%%%%%%
\section{Information}

%%%%%%%%%%%%%%%%%%%%%%%%%%%%%%%%%%%%%%%%%%%%%%%%%%%%%%%%%%%%%%%%%%%%%%%%%%%%%%%%
\subsection{Copyright}

Copyright \copyright{} 2017--2018 Niklas Beisert

This work may be distributed and/or modified under the
conditions of the \LaTeX{} Project Public License, either version 1.3
of this license or (at your option) any later version.
The latest version of this license is in
  \url{http://www.latex-project.org/lppl.txt}
and version 1.3 or later is part of all distributions of \LaTeX{}
version 2005/12/01 or later.

This work has the LPPL maintenance status `maintained'.

The Current Maintainer of this work is Niklas Beisert.

This work consists of the files |README.txt|, |childdoc.ins| and |childdoc.dtx|
as well as the derived files |childdoc.def|, |cdocsamp.tex|
with |cdocsch1.tex|, |cdocsch2.tex|, |cdocspt3.tex|, |cdocspt4.tex|,
|cdocsdrf.tex|, |cdocsfn1.tex|, |cdocsfn2.tex|
as well as |childdoc.pdf|.

%%%%%%%%%%%%%%%%%%%%%%%%%%%%%%%%%%%%%%%%%%%%%%%%%%%%%%%%%%%%%%%%%%%%%%%%%%%%%%%%
\subsection{Files and Installation}

The package consists of the files:
%
\begin{center}
\begin{tabular}{ll}
    |README.txt|   & readme file \\
    |childdoc.ins| & installation file \\
    |childdoc.dtx| & source file \\
    |childdoc.def| & definition file \\
    |cdocsamp.tex| & sample main file \\
    |cdocsch1.tex| & sample include file \\
    |cdocsch2.tex| & sample include file \\
    |cdocspt3.tex| & sample part file \\
    |cdocspt4.tex| & sample part file \\
    |cdocsdrf.tex| & sample redirection file \\
    |cdocsfn1.tex| & sample redirection file \\
    |cdocsfn2.tex| & sample redirection file \\
    |childdoc.pdf| & manual
\end{tabular}
\end{center}
%
The distribution consists of the files
|README.txt|, |childdoc.ins| and |childdoc.dtx|.
%
\begin{itemize}
\item
Run (pdf)\LaTeX{} on |childdoc.dtx|
to compile the manual |childdoc.pdf| (this file).
\item
Run \LaTeX{} on |childdoc.ins| to create the definitions file |childdoc.def|
and the sample |cdocsamp.tex| with include files
|cdocsch1.tex|, |cdocsch2.tex|, |cdocspt3.tex|, |cdocspt4.tex|,
|cdocsdrf.tex|, |cdocsfn1.tex|, |cdocsfn2.tex|.
Then copy the file |childdoc.def| to an appropriate directory of your \LaTeX{}
distribution, e.g.\ \textit{texmf-root}|/tex/latex/childdoc|.
\end{itemize}

%%%%%%%%%%%%%%%%%%%%%%%%%%%%%%%%%%%%%%%%%%%%%%%%%%%%%%%%%%%%%%%%%%%%%%%%%%%%%%%%
\subsection{Related CTAN Packages}

There are several other packages which offer a similar functionality:
%
\begin{itemize}
\item
The packages
\href{http://ctan.org/pkg/docmute}{\textsf{docmute}},
\href{http://ctan.org/pkg/includex}{\textsf{includex}} and
\href{http://ctan.org/pkg/standalone}{\textsf{standalone}}
provide commands to include only the document body of
a child file thus allowing both files to be compiled individually.
\item
The packages \href{http://ctan.org/pkg/subdocs}{\textsf{subdocs}}
and \href{http://ctan.org/pkg/subfiles}{\textsf{subfiles}}
provide structures in which the main and child documents can be
encapsulated and allowing them to be compiled individually.
The inclusion mechanism is different from the conventional |\include|.
\item
The package \href{http://ctan.org/pkg/combine}{\textsf{combine}}
is an elaborate solution to combine several documents into one.
\end{itemize}
%
See also the CTAN topic \href{http://ctan.org/topic/subdocs}{\textsf{subdocs}}
for further related packages.
The present package differs from the above solutions in that
a document structure constructed with the conventional |\include| mechanism
just needs two extra commands at the top of every file
such that all constituent files can be compiled individually.

%%%%%%%%%%%%%%%%%%%%%%%%%%%%%%%%%%%%%%%%%%%%%%%%%%%%%%%%%%%%%%%%%%%%%%%%%%%%%%%%
%\subsection{Feature Suggestions}
%
%The following is a list of features which may be useful for future
%versions of this package:
%%
%\begin{itemize}
%\item
%\ldots
%\end{itemize}

%%%%%%%%%%%%%%%%%%%%%%%%%%%%%%%%%%%%%%%%%%%%%%%%%%%%%%%%%%%%%%%%%%%%%%%%%%%%%%%%
\subsection{Revision History}

%%%%%%%%%%%%%%%%%%%%%%%%%%%%%%%%%%%%%%%%
\paragraph{v2.0:} 2018/12/30

\begin{itemize}
\item
immediate forward processing
\item
added |\childdocby| mechanism
\item
manual restructured
\end{itemize}

%%%%%%%%%%%%%%%%%%%%%%%%%%%%%%%%%%%%%%%%
\paragraph{v1.6:} 2018/01/17

\begin{itemize}
\item
application for development of include files
\item
corrections to manual
\end{itemize}

%%%%%%%%%%%%%%%%%%%%%%%%%%%%%%%%%%%%%%%%
\paragraph{v1.5:} 2017/05/21

\begin{itemize}
\item
more complete structuring introduced
\item
|\childdocof| introduced
\item
|\childdoc| renamed to |\childdocmain|
\item
|\childredirect| renamed to |\childdocforward| and |\childdocforwardprefix|
and functionality expanded
\end{itemize}

%%%%%%%%%%%%%%%%%%%%%%%%%%%%%%%%%%%%%%%%
\paragraph{v1.0:} 2017/04/27

\begin{itemize}
\item
manual and install package
\item
first version published on CTAN
\end{itemize}

%%%%%%%%%%%%%%%%%%%%%%%%%%%%%%%%%%%%%%%%
\paragraph{v0.6:} 2017/04/26

\begin{itemize}
\item
redirection mechanism added
\end{itemize}

%%%%%%%%%%%%%%%%%%%%%%%%%%%%%%%%%%%%%%%%
\paragraph{v0.5:} 2017/04/26

\begin{itemize}
\item
functionality in definition file
\end{itemize}


%%%%%%%%%%%%%%%%%%%%%%%%%%%%%%%%%%%%%%%%%%%%%%%%%%%%%%%%%%%%%%%%%%%%%%%%%%%%%%%%
%%%%%%%%%%%%%%%%%%%%%%%%%%%%%%%%%%%%%%%%%%%%%%%%%%%%%%%%%%%%%%%%%%%%%%%%%%%%%%%%
%%%%%%%%%%%%%%%%%%%%%%%%%%%%%%%%%%%%%%%%%%%%%%%%%%%%%%%%%%%%%%%%%%%%%%%%%%%%%%%%
\appendix

\settowidth\MacroIndent{\rmfamily\scriptsize 000\ }

 \DocInput{childdoc.dtx}

\end{document}
%</driver>
% \fi
%
% %%%%%%%%%%%%%%%%%%%%%%%%%%%%%%%%%%%%%%%%%%%%%%%%%%%%%%%%%%%%%%%%%%%%%%%%%%%%%%
% %%%%%%%%%%%%%%%%%%%%%%%%%%%%%%%%%%%%%%%%%%%%%%%%%%%%%%%%%%%%%%%%%%%%%%%%%%%%%%
% \section{Sample}
%\iffalse
%<*samplemain>
%\fi
%
% The following presents a sample document
% with two chapters, two parts, a title page,
% a compile flag as well as three forwarding files to set the flag.
% It consists of eight |.tex| files:
% \begin{center}
% \begin{tabular}{ll}
% |cdocsamp.tex|&main file\\
% |cdocsch1.tex|&include file for chapter 1\\
% |cdocsch2.tex|&include file for chapter 2\\
% |cdocspt3.tex|&include file for part 3\\
% |cdocspt4.tex|&include file for part 4\\
% |cdocsdrf.tex|&forwarding file for main file in draft mode\\
% |cdocsfi1.tex|&forwarding file for final version of chapter 1\\
% |cdocsfi2.tex|&forwarding file for final version of chapter 2\\
% \end{tabular}
% \end{center}
% Each of the eight files can be compiled directly by the \LaTeX{} compiler.
%
% %%%%%%%%%%%%%%%%%%%%%%%%%%%%%%%%%%%%%%
% \paragraph{Main File.}
%
% The main file is called |cdocsamp.tex|.
%
% Load the \textsf{childdoc} definitions and
% declare the filename for the main document:
%    \begin{macrocode}
\input{childdoc.def}
\childdocmain{}
%    \end{macrocode}

% Optional override for |\version| flag:
%    \begin{macrocode}
%%\ifchilddoc\else\providecommand{\version}{draft}\fi
%    \end{macrocode}

% Define the default values for the |\version| flag
% (|final| for the main file and |draft| for childs):
%    \begin{macrocode}
\ifchilddoc
\providecommand{\version}{draft}
\else
\providecommand{\version}{final}
\fi
%    \end{macrocode}

% Load the standard document class:
%    \begin{macrocode}
\documentclass[12pt]{article}
%    \end{macrocode}

% Start the document body:
%    \begin{macrocode}
\begin{document}
%    \end{macrocode}

% Declare a title page.
% Print title, part of document being processed and version flag:
%    \begin{macrocode}
\addtocounter{page}{-1}
\begin{center}
{\LARGE\bfseries{}childdoc example\par}
\vspace{1cm}
\ifchilddoc
\ifchilddocmanual part\else chapter\fi:
`\childdocname' of `\childdocjob'\par
\else
main document: `\childdocjob'\par
\fi
version: \version\par
\end{center}
\newpage
%    \end{macrocode}

% Manually include selected file,
% otherwise process as usual:
%    \begin{macrocode}
\ifchilddocmanual
\section*{part `\childdocname'}
\input{\childdocname}
\else
%    \end{macrocode}

% Include the two chapters:
%    \begin{macrocode}
\include{cdocsch1}
\include{cdocsch2}
%    \end{macrocode}

% Include the two parts unless only chapters should be displayed:
%    \begin{macrocode}
\ifchilddoc\else
\section{part three}
\input{cdocspt3}
\section{part four}
\input{cdocspt4}
\fi
%    \end{macrocode}

% Process as usual until here:
%    \begin{macrocode}
\fi
%    \end{macrocode}

% End of document body:
%    \begin{macrocode}
\end{document}
%    \end{macrocode}
%\iffalse
%</samplemain>
%\fi
%
% %%%%%%%%%%%%%%%%%%%%%%%%%%%%%%%%%%%%%%
% \paragraph{Chapter Include Files.}
%
% The include files are called |cdocsch1.tex| and |cdocsch2.tex|.
%
%\iffalse
%<*samplechap1|samplechap2>
%\fi

% Optional override for |\version| flag:
%    \begin{macrocode}
%%\providecommand{\version}{final}
%    \end{macrocode}

% Include the main document:
%    \begin{macrocode}
\input{childdoc.def}
\childdocof{cdocsamp}
%    \end{macrocode}

%\iffalse
%</samplechap1|samplechap2>
%\fi
%
%\iffalse
%<*samplechap1>
%\fi
% Some text for chapter 1:
%    \begin{macrocode}
\section{one}
some text in chapter one
%    \end{macrocode}

%\iffalse
%</samplechap1>
%\fi
% Some text for chapter 2:
%\iffalse
%<*samplechap2>
%\fi
%    \begin{macrocode}
\section{two}
more text in chapter two
%    \end{macrocode}

%\iffalse
%</samplechap2>
%\fi
%
% %%%%%%%%%%%%%%%%%%%%%%%%%%%%%%%%%%%%%%
% \paragraph{Part Include Files.}
%
% The include files are called |cdocspt3.tex| and |cdocspt4.tex|.
%
%\iffalse
%<*samplepart3|samplepart4>
%\fi

% Optional override for |\version| flag:
%    \begin{macrocode}
%%\providecommand{\version}{final}
%    \end{macrocode}

% Include the main document:
%    \begin{macrocode}
\input{childdoc.def}
\childdocby{cdocsamp}
%    \end{macrocode}

%\iffalse
%</samplepart3|samplepart4>
%\fi
%
%\iffalse
%<*samplepart3>
%\fi
% Some text for part 3:
%    \begin{macrocode}
some text in part three
%    \end{macrocode}

%\iffalse
%</samplepart3>
%\fi
% Some text for part 4:
%\iffalse
%<*samplepart4>
%\fi
%    \begin{macrocode}
more text in part four
%    \end{macrocode}

%\iffalse
%</samplepart4>
%\fi
%
% %%%%%%%%%%%%%%%%%%%%%%%%%%%%%%%%%%%%%%
% \paragraph{Forwarding for a Complete Draft.}
%
% The following forwarding file |cdocsdrf.tex|
% compiles the main document in draft mode:
%\iffalse
%<*sampledraft>
%\fi
%    \begin{macrocode}
\def\version{draft}
\input{childdoc.def}
\childdocforward{cdocsamp}
%    \end{macrocode}

%\iffalse
%</sampledraft>
%\fi
%
% %%%%%%%%%%%%%%%%%%%%%%%%%%%%%%%%%%%%%%
% \paragraph{Forwarding for Final Version of the Chapters.}
%
% The following forwarding files |cdocsfn1.tex| and |cdocsfn2.tex|
% (with identical content)
% compile the final versions of the child documents
% |cdocsch1.tex| and |cdocsch2.tex|, respectively:
%\iffalse
%<*samplefinal>
%\fi
%    \begin{macrocode}
\def\version{final}
\input{childdoc.def}
\childdocforwardprefix[cdocsamp]{cdocsfn}{cdocsch}
%    \end{macrocode}

%\iffalse
%</samplefinal>
%\fi
%
% %%%%%%%%%%%%%%%%%%%%%%%%%%%%%%%%%%%%%%
% \paragraph{Command Line Processing.}
%
% The following three command lines generate the output files
% |cdocscld|, |cdocscl1| and |cdocscl2|
% which should be identical to
% |cdocsdrf|, |cdocsch1| and |cdocsfn2|, respectively:
% \begin{center}
% \begin{tabular}{l}
% |latex -jobname cdocscld \|\\
% |  "\def\version{draft}\input{childdoc.def}\childdocforward{cdocsamp}"|\\
% |latex -jobname cdocscl1 \|\\
% |  "\input{childdoc.def}\childdocforward[cdocsamp]{cdocsch1}"|\\
% |latex -jobname cdocscl2 \|\\
% |  "\def\version{final}\input{childdoc.def}\childdocforward{cdocsch2}"|
% \end{tabular}
% \end{center}
% Note that the trailing backslash on each first line
% merely continues the input to the second line
% (for convenient cut ant paste).
% Furthermore, the command |latex| can be replaced by any
% of its alternative versions such as |pdflatex|.
%
% %%%%%%%%%%%%%%%%%%%%%%%%%%%%%%%%%%%%%%%%%%%%%%%%%%%%%%%%%%%%%%%%%%%%%%%%%%%%%%
% %%%%%%%%%%%%%%%%%%%%%%%%%%%%%%%%%%%%%%%%%%%%%%%%%%%%%%%%%%%%%%%%%%%%%%%%%%%%%%
% \section{Implementation}
%\iffalse
%<*package>
%\fi
%
% This section describes the definitions file |childdoc.def|.

% The definitions cannot be loaded using |\usepackage| or |\RequirePackage|
% which has a mechanism to prevent loading a style file more than once.
% When loading the definitions by means of |\input|
% multiple instances have to be prevented manually:
%\iffalse
%This code needs to be before the `\ProvidesFile' directive
%which is defined at the beginning of this file.
%Therefore it is also placed there and commented out here.
%</package>
%<*discard>
%\fi
%    \begin{macrocode}
\ifdefined\childdocmain\endinput\fi
%    \end{macrocode}
%\iffalse
%</discard>
%<*package>
%\fi
%
% \macro{\ifchilddoc}
% \macro{\ifchilddocmanual}
% The conditional |\ifchilddoc| tells whether a
% child (true) or main (false) document is being compiled.
% The conditional |\ifchilddocmanual| tells whether
% the |\includeonly| mechanism is used (false) or
% the selection of child files must be performed manually (true).
% The definitions initialise to false:
%    \begin{macrocode}
\newif\ifchilddoc
\newif\ifchilddocmanual
%    \end{macrocode}

% \macro{\childdocname}
% \macro{\childdocjob}
% The macro |\childdocname| stores the name of the main document
% to be compiled. The macro |\childdocjob| stores the name of
% the document on which the \LaTeX{} compiler was originally invoked.
% The content of |\jobname| cannot be compared
% to filenames specified in the source due to different catcodes.
% The following code rescans |\jobname|, stores the result
% in |\childdocname| and saves a copy in |\childdocjob|:
%    \begin{macrocode}
\edef\childdocname{\scantokens\expandafter{\jobname\noexpand}}
\let\childdocjob\childdocname
%    \end{macrocode}

% \macro{\childdocdisable}
% The macro |\childdocdisable| prevents the main file
% from being processed more than once.
% At this stage, the main document command |\childdocmain|
% is assumed to be called once again where it should do nothing.
% Any subsequent call to it should prevent
% a secondary processing of the main document
% It overwrites the forwarding commands
% |\childdocof| and |\childdocforward|
% with empty macros to prevent further inclusions of the main document:
%    \begin{macrocode}
\newcommand{\childdocdisable}
{
  \renewcommand{\childdocmain}[1]{\renewcommand{\childdocmain}[1]{\endinput}}
  \renewcommand{\childdocof}[1]{}
  \renewcommand{\childdocby}[2][]{}
  \renewcommand{\childdocforward}[2][]{}
  \renewcommand{\childdocdisable}{}
}
%    \end{macrocode}

% \macro{\childdocmain}
% The macro |\childdocmain| is to be called at the top of the main file
% with nothing or the main filename (without extension) as argument.
% First, it breaks loops.
% If the argument is not empty and does not match |\childdocname|
% (which is set by the first inclusion of |childdoc.def|),
% |\ifchilddoc| is set to true, |\includeonly| is applied to the child file
% and |\jobname| is set to the main file
% (for proper handling of |.aux| files):
%    \begin{macrocode}
\newcommand{\childdocmain}[1]
{
  \childdocdisable\childdocmain{}
  \if?#1?\else
    \begingroup
      \def\childdoctmp{#1}
      \ifx\childdoctmp\childdocname
        \def\childdoctmp{}
      \else
        \def\childdoctmp
        {
          \childdoctrue
          \includeonly{\childdocname}
          \def\childdocjob{#1}
          \def\jobname{#1}
        }
      \fi
      \expandafter
    \endgroup
    \childdoctmp
  \fi
}
%    \end{macrocode}

% \macro{\childdocof}
% The command |\childdocof| redirects
% compilation to the main file |#1|.
%    \begin{macrocode}
\newcommand{\childdocof}[1]
{
  \childdocdisable
  \childdoctrue
  \includeonly{\childdocname}
  \def\jobname{#1}
  \def\childdocjob{#1}
  \input{#1}
}
%    \end{macrocode}

% \macro{\childdocby}
% The command |\childdocby| ....
%    \begin{macrocode}
\newcommand{\childdocby}[2][]
{
  \childdocdisable
  \childdoctrue
  \childdocmanualtrue
  \if?#1?\else
    \def\jobname{#2}
  \fi
  \def\childdocjob{#2}
  \input{#2}
  \endinput
}
%    \end{macrocode}

% \macro{\childdocforward}
% The command |\childdocforward| redirects
% compilation to the main file or
% (if the optional argument is given) a child file.
% Parameters are set as if the main file
% or a child file starting with |\childdocof| was compiled.
% Then compilation is handed over to the main file:
%    \begin{macrocode}
\newcommand{\childdocforward}[2][]
{
  \begingroup
    \if?#1?
      \def\childdoctmp
      {
        \def\childdocname{#2}
        \def\childdocjob{#2}
        \def\jobname{#2}
        \input{#2}
        \endinput
      }
    \else
      \def\childdoctmp
      {
        \childdocdisable
        \def\childdocname{#2}
        \childdoctrue
        \includeonly{#2}
        \def\childdocjob{#1}
        \def\jobname{#1}
        \input{#1}
        \endinput
      }
    \fi
    \expandafter
  \endgroup
  \childdoctmp
}
%    \end{macrocode}

% \macro{\childdocforwardprefix}
% The command |\childdocforwardprefix| redirects
% compilation to the main or a child file by means of a pattern.
% The prefix |#1| in the current filename is replaced by |#2|
% and the suffix of the current filename is kept
% (it is assumed that the filename does not contain the substring `|~~~|'
% which is used as a delimiter).
% Compilation is handed over to the new file by |\childdocforward|:
%    \begin{macrocode}
\newcommand{\childdocforwardprefix}[3][]
{
  \begingroup
    \def\childdocextract #2##1~~~{\def\childdoctmp{\childdocforward[#1]{#3##1}}}
    \expandafter\childdocextract\childdocname~~~
    \expandafter
  \endgroup
  \childdoctmp
}
%    \end{macrocode}

% \macro{\childdoc}
% The deprecated macro |\childdoc| is a legacy version of |\childdocmain|:
%    \begin{macrocode}
\newcommand{\childdoc}{\childdocmain}
%    \end{macrocode}

% \macro{\childdocredirect}
% The deprecated macro |\childdocredirect| is a legacy version
% of |\childdocforward| and |\childdocforwardprefix|:
%    \begin{macrocode}
\newcommand{\childdocredirect}[2][]
{
  \begingroup
    \if?#1?
      \def\childdoctmp{\childdocforward{#2}}
    \else
      \def\childdoctmp{\childdocforwardprefix{#1}{#2}}
    \fi
    \expandafter
  \endgroup
  \childdoctmp
}
%    \end{macrocode}

%\iffalse
%</package>
%\fi
%
\endinput
|\\
|\childdocof{|\textit{main}|}|\\
\end{tabular}
\end{center}
at the top of every child file \textit{child}
which is included by |\include{|\textit{child}|}|
from within the main file
(or at least for those files to be compiled individually).
The argument \textit{main} must be the filename of the main file.

There are a couple of
considerations in setting up the main and child documents:

%%%%%%%%%%%%%%%%%%%%%%%%%%%%%%%%%%%%%%%%
\paragraph{Restrictions.}

Please note the following restrictions:
\begin{itemize}
\item
|\childdocmain| must be called with one argument \textit{main}
to ensure compatibility with earlier version of the package.
It must either be empty (|\childdocmain{}|)
or precisely match the filename of the main file in which it is specified.
See \secref{sec:detection} for further information.
\item
The filename \textit{main} must be specified without the |.tex| extension.
\item
The filename \textit{main} is case sensitive
(even in case-insensitive file systems)
due to internal string comparison.
\item
The argument \textit{main} should be fully expanded, it cannot be a macro.
\item
Subdirectories and special characters should be avoided in filenames.
\item
The command |\childdocmain{|\textit{main}|}| must be followed by a whitespace.
It should not be followed immediately by another command
or by a comment mark `|%|'.
This is because the \TeX{} parser reads the token immediately following
the argument of |\childdocmain| and puts it
at the beginning of every child section;
however, a white\-space is ignored.
\end{itemize}

%%%%%%%%%%%%%%%%%%%%%%%%%%%%%%%%%%%%%%%%
\paragraph{Content of Main File.}

It is advisable to place all content in the child files included by |\include|.
Any output contained in the main file will appear in all child documents
unless suppressed manually;
it cannot be suppressed automatically by the |\includeonly| directive
and thus should normally be avoided.
A method to include some content in the main file
by means of conditional processing is described in \secref{sec:conditional}.

%%%%%%%%%%%%%%%%%%%%%%%%%%%%%%%%%%%%%%%%
\paragraph{Page Numbering.}

When only a part of the document is compiled,
the appropriate numbering of pages
(as well as other status parameters)
is determined from the |.aux| files.
The latter contain information from previous passes.
However this information needs to propagate through
all intermediate child documents.
Therefore the page numbering in child documents may well
be inconsistent until the complete document is compiled at least once.

A useful (if unconventional) way to always ensure a consistent
page numbering is to restart the numbering in each child document
and denote the pages by `\textit{child}|.|\textit{page}'
where \textit{child} represents the chapter/section number of the child file.
This can be achieved by the command
|\numberwithin{page}{|\textit{child}|}|
of the \textsf{amsmath} package
where \textit{child} can be |chapter| or |section|
depending on the chosen structuring.
Alternatively, one can modify the macro |\thepage| appropriately
and reset the counter |page| at the start of each child file.

%%%%%%%%%%%%%%%%%%%%%%%%%%%%%%%%%%%%%%%%%%%%%%%%%%%%%%%%%%%%%%%%%%%%%%%%%%%%%%%%
\subsection{Conditional Processing}
\label{sec:conditional}

The package provides a mechanism to compile different versions
of a document. To customise the versions further some conditional processing
can come in handy to distinguish which version is being compiled.
The package provides two macros to describe the compilation context:

%%%%%%%%%%%%%%%%%%%%%%%%%%%%%%%%%%%%%%%%
\DescribeMacro{\ifchilddoc}
The conditional |\ifchilddoc| distinguishes between the compilation of
child documents and the main document:
%
\begin{center}
|\ifchilddoc |\textit{child-code}| |[|\||else |\textit{main-code}]| \||fi|
\end{center}

%%%%%%%%%%%%%%%%%%%%%%%%%%%%%%%%%%%%%%%%
\DescribeMacro{\childdocname}
\DescribeMacro{\childdocjob}
The macro |\childdocname| contains the filename (without extension)
of the main or child file being processed.
Note that |\childdocjob| will always contain the name of the main file.

%%%%%%%%%%%%%%%%%%%%%%%%%%%%%%%%%%%%%%%%
\paragraph{Title Page.}

Conditional processing can be used to include a title or banner page
in the main document when proper precautions are taken.
Importantly, the code in the main file should ensure that the page counter
(as well as other status parameters which are stored in the |.aux| files)
takes the same value after the conditional processing.
Otherwise the page numbers may take divergent values
depending on which part is compiled.

For example, a title page could be declared by:
%
\begin{center}
\begin{tabular}{l}
|\ifchilddoc\||else|\\
|\addtocounter{page}{-1}|\\
\textit{code for title page}\\
|\newpage|\\
|\||fi|
\end{tabular}
\end{center}
%
A banner page for the child documents can be generated by:
%
\begin{center}
\begin{tabular}{l}
|\ifchilddoc|\\
|\addtocounter{page}{-1}|\\
\textit{code for banner page}\\
|\newpage|\\
|\||fi|
\end{tabular}
\end{center}
%
Here one could write a message such as:
\begin{center}
|This is the part \childdocname{} of \childdocjob{}.|
\end{center}

%%%%%%%%%%%%%%%%%%%%%%%%%%%%%%%%%%%%%%%%%%%%%%%%%%%%%%%%%%%%%%%%%%%%%%%%%%%%%%%%
\subsection{Flags}
\label{sec:flags}

The package makes it easy to generate different versions
of the main or child documents.
To this end compilation flags can be defined
and assigned different default values.
They will be particularly useful in conjunction
with the forwarding mechanism described in \secref{sec:forward}.

For example, it may be useful to have a flag |\version|
which can be set to |draft| or |final|.
The document source will contain some conditional code
depending on the value of |\version|.
Suppose further, the flag should default to |final| for the main file
and to |draft| for child files
which is a natural assignment for editing the document.
This is achieved by placing the following code
in the preamble of the main document
(below the |\childdocmain| directive):
%
\begin{center}
\begin{tabular}{l}
|\ifchilddoc|\\
|\providecommand{\version}{draft}|\\
|\||else|\\
|\providecommand{\version}{final}|\\
|\||fi|
\end{tabular}
\end{center}
%
The definition by |\providecommand| makes sure
that previous definitions are not overwritten.
Further statements |\providecommand{\version}{...}|
can thus be added before the above code to override it.

For the main file, one might add a line
(between |\childdocmain| and the above block)
%
\begin{center}
|%\ifchilddoc\||else\providecommand{\version}{draft}\||fi|
\end{center}
%
which can be uncommented to produce a draft version.
Likewise one can add a line to the very top of a child file
(above the |\childdocof{|\textit{main}|}| directive)
%
\begin{center}
|%\providecommand{\version}{final}|
\end{center}
%
which can be uncommented to produce the final version of this child document.

%%%%%%%%%%%%%%%%%%%%%%%%%%%%%%%%%%%%%%%%%%%%%%%%%%%%%%%%%%%%%%%%%%%%%%%%%%%%%%%%
\subsection{Forwarding}
\label{sec:forward}

Different versions of the main or child documents
using compilation flags as described in \secref{sec:flags}
can be (permanently) stored in different files
for convenient compilation, viewing and distribution.
To this end, the package defines a command
to pass on compilation to a different file:

%%%%%%%%%%%%%%%%%%%%%%%%%%%%%%%%%%%%%%%%
\DescribeMacro{\childdocforward}
The command |\childdocforward| redirects processing to
another source file:
%
\begin{center}
\begin{tabular}{l}
|% \iffalse
%
% childdoc.dtx Copyright (C) 2017-2018 Niklas Beisert
%
% This work may be distributed and/or modified under the
% conditions of the LaTeX Project Public License, either version 1.3
% of this license or (at your option) any later version.
% The latest version of this license is in
%   http://www.latex-project.org/lppl.txt
% and version 1.3 or later is part of all distributions of LaTeX
% version 2005/12/01 or later.
%
% This work has the LPPL maintenance status `maintained'.
%
% The Current Maintainer of this work is Niklas Beisert.
%
% This work consists of the files childdoc.dtx and childdoc.ins
% and the derived files childdoc.def and cdocsamp.tex with
% cdocsch1.tex, cdocsch2.tex, cdocsdrf.tex, cdocsfn1.tex, cdocsfn2.tex.
%
%<package>\ifdefined\childdocmain\endinput\fi
%<package>\ProvidesFile{childdoc.def}[2018/12/30 v2.0 child document driver]
%<samplemain>\ProvidesFile{cdocsamp.tex}[2018/12/30 v2.0 sample for childdoc]
%<*driver>
%\ProvidesFile{childdoc.drv}[2018/12/30 v2.0 childdoc reference manual file]
\PassOptionsToClass{10pt,a4paper}{article}
\documentclass{ltxdoc}

\usepackage[margin=35mm]{geometry}
\usepackage{hyperref}
\usepackage{hyperxmp}
\usepackage[usenames]{color}

\hypersetup{colorlinks=true}
\hypersetup{pdfstartview=FitH}
\hypersetup{pdfpagemode=UseNone}
\hypersetup{pdfsource={}}
\hypersetup{pdflang={en-UK}}
\hypersetup{pdfcopyright={Copyright 2017-2018 Niklas Beisert.
  This work may be distributed and/or modified under the
  conditions of the LaTeX Project Public License, either version 1.3
  of this license or (at your option) any later version.}}
\hypersetup{pdflicenseurl={http://www.latex-project.org/lppl.txt}}
\hypersetup{pdfcontactaddress={ETH Zurich, ITP, HIT K,
  Wolfgang-Pauli-Strasse 27}}
\hypersetup{pdfcontactpostcode={8093}}
\hypersetup{pdfcontactcity={Zurich}}
\hypersetup{pdfcontactcountry={Switzerland}}
\hypersetup{pdfcontactemail={nbeisert@itp.phys.ethz.ch}}
\hypersetup{pdfcontacturl={http://people.phys.ethz.ch/\xmptilde nbeisert/}}

\newcommand{\secref}[1]{\hyperref[#1]{section \ref*{#1}}}

\parskip1ex
\parindent0pt
\let\olditemize\itemize
\def\itemize{\olditemize\parskip0pt}

\begin{document}

\title{The \textsf{childdoc} Package}
\hypersetup{pdftitle={The childdoc Package}}
\author{Niklas Beisert\\[2ex]
  Institut f\"ur Theoretische Physik\\
  Eidgen\"ossische Technische Hochschule Z\"urich\\
  Wolfgang-Pauli-Strasse 27, 8093 Z\"urich, Switzerland\\[1ex]
  \href{mailto:nbeisert@itp.phys.ethz.ch}
  {\texttt{nbeisert@itp.phys.ethz.ch}}}
\hypersetup{pdfauthor={Niklas Beisert}}
\hypersetup{pdfsubject={Manual for the LaTeX2e Package childdoc}}
\date{30 December 2018, \textsf{v2.0}}
\maketitle

\begin{abstract}\noindent
\textsf{childdoc} is a \LaTeXe{} package
that enables the direct compilation
of document sections included by |\include|
to individual files.
\end{abstract}

\begingroup
\parskip0ex
\tableofcontents
\endgroup

%%%%%%%%%%%%%%%%%%%%%%%%%%%%%%%%%%%%%%%%%%%%%%%%%%%%%%%%%%%%%%%%%%%%%%%%%%%%%%%%
%%%%%%%%%%%%%%%%%%%%%%%%%%%%%%%%%%%%%%%%%%%%%%%%%%%%%%%%%%%%%%%%%%%%%%%%%%%%%%%%
\section{Introduction}

\LaTeX{} provides a mechanism to structure a large document (such as a book)
into a main file and several child files (containing the chapters)
using the |\include| command.
This mechanism is beneficial for documents
which span hundreds of pages in order to
make the source file(s) more manageable.
Moreover, compilation can be restricted to
selected child files by means of the |\includeonly| command.
The latter feature can be used to reduce the compilation time while editing
(this was significantly more useful in the earlier days of \LaTeX{})
or to generate a smaller document which is easier to navigate.
Another application of |\includeonly| is to generate
documents consisting of selected parts of the complete document.

However, there are a few drawbacks of the plain |\include| mechanism:
\begin{itemize}
\item
The child files cannot be compiled on their own,
they can only be compiled via the main file.
A naive editing environment
(such as a text editor with an option
to have the current file processed by \LaTeX)
may require one to switch to the main file before compiling;
attempting to compile the child file produces errors.
\item
The main file must be modified (each time)
to adjust the |\includeonly| command
to the present needs. This easily leaves the main file in a messy state.
\item
The generated document will always carry the filename
of the main document. This is inconvenient if
several child files are to be compiled and
to be kept for distribution.
\end{itemize}

The present package provides a simple interface
to make child files individually compilable by \LaTeX{}.
Compiling a child file then has the same effect as compiling
the main file with an |\includeonly| command
to select the appropriate child.
Moreover the generated document will carry the name of the child
rather than the main file.
This resolves all three above issues.

This feature is meant to make the editing of books,
thesis documents and lecture notes somewhat more convenient.
However, the package can also be used efficiently for
composing a series of documents (such as exercise sheets)
which are typically distributed individually.
It then assists the author in generating the individual documents
(potentially in different versions)
as well as a document containing the collected series.
Another application is in developing style files
or other kinds of included material
where compilation of the style file could redirect
to a sample or test file.

%%%%%%%%%%%%%%%%%%%%%%%%%%%%%%%%%%%%%%%%%%%%%%%%%%%%%%%%%%%%%%%%%%%%%%%%%%%%%%%%
%%%%%%%%%%%%%%%%%%%%%%%%%%%%%%%%%%%%%%%%%%%%%%%%%%%%%%%%%%%%%%%%%%%%%%%%%%%%%%%%
\section{Usage}

First of all, the package \textsf{childdoc} is \emph{not} a standard
\LaTeXe{} |.sty| style file! Therefore it needs to be invoked in
a non-standard way.

%%%%%%%%%%%%%%%%%%%%%%%%%%%%%%%%%%%%%%%%%%%%%%%%%%%%%%%%%%%%%%%%%%%%%%%%%%%%%%%%
\subsection{Included Files}
\label{sec:include}

%%%%%%%%%%%%%%%%%%%%%%%%%%%%%%%%%%%%%%%%
\DescribeMacro{\childdocmain}
To use the package, add the commands
\begin{center}
\begin{tabular}{l}
|\input{childdoc.def}|\\
|\childdocmain{}|\\
\end{tabular}
\end{center}
at the very top of the main \LaTeX{} file,
in particular \emph{before} the |\documentclass| statement!
The argument of |\childdocmain| should be left empty
(but it must be present).

%%%%%%%%%%%%%%%%%%%%%%%%%%%%%%%%%%%%%%%%
\DescribeMacro{\childdocof}
Furthermore, add the commands
\begin{center}
\begin{tabular}{l}
|\input{childdoc.def}|\\
|\childdocof{|\textit{main}|}|\\
\end{tabular}
\end{center}
at the top of every child file \textit{child}
which is included by |\include{|\textit{child}|}|
from within the main file
(or at least for those files to be compiled individually).
The argument \textit{main} must be the filename of the main file.

There are a couple of
considerations in setting up the main and child documents:

%%%%%%%%%%%%%%%%%%%%%%%%%%%%%%%%%%%%%%%%
\paragraph{Restrictions.}

Please note the following restrictions:
\begin{itemize}
\item
|\childdocmain| must be called with one argument \textit{main}
to ensure compatibility with earlier version of the package.
It must either be empty (|\childdocmain{}|)
or precisely match the filename of the main file in which it is specified.
See \secref{sec:detection} for further information.
\item
The filename \textit{main} must be specified without the |.tex| extension.
\item
The filename \textit{main} is case sensitive
(even in case-insensitive file systems)
due to internal string comparison.
\item
The argument \textit{main} should be fully expanded, it cannot be a macro.
\item
Subdirectories and special characters should be avoided in filenames.
\item
The command |\childdocmain{|\textit{main}|}| must be followed by a whitespace.
It should not be followed immediately by another command
or by a comment mark `|%|'.
This is because the \TeX{} parser reads the token immediately following
the argument of |\childdocmain| and puts it
at the beginning of every child section;
however, a white\-space is ignored.
\end{itemize}

%%%%%%%%%%%%%%%%%%%%%%%%%%%%%%%%%%%%%%%%
\paragraph{Content of Main File.}

It is advisable to place all content in the child files included by |\include|.
Any output contained in the main file will appear in all child documents
unless suppressed manually;
it cannot be suppressed automatically by the |\includeonly| directive
and thus should normally be avoided.
A method to include some content in the main file
by means of conditional processing is described in \secref{sec:conditional}.

%%%%%%%%%%%%%%%%%%%%%%%%%%%%%%%%%%%%%%%%
\paragraph{Page Numbering.}

When only a part of the document is compiled,
the appropriate numbering of pages
(as well as other status parameters)
is determined from the |.aux| files.
The latter contain information from previous passes.
However this information needs to propagate through
all intermediate child documents.
Therefore the page numbering in child documents may well
be inconsistent until the complete document is compiled at least once.

A useful (if unconventional) way to always ensure a consistent
page numbering is to restart the numbering in each child document
and denote the pages by `\textit{child}|.|\textit{page}'
where \textit{child} represents the chapter/section number of the child file.
This can be achieved by the command
|\numberwithin{page}{|\textit{child}|}|
of the \textsf{amsmath} package
where \textit{child} can be |chapter| or |section|
depending on the chosen structuring.
Alternatively, one can modify the macro |\thepage| appropriately
and reset the counter |page| at the start of each child file.

%%%%%%%%%%%%%%%%%%%%%%%%%%%%%%%%%%%%%%%%%%%%%%%%%%%%%%%%%%%%%%%%%%%%%%%%%%%%%%%%
\subsection{Conditional Processing}
\label{sec:conditional}

The package provides a mechanism to compile different versions
of a document. To customise the versions further some conditional processing
can come in handy to distinguish which version is being compiled.
The package provides two macros to describe the compilation context:

%%%%%%%%%%%%%%%%%%%%%%%%%%%%%%%%%%%%%%%%
\DescribeMacro{\ifchilddoc}
The conditional |\ifchilddoc| distinguishes between the compilation of
child documents and the main document:
%
\begin{center}
|\ifchilddoc |\textit{child-code}| |[|\||else |\textit{main-code}]| \||fi|
\end{center}

%%%%%%%%%%%%%%%%%%%%%%%%%%%%%%%%%%%%%%%%
\DescribeMacro{\childdocname}
\DescribeMacro{\childdocjob}
The macro |\childdocname| contains the filename (without extension)
of the main or child file being processed.
Note that |\childdocjob| will always contain the name of the main file.

%%%%%%%%%%%%%%%%%%%%%%%%%%%%%%%%%%%%%%%%
\paragraph{Title Page.}

Conditional processing can be used to include a title or banner page
in the main document when proper precautions are taken.
Importantly, the code in the main file should ensure that the page counter
(as well as other status parameters which are stored in the |.aux| files)
takes the same value after the conditional processing.
Otherwise the page numbers may take divergent values
depending on which part is compiled.

For example, a title page could be declared by:
%
\begin{center}
\begin{tabular}{l}
|\ifchilddoc\||else|\\
|\addtocounter{page}{-1}|\\
\textit{code for title page}\\
|\newpage|\\
|\||fi|
\end{tabular}
\end{center}
%
A banner page for the child documents can be generated by:
%
\begin{center}
\begin{tabular}{l}
|\ifchilddoc|\\
|\addtocounter{page}{-1}|\\
\textit{code for banner page}\\
|\newpage|\\
|\||fi|
\end{tabular}
\end{center}
%
Here one could write a message such as:
\begin{center}
|This is the part \childdocname{} of \childdocjob{}.|
\end{center}

%%%%%%%%%%%%%%%%%%%%%%%%%%%%%%%%%%%%%%%%%%%%%%%%%%%%%%%%%%%%%%%%%%%%%%%%%%%%%%%%
\subsection{Flags}
\label{sec:flags}

The package makes it easy to generate different versions
of the main or child documents.
To this end compilation flags can be defined
and assigned different default values.
They will be particularly useful in conjunction
with the forwarding mechanism described in \secref{sec:forward}.

For example, it may be useful to have a flag |\version|
which can be set to |draft| or |final|.
The document source will contain some conditional code
depending on the value of |\version|.
Suppose further, the flag should default to |final| for the main file
and to |draft| for child files
which is a natural assignment for editing the document.
This is achieved by placing the following code
in the preamble of the main document
(below the |\childdocmain| directive):
%
\begin{center}
\begin{tabular}{l}
|\ifchilddoc|\\
|\providecommand{\version}{draft}|\\
|\||else|\\
|\providecommand{\version}{final}|\\
|\||fi|
\end{tabular}
\end{center}
%
The definition by |\providecommand| makes sure
that previous definitions are not overwritten.
Further statements |\providecommand{\version}{...}|
can thus be added before the above code to override it.

For the main file, one might add a line
(between |\childdocmain| and the above block)
%
\begin{center}
|%\ifchilddoc\||else\providecommand{\version}{draft}\||fi|
\end{center}
%
which can be uncommented to produce a draft version.
Likewise one can add a line to the very top of a child file
(above the |\childdocof{|\textit{main}|}| directive)
%
\begin{center}
|%\providecommand{\version}{final}|
\end{center}
%
which can be uncommented to produce the final version of this child document.

%%%%%%%%%%%%%%%%%%%%%%%%%%%%%%%%%%%%%%%%%%%%%%%%%%%%%%%%%%%%%%%%%%%%%%%%%%%%%%%%
\subsection{Forwarding}
\label{sec:forward}

Different versions of the main or child documents
using compilation flags as described in \secref{sec:flags}
can be (permanently) stored in different files
for convenient compilation, viewing and distribution.
To this end, the package defines a command
to pass on compilation to a different file:

%%%%%%%%%%%%%%%%%%%%%%%%%%%%%%%%%%%%%%%%
\DescribeMacro{\childdocforward}
The command |\childdocforward| redirects processing to
another source file:
%
\begin{center}
\begin{tabular}{l}
|\input{childdoc.def}|\\
|\childdocforward[|\textit{main}|]{|\textit{dest}|}|\\
\end{tabular}
\end{center}
%
The argument \textit{dest} is the destination file
(without extension).
It should be the main file or one of the child files.
Note that further \textsf{childdoc} directives
such as |\childdocof| and |\childdocforward|
in the indicated file will be processed in this form.
The optional argument \textit{main}
passes on directly to the main file \textit{main}
while pretending to compile the child \textit{dest}.
This form behaves as if \textit{dest}
issues |\childdocof{|\textit{main}|}| right away,
and no further \textsf{childdoc} directives will be processed.

%%%%%%%%%%%%%%%%%%%%%%%%%%%%%%%%%%%%%%%%
\DescribeMacro{\...prefix}
In the alternative form |\childdocforwardprefix|,
%
\begin{center}
\begin{tabular}{l}
|\input{childdoc.def}|\\
|\childdocforwardprefix[|\textit{main}|]{|\textit{prefix}|}{|\textit{dest}|}|
\end{tabular}
\end{center}
%
the destination file is determined by a pattern
depending on the current file:
To make this work, the current file must be called
`{\textit{prefix}\hspace{0.2em}\textit{suffix}}'
with \textit{prefix} matching precisely the argument.
Processing is then passed on to the file
`{\textit{dest}\hspace{0.2em}\textit{suffix}}'.
Surely, the same effect is achieved by
directly specifying the
argument `{\textit{dest}\hspace{0.2em}\textit{suffix}}'
in the first form.
However, that requires to set up a different file
for each child. With the alternative form of the command
all these files can have exactly the same content
which simplifies setting them up and maintaining them.

For example, the following file |draft.tex|
with a compilation flag |\version| as described in \secref{sec:flags}
compiles the main document as a draft:
%
\begin{center}
\begin{tabular}{l}
|\def\version{draft}|\\
|\input{childdoc.def}|\\
|\childdocforward{|\textit{main}|}|
\end{tabular}
\end{center}
%
Likewise, the following files |final|\textit{nn}|.tex|
compile the final version of the child document
|child|\textit{nn}|.tex|:
%
\begin{center}
\begin{tabular}{l}
|\def\version{final}|\\
|\input{childdoc.def}|\\
|\childdocforwardprefix{final}{child}|
\end{tabular}
\end{center}
%

Note that when several versions of a main file and/or of each child file
are to be generated, it may be convenient to set up a |Makefile| or
shell script to automatise the process.

%%%%%%%%%%%%%%%%%%%%%%%%%%%%%%%%%%%%%%%%%%%%%%%%%%%%%%%%%%%%%%%%%%%%%%%%%%%%%%%%
\subsection{Command Line Processing}
\label{sec:commandline}

The effect of redirection files can also be achieved by invoking
the \LaTeX{} compiler with a more elaborate command line.
Most conveniently this should be done as part
of a shell script or a |Makefile|.

When using \textsf{childdoc} in the main file, the following
command lines effectively perform a redirection
(note that depending on the shell being used,
backslashes may have to be doubled: `|\|' $\to$ `|\\|'):
%
\begin{center}
|... -jobname "|\textit{target}|" |\\|"|[\textit{flags}]%
|\input{childdoc.def}\childdocforward[|\textit{main}|]{|\textit{dest}|}"|
\end{center}
%
Here \textit{target} is the name of the output file,
\textit{main} is the name of the main file
and \textit{dest} is the name of the main or child file to be processed
(all filenames without extensions).
The optional argument \textit{main} can be omitted
if \textit{main} matches \textit{dest}.
Optionally, compilation \textit{flags} can be defined via |\def| commands.
This command line makes the \TeX{} engine believe
it is compiling the file \textit{target}
whose content is specified as the latter parameter.
The provided code then forwards the processing to
\textit{main} or \textit{dest} as described in \secref{sec:forward}.

%%%%%%%%%%%%%%%%%%%%%%%%%%%%%%%%%%%%%%%%%%%%%%%%%%%%%%%%%%%%%%%%%%%%%%%%%%%%%%%%
\subsection{Include by Input}
\label{sec:input}

Including child documents by |\include| has some restrictions by design.
Most notably, the content of a child document always occupies
its own set of pages; pages cannot be shared between child documents.
Usually, this behaviour makes perfect sense
because each child document contain an essential part of the document.
However, in some situations it may be desirable to compose
a document from a collection of parts
without having mandatory page breaks between then.
For this case, the package
provides a mechanism to include parts
by |\input| which can also be processed individually.
However, by construction this mechanism
requires manual handling of the content to be output.

%%%%%%%%%%%%%%%%%%%%%%%%%%%%%%%%%%%%%%%%
\DescribeMacro{\ifchilddocmanual}
The main file should be prepared as usual, see \secref{sec:include}.
However, the document body must make a distinction
between processing of an individual part and of the main document, e.g.:
%
\begin{center}
\begin{tabular}{l}
|\ifchilddocmanual|\\
|\input{\childdocname}|\\
|\||else|\\
\textit{document body with }|\input{|\textit{part}|}|\\
|\||fi|
\end{tabular}
\end{center}
%
The conditional |\ifchilddocmanual| is true whenever
a part to be included by |\input| is being compiled,
and the name of the part is stored in |\childdocname|.

%%%%%%%%%%%%%%%%%%%%%%%%%%%%%%%%%%%%%%%%
\DescribeMacro{\childdocby}
Each part to be included by |\input| should start with:
%
\begin{center}
\begin{tabular}{l}
|\input{childdoc.def}|\\
|\childdocby{|\textit{main}|}|\\
\end{tabular}
\end{center}
%
The directive |\childdocby| is similar to |\childdocof|
described in \secref{sec:include},
but the subsequent selection of content must be done manually.
To that end, both |\ifchilddoc| and |\ifchilddocmanual|
will be true upon processing of a part,
and the name of the part is stored in |\childdocname|.
Note that |\jobname| will be set to the filename of the current part
so that each part receives an individual |.aux| file
that does not interfere with the |.aux| file(s) of the main document.
This behaviour can be altered by the alternative form
|\childdocby[*]{|\textit{main}|}| (with a non-empty optional argument)
which uses the |.aux| file of the main document
by setting |\jobname| to \textit{main}.

%%%%%%%%%%%%%%%%%%%%%%%%%%%%%%%%%%%%%%%%%%%%%%%%%%%%%%%%%%%%%%%%%%%%%%%%%%%%%%%%
\subsection{Driver Development}
\label{sec:driver}

The \textsf{childdoc} mechanism can also be use for the development
of definition files such as \LaTeX{} styles or classes.
This case differs from the above setup with multiple parts
included by |\include| in that no |\includeonly| should be invoked.
This can be achieved by starting the include file
(before |\ProvidesPackage|) with:
%
\begin{center}
\begin{tabular}{l}
|\input{childdoc.def}|\\
|\childdocforward{|\textit{main}|}|\\
\end{tabular}
\end{center}
%
or alternatively with:
%
\begin{center}
\begin{tabular}{l}
|\input{childdoc.def}|\\
|\childdocby{|\textit{main}|}|\\
\end{tabular}
\end{center}
%
Both forms have slightly different effects as described above.
The main file is prepared as usual, see \secref{sec:include}.

%%%%%%%%%%%%%%%%%%%%%%%%%%%%%%%%%%%%%%%%%%%%%%%%%%%%%%%%%%%%%%%%%%%%%%%%%%%%%%%%
\subsection{Legacy Detection}
\label{sec:detection}

The directive |\childdocmain| in the main file can detect
whether the complete document or merely a child is to be compiled
even without using the directive |\childdocof|.
This method is deprecated because it is less robust
and there is no compelling reason to use it;
it is merely provided for backward compatibility
and it may be removed in future versions.

If the detection mechanism is to be used,
it is mandatory to correctly specify
the filename of the main file as the argument of |\childdocmain|:
%
\begin{center}
\begin{tabular}{l}
|\input{childdoc.def}|\\
|\childdocmain{|\textit{main}|}|\\
\end{tabular}
\end{center}
%
If |\jobname| does not match the argument \textit{main} of |\childdocmain|,
it is assumed that |\jobname| points to the child file to be compiled.
When using |\childdocmain| with the main file specified as argument,
it suffices to start a child file
with just |\input{|\textit{main}|}|
without loading of the package and using |\childdocof|.
If instead all processing is done
with the appropriate \textsf{childdoc} directives,
the argument of \textit{main} of |\childdocmain| can be empty.

An alternative version of the command line processing described
in \secref{sec:commandline} using the detection mechanism reads:
%
\begin{center}
|... -jobname "|\textit{target}|" "|[\textit{flags}]%
[|\def\jobname{|\textit{dest}|}|]|\input{|\textit{main}|}"|
\end{center}

%%%%%%%%%%%%%%%%%%%%%%%%%%%%%%%%%%%%%%%%%%%%%%%%%%%%%%%%%%%%%%%%%%%%%%%%%%%%%%%%
\subsection{Manual Code}
\label{sec:manual}

In case one cannot be certain whether the definitions file |childdoc.def|
is installed on the target \TeX{} distribution
and one prefers not to ship it,
it is conceivable to paste a few relevant commands into the sources.

To that end, drop all statements |\input{childdoc.def}|
and perform the replacements as outlined below.
Instead of |\childdocmain{|\textit{main}|}| add the following code
to the top of the main file:
%
\begin{center}
\begin{tabular}{l}
|\||ifdefined\childdocname\endinput\||fi\newif\ifchilddoc|\\
|\edef\childdocname{\scantokens\expandafter{\jobname\noexpand}}|\\
|\def\childdocmain{|\textit{main}|}\||ifx\childdocmain\childdocname\||else|\\
|\childdoctrue\includeonly{\childdocname}\let\jobname\childdocmain\||fi|\\
\end{tabular}
\end{center}
%
Instead of |\childdocof{|\textit{main}|}| just include the main file
at the top of each child file:
%
\begin{center}
|\input{|\textit{main}|}|
\end{center}
%
A simple redirection |\childdocforward{|\textit{dest}|}| is achieved by:
%
\begin{center}
|\def\jobname{|\textit{dest}|}\input{\jobname}|
\end{center}
%
The redirection with prefix
|\childdocforwardprefix[|\textit{prefix}|]{|\textit{dest}|}|
is accomplished by:
%
\begin{center}
\begin{tabular}{l}
|{\edef\jobname{\scantokens\expandafter{\jobname\noexpand}}|\\
|\def\redirectjob |\textit{prefix}|#1~~~{\gdef\jobname{|\textit{dest}|#1}}|\\
|\expandafter\redirectjob\jobname~~~}\input{\jobname}|
\end{tabular}
\end{center}

In an alternative approach,
child documents can be compiled by a specific command line
without additional code or specific definitions:
%
\begin{center}
|... -jobname "|\textit{target}|" "|[\textit{flags}]%
|\includeonly{|\textit{dest}|}\input{|\textit{main}|}"|
\end{center}
%

%%%%%%%%%%%%%%%%%%%%%%%%%%%%%%%%%%%%%%%%%%%%%%%%%%%%%%%%%%%%%%%%%%%%%%%%%%%%%%%%
%%%%%%%%%%%%%%%%%%%%%%%%%%%%%%%%%%%%%%%%%%%%%%%%%%%%%%%%%%%%%%%%%%%%%%%%%%%%%%%%
\section{Information}

%%%%%%%%%%%%%%%%%%%%%%%%%%%%%%%%%%%%%%%%%%%%%%%%%%%%%%%%%%%%%%%%%%%%%%%%%%%%%%%%
\subsection{Copyright}

Copyright \copyright{} 2017--2018 Niklas Beisert

This work may be distributed and/or modified under the
conditions of the \LaTeX{} Project Public License, either version 1.3
of this license or (at your option) any later version.
The latest version of this license is in
  \url{http://www.latex-project.org/lppl.txt}
and version 1.3 or later is part of all distributions of \LaTeX{}
version 2005/12/01 or later.

This work has the LPPL maintenance status `maintained'.

The Current Maintainer of this work is Niklas Beisert.

This work consists of the files |README.txt|, |childdoc.ins| and |childdoc.dtx|
as well as the derived files |childdoc.def|, |cdocsamp.tex|
with |cdocsch1.tex|, |cdocsch2.tex|, |cdocspt3.tex|, |cdocspt4.tex|,
|cdocsdrf.tex|, |cdocsfn1.tex|, |cdocsfn2.tex|
as well as |childdoc.pdf|.

%%%%%%%%%%%%%%%%%%%%%%%%%%%%%%%%%%%%%%%%%%%%%%%%%%%%%%%%%%%%%%%%%%%%%%%%%%%%%%%%
\subsection{Files and Installation}

The package consists of the files:
%
\begin{center}
\begin{tabular}{ll}
    |README.txt|   & readme file \\
    |childdoc.ins| & installation file \\
    |childdoc.dtx| & source file \\
    |childdoc.def| & definition file \\
    |cdocsamp.tex| & sample main file \\
    |cdocsch1.tex| & sample include file \\
    |cdocsch2.tex| & sample include file \\
    |cdocspt3.tex| & sample part file \\
    |cdocspt4.tex| & sample part file \\
    |cdocsdrf.tex| & sample redirection file \\
    |cdocsfn1.tex| & sample redirection file \\
    |cdocsfn2.tex| & sample redirection file \\
    |childdoc.pdf| & manual
\end{tabular}
\end{center}
%
The distribution consists of the files
|README.txt|, |childdoc.ins| and |childdoc.dtx|.
%
\begin{itemize}
\item
Run (pdf)\LaTeX{} on |childdoc.dtx|
to compile the manual |childdoc.pdf| (this file).
\item
Run \LaTeX{} on |childdoc.ins| to create the definitions file |childdoc.def|
and the sample |cdocsamp.tex| with include files
|cdocsch1.tex|, |cdocsch2.tex|, |cdocspt3.tex|, |cdocspt4.tex|,
|cdocsdrf.tex|, |cdocsfn1.tex|, |cdocsfn2.tex|.
Then copy the file |childdoc.def| to an appropriate directory of your \LaTeX{}
distribution, e.g.\ \textit{texmf-root}|/tex/latex/childdoc|.
\end{itemize}

%%%%%%%%%%%%%%%%%%%%%%%%%%%%%%%%%%%%%%%%%%%%%%%%%%%%%%%%%%%%%%%%%%%%%%%%%%%%%%%%
\subsection{Related CTAN Packages}

There are several other packages which offer a similar functionality:
%
\begin{itemize}
\item
The packages
\href{http://ctan.org/pkg/docmute}{\textsf{docmute}},
\href{http://ctan.org/pkg/includex}{\textsf{includex}} and
\href{http://ctan.org/pkg/standalone}{\textsf{standalone}}
provide commands to include only the document body of
a child file thus allowing both files to be compiled individually.
\item
The packages \href{http://ctan.org/pkg/subdocs}{\textsf{subdocs}}
and \href{http://ctan.org/pkg/subfiles}{\textsf{subfiles}}
provide structures in which the main and child documents can be
encapsulated and allowing them to be compiled individually.
The inclusion mechanism is different from the conventional |\include|.
\item
The package \href{http://ctan.org/pkg/combine}{\textsf{combine}}
is an elaborate solution to combine several documents into one.
\end{itemize}
%
See also the CTAN topic \href{http://ctan.org/topic/subdocs}{\textsf{subdocs}}
for further related packages.
The present package differs from the above solutions in that
a document structure constructed with the conventional |\include| mechanism
just needs two extra commands at the top of every file
such that all constituent files can be compiled individually.

%%%%%%%%%%%%%%%%%%%%%%%%%%%%%%%%%%%%%%%%%%%%%%%%%%%%%%%%%%%%%%%%%%%%%%%%%%%%%%%%
%\subsection{Feature Suggestions}
%
%The following is a list of features which may be useful for future
%versions of this package:
%%
%\begin{itemize}
%\item
%\ldots
%\end{itemize}

%%%%%%%%%%%%%%%%%%%%%%%%%%%%%%%%%%%%%%%%%%%%%%%%%%%%%%%%%%%%%%%%%%%%%%%%%%%%%%%%
\subsection{Revision History}

%%%%%%%%%%%%%%%%%%%%%%%%%%%%%%%%%%%%%%%%
\paragraph{v2.0:} 2018/12/30

\begin{itemize}
\item
immediate forward processing
\item
added |\childdocby| mechanism
\item
manual restructured
\end{itemize}

%%%%%%%%%%%%%%%%%%%%%%%%%%%%%%%%%%%%%%%%
\paragraph{v1.6:} 2018/01/17

\begin{itemize}
\item
application for development of include files
\item
corrections to manual
\end{itemize}

%%%%%%%%%%%%%%%%%%%%%%%%%%%%%%%%%%%%%%%%
\paragraph{v1.5:} 2017/05/21

\begin{itemize}
\item
more complete structuring introduced
\item
|\childdocof| introduced
\item
|\childdoc| renamed to |\childdocmain|
\item
|\childredirect| renamed to |\childdocforward| and |\childdocforwardprefix|
and functionality expanded
\end{itemize}

%%%%%%%%%%%%%%%%%%%%%%%%%%%%%%%%%%%%%%%%
\paragraph{v1.0:} 2017/04/27

\begin{itemize}
\item
manual and install package
\item
first version published on CTAN
\end{itemize}

%%%%%%%%%%%%%%%%%%%%%%%%%%%%%%%%%%%%%%%%
\paragraph{v0.6:} 2017/04/26

\begin{itemize}
\item
redirection mechanism added
\end{itemize}

%%%%%%%%%%%%%%%%%%%%%%%%%%%%%%%%%%%%%%%%
\paragraph{v0.5:} 2017/04/26

\begin{itemize}
\item
functionality in definition file
\end{itemize}


%%%%%%%%%%%%%%%%%%%%%%%%%%%%%%%%%%%%%%%%%%%%%%%%%%%%%%%%%%%%%%%%%%%%%%%%%%%%%%%%
%%%%%%%%%%%%%%%%%%%%%%%%%%%%%%%%%%%%%%%%%%%%%%%%%%%%%%%%%%%%%%%%%%%%%%%%%%%%%%%%
%%%%%%%%%%%%%%%%%%%%%%%%%%%%%%%%%%%%%%%%%%%%%%%%%%%%%%%%%%%%%%%%%%%%%%%%%%%%%%%%
\appendix

\settowidth\MacroIndent{\rmfamily\scriptsize 000\ }

 \DocInput{childdoc.dtx}

\end{document}
%</driver>
% \fi
%
% %%%%%%%%%%%%%%%%%%%%%%%%%%%%%%%%%%%%%%%%%%%%%%%%%%%%%%%%%%%%%%%%%%%%%%%%%%%%%%
% %%%%%%%%%%%%%%%%%%%%%%%%%%%%%%%%%%%%%%%%%%%%%%%%%%%%%%%%%%%%%%%%%%%%%%%%%%%%%%
% \section{Sample}
%\iffalse
%<*samplemain>
%\fi
%
% The following presents a sample document
% with two chapters, two parts, a title page,
% a compile flag as well as three forwarding files to set the flag.
% It consists of eight |.tex| files:
% \begin{center}
% \begin{tabular}{ll}
% |cdocsamp.tex|&main file\\
% |cdocsch1.tex|&include file for chapter 1\\
% |cdocsch2.tex|&include file for chapter 2\\
% |cdocspt3.tex|&include file for part 3\\
% |cdocspt4.tex|&include file for part 4\\
% |cdocsdrf.tex|&forwarding file for main file in draft mode\\
% |cdocsfi1.tex|&forwarding file for final version of chapter 1\\
% |cdocsfi2.tex|&forwarding file for final version of chapter 2\\
% \end{tabular}
% \end{center}
% Each of the eight files can be compiled directly by the \LaTeX{} compiler.
%
% %%%%%%%%%%%%%%%%%%%%%%%%%%%%%%%%%%%%%%
% \paragraph{Main File.}
%
% The main file is called |cdocsamp.tex|.
%
% Load the \textsf{childdoc} definitions and
% declare the filename for the main document:
%    \begin{macrocode}
\input{childdoc.def}
\childdocmain{}
%    \end{macrocode}

% Optional override for |\version| flag:
%    \begin{macrocode}
%%\ifchilddoc\else\providecommand{\version}{draft}\fi
%    \end{macrocode}

% Define the default values for the |\version| flag
% (|final| for the main file and |draft| for childs):
%    \begin{macrocode}
\ifchilddoc
\providecommand{\version}{draft}
\else
\providecommand{\version}{final}
\fi
%    \end{macrocode}

% Load the standard document class:
%    \begin{macrocode}
\documentclass[12pt]{article}
%    \end{macrocode}

% Start the document body:
%    \begin{macrocode}
\begin{document}
%    \end{macrocode}

% Declare a title page.
% Print title, part of document being processed and version flag:
%    \begin{macrocode}
\addtocounter{page}{-1}
\begin{center}
{\LARGE\bfseries{}childdoc example\par}
\vspace{1cm}
\ifchilddoc
\ifchilddocmanual part\else chapter\fi:
`\childdocname' of `\childdocjob'\par
\else
main document: `\childdocjob'\par
\fi
version: \version\par
\end{center}
\newpage
%    \end{macrocode}

% Manually include selected file,
% otherwise process as usual:
%    \begin{macrocode}
\ifchilddocmanual
\section*{part `\childdocname'}
\input{\childdocname}
\else
%    \end{macrocode}

% Include the two chapters:
%    \begin{macrocode}
\include{cdocsch1}
\include{cdocsch2}
%    \end{macrocode}

% Include the two parts unless only chapters should be displayed:
%    \begin{macrocode}
\ifchilddoc\else
\section{part three}
\input{cdocspt3}
\section{part four}
\input{cdocspt4}
\fi
%    \end{macrocode}

% Process as usual until here:
%    \begin{macrocode}
\fi
%    \end{macrocode}

% End of document body:
%    \begin{macrocode}
\end{document}
%    \end{macrocode}
%\iffalse
%</samplemain>
%\fi
%
% %%%%%%%%%%%%%%%%%%%%%%%%%%%%%%%%%%%%%%
% \paragraph{Chapter Include Files.}
%
% The include files are called |cdocsch1.tex| and |cdocsch2.tex|.
%
%\iffalse
%<*samplechap1|samplechap2>
%\fi

% Optional override for |\version| flag:
%    \begin{macrocode}
%%\providecommand{\version}{final}
%    \end{macrocode}

% Include the main document:
%    \begin{macrocode}
\input{childdoc.def}
\childdocof{cdocsamp}
%    \end{macrocode}

%\iffalse
%</samplechap1|samplechap2>
%\fi
%
%\iffalse
%<*samplechap1>
%\fi
% Some text for chapter 1:
%    \begin{macrocode}
\section{one}
some text in chapter one
%    \end{macrocode}

%\iffalse
%</samplechap1>
%\fi
% Some text for chapter 2:
%\iffalse
%<*samplechap2>
%\fi
%    \begin{macrocode}
\section{two}
more text in chapter two
%    \end{macrocode}

%\iffalse
%</samplechap2>
%\fi
%
% %%%%%%%%%%%%%%%%%%%%%%%%%%%%%%%%%%%%%%
% \paragraph{Part Include Files.}
%
% The include files are called |cdocspt3.tex| and |cdocspt4.tex|.
%
%\iffalse
%<*samplepart3|samplepart4>
%\fi

% Optional override for |\version| flag:
%    \begin{macrocode}
%%\providecommand{\version}{final}
%    \end{macrocode}

% Include the main document:
%    \begin{macrocode}
\input{childdoc.def}
\childdocby{cdocsamp}
%    \end{macrocode}

%\iffalse
%</samplepart3|samplepart4>
%\fi
%
%\iffalse
%<*samplepart3>
%\fi
% Some text for part 3:
%    \begin{macrocode}
some text in part three
%    \end{macrocode}

%\iffalse
%</samplepart3>
%\fi
% Some text for part 4:
%\iffalse
%<*samplepart4>
%\fi
%    \begin{macrocode}
more text in part four
%    \end{macrocode}

%\iffalse
%</samplepart4>
%\fi
%
% %%%%%%%%%%%%%%%%%%%%%%%%%%%%%%%%%%%%%%
% \paragraph{Forwarding for a Complete Draft.}
%
% The following forwarding file |cdocsdrf.tex|
% compiles the main document in draft mode:
%\iffalse
%<*sampledraft>
%\fi
%    \begin{macrocode}
\def\version{draft}
\input{childdoc.def}
\childdocforward{cdocsamp}
%    \end{macrocode}

%\iffalse
%</sampledraft>
%\fi
%
% %%%%%%%%%%%%%%%%%%%%%%%%%%%%%%%%%%%%%%
% \paragraph{Forwarding for Final Version of the Chapters.}
%
% The following forwarding files |cdocsfn1.tex| and |cdocsfn2.tex|
% (with identical content)
% compile the final versions of the child documents
% |cdocsch1.tex| and |cdocsch2.tex|, respectively:
%\iffalse
%<*samplefinal>
%\fi
%    \begin{macrocode}
\def\version{final}
\input{childdoc.def}
\childdocforwardprefix[cdocsamp]{cdocsfn}{cdocsch}
%    \end{macrocode}

%\iffalse
%</samplefinal>
%\fi
%
% %%%%%%%%%%%%%%%%%%%%%%%%%%%%%%%%%%%%%%
% \paragraph{Command Line Processing.}
%
% The following three command lines generate the output files
% |cdocscld|, |cdocscl1| and |cdocscl2|
% which should be identical to
% |cdocsdrf|, |cdocsch1| and |cdocsfn2|, respectively:
% \begin{center}
% \begin{tabular}{l}
% |latex -jobname cdocscld \|\\
% |  "\def\version{draft}\input{childdoc.def}\childdocforward{cdocsamp}"|\\
% |latex -jobname cdocscl1 \|\\
% |  "\input{childdoc.def}\childdocforward[cdocsamp]{cdocsch1}"|\\
% |latex -jobname cdocscl2 \|\\
% |  "\def\version{final}\input{childdoc.def}\childdocforward{cdocsch2}"|
% \end{tabular}
% \end{center}
% Note that the trailing backslash on each first line
% merely continues the input to the second line
% (for convenient cut ant paste).
% Furthermore, the command |latex| can be replaced by any
% of its alternative versions such as |pdflatex|.
%
% %%%%%%%%%%%%%%%%%%%%%%%%%%%%%%%%%%%%%%%%%%%%%%%%%%%%%%%%%%%%%%%%%%%%%%%%%%%%%%
% %%%%%%%%%%%%%%%%%%%%%%%%%%%%%%%%%%%%%%%%%%%%%%%%%%%%%%%%%%%%%%%%%%%%%%%%%%%%%%
% \section{Implementation}
%\iffalse
%<*package>
%\fi
%
% This section describes the definitions file |childdoc.def|.

% The definitions cannot be loaded using |\usepackage| or |\RequirePackage|
% which has a mechanism to prevent loading a style file more than once.
% When loading the definitions by means of |\input|
% multiple instances have to be prevented manually:
%\iffalse
%This code needs to be before the `\ProvidesFile' directive
%which is defined at the beginning of this file.
%Therefore it is also placed there and commented out here.
%</package>
%<*discard>
%\fi
%    \begin{macrocode}
\ifdefined\childdocmain\endinput\fi
%    \end{macrocode}
%\iffalse
%</discard>
%<*package>
%\fi
%
% \macro{\ifchilddoc}
% \macro{\ifchilddocmanual}
% The conditional |\ifchilddoc| tells whether a
% child (true) or main (false) document is being compiled.
% The conditional |\ifchilddocmanual| tells whether
% the |\includeonly| mechanism is used (false) or
% the selection of child files must be performed manually (true).
% The definitions initialise to false:
%    \begin{macrocode}
\newif\ifchilddoc
\newif\ifchilddocmanual
%    \end{macrocode}

% \macro{\childdocname}
% \macro{\childdocjob}
% The macro |\childdocname| stores the name of the main document
% to be compiled. The macro |\childdocjob| stores the name of
% the document on which the \LaTeX{} compiler was originally invoked.
% The content of |\jobname| cannot be compared
% to filenames specified in the source due to different catcodes.
% The following code rescans |\jobname|, stores the result
% in |\childdocname| and saves a copy in |\childdocjob|:
%    \begin{macrocode}
\edef\childdocname{\scantokens\expandafter{\jobname\noexpand}}
\let\childdocjob\childdocname
%    \end{macrocode}

% \macro{\childdocdisable}
% The macro |\childdocdisable| prevents the main file
% from being processed more than once.
% At this stage, the main document command |\childdocmain|
% is assumed to be called once again where it should do nothing.
% Any subsequent call to it should prevent
% a secondary processing of the main document
% It overwrites the forwarding commands
% |\childdocof| and |\childdocforward|
% with empty macros to prevent further inclusions of the main document:
%    \begin{macrocode}
\newcommand{\childdocdisable}
{
  \renewcommand{\childdocmain}[1]{\renewcommand{\childdocmain}[1]{\endinput}}
  \renewcommand{\childdocof}[1]{}
  \renewcommand{\childdocby}[2][]{}
  \renewcommand{\childdocforward}[2][]{}
  \renewcommand{\childdocdisable}{}
}
%    \end{macrocode}

% \macro{\childdocmain}
% The macro |\childdocmain| is to be called at the top of the main file
% with nothing or the main filename (without extension) as argument.
% First, it breaks loops.
% If the argument is not empty and does not match |\childdocname|
% (which is set by the first inclusion of |childdoc.def|),
% |\ifchilddoc| is set to true, |\includeonly| is applied to the child file
% and |\jobname| is set to the main file
% (for proper handling of |.aux| files):
%    \begin{macrocode}
\newcommand{\childdocmain}[1]
{
  \childdocdisable\childdocmain{}
  \if?#1?\else
    \begingroup
      \def\childdoctmp{#1}
      \ifx\childdoctmp\childdocname
        \def\childdoctmp{}
      \else
        \def\childdoctmp
        {
          \childdoctrue
          \includeonly{\childdocname}
          \def\childdocjob{#1}
          \def\jobname{#1}
        }
      \fi
      \expandafter
    \endgroup
    \childdoctmp
  \fi
}
%    \end{macrocode}

% \macro{\childdocof}
% The command |\childdocof| redirects
% compilation to the main file |#1|.
%    \begin{macrocode}
\newcommand{\childdocof}[1]
{
  \childdocdisable
  \childdoctrue
  \includeonly{\childdocname}
  \def\jobname{#1}
  \def\childdocjob{#1}
  \input{#1}
}
%    \end{macrocode}

% \macro{\childdocby}
% The command |\childdocby| ....
%    \begin{macrocode}
\newcommand{\childdocby}[2][]
{
  \childdocdisable
  \childdoctrue
  \childdocmanualtrue
  \if?#1?\else
    \def\jobname{#2}
  \fi
  \def\childdocjob{#2}
  \input{#2}
  \endinput
}
%    \end{macrocode}

% \macro{\childdocforward}
% The command |\childdocforward| redirects
% compilation to the main file or
% (if the optional argument is given) a child file.
% Parameters are set as if the main file
% or a child file starting with |\childdocof| was compiled.
% Then compilation is handed over to the main file:
%    \begin{macrocode}
\newcommand{\childdocforward}[2][]
{
  \begingroup
    \if?#1?
      \def\childdoctmp
      {
        \def\childdocname{#2}
        \def\childdocjob{#2}
        \def\jobname{#2}
        \input{#2}
        \endinput
      }
    \else
      \def\childdoctmp
      {
        \childdocdisable
        \def\childdocname{#2}
        \childdoctrue
        \includeonly{#2}
        \def\childdocjob{#1}
        \def\jobname{#1}
        \input{#1}
        \endinput
      }
    \fi
    \expandafter
  \endgroup
  \childdoctmp
}
%    \end{macrocode}

% \macro{\childdocforwardprefix}
% The command |\childdocforwardprefix| redirects
% compilation to the main or a child file by means of a pattern.
% The prefix |#1| in the current filename is replaced by |#2|
% and the suffix of the current filename is kept
% (it is assumed that the filename does not contain the substring `|~~~|'
% which is used as a delimiter).
% Compilation is handed over to the new file by |\childdocforward|:
%    \begin{macrocode}
\newcommand{\childdocforwardprefix}[3][]
{
  \begingroup
    \def\childdocextract #2##1~~~{\def\childdoctmp{\childdocforward[#1]{#3##1}}}
    \expandafter\childdocextract\childdocname~~~
    \expandafter
  \endgroup
  \childdoctmp
}
%    \end{macrocode}

% \macro{\childdoc}
% The deprecated macro |\childdoc| is a legacy version of |\childdocmain|:
%    \begin{macrocode}
\newcommand{\childdoc}{\childdocmain}
%    \end{macrocode}

% \macro{\childdocredirect}
% The deprecated macro |\childdocredirect| is a legacy version
% of |\childdocforward| and |\childdocforwardprefix|:
%    \begin{macrocode}
\newcommand{\childdocredirect}[2][]
{
  \begingroup
    \if?#1?
      \def\childdoctmp{\childdocforward{#2}}
    \else
      \def\childdoctmp{\childdocforwardprefix{#1}{#2}}
    \fi
    \expandafter
  \endgroup
  \childdoctmp
}
%    \end{macrocode}

%\iffalse
%</package>
%\fi
%
\endinput
|\\
|\childdocforward[|\textit{main}|]{|\textit{dest}|}|\\
\end{tabular}
\end{center}
%
The argument \textit{dest} is the destination file
(without extension).
It should be the main file or one of the child files.
Note that further \textsf{childdoc} directives
such as |\childdocof| and |\childdocforward|
in the indicated file will be processed in this form.
The optional argument \textit{main}
passes on directly to the main file \textit{main}
while pretending to compile the child \textit{dest}.
This form behaves as if \textit{dest}
issues |\childdocof{|\textit{main}|}| right away,
and no further \textsf{childdoc} directives will be processed.

%%%%%%%%%%%%%%%%%%%%%%%%%%%%%%%%%%%%%%%%
\DescribeMacro{\...prefix}
In the alternative form |\childdocforwardprefix|,
%
\begin{center}
\begin{tabular}{l}
|% \iffalse
%
% childdoc.dtx Copyright (C) 2017-2018 Niklas Beisert
%
% This work may be distributed and/or modified under the
% conditions of the LaTeX Project Public License, either version 1.3
% of this license or (at your option) any later version.
% The latest version of this license is in
%   http://www.latex-project.org/lppl.txt
% and version 1.3 or later is part of all distributions of LaTeX
% version 2005/12/01 or later.
%
% This work has the LPPL maintenance status `maintained'.
%
% The Current Maintainer of this work is Niklas Beisert.
%
% This work consists of the files childdoc.dtx and childdoc.ins
% and the derived files childdoc.def and cdocsamp.tex with
% cdocsch1.tex, cdocsch2.tex, cdocsdrf.tex, cdocsfn1.tex, cdocsfn2.tex.
%
%<package>\ifdefined\childdocmain\endinput\fi
%<package>\ProvidesFile{childdoc.def}[2018/12/30 v2.0 child document driver]
%<samplemain>\ProvidesFile{cdocsamp.tex}[2018/12/30 v2.0 sample for childdoc]
%<*driver>
%\ProvidesFile{childdoc.drv}[2018/12/30 v2.0 childdoc reference manual file]
\PassOptionsToClass{10pt,a4paper}{article}
\documentclass{ltxdoc}

\usepackage[margin=35mm]{geometry}
\usepackage{hyperref}
\usepackage{hyperxmp}
\usepackage[usenames]{color}

\hypersetup{colorlinks=true}
\hypersetup{pdfstartview=FitH}
\hypersetup{pdfpagemode=UseNone}
\hypersetup{pdfsource={}}
\hypersetup{pdflang={en-UK}}
\hypersetup{pdfcopyright={Copyright 2017-2018 Niklas Beisert.
  This work may be distributed and/or modified under the
  conditions of the LaTeX Project Public License, either version 1.3
  of this license or (at your option) any later version.}}
\hypersetup{pdflicenseurl={http://www.latex-project.org/lppl.txt}}
\hypersetup{pdfcontactaddress={ETH Zurich, ITP, HIT K,
  Wolfgang-Pauli-Strasse 27}}
\hypersetup{pdfcontactpostcode={8093}}
\hypersetup{pdfcontactcity={Zurich}}
\hypersetup{pdfcontactcountry={Switzerland}}
\hypersetup{pdfcontactemail={nbeisert@itp.phys.ethz.ch}}
\hypersetup{pdfcontacturl={http://people.phys.ethz.ch/\xmptilde nbeisert/}}

\newcommand{\secref}[1]{\hyperref[#1]{section \ref*{#1}}}

\parskip1ex
\parindent0pt
\let\olditemize\itemize
\def\itemize{\olditemize\parskip0pt}

\begin{document}

\title{The \textsf{childdoc} Package}
\hypersetup{pdftitle={The childdoc Package}}
\author{Niklas Beisert\\[2ex]
  Institut f\"ur Theoretische Physik\\
  Eidgen\"ossische Technische Hochschule Z\"urich\\
  Wolfgang-Pauli-Strasse 27, 8093 Z\"urich, Switzerland\\[1ex]
  \href{mailto:nbeisert@itp.phys.ethz.ch}
  {\texttt{nbeisert@itp.phys.ethz.ch}}}
\hypersetup{pdfauthor={Niklas Beisert}}
\hypersetup{pdfsubject={Manual for the LaTeX2e Package childdoc}}
\date{30 December 2018, \textsf{v2.0}}
\maketitle

\begin{abstract}\noindent
\textsf{childdoc} is a \LaTeXe{} package
that enables the direct compilation
of document sections included by |\include|
to individual files.
\end{abstract}

\begingroup
\parskip0ex
\tableofcontents
\endgroup

%%%%%%%%%%%%%%%%%%%%%%%%%%%%%%%%%%%%%%%%%%%%%%%%%%%%%%%%%%%%%%%%%%%%%%%%%%%%%%%%
%%%%%%%%%%%%%%%%%%%%%%%%%%%%%%%%%%%%%%%%%%%%%%%%%%%%%%%%%%%%%%%%%%%%%%%%%%%%%%%%
\section{Introduction}

\LaTeX{} provides a mechanism to structure a large document (such as a book)
into a main file and several child files (containing the chapters)
using the |\include| command.
This mechanism is beneficial for documents
which span hundreds of pages in order to
make the source file(s) more manageable.
Moreover, compilation can be restricted to
selected child files by means of the |\includeonly| command.
The latter feature can be used to reduce the compilation time while editing
(this was significantly more useful in the earlier days of \LaTeX{})
or to generate a smaller document which is easier to navigate.
Another application of |\includeonly| is to generate
documents consisting of selected parts of the complete document.

However, there are a few drawbacks of the plain |\include| mechanism:
\begin{itemize}
\item
The child files cannot be compiled on their own,
they can only be compiled via the main file.
A naive editing environment
(such as a text editor with an option
to have the current file processed by \LaTeX)
may require one to switch to the main file before compiling;
attempting to compile the child file produces errors.
\item
The main file must be modified (each time)
to adjust the |\includeonly| command
to the present needs. This easily leaves the main file in a messy state.
\item
The generated document will always carry the filename
of the main document. This is inconvenient if
several child files are to be compiled and
to be kept for distribution.
\end{itemize}

The present package provides a simple interface
to make child files individually compilable by \LaTeX{}.
Compiling a child file then has the same effect as compiling
the main file with an |\includeonly| command
to select the appropriate child.
Moreover the generated document will carry the name of the child
rather than the main file.
This resolves all three above issues.

This feature is meant to make the editing of books,
thesis documents and lecture notes somewhat more convenient.
However, the package can also be used efficiently for
composing a series of documents (such as exercise sheets)
which are typically distributed individually.
It then assists the author in generating the individual documents
(potentially in different versions)
as well as a document containing the collected series.
Another application is in developing style files
or other kinds of included material
where compilation of the style file could redirect
to a sample or test file.

%%%%%%%%%%%%%%%%%%%%%%%%%%%%%%%%%%%%%%%%%%%%%%%%%%%%%%%%%%%%%%%%%%%%%%%%%%%%%%%%
%%%%%%%%%%%%%%%%%%%%%%%%%%%%%%%%%%%%%%%%%%%%%%%%%%%%%%%%%%%%%%%%%%%%%%%%%%%%%%%%
\section{Usage}

First of all, the package \textsf{childdoc} is \emph{not} a standard
\LaTeXe{} |.sty| style file! Therefore it needs to be invoked in
a non-standard way.

%%%%%%%%%%%%%%%%%%%%%%%%%%%%%%%%%%%%%%%%%%%%%%%%%%%%%%%%%%%%%%%%%%%%%%%%%%%%%%%%
\subsection{Included Files}
\label{sec:include}

%%%%%%%%%%%%%%%%%%%%%%%%%%%%%%%%%%%%%%%%
\DescribeMacro{\childdocmain}
To use the package, add the commands
\begin{center}
\begin{tabular}{l}
|\input{childdoc.def}|\\
|\childdocmain{}|\\
\end{tabular}
\end{center}
at the very top of the main \LaTeX{} file,
in particular \emph{before} the |\documentclass| statement!
The argument of |\childdocmain| should be left empty
(but it must be present).

%%%%%%%%%%%%%%%%%%%%%%%%%%%%%%%%%%%%%%%%
\DescribeMacro{\childdocof}
Furthermore, add the commands
\begin{center}
\begin{tabular}{l}
|\input{childdoc.def}|\\
|\childdocof{|\textit{main}|}|\\
\end{tabular}
\end{center}
at the top of every child file \textit{child}
which is included by |\include{|\textit{child}|}|
from within the main file
(or at least for those files to be compiled individually).
The argument \textit{main} must be the filename of the main file.

There are a couple of
considerations in setting up the main and child documents:

%%%%%%%%%%%%%%%%%%%%%%%%%%%%%%%%%%%%%%%%
\paragraph{Restrictions.}

Please note the following restrictions:
\begin{itemize}
\item
|\childdocmain| must be called with one argument \textit{main}
to ensure compatibility with earlier version of the package.
It must either be empty (|\childdocmain{}|)
or precisely match the filename of the main file in which it is specified.
See \secref{sec:detection} for further information.
\item
The filename \textit{main} must be specified without the |.tex| extension.
\item
The filename \textit{main} is case sensitive
(even in case-insensitive file systems)
due to internal string comparison.
\item
The argument \textit{main} should be fully expanded, it cannot be a macro.
\item
Subdirectories and special characters should be avoided in filenames.
\item
The command |\childdocmain{|\textit{main}|}| must be followed by a whitespace.
It should not be followed immediately by another command
or by a comment mark `|%|'.
This is because the \TeX{} parser reads the token immediately following
the argument of |\childdocmain| and puts it
at the beginning of every child section;
however, a white\-space is ignored.
\end{itemize}

%%%%%%%%%%%%%%%%%%%%%%%%%%%%%%%%%%%%%%%%
\paragraph{Content of Main File.}

It is advisable to place all content in the child files included by |\include|.
Any output contained in the main file will appear in all child documents
unless suppressed manually;
it cannot be suppressed automatically by the |\includeonly| directive
and thus should normally be avoided.
A method to include some content in the main file
by means of conditional processing is described in \secref{sec:conditional}.

%%%%%%%%%%%%%%%%%%%%%%%%%%%%%%%%%%%%%%%%
\paragraph{Page Numbering.}

When only a part of the document is compiled,
the appropriate numbering of pages
(as well as other status parameters)
is determined from the |.aux| files.
The latter contain information from previous passes.
However this information needs to propagate through
all intermediate child documents.
Therefore the page numbering in child documents may well
be inconsistent until the complete document is compiled at least once.

A useful (if unconventional) way to always ensure a consistent
page numbering is to restart the numbering in each child document
and denote the pages by `\textit{child}|.|\textit{page}'
where \textit{child} represents the chapter/section number of the child file.
This can be achieved by the command
|\numberwithin{page}{|\textit{child}|}|
of the \textsf{amsmath} package
where \textit{child} can be |chapter| or |section|
depending on the chosen structuring.
Alternatively, one can modify the macro |\thepage| appropriately
and reset the counter |page| at the start of each child file.

%%%%%%%%%%%%%%%%%%%%%%%%%%%%%%%%%%%%%%%%%%%%%%%%%%%%%%%%%%%%%%%%%%%%%%%%%%%%%%%%
\subsection{Conditional Processing}
\label{sec:conditional}

The package provides a mechanism to compile different versions
of a document. To customise the versions further some conditional processing
can come in handy to distinguish which version is being compiled.
The package provides two macros to describe the compilation context:

%%%%%%%%%%%%%%%%%%%%%%%%%%%%%%%%%%%%%%%%
\DescribeMacro{\ifchilddoc}
The conditional |\ifchilddoc| distinguishes between the compilation of
child documents and the main document:
%
\begin{center}
|\ifchilddoc |\textit{child-code}| |[|\||else |\textit{main-code}]| \||fi|
\end{center}

%%%%%%%%%%%%%%%%%%%%%%%%%%%%%%%%%%%%%%%%
\DescribeMacro{\childdocname}
\DescribeMacro{\childdocjob}
The macro |\childdocname| contains the filename (without extension)
of the main or child file being processed.
Note that |\childdocjob| will always contain the name of the main file.

%%%%%%%%%%%%%%%%%%%%%%%%%%%%%%%%%%%%%%%%
\paragraph{Title Page.}

Conditional processing can be used to include a title or banner page
in the main document when proper precautions are taken.
Importantly, the code in the main file should ensure that the page counter
(as well as other status parameters which are stored in the |.aux| files)
takes the same value after the conditional processing.
Otherwise the page numbers may take divergent values
depending on which part is compiled.

For example, a title page could be declared by:
%
\begin{center}
\begin{tabular}{l}
|\ifchilddoc\||else|\\
|\addtocounter{page}{-1}|\\
\textit{code for title page}\\
|\newpage|\\
|\||fi|
\end{tabular}
\end{center}
%
A banner page for the child documents can be generated by:
%
\begin{center}
\begin{tabular}{l}
|\ifchilddoc|\\
|\addtocounter{page}{-1}|\\
\textit{code for banner page}\\
|\newpage|\\
|\||fi|
\end{tabular}
\end{center}
%
Here one could write a message such as:
\begin{center}
|This is the part \childdocname{} of \childdocjob{}.|
\end{center}

%%%%%%%%%%%%%%%%%%%%%%%%%%%%%%%%%%%%%%%%%%%%%%%%%%%%%%%%%%%%%%%%%%%%%%%%%%%%%%%%
\subsection{Flags}
\label{sec:flags}

The package makes it easy to generate different versions
of the main or child documents.
To this end compilation flags can be defined
and assigned different default values.
They will be particularly useful in conjunction
with the forwarding mechanism described in \secref{sec:forward}.

For example, it may be useful to have a flag |\version|
which can be set to |draft| or |final|.
The document source will contain some conditional code
depending on the value of |\version|.
Suppose further, the flag should default to |final| for the main file
and to |draft| for child files
which is a natural assignment for editing the document.
This is achieved by placing the following code
in the preamble of the main document
(below the |\childdocmain| directive):
%
\begin{center}
\begin{tabular}{l}
|\ifchilddoc|\\
|\providecommand{\version}{draft}|\\
|\||else|\\
|\providecommand{\version}{final}|\\
|\||fi|
\end{tabular}
\end{center}
%
The definition by |\providecommand| makes sure
that previous definitions are not overwritten.
Further statements |\providecommand{\version}{...}|
can thus be added before the above code to override it.

For the main file, one might add a line
(between |\childdocmain| and the above block)
%
\begin{center}
|%\ifchilddoc\||else\providecommand{\version}{draft}\||fi|
\end{center}
%
which can be uncommented to produce a draft version.
Likewise one can add a line to the very top of a child file
(above the |\childdocof{|\textit{main}|}| directive)
%
\begin{center}
|%\providecommand{\version}{final}|
\end{center}
%
which can be uncommented to produce the final version of this child document.

%%%%%%%%%%%%%%%%%%%%%%%%%%%%%%%%%%%%%%%%%%%%%%%%%%%%%%%%%%%%%%%%%%%%%%%%%%%%%%%%
\subsection{Forwarding}
\label{sec:forward}

Different versions of the main or child documents
using compilation flags as described in \secref{sec:flags}
can be (permanently) stored in different files
for convenient compilation, viewing and distribution.
To this end, the package defines a command
to pass on compilation to a different file:

%%%%%%%%%%%%%%%%%%%%%%%%%%%%%%%%%%%%%%%%
\DescribeMacro{\childdocforward}
The command |\childdocforward| redirects processing to
another source file:
%
\begin{center}
\begin{tabular}{l}
|\input{childdoc.def}|\\
|\childdocforward[|\textit{main}|]{|\textit{dest}|}|\\
\end{tabular}
\end{center}
%
The argument \textit{dest} is the destination file
(without extension).
It should be the main file or one of the child files.
Note that further \textsf{childdoc} directives
such as |\childdocof| and |\childdocforward|
in the indicated file will be processed in this form.
The optional argument \textit{main}
passes on directly to the main file \textit{main}
while pretending to compile the child \textit{dest}.
This form behaves as if \textit{dest}
issues |\childdocof{|\textit{main}|}| right away,
and no further \textsf{childdoc} directives will be processed.

%%%%%%%%%%%%%%%%%%%%%%%%%%%%%%%%%%%%%%%%
\DescribeMacro{\...prefix}
In the alternative form |\childdocforwardprefix|,
%
\begin{center}
\begin{tabular}{l}
|\input{childdoc.def}|\\
|\childdocforwardprefix[|\textit{main}|]{|\textit{prefix}|}{|\textit{dest}|}|
\end{tabular}
\end{center}
%
the destination file is determined by a pattern
depending on the current file:
To make this work, the current file must be called
`{\textit{prefix}\hspace{0.2em}\textit{suffix}}'
with \textit{prefix} matching precisely the argument.
Processing is then passed on to the file
`{\textit{dest}\hspace{0.2em}\textit{suffix}}'.
Surely, the same effect is achieved by
directly specifying the
argument `{\textit{dest}\hspace{0.2em}\textit{suffix}}'
in the first form.
However, that requires to set up a different file
for each child. With the alternative form of the command
all these files can have exactly the same content
which simplifies setting them up and maintaining them.

For example, the following file |draft.tex|
with a compilation flag |\version| as described in \secref{sec:flags}
compiles the main document as a draft:
%
\begin{center}
\begin{tabular}{l}
|\def\version{draft}|\\
|\input{childdoc.def}|\\
|\childdocforward{|\textit{main}|}|
\end{tabular}
\end{center}
%
Likewise, the following files |final|\textit{nn}|.tex|
compile the final version of the child document
|child|\textit{nn}|.tex|:
%
\begin{center}
\begin{tabular}{l}
|\def\version{final}|\\
|\input{childdoc.def}|\\
|\childdocforwardprefix{final}{child}|
\end{tabular}
\end{center}
%

Note that when several versions of a main file and/or of each child file
are to be generated, it may be convenient to set up a |Makefile| or
shell script to automatise the process.

%%%%%%%%%%%%%%%%%%%%%%%%%%%%%%%%%%%%%%%%%%%%%%%%%%%%%%%%%%%%%%%%%%%%%%%%%%%%%%%%
\subsection{Command Line Processing}
\label{sec:commandline}

The effect of redirection files can also be achieved by invoking
the \LaTeX{} compiler with a more elaborate command line.
Most conveniently this should be done as part
of a shell script or a |Makefile|.

When using \textsf{childdoc} in the main file, the following
command lines effectively perform a redirection
(note that depending on the shell being used,
backslashes may have to be doubled: `|\|' $\to$ `|\\|'):
%
\begin{center}
|... -jobname "|\textit{target}|" |\\|"|[\textit{flags}]%
|\input{childdoc.def}\childdocforward[|\textit{main}|]{|\textit{dest}|}"|
\end{center}
%
Here \textit{target} is the name of the output file,
\textit{main} is the name of the main file
and \textit{dest} is the name of the main or child file to be processed
(all filenames without extensions).
The optional argument \textit{main} can be omitted
if \textit{main} matches \textit{dest}.
Optionally, compilation \textit{flags} can be defined via |\def| commands.
This command line makes the \TeX{} engine believe
it is compiling the file \textit{target}
whose content is specified as the latter parameter.
The provided code then forwards the processing to
\textit{main} or \textit{dest} as described in \secref{sec:forward}.

%%%%%%%%%%%%%%%%%%%%%%%%%%%%%%%%%%%%%%%%%%%%%%%%%%%%%%%%%%%%%%%%%%%%%%%%%%%%%%%%
\subsection{Include by Input}
\label{sec:input}

Including child documents by |\include| has some restrictions by design.
Most notably, the content of a child document always occupies
its own set of pages; pages cannot be shared between child documents.
Usually, this behaviour makes perfect sense
because each child document contain an essential part of the document.
However, in some situations it may be desirable to compose
a document from a collection of parts
without having mandatory page breaks between then.
For this case, the package
provides a mechanism to include parts
by |\input| which can also be processed individually.
However, by construction this mechanism
requires manual handling of the content to be output.

%%%%%%%%%%%%%%%%%%%%%%%%%%%%%%%%%%%%%%%%
\DescribeMacro{\ifchilddocmanual}
The main file should be prepared as usual, see \secref{sec:include}.
However, the document body must make a distinction
between processing of an individual part and of the main document, e.g.:
%
\begin{center}
\begin{tabular}{l}
|\ifchilddocmanual|\\
|\input{\childdocname}|\\
|\||else|\\
\textit{document body with }|\input{|\textit{part}|}|\\
|\||fi|
\end{tabular}
\end{center}
%
The conditional |\ifchilddocmanual| is true whenever
a part to be included by |\input| is being compiled,
and the name of the part is stored in |\childdocname|.

%%%%%%%%%%%%%%%%%%%%%%%%%%%%%%%%%%%%%%%%
\DescribeMacro{\childdocby}
Each part to be included by |\input| should start with:
%
\begin{center}
\begin{tabular}{l}
|\input{childdoc.def}|\\
|\childdocby{|\textit{main}|}|\\
\end{tabular}
\end{center}
%
The directive |\childdocby| is similar to |\childdocof|
described in \secref{sec:include},
but the subsequent selection of content must be done manually.
To that end, both |\ifchilddoc| and |\ifchilddocmanual|
will be true upon processing of a part,
and the name of the part is stored in |\childdocname|.
Note that |\jobname| will be set to the filename of the current part
so that each part receives an individual |.aux| file
that does not interfere with the |.aux| file(s) of the main document.
This behaviour can be altered by the alternative form
|\childdocby[*]{|\textit{main}|}| (with a non-empty optional argument)
which uses the |.aux| file of the main document
by setting |\jobname| to \textit{main}.

%%%%%%%%%%%%%%%%%%%%%%%%%%%%%%%%%%%%%%%%%%%%%%%%%%%%%%%%%%%%%%%%%%%%%%%%%%%%%%%%
\subsection{Driver Development}
\label{sec:driver}

The \textsf{childdoc} mechanism can also be use for the development
of definition files such as \LaTeX{} styles or classes.
This case differs from the above setup with multiple parts
included by |\include| in that no |\includeonly| should be invoked.
This can be achieved by starting the include file
(before |\ProvidesPackage|) with:
%
\begin{center}
\begin{tabular}{l}
|\input{childdoc.def}|\\
|\childdocforward{|\textit{main}|}|\\
\end{tabular}
\end{center}
%
or alternatively with:
%
\begin{center}
\begin{tabular}{l}
|\input{childdoc.def}|\\
|\childdocby{|\textit{main}|}|\\
\end{tabular}
\end{center}
%
Both forms have slightly different effects as described above.
The main file is prepared as usual, see \secref{sec:include}.

%%%%%%%%%%%%%%%%%%%%%%%%%%%%%%%%%%%%%%%%%%%%%%%%%%%%%%%%%%%%%%%%%%%%%%%%%%%%%%%%
\subsection{Legacy Detection}
\label{sec:detection}

The directive |\childdocmain| in the main file can detect
whether the complete document or merely a child is to be compiled
even without using the directive |\childdocof|.
This method is deprecated because it is less robust
and there is no compelling reason to use it;
it is merely provided for backward compatibility
and it may be removed in future versions.

If the detection mechanism is to be used,
it is mandatory to correctly specify
the filename of the main file as the argument of |\childdocmain|:
%
\begin{center}
\begin{tabular}{l}
|\input{childdoc.def}|\\
|\childdocmain{|\textit{main}|}|\\
\end{tabular}
\end{center}
%
If |\jobname| does not match the argument \textit{main} of |\childdocmain|,
it is assumed that |\jobname| points to the child file to be compiled.
When using |\childdocmain| with the main file specified as argument,
it suffices to start a child file
with just |\input{|\textit{main}|}|
without loading of the package and using |\childdocof|.
If instead all processing is done
with the appropriate \textsf{childdoc} directives,
the argument of \textit{main} of |\childdocmain| can be empty.

An alternative version of the command line processing described
in \secref{sec:commandline} using the detection mechanism reads:
%
\begin{center}
|... -jobname "|\textit{target}|" "|[\textit{flags}]%
[|\def\jobname{|\textit{dest}|}|]|\input{|\textit{main}|}"|
\end{center}

%%%%%%%%%%%%%%%%%%%%%%%%%%%%%%%%%%%%%%%%%%%%%%%%%%%%%%%%%%%%%%%%%%%%%%%%%%%%%%%%
\subsection{Manual Code}
\label{sec:manual}

In case one cannot be certain whether the definitions file |childdoc.def|
is installed on the target \TeX{} distribution
and one prefers not to ship it,
it is conceivable to paste a few relevant commands into the sources.

To that end, drop all statements |\input{childdoc.def}|
and perform the replacements as outlined below.
Instead of |\childdocmain{|\textit{main}|}| add the following code
to the top of the main file:
%
\begin{center}
\begin{tabular}{l}
|\||ifdefined\childdocname\endinput\||fi\newif\ifchilddoc|\\
|\edef\childdocname{\scantokens\expandafter{\jobname\noexpand}}|\\
|\def\childdocmain{|\textit{main}|}\||ifx\childdocmain\childdocname\||else|\\
|\childdoctrue\includeonly{\childdocname}\let\jobname\childdocmain\||fi|\\
\end{tabular}
\end{center}
%
Instead of |\childdocof{|\textit{main}|}| just include the main file
at the top of each child file:
%
\begin{center}
|\input{|\textit{main}|}|
\end{center}
%
A simple redirection |\childdocforward{|\textit{dest}|}| is achieved by:
%
\begin{center}
|\def\jobname{|\textit{dest}|}\input{\jobname}|
\end{center}
%
The redirection with prefix
|\childdocforwardprefix[|\textit{prefix}|]{|\textit{dest}|}|
is accomplished by:
%
\begin{center}
\begin{tabular}{l}
|{\edef\jobname{\scantokens\expandafter{\jobname\noexpand}}|\\
|\def\redirectjob |\textit{prefix}|#1~~~{\gdef\jobname{|\textit{dest}|#1}}|\\
|\expandafter\redirectjob\jobname~~~}\input{\jobname}|
\end{tabular}
\end{center}

In an alternative approach,
child documents can be compiled by a specific command line
without additional code or specific definitions:
%
\begin{center}
|... -jobname "|\textit{target}|" "|[\textit{flags}]%
|\includeonly{|\textit{dest}|}\input{|\textit{main}|}"|
\end{center}
%

%%%%%%%%%%%%%%%%%%%%%%%%%%%%%%%%%%%%%%%%%%%%%%%%%%%%%%%%%%%%%%%%%%%%%%%%%%%%%%%%
%%%%%%%%%%%%%%%%%%%%%%%%%%%%%%%%%%%%%%%%%%%%%%%%%%%%%%%%%%%%%%%%%%%%%%%%%%%%%%%%
\section{Information}

%%%%%%%%%%%%%%%%%%%%%%%%%%%%%%%%%%%%%%%%%%%%%%%%%%%%%%%%%%%%%%%%%%%%%%%%%%%%%%%%
\subsection{Copyright}

Copyright \copyright{} 2017--2018 Niklas Beisert

This work may be distributed and/or modified under the
conditions of the \LaTeX{} Project Public License, either version 1.3
of this license or (at your option) any later version.
The latest version of this license is in
  \url{http://www.latex-project.org/lppl.txt}
and version 1.3 or later is part of all distributions of \LaTeX{}
version 2005/12/01 or later.

This work has the LPPL maintenance status `maintained'.

The Current Maintainer of this work is Niklas Beisert.

This work consists of the files |README.txt|, |childdoc.ins| and |childdoc.dtx|
as well as the derived files |childdoc.def|, |cdocsamp.tex|
with |cdocsch1.tex|, |cdocsch2.tex|, |cdocspt3.tex|, |cdocspt4.tex|,
|cdocsdrf.tex|, |cdocsfn1.tex|, |cdocsfn2.tex|
as well as |childdoc.pdf|.

%%%%%%%%%%%%%%%%%%%%%%%%%%%%%%%%%%%%%%%%%%%%%%%%%%%%%%%%%%%%%%%%%%%%%%%%%%%%%%%%
\subsection{Files and Installation}

The package consists of the files:
%
\begin{center}
\begin{tabular}{ll}
    |README.txt|   & readme file \\
    |childdoc.ins| & installation file \\
    |childdoc.dtx| & source file \\
    |childdoc.def| & definition file \\
    |cdocsamp.tex| & sample main file \\
    |cdocsch1.tex| & sample include file \\
    |cdocsch2.tex| & sample include file \\
    |cdocspt3.tex| & sample part file \\
    |cdocspt4.tex| & sample part file \\
    |cdocsdrf.tex| & sample redirection file \\
    |cdocsfn1.tex| & sample redirection file \\
    |cdocsfn2.tex| & sample redirection file \\
    |childdoc.pdf| & manual
\end{tabular}
\end{center}
%
The distribution consists of the files
|README.txt|, |childdoc.ins| and |childdoc.dtx|.
%
\begin{itemize}
\item
Run (pdf)\LaTeX{} on |childdoc.dtx|
to compile the manual |childdoc.pdf| (this file).
\item
Run \LaTeX{} on |childdoc.ins| to create the definitions file |childdoc.def|
and the sample |cdocsamp.tex| with include files
|cdocsch1.tex|, |cdocsch2.tex|, |cdocspt3.tex|, |cdocspt4.tex|,
|cdocsdrf.tex|, |cdocsfn1.tex|, |cdocsfn2.tex|.
Then copy the file |childdoc.def| to an appropriate directory of your \LaTeX{}
distribution, e.g.\ \textit{texmf-root}|/tex/latex/childdoc|.
\end{itemize}

%%%%%%%%%%%%%%%%%%%%%%%%%%%%%%%%%%%%%%%%%%%%%%%%%%%%%%%%%%%%%%%%%%%%%%%%%%%%%%%%
\subsection{Related CTAN Packages}

There are several other packages which offer a similar functionality:
%
\begin{itemize}
\item
The packages
\href{http://ctan.org/pkg/docmute}{\textsf{docmute}},
\href{http://ctan.org/pkg/includex}{\textsf{includex}} and
\href{http://ctan.org/pkg/standalone}{\textsf{standalone}}
provide commands to include only the document body of
a child file thus allowing both files to be compiled individually.
\item
The packages \href{http://ctan.org/pkg/subdocs}{\textsf{subdocs}}
and \href{http://ctan.org/pkg/subfiles}{\textsf{subfiles}}
provide structures in which the main and child documents can be
encapsulated and allowing them to be compiled individually.
The inclusion mechanism is different from the conventional |\include|.
\item
The package \href{http://ctan.org/pkg/combine}{\textsf{combine}}
is an elaborate solution to combine several documents into one.
\end{itemize}
%
See also the CTAN topic \href{http://ctan.org/topic/subdocs}{\textsf{subdocs}}
for further related packages.
The present package differs from the above solutions in that
a document structure constructed with the conventional |\include| mechanism
just needs two extra commands at the top of every file
such that all constituent files can be compiled individually.

%%%%%%%%%%%%%%%%%%%%%%%%%%%%%%%%%%%%%%%%%%%%%%%%%%%%%%%%%%%%%%%%%%%%%%%%%%%%%%%%
%\subsection{Feature Suggestions}
%
%The following is a list of features which may be useful for future
%versions of this package:
%%
%\begin{itemize}
%\item
%\ldots
%\end{itemize}

%%%%%%%%%%%%%%%%%%%%%%%%%%%%%%%%%%%%%%%%%%%%%%%%%%%%%%%%%%%%%%%%%%%%%%%%%%%%%%%%
\subsection{Revision History}

%%%%%%%%%%%%%%%%%%%%%%%%%%%%%%%%%%%%%%%%
\paragraph{v2.0:} 2018/12/30

\begin{itemize}
\item
immediate forward processing
\item
added |\childdocby| mechanism
\item
manual restructured
\end{itemize}

%%%%%%%%%%%%%%%%%%%%%%%%%%%%%%%%%%%%%%%%
\paragraph{v1.6:} 2018/01/17

\begin{itemize}
\item
application for development of include files
\item
corrections to manual
\end{itemize}

%%%%%%%%%%%%%%%%%%%%%%%%%%%%%%%%%%%%%%%%
\paragraph{v1.5:} 2017/05/21

\begin{itemize}
\item
more complete structuring introduced
\item
|\childdocof| introduced
\item
|\childdoc| renamed to |\childdocmain|
\item
|\childredirect| renamed to |\childdocforward| and |\childdocforwardprefix|
and functionality expanded
\end{itemize}

%%%%%%%%%%%%%%%%%%%%%%%%%%%%%%%%%%%%%%%%
\paragraph{v1.0:} 2017/04/27

\begin{itemize}
\item
manual and install package
\item
first version published on CTAN
\end{itemize}

%%%%%%%%%%%%%%%%%%%%%%%%%%%%%%%%%%%%%%%%
\paragraph{v0.6:} 2017/04/26

\begin{itemize}
\item
redirection mechanism added
\end{itemize}

%%%%%%%%%%%%%%%%%%%%%%%%%%%%%%%%%%%%%%%%
\paragraph{v0.5:} 2017/04/26

\begin{itemize}
\item
functionality in definition file
\end{itemize}


%%%%%%%%%%%%%%%%%%%%%%%%%%%%%%%%%%%%%%%%%%%%%%%%%%%%%%%%%%%%%%%%%%%%%%%%%%%%%%%%
%%%%%%%%%%%%%%%%%%%%%%%%%%%%%%%%%%%%%%%%%%%%%%%%%%%%%%%%%%%%%%%%%%%%%%%%%%%%%%%%
%%%%%%%%%%%%%%%%%%%%%%%%%%%%%%%%%%%%%%%%%%%%%%%%%%%%%%%%%%%%%%%%%%%%%%%%%%%%%%%%
\appendix

\settowidth\MacroIndent{\rmfamily\scriptsize 000\ }

 \DocInput{childdoc.dtx}

\end{document}
%</driver>
% \fi
%
% %%%%%%%%%%%%%%%%%%%%%%%%%%%%%%%%%%%%%%%%%%%%%%%%%%%%%%%%%%%%%%%%%%%%%%%%%%%%%%
% %%%%%%%%%%%%%%%%%%%%%%%%%%%%%%%%%%%%%%%%%%%%%%%%%%%%%%%%%%%%%%%%%%%%%%%%%%%%%%
% \section{Sample}
%\iffalse
%<*samplemain>
%\fi
%
% The following presents a sample document
% with two chapters, two parts, a title page,
% a compile flag as well as three forwarding files to set the flag.
% It consists of eight |.tex| files:
% \begin{center}
% \begin{tabular}{ll}
% |cdocsamp.tex|&main file\\
% |cdocsch1.tex|&include file for chapter 1\\
% |cdocsch2.tex|&include file for chapter 2\\
% |cdocspt3.tex|&include file for part 3\\
% |cdocspt4.tex|&include file for part 4\\
% |cdocsdrf.tex|&forwarding file for main file in draft mode\\
% |cdocsfi1.tex|&forwarding file for final version of chapter 1\\
% |cdocsfi2.tex|&forwarding file for final version of chapter 2\\
% \end{tabular}
% \end{center}
% Each of the eight files can be compiled directly by the \LaTeX{} compiler.
%
% %%%%%%%%%%%%%%%%%%%%%%%%%%%%%%%%%%%%%%
% \paragraph{Main File.}
%
% The main file is called |cdocsamp.tex|.
%
% Load the \textsf{childdoc} definitions and
% declare the filename for the main document:
%    \begin{macrocode}
\input{childdoc.def}
\childdocmain{}
%    \end{macrocode}

% Optional override for |\version| flag:
%    \begin{macrocode}
%%\ifchilddoc\else\providecommand{\version}{draft}\fi
%    \end{macrocode}

% Define the default values for the |\version| flag
% (|final| for the main file and |draft| for childs):
%    \begin{macrocode}
\ifchilddoc
\providecommand{\version}{draft}
\else
\providecommand{\version}{final}
\fi
%    \end{macrocode}

% Load the standard document class:
%    \begin{macrocode}
\documentclass[12pt]{article}
%    \end{macrocode}

% Start the document body:
%    \begin{macrocode}
\begin{document}
%    \end{macrocode}

% Declare a title page.
% Print title, part of document being processed and version flag:
%    \begin{macrocode}
\addtocounter{page}{-1}
\begin{center}
{\LARGE\bfseries{}childdoc example\par}
\vspace{1cm}
\ifchilddoc
\ifchilddocmanual part\else chapter\fi:
`\childdocname' of `\childdocjob'\par
\else
main document: `\childdocjob'\par
\fi
version: \version\par
\end{center}
\newpage
%    \end{macrocode}

% Manually include selected file,
% otherwise process as usual:
%    \begin{macrocode}
\ifchilddocmanual
\section*{part `\childdocname'}
\input{\childdocname}
\else
%    \end{macrocode}

% Include the two chapters:
%    \begin{macrocode}
\include{cdocsch1}
\include{cdocsch2}
%    \end{macrocode}

% Include the two parts unless only chapters should be displayed:
%    \begin{macrocode}
\ifchilddoc\else
\section{part three}
\input{cdocspt3}
\section{part four}
\input{cdocspt4}
\fi
%    \end{macrocode}

% Process as usual until here:
%    \begin{macrocode}
\fi
%    \end{macrocode}

% End of document body:
%    \begin{macrocode}
\end{document}
%    \end{macrocode}
%\iffalse
%</samplemain>
%\fi
%
% %%%%%%%%%%%%%%%%%%%%%%%%%%%%%%%%%%%%%%
% \paragraph{Chapter Include Files.}
%
% The include files are called |cdocsch1.tex| and |cdocsch2.tex|.
%
%\iffalse
%<*samplechap1|samplechap2>
%\fi

% Optional override for |\version| flag:
%    \begin{macrocode}
%%\providecommand{\version}{final}
%    \end{macrocode}

% Include the main document:
%    \begin{macrocode}
\input{childdoc.def}
\childdocof{cdocsamp}
%    \end{macrocode}

%\iffalse
%</samplechap1|samplechap2>
%\fi
%
%\iffalse
%<*samplechap1>
%\fi
% Some text for chapter 1:
%    \begin{macrocode}
\section{one}
some text in chapter one
%    \end{macrocode}

%\iffalse
%</samplechap1>
%\fi
% Some text for chapter 2:
%\iffalse
%<*samplechap2>
%\fi
%    \begin{macrocode}
\section{two}
more text in chapter two
%    \end{macrocode}

%\iffalse
%</samplechap2>
%\fi
%
% %%%%%%%%%%%%%%%%%%%%%%%%%%%%%%%%%%%%%%
% \paragraph{Part Include Files.}
%
% The include files are called |cdocspt3.tex| and |cdocspt4.tex|.
%
%\iffalse
%<*samplepart3|samplepart4>
%\fi

% Optional override for |\version| flag:
%    \begin{macrocode}
%%\providecommand{\version}{final}
%    \end{macrocode}

% Include the main document:
%    \begin{macrocode}
\input{childdoc.def}
\childdocby{cdocsamp}
%    \end{macrocode}

%\iffalse
%</samplepart3|samplepart4>
%\fi
%
%\iffalse
%<*samplepart3>
%\fi
% Some text for part 3:
%    \begin{macrocode}
some text in part three
%    \end{macrocode}

%\iffalse
%</samplepart3>
%\fi
% Some text for part 4:
%\iffalse
%<*samplepart4>
%\fi
%    \begin{macrocode}
more text in part four
%    \end{macrocode}

%\iffalse
%</samplepart4>
%\fi
%
% %%%%%%%%%%%%%%%%%%%%%%%%%%%%%%%%%%%%%%
% \paragraph{Forwarding for a Complete Draft.}
%
% The following forwarding file |cdocsdrf.tex|
% compiles the main document in draft mode:
%\iffalse
%<*sampledraft>
%\fi
%    \begin{macrocode}
\def\version{draft}
\input{childdoc.def}
\childdocforward{cdocsamp}
%    \end{macrocode}

%\iffalse
%</sampledraft>
%\fi
%
% %%%%%%%%%%%%%%%%%%%%%%%%%%%%%%%%%%%%%%
% \paragraph{Forwarding for Final Version of the Chapters.}
%
% The following forwarding files |cdocsfn1.tex| and |cdocsfn2.tex|
% (with identical content)
% compile the final versions of the child documents
% |cdocsch1.tex| and |cdocsch2.tex|, respectively:
%\iffalse
%<*samplefinal>
%\fi
%    \begin{macrocode}
\def\version{final}
\input{childdoc.def}
\childdocforwardprefix[cdocsamp]{cdocsfn}{cdocsch}
%    \end{macrocode}

%\iffalse
%</samplefinal>
%\fi
%
% %%%%%%%%%%%%%%%%%%%%%%%%%%%%%%%%%%%%%%
% \paragraph{Command Line Processing.}
%
% The following three command lines generate the output files
% |cdocscld|, |cdocscl1| and |cdocscl2|
% which should be identical to
% |cdocsdrf|, |cdocsch1| and |cdocsfn2|, respectively:
% \begin{center}
% \begin{tabular}{l}
% |latex -jobname cdocscld \|\\
% |  "\def\version{draft}\input{childdoc.def}\childdocforward{cdocsamp}"|\\
% |latex -jobname cdocscl1 \|\\
% |  "\input{childdoc.def}\childdocforward[cdocsamp]{cdocsch1}"|\\
% |latex -jobname cdocscl2 \|\\
% |  "\def\version{final}\input{childdoc.def}\childdocforward{cdocsch2}"|
% \end{tabular}
% \end{center}
% Note that the trailing backslash on each first line
% merely continues the input to the second line
% (for convenient cut ant paste).
% Furthermore, the command |latex| can be replaced by any
% of its alternative versions such as |pdflatex|.
%
% %%%%%%%%%%%%%%%%%%%%%%%%%%%%%%%%%%%%%%%%%%%%%%%%%%%%%%%%%%%%%%%%%%%%%%%%%%%%%%
% %%%%%%%%%%%%%%%%%%%%%%%%%%%%%%%%%%%%%%%%%%%%%%%%%%%%%%%%%%%%%%%%%%%%%%%%%%%%%%
% \section{Implementation}
%\iffalse
%<*package>
%\fi
%
% This section describes the definitions file |childdoc.def|.

% The definitions cannot be loaded using |\usepackage| or |\RequirePackage|
% which has a mechanism to prevent loading a style file more than once.
% When loading the definitions by means of |\input|
% multiple instances have to be prevented manually:
%\iffalse
%This code needs to be before the `\ProvidesFile' directive
%which is defined at the beginning of this file.
%Therefore it is also placed there and commented out here.
%</package>
%<*discard>
%\fi
%    \begin{macrocode}
\ifdefined\childdocmain\endinput\fi
%    \end{macrocode}
%\iffalse
%</discard>
%<*package>
%\fi
%
% \macro{\ifchilddoc}
% \macro{\ifchilddocmanual}
% The conditional |\ifchilddoc| tells whether a
% child (true) or main (false) document is being compiled.
% The conditional |\ifchilddocmanual| tells whether
% the |\includeonly| mechanism is used (false) or
% the selection of child files must be performed manually (true).
% The definitions initialise to false:
%    \begin{macrocode}
\newif\ifchilddoc
\newif\ifchilddocmanual
%    \end{macrocode}

% \macro{\childdocname}
% \macro{\childdocjob}
% The macro |\childdocname| stores the name of the main document
% to be compiled. The macro |\childdocjob| stores the name of
% the document on which the \LaTeX{} compiler was originally invoked.
% The content of |\jobname| cannot be compared
% to filenames specified in the source due to different catcodes.
% The following code rescans |\jobname|, stores the result
% in |\childdocname| and saves a copy in |\childdocjob|:
%    \begin{macrocode}
\edef\childdocname{\scantokens\expandafter{\jobname\noexpand}}
\let\childdocjob\childdocname
%    \end{macrocode}

% \macro{\childdocdisable}
% The macro |\childdocdisable| prevents the main file
% from being processed more than once.
% At this stage, the main document command |\childdocmain|
% is assumed to be called once again where it should do nothing.
% Any subsequent call to it should prevent
% a secondary processing of the main document
% It overwrites the forwarding commands
% |\childdocof| and |\childdocforward|
% with empty macros to prevent further inclusions of the main document:
%    \begin{macrocode}
\newcommand{\childdocdisable}
{
  \renewcommand{\childdocmain}[1]{\renewcommand{\childdocmain}[1]{\endinput}}
  \renewcommand{\childdocof}[1]{}
  \renewcommand{\childdocby}[2][]{}
  \renewcommand{\childdocforward}[2][]{}
  \renewcommand{\childdocdisable}{}
}
%    \end{macrocode}

% \macro{\childdocmain}
% The macro |\childdocmain| is to be called at the top of the main file
% with nothing or the main filename (without extension) as argument.
% First, it breaks loops.
% If the argument is not empty and does not match |\childdocname|
% (which is set by the first inclusion of |childdoc.def|),
% |\ifchilddoc| is set to true, |\includeonly| is applied to the child file
% and |\jobname| is set to the main file
% (for proper handling of |.aux| files):
%    \begin{macrocode}
\newcommand{\childdocmain}[1]
{
  \childdocdisable\childdocmain{}
  \if?#1?\else
    \begingroup
      \def\childdoctmp{#1}
      \ifx\childdoctmp\childdocname
        \def\childdoctmp{}
      \else
        \def\childdoctmp
        {
          \childdoctrue
          \includeonly{\childdocname}
          \def\childdocjob{#1}
          \def\jobname{#1}
        }
      \fi
      \expandafter
    \endgroup
    \childdoctmp
  \fi
}
%    \end{macrocode}

% \macro{\childdocof}
% The command |\childdocof| redirects
% compilation to the main file |#1|.
%    \begin{macrocode}
\newcommand{\childdocof}[1]
{
  \childdocdisable
  \childdoctrue
  \includeonly{\childdocname}
  \def\jobname{#1}
  \def\childdocjob{#1}
  \input{#1}
}
%    \end{macrocode}

% \macro{\childdocby}
% The command |\childdocby| ....
%    \begin{macrocode}
\newcommand{\childdocby}[2][]
{
  \childdocdisable
  \childdoctrue
  \childdocmanualtrue
  \if?#1?\else
    \def\jobname{#2}
  \fi
  \def\childdocjob{#2}
  \input{#2}
  \endinput
}
%    \end{macrocode}

% \macro{\childdocforward}
% The command |\childdocforward| redirects
% compilation to the main file or
% (if the optional argument is given) a child file.
% Parameters are set as if the main file
% or a child file starting with |\childdocof| was compiled.
% Then compilation is handed over to the main file:
%    \begin{macrocode}
\newcommand{\childdocforward}[2][]
{
  \begingroup
    \if?#1?
      \def\childdoctmp
      {
        \def\childdocname{#2}
        \def\childdocjob{#2}
        \def\jobname{#2}
        \input{#2}
        \endinput
      }
    \else
      \def\childdoctmp
      {
        \childdocdisable
        \def\childdocname{#2}
        \childdoctrue
        \includeonly{#2}
        \def\childdocjob{#1}
        \def\jobname{#1}
        \input{#1}
        \endinput
      }
    \fi
    \expandafter
  \endgroup
  \childdoctmp
}
%    \end{macrocode}

% \macro{\childdocforwardprefix}
% The command |\childdocforwardprefix| redirects
% compilation to the main or a child file by means of a pattern.
% The prefix |#1| in the current filename is replaced by |#2|
% and the suffix of the current filename is kept
% (it is assumed that the filename does not contain the substring `|~~~|'
% which is used as a delimiter).
% Compilation is handed over to the new file by |\childdocforward|:
%    \begin{macrocode}
\newcommand{\childdocforwardprefix}[3][]
{
  \begingroup
    \def\childdocextract #2##1~~~{\def\childdoctmp{\childdocforward[#1]{#3##1}}}
    \expandafter\childdocextract\childdocname~~~
    \expandafter
  \endgroup
  \childdoctmp
}
%    \end{macrocode}

% \macro{\childdoc}
% The deprecated macro |\childdoc| is a legacy version of |\childdocmain|:
%    \begin{macrocode}
\newcommand{\childdoc}{\childdocmain}
%    \end{macrocode}

% \macro{\childdocredirect}
% The deprecated macro |\childdocredirect| is a legacy version
% of |\childdocforward| and |\childdocforwardprefix|:
%    \begin{macrocode}
\newcommand{\childdocredirect}[2][]
{
  \begingroup
    \if?#1?
      \def\childdoctmp{\childdocforward{#2}}
    \else
      \def\childdoctmp{\childdocforwardprefix{#1}{#2}}
    \fi
    \expandafter
  \endgroup
  \childdoctmp
}
%    \end{macrocode}

%\iffalse
%</package>
%\fi
%
\endinput
|\\
|\childdocforwardprefix[|\textit{main}|]{|\textit{prefix}|}{|\textit{dest}|}|
\end{tabular}
\end{center}
%
the destination file is determined by a pattern
depending on the current file:
To make this work, the current file must be called
`{\textit{prefix}\hspace{0.2em}\textit{suffix}}'
with \textit{prefix} matching precisely the argument.
Processing is then passed on to the file
`{\textit{dest}\hspace{0.2em}\textit{suffix}}'.
Surely, the same effect is achieved by
directly specifying the
argument `{\textit{dest}\hspace{0.2em}\textit{suffix}}'
in the first form.
However, that requires to set up a different file
for each child. With the alternative form of the command
all these files can have exactly the same content
which simplifies setting them up and maintaining them.

For example, the following file |draft.tex|
with a compilation flag |\version| as described in \secref{sec:flags}
compiles the main document as a draft:
%
\begin{center}
\begin{tabular}{l}
|\def\version{draft}|\\
|% \iffalse
%
% childdoc.dtx Copyright (C) 2017-2018 Niklas Beisert
%
% This work may be distributed and/or modified under the
% conditions of the LaTeX Project Public License, either version 1.3
% of this license or (at your option) any later version.
% The latest version of this license is in
%   http://www.latex-project.org/lppl.txt
% and version 1.3 or later is part of all distributions of LaTeX
% version 2005/12/01 or later.
%
% This work has the LPPL maintenance status `maintained'.
%
% The Current Maintainer of this work is Niklas Beisert.
%
% This work consists of the files childdoc.dtx and childdoc.ins
% and the derived files childdoc.def and cdocsamp.tex with
% cdocsch1.tex, cdocsch2.tex, cdocsdrf.tex, cdocsfn1.tex, cdocsfn2.tex.
%
%<package>\ifdefined\childdocmain\endinput\fi
%<package>\ProvidesFile{childdoc.def}[2018/12/30 v2.0 child document driver]
%<samplemain>\ProvidesFile{cdocsamp.tex}[2018/12/30 v2.0 sample for childdoc]
%<*driver>
%\ProvidesFile{childdoc.drv}[2018/12/30 v2.0 childdoc reference manual file]
\PassOptionsToClass{10pt,a4paper}{article}
\documentclass{ltxdoc}

\usepackage[margin=35mm]{geometry}
\usepackage{hyperref}
\usepackage{hyperxmp}
\usepackage[usenames]{color}

\hypersetup{colorlinks=true}
\hypersetup{pdfstartview=FitH}
\hypersetup{pdfpagemode=UseNone}
\hypersetup{pdfsource={}}
\hypersetup{pdflang={en-UK}}
\hypersetup{pdfcopyright={Copyright 2017-2018 Niklas Beisert.
  This work may be distributed and/or modified under the
  conditions of the LaTeX Project Public License, either version 1.3
  of this license or (at your option) any later version.}}
\hypersetup{pdflicenseurl={http://www.latex-project.org/lppl.txt}}
\hypersetup{pdfcontactaddress={ETH Zurich, ITP, HIT K,
  Wolfgang-Pauli-Strasse 27}}
\hypersetup{pdfcontactpostcode={8093}}
\hypersetup{pdfcontactcity={Zurich}}
\hypersetup{pdfcontactcountry={Switzerland}}
\hypersetup{pdfcontactemail={nbeisert@itp.phys.ethz.ch}}
\hypersetup{pdfcontacturl={http://people.phys.ethz.ch/\xmptilde nbeisert/}}

\newcommand{\secref}[1]{\hyperref[#1]{section \ref*{#1}}}

\parskip1ex
\parindent0pt
\let\olditemize\itemize
\def\itemize{\olditemize\parskip0pt}

\begin{document}

\title{The \textsf{childdoc} Package}
\hypersetup{pdftitle={The childdoc Package}}
\author{Niklas Beisert\\[2ex]
  Institut f\"ur Theoretische Physik\\
  Eidgen\"ossische Technische Hochschule Z\"urich\\
  Wolfgang-Pauli-Strasse 27, 8093 Z\"urich, Switzerland\\[1ex]
  \href{mailto:nbeisert@itp.phys.ethz.ch}
  {\texttt{nbeisert@itp.phys.ethz.ch}}}
\hypersetup{pdfauthor={Niklas Beisert}}
\hypersetup{pdfsubject={Manual for the LaTeX2e Package childdoc}}
\date{30 December 2018, \textsf{v2.0}}
\maketitle

\begin{abstract}\noindent
\textsf{childdoc} is a \LaTeXe{} package
that enables the direct compilation
of document sections included by |\include|
to individual files.
\end{abstract}

\begingroup
\parskip0ex
\tableofcontents
\endgroup

%%%%%%%%%%%%%%%%%%%%%%%%%%%%%%%%%%%%%%%%%%%%%%%%%%%%%%%%%%%%%%%%%%%%%%%%%%%%%%%%
%%%%%%%%%%%%%%%%%%%%%%%%%%%%%%%%%%%%%%%%%%%%%%%%%%%%%%%%%%%%%%%%%%%%%%%%%%%%%%%%
\section{Introduction}

\LaTeX{} provides a mechanism to structure a large document (such as a book)
into a main file and several child files (containing the chapters)
using the |\include| command.
This mechanism is beneficial for documents
which span hundreds of pages in order to
make the source file(s) more manageable.
Moreover, compilation can be restricted to
selected child files by means of the |\includeonly| command.
The latter feature can be used to reduce the compilation time while editing
(this was significantly more useful in the earlier days of \LaTeX{})
or to generate a smaller document which is easier to navigate.
Another application of |\includeonly| is to generate
documents consisting of selected parts of the complete document.

However, there are a few drawbacks of the plain |\include| mechanism:
\begin{itemize}
\item
The child files cannot be compiled on their own,
they can only be compiled via the main file.
A naive editing environment
(such as a text editor with an option
to have the current file processed by \LaTeX)
may require one to switch to the main file before compiling;
attempting to compile the child file produces errors.
\item
The main file must be modified (each time)
to adjust the |\includeonly| command
to the present needs. This easily leaves the main file in a messy state.
\item
The generated document will always carry the filename
of the main document. This is inconvenient if
several child files are to be compiled and
to be kept for distribution.
\end{itemize}

The present package provides a simple interface
to make child files individually compilable by \LaTeX{}.
Compiling a child file then has the same effect as compiling
the main file with an |\includeonly| command
to select the appropriate child.
Moreover the generated document will carry the name of the child
rather than the main file.
This resolves all three above issues.

This feature is meant to make the editing of books,
thesis documents and lecture notes somewhat more convenient.
However, the package can also be used efficiently for
composing a series of documents (such as exercise sheets)
which are typically distributed individually.
It then assists the author in generating the individual documents
(potentially in different versions)
as well as a document containing the collected series.
Another application is in developing style files
or other kinds of included material
where compilation of the style file could redirect
to a sample or test file.

%%%%%%%%%%%%%%%%%%%%%%%%%%%%%%%%%%%%%%%%%%%%%%%%%%%%%%%%%%%%%%%%%%%%%%%%%%%%%%%%
%%%%%%%%%%%%%%%%%%%%%%%%%%%%%%%%%%%%%%%%%%%%%%%%%%%%%%%%%%%%%%%%%%%%%%%%%%%%%%%%
\section{Usage}

First of all, the package \textsf{childdoc} is \emph{not} a standard
\LaTeXe{} |.sty| style file! Therefore it needs to be invoked in
a non-standard way.

%%%%%%%%%%%%%%%%%%%%%%%%%%%%%%%%%%%%%%%%%%%%%%%%%%%%%%%%%%%%%%%%%%%%%%%%%%%%%%%%
\subsection{Included Files}
\label{sec:include}

%%%%%%%%%%%%%%%%%%%%%%%%%%%%%%%%%%%%%%%%
\DescribeMacro{\childdocmain}
To use the package, add the commands
\begin{center}
\begin{tabular}{l}
|\input{childdoc.def}|\\
|\childdocmain{}|\\
\end{tabular}
\end{center}
at the very top of the main \LaTeX{} file,
in particular \emph{before} the |\documentclass| statement!
The argument of |\childdocmain| should be left empty
(but it must be present).

%%%%%%%%%%%%%%%%%%%%%%%%%%%%%%%%%%%%%%%%
\DescribeMacro{\childdocof}
Furthermore, add the commands
\begin{center}
\begin{tabular}{l}
|\input{childdoc.def}|\\
|\childdocof{|\textit{main}|}|\\
\end{tabular}
\end{center}
at the top of every child file \textit{child}
which is included by |\include{|\textit{child}|}|
from within the main file
(or at least for those files to be compiled individually).
The argument \textit{main} must be the filename of the main file.

There are a couple of
considerations in setting up the main and child documents:

%%%%%%%%%%%%%%%%%%%%%%%%%%%%%%%%%%%%%%%%
\paragraph{Restrictions.}

Please note the following restrictions:
\begin{itemize}
\item
|\childdocmain| must be called with one argument \textit{main}
to ensure compatibility with earlier version of the package.
It must either be empty (|\childdocmain{}|)
or precisely match the filename of the main file in which it is specified.
See \secref{sec:detection} for further information.
\item
The filename \textit{main} must be specified without the |.tex| extension.
\item
The filename \textit{main} is case sensitive
(even in case-insensitive file systems)
due to internal string comparison.
\item
The argument \textit{main} should be fully expanded, it cannot be a macro.
\item
Subdirectories and special characters should be avoided in filenames.
\item
The command |\childdocmain{|\textit{main}|}| must be followed by a whitespace.
It should not be followed immediately by another command
or by a comment mark `|%|'.
This is because the \TeX{} parser reads the token immediately following
the argument of |\childdocmain| and puts it
at the beginning of every child section;
however, a white\-space is ignored.
\end{itemize}

%%%%%%%%%%%%%%%%%%%%%%%%%%%%%%%%%%%%%%%%
\paragraph{Content of Main File.}

It is advisable to place all content in the child files included by |\include|.
Any output contained in the main file will appear in all child documents
unless suppressed manually;
it cannot be suppressed automatically by the |\includeonly| directive
and thus should normally be avoided.
A method to include some content in the main file
by means of conditional processing is described in \secref{sec:conditional}.

%%%%%%%%%%%%%%%%%%%%%%%%%%%%%%%%%%%%%%%%
\paragraph{Page Numbering.}

When only a part of the document is compiled,
the appropriate numbering of pages
(as well as other status parameters)
is determined from the |.aux| files.
The latter contain information from previous passes.
However this information needs to propagate through
all intermediate child documents.
Therefore the page numbering in child documents may well
be inconsistent until the complete document is compiled at least once.

A useful (if unconventional) way to always ensure a consistent
page numbering is to restart the numbering in each child document
and denote the pages by `\textit{child}|.|\textit{page}'
where \textit{child} represents the chapter/section number of the child file.
This can be achieved by the command
|\numberwithin{page}{|\textit{child}|}|
of the \textsf{amsmath} package
where \textit{child} can be |chapter| or |section|
depending on the chosen structuring.
Alternatively, one can modify the macro |\thepage| appropriately
and reset the counter |page| at the start of each child file.

%%%%%%%%%%%%%%%%%%%%%%%%%%%%%%%%%%%%%%%%%%%%%%%%%%%%%%%%%%%%%%%%%%%%%%%%%%%%%%%%
\subsection{Conditional Processing}
\label{sec:conditional}

The package provides a mechanism to compile different versions
of a document. To customise the versions further some conditional processing
can come in handy to distinguish which version is being compiled.
The package provides two macros to describe the compilation context:

%%%%%%%%%%%%%%%%%%%%%%%%%%%%%%%%%%%%%%%%
\DescribeMacro{\ifchilddoc}
The conditional |\ifchilddoc| distinguishes between the compilation of
child documents and the main document:
%
\begin{center}
|\ifchilddoc |\textit{child-code}| |[|\||else |\textit{main-code}]| \||fi|
\end{center}

%%%%%%%%%%%%%%%%%%%%%%%%%%%%%%%%%%%%%%%%
\DescribeMacro{\childdocname}
\DescribeMacro{\childdocjob}
The macro |\childdocname| contains the filename (without extension)
of the main or child file being processed.
Note that |\childdocjob| will always contain the name of the main file.

%%%%%%%%%%%%%%%%%%%%%%%%%%%%%%%%%%%%%%%%
\paragraph{Title Page.}

Conditional processing can be used to include a title or banner page
in the main document when proper precautions are taken.
Importantly, the code in the main file should ensure that the page counter
(as well as other status parameters which are stored in the |.aux| files)
takes the same value after the conditional processing.
Otherwise the page numbers may take divergent values
depending on which part is compiled.

For example, a title page could be declared by:
%
\begin{center}
\begin{tabular}{l}
|\ifchilddoc\||else|\\
|\addtocounter{page}{-1}|\\
\textit{code for title page}\\
|\newpage|\\
|\||fi|
\end{tabular}
\end{center}
%
A banner page for the child documents can be generated by:
%
\begin{center}
\begin{tabular}{l}
|\ifchilddoc|\\
|\addtocounter{page}{-1}|\\
\textit{code for banner page}\\
|\newpage|\\
|\||fi|
\end{tabular}
\end{center}
%
Here one could write a message such as:
\begin{center}
|This is the part \childdocname{} of \childdocjob{}.|
\end{center}

%%%%%%%%%%%%%%%%%%%%%%%%%%%%%%%%%%%%%%%%%%%%%%%%%%%%%%%%%%%%%%%%%%%%%%%%%%%%%%%%
\subsection{Flags}
\label{sec:flags}

The package makes it easy to generate different versions
of the main or child documents.
To this end compilation flags can be defined
and assigned different default values.
They will be particularly useful in conjunction
with the forwarding mechanism described in \secref{sec:forward}.

For example, it may be useful to have a flag |\version|
which can be set to |draft| or |final|.
The document source will contain some conditional code
depending on the value of |\version|.
Suppose further, the flag should default to |final| for the main file
and to |draft| for child files
which is a natural assignment for editing the document.
This is achieved by placing the following code
in the preamble of the main document
(below the |\childdocmain| directive):
%
\begin{center}
\begin{tabular}{l}
|\ifchilddoc|\\
|\providecommand{\version}{draft}|\\
|\||else|\\
|\providecommand{\version}{final}|\\
|\||fi|
\end{tabular}
\end{center}
%
The definition by |\providecommand| makes sure
that previous definitions are not overwritten.
Further statements |\providecommand{\version}{...}|
can thus be added before the above code to override it.

For the main file, one might add a line
(between |\childdocmain| and the above block)
%
\begin{center}
|%\ifchilddoc\||else\providecommand{\version}{draft}\||fi|
\end{center}
%
which can be uncommented to produce a draft version.
Likewise one can add a line to the very top of a child file
(above the |\childdocof{|\textit{main}|}| directive)
%
\begin{center}
|%\providecommand{\version}{final}|
\end{center}
%
which can be uncommented to produce the final version of this child document.

%%%%%%%%%%%%%%%%%%%%%%%%%%%%%%%%%%%%%%%%%%%%%%%%%%%%%%%%%%%%%%%%%%%%%%%%%%%%%%%%
\subsection{Forwarding}
\label{sec:forward}

Different versions of the main or child documents
using compilation flags as described in \secref{sec:flags}
can be (permanently) stored in different files
for convenient compilation, viewing and distribution.
To this end, the package defines a command
to pass on compilation to a different file:

%%%%%%%%%%%%%%%%%%%%%%%%%%%%%%%%%%%%%%%%
\DescribeMacro{\childdocforward}
The command |\childdocforward| redirects processing to
another source file:
%
\begin{center}
\begin{tabular}{l}
|\input{childdoc.def}|\\
|\childdocforward[|\textit{main}|]{|\textit{dest}|}|\\
\end{tabular}
\end{center}
%
The argument \textit{dest} is the destination file
(without extension).
It should be the main file or one of the child files.
Note that further \textsf{childdoc} directives
such as |\childdocof| and |\childdocforward|
in the indicated file will be processed in this form.
The optional argument \textit{main}
passes on directly to the main file \textit{main}
while pretending to compile the child \textit{dest}.
This form behaves as if \textit{dest}
issues |\childdocof{|\textit{main}|}| right away,
and no further \textsf{childdoc} directives will be processed.

%%%%%%%%%%%%%%%%%%%%%%%%%%%%%%%%%%%%%%%%
\DescribeMacro{\...prefix}
In the alternative form |\childdocforwardprefix|,
%
\begin{center}
\begin{tabular}{l}
|\input{childdoc.def}|\\
|\childdocforwardprefix[|\textit{main}|]{|\textit{prefix}|}{|\textit{dest}|}|
\end{tabular}
\end{center}
%
the destination file is determined by a pattern
depending on the current file:
To make this work, the current file must be called
`{\textit{prefix}\hspace{0.2em}\textit{suffix}}'
with \textit{prefix} matching precisely the argument.
Processing is then passed on to the file
`{\textit{dest}\hspace{0.2em}\textit{suffix}}'.
Surely, the same effect is achieved by
directly specifying the
argument `{\textit{dest}\hspace{0.2em}\textit{suffix}}'
in the first form.
However, that requires to set up a different file
for each child. With the alternative form of the command
all these files can have exactly the same content
which simplifies setting them up and maintaining them.

For example, the following file |draft.tex|
with a compilation flag |\version| as described in \secref{sec:flags}
compiles the main document as a draft:
%
\begin{center}
\begin{tabular}{l}
|\def\version{draft}|\\
|\input{childdoc.def}|\\
|\childdocforward{|\textit{main}|}|
\end{tabular}
\end{center}
%
Likewise, the following files |final|\textit{nn}|.tex|
compile the final version of the child document
|child|\textit{nn}|.tex|:
%
\begin{center}
\begin{tabular}{l}
|\def\version{final}|\\
|\input{childdoc.def}|\\
|\childdocforwardprefix{final}{child}|
\end{tabular}
\end{center}
%

Note that when several versions of a main file and/or of each child file
are to be generated, it may be convenient to set up a |Makefile| or
shell script to automatise the process.

%%%%%%%%%%%%%%%%%%%%%%%%%%%%%%%%%%%%%%%%%%%%%%%%%%%%%%%%%%%%%%%%%%%%%%%%%%%%%%%%
\subsection{Command Line Processing}
\label{sec:commandline}

The effect of redirection files can also be achieved by invoking
the \LaTeX{} compiler with a more elaborate command line.
Most conveniently this should be done as part
of a shell script or a |Makefile|.

When using \textsf{childdoc} in the main file, the following
command lines effectively perform a redirection
(note that depending on the shell being used,
backslashes may have to be doubled: `|\|' $\to$ `|\\|'):
%
\begin{center}
|... -jobname "|\textit{target}|" |\\|"|[\textit{flags}]%
|\input{childdoc.def}\childdocforward[|\textit{main}|]{|\textit{dest}|}"|
\end{center}
%
Here \textit{target} is the name of the output file,
\textit{main} is the name of the main file
and \textit{dest} is the name of the main or child file to be processed
(all filenames without extensions).
The optional argument \textit{main} can be omitted
if \textit{main} matches \textit{dest}.
Optionally, compilation \textit{flags} can be defined via |\def| commands.
This command line makes the \TeX{} engine believe
it is compiling the file \textit{target}
whose content is specified as the latter parameter.
The provided code then forwards the processing to
\textit{main} or \textit{dest} as described in \secref{sec:forward}.

%%%%%%%%%%%%%%%%%%%%%%%%%%%%%%%%%%%%%%%%%%%%%%%%%%%%%%%%%%%%%%%%%%%%%%%%%%%%%%%%
\subsection{Include by Input}
\label{sec:input}

Including child documents by |\include| has some restrictions by design.
Most notably, the content of a child document always occupies
its own set of pages; pages cannot be shared between child documents.
Usually, this behaviour makes perfect sense
because each child document contain an essential part of the document.
However, in some situations it may be desirable to compose
a document from a collection of parts
without having mandatory page breaks between then.
For this case, the package
provides a mechanism to include parts
by |\input| which can also be processed individually.
However, by construction this mechanism
requires manual handling of the content to be output.

%%%%%%%%%%%%%%%%%%%%%%%%%%%%%%%%%%%%%%%%
\DescribeMacro{\ifchilddocmanual}
The main file should be prepared as usual, see \secref{sec:include}.
However, the document body must make a distinction
between processing of an individual part and of the main document, e.g.:
%
\begin{center}
\begin{tabular}{l}
|\ifchilddocmanual|\\
|\input{\childdocname}|\\
|\||else|\\
\textit{document body with }|\input{|\textit{part}|}|\\
|\||fi|
\end{tabular}
\end{center}
%
The conditional |\ifchilddocmanual| is true whenever
a part to be included by |\input| is being compiled,
and the name of the part is stored in |\childdocname|.

%%%%%%%%%%%%%%%%%%%%%%%%%%%%%%%%%%%%%%%%
\DescribeMacro{\childdocby}
Each part to be included by |\input| should start with:
%
\begin{center}
\begin{tabular}{l}
|\input{childdoc.def}|\\
|\childdocby{|\textit{main}|}|\\
\end{tabular}
\end{center}
%
The directive |\childdocby| is similar to |\childdocof|
described in \secref{sec:include},
but the subsequent selection of content must be done manually.
To that end, both |\ifchilddoc| and |\ifchilddocmanual|
will be true upon processing of a part,
and the name of the part is stored in |\childdocname|.
Note that |\jobname| will be set to the filename of the current part
so that each part receives an individual |.aux| file
that does not interfere with the |.aux| file(s) of the main document.
This behaviour can be altered by the alternative form
|\childdocby[*]{|\textit{main}|}| (with a non-empty optional argument)
which uses the |.aux| file of the main document
by setting |\jobname| to \textit{main}.

%%%%%%%%%%%%%%%%%%%%%%%%%%%%%%%%%%%%%%%%%%%%%%%%%%%%%%%%%%%%%%%%%%%%%%%%%%%%%%%%
\subsection{Driver Development}
\label{sec:driver}

The \textsf{childdoc} mechanism can also be use for the development
of definition files such as \LaTeX{} styles or classes.
This case differs from the above setup with multiple parts
included by |\include| in that no |\includeonly| should be invoked.
This can be achieved by starting the include file
(before |\ProvidesPackage|) with:
%
\begin{center}
\begin{tabular}{l}
|\input{childdoc.def}|\\
|\childdocforward{|\textit{main}|}|\\
\end{tabular}
\end{center}
%
or alternatively with:
%
\begin{center}
\begin{tabular}{l}
|\input{childdoc.def}|\\
|\childdocby{|\textit{main}|}|\\
\end{tabular}
\end{center}
%
Both forms have slightly different effects as described above.
The main file is prepared as usual, see \secref{sec:include}.

%%%%%%%%%%%%%%%%%%%%%%%%%%%%%%%%%%%%%%%%%%%%%%%%%%%%%%%%%%%%%%%%%%%%%%%%%%%%%%%%
\subsection{Legacy Detection}
\label{sec:detection}

The directive |\childdocmain| in the main file can detect
whether the complete document or merely a child is to be compiled
even without using the directive |\childdocof|.
This method is deprecated because it is less robust
and there is no compelling reason to use it;
it is merely provided for backward compatibility
and it may be removed in future versions.

If the detection mechanism is to be used,
it is mandatory to correctly specify
the filename of the main file as the argument of |\childdocmain|:
%
\begin{center}
\begin{tabular}{l}
|\input{childdoc.def}|\\
|\childdocmain{|\textit{main}|}|\\
\end{tabular}
\end{center}
%
If |\jobname| does not match the argument \textit{main} of |\childdocmain|,
it is assumed that |\jobname| points to the child file to be compiled.
When using |\childdocmain| with the main file specified as argument,
it suffices to start a child file
with just |\input{|\textit{main}|}|
without loading of the package and using |\childdocof|.
If instead all processing is done
with the appropriate \textsf{childdoc} directives,
the argument of \textit{main} of |\childdocmain| can be empty.

An alternative version of the command line processing described
in \secref{sec:commandline} using the detection mechanism reads:
%
\begin{center}
|... -jobname "|\textit{target}|" "|[\textit{flags}]%
[|\def\jobname{|\textit{dest}|}|]|\input{|\textit{main}|}"|
\end{center}

%%%%%%%%%%%%%%%%%%%%%%%%%%%%%%%%%%%%%%%%%%%%%%%%%%%%%%%%%%%%%%%%%%%%%%%%%%%%%%%%
\subsection{Manual Code}
\label{sec:manual}

In case one cannot be certain whether the definitions file |childdoc.def|
is installed on the target \TeX{} distribution
and one prefers not to ship it,
it is conceivable to paste a few relevant commands into the sources.

To that end, drop all statements |\input{childdoc.def}|
and perform the replacements as outlined below.
Instead of |\childdocmain{|\textit{main}|}| add the following code
to the top of the main file:
%
\begin{center}
\begin{tabular}{l}
|\||ifdefined\childdocname\endinput\||fi\newif\ifchilddoc|\\
|\edef\childdocname{\scantokens\expandafter{\jobname\noexpand}}|\\
|\def\childdocmain{|\textit{main}|}\||ifx\childdocmain\childdocname\||else|\\
|\childdoctrue\includeonly{\childdocname}\let\jobname\childdocmain\||fi|\\
\end{tabular}
\end{center}
%
Instead of |\childdocof{|\textit{main}|}| just include the main file
at the top of each child file:
%
\begin{center}
|\input{|\textit{main}|}|
\end{center}
%
A simple redirection |\childdocforward{|\textit{dest}|}| is achieved by:
%
\begin{center}
|\def\jobname{|\textit{dest}|}\input{\jobname}|
\end{center}
%
The redirection with prefix
|\childdocforwardprefix[|\textit{prefix}|]{|\textit{dest}|}|
is accomplished by:
%
\begin{center}
\begin{tabular}{l}
|{\edef\jobname{\scantokens\expandafter{\jobname\noexpand}}|\\
|\def\redirectjob |\textit{prefix}|#1~~~{\gdef\jobname{|\textit{dest}|#1}}|\\
|\expandafter\redirectjob\jobname~~~}\input{\jobname}|
\end{tabular}
\end{center}

In an alternative approach,
child documents can be compiled by a specific command line
without additional code or specific definitions:
%
\begin{center}
|... -jobname "|\textit{target}|" "|[\textit{flags}]%
|\includeonly{|\textit{dest}|}\input{|\textit{main}|}"|
\end{center}
%

%%%%%%%%%%%%%%%%%%%%%%%%%%%%%%%%%%%%%%%%%%%%%%%%%%%%%%%%%%%%%%%%%%%%%%%%%%%%%%%%
%%%%%%%%%%%%%%%%%%%%%%%%%%%%%%%%%%%%%%%%%%%%%%%%%%%%%%%%%%%%%%%%%%%%%%%%%%%%%%%%
\section{Information}

%%%%%%%%%%%%%%%%%%%%%%%%%%%%%%%%%%%%%%%%%%%%%%%%%%%%%%%%%%%%%%%%%%%%%%%%%%%%%%%%
\subsection{Copyright}

Copyright \copyright{} 2017--2018 Niklas Beisert

This work may be distributed and/or modified under the
conditions of the \LaTeX{} Project Public License, either version 1.3
of this license or (at your option) any later version.
The latest version of this license is in
  \url{http://www.latex-project.org/lppl.txt}
and version 1.3 or later is part of all distributions of \LaTeX{}
version 2005/12/01 or later.

This work has the LPPL maintenance status `maintained'.

The Current Maintainer of this work is Niklas Beisert.

This work consists of the files |README.txt|, |childdoc.ins| and |childdoc.dtx|
as well as the derived files |childdoc.def|, |cdocsamp.tex|
with |cdocsch1.tex|, |cdocsch2.tex|, |cdocspt3.tex|, |cdocspt4.tex|,
|cdocsdrf.tex|, |cdocsfn1.tex|, |cdocsfn2.tex|
as well as |childdoc.pdf|.

%%%%%%%%%%%%%%%%%%%%%%%%%%%%%%%%%%%%%%%%%%%%%%%%%%%%%%%%%%%%%%%%%%%%%%%%%%%%%%%%
\subsection{Files and Installation}

The package consists of the files:
%
\begin{center}
\begin{tabular}{ll}
    |README.txt|   & readme file \\
    |childdoc.ins| & installation file \\
    |childdoc.dtx| & source file \\
    |childdoc.def| & definition file \\
    |cdocsamp.tex| & sample main file \\
    |cdocsch1.tex| & sample include file \\
    |cdocsch2.tex| & sample include file \\
    |cdocspt3.tex| & sample part file \\
    |cdocspt4.tex| & sample part file \\
    |cdocsdrf.tex| & sample redirection file \\
    |cdocsfn1.tex| & sample redirection file \\
    |cdocsfn2.tex| & sample redirection file \\
    |childdoc.pdf| & manual
\end{tabular}
\end{center}
%
The distribution consists of the files
|README.txt|, |childdoc.ins| and |childdoc.dtx|.
%
\begin{itemize}
\item
Run (pdf)\LaTeX{} on |childdoc.dtx|
to compile the manual |childdoc.pdf| (this file).
\item
Run \LaTeX{} on |childdoc.ins| to create the definitions file |childdoc.def|
and the sample |cdocsamp.tex| with include files
|cdocsch1.tex|, |cdocsch2.tex|, |cdocspt3.tex|, |cdocspt4.tex|,
|cdocsdrf.tex|, |cdocsfn1.tex|, |cdocsfn2.tex|.
Then copy the file |childdoc.def| to an appropriate directory of your \LaTeX{}
distribution, e.g.\ \textit{texmf-root}|/tex/latex/childdoc|.
\end{itemize}

%%%%%%%%%%%%%%%%%%%%%%%%%%%%%%%%%%%%%%%%%%%%%%%%%%%%%%%%%%%%%%%%%%%%%%%%%%%%%%%%
\subsection{Related CTAN Packages}

There are several other packages which offer a similar functionality:
%
\begin{itemize}
\item
The packages
\href{http://ctan.org/pkg/docmute}{\textsf{docmute}},
\href{http://ctan.org/pkg/includex}{\textsf{includex}} and
\href{http://ctan.org/pkg/standalone}{\textsf{standalone}}
provide commands to include only the document body of
a child file thus allowing both files to be compiled individually.
\item
The packages \href{http://ctan.org/pkg/subdocs}{\textsf{subdocs}}
and \href{http://ctan.org/pkg/subfiles}{\textsf{subfiles}}
provide structures in which the main and child documents can be
encapsulated and allowing them to be compiled individually.
The inclusion mechanism is different from the conventional |\include|.
\item
The package \href{http://ctan.org/pkg/combine}{\textsf{combine}}
is an elaborate solution to combine several documents into one.
\end{itemize}
%
See also the CTAN topic \href{http://ctan.org/topic/subdocs}{\textsf{subdocs}}
for further related packages.
The present package differs from the above solutions in that
a document structure constructed with the conventional |\include| mechanism
just needs two extra commands at the top of every file
such that all constituent files can be compiled individually.

%%%%%%%%%%%%%%%%%%%%%%%%%%%%%%%%%%%%%%%%%%%%%%%%%%%%%%%%%%%%%%%%%%%%%%%%%%%%%%%%
%\subsection{Feature Suggestions}
%
%The following is a list of features which may be useful for future
%versions of this package:
%%
%\begin{itemize}
%\item
%\ldots
%\end{itemize}

%%%%%%%%%%%%%%%%%%%%%%%%%%%%%%%%%%%%%%%%%%%%%%%%%%%%%%%%%%%%%%%%%%%%%%%%%%%%%%%%
\subsection{Revision History}

%%%%%%%%%%%%%%%%%%%%%%%%%%%%%%%%%%%%%%%%
\paragraph{v2.0:} 2018/12/30

\begin{itemize}
\item
immediate forward processing
\item
added |\childdocby| mechanism
\item
manual restructured
\end{itemize}

%%%%%%%%%%%%%%%%%%%%%%%%%%%%%%%%%%%%%%%%
\paragraph{v1.6:} 2018/01/17

\begin{itemize}
\item
application for development of include files
\item
corrections to manual
\end{itemize}

%%%%%%%%%%%%%%%%%%%%%%%%%%%%%%%%%%%%%%%%
\paragraph{v1.5:} 2017/05/21

\begin{itemize}
\item
more complete structuring introduced
\item
|\childdocof| introduced
\item
|\childdoc| renamed to |\childdocmain|
\item
|\childredirect| renamed to |\childdocforward| and |\childdocforwardprefix|
and functionality expanded
\end{itemize}

%%%%%%%%%%%%%%%%%%%%%%%%%%%%%%%%%%%%%%%%
\paragraph{v1.0:} 2017/04/27

\begin{itemize}
\item
manual and install package
\item
first version published on CTAN
\end{itemize}

%%%%%%%%%%%%%%%%%%%%%%%%%%%%%%%%%%%%%%%%
\paragraph{v0.6:} 2017/04/26

\begin{itemize}
\item
redirection mechanism added
\end{itemize}

%%%%%%%%%%%%%%%%%%%%%%%%%%%%%%%%%%%%%%%%
\paragraph{v0.5:} 2017/04/26

\begin{itemize}
\item
functionality in definition file
\end{itemize}


%%%%%%%%%%%%%%%%%%%%%%%%%%%%%%%%%%%%%%%%%%%%%%%%%%%%%%%%%%%%%%%%%%%%%%%%%%%%%%%%
%%%%%%%%%%%%%%%%%%%%%%%%%%%%%%%%%%%%%%%%%%%%%%%%%%%%%%%%%%%%%%%%%%%%%%%%%%%%%%%%
%%%%%%%%%%%%%%%%%%%%%%%%%%%%%%%%%%%%%%%%%%%%%%%%%%%%%%%%%%%%%%%%%%%%%%%%%%%%%%%%
\appendix

\settowidth\MacroIndent{\rmfamily\scriptsize 000\ }

 \DocInput{childdoc.dtx}

\end{document}
%</driver>
% \fi
%
% %%%%%%%%%%%%%%%%%%%%%%%%%%%%%%%%%%%%%%%%%%%%%%%%%%%%%%%%%%%%%%%%%%%%%%%%%%%%%%
% %%%%%%%%%%%%%%%%%%%%%%%%%%%%%%%%%%%%%%%%%%%%%%%%%%%%%%%%%%%%%%%%%%%%%%%%%%%%%%
% \section{Sample}
%\iffalse
%<*samplemain>
%\fi
%
% The following presents a sample document
% with two chapters, two parts, a title page,
% a compile flag as well as three forwarding files to set the flag.
% It consists of eight |.tex| files:
% \begin{center}
% \begin{tabular}{ll}
% |cdocsamp.tex|&main file\\
% |cdocsch1.tex|&include file for chapter 1\\
% |cdocsch2.tex|&include file for chapter 2\\
% |cdocspt3.tex|&include file for part 3\\
% |cdocspt4.tex|&include file for part 4\\
% |cdocsdrf.tex|&forwarding file for main file in draft mode\\
% |cdocsfi1.tex|&forwarding file for final version of chapter 1\\
% |cdocsfi2.tex|&forwarding file for final version of chapter 2\\
% \end{tabular}
% \end{center}
% Each of the eight files can be compiled directly by the \LaTeX{} compiler.
%
% %%%%%%%%%%%%%%%%%%%%%%%%%%%%%%%%%%%%%%
% \paragraph{Main File.}
%
% The main file is called |cdocsamp.tex|.
%
% Load the \textsf{childdoc} definitions and
% declare the filename for the main document:
%    \begin{macrocode}
\input{childdoc.def}
\childdocmain{}
%    \end{macrocode}

% Optional override for |\version| flag:
%    \begin{macrocode}
%%\ifchilddoc\else\providecommand{\version}{draft}\fi
%    \end{macrocode}

% Define the default values for the |\version| flag
% (|final| for the main file and |draft| for childs):
%    \begin{macrocode}
\ifchilddoc
\providecommand{\version}{draft}
\else
\providecommand{\version}{final}
\fi
%    \end{macrocode}

% Load the standard document class:
%    \begin{macrocode}
\documentclass[12pt]{article}
%    \end{macrocode}

% Start the document body:
%    \begin{macrocode}
\begin{document}
%    \end{macrocode}

% Declare a title page.
% Print title, part of document being processed and version flag:
%    \begin{macrocode}
\addtocounter{page}{-1}
\begin{center}
{\LARGE\bfseries{}childdoc example\par}
\vspace{1cm}
\ifchilddoc
\ifchilddocmanual part\else chapter\fi:
`\childdocname' of `\childdocjob'\par
\else
main document: `\childdocjob'\par
\fi
version: \version\par
\end{center}
\newpage
%    \end{macrocode}

% Manually include selected file,
% otherwise process as usual:
%    \begin{macrocode}
\ifchilddocmanual
\section*{part `\childdocname'}
\input{\childdocname}
\else
%    \end{macrocode}

% Include the two chapters:
%    \begin{macrocode}
\include{cdocsch1}
\include{cdocsch2}
%    \end{macrocode}

% Include the two parts unless only chapters should be displayed:
%    \begin{macrocode}
\ifchilddoc\else
\section{part three}
\input{cdocspt3}
\section{part four}
\input{cdocspt4}
\fi
%    \end{macrocode}

% Process as usual until here:
%    \begin{macrocode}
\fi
%    \end{macrocode}

% End of document body:
%    \begin{macrocode}
\end{document}
%    \end{macrocode}
%\iffalse
%</samplemain>
%\fi
%
% %%%%%%%%%%%%%%%%%%%%%%%%%%%%%%%%%%%%%%
% \paragraph{Chapter Include Files.}
%
% The include files are called |cdocsch1.tex| and |cdocsch2.tex|.
%
%\iffalse
%<*samplechap1|samplechap2>
%\fi

% Optional override for |\version| flag:
%    \begin{macrocode}
%%\providecommand{\version}{final}
%    \end{macrocode}

% Include the main document:
%    \begin{macrocode}
\input{childdoc.def}
\childdocof{cdocsamp}
%    \end{macrocode}

%\iffalse
%</samplechap1|samplechap2>
%\fi
%
%\iffalse
%<*samplechap1>
%\fi
% Some text for chapter 1:
%    \begin{macrocode}
\section{one}
some text in chapter one
%    \end{macrocode}

%\iffalse
%</samplechap1>
%\fi
% Some text for chapter 2:
%\iffalse
%<*samplechap2>
%\fi
%    \begin{macrocode}
\section{two}
more text in chapter two
%    \end{macrocode}

%\iffalse
%</samplechap2>
%\fi
%
% %%%%%%%%%%%%%%%%%%%%%%%%%%%%%%%%%%%%%%
% \paragraph{Part Include Files.}
%
% The include files are called |cdocspt3.tex| and |cdocspt4.tex|.
%
%\iffalse
%<*samplepart3|samplepart4>
%\fi

% Optional override for |\version| flag:
%    \begin{macrocode}
%%\providecommand{\version}{final}
%    \end{macrocode}

% Include the main document:
%    \begin{macrocode}
\input{childdoc.def}
\childdocby{cdocsamp}
%    \end{macrocode}

%\iffalse
%</samplepart3|samplepart4>
%\fi
%
%\iffalse
%<*samplepart3>
%\fi
% Some text for part 3:
%    \begin{macrocode}
some text in part three
%    \end{macrocode}

%\iffalse
%</samplepart3>
%\fi
% Some text for part 4:
%\iffalse
%<*samplepart4>
%\fi
%    \begin{macrocode}
more text in part four
%    \end{macrocode}

%\iffalse
%</samplepart4>
%\fi
%
% %%%%%%%%%%%%%%%%%%%%%%%%%%%%%%%%%%%%%%
% \paragraph{Forwarding for a Complete Draft.}
%
% The following forwarding file |cdocsdrf.tex|
% compiles the main document in draft mode:
%\iffalse
%<*sampledraft>
%\fi
%    \begin{macrocode}
\def\version{draft}
\input{childdoc.def}
\childdocforward{cdocsamp}
%    \end{macrocode}

%\iffalse
%</sampledraft>
%\fi
%
% %%%%%%%%%%%%%%%%%%%%%%%%%%%%%%%%%%%%%%
% \paragraph{Forwarding for Final Version of the Chapters.}
%
% The following forwarding files |cdocsfn1.tex| and |cdocsfn2.tex|
% (with identical content)
% compile the final versions of the child documents
% |cdocsch1.tex| and |cdocsch2.tex|, respectively:
%\iffalse
%<*samplefinal>
%\fi
%    \begin{macrocode}
\def\version{final}
\input{childdoc.def}
\childdocforwardprefix[cdocsamp]{cdocsfn}{cdocsch}
%    \end{macrocode}

%\iffalse
%</samplefinal>
%\fi
%
% %%%%%%%%%%%%%%%%%%%%%%%%%%%%%%%%%%%%%%
% \paragraph{Command Line Processing.}
%
% The following three command lines generate the output files
% |cdocscld|, |cdocscl1| and |cdocscl2|
% which should be identical to
% |cdocsdrf|, |cdocsch1| and |cdocsfn2|, respectively:
% \begin{center}
% \begin{tabular}{l}
% |latex -jobname cdocscld \|\\
% |  "\def\version{draft}\input{childdoc.def}\childdocforward{cdocsamp}"|\\
% |latex -jobname cdocscl1 \|\\
% |  "\input{childdoc.def}\childdocforward[cdocsamp]{cdocsch1}"|\\
% |latex -jobname cdocscl2 \|\\
% |  "\def\version{final}\input{childdoc.def}\childdocforward{cdocsch2}"|
% \end{tabular}
% \end{center}
% Note that the trailing backslash on each first line
% merely continues the input to the second line
% (for convenient cut ant paste).
% Furthermore, the command |latex| can be replaced by any
% of its alternative versions such as |pdflatex|.
%
% %%%%%%%%%%%%%%%%%%%%%%%%%%%%%%%%%%%%%%%%%%%%%%%%%%%%%%%%%%%%%%%%%%%%%%%%%%%%%%
% %%%%%%%%%%%%%%%%%%%%%%%%%%%%%%%%%%%%%%%%%%%%%%%%%%%%%%%%%%%%%%%%%%%%%%%%%%%%%%
% \section{Implementation}
%\iffalse
%<*package>
%\fi
%
% This section describes the definitions file |childdoc.def|.

% The definitions cannot be loaded using |\usepackage| or |\RequirePackage|
% which has a mechanism to prevent loading a style file more than once.
% When loading the definitions by means of |\input|
% multiple instances have to be prevented manually:
%\iffalse
%This code needs to be before the `\ProvidesFile' directive
%which is defined at the beginning of this file.
%Therefore it is also placed there and commented out here.
%</package>
%<*discard>
%\fi
%    \begin{macrocode}
\ifdefined\childdocmain\endinput\fi
%    \end{macrocode}
%\iffalse
%</discard>
%<*package>
%\fi
%
% \macro{\ifchilddoc}
% \macro{\ifchilddocmanual}
% The conditional |\ifchilddoc| tells whether a
% child (true) or main (false) document is being compiled.
% The conditional |\ifchilddocmanual| tells whether
% the |\includeonly| mechanism is used (false) or
% the selection of child files must be performed manually (true).
% The definitions initialise to false:
%    \begin{macrocode}
\newif\ifchilddoc
\newif\ifchilddocmanual
%    \end{macrocode}

% \macro{\childdocname}
% \macro{\childdocjob}
% The macro |\childdocname| stores the name of the main document
% to be compiled. The macro |\childdocjob| stores the name of
% the document on which the \LaTeX{} compiler was originally invoked.
% The content of |\jobname| cannot be compared
% to filenames specified in the source due to different catcodes.
% The following code rescans |\jobname|, stores the result
% in |\childdocname| and saves a copy in |\childdocjob|:
%    \begin{macrocode}
\edef\childdocname{\scantokens\expandafter{\jobname\noexpand}}
\let\childdocjob\childdocname
%    \end{macrocode}

% \macro{\childdocdisable}
% The macro |\childdocdisable| prevents the main file
% from being processed more than once.
% At this stage, the main document command |\childdocmain|
% is assumed to be called once again where it should do nothing.
% Any subsequent call to it should prevent
% a secondary processing of the main document
% It overwrites the forwarding commands
% |\childdocof| and |\childdocforward|
% with empty macros to prevent further inclusions of the main document:
%    \begin{macrocode}
\newcommand{\childdocdisable}
{
  \renewcommand{\childdocmain}[1]{\renewcommand{\childdocmain}[1]{\endinput}}
  \renewcommand{\childdocof}[1]{}
  \renewcommand{\childdocby}[2][]{}
  \renewcommand{\childdocforward}[2][]{}
  \renewcommand{\childdocdisable}{}
}
%    \end{macrocode}

% \macro{\childdocmain}
% The macro |\childdocmain| is to be called at the top of the main file
% with nothing or the main filename (without extension) as argument.
% First, it breaks loops.
% If the argument is not empty and does not match |\childdocname|
% (which is set by the first inclusion of |childdoc.def|),
% |\ifchilddoc| is set to true, |\includeonly| is applied to the child file
% and |\jobname| is set to the main file
% (for proper handling of |.aux| files):
%    \begin{macrocode}
\newcommand{\childdocmain}[1]
{
  \childdocdisable\childdocmain{}
  \if?#1?\else
    \begingroup
      \def\childdoctmp{#1}
      \ifx\childdoctmp\childdocname
        \def\childdoctmp{}
      \else
        \def\childdoctmp
        {
          \childdoctrue
          \includeonly{\childdocname}
          \def\childdocjob{#1}
          \def\jobname{#1}
        }
      \fi
      \expandafter
    \endgroup
    \childdoctmp
  \fi
}
%    \end{macrocode}

% \macro{\childdocof}
% The command |\childdocof| redirects
% compilation to the main file |#1|.
%    \begin{macrocode}
\newcommand{\childdocof}[1]
{
  \childdocdisable
  \childdoctrue
  \includeonly{\childdocname}
  \def\jobname{#1}
  \def\childdocjob{#1}
  \input{#1}
}
%    \end{macrocode}

% \macro{\childdocby}
% The command |\childdocby| ....
%    \begin{macrocode}
\newcommand{\childdocby}[2][]
{
  \childdocdisable
  \childdoctrue
  \childdocmanualtrue
  \if?#1?\else
    \def\jobname{#2}
  \fi
  \def\childdocjob{#2}
  \input{#2}
  \endinput
}
%    \end{macrocode}

% \macro{\childdocforward}
% The command |\childdocforward| redirects
% compilation to the main file or
% (if the optional argument is given) a child file.
% Parameters are set as if the main file
% or a child file starting with |\childdocof| was compiled.
% Then compilation is handed over to the main file:
%    \begin{macrocode}
\newcommand{\childdocforward}[2][]
{
  \begingroup
    \if?#1?
      \def\childdoctmp
      {
        \def\childdocname{#2}
        \def\childdocjob{#2}
        \def\jobname{#2}
        \input{#2}
        \endinput
      }
    \else
      \def\childdoctmp
      {
        \childdocdisable
        \def\childdocname{#2}
        \childdoctrue
        \includeonly{#2}
        \def\childdocjob{#1}
        \def\jobname{#1}
        \input{#1}
        \endinput
      }
    \fi
    \expandafter
  \endgroup
  \childdoctmp
}
%    \end{macrocode}

% \macro{\childdocforwardprefix}
% The command |\childdocforwardprefix| redirects
% compilation to the main or a child file by means of a pattern.
% The prefix |#1| in the current filename is replaced by |#2|
% and the suffix of the current filename is kept
% (it is assumed that the filename does not contain the substring `|~~~|'
% which is used as a delimiter).
% Compilation is handed over to the new file by |\childdocforward|:
%    \begin{macrocode}
\newcommand{\childdocforwardprefix}[3][]
{
  \begingroup
    \def\childdocextract #2##1~~~{\def\childdoctmp{\childdocforward[#1]{#3##1}}}
    \expandafter\childdocextract\childdocname~~~
    \expandafter
  \endgroup
  \childdoctmp
}
%    \end{macrocode}

% \macro{\childdoc}
% The deprecated macro |\childdoc| is a legacy version of |\childdocmain|:
%    \begin{macrocode}
\newcommand{\childdoc}{\childdocmain}
%    \end{macrocode}

% \macro{\childdocredirect}
% The deprecated macro |\childdocredirect| is a legacy version
% of |\childdocforward| and |\childdocforwardprefix|:
%    \begin{macrocode}
\newcommand{\childdocredirect}[2][]
{
  \begingroup
    \if?#1?
      \def\childdoctmp{\childdocforward{#2}}
    \else
      \def\childdoctmp{\childdocforwardprefix{#1}{#2}}
    \fi
    \expandafter
  \endgroup
  \childdoctmp
}
%    \end{macrocode}

%\iffalse
%</package>
%\fi
%
\endinput
|\\
|\childdocforward{|\textit{main}|}|
\end{tabular}
\end{center}
%
Likewise, the following files |final|\textit{nn}|.tex|
compile the final version of the child document
|child|\textit{nn}|.tex|:
%
\begin{center}
\begin{tabular}{l}
|\def\version{final}|\\
|% \iffalse
%
% childdoc.dtx Copyright (C) 2017-2018 Niklas Beisert
%
% This work may be distributed and/or modified under the
% conditions of the LaTeX Project Public License, either version 1.3
% of this license or (at your option) any later version.
% The latest version of this license is in
%   http://www.latex-project.org/lppl.txt
% and version 1.3 or later is part of all distributions of LaTeX
% version 2005/12/01 or later.
%
% This work has the LPPL maintenance status `maintained'.
%
% The Current Maintainer of this work is Niklas Beisert.
%
% This work consists of the files childdoc.dtx and childdoc.ins
% and the derived files childdoc.def and cdocsamp.tex with
% cdocsch1.tex, cdocsch2.tex, cdocsdrf.tex, cdocsfn1.tex, cdocsfn2.tex.
%
%<package>\ifdefined\childdocmain\endinput\fi
%<package>\ProvidesFile{childdoc.def}[2018/12/30 v2.0 child document driver]
%<samplemain>\ProvidesFile{cdocsamp.tex}[2018/12/30 v2.0 sample for childdoc]
%<*driver>
%\ProvidesFile{childdoc.drv}[2018/12/30 v2.0 childdoc reference manual file]
\PassOptionsToClass{10pt,a4paper}{article}
\documentclass{ltxdoc}

\usepackage[margin=35mm]{geometry}
\usepackage{hyperref}
\usepackage{hyperxmp}
\usepackage[usenames]{color}

\hypersetup{colorlinks=true}
\hypersetup{pdfstartview=FitH}
\hypersetup{pdfpagemode=UseNone}
\hypersetup{pdfsource={}}
\hypersetup{pdflang={en-UK}}
\hypersetup{pdfcopyright={Copyright 2017-2018 Niklas Beisert.
  This work may be distributed and/or modified under the
  conditions of the LaTeX Project Public License, either version 1.3
  of this license or (at your option) any later version.}}
\hypersetup{pdflicenseurl={http://www.latex-project.org/lppl.txt}}
\hypersetup{pdfcontactaddress={ETH Zurich, ITP, HIT K,
  Wolfgang-Pauli-Strasse 27}}
\hypersetup{pdfcontactpostcode={8093}}
\hypersetup{pdfcontactcity={Zurich}}
\hypersetup{pdfcontactcountry={Switzerland}}
\hypersetup{pdfcontactemail={nbeisert@itp.phys.ethz.ch}}
\hypersetup{pdfcontacturl={http://people.phys.ethz.ch/\xmptilde nbeisert/}}

\newcommand{\secref}[1]{\hyperref[#1]{section \ref*{#1}}}

\parskip1ex
\parindent0pt
\let\olditemize\itemize
\def\itemize{\olditemize\parskip0pt}

\begin{document}

\title{The \textsf{childdoc} Package}
\hypersetup{pdftitle={The childdoc Package}}
\author{Niklas Beisert\\[2ex]
  Institut f\"ur Theoretische Physik\\
  Eidgen\"ossische Technische Hochschule Z\"urich\\
  Wolfgang-Pauli-Strasse 27, 8093 Z\"urich, Switzerland\\[1ex]
  \href{mailto:nbeisert@itp.phys.ethz.ch}
  {\texttt{nbeisert@itp.phys.ethz.ch}}}
\hypersetup{pdfauthor={Niklas Beisert}}
\hypersetup{pdfsubject={Manual for the LaTeX2e Package childdoc}}
\date{30 December 2018, \textsf{v2.0}}
\maketitle

\begin{abstract}\noindent
\textsf{childdoc} is a \LaTeXe{} package
that enables the direct compilation
of document sections included by |\include|
to individual files.
\end{abstract}

\begingroup
\parskip0ex
\tableofcontents
\endgroup

%%%%%%%%%%%%%%%%%%%%%%%%%%%%%%%%%%%%%%%%%%%%%%%%%%%%%%%%%%%%%%%%%%%%%%%%%%%%%%%%
%%%%%%%%%%%%%%%%%%%%%%%%%%%%%%%%%%%%%%%%%%%%%%%%%%%%%%%%%%%%%%%%%%%%%%%%%%%%%%%%
\section{Introduction}

\LaTeX{} provides a mechanism to structure a large document (such as a book)
into a main file and several child files (containing the chapters)
using the |\include| command.
This mechanism is beneficial for documents
which span hundreds of pages in order to
make the source file(s) more manageable.
Moreover, compilation can be restricted to
selected child files by means of the |\includeonly| command.
The latter feature can be used to reduce the compilation time while editing
(this was significantly more useful in the earlier days of \LaTeX{})
or to generate a smaller document which is easier to navigate.
Another application of |\includeonly| is to generate
documents consisting of selected parts of the complete document.

However, there are a few drawbacks of the plain |\include| mechanism:
\begin{itemize}
\item
The child files cannot be compiled on their own,
they can only be compiled via the main file.
A naive editing environment
(such as a text editor with an option
to have the current file processed by \LaTeX)
may require one to switch to the main file before compiling;
attempting to compile the child file produces errors.
\item
The main file must be modified (each time)
to adjust the |\includeonly| command
to the present needs. This easily leaves the main file in a messy state.
\item
The generated document will always carry the filename
of the main document. This is inconvenient if
several child files are to be compiled and
to be kept for distribution.
\end{itemize}

The present package provides a simple interface
to make child files individually compilable by \LaTeX{}.
Compiling a child file then has the same effect as compiling
the main file with an |\includeonly| command
to select the appropriate child.
Moreover the generated document will carry the name of the child
rather than the main file.
This resolves all three above issues.

This feature is meant to make the editing of books,
thesis documents and lecture notes somewhat more convenient.
However, the package can also be used efficiently for
composing a series of documents (such as exercise sheets)
which are typically distributed individually.
It then assists the author in generating the individual documents
(potentially in different versions)
as well as a document containing the collected series.
Another application is in developing style files
or other kinds of included material
where compilation of the style file could redirect
to a sample or test file.

%%%%%%%%%%%%%%%%%%%%%%%%%%%%%%%%%%%%%%%%%%%%%%%%%%%%%%%%%%%%%%%%%%%%%%%%%%%%%%%%
%%%%%%%%%%%%%%%%%%%%%%%%%%%%%%%%%%%%%%%%%%%%%%%%%%%%%%%%%%%%%%%%%%%%%%%%%%%%%%%%
\section{Usage}

First of all, the package \textsf{childdoc} is \emph{not} a standard
\LaTeXe{} |.sty| style file! Therefore it needs to be invoked in
a non-standard way.

%%%%%%%%%%%%%%%%%%%%%%%%%%%%%%%%%%%%%%%%%%%%%%%%%%%%%%%%%%%%%%%%%%%%%%%%%%%%%%%%
\subsection{Included Files}
\label{sec:include}

%%%%%%%%%%%%%%%%%%%%%%%%%%%%%%%%%%%%%%%%
\DescribeMacro{\childdocmain}
To use the package, add the commands
\begin{center}
\begin{tabular}{l}
|\input{childdoc.def}|\\
|\childdocmain{}|\\
\end{tabular}
\end{center}
at the very top of the main \LaTeX{} file,
in particular \emph{before} the |\documentclass| statement!
The argument of |\childdocmain| should be left empty
(but it must be present).

%%%%%%%%%%%%%%%%%%%%%%%%%%%%%%%%%%%%%%%%
\DescribeMacro{\childdocof}
Furthermore, add the commands
\begin{center}
\begin{tabular}{l}
|\input{childdoc.def}|\\
|\childdocof{|\textit{main}|}|\\
\end{tabular}
\end{center}
at the top of every child file \textit{child}
which is included by |\include{|\textit{child}|}|
from within the main file
(or at least for those files to be compiled individually).
The argument \textit{main} must be the filename of the main file.

There are a couple of
considerations in setting up the main and child documents:

%%%%%%%%%%%%%%%%%%%%%%%%%%%%%%%%%%%%%%%%
\paragraph{Restrictions.}

Please note the following restrictions:
\begin{itemize}
\item
|\childdocmain| must be called with one argument \textit{main}
to ensure compatibility with earlier version of the package.
It must either be empty (|\childdocmain{}|)
or precisely match the filename of the main file in which it is specified.
See \secref{sec:detection} for further information.
\item
The filename \textit{main} must be specified without the |.tex| extension.
\item
The filename \textit{main} is case sensitive
(even in case-insensitive file systems)
due to internal string comparison.
\item
The argument \textit{main} should be fully expanded, it cannot be a macro.
\item
Subdirectories and special characters should be avoided in filenames.
\item
The command |\childdocmain{|\textit{main}|}| must be followed by a whitespace.
It should not be followed immediately by another command
or by a comment mark `|%|'.
This is because the \TeX{} parser reads the token immediately following
the argument of |\childdocmain| and puts it
at the beginning of every child section;
however, a white\-space is ignored.
\end{itemize}

%%%%%%%%%%%%%%%%%%%%%%%%%%%%%%%%%%%%%%%%
\paragraph{Content of Main File.}

It is advisable to place all content in the child files included by |\include|.
Any output contained in the main file will appear in all child documents
unless suppressed manually;
it cannot be suppressed automatically by the |\includeonly| directive
and thus should normally be avoided.
A method to include some content in the main file
by means of conditional processing is described in \secref{sec:conditional}.

%%%%%%%%%%%%%%%%%%%%%%%%%%%%%%%%%%%%%%%%
\paragraph{Page Numbering.}

When only a part of the document is compiled,
the appropriate numbering of pages
(as well as other status parameters)
is determined from the |.aux| files.
The latter contain information from previous passes.
However this information needs to propagate through
all intermediate child documents.
Therefore the page numbering in child documents may well
be inconsistent until the complete document is compiled at least once.

A useful (if unconventional) way to always ensure a consistent
page numbering is to restart the numbering in each child document
and denote the pages by `\textit{child}|.|\textit{page}'
where \textit{child} represents the chapter/section number of the child file.
This can be achieved by the command
|\numberwithin{page}{|\textit{child}|}|
of the \textsf{amsmath} package
where \textit{child} can be |chapter| or |section|
depending on the chosen structuring.
Alternatively, one can modify the macro |\thepage| appropriately
and reset the counter |page| at the start of each child file.

%%%%%%%%%%%%%%%%%%%%%%%%%%%%%%%%%%%%%%%%%%%%%%%%%%%%%%%%%%%%%%%%%%%%%%%%%%%%%%%%
\subsection{Conditional Processing}
\label{sec:conditional}

The package provides a mechanism to compile different versions
of a document. To customise the versions further some conditional processing
can come in handy to distinguish which version is being compiled.
The package provides two macros to describe the compilation context:

%%%%%%%%%%%%%%%%%%%%%%%%%%%%%%%%%%%%%%%%
\DescribeMacro{\ifchilddoc}
The conditional |\ifchilddoc| distinguishes between the compilation of
child documents and the main document:
%
\begin{center}
|\ifchilddoc |\textit{child-code}| |[|\||else |\textit{main-code}]| \||fi|
\end{center}

%%%%%%%%%%%%%%%%%%%%%%%%%%%%%%%%%%%%%%%%
\DescribeMacro{\childdocname}
\DescribeMacro{\childdocjob}
The macro |\childdocname| contains the filename (without extension)
of the main or child file being processed.
Note that |\childdocjob| will always contain the name of the main file.

%%%%%%%%%%%%%%%%%%%%%%%%%%%%%%%%%%%%%%%%
\paragraph{Title Page.}

Conditional processing can be used to include a title or banner page
in the main document when proper precautions are taken.
Importantly, the code in the main file should ensure that the page counter
(as well as other status parameters which are stored in the |.aux| files)
takes the same value after the conditional processing.
Otherwise the page numbers may take divergent values
depending on which part is compiled.

For example, a title page could be declared by:
%
\begin{center}
\begin{tabular}{l}
|\ifchilddoc\||else|\\
|\addtocounter{page}{-1}|\\
\textit{code for title page}\\
|\newpage|\\
|\||fi|
\end{tabular}
\end{center}
%
A banner page for the child documents can be generated by:
%
\begin{center}
\begin{tabular}{l}
|\ifchilddoc|\\
|\addtocounter{page}{-1}|\\
\textit{code for banner page}\\
|\newpage|\\
|\||fi|
\end{tabular}
\end{center}
%
Here one could write a message such as:
\begin{center}
|This is the part \childdocname{} of \childdocjob{}.|
\end{center}

%%%%%%%%%%%%%%%%%%%%%%%%%%%%%%%%%%%%%%%%%%%%%%%%%%%%%%%%%%%%%%%%%%%%%%%%%%%%%%%%
\subsection{Flags}
\label{sec:flags}

The package makes it easy to generate different versions
of the main or child documents.
To this end compilation flags can be defined
and assigned different default values.
They will be particularly useful in conjunction
with the forwarding mechanism described in \secref{sec:forward}.

For example, it may be useful to have a flag |\version|
which can be set to |draft| or |final|.
The document source will contain some conditional code
depending on the value of |\version|.
Suppose further, the flag should default to |final| for the main file
and to |draft| for child files
which is a natural assignment for editing the document.
This is achieved by placing the following code
in the preamble of the main document
(below the |\childdocmain| directive):
%
\begin{center}
\begin{tabular}{l}
|\ifchilddoc|\\
|\providecommand{\version}{draft}|\\
|\||else|\\
|\providecommand{\version}{final}|\\
|\||fi|
\end{tabular}
\end{center}
%
The definition by |\providecommand| makes sure
that previous definitions are not overwritten.
Further statements |\providecommand{\version}{...}|
can thus be added before the above code to override it.

For the main file, one might add a line
(between |\childdocmain| and the above block)
%
\begin{center}
|%\ifchilddoc\||else\providecommand{\version}{draft}\||fi|
\end{center}
%
which can be uncommented to produce a draft version.
Likewise one can add a line to the very top of a child file
(above the |\childdocof{|\textit{main}|}| directive)
%
\begin{center}
|%\providecommand{\version}{final}|
\end{center}
%
which can be uncommented to produce the final version of this child document.

%%%%%%%%%%%%%%%%%%%%%%%%%%%%%%%%%%%%%%%%%%%%%%%%%%%%%%%%%%%%%%%%%%%%%%%%%%%%%%%%
\subsection{Forwarding}
\label{sec:forward}

Different versions of the main or child documents
using compilation flags as described in \secref{sec:flags}
can be (permanently) stored in different files
for convenient compilation, viewing and distribution.
To this end, the package defines a command
to pass on compilation to a different file:

%%%%%%%%%%%%%%%%%%%%%%%%%%%%%%%%%%%%%%%%
\DescribeMacro{\childdocforward}
The command |\childdocforward| redirects processing to
another source file:
%
\begin{center}
\begin{tabular}{l}
|\input{childdoc.def}|\\
|\childdocforward[|\textit{main}|]{|\textit{dest}|}|\\
\end{tabular}
\end{center}
%
The argument \textit{dest} is the destination file
(without extension).
It should be the main file or one of the child files.
Note that further \textsf{childdoc} directives
such as |\childdocof| and |\childdocforward|
in the indicated file will be processed in this form.
The optional argument \textit{main}
passes on directly to the main file \textit{main}
while pretending to compile the child \textit{dest}.
This form behaves as if \textit{dest}
issues |\childdocof{|\textit{main}|}| right away,
and no further \textsf{childdoc} directives will be processed.

%%%%%%%%%%%%%%%%%%%%%%%%%%%%%%%%%%%%%%%%
\DescribeMacro{\...prefix}
In the alternative form |\childdocforwardprefix|,
%
\begin{center}
\begin{tabular}{l}
|\input{childdoc.def}|\\
|\childdocforwardprefix[|\textit{main}|]{|\textit{prefix}|}{|\textit{dest}|}|
\end{tabular}
\end{center}
%
the destination file is determined by a pattern
depending on the current file:
To make this work, the current file must be called
`{\textit{prefix}\hspace{0.2em}\textit{suffix}}'
with \textit{prefix} matching precisely the argument.
Processing is then passed on to the file
`{\textit{dest}\hspace{0.2em}\textit{suffix}}'.
Surely, the same effect is achieved by
directly specifying the
argument `{\textit{dest}\hspace{0.2em}\textit{suffix}}'
in the first form.
However, that requires to set up a different file
for each child. With the alternative form of the command
all these files can have exactly the same content
which simplifies setting them up and maintaining them.

For example, the following file |draft.tex|
with a compilation flag |\version| as described in \secref{sec:flags}
compiles the main document as a draft:
%
\begin{center}
\begin{tabular}{l}
|\def\version{draft}|\\
|\input{childdoc.def}|\\
|\childdocforward{|\textit{main}|}|
\end{tabular}
\end{center}
%
Likewise, the following files |final|\textit{nn}|.tex|
compile the final version of the child document
|child|\textit{nn}|.tex|:
%
\begin{center}
\begin{tabular}{l}
|\def\version{final}|\\
|\input{childdoc.def}|\\
|\childdocforwardprefix{final}{child}|
\end{tabular}
\end{center}
%

Note that when several versions of a main file and/or of each child file
are to be generated, it may be convenient to set up a |Makefile| or
shell script to automatise the process.

%%%%%%%%%%%%%%%%%%%%%%%%%%%%%%%%%%%%%%%%%%%%%%%%%%%%%%%%%%%%%%%%%%%%%%%%%%%%%%%%
\subsection{Command Line Processing}
\label{sec:commandline}

The effect of redirection files can also be achieved by invoking
the \LaTeX{} compiler with a more elaborate command line.
Most conveniently this should be done as part
of a shell script or a |Makefile|.

When using \textsf{childdoc} in the main file, the following
command lines effectively perform a redirection
(note that depending on the shell being used,
backslashes may have to be doubled: `|\|' $\to$ `|\\|'):
%
\begin{center}
|... -jobname "|\textit{target}|" |\\|"|[\textit{flags}]%
|\input{childdoc.def}\childdocforward[|\textit{main}|]{|\textit{dest}|}"|
\end{center}
%
Here \textit{target} is the name of the output file,
\textit{main} is the name of the main file
and \textit{dest} is the name of the main or child file to be processed
(all filenames without extensions).
The optional argument \textit{main} can be omitted
if \textit{main} matches \textit{dest}.
Optionally, compilation \textit{flags} can be defined via |\def| commands.
This command line makes the \TeX{} engine believe
it is compiling the file \textit{target}
whose content is specified as the latter parameter.
The provided code then forwards the processing to
\textit{main} or \textit{dest} as described in \secref{sec:forward}.

%%%%%%%%%%%%%%%%%%%%%%%%%%%%%%%%%%%%%%%%%%%%%%%%%%%%%%%%%%%%%%%%%%%%%%%%%%%%%%%%
\subsection{Include by Input}
\label{sec:input}

Including child documents by |\include| has some restrictions by design.
Most notably, the content of a child document always occupies
its own set of pages; pages cannot be shared between child documents.
Usually, this behaviour makes perfect sense
because each child document contain an essential part of the document.
However, in some situations it may be desirable to compose
a document from a collection of parts
without having mandatory page breaks between then.
For this case, the package
provides a mechanism to include parts
by |\input| which can also be processed individually.
However, by construction this mechanism
requires manual handling of the content to be output.

%%%%%%%%%%%%%%%%%%%%%%%%%%%%%%%%%%%%%%%%
\DescribeMacro{\ifchilddocmanual}
The main file should be prepared as usual, see \secref{sec:include}.
However, the document body must make a distinction
between processing of an individual part and of the main document, e.g.:
%
\begin{center}
\begin{tabular}{l}
|\ifchilddocmanual|\\
|\input{\childdocname}|\\
|\||else|\\
\textit{document body with }|\input{|\textit{part}|}|\\
|\||fi|
\end{tabular}
\end{center}
%
The conditional |\ifchilddocmanual| is true whenever
a part to be included by |\input| is being compiled,
and the name of the part is stored in |\childdocname|.

%%%%%%%%%%%%%%%%%%%%%%%%%%%%%%%%%%%%%%%%
\DescribeMacro{\childdocby}
Each part to be included by |\input| should start with:
%
\begin{center}
\begin{tabular}{l}
|\input{childdoc.def}|\\
|\childdocby{|\textit{main}|}|\\
\end{tabular}
\end{center}
%
The directive |\childdocby| is similar to |\childdocof|
described in \secref{sec:include},
but the subsequent selection of content must be done manually.
To that end, both |\ifchilddoc| and |\ifchilddocmanual|
will be true upon processing of a part,
and the name of the part is stored in |\childdocname|.
Note that |\jobname| will be set to the filename of the current part
so that each part receives an individual |.aux| file
that does not interfere with the |.aux| file(s) of the main document.
This behaviour can be altered by the alternative form
|\childdocby[*]{|\textit{main}|}| (with a non-empty optional argument)
which uses the |.aux| file of the main document
by setting |\jobname| to \textit{main}.

%%%%%%%%%%%%%%%%%%%%%%%%%%%%%%%%%%%%%%%%%%%%%%%%%%%%%%%%%%%%%%%%%%%%%%%%%%%%%%%%
\subsection{Driver Development}
\label{sec:driver}

The \textsf{childdoc} mechanism can also be use for the development
of definition files such as \LaTeX{} styles or classes.
This case differs from the above setup with multiple parts
included by |\include| in that no |\includeonly| should be invoked.
This can be achieved by starting the include file
(before |\ProvidesPackage|) with:
%
\begin{center}
\begin{tabular}{l}
|\input{childdoc.def}|\\
|\childdocforward{|\textit{main}|}|\\
\end{tabular}
\end{center}
%
or alternatively with:
%
\begin{center}
\begin{tabular}{l}
|\input{childdoc.def}|\\
|\childdocby{|\textit{main}|}|\\
\end{tabular}
\end{center}
%
Both forms have slightly different effects as described above.
The main file is prepared as usual, see \secref{sec:include}.

%%%%%%%%%%%%%%%%%%%%%%%%%%%%%%%%%%%%%%%%%%%%%%%%%%%%%%%%%%%%%%%%%%%%%%%%%%%%%%%%
\subsection{Legacy Detection}
\label{sec:detection}

The directive |\childdocmain| in the main file can detect
whether the complete document or merely a child is to be compiled
even without using the directive |\childdocof|.
This method is deprecated because it is less robust
and there is no compelling reason to use it;
it is merely provided for backward compatibility
and it may be removed in future versions.

If the detection mechanism is to be used,
it is mandatory to correctly specify
the filename of the main file as the argument of |\childdocmain|:
%
\begin{center}
\begin{tabular}{l}
|\input{childdoc.def}|\\
|\childdocmain{|\textit{main}|}|\\
\end{tabular}
\end{center}
%
If |\jobname| does not match the argument \textit{main} of |\childdocmain|,
it is assumed that |\jobname| points to the child file to be compiled.
When using |\childdocmain| with the main file specified as argument,
it suffices to start a child file
with just |\input{|\textit{main}|}|
without loading of the package and using |\childdocof|.
If instead all processing is done
with the appropriate \textsf{childdoc} directives,
the argument of \textit{main} of |\childdocmain| can be empty.

An alternative version of the command line processing described
in \secref{sec:commandline} using the detection mechanism reads:
%
\begin{center}
|... -jobname "|\textit{target}|" "|[\textit{flags}]%
[|\def\jobname{|\textit{dest}|}|]|\input{|\textit{main}|}"|
\end{center}

%%%%%%%%%%%%%%%%%%%%%%%%%%%%%%%%%%%%%%%%%%%%%%%%%%%%%%%%%%%%%%%%%%%%%%%%%%%%%%%%
\subsection{Manual Code}
\label{sec:manual}

In case one cannot be certain whether the definitions file |childdoc.def|
is installed on the target \TeX{} distribution
and one prefers not to ship it,
it is conceivable to paste a few relevant commands into the sources.

To that end, drop all statements |\input{childdoc.def}|
and perform the replacements as outlined below.
Instead of |\childdocmain{|\textit{main}|}| add the following code
to the top of the main file:
%
\begin{center}
\begin{tabular}{l}
|\||ifdefined\childdocname\endinput\||fi\newif\ifchilddoc|\\
|\edef\childdocname{\scantokens\expandafter{\jobname\noexpand}}|\\
|\def\childdocmain{|\textit{main}|}\||ifx\childdocmain\childdocname\||else|\\
|\childdoctrue\includeonly{\childdocname}\let\jobname\childdocmain\||fi|\\
\end{tabular}
\end{center}
%
Instead of |\childdocof{|\textit{main}|}| just include the main file
at the top of each child file:
%
\begin{center}
|\input{|\textit{main}|}|
\end{center}
%
A simple redirection |\childdocforward{|\textit{dest}|}| is achieved by:
%
\begin{center}
|\def\jobname{|\textit{dest}|}\input{\jobname}|
\end{center}
%
The redirection with prefix
|\childdocforwardprefix[|\textit{prefix}|]{|\textit{dest}|}|
is accomplished by:
%
\begin{center}
\begin{tabular}{l}
|{\edef\jobname{\scantokens\expandafter{\jobname\noexpand}}|\\
|\def\redirectjob |\textit{prefix}|#1~~~{\gdef\jobname{|\textit{dest}|#1}}|\\
|\expandafter\redirectjob\jobname~~~}\input{\jobname}|
\end{tabular}
\end{center}

In an alternative approach,
child documents can be compiled by a specific command line
without additional code or specific definitions:
%
\begin{center}
|... -jobname "|\textit{target}|" "|[\textit{flags}]%
|\includeonly{|\textit{dest}|}\input{|\textit{main}|}"|
\end{center}
%

%%%%%%%%%%%%%%%%%%%%%%%%%%%%%%%%%%%%%%%%%%%%%%%%%%%%%%%%%%%%%%%%%%%%%%%%%%%%%%%%
%%%%%%%%%%%%%%%%%%%%%%%%%%%%%%%%%%%%%%%%%%%%%%%%%%%%%%%%%%%%%%%%%%%%%%%%%%%%%%%%
\section{Information}

%%%%%%%%%%%%%%%%%%%%%%%%%%%%%%%%%%%%%%%%%%%%%%%%%%%%%%%%%%%%%%%%%%%%%%%%%%%%%%%%
\subsection{Copyright}

Copyright \copyright{} 2017--2018 Niklas Beisert

This work may be distributed and/or modified under the
conditions of the \LaTeX{} Project Public License, either version 1.3
of this license or (at your option) any later version.
The latest version of this license is in
  \url{http://www.latex-project.org/lppl.txt}
and version 1.3 or later is part of all distributions of \LaTeX{}
version 2005/12/01 or later.

This work has the LPPL maintenance status `maintained'.

The Current Maintainer of this work is Niklas Beisert.

This work consists of the files |README.txt|, |childdoc.ins| and |childdoc.dtx|
as well as the derived files |childdoc.def|, |cdocsamp.tex|
with |cdocsch1.tex|, |cdocsch2.tex|, |cdocspt3.tex|, |cdocspt4.tex|,
|cdocsdrf.tex|, |cdocsfn1.tex|, |cdocsfn2.tex|
as well as |childdoc.pdf|.

%%%%%%%%%%%%%%%%%%%%%%%%%%%%%%%%%%%%%%%%%%%%%%%%%%%%%%%%%%%%%%%%%%%%%%%%%%%%%%%%
\subsection{Files and Installation}

The package consists of the files:
%
\begin{center}
\begin{tabular}{ll}
    |README.txt|   & readme file \\
    |childdoc.ins| & installation file \\
    |childdoc.dtx| & source file \\
    |childdoc.def| & definition file \\
    |cdocsamp.tex| & sample main file \\
    |cdocsch1.tex| & sample include file \\
    |cdocsch2.tex| & sample include file \\
    |cdocspt3.tex| & sample part file \\
    |cdocspt4.tex| & sample part file \\
    |cdocsdrf.tex| & sample redirection file \\
    |cdocsfn1.tex| & sample redirection file \\
    |cdocsfn2.tex| & sample redirection file \\
    |childdoc.pdf| & manual
\end{tabular}
\end{center}
%
The distribution consists of the files
|README.txt|, |childdoc.ins| and |childdoc.dtx|.
%
\begin{itemize}
\item
Run (pdf)\LaTeX{} on |childdoc.dtx|
to compile the manual |childdoc.pdf| (this file).
\item
Run \LaTeX{} on |childdoc.ins| to create the definitions file |childdoc.def|
and the sample |cdocsamp.tex| with include files
|cdocsch1.tex|, |cdocsch2.tex|, |cdocspt3.tex|, |cdocspt4.tex|,
|cdocsdrf.tex|, |cdocsfn1.tex|, |cdocsfn2.tex|.
Then copy the file |childdoc.def| to an appropriate directory of your \LaTeX{}
distribution, e.g.\ \textit{texmf-root}|/tex/latex/childdoc|.
\end{itemize}

%%%%%%%%%%%%%%%%%%%%%%%%%%%%%%%%%%%%%%%%%%%%%%%%%%%%%%%%%%%%%%%%%%%%%%%%%%%%%%%%
\subsection{Related CTAN Packages}

There are several other packages which offer a similar functionality:
%
\begin{itemize}
\item
The packages
\href{http://ctan.org/pkg/docmute}{\textsf{docmute}},
\href{http://ctan.org/pkg/includex}{\textsf{includex}} and
\href{http://ctan.org/pkg/standalone}{\textsf{standalone}}
provide commands to include only the document body of
a child file thus allowing both files to be compiled individually.
\item
The packages \href{http://ctan.org/pkg/subdocs}{\textsf{subdocs}}
and \href{http://ctan.org/pkg/subfiles}{\textsf{subfiles}}
provide structures in which the main and child documents can be
encapsulated and allowing them to be compiled individually.
The inclusion mechanism is different from the conventional |\include|.
\item
The package \href{http://ctan.org/pkg/combine}{\textsf{combine}}
is an elaborate solution to combine several documents into one.
\end{itemize}
%
See also the CTAN topic \href{http://ctan.org/topic/subdocs}{\textsf{subdocs}}
for further related packages.
The present package differs from the above solutions in that
a document structure constructed with the conventional |\include| mechanism
just needs two extra commands at the top of every file
such that all constituent files can be compiled individually.

%%%%%%%%%%%%%%%%%%%%%%%%%%%%%%%%%%%%%%%%%%%%%%%%%%%%%%%%%%%%%%%%%%%%%%%%%%%%%%%%
%\subsection{Feature Suggestions}
%
%The following is a list of features which may be useful for future
%versions of this package:
%%
%\begin{itemize}
%\item
%\ldots
%\end{itemize}

%%%%%%%%%%%%%%%%%%%%%%%%%%%%%%%%%%%%%%%%%%%%%%%%%%%%%%%%%%%%%%%%%%%%%%%%%%%%%%%%
\subsection{Revision History}

%%%%%%%%%%%%%%%%%%%%%%%%%%%%%%%%%%%%%%%%
\paragraph{v2.0:} 2018/12/30

\begin{itemize}
\item
immediate forward processing
\item
added |\childdocby| mechanism
\item
manual restructured
\end{itemize}

%%%%%%%%%%%%%%%%%%%%%%%%%%%%%%%%%%%%%%%%
\paragraph{v1.6:} 2018/01/17

\begin{itemize}
\item
application for development of include files
\item
corrections to manual
\end{itemize}

%%%%%%%%%%%%%%%%%%%%%%%%%%%%%%%%%%%%%%%%
\paragraph{v1.5:} 2017/05/21

\begin{itemize}
\item
more complete structuring introduced
\item
|\childdocof| introduced
\item
|\childdoc| renamed to |\childdocmain|
\item
|\childredirect| renamed to |\childdocforward| and |\childdocforwardprefix|
and functionality expanded
\end{itemize}

%%%%%%%%%%%%%%%%%%%%%%%%%%%%%%%%%%%%%%%%
\paragraph{v1.0:} 2017/04/27

\begin{itemize}
\item
manual and install package
\item
first version published on CTAN
\end{itemize}

%%%%%%%%%%%%%%%%%%%%%%%%%%%%%%%%%%%%%%%%
\paragraph{v0.6:} 2017/04/26

\begin{itemize}
\item
redirection mechanism added
\end{itemize}

%%%%%%%%%%%%%%%%%%%%%%%%%%%%%%%%%%%%%%%%
\paragraph{v0.5:} 2017/04/26

\begin{itemize}
\item
functionality in definition file
\end{itemize}


%%%%%%%%%%%%%%%%%%%%%%%%%%%%%%%%%%%%%%%%%%%%%%%%%%%%%%%%%%%%%%%%%%%%%%%%%%%%%%%%
%%%%%%%%%%%%%%%%%%%%%%%%%%%%%%%%%%%%%%%%%%%%%%%%%%%%%%%%%%%%%%%%%%%%%%%%%%%%%%%%
%%%%%%%%%%%%%%%%%%%%%%%%%%%%%%%%%%%%%%%%%%%%%%%%%%%%%%%%%%%%%%%%%%%%%%%%%%%%%%%%
\appendix

\settowidth\MacroIndent{\rmfamily\scriptsize 000\ }

 \DocInput{childdoc.dtx}

\end{document}
%</driver>
% \fi
%
% %%%%%%%%%%%%%%%%%%%%%%%%%%%%%%%%%%%%%%%%%%%%%%%%%%%%%%%%%%%%%%%%%%%%%%%%%%%%%%
% %%%%%%%%%%%%%%%%%%%%%%%%%%%%%%%%%%%%%%%%%%%%%%%%%%%%%%%%%%%%%%%%%%%%%%%%%%%%%%
% \section{Sample}
%\iffalse
%<*samplemain>
%\fi
%
% The following presents a sample document
% with two chapters, two parts, a title page,
% a compile flag as well as three forwarding files to set the flag.
% It consists of eight |.tex| files:
% \begin{center}
% \begin{tabular}{ll}
% |cdocsamp.tex|&main file\\
% |cdocsch1.tex|&include file for chapter 1\\
% |cdocsch2.tex|&include file for chapter 2\\
% |cdocspt3.tex|&include file for part 3\\
% |cdocspt4.tex|&include file for part 4\\
% |cdocsdrf.tex|&forwarding file for main file in draft mode\\
% |cdocsfi1.tex|&forwarding file for final version of chapter 1\\
% |cdocsfi2.tex|&forwarding file for final version of chapter 2\\
% \end{tabular}
% \end{center}
% Each of the eight files can be compiled directly by the \LaTeX{} compiler.
%
% %%%%%%%%%%%%%%%%%%%%%%%%%%%%%%%%%%%%%%
% \paragraph{Main File.}
%
% The main file is called |cdocsamp.tex|.
%
% Load the \textsf{childdoc} definitions and
% declare the filename for the main document:
%    \begin{macrocode}
\input{childdoc.def}
\childdocmain{}
%    \end{macrocode}

% Optional override for |\version| flag:
%    \begin{macrocode}
%%\ifchilddoc\else\providecommand{\version}{draft}\fi
%    \end{macrocode}

% Define the default values for the |\version| flag
% (|final| for the main file and |draft| for childs):
%    \begin{macrocode}
\ifchilddoc
\providecommand{\version}{draft}
\else
\providecommand{\version}{final}
\fi
%    \end{macrocode}

% Load the standard document class:
%    \begin{macrocode}
\documentclass[12pt]{article}
%    \end{macrocode}

% Start the document body:
%    \begin{macrocode}
\begin{document}
%    \end{macrocode}

% Declare a title page.
% Print title, part of document being processed and version flag:
%    \begin{macrocode}
\addtocounter{page}{-1}
\begin{center}
{\LARGE\bfseries{}childdoc example\par}
\vspace{1cm}
\ifchilddoc
\ifchilddocmanual part\else chapter\fi:
`\childdocname' of `\childdocjob'\par
\else
main document: `\childdocjob'\par
\fi
version: \version\par
\end{center}
\newpage
%    \end{macrocode}

% Manually include selected file,
% otherwise process as usual:
%    \begin{macrocode}
\ifchilddocmanual
\section*{part `\childdocname'}
\input{\childdocname}
\else
%    \end{macrocode}

% Include the two chapters:
%    \begin{macrocode}
\include{cdocsch1}
\include{cdocsch2}
%    \end{macrocode}

% Include the two parts unless only chapters should be displayed:
%    \begin{macrocode}
\ifchilddoc\else
\section{part three}
\input{cdocspt3}
\section{part four}
\input{cdocspt4}
\fi
%    \end{macrocode}

% Process as usual until here:
%    \begin{macrocode}
\fi
%    \end{macrocode}

% End of document body:
%    \begin{macrocode}
\end{document}
%    \end{macrocode}
%\iffalse
%</samplemain>
%\fi
%
% %%%%%%%%%%%%%%%%%%%%%%%%%%%%%%%%%%%%%%
% \paragraph{Chapter Include Files.}
%
% The include files are called |cdocsch1.tex| and |cdocsch2.tex|.
%
%\iffalse
%<*samplechap1|samplechap2>
%\fi

% Optional override for |\version| flag:
%    \begin{macrocode}
%%\providecommand{\version}{final}
%    \end{macrocode}

% Include the main document:
%    \begin{macrocode}
\input{childdoc.def}
\childdocof{cdocsamp}
%    \end{macrocode}

%\iffalse
%</samplechap1|samplechap2>
%\fi
%
%\iffalse
%<*samplechap1>
%\fi
% Some text for chapter 1:
%    \begin{macrocode}
\section{one}
some text in chapter one
%    \end{macrocode}

%\iffalse
%</samplechap1>
%\fi
% Some text for chapter 2:
%\iffalse
%<*samplechap2>
%\fi
%    \begin{macrocode}
\section{two}
more text in chapter two
%    \end{macrocode}

%\iffalse
%</samplechap2>
%\fi
%
% %%%%%%%%%%%%%%%%%%%%%%%%%%%%%%%%%%%%%%
% \paragraph{Part Include Files.}
%
% The include files are called |cdocspt3.tex| and |cdocspt4.tex|.
%
%\iffalse
%<*samplepart3|samplepart4>
%\fi

% Optional override for |\version| flag:
%    \begin{macrocode}
%%\providecommand{\version}{final}
%    \end{macrocode}

% Include the main document:
%    \begin{macrocode}
\input{childdoc.def}
\childdocby{cdocsamp}
%    \end{macrocode}

%\iffalse
%</samplepart3|samplepart4>
%\fi
%
%\iffalse
%<*samplepart3>
%\fi
% Some text for part 3:
%    \begin{macrocode}
some text in part three
%    \end{macrocode}

%\iffalse
%</samplepart3>
%\fi
% Some text for part 4:
%\iffalse
%<*samplepart4>
%\fi
%    \begin{macrocode}
more text in part four
%    \end{macrocode}

%\iffalse
%</samplepart4>
%\fi
%
% %%%%%%%%%%%%%%%%%%%%%%%%%%%%%%%%%%%%%%
% \paragraph{Forwarding for a Complete Draft.}
%
% The following forwarding file |cdocsdrf.tex|
% compiles the main document in draft mode:
%\iffalse
%<*sampledraft>
%\fi
%    \begin{macrocode}
\def\version{draft}
\input{childdoc.def}
\childdocforward{cdocsamp}
%    \end{macrocode}

%\iffalse
%</sampledraft>
%\fi
%
% %%%%%%%%%%%%%%%%%%%%%%%%%%%%%%%%%%%%%%
% \paragraph{Forwarding for Final Version of the Chapters.}
%
% The following forwarding files |cdocsfn1.tex| and |cdocsfn2.tex|
% (with identical content)
% compile the final versions of the child documents
% |cdocsch1.tex| and |cdocsch2.tex|, respectively:
%\iffalse
%<*samplefinal>
%\fi
%    \begin{macrocode}
\def\version{final}
\input{childdoc.def}
\childdocforwardprefix[cdocsamp]{cdocsfn}{cdocsch}
%    \end{macrocode}

%\iffalse
%</samplefinal>
%\fi
%
% %%%%%%%%%%%%%%%%%%%%%%%%%%%%%%%%%%%%%%
% \paragraph{Command Line Processing.}
%
% The following three command lines generate the output files
% |cdocscld|, |cdocscl1| and |cdocscl2|
% which should be identical to
% |cdocsdrf|, |cdocsch1| and |cdocsfn2|, respectively:
% \begin{center}
% \begin{tabular}{l}
% |latex -jobname cdocscld \|\\
% |  "\def\version{draft}\input{childdoc.def}\childdocforward{cdocsamp}"|\\
% |latex -jobname cdocscl1 \|\\
% |  "\input{childdoc.def}\childdocforward[cdocsamp]{cdocsch1}"|\\
% |latex -jobname cdocscl2 \|\\
% |  "\def\version{final}\input{childdoc.def}\childdocforward{cdocsch2}"|
% \end{tabular}
% \end{center}
% Note that the trailing backslash on each first line
% merely continues the input to the second line
% (for convenient cut ant paste).
% Furthermore, the command |latex| can be replaced by any
% of its alternative versions such as |pdflatex|.
%
% %%%%%%%%%%%%%%%%%%%%%%%%%%%%%%%%%%%%%%%%%%%%%%%%%%%%%%%%%%%%%%%%%%%%%%%%%%%%%%
% %%%%%%%%%%%%%%%%%%%%%%%%%%%%%%%%%%%%%%%%%%%%%%%%%%%%%%%%%%%%%%%%%%%%%%%%%%%%%%
% \section{Implementation}
%\iffalse
%<*package>
%\fi
%
% This section describes the definitions file |childdoc.def|.

% The definitions cannot be loaded using |\usepackage| or |\RequirePackage|
% which has a mechanism to prevent loading a style file more than once.
% When loading the definitions by means of |\input|
% multiple instances have to be prevented manually:
%\iffalse
%This code needs to be before the `\ProvidesFile' directive
%which is defined at the beginning of this file.
%Therefore it is also placed there and commented out here.
%</package>
%<*discard>
%\fi
%    \begin{macrocode}
\ifdefined\childdocmain\endinput\fi
%    \end{macrocode}
%\iffalse
%</discard>
%<*package>
%\fi
%
% \macro{\ifchilddoc}
% \macro{\ifchilddocmanual}
% The conditional |\ifchilddoc| tells whether a
% child (true) or main (false) document is being compiled.
% The conditional |\ifchilddocmanual| tells whether
% the |\includeonly| mechanism is used (false) or
% the selection of child files must be performed manually (true).
% The definitions initialise to false:
%    \begin{macrocode}
\newif\ifchilddoc
\newif\ifchilddocmanual
%    \end{macrocode}

% \macro{\childdocname}
% \macro{\childdocjob}
% The macro |\childdocname| stores the name of the main document
% to be compiled. The macro |\childdocjob| stores the name of
% the document on which the \LaTeX{} compiler was originally invoked.
% The content of |\jobname| cannot be compared
% to filenames specified in the source due to different catcodes.
% The following code rescans |\jobname|, stores the result
% in |\childdocname| and saves a copy in |\childdocjob|:
%    \begin{macrocode}
\edef\childdocname{\scantokens\expandafter{\jobname\noexpand}}
\let\childdocjob\childdocname
%    \end{macrocode}

% \macro{\childdocdisable}
% The macro |\childdocdisable| prevents the main file
% from being processed more than once.
% At this stage, the main document command |\childdocmain|
% is assumed to be called once again where it should do nothing.
% Any subsequent call to it should prevent
% a secondary processing of the main document
% It overwrites the forwarding commands
% |\childdocof| and |\childdocforward|
% with empty macros to prevent further inclusions of the main document:
%    \begin{macrocode}
\newcommand{\childdocdisable}
{
  \renewcommand{\childdocmain}[1]{\renewcommand{\childdocmain}[1]{\endinput}}
  \renewcommand{\childdocof}[1]{}
  \renewcommand{\childdocby}[2][]{}
  \renewcommand{\childdocforward}[2][]{}
  \renewcommand{\childdocdisable}{}
}
%    \end{macrocode}

% \macro{\childdocmain}
% The macro |\childdocmain| is to be called at the top of the main file
% with nothing or the main filename (without extension) as argument.
% First, it breaks loops.
% If the argument is not empty and does not match |\childdocname|
% (which is set by the first inclusion of |childdoc.def|),
% |\ifchilddoc| is set to true, |\includeonly| is applied to the child file
% and |\jobname| is set to the main file
% (for proper handling of |.aux| files):
%    \begin{macrocode}
\newcommand{\childdocmain}[1]
{
  \childdocdisable\childdocmain{}
  \if?#1?\else
    \begingroup
      \def\childdoctmp{#1}
      \ifx\childdoctmp\childdocname
        \def\childdoctmp{}
      \else
        \def\childdoctmp
        {
          \childdoctrue
          \includeonly{\childdocname}
          \def\childdocjob{#1}
          \def\jobname{#1}
        }
      \fi
      \expandafter
    \endgroup
    \childdoctmp
  \fi
}
%    \end{macrocode}

% \macro{\childdocof}
% The command |\childdocof| redirects
% compilation to the main file |#1|.
%    \begin{macrocode}
\newcommand{\childdocof}[1]
{
  \childdocdisable
  \childdoctrue
  \includeonly{\childdocname}
  \def\jobname{#1}
  \def\childdocjob{#1}
  \input{#1}
}
%    \end{macrocode}

% \macro{\childdocby}
% The command |\childdocby| ....
%    \begin{macrocode}
\newcommand{\childdocby}[2][]
{
  \childdocdisable
  \childdoctrue
  \childdocmanualtrue
  \if?#1?\else
    \def\jobname{#2}
  \fi
  \def\childdocjob{#2}
  \input{#2}
  \endinput
}
%    \end{macrocode}

% \macro{\childdocforward}
% The command |\childdocforward| redirects
% compilation to the main file or
% (if the optional argument is given) a child file.
% Parameters are set as if the main file
% or a child file starting with |\childdocof| was compiled.
% Then compilation is handed over to the main file:
%    \begin{macrocode}
\newcommand{\childdocforward}[2][]
{
  \begingroup
    \if?#1?
      \def\childdoctmp
      {
        \def\childdocname{#2}
        \def\childdocjob{#2}
        \def\jobname{#2}
        \input{#2}
        \endinput
      }
    \else
      \def\childdoctmp
      {
        \childdocdisable
        \def\childdocname{#2}
        \childdoctrue
        \includeonly{#2}
        \def\childdocjob{#1}
        \def\jobname{#1}
        \input{#1}
        \endinput
      }
    \fi
    \expandafter
  \endgroup
  \childdoctmp
}
%    \end{macrocode}

% \macro{\childdocforwardprefix}
% The command |\childdocforwardprefix| redirects
% compilation to the main or a child file by means of a pattern.
% The prefix |#1| in the current filename is replaced by |#2|
% and the suffix of the current filename is kept
% (it is assumed that the filename does not contain the substring `|~~~|'
% which is used as a delimiter).
% Compilation is handed over to the new file by |\childdocforward|:
%    \begin{macrocode}
\newcommand{\childdocforwardprefix}[3][]
{
  \begingroup
    \def\childdocextract #2##1~~~{\def\childdoctmp{\childdocforward[#1]{#3##1}}}
    \expandafter\childdocextract\childdocname~~~
    \expandafter
  \endgroup
  \childdoctmp
}
%    \end{macrocode}

% \macro{\childdoc}
% The deprecated macro |\childdoc| is a legacy version of |\childdocmain|:
%    \begin{macrocode}
\newcommand{\childdoc}{\childdocmain}
%    \end{macrocode}

% \macro{\childdocredirect}
% The deprecated macro |\childdocredirect| is a legacy version
% of |\childdocforward| and |\childdocforwardprefix|:
%    \begin{macrocode}
\newcommand{\childdocredirect}[2][]
{
  \begingroup
    \if?#1?
      \def\childdoctmp{\childdocforward{#2}}
    \else
      \def\childdoctmp{\childdocforwardprefix{#1}{#2}}
    \fi
    \expandafter
  \endgroup
  \childdoctmp
}
%    \end{macrocode}

%\iffalse
%</package>
%\fi
%
\endinput
|\\
|\childdocforwardprefix{final}{child}|
\end{tabular}
\end{center}
%

Note that when several versions of a main file and/or of each child file
are to be generated, it may be convenient to set up a |Makefile| or
shell script to automatise the process.

%%%%%%%%%%%%%%%%%%%%%%%%%%%%%%%%%%%%%%%%%%%%%%%%%%%%%%%%%%%%%%%%%%%%%%%%%%%%%%%%
\subsection{Command Line Processing}
\label{sec:commandline}

The effect of redirection files can also be achieved by invoking
the \LaTeX{} compiler with a more elaborate command line.
Most conveniently this should be done as part
of a shell script or a |Makefile|.

When using \textsf{childdoc} in the main file, the following
command lines effectively perform a redirection
(note that depending on the shell being used,
backslashes may have to be doubled: `|\|' $\to$ `|\\|'):
%
\begin{center}
|... -jobname "|\textit{target}|" |\\|"|[\textit{flags}]%
|% \iffalse
%
% childdoc.dtx Copyright (C) 2017-2018 Niklas Beisert
%
% This work may be distributed and/or modified under the
% conditions of the LaTeX Project Public License, either version 1.3
% of this license or (at your option) any later version.
% The latest version of this license is in
%   http://www.latex-project.org/lppl.txt
% and version 1.3 or later is part of all distributions of LaTeX
% version 2005/12/01 or later.
%
% This work has the LPPL maintenance status `maintained'.
%
% The Current Maintainer of this work is Niklas Beisert.
%
% This work consists of the files childdoc.dtx and childdoc.ins
% and the derived files childdoc.def and cdocsamp.tex with
% cdocsch1.tex, cdocsch2.tex, cdocsdrf.tex, cdocsfn1.tex, cdocsfn2.tex.
%
%<package>\ifdefined\childdocmain\endinput\fi
%<package>\ProvidesFile{childdoc.def}[2018/12/30 v2.0 child document driver]
%<samplemain>\ProvidesFile{cdocsamp.tex}[2018/12/30 v2.0 sample for childdoc]
%<*driver>
%\ProvidesFile{childdoc.drv}[2018/12/30 v2.0 childdoc reference manual file]
\PassOptionsToClass{10pt,a4paper}{article}
\documentclass{ltxdoc}

\usepackage[margin=35mm]{geometry}
\usepackage{hyperref}
\usepackage{hyperxmp}
\usepackage[usenames]{color}

\hypersetup{colorlinks=true}
\hypersetup{pdfstartview=FitH}
\hypersetup{pdfpagemode=UseNone}
\hypersetup{pdfsource={}}
\hypersetup{pdflang={en-UK}}
\hypersetup{pdfcopyright={Copyright 2017-2018 Niklas Beisert.
  This work may be distributed and/or modified under the
  conditions of the LaTeX Project Public License, either version 1.3
  of this license or (at your option) any later version.}}
\hypersetup{pdflicenseurl={http://www.latex-project.org/lppl.txt}}
\hypersetup{pdfcontactaddress={ETH Zurich, ITP, HIT K,
  Wolfgang-Pauli-Strasse 27}}
\hypersetup{pdfcontactpostcode={8093}}
\hypersetup{pdfcontactcity={Zurich}}
\hypersetup{pdfcontactcountry={Switzerland}}
\hypersetup{pdfcontactemail={nbeisert@itp.phys.ethz.ch}}
\hypersetup{pdfcontacturl={http://people.phys.ethz.ch/\xmptilde nbeisert/}}

\newcommand{\secref}[1]{\hyperref[#1]{section \ref*{#1}}}

\parskip1ex
\parindent0pt
\let\olditemize\itemize
\def\itemize{\olditemize\parskip0pt}

\begin{document}

\title{The \textsf{childdoc} Package}
\hypersetup{pdftitle={The childdoc Package}}
\author{Niklas Beisert\\[2ex]
  Institut f\"ur Theoretische Physik\\
  Eidgen\"ossische Technische Hochschule Z\"urich\\
  Wolfgang-Pauli-Strasse 27, 8093 Z\"urich, Switzerland\\[1ex]
  \href{mailto:nbeisert@itp.phys.ethz.ch}
  {\texttt{nbeisert@itp.phys.ethz.ch}}}
\hypersetup{pdfauthor={Niklas Beisert}}
\hypersetup{pdfsubject={Manual for the LaTeX2e Package childdoc}}
\date{30 December 2018, \textsf{v2.0}}
\maketitle

\begin{abstract}\noindent
\textsf{childdoc} is a \LaTeXe{} package
that enables the direct compilation
of document sections included by |\include|
to individual files.
\end{abstract}

\begingroup
\parskip0ex
\tableofcontents
\endgroup

%%%%%%%%%%%%%%%%%%%%%%%%%%%%%%%%%%%%%%%%%%%%%%%%%%%%%%%%%%%%%%%%%%%%%%%%%%%%%%%%
%%%%%%%%%%%%%%%%%%%%%%%%%%%%%%%%%%%%%%%%%%%%%%%%%%%%%%%%%%%%%%%%%%%%%%%%%%%%%%%%
\section{Introduction}

\LaTeX{} provides a mechanism to structure a large document (such as a book)
into a main file and several child files (containing the chapters)
using the |\include| command.
This mechanism is beneficial for documents
which span hundreds of pages in order to
make the source file(s) more manageable.
Moreover, compilation can be restricted to
selected child files by means of the |\includeonly| command.
The latter feature can be used to reduce the compilation time while editing
(this was significantly more useful in the earlier days of \LaTeX{})
or to generate a smaller document which is easier to navigate.
Another application of |\includeonly| is to generate
documents consisting of selected parts of the complete document.

However, there are a few drawbacks of the plain |\include| mechanism:
\begin{itemize}
\item
The child files cannot be compiled on their own,
they can only be compiled via the main file.
A naive editing environment
(such as a text editor with an option
to have the current file processed by \LaTeX)
may require one to switch to the main file before compiling;
attempting to compile the child file produces errors.
\item
The main file must be modified (each time)
to adjust the |\includeonly| command
to the present needs. This easily leaves the main file in a messy state.
\item
The generated document will always carry the filename
of the main document. This is inconvenient if
several child files are to be compiled and
to be kept for distribution.
\end{itemize}

The present package provides a simple interface
to make child files individually compilable by \LaTeX{}.
Compiling a child file then has the same effect as compiling
the main file with an |\includeonly| command
to select the appropriate child.
Moreover the generated document will carry the name of the child
rather than the main file.
This resolves all three above issues.

This feature is meant to make the editing of books,
thesis documents and lecture notes somewhat more convenient.
However, the package can also be used efficiently for
composing a series of documents (such as exercise sheets)
which are typically distributed individually.
It then assists the author in generating the individual documents
(potentially in different versions)
as well as a document containing the collected series.
Another application is in developing style files
or other kinds of included material
where compilation of the style file could redirect
to a sample or test file.

%%%%%%%%%%%%%%%%%%%%%%%%%%%%%%%%%%%%%%%%%%%%%%%%%%%%%%%%%%%%%%%%%%%%%%%%%%%%%%%%
%%%%%%%%%%%%%%%%%%%%%%%%%%%%%%%%%%%%%%%%%%%%%%%%%%%%%%%%%%%%%%%%%%%%%%%%%%%%%%%%
\section{Usage}

First of all, the package \textsf{childdoc} is \emph{not} a standard
\LaTeXe{} |.sty| style file! Therefore it needs to be invoked in
a non-standard way.

%%%%%%%%%%%%%%%%%%%%%%%%%%%%%%%%%%%%%%%%%%%%%%%%%%%%%%%%%%%%%%%%%%%%%%%%%%%%%%%%
\subsection{Included Files}
\label{sec:include}

%%%%%%%%%%%%%%%%%%%%%%%%%%%%%%%%%%%%%%%%
\DescribeMacro{\childdocmain}
To use the package, add the commands
\begin{center}
\begin{tabular}{l}
|\input{childdoc.def}|\\
|\childdocmain{}|\\
\end{tabular}
\end{center}
at the very top of the main \LaTeX{} file,
in particular \emph{before} the |\documentclass| statement!
The argument of |\childdocmain| should be left empty
(but it must be present).

%%%%%%%%%%%%%%%%%%%%%%%%%%%%%%%%%%%%%%%%
\DescribeMacro{\childdocof}
Furthermore, add the commands
\begin{center}
\begin{tabular}{l}
|\input{childdoc.def}|\\
|\childdocof{|\textit{main}|}|\\
\end{tabular}
\end{center}
at the top of every child file \textit{child}
which is included by |\include{|\textit{child}|}|
from within the main file
(or at least for those files to be compiled individually).
The argument \textit{main} must be the filename of the main file.

There are a couple of
considerations in setting up the main and child documents:

%%%%%%%%%%%%%%%%%%%%%%%%%%%%%%%%%%%%%%%%
\paragraph{Restrictions.}

Please note the following restrictions:
\begin{itemize}
\item
|\childdocmain| must be called with one argument \textit{main}
to ensure compatibility with earlier version of the package.
It must either be empty (|\childdocmain{}|)
or precisely match the filename of the main file in which it is specified.
See \secref{sec:detection} for further information.
\item
The filename \textit{main} must be specified without the |.tex| extension.
\item
The filename \textit{main} is case sensitive
(even in case-insensitive file systems)
due to internal string comparison.
\item
The argument \textit{main} should be fully expanded, it cannot be a macro.
\item
Subdirectories and special characters should be avoided in filenames.
\item
The command |\childdocmain{|\textit{main}|}| must be followed by a whitespace.
It should not be followed immediately by another command
or by a comment mark `|%|'.
This is because the \TeX{} parser reads the token immediately following
the argument of |\childdocmain| and puts it
at the beginning of every child section;
however, a white\-space is ignored.
\end{itemize}

%%%%%%%%%%%%%%%%%%%%%%%%%%%%%%%%%%%%%%%%
\paragraph{Content of Main File.}

It is advisable to place all content in the child files included by |\include|.
Any output contained in the main file will appear in all child documents
unless suppressed manually;
it cannot be suppressed automatically by the |\includeonly| directive
and thus should normally be avoided.
A method to include some content in the main file
by means of conditional processing is described in \secref{sec:conditional}.

%%%%%%%%%%%%%%%%%%%%%%%%%%%%%%%%%%%%%%%%
\paragraph{Page Numbering.}

When only a part of the document is compiled,
the appropriate numbering of pages
(as well as other status parameters)
is determined from the |.aux| files.
The latter contain information from previous passes.
However this information needs to propagate through
all intermediate child documents.
Therefore the page numbering in child documents may well
be inconsistent until the complete document is compiled at least once.

A useful (if unconventional) way to always ensure a consistent
page numbering is to restart the numbering in each child document
and denote the pages by `\textit{child}|.|\textit{page}'
where \textit{child} represents the chapter/section number of the child file.
This can be achieved by the command
|\numberwithin{page}{|\textit{child}|}|
of the \textsf{amsmath} package
where \textit{child} can be |chapter| or |section|
depending on the chosen structuring.
Alternatively, one can modify the macro |\thepage| appropriately
and reset the counter |page| at the start of each child file.

%%%%%%%%%%%%%%%%%%%%%%%%%%%%%%%%%%%%%%%%%%%%%%%%%%%%%%%%%%%%%%%%%%%%%%%%%%%%%%%%
\subsection{Conditional Processing}
\label{sec:conditional}

The package provides a mechanism to compile different versions
of a document. To customise the versions further some conditional processing
can come in handy to distinguish which version is being compiled.
The package provides two macros to describe the compilation context:

%%%%%%%%%%%%%%%%%%%%%%%%%%%%%%%%%%%%%%%%
\DescribeMacro{\ifchilddoc}
The conditional |\ifchilddoc| distinguishes between the compilation of
child documents and the main document:
%
\begin{center}
|\ifchilddoc |\textit{child-code}| |[|\||else |\textit{main-code}]| \||fi|
\end{center}

%%%%%%%%%%%%%%%%%%%%%%%%%%%%%%%%%%%%%%%%
\DescribeMacro{\childdocname}
\DescribeMacro{\childdocjob}
The macro |\childdocname| contains the filename (without extension)
of the main or child file being processed.
Note that |\childdocjob| will always contain the name of the main file.

%%%%%%%%%%%%%%%%%%%%%%%%%%%%%%%%%%%%%%%%
\paragraph{Title Page.}

Conditional processing can be used to include a title or banner page
in the main document when proper precautions are taken.
Importantly, the code in the main file should ensure that the page counter
(as well as other status parameters which are stored in the |.aux| files)
takes the same value after the conditional processing.
Otherwise the page numbers may take divergent values
depending on which part is compiled.

For example, a title page could be declared by:
%
\begin{center}
\begin{tabular}{l}
|\ifchilddoc\||else|\\
|\addtocounter{page}{-1}|\\
\textit{code for title page}\\
|\newpage|\\
|\||fi|
\end{tabular}
\end{center}
%
A banner page for the child documents can be generated by:
%
\begin{center}
\begin{tabular}{l}
|\ifchilddoc|\\
|\addtocounter{page}{-1}|\\
\textit{code for banner page}\\
|\newpage|\\
|\||fi|
\end{tabular}
\end{center}
%
Here one could write a message such as:
\begin{center}
|This is the part \childdocname{} of \childdocjob{}.|
\end{center}

%%%%%%%%%%%%%%%%%%%%%%%%%%%%%%%%%%%%%%%%%%%%%%%%%%%%%%%%%%%%%%%%%%%%%%%%%%%%%%%%
\subsection{Flags}
\label{sec:flags}

The package makes it easy to generate different versions
of the main or child documents.
To this end compilation flags can be defined
and assigned different default values.
They will be particularly useful in conjunction
with the forwarding mechanism described in \secref{sec:forward}.

For example, it may be useful to have a flag |\version|
which can be set to |draft| or |final|.
The document source will contain some conditional code
depending on the value of |\version|.
Suppose further, the flag should default to |final| for the main file
and to |draft| for child files
which is a natural assignment for editing the document.
This is achieved by placing the following code
in the preamble of the main document
(below the |\childdocmain| directive):
%
\begin{center}
\begin{tabular}{l}
|\ifchilddoc|\\
|\providecommand{\version}{draft}|\\
|\||else|\\
|\providecommand{\version}{final}|\\
|\||fi|
\end{tabular}
\end{center}
%
The definition by |\providecommand| makes sure
that previous definitions are not overwritten.
Further statements |\providecommand{\version}{...}|
can thus be added before the above code to override it.

For the main file, one might add a line
(between |\childdocmain| and the above block)
%
\begin{center}
|%\ifchilddoc\||else\providecommand{\version}{draft}\||fi|
\end{center}
%
which can be uncommented to produce a draft version.
Likewise one can add a line to the very top of a child file
(above the |\childdocof{|\textit{main}|}| directive)
%
\begin{center}
|%\providecommand{\version}{final}|
\end{center}
%
which can be uncommented to produce the final version of this child document.

%%%%%%%%%%%%%%%%%%%%%%%%%%%%%%%%%%%%%%%%%%%%%%%%%%%%%%%%%%%%%%%%%%%%%%%%%%%%%%%%
\subsection{Forwarding}
\label{sec:forward}

Different versions of the main or child documents
using compilation flags as described in \secref{sec:flags}
can be (permanently) stored in different files
for convenient compilation, viewing and distribution.
To this end, the package defines a command
to pass on compilation to a different file:

%%%%%%%%%%%%%%%%%%%%%%%%%%%%%%%%%%%%%%%%
\DescribeMacro{\childdocforward}
The command |\childdocforward| redirects processing to
another source file:
%
\begin{center}
\begin{tabular}{l}
|\input{childdoc.def}|\\
|\childdocforward[|\textit{main}|]{|\textit{dest}|}|\\
\end{tabular}
\end{center}
%
The argument \textit{dest} is the destination file
(without extension).
It should be the main file or one of the child files.
Note that further \textsf{childdoc} directives
such as |\childdocof| and |\childdocforward|
in the indicated file will be processed in this form.
The optional argument \textit{main}
passes on directly to the main file \textit{main}
while pretending to compile the child \textit{dest}.
This form behaves as if \textit{dest}
issues |\childdocof{|\textit{main}|}| right away,
and no further \textsf{childdoc} directives will be processed.

%%%%%%%%%%%%%%%%%%%%%%%%%%%%%%%%%%%%%%%%
\DescribeMacro{\...prefix}
In the alternative form |\childdocforwardprefix|,
%
\begin{center}
\begin{tabular}{l}
|\input{childdoc.def}|\\
|\childdocforwardprefix[|\textit{main}|]{|\textit{prefix}|}{|\textit{dest}|}|
\end{tabular}
\end{center}
%
the destination file is determined by a pattern
depending on the current file:
To make this work, the current file must be called
`{\textit{prefix}\hspace{0.2em}\textit{suffix}}'
with \textit{prefix} matching precisely the argument.
Processing is then passed on to the file
`{\textit{dest}\hspace{0.2em}\textit{suffix}}'.
Surely, the same effect is achieved by
directly specifying the
argument `{\textit{dest}\hspace{0.2em}\textit{suffix}}'
in the first form.
However, that requires to set up a different file
for each child. With the alternative form of the command
all these files can have exactly the same content
which simplifies setting them up and maintaining them.

For example, the following file |draft.tex|
with a compilation flag |\version| as described in \secref{sec:flags}
compiles the main document as a draft:
%
\begin{center}
\begin{tabular}{l}
|\def\version{draft}|\\
|\input{childdoc.def}|\\
|\childdocforward{|\textit{main}|}|
\end{tabular}
\end{center}
%
Likewise, the following files |final|\textit{nn}|.tex|
compile the final version of the child document
|child|\textit{nn}|.tex|:
%
\begin{center}
\begin{tabular}{l}
|\def\version{final}|\\
|\input{childdoc.def}|\\
|\childdocforwardprefix{final}{child}|
\end{tabular}
\end{center}
%

Note that when several versions of a main file and/or of each child file
are to be generated, it may be convenient to set up a |Makefile| or
shell script to automatise the process.

%%%%%%%%%%%%%%%%%%%%%%%%%%%%%%%%%%%%%%%%%%%%%%%%%%%%%%%%%%%%%%%%%%%%%%%%%%%%%%%%
\subsection{Command Line Processing}
\label{sec:commandline}

The effect of redirection files can also be achieved by invoking
the \LaTeX{} compiler with a more elaborate command line.
Most conveniently this should be done as part
of a shell script or a |Makefile|.

When using \textsf{childdoc} in the main file, the following
command lines effectively perform a redirection
(note that depending on the shell being used,
backslashes may have to be doubled: `|\|' $\to$ `|\\|'):
%
\begin{center}
|... -jobname "|\textit{target}|" |\\|"|[\textit{flags}]%
|\input{childdoc.def}\childdocforward[|\textit{main}|]{|\textit{dest}|}"|
\end{center}
%
Here \textit{target} is the name of the output file,
\textit{main} is the name of the main file
and \textit{dest} is the name of the main or child file to be processed
(all filenames without extensions).
The optional argument \textit{main} can be omitted
if \textit{main} matches \textit{dest}.
Optionally, compilation \textit{flags} can be defined via |\def| commands.
This command line makes the \TeX{} engine believe
it is compiling the file \textit{target}
whose content is specified as the latter parameter.
The provided code then forwards the processing to
\textit{main} or \textit{dest} as described in \secref{sec:forward}.

%%%%%%%%%%%%%%%%%%%%%%%%%%%%%%%%%%%%%%%%%%%%%%%%%%%%%%%%%%%%%%%%%%%%%%%%%%%%%%%%
\subsection{Include by Input}
\label{sec:input}

Including child documents by |\include| has some restrictions by design.
Most notably, the content of a child document always occupies
its own set of pages; pages cannot be shared between child documents.
Usually, this behaviour makes perfect sense
because each child document contain an essential part of the document.
However, in some situations it may be desirable to compose
a document from a collection of parts
without having mandatory page breaks between then.
For this case, the package
provides a mechanism to include parts
by |\input| which can also be processed individually.
However, by construction this mechanism
requires manual handling of the content to be output.

%%%%%%%%%%%%%%%%%%%%%%%%%%%%%%%%%%%%%%%%
\DescribeMacro{\ifchilddocmanual}
The main file should be prepared as usual, see \secref{sec:include}.
However, the document body must make a distinction
between processing of an individual part and of the main document, e.g.:
%
\begin{center}
\begin{tabular}{l}
|\ifchilddocmanual|\\
|\input{\childdocname}|\\
|\||else|\\
\textit{document body with }|\input{|\textit{part}|}|\\
|\||fi|
\end{tabular}
\end{center}
%
The conditional |\ifchilddocmanual| is true whenever
a part to be included by |\input| is being compiled,
and the name of the part is stored in |\childdocname|.

%%%%%%%%%%%%%%%%%%%%%%%%%%%%%%%%%%%%%%%%
\DescribeMacro{\childdocby}
Each part to be included by |\input| should start with:
%
\begin{center}
\begin{tabular}{l}
|\input{childdoc.def}|\\
|\childdocby{|\textit{main}|}|\\
\end{tabular}
\end{center}
%
The directive |\childdocby| is similar to |\childdocof|
described in \secref{sec:include},
but the subsequent selection of content must be done manually.
To that end, both |\ifchilddoc| and |\ifchilddocmanual|
will be true upon processing of a part,
and the name of the part is stored in |\childdocname|.
Note that |\jobname| will be set to the filename of the current part
so that each part receives an individual |.aux| file
that does not interfere with the |.aux| file(s) of the main document.
This behaviour can be altered by the alternative form
|\childdocby[*]{|\textit{main}|}| (with a non-empty optional argument)
which uses the |.aux| file of the main document
by setting |\jobname| to \textit{main}.

%%%%%%%%%%%%%%%%%%%%%%%%%%%%%%%%%%%%%%%%%%%%%%%%%%%%%%%%%%%%%%%%%%%%%%%%%%%%%%%%
\subsection{Driver Development}
\label{sec:driver}

The \textsf{childdoc} mechanism can also be use for the development
of definition files such as \LaTeX{} styles or classes.
This case differs from the above setup with multiple parts
included by |\include| in that no |\includeonly| should be invoked.
This can be achieved by starting the include file
(before |\ProvidesPackage|) with:
%
\begin{center}
\begin{tabular}{l}
|\input{childdoc.def}|\\
|\childdocforward{|\textit{main}|}|\\
\end{tabular}
\end{center}
%
or alternatively with:
%
\begin{center}
\begin{tabular}{l}
|\input{childdoc.def}|\\
|\childdocby{|\textit{main}|}|\\
\end{tabular}
\end{center}
%
Both forms have slightly different effects as described above.
The main file is prepared as usual, see \secref{sec:include}.

%%%%%%%%%%%%%%%%%%%%%%%%%%%%%%%%%%%%%%%%%%%%%%%%%%%%%%%%%%%%%%%%%%%%%%%%%%%%%%%%
\subsection{Legacy Detection}
\label{sec:detection}

The directive |\childdocmain| in the main file can detect
whether the complete document or merely a child is to be compiled
even without using the directive |\childdocof|.
This method is deprecated because it is less robust
and there is no compelling reason to use it;
it is merely provided for backward compatibility
and it may be removed in future versions.

If the detection mechanism is to be used,
it is mandatory to correctly specify
the filename of the main file as the argument of |\childdocmain|:
%
\begin{center}
\begin{tabular}{l}
|\input{childdoc.def}|\\
|\childdocmain{|\textit{main}|}|\\
\end{tabular}
\end{center}
%
If |\jobname| does not match the argument \textit{main} of |\childdocmain|,
it is assumed that |\jobname| points to the child file to be compiled.
When using |\childdocmain| with the main file specified as argument,
it suffices to start a child file
with just |\input{|\textit{main}|}|
without loading of the package and using |\childdocof|.
If instead all processing is done
with the appropriate \textsf{childdoc} directives,
the argument of \textit{main} of |\childdocmain| can be empty.

An alternative version of the command line processing described
in \secref{sec:commandline} using the detection mechanism reads:
%
\begin{center}
|... -jobname "|\textit{target}|" "|[\textit{flags}]%
[|\def\jobname{|\textit{dest}|}|]|\input{|\textit{main}|}"|
\end{center}

%%%%%%%%%%%%%%%%%%%%%%%%%%%%%%%%%%%%%%%%%%%%%%%%%%%%%%%%%%%%%%%%%%%%%%%%%%%%%%%%
\subsection{Manual Code}
\label{sec:manual}

In case one cannot be certain whether the definitions file |childdoc.def|
is installed on the target \TeX{} distribution
and one prefers not to ship it,
it is conceivable to paste a few relevant commands into the sources.

To that end, drop all statements |\input{childdoc.def}|
and perform the replacements as outlined below.
Instead of |\childdocmain{|\textit{main}|}| add the following code
to the top of the main file:
%
\begin{center}
\begin{tabular}{l}
|\||ifdefined\childdocname\endinput\||fi\newif\ifchilddoc|\\
|\edef\childdocname{\scantokens\expandafter{\jobname\noexpand}}|\\
|\def\childdocmain{|\textit{main}|}\||ifx\childdocmain\childdocname\||else|\\
|\childdoctrue\includeonly{\childdocname}\let\jobname\childdocmain\||fi|\\
\end{tabular}
\end{center}
%
Instead of |\childdocof{|\textit{main}|}| just include the main file
at the top of each child file:
%
\begin{center}
|\input{|\textit{main}|}|
\end{center}
%
A simple redirection |\childdocforward{|\textit{dest}|}| is achieved by:
%
\begin{center}
|\def\jobname{|\textit{dest}|}\input{\jobname}|
\end{center}
%
The redirection with prefix
|\childdocforwardprefix[|\textit{prefix}|]{|\textit{dest}|}|
is accomplished by:
%
\begin{center}
\begin{tabular}{l}
|{\edef\jobname{\scantokens\expandafter{\jobname\noexpand}}|\\
|\def\redirectjob |\textit{prefix}|#1~~~{\gdef\jobname{|\textit{dest}|#1}}|\\
|\expandafter\redirectjob\jobname~~~}\input{\jobname}|
\end{tabular}
\end{center}

In an alternative approach,
child documents can be compiled by a specific command line
without additional code or specific definitions:
%
\begin{center}
|... -jobname "|\textit{target}|" "|[\textit{flags}]%
|\includeonly{|\textit{dest}|}\input{|\textit{main}|}"|
\end{center}
%

%%%%%%%%%%%%%%%%%%%%%%%%%%%%%%%%%%%%%%%%%%%%%%%%%%%%%%%%%%%%%%%%%%%%%%%%%%%%%%%%
%%%%%%%%%%%%%%%%%%%%%%%%%%%%%%%%%%%%%%%%%%%%%%%%%%%%%%%%%%%%%%%%%%%%%%%%%%%%%%%%
\section{Information}

%%%%%%%%%%%%%%%%%%%%%%%%%%%%%%%%%%%%%%%%%%%%%%%%%%%%%%%%%%%%%%%%%%%%%%%%%%%%%%%%
\subsection{Copyright}

Copyright \copyright{} 2017--2018 Niklas Beisert

This work may be distributed and/or modified under the
conditions of the \LaTeX{} Project Public License, either version 1.3
of this license or (at your option) any later version.
The latest version of this license is in
  \url{http://www.latex-project.org/lppl.txt}
and version 1.3 or later is part of all distributions of \LaTeX{}
version 2005/12/01 or later.

This work has the LPPL maintenance status `maintained'.

The Current Maintainer of this work is Niklas Beisert.

This work consists of the files |README.txt|, |childdoc.ins| and |childdoc.dtx|
as well as the derived files |childdoc.def|, |cdocsamp.tex|
with |cdocsch1.tex|, |cdocsch2.tex|, |cdocspt3.tex|, |cdocspt4.tex|,
|cdocsdrf.tex|, |cdocsfn1.tex|, |cdocsfn2.tex|
as well as |childdoc.pdf|.

%%%%%%%%%%%%%%%%%%%%%%%%%%%%%%%%%%%%%%%%%%%%%%%%%%%%%%%%%%%%%%%%%%%%%%%%%%%%%%%%
\subsection{Files and Installation}

The package consists of the files:
%
\begin{center}
\begin{tabular}{ll}
    |README.txt|   & readme file \\
    |childdoc.ins| & installation file \\
    |childdoc.dtx| & source file \\
    |childdoc.def| & definition file \\
    |cdocsamp.tex| & sample main file \\
    |cdocsch1.tex| & sample include file \\
    |cdocsch2.tex| & sample include file \\
    |cdocspt3.tex| & sample part file \\
    |cdocspt4.tex| & sample part file \\
    |cdocsdrf.tex| & sample redirection file \\
    |cdocsfn1.tex| & sample redirection file \\
    |cdocsfn2.tex| & sample redirection file \\
    |childdoc.pdf| & manual
\end{tabular}
\end{center}
%
The distribution consists of the files
|README.txt|, |childdoc.ins| and |childdoc.dtx|.
%
\begin{itemize}
\item
Run (pdf)\LaTeX{} on |childdoc.dtx|
to compile the manual |childdoc.pdf| (this file).
\item
Run \LaTeX{} on |childdoc.ins| to create the definitions file |childdoc.def|
and the sample |cdocsamp.tex| with include files
|cdocsch1.tex|, |cdocsch2.tex|, |cdocspt3.tex|, |cdocspt4.tex|,
|cdocsdrf.tex|, |cdocsfn1.tex|, |cdocsfn2.tex|.
Then copy the file |childdoc.def| to an appropriate directory of your \LaTeX{}
distribution, e.g.\ \textit{texmf-root}|/tex/latex/childdoc|.
\end{itemize}

%%%%%%%%%%%%%%%%%%%%%%%%%%%%%%%%%%%%%%%%%%%%%%%%%%%%%%%%%%%%%%%%%%%%%%%%%%%%%%%%
\subsection{Related CTAN Packages}

There are several other packages which offer a similar functionality:
%
\begin{itemize}
\item
The packages
\href{http://ctan.org/pkg/docmute}{\textsf{docmute}},
\href{http://ctan.org/pkg/includex}{\textsf{includex}} and
\href{http://ctan.org/pkg/standalone}{\textsf{standalone}}
provide commands to include only the document body of
a child file thus allowing both files to be compiled individually.
\item
The packages \href{http://ctan.org/pkg/subdocs}{\textsf{subdocs}}
and \href{http://ctan.org/pkg/subfiles}{\textsf{subfiles}}
provide structures in which the main and child documents can be
encapsulated and allowing them to be compiled individually.
The inclusion mechanism is different from the conventional |\include|.
\item
The package \href{http://ctan.org/pkg/combine}{\textsf{combine}}
is an elaborate solution to combine several documents into one.
\end{itemize}
%
See also the CTAN topic \href{http://ctan.org/topic/subdocs}{\textsf{subdocs}}
for further related packages.
The present package differs from the above solutions in that
a document structure constructed with the conventional |\include| mechanism
just needs two extra commands at the top of every file
such that all constituent files can be compiled individually.

%%%%%%%%%%%%%%%%%%%%%%%%%%%%%%%%%%%%%%%%%%%%%%%%%%%%%%%%%%%%%%%%%%%%%%%%%%%%%%%%
%\subsection{Feature Suggestions}
%
%The following is a list of features which may be useful for future
%versions of this package:
%%
%\begin{itemize}
%\item
%\ldots
%\end{itemize}

%%%%%%%%%%%%%%%%%%%%%%%%%%%%%%%%%%%%%%%%%%%%%%%%%%%%%%%%%%%%%%%%%%%%%%%%%%%%%%%%
\subsection{Revision History}

%%%%%%%%%%%%%%%%%%%%%%%%%%%%%%%%%%%%%%%%
\paragraph{v2.0:} 2018/12/30

\begin{itemize}
\item
immediate forward processing
\item
added |\childdocby| mechanism
\item
manual restructured
\end{itemize}

%%%%%%%%%%%%%%%%%%%%%%%%%%%%%%%%%%%%%%%%
\paragraph{v1.6:} 2018/01/17

\begin{itemize}
\item
application for development of include files
\item
corrections to manual
\end{itemize}

%%%%%%%%%%%%%%%%%%%%%%%%%%%%%%%%%%%%%%%%
\paragraph{v1.5:} 2017/05/21

\begin{itemize}
\item
more complete structuring introduced
\item
|\childdocof| introduced
\item
|\childdoc| renamed to |\childdocmain|
\item
|\childredirect| renamed to |\childdocforward| and |\childdocforwardprefix|
and functionality expanded
\end{itemize}

%%%%%%%%%%%%%%%%%%%%%%%%%%%%%%%%%%%%%%%%
\paragraph{v1.0:} 2017/04/27

\begin{itemize}
\item
manual and install package
\item
first version published on CTAN
\end{itemize}

%%%%%%%%%%%%%%%%%%%%%%%%%%%%%%%%%%%%%%%%
\paragraph{v0.6:} 2017/04/26

\begin{itemize}
\item
redirection mechanism added
\end{itemize}

%%%%%%%%%%%%%%%%%%%%%%%%%%%%%%%%%%%%%%%%
\paragraph{v0.5:} 2017/04/26

\begin{itemize}
\item
functionality in definition file
\end{itemize}


%%%%%%%%%%%%%%%%%%%%%%%%%%%%%%%%%%%%%%%%%%%%%%%%%%%%%%%%%%%%%%%%%%%%%%%%%%%%%%%%
%%%%%%%%%%%%%%%%%%%%%%%%%%%%%%%%%%%%%%%%%%%%%%%%%%%%%%%%%%%%%%%%%%%%%%%%%%%%%%%%
%%%%%%%%%%%%%%%%%%%%%%%%%%%%%%%%%%%%%%%%%%%%%%%%%%%%%%%%%%%%%%%%%%%%%%%%%%%%%%%%
\appendix

\settowidth\MacroIndent{\rmfamily\scriptsize 000\ }

 \DocInput{childdoc.dtx}

\end{document}
%</driver>
% \fi
%
% %%%%%%%%%%%%%%%%%%%%%%%%%%%%%%%%%%%%%%%%%%%%%%%%%%%%%%%%%%%%%%%%%%%%%%%%%%%%%%
% %%%%%%%%%%%%%%%%%%%%%%%%%%%%%%%%%%%%%%%%%%%%%%%%%%%%%%%%%%%%%%%%%%%%%%%%%%%%%%
% \section{Sample}
%\iffalse
%<*samplemain>
%\fi
%
% The following presents a sample document
% with two chapters, two parts, a title page,
% a compile flag as well as three forwarding files to set the flag.
% It consists of eight |.tex| files:
% \begin{center}
% \begin{tabular}{ll}
% |cdocsamp.tex|&main file\\
% |cdocsch1.tex|&include file for chapter 1\\
% |cdocsch2.tex|&include file for chapter 2\\
% |cdocspt3.tex|&include file for part 3\\
% |cdocspt4.tex|&include file for part 4\\
% |cdocsdrf.tex|&forwarding file for main file in draft mode\\
% |cdocsfi1.tex|&forwarding file for final version of chapter 1\\
% |cdocsfi2.tex|&forwarding file for final version of chapter 2\\
% \end{tabular}
% \end{center}
% Each of the eight files can be compiled directly by the \LaTeX{} compiler.
%
% %%%%%%%%%%%%%%%%%%%%%%%%%%%%%%%%%%%%%%
% \paragraph{Main File.}
%
% The main file is called |cdocsamp.tex|.
%
% Load the \textsf{childdoc} definitions and
% declare the filename for the main document:
%    \begin{macrocode}
\input{childdoc.def}
\childdocmain{}
%    \end{macrocode}

% Optional override for |\version| flag:
%    \begin{macrocode}
%%\ifchilddoc\else\providecommand{\version}{draft}\fi
%    \end{macrocode}

% Define the default values for the |\version| flag
% (|final| for the main file and |draft| for childs):
%    \begin{macrocode}
\ifchilddoc
\providecommand{\version}{draft}
\else
\providecommand{\version}{final}
\fi
%    \end{macrocode}

% Load the standard document class:
%    \begin{macrocode}
\documentclass[12pt]{article}
%    \end{macrocode}

% Start the document body:
%    \begin{macrocode}
\begin{document}
%    \end{macrocode}

% Declare a title page.
% Print title, part of document being processed and version flag:
%    \begin{macrocode}
\addtocounter{page}{-1}
\begin{center}
{\LARGE\bfseries{}childdoc example\par}
\vspace{1cm}
\ifchilddoc
\ifchilddocmanual part\else chapter\fi:
`\childdocname' of `\childdocjob'\par
\else
main document: `\childdocjob'\par
\fi
version: \version\par
\end{center}
\newpage
%    \end{macrocode}

% Manually include selected file,
% otherwise process as usual:
%    \begin{macrocode}
\ifchilddocmanual
\section*{part `\childdocname'}
\input{\childdocname}
\else
%    \end{macrocode}

% Include the two chapters:
%    \begin{macrocode}
\include{cdocsch1}
\include{cdocsch2}
%    \end{macrocode}

% Include the two parts unless only chapters should be displayed:
%    \begin{macrocode}
\ifchilddoc\else
\section{part three}
\input{cdocspt3}
\section{part four}
\input{cdocspt4}
\fi
%    \end{macrocode}

% Process as usual until here:
%    \begin{macrocode}
\fi
%    \end{macrocode}

% End of document body:
%    \begin{macrocode}
\end{document}
%    \end{macrocode}
%\iffalse
%</samplemain>
%\fi
%
% %%%%%%%%%%%%%%%%%%%%%%%%%%%%%%%%%%%%%%
% \paragraph{Chapter Include Files.}
%
% The include files are called |cdocsch1.tex| and |cdocsch2.tex|.
%
%\iffalse
%<*samplechap1|samplechap2>
%\fi

% Optional override for |\version| flag:
%    \begin{macrocode}
%%\providecommand{\version}{final}
%    \end{macrocode}

% Include the main document:
%    \begin{macrocode}
\input{childdoc.def}
\childdocof{cdocsamp}
%    \end{macrocode}

%\iffalse
%</samplechap1|samplechap2>
%\fi
%
%\iffalse
%<*samplechap1>
%\fi
% Some text for chapter 1:
%    \begin{macrocode}
\section{one}
some text in chapter one
%    \end{macrocode}

%\iffalse
%</samplechap1>
%\fi
% Some text for chapter 2:
%\iffalse
%<*samplechap2>
%\fi
%    \begin{macrocode}
\section{two}
more text in chapter two
%    \end{macrocode}

%\iffalse
%</samplechap2>
%\fi
%
% %%%%%%%%%%%%%%%%%%%%%%%%%%%%%%%%%%%%%%
% \paragraph{Part Include Files.}
%
% The include files are called |cdocspt3.tex| and |cdocspt4.tex|.
%
%\iffalse
%<*samplepart3|samplepart4>
%\fi

% Optional override for |\version| flag:
%    \begin{macrocode}
%%\providecommand{\version}{final}
%    \end{macrocode}

% Include the main document:
%    \begin{macrocode}
\input{childdoc.def}
\childdocby{cdocsamp}
%    \end{macrocode}

%\iffalse
%</samplepart3|samplepart4>
%\fi
%
%\iffalse
%<*samplepart3>
%\fi
% Some text for part 3:
%    \begin{macrocode}
some text in part three
%    \end{macrocode}

%\iffalse
%</samplepart3>
%\fi
% Some text for part 4:
%\iffalse
%<*samplepart4>
%\fi
%    \begin{macrocode}
more text in part four
%    \end{macrocode}

%\iffalse
%</samplepart4>
%\fi
%
% %%%%%%%%%%%%%%%%%%%%%%%%%%%%%%%%%%%%%%
% \paragraph{Forwarding for a Complete Draft.}
%
% The following forwarding file |cdocsdrf.tex|
% compiles the main document in draft mode:
%\iffalse
%<*sampledraft>
%\fi
%    \begin{macrocode}
\def\version{draft}
\input{childdoc.def}
\childdocforward{cdocsamp}
%    \end{macrocode}

%\iffalse
%</sampledraft>
%\fi
%
% %%%%%%%%%%%%%%%%%%%%%%%%%%%%%%%%%%%%%%
% \paragraph{Forwarding for Final Version of the Chapters.}
%
% The following forwarding files |cdocsfn1.tex| and |cdocsfn2.tex|
% (with identical content)
% compile the final versions of the child documents
% |cdocsch1.tex| and |cdocsch2.tex|, respectively:
%\iffalse
%<*samplefinal>
%\fi
%    \begin{macrocode}
\def\version{final}
\input{childdoc.def}
\childdocforwardprefix[cdocsamp]{cdocsfn}{cdocsch}
%    \end{macrocode}

%\iffalse
%</samplefinal>
%\fi
%
% %%%%%%%%%%%%%%%%%%%%%%%%%%%%%%%%%%%%%%
% \paragraph{Command Line Processing.}
%
% The following three command lines generate the output files
% |cdocscld|, |cdocscl1| and |cdocscl2|
% which should be identical to
% |cdocsdrf|, |cdocsch1| and |cdocsfn2|, respectively:
% \begin{center}
% \begin{tabular}{l}
% |latex -jobname cdocscld \|\\
% |  "\def\version{draft}\input{childdoc.def}\childdocforward{cdocsamp}"|\\
% |latex -jobname cdocscl1 \|\\
% |  "\input{childdoc.def}\childdocforward[cdocsamp]{cdocsch1}"|\\
% |latex -jobname cdocscl2 \|\\
% |  "\def\version{final}\input{childdoc.def}\childdocforward{cdocsch2}"|
% \end{tabular}
% \end{center}
% Note that the trailing backslash on each first line
% merely continues the input to the second line
% (for convenient cut ant paste).
% Furthermore, the command |latex| can be replaced by any
% of its alternative versions such as |pdflatex|.
%
% %%%%%%%%%%%%%%%%%%%%%%%%%%%%%%%%%%%%%%%%%%%%%%%%%%%%%%%%%%%%%%%%%%%%%%%%%%%%%%
% %%%%%%%%%%%%%%%%%%%%%%%%%%%%%%%%%%%%%%%%%%%%%%%%%%%%%%%%%%%%%%%%%%%%%%%%%%%%%%
% \section{Implementation}
%\iffalse
%<*package>
%\fi
%
% This section describes the definitions file |childdoc.def|.

% The definitions cannot be loaded using |\usepackage| or |\RequirePackage|
% which has a mechanism to prevent loading a style file more than once.
% When loading the definitions by means of |\input|
% multiple instances have to be prevented manually:
%\iffalse
%This code needs to be before the `\ProvidesFile' directive
%which is defined at the beginning of this file.
%Therefore it is also placed there and commented out here.
%</package>
%<*discard>
%\fi
%    \begin{macrocode}
\ifdefined\childdocmain\endinput\fi
%    \end{macrocode}
%\iffalse
%</discard>
%<*package>
%\fi
%
% \macro{\ifchilddoc}
% \macro{\ifchilddocmanual}
% The conditional |\ifchilddoc| tells whether a
% child (true) or main (false) document is being compiled.
% The conditional |\ifchilddocmanual| tells whether
% the |\includeonly| mechanism is used (false) or
% the selection of child files must be performed manually (true).
% The definitions initialise to false:
%    \begin{macrocode}
\newif\ifchilddoc
\newif\ifchilddocmanual
%    \end{macrocode}

% \macro{\childdocname}
% \macro{\childdocjob}
% The macro |\childdocname| stores the name of the main document
% to be compiled. The macro |\childdocjob| stores the name of
% the document on which the \LaTeX{} compiler was originally invoked.
% The content of |\jobname| cannot be compared
% to filenames specified in the source due to different catcodes.
% The following code rescans |\jobname|, stores the result
% in |\childdocname| and saves a copy in |\childdocjob|:
%    \begin{macrocode}
\edef\childdocname{\scantokens\expandafter{\jobname\noexpand}}
\let\childdocjob\childdocname
%    \end{macrocode}

% \macro{\childdocdisable}
% The macro |\childdocdisable| prevents the main file
% from being processed more than once.
% At this stage, the main document command |\childdocmain|
% is assumed to be called once again where it should do nothing.
% Any subsequent call to it should prevent
% a secondary processing of the main document
% It overwrites the forwarding commands
% |\childdocof| and |\childdocforward|
% with empty macros to prevent further inclusions of the main document:
%    \begin{macrocode}
\newcommand{\childdocdisable}
{
  \renewcommand{\childdocmain}[1]{\renewcommand{\childdocmain}[1]{\endinput}}
  \renewcommand{\childdocof}[1]{}
  \renewcommand{\childdocby}[2][]{}
  \renewcommand{\childdocforward}[2][]{}
  \renewcommand{\childdocdisable}{}
}
%    \end{macrocode}

% \macro{\childdocmain}
% The macro |\childdocmain| is to be called at the top of the main file
% with nothing or the main filename (without extension) as argument.
% First, it breaks loops.
% If the argument is not empty and does not match |\childdocname|
% (which is set by the first inclusion of |childdoc.def|),
% |\ifchilddoc| is set to true, |\includeonly| is applied to the child file
% and |\jobname| is set to the main file
% (for proper handling of |.aux| files):
%    \begin{macrocode}
\newcommand{\childdocmain}[1]
{
  \childdocdisable\childdocmain{}
  \if?#1?\else
    \begingroup
      \def\childdoctmp{#1}
      \ifx\childdoctmp\childdocname
        \def\childdoctmp{}
      \else
        \def\childdoctmp
        {
          \childdoctrue
          \includeonly{\childdocname}
          \def\childdocjob{#1}
          \def\jobname{#1}
        }
      \fi
      \expandafter
    \endgroup
    \childdoctmp
  \fi
}
%    \end{macrocode}

% \macro{\childdocof}
% The command |\childdocof| redirects
% compilation to the main file |#1|.
%    \begin{macrocode}
\newcommand{\childdocof}[1]
{
  \childdocdisable
  \childdoctrue
  \includeonly{\childdocname}
  \def\jobname{#1}
  \def\childdocjob{#1}
  \input{#1}
}
%    \end{macrocode}

% \macro{\childdocby}
% The command |\childdocby| ....
%    \begin{macrocode}
\newcommand{\childdocby}[2][]
{
  \childdocdisable
  \childdoctrue
  \childdocmanualtrue
  \if?#1?\else
    \def\jobname{#2}
  \fi
  \def\childdocjob{#2}
  \input{#2}
  \endinput
}
%    \end{macrocode}

% \macro{\childdocforward}
% The command |\childdocforward| redirects
% compilation to the main file or
% (if the optional argument is given) a child file.
% Parameters are set as if the main file
% or a child file starting with |\childdocof| was compiled.
% Then compilation is handed over to the main file:
%    \begin{macrocode}
\newcommand{\childdocforward}[2][]
{
  \begingroup
    \if?#1?
      \def\childdoctmp
      {
        \def\childdocname{#2}
        \def\childdocjob{#2}
        \def\jobname{#2}
        \input{#2}
        \endinput
      }
    \else
      \def\childdoctmp
      {
        \childdocdisable
        \def\childdocname{#2}
        \childdoctrue
        \includeonly{#2}
        \def\childdocjob{#1}
        \def\jobname{#1}
        \input{#1}
        \endinput
      }
    \fi
    \expandafter
  \endgroup
  \childdoctmp
}
%    \end{macrocode}

% \macro{\childdocforwardprefix}
% The command |\childdocforwardprefix| redirects
% compilation to the main or a child file by means of a pattern.
% The prefix |#1| in the current filename is replaced by |#2|
% and the suffix of the current filename is kept
% (it is assumed that the filename does not contain the substring `|~~~|'
% which is used as a delimiter).
% Compilation is handed over to the new file by |\childdocforward|:
%    \begin{macrocode}
\newcommand{\childdocforwardprefix}[3][]
{
  \begingroup
    \def\childdocextract #2##1~~~{\def\childdoctmp{\childdocforward[#1]{#3##1}}}
    \expandafter\childdocextract\childdocname~~~
    \expandafter
  \endgroup
  \childdoctmp
}
%    \end{macrocode}

% \macro{\childdoc}
% The deprecated macro |\childdoc| is a legacy version of |\childdocmain|:
%    \begin{macrocode}
\newcommand{\childdoc}{\childdocmain}
%    \end{macrocode}

% \macro{\childdocredirect}
% The deprecated macro |\childdocredirect| is a legacy version
% of |\childdocforward| and |\childdocforwardprefix|:
%    \begin{macrocode}
\newcommand{\childdocredirect}[2][]
{
  \begingroup
    \if?#1?
      \def\childdoctmp{\childdocforward{#2}}
    \else
      \def\childdoctmp{\childdocforwardprefix{#1}{#2}}
    \fi
    \expandafter
  \endgroup
  \childdoctmp
}
%    \end{macrocode}

%\iffalse
%</package>
%\fi
%
\endinput
\childdocforward[|\textit{main}|]{|\textit{dest}|}"|
\end{center}
%
Here \textit{target} is the name of the output file,
\textit{main} is the name of the main file
and \textit{dest} is the name of the main or child file to be processed
(all filenames without extensions).
The optional argument \textit{main} can be omitted
if \textit{main} matches \textit{dest}.
Optionally, compilation \textit{flags} can be defined via |\def| commands.
This command line makes the \TeX{} engine believe
it is compiling the file \textit{target}
whose content is specified as the latter parameter.
The provided code then forwards the processing to
\textit{main} or \textit{dest} as described in \secref{sec:forward}.

%%%%%%%%%%%%%%%%%%%%%%%%%%%%%%%%%%%%%%%%%%%%%%%%%%%%%%%%%%%%%%%%%%%%%%%%%%%%%%%%
\subsection{Include by Input}
\label{sec:input}

Including child documents by |\include| has some restrictions by design.
Most notably, the content of a child document always occupies
its own set of pages; pages cannot be shared between child documents.
Usually, this behaviour makes perfect sense
because each child document contain an essential part of the document.
However, in some situations it may be desirable to compose
a document from a collection of parts
without having mandatory page breaks between then.
For this case, the package
provides a mechanism to include parts
by |\input| which can also be processed individually.
However, by construction this mechanism
requires manual handling of the content to be output.

%%%%%%%%%%%%%%%%%%%%%%%%%%%%%%%%%%%%%%%%
\DescribeMacro{\ifchilddocmanual}
The main file should be prepared as usual, see \secref{sec:include}.
However, the document body must make a distinction
between processing of an individual part and of the main document, e.g.:
%
\begin{center}
\begin{tabular}{l}
|\ifchilddocmanual|\\
|\input{\childdocname}|\\
|\||else|\\
\textit{document body with }|\input{|\textit{part}|}|\\
|\||fi|
\end{tabular}
\end{center}
%
The conditional |\ifchilddocmanual| is true whenever
a part to be included by |\input| is being compiled,
and the name of the part is stored in |\childdocname|.

%%%%%%%%%%%%%%%%%%%%%%%%%%%%%%%%%%%%%%%%
\DescribeMacro{\childdocby}
Each part to be included by |\input| should start with:
%
\begin{center}
\begin{tabular}{l}
|% \iffalse
%
% childdoc.dtx Copyright (C) 2017-2018 Niklas Beisert
%
% This work may be distributed and/or modified under the
% conditions of the LaTeX Project Public License, either version 1.3
% of this license or (at your option) any later version.
% The latest version of this license is in
%   http://www.latex-project.org/lppl.txt
% and version 1.3 or later is part of all distributions of LaTeX
% version 2005/12/01 or later.
%
% This work has the LPPL maintenance status `maintained'.
%
% The Current Maintainer of this work is Niklas Beisert.
%
% This work consists of the files childdoc.dtx and childdoc.ins
% and the derived files childdoc.def and cdocsamp.tex with
% cdocsch1.tex, cdocsch2.tex, cdocsdrf.tex, cdocsfn1.tex, cdocsfn2.tex.
%
%<package>\ifdefined\childdocmain\endinput\fi
%<package>\ProvidesFile{childdoc.def}[2018/12/30 v2.0 child document driver]
%<samplemain>\ProvidesFile{cdocsamp.tex}[2018/12/30 v2.0 sample for childdoc]
%<*driver>
%\ProvidesFile{childdoc.drv}[2018/12/30 v2.0 childdoc reference manual file]
\PassOptionsToClass{10pt,a4paper}{article}
\documentclass{ltxdoc}

\usepackage[margin=35mm]{geometry}
\usepackage{hyperref}
\usepackage{hyperxmp}
\usepackage[usenames]{color}

\hypersetup{colorlinks=true}
\hypersetup{pdfstartview=FitH}
\hypersetup{pdfpagemode=UseNone}
\hypersetup{pdfsource={}}
\hypersetup{pdflang={en-UK}}
\hypersetup{pdfcopyright={Copyright 2017-2018 Niklas Beisert.
  This work may be distributed and/or modified under the
  conditions of the LaTeX Project Public License, either version 1.3
  of this license or (at your option) any later version.}}
\hypersetup{pdflicenseurl={http://www.latex-project.org/lppl.txt}}
\hypersetup{pdfcontactaddress={ETH Zurich, ITP, HIT K,
  Wolfgang-Pauli-Strasse 27}}
\hypersetup{pdfcontactpostcode={8093}}
\hypersetup{pdfcontactcity={Zurich}}
\hypersetup{pdfcontactcountry={Switzerland}}
\hypersetup{pdfcontactemail={nbeisert@itp.phys.ethz.ch}}
\hypersetup{pdfcontacturl={http://people.phys.ethz.ch/\xmptilde nbeisert/}}

\newcommand{\secref}[1]{\hyperref[#1]{section \ref*{#1}}}

\parskip1ex
\parindent0pt
\let\olditemize\itemize
\def\itemize{\olditemize\parskip0pt}

\begin{document}

\title{The \textsf{childdoc} Package}
\hypersetup{pdftitle={The childdoc Package}}
\author{Niklas Beisert\\[2ex]
  Institut f\"ur Theoretische Physik\\
  Eidgen\"ossische Technische Hochschule Z\"urich\\
  Wolfgang-Pauli-Strasse 27, 8093 Z\"urich, Switzerland\\[1ex]
  \href{mailto:nbeisert@itp.phys.ethz.ch}
  {\texttt{nbeisert@itp.phys.ethz.ch}}}
\hypersetup{pdfauthor={Niklas Beisert}}
\hypersetup{pdfsubject={Manual for the LaTeX2e Package childdoc}}
\date{30 December 2018, \textsf{v2.0}}
\maketitle

\begin{abstract}\noindent
\textsf{childdoc} is a \LaTeXe{} package
that enables the direct compilation
of document sections included by |\include|
to individual files.
\end{abstract}

\begingroup
\parskip0ex
\tableofcontents
\endgroup

%%%%%%%%%%%%%%%%%%%%%%%%%%%%%%%%%%%%%%%%%%%%%%%%%%%%%%%%%%%%%%%%%%%%%%%%%%%%%%%%
%%%%%%%%%%%%%%%%%%%%%%%%%%%%%%%%%%%%%%%%%%%%%%%%%%%%%%%%%%%%%%%%%%%%%%%%%%%%%%%%
\section{Introduction}

\LaTeX{} provides a mechanism to structure a large document (such as a book)
into a main file and several child files (containing the chapters)
using the |\include| command.
This mechanism is beneficial for documents
which span hundreds of pages in order to
make the source file(s) more manageable.
Moreover, compilation can be restricted to
selected child files by means of the |\includeonly| command.
The latter feature can be used to reduce the compilation time while editing
(this was significantly more useful in the earlier days of \LaTeX{})
or to generate a smaller document which is easier to navigate.
Another application of |\includeonly| is to generate
documents consisting of selected parts of the complete document.

However, there are a few drawbacks of the plain |\include| mechanism:
\begin{itemize}
\item
The child files cannot be compiled on their own,
they can only be compiled via the main file.
A naive editing environment
(such as a text editor with an option
to have the current file processed by \LaTeX)
may require one to switch to the main file before compiling;
attempting to compile the child file produces errors.
\item
The main file must be modified (each time)
to adjust the |\includeonly| command
to the present needs. This easily leaves the main file in a messy state.
\item
The generated document will always carry the filename
of the main document. This is inconvenient if
several child files are to be compiled and
to be kept for distribution.
\end{itemize}

The present package provides a simple interface
to make child files individually compilable by \LaTeX{}.
Compiling a child file then has the same effect as compiling
the main file with an |\includeonly| command
to select the appropriate child.
Moreover the generated document will carry the name of the child
rather than the main file.
This resolves all three above issues.

This feature is meant to make the editing of books,
thesis documents and lecture notes somewhat more convenient.
However, the package can also be used efficiently for
composing a series of documents (such as exercise sheets)
which are typically distributed individually.
It then assists the author in generating the individual documents
(potentially in different versions)
as well as a document containing the collected series.
Another application is in developing style files
or other kinds of included material
where compilation of the style file could redirect
to a sample or test file.

%%%%%%%%%%%%%%%%%%%%%%%%%%%%%%%%%%%%%%%%%%%%%%%%%%%%%%%%%%%%%%%%%%%%%%%%%%%%%%%%
%%%%%%%%%%%%%%%%%%%%%%%%%%%%%%%%%%%%%%%%%%%%%%%%%%%%%%%%%%%%%%%%%%%%%%%%%%%%%%%%
\section{Usage}

First of all, the package \textsf{childdoc} is \emph{not} a standard
\LaTeXe{} |.sty| style file! Therefore it needs to be invoked in
a non-standard way.

%%%%%%%%%%%%%%%%%%%%%%%%%%%%%%%%%%%%%%%%%%%%%%%%%%%%%%%%%%%%%%%%%%%%%%%%%%%%%%%%
\subsection{Included Files}
\label{sec:include}

%%%%%%%%%%%%%%%%%%%%%%%%%%%%%%%%%%%%%%%%
\DescribeMacro{\childdocmain}
To use the package, add the commands
\begin{center}
\begin{tabular}{l}
|\input{childdoc.def}|\\
|\childdocmain{}|\\
\end{tabular}
\end{center}
at the very top of the main \LaTeX{} file,
in particular \emph{before} the |\documentclass| statement!
The argument of |\childdocmain| should be left empty
(but it must be present).

%%%%%%%%%%%%%%%%%%%%%%%%%%%%%%%%%%%%%%%%
\DescribeMacro{\childdocof}
Furthermore, add the commands
\begin{center}
\begin{tabular}{l}
|\input{childdoc.def}|\\
|\childdocof{|\textit{main}|}|\\
\end{tabular}
\end{center}
at the top of every child file \textit{child}
which is included by |\include{|\textit{child}|}|
from within the main file
(or at least for those files to be compiled individually).
The argument \textit{main} must be the filename of the main file.

There are a couple of
considerations in setting up the main and child documents:

%%%%%%%%%%%%%%%%%%%%%%%%%%%%%%%%%%%%%%%%
\paragraph{Restrictions.}

Please note the following restrictions:
\begin{itemize}
\item
|\childdocmain| must be called with one argument \textit{main}
to ensure compatibility with earlier version of the package.
It must either be empty (|\childdocmain{}|)
or precisely match the filename of the main file in which it is specified.
See \secref{sec:detection} for further information.
\item
The filename \textit{main} must be specified without the |.tex| extension.
\item
The filename \textit{main} is case sensitive
(even in case-insensitive file systems)
due to internal string comparison.
\item
The argument \textit{main} should be fully expanded, it cannot be a macro.
\item
Subdirectories and special characters should be avoided in filenames.
\item
The command |\childdocmain{|\textit{main}|}| must be followed by a whitespace.
It should not be followed immediately by another command
or by a comment mark `|%|'.
This is because the \TeX{} parser reads the token immediately following
the argument of |\childdocmain| and puts it
at the beginning of every child section;
however, a white\-space is ignored.
\end{itemize}

%%%%%%%%%%%%%%%%%%%%%%%%%%%%%%%%%%%%%%%%
\paragraph{Content of Main File.}

It is advisable to place all content in the child files included by |\include|.
Any output contained in the main file will appear in all child documents
unless suppressed manually;
it cannot be suppressed automatically by the |\includeonly| directive
and thus should normally be avoided.
A method to include some content in the main file
by means of conditional processing is described in \secref{sec:conditional}.

%%%%%%%%%%%%%%%%%%%%%%%%%%%%%%%%%%%%%%%%
\paragraph{Page Numbering.}

When only a part of the document is compiled,
the appropriate numbering of pages
(as well as other status parameters)
is determined from the |.aux| files.
The latter contain information from previous passes.
However this information needs to propagate through
all intermediate child documents.
Therefore the page numbering in child documents may well
be inconsistent until the complete document is compiled at least once.

A useful (if unconventional) way to always ensure a consistent
page numbering is to restart the numbering in each child document
and denote the pages by `\textit{child}|.|\textit{page}'
where \textit{child} represents the chapter/section number of the child file.
This can be achieved by the command
|\numberwithin{page}{|\textit{child}|}|
of the \textsf{amsmath} package
where \textit{child} can be |chapter| or |section|
depending on the chosen structuring.
Alternatively, one can modify the macro |\thepage| appropriately
and reset the counter |page| at the start of each child file.

%%%%%%%%%%%%%%%%%%%%%%%%%%%%%%%%%%%%%%%%%%%%%%%%%%%%%%%%%%%%%%%%%%%%%%%%%%%%%%%%
\subsection{Conditional Processing}
\label{sec:conditional}

The package provides a mechanism to compile different versions
of a document. To customise the versions further some conditional processing
can come in handy to distinguish which version is being compiled.
The package provides two macros to describe the compilation context:

%%%%%%%%%%%%%%%%%%%%%%%%%%%%%%%%%%%%%%%%
\DescribeMacro{\ifchilddoc}
The conditional |\ifchilddoc| distinguishes between the compilation of
child documents and the main document:
%
\begin{center}
|\ifchilddoc |\textit{child-code}| |[|\||else |\textit{main-code}]| \||fi|
\end{center}

%%%%%%%%%%%%%%%%%%%%%%%%%%%%%%%%%%%%%%%%
\DescribeMacro{\childdocname}
\DescribeMacro{\childdocjob}
The macro |\childdocname| contains the filename (without extension)
of the main or child file being processed.
Note that |\childdocjob| will always contain the name of the main file.

%%%%%%%%%%%%%%%%%%%%%%%%%%%%%%%%%%%%%%%%
\paragraph{Title Page.}

Conditional processing can be used to include a title or banner page
in the main document when proper precautions are taken.
Importantly, the code in the main file should ensure that the page counter
(as well as other status parameters which are stored in the |.aux| files)
takes the same value after the conditional processing.
Otherwise the page numbers may take divergent values
depending on which part is compiled.

For example, a title page could be declared by:
%
\begin{center}
\begin{tabular}{l}
|\ifchilddoc\||else|\\
|\addtocounter{page}{-1}|\\
\textit{code for title page}\\
|\newpage|\\
|\||fi|
\end{tabular}
\end{center}
%
A banner page for the child documents can be generated by:
%
\begin{center}
\begin{tabular}{l}
|\ifchilddoc|\\
|\addtocounter{page}{-1}|\\
\textit{code for banner page}\\
|\newpage|\\
|\||fi|
\end{tabular}
\end{center}
%
Here one could write a message such as:
\begin{center}
|This is the part \childdocname{} of \childdocjob{}.|
\end{center}

%%%%%%%%%%%%%%%%%%%%%%%%%%%%%%%%%%%%%%%%%%%%%%%%%%%%%%%%%%%%%%%%%%%%%%%%%%%%%%%%
\subsection{Flags}
\label{sec:flags}

The package makes it easy to generate different versions
of the main or child documents.
To this end compilation flags can be defined
and assigned different default values.
They will be particularly useful in conjunction
with the forwarding mechanism described in \secref{sec:forward}.

For example, it may be useful to have a flag |\version|
which can be set to |draft| or |final|.
The document source will contain some conditional code
depending on the value of |\version|.
Suppose further, the flag should default to |final| for the main file
and to |draft| for child files
which is a natural assignment for editing the document.
This is achieved by placing the following code
in the preamble of the main document
(below the |\childdocmain| directive):
%
\begin{center}
\begin{tabular}{l}
|\ifchilddoc|\\
|\providecommand{\version}{draft}|\\
|\||else|\\
|\providecommand{\version}{final}|\\
|\||fi|
\end{tabular}
\end{center}
%
The definition by |\providecommand| makes sure
that previous definitions are not overwritten.
Further statements |\providecommand{\version}{...}|
can thus be added before the above code to override it.

For the main file, one might add a line
(between |\childdocmain| and the above block)
%
\begin{center}
|%\ifchilddoc\||else\providecommand{\version}{draft}\||fi|
\end{center}
%
which can be uncommented to produce a draft version.
Likewise one can add a line to the very top of a child file
(above the |\childdocof{|\textit{main}|}| directive)
%
\begin{center}
|%\providecommand{\version}{final}|
\end{center}
%
which can be uncommented to produce the final version of this child document.

%%%%%%%%%%%%%%%%%%%%%%%%%%%%%%%%%%%%%%%%%%%%%%%%%%%%%%%%%%%%%%%%%%%%%%%%%%%%%%%%
\subsection{Forwarding}
\label{sec:forward}

Different versions of the main or child documents
using compilation flags as described in \secref{sec:flags}
can be (permanently) stored in different files
for convenient compilation, viewing and distribution.
To this end, the package defines a command
to pass on compilation to a different file:

%%%%%%%%%%%%%%%%%%%%%%%%%%%%%%%%%%%%%%%%
\DescribeMacro{\childdocforward}
The command |\childdocforward| redirects processing to
another source file:
%
\begin{center}
\begin{tabular}{l}
|\input{childdoc.def}|\\
|\childdocforward[|\textit{main}|]{|\textit{dest}|}|\\
\end{tabular}
\end{center}
%
The argument \textit{dest} is the destination file
(without extension).
It should be the main file or one of the child files.
Note that further \textsf{childdoc} directives
such as |\childdocof| and |\childdocforward|
in the indicated file will be processed in this form.
The optional argument \textit{main}
passes on directly to the main file \textit{main}
while pretending to compile the child \textit{dest}.
This form behaves as if \textit{dest}
issues |\childdocof{|\textit{main}|}| right away,
and no further \textsf{childdoc} directives will be processed.

%%%%%%%%%%%%%%%%%%%%%%%%%%%%%%%%%%%%%%%%
\DescribeMacro{\...prefix}
In the alternative form |\childdocforwardprefix|,
%
\begin{center}
\begin{tabular}{l}
|\input{childdoc.def}|\\
|\childdocforwardprefix[|\textit{main}|]{|\textit{prefix}|}{|\textit{dest}|}|
\end{tabular}
\end{center}
%
the destination file is determined by a pattern
depending on the current file:
To make this work, the current file must be called
`{\textit{prefix}\hspace{0.2em}\textit{suffix}}'
with \textit{prefix} matching precisely the argument.
Processing is then passed on to the file
`{\textit{dest}\hspace{0.2em}\textit{suffix}}'.
Surely, the same effect is achieved by
directly specifying the
argument `{\textit{dest}\hspace{0.2em}\textit{suffix}}'
in the first form.
However, that requires to set up a different file
for each child. With the alternative form of the command
all these files can have exactly the same content
which simplifies setting them up and maintaining them.

For example, the following file |draft.tex|
with a compilation flag |\version| as described in \secref{sec:flags}
compiles the main document as a draft:
%
\begin{center}
\begin{tabular}{l}
|\def\version{draft}|\\
|\input{childdoc.def}|\\
|\childdocforward{|\textit{main}|}|
\end{tabular}
\end{center}
%
Likewise, the following files |final|\textit{nn}|.tex|
compile the final version of the child document
|child|\textit{nn}|.tex|:
%
\begin{center}
\begin{tabular}{l}
|\def\version{final}|\\
|\input{childdoc.def}|\\
|\childdocforwardprefix{final}{child}|
\end{tabular}
\end{center}
%

Note that when several versions of a main file and/or of each child file
are to be generated, it may be convenient to set up a |Makefile| or
shell script to automatise the process.

%%%%%%%%%%%%%%%%%%%%%%%%%%%%%%%%%%%%%%%%%%%%%%%%%%%%%%%%%%%%%%%%%%%%%%%%%%%%%%%%
\subsection{Command Line Processing}
\label{sec:commandline}

The effect of redirection files can also be achieved by invoking
the \LaTeX{} compiler with a more elaborate command line.
Most conveniently this should be done as part
of a shell script or a |Makefile|.

When using \textsf{childdoc} in the main file, the following
command lines effectively perform a redirection
(note that depending on the shell being used,
backslashes may have to be doubled: `|\|' $\to$ `|\\|'):
%
\begin{center}
|... -jobname "|\textit{target}|" |\\|"|[\textit{flags}]%
|\input{childdoc.def}\childdocforward[|\textit{main}|]{|\textit{dest}|}"|
\end{center}
%
Here \textit{target} is the name of the output file,
\textit{main} is the name of the main file
and \textit{dest} is the name of the main or child file to be processed
(all filenames without extensions).
The optional argument \textit{main} can be omitted
if \textit{main} matches \textit{dest}.
Optionally, compilation \textit{flags} can be defined via |\def| commands.
This command line makes the \TeX{} engine believe
it is compiling the file \textit{target}
whose content is specified as the latter parameter.
The provided code then forwards the processing to
\textit{main} or \textit{dest} as described in \secref{sec:forward}.

%%%%%%%%%%%%%%%%%%%%%%%%%%%%%%%%%%%%%%%%%%%%%%%%%%%%%%%%%%%%%%%%%%%%%%%%%%%%%%%%
\subsection{Include by Input}
\label{sec:input}

Including child documents by |\include| has some restrictions by design.
Most notably, the content of a child document always occupies
its own set of pages; pages cannot be shared between child documents.
Usually, this behaviour makes perfect sense
because each child document contain an essential part of the document.
However, in some situations it may be desirable to compose
a document from a collection of parts
without having mandatory page breaks between then.
For this case, the package
provides a mechanism to include parts
by |\input| which can also be processed individually.
However, by construction this mechanism
requires manual handling of the content to be output.

%%%%%%%%%%%%%%%%%%%%%%%%%%%%%%%%%%%%%%%%
\DescribeMacro{\ifchilddocmanual}
The main file should be prepared as usual, see \secref{sec:include}.
However, the document body must make a distinction
between processing of an individual part and of the main document, e.g.:
%
\begin{center}
\begin{tabular}{l}
|\ifchilddocmanual|\\
|\input{\childdocname}|\\
|\||else|\\
\textit{document body with }|\input{|\textit{part}|}|\\
|\||fi|
\end{tabular}
\end{center}
%
The conditional |\ifchilddocmanual| is true whenever
a part to be included by |\input| is being compiled,
and the name of the part is stored in |\childdocname|.

%%%%%%%%%%%%%%%%%%%%%%%%%%%%%%%%%%%%%%%%
\DescribeMacro{\childdocby}
Each part to be included by |\input| should start with:
%
\begin{center}
\begin{tabular}{l}
|\input{childdoc.def}|\\
|\childdocby{|\textit{main}|}|\\
\end{tabular}
\end{center}
%
The directive |\childdocby| is similar to |\childdocof|
described in \secref{sec:include},
but the subsequent selection of content must be done manually.
To that end, both |\ifchilddoc| and |\ifchilddocmanual|
will be true upon processing of a part,
and the name of the part is stored in |\childdocname|.
Note that |\jobname| will be set to the filename of the current part
so that each part receives an individual |.aux| file
that does not interfere with the |.aux| file(s) of the main document.
This behaviour can be altered by the alternative form
|\childdocby[*]{|\textit{main}|}| (with a non-empty optional argument)
which uses the |.aux| file of the main document
by setting |\jobname| to \textit{main}.

%%%%%%%%%%%%%%%%%%%%%%%%%%%%%%%%%%%%%%%%%%%%%%%%%%%%%%%%%%%%%%%%%%%%%%%%%%%%%%%%
\subsection{Driver Development}
\label{sec:driver}

The \textsf{childdoc} mechanism can also be use for the development
of definition files such as \LaTeX{} styles or classes.
This case differs from the above setup with multiple parts
included by |\include| in that no |\includeonly| should be invoked.
This can be achieved by starting the include file
(before |\ProvidesPackage|) with:
%
\begin{center}
\begin{tabular}{l}
|\input{childdoc.def}|\\
|\childdocforward{|\textit{main}|}|\\
\end{tabular}
\end{center}
%
or alternatively with:
%
\begin{center}
\begin{tabular}{l}
|\input{childdoc.def}|\\
|\childdocby{|\textit{main}|}|\\
\end{tabular}
\end{center}
%
Both forms have slightly different effects as described above.
The main file is prepared as usual, see \secref{sec:include}.

%%%%%%%%%%%%%%%%%%%%%%%%%%%%%%%%%%%%%%%%%%%%%%%%%%%%%%%%%%%%%%%%%%%%%%%%%%%%%%%%
\subsection{Legacy Detection}
\label{sec:detection}

The directive |\childdocmain| in the main file can detect
whether the complete document or merely a child is to be compiled
even without using the directive |\childdocof|.
This method is deprecated because it is less robust
and there is no compelling reason to use it;
it is merely provided for backward compatibility
and it may be removed in future versions.

If the detection mechanism is to be used,
it is mandatory to correctly specify
the filename of the main file as the argument of |\childdocmain|:
%
\begin{center}
\begin{tabular}{l}
|\input{childdoc.def}|\\
|\childdocmain{|\textit{main}|}|\\
\end{tabular}
\end{center}
%
If |\jobname| does not match the argument \textit{main} of |\childdocmain|,
it is assumed that |\jobname| points to the child file to be compiled.
When using |\childdocmain| with the main file specified as argument,
it suffices to start a child file
with just |\input{|\textit{main}|}|
without loading of the package and using |\childdocof|.
If instead all processing is done
with the appropriate \textsf{childdoc} directives,
the argument of \textit{main} of |\childdocmain| can be empty.

An alternative version of the command line processing described
in \secref{sec:commandline} using the detection mechanism reads:
%
\begin{center}
|... -jobname "|\textit{target}|" "|[\textit{flags}]%
[|\def\jobname{|\textit{dest}|}|]|\input{|\textit{main}|}"|
\end{center}

%%%%%%%%%%%%%%%%%%%%%%%%%%%%%%%%%%%%%%%%%%%%%%%%%%%%%%%%%%%%%%%%%%%%%%%%%%%%%%%%
\subsection{Manual Code}
\label{sec:manual}

In case one cannot be certain whether the definitions file |childdoc.def|
is installed on the target \TeX{} distribution
and one prefers not to ship it,
it is conceivable to paste a few relevant commands into the sources.

To that end, drop all statements |\input{childdoc.def}|
and perform the replacements as outlined below.
Instead of |\childdocmain{|\textit{main}|}| add the following code
to the top of the main file:
%
\begin{center}
\begin{tabular}{l}
|\||ifdefined\childdocname\endinput\||fi\newif\ifchilddoc|\\
|\edef\childdocname{\scantokens\expandafter{\jobname\noexpand}}|\\
|\def\childdocmain{|\textit{main}|}\||ifx\childdocmain\childdocname\||else|\\
|\childdoctrue\includeonly{\childdocname}\let\jobname\childdocmain\||fi|\\
\end{tabular}
\end{center}
%
Instead of |\childdocof{|\textit{main}|}| just include the main file
at the top of each child file:
%
\begin{center}
|\input{|\textit{main}|}|
\end{center}
%
A simple redirection |\childdocforward{|\textit{dest}|}| is achieved by:
%
\begin{center}
|\def\jobname{|\textit{dest}|}\input{\jobname}|
\end{center}
%
The redirection with prefix
|\childdocforwardprefix[|\textit{prefix}|]{|\textit{dest}|}|
is accomplished by:
%
\begin{center}
\begin{tabular}{l}
|{\edef\jobname{\scantokens\expandafter{\jobname\noexpand}}|\\
|\def\redirectjob |\textit{prefix}|#1~~~{\gdef\jobname{|\textit{dest}|#1}}|\\
|\expandafter\redirectjob\jobname~~~}\input{\jobname}|
\end{tabular}
\end{center}

In an alternative approach,
child documents can be compiled by a specific command line
without additional code or specific definitions:
%
\begin{center}
|... -jobname "|\textit{target}|" "|[\textit{flags}]%
|\includeonly{|\textit{dest}|}\input{|\textit{main}|}"|
\end{center}
%

%%%%%%%%%%%%%%%%%%%%%%%%%%%%%%%%%%%%%%%%%%%%%%%%%%%%%%%%%%%%%%%%%%%%%%%%%%%%%%%%
%%%%%%%%%%%%%%%%%%%%%%%%%%%%%%%%%%%%%%%%%%%%%%%%%%%%%%%%%%%%%%%%%%%%%%%%%%%%%%%%
\section{Information}

%%%%%%%%%%%%%%%%%%%%%%%%%%%%%%%%%%%%%%%%%%%%%%%%%%%%%%%%%%%%%%%%%%%%%%%%%%%%%%%%
\subsection{Copyright}

Copyright \copyright{} 2017--2018 Niklas Beisert

This work may be distributed and/or modified under the
conditions of the \LaTeX{} Project Public License, either version 1.3
of this license or (at your option) any later version.
The latest version of this license is in
  \url{http://www.latex-project.org/lppl.txt}
and version 1.3 or later is part of all distributions of \LaTeX{}
version 2005/12/01 or later.

This work has the LPPL maintenance status `maintained'.

The Current Maintainer of this work is Niklas Beisert.

This work consists of the files |README.txt|, |childdoc.ins| and |childdoc.dtx|
as well as the derived files |childdoc.def|, |cdocsamp.tex|
with |cdocsch1.tex|, |cdocsch2.tex|, |cdocspt3.tex|, |cdocspt4.tex|,
|cdocsdrf.tex|, |cdocsfn1.tex|, |cdocsfn2.tex|
as well as |childdoc.pdf|.

%%%%%%%%%%%%%%%%%%%%%%%%%%%%%%%%%%%%%%%%%%%%%%%%%%%%%%%%%%%%%%%%%%%%%%%%%%%%%%%%
\subsection{Files and Installation}

The package consists of the files:
%
\begin{center}
\begin{tabular}{ll}
    |README.txt|   & readme file \\
    |childdoc.ins| & installation file \\
    |childdoc.dtx| & source file \\
    |childdoc.def| & definition file \\
    |cdocsamp.tex| & sample main file \\
    |cdocsch1.tex| & sample include file \\
    |cdocsch2.tex| & sample include file \\
    |cdocspt3.tex| & sample part file \\
    |cdocspt4.tex| & sample part file \\
    |cdocsdrf.tex| & sample redirection file \\
    |cdocsfn1.tex| & sample redirection file \\
    |cdocsfn2.tex| & sample redirection file \\
    |childdoc.pdf| & manual
\end{tabular}
\end{center}
%
The distribution consists of the files
|README.txt|, |childdoc.ins| and |childdoc.dtx|.
%
\begin{itemize}
\item
Run (pdf)\LaTeX{} on |childdoc.dtx|
to compile the manual |childdoc.pdf| (this file).
\item
Run \LaTeX{} on |childdoc.ins| to create the definitions file |childdoc.def|
and the sample |cdocsamp.tex| with include files
|cdocsch1.tex|, |cdocsch2.tex|, |cdocspt3.tex|, |cdocspt4.tex|,
|cdocsdrf.tex|, |cdocsfn1.tex|, |cdocsfn2.tex|.
Then copy the file |childdoc.def| to an appropriate directory of your \LaTeX{}
distribution, e.g.\ \textit{texmf-root}|/tex/latex/childdoc|.
\end{itemize}

%%%%%%%%%%%%%%%%%%%%%%%%%%%%%%%%%%%%%%%%%%%%%%%%%%%%%%%%%%%%%%%%%%%%%%%%%%%%%%%%
\subsection{Related CTAN Packages}

There are several other packages which offer a similar functionality:
%
\begin{itemize}
\item
The packages
\href{http://ctan.org/pkg/docmute}{\textsf{docmute}},
\href{http://ctan.org/pkg/includex}{\textsf{includex}} and
\href{http://ctan.org/pkg/standalone}{\textsf{standalone}}
provide commands to include only the document body of
a child file thus allowing both files to be compiled individually.
\item
The packages \href{http://ctan.org/pkg/subdocs}{\textsf{subdocs}}
and \href{http://ctan.org/pkg/subfiles}{\textsf{subfiles}}
provide structures in which the main and child documents can be
encapsulated and allowing them to be compiled individually.
The inclusion mechanism is different from the conventional |\include|.
\item
The package \href{http://ctan.org/pkg/combine}{\textsf{combine}}
is an elaborate solution to combine several documents into one.
\end{itemize}
%
See also the CTAN topic \href{http://ctan.org/topic/subdocs}{\textsf{subdocs}}
for further related packages.
The present package differs from the above solutions in that
a document structure constructed with the conventional |\include| mechanism
just needs two extra commands at the top of every file
such that all constituent files can be compiled individually.

%%%%%%%%%%%%%%%%%%%%%%%%%%%%%%%%%%%%%%%%%%%%%%%%%%%%%%%%%%%%%%%%%%%%%%%%%%%%%%%%
%\subsection{Feature Suggestions}
%
%The following is a list of features which may be useful for future
%versions of this package:
%%
%\begin{itemize}
%\item
%\ldots
%\end{itemize}

%%%%%%%%%%%%%%%%%%%%%%%%%%%%%%%%%%%%%%%%%%%%%%%%%%%%%%%%%%%%%%%%%%%%%%%%%%%%%%%%
\subsection{Revision History}

%%%%%%%%%%%%%%%%%%%%%%%%%%%%%%%%%%%%%%%%
\paragraph{v2.0:} 2018/12/30

\begin{itemize}
\item
immediate forward processing
\item
added |\childdocby| mechanism
\item
manual restructured
\end{itemize}

%%%%%%%%%%%%%%%%%%%%%%%%%%%%%%%%%%%%%%%%
\paragraph{v1.6:} 2018/01/17

\begin{itemize}
\item
application for development of include files
\item
corrections to manual
\end{itemize}

%%%%%%%%%%%%%%%%%%%%%%%%%%%%%%%%%%%%%%%%
\paragraph{v1.5:} 2017/05/21

\begin{itemize}
\item
more complete structuring introduced
\item
|\childdocof| introduced
\item
|\childdoc| renamed to |\childdocmain|
\item
|\childredirect| renamed to |\childdocforward| and |\childdocforwardprefix|
and functionality expanded
\end{itemize}

%%%%%%%%%%%%%%%%%%%%%%%%%%%%%%%%%%%%%%%%
\paragraph{v1.0:} 2017/04/27

\begin{itemize}
\item
manual and install package
\item
first version published on CTAN
\end{itemize}

%%%%%%%%%%%%%%%%%%%%%%%%%%%%%%%%%%%%%%%%
\paragraph{v0.6:} 2017/04/26

\begin{itemize}
\item
redirection mechanism added
\end{itemize}

%%%%%%%%%%%%%%%%%%%%%%%%%%%%%%%%%%%%%%%%
\paragraph{v0.5:} 2017/04/26

\begin{itemize}
\item
functionality in definition file
\end{itemize}


%%%%%%%%%%%%%%%%%%%%%%%%%%%%%%%%%%%%%%%%%%%%%%%%%%%%%%%%%%%%%%%%%%%%%%%%%%%%%%%%
%%%%%%%%%%%%%%%%%%%%%%%%%%%%%%%%%%%%%%%%%%%%%%%%%%%%%%%%%%%%%%%%%%%%%%%%%%%%%%%%
%%%%%%%%%%%%%%%%%%%%%%%%%%%%%%%%%%%%%%%%%%%%%%%%%%%%%%%%%%%%%%%%%%%%%%%%%%%%%%%%
\appendix

\settowidth\MacroIndent{\rmfamily\scriptsize 000\ }

 \DocInput{childdoc.dtx}

\end{document}
%</driver>
% \fi
%
% %%%%%%%%%%%%%%%%%%%%%%%%%%%%%%%%%%%%%%%%%%%%%%%%%%%%%%%%%%%%%%%%%%%%%%%%%%%%%%
% %%%%%%%%%%%%%%%%%%%%%%%%%%%%%%%%%%%%%%%%%%%%%%%%%%%%%%%%%%%%%%%%%%%%%%%%%%%%%%
% \section{Sample}
%\iffalse
%<*samplemain>
%\fi
%
% The following presents a sample document
% with two chapters, two parts, a title page,
% a compile flag as well as three forwarding files to set the flag.
% It consists of eight |.tex| files:
% \begin{center}
% \begin{tabular}{ll}
% |cdocsamp.tex|&main file\\
% |cdocsch1.tex|&include file for chapter 1\\
% |cdocsch2.tex|&include file for chapter 2\\
% |cdocspt3.tex|&include file for part 3\\
% |cdocspt4.tex|&include file for part 4\\
% |cdocsdrf.tex|&forwarding file for main file in draft mode\\
% |cdocsfi1.tex|&forwarding file for final version of chapter 1\\
% |cdocsfi2.tex|&forwarding file for final version of chapter 2\\
% \end{tabular}
% \end{center}
% Each of the eight files can be compiled directly by the \LaTeX{} compiler.
%
% %%%%%%%%%%%%%%%%%%%%%%%%%%%%%%%%%%%%%%
% \paragraph{Main File.}
%
% The main file is called |cdocsamp.tex|.
%
% Load the \textsf{childdoc} definitions and
% declare the filename for the main document:
%    \begin{macrocode}
\input{childdoc.def}
\childdocmain{}
%    \end{macrocode}

% Optional override for |\version| flag:
%    \begin{macrocode}
%%\ifchilddoc\else\providecommand{\version}{draft}\fi
%    \end{macrocode}

% Define the default values for the |\version| flag
% (|final| for the main file and |draft| for childs):
%    \begin{macrocode}
\ifchilddoc
\providecommand{\version}{draft}
\else
\providecommand{\version}{final}
\fi
%    \end{macrocode}

% Load the standard document class:
%    \begin{macrocode}
\documentclass[12pt]{article}
%    \end{macrocode}

% Start the document body:
%    \begin{macrocode}
\begin{document}
%    \end{macrocode}

% Declare a title page.
% Print title, part of document being processed and version flag:
%    \begin{macrocode}
\addtocounter{page}{-1}
\begin{center}
{\LARGE\bfseries{}childdoc example\par}
\vspace{1cm}
\ifchilddoc
\ifchilddocmanual part\else chapter\fi:
`\childdocname' of `\childdocjob'\par
\else
main document: `\childdocjob'\par
\fi
version: \version\par
\end{center}
\newpage
%    \end{macrocode}

% Manually include selected file,
% otherwise process as usual:
%    \begin{macrocode}
\ifchilddocmanual
\section*{part `\childdocname'}
\input{\childdocname}
\else
%    \end{macrocode}

% Include the two chapters:
%    \begin{macrocode}
\include{cdocsch1}
\include{cdocsch2}
%    \end{macrocode}

% Include the two parts unless only chapters should be displayed:
%    \begin{macrocode}
\ifchilddoc\else
\section{part three}
\input{cdocspt3}
\section{part four}
\input{cdocspt4}
\fi
%    \end{macrocode}

% Process as usual until here:
%    \begin{macrocode}
\fi
%    \end{macrocode}

% End of document body:
%    \begin{macrocode}
\end{document}
%    \end{macrocode}
%\iffalse
%</samplemain>
%\fi
%
% %%%%%%%%%%%%%%%%%%%%%%%%%%%%%%%%%%%%%%
% \paragraph{Chapter Include Files.}
%
% The include files are called |cdocsch1.tex| and |cdocsch2.tex|.
%
%\iffalse
%<*samplechap1|samplechap2>
%\fi

% Optional override for |\version| flag:
%    \begin{macrocode}
%%\providecommand{\version}{final}
%    \end{macrocode}

% Include the main document:
%    \begin{macrocode}
\input{childdoc.def}
\childdocof{cdocsamp}
%    \end{macrocode}

%\iffalse
%</samplechap1|samplechap2>
%\fi
%
%\iffalse
%<*samplechap1>
%\fi
% Some text for chapter 1:
%    \begin{macrocode}
\section{one}
some text in chapter one
%    \end{macrocode}

%\iffalse
%</samplechap1>
%\fi
% Some text for chapter 2:
%\iffalse
%<*samplechap2>
%\fi
%    \begin{macrocode}
\section{two}
more text in chapter two
%    \end{macrocode}

%\iffalse
%</samplechap2>
%\fi
%
% %%%%%%%%%%%%%%%%%%%%%%%%%%%%%%%%%%%%%%
% \paragraph{Part Include Files.}
%
% The include files are called |cdocspt3.tex| and |cdocspt4.tex|.
%
%\iffalse
%<*samplepart3|samplepart4>
%\fi

% Optional override for |\version| flag:
%    \begin{macrocode}
%%\providecommand{\version}{final}
%    \end{macrocode}

% Include the main document:
%    \begin{macrocode}
\input{childdoc.def}
\childdocby{cdocsamp}
%    \end{macrocode}

%\iffalse
%</samplepart3|samplepart4>
%\fi
%
%\iffalse
%<*samplepart3>
%\fi
% Some text for part 3:
%    \begin{macrocode}
some text in part three
%    \end{macrocode}

%\iffalse
%</samplepart3>
%\fi
% Some text for part 4:
%\iffalse
%<*samplepart4>
%\fi
%    \begin{macrocode}
more text in part four
%    \end{macrocode}

%\iffalse
%</samplepart4>
%\fi
%
% %%%%%%%%%%%%%%%%%%%%%%%%%%%%%%%%%%%%%%
% \paragraph{Forwarding for a Complete Draft.}
%
% The following forwarding file |cdocsdrf.tex|
% compiles the main document in draft mode:
%\iffalse
%<*sampledraft>
%\fi
%    \begin{macrocode}
\def\version{draft}
\input{childdoc.def}
\childdocforward{cdocsamp}
%    \end{macrocode}

%\iffalse
%</sampledraft>
%\fi
%
% %%%%%%%%%%%%%%%%%%%%%%%%%%%%%%%%%%%%%%
% \paragraph{Forwarding for Final Version of the Chapters.}
%
% The following forwarding files |cdocsfn1.tex| and |cdocsfn2.tex|
% (with identical content)
% compile the final versions of the child documents
% |cdocsch1.tex| and |cdocsch2.tex|, respectively:
%\iffalse
%<*samplefinal>
%\fi
%    \begin{macrocode}
\def\version{final}
\input{childdoc.def}
\childdocforwardprefix[cdocsamp]{cdocsfn}{cdocsch}
%    \end{macrocode}

%\iffalse
%</samplefinal>
%\fi
%
% %%%%%%%%%%%%%%%%%%%%%%%%%%%%%%%%%%%%%%
% \paragraph{Command Line Processing.}
%
% The following three command lines generate the output files
% |cdocscld|, |cdocscl1| and |cdocscl2|
% which should be identical to
% |cdocsdrf|, |cdocsch1| and |cdocsfn2|, respectively:
% \begin{center}
% \begin{tabular}{l}
% |latex -jobname cdocscld \|\\
% |  "\def\version{draft}\input{childdoc.def}\childdocforward{cdocsamp}"|\\
% |latex -jobname cdocscl1 \|\\
% |  "\input{childdoc.def}\childdocforward[cdocsamp]{cdocsch1}"|\\
% |latex -jobname cdocscl2 \|\\
% |  "\def\version{final}\input{childdoc.def}\childdocforward{cdocsch2}"|
% \end{tabular}
% \end{center}
% Note that the trailing backslash on each first line
% merely continues the input to the second line
% (for convenient cut ant paste).
% Furthermore, the command |latex| can be replaced by any
% of its alternative versions such as |pdflatex|.
%
% %%%%%%%%%%%%%%%%%%%%%%%%%%%%%%%%%%%%%%%%%%%%%%%%%%%%%%%%%%%%%%%%%%%%%%%%%%%%%%
% %%%%%%%%%%%%%%%%%%%%%%%%%%%%%%%%%%%%%%%%%%%%%%%%%%%%%%%%%%%%%%%%%%%%%%%%%%%%%%
% \section{Implementation}
%\iffalse
%<*package>
%\fi
%
% This section describes the definitions file |childdoc.def|.

% The definitions cannot be loaded using |\usepackage| or |\RequirePackage|
% which has a mechanism to prevent loading a style file more than once.
% When loading the definitions by means of |\input|
% multiple instances have to be prevented manually:
%\iffalse
%This code needs to be before the `\ProvidesFile' directive
%which is defined at the beginning of this file.
%Therefore it is also placed there and commented out here.
%</package>
%<*discard>
%\fi
%    \begin{macrocode}
\ifdefined\childdocmain\endinput\fi
%    \end{macrocode}
%\iffalse
%</discard>
%<*package>
%\fi
%
% \macro{\ifchilddoc}
% \macro{\ifchilddocmanual}
% The conditional |\ifchilddoc| tells whether a
% child (true) or main (false) document is being compiled.
% The conditional |\ifchilddocmanual| tells whether
% the |\includeonly| mechanism is used (false) or
% the selection of child files must be performed manually (true).
% The definitions initialise to false:
%    \begin{macrocode}
\newif\ifchilddoc
\newif\ifchilddocmanual
%    \end{macrocode}

% \macro{\childdocname}
% \macro{\childdocjob}
% The macro |\childdocname| stores the name of the main document
% to be compiled. The macro |\childdocjob| stores the name of
% the document on which the \LaTeX{} compiler was originally invoked.
% The content of |\jobname| cannot be compared
% to filenames specified in the source due to different catcodes.
% The following code rescans |\jobname|, stores the result
% in |\childdocname| and saves a copy in |\childdocjob|:
%    \begin{macrocode}
\edef\childdocname{\scantokens\expandafter{\jobname\noexpand}}
\let\childdocjob\childdocname
%    \end{macrocode}

% \macro{\childdocdisable}
% The macro |\childdocdisable| prevents the main file
% from being processed more than once.
% At this stage, the main document command |\childdocmain|
% is assumed to be called once again where it should do nothing.
% Any subsequent call to it should prevent
% a secondary processing of the main document
% It overwrites the forwarding commands
% |\childdocof| and |\childdocforward|
% with empty macros to prevent further inclusions of the main document:
%    \begin{macrocode}
\newcommand{\childdocdisable}
{
  \renewcommand{\childdocmain}[1]{\renewcommand{\childdocmain}[1]{\endinput}}
  \renewcommand{\childdocof}[1]{}
  \renewcommand{\childdocby}[2][]{}
  \renewcommand{\childdocforward}[2][]{}
  \renewcommand{\childdocdisable}{}
}
%    \end{macrocode}

% \macro{\childdocmain}
% The macro |\childdocmain| is to be called at the top of the main file
% with nothing or the main filename (without extension) as argument.
% First, it breaks loops.
% If the argument is not empty and does not match |\childdocname|
% (which is set by the first inclusion of |childdoc.def|),
% |\ifchilddoc| is set to true, |\includeonly| is applied to the child file
% and |\jobname| is set to the main file
% (for proper handling of |.aux| files):
%    \begin{macrocode}
\newcommand{\childdocmain}[1]
{
  \childdocdisable\childdocmain{}
  \if?#1?\else
    \begingroup
      \def\childdoctmp{#1}
      \ifx\childdoctmp\childdocname
        \def\childdoctmp{}
      \else
        \def\childdoctmp
        {
          \childdoctrue
          \includeonly{\childdocname}
          \def\childdocjob{#1}
          \def\jobname{#1}
        }
      \fi
      \expandafter
    \endgroup
    \childdoctmp
  \fi
}
%    \end{macrocode}

% \macro{\childdocof}
% The command |\childdocof| redirects
% compilation to the main file |#1|.
%    \begin{macrocode}
\newcommand{\childdocof}[1]
{
  \childdocdisable
  \childdoctrue
  \includeonly{\childdocname}
  \def\jobname{#1}
  \def\childdocjob{#1}
  \input{#1}
}
%    \end{macrocode}

% \macro{\childdocby}
% The command |\childdocby| ....
%    \begin{macrocode}
\newcommand{\childdocby}[2][]
{
  \childdocdisable
  \childdoctrue
  \childdocmanualtrue
  \if?#1?\else
    \def\jobname{#2}
  \fi
  \def\childdocjob{#2}
  \input{#2}
  \endinput
}
%    \end{macrocode}

% \macro{\childdocforward}
% The command |\childdocforward| redirects
% compilation to the main file or
% (if the optional argument is given) a child file.
% Parameters are set as if the main file
% or a child file starting with |\childdocof| was compiled.
% Then compilation is handed over to the main file:
%    \begin{macrocode}
\newcommand{\childdocforward}[2][]
{
  \begingroup
    \if?#1?
      \def\childdoctmp
      {
        \def\childdocname{#2}
        \def\childdocjob{#2}
        \def\jobname{#2}
        \input{#2}
        \endinput
      }
    \else
      \def\childdoctmp
      {
        \childdocdisable
        \def\childdocname{#2}
        \childdoctrue
        \includeonly{#2}
        \def\childdocjob{#1}
        \def\jobname{#1}
        \input{#1}
        \endinput
      }
    \fi
    \expandafter
  \endgroup
  \childdoctmp
}
%    \end{macrocode}

% \macro{\childdocforwardprefix}
% The command |\childdocforwardprefix| redirects
% compilation to the main or a child file by means of a pattern.
% The prefix |#1| in the current filename is replaced by |#2|
% and the suffix of the current filename is kept
% (it is assumed that the filename does not contain the substring `|~~~|'
% which is used as a delimiter).
% Compilation is handed over to the new file by |\childdocforward|:
%    \begin{macrocode}
\newcommand{\childdocforwardprefix}[3][]
{
  \begingroup
    \def\childdocextract #2##1~~~{\def\childdoctmp{\childdocforward[#1]{#3##1}}}
    \expandafter\childdocextract\childdocname~~~
    \expandafter
  \endgroup
  \childdoctmp
}
%    \end{macrocode}

% \macro{\childdoc}
% The deprecated macro |\childdoc| is a legacy version of |\childdocmain|:
%    \begin{macrocode}
\newcommand{\childdoc}{\childdocmain}
%    \end{macrocode}

% \macro{\childdocredirect}
% The deprecated macro |\childdocredirect| is a legacy version
% of |\childdocforward| and |\childdocforwardprefix|:
%    \begin{macrocode}
\newcommand{\childdocredirect}[2][]
{
  \begingroup
    \if?#1?
      \def\childdoctmp{\childdocforward{#2}}
    \else
      \def\childdoctmp{\childdocforwardprefix{#1}{#2}}
    \fi
    \expandafter
  \endgroup
  \childdoctmp
}
%    \end{macrocode}

%\iffalse
%</package>
%\fi
%
\endinput
|\\
|\childdocby{|\textit{main}|}|\\
\end{tabular}
\end{center}
%
The directive |\childdocby| is similar to |\childdocof|
described in \secref{sec:include},
but the subsequent selection of content must be done manually.
To that end, both |\ifchilddoc| and |\ifchilddocmanual|
will be true upon processing of a part,
and the name of the part is stored in |\childdocname|.
Note that |\jobname| will be set to the filename of the current part
so that each part receives an individual |.aux| file
that does not interfere with the |.aux| file(s) of the main document.
This behaviour can be altered by the alternative form
|\childdocby[*]{|\textit{main}|}| (with a non-empty optional argument)
which uses the |.aux| file of the main document
by setting |\jobname| to \textit{main}.

%%%%%%%%%%%%%%%%%%%%%%%%%%%%%%%%%%%%%%%%%%%%%%%%%%%%%%%%%%%%%%%%%%%%%%%%%%%%%%%%
\subsection{Driver Development}
\label{sec:driver}

The \textsf{childdoc} mechanism can also be use for the development
of definition files such as \LaTeX{} styles or classes.
This case differs from the above setup with multiple parts
included by |\include| in that no |\includeonly| should be invoked.
This can be achieved by starting the include file
(before |\ProvidesPackage|) with:
%
\begin{center}
\begin{tabular}{l}
|% \iffalse
%
% childdoc.dtx Copyright (C) 2017-2018 Niklas Beisert
%
% This work may be distributed and/or modified under the
% conditions of the LaTeX Project Public License, either version 1.3
% of this license or (at your option) any later version.
% The latest version of this license is in
%   http://www.latex-project.org/lppl.txt
% and version 1.3 or later is part of all distributions of LaTeX
% version 2005/12/01 or later.
%
% This work has the LPPL maintenance status `maintained'.
%
% The Current Maintainer of this work is Niklas Beisert.
%
% This work consists of the files childdoc.dtx and childdoc.ins
% and the derived files childdoc.def and cdocsamp.tex with
% cdocsch1.tex, cdocsch2.tex, cdocsdrf.tex, cdocsfn1.tex, cdocsfn2.tex.
%
%<package>\ifdefined\childdocmain\endinput\fi
%<package>\ProvidesFile{childdoc.def}[2018/12/30 v2.0 child document driver]
%<samplemain>\ProvidesFile{cdocsamp.tex}[2018/12/30 v2.0 sample for childdoc]
%<*driver>
%\ProvidesFile{childdoc.drv}[2018/12/30 v2.0 childdoc reference manual file]
\PassOptionsToClass{10pt,a4paper}{article}
\documentclass{ltxdoc}

\usepackage[margin=35mm]{geometry}
\usepackage{hyperref}
\usepackage{hyperxmp}
\usepackage[usenames]{color}

\hypersetup{colorlinks=true}
\hypersetup{pdfstartview=FitH}
\hypersetup{pdfpagemode=UseNone}
\hypersetup{pdfsource={}}
\hypersetup{pdflang={en-UK}}
\hypersetup{pdfcopyright={Copyright 2017-2018 Niklas Beisert.
  This work may be distributed and/or modified under the
  conditions of the LaTeX Project Public License, either version 1.3
  of this license or (at your option) any later version.}}
\hypersetup{pdflicenseurl={http://www.latex-project.org/lppl.txt}}
\hypersetup{pdfcontactaddress={ETH Zurich, ITP, HIT K,
  Wolfgang-Pauli-Strasse 27}}
\hypersetup{pdfcontactpostcode={8093}}
\hypersetup{pdfcontactcity={Zurich}}
\hypersetup{pdfcontactcountry={Switzerland}}
\hypersetup{pdfcontactemail={nbeisert@itp.phys.ethz.ch}}
\hypersetup{pdfcontacturl={http://people.phys.ethz.ch/\xmptilde nbeisert/}}

\newcommand{\secref}[1]{\hyperref[#1]{section \ref*{#1}}}

\parskip1ex
\parindent0pt
\let\olditemize\itemize
\def\itemize{\olditemize\parskip0pt}

\begin{document}

\title{The \textsf{childdoc} Package}
\hypersetup{pdftitle={The childdoc Package}}
\author{Niklas Beisert\\[2ex]
  Institut f\"ur Theoretische Physik\\
  Eidgen\"ossische Technische Hochschule Z\"urich\\
  Wolfgang-Pauli-Strasse 27, 8093 Z\"urich, Switzerland\\[1ex]
  \href{mailto:nbeisert@itp.phys.ethz.ch}
  {\texttt{nbeisert@itp.phys.ethz.ch}}}
\hypersetup{pdfauthor={Niklas Beisert}}
\hypersetup{pdfsubject={Manual for the LaTeX2e Package childdoc}}
\date{30 December 2018, \textsf{v2.0}}
\maketitle

\begin{abstract}\noindent
\textsf{childdoc} is a \LaTeXe{} package
that enables the direct compilation
of document sections included by |\include|
to individual files.
\end{abstract}

\begingroup
\parskip0ex
\tableofcontents
\endgroup

%%%%%%%%%%%%%%%%%%%%%%%%%%%%%%%%%%%%%%%%%%%%%%%%%%%%%%%%%%%%%%%%%%%%%%%%%%%%%%%%
%%%%%%%%%%%%%%%%%%%%%%%%%%%%%%%%%%%%%%%%%%%%%%%%%%%%%%%%%%%%%%%%%%%%%%%%%%%%%%%%
\section{Introduction}

\LaTeX{} provides a mechanism to structure a large document (such as a book)
into a main file and several child files (containing the chapters)
using the |\include| command.
This mechanism is beneficial for documents
which span hundreds of pages in order to
make the source file(s) more manageable.
Moreover, compilation can be restricted to
selected child files by means of the |\includeonly| command.
The latter feature can be used to reduce the compilation time while editing
(this was significantly more useful in the earlier days of \LaTeX{})
or to generate a smaller document which is easier to navigate.
Another application of |\includeonly| is to generate
documents consisting of selected parts of the complete document.

However, there are a few drawbacks of the plain |\include| mechanism:
\begin{itemize}
\item
The child files cannot be compiled on their own,
they can only be compiled via the main file.
A naive editing environment
(such as a text editor with an option
to have the current file processed by \LaTeX)
may require one to switch to the main file before compiling;
attempting to compile the child file produces errors.
\item
The main file must be modified (each time)
to adjust the |\includeonly| command
to the present needs. This easily leaves the main file in a messy state.
\item
The generated document will always carry the filename
of the main document. This is inconvenient if
several child files are to be compiled and
to be kept for distribution.
\end{itemize}

The present package provides a simple interface
to make child files individually compilable by \LaTeX{}.
Compiling a child file then has the same effect as compiling
the main file with an |\includeonly| command
to select the appropriate child.
Moreover the generated document will carry the name of the child
rather than the main file.
This resolves all three above issues.

This feature is meant to make the editing of books,
thesis documents and lecture notes somewhat more convenient.
However, the package can also be used efficiently for
composing a series of documents (such as exercise sheets)
which are typically distributed individually.
It then assists the author in generating the individual documents
(potentially in different versions)
as well as a document containing the collected series.
Another application is in developing style files
or other kinds of included material
where compilation of the style file could redirect
to a sample or test file.

%%%%%%%%%%%%%%%%%%%%%%%%%%%%%%%%%%%%%%%%%%%%%%%%%%%%%%%%%%%%%%%%%%%%%%%%%%%%%%%%
%%%%%%%%%%%%%%%%%%%%%%%%%%%%%%%%%%%%%%%%%%%%%%%%%%%%%%%%%%%%%%%%%%%%%%%%%%%%%%%%
\section{Usage}

First of all, the package \textsf{childdoc} is \emph{not} a standard
\LaTeXe{} |.sty| style file! Therefore it needs to be invoked in
a non-standard way.

%%%%%%%%%%%%%%%%%%%%%%%%%%%%%%%%%%%%%%%%%%%%%%%%%%%%%%%%%%%%%%%%%%%%%%%%%%%%%%%%
\subsection{Included Files}
\label{sec:include}

%%%%%%%%%%%%%%%%%%%%%%%%%%%%%%%%%%%%%%%%
\DescribeMacro{\childdocmain}
To use the package, add the commands
\begin{center}
\begin{tabular}{l}
|\input{childdoc.def}|\\
|\childdocmain{}|\\
\end{tabular}
\end{center}
at the very top of the main \LaTeX{} file,
in particular \emph{before} the |\documentclass| statement!
The argument of |\childdocmain| should be left empty
(but it must be present).

%%%%%%%%%%%%%%%%%%%%%%%%%%%%%%%%%%%%%%%%
\DescribeMacro{\childdocof}
Furthermore, add the commands
\begin{center}
\begin{tabular}{l}
|\input{childdoc.def}|\\
|\childdocof{|\textit{main}|}|\\
\end{tabular}
\end{center}
at the top of every child file \textit{child}
which is included by |\include{|\textit{child}|}|
from within the main file
(or at least for those files to be compiled individually).
The argument \textit{main} must be the filename of the main file.

There are a couple of
considerations in setting up the main and child documents:

%%%%%%%%%%%%%%%%%%%%%%%%%%%%%%%%%%%%%%%%
\paragraph{Restrictions.}

Please note the following restrictions:
\begin{itemize}
\item
|\childdocmain| must be called with one argument \textit{main}
to ensure compatibility with earlier version of the package.
It must either be empty (|\childdocmain{}|)
or precisely match the filename of the main file in which it is specified.
See \secref{sec:detection} for further information.
\item
The filename \textit{main} must be specified without the |.tex| extension.
\item
The filename \textit{main} is case sensitive
(even in case-insensitive file systems)
due to internal string comparison.
\item
The argument \textit{main} should be fully expanded, it cannot be a macro.
\item
Subdirectories and special characters should be avoided in filenames.
\item
The command |\childdocmain{|\textit{main}|}| must be followed by a whitespace.
It should not be followed immediately by another command
or by a comment mark `|%|'.
This is because the \TeX{} parser reads the token immediately following
the argument of |\childdocmain| and puts it
at the beginning of every child section;
however, a white\-space is ignored.
\end{itemize}

%%%%%%%%%%%%%%%%%%%%%%%%%%%%%%%%%%%%%%%%
\paragraph{Content of Main File.}

It is advisable to place all content in the child files included by |\include|.
Any output contained in the main file will appear in all child documents
unless suppressed manually;
it cannot be suppressed automatically by the |\includeonly| directive
and thus should normally be avoided.
A method to include some content in the main file
by means of conditional processing is described in \secref{sec:conditional}.

%%%%%%%%%%%%%%%%%%%%%%%%%%%%%%%%%%%%%%%%
\paragraph{Page Numbering.}

When only a part of the document is compiled,
the appropriate numbering of pages
(as well as other status parameters)
is determined from the |.aux| files.
The latter contain information from previous passes.
However this information needs to propagate through
all intermediate child documents.
Therefore the page numbering in child documents may well
be inconsistent until the complete document is compiled at least once.

A useful (if unconventional) way to always ensure a consistent
page numbering is to restart the numbering in each child document
and denote the pages by `\textit{child}|.|\textit{page}'
where \textit{child} represents the chapter/section number of the child file.
This can be achieved by the command
|\numberwithin{page}{|\textit{child}|}|
of the \textsf{amsmath} package
where \textit{child} can be |chapter| or |section|
depending on the chosen structuring.
Alternatively, one can modify the macro |\thepage| appropriately
and reset the counter |page| at the start of each child file.

%%%%%%%%%%%%%%%%%%%%%%%%%%%%%%%%%%%%%%%%%%%%%%%%%%%%%%%%%%%%%%%%%%%%%%%%%%%%%%%%
\subsection{Conditional Processing}
\label{sec:conditional}

The package provides a mechanism to compile different versions
of a document. To customise the versions further some conditional processing
can come in handy to distinguish which version is being compiled.
The package provides two macros to describe the compilation context:

%%%%%%%%%%%%%%%%%%%%%%%%%%%%%%%%%%%%%%%%
\DescribeMacro{\ifchilddoc}
The conditional |\ifchilddoc| distinguishes between the compilation of
child documents and the main document:
%
\begin{center}
|\ifchilddoc |\textit{child-code}| |[|\||else |\textit{main-code}]| \||fi|
\end{center}

%%%%%%%%%%%%%%%%%%%%%%%%%%%%%%%%%%%%%%%%
\DescribeMacro{\childdocname}
\DescribeMacro{\childdocjob}
The macro |\childdocname| contains the filename (without extension)
of the main or child file being processed.
Note that |\childdocjob| will always contain the name of the main file.

%%%%%%%%%%%%%%%%%%%%%%%%%%%%%%%%%%%%%%%%
\paragraph{Title Page.}

Conditional processing can be used to include a title or banner page
in the main document when proper precautions are taken.
Importantly, the code in the main file should ensure that the page counter
(as well as other status parameters which are stored in the |.aux| files)
takes the same value after the conditional processing.
Otherwise the page numbers may take divergent values
depending on which part is compiled.

For example, a title page could be declared by:
%
\begin{center}
\begin{tabular}{l}
|\ifchilddoc\||else|\\
|\addtocounter{page}{-1}|\\
\textit{code for title page}\\
|\newpage|\\
|\||fi|
\end{tabular}
\end{center}
%
A banner page for the child documents can be generated by:
%
\begin{center}
\begin{tabular}{l}
|\ifchilddoc|\\
|\addtocounter{page}{-1}|\\
\textit{code for banner page}\\
|\newpage|\\
|\||fi|
\end{tabular}
\end{center}
%
Here one could write a message such as:
\begin{center}
|This is the part \childdocname{} of \childdocjob{}.|
\end{center}

%%%%%%%%%%%%%%%%%%%%%%%%%%%%%%%%%%%%%%%%%%%%%%%%%%%%%%%%%%%%%%%%%%%%%%%%%%%%%%%%
\subsection{Flags}
\label{sec:flags}

The package makes it easy to generate different versions
of the main or child documents.
To this end compilation flags can be defined
and assigned different default values.
They will be particularly useful in conjunction
with the forwarding mechanism described in \secref{sec:forward}.

For example, it may be useful to have a flag |\version|
which can be set to |draft| or |final|.
The document source will contain some conditional code
depending on the value of |\version|.
Suppose further, the flag should default to |final| for the main file
and to |draft| for child files
which is a natural assignment for editing the document.
This is achieved by placing the following code
in the preamble of the main document
(below the |\childdocmain| directive):
%
\begin{center}
\begin{tabular}{l}
|\ifchilddoc|\\
|\providecommand{\version}{draft}|\\
|\||else|\\
|\providecommand{\version}{final}|\\
|\||fi|
\end{tabular}
\end{center}
%
The definition by |\providecommand| makes sure
that previous definitions are not overwritten.
Further statements |\providecommand{\version}{...}|
can thus be added before the above code to override it.

For the main file, one might add a line
(between |\childdocmain| and the above block)
%
\begin{center}
|%\ifchilddoc\||else\providecommand{\version}{draft}\||fi|
\end{center}
%
which can be uncommented to produce a draft version.
Likewise one can add a line to the very top of a child file
(above the |\childdocof{|\textit{main}|}| directive)
%
\begin{center}
|%\providecommand{\version}{final}|
\end{center}
%
which can be uncommented to produce the final version of this child document.

%%%%%%%%%%%%%%%%%%%%%%%%%%%%%%%%%%%%%%%%%%%%%%%%%%%%%%%%%%%%%%%%%%%%%%%%%%%%%%%%
\subsection{Forwarding}
\label{sec:forward}

Different versions of the main or child documents
using compilation flags as described in \secref{sec:flags}
can be (permanently) stored in different files
for convenient compilation, viewing and distribution.
To this end, the package defines a command
to pass on compilation to a different file:

%%%%%%%%%%%%%%%%%%%%%%%%%%%%%%%%%%%%%%%%
\DescribeMacro{\childdocforward}
The command |\childdocforward| redirects processing to
another source file:
%
\begin{center}
\begin{tabular}{l}
|\input{childdoc.def}|\\
|\childdocforward[|\textit{main}|]{|\textit{dest}|}|\\
\end{tabular}
\end{center}
%
The argument \textit{dest} is the destination file
(without extension).
It should be the main file or one of the child files.
Note that further \textsf{childdoc} directives
such as |\childdocof| and |\childdocforward|
in the indicated file will be processed in this form.
The optional argument \textit{main}
passes on directly to the main file \textit{main}
while pretending to compile the child \textit{dest}.
This form behaves as if \textit{dest}
issues |\childdocof{|\textit{main}|}| right away,
and no further \textsf{childdoc} directives will be processed.

%%%%%%%%%%%%%%%%%%%%%%%%%%%%%%%%%%%%%%%%
\DescribeMacro{\...prefix}
In the alternative form |\childdocforwardprefix|,
%
\begin{center}
\begin{tabular}{l}
|\input{childdoc.def}|\\
|\childdocforwardprefix[|\textit{main}|]{|\textit{prefix}|}{|\textit{dest}|}|
\end{tabular}
\end{center}
%
the destination file is determined by a pattern
depending on the current file:
To make this work, the current file must be called
`{\textit{prefix}\hspace{0.2em}\textit{suffix}}'
with \textit{prefix} matching precisely the argument.
Processing is then passed on to the file
`{\textit{dest}\hspace{0.2em}\textit{suffix}}'.
Surely, the same effect is achieved by
directly specifying the
argument `{\textit{dest}\hspace{0.2em}\textit{suffix}}'
in the first form.
However, that requires to set up a different file
for each child. With the alternative form of the command
all these files can have exactly the same content
which simplifies setting them up and maintaining them.

For example, the following file |draft.tex|
with a compilation flag |\version| as described in \secref{sec:flags}
compiles the main document as a draft:
%
\begin{center}
\begin{tabular}{l}
|\def\version{draft}|\\
|\input{childdoc.def}|\\
|\childdocforward{|\textit{main}|}|
\end{tabular}
\end{center}
%
Likewise, the following files |final|\textit{nn}|.tex|
compile the final version of the child document
|child|\textit{nn}|.tex|:
%
\begin{center}
\begin{tabular}{l}
|\def\version{final}|\\
|\input{childdoc.def}|\\
|\childdocforwardprefix{final}{child}|
\end{tabular}
\end{center}
%

Note that when several versions of a main file and/or of each child file
are to be generated, it may be convenient to set up a |Makefile| or
shell script to automatise the process.

%%%%%%%%%%%%%%%%%%%%%%%%%%%%%%%%%%%%%%%%%%%%%%%%%%%%%%%%%%%%%%%%%%%%%%%%%%%%%%%%
\subsection{Command Line Processing}
\label{sec:commandline}

The effect of redirection files can also be achieved by invoking
the \LaTeX{} compiler with a more elaborate command line.
Most conveniently this should be done as part
of a shell script or a |Makefile|.

When using \textsf{childdoc} in the main file, the following
command lines effectively perform a redirection
(note that depending on the shell being used,
backslashes may have to be doubled: `|\|' $\to$ `|\\|'):
%
\begin{center}
|... -jobname "|\textit{target}|" |\\|"|[\textit{flags}]%
|\input{childdoc.def}\childdocforward[|\textit{main}|]{|\textit{dest}|}"|
\end{center}
%
Here \textit{target} is the name of the output file,
\textit{main} is the name of the main file
and \textit{dest} is the name of the main or child file to be processed
(all filenames without extensions).
The optional argument \textit{main} can be omitted
if \textit{main} matches \textit{dest}.
Optionally, compilation \textit{flags} can be defined via |\def| commands.
This command line makes the \TeX{} engine believe
it is compiling the file \textit{target}
whose content is specified as the latter parameter.
The provided code then forwards the processing to
\textit{main} or \textit{dest} as described in \secref{sec:forward}.

%%%%%%%%%%%%%%%%%%%%%%%%%%%%%%%%%%%%%%%%%%%%%%%%%%%%%%%%%%%%%%%%%%%%%%%%%%%%%%%%
\subsection{Include by Input}
\label{sec:input}

Including child documents by |\include| has some restrictions by design.
Most notably, the content of a child document always occupies
its own set of pages; pages cannot be shared between child documents.
Usually, this behaviour makes perfect sense
because each child document contain an essential part of the document.
However, in some situations it may be desirable to compose
a document from a collection of parts
without having mandatory page breaks between then.
For this case, the package
provides a mechanism to include parts
by |\input| which can also be processed individually.
However, by construction this mechanism
requires manual handling of the content to be output.

%%%%%%%%%%%%%%%%%%%%%%%%%%%%%%%%%%%%%%%%
\DescribeMacro{\ifchilddocmanual}
The main file should be prepared as usual, see \secref{sec:include}.
However, the document body must make a distinction
between processing of an individual part and of the main document, e.g.:
%
\begin{center}
\begin{tabular}{l}
|\ifchilddocmanual|\\
|\input{\childdocname}|\\
|\||else|\\
\textit{document body with }|\input{|\textit{part}|}|\\
|\||fi|
\end{tabular}
\end{center}
%
The conditional |\ifchilddocmanual| is true whenever
a part to be included by |\input| is being compiled,
and the name of the part is stored in |\childdocname|.

%%%%%%%%%%%%%%%%%%%%%%%%%%%%%%%%%%%%%%%%
\DescribeMacro{\childdocby}
Each part to be included by |\input| should start with:
%
\begin{center}
\begin{tabular}{l}
|\input{childdoc.def}|\\
|\childdocby{|\textit{main}|}|\\
\end{tabular}
\end{center}
%
The directive |\childdocby| is similar to |\childdocof|
described in \secref{sec:include},
but the subsequent selection of content must be done manually.
To that end, both |\ifchilddoc| and |\ifchilddocmanual|
will be true upon processing of a part,
and the name of the part is stored in |\childdocname|.
Note that |\jobname| will be set to the filename of the current part
so that each part receives an individual |.aux| file
that does not interfere with the |.aux| file(s) of the main document.
This behaviour can be altered by the alternative form
|\childdocby[*]{|\textit{main}|}| (with a non-empty optional argument)
which uses the |.aux| file of the main document
by setting |\jobname| to \textit{main}.

%%%%%%%%%%%%%%%%%%%%%%%%%%%%%%%%%%%%%%%%%%%%%%%%%%%%%%%%%%%%%%%%%%%%%%%%%%%%%%%%
\subsection{Driver Development}
\label{sec:driver}

The \textsf{childdoc} mechanism can also be use for the development
of definition files such as \LaTeX{} styles or classes.
This case differs from the above setup with multiple parts
included by |\include| in that no |\includeonly| should be invoked.
This can be achieved by starting the include file
(before |\ProvidesPackage|) with:
%
\begin{center}
\begin{tabular}{l}
|\input{childdoc.def}|\\
|\childdocforward{|\textit{main}|}|\\
\end{tabular}
\end{center}
%
or alternatively with:
%
\begin{center}
\begin{tabular}{l}
|\input{childdoc.def}|\\
|\childdocby{|\textit{main}|}|\\
\end{tabular}
\end{center}
%
Both forms have slightly different effects as described above.
The main file is prepared as usual, see \secref{sec:include}.

%%%%%%%%%%%%%%%%%%%%%%%%%%%%%%%%%%%%%%%%%%%%%%%%%%%%%%%%%%%%%%%%%%%%%%%%%%%%%%%%
\subsection{Legacy Detection}
\label{sec:detection}

The directive |\childdocmain| in the main file can detect
whether the complete document or merely a child is to be compiled
even without using the directive |\childdocof|.
This method is deprecated because it is less robust
and there is no compelling reason to use it;
it is merely provided for backward compatibility
and it may be removed in future versions.

If the detection mechanism is to be used,
it is mandatory to correctly specify
the filename of the main file as the argument of |\childdocmain|:
%
\begin{center}
\begin{tabular}{l}
|\input{childdoc.def}|\\
|\childdocmain{|\textit{main}|}|\\
\end{tabular}
\end{center}
%
If |\jobname| does not match the argument \textit{main} of |\childdocmain|,
it is assumed that |\jobname| points to the child file to be compiled.
When using |\childdocmain| with the main file specified as argument,
it suffices to start a child file
with just |\input{|\textit{main}|}|
without loading of the package and using |\childdocof|.
If instead all processing is done
with the appropriate \textsf{childdoc} directives,
the argument of \textit{main} of |\childdocmain| can be empty.

An alternative version of the command line processing described
in \secref{sec:commandline} using the detection mechanism reads:
%
\begin{center}
|... -jobname "|\textit{target}|" "|[\textit{flags}]%
[|\def\jobname{|\textit{dest}|}|]|\input{|\textit{main}|}"|
\end{center}

%%%%%%%%%%%%%%%%%%%%%%%%%%%%%%%%%%%%%%%%%%%%%%%%%%%%%%%%%%%%%%%%%%%%%%%%%%%%%%%%
\subsection{Manual Code}
\label{sec:manual}

In case one cannot be certain whether the definitions file |childdoc.def|
is installed on the target \TeX{} distribution
and one prefers not to ship it,
it is conceivable to paste a few relevant commands into the sources.

To that end, drop all statements |\input{childdoc.def}|
and perform the replacements as outlined below.
Instead of |\childdocmain{|\textit{main}|}| add the following code
to the top of the main file:
%
\begin{center}
\begin{tabular}{l}
|\||ifdefined\childdocname\endinput\||fi\newif\ifchilddoc|\\
|\edef\childdocname{\scantokens\expandafter{\jobname\noexpand}}|\\
|\def\childdocmain{|\textit{main}|}\||ifx\childdocmain\childdocname\||else|\\
|\childdoctrue\includeonly{\childdocname}\let\jobname\childdocmain\||fi|\\
\end{tabular}
\end{center}
%
Instead of |\childdocof{|\textit{main}|}| just include the main file
at the top of each child file:
%
\begin{center}
|\input{|\textit{main}|}|
\end{center}
%
A simple redirection |\childdocforward{|\textit{dest}|}| is achieved by:
%
\begin{center}
|\def\jobname{|\textit{dest}|}\input{\jobname}|
\end{center}
%
The redirection with prefix
|\childdocforwardprefix[|\textit{prefix}|]{|\textit{dest}|}|
is accomplished by:
%
\begin{center}
\begin{tabular}{l}
|{\edef\jobname{\scantokens\expandafter{\jobname\noexpand}}|\\
|\def\redirectjob |\textit{prefix}|#1~~~{\gdef\jobname{|\textit{dest}|#1}}|\\
|\expandafter\redirectjob\jobname~~~}\input{\jobname}|
\end{tabular}
\end{center}

In an alternative approach,
child documents can be compiled by a specific command line
without additional code or specific definitions:
%
\begin{center}
|... -jobname "|\textit{target}|" "|[\textit{flags}]%
|\includeonly{|\textit{dest}|}\input{|\textit{main}|}"|
\end{center}
%

%%%%%%%%%%%%%%%%%%%%%%%%%%%%%%%%%%%%%%%%%%%%%%%%%%%%%%%%%%%%%%%%%%%%%%%%%%%%%%%%
%%%%%%%%%%%%%%%%%%%%%%%%%%%%%%%%%%%%%%%%%%%%%%%%%%%%%%%%%%%%%%%%%%%%%%%%%%%%%%%%
\section{Information}

%%%%%%%%%%%%%%%%%%%%%%%%%%%%%%%%%%%%%%%%%%%%%%%%%%%%%%%%%%%%%%%%%%%%%%%%%%%%%%%%
\subsection{Copyright}

Copyright \copyright{} 2017--2018 Niklas Beisert

This work may be distributed and/or modified under the
conditions of the \LaTeX{} Project Public License, either version 1.3
of this license or (at your option) any later version.
The latest version of this license is in
  \url{http://www.latex-project.org/lppl.txt}
and version 1.3 or later is part of all distributions of \LaTeX{}
version 2005/12/01 or later.

This work has the LPPL maintenance status `maintained'.

The Current Maintainer of this work is Niklas Beisert.

This work consists of the files |README.txt|, |childdoc.ins| and |childdoc.dtx|
as well as the derived files |childdoc.def|, |cdocsamp.tex|
with |cdocsch1.tex|, |cdocsch2.tex|, |cdocspt3.tex|, |cdocspt4.tex|,
|cdocsdrf.tex|, |cdocsfn1.tex|, |cdocsfn2.tex|
as well as |childdoc.pdf|.

%%%%%%%%%%%%%%%%%%%%%%%%%%%%%%%%%%%%%%%%%%%%%%%%%%%%%%%%%%%%%%%%%%%%%%%%%%%%%%%%
\subsection{Files and Installation}

The package consists of the files:
%
\begin{center}
\begin{tabular}{ll}
    |README.txt|   & readme file \\
    |childdoc.ins| & installation file \\
    |childdoc.dtx| & source file \\
    |childdoc.def| & definition file \\
    |cdocsamp.tex| & sample main file \\
    |cdocsch1.tex| & sample include file \\
    |cdocsch2.tex| & sample include file \\
    |cdocspt3.tex| & sample part file \\
    |cdocspt4.tex| & sample part file \\
    |cdocsdrf.tex| & sample redirection file \\
    |cdocsfn1.tex| & sample redirection file \\
    |cdocsfn2.tex| & sample redirection file \\
    |childdoc.pdf| & manual
\end{tabular}
\end{center}
%
The distribution consists of the files
|README.txt|, |childdoc.ins| and |childdoc.dtx|.
%
\begin{itemize}
\item
Run (pdf)\LaTeX{} on |childdoc.dtx|
to compile the manual |childdoc.pdf| (this file).
\item
Run \LaTeX{} on |childdoc.ins| to create the definitions file |childdoc.def|
and the sample |cdocsamp.tex| with include files
|cdocsch1.tex|, |cdocsch2.tex|, |cdocspt3.tex|, |cdocspt4.tex|,
|cdocsdrf.tex|, |cdocsfn1.tex|, |cdocsfn2.tex|.
Then copy the file |childdoc.def| to an appropriate directory of your \LaTeX{}
distribution, e.g.\ \textit{texmf-root}|/tex/latex/childdoc|.
\end{itemize}

%%%%%%%%%%%%%%%%%%%%%%%%%%%%%%%%%%%%%%%%%%%%%%%%%%%%%%%%%%%%%%%%%%%%%%%%%%%%%%%%
\subsection{Related CTAN Packages}

There are several other packages which offer a similar functionality:
%
\begin{itemize}
\item
The packages
\href{http://ctan.org/pkg/docmute}{\textsf{docmute}},
\href{http://ctan.org/pkg/includex}{\textsf{includex}} and
\href{http://ctan.org/pkg/standalone}{\textsf{standalone}}
provide commands to include only the document body of
a child file thus allowing both files to be compiled individually.
\item
The packages \href{http://ctan.org/pkg/subdocs}{\textsf{subdocs}}
and \href{http://ctan.org/pkg/subfiles}{\textsf{subfiles}}
provide structures in which the main and child documents can be
encapsulated and allowing them to be compiled individually.
The inclusion mechanism is different from the conventional |\include|.
\item
The package \href{http://ctan.org/pkg/combine}{\textsf{combine}}
is an elaborate solution to combine several documents into one.
\end{itemize}
%
See also the CTAN topic \href{http://ctan.org/topic/subdocs}{\textsf{subdocs}}
for further related packages.
The present package differs from the above solutions in that
a document structure constructed with the conventional |\include| mechanism
just needs two extra commands at the top of every file
such that all constituent files can be compiled individually.

%%%%%%%%%%%%%%%%%%%%%%%%%%%%%%%%%%%%%%%%%%%%%%%%%%%%%%%%%%%%%%%%%%%%%%%%%%%%%%%%
%\subsection{Feature Suggestions}
%
%The following is a list of features which may be useful for future
%versions of this package:
%%
%\begin{itemize}
%\item
%\ldots
%\end{itemize}

%%%%%%%%%%%%%%%%%%%%%%%%%%%%%%%%%%%%%%%%%%%%%%%%%%%%%%%%%%%%%%%%%%%%%%%%%%%%%%%%
\subsection{Revision History}

%%%%%%%%%%%%%%%%%%%%%%%%%%%%%%%%%%%%%%%%
\paragraph{v2.0:} 2018/12/30

\begin{itemize}
\item
immediate forward processing
\item
added |\childdocby| mechanism
\item
manual restructured
\end{itemize}

%%%%%%%%%%%%%%%%%%%%%%%%%%%%%%%%%%%%%%%%
\paragraph{v1.6:} 2018/01/17

\begin{itemize}
\item
application for development of include files
\item
corrections to manual
\end{itemize}

%%%%%%%%%%%%%%%%%%%%%%%%%%%%%%%%%%%%%%%%
\paragraph{v1.5:} 2017/05/21

\begin{itemize}
\item
more complete structuring introduced
\item
|\childdocof| introduced
\item
|\childdoc| renamed to |\childdocmain|
\item
|\childredirect| renamed to |\childdocforward| and |\childdocforwardprefix|
and functionality expanded
\end{itemize}

%%%%%%%%%%%%%%%%%%%%%%%%%%%%%%%%%%%%%%%%
\paragraph{v1.0:} 2017/04/27

\begin{itemize}
\item
manual and install package
\item
first version published on CTAN
\end{itemize}

%%%%%%%%%%%%%%%%%%%%%%%%%%%%%%%%%%%%%%%%
\paragraph{v0.6:} 2017/04/26

\begin{itemize}
\item
redirection mechanism added
\end{itemize}

%%%%%%%%%%%%%%%%%%%%%%%%%%%%%%%%%%%%%%%%
\paragraph{v0.5:} 2017/04/26

\begin{itemize}
\item
functionality in definition file
\end{itemize}


%%%%%%%%%%%%%%%%%%%%%%%%%%%%%%%%%%%%%%%%%%%%%%%%%%%%%%%%%%%%%%%%%%%%%%%%%%%%%%%%
%%%%%%%%%%%%%%%%%%%%%%%%%%%%%%%%%%%%%%%%%%%%%%%%%%%%%%%%%%%%%%%%%%%%%%%%%%%%%%%%
%%%%%%%%%%%%%%%%%%%%%%%%%%%%%%%%%%%%%%%%%%%%%%%%%%%%%%%%%%%%%%%%%%%%%%%%%%%%%%%%
\appendix

\settowidth\MacroIndent{\rmfamily\scriptsize 000\ }

 \DocInput{childdoc.dtx}

\end{document}
%</driver>
% \fi
%
% %%%%%%%%%%%%%%%%%%%%%%%%%%%%%%%%%%%%%%%%%%%%%%%%%%%%%%%%%%%%%%%%%%%%%%%%%%%%%%
% %%%%%%%%%%%%%%%%%%%%%%%%%%%%%%%%%%%%%%%%%%%%%%%%%%%%%%%%%%%%%%%%%%%%%%%%%%%%%%
% \section{Sample}
%\iffalse
%<*samplemain>
%\fi
%
% The following presents a sample document
% with two chapters, two parts, a title page,
% a compile flag as well as three forwarding files to set the flag.
% It consists of eight |.tex| files:
% \begin{center}
% \begin{tabular}{ll}
% |cdocsamp.tex|&main file\\
% |cdocsch1.tex|&include file for chapter 1\\
% |cdocsch2.tex|&include file for chapter 2\\
% |cdocspt3.tex|&include file for part 3\\
% |cdocspt4.tex|&include file for part 4\\
% |cdocsdrf.tex|&forwarding file for main file in draft mode\\
% |cdocsfi1.tex|&forwarding file for final version of chapter 1\\
% |cdocsfi2.tex|&forwarding file for final version of chapter 2\\
% \end{tabular}
% \end{center}
% Each of the eight files can be compiled directly by the \LaTeX{} compiler.
%
% %%%%%%%%%%%%%%%%%%%%%%%%%%%%%%%%%%%%%%
% \paragraph{Main File.}
%
% The main file is called |cdocsamp.tex|.
%
% Load the \textsf{childdoc} definitions and
% declare the filename for the main document:
%    \begin{macrocode}
\input{childdoc.def}
\childdocmain{}
%    \end{macrocode}

% Optional override for |\version| flag:
%    \begin{macrocode}
%%\ifchilddoc\else\providecommand{\version}{draft}\fi
%    \end{macrocode}

% Define the default values for the |\version| flag
% (|final| for the main file and |draft| for childs):
%    \begin{macrocode}
\ifchilddoc
\providecommand{\version}{draft}
\else
\providecommand{\version}{final}
\fi
%    \end{macrocode}

% Load the standard document class:
%    \begin{macrocode}
\documentclass[12pt]{article}
%    \end{macrocode}

% Start the document body:
%    \begin{macrocode}
\begin{document}
%    \end{macrocode}

% Declare a title page.
% Print title, part of document being processed and version flag:
%    \begin{macrocode}
\addtocounter{page}{-1}
\begin{center}
{\LARGE\bfseries{}childdoc example\par}
\vspace{1cm}
\ifchilddoc
\ifchilddocmanual part\else chapter\fi:
`\childdocname' of `\childdocjob'\par
\else
main document: `\childdocjob'\par
\fi
version: \version\par
\end{center}
\newpage
%    \end{macrocode}

% Manually include selected file,
% otherwise process as usual:
%    \begin{macrocode}
\ifchilddocmanual
\section*{part `\childdocname'}
\input{\childdocname}
\else
%    \end{macrocode}

% Include the two chapters:
%    \begin{macrocode}
\include{cdocsch1}
\include{cdocsch2}
%    \end{macrocode}

% Include the two parts unless only chapters should be displayed:
%    \begin{macrocode}
\ifchilddoc\else
\section{part three}
\input{cdocspt3}
\section{part four}
\input{cdocspt4}
\fi
%    \end{macrocode}

% Process as usual until here:
%    \begin{macrocode}
\fi
%    \end{macrocode}

% End of document body:
%    \begin{macrocode}
\end{document}
%    \end{macrocode}
%\iffalse
%</samplemain>
%\fi
%
% %%%%%%%%%%%%%%%%%%%%%%%%%%%%%%%%%%%%%%
% \paragraph{Chapter Include Files.}
%
% The include files are called |cdocsch1.tex| and |cdocsch2.tex|.
%
%\iffalse
%<*samplechap1|samplechap2>
%\fi

% Optional override for |\version| flag:
%    \begin{macrocode}
%%\providecommand{\version}{final}
%    \end{macrocode}

% Include the main document:
%    \begin{macrocode}
\input{childdoc.def}
\childdocof{cdocsamp}
%    \end{macrocode}

%\iffalse
%</samplechap1|samplechap2>
%\fi
%
%\iffalse
%<*samplechap1>
%\fi
% Some text for chapter 1:
%    \begin{macrocode}
\section{one}
some text in chapter one
%    \end{macrocode}

%\iffalse
%</samplechap1>
%\fi
% Some text for chapter 2:
%\iffalse
%<*samplechap2>
%\fi
%    \begin{macrocode}
\section{two}
more text in chapter two
%    \end{macrocode}

%\iffalse
%</samplechap2>
%\fi
%
% %%%%%%%%%%%%%%%%%%%%%%%%%%%%%%%%%%%%%%
% \paragraph{Part Include Files.}
%
% The include files are called |cdocspt3.tex| and |cdocspt4.tex|.
%
%\iffalse
%<*samplepart3|samplepart4>
%\fi

% Optional override for |\version| flag:
%    \begin{macrocode}
%%\providecommand{\version}{final}
%    \end{macrocode}

% Include the main document:
%    \begin{macrocode}
\input{childdoc.def}
\childdocby{cdocsamp}
%    \end{macrocode}

%\iffalse
%</samplepart3|samplepart4>
%\fi
%
%\iffalse
%<*samplepart3>
%\fi
% Some text for part 3:
%    \begin{macrocode}
some text in part three
%    \end{macrocode}

%\iffalse
%</samplepart3>
%\fi
% Some text for part 4:
%\iffalse
%<*samplepart4>
%\fi
%    \begin{macrocode}
more text in part four
%    \end{macrocode}

%\iffalse
%</samplepart4>
%\fi
%
% %%%%%%%%%%%%%%%%%%%%%%%%%%%%%%%%%%%%%%
% \paragraph{Forwarding for a Complete Draft.}
%
% The following forwarding file |cdocsdrf.tex|
% compiles the main document in draft mode:
%\iffalse
%<*sampledraft>
%\fi
%    \begin{macrocode}
\def\version{draft}
\input{childdoc.def}
\childdocforward{cdocsamp}
%    \end{macrocode}

%\iffalse
%</sampledraft>
%\fi
%
% %%%%%%%%%%%%%%%%%%%%%%%%%%%%%%%%%%%%%%
% \paragraph{Forwarding for Final Version of the Chapters.}
%
% The following forwarding files |cdocsfn1.tex| and |cdocsfn2.tex|
% (with identical content)
% compile the final versions of the child documents
% |cdocsch1.tex| and |cdocsch2.tex|, respectively:
%\iffalse
%<*samplefinal>
%\fi
%    \begin{macrocode}
\def\version{final}
\input{childdoc.def}
\childdocforwardprefix[cdocsamp]{cdocsfn}{cdocsch}
%    \end{macrocode}

%\iffalse
%</samplefinal>
%\fi
%
% %%%%%%%%%%%%%%%%%%%%%%%%%%%%%%%%%%%%%%
% \paragraph{Command Line Processing.}
%
% The following three command lines generate the output files
% |cdocscld|, |cdocscl1| and |cdocscl2|
% which should be identical to
% |cdocsdrf|, |cdocsch1| and |cdocsfn2|, respectively:
% \begin{center}
% \begin{tabular}{l}
% |latex -jobname cdocscld \|\\
% |  "\def\version{draft}\input{childdoc.def}\childdocforward{cdocsamp}"|\\
% |latex -jobname cdocscl1 \|\\
% |  "\input{childdoc.def}\childdocforward[cdocsamp]{cdocsch1}"|\\
% |latex -jobname cdocscl2 \|\\
% |  "\def\version{final}\input{childdoc.def}\childdocforward{cdocsch2}"|
% \end{tabular}
% \end{center}
% Note that the trailing backslash on each first line
% merely continues the input to the second line
% (for convenient cut ant paste).
% Furthermore, the command |latex| can be replaced by any
% of its alternative versions such as |pdflatex|.
%
% %%%%%%%%%%%%%%%%%%%%%%%%%%%%%%%%%%%%%%%%%%%%%%%%%%%%%%%%%%%%%%%%%%%%%%%%%%%%%%
% %%%%%%%%%%%%%%%%%%%%%%%%%%%%%%%%%%%%%%%%%%%%%%%%%%%%%%%%%%%%%%%%%%%%%%%%%%%%%%
% \section{Implementation}
%\iffalse
%<*package>
%\fi
%
% This section describes the definitions file |childdoc.def|.

% The definitions cannot be loaded using |\usepackage| or |\RequirePackage|
% which has a mechanism to prevent loading a style file more than once.
% When loading the definitions by means of |\input|
% multiple instances have to be prevented manually:
%\iffalse
%This code needs to be before the `\ProvidesFile' directive
%which is defined at the beginning of this file.
%Therefore it is also placed there and commented out here.
%</package>
%<*discard>
%\fi
%    \begin{macrocode}
\ifdefined\childdocmain\endinput\fi
%    \end{macrocode}
%\iffalse
%</discard>
%<*package>
%\fi
%
% \macro{\ifchilddoc}
% \macro{\ifchilddocmanual}
% The conditional |\ifchilddoc| tells whether a
% child (true) or main (false) document is being compiled.
% The conditional |\ifchilddocmanual| tells whether
% the |\includeonly| mechanism is used (false) or
% the selection of child files must be performed manually (true).
% The definitions initialise to false:
%    \begin{macrocode}
\newif\ifchilddoc
\newif\ifchilddocmanual
%    \end{macrocode}

% \macro{\childdocname}
% \macro{\childdocjob}
% The macro |\childdocname| stores the name of the main document
% to be compiled. The macro |\childdocjob| stores the name of
% the document on which the \LaTeX{} compiler was originally invoked.
% The content of |\jobname| cannot be compared
% to filenames specified in the source due to different catcodes.
% The following code rescans |\jobname|, stores the result
% in |\childdocname| and saves a copy in |\childdocjob|:
%    \begin{macrocode}
\edef\childdocname{\scantokens\expandafter{\jobname\noexpand}}
\let\childdocjob\childdocname
%    \end{macrocode}

% \macro{\childdocdisable}
% The macro |\childdocdisable| prevents the main file
% from being processed more than once.
% At this stage, the main document command |\childdocmain|
% is assumed to be called once again where it should do nothing.
% Any subsequent call to it should prevent
% a secondary processing of the main document
% It overwrites the forwarding commands
% |\childdocof| and |\childdocforward|
% with empty macros to prevent further inclusions of the main document:
%    \begin{macrocode}
\newcommand{\childdocdisable}
{
  \renewcommand{\childdocmain}[1]{\renewcommand{\childdocmain}[1]{\endinput}}
  \renewcommand{\childdocof}[1]{}
  \renewcommand{\childdocby}[2][]{}
  \renewcommand{\childdocforward}[2][]{}
  \renewcommand{\childdocdisable}{}
}
%    \end{macrocode}

% \macro{\childdocmain}
% The macro |\childdocmain| is to be called at the top of the main file
% with nothing or the main filename (without extension) as argument.
% First, it breaks loops.
% If the argument is not empty and does not match |\childdocname|
% (which is set by the first inclusion of |childdoc.def|),
% |\ifchilddoc| is set to true, |\includeonly| is applied to the child file
% and |\jobname| is set to the main file
% (for proper handling of |.aux| files):
%    \begin{macrocode}
\newcommand{\childdocmain}[1]
{
  \childdocdisable\childdocmain{}
  \if?#1?\else
    \begingroup
      \def\childdoctmp{#1}
      \ifx\childdoctmp\childdocname
        \def\childdoctmp{}
      \else
        \def\childdoctmp
        {
          \childdoctrue
          \includeonly{\childdocname}
          \def\childdocjob{#1}
          \def\jobname{#1}
        }
      \fi
      \expandafter
    \endgroup
    \childdoctmp
  \fi
}
%    \end{macrocode}

% \macro{\childdocof}
% The command |\childdocof| redirects
% compilation to the main file |#1|.
%    \begin{macrocode}
\newcommand{\childdocof}[1]
{
  \childdocdisable
  \childdoctrue
  \includeonly{\childdocname}
  \def\jobname{#1}
  \def\childdocjob{#1}
  \input{#1}
}
%    \end{macrocode}

% \macro{\childdocby}
% The command |\childdocby| ....
%    \begin{macrocode}
\newcommand{\childdocby}[2][]
{
  \childdocdisable
  \childdoctrue
  \childdocmanualtrue
  \if?#1?\else
    \def\jobname{#2}
  \fi
  \def\childdocjob{#2}
  \input{#2}
  \endinput
}
%    \end{macrocode}

% \macro{\childdocforward}
% The command |\childdocforward| redirects
% compilation to the main file or
% (if the optional argument is given) a child file.
% Parameters are set as if the main file
% or a child file starting with |\childdocof| was compiled.
% Then compilation is handed over to the main file:
%    \begin{macrocode}
\newcommand{\childdocforward}[2][]
{
  \begingroup
    \if?#1?
      \def\childdoctmp
      {
        \def\childdocname{#2}
        \def\childdocjob{#2}
        \def\jobname{#2}
        \input{#2}
        \endinput
      }
    \else
      \def\childdoctmp
      {
        \childdocdisable
        \def\childdocname{#2}
        \childdoctrue
        \includeonly{#2}
        \def\childdocjob{#1}
        \def\jobname{#1}
        \input{#1}
        \endinput
      }
    \fi
    \expandafter
  \endgroup
  \childdoctmp
}
%    \end{macrocode}

% \macro{\childdocforwardprefix}
% The command |\childdocforwardprefix| redirects
% compilation to the main or a child file by means of a pattern.
% The prefix |#1| in the current filename is replaced by |#2|
% and the suffix of the current filename is kept
% (it is assumed that the filename does not contain the substring `|~~~|'
% which is used as a delimiter).
% Compilation is handed over to the new file by |\childdocforward|:
%    \begin{macrocode}
\newcommand{\childdocforwardprefix}[3][]
{
  \begingroup
    \def\childdocextract #2##1~~~{\def\childdoctmp{\childdocforward[#1]{#3##1}}}
    \expandafter\childdocextract\childdocname~~~
    \expandafter
  \endgroup
  \childdoctmp
}
%    \end{macrocode}

% \macro{\childdoc}
% The deprecated macro |\childdoc| is a legacy version of |\childdocmain|:
%    \begin{macrocode}
\newcommand{\childdoc}{\childdocmain}
%    \end{macrocode}

% \macro{\childdocredirect}
% The deprecated macro |\childdocredirect| is a legacy version
% of |\childdocforward| and |\childdocforwardprefix|:
%    \begin{macrocode}
\newcommand{\childdocredirect}[2][]
{
  \begingroup
    \if?#1?
      \def\childdoctmp{\childdocforward{#2}}
    \else
      \def\childdoctmp{\childdocforwardprefix{#1}{#2}}
    \fi
    \expandafter
  \endgroup
  \childdoctmp
}
%    \end{macrocode}

%\iffalse
%</package>
%\fi
%
\endinput
|\\
|\childdocforward{|\textit{main}|}|\\
\end{tabular}
\end{center}
%
or alternatively with:
%
\begin{center}
\begin{tabular}{l}
|% \iffalse
%
% childdoc.dtx Copyright (C) 2017-2018 Niklas Beisert
%
% This work may be distributed and/or modified under the
% conditions of the LaTeX Project Public License, either version 1.3
% of this license or (at your option) any later version.
% The latest version of this license is in
%   http://www.latex-project.org/lppl.txt
% and version 1.3 or later is part of all distributions of LaTeX
% version 2005/12/01 or later.
%
% This work has the LPPL maintenance status `maintained'.
%
% The Current Maintainer of this work is Niklas Beisert.
%
% This work consists of the files childdoc.dtx and childdoc.ins
% and the derived files childdoc.def and cdocsamp.tex with
% cdocsch1.tex, cdocsch2.tex, cdocsdrf.tex, cdocsfn1.tex, cdocsfn2.tex.
%
%<package>\ifdefined\childdocmain\endinput\fi
%<package>\ProvidesFile{childdoc.def}[2018/12/30 v2.0 child document driver]
%<samplemain>\ProvidesFile{cdocsamp.tex}[2018/12/30 v2.0 sample for childdoc]
%<*driver>
%\ProvidesFile{childdoc.drv}[2018/12/30 v2.0 childdoc reference manual file]
\PassOptionsToClass{10pt,a4paper}{article}
\documentclass{ltxdoc}

\usepackage[margin=35mm]{geometry}
\usepackage{hyperref}
\usepackage{hyperxmp}
\usepackage[usenames]{color}

\hypersetup{colorlinks=true}
\hypersetup{pdfstartview=FitH}
\hypersetup{pdfpagemode=UseNone}
\hypersetup{pdfsource={}}
\hypersetup{pdflang={en-UK}}
\hypersetup{pdfcopyright={Copyright 2017-2018 Niklas Beisert.
  This work may be distributed and/or modified under the
  conditions of the LaTeX Project Public License, either version 1.3
  of this license or (at your option) any later version.}}
\hypersetup{pdflicenseurl={http://www.latex-project.org/lppl.txt}}
\hypersetup{pdfcontactaddress={ETH Zurich, ITP, HIT K,
  Wolfgang-Pauli-Strasse 27}}
\hypersetup{pdfcontactpostcode={8093}}
\hypersetup{pdfcontactcity={Zurich}}
\hypersetup{pdfcontactcountry={Switzerland}}
\hypersetup{pdfcontactemail={nbeisert@itp.phys.ethz.ch}}
\hypersetup{pdfcontacturl={http://people.phys.ethz.ch/\xmptilde nbeisert/}}

\newcommand{\secref}[1]{\hyperref[#1]{section \ref*{#1}}}

\parskip1ex
\parindent0pt
\let\olditemize\itemize
\def\itemize{\olditemize\parskip0pt}

\begin{document}

\title{The \textsf{childdoc} Package}
\hypersetup{pdftitle={The childdoc Package}}
\author{Niklas Beisert\\[2ex]
  Institut f\"ur Theoretische Physik\\
  Eidgen\"ossische Technische Hochschule Z\"urich\\
  Wolfgang-Pauli-Strasse 27, 8093 Z\"urich, Switzerland\\[1ex]
  \href{mailto:nbeisert@itp.phys.ethz.ch}
  {\texttt{nbeisert@itp.phys.ethz.ch}}}
\hypersetup{pdfauthor={Niklas Beisert}}
\hypersetup{pdfsubject={Manual for the LaTeX2e Package childdoc}}
\date{30 December 2018, \textsf{v2.0}}
\maketitle

\begin{abstract}\noindent
\textsf{childdoc} is a \LaTeXe{} package
that enables the direct compilation
of document sections included by |\include|
to individual files.
\end{abstract}

\begingroup
\parskip0ex
\tableofcontents
\endgroup

%%%%%%%%%%%%%%%%%%%%%%%%%%%%%%%%%%%%%%%%%%%%%%%%%%%%%%%%%%%%%%%%%%%%%%%%%%%%%%%%
%%%%%%%%%%%%%%%%%%%%%%%%%%%%%%%%%%%%%%%%%%%%%%%%%%%%%%%%%%%%%%%%%%%%%%%%%%%%%%%%
\section{Introduction}

\LaTeX{} provides a mechanism to structure a large document (such as a book)
into a main file and several child files (containing the chapters)
using the |\include| command.
This mechanism is beneficial for documents
which span hundreds of pages in order to
make the source file(s) more manageable.
Moreover, compilation can be restricted to
selected child files by means of the |\includeonly| command.
The latter feature can be used to reduce the compilation time while editing
(this was significantly more useful in the earlier days of \LaTeX{})
or to generate a smaller document which is easier to navigate.
Another application of |\includeonly| is to generate
documents consisting of selected parts of the complete document.

However, there are a few drawbacks of the plain |\include| mechanism:
\begin{itemize}
\item
The child files cannot be compiled on their own,
they can only be compiled via the main file.
A naive editing environment
(such as a text editor with an option
to have the current file processed by \LaTeX)
may require one to switch to the main file before compiling;
attempting to compile the child file produces errors.
\item
The main file must be modified (each time)
to adjust the |\includeonly| command
to the present needs. This easily leaves the main file in a messy state.
\item
The generated document will always carry the filename
of the main document. This is inconvenient if
several child files are to be compiled and
to be kept for distribution.
\end{itemize}

The present package provides a simple interface
to make child files individually compilable by \LaTeX{}.
Compiling a child file then has the same effect as compiling
the main file with an |\includeonly| command
to select the appropriate child.
Moreover the generated document will carry the name of the child
rather than the main file.
This resolves all three above issues.

This feature is meant to make the editing of books,
thesis documents and lecture notes somewhat more convenient.
However, the package can also be used efficiently for
composing a series of documents (such as exercise sheets)
which are typically distributed individually.
It then assists the author in generating the individual documents
(potentially in different versions)
as well as a document containing the collected series.
Another application is in developing style files
or other kinds of included material
where compilation of the style file could redirect
to a sample or test file.

%%%%%%%%%%%%%%%%%%%%%%%%%%%%%%%%%%%%%%%%%%%%%%%%%%%%%%%%%%%%%%%%%%%%%%%%%%%%%%%%
%%%%%%%%%%%%%%%%%%%%%%%%%%%%%%%%%%%%%%%%%%%%%%%%%%%%%%%%%%%%%%%%%%%%%%%%%%%%%%%%
\section{Usage}

First of all, the package \textsf{childdoc} is \emph{not} a standard
\LaTeXe{} |.sty| style file! Therefore it needs to be invoked in
a non-standard way.

%%%%%%%%%%%%%%%%%%%%%%%%%%%%%%%%%%%%%%%%%%%%%%%%%%%%%%%%%%%%%%%%%%%%%%%%%%%%%%%%
\subsection{Included Files}
\label{sec:include}

%%%%%%%%%%%%%%%%%%%%%%%%%%%%%%%%%%%%%%%%
\DescribeMacro{\childdocmain}
To use the package, add the commands
\begin{center}
\begin{tabular}{l}
|\input{childdoc.def}|\\
|\childdocmain{}|\\
\end{tabular}
\end{center}
at the very top of the main \LaTeX{} file,
in particular \emph{before} the |\documentclass| statement!
The argument of |\childdocmain| should be left empty
(but it must be present).

%%%%%%%%%%%%%%%%%%%%%%%%%%%%%%%%%%%%%%%%
\DescribeMacro{\childdocof}
Furthermore, add the commands
\begin{center}
\begin{tabular}{l}
|\input{childdoc.def}|\\
|\childdocof{|\textit{main}|}|\\
\end{tabular}
\end{center}
at the top of every child file \textit{child}
which is included by |\include{|\textit{child}|}|
from within the main file
(or at least for those files to be compiled individually).
The argument \textit{main} must be the filename of the main file.

There are a couple of
considerations in setting up the main and child documents:

%%%%%%%%%%%%%%%%%%%%%%%%%%%%%%%%%%%%%%%%
\paragraph{Restrictions.}

Please note the following restrictions:
\begin{itemize}
\item
|\childdocmain| must be called with one argument \textit{main}
to ensure compatibility with earlier version of the package.
It must either be empty (|\childdocmain{}|)
or precisely match the filename of the main file in which it is specified.
See \secref{sec:detection} for further information.
\item
The filename \textit{main} must be specified without the |.tex| extension.
\item
The filename \textit{main} is case sensitive
(even in case-insensitive file systems)
due to internal string comparison.
\item
The argument \textit{main} should be fully expanded, it cannot be a macro.
\item
Subdirectories and special characters should be avoided in filenames.
\item
The command |\childdocmain{|\textit{main}|}| must be followed by a whitespace.
It should not be followed immediately by another command
or by a comment mark `|%|'.
This is because the \TeX{} parser reads the token immediately following
the argument of |\childdocmain| and puts it
at the beginning of every child section;
however, a white\-space is ignored.
\end{itemize}

%%%%%%%%%%%%%%%%%%%%%%%%%%%%%%%%%%%%%%%%
\paragraph{Content of Main File.}

It is advisable to place all content in the child files included by |\include|.
Any output contained in the main file will appear in all child documents
unless suppressed manually;
it cannot be suppressed automatically by the |\includeonly| directive
and thus should normally be avoided.
A method to include some content in the main file
by means of conditional processing is described in \secref{sec:conditional}.

%%%%%%%%%%%%%%%%%%%%%%%%%%%%%%%%%%%%%%%%
\paragraph{Page Numbering.}

When only a part of the document is compiled,
the appropriate numbering of pages
(as well as other status parameters)
is determined from the |.aux| files.
The latter contain information from previous passes.
However this information needs to propagate through
all intermediate child documents.
Therefore the page numbering in child documents may well
be inconsistent until the complete document is compiled at least once.

A useful (if unconventional) way to always ensure a consistent
page numbering is to restart the numbering in each child document
and denote the pages by `\textit{child}|.|\textit{page}'
where \textit{child} represents the chapter/section number of the child file.
This can be achieved by the command
|\numberwithin{page}{|\textit{child}|}|
of the \textsf{amsmath} package
where \textit{child} can be |chapter| or |section|
depending on the chosen structuring.
Alternatively, one can modify the macro |\thepage| appropriately
and reset the counter |page| at the start of each child file.

%%%%%%%%%%%%%%%%%%%%%%%%%%%%%%%%%%%%%%%%%%%%%%%%%%%%%%%%%%%%%%%%%%%%%%%%%%%%%%%%
\subsection{Conditional Processing}
\label{sec:conditional}

The package provides a mechanism to compile different versions
of a document. To customise the versions further some conditional processing
can come in handy to distinguish which version is being compiled.
The package provides two macros to describe the compilation context:

%%%%%%%%%%%%%%%%%%%%%%%%%%%%%%%%%%%%%%%%
\DescribeMacro{\ifchilddoc}
The conditional |\ifchilddoc| distinguishes between the compilation of
child documents and the main document:
%
\begin{center}
|\ifchilddoc |\textit{child-code}| |[|\||else |\textit{main-code}]| \||fi|
\end{center}

%%%%%%%%%%%%%%%%%%%%%%%%%%%%%%%%%%%%%%%%
\DescribeMacro{\childdocname}
\DescribeMacro{\childdocjob}
The macro |\childdocname| contains the filename (without extension)
of the main or child file being processed.
Note that |\childdocjob| will always contain the name of the main file.

%%%%%%%%%%%%%%%%%%%%%%%%%%%%%%%%%%%%%%%%
\paragraph{Title Page.}

Conditional processing can be used to include a title or banner page
in the main document when proper precautions are taken.
Importantly, the code in the main file should ensure that the page counter
(as well as other status parameters which are stored in the |.aux| files)
takes the same value after the conditional processing.
Otherwise the page numbers may take divergent values
depending on which part is compiled.

For example, a title page could be declared by:
%
\begin{center}
\begin{tabular}{l}
|\ifchilddoc\||else|\\
|\addtocounter{page}{-1}|\\
\textit{code for title page}\\
|\newpage|\\
|\||fi|
\end{tabular}
\end{center}
%
A banner page for the child documents can be generated by:
%
\begin{center}
\begin{tabular}{l}
|\ifchilddoc|\\
|\addtocounter{page}{-1}|\\
\textit{code for banner page}\\
|\newpage|\\
|\||fi|
\end{tabular}
\end{center}
%
Here one could write a message such as:
\begin{center}
|This is the part \childdocname{} of \childdocjob{}.|
\end{center}

%%%%%%%%%%%%%%%%%%%%%%%%%%%%%%%%%%%%%%%%%%%%%%%%%%%%%%%%%%%%%%%%%%%%%%%%%%%%%%%%
\subsection{Flags}
\label{sec:flags}

The package makes it easy to generate different versions
of the main or child documents.
To this end compilation flags can be defined
and assigned different default values.
They will be particularly useful in conjunction
with the forwarding mechanism described in \secref{sec:forward}.

For example, it may be useful to have a flag |\version|
which can be set to |draft| or |final|.
The document source will contain some conditional code
depending on the value of |\version|.
Suppose further, the flag should default to |final| for the main file
and to |draft| for child files
which is a natural assignment for editing the document.
This is achieved by placing the following code
in the preamble of the main document
(below the |\childdocmain| directive):
%
\begin{center}
\begin{tabular}{l}
|\ifchilddoc|\\
|\providecommand{\version}{draft}|\\
|\||else|\\
|\providecommand{\version}{final}|\\
|\||fi|
\end{tabular}
\end{center}
%
The definition by |\providecommand| makes sure
that previous definitions are not overwritten.
Further statements |\providecommand{\version}{...}|
can thus be added before the above code to override it.

For the main file, one might add a line
(between |\childdocmain| and the above block)
%
\begin{center}
|%\ifchilddoc\||else\providecommand{\version}{draft}\||fi|
\end{center}
%
which can be uncommented to produce a draft version.
Likewise one can add a line to the very top of a child file
(above the |\childdocof{|\textit{main}|}| directive)
%
\begin{center}
|%\providecommand{\version}{final}|
\end{center}
%
which can be uncommented to produce the final version of this child document.

%%%%%%%%%%%%%%%%%%%%%%%%%%%%%%%%%%%%%%%%%%%%%%%%%%%%%%%%%%%%%%%%%%%%%%%%%%%%%%%%
\subsection{Forwarding}
\label{sec:forward}

Different versions of the main or child documents
using compilation flags as described in \secref{sec:flags}
can be (permanently) stored in different files
for convenient compilation, viewing and distribution.
To this end, the package defines a command
to pass on compilation to a different file:

%%%%%%%%%%%%%%%%%%%%%%%%%%%%%%%%%%%%%%%%
\DescribeMacro{\childdocforward}
The command |\childdocforward| redirects processing to
another source file:
%
\begin{center}
\begin{tabular}{l}
|\input{childdoc.def}|\\
|\childdocforward[|\textit{main}|]{|\textit{dest}|}|\\
\end{tabular}
\end{center}
%
The argument \textit{dest} is the destination file
(without extension).
It should be the main file or one of the child files.
Note that further \textsf{childdoc} directives
such as |\childdocof| and |\childdocforward|
in the indicated file will be processed in this form.
The optional argument \textit{main}
passes on directly to the main file \textit{main}
while pretending to compile the child \textit{dest}.
This form behaves as if \textit{dest}
issues |\childdocof{|\textit{main}|}| right away,
and no further \textsf{childdoc} directives will be processed.

%%%%%%%%%%%%%%%%%%%%%%%%%%%%%%%%%%%%%%%%
\DescribeMacro{\...prefix}
In the alternative form |\childdocforwardprefix|,
%
\begin{center}
\begin{tabular}{l}
|\input{childdoc.def}|\\
|\childdocforwardprefix[|\textit{main}|]{|\textit{prefix}|}{|\textit{dest}|}|
\end{tabular}
\end{center}
%
the destination file is determined by a pattern
depending on the current file:
To make this work, the current file must be called
`{\textit{prefix}\hspace{0.2em}\textit{suffix}}'
with \textit{prefix} matching precisely the argument.
Processing is then passed on to the file
`{\textit{dest}\hspace{0.2em}\textit{suffix}}'.
Surely, the same effect is achieved by
directly specifying the
argument `{\textit{dest}\hspace{0.2em}\textit{suffix}}'
in the first form.
However, that requires to set up a different file
for each child. With the alternative form of the command
all these files can have exactly the same content
which simplifies setting them up and maintaining them.

For example, the following file |draft.tex|
with a compilation flag |\version| as described in \secref{sec:flags}
compiles the main document as a draft:
%
\begin{center}
\begin{tabular}{l}
|\def\version{draft}|\\
|\input{childdoc.def}|\\
|\childdocforward{|\textit{main}|}|
\end{tabular}
\end{center}
%
Likewise, the following files |final|\textit{nn}|.tex|
compile the final version of the child document
|child|\textit{nn}|.tex|:
%
\begin{center}
\begin{tabular}{l}
|\def\version{final}|\\
|\input{childdoc.def}|\\
|\childdocforwardprefix{final}{child}|
\end{tabular}
\end{center}
%

Note that when several versions of a main file and/or of each child file
are to be generated, it may be convenient to set up a |Makefile| or
shell script to automatise the process.

%%%%%%%%%%%%%%%%%%%%%%%%%%%%%%%%%%%%%%%%%%%%%%%%%%%%%%%%%%%%%%%%%%%%%%%%%%%%%%%%
\subsection{Command Line Processing}
\label{sec:commandline}

The effect of redirection files can also be achieved by invoking
the \LaTeX{} compiler with a more elaborate command line.
Most conveniently this should be done as part
of a shell script or a |Makefile|.

When using \textsf{childdoc} in the main file, the following
command lines effectively perform a redirection
(note that depending on the shell being used,
backslashes may have to be doubled: `|\|' $\to$ `|\\|'):
%
\begin{center}
|... -jobname "|\textit{target}|" |\\|"|[\textit{flags}]%
|\input{childdoc.def}\childdocforward[|\textit{main}|]{|\textit{dest}|}"|
\end{center}
%
Here \textit{target} is the name of the output file,
\textit{main} is the name of the main file
and \textit{dest} is the name of the main or child file to be processed
(all filenames without extensions).
The optional argument \textit{main} can be omitted
if \textit{main} matches \textit{dest}.
Optionally, compilation \textit{flags} can be defined via |\def| commands.
This command line makes the \TeX{} engine believe
it is compiling the file \textit{target}
whose content is specified as the latter parameter.
The provided code then forwards the processing to
\textit{main} or \textit{dest} as described in \secref{sec:forward}.

%%%%%%%%%%%%%%%%%%%%%%%%%%%%%%%%%%%%%%%%%%%%%%%%%%%%%%%%%%%%%%%%%%%%%%%%%%%%%%%%
\subsection{Include by Input}
\label{sec:input}

Including child documents by |\include| has some restrictions by design.
Most notably, the content of a child document always occupies
its own set of pages; pages cannot be shared between child documents.
Usually, this behaviour makes perfect sense
because each child document contain an essential part of the document.
However, in some situations it may be desirable to compose
a document from a collection of parts
without having mandatory page breaks between then.
For this case, the package
provides a mechanism to include parts
by |\input| which can also be processed individually.
However, by construction this mechanism
requires manual handling of the content to be output.

%%%%%%%%%%%%%%%%%%%%%%%%%%%%%%%%%%%%%%%%
\DescribeMacro{\ifchilddocmanual}
The main file should be prepared as usual, see \secref{sec:include}.
However, the document body must make a distinction
between processing of an individual part and of the main document, e.g.:
%
\begin{center}
\begin{tabular}{l}
|\ifchilddocmanual|\\
|\input{\childdocname}|\\
|\||else|\\
\textit{document body with }|\input{|\textit{part}|}|\\
|\||fi|
\end{tabular}
\end{center}
%
The conditional |\ifchilddocmanual| is true whenever
a part to be included by |\input| is being compiled,
and the name of the part is stored in |\childdocname|.

%%%%%%%%%%%%%%%%%%%%%%%%%%%%%%%%%%%%%%%%
\DescribeMacro{\childdocby}
Each part to be included by |\input| should start with:
%
\begin{center}
\begin{tabular}{l}
|\input{childdoc.def}|\\
|\childdocby{|\textit{main}|}|\\
\end{tabular}
\end{center}
%
The directive |\childdocby| is similar to |\childdocof|
described in \secref{sec:include},
but the subsequent selection of content must be done manually.
To that end, both |\ifchilddoc| and |\ifchilddocmanual|
will be true upon processing of a part,
and the name of the part is stored in |\childdocname|.
Note that |\jobname| will be set to the filename of the current part
so that each part receives an individual |.aux| file
that does not interfere with the |.aux| file(s) of the main document.
This behaviour can be altered by the alternative form
|\childdocby[*]{|\textit{main}|}| (with a non-empty optional argument)
which uses the |.aux| file of the main document
by setting |\jobname| to \textit{main}.

%%%%%%%%%%%%%%%%%%%%%%%%%%%%%%%%%%%%%%%%%%%%%%%%%%%%%%%%%%%%%%%%%%%%%%%%%%%%%%%%
\subsection{Driver Development}
\label{sec:driver}

The \textsf{childdoc} mechanism can also be use for the development
of definition files such as \LaTeX{} styles or classes.
This case differs from the above setup with multiple parts
included by |\include| in that no |\includeonly| should be invoked.
This can be achieved by starting the include file
(before |\ProvidesPackage|) with:
%
\begin{center}
\begin{tabular}{l}
|\input{childdoc.def}|\\
|\childdocforward{|\textit{main}|}|\\
\end{tabular}
\end{center}
%
or alternatively with:
%
\begin{center}
\begin{tabular}{l}
|\input{childdoc.def}|\\
|\childdocby{|\textit{main}|}|\\
\end{tabular}
\end{center}
%
Both forms have slightly different effects as described above.
The main file is prepared as usual, see \secref{sec:include}.

%%%%%%%%%%%%%%%%%%%%%%%%%%%%%%%%%%%%%%%%%%%%%%%%%%%%%%%%%%%%%%%%%%%%%%%%%%%%%%%%
\subsection{Legacy Detection}
\label{sec:detection}

The directive |\childdocmain| in the main file can detect
whether the complete document or merely a child is to be compiled
even without using the directive |\childdocof|.
This method is deprecated because it is less robust
and there is no compelling reason to use it;
it is merely provided for backward compatibility
and it may be removed in future versions.

If the detection mechanism is to be used,
it is mandatory to correctly specify
the filename of the main file as the argument of |\childdocmain|:
%
\begin{center}
\begin{tabular}{l}
|\input{childdoc.def}|\\
|\childdocmain{|\textit{main}|}|\\
\end{tabular}
\end{center}
%
If |\jobname| does not match the argument \textit{main} of |\childdocmain|,
it is assumed that |\jobname| points to the child file to be compiled.
When using |\childdocmain| with the main file specified as argument,
it suffices to start a child file
with just |\input{|\textit{main}|}|
without loading of the package and using |\childdocof|.
If instead all processing is done
with the appropriate \textsf{childdoc} directives,
the argument of \textit{main} of |\childdocmain| can be empty.

An alternative version of the command line processing described
in \secref{sec:commandline} using the detection mechanism reads:
%
\begin{center}
|... -jobname "|\textit{target}|" "|[\textit{flags}]%
[|\def\jobname{|\textit{dest}|}|]|\input{|\textit{main}|}"|
\end{center}

%%%%%%%%%%%%%%%%%%%%%%%%%%%%%%%%%%%%%%%%%%%%%%%%%%%%%%%%%%%%%%%%%%%%%%%%%%%%%%%%
\subsection{Manual Code}
\label{sec:manual}

In case one cannot be certain whether the definitions file |childdoc.def|
is installed on the target \TeX{} distribution
and one prefers not to ship it,
it is conceivable to paste a few relevant commands into the sources.

To that end, drop all statements |\input{childdoc.def}|
and perform the replacements as outlined below.
Instead of |\childdocmain{|\textit{main}|}| add the following code
to the top of the main file:
%
\begin{center}
\begin{tabular}{l}
|\||ifdefined\childdocname\endinput\||fi\newif\ifchilddoc|\\
|\edef\childdocname{\scantokens\expandafter{\jobname\noexpand}}|\\
|\def\childdocmain{|\textit{main}|}\||ifx\childdocmain\childdocname\||else|\\
|\childdoctrue\includeonly{\childdocname}\let\jobname\childdocmain\||fi|\\
\end{tabular}
\end{center}
%
Instead of |\childdocof{|\textit{main}|}| just include the main file
at the top of each child file:
%
\begin{center}
|\input{|\textit{main}|}|
\end{center}
%
A simple redirection |\childdocforward{|\textit{dest}|}| is achieved by:
%
\begin{center}
|\def\jobname{|\textit{dest}|}\input{\jobname}|
\end{center}
%
The redirection with prefix
|\childdocforwardprefix[|\textit{prefix}|]{|\textit{dest}|}|
is accomplished by:
%
\begin{center}
\begin{tabular}{l}
|{\edef\jobname{\scantokens\expandafter{\jobname\noexpand}}|\\
|\def\redirectjob |\textit{prefix}|#1~~~{\gdef\jobname{|\textit{dest}|#1}}|\\
|\expandafter\redirectjob\jobname~~~}\input{\jobname}|
\end{tabular}
\end{center}

In an alternative approach,
child documents can be compiled by a specific command line
without additional code or specific definitions:
%
\begin{center}
|... -jobname "|\textit{target}|" "|[\textit{flags}]%
|\includeonly{|\textit{dest}|}\input{|\textit{main}|}"|
\end{center}
%

%%%%%%%%%%%%%%%%%%%%%%%%%%%%%%%%%%%%%%%%%%%%%%%%%%%%%%%%%%%%%%%%%%%%%%%%%%%%%%%%
%%%%%%%%%%%%%%%%%%%%%%%%%%%%%%%%%%%%%%%%%%%%%%%%%%%%%%%%%%%%%%%%%%%%%%%%%%%%%%%%
\section{Information}

%%%%%%%%%%%%%%%%%%%%%%%%%%%%%%%%%%%%%%%%%%%%%%%%%%%%%%%%%%%%%%%%%%%%%%%%%%%%%%%%
\subsection{Copyright}

Copyright \copyright{} 2017--2018 Niklas Beisert

This work may be distributed and/or modified under the
conditions of the \LaTeX{} Project Public License, either version 1.3
of this license or (at your option) any later version.
The latest version of this license is in
  \url{http://www.latex-project.org/lppl.txt}
and version 1.3 or later is part of all distributions of \LaTeX{}
version 2005/12/01 or later.

This work has the LPPL maintenance status `maintained'.

The Current Maintainer of this work is Niklas Beisert.

This work consists of the files |README.txt|, |childdoc.ins| and |childdoc.dtx|
as well as the derived files |childdoc.def|, |cdocsamp.tex|
with |cdocsch1.tex|, |cdocsch2.tex|, |cdocspt3.tex|, |cdocspt4.tex|,
|cdocsdrf.tex|, |cdocsfn1.tex|, |cdocsfn2.tex|
as well as |childdoc.pdf|.

%%%%%%%%%%%%%%%%%%%%%%%%%%%%%%%%%%%%%%%%%%%%%%%%%%%%%%%%%%%%%%%%%%%%%%%%%%%%%%%%
\subsection{Files and Installation}

The package consists of the files:
%
\begin{center}
\begin{tabular}{ll}
    |README.txt|   & readme file \\
    |childdoc.ins| & installation file \\
    |childdoc.dtx| & source file \\
    |childdoc.def| & definition file \\
    |cdocsamp.tex| & sample main file \\
    |cdocsch1.tex| & sample include file \\
    |cdocsch2.tex| & sample include file \\
    |cdocspt3.tex| & sample part file \\
    |cdocspt4.tex| & sample part file \\
    |cdocsdrf.tex| & sample redirection file \\
    |cdocsfn1.tex| & sample redirection file \\
    |cdocsfn2.tex| & sample redirection file \\
    |childdoc.pdf| & manual
\end{tabular}
\end{center}
%
The distribution consists of the files
|README.txt|, |childdoc.ins| and |childdoc.dtx|.
%
\begin{itemize}
\item
Run (pdf)\LaTeX{} on |childdoc.dtx|
to compile the manual |childdoc.pdf| (this file).
\item
Run \LaTeX{} on |childdoc.ins| to create the definitions file |childdoc.def|
and the sample |cdocsamp.tex| with include files
|cdocsch1.tex|, |cdocsch2.tex|, |cdocspt3.tex|, |cdocspt4.tex|,
|cdocsdrf.tex|, |cdocsfn1.tex|, |cdocsfn2.tex|.
Then copy the file |childdoc.def| to an appropriate directory of your \LaTeX{}
distribution, e.g.\ \textit{texmf-root}|/tex/latex/childdoc|.
\end{itemize}

%%%%%%%%%%%%%%%%%%%%%%%%%%%%%%%%%%%%%%%%%%%%%%%%%%%%%%%%%%%%%%%%%%%%%%%%%%%%%%%%
\subsection{Related CTAN Packages}

There are several other packages which offer a similar functionality:
%
\begin{itemize}
\item
The packages
\href{http://ctan.org/pkg/docmute}{\textsf{docmute}},
\href{http://ctan.org/pkg/includex}{\textsf{includex}} and
\href{http://ctan.org/pkg/standalone}{\textsf{standalone}}
provide commands to include only the document body of
a child file thus allowing both files to be compiled individually.
\item
The packages \href{http://ctan.org/pkg/subdocs}{\textsf{subdocs}}
and \href{http://ctan.org/pkg/subfiles}{\textsf{subfiles}}
provide structures in which the main and child documents can be
encapsulated and allowing them to be compiled individually.
The inclusion mechanism is different from the conventional |\include|.
\item
The package \href{http://ctan.org/pkg/combine}{\textsf{combine}}
is an elaborate solution to combine several documents into one.
\end{itemize}
%
See also the CTAN topic \href{http://ctan.org/topic/subdocs}{\textsf{subdocs}}
for further related packages.
The present package differs from the above solutions in that
a document structure constructed with the conventional |\include| mechanism
just needs two extra commands at the top of every file
such that all constituent files can be compiled individually.

%%%%%%%%%%%%%%%%%%%%%%%%%%%%%%%%%%%%%%%%%%%%%%%%%%%%%%%%%%%%%%%%%%%%%%%%%%%%%%%%
%\subsection{Feature Suggestions}
%
%The following is a list of features which may be useful for future
%versions of this package:
%%
%\begin{itemize}
%\item
%\ldots
%\end{itemize}

%%%%%%%%%%%%%%%%%%%%%%%%%%%%%%%%%%%%%%%%%%%%%%%%%%%%%%%%%%%%%%%%%%%%%%%%%%%%%%%%
\subsection{Revision History}

%%%%%%%%%%%%%%%%%%%%%%%%%%%%%%%%%%%%%%%%
\paragraph{v2.0:} 2018/12/30

\begin{itemize}
\item
immediate forward processing
\item
added |\childdocby| mechanism
\item
manual restructured
\end{itemize}

%%%%%%%%%%%%%%%%%%%%%%%%%%%%%%%%%%%%%%%%
\paragraph{v1.6:} 2018/01/17

\begin{itemize}
\item
application for development of include files
\item
corrections to manual
\end{itemize}

%%%%%%%%%%%%%%%%%%%%%%%%%%%%%%%%%%%%%%%%
\paragraph{v1.5:} 2017/05/21

\begin{itemize}
\item
more complete structuring introduced
\item
|\childdocof| introduced
\item
|\childdoc| renamed to |\childdocmain|
\item
|\childredirect| renamed to |\childdocforward| and |\childdocforwardprefix|
and functionality expanded
\end{itemize}

%%%%%%%%%%%%%%%%%%%%%%%%%%%%%%%%%%%%%%%%
\paragraph{v1.0:} 2017/04/27

\begin{itemize}
\item
manual and install package
\item
first version published on CTAN
\end{itemize}

%%%%%%%%%%%%%%%%%%%%%%%%%%%%%%%%%%%%%%%%
\paragraph{v0.6:} 2017/04/26

\begin{itemize}
\item
redirection mechanism added
\end{itemize}

%%%%%%%%%%%%%%%%%%%%%%%%%%%%%%%%%%%%%%%%
\paragraph{v0.5:} 2017/04/26

\begin{itemize}
\item
functionality in definition file
\end{itemize}


%%%%%%%%%%%%%%%%%%%%%%%%%%%%%%%%%%%%%%%%%%%%%%%%%%%%%%%%%%%%%%%%%%%%%%%%%%%%%%%%
%%%%%%%%%%%%%%%%%%%%%%%%%%%%%%%%%%%%%%%%%%%%%%%%%%%%%%%%%%%%%%%%%%%%%%%%%%%%%%%%
%%%%%%%%%%%%%%%%%%%%%%%%%%%%%%%%%%%%%%%%%%%%%%%%%%%%%%%%%%%%%%%%%%%%%%%%%%%%%%%%
\appendix

\settowidth\MacroIndent{\rmfamily\scriptsize 000\ }

 \DocInput{childdoc.dtx}

\end{document}
%</driver>
% \fi
%
% %%%%%%%%%%%%%%%%%%%%%%%%%%%%%%%%%%%%%%%%%%%%%%%%%%%%%%%%%%%%%%%%%%%%%%%%%%%%%%
% %%%%%%%%%%%%%%%%%%%%%%%%%%%%%%%%%%%%%%%%%%%%%%%%%%%%%%%%%%%%%%%%%%%%%%%%%%%%%%
% \section{Sample}
%\iffalse
%<*samplemain>
%\fi
%
% The following presents a sample document
% with two chapters, two parts, a title page,
% a compile flag as well as three forwarding files to set the flag.
% It consists of eight |.tex| files:
% \begin{center}
% \begin{tabular}{ll}
% |cdocsamp.tex|&main file\\
% |cdocsch1.tex|&include file for chapter 1\\
% |cdocsch2.tex|&include file for chapter 2\\
% |cdocspt3.tex|&include file for part 3\\
% |cdocspt4.tex|&include file for part 4\\
% |cdocsdrf.tex|&forwarding file for main file in draft mode\\
% |cdocsfi1.tex|&forwarding file for final version of chapter 1\\
% |cdocsfi2.tex|&forwarding file for final version of chapter 2\\
% \end{tabular}
% \end{center}
% Each of the eight files can be compiled directly by the \LaTeX{} compiler.
%
% %%%%%%%%%%%%%%%%%%%%%%%%%%%%%%%%%%%%%%
% \paragraph{Main File.}
%
% The main file is called |cdocsamp.tex|.
%
% Load the \textsf{childdoc} definitions and
% declare the filename for the main document:
%    \begin{macrocode}
\input{childdoc.def}
\childdocmain{}
%    \end{macrocode}

% Optional override for |\version| flag:
%    \begin{macrocode}
%%\ifchilddoc\else\providecommand{\version}{draft}\fi
%    \end{macrocode}

% Define the default values for the |\version| flag
% (|final| for the main file and |draft| for childs):
%    \begin{macrocode}
\ifchilddoc
\providecommand{\version}{draft}
\else
\providecommand{\version}{final}
\fi
%    \end{macrocode}

% Load the standard document class:
%    \begin{macrocode}
\documentclass[12pt]{article}
%    \end{macrocode}

% Start the document body:
%    \begin{macrocode}
\begin{document}
%    \end{macrocode}

% Declare a title page.
% Print title, part of document being processed and version flag:
%    \begin{macrocode}
\addtocounter{page}{-1}
\begin{center}
{\LARGE\bfseries{}childdoc example\par}
\vspace{1cm}
\ifchilddoc
\ifchilddocmanual part\else chapter\fi:
`\childdocname' of `\childdocjob'\par
\else
main document: `\childdocjob'\par
\fi
version: \version\par
\end{center}
\newpage
%    \end{macrocode}

% Manually include selected file,
% otherwise process as usual:
%    \begin{macrocode}
\ifchilddocmanual
\section*{part `\childdocname'}
\input{\childdocname}
\else
%    \end{macrocode}

% Include the two chapters:
%    \begin{macrocode}
\include{cdocsch1}
\include{cdocsch2}
%    \end{macrocode}

% Include the two parts unless only chapters should be displayed:
%    \begin{macrocode}
\ifchilddoc\else
\section{part three}
\input{cdocspt3}
\section{part four}
\input{cdocspt4}
\fi
%    \end{macrocode}

% Process as usual until here:
%    \begin{macrocode}
\fi
%    \end{macrocode}

% End of document body:
%    \begin{macrocode}
\end{document}
%    \end{macrocode}
%\iffalse
%</samplemain>
%\fi
%
% %%%%%%%%%%%%%%%%%%%%%%%%%%%%%%%%%%%%%%
% \paragraph{Chapter Include Files.}
%
% The include files are called |cdocsch1.tex| and |cdocsch2.tex|.
%
%\iffalse
%<*samplechap1|samplechap2>
%\fi

% Optional override for |\version| flag:
%    \begin{macrocode}
%%\providecommand{\version}{final}
%    \end{macrocode}

% Include the main document:
%    \begin{macrocode}
\input{childdoc.def}
\childdocof{cdocsamp}
%    \end{macrocode}

%\iffalse
%</samplechap1|samplechap2>
%\fi
%
%\iffalse
%<*samplechap1>
%\fi
% Some text for chapter 1:
%    \begin{macrocode}
\section{one}
some text in chapter one
%    \end{macrocode}

%\iffalse
%</samplechap1>
%\fi
% Some text for chapter 2:
%\iffalse
%<*samplechap2>
%\fi
%    \begin{macrocode}
\section{two}
more text in chapter two
%    \end{macrocode}

%\iffalse
%</samplechap2>
%\fi
%
% %%%%%%%%%%%%%%%%%%%%%%%%%%%%%%%%%%%%%%
% \paragraph{Part Include Files.}
%
% The include files are called |cdocspt3.tex| and |cdocspt4.tex|.
%
%\iffalse
%<*samplepart3|samplepart4>
%\fi

% Optional override for |\version| flag:
%    \begin{macrocode}
%%\providecommand{\version}{final}
%    \end{macrocode}

% Include the main document:
%    \begin{macrocode}
\input{childdoc.def}
\childdocby{cdocsamp}
%    \end{macrocode}

%\iffalse
%</samplepart3|samplepart4>
%\fi
%
%\iffalse
%<*samplepart3>
%\fi
% Some text for part 3:
%    \begin{macrocode}
some text in part three
%    \end{macrocode}

%\iffalse
%</samplepart3>
%\fi
% Some text for part 4:
%\iffalse
%<*samplepart4>
%\fi
%    \begin{macrocode}
more text in part four
%    \end{macrocode}

%\iffalse
%</samplepart4>
%\fi
%
% %%%%%%%%%%%%%%%%%%%%%%%%%%%%%%%%%%%%%%
% \paragraph{Forwarding for a Complete Draft.}
%
% The following forwarding file |cdocsdrf.tex|
% compiles the main document in draft mode:
%\iffalse
%<*sampledraft>
%\fi
%    \begin{macrocode}
\def\version{draft}
\input{childdoc.def}
\childdocforward{cdocsamp}
%    \end{macrocode}

%\iffalse
%</sampledraft>
%\fi
%
% %%%%%%%%%%%%%%%%%%%%%%%%%%%%%%%%%%%%%%
% \paragraph{Forwarding for Final Version of the Chapters.}
%
% The following forwarding files |cdocsfn1.tex| and |cdocsfn2.tex|
% (with identical content)
% compile the final versions of the child documents
% |cdocsch1.tex| and |cdocsch2.tex|, respectively:
%\iffalse
%<*samplefinal>
%\fi
%    \begin{macrocode}
\def\version{final}
\input{childdoc.def}
\childdocforwardprefix[cdocsamp]{cdocsfn}{cdocsch}
%    \end{macrocode}

%\iffalse
%</samplefinal>
%\fi
%
% %%%%%%%%%%%%%%%%%%%%%%%%%%%%%%%%%%%%%%
% \paragraph{Command Line Processing.}
%
% The following three command lines generate the output files
% |cdocscld|, |cdocscl1| and |cdocscl2|
% which should be identical to
% |cdocsdrf|, |cdocsch1| and |cdocsfn2|, respectively:
% \begin{center}
% \begin{tabular}{l}
% |latex -jobname cdocscld \|\\
% |  "\def\version{draft}\input{childdoc.def}\childdocforward{cdocsamp}"|\\
% |latex -jobname cdocscl1 \|\\
% |  "\input{childdoc.def}\childdocforward[cdocsamp]{cdocsch1}"|\\
% |latex -jobname cdocscl2 \|\\
% |  "\def\version{final}\input{childdoc.def}\childdocforward{cdocsch2}"|
% \end{tabular}
% \end{center}
% Note that the trailing backslash on each first line
% merely continues the input to the second line
% (for convenient cut ant paste).
% Furthermore, the command |latex| can be replaced by any
% of its alternative versions such as |pdflatex|.
%
% %%%%%%%%%%%%%%%%%%%%%%%%%%%%%%%%%%%%%%%%%%%%%%%%%%%%%%%%%%%%%%%%%%%%%%%%%%%%%%
% %%%%%%%%%%%%%%%%%%%%%%%%%%%%%%%%%%%%%%%%%%%%%%%%%%%%%%%%%%%%%%%%%%%%%%%%%%%%%%
% \section{Implementation}
%\iffalse
%<*package>
%\fi
%
% This section describes the definitions file |childdoc.def|.

% The definitions cannot be loaded using |\usepackage| or |\RequirePackage|
% which has a mechanism to prevent loading a style file more than once.
% When loading the definitions by means of |\input|
% multiple instances have to be prevented manually:
%\iffalse
%This code needs to be before the `\ProvidesFile' directive
%which is defined at the beginning of this file.
%Therefore it is also placed there and commented out here.
%</package>
%<*discard>
%\fi
%    \begin{macrocode}
\ifdefined\childdocmain\endinput\fi
%    \end{macrocode}
%\iffalse
%</discard>
%<*package>
%\fi
%
% \macro{\ifchilddoc}
% \macro{\ifchilddocmanual}
% The conditional |\ifchilddoc| tells whether a
% child (true) or main (false) document is being compiled.
% The conditional |\ifchilddocmanual| tells whether
% the |\includeonly| mechanism is used (false) or
% the selection of child files must be performed manually (true).
% The definitions initialise to false:
%    \begin{macrocode}
\newif\ifchilddoc
\newif\ifchilddocmanual
%    \end{macrocode}

% \macro{\childdocname}
% \macro{\childdocjob}
% The macro |\childdocname| stores the name of the main document
% to be compiled. The macro |\childdocjob| stores the name of
% the document on which the \LaTeX{} compiler was originally invoked.
% The content of |\jobname| cannot be compared
% to filenames specified in the source due to different catcodes.
% The following code rescans |\jobname|, stores the result
% in |\childdocname| and saves a copy in |\childdocjob|:
%    \begin{macrocode}
\edef\childdocname{\scantokens\expandafter{\jobname\noexpand}}
\let\childdocjob\childdocname
%    \end{macrocode}

% \macro{\childdocdisable}
% The macro |\childdocdisable| prevents the main file
% from being processed more than once.
% At this stage, the main document command |\childdocmain|
% is assumed to be called once again where it should do nothing.
% Any subsequent call to it should prevent
% a secondary processing of the main document
% It overwrites the forwarding commands
% |\childdocof| and |\childdocforward|
% with empty macros to prevent further inclusions of the main document:
%    \begin{macrocode}
\newcommand{\childdocdisable}
{
  \renewcommand{\childdocmain}[1]{\renewcommand{\childdocmain}[1]{\endinput}}
  \renewcommand{\childdocof}[1]{}
  \renewcommand{\childdocby}[2][]{}
  \renewcommand{\childdocforward}[2][]{}
  \renewcommand{\childdocdisable}{}
}
%    \end{macrocode}

% \macro{\childdocmain}
% The macro |\childdocmain| is to be called at the top of the main file
% with nothing or the main filename (without extension) as argument.
% First, it breaks loops.
% If the argument is not empty and does not match |\childdocname|
% (which is set by the first inclusion of |childdoc.def|),
% |\ifchilddoc| is set to true, |\includeonly| is applied to the child file
% and |\jobname| is set to the main file
% (for proper handling of |.aux| files):
%    \begin{macrocode}
\newcommand{\childdocmain}[1]
{
  \childdocdisable\childdocmain{}
  \if?#1?\else
    \begingroup
      \def\childdoctmp{#1}
      \ifx\childdoctmp\childdocname
        \def\childdoctmp{}
      \else
        \def\childdoctmp
        {
          \childdoctrue
          \includeonly{\childdocname}
          \def\childdocjob{#1}
          \def\jobname{#1}
        }
      \fi
      \expandafter
    \endgroup
    \childdoctmp
  \fi
}
%    \end{macrocode}

% \macro{\childdocof}
% The command |\childdocof| redirects
% compilation to the main file |#1|.
%    \begin{macrocode}
\newcommand{\childdocof}[1]
{
  \childdocdisable
  \childdoctrue
  \includeonly{\childdocname}
  \def\jobname{#1}
  \def\childdocjob{#1}
  \input{#1}
}
%    \end{macrocode}

% \macro{\childdocby}
% The command |\childdocby| ....
%    \begin{macrocode}
\newcommand{\childdocby}[2][]
{
  \childdocdisable
  \childdoctrue
  \childdocmanualtrue
  \if?#1?\else
    \def\jobname{#2}
  \fi
  \def\childdocjob{#2}
  \input{#2}
  \endinput
}
%    \end{macrocode}

% \macro{\childdocforward}
% The command |\childdocforward| redirects
% compilation to the main file or
% (if the optional argument is given) a child file.
% Parameters are set as if the main file
% or a child file starting with |\childdocof| was compiled.
% Then compilation is handed over to the main file:
%    \begin{macrocode}
\newcommand{\childdocforward}[2][]
{
  \begingroup
    \if?#1?
      \def\childdoctmp
      {
        \def\childdocname{#2}
        \def\childdocjob{#2}
        \def\jobname{#2}
        \input{#2}
        \endinput
      }
    \else
      \def\childdoctmp
      {
        \childdocdisable
        \def\childdocname{#2}
        \childdoctrue
        \includeonly{#2}
        \def\childdocjob{#1}
        \def\jobname{#1}
        \input{#1}
        \endinput
      }
    \fi
    \expandafter
  \endgroup
  \childdoctmp
}
%    \end{macrocode}

% \macro{\childdocforwardprefix}
% The command |\childdocforwardprefix| redirects
% compilation to the main or a child file by means of a pattern.
% The prefix |#1| in the current filename is replaced by |#2|
% and the suffix of the current filename is kept
% (it is assumed that the filename does not contain the substring `|~~~|'
% which is used as a delimiter).
% Compilation is handed over to the new file by |\childdocforward|:
%    \begin{macrocode}
\newcommand{\childdocforwardprefix}[3][]
{
  \begingroup
    \def\childdocextract #2##1~~~{\def\childdoctmp{\childdocforward[#1]{#3##1}}}
    \expandafter\childdocextract\childdocname~~~
    \expandafter
  \endgroup
  \childdoctmp
}
%    \end{macrocode}

% \macro{\childdoc}
% The deprecated macro |\childdoc| is a legacy version of |\childdocmain|:
%    \begin{macrocode}
\newcommand{\childdoc}{\childdocmain}
%    \end{macrocode}

% \macro{\childdocredirect}
% The deprecated macro |\childdocredirect| is a legacy version
% of |\childdocforward| and |\childdocforwardprefix|:
%    \begin{macrocode}
\newcommand{\childdocredirect}[2][]
{
  \begingroup
    \if?#1?
      \def\childdoctmp{\childdocforward{#2}}
    \else
      \def\childdoctmp{\childdocforwardprefix{#1}{#2}}
    \fi
    \expandafter
  \endgroup
  \childdoctmp
}
%    \end{macrocode}

%\iffalse
%</package>
%\fi
%
\endinput
|\\
|\childdocby{|\textit{main}|}|\\
\end{tabular}
\end{center}
%
Both forms have slightly different effects as described above.
The main file is prepared as usual, see \secref{sec:include}.

%%%%%%%%%%%%%%%%%%%%%%%%%%%%%%%%%%%%%%%%%%%%%%%%%%%%%%%%%%%%%%%%%%%%%%%%%%%%%%%%
\subsection{Legacy Detection}
\label{sec:detection}

The directive |\childdocmain| in the main file can detect
whether the complete document or merely a child is to be compiled
even without using the directive |\childdocof|.
This method is deprecated because it is less robust
and there is no compelling reason to use it;
it is merely provided for backward compatibility
and it may be removed in future versions.

If the detection mechanism is to be used,
it is mandatory to correctly specify
the filename of the main file as the argument of |\childdocmain|:
%
\begin{center}
\begin{tabular}{l}
|% \iffalse
%
% childdoc.dtx Copyright (C) 2017-2018 Niklas Beisert
%
% This work may be distributed and/or modified under the
% conditions of the LaTeX Project Public License, either version 1.3
% of this license or (at your option) any later version.
% The latest version of this license is in
%   http://www.latex-project.org/lppl.txt
% and version 1.3 or later is part of all distributions of LaTeX
% version 2005/12/01 or later.
%
% This work has the LPPL maintenance status `maintained'.
%
% The Current Maintainer of this work is Niklas Beisert.
%
% This work consists of the files childdoc.dtx and childdoc.ins
% and the derived files childdoc.def and cdocsamp.tex with
% cdocsch1.tex, cdocsch2.tex, cdocsdrf.tex, cdocsfn1.tex, cdocsfn2.tex.
%
%<package>\ifdefined\childdocmain\endinput\fi
%<package>\ProvidesFile{childdoc.def}[2018/12/30 v2.0 child document driver]
%<samplemain>\ProvidesFile{cdocsamp.tex}[2018/12/30 v2.0 sample for childdoc]
%<*driver>
%\ProvidesFile{childdoc.drv}[2018/12/30 v2.0 childdoc reference manual file]
\PassOptionsToClass{10pt,a4paper}{article}
\documentclass{ltxdoc}

\usepackage[margin=35mm]{geometry}
\usepackage{hyperref}
\usepackage{hyperxmp}
\usepackage[usenames]{color}

\hypersetup{colorlinks=true}
\hypersetup{pdfstartview=FitH}
\hypersetup{pdfpagemode=UseNone}
\hypersetup{pdfsource={}}
\hypersetup{pdflang={en-UK}}
\hypersetup{pdfcopyright={Copyright 2017-2018 Niklas Beisert.
  This work may be distributed and/or modified under the
  conditions of the LaTeX Project Public License, either version 1.3
  of this license or (at your option) any later version.}}
\hypersetup{pdflicenseurl={http://www.latex-project.org/lppl.txt}}
\hypersetup{pdfcontactaddress={ETH Zurich, ITP, HIT K,
  Wolfgang-Pauli-Strasse 27}}
\hypersetup{pdfcontactpostcode={8093}}
\hypersetup{pdfcontactcity={Zurich}}
\hypersetup{pdfcontactcountry={Switzerland}}
\hypersetup{pdfcontactemail={nbeisert@itp.phys.ethz.ch}}
\hypersetup{pdfcontacturl={http://people.phys.ethz.ch/\xmptilde nbeisert/}}

\newcommand{\secref}[1]{\hyperref[#1]{section \ref*{#1}}}

\parskip1ex
\parindent0pt
\let\olditemize\itemize
\def\itemize{\olditemize\parskip0pt}

\begin{document}

\title{The \textsf{childdoc} Package}
\hypersetup{pdftitle={The childdoc Package}}
\author{Niklas Beisert\\[2ex]
  Institut f\"ur Theoretische Physik\\
  Eidgen\"ossische Technische Hochschule Z\"urich\\
  Wolfgang-Pauli-Strasse 27, 8093 Z\"urich, Switzerland\\[1ex]
  \href{mailto:nbeisert@itp.phys.ethz.ch}
  {\texttt{nbeisert@itp.phys.ethz.ch}}}
\hypersetup{pdfauthor={Niklas Beisert}}
\hypersetup{pdfsubject={Manual for the LaTeX2e Package childdoc}}
\date{30 December 2018, \textsf{v2.0}}
\maketitle

\begin{abstract}\noindent
\textsf{childdoc} is a \LaTeXe{} package
that enables the direct compilation
of document sections included by |\include|
to individual files.
\end{abstract}

\begingroup
\parskip0ex
\tableofcontents
\endgroup

%%%%%%%%%%%%%%%%%%%%%%%%%%%%%%%%%%%%%%%%%%%%%%%%%%%%%%%%%%%%%%%%%%%%%%%%%%%%%%%%
%%%%%%%%%%%%%%%%%%%%%%%%%%%%%%%%%%%%%%%%%%%%%%%%%%%%%%%%%%%%%%%%%%%%%%%%%%%%%%%%
\section{Introduction}

\LaTeX{} provides a mechanism to structure a large document (such as a book)
into a main file and several child files (containing the chapters)
using the |\include| command.
This mechanism is beneficial for documents
which span hundreds of pages in order to
make the source file(s) more manageable.
Moreover, compilation can be restricted to
selected child files by means of the |\includeonly| command.
The latter feature can be used to reduce the compilation time while editing
(this was significantly more useful in the earlier days of \LaTeX{})
or to generate a smaller document which is easier to navigate.
Another application of |\includeonly| is to generate
documents consisting of selected parts of the complete document.

However, there are a few drawbacks of the plain |\include| mechanism:
\begin{itemize}
\item
The child files cannot be compiled on their own,
they can only be compiled via the main file.
A naive editing environment
(such as a text editor with an option
to have the current file processed by \LaTeX)
may require one to switch to the main file before compiling;
attempting to compile the child file produces errors.
\item
The main file must be modified (each time)
to adjust the |\includeonly| command
to the present needs. This easily leaves the main file in a messy state.
\item
The generated document will always carry the filename
of the main document. This is inconvenient if
several child files are to be compiled and
to be kept for distribution.
\end{itemize}

The present package provides a simple interface
to make child files individually compilable by \LaTeX{}.
Compiling a child file then has the same effect as compiling
the main file with an |\includeonly| command
to select the appropriate child.
Moreover the generated document will carry the name of the child
rather than the main file.
This resolves all three above issues.

This feature is meant to make the editing of books,
thesis documents and lecture notes somewhat more convenient.
However, the package can also be used efficiently for
composing a series of documents (such as exercise sheets)
which are typically distributed individually.
It then assists the author in generating the individual documents
(potentially in different versions)
as well as a document containing the collected series.
Another application is in developing style files
or other kinds of included material
where compilation of the style file could redirect
to a sample or test file.

%%%%%%%%%%%%%%%%%%%%%%%%%%%%%%%%%%%%%%%%%%%%%%%%%%%%%%%%%%%%%%%%%%%%%%%%%%%%%%%%
%%%%%%%%%%%%%%%%%%%%%%%%%%%%%%%%%%%%%%%%%%%%%%%%%%%%%%%%%%%%%%%%%%%%%%%%%%%%%%%%
\section{Usage}

First of all, the package \textsf{childdoc} is \emph{not} a standard
\LaTeXe{} |.sty| style file! Therefore it needs to be invoked in
a non-standard way.

%%%%%%%%%%%%%%%%%%%%%%%%%%%%%%%%%%%%%%%%%%%%%%%%%%%%%%%%%%%%%%%%%%%%%%%%%%%%%%%%
\subsection{Included Files}
\label{sec:include}

%%%%%%%%%%%%%%%%%%%%%%%%%%%%%%%%%%%%%%%%
\DescribeMacro{\childdocmain}
To use the package, add the commands
\begin{center}
\begin{tabular}{l}
|\input{childdoc.def}|\\
|\childdocmain{}|\\
\end{tabular}
\end{center}
at the very top of the main \LaTeX{} file,
in particular \emph{before} the |\documentclass| statement!
The argument of |\childdocmain| should be left empty
(but it must be present).

%%%%%%%%%%%%%%%%%%%%%%%%%%%%%%%%%%%%%%%%
\DescribeMacro{\childdocof}
Furthermore, add the commands
\begin{center}
\begin{tabular}{l}
|\input{childdoc.def}|\\
|\childdocof{|\textit{main}|}|\\
\end{tabular}
\end{center}
at the top of every child file \textit{child}
which is included by |\include{|\textit{child}|}|
from within the main file
(or at least for those files to be compiled individually).
The argument \textit{main} must be the filename of the main file.

There are a couple of
considerations in setting up the main and child documents:

%%%%%%%%%%%%%%%%%%%%%%%%%%%%%%%%%%%%%%%%
\paragraph{Restrictions.}

Please note the following restrictions:
\begin{itemize}
\item
|\childdocmain| must be called with one argument \textit{main}
to ensure compatibility with earlier version of the package.
It must either be empty (|\childdocmain{}|)
or precisely match the filename of the main file in which it is specified.
See \secref{sec:detection} for further information.
\item
The filename \textit{main} must be specified without the |.tex| extension.
\item
The filename \textit{main} is case sensitive
(even in case-insensitive file systems)
due to internal string comparison.
\item
The argument \textit{main} should be fully expanded, it cannot be a macro.
\item
Subdirectories and special characters should be avoided in filenames.
\item
The command |\childdocmain{|\textit{main}|}| must be followed by a whitespace.
It should not be followed immediately by another command
or by a comment mark `|%|'.
This is because the \TeX{} parser reads the token immediately following
the argument of |\childdocmain| and puts it
at the beginning of every child section;
however, a white\-space is ignored.
\end{itemize}

%%%%%%%%%%%%%%%%%%%%%%%%%%%%%%%%%%%%%%%%
\paragraph{Content of Main File.}

It is advisable to place all content in the child files included by |\include|.
Any output contained in the main file will appear in all child documents
unless suppressed manually;
it cannot be suppressed automatically by the |\includeonly| directive
and thus should normally be avoided.
A method to include some content in the main file
by means of conditional processing is described in \secref{sec:conditional}.

%%%%%%%%%%%%%%%%%%%%%%%%%%%%%%%%%%%%%%%%
\paragraph{Page Numbering.}

When only a part of the document is compiled,
the appropriate numbering of pages
(as well as other status parameters)
is determined from the |.aux| files.
The latter contain information from previous passes.
However this information needs to propagate through
all intermediate child documents.
Therefore the page numbering in child documents may well
be inconsistent until the complete document is compiled at least once.

A useful (if unconventional) way to always ensure a consistent
page numbering is to restart the numbering in each child document
and denote the pages by `\textit{child}|.|\textit{page}'
where \textit{child} represents the chapter/section number of the child file.
This can be achieved by the command
|\numberwithin{page}{|\textit{child}|}|
of the \textsf{amsmath} package
where \textit{child} can be |chapter| or |section|
depending on the chosen structuring.
Alternatively, one can modify the macro |\thepage| appropriately
and reset the counter |page| at the start of each child file.

%%%%%%%%%%%%%%%%%%%%%%%%%%%%%%%%%%%%%%%%%%%%%%%%%%%%%%%%%%%%%%%%%%%%%%%%%%%%%%%%
\subsection{Conditional Processing}
\label{sec:conditional}

The package provides a mechanism to compile different versions
of a document. To customise the versions further some conditional processing
can come in handy to distinguish which version is being compiled.
The package provides two macros to describe the compilation context:

%%%%%%%%%%%%%%%%%%%%%%%%%%%%%%%%%%%%%%%%
\DescribeMacro{\ifchilddoc}
The conditional |\ifchilddoc| distinguishes between the compilation of
child documents and the main document:
%
\begin{center}
|\ifchilddoc |\textit{child-code}| |[|\||else |\textit{main-code}]| \||fi|
\end{center}

%%%%%%%%%%%%%%%%%%%%%%%%%%%%%%%%%%%%%%%%
\DescribeMacro{\childdocname}
\DescribeMacro{\childdocjob}
The macro |\childdocname| contains the filename (without extension)
of the main or child file being processed.
Note that |\childdocjob| will always contain the name of the main file.

%%%%%%%%%%%%%%%%%%%%%%%%%%%%%%%%%%%%%%%%
\paragraph{Title Page.}

Conditional processing can be used to include a title or banner page
in the main document when proper precautions are taken.
Importantly, the code in the main file should ensure that the page counter
(as well as other status parameters which are stored in the |.aux| files)
takes the same value after the conditional processing.
Otherwise the page numbers may take divergent values
depending on which part is compiled.

For example, a title page could be declared by:
%
\begin{center}
\begin{tabular}{l}
|\ifchilddoc\||else|\\
|\addtocounter{page}{-1}|\\
\textit{code for title page}\\
|\newpage|\\
|\||fi|
\end{tabular}
\end{center}
%
A banner page for the child documents can be generated by:
%
\begin{center}
\begin{tabular}{l}
|\ifchilddoc|\\
|\addtocounter{page}{-1}|\\
\textit{code for banner page}\\
|\newpage|\\
|\||fi|
\end{tabular}
\end{center}
%
Here one could write a message such as:
\begin{center}
|This is the part \childdocname{} of \childdocjob{}.|
\end{center}

%%%%%%%%%%%%%%%%%%%%%%%%%%%%%%%%%%%%%%%%%%%%%%%%%%%%%%%%%%%%%%%%%%%%%%%%%%%%%%%%
\subsection{Flags}
\label{sec:flags}

The package makes it easy to generate different versions
of the main or child documents.
To this end compilation flags can be defined
and assigned different default values.
They will be particularly useful in conjunction
with the forwarding mechanism described in \secref{sec:forward}.

For example, it may be useful to have a flag |\version|
which can be set to |draft| or |final|.
The document source will contain some conditional code
depending on the value of |\version|.
Suppose further, the flag should default to |final| for the main file
and to |draft| for child files
which is a natural assignment for editing the document.
This is achieved by placing the following code
in the preamble of the main document
(below the |\childdocmain| directive):
%
\begin{center}
\begin{tabular}{l}
|\ifchilddoc|\\
|\providecommand{\version}{draft}|\\
|\||else|\\
|\providecommand{\version}{final}|\\
|\||fi|
\end{tabular}
\end{center}
%
The definition by |\providecommand| makes sure
that previous definitions are not overwritten.
Further statements |\providecommand{\version}{...}|
can thus be added before the above code to override it.

For the main file, one might add a line
(between |\childdocmain| and the above block)
%
\begin{center}
|%\ifchilddoc\||else\providecommand{\version}{draft}\||fi|
\end{center}
%
which can be uncommented to produce a draft version.
Likewise one can add a line to the very top of a child file
(above the |\childdocof{|\textit{main}|}| directive)
%
\begin{center}
|%\providecommand{\version}{final}|
\end{center}
%
which can be uncommented to produce the final version of this child document.

%%%%%%%%%%%%%%%%%%%%%%%%%%%%%%%%%%%%%%%%%%%%%%%%%%%%%%%%%%%%%%%%%%%%%%%%%%%%%%%%
\subsection{Forwarding}
\label{sec:forward}

Different versions of the main or child documents
using compilation flags as described in \secref{sec:flags}
can be (permanently) stored in different files
for convenient compilation, viewing and distribution.
To this end, the package defines a command
to pass on compilation to a different file:

%%%%%%%%%%%%%%%%%%%%%%%%%%%%%%%%%%%%%%%%
\DescribeMacro{\childdocforward}
The command |\childdocforward| redirects processing to
another source file:
%
\begin{center}
\begin{tabular}{l}
|\input{childdoc.def}|\\
|\childdocforward[|\textit{main}|]{|\textit{dest}|}|\\
\end{tabular}
\end{center}
%
The argument \textit{dest} is the destination file
(without extension).
It should be the main file or one of the child files.
Note that further \textsf{childdoc} directives
such as |\childdocof| and |\childdocforward|
in the indicated file will be processed in this form.
The optional argument \textit{main}
passes on directly to the main file \textit{main}
while pretending to compile the child \textit{dest}.
This form behaves as if \textit{dest}
issues |\childdocof{|\textit{main}|}| right away,
and no further \textsf{childdoc} directives will be processed.

%%%%%%%%%%%%%%%%%%%%%%%%%%%%%%%%%%%%%%%%
\DescribeMacro{\...prefix}
In the alternative form |\childdocforwardprefix|,
%
\begin{center}
\begin{tabular}{l}
|\input{childdoc.def}|\\
|\childdocforwardprefix[|\textit{main}|]{|\textit{prefix}|}{|\textit{dest}|}|
\end{tabular}
\end{center}
%
the destination file is determined by a pattern
depending on the current file:
To make this work, the current file must be called
`{\textit{prefix}\hspace{0.2em}\textit{suffix}}'
with \textit{prefix} matching precisely the argument.
Processing is then passed on to the file
`{\textit{dest}\hspace{0.2em}\textit{suffix}}'.
Surely, the same effect is achieved by
directly specifying the
argument `{\textit{dest}\hspace{0.2em}\textit{suffix}}'
in the first form.
However, that requires to set up a different file
for each child. With the alternative form of the command
all these files can have exactly the same content
which simplifies setting them up and maintaining them.

For example, the following file |draft.tex|
with a compilation flag |\version| as described in \secref{sec:flags}
compiles the main document as a draft:
%
\begin{center}
\begin{tabular}{l}
|\def\version{draft}|\\
|\input{childdoc.def}|\\
|\childdocforward{|\textit{main}|}|
\end{tabular}
\end{center}
%
Likewise, the following files |final|\textit{nn}|.tex|
compile the final version of the child document
|child|\textit{nn}|.tex|:
%
\begin{center}
\begin{tabular}{l}
|\def\version{final}|\\
|\input{childdoc.def}|\\
|\childdocforwardprefix{final}{child}|
\end{tabular}
\end{center}
%

Note that when several versions of a main file and/or of each child file
are to be generated, it may be convenient to set up a |Makefile| or
shell script to automatise the process.

%%%%%%%%%%%%%%%%%%%%%%%%%%%%%%%%%%%%%%%%%%%%%%%%%%%%%%%%%%%%%%%%%%%%%%%%%%%%%%%%
\subsection{Command Line Processing}
\label{sec:commandline}

The effect of redirection files can also be achieved by invoking
the \LaTeX{} compiler with a more elaborate command line.
Most conveniently this should be done as part
of a shell script or a |Makefile|.

When using \textsf{childdoc} in the main file, the following
command lines effectively perform a redirection
(note that depending on the shell being used,
backslashes may have to be doubled: `|\|' $\to$ `|\\|'):
%
\begin{center}
|... -jobname "|\textit{target}|" |\\|"|[\textit{flags}]%
|\input{childdoc.def}\childdocforward[|\textit{main}|]{|\textit{dest}|}"|
\end{center}
%
Here \textit{target} is the name of the output file,
\textit{main} is the name of the main file
and \textit{dest} is the name of the main or child file to be processed
(all filenames without extensions).
The optional argument \textit{main} can be omitted
if \textit{main} matches \textit{dest}.
Optionally, compilation \textit{flags} can be defined via |\def| commands.
This command line makes the \TeX{} engine believe
it is compiling the file \textit{target}
whose content is specified as the latter parameter.
The provided code then forwards the processing to
\textit{main} or \textit{dest} as described in \secref{sec:forward}.

%%%%%%%%%%%%%%%%%%%%%%%%%%%%%%%%%%%%%%%%%%%%%%%%%%%%%%%%%%%%%%%%%%%%%%%%%%%%%%%%
\subsection{Include by Input}
\label{sec:input}

Including child documents by |\include| has some restrictions by design.
Most notably, the content of a child document always occupies
its own set of pages; pages cannot be shared between child documents.
Usually, this behaviour makes perfect sense
because each child document contain an essential part of the document.
However, in some situations it may be desirable to compose
a document from a collection of parts
without having mandatory page breaks between then.
For this case, the package
provides a mechanism to include parts
by |\input| which can also be processed individually.
However, by construction this mechanism
requires manual handling of the content to be output.

%%%%%%%%%%%%%%%%%%%%%%%%%%%%%%%%%%%%%%%%
\DescribeMacro{\ifchilddocmanual}
The main file should be prepared as usual, see \secref{sec:include}.
However, the document body must make a distinction
between processing of an individual part and of the main document, e.g.:
%
\begin{center}
\begin{tabular}{l}
|\ifchilddocmanual|\\
|\input{\childdocname}|\\
|\||else|\\
\textit{document body with }|\input{|\textit{part}|}|\\
|\||fi|
\end{tabular}
\end{center}
%
The conditional |\ifchilddocmanual| is true whenever
a part to be included by |\input| is being compiled,
and the name of the part is stored in |\childdocname|.

%%%%%%%%%%%%%%%%%%%%%%%%%%%%%%%%%%%%%%%%
\DescribeMacro{\childdocby}
Each part to be included by |\input| should start with:
%
\begin{center}
\begin{tabular}{l}
|\input{childdoc.def}|\\
|\childdocby{|\textit{main}|}|\\
\end{tabular}
\end{center}
%
The directive |\childdocby| is similar to |\childdocof|
described in \secref{sec:include},
but the subsequent selection of content must be done manually.
To that end, both |\ifchilddoc| and |\ifchilddocmanual|
will be true upon processing of a part,
and the name of the part is stored in |\childdocname|.
Note that |\jobname| will be set to the filename of the current part
so that each part receives an individual |.aux| file
that does not interfere with the |.aux| file(s) of the main document.
This behaviour can be altered by the alternative form
|\childdocby[*]{|\textit{main}|}| (with a non-empty optional argument)
which uses the |.aux| file of the main document
by setting |\jobname| to \textit{main}.

%%%%%%%%%%%%%%%%%%%%%%%%%%%%%%%%%%%%%%%%%%%%%%%%%%%%%%%%%%%%%%%%%%%%%%%%%%%%%%%%
\subsection{Driver Development}
\label{sec:driver}

The \textsf{childdoc} mechanism can also be use for the development
of definition files such as \LaTeX{} styles or classes.
This case differs from the above setup with multiple parts
included by |\include| in that no |\includeonly| should be invoked.
This can be achieved by starting the include file
(before |\ProvidesPackage|) with:
%
\begin{center}
\begin{tabular}{l}
|\input{childdoc.def}|\\
|\childdocforward{|\textit{main}|}|\\
\end{tabular}
\end{center}
%
or alternatively with:
%
\begin{center}
\begin{tabular}{l}
|\input{childdoc.def}|\\
|\childdocby{|\textit{main}|}|\\
\end{tabular}
\end{center}
%
Both forms have slightly different effects as described above.
The main file is prepared as usual, see \secref{sec:include}.

%%%%%%%%%%%%%%%%%%%%%%%%%%%%%%%%%%%%%%%%%%%%%%%%%%%%%%%%%%%%%%%%%%%%%%%%%%%%%%%%
\subsection{Legacy Detection}
\label{sec:detection}

The directive |\childdocmain| in the main file can detect
whether the complete document or merely a child is to be compiled
even without using the directive |\childdocof|.
This method is deprecated because it is less robust
and there is no compelling reason to use it;
it is merely provided for backward compatibility
and it may be removed in future versions.

If the detection mechanism is to be used,
it is mandatory to correctly specify
the filename of the main file as the argument of |\childdocmain|:
%
\begin{center}
\begin{tabular}{l}
|\input{childdoc.def}|\\
|\childdocmain{|\textit{main}|}|\\
\end{tabular}
\end{center}
%
If |\jobname| does not match the argument \textit{main} of |\childdocmain|,
it is assumed that |\jobname| points to the child file to be compiled.
When using |\childdocmain| with the main file specified as argument,
it suffices to start a child file
with just |\input{|\textit{main}|}|
without loading of the package and using |\childdocof|.
If instead all processing is done
with the appropriate \textsf{childdoc} directives,
the argument of \textit{main} of |\childdocmain| can be empty.

An alternative version of the command line processing described
in \secref{sec:commandline} using the detection mechanism reads:
%
\begin{center}
|... -jobname "|\textit{target}|" "|[\textit{flags}]%
[|\def\jobname{|\textit{dest}|}|]|\input{|\textit{main}|}"|
\end{center}

%%%%%%%%%%%%%%%%%%%%%%%%%%%%%%%%%%%%%%%%%%%%%%%%%%%%%%%%%%%%%%%%%%%%%%%%%%%%%%%%
\subsection{Manual Code}
\label{sec:manual}

In case one cannot be certain whether the definitions file |childdoc.def|
is installed on the target \TeX{} distribution
and one prefers not to ship it,
it is conceivable to paste a few relevant commands into the sources.

To that end, drop all statements |\input{childdoc.def}|
and perform the replacements as outlined below.
Instead of |\childdocmain{|\textit{main}|}| add the following code
to the top of the main file:
%
\begin{center}
\begin{tabular}{l}
|\||ifdefined\childdocname\endinput\||fi\newif\ifchilddoc|\\
|\edef\childdocname{\scantokens\expandafter{\jobname\noexpand}}|\\
|\def\childdocmain{|\textit{main}|}\||ifx\childdocmain\childdocname\||else|\\
|\childdoctrue\includeonly{\childdocname}\let\jobname\childdocmain\||fi|\\
\end{tabular}
\end{center}
%
Instead of |\childdocof{|\textit{main}|}| just include the main file
at the top of each child file:
%
\begin{center}
|\input{|\textit{main}|}|
\end{center}
%
A simple redirection |\childdocforward{|\textit{dest}|}| is achieved by:
%
\begin{center}
|\def\jobname{|\textit{dest}|}\input{\jobname}|
\end{center}
%
The redirection with prefix
|\childdocforwardprefix[|\textit{prefix}|]{|\textit{dest}|}|
is accomplished by:
%
\begin{center}
\begin{tabular}{l}
|{\edef\jobname{\scantokens\expandafter{\jobname\noexpand}}|\\
|\def\redirectjob |\textit{prefix}|#1~~~{\gdef\jobname{|\textit{dest}|#1}}|\\
|\expandafter\redirectjob\jobname~~~}\input{\jobname}|
\end{tabular}
\end{center}

In an alternative approach,
child documents can be compiled by a specific command line
without additional code or specific definitions:
%
\begin{center}
|... -jobname "|\textit{target}|" "|[\textit{flags}]%
|\includeonly{|\textit{dest}|}\input{|\textit{main}|}"|
\end{center}
%

%%%%%%%%%%%%%%%%%%%%%%%%%%%%%%%%%%%%%%%%%%%%%%%%%%%%%%%%%%%%%%%%%%%%%%%%%%%%%%%%
%%%%%%%%%%%%%%%%%%%%%%%%%%%%%%%%%%%%%%%%%%%%%%%%%%%%%%%%%%%%%%%%%%%%%%%%%%%%%%%%
\section{Information}

%%%%%%%%%%%%%%%%%%%%%%%%%%%%%%%%%%%%%%%%%%%%%%%%%%%%%%%%%%%%%%%%%%%%%%%%%%%%%%%%
\subsection{Copyright}

Copyright \copyright{} 2017--2018 Niklas Beisert

This work may be distributed and/or modified under the
conditions of the \LaTeX{} Project Public License, either version 1.3
of this license or (at your option) any later version.
The latest version of this license is in
  \url{http://www.latex-project.org/lppl.txt}
and version 1.3 or later is part of all distributions of \LaTeX{}
version 2005/12/01 or later.

This work has the LPPL maintenance status `maintained'.

The Current Maintainer of this work is Niklas Beisert.

This work consists of the files |README.txt|, |childdoc.ins| and |childdoc.dtx|
as well as the derived files |childdoc.def|, |cdocsamp.tex|
with |cdocsch1.tex|, |cdocsch2.tex|, |cdocspt3.tex|, |cdocspt4.tex|,
|cdocsdrf.tex|, |cdocsfn1.tex|, |cdocsfn2.tex|
as well as |childdoc.pdf|.

%%%%%%%%%%%%%%%%%%%%%%%%%%%%%%%%%%%%%%%%%%%%%%%%%%%%%%%%%%%%%%%%%%%%%%%%%%%%%%%%
\subsection{Files and Installation}

The package consists of the files:
%
\begin{center}
\begin{tabular}{ll}
    |README.txt|   & readme file \\
    |childdoc.ins| & installation file \\
    |childdoc.dtx| & source file \\
    |childdoc.def| & definition file \\
    |cdocsamp.tex| & sample main file \\
    |cdocsch1.tex| & sample include file \\
    |cdocsch2.tex| & sample include file \\
    |cdocspt3.tex| & sample part file \\
    |cdocspt4.tex| & sample part file \\
    |cdocsdrf.tex| & sample redirection file \\
    |cdocsfn1.tex| & sample redirection file \\
    |cdocsfn2.tex| & sample redirection file \\
    |childdoc.pdf| & manual
\end{tabular}
\end{center}
%
The distribution consists of the files
|README.txt|, |childdoc.ins| and |childdoc.dtx|.
%
\begin{itemize}
\item
Run (pdf)\LaTeX{} on |childdoc.dtx|
to compile the manual |childdoc.pdf| (this file).
\item
Run \LaTeX{} on |childdoc.ins| to create the definitions file |childdoc.def|
and the sample |cdocsamp.tex| with include files
|cdocsch1.tex|, |cdocsch2.tex|, |cdocspt3.tex|, |cdocspt4.tex|,
|cdocsdrf.tex|, |cdocsfn1.tex|, |cdocsfn2.tex|.
Then copy the file |childdoc.def| to an appropriate directory of your \LaTeX{}
distribution, e.g.\ \textit{texmf-root}|/tex/latex/childdoc|.
\end{itemize}

%%%%%%%%%%%%%%%%%%%%%%%%%%%%%%%%%%%%%%%%%%%%%%%%%%%%%%%%%%%%%%%%%%%%%%%%%%%%%%%%
\subsection{Related CTAN Packages}

There are several other packages which offer a similar functionality:
%
\begin{itemize}
\item
The packages
\href{http://ctan.org/pkg/docmute}{\textsf{docmute}},
\href{http://ctan.org/pkg/includex}{\textsf{includex}} and
\href{http://ctan.org/pkg/standalone}{\textsf{standalone}}
provide commands to include only the document body of
a child file thus allowing both files to be compiled individually.
\item
The packages \href{http://ctan.org/pkg/subdocs}{\textsf{subdocs}}
and \href{http://ctan.org/pkg/subfiles}{\textsf{subfiles}}
provide structures in which the main and child documents can be
encapsulated and allowing them to be compiled individually.
The inclusion mechanism is different from the conventional |\include|.
\item
The package \href{http://ctan.org/pkg/combine}{\textsf{combine}}
is an elaborate solution to combine several documents into one.
\end{itemize}
%
See also the CTAN topic \href{http://ctan.org/topic/subdocs}{\textsf{subdocs}}
for further related packages.
The present package differs from the above solutions in that
a document structure constructed with the conventional |\include| mechanism
just needs two extra commands at the top of every file
such that all constituent files can be compiled individually.

%%%%%%%%%%%%%%%%%%%%%%%%%%%%%%%%%%%%%%%%%%%%%%%%%%%%%%%%%%%%%%%%%%%%%%%%%%%%%%%%
%\subsection{Feature Suggestions}
%
%The following is a list of features which may be useful for future
%versions of this package:
%%
%\begin{itemize}
%\item
%\ldots
%\end{itemize}

%%%%%%%%%%%%%%%%%%%%%%%%%%%%%%%%%%%%%%%%%%%%%%%%%%%%%%%%%%%%%%%%%%%%%%%%%%%%%%%%
\subsection{Revision History}

%%%%%%%%%%%%%%%%%%%%%%%%%%%%%%%%%%%%%%%%
\paragraph{v2.0:} 2018/12/30

\begin{itemize}
\item
immediate forward processing
\item
added |\childdocby| mechanism
\item
manual restructured
\end{itemize}

%%%%%%%%%%%%%%%%%%%%%%%%%%%%%%%%%%%%%%%%
\paragraph{v1.6:} 2018/01/17

\begin{itemize}
\item
application for development of include files
\item
corrections to manual
\end{itemize}

%%%%%%%%%%%%%%%%%%%%%%%%%%%%%%%%%%%%%%%%
\paragraph{v1.5:} 2017/05/21

\begin{itemize}
\item
more complete structuring introduced
\item
|\childdocof| introduced
\item
|\childdoc| renamed to |\childdocmain|
\item
|\childredirect| renamed to |\childdocforward| and |\childdocforwardprefix|
and functionality expanded
\end{itemize}

%%%%%%%%%%%%%%%%%%%%%%%%%%%%%%%%%%%%%%%%
\paragraph{v1.0:} 2017/04/27

\begin{itemize}
\item
manual and install package
\item
first version published on CTAN
\end{itemize}

%%%%%%%%%%%%%%%%%%%%%%%%%%%%%%%%%%%%%%%%
\paragraph{v0.6:} 2017/04/26

\begin{itemize}
\item
redirection mechanism added
\end{itemize}

%%%%%%%%%%%%%%%%%%%%%%%%%%%%%%%%%%%%%%%%
\paragraph{v0.5:} 2017/04/26

\begin{itemize}
\item
functionality in definition file
\end{itemize}


%%%%%%%%%%%%%%%%%%%%%%%%%%%%%%%%%%%%%%%%%%%%%%%%%%%%%%%%%%%%%%%%%%%%%%%%%%%%%%%%
%%%%%%%%%%%%%%%%%%%%%%%%%%%%%%%%%%%%%%%%%%%%%%%%%%%%%%%%%%%%%%%%%%%%%%%%%%%%%%%%
%%%%%%%%%%%%%%%%%%%%%%%%%%%%%%%%%%%%%%%%%%%%%%%%%%%%%%%%%%%%%%%%%%%%%%%%%%%%%%%%
\appendix

\settowidth\MacroIndent{\rmfamily\scriptsize 000\ }

 \DocInput{childdoc.dtx}

\end{document}
%</driver>
% \fi
%
% %%%%%%%%%%%%%%%%%%%%%%%%%%%%%%%%%%%%%%%%%%%%%%%%%%%%%%%%%%%%%%%%%%%%%%%%%%%%%%
% %%%%%%%%%%%%%%%%%%%%%%%%%%%%%%%%%%%%%%%%%%%%%%%%%%%%%%%%%%%%%%%%%%%%%%%%%%%%%%
% \section{Sample}
%\iffalse
%<*samplemain>
%\fi
%
% The following presents a sample document
% with two chapters, two parts, a title page,
% a compile flag as well as three forwarding files to set the flag.
% It consists of eight |.tex| files:
% \begin{center}
% \begin{tabular}{ll}
% |cdocsamp.tex|&main file\\
% |cdocsch1.tex|&include file for chapter 1\\
% |cdocsch2.tex|&include file for chapter 2\\
% |cdocspt3.tex|&include file for part 3\\
% |cdocspt4.tex|&include file for part 4\\
% |cdocsdrf.tex|&forwarding file for main file in draft mode\\
% |cdocsfi1.tex|&forwarding file for final version of chapter 1\\
% |cdocsfi2.tex|&forwarding file for final version of chapter 2\\
% \end{tabular}
% \end{center}
% Each of the eight files can be compiled directly by the \LaTeX{} compiler.
%
% %%%%%%%%%%%%%%%%%%%%%%%%%%%%%%%%%%%%%%
% \paragraph{Main File.}
%
% The main file is called |cdocsamp.tex|.
%
% Load the \textsf{childdoc} definitions and
% declare the filename for the main document:
%    \begin{macrocode}
\input{childdoc.def}
\childdocmain{}
%    \end{macrocode}

% Optional override for |\version| flag:
%    \begin{macrocode}
%%\ifchilddoc\else\providecommand{\version}{draft}\fi
%    \end{macrocode}

% Define the default values for the |\version| flag
% (|final| for the main file and |draft| for childs):
%    \begin{macrocode}
\ifchilddoc
\providecommand{\version}{draft}
\else
\providecommand{\version}{final}
\fi
%    \end{macrocode}

% Load the standard document class:
%    \begin{macrocode}
\documentclass[12pt]{article}
%    \end{macrocode}

% Start the document body:
%    \begin{macrocode}
\begin{document}
%    \end{macrocode}

% Declare a title page.
% Print title, part of document being processed and version flag:
%    \begin{macrocode}
\addtocounter{page}{-1}
\begin{center}
{\LARGE\bfseries{}childdoc example\par}
\vspace{1cm}
\ifchilddoc
\ifchilddocmanual part\else chapter\fi:
`\childdocname' of `\childdocjob'\par
\else
main document: `\childdocjob'\par
\fi
version: \version\par
\end{center}
\newpage
%    \end{macrocode}

% Manually include selected file,
% otherwise process as usual:
%    \begin{macrocode}
\ifchilddocmanual
\section*{part `\childdocname'}
\input{\childdocname}
\else
%    \end{macrocode}

% Include the two chapters:
%    \begin{macrocode}
\include{cdocsch1}
\include{cdocsch2}
%    \end{macrocode}

% Include the two parts unless only chapters should be displayed:
%    \begin{macrocode}
\ifchilddoc\else
\section{part three}
\input{cdocspt3}
\section{part four}
\input{cdocspt4}
\fi
%    \end{macrocode}

% Process as usual until here:
%    \begin{macrocode}
\fi
%    \end{macrocode}

% End of document body:
%    \begin{macrocode}
\end{document}
%    \end{macrocode}
%\iffalse
%</samplemain>
%\fi
%
% %%%%%%%%%%%%%%%%%%%%%%%%%%%%%%%%%%%%%%
% \paragraph{Chapter Include Files.}
%
% The include files are called |cdocsch1.tex| and |cdocsch2.tex|.
%
%\iffalse
%<*samplechap1|samplechap2>
%\fi

% Optional override for |\version| flag:
%    \begin{macrocode}
%%\providecommand{\version}{final}
%    \end{macrocode}

% Include the main document:
%    \begin{macrocode}
\input{childdoc.def}
\childdocof{cdocsamp}
%    \end{macrocode}

%\iffalse
%</samplechap1|samplechap2>
%\fi
%
%\iffalse
%<*samplechap1>
%\fi
% Some text for chapter 1:
%    \begin{macrocode}
\section{one}
some text in chapter one
%    \end{macrocode}

%\iffalse
%</samplechap1>
%\fi
% Some text for chapter 2:
%\iffalse
%<*samplechap2>
%\fi
%    \begin{macrocode}
\section{two}
more text in chapter two
%    \end{macrocode}

%\iffalse
%</samplechap2>
%\fi
%
% %%%%%%%%%%%%%%%%%%%%%%%%%%%%%%%%%%%%%%
% \paragraph{Part Include Files.}
%
% The include files are called |cdocspt3.tex| and |cdocspt4.tex|.
%
%\iffalse
%<*samplepart3|samplepart4>
%\fi

% Optional override for |\version| flag:
%    \begin{macrocode}
%%\providecommand{\version}{final}
%    \end{macrocode}

% Include the main document:
%    \begin{macrocode}
\input{childdoc.def}
\childdocby{cdocsamp}
%    \end{macrocode}

%\iffalse
%</samplepart3|samplepart4>
%\fi
%
%\iffalse
%<*samplepart3>
%\fi
% Some text for part 3:
%    \begin{macrocode}
some text in part three
%    \end{macrocode}

%\iffalse
%</samplepart3>
%\fi
% Some text for part 4:
%\iffalse
%<*samplepart4>
%\fi
%    \begin{macrocode}
more text in part four
%    \end{macrocode}

%\iffalse
%</samplepart4>
%\fi
%
% %%%%%%%%%%%%%%%%%%%%%%%%%%%%%%%%%%%%%%
% \paragraph{Forwarding for a Complete Draft.}
%
% The following forwarding file |cdocsdrf.tex|
% compiles the main document in draft mode:
%\iffalse
%<*sampledraft>
%\fi
%    \begin{macrocode}
\def\version{draft}
\input{childdoc.def}
\childdocforward{cdocsamp}
%    \end{macrocode}

%\iffalse
%</sampledraft>
%\fi
%
% %%%%%%%%%%%%%%%%%%%%%%%%%%%%%%%%%%%%%%
% \paragraph{Forwarding for Final Version of the Chapters.}
%
% The following forwarding files |cdocsfn1.tex| and |cdocsfn2.tex|
% (with identical content)
% compile the final versions of the child documents
% |cdocsch1.tex| and |cdocsch2.tex|, respectively:
%\iffalse
%<*samplefinal>
%\fi
%    \begin{macrocode}
\def\version{final}
\input{childdoc.def}
\childdocforwardprefix[cdocsamp]{cdocsfn}{cdocsch}
%    \end{macrocode}

%\iffalse
%</samplefinal>
%\fi
%
% %%%%%%%%%%%%%%%%%%%%%%%%%%%%%%%%%%%%%%
% \paragraph{Command Line Processing.}
%
% The following three command lines generate the output files
% |cdocscld|, |cdocscl1| and |cdocscl2|
% which should be identical to
% |cdocsdrf|, |cdocsch1| and |cdocsfn2|, respectively:
% \begin{center}
% \begin{tabular}{l}
% |latex -jobname cdocscld \|\\
% |  "\def\version{draft}\input{childdoc.def}\childdocforward{cdocsamp}"|\\
% |latex -jobname cdocscl1 \|\\
% |  "\input{childdoc.def}\childdocforward[cdocsamp]{cdocsch1}"|\\
% |latex -jobname cdocscl2 \|\\
% |  "\def\version{final}\input{childdoc.def}\childdocforward{cdocsch2}"|
% \end{tabular}
% \end{center}
% Note that the trailing backslash on each first line
% merely continues the input to the second line
% (for convenient cut ant paste).
% Furthermore, the command |latex| can be replaced by any
% of its alternative versions such as |pdflatex|.
%
% %%%%%%%%%%%%%%%%%%%%%%%%%%%%%%%%%%%%%%%%%%%%%%%%%%%%%%%%%%%%%%%%%%%%%%%%%%%%%%
% %%%%%%%%%%%%%%%%%%%%%%%%%%%%%%%%%%%%%%%%%%%%%%%%%%%%%%%%%%%%%%%%%%%%%%%%%%%%%%
% \section{Implementation}
%\iffalse
%<*package>
%\fi
%
% This section describes the definitions file |childdoc.def|.

% The definitions cannot be loaded using |\usepackage| or |\RequirePackage|
% which has a mechanism to prevent loading a style file more than once.
% When loading the definitions by means of |\input|
% multiple instances have to be prevented manually:
%\iffalse
%This code needs to be before the `\ProvidesFile' directive
%which is defined at the beginning of this file.
%Therefore it is also placed there and commented out here.
%</package>
%<*discard>
%\fi
%    \begin{macrocode}
\ifdefined\childdocmain\endinput\fi
%    \end{macrocode}
%\iffalse
%</discard>
%<*package>
%\fi
%
% \macro{\ifchilddoc}
% \macro{\ifchilddocmanual}
% The conditional |\ifchilddoc| tells whether a
% child (true) or main (false) document is being compiled.
% The conditional |\ifchilddocmanual| tells whether
% the |\includeonly| mechanism is used (false) or
% the selection of child files must be performed manually (true).
% The definitions initialise to false:
%    \begin{macrocode}
\newif\ifchilddoc
\newif\ifchilddocmanual
%    \end{macrocode}

% \macro{\childdocname}
% \macro{\childdocjob}
% The macro |\childdocname| stores the name of the main document
% to be compiled. The macro |\childdocjob| stores the name of
% the document on which the \LaTeX{} compiler was originally invoked.
% The content of |\jobname| cannot be compared
% to filenames specified in the source due to different catcodes.
% The following code rescans |\jobname|, stores the result
% in |\childdocname| and saves a copy in |\childdocjob|:
%    \begin{macrocode}
\edef\childdocname{\scantokens\expandafter{\jobname\noexpand}}
\let\childdocjob\childdocname
%    \end{macrocode}

% \macro{\childdocdisable}
% The macro |\childdocdisable| prevents the main file
% from being processed more than once.
% At this stage, the main document command |\childdocmain|
% is assumed to be called once again where it should do nothing.
% Any subsequent call to it should prevent
% a secondary processing of the main document
% It overwrites the forwarding commands
% |\childdocof| and |\childdocforward|
% with empty macros to prevent further inclusions of the main document:
%    \begin{macrocode}
\newcommand{\childdocdisable}
{
  \renewcommand{\childdocmain}[1]{\renewcommand{\childdocmain}[1]{\endinput}}
  \renewcommand{\childdocof}[1]{}
  \renewcommand{\childdocby}[2][]{}
  \renewcommand{\childdocforward}[2][]{}
  \renewcommand{\childdocdisable}{}
}
%    \end{macrocode}

% \macro{\childdocmain}
% The macro |\childdocmain| is to be called at the top of the main file
% with nothing or the main filename (without extension) as argument.
% First, it breaks loops.
% If the argument is not empty and does not match |\childdocname|
% (which is set by the first inclusion of |childdoc.def|),
% |\ifchilddoc| is set to true, |\includeonly| is applied to the child file
% and |\jobname| is set to the main file
% (for proper handling of |.aux| files):
%    \begin{macrocode}
\newcommand{\childdocmain}[1]
{
  \childdocdisable\childdocmain{}
  \if?#1?\else
    \begingroup
      \def\childdoctmp{#1}
      \ifx\childdoctmp\childdocname
        \def\childdoctmp{}
      \else
        \def\childdoctmp
        {
          \childdoctrue
          \includeonly{\childdocname}
          \def\childdocjob{#1}
          \def\jobname{#1}
        }
      \fi
      \expandafter
    \endgroup
    \childdoctmp
  \fi
}
%    \end{macrocode}

% \macro{\childdocof}
% The command |\childdocof| redirects
% compilation to the main file |#1|.
%    \begin{macrocode}
\newcommand{\childdocof}[1]
{
  \childdocdisable
  \childdoctrue
  \includeonly{\childdocname}
  \def\jobname{#1}
  \def\childdocjob{#1}
  \input{#1}
}
%    \end{macrocode}

% \macro{\childdocby}
% The command |\childdocby| ....
%    \begin{macrocode}
\newcommand{\childdocby}[2][]
{
  \childdocdisable
  \childdoctrue
  \childdocmanualtrue
  \if?#1?\else
    \def\jobname{#2}
  \fi
  \def\childdocjob{#2}
  \input{#2}
  \endinput
}
%    \end{macrocode}

% \macro{\childdocforward}
% The command |\childdocforward| redirects
% compilation to the main file or
% (if the optional argument is given) a child file.
% Parameters are set as if the main file
% or a child file starting with |\childdocof| was compiled.
% Then compilation is handed over to the main file:
%    \begin{macrocode}
\newcommand{\childdocforward}[2][]
{
  \begingroup
    \if?#1?
      \def\childdoctmp
      {
        \def\childdocname{#2}
        \def\childdocjob{#2}
        \def\jobname{#2}
        \input{#2}
        \endinput
      }
    \else
      \def\childdoctmp
      {
        \childdocdisable
        \def\childdocname{#2}
        \childdoctrue
        \includeonly{#2}
        \def\childdocjob{#1}
        \def\jobname{#1}
        \input{#1}
        \endinput
      }
    \fi
    \expandafter
  \endgroup
  \childdoctmp
}
%    \end{macrocode}

% \macro{\childdocforwardprefix}
% The command |\childdocforwardprefix| redirects
% compilation to the main or a child file by means of a pattern.
% The prefix |#1| in the current filename is replaced by |#2|
% and the suffix of the current filename is kept
% (it is assumed that the filename does not contain the substring `|~~~|'
% which is used as a delimiter).
% Compilation is handed over to the new file by |\childdocforward|:
%    \begin{macrocode}
\newcommand{\childdocforwardprefix}[3][]
{
  \begingroup
    \def\childdocextract #2##1~~~{\def\childdoctmp{\childdocforward[#1]{#3##1}}}
    \expandafter\childdocextract\childdocname~~~
    \expandafter
  \endgroup
  \childdoctmp
}
%    \end{macrocode}

% \macro{\childdoc}
% The deprecated macro |\childdoc| is a legacy version of |\childdocmain|:
%    \begin{macrocode}
\newcommand{\childdoc}{\childdocmain}
%    \end{macrocode}

% \macro{\childdocredirect}
% The deprecated macro |\childdocredirect| is a legacy version
% of |\childdocforward| and |\childdocforwardprefix|:
%    \begin{macrocode}
\newcommand{\childdocredirect}[2][]
{
  \begingroup
    \if?#1?
      \def\childdoctmp{\childdocforward{#2}}
    \else
      \def\childdoctmp{\childdocforwardprefix{#1}{#2}}
    \fi
    \expandafter
  \endgroup
  \childdoctmp
}
%    \end{macrocode}

%\iffalse
%</package>
%\fi
%
\endinput
|\\
|\childdocmain{|\textit{main}|}|\\
\end{tabular}
\end{center}
%
If |\jobname| does not match the argument \textit{main} of |\childdocmain|,
it is assumed that |\jobname| points to the child file to be compiled.
When using |\childdocmain| with the main file specified as argument,
it suffices to start a child file
with just |\input{|\textit{main}|}|
without loading of the package and using |\childdocof|.
If instead all processing is done
with the appropriate \textsf{childdoc} directives,
the argument of \textit{main} of |\childdocmain| can be empty.

An alternative version of the command line processing described
in \secref{sec:commandline} using the detection mechanism reads:
%
\begin{center}
|... -jobname "|\textit{target}|" "|[\textit{flags}]%
[|\def\jobname{|\textit{dest}|}|]|\input{|\textit{main}|}"|
\end{center}

%%%%%%%%%%%%%%%%%%%%%%%%%%%%%%%%%%%%%%%%%%%%%%%%%%%%%%%%%%%%%%%%%%%%%%%%%%%%%%%%
\subsection{Manual Code}
\label{sec:manual}

In case one cannot be certain whether the definitions file |childdoc.def|
is installed on the target \TeX{} distribution
and one prefers not to ship it,
it is conceivable to paste a few relevant commands into the sources.

To that end, drop all statements |% \iffalse
%
% childdoc.dtx Copyright (C) 2017-2018 Niklas Beisert
%
% This work may be distributed and/or modified under the
% conditions of the LaTeX Project Public License, either version 1.3
% of this license or (at your option) any later version.
% The latest version of this license is in
%   http://www.latex-project.org/lppl.txt
% and version 1.3 or later is part of all distributions of LaTeX
% version 2005/12/01 or later.
%
% This work has the LPPL maintenance status `maintained'.
%
% The Current Maintainer of this work is Niklas Beisert.
%
% This work consists of the files childdoc.dtx and childdoc.ins
% and the derived files childdoc.def and cdocsamp.tex with
% cdocsch1.tex, cdocsch2.tex, cdocsdrf.tex, cdocsfn1.tex, cdocsfn2.tex.
%
%<package>\ifdefined\childdocmain\endinput\fi
%<package>\ProvidesFile{childdoc.def}[2018/12/30 v2.0 child document driver]
%<samplemain>\ProvidesFile{cdocsamp.tex}[2018/12/30 v2.0 sample for childdoc]
%<*driver>
%\ProvidesFile{childdoc.drv}[2018/12/30 v2.0 childdoc reference manual file]
\PassOptionsToClass{10pt,a4paper}{article}
\documentclass{ltxdoc}

\usepackage[margin=35mm]{geometry}
\usepackage{hyperref}
\usepackage{hyperxmp}
\usepackage[usenames]{color}

\hypersetup{colorlinks=true}
\hypersetup{pdfstartview=FitH}
\hypersetup{pdfpagemode=UseNone}
\hypersetup{pdfsource={}}
\hypersetup{pdflang={en-UK}}
\hypersetup{pdfcopyright={Copyright 2017-2018 Niklas Beisert.
  This work may be distributed and/or modified under the
  conditions of the LaTeX Project Public License, either version 1.3
  of this license or (at your option) any later version.}}
\hypersetup{pdflicenseurl={http://www.latex-project.org/lppl.txt}}
\hypersetup{pdfcontactaddress={ETH Zurich, ITP, HIT K,
  Wolfgang-Pauli-Strasse 27}}
\hypersetup{pdfcontactpostcode={8093}}
\hypersetup{pdfcontactcity={Zurich}}
\hypersetup{pdfcontactcountry={Switzerland}}
\hypersetup{pdfcontactemail={nbeisert@itp.phys.ethz.ch}}
\hypersetup{pdfcontacturl={http://people.phys.ethz.ch/\xmptilde nbeisert/}}

\newcommand{\secref}[1]{\hyperref[#1]{section \ref*{#1}}}

\parskip1ex
\parindent0pt
\let\olditemize\itemize
\def\itemize{\olditemize\parskip0pt}

\begin{document}

\title{The \textsf{childdoc} Package}
\hypersetup{pdftitle={The childdoc Package}}
\author{Niklas Beisert\\[2ex]
  Institut f\"ur Theoretische Physik\\
  Eidgen\"ossische Technische Hochschule Z\"urich\\
  Wolfgang-Pauli-Strasse 27, 8093 Z\"urich, Switzerland\\[1ex]
  \href{mailto:nbeisert@itp.phys.ethz.ch}
  {\texttt{nbeisert@itp.phys.ethz.ch}}}
\hypersetup{pdfauthor={Niklas Beisert}}
\hypersetup{pdfsubject={Manual for the LaTeX2e Package childdoc}}
\date{30 December 2018, \textsf{v2.0}}
\maketitle

\begin{abstract}\noindent
\textsf{childdoc} is a \LaTeXe{} package
that enables the direct compilation
of document sections included by |\include|
to individual files.
\end{abstract}

\begingroup
\parskip0ex
\tableofcontents
\endgroup

%%%%%%%%%%%%%%%%%%%%%%%%%%%%%%%%%%%%%%%%%%%%%%%%%%%%%%%%%%%%%%%%%%%%%%%%%%%%%%%%
%%%%%%%%%%%%%%%%%%%%%%%%%%%%%%%%%%%%%%%%%%%%%%%%%%%%%%%%%%%%%%%%%%%%%%%%%%%%%%%%
\section{Introduction}

\LaTeX{} provides a mechanism to structure a large document (such as a book)
into a main file and several child files (containing the chapters)
using the |\include| command.
This mechanism is beneficial for documents
which span hundreds of pages in order to
make the source file(s) more manageable.
Moreover, compilation can be restricted to
selected child files by means of the |\includeonly| command.
The latter feature can be used to reduce the compilation time while editing
(this was significantly more useful in the earlier days of \LaTeX{})
or to generate a smaller document which is easier to navigate.
Another application of |\includeonly| is to generate
documents consisting of selected parts of the complete document.

However, there are a few drawbacks of the plain |\include| mechanism:
\begin{itemize}
\item
The child files cannot be compiled on their own,
they can only be compiled via the main file.
A naive editing environment
(such as a text editor with an option
to have the current file processed by \LaTeX)
may require one to switch to the main file before compiling;
attempting to compile the child file produces errors.
\item
The main file must be modified (each time)
to adjust the |\includeonly| command
to the present needs. This easily leaves the main file in a messy state.
\item
The generated document will always carry the filename
of the main document. This is inconvenient if
several child files are to be compiled and
to be kept for distribution.
\end{itemize}

The present package provides a simple interface
to make child files individually compilable by \LaTeX{}.
Compiling a child file then has the same effect as compiling
the main file with an |\includeonly| command
to select the appropriate child.
Moreover the generated document will carry the name of the child
rather than the main file.
This resolves all three above issues.

This feature is meant to make the editing of books,
thesis documents and lecture notes somewhat more convenient.
However, the package can also be used efficiently for
composing a series of documents (such as exercise sheets)
which are typically distributed individually.
It then assists the author in generating the individual documents
(potentially in different versions)
as well as a document containing the collected series.
Another application is in developing style files
or other kinds of included material
where compilation of the style file could redirect
to a sample or test file.

%%%%%%%%%%%%%%%%%%%%%%%%%%%%%%%%%%%%%%%%%%%%%%%%%%%%%%%%%%%%%%%%%%%%%%%%%%%%%%%%
%%%%%%%%%%%%%%%%%%%%%%%%%%%%%%%%%%%%%%%%%%%%%%%%%%%%%%%%%%%%%%%%%%%%%%%%%%%%%%%%
\section{Usage}

First of all, the package \textsf{childdoc} is \emph{not} a standard
\LaTeXe{} |.sty| style file! Therefore it needs to be invoked in
a non-standard way.

%%%%%%%%%%%%%%%%%%%%%%%%%%%%%%%%%%%%%%%%%%%%%%%%%%%%%%%%%%%%%%%%%%%%%%%%%%%%%%%%
\subsection{Included Files}
\label{sec:include}

%%%%%%%%%%%%%%%%%%%%%%%%%%%%%%%%%%%%%%%%
\DescribeMacro{\childdocmain}
To use the package, add the commands
\begin{center}
\begin{tabular}{l}
|\input{childdoc.def}|\\
|\childdocmain{}|\\
\end{tabular}
\end{center}
at the very top of the main \LaTeX{} file,
in particular \emph{before} the |\documentclass| statement!
The argument of |\childdocmain| should be left empty
(but it must be present).

%%%%%%%%%%%%%%%%%%%%%%%%%%%%%%%%%%%%%%%%
\DescribeMacro{\childdocof}
Furthermore, add the commands
\begin{center}
\begin{tabular}{l}
|\input{childdoc.def}|\\
|\childdocof{|\textit{main}|}|\\
\end{tabular}
\end{center}
at the top of every child file \textit{child}
which is included by |\include{|\textit{child}|}|
from within the main file
(or at least for those files to be compiled individually).
The argument \textit{main} must be the filename of the main file.

There are a couple of
considerations in setting up the main and child documents:

%%%%%%%%%%%%%%%%%%%%%%%%%%%%%%%%%%%%%%%%
\paragraph{Restrictions.}

Please note the following restrictions:
\begin{itemize}
\item
|\childdocmain| must be called with one argument \textit{main}
to ensure compatibility with earlier version of the package.
It must either be empty (|\childdocmain{}|)
or precisely match the filename of the main file in which it is specified.
See \secref{sec:detection} for further information.
\item
The filename \textit{main} must be specified without the |.tex| extension.
\item
The filename \textit{main} is case sensitive
(even in case-insensitive file systems)
due to internal string comparison.
\item
The argument \textit{main} should be fully expanded, it cannot be a macro.
\item
Subdirectories and special characters should be avoided in filenames.
\item
The command |\childdocmain{|\textit{main}|}| must be followed by a whitespace.
It should not be followed immediately by another command
or by a comment mark `|%|'.
This is because the \TeX{} parser reads the token immediately following
the argument of |\childdocmain| and puts it
at the beginning of every child section;
however, a white\-space is ignored.
\end{itemize}

%%%%%%%%%%%%%%%%%%%%%%%%%%%%%%%%%%%%%%%%
\paragraph{Content of Main File.}

It is advisable to place all content in the child files included by |\include|.
Any output contained in the main file will appear in all child documents
unless suppressed manually;
it cannot be suppressed automatically by the |\includeonly| directive
and thus should normally be avoided.
A method to include some content in the main file
by means of conditional processing is described in \secref{sec:conditional}.

%%%%%%%%%%%%%%%%%%%%%%%%%%%%%%%%%%%%%%%%
\paragraph{Page Numbering.}

When only a part of the document is compiled,
the appropriate numbering of pages
(as well as other status parameters)
is determined from the |.aux| files.
The latter contain information from previous passes.
However this information needs to propagate through
all intermediate child documents.
Therefore the page numbering in child documents may well
be inconsistent until the complete document is compiled at least once.

A useful (if unconventional) way to always ensure a consistent
page numbering is to restart the numbering in each child document
and denote the pages by `\textit{child}|.|\textit{page}'
where \textit{child} represents the chapter/section number of the child file.
This can be achieved by the command
|\numberwithin{page}{|\textit{child}|}|
of the \textsf{amsmath} package
where \textit{child} can be |chapter| or |section|
depending on the chosen structuring.
Alternatively, one can modify the macro |\thepage| appropriately
and reset the counter |page| at the start of each child file.

%%%%%%%%%%%%%%%%%%%%%%%%%%%%%%%%%%%%%%%%%%%%%%%%%%%%%%%%%%%%%%%%%%%%%%%%%%%%%%%%
\subsection{Conditional Processing}
\label{sec:conditional}

The package provides a mechanism to compile different versions
of a document. To customise the versions further some conditional processing
can come in handy to distinguish which version is being compiled.
The package provides two macros to describe the compilation context:

%%%%%%%%%%%%%%%%%%%%%%%%%%%%%%%%%%%%%%%%
\DescribeMacro{\ifchilddoc}
The conditional |\ifchilddoc| distinguishes between the compilation of
child documents and the main document:
%
\begin{center}
|\ifchilddoc |\textit{child-code}| |[|\||else |\textit{main-code}]| \||fi|
\end{center}

%%%%%%%%%%%%%%%%%%%%%%%%%%%%%%%%%%%%%%%%
\DescribeMacro{\childdocname}
\DescribeMacro{\childdocjob}
The macro |\childdocname| contains the filename (without extension)
of the main or child file being processed.
Note that |\childdocjob| will always contain the name of the main file.

%%%%%%%%%%%%%%%%%%%%%%%%%%%%%%%%%%%%%%%%
\paragraph{Title Page.}

Conditional processing can be used to include a title or banner page
in the main document when proper precautions are taken.
Importantly, the code in the main file should ensure that the page counter
(as well as other status parameters which are stored in the |.aux| files)
takes the same value after the conditional processing.
Otherwise the page numbers may take divergent values
depending on which part is compiled.

For example, a title page could be declared by:
%
\begin{center}
\begin{tabular}{l}
|\ifchilddoc\||else|\\
|\addtocounter{page}{-1}|\\
\textit{code for title page}\\
|\newpage|\\
|\||fi|
\end{tabular}
\end{center}
%
A banner page for the child documents can be generated by:
%
\begin{center}
\begin{tabular}{l}
|\ifchilddoc|\\
|\addtocounter{page}{-1}|\\
\textit{code for banner page}\\
|\newpage|\\
|\||fi|
\end{tabular}
\end{center}
%
Here one could write a message such as:
\begin{center}
|This is the part \childdocname{} of \childdocjob{}.|
\end{center}

%%%%%%%%%%%%%%%%%%%%%%%%%%%%%%%%%%%%%%%%%%%%%%%%%%%%%%%%%%%%%%%%%%%%%%%%%%%%%%%%
\subsection{Flags}
\label{sec:flags}

The package makes it easy to generate different versions
of the main or child documents.
To this end compilation flags can be defined
and assigned different default values.
They will be particularly useful in conjunction
with the forwarding mechanism described in \secref{sec:forward}.

For example, it may be useful to have a flag |\version|
which can be set to |draft| or |final|.
The document source will contain some conditional code
depending on the value of |\version|.
Suppose further, the flag should default to |final| for the main file
and to |draft| for child files
which is a natural assignment for editing the document.
This is achieved by placing the following code
in the preamble of the main document
(below the |\childdocmain| directive):
%
\begin{center}
\begin{tabular}{l}
|\ifchilddoc|\\
|\providecommand{\version}{draft}|\\
|\||else|\\
|\providecommand{\version}{final}|\\
|\||fi|
\end{tabular}
\end{center}
%
The definition by |\providecommand| makes sure
that previous definitions are not overwritten.
Further statements |\providecommand{\version}{...}|
can thus be added before the above code to override it.

For the main file, one might add a line
(between |\childdocmain| and the above block)
%
\begin{center}
|%\ifchilddoc\||else\providecommand{\version}{draft}\||fi|
\end{center}
%
which can be uncommented to produce a draft version.
Likewise one can add a line to the very top of a child file
(above the |\childdocof{|\textit{main}|}| directive)
%
\begin{center}
|%\providecommand{\version}{final}|
\end{center}
%
which can be uncommented to produce the final version of this child document.

%%%%%%%%%%%%%%%%%%%%%%%%%%%%%%%%%%%%%%%%%%%%%%%%%%%%%%%%%%%%%%%%%%%%%%%%%%%%%%%%
\subsection{Forwarding}
\label{sec:forward}

Different versions of the main or child documents
using compilation flags as described in \secref{sec:flags}
can be (permanently) stored in different files
for convenient compilation, viewing and distribution.
To this end, the package defines a command
to pass on compilation to a different file:

%%%%%%%%%%%%%%%%%%%%%%%%%%%%%%%%%%%%%%%%
\DescribeMacro{\childdocforward}
The command |\childdocforward| redirects processing to
another source file:
%
\begin{center}
\begin{tabular}{l}
|\input{childdoc.def}|\\
|\childdocforward[|\textit{main}|]{|\textit{dest}|}|\\
\end{tabular}
\end{center}
%
The argument \textit{dest} is the destination file
(without extension).
It should be the main file or one of the child files.
Note that further \textsf{childdoc} directives
such as |\childdocof| and |\childdocforward|
in the indicated file will be processed in this form.
The optional argument \textit{main}
passes on directly to the main file \textit{main}
while pretending to compile the child \textit{dest}.
This form behaves as if \textit{dest}
issues |\childdocof{|\textit{main}|}| right away,
and no further \textsf{childdoc} directives will be processed.

%%%%%%%%%%%%%%%%%%%%%%%%%%%%%%%%%%%%%%%%
\DescribeMacro{\...prefix}
In the alternative form |\childdocforwardprefix|,
%
\begin{center}
\begin{tabular}{l}
|\input{childdoc.def}|\\
|\childdocforwardprefix[|\textit{main}|]{|\textit{prefix}|}{|\textit{dest}|}|
\end{tabular}
\end{center}
%
the destination file is determined by a pattern
depending on the current file:
To make this work, the current file must be called
`{\textit{prefix}\hspace{0.2em}\textit{suffix}}'
with \textit{prefix} matching precisely the argument.
Processing is then passed on to the file
`{\textit{dest}\hspace{0.2em}\textit{suffix}}'.
Surely, the same effect is achieved by
directly specifying the
argument `{\textit{dest}\hspace{0.2em}\textit{suffix}}'
in the first form.
However, that requires to set up a different file
for each child. With the alternative form of the command
all these files can have exactly the same content
which simplifies setting them up and maintaining them.

For example, the following file |draft.tex|
with a compilation flag |\version| as described in \secref{sec:flags}
compiles the main document as a draft:
%
\begin{center}
\begin{tabular}{l}
|\def\version{draft}|\\
|\input{childdoc.def}|\\
|\childdocforward{|\textit{main}|}|
\end{tabular}
\end{center}
%
Likewise, the following files |final|\textit{nn}|.tex|
compile the final version of the child document
|child|\textit{nn}|.tex|:
%
\begin{center}
\begin{tabular}{l}
|\def\version{final}|\\
|\input{childdoc.def}|\\
|\childdocforwardprefix{final}{child}|
\end{tabular}
\end{center}
%

Note that when several versions of a main file and/or of each child file
are to be generated, it may be convenient to set up a |Makefile| or
shell script to automatise the process.

%%%%%%%%%%%%%%%%%%%%%%%%%%%%%%%%%%%%%%%%%%%%%%%%%%%%%%%%%%%%%%%%%%%%%%%%%%%%%%%%
\subsection{Command Line Processing}
\label{sec:commandline}

The effect of redirection files can also be achieved by invoking
the \LaTeX{} compiler with a more elaborate command line.
Most conveniently this should be done as part
of a shell script or a |Makefile|.

When using \textsf{childdoc} in the main file, the following
command lines effectively perform a redirection
(note that depending on the shell being used,
backslashes may have to be doubled: `|\|' $\to$ `|\\|'):
%
\begin{center}
|... -jobname "|\textit{target}|" |\\|"|[\textit{flags}]%
|\input{childdoc.def}\childdocforward[|\textit{main}|]{|\textit{dest}|}"|
\end{center}
%
Here \textit{target} is the name of the output file,
\textit{main} is the name of the main file
and \textit{dest} is the name of the main or child file to be processed
(all filenames without extensions).
The optional argument \textit{main} can be omitted
if \textit{main} matches \textit{dest}.
Optionally, compilation \textit{flags} can be defined via |\def| commands.
This command line makes the \TeX{} engine believe
it is compiling the file \textit{target}
whose content is specified as the latter parameter.
The provided code then forwards the processing to
\textit{main} or \textit{dest} as described in \secref{sec:forward}.

%%%%%%%%%%%%%%%%%%%%%%%%%%%%%%%%%%%%%%%%%%%%%%%%%%%%%%%%%%%%%%%%%%%%%%%%%%%%%%%%
\subsection{Include by Input}
\label{sec:input}

Including child documents by |\include| has some restrictions by design.
Most notably, the content of a child document always occupies
its own set of pages; pages cannot be shared between child documents.
Usually, this behaviour makes perfect sense
because each child document contain an essential part of the document.
However, in some situations it may be desirable to compose
a document from a collection of parts
without having mandatory page breaks between then.
For this case, the package
provides a mechanism to include parts
by |\input| which can also be processed individually.
However, by construction this mechanism
requires manual handling of the content to be output.

%%%%%%%%%%%%%%%%%%%%%%%%%%%%%%%%%%%%%%%%
\DescribeMacro{\ifchilddocmanual}
The main file should be prepared as usual, see \secref{sec:include}.
However, the document body must make a distinction
between processing of an individual part and of the main document, e.g.:
%
\begin{center}
\begin{tabular}{l}
|\ifchilddocmanual|\\
|\input{\childdocname}|\\
|\||else|\\
\textit{document body with }|\input{|\textit{part}|}|\\
|\||fi|
\end{tabular}
\end{center}
%
The conditional |\ifchilddocmanual| is true whenever
a part to be included by |\input| is being compiled,
and the name of the part is stored in |\childdocname|.

%%%%%%%%%%%%%%%%%%%%%%%%%%%%%%%%%%%%%%%%
\DescribeMacro{\childdocby}
Each part to be included by |\input| should start with:
%
\begin{center}
\begin{tabular}{l}
|\input{childdoc.def}|\\
|\childdocby{|\textit{main}|}|\\
\end{tabular}
\end{center}
%
The directive |\childdocby| is similar to |\childdocof|
described in \secref{sec:include},
but the subsequent selection of content must be done manually.
To that end, both |\ifchilddoc| and |\ifchilddocmanual|
will be true upon processing of a part,
and the name of the part is stored in |\childdocname|.
Note that |\jobname| will be set to the filename of the current part
so that each part receives an individual |.aux| file
that does not interfere with the |.aux| file(s) of the main document.
This behaviour can be altered by the alternative form
|\childdocby[*]{|\textit{main}|}| (with a non-empty optional argument)
which uses the |.aux| file of the main document
by setting |\jobname| to \textit{main}.

%%%%%%%%%%%%%%%%%%%%%%%%%%%%%%%%%%%%%%%%%%%%%%%%%%%%%%%%%%%%%%%%%%%%%%%%%%%%%%%%
\subsection{Driver Development}
\label{sec:driver}

The \textsf{childdoc} mechanism can also be use for the development
of definition files such as \LaTeX{} styles or classes.
This case differs from the above setup with multiple parts
included by |\include| in that no |\includeonly| should be invoked.
This can be achieved by starting the include file
(before |\ProvidesPackage|) with:
%
\begin{center}
\begin{tabular}{l}
|\input{childdoc.def}|\\
|\childdocforward{|\textit{main}|}|\\
\end{tabular}
\end{center}
%
or alternatively with:
%
\begin{center}
\begin{tabular}{l}
|\input{childdoc.def}|\\
|\childdocby{|\textit{main}|}|\\
\end{tabular}
\end{center}
%
Both forms have slightly different effects as described above.
The main file is prepared as usual, see \secref{sec:include}.

%%%%%%%%%%%%%%%%%%%%%%%%%%%%%%%%%%%%%%%%%%%%%%%%%%%%%%%%%%%%%%%%%%%%%%%%%%%%%%%%
\subsection{Legacy Detection}
\label{sec:detection}

The directive |\childdocmain| in the main file can detect
whether the complete document or merely a child is to be compiled
even without using the directive |\childdocof|.
This method is deprecated because it is less robust
and there is no compelling reason to use it;
it is merely provided for backward compatibility
and it may be removed in future versions.

If the detection mechanism is to be used,
it is mandatory to correctly specify
the filename of the main file as the argument of |\childdocmain|:
%
\begin{center}
\begin{tabular}{l}
|\input{childdoc.def}|\\
|\childdocmain{|\textit{main}|}|\\
\end{tabular}
\end{center}
%
If |\jobname| does not match the argument \textit{main} of |\childdocmain|,
it is assumed that |\jobname| points to the child file to be compiled.
When using |\childdocmain| with the main file specified as argument,
it suffices to start a child file
with just |\input{|\textit{main}|}|
without loading of the package and using |\childdocof|.
If instead all processing is done
with the appropriate \textsf{childdoc} directives,
the argument of \textit{main} of |\childdocmain| can be empty.

An alternative version of the command line processing described
in \secref{sec:commandline} using the detection mechanism reads:
%
\begin{center}
|... -jobname "|\textit{target}|" "|[\textit{flags}]%
[|\def\jobname{|\textit{dest}|}|]|\input{|\textit{main}|}"|
\end{center}

%%%%%%%%%%%%%%%%%%%%%%%%%%%%%%%%%%%%%%%%%%%%%%%%%%%%%%%%%%%%%%%%%%%%%%%%%%%%%%%%
\subsection{Manual Code}
\label{sec:manual}

In case one cannot be certain whether the definitions file |childdoc.def|
is installed on the target \TeX{} distribution
and one prefers not to ship it,
it is conceivable to paste a few relevant commands into the sources.

To that end, drop all statements |\input{childdoc.def}|
and perform the replacements as outlined below.
Instead of |\childdocmain{|\textit{main}|}| add the following code
to the top of the main file:
%
\begin{center}
\begin{tabular}{l}
|\||ifdefined\childdocname\endinput\||fi\newif\ifchilddoc|\\
|\edef\childdocname{\scantokens\expandafter{\jobname\noexpand}}|\\
|\def\childdocmain{|\textit{main}|}\||ifx\childdocmain\childdocname\||else|\\
|\childdoctrue\includeonly{\childdocname}\let\jobname\childdocmain\||fi|\\
\end{tabular}
\end{center}
%
Instead of |\childdocof{|\textit{main}|}| just include the main file
at the top of each child file:
%
\begin{center}
|\input{|\textit{main}|}|
\end{center}
%
A simple redirection |\childdocforward{|\textit{dest}|}| is achieved by:
%
\begin{center}
|\def\jobname{|\textit{dest}|}\input{\jobname}|
\end{center}
%
The redirection with prefix
|\childdocforwardprefix[|\textit{prefix}|]{|\textit{dest}|}|
is accomplished by:
%
\begin{center}
\begin{tabular}{l}
|{\edef\jobname{\scantokens\expandafter{\jobname\noexpand}}|\\
|\def\redirectjob |\textit{prefix}|#1~~~{\gdef\jobname{|\textit{dest}|#1}}|\\
|\expandafter\redirectjob\jobname~~~}\input{\jobname}|
\end{tabular}
\end{center}

In an alternative approach,
child documents can be compiled by a specific command line
without additional code or specific definitions:
%
\begin{center}
|... -jobname "|\textit{target}|" "|[\textit{flags}]%
|\includeonly{|\textit{dest}|}\input{|\textit{main}|}"|
\end{center}
%

%%%%%%%%%%%%%%%%%%%%%%%%%%%%%%%%%%%%%%%%%%%%%%%%%%%%%%%%%%%%%%%%%%%%%%%%%%%%%%%%
%%%%%%%%%%%%%%%%%%%%%%%%%%%%%%%%%%%%%%%%%%%%%%%%%%%%%%%%%%%%%%%%%%%%%%%%%%%%%%%%
\section{Information}

%%%%%%%%%%%%%%%%%%%%%%%%%%%%%%%%%%%%%%%%%%%%%%%%%%%%%%%%%%%%%%%%%%%%%%%%%%%%%%%%
\subsection{Copyright}

Copyright \copyright{} 2017--2018 Niklas Beisert

This work may be distributed and/or modified under the
conditions of the \LaTeX{} Project Public License, either version 1.3
of this license or (at your option) any later version.
The latest version of this license is in
  \url{http://www.latex-project.org/lppl.txt}
and version 1.3 or later is part of all distributions of \LaTeX{}
version 2005/12/01 or later.

This work has the LPPL maintenance status `maintained'.

The Current Maintainer of this work is Niklas Beisert.

This work consists of the files |README.txt|, |childdoc.ins| and |childdoc.dtx|
as well as the derived files |childdoc.def|, |cdocsamp.tex|
with |cdocsch1.tex|, |cdocsch2.tex|, |cdocspt3.tex|, |cdocspt4.tex|,
|cdocsdrf.tex|, |cdocsfn1.tex|, |cdocsfn2.tex|
as well as |childdoc.pdf|.

%%%%%%%%%%%%%%%%%%%%%%%%%%%%%%%%%%%%%%%%%%%%%%%%%%%%%%%%%%%%%%%%%%%%%%%%%%%%%%%%
\subsection{Files and Installation}

The package consists of the files:
%
\begin{center}
\begin{tabular}{ll}
    |README.txt|   & readme file \\
    |childdoc.ins| & installation file \\
    |childdoc.dtx| & source file \\
    |childdoc.def| & definition file \\
    |cdocsamp.tex| & sample main file \\
    |cdocsch1.tex| & sample include file \\
    |cdocsch2.tex| & sample include file \\
    |cdocspt3.tex| & sample part file \\
    |cdocspt4.tex| & sample part file \\
    |cdocsdrf.tex| & sample redirection file \\
    |cdocsfn1.tex| & sample redirection file \\
    |cdocsfn2.tex| & sample redirection file \\
    |childdoc.pdf| & manual
\end{tabular}
\end{center}
%
The distribution consists of the files
|README.txt|, |childdoc.ins| and |childdoc.dtx|.
%
\begin{itemize}
\item
Run (pdf)\LaTeX{} on |childdoc.dtx|
to compile the manual |childdoc.pdf| (this file).
\item
Run \LaTeX{} on |childdoc.ins| to create the definitions file |childdoc.def|
and the sample |cdocsamp.tex| with include files
|cdocsch1.tex|, |cdocsch2.tex|, |cdocspt3.tex|, |cdocspt4.tex|,
|cdocsdrf.tex|, |cdocsfn1.tex|, |cdocsfn2.tex|.
Then copy the file |childdoc.def| to an appropriate directory of your \LaTeX{}
distribution, e.g.\ \textit{texmf-root}|/tex/latex/childdoc|.
\end{itemize}

%%%%%%%%%%%%%%%%%%%%%%%%%%%%%%%%%%%%%%%%%%%%%%%%%%%%%%%%%%%%%%%%%%%%%%%%%%%%%%%%
\subsection{Related CTAN Packages}

There are several other packages which offer a similar functionality:
%
\begin{itemize}
\item
The packages
\href{http://ctan.org/pkg/docmute}{\textsf{docmute}},
\href{http://ctan.org/pkg/includex}{\textsf{includex}} and
\href{http://ctan.org/pkg/standalone}{\textsf{standalone}}
provide commands to include only the document body of
a child file thus allowing both files to be compiled individually.
\item
The packages \href{http://ctan.org/pkg/subdocs}{\textsf{subdocs}}
and \href{http://ctan.org/pkg/subfiles}{\textsf{subfiles}}
provide structures in which the main and child documents can be
encapsulated and allowing them to be compiled individually.
The inclusion mechanism is different from the conventional |\include|.
\item
The package \href{http://ctan.org/pkg/combine}{\textsf{combine}}
is an elaborate solution to combine several documents into one.
\end{itemize}
%
See also the CTAN topic \href{http://ctan.org/topic/subdocs}{\textsf{subdocs}}
for further related packages.
The present package differs from the above solutions in that
a document structure constructed with the conventional |\include| mechanism
just needs two extra commands at the top of every file
such that all constituent files can be compiled individually.

%%%%%%%%%%%%%%%%%%%%%%%%%%%%%%%%%%%%%%%%%%%%%%%%%%%%%%%%%%%%%%%%%%%%%%%%%%%%%%%%
%\subsection{Feature Suggestions}
%
%The following is a list of features which may be useful for future
%versions of this package:
%%
%\begin{itemize}
%\item
%\ldots
%\end{itemize}

%%%%%%%%%%%%%%%%%%%%%%%%%%%%%%%%%%%%%%%%%%%%%%%%%%%%%%%%%%%%%%%%%%%%%%%%%%%%%%%%
\subsection{Revision History}

%%%%%%%%%%%%%%%%%%%%%%%%%%%%%%%%%%%%%%%%
\paragraph{v2.0:} 2018/12/30

\begin{itemize}
\item
immediate forward processing
\item
added |\childdocby| mechanism
\item
manual restructured
\end{itemize}

%%%%%%%%%%%%%%%%%%%%%%%%%%%%%%%%%%%%%%%%
\paragraph{v1.6:} 2018/01/17

\begin{itemize}
\item
application for development of include files
\item
corrections to manual
\end{itemize}

%%%%%%%%%%%%%%%%%%%%%%%%%%%%%%%%%%%%%%%%
\paragraph{v1.5:} 2017/05/21

\begin{itemize}
\item
more complete structuring introduced
\item
|\childdocof| introduced
\item
|\childdoc| renamed to |\childdocmain|
\item
|\childredirect| renamed to |\childdocforward| and |\childdocforwardprefix|
and functionality expanded
\end{itemize}

%%%%%%%%%%%%%%%%%%%%%%%%%%%%%%%%%%%%%%%%
\paragraph{v1.0:} 2017/04/27

\begin{itemize}
\item
manual and install package
\item
first version published on CTAN
\end{itemize}

%%%%%%%%%%%%%%%%%%%%%%%%%%%%%%%%%%%%%%%%
\paragraph{v0.6:} 2017/04/26

\begin{itemize}
\item
redirection mechanism added
\end{itemize}

%%%%%%%%%%%%%%%%%%%%%%%%%%%%%%%%%%%%%%%%
\paragraph{v0.5:} 2017/04/26

\begin{itemize}
\item
functionality in definition file
\end{itemize}


%%%%%%%%%%%%%%%%%%%%%%%%%%%%%%%%%%%%%%%%%%%%%%%%%%%%%%%%%%%%%%%%%%%%%%%%%%%%%%%%
%%%%%%%%%%%%%%%%%%%%%%%%%%%%%%%%%%%%%%%%%%%%%%%%%%%%%%%%%%%%%%%%%%%%%%%%%%%%%%%%
%%%%%%%%%%%%%%%%%%%%%%%%%%%%%%%%%%%%%%%%%%%%%%%%%%%%%%%%%%%%%%%%%%%%%%%%%%%%%%%%
\appendix

\settowidth\MacroIndent{\rmfamily\scriptsize 000\ }

 \DocInput{childdoc.dtx}

\end{document}
%</driver>
% \fi
%
% %%%%%%%%%%%%%%%%%%%%%%%%%%%%%%%%%%%%%%%%%%%%%%%%%%%%%%%%%%%%%%%%%%%%%%%%%%%%%%
% %%%%%%%%%%%%%%%%%%%%%%%%%%%%%%%%%%%%%%%%%%%%%%%%%%%%%%%%%%%%%%%%%%%%%%%%%%%%%%
% \section{Sample}
%\iffalse
%<*samplemain>
%\fi
%
% The following presents a sample document
% with two chapters, two parts, a title page,
% a compile flag as well as three forwarding files to set the flag.
% It consists of eight |.tex| files:
% \begin{center}
% \begin{tabular}{ll}
% |cdocsamp.tex|&main file\\
% |cdocsch1.tex|&include file for chapter 1\\
% |cdocsch2.tex|&include file for chapter 2\\
% |cdocspt3.tex|&include file for part 3\\
% |cdocspt4.tex|&include file for part 4\\
% |cdocsdrf.tex|&forwarding file for main file in draft mode\\
% |cdocsfi1.tex|&forwarding file for final version of chapter 1\\
% |cdocsfi2.tex|&forwarding file for final version of chapter 2\\
% \end{tabular}
% \end{center}
% Each of the eight files can be compiled directly by the \LaTeX{} compiler.
%
% %%%%%%%%%%%%%%%%%%%%%%%%%%%%%%%%%%%%%%
% \paragraph{Main File.}
%
% The main file is called |cdocsamp.tex|.
%
% Load the \textsf{childdoc} definitions and
% declare the filename for the main document:
%    \begin{macrocode}
\input{childdoc.def}
\childdocmain{}
%    \end{macrocode}

% Optional override for |\version| flag:
%    \begin{macrocode}
%%\ifchilddoc\else\providecommand{\version}{draft}\fi
%    \end{macrocode}

% Define the default values for the |\version| flag
% (|final| for the main file and |draft| for childs):
%    \begin{macrocode}
\ifchilddoc
\providecommand{\version}{draft}
\else
\providecommand{\version}{final}
\fi
%    \end{macrocode}

% Load the standard document class:
%    \begin{macrocode}
\documentclass[12pt]{article}
%    \end{macrocode}

% Start the document body:
%    \begin{macrocode}
\begin{document}
%    \end{macrocode}

% Declare a title page.
% Print title, part of document being processed and version flag:
%    \begin{macrocode}
\addtocounter{page}{-1}
\begin{center}
{\LARGE\bfseries{}childdoc example\par}
\vspace{1cm}
\ifchilddoc
\ifchilddocmanual part\else chapter\fi:
`\childdocname' of `\childdocjob'\par
\else
main document: `\childdocjob'\par
\fi
version: \version\par
\end{center}
\newpage
%    \end{macrocode}

% Manually include selected file,
% otherwise process as usual:
%    \begin{macrocode}
\ifchilddocmanual
\section*{part `\childdocname'}
\input{\childdocname}
\else
%    \end{macrocode}

% Include the two chapters:
%    \begin{macrocode}
\include{cdocsch1}
\include{cdocsch2}
%    \end{macrocode}

% Include the two parts unless only chapters should be displayed:
%    \begin{macrocode}
\ifchilddoc\else
\section{part three}
\input{cdocspt3}
\section{part four}
\input{cdocspt4}
\fi
%    \end{macrocode}

% Process as usual until here:
%    \begin{macrocode}
\fi
%    \end{macrocode}

% End of document body:
%    \begin{macrocode}
\end{document}
%    \end{macrocode}
%\iffalse
%</samplemain>
%\fi
%
% %%%%%%%%%%%%%%%%%%%%%%%%%%%%%%%%%%%%%%
% \paragraph{Chapter Include Files.}
%
% The include files are called |cdocsch1.tex| and |cdocsch2.tex|.
%
%\iffalse
%<*samplechap1|samplechap2>
%\fi

% Optional override for |\version| flag:
%    \begin{macrocode}
%%\providecommand{\version}{final}
%    \end{macrocode}

% Include the main document:
%    \begin{macrocode}
\input{childdoc.def}
\childdocof{cdocsamp}
%    \end{macrocode}

%\iffalse
%</samplechap1|samplechap2>
%\fi
%
%\iffalse
%<*samplechap1>
%\fi
% Some text for chapter 1:
%    \begin{macrocode}
\section{one}
some text in chapter one
%    \end{macrocode}

%\iffalse
%</samplechap1>
%\fi
% Some text for chapter 2:
%\iffalse
%<*samplechap2>
%\fi
%    \begin{macrocode}
\section{two}
more text in chapter two
%    \end{macrocode}

%\iffalse
%</samplechap2>
%\fi
%
% %%%%%%%%%%%%%%%%%%%%%%%%%%%%%%%%%%%%%%
% \paragraph{Part Include Files.}
%
% The include files are called |cdocspt3.tex| and |cdocspt4.tex|.
%
%\iffalse
%<*samplepart3|samplepart4>
%\fi

% Optional override for |\version| flag:
%    \begin{macrocode}
%%\providecommand{\version}{final}
%    \end{macrocode}

% Include the main document:
%    \begin{macrocode}
\input{childdoc.def}
\childdocby{cdocsamp}
%    \end{macrocode}

%\iffalse
%</samplepart3|samplepart4>
%\fi
%
%\iffalse
%<*samplepart3>
%\fi
% Some text for part 3:
%    \begin{macrocode}
some text in part three
%    \end{macrocode}

%\iffalse
%</samplepart3>
%\fi
% Some text for part 4:
%\iffalse
%<*samplepart4>
%\fi
%    \begin{macrocode}
more text in part four
%    \end{macrocode}

%\iffalse
%</samplepart4>
%\fi
%
% %%%%%%%%%%%%%%%%%%%%%%%%%%%%%%%%%%%%%%
% \paragraph{Forwarding for a Complete Draft.}
%
% The following forwarding file |cdocsdrf.tex|
% compiles the main document in draft mode:
%\iffalse
%<*sampledraft>
%\fi
%    \begin{macrocode}
\def\version{draft}
\input{childdoc.def}
\childdocforward{cdocsamp}
%    \end{macrocode}

%\iffalse
%</sampledraft>
%\fi
%
% %%%%%%%%%%%%%%%%%%%%%%%%%%%%%%%%%%%%%%
% \paragraph{Forwarding for Final Version of the Chapters.}
%
% The following forwarding files |cdocsfn1.tex| and |cdocsfn2.tex|
% (with identical content)
% compile the final versions of the child documents
% |cdocsch1.tex| and |cdocsch2.tex|, respectively:
%\iffalse
%<*samplefinal>
%\fi
%    \begin{macrocode}
\def\version{final}
\input{childdoc.def}
\childdocforwardprefix[cdocsamp]{cdocsfn}{cdocsch}
%    \end{macrocode}

%\iffalse
%</samplefinal>
%\fi
%
% %%%%%%%%%%%%%%%%%%%%%%%%%%%%%%%%%%%%%%
% \paragraph{Command Line Processing.}
%
% The following three command lines generate the output files
% |cdocscld|, |cdocscl1| and |cdocscl2|
% which should be identical to
% |cdocsdrf|, |cdocsch1| and |cdocsfn2|, respectively:
% \begin{center}
% \begin{tabular}{l}
% |latex -jobname cdocscld \|\\
% |  "\def\version{draft}\input{childdoc.def}\childdocforward{cdocsamp}"|\\
% |latex -jobname cdocscl1 \|\\
% |  "\input{childdoc.def}\childdocforward[cdocsamp]{cdocsch1}"|\\
% |latex -jobname cdocscl2 \|\\
% |  "\def\version{final}\input{childdoc.def}\childdocforward{cdocsch2}"|
% \end{tabular}
% \end{center}
% Note that the trailing backslash on each first line
% merely continues the input to the second line
% (for convenient cut ant paste).
% Furthermore, the command |latex| can be replaced by any
% of its alternative versions such as |pdflatex|.
%
% %%%%%%%%%%%%%%%%%%%%%%%%%%%%%%%%%%%%%%%%%%%%%%%%%%%%%%%%%%%%%%%%%%%%%%%%%%%%%%
% %%%%%%%%%%%%%%%%%%%%%%%%%%%%%%%%%%%%%%%%%%%%%%%%%%%%%%%%%%%%%%%%%%%%%%%%%%%%%%
% \section{Implementation}
%\iffalse
%<*package>
%\fi
%
% This section describes the definitions file |childdoc.def|.

% The definitions cannot be loaded using |\usepackage| or |\RequirePackage|
% which has a mechanism to prevent loading a style file more than once.
% When loading the definitions by means of |\input|
% multiple instances have to be prevented manually:
%\iffalse
%This code needs to be before the `\ProvidesFile' directive
%which is defined at the beginning of this file.
%Therefore it is also placed there and commented out here.
%</package>
%<*discard>
%\fi
%    \begin{macrocode}
\ifdefined\childdocmain\endinput\fi
%    \end{macrocode}
%\iffalse
%</discard>
%<*package>
%\fi
%
% \macro{\ifchilddoc}
% \macro{\ifchilddocmanual}
% The conditional |\ifchilddoc| tells whether a
% child (true) or main (false) document is being compiled.
% The conditional |\ifchilddocmanual| tells whether
% the |\includeonly| mechanism is used (false) or
% the selection of child files must be performed manually (true).
% The definitions initialise to false:
%    \begin{macrocode}
\newif\ifchilddoc
\newif\ifchilddocmanual
%    \end{macrocode}

% \macro{\childdocname}
% \macro{\childdocjob}
% The macro |\childdocname| stores the name of the main document
% to be compiled. The macro |\childdocjob| stores the name of
% the document on which the \LaTeX{} compiler was originally invoked.
% The content of |\jobname| cannot be compared
% to filenames specified in the source due to different catcodes.
% The following code rescans |\jobname|, stores the result
% in |\childdocname| and saves a copy in |\childdocjob|:
%    \begin{macrocode}
\edef\childdocname{\scantokens\expandafter{\jobname\noexpand}}
\let\childdocjob\childdocname
%    \end{macrocode}

% \macro{\childdocdisable}
% The macro |\childdocdisable| prevents the main file
% from being processed more than once.
% At this stage, the main document command |\childdocmain|
% is assumed to be called once again where it should do nothing.
% Any subsequent call to it should prevent
% a secondary processing of the main document
% It overwrites the forwarding commands
% |\childdocof| and |\childdocforward|
% with empty macros to prevent further inclusions of the main document:
%    \begin{macrocode}
\newcommand{\childdocdisable}
{
  \renewcommand{\childdocmain}[1]{\renewcommand{\childdocmain}[1]{\endinput}}
  \renewcommand{\childdocof}[1]{}
  \renewcommand{\childdocby}[2][]{}
  \renewcommand{\childdocforward}[2][]{}
  \renewcommand{\childdocdisable}{}
}
%    \end{macrocode}

% \macro{\childdocmain}
% The macro |\childdocmain| is to be called at the top of the main file
% with nothing or the main filename (without extension) as argument.
% First, it breaks loops.
% If the argument is not empty and does not match |\childdocname|
% (which is set by the first inclusion of |childdoc.def|),
% |\ifchilddoc| is set to true, |\includeonly| is applied to the child file
% and |\jobname| is set to the main file
% (for proper handling of |.aux| files):
%    \begin{macrocode}
\newcommand{\childdocmain}[1]
{
  \childdocdisable\childdocmain{}
  \if?#1?\else
    \begingroup
      \def\childdoctmp{#1}
      \ifx\childdoctmp\childdocname
        \def\childdoctmp{}
      \else
        \def\childdoctmp
        {
          \childdoctrue
          \includeonly{\childdocname}
          \def\childdocjob{#1}
          \def\jobname{#1}
        }
      \fi
      \expandafter
    \endgroup
    \childdoctmp
  \fi
}
%    \end{macrocode}

% \macro{\childdocof}
% The command |\childdocof| redirects
% compilation to the main file |#1|.
%    \begin{macrocode}
\newcommand{\childdocof}[1]
{
  \childdocdisable
  \childdoctrue
  \includeonly{\childdocname}
  \def\jobname{#1}
  \def\childdocjob{#1}
  \input{#1}
}
%    \end{macrocode}

% \macro{\childdocby}
% The command |\childdocby| ....
%    \begin{macrocode}
\newcommand{\childdocby}[2][]
{
  \childdocdisable
  \childdoctrue
  \childdocmanualtrue
  \if?#1?\else
    \def\jobname{#2}
  \fi
  \def\childdocjob{#2}
  \input{#2}
  \endinput
}
%    \end{macrocode}

% \macro{\childdocforward}
% The command |\childdocforward| redirects
% compilation to the main file or
% (if the optional argument is given) a child file.
% Parameters are set as if the main file
% or a child file starting with |\childdocof| was compiled.
% Then compilation is handed over to the main file:
%    \begin{macrocode}
\newcommand{\childdocforward}[2][]
{
  \begingroup
    \if?#1?
      \def\childdoctmp
      {
        \def\childdocname{#2}
        \def\childdocjob{#2}
        \def\jobname{#2}
        \input{#2}
        \endinput
      }
    \else
      \def\childdoctmp
      {
        \childdocdisable
        \def\childdocname{#2}
        \childdoctrue
        \includeonly{#2}
        \def\childdocjob{#1}
        \def\jobname{#1}
        \input{#1}
        \endinput
      }
    \fi
    \expandafter
  \endgroup
  \childdoctmp
}
%    \end{macrocode}

% \macro{\childdocforwardprefix}
% The command |\childdocforwardprefix| redirects
% compilation to the main or a child file by means of a pattern.
% The prefix |#1| in the current filename is replaced by |#2|
% and the suffix of the current filename is kept
% (it is assumed that the filename does not contain the substring `|~~~|'
% which is used as a delimiter).
% Compilation is handed over to the new file by |\childdocforward|:
%    \begin{macrocode}
\newcommand{\childdocforwardprefix}[3][]
{
  \begingroup
    \def\childdocextract #2##1~~~{\def\childdoctmp{\childdocforward[#1]{#3##1}}}
    \expandafter\childdocextract\childdocname~~~
    \expandafter
  \endgroup
  \childdoctmp
}
%    \end{macrocode}

% \macro{\childdoc}
% The deprecated macro |\childdoc| is a legacy version of |\childdocmain|:
%    \begin{macrocode}
\newcommand{\childdoc}{\childdocmain}
%    \end{macrocode}

% \macro{\childdocredirect}
% The deprecated macro |\childdocredirect| is a legacy version
% of |\childdocforward| and |\childdocforwardprefix|:
%    \begin{macrocode}
\newcommand{\childdocredirect}[2][]
{
  \begingroup
    \if?#1?
      \def\childdoctmp{\childdocforward{#2}}
    \else
      \def\childdoctmp{\childdocforwardprefix{#1}{#2}}
    \fi
    \expandafter
  \endgroup
  \childdoctmp
}
%    \end{macrocode}

%\iffalse
%</package>
%\fi
%
\endinput
|
and perform the replacements as outlined below.
Instead of |\childdocmain{|\textit{main}|}| add the following code
to the top of the main file:
%
\begin{center}
\begin{tabular}{l}
|\||ifdefined\childdocname\endinput\||fi\newif\ifchilddoc|\\
|\edef\childdocname{\scantokens\expandafter{\jobname\noexpand}}|\\
|\def\childdocmain{|\textit{main}|}\||ifx\childdocmain\childdocname\||else|\\
|\childdoctrue\includeonly{\childdocname}\let\jobname\childdocmain\||fi|\\
\end{tabular}
\end{center}
%
Instead of |\childdocof{|\textit{main}|}| just include the main file
at the top of each child file:
%
\begin{center}
|\input{|\textit{main}|}|
\end{center}
%
A simple redirection |\childdocforward{|\textit{dest}|}| is achieved by:
%
\begin{center}
|\def\jobname{|\textit{dest}|}\input{\jobname}|
\end{center}
%
The redirection with prefix
|\childdocforwardprefix[|\textit{prefix}|]{|\textit{dest}|}|
is accomplished by:
%
\begin{center}
\begin{tabular}{l}
|{\edef\jobname{\scantokens\expandafter{\jobname\noexpand}}|\\
|\def\redirectjob |\textit{prefix}|#1~~~{\gdef\jobname{|\textit{dest}|#1}}|\\
|\expandafter\redirectjob\jobname~~~}\input{\jobname}|
\end{tabular}
\end{center}

In an alternative approach,
child documents can be compiled by a specific command line
without additional code or specific definitions:
%
\begin{center}
|... -jobname "|\textit{target}|" "|[\textit{flags}]%
|\includeonly{|\textit{dest}|}\input{|\textit{main}|}"|
\end{center}
%

%%%%%%%%%%%%%%%%%%%%%%%%%%%%%%%%%%%%%%%%%%%%%%%%%%%%%%%%%%%%%%%%%%%%%%%%%%%%%%%%
%%%%%%%%%%%%%%%%%%%%%%%%%%%%%%%%%%%%%%%%%%%%%%%%%%%%%%%%%%%%%%%%%%%%%%%%%%%%%%%%
\section{Information}

%%%%%%%%%%%%%%%%%%%%%%%%%%%%%%%%%%%%%%%%%%%%%%%%%%%%%%%%%%%%%%%%%%%%%%%%%%%%%%%%
\subsection{Copyright}

Copyright \copyright{} 2017--2018 Niklas Beisert

This work may be distributed and/or modified under the
conditions of the \LaTeX{} Project Public License, either version 1.3
of this license or (at your option) any later version.
The latest version of this license is in
  \url{http://www.latex-project.org/lppl.txt}
and version 1.3 or later is part of all distributions of \LaTeX{}
version 2005/12/01 or later.

This work has the LPPL maintenance status `maintained'.

The Current Maintainer of this work is Niklas Beisert.

This work consists of the files |README.txt|, |childdoc.ins| and |childdoc.dtx|
as well as the derived files |childdoc.def|, |cdocsamp.tex|
with |cdocsch1.tex|, |cdocsch2.tex|, |cdocspt3.tex|, |cdocspt4.tex|,
|cdocsdrf.tex|, |cdocsfn1.tex|, |cdocsfn2.tex|
as well as |childdoc.pdf|.

%%%%%%%%%%%%%%%%%%%%%%%%%%%%%%%%%%%%%%%%%%%%%%%%%%%%%%%%%%%%%%%%%%%%%%%%%%%%%%%%
\subsection{Files and Installation}

The package consists of the files:
%
\begin{center}
\begin{tabular}{ll}
    |README.txt|   & readme file \\
    |childdoc.ins| & installation file \\
    |childdoc.dtx| & source file \\
    |childdoc.def| & definition file \\
    |cdocsamp.tex| & sample main file \\
    |cdocsch1.tex| & sample include file \\
    |cdocsch2.tex| & sample include file \\
    |cdocspt3.tex| & sample part file \\
    |cdocspt4.tex| & sample part file \\
    |cdocsdrf.tex| & sample redirection file \\
    |cdocsfn1.tex| & sample redirection file \\
    |cdocsfn2.tex| & sample redirection file \\
    |childdoc.pdf| & manual
\end{tabular}
\end{center}
%
The distribution consists of the files
|README.txt|, |childdoc.ins| and |childdoc.dtx|.
%
\begin{itemize}
\item
Run (pdf)\LaTeX{} on |childdoc.dtx|
to compile the manual |childdoc.pdf| (this file).
\item
Run \LaTeX{} on |childdoc.ins| to create the definitions file |childdoc.def|
and the sample |cdocsamp.tex| with include files
|cdocsch1.tex|, |cdocsch2.tex|, |cdocspt3.tex|, |cdocspt4.tex|,
|cdocsdrf.tex|, |cdocsfn1.tex|, |cdocsfn2.tex|.
Then copy the file |childdoc.def| to an appropriate directory of your \LaTeX{}
distribution, e.g.\ \textit{texmf-root}|/tex/latex/childdoc|.
\end{itemize}

%%%%%%%%%%%%%%%%%%%%%%%%%%%%%%%%%%%%%%%%%%%%%%%%%%%%%%%%%%%%%%%%%%%%%%%%%%%%%%%%
\subsection{Related CTAN Packages}

There are several other packages which offer a similar functionality:
%
\begin{itemize}
\item
The packages
\href{http://ctan.org/pkg/docmute}{\textsf{docmute}},
\href{http://ctan.org/pkg/includex}{\textsf{includex}} and
\href{http://ctan.org/pkg/standalone}{\textsf{standalone}}
provide commands to include only the document body of
a child file thus allowing both files to be compiled individually.
\item
The packages \href{http://ctan.org/pkg/subdocs}{\textsf{subdocs}}
and \href{http://ctan.org/pkg/subfiles}{\textsf{subfiles}}
provide structures in which the main and child documents can be
encapsulated and allowing them to be compiled individually.
The inclusion mechanism is different from the conventional |\include|.
\item
The package \href{http://ctan.org/pkg/combine}{\textsf{combine}}
is an elaborate solution to combine several documents into one.
\end{itemize}
%
See also the CTAN topic \href{http://ctan.org/topic/subdocs}{\textsf{subdocs}}
for further related packages.
The present package differs from the above solutions in that
a document structure constructed with the conventional |\include| mechanism
just needs two extra commands at the top of every file
such that all constituent files can be compiled individually.

%%%%%%%%%%%%%%%%%%%%%%%%%%%%%%%%%%%%%%%%%%%%%%%%%%%%%%%%%%%%%%%%%%%%%%%%%%%%%%%%
%\subsection{Feature Suggestions}
%
%The following is a list of features which may be useful for future
%versions of this package:
%%
%\begin{itemize}
%\item
%\ldots
%\end{itemize}

%%%%%%%%%%%%%%%%%%%%%%%%%%%%%%%%%%%%%%%%%%%%%%%%%%%%%%%%%%%%%%%%%%%%%%%%%%%%%%%%
\subsection{Revision History}

%%%%%%%%%%%%%%%%%%%%%%%%%%%%%%%%%%%%%%%%
\paragraph{v2.0:} 2018/12/30

\begin{itemize}
\item
immediate forward processing
\item
added |\childdocby| mechanism
\item
manual restructured
\end{itemize}

%%%%%%%%%%%%%%%%%%%%%%%%%%%%%%%%%%%%%%%%
\paragraph{v1.6:} 2018/01/17

\begin{itemize}
\item
application for development of include files
\item
corrections to manual
\end{itemize}

%%%%%%%%%%%%%%%%%%%%%%%%%%%%%%%%%%%%%%%%
\paragraph{v1.5:} 2017/05/21

\begin{itemize}
\item
more complete structuring introduced
\item
|\childdocof| introduced
\item
|\childdoc| renamed to |\childdocmain|
\item
|\childredirect| renamed to |\childdocforward| and |\childdocforwardprefix|
and functionality expanded
\end{itemize}

%%%%%%%%%%%%%%%%%%%%%%%%%%%%%%%%%%%%%%%%
\paragraph{v1.0:} 2017/04/27

\begin{itemize}
\item
manual and install package
\item
first version published on CTAN
\end{itemize}

%%%%%%%%%%%%%%%%%%%%%%%%%%%%%%%%%%%%%%%%
\paragraph{v0.6:} 2017/04/26

\begin{itemize}
\item
redirection mechanism added
\end{itemize}

%%%%%%%%%%%%%%%%%%%%%%%%%%%%%%%%%%%%%%%%
\paragraph{v0.5:} 2017/04/26

\begin{itemize}
\item
functionality in definition file
\end{itemize}


%%%%%%%%%%%%%%%%%%%%%%%%%%%%%%%%%%%%%%%%%%%%%%%%%%%%%%%%%%%%%%%%%%%%%%%%%%%%%%%%
%%%%%%%%%%%%%%%%%%%%%%%%%%%%%%%%%%%%%%%%%%%%%%%%%%%%%%%%%%%%%%%%%%%%%%%%%%%%%%%%
%%%%%%%%%%%%%%%%%%%%%%%%%%%%%%%%%%%%%%%%%%%%%%%%%%%%%%%%%%%%%%%%%%%%%%%%%%%%%%%%
\appendix

\settowidth\MacroIndent{\rmfamily\scriptsize 000\ }

 \DocInput{childdoc.dtx}

\end{document}
%</driver>
% \fi
%
% %%%%%%%%%%%%%%%%%%%%%%%%%%%%%%%%%%%%%%%%%%%%%%%%%%%%%%%%%%%%%%%%%%%%%%%%%%%%%%
% %%%%%%%%%%%%%%%%%%%%%%%%%%%%%%%%%%%%%%%%%%%%%%%%%%%%%%%%%%%%%%%%%%%%%%%%%%%%%%
% \section{Sample}
%\iffalse
%<*samplemain>
%\fi
%
% The following presents a sample document
% with two chapters, two parts, a title page,
% a compile flag as well as three forwarding files to set the flag.
% It consists of eight |.tex| files:
% \begin{center}
% \begin{tabular}{ll}
% |cdocsamp.tex|&main file\\
% |cdocsch1.tex|&include file for chapter 1\\
% |cdocsch2.tex|&include file for chapter 2\\
% |cdocspt3.tex|&include file for part 3\\
% |cdocspt4.tex|&include file for part 4\\
% |cdocsdrf.tex|&forwarding file for main file in draft mode\\
% |cdocsfi1.tex|&forwarding file for final version of chapter 1\\
% |cdocsfi2.tex|&forwarding file for final version of chapter 2\\
% \end{tabular}
% \end{center}
% Each of the eight files can be compiled directly by the \LaTeX{} compiler.
%
% %%%%%%%%%%%%%%%%%%%%%%%%%%%%%%%%%%%%%%
% \paragraph{Main File.}
%
% The main file is called |cdocsamp.tex|.
%
% Load the \textsf{childdoc} definitions and
% declare the filename for the main document:
%    \begin{macrocode}
% \iffalse
%
% childdoc.dtx Copyright (C) 2017-2018 Niklas Beisert
%
% This work may be distributed and/or modified under the
% conditions of the LaTeX Project Public License, either version 1.3
% of this license or (at your option) any later version.
% The latest version of this license is in
%   http://www.latex-project.org/lppl.txt
% and version 1.3 or later is part of all distributions of LaTeX
% version 2005/12/01 or later.
%
% This work has the LPPL maintenance status `maintained'.
%
% The Current Maintainer of this work is Niklas Beisert.
%
% This work consists of the files childdoc.dtx and childdoc.ins
% and the derived files childdoc.def and cdocsamp.tex with
% cdocsch1.tex, cdocsch2.tex, cdocsdrf.tex, cdocsfn1.tex, cdocsfn2.tex.
%
%<package>\ifdefined\childdocmain\endinput\fi
%<package>\ProvidesFile{childdoc.def}[2018/12/30 v2.0 child document driver]
%<samplemain>\ProvidesFile{cdocsamp.tex}[2018/12/30 v2.0 sample for childdoc]
%<*driver>
%\ProvidesFile{childdoc.drv}[2018/12/30 v2.0 childdoc reference manual file]
\PassOptionsToClass{10pt,a4paper}{article}
\documentclass{ltxdoc}

\usepackage[margin=35mm]{geometry}
\usepackage{hyperref}
\usepackage{hyperxmp}
\usepackage[usenames]{color}

\hypersetup{colorlinks=true}
\hypersetup{pdfstartview=FitH}
\hypersetup{pdfpagemode=UseNone}
\hypersetup{pdfsource={}}
\hypersetup{pdflang={en-UK}}
\hypersetup{pdfcopyright={Copyright 2017-2018 Niklas Beisert.
  This work may be distributed and/or modified under the
  conditions of the LaTeX Project Public License, either version 1.3
  of this license or (at your option) any later version.}}
\hypersetup{pdflicenseurl={http://www.latex-project.org/lppl.txt}}
\hypersetup{pdfcontactaddress={ETH Zurich, ITP, HIT K,
  Wolfgang-Pauli-Strasse 27}}
\hypersetup{pdfcontactpostcode={8093}}
\hypersetup{pdfcontactcity={Zurich}}
\hypersetup{pdfcontactcountry={Switzerland}}
\hypersetup{pdfcontactemail={nbeisert@itp.phys.ethz.ch}}
\hypersetup{pdfcontacturl={http://people.phys.ethz.ch/\xmptilde nbeisert/}}

\newcommand{\secref}[1]{\hyperref[#1]{section \ref*{#1}}}

\parskip1ex
\parindent0pt
\let\olditemize\itemize
\def\itemize{\olditemize\parskip0pt}

\begin{document}

\title{The \textsf{childdoc} Package}
\hypersetup{pdftitle={The childdoc Package}}
\author{Niklas Beisert\\[2ex]
  Institut f\"ur Theoretische Physik\\
  Eidgen\"ossische Technische Hochschule Z\"urich\\
  Wolfgang-Pauli-Strasse 27, 8093 Z\"urich, Switzerland\\[1ex]
  \href{mailto:nbeisert@itp.phys.ethz.ch}
  {\texttt{nbeisert@itp.phys.ethz.ch}}}
\hypersetup{pdfauthor={Niklas Beisert}}
\hypersetup{pdfsubject={Manual for the LaTeX2e Package childdoc}}
\date{30 December 2018, \textsf{v2.0}}
\maketitle

\begin{abstract}\noindent
\textsf{childdoc} is a \LaTeXe{} package
that enables the direct compilation
of document sections included by |\include|
to individual files.
\end{abstract}

\begingroup
\parskip0ex
\tableofcontents
\endgroup

%%%%%%%%%%%%%%%%%%%%%%%%%%%%%%%%%%%%%%%%%%%%%%%%%%%%%%%%%%%%%%%%%%%%%%%%%%%%%%%%
%%%%%%%%%%%%%%%%%%%%%%%%%%%%%%%%%%%%%%%%%%%%%%%%%%%%%%%%%%%%%%%%%%%%%%%%%%%%%%%%
\section{Introduction}

\LaTeX{} provides a mechanism to structure a large document (such as a book)
into a main file and several child files (containing the chapters)
using the |\include| command.
This mechanism is beneficial for documents
which span hundreds of pages in order to
make the source file(s) more manageable.
Moreover, compilation can be restricted to
selected child files by means of the |\includeonly| command.
The latter feature can be used to reduce the compilation time while editing
(this was significantly more useful in the earlier days of \LaTeX{})
or to generate a smaller document which is easier to navigate.
Another application of |\includeonly| is to generate
documents consisting of selected parts of the complete document.

However, there are a few drawbacks of the plain |\include| mechanism:
\begin{itemize}
\item
The child files cannot be compiled on their own,
they can only be compiled via the main file.
A naive editing environment
(such as a text editor with an option
to have the current file processed by \LaTeX)
may require one to switch to the main file before compiling;
attempting to compile the child file produces errors.
\item
The main file must be modified (each time)
to adjust the |\includeonly| command
to the present needs. This easily leaves the main file in a messy state.
\item
The generated document will always carry the filename
of the main document. This is inconvenient if
several child files are to be compiled and
to be kept for distribution.
\end{itemize}

The present package provides a simple interface
to make child files individually compilable by \LaTeX{}.
Compiling a child file then has the same effect as compiling
the main file with an |\includeonly| command
to select the appropriate child.
Moreover the generated document will carry the name of the child
rather than the main file.
This resolves all three above issues.

This feature is meant to make the editing of books,
thesis documents and lecture notes somewhat more convenient.
However, the package can also be used efficiently for
composing a series of documents (such as exercise sheets)
which are typically distributed individually.
It then assists the author in generating the individual documents
(potentially in different versions)
as well as a document containing the collected series.
Another application is in developing style files
or other kinds of included material
where compilation of the style file could redirect
to a sample or test file.

%%%%%%%%%%%%%%%%%%%%%%%%%%%%%%%%%%%%%%%%%%%%%%%%%%%%%%%%%%%%%%%%%%%%%%%%%%%%%%%%
%%%%%%%%%%%%%%%%%%%%%%%%%%%%%%%%%%%%%%%%%%%%%%%%%%%%%%%%%%%%%%%%%%%%%%%%%%%%%%%%
\section{Usage}

First of all, the package \textsf{childdoc} is \emph{not} a standard
\LaTeXe{} |.sty| style file! Therefore it needs to be invoked in
a non-standard way.

%%%%%%%%%%%%%%%%%%%%%%%%%%%%%%%%%%%%%%%%%%%%%%%%%%%%%%%%%%%%%%%%%%%%%%%%%%%%%%%%
\subsection{Included Files}
\label{sec:include}

%%%%%%%%%%%%%%%%%%%%%%%%%%%%%%%%%%%%%%%%
\DescribeMacro{\childdocmain}
To use the package, add the commands
\begin{center}
\begin{tabular}{l}
|\input{childdoc.def}|\\
|\childdocmain{}|\\
\end{tabular}
\end{center}
at the very top of the main \LaTeX{} file,
in particular \emph{before} the |\documentclass| statement!
The argument of |\childdocmain| should be left empty
(but it must be present).

%%%%%%%%%%%%%%%%%%%%%%%%%%%%%%%%%%%%%%%%
\DescribeMacro{\childdocof}
Furthermore, add the commands
\begin{center}
\begin{tabular}{l}
|\input{childdoc.def}|\\
|\childdocof{|\textit{main}|}|\\
\end{tabular}
\end{center}
at the top of every child file \textit{child}
which is included by |\include{|\textit{child}|}|
from within the main file
(or at least for those files to be compiled individually).
The argument \textit{main} must be the filename of the main file.

There are a couple of
considerations in setting up the main and child documents:

%%%%%%%%%%%%%%%%%%%%%%%%%%%%%%%%%%%%%%%%
\paragraph{Restrictions.}

Please note the following restrictions:
\begin{itemize}
\item
|\childdocmain| must be called with one argument \textit{main}
to ensure compatibility with earlier version of the package.
It must either be empty (|\childdocmain{}|)
or precisely match the filename of the main file in which it is specified.
See \secref{sec:detection} for further information.
\item
The filename \textit{main} must be specified without the |.tex| extension.
\item
The filename \textit{main} is case sensitive
(even in case-insensitive file systems)
due to internal string comparison.
\item
The argument \textit{main} should be fully expanded, it cannot be a macro.
\item
Subdirectories and special characters should be avoided in filenames.
\item
The command |\childdocmain{|\textit{main}|}| must be followed by a whitespace.
It should not be followed immediately by another command
or by a comment mark `|%|'.
This is because the \TeX{} parser reads the token immediately following
the argument of |\childdocmain| and puts it
at the beginning of every child section;
however, a white\-space is ignored.
\end{itemize}

%%%%%%%%%%%%%%%%%%%%%%%%%%%%%%%%%%%%%%%%
\paragraph{Content of Main File.}

It is advisable to place all content in the child files included by |\include|.
Any output contained in the main file will appear in all child documents
unless suppressed manually;
it cannot be suppressed automatically by the |\includeonly| directive
and thus should normally be avoided.
A method to include some content in the main file
by means of conditional processing is described in \secref{sec:conditional}.

%%%%%%%%%%%%%%%%%%%%%%%%%%%%%%%%%%%%%%%%
\paragraph{Page Numbering.}

When only a part of the document is compiled,
the appropriate numbering of pages
(as well as other status parameters)
is determined from the |.aux| files.
The latter contain information from previous passes.
However this information needs to propagate through
all intermediate child documents.
Therefore the page numbering in child documents may well
be inconsistent until the complete document is compiled at least once.

A useful (if unconventional) way to always ensure a consistent
page numbering is to restart the numbering in each child document
and denote the pages by `\textit{child}|.|\textit{page}'
where \textit{child} represents the chapter/section number of the child file.
This can be achieved by the command
|\numberwithin{page}{|\textit{child}|}|
of the \textsf{amsmath} package
where \textit{child} can be |chapter| or |section|
depending on the chosen structuring.
Alternatively, one can modify the macro |\thepage| appropriately
and reset the counter |page| at the start of each child file.

%%%%%%%%%%%%%%%%%%%%%%%%%%%%%%%%%%%%%%%%%%%%%%%%%%%%%%%%%%%%%%%%%%%%%%%%%%%%%%%%
\subsection{Conditional Processing}
\label{sec:conditional}

The package provides a mechanism to compile different versions
of a document. To customise the versions further some conditional processing
can come in handy to distinguish which version is being compiled.
The package provides two macros to describe the compilation context:

%%%%%%%%%%%%%%%%%%%%%%%%%%%%%%%%%%%%%%%%
\DescribeMacro{\ifchilddoc}
The conditional |\ifchilddoc| distinguishes between the compilation of
child documents and the main document:
%
\begin{center}
|\ifchilddoc |\textit{child-code}| |[|\||else |\textit{main-code}]| \||fi|
\end{center}

%%%%%%%%%%%%%%%%%%%%%%%%%%%%%%%%%%%%%%%%
\DescribeMacro{\childdocname}
\DescribeMacro{\childdocjob}
The macro |\childdocname| contains the filename (without extension)
of the main or child file being processed.
Note that |\childdocjob| will always contain the name of the main file.

%%%%%%%%%%%%%%%%%%%%%%%%%%%%%%%%%%%%%%%%
\paragraph{Title Page.}

Conditional processing can be used to include a title or banner page
in the main document when proper precautions are taken.
Importantly, the code in the main file should ensure that the page counter
(as well as other status parameters which are stored in the |.aux| files)
takes the same value after the conditional processing.
Otherwise the page numbers may take divergent values
depending on which part is compiled.

For example, a title page could be declared by:
%
\begin{center}
\begin{tabular}{l}
|\ifchilddoc\||else|\\
|\addtocounter{page}{-1}|\\
\textit{code for title page}\\
|\newpage|\\
|\||fi|
\end{tabular}
\end{center}
%
A banner page for the child documents can be generated by:
%
\begin{center}
\begin{tabular}{l}
|\ifchilddoc|\\
|\addtocounter{page}{-1}|\\
\textit{code for banner page}\\
|\newpage|\\
|\||fi|
\end{tabular}
\end{center}
%
Here one could write a message such as:
\begin{center}
|This is the part \childdocname{} of \childdocjob{}.|
\end{center}

%%%%%%%%%%%%%%%%%%%%%%%%%%%%%%%%%%%%%%%%%%%%%%%%%%%%%%%%%%%%%%%%%%%%%%%%%%%%%%%%
\subsection{Flags}
\label{sec:flags}

The package makes it easy to generate different versions
of the main or child documents.
To this end compilation flags can be defined
and assigned different default values.
They will be particularly useful in conjunction
with the forwarding mechanism described in \secref{sec:forward}.

For example, it may be useful to have a flag |\version|
which can be set to |draft| or |final|.
The document source will contain some conditional code
depending on the value of |\version|.
Suppose further, the flag should default to |final| for the main file
and to |draft| for child files
which is a natural assignment for editing the document.
This is achieved by placing the following code
in the preamble of the main document
(below the |\childdocmain| directive):
%
\begin{center}
\begin{tabular}{l}
|\ifchilddoc|\\
|\providecommand{\version}{draft}|\\
|\||else|\\
|\providecommand{\version}{final}|\\
|\||fi|
\end{tabular}
\end{center}
%
The definition by |\providecommand| makes sure
that previous definitions are not overwritten.
Further statements |\providecommand{\version}{...}|
can thus be added before the above code to override it.

For the main file, one might add a line
(between |\childdocmain| and the above block)
%
\begin{center}
|%\ifchilddoc\||else\providecommand{\version}{draft}\||fi|
\end{center}
%
which can be uncommented to produce a draft version.
Likewise one can add a line to the very top of a child file
(above the |\childdocof{|\textit{main}|}| directive)
%
\begin{center}
|%\providecommand{\version}{final}|
\end{center}
%
which can be uncommented to produce the final version of this child document.

%%%%%%%%%%%%%%%%%%%%%%%%%%%%%%%%%%%%%%%%%%%%%%%%%%%%%%%%%%%%%%%%%%%%%%%%%%%%%%%%
\subsection{Forwarding}
\label{sec:forward}

Different versions of the main or child documents
using compilation flags as described in \secref{sec:flags}
can be (permanently) stored in different files
for convenient compilation, viewing and distribution.
To this end, the package defines a command
to pass on compilation to a different file:

%%%%%%%%%%%%%%%%%%%%%%%%%%%%%%%%%%%%%%%%
\DescribeMacro{\childdocforward}
The command |\childdocforward| redirects processing to
another source file:
%
\begin{center}
\begin{tabular}{l}
|\input{childdoc.def}|\\
|\childdocforward[|\textit{main}|]{|\textit{dest}|}|\\
\end{tabular}
\end{center}
%
The argument \textit{dest} is the destination file
(without extension).
It should be the main file or one of the child files.
Note that further \textsf{childdoc} directives
such as |\childdocof| and |\childdocforward|
in the indicated file will be processed in this form.
The optional argument \textit{main}
passes on directly to the main file \textit{main}
while pretending to compile the child \textit{dest}.
This form behaves as if \textit{dest}
issues |\childdocof{|\textit{main}|}| right away,
and no further \textsf{childdoc} directives will be processed.

%%%%%%%%%%%%%%%%%%%%%%%%%%%%%%%%%%%%%%%%
\DescribeMacro{\...prefix}
In the alternative form |\childdocforwardprefix|,
%
\begin{center}
\begin{tabular}{l}
|\input{childdoc.def}|\\
|\childdocforwardprefix[|\textit{main}|]{|\textit{prefix}|}{|\textit{dest}|}|
\end{tabular}
\end{center}
%
the destination file is determined by a pattern
depending on the current file:
To make this work, the current file must be called
`{\textit{prefix}\hspace{0.2em}\textit{suffix}}'
with \textit{prefix} matching precisely the argument.
Processing is then passed on to the file
`{\textit{dest}\hspace{0.2em}\textit{suffix}}'.
Surely, the same effect is achieved by
directly specifying the
argument `{\textit{dest}\hspace{0.2em}\textit{suffix}}'
in the first form.
However, that requires to set up a different file
for each child. With the alternative form of the command
all these files can have exactly the same content
which simplifies setting them up and maintaining them.

For example, the following file |draft.tex|
with a compilation flag |\version| as described in \secref{sec:flags}
compiles the main document as a draft:
%
\begin{center}
\begin{tabular}{l}
|\def\version{draft}|\\
|\input{childdoc.def}|\\
|\childdocforward{|\textit{main}|}|
\end{tabular}
\end{center}
%
Likewise, the following files |final|\textit{nn}|.tex|
compile the final version of the child document
|child|\textit{nn}|.tex|:
%
\begin{center}
\begin{tabular}{l}
|\def\version{final}|\\
|\input{childdoc.def}|\\
|\childdocforwardprefix{final}{child}|
\end{tabular}
\end{center}
%

Note that when several versions of a main file and/or of each child file
are to be generated, it may be convenient to set up a |Makefile| or
shell script to automatise the process.

%%%%%%%%%%%%%%%%%%%%%%%%%%%%%%%%%%%%%%%%%%%%%%%%%%%%%%%%%%%%%%%%%%%%%%%%%%%%%%%%
\subsection{Command Line Processing}
\label{sec:commandline}

The effect of redirection files can also be achieved by invoking
the \LaTeX{} compiler with a more elaborate command line.
Most conveniently this should be done as part
of a shell script or a |Makefile|.

When using \textsf{childdoc} in the main file, the following
command lines effectively perform a redirection
(note that depending on the shell being used,
backslashes may have to be doubled: `|\|' $\to$ `|\\|'):
%
\begin{center}
|... -jobname "|\textit{target}|" |\\|"|[\textit{flags}]%
|\input{childdoc.def}\childdocforward[|\textit{main}|]{|\textit{dest}|}"|
\end{center}
%
Here \textit{target} is the name of the output file,
\textit{main} is the name of the main file
and \textit{dest} is the name of the main or child file to be processed
(all filenames without extensions).
The optional argument \textit{main} can be omitted
if \textit{main} matches \textit{dest}.
Optionally, compilation \textit{flags} can be defined via |\def| commands.
This command line makes the \TeX{} engine believe
it is compiling the file \textit{target}
whose content is specified as the latter parameter.
The provided code then forwards the processing to
\textit{main} or \textit{dest} as described in \secref{sec:forward}.

%%%%%%%%%%%%%%%%%%%%%%%%%%%%%%%%%%%%%%%%%%%%%%%%%%%%%%%%%%%%%%%%%%%%%%%%%%%%%%%%
\subsection{Include by Input}
\label{sec:input}

Including child documents by |\include| has some restrictions by design.
Most notably, the content of a child document always occupies
its own set of pages; pages cannot be shared between child documents.
Usually, this behaviour makes perfect sense
because each child document contain an essential part of the document.
However, in some situations it may be desirable to compose
a document from a collection of parts
without having mandatory page breaks between then.
For this case, the package
provides a mechanism to include parts
by |\input| which can also be processed individually.
However, by construction this mechanism
requires manual handling of the content to be output.

%%%%%%%%%%%%%%%%%%%%%%%%%%%%%%%%%%%%%%%%
\DescribeMacro{\ifchilddocmanual}
The main file should be prepared as usual, see \secref{sec:include}.
However, the document body must make a distinction
between processing of an individual part and of the main document, e.g.:
%
\begin{center}
\begin{tabular}{l}
|\ifchilddocmanual|\\
|\input{\childdocname}|\\
|\||else|\\
\textit{document body with }|\input{|\textit{part}|}|\\
|\||fi|
\end{tabular}
\end{center}
%
The conditional |\ifchilddocmanual| is true whenever
a part to be included by |\input| is being compiled,
and the name of the part is stored in |\childdocname|.

%%%%%%%%%%%%%%%%%%%%%%%%%%%%%%%%%%%%%%%%
\DescribeMacro{\childdocby}
Each part to be included by |\input| should start with:
%
\begin{center}
\begin{tabular}{l}
|\input{childdoc.def}|\\
|\childdocby{|\textit{main}|}|\\
\end{tabular}
\end{center}
%
The directive |\childdocby| is similar to |\childdocof|
described in \secref{sec:include},
but the subsequent selection of content must be done manually.
To that end, both |\ifchilddoc| and |\ifchilddocmanual|
will be true upon processing of a part,
and the name of the part is stored in |\childdocname|.
Note that |\jobname| will be set to the filename of the current part
so that each part receives an individual |.aux| file
that does not interfere with the |.aux| file(s) of the main document.
This behaviour can be altered by the alternative form
|\childdocby[*]{|\textit{main}|}| (with a non-empty optional argument)
which uses the |.aux| file of the main document
by setting |\jobname| to \textit{main}.

%%%%%%%%%%%%%%%%%%%%%%%%%%%%%%%%%%%%%%%%%%%%%%%%%%%%%%%%%%%%%%%%%%%%%%%%%%%%%%%%
\subsection{Driver Development}
\label{sec:driver}

The \textsf{childdoc} mechanism can also be use for the development
of definition files such as \LaTeX{} styles or classes.
This case differs from the above setup with multiple parts
included by |\include| in that no |\includeonly| should be invoked.
This can be achieved by starting the include file
(before |\ProvidesPackage|) with:
%
\begin{center}
\begin{tabular}{l}
|\input{childdoc.def}|\\
|\childdocforward{|\textit{main}|}|\\
\end{tabular}
\end{center}
%
or alternatively with:
%
\begin{center}
\begin{tabular}{l}
|\input{childdoc.def}|\\
|\childdocby{|\textit{main}|}|\\
\end{tabular}
\end{center}
%
Both forms have slightly different effects as described above.
The main file is prepared as usual, see \secref{sec:include}.

%%%%%%%%%%%%%%%%%%%%%%%%%%%%%%%%%%%%%%%%%%%%%%%%%%%%%%%%%%%%%%%%%%%%%%%%%%%%%%%%
\subsection{Legacy Detection}
\label{sec:detection}

The directive |\childdocmain| in the main file can detect
whether the complete document or merely a child is to be compiled
even without using the directive |\childdocof|.
This method is deprecated because it is less robust
and there is no compelling reason to use it;
it is merely provided for backward compatibility
and it may be removed in future versions.

If the detection mechanism is to be used,
it is mandatory to correctly specify
the filename of the main file as the argument of |\childdocmain|:
%
\begin{center}
\begin{tabular}{l}
|\input{childdoc.def}|\\
|\childdocmain{|\textit{main}|}|\\
\end{tabular}
\end{center}
%
If |\jobname| does not match the argument \textit{main} of |\childdocmain|,
it is assumed that |\jobname| points to the child file to be compiled.
When using |\childdocmain| with the main file specified as argument,
it suffices to start a child file
with just |\input{|\textit{main}|}|
without loading of the package and using |\childdocof|.
If instead all processing is done
with the appropriate \textsf{childdoc} directives,
the argument of \textit{main} of |\childdocmain| can be empty.

An alternative version of the command line processing described
in \secref{sec:commandline} using the detection mechanism reads:
%
\begin{center}
|... -jobname "|\textit{target}|" "|[\textit{flags}]%
[|\def\jobname{|\textit{dest}|}|]|\input{|\textit{main}|}"|
\end{center}

%%%%%%%%%%%%%%%%%%%%%%%%%%%%%%%%%%%%%%%%%%%%%%%%%%%%%%%%%%%%%%%%%%%%%%%%%%%%%%%%
\subsection{Manual Code}
\label{sec:manual}

In case one cannot be certain whether the definitions file |childdoc.def|
is installed on the target \TeX{} distribution
and one prefers not to ship it,
it is conceivable to paste a few relevant commands into the sources.

To that end, drop all statements |\input{childdoc.def}|
and perform the replacements as outlined below.
Instead of |\childdocmain{|\textit{main}|}| add the following code
to the top of the main file:
%
\begin{center}
\begin{tabular}{l}
|\||ifdefined\childdocname\endinput\||fi\newif\ifchilddoc|\\
|\edef\childdocname{\scantokens\expandafter{\jobname\noexpand}}|\\
|\def\childdocmain{|\textit{main}|}\||ifx\childdocmain\childdocname\||else|\\
|\childdoctrue\includeonly{\childdocname}\let\jobname\childdocmain\||fi|\\
\end{tabular}
\end{center}
%
Instead of |\childdocof{|\textit{main}|}| just include the main file
at the top of each child file:
%
\begin{center}
|\input{|\textit{main}|}|
\end{center}
%
A simple redirection |\childdocforward{|\textit{dest}|}| is achieved by:
%
\begin{center}
|\def\jobname{|\textit{dest}|}\input{\jobname}|
\end{center}
%
The redirection with prefix
|\childdocforwardprefix[|\textit{prefix}|]{|\textit{dest}|}|
is accomplished by:
%
\begin{center}
\begin{tabular}{l}
|{\edef\jobname{\scantokens\expandafter{\jobname\noexpand}}|\\
|\def\redirectjob |\textit{prefix}|#1~~~{\gdef\jobname{|\textit{dest}|#1}}|\\
|\expandafter\redirectjob\jobname~~~}\input{\jobname}|
\end{tabular}
\end{center}

In an alternative approach,
child documents can be compiled by a specific command line
without additional code or specific definitions:
%
\begin{center}
|... -jobname "|\textit{target}|" "|[\textit{flags}]%
|\includeonly{|\textit{dest}|}\input{|\textit{main}|}"|
\end{center}
%

%%%%%%%%%%%%%%%%%%%%%%%%%%%%%%%%%%%%%%%%%%%%%%%%%%%%%%%%%%%%%%%%%%%%%%%%%%%%%%%%
%%%%%%%%%%%%%%%%%%%%%%%%%%%%%%%%%%%%%%%%%%%%%%%%%%%%%%%%%%%%%%%%%%%%%%%%%%%%%%%%
\section{Information}

%%%%%%%%%%%%%%%%%%%%%%%%%%%%%%%%%%%%%%%%%%%%%%%%%%%%%%%%%%%%%%%%%%%%%%%%%%%%%%%%
\subsection{Copyright}

Copyright \copyright{} 2017--2018 Niklas Beisert

This work may be distributed and/or modified under the
conditions of the \LaTeX{} Project Public License, either version 1.3
of this license or (at your option) any later version.
The latest version of this license is in
  \url{http://www.latex-project.org/lppl.txt}
and version 1.3 or later is part of all distributions of \LaTeX{}
version 2005/12/01 or later.

This work has the LPPL maintenance status `maintained'.

The Current Maintainer of this work is Niklas Beisert.

This work consists of the files |README.txt|, |childdoc.ins| and |childdoc.dtx|
as well as the derived files |childdoc.def|, |cdocsamp.tex|
with |cdocsch1.tex|, |cdocsch2.tex|, |cdocspt3.tex|, |cdocspt4.tex|,
|cdocsdrf.tex|, |cdocsfn1.tex|, |cdocsfn2.tex|
as well as |childdoc.pdf|.

%%%%%%%%%%%%%%%%%%%%%%%%%%%%%%%%%%%%%%%%%%%%%%%%%%%%%%%%%%%%%%%%%%%%%%%%%%%%%%%%
\subsection{Files and Installation}

The package consists of the files:
%
\begin{center}
\begin{tabular}{ll}
    |README.txt|   & readme file \\
    |childdoc.ins| & installation file \\
    |childdoc.dtx| & source file \\
    |childdoc.def| & definition file \\
    |cdocsamp.tex| & sample main file \\
    |cdocsch1.tex| & sample include file \\
    |cdocsch2.tex| & sample include file \\
    |cdocspt3.tex| & sample part file \\
    |cdocspt4.tex| & sample part file \\
    |cdocsdrf.tex| & sample redirection file \\
    |cdocsfn1.tex| & sample redirection file \\
    |cdocsfn2.tex| & sample redirection file \\
    |childdoc.pdf| & manual
\end{tabular}
\end{center}
%
The distribution consists of the files
|README.txt|, |childdoc.ins| and |childdoc.dtx|.
%
\begin{itemize}
\item
Run (pdf)\LaTeX{} on |childdoc.dtx|
to compile the manual |childdoc.pdf| (this file).
\item
Run \LaTeX{} on |childdoc.ins| to create the definitions file |childdoc.def|
and the sample |cdocsamp.tex| with include files
|cdocsch1.tex|, |cdocsch2.tex|, |cdocspt3.tex|, |cdocspt4.tex|,
|cdocsdrf.tex|, |cdocsfn1.tex|, |cdocsfn2.tex|.
Then copy the file |childdoc.def| to an appropriate directory of your \LaTeX{}
distribution, e.g.\ \textit{texmf-root}|/tex/latex/childdoc|.
\end{itemize}

%%%%%%%%%%%%%%%%%%%%%%%%%%%%%%%%%%%%%%%%%%%%%%%%%%%%%%%%%%%%%%%%%%%%%%%%%%%%%%%%
\subsection{Related CTAN Packages}

There are several other packages which offer a similar functionality:
%
\begin{itemize}
\item
The packages
\href{http://ctan.org/pkg/docmute}{\textsf{docmute}},
\href{http://ctan.org/pkg/includex}{\textsf{includex}} and
\href{http://ctan.org/pkg/standalone}{\textsf{standalone}}
provide commands to include only the document body of
a child file thus allowing both files to be compiled individually.
\item
The packages \href{http://ctan.org/pkg/subdocs}{\textsf{subdocs}}
and \href{http://ctan.org/pkg/subfiles}{\textsf{subfiles}}
provide structures in which the main and child documents can be
encapsulated and allowing them to be compiled individually.
The inclusion mechanism is different from the conventional |\include|.
\item
The package \href{http://ctan.org/pkg/combine}{\textsf{combine}}
is an elaborate solution to combine several documents into one.
\end{itemize}
%
See also the CTAN topic \href{http://ctan.org/topic/subdocs}{\textsf{subdocs}}
for further related packages.
The present package differs from the above solutions in that
a document structure constructed with the conventional |\include| mechanism
just needs two extra commands at the top of every file
such that all constituent files can be compiled individually.

%%%%%%%%%%%%%%%%%%%%%%%%%%%%%%%%%%%%%%%%%%%%%%%%%%%%%%%%%%%%%%%%%%%%%%%%%%%%%%%%
%\subsection{Feature Suggestions}
%
%The following is a list of features which may be useful for future
%versions of this package:
%%
%\begin{itemize}
%\item
%\ldots
%\end{itemize}

%%%%%%%%%%%%%%%%%%%%%%%%%%%%%%%%%%%%%%%%%%%%%%%%%%%%%%%%%%%%%%%%%%%%%%%%%%%%%%%%
\subsection{Revision History}

%%%%%%%%%%%%%%%%%%%%%%%%%%%%%%%%%%%%%%%%
\paragraph{v2.0:} 2018/12/30

\begin{itemize}
\item
immediate forward processing
\item
added |\childdocby| mechanism
\item
manual restructured
\end{itemize}

%%%%%%%%%%%%%%%%%%%%%%%%%%%%%%%%%%%%%%%%
\paragraph{v1.6:} 2018/01/17

\begin{itemize}
\item
application for development of include files
\item
corrections to manual
\end{itemize}

%%%%%%%%%%%%%%%%%%%%%%%%%%%%%%%%%%%%%%%%
\paragraph{v1.5:} 2017/05/21

\begin{itemize}
\item
more complete structuring introduced
\item
|\childdocof| introduced
\item
|\childdoc| renamed to |\childdocmain|
\item
|\childredirect| renamed to |\childdocforward| and |\childdocforwardprefix|
and functionality expanded
\end{itemize}

%%%%%%%%%%%%%%%%%%%%%%%%%%%%%%%%%%%%%%%%
\paragraph{v1.0:} 2017/04/27

\begin{itemize}
\item
manual and install package
\item
first version published on CTAN
\end{itemize}

%%%%%%%%%%%%%%%%%%%%%%%%%%%%%%%%%%%%%%%%
\paragraph{v0.6:} 2017/04/26

\begin{itemize}
\item
redirection mechanism added
\end{itemize}

%%%%%%%%%%%%%%%%%%%%%%%%%%%%%%%%%%%%%%%%
\paragraph{v0.5:} 2017/04/26

\begin{itemize}
\item
functionality in definition file
\end{itemize}


%%%%%%%%%%%%%%%%%%%%%%%%%%%%%%%%%%%%%%%%%%%%%%%%%%%%%%%%%%%%%%%%%%%%%%%%%%%%%%%%
%%%%%%%%%%%%%%%%%%%%%%%%%%%%%%%%%%%%%%%%%%%%%%%%%%%%%%%%%%%%%%%%%%%%%%%%%%%%%%%%
%%%%%%%%%%%%%%%%%%%%%%%%%%%%%%%%%%%%%%%%%%%%%%%%%%%%%%%%%%%%%%%%%%%%%%%%%%%%%%%%
\appendix

\settowidth\MacroIndent{\rmfamily\scriptsize 000\ }

 \DocInput{childdoc.dtx}

\end{document}
%</driver>
% \fi
%
% %%%%%%%%%%%%%%%%%%%%%%%%%%%%%%%%%%%%%%%%%%%%%%%%%%%%%%%%%%%%%%%%%%%%%%%%%%%%%%
% %%%%%%%%%%%%%%%%%%%%%%%%%%%%%%%%%%%%%%%%%%%%%%%%%%%%%%%%%%%%%%%%%%%%%%%%%%%%%%
% \section{Sample}
%\iffalse
%<*samplemain>
%\fi
%
% The following presents a sample document
% with two chapters, two parts, a title page,
% a compile flag as well as three forwarding files to set the flag.
% It consists of eight |.tex| files:
% \begin{center}
% \begin{tabular}{ll}
% |cdocsamp.tex|&main file\\
% |cdocsch1.tex|&include file for chapter 1\\
% |cdocsch2.tex|&include file for chapter 2\\
% |cdocspt3.tex|&include file for part 3\\
% |cdocspt4.tex|&include file for part 4\\
% |cdocsdrf.tex|&forwarding file for main file in draft mode\\
% |cdocsfi1.tex|&forwarding file for final version of chapter 1\\
% |cdocsfi2.tex|&forwarding file for final version of chapter 2\\
% \end{tabular}
% \end{center}
% Each of the eight files can be compiled directly by the \LaTeX{} compiler.
%
% %%%%%%%%%%%%%%%%%%%%%%%%%%%%%%%%%%%%%%
% \paragraph{Main File.}
%
% The main file is called |cdocsamp.tex|.
%
% Load the \textsf{childdoc} definitions and
% declare the filename for the main document:
%    \begin{macrocode}
\input{childdoc.def}
\childdocmain{}
%    \end{macrocode}

% Optional override for |\version| flag:
%    \begin{macrocode}
%%\ifchilddoc\else\providecommand{\version}{draft}\fi
%    \end{macrocode}

% Define the default values for the |\version| flag
% (|final| for the main file and |draft| for childs):
%    \begin{macrocode}
\ifchilddoc
\providecommand{\version}{draft}
\else
\providecommand{\version}{final}
\fi
%    \end{macrocode}

% Load the standard document class:
%    \begin{macrocode}
\documentclass[12pt]{article}
%    \end{macrocode}

% Start the document body:
%    \begin{macrocode}
\begin{document}
%    \end{macrocode}

% Declare a title page.
% Print title, part of document being processed and version flag:
%    \begin{macrocode}
\addtocounter{page}{-1}
\begin{center}
{\LARGE\bfseries{}childdoc example\par}
\vspace{1cm}
\ifchilddoc
\ifchilddocmanual part\else chapter\fi:
`\childdocname' of `\childdocjob'\par
\else
main document: `\childdocjob'\par
\fi
version: \version\par
\end{center}
\newpage
%    \end{macrocode}

% Manually include selected file,
% otherwise process as usual:
%    \begin{macrocode}
\ifchilddocmanual
\section*{part `\childdocname'}
\input{\childdocname}
\else
%    \end{macrocode}

% Include the two chapters:
%    \begin{macrocode}
\include{cdocsch1}
\include{cdocsch2}
%    \end{macrocode}

% Include the two parts unless only chapters should be displayed:
%    \begin{macrocode}
\ifchilddoc\else
\section{part three}
\input{cdocspt3}
\section{part four}
\input{cdocspt4}
\fi
%    \end{macrocode}

% Process as usual until here:
%    \begin{macrocode}
\fi
%    \end{macrocode}

% End of document body:
%    \begin{macrocode}
\end{document}
%    \end{macrocode}
%\iffalse
%</samplemain>
%\fi
%
% %%%%%%%%%%%%%%%%%%%%%%%%%%%%%%%%%%%%%%
% \paragraph{Chapter Include Files.}
%
% The include files are called |cdocsch1.tex| and |cdocsch2.tex|.
%
%\iffalse
%<*samplechap1|samplechap2>
%\fi

% Optional override for |\version| flag:
%    \begin{macrocode}
%%\providecommand{\version}{final}
%    \end{macrocode}

% Include the main document:
%    \begin{macrocode}
\input{childdoc.def}
\childdocof{cdocsamp}
%    \end{macrocode}

%\iffalse
%</samplechap1|samplechap2>
%\fi
%
%\iffalse
%<*samplechap1>
%\fi
% Some text for chapter 1:
%    \begin{macrocode}
\section{one}
some text in chapter one
%    \end{macrocode}

%\iffalse
%</samplechap1>
%\fi
% Some text for chapter 2:
%\iffalse
%<*samplechap2>
%\fi
%    \begin{macrocode}
\section{two}
more text in chapter two
%    \end{macrocode}

%\iffalse
%</samplechap2>
%\fi
%
% %%%%%%%%%%%%%%%%%%%%%%%%%%%%%%%%%%%%%%
% \paragraph{Part Include Files.}
%
% The include files are called |cdocspt3.tex| and |cdocspt4.tex|.
%
%\iffalse
%<*samplepart3|samplepart4>
%\fi

% Optional override for |\version| flag:
%    \begin{macrocode}
%%\providecommand{\version}{final}
%    \end{macrocode}

% Include the main document:
%    \begin{macrocode}
\input{childdoc.def}
\childdocby{cdocsamp}
%    \end{macrocode}

%\iffalse
%</samplepart3|samplepart4>
%\fi
%
%\iffalse
%<*samplepart3>
%\fi
% Some text for part 3:
%    \begin{macrocode}
some text in part three
%    \end{macrocode}

%\iffalse
%</samplepart3>
%\fi
% Some text for part 4:
%\iffalse
%<*samplepart4>
%\fi
%    \begin{macrocode}
more text in part four
%    \end{macrocode}

%\iffalse
%</samplepart4>
%\fi
%
% %%%%%%%%%%%%%%%%%%%%%%%%%%%%%%%%%%%%%%
% \paragraph{Forwarding for a Complete Draft.}
%
% The following forwarding file |cdocsdrf.tex|
% compiles the main document in draft mode:
%\iffalse
%<*sampledraft>
%\fi
%    \begin{macrocode}
\def\version{draft}
\input{childdoc.def}
\childdocforward{cdocsamp}
%    \end{macrocode}

%\iffalse
%</sampledraft>
%\fi
%
% %%%%%%%%%%%%%%%%%%%%%%%%%%%%%%%%%%%%%%
% \paragraph{Forwarding for Final Version of the Chapters.}
%
% The following forwarding files |cdocsfn1.tex| and |cdocsfn2.tex|
% (with identical content)
% compile the final versions of the child documents
% |cdocsch1.tex| and |cdocsch2.tex|, respectively:
%\iffalse
%<*samplefinal>
%\fi
%    \begin{macrocode}
\def\version{final}
\input{childdoc.def}
\childdocforwardprefix[cdocsamp]{cdocsfn}{cdocsch}
%    \end{macrocode}

%\iffalse
%</samplefinal>
%\fi
%
% %%%%%%%%%%%%%%%%%%%%%%%%%%%%%%%%%%%%%%
% \paragraph{Command Line Processing.}
%
% The following three command lines generate the output files
% |cdocscld|, |cdocscl1| and |cdocscl2|
% which should be identical to
% |cdocsdrf|, |cdocsch1| and |cdocsfn2|, respectively:
% \begin{center}
% \begin{tabular}{l}
% |latex -jobname cdocscld \|\\
% |  "\def\version{draft}\input{childdoc.def}\childdocforward{cdocsamp}"|\\
% |latex -jobname cdocscl1 \|\\
% |  "\input{childdoc.def}\childdocforward[cdocsamp]{cdocsch1}"|\\
% |latex -jobname cdocscl2 \|\\
% |  "\def\version{final}\input{childdoc.def}\childdocforward{cdocsch2}"|
% \end{tabular}
% \end{center}
% Note that the trailing backslash on each first line
% merely continues the input to the second line
% (for convenient cut ant paste).
% Furthermore, the command |latex| can be replaced by any
% of its alternative versions such as |pdflatex|.
%
% %%%%%%%%%%%%%%%%%%%%%%%%%%%%%%%%%%%%%%%%%%%%%%%%%%%%%%%%%%%%%%%%%%%%%%%%%%%%%%
% %%%%%%%%%%%%%%%%%%%%%%%%%%%%%%%%%%%%%%%%%%%%%%%%%%%%%%%%%%%%%%%%%%%%%%%%%%%%%%
% \section{Implementation}
%\iffalse
%<*package>
%\fi
%
% This section describes the definitions file |childdoc.def|.

% The definitions cannot be loaded using |\usepackage| or |\RequirePackage|
% which has a mechanism to prevent loading a style file more than once.
% When loading the definitions by means of |\input|
% multiple instances have to be prevented manually:
%\iffalse
%This code needs to be before the `\ProvidesFile' directive
%which is defined at the beginning of this file.
%Therefore it is also placed there and commented out here.
%</package>
%<*discard>
%\fi
%    \begin{macrocode}
\ifdefined\childdocmain\endinput\fi
%    \end{macrocode}
%\iffalse
%</discard>
%<*package>
%\fi
%
% \macro{\ifchilddoc}
% \macro{\ifchilddocmanual}
% The conditional |\ifchilddoc| tells whether a
% child (true) or main (false) document is being compiled.
% The conditional |\ifchilddocmanual| tells whether
% the |\includeonly| mechanism is used (false) or
% the selection of child files must be performed manually (true).
% The definitions initialise to false:
%    \begin{macrocode}
\newif\ifchilddoc
\newif\ifchilddocmanual
%    \end{macrocode}

% \macro{\childdocname}
% \macro{\childdocjob}
% The macro |\childdocname| stores the name of the main document
% to be compiled. The macro |\childdocjob| stores the name of
% the document on which the \LaTeX{} compiler was originally invoked.
% The content of |\jobname| cannot be compared
% to filenames specified in the source due to different catcodes.
% The following code rescans |\jobname|, stores the result
% in |\childdocname| and saves a copy in |\childdocjob|:
%    \begin{macrocode}
\edef\childdocname{\scantokens\expandafter{\jobname\noexpand}}
\let\childdocjob\childdocname
%    \end{macrocode}

% \macro{\childdocdisable}
% The macro |\childdocdisable| prevents the main file
% from being processed more than once.
% At this stage, the main document command |\childdocmain|
% is assumed to be called once again where it should do nothing.
% Any subsequent call to it should prevent
% a secondary processing of the main document
% It overwrites the forwarding commands
% |\childdocof| and |\childdocforward|
% with empty macros to prevent further inclusions of the main document:
%    \begin{macrocode}
\newcommand{\childdocdisable}
{
  \renewcommand{\childdocmain}[1]{\renewcommand{\childdocmain}[1]{\endinput}}
  \renewcommand{\childdocof}[1]{}
  \renewcommand{\childdocby}[2][]{}
  \renewcommand{\childdocforward}[2][]{}
  \renewcommand{\childdocdisable}{}
}
%    \end{macrocode}

% \macro{\childdocmain}
% The macro |\childdocmain| is to be called at the top of the main file
% with nothing or the main filename (without extension) as argument.
% First, it breaks loops.
% If the argument is not empty and does not match |\childdocname|
% (which is set by the first inclusion of |childdoc.def|),
% |\ifchilddoc| is set to true, |\includeonly| is applied to the child file
% and |\jobname| is set to the main file
% (for proper handling of |.aux| files):
%    \begin{macrocode}
\newcommand{\childdocmain}[1]
{
  \childdocdisable\childdocmain{}
  \if?#1?\else
    \begingroup
      \def\childdoctmp{#1}
      \ifx\childdoctmp\childdocname
        \def\childdoctmp{}
      \else
        \def\childdoctmp
        {
          \childdoctrue
          \includeonly{\childdocname}
          \def\childdocjob{#1}
          \def\jobname{#1}
        }
      \fi
      \expandafter
    \endgroup
    \childdoctmp
  \fi
}
%    \end{macrocode}

% \macro{\childdocof}
% The command |\childdocof| redirects
% compilation to the main file |#1|.
%    \begin{macrocode}
\newcommand{\childdocof}[1]
{
  \childdocdisable
  \childdoctrue
  \includeonly{\childdocname}
  \def\jobname{#1}
  \def\childdocjob{#1}
  \input{#1}
}
%    \end{macrocode}

% \macro{\childdocby}
% The command |\childdocby| ....
%    \begin{macrocode}
\newcommand{\childdocby}[2][]
{
  \childdocdisable
  \childdoctrue
  \childdocmanualtrue
  \if?#1?\else
    \def\jobname{#2}
  \fi
  \def\childdocjob{#2}
  \input{#2}
  \endinput
}
%    \end{macrocode}

% \macro{\childdocforward}
% The command |\childdocforward| redirects
% compilation to the main file or
% (if the optional argument is given) a child file.
% Parameters are set as if the main file
% or a child file starting with |\childdocof| was compiled.
% Then compilation is handed over to the main file:
%    \begin{macrocode}
\newcommand{\childdocforward}[2][]
{
  \begingroup
    \if?#1?
      \def\childdoctmp
      {
        \def\childdocname{#2}
        \def\childdocjob{#2}
        \def\jobname{#2}
        \input{#2}
        \endinput
      }
    \else
      \def\childdoctmp
      {
        \childdocdisable
        \def\childdocname{#2}
        \childdoctrue
        \includeonly{#2}
        \def\childdocjob{#1}
        \def\jobname{#1}
        \input{#1}
        \endinput
      }
    \fi
    \expandafter
  \endgroup
  \childdoctmp
}
%    \end{macrocode}

% \macro{\childdocforwardprefix}
% The command |\childdocforwardprefix| redirects
% compilation to the main or a child file by means of a pattern.
% The prefix |#1| in the current filename is replaced by |#2|
% and the suffix of the current filename is kept
% (it is assumed that the filename does not contain the substring `|~~~|'
% which is used as a delimiter).
% Compilation is handed over to the new file by |\childdocforward|:
%    \begin{macrocode}
\newcommand{\childdocforwardprefix}[3][]
{
  \begingroup
    \def\childdocextract #2##1~~~{\def\childdoctmp{\childdocforward[#1]{#3##1}}}
    \expandafter\childdocextract\childdocname~~~
    \expandafter
  \endgroup
  \childdoctmp
}
%    \end{macrocode}

% \macro{\childdoc}
% The deprecated macro |\childdoc| is a legacy version of |\childdocmain|:
%    \begin{macrocode}
\newcommand{\childdoc}{\childdocmain}
%    \end{macrocode}

% \macro{\childdocredirect}
% The deprecated macro |\childdocredirect| is a legacy version
% of |\childdocforward| and |\childdocforwardprefix|:
%    \begin{macrocode}
\newcommand{\childdocredirect}[2][]
{
  \begingroup
    \if?#1?
      \def\childdoctmp{\childdocforward{#2}}
    \else
      \def\childdoctmp{\childdocforwardprefix{#1}{#2}}
    \fi
    \expandafter
  \endgroup
  \childdoctmp
}
%    \end{macrocode}

%\iffalse
%</package>
%\fi
%
\endinput

\childdocmain{}
%    \end{macrocode}

% Optional override for |\version| flag:
%    \begin{macrocode}
%%\ifchilddoc\else\providecommand{\version}{draft}\fi
%    \end{macrocode}

% Define the default values for the |\version| flag
% (|final| for the main file and |draft| for childs):
%    \begin{macrocode}
\ifchilddoc
\providecommand{\version}{draft}
\else
\providecommand{\version}{final}
\fi
%    \end{macrocode}

% Load the standard document class:
%    \begin{macrocode}
\documentclass[12pt]{article}
%    \end{macrocode}

% Start the document body:
%    \begin{macrocode}
\begin{document}
%    \end{macrocode}

% Declare a title page.
% Print title, part of document being processed and version flag:
%    \begin{macrocode}
\addtocounter{page}{-1}
\begin{center}
{\LARGE\bfseries{}childdoc example\par}
\vspace{1cm}
\ifchilddoc
\ifchilddocmanual part\else chapter\fi:
`\childdocname' of `\childdocjob'\par
\else
main document: `\childdocjob'\par
\fi
version: \version\par
\end{center}
\newpage
%    \end{macrocode}

% Manually include selected file,
% otherwise process as usual:
%    \begin{macrocode}
\ifchilddocmanual
\section*{part `\childdocname'}
\input{\childdocname}
\else
%    \end{macrocode}

% Include the two chapters:
%    \begin{macrocode}
\include{cdocsch1}
\include{cdocsch2}
%    \end{macrocode}

% Include the two parts unless only chapters should be displayed:
%    \begin{macrocode}
\ifchilddoc\else
\section{part three}
\input{cdocspt3}
\section{part four}
\input{cdocspt4}
\fi
%    \end{macrocode}

% Process as usual until here:
%    \begin{macrocode}
\fi
%    \end{macrocode}

% End of document body:
%    \begin{macrocode}
\end{document}
%    \end{macrocode}
%\iffalse
%</samplemain>
%\fi
%
% %%%%%%%%%%%%%%%%%%%%%%%%%%%%%%%%%%%%%%
% \paragraph{Chapter Include Files.}
%
% The include files are called |cdocsch1.tex| and |cdocsch2.tex|.
%
%\iffalse
%<*samplechap1|samplechap2>
%\fi

% Optional override for |\version| flag:
%    \begin{macrocode}
%%\providecommand{\version}{final}
%    \end{macrocode}

% Include the main document:
%    \begin{macrocode}
% \iffalse
%
% childdoc.dtx Copyright (C) 2017-2018 Niklas Beisert
%
% This work may be distributed and/or modified under the
% conditions of the LaTeX Project Public License, either version 1.3
% of this license or (at your option) any later version.
% The latest version of this license is in
%   http://www.latex-project.org/lppl.txt
% and version 1.3 or later is part of all distributions of LaTeX
% version 2005/12/01 or later.
%
% This work has the LPPL maintenance status `maintained'.
%
% The Current Maintainer of this work is Niklas Beisert.
%
% This work consists of the files childdoc.dtx and childdoc.ins
% and the derived files childdoc.def and cdocsamp.tex with
% cdocsch1.tex, cdocsch2.tex, cdocsdrf.tex, cdocsfn1.tex, cdocsfn2.tex.
%
%<package>\ifdefined\childdocmain\endinput\fi
%<package>\ProvidesFile{childdoc.def}[2018/12/30 v2.0 child document driver]
%<samplemain>\ProvidesFile{cdocsamp.tex}[2018/12/30 v2.0 sample for childdoc]
%<*driver>
%\ProvidesFile{childdoc.drv}[2018/12/30 v2.0 childdoc reference manual file]
\PassOptionsToClass{10pt,a4paper}{article}
\documentclass{ltxdoc}

\usepackage[margin=35mm]{geometry}
\usepackage{hyperref}
\usepackage{hyperxmp}
\usepackage[usenames]{color}

\hypersetup{colorlinks=true}
\hypersetup{pdfstartview=FitH}
\hypersetup{pdfpagemode=UseNone}
\hypersetup{pdfsource={}}
\hypersetup{pdflang={en-UK}}
\hypersetup{pdfcopyright={Copyright 2017-2018 Niklas Beisert.
  This work may be distributed and/or modified under the
  conditions of the LaTeX Project Public License, either version 1.3
  of this license or (at your option) any later version.}}
\hypersetup{pdflicenseurl={http://www.latex-project.org/lppl.txt}}
\hypersetup{pdfcontactaddress={ETH Zurich, ITP, HIT K,
  Wolfgang-Pauli-Strasse 27}}
\hypersetup{pdfcontactpostcode={8093}}
\hypersetup{pdfcontactcity={Zurich}}
\hypersetup{pdfcontactcountry={Switzerland}}
\hypersetup{pdfcontactemail={nbeisert@itp.phys.ethz.ch}}
\hypersetup{pdfcontacturl={http://people.phys.ethz.ch/\xmptilde nbeisert/}}

\newcommand{\secref}[1]{\hyperref[#1]{section \ref*{#1}}}

\parskip1ex
\parindent0pt
\let\olditemize\itemize
\def\itemize{\olditemize\parskip0pt}

\begin{document}

\title{The \textsf{childdoc} Package}
\hypersetup{pdftitle={The childdoc Package}}
\author{Niklas Beisert\\[2ex]
  Institut f\"ur Theoretische Physik\\
  Eidgen\"ossische Technische Hochschule Z\"urich\\
  Wolfgang-Pauli-Strasse 27, 8093 Z\"urich, Switzerland\\[1ex]
  \href{mailto:nbeisert@itp.phys.ethz.ch}
  {\texttt{nbeisert@itp.phys.ethz.ch}}}
\hypersetup{pdfauthor={Niklas Beisert}}
\hypersetup{pdfsubject={Manual for the LaTeX2e Package childdoc}}
\date{30 December 2018, \textsf{v2.0}}
\maketitle

\begin{abstract}\noindent
\textsf{childdoc} is a \LaTeXe{} package
that enables the direct compilation
of document sections included by |\include|
to individual files.
\end{abstract}

\begingroup
\parskip0ex
\tableofcontents
\endgroup

%%%%%%%%%%%%%%%%%%%%%%%%%%%%%%%%%%%%%%%%%%%%%%%%%%%%%%%%%%%%%%%%%%%%%%%%%%%%%%%%
%%%%%%%%%%%%%%%%%%%%%%%%%%%%%%%%%%%%%%%%%%%%%%%%%%%%%%%%%%%%%%%%%%%%%%%%%%%%%%%%
\section{Introduction}

\LaTeX{} provides a mechanism to structure a large document (such as a book)
into a main file and several child files (containing the chapters)
using the |\include| command.
This mechanism is beneficial for documents
which span hundreds of pages in order to
make the source file(s) more manageable.
Moreover, compilation can be restricted to
selected child files by means of the |\includeonly| command.
The latter feature can be used to reduce the compilation time while editing
(this was significantly more useful in the earlier days of \LaTeX{})
or to generate a smaller document which is easier to navigate.
Another application of |\includeonly| is to generate
documents consisting of selected parts of the complete document.

However, there are a few drawbacks of the plain |\include| mechanism:
\begin{itemize}
\item
The child files cannot be compiled on their own,
they can only be compiled via the main file.
A naive editing environment
(such as a text editor with an option
to have the current file processed by \LaTeX)
may require one to switch to the main file before compiling;
attempting to compile the child file produces errors.
\item
The main file must be modified (each time)
to adjust the |\includeonly| command
to the present needs. This easily leaves the main file in a messy state.
\item
The generated document will always carry the filename
of the main document. This is inconvenient if
several child files are to be compiled and
to be kept for distribution.
\end{itemize}

The present package provides a simple interface
to make child files individually compilable by \LaTeX{}.
Compiling a child file then has the same effect as compiling
the main file with an |\includeonly| command
to select the appropriate child.
Moreover the generated document will carry the name of the child
rather than the main file.
This resolves all three above issues.

This feature is meant to make the editing of books,
thesis documents and lecture notes somewhat more convenient.
However, the package can also be used efficiently for
composing a series of documents (such as exercise sheets)
which are typically distributed individually.
It then assists the author in generating the individual documents
(potentially in different versions)
as well as a document containing the collected series.
Another application is in developing style files
or other kinds of included material
where compilation of the style file could redirect
to a sample or test file.

%%%%%%%%%%%%%%%%%%%%%%%%%%%%%%%%%%%%%%%%%%%%%%%%%%%%%%%%%%%%%%%%%%%%%%%%%%%%%%%%
%%%%%%%%%%%%%%%%%%%%%%%%%%%%%%%%%%%%%%%%%%%%%%%%%%%%%%%%%%%%%%%%%%%%%%%%%%%%%%%%
\section{Usage}

First of all, the package \textsf{childdoc} is \emph{not} a standard
\LaTeXe{} |.sty| style file! Therefore it needs to be invoked in
a non-standard way.

%%%%%%%%%%%%%%%%%%%%%%%%%%%%%%%%%%%%%%%%%%%%%%%%%%%%%%%%%%%%%%%%%%%%%%%%%%%%%%%%
\subsection{Included Files}
\label{sec:include}

%%%%%%%%%%%%%%%%%%%%%%%%%%%%%%%%%%%%%%%%
\DescribeMacro{\childdocmain}
To use the package, add the commands
\begin{center}
\begin{tabular}{l}
|\input{childdoc.def}|\\
|\childdocmain{}|\\
\end{tabular}
\end{center}
at the very top of the main \LaTeX{} file,
in particular \emph{before} the |\documentclass| statement!
The argument of |\childdocmain| should be left empty
(but it must be present).

%%%%%%%%%%%%%%%%%%%%%%%%%%%%%%%%%%%%%%%%
\DescribeMacro{\childdocof}
Furthermore, add the commands
\begin{center}
\begin{tabular}{l}
|\input{childdoc.def}|\\
|\childdocof{|\textit{main}|}|\\
\end{tabular}
\end{center}
at the top of every child file \textit{child}
which is included by |\include{|\textit{child}|}|
from within the main file
(or at least for those files to be compiled individually).
The argument \textit{main} must be the filename of the main file.

There are a couple of
considerations in setting up the main and child documents:

%%%%%%%%%%%%%%%%%%%%%%%%%%%%%%%%%%%%%%%%
\paragraph{Restrictions.}

Please note the following restrictions:
\begin{itemize}
\item
|\childdocmain| must be called with one argument \textit{main}
to ensure compatibility with earlier version of the package.
It must either be empty (|\childdocmain{}|)
or precisely match the filename of the main file in which it is specified.
See \secref{sec:detection} for further information.
\item
The filename \textit{main} must be specified without the |.tex| extension.
\item
The filename \textit{main} is case sensitive
(even in case-insensitive file systems)
due to internal string comparison.
\item
The argument \textit{main} should be fully expanded, it cannot be a macro.
\item
Subdirectories and special characters should be avoided in filenames.
\item
The command |\childdocmain{|\textit{main}|}| must be followed by a whitespace.
It should not be followed immediately by another command
or by a comment mark `|%|'.
This is because the \TeX{} parser reads the token immediately following
the argument of |\childdocmain| and puts it
at the beginning of every child section;
however, a white\-space is ignored.
\end{itemize}

%%%%%%%%%%%%%%%%%%%%%%%%%%%%%%%%%%%%%%%%
\paragraph{Content of Main File.}

It is advisable to place all content in the child files included by |\include|.
Any output contained in the main file will appear in all child documents
unless suppressed manually;
it cannot be suppressed automatically by the |\includeonly| directive
and thus should normally be avoided.
A method to include some content in the main file
by means of conditional processing is described in \secref{sec:conditional}.

%%%%%%%%%%%%%%%%%%%%%%%%%%%%%%%%%%%%%%%%
\paragraph{Page Numbering.}

When only a part of the document is compiled,
the appropriate numbering of pages
(as well as other status parameters)
is determined from the |.aux| files.
The latter contain information from previous passes.
However this information needs to propagate through
all intermediate child documents.
Therefore the page numbering in child documents may well
be inconsistent until the complete document is compiled at least once.

A useful (if unconventional) way to always ensure a consistent
page numbering is to restart the numbering in each child document
and denote the pages by `\textit{child}|.|\textit{page}'
where \textit{child} represents the chapter/section number of the child file.
This can be achieved by the command
|\numberwithin{page}{|\textit{child}|}|
of the \textsf{amsmath} package
where \textit{child} can be |chapter| or |section|
depending on the chosen structuring.
Alternatively, one can modify the macro |\thepage| appropriately
and reset the counter |page| at the start of each child file.

%%%%%%%%%%%%%%%%%%%%%%%%%%%%%%%%%%%%%%%%%%%%%%%%%%%%%%%%%%%%%%%%%%%%%%%%%%%%%%%%
\subsection{Conditional Processing}
\label{sec:conditional}

The package provides a mechanism to compile different versions
of a document. To customise the versions further some conditional processing
can come in handy to distinguish which version is being compiled.
The package provides two macros to describe the compilation context:

%%%%%%%%%%%%%%%%%%%%%%%%%%%%%%%%%%%%%%%%
\DescribeMacro{\ifchilddoc}
The conditional |\ifchilddoc| distinguishes between the compilation of
child documents and the main document:
%
\begin{center}
|\ifchilddoc |\textit{child-code}| |[|\||else |\textit{main-code}]| \||fi|
\end{center}

%%%%%%%%%%%%%%%%%%%%%%%%%%%%%%%%%%%%%%%%
\DescribeMacro{\childdocname}
\DescribeMacro{\childdocjob}
The macro |\childdocname| contains the filename (without extension)
of the main or child file being processed.
Note that |\childdocjob| will always contain the name of the main file.

%%%%%%%%%%%%%%%%%%%%%%%%%%%%%%%%%%%%%%%%
\paragraph{Title Page.}

Conditional processing can be used to include a title or banner page
in the main document when proper precautions are taken.
Importantly, the code in the main file should ensure that the page counter
(as well as other status parameters which are stored in the |.aux| files)
takes the same value after the conditional processing.
Otherwise the page numbers may take divergent values
depending on which part is compiled.

For example, a title page could be declared by:
%
\begin{center}
\begin{tabular}{l}
|\ifchilddoc\||else|\\
|\addtocounter{page}{-1}|\\
\textit{code for title page}\\
|\newpage|\\
|\||fi|
\end{tabular}
\end{center}
%
A banner page for the child documents can be generated by:
%
\begin{center}
\begin{tabular}{l}
|\ifchilddoc|\\
|\addtocounter{page}{-1}|\\
\textit{code for banner page}\\
|\newpage|\\
|\||fi|
\end{tabular}
\end{center}
%
Here one could write a message such as:
\begin{center}
|This is the part \childdocname{} of \childdocjob{}.|
\end{center}

%%%%%%%%%%%%%%%%%%%%%%%%%%%%%%%%%%%%%%%%%%%%%%%%%%%%%%%%%%%%%%%%%%%%%%%%%%%%%%%%
\subsection{Flags}
\label{sec:flags}

The package makes it easy to generate different versions
of the main or child documents.
To this end compilation flags can be defined
and assigned different default values.
They will be particularly useful in conjunction
with the forwarding mechanism described in \secref{sec:forward}.

For example, it may be useful to have a flag |\version|
which can be set to |draft| or |final|.
The document source will contain some conditional code
depending on the value of |\version|.
Suppose further, the flag should default to |final| for the main file
and to |draft| for child files
which is a natural assignment for editing the document.
This is achieved by placing the following code
in the preamble of the main document
(below the |\childdocmain| directive):
%
\begin{center}
\begin{tabular}{l}
|\ifchilddoc|\\
|\providecommand{\version}{draft}|\\
|\||else|\\
|\providecommand{\version}{final}|\\
|\||fi|
\end{tabular}
\end{center}
%
The definition by |\providecommand| makes sure
that previous definitions are not overwritten.
Further statements |\providecommand{\version}{...}|
can thus be added before the above code to override it.

For the main file, one might add a line
(between |\childdocmain| and the above block)
%
\begin{center}
|%\ifchilddoc\||else\providecommand{\version}{draft}\||fi|
\end{center}
%
which can be uncommented to produce a draft version.
Likewise one can add a line to the very top of a child file
(above the |\childdocof{|\textit{main}|}| directive)
%
\begin{center}
|%\providecommand{\version}{final}|
\end{center}
%
which can be uncommented to produce the final version of this child document.

%%%%%%%%%%%%%%%%%%%%%%%%%%%%%%%%%%%%%%%%%%%%%%%%%%%%%%%%%%%%%%%%%%%%%%%%%%%%%%%%
\subsection{Forwarding}
\label{sec:forward}

Different versions of the main or child documents
using compilation flags as described in \secref{sec:flags}
can be (permanently) stored in different files
for convenient compilation, viewing and distribution.
To this end, the package defines a command
to pass on compilation to a different file:

%%%%%%%%%%%%%%%%%%%%%%%%%%%%%%%%%%%%%%%%
\DescribeMacro{\childdocforward}
The command |\childdocforward| redirects processing to
another source file:
%
\begin{center}
\begin{tabular}{l}
|\input{childdoc.def}|\\
|\childdocforward[|\textit{main}|]{|\textit{dest}|}|\\
\end{tabular}
\end{center}
%
The argument \textit{dest} is the destination file
(without extension).
It should be the main file or one of the child files.
Note that further \textsf{childdoc} directives
such as |\childdocof| and |\childdocforward|
in the indicated file will be processed in this form.
The optional argument \textit{main}
passes on directly to the main file \textit{main}
while pretending to compile the child \textit{dest}.
This form behaves as if \textit{dest}
issues |\childdocof{|\textit{main}|}| right away,
and no further \textsf{childdoc} directives will be processed.

%%%%%%%%%%%%%%%%%%%%%%%%%%%%%%%%%%%%%%%%
\DescribeMacro{\...prefix}
In the alternative form |\childdocforwardprefix|,
%
\begin{center}
\begin{tabular}{l}
|\input{childdoc.def}|\\
|\childdocforwardprefix[|\textit{main}|]{|\textit{prefix}|}{|\textit{dest}|}|
\end{tabular}
\end{center}
%
the destination file is determined by a pattern
depending on the current file:
To make this work, the current file must be called
`{\textit{prefix}\hspace{0.2em}\textit{suffix}}'
with \textit{prefix} matching precisely the argument.
Processing is then passed on to the file
`{\textit{dest}\hspace{0.2em}\textit{suffix}}'.
Surely, the same effect is achieved by
directly specifying the
argument `{\textit{dest}\hspace{0.2em}\textit{suffix}}'
in the first form.
However, that requires to set up a different file
for each child. With the alternative form of the command
all these files can have exactly the same content
which simplifies setting them up and maintaining them.

For example, the following file |draft.tex|
with a compilation flag |\version| as described in \secref{sec:flags}
compiles the main document as a draft:
%
\begin{center}
\begin{tabular}{l}
|\def\version{draft}|\\
|\input{childdoc.def}|\\
|\childdocforward{|\textit{main}|}|
\end{tabular}
\end{center}
%
Likewise, the following files |final|\textit{nn}|.tex|
compile the final version of the child document
|child|\textit{nn}|.tex|:
%
\begin{center}
\begin{tabular}{l}
|\def\version{final}|\\
|\input{childdoc.def}|\\
|\childdocforwardprefix{final}{child}|
\end{tabular}
\end{center}
%

Note that when several versions of a main file and/or of each child file
are to be generated, it may be convenient to set up a |Makefile| or
shell script to automatise the process.

%%%%%%%%%%%%%%%%%%%%%%%%%%%%%%%%%%%%%%%%%%%%%%%%%%%%%%%%%%%%%%%%%%%%%%%%%%%%%%%%
\subsection{Command Line Processing}
\label{sec:commandline}

The effect of redirection files can also be achieved by invoking
the \LaTeX{} compiler with a more elaborate command line.
Most conveniently this should be done as part
of a shell script or a |Makefile|.

When using \textsf{childdoc} in the main file, the following
command lines effectively perform a redirection
(note that depending on the shell being used,
backslashes may have to be doubled: `|\|' $\to$ `|\\|'):
%
\begin{center}
|... -jobname "|\textit{target}|" |\\|"|[\textit{flags}]%
|\input{childdoc.def}\childdocforward[|\textit{main}|]{|\textit{dest}|}"|
\end{center}
%
Here \textit{target} is the name of the output file,
\textit{main} is the name of the main file
and \textit{dest} is the name of the main or child file to be processed
(all filenames without extensions).
The optional argument \textit{main} can be omitted
if \textit{main} matches \textit{dest}.
Optionally, compilation \textit{flags} can be defined via |\def| commands.
This command line makes the \TeX{} engine believe
it is compiling the file \textit{target}
whose content is specified as the latter parameter.
The provided code then forwards the processing to
\textit{main} or \textit{dest} as described in \secref{sec:forward}.

%%%%%%%%%%%%%%%%%%%%%%%%%%%%%%%%%%%%%%%%%%%%%%%%%%%%%%%%%%%%%%%%%%%%%%%%%%%%%%%%
\subsection{Include by Input}
\label{sec:input}

Including child documents by |\include| has some restrictions by design.
Most notably, the content of a child document always occupies
its own set of pages; pages cannot be shared between child documents.
Usually, this behaviour makes perfect sense
because each child document contain an essential part of the document.
However, in some situations it may be desirable to compose
a document from a collection of parts
without having mandatory page breaks between then.
For this case, the package
provides a mechanism to include parts
by |\input| which can also be processed individually.
However, by construction this mechanism
requires manual handling of the content to be output.

%%%%%%%%%%%%%%%%%%%%%%%%%%%%%%%%%%%%%%%%
\DescribeMacro{\ifchilddocmanual}
The main file should be prepared as usual, see \secref{sec:include}.
However, the document body must make a distinction
between processing of an individual part and of the main document, e.g.:
%
\begin{center}
\begin{tabular}{l}
|\ifchilddocmanual|\\
|\input{\childdocname}|\\
|\||else|\\
\textit{document body with }|\input{|\textit{part}|}|\\
|\||fi|
\end{tabular}
\end{center}
%
The conditional |\ifchilddocmanual| is true whenever
a part to be included by |\input| is being compiled,
and the name of the part is stored in |\childdocname|.

%%%%%%%%%%%%%%%%%%%%%%%%%%%%%%%%%%%%%%%%
\DescribeMacro{\childdocby}
Each part to be included by |\input| should start with:
%
\begin{center}
\begin{tabular}{l}
|\input{childdoc.def}|\\
|\childdocby{|\textit{main}|}|\\
\end{tabular}
\end{center}
%
The directive |\childdocby| is similar to |\childdocof|
described in \secref{sec:include},
but the subsequent selection of content must be done manually.
To that end, both |\ifchilddoc| and |\ifchilddocmanual|
will be true upon processing of a part,
and the name of the part is stored in |\childdocname|.
Note that |\jobname| will be set to the filename of the current part
so that each part receives an individual |.aux| file
that does not interfere with the |.aux| file(s) of the main document.
This behaviour can be altered by the alternative form
|\childdocby[*]{|\textit{main}|}| (with a non-empty optional argument)
which uses the |.aux| file of the main document
by setting |\jobname| to \textit{main}.

%%%%%%%%%%%%%%%%%%%%%%%%%%%%%%%%%%%%%%%%%%%%%%%%%%%%%%%%%%%%%%%%%%%%%%%%%%%%%%%%
\subsection{Driver Development}
\label{sec:driver}

The \textsf{childdoc} mechanism can also be use for the development
of definition files such as \LaTeX{} styles or classes.
This case differs from the above setup with multiple parts
included by |\include| in that no |\includeonly| should be invoked.
This can be achieved by starting the include file
(before |\ProvidesPackage|) with:
%
\begin{center}
\begin{tabular}{l}
|\input{childdoc.def}|\\
|\childdocforward{|\textit{main}|}|\\
\end{tabular}
\end{center}
%
or alternatively with:
%
\begin{center}
\begin{tabular}{l}
|\input{childdoc.def}|\\
|\childdocby{|\textit{main}|}|\\
\end{tabular}
\end{center}
%
Both forms have slightly different effects as described above.
The main file is prepared as usual, see \secref{sec:include}.

%%%%%%%%%%%%%%%%%%%%%%%%%%%%%%%%%%%%%%%%%%%%%%%%%%%%%%%%%%%%%%%%%%%%%%%%%%%%%%%%
\subsection{Legacy Detection}
\label{sec:detection}

The directive |\childdocmain| in the main file can detect
whether the complete document or merely a child is to be compiled
even without using the directive |\childdocof|.
This method is deprecated because it is less robust
and there is no compelling reason to use it;
it is merely provided for backward compatibility
and it may be removed in future versions.

If the detection mechanism is to be used,
it is mandatory to correctly specify
the filename of the main file as the argument of |\childdocmain|:
%
\begin{center}
\begin{tabular}{l}
|\input{childdoc.def}|\\
|\childdocmain{|\textit{main}|}|\\
\end{tabular}
\end{center}
%
If |\jobname| does not match the argument \textit{main} of |\childdocmain|,
it is assumed that |\jobname| points to the child file to be compiled.
When using |\childdocmain| with the main file specified as argument,
it suffices to start a child file
with just |\input{|\textit{main}|}|
without loading of the package and using |\childdocof|.
If instead all processing is done
with the appropriate \textsf{childdoc} directives,
the argument of \textit{main} of |\childdocmain| can be empty.

An alternative version of the command line processing described
in \secref{sec:commandline} using the detection mechanism reads:
%
\begin{center}
|... -jobname "|\textit{target}|" "|[\textit{flags}]%
[|\def\jobname{|\textit{dest}|}|]|\input{|\textit{main}|}"|
\end{center}

%%%%%%%%%%%%%%%%%%%%%%%%%%%%%%%%%%%%%%%%%%%%%%%%%%%%%%%%%%%%%%%%%%%%%%%%%%%%%%%%
\subsection{Manual Code}
\label{sec:manual}

In case one cannot be certain whether the definitions file |childdoc.def|
is installed on the target \TeX{} distribution
and one prefers not to ship it,
it is conceivable to paste a few relevant commands into the sources.

To that end, drop all statements |\input{childdoc.def}|
and perform the replacements as outlined below.
Instead of |\childdocmain{|\textit{main}|}| add the following code
to the top of the main file:
%
\begin{center}
\begin{tabular}{l}
|\||ifdefined\childdocname\endinput\||fi\newif\ifchilddoc|\\
|\edef\childdocname{\scantokens\expandafter{\jobname\noexpand}}|\\
|\def\childdocmain{|\textit{main}|}\||ifx\childdocmain\childdocname\||else|\\
|\childdoctrue\includeonly{\childdocname}\let\jobname\childdocmain\||fi|\\
\end{tabular}
\end{center}
%
Instead of |\childdocof{|\textit{main}|}| just include the main file
at the top of each child file:
%
\begin{center}
|\input{|\textit{main}|}|
\end{center}
%
A simple redirection |\childdocforward{|\textit{dest}|}| is achieved by:
%
\begin{center}
|\def\jobname{|\textit{dest}|}\input{\jobname}|
\end{center}
%
The redirection with prefix
|\childdocforwardprefix[|\textit{prefix}|]{|\textit{dest}|}|
is accomplished by:
%
\begin{center}
\begin{tabular}{l}
|{\edef\jobname{\scantokens\expandafter{\jobname\noexpand}}|\\
|\def\redirectjob |\textit{prefix}|#1~~~{\gdef\jobname{|\textit{dest}|#1}}|\\
|\expandafter\redirectjob\jobname~~~}\input{\jobname}|
\end{tabular}
\end{center}

In an alternative approach,
child documents can be compiled by a specific command line
without additional code or specific definitions:
%
\begin{center}
|... -jobname "|\textit{target}|" "|[\textit{flags}]%
|\includeonly{|\textit{dest}|}\input{|\textit{main}|}"|
\end{center}
%

%%%%%%%%%%%%%%%%%%%%%%%%%%%%%%%%%%%%%%%%%%%%%%%%%%%%%%%%%%%%%%%%%%%%%%%%%%%%%%%%
%%%%%%%%%%%%%%%%%%%%%%%%%%%%%%%%%%%%%%%%%%%%%%%%%%%%%%%%%%%%%%%%%%%%%%%%%%%%%%%%
\section{Information}

%%%%%%%%%%%%%%%%%%%%%%%%%%%%%%%%%%%%%%%%%%%%%%%%%%%%%%%%%%%%%%%%%%%%%%%%%%%%%%%%
\subsection{Copyright}

Copyright \copyright{} 2017--2018 Niklas Beisert

This work may be distributed and/or modified under the
conditions of the \LaTeX{} Project Public License, either version 1.3
of this license or (at your option) any later version.
The latest version of this license is in
  \url{http://www.latex-project.org/lppl.txt}
and version 1.3 or later is part of all distributions of \LaTeX{}
version 2005/12/01 or later.

This work has the LPPL maintenance status `maintained'.

The Current Maintainer of this work is Niklas Beisert.

This work consists of the files |README.txt|, |childdoc.ins| and |childdoc.dtx|
as well as the derived files |childdoc.def|, |cdocsamp.tex|
with |cdocsch1.tex|, |cdocsch2.tex|, |cdocspt3.tex|, |cdocspt4.tex|,
|cdocsdrf.tex|, |cdocsfn1.tex|, |cdocsfn2.tex|
as well as |childdoc.pdf|.

%%%%%%%%%%%%%%%%%%%%%%%%%%%%%%%%%%%%%%%%%%%%%%%%%%%%%%%%%%%%%%%%%%%%%%%%%%%%%%%%
\subsection{Files and Installation}

The package consists of the files:
%
\begin{center}
\begin{tabular}{ll}
    |README.txt|   & readme file \\
    |childdoc.ins| & installation file \\
    |childdoc.dtx| & source file \\
    |childdoc.def| & definition file \\
    |cdocsamp.tex| & sample main file \\
    |cdocsch1.tex| & sample include file \\
    |cdocsch2.tex| & sample include file \\
    |cdocspt3.tex| & sample part file \\
    |cdocspt4.tex| & sample part file \\
    |cdocsdrf.tex| & sample redirection file \\
    |cdocsfn1.tex| & sample redirection file \\
    |cdocsfn2.tex| & sample redirection file \\
    |childdoc.pdf| & manual
\end{tabular}
\end{center}
%
The distribution consists of the files
|README.txt|, |childdoc.ins| and |childdoc.dtx|.
%
\begin{itemize}
\item
Run (pdf)\LaTeX{} on |childdoc.dtx|
to compile the manual |childdoc.pdf| (this file).
\item
Run \LaTeX{} on |childdoc.ins| to create the definitions file |childdoc.def|
and the sample |cdocsamp.tex| with include files
|cdocsch1.tex|, |cdocsch2.tex|, |cdocspt3.tex|, |cdocspt4.tex|,
|cdocsdrf.tex|, |cdocsfn1.tex|, |cdocsfn2.tex|.
Then copy the file |childdoc.def| to an appropriate directory of your \LaTeX{}
distribution, e.g.\ \textit{texmf-root}|/tex/latex/childdoc|.
\end{itemize}

%%%%%%%%%%%%%%%%%%%%%%%%%%%%%%%%%%%%%%%%%%%%%%%%%%%%%%%%%%%%%%%%%%%%%%%%%%%%%%%%
\subsection{Related CTAN Packages}

There are several other packages which offer a similar functionality:
%
\begin{itemize}
\item
The packages
\href{http://ctan.org/pkg/docmute}{\textsf{docmute}},
\href{http://ctan.org/pkg/includex}{\textsf{includex}} and
\href{http://ctan.org/pkg/standalone}{\textsf{standalone}}
provide commands to include only the document body of
a child file thus allowing both files to be compiled individually.
\item
The packages \href{http://ctan.org/pkg/subdocs}{\textsf{subdocs}}
and \href{http://ctan.org/pkg/subfiles}{\textsf{subfiles}}
provide structures in which the main and child documents can be
encapsulated and allowing them to be compiled individually.
The inclusion mechanism is different from the conventional |\include|.
\item
The package \href{http://ctan.org/pkg/combine}{\textsf{combine}}
is an elaborate solution to combine several documents into one.
\end{itemize}
%
See also the CTAN topic \href{http://ctan.org/topic/subdocs}{\textsf{subdocs}}
for further related packages.
The present package differs from the above solutions in that
a document structure constructed with the conventional |\include| mechanism
just needs two extra commands at the top of every file
such that all constituent files can be compiled individually.

%%%%%%%%%%%%%%%%%%%%%%%%%%%%%%%%%%%%%%%%%%%%%%%%%%%%%%%%%%%%%%%%%%%%%%%%%%%%%%%%
%\subsection{Feature Suggestions}
%
%The following is a list of features which may be useful for future
%versions of this package:
%%
%\begin{itemize}
%\item
%\ldots
%\end{itemize}

%%%%%%%%%%%%%%%%%%%%%%%%%%%%%%%%%%%%%%%%%%%%%%%%%%%%%%%%%%%%%%%%%%%%%%%%%%%%%%%%
\subsection{Revision History}

%%%%%%%%%%%%%%%%%%%%%%%%%%%%%%%%%%%%%%%%
\paragraph{v2.0:} 2018/12/30

\begin{itemize}
\item
immediate forward processing
\item
added |\childdocby| mechanism
\item
manual restructured
\end{itemize}

%%%%%%%%%%%%%%%%%%%%%%%%%%%%%%%%%%%%%%%%
\paragraph{v1.6:} 2018/01/17

\begin{itemize}
\item
application for development of include files
\item
corrections to manual
\end{itemize}

%%%%%%%%%%%%%%%%%%%%%%%%%%%%%%%%%%%%%%%%
\paragraph{v1.5:} 2017/05/21

\begin{itemize}
\item
more complete structuring introduced
\item
|\childdocof| introduced
\item
|\childdoc| renamed to |\childdocmain|
\item
|\childredirect| renamed to |\childdocforward| and |\childdocforwardprefix|
and functionality expanded
\end{itemize}

%%%%%%%%%%%%%%%%%%%%%%%%%%%%%%%%%%%%%%%%
\paragraph{v1.0:} 2017/04/27

\begin{itemize}
\item
manual and install package
\item
first version published on CTAN
\end{itemize}

%%%%%%%%%%%%%%%%%%%%%%%%%%%%%%%%%%%%%%%%
\paragraph{v0.6:} 2017/04/26

\begin{itemize}
\item
redirection mechanism added
\end{itemize}

%%%%%%%%%%%%%%%%%%%%%%%%%%%%%%%%%%%%%%%%
\paragraph{v0.5:} 2017/04/26

\begin{itemize}
\item
functionality in definition file
\end{itemize}


%%%%%%%%%%%%%%%%%%%%%%%%%%%%%%%%%%%%%%%%%%%%%%%%%%%%%%%%%%%%%%%%%%%%%%%%%%%%%%%%
%%%%%%%%%%%%%%%%%%%%%%%%%%%%%%%%%%%%%%%%%%%%%%%%%%%%%%%%%%%%%%%%%%%%%%%%%%%%%%%%
%%%%%%%%%%%%%%%%%%%%%%%%%%%%%%%%%%%%%%%%%%%%%%%%%%%%%%%%%%%%%%%%%%%%%%%%%%%%%%%%
\appendix

\settowidth\MacroIndent{\rmfamily\scriptsize 000\ }

 \DocInput{childdoc.dtx}

\end{document}
%</driver>
% \fi
%
% %%%%%%%%%%%%%%%%%%%%%%%%%%%%%%%%%%%%%%%%%%%%%%%%%%%%%%%%%%%%%%%%%%%%%%%%%%%%%%
% %%%%%%%%%%%%%%%%%%%%%%%%%%%%%%%%%%%%%%%%%%%%%%%%%%%%%%%%%%%%%%%%%%%%%%%%%%%%%%
% \section{Sample}
%\iffalse
%<*samplemain>
%\fi
%
% The following presents a sample document
% with two chapters, two parts, a title page,
% a compile flag as well as three forwarding files to set the flag.
% It consists of eight |.tex| files:
% \begin{center}
% \begin{tabular}{ll}
% |cdocsamp.tex|&main file\\
% |cdocsch1.tex|&include file for chapter 1\\
% |cdocsch2.tex|&include file for chapter 2\\
% |cdocspt3.tex|&include file for part 3\\
% |cdocspt4.tex|&include file for part 4\\
% |cdocsdrf.tex|&forwarding file for main file in draft mode\\
% |cdocsfi1.tex|&forwarding file for final version of chapter 1\\
% |cdocsfi2.tex|&forwarding file for final version of chapter 2\\
% \end{tabular}
% \end{center}
% Each of the eight files can be compiled directly by the \LaTeX{} compiler.
%
% %%%%%%%%%%%%%%%%%%%%%%%%%%%%%%%%%%%%%%
% \paragraph{Main File.}
%
% The main file is called |cdocsamp.tex|.
%
% Load the \textsf{childdoc} definitions and
% declare the filename for the main document:
%    \begin{macrocode}
\input{childdoc.def}
\childdocmain{}
%    \end{macrocode}

% Optional override for |\version| flag:
%    \begin{macrocode}
%%\ifchilddoc\else\providecommand{\version}{draft}\fi
%    \end{macrocode}

% Define the default values for the |\version| flag
% (|final| for the main file and |draft| for childs):
%    \begin{macrocode}
\ifchilddoc
\providecommand{\version}{draft}
\else
\providecommand{\version}{final}
\fi
%    \end{macrocode}

% Load the standard document class:
%    \begin{macrocode}
\documentclass[12pt]{article}
%    \end{macrocode}

% Start the document body:
%    \begin{macrocode}
\begin{document}
%    \end{macrocode}

% Declare a title page.
% Print title, part of document being processed and version flag:
%    \begin{macrocode}
\addtocounter{page}{-1}
\begin{center}
{\LARGE\bfseries{}childdoc example\par}
\vspace{1cm}
\ifchilddoc
\ifchilddocmanual part\else chapter\fi:
`\childdocname' of `\childdocjob'\par
\else
main document: `\childdocjob'\par
\fi
version: \version\par
\end{center}
\newpage
%    \end{macrocode}

% Manually include selected file,
% otherwise process as usual:
%    \begin{macrocode}
\ifchilddocmanual
\section*{part `\childdocname'}
\input{\childdocname}
\else
%    \end{macrocode}

% Include the two chapters:
%    \begin{macrocode}
\include{cdocsch1}
\include{cdocsch2}
%    \end{macrocode}

% Include the two parts unless only chapters should be displayed:
%    \begin{macrocode}
\ifchilddoc\else
\section{part three}
\input{cdocspt3}
\section{part four}
\input{cdocspt4}
\fi
%    \end{macrocode}

% Process as usual until here:
%    \begin{macrocode}
\fi
%    \end{macrocode}

% End of document body:
%    \begin{macrocode}
\end{document}
%    \end{macrocode}
%\iffalse
%</samplemain>
%\fi
%
% %%%%%%%%%%%%%%%%%%%%%%%%%%%%%%%%%%%%%%
% \paragraph{Chapter Include Files.}
%
% The include files are called |cdocsch1.tex| and |cdocsch2.tex|.
%
%\iffalse
%<*samplechap1|samplechap2>
%\fi

% Optional override for |\version| flag:
%    \begin{macrocode}
%%\providecommand{\version}{final}
%    \end{macrocode}

% Include the main document:
%    \begin{macrocode}
\input{childdoc.def}
\childdocof{cdocsamp}
%    \end{macrocode}

%\iffalse
%</samplechap1|samplechap2>
%\fi
%
%\iffalse
%<*samplechap1>
%\fi
% Some text for chapter 1:
%    \begin{macrocode}
\section{one}
some text in chapter one
%    \end{macrocode}

%\iffalse
%</samplechap1>
%\fi
% Some text for chapter 2:
%\iffalse
%<*samplechap2>
%\fi
%    \begin{macrocode}
\section{two}
more text in chapter two
%    \end{macrocode}

%\iffalse
%</samplechap2>
%\fi
%
% %%%%%%%%%%%%%%%%%%%%%%%%%%%%%%%%%%%%%%
% \paragraph{Part Include Files.}
%
% The include files are called |cdocspt3.tex| and |cdocspt4.tex|.
%
%\iffalse
%<*samplepart3|samplepart4>
%\fi

% Optional override for |\version| flag:
%    \begin{macrocode}
%%\providecommand{\version}{final}
%    \end{macrocode}

% Include the main document:
%    \begin{macrocode}
\input{childdoc.def}
\childdocby{cdocsamp}
%    \end{macrocode}

%\iffalse
%</samplepart3|samplepart4>
%\fi
%
%\iffalse
%<*samplepart3>
%\fi
% Some text for part 3:
%    \begin{macrocode}
some text in part three
%    \end{macrocode}

%\iffalse
%</samplepart3>
%\fi
% Some text for part 4:
%\iffalse
%<*samplepart4>
%\fi
%    \begin{macrocode}
more text in part four
%    \end{macrocode}

%\iffalse
%</samplepart4>
%\fi
%
% %%%%%%%%%%%%%%%%%%%%%%%%%%%%%%%%%%%%%%
% \paragraph{Forwarding for a Complete Draft.}
%
% The following forwarding file |cdocsdrf.tex|
% compiles the main document in draft mode:
%\iffalse
%<*sampledraft>
%\fi
%    \begin{macrocode}
\def\version{draft}
\input{childdoc.def}
\childdocforward{cdocsamp}
%    \end{macrocode}

%\iffalse
%</sampledraft>
%\fi
%
% %%%%%%%%%%%%%%%%%%%%%%%%%%%%%%%%%%%%%%
% \paragraph{Forwarding for Final Version of the Chapters.}
%
% The following forwarding files |cdocsfn1.tex| and |cdocsfn2.tex|
% (with identical content)
% compile the final versions of the child documents
% |cdocsch1.tex| and |cdocsch2.tex|, respectively:
%\iffalse
%<*samplefinal>
%\fi
%    \begin{macrocode}
\def\version{final}
\input{childdoc.def}
\childdocforwardprefix[cdocsamp]{cdocsfn}{cdocsch}
%    \end{macrocode}

%\iffalse
%</samplefinal>
%\fi
%
% %%%%%%%%%%%%%%%%%%%%%%%%%%%%%%%%%%%%%%
% \paragraph{Command Line Processing.}
%
% The following three command lines generate the output files
% |cdocscld|, |cdocscl1| and |cdocscl2|
% which should be identical to
% |cdocsdrf|, |cdocsch1| and |cdocsfn2|, respectively:
% \begin{center}
% \begin{tabular}{l}
% |latex -jobname cdocscld \|\\
% |  "\def\version{draft}\input{childdoc.def}\childdocforward{cdocsamp}"|\\
% |latex -jobname cdocscl1 \|\\
% |  "\input{childdoc.def}\childdocforward[cdocsamp]{cdocsch1}"|\\
% |latex -jobname cdocscl2 \|\\
% |  "\def\version{final}\input{childdoc.def}\childdocforward{cdocsch2}"|
% \end{tabular}
% \end{center}
% Note that the trailing backslash on each first line
% merely continues the input to the second line
% (for convenient cut ant paste).
% Furthermore, the command |latex| can be replaced by any
% of its alternative versions such as |pdflatex|.
%
% %%%%%%%%%%%%%%%%%%%%%%%%%%%%%%%%%%%%%%%%%%%%%%%%%%%%%%%%%%%%%%%%%%%%%%%%%%%%%%
% %%%%%%%%%%%%%%%%%%%%%%%%%%%%%%%%%%%%%%%%%%%%%%%%%%%%%%%%%%%%%%%%%%%%%%%%%%%%%%
% \section{Implementation}
%\iffalse
%<*package>
%\fi
%
% This section describes the definitions file |childdoc.def|.

% The definitions cannot be loaded using |\usepackage| or |\RequirePackage|
% which has a mechanism to prevent loading a style file more than once.
% When loading the definitions by means of |\input|
% multiple instances have to be prevented manually:
%\iffalse
%This code needs to be before the `\ProvidesFile' directive
%which is defined at the beginning of this file.
%Therefore it is also placed there and commented out here.
%</package>
%<*discard>
%\fi
%    \begin{macrocode}
\ifdefined\childdocmain\endinput\fi
%    \end{macrocode}
%\iffalse
%</discard>
%<*package>
%\fi
%
% \macro{\ifchilddoc}
% \macro{\ifchilddocmanual}
% The conditional |\ifchilddoc| tells whether a
% child (true) or main (false) document is being compiled.
% The conditional |\ifchilddocmanual| tells whether
% the |\includeonly| mechanism is used (false) or
% the selection of child files must be performed manually (true).
% The definitions initialise to false:
%    \begin{macrocode}
\newif\ifchilddoc
\newif\ifchilddocmanual
%    \end{macrocode}

% \macro{\childdocname}
% \macro{\childdocjob}
% The macro |\childdocname| stores the name of the main document
% to be compiled. The macro |\childdocjob| stores the name of
% the document on which the \LaTeX{} compiler was originally invoked.
% The content of |\jobname| cannot be compared
% to filenames specified in the source due to different catcodes.
% The following code rescans |\jobname|, stores the result
% in |\childdocname| and saves a copy in |\childdocjob|:
%    \begin{macrocode}
\edef\childdocname{\scantokens\expandafter{\jobname\noexpand}}
\let\childdocjob\childdocname
%    \end{macrocode}

% \macro{\childdocdisable}
% The macro |\childdocdisable| prevents the main file
% from being processed more than once.
% At this stage, the main document command |\childdocmain|
% is assumed to be called once again where it should do nothing.
% Any subsequent call to it should prevent
% a secondary processing of the main document
% It overwrites the forwarding commands
% |\childdocof| and |\childdocforward|
% with empty macros to prevent further inclusions of the main document:
%    \begin{macrocode}
\newcommand{\childdocdisable}
{
  \renewcommand{\childdocmain}[1]{\renewcommand{\childdocmain}[1]{\endinput}}
  \renewcommand{\childdocof}[1]{}
  \renewcommand{\childdocby}[2][]{}
  \renewcommand{\childdocforward}[2][]{}
  \renewcommand{\childdocdisable}{}
}
%    \end{macrocode}

% \macro{\childdocmain}
% The macro |\childdocmain| is to be called at the top of the main file
% with nothing or the main filename (without extension) as argument.
% First, it breaks loops.
% If the argument is not empty and does not match |\childdocname|
% (which is set by the first inclusion of |childdoc.def|),
% |\ifchilddoc| is set to true, |\includeonly| is applied to the child file
% and |\jobname| is set to the main file
% (for proper handling of |.aux| files):
%    \begin{macrocode}
\newcommand{\childdocmain}[1]
{
  \childdocdisable\childdocmain{}
  \if?#1?\else
    \begingroup
      \def\childdoctmp{#1}
      \ifx\childdoctmp\childdocname
        \def\childdoctmp{}
      \else
        \def\childdoctmp
        {
          \childdoctrue
          \includeonly{\childdocname}
          \def\childdocjob{#1}
          \def\jobname{#1}
        }
      \fi
      \expandafter
    \endgroup
    \childdoctmp
  \fi
}
%    \end{macrocode}

% \macro{\childdocof}
% The command |\childdocof| redirects
% compilation to the main file |#1|.
%    \begin{macrocode}
\newcommand{\childdocof}[1]
{
  \childdocdisable
  \childdoctrue
  \includeonly{\childdocname}
  \def\jobname{#1}
  \def\childdocjob{#1}
  \input{#1}
}
%    \end{macrocode}

% \macro{\childdocby}
% The command |\childdocby| ....
%    \begin{macrocode}
\newcommand{\childdocby}[2][]
{
  \childdocdisable
  \childdoctrue
  \childdocmanualtrue
  \if?#1?\else
    \def\jobname{#2}
  \fi
  \def\childdocjob{#2}
  \input{#2}
  \endinput
}
%    \end{macrocode}

% \macro{\childdocforward}
% The command |\childdocforward| redirects
% compilation to the main file or
% (if the optional argument is given) a child file.
% Parameters are set as if the main file
% or a child file starting with |\childdocof| was compiled.
% Then compilation is handed over to the main file:
%    \begin{macrocode}
\newcommand{\childdocforward}[2][]
{
  \begingroup
    \if?#1?
      \def\childdoctmp
      {
        \def\childdocname{#2}
        \def\childdocjob{#2}
        \def\jobname{#2}
        \input{#2}
        \endinput
      }
    \else
      \def\childdoctmp
      {
        \childdocdisable
        \def\childdocname{#2}
        \childdoctrue
        \includeonly{#2}
        \def\childdocjob{#1}
        \def\jobname{#1}
        \input{#1}
        \endinput
      }
    \fi
    \expandafter
  \endgroup
  \childdoctmp
}
%    \end{macrocode}

% \macro{\childdocforwardprefix}
% The command |\childdocforwardprefix| redirects
% compilation to the main or a child file by means of a pattern.
% The prefix |#1| in the current filename is replaced by |#2|
% and the suffix of the current filename is kept
% (it is assumed that the filename does not contain the substring `|~~~|'
% which is used as a delimiter).
% Compilation is handed over to the new file by |\childdocforward|:
%    \begin{macrocode}
\newcommand{\childdocforwardprefix}[3][]
{
  \begingroup
    \def\childdocextract #2##1~~~{\def\childdoctmp{\childdocforward[#1]{#3##1}}}
    \expandafter\childdocextract\childdocname~~~
    \expandafter
  \endgroup
  \childdoctmp
}
%    \end{macrocode}

% \macro{\childdoc}
% The deprecated macro |\childdoc| is a legacy version of |\childdocmain|:
%    \begin{macrocode}
\newcommand{\childdoc}{\childdocmain}
%    \end{macrocode}

% \macro{\childdocredirect}
% The deprecated macro |\childdocredirect| is a legacy version
% of |\childdocforward| and |\childdocforwardprefix|:
%    \begin{macrocode}
\newcommand{\childdocredirect}[2][]
{
  \begingroup
    \if?#1?
      \def\childdoctmp{\childdocforward{#2}}
    \else
      \def\childdoctmp{\childdocforwardprefix{#1}{#2}}
    \fi
    \expandafter
  \endgroup
  \childdoctmp
}
%    \end{macrocode}

%\iffalse
%</package>
%\fi
%
\endinput

\childdocof{cdocsamp}
%    \end{macrocode}

%\iffalse
%</samplechap1|samplechap2>
%\fi
%
%\iffalse
%<*samplechap1>
%\fi
% Some text for chapter 1:
%    \begin{macrocode}
\section{one}
some text in chapter one
%    \end{macrocode}

%\iffalse
%</samplechap1>
%\fi
% Some text for chapter 2:
%\iffalse
%<*samplechap2>
%\fi
%    \begin{macrocode}
\section{two}
more text in chapter two
%    \end{macrocode}

%\iffalse
%</samplechap2>
%\fi
%
% %%%%%%%%%%%%%%%%%%%%%%%%%%%%%%%%%%%%%%
% \paragraph{Part Include Files.}
%
% The include files are called |cdocspt3.tex| and |cdocspt4.tex|.
%
%\iffalse
%<*samplepart3|samplepart4>
%\fi

% Optional override for |\version| flag:
%    \begin{macrocode}
%%\providecommand{\version}{final}
%    \end{macrocode}

% Include the main document:
%    \begin{macrocode}
% \iffalse
%
% childdoc.dtx Copyright (C) 2017-2018 Niklas Beisert
%
% This work may be distributed and/or modified under the
% conditions of the LaTeX Project Public License, either version 1.3
% of this license or (at your option) any later version.
% The latest version of this license is in
%   http://www.latex-project.org/lppl.txt
% and version 1.3 or later is part of all distributions of LaTeX
% version 2005/12/01 or later.
%
% This work has the LPPL maintenance status `maintained'.
%
% The Current Maintainer of this work is Niklas Beisert.
%
% This work consists of the files childdoc.dtx and childdoc.ins
% and the derived files childdoc.def and cdocsamp.tex with
% cdocsch1.tex, cdocsch2.tex, cdocsdrf.tex, cdocsfn1.tex, cdocsfn2.tex.
%
%<package>\ifdefined\childdocmain\endinput\fi
%<package>\ProvidesFile{childdoc.def}[2018/12/30 v2.0 child document driver]
%<samplemain>\ProvidesFile{cdocsamp.tex}[2018/12/30 v2.0 sample for childdoc]
%<*driver>
%\ProvidesFile{childdoc.drv}[2018/12/30 v2.0 childdoc reference manual file]
\PassOptionsToClass{10pt,a4paper}{article}
\documentclass{ltxdoc}

\usepackage[margin=35mm]{geometry}
\usepackage{hyperref}
\usepackage{hyperxmp}
\usepackage[usenames]{color}

\hypersetup{colorlinks=true}
\hypersetup{pdfstartview=FitH}
\hypersetup{pdfpagemode=UseNone}
\hypersetup{pdfsource={}}
\hypersetup{pdflang={en-UK}}
\hypersetup{pdfcopyright={Copyright 2017-2018 Niklas Beisert.
  This work may be distributed and/or modified under the
  conditions of the LaTeX Project Public License, either version 1.3
  of this license or (at your option) any later version.}}
\hypersetup{pdflicenseurl={http://www.latex-project.org/lppl.txt}}
\hypersetup{pdfcontactaddress={ETH Zurich, ITP, HIT K,
  Wolfgang-Pauli-Strasse 27}}
\hypersetup{pdfcontactpostcode={8093}}
\hypersetup{pdfcontactcity={Zurich}}
\hypersetup{pdfcontactcountry={Switzerland}}
\hypersetup{pdfcontactemail={nbeisert@itp.phys.ethz.ch}}
\hypersetup{pdfcontacturl={http://people.phys.ethz.ch/\xmptilde nbeisert/}}

\newcommand{\secref}[1]{\hyperref[#1]{section \ref*{#1}}}

\parskip1ex
\parindent0pt
\let\olditemize\itemize
\def\itemize{\olditemize\parskip0pt}

\begin{document}

\title{The \textsf{childdoc} Package}
\hypersetup{pdftitle={The childdoc Package}}
\author{Niklas Beisert\\[2ex]
  Institut f\"ur Theoretische Physik\\
  Eidgen\"ossische Technische Hochschule Z\"urich\\
  Wolfgang-Pauli-Strasse 27, 8093 Z\"urich, Switzerland\\[1ex]
  \href{mailto:nbeisert@itp.phys.ethz.ch}
  {\texttt{nbeisert@itp.phys.ethz.ch}}}
\hypersetup{pdfauthor={Niklas Beisert}}
\hypersetup{pdfsubject={Manual for the LaTeX2e Package childdoc}}
\date{30 December 2018, \textsf{v2.0}}
\maketitle

\begin{abstract}\noindent
\textsf{childdoc} is a \LaTeXe{} package
that enables the direct compilation
of document sections included by |\include|
to individual files.
\end{abstract}

\begingroup
\parskip0ex
\tableofcontents
\endgroup

%%%%%%%%%%%%%%%%%%%%%%%%%%%%%%%%%%%%%%%%%%%%%%%%%%%%%%%%%%%%%%%%%%%%%%%%%%%%%%%%
%%%%%%%%%%%%%%%%%%%%%%%%%%%%%%%%%%%%%%%%%%%%%%%%%%%%%%%%%%%%%%%%%%%%%%%%%%%%%%%%
\section{Introduction}

\LaTeX{} provides a mechanism to structure a large document (such as a book)
into a main file and several child files (containing the chapters)
using the |\include| command.
This mechanism is beneficial for documents
which span hundreds of pages in order to
make the source file(s) more manageable.
Moreover, compilation can be restricted to
selected child files by means of the |\includeonly| command.
The latter feature can be used to reduce the compilation time while editing
(this was significantly more useful in the earlier days of \LaTeX{})
or to generate a smaller document which is easier to navigate.
Another application of |\includeonly| is to generate
documents consisting of selected parts of the complete document.

However, there are a few drawbacks of the plain |\include| mechanism:
\begin{itemize}
\item
The child files cannot be compiled on their own,
they can only be compiled via the main file.
A naive editing environment
(such as a text editor with an option
to have the current file processed by \LaTeX)
may require one to switch to the main file before compiling;
attempting to compile the child file produces errors.
\item
The main file must be modified (each time)
to adjust the |\includeonly| command
to the present needs. This easily leaves the main file in a messy state.
\item
The generated document will always carry the filename
of the main document. This is inconvenient if
several child files are to be compiled and
to be kept for distribution.
\end{itemize}

The present package provides a simple interface
to make child files individually compilable by \LaTeX{}.
Compiling a child file then has the same effect as compiling
the main file with an |\includeonly| command
to select the appropriate child.
Moreover the generated document will carry the name of the child
rather than the main file.
This resolves all three above issues.

This feature is meant to make the editing of books,
thesis documents and lecture notes somewhat more convenient.
However, the package can also be used efficiently for
composing a series of documents (such as exercise sheets)
which are typically distributed individually.
It then assists the author in generating the individual documents
(potentially in different versions)
as well as a document containing the collected series.
Another application is in developing style files
or other kinds of included material
where compilation of the style file could redirect
to a sample or test file.

%%%%%%%%%%%%%%%%%%%%%%%%%%%%%%%%%%%%%%%%%%%%%%%%%%%%%%%%%%%%%%%%%%%%%%%%%%%%%%%%
%%%%%%%%%%%%%%%%%%%%%%%%%%%%%%%%%%%%%%%%%%%%%%%%%%%%%%%%%%%%%%%%%%%%%%%%%%%%%%%%
\section{Usage}

First of all, the package \textsf{childdoc} is \emph{not} a standard
\LaTeXe{} |.sty| style file! Therefore it needs to be invoked in
a non-standard way.

%%%%%%%%%%%%%%%%%%%%%%%%%%%%%%%%%%%%%%%%%%%%%%%%%%%%%%%%%%%%%%%%%%%%%%%%%%%%%%%%
\subsection{Included Files}
\label{sec:include}

%%%%%%%%%%%%%%%%%%%%%%%%%%%%%%%%%%%%%%%%
\DescribeMacro{\childdocmain}
To use the package, add the commands
\begin{center}
\begin{tabular}{l}
|\input{childdoc.def}|\\
|\childdocmain{}|\\
\end{tabular}
\end{center}
at the very top of the main \LaTeX{} file,
in particular \emph{before} the |\documentclass| statement!
The argument of |\childdocmain| should be left empty
(but it must be present).

%%%%%%%%%%%%%%%%%%%%%%%%%%%%%%%%%%%%%%%%
\DescribeMacro{\childdocof}
Furthermore, add the commands
\begin{center}
\begin{tabular}{l}
|\input{childdoc.def}|\\
|\childdocof{|\textit{main}|}|\\
\end{tabular}
\end{center}
at the top of every child file \textit{child}
which is included by |\include{|\textit{child}|}|
from within the main file
(or at least for those files to be compiled individually).
The argument \textit{main} must be the filename of the main file.

There are a couple of
considerations in setting up the main and child documents:

%%%%%%%%%%%%%%%%%%%%%%%%%%%%%%%%%%%%%%%%
\paragraph{Restrictions.}

Please note the following restrictions:
\begin{itemize}
\item
|\childdocmain| must be called with one argument \textit{main}
to ensure compatibility with earlier version of the package.
It must either be empty (|\childdocmain{}|)
or precisely match the filename of the main file in which it is specified.
See \secref{sec:detection} for further information.
\item
The filename \textit{main} must be specified without the |.tex| extension.
\item
The filename \textit{main} is case sensitive
(even in case-insensitive file systems)
due to internal string comparison.
\item
The argument \textit{main} should be fully expanded, it cannot be a macro.
\item
Subdirectories and special characters should be avoided in filenames.
\item
The command |\childdocmain{|\textit{main}|}| must be followed by a whitespace.
It should not be followed immediately by another command
or by a comment mark `|%|'.
This is because the \TeX{} parser reads the token immediately following
the argument of |\childdocmain| and puts it
at the beginning of every child section;
however, a white\-space is ignored.
\end{itemize}

%%%%%%%%%%%%%%%%%%%%%%%%%%%%%%%%%%%%%%%%
\paragraph{Content of Main File.}

It is advisable to place all content in the child files included by |\include|.
Any output contained in the main file will appear in all child documents
unless suppressed manually;
it cannot be suppressed automatically by the |\includeonly| directive
and thus should normally be avoided.
A method to include some content in the main file
by means of conditional processing is described in \secref{sec:conditional}.

%%%%%%%%%%%%%%%%%%%%%%%%%%%%%%%%%%%%%%%%
\paragraph{Page Numbering.}

When only a part of the document is compiled,
the appropriate numbering of pages
(as well as other status parameters)
is determined from the |.aux| files.
The latter contain information from previous passes.
However this information needs to propagate through
all intermediate child documents.
Therefore the page numbering in child documents may well
be inconsistent until the complete document is compiled at least once.

A useful (if unconventional) way to always ensure a consistent
page numbering is to restart the numbering in each child document
and denote the pages by `\textit{child}|.|\textit{page}'
where \textit{child} represents the chapter/section number of the child file.
This can be achieved by the command
|\numberwithin{page}{|\textit{child}|}|
of the \textsf{amsmath} package
where \textit{child} can be |chapter| or |section|
depending on the chosen structuring.
Alternatively, one can modify the macro |\thepage| appropriately
and reset the counter |page| at the start of each child file.

%%%%%%%%%%%%%%%%%%%%%%%%%%%%%%%%%%%%%%%%%%%%%%%%%%%%%%%%%%%%%%%%%%%%%%%%%%%%%%%%
\subsection{Conditional Processing}
\label{sec:conditional}

The package provides a mechanism to compile different versions
of a document. To customise the versions further some conditional processing
can come in handy to distinguish which version is being compiled.
The package provides two macros to describe the compilation context:

%%%%%%%%%%%%%%%%%%%%%%%%%%%%%%%%%%%%%%%%
\DescribeMacro{\ifchilddoc}
The conditional |\ifchilddoc| distinguishes between the compilation of
child documents and the main document:
%
\begin{center}
|\ifchilddoc |\textit{child-code}| |[|\||else |\textit{main-code}]| \||fi|
\end{center}

%%%%%%%%%%%%%%%%%%%%%%%%%%%%%%%%%%%%%%%%
\DescribeMacro{\childdocname}
\DescribeMacro{\childdocjob}
The macro |\childdocname| contains the filename (without extension)
of the main or child file being processed.
Note that |\childdocjob| will always contain the name of the main file.

%%%%%%%%%%%%%%%%%%%%%%%%%%%%%%%%%%%%%%%%
\paragraph{Title Page.}

Conditional processing can be used to include a title or banner page
in the main document when proper precautions are taken.
Importantly, the code in the main file should ensure that the page counter
(as well as other status parameters which are stored in the |.aux| files)
takes the same value after the conditional processing.
Otherwise the page numbers may take divergent values
depending on which part is compiled.

For example, a title page could be declared by:
%
\begin{center}
\begin{tabular}{l}
|\ifchilddoc\||else|\\
|\addtocounter{page}{-1}|\\
\textit{code for title page}\\
|\newpage|\\
|\||fi|
\end{tabular}
\end{center}
%
A banner page for the child documents can be generated by:
%
\begin{center}
\begin{tabular}{l}
|\ifchilddoc|\\
|\addtocounter{page}{-1}|\\
\textit{code for banner page}\\
|\newpage|\\
|\||fi|
\end{tabular}
\end{center}
%
Here one could write a message such as:
\begin{center}
|This is the part \childdocname{} of \childdocjob{}.|
\end{center}

%%%%%%%%%%%%%%%%%%%%%%%%%%%%%%%%%%%%%%%%%%%%%%%%%%%%%%%%%%%%%%%%%%%%%%%%%%%%%%%%
\subsection{Flags}
\label{sec:flags}

The package makes it easy to generate different versions
of the main or child documents.
To this end compilation flags can be defined
and assigned different default values.
They will be particularly useful in conjunction
with the forwarding mechanism described in \secref{sec:forward}.

For example, it may be useful to have a flag |\version|
which can be set to |draft| or |final|.
The document source will contain some conditional code
depending on the value of |\version|.
Suppose further, the flag should default to |final| for the main file
and to |draft| for child files
which is a natural assignment for editing the document.
This is achieved by placing the following code
in the preamble of the main document
(below the |\childdocmain| directive):
%
\begin{center}
\begin{tabular}{l}
|\ifchilddoc|\\
|\providecommand{\version}{draft}|\\
|\||else|\\
|\providecommand{\version}{final}|\\
|\||fi|
\end{tabular}
\end{center}
%
The definition by |\providecommand| makes sure
that previous definitions are not overwritten.
Further statements |\providecommand{\version}{...}|
can thus be added before the above code to override it.

For the main file, one might add a line
(between |\childdocmain| and the above block)
%
\begin{center}
|%\ifchilddoc\||else\providecommand{\version}{draft}\||fi|
\end{center}
%
which can be uncommented to produce a draft version.
Likewise one can add a line to the very top of a child file
(above the |\childdocof{|\textit{main}|}| directive)
%
\begin{center}
|%\providecommand{\version}{final}|
\end{center}
%
which can be uncommented to produce the final version of this child document.

%%%%%%%%%%%%%%%%%%%%%%%%%%%%%%%%%%%%%%%%%%%%%%%%%%%%%%%%%%%%%%%%%%%%%%%%%%%%%%%%
\subsection{Forwarding}
\label{sec:forward}

Different versions of the main or child documents
using compilation flags as described in \secref{sec:flags}
can be (permanently) stored in different files
for convenient compilation, viewing and distribution.
To this end, the package defines a command
to pass on compilation to a different file:

%%%%%%%%%%%%%%%%%%%%%%%%%%%%%%%%%%%%%%%%
\DescribeMacro{\childdocforward}
The command |\childdocforward| redirects processing to
another source file:
%
\begin{center}
\begin{tabular}{l}
|\input{childdoc.def}|\\
|\childdocforward[|\textit{main}|]{|\textit{dest}|}|\\
\end{tabular}
\end{center}
%
The argument \textit{dest} is the destination file
(without extension).
It should be the main file or one of the child files.
Note that further \textsf{childdoc} directives
such as |\childdocof| and |\childdocforward|
in the indicated file will be processed in this form.
The optional argument \textit{main}
passes on directly to the main file \textit{main}
while pretending to compile the child \textit{dest}.
This form behaves as if \textit{dest}
issues |\childdocof{|\textit{main}|}| right away,
and no further \textsf{childdoc} directives will be processed.

%%%%%%%%%%%%%%%%%%%%%%%%%%%%%%%%%%%%%%%%
\DescribeMacro{\...prefix}
In the alternative form |\childdocforwardprefix|,
%
\begin{center}
\begin{tabular}{l}
|\input{childdoc.def}|\\
|\childdocforwardprefix[|\textit{main}|]{|\textit{prefix}|}{|\textit{dest}|}|
\end{tabular}
\end{center}
%
the destination file is determined by a pattern
depending on the current file:
To make this work, the current file must be called
`{\textit{prefix}\hspace{0.2em}\textit{suffix}}'
with \textit{prefix} matching precisely the argument.
Processing is then passed on to the file
`{\textit{dest}\hspace{0.2em}\textit{suffix}}'.
Surely, the same effect is achieved by
directly specifying the
argument `{\textit{dest}\hspace{0.2em}\textit{suffix}}'
in the first form.
However, that requires to set up a different file
for each child. With the alternative form of the command
all these files can have exactly the same content
which simplifies setting them up and maintaining them.

For example, the following file |draft.tex|
with a compilation flag |\version| as described in \secref{sec:flags}
compiles the main document as a draft:
%
\begin{center}
\begin{tabular}{l}
|\def\version{draft}|\\
|\input{childdoc.def}|\\
|\childdocforward{|\textit{main}|}|
\end{tabular}
\end{center}
%
Likewise, the following files |final|\textit{nn}|.tex|
compile the final version of the child document
|child|\textit{nn}|.tex|:
%
\begin{center}
\begin{tabular}{l}
|\def\version{final}|\\
|\input{childdoc.def}|\\
|\childdocforwardprefix{final}{child}|
\end{tabular}
\end{center}
%

Note that when several versions of a main file and/or of each child file
are to be generated, it may be convenient to set up a |Makefile| or
shell script to automatise the process.

%%%%%%%%%%%%%%%%%%%%%%%%%%%%%%%%%%%%%%%%%%%%%%%%%%%%%%%%%%%%%%%%%%%%%%%%%%%%%%%%
\subsection{Command Line Processing}
\label{sec:commandline}

The effect of redirection files can also be achieved by invoking
the \LaTeX{} compiler with a more elaborate command line.
Most conveniently this should be done as part
of a shell script or a |Makefile|.

When using \textsf{childdoc} in the main file, the following
command lines effectively perform a redirection
(note that depending on the shell being used,
backslashes may have to be doubled: `|\|' $\to$ `|\\|'):
%
\begin{center}
|... -jobname "|\textit{target}|" |\\|"|[\textit{flags}]%
|\input{childdoc.def}\childdocforward[|\textit{main}|]{|\textit{dest}|}"|
\end{center}
%
Here \textit{target} is the name of the output file,
\textit{main} is the name of the main file
and \textit{dest} is the name of the main or child file to be processed
(all filenames without extensions).
The optional argument \textit{main} can be omitted
if \textit{main} matches \textit{dest}.
Optionally, compilation \textit{flags} can be defined via |\def| commands.
This command line makes the \TeX{} engine believe
it is compiling the file \textit{target}
whose content is specified as the latter parameter.
The provided code then forwards the processing to
\textit{main} or \textit{dest} as described in \secref{sec:forward}.

%%%%%%%%%%%%%%%%%%%%%%%%%%%%%%%%%%%%%%%%%%%%%%%%%%%%%%%%%%%%%%%%%%%%%%%%%%%%%%%%
\subsection{Include by Input}
\label{sec:input}

Including child documents by |\include| has some restrictions by design.
Most notably, the content of a child document always occupies
its own set of pages; pages cannot be shared between child documents.
Usually, this behaviour makes perfect sense
because each child document contain an essential part of the document.
However, in some situations it may be desirable to compose
a document from a collection of parts
without having mandatory page breaks between then.
For this case, the package
provides a mechanism to include parts
by |\input| which can also be processed individually.
However, by construction this mechanism
requires manual handling of the content to be output.

%%%%%%%%%%%%%%%%%%%%%%%%%%%%%%%%%%%%%%%%
\DescribeMacro{\ifchilddocmanual}
The main file should be prepared as usual, see \secref{sec:include}.
However, the document body must make a distinction
between processing of an individual part and of the main document, e.g.:
%
\begin{center}
\begin{tabular}{l}
|\ifchilddocmanual|\\
|\input{\childdocname}|\\
|\||else|\\
\textit{document body with }|\input{|\textit{part}|}|\\
|\||fi|
\end{tabular}
\end{center}
%
The conditional |\ifchilddocmanual| is true whenever
a part to be included by |\input| is being compiled,
and the name of the part is stored in |\childdocname|.

%%%%%%%%%%%%%%%%%%%%%%%%%%%%%%%%%%%%%%%%
\DescribeMacro{\childdocby}
Each part to be included by |\input| should start with:
%
\begin{center}
\begin{tabular}{l}
|\input{childdoc.def}|\\
|\childdocby{|\textit{main}|}|\\
\end{tabular}
\end{center}
%
The directive |\childdocby| is similar to |\childdocof|
described in \secref{sec:include},
but the subsequent selection of content must be done manually.
To that end, both |\ifchilddoc| and |\ifchilddocmanual|
will be true upon processing of a part,
and the name of the part is stored in |\childdocname|.
Note that |\jobname| will be set to the filename of the current part
so that each part receives an individual |.aux| file
that does not interfere with the |.aux| file(s) of the main document.
This behaviour can be altered by the alternative form
|\childdocby[*]{|\textit{main}|}| (with a non-empty optional argument)
which uses the |.aux| file of the main document
by setting |\jobname| to \textit{main}.

%%%%%%%%%%%%%%%%%%%%%%%%%%%%%%%%%%%%%%%%%%%%%%%%%%%%%%%%%%%%%%%%%%%%%%%%%%%%%%%%
\subsection{Driver Development}
\label{sec:driver}

The \textsf{childdoc} mechanism can also be use for the development
of definition files such as \LaTeX{} styles or classes.
This case differs from the above setup with multiple parts
included by |\include| in that no |\includeonly| should be invoked.
This can be achieved by starting the include file
(before |\ProvidesPackage|) with:
%
\begin{center}
\begin{tabular}{l}
|\input{childdoc.def}|\\
|\childdocforward{|\textit{main}|}|\\
\end{tabular}
\end{center}
%
or alternatively with:
%
\begin{center}
\begin{tabular}{l}
|\input{childdoc.def}|\\
|\childdocby{|\textit{main}|}|\\
\end{tabular}
\end{center}
%
Both forms have slightly different effects as described above.
The main file is prepared as usual, see \secref{sec:include}.

%%%%%%%%%%%%%%%%%%%%%%%%%%%%%%%%%%%%%%%%%%%%%%%%%%%%%%%%%%%%%%%%%%%%%%%%%%%%%%%%
\subsection{Legacy Detection}
\label{sec:detection}

The directive |\childdocmain| in the main file can detect
whether the complete document or merely a child is to be compiled
even without using the directive |\childdocof|.
This method is deprecated because it is less robust
and there is no compelling reason to use it;
it is merely provided for backward compatibility
and it may be removed in future versions.

If the detection mechanism is to be used,
it is mandatory to correctly specify
the filename of the main file as the argument of |\childdocmain|:
%
\begin{center}
\begin{tabular}{l}
|\input{childdoc.def}|\\
|\childdocmain{|\textit{main}|}|\\
\end{tabular}
\end{center}
%
If |\jobname| does not match the argument \textit{main} of |\childdocmain|,
it is assumed that |\jobname| points to the child file to be compiled.
When using |\childdocmain| with the main file specified as argument,
it suffices to start a child file
with just |\input{|\textit{main}|}|
without loading of the package and using |\childdocof|.
If instead all processing is done
with the appropriate \textsf{childdoc} directives,
the argument of \textit{main} of |\childdocmain| can be empty.

An alternative version of the command line processing described
in \secref{sec:commandline} using the detection mechanism reads:
%
\begin{center}
|... -jobname "|\textit{target}|" "|[\textit{flags}]%
[|\def\jobname{|\textit{dest}|}|]|\input{|\textit{main}|}"|
\end{center}

%%%%%%%%%%%%%%%%%%%%%%%%%%%%%%%%%%%%%%%%%%%%%%%%%%%%%%%%%%%%%%%%%%%%%%%%%%%%%%%%
\subsection{Manual Code}
\label{sec:manual}

In case one cannot be certain whether the definitions file |childdoc.def|
is installed on the target \TeX{} distribution
and one prefers not to ship it,
it is conceivable to paste a few relevant commands into the sources.

To that end, drop all statements |\input{childdoc.def}|
and perform the replacements as outlined below.
Instead of |\childdocmain{|\textit{main}|}| add the following code
to the top of the main file:
%
\begin{center}
\begin{tabular}{l}
|\||ifdefined\childdocname\endinput\||fi\newif\ifchilddoc|\\
|\edef\childdocname{\scantokens\expandafter{\jobname\noexpand}}|\\
|\def\childdocmain{|\textit{main}|}\||ifx\childdocmain\childdocname\||else|\\
|\childdoctrue\includeonly{\childdocname}\let\jobname\childdocmain\||fi|\\
\end{tabular}
\end{center}
%
Instead of |\childdocof{|\textit{main}|}| just include the main file
at the top of each child file:
%
\begin{center}
|\input{|\textit{main}|}|
\end{center}
%
A simple redirection |\childdocforward{|\textit{dest}|}| is achieved by:
%
\begin{center}
|\def\jobname{|\textit{dest}|}\input{\jobname}|
\end{center}
%
The redirection with prefix
|\childdocforwardprefix[|\textit{prefix}|]{|\textit{dest}|}|
is accomplished by:
%
\begin{center}
\begin{tabular}{l}
|{\edef\jobname{\scantokens\expandafter{\jobname\noexpand}}|\\
|\def\redirectjob |\textit{prefix}|#1~~~{\gdef\jobname{|\textit{dest}|#1}}|\\
|\expandafter\redirectjob\jobname~~~}\input{\jobname}|
\end{tabular}
\end{center}

In an alternative approach,
child documents can be compiled by a specific command line
without additional code or specific definitions:
%
\begin{center}
|... -jobname "|\textit{target}|" "|[\textit{flags}]%
|\includeonly{|\textit{dest}|}\input{|\textit{main}|}"|
\end{center}
%

%%%%%%%%%%%%%%%%%%%%%%%%%%%%%%%%%%%%%%%%%%%%%%%%%%%%%%%%%%%%%%%%%%%%%%%%%%%%%%%%
%%%%%%%%%%%%%%%%%%%%%%%%%%%%%%%%%%%%%%%%%%%%%%%%%%%%%%%%%%%%%%%%%%%%%%%%%%%%%%%%
\section{Information}

%%%%%%%%%%%%%%%%%%%%%%%%%%%%%%%%%%%%%%%%%%%%%%%%%%%%%%%%%%%%%%%%%%%%%%%%%%%%%%%%
\subsection{Copyright}

Copyright \copyright{} 2017--2018 Niklas Beisert

This work may be distributed and/or modified under the
conditions of the \LaTeX{} Project Public License, either version 1.3
of this license or (at your option) any later version.
The latest version of this license is in
  \url{http://www.latex-project.org/lppl.txt}
and version 1.3 or later is part of all distributions of \LaTeX{}
version 2005/12/01 or later.

This work has the LPPL maintenance status `maintained'.

The Current Maintainer of this work is Niklas Beisert.

This work consists of the files |README.txt|, |childdoc.ins| and |childdoc.dtx|
as well as the derived files |childdoc.def|, |cdocsamp.tex|
with |cdocsch1.tex|, |cdocsch2.tex|, |cdocspt3.tex|, |cdocspt4.tex|,
|cdocsdrf.tex|, |cdocsfn1.tex|, |cdocsfn2.tex|
as well as |childdoc.pdf|.

%%%%%%%%%%%%%%%%%%%%%%%%%%%%%%%%%%%%%%%%%%%%%%%%%%%%%%%%%%%%%%%%%%%%%%%%%%%%%%%%
\subsection{Files and Installation}

The package consists of the files:
%
\begin{center}
\begin{tabular}{ll}
    |README.txt|   & readme file \\
    |childdoc.ins| & installation file \\
    |childdoc.dtx| & source file \\
    |childdoc.def| & definition file \\
    |cdocsamp.tex| & sample main file \\
    |cdocsch1.tex| & sample include file \\
    |cdocsch2.tex| & sample include file \\
    |cdocspt3.tex| & sample part file \\
    |cdocspt4.tex| & sample part file \\
    |cdocsdrf.tex| & sample redirection file \\
    |cdocsfn1.tex| & sample redirection file \\
    |cdocsfn2.tex| & sample redirection file \\
    |childdoc.pdf| & manual
\end{tabular}
\end{center}
%
The distribution consists of the files
|README.txt|, |childdoc.ins| and |childdoc.dtx|.
%
\begin{itemize}
\item
Run (pdf)\LaTeX{} on |childdoc.dtx|
to compile the manual |childdoc.pdf| (this file).
\item
Run \LaTeX{} on |childdoc.ins| to create the definitions file |childdoc.def|
and the sample |cdocsamp.tex| with include files
|cdocsch1.tex|, |cdocsch2.tex|, |cdocspt3.tex|, |cdocspt4.tex|,
|cdocsdrf.tex|, |cdocsfn1.tex|, |cdocsfn2.tex|.
Then copy the file |childdoc.def| to an appropriate directory of your \LaTeX{}
distribution, e.g.\ \textit{texmf-root}|/tex/latex/childdoc|.
\end{itemize}

%%%%%%%%%%%%%%%%%%%%%%%%%%%%%%%%%%%%%%%%%%%%%%%%%%%%%%%%%%%%%%%%%%%%%%%%%%%%%%%%
\subsection{Related CTAN Packages}

There are several other packages which offer a similar functionality:
%
\begin{itemize}
\item
The packages
\href{http://ctan.org/pkg/docmute}{\textsf{docmute}},
\href{http://ctan.org/pkg/includex}{\textsf{includex}} and
\href{http://ctan.org/pkg/standalone}{\textsf{standalone}}
provide commands to include only the document body of
a child file thus allowing both files to be compiled individually.
\item
The packages \href{http://ctan.org/pkg/subdocs}{\textsf{subdocs}}
and \href{http://ctan.org/pkg/subfiles}{\textsf{subfiles}}
provide structures in which the main and child documents can be
encapsulated and allowing them to be compiled individually.
The inclusion mechanism is different from the conventional |\include|.
\item
The package \href{http://ctan.org/pkg/combine}{\textsf{combine}}
is an elaborate solution to combine several documents into one.
\end{itemize}
%
See also the CTAN topic \href{http://ctan.org/topic/subdocs}{\textsf{subdocs}}
for further related packages.
The present package differs from the above solutions in that
a document structure constructed with the conventional |\include| mechanism
just needs two extra commands at the top of every file
such that all constituent files can be compiled individually.

%%%%%%%%%%%%%%%%%%%%%%%%%%%%%%%%%%%%%%%%%%%%%%%%%%%%%%%%%%%%%%%%%%%%%%%%%%%%%%%%
%\subsection{Feature Suggestions}
%
%The following is a list of features which may be useful for future
%versions of this package:
%%
%\begin{itemize}
%\item
%\ldots
%\end{itemize}

%%%%%%%%%%%%%%%%%%%%%%%%%%%%%%%%%%%%%%%%%%%%%%%%%%%%%%%%%%%%%%%%%%%%%%%%%%%%%%%%
\subsection{Revision History}

%%%%%%%%%%%%%%%%%%%%%%%%%%%%%%%%%%%%%%%%
\paragraph{v2.0:} 2018/12/30

\begin{itemize}
\item
immediate forward processing
\item
added |\childdocby| mechanism
\item
manual restructured
\end{itemize}

%%%%%%%%%%%%%%%%%%%%%%%%%%%%%%%%%%%%%%%%
\paragraph{v1.6:} 2018/01/17

\begin{itemize}
\item
application for development of include files
\item
corrections to manual
\end{itemize}

%%%%%%%%%%%%%%%%%%%%%%%%%%%%%%%%%%%%%%%%
\paragraph{v1.5:} 2017/05/21

\begin{itemize}
\item
more complete structuring introduced
\item
|\childdocof| introduced
\item
|\childdoc| renamed to |\childdocmain|
\item
|\childredirect| renamed to |\childdocforward| and |\childdocforwardprefix|
and functionality expanded
\end{itemize}

%%%%%%%%%%%%%%%%%%%%%%%%%%%%%%%%%%%%%%%%
\paragraph{v1.0:} 2017/04/27

\begin{itemize}
\item
manual and install package
\item
first version published on CTAN
\end{itemize}

%%%%%%%%%%%%%%%%%%%%%%%%%%%%%%%%%%%%%%%%
\paragraph{v0.6:} 2017/04/26

\begin{itemize}
\item
redirection mechanism added
\end{itemize}

%%%%%%%%%%%%%%%%%%%%%%%%%%%%%%%%%%%%%%%%
\paragraph{v0.5:} 2017/04/26

\begin{itemize}
\item
functionality in definition file
\end{itemize}


%%%%%%%%%%%%%%%%%%%%%%%%%%%%%%%%%%%%%%%%%%%%%%%%%%%%%%%%%%%%%%%%%%%%%%%%%%%%%%%%
%%%%%%%%%%%%%%%%%%%%%%%%%%%%%%%%%%%%%%%%%%%%%%%%%%%%%%%%%%%%%%%%%%%%%%%%%%%%%%%%
%%%%%%%%%%%%%%%%%%%%%%%%%%%%%%%%%%%%%%%%%%%%%%%%%%%%%%%%%%%%%%%%%%%%%%%%%%%%%%%%
\appendix

\settowidth\MacroIndent{\rmfamily\scriptsize 000\ }

 \DocInput{childdoc.dtx}

\end{document}
%</driver>
% \fi
%
% %%%%%%%%%%%%%%%%%%%%%%%%%%%%%%%%%%%%%%%%%%%%%%%%%%%%%%%%%%%%%%%%%%%%%%%%%%%%%%
% %%%%%%%%%%%%%%%%%%%%%%%%%%%%%%%%%%%%%%%%%%%%%%%%%%%%%%%%%%%%%%%%%%%%%%%%%%%%%%
% \section{Sample}
%\iffalse
%<*samplemain>
%\fi
%
% The following presents a sample document
% with two chapters, two parts, a title page,
% a compile flag as well as three forwarding files to set the flag.
% It consists of eight |.tex| files:
% \begin{center}
% \begin{tabular}{ll}
% |cdocsamp.tex|&main file\\
% |cdocsch1.tex|&include file for chapter 1\\
% |cdocsch2.tex|&include file for chapter 2\\
% |cdocspt3.tex|&include file for part 3\\
% |cdocspt4.tex|&include file for part 4\\
% |cdocsdrf.tex|&forwarding file for main file in draft mode\\
% |cdocsfi1.tex|&forwarding file for final version of chapter 1\\
% |cdocsfi2.tex|&forwarding file for final version of chapter 2\\
% \end{tabular}
% \end{center}
% Each of the eight files can be compiled directly by the \LaTeX{} compiler.
%
% %%%%%%%%%%%%%%%%%%%%%%%%%%%%%%%%%%%%%%
% \paragraph{Main File.}
%
% The main file is called |cdocsamp.tex|.
%
% Load the \textsf{childdoc} definitions and
% declare the filename for the main document:
%    \begin{macrocode}
\input{childdoc.def}
\childdocmain{}
%    \end{macrocode}

% Optional override for |\version| flag:
%    \begin{macrocode}
%%\ifchilddoc\else\providecommand{\version}{draft}\fi
%    \end{macrocode}

% Define the default values for the |\version| flag
% (|final| for the main file and |draft| for childs):
%    \begin{macrocode}
\ifchilddoc
\providecommand{\version}{draft}
\else
\providecommand{\version}{final}
\fi
%    \end{macrocode}

% Load the standard document class:
%    \begin{macrocode}
\documentclass[12pt]{article}
%    \end{macrocode}

% Start the document body:
%    \begin{macrocode}
\begin{document}
%    \end{macrocode}

% Declare a title page.
% Print title, part of document being processed and version flag:
%    \begin{macrocode}
\addtocounter{page}{-1}
\begin{center}
{\LARGE\bfseries{}childdoc example\par}
\vspace{1cm}
\ifchilddoc
\ifchilddocmanual part\else chapter\fi:
`\childdocname' of `\childdocjob'\par
\else
main document: `\childdocjob'\par
\fi
version: \version\par
\end{center}
\newpage
%    \end{macrocode}

% Manually include selected file,
% otherwise process as usual:
%    \begin{macrocode}
\ifchilddocmanual
\section*{part `\childdocname'}
\input{\childdocname}
\else
%    \end{macrocode}

% Include the two chapters:
%    \begin{macrocode}
\include{cdocsch1}
\include{cdocsch2}
%    \end{macrocode}

% Include the two parts unless only chapters should be displayed:
%    \begin{macrocode}
\ifchilddoc\else
\section{part three}
\input{cdocspt3}
\section{part four}
\input{cdocspt4}
\fi
%    \end{macrocode}

% Process as usual until here:
%    \begin{macrocode}
\fi
%    \end{macrocode}

% End of document body:
%    \begin{macrocode}
\end{document}
%    \end{macrocode}
%\iffalse
%</samplemain>
%\fi
%
% %%%%%%%%%%%%%%%%%%%%%%%%%%%%%%%%%%%%%%
% \paragraph{Chapter Include Files.}
%
% The include files are called |cdocsch1.tex| and |cdocsch2.tex|.
%
%\iffalse
%<*samplechap1|samplechap2>
%\fi

% Optional override for |\version| flag:
%    \begin{macrocode}
%%\providecommand{\version}{final}
%    \end{macrocode}

% Include the main document:
%    \begin{macrocode}
\input{childdoc.def}
\childdocof{cdocsamp}
%    \end{macrocode}

%\iffalse
%</samplechap1|samplechap2>
%\fi
%
%\iffalse
%<*samplechap1>
%\fi
% Some text for chapter 1:
%    \begin{macrocode}
\section{one}
some text in chapter one
%    \end{macrocode}

%\iffalse
%</samplechap1>
%\fi
% Some text for chapter 2:
%\iffalse
%<*samplechap2>
%\fi
%    \begin{macrocode}
\section{two}
more text in chapter two
%    \end{macrocode}

%\iffalse
%</samplechap2>
%\fi
%
% %%%%%%%%%%%%%%%%%%%%%%%%%%%%%%%%%%%%%%
% \paragraph{Part Include Files.}
%
% The include files are called |cdocspt3.tex| and |cdocspt4.tex|.
%
%\iffalse
%<*samplepart3|samplepart4>
%\fi

% Optional override for |\version| flag:
%    \begin{macrocode}
%%\providecommand{\version}{final}
%    \end{macrocode}

% Include the main document:
%    \begin{macrocode}
\input{childdoc.def}
\childdocby{cdocsamp}
%    \end{macrocode}

%\iffalse
%</samplepart3|samplepart4>
%\fi
%
%\iffalse
%<*samplepart3>
%\fi
% Some text for part 3:
%    \begin{macrocode}
some text in part three
%    \end{macrocode}

%\iffalse
%</samplepart3>
%\fi
% Some text for part 4:
%\iffalse
%<*samplepart4>
%\fi
%    \begin{macrocode}
more text in part four
%    \end{macrocode}

%\iffalse
%</samplepart4>
%\fi
%
% %%%%%%%%%%%%%%%%%%%%%%%%%%%%%%%%%%%%%%
% \paragraph{Forwarding for a Complete Draft.}
%
% The following forwarding file |cdocsdrf.tex|
% compiles the main document in draft mode:
%\iffalse
%<*sampledraft>
%\fi
%    \begin{macrocode}
\def\version{draft}
\input{childdoc.def}
\childdocforward{cdocsamp}
%    \end{macrocode}

%\iffalse
%</sampledraft>
%\fi
%
% %%%%%%%%%%%%%%%%%%%%%%%%%%%%%%%%%%%%%%
% \paragraph{Forwarding for Final Version of the Chapters.}
%
% The following forwarding files |cdocsfn1.tex| and |cdocsfn2.tex|
% (with identical content)
% compile the final versions of the child documents
% |cdocsch1.tex| and |cdocsch2.tex|, respectively:
%\iffalse
%<*samplefinal>
%\fi
%    \begin{macrocode}
\def\version{final}
\input{childdoc.def}
\childdocforwardprefix[cdocsamp]{cdocsfn}{cdocsch}
%    \end{macrocode}

%\iffalse
%</samplefinal>
%\fi
%
% %%%%%%%%%%%%%%%%%%%%%%%%%%%%%%%%%%%%%%
% \paragraph{Command Line Processing.}
%
% The following three command lines generate the output files
% |cdocscld|, |cdocscl1| and |cdocscl2|
% which should be identical to
% |cdocsdrf|, |cdocsch1| and |cdocsfn2|, respectively:
% \begin{center}
% \begin{tabular}{l}
% |latex -jobname cdocscld \|\\
% |  "\def\version{draft}\input{childdoc.def}\childdocforward{cdocsamp}"|\\
% |latex -jobname cdocscl1 \|\\
% |  "\input{childdoc.def}\childdocforward[cdocsamp]{cdocsch1}"|\\
% |latex -jobname cdocscl2 \|\\
% |  "\def\version{final}\input{childdoc.def}\childdocforward{cdocsch2}"|
% \end{tabular}
% \end{center}
% Note that the trailing backslash on each first line
% merely continues the input to the second line
% (for convenient cut ant paste).
% Furthermore, the command |latex| can be replaced by any
% of its alternative versions such as |pdflatex|.
%
% %%%%%%%%%%%%%%%%%%%%%%%%%%%%%%%%%%%%%%%%%%%%%%%%%%%%%%%%%%%%%%%%%%%%%%%%%%%%%%
% %%%%%%%%%%%%%%%%%%%%%%%%%%%%%%%%%%%%%%%%%%%%%%%%%%%%%%%%%%%%%%%%%%%%%%%%%%%%%%
% \section{Implementation}
%\iffalse
%<*package>
%\fi
%
% This section describes the definitions file |childdoc.def|.

% The definitions cannot be loaded using |\usepackage| or |\RequirePackage|
% which has a mechanism to prevent loading a style file more than once.
% When loading the definitions by means of |\input|
% multiple instances have to be prevented manually:
%\iffalse
%This code needs to be before the `\ProvidesFile' directive
%which is defined at the beginning of this file.
%Therefore it is also placed there and commented out here.
%</package>
%<*discard>
%\fi
%    \begin{macrocode}
\ifdefined\childdocmain\endinput\fi
%    \end{macrocode}
%\iffalse
%</discard>
%<*package>
%\fi
%
% \macro{\ifchilddoc}
% \macro{\ifchilddocmanual}
% The conditional |\ifchilddoc| tells whether a
% child (true) or main (false) document is being compiled.
% The conditional |\ifchilddocmanual| tells whether
% the |\includeonly| mechanism is used (false) or
% the selection of child files must be performed manually (true).
% The definitions initialise to false:
%    \begin{macrocode}
\newif\ifchilddoc
\newif\ifchilddocmanual
%    \end{macrocode}

% \macro{\childdocname}
% \macro{\childdocjob}
% The macro |\childdocname| stores the name of the main document
% to be compiled. The macro |\childdocjob| stores the name of
% the document on which the \LaTeX{} compiler was originally invoked.
% The content of |\jobname| cannot be compared
% to filenames specified in the source due to different catcodes.
% The following code rescans |\jobname|, stores the result
% in |\childdocname| and saves a copy in |\childdocjob|:
%    \begin{macrocode}
\edef\childdocname{\scantokens\expandafter{\jobname\noexpand}}
\let\childdocjob\childdocname
%    \end{macrocode}

% \macro{\childdocdisable}
% The macro |\childdocdisable| prevents the main file
% from being processed more than once.
% At this stage, the main document command |\childdocmain|
% is assumed to be called once again where it should do nothing.
% Any subsequent call to it should prevent
% a secondary processing of the main document
% It overwrites the forwarding commands
% |\childdocof| and |\childdocforward|
% with empty macros to prevent further inclusions of the main document:
%    \begin{macrocode}
\newcommand{\childdocdisable}
{
  \renewcommand{\childdocmain}[1]{\renewcommand{\childdocmain}[1]{\endinput}}
  \renewcommand{\childdocof}[1]{}
  \renewcommand{\childdocby}[2][]{}
  \renewcommand{\childdocforward}[2][]{}
  \renewcommand{\childdocdisable}{}
}
%    \end{macrocode}

% \macro{\childdocmain}
% The macro |\childdocmain| is to be called at the top of the main file
% with nothing or the main filename (without extension) as argument.
% First, it breaks loops.
% If the argument is not empty and does not match |\childdocname|
% (which is set by the first inclusion of |childdoc.def|),
% |\ifchilddoc| is set to true, |\includeonly| is applied to the child file
% and |\jobname| is set to the main file
% (for proper handling of |.aux| files):
%    \begin{macrocode}
\newcommand{\childdocmain}[1]
{
  \childdocdisable\childdocmain{}
  \if?#1?\else
    \begingroup
      \def\childdoctmp{#1}
      \ifx\childdoctmp\childdocname
        \def\childdoctmp{}
      \else
        \def\childdoctmp
        {
          \childdoctrue
          \includeonly{\childdocname}
          \def\childdocjob{#1}
          \def\jobname{#1}
        }
      \fi
      \expandafter
    \endgroup
    \childdoctmp
  \fi
}
%    \end{macrocode}

% \macro{\childdocof}
% The command |\childdocof| redirects
% compilation to the main file |#1|.
%    \begin{macrocode}
\newcommand{\childdocof}[1]
{
  \childdocdisable
  \childdoctrue
  \includeonly{\childdocname}
  \def\jobname{#1}
  \def\childdocjob{#1}
  \input{#1}
}
%    \end{macrocode}

% \macro{\childdocby}
% The command |\childdocby| ....
%    \begin{macrocode}
\newcommand{\childdocby}[2][]
{
  \childdocdisable
  \childdoctrue
  \childdocmanualtrue
  \if?#1?\else
    \def\jobname{#2}
  \fi
  \def\childdocjob{#2}
  \input{#2}
  \endinput
}
%    \end{macrocode}

% \macro{\childdocforward}
% The command |\childdocforward| redirects
% compilation to the main file or
% (if the optional argument is given) a child file.
% Parameters are set as if the main file
% or a child file starting with |\childdocof| was compiled.
% Then compilation is handed over to the main file:
%    \begin{macrocode}
\newcommand{\childdocforward}[2][]
{
  \begingroup
    \if?#1?
      \def\childdoctmp
      {
        \def\childdocname{#2}
        \def\childdocjob{#2}
        \def\jobname{#2}
        \input{#2}
        \endinput
      }
    \else
      \def\childdoctmp
      {
        \childdocdisable
        \def\childdocname{#2}
        \childdoctrue
        \includeonly{#2}
        \def\childdocjob{#1}
        \def\jobname{#1}
        \input{#1}
        \endinput
      }
    \fi
    \expandafter
  \endgroup
  \childdoctmp
}
%    \end{macrocode}

% \macro{\childdocforwardprefix}
% The command |\childdocforwardprefix| redirects
% compilation to the main or a child file by means of a pattern.
% The prefix |#1| in the current filename is replaced by |#2|
% and the suffix of the current filename is kept
% (it is assumed that the filename does not contain the substring `|~~~|'
% which is used as a delimiter).
% Compilation is handed over to the new file by |\childdocforward|:
%    \begin{macrocode}
\newcommand{\childdocforwardprefix}[3][]
{
  \begingroup
    \def\childdocextract #2##1~~~{\def\childdoctmp{\childdocforward[#1]{#3##1}}}
    \expandafter\childdocextract\childdocname~~~
    \expandafter
  \endgroup
  \childdoctmp
}
%    \end{macrocode}

% \macro{\childdoc}
% The deprecated macro |\childdoc| is a legacy version of |\childdocmain|:
%    \begin{macrocode}
\newcommand{\childdoc}{\childdocmain}
%    \end{macrocode}

% \macro{\childdocredirect}
% The deprecated macro |\childdocredirect| is a legacy version
% of |\childdocforward| and |\childdocforwardprefix|:
%    \begin{macrocode}
\newcommand{\childdocredirect}[2][]
{
  \begingroup
    \if?#1?
      \def\childdoctmp{\childdocforward{#2}}
    \else
      \def\childdoctmp{\childdocforwardprefix{#1}{#2}}
    \fi
    \expandafter
  \endgroup
  \childdoctmp
}
%    \end{macrocode}

%\iffalse
%</package>
%\fi
%
\endinput

\childdocby{cdocsamp}
%    \end{macrocode}

%\iffalse
%</samplepart3|samplepart4>
%\fi
%
%\iffalse
%<*samplepart3>
%\fi
% Some text for part 3:
%    \begin{macrocode}
some text in part three
%    \end{macrocode}

%\iffalse
%</samplepart3>
%\fi
% Some text for part 4:
%\iffalse
%<*samplepart4>
%\fi
%    \begin{macrocode}
more text in part four
%    \end{macrocode}

%\iffalse
%</samplepart4>
%\fi
%
% %%%%%%%%%%%%%%%%%%%%%%%%%%%%%%%%%%%%%%
% \paragraph{Forwarding for a Complete Draft.}
%
% The following forwarding file |cdocsdrf.tex|
% compiles the main document in draft mode:
%\iffalse
%<*sampledraft>
%\fi
%    \begin{macrocode}
\def\version{draft}
% \iffalse
%
% childdoc.dtx Copyright (C) 2017-2018 Niklas Beisert
%
% This work may be distributed and/or modified under the
% conditions of the LaTeX Project Public License, either version 1.3
% of this license or (at your option) any later version.
% The latest version of this license is in
%   http://www.latex-project.org/lppl.txt
% and version 1.3 or later is part of all distributions of LaTeX
% version 2005/12/01 or later.
%
% This work has the LPPL maintenance status `maintained'.
%
% The Current Maintainer of this work is Niklas Beisert.
%
% This work consists of the files childdoc.dtx and childdoc.ins
% and the derived files childdoc.def and cdocsamp.tex with
% cdocsch1.tex, cdocsch2.tex, cdocsdrf.tex, cdocsfn1.tex, cdocsfn2.tex.
%
%<package>\ifdefined\childdocmain\endinput\fi
%<package>\ProvidesFile{childdoc.def}[2018/12/30 v2.0 child document driver]
%<samplemain>\ProvidesFile{cdocsamp.tex}[2018/12/30 v2.0 sample for childdoc]
%<*driver>
%\ProvidesFile{childdoc.drv}[2018/12/30 v2.0 childdoc reference manual file]
\PassOptionsToClass{10pt,a4paper}{article}
\documentclass{ltxdoc}

\usepackage[margin=35mm]{geometry}
\usepackage{hyperref}
\usepackage{hyperxmp}
\usepackage[usenames]{color}

\hypersetup{colorlinks=true}
\hypersetup{pdfstartview=FitH}
\hypersetup{pdfpagemode=UseNone}
\hypersetup{pdfsource={}}
\hypersetup{pdflang={en-UK}}
\hypersetup{pdfcopyright={Copyright 2017-2018 Niklas Beisert.
  This work may be distributed and/or modified under the
  conditions of the LaTeX Project Public License, either version 1.3
  of this license or (at your option) any later version.}}
\hypersetup{pdflicenseurl={http://www.latex-project.org/lppl.txt}}
\hypersetup{pdfcontactaddress={ETH Zurich, ITP, HIT K,
  Wolfgang-Pauli-Strasse 27}}
\hypersetup{pdfcontactpostcode={8093}}
\hypersetup{pdfcontactcity={Zurich}}
\hypersetup{pdfcontactcountry={Switzerland}}
\hypersetup{pdfcontactemail={nbeisert@itp.phys.ethz.ch}}
\hypersetup{pdfcontacturl={http://people.phys.ethz.ch/\xmptilde nbeisert/}}

\newcommand{\secref}[1]{\hyperref[#1]{section \ref*{#1}}}

\parskip1ex
\parindent0pt
\let\olditemize\itemize
\def\itemize{\olditemize\parskip0pt}

\begin{document}

\title{The \textsf{childdoc} Package}
\hypersetup{pdftitle={The childdoc Package}}
\author{Niklas Beisert\\[2ex]
  Institut f\"ur Theoretische Physik\\
  Eidgen\"ossische Technische Hochschule Z\"urich\\
  Wolfgang-Pauli-Strasse 27, 8093 Z\"urich, Switzerland\\[1ex]
  \href{mailto:nbeisert@itp.phys.ethz.ch}
  {\texttt{nbeisert@itp.phys.ethz.ch}}}
\hypersetup{pdfauthor={Niklas Beisert}}
\hypersetup{pdfsubject={Manual for the LaTeX2e Package childdoc}}
\date{30 December 2018, \textsf{v2.0}}
\maketitle

\begin{abstract}\noindent
\textsf{childdoc} is a \LaTeXe{} package
that enables the direct compilation
of document sections included by |\include|
to individual files.
\end{abstract}

\begingroup
\parskip0ex
\tableofcontents
\endgroup

%%%%%%%%%%%%%%%%%%%%%%%%%%%%%%%%%%%%%%%%%%%%%%%%%%%%%%%%%%%%%%%%%%%%%%%%%%%%%%%%
%%%%%%%%%%%%%%%%%%%%%%%%%%%%%%%%%%%%%%%%%%%%%%%%%%%%%%%%%%%%%%%%%%%%%%%%%%%%%%%%
\section{Introduction}

\LaTeX{} provides a mechanism to structure a large document (such as a book)
into a main file and several child files (containing the chapters)
using the |\include| command.
This mechanism is beneficial for documents
which span hundreds of pages in order to
make the source file(s) more manageable.
Moreover, compilation can be restricted to
selected child files by means of the |\includeonly| command.
The latter feature can be used to reduce the compilation time while editing
(this was significantly more useful in the earlier days of \LaTeX{})
or to generate a smaller document which is easier to navigate.
Another application of |\includeonly| is to generate
documents consisting of selected parts of the complete document.

However, there are a few drawbacks of the plain |\include| mechanism:
\begin{itemize}
\item
The child files cannot be compiled on their own,
they can only be compiled via the main file.
A naive editing environment
(such as a text editor with an option
to have the current file processed by \LaTeX)
may require one to switch to the main file before compiling;
attempting to compile the child file produces errors.
\item
The main file must be modified (each time)
to adjust the |\includeonly| command
to the present needs. This easily leaves the main file in a messy state.
\item
The generated document will always carry the filename
of the main document. This is inconvenient if
several child files are to be compiled and
to be kept for distribution.
\end{itemize}

The present package provides a simple interface
to make child files individually compilable by \LaTeX{}.
Compiling a child file then has the same effect as compiling
the main file with an |\includeonly| command
to select the appropriate child.
Moreover the generated document will carry the name of the child
rather than the main file.
This resolves all three above issues.

This feature is meant to make the editing of books,
thesis documents and lecture notes somewhat more convenient.
However, the package can also be used efficiently for
composing a series of documents (such as exercise sheets)
which are typically distributed individually.
It then assists the author in generating the individual documents
(potentially in different versions)
as well as a document containing the collected series.
Another application is in developing style files
or other kinds of included material
where compilation of the style file could redirect
to a sample or test file.

%%%%%%%%%%%%%%%%%%%%%%%%%%%%%%%%%%%%%%%%%%%%%%%%%%%%%%%%%%%%%%%%%%%%%%%%%%%%%%%%
%%%%%%%%%%%%%%%%%%%%%%%%%%%%%%%%%%%%%%%%%%%%%%%%%%%%%%%%%%%%%%%%%%%%%%%%%%%%%%%%
\section{Usage}

First of all, the package \textsf{childdoc} is \emph{not} a standard
\LaTeXe{} |.sty| style file! Therefore it needs to be invoked in
a non-standard way.

%%%%%%%%%%%%%%%%%%%%%%%%%%%%%%%%%%%%%%%%%%%%%%%%%%%%%%%%%%%%%%%%%%%%%%%%%%%%%%%%
\subsection{Included Files}
\label{sec:include}

%%%%%%%%%%%%%%%%%%%%%%%%%%%%%%%%%%%%%%%%
\DescribeMacro{\childdocmain}
To use the package, add the commands
\begin{center}
\begin{tabular}{l}
|\input{childdoc.def}|\\
|\childdocmain{}|\\
\end{tabular}
\end{center}
at the very top of the main \LaTeX{} file,
in particular \emph{before} the |\documentclass| statement!
The argument of |\childdocmain| should be left empty
(but it must be present).

%%%%%%%%%%%%%%%%%%%%%%%%%%%%%%%%%%%%%%%%
\DescribeMacro{\childdocof}
Furthermore, add the commands
\begin{center}
\begin{tabular}{l}
|\input{childdoc.def}|\\
|\childdocof{|\textit{main}|}|\\
\end{tabular}
\end{center}
at the top of every child file \textit{child}
which is included by |\include{|\textit{child}|}|
from within the main file
(or at least for those files to be compiled individually).
The argument \textit{main} must be the filename of the main file.

There are a couple of
considerations in setting up the main and child documents:

%%%%%%%%%%%%%%%%%%%%%%%%%%%%%%%%%%%%%%%%
\paragraph{Restrictions.}

Please note the following restrictions:
\begin{itemize}
\item
|\childdocmain| must be called with one argument \textit{main}
to ensure compatibility with earlier version of the package.
It must either be empty (|\childdocmain{}|)
or precisely match the filename of the main file in which it is specified.
See \secref{sec:detection} for further information.
\item
The filename \textit{main} must be specified without the |.tex| extension.
\item
The filename \textit{main} is case sensitive
(even in case-insensitive file systems)
due to internal string comparison.
\item
The argument \textit{main} should be fully expanded, it cannot be a macro.
\item
Subdirectories and special characters should be avoided in filenames.
\item
The command |\childdocmain{|\textit{main}|}| must be followed by a whitespace.
It should not be followed immediately by another command
or by a comment mark `|%|'.
This is because the \TeX{} parser reads the token immediately following
the argument of |\childdocmain| and puts it
at the beginning of every child section;
however, a white\-space is ignored.
\end{itemize}

%%%%%%%%%%%%%%%%%%%%%%%%%%%%%%%%%%%%%%%%
\paragraph{Content of Main File.}

It is advisable to place all content in the child files included by |\include|.
Any output contained in the main file will appear in all child documents
unless suppressed manually;
it cannot be suppressed automatically by the |\includeonly| directive
and thus should normally be avoided.
A method to include some content in the main file
by means of conditional processing is described in \secref{sec:conditional}.

%%%%%%%%%%%%%%%%%%%%%%%%%%%%%%%%%%%%%%%%
\paragraph{Page Numbering.}

When only a part of the document is compiled,
the appropriate numbering of pages
(as well as other status parameters)
is determined from the |.aux| files.
The latter contain information from previous passes.
However this information needs to propagate through
all intermediate child documents.
Therefore the page numbering in child documents may well
be inconsistent until the complete document is compiled at least once.

A useful (if unconventional) way to always ensure a consistent
page numbering is to restart the numbering in each child document
and denote the pages by `\textit{child}|.|\textit{page}'
where \textit{child} represents the chapter/section number of the child file.
This can be achieved by the command
|\numberwithin{page}{|\textit{child}|}|
of the \textsf{amsmath} package
where \textit{child} can be |chapter| or |section|
depending on the chosen structuring.
Alternatively, one can modify the macro |\thepage| appropriately
and reset the counter |page| at the start of each child file.

%%%%%%%%%%%%%%%%%%%%%%%%%%%%%%%%%%%%%%%%%%%%%%%%%%%%%%%%%%%%%%%%%%%%%%%%%%%%%%%%
\subsection{Conditional Processing}
\label{sec:conditional}

The package provides a mechanism to compile different versions
of a document. To customise the versions further some conditional processing
can come in handy to distinguish which version is being compiled.
The package provides two macros to describe the compilation context:

%%%%%%%%%%%%%%%%%%%%%%%%%%%%%%%%%%%%%%%%
\DescribeMacro{\ifchilddoc}
The conditional |\ifchilddoc| distinguishes between the compilation of
child documents and the main document:
%
\begin{center}
|\ifchilddoc |\textit{child-code}| |[|\||else |\textit{main-code}]| \||fi|
\end{center}

%%%%%%%%%%%%%%%%%%%%%%%%%%%%%%%%%%%%%%%%
\DescribeMacro{\childdocname}
\DescribeMacro{\childdocjob}
The macro |\childdocname| contains the filename (without extension)
of the main or child file being processed.
Note that |\childdocjob| will always contain the name of the main file.

%%%%%%%%%%%%%%%%%%%%%%%%%%%%%%%%%%%%%%%%
\paragraph{Title Page.}

Conditional processing can be used to include a title or banner page
in the main document when proper precautions are taken.
Importantly, the code in the main file should ensure that the page counter
(as well as other status parameters which are stored in the |.aux| files)
takes the same value after the conditional processing.
Otherwise the page numbers may take divergent values
depending on which part is compiled.

For example, a title page could be declared by:
%
\begin{center}
\begin{tabular}{l}
|\ifchilddoc\||else|\\
|\addtocounter{page}{-1}|\\
\textit{code for title page}\\
|\newpage|\\
|\||fi|
\end{tabular}
\end{center}
%
A banner page for the child documents can be generated by:
%
\begin{center}
\begin{tabular}{l}
|\ifchilddoc|\\
|\addtocounter{page}{-1}|\\
\textit{code for banner page}\\
|\newpage|\\
|\||fi|
\end{tabular}
\end{center}
%
Here one could write a message such as:
\begin{center}
|This is the part \childdocname{} of \childdocjob{}.|
\end{center}

%%%%%%%%%%%%%%%%%%%%%%%%%%%%%%%%%%%%%%%%%%%%%%%%%%%%%%%%%%%%%%%%%%%%%%%%%%%%%%%%
\subsection{Flags}
\label{sec:flags}

The package makes it easy to generate different versions
of the main or child documents.
To this end compilation flags can be defined
and assigned different default values.
They will be particularly useful in conjunction
with the forwarding mechanism described in \secref{sec:forward}.

For example, it may be useful to have a flag |\version|
which can be set to |draft| or |final|.
The document source will contain some conditional code
depending on the value of |\version|.
Suppose further, the flag should default to |final| for the main file
and to |draft| for child files
which is a natural assignment for editing the document.
This is achieved by placing the following code
in the preamble of the main document
(below the |\childdocmain| directive):
%
\begin{center}
\begin{tabular}{l}
|\ifchilddoc|\\
|\providecommand{\version}{draft}|\\
|\||else|\\
|\providecommand{\version}{final}|\\
|\||fi|
\end{tabular}
\end{center}
%
The definition by |\providecommand| makes sure
that previous definitions are not overwritten.
Further statements |\providecommand{\version}{...}|
can thus be added before the above code to override it.

For the main file, one might add a line
(between |\childdocmain| and the above block)
%
\begin{center}
|%\ifchilddoc\||else\providecommand{\version}{draft}\||fi|
\end{center}
%
which can be uncommented to produce a draft version.
Likewise one can add a line to the very top of a child file
(above the |\childdocof{|\textit{main}|}| directive)
%
\begin{center}
|%\providecommand{\version}{final}|
\end{center}
%
which can be uncommented to produce the final version of this child document.

%%%%%%%%%%%%%%%%%%%%%%%%%%%%%%%%%%%%%%%%%%%%%%%%%%%%%%%%%%%%%%%%%%%%%%%%%%%%%%%%
\subsection{Forwarding}
\label{sec:forward}

Different versions of the main or child documents
using compilation flags as described in \secref{sec:flags}
can be (permanently) stored in different files
for convenient compilation, viewing and distribution.
To this end, the package defines a command
to pass on compilation to a different file:

%%%%%%%%%%%%%%%%%%%%%%%%%%%%%%%%%%%%%%%%
\DescribeMacro{\childdocforward}
The command |\childdocforward| redirects processing to
another source file:
%
\begin{center}
\begin{tabular}{l}
|\input{childdoc.def}|\\
|\childdocforward[|\textit{main}|]{|\textit{dest}|}|\\
\end{tabular}
\end{center}
%
The argument \textit{dest} is the destination file
(without extension).
It should be the main file or one of the child files.
Note that further \textsf{childdoc} directives
such as |\childdocof| and |\childdocforward|
in the indicated file will be processed in this form.
The optional argument \textit{main}
passes on directly to the main file \textit{main}
while pretending to compile the child \textit{dest}.
This form behaves as if \textit{dest}
issues |\childdocof{|\textit{main}|}| right away,
and no further \textsf{childdoc} directives will be processed.

%%%%%%%%%%%%%%%%%%%%%%%%%%%%%%%%%%%%%%%%
\DescribeMacro{\...prefix}
In the alternative form |\childdocforwardprefix|,
%
\begin{center}
\begin{tabular}{l}
|\input{childdoc.def}|\\
|\childdocforwardprefix[|\textit{main}|]{|\textit{prefix}|}{|\textit{dest}|}|
\end{tabular}
\end{center}
%
the destination file is determined by a pattern
depending on the current file:
To make this work, the current file must be called
`{\textit{prefix}\hspace{0.2em}\textit{suffix}}'
with \textit{prefix} matching precisely the argument.
Processing is then passed on to the file
`{\textit{dest}\hspace{0.2em}\textit{suffix}}'.
Surely, the same effect is achieved by
directly specifying the
argument `{\textit{dest}\hspace{0.2em}\textit{suffix}}'
in the first form.
However, that requires to set up a different file
for each child. With the alternative form of the command
all these files can have exactly the same content
which simplifies setting them up and maintaining them.

For example, the following file |draft.tex|
with a compilation flag |\version| as described in \secref{sec:flags}
compiles the main document as a draft:
%
\begin{center}
\begin{tabular}{l}
|\def\version{draft}|\\
|\input{childdoc.def}|\\
|\childdocforward{|\textit{main}|}|
\end{tabular}
\end{center}
%
Likewise, the following files |final|\textit{nn}|.tex|
compile the final version of the child document
|child|\textit{nn}|.tex|:
%
\begin{center}
\begin{tabular}{l}
|\def\version{final}|\\
|\input{childdoc.def}|\\
|\childdocforwardprefix{final}{child}|
\end{tabular}
\end{center}
%

Note that when several versions of a main file and/or of each child file
are to be generated, it may be convenient to set up a |Makefile| or
shell script to automatise the process.

%%%%%%%%%%%%%%%%%%%%%%%%%%%%%%%%%%%%%%%%%%%%%%%%%%%%%%%%%%%%%%%%%%%%%%%%%%%%%%%%
\subsection{Command Line Processing}
\label{sec:commandline}

The effect of redirection files can also be achieved by invoking
the \LaTeX{} compiler with a more elaborate command line.
Most conveniently this should be done as part
of a shell script or a |Makefile|.

When using \textsf{childdoc} in the main file, the following
command lines effectively perform a redirection
(note that depending on the shell being used,
backslashes may have to be doubled: `|\|' $\to$ `|\\|'):
%
\begin{center}
|... -jobname "|\textit{target}|" |\\|"|[\textit{flags}]%
|\input{childdoc.def}\childdocforward[|\textit{main}|]{|\textit{dest}|}"|
\end{center}
%
Here \textit{target} is the name of the output file,
\textit{main} is the name of the main file
and \textit{dest} is the name of the main or child file to be processed
(all filenames without extensions).
The optional argument \textit{main} can be omitted
if \textit{main} matches \textit{dest}.
Optionally, compilation \textit{flags} can be defined via |\def| commands.
This command line makes the \TeX{} engine believe
it is compiling the file \textit{target}
whose content is specified as the latter parameter.
The provided code then forwards the processing to
\textit{main} or \textit{dest} as described in \secref{sec:forward}.

%%%%%%%%%%%%%%%%%%%%%%%%%%%%%%%%%%%%%%%%%%%%%%%%%%%%%%%%%%%%%%%%%%%%%%%%%%%%%%%%
\subsection{Include by Input}
\label{sec:input}

Including child documents by |\include| has some restrictions by design.
Most notably, the content of a child document always occupies
its own set of pages; pages cannot be shared between child documents.
Usually, this behaviour makes perfect sense
because each child document contain an essential part of the document.
However, in some situations it may be desirable to compose
a document from a collection of parts
without having mandatory page breaks between then.
For this case, the package
provides a mechanism to include parts
by |\input| which can also be processed individually.
However, by construction this mechanism
requires manual handling of the content to be output.

%%%%%%%%%%%%%%%%%%%%%%%%%%%%%%%%%%%%%%%%
\DescribeMacro{\ifchilddocmanual}
The main file should be prepared as usual, see \secref{sec:include}.
However, the document body must make a distinction
between processing of an individual part and of the main document, e.g.:
%
\begin{center}
\begin{tabular}{l}
|\ifchilddocmanual|\\
|\input{\childdocname}|\\
|\||else|\\
\textit{document body with }|\input{|\textit{part}|}|\\
|\||fi|
\end{tabular}
\end{center}
%
The conditional |\ifchilddocmanual| is true whenever
a part to be included by |\input| is being compiled,
and the name of the part is stored in |\childdocname|.

%%%%%%%%%%%%%%%%%%%%%%%%%%%%%%%%%%%%%%%%
\DescribeMacro{\childdocby}
Each part to be included by |\input| should start with:
%
\begin{center}
\begin{tabular}{l}
|\input{childdoc.def}|\\
|\childdocby{|\textit{main}|}|\\
\end{tabular}
\end{center}
%
The directive |\childdocby| is similar to |\childdocof|
described in \secref{sec:include},
but the subsequent selection of content must be done manually.
To that end, both |\ifchilddoc| and |\ifchilddocmanual|
will be true upon processing of a part,
and the name of the part is stored in |\childdocname|.
Note that |\jobname| will be set to the filename of the current part
so that each part receives an individual |.aux| file
that does not interfere with the |.aux| file(s) of the main document.
This behaviour can be altered by the alternative form
|\childdocby[*]{|\textit{main}|}| (with a non-empty optional argument)
which uses the |.aux| file of the main document
by setting |\jobname| to \textit{main}.

%%%%%%%%%%%%%%%%%%%%%%%%%%%%%%%%%%%%%%%%%%%%%%%%%%%%%%%%%%%%%%%%%%%%%%%%%%%%%%%%
\subsection{Driver Development}
\label{sec:driver}

The \textsf{childdoc} mechanism can also be use for the development
of definition files such as \LaTeX{} styles or classes.
This case differs from the above setup with multiple parts
included by |\include| in that no |\includeonly| should be invoked.
This can be achieved by starting the include file
(before |\ProvidesPackage|) with:
%
\begin{center}
\begin{tabular}{l}
|\input{childdoc.def}|\\
|\childdocforward{|\textit{main}|}|\\
\end{tabular}
\end{center}
%
or alternatively with:
%
\begin{center}
\begin{tabular}{l}
|\input{childdoc.def}|\\
|\childdocby{|\textit{main}|}|\\
\end{tabular}
\end{center}
%
Both forms have slightly different effects as described above.
The main file is prepared as usual, see \secref{sec:include}.

%%%%%%%%%%%%%%%%%%%%%%%%%%%%%%%%%%%%%%%%%%%%%%%%%%%%%%%%%%%%%%%%%%%%%%%%%%%%%%%%
\subsection{Legacy Detection}
\label{sec:detection}

The directive |\childdocmain| in the main file can detect
whether the complete document or merely a child is to be compiled
even without using the directive |\childdocof|.
This method is deprecated because it is less robust
and there is no compelling reason to use it;
it is merely provided for backward compatibility
and it may be removed in future versions.

If the detection mechanism is to be used,
it is mandatory to correctly specify
the filename of the main file as the argument of |\childdocmain|:
%
\begin{center}
\begin{tabular}{l}
|\input{childdoc.def}|\\
|\childdocmain{|\textit{main}|}|\\
\end{tabular}
\end{center}
%
If |\jobname| does not match the argument \textit{main} of |\childdocmain|,
it is assumed that |\jobname| points to the child file to be compiled.
When using |\childdocmain| with the main file specified as argument,
it suffices to start a child file
with just |\input{|\textit{main}|}|
without loading of the package and using |\childdocof|.
If instead all processing is done
with the appropriate \textsf{childdoc} directives,
the argument of \textit{main} of |\childdocmain| can be empty.

An alternative version of the command line processing described
in \secref{sec:commandline} using the detection mechanism reads:
%
\begin{center}
|... -jobname "|\textit{target}|" "|[\textit{flags}]%
[|\def\jobname{|\textit{dest}|}|]|\input{|\textit{main}|}"|
\end{center}

%%%%%%%%%%%%%%%%%%%%%%%%%%%%%%%%%%%%%%%%%%%%%%%%%%%%%%%%%%%%%%%%%%%%%%%%%%%%%%%%
\subsection{Manual Code}
\label{sec:manual}

In case one cannot be certain whether the definitions file |childdoc.def|
is installed on the target \TeX{} distribution
and one prefers not to ship it,
it is conceivable to paste a few relevant commands into the sources.

To that end, drop all statements |\input{childdoc.def}|
and perform the replacements as outlined below.
Instead of |\childdocmain{|\textit{main}|}| add the following code
to the top of the main file:
%
\begin{center}
\begin{tabular}{l}
|\||ifdefined\childdocname\endinput\||fi\newif\ifchilddoc|\\
|\edef\childdocname{\scantokens\expandafter{\jobname\noexpand}}|\\
|\def\childdocmain{|\textit{main}|}\||ifx\childdocmain\childdocname\||else|\\
|\childdoctrue\includeonly{\childdocname}\let\jobname\childdocmain\||fi|\\
\end{tabular}
\end{center}
%
Instead of |\childdocof{|\textit{main}|}| just include the main file
at the top of each child file:
%
\begin{center}
|\input{|\textit{main}|}|
\end{center}
%
A simple redirection |\childdocforward{|\textit{dest}|}| is achieved by:
%
\begin{center}
|\def\jobname{|\textit{dest}|}\input{\jobname}|
\end{center}
%
The redirection with prefix
|\childdocforwardprefix[|\textit{prefix}|]{|\textit{dest}|}|
is accomplished by:
%
\begin{center}
\begin{tabular}{l}
|{\edef\jobname{\scantokens\expandafter{\jobname\noexpand}}|\\
|\def\redirectjob |\textit{prefix}|#1~~~{\gdef\jobname{|\textit{dest}|#1}}|\\
|\expandafter\redirectjob\jobname~~~}\input{\jobname}|
\end{tabular}
\end{center}

In an alternative approach,
child documents can be compiled by a specific command line
without additional code or specific definitions:
%
\begin{center}
|... -jobname "|\textit{target}|" "|[\textit{flags}]%
|\includeonly{|\textit{dest}|}\input{|\textit{main}|}"|
\end{center}
%

%%%%%%%%%%%%%%%%%%%%%%%%%%%%%%%%%%%%%%%%%%%%%%%%%%%%%%%%%%%%%%%%%%%%%%%%%%%%%%%%
%%%%%%%%%%%%%%%%%%%%%%%%%%%%%%%%%%%%%%%%%%%%%%%%%%%%%%%%%%%%%%%%%%%%%%%%%%%%%%%%
\section{Information}

%%%%%%%%%%%%%%%%%%%%%%%%%%%%%%%%%%%%%%%%%%%%%%%%%%%%%%%%%%%%%%%%%%%%%%%%%%%%%%%%
\subsection{Copyright}

Copyright \copyright{} 2017--2018 Niklas Beisert

This work may be distributed and/or modified under the
conditions of the \LaTeX{} Project Public License, either version 1.3
of this license or (at your option) any later version.
The latest version of this license is in
  \url{http://www.latex-project.org/lppl.txt}
and version 1.3 or later is part of all distributions of \LaTeX{}
version 2005/12/01 or later.

This work has the LPPL maintenance status `maintained'.

The Current Maintainer of this work is Niklas Beisert.

This work consists of the files |README.txt|, |childdoc.ins| and |childdoc.dtx|
as well as the derived files |childdoc.def|, |cdocsamp.tex|
with |cdocsch1.tex|, |cdocsch2.tex|, |cdocspt3.tex|, |cdocspt4.tex|,
|cdocsdrf.tex|, |cdocsfn1.tex|, |cdocsfn2.tex|
as well as |childdoc.pdf|.

%%%%%%%%%%%%%%%%%%%%%%%%%%%%%%%%%%%%%%%%%%%%%%%%%%%%%%%%%%%%%%%%%%%%%%%%%%%%%%%%
\subsection{Files and Installation}

The package consists of the files:
%
\begin{center}
\begin{tabular}{ll}
    |README.txt|   & readme file \\
    |childdoc.ins| & installation file \\
    |childdoc.dtx| & source file \\
    |childdoc.def| & definition file \\
    |cdocsamp.tex| & sample main file \\
    |cdocsch1.tex| & sample include file \\
    |cdocsch2.tex| & sample include file \\
    |cdocspt3.tex| & sample part file \\
    |cdocspt4.tex| & sample part file \\
    |cdocsdrf.tex| & sample redirection file \\
    |cdocsfn1.tex| & sample redirection file \\
    |cdocsfn2.tex| & sample redirection file \\
    |childdoc.pdf| & manual
\end{tabular}
\end{center}
%
The distribution consists of the files
|README.txt|, |childdoc.ins| and |childdoc.dtx|.
%
\begin{itemize}
\item
Run (pdf)\LaTeX{} on |childdoc.dtx|
to compile the manual |childdoc.pdf| (this file).
\item
Run \LaTeX{} on |childdoc.ins| to create the definitions file |childdoc.def|
and the sample |cdocsamp.tex| with include files
|cdocsch1.tex|, |cdocsch2.tex|, |cdocspt3.tex|, |cdocspt4.tex|,
|cdocsdrf.tex|, |cdocsfn1.tex|, |cdocsfn2.tex|.
Then copy the file |childdoc.def| to an appropriate directory of your \LaTeX{}
distribution, e.g.\ \textit{texmf-root}|/tex/latex/childdoc|.
\end{itemize}

%%%%%%%%%%%%%%%%%%%%%%%%%%%%%%%%%%%%%%%%%%%%%%%%%%%%%%%%%%%%%%%%%%%%%%%%%%%%%%%%
\subsection{Related CTAN Packages}

There are several other packages which offer a similar functionality:
%
\begin{itemize}
\item
The packages
\href{http://ctan.org/pkg/docmute}{\textsf{docmute}},
\href{http://ctan.org/pkg/includex}{\textsf{includex}} and
\href{http://ctan.org/pkg/standalone}{\textsf{standalone}}
provide commands to include only the document body of
a child file thus allowing both files to be compiled individually.
\item
The packages \href{http://ctan.org/pkg/subdocs}{\textsf{subdocs}}
and \href{http://ctan.org/pkg/subfiles}{\textsf{subfiles}}
provide structures in which the main and child documents can be
encapsulated and allowing them to be compiled individually.
The inclusion mechanism is different from the conventional |\include|.
\item
The package \href{http://ctan.org/pkg/combine}{\textsf{combine}}
is an elaborate solution to combine several documents into one.
\end{itemize}
%
See also the CTAN topic \href{http://ctan.org/topic/subdocs}{\textsf{subdocs}}
for further related packages.
The present package differs from the above solutions in that
a document structure constructed with the conventional |\include| mechanism
just needs two extra commands at the top of every file
such that all constituent files can be compiled individually.

%%%%%%%%%%%%%%%%%%%%%%%%%%%%%%%%%%%%%%%%%%%%%%%%%%%%%%%%%%%%%%%%%%%%%%%%%%%%%%%%
%\subsection{Feature Suggestions}
%
%The following is a list of features which may be useful for future
%versions of this package:
%%
%\begin{itemize}
%\item
%\ldots
%\end{itemize}

%%%%%%%%%%%%%%%%%%%%%%%%%%%%%%%%%%%%%%%%%%%%%%%%%%%%%%%%%%%%%%%%%%%%%%%%%%%%%%%%
\subsection{Revision History}

%%%%%%%%%%%%%%%%%%%%%%%%%%%%%%%%%%%%%%%%
\paragraph{v2.0:} 2018/12/30

\begin{itemize}
\item
immediate forward processing
\item
added |\childdocby| mechanism
\item
manual restructured
\end{itemize}

%%%%%%%%%%%%%%%%%%%%%%%%%%%%%%%%%%%%%%%%
\paragraph{v1.6:} 2018/01/17

\begin{itemize}
\item
application for development of include files
\item
corrections to manual
\end{itemize}

%%%%%%%%%%%%%%%%%%%%%%%%%%%%%%%%%%%%%%%%
\paragraph{v1.5:} 2017/05/21

\begin{itemize}
\item
more complete structuring introduced
\item
|\childdocof| introduced
\item
|\childdoc| renamed to |\childdocmain|
\item
|\childredirect| renamed to |\childdocforward| and |\childdocforwardprefix|
and functionality expanded
\end{itemize}

%%%%%%%%%%%%%%%%%%%%%%%%%%%%%%%%%%%%%%%%
\paragraph{v1.0:} 2017/04/27

\begin{itemize}
\item
manual and install package
\item
first version published on CTAN
\end{itemize}

%%%%%%%%%%%%%%%%%%%%%%%%%%%%%%%%%%%%%%%%
\paragraph{v0.6:} 2017/04/26

\begin{itemize}
\item
redirection mechanism added
\end{itemize}

%%%%%%%%%%%%%%%%%%%%%%%%%%%%%%%%%%%%%%%%
\paragraph{v0.5:} 2017/04/26

\begin{itemize}
\item
functionality in definition file
\end{itemize}


%%%%%%%%%%%%%%%%%%%%%%%%%%%%%%%%%%%%%%%%%%%%%%%%%%%%%%%%%%%%%%%%%%%%%%%%%%%%%%%%
%%%%%%%%%%%%%%%%%%%%%%%%%%%%%%%%%%%%%%%%%%%%%%%%%%%%%%%%%%%%%%%%%%%%%%%%%%%%%%%%
%%%%%%%%%%%%%%%%%%%%%%%%%%%%%%%%%%%%%%%%%%%%%%%%%%%%%%%%%%%%%%%%%%%%%%%%%%%%%%%%
\appendix

\settowidth\MacroIndent{\rmfamily\scriptsize 000\ }

 \DocInput{childdoc.dtx}

\end{document}
%</driver>
% \fi
%
% %%%%%%%%%%%%%%%%%%%%%%%%%%%%%%%%%%%%%%%%%%%%%%%%%%%%%%%%%%%%%%%%%%%%%%%%%%%%%%
% %%%%%%%%%%%%%%%%%%%%%%%%%%%%%%%%%%%%%%%%%%%%%%%%%%%%%%%%%%%%%%%%%%%%%%%%%%%%%%
% \section{Sample}
%\iffalse
%<*samplemain>
%\fi
%
% The following presents a sample document
% with two chapters, two parts, a title page,
% a compile flag as well as three forwarding files to set the flag.
% It consists of eight |.tex| files:
% \begin{center}
% \begin{tabular}{ll}
% |cdocsamp.tex|&main file\\
% |cdocsch1.tex|&include file for chapter 1\\
% |cdocsch2.tex|&include file for chapter 2\\
% |cdocspt3.tex|&include file for part 3\\
% |cdocspt4.tex|&include file for part 4\\
% |cdocsdrf.tex|&forwarding file for main file in draft mode\\
% |cdocsfi1.tex|&forwarding file for final version of chapter 1\\
% |cdocsfi2.tex|&forwarding file for final version of chapter 2\\
% \end{tabular}
% \end{center}
% Each of the eight files can be compiled directly by the \LaTeX{} compiler.
%
% %%%%%%%%%%%%%%%%%%%%%%%%%%%%%%%%%%%%%%
% \paragraph{Main File.}
%
% The main file is called |cdocsamp.tex|.
%
% Load the \textsf{childdoc} definitions and
% declare the filename for the main document:
%    \begin{macrocode}
\input{childdoc.def}
\childdocmain{}
%    \end{macrocode}

% Optional override for |\version| flag:
%    \begin{macrocode}
%%\ifchilddoc\else\providecommand{\version}{draft}\fi
%    \end{macrocode}

% Define the default values for the |\version| flag
% (|final| for the main file and |draft| for childs):
%    \begin{macrocode}
\ifchilddoc
\providecommand{\version}{draft}
\else
\providecommand{\version}{final}
\fi
%    \end{macrocode}

% Load the standard document class:
%    \begin{macrocode}
\documentclass[12pt]{article}
%    \end{macrocode}

% Start the document body:
%    \begin{macrocode}
\begin{document}
%    \end{macrocode}

% Declare a title page.
% Print title, part of document being processed and version flag:
%    \begin{macrocode}
\addtocounter{page}{-1}
\begin{center}
{\LARGE\bfseries{}childdoc example\par}
\vspace{1cm}
\ifchilddoc
\ifchilddocmanual part\else chapter\fi:
`\childdocname' of `\childdocjob'\par
\else
main document: `\childdocjob'\par
\fi
version: \version\par
\end{center}
\newpage
%    \end{macrocode}

% Manually include selected file,
% otherwise process as usual:
%    \begin{macrocode}
\ifchilddocmanual
\section*{part `\childdocname'}
\input{\childdocname}
\else
%    \end{macrocode}

% Include the two chapters:
%    \begin{macrocode}
\include{cdocsch1}
\include{cdocsch2}
%    \end{macrocode}

% Include the two parts unless only chapters should be displayed:
%    \begin{macrocode}
\ifchilddoc\else
\section{part three}
\input{cdocspt3}
\section{part four}
\input{cdocspt4}
\fi
%    \end{macrocode}

% Process as usual until here:
%    \begin{macrocode}
\fi
%    \end{macrocode}

% End of document body:
%    \begin{macrocode}
\end{document}
%    \end{macrocode}
%\iffalse
%</samplemain>
%\fi
%
% %%%%%%%%%%%%%%%%%%%%%%%%%%%%%%%%%%%%%%
% \paragraph{Chapter Include Files.}
%
% The include files are called |cdocsch1.tex| and |cdocsch2.tex|.
%
%\iffalse
%<*samplechap1|samplechap2>
%\fi

% Optional override for |\version| flag:
%    \begin{macrocode}
%%\providecommand{\version}{final}
%    \end{macrocode}

% Include the main document:
%    \begin{macrocode}
\input{childdoc.def}
\childdocof{cdocsamp}
%    \end{macrocode}

%\iffalse
%</samplechap1|samplechap2>
%\fi
%
%\iffalse
%<*samplechap1>
%\fi
% Some text for chapter 1:
%    \begin{macrocode}
\section{one}
some text in chapter one
%    \end{macrocode}

%\iffalse
%</samplechap1>
%\fi
% Some text for chapter 2:
%\iffalse
%<*samplechap2>
%\fi
%    \begin{macrocode}
\section{two}
more text in chapter two
%    \end{macrocode}

%\iffalse
%</samplechap2>
%\fi
%
% %%%%%%%%%%%%%%%%%%%%%%%%%%%%%%%%%%%%%%
% \paragraph{Part Include Files.}
%
% The include files are called |cdocspt3.tex| and |cdocspt4.tex|.
%
%\iffalse
%<*samplepart3|samplepart4>
%\fi

% Optional override for |\version| flag:
%    \begin{macrocode}
%%\providecommand{\version}{final}
%    \end{macrocode}

% Include the main document:
%    \begin{macrocode}
\input{childdoc.def}
\childdocby{cdocsamp}
%    \end{macrocode}

%\iffalse
%</samplepart3|samplepart4>
%\fi
%
%\iffalse
%<*samplepart3>
%\fi
% Some text for part 3:
%    \begin{macrocode}
some text in part three
%    \end{macrocode}

%\iffalse
%</samplepart3>
%\fi
% Some text for part 4:
%\iffalse
%<*samplepart4>
%\fi
%    \begin{macrocode}
more text in part four
%    \end{macrocode}

%\iffalse
%</samplepart4>
%\fi
%
% %%%%%%%%%%%%%%%%%%%%%%%%%%%%%%%%%%%%%%
% \paragraph{Forwarding for a Complete Draft.}
%
% The following forwarding file |cdocsdrf.tex|
% compiles the main document in draft mode:
%\iffalse
%<*sampledraft>
%\fi
%    \begin{macrocode}
\def\version{draft}
\input{childdoc.def}
\childdocforward{cdocsamp}
%    \end{macrocode}

%\iffalse
%</sampledraft>
%\fi
%
% %%%%%%%%%%%%%%%%%%%%%%%%%%%%%%%%%%%%%%
% \paragraph{Forwarding for Final Version of the Chapters.}
%
% The following forwarding files |cdocsfn1.tex| and |cdocsfn2.tex|
% (with identical content)
% compile the final versions of the child documents
% |cdocsch1.tex| and |cdocsch2.tex|, respectively:
%\iffalse
%<*samplefinal>
%\fi
%    \begin{macrocode}
\def\version{final}
\input{childdoc.def}
\childdocforwardprefix[cdocsamp]{cdocsfn}{cdocsch}
%    \end{macrocode}

%\iffalse
%</samplefinal>
%\fi
%
% %%%%%%%%%%%%%%%%%%%%%%%%%%%%%%%%%%%%%%
% \paragraph{Command Line Processing.}
%
% The following three command lines generate the output files
% |cdocscld|, |cdocscl1| and |cdocscl2|
% which should be identical to
% |cdocsdrf|, |cdocsch1| and |cdocsfn2|, respectively:
% \begin{center}
% \begin{tabular}{l}
% |latex -jobname cdocscld \|\\
% |  "\def\version{draft}\input{childdoc.def}\childdocforward{cdocsamp}"|\\
% |latex -jobname cdocscl1 \|\\
% |  "\input{childdoc.def}\childdocforward[cdocsamp]{cdocsch1}"|\\
% |latex -jobname cdocscl2 \|\\
% |  "\def\version{final}\input{childdoc.def}\childdocforward{cdocsch2}"|
% \end{tabular}
% \end{center}
% Note that the trailing backslash on each first line
% merely continues the input to the second line
% (for convenient cut ant paste).
% Furthermore, the command |latex| can be replaced by any
% of its alternative versions such as |pdflatex|.
%
% %%%%%%%%%%%%%%%%%%%%%%%%%%%%%%%%%%%%%%%%%%%%%%%%%%%%%%%%%%%%%%%%%%%%%%%%%%%%%%
% %%%%%%%%%%%%%%%%%%%%%%%%%%%%%%%%%%%%%%%%%%%%%%%%%%%%%%%%%%%%%%%%%%%%%%%%%%%%%%
% \section{Implementation}
%\iffalse
%<*package>
%\fi
%
% This section describes the definitions file |childdoc.def|.

% The definitions cannot be loaded using |\usepackage| or |\RequirePackage|
% which has a mechanism to prevent loading a style file more than once.
% When loading the definitions by means of |\input|
% multiple instances have to be prevented manually:
%\iffalse
%This code needs to be before the `\ProvidesFile' directive
%which is defined at the beginning of this file.
%Therefore it is also placed there and commented out here.
%</package>
%<*discard>
%\fi
%    \begin{macrocode}
\ifdefined\childdocmain\endinput\fi
%    \end{macrocode}
%\iffalse
%</discard>
%<*package>
%\fi
%
% \macro{\ifchilddoc}
% \macro{\ifchilddocmanual}
% The conditional |\ifchilddoc| tells whether a
% child (true) or main (false) document is being compiled.
% The conditional |\ifchilddocmanual| tells whether
% the |\includeonly| mechanism is used (false) or
% the selection of child files must be performed manually (true).
% The definitions initialise to false:
%    \begin{macrocode}
\newif\ifchilddoc
\newif\ifchilddocmanual
%    \end{macrocode}

% \macro{\childdocname}
% \macro{\childdocjob}
% The macro |\childdocname| stores the name of the main document
% to be compiled. The macro |\childdocjob| stores the name of
% the document on which the \LaTeX{} compiler was originally invoked.
% The content of |\jobname| cannot be compared
% to filenames specified in the source due to different catcodes.
% The following code rescans |\jobname|, stores the result
% in |\childdocname| and saves a copy in |\childdocjob|:
%    \begin{macrocode}
\edef\childdocname{\scantokens\expandafter{\jobname\noexpand}}
\let\childdocjob\childdocname
%    \end{macrocode}

% \macro{\childdocdisable}
% The macro |\childdocdisable| prevents the main file
% from being processed more than once.
% At this stage, the main document command |\childdocmain|
% is assumed to be called once again where it should do nothing.
% Any subsequent call to it should prevent
% a secondary processing of the main document
% It overwrites the forwarding commands
% |\childdocof| and |\childdocforward|
% with empty macros to prevent further inclusions of the main document:
%    \begin{macrocode}
\newcommand{\childdocdisable}
{
  \renewcommand{\childdocmain}[1]{\renewcommand{\childdocmain}[1]{\endinput}}
  \renewcommand{\childdocof}[1]{}
  \renewcommand{\childdocby}[2][]{}
  \renewcommand{\childdocforward}[2][]{}
  \renewcommand{\childdocdisable}{}
}
%    \end{macrocode}

% \macro{\childdocmain}
% The macro |\childdocmain| is to be called at the top of the main file
% with nothing or the main filename (without extension) as argument.
% First, it breaks loops.
% If the argument is not empty and does not match |\childdocname|
% (which is set by the first inclusion of |childdoc.def|),
% |\ifchilddoc| is set to true, |\includeonly| is applied to the child file
% and |\jobname| is set to the main file
% (for proper handling of |.aux| files):
%    \begin{macrocode}
\newcommand{\childdocmain}[1]
{
  \childdocdisable\childdocmain{}
  \if?#1?\else
    \begingroup
      \def\childdoctmp{#1}
      \ifx\childdoctmp\childdocname
        \def\childdoctmp{}
      \else
        \def\childdoctmp
        {
          \childdoctrue
          \includeonly{\childdocname}
          \def\childdocjob{#1}
          \def\jobname{#1}
        }
      \fi
      \expandafter
    \endgroup
    \childdoctmp
  \fi
}
%    \end{macrocode}

% \macro{\childdocof}
% The command |\childdocof| redirects
% compilation to the main file |#1|.
%    \begin{macrocode}
\newcommand{\childdocof}[1]
{
  \childdocdisable
  \childdoctrue
  \includeonly{\childdocname}
  \def\jobname{#1}
  \def\childdocjob{#1}
  \input{#1}
}
%    \end{macrocode}

% \macro{\childdocby}
% The command |\childdocby| ....
%    \begin{macrocode}
\newcommand{\childdocby}[2][]
{
  \childdocdisable
  \childdoctrue
  \childdocmanualtrue
  \if?#1?\else
    \def\jobname{#2}
  \fi
  \def\childdocjob{#2}
  \input{#2}
  \endinput
}
%    \end{macrocode}

% \macro{\childdocforward}
% The command |\childdocforward| redirects
% compilation to the main file or
% (if the optional argument is given) a child file.
% Parameters are set as if the main file
% or a child file starting with |\childdocof| was compiled.
% Then compilation is handed over to the main file:
%    \begin{macrocode}
\newcommand{\childdocforward}[2][]
{
  \begingroup
    \if?#1?
      \def\childdoctmp
      {
        \def\childdocname{#2}
        \def\childdocjob{#2}
        \def\jobname{#2}
        \input{#2}
        \endinput
      }
    \else
      \def\childdoctmp
      {
        \childdocdisable
        \def\childdocname{#2}
        \childdoctrue
        \includeonly{#2}
        \def\childdocjob{#1}
        \def\jobname{#1}
        \input{#1}
        \endinput
      }
    \fi
    \expandafter
  \endgroup
  \childdoctmp
}
%    \end{macrocode}

% \macro{\childdocforwardprefix}
% The command |\childdocforwardprefix| redirects
% compilation to the main or a child file by means of a pattern.
% The prefix |#1| in the current filename is replaced by |#2|
% and the suffix of the current filename is kept
% (it is assumed that the filename does not contain the substring `|~~~|'
% which is used as a delimiter).
% Compilation is handed over to the new file by |\childdocforward|:
%    \begin{macrocode}
\newcommand{\childdocforwardprefix}[3][]
{
  \begingroup
    \def\childdocextract #2##1~~~{\def\childdoctmp{\childdocforward[#1]{#3##1}}}
    \expandafter\childdocextract\childdocname~~~
    \expandafter
  \endgroup
  \childdoctmp
}
%    \end{macrocode}

% \macro{\childdoc}
% The deprecated macro |\childdoc| is a legacy version of |\childdocmain|:
%    \begin{macrocode}
\newcommand{\childdoc}{\childdocmain}
%    \end{macrocode}

% \macro{\childdocredirect}
% The deprecated macro |\childdocredirect| is a legacy version
% of |\childdocforward| and |\childdocforwardprefix|:
%    \begin{macrocode}
\newcommand{\childdocredirect}[2][]
{
  \begingroup
    \if?#1?
      \def\childdoctmp{\childdocforward{#2}}
    \else
      \def\childdoctmp{\childdocforwardprefix{#1}{#2}}
    \fi
    \expandafter
  \endgroup
  \childdoctmp
}
%    \end{macrocode}

%\iffalse
%</package>
%\fi
%
\endinput

\childdocforward{cdocsamp}
%    \end{macrocode}

%\iffalse
%</sampledraft>
%\fi
%
% %%%%%%%%%%%%%%%%%%%%%%%%%%%%%%%%%%%%%%
% \paragraph{Forwarding for Final Version of the Chapters.}
%
% The following forwarding files |cdocsfn1.tex| and |cdocsfn2.tex|
% (with identical content)
% compile the final versions of the child documents
% |cdocsch1.tex| and |cdocsch2.tex|, respectively:
%\iffalse
%<*samplefinal>
%\fi
%    \begin{macrocode}
\def\version{final}
% \iffalse
%
% childdoc.dtx Copyright (C) 2017-2018 Niklas Beisert
%
% This work may be distributed and/or modified under the
% conditions of the LaTeX Project Public License, either version 1.3
% of this license or (at your option) any later version.
% The latest version of this license is in
%   http://www.latex-project.org/lppl.txt
% and version 1.3 or later is part of all distributions of LaTeX
% version 2005/12/01 or later.
%
% This work has the LPPL maintenance status `maintained'.
%
% The Current Maintainer of this work is Niklas Beisert.
%
% This work consists of the files childdoc.dtx and childdoc.ins
% and the derived files childdoc.def and cdocsamp.tex with
% cdocsch1.tex, cdocsch2.tex, cdocsdrf.tex, cdocsfn1.tex, cdocsfn2.tex.
%
%<package>\ifdefined\childdocmain\endinput\fi
%<package>\ProvidesFile{childdoc.def}[2018/12/30 v2.0 child document driver]
%<samplemain>\ProvidesFile{cdocsamp.tex}[2018/12/30 v2.0 sample for childdoc]
%<*driver>
%\ProvidesFile{childdoc.drv}[2018/12/30 v2.0 childdoc reference manual file]
\PassOptionsToClass{10pt,a4paper}{article}
\documentclass{ltxdoc}

\usepackage[margin=35mm]{geometry}
\usepackage{hyperref}
\usepackage{hyperxmp}
\usepackage[usenames]{color}

\hypersetup{colorlinks=true}
\hypersetup{pdfstartview=FitH}
\hypersetup{pdfpagemode=UseNone}
\hypersetup{pdfsource={}}
\hypersetup{pdflang={en-UK}}
\hypersetup{pdfcopyright={Copyright 2017-2018 Niklas Beisert.
  This work may be distributed and/or modified under the
  conditions of the LaTeX Project Public License, either version 1.3
  of this license or (at your option) any later version.}}
\hypersetup{pdflicenseurl={http://www.latex-project.org/lppl.txt}}
\hypersetup{pdfcontactaddress={ETH Zurich, ITP, HIT K,
  Wolfgang-Pauli-Strasse 27}}
\hypersetup{pdfcontactpostcode={8093}}
\hypersetup{pdfcontactcity={Zurich}}
\hypersetup{pdfcontactcountry={Switzerland}}
\hypersetup{pdfcontactemail={nbeisert@itp.phys.ethz.ch}}
\hypersetup{pdfcontacturl={http://people.phys.ethz.ch/\xmptilde nbeisert/}}

\newcommand{\secref}[1]{\hyperref[#1]{section \ref*{#1}}}

\parskip1ex
\parindent0pt
\let\olditemize\itemize
\def\itemize{\olditemize\parskip0pt}

\begin{document}

\title{The \textsf{childdoc} Package}
\hypersetup{pdftitle={The childdoc Package}}
\author{Niklas Beisert\\[2ex]
  Institut f\"ur Theoretische Physik\\
  Eidgen\"ossische Technische Hochschule Z\"urich\\
  Wolfgang-Pauli-Strasse 27, 8093 Z\"urich, Switzerland\\[1ex]
  \href{mailto:nbeisert@itp.phys.ethz.ch}
  {\texttt{nbeisert@itp.phys.ethz.ch}}}
\hypersetup{pdfauthor={Niklas Beisert}}
\hypersetup{pdfsubject={Manual for the LaTeX2e Package childdoc}}
\date{30 December 2018, \textsf{v2.0}}
\maketitle

\begin{abstract}\noindent
\textsf{childdoc} is a \LaTeXe{} package
that enables the direct compilation
of document sections included by |\include|
to individual files.
\end{abstract}

\begingroup
\parskip0ex
\tableofcontents
\endgroup

%%%%%%%%%%%%%%%%%%%%%%%%%%%%%%%%%%%%%%%%%%%%%%%%%%%%%%%%%%%%%%%%%%%%%%%%%%%%%%%%
%%%%%%%%%%%%%%%%%%%%%%%%%%%%%%%%%%%%%%%%%%%%%%%%%%%%%%%%%%%%%%%%%%%%%%%%%%%%%%%%
\section{Introduction}

\LaTeX{} provides a mechanism to structure a large document (such as a book)
into a main file and several child files (containing the chapters)
using the |\include| command.
This mechanism is beneficial for documents
which span hundreds of pages in order to
make the source file(s) more manageable.
Moreover, compilation can be restricted to
selected child files by means of the |\includeonly| command.
The latter feature can be used to reduce the compilation time while editing
(this was significantly more useful in the earlier days of \LaTeX{})
or to generate a smaller document which is easier to navigate.
Another application of |\includeonly| is to generate
documents consisting of selected parts of the complete document.

However, there are a few drawbacks of the plain |\include| mechanism:
\begin{itemize}
\item
The child files cannot be compiled on their own,
they can only be compiled via the main file.
A naive editing environment
(such as a text editor with an option
to have the current file processed by \LaTeX)
may require one to switch to the main file before compiling;
attempting to compile the child file produces errors.
\item
The main file must be modified (each time)
to adjust the |\includeonly| command
to the present needs. This easily leaves the main file in a messy state.
\item
The generated document will always carry the filename
of the main document. This is inconvenient if
several child files are to be compiled and
to be kept for distribution.
\end{itemize}

The present package provides a simple interface
to make child files individually compilable by \LaTeX{}.
Compiling a child file then has the same effect as compiling
the main file with an |\includeonly| command
to select the appropriate child.
Moreover the generated document will carry the name of the child
rather than the main file.
This resolves all three above issues.

This feature is meant to make the editing of books,
thesis documents and lecture notes somewhat more convenient.
However, the package can also be used efficiently for
composing a series of documents (such as exercise sheets)
which are typically distributed individually.
It then assists the author in generating the individual documents
(potentially in different versions)
as well as a document containing the collected series.
Another application is in developing style files
or other kinds of included material
where compilation of the style file could redirect
to a sample or test file.

%%%%%%%%%%%%%%%%%%%%%%%%%%%%%%%%%%%%%%%%%%%%%%%%%%%%%%%%%%%%%%%%%%%%%%%%%%%%%%%%
%%%%%%%%%%%%%%%%%%%%%%%%%%%%%%%%%%%%%%%%%%%%%%%%%%%%%%%%%%%%%%%%%%%%%%%%%%%%%%%%
\section{Usage}

First of all, the package \textsf{childdoc} is \emph{not} a standard
\LaTeXe{} |.sty| style file! Therefore it needs to be invoked in
a non-standard way.

%%%%%%%%%%%%%%%%%%%%%%%%%%%%%%%%%%%%%%%%%%%%%%%%%%%%%%%%%%%%%%%%%%%%%%%%%%%%%%%%
\subsection{Included Files}
\label{sec:include}

%%%%%%%%%%%%%%%%%%%%%%%%%%%%%%%%%%%%%%%%
\DescribeMacro{\childdocmain}
To use the package, add the commands
\begin{center}
\begin{tabular}{l}
|\input{childdoc.def}|\\
|\childdocmain{}|\\
\end{tabular}
\end{center}
at the very top of the main \LaTeX{} file,
in particular \emph{before} the |\documentclass| statement!
The argument of |\childdocmain| should be left empty
(but it must be present).

%%%%%%%%%%%%%%%%%%%%%%%%%%%%%%%%%%%%%%%%
\DescribeMacro{\childdocof}
Furthermore, add the commands
\begin{center}
\begin{tabular}{l}
|\input{childdoc.def}|\\
|\childdocof{|\textit{main}|}|\\
\end{tabular}
\end{center}
at the top of every child file \textit{child}
which is included by |\include{|\textit{child}|}|
from within the main file
(or at least for those files to be compiled individually).
The argument \textit{main} must be the filename of the main file.

There are a couple of
considerations in setting up the main and child documents:

%%%%%%%%%%%%%%%%%%%%%%%%%%%%%%%%%%%%%%%%
\paragraph{Restrictions.}

Please note the following restrictions:
\begin{itemize}
\item
|\childdocmain| must be called with one argument \textit{main}
to ensure compatibility with earlier version of the package.
It must either be empty (|\childdocmain{}|)
or precisely match the filename of the main file in which it is specified.
See \secref{sec:detection} for further information.
\item
The filename \textit{main} must be specified without the |.tex| extension.
\item
The filename \textit{main} is case sensitive
(even in case-insensitive file systems)
due to internal string comparison.
\item
The argument \textit{main} should be fully expanded, it cannot be a macro.
\item
Subdirectories and special characters should be avoided in filenames.
\item
The command |\childdocmain{|\textit{main}|}| must be followed by a whitespace.
It should not be followed immediately by another command
or by a comment mark `|%|'.
This is because the \TeX{} parser reads the token immediately following
the argument of |\childdocmain| and puts it
at the beginning of every child section;
however, a white\-space is ignored.
\end{itemize}

%%%%%%%%%%%%%%%%%%%%%%%%%%%%%%%%%%%%%%%%
\paragraph{Content of Main File.}

It is advisable to place all content in the child files included by |\include|.
Any output contained in the main file will appear in all child documents
unless suppressed manually;
it cannot be suppressed automatically by the |\includeonly| directive
and thus should normally be avoided.
A method to include some content in the main file
by means of conditional processing is described in \secref{sec:conditional}.

%%%%%%%%%%%%%%%%%%%%%%%%%%%%%%%%%%%%%%%%
\paragraph{Page Numbering.}

When only a part of the document is compiled,
the appropriate numbering of pages
(as well as other status parameters)
is determined from the |.aux| files.
The latter contain information from previous passes.
However this information needs to propagate through
all intermediate child documents.
Therefore the page numbering in child documents may well
be inconsistent until the complete document is compiled at least once.

A useful (if unconventional) way to always ensure a consistent
page numbering is to restart the numbering in each child document
and denote the pages by `\textit{child}|.|\textit{page}'
where \textit{child} represents the chapter/section number of the child file.
This can be achieved by the command
|\numberwithin{page}{|\textit{child}|}|
of the \textsf{amsmath} package
where \textit{child} can be |chapter| or |section|
depending on the chosen structuring.
Alternatively, one can modify the macro |\thepage| appropriately
and reset the counter |page| at the start of each child file.

%%%%%%%%%%%%%%%%%%%%%%%%%%%%%%%%%%%%%%%%%%%%%%%%%%%%%%%%%%%%%%%%%%%%%%%%%%%%%%%%
\subsection{Conditional Processing}
\label{sec:conditional}

The package provides a mechanism to compile different versions
of a document. To customise the versions further some conditional processing
can come in handy to distinguish which version is being compiled.
The package provides two macros to describe the compilation context:

%%%%%%%%%%%%%%%%%%%%%%%%%%%%%%%%%%%%%%%%
\DescribeMacro{\ifchilddoc}
The conditional |\ifchilddoc| distinguishes between the compilation of
child documents and the main document:
%
\begin{center}
|\ifchilddoc |\textit{child-code}| |[|\||else |\textit{main-code}]| \||fi|
\end{center}

%%%%%%%%%%%%%%%%%%%%%%%%%%%%%%%%%%%%%%%%
\DescribeMacro{\childdocname}
\DescribeMacro{\childdocjob}
The macro |\childdocname| contains the filename (without extension)
of the main or child file being processed.
Note that |\childdocjob| will always contain the name of the main file.

%%%%%%%%%%%%%%%%%%%%%%%%%%%%%%%%%%%%%%%%
\paragraph{Title Page.}

Conditional processing can be used to include a title or banner page
in the main document when proper precautions are taken.
Importantly, the code in the main file should ensure that the page counter
(as well as other status parameters which are stored in the |.aux| files)
takes the same value after the conditional processing.
Otherwise the page numbers may take divergent values
depending on which part is compiled.

For example, a title page could be declared by:
%
\begin{center}
\begin{tabular}{l}
|\ifchilddoc\||else|\\
|\addtocounter{page}{-1}|\\
\textit{code for title page}\\
|\newpage|\\
|\||fi|
\end{tabular}
\end{center}
%
A banner page for the child documents can be generated by:
%
\begin{center}
\begin{tabular}{l}
|\ifchilddoc|\\
|\addtocounter{page}{-1}|\\
\textit{code for banner page}\\
|\newpage|\\
|\||fi|
\end{tabular}
\end{center}
%
Here one could write a message such as:
\begin{center}
|This is the part \childdocname{} of \childdocjob{}.|
\end{center}

%%%%%%%%%%%%%%%%%%%%%%%%%%%%%%%%%%%%%%%%%%%%%%%%%%%%%%%%%%%%%%%%%%%%%%%%%%%%%%%%
\subsection{Flags}
\label{sec:flags}

The package makes it easy to generate different versions
of the main or child documents.
To this end compilation flags can be defined
and assigned different default values.
They will be particularly useful in conjunction
with the forwarding mechanism described in \secref{sec:forward}.

For example, it may be useful to have a flag |\version|
which can be set to |draft| or |final|.
The document source will contain some conditional code
depending on the value of |\version|.
Suppose further, the flag should default to |final| for the main file
and to |draft| for child files
which is a natural assignment for editing the document.
This is achieved by placing the following code
in the preamble of the main document
(below the |\childdocmain| directive):
%
\begin{center}
\begin{tabular}{l}
|\ifchilddoc|\\
|\providecommand{\version}{draft}|\\
|\||else|\\
|\providecommand{\version}{final}|\\
|\||fi|
\end{tabular}
\end{center}
%
The definition by |\providecommand| makes sure
that previous definitions are not overwritten.
Further statements |\providecommand{\version}{...}|
can thus be added before the above code to override it.

For the main file, one might add a line
(between |\childdocmain| and the above block)
%
\begin{center}
|%\ifchilddoc\||else\providecommand{\version}{draft}\||fi|
\end{center}
%
which can be uncommented to produce a draft version.
Likewise one can add a line to the very top of a child file
(above the |\childdocof{|\textit{main}|}| directive)
%
\begin{center}
|%\providecommand{\version}{final}|
\end{center}
%
which can be uncommented to produce the final version of this child document.

%%%%%%%%%%%%%%%%%%%%%%%%%%%%%%%%%%%%%%%%%%%%%%%%%%%%%%%%%%%%%%%%%%%%%%%%%%%%%%%%
\subsection{Forwarding}
\label{sec:forward}

Different versions of the main or child documents
using compilation flags as described in \secref{sec:flags}
can be (permanently) stored in different files
for convenient compilation, viewing and distribution.
To this end, the package defines a command
to pass on compilation to a different file:

%%%%%%%%%%%%%%%%%%%%%%%%%%%%%%%%%%%%%%%%
\DescribeMacro{\childdocforward}
The command |\childdocforward| redirects processing to
another source file:
%
\begin{center}
\begin{tabular}{l}
|\input{childdoc.def}|\\
|\childdocforward[|\textit{main}|]{|\textit{dest}|}|\\
\end{tabular}
\end{center}
%
The argument \textit{dest} is the destination file
(without extension).
It should be the main file or one of the child files.
Note that further \textsf{childdoc} directives
such as |\childdocof| and |\childdocforward|
in the indicated file will be processed in this form.
The optional argument \textit{main}
passes on directly to the main file \textit{main}
while pretending to compile the child \textit{dest}.
This form behaves as if \textit{dest}
issues |\childdocof{|\textit{main}|}| right away,
and no further \textsf{childdoc} directives will be processed.

%%%%%%%%%%%%%%%%%%%%%%%%%%%%%%%%%%%%%%%%
\DescribeMacro{\...prefix}
In the alternative form |\childdocforwardprefix|,
%
\begin{center}
\begin{tabular}{l}
|\input{childdoc.def}|\\
|\childdocforwardprefix[|\textit{main}|]{|\textit{prefix}|}{|\textit{dest}|}|
\end{tabular}
\end{center}
%
the destination file is determined by a pattern
depending on the current file:
To make this work, the current file must be called
`{\textit{prefix}\hspace{0.2em}\textit{suffix}}'
with \textit{prefix} matching precisely the argument.
Processing is then passed on to the file
`{\textit{dest}\hspace{0.2em}\textit{suffix}}'.
Surely, the same effect is achieved by
directly specifying the
argument `{\textit{dest}\hspace{0.2em}\textit{suffix}}'
in the first form.
However, that requires to set up a different file
for each child. With the alternative form of the command
all these files can have exactly the same content
which simplifies setting them up and maintaining them.

For example, the following file |draft.tex|
with a compilation flag |\version| as described in \secref{sec:flags}
compiles the main document as a draft:
%
\begin{center}
\begin{tabular}{l}
|\def\version{draft}|\\
|\input{childdoc.def}|\\
|\childdocforward{|\textit{main}|}|
\end{tabular}
\end{center}
%
Likewise, the following files |final|\textit{nn}|.tex|
compile the final version of the child document
|child|\textit{nn}|.tex|:
%
\begin{center}
\begin{tabular}{l}
|\def\version{final}|\\
|\input{childdoc.def}|\\
|\childdocforwardprefix{final}{child}|
\end{tabular}
\end{center}
%

Note that when several versions of a main file and/or of each child file
are to be generated, it may be convenient to set up a |Makefile| or
shell script to automatise the process.

%%%%%%%%%%%%%%%%%%%%%%%%%%%%%%%%%%%%%%%%%%%%%%%%%%%%%%%%%%%%%%%%%%%%%%%%%%%%%%%%
\subsection{Command Line Processing}
\label{sec:commandline}

The effect of redirection files can also be achieved by invoking
the \LaTeX{} compiler with a more elaborate command line.
Most conveniently this should be done as part
of a shell script or a |Makefile|.

When using \textsf{childdoc} in the main file, the following
command lines effectively perform a redirection
(note that depending on the shell being used,
backslashes may have to be doubled: `|\|' $\to$ `|\\|'):
%
\begin{center}
|... -jobname "|\textit{target}|" |\\|"|[\textit{flags}]%
|\input{childdoc.def}\childdocforward[|\textit{main}|]{|\textit{dest}|}"|
\end{center}
%
Here \textit{target} is the name of the output file,
\textit{main} is the name of the main file
and \textit{dest} is the name of the main or child file to be processed
(all filenames without extensions).
The optional argument \textit{main} can be omitted
if \textit{main} matches \textit{dest}.
Optionally, compilation \textit{flags} can be defined via |\def| commands.
This command line makes the \TeX{} engine believe
it is compiling the file \textit{target}
whose content is specified as the latter parameter.
The provided code then forwards the processing to
\textit{main} or \textit{dest} as described in \secref{sec:forward}.

%%%%%%%%%%%%%%%%%%%%%%%%%%%%%%%%%%%%%%%%%%%%%%%%%%%%%%%%%%%%%%%%%%%%%%%%%%%%%%%%
\subsection{Include by Input}
\label{sec:input}

Including child documents by |\include| has some restrictions by design.
Most notably, the content of a child document always occupies
its own set of pages; pages cannot be shared between child documents.
Usually, this behaviour makes perfect sense
because each child document contain an essential part of the document.
However, in some situations it may be desirable to compose
a document from a collection of parts
without having mandatory page breaks between then.
For this case, the package
provides a mechanism to include parts
by |\input| which can also be processed individually.
However, by construction this mechanism
requires manual handling of the content to be output.

%%%%%%%%%%%%%%%%%%%%%%%%%%%%%%%%%%%%%%%%
\DescribeMacro{\ifchilddocmanual}
The main file should be prepared as usual, see \secref{sec:include}.
However, the document body must make a distinction
between processing of an individual part and of the main document, e.g.:
%
\begin{center}
\begin{tabular}{l}
|\ifchilddocmanual|\\
|\input{\childdocname}|\\
|\||else|\\
\textit{document body with }|\input{|\textit{part}|}|\\
|\||fi|
\end{tabular}
\end{center}
%
The conditional |\ifchilddocmanual| is true whenever
a part to be included by |\input| is being compiled,
and the name of the part is stored in |\childdocname|.

%%%%%%%%%%%%%%%%%%%%%%%%%%%%%%%%%%%%%%%%
\DescribeMacro{\childdocby}
Each part to be included by |\input| should start with:
%
\begin{center}
\begin{tabular}{l}
|\input{childdoc.def}|\\
|\childdocby{|\textit{main}|}|\\
\end{tabular}
\end{center}
%
The directive |\childdocby| is similar to |\childdocof|
described in \secref{sec:include},
but the subsequent selection of content must be done manually.
To that end, both |\ifchilddoc| and |\ifchilddocmanual|
will be true upon processing of a part,
and the name of the part is stored in |\childdocname|.
Note that |\jobname| will be set to the filename of the current part
so that each part receives an individual |.aux| file
that does not interfere with the |.aux| file(s) of the main document.
This behaviour can be altered by the alternative form
|\childdocby[*]{|\textit{main}|}| (with a non-empty optional argument)
which uses the |.aux| file of the main document
by setting |\jobname| to \textit{main}.

%%%%%%%%%%%%%%%%%%%%%%%%%%%%%%%%%%%%%%%%%%%%%%%%%%%%%%%%%%%%%%%%%%%%%%%%%%%%%%%%
\subsection{Driver Development}
\label{sec:driver}

The \textsf{childdoc} mechanism can also be use for the development
of definition files such as \LaTeX{} styles or classes.
This case differs from the above setup with multiple parts
included by |\include| in that no |\includeonly| should be invoked.
This can be achieved by starting the include file
(before |\ProvidesPackage|) with:
%
\begin{center}
\begin{tabular}{l}
|\input{childdoc.def}|\\
|\childdocforward{|\textit{main}|}|\\
\end{tabular}
\end{center}
%
or alternatively with:
%
\begin{center}
\begin{tabular}{l}
|\input{childdoc.def}|\\
|\childdocby{|\textit{main}|}|\\
\end{tabular}
\end{center}
%
Both forms have slightly different effects as described above.
The main file is prepared as usual, see \secref{sec:include}.

%%%%%%%%%%%%%%%%%%%%%%%%%%%%%%%%%%%%%%%%%%%%%%%%%%%%%%%%%%%%%%%%%%%%%%%%%%%%%%%%
\subsection{Legacy Detection}
\label{sec:detection}

The directive |\childdocmain| in the main file can detect
whether the complete document or merely a child is to be compiled
even without using the directive |\childdocof|.
This method is deprecated because it is less robust
and there is no compelling reason to use it;
it is merely provided for backward compatibility
and it may be removed in future versions.

If the detection mechanism is to be used,
it is mandatory to correctly specify
the filename of the main file as the argument of |\childdocmain|:
%
\begin{center}
\begin{tabular}{l}
|\input{childdoc.def}|\\
|\childdocmain{|\textit{main}|}|\\
\end{tabular}
\end{center}
%
If |\jobname| does not match the argument \textit{main} of |\childdocmain|,
it is assumed that |\jobname| points to the child file to be compiled.
When using |\childdocmain| with the main file specified as argument,
it suffices to start a child file
with just |\input{|\textit{main}|}|
without loading of the package and using |\childdocof|.
If instead all processing is done
with the appropriate \textsf{childdoc} directives,
the argument of \textit{main} of |\childdocmain| can be empty.

An alternative version of the command line processing described
in \secref{sec:commandline} using the detection mechanism reads:
%
\begin{center}
|... -jobname "|\textit{target}|" "|[\textit{flags}]%
[|\def\jobname{|\textit{dest}|}|]|\input{|\textit{main}|}"|
\end{center}

%%%%%%%%%%%%%%%%%%%%%%%%%%%%%%%%%%%%%%%%%%%%%%%%%%%%%%%%%%%%%%%%%%%%%%%%%%%%%%%%
\subsection{Manual Code}
\label{sec:manual}

In case one cannot be certain whether the definitions file |childdoc.def|
is installed on the target \TeX{} distribution
and one prefers not to ship it,
it is conceivable to paste a few relevant commands into the sources.

To that end, drop all statements |\input{childdoc.def}|
and perform the replacements as outlined below.
Instead of |\childdocmain{|\textit{main}|}| add the following code
to the top of the main file:
%
\begin{center}
\begin{tabular}{l}
|\||ifdefined\childdocname\endinput\||fi\newif\ifchilddoc|\\
|\edef\childdocname{\scantokens\expandafter{\jobname\noexpand}}|\\
|\def\childdocmain{|\textit{main}|}\||ifx\childdocmain\childdocname\||else|\\
|\childdoctrue\includeonly{\childdocname}\let\jobname\childdocmain\||fi|\\
\end{tabular}
\end{center}
%
Instead of |\childdocof{|\textit{main}|}| just include the main file
at the top of each child file:
%
\begin{center}
|\input{|\textit{main}|}|
\end{center}
%
A simple redirection |\childdocforward{|\textit{dest}|}| is achieved by:
%
\begin{center}
|\def\jobname{|\textit{dest}|}\input{\jobname}|
\end{center}
%
The redirection with prefix
|\childdocforwardprefix[|\textit{prefix}|]{|\textit{dest}|}|
is accomplished by:
%
\begin{center}
\begin{tabular}{l}
|{\edef\jobname{\scantokens\expandafter{\jobname\noexpand}}|\\
|\def\redirectjob |\textit{prefix}|#1~~~{\gdef\jobname{|\textit{dest}|#1}}|\\
|\expandafter\redirectjob\jobname~~~}\input{\jobname}|
\end{tabular}
\end{center}

In an alternative approach,
child documents can be compiled by a specific command line
without additional code or specific definitions:
%
\begin{center}
|... -jobname "|\textit{target}|" "|[\textit{flags}]%
|\includeonly{|\textit{dest}|}\input{|\textit{main}|}"|
\end{center}
%

%%%%%%%%%%%%%%%%%%%%%%%%%%%%%%%%%%%%%%%%%%%%%%%%%%%%%%%%%%%%%%%%%%%%%%%%%%%%%%%%
%%%%%%%%%%%%%%%%%%%%%%%%%%%%%%%%%%%%%%%%%%%%%%%%%%%%%%%%%%%%%%%%%%%%%%%%%%%%%%%%
\section{Information}

%%%%%%%%%%%%%%%%%%%%%%%%%%%%%%%%%%%%%%%%%%%%%%%%%%%%%%%%%%%%%%%%%%%%%%%%%%%%%%%%
\subsection{Copyright}

Copyright \copyright{} 2017--2018 Niklas Beisert

This work may be distributed and/or modified under the
conditions of the \LaTeX{} Project Public License, either version 1.3
of this license or (at your option) any later version.
The latest version of this license is in
  \url{http://www.latex-project.org/lppl.txt}
and version 1.3 or later is part of all distributions of \LaTeX{}
version 2005/12/01 or later.

This work has the LPPL maintenance status `maintained'.

The Current Maintainer of this work is Niklas Beisert.

This work consists of the files |README.txt|, |childdoc.ins| and |childdoc.dtx|
as well as the derived files |childdoc.def|, |cdocsamp.tex|
with |cdocsch1.tex|, |cdocsch2.tex|, |cdocspt3.tex|, |cdocspt4.tex|,
|cdocsdrf.tex|, |cdocsfn1.tex|, |cdocsfn2.tex|
as well as |childdoc.pdf|.

%%%%%%%%%%%%%%%%%%%%%%%%%%%%%%%%%%%%%%%%%%%%%%%%%%%%%%%%%%%%%%%%%%%%%%%%%%%%%%%%
\subsection{Files and Installation}

The package consists of the files:
%
\begin{center}
\begin{tabular}{ll}
    |README.txt|   & readme file \\
    |childdoc.ins| & installation file \\
    |childdoc.dtx| & source file \\
    |childdoc.def| & definition file \\
    |cdocsamp.tex| & sample main file \\
    |cdocsch1.tex| & sample include file \\
    |cdocsch2.tex| & sample include file \\
    |cdocspt3.tex| & sample part file \\
    |cdocspt4.tex| & sample part file \\
    |cdocsdrf.tex| & sample redirection file \\
    |cdocsfn1.tex| & sample redirection file \\
    |cdocsfn2.tex| & sample redirection file \\
    |childdoc.pdf| & manual
\end{tabular}
\end{center}
%
The distribution consists of the files
|README.txt|, |childdoc.ins| and |childdoc.dtx|.
%
\begin{itemize}
\item
Run (pdf)\LaTeX{} on |childdoc.dtx|
to compile the manual |childdoc.pdf| (this file).
\item
Run \LaTeX{} on |childdoc.ins| to create the definitions file |childdoc.def|
and the sample |cdocsamp.tex| with include files
|cdocsch1.tex|, |cdocsch2.tex|, |cdocspt3.tex|, |cdocspt4.tex|,
|cdocsdrf.tex|, |cdocsfn1.tex|, |cdocsfn2.tex|.
Then copy the file |childdoc.def| to an appropriate directory of your \LaTeX{}
distribution, e.g.\ \textit{texmf-root}|/tex/latex/childdoc|.
\end{itemize}

%%%%%%%%%%%%%%%%%%%%%%%%%%%%%%%%%%%%%%%%%%%%%%%%%%%%%%%%%%%%%%%%%%%%%%%%%%%%%%%%
\subsection{Related CTAN Packages}

There are several other packages which offer a similar functionality:
%
\begin{itemize}
\item
The packages
\href{http://ctan.org/pkg/docmute}{\textsf{docmute}},
\href{http://ctan.org/pkg/includex}{\textsf{includex}} and
\href{http://ctan.org/pkg/standalone}{\textsf{standalone}}
provide commands to include only the document body of
a child file thus allowing both files to be compiled individually.
\item
The packages \href{http://ctan.org/pkg/subdocs}{\textsf{subdocs}}
and \href{http://ctan.org/pkg/subfiles}{\textsf{subfiles}}
provide structures in which the main and child documents can be
encapsulated and allowing them to be compiled individually.
The inclusion mechanism is different from the conventional |\include|.
\item
The package \href{http://ctan.org/pkg/combine}{\textsf{combine}}
is an elaborate solution to combine several documents into one.
\end{itemize}
%
See also the CTAN topic \href{http://ctan.org/topic/subdocs}{\textsf{subdocs}}
for further related packages.
The present package differs from the above solutions in that
a document structure constructed with the conventional |\include| mechanism
just needs two extra commands at the top of every file
such that all constituent files can be compiled individually.

%%%%%%%%%%%%%%%%%%%%%%%%%%%%%%%%%%%%%%%%%%%%%%%%%%%%%%%%%%%%%%%%%%%%%%%%%%%%%%%%
%\subsection{Feature Suggestions}
%
%The following is a list of features which may be useful for future
%versions of this package:
%%
%\begin{itemize}
%\item
%\ldots
%\end{itemize}

%%%%%%%%%%%%%%%%%%%%%%%%%%%%%%%%%%%%%%%%%%%%%%%%%%%%%%%%%%%%%%%%%%%%%%%%%%%%%%%%
\subsection{Revision History}

%%%%%%%%%%%%%%%%%%%%%%%%%%%%%%%%%%%%%%%%
\paragraph{v2.0:} 2018/12/30

\begin{itemize}
\item
immediate forward processing
\item
added |\childdocby| mechanism
\item
manual restructured
\end{itemize}

%%%%%%%%%%%%%%%%%%%%%%%%%%%%%%%%%%%%%%%%
\paragraph{v1.6:} 2018/01/17

\begin{itemize}
\item
application for development of include files
\item
corrections to manual
\end{itemize}

%%%%%%%%%%%%%%%%%%%%%%%%%%%%%%%%%%%%%%%%
\paragraph{v1.5:} 2017/05/21

\begin{itemize}
\item
more complete structuring introduced
\item
|\childdocof| introduced
\item
|\childdoc| renamed to |\childdocmain|
\item
|\childredirect| renamed to |\childdocforward| and |\childdocforwardprefix|
and functionality expanded
\end{itemize}

%%%%%%%%%%%%%%%%%%%%%%%%%%%%%%%%%%%%%%%%
\paragraph{v1.0:} 2017/04/27

\begin{itemize}
\item
manual and install package
\item
first version published on CTAN
\end{itemize}

%%%%%%%%%%%%%%%%%%%%%%%%%%%%%%%%%%%%%%%%
\paragraph{v0.6:} 2017/04/26

\begin{itemize}
\item
redirection mechanism added
\end{itemize}

%%%%%%%%%%%%%%%%%%%%%%%%%%%%%%%%%%%%%%%%
\paragraph{v0.5:} 2017/04/26

\begin{itemize}
\item
functionality in definition file
\end{itemize}


%%%%%%%%%%%%%%%%%%%%%%%%%%%%%%%%%%%%%%%%%%%%%%%%%%%%%%%%%%%%%%%%%%%%%%%%%%%%%%%%
%%%%%%%%%%%%%%%%%%%%%%%%%%%%%%%%%%%%%%%%%%%%%%%%%%%%%%%%%%%%%%%%%%%%%%%%%%%%%%%%
%%%%%%%%%%%%%%%%%%%%%%%%%%%%%%%%%%%%%%%%%%%%%%%%%%%%%%%%%%%%%%%%%%%%%%%%%%%%%%%%
\appendix

\settowidth\MacroIndent{\rmfamily\scriptsize 000\ }

 \DocInput{childdoc.dtx}

\end{document}
%</driver>
% \fi
%
% %%%%%%%%%%%%%%%%%%%%%%%%%%%%%%%%%%%%%%%%%%%%%%%%%%%%%%%%%%%%%%%%%%%%%%%%%%%%%%
% %%%%%%%%%%%%%%%%%%%%%%%%%%%%%%%%%%%%%%%%%%%%%%%%%%%%%%%%%%%%%%%%%%%%%%%%%%%%%%
% \section{Sample}
%\iffalse
%<*samplemain>
%\fi
%
% The following presents a sample document
% with two chapters, two parts, a title page,
% a compile flag as well as three forwarding files to set the flag.
% It consists of eight |.tex| files:
% \begin{center}
% \begin{tabular}{ll}
% |cdocsamp.tex|&main file\\
% |cdocsch1.tex|&include file for chapter 1\\
% |cdocsch2.tex|&include file for chapter 2\\
% |cdocspt3.tex|&include file for part 3\\
% |cdocspt4.tex|&include file for part 4\\
% |cdocsdrf.tex|&forwarding file for main file in draft mode\\
% |cdocsfi1.tex|&forwarding file for final version of chapter 1\\
% |cdocsfi2.tex|&forwarding file for final version of chapter 2\\
% \end{tabular}
% \end{center}
% Each of the eight files can be compiled directly by the \LaTeX{} compiler.
%
% %%%%%%%%%%%%%%%%%%%%%%%%%%%%%%%%%%%%%%
% \paragraph{Main File.}
%
% The main file is called |cdocsamp.tex|.
%
% Load the \textsf{childdoc} definitions and
% declare the filename for the main document:
%    \begin{macrocode}
\input{childdoc.def}
\childdocmain{}
%    \end{macrocode}

% Optional override for |\version| flag:
%    \begin{macrocode}
%%\ifchilddoc\else\providecommand{\version}{draft}\fi
%    \end{macrocode}

% Define the default values for the |\version| flag
% (|final| for the main file and |draft| for childs):
%    \begin{macrocode}
\ifchilddoc
\providecommand{\version}{draft}
\else
\providecommand{\version}{final}
\fi
%    \end{macrocode}

% Load the standard document class:
%    \begin{macrocode}
\documentclass[12pt]{article}
%    \end{macrocode}

% Start the document body:
%    \begin{macrocode}
\begin{document}
%    \end{macrocode}

% Declare a title page.
% Print title, part of document being processed and version flag:
%    \begin{macrocode}
\addtocounter{page}{-1}
\begin{center}
{\LARGE\bfseries{}childdoc example\par}
\vspace{1cm}
\ifchilddoc
\ifchilddocmanual part\else chapter\fi:
`\childdocname' of `\childdocjob'\par
\else
main document: `\childdocjob'\par
\fi
version: \version\par
\end{center}
\newpage
%    \end{macrocode}

% Manually include selected file,
% otherwise process as usual:
%    \begin{macrocode}
\ifchilddocmanual
\section*{part `\childdocname'}
\input{\childdocname}
\else
%    \end{macrocode}

% Include the two chapters:
%    \begin{macrocode}
\include{cdocsch1}
\include{cdocsch2}
%    \end{macrocode}

% Include the two parts unless only chapters should be displayed:
%    \begin{macrocode}
\ifchilddoc\else
\section{part three}
\input{cdocspt3}
\section{part four}
\input{cdocspt4}
\fi
%    \end{macrocode}

% Process as usual until here:
%    \begin{macrocode}
\fi
%    \end{macrocode}

% End of document body:
%    \begin{macrocode}
\end{document}
%    \end{macrocode}
%\iffalse
%</samplemain>
%\fi
%
% %%%%%%%%%%%%%%%%%%%%%%%%%%%%%%%%%%%%%%
% \paragraph{Chapter Include Files.}
%
% The include files are called |cdocsch1.tex| and |cdocsch2.tex|.
%
%\iffalse
%<*samplechap1|samplechap2>
%\fi

% Optional override for |\version| flag:
%    \begin{macrocode}
%%\providecommand{\version}{final}
%    \end{macrocode}

% Include the main document:
%    \begin{macrocode}
\input{childdoc.def}
\childdocof{cdocsamp}
%    \end{macrocode}

%\iffalse
%</samplechap1|samplechap2>
%\fi
%
%\iffalse
%<*samplechap1>
%\fi
% Some text for chapter 1:
%    \begin{macrocode}
\section{one}
some text in chapter one
%    \end{macrocode}

%\iffalse
%</samplechap1>
%\fi
% Some text for chapter 2:
%\iffalse
%<*samplechap2>
%\fi
%    \begin{macrocode}
\section{two}
more text in chapter two
%    \end{macrocode}

%\iffalse
%</samplechap2>
%\fi
%
% %%%%%%%%%%%%%%%%%%%%%%%%%%%%%%%%%%%%%%
% \paragraph{Part Include Files.}
%
% The include files are called |cdocspt3.tex| and |cdocspt4.tex|.
%
%\iffalse
%<*samplepart3|samplepart4>
%\fi

% Optional override for |\version| flag:
%    \begin{macrocode}
%%\providecommand{\version}{final}
%    \end{macrocode}

% Include the main document:
%    \begin{macrocode}
\input{childdoc.def}
\childdocby{cdocsamp}
%    \end{macrocode}

%\iffalse
%</samplepart3|samplepart4>
%\fi
%
%\iffalse
%<*samplepart3>
%\fi
% Some text for part 3:
%    \begin{macrocode}
some text in part three
%    \end{macrocode}

%\iffalse
%</samplepart3>
%\fi
% Some text for part 4:
%\iffalse
%<*samplepart4>
%\fi
%    \begin{macrocode}
more text in part four
%    \end{macrocode}

%\iffalse
%</samplepart4>
%\fi
%
% %%%%%%%%%%%%%%%%%%%%%%%%%%%%%%%%%%%%%%
% \paragraph{Forwarding for a Complete Draft.}
%
% The following forwarding file |cdocsdrf.tex|
% compiles the main document in draft mode:
%\iffalse
%<*sampledraft>
%\fi
%    \begin{macrocode}
\def\version{draft}
\input{childdoc.def}
\childdocforward{cdocsamp}
%    \end{macrocode}

%\iffalse
%</sampledraft>
%\fi
%
% %%%%%%%%%%%%%%%%%%%%%%%%%%%%%%%%%%%%%%
% \paragraph{Forwarding for Final Version of the Chapters.}
%
% The following forwarding files |cdocsfn1.tex| and |cdocsfn2.tex|
% (with identical content)
% compile the final versions of the child documents
% |cdocsch1.tex| and |cdocsch2.tex|, respectively:
%\iffalse
%<*samplefinal>
%\fi
%    \begin{macrocode}
\def\version{final}
\input{childdoc.def}
\childdocforwardprefix[cdocsamp]{cdocsfn}{cdocsch}
%    \end{macrocode}

%\iffalse
%</samplefinal>
%\fi
%
% %%%%%%%%%%%%%%%%%%%%%%%%%%%%%%%%%%%%%%
% \paragraph{Command Line Processing.}
%
% The following three command lines generate the output files
% |cdocscld|, |cdocscl1| and |cdocscl2|
% which should be identical to
% |cdocsdrf|, |cdocsch1| and |cdocsfn2|, respectively:
% \begin{center}
% \begin{tabular}{l}
% |latex -jobname cdocscld \|\\
% |  "\def\version{draft}\input{childdoc.def}\childdocforward{cdocsamp}"|\\
% |latex -jobname cdocscl1 \|\\
% |  "\input{childdoc.def}\childdocforward[cdocsamp]{cdocsch1}"|\\
% |latex -jobname cdocscl2 \|\\
% |  "\def\version{final}\input{childdoc.def}\childdocforward{cdocsch2}"|
% \end{tabular}
% \end{center}
% Note that the trailing backslash on each first line
% merely continues the input to the second line
% (for convenient cut ant paste).
% Furthermore, the command |latex| can be replaced by any
% of its alternative versions such as |pdflatex|.
%
% %%%%%%%%%%%%%%%%%%%%%%%%%%%%%%%%%%%%%%%%%%%%%%%%%%%%%%%%%%%%%%%%%%%%%%%%%%%%%%
% %%%%%%%%%%%%%%%%%%%%%%%%%%%%%%%%%%%%%%%%%%%%%%%%%%%%%%%%%%%%%%%%%%%%%%%%%%%%%%
% \section{Implementation}
%\iffalse
%<*package>
%\fi
%
% This section describes the definitions file |childdoc.def|.

% The definitions cannot be loaded using |\usepackage| or |\RequirePackage|
% which has a mechanism to prevent loading a style file more than once.
% When loading the definitions by means of |\input|
% multiple instances have to be prevented manually:
%\iffalse
%This code needs to be before the `\ProvidesFile' directive
%which is defined at the beginning of this file.
%Therefore it is also placed there and commented out here.
%</package>
%<*discard>
%\fi
%    \begin{macrocode}
\ifdefined\childdocmain\endinput\fi
%    \end{macrocode}
%\iffalse
%</discard>
%<*package>
%\fi
%
% \macro{\ifchilddoc}
% \macro{\ifchilddocmanual}
% The conditional |\ifchilddoc| tells whether a
% child (true) or main (false) document is being compiled.
% The conditional |\ifchilddocmanual| tells whether
% the |\includeonly| mechanism is used (false) or
% the selection of child files must be performed manually (true).
% The definitions initialise to false:
%    \begin{macrocode}
\newif\ifchilddoc
\newif\ifchilddocmanual
%    \end{macrocode}

% \macro{\childdocname}
% \macro{\childdocjob}
% The macro |\childdocname| stores the name of the main document
% to be compiled. The macro |\childdocjob| stores the name of
% the document on which the \LaTeX{} compiler was originally invoked.
% The content of |\jobname| cannot be compared
% to filenames specified in the source due to different catcodes.
% The following code rescans |\jobname|, stores the result
% in |\childdocname| and saves a copy in |\childdocjob|:
%    \begin{macrocode}
\edef\childdocname{\scantokens\expandafter{\jobname\noexpand}}
\let\childdocjob\childdocname
%    \end{macrocode}

% \macro{\childdocdisable}
% The macro |\childdocdisable| prevents the main file
% from being processed more than once.
% At this stage, the main document command |\childdocmain|
% is assumed to be called once again where it should do nothing.
% Any subsequent call to it should prevent
% a secondary processing of the main document
% It overwrites the forwarding commands
% |\childdocof| and |\childdocforward|
% with empty macros to prevent further inclusions of the main document:
%    \begin{macrocode}
\newcommand{\childdocdisable}
{
  \renewcommand{\childdocmain}[1]{\renewcommand{\childdocmain}[1]{\endinput}}
  \renewcommand{\childdocof}[1]{}
  \renewcommand{\childdocby}[2][]{}
  \renewcommand{\childdocforward}[2][]{}
  \renewcommand{\childdocdisable}{}
}
%    \end{macrocode}

% \macro{\childdocmain}
% The macro |\childdocmain| is to be called at the top of the main file
% with nothing or the main filename (without extension) as argument.
% First, it breaks loops.
% If the argument is not empty and does not match |\childdocname|
% (which is set by the first inclusion of |childdoc.def|),
% |\ifchilddoc| is set to true, |\includeonly| is applied to the child file
% and |\jobname| is set to the main file
% (for proper handling of |.aux| files):
%    \begin{macrocode}
\newcommand{\childdocmain}[1]
{
  \childdocdisable\childdocmain{}
  \if?#1?\else
    \begingroup
      \def\childdoctmp{#1}
      \ifx\childdoctmp\childdocname
        \def\childdoctmp{}
      \else
        \def\childdoctmp
        {
          \childdoctrue
          \includeonly{\childdocname}
          \def\childdocjob{#1}
          \def\jobname{#1}
        }
      \fi
      \expandafter
    \endgroup
    \childdoctmp
  \fi
}
%    \end{macrocode}

% \macro{\childdocof}
% The command |\childdocof| redirects
% compilation to the main file |#1|.
%    \begin{macrocode}
\newcommand{\childdocof}[1]
{
  \childdocdisable
  \childdoctrue
  \includeonly{\childdocname}
  \def\jobname{#1}
  \def\childdocjob{#1}
  \input{#1}
}
%    \end{macrocode}

% \macro{\childdocby}
% The command |\childdocby| ....
%    \begin{macrocode}
\newcommand{\childdocby}[2][]
{
  \childdocdisable
  \childdoctrue
  \childdocmanualtrue
  \if?#1?\else
    \def\jobname{#2}
  \fi
  \def\childdocjob{#2}
  \input{#2}
  \endinput
}
%    \end{macrocode}

% \macro{\childdocforward}
% The command |\childdocforward| redirects
% compilation to the main file or
% (if the optional argument is given) a child file.
% Parameters are set as if the main file
% or a child file starting with |\childdocof| was compiled.
% Then compilation is handed over to the main file:
%    \begin{macrocode}
\newcommand{\childdocforward}[2][]
{
  \begingroup
    \if?#1?
      \def\childdoctmp
      {
        \def\childdocname{#2}
        \def\childdocjob{#2}
        \def\jobname{#2}
        \input{#2}
        \endinput
      }
    \else
      \def\childdoctmp
      {
        \childdocdisable
        \def\childdocname{#2}
        \childdoctrue
        \includeonly{#2}
        \def\childdocjob{#1}
        \def\jobname{#1}
        \input{#1}
        \endinput
      }
    \fi
    \expandafter
  \endgroup
  \childdoctmp
}
%    \end{macrocode}

% \macro{\childdocforwardprefix}
% The command |\childdocforwardprefix| redirects
% compilation to the main or a child file by means of a pattern.
% The prefix |#1| in the current filename is replaced by |#2|
% and the suffix of the current filename is kept
% (it is assumed that the filename does not contain the substring `|~~~|'
% which is used as a delimiter).
% Compilation is handed over to the new file by |\childdocforward|:
%    \begin{macrocode}
\newcommand{\childdocforwardprefix}[3][]
{
  \begingroup
    \def\childdocextract #2##1~~~{\def\childdoctmp{\childdocforward[#1]{#3##1}}}
    \expandafter\childdocextract\childdocname~~~
    \expandafter
  \endgroup
  \childdoctmp
}
%    \end{macrocode}

% \macro{\childdoc}
% The deprecated macro |\childdoc| is a legacy version of |\childdocmain|:
%    \begin{macrocode}
\newcommand{\childdoc}{\childdocmain}
%    \end{macrocode}

% \macro{\childdocredirect}
% The deprecated macro |\childdocredirect| is a legacy version
% of |\childdocforward| and |\childdocforwardprefix|:
%    \begin{macrocode}
\newcommand{\childdocredirect}[2][]
{
  \begingroup
    \if?#1?
      \def\childdoctmp{\childdocforward{#2}}
    \else
      \def\childdoctmp{\childdocforwardprefix{#1}{#2}}
    \fi
    \expandafter
  \endgroup
  \childdoctmp
}
%    \end{macrocode}

%\iffalse
%</package>
%\fi
%
\endinput

\childdocforwardprefix[cdocsamp]{cdocsfn}{cdocsch}
%    \end{macrocode}

%\iffalse
%</samplefinal>
%\fi
%
% %%%%%%%%%%%%%%%%%%%%%%%%%%%%%%%%%%%%%%
% \paragraph{Command Line Processing.}
%
% The following three command lines generate the output files
% |cdocscld|, |cdocscl1| and |cdocscl2|
% which should be identical to
% |cdocsdrf|, |cdocsch1| and |cdocsfn2|, respectively:
% \begin{center}
% \begin{tabular}{l}
% |latex -jobname cdocscld \|\\
% |  "\def\version{draft}% \iffalse
%
% childdoc.dtx Copyright (C) 2017-2018 Niklas Beisert
%
% This work may be distributed and/or modified under the
% conditions of the LaTeX Project Public License, either version 1.3
% of this license or (at your option) any later version.
% The latest version of this license is in
%   http://www.latex-project.org/lppl.txt
% and version 1.3 or later is part of all distributions of LaTeX
% version 2005/12/01 or later.
%
% This work has the LPPL maintenance status `maintained'.
%
% The Current Maintainer of this work is Niklas Beisert.
%
% This work consists of the files childdoc.dtx and childdoc.ins
% and the derived files childdoc.def and cdocsamp.tex with
% cdocsch1.tex, cdocsch2.tex, cdocsdrf.tex, cdocsfn1.tex, cdocsfn2.tex.
%
%<package>\ifdefined\childdocmain\endinput\fi
%<package>\ProvidesFile{childdoc.def}[2018/12/30 v2.0 child document driver]
%<samplemain>\ProvidesFile{cdocsamp.tex}[2018/12/30 v2.0 sample for childdoc]
%<*driver>
%\ProvidesFile{childdoc.drv}[2018/12/30 v2.0 childdoc reference manual file]
\PassOptionsToClass{10pt,a4paper}{article}
\documentclass{ltxdoc}

\usepackage[margin=35mm]{geometry}
\usepackage{hyperref}
\usepackage{hyperxmp}
\usepackage[usenames]{color}

\hypersetup{colorlinks=true}
\hypersetup{pdfstartview=FitH}
\hypersetup{pdfpagemode=UseNone}
\hypersetup{pdfsource={}}
\hypersetup{pdflang={en-UK}}
\hypersetup{pdfcopyright={Copyright 2017-2018 Niklas Beisert.
  This work may be distributed and/or modified under the
  conditions of the LaTeX Project Public License, either version 1.3
  of this license or (at your option) any later version.}}
\hypersetup{pdflicenseurl={http://www.latex-project.org/lppl.txt}}
\hypersetup{pdfcontactaddress={ETH Zurich, ITP, HIT K,
  Wolfgang-Pauli-Strasse 27}}
\hypersetup{pdfcontactpostcode={8093}}
\hypersetup{pdfcontactcity={Zurich}}
\hypersetup{pdfcontactcountry={Switzerland}}
\hypersetup{pdfcontactemail={nbeisert@itp.phys.ethz.ch}}
\hypersetup{pdfcontacturl={http://people.phys.ethz.ch/\xmptilde nbeisert/}}

\newcommand{\secref}[1]{\hyperref[#1]{section \ref*{#1}}}

\parskip1ex
\parindent0pt
\let\olditemize\itemize
\def\itemize{\olditemize\parskip0pt}

\begin{document}

\title{The \textsf{childdoc} Package}
\hypersetup{pdftitle={The childdoc Package}}
\author{Niklas Beisert\\[2ex]
  Institut f\"ur Theoretische Physik\\
  Eidgen\"ossische Technische Hochschule Z\"urich\\
  Wolfgang-Pauli-Strasse 27, 8093 Z\"urich, Switzerland\\[1ex]
  \href{mailto:nbeisert@itp.phys.ethz.ch}
  {\texttt{nbeisert@itp.phys.ethz.ch}}}
\hypersetup{pdfauthor={Niklas Beisert}}
\hypersetup{pdfsubject={Manual for the LaTeX2e Package childdoc}}
\date{30 December 2018, \textsf{v2.0}}
\maketitle

\begin{abstract}\noindent
\textsf{childdoc} is a \LaTeXe{} package
that enables the direct compilation
of document sections included by |\include|
to individual files.
\end{abstract}

\begingroup
\parskip0ex
\tableofcontents
\endgroup

%%%%%%%%%%%%%%%%%%%%%%%%%%%%%%%%%%%%%%%%%%%%%%%%%%%%%%%%%%%%%%%%%%%%%%%%%%%%%%%%
%%%%%%%%%%%%%%%%%%%%%%%%%%%%%%%%%%%%%%%%%%%%%%%%%%%%%%%%%%%%%%%%%%%%%%%%%%%%%%%%
\section{Introduction}

\LaTeX{} provides a mechanism to structure a large document (such as a book)
into a main file and several child files (containing the chapters)
using the |\include| command.
This mechanism is beneficial for documents
which span hundreds of pages in order to
make the source file(s) more manageable.
Moreover, compilation can be restricted to
selected child files by means of the |\includeonly| command.
The latter feature can be used to reduce the compilation time while editing
(this was significantly more useful in the earlier days of \LaTeX{})
or to generate a smaller document which is easier to navigate.
Another application of |\includeonly| is to generate
documents consisting of selected parts of the complete document.

However, there are a few drawbacks of the plain |\include| mechanism:
\begin{itemize}
\item
The child files cannot be compiled on their own,
they can only be compiled via the main file.
A naive editing environment
(such as a text editor with an option
to have the current file processed by \LaTeX)
may require one to switch to the main file before compiling;
attempting to compile the child file produces errors.
\item
The main file must be modified (each time)
to adjust the |\includeonly| command
to the present needs. This easily leaves the main file in a messy state.
\item
The generated document will always carry the filename
of the main document. This is inconvenient if
several child files are to be compiled and
to be kept for distribution.
\end{itemize}

The present package provides a simple interface
to make child files individually compilable by \LaTeX{}.
Compiling a child file then has the same effect as compiling
the main file with an |\includeonly| command
to select the appropriate child.
Moreover the generated document will carry the name of the child
rather than the main file.
This resolves all three above issues.

This feature is meant to make the editing of books,
thesis documents and lecture notes somewhat more convenient.
However, the package can also be used efficiently for
composing a series of documents (such as exercise sheets)
which are typically distributed individually.
It then assists the author in generating the individual documents
(potentially in different versions)
as well as a document containing the collected series.
Another application is in developing style files
or other kinds of included material
where compilation of the style file could redirect
to a sample or test file.

%%%%%%%%%%%%%%%%%%%%%%%%%%%%%%%%%%%%%%%%%%%%%%%%%%%%%%%%%%%%%%%%%%%%%%%%%%%%%%%%
%%%%%%%%%%%%%%%%%%%%%%%%%%%%%%%%%%%%%%%%%%%%%%%%%%%%%%%%%%%%%%%%%%%%%%%%%%%%%%%%
\section{Usage}

First of all, the package \textsf{childdoc} is \emph{not} a standard
\LaTeXe{} |.sty| style file! Therefore it needs to be invoked in
a non-standard way.

%%%%%%%%%%%%%%%%%%%%%%%%%%%%%%%%%%%%%%%%%%%%%%%%%%%%%%%%%%%%%%%%%%%%%%%%%%%%%%%%
\subsection{Included Files}
\label{sec:include}

%%%%%%%%%%%%%%%%%%%%%%%%%%%%%%%%%%%%%%%%
\DescribeMacro{\childdocmain}
To use the package, add the commands
\begin{center}
\begin{tabular}{l}
|\input{childdoc.def}|\\
|\childdocmain{}|\\
\end{tabular}
\end{center}
at the very top of the main \LaTeX{} file,
in particular \emph{before} the |\documentclass| statement!
The argument of |\childdocmain| should be left empty
(but it must be present).

%%%%%%%%%%%%%%%%%%%%%%%%%%%%%%%%%%%%%%%%
\DescribeMacro{\childdocof}
Furthermore, add the commands
\begin{center}
\begin{tabular}{l}
|\input{childdoc.def}|\\
|\childdocof{|\textit{main}|}|\\
\end{tabular}
\end{center}
at the top of every child file \textit{child}
which is included by |\include{|\textit{child}|}|
from within the main file
(or at least for those files to be compiled individually).
The argument \textit{main} must be the filename of the main file.

There are a couple of
considerations in setting up the main and child documents:

%%%%%%%%%%%%%%%%%%%%%%%%%%%%%%%%%%%%%%%%
\paragraph{Restrictions.}

Please note the following restrictions:
\begin{itemize}
\item
|\childdocmain| must be called with one argument \textit{main}
to ensure compatibility with earlier version of the package.
It must either be empty (|\childdocmain{}|)
or precisely match the filename of the main file in which it is specified.
See \secref{sec:detection} for further information.
\item
The filename \textit{main} must be specified without the |.tex| extension.
\item
The filename \textit{main} is case sensitive
(even in case-insensitive file systems)
due to internal string comparison.
\item
The argument \textit{main} should be fully expanded, it cannot be a macro.
\item
Subdirectories and special characters should be avoided in filenames.
\item
The command |\childdocmain{|\textit{main}|}| must be followed by a whitespace.
It should not be followed immediately by another command
or by a comment mark `|%|'.
This is because the \TeX{} parser reads the token immediately following
the argument of |\childdocmain| and puts it
at the beginning of every child section;
however, a white\-space is ignored.
\end{itemize}

%%%%%%%%%%%%%%%%%%%%%%%%%%%%%%%%%%%%%%%%
\paragraph{Content of Main File.}

It is advisable to place all content in the child files included by |\include|.
Any output contained in the main file will appear in all child documents
unless suppressed manually;
it cannot be suppressed automatically by the |\includeonly| directive
and thus should normally be avoided.
A method to include some content in the main file
by means of conditional processing is described in \secref{sec:conditional}.

%%%%%%%%%%%%%%%%%%%%%%%%%%%%%%%%%%%%%%%%
\paragraph{Page Numbering.}

When only a part of the document is compiled,
the appropriate numbering of pages
(as well as other status parameters)
is determined from the |.aux| files.
The latter contain information from previous passes.
However this information needs to propagate through
all intermediate child documents.
Therefore the page numbering in child documents may well
be inconsistent until the complete document is compiled at least once.

A useful (if unconventional) way to always ensure a consistent
page numbering is to restart the numbering in each child document
and denote the pages by `\textit{child}|.|\textit{page}'
where \textit{child} represents the chapter/section number of the child file.
This can be achieved by the command
|\numberwithin{page}{|\textit{child}|}|
of the \textsf{amsmath} package
where \textit{child} can be |chapter| or |section|
depending on the chosen structuring.
Alternatively, one can modify the macro |\thepage| appropriately
and reset the counter |page| at the start of each child file.

%%%%%%%%%%%%%%%%%%%%%%%%%%%%%%%%%%%%%%%%%%%%%%%%%%%%%%%%%%%%%%%%%%%%%%%%%%%%%%%%
\subsection{Conditional Processing}
\label{sec:conditional}

The package provides a mechanism to compile different versions
of a document. To customise the versions further some conditional processing
can come in handy to distinguish which version is being compiled.
The package provides two macros to describe the compilation context:

%%%%%%%%%%%%%%%%%%%%%%%%%%%%%%%%%%%%%%%%
\DescribeMacro{\ifchilddoc}
The conditional |\ifchilddoc| distinguishes between the compilation of
child documents and the main document:
%
\begin{center}
|\ifchilddoc |\textit{child-code}| |[|\||else |\textit{main-code}]| \||fi|
\end{center}

%%%%%%%%%%%%%%%%%%%%%%%%%%%%%%%%%%%%%%%%
\DescribeMacro{\childdocname}
\DescribeMacro{\childdocjob}
The macro |\childdocname| contains the filename (without extension)
of the main or child file being processed.
Note that |\childdocjob| will always contain the name of the main file.

%%%%%%%%%%%%%%%%%%%%%%%%%%%%%%%%%%%%%%%%
\paragraph{Title Page.}

Conditional processing can be used to include a title or banner page
in the main document when proper precautions are taken.
Importantly, the code in the main file should ensure that the page counter
(as well as other status parameters which are stored in the |.aux| files)
takes the same value after the conditional processing.
Otherwise the page numbers may take divergent values
depending on which part is compiled.

For example, a title page could be declared by:
%
\begin{center}
\begin{tabular}{l}
|\ifchilddoc\||else|\\
|\addtocounter{page}{-1}|\\
\textit{code for title page}\\
|\newpage|\\
|\||fi|
\end{tabular}
\end{center}
%
A banner page for the child documents can be generated by:
%
\begin{center}
\begin{tabular}{l}
|\ifchilddoc|\\
|\addtocounter{page}{-1}|\\
\textit{code for banner page}\\
|\newpage|\\
|\||fi|
\end{tabular}
\end{center}
%
Here one could write a message such as:
\begin{center}
|This is the part \childdocname{} of \childdocjob{}.|
\end{center}

%%%%%%%%%%%%%%%%%%%%%%%%%%%%%%%%%%%%%%%%%%%%%%%%%%%%%%%%%%%%%%%%%%%%%%%%%%%%%%%%
\subsection{Flags}
\label{sec:flags}

The package makes it easy to generate different versions
of the main or child documents.
To this end compilation flags can be defined
and assigned different default values.
They will be particularly useful in conjunction
with the forwarding mechanism described in \secref{sec:forward}.

For example, it may be useful to have a flag |\version|
which can be set to |draft| or |final|.
The document source will contain some conditional code
depending on the value of |\version|.
Suppose further, the flag should default to |final| for the main file
and to |draft| for child files
which is a natural assignment for editing the document.
This is achieved by placing the following code
in the preamble of the main document
(below the |\childdocmain| directive):
%
\begin{center}
\begin{tabular}{l}
|\ifchilddoc|\\
|\providecommand{\version}{draft}|\\
|\||else|\\
|\providecommand{\version}{final}|\\
|\||fi|
\end{tabular}
\end{center}
%
The definition by |\providecommand| makes sure
that previous definitions are not overwritten.
Further statements |\providecommand{\version}{...}|
can thus be added before the above code to override it.

For the main file, one might add a line
(between |\childdocmain| and the above block)
%
\begin{center}
|%\ifchilddoc\||else\providecommand{\version}{draft}\||fi|
\end{center}
%
which can be uncommented to produce a draft version.
Likewise one can add a line to the very top of a child file
(above the |\childdocof{|\textit{main}|}| directive)
%
\begin{center}
|%\providecommand{\version}{final}|
\end{center}
%
which can be uncommented to produce the final version of this child document.

%%%%%%%%%%%%%%%%%%%%%%%%%%%%%%%%%%%%%%%%%%%%%%%%%%%%%%%%%%%%%%%%%%%%%%%%%%%%%%%%
\subsection{Forwarding}
\label{sec:forward}

Different versions of the main or child documents
using compilation flags as described in \secref{sec:flags}
can be (permanently) stored in different files
for convenient compilation, viewing and distribution.
To this end, the package defines a command
to pass on compilation to a different file:

%%%%%%%%%%%%%%%%%%%%%%%%%%%%%%%%%%%%%%%%
\DescribeMacro{\childdocforward}
The command |\childdocforward| redirects processing to
another source file:
%
\begin{center}
\begin{tabular}{l}
|\input{childdoc.def}|\\
|\childdocforward[|\textit{main}|]{|\textit{dest}|}|\\
\end{tabular}
\end{center}
%
The argument \textit{dest} is the destination file
(without extension).
It should be the main file or one of the child files.
Note that further \textsf{childdoc} directives
such as |\childdocof| and |\childdocforward|
in the indicated file will be processed in this form.
The optional argument \textit{main}
passes on directly to the main file \textit{main}
while pretending to compile the child \textit{dest}.
This form behaves as if \textit{dest}
issues |\childdocof{|\textit{main}|}| right away,
and no further \textsf{childdoc} directives will be processed.

%%%%%%%%%%%%%%%%%%%%%%%%%%%%%%%%%%%%%%%%
\DescribeMacro{\...prefix}
In the alternative form |\childdocforwardprefix|,
%
\begin{center}
\begin{tabular}{l}
|\input{childdoc.def}|\\
|\childdocforwardprefix[|\textit{main}|]{|\textit{prefix}|}{|\textit{dest}|}|
\end{tabular}
\end{center}
%
the destination file is determined by a pattern
depending on the current file:
To make this work, the current file must be called
`{\textit{prefix}\hspace{0.2em}\textit{suffix}}'
with \textit{prefix} matching precisely the argument.
Processing is then passed on to the file
`{\textit{dest}\hspace{0.2em}\textit{suffix}}'.
Surely, the same effect is achieved by
directly specifying the
argument `{\textit{dest}\hspace{0.2em}\textit{suffix}}'
in the first form.
However, that requires to set up a different file
for each child. With the alternative form of the command
all these files can have exactly the same content
which simplifies setting them up and maintaining them.

For example, the following file |draft.tex|
with a compilation flag |\version| as described in \secref{sec:flags}
compiles the main document as a draft:
%
\begin{center}
\begin{tabular}{l}
|\def\version{draft}|\\
|\input{childdoc.def}|\\
|\childdocforward{|\textit{main}|}|
\end{tabular}
\end{center}
%
Likewise, the following files |final|\textit{nn}|.tex|
compile the final version of the child document
|child|\textit{nn}|.tex|:
%
\begin{center}
\begin{tabular}{l}
|\def\version{final}|\\
|\input{childdoc.def}|\\
|\childdocforwardprefix{final}{child}|
\end{tabular}
\end{center}
%

Note that when several versions of a main file and/or of each child file
are to be generated, it may be convenient to set up a |Makefile| or
shell script to automatise the process.

%%%%%%%%%%%%%%%%%%%%%%%%%%%%%%%%%%%%%%%%%%%%%%%%%%%%%%%%%%%%%%%%%%%%%%%%%%%%%%%%
\subsection{Command Line Processing}
\label{sec:commandline}

The effect of redirection files can also be achieved by invoking
the \LaTeX{} compiler with a more elaborate command line.
Most conveniently this should be done as part
of a shell script or a |Makefile|.

When using \textsf{childdoc} in the main file, the following
command lines effectively perform a redirection
(note that depending on the shell being used,
backslashes may have to be doubled: `|\|' $\to$ `|\\|'):
%
\begin{center}
|... -jobname "|\textit{target}|" |\\|"|[\textit{flags}]%
|\input{childdoc.def}\childdocforward[|\textit{main}|]{|\textit{dest}|}"|
\end{center}
%
Here \textit{target} is the name of the output file,
\textit{main} is the name of the main file
and \textit{dest} is the name of the main or child file to be processed
(all filenames without extensions).
The optional argument \textit{main} can be omitted
if \textit{main} matches \textit{dest}.
Optionally, compilation \textit{flags} can be defined via |\def| commands.
This command line makes the \TeX{} engine believe
it is compiling the file \textit{target}
whose content is specified as the latter parameter.
The provided code then forwards the processing to
\textit{main} or \textit{dest} as described in \secref{sec:forward}.

%%%%%%%%%%%%%%%%%%%%%%%%%%%%%%%%%%%%%%%%%%%%%%%%%%%%%%%%%%%%%%%%%%%%%%%%%%%%%%%%
\subsection{Include by Input}
\label{sec:input}

Including child documents by |\include| has some restrictions by design.
Most notably, the content of a child document always occupies
its own set of pages; pages cannot be shared between child documents.
Usually, this behaviour makes perfect sense
because each child document contain an essential part of the document.
However, in some situations it may be desirable to compose
a document from a collection of parts
without having mandatory page breaks between then.
For this case, the package
provides a mechanism to include parts
by |\input| which can also be processed individually.
However, by construction this mechanism
requires manual handling of the content to be output.

%%%%%%%%%%%%%%%%%%%%%%%%%%%%%%%%%%%%%%%%
\DescribeMacro{\ifchilddocmanual}
The main file should be prepared as usual, see \secref{sec:include}.
However, the document body must make a distinction
between processing of an individual part and of the main document, e.g.:
%
\begin{center}
\begin{tabular}{l}
|\ifchilddocmanual|\\
|\input{\childdocname}|\\
|\||else|\\
\textit{document body with }|\input{|\textit{part}|}|\\
|\||fi|
\end{tabular}
\end{center}
%
The conditional |\ifchilddocmanual| is true whenever
a part to be included by |\input| is being compiled,
and the name of the part is stored in |\childdocname|.

%%%%%%%%%%%%%%%%%%%%%%%%%%%%%%%%%%%%%%%%
\DescribeMacro{\childdocby}
Each part to be included by |\input| should start with:
%
\begin{center}
\begin{tabular}{l}
|\input{childdoc.def}|\\
|\childdocby{|\textit{main}|}|\\
\end{tabular}
\end{center}
%
The directive |\childdocby| is similar to |\childdocof|
described in \secref{sec:include},
but the subsequent selection of content must be done manually.
To that end, both |\ifchilddoc| and |\ifchilddocmanual|
will be true upon processing of a part,
and the name of the part is stored in |\childdocname|.
Note that |\jobname| will be set to the filename of the current part
so that each part receives an individual |.aux| file
that does not interfere with the |.aux| file(s) of the main document.
This behaviour can be altered by the alternative form
|\childdocby[*]{|\textit{main}|}| (with a non-empty optional argument)
which uses the |.aux| file of the main document
by setting |\jobname| to \textit{main}.

%%%%%%%%%%%%%%%%%%%%%%%%%%%%%%%%%%%%%%%%%%%%%%%%%%%%%%%%%%%%%%%%%%%%%%%%%%%%%%%%
\subsection{Driver Development}
\label{sec:driver}

The \textsf{childdoc} mechanism can also be use for the development
of definition files such as \LaTeX{} styles or classes.
This case differs from the above setup with multiple parts
included by |\include| in that no |\includeonly| should be invoked.
This can be achieved by starting the include file
(before |\ProvidesPackage|) with:
%
\begin{center}
\begin{tabular}{l}
|\input{childdoc.def}|\\
|\childdocforward{|\textit{main}|}|\\
\end{tabular}
\end{center}
%
or alternatively with:
%
\begin{center}
\begin{tabular}{l}
|\input{childdoc.def}|\\
|\childdocby{|\textit{main}|}|\\
\end{tabular}
\end{center}
%
Both forms have slightly different effects as described above.
The main file is prepared as usual, see \secref{sec:include}.

%%%%%%%%%%%%%%%%%%%%%%%%%%%%%%%%%%%%%%%%%%%%%%%%%%%%%%%%%%%%%%%%%%%%%%%%%%%%%%%%
\subsection{Legacy Detection}
\label{sec:detection}

The directive |\childdocmain| in the main file can detect
whether the complete document or merely a child is to be compiled
even without using the directive |\childdocof|.
This method is deprecated because it is less robust
and there is no compelling reason to use it;
it is merely provided for backward compatibility
and it may be removed in future versions.

If the detection mechanism is to be used,
it is mandatory to correctly specify
the filename of the main file as the argument of |\childdocmain|:
%
\begin{center}
\begin{tabular}{l}
|\input{childdoc.def}|\\
|\childdocmain{|\textit{main}|}|\\
\end{tabular}
\end{center}
%
If |\jobname| does not match the argument \textit{main} of |\childdocmain|,
it is assumed that |\jobname| points to the child file to be compiled.
When using |\childdocmain| with the main file specified as argument,
it suffices to start a child file
with just |\input{|\textit{main}|}|
without loading of the package and using |\childdocof|.
If instead all processing is done
with the appropriate \textsf{childdoc} directives,
the argument of \textit{main} of |\childdocmain| can be empty.

An alternative version of the command line processing described
in \secref{sec:commandline} using the detection mechanism reads:
%
\begin{center}
|... -jobname "|\textit{target}|" "|[\textit{flags}]%
[|\def\jobname{|\textit{dest}|}|]|\input{|\textit{main}|}"|
\end{center}

%%%%%%%%%%%%%%%%%%%%%%%%%%%%%%%%%%%%%%%%%%%%%%%%%%%%%%%%%%%%%%%%%%%%%%%%%%%%%%%%
\subsection{Manual Code}
\label{sec:manual}

In case one cannot be certain whether the definitions file |childdoc.def|
is installed on the target \TeX{} distribution
and one prefers not to ship it,
it is conceivable to paste a few relevant commands into the sources.

To that end, drop all statements |\input{childdoc.def}|
and perform the replacements as outlined below.
Instead of |\childdocmain{|\textit{main}|}| add the following code
to the top of the main file:
%
\begin{center}
\begin{tabular}{l}
|\||ifdefined\childdocname\endinput\||fi\newif\ifchilddoc|\\
|\edef\childdocname{\scantokens\expandafter{\jobname\noexpand}}|\\
|\def\childdocmain{|\textit{main}|}\||ifx\childdocmain\childdocname\||else|\\
|\childdoctrue\includeonly{\childdocname}\let\jobname\childdocmain\||fi|\\
\end{tabular}
\end{center}
%
Instead of |\childdocof{|\textit{main}|}| just include the main file
at the top of each child file:
%
\begin{center}
|\input{|\textit{main}|}|
\end{center}
%
A simple redirection |\childdocforward{|\textit{dest}|}| is achieved by:
%
\begin{center}
|\def\jobname{|\textit{dest}|}\input{\jobname}|
\end{center}
%
The redirection with prefix
|\childdocforwardprefix[|\textit{prefix}|]{|\textit{dest}|}|
is accomplished by:
%
\begin{center}
\begin{tabular}{l}
|{\edef\jobname{\scantokens\expandafter{\jobname\noexpand}}|\\
|\def\redirectjob |\textit{prefix}|#1~~~{\gdef\jobname{|\textit{dest}|#1}}|\\
|\expandafter\redirectjob\jobname~~~}\input{\jobname}|
\end{tabular}
\end{center}

In an alternative approach,
child documents can be compiled by a specific command line
without additional code or specific definitions:
%
\begin{center}
|... -jobname "|\textit{target}|" "|[\textit{flags}]%
|\includeonly{|\textit{dest}|}\input{|\textit{main}|}"|
\end{center}
%

%%%%%%%%%%%%%%%%%%%%%%%%%%%%%%%%%%%%%%%%%%%%%%%%%%%%%%%%%%%%%%%%%%%%%%%%%%%%%%%%
%%%%%%%%%%%%%%%%%%%%%%%%%%%%%%%%%%%%%%%%%%%%%%%%%%%%%%%%%%%%%%%%%%%%%%%%%%%%%%%%
\section{Information}

%%%%%%%%%%%%%%%%%%%%%%%%%%%%%%%%%%%%%%%%%%%%%%%%%%%%%%%%%%%%%%%%%%%%%%%%%%%%%%%%
\subsection{Copyright}

Copyright \copyright{} 2017--2018 Niklas Beisert

This work may be distributed and/or modified under the
conditions of the \LaTeX{} Project Public License, either version 1.3
of this license or (at your option) any later version.
The latest version of this license is in
  \url{http://www.latex-project.org/lppl.txt}
and version 1.3 or later is part of all distributions of \LaTeX{}
version 2005/12/01 or later.

This work has the LPPL maintenance status `maintained'.

The Current Maintainer of this work is Niklas Beisert.

This work consists of the files |README.txt|, |childdoc.ins| and |childdoc.dtx|
as well as the derived files |childdoc.def|, |cdocsamp.tex|
with |cdocsch1.tex|, |cdocsch2.tex|, |cdocspt3.tex|, |cdocspt4.tex|,
|cdocsdrf.tex|, |cdocsfn1.tex|, |cdocsfn2.tex|
as well as |childdoc.pdf|.

%%%%%%%%%%%%%%%%%%%%%%%%%%%%%%%%%%%%%%%%%%%%%%%%%%%%%%%%%%%%%%%%%%%%%%%%%%%%%%%%
\subsection{Files and Installation}

The package consists of the files:
%
\begin{center}
\begin{tabular}{ll}
    |README.txt|   & readme file \\
    |childdoc.ins| & installation file \\
    |childdoc.dtx| & source file \\
    |childdoc.def| & definition file \\
    |cdocsamp.tex| & sample main file \\
    |cdocsch1.tex| & sample include file \\
    |cdocsch2.tex| & sample include file \\
    |cdocspt3.tex| & sample part file \\
    |cdocspt4.tex| & sample part file \\
    |cdocsdrf.tex| & sample redirection file \\
    |cdocsfn1.tex| & sample redirection file \\
    |cdocsfn2.tex| & sample redirection file \\
    |childdoc.pdf| & manual
\end{tabular}
\end{center}
%
The distribution consists of the files
|README.txt|, |childdoc.ins| and |childdoc.dtx|.
%
\begin{itemize}
\item
Run (pdf)\LaTeX{} on |childdoc.dtx|
to compile the manual |childdoc.pdf| (this file).
\item
Run \LaTeX{} on |childdoc.ins| to create the definitions file |childdoc.def|
and the sample |cdocsamp.tex| with include files
|cdocsch1.tex|, |cdocsch2.tex|, |cdocspt3.tex|, |cdocspt4.tex|,
|cdocsdrf.tex|, |cdocsfn1.tex|, |cdocsfn2.tex|.
Then copy the file |childdoc.def| to an appropriate directory of your \LaTeX{}
distribution, e.g.\ \textit{texmf-root}|/tex/latex/childdoc|.
\end{itemize}

%%%%%%%%%%%%%%%%%%%%%%%%%%%%%%%%%%%%%%%%%%%%%%%%%%%%%%%%%%%%%%%%%%%%%%%%%%%%%%%%
\subsection{Related CTAN Packages}

There are several other packages which offer a similar functionality:
%
\begin{itemize}
\item
The packages
\href{http://ctan.org/pkg/docmute}{\textsf{docmute}},
\href{http://ctan.org/pkg/includex}{\textsf{includex}} and
\href{http://ctan.org/pkg/standalone}{\textsf{standalone}}
provide commands to include only the document body of
a child file thus allowing both files to be compiled individually.
\item
The packages \href{http://ctan.org/pkg/subdocs}{\textsf{subdocs}}
and \href{http://ctan.org/pkg/subfiles}{\textsf{subfiles}}
provide structures in which the main and child documents can be
encapsulated and allowing them to be compiled individually.
The inclusion mechanism is different from the conventional |\include|.
\item
The package \href{http://ctan.org/pkg/combine}{\textsf{combine}}
is an elaborate solution to combine several documents into one.
\end{itemize}
%
See also the CTAN topic \href{http://ctan.org/topic/subdocs}{\textsf{subdocs}}
for further related packages.
The present package differs from the above solutions in that
a document structure constructed with the conventional |\include| mechanism
just needs two extra commands at the top of every file
such that all constituent files can be compiled individually.

%%%%%%%%%%%%%%%%%%%%%%%%%%%%%%%%%%%%%%%%%%%%%%%%%%%%%%%%%%%%%%%%%%%%%%%%%%%%%%%%
%\subsection{Feature Suggestions}
%
%The following is a list of features which may be useful for future
%versions of this package:
%%
%\begin{itemize}
%\item
%\ldots
%\end{itemize}

%%%%%%%%%%%%%%%%%%%%%%%%%%%%%%%%%%%%%%%%%%%%%%%%%%%%%%%%%%%%%%%%%%%%%%%%%%%%%%%%
\subsection{Revision History}

%%%%%%%%%%%%%%%%%%%%%%%%%%%%%%%%%%%%%%%%
\paragraph{v2.0:} 2018/12/30

\begin{itemize}
\item
immediate forward processing
\item
added |\childdocby| mechanism
\item
manual restructured
\end{itemize}

%%%%%%%%%%%%%%%%%%%%%%%%%%%%%%%%%%%%%%%%
\paragraph{v1.6:} 2018/01/17

\begin{itemize}
\item
application for development of include files
\item
corrections to manual
\end{itemize}

%%%%%%%%%%%%%%%%%%%%%%%%%%%%%%%%%%%%%%%%
\paragraph{v1.5:} 2017/05/21

\begin{itemize}
\item
more complete structuring introduced
\item
|\childdocof| introduced
\item
|\childdoc| renamed to |\childdocmain|
\item
|\childredirect| renamed to |\childdocforward| and |\childdocforwardprefix|
and functionality expanded
\end{itemize}

%%%%%%%%%%%%%%%%%%%%%%%%%%%%%%%%%%%%%%%%
\paragraph{v1.0:} 2017/04/27

\begin{itemize}
\item
manual and install package
\item
first version published on CTAN
\end{itemize}

%%%%%%%%%%%%%%%%%%%%%%%%%%%%%%%%%%%%%%%%
\paragraph{v0.6:} 2017/04/26

\begin{itemize}
\item
redirection mechanism added
\end{itemize}

%%%%%%%%%%%%%%%%%%%%%%%%%%%%%%%%%%%%%%%%
\paragraph{v0.5:} 2017/04/26

\begin{itemize}
\item
functionality in definition file
\end{itemize}


%%%%%%%%%%%%%%%%%%%%%%%%%%%%%%%%%%%%%%%%%%%%%%%%%%%%%%%%%%%%%%%%%%%%%%%%%%%%%%%%
%%%%%%%%%%%%%%%%%%%%%%%%%%%%%%%%%%%%%%%%%%%%%%%%%%%%%%%%%%%%%%%%%%%%%%%%%%%%%%%%
%%%%%%%%%%%%%%%%%%%%%%%%%%%%%%%%%%%%%%%%%%%%%%%%%%%%%%%%%%%%%%%%%%%%%%%%%%%%%%%%
\appendix

\settowidth\MacroIndent{\rmfamily\scriptsize 000\ }

 \DocInput{childdoc.dtx}

\end{document}
%</driver>
% \fi
%
% %%%%%%%%%%%%%%%%%%%%%%%%%%%%%%%%%%%%%%%%%%%%%%%%%%%%%%%%%%%%%%%%%%%%%%%%%%%%%%
% %%%%%%%%%%%%%%%%%%%%%%%%%%%%%%%%%%%%%%%%%%%%%%%%%%%%%%%%%%%%%%%%%%%%%%%%%%%%%%
% \section{Sample}
%\iffalse
%<*samplemain>
%\fi
%
% The following presents a sample document
% with two chapters, two parts, a title page,
% a compile flag as well as three forwarding files to set the flag.
% It consists of eight |.tex| files:
% \begin{center}
% \begin{tabular}{ll}
% |cdocsamp.tex|&main file\\
% |cdocsch1.tex|&include file for chapter 1\\
% |cdocsch2.tex|&include file for chapter 2\\
% |cdocspt3.tex|&include file for part 3\\
% |cdocspt4.tex|&include file for part 4\\
% |cdocsdrf.tex|&forwarding file for main file in draft mode\\
% |cdocsfi1.tex|&forwarding file for final version of chapter 1\\
% |cdocsfi2.tex|&forwarding file for final version of chapter 2\\
% \end{tabular}
% \end{center}
% Each of the eight files can be compiled directly by the \LaTeX{} compiler.
%
% %%%%%%%%%%%%%%%%%%%%%%%%%%%%%%%%%%%%%%
% \paragraph{Main File.}
%
% The main file is called |cdocsamp.tex|.
%
% Load the \textsf{childdoc} definitions and
% declare the filename for the main document:
%    \begin{macrocode}
\input{childdoc.def}
\childdocmain{}
%    \end{macrocode}

% Optional override for |\version| flag:
%    \begin{macrocode}
%%\ifchilddoc\else\providecommand{\version}{draft}\fi
%    \end{macrocode}

% Define the default values for the |\version| flag
% (|final| for the main file and |draft| for childs):
%    \begin{macrocode}
\ifchilddoc
\providecommand{\version}{draft}
\else
\providecommand{\version}{final}
\fi
%    \end{macrocode}

% Load the standard document class:
%    \begin{macrocode}
\documentclass[12pt]{article}
%    \end{macrocode}

% Start the document body:
%    \begin{macrocode}
\begin{document}
%    \end{macrocode}

% Declare a title page.
% Print title, part of document being processed and version flag:
%    \begin{macrocode}
\addtocounter{page}{-1}
\begin{center}
{\LARGE\bfseries{}childdoc example\par}
\vspace{1cm}
\ifchilddoc
\ifchilddocmanual part\else chapter\fi:
`\childdocname' of `\childdocjob'\par
\else
main document: `\childdocjob'\par
\fi
version: \version\par
\end{center}
\newpage
%    \end{macrocode}

% Manually include selected file,
% otherwise process as usual:
%    \begin{macrocode}
\ifchilddocmanual
\section*{part `\childdocname'}
\input{\childdocname}
\else
%    \end{macrocode}

% Include the two chapters:
%    \begin{macrocode}
\include{cdocsch1}
\include{cdocsch2}
%    \end{macrocode}

% Include the two parts unless only chapters should be displayed:
%    \begin{macrocode}
\ifchilddoc\else
\section{part three}
\input{cdocspt3}
\section{part four}
\input{cdocspt4}
\fi
%    \end{macrocode}

% Process as usual until here:
%    \begin{macrocode}
\fi
%    \end{macrocode}

% End of document body:
%    \begin{macrocode}
\end{document}
%    \end{macrocode}
%\iffalse
%</samplemain>
%\fi
%
% %%%%%%%%%%%%%%%%%%%%%%%%%%%%%%%%%%%%%%
% \paragraph{Chapter Include Files.}
%
% The include files are called |cdocsch1.tex| and |cdocsch2.tex|.
%
%\iffalse
%<*samplechap1|samplechap2>
%\fi

% Optional override for |\version| flag:
%    \begin{macrocode}
%%\providecommand{\version}{final}
%    \end{macrocode}

% Include the main document:
%    \begin{macrocode}
\input{childdoc.def}
\childdocof{cdocsamp}
%    \end{macrocode}

%\iffalse
%</samplechap1|samplechap2>
%\fi
%
%\iffalse
%<*samplechap1>
%\fi
% Some text for chapter 1:
%    \begin{macrocode}
\section{one}
some text in chapter one
%    \end{macrocode}

%\iffalse
%</samplechap1>
%\fi
% Some text for chapter 2:
%\iffalse
%<*samplechap2>
%\fi
%    \begin{macrocode}
\section{two}
more text in chapter two
%    \end{macrocode}

%\iffalse
%</samplechap2>
%\fi
%
% %%%%%%%%%%%%%%%%%%%%%%%%%%%%%%%%%%%%%%
% \paragraph{Part Include Files.}
%
% The include files are called |cdocspt3.tex| and |cdocspt4.tex|.
%
%\iffalse
%<*samplepart3|samplepart4>
%\fi

% Optional override for |\version| flag:
%    \begin{macrocode}
%%\providecommand{\version}{final}
%    \end{macrocode}

% Include the main document:
%    \begin{macrocode}
\input{childdoc.def}
\childdocby{cdocsamp}
%    \end{macrocode}

%\iffalse
%</samplepart3|samplepart4>
%\fi
%
%\iffalse
%<*samplepart3>
%\fi
% Some text for part 3:
%    \begin{macrocode}
some text in part three
%    \end{macrocode}

%\iffalse
%</samplepart3>
%\fi
% Some text for part 4:
%\iffalse
%<*samplepart4>
%\fi
%    \begin{macrocode}
more text in part four
%    \end{macrocode}

%\iffalse
%</samplepart4>
%\fi
%
% %%%%%%%%%%%%%%%%%%%%%%%%%%%%%%%%%%%%%%
% \paragraph{Forwarding for a Complete Draft.}
%
% The following forwarding file |cdocsdrf.tex|
% compiles the main document in draft mode:
%\iffalse
%<*sampledraft>
%\fi
%    \begin{macrocode}
\def\version{draft}
\input{childdoc.def}
\childdocforward{cdocsamp}
%    \end{macrocode}

%\iffalse
%</sampledraft>
%\fi
%
% %%%%%%%%%%%%%%%%%%%%%%%%%%%%%%%%%%%%%%
% \paragraph{Forwarding for Final Version of the Chapters.}
%
% The following forwarding files |cdocsfn1.tex| and |cdocsfn2.tex|
% (with identical content)
% compile the final versions of the child documents
% |cdocsch1.tex| and |cdocsch2.tex|, respectively:
%\iffalse
%<*samplefinal>
%\fi
%    \begin{macrocode}
\def\version{final}
\input{childdoc.def}
\childdocforwardprefix[cdocsamp]{cdocsfn}{cdocsch}
%    \end{macrocode}

%\iffalse
%</samplefinal>
%\fi
%
% %%%%%%%%%%%%%%%%%%%%%%%%%%%%%%%%%%%%%%
% \paragraph{Command Line Processing.}
%
% The following three command lines generate the output files
% |cdocscld|, |cdocscl1| and |cdocscl2|
% which should be identical to
% |cdocsdrf|, |cdocsch1| and |cdocsfn2|, respectively:
% \begin{center}
% \begin{tabular}{l}
% |latex -jobname cdocscld \|\\
% |  "\def\version{draft}\input{childdoc.def}\childdocforward{cdocsamp}"|\\
% |latex -jobname cdocscl1 \|\\
% |  "\input{childdoc.def}\childdocforward[cdocsamp]{cdocsch1}"|\\
% |latex -jobname cdocscl2 \|\\
% |  "\def\version{final}\input{childdoc.def}\childdocforward{cdocsch2}"|
% \end{tabular}
% \end{center}
% Note that the trailing backslash on each first line
% merely continues the input to the second line
% (for convenient cut ant paste).
% Furthermore, the command |latex| can be replaced by any
% of its alternative versions such as |pdflatex|.
%
% %%%%%%%%%%%%%%%%%%%%%%%%%%%%%%%%%%%%%%%%%%%%%%%%%%%%%%%%%%%%%%%%%%%%%%%%%%%%%%
% %%%%%%%%%%%%%%%%%%%%%%%%%%%%%%%%%%%%%%%%%%%%%%%%%%%%%%%%%%%%%%%%%%%%%%%%%%%%%%
% \section{Implementation}
%\iffalse
%<*package>
%\fi
%
% This section describes the definitions file |childdoc.def|.

% The definitions cannot be loaded using |\usepackage| or |\RequirePackage|
% which has a mechanism to prevent loading a style file more than once.
% When loading the definitions by means of |\input|
% multiple instances have to be prevented manually:
%\iffalse
%This code needs to be before the `\ProvidesFile' directive
%which is defined at the beginning of this file.
%Therefore it is also placed there and commented out here.
%</package>
%<*discard>
%\fi
%    \begin{macrocode}
\ifdefined\childdocmain\endinput\fi
%    \end{macrocode}
%\iffalse
%</discard>
%<*package>
%\fi
%
% \macro{\ifchilddoc}
% \macro{\ifchilddocmanual}
% The conditional |\ifchilddoc| tells whether a
% child (true) or main (false) document is being compiled.
% The conditional |\ifchilddocmanual| tells whether
% the |\includeonly| mechanism is used (false) or
% the selection of child files must be performed manually (true).
% The definitions initialise to false:
%    \begin{macrocode}
\newif\ifchilddoc
\newif\ifchilddocmanual
%    \end{macrocode}

% \macro{\childdocname}
% \macro{\childdocjob}
% The macro |\childdocname| stores the name of the main document
% to be compiled. The macro |\childdocjob| stores the name of
% the document on which the \LaTeX{} compiler was originally invoked.
% The content of |\jobname| cannot be compared
% to filenames specified in the source due to different catcodes.
% The following code rescans |\jobname|, stores the result
% in |\childdocname| and saves a copy in |\childdocjob|:
%    \begin{macrocode}
\edef\childdocname{\scantokens\expandafter{\jobname\noexpand}}
\let\childdocjob\childdocname
%    \end{macrocode}

% \macro{\childdocdisable}
% The macro |\childdocdisable| prevents the main file
% from being processed more than once.
% At this stage, the main document command |\childdocmain|
% is assumed to be called once again where it should do nothing.
% Any subsequent call to it should prevent
% a secondary processing of the main document
% It overwrites the forwarding commands
% |\childdocof| and |\childdocforward|
% with empty macros to prevent further inclusions of the main document:
%    \begin{macrocode}
\newcommand{\childdocdisable}
{
  \renewcommand{\childdocmain}[1]{\renewcommand{\childdocmain}[1]{\endinput}}
  \renewcommand{\childdocof}[1]{}
  \renewcommand{\childdocby}[2][]{}
  \renewcommand{\childdocforward}[2][]{}
  \renewcommand{\childdocdisable}{}
}
%    \end{macrocode}

% \macro{\childdocmain}
% The macro |\childdocmain| is to be called at the top of the main file
% with nothing or the main filename (without extension) as argument.
% First, it breaks loops.
% If the argument is not empty and does not match |\childdocname|
% (which is set by the first inclusion of |childdoc.def|),
% |\ifchilddoc| is set to true, |\includeonly| is applied to the child file
% and |\jobname| is set to the main file
% (for proper handling of |.aux| files):
%    \begin{macrocode}
\newcommand{\childdocmain}[1]
{
  \childdocdisable\childdocmain{}
  \if?#1?\else
    \begingroup
      \def\childdoctmp{#1}
      \ifx\childdoctmp\childdocname
        \def\childdoctmp{}
      \else
        \def\childdoctmp
        {
          \childdoctrue
          \includeonly{\childdocname}
          \def\childdocjob{#1}
          \def\jobname{#1}
        }
      \fi
      \expandafter
    \endgroup
    \childdoctmp
  \fi
}
%    \end{macrocode}

% \macro{\childdocof}
% The command |\childdocof| redirects
% compilation to the main file |#1|.
%    \begin{macrocode}
\newcommand{\childdocof}[1]
{
  \childdocdisable
  \childdoctrue
  \includeonly{\childdocname}
  \def\jobname{#1}
  \def\childdocjob{#1}
  \input{#1}
}
%    \end{macrocode}

% \macro{\childdocby}
% The command |\childdocby| ....
%    \begin{macrocode}
\newcommand{\childdocby}[2][]
{
  \childdocdisable
  \childdoctrue
  \childdocmanualtrue
  \if?#1?\else
    \def\jobname{#2}
  \fi
  \def\childdocjob{#2}
  \input{#2}
  \endinput
}
%    \end{macrocode}

% \macro{\childdocforward}
% The command |\childdocforward| redirects
% compilation to the main file or
% (if the optional argument is given) a child file.
% Parameters are set as if the main file
% or a child file starting with |\childdocof| was compiled.
% Then compilation is handed over to the main file:
%    \begin{macrocode}
\newcommand{\childdocforward}[2][]
{
  \begingroup
    \if?#1?
      \def\childdoctmp
      {
        \def\childdocname{#2}
        \def\childdocjob{#2}
        \def\jobname{#2}
        \input{#2}
        \endinput
      }
    \else
      \def\childdoctmp
      {
        \childdocdisable
        \def\childdocname{#2}
        \childdoctrue
        \includeonly{#2}
        \def\childdocjob{#1}
        \def\jobname{#1}
        \input{#1}
        \endinput
      }
    \fi
    \expandafter
  \endgroup
  \childdoctmp
}
%    \end{macrocode}

% \macro{\childdocforwardprefix}
% The command |\childdocforwardprefix| redirects
% compilation to the main or a child file by means of a pattern.
% The prefix |#1| in the current filename is replaced by |#2|
% and the suffix of the current filename is kept
% (it is assumed that the filename does not contain the substring `|~~~|'
% which is used as a delimiter).
% Compilation is handed over to the new file by |\childdocforward|:
%    \begin{macrocode}
\newcommand{\childdocforwardprefix}[3][]
{
  \begingroup
    \def\childdocextract #2##1~~~{\def\childdoctmp{\childdocforward[#1]{#3##1}}}
    \expandafter\childdocextract\childdocname~~~
    \expandafter
  \endgroup
  \childdoctmp
}
%    \end{macrocode}

% \macro{\childdoc}
% The deprecated macro |\childdoc| is a legacy version of |\childdocmain|:
%    \begin{macrocode}
\newcommand{\childdoc}{\childdocmain}
%    \end{macrocode}

% \macro{\childdocredirect}
% The deprecated macro |\childdocredirect| is a legacy version
% of |\childdocforward| and |\childdocforwardprefix|:
%    \begin{macrocode}
\newcommand{\childdocredirect}[2][]
{
  \begingroup
    \if?#1?
      \def\childdoctmp{\childdocforward{#2}}
    \else
      \def\childdoctmp{\childdocforwardprefix{#1}{#2}}
    \fi
    \expandafter
  \endgroup
  \childdoctmp
}
%    \end{macrocode}

%\iffalse
%</package>
%\fi
%
\endinput
\childdocforward{cdocsamp}"|\\
% |latex -jobname cdocscl1 \|\\
% |  "% \iffalse
%
% childdoc.dtx Copyright (C) 2017-2018 Niklas Beisert
%
% This work may be distributed and/or modified under the
% conditions of the LaTeX Project Public License, either version 1.3
% of this license or (at your option) any later version.
% The latest version of this license is in
%   http://www.latex-project.org/lppl.txt
% and version 1.3 or later is part of all distributions of LaTeX
% version 2005/12/01 or later.
%
% This work has the LPPL maintenance status `maintained'.
%
% The Current Maintainer of this work is Niklas Beisert.
%
% This work consists of the files childdoc.dtx and childdoc.ins
% and the derived files childdoc.def and cdocsamp.tex with
% cdocsch1.tex, cdocsch2.tex, cdocsdrf.tex, cdocsfn1.tex, cdocsfn2.tex.
%
%<package>\ifdefined\childdocmain\endinput\fi
%<package>\ProvidesFile{childdoc.def}[2018/12/30 v2.0 child document driver]
%<samplemain>\ProvidesFile{cdocsamp.tex}[2018/12/30 v2.0 sample for childdoc]
%<*driver>
%\ProvidesFile{childdoc.drv}[2018/12/30 v2.0 childdoc reference manual file]
\PassOptionsToClass{10pt,a4paper}{article}
\documentclass{ltxdoc}

\usepackage[margin=35mm]{geometry}
\usepackage{hyperref}
\usepackage{hyperxmp}
\usepackage[usenames]{color}

\hypersetup{colorlinks=true}
\hypersetup{pdfstartview=FitH}
\hypersetup{pdfpagemode=UseNone}
\hypersetup{pdfsource={}}
\hypersetup{pdflang={en-UK}}
\hypersetup{pdfcopyright={Copyright 2017-2018 Niklas Beisert.
  This work may be distributed and/or modified under the
  conditions of the LaTeX Project Public License, either version 1.3
  of this license or (at your option) any later version.}}
\hypersetup{pdflicenseurl={http://www.latex-project.org/lppl.txt}}
\hypersetup{pdfcontactaddress={ETH Zurich, ITP, HIT K,
  Wolfgang-Pauli-Strasse 27}}
\hypersetup{pdfcontactpostcode={8093}}
\hypersetup{pdfcontactcity={Zurich}}
\hypersetup{pdfcontactcountry={Switzerland}}
\hypersetup{pdfcontactemail={nbeisert@itp.phys.ethz.ch}}
\hypersetup{pdfcontacturl={http://people.phys.ethz.ch/\xmptilde nbeisert/}}

\newcommand{\secref}[1]{\hyperref[#1]{section \ref*{#1}}}

\parskip1ex
\parindent0pt
\let\olditemize\itemize
\def\itemize{\olditemize\parskip0pt}

\begin{document}

\title{The \textsf{childdoc} Package}
\hypersetup{pdftitle={The childdoc Package}}
\author{Niklas Beisert\\[2ex]
  Institut f\"ur Theoretische Physik\\
  Eidgen\"ossische Technische Hochschule Z\"urich\\
  Wolfgang-Pauli-Strasse 27, 8093 Z\"urich, Switzerland\\[1ex]
  \href{mailto:nbeisert@itp.phys.ethz.ch}
  {\texttt{nbeisert@itp.phys.ethz.ch}}}
\hypersetup{pdfauthor={Niklas Beisert}}
\hypersetup{pdfsubject={Manual for the LaTeX2e Package childdoc}}
\date{30 December 2018, \textsf{v2.0}}
\maketitle

\begin{abstract}\noindent
\textsf{childdoc} is a \LaTeXe{} package
that enables the direct compilation
of document sections included by |\include|
to individual files.
\end{abstract}

\begingroup
\parskip0ex
\tableofcontents
\endgroup

%%%%%%%%%%%%%%%%%%%%%%%%%%%%%%%%%%%%%%%%%%%%%%%%%%%%%%%%%%%%%%%%%%%%%%%%%%%%%%%%
%%%%%%%%%%%%%%%%%%%%%%%%%%%%%%%%%%%%%%%%%%%%%%%%%%%%%%%%%%%%%%%%%%%%%%%%%%%%%%%%
\section{Introduction}

\LaTeX{} provides a mechanism to structure a large document (such as a book)
into a main file and several child files (containing the chapters)
using the |\include| command.
This mechanism is beneficial for documents
which span hundreds of pages in order to
make the source file(s) more manageable.
Moreover, compilation can be restricted to
selected child files by means of the |\includeonly| command.
The latter feature can be used to reduce the compilation time while editing
(this was significantly more useful in the earlier days of \LaTeX{})
or to generate a smaller document which is easier to navigate.
Another application of |\includeonly| is to generate
documents consisting of selected parts of the complete document.

However, there are a few drawbacks of the plain |\include| mechanism:
\begin{itemize}
\item
The child files cannot be compiled on their own,
they can only be compiled via the main file.
A naive editing environment
(such as a text editor with an option
to have the current file processed by \LaTeX)
may require one to switch to the main file before compiling;
attempting to compile the child file produces errors.
\item
The main file must be modified (each time)
to adjust the |\includeonly| command
to the present needs. This easily leaves the main file in a messy state.
\item
The generated document will always carry the filename
of the main document. This is inconvenient if
several child files are to be compiled and
to be kept for distribution.
\end{itemize}

The present package provides a simple interface
to make child files individually compilable by \LaTeX{}.
Compiling a child file then has the same effect as compiling
the main file with an |\includeonly| command
to select the appropriate child.
Moreover the generated document will carry the name of the child
rather than the main file.
This resolves all three above issues.

This feature is meant to make the editing of books,
thesis documents and lecture notes somewhat more convenient.
However, the package can also be used efficiently for
composing a series of documents (such as exercise sheets)
which are typically distributed individually.
It then assists the author in generating the individual documents
(potentially in different versions)
as well as a document containing the collected series.
Another application is in developing style files
or other kinds of included material
where compilation of the style file could redirect
to a sample or test file.

%%%%%%%%%%%%%%%%%%%%%%%%%%%%%%%%%%%%%%%%%%%%%%%%%%%%%%%%%%%%%%%%%%%%%%%%%%%%%%%%
%%%%%%%%%%%%%%%%%%%%%%%%%%%%%%%%%%%%%%%%%%%%%%%%%%%%%%%%%%%%%%%%%%%%%%%%%%%%%%%%
\section{Usage}

First of all, the package \textsf{childdoc} is \emph{not} a standard
\LaTeXe{} |.sty| style file! Therefore it needs to be invoked in
a non-standard way.

%%%%%%%%%%%%%%%%%%%%%%%%%%%%%%%%%%%%%%%%%%%%%%%%%%%%%%%%%%%%%%%%%%%%%%%%%%%%%%%%
\subsection{Included Files}
\label{sec:include}

%%%%%%%%%%%%%%%%%%%%%%%%%%%%%%%%%%%%%%%%
\DescribeMacro{\childdocmain}
To use the package, add the commands
\begin{center}
\begin{tabular}{l}
|\input{childdoc.def}|\\
|\childdocmain{}|\\
\end{tabular}
\end{center}
at the very top of the main \LaTeX{} file,
in particular \emph{before} the |\documentclass| statement!
The argument of |\childdocmain| should be left empty
(but it must be present).

%%%%%%%%%%%%%%%%%%%%%%%%%%%%%%%%%%%%%%%%
\DescribeMacro{\childdocof}
Furthermore, add the commands
\begin{center}
\begin{tabular}{l}
|\input{childdoc.def}|\\
|\childdocof{|\textit{main}|}|\\
\end{tabular}
\end{center}
at the top of every child file \textit{child}
which is included by |\include{|\textit{child}|}|
from within the main file
(or at least for those files to be compiled individually).
The argument \textit{main} must be the filename of the main file.

There are a couple of
considerations in setting up the main and child documents:

%%%%%%%%%%%%%%%%%%%%%%%%%%%%%%%%%%%%%%%%
\paragraph{Restrictions.}

Please note the following restrictions:
\begin{itemize}
\item
|\childdocmain| must be called with one argument \textit{main}
to ensure compatibility with earlier version of the package.
It must either be empty (|\childdocmain{}|)
or precisely match the filename of the main file in which it is specified.
See \secref{sec:detection} for further information.
\item
The filename \textit{main} must be specified without the |.tex| extension.
\item
The filename \textit{main} is case sensitive
(even in case-insensitive file systems)
due to internal string comparison.
\item
The argument \textit{main} should be fully expanded, it cannot be a macro.
\item
Subdirectories and special characters should be avoided in filenames.
\item
The command |\childdocmain{|\textit{main}|}| must be followed by a whitespace.
It should not be followed immediately by another command
or by a comment mark `|%|'.
This is because the \TeX{} parser reads the token immediately following
the argument of |\childdocmain| and puts it
at the beginning of every child section;
however, a white\-space is ignored.
\end{itemize}

%%%%%%%%%%%%%%%%%%%%%%%%%%%%%%%%%%%%%%%%
\paragraph{Content of Main File.}

It is advisable to place all content in the child files included by |\include|.
Any output contained in the main file will appear in all child documents
unless suppressed manually;
it cannot be suppressed automatically by the |\includeonly| directive
and thus should normally be avoided.
A method to include some content in the main file
by means of conditional processing is described in \secref{sec:conditional}.

%%%%%%%%%%%%%%%%%%%%%%%%%%%%%%%%%%%%%%%%
\paragraph{Page Numbering.}

When only a part of the document is compiled,
the appropriate numbering of pages
(as well as other status parameters)
is determined from the |.aux| files.
The latter contain information from previous passes.
However this information needs to propagate through
all intermediate child documents.
Therefore the page numbering in child documents may well
be inconsistent until the complete document is compiled at least once.

A useful (if unconventional) way to always ensure a consistent
page numbering is to restart the numbering in each child document
and denote the pages by `\textit{child}|.|\textit{page}'
where \textit{child} represents the chapter/section number of the child file.
This can be achieved by the command
|\numberwithin{page}{|\textit{child}|}|
of the \textsf{amsmath} package
where \textit{child} can be |chapter| or |section|
depending on the chosen structuring.
Alternatively, one can modify the macro |\thepage| appropriately
and reset the counter |page| at the start of each child file.

%%%%%%%%%%%%%%%%%%%%%%%%%%%%%%%%%%%%%%%%%%%%%%%%%%%%%%%%%%%%%%%%%%%%%%%%%%%%%%%%
\subsection{Conditional Processing}
\label{sec:conditional}

The package provides a mechanism to compile different versions
of a document. To customise the versions further some conditional processing
can come in handy to distinguish which version is being compiled.
The package provides two macros to describe the compilation context:

%%%%%%%%%%%%%%%%%%%%%%%%%%%%%%%%%%%%%%%%
\DescribeMacro{\ifchilddoc}
The conditional |\ifchilddoc| distinguishes between the compilation of
child documents and the main document:
%
\begin{center}
|\ifchilddoc |\textit{child-code}| |[|\||else |\textit{main-code}]| \||fi|
\end{center}

%%%%%%%%%%%%%%%%%%%%%%%%%%%%%%%%%%%%%%%%
\DescribeMacro{\childdocname}
\DescribeMacro{\childdocjob}
The macro |\childdocname| contains the filename (without extension)
of the main or child file being processed.
Note that |\childdocjob| will always contain the name of the main file.

%%%%%%%%%%%%%%%%%%%%%%%%%%%%%%%%%%%%%%%%
\paragraph{Title Page.}

Conditional processing can be used to include a title or banner page
in the main document when proper precautions are taken.
Importantly, the code in the main file should ensure that the page counter
(as well as other status parameters which are stored in the |.aux| files)
takes the same value after the conditional processing.
Otherwise the page numbers may take divergent values
depending on which part is compiled.

For example, a title page could be declared by:
%
\begin{center}
\begin{tabular}{l}
|\ifchilddoc\||else|\\
|\addtocounter{page}{-1}|\\
\textit{code for title page}\\
|\newpage|\\
|\||fi|
\end{tabular}
\end{center}
%
A banner page for the child documents can be generated by:
%
\begin{center}
\begin{tabular}{l}
|\ifchilddoc|\\
|\addtocounter{page}{-1}|\\
\textit{code for banner page}\\
|\newpage|\\
|\||fi|
\end{tabular}
\end{center}
%
Here one could write a message such as:
\begin{center}
|This is the part \childdocname{} of \childdocjob{}.|
\end{center}

%%%%%%%%%%%%%%%%%%%%%%%%%%%%%%%%%%%%%%%%%%%%%%%%%%%%%%%%%%%%%%%%%%%%%%%%%%%%%%%%
\subsection{Flags}
\label{sec:flags}

The package makes it easy to generate different versions
of the main or child documents.
To this end compilation flags can be defined
and assigned different default values.
They will be particularly useful in conjunction
with the forwarding mechanism described in \secref{sec:forward}.

For example, it may be useful to have a flag |\version|
which can be set to |draft| or |final|.
The document source will contain some conditional code
depending on the value of |\version|.
Suppose further, the flag should default to |final| for the main file
and to |draft| for child files
which is a natural assignment for editing the document.
This is achieved by placing the following code
in the preamble of the main document
(below the |\childdocmain| directive):
%
\begin{center}
\begin{tabular}{l}
|\ifchilddoc|\\
|\providecommand{\version}{draft}|\\
|\||else|\\
|\providecommand{\version}{final}|\\
|\||fi|
\end{tabular}
\end{center}
%
The definition by |\providecommand| makes sure
that previous definitions are not overwritten.
Further statements |\providecommand{\version}{...}|
can thus be added before the above code to override it.

For the main file, one might add a line
(between |\childdocmain| and the above block)
%
\begin{center}
|%\ifchilddoc\||else\providecommand{\version}{draft}\||fi|
\end{center}
%
which can be uncommented to produce a draft version.
Likewise one can add a line to the very top of a child file
(above the |\childdocof{|\textit{main}|}| directive)
%
\begin{center}
|%\providecommand{\version}{final}|
\end{center}
%
which can be uncommented to produce the final version of this child document.

%%%%%%%%%%%%%%%%%%%%%%%%%%%%%%%%%%%%%%%%%%%%%%%%%%%%%%%%%%%%%%%%%%%%%%%%%%%%%%%%
\subsection{Forwarding}
\label{sec:forward}

Different versions of the main or child documents
using compilation flags as described in \secref{sec:flags}
can be (permanently) stored in different files
for convenient compilation, viewing and distribution.
To this end, the package defines a command
to pass on compilation to a different file:

%%%%%%%%%%%%%%%%%%%%%%%%%%%%%%%%%%%%%%%%
\DescribeMacro{\childdocforward}
The command |\childdocforward| redirects processing to
another source file:
%
\begin{center}
\begin{tabular}{l}
|\input{childdoc.def}|\\
|\childdocforward[|\textit{main}|]{|\textit{dest}|}|\\
\end{tabular}
\end{center}
%
The argument \textit{dest} is the destination file
(without extension).
It should be the main file or one of the child files.
Note that further \textsf{childdoc} directives
such as |\childdocof| and |\childdocforward|
in the indicated file will be processed in this form.
The optional argument \textit{main}
passes on directly to the main file \textit{main}
while pretending to compile the child \textit{dest}.
This form behaves as if \textit{dest}
issues |\childdocof{|\textit{main}|}| right away,
and no further \textsf{childdoc} directives will be processed.

%%%%%%%%%%%%%%%%%%%%%%%%%%%%%%%%%%%%%%%%
\DescribeMacro{\...prefix}
In the alternative form |\childdocforwardprefix|,
%
\begin{center}
\begin{tabular}{l}
|\input{childdoc.def}|\\
|\childdocforwardprefix[|\textit{main}|]{|\textit{prefix}|}{|\textit{dest}|}|
\end{tabular}
\end{center}
%
the destination file is determined by a pattern
depending on the current file:
To make this work, the current file must be called
`{\textit{prefix}\hspace{0.2em}\textit{suffix}}'
with \textit{prefix} matching precisely the argument.
Processing is then passed on to the file
`{\textit{dest}\hspace{0.2em}\textit{suffix}}'.
Surely, the same effect is achieved by
directly specifying the
argument `{\textit{dest}\hspace{0.2em}\textit{suffix}}'
in the first form.
However, that requires to set up a different file
for each child. With the alternative form of the command
all these files can have exactly the same content
which simplifies setting them up and maintaining them.

For example, the following file |draft.tex|
with a compilation flag |\version| as described in \secref{sec:flags}
compiles the main document as a draft:
%
\begin{center}
\begin{tabular}{l}
|\def\version{draft}|\\
|\input{childdoc.def}|\\
|\childdocforward{|\textit{main}|}|
\end{tabular}
\end{center}
%
Likewise, the following files |final|\textit{nn}|.tex|
compile the final version of the child document
|child|\textit{nn}|.tex|:
%
\begin{center}
\begin{tabular}{l}
|\def\version{final}|\\
|\input{childdoc.def}|\\
|\childdocforwardprefix{final}{child}|
\end{tabular}
\end{center}
%

Note that when several versions of a main file and/or of each child file
are to be generated, it may be convenient to set up a |Makefile| or
shell script to automatise the process.

%%%%%%%%%%%%%%%%%%%%%%%%%%%%%%%%%%%%%%%%%%%%%%%%%%%%%%%%%%%%%%%%%%%%%%%%%%%%%%%%
\subsection{Command Line Processing}
\label{sec:commandline}

The effect of redirection files can also be achieved by invoking
the \LaTeX{} compiler with a more elaborate command line.
Most conveniently this should be done as part
of a shell script or a |Makefile|.

When using \textsf{childdoc} in the main file, the following
command lines effectively perform a redirection
(note that depending on the shell being used,
backslashes may have to be doubled: `|\|' $\to$ `|\\|'):
%
\begin{center}
|... -jobname "|\textit{target}|" |\\|"|[\textit{flags}]%
|\input{childdoc.def}\childdocforward[|\textit{main}|]{|\textit{dest}|}"|
\end{center}
%
Here \textit{target} is the name of the output file,
\textit{main} is the name of the main file
and \textit{dest} is the name of the main or child file to be processed
(all filenames without extensions).
The optional argument \textit{main} can be omitted
if \textit{main} matches \textit{dest}.
Optionally, compilation \textit{flags} can be defined via |\def| commands.
This command line makes the \TeX{} engine believe
it is compiling the file \textit{target}
whose content is specified as the latter parameter.
The provided code then forwards the processing to
\textit{main} or \textit{dest} as described in \secref{sec:forward}.

%%%%%%%%%%%%%%%%%%%%%%%%%%%%%%%%%%%%%%%%%%%%%%%%%%%%%%%%%%%%%%%%%%%%%%%%%%%%%%%%
\subsection{Include by Input}
\label{sec:input}

Including child documents by |\include| has some restrictions by design.
Most notably, the content of a child document always occupies
its own set of pages; pages cannot be shared between child documents.
Usually, this behaviour makes perfect sense
because each child document contain an essential part of the document.
However, in some situations it may be desirable to compose
a document from a collection of parts
without having mandatory page breaks between then.
For this case, the package
provides a mechanism to include parts
by |\input| which can also be processed individually.
However, by construction this mechanism
requires manual handling of the content to be output.

%%%%%%%%%%%%%%%%%%%%%%%%%%%%%%%%%%%%%%%%
\DescribeMacro{\ifchilddocmanual}
The main file should be prepared as usual, see \secref{sec:include}.
However, the document body must make a distinction
between processing of an individual part and of the main document, e.g.:
%
\begin{center}
\begin{tabular}{l}
|\ifchilddocmanual|\\
|\input{\childdocname}|\\
|\||else|\\
\textit{document body with }|\input{|\textit{part}|}|\\
|\||fi|
\end{tabular}
\end{center}
%
The conditional |\ifchilddocmanual| is true whenever
a part to be included by |\input| is being compiled,
and the name of the part is stored in |\childdocname|.

%%%%%%%%%%%%%%%%%%%%%%%%%%%%%%%%%%%%%%%%
\DescribeMacro{\childdocby}
Each part to be included by |\input| should start with:
%
\begin{center}
\begin{tabular}{l}
|\input{childdoc.def}|\\
|\childdocby{|\textit{main}|}|\\
\end{tabular}
\end{center}
%
The directive |\childdocby| is similar to |\childdocof|
described in \secref{sec:include},
but the subsequent selection of content must be done manually.
To that end, both |\ifchilddoc| and |\ifchilddocmanual|
will be true upon processing of a part,
and the name of the part is stored in |\childdocname|.
Note that |\jobname| will be set to the filename of the current part
so that each part receives an individual |.aux| file
that does not interfere with the |.aux| file(s) of the main document.
This behaviour can be altered by the alternative form
|\childdocby[*]{|\textit{main}|}| (with a non-empty optional argument)
which uses the |.aux| file of the main document
by setting |\jobname| to \textit{main}.

%%%%%%%%%%%%%%%%%%%%%%%%%%%%%%%%%%%%%%%%%%%%%%%%%%%%%%%%%%%%%%%%%%%%%%%%%%%%%%%%
\subsection{Driver Development}
\label{sec:driver}

The \textsf{childdoc} mechanism can also be use for the development
of definition files such as \LaTeX{} styles or classes.
This case differs from the above setup with multiple parts
included by |\include| in that no |\includeonly| should be invoked.
This can be achieved by starting the include file
(before |\ProvidesPackage|) with:
%
\begin{center}
\begin{tabular}{l}
|\input{childdoc.def}|\\
|\childdocforward{|\textit{main}|}|\\
\end{tabular}
\end{center}
%
or alternatively with:
%
\begin{center}
\begin{tabular}{l}
|\input{childdoc.def}|\\
|\childdocby{|\textit{main}|}|\\
\end{tabular}
\end{center}
%
Both forms have slightly different effects as described above.
The main file is prepared as usual, see \secref{sec:include}.

%%%%%%%%%%%%%%%%%%%%%%%%%%%%%%%%%%%%%%%%%%%%%%%%%%%%%%%%%%%%%%%%%%%%%%%%%%%%%%%%
\subsection{Legacy Detection}
\label{sec:detection}

The directive |\childdocmain| in the main file can detect
whether the complete document or merely a child is to be compiled
even without using the directive |\childdocof|.
This method is deprecated because it is less robust
and there is no compelling reason to use it;
it is merely provided for backward compatibility
and it may be removed in future versions.

If the detection mechanism is to be used,
it is mandatory to correctly specify
the filename of the main file as the argument of |\childdocmain|:
%
\begin{center}
\begin{tabular}{l}
|\input{childdoc.def}|\\
|\childdocmain{|\textit{main}|}|\\
\end{tabular}
\end{center}
%
If |\jobname| does not match the argument \textit{main} of |\childdocmain|,
it is assumed that |\jobname| points to the child file to be compiled.
When using |\childdocmain| with the main file specified as argument,
it suffices to start a child file
with just |\input{|\textit{main}|}|
without loading of the package and using |\childdocof|.
If instead all processing is done
with the appropriate \textsf{childdoc} directives,
the argument of \textit{main} of |\childdocmain| can be empty.

An alternative version of the command line processing described
in \secref{sec:commandline} using the detection mechanism reads:
%
\begin{center}
|... -jobname "|\textit{target}|" "|[\textit{flags}]%
[|\def\jobname{|\textit{dest}|}|]|\input{|\textit{main}|}"|
\end{center}

%%%%%%%%%%%%%%%%%%%%%%%%%%%%%%%%%%%%%%%%%%%%%%%%%%%%%%%%%%%%%%%%%%%%%%%%%%%%%%%%
\subsection{Manual Code}
\label{sec:manual}

In case one cannot be certain whether the definitions file |childdoc.def|
is installed on the target \TeX{} distribution
and one prefers not to ship it,
it is conceivable to paste a few relevant commands into the sources.

To that end, drop all statements |\input{childdoc.def}|
and perform the replacements as outlined below.
Instead of |\childdocmain{|\textit{main}|}| add the following code
to the top of the main file:
%
\begin{center}
\begin{tabular}{l}
|\||ifdefined\childdocname\endinput\||fi\newif\ifchilddoc|\\
|\edef\childdocname{\scantokens\expandafter{\jobname\noexpand}}|\\
|\def\childdocmain{|\textit{main}|}\||ifx\childdocmain\childdocname\||else|\\
|\childdoctrue\includeonly{\childdocname}\let\jobname\childdocmain\||fi|\\
\end{tabular}
\end{center}
%
Instead of |\childdocof{|\textit{main}|}| just include the main file
at the top of each child file:
%
\begin{center}
|\input{|\textit{main}|}|
\end{center}
%
A simple redirection |\childdocforward{|\textit{dest}|}| is achieved by:
%
\begin{center}
|\def\jobname{|\textit{dest}|}\input{\jobname}|
\end{center}
%
The redirection with prefix
|\childdocforwardprefix[|\textit{prefix}|]{|\textit{dest}|}|
is accomplished by:
%
\begin{center}
\begin{tabular}{l}
|{\edef\jobname{\scantokens\expandafter{\jobname\noexpand}}|\\
|\def\redirectjob |\textit{prefix}|#1~~~{\gdef\jobname{|\textit{dest}|#1}}|\\
|\expandafter\redirectjob\jobname~~~}\input{\jobname}|
\end{tabular}
\end{center}

In an alternative approach,
child documents can be compiled by a specific command line
without additional code or specific definitions:
%
\begin{center}
|... -jobname "|\textit{target}|" "|[\textit{flags}]%
|\includeonly{|\textit{dest}|}\input{|\textit{main}|}"|
\end{center}
%

%%%%%%%%%%%%%%%%%%%%%%%%%%%%%%%%%%%%%%%%%%%%%%%%%%%%%%%%%%%%%%%%%%%%%%%%%%%%%%%%
%%%%%%%%%%%%%%%%%%%%%%%%%%%%%%%%%%%%%%%%%%%%%%%%%%%%%%%%%%%%%%%%%%%%%%%%%%%%%%%%
\section{Information}

%%%%%%%%%%%%%%%%%%%%%%%%%%%%%%%%%%%%%%%%%%%%%%%%%%%%%%%%%%%%%%%%%%%%%%%%%%%%%%%%
\subsection{Copyright}

Copyright \copyright{} 2017--2018 Niklas Beisert

This work may be distributed and/or modified under the
conditions of the \LaTeX{} Project Public License, either version 1.3
of this license or (at your option) any later version.
The latest version of this license is in
  \url{http://www.latex-project.org/lppl.txt}
and version 1.3 or later is part of all distributions of \LaTeX{}
version 2005/12/01 or later.

This work has the LPPL maintenance status `maintained'.

The Current Maintainer of this work is Niklas Beisert.

This work consists of the files |README.txt|, |childdoc.ins| and |childdoc.dtx|
as well as the derived files |childdoc.def|, |cdocsamp.tex|
with |cdocsch1.tex|, |cdocsch2.tex|, |cdocspt3.tex|, |cdocspt4.tex|,
|cdocsdrf.tex|, |cdocsfn1.tex|, |cdocsfn2.tex|
as well as |childdoc.pdf|.

%%%%%%%%%%%%%%%%%%%%%%%%%%%%%%%%%%%%%%%%%%%%%%%%%%%%%%%%%%%%%%%%%%%%%%%%%%%%%%%%
\subsection{Files and Installation}

The package consists of the files:
%
\begin{center}
\begin{tabular}{ll}
    |README.txt|   & readme file \\
    |childdoc.ins| & installation file \\
    |childdoc.dtx| & source file \\
    |childdoc.def| & definition file \\
    |cdocsamp.tex| & sample main file \\
    |cdocsch1.tex| & sample include file \\
    |cdocsch2.tex| & sample include file \\
    |cdocspt3.tex| & sample part file \\
    |cdocspt4.tex| & sample part file \\
    |cdocsdrf.tex| & sample redirection file \\
    |cdocsfn1.tex| & sample redirection file \\
    |cdocsfn2.tex| & sample redirection file \\
    |childdoc.pdf| & manual
\end{tabular}
\end{center}
%
The distribution consists of the files
|README.txt|, |childdoc.ins| and |childdoc.dtx|.
%
\begin{itemize}
\item
Run (pdf)\LaTeX{} on |childdoc.dtx|
to compile the manual |childdoc.pdf| (this file).
\item
Run \LaTeX{} on |childdoc.ins| to create the definitions file |childdoc.def|
and the sample |cdocsamp.tex| with include files
|cdocsch1.tex|, |cdocsch2.tex|, |cdocspt3.tex|, |cdocspt4.tex|,
|cdocsdrf.tex|, |cdocsfn1.tex|, |cdocsfn2.tex|.
Then copy the file |childdoc.def| to an appropriate directory of your \LaTeX{}
distribution, e.g.\ \textit{texmf-root}|/tex/latex/childdoc|.
\end{itemize}

%%%%%%%%%%%%%%%%%%%%%%%%%%%%%%%%%%%%%%%%%%%%%%%%%%%%%%%%%%%%%%%%%%%%%%%%%%%%%%%%
\subsection{Related CTAN Packages}

There are several other packages which offer a similar functionality:
%
\begin{itemize}
\item
The packages
\href{http://ctan.org/pkg/docmute}{\textsf{docmute}},
\href{http://ctan.org/pkg/includex}{\textsf{includex}} and
\href{http://ctan.org/pkg/standalone}{\textsf{standalone}}
provide commands to include only the document body of
a child file thus allowing both files to be compiled individually.
\item
The packages \href{http://ctan.org/pkg/subdocs}{\textsf{subdocs}}
and \href{http://ctan.org/pkg/subfiles}{\textsf{subfiles}}
provide structures in which the main and child documents can be
encapsulated and allowing them to be compiled individually.
The inclusion mechanism is different from the conventional |\include|.
\item
The package \href{http://ctan.org/pkg/combine}{\textsf{combine}}
is an elaborate solution to combine several documents into one.
\end{itemize}
%
See also the CTAN topic \href{http://ctan.org/topic/subdocs}{\textsf{subdocs}}
for further related packages.
The present package differs from the above solutions in that
a document structure constructed with the conventional |\include| mechanism
just needs two extra commands at the top of every file
such that all constituent files can be compiled individually.

%%%%%%%%%%%%%%%%%%%%%%%%%%%%%%%%%%%%%%%%%%%%%%%%%%%%%%%%%%%%%%%%%%%%%%%%%%%%%%%%
%\subsection{Feature Suggestions}
%
%The following is a list of features which may be useful for future
%versions of this package:
%%
%\begin{itemize}
%\item
%\ldots
%\end{itemize}

%%%%%%%%%%%%%%%%%%%%%%%%%%%%%%%%%%%%%%%%%%%%%%%%%%%%%%%%%%%%%%%%%%%%%%%%%%%%%%%%
\subsection{Revision History}

%%%%%%%%%%%%%%%%%%%%%%%%%%%%%%%%%%%%%%%%
\paragraph{v2.0:} 2018/12/30

\begin{itemize}
\item
immediate forward processing
\item
added |\childdocby| mechanism
\item
manual restructured
\end{itemize}

%%%%%%%%%%%%%%%%%%%%%%%%%%%%%%%%%%%%%%%%
\paragraph{v1.6:} 2018/01/17

\begin{itemize}
\item
application for development of include files
\item
corrections to manual
\end{itemize}

%%%%%%%%%%%%%%%%%%%%%%%%%%%%%%%%%%%%%%%%
\paragraph{v1.5:} 2017/05/21

\begin{itemize}
\item
more complete structuring introduced
\item
|\childdocof| introduced
\item
|\childdoc| renamed to |\childdocmain|
\item
|\childredirect| renamed to |\childdocforward| and |\childdocforwardprefix|
and functionality expanded
\end{itemize}

%%%%%%%%%%%%%%%%%%%%%%%%%%%%%%%%%%%%%%%%
\paragraph{v1.0:} 2017/04/27

\begin{itemize}
\item
manual and install package
\item
first version published on CTAN
\end{itemize}

%%%%%%%%%%%%%%%%%%%%%%%%%%%%%%%%%%%%%%%%
\paragraph{v0.6:} 2017/04/26

\begin{itemize}
\item
redirection mechanism added
\end{itemize}

%%%%%%%%%%%%%%%%%%%%%%%%%%%%%%%%%%%%%%%%
\paragraph{v0.5:} 2017/04/26

\begin{itemize}
\item
functionality in definition file
\end{itemize}


%%%%%%%%%%%%%%%%%%%%%%%%%%%%%%%%%%%%%%%%%%%%%%%%%%%%%%%%%%%%%%%%%%%%%%%%%%%%%%%%
%%%%%%%%%%%%%%%%%%%%%%%%%%%%%%%%%%%%%%%%%%%%%%%%%%%%%%%%%%%%%%%%%%%%%%%%%%%%%%%%
%%%%%%%%%%%%%%%%%%%%%%%%%%%%%%%%%%%%%%%%%%%%%%%%%%%%%%%%%%%%%%%%%%%%%%%%%%%%%%%%
\appendix

\settowidth\MacroIndent{\rmfamily\scriptsize 000\ }

 \DocInput{childdoc.dtx}

\end{document}
%</driver>
% \fi
%
% %%%%%%%%%%%%%%%%%%%%%%%%%%%%%%%%%%%%%%%%%%%%%%%%%%%%%%%%%%%%%%%%%%%%%%%%%%%%%%
% %%%%%%%%%%%%%%%%%%%%%%%%%%%%%%%%%%%%%%%%%%%%%%%%%%%%%%%%%%%%%%%%%%%%%%%%%%%%%%
% \section{Sample}
%\iffalse
%<*samplemain>
%\fi
%
% The following presents a sample document
% with two chapters, two parts, a title page,
% a compile flag as well as three forwarding files to set the flag.
% It consists of eight |.tex| files:
% \begin{center}
% \begin{tabular}{ll}
% |cdocsamp.tex|&main file\\
% |cdocsch1.tex|&include file for chapter 1\\
% |cdocsch2.tex|&include file for chapter 2\\
% |cdocspt3.tex|&include file for part 3\\
% |cdocspt4.tex|&include file for part 4\\
% |cdocsdrf.tex|&forwarding file for main file in draft mode\\
% |cdocsfi1.tex|&forwarding file for final version of chapter 1\\
% |cdocsfi2.tex|&forwarding file for final version of chapter 2\\
% \end{tabular}
% \end{center}
% Each of the eight files can be compiled directly by the \LaTeX{} compiler.
%
% %%%%%%%%%%%%%%%%%%%%%%%%%%%%%%%%%%%%%%
% \paragraph{Main File.}
%
% The main file is called |cdocsamp.tex|.
%
% Load the \textsf{childdoc} definitions and
% declare the filename for the main document:
%    \begin{macrocode}
\input{childdoc.def}
\childdocmain{}
%    \end{macrocode}

% Optional override for |\version| flag:
%    \begin{macrocode}
%%\ifchilddoc\else\providecommand{\version}{draft}\fi
%    \end{macrocode}

% Define the default values for the |\version| flag
% (|final| for the main file and |draft| for childs):
%    \begin{macrocode}
\ifchilddoc
\providecommand{\version}{draft}
\else
\providecommand{\version}{final}
\fi
%    \end{macrocode}

% Load the standard document class:
%    \begin{macrocode}
\documentclass[12pt]{article}
%    \end{macrocode}

% Start the document body:
%    \begin{macrocode}
\begin{document}
%    \end{macrocode}

% Declare a title page.
% Print title, part of document being processed and version flag:
%    \begin{macrocode}
\addtocounter{page}{-1}
\begin{center}
{\LARGE\bfseries{}childdoc example\par}
\vspace{1cm}
\ifchilddoc
\ifchilddocmanual part\else chapter\fi:
`\childdocname' of `\childdocjob'\par
\else
main document: `\childdocjob'\par
\fi
version: \version\par
\end{center}
\newpage
%    \end{macrocode}

% Manually include selected file,
% otherwise process as usual:
%    \begin{macrocode}
\ifchilddocmanual
\section*{part `\childdocname'}
\input{\childdocname}
\else
%    \end{macrocode}

% Include the two chapters:
%    \begin{macrocode}
\include{cdocsch1}
\include{cdocsch2}
%    \end{macrocode}

% Include the two parts unless only chapters should be displayed:
%    \begin{macrocode}
\ifchilddoc\else
\section{part three}
\input{cdocspt3}
\section{part four}
\input{cdocspt4}
\fi
%    \end{macrocode}

% Process as usual until here:
%    \begin{macrocode}
\fi
%    \end{macrocode}

% End of document body:
%    \begin{macrocode}
\end{document}
%    \end{macrocode}
%\iffalse
%</samplemain>
%\fi
%
% %%%%%%%%%%%%%%%%%%%%%%%%%%%%%%%%%%%%%%
% \paragraph{Chapter Include Files.}
%
% The include files are called |cdocsch1.tex| and |cdocsch2.tex|.
%
%\iffalse
%<*samplechap1|samplechap2>
%\fi

% Optional override for |\version| flag:
%    \begin{macrocode}
%%\providecommand{\version}{final}
%    \end{macrocode}

% Include the main document:
%    \begin{macrocode}
\input{childdoc.def}
\childdocof{cdocsamp}
%    \end{macrocode}

%\iffalse
%</samplechap1|samplechap2>
%\fi
%
%\iffalse
%<*samplechap1>
%\fi
% Some text for chapter 1:
%    \begin{macrocode}
\section{one}
some text in chapter one
%    \end{macrocode}

%\iffalse
%</samplechap1>
%\fi
% Some text for chapter 2:
%\iffalse
%<*samplechap2>
%\fi
%    \begin{macrocode}
\section{two}
more text in chapter two
%    \end{macrocode}

%\iffalse
%</samplechap2>
%\fi
%
% %%%%%%%%%%%%%%%%%%%%%%%%%%%%%%%%%%%%%%
% \paragraph{Part Include Files.}
%
% The include files are called |cdocspt3.tex| and |cdocspt4.tex|.
%
%\iffalse
%<*samplepart3|samplepart4>
%\fi

% Optional override for |\version| flag:
%    \begin{macrocode}
%%\providecommand{\version}{final}
%    \end{macrocode}

% Include the main document:
%    \begin{macrocode}
\input{childdoc.def}
\childdocby{cdocsamp}
%    \end{macrocode}

%\iffalse
%</samplepart3|samplepart4>
%\fi
%
%\iffalse
%<*samplepart3>
%\fi
% Some text for part 3:
%    \begin{macrocode}
some text in part three
%    \end{macrocode}

%\iffalse
%</samplepart3>
%\fi
% Some text for part 4:
%\iffalse
%<*samplepart4>
%\fi
%    \begin{macrocode}
more text in part four
%    \end{macrocode}

%\iffalse
%</samplepart4>
%\fi
%
% %%%%%%%%%%%%%%%%%%%%%%%%%%%%%%%%%%%%%%
% \paragraph{Forwarding for a Complete Draft.}
%
% The following forwarding file |cdocsdrf.tex|
% compiles the main document in draft mode:
%\iffalse
%<*sampledraft>
%\fi
%    \begin{macrocode}
\def\version{draft}
\input{childdoc.def}
\childdocforward{cdocsamp}
%    \end{macrocode}

%\iffalse
%</sampledraft>
%\fi
%
% %%%%%%%%%%%%%%%%%%%%%%%%%%%%%%%%%%%%%%
% \paragraph{Forwarding for Final Version of the Chapters.}
%
% The following forwarding files |cdocsfn1.tex| and |cdocsfn2.tex|
% (with identical content)
% compile the final versions of the child documents
% |cdocsch1.tex| and |cdocsch2.tex|, respectively:
%\iffalse
%<*samplefinal>
%\fi
%    \begin{macrocode}
\def\version{final}
\input{childdoc.def}
\childdocforwardprefix[cdocsamp]{cdocsfn}{cdocsch}
%    \end{macrocode}

%\iffalse
%</samplefinal>
%\fi
%
% %%%%%%%%%%%%%%%%%%%%%%%%%%%%%%%%%%%%%%
% \paragraph{Command Line Processing.}
%
% The following three command lines generate the output files
% |cdocscld|, |cdocscl1| and |cdocscl2|
% which should be identical to
% |cdocsdrf|, |cdocsch1| and |cdocsfn2|, respectively:
% \begin{center}
% \begin{tabular}{l}
% |latex -jobname cdocscld \|\\
% |  "\def\version{draft}\input{childdoc.def}\childdocforward{cdocsamp}"|\\
% |latex -jobname cdocscl1 \|\\
% |  "\input{childdoc.def}\childdocforward[cdocsamp]{cdocsch1}"|\\
% |latex -jobname cdocscl2 \|\\
% |  "\def\version{final}\input{childdoc.def}\childdocforward{cdocsch2}"|
% \end{tabular}
% \end{center}
% Note that the trailing backslash on each first line
% merely continues the input to the second line
% (for convenient cut ant paste).
% Furthermore, the command |latex| can be replaced by any
% of its alternative versions such as |pdflatex|.
%
% %%%%%%%%%%%%%%%%%%%%%%%%%%%%%%%%%%%%%%%%%%%%%%%%%%%%%%%%%%%%%%%%%%%%%%%%%%%%%%
% %%%%%%%%%%%%%%%%%%%%%%%%%%%%%%%%%%%%%%%%%%%%%%%%%%%%%%%%%%%%%%%%%%%%%%%%%%%%%%
% \section{Implementation}
%\iffalse
%<*package>
%\fi
%
% This section describes the definitions file |childdoc.def|.

% The definitions cannot be loaded using |\usepackage| or |\RequirePackage|
% which has a mechanism to prevent loading a style file more than once.
% When loading the definitions by means of |\input|
% multiple instances have to be prevented manually:
%\iffalse
%This code needs to be before the `\ProvidesFile' directive
%which is defined at the beginning of this file.
%Therefore it is also placed there and commented out here.
%</package>
%<*discard>
%\fi
%    \begin{macrocode}
\ifdefined\childdocmain\endinput\fi
%    \end{macrocode}
%\iffalse
%</discard>
%<*package>
%\fi
%
% \macro{\ifchilddoc}
% \macro{\ifchilddocmanual}
% The conditional |\ifchilddoc| tells whether a
% child (true) or main (false) document is being compiled.
% The conditional |\ifchilddocmanual| tells whether
% the |\includeonly| mechanism is used (false) or
% the selection of child files must be performed manually (true).
% The definitions initialise to false:
%    \begin{macrocode}
\newif\ifchilddoc
\newif\ifchilddocmanual
%    \end{macrocode}

% \macro{\childdocname}
% \macro{\childdocjob}
% The macro |\childdocname| stores the name of the main document
% to be compiled. The macro |\childdocjob| stores the name of
% the document on which the \LaTeX{} compiler was originally invoked.
% The content of |\jobname| cannot be compared
% to filenames specified in the source due to different catcodes.
% The following code rescans |\jobname|, stores the result
% in |\childdocname| and saves a copy in |\childdocjob|:
%    \begin{macrocode}
\edef\childdocname{\scantokens\expandafter{\jobname\noexpand}}
\let\childdocjob\childdocname
%    \end{macrocode}

% \macro{\childdocdisable}
% The macro |\childdocdisable| prevents the main file
% from being processed more than once.
% At this stage, the main document command |\childdocmain|
% is assumed to be called once again where it should do nothing.
% Any subsequent call to it should prevent
% a secondary processing of the main document
% It overwrites the forwarding commands
% |\childdocof| and |\childdocforward|
% with empty macros to prevent further inclusions of the main document:
%    \begin{macrocode}
\newcommand{\childdocdisable}
{
  \renewcommand{\childdocmain}[1]{\renewcommand{\childdocmain}[1]{\endinput}}
  \renewcommand{\childdocof}[1]{}
  \renewcommand{\childdocby}[2][]{}
  \renewcommand{\childdocforward}[2][]{}
  \renewcommand{\childdocdisable}{}
}
%    \end{macrocode}

% \macro{\childdocmain}
% The macro |\childdocmain| is to be called at the top of the main file
% with nothing or the main filename (without extension) as argument.
% First, it breaks loops.
% If the argument is not empty and does not match |\childdocname|
% (which is set by the first inclusion of |childdoc.def|),
% |\ifchilddoc| is set to true, |\includeonly| is applied to the child file
% and |\jobname| is set to the main file
% (for proper handling of |.aux| files):
%    \begin{macrocode}
\newcommand{\childdocmain}[1]
{
  \childdocdisable\childdocmain{}
  \if?#1?\else
    \begingroup
      \def\childdoctmp{#1}
      \ifx\childdoctmp\childdocname
        \def\childdoctmp{}
      \else
        \def\childdoctmp
        {
          \childdoctrue
          \includeonly{\childdocname}
          \def\childdocjob{#1}
          \def\jobname{#1}
        }
      \fi
      \expandafter
    \endgroup
    \childdoctmp
  \fi
}
%    \end{macrocode}

% \macro{\childdocof}
% The command |\childdocof| redirects
% compilation to the main file |#1|.
%    \begin{macrocode}
\newcommand{\childdocof}[1]
{
  \childdocdisable
  \childdoctrue
  \includeonly{\childdocname}
  \def\jobname{#1}
  \def\childdocjob{#1}
  \input{#1}
}
%    \end{macrocode}

% \macro{\childdocby}
% The command |\childdocby| ....
%    \begin{macrocode}
\newcommand{\childdocby}[2][]
{
  \childdocdisable
  \childdoctrue
  \childdocmanualtrue
  \if?#1?\else
    \def\jobname{#2}
  \fi
  \def\childdocjob{#2}
  \input{#2}
  \endinput
}
%    \end{macrocode}

% \macro{\childdocforward}
% The command |\childdocforward| redirects
% compilation to the main file or
% (if the optional argument is given) a child file.
% Parameters are set as if the main file
% or a child file starting with |\childdocof| was compiled.
% Then compilation is handed over to the main file:
%    \begin{macrocode}
\newcommand{\childdocforward}[2][]
{
  \begingroup
    \if?#1?
      \def\childdoctmp
      {
        \def\childdocname{#2}
        \def\childdocjob{#2}
        \def\jobname{#2}
        \input{#2}
        \endinput
      }
    \else
      \def\childdoctmp
      {
        \childdocdisable
        \def\childdocname{#2}
        \childdoctrue
        \includeonly{#2}
        \def\childdocjob{#1}
        \def\jobname{#1}
        \input{#1}
        \endinput
      }
    \fi
    \expandafter
  \endgroup
  \childdoctmp
}
%    \end{macrocode}

% \macro{\childdocforwardprefix}
% The command |\childdocforwardprefix| redirects
% compilation to the main or a child file by means of a pattern.
% The prefix |#1| in the current filename is replaced by |#2|
% and the suffix of the current filename is kept
% (it is assumed that the filename does not contain the substring `|~~~|'
% which is used as a delimiter).
% Compilation is handed over to the new file by |\childdocforward|:
%    \begin{macrocode}
\newcommand{\childdocforwardprefix}[3][]
{
  \begingroup
    \def\childdocextract #2##1~~~{\def\childdoctmp{\childdocforward[#1]{#3##1}}}
    \expandafter\childdocextract\childdocname~~~
    \expandafter
  \endgroup
  \childdoctmp
}
%    \end{macrocode}

% \macro{\childdoc}
% The deprecated macro |\childdoc| is a legacy version of |\childdocmain|:
%    \begin{macrocode}
\newcommand{\childdoc}{\childdocmain}
%    \end{macrocode}

% \macro{\childdocredirect}
% The deprecated macro |\childdocredirect| is a legacy version
% of |\childdocforward| and |\childdocforwardprefix|:
%    \begin{macrocode}
\newcommand{\childdocredirect}[2][]
{
  \begingroup
    \if?#1?
      \def\childdoctmp{\childdocforward{#2}}
    \else
      \def\childdoctmp{\childdocforwardprefix{#1}{#2}}
    \fi
    \expandafter
  \endgroup
  \childdoctmp
}
%    \end{macrocode}

%\iffalse
%</package>
%\fi
%
\endinput
\childdocforward[cdocsamp]{cdocsch1}"|\\
% |latex -jobname cdocscl2 \|\\
% |  "\def\version{final}% \iffalse
%
% childdoc.dtx Copyright (C) 2017-2018 Niklas Beisert
%
% This work may be distributed and/or modified under the
% conditions of the LaTeX Project Public License, either version 1.3
% of this license or (at your option) any later version.
% The latest version of this license is in
%   http://www.latex-project.org/lppl.txt
% and version 1.3 or later is part of all distributions of LaTeX
% version 2005/12/01 or later.
%
% This work has the LPPL maintenance status `maintained'.
%
% The Current Maintainer of this work is Niklas Beisert.
%
% This work consists of the files childdoc.dtx and childdoc.ins
% and the derived files childdoc.def and cdocsamp.tex with
% cdocsch1.tex, cdocsch2.tex, cdocsdrf.tex, cdocsfn1.tex, cdocsfn2.tex.
%
%<package>\ifdefined\childdocmain\endinput\fi
%<package>\ProvidesFile{childdoc.def}[2018/12/30 v2.0 child document driver]
%<samplemain>\ProvidesFile{cdocsamp.tex}[2018/12/30 v2.0 sample for childdoc]
%<*driver>
%\ProvidesFile{childdoc.drv}[2018/12/30 v2.0 childdoc reference manual file]
\PassOptionsToClass{10pt,a4paper}{article}
\documentclass{ltxdoc}

\usepackage[margin=35mm]{geometry}
\usepackage{hyperref}
\usepackage{hyperxmp}
\usepackage[usenames]{color}

\hypersetup{colorlinks=true}
\hypersetup{pdfstartview=FitH}
\hypersetup{pdfpagemode=UseNone}
\hypersetup{pdfsource={}}
\hypersetup{pdflang={en-UK}}
\hypersetup{pdfcopyright={Copyright 2017-2018 Niklas Beisert.
  This work may be distributed and/or modified under the
  conditions of the LaTeX Project Public License, either version 1.3
  of this license or (at your option) any later version.}}
\hypersetup{pdflicenseurl={http://www.latex-project.org/lppl.txt}}
\hypersetup{pdfcontactaddress={ETH Zurich, ITP, HIT K,
  Wolfgang-Pauli-Strasse 27}}
\hypersetup{pdfcontactpostcode={8093}}
\hypersetup{pdfcontactcity={Zurich}}
\hypersetup{pdfcontactcountry={Switzerland}}
\hypersetup{pdfcontactemail={nbeisert@itp.phys.ethz.ch}}
\hypersetup{pdfcontacturl={http://people.phys.ethz.ch/\xmptilde nbeisert/}}

\newcommand{\secref}[1]{\hyperref[#1]{section \ref*{#1}}}

\parskip1ex
\parindent0pt
\let\olditemize\itemize
\def\itemize{\olditemize\parskip0pt}

\begin{document}

\title{The \textsf{childdoc} Package}
\hypersetup{pdftitle={The childdoc Package}}
\author{Niklas Beisert\\[2ex]
  Institut f\"ur Theoretische Physik\\
  Eidgen\"ossische Technische Hochschule Z\"urich\\
  Wolfgang-Pauli-Strasse 27, 8093 Z\"urich, Switzerland\\[1ex]
  \href{mailto:nbeisert@itp.phys.ethz.ch}
  {\texttt{nbeisert@itp.phys.ethz.ch}}}
\hypersetup{pdfauthor={Niklas Beisert}}
\hypersetup{pdfsubject={Manual for the LaTeX2e Package childdoc}}
\date{30 December 2018, \textsf{v2.0}}
\maketitle

\begin{abstract}\noindent
\textsf{childdoc} is a \LaTeXe{} package
that enables the direct compilation
of document sections included by |\include|
to individual files.
\end{abstract}

\begingroup
\parskip0ex
\tableofcontents
\endgroup

%%%%%%%%%%%%%%%%%%%%%%%%%%%%%%%%%%%%%%%%%%%%%%%%%%%%%%%%%%%%%%%%%%%%%%%%%%%%%%%%
%%%%%%%%%%%%%%%%%%%%%%%%%%%%%%%%%%%%%%%%%%%%%%%%%%%%%%%%%%%%%%%%%%%%%%%%%%%%%%%%
\section{Introduction}

\LaTeX{} provides a mechanism to structure a large document (such as a book)
into a main file and several child files (containing the chapters)
using the |\include| command.
This mechanism is beneficial for documents
which span hundreds of pages in order to
make the source file(s) more manageable.
Moreover, compilation can be restricted to
selected child files by means of the |\includeonly| command.
The latter feature can be used to reduce the compilation time while editing
(this was significantly more useful in the earlier days of \LaTeX{})
or to generate a smaller document which is easier to navigate.
Another application of |\includeonly| is to generate
documents consisting of selected parts of the complete document.

However, there are a few drawbacks of the plain |\include| mechanism:
\begin{itemize}
\item
The child files cannot be compiled on their own,
they can only be compiled via the main file.
A naive editing environment
(such as a text editor with an option
to have the current file processed by \LaTeX)
may require one to switch to the main file before compiling;
attempting to compile the child file produces errors.
\item
The main file must be modified (each time)
to adjust the |\includeonly| command
to the present needs. This easily leaves the main file in a messy state.
\item
The generated document will always carry the filename
of the main document. This is inconvenient if
several child files are to be compiled and
to be kept for distribution.
\end{itemize}

The present package provides a simple interface
to make child files individually compilable by \LaTeX{}.
Compiling a child file then has the same effect as compiling
the main file with an |\includeonly| command
to select the appropriate child.
Moreover the generated document will carry the name of the child
rather than the main file.
This resolves all three above issues.

This feature is meant to make the editing of books,
thesis documents and lecture notes somewhat more convenient.
However, the package can also be used efficiently for
composing a series of documents (such as exercise sheets)
which are typically distributed individually.
It then assists the author in generating the individual documents
(potentially in different versions)
as well as a document containing the collected series.
Another application is in developing style files
or other kinds of included material
where compilation of the style file could redirect
to a sample or test file.

%%%%%%%%%%%%%%%%%%%%%%%%%%%%%%%%%%%%%%%%%%%%%%%%%%%%%%%%%%%%%%%%%%%%%%%%%%%%%%%%
%%%%%%%%%%%%%%%%%%%%%%%%%%%%%%%%%%%%%%%%%%%%%%%%%%%%%%%%%%%%%%%%%%%%%%%%%%%%%%%%
\section{Usage}

First of all, the package \textsf{childdoc} is \emph{not} a standard
\LaTeXe{} |.sty| style file! Therefore it needs to be invoked in
a non-standard way.

%%%%%%%%%%%%%%%%%%%%%%%%%%%%%%%%%%%%%%%%%%%%%%%%%%%%%%%%%%%%%%%%%%%%%%%%%%%%%%%%
\subsection{Included Files}
\label{sec:include}

%%%%%%%%%%%%%%%%%%%%%%%%%%%%%%%%%%%%%%%%
\DescribeMacro{\childdocmain}
To use the package, add the commands
\begin{center}
\begin{tabular}{l}
|\input{childdoc.def}|\\
|\childdocmain{}|\\
\end{tabular}
\end{center}
at the very top of the main \LaTeX{} file,
in particular \emph{before} the |\documentclass| statement!
The argument of |\childdocmain| should be left empty
(but it must be present).

%%%%%%%%%%%%%%%%%%%%%%%%%%%%%%%%%%%%%%%%
\DescribeMacro{\childdocof}
Furthermore, add the commands
\begin{center}
\begin{tabular}{l}
|\input{childdoc.def}|\\
|\childdocof{|\textit{main}|}|\\
\end{tabular}
\end{center}
at the top of every child file \textit{child}
which is included by |\include{|\textit{child}|}|
from within the main file
(or at least for those files to be compiled individually).
The argument \textit{main} must be the filename of the main file.

There are a couple of
considerations in setting up the main and child documents:

%%%%%%%%%%%%%%%%%%%%%%%%%%%%%%%%%%%%%%%%
\paragraph{Restrictions.}

Please note the following restrictions:
\begin{itemize}
\item
|\childdocmain| must be called with one argument \textit{main}
to ensure compatibility with earlier version of the package.
It must either be empty (|\childdocmain{}|)
or precisely match the filename of the main file in which it is specified.
See \secref{sec:detection} for further information.
\item
The filename \textit{main} must be specified without the |.tex| extension.
\item
The filename \textit{main} is case sensitive
(even in case-insensitive file systems)
due to internal string comparison.
\item
The argument \textit{main} should be fully expanded, it cannot be a macro.
\item
Subdirectories and special characters should be avoided in filenames.
\item
The command |\childdocmain{|\textit{main}|}| must be followed by a whitespace.
It should not be followed immediately by another command
or by a comment mark `|%|'.
This is because the \TeX{} parser reads the token immediately following
the argument of |\childdocmain| and puts it
at the beginning of every child section;
however, a white\-space is ignored.
\end{itemize}

%%%%%%%%%%%%%%%%%%%%%%%%%%%%%%%%%%%%%%%%
\paragraph{Content of Main File.}

It is advisable to place all content in the child files included by |\include|.
Any output contained in the main file will appear in all child documents
unless suppressed manually;
it cannot be suppressed automatically by the |\includeonly| directive
and thus should normally be avoided.
A method to include some content in the main file
by means of conditional processing is described in \secref{sec:conditional}.

%%%%%%%%%%%%%%%%%%%%%%%%%%%%%%%%%%%%%%%%
\paragraph{Page Numbering.}

When only a part of the document is compiled,
the appropriate numbering of pages
(as well as other status parameters)
is determined from the |.aux| files.
The latter contain information from previous passes.
However this information needs to propagate through
all intermediate child documents.
Therefore the page numbering in child documents may well
be inconsistent until the complete document is compiled at least once.

A useful (if unconventional) way to always ensure a consistent
page numbering is to restart the numbering in each child document
and denote the pages by `\textit{child}|.|\textit{page}'
where \textit{child} represents the chapter/section number of the child file.
This can be achieved by the command
|\numberwithin{page}{|\textit{child}|}|
of the \textsf{amsmath} package
where \textit{child} can be |chapter| or |section|
depending on the chosen structuring.
Alternatively, one can modify the macro |\thepage| appropriately
and reset the counter |page| at the start of each child file.

%%%%%%%%%%%%%%%%%%%%%%%%%%%%%%%%%%%%%%%%%%%%%%%%%%%%%%%%%%%%%%%%%%%%%%%%%%%%%%%%
\subsection{Conditional Processing}
\label{sec:conditional}

The package provides a mechanism to compile different versions
of a document. To customise the versions further some conditional processing
can come in handy to distinguish which version is being compiled.
The package provides two macros to describe the compilation context:

%%%%%%%%%%%%%%%%%%%%%%%%%%%%%%%%%%%%%%%%
\DescribeMacro{\ifchilddoc}
The conditional |\ifchilddoc| distinguishes between the compilation of
child documents and the main document:
%
\begin{center}
|\ifchilddoc |\textit{child-code}| |[|\||else |\textit{main-code}]| \||fi|
\end{center}

%%%%%%%%%%%%%%%%%%%%%%%%%%%%%%%%%%%%%%%%
\DescribeMacro{\childdocname}
\DescribeMacro{\childdocjob}
The macro |\childdocname| contains the filename (without extension)
of the main or child file being processed.
Note that |\childdocjob| will always contain the name of the main file.

%%%%%%%%%%%%%%%%%%%%%%%%%%%%%%%%%%%%%%%%
\paragraph{Title Page.}

Conditional processing can be used to include a title or banner page
in the main document when proper precautions are taken.
Importantly, the code in the main file should ensure that the page counter
(as well as other status parameters which are stored in the |.aux| files)
takes the same value after the conditional processing.
Otherwise the page numbers may take divergent values
depending on which part is compiled.

For example, a title page could be declared by:
%
\begin{center}
\begin{tabular}{l}
|\ifchilddoc\||else|\\
|\addtocounter{page}{-1}|\\
\textit{code for title page}\\
|\newpage|\\
|\||fi|
\end{tabular}
\end{center}
%
A banner page for the child documents can be generated by:
%
\begin{center}
\begin{tabular}{l}
|\ifchilddoc|\\
|\addtocounter{page}{-1}|\\
\textit{code for banner page}\\
|\newpage|\\
|\||fi|
\end{tabular}
\end{center}
%
Here one could write a message such as:
\begin{center}
|This is the part \childdocname{} of \childdocjob{}.|
\end{center}

%%%%%%%%%%%%%%%%%%%%%%%%%%%%%%%%%%%%%%%%%%%%%%%%%%%%%%%%%%%%%%%%%%%%%%%%%%%%%%%%
\subsection{Flags}
\label{sec:flags}

The package makes it easy to generate different versions
of the main or child documents.
To this end compilation flags can be defined
and assigned different default values.
They will be particularly useful in conjunction
with the forwarding mechanism described in \secref{sec:forward}.

For example, it may be useful to have a flag |\version|
which can be set to |draft| or |final|.
The document source will contain some conditional code
depending on the value of |\version|.
Suppose further, the flag should default to |final| for the main file
and to |draft| for child files
which is a natural assignment for editing the document.
This is achieved by placing the following code
in the preamble of the main document
(below the |\childdocmain| directive):
%
\begin{center}
\begin{tabular}{l}
|\ifchilddoc|\\
|\providecommand{\version}{draft}|\\
|\||else|\\
|\providecommand{\version}{final}|\\
|\||fi|
\end{tabular}
\end{center}
%
The definition by |\providecommand| makes sure
that previous definitions are not overwritten.
Further statements |\providecommand{\version}{...}|
can thus be added before the above code to override it.

For the main file, one might add a line
(between |\childdocmain| and the above block)
%
\begin{center}
|%\ifchilddoc\||else\providecommand{\version}{draft}\||fi|
\end{center}
%
which can be uncommented to produce a draft version.
Likewise one can add a line to the very top of a child file
(above the |\childdocof{|\textit{main}|}| directive)
%
\begin{center}
|%\providecommand{\version}{final}|
\end{center}
%
which can be uncommented to produce the final version of this child document.

%%%%%%%%%%%%%%%%%%%%%%%%%%%%%%%%%%%%%%%%%%%%%%%%%%%%%%%%%%%%%%%%%%%%%%%%%%%%%%%%
\subsection{Forwarding}
\label{sec:forward}

Different versions of the main or child documents
using compilation flags as described in \secref{sec:flags}
can be (permanently) stored in different files
for convenient compilation, viewing and distribution.
To this end, the package defines a command
to pass on compilation to a different file:

%%%%%%%%%%%%%%%%%%%%%%%%%%%%%%%%%%%%%%%%
\DescribeMacro{\childdocforward}
The command |\childdocforward| redirects processing to
another source file:
%
\begin{center}
\begin{tabular}{l}
|\input{childdoc.def}|\\
|\childdocforward[|\textit{main}|]{|\textit{dest}|}|\\
\end{tabular}
\end{center}
%
The argument \textit{dest} is the destination file
(without extension).
It should be the main file or one of the child files.
Note that further \textsf{childdoc} directives
such as |\childdocof| and |\childdocforward|
in the indicated file will be processed in this form.
The optional argument \textit{main}
passes on directly to the main file \textit{main}
while pretending to compile the child \textit{dest}.
This form behaves as if \textit{dest}
issues |\childdocof{|\textit{main}|}| right away,
and no further \textsf{childdoc} directives will be processed.

%%%%%%%%%%%%%%%%%%%%%%%%%%%%%%%%%%%%%%%%
\DescribeMacro{\...prefix}
In the alternative form |\childdocforwardprefix|,
%
\begin{center}
\begin{tabular}{l}
|\input{childdoc.def}|\\
|\childdocforwardprefix[|\textit{main}|]{|\textit{prefix}|}{|\textit{dest}|}|
\end{tabular}
\end{center}
%
the destination file is determined by a pattern
depending on the current file:
To make this work, the current file must be called
`{\textit{prefix}\hspace{0.2em}\textit{suffix}}'
with \textit{prefix} matching precisely the argument.
Processing is then passed on to the file
`{\textit{dest}\hspace{0.2em}\textit{suffix}}'.
Surely, the same effect is achieved by
directly specifying the
argument `{\textit{dest}\hspace{0.2em}\textit{suffix}}'
in the first form.
However, that requires to set up a different file
for each child. With the alternative form of the command
all these files can have exactly the same content
which simplifies setting them up and maintaining them.

For example, the following file |draft.tex|
with a compilation flag |\version| as described in \secref{sec:flags}
compiles the main document as a draft:
%
\begin{center}
\begin{tabular}{l}
|\def\version{draft}|\\
|\input{childdoc.def}|\\
|\childdocforward{|\textit{main}|}|
\end{tabular}
\end{center}
%
Likewise, the following files |final|\textit{nn}|.tex|
compile the final version of the child document
|child|\textit{nn}|.tex|:
%
\begin{center}
\begin{tabular}{l}
|\def\version{final}|\\
|\input{childdoc.def}|\\
|\childdocforwardprefix{final}{child}|
\end{tabular}
\end{center}
%

Note that when several versions of a main file and/or of each child file
are to be generated, it may be convenient to set up a |Makefile| or
shell script to automatise the process.

%%%%%%%%%%%%%%%%%%%%%%%%%%%%%%%%%%%%%%%%%%%%%%%%%%%%%%%%%%%%%%%%%%%%%%%%%%%%%%%%
\subsection{Command Line Processing}
\label{sec:commandline}

The effect of redirection files can also be achieved by invoking
the \LaTeX{} compiler with a more elaborate command line.
Most conveniently this should be done as part
of a shell script or a |Makefile|.

When using \textsf{childdoc} in the main file, the following
command lines effectively perform a redirection
(note that depending on the shell being used,
backslashes may have to be doubled: `|\|' $\to$ `|\\|'):
%
\begin{center}
|... -jobname "|\textit{target}|" |\\|"|[\textit{flags}]%
|\input{childdoc.def}\childdocforward[|\textit{main}|]{|\textit{dest}|}"|
\end{center}
%
Here \textit{target} is the name of the output file,
\textit{main} is the name of the main file
and \textit{dest} is the name of the main or child file to be processed
(all filenames without extensions).
The optional argument \textit{main} can be omitted
if \textit{main} matches \textit{dest}.
Optionally, compilation \textit{flags} can be defined via |\def| commands.
This command line makes the \TeX{} engine believe
it is compiling the file \textit{target}
whose content is specified as the latter parameter.
The provided code then forwards the processing to
\textit{main} or \textit{dest} as described in \secref{sec:forward}.

%%%%%%%%%%%%%%%%%%%%%%%%%%%%%%%%%%%%%%%%%%%%%%%%%%%%%%%%%%%%%%%%%%%%%%%%%%%%%%%%
\subsection{Include by Input}
\label{sec:input}

Including child documents by |\include| has some restrictions by design.
Most notably, the content of a child document always occupies
its own set of pages; pages cannot be shared between child documents.
Usually, this behaviour makes perfect sense
because each child document contain an essential part of the document.
However, in some situations it may be desirable to compose
a document from a collection of parts
without having mandatory page breaks between then.
For this case, the package
provides a mechanism to include parts
by |\input| which can also be processed individually.
However, by construction this mechanism
requires manual handling of the content to be output.

%%%%%%%%%%%%%%%%%%%%%%%%%%%%%%%%%%%%%%%%
\DescribeMacro{\ifchilddocmanual}
The main file should be prepared as usual, see \secref{sec:include}.
However, the document body must make a distinction
between processing of an individual part and of the main document, e.g.:
%
\begin{center}
\begin{tabular}{l}
|\ifchilddocmanual|\\
|\input{\childdocname}|\\
|\||else|\\
\textit{document body with }|\input{|\textit{part}|}|\\
|\||fi|
\end{tabular}
\end{center}
%
The conditional |\ifchilddocmanual| is true whenever
a part to be included by |\input| is being compiled,
and the name of the part is stored in |\childdocname|.

%%%%%%%%%%%%%%%%%%%%%%%%%%%%%%%%%%%%%%%%
\DescribeMacro{\childdocby}
Each part to be included by |\input| should start with:
%
\begin{center}
\begin{tabular}{l}
|\input{childdoc.def}|\\
|\childdocby{|\textit{main}|}|\\
\end{tabular}
\end{center}
%
The directive |\childdocby| is similar to |\childdocof|
described in \secref{sec:include},
but the subsequent selection of content must be done manually.
To that end, both |\ifchilddoc| and |\ifchilddocmanual|
will be true upon processing of a part,
and the name of the part is stored in |\childdocname|.
Note that |\jobname| will be set to the filename of the current part
so that each part receives an individual |.aux| file
that does not interfere with the |.aux| file(s) of the main document.
This behaviour can be altered by the alternative form
|\childdocby[*]{|\textit{main}|}| (with a non-empty optional argument)
which uses the |.aux| file of the main document
by setting |\jobname| to \textit{main}.

%%%%%%%%%%%%%%%%%%%%%%%%%%%%%%%%%%%%%%%%%%%%%%%%%%%%%%%%%%%%%%%%%%%%%%%%%%%%%%%%
\subsection{Driver Development}
\label{sec:driver}

The \textsf{childdoc} mechanism can also be use for the development
of definition files such as \LaTeX{} styles or classes.
This case differs from the above setup with multiple parts
included by |\include| in that no |\includeonly| should be invoked.
This can be achieved by starting the include file
(before |\ProvidesPackage|) with:
%
\begin{center}
\begin{tabular}{l}
|\input{childdoc.def}|\\
|\childdocforward{|\textit{main}|}|\\
\end{tabular}
\end{center}
%
or alternatively with:
%
\begin{center}
\begin{tabular}{l}
|\input{childdoc.def}|\\
|\childdocby{|\textit{main}|}|\\
\end{tabular}
\end{center}
%
Both forms have slightly different effects as described above.
The main file is prepared as usual, see \secref{sec:include}.

%%%%%%%%%%%%%%%%%%%%%%%%%%%%%%%%%%%%%%%%%%%%%%%%%%%%%%%%%%%%%%%%%%%%%%%%%%%%%%%%
\subsection{Legacy Detection}
\label{sec:detection}

The directive |\childdocmain| in the main file can detect
whether the complete document or merely a child is to be compiled
even without using the directive |\childdocof|.
This method is deprecated because it is less robust
and there is no compelling reason to use it;
it is merely provided for backward compatibility
and it may be removed in future versions.

If the detection mechanism is to be used,
it is mandatory to correctly specify
the filename of the main file as the argument of |\childdocmain|:
%
\begin{center}
\begin{tabular}{l}
|\input{childdoc.def}|\\
|\childdocmain{|\textit{main}|}|\\
\end{tabular}
\end{center}
%
If |\jobname| does not match the argument \textit{main} of |\childdocmain|,
it is assumed that |\jobname| points to the child file to be compiled.
When using |\childdocmain| with the main file specified as argument,
it suffices to start a child file
with just |\input{|\textit{main}|}|
without loading of the package and using |\childdocof|.
If instead all processing is done
with the appropriate \textsf{childdoc} directives,
the argument of \textit{main} of |\childdocmain| can be empty.

An alternative version of the command line processing described
in \secref{sec:commandline} using the detection mechanism reads:
%
\begin{center}
|... -jobname "|\textit{target}|" "|[\textit{flags}]%
[|\def\jobname{|\textit{dest}|}|]|\input{|\textit{main}|}"|
\end{center}

%%%%%%%%%%%%%%%%%%%%%%%%%%%%%%%%%%%%%%%%%%%%%%%%%%%%%%%%%%%%%%%%%%%%%%%%%%%%%%%%
\subsection{Manual Code}
\label{sec:manual}

In case one cannot be certain whether the definitions file |childdoc.def|
is installed on the target \TeX{} distribution
and one prefers not to ship it,
it is conceivable to paste a few relevant commands into the sources.

To that end, drop all statements |\input{childdoc.def}|
and perform the replacements as outlined below.
Instead of |\childdocmain{|\textit{main}|}| add the following code
to the top of the main file:
%
\begin{center}
\begin{tabular}{l}
|\||ifdefined\childdocname\endinput\||fi\newif\ifchilddoc|\\
|\edef\childdocname{\scantokens\expandafter{\jobname\noexpand}}|\\
|\def\childdocmain{|\textit{main}|}\||ifx\childdocmain\childdocname\||else|\\
|\childdoctrue\includeonly{\childdocname}\let\jobname\childdocmain\||fi|\\
\end{tabular}
\end{center}
%
Instead of |\childdocof{|\textit{main}|}| just include the main file
at the top of each child file:
%
\begin{center}
|\input{|\textit{main}|}|
\end{center}
%
A simple redirection |\childdocforward{|\textit{dest}|}| is achieved by:
%
\begin{center}
|\def\jobname{|\textit{dest}|}\input{\jobname}|
\end{center}
%
The redirection with prefix
|\childdocforwardprefix[|\textit{prefix}|]{|\textit{dest}|}|
is accomplished by:
%
\begin{center}
\begin{tabular}{l}
|{\edef\jobname{\scantokens\expandafter{\jobname\noexpand}}|\\
|\def\redirectjob |\textit{prefix}|#1~~~{\gdef\jobname{|\textit{dest}|#1}}|\\
|\expandafter\redirectjob\jobname~~~}\input{\jobname}|
\end{tabular}
\end{center}

In an alternative approach,
child documents can be compiled by a specific command line
without additional code or specific definitions:
%
\begin{center}
|... -jobname "|\textit{target}|" "|[\textit{flags}]%
|\includeonly{|\textit{dest}|}\input{|\textit{main}|}"|
\end{center}
%

%%%%%%%%%%%%%%%%%%%%%%%%%%%%%%%%%%%%%%%%%%%%%%%%%%%%%%%%%%%%%%%%%%%%%%%%%%%%%%%%
%%%%%%%%%%%%%%%%%%%%%%%%%%%%%%%%%%%%%%%%%%%%%%%%%%%%%%%%%%%%%%%%%%%%%%%%%%%%%%%%
\section{Information}

%%%%%%%%%%%%%%%%%%%%%%%%%%%%%%%%%%%%%%%%%%%%%%%%%%%%%%%%%%%%%%%%%%%%%%%%%%%%%%%%
\subsection{Copyright}

Copyright \copyright{} 2017--2018 Niklas Beisert

This work may be distributed and/or modified under the
conditions of the \LaTeX{} Project Public License, either version 1.3
of this license or (at your option) any later version.
The latest version of this license is in
  \url{http://www.latex-project.org/lppl.txt}
and version 1.3 or later is part of all distributions of \LaTeX{}
version 2005/12/01 or later.

This work has the LPPL maintenance status `maintained'.

The Current Maintainer of this work is Niklas Beisert.

This work consists of the files |README.txt|, |childdoc.ins| and |childdoc.dtx|
as well as the derived files |childdoc.def|, |cdocsamp.tex|
with |cdocsch1.tex|, |cdocsch2.tex|, |cdocspt3.tex|, |cdocspt4.tex|,
|cdocsdrf.tex|, |cdocsfn1.tex|, |cdocsfn2.tex|
as well as |childdoc.pdf|.

%%%%%%%%%%%%%%%%%%%%%%%%%%%%%%%%%%%%%%%%%%%%%%%%%%%%%%%%%%%%%%%%%%%%%%%%%%%%%%%%
\subsection{Files and Installation}

The package consists of the files:
%
\begin{center}
\begin{tabular}{ll}
    |README.txt|   & readme file \\
    |childdoc.ins| & installation file \\
    |childdoc.dtx| & source file \\
    |childdoc.def| & definition file \\
    |cdocsamp.tex| & sample main file \\
    |cdocsch1.tex| & sample include file \\
    |cdocsch2.tex| & sample include file \\
    |cdocspt3.tex| & sample part file \\
    |cdocspt4.tex| & sample part file \\
    |cdocsdrf.tex| & sample redirection file \\
    |cdocsfn1.tex| & sample redirection file \\
    |cdocsfn2.tex| & sample redirection file \\
    |childdoc.pdf| & manual
\end{tabular}
\end{center}
%
The distribution consists of the files
|README.txt|, |childdoc.ins| and |childdoc.dtx|.
%
\begin{itemize}
\item
Run (pdf)\LaTeX{} on |childdoc.dtx|
to compile the manual |childdoc.pdf| (this file).
\item
Run \LaTeX{} on |childdoc.ins| to create the definitions file |childdoc.def|
and the sample |cdocsamp.tex| with include files
|cdocsch1.tex|, |cdocsch2.tex|, |cdocspt3.tex|, |cdocspt4.tex|,
|cdocsdrf.tex|, |cdocsfn1.tex|, |cdocsfn2.tex|.
Then copy the file |childdoc.def| to an appropriate directory of your \LaTeX{}
distribution, e.g.\ \textit{texmf-root}|/tex/latex/childdoc|.
\end{itemize}

%%%%%%%%%%%%%%%%%%%%%%%%%%%%%%%%%%%%%%%%%%%%%%%%%%%%%%%%%%%%%%%%%%%%%%%%%%%%%%%%
\subsection{Related CTAN Packages}

There are several other packages which offer a similar functionality:
%
\begin{itemize}
\item
The packages
\href{http://ctan.org/pkg/docmute}{\textsf{docmute}},
\href{http://ctan.org/pkg/includex}{\textsf{includex}} and
\href{http://ctan.org/pkg/standalone}{\textsf{standalone}}
provide commands to include only the document body of
a child file thus allowing both files to be compiled individually.
\item
The packages \href{http://ctan.org/pkg/subdocs}{\textsf{subdocs}}
and \href{http://ctan.org/pkg/subfiles}{\textsf{subfiles}}
provide structures in which the main and child documents can be
encapsulated and allowing them to be compiled individually.
The inclusion mechanism is different from the conventional |\include|.
\item
The package \href{http://ctan.org/pkg/combine}{\textsf{combine}}
is an elaborate solution to combine several documents into one.
\end{itemize}
%
See also the CTAN topic \href{http://ctan.org/topic/subdocs}{\textsf{subdocs}}
for further related packages.
The present package differs from the above solutions in that
a document structure constructed with the conventional |\include| mechanism
just needs two extra commands at the top of every file
such that all constituent files can be compiled individually.

%%%%%%%%%%%%%%%%%%%%%%%%%%%%%%%%%%%%%%%%%%%%%%%%%%%%%%%%%%%%%%%%%%%%%%%%%%%%%%%%
%\subsection{Feature Suggestions}
%
%The following is a list of features which may be useful for future
%versions of this package:
%%
%\begin{itemize}
%\item
%\ldots
%\end{itemize}

%%%%%%%%%%%%%%%%%%%%%%%%%%%%%%%%%%%%%%%%%%%%%%%%%%%%%%%%%%%%%%%%%%%%%%%%%%%%%%%%
\subsection{Revision History}

%%%%%%%%%%%%%%%%%%%%%%%%%%%%%%%%%%%%%%%%
\paragraph{v2.0:} 2018/12/30

\begin{itemize}
\item
immediate forward processing
\item
added |\childdocby| mechanism
\item
manual restructured
\end{itemize}

%%%%%%%%%%%%%%%%%%%%%%%%%%%%%%%%%%%%%%%%
\paragraph{v1.6:} 2018/01/17

\begin{itemize}
\item
application for development of include files
\item
corrections to manual
\end{itemize}

%%%%%%%%%%%%%%%%%%%%%%%%%%%%%%%%%%%%%%%%
\paragraph{v1.5:} 2017/05/21

\begin{itemize}
\item
more complete structuring introduced
\item
|\childdocof| introduced
\item
|\childdoc| renamed to |\childdocmain|
\item
|\childredirect| renamed to |\childdocforward| and |\childdocforwardprefix|
and functionality expanded
\end{itemize}

%%%%%%%%%%%%%%%%%%%%%%%%%%%%%%%%%%%%%%%%
\paragraph{v1.0:} 2017/04/27

\begin{itemize}
\item
manual and install package
\item
first version published on CTAN
\end{itemize}

%%%%%%%%%%%%%%%%%%%%%%%%%%%%%%%%%%%%%%%%
\paragraph{v0.6:} 2017/04/26

\begin{itemize}
\item
redirection mechanism added
\end{itemize}

%%%%%%%%%%%%%%%%%%%%%%%%%%%%%%%%%%%%%%%%
\paragraph{v0.5:} 2017/04/26

\begin{itemize}
\item
functionality in definition file
\end{itemize}


%%%%%%%%%%%%%%%%%%%%%%%%%%%%%%%%%%%%%%%%%%%%%%%%%%%%%%%%%%%%%%%%%%%%%%%%%%%%%%%%
%%%%%%%%%%%%%%%%%%%%%%%%%%%%%%%%%%%%%%%%%%%%%%%%%%%%%%%%%%%%%%%%%%%%%%%%%%%%%%%%
%%%%%%%%%%%%%%%%%%%%%%%%%%%%%%%%%%%%%%%%%%%%%%%%%%%%%%%%%%%%%%%%%%%%%%%%%%%%%%%%
\appendix

\settowidth\MacroIndent{\rmfamily\scriptsize 000\ }

 \DocInput{childdoc.dtx}

\end{document}
%</driver>
% \fi
%
% %%%%%%%%%%%%%%%%%%%%%%%%%%%%%%%%%%%%%%%%%%%%%%%%%%%%%%%%%%%%%%%%%%%%%%%%%%%%%%
% %%%%%%%%%%%%%%%%%%%%%%%%%%%%%%%%%%%%%%%%%%%%%%%%%%%%%%%%%%%%%%%%%%%%%%%%%%%%%%
% \section{Sample}
%\iffalse
%<*samplemain>
%\fi
%
% The following presents a sample document
% with two chapters, two parts, a title page,
% a compile flag as well as three forwarding files to set the flag.
% It consists of eight |.tex| files:
% \begin{center}
% \begin{tabular}{ll}
% |cdocsamp.tex|&main file\\
% |cdocsch1.tex|&include file for chapter 1\\
% |cdocsch2.tex|&include file for chapter 2\\
% |cdocspt3.tex|&include file for part 3\\
% |cdocspt4.tex|&include file for part 4\\
% |cdocsdrf.tex|&forwarding file for main file in draft mode\\
% |cdocsfi1.tex|&forwarding file for final version of chapter 1\\
% |cdocsfi2.tex|&forwarding file for final version of chapter 2\\
% \end{tabular}
% \end{center}
% Each of the eight files can be compiled directly by the \LaTeX{} compiler.
%
% %%%%%%%%%%%%%%%%%%%%%%%%%%%%%%%%%%%%%%
% \paragraph{Main File.}
%
% The main file is called |cdocsamp.tex|.
%
% Load the \textsf{childdoc} definitions and
% declare the filename for the main document:
%    \begin{macrocode}
\input{childdoc.def}
\childdocmain{}
%    \end{macrocode}

% Optional override for |\version| flag:
%    \begin{macrocode}
%%\ifchilddoc\else\providecommand{\version}{draft}\fi
%    \end{macrocode}

% Define the default values for the |\version| flag
% (|final| for the main file and |draft| for childs):
%    \begin{macrocode}
\ifchilddoc
\providecommand{\version}{draft}
\else
\providecommand{\version}{final}
\fi
%    \end{macrocode}

% Load the standard document class:
%    \begin{macrocode}
\documentclass[12pt]{article}
%    \end{macrocode}

% Start the document body:
%    \begin{macrocode}
\begin{document}
%    \end{macrocode}

% Declare a title page.
% Print title, part of document being processed and version flag:
%    \begin{macrocode}
\addtocounter{page}{-1}
\begin{center}
{\LARGE\bfseries{}childdoc example\par}
\vspace{1cm}
\ifchilddoc
\ifchilddocmanual part\else chapter\fi:
`\childdocname' of `\childdocjob'\par
\else
main document: `\childdocjob'\par
\fi
version: \version\par
\end{center}
\newpage
%    \end{macrocode}

% Manually include selected file,
% otherwise process as usual:
%    \begin{macrocode}
\ifchilddocmanual
\section*{part `\childdocname'}
\input{\childdocname}
\else
%    \end{macrocode}

% Include the two chapters:
%    \begin{macrocode}
\include{cdocsch1}
\include{cdocsch2}
%    \end{macrocode}

% Include the two parts unless only chapters should be displayed:
%    \begin{macrocode}
\ifchilddoc\else
\section{part three}
\input{cdocspt3}
\section{part four}
\input{cdocspt4}
\fi
%    \end{macrocode}

% Process as usual until here:
%    \begin{macrocode}
\fi
%    \end{macrocode}

% End of document body:
%    \begin{macrocode}
\end{document}
%    \end{macrocode}
%\iffalse
%</samplemain>
%\fi
%
% %%%%%%%%%%%%%%%%%%%%%%%%%%%%%%%%%%%%%%
% \paragraph{Chapter Include Files.}
%
% The include files are called |cdocsch1.tex| and |cdocsch2.tex|.
%
%\iffalse
%<*samplechap1|samplechap2>
%\fi

% Optional override for |\version| flag:
%    \begin{macrocode}
%%\providecommand{\version}{final}
%    \end{macrocode}

% Include the main document:
%    \begin{macrocode}
\input{childdoc.def}
\childdocof{cdocsamp}
%    \end{macrocode}

%\iffalse
%</samplechap1|samplechap2>
%\fi
%
%\iffalse
%<*samplechap1>
%\fi
% Some text for chapter 1:
%    \begin{macrocode}
\section{one}
some text in chapter one
%    \end{macrocode}

%\iffalse
%</samplechap1>
%\fi
% Some text for chapter 2:
%\iffalse
%<*samplechap2>
%\fi
%    \begin{macrocode}
\section{two}
more text in chapter two
%    \end{macrocode}

%\iffalse
%</samplechap2>
%\fi
%
% %%%%%%%%%%%%%%%%%%%%%%%%%%%%%%%%%%%%%%
% \paragraph{Part Include Files.}
%
% The include files are called |cdocspt3.tex| and |cdocspt4.tex|.
%
%\iffalse
%<*samplepart3|samplepart4>
%\fi

% Optional override for |\version| flag:
%    \begin{macrocode}
%%\providecommand{\version}{final}
%    \end{macrocode}

% Include the main document:
%    \begin{macrocode}
\input{childdoc.def}
\childdocby{cdocsamp}
%    \end{macrocode}

%\iffalse
%</samplepart3|samplepart4>
%\fi
%
%\iffalse
%<*samplepart3>
%\fi
% Some text for part 3:
%    \begin{macrocode}
some text in part three
%    \end{macrocode}

%\iffalse
%</samplepart3>
%\fi
% Some text for part 4:
%\iffalse
%<*samplepart4>
%\fi
%    \begin{macrocode}
more text in part four
%    \end{macrocode}

%\iffalse
%</samplepart4>
%\fi
%
% %%%%%%%%%%%%%%%%%%%%%%%%%%%%%%%%%%%%%%
% \paragraph{Forwarding for a Complete Draft.}
%
% The following forwarding file |cdocsdrf.tex|
% compiles the main document in draft mode:
%\iffalse
%<*sampledraft>
%\fi
%    \begin{macrocode}
\def\version{draft}
\input{childdoc.def}
\childdocforward{cdocsamp}
%    \end{macrocode}

%\iffalse
%</sampledraft>
%\fi
%
% %%%%%%%%%%%%%%%%%%%%%%%%%%%%%%%%%%%%%%
% \paragraph{Forwarding for Final Version of the Chapters.}
%
% The following forwarding files |cdocsfn1.tex| and |cdocsfn2.tex|
% (with identical content)
% compile the final versions of the child documents
% |cdocsch1.tex| and |cdocsch2.tex|, respectively:
%\iffalse
%<*samplefinal>
%\fi
%    \begin{macrocode}
\def\version{final}
\input{childdoc.def}
\childdocforwardprefix[cdocsamp]{cdocsfn}{cdocsch}
%    \end{macrocode}

%\iffalse
%</samplefinal>
%\fi
%
% %%%%%%%%%%%%%%%%%%%%%%%%%%%%%%%%%%%%%%
% \paragraph{Command Line Processing.}
%
% The following three command lines generate the output files
% |cdocscld|, |cdocscl1| and |cdocscl2|
% which should be identical to
% |cdocsdrf|, |cdocsch1| and |cdocsfn2|, respectively:
% \begin{center}
% \begin{tabular}{l}
% |latex -jobname cdocscld \|\\
% |  "\def\version{draft}\input{childdoc.def}\childdocforward{cdocsamp}"|\\
% |latex -jobname cdocscl1 \|\\
% |  "\input{childdoc.def}\childdocforward[cdocsamp]{cdocsch1}"|\\
% |latex -jobname cdocscl2 \|\\
% |  "\def\version{final}\input{childdoc.def}\childdocforward{cdocsch2}"|
% \end{tabular}
% \end{center}
% Note that the trailing backslash on each first line
% merely continues the input to the second line
% (for convenient cut ant paste).
% Furthermore, the command |latex| can be replaced by any
% of its alternative versions such as |pdflatex|.
%
% %%%%%%%%%%%%%%%%%%%%%%%%%%%%%%%%%%%%%%%%%%%%%%%%%%%%%%%%%%%%%%%%%%%%%%%%%%%%%%
% %%%%%%%%%%%%%%%%%%%%%%%%%%%%%%%%%%%%%%%%%%%%%%%%%%%%%%%%%%%%%%%%%%%%%%%%%%%%%%
% \section{Implementation}
%\iffalse
%<*package>
%\fi
%
% This section describes the definitions file |childdoc.def|.

% The definitions cannot be loaded using |\usepackage| or |\RequirePackage|
% which has a mechanism to prevent loading a style file more than once.
% When loading the definitions by means of |\input|
% multiple instances have to be prevented manually:
%\iffalse
%This code needs to be before the `\ProvidesFile' directive
%which is defined at the beginning of this file.
%Therefore it is also placed there and commented out here.
%</package>
%<*discard>
%\fi
%    \begin{macrocode}
\ifdefined\childdocmain\endinput\fi
%    \end{macrocode}
%\iffalse
%</discard>
%<*package>
%\fi
%
% \macro{\ifchilddoc}
% \macro{\ifchilddocmanual}
% The conditional |\ifchilddoc| tells whether a
% child (true) or main (false) document is being compiled.
% The conditional |\ifchilddocmanual| tells whether
% the |\includeonly| mechanism is used (false) or
% the selection of child files must be performed manually (true).
% The definitions initialise to false:
%    \begin{macrocode}
\newif\ifchilddoc
\newif\ifchilddocmanual
%    \end{macrocode}

% \macro{\childdocname}
% \macro{\childdocjob}
% The macro |\childdocname| stores the name of the main document
% to be compiled. The macro |\childdocjob| stores the name of
% the document on which the \LaTeX{} compiler was originally invoked.
% The content of |\jobname| cannot be compared
% to filenames specified in the source due to different catcodes.
% The following code rescans |\jobname|, stores the result
% in |\childdocname| and saves a copy in |\childdocjob|:
%    \begin{macrocode}
\edef\childdocname{\scantokens\expandafter{\jobname\noexpand}}
\let\childdocjob\childdocname
%    \end{macrocode}

% \macro{\childdocdisable}
% The macro |\childdocdisable| prevents the main file
% from being processed more than once.
% At this stage, the main document command |\childdocmain|
% is assumed to be called once again where it should do nothing.
% Any subsequent call to it should prevent
% a secondary processing of the main document
% It overwrites the forwarding commands
% |\childdocof| and |\childdocforward|
% with empty macros to prevent further inclusions of the main document:
%    \begin{macrocode}
\newcommand{\childdocdisable}
{
  \renewcommand{\childdocmain}[1]{\renewcommand{\childdocmain}[1]{\endinput}}
  \renewcommand{\childdocof}[1]{}
  \renewcommand{\childdocby}[2][]{}
  \renewcommand{\childdocforward}[2][]{}
  \renewcommand{\childdocdisable}{}
}
%    \end{macrocode}

% \macro{\childdocmain}
% The macro |\childdocmain| is to be called at the top of the main file
% with nothing or the main filename (without extension) as argument.
% First, it breaks loops.
% If the argument is not empty and does not match |\childdocname|
% (which is set by the first inclusion of |childdoc.def|),
% |\ifchilddoc| is set to true, |\includeonly| is applied to the child file
% and |\jobname| is set to the main file
% (for proper handling of |.aux| files):
%    \begin{macrocode}
\newcommand{\childdocmain}[1]
{
  \childdocdisable\childdocmain{}
  \if?#1?\else
    \begingroup
      \def\childdoctmp{#1}
      \ifx\childdoctmp\childdocname
        \def\childdoctmp{}
      \else
        \def\childdoctmp
        {
          \childdoctrue
          \includeonly{\childdocname}
          \def\childdocjob{#1}
          \def\jobname{#1}
        }
      \fi
      \expandafter
    \endgroup
    \childdoctmp
  \fi
}
%    \end{macrocode}

% \macro{\childdocof}
% The command |\childdocof| redirects
% compilation to the main file |#1|.
%    \begin{macrocode}
\newcommand{\childdocof}[1]
{
  \childdocdisable
  \childdoctrue
  \includeonly{\childdocname}
  \def\jobname{#1}
  \def\childdocjob{#1}
  \input{#1}
}
%    \end{macrocode}

% \macro{\childdocby}
% The command |\childdocby| ....
%    \begin{macrocode}
\newcommand{\childdocby}[2][]
{
  \childdocdisable
  \childdoctrue
  \childdocmanualtrue
  \if?#1?\else
    \def\jobname{#2}
  \fi
  \def\childdocjob{#2}
  \input{#2}
  \endinput
}
%    \end{macrocode}

% \macro{\childdocforward}
% The command |\childdocforward| redirects
% compilation to the main file or
% (if the optional argument is given) a child file.
% Parameters are set as if the main file
% or a child file starting with |\childdocof| was compiled.
% Then compilation is handed over to the main file:
%    \begin{macrocode}
\newcommand{\childdocforward}[2][]
{
  \begingroup
    \if?#1?
      \def\childdoctmp
      {
        \def\childdocname{#2}
        \def\childdocjob{#2}
        \def\jobname{#2}
        \input{#2}
        \endinput
      }
    \else
      \def\childdoctmp
      {
        \childdocdisable
        \def\childdocname{#2}
        \childdoctrue
        \includeonly{#2}
        \def\childdocjob{#1}
        \def\jobname{#1}
        \input{#1}
        \endinput
      }
    \fi
    \expandafter
  \endgroup
  \childdoctmp
}
%    \end{macrocode}

% \macro{\childdocforwardprefix}
% The command |\childdocforwardprefix| redirects
% compilation to the main or a child file by means of a pattern.
% The prefix |#1| in the current filename is replaced by |#2|
% and the suffix of the current filename is kept
% (it is assumed that the filename does not contain the substring `|~~~|'
% which is used as a delimiter).
% Compilation is handed over to the new file by |\childdocforward|:
%    \begin{macrocode}
\newcommand{\childdocforwardprefix}[3][]
{
  \begingroup
    \def\childdocextract #2##1~~~{\def\childdoctmp{\childdocforward[#1]{#3##1}}}
    \expandafter\childdocextract\childdocname~~~
    \expandafter
  \endgroup
  \childdoctmp
}
%    \end{macrocode}

% \macro{\childdoc}
% The deprecated macro |\childdoc| is a legacy version of |\childdocmain|:
%    \begin{macrocode}
\newcommand{\childdoc}{\childdocmain}
%    \end{macrocode}

% \macro{\childdocredirect}
% The deprecated macro |\childdocredirect| is a legacy version
% of |\childdocforward| and |\childdocforwardprefix|:
%    \begin{macrocode}
\newcommand{\childdocredirect}[2][]
{
  \begingroup
    \if?#1?
      \def\childdoctmp{\childdocforward{#2}}
    \else
      \def\childdoctmp{\childdocforwardprefix{#1}{#2}}
    \fi
    \expandafter
  \endgroup
  \childdoctmp
}
%    \end{macrocode}

%\iffalse
%</package>
%\fi
%
\endinput
\childdocforward{cdocsch2}"|
% \end{tabular}
% \end{center}
% Note that the trailing backslash on each first line
% merely continues the input to the second line
% (for convenient cut ant paste).
% Furthermore, the command |latex| can be replaced by any
% of its alternative versions such as |pdflatex|.
%
% %%%%%%%%%%%%%%%%%%%%%%%%%%%%%%%%%%%%%%%%%%%%%%%%%%%%%%%%%%%%%%%%%%%%%%%%%%%%%%
% %%%%%%%%%%%%%%%%%%%%%%%%%%%%%%%%%%%%%%%%%%%%%%%%%%%%%%%%%%%%%%%%%%%%%%%%%%%%%%
% \section{Implementation}
%\iffalse
%<*package>
%\fi
%
% This section describes the definitions file |childdoc.def|.

% The definitions cannot be loaded using |\usepackage| or |\RequirePackage|
% which has a mechanism to prevent loading a style file more than once.
% When loading the definitions by means of |\input|
% multiple instances have to be prevented manually:
%\iffalse
%This code needs to be before the `\ProvidesFile' directive
%which is defined at the beginning of this file.
%Therefore it is also placed there and commented out here.
%</package>
%<*discard>
%\fi
%    \begin{macrocode}
\ifdefined\childdocmain\endinput\fi
%    \end{macrocode}
%\iffalse
%</discard>
%<*package>
%\fi
%
% \macro{\ifchilddoc}
% \macro{\ifchilddocmanual}
% The conditional |\ifchilddoc| tells whether a
% child (true) or main (false) document is being compiled.
% The conditional |\ifchilddocmanual| tells whether
% the |\includeonly| mechanism is used (false) or
% the selection of child files must be performed manually (true).
% The definitions initialise to false:
%    \begin{macrocode}
\newif\ifchilddoc
\newif\ifchilddocmanual
%    \end{macrocode}

% \macro{\childdocname}
% \macro{\childdocjob}
% The macro |\childdocname| stores the name of the main document
% to be compiled. The macro |\childdocjob| stores the name of
% the document on which the \LaTeX{} compiler was originally invoked.
% The content of |\jobname| cannot be compared
% to filenames specified in the source due to different catcodes.
% The following code rescans |\jobname|, stores the result
% in |\childdocname| and saves a copy in |\childdocjob|:
%    \begin{macrocode}
\edef\childdocname{\scantokens\expandafter{\jobname\noexpand}}
\let\childdocjob\childdocname
%    \end{macrocode}

% \macro{\childdocdisable}
% The macro |\childdocdisable| prevents the main file
% from being processed more than once.
% At this stage, the main document command |\childdocmain|
% is assumed to be called once again where it should do nothing.
% Any subsequent call to it should prevent
% a secondary processing of the main document
% It overwrites the forwarding commands
% |\childdocof| and |\childdocforward|
% with empty macros to prevent further inclusions of the main document:
%    \begin{macrocode}
\newcommand{\childdocdisable}
{
  \renewcommand{\childdocmain}[1]{\renewcommand{\childdocmain}[1]{\endinput}}
  \renewcommand{\childdocof}[1]{}
  \renewcommand{\childdocby}[2][]{}
  \renewcommand{\childdocforward}[2][]{}
  \renewcommand{\childdocdisable}{}
}
%    \end{macrocode}

% \macro{\childdocmain}
% The macro |\childdocmain| is to be called at the top of the main file
% with nothing or the main filename (without extension) as argument.
% First, it breaks loops.
% If the argument is not empty and does not match |\childdocname|
% (which is set by the first inclusion of |childdoc.def|),
% |\ifchilddoc| is set to true, |\includeonly| is applied to the child file
% and |\jobname| is set to the main file
% (for proper handling of |.aux| files):
%    \begin{macrocode}
\newcommand{\childdocmain}[1]
{
  \childdocdisable\childdocmain{}
  \if?#1?\else
    \begingroup
      \def\childdoctmp{#1}
      \ifx\childdoctmp\childdocname
        \def\childdoctmp{}
      \else
        \def\childdoctmp
        {
          \childdoctrue
          \includeonly{\childdocname}
          \def\childdocjob{#1}
          \def\jobname{#1}
        }
      \fi
      \expandafter
    \endgroup
    \childdoctmp
  \fi
}
%    \end{macrocode}

% \macro{\childdocof}
% The command |\childdocof| redirects
% compilation to the main file |#1|.
%    \begin{macrocode}
\newcommand{\childdocof}[1]
{
  \childdocdisable
  \childdoctrue
  \includeonly{\childdocname}
  \def\jobname{#1}
  \def\childdocjob{#1}
  \input{#1}
}
%    \end{macrocode}

% \macro{\childdocby}
% The command |\childdocby| ....
%    \begin{macrocode}
\newcommand{\childdocby}[2][]
{
  \childdocdisable
  \childdoctrue
  \childdocmanualtrue
  \if?#1?\else
    \def\jobname{#2}
  \fi
  \def\childdocjob{#2}
  \input{#2}
  \endinput
}
%    \end{macrocode}

% \macro{\childdocforward}
% The command |\childdocforward| redirects
% compilation to the main file or
% (if the optional argument is given) a child file.
% Parameters are set as if the main file
% or a child file starting with |\childdocof| was compiled.
% Then compilation is handed over to the main file:
%    \begin{macrocode}
\newcommand{\childdocforward}[2][]
{
  \begingroup
    \if?#1?
      \def\childdoctmp
      {
        \def\childdocname{#2}
        \def\childdocjob{#2}
        \def\jobname{#2}
        \input{#2}
        \endinput
      }
    \else
      \def\childdoctmp
      {
        \childdocdisable
        \def\childdocname{#2}
        \childdoctrue
        \includeonly{#2}
        \def\childdocjob{#1}
        \def\jobname{#1}
        \input{#1}
        \endinput
      }
    \fi
    \expandafter
  \endgroup
  \childdoctmp
}
%    \end{macrocode}

% \macro{\childdocforwardprefix}
% The command |\childdocforwardprefix| redirects
% compilation to the main or a child file by means of a pattern.
% The prefix |#1| in the current filename is replaced by |#2|
% and the suffix of the current filename is kept
% (it is assumed that the filename does not contain the substring `|~~~|'
% which is used as a delimiter).
% Compilation is handed over to the new file by |\childdocforward|:
%    \begin{macrocode}
\newcommand{\childdocforwardprefix}[3][]
{
  \begingroup
    \def\childdocextract #2##1~~~{\def\childdoctmp{\childdocforward[#1]{#3##1}}}
    \expandafter\childdocextract\childdocname~~~
    \expandafter
  \endgroup
  \childdoctmp
}
%    \end{macrocode}

% \macro{\childdoc}
% The deprecated macro |\childdoc| is a legacy version of |\childdocmain|:
%    \begin{macrocode}
\newcommand{\childdoc}{\childdocmain}
%    \end{macrocode}

% \macro{\childdocredirect}
% The deprecated macro |\childdocredirect| is a legacy version
% of |\childdocforward| and |\childdocforwardprefix|:
%    \begin{macrocode}
\newcommand{\childdocredirect}[2][]
{
  \begingroup
    \if?#1?
      \def\childdoctmp{\childdocforward{#2}}
    \else
      \def\childdoctmp{\childdocforwardprefix{#1}{#2}}
    \fi
    \expandafter
  \endgroup
  \childdoctmp
}
%    \end{macrocode}

%\iffalse
%</package>
%\fi
%
\endinput
\childdocforward[|\textit{main}|]{|\textit{dest}|}"|
\end{center}
%
Here \textit{target} is the name of the output file,
\textit{main} is the name of the main file
and \textit{dest} is the name of the main or child file to be processed
(all filenames without extensions).
The optional argument \textit{main} can be omitted
if \textit{main} matches \textit{dest}.
Optionally, compilation \textit{flags} can be defined via |\def| commands.
This command line makes the \TeX{} engine believe
it is compiling the file \textit{target}
whose content is specified as the latter parameter.
The provided code then forwards the processing to
\textit{main} or \textit{dest} as described in \secref{sec:forward}.

%%%%%%%%%%%%%%%%%%%%%%%%%%%%%%%%%%%%%%%%%%%%%%%%%%%%%%%%%%%%%%%%%%%%%%%%%%%%%%%%
\subsection{Include by Input}
\label{sec:input}

Including child documents by |\include| has some restrictions by design.
Most notably, the content of a child document always occupies
its own set of pages; pages cannot be shared between child documents.
Usually, this behaviour makes perfect sense
because each child document contain an essential part of the document.
However, in some situations it may be desirable to compose
a document from a collection of parts
without having mandatory page breaks between then.
For this case, the package
provides a mechanism to include parts
by |\input| which can also be processed individually.
However, by construction this mechanism
requires manual handling of the content to be output.

%%%%%%%%%%%%%%%%%%%%%%%%%%%%%%%%%%%%%%%%
\DescribeMacro{\ifchilddocmanual}
The main file should be prepared as usual, see \secref{sec:include}.
However, the document body must make a distinction
between processing of an individual part and of the main document, e.g.:
%
\begin{center}
\begin{tabular}{l}
|\ifchilddocmanual|\\
|\input{\childdocname}|\\
|\||else|\\
\textit{document body with }|\input{|\textit{part}|}|\\
|\||fi|
\end{tabular}
\end{center}
%
The conditional |\ifchilddocmanual| is true whenever
a part to be included by |\input| is being compiled,
and the name of the part is stored in |\childdocname|.

%%%%%%%%%%%%%%%%%%%%%%%%%%%%%%%%%%%%%%%%
\DescribeMacro{\childdocby}
Each part to be included by |\input| should start with:
%
\begin{center}
\begin{tabular}{l}
|% \iffalse
%
% childdoc.dtx Copyright (C) 2017-2018 Niklas Beisert
%
% This work may be distributed and/or modified under the
% conditions of the LaTeX Project Public License, either version 1.3
% of this license or (at your option) any later version.
% The latest version of this license is in
%   http://www.latex-project.org/lppl.txt
% and version 1.3 or later is part of all distributions of LaTeX
% version 2005/12/01 or later.
%
% This work has the LPPL maintenance status `maintained'.
%
% The Current Maintainer of this work is Niklas Beisert.
%
% This work consists of the files childdoc.dtx and childdoc.ins
% and the derived files childdoc.def and cdocsamp.tex with
% cdocsch1.tex, cdocsch2.tex, cdocsdrf.tex, cdocsfn1.tex, cdocsfn2.tex.
%
%<package>\ifdefined\childdocmain\endinput\fi
%<package>\ProvidesFile{childdoc.def}[2018/12/30 v2.0 child document driver]
%<samplemain>\ProvidesFile{cdocsamp.tex}[2018/12/30 v2.0 sample for childdoc]
%<*driver>
%\ProvidesFile{childdoc.drv}[2018/12/30 v2.0 childdoc reference manual file]
\PassOptionsToClass{10pt,a4paper}{article}
\documentclass{ltxdoc}

\usepackage[margin=35mm]{geometry}
\usepackage{hyperref}
\usepackage{hyperxmp}
\usepackage[usenames]{color}

\hypersetup{colorlinks=true}
\hypersetup{pdfstartview=FitH}
\hypersetup{pdfpagemode=UseNone}
\hypersetup{pdfsource={}}
\hypersetup{pdflang={en-UK}}
\hypersetup{pdfcopyright={Copyright 2017-2018 Niklas Beisert.
  This work may be distributed and/or modified under the
  conditions of the LaTeX Project Public License, either version 1.3
  of this license or (at your option) any later version.}}
\hypersetup{pdflicenseurl={http://www.latex-project.org/lppl.txt}}
\hypersetup{pdfcontactaddress={ETH Zurich, ITP, HIT K,
  Wolfgang-Pauli-Strasse 27}}
\hypersetup{pdfcontactpostcode={8093}}
\hypersetup{pdfcontactcity={Zurich}}
\hypersetup{pdfcontactcountry={Switzerland}}
\hypersetup{pdfcontactemail={nbeisert@itp.phys.ethz.ch}}
\hypersetup{pdfcontacturl={http://people.phys.ethz.ch/\xmptilde nbeisert/}}

\newcommand{\secref}[1]{\hyperref[#1]{section \ref*{#1}}}

\parskip1ex
\parindent0pt
\let\olditemize\itemize
\def\itemize{\olditemize\parskip0pt}

\begin{document}

\title{The \textsf{childdoc} Package}
\hypersetup{pdftitle={The childdoc Package}}
\author{Niklas Beisert\\[2ex]
  Institut f\"ur Theoretische Physik\\
  Eidgen\"ossische Technische Hochschule Z\"urich\\
  Wolfgang-Pauli-Strasse 27, 8093 Z\"urich, Switzerland\\[1ex]
  \href{mailto:nbeisert@itp.phys.ethz.ch}
  {\texttt{nbeisert@itp.phys.ethz.ch}}}
\hypersetup{pdfauthor={Niklas Beisert}}
\hypersetup{pdfsubject={Manual for the LaTeX2e Package childdoc}}
\date{30 December 2018, \textsf{v2.0}}
\maketitle

\begin{abstract}\noindent
\textsf{childdoc} is a \LaTeXe{} package
that enables the direct compilation
of document sections included by |\include|
to individual files.
\end{abstract}

\begingroup
\parskip0ex
\tableofcontents
\endgroup

%%%%%%%%%%%%%%%%%%%%%%%%%%%%%%%%%%%%%%%%%%%%%%%%%%%%%%%%%%%%%%%%%%%%%%%%%%%%%%%%
%%%%%%%%%%%%%%%%%%%%%%%%%%%%%%%%%%%%%%%%%%%%%%%%%%%%%%%%%%%%%%%%%%%%%%%%%%%%%%%%
\section{Introduction}

\LaTeX{} provides a mechanism to structure a large document (such as a book)
into a main file and several child files (containing the chapters)
using the |\include| command.
This mechanism is beneficial for documents
which span hundreds of pages in order to
make the source file(s) more manageable.
Moreover, compilation can be restricted to
selected child files by means of the |\includeonly| command.
The latter feature can be used to reduce the compilation time while editing
(this was significantly more useful in the earlier days of \LaTeX{})
or to generate a smaller document which is easier to navigate.
Another application of |\includeonly| is to generate
documents consisting of selected parts of the complete document.

However, there are a few drawbacks of the plain |\include| mechanism:
\begin{itemize}
\item
The child files cannot be compiled on their own,
they can only be compiled via the main file.
A naive editing environment
(such as a text editor with an option
to have the current file processed by \LaTeX)
may require one to switch to the main file before compiling;
attempting to compile the child file produces errors.
\item
The main file must be modified (each time)
to adjust the |\includeonly| command
to the present needs. This easily leaves the main file in a messy state.
\item
The generated document will always carry the filename
of the main document. This is inconvenient if
several child files are to be compiled and
to be kept for distribution.
\end{itemize}

The present package provides a simple interface
to make child files individually compilable by \LaTeX{}.
Compiling a child file then has the same effect as compiling
the main file with an |\includeonly| command
to select the appropriate child.
Moreover the generated document will carry the name of the child
rather than the main file.
This resolves all three above issues.

This feature is meant to make the editing of books,
thesis documents and lecture notes somewhat more convenient.
However, the package can also be used efficiently for
composing a series of documents (such as exercise sheets)
which are typically distributed individually.
It then assists the author in generating the individual documents
(potentially in different versions)
as well as a document containing the collected series.
Another application is in developing style files
or other kinds of included material
where compilation of the style file could redirect
to a sample or test file.

%%%%%%%%%%%%%%%%%%%%%%%%%%%%%%%%%%%%%%%%%%%%%%%%%%%%%%%%%%%%%%%%%%%%%%%%%%%%%%%%
%%%%%%%%%%%%%%%%%%%%%%%%%%%%%%%%%%%%%%%%%%%%%%%%%%%%%%%%%%%%%%%%%%%%%%%%%%%%%%%%
\section{Usage}

First of all, the package \textsf{childdoc} is \emph{not} a standard
\LaTeXe{} |.sty| style file! Therefore it needs to be invoked in
a non-standard way.

%%%%%%%%%%%%%%%%%%%%%%%%%%%%%%%%%%%%%%%%%%%%%%%%%%%%%%%%%%%%%%%%%%%%%%%%%%%%%%%%
\subsection{Included Files}
\label{sec:include}

%%%%%%%%%%%%%%%%%%%%%%%%%%%%%%%%%%%%%%%%
\DescribeMacro{\childdocmain}
To use the package, add the commands
\begin{center}
\begin{tabular}{l}
|% \iffalse
%
% childdoc.dtx Copyright (C) 2017-2018 Niklas Beisert
%
% This work may be distributed and/or modified under the
% conditions of the LaTeX Project Public License, either version 1.3
% of this license or (at your option) any later version.
% The latest version of this license is in
%   http://www.latex-project.org/lppl.txt
% and version 1.3 or later is part of all distributions of LaTeX
% version 2005/12/01 or later.
%
% This work has the LPPL maintenance status `maintained'.
%
% The Current Maintainer of this work is Niklas Beisert.
%
% This work consists of the files childdoc.dtx and childdoc.ins
% and the derived files childdoc.def and cdocsamp.tex with
% cdocsch1.tex, cdocsch2.tex, cdocsdrf.tex, cdocsfn1.tex, cdocsfn2.tex.
%
%<package>\ifdefined\childdocmain\endinput\fi
%<package>\ProvidesFile{childdoc.def}[2018/12/30 v2.0 child document driver]
%<samplemain>\ProvidesFile{cdocsamp.tex}[2018/12/30 v2.0 sample for childdoc]
%<*driver>
%\ProvidesFile{childdoc.drv}[2018/12/30 v2.0 childdoc reference manual file]
\PassOptionsToClass{10pt,a4paper}{article}
\documentclass{ltxdoc}

\usepackage[margin=35mm]{geometry}
\usepackage{hyperref}
\usepackage{hyperxmp}
\usepackage[usenames]{color}

\hypersetup{colorlinks=true}
\hypersetup{pdfstartview=FitH}
\hypersetup{pdfpagemode=UseNone}
\hypersetup{pdfsource={}}
\hypersetup{pdflang={en-UK}}
\hypersetup{pdfcopyright={Copyright 2017-2018 Niklas Beisert.
  This work may be distributed and/or modified under the
  conditions of the LaTeX Project Public License, either version 1.3
  of this license or (at your option) any later version.}}
\hypersetup{pdflicenseurl={http://www.latex-project.org/lppl.txt}}
\hypersetup{pdfcontactaddress={ETH Zurich, ITP, HIT K,
  Wolfgang-Pauli-Strasse 27}}
\hypersetup{pdfcontactpostcode={8093}}
\hypersetup{pdfcontactcity={Zurich}}
\hypersetup{pdfcontactcountry={Switzerland}}
\hypersetup{pdfcontactemail={nbeisert@itp.phys.ethz.ch}}
\hypersetup{pdfcontacturl={http://people.phys.ethz.ch/\xmptilde nbeisert/}}

\newcommand{\secref}[1]{\hyperref[#1]{section \ref*{#1}}}

\parskip1ex
\parindent0pt
\let\olditemize\itemize
\def\itemize{\olditemize\parskip0pt}

\begin{document}

\title{The \textsf{childdoc} Package}
\hypersetup{pdftitle={The childdoc Package}}
\author{Niklas Beisert\\[2ex]
  Institut f\"ur Theoretische Physik\\
  Eidgen\"ossische Technische Hochschule Z\"urich\\
  Wolfgang-Pauli-Strasse 27, 8093 Z\"urich, Switzerland\\[1ex]
  \href{mailto:nbeisert@itp.phys.ethz.ch}
  {\texttt{nbeisert@itp.phys.ethz.ch}}}
\hypersetup{pdfauthor={Niklas Beisert}}
\hypersetup{pdfsubject={Manual for the LaTeX2e Package childdoc}}
\date{30 December 2018, \textsf{v2.0}}
\maketitle

\begin{abstract}\noindent
\textsf{childdoc} is a \LaTeXe{} package
that enables the direct compilation
of document sections included by |\include|
to individual files.
\end{abstract}

\begingroup
\parskip0ex
\tableofcontents
\endgroup

%%%%%%%%%%%%%%%%%%%%%%%%%%%%%%%%%%%%%%%%%%%%%%%%%%%%%%%%%%%%%%%%%%%%%%%%%%%%%%%%
%%%%%%%%%%%%%%%%%%%%%%%%%%%%%%%%%%%%%%%%%%%%%%%%%%%%%%%%%%%%%%%%%%%%%%%%%%%%%%%%
\section{Introduction}

\LaTeX{} provides a mechanism to structure a large document (such as a book)
into a main file and several child files (containing the chapters)
using the |\include| command.
This mechanism is beneficial for documents
which span hundreds of pages in order to
make the source file(s) more manageable.
Moreover, compilation can be restricted to
selected child files by means of the |\includeonly| command.
The latter feature can be used to reduce the compilation time while editing
(this was significantly more useful in the earlier days of \LaTeX{})
or to generate a smaller document which is easier to navigate.
Another application of |\includeonly| is to generate
documents consisting of selected parts of the complete document.

However, there are a few drawbacks of the plain |\include| mechanism:
\begin{itemize}
\item
The child files cannot be compiled on their own,
they can only be compiled via the main file.
A naive editing environment
(such as a text editor with an option
to have the current file processed by \LaTeX)
may require one to switch to the main file before compiling;
attempting to compile the child file produces errors.
\item
The main file must be modified (each time)
to adjust the |\includeonly| command
to the present needs. This easily leaves the main file in a messy state.
\item
The generated document will always carry the filename
of the main document. This is inconvenient if
several child files are to be compiled and
to be kept for distribution.
\end{itemize}

The present package provides a simple interface
to make child files individually compilable by \LaTeX{}.
Compiling a child file then has the same effect as compiling
the main file with an |\includeonly| command
to select the appropriate child.
Moreover the generated document will carry the name of the child
rather than the main file.
This resolves all three above issues.

This feature is meant to make the editing of books,
thesis documents and lecture notes somewhat more convenient.
However, the package can also be used efficiently for
composing a series of documents (such as exercise sheets)
which are typically distributed individually.
It then assists the author in generating the individual documents
(potentially in different versions)
as well as a document containing the collected series.
Another application is in developing style files
or other kinds of included material
where compilation of the style file could redirect
to a sample or test file.

%%%%%%%%%%%%%%%%%%%%%%%%%%%%%%%%%%%%%%%%%%%%%%%%%%%%%%%%%%%%%%%%%%%%%%%%%%%%%%%%
%%%%%%%%%%%%%%%%%%%%%%%%%%%%%%%%%%%%%%%%%%%%%%%%%%%%%%%%%%%%%%%%%%%%%%%%%%%%%%%%
\section{Usage}

First of all, the package \textsf{childdoc} is \emph{not} a standard
\LaTeXe{} |.sty| style file! Therefore it needs to be invoked in
a non-standard way.

%%%%%%%%%%%%%%%%%%%%%%%%%%%%%%%%%%%%%%%%%%%%%%%%%%%%%%%%%%%%%%%%%%%%%%%%%%%%%%%%
\subsection{Included Files}
\label{sec:include}

%%%%%%%%%%%%%%%%%%%%%%%%%%%%%%%%%%%%%%%%
\DescribeMacro{\childdocmain}
To use the package, add the commands
\begin{center}
\begin{tabular}{l}
|\input{childdoc.def}|\\
|\childdocmain{}|\\
\end{tabular}
\end{center}
at the very top of the main \LaTeX{} file,
in particular \emph{before} the |\documentclass| statement!
The argument of |\childdocmain| should be left empty
(but it must be present).

%%%%%%%%%%%%%%%%%%%%%%%%%%%%%%%%%%%%%%%%
\DescribeMacro{\childdocof}
Furthermore, add the commands
\begin{center}
\begin{tabular}{l}
|\input{childdoc.def}|\\
|\childdocof{|\textit{main}|}|\\
\end{tabular}
\end{center}
at the top of every child file \textit{child}
which is included by |\include{|\textit{child}|}|
from within the main file
(or at least for those files to be compiled individually).
The argument \textit{main} must be the filename of the main file.

There are a couple of
considerations in setting up the main and child documents:

%%%%%%%%%%%%%%%%%%%%%%%%%%%%%%%%%%%%%%%%
\paragraph{Restrictions.}

Please note the following restrictions:
\begin{itemize}
\item
|\childdocmain| must be called with one argument \textit{main}
to ensure compatibility with earlier version of the package.
It must either be empty (|\childdocmain{}|)
or precisely match the filename of the main file in which it is specified.
See \secref{sec:detection} for further information.
\item
The filename \textit{main} must be specified without the |.tex| extension.
\item
The filename \textit{main} is case sensitive
(even in case-insensitive file systems)
due to internal string comparison.
\item
The argument \textit{main} should be fully expanded, it cannot be a macro.
\item
Subdirectories and special characters should be avoided in filenames.
\item
The command |\childdocmain{|\textit{main}|}| must be followed by a whitespace.
It should not be followed immediately by another command
or by a comment mark `|%|'.
This is because the \TeX{} parser reads the token immediately following
the argument of |\childdocmain| and puts it
at the beginning of every child section;
however, a white\-space is ignored.
\end{itemize}

%%%%%%%%%%%%%%%%%%%%%%%%%%%%%%%%%%%%%%%%
\paragraph{Content of Main File.}

It is advisable to place all content in the child files included by |\include|.
Any output contained in the main file will appear in all child documents
unless suppressed manually;
it cannot be suppressed automatically by the |\includeonly| directive
and thus should normally be avoided.
A method to include some content in the main file
by means of conditional processing is described in \secref{sec:conditional}.

%%%%%%%%%%%%%%%%%%%%%%%%%%%%%%%%%%%%%%%%
\paragraph{Page Numbering.}

When only a part of the document is compiled,
the appropriate numbering of pages
(as well as other status parameters)
is determined from the |.aux| files.
The latter contain information from previous passes.
However this information needs to propagate through
all intermediate child documents.
Therefore the page numbering in child documents may well
be inconsistent until the complete document is compiled at least once.

A useful (if unconventional) way to always ensure a consistent
page numbering is to restart the numbering in each child document
and denote the pages by `\textit{child}|.|\textit{page}'
where \textit{child} represents the chapter/section number of the child file.
This can be achieved by the command
|\numberwithin{page}{|\textit{child}|}|
of the \textsf{amsmath} package
where \textit{child} can be |chapter| or |section|
depending on the chosen structuring.
Alternatively, one can modify the macro |\thepage| appropriately
and reset the counter |page| at the start of each child file.

%%%%%%%%%%%%%%%%%%%%%%%%%%%%%%%%%%%%%%%%%%%%%%%%%%%%%%%%%%%%%%%%%%%%%%%%%%%%%%%%
\subsection{Conditional Processing}
\label{sec:conditional}

The package provides a mechanism to compile different versions
of a document. To customise the versions further some conditional processing
can come in handy to distinguish which version is being compiled.
The package provides two macros to describe the compilation context:

%%%%%%%%%%%%%%%%%%%%%%%%%%%%%%%%%%%%%%%%
\DescribeMacro{\ifchilddoc}
The conditional |\ifchilddoc| distinguishes between the compilation of
child documents and the main document:
%
\begin{center}
|\ifchilddoc |\textit{child-code}| |[|\||else |\textit{main-code}]| \||fi|
\end{center}

%%%%%%%%%%%%%%%%%%%%%%%%%%%%%%%%%%%%%%%%
\DescribeMacro{\childdocname}
\DescribeMacro{\childdocjob}
The macro |\childdocname| contains the filename (without extension)
of the main or child file being processed.
Note that |\childdocjob| will always contain the name of the main file.

%%%%%%%%%%%%%%%%%%%%%%%%%%%%%%%%%%%%%%%%
\paragraph{Title Page.}

Conditional processing can be used to include a title or banner page
in the main document when proper precautions are taken.
Importantly, the code in the main file should ensure that the page counter
(as well as other status parameters which are stored in the |.aux| files)
takes the same value after the conditional processing.
Otherwise the page numbers may take divergent values
depending on which part is compiled.

For example, a title page could be declared by:
%
\begin{center}
\begin{tabular}{l}
|\ifchilddoc\||else|\\
|\addtocounter{page}{-1}|\\
\textit{code for title page}\\
|\newpage|\\
|\||fi|
\end{tabular}
\end{center}
%
A banner page for the child documents can be generated by:
%
\begin{center}
\begin{tabular}{l}
|\ifchilddoc|\\
|\addtocounter{page}{-1}|\\
\textit{code for banner page}\\
|\newpage|\\
|\||fi|
\end{tabular}
\end{center}
%
Here one could write a message such as:
\begin{center}
|This is the part \childdocname{} of \childdocjob{}.|
\end{center}

%%%%%%%%%%%%%%%%%%%%%%%%%%%%%%%%%%%%%%%%%%%%%%%%%%%%%%%%%%%%%%%%%%%%%%%%%%%%%%%%
\subsection{Flags}
\label{sec:flags}

The package makes it easy to generate different versions
of the main or child documents.
To this end compilation flags can be defined
and assigned different default values.
They will be particularly useful in conjunction
with the forwarding mechanism described in \secref{sec:forward}.

For example, it may be useful to have a flag |\version|
which can be set to |draft| or |final|.
The document source will contain some conditional code
depending on the value of |\version|.
Suppose further, the flag should default to |final| for the main file
and to |draft| for child files
which is a natural assignment for editing the document.
This is achieved by placing the following code
in the preamble of the main document
(below the |\childdocmain| directive):
%
\begin{center}
\begin{tabular}{l}
|\ifchilddoc|\\
|\providecommand{\version}{draft}|\\
|\||else|\\
|\providecommand{\version}{final}|\\
|\||fi|
\end{tabular}
\end{center}
%
The definition by |\providecommand| makes sure
that previous definitions are not overwritten.
Further statements |\providecommand{\version}{...}|
can thus be added before the above code to override it.

For the main file, one might add a line
(between |\childdocmain| and the above block)
%
\begin{center}
|%\ifchilddoc\||else\providecommand{\version}{draft}\||fi|
\end{center}
%
which can be uncommented to produce a draft version.
Likewise one can add a line to the very top of a child file
(above the |\childdocof{|\textit{main}|}| directive)
%
\begin{center}
|%\providecommand{\version}{final}|
\end{center}
%
which can be uncommented to produce the final version of this child document.

%%%%%%%%%%%%%%%%%%%%%%%%%%%%%%%%%%%%%%%%%%%%%%%%%%%%%%%%%%%%%%%%%%%%%%%%%%%%%%%%
\subsection{Forwarding}
\label{sec:forward}

Different versions of the main or child documents
using compilation flags as described in \secref{sec:flags}
can be (permanently) stored in different files
for convenient compilation, viewing and distribution.
To this end, the package defines a command
to pass on compilation to a different file:

%%%%%%%%%%%%%%%%%%%%%%%%%%%%%%%%%%%%%%%%
\DescribeMacro{\childdocforward}
The command |\childdocforward| redirects processing to
another source file:
%
\begin{center}
\begin{tabular}{l}
|\input{childdoc.def}|\\
|\childdocforward[|\textit{main}|]{|\textit{dest}|}|\\
\end{tabular}
\end{center}
%
The argument \textit{dest} is the destination file
(without extension).
It should be the main file or one of the child files.
Note that further \textsf{childdoc} directives
such as |\childdocof| and |\childdocforward|
in the indicated file will be processed in this form.
The optional argument \textit{main}
passes on directly to the main file \textit{main}
while pretending to compile the child \textit{dest}.
This form behaves as if \textit{dest}
issues |\childdocof{|\textit{main}|}| right away,
and no further \textsf{childdoc} directives will be processed.

%%%%%%%%%%%%%%%%%%%%%%%%%%%%%%%%%%%%%%%%
\DescribeMacro{\...prefix}
In the alternative form |\childdocforwardprefix|,
%
\begin{center}
\begin{tabular}{l}
|\input{childdoc.def}|\\
|\childdocforwardprefix[|\textit{main}|]{|\textit{prefix}|}{|\textit{dest}|}|
\end{tabular}
\end{center}
%
the destination file is determined by a pattern
depending on the current file:
To make this work, the current file must be called
`{\textit{prefix}\hspace{0.2em}\textit{suffix}}'
with \textit{prefix} matching precisely the argument.
Processing is then passed on to the file
`{\textit{dest}\hspace{0.2em}\textit{suffix}}'.
Surely, the same effect is achieved by
directly specifying the
argument `{\textit{dest}\hspace{0.2em}\textit{suffix}}'
in the first form.
However, that requires to set up a different file
for each child. With the alternative form of the command
all these files can have exactly the same content
which simplifies setting them up and maintaining them.

For example, the following file |draft.tex|
with a compilation flag |\version| as described in \secref{sec:flags}
compiles the main document as a draft:
%
\begin{center}
\begin{tabular}{l}
|\def\version{draft}|\\
|\input{childdoc.def}|\\
|\childdocforward{|\textit{main}|}|
\end{tabular}
\end{center}
%
Likewise, the following files |final|\textit{nn}|.tex|
compile the final version of the child document
|child|\textit{nn}|.tex|:
%
\begin{center}
\begin{tabular}{l}
|\def\version{final}|\\
|\input{childdoc.def}|\\
|\childdocforwardprefix{final}{child}|
\end{tabular}
\end{center}
%

Note that when several versions of a main file and/or of each child file
are to be generated, it may be convenient to set up a |Makefile| or
shell script to automatise the process.

%%%%%%%%%%%%%%%%%%%%%%%%%%%%%%%%%%%%%%%%%%%%%%%%%%%%%%%%%%%%%%%%%%%%%%%%%%%%%%%%
\subsection{Command Line Processing}
\label{sec:commandline}

The effect of redirection files can also be achieved by invoking
the \LaTeX{} compiler with a more elaborate command line.
Most conveniently this should be done as part
of a shell script or a |Makefile|.

When using \textsf{childdoc} in the main file, the following
command lines effectively perform a redirection
(note that depending on the shell being used,
backslashes may have to be doubled: `|\|' $\to$ `|\\|'):
%
\begin{center}
|... -jobname "|\textit{target}|" |\\|"|[\textit{flags}]%
|\input{childdoc.def}\childdocforward[|\textit{main}|]{|\textit{dest}|}"|
\end{center}
%
Here \textit{target} is the name of the output file,
\textit{main} is the name of the main file
and \textit{dest} is the name of the main or child file to be processed
(all filenames without extensions).
The optional argument \textit{main} can be omitted
if \textit{main} matches \textit{dest}.
Optionally, compilation \textit{flags} can be defined via |\def| commands.
This command line makes the \TeX{} engine believe
it is compiling the file \textit{target}
whose content is specified as the latter parameter.
The provided code then forwards the processing to
\textit{main} or \textit{dest} as described in \secref{sec:forward}.

%%%%%%%%%%%%%%%%%%%%%%%%%%%%%%%%%%%%%%%%%%%%%%%%%%%%%%%%%%%%%%%%%%%%%%%%%%%%%%%%
\subsection{Include by Input}
\label{sec:input}

Including child documents by |\include| has some restrictions by design.
Most notably, the content of a child document always occupies
its own set of pages; pages cannot be shared between child documents.
Usually, this behaviour makes perfect sense
because each child document contain an essential part of the document.
However, in some situations it may be desirable to compose
a document from a collection of parts
without having mandatory page breaks between then.
For this case, the package
provides a mechanism to include parts
by |\input| which can also be processed individually.
However, by construction this mechanism
requires manual handling of the content to be output.

%%%%%%%%%%%%%%%%%%%%%%%%%%%%%%%%%%%%%%%%
\DescribeMacro{\ifchilddocmanual}
The main file should be prepared as usual, see \secref{sec:include}.
However, the document body must make a distinction
between processing of an individual part and of the main document, e.g.:
%
\begin{center}
\begin{tabular}{l}
|\ifchilddocmanual|\\
|\input{\childdocname}|\\
|\||else|\\
\textit{document body with }|\input{|\textit{part}|}|\\
|\||fi|
\end{tabular}
\end{center}
%
The conditional |\ifchilddocmanual| is true whenever
a part to be included by |\input| is being compiled,
and the name of the part is stored in |\childdocname|.

%%%%%%%%%%%%%%%%%%%%%%%%%%%%%%%%%%%%%%%%
\DescribeMacro{\childdocby}
Each part to be included by |\input| should start with:
%
\begin{center}
\begin{tabular}{l}
|\input{childdoc.def}|\\
|\childdocby{|\textit{main}|}|\\
\end{tabular}
\end{center}
%
The directive |\childdocby| is similar to |\childdocof|
described in \secref{sec:include},
but the subsequent selection of content must be done manually.
To that end, both |\ifchilddoc| and |\ifchilddocmanual|
will be true upon processing of a part,
and the name of the part is stored in |\childdocname|.
Note that |\jobname| will be set to the filename of the current part
so that each part receives an individual |.aux| file
that does not interfere with the |.aux| file(s) of the main document.
This behaviour can be altered by the alternative form
|\childdocby[*]{|\textit{main}|}| (with a non-empty optional argument)
which uses the |.aux| file of the main document
by setting |\jobname| to \textit{main}.

%%%%%%%%%%%%%%%%%%%%%%%%%%%%%%%%%%%%%%%%%%%%%%%%%%%%%%%%%%%%%%%%%%%%%%%%%%%%%%%%
\subsection{Driver Development}
\label{sec:driver}

The \textsf{childdoc} mechanism can also be use for the development
of definition files such as \LaTeX{} styles or classes.
This case differs from the above setup with multiple parts
included by |\include| in that no |\includeonly| should be invoked.
This can be achieved by starting the include file
(before |\ProvidesPackage|) with:
%
\begin{center}
\begin{tabular}{l}
|\input{childdoc.def}|\\
|\childdocforward{|\textit{main}|}|\\
\end{tabular}
\end{center}
%
or alternatively with:
%
\begin{center}
\begin{tabular}{l}
|\input{childdoc.def}|\\
|\childdocby{|\textit{main}|}|\\
\end{tabular}
\end{center}
%
Both forms have slightly different effects as described above.
The main file is prepared as usual, see \secref{sec:include}.

%%%%%%%%%%%%%%%%%%%%%%%%%%%%%%%%%%%%%%%%%%%%%%%%%%%%%%%%%%%%%%%%%%%%%%%%%%%%%%%%
\subsection{Legacy Detection}
\label{sec:detection}

The directive |\childdocmain| in the main file can detect
whether the complete document or merely a child is to be compiled
even without using the directive |\childdocof|.
This method is deprecated because it is less robust
and there is no compelling reason to use it;
it is merely provided for backward compatibility
and it may be removed in future versions.

If the detection mechanism is to be used,
it is mandatory to correctly specify
the filename of the main file as the argument of |\childdocmain|:
%
\begin{center}
\begin{tabular}{l}
|\input{childdoc.def}|\\
|\childdocmain{|\textit{main}|}|\\
\end{tabular}
\end{center}
%
If |\jobname| does not match the argument \textit{main} of |\childdocmain|,
it is assumed that |\jobname| points to the child file to be compiled.
When using |\childdocmain| with the main file specified as argument,
it suffices to start a child file
with just |\input{|\textit{main}|}|
without loading of the package and using |\childdocof|.
If instead all processing is done
with the appropriate \textsf{childdoc} directives,
the argument of \textit{main} of |\childdocmain| can be empty.

An alternative version of the command line processing described
in \secref{sec:commandline} using the detection mechanism reads:
%
\begin{center}
|... -jobname "|\textit{target}|" "|[\textit{flags}]%
[|\def\jobname{|\textit{dest}|}|]|\input{|\textit{main}|}"|
\end{center}

%%%%%%%%%%%%%%%%%%%%%%%%%%%%%%%%%%%%%%%%%%%%%%%%%%%%%%%%%%%%%%%%%%%%%%%%%%%%%%%%
\subsection{Manual Code}
\label{sec:manual}

In case one cannot be certain whether the definitions file |childdoc.def|
is installed on the target \TeX{} distribution
and one prefers not to ship it,
it is conceivable to paste a few relevant commands into the sources.

To that end, drop all statements |\input{childdoc.def}|
and perform the replacements as outlined below.
Instead of |\childdocmain{|\textit{main}|}| add the following code
to the top of the main file:
%
\begin{center}
\begin{tabular}{l}
|\||ifdefined\childdocname\endinput\||fi\newif\ifchilddoc|\\
|\edef\childdocname{\scantokens\expandafter{\jobname\noexpand}}|\\
|\def\childdocmain{|\textit{main}|}\||ifx\childdocmain\childdocname\||else|\\
|\childdoctrue\includeonly{\childdocname}\let\jobname\childdocmain\||fi|\\
\end{tabular}
\end{center}
%
Instead of |\childdocof{|\textit{main}|}| just include the main file
at the top of each child file:
%
\begin{center}
|\input{|\textit{main}|}|
\end{center}
%
A simple redirection |\childdocforward{|\textit{dest}|}| is achieved by:
%
\begin{center}
|\def\jobname{|\textit{dest}|}\input{\jobname}|
\end{center}
%
The redirection with prefix
|\childdocforwardprefix[|\textit{prefix}|]{|\textit{dest}|}|
is accomplished by:
%
\begin{center}
\begin{tabular}{l}
|{\edef\jobname{\scantokens\expandafter{\jobname\noexpand}}|\\
|\def\redirectjob |\textit{prefix}|#1~~~{\gdef\jobname{|\textit{dest}|#1}}|\\
|\expandafter\redirectjob\jobname~~~}\input{\jobname}|
\end{tabular}
\end{center}

In an alternative approach,
child documents can be compiled by a specific command line
without additional code or specific definitions:
%
\begin{center}
|... -jobname "|\textit{target}|" "|[\textit{flags}]%
|\includeonly{|\textit{dest}|}\input{|\textit{main}|}"|
\end{center}
%

%%%%%%%%%%%%%%%%%%%%%%%%%%%%%%%%%%%%%%%%%%%%%%%%%%%%%%%%%%%%%%%%%%%%%%%%%%%%%%%%
%%%%%%%%%%%%%%%%%%%%%%%%%%%%%%%%%%%%%%%%%%%%%%%%%%%%%%%%%%%%%%%%%%%%%%%%%%%%%%%%
\section{Information}

%%%%%%%%%%%%%%%%%%%%%%%%%%%%%%%%%%%%%%%%%%%%%%%%%%%%%%%%%%%%%%%%%%%%%%%%%%%%%%%%
\subsection{Copyright}

Copyright \copyright{} 2017--2018 Niklas Beisert

This work may be distributed and/or modified under the
conditions of the \LaTeX{} Project Public License, either version 1.3
of this license or (at your option) any later version.
The latest version of this license is in
  \url{http://www.latex-project.org/lppl.txt}
and version 1.3 or later is part of all distributions of \LaTeX{}
version 2005/12/01 or later.

This work has the LPPL maintenance status `maintained'.

The Current Maintainer of this work is Niklas Beisert.

This work consists of the files |README.txt|, |childdoc.ins| and |childdoc.dtx|
as well as the derived files |childdoc.def|, |cdocsamp.tex|
with |cdocsch1.tex|, |cdocsch2.tex|, |cdocspt3.tex|, |cdocspt4.tex|,
|cdocsdrf.tex|, |cdocsfn1.tex|, |cdocsfn2.tex|
as well as |childdoc.pdf|.

%%%%%%%%%%%%%%%%%%%%%%%%%%%%%%%%%%%%%%%%%%%%%%%%%%%%%%%%%%%%%%%%%%%%%%%%%%%%%%%%
\subsection{Files and Installation}

The package consists of the files:
%
\begin{center}
\begin{tabular}{ll}
    |README.txt|   & readme file \\
    |childdoc.ins| & installation file \\
    |childdoc.dtx| & source file \\
    |childdoc.def| & definition file \\
    |cdocsamp.tex| & sample main file \\
    |cdocsch1.tex| & sample include file \\
    |cdocsch2.tex| & sample include file \\
    |cdocspt3.tex| & sample part file \\
    |cdocspt4.tex| & sample part file \\
    |cdocsdrf.tex| & sample redirection file \\
    |cdocsfn1.tex| & sample redirection file \\
    |cdocsfn2.tex| & sample redirection file \\
    |childdoc.pdf| & manual
\end{tabular}
\end{center}
%
The distribution consists of the files
|README.txt|, |childdoc.ins| and |childdoc.dtx|.
%
\begin{itemize}
\item
Run (pdf)\LaTeX{} on |childdoc.dtx|
to compile the manual |childdoc.pdf| (this file).
\item
Run \LaTeX{} on |childdoc.ins| to create the definitions file |childdoc.def|
and the sample |cdocsamp.tex| with include files
|cdocsch1.tex|, |cdocsch2.tex|, |cdocspt3.tex|, |cdocspt4.tex|,
|cdocsdrf.tex|, |cdocsfn1.tex|, |cdocsfn2.tex|.
Then copy the file |childdoc.def| to an appropriate directory of your \LaTeX{}
distribution, e.g.\ \textit{texmf-root}|/tex/latex/childdoc|.
\end{itemize}

%%%%%%%%%%%%%%%%%%%%%%%%%%%%%%%%%%%%%%%%%%%%%%%%%%%%%%%%%%%%%%%%%%%%%%%%%%%%%%%%
\subsection{Related CTAN Packages}

There are several other packages which offer a similar functionality:
%
\begin{itemize}
\item
The packages
\href{http://ctan.org/pkg/docmute}{\textsf{docmute}},
\href{http://ctan.org/pkg/includex}{\textsf{includex}} and
\href{http://ctan.org/pkg/standalone}{\textsf{standalone}}
provide commands to include only the document body of
a child file thus allowing both files to be compiled individually.
\item
The packages \href{http://ctan.org/pkg/subdocs}{\textsf{subdocs}}
and \href{http://ctan.org/pkg/subfiles}{\textsf{subfiles}}
provide structures in which the main and child documents can be
encapsulated and allowing them to be compiled individually.
The inclusion mechanism is different from the conventional |\include|.
\item
The package \href{http://ctan.org/pkg/combine}{\textsf{combine}}
is an elaborate solution to combine several documents into one.
\end{itemize}
%
See also the CTAN topic \href{http://ctan.org/topic/subdocs}{\textsf{subdocs}}
for further related packages.
The present package differs from the above solutions in that
a document structure constructed with the conventional |\include| mechanism
just needs two extra commands at the top of every file
such that all constituent files can be compiled individually.

%%%%%%%%%%%%%%%%%%%%%%%%%%%%%%%%%%%%%%%%%%%%%%%%%%%%%%%%%%%%%%%%%%%%%%%%%%%%%%%%
%\subsection{Feature Suggestions}
%
%The following is a list of features which may be useful for future
%versions of this package:
%%
%\begin{itemize}
%\item
%\ldots
%\end{itemize}

%%%%%%%%%%%%%%%%%%%%%%%%%%%%%%%%%%%%%%%%%%%%%%%%%%%%%%%%%%%%%%%%%%%%%%%%%%%%%%%%
\subsection{Revision History}

%%%%%%%%%%%%%%%%%%%%%%%%%%%%%%%%%%%%%%%%
\paragraph{v2.0:} 2018/12/30

\begin{itemize}
\item
immediate forward processing
\item
added |\childdocby| mechanism
\item
manual restructured
\end{itemize}

%%%%%%%%%%%%%%%%%%%%%%%%%%%%%%%%%%%%%%%%
\paragraph{v1.6:} 2018/01/17

\begin{itemize}
\item
application for development of include files
\item
corrections to manual
\end{itemize}

%%%%%%%%%%%%%%%%%%%%%%%%%%%%%%%%%%%%%%%%
\paragraph{v1.5:} 2017/05/21

\begin{itemize}
\item
more complete structuring introduced
\item
|\childdocof| introduced
\item
|\childdoc| renamed to |\childdocmain|
\item
|\childredirect| renamed to |\childdocforward| and |\childdocforwardprefix|
and functionality expanded
\end{itemize}

%%%%%%%%%%%%%%%%%%%%%%%%%%%%%%%%%%%%%%%%
\paragraph{v1.0:} 2017/04/27

\begin{itemize}
\item
manual and install package
\item
first version published on CTAN
\end{itemize}

%%%%%%%%%%%%%%%%%%%%%%%%%%%%%%%%%%%%%%%%
\paragraph{v0.6:} 2017/04/26

\begin{itemize}
\item
redirection mechanism added
\end{itemize}

%%%%%%%%%%%%%%%%%%%%%%%%%%%%%%%%%%%%%%%%
\paragraph{v0.5:} 2017/04/26

\begin{itemize}
\item
functionality in definition file
\end{itemize}


%%%%%%%%%%%%%%%%%%%%%%%%%%%%%%%%%%%%%%%%%%%%%%%%%%%%%%%%%%%%%%%%%%%%%%%%%%%%%%%%
%%%%%%%%%%%%%%%%%%%%%%%%%%%%%%%%%%%%%%%%%%%%%%%%%%%%%%%%%%%%%%%%%%%%%%%%%%%%%%%%
%%%%%%%%%%%%%%%%%%%%%%%%%%%%%%%%%%%%%%%%%%%%%%%%%%%%%%%%%%%%%%%%%%%%%%%%%%%%%%%%
\appendix

\settowidth\MacroIndent{\rmfamily\scriptsize 000\ }

 \DocInput{childdoc.dtx}

\end{document}
%</driver>
% \fi
%
% %%%%%%%%%%%%%%%%%%%%%%%%%%%%%%%%%%%%%%%%%%%%%%%%%%%%%%%%%%%%%%%%%%%%%%%%%%%%%%
% %%%%%%%%%%%%%%%%%%%%%%%%%%%%%%%%%%%%%%%%%%%%%%%%%%%%%%%%%%%%%%%%%%%%%%%%%%%%%%
% \section{Sample}
%\iffalse
%<*samplemain>
%\fi
%
% The following presents a sample document
% with two chapters, two parts, a title page,
% a compile flag as well as three forwarding files to set the flag.
% It consists of eight |.tex| files:
% \begin{center}
% \begin{tabular}{ll}
% |cdocsamp.tex|&main file\\
% |cdocsch1.tex|&include file for chapter 1\\
% |cdocsch2.tex|&include file for chapter 2\\
% |cdocspt3.tex|&include file for part 3\\
% |cdocspt4.tex|&include file for part 4\\
% |cdocsdrf.tex|&forwarding file for main file in draft mode\\
% |cdocsfi1.tex|&forwarding file for final version of chapter 1\\
% |cdocsfi2.tex|&forwarding file for final version of chapter 2\\
% \end{tabular}
% \end{center}
% Each of the eight files can be compiled directly by the \LaTeX{} compiler.
%
% %%%%%%%%%%%%%%%%%%%%%%%%%%%%%%%%%%%%%%
% \paragraph{Main File.}
%
% The main file is called |cdocsamp.tex|.
%
% Load the \textsf{childdoc} definitions and
% declare the filename for the main document:
%    \begin{macrocode}
\input{childdoc.def}
\childdocmain{}
%    \end{macrocode}

% Optional override for |\version| flag:
%    \begin{macrocode}
%%\ifchilddoc\else\providecommand{\version}{draft}\fi
%    \end{macrocode}

% Define the default values for the |\version| flag
% (|final| for the main file and |draft| for childs):
%    \begin{macrocode}
\ifchilddoc
\providecommand{\version}{draft}
\else
\providecommand{\version}{final}
\fi
%    \end{macrocode}

% Load the standard document class:
%    \begin{macrocode}
\documentclass[12pt]{article}
%    \end{macrocode}

% Start the document body:
%    \begin{macrocode}
\begin{document}
%    \end{macrocode}

% Declare a title page.
% Print title, part of document being processed and version flag:
%    \begin{macrocode}
\addtocounter{page}{-1}
\begin{center}
{\LARGE\bfseries{}childdoc example\par}
\vspace{1cm}
\ifchilddoc
\ifchilddocmanual part\else chapter\fi:
`\childdocname' of `\childdocjob'\par
\else
main document: `\childdocjob'\par
\fi
version: \version\par
\end{center}
\newpage
%    \end{macrocode}

% Manually include selected file,
% otherwise process as usual:
%    \begin{macrocode}
\ifchilddocmanual
\section*{part `\childdocname'}
\input{\childdocname}
\else
%    \end{macrocode}

% Include the two chapters:
%    \begin{macrocode}
\include{cdocsch1}
\include{cdocsch2}
%    \end{macrocode}

% Include the two parts unless only chapters should be displayed:
%    \begin{macrocode}
\ifchilddoc\else
\section{part three}
\input{cdocspt3}
\section{part four}
\input{cdocspt4}
\fi
%    \end{macrocode}

% Process as usual until here:
%    \begin{macrocode}
\fi
%    \end{macrocode}

% End of document body:
%    \begin{macrocode}
\end{document}
%    \end{macrocode}
%\iffalse
%</samplemain>
%\fi
%
% %%%%%%%%%%%%%%%%%%%%%%%%%%%%%%%%%%%%%%
% \paragraph{Chapter Include Files.}
%
% The include files are called |cdocsch1.tex| and |cdocsch2.tex|.
%
%\iffalse
%<*samplechap1|samplechap2>
%\fi

% Optional override for |\version| flag:
%    \begin{macrocode}
%%\providecommand{\version}{final}
%    \end{macrocode}

% Include the main document:
%    \begin{macrocode}
\input{childdoc.def}
\childdocof{cdocsamp}
%    \end{macrocode}

%\iffalse
%</samplechap1|samplechap2>
%\fi
%
%\iffalse
%<*samplechap1>
%\fi
% Some text for chapter 1:
%    \begin{macrocode}
\section{one}
some text in chapter one
%    \end{macrocode}

%\iffalse
%</samplechap1>
%\fi
% Some text for chapter 2:
%\iffalse
%<*samplechap2>
%\fi
%    \begin{macrocode}
\section{two}
more text in chapter two
%    \end{macrocode}

%\iffalse
%</samplechap2>
%\fi
%
% %%%%%%%%%%%%%%%%%%%%%%%%%%%%%%%%%%%%%%
% \paragraph{Part Include Files.}
%
% The include files are called |cdocspt3.tex| and |cdocspt4.tex|.
%
%\iffalse
%<*samplepart3|samplepart4>
%\fi

% Optional override for |\version| flag:
%    \begin{macrocode}
%%\providecommand{\version}{final}
%    \end{macrocode}

% Include the main document:
%    \begin{macrocode}
\input{childdoc.def}
\childdocby{cdocsamp}
%    \end{macrocode}

%\iffalse
%</samplepart3|samplepart4>
%\fi
%
%\iffalse
%<*samplepart3>
%\fi
% Some text for part 3:
%    \begin{macrocode}
some text in part three
%    \end{macrocode}

%\iffalse
%</samplepart3>
%\fi
% Some text for part 4:
%\iffalse
%<*samplepart4>
%\fi
%    \begin{macrocode}
more text in part four
%    \end{macrocode}

%\iffalse
%</samplepart4>
%\fi
%
% %%%%%%%%%%%%%%%%%%%%%%%%%%%%%%%%%%%%%%
% \paragraph{Forwarding for a Complete Draft.}
%
% The following forwarding file |cdocsdrf.tex|
% compiles the main document in draft mode:
%\iffalse
%<*sampledraft>
%\fi
%    \begin{macrocode}
\def\version{draft}
\input{childdoc.def}
\childdocforward{cdocsamp}
%    \end{macrocode}

%\iffalse
%</sampledraft>
%\fi
%
% %%%%%%%%%%%%%%%%%%%%%%%%%%%%%%%%%%%%%%
% \paragraph{Forwarding for Final Version of the Chapters.}
%
% The following forwarding files |cdocsfn1.tex| and |cdocsfn2.tex|
% (with identical content)
% compile the final versions of the child documents
% |cdocsch1.tex| and |cdocsch2.tex|, respectively:
%\iffalse
%<*samplefinal>
%\fi
%    \begin{macrocode}
\def\version{final}
\input{childdoc.def}
\childdocforwardprefix[cdocsamp]{cdocsfn}{cdocsch}
%    \end{macrocode}

%\iffalse
%</samplefinal>
%\fi
%
% %%%%%%%%%%%%%%%%%%%%%%%%%%%%%%%%%%%%%%
% \paragraph{Command Line Processing.}
%
% The following three command lines generate the output files
% |cdocscld|, |cdocscl1| and |cdocscl2|
% which should be identical to
% |cdocsdrf|, |cdocsch1| and |cdocsfn2|, respectively:
% \begin{center}
% \begin{tabular}{l}
% |latex -jobname cdocscld \|\\
% |  "\def\version{draft}\input{childdoc.def}\childdocforward{cdocsamp}"|\\
% |latex -jobname cdocscl1 \|\\
% |  "\input{childdoc.def}\childdocforward[cdocsamp]{cdocsch1}"|\\
% |latex -jobname cdocscl2 \|\\
% |  "\def\version{final}\input{childdoc.def}\childdocforward{cdocsch2}"|
% \end{tabular}
% \end{center}
% Note that the trailing backslash on each first line
% merely continues the input to the second line
% (for convenient cut ant paste).
% Furthermore, the command |latex| can be replaced by any
% of its alternative versions such as |pdflatex|.
%
% %%%%%%%%%%%%%%%%%%%%%%%%%%%%%%%%%%%%%%%%%%%%%%%%%%%%%%%%%%%%%%%%%%%%%%%%%%%%%%
% %%%%%%%%%%%%%%%%%%%%%%%%%%%%%%%%%%%%%%%%%%%%%%%%%%%%%%%%%%%%%%%%%%%%%%%%%%%%%%
% \section{Implementation}
%\iffalse
%<*package>
%\fi
%
% This section describes the definitions file |childdoc.def|.

% The definitions cannot be loaded using |\usepackage| or |\RequirePackage|
% which has a mechanism to prevent loading a style file more than once.
% When loading the definitions by means of |\input|
% multiple instances have to be prevented manually:
%\iffalse
%This code needs to be before the `\ProvidesFile' directive
%which is defined at the beginning of this file.
%Therefore it is also placed there and commented out here.
%</package>
%<*discard>
%\fi
%    \begin{macrocode}
\ifdefined\childdocmain\endinput\fi
%    \end{macrocode}
%\iffalse
%</discard>
%<*package>
%\fi
%
% \macro{\ifchilddoc}
% \macro{\ifchilddocmanual}
% The conditional |\ifchilddoc| tells whether a
% child (true) or main (false) document is being compiled.
% The conditional |\ifchilddocmanual| tells whether
% the |\includeonly| mechanism is used (false) or
% the selection of child files must be performed manually (true).
% The definitions initialise to false:
%    \begin{macrocode}
\newif\ifchilddoc
\newif\ifchilddocmanual
%    \end{macrocode}

% \macro{\childdocname}
% \macro{\childdocjob}
% The macro |\childdocname| stores the name of the main document
% to be compiled. The macro |\childdocjob| stores the name of
% the document on which the \LaTeX{} compiler was originally invoked.
% The content of |\jobname| cannot be compared
% to filenames specified in the source due to different catcodes.
% The following code rescans |\jobname|, stores the result
% in |\childdocname| and saves a copy in |\childdocjob|:
%    \begin{macrocode}
\edef\childdocname{\scantokens\expandafter{\jobname\noexpand}}
\let\childdocjob\childdocname
%    \end{macrocode}

% \macro{\childdocdisable}
% The macro |\childdocdisable| prevents the main file
% from being processed more than once.
% At this stage, the main document command |\childdocmain|
% is assumed to be called once again where it should do nothing.
% Any subsequent call to it should prevent
% a secondary processing of the main document
% It overwrites the forwarding commands
% |\childdocof| and |\childdocforward|
% with empty macros to prevent further inclusions of the main document:
%    \begin{macrocode}
\newcommand{\childdocdisable}
{
  \renewcommand{\childdocmain}[1]{\renewcommand{\childdocmain}[1]{\endinput}}
  \renewcommand{\childdocof}[1]{}
  \renewcommand{\childdocby}[2][]{}
  \renewcommand{\childdocforward}[2][]{}
  \renewcommand{\childdocdisable}{}
}
%    \end{macrocode}

% \macro{\childdocmain}
% The macro |\childdocmain| is to be called at the top of the main file
% with nothing or the main filename (without extension) as argument.
% First, it breaks loops.
% If the argument is not empty and does not match |\childdocname|
% (which is set by the first inclusion of |childdoc.def|),
% |\ifchilddoc| is set to true, |\includeonly| is applied to the child file
% and |\jobname| is set to the main file
% (for proper handling of |.aux| files):
%    \begin{macrocode}
\newcommand{\childdocmain}[1]
{
  \childdocdisable\childdocmain{}
  \if?#1?\else
    \begingroup
      \def\childdoctmp{#1}
      \ifx\childdoctmp\childdocname
        \def\childdoctmp{}
      \else
        \def\childdoctmp
        {
          \childdoctrue
          \includeonly{\childdocname}
          \def\childdocjob{#1}
          \def\jobname{#1}
        }
      \fi
      \expandafter
    \endgroup
    \childdoctmp
  \fi
}
%    \end{macrocode}

% \macro{\childdocof}
% The command |\childdocof| redirects
% compilation to the main file |#1|.
%    \begin{macrocode}
\newcommand{\childdocof}[1]
{
  \childdocdisable
  \childdoctrue
  \includeonly{\childdocname}
  \def\jobname{#1}
  \def\childdocjob{#1}
  \input{#1}
}
%    \end{macrocode}

% \macro{\childdocby}
% The command |\childdocby| ....
%    \begin{macrocode}
\newcommand{\childdocby}[2][]
{
  \childdocdisable
  \childdoctrue
  \childdocmanualtrue
  \if?#1?\else
    \def\jobname{#2}
  \fi
  \def\childdocjob{#2}
  \input{#2}
  \endinput
}
%    \end{macrocode}

% \macro{\childdocforward}
% The command |\childdocforward| redirects
% compilation to the main file or
% (if the optional argument is given) a child file.
% Parameters are set as if the main file
% or a child file starting with |\childdocof| was compiled.
% Then compilation is handed over to the main file:
%    \begin{macrocode}
\newcommand{\childdocforward}[2][]
{
  \begingroup
    \if?#1?
      \def\childdoctmp
      {
        \def\childdocname{#2}
        \def\childdocjob{#2}
        \def\jobname{#2}
        \input{#2}
        \endinput
      }
    \else
      \def\childdoctmp
      {
        \childdocdisable
        \def\childdocname{#2}
        \childdoctrue
        \includeonly{#2}
        \def\childdocjob{#1}
        \def\jobname{#1}
        \input{#1}
        \endinput
      }
    \fi
    \expandafter
  \endgroup
  \childdoctmp
}
%    \end{macrocode}

% \macro{\childdocforwardprefix}
% The command |\childdocforwardprefix| redirects
% compilation to the main or a child file by means of a pattern.
% The prefix |#1| in the current filename is replaced by |#2|
% and the suffix of the current filename is kept
% (it is assumed that the filename does not contain the substring `|~~~|'
% which is used as a delimiter).
% Compilation is handed over to the new file by |\childdocforward|:
%    \begin{macrocode}
\newcommand{\childdocforwardprefix}[3][]
{
  \begingroup
    \def\childdocextract #2##1~~~{\def\childdoctmp{\childdocforward[#1]{#3##1}}}
    \expandafter\childdocextract\childdocname~~~
    \expandafter
  \endgroup
  \childdoctmp
}
%    \end{macrocode}

% \macro{\childdoc}
% The deprecated macro |\childdoc| is a legacy version of |\childdocmain|:
%    \begin{macrocode}
\newcommand{\childdoc}{\childdocmain}
%    \end{macrocode}

% \macro{\childdocredirect}
% The deprecated macro |\childdocredirect| is a legacy version
% of |\childdocforward| and |\childdocforwardprefix|:
%    \begin{macrocode}
\newcommand{\childdocredirect}[2][]
{
  \begingroup
    \if?#1?
      \def\childdoctmp{\childdocforward{#2}}
    \else
      \def\childdoctmp{\childdocforwardprefix{#1}{#2}}
    \fi
    \expandafter
  \endgroup
  \childdoctmp
}
%    \end{macrocode}

%\iffalse
%</package>
%\fi
%
\endinput
|\\
|\childdocmain{}|\\
\end{tabular}
\end{center}
at the very top of the main \LaTeX{} file,
in particular \emph{before} the |\documentclass| statement!
The argument of |\childdocmain| should be left empty
(but it must be present).

%%%%%%%%%%%%%%%%%%%%%%%%%%%%%%%%%%%%%%%%
\DescribeMacro{\childdocof}
Furthermore, add the commands
\begin{center}
\begin{tabular}{l}
|% \iffalse
%
% childdoc.dtx Copyright (C) 2017-2018 Niklas Beisert
%
% This work may be distributed and/or modified under the
% conditions of the LaTeX Project Public License, either version 1.3
% of this license or (at your option) any later version.
% The latest version of this license is in
%   http://www.latex-project.org/lppl.txt
% and version 1.3 or later is part of all distributions of LaTeX
% version 2005/12/01 or later.
%
% This work has the LPPL maintenance status `maintained'.
%
% The Current Maintainer of this work is Niklas Beisert.
%
% This work consists of the files childdoc.dtx and childdoc.ins
% and the derived files childdoc.def and cdocsamp.tex with
% cdocsch1.tex, cdocsch2.tex, cdocsdrf.tex, cdocsfn1.tex, cdocsfn2.tex.
%
%<package>\ifdefined\childdocmain\endinput\fi
%<package>\ProvidesFile{childdoc.def}[2018/12/30 v2.0 child document driver]
%<samplemain>\ProvidesFile{cdocsamp.tex}[2018/12/30 v2.0 sample for childdoc]
%<*driver>
%\ProvidesFile{childdoc.drv}[2018/12/30 v2.0 childdoc reference manual file]
\PassOptionsToClass{10pt,a4paper}{article}
\documentclass{ltxdoc}

\usepackage[margin=35mm]{geometry}
\usepackage{hyperref}
\usepackage{hyperxmp}
\usepackage[usenames]{color}

\hypersetup{colorlinks=true}
\hypersetup{pdfstartview=FitH}
\hypersetup{pdfpagemode=UseNone}
\hypersetup{pdfsource={}}
\hypersetup{pdflang={en-UK}}
\hypersetup{pdfcopyright={Copyright 2017-2018 Niklas Beisert.
  This work may be distributed and/or modified under the
  conditions of the LaTeX Project Public License, either version 1.3
  of this license or (at your option) any later version.}}
\hypersetup{pdflicenseurl={http://www.latex-project.org/lppl.txt}}
\hypersetup{pdfcontactaddress={ETH Zurich, ITP, HIT K,
  Wolfgang-Pauli-Strasse 27}}
\hypersetup{pdfcontactpostcode={8093}}
\hypersetup{pdfcontactcity={Zurich}}
\hypersetup{pdfcontactcountry={Switzerland}}
\hypersetup{pdfcontactemail={nbeisert@itp.phys.ethz.ch}}
\hypersetup{pdfcontacturl={http://people.phys.ethz.ch/\xmptilde nbeisert/}}

\newcommand{\secref}[1]{\hyperref[#1]{section \ref*{#1}}}

\parskip1ex
\parindent0pt
\let\olditemize\itemize
\def\itemize{\olditemize\parskip0pt}

\begin{document}

\title{The \textsf{childdoc} Package}
\hypersetup{pdftitle={The childdoc Package}}
\author{Niklas Beisert\\[2ex]
  Institut f\"ur Theoretische Physik\\
  Eidgen\"ossische Technische Hochschule Z\"urich\\
  Wolfgang-Pauli-Strasse 27, 8093 Z\"urich, Switzerland\\[1ex]
  \href{mailto:nbeisert@itp.phys.ethz.ch}
  {\texttt{nbeisert@itp.phys.ethz.ch}}}
\hypersetup{pdfauthor={Niklas Beisert}}
\hypersetup{pdfsubject={Manual for the LaTeX2e Package childdoc}}
\date{30 December 2018, \textsf{v2.0}}
\maketitle

\begin{abstract}\noindent
\textsf{childdoc} is a \LaTeXe{} package
that enables the direct compilation
of document sections included by |\include|
to individual files.
\end{abstract}

\begingroup
\parskip0ex
\tableofcontents
\endgroup

%%%%%%%%%%%%%%%%%%%%%%%%%%%%%%%%%%%%%%%%%%%%%%%%%%%%%%%%%%%%%%%%%%%%%%%%%%%%%%%%
%%%%%%%%%%%%%%%%%%%%%%%%%%%%%%%%%%%%%%%%%%%%%%%%%%%%%%%%%%%%%%%%%%%%%%%%%%%%%%%%
\section{Introduction}

\LaTeX{} provides a mechanism to structure a large document (such as a book)
into a main file and several child files (containing the chapters)
using the |\include| command.
This mechanism is beneficial for documents
which span hundreds of pages in order to
make the source file(s) more manageable.
Moreover, compilation can be restricted to
selected child files by means of the |\includeonly| command.
The latter feature can be used to reduce the compilation time while editing
(this was significantly more useful in the earlier days of \LaTeX{})
or to generate a smaller document which is easier to navigate.
Another application of |\includeonly| is to generate
documents consisting of selected parts of the complete document.

However, there are a few drawbacks of the plain |\include| mechanism:
\begin{itemize}
\item
The child files cannot be compiled on their own,
they can only be compiled via the main file.
A naive editing environment
(such as a text editor with an option
to have the current file processed by \LaTeX)
may require one to switch to the main file before compiling;
attempting to compile the child file produces errors.
\item
The main file must be modified (each time)
to adjust the |\includeonly| command
to the present needs. This easily leaves the main file in a messy state.
\item
The generated document will always carry the filename
of the main document. This is inconvenient if
several child files are to be compiled and
to be kept for distribution.
\end{itemize}

The present package provides a simple interface
to make child files individually compilable by \LaTeX{}.
Compiling a child file then has the same effect as compiling
the main file with an |\includeonly| command
to select the appropriate child.
Moreover the generated document will carry the name of the child
rather than the main file.
This resolves all three above issues.

This feature is meant to make the editing of books,
thesis documents and lecture notes somewhat more convenient.
However, the package can also be used efficiently for
composing a series of documents (such as exercise sheets)
which are typically distributed individually.
It then assists the author in generating the individual documents
(potentially in different versions)
as well as a document containing the collected series.
Another application is in developing style files
or other kinds of included material
where compilation of the style file could redirect
to a sample or test file.

%%%%%%%%%%%%%%%%%%%%%%%%%%%%%%%%%%%%%%%%%%%%%%%%%%%%%%%%%%%%%%%%%%%%%%%%%%%%%%%%
%%%%%%%%%%%%%%%%%%%%%%%%%%%%%%%%%%%%%%%%%%%%%%%%%%%%%%%%%%%%%%%%%%%%%%%%%%%%%%%%
\section{Usage}

First of all, the package \textsf{childdoc} is \emph{not} a standard
\LaTeXe{} |.sty| style file! Therefore it needs to be invoked in
a non-standard way.

%%%%%%%%%%%%%%%%%%%%%%%%%%%%%%%%%%%%%%%%%%%%%%%%%%%%%%%%%%%%%%%%%%%%%%%%%%%%%%%%
\subsection{Included Files}
\label{sec:include}

%%%%%%%%%%%%%%%%%%%%%%%%%%%%%%%%%%%%%%%%
\DescribeMacro{\childdocmain}
To use the package, add the commands
\begin{center}
\begin{tabular}{l}
|\input{childdoc.def}|\\
|\childdocmain{}|\\
\end{tabular}
\end{center}
at the very top of the main \LaTeX{} file,
in particular \emph{before} the |\documentclass| statement!
The argument of |\childdocmain| should be left empty
(but it must be present).

%%%%%%%%%%%%%%%%%%%%%%%%%%%%%%%%%%%%%%%%
\DescribeMacro{\childdocof}
Furthermore, add the commands
\begin{center}
\begin{tabular}{l}
|\input{childdoc.def}|\\
|\childdocof{|\textit{main}|}|\\
\end{tabular}
\end{center}
at the top of every child file \textit{child}
which is included by |\include{|\textit{child}|}|
from within the main file
(or at least for those files to be compiled individually).
The argument \textit{main} must be the filename of the main file.

There are a couple of
considerations in setting up the main and child documents:

%%%%%%%%%%%%%%%%%%%%%%%%%%%%%%%%%%%%%%%%
\paragraph{Restrictions.}

Please note the following restrictions:
\begin{itemize}
\item
|\childdocmain| must be called with one argument \textit{main}
to ensure compatibility with earlier version of the package.
It must either be empty (|\childdocmain{}|)
or precisely match the filename of the main file in which it is specified.
See \secref{sec:detection} for further information.
\item
The filename \textit{main} must be specified without the |.tex| extension.
\item
The filename \textit{main} is case sensitive
(even in case-insensitive file systems)
due to internal string comparison.
\item
The argument \textit{main} should be fully expanded, it cannot be a macro.
\item
Subdirectories and special characters should be avoided in filenames.
\item
The command |\childdocmain{|\textit{main}|}| must be followed by a whitespace.
It should not be followed immediately by another command
or by a comment mark `|%|'.
This is because the \TeX{} parser reads the token immediately following
the argument of |\childdocmain| and puts it
at the beginning of every child section;
however, a white\-space is ignored.
\end{itemize}

%%%%%%%%%%%%%%%%%%%%%%%%%%%%%%%%%%%%%%%%
\paragraph{Content of Main File.}

It is advisable to place all content in the child files included by |\include|.
Any output contained in the main file will appear in all child documents
unless suppressed manually;
it cannot be suppressed automatically by the |\includeonly| directive
and thus should normally be avoided.
A method to include some content in the main file
by means of conditional processing is described in \secref{sec:conditional}.

%%%%%%%%%%%%%%%%%%%%%%%%%%%%%%%%%%%%%%%%
\paragraph{Page Numbering.}

When only a part of the document is compiled,
the appropriate numbering of pages
(as well as other status parameters)
is determined from the |.aux| files.
The latter contain information from previous passes.
However this information needs to propagate through
all intermediate child documents.
Therefore the page numbering in child documents may well
be inconsistent until the complete document is compiled at least once.

A useful (if unconventional) way to always ensure a consistent
page numbering is to restart the numbering in each child document
and denote the pages by `\textit{child}|.|\textit{page}'
where \textit{child} represents the chapter/section number of the child file.
This can be achieved by the command
|\numberwithin{page}{|\textit{child}|}|
of the \textsf{amsmath} package
where \textit{child} can be |chapter| or |section|
depending on the chosen structuring.
Alternatively, one can modify the macro |\thepage| appropriately
and reset the counter |page| at the start of each child file.

%%%%%%%%%%%%%%%%%%%%%%%%%%%%%%%%%%%%%%%%%%%%%%%%%%%%%%%%%%%%%%%%%%%%%%%%%%%%%%%%
\subsection{Conditional Processing}
\label{sec:conditional}

The package provides a mechanism to compile different versions
of a document. To customise the versions further some conditional processing
can come in handy to distinguish which version is being compiled.
The package provides two macros to describe the compilation context:

%%%%%%%%%%%%%%%%%%%%%%%%%%%%%%%%%%%%%%%%
\DescribeMacro{\ifchilddoc}
The conditional |\ifchilddoc| distinguishes between the compilation of
child documents and the main document:
%
\begin{center}
|\ifchilddoc |\textit{child-code}| |[|\||else |\textit{main-code}]| \||fi|
\end{center}

%%%%%%%%%%%%%%%%%%%%%%%%%%%%%%%%%%%%%%%%
\DescribeMacro{\childdocname}
\DescribeMacro{\childdocjob}
The macro |\childdocname| contains the filename (without extension)
of the main or child file being processed.
Note that |\childdocjob| will always contain the name of the main file.

%%%%%%%%%%%%%%%%%%%%%%%%%%%%%%%%%%%%%%%%
\paragraph{Title Page.}

Conditional processing can be used to include a title or banner page
in the main document when proper precautions are taken.
Importantly, the code in the main file should ensure that the page counter
(as well as other status parameters which are stored in the |.aux| files)
takes the same value after the conditional processing.
Otherwise the page numbers may take divergent values
depending on which part is compiled.

For example, a title page could be declared by:
%
\begin{center}
\begin{tabular}{l}
|\ifchilddoc\||else|\\
|\addtocounter{page}{-1}|\\
\textit{code for title page}\\
|\newpage|\\
|\||fi|
\end{tabular}
\end{center}
%
A banner page for the child documents can be generated by:
%
\begin{center}
\begin{tabular}{l}
|\ifchilddoc|\\
|\addtocounter{page}{-1}|\\
\textit{code for banner page}\\
|\newpage|\\
|\||fi|
\end{tabular}
\end{center}
%
Here one could write a message such as:
\begin{center}
|This is the part \childdocname{} of \childdocjob{}.|
\end{center}

%%%%%%%%%%%%%%%%%%%%%%%%%%%%%%%%%%%%%%%%%%%%%%%%%%%%%%%%%%%%%%%%%%%%%%%%%%%%%%%%
\subsection{Flags}
\label{sec:flags}

The package makes it easy to generate different versions
of the main or child documents.
To this end compilation flags can be defined
and assigned different default values.
They will be particularly useful in conjunction
with the forwarding mechanism described in \secref{sec:forward}.

For example, it may be useful to have a flag |\version|
which can be set to |draft| or |final|.
The document source will contain some conditional code
depending on the value of |\version|.
Suppose further, the flag should default to |final| for the main file
and to |draft| for child files
which is a natural assignment for editing the document.
This is achieved by placing the following code
in the preamble of the main document
(below the |\childdocmain| directive):
%
\begin{center}
\begin{tabular}{l}
|\ifchilddoc|\\
|\providecommand{\version}{draft}|\\
|\||else|\\
|\providecommand{\version}{final}|\\
|\||fi|
\end{tabular}
\end{center}
%
The definition by |\providecommand| makes sure
that previous definitions are not overwritten.
Further statements |\providecommand{\version}{...}|
can thus be added before the above code to override it.

For the main file, one might add a line
(between |\childdocmain| and the above block)
%
\begin{center}
|%\ifchilddoc\||else\providecommand{\version}{draft}\||fi|
\end{center}
%
which can be uncommented to produce a draft version.
Likewise one can add a line to the very top of a child file
(above the |\childdocof{|\textit{main}|}| directive)
%
\begin{center}
|%\providecommand{\version}{final}|
\end{center}
%
which can be uncommented to produce the final version of this child document.

%%%%%%%%%%%%%%%%%%%%%%%%%%%%%%%%%%%%%%%%%%%%%%%%%%%%%%%%%%%%%%%%%%%%%%%%%%%%%%%%
\subsection{Forwarding}
\label{sec:forward}

Different versions of the main or child documents
using compilation flags as described in \secref{sec:flags}
can be (permanently) stored in different files
for convenient compilation, viewing and distribution.
To this end, the package defines a command
to pass on compilation to a different file:

%%%%%%%%%%%%%%%%%%%%%%%%%%%%%%%%%%%%%%%%
\DescribeMacro{\childdocforward}
The command |\childdocforward| redirects processing to
another source file:
%
\begin{center}
\begin{tabular}{l}
|\input{childdoc.def}|\\
|\childdocforward[|\textit{main}|]{|\textit{dest}|}|\\
\end{tabular}
\end{center}
%
The argument \textit{dest} is the destination file
(without extension).
It should be the main file or one of the child files.
Note that further \textsf{childdoc} directives
such as |\childdocof| and |\childdocforward|
in the indicated file will be processed in this form.
The optional argument \textit{main}
passes on directly to the main file \textit{main}
while pretending to compile the child \textit{dest}.
This form behaves as if \textit{dest}
issues |\childdocof{|\textit{main}|}| right away,
and no further \textsf{childdoc} directives will be processed.

%%%%%%%%%%%%%%%%%%%%%%%%%%%%%%%%%%%%%%%%
\DescribeMacro{\...prefix}
In the alternative form |\childdocforwardprefix|,
%
\begin{center}
\begin{tabular}{l}
|\input{childdoc.def}|\\
|\childdocforwardprefix[|\textit{main}|]{|\textit{prefix}|}{|\textit{dest}|}|
\end{tabular}
\end{center}
%
the destination file is determined by a pattern
depending on the current file:
To make this work, the current file must be called
`{\textit{prefix}\hspace{0.2em}\textit{suffix}}'
with \textit{prefix} matching precisely the argument.
Processing is then passed on to the file
`{\textit{dest}\hspace{0.2em}\textit{suffix}}'.
Surely, the same effect is achieved by
directly specifying the
argument `{\textit{dest}\hspace{0.2em}\textit{suffix}}'
in the first form.
However, that requires to set up a different file
for each child. With the alternative form of the command
all these files can have exactly the same content
which simplifies setting them up and maintaining them.

For example, the following file |draft.tex|
with a compilation flag |\version| as described in \secref{sec:flags}
compiles the main document as a draft:
%
\begin{center}
\begin{tabular}{l}
|\def\version{draft}|\\
|\input{childdoc.def}|\\
|\childdocforward{|\textit{main}|}|
\end{tabular}
\end{center}
%
Likewise, the following files |final|\textit{nn}|.tex|
compile the final version of the child document
|child|\textit{nn}|.tex|:
%
\begin{center}
\begin{tabular}{l}
|\def\version{final}|\\
|\input{childdoc.def}|\\
|\childdocforwardprefix{final}{child}|
\end{tabular}
\end{center}
%

Note that when several versions of a main file and/or of each child file
are to be generated, it may be convenient to set up a |Makefile| or
shell script to automatise the process.

%%%%%%%%%%%%%%%%%%%%%%%%%%%%%%%%%%%%%%%%%%%%%%%%%%%%%%%%%%%%%%%%%%%%%%%%%%%%%%%%
\subsection{Command Line Processing}
\label{sec:commandline}

The effect of redirection files can also be achieved by invoking
the \LaTeX{} compiler with a more elaborate command line.
Most conveniently this should be done as part
of a shell script or a |Makefile|.

When using \textsf{childdoc} in the main file, the following
command lines effectively perform a redirection
(note that depending on the shell being used,
backslashes may have to be doubled: `|\|' $\to$ `|\\|'):
%
\begin{center}
|... -jobname "|\textit{target}|" |\\|"|[\textit{flags}]%
|\input{childdoc.def}\childdocforward[|\textit{main}|]{|\textit{dest}|}"|
\end{center}
%
Here \textit{target} is the name of the output file,
\textit{main} is the name of the main file
and \textit{dest} is the name of the main or child file to be processed
(all filenames without extensions).
The optional argument \textit{main} can be omitted
if \textit{main} matches \textit{dest}.
Optionally, compilation \textit{flags} can be defined via |\def| commands.
This command line makes the \TeX{} engine believe
it is compiling the file \textit{target}
whose content is specified as the latter parameter.
The provided code then forwards the processing to
\textit{main} or \textit{dest} as described in \secref{sec:forward}.

%%%%%%%%%%%%%%%%%%%%%%%%%%%%%%%%%%%%%%%%%%%%%%%%%%%%%%%%%%%%%%%%%%%%%%%%%%%%%%%%
\subsection{Include by Input}
\label{sec:input}

Including child documents by |\include| has some restrictions by design.
Most notably, the content of a child document always occupies
its own set of pages; pages cannot be shared between child documents.
Usually, this behaviour makes perfect sense
because each child document contain an essential part of the document.
However, in some situations it may be desirable to compose
a document from a collection of parts
without having mandatory page breaks between then.
For this case, the package
provides a mechanism to include parts
by |\input| which can also be processed individually.
However, by construction this mechanism
requires manual handling of the content to be output.

%%%%%%%%%%%%%%%%%%%%%%%%%%%%%%%%%%%%%%%%
\DescribeMacro{\ifchilddocmanual}
The main file should be prepared as usual, see \secref{sec:include}.
However, the document body must make a distinction
between processing of an individual part and of the main document, e.g.:
%
\begin{center}
\begin{tabular}{l}
|\ifchilddocmanual|\\
|\input{\childdocname}|\\
|\||else|\\
\textit{document body with }|\input{|\textit{part}|}|\\
|\||fi|
\end{tabular}
\end{center}
%
The conditional |\ifchilddocmanual| is true whenever
a part to be included by |\input| is being compiled,
and the name of the part is stored in |\childdocname|.

%%%%%%%%%%%%%%%%%%%%%%%%%%%%%%%%%%%%%%%%
\DescribeMacro{\childdocby}
Each part to be included by |\input| should start with:
%
\begin{center}
\begin{tabular}{l}
|\input{childdoc.def}|\\
|\childdocby{|\textit{main}|}|\\
\end{tabular}
\end{center}
%
The directive |\childdocby| is similar to |\childdocof|
described in \secref{sec:include},
but the subsequent selection of content must be done manually.
To that end, both |\ifchilddoc| and |\ifchilddocmanual|
will be true upon processing of a part,
and the name of the part is stored in |\childdocname|.
Note that |\jobname| will be set to the filename of the current part
so that each part receives an individual |.aux| file
that does not interfere with the |.aux| file(s) of the main document.
This behaviour can be altered by the alternative form
|\childdocby[*]{|\textit{main}|}| (with a non-empty optional argument)
which uses the |.aux| file of the main document
by setting |\jobname| to \textit{main}.

%%%%%%%%%%%%%%%%%%%%%%%%%%%%%%%%%%%%%%%%%%%%%%%%%%%%%%%%%%%%%%%%%%%%%%%%%%%%%%%%
\subsection{Driver Development}
\label{sec:driver}

The \textsf{childdoc} mechanism can also be use for the development
of definition files such as \LaTeX{} styles or classes.
This case differs from the above setup with multiple parts
included by |\include| in that no |\includeonly| should be invoked.
This can be achieved by starting the include file
(before |\ProvidesPackage|) with:
%
\begin{center}
\begin{tabular}{l}
|\input{childdoc.def}|\\
|\childdocforward{|\textit{main}|}|\\
\end{tabular}
\end{center}
%
or alternatively with:
%
\begin{center}
\begin{tabular}{l}
|\input{childdoc.def}|\\
|\childdocby{|\textit{main}|}|\\
\end{tabular}
\end{center}
%
Both forms have slightly different effects as described above.
The main file is prepared as usual, see \secref{sec:include}.

%%%%%%%%%%%%%%%%%%%%%%%%%%%%%%%%%%%%%%%%%%%%%%%%%%%%%%%%%%%%%%%%%%%%%%%%%%%%%%%%
\subsection{Legacy Detection}
\label{sec:detection}

The directive |\childdocmain| in the main file can detect
whether the complete document or merely a child is to be compiled
even without using the directive |\childdocof|.
This method is deprecated because it is less robust
and there is no compelling reason to use it;
it is merely provided for backward compatibility
and it may be removed in future versions.

If the detection mechanism is to be used,
it is mandatory to correctly specify
the filename of the main file as the argument of |\childdocmain|:
%
\begin{center}
\begin{tabular}{l}
|\input{childdoc.def}|\\
|\childdocmain{|\textit{main}|}|\\
\end{tabular}
\end{center}
%
If |\jobname| does not match the argument \textit{main} of |\childdocmain|,
it is assumed that |\jobname| points to the child file to be compiled.
When using |\childdocmain| with the main file specified as argument,
it suffices to start a child file
with just |\input{|\textit{main}|}|
without loading of the package and using |\childdocof|.
If instead all processing is done
with the appropriate \textsf{childdoc} directives,
the argument of \textit{main} of |\childdocmain| can be empty.

An alternative version of the command line processing described
in \secref{sec:commandline} using the detection mechanism reads:
%
\begin{center}
|... -jobname "|\textit{target}|" "|[\textit{flags}]%
[|\def\jobname{|\textit{dest}|}|]|\input{|\textit{main}|}"|
\end{center}

%%%%%%%%%%%%%%%%%%%%%%%%%%%%%%%%%%%%%%%%%%%%%%%%%%%%%%%%%%%%%%%%%%%%%%%%%%%%%%%%
\subsection{Manual Code}
\label{sec:manual}

In case one cannot be certain whether the definitions file |childdoc.def|
is installed on the target \TeX{} distribution
and one prefers not to ship it,
it is conceivable to paste a few relevant commands into the sources.

To that end, drop all statements |\input{childdoc.def}|
and perform the replacements as outlined below.
Instead of |\childdocmain{|\textit{main}|}| add the following code
to the top of the main file:
%
\begin{center}
\begin{tabular}{l}
|\||ifdefined\childdocname\endinput\||fi\newif\ifchilddoc|\\
|\edef\childdocname{\scantokens\expandafter{\jobname\noexpand}}|\\
|\def\childdocmain{|\textit{main}|}\||ifx\childdocmain\childdocname\||else|\\
|\childdoctrue\includeonly{\childdocname}\let\jobname\childdocmain\||fi|\\
\end{tabular}
\end{center}
%
Instead of |\childdocof{|\textit{main}|}| just include the main file
at the top of each child file:
%
\begin{center}
|\input{|\textit{main}|}|
\end{center}
%
A simple redirection |\childdocforward{|\textit{dest}|}| is achieved by:
%
\begin{center}
|\def\jobname{|\textit{dest}|}\input{\jobname}|
\end{center}
%
The redirection with prefix
|\childdocforwardprefix[|\textit{prefix}|]{|\textit{dest}|}|
is accomplished by:
%
\begin{center}
\begin{tabular}{l}
|{\edef\jobname{\scantokens\expandafter{\jobname\noexpand}}|\\
|\def\redirectjob |\textit{prefix}|#1~~~{\gdef\jobname{|\textit{dest}|#1}}|\\
|\expandafter\redirectjob\jobname~~~}\input{\jobname}|
\end{tabular}
\end{center}

In an alternative approach,
child documents can be compiled by a specific command line
without additional code or specific definitions:
%
\begin{center}
|... -jobname "|\textit{target}|" "|[\textit{flags}]%
|\includeonly{|\textit{dest}|}\input{|\textit{main}|}"|
\end{center}
%

%%%%%%%%%%%%%%%%%%%%%%%%%%%%%%%%%%%%%%%%%%%%%%%%%%%%%%%%%%%%%%%%%%%%%%%%%%%%%%%%
%%%%%%%%%%%%%%%%%%%%%%%%%%%%%%%%%%%%%%%%%%%%%%%%%%%%%%%%%%%%%%%%%%%%%%%%%%%%%%%%
\section{Information}

%%%%%%%%%%%%%%%%%%%%%%%%%%%%%%%%%%%%%%%%%%%%%%%%%%%%%%%%%%%%%%%%%%%%%%%%%%%%%%%%
\subsection{Copyright}

Copyright \copyright{} 2017--2018 Niklas Beisert

This work may be distributed and/or modified under the
conditions of the \LaTeX{} Project Public License, either version 1.3
of this license or (at your option) any later version.
The latest version of this license is in
  \url{http://www.latex-project.org/lppl.txt}
and version 1.3 or later is part of all distributions of \LaTeX{}
version 2005/12/01 or later.

This work has the LPPL maintenance status `maintained'.

The Current Maintainer of this work is Niklas Beisert.

This work consists of the files |README.txt|, |childdoc.ins| and |childdoc.dtx|
as well as the derived files |childdoc.def|, |cdocsamp.tex|
with |cdocsch1.tex|, |cdocsch2.tex|, |cdocspt3.tex|, |cdocspt4.tex|,
|cdocsdrf.tex|, |cdocsfn1.tex|, |cdocsfn2.tex|
as well as |childdoc.pdf|.

%%%%%%%%%%%%%%%%%%%%%%%%%%%%%%%%%%%%%%%%%%%%%%%%%%%%%%%%%%%%%%%%%%%%%%%%%%%%%%%%
\subsection{Files and Installation}

The package consists of the files:
%
\begin{center}
\begin{tabular}{ll}
    |README.txt|   & readme file \\
    |childdoc.ins| & installation file \\
    |childdoc.dtx| & source file \\
    |childdoc.def| & definition file \\
    |cdocsamp.tex| & sample main file \\
    |cdocsch1.tex| & sample include file \\
    |cdocsch2.tex| & sample include file \\
    |cdocspt3.tex| & sample part file \\
    |cdocspt4.tex| & sample part file \\
    |cdocsdrf.tex| & sample redirection file \\
    |cdocsfn1.tex| & sample redirection file \\
    |cdocsfn2.tex| & sample redirection file \\
    |childdoc.pdf| & manual
\end{tabular}
\end{center}
%
The distribution consists of the files
|README.txt|, |childdoc.ins| and |childdoc.dtx|.
%
\begin{itemize}
\item
Run (pdf)\LaTeX{} on |childdoc.dtx|
to compile the manual |childdoc.pdf| (this file).
\item
Run \LaTeX{} on |childdoc.ins| to create the definitions file |childdoc.def|
and the sample |cdocsamp.tex| with include files
|cdocsch1.tex|, |cdocsch2.tex|, |cdocspt3.tex|, |cdocspt4.tex|,
|cdocsdrf.tex|, |cdocsfn1.tex|, |cdocsfn2.tex|.
Then copy the file |childdoc.def| to an appropriate directory of your \LaTeX{}
distribution, e.g.\ \textit{texmf-root}|/tex/latex/childdoc|.
\end{itemize}

%%%%%%%%%%%%%%%%%%%%%%%%%%%%%%%%%%%%%%%%%%%%%%%%%%%%%%%%%%%%%%%%%%%%%%%%%%%%%%%%
\subsection{Related CTAN Packages}

There are several other packages which offer a similar functionality:
%
\begin{itemize}
\item
The packages
\href{http://ctan.org/pkg/docmute}{\textsf{docmute}},
\href{http://ctan.org/pkg/includex}{\textsf{includex}} and
\href{http://ctan.org/pkg/standalone}{\textsf{standalone}}
provide commands to include only the document body of
a child file thus allowing both files to be compiled individually.
\item
The packages \href{http://ctan.org/pkg/subdocs}{\textsf{subdocs}}
and \href{http://ctan.org/pkg/subfiles}{\textsf{subfiles}}
provide structures in which the main and child documents can be
encapsulated and allowing them to be compiled individually.
The inclusion mechanism is different from the conventional |\include|.
\item
The package \href{http://ctan.org/pkg/combine}{\textsf{combine}}
is an elaborate solution to combine several documents into one.
\end{itemize}
%
See also the CTAN topic \href{http://ctan.org/topic/subdocs}{\textsf{subdocs}}
for further related packages.
The present package differs from the above solutions in that
a document structure constructed with the conventional |\include| mechanism
just needs two extra commands at the top of every file
such that all constituent files can be compiled individually.

%%%%%%%%%%%%%%%%%%%%%%%%%%%%%%%%%%%%%%%%%%%%%%%%%%%%%%%%%%%%%%%%%%%%%%%%%%%%%%%%
%\subsection{Feature Suggestions}
%
%The following is a list of features which may be useful for future
%versions of this package:
%%
%\begin{itemize}
%\item
%\ldots
%\end{itemize}

%%%%%%%%%%%%%%%%%%%%%%%%%%%%%%%%%%%%%%%%%%%%%%%%%%%%%%%%%%%%%%%%%%%%%%%%%%%%%%%%
\subsection{Revision History}

%%%%%%%%%%%%%%%%%%%%%%%%%%%%%%%%%%%%%%%%
\paragraph{v2.0:} 2018/12/30

\begin{itemize}
\item
immediate forward processing
\item
added |\childdocby| mechanism
\item
manual restructured
\end{itemize}

%%%%%%%%%%%%%%%%%%%%%%%%%%%%%%%%%%%%%%%%
\paragraph{v1.6:} 2018/01/17

\begin{itemize}
\item
application for development of include files
\item
corrections to manual
\end{itemize}

%%%%%%%%%%%%%%%%%%%%%%%%%%%%%%%%%%%%%%%%
\paragraph{v1.5:} 2017/05/21

\begin{itemize}
\item
more complete structuring introduced
\item
|\childdocof| introduced
\item
|\childdoc| renamed to |\childdocmain|
\item
|\childredirect| renamed to |\childdocforward| and |\childdocforwardprefix|
and functionality expanded
\end{itemize}

%%%%%%%%%%%%%%%%%%%%%%%%%%%%%%%%%%%%%%%%
\paragraph{v1.0:} 2017/04/27

\begin{itemize}
\item
manual and install package
\item
first version published on CTAN
\end{itemize}

%%%%%%%%%%%%%%%%%%%%%%%%%%%%%%%%%%%%%%%%
\paragraph{v0.6:} 2017/04/26

\begin{itemize}
\item
redirection mechanism added
\end{itemize}

%%%%%%%%%%%%%%%%%%%%%%%%%%%%%%%%%%%%%%%%
\paragraph{v0.5:} 2017/04/26

\begin{itemize}
\item
functionality in definition file
\end{itemize}


%%%%%%%%%%%%%%%%%%%%%%%%%%%%%%%%%%%%%%%%%%%%%%%%%%%%%%%%%%%%%%%%%%%%%%%%%%%%%%%%
%%%%%%%%%%%%%%%%%%%%%%%%%%%%%%%%%%%%%%%%%%%%%%%%%%%%%%%%%%%%%%%%%%%%%%%%%%%%%%%%
%%%%%%%%%%%%%%%%%%%%%%%%%%%%%%%%%%%%%%%%%%%%%%%%%%%%%%%%%%%%%%%%%%%%%%%%%%%%%%%%
\appendix

\settowidth\MacroIndent{\rmfamily\scriptsize 000\ }

 \DocInput{childdoc.dtx}

\end{document}
%</driver>
% \fi
%
% %%%%%%%%%%%%%%%%%%%%%%%%%%%%%%%%%%%%%%%%%%%%%%%%%%%%%%%%%%%%%%%%%%%%%%%%%%%%%%
% %%%%%%%%%%%%%%%%%%%%%%%%%%%%%%%%%%%%%%%%%%%%%%%%%%%%%%%%%%%%%%%%%%%%%%%%%%%%%%
% \section{Sample}
%\iffalse
%<*samplemain>
%\fi
%
% The following presents a sample document
% with two chapters, two parts, a title page,
% a compile flag as well as three forwarding files to set the flag.
% It consists of eight |.tex| files:
% \begin{center}
% \begin{tabular}{ll}
% |cdocsamp.tex|&main file\\
% |cdocsch1.tex|&include file for chapter 1\\
% |cdocsch2.tex|&include file for chapter 2\\
% |cdocspt3.tex|&include file for part 3\\
% |cdocspt4.tex|&include file for part 4\\
% |cdocsdrf.tex|&forwarding file for main file in draft mode\\
% |cdocsfi1.tex|&forwarding file for final version of chapter 1\\
% |cdocsfi2.tex|&forwarding file for final version of chapter 2\\
% \end{tabular}
% \end{center}
% Each of the eight files can be compiled directly by the \LaTeX{} compiler.
%
% %%%%%%%%%%%%%%%%%%%%%%%%%%%%%%%%%%%%%%
% \paragraph{Main File.}
%
% The main file is called |cdocsamp.tex|.
%
% Load the \textsf{childdoc} definitions and
% declare the filename for the main document:
%    \begin{macrocode}
\input{childdoc.def}
\childdocmain{}
%    \end{macrocode}

% Optional override for |\version| flag:
%    \begin{macrocode}
%%\ifchilddoc\else\providecommand{\version}{draft}\fi
%    \end{macrocode}

% Define the default values for the |\version| flag
% (|final| for the main file and |draft| for childs):
%    \begin{macrocode}
\ifchilddoc
\providecommand{\version}{draft}
\else
\providecommand{\version}{final}
\fi
%    \end{macrocode}

% Load the standard document class:
%    \begin{macrocode}
\documentclass[12pt]{article}
%    \end{macrocode}

% Start the document body:
%    \begin{macrocode}
\begin{document}
%    \end{macrocode}

% Declare a title page.
% Print title, part of document being processed and version flag:
%    \begin{macrocode}
\addtocounter{page}{-1}
\begin{center}
{\LARGE\bfseries{}childdoc example\par}
\vspace{1cm}
\ifchilddoc
\ifchilddocmanual part\else chapter\fi:
`\childdocname' of `\childdocjob'\par
\else
main document: `\childdocjob'\par
\fi
version: \version\par
\end{center}
\newpage
%    \end{macrocode}

% Manually include selected file,
% otherwise process as usual:
%    \begin{macrocode}
\ifchilddocmanual
\section*{part `\childdocname'}
\input{\childdocname}
\else
%    \end{macrocode}

% Include the two chapters:
%    \begin{macrocode}
\include{cdocsch1}
\include{cdocsch2}
%    \end{macrocode}

% Include the two parts unless only chapters should be displayed:
%    \begin{macrocode}
\ifchilddoc\else
\section{part three}
\input{cdocspt3}
\section{part four}
\input{cdocspt4}
\fi
%    \end{macrocode}

% Process as usual until here:
%    \begin{macrocode}
\fi
%    \end{macrocode}

% End of document body:
%    \begin{macrocode}
\end{document}
%    \end{macrocode}
%\iffalse
%</samplemain>
%\fi
%
% %%%%%%%%%%%%%%%%%%%%%%%%%%%%%%%%%%%%%%
% \paragraph{Chapter Include Files.}
%
% The include files are called |cdocsch1.tex| and |cdocsch2.tex|.
%
%\iffalse
%<*samplechap1|samplechap2>
%\fi

% Optional override for |\version| flag:
%    \begin{macrocode}
%%\providecommand{\version}{final}
%    \end{macrocode}

% Include the main document:
%    \begin{macrocode}
\input{childdoc.def}
\childdocof{cdocsamp}
%    \end{macrocode}

%\iffalse
%</samplechap1|samplechap2>
%\fi
%
%\iffalse
%<*samplechap1>
%\fi
% Some text for chapter 1:
%    \begin{macrocode}
\section{one}
some text in chapter one
%    \end{macrocode}

%\iffalse
%</samplechap1>
%\fi
% Some text for chapter 2:
%\iffalse
%<*samplechap2>
%\fi
%    \begin{macrocode}
\section{two}
more text in chapter two
%    \end{macrocode}

%\iffalse
%</samplechap2>
%\fi
%
% %%%%%%%%%%%%%%%%%%%%%%%%%%%%%%%%%%%%%%
% \paragraph{Part Include Files.}
%
% The include files are called |cdocspt3.tex| and |cdocspt4.tex|.
%
%\iffalse
%<*samplepart3|samplepart4>
%\fi

% Optional override for |\version| flag:
%    \begin{macrocode}
%%\providecommand{\version}{final}
%    \end{macrocode}

% Include the main document:
%    \begin{macrocode}
\input{childdoc.def}
\childdocby{cdocsamp}
%    \end{macrocode}

%\iffalse
%</samplepart3|samplepart4>
%\fi
%
%\iffalse
%<*samplepart3>
%\fi
% Some text for part 3:
%    \begin{macrocode}
some text in part three
%    \end{macrocode}

%\iffalse
%</samplepart3>
%\fi
% Some text for part 4:
%\iffalse
%<*samplepart4>
%\fi
%    \begin{macrocode}
more text in part four
%    \end{macrocode}

%\iffalse
%</samplepart4>
%\fi
%
% %%%%%%%%%%%%%%%%%%%%%%%%%%%%%%%%%%%%%%
% \paragraph{Forwarding for a Complete Draft.}
%
% The following forwarding file |cdocsdrf.tex|
% compiles the main document in draft mode:
%\iffalse
%<*sampledraft>
%\fi
%    \begin{macrocode}
\def\version{draft}
\input{childdoc.def}
\childdocforward{cdocsamp}
%    \end{macrocode}

%\iffalse
%</sampledraft>
%\fi
%
% %%%%%%%%%%%%%%%%%%%%%%%%%%%%%%%%%%%%%%
% \paragraph{Forwarding for Final Version of the Chapters.}
%
% The following forwarding files |cdocsfn1.tex| and |cdocsfn2.tex|
% (with identical content)
% compile the final versions of the child documents
% |cdocsch1.tex| and |cdocsch2.tex|, respectively:
%\iffalse
%<*samplefinal>
%\fi
%    \begin{macrocode}
\def\version{final}
\input{childdoc.def}
\childdocforwardprefix[cdocsamp]{cdocsfn}{cdocsch}
%    \end{macrocode}

%\iffalse
%</samplefinal>
%\fi
%
% %%%%%%%%%%%%%%%%%%%%%%%%%%%%%%%%%%%%%%
% \paragraph{Command Line Processing.}
%
% The following three command lines generate the output files
% |cdocscld|, |cdocscl1| and |cdocscl2|
% which should be identical to
% |cdocsdrf|, |cdocsch1| and |cdocsfn2|, respectively:
% \begin{center}
% \begin{tabular}{l}
% |latex -jobname cdocscld \|\\
% |  "\def\version{draft}\input{childdoc.def}\childdocforward{cdocsamp}"|\\
% |latex -jobname cdocscl1 \|\\
% |  "\input{childdoc.def}\childdocforward[cdocsamp]{cdocsch1}"|\\
% |latex -jobname cdocscl2 \|\\
% |  "\def\version{final}\input{childdoc.def}\childdocforward{cdocsch2}"|
% \end{tabular}
% \end{center}
% Note that the trailing backslash on each first line
% merely continues the input to the second line
% (for convenient cut ant paste).
% Furthermore, the command |latex| can be replaced by any
% of its alternative versions such as |pdflatex|.
%
% %%%%%%%%%%%%%%%%%%%%%%%%%%%%%%%%%%%%%%%%%%%%%%%%%%%%%%%%%%%%%%%%%%%%%%%%%%%%%%
% %%%%%%%%%%%%%%%%%%%%%%%%%%%%%%%%%%%%%%%%%%%%%%%%%%%%%%%%%%%%%%%%%%%%%%%%%%%%%%
% \section{Implementation}
%\iffalse
%<*package>
%\fi
%
% This section describes the definitions file |childdoc.def|.

% The definitions cannot be loaded using |\usepackage| or |\RequirePackage|
% which has a mechanism to prevent loading a style file more than once.
% When loading the definitions by means of |\input|
% multiple instances have to be prevented manually:
%\iffalse
%This code needs to be before the `\ProvidesFile' directive
%which is defined at the beginning of this file.
%Therefore it is also placed there and commented out here.
%</package>
%<*discard>
%\fi
%    \begin{macrocode}
\ifdefined\childdocmain\endinput\fi
%    \end{macrocode}
%\iffalse
%</discard>
%<*package>
%\fi
%
% \macro{\ifchilddoc}
% \macro{\ifchilddocmanual}
% The conditional |\ifchilddoc| tells whether a
% child (true) or main (false) document is being compiled.
% The conditional |\ifchilddocmanual| tells whether
% the |\includeonly| mechanism is used (false) or
% the selection of child files must be performed manually (true).
% The definitions initialise to false:
%    \begin{macrocode}
\newif\ifchilddoc
\newif\ifchilddocmanual
%    \end{macrocode}

% \macro{\childdocname}
% \macro{\childdocjob}
% The macro |\childdocname| stores the name of the main document
% to be compiled. The macro |\childdocjob| stores the name of
% the document on which the \LaTeX{} compiler was originally invoked.
% The content of |\jobname| cannot be compared
% to filenames specified in the source due to different catcodes.
% The following code rescans |\jobname|, stores the result
% in |\childdocname| and saves a copy in |\childdocjob|:
%    \begin{macrocode}
\edef\childdocname{\scantokens\expandafter{\jobname\noexpand}}
\let\childdocjob\childdocname
%    \end{macrocode}

% \macro{\childdocdisable}
% The macro |\childdocdisable| prevents the main file
% from being processed more than once.
% At this stage, the main document command |\childdocmain|
% is assumed to be called once again where it should do nothing.
% Any subsequent call to it should prevent
% a secondary processing of the main document
% It overwrites the forwarding commands
% |\childdocof| and |\childdocforward|
% with empty macros to prevent further inclusions of the main document:
%    \begin{macrocode}
\newcommand{\childdocdisable}
{
  \renewcommand{\childdocmain}[1]{\renewcommand{\childdocmain}[1]{\endinput}}
  \renewcommand{\childdocof}[1]{}
  \renewcommand{\childdocby}[2][]{}
  \renewcommand{\childdocforward}[2][]{}
  \renewcommand{\childdocdisable}{}
}
%    \end{macrocode}

% \macro{\childdocmain}
% The macro |\childdocmain| is to be called at the top of the main file
% with nothing or the main filename (without extension) as argument.
% First, it breaks loops.
% If the argument is not empty and does not match |\childdocname|
% (which is set by the first inclusion of |childdoc.def|),
% |\ifchilddoc| is set to true, |\includeonly| is applied to the child file
% and |\jobname| is set to the main file
% (for proper handling of |.aux| files):
%    \begin{macrocode}
\newcommand{\childdocmain}[1]
{
  \childdocdisable\childdocmain{}
  \if?#1?\else
    \begingroup
      \def\childdoctmp{#1}
      \ifx\childdoctmp\childdocname
        \def\childdoctmp{}
      \else
        \def\childdoctmp
        {
          \childdoctrue
          \includeonly{\childdocname}
          \def\childdocjob{#1}
          \def\jobname{#1}
        }
      \fi
      \expandafter
    \endgroup
    \childdoctmp
  \fi
}
%    \end{macrocode}

% \macro{\childdocof}
% The command |\childdocof| redirects
% compilation to the main file |#1|.
%    \begin{macrocode}
\newcommand{\childdocof}[1]
{
  \childdocdisable
  \childdoctrue
  \includeonly{\childdocname}
  \def\jobname{#1}
  \def\childdocjob{#1}
  \input{#1}
}
%    \end{macrocode}

% \macro{\childdocby}
% The command |\childdocby| ....
%    \begin{macrocode}
\newcommand{\childdocby}[2][]
{
  \childdocdisable
  \childdoctrue
  \childdocmanualtrue
  \if?#1?\else
    \def\jobname{#2}
  \fi
  \def\childdocjob{#2}
  \input{#2}
  \endinput
}
%    \end{macrocode}

% \macro{\childdocforward}
% The command |\childdocforward| redirects
% compilation to the main file or
% (if the optional argument is given) a child file.
% Parameters are set as if the main file
% or a child file starting with |\childdocof| was compiled.
% Then compilation is handed over to the main file:
%    \begin{macrocode}
\newcommand{\childdocforward}[2][]
{
  \begingroup
    \if?#1?
      \def\childdoctmp
      {
        \def\childdocname{#2}
        \def\childdocjob{#2}
        \def\jobname{#2}
        \input{#2}
        \endinput
      }
    \else
      \def\childdoctmp
      {
        \childdocdisable
        \def\childdocname{#2}
        \childdoctrue
        \includeonly{#2}
        \def\childdocjob{#1}
        \def\jobname{#1}
        \input{#1}
        \endinput
      }
    \fi
    \expandafter
  \endgroup
  \childdoctmp
}
%    \end{macrocode}

% \macro{\childdocforwardprefix}
% The command |\childdocforwardprefix| redirects
% compilation to the main or a child file by means of a pattern.
% The prefix |#1| in the current filename is replaced by |#2|
% and the suffix of the current filename is kept
% (it is assumed that the filename does not contain the substring `|~~~|'
% which is used as a delimiter).
% Compilation is handed over to the new file by |\childdocforward|:
%    \begin{macrocode}
\newcommand{\childdocforwardprefix}[3][]
{
  \begingroup
    \def\childdocextract #2##1~~~{\def\childdoctmp{\childdocforward[#1]{#3##1}}}
    \expandafter\childdocextract\childdocname~~~
    \expandafter
  \endgroup
  \childdoctmp
}
%    \end{macrocode}

% \macro{\childdoc}
% The deprecated macro |\childdoc| is a legacy version of |\childdocmain|:
%    \begin{macrocode}
\newcommand{\childdoc}{\childdocmain}
%    \end{macrocode}

% \macro{\childdocredirect}
% The deprecated macro |\childdocredirect| is a legacy version
% of |\childdocforward| and |\childdocforwardprefix|:
%    \begin{macrocode}
\newcommand{\childdocredirect}[2][]
{
  \begingroup
    \if?#1?
      \def\childdoctmp{\childdocforward{#2}}
    \else
      \def\childdoctmp{\childdocforwardprefix{#1}{#2}}
    \fi
    \expandafter
  \endgroup
  \childdoctmp
}
%    \end{macrocode}

%\iffalse
%</package>
%\fi
%
\endinput
|\\
|\childdocof{|\textit{main}|}|\\
\end{tabular}
\end{center}
at the top of every child file \textit{child}
which is included by |\include{|\textit{child}|}|
from within the main file
(or at least for those files to be compiled individually).
The argument \textit{main} must be the filename of the main file.

There are a couple of
considerations in setting up the main and child documents:

%%%%%%%%%%%%%%%%%%%%%%%%%%%%%%%%%%%%%%%%
\paragraph{Restrictions.}

Please note the following restrictions:
\begin{itemize}
\item
|\childdocmain| must be called with one argument \textit{main}
to ensure compatibility with earlier version of the package.
It must either be empty (|\childdocmain{}|)
or precisely match the filename of the main file in which it is specified.
See \secref{sec:detection} for further information.
\item
The filename \textit{main} must be specified without the |.tex| extension.
\item
The filename \textit{main} is case sensitive
(even in case-insensitive file systems)
due to internal string comparison.
\item
The argument \textit{main} should be fully expanded, it cannot be a macro.
\item
Subdirectories and special characters should be avoided in filenames.
\item
The command |\childdocmain{|\textit{main}|}| must be followed by a whitespace.
It should not be followed immediately by another command
or by a comment mark `|%|'.
This is because the \TeX{} parser reads the token immediately following
the argument of |\childdocmain| and puts it
at the beginning of every child section;
however, a white\-space is ignored.
\end{itemize}

%%%%%%%%%%%%%%%%%%%%%%%%%%%%%%%%%%%%%%%%
\paragraph{Content of Main File.}

It is advisable to place all content in the child files included by |\include|.
Any output contained in the main file will appear in all child documents
unless suppressed manually;
it cannot be suppressed automatically by the |\includeonly| directive
and thus should normally be avoided.
A method to include some content in the main file
by means of conditional processing is described in \secref{sec:conditional}.

%%%%%%%%%%%%%%%%%%%%%%%%%%%%%%%%%%%%%%%%
\paragraph{Page Numbering.}

When only a part of the document is compiled,
the appropriate numbering of pages
(as well as other status parameters)
is determined from the |.aux| files.
The latter contain information from previous passes.
However this information needs to propagate through
all intermediate child documents.
Therefore the page numbering in child documents may well
be inconsistent until the complete document is compiled at least once.

A useful (if unconventional) way to always ensure a consistent
page numbering is to restart the numbering in each child document
and denote the pages by `\textit{child}|.|\textit{page}'
where \textit{child} represents the chapter/section number of the child file.
This can be achieved by the command
|\numberwithin{page}{|\textit{child}|}|
of the \textsf{amsmath} package
where \textit{child} can be |chapter| or |section|
depending on the chosen structuring.
Alternatively, one can modify the macro |\thepage| appropriately
and reset the counter |page| at the start of each child file.

%%%%%%%%%%%%%%%%%%%%%%%%%%%%%%%%%%%%%%%%%%%%%%%%%%%%%%%%%%%%%%%%%%%%%%%%%%%%%%%%
\subsection{Conditional Processing}
\label{sec:conditional}

The package provides a mechanism to compile different versions
of a document. To customise the versions further some conditional processing
can come in handy to distinguish which version is being compiled.
The package provides two macros to describe the compilation context:

%%%%%%%%%%%%%%%%%%%%%%%%%%%%%%%%%%%%%%%%
\DescribeMacro{\ifchilddoc}
The conditional |\ifchilddoc| distinguishes between the compilation of
child documents and the main document:
%
\begin{center}
|\ifchilddoc |\textit{child-code}| |[|\||else |\textit{main-code}]| \||fi|
\end{center}

%%%%%%%%%%%%%%%%%%%%%%%%%%%%%%%%%%%%%%%%
\DescribeMacro{\childdocname}
\DescribeMacro{\childdocjob}
The macro |\childdocname| contains the filename (without extension)
of the main or child file being processed.
Note that |\childdocjob| will always contain the name of the main file.

%%%%%%%%%%%%%%%%%%%%%%%%%%%%%%%%%%%%%%%%
\paragraph{Title Page.}

Conditional processing can be used to include a title or banner page
in the main document when proper precautions are taken.
Importantly, the code in the main file should ensure that the page counter
(as well as other status parameters which are stored in the |.aux| files)
takes the same value after the conditional processing.
Otherwise the page numbers may take divergent values
depending on which part is compiled.

For example, a title page could be declared by:
%
\begin{center}
\begin{tabular}{l}
|\ifchilddoc\||else|\\
|\addtocounter{page}{-1}|\\
\textit{code for title page}\\
|\newpage|\\
|\||fi|
\end{tabular}
\end{center}
%
A banner page for the child documents can be generated by:
%
\begin{center}
\begin{tabular}{l}
|\ifchilddoc|\\
|\addtocounter{page}{-1}|\\
\textit{code for banner page}\\
|\newpage|\\
|\||fi|
\end{tabular}
\end{center}
%
Here one could write a message such as:
\begin{center}
|This is the part \childdocname{} of \childdocjob{}.|
\end{center}

%%%%%%%%%%%%%%%%%%%%%%%%%%%%%%%%%%%%%%%%%%%%%%%%%%%%%%%%%%%%%%%%%%%%%%%%%%%%%%%%
\subsection{Flags}
\label{sec:flags}

The package makes it easy to generate different versions
of the main or child documents.
To this end compilation flags can be defined
and assigned different default values.
They will be particularly useful in conjunction
with the forwarding mechanism described in \secref{sec:forward}.

For example, it may be useful to have a flag |\version|
which can be set to |draft| or |final|.
The document source will contain some conditional code
depending on the value of |\version|.
Suppose further, the flag should default to |final| for the main file
and to |draft| for child files
which is a natural assignment for editing the document.
This is achieved by placing the following code
in the preamble of the main document
(below the |\childdocmain| directive):
%
\begin{center}
\begin{tabular}{l}
|\ifchilddoc|\\
|\providecommand{\version}{draft}|\\
|\||else|\\
|\providecommand{\version}{final}|\\
|\||fi|
\end{tabular}
\end{center}
%
The definition by |\providecommand| makes sure
that previous definitions are not overwritten.
Further statements |\providecommand{\version}{...}|
can thus be added before the above code to override it.

For the main file, one might add a line
(between |\childdocmain| and the above block)
%
\begin{center}
|%\ifchilddoc\||else\providecommand{\version}{draft}\||fi|
\end{center}
%
which can be uncommented to produce a draft version.
Likewise one can add a line to the very top of a child file
(above the |\childdocof{|\textit{main}|}| directive)
%
\begin{center}
|%\providecommand{\version}{final}|
\end{center}
%
which can be uncommented to produce the final version of this child document.

%%%%%%%%%%%%%%%%%%%%%%%%%%%%%%%%%%%%%%%%%%%%%%%%%%%%%%%%%%%%%%%%%%%%%%%%%%%%%%%%
\subsection{Forwarding}
\label{sec:forward}

Different versions of the main or child documents
using compilation flags as described in \secref{sec:flags}
can be (permanently) stored in different files
for convenient compilation, viewing and distribution.
To this end, the package defines a command
to pass on compilation to a different file:

%%%%%%%%%%%%%%%%%%%%%%%%%%%%%%%%%%%%%%%%
\DescribeMacro{\childdocforward}
The command |\childdocforward| redirects processing to
another source file:
%
\begin{center}
\begin{tabular}{l}
|% \iffalse
%
% childdoc.dtx Copyright (C) 2017-2018 Niklas Beisert
%
% This work may be distributed and/or modified under the
% conditions of the LaTeX Project Public License, either version 1.3
% of this license or (at your option) any later version.
% The latest version of this license is in
%   http://www.latex-project.org/lppl.txt
% and version 1.3 or later is part of all distributions of LaTeX
% version 2005/12/01 or later.
%
% This work has the LPPL maintenance status `maintained'.
%
% The Current Maintainer of this work is Niklas Beisert.
%
% This work consists of the files childdoc.dtx and childdoc.ins
% and the derived files childdoc.def and cdocsamp.tex with
% cdocsch1.tex, cdocsch2.tex, cdocsdrf.tex, cdocsfn1.tex, cdocsfn2.tex.
%
%<package>\ifdefined\childdocmain\endinput\fi
%<package>\ProvidesFile{childdoc.def}[2018/12/30 v2.0 child document driver]
%<samplemain>\ProvidesFile{cdocsamp.tex}[2018/12/30 v2.0 sample for childdoc]
%<*driver>
%\ProvidesFile{childdoc.drv}[2018/12/30 v2.0 childdoc reference manual file]
\PassOptionsToClass{10pt,a4paper}{article}
\documentclass{ltxdoc}

\usepackage[margin=35mm]{geometry}
\usepackage{hyperref}
\usepackage{hyperxmp}
\usepackage[usenames]{color}

\hypersetup{colorlinks=true}
\hypersetup{pdfstartview=FitH}
\hypersetup{pdfpagemode=UseNone}
\hypersetup{pdfsource={}}
\hypersetup{pdflang={en-UK}}
\hypersetup{pdfcopyright={Copyright 2017-2018 Niklas Beisert.
  This work may be distributed and/or modified under the
  conditions of the LaTeX Project Public License, either version 1.3
  of this license or (at your option) any later version.}}
\hypersetup{pdflicenseurl={http://www.latex-project.org/lppl.txt}}
\hypersetup{pdfcontactaddress={ETH Zurich, ITP, HIT K,
  Wolfgang-Pauli-Strasse 27}}
\hypersetup{pdfcontactpostcode={8093}}
\hypersetup{pdfcontactcity={Zurich}}
\hypersetup{pdfcontactcountry={Switzerland}}
\hypersetup{pdfcontactemail={nbeisert@itp.phys.ethz.ch}}
\hypersetup{pdfcontacturl={http://people.phys.ethz.ch/\xmptilde nbeisert/}}

\newcommand{\secref}[1]{\hyperref[#1]{section \ref*{#1}}}

\parskip1ex
\parindent0pt
\let\olditemize\itemize
\def\itemize{\olditemize\parskip0pt}

\begin{document}

\title{The \textsf{childdoc} Package}
\hypersetup{pdftitle={The childdoc Package}}
\author{Niklas Beisert\\[2ex]
  Institut f\"ur Theoretische Physik\\
  Eidgen\"ossische Technische Hochschule Z\"urich\\
  Wolfgang-Pauli-Strasse 27, 8093 Z\"urich, Switzerland\\[1ex]
  \href{mailto:nbeisert@itp.phys.ethz.ch}
  {\texttt{nbeisert@itp.phys.ethz.ch}}}
\hypersetup{pdfauthor={Niklas Beisert}}
\hypersetup{pdfsubject={Manual for the LaTeX2e Package childdoc}}
\date{30 December 2018, \textsf{v2.0}}
\maketitle

\begin{abstract}\noindent
\textsf{childdoc} is a \LaTeXe{} package
that enables the direct compilation
of document sections included by |\include|
to individual files.
\end{abstract}

\begingroup
\parskip0ex
\tableofcontents
\endgroup

%%%%%%%%%%%%%%%%%%%%%%%%%%%%%%%%%%%%%%%%%%%%%%%%%%%%%%%%%%%%%%%%%%%%%%%%%%%%%%%%
%%%%%%%%%%%%%%%%%%%%%%%%%%%%%%%%%%%%%%%%%%%%%%%%%%%%%%%%%%%%%%%%%%%%%%%%%%%%%%%%
\section{Introduction}

\LaTeX{} provides a mechanism to structure a large document (such as a book)
into a main file and several child files (containing the chapters)
using the |\include| command.
This mechanism is beneficial for documents
which span hundreds of pages in order to
make the source file(s) more manageable.
Moreover, compilation can be restricted to
selected child files by means of the |\includeonly| command.
The latter feature can be used to reduce the compilation time while editing
(this was significantly more useful in the earlier days of \LaTeX{})
or to generate a smaller document which is easier to navigate.
Another application of |\includeonly| is to generate
documents consisting of selected parts of the complete document.

However, there are a few drawbacks of the plain |\include| mechanism:
\begin{itemize}
\item
The child files cannot be compiled on their own,
they can only be compiled via the main file.
A naive editing environment
(such as a text editor with an option
to have the current file processed by \LaTeX)
may require one to switch to the main file before compiling;
attempting to compile the child file produces errors.
\item
The main file must be modified (each time)
to adjust the |\includeonly| command
to the present needs. This easily leaves the main file in a messy state.
\item
The generated document will always carry the filename
of the main document. This is inconvenient if
several child files are to be compiled and
to be kept for distribution.
\end{itemize}

The present package provides a simple interface
to make child files individually compilable by \LaTeX{}.
Compiling a child file then has the same effect as compiling
the main file with an |\includeonly| command
to select the appropriate child.
Moreover the generated document will carry the name of the child
rather than the main file.
This resolves all three above issues.

This feature is meant to make the editing of books,
thesis documents and lecture notes somewhat more convenient.
However, the package can also be used efficiently for
composing a series of documents (such as exercise sheets)
which are typically distributed individually.
It then assists the author in generating the individual documents
(potentially in different versions)
as well as a document containing the collected series.
Another application is in developing style files
or other kinds of included material
where compilation of the style file could redirect
to a sample or test file.

%%%%%%%%%%%%%%%%%%%%%%%%%%%%%%%%%%%%%%%%%%%%%%%%%%%%%%%%%%%%%%%%%%%%%%%%%%%%%%%%
%%%%%%%%%%%%%%%%%%%%%%%%%%%%%%%%%%%%%%%%%%%%%%%%%%%%%%%%%%%%%%%%%%%%%%%%%%%%%%%%
\section{Usage}

First of all, the package \textsf{childdoc} is \emph{not} a standard
\LaTeXe{} |.sty| style file! Therefore it needs to be invoked in
a non-standard way.

%%%%%%%%%%%%%%%%%%%%%%%%%%%%%%%%%%%%%%%%%%%%%%%%%%%%%%%%%%%%%%%%%%%%%%%%%%%%%%%%
\subsection{Included Files}
\label{sec:include}

%%%%%%%%%%%%%%%%%%%%%%%%%%%%%%%%%%%%%%%%
\DescribeMacro{\childdocmain}
To use the package, add the commands
\begin{center}
\begin{tabular}{l}
|\input{childdoc.def}|\\
|\childdocmain{}|\\
\end{tabular}
\end{center}
at the very top of the main \LaTeX{} file,
in particular \emph{before} the |\documentclass| statement!
The argument of |\childdocmain| should be left empty
(but it must be present).

%%%%%%%%%%%%%%%%%%%%%%%%%%%%%%%%%%%%%%%%
\DescribeMacro{\childdocof}
Furthermore, add the commands
\begin{center}
\begin{tabular}{l}
|\input{childdoc.def}|\\
|\childdocof{|\textit{main}|}|\\
\end{tabular}
\end{center}
at the top of every child file \textit{child}
which is included by |\include{|\textit{child}|}|
from within the main file
(or at least for those files to be compiled individually).
The argument \textit{main} must be the filename of the main file.

There are a couple of
considerations in setting up the main and child documents:

%%%%%%%%%%%%%%%%%%%%%%%%%%%%%%%%%%%%%%%%
\paragraph{Restrictions.}

Please note the following restrictions:
\begin{itemize}
\item
|\childdocmain| must be called with one argument \textit{main}
to ensure compatibility with earlier version of the package.
It must either be empty (|\childdocmain{}|)
or precisely match the filename of the main file in which it is specified.
See \secref{sec:detection} for further information.
\item
The filename \textit{main} must be specified without the |.tex| extension.
\item
The filename \textit{main} is case sensitive
(even in case-insensitive file systems)
due to internal string comparison.
\item
The argument \textit{main} should be fully expanded, it cannot be a macro.
\item
Subdirectories and special characters should be avoided in filenames.
\item
The command |\childdocmain{|\textit{main}|}| must be followed by a whitespace.
It should not be followed immediately by another command
or by a comment mark `|%|'.
This is because the \TeX{} parser reads the token immediately following
the argument of |\childdocmain| and puts it
at the beginning of every child section;
however, a white\-space is ignored.
\end{itemize}

%%%%%%%%%%%%%%%%%%%%%%%%%%%%%%%%%%%%%%%%
\paragraph{Content of Main File.}

It is advisable to place all content in the child files included by |\include|.
Any output contained in the main file will appear in all child documents
unless suppressed manually;
it cannot be suppressed automatically by the |\includeonly| directive
and thus should normally be avoided.
A method to include some content in the main file
by means of conditional processing is described in \secref{sec:conditional}.

%%%%%%%%%%%%%%%%%%%%%%%%%%%%%%%%%%%%%%%%
\paragraph{Page Numbering.}

When only a part of the document is compiled,
the appropriate numbering of pages
(as well as other status parameters)
is determined from the |.aux| files.
The latter contain information from previous passes.
However this information needs to propagate through
all intermediate child documents.
Therefore the page numbering in child documents may well
be inconsistent until the complete document is compiled at least once.

A useful (if unconventional) way to always ensure a consistent
page numbering is to restart the numbering in each child document
and denote the pages by `\textit{child}|.|\textit{page}'
where \textit{child} represents the chapter/section number of the child file.
This can be achieved by the command
|\numberwithin{page}{|\textit{child}|}|
of the \textsf{amsmath} package
where \textit{child} can be |chapter| or |section|
depending on the chosen structuring.
Alternatively, one can modify the macro |\thepage| appropriately
and reset the counter |page| at the start of each child file.

%%%%%%%%%%%%%%%%%%%%%%%%%%%%%%%%%%%%%%%%%%%%%%%%%%%%%%%%%%%%%%%%%%%%%%%%%%%%%%%%
\subsection{Conditional Processing}
\label{sec:conditional}

The package provides a mechanism to compile different versions
of a document. To customise the versions further some conditional processing
can come in handy to distinguish which version is being compiled.
The package provides two macros to describe the compilation context:

%%%%%%%%%%%%%%%%%%%%%%%%%%%%%%%%%%%%%%%%
\DescribeMacro{\ifchilddoc}
The conditional |\ifchilddoc| distinguishes between the compilation of
child documents and the main document:
%
\begin{center}
|\ifchilddoc |\textit{child-code}| |[|\||else |\textit{main-code}]| \||fi|
\end{center}

%%%%%%%%%%%%%%%%%%%%%%%%%%%%%%%%%%%%%%%%
\DescribeMacro{\childdocname}
\DescribeMacro{\childdocjob}
The macro |\childdocname| contains the filename (without extension)
of the main or child file being processed.
Note that |\childdocjob| will always contain the name of the main file.

%%%%%%%%%%%%%%%%%%%%%%%%%%%%%%%%%%%%%%%%
\paragraph{Title Page.}

Conditional processing can be used to include a title or banner page
in the main document when proper precautions are taken.
Importantly, the code in the main file should ensure that the page counter
(as well as other status parameters which are stored in the |.aux| files)
takes the same value after the conditional processing.
Otherwise the page numbers may take divergent values
depending on which part is compiled.

For example, a title page could be declared by:
%
\begin{center}
\begin{tabular}{l}
|\ifchilddoc\||else|\\
|\addtocounter{page}{-1}|\\
\textit{code for title page}\\
|\newpage|\\
|\||fi|
\end{tabular}
\end{center}
%
A banner page for the child documents can be generated by:
%
\begin{center}
\begin{tabular}{l}
|\ifchilddoc|\\
|\addtocounter{page}{-1}|\\
\textit{code for banner page}\\
|\newpage|\\
|\||fi|
\end{tabular}
\end{center}
%
Here one could write a message such as:
\begin{center}
|This is the part \childdocname{} of \childdocjob{}.|
\end{center}

%%%%%%%%%%%%%%%%%%%%%%%%%%%%%%%%%%%%%%%%%%%%%%%%%%%%%%%%%%%%%%%%%%%%%%%%%%%%%%%%
\subsection{Flags}
\label{sec:flags}

The package makes it easy to generate different versions
of the main or child documents.
To this end compilation flags can be defined
and assigned different default values.
They will be particularly useful in conjunction
with the forwarding mechanism described in \secref{sec:forward}.

For example, it may be useful to have a flag |\version|
which can be set to |draft| or |final|.
The document source will contain some conditional code
depending on the value of |\version|.
Suppose further, the flag should default to |final| for the main file
and to |draft| for child files
which is a natural assignment for editing the document.
This is achieved by placing the following code
in the preamble of the main document
(below the |\childdocmain| directive):
%
\begin{center}
\begin{tabular}{l}
|\ifchilddoc|\\
|\providecommand{\version}{draft}|\\
|\||else|\\
|\providecommand{\version}{final}|\\
|\||fi|
\end{tabular}
\end{center}
%
The definition by |\providecommand| makes sure
that previous definitions are not overwritten.
Further statements |\providecommand{\version}{...}|
can thus be added before the above code to override it.

For the main file, one might add a line
(between |\childdocmain| and the above block)
%
\begin{center}
|%\ifchilddoc\||else\providecommand{\version}{draft}\||fi|
\end{center}
%
which can be uncommented to produce a draft version.
Likewise one can add a line to the very top of a child file
(above the |\childdocof{|\textit{main}|}| directive)
%
\begin{center}
|%\providecommand{\version}{final}|
\end{center}
%
which can be uncommented to produce the final version of this child document.

%%%%%%%%%%%%%%%%%%%%%%%%%%%%%%%%%%%%%%%%%%%%%%%%%%%%%%%%%%%%%%%%%%%%%%%%%%%%%%%%
\subsection{Forwarding}
\label{sec:forward}

Different versions of the main or child documents
using compilation flags as described in \secref{sec:flags}
can be (permanently) stored in different files
for convenient compilation, viewing and distribution.
To this end, the package defines a command
to pass on compilation to a different file:

%%%%%%%%%%%%%%%%%%%%%%%%%%%%%%%%%%%%%%%%
\DescribeMacro{\childdocforward}
The command |\childdocforward| redirects processing to
another source file:
%
\begin{center}
\begin{tabular}{l}
|\input{childdoc.def}|\\
|\childdocforward[|\textit{main}|]{|\textit{dest}|}|\\
\end{tabular}
\end{center}
%
The argument \textit{dest} is the destination file
(without extension).
It should be the main file or one of the child files.
Note that further \textsf{childdoc} directives
such as |\childdocof| and |\childdocforward|
in the indicated file will be processed in this form.
The optional argument \textit{main}
passes on directly to the main file \textit{main}
while pretending to compile the child \textit{dest}.
This form behaves as if \textit{dest}
issues |\childdocof{|\textit{main}|}| right away,
and no further \textsf{childdoc} directives will be processed.

%%%%%%%%%%%%%%%%%%%%%%%%%%%%%%%%%%%%%%%%
\DescribeMacro{\...prefix}
In the alternative form |\childdocforwardprefix|,
%
\begin{center}
\begin{tabular}{l}
|\input{childdoc.def}|\\
|\childdocforwardprefix[|\textit{main}|]{|\textit{prefix}|}{|\textit{dest}|}|
\end{tabular}
\end{center}
%
the destination file is determined by a pattern
depending on the current file:
To make this work, the current file must be called
`{\textit{prefix}\hspace{0.2em}\textit{suffix}}'
with \textit{prefix} matching precisely the argument.
Processing is then passed on to the file
`{\textit{dest}\hspace{0.2em}\textit{suffix}}'.
Surely, the same effect is achieved by
directly specifying the
argument `{\textit{dest}\hspace{0.2em}\textit{suffix}}'
in the first form.
However, that requires to set up a different file
for each child. With the alternative form of the command
all these files can have exactly the same content
which simplifies setting them up and maintaining them.

For example, the following file |draft.tex|
with a compilation flag |\version| as described in \secref{sec:flags}
compiles the main document as a draft:
%
\begin{center}
\begin{tabular}{l}
|\def\version{draft}|\\
|\input{childdoc.def}|\\
|\childdocforward{|\textit{main}|}|
\end{tabular}
\end{center}
%
Likewise, the following files |final|\textit{nn}|.tex|
compile the final version of the child document
|child|\textit{nn}|.tex|:
%
\begin{center}
\begin{tabular}{l}
|\def\version{final}|\\
|\input{childdoc.def}|\\
|\childdocforwardprefix{final}{child}|
\end{tabular}
\end{center}
%

Note that when several versions of a main file and/or of each child file
are to be generated, it may be convenient to set up a |Makefile| or
shell script to automatise the process.

%%%%%%%%%%%%%%%%%%%%%%%%%%%%%%%%%%%%%%%%%%%%%%%%%%%%%%%%%%%%%%%%%%%%%%%%%%%%%%%%
\subsection{Command Line Processing}
\label{sec:commandline}

The effect of redirection files can also be achieved by invoking
the \LaTeX{} compiler with a more elaborate command line.
Most conveniently this should be done as part
of a shell script or a |Makefile|.

When using \textsf{childdoc} in the main file, the following
command lines effectively perform a redirection
(note that depending on the shell being used,
backslashes may have to be doubled: `|\|' $\to$ `|\\|'):
%
\begin{center}
|... -jobname "|\textit{target}|" |\\|"|[\textit{flags}]%
|\input{childdoc.def}\childdocforward[|\textit{main}|]{|\textit{dest}|}"|
\end{center}
%
Here \textit{target} is the name of the output file,
\textit{main} is the name of the main file
and \textit{dest} is the name of the main or child file to be processed
(all filenames without extensions).
The optional argument \textit{main} can be omitted
if \textit{main} matches \textit{dest}.
Optionally, compilation \textit{flags} can be defined via |\def| commands.
This command line makes the \TeX{} engine believe
it is compiling the file \textit{target}
whose content is specified as the latter parameter.
The provided code then forwards the processing to
\textit{main} or \textit{dest} as described in \secref{sec:forward}.

%%%%%%%%%%%%%%%%%%%%%%%%%%%%%%%%%%%%%%%%%%%%%%%%%%%%%%%%%%%%%%%%%%%%%%%%%%%%%%%%
\subsection{Include by Input}
\label{sec:input}

Including child documents by |\include| has some restrictions by design.
Most notably, the content of a child document always occupies
its own set of pages; pages cannot be shared between child documents.
Usually, this behaviour makes perfect sense
because each child document contain an essential part of the document.
However, in some situations it may be desirable to compose
a document from a collection of parts
without having mandatory page breaks between then.
For this case, the package
provides a mechanism to include parts
by |\input| which can also be processed individually.
However, by construction this mechanism
requires manual handling of the content to be output.

%%%%%%%%%%%%%%%%%%%%%%%%%%%%%%%%%%%%%%%%
\DescribeMacro{\ifchilddocmanual}
The main file should be prepared as usual, see \secref{sec:include}.
However, the document body must make a distinction
between processing of an individual part and of the main document, e.g.:
%
\begin{center}
\begin{tabular}{l}
|\ifchilddocmanual|\\
|\input{\childdocname}|\\
|\||else|\\
\textit{document body with }|\input{|\textit{part}|}|\\
|\||fi|
\end{tabular}
\end{center}
%
The conditional |\ifchilddocmanual| is true whenever
a part to be included by |\input| is being compiled,
and the name of the part is stored in |\childdocname|.

%%%%%%%%%%%%%%%%%%%%%%%%%%%%%%%%%%%%%%%%
\DescribeMacro{\childdocby}
Each part to be included by |\input| should start with:
%
\begin{center}
\begin{tabular}{l}
|\input{childdoc.def}|\\
|\childdocby{|\textit{main}|}|\\
\end{tabular}
\end{center}
%
The directive |\childdocby| is similar to |\childdocof|
described in \secref{sec:include},
but the subsequent selection of content must be done manually.
To that end, both |\ifchilddoc| and |\ifchilddocmanual|
will be true upon processing of a part,
and the name of the part is stored in |\childdocname|.
Note that |\jobname| will be set to the filename of the current part
so that each part receives an individual |.aux| file
that does not interfere with the |.aux| file(s) of the main document.
This behaviour can be altered by the alternative form
|\childdocby[*]{|\textit{main}|}| (with a non-empty optional argument)
which uses the |.aux| file of the main document
by setting |\jobname| to \textit{main}.

%%%%%%%%%%%%%%%%%%%%%%%%%%%%%%%%%%%%%%%%%%%%%%%%%%%%%%%%%%%%%%%%%%%%%%%%%%%%%%%%
\subsection{Driver Development}
\label{sec:driver}

The \textsf{childdoc} mechanism can also be use for the development
of definition files such as \LaTeX{} styles or classes.
This case differs from the above setup with multiple parts
included by |\include| in that no |\includeonly| should be invoked.
This can be achieved by starting the include file
(before |\ProvidesPackage|) with:
%
\begin{center}
\begin{tabular}{l}
|\input{childdoc.def}|\\
|\childdocforward{|\textit{main}|}|\\
\end{tabular}
\end{center}
%
or alternatively with:
%
\begin{center}
\begin{tabular}{l}
|\input{childdoc.def}|\\
|\childdocby{|\textit{main}|}|\\
\end{tabular}
\end{center}
%
Both forms have slightly different effects as described above.
The main file is prepared as usual, see \secref{sec:include}.

%%%%%%%%%%%%%%%%%%%%%%%%%%%%%%%%%%%%%%%%%%%%%%%%%%%%%%%%%%%%%%%%%%%%%%%%%%%%%%%%
\subsection{Legacy Detection}
\label{sec:detection}

The directive |\childdocmain| in the main file can detect
whether the complete document or merely a child is to be compiled
even without using the directive |\childdocof|.
This method is deprecated because it is less robust
and there is no compelling reason to use it;
it is merely provided for backward compatibility
and it may be removed in future versions.

If the detection mechanism is to be used,
it is mandatory to correctly specify
the filename of the main file as the argument of |\childdocmain|:
%
\begin{center}
\begin{tabular}{l}
|\input{childdoc.def}|\\
|\childdocmain{|\textit{main}|}|\\
\end{tabular}
\end{center}
%
If |\jobname| does not match the argument \textit{main} of |\childdocmain|,
it is assumed that |\jobname| points to the child file to be compiled.
When using |\childdocmain| with the main file specified as argument,
it suffices to start a child file
with just |\input{|\textit{main}|}|
without loading of the package and using |\childdocof|.
If instead all processing is done
with the appropriate \textsf{childdoc} directives,
the argument of \textit{main} of |\childdocmain| can be empty.

An alternative version of the command line processing described
in \secref{sec:commandline} using the detection mechanism reads:
%
\begin{center}
|... -jobname "|\textit{target}|" "|[\textit{flags}]%
[|\def\jobname{|\textit{dest}|}|]|\input{|\textit{main}|}"|
\end{center}

%%%%%%%%%%%%%%%%%%%%%%%%%%%%%%%%%%%%%%%%%%%%%%%%%%%%%%%%%%%%%%%%%%%%%%%%%%%%%%%%
\subsection{Manual Code}
\label{sec:manual}

In case one cannot be certain whether the definitions file |childdoc.def|
is installed on the target \TeX{} distribution
and one prefers not to ship it,
it is conceivable to paste a few relevant commands into the sources.

To that end, drop all statements |\input{childdoc.def}|
and perform the replacements as outlined below.
Instead of |\childdocmain{|\textit{main}|}| add the following code
to the top of the main file:
%
\begin{center}
\begin{tabular}{l}
|\||ifdefined\childdocname\endinput\||fi\newif\ifchilddoc|\\
|\edef\childdocname{\scantokens\expandafter{\jobname\noexpand}}|\\
|\def\childdocmain{|\textit{main}|}\||ifx\childdocmain\childdocname\||else|\\
|\childdoctrue\includeonly{\childdocname}\let\jobname\childdocmain\||fi|\\
\end{tabular}
\end{center}
%
Instead of |\childdocof{|\textit{main}|}| just include the main file
at the top of each child file:
%
\begin{center}
|\input{|\textit{main}|}|
\end{center}
%
A simple redirection |\childdocforward{|\textit{dest}|}| is achieved by:
%
\begin{center}
|\def\jobname{|\textit{dest}|}\input{\jobname}|
\end{center}
%
The redirection with prefix
|\childdocforwardprefix[|\textit{prefix}|]{|\textit{dest}|}|
is accomplished by:
%
\begin{center}
\begin{tabular}{l}
|{\edef\jobname{\scantokens\expandafter{\jobname\noexpand}}|\\
|\def\redirectjob |\textit{prefix}|#1~~~{\gdef\jobname{|\textit{dest}|#1}}|\\
|\expandafter\redirectjob\jobname~~~}\input{\jobname}|
\end{tabular}
\end{center}

In an alternative approach,
child documents can be compiled by a specific command line
without additional code or specific definitions:
%
\begin{center}
|... -jobname "|\textit{target}|" "|[\textit{flags}]%
|\includeonly{|\textit{dest}|}\input{|\textit{main}|}"|
\end{center}
%

%%%%%%%%%%%%%%%%%%%%%%%%%%%%%%%%%%%%%%%%%%%%%%%%%%%%%%%%%%%%%%%%%%%%%%%%%%%%%%%%
%%%%%%%%%%%%%%%%%%%%%%%%%%%%%%%%%%%%%%%%%%%%%%%%%%%%%%%%%%%%%%%%%%%%%%%%%%%%%%%%
\section{Information}

%%%%%%%%%%%%%%%%%%%%%%%%%%%%%%%%%%%%%%%%%%%%%%%%%%%%%%%%%%%%%%%%%%%%%%%%%%%%%%%%
\subsection{Copyright}

Copyright \copyright{} 2017--2018 Niklas Beisert

This work may be distributed and/or modified under the
conditions of the \LaTeX{} Project Public License, either version 1.3
of this license or (at your option) any later version.
The latest version of this license is in
  \url{http://www.latex-project.org/lppl.txt}
and version 1.3 or later is part of all distributions of \LaTeX{}
version 2005/12/01 or later.

This work has the LPPL maintenance status `maintained'.

The Current Maintainer of this work is Niklas Beisert.

This work consists of the files |README.txt|, |childdoc.ins| and |childdoc.dtx|
as well as the derived files |childdoc.def|, |cdocsamp.tex|
with |cdocsch1.tex|, |cdocsch2.tex|, |cdocspt3.tex|, |cdocspt4.tex|,
|cdocsdrf.tex|, |cdocsfn1.tex|, |cdocsfn2.tex|
as well as |childdoc.pdf|.

%%%%%%%%%%%%%%%%%%%%%%%%%%%%%%%%%%%%%%%%%%%%%%%%%%%%%%%%%%%%%%%%%%%%%%%%%%%%%%%%
\subsection{Files and Installation}

The package consists of the files:
%
\begin{center}
\begin{tabular}{ll}
    |README.txt|   & readme file \\
    |childdoc.ins| & installation file \\
    |childdoc.dtx| & source file \\
    |childdoc.def| & definition file \\
    |cdocsamp.tex| & sample main file \\
    |cdocsch1.tex| & sample include file \\
    |cdocsch2.tex| & sample include file \\
    |cdocspt3.tex| & sample part file \\
    |cdocspt4.tex| & sample part file \\
    |cdocsdrf.tex| & sample redirection file \\
    |cdocsfn1.tex| & sample redirection file \\
    |cdocsfn2.tex| & sample redirection file \\
    |childdoc.pdf| & manual
\end{tabular}
\end{center}
%
The distribution consists of the files
|README.txt|, |childdoc.ins| and |childdoc.dtx|.
%
\begin{itemize}
\item
Run (pdf)\LaTeX{} on |childdoc.dtx|
to compile the manual |childdoc.pdf| (this file).
\item
Run \LaTeX{} on |childdoc.ins| to create the definitions file |childdoc.def|
and the sample |cdocsamp.tex| with include files
|cdocsch1.tex|, |cdocsch2.tex|, |cdocspt3.tex|, |cdocspt4.tex|,
|cdocsdrf.tex|, |cdocsfn1.tex|, |cdocsfn2.tex|.
Then copy the file |childdoc.def| to an appropriate directory of your \LaTeX{}
distribution, e.g.\ \textit{texmf-root}|/tex/latex/childdoc|.
\end{itemize}

%%%%%%%%%%%%%%%%%%%%%%%%%%%%%%%%%%%%%%%%%%%%%%%%%%%%%%%%%%%%%%%%%%%%%%%%%%%%%%%%
\subsection{Related CTAN Packages}

There are several other packages which offer a similar functionality:
%
\begin{itemize}
\item
The packages
\href{http://ctan.org/pkg/docmute}{\textsf{docmute}},
\href{http://ctan.org/pkg/includex}{\textsf{includex}} and
\href{http://ctan.org/pkg/standalone}{\textsf{standalone}}
provide commands to include only the document body of
a child file thus allowing both files to be compiled individually.
\item
The packages \href{http://ctan.org/pkg/subdocs}{\textsf{subdocs}}
and \href{http://ctan.org/pkg/subfiles}{\textsf{subfiles}}
provide structures in which the main and child documents can be
encapsulated and allowing them to be compiled individually.
The inclusion mechanism is different from the conventional |\include|.
\item
The package \href{http://ctan.org/pkg/combine}{\textsf{combine}}
is an elaborate solution to combine several documents into one.
\end{itemize}
%
See also the CTAN topic \href{http://ctan.org/topic/subdocs}{\textsf{subdocs}}
for further related packages.
The present package differs from the above solutions in that
a document structure constructed with the conventional |\include| mechanism
just needs two extra commands at the top of every file
such that all constituent files can be compiled individually.

%%%%%%%%%%%%%%%%%%%%%%%%%%%%%%%%%%%%%%%%%%%%%%%%%%%%%%%%%%%%%%%%%%%%%%%%%%%%%%%%
%\subsection{Feature Suggestions}
%
%The following is a list of features which may be useful for future
%versions of this package:
%%
%\begin{itemize}
%\item
%\ldots
%\end{itemize}

%%%%%%%%%%%%%%%%%%%%%%%%%%%%%%%%%%%%%%%%%%%%%%%%%%%%%%%%%%%%%%%%%%%%%%%%%%%%%%%%
\subsection{Revision History}

%%%%%%%%%%%%%%%%%%%%%%%%%%%%%%%%%%%%%%%%
\paragraph{v2.0:} 2018/12/30

\begin{itemize}
\item
immediate forward processing
\item
added |\childdocby| mechanism
\item
manual restructured
\end{itemize}

%%%%%%%%%%%%%%%%%%%%%%%%%%%%%%%%%%%%%%%%
\paragraph{v1.6:} 2018/01/17

\begin{itemize}
\item
application for development of include files
\item
corrections to manual
\end{itemize}

%%%%%%%%%%%%%%%%%%%%%%%%%%%%%%%%%%%%%%%%
\paragraph{v1.5:} 2017/05/21

\begin{itemize}
\item
more complete structuring introduced
\item
|\childdocof| introduced
\item
|\childdoc| renamed to |\childdocmain|
\item
|\childredirect| renamed to |\childdocforward| and |\childdocforwardprefix|
and functionality expanded
\end{itemize}

%%%%%%%%%%%%%%%%%%%%%%%%%%%%%%%%%%%%%%%%
\paragraph{v1.0:} 2017/04/27

\begin{itemize}
\item
manual and install package
\item
first version published on CTAN
\end{itemize}

%%%%%%%%%%%%%%%%%%%%%%%%%%%%%%%%%%%%%%%%
\paragraph{v0.6:} 2017/04/26

\begin{itemize}
\item
redirection mechanism added
\end{itemize}

%%%%%%%%%%%%%%%%%%%%%%%%%%%%%%%%%%%%%%%%
\paragraph{v0.5:} 2017/04/26

\begin{itemize}
\item
functionality in definition file
\end{itemize}


%%%%%%%%%%%%%%%%%%%%%%%%%%%%%%%%%%%%%%%%%%%%%%%%%%%%%%%%%%%%%%%%%%%%%%%%%%%%%%%%
%%%%%%%%%%%%%%%%%%%%%%%%%%%%%%%%%%%%%%%%%%%%%%%%%%%%%%%%%%%%%%%%%%%%%%%%%%%%%%%%
%%%%%%%%%%%%%%%%%%%%%%%%%%%%%%%%%%%%%%%%%%%%%%%%%%%%%%%%%%%%%%%%%%%%%%%%%%%%%%%%
\appendix

\settowidth\MacroIndent{\rmfamily\scriptsize 000\ }

 \DocInput{childdoc.dtx}

\end{document}
%</driver>
% \fi
%
% %%%%%%%%%%%%%%%%%%%%%%%%%%%%%%%%%%%%%%%%%%%%%%%%%%%%%%%%%%%%%%%%%%%%%%%%%%%%%%
% %%%%%%%%%%%%%%%%%%%%%%%%%%%%%%%%%%%%%%%%%%%%%%%%%%%%%%%%%%%%%%%%%%%%%%%%%%%%%%
% \section{Sample}
%\iffalse
%<*samplemain>
%\fi
%
% The following presents a sample document
% with two chapters, two parts, a title page,
% a compile flag as well as three forwarding files to set the flag.
% It consists of eight |.tex| files:
% \begin{center}
% \begin{tabular}{ll}
% |cdocsamp.tex|&main file\\
% |cdocsch1.tex|&include file for chapter 1\\
% |cdocsch2.tex|&include file for chapter 2\\
% |cdocspt3.tex|&include file for part 3\\
% |cdocspt4.tex|&include file for part 4\\
% |cdocsdrf.tex|&forwarding file for main file in draft mode\\
% |cdocsfi1.tex|&forwarding file for final version of chapter 1\\
% |cdocsfi2.tex|&forwarding file for final version of chapter 2\\
% \end{tabular}
% \end{center}
% Each of the eight files can be compiled directly by the \LaTeX{} compiler.
%
% %%%%%%%%%%%%%%%%%%%%%%%%%%%%%%%%%%%%%%
% \paragraph{Main File.}
%
% The main file is called |cdocsamp.tex|.
%
% Load the \textsf{childdoc} definitions and
% declare the filename for the main document:
%    \begin{macrocode}
\input{childdoc.def}
\childdocmain{}
%    \end{macrocode}

% Optional override for |\version| flag:
%    \begin{macrocode}
%%\ifchilddoc\else\providecommand{\version}{draft}\fi
%    \end{macrocode}

% Define the default values for the |\version| flag
% (|final| for the main file and |draft| for childs):
%    \begin{macrocode}
\ifchilddoc
\providecommand{\version}{draft}
\else
\providecommand{\version}{final}
\fi
%    \end{macrocode}

% Load the standard document class:
%    \begin{macrocode}
\documentclass[12pt]{article}
%    \end{macrocode}

% Start the document body:
%    \begin{macrocode}
\begin{document}
%    \end{macrocode}

% Declare a title page.
% Print title, part of document being processed and version flag:
%    \begin{macrocode}
\addtocounter{page}{-1}
\begin{center}
{\LARGE\bfseries{}childdoc example\par}
\vspace{1cm}
\ifchilddoc
\ifchilddocmanual part\else chapter\fi:
`\childdocname' of `\childdocjob'\par
\else
main document: `\childdocjob'\par
\fi
version: \version\par
\end{center}
\newpage
%    \end{macrocode}

% Manually include selected file,
% otherwise process as usual:
%    \begin{macrocode}
\ifchilddocmanual
\section*{part `\childdocname'}
\input{\childdocname}
\else
%    \end{macrocode}

% Include the two chapters:
%    \begin{macrocode}
\include{cdocsch1}
\include{cdocsch2}
%    \end{macrocode}

% Include the two parts unless only chapters should be displayed:
%    \begin{macrocode}
\ifchilddoc\else
\section{part three}
\input{cdocspt3}
\section{part four}
\input{cdocspt4}
\fi
%    \end{macrocode}

% Process as usual until here:
%    \begin{macrocode}
\fi
%    \end{macrocode}

% End of document body:
%    \begin{macrocode}
\end{document}
%    \end{macrocode}
%\iffalse
%</samplemain>
%\fi
%
% %%%%%%%%%%%%%%%%%%%%%%%%%%%%%%%%%%%%%%
% \paragraph{Chapter Include Files.}
%
% The include files are called |cdocsch1.tex| and |cdocsch2.tex|.
%
%\iffalse
%<*samplechap1|samplechap2>
%\fi

% Optional override for |\version| flag:
%    \begin{macrocode}
%%\providecommand{\version}{final}
%    \end{macrocode}

% Include the main document:
%    \begin{macrocode}
\input{childdoc.def}
\childdocof{cdocsamp}
%    \end{macrocode}

%\iffalse
%</samplechap1|samplechap2>
%\fi
%
%\iffalse
%<*samplechap1>
%\fi
% Some text for chapter 1:
%    \begin{macrocode}
\section{one}
some text in chapter one
%    \end{macrocode}

%\iffalse
%</samplechap1>
%\fi
% Some text for chapter 2:
%\iffalse
%<*samplechap2>
%\fi
%    \begin{macrocode}
\section{two}
more text in chapter two
%    \end{macrocode}

%\iffalse
%</samplechap2>
%\fi
%
% %%%%%%%%%%%%%%%%%%%%%%%%%%%%%%%%%%%%%%
% \paragraph{Part Include Files.}
%
% The include files are called |cdocspt3.tex| and |cdocspt4.tex|.
%
%\iffalse
%<*samplepart3|samplepart4>
%\fi

% Optional override for |\version| flag:
%    \begin{macrocode}
%%\providecommand{\version}{final}
%    \end{macrocode}

% Include the main document:
%    \begin{macrocode}
\input{childdoc.def}
\childdocby{cdocsamp}
%    \end{macrocode}

%\iffalse
%</samplepart3|samplepart4>
%\fi
%
%\iffalse
%<*samplepart3>
%\fi
% Some text for part 3:
%    \begin{macrocode}
some text in part three
%    \end{macrocode}

%\iffalse
%</samplepart3>
%\fi
% Some text for part 4:
%\iffalse
%<*samplepart4>
%\fi
%    \begin{macrocode}
more text in part four
%    \end{macrocode}

%\iffalse
%</samplepart4>
%\fi
%
% %%%%%%%%%%%%%%%%%%%%%%%%%%%%%%%%%%%%%%
% \paragraph{Forwarding for a Complete Draft.}
%
% The following forwarding file |cdocsdrf.tex|
% compiles the main document in draft mode:
%\iffalse
%<*sampledraft>
%\fi
%    \begin{macrocode}
\def\version{draft}
\input{childdoc.def}
\childdocforward{cdocsamp}
%    \end{macrocode}

%\iffalse
%</sampledraft>
%\fi
%
% %%%%%%%%%%%%%%%%%%%%%%%%%%%%%%%%%%%%%%
% \paragraph{Forwarding for Final Version of the Chapters.}
%
% The following forwarding files |cdocsfn1.tex| and |cdocsfn2.tex|
% (with identical content)
% compile the final versions of the child documents
% |cdocsch1.tex| and |cdocsch2.tex|, respectively:
%\iffalse
%<*samplefinal>
%\fi
%    \begin{macrocode}
\def\version{final}
\input{childdoc.def}
\childdocforwardprefix[cdocsamp]{cdocsfn}{cdocsch}
%    \end{macrocode}

%\iffalse
%</samplefinal>
%\fi
%
% %%%%%%%%%%%%%%%%%%%%%%%%%%%%%%%%%%%%%%
% \paragraph{Command Line Processing.}
%
% The following three command lines generate the output files
% |cdocscld|, |cdocscl1| and |cdocscl2|
% which should be identical to
% |cdocsdrf|, |cdocsch1| and |cdocsfn2|, respectively:
% \begin{center}
% \begin{tabular}{l}
% |latex -jobname cdocscld \|\\
% |  "\def\version{draft}\input{childdoc.def}\childdocforward{cdocsamp}"|\\
% |latex -jobname cdocscl1 \|\\
% |  "\input{childdoc.def}\childdocforward[cdocsamp]{cdocsch1}"|\\
% |latex -jobname cdocscl2 \|\\
% |  "\def\version{final}\input{childdoc.def}\childdocforward{cdocsch2}"|
% \end{tabular}
% \end{center}
% Note that the trailing backslash on each first line
% merely continues the input to the second line
% (for convenient cut ant paste).
% Furthermore, the command |latex| can be replaced by any
% of its alternative versions such as |pdflatex|.
%
% %%%%%%%%%%%%%%%%%%%%%%%%%%%%%%%%%%%%%%%%%%%%%%%%%%%%%%%%%%%%%%%%%%%%%%%%%%%%%%
% %%%%%%%%%%%%%%%%%%%%%%%%%%%%%%%%%%%%%%%%%%%%%%%%%%%%%%%%%%%%%%%%%%%%%%%%%%%%%%
% \section{Implementation}
%\iffalse
%<*package>
%\fi
%
% This section describes the definitions file |childdoc.def|.

% The definitions cannot be loaded using |\usepackage| or |\RequirePackage|
% which has a mechanism to prevent loading a style file more than once.
% When loading the definitions by means of |\input|
% multiple instances have to be prevented manually:
%\iffalse
%This code needs to be before the `\ProvidesFile' directive
%which is defined at the beginning of this file.
%Therefore it is also placed there and commented out here.
%</package>
%<*discard>
%\fi
%    \begin{macrocode}
\ifdefined\childdocmain\endinput\fi
%    \end{macrocode}
%\iffalse
%</discard>
%<*package>
%\fi
%
% \macro{\ifchilddoc}
% \macro{\ifchilddocmanual}
% The conditional |\ifchilddoc| tells whether a
% child (true) or main (false) document is being compiled.
% The conditional |\ifchilddocmanual| tells whether
% the |\includeonly| mechanism is used (false) or
% the selection of child files must be performed manually (true).
% The definitions initialise to false:
%    \begin{macrocode}
\newif\ifchilddoc
\newif\ifchilddocmanual
%    \end{macrocode}

% \macro{\childdocname}
% \macro{\childdocjob}
% The macro |\childdocname| stores the name of the main document
% to be compiled. The macro |\childdocjob| stores the name of
% the document on which the \LaTeX{} compiler was originally invoked.
% The content of |\jobname| cannot be compared
% to filenames specified in the source due to different catcodes.
% The following code rescans |\jobname|, stores the result
% in |\childdocname| and saves a copy in |\childdocjob|:
%    \begin{macrocode}
\edef\childdocname{\scantokens\expandafter{\jobname\noexpand}}
\let\childdocjob\childdocname
%    \end{macrocode}

% \macro{\childdocdisable}
% The macro |\childdocdisable| prevents the main file
% from being processed more than once.
% At this stage, the main document command |\childdocmain|
% is assumed to be called once again where it should do nothing.
% Any subsequent call to it should prevent
% a secondary processing of the main document
% It overwrites the forwarding commands
% |\childdocof| and |\childdocforward|
% with empty macros to prevent further inclusions of the main document:
%    \begin{macrocode}
\newcommand{\childdocdisable}
{
  \renewcommand{\childdocmain}[1]{\renewcommand{\childdocmain}[1]{\endinput}}
  \renewcommand{\childdocof}[1]{}
  \renewcommand{\childdocby}[2][]{}
  \renewcommand{\childdocforward}[2][]{}
  \renewcommand{\childdocdisable}{}
}
%    \end{macrocode}

% \macro{\childdocmain}
% The macro |\childdocmain| is to be called at the top of the main file
% with nothing or the main filename (without extension) as argument.
% First, it breaks loops.
% If the argument is not empty and does not match |\childdocname|
% (which is set by the first inclusion of |childdoc.def|),
% |\ifchilddoc| is set to true, |\includeonly| is applied to the child file
% and |\jobname| is set to the main file
% (for proper handling of |.aux| files):
%    \begin{macrocode}
\newcommand{\childdocmain}[1]
{
  \childdocdisable\childdocmain{}
  \if?#1?\else
    \begingroup
      \def\childdoctmp{#1}
      \ifx\childdoctmp\childdocname
        \def\childdoctmp{}
      \else
        \def\childdoctmp
        {
          \childdoctrue
          \includeonly{\childdocname}
          \def\childdocjob{#1}
          \def\jobname{#1}
        }
      \fi
      \expandafter
    \endgroup
    \childdoctmp
  \fi
}
%    \end{macrocode}

% \macro{\childdocof}
% The command |\childdocof| redirects
% compilation to the main file |#1|.
%    \begin{macrocode}
\newcommand{\childdocof}[1]
{
  \childdocdisable
  \childdoctrue
  \includeonly{\childdocname}
  \def\jobname{#1}
  \def\childdocjob{#1}
  \input{#1}
}
%    \end{macrocode}

% \macro{\childdocby}
% The command |\childdocby| ....
%    \begin{macrocode}
\newcommand{\childdocby}[2][]
{
  \childdocdisable
  \childdoctrue
  \childdocmanualtrue
  \if?#1?\else
    \def\jobname{#2}
  \fi
  \def\childdocjob{#2}
  \input{#2}
  \endinput
}
%    \end{macrocode}

% \macro{\childdocforward}
% The command |\childdocforward| redirects
% compilation to the main file or
% (if the optional argument is given) a child file.
% Parameters are set as if the main file
% or a child file starting with |\childdocof| was compiled.
% Then compilation is handed over to the main file:
%    \begin{macrocode}
\newcommand{\childdocforward}[2][]
{
  \begingroup
    \if?#1?
      \def\childdoctmp
      {
        \def\childdocname{#2}
        \def\childdocjob{#2}
        \def\jobname{#2}
        \input{#2}
        \endinput
      }
    \else
      \def\childdoctmp
      {
        \childdocdisable
        \def\childdocname{#2}
        \childdoctrue
        \includeonly{#2}
        \def\childdocjob{#1}
        \def\jobname{#1}
        \input{#1}
        \endinput
      }
    \fi
    \expandafter
  \endgroup
  \childdoctmp
}
%    \end{macrocode}

% \macro{\childdocforwardprefix}
% The command |\childdocforwardprefix| redirects
% compilation to the main or a child file by means of a pattern.
% The prefix |#1| in the current filename is replaced by |#2|
% and the suffix of the current filename is kept
% (it is assumed that the filename does not contain the substring `|~~~|'
% which is used as a delimiter).
% Compilation is handed over to the new file by |\childdocforward|:
%    \begin{macrocode}
\newcommand{\childdocforwardprefix}[3][]
{
  \begingroup
    \def\childdocextract #2##1~~~{\def\childdoctmp{\childdocforward[#1]{#3##1}}}
    \expandafter\childdocextract\childdocname~~~
    \expandafter
  \endgroup
  \childdoctmp
}
%    \end{macrocode}

% \macro{\childdoc}
% The deprecated macro |\childdoc| is a legacy version of |\childdocmain|:
%    \begin{macrocode}
\newcommand{\childdoc}{\childdocmain}
%    \end{macrocode}

% \macro{\childdocredirect}
% The deprecated macro |\childdocredirect| is a legacy version
% of |\childdocforward| and |\childdocforwardprefix|:
%    \begin{macrocode}
\newcommand{\childdocredirect}[2][]
{
  \begingroup
    \if?#1?
      \def\childdoctmp{\childdocforward{#2}}
    \else
      \def\childdoctmp{\childdocforwardprefix{#1}{#2}}
    \fi
    \expandafter
  \endgroup
  \childdoctmp
}
%    \end{macrocode}

%\iffalse
%</package>
%\fi
%
\endinput
|\\
|\childdocforward[|\textit{main}|]{|\textit{dest}|}|\\
\end{tabular}
\end{center}
%
The argument \textit{dest} is the destination file
(without extension).
It should be the main file or one of the child files.
Note that further \textsf{childdoc} directives
such as |\childdocof| and |\childdocforward|
in the indicated file will be processed in this form.
The optional argument \textit{main}
passes on directly to the main file \textit{main}
while pretending to compile the child \textit{dest}.
This form behaves as if \textit{dest}
issues |\childdocof{|\textit{main}|}| right away,
and no further \textsf{childdoc} directives will be processed.

%%%%%%%%%%%%%%%%%%%%%%%%%%%%%%%%%%%%%%%%
\DescribeMacro{\...prefix}
In the alternative form |\childdocforwardprefix|,
%
\begin{center}
\begin{tabular}{l}
|% \iffalse
%
% childdoc.dtx Copyright (C) 2017-2018 Niklas Beisert
%
% This work may be distributed and/or modified under the
% conditions of the LaTeX Project Public License, either version 1.3
% of this license or (at your option) any later version.
% The latest version of this license is in
%   http://www.latex-project.org/lppl.txt
% and version 1.3 or later is part of all distributions of LaTeX
% version 2005/12/01 or later.
%
% This work has the LPPL maintenance status `maintained'.
%
% The Current Maintainer of this work is Niklas Beisert.
%
% This work consists of the files childdoc.dtx and childdoc.ins
% and the derived files childdoc.def and cdocsamp.tex with
% cdocsch1.tex, cdocsch2.tex, cdocsdrf.tex, cdocsfn1.tex, cdocsfn2.tex.
%
%<package>\ifdefined\childdocmain\endinput\fi
%<package>\ProvidesFile{childdoc.def}[2018/12/30 v2.0 child document driver]
%<samplemain>\ProvidesFile{cdocsamp.tex}[2018/12/30 v2.0 sample for childdoc]
%<*driver>
%\ProvidesFile{childdoc.drv}[2018/12/30 v2.0 childdoc reference manual file]
\PassOptionsToClass{10pt,a4paper}{article}
\documentclass{ltxdoc}

\usepackage[margin=35mm]{geometry}
\usepackage{hyperref}
\usepackage{hyperxmp}
\usepackage[usenames]{color}

\hypersetup{colorlinks=true}
\hypersetup{pdfstartview=FitH}
\hypersetup{pdfpagemode=UseNone}
\hypersetup{pdfsource={}}
\hypersetup{pdflang={en-UK}}
\hypersetup{pdfcopyright={Copyright 2017-2018 Niklas Beisert.
  This work may be distributed and/or modified under the
  conditions of the LaTeX Project Public License, either version 1.3
  of this license or (at your option) any later version.}}
\hypersetup{pdflicenseurl={http://www.latex-project.org/lppl.txt}}
\hypersetup{pdfcontactaddress={ETH Zurich, ITP, HIT K,
  Wolfgang-Pauli-Strasse 27}}
\hypersetup{pdfcontactpostcode={8093}}
\hypersetup{pdfcontactcity={Zurich}}
\hypersetup{pdfcontactcountry={Switzerland}}
\hypersetup{pdfcontactemail={nbeisert@itp.phys.ethz.ch}}
\hypersetup{pdfcontacturl={http://people.phys.ethz.ch/\xmptilde nbeisert/}}

\newcommand{\secref}[1]{\hyperref[#1]{section \ref*{#1}}}

\parskip1ex
\parindent0pt
\let\olditemize\itemize
\def\itemize{\olditemize\parskip0pt}

\begin{document}

\title{The \textsf{childdoc} Package}
\hypersetup{pdftitle={The childdoc Package}}
\author{Niklas Beisert\\[2ex]
  Institut f\"ur Theoretische Physik\\
  Eidgen\"ossische Technische Hochschule Z\"urich\\
  Wolfgang-Pauli-Strasse 27, 8093 Z\"urich, Switzerland\\[1ex]
  \href{mailto:nbeisert@itp.phys.ethz.ch}
  {\texttt{nbeisert@itp.phys.ethz.ch}}}
\hypersetup{pdfauthor={Niklas Beisert}}
\hypersetup{pdfsubject={Manual for the LaTeX2e Package childdoc}}
\date{30 December 2018, \textsf{v2.0}}
\maketitle

\begin{abstract}\noindent
\textsf{childdoc} is a \LaTeXe{} package
that enables the direct compilation
of document sections included by |\include|
to individual files.
\end{abstract}

\begingroup
\parskip0ex
\tableofcontents
\endgroup

%%%%%%%%%%%%%%%%%%%%%%%%%%%%%%%%%%%%%%%%%%%%%%%%%%%%%%%%%%%%%%%%%%%%%%%%%%%%%%%%
%%%%%%%%%%%%%%%%%%%%%%%%%%%%%%%%%%%%%%%%%%%%%%%%%%%%%%%%%%%%%%%%%%%%%%%%%%%%%%%%
\section{Introduction}

\LaTeX{} provides a mechanism to structure a large document (such as a book)
into a main file and several child files (containing the chapters)
using the |\include| command.
This mechanism is beneficial for documents
which span hundreds of pages in order to
make the source file(s) more manageable.
Moreover, compilation can be restricted to
selected child files by means of the |\includeonly| command.
The latter feature can be used to reduce the compilation time while editing
(this was significantly more useful in the earlier days of \LaTeX{})
or to generate a smaller document which is easier to navigate.
Another application of |\includeonly| is to generate
documents consisting of selected parts of the complete document.

However, there are a few drawbacks of the plain |\include| mechanism:
\begin{itemize}
\item
The child files cannot be compiled on their own,
they can only be compiled via the main file.
A naive editing environment
(such as a text editor with an option
to have the current file processed by \LaTeX)
may require one to switch to the main file before compiling;
attempting to compile the child file produces errors.
\item
The main file must be modified (each time)
to adjust the |\includeonly| command
to the present needs. This easily leaves the main file in a messy state.
\item
The generated document will always carry the filename
of the main document. This is inconvenient if
several child files are to be compiled and
to be kept for distribution.
\end{itemize}

The present package provides a simple interface
to make child files individually compilable by \LaTeX{}.
Compiling a child file then has the same effect as compiling
the main file with an |\includeonly| command
to select the appropriate child.
Moreover the generated document will carry the name of the child
rather than the main file.
This resolves all three above issues.

This feature is meant to make the editing of books,
thesis documents and lecture notes somewhat more convenient.
However, the package can also be used efficiently for
composing a series of documents (such as exercise sheets)
which are typically distributed individually.
It then assists the author in generating the individual documents
(potentially in different versions)
as well as a document containing the collected series.
Another application is in developing style files
or other kinds of included material
where compilation of the style file could redirect
to a sample or test file.

%%%%%%%%%%%%%%%%%%%%%%%%%%%%%%%%%%%%%%%%%%%%%%%%%%%%%%%%%%%%%%%%%%%%%%%%%%%%%%%%
%%%%%%%%%%%%%%%%%%%%%%%%%%%%%%%%%%%%%%%%%%%%%%%%%%%%%%%%%%%%%%%%%%%%%%%%%%%%%%%%
\section{Usage}

First of all, the package \textsf{childdoc} is \emph{not} a standard
\LaTeXe{} |.sty| style file! Therefore it needs to be invoked in
a non-standard way.

%%%%%%%%%%%%%%%%%%%%%%%%%%%%%%%%%%%%%%%%%%%%%%%%%%%%%%%%%%%%%%%%%%%%%%%%%%%%%%%%
\subsection{Included Files}
\label{sec:include}

%%%%%%%%%%%%%%%%%%%%%%%%%%%%%%%%%%%%%%%%
\DescribeMacro{\childdocmain}
To use the package, add the commands
\begin{center}
\begin{tabular}{l}
|\input{childdoc.def}|\\
|\childdocmain{}|\\
\end{tabular}
\end{center}
at the very top of the main \LaTeX{} file,
in particular \emph{before} the |\documentclass| statement!
The argument of |\childdocmain| should be left empty
(but it must be present).

%%%%%%%%%%%%%%%%%%%%%%%%%%%%%%%%%%%%%%%%
\DescribeMacro{\childdocof}
Furthermore, add the commands
\begin{center}
\begin{tabular}{l}
|\input{childdoc.def}|\\
|\childdocof{|\textit{main}|}|\\
\end{tabular}
\end{center}
at the top of every child file \textit{child}
which is included by |\include{|\textit{child}|}|
from within the main file
(or at least for those files to be compiled individually).
The argument \textit{main} must be the filename of the main file.

There are a couple of
considerations in setting up the main and child documents:

%%%%%%%%%%%%%%%%%%%%%%%%%%%%%%%%%%%%%%%%
\paragraph{Restrictions.}

Please note the following restrictions:
\begin{itemize}
\item
|\childdocmain| must be called with one argument \textit{main}
to ensure compatibility with earlier version of the package.
It must either be empty (|\childdocmain{}|)
or precisely match the filename of the main file in which it is specified.
See \secref{sec:detection} for further information.
\item
The filename \textit{main} must be specified without the |.tex| extension.
\item
The filename \textit{main} is case sensitive
(even in case-insensitive file systems)
due to internal string comparison.
\item
The argument \textit{main} should be fully expanded, it cannot be a macro.
\item
Subdirectories and special characters should be avoided in filenames.
\item
The command |\childdocmain{|\textit{main}|}| must be followed by a whitespace.
It should not be followed immediately by another command
or by a comment mark `|%|'.
This is because the \TeX{} parser reads the token immediately following
the argument of |\childdocmain| and puts it
at the beginning of every child section;
however, a white\-space is ignored.
\end{itemize}

%%%%%%%%%%%%%%%%%%%%%%%%%%%%%%%%%%%%%%%%
\paragraph{Content of Main File.}

It is advisable to place all content in the child files included by |\include|.
Any output contained in the main file will appear in all child documents
unless suppressed manually;
it cannot be suppressed automatically by the |\includeonly| directive
and thus should normally be avoided.
A method to include some content in the main file
by means of conditional processing is described in \secref{sec:conditional}.

%%%%%%%%%%%%%%%%%%%%%%%%%%%%%%%%%%%%%%%%
\paragraph{Page Numbering.}

When only a part of the document is compiled,
the appropriate numbering of pages
(as well as other status parameters)
is determined from the |.aux| files.
The latter contain information from previous passes.
However this information needs to propagate through
all intermediate child documents.
Therefore the page numbering in child documents may well
be inconsistent until the complete document is compiled at least once.

A useful (if unconventional) way to always ensure a consistent
page numbering is to restart the numbering in each child document
and denote the pages by `\textit{child}|.|\textit{page}'
where \textit{child} represents the chapter/section number of the child file.
This can be achieved by the command
|\numberwithin{page}{|\textit{child}|}|
of the \textsf{amsmath} package
where \textit{child} can be |chapter| or |section|
depending on the chosen structuring.
Alternatively, one can modify the macro |\thepage| appropriately
and reset the counter |page| at the start of each child file.

%%%%%%%%%%%%%%%%%%%%%%%%%%%%%%%%%%%%%%%%%%%%%%%%%%%%%%%%%%%%%%%%%%%%%%%%%%%%%%%%
\subsection{Conditional Processing}
\label{sec:conditional}

The package provides a mechanism to compile different versions
of a document. To customise the versions further some conditional processing
can come in handy to distinguish which version is being compiled.
The package provides two macros to describe the compilation context:

%%%%%%%%%%%%%%%%%%%%%%%%%%%%%%%%%%%%%%%%
\DescribeMacro{\ifchilddoc}
The conditional |\ifchilddoc| distinguishes between the compilation of
child documents and the main document:
%
\begin{center}
|\ifchilddoc |\textit{child-code}| |[|\||else |\textit{main-code}]| \||fi|
\end{center}

%%%%%%%%%%%%%%%%%%%%%%%%%%%%%%%%%%%%%%%%
\DescribeMacro{\childdocname}
\DescribeMacro{\childdocjob}
The macro |\childdocname| contains the filename (without extension)
of the main or child file being processed.
Note that |\childdocjob| will always contain the name of the main file.

%%%%%%%%%%%%%%%%%%%%%%%%%%%%%%%%%%%%%%%%
\paragraph{Title Page.}

Conditional processing can be used to include a title or banner page
in the main document when proper precautions are taken.
Importantly, the code in the main file should ensure that the page counter
(as well as other status parameters which are stored in the |.aux| files)
takes the same value after the conditional processing.
Otherwise the page numbers may take divergent values
depending on which part is compiled.

For example, a title page could be declared by:
%
\begin{center}
\begin{tabular}{l}
|\ifchilddoc\||else|\\
|\addtocounter{page}{-1}|\\
\textit{code for title page}\\
|\newpage|\\
|\||fi|
\end{tabular}
\end{center}
%
A banner page for the child documents can be generated by:
%
\begin{center}
\begin{tabular}{l}
|\ifchilddoc|\\
|\addtocounter{page}{-1}|\\
\textit{code for banner page}\\
|\newpage|\\
|\||fi|
\end{tabular}
\end{center}
%
Here one could write a message such as:
\begin{center}
|This is the part \childdocname{} of \childdocjob{}.|
\end{center}

%%%%%%%%%%%%%%%%%%%%%%%%%%%%%%%%%%%%%%%%%%%%%%%%%%%%%%%%%%%%%%%%%%%%%%%%%%%%%%%%
\subsection{Flags}
\label{sec:flags}

The package makes it easy to generate different versions
of the main or child documents.
To this end compilation flags can be defined
and assigned different default values.
They will be particularly useful in conjunction
with the forwarding mechanism described in \secref{sec:forward}.

For example, it may be useful to have a flag |\version|
which can be set to |draft| or |final|.
The document source will contain some conditional code
depending on the value of |\version|.
Suppose further, the flag should default to |final| for the main file
and to |draft| for child files
which is a natural assignment for editing the document.
This is achieved by placing the following code
in the preamble of the main document
(below the |\childdocmain| directive):
%
\begin{center}
\begin{tabular}{l}
|\ifchilddoc|\\
|\providecommand{\version}{draft}|\\
|\||else|\\
|\providecommand{\version}{final}|\\
|\||fi|
\end{tabular}
\end{center}
%
The definition by |\providecommand| makes sure
that previous definitions are not overwritten.
Further statements |\providecommand{\version}{...}|
can thus be added before the above code to override it.

For the main file, one might add a line
(between |\childdocmain| and the above block)
%
\begin{center}
|%\ifchilddoc\||else\providecommand{\version}{draft}\||fi|
\end{center}
%
which can be uncommented to produce a draft version.
Likewise one can add a line to the very top of a child file
(above the |\childdocof{|\textit{main}|}| directive)
%
\begin{center}
|%\providecommand{\version}{final}|
\end{center}
%
which can be uncommented to produce the final version of this child document.

%%%%%%%%%%%%%%%%%%%%%%%%%%%%%%%%%%%%%%%%%%%%%%%%%%%%%%%%%%%%%%%%%%%%%%%%%%%%%%%%
\subsection{Forwarding}
\label{sec:forward}

Different versions of the main or child documents
using compilation flags as described in \secref{sec:flags}
can be (permanently) stored in different files
for convenient compilation, viewing and distribution.
To this end, the package defines a command
to pass on compilation to a different file:

%%%%%%%%%%%%%%%%%%%%%%%%%%%%%%%%%%%%%%%%
\DescribeMacro{\childdocforward}
The command |\childdocforward| redirects processing to
another source file:
%
\begin{center}
\begin{tabular}{l}
|\input{childdoc.def}|\\
|\childdocforward[|\textit{main}|]{|\textit{dest}|}|\\
\end{tabular}
\end{center}
%
The argument \textit{dest} is the destination file
(without extension).
It should be the main file or one of the child files.
Note that further \textsf{childdoc} directives
such as |\childdocof| and |\childdocforward|
in the indicated file will be processed in this form.
The optional argument \textit{main}
passes on directly to the main file \textit{main}
while pretending to compile the child \textit{dest}.
This form behaves as if \textit{dest}
issues |\childdocof{|\textit{main}|}| right away,
and no further \textsf{childdoc} directives will be processed.

%%%%%%%%%%%%%%%%%%%%%%%%%%%%%%%%%%%%%%%%
\DescribeMacro{\...prefix}
In the alternative form |\childdocforwardprefix|,
%
\begin{center}
\begin{tabular}{l}
|\input{childdoc.def}|\\
|\childdocforwardprefix[|\textit{main}|]{|\textit{prefix}|}{|\textit{dest}|}|
\end{tabular}
\end{center}
%
the destination file is determined by a pattern
depending on the current file:
To make this work, the current file must be called
`{\textit{prefix}\hspace{0.2em}\textit{suffix}}'
with \textit{prefix} matching precisely the argument.
Processing is then passed on to the file
`{\textit{dest}\hspace{0.2em}\textit{suffix}}'.
Surely, the same effect is achieved by
directly specifying the
argument `{\textit{dest}\hspace{0.2em}\textit{suffix}}'
in the first form.
However, that requires to set up a different file
for each child. With the alternative form of the command
all these files can have exactly the same content
which simplifies setting them up and maintaining them.

For example, the following file |draft.tex|
with a compilation flag |\version| as described in \secref{sec:flags}
compiles the main document as a draft:
%
\begin{center}
\begin{tabular}{l}
|\def\version{draft}|\\
|\input{childdoc.def}|\\
|\childdocforward{|\textit{main}|}|
\end{tabular}
\end{center}
%
Likewise, the following files |final|\textit{nn}|.tex|
compile the final version of the child document
|child|\textit{nn}|.tex|:
%
\begin{center}
\begin{tabular}{l}
|\def\version{final}|\\
|\input{childdoc.def}|\\
|\childdocforwardprefix{final}{child}|
\end{tabular}
\end{center}
%

Note that when several versions of a main file and/or of each child file
are to be generated, it may be convenient to set up a |Makefile| or
shell script to automatise the process.

%%%%%%%%%%%%%%%%%%%%%%%%%%%%%%%%%%%%%%%%%%%%%%%%%%%%%%%%%%%%%%%%%%%%%%%%%%%%%%%%
\subsection{Command Line Processing}
\label{sec:commandline}

The effect of redirection files can also be achieved by invoking
the \LaTeX{} compiler with a more elaborate command line.
Most conveniently this should be done as part
of a shell script or a |Makefile|.

When using \textsf{childdoc} in the main file, the following
command lines effectively perform a redirection
(note that depending on the shell being used,
backslashes may have to be doubled: `|\|' $\to$ `|\\|'):
%
\begin{center}
|... -jobname "|\textit{target}|" |\\|"|[\textit{flags}]%
|\input{childdoc.def}\childdocforward[|\textit{main}|]{|\textit{dest}|}"|
\end{center}
%
Here \textit{target} is the name of the output file,
\textit{main} is the name of the main file
and \textit{dest} is the name of the main or child file to be processed
(all filenames without extensions).
The optional argument \textit{main} can be omitted
if \textit{main} matches \textit{dest}.
Optionally, compilation \textit{flags} can be defined via |\def| commands.
This command line makes the \TeX{} engine believe
it is compiling the file \textit{target}
whose content is specified as the latter parameter.
The provided code then forwards the processing to
\textit{main} or \textit{dest} as described in \secref{sec:forward}.

%%%%%%%%%%%%%%%%%%%%%%%%%%%%%%%%%%%%%%%%%%%%%%%%%%%%%%%%%%%%%%%%%%%%%%%%%%%%%%%%
\subsection{Include by Input}
\label{sec:input}

Including child documents by |\include| has some restrictions by design.
Most notably, the content of a child document always occupies
its own set of pages; pages cannot be shared between child documents.
Usually, this behaviour makes perfect sense
because each child document contain an essential part of the document.
However, in some situations it may be desirable to compose
a document from a collection of parts
without having mandatory page breaks between then.
For this case, the package
provides a mechanism to include parts
by |\input| which can also be processed individually.
However, by construction this mechanism
requires manual handling of the content to be output.

%%%%%%%%%%%%%%%%%%%%%%%%%%%%%%%%%%%%%%%%
\DescribeMacro{\ifchilddocmanual}
The main file should be prepared as usual, see \secref{sec:include}.
However, the document body must make a distinction
between processing of an individual part and of the main document, e.g.:
%
\begin{center}
\begin{tabular}{l}
|\ifchilddocmanual|\\
|\input{\childdocname}|\\
|\||else|\\
\textit{document body with }|\input{|\textit{part}|}|\\
|\||fi|
\end{tabular}
\end{center}
%
The conditional |\ifchilddocmanual| is true whenever
a part to be included by |\input| is being compiled,
and the name of the part is stored in |\childdocname|.

%%%%%%%%%%%%%%%%%%%%%%%%%%%%%%%%%%%%%%%%
\DescribeMacro{\childdocby}
Each part to be included by |\input| should start with:
%
\begin{center}
\begin{tabular}{l}
|\input{childdoc.def}|\\
|\childdocby{|\textit{main}|}|\\
\end{tabular}
\end{center}
%
The directive |\childdocby| is similar to |\childdocof|
described in \secref{sec:include},
but the subsequent selection of content must be done manually.
To that end, both |\ifchilddoc| and |\ifchilddocmanual|
will be true upon processing of a part,
and the name of the part is stored in |\childdocname|.
Note that |\jobname| will be set to the filename of the current part
so that each part receives an individual |.aux| file
that does not interfere with the |.aux| file(s) of the main document.
This behaviour can be altered by the alternative form
|\childdocby[*]{|\textit{main}|}| (with a non-empty optional argument)
which uses the |.aux| file of the main document
by setting |\jobname| to \textit{main}.

%%%%%%%%%%%%%%%%%%%%%%%%%%%%%%%%%%%%%%%%%%%%%%%%%%%%%%%%%%%%%%%%%%%%%%%%%%%%%%%%
\subsection{Driver Development}
\label{sec:driver}

The \textsf{childdoc} mechanism can also be use for the development
of definition files such as \LaTeX{} styles or classes.
This case differs from the above setup with multiple parts
included by |\include| in that no |\includeonly| should be invoked.
This can be achieved by starting the include file
(before |\ProvidesPackage|) with:
%
\begin{center}
\begin{tabular}{l}
|\input{childdoc.def}|\\
|\childdocforward{|\textit{main}|}|\\
\end{tabular}
\end{center}
%
or alternatively with:
%
\begin{center}
\begin{tabular}{l}
|\input{childdoc.def}|\\
|\childdocby{|\textit{main}|}|\\
\end{tabular}
\end{center}
%
Both forms have slightly different effects as described above.
The main file is prepared as usual, see \secref{sec:include}.

%%%%%%%%%%%%%%%%%%%%%%%%%%%%%%%%%%%%%%%%%%%%%%%%%%%%%%%%%%%%%%%%%%%%%%%%%%%%%%%%
\subsection{Legacy Detection}
\label{sec:detection}

The directive |\childdocmain| in the main file can detect
whether the complete document or merely a child is to be compiled
even without using the directive |\childdocof|.
This method is deprecated because it is less robust
and there is no compelling reason to use it;
it is merely provided for backward compatibility
and it may be removed in future versions.

If the detection mechanism is to be used,
it is mandatory to correctly specify
the filename of the main file as the argument of |\childdocmain|:
%
\begin{center}
\begin{tabular}{l}
|\input{childdoc.def}|\\
|\childdocmain{|\textit{main}|}|\\
\end{tabular}
\end{center}
%
If |\jobname| does not match the argument \textit{main} of |\childdocmain|,
it is assumed that |\jobname| points to the child file to be compiled.
When using |\childdocmain| with the main file specified as argument,
it suffices to start a child file
with just |\input{|\textit{main}|}|
without loading of the package and using |\childdocof|.
If instead all processing is done
with the appropriate \textsf{childdoc} directives,
the argument of \textit{main} of |\childdocmain| can be empty.

An alternative version of the command line processing described
in \secref{sec:commandline} using the detection mechanism reads:
%
\begin{center}
|... -jobname "|\textit{target}|" "|[\textit{flags}]%
[|\def\jobname{|\textit{dest}|}|]|\input{|\textit{main}|}"|
\end{center}

%%%%%%%%%%%%%%%%%%%%%%%%%%%%%%%%%%%%%%%%%%%%%%%%%%%%%%%%%%%%%%%%%%%%%%%%%%%%%%%%
\subsection{Manual Code}
\label{sec:manual}

In case one cannot be certain whether the definitions file |childdoc.def|
is installed on the target \TeX{} distribution
and one prefers not to ship it,
it is conceivable to paste a few relevant commands into the sources.

To that end, drop all statements |\input{childdoc.def}|
and perform the replacements as outlined below.
Instead of |\childdocmain{|\textit{main}|}| add the following code
to the top of the main file:
%
\begin{center}
\begin{tabular}{l}
|\||ifdefined\childdocname\endinput\||fi\newif\ifchilddoc|\\
|\edef\childdocname{\scantokens\expandafter{\jobname\noexpand}}|\\
|\def\childdocmain{|\textit{main}|}\||ifx\childdocmain\childdocname\||else|\\
|\childdoctrue\includeonly{\childdocname}\let\jobname\childdocmain\||fi|\\
\end{tabular}
\end{center}
%
Instead of |\childdocof{|\textit{main}|}| just include the main file
at the top of each child file:
%
\begin{center}
|\input{|\textit{main}|}|
\end{center}
%
A simple redirection |\childdocforward{|\textit{dest}|}| is achieved by:
%
\begin{center}
|\def\jobname{|\textit{dest}|}\input{\jobname}|
\end{center}
%
The redirection with prefix
|\childdocforwardprefix[|\textit{prefix}|]{|\textit{dest}|}|
is accomplished by:
%
\begin{center}
\begin{tabular}{l}
|{\edef\jobname{\scantokens\expandafter{\jobname\noexpand}}|\\
|\def\redirectjob |\textit{prefix}|#1~~~{\gdef\jobname{|\textit{dest}|#1}}|\\
|\expandafter\redirectjob\jobname~~~}\input{\jobname}|
\end{tabular}
\end{center}

In an alternative approach,
child documents can be compiled by a specific command line
without additional code or specific definitions:
%
\begin{center}
|... -jobname "|\textit{target}|" "|[\textit{flags}]%
|\includeonly{|\textit{dest}|}\input{|\textit{main}|}"|
\end{center}
%

%%%%%%%%%%%%%%%%%%%%%%%%%%%%%%%%%%%%%%%%%%%%%%%%%%%%%%%%%%%%%%%%%%%%%%%%%%%%%%%%
%%%%%%%%%%%%%%%%%%%%%%%%%%%%%%%%%%%%%%%%%%%%%%%%%%%%%%%%%%%%%%%%%%%%%%%%%%%%%%%%
\section{Information}

%%%%%%%%%%%%%%%%%%%%%%%%%%%%%%%%%%%%%%%%%%%%%%%%%%%%%%%%%%%%%%%%%%%%%%%%%%%%%%%%
\subsection{Copyright}

Copyright \copyright{} 2017--2018 Niklas Beisert

This work may be distributed and/or modified under the
conditions of the \LaTeX{} Project Public License, either version 1.3
of this license or (at your option) any later version.
The latest version of this license is in
  \url{http://www.latex-project.org/lppl.txt}
and version 1.3 or later is part of all distributions of \LaTeX{}
version 2005/12/01 or later.

This work has the LPPL maintenance status `maintained'.

The Current Maintainer of this work is Niklas Beisert.

This work consists of the files |README.txt|, |childdoc.ins| and |childdoc.dtx|
as well as the derived files |childdoc.def|, |cdocsamp.tex|
with |cdocsch1.tex|, |cdocsch2.tex|, |cdocspt3.tex|, |cdocspt4.tex|,
|cdocsdrf.tex|, |cdocsfn1.tex|, |cdocsfn2.tex|
as well as |childdoc.pdf|.

%%%%%%%%%%%%%%%%%%%%%%%%%%%%%%%%%%%%%%%%%%%%%%%%%%%%%%%%%%%%%%%%%%%%%%%%%%%%%%%%
\subsection{Files and Installation}

The package consists of the files:
%
\begin{center}
\begin{tabular}{ll}
    |README.txt|   & readme file \\
    |childdoc.ins| & installation file \\
    |childdoc.dtx| & source file \\
    |childdoc.def| & definition file \\
    |cdocsamp.tex| & sample main file \\
    |cdocsch1.tex| & sample include file \\
    |cdocsch2.tex| & sample include file \\
    |cdocspt3.tex| & sample part file \\
    |cdocspt4.tex| & sample part file \\
    |cdocsdrf.tex| & sample redirection file \\
    |cdocsfn1.tex| & sample redirection file \\
    |cdocsfn2.tex| & sample redirection file \\
    |childdoc.pdf| & manual
\end{tabular}
\end{center}
%
The distribution consists of the files
|README.txt|, |childdoc.ins| and |childdoc.dtx|.
%
\begin{itemize}
\item
Run (pdf)\LaTeX{} on |childdoc.dtx|
to compile the manual |childdoc.pdf| (this file).
\item
Run \LaTeX{} on |childdoc.ins| to create the definitions file |childdoc.def|
and the sample |cdocsamp.tex| with include files
|cdocsch1.tex|, |cdocsch2.tex|, |cdocspt3.tex|, |cdocspt4.tex|,
|cdocsdrf.tex|, |cdocsfn1.tex|, |cdocsfn2.tex|.
Then copy the file |childdoc.def| to an appropriate directory of your \LaTeX{}
distribution, e.g.\ \textit{texmf-root}|/tex/latex/childdoc|.
\end{itemize}

%%%%%%%%%%%%%%%%%%%%%%%%%%%%%%%%%%%%%%%%%%%%%%%%%%%%%%%%%%%%%%%%%%%%%%%%%%%%%%%%
\subsection{Related CTAN Packages}

There are several other packages which offer a similar functionality:
%
\begin{itemize}
\item
The packages
\href{http://ctan.org/pkg/docmute}{\textsf{docmute}},
\href{http://ctan.org/pkg/includex}{\textsf{includex}} and
\href{http://ctan.org/pkg/standalone}{\textsf{standalone}}
provide commands to include only the document body of
a child file thus allowing both files to be compiled individually.
\item
The packages \href{http://ctan.org/pkg/subdocs}{\textsf{subdocs}}
and \href{http://ctan.org/pkg/subfiles}{\textsf{subfiles}}
provide structures in which the main and child documents can be
encapsulated and allowing them to be compiled individually.
The inclusion mechanism is different from the conventional |\include|.
\item
The package \href{http://ctan.org/pkg/combine}{\textsf{combine}}
is an elaborate solution to combine several documents into one.
\end{itemize}
%
See also the CTAN topic \href{http://ctan.org/topic/subdocs}{\textsf{subdocs}}
for further related packages.
The present package differs from the above solutions in that
a document structure constructed with the conventional |\include| mechanism
just needs two extra commands at the top of every file
such that all constituent files can be compiled individually.

%%%%%%%%%%%%%%%%%%%%%%%%%%%%%%%%%%%%%%%%%%%%%%%%%%%%%%%%%%%%%%%%%%%%%%%%%%%%%%%%
%\subsection{Feature Suggestions}
%
%The following is a list of features which may be useful for future
%versions of this package:
%%
%\begin{itemize}
%\item
%\ldots
%\end{itemize}

%%%%%%%%%%%%%%%%%%%%%%%%%%%%%%%%%%%%%%%%%%%%%%%%%%%%%%%%%%%%%%%%%%%%%%%%%%%%%%%%
\subsection{Revision History}

%%%%%%%%%%%%%%%%%%%%%%%%%%%%%%%%%%%%%%%%
\paragraph{v2.0:} 2018/12/30

\begin{itemize}
\item
immediate forward processing
\item
added |\childdocby| mechanism
\item
manual restructured
\end{itemize}

%%%%%%%%%%%%%%%%%%%%%%%%%%%%%%%%%%%%%%%%
\paragraph{v1.6:} 2018/01/17

\begin{itemize}
\item
application for development of include files
\item
corrections to manual
\end{itemize}

%%%%%%%%%%%%%%%%%%%%%%%%%%%%%%%%%%%%%%%%
\paragraph{v1.5:} 2017/05/21

\begin{itemize}
\item
more complete structuring introduced
\item
|\childdocof| introduced
\item
|\childdoc| renamed to |\childdocmain|
\item
|\childredirect| renamed to |\childdocforward| and |\childdocforwardprefix|
and functionality expanded
\end{itemize}

%%%%%%%%%%%%%%%%%%%%%%%%%%%%%%%%%%%%%%%%
\paragraph{v1.0:} 2017/04/27

\begin{itemize}
\item
manual and install package
\item
first version published on CTAN
\end{itemize}

%%%%%%%%%%%%%%%%%%%%%%%%%%%%%%%%%%%%%%%%
\paragraph{v0.6:} 2017/04/26

\begin{itemize}
\item
redirection mechanism added
\end{itemize}

%%%%%%%%%%%%%%%%%%%%%%%%%%%%%%%%%%%%%%%%
\paragraph{v0.5:} 2017/04/26

\begin{itemize}
\item
functionality in definition file
\end{itemize}


%%%%%%%%%%%%%%%%%%%%%%%%%%%%%%%%%%%%%%%%%%%%%%%%%%%%%%%%%%%%%%%%%%%%%%%%%%%%%%%%
%%%%%%%%%%%%%%%%%%%%%%%%%%%%%%%%%%%%%%%%%%%%%%%%%%%%%%%%%%%%%%%%%%%%%%%%%%%%%%%%
%%%%%%%%%%%%%%%%%%%%%%%%%%%%%%%%%%%%%%%%%%%%%%%%%%%%%%%%%%%%%%%%%%%%%%%%%%%%%%%%
\appendix

\settowidth\MacroIndent{\rmfamily\scriptsize 000\ }

 \DocInput{childdoc.dtx}

\end{document}
%</driver>
% \fi
%
% %%%%%%%%%%%%%%%%%%%%%%%%%%%%%%%%%%%%%%%%%%%%%%%%%%%%%%%%%%%%%%%%%%%%%%%%%%%%%%
% %%%%%%%%%%%%%%%%%%%%%%%%%%%%%%%%%%%%%%%%%%%%%%%%%%%%%%%%%%%%%%%%%%%%%%%%%%%%%%
% \section{Sample}
%\iffalse
%<*samplemain>
%\fi
%
% The following presents a sample document
% with two chapters, two parts, a title page,
% a compile flag as well as three forwarding files to set the flag.
% It consists of eight |.tex| files:
% \begin{center}
% \begin{tabular}{ll}
% |cdocsamp.tex|&main file\\
% |cdocsch1.tex|&include file for chapter 1\\
% |cdocsch2.tex|&include file for chapter 2\\
% |cdocspt3.tex|&include file for part 3\\
% |cdocspt4.tex|&include file for part 4\\
% |cdocsdrf.tex|&forwarding file for main file in draft mode\\
% |cdocsfi1.tex|&forwarding file for final version of chapter 1\\
% |cdocsfi2.tex|&forwarding file for final version of chapter 2\\
% \end{tabular}
% \end{center}
% Each of the eight files can be compiled directly by the \LaTeX{} compiler.
%
% %%%%%%%%%%%%%%%%%%%%%%%%%%%%%%%%%%%%%%
% \paragraph{Main File.}
%
% The main file is called |cdocsamp.tex|.
%
% Load the \textsf{childdoc} definitions and
% declare the filename for the main document:
%    \begin{macrocode}
\input{childdoc.def}
\childdocmain{}
%    \end{macrocode}

% Optional override for |\version| flag:
%    \begin{macrocode}
%%\ifchilddoc\else\providecommand{\version}{draft}\fi
%    \end{macrocode}

% Define the default values for the |\version| flag
% (|final| for the main file and |draft| for childs):
%    \begin{macrocode}
\ifchilddoc
\providecommand{\version}{draft}
\else
\providecommand{\version}{final}
\fi
%    \end{macrocode}

% Load the standard document class:
%    \begin{macrocode}
\documentclass[12pt]{article}
%    \end{macrocode}

% Start the document body:
%    \begin{macrocode}
\begin{document}
%    \end{macrocode}

% Declare a title page.
% Print title, part of document being processed and version flag:
%    \begin{macrocode}
\addtocounter{page}{-1}
\begin{center}
{\LARGE\bfseries{}childdoc example\par}
\vspace{1cm}
\ifchilddoc
\ifchilddocmanual part\else chapter\fi:
`\childdocname' of `\childdocjob'\par
\else
main document: `\childdocjob'\par
\fi
version: \version\par
\end{center}
\newpage
%    \end{macrocode}

% Manually include selected file,
% otherwise process as usual:
%    \begin{macrocode}
\ifchilddocmanual
\section*{part `\childdocname'}
\input{\childdocname}
\else
%    \end{macrocode}

% Include the two chapters:
%    \begin{macrocode}
\include{cdocsch1}
\include{cdocsch2}
%    \end{macrocode}

% Include the two parts unless only chapters should be displayed:
%    \begin{macrocode}
\ifchilddoc\else
\section{part three}
\input{cdocspt3}
\section{part four}
\input{cdocspt4}
\fi
%    \end{macrocode}

% Process as usual until here:
%    \begin{macrocode}
\fi
%    \end{macrocode}

% End of document body:
%    \begin{macrocode}
\end{document}
%    \end{macrocode}
%\iffalse
%</samplemain>
%\fi
%
% %%%%%%%%%%%%%%%%%%%%%%%%%%%%%%%%%%%%%%
% \paragraph{Chapter Include Files.}
%
% The include files are called |cdocsch1.tex| and |cdocsch2.tex|.
%
%\iffalse
%<*samplechap1|samplechap2>
%\fi

% Optional override for |\version| flag:
%    \begin{macrocode}
%%\providecommand{\version}{final}
%    \end{macrocode}

% Include the main document:
%    \begin{macrocode}
\input{childdoc.def}
\childdocof{cdocsamp}
%    \end{macrocode}

%\iffalse
%</samplechap1|samplechap2>
%\fi
%
%\iffalse
%<*samplechap1>
%\fi
% Some text for chapter 1:
%    \begin{macrocode}
\section{one}
some text in chapter one
%    \end{macrocode}

%\iffalse
%</samplechap1>
%\fi
% Some text for chapter 2:
%\iffalse
%<*samplechap2>
%\fi
%    \begin{macrocode}
\section{two}
more text in chapter two
%    \end{macrocode}

%\iffalse
%</samplechap2>
%\fi
%
% %%%%%%%%%%%%%%%%%%%%%%%%%%%%%%%%%%%%%%
% \paragraph{Part Include Files.}
%
% The include files are called |cdocspt3.tex| and |cdocspt4.tex|.
%
%\iffalse
%<*samplepart3|samplepart4>
%\fi

% Optional override for |\version| flag:
%    \begin{macrocode}
%%\providecommand{\version}{final}
%    \end{macrocode}

% Include the main document:
%    \begin{macrocode}
\input{childdoc.def}
\childdocby{cdocsamp}
%    \end{macrocode}

%\iffalse
%</samplepart3|samplepart4>
%\fi
%
%\iffalse
%<*samplepart3>
%\fi
% Some text for part 3:
%    \begin{macrocode}
some text in part three
%    \end{macrocode}

%\iffalse
%</samplepart3>
%\fi
% Some text for part 4:
%\iffalse
%<*samplepart4>
%\fi
%    \begin{macrocode}
more text in part four
%    \end{macrocode}

%\iffalse
%</samplepart4>
%\fi
%
% %%%%%%%%%%%%%%%%%%%%%%%%%%%%%%%%%%%%%%
% \paragraph{Forwarding for a Complete Draft.}
%
% The following forwarding file |cdocsdrf.tex|
% compiles the main document in draft mode:
%\iffalse
%<*sampledraft>
%\fi
%    \begin{macrocode}
\def\version{draft}
\input{childdoc.def}
\childdocforward{cdocsamp}
%    \end{macrocode}

%\iffalse
%</sampledraft>
%\fi
%
% %%%%%%%%%%%%%%%%%%%%%%%%%%%%%%%%%%%%%%
% \paragraph{Forwarding for Final Version of the Chapters.}
%
% The following forwarding files |cdocsfn1.tex| and |cdocsfn2.tex|
% (with identical content)
% compile the final versions of the child documents
% |cdocsch1.tex| and |cdocsch2.tex|, respectively:
%\iffalse
%<*samplefinal>
%\fi
%    \begin{macrocode}
\def\version{final}
\input{childdoc.def}
\childdocforwardprefix[cdocsamp]{cdocsfn}{cdocsch}
%    \end{macrocode}

%\iffalse
%</samplefinal>
%\fi
%
% %%%%%%%%%%%%%%%%%%%%%%%%%%%%%%%%%%%%%%
% \paragraph{Command Line Processing.}
%
% The following three command lines generate the output files
% |cdocscld|, |cdocscl1| and |cdocscl2|
% which should be identical to
% |cdocsdrf|, |cdocsch1| and |cdocsfn2|, respectively:
% \begin{center}
% \begin{tabular}{l}
% |latex -jobname cdocscld \|\\
% |  "\def\version{draft}\input{childdoc.def}\childdocforward{cdocsamp}"|\\
% |latex -jobname cdocscl1 \|\\
% |  "\input{childdoc.def}\childdocforward[cdocsamp]{cdocsch1}"|\\
% |latex -jobname cdocscl2 \|\\
% |  "\def\version{final}\input{childdoc.def}\childdocforward{cdocsch2}"|
% \end{tabular}
% \end{center}
% Note that the trailing backslash on each first line
% merely continues the input to the second line
% (for convenient cut ant paste).
% Furthermore, the command |latex| can be replaced by any
% of its alternative versions such as |pdflatex|.
%
% %%%%%%%%%%%%%%%%%%%%%%%%%%%%%%%%%%%%%%%%%%%%%%%%%%%%%%%%%%%%%%%%%%%%%%%%%%%%%%
% %%%%%%%%%%%%%%%%%%%%%%%%%%%%%%%%%%%%%%%%%%%%%%%%%%%%%%%%%%%%%%%%%%%%%%%%%%%%%%
% \section{Implementation}
%\iffalse
%<*package>
%\fi
%
% This section describes the definitions file |childdoc.def|.

% The definitions cannot be loaded using |\usepackage| or |\RequirePackage|
% which has a mechanism to prevent loading a style file more than once.
% When loading the definitions by means of |\input|
% multiple instances have to be prevented manually:
%\iffalse
%This code needs to be before the `\ProvidesFile' directive
%which is defined at the beginning of this file.
%Therefore it is also placed there and commented out here.
%</package>
%<*discard>
%\fi
%    \begin{macrocode}
\ifdefined\childdocmain\endinput\fi
%    \end{macrocode}
%\iffalse
%</discard>
%<*package>
%\fi
%
% \macro{\ifchilddoc}
% \macro{\ifchilddocmanual}
% The conditional |\ifchilddoc| tells whether a
% child (true) or main (false) document is being compiled.
% The conditional |\ifchilddocmanual| tells whether
% the |\includeonly| mechanism is used (false) or
% the selection of child files must be performed manually (true).
% The definitions initialise to false:
%    \begin{macrocode}
\newif\ifchilddoc
\newif\ifchilddocmanual
%    \end{macrocode}

% \macro{\childdocname}
% \macro{\childdocjob}
% The macro |\childdocname| stores the name of the main document
% to be compiled. The macro |\childdocjob| stores the name of
% the document on which the \LaTeX{} compiler was originally invoked.
% The content of |\jobname| cannot be compared
% to filenames specified in the source due to different catcodes.
% The following code rescans |\jobname|, stores the result
% in |\childdocname| and saves a copy in |\childdocjob|:
%    \begin{macrocode}
\edef\childdocname{\scantokens\expandafter{\jobname\noexpand}}
\let\childdocjob\childdocname
%    \end{macrocode}

% \macro{\childdocdisable}
% The macro |\childdocdisable| prevents the main file
% from being processed more than once.
% At this stage, the main document command |\childdocmain|
% is assumed to be called once again where it should do nothing.
% Any subsequent call to it should prevent
% a secondary processing of the main document
% It overwrites the forwarding commands
% |\childdocof| and |\childdocforward|
% with empty macros to prevent further inclusions of the main document:
%    \begin{macrocode}
\newcommand{\childdocdisable}
{
  \renewcommand{\childdocmain}[1]{\renewcommand{\childdocmain}[1]{\endinput}}
  \renewcommand{\childdocof}[1]{}
  \renewcommand{\childdocby}[2][]{}
  \renewcommand{\childdocforward}[2][]{}
  \renewcommand{\childdocdisable}{}
}
%    \end{macrocode}

% \macro{\childdocmain}
% The macro |\childdocmain| is to be called at the top of the main file
% with nothing or the main filename (without extension) as argument.
% First, it breaks loops.
% If the argument is not empty and does not match |\childdocname|
% (which is set by the first inclusion of |childdoc.def|),
% |\ifchilddoc| is set to true, |\includeonly| is applied to the child file
% and |\jobname| is set to the main file
% (for proper handling of |.aux| files):
%    \begin{macrocode}
\newcommand{\childdocmain}[1]
{
  \childdocdisable\childdocmain{}
  \if?#1?\else
    \begingroup
      \def\childdoctmp{#1}
      \ifx\childdoctmp\childdocname
        \def\childdoctmp{}
      \else
        \def\childdoctmp
        {
          \childdoctrue
          \includeonly{\childdocname}
          \def\childdocjob{#1}
          \def\jobname{#1}
        }
      \fi
      \expandafter
    \endgroup
    \childdoctmp
  \fi
}
%    \end{macrocode}

% \macro{\childdocof}
% The command |\childdocof| redirects
% compilation to the main file |#1|.
%    \begin{macrocode}
\newcommand{\childdocof}[1]
{
  \childdocdisable
  \childdoctrue
  \includeonly{\childdocname}
  \def\jobname{#1}
  \def\childdocjob{#1}
  \input{#1}
}
%    \end{macrocode}

% \macro{\childdocby}
% The command |\childdocby| ....
%    \begin{macrocode}
\newcommand{\childdocby}[2][]
{
  \childdocdisable
  \childdoctrue
  \childdocmanualtrue
  \if?#1?\else
    \def\jobname{#2}
  \fi
  \def\childdocjob{#2}
  \input{#2}
  \endinput
}
%    \end{macrocode}

% \macro{\childdocforward}
% The command |\childdocforward| redirects
% compilation to the main file or
% (if the optional argument is given) a child file.
% Parameters are set as if the main file
% or a child file starting with |\childdocof| was compiled.
% Then compilation is handed over to the main file:
%    \begin{macrocode}
\newcommand{\childdocforward}[2][]
{
  \begingroup
    \if?#1?
      \def\childdoctmp
      {
        \def\childdocname{#2}
        \def\childdocjob{#2}
        \def\jobname{#2}
        \input{#2}
        \endinput
      }
    \else
      \def\childdoctmp
      {
        \childdocdisable
        \def\childdocname{#2}
        \childdoctrue
        \includeonly{#2}
        \def\childdocjob{#1}
        \def\jobname{#1}
        \input{#1}
        \endinput
      }
    \fi
    \expandafter
  \endgroup
  \childdoctmp
}
%    \end{macrocode}

% \macro{\childdocforwardprefix}
% The command |\childdocforwardprefix| redirects
% compilation to the main or a child file by means of a pattern.
% The prefix |#1| in the current filename is replaced by |#2|
% and the suffix of the current filename is kept
% (it is assumed that the filename does not contain the substring `|~~~|'
% which is used as a delimiter).
% Compilation is handed over to the new file by |\childdocforward|:
%    \begin{macrocode}
\newcommand{\childdocforwardprefix}[3][]
{
  \begingroup
    \def\childdocextract #2##1~~~{\def\childdoctmp{\childdocforward[#1]{#3##1}}}
    \expandafter\childdocextract\childdocname~~~
    \expandafter
  \endgroup
  \childdoctmp
}
%    \end{macrocode}

% \macro{\childdoc}
% The deprecated macro |\childdoc| is a legacy version of |\childdocmain|:
%    \begin{macrocode}
\newcommand{\childdoc}{\childdocmain}
%    \end{macrocode}

% \macro{\childdocredirect}
% The deprecated macro |\childdocredirect| is a legacy version
% of |\childdocforward| and |\childdocforwardprefix|:
%    \begin{macrocode}
\newcommand{\childdocredirect}[2][]
{
  \begingroup
    \if?#1?
      \def\childdoctmp{\childdocforward{#2}}
    \else
      \def\childdoctmp{\childdocforwardprefix{#1}{#2}}
    \fi
    \expandafter
  \endgroup
  \childdoctmp
}
%    \end{macrocode}

%\iffalse
%</package>
%\fi
%
\endinput
|\\
|\childdocforwardprefix[|\textit{main}|]{|\textit{prefix}|}{|\textit{dest}|}|
\end{tabular}
\end{center}
%
the destination file is determined by a pattern
depending on the current file:
To make this work, the current file must be called
`{\textit{prefix}\hspace{0.2em}\textit{suffix}}'
with \textit{prefix} matching precisely the argument.
Processing is then passed on to the file
`{\textit{dest}\hspace{0.2em}\textit{suffix}}'.
Surely, the same effect is achieved by
directly specifying the
argument `{\textit{dest}\hspace{0.2em}\textit{suffix}}'
in the first form.
However, that requires to set up a different file
for each child. With the alternative form of the command
all these files can have exactly the same content
which simplifies setting them up and maintaining them.

For example, the following file |draft.tex|
with a compilation flag |\version| as described in \secref{sec:flags}
compiles the main document as a draft:
%
\begin{center}
\begin{tabular}{l}
|\def\version{draft}|\\
|% \iffalse
%
% childdoc.dtx Copyright (C) 2017-2018 Niklas Beisert
%
% This work may be distributed and/or modified under the
% conditions of the LaTeX Project Public License, either version 1.3
% of this license or (at your option) any later version.
% The latest version of this license is in
%   http://www.latex-project.org/lppl.txt
% and version 1.3 or later is part of all distributions of LaTeX
% version 2005/12/01 or later.
%
% This work has the LPPL maintenance status `maintained'.
%
% The Current Maintainer of this work is Niklas Beisert.
%
% This work consists of the files childdoc.dtx and childdoc.ins
% and the derived files childdoc.def and cdocsamp.tex with
% cdocsch1.tex, cdocsch2.tex, cdocsdrf.tex, cdocsfn1.tex, cdocsfn2.tex.
%
%<package>\ifdefined\childdocmain\endinput\fi
%<package>\ProvidesFile{childdoc.def}[2018/12/30 v2.0 child document driver]
%<samplemain>\ProvidesFile{cdocsamp.tex}[2018/12/30 v2.0 sample for childdoc]
%<*driver>
%\ProvidesFile{childdoc.drv}[2018/12/30 v2.0 childdoc reference manual file]
\PassOptionsToClass{10pt,a4paper}{article}
\documentclass{ltxdoc}

\usepackage[margin=35mm]{geometry}
\usepackage{hyperref}
\usepackage{hyperxmp}
\usepackage[usenames]{color}

\hypersetup{colorlinks=true}
\hypersetup{pdfstartview=FitH}
\hypersetup{pdfpagemode=UseNone}
\hypersetup{pdfsource={}}
\hypersetup{pdflang={en-UK}}
\hypersetup{pdfcopyright={Copyright 2017-2018 Niklas Beisert.
  This work may be distributed and/or modified under the
  conditions of the LaTeX Project Public License, either version 1.3
  of this license or (at your option) any later version.}}
\hypersetup{pdflicenseurl={http://www.latex-project.org/lppl.txt}}
\hypersetup{pdfcontactaddress={ETH Zurich, ITP, HIT K,
  Wolfgang-Pauli-Strasse 27}}
\hypersetup{pdfcontactpostcode={8093}}
\hypersetup{pdfcontactcity={Zurich}}
\hypersetup{pdfcontactcountry={Switzerland}}
\hypersetup{pdfcontactemail={nbeisert@itp.phys.ethz.ch}}
\hypersetup{pdfcontacturl={http://people.phys.ethz.ch/\xmptilde nbeisert/}}

\newcommand{\secref}[1]{\hyperref[#1]{section \ref*{#1}}}

\parskip1ex
\parindent0pt
\let\olditemize\itemize
\def\itemize{\olditemize\parskip0pt}

\begin{document}

\title{The \textsf{childdoc} Package}
\hypersetup{pdftitle={The childdoc Package}}
\author{Niklas Beisert\\[2ex]
  Institut f\"ur Theoretische Physik\\
  Eidgen\"ossische Technische Hochschule Z\"urich\\
  Wolfgang-Pauli-Strasse 27, 8093 Z\"urich, Switzerland\\[1ex]
  \href{mailto:nbeisert@itp.phys.ethz.ch}
  {\texttt{nbeisert@itp.phys.ethz.ch}}}
\hypersetup{pdfauthor={Niklas Beisert}}
\hypersetup{pdfsubject={Manual for the LaTeX2e Package childdoc}}
\date{30 December 2018, \textsf{v2.0}}
\maketitle

\begin{abstract}\noindent
\textsf{childdoc} is a \LaTeXe{} package
that enables the direct compilation
of document sections included by |\include|
to individual files.
\end{abstract}

\begingroup
\parskip0ex
\tableofcontents
\endgroup

%%%%%%%%%%%%%%%%%%%%%%%%%%%%%%%%%%%%%%%%%%%%%%%%%%%%%%%%%%%%%%%%%%%%%%%%%%%%%%%%
%%%%%%%%%%%%%%%%%%%%%%%%%%%%%%%%%%%%%%%%%%%%%%%%%%%%%%%%%%%%%%%%%%%%%%%%%%%%%%%%
\section{Introduction}

\LaTeX{} provides a mechanism to structure a large document (such as a book)
into a main file and several child files (containing the chapters)
using the |\include| command.
This mechanism is beneficial for documents
which span hundreds of pages in order to
make the source file(s) more manageable.
Moreover, compilation can be restricted to
selected child files by means of the |\includeonly| command.
The latter feature can be used to reduce the compilation time while editing
(this was significantly more useful in the earlier days of \LaTeX{})
or to generate a smaller document which is easier to navigate.
Another application of |\includeonly| is to generate
documents consisting of selected parts of the complete document.

However, there are a few drawbacks of the plain |\include| mechanism:
\begin{itemize}
\item
The child files cannot be compiled on their own,
they can only be compiled via the main file.
A naive editing environment
(such as a text editor with an option
to have the current file processed by \LaTeX)
may require one to switch to the main file before compiling;
attempting to compile the child file produces errors.
\item
The main file must be modified (each time)
to adjust the |\includeonly| command
to the present needs. This easily leaves the main file in a messy state.
\item
The generated document will always carry the filename
of the main document. This is inconvenient if
several child files are to be compiled and
to be kept for distribution.
\end{itemize}

The present package provides a simple interface
to make child files individually compilable by \LaTeX{}.
Compiling a child file then has the same effect as compiling
the main file with an |\includeonly| command
to select the appropriate child.
Moreover the generated document will carry the name of the child
rather than the main file.
This resolves all three above issues.

This feature is meant to make the editing of books,
thesis documents and lecture notes somewhat more convenient.
However, the package can also be used efficiently for
composing a series of documents (such as exercise sheets)
which are typically distributed individually.
It then assists the author in generating the individual documents
(potentially in different versions)
as well as a document containing the collected series.
Another application is in developing style files
or other kinds of included material
where compilation of the style file could redirect
to a sample or test file.

%%%%%%%%%%%%%%%%%%%%%%%%%%%%%%%%%%%%%%%%%%%%%%%%%%%%%%%%%%%%%%%%%%%%%%%%%%%%%%%%
%%%%%%%%%%%%%%%%%%%%%%%%%%%%%%%%%%%%%%%%%%%%%%%%%%%%%%%%%%%%%%%%%%%%%%%%%%%%%%%%
\section{Usage}

First of all, the package \textsf{childdoc} is \emph{not} a standard
\LaTeXe{} |.sty| style file! Therefore it needs to be invoked in
a non-standard way.

%%%%%%%%%%%%%%%%%%%%%%%%%%%%%%%%%%%%%%%%%%%%%%%%%%%%%%%%%%%%%%%%%%%%%%%%%%%%%%%%
\subsection{Included Files}
\label{sec:include}

%%%%%%%%%%%%%%%%%%%%%%%%%%%%%%%%%%%%%%%%
\DescribeMacro{\childdocmain}
To use the package, add the commands
\begin{center}
\begin{tabular}{l}
|\input{childdoc.def}|\\
|\childdocmain{}|\\
\end{tabular}
\end{center}
at the very top of the main \LaTeX{} file,
in particular \emph{before} the |\documentclass| statement!
The argument of |\childdocmain| should be left empty
(but it must be present).

%%%%%%%%%%%%%%%%%%%%%%%%%%%%%%%%%%%%%%%%
\DescribeMacro{\childdocof}
Furthermore, add the commands
\begin{center}
\begin{tabular}{l}
|\input{childdoc.def}|\\
|\childdocof{|\textit{main}|}|\\
\end{tabular}
\end{center}
at the top of every child file \textit{child}
which is included by |\include{|\textit{child}|}|
from within the main file
(or at least for those files to be compiled individually).
The argument \textit{main} must be the filename of the main file.

There are a couple of
considerations in setting up the main and child documents:

%%%%%%%%%%%%%%%%%%%%%%%%%%%%%%%%%%%%%%%%
\paragraph{Restrictions.}

Please note the following restrictions:
\begin{itemize}
\item
|\childdocmain| must be called with one argument \textit{main}
to ensure compatibility with earlier version of the package.
It must either be empty (|\childdocmain{}|)
or precisely match the filename of the main file in which it is specified.
See \secref{sec:detection} for further information.
\item
The filename \textit{main} must be specified without the |.tex| extension.
\item
The filename \textit{main} is case sensitive
(even in case-insensitive file systems)
due to internal string comparison.
\item
The argument \textit{main} should be fully expanded, it cannot be a macro.
\item
Subdirectories and special characters should be avoided in filenames.
\item
The command |\childdocmain{|\textit{main}|}| must be followed by a whitespace.
It should not be followed immediately by another command
or by a comment mark `|%|'.
This is because the \TeX{} parser reads the token immediately following
the argument of |\childdocmain| and puts it
at the beginning of every child section;
however, a white\-space is ignored.
\end{itemize}

%%%%%%%%%%%%%%%%%%%%%%%%%%%%%%%%%%%%%%%%
\paragraph{Content of Main File.}

It is advisable to place all content in the child files included by |\include|.
Any output contained in the main file will appear in all child documents
unless suppressed manually;
it cannot be suppressed automatically by the |\includeonly| directive
and thus should normally be avoided.
A method to include some content in the main file
by means of conditional processing is described in \secref{sec:conditional}.

%%%%%%%%%%%%%%%%%%%%%%%%%%%%%%%%%%%%%%%%
\paragraph{Page Numbering.}

When only a part of the document is compiled,
the appropriate numbering of pages
(as well as other status parameters)
is determined from the |.aux| files.
The latter contain information from previous passes.
However this information needs to propagate through
all intermediate child documents.
Therefore the page numbering in child documents may well
be inconsistent until the complete document is compiled at least once.

A useful (if unconventional) way to always ensure a consistent
page numbering is to restart the numbering in each child document
and denote the pages by `\textit{child}|.|\textit{page}'
where \textit{child} represents the chapter/section number of the child file.
This can be achieved by the command
|\numberwithin{page}{|\textit{child}|}|
of the \textsf{amsmath} package
where \textit{child} can be |chapter| or |section|
depending on the chosen structuring.
Alternatively, one can modify the macro |\thepage| appropriately
and reset the counter |page| at the start of each child file.

%%%%%%%%%%%%%%%%%%%%%%%%%%%%%%%%%%%%%%%%%%%%%%%%%%%%%%%%%%%%%%%%%%%%%%%%%%%%%%%%
\subsection{Conditional Processing}
\label{sec:conditional}

The package provides a mechanism to compile different versions
of a document. To customise the versions further some conditional processing
can come in handy to distinguish which version is being compiled.
The package provides two macros to describe the compilation context:

%%%%%%%%%%%%%%%%%%%%%%%%%%%%%%%%%%%%%%%%
\DescribeMacro{\ifchilddoc}
The conditional |\ifchilddoc| distinguishes between the compilation of
child documents and the main document:
%
\begin{center}
|\ifchilddoc |\textit{child-code}| |[|\||else |\textit{main-code}]| \||fi|
\end{center}

%%%%%%%%%%%%%%%%%%%%%%%%%%%%%%%%%%%%%%%%
\DescribeMacro{\childdocname}
\DescribeMacro{\childdocjob}
The macro |\childdocname| contains the filename (without extension)
of the main or child file being processed.
Note that |\childdocjob| will always contain the name of the main file.

%%%%%%%%%%%%%%%%%%%%%%%%%%%%%%%%%%%%%%%%
\paragraph{Title Page.}

Conditional processing can be used to include a title or banner page
in the main document when proper precautions are taken.
Importantly, the code in the main file should ensure that the page counter
(as well as other status parameters which are stored in the |.aux| files)
takes the same value after the conditional processing.
Otherwise the page numbers may take divergent values
depending on which part is compiled.

For example, a title page could be declared by:
%
\begin{center}
\begin{tabular}{l}
|\ifchilddoc\||else|\\
|\addtocounter{page}{-1}|\\
\textit{code for title page}\\
|\newpage|\\
|\||fi|
\end{tabular}
\end{center}
%
A banner page for the child documents can be generated by:
%
\begin{center}
\begin{tabular}{l}
|\ifchilddoc|\\
|\addtocounter{page}{-1}|\\
\textit{code for banner page}\\
|\newpage|\\
|\||fi|
\end{tabular}
\end{center}
%
Here one could write a message such as:
\begin{center}
|This is the part \childdocname{} of \childdocjob{}.|
\end{center}

%%%%%%%%%%%%%%%%%%%%%%%%%%%%%%%%%%%%%%%%%%%%%%%%%%%%%%%%%%%%%%%%%%%%%%%%%%%%%%%%
\subsection{Flags}
\label{sec:flags}

The package makes it easy to generate different versions
of the main or child documents.
To this end compilation flags can be defined
and assigned different default values.
They will be particularly useful in conjunction
with the forwarding mechanism described in \secref{sec:forward}.

For example, it may be useful to have a flag |\version|
which can be set to |draft| or |final|.
The document source will contain some conditional code
depending on the value of |\version|.
Suppose further, the flag should default to |final| for the main file
and to |draft| for child files
which is a natural assignment for editing the document.
This is achieved by placing the following code
in the preamble of the main document
(below the |\childdocmain| directive):
%
\begin{center}
\begin{tabular}{l}
|\ifchilddoc|\\
|\providecommand{\version}{draft}|\\
|\||else|\\
|\providecommand{\version}{final}|\\
|\||fi|
\end{tabular}
\end{center}
%
The definition by |\providecommand| makes sure
that previous definitions are not overwritten.
Further statements |\providecommand{\version}{...}|
can thus be added before the above code to override it.

For the main file, one might add a line
(between |\childdocmain| and the above block)
%
\begin{center}
|%\ifchilddoc\||else\providecommand{\version}{draft}\||fi|
\end{center}
%
which can be uncommented to produce a draft version.
Likewise one can add a line to the very top of a child file
(above the |\childdocof{|\textit{main}|}| directive)
%
\begin{center}
|%\providecommand{\version}{final}|
\end{center}
%
which can be uncommented to produce the final version of this child document.

%%%%%%%%%%%%%%%%%%%%%%%%%%%%%%%%%%%%%%%%%%%%%%%%%%%%%%%%%%%%%%%%%%%%%%%%%%%%%%%%
\subsection{Forwarding}
\label{sec:forward}

Different versions of the main or child documents
using compilation flags as described in \secref{sec:flags}
can be (permanently) stored in different files
for convenient compilation, viewing and distribution.
To this end, the package defines a command
to pass on compilation to a different file:

%%%%%%%%%%%%%%%%%%%%%%%%%%%%%%%%%%%%%%%%
\DescribeMacro{\childdocforward}
The command |\childdocforward| redirects processing to
another source file:
%
\begin{center}
\begin{tabular}{l}
|\input{childdoc.def}|\\
|\childdocforward[|\textit{main}|]{|\textit{dest}|}|\\
\end{tabular}
\end{center}
%
The argument \textit{dest} is the destination file
(without extension).
It should be the main file or one of the child files.
Note that further \textsf{childdoc} directives
such as |\childdocof| and |\childdocforward|
in the indicated file will be processed in this form.
The optional argument \textit{main}
passes on directly to the main file \textit{main}
while pretending to compile the child \textit{dest}.
This form behaves as if \textit{dest}
issues |\childdocof{|\textit{main}|}| right away,
and no further \textsf{childdoc} directives will be processed.

%%%%%%%%%%%%%%%%%%%%%%%%%%%%%%%%%%%%%%%%
\DescribeMacro{\...prefix}
In the alternative form |\childdocforwardprefix|,
%
\begin{center}
\begin{tabular}{l}
|\input{childdoc.def}|\\
|\childdocforwardprefix[|\textit{main}|]{|\textit{prefix}|}{|\textit{dest}|}|
\end{tabular}
\end{center}
%
the destination file is determined by a pattern
depending on the current file:
To make this work, the current file must be called
`{\textit{prefix}\hspace{0.2em}\textit{suffix}}'
with \textit{prefix} matching precisely the argument.
Processing is then passed on to the file
`{\textit{dest}\hspace{0.2em}\textit{suffix}}'.
Surely, the same effect is achieved by
directly specifying the
argument `{\textit{dest}\hspace{0.2em}\textit{suffix}}'
in the first form.
However, that requires to set up a different file
for each child. With the alternative form of the command
all these files can have exactly the same content
which simplifies setting them up and maintaining them.

For example, the following file |draft.tex|
with a compilation flag |\version| as described in \secref{sec:flags}
compiles the main document as a draft:
%
\begin{center}
\begin{tabular}{l}
|\def\version{draft}|\\
|\input{childdoc.def}|\\
|\childdocforward{|\textit{main}|}|
\end{tabular}
\end{center}
%
Likewise, the following files |final|\textit{nn}|.tex|
compile the final version of the child document
|child|\textit{nn}|.tex|:
%
\begin{center}
\begin{tabular}{l}
|\def\version{final}|\\
|\input{childdoc.def}|\\
|\childdocforwardprefix{final}{child}|
\end{tabular}
\end{center}
%

Note that when several versions of a main file and/or of each child file
are to be generated, it may be convenient to set up a |Makefile| or
shell script to automatise the process.

%%%%%%%%%%%%%%%%%%%%%%%%%%%%%%%%%%%%%%%%%%%%%%%%%%%%%%%%%%%%%%%%%%%%%%%%%%%%%%%%
\subsection{Command Line Processing}
\label{sec:commandline}

The effect of redirection files can also be achieved by invoking
the \LaTeX{} compiler with a more elaborate command line.
Most conveniently this should be done as part
of a shell script or a |Makefile|.

When using \textsf{childdoc} in the main file, the following
command lines effectively perform a redirection
(note that depending on the shell being used,
backslashes may have to be doubled: `|\|' $\to$ `|\\|'):
%
\begin{center}
|... -jobname "|\textit{target}|" |\\|"|[\textit{flags}]%
|\input{childdoc.def}\childdocforward[|\textit{main}|]{|\textit{dest}|}"|
\end{center}
%
Here \textit{target} is the name of the output file,
\textit{main} is the name of the main file
and \textit{dest} is the name of the main or child file to be processed
(all filenames without extensions).
The optional argument \textit{main} can be omitted
if \textit{main} matches \textit{dest}.
Optionally, compilation \textit{flags} can be defined via |\def| commands.
This command line makes the \TeX{} engine believe
it is compiling the file \textit{target}
whose content is specified as the latter parameter.
The provided code then forwards the processing to
\textit{main} or \textit{dest} as described in \secref{sec:forward}.

%%%%%%%%%%%%%%%%%%%%%%%%%%%%%%%%%%%%%%%%%%%%%%%%%%%%%%%%%%%%%%%%%%%%%%%%%%%%%%%%
\subsection{Include by Input}
\label{sec:input}

Including child documents by |\include| has some restrictions by design.
Most notably, the content of a child document always occupies
its own set of pages; pages cannot be shared between child documents.
Usually, this behaviour makes perfect sense
because each child document contain an essential part of the document.
However, in some situations it may be desirable to compose
a document from a collection of parts
without having mandatory page breaks between then.
For this case, the package
provides a mechanism to include parts
by |\input| which can also be processed individually.
However, by construction this mechanism
requires manual handling of the content to be output.

%%%%%%%%%%%%%%%%%%%%%%%%%%%%%%%%%%%%%%%%
\DescribeMacro{\ifchilddocmanual}
The main file should be prepared as usual, see \secref{sec:include}.
However, the document body must make a distinction
between processing of an individual part and of the main document, e.g.:
%
\begin{center}
\begin{tabular}{l}
|\ifchilddocmanual|\\
|\input{\childdocname}|\\
|\||else|\\
\textit{document body with }|\input{|\textit{part}|}|\\
|\||fi|
\end{tabular}
\end{center}
%
The conditional |\ifchilddocmanual| is true whenever
a part to be included by |\input| is being compiled,
and the name of the part is stored in |\childdocname|.

%%%%%%%%%%%%%%%%%%%%%%%%%%%%%%%%%%%%%%%%
\DescribeMacro{\childdocby}
Each part to be included by |\input| should start with:
%
\begin{center}
\begin{tabular}{l}
|\input{childdoc.def}|\\
|\childdocby{|\textit{main}|}|\\
\end{tabular}
\end{center}
%
The directive |\childdocby| is similar to |\childdocof|
described in \secref{sec:include},
but the subsequent selection of content must be done manually.
To that end, both |\ifchilddoc| and |\ifchilddocmanual|
will be true upon processing of a part,
and the name of the part is stored in |\childdocname|.
Note that |\jobname| will be set to the filename of the current part
so that each part receives an individual |.aux| file
that does not interfere with the |.aux| file(s) of the main document.
This behaviour can be altered by the alternative form
|\childdocby[*]{|\textit{main}|}| (with a non-empty optional argument)
which uses the |.aux| file of the main document
by setting |\jobname| to \textit{main}.

%%%%%%%%%%%%%%%%%%%%%%%%%%%%%%%%%%%%%%%%%%%%%%%%%%%%%%%%%%%%%%%%%%%%%%%%%%%%%%%%
\subsection{Driver Development}
\label{sec:driver}

The \textsf{childdoc} mechanism can also be use for the development
of definition files such as \LaTeX{} styles or classes.
This case differs from the above setup with multiple parts
included by |\include| in that no |\includeonly| should be invoked.
This can be achieved by starting the include file
(before |\ProvidesPackage|) with:
%
\begin{center}
\begin{tabular}{l}
|\input{childdoc.def}|\\
|\childdocforward{|\textit{main}|}|\\
\end{tabular}
\end{center}
%
or alternatively with:
%
\begin{center}
\begin{tabular}{l}
|\input{childdoc.def}|\\
|\childdocby{|\textit{main}|}|\\
\end{tabular}
\end{center}
%
Both forms have slightly different effects as described above.
The main file is prepared as usual, see \secref{sec:include}.

%%%%%%%%%%%%%%%%%%%%%%%%%%%%%%%%%%%%%%%%%%%%%%%%%%%%%%%%%%%%%%%%%%%%%%%%%%%%%%%%
\subsection{Legacy Detection}
\label{sec:detection}

The directive |\childdocmain| in the main file can detect
whether the complete document or merely a child is to be compiled
even without using the directive |\childdocof|.
This method is deprecated because it is less robust
and there is no compelling reason to use it;
it is merely provided for backward compatibility
and it may be removed in future versions.

If the detection mechanism is to be used,
it is mandatory to correctly specify
the filename of the main file as the argument of |\childdocmain|:
%
\begin{center}
\begin{tabular}{l}
|\input{childdoc.def}|\\
|\childdocmain{|\textit{main}|}|\\
\end{tabular}
\end{center}
%
If |\jobname| does not match the argument \textit{main} of |\childdocmain|,
it is assumed that |\jobname| points to the child file to be compiled.
When using |\childdocmain| with the main file specified as argument,
it suffices to start a child file
with just |\input{|\textit{main}|}|
without loading of the package and using |\childdocof|.
If instead all processing is done
with the appropriate \textsf{childdoc} directives,
the argument of \textit{main} of |\childdocmain| can be empty.

An alternative version of the command line processing described
in \secref{sec:commandline} using the detection mechanism reads:
%
\begin{center}
|... -jobname "|\textit{target}|" "|[\textit{flags}]%
[|\def\jobname{|\textit{dest}|}|]|\input{|\textit{main}|}"|
\end{center}

%%%%%%%%%%%%%%%%%%%%%%%%%%%%%%%%%%%%%%%%%%%%%%%%%%%%%%%%%%%%%%%%%%%%%%%%%%%%%%%%
\subsection{Manual Code}
\label{sec:manual}

In case one cannot be certain whether the definitions file |childdoc.def|
is installed on the target \TeX{} distribution
and one prefers not to ship it,
it is conceivable to paste a few relevant commands into the sources.

To that end, drop all statements |\input{childdoc.def}|
and perform the replacements as outlined below.
Instead of |\childdocmain{|\textit{main}|}| add the following code
to the top of the main file:
%
\begin{center}
\begin{tabular}{l}
|\||ifdefined\childdocname\endinput\||fi\newif\ifchilddoc|\\
|\edef\childdocname{\scantokens\expandafter{\jobname\noexpand}}|\\
|\def\childdocmain{|\textit{main}|}\||ifx\childdocmain\childdocname\||else|\\
|\childdoctrue\includeonly{\childdocname}\let\jobname\childdocmain\||fi|\\
\end{tabular}
\end{center}
%
Instead of |\childdocof{|\textit{main}|}| just include the main file
at the top of each child file:
%
\begin{center}
|\input{|\textit{main}|}|
\end{center}
%
A simple redirection |\childdocforward{|\textit{dest}|}| is achieved by:
%
\begin{center}
|\def\jobname{|\textit{dest}|}\input{\jobname}|
\end{center}
%
The redirection with prefix
|\childdocforwardprefix[|\textit{prefix}|]{|\textit{dest}|}|
is accomplished by:
%
\begin{center}
\begin{tabular}{l}
|{\edef\jobname{\scantokens\expandafter{\jobname\noexpand}}|\\
|\def\redirectjob |\textit{prefix}|#1~~~{\gdef\jobname{|\textit{dest}|#1}}|\\
|\expandafter\redirectjob\jobname~~~}\input{\jobname}|
\end{tabular}
\end{center}

In an alternative approach,
child documents can be compiled by a specific command line
without additional code or specific definitions:
%
\begin{center}
|... -jobname "|\textit{target}|" "|[\textit{flags}]%
|\includeonly{|\textit{dest}|}\input{|\textit{main}|}"|
\end{center}
%

%%%%%%%%%%%%%%%%%%%%%%%%%%%%%%%%%%%%%%%%%%%%%%%%%%%%%%%%%%%%%%%%%%%%%%%%%%%%%%%%
%%%%%%%%%%%%%%%%%%%%%%%%%%%%%%%%%%%%%%%%%%%%%%%%%%%%%%%%%%%%%%%%%%%%%%%%%%%%%%%%
\section{Information}

%%%%%%%%%%%%%%%%%%%%%%%%%%%%%%%%%%%%%%%%%%%%%%%%%%%%%%%%%%%%%%%%%%%%%%%%%%%%%%%%
\subsection{Copyright}

Copyright \copyright{} 2017--2018 Niklas Beisert

This work may be distributed and/or modified under the
conditions of the \LaTeX{} Project Public License, either version 1.3
of this license or (at your option) any later version.
The latest version of this license is in
  \url{http://www.latex-project.org/lppl.txt}
and version 1.3 or later is part of all distributions of \LaTeX{}
version 2005/12/01 or later.

This work has the LPPL maintenance status `maintained'.

The Current Maintainer of this work is Niklas Beisert.

This work consists of the files |README.txt|, |childdoc.ins| and |childdoc.dtx|
as well as the derived files |childdoc.def|, |cdocsamp.tex|
with |cdocsch1.tex|, |cdocsch2.tex|, |cdocspt3.tex|, |cdocspt4.tex|,
|cdocsdrf.tex|, |cdocsfn1.tex|, |cdocsfn2.tex|
as well as |childdoc.pdf|.

%%%%%%%%%%%%%%%%%%%%%%%%%%%%%%%%%%%%%%%%%%%%%%%%%%%%%%%%%%%%%%%%%%%%%%%%%%%%%%%%
\subsection{Files and Installation}

The package consists of the files:
%
\begin{center}
\begin{tabular}{ll}
    |README.txt|   & readme file \\
    |childdoc.ins| & installation file \\
    |childdoc.dtx| & source file \\
    |childdoc.def| & definition file \\
    |cdocsamp.tex| & sample main file \\
    |cdocsch1.tex| & sample include file \\
    |cdocsch2.tex| & sample include file \\
    |cdocspt3.tex| & sample part file \\
    |cdocspt4.tex| & sample part file \\
    |cdocsdrf.tex| & sample redirection file \\
    |cdocsfn1.tex| & sample redirection file \\
    |cdocsfn2.tex| & sample redirection file \\
    |childdoc.pdf| & manual
\end{tabular}
\end{center}
%
The distribution consists of the files
|README.txt|, |childdoc.ins| and |childdoc.dtx|.
%
\begin{itemize}
\item
Run (pdf)\LaTeX{} on |childdoc.dtx|
to compile the manual |childdoc.pdf| (this file).
\item
Run \LaTeX{} on |childdoc.ins| to create the definitions file |childdoc.def|
and the sample |cdocsamp.tex| with include files
|cdocsch1.tex|, |cdocsch2.tex|, |cdocspt3.tex|, |cdocspt4.tex|,
|cdocsdrf.tex|, |cdocsfn1.tex|, |cdocsfn2.tex|.
Then copy the file |childdoc.def| to an appropriate directory of your \LaTeX{}
distribution, e.g.\ \textit{texmf-root}|/tex/latex/childdoc|.
\end{itemize}

%%%%%%%%%%%%%%%%%%%%%%%%%%%%%%%%%%%%%%%%%%%%%%%%%%%%%%%%%%%%%%%%%%%%%%%%%%%%%%%%
\subsection{Related CTAN Packages}

There are several other packages which offer a similar functionality:
%
\begin{itemize}
\item
The packages
\href{http://ctan.org/pkg/docmute}{\textsf{docmute}},
\href{http://ctan.org/pkg/includex}{\textsf{includex}} and
\href{http://ctan.org/pkg/standalone}{\textsf{standalone}}
provide commands to include only the document body of
a child file thus allowing both files to be compiled individually.
\item
The packages \href{http://ctan.org/pkg/subdocs}{\textsf{subdocs}}
and \href{http://ctan.org/pkg/subfiles}{\textsf{subfiles}}
provide structures in which the main and child documents can be
encapsulated and allowing them to be compiled individually.
The inclusion mechanism is different from the conventional |\include|.
\item
The package \href{http://ctan.org/pkg/combine}{\textsf{combine}}
is an elaborate solution to combine several documents into one.
\end{itemize}
%
See also the CTAN topic \href{http://ctan.org/topic/subdocs}{\textsf{subdocs}}
for further related packages.
The present package differs from the above solutions in that
a document structure constructed with the conventional |\include| mechanism
just needs two extra commands at the top of every file
such that all constituent files can be compiled individually.

%%%%%%%%%%%%%%%%%%%%%%%%%%%%%%%%%%%%%%%%%%%%%%%%%%%%%%%%%%%%%%%%%%%%%%%%%%%%%%%%
%\subsection{Feature Suggestions}
%
%The following is a list of features which may be useful for future
%versions of this package:
%%
%\begin{itemize}
%\item
%\ldots
%\end{itemize}

%%%%%%%%%%%%%%%%%%%%%%%%%%%%%%%%%%%%%%%%%%%%%%%%%%%%%%%%%%%%%%%%%%%%%%%%%%%%%%%%
\subsection{Revision History}

%%%%%%%%%%%%%%%%%%%%%%%%%%%%%%%%%%%%%%%%
\paragraph{v2.0:} 2018/12/30

\begin{itemize}
\item
immediate forward processing
\item
added |\childdocby| mechanism
\item
manual restructured
\end{itemize}

%%%%%%%%%%%%%%%%%%%%%%%%%%%%%%%%%%%%%%%%
\paragraph{v1.6:} 2018/01/17

\begin{itemize}
\item
application for development of include files
\item
corrections to manual
\end{itemize}

%%%%%%%%%%%%%%%%%%%%%%%%%%%%%%%%%%%%%%%%
\paragraph{v1.5:} 2017/05/21

\begin{itemize}
\item
more complete structuring introduced
\item
|\childdocof| introduced
\item
|\childdoc| renamed to |\childdocmain|
\item
|\childredirect| renamed to |\childdocforward| and |\childdocforwardprefix|
and functionality expanded
\end{itemize}

%%%%%%%%%%%%%%%%%%%%%%%%%%%%%%%%%%%%%%%%
\paragraph{v1.0:} 2017/04/27

\begin{itemize}
\item
manual and install package
\item
first version published on CTAN
\end{itemize}

%%%%%%%%%%%%%%%%%%%%%%%%%%%%%%%%%%%%%%%%
\paragraph{v0.6:} 2017/04/26

\begin{itemize}
\item
redirection mechanism added
\end{itemize}

%%%%%%%%%%%%%%%%%%%%%%%%%%%%%%%%%%%%%%%%
\paragraph{v0.5:} 2017/04/26

\begin{itemize}
\item
functionality in definition file
\end{itemize}


%%%%%%%%%%%%%%%%%%%%%%%%%%%%%%%%%%%%%%%%%%%%%%%%%%%%%%%%%%%%%%%%%%%%%%%%%%%%%%%%
%%%%%%%%%%%%%%%%%%%%%%%%%%%%%%%%%%%%%%%%%%%%%%%%%%%%%%%%%%%%%%%%%%%%%%%%%%%%%%%%
%%%%%%%%%%%%%%%%%%%%%%%%%%%%%%%%%%%%%%%%%%%%%%%%%%%%%%%%%%%%%%%%%%%%%%%%%%%%%%%%
\appendix

\settowidth\MacroIndent{\rmfamily\scriptsize 000\ }

 \DocInput{childdoc.dtx}

\end{document}
%</driver>
% \fi
%
% %%%%%%%%%%%%%%%%%%%%%%%%%%%%%%%%%%%%%%%%%%%%%%%%%%%%%%%%%%%%%%%%%%%%%%%%%%%%%%
% %%%%%%%%%%%%%%%%%%%%%%%%%%%%%%%%%%%%%%%%%%%%%%%%%%%%%%%%%%%%%%%%%%%%%%%%%%%%%%
% \section{Sample}
%\iffalse
%<*samplemain>
%\fi
%
% The following presents a sample document
% with two chapters, two parts, a title page,
% a compile flag as well as three forwarding files to set the flag.
% It consists of eight |.tex| files:
% \begin{center}
% \begin{tabular}{ll}
% |cdocsamp.tex|&main file\\
% |cdocsch1.tex|&include file for chapter 1\\
% |cdocsch2.tex|&include file for chapter 2\\
% |cdocspt3.tex|&include file for part 3\\
% |cdocspt4.tex|&include file for part 4\\
% |cdocsdrf.tex|&forwarding file for main file in draft mode\\
% |cdocsfi1.tex|&forwarding file for final version of chapter 1\\
% |cdocsfi2.tex|&forwarding file for final version of chapter 2\\
% \end{tabular}
% \end{center}
% Each of the eight files can be compiled directly by the \LaTeX{} compiler.
%
% %%%%%%%%%%%%%%%%%%%%%%%%%%%%%%%%%%%%%%
% \paragraph{Main File.}
%
% The main file is called |cdocsamp.tex|.
%
% Load the \textsf{childdoc} definitions and
% declare the filename for the main document:
%    \begin{macrocode}
\input{childdoc.def}
\childdocmain{}
%    \end{macrocode}

% Optional override for |\version| flag:
%    \begin{macrocode}
%%\ifchilddoc\else\providecommand{\version}{draft}\fi
%    \end{macrocode}

% Define the default values for the |\version| flag
% (|final| for the main file and |draft| for childs):
%    \begin{macrocode}
\ifchilddoc
\providecommand{\version}{draft}
\else
\providecommand{\version}{final}
\fi
%    \end{macrocode}

% Load the standard document class:
%    \begin{macrocode}
\documentclass[12pt]{article}
%    \end{macrocode}

% Start the document body:
%    \begin{macrocode}
\begin{document}
%    \end{macrocode}

% Declare a title page.
% Print title, part of document being processed and version flag:
%    \begin{macrocode}
\addtocounter{page}{-1}
\begin{center}
{\LARGE\bfseries{}childdoc example\par}
\vspace{1cm}
\ifchilddoc
\ifchilddocmanual part\else chapter\fi:
`\childdocname' of `\childdocjob'\par
\else
main document: `\childdocjob'\par
\fi
version: \version\par
\end{center}
\newpage
%    \end{macrocode}

% Manually include selected file,
% otherwise process as usual:
%    \begin{macrocode}
\ifchilddocmanual
\section*{part `\childdocname'}
\input{\childdocname}
\else
%    \end{macrocode}

% Include the two chapters:
%    \begin{macrocode}
\include{cdocsch1}
\include{cdocsch2}
%    \end{macrocode}

% Include the two parts unless only chapters should be displayed:
%    \begin{macrocode}
\ifchilddoc\else
\section{part three}
\input{cdocspt3}
\section{part four}
\input{cdocspt4}
\fi
%    \end{macrocode}

% Process as usual until here:
%    \begin{macrocode}
\fi
%    \end{macrocode}

% End of document body:
%    \begin{macrocode}
\end{document}
%    \end{macrocode}
%\iffalse
%</samplemain>
%\fi
%
% %%%%%%%%%%%%%%%%%%%%%%%%%%%%%%%%%%%%%%
% \paragraph{Chapter Include Files.}
%
% The include files are called |cdocsch1.tex| and |cdocsch2.tex|.
%
%\iffalse
%<*samplechap1|samplechap2>
%\fi

% Optional override for |\version| flag:
%    \begin{macrocode}
%%\providecommand{\version}{final}
%    \end{macrocode}

% Include the main document:
%    \begin{macrocode}
\input{childdoc.def}
\childdocof{cdocsamp}
%    \end{macrocode}

%\iffalse
%</samplechap1|samplechap2>
%\fi
%
%\iffalse
%<*samplechap1>
%\fi
% Some text for chapter 1:
%    \begin{macrocode}
\section{one}
some text in chapter one
%    \end{macrocode}

%\iffalse
%</samplechap1>
%\fi
% Some text for chapter 2:
%\iffalse
%<*samplechap2>
%\fi
%    \begin{macrocode}
\section{two}
more text in chapter two
%    \end{macrocode}

%\iffalse
%</samplechap2>
%\fi
%
% %%%%%%%%%%%%%%%%%%%%%%%%%%%%%%%%%%%%%%
% \paragraph{Part Include Files.}
%
% The include files are called |cdocspt3.tex| and |cdocspt4.tex|.
%
%\iffalse
%<*samplepart3|samplepart4>
%\fi

% Optional override for |\version| flag:
%    \begin{macrocode}
%%\providecommand{\version}{final}
%    \end{macrocode}

% Include the main document:
%    \begin{macrocode}
\input{childdoc.def}
\childdocby{cdocsamp}
%    \end{macrocode}

%\iffalse
%</samplepart3|samplepart4>
%\fi
%
%\iffalse
%<*samplepart3>
%\fi
% Some text for part 3:
%    \begin{macrocode}
some text in part three
%    \end{macrocode}

%\iffalse
%</samplepart3>
%\fi
% Some text for part 4:
%\iffalse
%<*samplepart4>
%\fi
%    \begin{macrocode}
more text in part four
%    \end{macrocode}

%\iffalse
%</samplepart4>
%\fi
%
% %%%%%%%%%%%%%%%%%%%%%%%%%%%%%%%%%%%%%%
% \paragraph{Forwarding for a Complete Draft.}
%
% The following forwarding file |cdocsdrf.tex|
% compiles the main document in draft mode:
%\iffalse
%<*sampledraft>
%\fi
%    \begin{macrocode}
\def\version{draft}
\input{childdoc.def}
\childdocforward{cdocsamp}
%    \end{macrocode}

%\iffalse
%</sampledraft>
%\fi
%
% %%%%%%%%%%%%%%%%%%%%%%%%%%%%%%%%%%%%%%
% \paragraph{Forwarding for Final Version of the Chapters.}
%
% The following forwarding files |cdocsfn1.tex| and |cdocsfn2.tex|
% (with identical content)
% compile the final versions of the child documents
% |cdocsch1.tex| and |cdocsch2.tex|, respectively:
%\iffalse
%<*samplefinal>
%\fi
%    \begin{macrocode}
\def\version{final}
\input{childdoc.def}
\childdocforwardprefix[cdocsamp]{cdocsfn}{cdocsch}
%    \end{macrocode}

%\iffalse
%</samplefinal>
%\fi
%
% %%%%%%%%%%%%%%%%%%%%%%%%%%%%%%%%%%%%%%
% \paragraph{Command Line Processing.}
%
% The following three command lines generate the output files
% |cdocscld|, |cdocscl1| and |cdocscl2|
% which should be identical to
% |cdocsdrf|, |cdocsch1| and |cdocsfn2|, respectively:
% \begin{center}
% \begin{tabular}{l}
% |latex -jobname cdocscld \|\\
% |  "\def\version{draft}\input{childdoc.def}\childdocforward{cdocsamp}"|\\
% |latex -jobname cdocscl1 \|\\
% |  "\input{childdoc.def}\childdocforward[cdocsamp]{cdocsch1}"|\\
% |latex -jobname cdocscl2 \|\\
% |  "\def\version{final}\input{childdoc.def}\childdocforward{cdocsch2}"|
% \end{tabular}
% \end{center}
% Note that the trailing backslash on each first line
% merely continues the input to the second line
% (for convenient cut ant paste).
% Furthermore, the command |latex| can be replaced by any
% of its alternative versions such as |pdflatex|.
%
% %%%%%%%%%%%%%%%%%%%%%%%%%%%%%%%%%%%%%%%%%%%%%%%%%%%%%%%%%%%%%%%%%%%%%%%%%%%%%%
% %%%%%%%%%%%%%%%%%%%%%%%%%%%%%%%%%%%%%%%%%%%%%%%%%%%%%%%%%%%%%%%%%%%%%%%%%%%%%%
% \section{Implementation}
%\iffalse
%<*package>
%\fi
%
% This section describes the definitions file |childdoc.def|.

% The definitions cannot be loaded using |\usepackage| or |\RequirePackage|
% which has a mechanism to prevent loading a style file more than once.
% When loading the definitions by means of |\input|
% multiple instances have to be prevented manually:
%\iffalse
%This code needs to be before the `\ProvidesFile' directive
%which is defined at the beginning of this file.
%Therefore it is also placed there and commented out here.
%</package>
%<*discard>
%\fi
%    \begin{macrocode}
\ifdefined\childdocmain\endinput\fi
%    \end{macrocode}
%\iffalse
%</discard>
%<*package>
%\fi
%
% \macro{\ifchilddoc}
% \macro{\ifchilddocmanual}
% The conditional |\ifchilddoc| tells whether a
% child (true) or main (false) document is being compiled.
% The conditional |\ifchilddocmanual| tells whether
% the |\includeonly| mechanism is used (false) or
% the selection of child files must be performed manually (true).
% The definitions initialise to false:
%    \begin{macrocode}
\newif\ifchilddoc
\newif\ifchilddocmanual
%    \end{macrocode}

% \macro{\childdocname}
% \macro{\childdocjob}
% The macro |\childdocname| stores the name of the main document
% to be compiled. The macro |\childdocjob| stores the name of
% the document on which the \LaTeX{} compiler was originally invoked.
% The content of |\jobname| cannot be compared
% to filenames specified in the source due to different catcodes.
% The following code rescans |\jobname|, stores the result
% in |\childdocname| and saves a copy in |\childdocjob|:
%    \begin{macrocode}
\edef\childdocname{\scantokens\expandafter{\jobname\noexpand}}
\let\childdocjob\childdocname
%    \end{macrocode}

% \macro{\childdocdisable}
% The macro |\childdocdisable| prevents the main file
% from being processed more than once.
% At this stage, the main document command |\childdocmain|
% is assumed to be called once again where it should do nothing.
% Any subsequent call to it should prevent
% a secondary processing of the main document
% It overwrites the forwarding commands
% |\childdocof| and |\childdocforward|
% with empty macros to prevent further inclusions of the main document:
%    \begin{macrocode}
\newcommand{\childdocdisable}
{
  \renewcommand{\childdocmain}[1]{\renewcommand{\childdocmain}[1]{\endinput}}
  \renewcommand{\childdocof}[1]{}
  \renewcommand{\childdocby}[2][]{}
  \renewcommand{\childdocforward}[2][]{}
  \renewcommand{\childdocdisable}{}
}
%    \end{macrocode}

% \macro{\childdocmain}
% The macro |\childdocmain| is to be called at the top of the main file
% with nothing or the main filename (without extension) as argument.
% First, it breaks loops.
% If the argument is not empty and does not match |\childdocname|
% (which is set by the first inclusion of |childdoc.def|),
% |\ifchilddoc| is set to true, |\includeonly| is applied to the child file
% and |\jobname| is set to the main file
% (for proper handling of |.aux| files):
%    \begin{macrocode}
\newcommand{\childdocmain}[1]
{
  \childdocdisable\childdocmain{}
  \if?#1?\else
    \begingroup
      \def\childdoctmp{#1}
      \ifx\childdoctmp\childdocname
        \def\childdoctmp{}
      \else
        \def\childdoctmp
        {
          \childdoctrue
          \includeonly{\childdocname}
          \def\childdocjob{#1}
          \def\jobname{#1}
        }
      \fi
      \expandafter
    \endgroup
    \childdoctmp
  \fi
}
%    \end{macrocode}

% \macro{\childdocof}
% The command |\childdocof| redirects
% compilation to the main file |#1|.
%    \begin{macrocode}
\newcommand{\childdocof}[1]
{
  \childdocdisable
  \childdoctrue
  \includeonly{\childdocname}
  \def\jobname{#1}
  \def\childdocjob{#1}
  \input{#1}
}
%    \end{macrocode}

% \macro{\childdocby}
% The command |\childdocby| ....
%    \begin{macrocode}
\newcommand{\childdocby}[2][]
{
  \childdocdisable
  \childdoctrue
  \childdocmanualtrue
  \if?#1?\else
    \def\jobname{#2}
  \fi
  \def\childdocjob{#2}
  \input{#2}
  \endinput
}
%    \end{macrocode}

% \macro{\childdocforward}
% The command |\childdocforward| redirects
% compilation to the main file or
% (if the optional argument is given) a child file.
% Parameters are set as if the main file
% or a child file starting with |\childdocof| was compiled.
% Then compilation is handed over to the main file:
%    \begin{macrocode}
\newcommand{\childdocforward}[2][]
{
  \begingroup
    \if?#1?
      \def\childdoctmp
      {
        \def\childdocname{#2}
        \def\childdocjob{#2}
        \def\jobname{#2}
        \input{#2}
        \endinput
      }
    \else
      \def\childdoctmp
      {
        \childdocdisable
        \def\childdocname{#2}
        \childdoctrue
        \includeonly{#2}
        \def\childdocjob{#1}
        \def\jobname{#1}
        \input{#1}
        \endinput
      }
    \fi
    \expandafter
  \endgroup
  \childdoctmp
}
%    \end{macrocode}

% \macro{\childdocforwardprefix}
% The command |\childdocforwardprefix| redirects
% compilation to the main or a child file by means of a pattern.
% The prefix |#1| in the current filename is replaced by |#2|
% and the suffix of the current filename is kept
% (it is assumed that the filename does not contain the substring `|~~~|'
% which is used as a delimiter).
% Compilation is handed over to the new file by |\childdocforward|:
%    \begin{macrocode}
\newcommand{\childdocforwardprefix}[3][]
{
  \begingroup
    \def\childdocextract #2##1~~~{\def\childdoctmp{\childdocforward[#1]{#3##1}}}
    \expandafter\childdocextract\childdocname~~~
    \expandafter
  \endgroup
  \childdoctmp
}
%    \end{macrocode}

% \macro{\childdoc}
% The deprecated macro |\childdoc| is a legacy version of |\childdocmain|:
%    \begin{macrocode}
\newcommand{\childdoc}{\childdocmain}
%    \end{macrocode}

% \macro{\childdocredirect}
% The deprecated macro |\childdocredirect| is a legacy version
% of |\childdocforward| and |\childdocforwardprefix|:
%    \begin{macrocode}
\newcommand{\childdocredirect}[2][]
{
  \begingroup
    \if?#1?
      \def\childdoctmp{\childdocforward{#2}}
    \else
      \def\childdoctmp{\childdocforwardprefix{#1}{#2}}
    \fi
    \expandafter
  \endgroup
  \childdoctmp
}
%    \end{macrocode}

%\iffalse
%</package>
%\fi
%
\endinput
|\\
|\childdocforward{|\textit{main}|}|
\end{tabular}
\end{center}
%
Likewise, the following files |final|\textit{nn}|.tex|
compile the final version of the child document
|child|\textit{nn}|.tex|:
%
\begin{center}
\begin{tabular}{l}
|\def\version{final}|\\
|% \iffalse
%
% childdoc.dtx Copyright (C) 2017-2018 Niklas Beisert
%
% This work may be distributed and/or modified under the
% conditions of the LaTeX Project Public License, either version 1.3
% of this license or (at your option) any later version.
% The latest version of this license is in
%   http://www.latex-project.org/lppl.txt
% and version 1.3 or later is part of all distributions of LaTeX
% version 2005/12/01 or later.
%
% This work has the LPPL maintenance status `maintained'.
%
% The Current Maintainer of this work is Niklas Beisert.
%
% This work consists of the files childdoc.dtx and childdoc.ins
% and the derived files childdoc.def and cdocsamp.tex with
% cdocsch1.tex, cdocsch2.tex, cdocsdrf.tex, cdocsfn1.tex, cdocsfn2.tex.
%
%<package>\ifdefined\childdocmain\endinput\fi
%<package>\ProvidesFile{childdoc.def}[2018/12/30 v2.0 child document driver]
%<samplemain>\ProvidesFile{cdocsamp.tex}[2018/12/30 v2.0 sample for childdoc]
%<*driver>
%\ProvidesFile{childdoc.drv}[2018/12/30 v2.0 childdoc reference manual file]
\PassOptionsToClass{10pt,a4paper}{article}
\documentclass{ltxdoc}

\usepackage[margin=35mm]{geometry}
\usepackage{hyperref}
\usepackage{hyperxmp}
\usepackage[usenames]{color}

\hypersetup{colorlinks=true}
\hypersetup{pdfstartview=FitH}
\hypersetup{pdfpagemode=UseNone}
\hypersetup{pdfsource={}}
\hypersetup{pdflang={en-UK}}
\hypersetup{pdfcopyright={Copyright 2017-2018 Niklas Beisert.
  This work may be distributed and/or modified under the
  conditions of the LaTeX Project Public License, either version 1.3
  of this license or (at your option) any later version.}}
\hypersetup{pdflicenseurl={http://www.latex-project.org/lppl.txt}}
\hypersetup{pdfcontactaddress={ETH Zurich, ITP, HIT K,
  Wolfgang-Pauli-Strasse 27}}
\hypersetup{pdfcontactpostcode={8093}}
\hypersetup{pdfcontactcity={Zurich}}
\hypersetup{pdfcontactcountry={Switzerland}}
\hypersetup{pdfcontactemail={nbeisert@itp.phys.ethz.ch}}
\hypersetup{pdfcontacturl={http://people.phys.ethz.ch/\xmptilde nbeisert/}}

\newcommand{\secref}[1]{\hyperref[#1]{section \ref*{#1}}}

\parskip1ex
\parindent0pt
\let\olditemize\itemize
\def\itemize{\olditemize\parskip0pt}

\begin{document}

\title{The \textsf{childdoc} Package}
\hypersetup{pdftitle={The childdoc Package}}
\author{Niklas Beisert\\[2ex]
  Institut f\"ur Theoretische Physik\\
  Eidgen\"ossische Technische Hochschule Z\"urich\\
  Wolfgang-Pauli-Strasse 27, 8093 Z\"urich, Switzerland\\[1ex]
  \href{mailto:nbeisert@itp.phys.ethz.ch}
  {\texttt{nbeisert@itp.phys.ethz.ch}}}
\hypersetup{pdfauthor={Niklas Beisert}}
\hypersetup{pdfsubject={Manual for the LaTeX2e Package childdoc}}
\date{30 December 2018, \textsf{v2.0}}
\maketitle

\begin{abstract}\noindent
\textsf{childdoc} is a \LaTeXe{} package
that enables the direct compilation
of document sections included by |\include|
to individual files.
\end{abstract}

\begingroup
\parskip0ex
\tableofcontents
\endgroup

%%%%%%%%%%%%%%%%%%%%%%%%%%%%%%%%%%%%%%%%%%%%%%%%%%%%%%%%%%%%%%%%%%%%%%%%%%%%%%%%
%%%%%%%%%%%%%%%%%%%%%%%%%%%%%%%%%%%%%%%%%%%%%%%%%%%%%%%%%%%%%%%%%%%%%%%%%%%%%%%%
\section{Introduction}

\LaTeX{} provides a mechanism to structure a large document (such as a book)
into a main file and several child files (containing the chapters)
using the |\include| command.
This mechanism is beneficial for documents
which span hundreds of pages in order to
make the source file(s) more manageable.
Moreover, compilation can be restricted to
selected child files by means of the |\includeonly| command.
The latter feature can be used to reduce the compilation time while editing
(this was significantly more useful in the earlier days of \LaTeX{})
or to generate a smaller document which is easier to navigate.
Another application of |\includeonly| is to generate
documents consisting of selected parts of the complete document.

However, there are a few drawbacks of the plain |\include| mechanism:
\begin{itemize}
\item
The child files cannot be compiled on their own,
they can only be compiled via the main file.
A naive editing environment
(such as a text editor with an option
to have the current file processed by \LaTeX)
may require one to switch to the main file before compiling;
attempting to compile the child file produces errors.
\item
The main file must be modified (each time)
to adjust the |\includeonly| command
to the present needs. This easily leaves the main file in a messy state.
\item
The generated document will always carry the filename
of the main document. This is inconvenient if
several child files are to be compiled and
to be kept for distribution.
\end{itemize}

The present package provides a simple interface
to make child files individually compilable by \LaTeX{}.
Compiling a child file then has the same effect as compiling
the main file with an |\includeonly| command
to select the appropriate child.
Moreover the generated document will carry the name of the child
rather than the main file.
This resolves all three above issues.

This feature is meant to make the editing of books,
thesis documents and lecture notes somewhat more convenient.
However, the package can also be used efficiently for
composing a series of documents (such as exercise sheets)
which are typically distributed individually.
It then assists the author in generating the individual documents
(potentially in different versions)
as well as a document containing the collected series.
Another application is in developing style files
or other kinds of included material
where compilation of the style file could redirect
to a sample or test file.

%%%%%%%%%%%%%%%%%%%%%%%%%%%%%%%%%%%%%%%%%%%%%%%%%%%%%%%%%%%%%%%%%%%%%%%%%%%%%%%%
%%%%%%%%%%%%%%%%%%%%%%%%%%%%%%%%%%%%%%%%%%%%%%%%%%%%%%%%%%%%%%%%%%%%%%%%%%%%%%%%
\section{Usage}

First of all, the package \textsf{childdoc} is \emph{not} a standard
\LaTeXe{} |.sty| style file! Therefore it needs to be invoked in
a non-standard way.

%%%%%%%%%%%%%%%%%%%%%%%%%%%%%%%%%%%%%%%%%%%%%%%%%%%%%%%%%%%%%%%%%%%%%%%%%%%%%%%%
\subsection{Included Files}
\label{sec:include}

%%%%%%%%%%%%%%%%%%%%%%%%%%%%%%%%%%%%%%%%
\DescribeMacro{\childdocmain}
To use the package, add the commands
\begin{center}
\begin{tabular}{l}
|\input{childdoc.def}|\\
|\childdocmain{}|\\
\end{tabular}
\end{center}
at the very top of the main \LaTeX{} file,
in particular \emph{before} the |\documentclass| statement!
The argument of |\childdocmain| should be left empty
(but it must be present).

%%%%%%%%%%%%%%%%%%%%%%%%%%%%%%%%%%%%%%%%
\DescribeMacro{\childdocof}
Furthermore, add the commands
\begin{center}
\begin{tabular}{l}
|\input{childdoc.def}|\\
|\childdocof{|\textit{main}|}|\\
\end{tabular}
\end{center}
at the top of every child file \textit{child}
which is included by |\include{|\textit{child}|}|
from within the main file
(or at least for those files to be compiled individually).
The argument \textit{main} must be the filename of the main file.

There are a couple of
considerations in setting up the main and child documents:

%%%%%%%%%%%%%%%%%%%%%%%%%%%%%%%%%%%%%%%%
\paragraph{Restrictions.}

Please note the following restrictions:
\begin{itemize}
\item
|\childdocmain| must be called with one argument \textit{main}
to ensure compatibility with earlier version of the package.
It must either be empty (|\childdocmain{}|)
or precisely match the filename of the main file in which it is specified.
See \secref{sec:detection} for further information.
\item
The filename \textit{main} must be specified without the |.tex| extension.
\item
The filename \textit{main} is case sensitive
(even in case-insensitive file systems)
due to internal string comparison.
\item
The argument \textit{main} should be fully expanded, it cannot be a macro.
\item
Subdirectories and special characters should be avoided in filenames.
\item
The command |\childdocmain{|\textit{main}|}| must be followed by a whitespace.
It should not be followed immediately by another command
or by a comment mark `|%|'.
This is because the \TeX{} parser reads the token immediately following
the argument of |\childdocmain| and puts it
at the beginning of every child section;
however, a white\-space is ignored.
\end{itemize}

%%%%%%%%%%%%%%%%%%%%%%%%%%%%%%%%%%%%%%%%
\paragraph{Content of Main File.}

It is advisable to place all content in the child files included by |\include|.
Any output contained in the main file will appear in all child documents
unless suppressed manually;
it cannot be suppressed automatically by the |\includeonly| directive
and thus should normally be avoided.
A method to include some content in the main file
by means of conditional processing is described in \secref{sec:conditional}.

%%%%%%%%%%%%%%%%%%%%%%%%%%%%%%%%%%%%%%%%
\paragraph{Page Numbering.}

When only a part of the document is compiled,
the appropriate numbering of pages
(as well as other status parameters)
is determined from the |.aux| files.
The latter contain information from previous passes.
However this information needs to propagate through
all intermediate child documents.
Therefore the page numbering in child documents may well
be inconsistent until the complete document is compiled at least once.

A useful (if unconventional) way to always ensure a consistent
page numbering is to restart the numbering in each child document
and denote the pages by `\textit{child}|.|\textit{page}'
where \textit{child} represents the chapter/section number of the child file.
This can be achieved by the command
|\numberwithin{page}{|\textit{child}|}|
of the \textsf{amsmath} package
where \textit{child} can be |chapter| or |section|
depending on the chosen structuring.
Alternatively, one can modify the macro |\thepage| appropriately
and reset the counter |page| at the start of each child file.

%%%%%%%%%%%%%%%%%%%%%%%%%%%%%%%%%%%%%%%%%%%%%%%%%%%%%%%%%%%%%%%%%%%%%%%%%%%%%%%%
\subsection{Conditional Processing}
\label{sec:conditional}

The package provides a mechanism to compile different versions
of a document. To customise the versions further some conditional processing
can come in handy to distinguish which version is being compiled.
The package provides two macros to describe the compilation context:

%%%%%%%%%%%%%%%%%%%%%%%%%%%%%%%%%%%%%%%%
\DescribeMacro{\ifchilddoc}
The conditional |\ifchilddoc| distinguishes between the compilation of
child documents and the main document:
%
\begin{center}
|\ifchilddoc |\textit{child-code}| |[|\||else |\textit{main-code}]| \||fi|
\end{center}

%%%%%%%%%%%%%%%%%%%%%%%%%%%%%%%%%%%%%%%%
\DescribeMacro{\childdocname}
\DescribeMacro{\childdocjob}
The macro |\childdocname| contains the filename (without extension)
of the main or child file being processed.
Note that |\childdocjob| will always contain the name of the main file.

%%%%%%%%%%%%%%%%%%%%%%%%%%%%%%%%%%%%%%%%
\paragraph{Title Page.}

Conditional processing can be used to include a title or banner page
in the main document when proper precautions are taken.
Importantly, the code in the main file should ensure that the page counter
(as well as other status parameters which are stored in the |.aux| files)
takes the same value after the conditional processing.
Otherwise the page numbers may take divergent values
depending on which part is compiled.

For example, a title page could be declared by:
%
\begin{center}
\begin{tabular}{l}
|\ifchilddoc\||else|\\
|\addtocounter{page}{-1}|\\
\textit{code for title page}\\
|\newpage|\\
|\||fi|
\end{tabular}
\end{center}
%
A banner page for the child documents can be generated by:
%
\begin{center}
\begin{tabular}{l}
|\ifchilddoc|\\
|\addtocounter{page}{-1}|\\
\textit{code for banner page}\\
|\newpage|\\
|\||fi|
\end{tabular}
\end{center}
%
Here one could write a message such as:
\begin{center}
|This is the part \childdocname{} of \childdocjob{}.|
\end{center}

%%%%%%%%%%%%%%%%%%%%%%%%%%%%%%%%%%%%%%%%%%%%%%%%%%%%%%%%%%%%%%%%%%%%%%%%%%%%%%%%
\subsection{Flags}
\label{sec:flags}

The package makes it easy to generate different versions
of the main or child documents.
To this end compilation flags can be defined
and assigned different default values.
They will be particularly useful in conjunction
with the forwarding mechanism described in \secref{sec:forward}.

For example, it may be useful to have a flag |\version|
which can be set to |draft| or |final|.
The document source will contain some conditional code
depending on the value of |\version|.
Suppose further, the flag should default to |final| for the main file
and to |draft| for child files
which is a natural assignment for editing the document.
This is achieved by placing the following code
in the preamble of the main document
(below the |\childdocmain| directive):
%
\begin{center}
\begin{tabular}{l}
|\ifchilddoc|\\
|\providecommand{\version}{draft}|\\
|\||else|\\
|\providecommand{\version}{final}|\\
|\||fi|
\end{tabular}
\end{center}
%
The definition by |\providecommand| makes sure
that previous definitions are not overwritten.
Further statements |\providecommand{\version}{...}|
can thus be added before the above code to override it.

For the main file, one might add a line
(between |\childdocmain| and the above block)
%
\begin{center}
|%\ifchilddoc\||else\providecommand{\version}{draft}\||fi|
\end{center}
%
which can be uncommented to produce a draft version.
Likewise one can add a line to the very top of a child file
(above the |\childdocof{|\textit{main}|}| directive)
%
\begin{center}
|%\providecommand{\version}{final}|
\end{center}
%
which can be uncommented to produce the final version of this child document.

%%%%%%%%%%%%%%%%%%%%%%%%%%%%%%%%%%%%%%%%%%%%%%%%%%%%%%%%%%%%%%%%%%%%%%%%%%%%%%%%
\subsection{Forwarding}
\label{sec:forward}

Different versions of the main or child documents
using compilation flags as described in \secref{sec:flags}
can be (permanently) stored in different files
for convenient compilation, viewing and distribution.
To this end, the package defines a command
to pass on compilation to a different file:

%%%%%%%%%%%%%%%%%%%%%%%%%%%%%%%%%%%%%%%%
\DescribeMacro{\childdocforward}
The command |\childdocforward| redirects processing to
another source file:
%
\begin{center}
\begin{tabular}{l}
|\input{childdoc.def}|\\
|\childdocforward[|\textit{main}|]{|\textit{dest}|}|\\
\end{tabular}
\end{center}
%
The argument \textit{dest} is the destination file
(without extension).
It should be the main file or one of the child files.
Note that further \textsf{childdoc} directives
such as |\childdocof| and |\childdocforward|
in the indicated file will be processed in this form.
The optional argument \textit{main}
passes on directly to the main file \textit{main}
while pretending to compile the child \textit{dest}.
This form behaves as if \textit{dest}
issues |\childdocof{|\textit{main}|}| right away,
and no further \textsf{childdoc} directives will be processed.

%%%%%%%%%%%%%%%%%%%%%%%%%%%%%%%%%%%%%%%%
\DescribeMacro{\...prefix}
In the alternative form |\childdocforwardprefix|,
%
\begin{center}
\begin{tabular}{l}
|\input{childdoc.def}|\\
|\childdocforwardprefix[|\textit{main}|]{|\textit{prefix}|}{|\textit{dest}|}|
\end{tabular}
\end{center}
%
the destination file is determined by a pattern
depending on the current file:
To make this work, the current file must be called
`{\textit{prefix}\hspace{0.2em}\textit{suffix}}'
with \textit{prefix} matching precisely the argument.
Processing is then passed on to the file
`{\textit{dest}\hspace{0.2em}\textit{suffix}}'.
Surely, the same effect is achieved by
directly specifying the
argument `{\textit{dest}\hspace{0.2em}\textit{suffix}}'
in the first form.
However, that requires to set up a different file
for each child. With the alternative form of the command
all these files can have exactly the same content
which simplifies setting them up and maintaining them.

For example, the following file |draft.tex|
with a compilation flag |\version| as described in \secref{sec:flags}
compiles the main document as a draft:
%
\begin{center}
\begin{tabular}{l}
|\def\version{draft}|\\
|\input{childdoc.def}|\\
|\childdocforward{|\textit{main}|}|
\end{tabular}
\end{center}
%
Likewise, the following files |final|\textit{nn}|.tex|
compile the final version of the child document
|child|\textit{nn}|.tex|:
%
\begin{center}
\begin{tabular}{l}
|\def\version{final}|\\
|\input{childdoc.def}|\\
|\childdocforwardprefix{final}{child}|
\end{tabular}
\end{center}
%

Note that when several versions of a main file and/or of each child file
are to be generated, it may be convenient to set up a |Makefile| or
shell script to automatise the process.

%%%%%%%%%%%%%%%%%%%%%%%%%%%%%%%%%%%%%%%%%%%%%%%%%%%%%%%%%%%%%%%%%%%%%%%%%%%%%%%%
\subsection{Command Line Processing}
\label{sec:commandline}

The effect of redirection files can also be achieved by invoking
the \LaTeX{} compiler with a more elaborate command line.
Most conveniently this should be done as part
of a shell script or a |Makefile|.

When using \textsf{childdoc} in the main file, the following
command lines effectively perform a redirection
(note that depending on the shell being used,
backslashes may have to be doubled: `|\|' $\to$ `|\\|'):
%
\begin{center}
|... -jobname "|\textit{target}|" |\\|"|[\textit{flags}]%
|\input{childdoc.def}\childdocforward[|\textit{main}|]{|\textit{dest}|}"|
\end{center}
%
Here \textit{target} is the name of the output file,
\textit{main} is the name of the main file
and \textit{dest} is the name of the main or child file to be processed
(all filenames without extensions).
The optional argument \textit{main} can be omitted
if \textit{main} matches \textit{dest}.
Optionally, compilation \textit{flags} can be defined via |\def| commands.
This command line makes the \TeX{} engine believe
it is compiling the file \textit{target}
whose content is specified as the latter parameter.
The provided code then forwards the processing to
\textit{main} or \textit{dest} as described in \secref{sec:forward}.

%%%%%%%%%%%%%%%%%%%%%%%%%%%%%%%%%%%%%%%%%%%%%%%%%%%%%%%%%%%%%%%%%%%%%%%%%%%%%%%%
\subsection{Include by Input}
\label{sec:input}

Including child documents by |\include| has some restrictions by design.
Most notably, the content of a child document always occupies
its own set of pages; pages cannot be shared between child documents.
Usually, this behaviour makes perfect sense
because each child document contain an essential part of the document.
However, in some situations it may be desirable to compose
a document from a collection of parts
without having mandatory page breaks between then.
For this case, the package
provides a mechanism to include parts
by |\input| which can also be processed individually.
However, by construction this mechanism
requires manual handling of the content to be output.

%%%%%%%%%%%%%%%%%%%%%%%%%%%%%%%%%%%%%%%%
\DescribeMacro{\ifchilddocmanual}
The main file should be prepared as usual, see \secref{sec:include}.
However, the document body must make a distinction
between processing of an individual part and of the main document, e.g.:
%
\begin{center}
\begin{tabular}{l}
|\ifchilddocmanual|\\
|\input{\childdocname}|\\
|\||else|\\
\textit{document body with }|\input{|\textit{part}|}|\\
|\||fi|
\end{tabular}
\end{center}
%
The conditional |\ifchilddocmanual| is true whenever
a part to be included by |\input| is being compiled,
and the name of the part is stored in |\childdocname|.

%%%%%%%%%%%%%%%%%%%%%%%%%%%%%%%%%%%%%%%%
\DescribeMacro{\childdocby}
Each part to be included by |\input| should start with:
%
\begin{center}
\begin{tabular}{l}
|\input{childdoc.def}|\\
|\childdocby{|\textit{main}|}|\\
\end{tabular}
\end{center}
%
The directive |\childdocby| is similar to |\childdocof|
described in \secref{sec:include},
but the subsequent selection of content must be done manually.
To that end, both |\ifchilddoc| and |\ifchilddocmanual|
will be true upon processing of a part,
and the name of the part is stored in |\childdocname|.
Note that |\jobname| will be set to the filename of the current part
so that each part receives an individual |.aux| file
that does not interfere with the |.aux| file(s) of the main document.
This behaviour can be altered by the alternative form
|\childdocby[*]{|\textit{main}|}| (with a non-empty optional argument)
which uses the |.aux| file of the main document
by setting |\jobname| to \textit{main}.

%%%%%%%%%%%%%%%%%%%%%%%%%%%%%%%%%%%%%%%%%%%%%%%%%%%%%%%%%%%%%%%%%%%%%%%%%%%%%%%%
\subsection{Driver Development}
\label{sec:driver}

The \textsf{childdoc} mechanism can also be use for the development
of definition files such as \LaTeX{} styles or classes.
This case differs from the above setup with multiple parts
included by |\include| in that no |\includeonly| should be invoked.
This can be achieved by starting the include file
(before |\ProvidesPackage|) with:
%
\begin{center}
\begin{tabular}{l}
|\input{childdoc.def}|\\
|\childdocforward{|\textit{main}|}|\\
\end{tabular}
\end{center}
%
or alternatively with:
%
\begin{center}
\begin{tabular}{l}
|\input{childdoc.def}|\\
|\childdocby{|\textit{main}|}|\\
\end{tabular}
\end{center}
%
Both forms have slightly different effects as described above.
The main file is prepared as usual, see \secref{sec:include}.

%%%%%%%%%%%%%%%%%%%%%%%%%%%%%%%%%%%%%%%%%%%%%%%%%%%%%%%%%%%%%%%%%%%%%%%%%%%%%%%%
\subsection{Legacy Detection}
\label{sec:detection}

The directive |\childdocmain| in the main file can detect
whether the complete document or merely a child is to be compiled
even without using the directive |\childdocof|.
This method is deprecated because it is less robust
and there is no compelling reason to use it;
it is merely provided for backward compatibility
and it may be removed in future versions.

If the detection mechanism is to be used,
it is mandatory to correctly specify
the filename of the main file as the argument of |\childdocmain|:
%
\begin{center}
\begin{tabular}{l}
|\input{childdoc.def}|\\
|\childdocmain{|\textit{main}|}|\\
\end{tabular}
\end{center}
%
If |\jobname| does not match the argument \textit{main} of |\childdocmain|,
it is assumed that |\jobname| points to the child file to be compiled.
When using |\childdocmain| with the main file specified as argument,
it suffices to start a child file
with just |\input{|\textit{main}|}|
without loading of the package and using |\childdocof|.
If instead all processing is done
with the appropriate \textsf{childdoc} directives,
the argument of \textit{main} of |\childdocmain| can be empty.

An alternative version of the command line processing described
in \secref{sec:commandline} using the detection mechanism reads:
%
\begin{center}
|... -jobname "|\textit{target}|" "|[\textit{flags}]%
[|\def\jobname{|\textit{dest}|}|]|\input{|\textit{main}|}"|
\end{center}

%%%%%%%%%%%%%%%%%%%%%%%%%%%%%%%%%%%%%%%%%%%%%%%%%%%%%%%%%%%%%%%%%%%%%%%%%%%%%%%%
\subsection{Manual Code}
\label{sec:manual}

In case one cannot be certain whether the definitions file |childdoc.def|
is installed on the target \TeX{} distribution
and one prefers not to ship it,
it is conceivable to paste a few relevant commands into the sources.

To that end, drop all statements |\input{childdoc.def}|
and perform the replacements as outlined below.
Instead of |\childdocmain{|\textit{main}|}| add the following code
to the top of the main file:
%
\begin{center}
\begin{tabular}{l}
|\||ifdefined\childdocname\endinput\||fi\newif\ifchilddoc|\\
|\edef\childdocname{\scantokens\expandafter{\jobname\noexpand}}|\\
|\def\childdocmain{|\textit{main}|}\||ifx\childdocmain\childdocname\||else|\\
|\childdoctrue\includeonly{\childdocname}\let\jobname\childdocmain\||fi|\\
\end{tabular}
\end{center}
%
Instead of |\childdocof{|\textit{main}|}| just include the main file
at the top of each child file:
%
\begin{center}
|\input{|\textit{main}|}|
\end{center}
%
A simple redirection |\childdocforward{|\textit{dest}|}| is achieved by:
%
\begin{center}
|\def\jobname{|\textit{dest}|}\input{\jobname}|
\end{center}
%
The redirection with prefix
|\childdocforwardprefix[|\textit{prefix}|]{|\textit{dest}|}|
is accomplished by:
%
\begin{center}
\begin{tabular}{l}
|{\edef\jobname{\scantokens\expandafter{\jobname\noexpand}}|\\
|\def\redirectjob |\textit{prefix}|#1~~~{\gdef\jobname{|\textit{dest}|#1}}|\\
|\expandafter\redirectjob\jobname~~~}\input{\jobname}|
\end{tabular}
\end{center}

In an alternative approach,
child documents can be compiled by a specific command line
without additional code or specific definitions:
%
\begin{center}
|... -jobname "|\textit{target}|" "|[\textit{flags}]%
|\includeonly{|\textit{dest}|}\input{|\textit{main}|}"|
\end{center}
%

%%%%%%%%%%%%%%%%%%%%%%%%%%%%%%%%%%%%%%%%%%%%%%%%%%%%%%%%%%%%%%%%%%%%%%%%%%%%%%%%
%%%%%%%%%%%%%%%%%%%%%%%%%%%%%%%%%%%%%%%%%%%%%%%%%%%%%%%%%%%%%%%%%%%%%%%%%%%%%%%%
\section{Information}

%%%%%%%%%%%%%%%%%%%%%%%%%%%%%%%%%%%%%%%%%%%%%%%%%%%%%%%%%%%%%%%%%%%%%%%%%%%%%%%%
\subsection{Copyright}

Copyright \copyright{} 2017--2018 Niklas Beisert

This work may be distributed and/or modified under the
conditions of the \LaTeX{} Project Public License, either version 1.3
of this license or (at your option) any later version.
The latest version of this license is in
  \url{http://www.latex-project.org/lppl.txt}
and version 1.3 or later is part of all distributions of \LaTeX{}
version 2005/12/01 or later.

This work has the LPPL maintenance status `maintained'.

The Current Maintainer of this work is Niklas Beisert.

This work consists of the files |README.txt|, |childdoc.ins| and |childdoc.dtx|
as well as the derived files |childdoc.def|, |cdocsamp.tex|
with |cdocsch1.tex|, |cdocsch2.tex|, |cdocspt3.tex|, |cdocspt4.tex|,
|cdocsdrf.tex|, |cdocsfn1.tex|, |cdocsfn2.tex|
as well as |childdoc.pdf|.

%%%%%%%%%%%%%%%%%%%%%%%%%%%%%%%%%%%%%%%%%%%%%%%%%%%%%%%%%%%%%%%%%%%%%%%%%%%%%%%%
\subsection{Files and Installation}

The package consists of the files:
%
\begin{center}
\begin{tabular}{ll}
    |README.txt|   & readme file \\
    |childdoc.ins| & installation file \\
    |childdoc.dtx| & source file \\
    |childdoc.def| & definition file \\
    |cdocsamp.tex| & sample main file \\
    |cdocsch1.tex| & sample include file \\
    |cdocsch2.tex| & sample include file \\
    |cdocspt3.tex| & sample part file \\
    |cdocspt4.tex| & sample part file \\
    |cdocsdrf.tex| & sample redirection file \\
    |cdocsfn1.tex| & sample redirection file \\
    |cdocsfn2.tex| & sample redirection file \\
    |childdoc.pdf| & manual
\end{tabular}
\end{center}
%
The distribution consists of the files
|README.txt|, |childdoc.ins| and |childdoc.dtx|.
%
\begin{itemize}
\item
Run (pdf)\LaTeX{} on |childdoc.dtx|
to compile the manual |childdoc.pdf| (this file).
\item
Run \LaTeX{} on |childdoc.ins| to create the definitions file |childdoc.def|
and the sample |cdocsamp.tex| with include files
|cdocsch1.tex|, |cdocsch2.tex|, |cdocspt3.tex|, |cdocspt4.tex|,
|cdocsdrf.tex|, |cdocsfn1.tex|, |cdocsfn2.tex|.
Then copy the file |childdoc.def| to an appropriate directory of your \LaTeX{}
distribution, e.g.\ \textit{texmf-root}|/tex/latex/childdoc|.
\end{itemize}

%%%%%%%%%%%%%%%%%%%%%%%%%%%%%%%%%%%%%%%%%%%%%%%%%%%%%%%%%%%%%%%%%%%%%%%%%%%%%%%%
\subsection{Related CTAN Packages}

There are several other packages which offer a similar functionality:
%
\begin{itemize}
\item
The packages
\href{http://ctan.org/pkg/docmute}{\textsf{docmute}},
\href{http://ctan.org/pkg/includex}{\textsf{includex}} and
\href{http://ctan.org/pkg/standalone}{\textsf{standalone}}
provide commands to include only the document body of
a child file thus allowing both files to be compiled individually.
\item
The packages \href{http://ctan.org/pkg/subdocs}{\textsf{subdocs}}
and \href{http://ctan.org/pkg/subfiles}{\textsf{subfiles}}
provide structures in which the main and child documents can be
encapsulated and allowing them to be compiled individually.
The inclusion mechanism is different from the conventional |\include|.
\item
The package \href{http://ctan.org/pkg/combine}{\textsf{combine}}
is an elaborate solution to combine several documents into one.
\end{itemize}
%
See also the CTAN topic \href{http://ctan.org/topic/subdocs}{\textsf{subdocs}}
for further related packages.
The present package differs from the above solutions in that
a document structure constructed with the conventional |\include| mechanism
just needs two extra commands at the top of every file
such that all constituent files can be compiled individually.

%%%%%%%%%%%%%%%%%%%%%%%%%%%%%%%%%%%%%%%%%%%%%%%%%%%%%%%%%%%%%%%%%%%%%%%%%%%%%%%%
%\subsection{Feature Suggestions}
%
%The following is a list of features which may be useful for future
%versions of this package:
%%
%\begin{itemize}
%\item
%\ldots
%\end{itemize}

%%%%%%%%%%%%%%%%%%%%%%%%%%%%%%%%%%%%%%%%%%%%%%%%%%%%%%%%%%%%%%%%%%%%%%%%%%%%%%%%
\subsection{Revision History}

%%%%%%%%%%%%%%%%%%%%%%%%%%%%%%%%%%%%%%%%
\paragraph{v2.0:} 2018/12/30

\begin{itemize}
\item
immediate forward processing
\item
added |\childdocby| mechanism
\item
manual restructured
\end{itemize}

%%%%%%%%%%%%%%%%%%%%%%%%%%%%%%%%%%%%%%%%
\paragraph{v1.6:} 2018/01/17

\begin{itemize}
\item
application for development of include files
\item
corrections to manual
\end{itemize}

%%%%%%%%%%%%%%%%%%%%%%%%%%%%%%%%%%%%%%%%
\paragraph{v1.5:} 2017/05/21

\begin{itemize}
\item
more complete structuring introduced
\item
|\childdocof| introduced
\item
|\childdoc| renamed to |\childdocmain|
\item
|\childredirect| renamed to |\childdocforward| and |\childdocforwardprefix|
and functionality expanded
\end{itemize}

%%%%%%%%%%%%%%%%%%%%%%%%%%%%%%%%%%%%%%%%
\paragraph{v1.0:} 2017/04/27

\begin{itemize}
\item
manual and install package
\item
first version published on CTAN
\end{itemize}

%%%%%%%%%%%%%%%%%%%%%%%%%%%%%%%%%%%%%%%%
\paragraph{v0.6:} 2017/04/26

\begin{itemize}
\item
redirection mechanism added
\end{itemize}

%%%%%%%%%%%%%%%%%%%%%%%%%%%%%%%%%%%%%%%%
\paragraph{v0.5:} 2017/04/26

\begin{itemize}
\item
functionality in definition file
\end{itemize}


%%%%%%%%%%%%%%%%%%%%%%%%%%%%%%%%%%%%%%%%%%%%%%%%%%%%%%%%%%%%%%%%%%%%%%%%%%%%%%%%
%%%%%%%%%%%%%%%%%%%%%%%%%%%%%%%%%%%%%%%%%%%%%%%%%%%%%%%%%%%%%%%%%%%%%%%%%%%%%%%%
%%%%%%%%%%%%%%%%%%%%%%%%%%%%%%%%%%%%%%%%%%%%%%%%%%%%%%%%%%%%%%%%%%%%%%%%%%%%%%%%
\appendix

\settowidth\MacroIndent{\rmfamily\scriptsize 000\ }

 \DocInput{childdoc.dtx}

\end{document}
%</driver>
% \fi
%
% %%%%%%%%%%%%%%%%%%%%%%%%%%%%%%%%%%%%%%%%%%%%%%%%%%%%%%%%%%%%%%%%%%%%%%%%%%%%%%
% %%%%%%%%%%%%%%%%%%%%%%%%%%%%%%%%%%%%%%%%%%%%%%%%%%%%%%%%%%%%%%%%%%%%%%%%%%%%%%
% \section{Sample}
%\iffalse
%<*samplemain>
%\fi
%
% The following presents a sample document
% with two chapters, two parts, a title page,
% a compile flag as well as three forwarding files to set the flag.
% It consists of eight |.tex| files:
% \begin{center}
% \begin{tabular}{ll}
% |cdocsamp.tex|&main file\\
% |cdocsch1.tex|&include file for chapter 1\\
% |cdocsch2.tex|&include file for chapter 2\\
% |cdocspt3.tex|&include file for part 3\\
% |cdocspt4.tex|&include file for part 4\\
% |cdocsdrf.tex|&forwarding file for main file in draft mode\\
% |cdocsfi1.tex|&forwarding file for final version of chapter 1\\
% |cdocsfi2.tex|&forwarding file for final version of chapter 2\\
% \end{tabular}
% \end{center}
% Each of the eight files can be compiled directly by the \LaTeX{} compiler.
%
% %%%%%%%%%%%%%%%%%%%%%%%%%%%%%%%%%%%%%%
% \paragraph{Main File.}
%
% The main file is called |cdocsamp.tex|.
%
% Load the \textsf{childdoc} definitions and
% declare the filename for the main document:
%    \begin{macrocode}
\input{childdoc.def}
\childdocmain{}
%    \end{macrocode}

% Optional override for |\version| flag:
%    \begin{macrocode}
%%\ifchilddoc\else\providecommand{\version}{draft}\fi
%    \end{macrocode}

% Define the default values for the |\version| flag
% (|final| for the main file and |draft| for childs):
%    \begin{macrocode}
\ifchilddoc
\providecommand{\version}{draft}
\else
\providecommand{\version}{final}
\fi
%    \end{macrocode}

% Load the standard document class:
%    \begin{macrocode}
\documentclass[12pt]{article}
%    \end{macrocode}

% Start the document body:
%    \begin{macrocode}
\begin{document}
%    \end{macrocode}

% Declare a title page.
% Print title, part of document being processed and version flag:
%    \begin{macrocode}
\addtocounter{page}{-1}
\begin{center}
{\LARGE\bfseries{}childdoc example\par}
\vspace{1cm}
\ifchilddoc
\ifchilddocmanual part\else chapter\fi:
`\childdocname' of `\childdocjob'\par
\else
main document: `\childdocjob'\par
\fi
version: \version\par
\end{center}
\newpage
%    \end{macrocode}

% Manually include selected file,
% otherwise process as usual:
%    \begin{macrocode}
\ifchilddocmanual
\section*{part `\childdocname'}
\input{\childdocname}
\else
%    \end{macrocode}

% Include the two chapters:
%    \begin{macrocode}
\include{cdocsch1}
\include{cdocsch2}
%    \end{macrocode}

% Include the two parts unless only chapters should be displayed:
%    \begin{macrocode}
\ifchilddoc\else
\section{part three}
\input{cdocspt3}
\section{part four}
\input{cdocspt4}
\fi
%    \end{macrocode}

% Process as usual until here:
%    \begin{macrocode}
\fi
%    \end{macrocode}

% End of document body:
%    \begin{macrocode}
\end{document}
%    \end{macrocode}
%\iffalse
%</samplemain>
%\fi
%
% %%%%%%%%%%%%%%%%%%%%%%%%%%%%%%%%%%%%%%
% \paragraph{Chapter Include Files.}
%
% The include files are called |cdocsch1.tex| and |cdocsch2.tex|.
%
%\iffalse
%<*samplechap1|samplechap2>
%\fi

% Optional override for |\version| flag:
%    \begin{macrocode}
%%\providecommand{\version}{final}
%    \end{macrocode}

% Include the main document:
%    \begin{macrocode}
\input{childdoc.def}
\childdocof{cdocsamp}
%    \end{macrocode}

%\iffalse
%</samplechap1|samplechap2>
%\fi
%
%\iffalse
%<*samplechap1>
%\fi
% Some text for chapter 1:
%    \begin{macrocode}
\section{one}
some text in chapter one
%    \end{macrocode}

%\iffalse
%</samplechap1>
%\fi
% Some text for chapter 2:
%\iffalse
%<*samplechap2>
%\fi
%    \begin{macrocode}
\section{two}
more text in chapter two
%    \end{macrocode}

%\iffalse
%</samplechap2>
%\fi
%
% %%%%%%%%%%%%%%%%%%%%%%%%%%%%%%%%%%%%%%
% \paragraph{Part Include Files.}
%
% The include files are called |cdocspt3.tex| and |cdocspt4.tex|.
%
%\iffalse
%<*samplepart3|samplepart4>
%\fi

% Optional override for |\version| flag:
%    \begin{macrocode}
%%\providecommand{\version}{final}
%    \end{macrocode}

% Include the main document:
%    \begin{macrocode}
\input{childdoc.def}
\childdocby{cdocsamp}
%    \end{macrocode}

%\iffalse
%</samplepart3|samplepart4>
%\fi
%
%\iffalse
%<*samplepart3>
%\fi
% Some text for part 3:
%    \begin{macrocode}
some text in part three
%    \end{macrocode}

%\iffalse
%</samplepart3>
%\fi
% Some text for part 4:
%\iffalse
%<*samplepart4>
%\fi
%    \begin{macrocode}
more text in part four
%    \end{macrocode}

%\iffalse
%</samplepart4>
%\fi
%
% %%%%%%%%%%%%%%%%%%%%%%%%%%%%%%%%%%%%%%
% \paragraph{Forwarding for a Complete Draft.}
%
% The following forwarding file |cdocsdrf.tex|
% compiles the main document in draft mode:
%\iffalse
%<*sampledraft>
%\fi
%    \begin{macrocode}
\def\version{draft}
\input{childdoc.def}
\childdocforward{cdocsamp}
%    \end{macrocode}

%\iffalse
%</sampledraft>
%\fi
%
% %%%%%%%%%%%%%%%%%%%%%%%%%%%%%%%%%%%%%%
% \paragraph{Forwarding for Final Version of the Chapters.}
%
% The following forwarding files |cdocsfn1.tex| and |cdocsfn2.tex|
% (with identical content)
% compile the final versions of the child documents
% |cdocsch1.tex| and |cdocsch2.tex|, respectively:
%\iffalse
%<*samplefinal>
%\fi
%    \begin{macrocode}
\def\version{final}
\input{childdoc.def}
\childdocforwardprefix[cdocsamp]{cdocsfn}{cdocsch}
%    \end{macrocode}

%\iffalse
%</samplefinal>
%\fi
%
% %%%%%%%%%%%%%%%%%%%%%%%%%%%%%%%%%%%%%%
% \paragraph{Command Line Processing.}
%
% The following three command lines generate the output files
% |cdocscld|, |cdocscl1| and |cdocscl2|
% which should be identical to
% |cdocsdrf|, |cdocsch1| and |cdocsfn2|, respectively:
% \begin{center}
% \begin{tabular}{l}
% |latex -jobname cdocscld \|\\
% |  "\def\version{draft}\input{childdoc.def}\childdocforward{cdocsamp}"|\\
% |latex -jobname cdocscl1 \|\\
% |  "\input{childdoc.def}\childdocforward[cdocsamp]{cdocsch1}"|\\
% |latex -jobname cdocscl2 \|\\
% |  "\def\version{final}\input{childdoc.def}\childdocforward{cdocsch2}"|
% \end{tabular}
% \end{center}
% Note that the trailing backslash on each first line
% merely continues the input to the second line
% (for convenient cut ant paste).
% Furthermore, the command |latex| can be replaced by any
% of its alternative versions such as |pdflatex|.
%
% %%%%%%%%%%%%%%%%%%%%%%%%%%%%%%%%%%%%%%%%%%%%%%%%%%%%%%%%%%%%%%%%%%%%%%%%%%%%%%
% %%%%%%%%%%%%%%%%%%%%%%%%%%%%%%%%%%%%%%%%%%%%%%%%%%%%%%%%%%%%%%%%%%%%%%%%%%%%%%
% \section{Implementation}
%\iffalse
%<*package>
%\fi
%
% This section describes the definitions file |childdoc.def|.

% The definitions cannot be loaded using |\usepackage| or |\RequirePackage|
% which has a mechanism to prevent loading a style file more than once.
% When loading the definitions by means of |\input|
% multiple instances have to be prevented manually:
%\iffalse
%This code needs to be before the `\ProvidesFile' directive
%which is defined at the beginning of this file.
%Therefore it is also placed there and commented out here.
%</package>
%<*discard>
%\fi
%    \begin{macrocode}
\ifdefined\childdocmain\endinput\fi
%    \end{macrocode}
%\iffalse
%</discard>
%<*package>
%\fi
%
% \macro{\ifchilddoc}
% \macro{\ifchilddocmanual}
% The conditional |\ifchilddoc| tells whether a
% child (true) or main (false) document is being compiled.
% The conditional |\ifchilddocmanual| tells whether
% the |\includeonly| mechanism is used (false) or
% the selection of child files must be performed manually (true).
% The definitions initialise to false:
%    \begin{macrocode}
\newif\ifchilddoc
\newif\ifchilddocmanual
%    \end{macrocode}

% \macro{\childdocname}
% \macro{\childdocjob}
% The macro |\childdocname| stores the name of the main document
% to be compiled. The macro |\childdocjob| stores the name of
% the document on which the \LaTeX{} compiler was originally invoked.
% The content of |\jobname| cannot be compared
% to filenames specified in the source due to different catcodes.
% The following code rescans |\jobname|, stores the result
% in |\childdocname| and saves a copy in |\childdocjob|:
%    \begin{macrocode}
\edef\childdocname{\scantokens\expandafter{\jobname\noexpand}}
\let\childdocjob\childdocname
%    \end{macrocode}

% \macro{\childdocdisable}
% The macro |\childdocdisable| prevents the main file
% from being processed more than once.
% At this stage, the main document command |\childdocmain|
% is assumed to be called once again where it should do nothing.
% Any subsequent call to it should prevent
% a secondary processing of the main document
% It overwrites the forwarding commands
% |\childdocof| and |\childdocforward|
% with empty macros to prevent further inclusions of the main document:
%    \begin{macrocode}
\newcommand{\childdocdisable}
{
  \renewcommand{\childdocmain}[1]{\renewcommand{\childdocmain}[1]{\endinput}}
  \renewcommand{\childdocof}[1]{}
  \renewcommand{\childdocby}[2][]{}
  \renewcommand{\childdocforward}[2][]{}
  \renewcommand{\childdocdisable}{}
}
%    \end{macrocode}

% \macro{\childdocmain}
% The macro |\childdocmain| is to be called at the top of the main file
% with nothing or the main filename (without extension) as argument.
% First, it breaks loops.
% If the argument is not empty and does not match |\childdocname|
% (which is set by the first inclusion of |childdoc.def|),
% |\ifchilddoc| is set to true, |\includeonly| is applied to the child file
% and |\jobname| is set to the main file
% (for proper handling of |.aux| files):
%    \begin{macrocode}
\newcommand{\childdocmain}[1]
{
  \childdocdisable\childdocmain{}
  \if?#1?\else
    \begingroup
      \def\childdoctmp{#1}
      \ifx\childdoctmp\childdocname
        \def\childdoctmp{}
      \else
        \def\childdoctmp
        {
          \childdoctrue
          \includeonly{\childdocname}
          \def\childdocjob{#1}
          \def\jobname{#1}
        }
      \fi
      \expandafter
    \endgroup
    \childdoctmp
  \fi
}
%    \end{macrocode}

% \macro{\childdocof}
% The command |\childdocof| redirects
% compilation to the main file |#1|.
%    \begin{macrocode}
\newcommand{\childdocof}[1]
{
  \childdocdisable
  \childdoctrue
  \includeonly{\childdocname}
  \def\jobname{#1}
  \def\childdocjob{#1}
  \input{#1}
}
%    \end{macrocode}

% \macro{\childdocby}
% The command |\childdocby| ....
%    \begin{macrocode}
\newcommand{\childdocby}[2][]
{
  \childdocdisable
  \childdoctrue
  \childdocmanualtrue
  \if?#1?\else
    \def\jobname{#2}
  \fi
  \def\childdocjob{#2}
  \input{#2}
  \endinput
}
%    \end{macrocode}

% \macro{\childdocforward}
% The command |\childdocforward| redirects
% compilation to the main file or
% (if the optional argument is given) a child file.
% Parameters are set as if the main file
% or a child file starting with |\childdocof| was compiled.
% Then compilation is handed over to the main file:
%    \begin{macrocode}
\newcommand{\childdocforward}[2][]
{
  \begingroup
    \if?#1?
      \def\childdoctmp
      {
        \def\childdocname{#2}
        \def\childdocjob{#2}
        \def\jobname{#2}
        \input{#2}
        \endinput
      }
    \else
      \def\childdoctmp
      {
        \childdocdisable
        \def\childdocname{#2}
        \childdoctrue
        \includeonly{#2}
        \def\childdocjob{#1}
        \def\jobname{#1}
        \input{#1}
        \endinput
      }
    \fi
    \expandafter
  \endgroup
  \childdoctmp
}
%    \end{macrocode}

% \macro{\childdocforwardprefix}
% The command |\childdocforwardprefix| redirects
% compilation to the main or a child file by means of a pattern.
% The prefix |#1| in the current filename is replaced by |#2|
% and the suffix of the current filename is kept
% (it is assumed that the filename does not contain the substring `|~~~|'
% which is used as a delimiter).
% Compilation is handed over to the new file by |\childdocforward|:
%    \begin{macrocode}
\newcommand{\childdocforwardprefix}[3][]
{
  \begingroup
    \def\childdocextract #2##1~~~{\def\childdoctmp{\childdocforward[#1]{#3##1}}}
    \expandafter\childdocextract\childdocname~~~
    \expandafter
  \endgroup
  \childdoctmp
}
%    \end{macrocode}

% \macro{\childdoc}
% The deprecated macro |\childdoc| is a legacy version of |\childdocmain|:
%    \begin{macrocode}
\newcommand{\childdoc}{\childdocmain}
%    \end{macrocode}

% \macro{\childdocredirect}
% The deprecated macro |\childdocredirect| is a legacy version
% of |\childdocforward| and |\childdocforwardprefix|:
%    \begin{macrocode}
\newcommand{\childdocredirect}[2][]
{
  \begingroup
    \if?#1?
      \def\childdoctmp{\childdocforward{#2}}
    \else
      \def\childdoctmp{\childdocforwardprefix{#1}{#2}}
    \fi
    \expandafter
  \endgroup
  \childdoctmp
}
%    \end{macrocode}

%\iffalse
%</package>
%\fi
%
\endinput
|\\
|\childdocforwardprefix{final}{child}|
\end{tabular}
\end{center}
%

Note that when several versions of a main file and/or of each child file
are to be generated, it may be convenient to set up a |Makefile| or
shell script to automatise the process.

%%%%%%%%%%%%%%%%%%%%%%%%%%%%%%%%%%%%%%%%%%%%%%%%%%%%%%%%%%%%%%%%%%%%%%%%%%%%%%%%
\subsection{Command Line Processing}
\label{sec:commandline}

The effect of redirection files can also be achieved by invoking
the \LaTeX{} compiler with a more elaborate command line.
Most conveniently this should be done as part
of a shell script or a |Makefile|.

When using \textsf{childdoc} in the main file, the following
command lines effectively perform a redirection
(note that depending on the shell being used,
backslashes may have to be doubled: `|\|' $\to$ `|\\|'):
%
\begin{center}
|... -jobname "|\textit{target}|" |\\|"|[\textit{flags}]%
|% \iffalse
%
% childdoc.dtx Copyright (C) 2017-2018 Niklas Beisert
%
% This work may be distributed and/or modified under the
% conditions of the LaTeX Project Public License, either version 1.3
% of this license or (at your option) any later version.
% The latest version of this license is in
%   http://www.latex-project.org/lppl.txt
% and version 1.3 or later is part of all distributions of LaTeX
% version 2005/12/01 or later.
%
% This work has the LPPL maintenance status `maintained'.
%
% The Current Maintainer of this work is Niklas Beisert.
%
% This work consists of the files childdoc.dtx and childdoc.ins
% and the derived files childdoc.def and cdocsamp.tex with
% cdocsch1.tex, cdocsch2.tex, cdocsdrf.tex, cdocsfn1.tex, cdocsfn2.tex.
%
%<package>\ifdefined\childdocmain\endinput\fi
%<package>\ProvidesFile{childdoc.def}[2018/12/30 v2.0 child document driver]
%<samplemain>\ProvidesFile{cdocsamp.tex}[2018/12/30 v2.0 sample for childdoc]
%<*driver>
%\ProvidesFile{childdoc.drv}[2018/12/30 v2.0 childdoc reference manual file]
\PassOptionsToClass{10pt,a4paper}{article}
\documentclass{ltxdoc}

\usepackage[margin=35mm]{geometry}
\usepackage{hyperref}
\usepackage{hyperxmp}
\usepackage[usenames]{color}

\hypersetup{colorlinks=true}
\hypersetup{pdfstartview=FitH}
\hypersetup{pdfpagemode=UseNone}
\hypersetup{pdfsource={}}
\hypersetup{pdflang={en-UK}}
\hypersetup{pdfcopyright={Copyright 2017-2018 Niklas Beisert.
  This work may be distributed and/or modified under the
  conditions of the LaTeX Project Public License, either version 1.3
  of this license or (at your option) any later version.}}
\hypersetup{pdflicenseurl={http://www.latex-project.org/lppl.txt}}
\hypersetup{pdfcontactaddress={ETH Zurich, ITP, HIT K,
  Wolfgang-Pauli-Strasse 27}}
\hypersetup{pdfcontactpostcode={8093}}
\hypersetup{pdfcontactcity={Zurich}}
\hypersetup{pdfcontactcountry={Switzerland}}
\hypersetup{pdfcontactemail={nbeisert@itp.phys.ethz.ch}}
\hypersetup{pdfcontacturl={http://people.phys.ethz.ch/\xmptilde nbeisert/}}

\newcommand{\secref}[1]{\hyperref[#1]{section \ref*{#1}}}

\parskip1ex
\parindent0pt
\let\olditemize\itemize
\def\itemize{\olditemize\parskip0pt}

\begin{document}

\title{The \textsf{childdoc} Package}
\hypersetup{pdftitle={The childdoc Package}}
\author{Niklas Beisert\\[2ex]
  Institut f\"ur Theoretische Physik\\
  Eidgen\"ossische Technische Hochschule Z\"urich\\
  Wolfgang-Pauli-Strasse 27, 8093 Z\"urich, Switzerland\\[1ex]
  \href{mailto:nbeisert@itp.phys.ethz.ch}
  {\texttt{nbeisert@itp.phys.ethz.ch}}}
\hypersetup{pdfauthor={Niklas Beisert}}
\hypersetup{pdfsubject={Manual for the LaTeX2e Package childdoc}}
\date{30 December 2018, \textsf{v2.0}}
\maketitle

\begin{abstract}\noindent
\textsf{childdoc} is a \LaTeXe{} package
that enables the direct compilation
of document sections included by |\include|
to individual files.
\end{abstract}

\begingroup
\parskip0ex
\tableofcontents
\endgroup

%%%%%%%%%%%%%%%%%%%%%%%%%%%%%%%%%%%%%%%%%%%%%%%%%%%%%%%%%%%%%%%%%%%%%%%%%%%%%%%%
%%%%%%%%%%%%%%%%%%%%%%%%%%%%%%%%%%%%%%%%%%%%%%%%%%%%%%%%%%%%%%%%%%%%%%%%%%%%%%%%
\section{Introduction}

\LaTeX{} provides a mechanism to structure a large document (such as a book)
into a main file and several child files (containing the chapters)
using the |\include| command.
This mechanism is beneficial for documents
which span hundreds of pages in order to
make the source file(s) more manageable.
Moreover, compilation can be restricted to
selected child files by means of the |\includeonly| command.
The latter feature can be used to reduce the compilation time while editing
(this was significantly more useful in the earlier days of \LaTeX{})
or to generate a smaller document which is easier to navigate.
Another application of |\includeonly| is to generate
documents consisting of selected parts of the complete document.

However, there are a few drawbacks of the plain |\include| mechanism:
\begin{itemize}
\item
The child files cannot be compiled on their own,
they can only be compiled via the main file.
A naive editing environment
(such as a text editor with an option
to have the current file processed by \LaTeX)
may require one to switch to the main file before compiling;
attempting to compile the child file produces errors.
\item
The main file must be modified (each time)
to adjust the |\includeonly| command
to the present needs. This easily leaves the main file in a messy state.
\item
The generated document will always carry the filename
of the main document. This is inconvenient if
several child files are to be compiled and
to be kept for distribution.
\end{itemize}

The present package provides a simple interface
to make child files individually compilable by \LaTeX{}.
Compiling a child file then has the same effect as compiling
the main file with an |\includeonly| command
to select the appropriate child.
Moreover the generated document will carry the name of the child
rather than the main file.
This resolves all three above issues.

This feature is meant to make the editing of books,
thesis documents and lecture notes somewhat more convenient.
However, the package can also be used efficiently for
composing a series of documents (such as exercise sheets)
which are typically distributed individually.
It then assists the author in generating the individual documents
(potentially in different versions)
as well as a document containing the collected series.
Another application is in developing style files
or other kinds of included material
where compilation of the style file could redirect
to a sample or test file.

%%%%%%%%%%%%%%%%%%%%%%%%%%%%%%%%%%%%%%%%%%%%%%%%%%%%%%%%%%%%%%%%%%%%%%%%%%%%%%%%
%%%%%%%%%%%%%%%%%%%%%%%%%%%%%%%%%%%%%%%%%%%%%%%%%%%%%%%%%%%%%%%%%%%%%%%%%%%%%%%%
\section{Usage}

First of all, the package \textsf{childdoc} is \emph{not} a standard
\LaTeXe{} |.sty| style file! Therefore it needs to be invoked in
a non-standard way.

%%%%%%%%%%%%%%%%%%%%%%%%%%%%%%%%%%%%%%%%%%%%%%%%%%%%%%%%%%%%%%%%%%%%%%%%%%%%%%%%
\subsection{Included Files}
\label{sec:include}

%%%%%%%%%%%%%%%%%%%%%%%%%%%%%%%%%%%%%%%%
\DescribeMacro{\childdocmain}
To use the package, add the commands
\begin{center}
\begin{tabular}{l}
|\input{childdoc.def}|\\
|\childdocmain{}|\\
\end{tabular}
\end{center}
at the very top of the main \LaTeX{} file,
in particular \emph{before} the |\documentclass| statement!
The argument of |\childdocmain| should be left empty
(but it must be present).

%%%%%%%%%%%%%%%%%%%%%%%%%%%%%%%%%%%%%%%%
\DescribeMacro{\childdocof}
Furthermore, add the commands
\begin{center}
\begin{tabular}{l}
|\input{childdoc.def}|\\
|\childdocof{|\textit{main}|}|\\
\end{tabular}
\end{center}
at the top of every child file \textit{child}
which is included by |\include{|\textit{child}|}|
from within the main file
(or at least for those files to be compiled individually).
The argument \textit{main} must be the filename of the main file.

There are a couple of
considerations in setting up the main and child documents:

%%%%%%%%%%%%%%%%%%%%%%%%%%%%%%%%%%%%%%%%
\paragraph{Restrictions.}

Please note the following restrictions:
\begin{itemize}
\item
|\childdocmain| must be called with one argument \textit{main}
to ensure compatibility with earlier version of the package.
It must either be empty (|\childdocmain{}|)
or precisely match the filename of the main file in which it is specified.
See \secref{sec:detection} for further information.
\item
The filename \textit{main} must be specified without the |.tex| extension.
\item
The filename \textit{main} is case sensitive
(even in case-insensitive file systems)
due to internal string comparison.
\item
The argument \textit{main} should be fully expanded, it cannot be a macro.
\item
Subdirectories and special characters should be avoided in filenames.
\item
The command |\childdocmain{|\textit{main}|}| must be followed by a whitespace.
It should not be followed immediately by another command
or by a comment mark `|%|'.
This is because the \TeX{} parser reads the token immediately following
the argument of |\childdocmain| and puts it
at the beginning of every child section;
however, a white\-space is ignored.
\end{itemize}

%%%%%%%%%%%%%%%%%%%%%%%%%%%%%%%%%%%%%%%%
\paragraph{Content of Main File.}

It is advisable to place all content in the child files included by |\include|.
Any output contained in the main file will appear in all child documents
unless suppressed manually;
it cannot be suppressed automatically by the |\includeonly| directive
and thus should normally be avoided.
A method to include some content in the main file
by means of conditional processing is described in \secref{sec:conditional}.

%%%%%%%%%%%%%%%%%%%%%%%%%%%%%%%%%%%%%%%%
\paragraph{Page Numbering.}

When only a part of the document is compiled,
the appropriate numbering of pages
(as well as other status parameters)
is determined from the |.aux| files.
The latter contain information from previous passes.
However this information needs to propagate through
all intermediate child documents.
Therefore the page numbering in child documents may well
be inconsistent until the complete document is compiled at least once.

A useful (if unconventional) way to always ensure a consistent
page numbering is to restart the numbering in each child document
and denote the pages by `\textit{child}|.|\textit{page}'
where \textit{child} represents the chapter/section number of the child file.
This can be achieved by the command
|\numberwithin{page}{|\textit{child}|}|
of the \textsf{amsmath} package
where \textit{child} can be |chapter| or |section|
depending on the chosen structuring.
Alternatively, one can modify the macro |\thepage| appropriately
and reset the counter |page| at the start of each child file.

%%%%%%%%%%%%%%%%%%%%%%%%%%%%%%%%%%%%%%%%%%%%%%%%%%%%%%%%%%%%%%%%%%%%%%%%%%%%%%%%
\subsection{Conditional Processing}
\label{sec:conditional}

The package provides a mechanism to compile different versions
of a document. To customise the versions further some conditional processing
can come in handy to distinguish which version is being compiled.
The package provides two macros to describe the compilation context:

%%%%%%%%%%%%%%%%%%%%%%%%%%%%%%%%%%%%%%%%
\DescribeMacro{\ifchilddoc}
The conditional |\ifchilddoc| distinguishes between the compilation of
child documents and the main document:
%
\begin{center}
|\ifchilddoc |\textit{child-code}| |[|\||else |\textit{main-code}]| \||fi|
\end{center}

%%%%%%%%%%%%%%%%%%%%%%%%%%%%%%%%%%%%%%%%
\DescribeMacro{\childdocname}
\DescribeMacro{\childdocjob}
The macro |\childdocname| contains the filename (without extension)
of the main or child file being processed.
Note that |\childdocjob| will always contain the name of the main file.

%%%%%%%%%%%%%%%%%%%%%%%%%%%%%%%%%%%%%%%%
\paragraph{Title Page.}

Conditional processing can be used to include a title or banner page
in the main document when proper precautions are taken.
Importantly, the code in the main file should ensure that the page counter
(as well as other status parameters which are stored in the |.aux| files)
takes the same value after the conditional processing.
Otherwise the page numbers may take divergent values
depending on which part is compiled.

For example, a title page could be declared by:
%
\begin{center}
\begin{tabular}{l}
|\ifchilddoc\||else|\\
|\addtocounter{page}{-1}|\\
\textit{code for title page}\\
|\newpage|\\
|\||fi|
\end{tabular}
\end{center}
%
A banner page for the child documents can be generated by:
%
\begin{center}
\begin{tabular}{l}
|\ifchilddoc|\\
|\addtocounter{page}{-1}|\\
\textit{code for banner page}\\
|\newpage|\\
|\||fi|
\end{tabular}
\end{center}
%
Here one could write a message such as:
\begin{center}
|This is the part \childdocname{} of \childdocjob{}.|
\end{center}

%%%%%%%%%%%%%%%%%%%%%%%%%%%%%%%%%%%%%%%%%%%%%%%%%%%%%%%%%%%%%%%%%%%%%%%%%%%%%%%%
\subsection{Flags}
\label{sec:flags}

The package makes it easy to generate different versions
of the main or child documents.
To this end compilation flags can be defined
and assigned different default values.
They will be particularly useful in conjunction
with the forwarding mechanism described in \secref{sec:forward}.

For example, it may be useful to have a flag |\version|
which can be set to |draft| or |final|.
The document source will contain some conditional code
depending on the value of |\version|.
Suppose further, the flag should default to |final| for the main file
and to |draft| for child files
which is a natural assignment for editing the document.
This is achieved by placing the following code
in the preamble of the main document
(below the |\childdocmain| directive):
%
\begin{center}
\begin{tabular}{l}
|\ifchilddoc|\\
|\providecommand{\version}{draft}|\\
|\||else|\\
|\providecommand{\version}{final}|\\
|\||fi|
\end{tabular}
\end{center}
%
The definition by |\providecommand| makes sure
that previous definitions are not overwritten.
Further statements |\providecommand{\version}{...}|
can thus be added before the above code to override it.

For the main file, one might add a line
(between |\childdocmain| and the above block)
%
\begin{center}
|%\ifchilddoc\||else\providecommand{\version}{draft}\||fi|
\end{center}
%
which can be uncommented to produce a draft version.
Likewise one can add a line to the very top of a child file
(above the |\childdocof{|\textit{main}|}| directive)
%
\begin{center}
|%\providecommand{\version}{final}|
\end{center}
%
which can be uncommented to produce the final version of this child document.

%%%%%%%%%%%%%%%%%%%%%%%%%%%%%%%%%%%%%%%%%%%%%%%%%%%%%%%%%%%%%%%%%%%%%%%%%%%%%%%%
\subsection{Forwarding}
\label{sec:forward}

Different versions of the main or child documents
using compilation flags as described in \secref{sec:flags}
can be (permanently) stored in different files
for convenient compilation, viewing and distribution.
To this end, the package defines a command
to pass on compilation to a different file:

%%%%%%%%%%%%%%%%%%%%%%%%%%%%%%%%%%%%%%%%
\DescribeMacro{\childdocforward}
The command |\childdocforward| redirects processing to
another source file:
%
\begin{center}
\begin{tabular}{l}
|\input{childdoc.def}|\\
|\childdocforward[|\textit{main}|]{|\textit{dest}|}|\\
\end{tabular}
\end{center}
%
The argument \textit{dest} is the destination file
(without extension).
It should be the main file or one of the child files.
Note that further \textsf{childdoc} directives
such as |\childdocof| and |\childdocforward|
in the indicated file will be processed in this form.
The optional argument \textit{main}
passes on directly to the main file \textit{main}
while pretending to compile the child \textit{dest}.
This form behaves as if \textit{dest}
issues |\childdocof{|\textit{main}|}| right away,
and no further \textsf{childdoc} directives will be processed.

%%%%%%%%%%%%%%%%%%%%%%%%%%%%%%%%%%%%%%%%
\DescribeMacro{\...prefix}
In the alternative form |\childdocforwardprefix|,
%
\begin{center}
\begin{tabular}{l}
|\input{childdoc.def}|\\
|\childdocforwardprefix[|\textit{main}|]{|\textit{prefix}|}{|\textit{dest}|}|
\end{tabular}
\end{center}
%
the destination file is determined by a pattern
depending on the current file:
To make this work, the current file must be called
`{\textit{prefix}\hspace{0.2em}\textit{suffix}}'
with \textit{prefix} matching precisely the argument.
Processing is then passed on to the file
`{\textit{dest}\hspace{0.2em}\textit{suffix}}'.
Surely, the same effect is achieved by
directly specifying the
argument `{\textit{dest}\hspace{0.2em}\textit{suffix}}'
in the first form.
However, that requires to set up a different file
for each child. With the alternative form of the command
all these files can have exactly the same content
which simplifies setting them up and maintaining them.

For example, the following file |draft.tex|
with a compilation flag |\version| as described in \secref{sec:flags}
compiles the main document as a draft:
%
\begin{center}
\begin{tabular}{l}
|\def\version{draft}|\\
|\input{childdoc.def}|\\
|\childdocforward{|\textit{main}|}|
\end{tabular}
\end{center}
%
Likewise, the following files |final|\textit{nn}|.tex|
compile the final version of the child document
|child|\textit{nn}|.tex|:
%
\begin{center}
\begin{tabular}{l}
|\def\version{final}|\\
|\input{childdoc.def}|\\
|\childdocforwardprefix{final}{child}|
\end{tabular}
\end{center}
%

Note that when several versions of a main file and/or of each child file
are to be generated, it may be convenient to set up a |Makefile| or
shell script to automatise the process.

%%%%%%%%%%%%%%%%%%%%%%%%%%%%%%%%%%%%%%%%%%%%%%%%%%%%%%%%%%%%%%%%%%%%%%%%%%%%%%%%
\subsection{Command Line Processing}
\label{sec:commandline}

The effect of redirection files can also be achieved by invoking
the \LaTeX{} compiler with a more elaborate command line.
Most conveniently this should be done as part
of a shell script or a |Makefile|.

When using \textsf{childdoc} in the main file, the following
command lines effectively perform a redirection
(note that depending on the shell being used,
backslashes may have to be doubled: `|\|' $\to$ `|\\|'):
%
\begin{center}
|... -jobname "|\textit{target}|" |\\|"|[\textit{flags}]%
|\input{childdoc.def}\childdocforward[|\textit{main}|]{|\textit{dest}|}"|
\end{center}
%
Here \textit{target} is the name of the output file,
\textit{main} is the name of the main file
and \textit{dest} is the name of the main or child file to be processed
(all filenames without extensions).
The optional argument \textit{main} can be omitted
if \textit{main} matches \textit{dest}.
Optionally, compilation \textit{flags} can be defined via |\def| commands.
This command line makes the \TeX{} engine believe
it is compiling the file \textit{target}
whose content is specified as the latter parameter.
The provided code then forwards the processing to
\textit{main} or \textit{dest} as described in \secref{sec:forward}.

%%%%%%%%%%%%%%%%%%%%%%%%%%%%%%%%%%%%%%%%%%%%%%%%%%%%%%%%%%%%%%%%%%%%%%%%%%%%%%%%
\subsection{Include by Input}
\label{sec:input}

Including child documents by |\include| has some restrictions by design.
Most notably, the content of a child document always occupies
its own set of pages; pages cannot be shared between child documents.
Usually, this behaviour makes perfect sense
because each child document contain an essential part of the document.
However, in some situations it may be desirable to compose
a document from a collection of parts
without having mandatory page breaks between then.
For this case, the package
provides a mechanism to include parts
by |\input| which can also be processed individually.
However, by construction this mechanism
requires manual handling of the content to be output.

%%%%%%%%%%%%%%%%%%%%%%%%%%%%%%%%%%%%%%%%
\DescribeMacro{\ifchilddocmanual}
The main file should be prepared as usual, see \secref{sec:include}.
However, the document body must make a distinction
between processing of an individual part and of the main document, e.g.:
%
\begin{center}
\begin{tabular}{l}
|\ifchilddocmanual|\\
|\input{\childdocname}|\\
|\||else|\\
\textit{document body with }|\input{|\textit{part}|}|\\
|\||fi|
\end{tabular}
\end{center}
%
The conditional |\ifchilddocmanual| is true whenever
a part to be included by |\input| is being compiled,
and the name of the part is stored in |\childdocname|.

%%%%%%%%%%%%%%%%%%%%%%%%%%%%%%%%%%%%%%%%
\DescribeMacro{\childdocby}
Each part to be included by |\input| should start with:
%
\begin{center}
\begin{tabular}{l}
|\input{childdoc.def}|\\
|\childdocby{|\textit{main}|}|\\
\end{tabular}
\end{center}
%
The directive |\childdocby| is similar to |\childdocof|
described in \secref{sec:include},
but the subsequent selection of content must be done manually.
To that end, both |\ifchilddoc| and |\ifchilddocmanual|
will be true upon processing of a part,
and the name of the part is stored in |\childdocname|.
Note that |\jobname| will be set to the filename of the current part
so that each part receives an individual |.aux| file
that does not interfere with the |.aux| file(s) of the main document.
This behaviour can be altered by the alternative form
|\childdocby[*]{|\textit{main}|}| (with a non-empty optional argument)
which uses the |.aux| file of the main document
by setting |\jobname| to \textit{main}.

%%%%%%%%%%%%%%%%%%%%%%%%%%%%%%%%%%%%%%%%%%%%%%%%%%%%%%%%%%%%%%%%%%%%%%%%%%%%%%%%
\subsection{Driver Development}
\label{sec:driver}

The \textsf{childdoc} mechanism can also be use for the development
of definition files such as \LaTeX{} styles or classes.
This case differs from the above setup with multiple parts
included by |\include| in that no |\includeonly| should be invoked.
This can be achieved by starting the include file
(before |\ProvidesPackage|) with:
%
\begin{center}
\begin{tabular}{l}
|\input{childdoc.def}|\\
|\childdocforward{|\textit{main}|}|\\
\end{tabular}
\end{center}
%
or alternatively with:
%
\begin{center}
\begin{tabular}{l}
|\input{childdoc.def}|\\
|\childdocby{|\textit{main}|}|\\
\end{tabular}
\end{center}
%
Both forms have slightly different effects as described above.
The main file is prepared as usual, see \secref{sec:include}.

%%%%%%%%%%%%%%%%%%%%%%%%%%%%%%%%%%%%%%%%%%%%%%%%%%%%%%%%%%%%%%%%%%%%%%%%%%%%%%%%
\subsection{Legacy Detection}
\label{sec:detection}

The directive |\childdocmain| in the main file can detect
whether the complete document or merely a child is to be compiled
even without using the directive |\childdocof|.
This method is deprecated because it is less robust
and there is no compelling reason to use it;
it is merely provided for backward compatibility
and it may be removed in future versions.

If the detection mechanism is to be used,
it is mandatory to correctly specify
the filename of the main file as the argument of |\childdocmain|:
%
\begin{center}
\begin{tabular}{l}
|\input{childdoc.def}|\\
|\childdocmain{|\textit{main}|}|\\
\end{tabular}
\end{center}
%
If |\jobname| does not match the argument \textit{main} of |\childdocmain|,
it is assumed that |\jobname| points to the child file to be compiled.
When using |\childdocmain| with the main file specified as argument,
it suffices to start a child file
with just |\input{|\textit{main}|}|
without loading of the package and using |\childdocof|.
If instead all processing is done
with the appropriate \textsf{childdoc} directives,
the argument of \textit{main} of |\childdocmain| can be empty.

An alternative version of the command line processing described
in \secref{sec:commandline} using the detection mechanism reads:
%
\begin{center}
|... -jobname "|\textit{target}|" "|[\textit{flags}]%
[|\def\jobname{|\textit{dest}|}|]|\input{|\textit{main}|}"|
\end{center}

%%%%%%%%%%%%%%%%%%%%%%%%%%%%%%%%%%%%%%%%%%%%%%%%%%%%%%%%%%%%%%%%%%%%%%%%%%%%%%%%
\subsection{Manual Code}
\label{sec:manual}

In case one cannot be certain whether the definitions file |childdoc.def|
is installed on the target \TeX{} distribution
and one prefers not to ship it,
it is conceivable to paste a few relevant commands into the sources.

To that end, drop all statements |\input{childdoc.def}|
and perform the replacements as outlined below.
Instead of |\childdocmain{|\textit{main}|}| add the following code
to the top of the main file:
%
\begin{center}
\begin{tabular}{l}
|\||ifdefined\childdocname\endinput\||fi\newif\ifchilddoc|\\
|\edef\childdocname{\scantokens\expandafter{\jobname\noexpand}}|\\
|\def\childdocmain{|\textit{main}|}\||ifx\childdocmain\childdocname\||else|\\
|\childdoctrue\includeonly{\childdocname}\let\jobname\childdocmain\||fi|\\
\end{tabular}
\end{center}
%
Instead of |\childdocof{|\textit{main}|}| just include the main file
at the top of each child file:
%
\begin{center}
|\input{|\textit{main}|}|
\end{center}
%
A simple redirection |\childdocforward{|\textit{dest}|}| is achieved by:
%
\begin{center}
|\def\jobname{|\textit{dest}|}\input{\jobname}|
\end{center}
%
The redirection with prefix
|\childdocforwardprefix[|\textit{prefix}|]{|\textit{dest}|}|
is accomplished by:
%
\begin{center}
\begin{tabular}{l}
|{\edef\jobname{\scantokens\expandafter{\jobname\noexpand}}|\\
|\def\redirectjob |\textit{prefix}|#1~~~{\gdef\jobname{|\textit{dest}|#1}}|\\
|\expandafter\redirectjob\jobname~~~}\input{\jobname}|
\end{tabular}
\end{center}

In an alternative approach,
child documents can be compiled by a specific command line
without additional code or specific definitions:
%
\begin{center}
|... -jobname "|\textit{target}|" "|[\textit{flags}]%
|\includeonly{|\textit{dest}|}\input{|\textit{main}|}"|
\end{center}
%

%%%%%%%%%%%%%%%%%%%%%%%%%%%%%%%%%%%%%%%%%%%%%%%%%%%%%%%%%%%%%%%%%%%%%%%%%%%%%%%%
%%%%%%%%%%%%%%%%%%%%%%%%%%%%%%%%%%%%%%%%%%%%%%%%%%%%%%%%%%%%%%%%%%%%%%%%%%%%%%%%
\section{Information}

%%%%%%%%%%%%%%%%%%%%%%%%%%%%%%%%%%%%%%%%%%%%%%%%%%%%%%%%%%%%%%%%%%%%%%%%%%%%%%%%
\subsection{Copyright}

Copyright \copyright{} 2017--2018 Niklas Beisert

This work may be distributed and/or modified under the
conditions of the \LaTeX{} Project Public License, either version 1.3
of this license or (at your option) any later version.
The latest version of this license is in
  \url{http://www.latex-project.org/lppl.txt}
and version 1.3 or later is part of all distributions of \LaTeX{}
version 2005/12/01 or later.

This work has the LPPL maintenance status `maintained'.

The Current Maintainer of this work is Niklas Beisert.

This work consists of the files |README.txt|, |childdoc.ins| and |childdoc.dtx|
as well as the derived files |childdoc.def|, |cdocsamp.tex|
with |cdocsch1.tex|, |cdocsch2.tex|, |cdocspt3.tex|, |cdocspt4.tex|,
|cdocsdrf.tex|, |cdocsfn1.tex|, |cdocsfn2.tex|
as well as |childdoc.pdf|.

%%%%%%%%%%%%%%%%%%%%%%%%%%%%%%%%%%%%%%%%%%%%%%%%%%%%%%%%%%%%%%%%%%%%%%%%%%%%%%%%
\subsection{Files and Installation}

The package consists of the files:
%
\begin{center}
\begin{tabular}{ll}
    |README.txt|   & readme file \\
    |childdoc.ins| & installation file \\
    |childdoc.dtx| & source file \\
    |childdoc.def| & definition file \\
    |cdocsamp.tex| & sample main file \\
    |cdocsch1.tex| & sample include file \\
    |cdocsch2.tex| & sample include file \\
    |cdocspt3.tex| & sample part file \\
    |cdocspt4.tex| & sample part file \\
    |cdocsdrf.tex| & sample redirection file \\
    |cdocsfn1.tex| & sample redirection file \\
    |cdocsfn2.tex| & sample redirection file \\
    |childdoc.pdf| & manual
\end{tabular}
\end{center}
%
The distribution consists of the files
|README.txt|, |childdoc.ins| and |childdoc.dtx|.
%
\begin{itemize}
\item
Run (pdf)\LaTeX{} on |childdoc.dtx|
to compile the manual |childdoc.pdf| (this file).
\item
Run \LaTeX{} on |childdoc.ins| to create the definitions file |childdoc.def|
and the sample |cdocsamp.tex| with include files
|cdocsch1.tex|, |cdocsch2.tex|, |cdocspt3.tex|, |cdocspt4.tex|,
|cdocsdrf.tex|, |cdocsfn1.tex|, |cdocsfn2.tex|.
Then copy the file |childdoc.def| to an appropriate directory of your \LaTeX{}
distribution, e.g.\ \textit{texmf-root}|/tex/latex/childdoc|.
\end{itemize}

%%%%%%%%%%%%%%%%%%%%%%%%%%%%%%%%%%%%%%%%%%%%%%%%%%%%%%%%%%%%%%%%%%%%%%%%%%%%%%%%
\subsection{Related CTAN Packages}

There are several other packages which offer a similar functionality:
%
\begin{itemize}
\item
The packages
\href{http://ctan.org/pkg/docmute}{\textsf{docmute}},
\href{http://ctan.org/pkg/includex}{\textsf{includex}} and
\href{http://ctan.org/pkg/standalone}{\textsf{standalone}}
provide commands to include only the document body of
a child file thus allowing both files to be compiled individually.
\item
The packages \href{http://ctan.org/pkg/subdocs}{\textsf{subdocs}}
and \href{http://ctan.org/pkg/subfiles}{\textsf{subfiles}}
provide structures in which the main and child documents can be
encapsulated and allowing them to be compiled individually.
The inclusion mechanism is different from the conventional |\include|.
\item
The package \href{http://ctan.org/pkg/combine}{\textsf{combine}}
is an elaborate solution to combine several documents into one.
\end{itemize}
%
See also the CTAN topic \href{http://ctan.org/topic/subdocs}{\textsf{subdocs}}
for further related packages.
The present package differs from the above solutions in that
a document structure constructed with the conventional |\include| mechanism
just needs two extra commands at the top of every file
such that all constituent files can be compiled individually.

%%%%%%%%%%%%%%%%%%%%%%%%%%%%%%%%%%%%%%%%%%%%%%%%%%%%%%%%%%%%%%%%%%%%%%%%%%%%%%%%
%\subsection{Feature Suggestions}
%
%The following is a list of features which may be useful for future
%versions of this package:
%%
%\begin{itemize}
%\item
%\ldots
%\end{itemize}

%%%%%%%%%%%%%%%%%%%%%%%%%%%%%%%%%%%%%%%%%%%%%%%%%%%%%%%%%%%%%%%%%%%%%%%%%%%%%%%%
\subsection{Revision History}

%%%%%%%%%%%%%%%%%%%%%%%%%%%%%%%%%%%%%%%%
\paragraph{v2.0:} 2018/12/30

\begin{itemize}
\item
immediate forward processing
\item
added |\childdocby| mechanism
\item
manual restructured
\end{itemize}

%%%%%%%%%%%%%%%%%%%%%%%%%%%%%%%%%%%%%%%%
\paragraph{v1.6:} 2018/01/17

\begin{itemize}
\item
application for development of include files
\item
corrections to manual
\end{itemize}

%%%%%%%%%%%%%%%%%%%%%%%%%%%%%%%%%%%%%%%%
\paragraph{v1.5:} 2017/05/21

\begin{itemize}
\item
more complete structuring introduced
\item
|\childdocof| introduced
\item
|\childdoc| renamed to |\childdocmain|
\item
|\childredirect| renamed to |\childdocforward| and |\childdocforwardprefix|
and functionality expanded
\end{itemize}

%%%%%%%%%%%%%%%%%%%%%%%%%%%%%%%%%%%%%%%%
\paragraph{v1.0:} 2017/04/27

\begin{itemize}
\item
manual and install package
\item
first version published on CTAN
\end{itemize}

%%%%%%%%%%%%%%%%%%%%%%%%%%%%%%%%%%%%%%%%
\paragraph{v0.6:} 2017/04/26

\begin{itemize}
\item
redirection mechanism added
\end{itemize}

%%%%%%%%%%%%%%%%%%%%%%%%%%%%%%%%%%%%%%%%
\paragraph{v0.5:} 2017/04/26

\begin{itemize}
\item
functionality in definition file
\end{itemize}


%%%%%%%%%%%%%%%%%%%%%%%%%%%%%%%%%%%%%%%%%%%%%%%%%%%%%%%%%%%%%%%%%%%%%%%%%%%%%%%%
%%%%%%%%%%%%%%%%%%%%%%%%%%%%%%%%%%%%%%%%%%%%%%%%%%%%%%%%%%%%%%%%%%%%%%%%%%%%%%%%
%%%%%%%%%%%%%%%%%%%%%%%%%%%%%%%%%%%%%%%%%%%%%%%%%%%%%%%%%%%%%%%%%%%%%%%%%%%%%%%%
\appendix

\settowidth\MacroIndent{\rmfamily\scriptsize 000\ }

 \DocInput{childdoc.dtx}

\end{document}
%</driver>
% \fi
%
% %%%%%%%%%%%%%%%%%%%%%%%%%%%%%%%%%%%%%%%%%%%%%%%%%%%%%%%%%%%%%%%%%%%%%%%%%%%%%%
% %%%%%%%%%%%%%%%%%%%%%%%%%%%%%%%%%%%%%%%%%%%%%%%%%%%%%%%%%%%%%%%%%%%%%%%%%%%%%%
% \section{Sample}
%\iffalse
%<*samplemain>
%\fi
%
% The following presents a sample document
% with two chapters, two parts, a title page,
% a compile flag as well as three forwarding files to set the flag.
% It consists of eight |.tex| files:
% \begin{center}
% \begin{tabular}{ll}
% |cdocsamp.tex|&main file\\
% |cdocsch1.tex|&include file for chapter 1\\
% |cdocsch2.tex|&include file for chapter 2\\
% |cdocspt3.tex|&include file for part 3\\
% |cdocspt4.tex|&include file for part 4\\
% |cdocsdrf.tex|&forwarding file for main file in draft mode\\
% |cdocsfi1.tex|&forwarding file for final version of chapter 1\\
% |cdocsfi2.tex|&forwarding file for final version of chapter 2\\
% \end{tabular}
% \end{center}
% Each of the eight files can be compiled directly by the \LaTeX{} compiler.
%
% %%%%%%%%%%%%%%%%%%%%%%%%%%%%%%%%%%%%%%
% \paragraph{Main File.}
%
% The main file is called |cdocsamp.tex|.
%
% Load the \textsf{childdoc} definitions and
% declare the filename for the main document:
%    \begin{macrocode}
\input{childdoc.def}
\childdocmain{}
%    \end{macrocode}

% Optional override for |\version| flag:
%    \begin{macrocode}
%%\ifchilddoc\else\providecommand{\version}{draft}\fi
%    \end{macrocode}

% Define the default values for the |\version| flag
% (|final| for the main file and |draft| for childs):
%    \begin{macrocode}
\ifchilddoc
\providecommand{\version}{draft}
\else
\providecommand{\version}{final}
\fi
%    \end{macrocode}

% Load the standard document class:
%    \begin{macrocode}
\documentclass[12pt]{article}
%    \end{macrocode}

% Start the document body:
%    \begin{macrocode}
\begin{document}
%    \end{macrocode}

% Declare a title page.
% Print title, part of document being processed and version flag:
%    \begin{macrocode}
\addtocounter{page}{-1}
\begin{center}
{\LARGE\bfseries{}childdoc example\par}
\vspace{1cm}
\ifchilddoc
\ifchilddocmanual part\else chapter\fi:
`\childdocname' of `\childdocjob'\par
\else
main document: `\childdocjob'\par
\fi
version: \version\par
\end{center}
\newpage
%    \end{macrocode}

% Manually include selected file,
% otherwise process as usual:
%    \begin{macrocode}
\ifchilddocmanual
\section*{part `\childdocname'}
\input{\childdocname}
\else
%    \end{macrocode}

% Include the two chapters:
%    \begin{macrocode}
\include{cdocsch1}
\include{cdocsch2}
%    \end{macrocode}

% Include the two parts unless only chapters should be displayed:
%    \begin{macrocode}
\ifchilddoc\else
\section{part three}
\input{cdocspt3}
\section{part four}
\input{cdocspt4}
\fi
%    \end{macrocode}

% Process as usual until here:
%    \begin{macrocode}
\fi
%    \end{macrocode}

% End of document body:
%    \begin{macrocode}
\end{document}
%    \end{macrocode}
%\iffalse
%</samplemain>
%\fi
%
% %%%%%%%%%%%%%%%%%%%%%%%%%%%%%%%%%%%%%%
% \paragraph{Chapter Include Files.}
%
% The include files are called |cdocsch1.tex| and |cdocsch2.tex|.
%
%\iffalse
%<*samplechap1|samplechap2>
%\fi

% Optional override for |\version| flag:
%    \begin{macrocode}
%%\providecommand{\version}{final}
%    \end{macrocode}

% Include the main document:
%    \begin{macrocode}
\input{childdoc.def}
\childdocof{cdocsamp}
%    \end{macrocode}

%\iffalse
%</samplechap1|samplechap2>
%\fi
%
%\iffalse
%<*samplechap1>
%\fi
% Some text for chapter 1:
%    \begin{macrocode}
\section{one}
some text in chapter one
%    \end{macrocode}

%\iffalse
%</samplechap1>
%\fi
% Some text for chapter 2:
%\iffalse
%<*samplechap2>
%\fi
%    \begin{macrocode}
\section{two}
more text in chapter two
%    \end{macrocode}

%\iffalse
%</samplechap2>
%\fi
%
% %%%%%%%%%%%%%%%%%%%%%%%%%%%%%%%%%%%%%%
% \paragraph{Part Include Files.}
%
% The include files are called |cdocspt3.tex| and |cdocspt4.tex|.
%
%\iffalse
%<*samplepart3|samplepart4>
%\fi

% Optional override for |\version| flag:
%    \begin{macrocode}
%%\providecommand{\version}{final}
%    \end{macrocode}

% Include the main document:
%    \begin{macrocode}
\input{childdoc.def}
\childdocby{cdocsamp}
%    \end{macrocode}

%\iffalse
%</samplepart3|samplepart4>
%\fi
%
%\iffalse
%<*samplepart3>
%\fi
% Some text for part 3:
%    \begin{macrocode}
some text in part three
%    \end{macrocode}

%\iffalse
%</samplepart3>
%\fi
% Some text for part 4:
%\iffalse
%<*samplepart4>
%\fi
%    \begin{macrocode}
more text in part four
%    \end{macrocode}

%\iffalse
%</samplepart4>
%\fi
%
% %%%%%%%%%%%%%%%%%%%%%%%%%%%%%%%%%%%%%%
% \paragraph{Forwarding for a Complete Draft.}
%
% The following forwarding file |cdocsdrf.tex|
% compiles the main document in draft mode:
%\iffalse
%<*sampledraft>
%\fi
%    \begin{macrocode}
\def\version{draft}
\input{childdoc.def}
\childdocforward{cdocsamp}
%    \end{macrocode}

%\iffalse
%</sampledraft>
%\fi
%
% %%%%%%%%%%%%%%%%%%%%%%%%%%%%%%%%%%%%%%
% \paragraph{Forwarding for Final Version of the Chapters.}
%
% The following forwarding files |cdocsfn1.tex| and |cdocsfn2.tex|
% (with identical content)
% compile the final versions of the child documents
% |cdocsch1.tex| and |cdocsch2.tex|, respectively:
%\iffalse
%<*samplefinal>
%\fi
%    \begin{macrocode}
\def\version{final}
\input{childdoc.def}
\childdocforwardprefix[cdocsamp]{cdocsfn}{cdocsch}
%    \end{macrocode}

%\iffalse
%</samplefinal>
%\fi
%
% %%%%%%%%%%%%%%%%%%%%%%%%%%%%%%%%%%%%%%
% \paragraph{Command Line Processing.}
%
% The following three command lines generate the output files
% |cdocscld|, |cdocscl1| and |cdocscl2|
% which should be identical to
% |cdocsdrf|, |cdocsch1| and |cdocsfn2|, respectively:
% \begin{center}
% \begin{tabular}{l}
% |latex -jobname cdocscld \|\\
% |  "\def\version{draft}\input{childdoc.def}\childdocforward{cdocsamp}"|\\
% |latex -jobname cdocscl1 \|\\
% |  "\input{childdoc.def}\childdocforward[cdocsamp]{cdocsch1}"|\\
% |latex -jobname cdocscl2 \|\\
% |  "\def\version{final}\input{childdoc.def}\childdocforward{cdocsch2}"|
% \end{tabular}
% \end{center}
% Note that the trailing backslash on each first line
% merely continues the input to the second line
% (for convenient cut ant paste).
% Furthermore, the command |latex| can be replaced by any
% of its alternative versions such as |pdflatex|.
%
% %%%%%%%%%%%%%%%%%%%%%%%%%%%%%%%%%%%%%%%%%%%%%%%%%%%%%%%%%%%%%%%%%%%%%%%%%%%%%%
% %%%%%%%%%%%%%%%%%%%%%%%%%%%%%%%%%%%%%%%%%%%%%%%%%%%%%%%%%%%%%%%%%%%%%%%%%%%%%%
% \section{Implementation}
%\iffalse
%<*package>
%\fi
%
% This section describes the definitions file |childdoc.def|.

% The definitions cannot be loaded using |\usepackage| or |\RequirePackage|
% which has a mechanism to prevent loading a style file more than once.
% When loading the definitions by means of |\input|
% multiple instances have to be prevented manually:
%\iffalse
%This code needs to be before the `\ProvidesFile' directive
%which is defined at the beginning of this file.
%Therefore it is also placed there and commented out here.
%</package>
%<*discard>
%\fi
%    \begin{macrocode}
\ifdefined\childdocmain\endinput\fi
%    \end{macrocode}
%\iffalse
%</discard>
%<*package>
%\fi
%
% \macro{\ifchilddoc}
% \macro{\ifchilddocmanual}
% The conditional |\ifchilddoc| tells whether a
% child (true) or main (false) document is being compiled.
% The conditional |\ifchilddocmanual| tells whether
% the |\includeonly| mechanism is used (false) or
% the selection of child files must be performed manually (true).
% The definitions initialise to false:
%    \begin{macrocode}
\newif\ifchilddoc
\newif\ifchilddocmanual
%    \end{macrocode}

% \macro{\childdocname}
% \macro{\childdocjob}
% The macro |\childdocname| stores the name of the main document
% to be compiled. The macro |\childdocjob| stores the name of
% the document on which the \LaTeX{} compiler was originally invoked.
% The content of |\jobname| cannot be compared
% to filenames specified in the source due to different catcodes.
% The following code rescans |\jobname|, stores the result
% in |\childdocname| and saves a copy in |\childdocjob|:
%    \begin{macrocode}
\edef\childdocname{\scantokens\expandafter{\jobname\noexpand}}
\let\childdocjob\childdocname
%    \end{macrocode}

% \macro{\childdocdisable}
% The macro |\childdocdisable| prevents the main file
% from being processed more than once.
% At this stage, the main document command |\childdocmain|
% is assumed to be called once again where it should do nothing.
% Any subsequent call to it should prevent
% a secondary processing of the main document
% It overwrites the forwarding commands
% |\childdocof| and |\childdocforward|
% with empty macros to prevent further inclusions of the main document:
%    \begin{macrocode}
\newcommand{\childdocdisable}
{
  \renewcommand{\childdocmain}[1]{\renewcommand{\childdocmain}[1]{\endinput}}
  \renewcommand{\childdocof}[1]{}
  \renewcommand{\childdocby}[2][]{}
  \renewcommand{\childdocforward}[2][]{}
  \renewcommand{\childdocdisable}{}
}
%    \end{macrocode}

% \macro{\childdocmain}
% The macro |\childdocmain| is to be called at the top of the main file
% with nothing or the main filename (without extension) as argument.
% First, it breaks loops.
% If the argument is not empty and does not match |\childdocname|
% (which is set by the first inclusion of |childdoc.def|),
% |\ifchilddoc| is set to true, |\includeonly| is applied to the child file
% and |\jobname| is set to the main file
% (for proper handling of |.aux| files):
%    \begin{macrocode}
\newcommand{\childdocmain}[1]
{
  \childdocdisable\childdocmain{}
  \if?#1?\else
    \begingroup
      \def\childdoctmp{#1}
      \ifx\childdoctmp\childdocname
        \def\childdoctmp{}
      \else
        \def\childdoctmp
        {
          \childdoctrue
          \includeonly{\childdocname}
          \def\childdocjob{#1}
          \def\jobname{#1}
        }
      \fi
      \expandafter
    \endgroup
    \childdoctmp
  \fi
}
%    \end{macrocode}

% \macro{\childdocof}
% The command |\childdocof| redirects
% compilation to the main file |#1|.
%    \begin{macrocode}
\newcommand{\childdocof}[1]
{
  \childdocdisable
  \childdoctrue
  \includeonly{\childdocname}
  \def\jobname{#1}
  \def\childdocjob{#1}
  \input{#1}
}
%    \end{macrocode}

% \macro{\childdocby}
% The command |\childdocby| ....
%    \begin{macrocode}
\newcommand{\childdocby}[2][]
{
  \childdocdisable
  \childdoctrue
  \childdocmanualtrue
  \if?#1?\else
    \def\jobname{#2}
  \fi
  \def\childdocjob{#2}
  \input{#2}
  \endinput
}
%    \end{macrocode}

% \macro{\childdocforward}
% The command |\childdocforward| redirects
% compilation to the main file or
% (if the optional argument is given) a child file.
% Parameters are set as if the main file
% or a child file starting with |\childdocof| was compiled.
% Then compilation is handed over to the main file:
%    \begin{macrocode}
\newcommand{\childdocforward}[2][]
{
  \begingroup
    \if?#1?
      \def\childdoctmp
      {
        \def\childdocname{#2}
        \def\childdocjob{#2}
        \def\jobname{#2}
        \input{#2}
        \endinput
      }
    \else
      \def\childdoctmp
      {
        \childdocdisable
        \def\childdocname{#2}
        \childdoctrue
        \includeonly{#2}
        \def\childdocjob{#1}
        \def\jobname{#1}
        \input{#1}
        \endinput
      }
    \fi
    \expandafter
  \endgroup
  \childdoctmp
}
%    \end{macrocode}

% \macro{\childdocforwardprefix}
% The command |\childdocforwardprefix| redirects
% compilation to the main or a child file by means of a pattern.
% The prefix |#1| in the current filename is replaced by |#2|
% and the suffix of the current filename is kept
% (it is assumed that the filename does not contain the substring `|~~~|'
% which is used as a delimiter).
% Compilation is handed over to the new file by |\childdocforward|:
%    \begin{macrocode}
\newcommand{\childdocforwardprefix}[3][]
{
  \begingroup
    \def\childdocextract #2##1~~~{\def\childdoctmp{\childdocforward[#1]{#3##1}}}
    \expandafter\childdocextract\childdocname~~~
    \expandafter
  \endgroup
  \childdoctmp
}
%    \end{macrocode}

% \macro{\childdoc}
% The deprecated macro |\childdoc| is a legacy version of |\childdocmain|:
%    \begin{macrocode}
\newcommand{\childdoc}{\childdocmain}
%    \end{macrocode}

% \macro{\childdocredirect}
% The deprecated macro |\childdocredirect| is a legacy version
% of |\childdocforward| and |\childdocforwardprefix|:
%    \begin{macrocode}
\newcommand{\childdocredirect}[2][]
{
  \begingroup
    \if?#1?
      \def\childdoctmp{\childdocforward{#2}}
    \else
      \def\childdoctmp{\childdocforwardprefix{#1}{#2}}
    \fi
    \expandafter
  \endgroup
  \childdoctmp
}
%    \end{macrocode}

%\iffalse
%</package>
%\fi
%
\endinput
\childdocforward[|\textit{main}|]{|\textit{dest}|}"|
\end{center}
%
Here \textit{target} is the name of the output file,
\textit{main} is the name of the main file
and \textit{dest} is the name of the main or child file to be processed
(all filenames without extensions).
The optional argument \textit{main} can be omitted
if \textit{main} matches \textit{dest}.
Optionally, compilation \textit{flags} can be defined via |\def| commands.
This command line makes the \TeX{} engine believe
it is compiling the file \textit{target}
whose content is specified as the latter parameter.
The provided code then forwards the processing to
\textit{main} or \textit{dest} as described in \secref{sec:forward}.

%%%%%%%%%%%%%%%%%%%%%%%%%%%%%%%%%%%%%%%%%%%%%%%%%%%%%%%%%%%%%%%%%%%%%%%%%%%%%%%%
\subsection{Include by Input}
\label{sec:input}

Including child documents by |\include| has some restrictions by design.
Most notably, the content of a child document always occupies
its own set of pages; pages cannot be shared between child documents.
Usually, this behaviour makes perfect sense
because each child document contain an essential part of the document.
However, in some situations it may be desirable to compose
a document from a collection of parts
without having mandatory page breaks between then.
For this case, the package
provides a mechanism to include parts
by |\input| which can also be processed individually.
However, by construction this mechanism
requires manual handling of the content to be output.

%%%%%%%%%%%%%%%%%%%%%%%%%%%%%%%%%%%%%%%%
\DescribeMacro{\ifchilddocmanual}
The main file should be prepared as usual, see \secref{sec:include}.
However, the document body must make a distinction
between processing of an individual part and of the main document, e.g.:
%
\begin{center}
\begin{tabular}{l}
|\ifchilddocmanual|\\
|\input{\childdocname}|\\
|\||else|\\
\textit{document body with }|\input{|\textit{part}|}|\\
|\||fi|
\end{tabular}
\end{center}
%
The conditional |\ifchilddocmanual| is true whenever
a part to be included by |\input| is being compiled,
and the name of the part is stored in |\childdocname|.

%%%%%%%%%%%%%%%%%%%%%%%%%%%%%%%%%%%%%%%%
\DescribeMacro{\childdocby}
Each part to be included by |\input| should start with:
%
\begin{center}
\begin{tabular}{l}
|% \iffalse
%
% childdoc.dtx Copyright (C) 2017-2018 Niklas Beisert
%
% This work may be distributed and/or modified under the
% conditions of the LaTeX Project Public License, either version 1.3
% of this license or (at your option) any later version.
% The latest version of this license is in
%   http://www.latex-project.org/lppl.txt
% and version 1.3 or later is part of all distributions of LaTeX
% version 2005/12/01 or later.
%
% This work has the LPPL maintenance status `maintained'.
%
% The Current Maintainer of this work is Niklas Beisert.
%
% This work consists of the files childdoc.dtx and childdoc.ins
% and the derived files childdoc.def and cdocsamp.tex with
% cdocsch1.tex, cdocsch2.tex, cdocsdrf.tex, cdocsfn1.tex, cdocsfn2.tex.
%
%<package>\ifdefined\childdocmain\endinput\fi
%<package>\ProvidesFile{childdoc.def}[2018/12/30 v2.0 child document driver]
%<samplemain>\ProvidesFile{cdocsamp.tex}[2018/12/30 v2.0 sample for childdoc]
%<*driver>
%\ProvidesFile{childdoc.drv}[2018/12/30 v2.0 childdoc reference manual file]
\PassOptionsToClass{10pt,a4paper}{article}
\documentclass{ltxdoc}

\usepackage[margin=35mm]{geometry}
\usepackage{hyperref}
\usepackage{hyperxmp}
\usepackage[usenames]{color}

\hypersetup{colorlinks=true}
\hypersetup{pdfstartview=FitH}
\hypersetup{pdfpagemode=UseNone}
\hypersetup{pdfsource={}}
\hypersetup{pdflang={en-UK}}
\hypersetup{pdfcopyright={Copyright 2017-2018 Niklas Beisert.
  This work may be distributed and/or modified under the
  conditions of the LaTeX Project Public License, either version 1.3
  of this license or (at your option) any later version.}}
\hypersetup{pdflicenseurl={http://www.latex-project.org/lppl.txt}}
\hypersetup{pdfcontactaddress={ETH Zurich, ITP, HIT K,
  Wolfgang-Pauli-Strasse 27}}
\hypersetup{pdfcontactpostcode={8093}}
\hypersetup{pdfcontactcity={Zurich}}
\hypersetup{pdfcontactcountry={Switzerland}}
\hypersetup{pdfcontactemail={nbeisert@itp.phys.ethz.ch}}
\hypersetup{pdfcontacturl={http://people.phys.ethz.ch/\xmptilde nbeisert/}}

\newcommand{\secref}[1]{\hyperref[#1]{section \ref*{#1}}}

\parskip1ex
\parindent0pt
\let\olditemize\itemize
\def\itemize{\olditemize\parskip0pt}

\begin{document}

\title{The \textsf{childdoc} Package}
\hypersetup{pdftitle={The childdoc Package}}
\author{Niklas Beisert\\[2ex]
  Institut f\"ur Theoretische Physik\\
  Eidgen\"ossische Technische Hochschule Z\"urich\\
  Wolfgang-Pauli-Strasse 27, 8093 Z\"urich, Switzerland\\[1ex]
  \href{mailto:nbeisert@itp.phys.ethz.ch}
  {\texttt{nbeisert@itp.phys.ethz.ch}}}
\hypersetup{pdfauthor={Niklas Beisert}}
\hypersetup{pdfsubject={Manual for the LaTeX2e Package childdoc}}
\date{30 December 2018, \textsf{v2.0}}
\maketitle

\begin{abstract}\noindent
\textsf{childdoc} is a \LaTeXe{} package
that enables the direct compilation
of document sections included by |\include|
to individual files.
\end{abstract}

\begingroup
\parskip0ex
\tableofcontents
\endgroup

%%%%%%%%%%%%%%%%%%%%%%%%%%%%%%%%%%%%%%%%%%%%%%%%%%%%%%%%%%%%%%%%%%%%%%%%%%%%%%%%
%%%%%%%%%%%%%%%%%%%%%%%%%%%%%%%%%%%%%%%%%%%%%%%%%%%%%%%%%%%%%%%%%%%%%%%%%%%%%%%%
\section{Introduction}

\LaTeX{} provides a mechanism to structure a large document (such as a book)
into a main file and several child files (containing the chapters)
using the |\include| command.
This mechanism is beneficial for documents
which span hundreds of pages in order to
make the source file(s) more manageable.
Moreover, compilation can be restricted to
selected child files by means of the |\includeonly| command.
The latter feature can be used to reduce the compilation time while editing
(this was significantly more useful in the earlier days of \LaTeX{})
or to generate a smaller document which is easier to navigate.
Another application of |\includeonly| is to generate
documents consisting of selected parts of the complete document.

However, there are a few drawbacks of the plain |\include| mechanism:
\begin{itemize}
\item
The child files cannot be compiled on their own,
they can only be compiled via the main file.
A naive editing environment
(such as a text editor with an option
to have the current file processed by \LaTeX)
may require one to switch to the main file before compiling;
attempting to compile the child file produces errors.
\item
The main file must be modified (each time)
to adjust the |\includeonly| command
to the present needs. This easily leaves the main file in a messy state.
\item
The generated document will always carry the filename
of the main document. This is inconvenient if
several child files are to be compiled and
to be kept for distribution.
\end{itemize}

The present package provides a simple interface
to make child files individually compilable by \LaTeX{}.
Compiling a child file then has the same effect as compiling
the main file with an |\includeonly| command
to select the appropriate child.
Moreover the generated document will carry the name of the child
rather than the main file.
This resolves all three above issues.

This feature is meant to make the editing of books,
thesis documents and lecture notes somewhat more convenient.
However, the package can also be used efficiently for
composing a series of documents (such as exercise sheets)
which are typically distributed individually.
It then assists the author in generating the individual documents
(potentially in different versions)
as well as a document containing the collected series.
Another application is in developing style files
or other kinds of included material
where compilation of the style file could redirect
to a sample or test file.

%%%%%%%%%%%%%%%%%%%%%%%%%%%%%%%%%%%%%%%%%%%%%%%%%%%%%%%%%%%%%%%%%%%%%%%%%%%%%%%%
%%%%%%%%%%%%%%%%%%%%%%%%%%%%%%%%%%%%%%%%%%%%%%%%%%%%%%%%%%%%%%%%%%%%%%%%%%%%%%%%
\section{Usage}

First of all, the package \textsf{childdoc} is \emph{not} a standard
\LaTeXe{} |.sty| style file! Therefore it needs to be invoked in
a non-standard way.

%%%%%%%%%%%%%%%%%%%%%%%%%%%%%%%%%%%%%%%%%%%%%%%%%%%%%%%%%%%%%%%%%%%%%%%%%%%%%%%%
\subsection{Included Files}
\label{sec:include}

%%%%%%%%%%%%%%%%%%%%%%%%%%%%%%%%%%%%%%%%
\DescribeMacro{\childdocmain}
To use the package, add the commands
\begin{center}
\begin{tabular}{l}
|\input{childdoc.def}|\\
|\childdocmain{}|\\
\end{tabular}
\end{center}
at the very top of the main \LaTeX{} file,
in particular \emph{before} the |\documentclass| statement!
The argument of |\childdocmain| should be left empty
(but it must be present).

%%%%%%%%%%%%%%%%%%%%%%%%%%%%%%%%%%%%%%%%
\DescribeMacro{\childdocof}
Furthermore, add the commands
\begin{center}
\begin{tabular}{l}
|\input{childdoc.def}|\\
|\childdocof{|\textit{main}|}|\\
\end{tabular}
\end{center}
at the top of every child file \textit{child}
which is included by |\include{|\textit{child}|}|
from within the main file
(or at least for those files to be compiled individually).
The argument \textit{main} must be the filename of the main file.

There are a couple of
considerations in setting up the main and child documents:

%%%%%%%%%%%%%%%%%%%%%%%%%%%%%%%%%%%%%%%%
\paragraph{Restrictions.}

Please note the following restrictions:
\begin{itemize}
\item
|\childdocmain| must be called with one argument \textit{main}
to ensure compatibility with earlier version of the package.
It must either be empty (|\childdocmain{}|)
or precisely match the filename of the main file in which it is specified.
See \secref{sec:detection} for further information.
\item
The filename \textit{main} must be specified without the |.tex| extension.
\item
The filename \textit{main} is case sensitive
(even in case-insensitive file systems)
due to internal string comparison.
\item
The argument \textit{main} should be fully expanded, it cannot be a macro.
\item
Subdirectories and special characters should be avoided in filenames.
\item
The command |\childdocmain{|\textit{main}|}| must be followed by a whitespace.
It should not be followed immediately by another command
or by a comment mark `|%|'.
This is because the \TeX{} parser reads the token immediately following
the argument of |\childdocmain| and puts it
at the beginning of every child section;
however, a white\-space is ignored.
\end{itemize}

%%%%%%%%%%%%%%%%%%%%%%%%%%%%%%%%%%%%%%%%
\paragraph{Content of Main File.}

It is advisable to place all content in the child files included by |\include|.
Any output contained in the main file will appear in all child documents
unless suppressed manually;
it cannot be suppressed automatically by the |\includeonly| directive
and thus should normally be avoided.
A method to include some content in the main file
by means of conditional processing is described in \secref{sec:conditional}.

%%%%%%%%%%%%%%%%%%%%%%%%%%%%%%%%%%%%%%%%
\paragraph{Page Numbering.}

When only a part of the document is compiled,
the appropriate numbering of pages
(as well as other status parameters)
is determined from the |.aux| files.
The latter contain information from previous passes.
However this information needs to propagate through
all intermediate child documents.
Therefore the page numbering in child documents may well
be inconsistent until the complete document is compiled at least once.

A useful (if unconventional) way to always ensure a consistent
page numbering is to restart the numbering in each child document
and denote the pages by `\textit{child}|.|\textit{page}'
where \textit{child} represents the chapter/section number of the child file.
This can be achieved by the command
|\numberwithin{page}{|\textit{child}|}|
of the \textsf{amsmath} package
where \textit{child} can be |chapter| or |section|
depending on the chosen structuring.
Alternatively, one can modify the macro |\thepage| appropriately
and reset the counter |page| at the start of each child file.

%%%%%%%%%%%%%%%%%%%%%%%%%%%%%%%%%%%%%%%%%%%%%%%%%%%%%%%%%%%%%%%%%%%%%%%%%%%%%%%%
\subsection{Conditional Processing}
\label{sec:conditional}

The package provides a mechanism to compile different versions
of a document. To customise the versions further some conditional processing
can come in handy to distinguish which version is being compiled.
The package provides two macros to describe the compilation context:

%%%%%%%%%%%%%%%%%%%%%%%%%%%%%%%%%%%%%%%%
\DescribeMacro{\ifchilddoc}
The conditional |\ifchilddoc| distinguishes between the compilation of
child documents and the main document:
%
\begin{center}
|\ifchilddoc |\textit{child-code}| |[|\||else |\textit{main-code}]| \||fi|
\end{center}

%%%%%%%%%%%%%%%%%%%%%%%%%%%%%%%%%%%%%%%%
\DescribeMacro{\childdocname}
\DescribeMacro{\childdocjob}
The macro |\childdocname| contains the filename (without extension)
of the main or child file being processed.
Note that |\childdocjob| will always contain the name of the main file.

%%%%%%%%%%%%%%%%%%%%%%%%%%%%%%%%%%%%%%%%
\paragraph{Title Page.}

Conditional processing can be used to include a title or banner page
in the main document when proper precautions are taken.
Importantly, the code in the main file should ensure that the page counter
(as well as other status parameters which are stored in the |.aux| files)
takes the same value after the conditional processing.
Otherwise the page numbers may take divergent values
depending on which part is compiled.

For example, a title page could be declared by:
%
\begin{center}
\begin{tabular}{l}
|\ifchilddoc\||else|\\
|\addtocounter{page}{-1}|\\
\textit{code for title page}\\
|\newpage|\\
|\||fi|
\end{tabular}
\end{center}
%
A banner page for the child documents can be generated by:
%
\begin{center}
\begin{tabular}{l}
|\ifchilddoc|\\
|\addtocounter{page}{-1}|\\
\textit{code for banner page}\\
|\newpage|\\
|\||fi|
\end{tabular}
\end{center}
%
Here one could write a message such as:
\begin{center}
|This is the part \childdocname{} of \childdocjob{}.|
\end{center}

%%%%%%%%%%%%%%%%%%%%%%%%%%%%%%%%%%%%%%%%%%%%%%%%%%%%%%%%%%%%%%%%%%%%%%%%%%%%%%%%
\subsection{Flags}
\label{sec:flags}

The package makes it easy to generate different versions
of the main or child documents.
To this end compilation flags can be defined
and assigned different default values.
They will be particularly useful in conjunction
with the forwarding mechanism described in \secref{sec:forward}.

For example, it may be useful to have a flag |\version|
which can be set to |draft| or |final|.
The document source will contain some conditional code
depending on the value of |\version|.
Suppose further, the flag should default to |final| for the main file
and to |draft| for child files
which is a natural assignment for editing the document.
This is achieved by placing the following code
in the preamble of the main document
(below the |\childdocmain| directive):
%
\begin{center}
\begin{tabular}{l}
|\ifchilddoc|\\
|\providecommand{\version}{draft}|\\
|\||else|\\
|\providecommand{\version}{final}|\\
|\||fi|
\end{tabular}
\end{center}
%
The definition by |\providecommand| makes sure
that previous definitions are not overwritten.
Further statements |\providecommand{\version}{...}|
can thus be added before the above code to override it.

For the main file, one might add a line
(between |\childdocmain| and the above block)
%
\begin{center}
|%\ifchilddoc\||else\providecommand{\version}{draft}\||fi|
\end{center}
%
which can be uncommented to produce a draft version.
Likewise one can add a line to the very top of a child file
(above the |\childdocof{|\textit{main}|}| directive)
%
\begin{center}
|%\providecommand{\version}{final}|
\end{center}
%
which can be uncommented to produce the final version of this child document.

%%%%%%%%%%%%%%%%%%%%%%%%%%%%%%%%%%%%%%%%%%%%%%%%%%%%%%%%%%%%%%%%%%%%%%%%%%%%%%%%
\subsection{Forwarding}
\label{sec:forward}

Different versions of the main or child documents
using compilation flags as described in \secref{sec:flags}
can be (permanently) stored in different files
for convenient compilation, viewing and distribution.
To this end, the package defines a command
to pass on compilation to a different file:

%%%%%%%%%%%%%%%%%%%%%%%%%%%%%%%%%%%%%%%%
\DescribeMacro{\childdocforward}
The command |\childdocforward| redirects processing to
another source file:
%
\begin{center}
\begin{tabular}{l}
|\input{childdoc.def}|\\
|\childdocforward[|\textit{main}|]{|\textit{dest}|}|\\
\end{tabular}
\end{center}
%
The argument \textit{dest} is the destination file
(without extension).
It should be the main file or one of the child files.
Note that further \textsf{childdoc} directives
such as |\childdocof| and |\childdocforward|
in the indicated file will be processed in this form.
The optional argument \textit{main}
passes on directly to the main file \textit{main}
while pretending to compile the child \textit{dest}.
This form behaves as if \textit{dest}
issues |\childdocof{|\textit{main}|}| right away,
and no further \textsf{childdoc} directives will be processed.

%%%%%%%%%%%%%%%%%%%%%%%%%%%%%%%%%%%%%%%%
\DescribeMacro{\...prefix}
In the alternative form |\childdocforwardprefix|,
%
\begin{center}
\begin{tabular}{l}
|\input{childdoc.def}|\\
|\childdocforwardprefix[|\textit{main}|]{|\textit{prefix}|}{|\textit{dest}|}|
\end{tabular}
\end{center}
%
the destination file is determined by a pattern
depending on the current file:
To make this work, the current file must be called
`{\textit{prefix}\hspace{0.2em}\textit{suffix}}'
with \textit{prefix} matching precisely the argument.
Processing is then passed on to the file
`{\textit{dest}\hspace{0.2em}\textit{suffix}}'.
Surely, the same effect is achieved by
directly specifying the
argument `{\textit{dest}\hspace{0.2em}\textit{suffix}}'
in the first form.
However, that requires to set up a different file
for each child. With the alternative form of the command
all these files can have exactly the same content
which simplifies setting them up and maintaining them.

For example, the following file |draft.tex|
with a compilation flag |\version| as described in \secref{sec:flags}
compiles the main document as a draft:
%
\begin{center}
\begin{tabular}{l}
|\def\version{draft}|\\
|\input{childdoc.def}|\\
|\childdocforward{|\textit{main}|}|
\end{tabular}
\end{center}
%
Likewise, the following files |final|\textit{nn}|.tex|
compile the final version of the child document
|child|\textit{nn}|.tex|:
%
\begin{center}
\begin{tabular}{l}
|\def\version{final}|\\
|\input{childdoc.def}|\\
|\childdocforwardprefix{final}{child}|
\end{tabular}
\end{center}
%

Note that when several versions of a main file and/or of each child file
are to be generated, it may be convenient to set up a |Makefile| or
shell script to automatise the process.

%%%%%%%%%%%%%%%%%%%%%%%%%%%%%%%%%%%%%%%%%%%%%%%%%%%%%%%%%%%%%%%%%%%%%%%%%%%%%%%%
\subsection{Command Line Processing}
\label{sec:commandline}

The effect of redirection files can also be achieved by invoking
the \LaTeX{} compiler with a more elaborate command line.
Most conveniently this should be done as part
of a shell script or a |Makefile|.

When using \textsf{childdoc} in the main file, the following
command lines effectively perform a redirection
(note that depending on the shell being used,
backslashes may have to be doubled: `|\|' $\to$ `|\\|'):
%
\begin{center}
|... -jobname "|\textit{target}|" |\\|"|[\textit{flags}]%
|\input{childdoc.def}\childdocforward[|\textit{main}|]{|\textit{dest}|}"|
\end{center}
%
Here \textit{target} is the name of the output file,
\textit{main} is the name of the main file
and \textit{dest} is the name of the main or child file to be processed
(all filenames without extensions).
The optional argument \textit{main} can be omitted
if \textit{main} matches \textit{dest}.
Optionally, compilation \textit{flags} can be defined via |\def| commands.
This command line makes the \TeX{} engine believe
it is compiling the file \textit{target}
whose content is specified as the latter parameter.
The provided code then forwards the processing to
\textit{main} or \textit{dest} as described in \secref{sec:forward}.

%%%%%%%%%%%%%%%%%%%%%%%%%%%%%%%%%%%%%%%%%%%%%%%%%%%%%%%%%%%%%%%%%%%%%%%%%%%%%%%%
\subsection{Include by Input}
\label{sec:input}

Including child documents by |\include| has some restrictions by design.
Most notably, the content of a child document always occupies
its own set of pages; pages cannot be shared between child documents.
Usually, this behaviour makes perfect sense
because each child document contain an essential part of the document.
However, in some situations it may be desirable to compose
a document from a collection of parts
without having mandatory page breaks between then.
For this case, the package
provides a mechanism to include parts
by |\input| which can also be processed individually.
However, by construction this mechanism
requires manual handling of the content to be output.

%%%%%%%%%%%%%%%%%%%%%%%%%%%%%%%%%%%%%%%%
\DescribeMacro{\ifchilddocmanual}
The main file should be prepared as usual, see \secref{sec:include}.
However, the document body must make a distinction
between processing of an individual part and of the main document, e.g.:
%
\begin{center}
\begin{tabular}{l}
|\ifchilddocmanual|\\
|\input{\childdocname}|\\
|\||else|\\
\textit{document body with }|\input{|\textit{part}|}|\\
|\||fi|
\end{tabular}
\end{center}
%
The conditional |\ifchilddocmanual| is true whenever
a part to be included by |\input| is being compiled,
and the name of the part is stored in |\childdocname|.

%%%%%%%%%%%%%%%%%%%%%%%%%%%%%%%%%%%%%%%%
\DescribeMacro{\childdocby}
Each part to be included by |\input| should start with:
%
\begin{center}
\begin{tabular}{l}
|\input{childdoc.def}|\\
|\childdocby{|\textit{main}|}|\\
\end{tabular}
\end{center}
%
The directive |\childdocby| is similar to |\childdocof|
described in \secref{sec:include},
but the subsequent selection of content must be done manually.
To that end, both |\ifchilddoc| and |\ifchilddocmanual|
will be true upon processing of a part,
and the name of the part is stored in |\childdocname|.
Note that |\jobname| will be set to the filename of the current part
so that each part receives an individual |.aux| file
that does not interfere with the |.aux| file(s) of the main document.
This behaviour can be altered by the alternative form
|\childdocby[*]{|\textit{main}|}| (with a non-empty optional argument)
which uses the |.aux| file of the main document
by setting |\jobname| to \textit{main}.

%%%%%%%%%%%%%%%%%%%%%%%%%%%%%%%%%%%%%%%%%%%%%%%%%%%%%%%%%%%%%%%%%%%%%%%%%%%%%%%%
\subsection{Driver Development}
\label{sec:driver}

The \textsf{childdoc} mechanism can also be use for the development
of definition files such as \LaTeX{} styles or classes.
This case differs from the above setup with multiple parts
included by |\include| in that no |\includeonly| should be invoked.
This can be achieved by starting the include file
(before |\ProvidesPackage|) with:
%
\begin{center}
\begin{tabular}{l}
|\input{childdoc.def}|\\
|\childdocforward{|\textit{main}|}|\\
\end{tabular}
\end{center}
%
or alternatively with:
%
\begin{center}
\begin{tabular}{l}
|\input{childdoc.def}|\\
|\childdocby{|\textit{main}|}|\\
\end{tabular}
\end{center}
%
Both forms have slightly different effects as described above.
The main file is prepared as usual, see \secref{sec:include}.

%%%%%%%%%%%%%%%%%%%%%%%%%%%%%%%%%%%%%%%%%%%%%%%%%%%%%%%%%%%%%%%%%%%%%%%%%%%%%%%%
\subsection{Legacy Detection}
\label{sec:detection}

The directive |\childdocmain| in the main file can detect
whether the complete document or merely a child is to be compiled
even without using the directive |\childdocof|.
This method is deprecated because it is less robust
and there is no compelling reason to use it;
it is merely provided for backward compatibility
and it may be removed in future versions.

If the detection mechanism is to be used,
it is mandatory to correctly specify
the filename of the main file as the argument of |\childdocmain|:
%
\begin{center}
\begin{tabular}{l}
|\input{childdoc.def}|\\
|\childdocmain{|\textit{main}|}|\\
\end{tabular}
\end{center}
%
If |\jobname| does not match the argument \textit{main} of |\childdocmain|,
it is assumed that |\jobname| points to the child file to be compiled.
When using |\childdocmain| with the main file specified as argument,
it suffices to start a child file
with just |\input{|\textit{main}|}|
without loading of the package and using |\childdocof|.
If instead all processing is done
with the appropriate \textsf{childdoc} directives,
the argument of \textit{main} of |\childdocmain| can be empty.

An alternative version of the command line processing described
in \secref{sec:commandline} using the detection mechanism reads:
%
\begin{center}
|... -jobname "|\textit{target}|" "|[\textit{flags}]%
[|\def\jobname{|\textit{dest}|}|]|\input{|\textit{main}|}"|
\end{center}

%%%%%%%%%%%%%%%%%%%%%%%%%%%%%%%%%%%%%%%%%%%%%%%%%%%%%%%%%%%%%%%%%%%%%%%%%%%%%%%%
\subsection{Manual Code}
\label{sec:manual}

In case one cannot be certain whether the definitions file |childdoc.def|
is installed on the target \TeX{} distribution
and one prefers not to ship it,
it is conceivable to paste a few relevant commands into the sources.

To that end, drop all statements |\input{childdoc.def}|
and perform the replacements as outlined below.
Instead of |\childdocmain{|\textit{main}|}| add the following code
to the top of the main file:
%
\begin{center}
\begin{tabular}{l}
|\||ifdefined\childdocname\endinput\||fi\newif\ifchilddoc|\\
|\edef\childdocname{\scantokens\expandafter{\jobname\noexpand}}|\\
|\def\childdocmain{|\textit{main}|}\||ifx\childdocmain\childdocname\||else|\\
|\childdoctrue\includeonly{\childdocname}\let\jobname\childdocmain\||fi|\\
\end{tabular}
\end{center}
%
Instead of |\childdocof{|\textit{main}|}| just include the main file
at the top of each child file:
%
\begin{center}
|\input{|\textit{main}|}|
\end{center}
%
A simple redirection |\childdocforward{|\textit{dest}|}| is achieved by:
%
\begin{center}
|\def\jobname{|\textit{dest}|}\input{\jobname}|
\end{center}
%
The redirection with prefix
|\childdocforwardprefix[|\textit{prefix}|]{|\textit{dest}|}|
is accomplished by:
%
\begin{center}
\begin{tabular}{l}
|{\edef\jobname{\scantokens\expandafter{\jobname\noexpand}}|\\
|\def\redirectjob |\textit{prefix}|#1~~~{\gdef\jobname{|\textit{dest}|#1}}|\\
|\expandafter\redirectjob\jobname~~~}\input{\jobname}|
\end{tabular}
\end{center}

In an alternative approach,
child documents can be compiled by a specific command line
without additional code or specific definitions:
%
\begin{center}
|... -jobname "|\textit{target}|" "|[\textit{flags}]%
|\includeonly{|\textit{dest}|}\input{|\textit{main}|}"|
\end{center}
%

%%%%%%%%%%%%%%%%%%%%%%%%%%%%%%%%%%%%%%%%%%%%%%%%%%%%%%%%%%%%%%%%%%%%%%%%%%%%%%%%
%%%%%%%%%%%%%%%%%%%%%%%%%%%%%%%%%%%%%%%%%%%%%%%%%%%%%%%%%%%%%%%%%%%%%%%%%%%%%%%%
\section{Information}

%%%%%%%%%%%%%%%%%%%%%%%%%%%%%%%%%%%%%%%%%%%%%%%%%%%%%%%%%%%%%%%%%%%%%%%%%%%%%%%%
\subsection{Copyright}

Copyright \copyright{} 2017--2018 Niklas Beisert

This work may be distributed and/or modified under the
conditions of the \LaTeX{} Project Public License, either version 1.3
of this license or (at your option) any later version.
The latest version of this license is in
  \url{http://www.latex-project.org/lppl.txt}
and version 1.3 or later is part of all distributions of \LaTeX{}
version 2005/12/01 or later.

This work has the LPPL maintenance status `maintained'.

The Current Maintainer of this work is Niklas Beisert.

This work consists of the files |README.txt|, |childdoc.ins| and |childdoc.dtx|
as well as the derived files |childdoc.def|, |cdocsamp.tex|
with |cdocsch1.tex|, |cdocsch2.tex|, |cdocspt3.tex|, |cdocspt4.tex|,
|cdocsdrf.tex|, |cdocsfn1.tex|, |cdocsfn2.tex|
as well as |childdoc.pdf|.

%%%%%%%%%%%%%%%%%%%%%%%%%%%%%%%%%%%%%%%%%%%%%%%%%%%%%%%%%%%%%%%%%%%%%%%%%%%%%%%%
\subsection{Files and Installation}

The package consists of the files:
%
\begin{center}
\begin{tabular}{ll}
    |README.txt|   & readme file \\
    |childdoc.ins| & installation file \\
    |childdoc.dtx| & source file \\
    |childdoc.def| & definition file \\
    |cdocsamp.tex| & sample main file \\
    |cdocsch1.tex| & sample include file \\
    |cdocsch2.tex| & sample include file \\
    |cdocspt3.tex| & sample part file \\
    |cdocspt4.tex| & sample part file \\
    |cdocsdrf.tex| & sample redirection file \\
    |cdocsfn1.tex| & sample redirection file \\
    |cdocsfn2.tex| & sample redirection file \\
    |childdoc.pdf| & manual
\end{tabular}
\end{center}
%
The distribution consists of the files
|README.txt|, |childdoc.ins| and |childdoc.dtx|.
%
\begin{itemize}
\item
Run (pdf)\LaTeX{} on |childdoc.dtx|
to compile the manual |childdoc.pdf| (this file).
\item
Run \LaTeX{} on |childdoc.ins| to create the definitions file |childdoc.def|
and the sample |cdocsamp.tex| with include files
|cdocsch1.tex|, |cdocsch2.tex|, |cdocspt3.tex|, |cdocspt4.tex|,
|cdocsdrf.tex|, |cdocsfn1.tex|, |cdocsfn2.tex|.
Then copy the file |childdoc.def| to an appropriate directory of your \LaTeX{}
distribution, e.g.\ \textit{texmf-root}|/tex/latex/childdoc|.
\end{itemize}

%%%%%%%%%%%%%%%%%%%%%%%%%%%%%%%%%%%%%%%%%%%%%%%%%%%%%%%%%%%%%%%%%%%%%%%%%%%%%%%%
\subsection{Related CTAN Packages}

There are several other packages which offer a similar functionality:
%
\begin{itemize}
\item
The packages
\href{http://ctan.org/pkg/docmute}{\textsf{docmute}},
\href{http://ctan.org/pkg/includex}{\textsf{includex}} and
\href{http://ctan.org/pkg/standalone}{\textsf{standalone}}
provide commands to include only the document body of
a child file thus allowing both files to be compiled individually.
\item
The packages \href{http://ctan.org/pkg/subdocs}{\textsf{subdocs}}
and \href{http://ctan.org/pkg/subfiles}{\textsf{subfiles}}
provide structures in which the main and child documents can be
encapsulated and allowing them to be compiled individually.
The inclusion mechanism is different from the conventional |\include|.
\item
The package \href{http://ctan.org/pkg/combine}{\textsf{combine}}
is an elaborate solution to combine several documents into one.
\end{itemize}
%
See also the CTAN topic \href{http://ctan.org/topic/subdocs}{\textsf{subdocs}}
for further related packages.
The present package differs from the above solutions in that
a document structure constructed with the conventional |\include| mechanism
just needs two extra commands at the top of every file
such that all constituent files can be compiled individually.

%%%%%%%%%%%%%%%%%%%%%%%%%%%%%%%%%%%%%%%%%%%%%%%%%%%%%%%%%%%%%%%%%%%%%%%%%%%%%%%%
%\subsection{Feature Suggestions}
%
%The following is a list of features which may be useful for future
%versions of this package:
%%
%\begin{itemize}
%\item
%\ldots
%\end{itemize}

%%%%%%%%%%%%%%%%%%%%%%%%%%%%%%%%%%%%%%%%%%%%%%%%%%%%%%%%%%%%%%%%%%%%%%%%%%%%%%%%
\subsection{Revision History}

%%%%%%%%%%%%%%%%%%%%%%%%%%%%%%%%%%%%%%%%
\paragraph{v2.0:} 2018/12/30

\begin{itemize}
\item
immediate forward processing
\item
added |\childdocby| mechanism
\item
manual restructured
\end{itemize}

%%%%%%%%%%%%%%%%%%%%%%%%%%%%%%%%%%%%%%%%
\paragraph{v1.6:} 2018/01/17

\begin{itemize}
\item
application for development of include files
\item
corrections to manual
\end{itemize}

%%%%%%%%%%%%%%%%%%%%%%%%%%%%%%%%%%%%%%%%
\paragraph{v1.5:} 2017/05/21

\begin{itemize}
\item
more complete structuring introduced
\item
|\childdocof| introduced
\item
|\childdoc| renamed to |\childdocmain|
\item
|\childredirect| renamed to |\childdocforward| and |\childdocforwardprefix|
and functionality expanded
\end{itemize}

%%%%%%%%%%%%%%%%%%%%%%%%%%%%%%%%%%%%%%%%
\paragraph{v1.0:} 2017/04/27

\begin{itemize}
\item
manual and install package
\item
first version published on CTAN
\end{itemize}

%%%%%%%%%%%%%%%%%%%%%%%%%%%%%%%%%%%%%%%%
\paragraph{v0.6:} 2017/04/26

\begin{itemize}
\item
redirection mechanism added
\end{itemize}

%%%%%%%%%%%%%%%%%%%%%%%%%%%%%%%%%%%%%%%%
\paragraph{v0.5:} 2017/04/26

\begin{itemize}
\item
functionality in definition file
\end{itemize}


%%%%%%%%%%%%%%%%%%%%%%%%%%%%%%%%%%%%%%%%%%%%%%%%%%%%%%%%%%%%%%%%%%%%%%%%%%%%%%%%
%%%%%%%%%%%%%%%%%%%%%%%%%%%%%%%%%%%%%%%%%%%%%%%%%%%%%%%%%%%%%%%%%%%%%%%%%%%%%%%%
%%%%%%%%%%%%%%%%%%%%%%%%%%%%%%%%%%%%%%%%%%%%%%%%%%%%%%%%%%%%%%%%%%%%%%%%%%%%%%%%
\appendix

\settowidth\MacroIndent{\rmfamily\scriptsize 000\ }

 \DocInput{childdoc.dtx}

\end{document}
%</driver>
% \fi
%
% %%%%%%%%%%%%%%%%%%%%%%%%%%%%%%%%%%%%%%%%%%%%%%%%%%%%%%%%%%%%%%%%%%%%%%%%%%%%%%
% %%%%%%%%%%%%%%%%%%%%%%%%%%%%%%%%%%%%%%%%%%%%%%%%%%%%%%%%%%%%%%%%%%%%%%%%%%%%%%
% \section{Sample}
%\iffalse
%<*samplemain>
%\fi
%
% The following presents a sample document
% with two chapters, two parts, a title page,
% a compile flag as well as three forwarding files to set the flag.
% It consists of eight |.tex| files:
% \begin{center}
% \begin{tabular}{ll}
% |cdocsamp.tex|&main file\\
% |cdocsch1.tex|&include file for chapter 1\\
% |cdocsch2.tex|&include file for chapter 2\\
% |cdocspt3.tex|&include file for part 3\\
% |cdocspt4.tex|&include file for part 4\\
% |cdocsdrf.tex|&forwarding file for main file in draft mode\\
% |cdocsfi1.tex|&forwarding file for final version of chapter 1\\
% |cdocsfi2.tex|&forwarding file for final version of chapter 2\\
% \end{tabular}
% \end{center}
% Each of the eight files can be compiled directly by the \LaTeX{} compiler.
%
% %%%%%%%%%%%%%%%%%%%%%%%%%%%%%%%%%%%%%%
% \paragraph{Main File.}
%
% The main file is called |cdocsamp.tex|.
%
% Load the \textsf{childdoc} definitions and
% declare the filename for the main document:
%    \begin{macrocode}
\input{childdoc.def}
\childdocmain{}
%    \end{macrocode}

% Optional override for |\version| flag:
%    \begin{macrocode}
%%\ifchilddoc\else\providecommand{\version}{draft}\fi
%    \end{macrocode}

% Define the default values for the |\version| flag
% (|final| for the main file and |draft| for childs):
%    \begin{macrocode}
\ifchilddoc
\providecommand{\version}{draft}
\else
\providecommand{\version}{final}
\fi
%    \end{macrocode}

% Load the standard document class:
%    \begin{macrocode}
\documentclass[12pt]{article}
%    \end{macrocode}

% Start the document body:
%    \begin{macrocode}
\begin{document}
%    \end{macrocode}

% Declare a title page.
% Print title, part of document being processed and version flag:
%    \begin{macrocode}
\addtocounter{page}{-1}
\begin{center}
{\LARGE\bfseries{}childdoc example\par}
\vspace{1cm}
\ifchilddoc
\ifchilddocmanual part\else chapter\fi:
`\childdocname' of `\childdocjob'\par
\else
main document: `\childdocjob'\par
\fi
version: \version\par
\end{center}
\newpage
%    \end{macrocode}

% Manually include selected file,
% otherwise process as usual:
%    \begin{macrocode}
\ifchilddocmanual
\section*{part `\childdocname'}
\input{\childdocname}
\else
%    \end{macrocode}

% Include the two chapters:
%    \begin{macrocode}
\include{cdocsch1}
\include{cdocsch2}
%    \end{macrocode}

% Include the two parts unless only chapters should be displayed:
%    \begin{macrocode}
\ifchilddoc\else
\section{part three}
\input{cdocspt3}
\section{part four}
\input{cdocspt4}
\fi
%    \end{macrocode}

% Process as usual until here:
%    \begin{macrocode}
\fi
%    \end{macrocode}

% End of document body:
%    \begin{macrocode}
\end{document}
%    \end{macrocode}
%\iffalse
%</samplemain>
%\fi
%
% %%%%%%%%%%%%%%%%%%%%%%%%%%%%%%%%%%%%%%
% \paragraph{Chapter Include Files.}
%
% The include files are called |cdocsch1.tex| and |cdocsch2.tex|.
%
%\iffalse
%<*samplechap1|samplechap2>
%\fi

% Optional override for |\version| flag:
%    \begin{macrocode}
%%\providecommand{\version}{final}
%    \end{macrocode}

% Include the main document:
%    \begin{macrocode}
\input{childdoc.def}
\childdocof{cdocsamp}
%    \end{macrocode}

%\iffalse
%</samplechap1|samplechap2>
%\fi
%
%\iffalse
%<*samplechap1>
%\fi
% Some text for chapter 1:
%    \begin{macrocode}
\section{one}
some text in chapter one
%    \end{macrocode}

%\iffalse
%</samplechap1>
%\fi
% Some text for chapter 2:
%\iffalse
%<*samplechap2>
%\fi
%    \begin{macrocode}
\section{two}
more text in chapter two
%    \end{macrocode}

%\iffalse
%</samplechap2>
%\fi
%
% %%%%%%%%%%%%%%%%%%%%%%%%%%%%%%%%%%%%%%
% \paragraph{Part Include Files.}
%
% The include files are called |cdocspt3.tex| and |cdocspt4.tex|.
%
%\iffalse
%<*samplepart3|samplepart4>
%\fi

% Optional override for |\version| flag:
%    \begin{macrocode}
%%\providecommand{\version}{final}
%    \end{macrocode}

% Include the main document:
%    \begin{macrocode}
\input{childdoc.def}
\childdocby{cdocsamp}
%    \end{macrocode}

%\iffalse
%</samplepart3|samplepart4>
%\fi
%
%\iffalse
%<*samplepart3>
%\fi
% Some text for part 3:
%    \begin{macrocode}
some text in part three
%    \end{macrocode}

%\iffalse
%</samplepart3>
%\fi
% Some text for part 4:
%\iffalse
%<*samplepart4>
%\fi
%    \begin{macrocode}
more text in part four
%    \end{macrocode}

%\iffalse
%</samplepart4>
%\fi
%
% %%%%%%%%%%%%%%%%%%%%%%%%%%%%%%%%%%%%%%
% \paragraph{Forwarding for a Complete Draft.}
%
% The following forwarding file |cdocsdrf.tex|
% compiles the main document in draft mode:
%\iffalse
%<*sampledraft>
%\fi
%    \begin{macrocode}
\def\version{draft}
\input{childdoc.def}
\childdocforward{cdocsamp}
%    \end{macrocode}

%\iffalse
%</sampledraft>
%\fi
%
% %%%%%%%%%%%%%%%%%%%%%%%%%%%%%%%%%%%%%%
% \paragraph{Forwarding for Final Version of the Chapters.}
%
% The following forwarding files |cdocsfn1.tex| and |cdocsfn2.tex|
% (with identical content)
% compile the final versions of the child documents
% |cdocsch1.tex| and |cdocsch2.tex|, respectively:
%\iffalse
%<*samplefinal>
%\fi
%    \begin{macrocode}
\def\version{final}
\input{childdoc.def}
\childdocforwardprefix[cdocsamp]{cdocsfn}{cdocsch}
%    \end{macrocode}

%\iffalse
%</samplefinal>
%\fi
%
% %%%%%%%%%%%%%%%%%%%%%%%%%%%%%%%%%%%%%%
% \paragraph{Command Line Processing.}
%
% The following three command lines generate the output files
% |cdocscld|, |cdocscl1| and |cdocscl2|
% which should be identical to
% |cdocsdrf|, |cdocsch1| and |cdocsfn2|, respectively:
% \begin{center}
% \begin{tabular}{l}
% |latex -jobname cdocscld \|\\
% |  "\def\version{draft}\input{childdoc.def}\childdocforward{cdocsamp}"|\\
% |latex -jobname cdocscl1 \|\\
% |  "\input{childdoc.def}\childdocforward[cdocsamp]{cdocsch1}"|\\
% |latex -jobname cdocscl2 \|\\
% |  "\def\version{final}\input{childdoc.def}\childdocforward{cdocsch2}"|
% \end{tabular}
% \end{center}
% Note that the trailing backslash on each first line
% merely continues the input to the second line
% (for convenient cut ant paste).
% Furthermore, the command |latex| can be replaced by any
% of its alternative versions such as |pdflatex|.
%
% %%%%%%%%%%%%%%%%%%%%%%%%%%%%%%%%%%%%%%%%%%%%%%%%%%%%%%%%%%%%%%%%%%%%%%%%%%%%%%
% %%%%%%%%%%%%%%%%%%%%%%%%%%%%%%%%%%%%%%%%%%%%%%%%%%%%%%%%%%%%%%%%%%%%%%%%%%%%%%
% \section{Implementation}
%\iffalse
%<*package>
%\fi
%
% This section describes the definitions file |childdoc.def|.

% The definitions cannot be loaded using |\usepackage| or |\RequirePackage|
% which has a mechanism to prevent loading a style file more than once.
% When loading the definitions by means of |\input|
% multiple instances have to be prevented manually:
%\iffalse
%This code needs to be before the `\ProvidesFile' directive
%which is defined at the beginning of this file.
%Therefore it is also placed there and commented out here.
%</package>
%<*discard>
%\fi
%    \begin{macrocode}
\ifdefined\childdocmain\endinput\fi
%    \end{macrocode}
%\iffalse
%</discard>
%<*package>
%\fi
%
% \macro{\ifchilddoc}
% \macro{\ifchilddocmanual}
% The conditional |\ifchilddoc| tells whether a
% child (true) or main (false) document is being compiled.
% The conditional |\ifchilddocmanual| tells whether
% the |\includeonly| mechanism is used (false) or
% the selection of child files must be performed manually (true).
% The definitions initialise to false:
%    \begin{macrocode}
\newif\ifchilddoc
\newif\ifchilddocmanual
%    \end{macrocode}

% \macro{\childdocname}
% \macro{\childdocjob}
% The macro |\childdocname| stores the name of the main document
% to be compiled. The macro |\childdocjob| stores the name of
% the document on which the \LaTeX{} compiler was originally invoked.
% The content of |\jobname| cannot be compared
% to filenames specified in the source due to different catcodes.
% The following code rescans |\jobname|, stores the result
% in |\childdocname| and saves a copy in |\childdocjob|:
%    \begin{macrocode}
\edef\childdocname{\scantokens\expandafter{\jobname\noexpand}}
\let\childdocjob\childdocname
%    \end{macrocode}

% \macro{\childdocdisable}
% The macro |\childdocdisable| prevents the main file
% from being processed more than once.
% At this stage, the main document command |\childdocmain|
% is assumed to be called once again where it should do nothing.
% Any subsequent call to it should prevent
% a secondary processing of the main document
% It overwrites the forwarding commands
% |\childdocof| and |\childdocforward|
% with empty macros to prevent further inclusions of the main document:
%    \begin{macrocode}
\newcommand{\childdocdisable}
{
  \renewcommand{\childdocmain}[1]{\renewcommand{\childdocmain}[1]{\endinput}}
  \renewcommand{\childdocof}[1]{}
  \renewcommand{\childdocby}[2][]{}
  \renewcommand{\childdocforward}[2][]{}
  \renewcommand{\childdocdisable}{}
}
%    \end{macrocode}

% \macro{\childdocmain}
% The macro |\childdocmain| is to be called at the top of the main file
% with nothing or the main filename (without extension) as argument.
% First, it breaks loops.
% If the argument is not empty and does not match |\childdocname|
% (which is set by the first inclusion of |childdoc.def|),
% |\ifchilddoc| is set to true, |\includeonly| is applied to the child file
% and |\jobname| is set to the main file
% (for proper handling of |.aux| files):
%    \begin{macrocode}
\newcommand{\childdocmain}[1]
{
  \childdocdisable\childdocmain{}
  \if?#1?\else
    \begingroup
      \def\childdoctmp{#1}
      \ifx\childdoctmp\childdocname
        \def\childdoctmp{}
      \else
        \def\childdoctmp
        {
          \childdoctrue
          \includeonly{\childdocname}
          \def\childdocjob{#1}
          \def\jobname{#1}
        }
      \fi
      \expandafter
    \endgroup
    \childdoctmp
  \fi
}
%    \end{macrocode}

% \macro{\childdocof}
% The command |\childdocof| redirects
% compilation to the main file |#1|.
%    \begin{macrocode}
\newcommand{\childdocof}[1]
{
  \childdocdisable
  \childdoctrue
  \includeonly{\childdocname}
  \def\jobname{#1}
  \def\childdocjob{#1}
  \input{#1}
}
%    \end{macrocode}

% \macro{\childdocby}
% The command |\childdocby| ....
%    \begin{macrocode}
\newcommand{\childdocby}[2][]
{
  \childdocdisable
  \childdoctrue
  \childdocmanualtrue
  \if?#1?\else
    \def\jobname{#2}
  \fi
  \def\childdocjob{#2}
  \input{#2}
  \endinput
}
%    \end{macrocode}

% \macro{\childdocforward}
% The command |\childdocforward| redirects
% compilation to the main file or
% (if the optional argument is given) a child file.
% Parameters are set as if the main file
% or a child file starting with |\childdocof| was compiled.
% Then compilation is handed over to the main file:
%    \begin{macrocode}
\newcommand{\childdocforward}[2][]
{
  \begingroup
    \if?#1?
      \def\childdoctmp
      {
        \def\childdocname{#2}
        \def\childdocjob{#2}
        \def\jobname{#2}
        \input{#2}
        \endinput
      }
    \else
      \def\childdoctmp
      {
        \childdocdisable
        \def\childdocname{#2}
        \childdoctrue
        \includeonly{#2}
        \def\childdocjob{#1}
        \def\jobname{#1}
        \input{#1}
        \endinput
      }
    \fi
    \expandafter
  \endgroup
  \childdoctmp
}
%    \end{macrocode}

% \macro{\childdocforwardprefix}
% The command |\childdocforwardprefix| redirects
% compilation to the main or a child file by means of a pattern.
% The prefix |#1| in the current filename is replaced by |#2|
% and the suffix of the current filename is kept
% (it is assumed that the filename does not contain the substring `|~~~|'
% which is used as a delimiter).
% Compilation is handed over to the new file by |\childdocforward|:
%    \begin{macrocode}
\newcommand{\childdocforwardprefix}[3][]
{
  \begingroup
    \def\childdocextract #2##1~~~{\def\childdoctmp{\childdocforward[#1]{#3##1}}}
    \expandafter\childdocextract\childdocname~~~
    \expandafter
  \endgroup
  \childdoctmp
}
%    \end{macrocode}

% \macro{\childdoc}
% The deprecated macro |\childdoc| is a legacy version of |\childdocmain|:
%    \begin{macrocode}
\newcommand{\childdoc}{\childdocmain}
%    \end{macrocode}

% \macro{\childdocredirect}
% The deprecated macro |\childdocredirect| is a legacy version
% of |\childdocforward| and |\childdocforwardprefix|:
%    \begin{macrocode}
\newcommand{\childdocredirect}[2][]
{
  \begingroup
    \if?#1?
      \def\childdoctmp{\childdocforward{#2}}
    \else
      \def\childdoctmp{\childdocforwardprefix{#1}{#2}}
    \fi
    \expandafter
  \endgroup
  \childdoctmp
}
%    \end{macrocode}

%\iffalse
%</package>
%\fi
%
\endinput
|\\
|\childdocby{|\textit{main}|}|\\
\end{tabular}
\end{center}
%
The directive |\childdocby| is similar to |\childdocof|
described in \secref{sec:include},
but the subsequent selection of content must be done manually.
To that end, both |\ifchilddoc| and |\ifchilddocmanual|
will be true upon processing of a part,
and the name of the part is stored in |\childdocname|.
Note that |\jobname| will be set to the filename of the current part
so that each part receives an individual |.aux| file
that does not interfere with the |.aux| file(s) of the main document.
This behaviour can be altered by the alternative form
|\childdocby[*]{|\textit{main}|}| (with a non-empty optional argument)
which uses the |.aux| file of the main document
by setting |\jobname| to \textit{main}.

%%%%%%%%%%%%%%%%%%%%%%%%%%%%%%%%%%%%%%%%%%%%%%%%%%%%%%%%%%%%%%%%%%%%%%%%%%%%%%%%
\subsection{Driver Development}
\label{sec:driver}

The \textsf{childdoc} mechanism can also be use for the development
of definition files such as \LaTeX{} styles or classes.
This case differs from the above setup with multiple parts
included by |\include| in that no |\includeonly| should be invoked.
This can be achieved by starting the include file
(before |\ProvidesPackage|) with:
%
\begin{center}
\begin{tabular}{l}
|% \iffalse
%
% childdoc.dtx Copyright (C) 2017-2018 Niklas Beisert
%
% This work may be distributed and/or modified under the
% conditions of the LaTeX Project Public License, either version 1.3
% of this license or (at your option) any later version.
% The latest version of this license is in
%   http://www.latex-project.org/lppl.txt
% and version 1.3 or later is part of all distributions of LaTeX
% version 2005/12/01 or later.
%
% This work has the LPPL maintenance status `maintained'.
%
% The Current Maintainer of this work is Niklas Beisert.
%
% This work consists of the files childdoc.dtx and childdoc.ins
% and the derived files childdoc.def and cdocsamp.tex with
% cdocsch1.tex, cdocsch2.tex, cdocsdrf.tex, cdocsfn1.tex, cdocsfn2.tex.
%
%<package>\ifdefined\childdocmain\endinput\fi
%<package>\ProvidesFile{childdoc.def}[2018/12/30 v2.0 child document driver]
%<samplemain>\ProvidesFile{cdocsamp.tex}[2018/12/30 v2.0 sample for childdoc]
%<*driver>
%\ProvidesFile{childdoc.drv}[2018/12/30 v2.0 childdoc reference manual file]
\PassOptionsToClass{10pt,a4paper}{article}
\documentclass{ltxdoc}

\usepackage[margin=35mm]{geometry}
\usepackage{hyperref}
\usepackage{hyperxmp}
\usepackage[usenames]{color}

\hypersetup{colorlinks=true}
\hypersetup{pdfstartview=FitH}
\hypersetup{pdfpagemode=UseNone}
\hypersetup{pdfsource={}}
\hypersetup{pdflang={en-UK}}
\hypersetup{pdfcopyright={Copyright 2017-2018 Niklas Beisert.
  This work may be distributed and/or modified under the
  conditions of the LaTeX Project Public License, either version 1.3
  of this license or (at your option) any later version.}}
\hypersetup{pdflicenseurl={http://www.latex-project.org/lppl.txt}}
\hypersetup{pdfcontactaddress={ETH Zurich, ITP, HIT K,
  Wolfgang-Pauli-Strasse 27}}
\hypersetup{pdfcontactpostcode={8093}}
\hypersetup{pdfcontactcity={Zurich}}
\hypersetup{pdfcontactcountry={Switzerland}}
\hypersetup{pdfcontactemail={nbeisert@itp.phys.ethz.ch}}
\hypersetup{pdfcontacturl={http://people.phys.ethz.ch/\xmptilde nbeisert/}}

\newcommand{\secref}[1]{\hyperref[#1]{section \ref*{#1}}}

\parskip1ex
\parindent0pt
\let\olditemize\itemize
\def\itemize{\olditemize\parskip0pt}

\begin{document}

\title{The \textsf{childdoc} Package}
\hypersetup{pdftitle={The childdoc Package}}
\author{Niklas Beisert\\[2ex]
  Institut f\"ur Theoretische Physik\\
  Eidgen\"ossische Technische Hochschule Z\"urich\\
  Wolfgang-Pauli-Strasse 27, 8093 Z\"urich, Switzerland\\[1ex]
  \href{mailto:nbeisert@itp.phys.ethz.ch}
  {\texttt{nbeisert@itp.phys.ethz.ch}}}
\hypersetup{pdfauthor={Niklas Beisert}}
\hypersetup{pdfsubject={Manual for the LaTeX2e Package childdoc}}
\date{30 December 2018, \textsf{v2.0}}
\maketitle

\begin{abstract}\noindent
\textsf{childdoc} is a \LaTeXe{} package
that enables the direct compilation
of document sections included by |\include|
to individual files.
\end{abstract}

\begingroup
\parskip0ex
\tableofcontents
\endgroup

%%%%%%%%%%%%%%%%%%%%%%%%%%%%%%%%%%%%%%%%%%%%%%%%%%%%%%%%%%%%%%%%%%%%%%%%%%%%%%%%
%%%%%%%%%%%%%%%%%%%%%%%%%%%%%%%%%%%%%%%%%%%%%%%%%%%%%%%%%%%%%%%%%%%%%%%%%%%%%%%%
\section{Introduction}

\LaTeX{} provides a mechanism to structure a large document (such as a book)
into a main file and several child files (containing the chapters)
using the |\include| command.
This mechanism is beneficial for documents
which span hundreds of pages in order to
make the source file(s) more manageable.
Moreover, compilation can be restricted to
selected child files by means of the |\includeonly| command.
The latter feature can be used to reduce the compilation time while editing
(this was significantly more useful in the earlier days of \LaTeX{})
or to generate a smaller document which is easier to navigate.
Another application of |\includeonly| is to generate
documents consisting of selected parts of the complete document.

However, there are a few drawbacks of the plain |\include| mechanism:
\begin{itemize}
\item
The child files cannot be compiled on their own,
they can only be compiled via the main file.
A naive editing environment
(such as a text editor with an option
to have the current file processed by \LaTeX)
may require one to switch to the main file before compiling;
attempting to compile the child file produces errors.
\item
The main file must be modified (each time)
to adjust the |\includeonly| command
to the present needs. This easily leaves the main file in a messy state.
\item
The generated document will always carry the filename
of the main document. This is inconvenient if
several child files are to be compiled and
to be kept for distribution.
\end{itemize}

The present package provides a simple interface
to make child files individually compilable by \LaTeX{}.
Compiling a child file then has the same effect as compiling
the main file with an |\includeonly| command
to select the appropriate child.
Moreover the generated document will carry the name of the child
rather than the main file.
This resolves all three above issues.

This feature is meant to make the editing of books,
thesis documents and lecture notes somewhat more convenient.
However, the package can also be used efficiently for
composing a series of documents (such as exercise sheets)
which are typically distributed individually.
It then assists the author in generating the individual documents
(potentially in different versions)
as well as a document containing the collected series.
Another application is in developing style files
or other kinds of included material
where compilation of the style file could redirect
to a sample or test file.

%%%%%%%%%%%%%%%%%%%%%%%%%%%%%%%%%%%%%%%%%%%%%%%%%%%%%%%%%%%%%%%%%%%%%%%%%%%%%%%%
%%%%%%%%%%%%%%%%%%%%%%%%%%%%%%%%%%%%%%%%%%%%%%%%%%%%%%%%%%%%%%%%%%%%%%%%%%%%%%%%
\section{Usage}

First of all, the package \textsf{childdoc} is \emph{not} a standard
\LaTeXe{} |.sty| style file! Therefore it needs to be invoked in
a non-standard way.

%%%%%%%%%%%%%%%%%%%%%%%%%%%%%%%%%%%%%%%%%%%%%%%%%%%%%%%%%%%%%%%%%%%%%%%%%%%%%%%%
\subsection{Included Files}
\label{sec:include}

%%%%%%%%%%%%%%%%%%%%%%%%%%%%%%%%%%%%%%%%
\DescribeMacro{\childdocmain}
To use the package, add the commands
\begin{center}
\begin{tabular}{l}
|\input{childdoc.def}|\\
|\childdocmain{}|\\
\end{tabular}
\end{center}
at the very top of the main \LaTeX{} file,
in particular \emph{before} the |\documentclass| statement!
The argument of |\childdocmain| should be left empty
(but it must be present).

%%%%%%%%%%%%%%%%%%%%%%%%%%%%%%%%%%%%%%%%
\DescribeMacro{\childdocof}
Furthermore, add the commands
\begin{center}
\begin{tabular}{l}
|\input{childdoc.def}|\\
|\childdocof{|\textit{main}|}|\\
\end{tabular}
\end{center}
at the top of every child file \textit{child}
which is included by |\include{|\textit{child}|}|
from within the main file
(or at least for those files to be compiled individually).
The argument \textit{main} must be the filename of the main file.

There are a couple of
considerations in setting up the main and child documents:

%%%%%%%%%%%%%%%%%%%%%%%%%%%%%%%%%%%%%%%%
\paragraph{Restrictions.}

Please note the following restrictions:
\begin{itemize}
\item
|\childdocmain| must be called with one argument \textit{main}
to ensure compatibility with earlier version of the package.
It must either be empty (|\childdocmain{}|)
or precisely match the filename of the main file in which it is specified.
See \secref{sec:detection} for further information.
\item
The filename \textit{main} must be specified without the |.tex| extension.
\item
The filename \textit{main} is case sensitive
(even in case-insensitive file systems)
due to internal string comparison.
\item
The argument \textit{main} should be fully expanded, it cannot be a macro.
\item
Subdirectories and special characters should be avoided in filenames.
\item
The command |\childdocmain{|\textit{main}|}| must be followed by a whitespace.
It should not be followed immediately by another command
or by a comment mark `|%|'.
This is because the \TeX{} parser reads the token immediately following
the argument of |\childdocmain| and puts it
at the beginning of every child section;
however, a white\-space is ignored.
\end{itemize}

%%%%%%%%%%%%%%%%%%%%%%%%%%%%%%%%%%%%%%%%
\paragraph{Content of Main File.}

It is advisable to place all content in the child files included by |\include|.
Any output contained in the main file will appear in all child documents
unless suppressed manually;
it cannot be suppressed automatically by the |\includeonly| directive
and thus should normally be avoided.
A method to include some content in the main file
by means of conditional processing is described in \secref{sec:conditional}.

%%%%%%%%%%%%%%%%%%%%%%%%%%%%%%%%%%%%%%%%
\paragraph{Page Numbering.}

When only a part of the document is compiled,
the appropriate numbering of pages
(as well as other status parameters)
is determined from the |.aux| files.
The latter contain information from previous passes.
However this information needs to propagate through
all intermediate child documents.
Therefore the page numbering in child documents may well
be inconsistent until the complete document is compiled at least once.

A useful (if unconventional) way to always ensure a consistent
page numbering is to restart the numbering in each child document
and denote the pages by `\textit{child}|.|\textit{page}'
where \textit{child} represents the chapter/section number of the child file.
This can be achieved by the command
|\numberwithin{page}{|\textit{child}|}|
of the \textsf{amsmath} package
where \textit{child} can be |chapter| or |section|
depending on the chosen structuring.
Alternatively, one can modify the macro |\thepage| appropriately
and reset the counter |page| at the start of each child file.

%%%%%%%%%%%%%%%%%%%%%%%%%%%%%%%%%%%%%%%%%%%%%%%%%%%%%%%%%%%%%%%%%%%%%%%%%%%%%%%%
\subsection{Conditional Processing}
\label{sec:conditional}

The package provides a mechanism to compile different versions
of a document. To customise the versions further some conditional processing
can come in handy to distinguish which version is being compiled.
The package provides two macros to describe the compilation context:

%%%%%%%%%%%%%%%%%%%%%%%%%%%%%%%%%%%%%%%%
\DescribeMacro{\ifchilddoc}
The conditional |\ifchilddoc| distinguishes between the compilation of
child documents and the main document:
%
\begin{center}
|\ifchilddoc |\textit{child-code}| |[|\||else |\textit{main-code}]| \||fi|
\end{center}

%%%%%%%%%%%%%%%%%%%%%%%%%%%%%%%%%%%%%%%%
\DescribeMacro{\childdocname}
\DescribeMacro{\childdocjob}
The macro |\childdocname| contains the filename (without extension)
of the main or child file being processed.
Note that |\childdocjob| will always contain the name of the main file.

%%%%%%%%%%%%%%%%%%%%%%%%%%%%%%%%%%%%%%%%
\paragraph{Title Page.}

Conditional processing can be used to include a title or banner page
in the main document when proper precautions are taken.
Importantly, the code in the main file should ensure that the page counter
(as well as other status parameters which are stored in the |.aux| files)
takes the same value after the conditional processing.
Otherwise the page numbers may take divergent values
depending on which part is compiled.

For example, a title page could be declared by:
%
\begin{center}
\begin{tabular}{l}
|\ifchilddoc\||else|\\
|\addtocounter{page}{-1}|\\
\textit{code for title page}\\
|\newpage|\\
|\||fi|
\end{tabular}
\end{center}
%
A banner page for the child documents can be generated by:
%
\begin{center}
\begin{tabular}{l}
|\ifchilddoc|\\
|\addtocounter{page}{-1}|\\
\textit{code for banner page}\\
|\newpage|\\
|\||fi|
\end{tabular}
\end{center}
%
Here one could write a message such as:
\begin{center}
|This is the part \childdocname{} of \childdocjob{}.|
\end{center}

%%%%%%%%%%%%%%%%%%%%%%%%%%%%%%%%%%%%%%%%%%%%%%%%%%%%%%%%%%%%%%%%%%%%%%%%%%%%%%%%
\subsection{Flags}
\label{sec:flags}

The package makes it easy to generate different versions
of the main or child documents.
To this end compilation flags can be defined
and assigned different default values.
They will be particularly useful in conjunction
with the forwarding mechanism described in \secref{sec:forward}.

For example, it may be useful to have a flag |\version|
which can be set to |draft| or |final|.
The document source will contain some conditional code
depending on the value of |\version|.
Suppose further, the flag should default to |final| for the main file
and to |draft| for child files
which is a natural assignment for editing the document.
This is achieved by placing the following code
in the preamble of the main document
(below the |\childdocmain| directive):
%
\begin{center}
\begin{tabular}{l}
|\ifchilddoc|\\
|\providecommand{\version}{draft}|\\
|\||else|\\
|\providecommand{\version}{final}|\\
|\||fi|
\end{tabular}
\end{center}
%
The definition by |\providecommand| makes sure
that previous definitions are not overwritten.
Further statements |\providecommand{\version}{...}|
can thus be added before the above code to override it.

For the main file, one might add a line
(between |\childdocmain| and the above block)
%
\begin{center}
|%\ifchilddoc\||else\providecommand{\version}{draft}\||fi|
\end{center}
%
which can be uncommented to produce a draft version.
Likewise one can add a line to the very top of a child file
(above the |\childdocof{|\textit{main}|}| directive)
%
\begin{center}
|%\providecommand{\version}{final}|
\end{center}
%
which can be uncommented to produce the final version of this child document.

%%%%%%%%%%%%%%%%%%%%%%%%%%%%%%%%%%%%%%%%%%%%%%%%%%%%%%%%%%%%%%%%%%%%%%%%%%%%%%%%
\subsection{Forwarding}
\label{sec:forward}

Different versions of the main or child documents
using compilation flags as described in \secref{sec:flags}
can be (permanently) stored in different files
for convenient compilation, viewing and distribution.
To this end, the package defines a command
to pass on compilation to a different file:

%%%%%%%%%%%%%%%%%%%%%%%%%%%%%%%%%%%%%%%%
\DescribeMacro{\childdocforward}
The command |\childdocforward| redirects processing to
another source file:
%
\begin{center}
\begin{tabular}{l}
|\input{childdoc.def}|\\
|\childdocforward[|\textit{main}|]{|\textit{dest}|}|\\
\end{tabular}
\end{center}
%
The argument \textit{dest} is the destination file
(without extension).
It should be the main file or one of the child files.
Note that further \textsf{childdoc} directives
such as |\childdocof| and |\childdocforward|
in the indicated file will be processed in this form.
The optional argument \textit{main}
passes on directly to the main file \textit{main}
while pretending to compile the child \textit{dest}.
This form behaves as if \textit{dest}
issues |\childdocof{|\textit{main}|}| right away,
and no further \textsf{childdoc} directives will be processed.

%%%%%%%%%%%%%%%%%%%%%%%%%%%%%%%%%%%%%%%%
\DescribeMacro{\...prefix}
In the alternative form |\childdocforwardprefix|,
%
\begin{center}
\begin{tabular}{l}
|\input{childdoc.def}|\\
|\childdocforwardprefix[|\textit{main}|]{|\textit{prefix}|}{|\textit{dest}|}|
\end{tabular}
\end{center}
%
the destination file is determined by a pattern
depending on the current file:
To make this work, the current file must be called
`{\textit{prefix}\hspace{0.2em}\textit{suffix}}'
with \textit{prefix} matching precisely the argument.
Processing is then passed on to the file
`{\textit{dest}\hspace{0.2em}\textit{suffix}}'.
Surely, the same effect is achieved by
directly specifying the
argument `{\textit{dest}\hspace{0.2em}\textit{suffix}}'
in the first form.
However, that requires to set up a different file
for each child. With the alternative form of the command
all these files can have exactly the same content
which simplifies setting them up and maintaining them.

For example, the following file |draft.tex|
with a compilation flag |\version| as described in \secref{sec:flags}
compiles the main document as a draft:
%
\begin{center}
\begin{tabular}{l}
|\def\version{draft}|\\
|\input{childdoc.def}|\\
|\childdocforward{|\textit{main}|}|
\end{tabular}
\end{center}
%
Likewise, the following files |final|\textit{nn}|.tex|
compile the final version of the child document
|child|\textit{nn}|.tex|:
%
\begin{center}
\begin{tabular}{l}
|\def\version{final}|\\
|\input{childdoc.def}|\\
|\childdocforwardprefix{final}{child}|
\end{tabular}
\end{center}
%

Note that when several versions of a main file and/or of each child file
are to be generated, it may be convenient to set up a |Makefile| or
shell script to automatise the process.

%%%%%%%%%%%%%%%%%%%%%%%%%%%%%%%%%%%%%%%%%%%%%%%%%%%%%%%%%%%%%%%%%%%%%%%%%%%%%%%%
\subsection{Command Line Processing}
\label{sec:commandline}

The effect of redirection files can also be achieved by invoking
the \LaTeX{} compiler with a more elaborate command line.
Most conveniently this should be done as part
of a shell script or a |Makefile|.

When using \textsf{childdoc} in the main file, the following
command lines effectively perform a redirection
(note that depending on the shell being used,
backslashes may have to be doubled: `|\|' $\to$ `|\\|'):
%
\begin{center}
|... -jobname "|\textit{target}|" |\\|"|[\textit{flags}]%
|\input{childdoc.def}\childdocforward[|\textit{main}|]{|\textit{dest}|}"|
\end{center}
%
Here \textit{target} is the name of the output file,
\textit{main} is the name of the main file
and \textit{dest} is the name of the main or child file to be processed
(all filenames without extensions).
The optional argument \textit{main} can be omitted
if \textit{main} matches \textit{dest}.
Optionally, compilation \textit{flags} can be defined via |\def| commands.
This command line makes the \TeX{} engine believe
it is compiling the file \textit{target}
whose content is specified as the latter parameter.
The provided code then forwards the processing to
\textit{main} or \textit{dest} as described in \secref{sec:forward}.

%%%%%%%%%%%%%%%%%%%%%%%%%%%%%%%%%%%%%%%%%%%%%%%%%%%%%%%%%%%%%%%%%%%%%%%%%%%%%%%%
\subsection{Include by Input}
\label{sec:input}

Including child documents by |\include| has some restrictions by design.
Most notably, the content of a child document always occupies
its own set of pages; pages cannot be shared between child documents.
Usually, this behaviour makes perfect sense
because each child document contain an essential part of the document.
However, in some situations it may be desirable to compose
a document from a collection of parts
without having mandatory page breaks between then.
For this case, the package
provides a mechanism to include parts
by |\input| which can also be processed individually.
However, by construction this mechanism
requires manual handling of the content to be output.

%%%%%%%%%%%%%%%%%%%%%%%%%%%%%%%%%%%%%%%%
\DescribeMacro{\ifchilddocmanual}
The main file should be prepared as usual, see \secref{sec:include}.
However, the document body must make a distinction
between processing of an individual part and of the main document, e.g.:
%
\begin{center}
\begin{tabular}{l}
|\ifchilddocmanual|\\
|\input{\childdocname}|\\
|\||else|\\
\textit{document body with }|\input{|\textit{part}|}|\\
|\||fi|
\end{tabular}
\end{center}
%
The conditional |\ifchilddocmanual| is true whenever
a part to be included by |\input| is being compiled,
and the name of the part is stored in |\childdocname|.

%%%%%%%%%%%%%%%%%%%%%%%%%%%%%%%%%%%%%%%%
\DescribeMacro{\childdocby}
Each part to be included by |\input| should start with:
%
\begin{center}
\begin{tabular}{l}
|\input{childdoc.def}|\\
|\childdocby{|\textit{main}|}|\\
\end{tabular}
\end{center}
%
The directive |\childdocby| is similar to |\childdocof|
described in \secref{sec:include},
but the subsequent selection of content must be done manually.
To that end, both |\ifchilddoc| and |\ifchilddocmanual|
will be true upon processing of a part,
and the name of the part is stored in |\childdocname|.
Note that |\jobname| will be set to the filename of the current part
so that each part receives an individual |.aux| file
that does not interfere with the |.aux| file(s) of the main document.
This behaviour can be altered by the alternative form
|\childdocby[*]{|\textit{main}|}| (with a non-empty optional argument)
which uses the |.aux| file of the main document
by setting |\jobname| to \textit{main}.

%%%%%%%%%%%%%%%%%%%%%%%%%%%%%%%%%%%%%%%%%%%%%%%%%%%%%%%%%%%%%%%%%%%%%%%%%%%%%%%%
\subsection{Driver Development}
\label{sec:driver}

The \textsf{childdoc} mechanism can also be use for the development
of definition files such as \LaTeX{} styles or classes.
This case differs from the above setup with multiple parts
included by |\include| in that no |\includeonly| should be invoked.
This can be achieved by starting the include file
(before |\ProvidesPackage|) with:
%
\begin{center}
\begin{tabular}{l}
|\input{childdoc.def}|\\
|\childdocforward{|\textit{main}|}|\\
\end{tabular}
\end{center}
%
or alternatively with:
%
\begin{center}
\begin{tabular}{l}
|\input{childdoc.def}|\\
|\childdocby{|\textit{main}|}|\\
\end{tabular}
\end{center}
%
Both forms have slightly different effects as described above.
The main file is prepared as usual, see \secref{sec:include}.

%%%%%%%%%%%%%%%%%%%%%%%%%%%%%%%%%%%%%%%%%%%%%%%%%%%%%%%%%%%%%%%%%%%%%%%%%%%%%%%%
\subsection{Legacy Detection}
\label{sec:detection}

The directive |\childdocmain| in the main file can detect
whether the complete document or merely a child is to be compiled
even without using the directive |\childdocof|.
This method is deprecated because it is less robust
and there is no compelling reason to use it;
it is merely provided for backward compatibility
and it may be removed in future versions.

If the detection mechanism is to be used,
it is mandatory to correctly specify
the filename of the main file as the argument of |\childdocmain|:
%
\begin{center}
\begin{tabular}{l}
|\input{childdoc.def}|\\
|\childdocmain{|\textit{main}|}|\\
\end{tabular}
\end{center}
%
If |\jobname| does not match the argument \textit{main} of |\childdocmain|,
it is assumed that |\jobname| points to the child file to be compiled.
When using |\childdocmain| with the main file specified as argument,
it suffices to start a child file
with just |\input{|\textit{main}|}|
without loading of the package and using |\childdocof|.
If instead all processing is done
with the appropriate \textsf{childdoc} directives,
the argument of \textit{main} of |\childdocmain| can be empty.

An alternative version of the command line processing described
in \secref{sec:commandline} using the detection mechanism reads:
%
\begin{center}
|... -jobname "|\textit{target}|" "|[\textit{flags}]%
[|\def\jobname{|\textit{dest}|}|]|\input{|\textit{main}|}"|
\end{center}

%%%%%%%%%%%%%%%%%%%%%%%%%%%%%%%%%%%%%%%%%%%%%%%%%%%%%%%%%%%%%%%%%%%%%%%%%%%%%%%%
\subsection{Manual Code}
\label{sec:manual}

In case one cannot be certain whether the definitions file |childdoc.def|
is installed on the target \TeX{} distribution
and one prefers not to ship it,
it is conceivable to paste a few relevant commands into the sources.

To that end, drop all statements |\input{childdoc.def}|
and perform the replacements as outlined below.
Instead of |\childdocmain{|\textit{main}|}| add the following code
to the top of the main file:
%
\begin{center}
\begin{tabular}{l}
|\||ifdefined\childdocname\endinput\||fi\newif\ifchilddoc|\\
|\edef\childdocname{\scantokens\expandafter{\jobname\noexpand}}|\\
|\def\childdocmain{|\textit{main}|}\||ifx\childdocmain\childdocname\||else|\\
|\childdoctrue\includeonly{\childdocname}\let\jobname\childdocmain\||fi|\\
\end{tabular}
\end{center}
%
Instead of |\childdocof{|\textit{main}|}| just include the main file
at the top of each child file:
%
\begin{center}
|\input{|\textit{main}|}|
\end{center}
%
A simple redirection |\childdocforward{|\textit{dest}|}| is achieved by:
%
\begin{center}
|\def\jobname{|\textit{dest}|}\input{\jobname}|
\end{center}
%
The redirection with prefix
|\childdocforwardprefix[|\textit{prefix}|]{|\textit{dest}|}|
is accomplished by:
%
\begin{center}
\begin{tabular}{l}
|{\edef\jobname{\scantokens\expandafter{\jobname\noexpand}}|\\
|\def\redirectjob |\textit{prefix}|#1~~~{\gdef\jobname{|\textit{dest}|#1}}|\\
|\expandafter\redirectjob\jobname~~~}\input{\jobname}|
\end{tabular}
\end{center}

In an alternative approach,
child documents can be compiled by a specific command line
without additional code or specific definitions:
%
\begin{center}
|... -jobname "|\textit{target}|" "|[\textit{flags}]%
|\includeonly{|\textit{dest}|}\input{|\textit{main}|}"|
\end{center}
%

%%%%%%%%%%%%%%%%%%%%%%%%%%%%%%%%%%%%%%%%%%%%%%%%%%%%%%%%%%%%%%%%%%%%%%%%%%%%%%%%
%%%%%%%%%%%%%%%%%%%%%%%%%%%%%%%%%%%%%%%%%%%%%%%%%%%%%%%%%%%%%%%%%%%%%%%%%%%%%%%%
\section{Information}

%%%%%%%%%%%%%%%%%%%%%%%%%%%%%%%%%%%%%%%%%%%%%%%%%%%%%%%%%%%%%%%%%%%%%%%%%%%%%%%%
\subsection{Copyright}

Copyright \copyright{} 2017--2018 Niklas Beisert

This work may be distributed and/or modified under the
conditions of the \LaTeX{} Project Public License, either version 1.3
of this license or (at your option) any later version.
The latest version of this license is in
  \url{http://www.latex-project.org/lppl.txt}
and version 1.3 or later is part of all distributions of \LaTeX{}
version 2005/12/01 or later.

This work has the LPPL maintenance status `maintained'.

The Current Maintainer of this work is Niklas Beisert.

This work consists of the files |README.txt|, |childdoc.ins| and |childdoc.dtx|
as well as the derived files |childdoc.def|, |cdocsamp.tex|
with |cdocsch1.tex|, |cdocsch2.tex|, |cdocspt3.tex|, |cdocspt4.tex|,
|cdocsdrf.tex|, |cdocsfn1.tex|, |cdocsfn2.tex|
as well as |childdoc.pdf|.

%%%%%%%%%%%%%%%%%%%%%%%%%%%%%%%%%%%%%%%%%%%%%%%%%%%%%%%%%%%%%%%%%%%%%%%%%%%%%%%%
\subsection{Files and Installation}

The package consists of the files:
%
\begin{center}
\begin{tabular}{ll}
    |README.txt|   & readme file \\
    |childdoc.ins| & installation file \\
    |childdoc.dtx| & source file \\
    |childdoc.def| & definition file \\
    |cdocsamp.tex| & sample main file \\
    |cdocsch1.tex| & sample include file \\
    |cdocsch2.tex| & sample include file \\
    |cdocspt3.tex| & sample part file \\
    |cdocspt4.tex| & sample part file \\
    |cdocsdrf.tex| & sample redirection file \\
    |cdocsfn1.tex| & sample redirection file \\
    |cdocsfn2.tex| & sample redirection file \\
    |childdoc.pdf| & manual
\end{tabular}
\end{center}
%
The distribution consists of the files
|README.txt|, |childdoc.ins| and |childdoc.dtx|.
%
\begin{itemize}
\item
Run (pdf)\LaTeX{} on |childdoc.dtx|
to compile the manual |childdoc.pdf| (this file).
\item
Run \LaTeX{} on |childdoc.ins| to create the definitions file |childdoc.def|
and the sample |cdocsamp.tex| with include files
|cdocsch1.tex|, |cdocsch2.tex|, |cdocspt3.tex|, |cdocspt4.tex|,
|cdocsdrf.tex|, |cdocsfn1.tex|, |cdocsfn2.tex|.
Then copy the file |childdoc.def| to an appropriate directory of your \LaTeX{}
distribution, e.g.\ \textit{texmf-root}|/tex/latex/childdoc|.
\end{itemize}

%%%%%%%%%%%%%%%%%%%%%%%%%%%%%%%%%%%%%%%%%%%%%%%%%%%%%%%%%%%%%%%%%%%%%%%%%%%%%%%%
\subsection{Related CTAN Packages}

There are several other packages which offer a similar functionality:
%
\begin{itemize}
\item
The packages
\href{http://ctan.org/pkg/docmute}{\textsf{docmute}},
\href{http://ctan.org/pkg/includex}{\textsf{includex}} and
\href{http://ctan.org/pkg/standalone}{\textsf{standalone}}
provide commands to include only the document body of
a child file thus allowing both files to be compiled individually.
\item
The packages \href{http://ctan.org/pkg/subdocs}{\textsf{subdocs}}
and \href{http://ctan.org/pkg/subfiles}{\textsf{subfiles}}
provide structures in which the main and child documents can be
encapsulated and allowing them to be compiled individually.
The inclusion mechanism is different from the conventional |\include|.
\item
The package \href{http://ctan.org/pkg/combine}{\textsf{combine}}
is an elaborate solution to combine several documents into one.
\end{itemize}
%
See also the CTAN topic \href{http://ctan.org/topic/subdocs}{\textsf{subdocs}}
for further related packages.
The present package differs from the above solutions in that
a document structure constructed with the conventional |\include| mechanism
just needs two extra commands at the top of every file
such that all constituent files can be compiled individually.

%%%%%%%%%%%%%%%%%%%%%%%%%%%%%%%%%%%%%%%%%%%%%%%%%%%%%%%%%%%%%%%%%%%%%%%%%%%%%%%%
%\subsection{Feature Suggestions}
%
%The following is a list of features which may be useful for future
%versions of this package:
%%
%\begin{itemize}
%\item
%\ldots
%\end{itemize}

%%%%%%%%%%%%%%%%%%%%%%%%%%%%%%%%%%%%%%%%%%%%%%%%%%%%%%%%%%%%%%%%%%%%%%%%%%%%%%%%
\subsection{Revision History}

%%%%%%%%%%%%%%%%%%%%%%%%%%%%%%%%%%%%%%%%
\paragraph{v2.0:} 2018/12/30

\begin{itemize}
\item
immediate forward processing
\item
added |\childdocby| mechanism
\item
manual restructured
\end{itemize}

%%%%%%%%%%%%%%%%%%%%%%%%%%%%%%%%%%%%%%%%
\paragraph{v1.6:} 2018/01/17

\begin{itemize}
\item
application for development of include files
\item
corrections to manual
\end{itemize}

%%%%%%%%%%%%%%%%%%%%%%%%%%%%%%%%%%%%%%%%
\paragraph{v1.5:} 2017/05/21

\begin{itemize}
\item
more complete structuring introduced
\item
|\childdocof| introduced
\item
|\childdoc| renamed to |\childdocmain|
\item
|\childredirect| renamed to |\childdocforward| and |\childdocforwardprefix|
and functionality expanded
\end{itemize}

%%%%%%%%%%%%%%%%%%%%%%%%%%%%%%%%%%%%%%%%
\paragraph{v1.0:} 2017/04/27

\begin{itemize}
\item
manual and install package
\item
first version published on CTAN
\end{itemize}

%%%%%%%%%%%%%%%%%%%%%%%%%%%%%%%%%%%%%%%%
\paragraph{v0.6:} 2017/04/26

\begin{itemize}
\item
redirection mechanism added
\end{itemize}

%%%%%%%%%%%%%%%%%%%%%%%%%%%%%%%%%%%%%%%%
\paragraph{v0.5:} 2017/04/26

\begin{itemize}
\item
functionality in definition file
\end{itemize}


%%%%%%%%%%%%%%%%%%%%%%%%%%%%%%%%%%%%%%%%%%%%%%%%%%%%%%%%%%%%%%%%%%%%%%%%%%%%%%%%
%%%%%%%%%%%%%%%%%%%%%%%%%%%%%%%%%%%%%%%%%%%%%%%%%%%%%%%%%%%%%%%%%%%%%%%%%%%%%%%%
%%%%%%%%%%%%%%%%%%%%%%%%%%%%%%%%%%%%%%%%%%%%%%%%%%%%%%%%%%%%%%%%%%%%%%%%%%%%%%%%
\appendix

\settowidth\MacroIndent{\rmfamily\scriptsize 000\ }

 \DocInput{childdoc.dtx}

\end{document}
%</driver>
% \fi
%
% %%%%%%%%%%%%%%%%%%%%%%%%%%%%%%%%%%%%%%%%%%%%%%%%%%%%%%%%%%%%%%%%%%%%%%%%%%%%%%
% %%%%%%%%%%%%%%%%%%%%%%%%%%%%%%%%%%%%%%%%%%%%%%%%%%%%%%%%%%%%%%%%%%%%%%%%%%%%%%
% \section{Sample}
%\iffalse
%<*samplemain>
%\fi
%
% The following presents a sample document
% with two chapters, two parts, a title page,
% a compile flag as well as three forwarding files to set the flag.
% It consists of eight |.tex| files:
% \begin{center}
% \begin{tabular}{ll}
% |cdocsamp.tex|&main file\\
% |cdocsch1.tex|&include file for chapter 1\\
% |cdocsch2.tex|&include file for chapter 2\\
% |cdocspt3.tex|&include file for part 3\\
% |cdocspt4.tex|&include file for part 4\\
% |cdocsdrf.tex|&forwarding file for main file in draft mode\\
% |cdocsfi1.tex|&forwarding file for final version of chapter 1\\
% |cdocsfi2.tex|&forwarding file for final version of chapter 2\\
% \end{tabular}
% \end{center}
% Each of the eight files can be compiled directly by the \LaTeX{} compiler.
%
% %%%%%%%%%%%%%%%%%%%%%%%%%%%%%%%%%%%%%%
% \paragraph{Main File.}
%
% The main file is called |cdocsamp.tex|.
%
% Load the \textsf{childdoc} definitions and
% declare the filename for the main document:
%    \begin{macrocode}
\input{childdoc.def}
\childdocmain{}
%    \end{macrocode}

% Optional override for |\version| flag:
%    \begin{macrocode}
%%\ifchilddoc\else\providecommand{\version}{draft}\fi
%    \end{macrocode}

% Define the default values for the |\version| flag
% (|final| for the main file and |draft| for childs):
%    \begin{macrocode}
\ifchilddoc
\providecommand{\version}{draft}
\else
\providecommand{\version}{final}
\fi
%    \end{macrocode}

% Load the standard document class:
%    \begin{macrocode}
\documentclass[12pt]{article}
%    \end{macrocode}

% Start the document body:
%    \begin{macrocode}
\begin{document}
%    \end{macrocode}

% Declare a title page.
% Print title, part of document being processed and version flag:
%    \begin{macrocode}
\addtocounter{page}{-1}
\begin{center}
{\LARGE\bfseries{}childdoc example\par}
\vspace{1cm}
\ifchilddoc
\ifchilddocmanual part\else chapter\fi:
`\childdocname' of `\childdocjob'\par
\else
main document: `\childdocjob'\par
\fi
version: \version\par
\end{center}
\newpage
%    \end{macrocode}

% Manually include selected file,
% otherwise process as usual:
%    \begin{macrocode}
\ifchilddocmanual
\section*{part `\childdocname'}
\input{\childdocname}
\else
%    \end{macrocode}

% Include the two chapters:
%    \begin{macrocode}
\include{cdocsch1}
\include{cdocsch2}
%    \end{macrocode}

% Include the two parts unless only chapters should be displayed:
%    \begin{macrocode}
\ifchilddoc\else
\section{part three}
\input{cdocspt3}
\section{part four}
\input{cdocspt4}
\fi
%    \end{macrocode}

% Process as usual until here:
%    \begin{macrocode}
\fi
%    \end{macrocode}

% End of document body:
%    \begin{macrocode}
\end{document}
%    \end{macrocode}
%\iffalse
%</samplemain>
%\fi
%
% %%%%%%%%%%%%%%%%%%%%%%%%%%%%%%%%%%%%%%
% \paragraph{Chapter Include Files.}
%
% The include files are called |cdocsch1.tex| and |cdocsch2.tex|.
%
%\iffalse
%<*samplechap1|samplechap2>
%\fi

% Optional override for |\version| flag:
%    \begin{macrocode}
%%\providecommand{\version}{final}
%    \end{macrocode}

% Include the main document:
%    \begin{macrocode}
\input{childdoc.def}
\childdocof{cdocsamp}
%    \end{macrocode}

%\iffalse
%</samplechap1|samplechap2>
%\fi
%
%\iffalse
%<*samplechap1>
%\fi
% Some text for chapter 1:
%    \begin{macrocode}
\section{one}
some text in chapter one
%    \end{macrocode}

%\iffalse
%</samplechap1>
%\fi
% Some text for chapter 2:
%\iffalse
%<*samplechap2>
%\fi
%    \begin{macrocode}
\section{two}
more text in chapter two
%    \end{macrocode}

%\iffalse
%</samplechap2>
%\fi
%
% %%%%%%%%%%%%%%%%%%%%%%%%%%%%%%%%%%%%%%
% \paragraph{Part Include Files.}
%
% The include files are called |cdocspt3.tex| and |cdocspt4.tex|.
%
%\iffalse
%<*samplepart3|samplepart4>
%\fi

% Optional override for |\version| flag:
%    \begin{macrocode}
%%\providecommand{\version}{final}
%    \end{macrocode}

% Include the main document:
%    \begin{macrocode}
\input{childdoc.def}
\childdocby{cdocsamp}
%    \end{macrocode}

%\iffalse
%</samplepart3|samplepart4>
%\fi
%
%\iffalse
%<*samplepart3>
%\fi
% Some text for part 3:
%    \begin{macrocode}
some text in part three
%    \end{macrocode}

%\iffalse
%</samplepart3>
%\fi
% Some text for part 4:
%\iffalse
%<*samplepart4>
%\fi
%    \begin{macrocode}
more text in part four
%    \end{macrocode}

%\iffalse
%</samplepart4>
%\fi
%
% %%%%%%%%%%%%%%%%%%%%%%%%%%%%%%%%%%%%%%
% \paragraph{Forwarding for a Complete Draft.}
%
% The following forwarding file |cdocsdrf.tex|
% compiles the main document in draft mode:
%\iffalse
%<*sampledraft>
%\fi
%    \begin{macrocode}
\def\version{draft}
\input{childdoc.def}
\childdocforward{cdocsamp}
%    \end{macrocode}

%\iffalse
%</sampledraft>
%\fi
%
% %%%%%%%%%%%%%%%%%%%%%%%%%%%%%%%%%%%%%%
% \paragraph{Forwarding for Final Version of the Chapters.}
%
% The following forwarding files |cdocsfn1.tex| and |cdocsfn2.tex|
% (with identical content)
% compile the final versions of the child documents
% |cdocsch1.tex| and |cdocsch2.tex|, respectively:
%\iffalse
%<*samplefinal>
%\fi
%    \begin{macrocode}
\def\version{final}
\input{childdoc.def}
\childdocforwardprefix[cdocsamp]{cdocsfn}{cdocsch}
%    \end{macrocode}

%\iffalse
%</samplefinal>
%\fi
%
% %%%%%%%%%%%%%%%%%%%%%%%%%%%%%%%%%%%%%%
% \paragraph{Command Line Processing.}
%
% The following three command lines generate the output files
% |cdocscld|, |cdocscl1| and |cdocscl2|
% which should be identical to
% |cdocsdrf|, |cdocsch1| and |cdocsfn2|, respectively:
% \begin{center}
% \begin{tabular}{l}
% |latex -jobname cdocscld \|\\
% |  "\def\version{draft}\input{childdoc.def}\childdocforward{cdocsamp}"|\\
% |latex -jobname cdocscl1 \|\\
% |  "\input{childdoc.def}\childdocforward[cdocsamp]{cdocsch1}"|\\
% |latex -jobname cdocscl2 \|\\
% |  "\def\version{final}\input{childdoc.def}\childdocforward{cdocsch2}"|
% \end{tabular}
% \end{center}
% Note that the trailing backslash on each first line
% merely continues the input to the second line
% (for convenient cut ant paste).
% Furthermore, the command |latex| can be replaced by any
% of its alternative versions such as |pdflatex|.
%
% %%%%%%%%%%%%%%%%%%%%%%%%%%%%%%%%%%%%%%%%%%%%%%%%%%%%%%%%%%%%%%%%%%%%%%%%%%%%%%
% %%%%%%%%%%%%%%%%%%%%%%%%%%%%%%%%%%%%%%%%%%%%%%%%%%%%%%%%%%%%%%%%%%%%%%%%%%%%%%
% \section{Implementation}
%\iffalse
%<*package>
%\fi
%
% This section describes the definitions file |childdoc.def|.

% The definitions cannot be loaded using |\usepackage| or |\RequirePackage|
% which has a mechanism to prevent loading a style file more than once.
% When loading the definitions by means of |\input|
% multiple instances have to be prevented manually:
%\iffalse
%This code needs to be before the `\ProvidesFile' directive
%which is defined at the beginning of this file.
%Therefore it is also placed there and commented out here.
%</package>
%<*discard>
%\fi
%    \begin{macrocode}
\ifdefined\childdocmain\endinput\fi
%    \end{macrocode}
%\iffalse
%</discard>
%<*package>
%\fi
%
% \macro{\ifchilddoc}
% \macro{\ifchilddocmanual}
% The conditional |\ifchilddoc| tells whether a
% child (true) or main (false) document is being compiled.
% The conditional |\ifchilddocmanual| tells whether
% the |\includeonly| mechanism is used (false) or
% the selection of child files must be performed manually (true).
% The definitions initialise to false:
%    \begin{macrocode}
\newif\ifchilddoc
\newif\ifchilddocmanual
%    \end{macrocode}

% \macro{\childdocname}
% \macro{\childdocjob}
% The macro |\childdocname| stores the name of the main document
% to be compiled. The macro |\childdocjob| stores the name of
% the document on which the \LaTeX{} compiler was originally invoked.
% The content of |\jobname| cannot be compared
% to filenames specified in the source due to different catcodes.
% The following code rescans |\jobname|, stores the result
% in |\childdocname| and saves a copy in |\childdocjob|:
%    \begin{macrocode}
\edef\childdocname{\scantokens\expandafter{\jobname\noexpand}}
\let\childdocjob\childdocname
%    \end{macrocode}

% \macro{\childdocdisable}
% The macro |\childdocdisable| prevents the main file
% from being processed more than once.
% At this stage, the main document command |\childdocmain|
% is assumed to be called once again where it should do nothing.
% Any subsequent call to it should prevent
% a secondary processing of the main document
% It overwrites the forwarding commands
% |\childdocof| and |\childdocforward|
% with empty macros to prevent further inclusions of the main document:
%    \begin{macrocode}
\newcommand{\childdocdisable}
{
  \renewcommand{\childdocmain}[1]{\renewcommand{\childdocmain}[1]{\endinput}}
  \renewcommand{\childdocof}[1]{}
  \renewcommand{\childdocby}[2][]{}
  \renewcommand{\childdocforward}[2][]{}
  \renewcommand{\childdocdisable}{}
}
%    \end{macrocode}

% \macro{\childdocmain}
% The macro |\childdocmain| is to be called at the top of the main file
% with nothing or the main filename (without extension) as argument.
% First, it breaks loops.
% If the argument is not empty and does not match |\childdocname|
% (which is set by the first inclusion of |childdoc.def|),
% |\ifchilddoc| is set to true, |\includeonly| is applied to the child file
% and |\jobname| is set to the main file
% (for proper handling of |.aux| files):
%    \begin{macrocode}
\newcommand{\childdocmain}[1]
{
  \childdocdisable\childdocmain{}
  \if?#1?\else
    \begingroup
      \def\childdoctmp{#1}
      \ifx\childdoctmp\childdocname
        \def\childdoctmp{}
      \else
        \def\childdoctmp
        {
          \childdoctrue
          \includeonly{\childdocname}
          \def\childdocjob{#1}
          \def\jobname{#1}
        }
      \fi
      \expandafter
    \endgroup
    \childdoctmp
  \fi
}
%    \end{macrocode}

% \macro{\childdocof}
% The command |\childdocof| redirects
% compilation to the main file |#1|.
%    \begin{macrocode}
\newcommand{\childdocof}[1]
{
  \childdocdisable
  \childdoctrue
  \includeonly{\childdocname}
  \def\jobname{#1}
  \def\childdocjob{#1}
  \input{#1}
}
%    \end{macrocode}

% \macro{\childdocby}
% The command |\childdocby| ....
%    \begin{macrocode}
\newcommand{\childdocby}[2][]
{
  \childdocdisable
  \childdoctrue
  \childdocmanualtrue
  \if?#1?\else
    \def\jobname{#2}
  \fi
  \def\childdocjob{#2}
  \input{#2}
  \endinput
}
%    \end{macrocode}

% \macro{\childdocforward}
% The command |\childdocforward| redirects
% compilation to the main file or
% (if the optional argument is given) a child file.
% Parameters are set as if the main file
% or a child file starting with |\childdocof| was compiled.
% Then compilation is handed over to the main file:
%    \begin{macrocode}
\newcommand{\childdocforward}[2][]
{
  \begingroup
    \if?#1?
      \def\childdoctmp
      {
        \def\childdocname{#2}
        \def\childdocjob{#2}
        \def\jobname{#2}
        \input{#2}
        \endinput
      }
    \else
      \def\childdoctmp
      {
        \childdocdisable
        \def\childdocname{#2}
        \childdoctrue
        \includeonly{#2}
        \def\childdocjob{#1}
        \def\jobname{#1}
        \input{#1}
        \endinput
      }
    \fi
    \expandafter
  \endgroup
  \childdoctmp
}
%    \end{macrocode}

% \macro{\childdocforwardprefix}
% The command |\childdocforwardprefix| redirects
% compilation to the main or a child file by means of a pattern.
% The prefix |#1| in the current filename is replaced by |#2|
% and the suffix of the current filename is kept
% (it is assumed that the filename does not contain the substring `|~~~|'
% which is used as a delimiter).
% Compilation is handed over to the new file by |\childdocforward|:
%    \begin{macrocode}
\newcommand{\childdocforwardprefix}[3][]
{
  \begingroup
    \def\childdocextract #2##1~~~{\def\childdoctmp{\childdocforward[#1]{#3##1}}}
    \expandafter\childdocextract\childdocname~~~
    \expandafter
  \endgroup
  \childdoctmp
}
%    \end{macrocode}

% \macro{\childdoc}
% The deprecated macro |\childdoc| is a legacy version of |\childdocmain|:
%    \begin{macrocode}
\newcommand{\childdoc}{\childdocmain}
%    \end{macrocode}

% \macro{\childdocredirect}
% The deprecated macro |\childdocredirect| is a legacy version
% of |\childdocforward| and |\childdocforwardprefix|:
%    \begin{macrocode}
\newcommand{\childdocredirect}[2][]
{
  \begingroup
    \if?#1?
      \def\childdoctmp{\childdocforward{#2}}
    \else
      \def\childdoctmp{\childdocforwardprefix{#1}{#2}}
    \fi
    \expandafter
  \endgroup
  \childdoctmp
}
%    \end{macrocode}

%\iffalse
%</package>
%\fi
%
\endinput
|\\
|\childdocforward{|\textit{main}|}|\\
\end{tabular}
\end{center}
%
or alternatively with:
%
\begin{center}
\begin{tabular}{l}
|% \iffalse
%
% childdoc.dtx Copyright (C) 2017-2018 Niklas Beisert
%
% This work may be distributed and/or modified under the
% conditions of the LaTeX Project Public License, either version 1.3
% of this license or (at your option) any later version.
% The latest version of this license is in
%   http://www.latex-project.org/lppl.txt
% and version 1.3 or later is part of all distributions of LaTeX
% version 2005/12/01 or later.
%
% This work has the LPPL maintenance status `maintained'.
%
% The Current Maintainer of this work is Niklas Beisert.
%
% This work consists of the files childdoc.dtx and childdoc.ins
% and the derived files childdoc.def and cdocsamp.tex with
% cdocsch1.tex, cdocsch2.tex, cdocsdrf.tex, cdocsfn1.tex, cdocsfn2.tex.
%
%<package>\ifdefined\childdocmain\endinput\fi
%<package>\ProvidesFile{childdoc.def}[2018/12/30 v2.0 child document driver]
%<samplemain>\ProvidesFile{cdocsamp.tex}[2018/12/30 v2.0 sample for childdoc]
%<*driver>
%\ProvidesFile{childdoc.drv}[2018/12/30 v2.0 childdoc reference manual file]
\PassOptionsToClass{10pt,a4paper}{article}
\documentclass{ltxdoc}

\usepackage[margin=35mm]{geometry}
\usepackage{hyperref}
\usepackage{hyperxmp}
\usepackage[usenames]{color}

\hypersetup{colorlinks=true}
\hypersetup{pdfstartview=FitH}
\hypersetup{pdfpagemode=UseNone}
\hypersetup{pdfsource={}}
\hypersetup{pdflang={en-UK}}
\hypersetup{pdfcopyright={Copyright 2017-2018 Niklas Beisert.
  This work may be distributed and/or modified under the
  conditions of the LaTeX Project Public License, either version 1.3
  of this license or (at your option) any later version.}}
\hypersetup{pdflicenseurl={http://www.latex-project.org/lppl.txt}}
\hypersetup{pdfcontactaddress={ETH Zurich, ITP, HIT K,
  Wolfgang-Pauli-Strasse 27}}
\hypersetup{pdfcontactpostcode={8093}}
\hypersetup{pdfcontactcity={Zurich}}
\hypersetup{pdfcontactcountry={Switzerland}}
\hypersetup{pdfcontactemail={nbeisert@itp.phys.ethz.ch}}
\hypersetup{pdfcontacturl={http://people.phys.ethz.ch/\xmptilde nbeisert/}}

\newcommand{\secref}[1]{\hyperref[#1]{section \ref*{#1}}}

\parskip1ex
\parindent0pt
\let\olditemize\itemize
\def\itemize{\olditemize\parskip0pt}

\begin{document}

\title{The \textsf{childdoc} Package}
\hypersetup{pdftitle={The childdoc Package}}
\author{Niklas Beisert\\[2ex]
  Institut f\"ur Theoretische Physik\\
  Eidgen\"ossische Technische Hochschule Z\"urich\\
  Wolfgang-Pauli-Strasse 27, 8093 Z\"urich, Switzerland\\[1ex]
  \href{mailto:nbeisert@itp.phys.ethz.ch}
  {\texttt{nbeisert@itp.phys.ethz.ch}}}
\hypersetup{pdfauthor={Niklas Beisert}}
\hypersetup{pdfsubject={Manual for the LaTeX2e Package childdoc}}
\date{30 December 2018, \textsf{v2.0}}
\maketitle

\begin{abstract}\noindent
\textsf{childdoc} is a \LaTeXe{} package
that enables the direct compilation
of document sections included by |\include|
to individual files.
\end{abstract}

\begingroup
\parskip0ex
\tableofcontents
\endgroup

%%%%%%%%%%%%%%%%%%%%%%%%%%%%%%%%%%%%%%%%%%%%%%%%%%%%%%%%%%%%%%%%%%%%%%%%%%%%%%%%
%%%%%%%%%%%%%%%%%%%%%%%%%%%%%%%%%%%%%%%%%%%%%%%%%%%%%%%%%%%%%%%%%%%%%%%%%%%%%%%%
\section{Introduction}

\LaTeX{} provides a mechanism to structure a large document (such as a book)
into a main file and several child files (containing the chapters)
using the |\include| command.
This mechanism is beneficial for documents
which span hundreds of pages in order to
make the source file(s) more manageable.
Moreover, compilation can be restricted to
selected child files by means of the |\includeonly| command.
The latter feature can be used to reduce the compilation time while editing
(this was significantly more useful in the earlier days of \LaTeX{})
or to generate a smaller document which is easier to navigate.
Another application of |\includeonly| is to generate
documents consisting of selected parts of the complete document.

However, there are a few drawbacks of the plain |\include| mechanism:
\begin{itemize}
\item
The child files cannot be compiled on their own,
they can only be compiled via the main file.
A naive editing environment
(such as a text editor with an option
to have the current file processed by \LaTeX)
may require one to switch to the main file before compiling;
attempting to compile the child file produces errors.
\item
The main file must be modified (each time)
to adjust the |\includeonly| command
to the present needs. This easily leaves the main file in a messy state.
\item
The generated document will always carry the filename
of the main document. This is inconvenient if
several child files are to be compiled and
to be kept for distribution.
\end{itemize}

The present package provides a simple interface
to make child files individually compilable by \LaTeX{}.
Compiling a child file then has the same effect as compiling
the main file with an |\includeonly| command
to select the appropriate child.
Moreover the generated document will carry the name of the child
rather than the main file.
This resolves all three above issues.

This feature is meant to make the editing of books,
thesis documents and lecture notes somewhat more convenient.
However, the package can also be used efficiently for
composing a series of documents (such as exercise sheets)
which are typically distributed individually.
It then assists the author in generating the individual documents
(potentially in different versions)
as well as a document containing the collected series.
Another application is in developing style files
or other kinds of included material
where compilation of the style file could redirect
to a sample or test file.

%%%%%%%%%%%%%%%%%%%%%%%%%%%%%%%%%%%%%%%%%%%%%%%%%%%%%%%%%%%%%%%%%%%%%%%%%%%%%%%%
%%%%%%%%%%%%%%%%%%%%%%%%%%%%%%%%%%%%%%%%%%%%%%%%%%%%%%%%%%%%%%%%%%%%%%%%%%%%%%%%
\section{Usage}

First of all, the package \textsf{childdoc} is \emph{not} a standard
\LaTeXe{} |.sty| style file! Therefore it needs to be invoked in
a non-standard way.

%%%%%%%%%%%%%%%%%%%%%%%%%%%%%%%%%%%%%%%%%%%%%%%%%%%%%%%%%%%%%%%%%%%%%%%%%%%%%%%%
\subsection{Included Files}
\label{sec:include}

%%%%%%%%%%%%%%%%%%%%%%%%%%%%%%%%%%%%%%%%
\DescribeMacro{\childdocmain}
To use the package, add the commands
\begin{center}
\begin{tabular}{l}
|\input{childdoc.def}|\\
|\childdocmain{}|\\
\end{tabular}
\end{center}
at the very top of the main \LaTeX{} file,
in particular \emph{before} the |\documentclass| statement!
The argument of |\childdocmain| should be left empty
(but it must be present).

%%%%%%%%%%%%%%%%%%%%%%%%%%%%%%%%%%%%%%%%
\DescribeMacro{\childdocof}
Furthermore, add the commands
\begin{center}
\begin{tabular}{l}
|\input{childdoc.def}|\\
|\childdocof{|\textit{main}|}|\\
\end{tabular}
\end{center}
at the top of every child file \textit{child}
which is included by |\include{|\textit{child}|}|
from within the main file
(or at least for those files to be compiled individually).
The argument \textit{main} must be the filename of the main file.

There are a couple of
considerations in setting up the main and child documents:

%%%%%%%%%%%%%%%%%%%%%%%%%%%%%%%%%%%%%%%%
\paragraph{Restrictions.}

Please note the following restrictions:
\begin{itemize}
\item
|\childdocmain| must be called with one argument \textit{main}
to ensure compatibility with earlier version of the package.
It must either be empty (|\childdocmain{}|)
or precisely match the filename of the main file in which it is specified.
See \secref{sec:detection} for further information.
\item
The filename \textit{main} must be specified without the |.tex| extension.
\item
The filename \textit{main} is case sensitive
(even in case-insensitive file systems)
due to internal string comparison.
\item
The argument \textit{main} should be fully expanded, it cannot be a macro.
\item
Subdirectories and special characters should be avoided in filenames.
\item
The command |\childdocmain{|\textit{main}|}| must be followed by a whitespace.
It should not be followed immediately by another command
or by a comment mark `|%|'.
This is because the \TeX{} parser reads the token immediately following
the argument of |\childdocmain| and puts it
at the beginning of every child section;
however, a white\-space is ignored.
\end{itemize}

%%%%%%%%%%%%%%%%%%%%%%%%%%%%%%%%%%%%%%%%
\paragraph{Content of Main File.}

It is advisable to place all content in the child files included by |\include|.
Any output contained in the main file will appear in all child documents
unless suppressed manually;
it cannot be suppressed automatically by the |\includeonly| directive
and thus should normally be avoided.
A method to include some content in the main file
by means of conditional processing is described in \secref{sec:conditional}.

%%%%%%%%%%%%%%%%%%%%%%%%%%%%%%%%%%%%%%%%
\paragraph{Page Numbering.}

When only a part of the document is compiled,
the appropriate numbering of pages
(as well as other status parameters)
is determined from the |.aux| files.
The latter contain information from previous passes.
However this information needs to propagate through
all intermediate child documents.
Therefore the page numbering in child documents may well
be inconsistent until the complete document is compiled at least once.

A useful (if unconventional) way to always ensure a consistent
page numbering is to restart the numbering in each child document
and denote the pages by `\textit{child}|.|\textit{page}'
where \textit{child} represents the chapter/section number of the child file.
This can be achieved by the command
|\numberwithin{page}{|\textit{child}|}|
of the \textsf{amsmath} package
where \textit{child} can be |chapter| or |section|
depending on the chosen structuring.
Alternatively, one can modify the macro |\thepage| appropriately
and reset the counter |page| at the start of each child file.

%%%%%%%%%%%%%%%%%%%%%%%%%%%%%%%%%%%%%%%%%%%%%%%%%%%%%%%%%%%%%%%%%%%%%%%%%%%%%%%%
\subsection{Conditional Processing}
\label{sec:conditional}

The package provides a mechanism to compile different versions
of a document. To customise the versions further some conditional processing
can come in handy to distinguish which version is being compiled.
The package provides two macros to describe the compilation context:

%%%%%%%%%%%%%%%%%%%%%%%%%%%%%%%%%%%%%%%%
\DescribeMacro{\ifchilddoc}
The conditional |\ifchilddoc| distinguishes between the compilation of
child documents and the main document:
%
\begin{center}
|\ifchilddoc |\textit{child-code}| |[|\||else |\textit{main-code}]| \||fi|
\end{center}

%%%%%%%%%%%%%%%%%%%%%%%%%%%%%%%%%%%%%%%%
\DescribeMacro{\childdocname}
\DescribeMacro{\childdocjob}
The macro |\childdocname| contains the filename (without extension)
of the main or child file being processed.
Note that |\childdocjob| will always contain the name of the main file.

%%%%%%%%%%%%%%%%%%%%%%%%%%%%%%%%%%%%%%%%
\paragraph{Title Page.}

Conditional processing can be used to include a title or banner page
in the main document when proper precautions are taken.
Importantly, the code in the main file should ensure that the page counter
(as well as other status parameters which are stored in the |.aux| files)
takes the same value after the conditional processing.
Otherwise the page numbers may take divergent values
depending on which part is compiled.

For example, a title page could be declared by:
%
\begin{center}
\begin{tabular}{l}
|\ifchilddoc\||else|\\
|\addtocounter{page}{-1}|\\
\textit{code for title page}\\
|\newpage|\\
|\||fi|
\end{tabular}
\end{center}
%
A banner page for the child documents can be generated by:
%
\begin{center}
\begin{tabular}{l}
|\ifchilddoc|\\
|\addtocounter{page}{-1}|\\
\textit{code for banner page}\\
|\newpage|\\
|\||fi|
\end{tabular}
\end{center}
%
Here one could write a message such as:
\begin{center}
|This is the part \childdocname{} of \childdocjob{}.|
\end{center}

%%%%%%%%%%%%%%%%%%%%%%%%%%%%%%%%%%%%%%%%%%%%%%%%%%%%%%%%%%%%%%%%%%%%%%%%%%%%%%%%
\subsection{Flags}
\label{sec:flags}

The package makes it easy to generate different versions
of the main or child documents.
To this end compilation flags can be defined
and assigned different default values.
They will be particularly useful in conjunction
with the forwarding mechanism described in \secref{sec:forward}.

For example, it may be useful to have a flag |\version|
which can be set to |draft| or |final|.
The document source will contain some conditional code
depending on the value of |\version|.
Suppose further, the flag should default to |final| for the main file
and to |draft| for child files
which is a natural assignment for editing the document.
This is achieved by placing the following code
in the preamble of the main document
(below the |\childdocmain| directive):
%
\begin{center}
\begin{tabular}{l}
|\ifchilddoc|\\
|\providecommand{\version}{draft}|\\
|\||else|\\
|\providecommand{\version}{final}|\\
|\||fi|
\end{tabular}
\end{center}
%
The definition by |\providecommand| makes sure
that previous definitions are not overwritten.
Further statements |\providecommand{\version}{...}|
can thus be added before the above code to override it.

For the main file, one might add a line
(between |\childdocmain| and the above block)
%
\begin{center}
|%\ifchilddoc\||else\providecommand{\version}{draft}\||fi|
\end{center}
%
which can be uncommented to produce a draft version.
Likewise one can add a line to the very top of a child file
(above the |\childdocof{|\textit{main}|}| directive)
%
\begin{center}
|%\providecommand{\version}{final}|
\end{center}
%
which can be uncommented to produce the final version of this child document.

%%%%%%%%%%%%%%%%%%%%%%%%%%%%%%%%%%%%%%%%%%%%%%%%%%%%%%%%%%%%%%%%%%%%%%%%%%%%%%%%
\subsection{Forwarding}
\label{sec:forward}

Different versions of the main or child documents
using compilation flags as described in \secref{sec:flags}
can be (permanently) stored in different files
for convenient compilation, viewing and distribution.
To this end, the package defines a command
to pass on compilation to a different file:

%%%%%%%%%%%%%%%%%%%%%%%%%%%%%%%%%%%%%%%%
\DescribeMacro{\childdocforward}
The command |\childdocforward| redirects processing to
another source file:
%
\begin{center}
\begin{tabular}{l}
|\input{childdoc.def}|\\
|\childdocforward[|\textit{main}|]{|\textit{dest}|}|\\
\end{tabular}
\end{center}
%
The argument \textit{dest} is the destination file
(without extension).
It should be the main file or one of the child files.
Note that further \textsf{childdoc} directives
such as |\childdocof| and |\childdocforward|
in the indicated file will be processed in this form.
The optional argument \textit{main}
passes on directly to the main file \textit{main}
while pretending to compile the child \textit{dest}.
This form behaves as if \textit{dest}
issues |\childdocof{|\textit{main}|}| right away,
and no further \textsf{childdoc} directives will be processed.

%%%%%%%%%%%%%%%%%%%%%%%%%%%%%%%%%%%%%%%%
\DescribeMacro{\...prefix}
In the alternative form |\childdocforwardprefix|,
%
\begin{center}
\begin{tabular}{l}
|\input{childdoc.def}|\\
|\childdocforwardprefix[|\textit{main}|]{|\textit{prefix}|}{|\textit{dest}|}|
\end{tabular}
\end{center}
%
the destination file is determined by a pattern
depending on the current file:
To make this work, the current file must be called
`{\textit{prefix}\hspace{0.2em}\textit{suffix}}'
with \textit{prefix} matching precisely the argument.
Processing is then passed on to the file
`{\textit{dest}\hspace{0.2em}\textit{suffix}}'.
Surely, the same effect is achieved by
directly specifying the
argument `{\textit{dest}\hspace{0.2em}\textit{suffix}}'
in the first form.
However, that requires to set up a different file
for each child. With the alternative form of the command
all these files can have exactly the same content
which simplifies setting them up and maintaining them.

For example, the following file |draft.tex|
with a compilation flag |\version| as described in \secref{sec:flags}
compiles the main document as a draft:
%
\begin{center}
\begin{tabular}{l}
|\def\version{draft}|\\
|\input{childdoc.def}|\\
|\childdocforward{|\textit{main}|}|
\end{tabular}
\end{center}
%
Likewise, the following files |final|\textit{nn}|.tex|
compile the final version of the child document
|child|\textit{nn}|.tex|:
%
\begin{center}
\begin{tabular}{l}
|\def\version{final}|\\
|\input{childdoc.def}|\\
|\childdocforwardprefix{final}{child}|
\end{tabular}
\end{center}
%

Note that when several versions of a main file and/or of each child file
are to be generated, it may be convenient to set up a |Makefile| or
shell script to automatise the process.

%%%%%%%%%%%%%%%%%%%%%%%%%%%%%%%%%%%%%%%%%%%%%%%%%%%%%%%%%%%%%%%%%%%%%%%%%%%%%%%%
\subsection{Command Line Processing}
\label{sec:commandline}

The effect of redirection files can also be achieved by invoking
the \LaTeX{} compiler with a more elaborate command line.
Most conveniently this should be done as part
of a shell script or a |Makefile|.

When using \textsf{childdoc} in the main file, the following
command lines effectively perform a redirection
(note that depending on the shell being used,
backslashes may have to be doubled: `|\|' $\to$ `|\\|'):
%
\begin{center}
|... -jobname "|\textit{target}|" |\\|"|[\textit{flags}]%
|\input{childdoc.def}\childdocforward[|\textit{main}|]{|\textit{dest}|}"|
\end{center}
%
Here \textit{target} is the name of the output file,
\textit{main} is the name of the main file
and \textit{dest} is the name of the main or child file to be processed
(all filenames without extensions).
The optional argument \textit{main} can be omitted
if \textit{main} matches \textit{dest}.
Optionally, compilation \textit{flags} can be defined via |\def| commands.
This command line makes the \TeX{} engine believe
it is compiling the file \textit{target}
whose content is specified as the latter parameter.
The provided code then forwards the processing to
\textit{main} or \textit{dest} as described in \secref{sec:forward}.

%%%%%%%%%%%%%%%%%%%%%%%%%%%%%%%%%%%%%%%%%%%%%%%%%%%%%%%%%%%%%%%%%%%%%%%%%%%%%%%%
\subsection{Include by Input}
\label{sec:input}

Including child documents by |\include| has some restrictions by design.
Most notably, the content of a child document always occupies
its own set of pages; pages cannot be shared between child documents.
Usually, this behaviour makes perfect sense
because each child document contain an essential part of the document.
However, in some situations it may be desirable to compose
a document from a collection of parts
without having mandatory page breaks between then.
For this case, the package
provides a mechanism to include parts
by |\input| which can also be processed individually.
However, by construction this mechanism
requires manual handling of the content to be output.

%%%%%%%%%%%%%%%%%%%%%%%%%%%%%%%%%%%%%%%%
\DescribeMacro{\ifchilddocmanual}
The main file should be prepared as usual, see \secref{sec:include}.
However, the document body must make a distinction
between processing of an individual part and of the main document, e.g.:
%
\begin{center}
\begin{tabular}{l}
|\ifchilddocmanual|\\
|\input{\childdocname}|\\
|\||else|\\
\textit{document body with }|\input{|\textit{part}|}|\\
|\||fi|
\end{tabular}
\end{center}
%
The conditional |\ifchilddocmanual| is true whenever
a part to be included by |\input| is being compiled,
and the name of the part is stored in |\childdocname|.

%%%%%%%%%%%%%%%%%%%%%%%%%%%%%%%%%%%%%%%%
\DescribeMacro{\childdocby}
Each part to be included by |\input| should start with:
%
\begin{center}
\begin{tabular}{l}
|\input{childdoc.def}|\\
|\childdocby{|\textit{main}|}|\\
\end{tabular}
\end{center}
%
The directive |\childdocby| is similar to |\childdocof|
described in \secref{sec:include},
but the subsequent selection of content must be done manually.
To that end, both |\ifchilddoc| and |\ifchilddocmanual|
will be true upon processing of a part,
and the name of the part is stored in |\childdocname|.
Note that |\jobname| will be set to the filename of the current part
so that each part receives an individual |.aux| file
that does not interfere with the |.aux| file(s) of the main document.
This behaviour can be altered by the alternative form
|\childdocby[*]{|\textit{main}|}| (with a non-empty optional argument)
which uses the |.aux| file of the main document
by setting |\jobname| to \textit{main}.

%%%%%%%%%%%%%%%%%%%%%%%%%%%%%%%%%%%%%%%%%%%%%%%%%%%%%%%%%%%%%%%%%%%%%%%%%%%%%%%%
\subsection{Driver Development}
\label{sec:driver}

The \textsf{childdoc} mechanism can also be use for the development
of definition files such as \LaTeX{} styles or classes.
This case differs from the above setup with multiple parts
included by |\include| in that no |\includeonly| should be invoked.
This can be achieved by starting the include file
(before |\ProvidesPackage|) with:
%
\begin{center}
\begin{tabular}{l}
|\input{childdoc.def}|\\
|\childdocforward{|\textit{main}|}|\\
\end{tabular}
\end{center}
%
or alternatively with:
%
\begin{center}
\begin{tabular}{l}
|\input{childdoc.def}|\\
|\childdocby{|\textit{main}|}|\\
\end{tabular}
\end{center}
%
Both forms have slightly different effects as described above.
The main file is prepared as usual, see \secref{sec:include}.

%%%%%%%%%%%%%%%%%%%%%%%%%%%%%%%%%%%%%%%%%%%%%%%%%%%%%%%%%%%%%%%%%%%%%%%%%%%%%%%%
\subsection{Legacy Detection}
\label{sec:detection}

The directive |\childdocmain| in the main file can detect
whether the complete document or merely a child is to be compiled
even without using the directive |\childdocof|.
This method is deprecated because it is less robust
and there is no compelling reason to use it;
it is merely provided for backward compatibility
and it may be removed in future versions.

If the detection mechanism is to be used,
it is mandatory to correctly specify
the filename of the main file as the argument of |\childdocmain|:
%
\begin{center}
\begin{tabular}{l}
|\input{childdoc.def}|\\
|\childdocmain{|\textit{main}|}|\\
\end{tabular}
\end{center}
%
If |\jobname| does not match the argument \textit{main} of |\childdocmain|,
it is assumed that |\jobname| points to the child file to be compiled.
When using |\childdocmain| with the main file specified as argument,
it suffices to start a child file
with just |\input{|\textit{main}|}|
without loading of the package and using |\childdocof|.
If instead all processing is done
with the appropriate \textsf{childdoc} directives,
the argument of \textit{main} of |\childdocmain| can be empty.

An alternative version of the command line processing described
in \secref{sec:commandline} using the detection mechanism reads:
%
\begin{center}
|... -jobname "|\textit{target}|" "|[\textit{flags}]%
[|\def\jobname{|\textit{dest}|}|]|\input{|\textit{main}|}"|
\end{center}

%%%%%%%%%%%%%%%%%%%%%%%%%%%%%%%%%%%%%%%%%%%%%%%%%%%%%%%%%%%%%%%%%%%%%%%%%%%%%%%%
\subsection{Manual Code}
\label{sec:manual}

In case one cannot be certain whether the definitions file |childdoc.def|
is installed on the target \TeX{} distribution
and one prefers not to ship it,
it is conceivable to paste a few relevant commands into the sources.

To that end, drop all statements |\input{childdoc.def}|
and perform the replacements as outlined below.
Instead of |\childdocmain{|\textit{main}|}| add the following code
to the top of the main file:
%
\begin{center}
\begin{tabular}{l}
|\||ifdefined\childdocname\endinput\||fi\newif\ifchilddoc|\\
|\edef\childdocname{\scantokens\expandafter{\jobname\noexpand}}|\\
|\def\childdocmain{|\textit{main}|}\||ifx\childdocmain\childdocname\||else|\\
|\childdoctrue\includeonly{\childdocname}\let\jobname\childdocmain\||fi|\\
\end{tabular}
\end{center}
%
Instead of |\childdocof{|\textit{main}|}| just include the main file
at the top of each child file:
%
\begin{center}
|\input{|\textit{main}|}|
\end{center}
%
A simple redirection |\childdocforward{|\textit{dest}|}| is achieved by:
%
\begin{center}
|\def\jobname{|\textit{dest}|}\input{\jobname}|
\end{center}
%
The redirection with prefix
|\childdocforwardprefix[|\textit{prefix}|]{|\textit{dest}|}|
is accomplished by:
%
\begin{center}
\begin{tabular}{l}
|{\edef\jobname{\scantokens\expandafter{\jobname\noexpand}}|\\
|\def\redirectjob |\textit{prefix}|#1~~~{\gdef\jobname{|\textit{dest}|#1}}|\\
|\expandafter\redirectjob\jobname~~~}\input{\jobname}|
\end{tabular}
\end{center}

In an alternative approach,
child documents can be compiled by a specific command line
without additional code or specific definitions:
%
\begin{center}
|... -jobname "|\textit{target}|" "|[\textit{flags}]%
|\includeonly{|\textit{dest}|}\input{|\textit{main}|}"|
\end{center}
%

%%%%%%%%%%%%%%%%%%%%%%%%%%%%%%%%%%%%%%%%%%%%%%%%%%%%%%%%%%%%%%%%%%%%%%%%%%%%%%%%
%%%%%%%%%%%%%%%%%%%%%%%%%%%%%%%%%%%%%%%%%%%%%%%%%%%%%%%%%%%%%%%%%%%%%%%%%%%%%%%%
\section{Information}

%%%%%%%%%%%%%%%%%%%%%%%%%%%%%%%%%%%%%%%%%%%%%%%%%%%%%%%%%%%%%%%%%%%%%%%%%%%%%%%%
\subsection{Copyright}

Copyright \copyright{} 2017--2018 Niklas Beisert

This work may be distributed and/or modified under the
conditions of the \LaTeX{} Project Public License, either version 1.3
of this license or (at your option) any later version.
The latest version of this license is in
  \url{http://www.latex-project.org/lppl.txt}
and version 1.3 or later is part of all distributions of \LaTeX{}
version 2005/12/01 or later.

This work has the LPPL maintenance status `maintained'.

The Current Maintainer of this work is Niklas Beisert.

This work consists of the files |README.txt|, |childdoc.ins| and |childdoc.dtx|
as well as the derived files |childdoc.def|, |cdocsamp.tex|
with |cdocsch1.tex|, |cdocsch2.tex|, |cdocspt3.tex|, |cdocspt4.tex|,
|cdocsdrf.tex|, |cdocsfn1.tex|, |cdocsfn2.tex|
as well as |childdoc.pdf|.

%%%%%%%%%%%%%%%%%%%%%%%%%%%%%%%%%%%%%%%%%%%%%%%%%%%%%%%%%%%%%%%%%%%%%%%%%%%%%%%%
\subsection{Files and Installation}

The package consists of the files:
%
\begin{center}
\begin{tabular}{ll}
    |README.txt|   & readme file \\
    |childdoc.ins| & installation file \\
    |childdoc.dtx| & source file \\
    |childdoc.def| & definition file \\
    |cdocsamp.tex| & sample main file \\
    |cdocsch1.tex| & sample include file \\
    |cdocsch2.tex| & sample include file \\
    |cdocspt3.tex| & sample part file \\
    |cdocspt4.tex| & sample part file \\
    |cdocsdrf.tex| & sample redirection file \\
    |cdocsfn1.tex| & sample redirection file \\
    |cdocsfn2.tex| & sample redirection file \\
    |childdoc.pdf| & manual
\end{tabular}
\end{center}
%
The distribution consists of the files
|README.txt|, |childdoc.ins| and |childdoc.dtx|.
%
\begin{itemize}
\item
Run (pdf)\LaTeX{} on |childdoc.dtx|
to compile the manual |childdoc.pdf| (this file).
\item
Run \LaTeX{} on |childdoc.ins| to create the definitions file |childdoc.def|
and the sample |cdocsamp.tex| with include files
|cdocsch1.tex|, |cdocsch2.tex|, |cdocspt3.tex|, |cdocspt4.tex|,
|cdocsdrf.tex|, |cdocsfn1.tex|, |cdocsfn2.tex|.
Then copy the file |childdoc.def| to an appropriate directory of your \LaTeX{}
distribution, e.g.\ \textit{texmf-root}|/tex/latex/childdoc|.
\end{itemize}

%%%%%%%%%%%%%%%%%%%%%%%%%%%%%%%%%%%%%%%%%%%%%%%%%%%%%%%%%%%%%%%%%%%%%%%%%%%%%%%%
\subsection{Related CTAN Packages}

There are several other packages which offer a similar functionality:
%
\begin{itemize}
\item
The packages
\href{http://ctan.org/pkg/docmute}{\textsf{docmute}},
\href{http://ctan.org/pkg/includex}{\textsf{includex}} and
\href{http://ctan.org/pkg/standalone}{\textsf{standalone}}
provide commands to include only the document body of
a child file thus allowing both files to be compiled individually.
\item
The packages \href{http://ctan.org/pkg/subdocs}{\textsf{subdocs}}
and \href{http://ctan.org/pkg/subfiles}{\textsf{subfiles}}
provide structures in which the main and child documents can be
encapsulated and allowing them to be compiled individually.
The inclusion mechanism is different from the conventional |\include|.
\item
The package \href{http://ctan.org/pkg/combine}{\textsf{combine}}
is an elaborate solution to combine several documents into one.
\end{itemize}
%
See also the CTAN topic \href{http://ctan.org/topic/subdocs}{\textsf{subdocs}}
for further related packages.
The present package differs from the above solutions in that
a document structure constructed with the conventional |\include| mechanism
just needs two extra commands at the top of every file
such that all constituent files can be compiled individually.

%%%%%%%%%%%%%%%%%%%%%%%%%%%%%%%%%%%%%%%%%%%%%%%%%%%%%%%%%%%%%%%%%%%%%%%%%%%%%%%%
%\subsection{Feature Suggestions}
%
%The following is a list of features which may be useful for future
%versions of this package:
%%
%\begin{itemize}
%\item
%\ldots
%\end{itemize}

%%%%%%%%%%%%%%%%%%%%%%%%%%%%%%%%%%%%%%%%%%%%%%%%%%%%%%%%%%%%%%%%%%%%%%%%%%%%%%%%
\subsection{Revision History}

%%%%%%%%%%%%%%%%%%%%%%%%%%%%%%%%%%%%%%%%
\paragraph{v2.0:} 2018/12/30

\begin{itemize}
\item
immediate forward processing
\item
added |\childdocby| mechanism
\item
manual restructured
\end{itemize}

%%%%%%%%%%%%%%%%%%%%%%%%%%%%%%%%%%%%%%%%
\paragraph{v1.6:} 2018/01/17

\begin{itemize}
\item
application for development of include files
\item
corrections to manual
\end{itemize}

%%%%%%%%%%%%%%%%%%%%%%%%%%%%%%%%%%%%%%%%
\paragraph{v1.5:} 2017/05/21

\begin{itemize}
\item
more complete structuring introduced
\item
|\childdocof| introduced
\item
|\childdoc| renamed to |\childdocmain|
\item
|\childredirect| renamed to |\childdocforward| and |\childdocforwardprefix|
and functionality expanded
\end{itemize}

%%%%%%%%%%%%%%%%%%%%%%%%%%%%%%%%%%%%%%%%
\paragraph{v1.0:} 2017/04/27

\begin{itemize}
\item
manual and install package
\item
first version published on CTAN
\end{itemize}

%%%%%%%%%%%%%%%%%%%%%%%%%%%%%%%%%%%%%%%%
\paragraph{v0.6:} 2017/04/26

\begin{itemize}
\item
redirection mechanism added
\end{itemize}

%%%%%%%%%%%%%%%%%%%%%%%%%%%%%%%%%%%%%%%%
\paragraph{v0.5:} 2017/04/26

\begin{itemize}
\item
functionality in definition file
\end{itemize}


%%%%%%%%%%%%%%%%%%%%%%%%%%%%%%%%%%%%%%%%%%%%%%%%%%%%%%%%%%%%%%%%%%%%%%%%%%%%%%%%
%%%%%%%%%%%%%%%%%%%%%%%%%%%%%%%%%%%%%%%%%%%%%%%%%%%%%%%%%%%%%%%%%%%%%%%%%%%%%%%%
%%%%%%%%%%%%%%%%%%%%%%%%%%%%%%%%%%%%%%%%%%%%%%%%%%%%%%%%%%%%%%%%%%%%%%%%%%%%%%%%
\appendix

\settowidth\MacroIndent{\rmfamily\scriptsize 000\ }

 \DocInput{childdoc.dtx}

\end{document}
%</driver>
% \fi
%
% %%%%%%%%%%%%%%%%%%%%%%%%%%%%%%%%%%%%%%%%%%%%%%%%%%%%%%%%%%%%%%%%%%%%%%%%%%%%%%
% %%%%%%%%%%%%%%%%%%%%%%%%%%%%%%%%%%%%%%%%%%%%%%%%%%%%%%%%%%%%%%%%%%%%%%%%%%%%%%
% \section{Sample}
%\iffalse
%<*samplemain>
%\fi
%
% The following presents a sample document
% with two chapters, two parts, a title page,
% a compile flag as well as three forwarding files to set the flag.
% It consists of eight |.tex| files:
% \begin{center}
% \begin{tabular}{ll}
% |cdocsamp.tex|&main file\\
% |cdocsch1.tex|&include file for chapter 1\\
% |cdocsch2.tex|&include file for chapter 2\\
% |cdocspt3.tex|&include file for part 3\\
% |cdocspt4.tex|&include file for part 4\\
% |cdocsdrf.tex|&forwarding file for main file in draft mode\\
% |cdocsfi1.tex|&forwarding file for final version of chapter 1\\
% |cdocsfi2.tex|&forwarding file for final version of chapter 2\\
% \end{tabular}
% \end{center}
% Each of the eight files can be compiled directly by the \LaTeX{} compiler.
%
% %%%%%%%%%%%%%%%%%%%%%%%%%%%%%%%%%%%%%%
% \paragraph{Main File.}
%
% The main file is called |cdocsamp.tex|.
%
% Load the \textsf{childdoc} definitions and
% declare the filename for the main document:
%    \begin{macrocode}
\input{childdoc.def}
\childdocmain{}
%    \end{macrocode}

% Optional override for |\version| flag:
%    \begin{macrocode}
%%\ifchilddoc\else\providecommand{\version}{draft}\fi
%    \end{macrocode}

% Define the default values for the |\version| flag
% (|final| for the main file and |draft| for childs):
%    \begin{macrocode}
\ifchilddoc
\providecommand{\version}{draft}
\else
\providecommand{\version}{final}
\fi
%    \end{macrocode}

% Load the standard document class:
%    \begin{macrocode}
\documentclass[12pt]{article}
%    \end{macrocode}

% Start the document body:
%    \begin{macrocode}
\begin{document}
%    \end{macrocode}

% Declare a title page.
% Print title, part of document being processed and version flag:
%    \begin{macrocode}
\addtocounter{page}{-1}
\begin{center}
{\LARGE\bfseries{}childdoc example\par}
\vspace{1cm}
\ifchilddoc
\ifchilddocmanual part\else chapter\fi:
`\childdocname' of `\childdocjob'\par
\else
main document: `\childdocjob'\par
\fi
version: \version\par
\end{center}
\newpage
%    \end{macrocode}

% Manually include selected file,
% otherwise process as usual:
%    \begin{macrocode}
\ifchilddocmanual
\section*{part `\childdocname'}
\input{\childdocname}
\else
%    \end{macrocode}

% Include the two chapters:
%    \begin{macrocode}
\include{cdocsch1}
\include{cdocsch2}
%    \end{macrocode}

% Include the two parts unless only chapters should be displayed:
%    \begin{macrocode}
\ifchilddoc\else
\section{part three}
\input{cdocspt3}
\section{part four}
\input{cdocspt4}
\fi
%    \end{macrocode}

% Process as usual until here:
%    \begin{macrocode}
\fi
%    \end{macrocode}

% End of document body:
%    \begin{macrocode}
\end{document}
%    \end{macrocode}
%\iffalse
%</samplemain>
%\fi
%
% %%%%%%%%%%%%%%%%%%%%%%%%%%%%%%%%%%%%%%
% \paragraph{Chapter Include Files.}
%
% The include files are called |cdocsch1.tex| and |cdocsch2.tex|.
%
%\iffalse
%<*samplechap1|samplechap2>
%\fi

% Optional override for |\version| flag:
%    \begin{macrocode}
%%\providecommand{\version}{final}
%    \end{macrocode}

% Include the main document:
%    \begin{macrocode}
\input{childdoc.def}
\childdocof{cdocsamp}
%    \end{macrocode}

%\iffalse
%</samplechap1|samplechap2>
%\fi
%
%\iffalse
%<*samplechap1>
%\fi
% Some text for chapter 1:
%    \begin{macrocode}
\section{one}
some text in chapter one
%    \end{macrocode}

%\iffalse
%</samplechap1>
%\fi
% Some text for chapter 2:
%\iffalse
%<*samplechap2>
%\fi
%    \begin{macrocode}
\section{two}
more text in chapter two
%    \end{macrocode}

%\iffalse
%</samplechap2>
%\fi
%
% %%%%%%%%%%%%%%%%%%%%%%%%%%%%%%%%%%%%%%
% \paragraph{Part Include Files.}
%
% The include files are called |cdocspt3.tex| and |cdocspt4.tex|.
%
%\iffalse
%<*samplepart3|samplepart4>
%\fi

% Optional override for |\version| flag:
%    \begin{macrocode}
%%\providecommand{\version}{final}
%    \end{macrocode}

% Include the main document:
%    \begin{macrocode}
\input{childdoc.def}
\childdocby{cdocsamp}
%    \end{macrocode}

%\iffalse
%</samplepart3|samplepart4>
%\fi
%
%\iffalse
%<*samplepart3>
%\fi
% Some text for part 3:
%    \begin{macrocode}
some text in part three
%    \end{macrocode}

%\iffalse
%</samplepart3>
%\fi
% Some text for part 4:
%\iffalse
%<*samplepart4>
%\fi
%    \begin{macrocode}
more text in part four
%    \end{macrocode}

%\iffalse
%</samplepart4>
%\fi
%
% %%%%%%%%%%%%%%%%%%%%%%%%%%%%%%%%%%%%%%
% \paragraph{Forwarding for a Complete Draft.}
%
% The following forwarding file |cdocsdrf.tex|
% compiles the main document in draft mode:
%\iffalse
%<*sampledraft>
%\fi
%    \begin{macrocode}
\def\version{draft}
\input{childdoc.def}
\childdocforward{cdocsamp}
%    \end{macrocode}

%\iffalse
%</sampledraft>
%\fi
%
% %%%%%%%%%%%%%%%%%%%%%%%%%%%%%%%%%%%%%%
% \paragraph{Forwarding for Final Version of the Chapters.}
%
% The following forwarding files |cdocsfn1.tex| and |cdocsfn2.tex|
% (with identical content)
% compile the final versions of the child documents
% |cdocsch1.tex| and |cdocsch2.tex|, respectively:
%\iffalse
%<*samplefinal>
%\fi
%    \begin{macrocode}
\def\version{final}
\input{childdoc.def}
\childdocforwardprefix[cdocsamp]{cdocsfn}{cdocsch}
%    \end{macrocode}

%\iffalse
%</samplefinal>
%\fi
%
% %%%%%%%%%%%%%%%%%%%%%%%%%%%%%%%%%%%%%%
% \paragraph{Command Line Processing.}
%
% The following three command lines generate the output files
% |cdocscld|, |cdocscl1| and |cdocscl2|
% which should be identical to
% |cdocsdrf|, |cdocsch1| and |cdocsfn2|, respectively:
% \begin{center}
% \begin{tabular}{l}
% |latex -jobname cdocscld \|\\
% |  "\def\version{draft}\input{childdoc.def}\childdocforward{cdocsamp}"|\\
% |latex -jobname cdocscl1 \|\\
% |  "\input{childdoc.def}\childdocforward[cdocsamp]{cdocsch1}"|\\
% |latex -jobname cdocscl2 \|\\
% |  "\def\version{final}\input{childdoc.def}\childdocforward{cdocsch2}"|
% \end{tabular}
% \end{center}
% Note that the trailing backslash on each first line
% merely continues the input to the second line
% (for convenient cut ant paste).
% Furthermore, the command |latex| can be replaced by any
% of its alternative versions such as |pdflatex|.
%
% %%%%%%%%%%%%%%%%%%%%%%%%%%%%%%%%%%%%%%%%%%%%%%%%%%%%%%%%%%%%%%%%%%%%%%%%%%%%%%
% %%%%%%%%%%%%%%%%%%%%%%%%%%%%%%%%%%%%%%%%%%%%%%%%%%%%%%%%%%%%%%%%%%%%%%%%%%%%%%
% \section{Implementation}
%\iffalse
%<*package>
%\fi
%
% This section describes the definitions file |childdoc.def|.

% The definitions cannot be loaded using |\usepackage| or |\RequirePackage|
% which has a mechanism to prevent loading a style file more than once.
% When loading the definitions by means of |\input|
% multiple instances have to be prevented manually:
%\iffalse
%This code needs to be before the `\ProvidesFile' directive
%which is defined at the beginning of this file.
%Therefore it is also placed there and commented out here.
%</package>
%<*discard>
%\fi
%    \begin{macrocode}
\ifdefined\childdocmain\endinput\fi
%    \end{macrocode}
%\iffalse
%</discard>
%<*package>
%\fi
%
% \macro{\ifchilddoc}
% \macro{\ifchilddocmanual}
% The conditional |\ifchilddoc| tells whether a
% child (true) or main (false) document is being compiled.
% The conditional |\ifchilddocmanual| tells whether
% the |\includeonly| mechanism is used (false) or
% the selection of child files must be performed manually (true).
% The definitions initialise to false:
%    \begin{macrocode}
\newif\ifchilddoc
\newif\ifchilddocmanual
%    \end{macrocode}

% \macro{\childdocname}
% \macro{\childdocjob}
% The macro |\childdocname| stores the name of the main document
% to be compiled. The macro |\childdocjob| stores the name of
% the document on which the \LaTeX{} compiler was originally invoked.
% The content of |\jobname| cannot be compared
% to filenames specified in the source due to different catcodes.
% The following code rescans |\jobname|, stores the result
% in |\childdocname| and saves a copy in |\childdocjob|:
%    \begin{macrocode}
\edef\childdocname{\scantokens\expandafter{\jobname\noexpand}}
\let\childdocjob\childdocname
%    \end{macrocode}

% \macro{\childdocdisable}
% The macro |\childdocdisable| prevents the main file
% from being processed more than once.
% At this stage, the main document command |\childdocmain|
% is assumed to be called once again where it should do nothing.
% Any subsequent call to it should prevent
% a secondary processing of the main document
% It overwrites the forwarding commands
% |\childdocof| and |\childdocforward|
% with empty macros to prevent further inclusions of the main document:
%    \begin{macrocode}
\newcommand{\childdocdisable}
{
  \renewcommand{\childdocmain}[1]{\renewcommand{\childdocmain}[1]{\endinput}}
  \renewcommand{\childdocof}[1]{}
  \renewcommand{\childdocby}[2][]{}
  \renewcommand{\childdocforward}[2][]{}
  \renewcommand{\childdocdisable}{}
}
%    \end{macrocode}

% \macro{\childdocmain}
% The macro |\childdocmain| is to be called at the top of the main file
% with nothing or the main filename (without extension) as argument.
% First, it breaks loops.
% If the argument is not empty and does not match |\childdocname|
% (which is set by the first inclusion of |childdoc.def|),
% |\ifchilddoc| is set to true, |\includeonly| is applied to the child file
% and |\jobname| is set to the main file
% (for proper handling of |.aux| files):
%    \begin{macrocode}
\newcommand{\childdocmain}[1]
{
  \childdocdisable\childdocmain{}
  \if?#1?\else
    \begingroup
      \def\childdoctmp{#1}
      \ifx\childdoctmp\childdocname
        \def\childdoctmp{}
      \else
        \def\childdoctmp
        {
          \childdoctrue
          \includeonly{\childdocname}
          \def\childdocjob{#1}
          \def\jobname{#1}
        }
      \fi
      \expandafter
    \endgroup
    \childdoctmp
  \fi
}
%    \end{macrocode}

% \macro{\childdocof}
% The command |\childdocof| redirects
% compilation to the main file |#1|.
%    \begin{macrocode}
\newcommand{\childdocof}[1]
{
  \childdocdisable
  \childdoctrue
  \includeonly{\childdocname}
  \def\jobname{#1}
  \def\childdocjob{#1}
  \input{#1}
}
%    \end{macrocode}

% \macro{\childdocby}
% The command |\childdocby| ....
%    \begin{macrocode}
\newcommand{\childdocby}[2][]
{
  \childdocdisable
  \childdoctrue
  \childdocmanualtrue
  \if?#1?\else
    \def\jobname{#2}
  \fi
  \def\childdocjob{#2}
  \input{#2}
  \endinput
}
%    \end{macrocode}

% \macro{\childdocforward}
% The command |\childdocforward| redirects
% compilation to the main file or
% (if the optional argument is given) a child file.
% Parameters are set as if the main file
% or a child file starting with |\childdocof| was compiled.
% Then compilation is handed over to the main file:
%    \begin{macrocode}
\newcommand{\childdocforward}[2][]
{
  \begingroup
    \if?#1?
      \def\childdoctmp
      {
        \def\childdocname{#2}
        \def\childdocjob{#2}
        \def\jobname{#2}
        \input{#2}
        \endinput
      }
    \else
      \def\childdoctmp
      {
        \childdocdisable
        \def\childdocname{#2}
        \childdoctrue
        \includeonly{#2}
        \def\childdocjob{#1}
        \def\jobname{#1}
        \input{#1}
        \endinput
      }
    \fi
    \expandafter
  \endgroup
  \childdoctmp
}
%    \end{macrocode}

% \macro{\childdocforwardprefix}
% The command |\childdocforwardprefix| redirects
% compilation to the main or a child file by means of a pattern.
% The prefix |#1| in the current filename is replaced by |#2|
% and the suffix of the current filename is kept
% (it is assumed that the filename does not contain the substring `|~~~|'
% which is used as a delimiter).
% Compilation is handed over to the new file by |\childdocforward|:
%    \begin{macrocode}
\newcommand{\childdocforwardprefix}[3][]
{
  \begingroup
    \def\childdocextract #2##1~~~{\def\childdoctmp{\childdocforward[#1]{#3##1}}}
    \expandafter\childdocextract\childdocname~~~
    \expandafter
  \endgroup
  \childdoctmp
}
%    \end{macrocode}

% \macro{\childdoc}
% The deprecated macro |\childdoc| is a legacy version of |\childdocmain|:
%    \begin{macrocode}
\newcommand{\childdoc}{\childdocmain}
%    \end{macrocode}

% \macro{\childdocredirect}
% The deprecated macro |\childdocredirect| is a legacy version
% of |\childdocforward| and |\childdocforwardprefix|:
%    \begin{macrocode}
\newcommand{\childdocredirect}[2][]
{
  \begingroup
    \if?#1?
      \def\childdoctmp{\childdocforward{#2}}
    \else
      \def\childdoctmp{\childdocforwardprefix{#1}{#2}}
    \fi
    \expandafter
  \endgroup
  \childdoctmp
}
%    \end{macrocode}

%\iffalse
%</package>
%\fi
%
\endinput
|\\
|\childdocby{|\textit{main}|}|\\
\end{tabular}
\end{center}
%
Both forms have slightly different effects as described above.
The main file is prepared as usual, see \secref{sec:include}.

%%%%%%%%%%%%%%%%%%%%%%%%%%%%%%%%%%%%%%%%%%%%%%%%%%%%%%%%%%%%%%%%%%%%%%%%%%%%%%%%
\subsection{Legacy Detection}
\label{sec:detection}

The directive |\childdocmain| in the main file can detect
whether the complete document or merely a child is to be compiled
even without using the directive |\childdocof|.
This method is deprecated because it is less robust
and there is no compelling reason to use it;
it is merely provided for backward compatibility
and it may be removed in future versions.

If the detection mechanism is to be used,
it is mandatory to correctly specify
the filename of the main file as the argument of |\childdocmain|:
%
\begin{center}
\begin{tabular}{l}
|% \iffalse
%
% childdoc.dtx Copyright (C) 2017-2018 Niklas Beisert
%
% This work may be distributed and/or modified under the
% conditions of the LaTeX Project Public License, either version 1.3
% of this license or (at your option) any later version.
% The latest version of this license is in
%   http://www.latex-project.org/lppl.txt
% and version 1.3 or later is part of all distributions of LaTeX
% version 2005/12/01 or later.
%
% This work has the LPPL maintenance status `maintained'.
%
% The Current Maintainer of this work is Niklas Beisert.
%
% This work consists of the files childdoc.dtx and childdoc.ins
% and the derived files childdoc.def and cdocsamp.tex with
% cdocsch1.tex, cdocsch2.tex, cdocsdrf.tex, cdocsfn1.tex, cdocsfn2.tex.
%
%<package>\ifdefined\childdocmain\endinput\fi
%<package>\ProvidesFile{childdoc.def}[2018/12/30 v2.0 child document driver]
%<samplemain>\ProvidesFile{cdocsamp.tex}[2018/12/30 v2.0 sample for childdoc]
%<*driver>
%\ProvidesFile{childdoc.drv}[2018/12/30 v2.0 childdoc reference manual file]
\PassOptionsToClass{10pt,a4paper}{article}
\documentclass{ltxdoc}

\usepackage[margin=35mm]{geometry}
\usepackage{hyperref}
\usepackage{hyperxmp}
\usepackage[usenames]{color}

\hypersetup{colorlinks=true}
\hypersetup{pdfstartview=FitH}
\hypersetup{pdfpagemode=UseNone}
\hypersetup{pdfsource={}}
\hypersetup{pdflang={en-UK}}
\hypersetup{pdfcopyright={Copyright 2017-2018 Niklas Beisert.
  This work may be distributed and/or modified under the
  conditions of the LaTeX Project Public License, either version 1.3
  of this license or (at your option) any later version.}}
\hypersetup{pdflicenseurl={http://www.latex-project.org/lppl.txt}}
\hypersetup{pdfcontactaddress={ETH Zurich, ITP, HIT K,
  Wolfgang-Pauli-Strasse 27}}
\hypersetup{pdfcontactpostcode={8093}}
\hypersetup{pdfcontactcity={Zurich}}
\hypersetup{pdfcontactcountry={Switzerland}}
\hypersetup{pdfcontactemail={nbeisert@itp.phys.ethz.ch}}
\hypersetup{pdfcontacturl={http://people.phys.ethz.ch/\xmptilde nbeisert/}}

\newcommand{\secref}[1]{\hyperref[#1]{section \ref*{#1}}}

\parskip1ex
\parindent0pt
\let\olditemize\itemize
\def\itemize{\olditemize\parskip0pt}

\begin{document}

\title{The \textsf{childdoc} Package}
\hypersetup{pdftitle={The childdoc Package}}
\author{Niklas Beisert\\[2ex]
  Institut f\"ur Theoretische Physik\\
  Eidgen\"ossische Technische Hochschule Z\"urich\\
  Wolfgang-Pauli-Strasse 27, 8093 Z\"urich, Switzerland\\[1ex]
  \href{mailto:nbeisert@itp.phys.ethz.ch}
  {\texttt{nbeisert@itp.phys.ethz.ch}}}
\hypersetup{pdfauthor={Niklas Beisert}}
\hypersetup{pdfsubject={Manual for the LaTeX2e Package childdoc}}
\date{30 December 2018, \textsf{v2.0}}
\maketitle

\begin{abstract}\noindent
\textsf{childdoc} is a \LaTeXe{} package
that enables the direct compilation
of document sections included by |\include|
to individual files.
\end{abstract}

\begingroup
\parskip0ex
\tableofcontents
\endgroup

%%%%%%%%%%%%%%%%%%%%%%%%%%%%%%%%%%%%%%%%%%%%%%%%%%%%%%%%%%%%%%%%%%%%%%%%%%%%%%%%
%%%%%%%%%%%%%%%%%%%%%%%%%%%%%%%%%%%%%%%%%%%%%%%%%%%%%%%%%%%%%%%%%%%%%%%%%%%%%%%%
\section{Introduction}

\LaTeX{} provides a mechanism to structure a large document (such as a book)
into a main file and several child files (containing the chapters)
using the |\include| command.
This mechanism is beneficial for documents
which span hundreds of pages in order to
make the source file(s) more manageable.
Moreover, compilation can be restricted to
selected child files by means of the |\includeonly| command.
The latter feature can be used to reduce the compilation time while editing
(this was significantly more useful in the earlier days of \LaTeX{})
or to generate a smaller document which is easier to navigate.
Another application of |\includeonly| is to generate
documents consisting of selected parts of the complete document.

However, there are a few drawbacks of the plain |\include| mechanism:
\begin{itemize}
\item
The child files cannot be compiled on their own,
they can only be compiled via the main file.
A naive editing environment
(such as a text editor with an option
to have the current file processed by \LaTeX)
may require one to switch to the main file before compiling;
attempting to compile the child file produces errors.
\item
The main file must be modified (each time)
to adjust the |\includeonly| command
to the present needs. This easily leaves the main file in a messy state.
\item
The generated document will always carry the filename
of the main document. This is inconvenient if
several child files are to be compiled and
to be kept for distribution.
\end{itemize}

The present package provides a simple interface
to make child files individually compilable by \LaTeX{}.
Compiling a child file then has the same effect as compiling
the main file with an |\includeonly| command
to select the appropriate child.
Moreover the generated document will carry the name of the child
rather than the main file.
This resolves all three above issues.

This feature is meant to make the editing of books,
thesis documents and lecture notes somewhat more convenient.
However, the package can also be used efficiently for
composing a series of documents (such as exercise sheets)
which are typically distributed individually.
It then assists the author in generating the individual documents
(potentially in different versions)
as well as a document containing the collected series.
Another application is in developing style files
or other kinds of included material
where compilation of the style file could redirect
to a sample or test file.

%%%%%%%%%%%%%%%%%%%%%%%%%%%%%%%%%%%%%%%%%%%%%%%%%%%%%%%%%%%%%%%%%%%%%%%%%%%%%%%%
%%%%%%%%%%%%%%%%%%%%%%%%%%%%%%%%%%%%%%%%%%%%%%%%%%%%%%%%%%%%%%%%%%%%%%%%%%%%%%%%
\section{Usage}

First of all, the package \textsf{childdoc} is \emph{not} a standard
\LaTeXe{} |.sty| style file! Therefore it needs to be invoked in
a non-standard way.

%%%%%%%%%%%%%%%%%%%%%%%%%%%%%%%%%%%%%%%%%%%%%%%%%%%%%%%%%%%%%%%%%%%%%%%%%%%%%%%%
\subsection{Included Files}
\label{sec:include}

%%%%%%%%%%%%%%%%%%%%%%%%%%%%%%%%%%%%%%%%
\DescribeMacro{\childdocmain}
To use the package, add the commands
\begin{center}
\begin{tabular}{l}
|\input{childdoc.def}|\\
|\childdocmain{}|\\
\end{tabular}
\end{center}
at the very top of the main \LaTeX{} file,
in particular \emph{before} the |\documentclass| statement!
The argument of |\childdocmain| should be left empty
(but it must be present).

%%%%%%%%%%%%%%%%%%%%%%%%%%%%%%%%%%%%%%%%
\DescribeMacro{\childdocof}
Furthermore, add the commands
\begin{center}
\begin{tabular}{l}
|\input{childdoc.def}|\\
|\childdocof{|\textit{main}|}|\\
\end{tabular}
\end{center}
at the top of every child file \textit{child}
which is included by |\include{|\textit{child}|}|
from within the main file
(or at least for those files to be compiled individually).
The argument \textit{main} must be the filename of the main file.

There are a couple of
considerations in setting up the main and child documents:

%%%%%%%%%%%%%%%%%%%%%%%%%%%%%%%%%%%%%%%%
\paragraph{Restrictions.}

Please note the following restrictions:
\begin{itemize}
\item
|\childdocmain| must be called with one argument \textit{main}
to ensure compatibility with earlier version of the package.
It must either be empty (|\childdocmain{}|)
or precisely match the filename of the main file in which it is specified.
See \secref{sec:detection} for further information.
\item
The filename \textit{main} must be specified without the |.tex| extension.
\item
The filename \textit{main} is case sensitive
(even in case-insensitive file systems)
due to internal string comparison.
\item
The argument \textit{main} should be fully expanded, it cannot be a macro.
\item
Subdirectories and special characters should be avoided in filenames.
\item
The command |\childdocmain{|\textit{main}|}| must be followed by a whitespace.
It should not be followed immediately by another command
or by a comment mark `|%|'.
This is because the \TeX{} parser reads the token immediately following
the argument of |\childdocmain| and puts it
at the beginning of every child section;
however, a white\-space is ignored.
\end{itemize}

%%%%%%%%%%%%%%%%%%%%%%%%%%%%%%%%%%%%%%%%
\paragraph{Content of Main File.}

It is advisable to place all content in the child files included by |\include|.
Any output contained in the main file will appear in all child documents
unless suppressed manually;
it cannot be suppressed automatically by the |\includeonly| directive
and thus should normally be avoided.
A method to include some content in the main file
by means of conditional processing is described in \secref{sec:conditional}.

%%%%%%%%%%%%%%%%%%%%%%%%%%%%%%%%%%%%%%%%
\paragraph{Page Numbering.}

When only a part of the document is compiled,
the appropriate numbering of pages
(as well as other status parameters)
is determined from the |.aux| files.
The latter contain information from previous passes.
However this information needs to propagate through
all intermediate child documents.
Therefore the page numbering in child documents may well
be inconsistent until the complete document is compiled at least once.

A useful (if unconventional) way to always ensure a consistent
page numbering is to restart the numbering in each child document
and denote the pages by `\textit{child}|.|\textit{page}'
where \textit{child} represents the chapter/section number of the child file.
This can be achieved by the command
|\numberwithin{page}{|\textit{child}|}|
of the \textsf{amsmath} package
where \textit{child} can be |chapter| or |section|
depending on the chosen structuring.
Alternatively, one can modify the macro |\thepage| appropriately
and reset the counter |page| at the start of each child file.

%%%%%%%%%%%%%%%%%%%%%%%%%%%%%%%%%%%%%%%%%%%%%%%%%%%%%%%%%%%%%%%%%%%%%%%%%%%%%%%%
\subsection{Conditional Processing}
\label{sec:conditional}

The package provides a mechanism to compile different versions
of a document. To customise the versions further some conditional processing
can come in handy to distinguish which version is being compiled.
The package provides two macros to describe the compilation context:

%%%%%%%%%%%%%%%%%%%%%%%%%%%%%%%%%%%%%%%%
\DescribeMacro{\ifchilddoc}
The conditional |\ifchilddoc| distinguishes between the compilation of
child documents and the main document:
%
\begin{center}
|\ifchilddoc |\textit{child-code}| |[|\||else |\textit{main-code}]| \||fi|
\end{center}

%%%%%%%%%%%%%%%%%%%%%%%%%%%%%%%%%%%%%%%%
\DescribeMacro{\childdocname}
\DescribeMacro{\childdocjob}
The macro |\childdocname| contains the filename (without extension)
of the main or child file being processed.
Note that |\childdocjob| will always contain the name of the main file.

%%%%%%%%%%%%%%%%%%%%%%%%%%%%%%%%%%%%%%%%
\paragraph{Title Page.}

Conditional processing can be used to include a title or banner page
in the main document when proper precautions are taken.
Importantly, the code in the main file should ensure that the page counter
(as well as other status parameters which are stored in the |.aux| files)
takes the same value after the conditional processing.
Otherwise the page numbers may take divergent values
depending on which part is compiled.

For example, a title page could be declared by:
%
\begin{center}
\begin{tabular}{l}
|\ifchilddoc\||else|\\
|\addtocounter{page}{-1}|\\
\textit{code for title page}\\
|\newpage|\\
|\||fi|
\end{tabular}
\end{center}
%
A banner page for the child documents can be generated by:
%
\begin{center}
\begin{tabular}{l}
|\ifchilddoc|\\
|\addtocounter{page}{-1}|\\
\textit{code for banner page}\\
|\newpage|\\
|\||fi|
\end{tabular}
\end{center}
%
Here one could write a message such as:
\begin{center}
|This is the part \childdocname{} of \childdocjob{}.|
\end{center}

%%%%%%%%%%%%%%%%%%%%%%%%%%%%%%%%%%%%%%%%%%%%%%%%%%%%%%%%%%%%%%%%%%%%%%%%%%%%%%%%
\subsection{Flags}
\label{sec:flags}

The package makes it easy to generate different versions
of the main or child documents.
To this end compilation flags can be defined
and assigned different default values.
They will be particularly useful in conjunction
with the forwarding mechanism described in \secref{sec:forward}.

For example, it may be useful to have a flag |\version|
which can be set to |draft| or |final|.
The document source will contain some conditional code
depending on the value of |\version|.
Suppose further, the flag should default to |final| for the main file
and to |draft| for child files
which is a natural assignment for editing the document.
This is achieved by placing the following code
in the preamble of the main document
(below the |\childdocmain| directive):
%
\begin{center}
\begin{tabular}{l}
|\ifchilddoc|\\
|\providecommand{\version}{draft}|\\
|\||else|\\
|\providecommand{\version}{final}|\\
|\||fi|
\end{tabular}
\end{center}
%
The definition by |\providecommand| makes sure
that previous definitions are not overwritten.
Further statements |\providecommand{\version}{...}|
can thus be added before the above code to override it.

For the main file, one might add a line
(between |\childdocmain| and the above block)
%
\begin{center}
|%\ifchilddoc\||else\providecommand{\version}{draft}\||fi|
\end{center}
%
which can be uncommented to produce a draft version.
Likewise one can add a line to the very top of a child file
(above the |\childdocof{|\textit{main}|}| directive)
%
\begin{center}
|%\providecommand{\version}{final}|
\end{center}
%
which can be uncommented to produce the final version of this child document.

%%%%%%%%%%%%%%%%%%%%%%%%%%%%%%%%%%%%%%%%%%%%%%%%%%%%%%%%%%%%%%%%%%%%%%%%%%%%%%%%
\subsection{Forwarding}
\label{sec:forward}

Different versions of the main or child documents
using compilation flags as described in \secref{sec:flags}
can be (permanently) stored in different files
for convenient compilation, viewing and distribution.
To this end, the package defines a command
to pass on compilation to a different file:

%%%%%%%%%%%%%%%%%%%%%%%%%%%%%%%%%%%%%%%%
\DescribeMacro{\childdocforward}
The command |\childdocforward| redirects processing to
another source file:
%
\begin{center}
\begin{tabular}{l}
|\input{childdoc.def}|\\
|\childdocforward[|\textit{main}|]{|\textit{dest}|}|\\
\end{tabular}
\end{center}
%
The argument \textit{dest} is the destination file
(without extension).
It should be the main file or one of the child files.
Note that further \textsf{childdoc} directives
such as |\childdocof| and |\childdocforward|
in the indicated file will be processed in this form.
The optional argument \textit{main}
passes on directly to the main file \textit{main}
while pretending to compile the child \textit{dest}.
This form behaves as if \textit{dest}
issues |\childdocof{|\textit{main}|}| right away,
and no further \textsf{childdoc} directives will be processed.

%%%%%%%%%%%%%%%%%%%%%%%%%%%%%%%%%%%%%%%%
\DescribeMacro{\...prefix}
In the alternative form |\childdocforwardprefix|,
%
\begin{center}
\begin{tabular}{l}
|\input{childdoc.def}|\\
|\childdocforwardprefix[|\textit{main}|]{|\textit{prefix}|}{|\textit{dest}|}|
\end{tabular}
\end{center}
%
the destination file is determined by a pattern
depending on the current file:
To make this work, the current file must be called
`{\textit{prefix}\hspace{0.2em}\textit{suffix}}'
with \textit{prefix} matching precisely the argument.
Processing is then passed on to the file
`{\textit{dest}\hspace{0.2em}\textit{suffix}}'.
Surely, the same effect is achieved by
directly specifying the
argument `{\textit{dest}\hspace{0.2em}\textit{suffix}}'
in the first form.
However, that requires to set up a different file
for each child. With the alternative form of the command
all these files can have exactly the same content
which simplifies setting them up and maintaining them.

For example, the following file |draft.tex|
with a compilation flag |\version| as described in \secref{sec:flags}
compiles the main document as a draft:
%
\begin{center}
\begin{tabular}{l}
|\def\version{draft}|\\
|\input{childdoc.def}|\\
|\childdocforward{|\textit{main}|}|
\end{tabular}
\end{center}
%
Likewise, the following files |final|\textit{nn}|.tex|
compile the final version of the child document
|child|\textit{nn}|.tex|:
%
\begin{center}
\begin{tabular}{l}
|\def\version{final}|\\
|\input{childdoc.def}|\\
|\childdocforwardprefix{final}{child}|
\end{tabular}
\end{center}
%

Note that when several versions of a main file and/or of each child file
are to be generated, it may be convenient to set up a |Makefile| or
shell script to automatise the process.

%%%%%%%%%%%%%%%%%%%%%%%%%%%%%%%%%%%%%%%%%%%%%%%%%%%%%%%%%%%%%%%%%%%%%%%%%%%%%%%%
\subsection{Command Line Processing}
\label{sec:commandline}

The effect of redirection files can also be achieved by invoking
the \LaTeX{} compiler with a more elaborate command line.
Most conveniently this should be done as part
of a shell script or a |Makefile|.

When using \textsf{childdoc} in the main file, the following
command lines effectively perform a redirection
(note that depending on the shell being used,
backslashes may have to be doubled: `|\|' $\to$ `|\\|'):
%
\begin{center}
|... -jobname "|\textit{target}|" |\\|"|[\textit{flags}]%
|\input{childdoc.def}\childdocforward[|\textit{main}|]{|\textit{dest}|}"|
\end{center}
%
Here \textit{target} is the name of the output file,
\textit{main} is the name of the main file
and \textit{dest} is the name of the main or child file to be processed
(all filenames without extensions).
The optional argument \textit{main} can be omitted
if \textit{main} matches \textit{dest}.
Optionally, compilation \textit{flags} can be defined via |\def| commands.
This command line makes the \TeX{} engine believe
it is compiling the file \textit{target}
whose content is specified as the latter parameter.
The provided code then forwards the processing to
\textit{main} or \textit{dest} as described in \secref{sec:forward}.

%%%%%%%%%%%%%%%%%%%%%%%%%%%%%%%%%%%%%%%%%%%%%%%%%%%%%%%%%%%%%%%%%%%%%%%%%%%%%%%%
\subsection{Include by Input}
\label{sec:input}

Including child documents by |\include| has some restrictions by design.
Most notably, the content of a child document always occupies
its own set of pages; pages cannot be shared between child documents.
Usually, this behaviour makes perfect sense
because each child document contain an essential part of the document.
However, in some situations it may be desirable to compose
a document from a collection of parts
without having mandatory page breaks between then.
For this case, the package
provides a mechanism to include parts
by |\input| which can also be processed individually.
However, by construction this mechanism
requires manual handling of the content to be output.

%%%%%%%%%%%%%%%%%%%%%%%%%%%%%%%%%%%%%%%%
\DescribeMacro{\ifchilddocmanual}
The main file should be prepared as usual, see \secref{sec:include}.
However, the document body must make a distinction
between processing of an individual part and of the main document, e.g.:
%
\begin{center}
\begin{tabular}{l}
|\ifchilddocmanual|\\
|\input{\childdocname}|\\
|\||else|\\
\textit{document body with }|\input{|\textit{part}|}|\\
|\||fi|
\end{tabular}
\end{center}
%
The conditional |\ifchilddocmanual| is true whenever
a part to be included by |\input| is being compiled,
and the name of the part is stored in |\childdocname|.

%%%%%%%%%%%%%%%%%%%%%%%%%%%%%%%%%%%%%%%%
\DescribeMacro{\childdocby}
Each part to be included by |\input| should start with:
%
\begin{center}
\begin{tabular}{l}
|\input{childdoc.def}|\\
|\childdocby{|\textit{main}|}|\\
\end{tabular}
\end{center}
%
The directive |\childdocby| is similar to |\childdocof|
described in \secref{sec:include},
but the subsequent selection of content must be done manually.
To that end, both |\ifchilddoc| and |\ifchilddocmanual|
will be true upon processing of a part,
and the name of the part is stored in |\childdocname|.
Note that |\jobname| will be set to the filename of the current part
so that each part receives an individual |.aux| file
that does not interfere with the |.aux| file(s) of the main document.
This behaviour can be altered by the alternative form
|\childdocby[*]{|\textit{main}|}| (with a non-empty optional argument)
which uses the |.aux| file of the main document
by setting |\jobname| to \textit{main}.

%%%%%%%%%%%%%%%%%%%%%%%%%%%%%%%%%%%%%%%%%%%%%%%%%%%%%%%%%%%%%%%%%%%%%%%%%%%%%%%%
\subsection{Driver Development}
\label{sec:driver}

The \textsf{childdoc} mechanism can also be use for the development
of definition files such as \LaTeX{} styles or classes.
This case differs from the above setup with multiple parts
included by |\include| in that no |\includeonly| should be invoked.
This can be achieved by starting the include file
(before |\ProvidesPackage|) with:
%
\begin{center}
\begin{tabular}{l}
|\input{childdoc.def}|\\
|\childdocforward{|\textit{main}|}|\\
\end{tabular}
\end{center}
%
or alternatively with:
%
\begin{center}
\begin{tabular}{l}
|\input{childdoc.def}|\\
|\childdocby{|\textit{main}|}|\\
\end{tabular}
\end{center}
%
Both forms have slightly different effects as described above.
The main file is prepared as usual, see \secref{sec:include}.

%%%%%%%%%%%%%%%%%%%%%%%%%%%%%%%%%%%%%%%%%%%%%%%%%%%%%%%%%%%%%%%%%%%%%%%%%%%%%%%%
\subsection{Legacy Detection}
\label{sec:detection}

The directive |\childdocmain| in the main file can detect
whether the complete document or merely a child is to be compiled
even without using the directive |\childdocof|.
This method is deprecated because it is less robust
and there is no compelling reason to use it;
it is merely provided for backward compatibility
and it may be removed in future versions.

If the detection mechanism is to be used,
it is mandatory to correctly specify
the filename of the main file as the argument of |\childdocmain|:
%
\begin{center}
\begin{tabular}{l}
|\input{childdoc.def}|\\
|\childdocmain{|\textit{main}|}|\\
\end{tabular}
\end{center}
%
If |\jobname| does not match the argument \textit{main} of |\childdocmain|,
it is assumed that |\jobname| points to the child file to be compiled.
When using |\childdocmain| with the main file specified as argument,
it suffices to start a child file
with just |\input{|\textit{main}|}|
without loading of the package and using |\childdocof|.
If instead all processing is done
with the appropriate \textsf{childdoc} directives,
the argument of \textit{main} of |\childdocmain| can be empty.

An alternative version of the command line processing described
in \secref{sec:commandline} using the detection mechanism reads:
%
\begin{center}
|... -jobname "|\textit{target}|" "|[\textit{flags}]%
[|\def\jobname{|\textit{dest}|}|]|\input{|\textit{main}|}"|
\end{center}

%%%%%%%%%%%%%%%%%%%%%%%%%%%%%%%%%%%%%%%%%%%%%%%%%%%%%%%%%%%%%%%%%%%%%%%%%%%%%%%%
\subsection{Manual Code}
\label{sec:manual}

In case one cannot be certain whether the definitions file |childdoc.def|
is installed on the target \TeX{} distribution
and one prefers not to ship it,
it is conceivable to paste a few relevant commands into the sources.

To that end, drop all statements |\input{childdoc.def}|
and perform the replacements as outlined below.
Instead of |\childdocmain{|\textit{main}|}| add the following code
to the top of the main file:
%
\begin{center}
\begin{tabular}{l}
|\||ifdefined\childdocname\endinput\||fi\newif\ifchilddoc|\\
|\edef\childdocname{\scantokens\expandafter{\jobname\noexpand}}|\\
|\def\childdocmain{|\textit{main}|}\||ifx\childdocmain\childdocname\||else|\\
|\childdoctrue\includeonly{\childdocname}\let\jobname\childdocmain\||fi|\\
\end{tabular}
\end{center}
%
Instead of |\childdocof{|\textit{main}|}| just include the main file
at the top of each child file:
%
\begin{center}
|\input{|\textit{main}|}|
\end{center}
%
A simple redirection |\childdocforward{|\textit{dest}|}| is achieved by:
%
\begin{center}
|\def\jobname{|\textit{dest}|}\input{\jobname}|
\end{center}
%
The redirection with prefix
|\childdocforwardprefix[|\textit{prefix}|]{|\textit{dest}|}|
is accomplished by:
%
\begin{center}
\begin{tabular}{l}
|{\edef\jobname{\scantokens\expandafter{\jobname\noexpand}}|\\
|\def\redirectjob |\textit{prefix}|#1~~~{\gdef\jobname{|\textit{dest}|#1}}|\\
|\expandafter\redirectjob\jobname~~~}\input{\jobname}|
\end{tabular}
\end{center}

In an alternative approach,
child documents can be compiled by a specific command line
without additional code or specific definitions:
%
\begin{center}
|... -jobname "|\textit{target}|" "|[\textit{flags}]%
|\includeonly{|\textit{dest}|}\input{|\textit{main}|}"|
\end{center}
%

%%%%%%%%%%%%%%%%%%%%%%%%%%%%%%%%%%%%%%%%%%%%%%%%%%%%%%%%%%%%%%%%%%%%%%%%%%%%%%%%
%%%%%%%%%%%%%%%%%%%%%%%%%%%%%%%%%%%%%%%%%%%%%%%%%%%%%%%%%%%%%%%%%%%%%%%%%%%%%%%%
\section{Information}

%%%%%%%%%%%%%%%%%%%%%%%%%%%%%%%%%%%%%%%%%%%%%%%%%%%%%%%%%%%%%%%%%%%%%%%%%%%%%%%%
\subsection{Copyright}

Copyright \copyright{} 2017--2018 Niklas Beisert

This work may be distributed and/or modified under the
conditions of the \LaTeX{} Project Public License, either version 1.3
of this license or (at your option) any later version.
The latest version of this license is in
  \url{http://www.latex-project.org/lppl.txt}
and version 1.3 or later is part of all distributions of \LaTeX{}
version 2005/12/01 or later.

This work has the LPPL maintenance status `maintained'.

The Current Maintainer of this work is Niklas Beisert.

This work consists of the files |README.txt|, |childdoc.ins| and |childdoc.dtx|
as well as the derived files |childdoc.def|, |cdocsamp.tex|
with |cdocsch1.tex|, |cdocsch2.tex|, |cdocspt3.tex|, |cdocspt4.tex|,
|cdocsdrf.tex|, |cdocsfn1.tex|, |cdocsfn2.tex|
as well as |childdoc.pdf|.

%%%%%%%%%%%%%%%%%%%%%%%%%%%%%%%%%%%%%%%%%%%%%%%%%%%%%%%%%%%%%%%%%%%%%%%%%%%%%%%%
\subsection{Files and Installation}

The package consists of the files:
%
\begin{center}
\begin{tabular}{ll}
    |README.txt|   & readme file \\
    |childdoc.ins| & installation file \\
    |childdoc.dtx| & source file \\
    |childdoc.def| & definition file \\
    |cdocsamp.tex| & sample main file \\
    |cdocsch1.tex| & sample include file \\
    |cdocsch2.tex| & sample include file \\
    |cdocspt3.tex| & sample part file \\
    |cdocspt4.tex| & sample part file \\
    |cdocsdrf.tex| & sample redirection file \\
    |cdocsfn1.tex| & sample redirection file \\
    |cdocsfn2.tex| & sample redirection file \\
    |childdoc.pdf| & manual
\end{tabular}
\end{center}
%
The distribution consists of the files
|README.txt|, |childdoc.ins| and |childdoc.dtx|.
%
\begin{itemize}
\item
Run (pdf)\LaTeX{} on |childdoc.dtx|
to compile the manual |childdoc.pdf| (this file).
\item
Run \LaTeX{} on |childdoc.ins| to create the definitions file |childdoc.def|
and the sample |cdocsamp.tex| with include files
|cdocsch1.tex|, |cdocsch2.tex|, |cdocspt3.tex|, |cdocspt4.tex|,
|cdocsdrf.tex|, |cdocsfn1.tex|, |cdocsfn2.tex|.
Then copy the file |childdoc.def| to an appropriate directory of your \LaTeX{}
distribution, e.g.\ \textit{texmf-root}|/tex/latex/childdoc|.
\end{itemize}

%%%%%%%%%%%%%%%%%%%%%%%%%%%%%%%%%%%%%%%%%%%%%%%%%%%%%%%%%%%%%%%%%%%%%%%%%%%%%%%%
\subsection{Related CTAN Packages}

There are several other packages which offer a similar functionality:
%
\begin{itemize}
\item
The packages
\href{http://ctan.org/pkg/docmute}{\textsf{docmute}},
\href{http://ctan.org/pkg/includex}{\textsf{includex}} and
\href{http://ctan.org/pkg/standalone}{\textsf{standalone}}
provide commands to include only the document body of
a child file thus allowing both files to be compiled individually.
\item
The packages \href{http://ctan.org/pkg/subdocs}{\textsf{subdocs}}
and \href{http://ctan.org/pkg/subfiles}{\textsf{subfiles}}
provide structures in which the main and child documents can be
encapsulated and allowing them to be compiled individually.
The inclusion mechanism is different from the conventional |\include|.
\item
The package \href{http://ctan.org/pkg/combine}{\textsf{combine}}
is an elaborate solution to combine several documents into one.
\end{itemize}
%
See also the CTAN topic \href{http://ctan.org/topic/subdocs}{\textsf{subdocs}}
for further related packages.
The present package differs from the above solutions in that
a document structure constructed with the conventional |\include| mechanism
just needs two extra commands at the top of every file
such that all constituent files can be compiled individually.

%%%%%%%%%%%%%%%%%%%%%%%%%%%%%%%%%%%%%%%%%%%%%%%%%%%%%%%%%%%%%%%%%%%%%%%%%%%%%%%%
%\subsection{Feature Suggestions}
%
%The following is a list of features which may be useful for future
%versions of this package:
%%
%\begin{itemize}
%\item
%\ldots
%\end{itemize}

%%%%%%%%%%%%%%%%%%%%%%%%%%%%%%%%%%%%%%%%%%%%%%%%%%%%%%%%%%%%%%%%%%%%%%%%%%%%%%%%
\subsection{Revision History}

%%%%%%%%%%%%%%%%%%%%%%%%%%%%%%%%%%%%%%%%
\paragraph{v2.0:} 2018/12/30

\begin{itemize}
\item
immediate forward processing
\item
added |\childdocby| mechanism
\item
manual restructured
\end{itemize}

%%%%%%%%%%%%%%%%%%%%%%%%%%%%%%%%%%%%%%%%
\paragraph{v1.6:} 2018/01/17

\begin{itemize}
\item
application for development of include files
\item
corrections to manual
\end{itemize}

%%%%%%%%%%%%%%%%%%%%%%%%%%%%%%%%%%%%%%%%
\paragraph{v1.5:} 2017/05/21

\begin{itemize}
\item
more complete structuring introduced
\item
|\childdocof| introduced
\item
|\childdoc| renamed to |\childdocmain|
\item
|\childredirect| renamed to |\childdocforward| and |\childdocforwardprefix|
and functionality expanded
\end{itemize}

%%%%%%%%%%%%%%%%%%%%%%%%%%%%%%%%%%%%%%%%
\paragraph{v1.0:} 2017/04/27

\begin{itemize}
\item
manual and install package
\item
first version published on CTAN
\end{itemize}

%%%%%%%%%%%%%%%%%%%%%%%%%%%%%%%%%%%%%%%%
\paragraph{v0.6:} 2017/04/26

\begin{itemize}
\item
redirection mechanism added
\end{itemize}

%%%%%%%%%%%%%%%%%%%%%%%%%%%%%%%%%%%%%%%%
\paragraph{v0.5:} 2017/04/26

\begin{itemize}
\item
functionality in definition file
\end{itemize}


%%%%%%%%%%%%%%%%%%%%%%%%%%%%%%%%%%%%%%%%%%%%%%%%%%%%%%%%%%%%%%%%%%%%%%%%%%%%%%%%
%%%%%%%%%%%%%%%%%%%%%%%%%%%%%%%%%%%%%%%%%%%%%%%%%%%%%%%%%%%%%%%%%%%%%%%%%%%%%%%%
%%%%%%%%%%%%%%%%%%%%%%%%%%%%%%%%%%%%%%%%%%%%%%%%%%%%%%%%%%%%%%%%%%%%%%%%%%%%%%%%
\appendix

\settowidth\MacroIndent{\rmfamily\scriptsize 000\ }

 \DocInput{childdoc.dtx}

\end{document}
%</driver>
% \fi
%
% %%%%%%%%%%%%%%%%%%%%%%%%%%%%%%%%%%%%%%%%%%%%%%%%%%%%%%%%%%%%%%%%%%%%%%%%%%%%%%
% %%%%%%%%%%%%%%%%%%%%%%%%%%%%%%%%%%%%%%%%%%%%%%%%%%%%%%%%%%%%%%%%%%%%%%%%%%%%%%
% \section{Sample}
%\iffalse
%<*samplemain>
%\fi
%
% The following presents a sample document
% with two chapters, two parts, a title page,
% a compile flag as well as three forwarding files to set the flag.
% It consists of eight |.tex| files:
% \begin{center}
% \begin{tabular}{ll}
% |cdocsamp.tex|&main file\\
% |cdocsch1.tex|&include file for chapter 1\\
% |cdocsch2.tex|&include file for chapter 2\\
% |cdocspt3.tex|&include file for part 3\\
% |cdocspt4.tex|&include file for part 4\\
% |cdocsdrf.tex|&forwarding file for main file in draft mode\\
% |cdocsfi1.tex|&forwarding file for final version of chapter 1\\
% |cdocsfi2.tex|&forwarding file for final version of chapter 2\\
% \end{tabular}
% \end{center}
% Each of the eight files can be compiled directly by the \LaTeX{} compiler.
%
% %%%%%%%%%%%%%%%%%%%%%%%%%%%%%%%%%%%%%%
% \paragraph{Main File.}
%
% The main file is called |cdocsamp.tex|.
%
% Load the \textsf{childdoc} definitions and
% declare the filename for the main document:
%    \begin{macrocode}
\input{childdoc.def}
\childdocmain{}
%    \end{macrocode}

% Optional override for |\version| flag:
%    \begin{macrocode}
%%\ifchilddoc\else\providecommand{\version}{draft}\fi
%    \end{macrocode}

% Define the default values for the |\version| flag
% (|final| for the main file and |draft| for childs):
%    \begin{macrocode}
\ifchilddoc
\providecommand{\version}{draft}
\else
\providecommand{\version}{final}
\fi
%    \end{macrocode}

% Load the standard document class:
%    \begin{macrocode}
\documentclass[12pt]{article}
%    \end{macrocode}

% Start the document body:
%    \begin{macrocode}
\begin{document}
%    \end{macrocode}

% Declare a title page.
% Print title, part of document being processed and version flag:
%    \begin{macrocode}
\addtocounter{page}{-1}
\begin{center}
{\LARGE\bfseries{}childdoc example\par}
\vspace{1cm}
\ifchilddoc
\ifchilddocmanual part\else chapter\fi:
`\childdocname' of `\childdocjob'\par
\else
main document: `\childdocjob'\par
\fi
version: \version\par
\end{center}
\newpage
%    \end{macrocode}

% Manually include selected file,
% otherwise process as usual:
%    \begin{macrocode}
\ifchilddocmanual
\section*{part `\childdocname'}
\input{\childdocname}
\else
%    \end{macrocode}

% Include the two chapters:
%    \begin{macrocode}
\include{cdocsch1}
\include{cdocsch2}
%    \end{macrocode}

% Include the two parts unless only chapters should be displayed:
%    \begin{macrocode}
\ifchilddoc\else
\section{part three}
\input{cdocspt3}
\section{part four}
\input{cdocspt4}
\fi
%    \end{macrocode}

% Process as usual until here:
%    \begin{macrocode}
\fi
%    \end{macrocode}

% End of document body:
%    \begin{macrocode}
\end{document}
%    \end{macrocode}
%\iffalse
%</samplemain>
%\fi
%
% %%%%%%%%%%%%%%%%%%%%%%%%%%%%%%%%%%%%%%
% \paragraph{Chapter Include Files.}
%
% The include files are called |cdocsch1.tex| and |cdocsch2.tex|.
%
%\iffalse
%<*samplechap1|samplechap2>
%\fi

% Optional override for |\version| flag:
%    \begin{macrocode}
%%\providecommand{\version}{final}
%    \end{macrocode}

% Include the main document:
%    \begin{macrocode}
\input{childdoc.def}
\childdocof{cdocsamp}
%    \end{macrocode}

%\iffalse
%</samplechap1|samplechap2>
%\fi
%
%\iffalse
%<*samplechap1>
%\fi
% Some text for chapter 1:
%    \begin{macrocode}
\section{one}
some text in chapter one
%    \end{macrocode}

%\iffalse
%</samplechap1>
%\fi
% Some text for chapter 2:
%\iffalse
%<*samplechap2>
%\fi
%    \begin{macrocode}
\section{two}
more text in chapter two
%    \end{macrocode}

%\iffalse
%</samplechap2>
%\fi
%
% %%%%%%%%%%%%%%%%%%%%%%%%%%%%%%%%%%%%%%
% \paragraph{Part Include Files.}
%
% The include files are called |cdocspt3.tex| and |cdocspt4.tex|.
%
%\iffalse
%<*samplepart3|samplepart4>
%\fi

% Optional override for |\version| flag:
%    \begin{macrocode}
%%\providecommand{\version}{final}
%    \end{macrocode}

% Include the main document:
%    \begin{macrocode}
\input{childdoc.def}
\childdocby{cdocsamp}
%    \end{macrocode}

%\iffalse
%</samplepart3|samplepart4>
%\fi
%
%\iffalse
%<*samplepart3>
%\fi
% Some text for part 3:
%    \begin{macrocode}
some text in part three
%    \end{macrocode}

%\iffalse
%</samplepart3>
%\fi
% Some text for part 4:
%\iffalse
%<*samplepart4>
%\fi
%    \begin{macrocode}
more text in part four
%    \end{macrocode}

%\iffalse
%</samplepart4>
%\fi
%
% %%%%%%%%%%%%%%%%%%%%%%%%%%%%%%%%%%%%%%
% \paragraph{Forwarding for a Complete Draft.}
%
% The following forwarding file |cdocsdrf.tex|
% compiles the main document in draft mode:
%\iffalse
%<*sampledraft>
%\fi
%    \begin{macrocode}
\def\version{draft}
\input{childdoc.def}
\childdocforward{cdocsamp}
%    \end{macrocode}

%\iffalse
%</sampledraft>
%\fi
%
% %%%%%%%%%%%%%%%%%%%%%%%%%%%%%%%%%%%%%%
% \paragraph{Forwarding for Final Version of the Chapters.}
%
% The following forwarding files |cdocsfn1.tex| and |cdocsfn2.tex|
% (with identical content)
% compile the final versions of the child documents
% |cdocsch1.tex| and |cdocsch2.tex|, respectively:
%\iffalse
%<*samplefinal>
%\fi
%    \begin{macrocode}
\def\version{final}
\input{childdoc.def}
\childdocforwardprefix[cdocsamp]{cdocsfn}{cdocsch}
%    \end{macrocode}

%\iffalse
%</samplefinal>
%\fi
%
% %%%%%%%%%%%%%%%%%%%%%%%%%%%%%%%%%%%%%%
% \paragraph{Command Line Processing.}
%
% The following three command lines generate the output files
% |cdocscld|, |cdocscl1| and |cdocscl2|
% which should be identical to
% |cdocsdrf|, |cdocsch1| and |cdocsfn2|, respectively:
% \begin{center}
% \begin{tabular}{l}
% |latex -jobname cdocscld \|\\
% |  "\def\version{draft}\input{childdoc.def}\childdocforward{cdocsamp}"|\\
% |latex -jobname cdocscl1 \|\\
% |  "\input{childdoc.def}\childdocforward[cdocsamp]{cdocsch1}"|\\
% |latex -jobname cdocscl2 \|\\
% |  "\def\version{final}\input{childdoc.def}\childdocforward{cdocsch2}"|
% \end{tabular}
% \end{center}
% Note that the trailing backslash on each first line
% merely continues the input to the second line
% (for convenient cut ant paste).
% Furthermore, the command |latex| can be replaced by any
% of its alternative versions such as |pdflatex|.
%
% %%%%%%%%%%%%%%%%%%%%%%%%%%%%%%%%%%%%%%%%%%%%%%%%%%%%%%%%%%%%%%%%%%%%%%%%%%%%%%
% %%%%%%%%%%%%%%%%%%%%%%%%%%%%%%%%%%%%%%%%%%%%%%%%%%%%%%%%%%%%%%%%%%%%%%%%%%%%%%
% \section{Implementation}
%\iffalse
%<*package>
%\fi
%
% This section describes the definitions file |childdoc.def|.

% The definitions cannot be loaded using |\usepackage| or |\RequirePackage|
% which has a mechanism to prevent loading a style file more than once.
% When loading the definitions by means of |\input|
% multiple instances have to be prevented manually:
%\iffalse
%This code needs to be before the `\ProvidesFile' directive
%which is defined at the beginning of this file.
%Therefore it is also placed there and commented out here.
%</package>
%<*discard>
%\fi
%    \begin{macrocode}
\ifdefined\childdocmain\endinput\fi
%    \end{macrocode}
%\iffalse
%</discard>
%<*package>
%\fi
%
% \macro{\ifchilddoc}
% \macro{\ifchilddocmanual}
% The conditional |\ifchilddoc| tells whether a
% child (true) or main (false) document is being compiled.
% The conditional |\ifchilddocmanual| tells whether
% the |\includeonly| mechanism is used (false) or
% the selection of child files must be performed manually (true).
% The definitions initialise to false:
%    \begin{macrocode}
\newif\ifchilddoc
\newif\ifchilddocmanual
%    \end{macrocode}

% \macro{\childdocname}
% \macro{\childdocjob}
% The macro |\childdocname| stores the name of the main document
% to be compiled. The macro |\childdocjob| stores the name of
% the document on which the \LaTeX{} compiler was originally invoked.
% The content of |\jobname| cannot be compared
% to filenames specified in the source due to different catcodes.
% The following code rescans |\jobname|, stores the result
% in |\childdocname| and saves a copy in |\childdocjob|:
%    \begin{macrocode}
\edef\childdocname{\scantokens\expandafter{\jobname\noexpand}}
\let\childdocjob\childdocname
%    \end{macrocode}

% \macro{\childdocdisable}
% The macro |\childdocdisable| prevents the main file
% from being processed more than once.
% At this stage, the main document command |\childdocmain|
% is assumed to be called once again where it should do nothing.
% Any subsequent call to it should prevent
% a secondary processing of the main document
% It overwrites the forwarding commands
% |\childdocof| and |\childdocforward|
% with empty macros to prevent further inclusions of the main document:
%    \begin{macrocode}
\newcommand{\childdocdisable}
{
  \renewcommand{\childdocmain}[1]{\renewcommand{\childdocmain}[1]{\endinput}}
  \renewcommand{\childdocof}[1]{}
  \renewcommand{\childdocby}[2][]{}
  \renewcommand{\childdocforward}[2][]{}
  \renewcommand{\childdocdisable}{}
}
%    \end{macrocode}

% \macro{\childdocmain}
% The macro |\childdocmain| is to be called at the top of the main file
% with nothing or the main filename (without extension) as argument.
% First, it breaks loops.
% If the argument is not empty and does not match |\childdocname|
% (which is set by the first inclusion of |childdoc.def|),
% |\ifchilddoc| is set to true, |\includeonly| is applied to the child file
% and |\jobname| is set to the main file
% (for proper handling of |.aux| files):
%    \begin{macrocode}
\newcommand{\childdocmain}[1]
{
  \childdocdisable\childdocmain{}
  \if?#1?\else
    \begingroup
      \def\childdoctmp{#1}
      \ifx\childdoctmp\childdocname
        \def\childdoctmp{}
      \else
        \def\childdoctmp
        {
          \childdoctrue
          \includeonly{\childdocname}
          \def\childdocjob{#1}
          \def\jobname{#1}
        }
      \fi
      \expandafter
    \endgroup
    \childdoctmp
  \fi
}
%    \end{macrocode}

% \macro{\childdocof}
% The command |\childdocof| redirects
% compilation to the main file |#1|.
%    \begin{macrocode}
\newcommand{\childdocof}[1]
{
  \childdocdisable
  \childdoctrue
  \includeonly{\childdocname}
  \def\jobname{#1}
  \def\childdocjob{#1}
  \input{#1}
}
%    \end{macrocode}

% \macro{\childdocby}
% The command |\childdocby| ....
%    \begin{macrocode}
\newcommand{\childdocby}[2][]
{
  \childdocdisable
  \childdoctrue
  \childdocmanualtrue
  \if?#1?\else
    \def\jobname{#2}
  \fi
  \def\childdocjob{#2}
  \input{#2}
  \endinput
}
%    \end{macrocode}

% \macro{\childdocforward}
% The command |\childdocforward| redirects
% compilation to the main file or
% (if the optional argument is given) a child file.
% Parameters are set as if the main file
% or a child file starting with |\childdocof| was compiled.
% Then compilation is handed over to the main file:
%    \begin{macrocode}
\newcommand{\childdocforward}[2][]
{
  \begingroup
    \if?#1?
      \def\childdoctmp
      {
        \def\childdocname{#2}
        \def\childdocjob{#2}
        \def\jobname{#2}
        \input{#2}
        \endinput
      }
    \else
      \def\childdoctmp
      {
        \childdocdisable
        \def\childdocname{#2}
        \childdoctrue
        \includeonly{#2}
        \def\childdocjob{#1}
        \def\jobname{#1}
        \input{#1}
        \endinput
      }
    \fi
    \expandafter
  \endgroup
  \childdoctmp
}
%    \end{macrocode}

% \macro{\childdocforwardprefix}
% The command |\childdocforwardprefix| redirects
% compilation to the main or a child file by means of a pattern.
% The prefix |#1| in the current filename is replaced by |#2|
% and the suffix of the current filename is kept
% (it is assumed that the filename does not contain the substring `|~~~|'
% which is used as a delimiter).
% Compilation is handed over to the new file by |\childdocforward|:
%    \begin{macrocode}
\newcommand{\childdocforwardprefix}[3][]
{
  \begingroup
    \def\childdocextract #2##1~~~{\def\childdoctmp{\childdocforward[#1]{#3##1}}}
    \expandafter\childdocextract\childdocname~~~
    \expandafter
  \endgroup
  \childdoctmp
}
%    \end{macrocode}

% \macro{\childdoc}
% The deprecated macro |\childdoc| is a legacy version of |\childdocmain|:
%    \begin{macrocode}
\newcommand{\childdoc}{\childdocmain}
%    \end{macrocode}

% \macro{\childdocredirect}
% The deprecated macro |\childdocredirect| is a legacy version
% of |\childdocforward| and |\childdocforwardprefix|:
%    \begin{macrocode}
\newcommand{\childdocredirect}[2][]
{
  \begingroup
    \if?#1?
      \def\childdoctmp{\childdocforward{#2}}
    \else
      \def\childdoctmp{\childdocforwardprefix{#1}{#2}}
    \fi
    \expandafter
  \endgroup
  \childdoctmp
}
%    \end{macrocode}

%\iffalse
%</package>
%\fi
%
\endinput
|\\
|\childdocmain{|\textit{main}|}|\\
\end{tabular}
\end{center}
%
If |\jobname| does not match the argument \textit{main} of |\childdocmain|,
it is assumed that |\jobname| points to the child file to be compiled.
When using |\childdocmain| with the main file specified as argument,
it suffices to start a child file
with just |\input{|\textit{main}|}|
without loading of the package and using |\childdocof|.
If instead all processing is done
with the appropriate \textsf{childdoc} directives,
the argument of \textit{main} of |\childdocmain| can be empty.

An alternative version of the command line processing described
in \secref{sec:commandline} using the detection mechanism reads:
%
\begin{center}
|... -jobname "|\textit{target}|" "|[\textit{flags}]%
[|\def\jobname{|\textit{dest}|}|]|\input{|\textit{main}|}"|
\end{center}

%%%%%%%%%%%%%%%%%%%%%%%%%%%%%%%%%%%%%%%%%%%%%%%%%%%%%%%%%%%%%%%%%%%%%%%%%%%%%%%%
\subsection{Manual Code}
\label{sec:manual}

In case one cannot be certain whether the definitions file |childdoc.def|
is installed on the target \TeX{} distribution
and one prefers not to ship it,
it is conceivable to paste a few relevant commands into the sources.

To that end, drop all statements |% \iffalse
%
% childdoc.dtx Copyright (C) 2017-2018 Niklas Beisert
%
% This work may be distributed and/or modified under the
% conditions of the LaTeX Project Public License, either version 1.3
% of this license or (at your option) any later version.
% The latest version of this license is in
%   http://www.latex-project.org/lppl.txt
% and version 1.3 or later is part of all distributions of LaTeX
% version 2005/12/01 or later.
%
% This work has the LPPL maintenance status `maintained'.
%
% The Current Maintainer of this work is Niklas Beisert.
%
% This work consists of the files childdoc.dtx and childdoc.ins
% and the derived files childdoc.def and cdocsamp.tex with
% cdocsch1.tex, cdocsch2.tex, cdocsdrf.tex, cdocsfn1.tex, cdocsfn2.tex.
%
%<package>\ifdefined\childdocmain\endinput\fi
%<package>\ProvidesFile{childdoc.def}[2018/12/30 v2.0 child document driver]
%<samplemain>\ProvidesFile{cdocsamp.tex}[2018/12/30 v2.0 sample for childdoc]
%<*driver>
%\ProvidesFile{childdoc.drv}[2018/12/30 v2.0 childdoc reference manual file]
\PassOptionsToClass{10pt,a4paper}{article}
\documentclass{ltxdoc}

\usepackage[margin=35mm]{geometry}
\usepackage{hyperref}
\usepackage{hyperxmp}
\usepackage[usenames]{color}

\hypersetup{colorlinks=true}
\hypersetup{pdfstartview=FitH}
\hypersetup{pdfpagemode=UseNone}
\hypersetup{pdfsource={}}
\hypersetup{pdflang={en-UK}}
\hypersetup{pdfcopyright={Copyright 2017-2018 Niklas Beisert.
  This work may be distributed and/or modified under the
  conditions of the LaTeX Project Public License, either version 1.3
  of this license or (at your option) any later version.}}
\hypersetup{pdflicenseurl={http://www.latex-project.org/lppl.txt}}
\hypersetup{pdfcontactaddress={ETH Zurich, ITP, HIT K,
  Wolfgang-Pauli-Strasse 27}}
\hypersetup{pdfcontactpostcode={8093}}
\hypersetup{pdfcontactcity={Zurich}}
\hypersetup{pdfcontactcountry={Switzerland}}
\hypersetup{pdfcontactemail={nbeisert@itp.phys.ethz.ch}}
\hypersetup{pdfcontacturl={http://people.phys.ethz.ch/\xmptilde nbeisert/}}

\newcommand{\secref}[1]{\hyperref[#1]{section \ref*{#1}}}

\parskip1ex
\parindent0pt
\let\olditemize\itemize
\def\itemize{\olditemize\parskip0pt}

\begin{document}

\title{The \textsf{childdoc} Package}
\hypersetup{pdftitle={The childdoc Package}}
\author{Niklas Beisert\\[2ex]
  Institut f\"ur Theoretische Physik\\
  Eidgen\"ossische Technische Hochschule Z\"urich\\
  Wolfgang-Pauli-Strasse 27, 8093 Z\"urich, Switzerland\\[1ex]
  \href{mailto:nbeisert@itp.phys.ethz.ch}
  {\texttt{nbeisert@itp.phys.ethz.ch}}}
\hypersetup{pdfauthor={Niklas Beisert}}
\hypersetup{pdfsubject={Manual for the LaTeX2e Package childdoc}}
\date{30 December 2018, \textsf{v2.0}}
\maketitle

\begin{abstract}\noindent
\textsf{childdoc} is a \LaTeXe{} package
that enables the direct compilation
of document sections included by |\include|
to individual files.
\end{abstract}

\begingroup
\parskip0ex
\tableofcontents
\endgroup

%%%%%%%%%%%%%%%%%%%%%%%%%%%%%%%%%%%%%%%%%%%%%%%%%%%%%%%%%%%%%%%%%%%%%%%%%%%%%%%%
%%%%%%%%%%%%%%%%%%%%%%%%%%%%%%%%%%%%%%%%%%%%%%%%%%%%%%%%%%%%%%%%%%%%%%%%%%%%%%%%
\section{Introduction}

\LaTeX{} provides a mechanism to structure a large document (such as a book)
into a main file and several child files (containing the chapters)
using the |\include| command.
This mechanism is beneficial for documents
which span hundreds of pages in order to
make the source file(s) more manageable.
Moreover, compilation can be restricted to
selected child files by means of the |\includeonly| command.
The latter feature can be used to reduce the compilation time while editing
(this was significantly more useful in the earlier days of \LaTeX{})
or to generate a smaller document which is easier to navigate.
Another application of |\includeonly| is to generate
documents consisting of selected parts of the complete document.

However, there are a few drawbacks of the plain |\include| mechanism:
\begin{itemize}
\item
The child files cannot be compiled on their own,
they can only be compiled via the main file.
A naive editing environment
(such as a text editor with an option
to have the current file processed by \LaTeX)
may require one to switch to the main file before compiling;
attempting to compile the child file produces errors.
\item
The main file must be modified (each time)
to adjust the |\includeonly| command
to the present needs. This easily leaves the main file in a messy state.
\item
The generated document will always carry the filename
of the main document. This is inconvenient if
several child files are to be compiled and
to be kept for distribution.
\end{itemize}

The present package provides a simple interface
to make child files individually compilable by \LaTeX{}.
Compiling a child file then has the same effect as compiling
the main file with an |\includeonly| command
to select the appropriate child.
Moreover the generated document will carry the name of the child
rather than the main file.
This resolves all three above issues.

This feature is meant to make the editing of books,
thesis documents and lecture notes somewhat more convenient.
However, the package can also be used efficiently for
composing a series of documents (such as exercise sheets)
which are typically distributed individually.
It then assists the author in generating the individual documents
(potentially in different versions)
as well as a document containing the collected series.
Another application is in developing style files
or other kinds of included material
where compilation of the style file could redirect
to a sample or test file.

%%%%%%%%%%%%%%%%%%%%%%%%%%%%%%%%%%%%%%%%%%%%%%%%%%%%%%%%%%%%%%%%%%%%%%%%%%%%%%%%
%%%%%%%%%%%%%%%%%%%%%%%%%%%%%%%%%%%%%%%%%%%%%%%%%%%%%%%%%%%%%%%%%%%%%%%%%%%%%%%%
\section{Usage}

First of all, the package \textsf{childdoc} is \emph{not} a standard
\LaTeXe{} |.sty| style file! Therefore it needs to be invoked in
a non-standard way.

%%%%%%%%%%%%%%%%%%%%%%%%%%%%%%%%%%%%%%%%%%%%%%%%%%%%%%%%%%%%%%%%%%%%%%%%%%%%%%%%
\subsection{Included Files}
\label{sec:include}

%%%%%%%%%%%%%%%%%%%%%%%%%%%%%%%%%%%%%%%%
\DescribeMacro{\childdocmain}
To use the package, add the commands
\begin{center}
\begin{tabular}{l}
|\input{childdoc.def}|\\
|\childdocmain{}|\\
\end{tabular}
\end{center}
at the very top of the main \LaTeX{} file,
in particular \emph{before} the |\documentclass| statement!
The argument of |\childdocmain| should be left empty
(but it must be present).

%%%%%%%%%%%%%%%%%%%%%%%%%%%%%%%%%%%%%%%%
\DescribeMacro{\childdocof}
Furthermore, add the commands
\begin{center}
\begin{tabular}{l}
|\input{childdoc.def}|\\
|\childdocof{|\textit{main}|}|\\
\end{tabular}
\end{center}
at the top of every child file \textit{child}
which is included by |\include{|\textit{child}|}|
from within the main file
(or at least for those files to be compiled individually).
The argument \textit{main} must be the filename of the main file.

There are a couple of
considerations in setting up the main and child documents:

%%%%%%%%%%%%%%%%%%%%%%%%%%%%%%%%%%%%%%%%
\paragraph{Restrictions.}

Please note the following restrictions:
\begin{itemize}
\item
|\childdocmain| must be called with one argument \textit{main}
to ensure compatibility with earlier version of the package.
It must either be empty (|\childdocmain{}|)
or precisely match the filename of the main file in which it is specified.
See \secref{sec:detection} for further information.
\item
The filename \textit{main} must be specified without the |.tex| extension.
\item
The filename \textit{main} is case sensitive
(even in case-insensitive file systems)
due to internal string comparison.
\item
The argument \textit{main} should be fully expanded, it cannot be a macro.
\item
Subdirectories and special characters should be avoided in filenames.
\item
The command |\childdocmain{|\textit{main}|}| must be followed by a whitespace.
It should not be followed immediately by another command
or by a comment mark `|%|'.
This is because the \TeX{} parser reads the token immediately following
the argument of |\childdocmain| and puts it
at the beginning of every child section;
however, a white\-space is ignored.
\end{itemize}

%%%%%%%%%%%%%%%%%%%%%%%%%%%%%%%%%%%%%%%%
\paragraph{Content of Main File.}

It is advisable to place all content in the child files included by |\include|.
Any output contained in the main file will appear in all child documents
unless suppressed manually;
it cannot be suppressed automatically by the |\includeonly| directive
and thus should normally be avoided.
A method to include some content in the main file
by means of conditional processing is described in \secref{sec:conditional}.

%%%%%%%%%%%%%%%%%%%%%%%%%%%%%%%%%%%%%%%%
\paragraph{Page Numbering.}

When only a part of the document is compiled,
the appropriate numbering of pages
(as well as other status parameters)
is determined from the |.aux| files.
The latter contain information from previous passes.
However this information needs to propagate through
all intermediate child documents.
Therefore the page numbering in child documents may well
be inconsistent until the complete document is compiled at least once.

A useful (if unconventional) way to always ensure a consistent
page numbering is to restart the numbering in each child document
and denote the pages by `\textit{child}|.|\textit{page}'
where \textit{child} represents the chapter/section number of the child file.
This can be achieved by the command
|\numberwithin{page}{|\textit{child}|}|
of the \textsf{amsmath} package
where \textit{child} can be |chapter| or |section|
depending on the chosen structuring.
Alternatively, one can modify the macro |\thepage| appropriately
and reset the counter |page| at the start of each child file.

%%%%%%%%%%%%%%%%%%%%%%%%%%%%%%%%%%%%%%%%%%%%%%%%%%%%%%%%%%%%%%%%%%%%%%%%%%%%%%%%
\subsection{Conditional Processing}
\label{sec:conditional}

The package provides a mechanism to compile different versions
of a document. To customise the versions further some conditional processing
can come in handy to distinguish which version is being compiled.
The package provides two macros to describe the compilation context:

%%%%%%%%%%%%%%%%%%%%%%%%%%%%%%%%%%%%%%%%
\DescribeMacro{\ifchilddoc}
The conditional |\ifchilddoc| distinguishes between the compilation of
child documents and the main document:
%
\begin{center}
|\ifchilddoc |\textit{child-code}| |[|\||else |\textit{main-code}]| \||fi|
\end{center}

%%%%%%%%%%%%%%%%%%%%%%%%%%%%%%%%%%%%%%%%
\DescribeMacro{\childdocname}
\DescribeMacro{\childdocjob}
The macro |\childdocname| contains the filename (without extension)
of the main or child file being processed.
Note that |\childdocjob| will always contain the name of the main file.

%%%%%%%%%%%%%%%%%%%%%%%%%%%%%%%%%%%%%%%%
\paragraph{Title Page.}

Conditional processing can be used to include a title or banner page
in the main document when proper precautions are taken.
Importantly, the code in the main file should ensure that the page counter
(as well as other status parameters which are stored in the |.aux| files)
takes the same value after the conditional processing.
Otherwise the page numbers may take divergent values
depending on which part is compiled.

For example, a title page could be declared by:
%
\begin{center}
\begin{tabular}{l}
|\ifchilddoc\||else|\\
|\addtocounter{page}{-1}|\\
\textit{code for title page}\\
|\newpage|\\
|\||fi|
\end{tabular}
\end{center}
%
A banner page for the child documents can be generated by:
%
\begin{center}
\begin{tabular}{l}
|\ifchilddoc|\\
|\addtocounter{page}{-1}|\\
\textit{code for banner page}\\
|\newpage|\\
|\||fi|
\end{tabular}
\end{center}
%
Here one could write a message such as:
\begin{center}
|This is the part \childdocname{} of \childdocjob{}.|
\end{center}

%%%%%%%%%%%%%%%%%%%%%%%%%%%%%%%%%%%%%%%%%%%%%%%%%%%%%%%%%%%%%%%%%%%%%%%%%%%%%%%%
\subsection{Flags}
\label{sec:flags}

The package makes it easy to generate different versions
of the main or child documents.
To this end compilation flags can be defined
and assigned different default values.
They will be particularly useful in conjunction
with the forwarding mechanism described in \secref{sec:forward}.

For example, it may be useful to have a flag |\version|
which can be set to |draft| or |final|.
The document source will contain some conditional code
depending on the value of |\version|.
Suppose further, the flag should default to |final| for the main file
and to |draft| for child files
which is a natural assignment for editing the document.
This is achieved by placing the following code
in the preamble of the main document
(below the |\childdocmain| directive):
%
\begin{center}
\begin{tabular}{l}
|\ifchilddoc|\\
|\providecommand{\version}{draft}|\\
|\||else|\\
|\providecommand{\version}{final}|\\
|\||fi|
\end{tabular}
\end{center}
%
The definition by |\providecommand| makes sure
that previous definitions are not overwritten.
Further statements |\providecommand{\version}{...}|
can thus be added before the above code to override it.

For the main file, one might add a line
(between |\childdocmain| and the above block)
%
\begin{center}
|%\ifchilddoc\||else\providecommand{\version}{draft}\||fi|
\end{center}
%
which can be uncommented to produce a draft version.
Likewise one can add a line to the very top of a child file
(above the |\childdocof{|\textit{main}|}| directive)
%
\begin{center}
|%\providecommand{\version}{final}|
\end{center}
%
which can be uncommented to produce the final version of this child document.

%%%%%%%%%%%%%%%%%%%%%%%%%%%%%%%%%%%%%%%%%%%%%%%%%%%%%%%%%%%%%%%%%%%%%%%%%%%%%%%%
\subsection{Forwarding}
\label{sec:forward}

Different versions of the main or child documents
using compilation flags as described in \secref{sec:flags}
can be (permanently) stored in different files
for convenient compilation, viewing and distribution.
To this end, the package defines a command
to pass on compilation to a different file:

%%%%%%%%%%%%%%%%%%%%%%%%%%%%%%%%%%%%%%%%
\DescribeMacro{\childdocforward}
The command |\childdocforward| redirects processing to
another source file:
%
\begin{center}
\begin{tabular}{l}
|\input{childdoc.def}|\\
|\childdocforward[|\textit{main}|]{|\textit{dest}|}|\\
\end{tabular}
\end{center}
%
The argument \textit{dest} is the destination file
(without extension).
It should be the main file or one of the child files.
Note that further \textsf{childdoc} directives
such as |\childdocof| and |\childdocforward|
in the indicated file will be processed in this form.
The optional argument \textit{main}
passes on directly to the main file \textit{main}
while pretending to compile the child \textit{dest}.
This form behaves as if \textit{dest}
issues |\childdocof{|\textit{main}|}| right away,
and no further \textsf{childdoc} directives will be processed.

%%%%%%%%%%%%%%%%%%%%%%%%%%%%%%%%%%%%%%%%
\DescribeMacro{\...prefix}
In the alternative form |\childdocforwardprefix|,
%
\begin{center}
\begin{tabular}{l}
|\input{childdoc.def}|\\
|\childdocforwardprefix[|\textit{main}|]{|\textit{prefix}|}{|\textit{dest}|}|
\end{tabular}
\end{center}
%
the destination file is determined by a pattern
depending on the current file:
To make this work, the current file must be called
`{\textit{prefix}\hspace{0.2em}\textit{suffix}}'
with \textit{prefix} matching precisely the argument.
Processing is then passed on to the file
`{\textit{dest}\hspace{0.2em}\textit{suffix}}'.
Surely, the same effect is achieved by
directly specifying the
argument `{\textit{dest}\hspace{0.2em}\textit{suffix}}'
in the first form.
However, that requires to set up a different file
for each child. With the alternative form of the command
all these files can have exactly the same content
which simplifies setting them up and maintaining them.

For example, the following file |draft.tex|
with a compilation flag |\version| as described in \secref{sec:flags}
compiles the main document as a draft:
%
\begin{center}
\begin{tabular}{l}
|\def\version{draft}|\\
|\input{childdoc.def}|\\
|\childdocforward{|\textit{main}|}|
\end{tabular}
\end{center}
%
Likewise, the following files |final|\textit{nn}|.tex|
compile the final version of the child document
|child|\textit{nn}|.tex|:
%
\begin{center}
\begin{tabular}{l}
|\def\version{final}|\\
|\input{childdoc.def}|\\
|\childdocforwardprefix{final}{child}|
\end{tabular}
\end{center}
%

Note that when several versions of a main file and/or of each child file
are to be generated, it may be convenient to set up a |Makefile| or
shell script to automatise the process.

%%%%%%%%%%%%%%%%%%%%%%%%%%%%%%%%%%%%%%%%%%%%%%%%%%%%%%%%%%%%%%%%%%%%%%%%%%%%%%%%
\subsection{Command Line Processing}
\label{sec:commandline}

The effect of redirection files can also be achieved by invoking
the \LaTeX{} compiler with a more elaborate command line.
Most conveniently this should be done as part
of a shell script or a |Makefile|.

When using \textsf{childdoc} in the main file, the following
command lines effectively perform a redirection
(note that depending on the shell being used,
backslashes may have to be doubled: `|\|' $\to$ `|\\|'):
%
\begin{center}
|... -jobname "|\textit{target}|" |\\|"|[\textit{flags}]%
|\input{childdoc.def}\childdocforward[|\textit{main}|]{|\textit{dest}|}"|
\end{center}
%
Here \textit{target} is the name of the output file,
\textit{main} is the name of the main file
and \textit{dest} is the name of the main or child file to be processed
(all filenames without extensions).
The optional argument \textit{main} can be omitted
if \textit{main} matches \textit{dest}.
Optionally, compilation \textit{flags} can be defined via |\def| commands.
This command line makes the \TeX{} engine believe
it is compiling the file \textit{target}
whose content is specified as the latter parameter.
The provided code then forwards the processing to
\textit{main} or \textit{dest} as described in \secref{sec:forward}.

%%%%%%%%%%%%%%%%%%%%%%%%%%%%%%%%%%%%%%%%%%%%%%%%%%%%%%%%%%%%%%%%%%%%%%%%%%%%%%%%
\subsection{Include by Input}
\label{sec:input}

Including child documents by |\include| has some restrictions by design.
Most notably, the content of a child document always occupies
its own set of pages; pages cannot be shared between child documents.
Usually, this behaviour makes perfect sense
because each child document contain an essential part of the document.
However, in some situations it may be desirable to compose
a document from a collection of parts
without having mandatory page breaks between then.
For this case, the package
provides a mechanism to include parts
by |\input| which can also be processed individually.
However, by construction this mechanism
requires manual handling of the content to be output.

%%%%%%%%%%%%%%%%%%%%%%%%%%%%%%%%%%%%%%%%
\DescribeMacro{\ifchilddocmanual}
The main file should be prepared as usual, see \secref{sec:include}.
However, the document body must make a distinction
between processing of an individual part and of the main document, e.g.:
%
\begin{center}
\begin{tabular}{l}
|\ifchilddocmanual|\\
|\input{\childdocname}|\\
|\||else|\\
\textit{document body with }|\input{|\textit{part}|}|\\
|\||fi|
\end{tabular}
\end{center}
%
The conditional |\ifchilddocmanual| is true whenever
a part to be included by |\input| is being compiled,
and the name of the part is stored in |\childdocname|.

%%%%%%%%%%%%%%%%%%%%%%%%%%%%%%%%%%%%%%%%
\DescribeMacro{\childdocby}
Each part to be included by |\input| should start with:
%
\begin{center}
\begin{tabular}{l}
|\input{childdoc.def}|\\
|\childdocby{|\textit{main}|}|\\
\end{tabular}
\end{center}
%
The directive |\childdocby| is similar to |\childdocof|
described in \secref{sec:include},
but the subsequent selection of content must be done manually.
To that end, both |\ifchilddoc| and |\ifchilddocmanual|
will be true upon processing of a part,
and the name of the part is stored in |\childdocname|.
Note that |\jobname| will be set to the filename of the current part
so that each part receives an individual |.aux| file
that does not interfere with the |.aux| file(s) of the main document.
This behaviour can be altered by the alternative form
|\childdocby[*]{|\textit{main}|}| (with a non-empty optional argument)
which uses the |.aux| file of the main document
by setting |\jobname| to \textit{main}.

%%%%%%%%%%%%%%%%%%%%%%%%%%%%%%%%%%%%%%%%%%%%%%%%%%%%%%%%%%%%%%%%%%%%%%%%%%%%%%%%
\subsection{Driver Development}
\label{sec:driver}

The \textsf{childdoc} mechanism can also be use for the development
of definition files such as \LaTeX{} styles or classes.
This case differs from the above setup with multiple parts
included by |\include| in that no |\includeonly| should be invoked.
This can be achieved by starting the include file
(before |\ProvidesPackage|) with:
%
\begin{center}
\begin{tabular}{l}
|\input{childdoc.def}|\\
|\childdocforward{|\textit{main}|}|\\
\end{tabular}
\end{center}
%
or alternatively with:
%
\begin{center}
\begin{tabular}{l}
|\input{childdoc.def}|\\
|\childdocby{|\textit{main}|}|\\
\end{tabular}
\end{center}
%
Both forms have slightly different effects as described above.
The main file is prepared as usual, see \secref{sec:include}.

%%%%%%%%%%%%%%%%%%%%%%%%%%%%%%%%%%%%%%%%%%%%%%%%%%%%%%%%%%%%%%%%%%%%%%%%%%%%%%%%
\subsection{Legacy Detection}
\label{sec:detection}

The directive |\childdocmain| in the main file can detect
whether the complete document or merely a child is to be compiled
even without using the directive |\childdocof|.
This method is deprecated because it is less robust
and there is no compelling reason to use it;
it is merely provided for backward compatibility
and it may be removed in future versions.

If the detection mechanism is to be used,
it is mandatory to correctly specify
the filename of the main file as the argument of |\childdocmain|:
%
\begin{center}
\begin{tabular}{l}
|\input{childdoc.def}|\\
|\childdocmain{|\textit{main}|}|\\
\end{tabular}
\end{center}
%
If |\jobname| does not match the argument \textit{main} of |\childdocmain|,
it is assumed that |\jobname| points to the child file to be compiled.
When using |\childdocmain| with the main file specified as argument,
it suffices to start a child file
with just |\input{|\textit{main}|}|
without loading of the package and using |\childdocof|.
If instead all processing is done
with the appropriate \textsf{childdoc} directives,
the argument of \textit{main} of |\childdocmain| can be empty.

An alternative version of the command line processing described
in \secref{sec:commandline} using the detection mechanism reads:
%
\begin{center}
|... -jobname "|\textit{target}|" "|[\textit{flags}]%
[|\def\jobname{|\textit{dest}|}|]|\input{|\textit{main}|}"|
\end{center}

%%%%%%%%%%%%%%%%%%%%%%%%%%%%%%%%%%%%%%%%%%%%%%%%%%%%%%%%%%%%%%%%%%%%%%%%%%%%%%%%
\subsection{Manual Code}
\label{sec:manual}

In case one cannot be certain whether the definitions file |childdoc.def|
is installed on the target \TeX{} distribution
and one prefers not to ship it,
it is conceivable to paste a few relevant commands into the sources.

To that end, drop all statements |\input{childdoc.def}|
and perform the replacements as outlined below.
Instead of |\childdocmain{|\textit{main}|}| add the following code
to the top of the main file:
%
\begin{center}
\begin{tabular}{l}
|\||ifdefined\childdocname\endinput\||fi\newif\ifchilddoc|\\
|\edef\childdocname{\scantokens\expandafter{\jobname\noexpand}}|\\
|\def\childdocmain{|\textit{main}|}\||ifx\childdocmain\childdocname\||else|\\
|\childdoctrue\includeonly{\childdocname}\let\jobname\childdocmain\||fi|\\
\end{tabular}
\end{center}
%
Instead of |\childdocof{|\textit{main}|}| just include the main file
at the top of each child file:
%
\begin{center}
|\input{|\textit{main}|}|
\end{center}
%
A simple redirection |\childdocforward{|\textit{dest}|}| is achieved by:
%
\begin{center}
|\def\jobname{|\textit{dest}|}\input{\jobname}|
\end{center}
%
The redirection with prefix
|\childdocforwardprefix[|\textit{prefix}|]{|\textit{dest}|}|
is accomplished by:
%
\begin{center}
\begin{tabular}{l}
|{\edef\jobname{\scantokens\expandafter{\jobname\noexpand}}|\\
|\def\redirectjob |\textit{prefix}|#1~~~{\gdef\jobname{|\textit{dest}|#1}}|\\
|\expandafter\redirectjob\jobname~~~}\input{\jobname}|
\end{tabular}
\end{center}

In an alternative approach,
child documents can be compiled by a specific command line
without additional code or specific definitions:
%
\begin{center}
|... -jobname "|\textit{target}|" "|[\textit{flags}]%
|\includeonly{|\textit{dest}|}\input{|\textit{main}|}"|
\end{center}
%

%%%%%%%%%%%%%%%%%%%%%%%%%%%%%%%%%%%%%%%%%%%%%%%%%%%%%%%%%%%%%%%%%%%%%%%%%%%%%%%%
%%%%%%%%%%%%%%%%%%%%%%%%%%%%%%%%%%%%%%%%%%%%%%%%%%%%%%%%%%%%%%%%%%%%%%%%%%%%%%%%
\section{Information}

%%%%%%%%%%%%%%%%%%%%%%%%%%%%%%%%%%%%%%%%%%%%%%%%%%%%%%%%%%%%%%%%%%%%%%%%%%%%%%%%
\subsection{Copyright}

Copyright \copyright{} 2017--2018 Niklas Beisert

This work may be distributed and/or modified under the
conditions of the \LaTeX{} Project Public License, either version 1.3
of this license or (at your option) any later version.
The latest version of this license is in
  \url{http://www.latex-project.org/lppl.txt}
and version 1.3 or later is part of all distributions of \LaTeX{}
version 2005/12/01 or later.

This work has the LPPL maintenance status `maintained'.

The Current Maintainer of this work is Niklas Beisert.

This work consists of the files |README.txt|, |childdoc.ins| and |childdoc.dtx|
as well as the derived files |childdoc.def|, |cdocsamp.tex|
with |cdocsch1.tex|, |cdocsch2.tex|, |cdocspt3.tex|, |cdocspt4.tex|,
|cdocsdrf.tex|, |cdocsfn1.tex|, |cdocsfn2.tex|
as well as |childdoc.pdf|.

%%%%%%%%%%%%%%%%%%%%%%%%%%%%%%%%%%%%%%%%%%%%%%%%%%%%%%%%%%%%%%%%%%%%%%%%%%%%%%%%
\subsection{Files and Installation}

The package consists of the files:
%
\begin{center}
\begin{tabular}{ll}
    |README.txt|   & readme file \\
    |childdoc.ins| & installation file \\
    |childdoc.dtx| & source file \\
    |childdoc.def| & definition file \\
    |cdocsamp.tex| & sample main file \\
    |cdocsch1.tex| & sample include file \\
    |cdocsch2.tex| & sample include file \\
    |cdocspt3.tex| & sample part file \\
    |cdocspt4.tex| & sample part file \\
    |cdocsdrf.tex| & sample redirection file \\
    |cdocsfn1.tex| & sample redirection file \\
    |cdocsfn2.tex| & sample redirection file \\
    |childdoc.pdf| & manual
\end{tabular}
\end{center}
%
The distribution consists of the files
|README.txt|, |childdoc.ins| and |childdoc.dtx|.
%
\begin{itemize}
\item
Run (pdf)\LaTeX{} on |childdoc.dtx|
to compile the manual |childdoc.pdf| (this file).
\item
Run \LaTeX{} on |childdoc.ins| to create the definitions file |childdoc.def|
and the sample |cdocsamp.tex| with include files
|cdocsch1.tex|, |cdocsch2.tex|, |cdocspt3.tex|, |cdocspt4.tex|,
|cdocsdrf.tex|, |cdocsfn1.tex|, |cdocsfn2.tex|.
Then copy the file |childdoc.def| to an appropriate directory of your \LaTeX{}
distribution, e.g.\ \textit{texmf-root}|/tex/latex/childdoc|.
\end{itemize}

%%%%%%%%%%%%%%%%%%%%%%%%%%%%%%%%%%%%%%%%%%%%%%%%%%%%%%%%%%%%%%%%%%%%%%%%%%%%%%%%
\subsection{Related CTAN Packages}

There are several other packages which offer a similar functionality:
%
\begin{itemize}
\item
The packages
\href{http://ctan.org/pkg/docmute}{\textsf{docmute}},
\href{http://ctan.org/pkg/includex}{\textsf{includex}} and
\href{http://ctan.org/pkg/standalone}{\textsf{standalone}}
provide commands to include only the document body of
a child file thus allowing both files to be compiled individually.
\item
The packages \href{http://ctan.org/pkg/subdocs}{\textsf{subdocs}}
and \href{http://ctan.org/pkg/subfiles}{\textsf{subfiles}}
provide structures in which the main and child documents can be
encapsulated and allowing them to be compiled individually.
The inclusion mechanism is different from the conventional |\include|.
\item
The package \href{http://ctan.org/pkg/combine}{\textsf{combine}}
is an elaborate solution to combine several documents into one.
\end{itemize}
%
See also the CTAN topic \href{http://ctan.org/topic/subdocs}{\textsf{subdocs}}
for further related packages.
The present package differs from the above solutions in that
a document structure constructed with the conventional |\include| mechanism
just needs two extra commands at the top of every file
such that all constituent files can be compiled individually.

%%%%%%%%%%%%%%%%%%%%%%%%%%%%%%%%%%%%%%%%%%%%%%%%%%%%%%%%%%%%%%%%%%%%%%%%%%%%%%%%
%\subsection{Feature Suggestions}
%
%The following is a list of features which may be useful for future
%versions of this package:
%%
%\begin{itemize}
%\item
%\ldots
%\end{itemize}

%%%%%%%%%%%%%%%%%%%%%%%%%%%%%%%%%%%%%%%%%%%%%%%%%%%%%%%%%%%%%%%%%%%%%%%%%%%%%%%%
\subsection{Revision History}

%%%%%%%%%%%%%%%%%%%%%%%%%%%%%%%%%%%%%%%%
\paragraph{v2.0:} 2018/12/30

\begin{itemize}
\item
immediate forward processing
\item
added |\childdocby| mechanism
\item
manual restructured
\end{itemize}

%%%%%%%%%%%%%%%%%%%%%%%%%%%%%%%%%%%%%%%%
\paragraph{v1.6:} 2018/01/17

\begin{itemize}
\item
application for development of include files
\item
corrections to manual
\end{itemize}

%%%%%%%%%%%%%%%%%%%%%%%%%%%%%%%%%%%%%%%%
\paragraph{v1.5:} 2017/05/21

\begin{itemize}
\item
more complete structuring introduced
\item
|\childdocof| introduced
\item
|\childdoc| renamed to |\childdocmain|
\item
|\childredirect| renamed to |\childdocforward| and |\childdocforwardprefix|
and functionality expanded
\end{itemize}

%%%%%%%%%%%%%%%%%%%%%%%%%%%%%%%%%%%%%%%%
\paragraph{v1.0:} 2017/04/27

\begin{itemize}
\item
manual and install package
\item
first version published on CTAN
\end{itemize}

%%%%%%%%%%%%%%%%%%%%%%%%%%%%%%%%%%%%%%%%
\paragraph{v0.6:} 2017/04/26

\begin{itemize}
\item
redirection mechanism added
\end{itemize}

%%%%%%%%%%%%%%%%%%%%%%%%%%%%%%%%%%%%%%%%
\paragraph{v0.5:} 2017/04/26

\begin{itemize}
\item
functionality in definition file
\end{itemize}


%%%%%%%%%%%%%%%%%%%%%%%%%%%%%%%%%%%%%%%%%%%%%%%%%%%%%%%%%%%%%%%%%%%%%%%%%%%%%%%%
%%%%%%%%%%%%%%%%%%%%%%%%%%%%%%%%%%%%%%%%%%%%%%%%%%%%%%%%%%%%%%%%%%%%%%%%%%%%%%%%
%%%%%%%%%%%%%%%%%%%%%%%%%%%%%%%%%%%%%%%%%%%%%%%%%%%%%%%%%%%%%%%%%%%%%%%%%%%%%%%%
\appendix

\settowidth\MacroIndent{\rmfamily\scriptsize 000\ }

 \DocInput{childdoc.dtx}

\end{document}
%</driver>
% \fi
%
% %%%%%%%%%%%%%%%%%%%%%%%%%%%%%%%%%%%%%%%%%%%%%%%%%%%%%%%%%%%%%%%%%%%%%%%%%%%%%%
% %%%%%%%%%%%%%%%%%%%%%%%%%%%%%%%%%%%%%%%%%%%%%%%%%%%%%%%%%%%%%%%%%%%%%%%%%%%%%%
% \section{Sample}
%\iffalse
%<*samplemain>
%\fi
%
% The following presents a sample document
% with two chapters, two parts, a title page,
% a compile flag as well as three forwarding files to set the flag.
% It consists of eight |.tex| files:
% \begin{center}
% \begin{tabular}{ll}
% |cdocsamp.tex|&main file\\
% |cdocsch1.tex|&include file for chapter 1\\
% |cdocsch2.tex|&include file for chapter 2\\
% |cdocspt3.tex|&include file for part 3\\
% |cdocspt4.tex|&include file for part 4\\
% |cdocsdrf.tex|&forwarding file for main file in draft mode\\
% |cdocsfi1.tex|&forwarding file for final version of chapter 1\\
% |cdocsfi2.tex|&forwarding file for final version of chapter 2\\
% \end{tabular}
% \end{center}
% Each of the eight files can be compiled directly by the \LaTeX{} compiler.
%
% %%%%%%%%%%%%%%%%%%%%%%%%%%%%%%%%%%%%%%
% \paragraph{Main File.}
%
% The main file is called |cdocsamp.tex|.
%
% Load the \textsf{childdoc} definitions and
% declare the filename for the main document:
%    \begin{macrocode}
\input{childdoc.def}
\childdocmain{}
%    \end{macrocode}

% Optional override for |\version| flag:
%    \begin{macrocode}
%%\ifchilddoc\else\providecommand{\version}{draft}\fi
%    \end{macrocode}

% Define the default values for the |\version| flag
% (|final| for the main file and |draft| for childs):
%    \begin{macrocode}
\ifchilddoc
\providecommand{\version}{draft}
\else
\providecommand{\version}{final}
\fi
%    \end{macrocode}

% Load the standard document class:
%    \begin{macrocode}
\documentclass[12pt]{article}
%    \end{macrocode}

% Start the document body:
%    \begin{macrocode}
\begin{document}
%    \end{macrocode}

% Declare a title page.
% Print title, part of document being processed and version flag:
%    \begin{macrocode}
\addtocounter{page}{-1}
\begin{center}
{\LARGE\bfseries{}childdoc example\par}
\vspace{1cm}
\ifchilddoc
\ifchilddocmanual part\else chapter\fi:
`\childdocname' of `\childdocjob'\par
\else
main document: `\childdocjob'\par
\fi
version: \version\par
\end{center}
\newpage
%    \end{macrocode}

% Manually include selected file,
% otherwise process as usual:
%    \begin{macrocode}
\ifchilddocmanual
\section*{part `\childdocname'}
\input{\childdocname}
\else
%    \end{macrocode}

% Include the two chapters:
%    \begin{macrocode}
\include{cdocsch1}
\include{cdocsch2}
%    \end{macrocode}

% Include the two parts unless only chapters should be displayed:
%    \begin{macrocode}
\ifchilddoc\else
\section{part three}
\input{cdocspt3}
\section{part four}
\input{cdocspt4}
\fi
%    \end{macrocode}

% Process as usual until here:
%    \begin{macrocode}
\fi
%    \end{macrocode}

% End of document body:
%    \begin{macrocode}
\end{document}
%    \end{macrocode}
%\iffalse
%</samplemain>
%\fi
%
% %%%%%%%%%%%%%%%%%%%%%%%%%%%%%%%%%%%%%%
% \paragraph{Chapter Include Files.}
%
% The include files are called |cdocsch1.tex| and |cdocsch2.tex|.
%
%\iffalse
%<*samplechap1|samplechap2>
%\fi

% Optional override for |\version| flag:
%    \begin{macrocode}
%%\providecommand{\version}{final}
%    \end{macrocode}

% Include the main document:
%    \begin{macrocode}
\input{childdoc.def}
\childdocof{cdocsamp}
%    \end{macrocode}

%\iffalse
%</samplechap1|samplechap2>
%\fi
%
%\iffalse
%<*samplechap1>
%\fi
% Some text for chapter 1:
%    \begin{macrocode}
\section{one}
some text in chapter one
%    \end{macrocode}

%\iffalse
%</samplechap1>
%\fi
% Some text for chapter 2:
%\iffalse
%<*samplechap2>
%\fi
%    \begin{macrocode}
\section{two}
more text in chapter two
%    \end{macrocode}

%\iffalse
%</samplechap2>
%\fi
%
% %%%%%%%%%%%%%%%%%%%%%%%%%%%%%%%%%%%%%%
% \paragraph{Part Include Files.}
%
% The include files are called |cdocspt3.tex| and |cdocspt4.tex|.
%
%\iffalse
%<*samplepart3|samplepart4>
%\fi

% Optional override for |\version| flag:
%    \begin{macrocode}
%%\providecommand{\version}{final}
%    \end{macrocode}

% Include the main document:
%    \begin{macrocode}
\input{childdoc.def}
\childdocby{cdocsamp}
%    \end{macrocode}

%\iffalse
%</samplepart3|samplepart4>
%\fi
%
%\iffalse
%<*samplepart3>
%\fi
% Some text for part 3:
%    \begin{macrocode}
some text in part three
%    \end{macrocode}

%\iffalse
%</samplepart3>
%\fi
% Some text for part 4:
%\iffalse
%<*samplepart4>
%\fi
%    \begin{macrocode}
more text in part four
%    \end{macrocode}

%\iffalse
%</samplepart4>
%\fi
%
% %%%%%%%%%%%%%%%%%%%%%%%%%%%%%%%%%%%%%%
% \paragraph{Forwarding for a Complete Draft.}
%
% The following forwarding file |cdocsdrf.tex|
% compiles the main document in draft mode:
%\iffalse
%<*sampledraft>
%\fi
%    \begin{macrocode}
\def\version{draft}
\input{childdoc.def}
\childdocforward{cdocsamp}
%    \end{macrocode}

%\iffalse
%</sampledraft>
%\fi
%
% %%%%%%%%%%%%%%%%%%%%%%%%%%%%%%%%%%%%%%
% \paragraph{Forwarding for Final Version of the Chapters.}
%
% The following forwarding files |cdocsfn1.tex| and |cdocsfn2.tex|
% (with identical content)
% compile the final versions of the child documents
% |cdocsch1.tex| and |cdocsch2.tex|, respectively:
%\iffalse
%<*samplefinal>
%\fi
%    \begin{macrocode}
\def\version{final}
\input{childdoc.def}
\childdocforwardprefix[cdocsamp]{cdocsfn}{cdocsch}
%    \end{macrocode}

%\iffalse
%</samplefinal>
%\fi
%
% %%%%%%%%%%%%%%%%%%%%%%%%%%%%%%%%%%%%%%
% \paragraph{Command Line Processing.}
%
% The following three command lines generate the output files
% |cdocscld|, |cdocscl1| and |cdocscl2|
% which should be identical to
% |cdocsdrf|, |cdocsch1| and |cdocsfn2|, respectively:
% \begin{center}
% \begin{tabular}{l}
% |latex -jobname cdocscld \|\\
% |  "\def\version{draft}\input{childdoc.def}\childdocforward{cdocsamp}"|\\
% |latex -jobname cdocscl1 \|\\
% |  "\input{childdoc.def}\childdocforward[cdocsamp]{cdocsch1}"|\\
% |latex -jobname cdocscl2 \|\\
% |  "\def\version{final}\input{childdoc.def}\childdocforward{cdocsch2}"|
% \end{tabular}
% \end{center}
% Note that the trailing backslash on each first line
% merely continues the input to the second line
% (for convenient cut ant paste).
% Furthermore, the command |latex| can be replaced by any
% of its alternative versions such as |pdflatex|.
%
% %%%%%%%%%%%%%%%%%%%%%%%%%%%%%%%%%%%%%%%%%%%%%%%%%%%%%%%%%%%%%%%%%%%%%%%%%%%%%%
% %%%%%%%%%%%%%%%%%%%%%%%%%%%%%%%%%%%%%%%%%%%%%%%%%%%%%%%%%%%%%%%%%%%%%%%%%%%%%%
% \section{Implementation}
%\iffalse
%<*package>
%\fi
%
% This section describes the definitions file |childdoc.def|.

% The definitions cannot be loaded using |\usepackage| or |\RequirePackage|
% which has a mechanism to prevent loading a style file more than once.
% When loading the definitions by means of |\input|
% multiple instances have to be prevented manually:
%\iffalse
%This code needs to be before the `\ProvidesFile' directive
%which is defined at the beginning of this file.
%Therefore it is also placed there and commented out here.
%</package>
%<*discard>
%\fi
%    \begin{macrocode}
\ifdefined\childdocmain\endinput\fi
%    \end{macrocode}
%\iffalse
%</discard>
%<*package>
%\fi
%
% \macro{\ifchilddoc}
% \macro{\ifchilddocmanual}
% The conditional |\ifchilddoc| tells whether a
% child (true) or main (false) document is being compiled.
% The conditional |\ifchilddocmanual| tells whether
% the |\includeonly| mechanism is used (false) or
% the selection of child files must be performed manually (true).
% The definitions initialise to false:
%    \begin{macrocode}
\newif\ifchilddoc
\newif\ifchilddocmanual
%    \end{macrocode}

% \macro{\childdocname}
% \macro{\childdocjob}
% The macro |\childdocname| stores the name of the main document
% to be compiled. The macro |\childdocjob| stores the name of
% the document on which the \LaTeX{} compiler was originally invoked.
% The content of |\jobname| cannot be compared
% to filenames specified in the source due to different catcodes.
% The following code rescans |\jobname|, stores the result
% in |\childdocname| and saves a copy in |\childdocjob|:
%    \begin{macrocode}
\edef\childdocname{\scantokens\expandafter{\jobname\noexpand}}
\let\childdocjob\childdocname
%    \end{macrocode}

% \macro{\childdocdisable}
% The macro |\childdocdisable| prevents the main file
% from being processed more than once.
% At this stage, the main document command |\childdocmain|
% is assumed to be called once again where it should do nothing.
% Any subsequent call to it should prevent
% a secondary processing of the main document
% It overwrites the forwarding commands
% |\childdocof| and |\childdocforward|
% with empty macros to prevent further inclusions of the main document:
%    \begin{macrocode}
\newcommand{\childdocdisable}
{
  \renewcommand{\childdocmain}[1]{\renewcommand{\childdocmain}[1]{\endinput}}
  \renewcommand{\childdocof}[1]{}
  \renewcommand{\childdocby}[2][]{}
  \renewcommand{\childdocforward}[2][]{}
  \renewcommand{\childdocdisable}{}
}
%    \end{macrocode}

% \macro{\childdocmain}
% The macro |\childdocmain| is to be called at the top of the main file
% with nothing or the main filename (without extension) as argument.
% First, it breaks loops.
% If the argument is not empty and does not match |\childdocname|
% (which is set by the first inclusion of |childdoc.def|),
% |\ifchilddoc| is set to true, |\includeonly| is applied to the child file
% and |\jobname| is set to the main file
% (for proper handling of |.aux| files):
%    \begin{macrocode}
\newcommand{\childdocmain}[1]
{
  \childdocdisable\childdocmain{}
  \if?#1?\else
    \begingroup
      \def\childdoctmp{#1}
      \ifx\childdoctmp\childdocname
        \def\childdoctmp{}
      \else
        \def\childdoctmp
        {
          \childdoctrue
          \includeonly{\childdocname}
          \def\childdocjob{#1}
          \def\jobname{#1}
        }
      \fi
      \expandafter
    \endgroup
    \childdoctmp
  \fi
}
%    \end{macrocode}

% \macro{\childdocof}
% The command |\childdocof| redirects
% compilation to the main file |#1|.
%    \begin{macrocode}
\newcommand{\childdocof}[1]
{
  \childdocdisable
  \childdoctrue
  \includeonly{\childdocname}
  \def\jobname{#1}
  \def\childdocjob{#1}
  \input{#1}
}
%    \end{macrocode}

% \macro{\childdocby}
% The command |\childdocby| ....
%    \begin{macrocode}
\newcommand{\childdocby}[2][]
{
  \childdocdisable
  \childdoctrue
  \childdocmanualtrue
  \if?#1?\else
    \def\jobname{#2}
  \fi
  \def\childdocjob{#2}
  \input{#2}
  \endinput
}
%    \end{macrocode}

% \macro{\childdocforward}
% The command |\childdocforward| redirects
% compilation to the main file or
% (if the optional argument is given) a child file.
% Parameters are set as if the main file
% or a child file starting with |\childdocof| was compiled.
% Then compilation is handed over to the main file:
%    \begin{macrocode}
\newcommand{\childdocforward}[2][]
{
  \begingroup
    \if?#1?
      \def\childdoctmp
      {
        \def\childdocname{#2}
        \def\childdocjob{#2}
        \def\jobname{#2}
        \input{#2}
        \endinput
      }
    \else
      \def\childdoctmp
      {
        \childdocdisable
        \def\childdocname{#2}
        \childdoctrue
        \includeonly{#2}
        \def\childdocjob{#1}
        \def\jobname{#1}
        \input{#1}
        \endinput
      }
    \fi
    \expandafter
  \endgroup
  \childdoctmp
}
%    \end{macrocode}

% \macro{\childdocforwardprefix}
% The command |\childdocforwardprefix| redirects
% compilation to the main or a child file by means of a pattern.
% The prefix |#1| in the current filename is replaced by |#2|
% and the suffix of the current filename is kept
% (it is assumed that the filename does not contain the substring `|~~~|'
% which is used as a delimiter).
% Compilation is handed over to the new file by |\childdocforward|:
%    \begin{macrocode}
\newcommand{\childdocforwardprefix}[3][]
{
  \begingroup
    \def\childdocextract #2##1~~~{\def\childdoctmp{\childdocforward[#1]{#3##1}}}
    \expandafter\childdocextract\childdocname~~~
    \expandafter
  \endgroup
  \childdoctmp
}
%    \end{macrocode}

% \macro{\childdoc}
% The deprecated macro |\childdoc| is a legacy version of |\childdocmain|:
%    \begin{macrocode}
\newcommand{\childdoc}{\childdocmain}
%    \end{macrocode}

% \macro{\childdocredirect}
% The deprecated macro |\childdocredirect| is a legacy version
% of |\childdocforward| and |\childdocforwardprefix|:
%    \begin{macrocode}
\newcommand{\childdocredirect}[2][]
{
  \begingroup
    \if?#1?
      \def\childdoctmp{\childdocforward{#2}}
    \else
      \def\childdoctmp{\childdocforwardprefix{#1}{#2}}
    \fi
    \expandafter
  \endgroup
  \childdoctmp
}
%    \end{macrocode}

%\iffalse
%</package>
%\fi
%
\endinput
|
and perform the replacements as outlined below.
Instead of |\childdocmain{|\textit{main}|}| add the following code
to the top of the main file:
%
\begin{center}
\begin{tabular}{l}
|\||ifdefined\childdocname\endinput\||fi\newif\ifchilddoc|\\
|\edef\childdocname{\scantokens\expandafter{\jobname\noexpand}}|\\
|\def\childdocmain{|\textit{main}|}\||ifx\childdocmain\childdocname\||else|\\
|\childdoctrue\includeonly{\childdocname}\let\jobname\childdocmain\||fi|\\
\end{tabular}
\end{center}
%
Instead of |\childdocof{|\textit{main}|}| just include the main file
at the top of each child file:
%
\begin{center}
|\input{|\textit{main}|}|
\end{center}
%
A simple redirection |\childdocforward{|\textit{dest}|}| is achieved by:
%
\begin{center}
|\def\jobname{|\textit{dest}|}\input{\jobname}|
\end{center}
%
The redirection with prefix
|\childdocforwardprefix[|\textit{prefix}|]{|\textit{dest}|}|
is accomplished by:
%
\begin{center}
\begin{tabular}{l}
|{\edef\jobname{\scantokens\expandafter{\jobname\noexpand}}|\\
|\def\redirectjob |\textit{prefix}|#1~~~{\gdef\jobname{|\textit{dest}|#1}}|\\
|\expandafter\redirectjob\jobname~~~}\input{\jobname}|
\end{tabular}
\end{center}

In an alternative approach,
child documents can be compiled by a specific command line
without additional code or specific definitions:
%
\begin{center}
|... -jobname "|\textit{target}|" "|[\textit{flags}]%
|\includeonly{|\textit{dest}|}\input{|\textit{main}|}"|
\end{center}
%

%%%%%%%%%%%%%%%%%%%%%%%%%%%%%%%%%%%%%%%%%%%%%%%%%%%%%%%%%%%%%%%%%%%%%%%%%%%%%%%%
%%%%%%%%%%%%%%%%%%%%%%%%%%%%%%%%%%%%%%%%%%%%%%%%%%%%%%%%%%%%%%%%%%%%%%%%%%%%%%%%
\section{Information}

%%%%%%%%%%%%%%%%%%%%%%%%%%%%%%%%%%%%%%%%%%%%%%%%%%%%%%%%%%%%%%%%%%%%%%%%%%%%%%%%
\subsection{Copyright}

Copyright \copyright{} 2017--2018 Niklas Beisert

This work may be distributed and/or modified under the
conditions of the \LaTeX{} Project Public License, either version 1.3
of this license or (at your option) any later version.
The latest version of this license is in
  \url{http://www.latex-project.org/lppl.txt}
and version 1.3 or later is part of all distributions of \LaTeX{}
version 2005/12/01 or later.

This work has the LPPL maintenance status `maintained'.

The Current Maintainer of this work is Niklas Beisert.

This work consists of the files |README.txt|, |childdoc.ins| and |childdoc.dtx|
as well as the derived files |childdoc.def|, |cdocsamp.tex|
with |cdocsch1.tex|, |cdocsch2.tex|, |cdocspt3.tex|, |cdocspt4.tex|,
|cdocsdrf.tex|, |cdocsfn1.tex|, |cdocsfn2.tex|
as well as |childdoc.pdf|.

%%%%%%%%%%%%%%%%%%%%%%%%%%%%%%%%%%%%%%%%%%%%%%%%%%%%%%%%%%%%%%%%%%%%%%%%%%%%%%%%
\subsection{Files and Installation}

The package consists of the files:
%
\begin{center}
\begin{tabular}{ll}
    |README.txt|   & readme file \\
    |childdoc.ins| & installation file \\
    |childdoc.dtx| & source file \\
    |childdoc.def| & definition file \\
    |cdocsamp.tex| & sample main file \\
    |cdocsch1.tex| & sample include file \\
    |cdocsch2.tex| & sample include file \\
    |cdocspt3.tex| & sample part file \\
    |cdocspt4.tex| & sample part file \\
    |cdocsdrf.tex| & sample redirection file \\
    |cdocsfn1.tex| & sample redirection file \\
    |cdocsfn2.tex| & sample redirection file \\
    |childdoc.pdf| & manual
\end{tabular}
\end{center}
%
The distribution consists of the files
|README.txt|, |childdoc.ins| and |childdoc.dtx|.
%
\begin{itemize}
\item
Run (pdf)\LaTeX{} on |childdoc.dtx|
to compile the manual |childdoc.pdf| (this file).
\item
Run \LaTeX{} on |childdoc.ins| to create the definitions file |childdoc.def|
and the sample |cdocsamp.tex| with include files
|cdocsch1.tex|, |cdocsch2.tex|, |cdocspt3.tex|, |cdocspt4.tex|,
|cdocsdrf.tex|, |cdocsfn1.tex|, |cdocsfn2.tex|.
Then copy the file |childdoc.def| to an appropriate directory of your \LaTeX{}
distribution, e.g.\ \textit{texmf-root}|/tex/latex/childdoc|.
\end{itemize}

%%%%%%%%%%%%%%%%%%%%%%%%%%%%%%%%%%%%%%%%%%%%%%%%%%%%%%%%%%%%%%%%%%%%%%%%%%%%%%%%
\subsection{Related CTAN Packages}

There are several other packages which offer a similar functionality:
%
\begin{itemize}
\item
The packages
\href{http://ctan.org/pkg/docmute}{\textsf{docmute}},
\href{http://ctan.org/pkg/includex}{\textsf{includex}} and
\href{http://ctan.org/pkg/standalone}{\textsf{standalone}}
provide commands to include only the document body of
a child file thus allowing both files to be compiled individually.
\item
The packages \href{http://ctan.org/pkg/subdocs}{\textsf{subdocs}}
and \href{http://ctan.org/pkg/subfiles}{\textsf{subfiles}}
provide structures in which the main and child documents can be
encapsulated and allowing them to be compiled individually.
The inclusion mechanism is different from the conventional |\include|.
\item
The package \href{http://ctan.org/pkg/combine}{\textsf{combine}}
is an elaborate solution to combine several documents into one.
\end{itemize}
%
See also the CTAN topic \href{http://ctan.org/topic/subdocs}{\textsf{subdocs}}
for further related packages.
The present package differs from the above solutions in that
a document structure constructed with the conventional |\include| mechanism
just needs two extra commands at the top of every file
such that all constituent files can be compiled individually.

%%%%%%%%%%%%%%%%%%%%%%%%%%%%%%%%%%%%%%%%%%%%%%%%%%%%%%%%%%%%%%%%%%%%%%%%%%%%%%%%
%\subsection{Feature Suggestions}
%
%The following is a list of features which may be useful for future
%versions of this package:
%%
%\begin{itemize}
%\item
%\ldots
%\end{itemize}

%%%%%%%%%%%%%%%%%%%%%%%%%%%%%%%%%%%%%%%%%%%%%%%%%%%%%%%%%%%%%%%%%%%%%%%%%%%%%%%%
\subsection{Revision History}

%%%%%%%%%%%%%%%%%%%%%%%%%%%%%%%%%%%%%%%%
\paragraph{v2.0:} 2018/12/30

\begin{itemize}
\item
immediate forward processing
\item
added |\childdocby| mechanism
\item
manual restructured
\end{itemize}

%%%%%%%%%%%%%%%%%%%%%%%%%%%%%%%%%%%%%%%%
\paragraph{v1.6:} 2018/01/17

\begin{itemize}
\item
application for development of include files
\item
corrections to manual
\end{itemize}

%%%%%%%%%%%%%%%%%%%%%%%%%%%%%%%%%%%%%%%%
\paragraph{v1.5:} 2017/05/21

\begin{itemize}
\item
more complete structuring introduced
\item
|\childdocof| introduced
\item
|\childdoc| renamed to |\childdocmain|
\item
|\childredirect| renamed to |\childdocforward| and |\childdocforwardprefix|
and functionality expanded
\end{itemize}

%%%%%%%%%%%%%%%%%%%%%%%%%%%%%%%%%%%%%%%%
\paragraph{v1.0:} 2017/04/27

\begin{itemize}
\item
manual and install package
\item
first version published on CTAN
\end{itemize}

%%%%%%%%%%%%%%%%%%%%%%%%%%%%%%%%%%%%%%%%
\paragraph{v0.6:} 2017/04/26

\begin{itemize}
\item
redirection mechanism added
\end{itemize}

%%%%%%%%%%%%%%%%%%%%%%%%%%%%%%%%%%%%%%%%
\paragraph{v0.5:} 2017/04/26

\begin{itemize}
\item
functionality in definition file
\end{itemize}


%%%%%%%%%%%%%%%%%%%%%%%%%%%%%%%%%%%%%%%%%%%%%%%%%%%%%%%%%%%%%%%%%%%%%%%%%%%%%%%%
%%%%%%%%%%%%%%%%%%%%%%%%%%%%%%%%%%%%%%%%%%%%%%%%%%%%%%%%%%%%%%%%%%%%%%%%%%%%%%%%
%%%%%%%%%%%%%%%%%%%%%%%%%%%%%%%%%%%%%%%%%%%%%%%%%%%%%%%%%%%%%%%%%%%%%%%%%%%%%%%%
\appendix

\settowidth\MacroIndent{\rmfamily\scriptsize 000\ }

 \DocInput{childdoc.dtx}

\end{document}
%</driver>
% \fi
%
% %%%%%%%%%%%%%%%%%%%%%%%%%%%%%%%%%%%%%%%%%%%%%%%%%%%%%%%%%%%%%%%%%%%%%%%%%%%%%%
% %%%%%%%%%%%%%%%%%%%%%%%%%%%%%%%%%%%%%%%%%%%%%%%%%%%%%%%%%%%%%%%%%%%%%%%%%%%%%%
% \section{Sample}
%\iffalse
%<*samplemain>
%\fi
%
% The following presents a sample document
% with two chapters, two parts, a title page,
% a compile flag as well as three forwarding files to set the flag.
% It consists of eight |.tex| files:
% \begin{center}
% \begin{tabular}{ll}
% |cdocsamp.tex|&main file\\
% |cdocsch1.tex|&include file for chapter 1\\
% |cdocsch2.tex|&include file for chapter 2\\
% |cdocspt3.tex|&include file for part 3\\
% |cdocspt4.tex|&include file for part 4\\
% |cdocsdrf.tex|&forwarding file for main file in draft mode\\
% |cdocsfi1.tex|&forwarding file for final version of chapter 1\\
% |cdocsfi2.tex|&forwarding file for final version of chapter 2\\
% \end{tabular}
% \end{center}
% Each of the eight files can be compiled directly by the \LaTeX{} compiler.
%
% %%%%%%%%%%%%%%%%%%%%%%%%%%%%%%%%%%%%%%
% \paragraph{Main File.}
%
% The main file is called |cdocsamp.tex|.
%
% Load the \textsf{childdoc} definitions and
% declare the filename for the main document:
%    \begin{macrocode}
% \iffalse
%
% childdoc.dtx Copyright (C) 2017-2018 Niklas Beisert
%
% This work may be distributed and/or modified under the
% conditions of the LaTeX Project Public License, either version 1.3
% of this license or (at your option) any later version.
% The latest version of this license is in
%   http://www.latex-project.org/lppl.txt
% and version 1.3 or later is part of all distributions of LaTeX
% version 2005/12/01 or later.
%
% This work has the LPPL maintenance status `maintained'.
%
% The Current Maintainer of this work is Niklas Beisert.
%
% This work consists of the files childdoc.dtx and childdoc.ins
% and the derived files childdoc.def and cdocsamp.tex with
% cdocsch1.tex, cdocsch2.tex, cdocsdrf.tex, cdocsfn1.tex, cdocsfn2.tex.
%
%<package>\ifdefined\childdocmain\endinput\fi
%<package>\ProvidesFile{childdoc.def}[2018/12/30 v2.0 child document driver]
%<samplemain>\ProvidesFile{cdocsamp.tex}[2018/12/30 v2.0 sample for childdoc]
%<*driver>
%\ProvidesFile{childdoc.drv}[2018/12/30 v2.0 childdoc reference manual file]
\PassOptionsToClass{10pt,a4paper}{article}
\documentclass{ltxdoc}

\usepackage[margin=35mm]{geometry}
\usepackage{hyperref}
\usepackage{hyperxmp}
\usepackage[usenames]{color}

\hypersetup{colorlinks=true}
\hypersetup{pdfstartview=FitH}
\hypersetup{pdfpagemode=UseNone}
\hypersetup{pdfsource={}}
\hypersetup{pdflang={en-UK}}
\hypersetup{pdfcopyright={Copyright 2017-2018 Niklas Beisert.
  This work may be distributed and/or modified under the
  conditions of the LaTeX Project Public License, either version 1.3
  of this license or (at your option) any later version.}}
\hypersetup{pdflicenseurl={http://www.latex-project.org/lppl.txt}}
\hypersetup{pdfcontactaddress={ETH Zurich, ITP, HIT K,
  Wolfgang-Pauli-Strasse 27}}
\hypersetup{pdfcontactpostcode={8093}}
\hypersetup{pdfcontactcity={Zurich}}
\hypersetup{pdfcontactcountry={Switzerland}}
\hypersetup{pdfcontactemail={nbeisert@itp.phys.ethz.ch}}
\hypersetup{pdfcontacturl={http://people.phys.ethz.ch/\xmptilde nbeisert/}}

\newcommand{\secref}[1]{\hyperref[#1]{section \ref*{#1}}}

\parskip1ex
\parindent0pt
\let\olditemize\itemize
\def\itemize{\olditemize\parskip0pt}

\begin{document}

\title{The \textsf{childdoc} Package}
\hypersetup{pdftitle={The childdoc Package}}
\author{Niklas Beisert\\[2ex]
  Institut f\"ur Theoretische Physik\\
  Eidgen\"ossische Technische Hochschule Z\"urich\\
  Wolfgang-Pauli-Strasse 27, 8093 Z\"urich, Switzerland\\[1ex]
  \href{mailto:nbeisert@itp.phys.ethz.ch}
  {\texttt{nbeisert@itp.phys.ethz.ch}}}
\hypersetup{pdfauthor={Niklas Beisert}}
\hypersetup{pdfsubject={Manual for the LaTeX2e Package childdoc}}
\date{30 December 2018, \textsf{v2.0}}
\maketitle

\begin{abstract}\noindent
\textsf{childdoc} is a \LaTeXe{} package
that enables the direct compilation
of document sections included by |\include|
to individual files.
\end{abstract}

\begingroup
\parskip0ex
\tableofcontents
\endgroup

%%%%%%%%%%%%%%%%%%%%%%%%%%%%%%%%%%%%%%%%%%%%%%%%%%%%%%%%%%%%%%%%%%%%%%%%%%%%%%%%
%%%%%%%%%%%%%%%%%%%%%%%%%%%%%%%%%%%%%%%%%%%%%%%%%%%%%%%%%%%%%%%%%%%%%%%%%%%%%%%%
\section{Introduction}

\LaTeX{} provides a mechanism to structure a large document (such as a book)
into a main file and several child files (containing the chapters)
using the |\include| command.
This mechanism is beneficial for documents
which span hundreds of pages in order to
make the source file(s) more manageable.
Moreover, compilation can be restricted to
selected child files by means of the |\includeonly| command.
The latter feature can be used to reduce the compilation time while editing
(this was significantly more useful in the earlier days of \LaTeX{})
or to generate a smaller document which is easier to navigate.
Another application of |\includeonly| is to generate
documents consisting of selected parts of the complete document.

However, there are a few drawbacks of the plain |\include| mechanism:
\begin{itemize}
\item
The child files cannot be compiled on their own,
they can only be compiled via the main file.
A naive editing environment
(such as a text editor with an option
to have the current file processed by \LaTeX)
may require one to switch to the main file before compiling;
attempting to compile the child file produces errors.
\item
The main file must be modified (each time)
to adjust the |\includeonly| command
to the present needs. This easily leaves the main file in a messy state.
\item
The generated document will always carry the filename
of the main document. This is inconvenient if
several child files are to be compiled and
to be kept for distribution.
\end{itemize}

The present package provides a simple interface
to make child files individually compilable by \LaTeX{}.
Compiling a child file then has the same effect as compiling
the main file with an |\includeonly| command
to select the appropriate child.
Moreover the generated document will carry the name of the child
rather than the main file.
This resolves all three above issues.

This feature is meant to make the editing of books,
thesis documents and lecture notes somewhat more convenient.
However, the package can also be used efficiently for
composing a series of documents (such as exercise sheets)
which are typically distributed individually.
It then assists the author in generating the individual documents
(potentially in different versions)
as well as a document containing the collected series.
Another application is in developing style files
or other kinds of included material
where compilation of the style file could redirect
to a sample or test file.

%%%%%%%%%%%%%%%%%%%%%%%%%%%%%%%%%%%%%%%%%%%%%%%%%%%%%%%%%%%%%%%%%%%%%%%%%%%%%%%%
%%%%%%%%%%%%%%%%%%%%%%%%%%%%%%%%%%%%%%%%%%%%%%%%%%%%%%%%%%%%%%%%%%%%%%%%%%%%%%%%
\section{Usage}

First of all, the package \textsf{childdoc} is \emph{not} a standard
\LaTeXe{} |.sty| style file! Therefore it needs to be invoked in
a non-standard way.

%%%%%%%%%%%%%%%%%%%%%%%%%%%%%%%%%%%%%%%%%%%%%%%%%%%%%%%%%%%%%%%%%%%%%%%%%%%%%%%%
\subsection{Included Files}
\label{sec:include}

%%%%%%%%%%%%%%%%%%%%%%%%%%%%%%%%%%%%%%%%
\DescribeMacro{\childdocmain}
To use the package, add the commands
\begin{center}
\begin{tabular}{l}
|\input{childdoc.def}|\\
|\childdocmain{}|\\
\end{tabular}
\end{center}
at the very top of the main \LaTeX{} file,
in particular \emph{before} the |\documentclass| statement!
The argument of |\childdocmain| should be left empty
(but it must be present).

%%%%%%%%%%%%%%%%%%%%%%%%%%%%%%%%%%%%%%%%
\DescribeMacro{\childdocof}
Furthermore, add the commands
\begin{center}
\begin{tabular}{l}
|\input{childdoc.def}|\\
|\childdocof{|\textit{main}|}|\\
\end{tabular}
\end{center}
at the top of every child file \textit{child}
which is included by |\include{|\textit{child}|}|
from within the main file
(or at least for those files to be compiled individually).
The argument \textit{main} must be the filename of the main file.

There are a couple of
considerations in setting up the main and child documents:

%%%%%%%%%%%%%%%%%%%%%%%%%%%%%%%%%%%%%%%%
\paragraph{Restrictions.}

Please note the following restrictions:
\begin{itemize}
\item
|\childdocmain| must be called with one argument \textit{main}
to ensure compatibility with earlier version of the package.
It must either be empty (|\childdocmain{}|)
or precisely match the filename of the main file in which it is specified.
See \secref{sec:detection} for further information.
\item
The filename \textit{main} must be specified without the |.tex| extension.
\item
The filename \textit{main} is case sensitive
(even in case-insensitive file systems)
due to internal string comparison.
\item
The argument \textit{main} should be fully expanded, it cannot be a macro.
\item
Subdirectories and special characters should be avoided in filenames.
\item
The command |\childdocmain{|\textit{main}|}| must be followed by a whitespace.
It should not be followed immediately by another command
or by a comment mark `|%|'.
This is because the \TeX{} parser reads the token immediately following
the argument of |\childdocmain| and puts it
at the beginning of every child section;
however, a white\-space is ignored.
\end{itemize}

%%%%%%%%%%%%%%%%%%%%%%%%%%%%%%%%%%%%%%%%
\paragraph{Content of Main File.}

It is advisable to place all content in the child files included by |\include|.
Any output contained in the main file will appear in all child documents
unless suppressed manually;
it cannot be suppressed automatically by the |\includeonly| directive
and thus should normally be avoided.
A method to include some content in the main file
by means of conditional processing is described in \secref{sec:conditional}.

%%%%%%%%%%%%%%%%%%%%%%%%%%%%%%%%%%%%%%%%
\paragraph{Page Numbering.}

When only a part of the document is compiled,
the appropriate numbering of pages
(as well as other status parameters)
is determined from the |.aux| files.
The latter contain information from previous passes.
However this information needs to propagate through
all intermediate child documents.
Therefore the page numbering in child documents may well
be inconsistent until the complete document is compiled at least once.

A useful (if unconventional) way to always ensure a consistent
page numbering is to restart the numbering in each child document
and denote the pages by `\textit{child}|.|\textit{page}'
where \textit{child} represents the chapter/section number of the child file.
This can be achieved by the command
|\numberwithin{page}{|\textit{child}|}|
of the \textsf{amsmath} package
where \textit{child} can be |chapter| or |section|
depending on the chosen structuring.
Alternatively, one can modify the macro |\thepage| appropriately
and reset the counter |page| at the start of each child file.

%%%%%%%%%%%%%%%%%%%%%%%%%%%%%%%%%%%%%%%%%%%%%%%%%%%%%%%%%%%%%%%%%%%%%%%%%%%%%%%%
\subsection{Conditional Processing}
\label{sec:conditional}

The package provides a mechanism to compile different versions
of a document. To customise the versions further some conditional processing
can come in handy to distinguish which version is being compiled.
The package provides two macros to describe the compilation context:

%%%%%%%%%%%%%%%%%%%%%%%%%%%%%%%%%%%%%%%%
\DescribeMacro{\ifchilddoc}
The conditional |\ifchilddoc| distinguishes between the compilation of
child documents and the main document:
%
\begin{center}
|\ifchilddoc |\textit{child-code}| |[|\||else |\textit{main-code}]| \||fi|
\end{center}

%%%%%%%%%%%%%%%%%%%%%%%%%%%%%%%%%%%%%%%%
\DescribeMacro{\childdocname}
\DescribeMacro{\childdocjob}
The macro |\childdocname| contains the filename (without extension)
of the main or child file being processed.
Note that |\childdocjob| will always contain the name of the main file.

%%%%%%%%%%%%%%%%%%%%%%%%%%%%%%%%%%%%%%%%
\paragraph{Title Page.}

Conditional processing can be used to include a title or banner page
in the main document when proper precautions are taken.
Importantly, the code in the main file should ensure that the page counter
(as well as other status parameters which are stored in the |.aux| files)
takes the same value after the conditional processing.
Otherwise the page numbers may take divergent values
depending on which part is compiled.

For example, a title page could be declared by:
%
\begin{center}
\begin{tabular}{l}
|\ifchilddoc\||else|\\
|\addtocounter{page}{-1}|\\
\textit{code for title page}\\
|\newpage|\\
|\||fi|
\end{tabular}
\end{center}
%
A banner page for the child documents can be generated by:
%
\begin{center}
\begin{tabular}{l}
|\ifchilddoc|\\
|\addtocounter{page}{-1}|\\
\textit{code for banner page}\\
|\newpage|\\
|\||fi|
\end{tabular}
\end{center}
%
Here one could write a message such as:
\begin{center}
|This is the part \childdocname{} of \childdocjob{}.|
\end{center}

%%%%%%%%%%%%%%%%%%%%%%%%%%%%%%%%%%%%%%%%%%%%%%%%%%%%%%%%%%%%%%%%%%%%%%%%%%%%%%%%
\subsection{Flags}
\label{sec:flags}

The package makes it easy to generate different versions
of the main or child documents.
To this end compilation flags can be defined
and assigned different default values.
They will be particularly useful in conjunction
with the forwarding mechanism described in \secref{sec:forward}.

For example, it may be useful to have a flag |\version|
which can be set to |draft| or |final|.
The document source will contain some conditional code
depending on the value of |\version|.
Suppose further, the flag should default to |final| for the main file
and to |draft| for child files
which is a natural assignment for editing the document.
This is achieved by placing the following code
in the preamble of the main document
(below the |\childdocmain| directive):
%
\begin{center}
\begin{tabular}{l}
|\ifchilddoc|\\
|\providecommand{\version}{draft}|\\
|\||else|\\
|\providecommand{\version}{final}|\\
|\||fi|
\end{tabular}
\end{center}
%
The definition by |\providecommand| makes sure
that previous definitions are not overwritten.
Further statements |\providecommand{\version}{...}|
can thus be added before the above code to override it.

For the main file, one might add a line
(between |\childdocmain| and the above block)
%
\begin{center}
|%\ifchilddoc\||else\providecommand{\version}{draft}\||fi|
\end{center}
%
which can be uncommented to produce a draft version.
Likewise one can add a line to the very top of a child file
(above the |\childdocof{|\textit{main}|}| directive)
%
\begin{center}
|%\providecommand{\version}{final}|
\end{center}
%
which can be uncommented to produce the final version of this child document.

%%%%%%%%%%%%%%%%%%%%%%%%%%%%%%%%%%%%%%%%%%%%%%%%%%%%%%%%%%%%%%%%%%%%%%%%%%%%%%%%
\subsection{Forwarding}
\label{sec:forward}

Different versions of the main or child documents
using compilation flags as described in \secref{sec:flags}
can be (permanently) stored in different files
for convenient compilation, viewing and distribution.
To this end, the package defines a command
to pass on compilation to a different file:

%%%%%%%%%%%%%%%%%%%%%%%%%%%%%%%%%%%%%%%%
\DescribeMacro{\childdocforward}
The command |\childdocforward| redirects processing to
another source file:
%
\begin{center}
\begin{tabular}{l}
|\input{childdoc.def}|\\
|\childdocforward[|\textit{main}|]{|\textit{dest}|}|\\
\end{tabular}
\end{center}
%
The argument \textit{dest} is the destination file
(without extension).
It should be the main file or one of the child files.
Note that further \textsf{childdoc} directives
such as |\childdocof| and |\childdocforward|
in the indicated file will be processed in this form.
The optional argument \textit{main}
passes on directly to the main file \textit{main}
while pretending to compile the child \textit{dest}.
This form behaves as if \textit{dest}
issues |\childdocof{|\textit{main}|}| right away,
and no further \textsf{childdoc} directives will be processed.

%%%%%%%%%%%%%%%%%%%%%%%%%%%%%%%%%%%%%%%%
\DescribeMacro{\...prefix}
In the alternative form |\childdocforwardprefix|,
%
\begin{center}
\begin{tabular}{l}
|\input{childdoc.def}|\\
|\childdocforwardprefix[|\textit{main}|]{|\textit{prefix}|}{|\textit{dest}|}|
\end{tabular}
\end{center}
%
the destination file is determined by a pattern
depending on the current file:
To make this work, the current file must be called
`{\textit{prefix}\hspace{0.2em}\textit{suffix}}'
with \textit{prefix} matching precisely the argument.
Processing is then passed on to the file
`{\textit{dest}\hspace{0.2em}\textit{suffix}}'.
Surely, the same effect is achieved by
directly specifying the
argument `{\textit{dest}\hspace{0.2em}\textit{suffix}}'
in the first form.
However, that requires to set up a different file
for each child. With the alternative form of the command
all these files can have exactly the same content
which simplifies setting them up and maintaining them.

For example, the following file |draft.tex|
with a compilation flag |\version| as described in \secref{sec:flags}
compiles the main document as a draft:
%
\begin{center}
\begin{tabular}{l}
|\def\version{draft}|\\
|\input{childdoc.def}|\\
|\childdocforward{|\textit{main}|}|
\end{tabular}
\end{center}
%
Likewise, the following files |final|\textit{nn}|.tex|
compile the final version of the child document
|child|\textit{nn}|.tex|:
%
\begin{center}
\begin{tabular}{l}
|\def\version{final}|\\
|\input{childdoc.def}|\\
|\childdocforwardprefix{final}{child}|
\end{tabular}
\end{center}
%

Note that when several versions of a main file and/or of each child file
are to be generated, it may be convenient to set up a |Makefile| or
shell script to automatise the process.

%%%%%%%%%%%%%%%%%%%%%%%%%%%%%%%%%%%%%%%%%%%%%%%%%%%%%%%%%%%%%%%%%%%%%%%%%%%%%%%%
\subsection{Command Line Processing}
\label{sec:commandline}

The effect of redirection files can also be achieved by invoking
the \LaTeX{} compiler with a more elaborate command line.
Most conveniently this should be done as part
of a shell script or a |Makefile|.

When using \textsf{childdoc} in the main file, the following
command lines effectively perform a redirection
(note that depending on the shell being used,
backslashes may have to be doubled: `|\|' $\to$ `|\\|'):
%
\begin{center}
|... -jobname "|\textit{target}|" |\\|"|[\textit{flags}]%
|\input{childdoc.def}\childdocforward[|\textit{main}|]{|\textit{dest}|}"|
\end{center}
%
Here \textit{target} is the name of the output file,
\textit{main} is the name of the main file
and \textit{dest} is the name of the main or child file to be processed
(all filenames without extensions).
The optional argument \textit{main} can be omitted
if \textit{main} matches \textit{dest}.
Optionally, compilation \textit{flags} can be defined via |\def| commands.
This command line makes the \TeX{} engine believe
it is compiling the file \textit{target}
whose content is specified as the latter parameter.
The provided code then forwards the processing to
\textit{main} or \textit{dest} as described in \secref{sec:forward}.

%%%%%%%%%%%%%%%%%%%%%%%%%%%%%%%%%%%%%%%%%%%%%%%%%%%%%%%%%%%%%%%%%%%%%%%%%%%%%%%%
\subsection{Include by Input}
\label{sec:input}

Including child documents by |\include| has some restrictions by design.
Most notably, the content of a child document always occupies
its own set of pages; pages cannot be shared between child documents.
Usually, this behaviour makes perfect sense
because each child document contain an essential part of the document.
However, in some situations it may be desirable to compose
a document from a collection of parts
without having mandatory page breaks between then.
For this case, the package
provides a mechanism to include parts
by |\input| which can also be processed individually.
However, by construction this mechanism
requires manual handling of the content to be output.

%%%%%%%%%%%%%%%%%%%%%%%%%%%%%%%%%%%%%%%%
\DescribeMacro{\ifchilddocmanual}
The main file should be prepared as usual, see \secref{sec:include}.
However, the document body must make a distinction
between processing of an individual part and of the main document, e.g.:
%
\begin{center}
\begin{tabular}{l}
|\ifchilddocmanual|\\
|\input{\childdocname}|\\
|\||else|\\
\textit{document body with }|\input{|\textit{part}|}|\\
|\||fi|
\end{tabular}
\end{center}
%
The conditional |\ifchilddocmanual| is true whenever
a part to be included by |\input| is being compiled,
and the name of the part is stored in |\childdocname|.

%%%%%%%%%%%%%%%%%%%%%%%%%%%%%%%%%%%%%%%%
\DescribeMacro{\childdocby}
Each part to be included by |\input| should start with:
%
\begin{center}
\begin{tabular}{l}
|\input{childdoc.def}|\\
|\childdocby{|\textit{main}|}|\\
\end{tabular}
\end{center}
%
The directive |\childdocby| is similar to |\childdocof|
described in \secref{sec:include},
but the subsequent selection of content must be done manually.
To that end, both |\ifchilddoc| and |\ifchilddocmanual|
will be true upon processing of a part,
and the name of the part is stored in |\childdocname|.
Note that |\jobname| will be set to the filename of the current part
so that each part receives an individual |.aux| file
that does not interfere with the |.aux| file(s) of the main document.
This behaviour can be altered by the alternative form
|\childdocby[*]{|\textit{main}|}| (with a non-empty optional argument)
which uses the |.aux| file of the main document
by setting |\jobname| to \textit{main}.

%%%%%%%%%%%%%%%%%%%%%%%%%%%%%%%%%%%%%%%%%%%%%%%%%%%%%%%%%%%%%%%%%%%%%%%%%%%%%%%%
\subsection{Driver Development}
\label{sec:driver}

The \textsf{childdoc} mechanism can also be use for the development
of definition files such as \LaTeX{} styles or classes.
This case differs from the above setup with multiple parts
included by |\include| in that no |\includeonly| should be invoked.
This can be achieved by starting the include file
(before |\ProvidesPackage|) with:
%
\begin{center}
\begin{tabular}{l}
|\input{childdoc.def}|\\
|\childdocforward{|\textit{main}|}|\\
\end{tabular}
\end{center}
%
or alternatively with:
%
\begin{center}
\begin{tabular}{l}
|\input{childdoc.def}|\\
|\childdocby{|\textit{main}|}|\\
\end{tabular}
\end{center}
%
Both forms have slightly different effects as described above.
The main file is prepared as usual, see \secref{sec:include}.

%%%%%%%%%%%%%%%%%%%%%%%%%%%%%%%%%%%%%%%%%%%%%%%%%%%%%%%%%%%%%%%%%%%%%%%%%%%%%%%%
\subsection{Legacy Detection}
\label{sec:detection}

The directive |\childdocmain| in the main file can detect
whether the complete document or merely a child is to be compiled
even without using the directive |\childdocof|.
This method is deprecated because it is less robust
and there is no compelling reason to use it;
it is merely provided for backward compatibility
and it may be removed in future versions.

If the detection mechanism is to be used,
it is mandatory to correctly specify
the filename of the main file as the argument of |\childdocmain|:
%
\begin{center}
\begin{tabular}{l}
|\input{childdoc.def}|\\
|\childdocmain{|\textit{main}|}|\\
\end{tabular}
\end{center}
%
If |\jobname| does not match the argument \textit{main} of |\childdocmain|,
it is assumed that |\jobname| points to the child file to be compiled.
When using |\childdocmain| with the main file specified as argument,
it suffices to start a child file
with just |\input{|\textit{main}|}|
without loading of the package and using |\childdocof|.
If instead all processing is done
with the appropriate \textsf{childdoc} directives,
the argument of \textit{main} of |\childdocmain| can be empty.

An alternative version of the command line processing described
in \secref{sec:commandline} using the detection mechanism reads:
%
\begin{center}
|... -jobname "|\textit{target}|" "|[\textit{flags}]%
[|\def\jobname{|\textit{dest}|}|]|\input{|\textit{main}|}"|
\end{center}

%%%%%%%%%%%%%%%%%%%%%%%%%%%%%%%%%%%%%%%%%%%%%%%%%%%%%%%%%%%%%%%%%%%%%%%%%%%%%%%%
\subsection{Manual Code}
\label{sec:manual}

In case one cannot be certain whether the definitions file |childdoc.def|
is installed on the target \TeX{} distribution
and one prefers not to ship it,
it is conceivable to paste a few relevant commands into the sources.

To that end, drop all statements |\input{childdoc.def}|
and perform the replacements as outlined below.
Instead of |\childdocmain{|\textit{main}|}| add the following code
to the top of the main file:
%
\begin{center}
\begin{tabular}{l}
|\||ifdefined\childdocname\endinput\||fi\newif\ifchilddoc|\\
|\edef\childdocname{\scantokens\expandafter{\jobname\noexpand}}|\\
|\def\childdocmain{|\textit{main}|}\||ifx\childdocmain\childdocname\||else|\\
|\childdoctrue\includeonly{\childdocname}\let\jobname\childdocmain\||fi|\\
\end{tabular}
\end{center}
%
Instead of |\childdocof{|\textit{main}|}| just include the main file
at the top of each child file:
%
\begin{center}
|\input{|\textit{main}|}|
\end{center}
%
A simple redirection |\childdocforward{|\textit{dest}|}| is achieved by:
%
\begin{center}
|\def\jobname{|\textit{dest}|}\input{\jobname}|
\end{center}
%
The redirection with prefix
|\childdocforwardprefix[|\textit{prefix}|]{|\textit{dest}|}|
is accomplished by:
%
\begin{center}
\begin{tabular}{l}
|{\edef\jobname{\scantokens\expandafter{\jobname\noexpand}}|\\
|\def\redirectjob |\textit{prefix}|#1~~~{\gdef\jobname{|\textit{dest}|#1}}|\\
|\expandafter\redirectjob\jobname~~~}\input{\jobname}|
\end{tabular}
\end{center}

In an alternative approach,
child documents can be compiled by a specific command line
without additional code or specific definitions:
%
\begin{center}
|... -jobname "|\textit{target}|" "|[\textit{flags}]%
|\includeonly{|\textit{dest}|}\input{|\textit{main}|}"|
\end{center}
%

%%%%%%%%%%%%%%%%%%%%%%%%%%%%%%%%%%%%%%%%%%%%%%%%%%%%%%%%%%%%%%%%%%%%%%%%%%%%%%%%
%%%%%%%%%%%%%%%%%%%%%%%%%%%%%%%%%%%%%%%%%%%%%%%%%%%%%%%%%%%%%%%%%%%%%%%%%%%%%%%%
\section{Information}

%%%%%%%%%%%%%%%%%%%%%%%%%%%%%%%%%%%%%%%%%%%%%%%%%%%%%%%%%%%%%%%%%%%%%%%%%%%%%%%%
\subsection{Copyright}

Copyright \copyright{} 2017--2018 Niklas Beisert

This work may be distributed and/or modified under the
conditions of the \LaTeX{} Project Public License, either version 1.3
of this license or (at your option) any later version.
The latest version of this license is in
  \url{http://www.latex-project.org/lppl.txt}
and version 1.3 or later is part of all distributions of \LaTeX{}
version 2005/12/01 or later.

This work has the LPPL maintenance status `maintained'.

The Current Maintainer of this work is Niklas Beisert.

This work consists of the files |README.txt|, |childdoc.ins| and |childdoc.dtx|
as well as the derived files |childdoc.def|, |cdocsamp.tex|
with |cdocsch1.tex|, |cdocsch2.tex|, |cdocspt3.tex|, |cdocspt4.tex|,
|cdocsdrf.tex|, |cdocsfn1.tex|, |cdocsfn2.tex|
as well as |childdoc.pdf|.

%%%%%%%%%%%%%%%%%%%%%%%%%%%%%%%%%%%%%%%%%%%%%%%%%%%%%%%%%%%%%%%%%%%%%%%%%%%%%%%%
\subsection{Files and Installation}

The package consists of the files:
%
\begin{center}
\begin{tabular}{ll}
    |README.txt|   & readme file \\
    |childdoc.ins| & installation file \\
    |childdoc.dtx| & source file \\
    |childdoc.def| & definition file \\
    |cdocsamp.tex| & sample main file \\
    |cdocsch1.tex| & sample include file \\
    |cdocsch2.tex| & sample include file \\
    |cdocspt3.tex| & sample part file \\
    |cdocspt4.tex| & sample part file \\
    |cdocsdrf.tex| & sample redirection file \\
    |cdocsfn1.tex| & sample redirection file \\
    |cdocsfn2.tex| & sample redirection file \\
    |childdoc.pdf| & manual
\end{tabular}
\end{center}
%
The distribution consists of the files
|README.txt|, |childdoc.ins| and |childdoc.dtx|.
%
\begin{itemize}
\item
Run (pdf)\LaTeX{} on |childdoc.dtx|
to compile the manual |childdoc.pdf| (this file).
\item
Run \LaTeX{} on |childdoc.ins| to create the definitions file |childdoc.def|
and the sample |cdocsamp.tex| with include files
|cdocsch1.tex|, |cdocsch2.tex|, |cdocspt3.tex|, |cdocspt4.tex|,
|cdocsdrf.tex|, |cdocsfn1.tex|, |cdocsfn2.tex|.
Then copy the file |childdoc.def| to an appropriate directory of your \LaTeX{}
distribution, e.g.\ \textit{texmf-root}|/tex/latex/childdoc|.
\end{itemize}

%%%%%%%%%%%%%%%%%%%%%%%%%%%%%%%%%%%%%%%%%%%%%%%%%%%%%%%%%%%%%%%%%%%%%%%%%%%%%%%%
\subsection{Related CTAN Packages}

There are several other packages which offer a similar functionality:
%
\begin{itemize}
\item
The packages
\href{http://ctan.org/pkg/docmute}{\textsf{docmute}},
\href{http://ctan.org/pkg/includex}{\textsf{includex}} and
\href{http://ctan.org/pkg/standalone}{\textsf{standalone}}
provide commands to include only the document body of
a child file thus allowing both files to be compiled individually.
\item
The packages \href{http://ctan.org/pkg/subdocs}{\textsf{subdocs}}
and \href{http://ctan.org/pkg/subfiles}{\textsf{subfiles}}
provide structures in which the main and child documents can be
encapsulated and allowing them to be compiled individually.
The inclusion mechanism is different from the conventional |\include|.
\item
The package \href{http://ctan.org/pkg/combine}{\textsf{combine}}
is an elaborate solution to combine several documents into one.
\end{itemize}
%
See also the CTAN topic \href{http://ctan.org/topic/subdocs}{\textsf{subdocs}}
for further related packages.
The present package differs from the above solutions in that
a document structure constructed with the conventional |\include| mechanism
just needs two extra commands at the top of every file
such that all constituent files can be compiled individually.

%%%%%%%%%%%%%%%%%%%%%%%%%%%%%%%%%%%%%%%%%%%%%%%%%%%%%%%%%%%%%%%%%%%%%%%%%%%%%%%%
%\subsection{Feature Suggestions}
%
%The following is a list of features which may be useful for future
%versions of this package:
%%
%\begin{itemize}
%\item
%\ldots
%\end{itemize}

%%%%%%%%%%%%%%%%%%%%%%%%%%%%%%%%%%%%%%%%%%%%%%%%%%%%%%%%%%%%%%%%%%%%%%%%%%%%%%%%
\subsection{Revision History}

%%%%%%%%%%%%%%%%%%%%%%%%%%%%%%%%%%%%%%%%
\paragraph{v2.0:} 2018/12/30

\begin{itemize}
\item
immediate forward processing
\item
added |\childdocby| mechanism
\item
manual restructured
\end{itemize}

%%%%%%%%%%%%%%%%%%%%%%%%%%%%%%%%%%%%%%%%
\paragraph{v1.6:} 2018/01/17

\begin{itemize}
\item
application for development of include files
\item
corrections to manual
\end{itemize}

%%%%%%%%%%%%%%%%%%%%%%%%%%%%%%%%%%%%%%%%
\paragraph{v1.5:} 2017/05/21

\begin{itemize}
\item
more complete structuring introduced
\item
|\childdocof| introduced
\item
|\childdoc| renamed to |\childdocmain|
\item
|\childredirect| renamed to |\childdocforward| and |\childdocforwardprefix|
and functionality expanded
\end{itemize}

%%%%%%%%%%%%%%%%%%%%%%%%%%%%%%%%%%%%%%%%
\paragraph{v1.0:} 2017/04/27

\begin{itemize}
\item
manual and install package
\item
first version published on CTAN
\end{itemize}

%%%%%%%%%%%%%%%%%%%%%%%%%%%%%%%%%%%%%%%%
\paragraph{v0.6:} 2017/04/26

\begin{itemize}
\item
redirection mechanism added
\end{itemize}

%%%%%%%%%%%%%%%%%%%%%%%%%%%%%%%%%%%%%%%%
\paragraph{v0.5:} 2017/04/26

\begin{itemize}
\item
functionality in definition file
\end{itemize}


%%%%%%%%%%%%%%%%%%%%%%%%%%%%%%%%%%%%%%%%%%%%%%%%%%%%%%%%%%%%%%%%%%%%%%%%%%%%%%%%
%%%%%%%%%%%%%%%%%%%%%%%%%%%%%%%%%%%%%%%%%%%%%%%%%%%%%%%%%%%%%%%%%%%%%%%%%%%%%%%%
%%%%%%%%%%%%%%%%%%%%%%%%%%%%%%%%%%%%%%%%%%%%%%%%%%%%%%%%%%%%%%%%%%%%%%%%%%%%%%%%
\appendix

\settowidth\MacroIndent{\rmfamily\scriptsize 000\ }

 \DocInput{childdoc.dtx}

\end{document}
%</driver>
% \fi
%
% %%%%%%%%%%%%%%%%%%%%%%%%%%%%%%%%%%%%%%%%%%%%%%%%%%%%%%%%%%%%%%%%%%%%%%%%%%%%%%
% %%%%%%%%%%%%%%%%%%%%%%%%%%%%%%%%%%%%%%%%%%%%%%%%%%%%%%%%%%%%%%%%%%%%%%%%%%%%%%
% \section{Sample}
%\iffalse
%<*samplemain>
%\fi
%
% The following presents a sample document
% with two chapters, two parts, a title page,
% a compile flag as well as three forwarding files to set the flag.
% It consists of eight |.tex| files:
% \begin{center}
% \begin{tabular}{ll}
% |cdocsamp.tex|&main file\\
% |cdocsch1.tex|&include file for chapter 1\\
% |cdocsch2.tex|&include file for chapter 2\\
% |cdocspt3.tex|&include file for part 3\\
% |cdocspt4.tex|&include file for part 4\\
% |cdocsdrf.tex|&forwarding file for main file in draft mode\\
% |cdocsfi1.tex|&forwarding file for final version of chapter 1\\
% |cdocsfi2.tex|&forwarding file for final version of chapter 2\\
% \end{tabular}
% \end{center}
% Each of the eight files can be compiled directly by the \LaTeX{} compiler.
%
% %%%%%%%%%%%%%%%%%%%%%%%%%%%%%%%%%%%%%%
% \paragraph{Main File.}
%
% The main file is called |cdocsamp.tex|.
%
% Load the \textsf{childdoc} definitions and
% declare the filename for the main document:
%    \begin{macrocode}
\input{childdoc.def}
\childdocmain{}
%    \end{macrocode}

% Optional override for |\version| flag:
%    \begin{macrocode}
%%\ifchilddoc\else\providecommand{\version}{draft}\fi
%    \end{macrocode}

% Define the default values for the |\version| flag
% (|final| for the main file and |draft| for childs):
%    \begin{macrocode}
\ifchilddoc
\providecommand{\version}{draft}
\else
\providecommand{\version}{final}
\fi
%    \end{macrocode}

% Load the standard document class:
%    \begin{macrocode}
\documentclass[12pt]{article}
%    \end{macrocode}

% Start the document body:
%    \begin{macrocode}
\begin{document}
%    \end{macrocode}

% Declare a title page.
% Print title, part of document being processed and version flag:
%    \begin{macrocode}
\addtocounter{page}{-1}
\begin{center}
{\LARGE\bfseries{}childdoc example\par}
\vspace{1cm}
\ifchilddoc
\ifchilddocmanual part\else chapter\fi:
`\childdocname' of `\childdocjob'\par
\else
main document: `\childdocjob'\par
\fi
version: \version\par
\end{center}
\newpage
%    \end{macrocode}

% Manually include selected file,
% otherwise process as usual:
%    \begin{macrocode}
\ifchilddocmanual
\section*{part `\childdocname'}
\input{\childdocname}
\else
%    \end{macrocode}

% Include the two chapters:
%    \begin{macrocode}
\include{cdocsch1}
\include{cdocsch2}
%    \end{macrocode}

% Include the two parts unless only chapters should be displayed:
%    \begin{macrocode}
\ifchilddoc\else
\section{part three}
\input{cdocspt3}
\section{part four}
\input{cdocspt4}
\fi
%    \end{macrocode}

% Process as usual until here:
%    \begin{macrocode}
\fi
%    \end{macrocode}

% End of document body:
%    \begin{macrocode}
\end{document}
%    \end{macrocode}
%\iffalse
%</samplemain>
%\fi
%
% %%%%%%%%%%%%%%%%%%%%%%%%%%%%%%%%%%%%%%
% \paragraph{Chapter Include Files.}
%
% The include files are called |cdocsch1.tex| and |cdocsch2.tex|.
%
%\iffalse
%<*samplechap1|samplechap2>
%\fi

% Optional override for |\version| flag:
%    \begin{macrocode}
%%\providecommand{\version}{final}
%    \end{macrocode}

% Include the main document:
%    \begin{macrocode}
\input{childdoc.def}
\childdocof{cdocsamp}
%    \end{macrocode}

%\iffalse
%</samplechap1|samplechap2>
%\fi
%
%\iffalse
%<*samplechap1>
%\fi
% Some text for chapter 1:
%    \begin{macrocode}
\section{one}
some text in chapter one
%    \end{macrocode}

%\iffalse
%</samplechap1>
%\fi
% Some text for chapter 2:
%\iffalse
%<*samplechap2>
%\fi
%    \begin{macrocode}
\section{two}
more text in chapter two
%    \end{macrocode}

%\iffalse
%</samplechap2>
%\fi
%
% %%%%%%%%%%%%%%%%%%%%%%%%%%%%%%%%%%%%%%
% \paragraph{Part Include Files.}
%
% The include files are called |cdocspt3.tex| and |cdocspt4.tex|.
%
%\iffalse
%<*samplepart3|samplepart4>
%\fi

% Optional override for |\version| flag:
%    \begin{macrocode}
%%\providecommand{\version}{final}
%    \end{macrocode}

% Include the main document:
%    \begin{macrocode}
\input{childdoc.def}
\childdocby{cdocsamp}
%    \end{macrocode}

%\iffalse
%</samplepart3|samplepart4>
%\fi
%
%\iffalse
%<*samplepart3>
%\fi
% Some text for part 3:
%    \begin{macrocode}
some text in part three
%    \end{macrocode}

%\iffalse
%</samplepart3>
%\fi
% Some text for part 4:
%\iffalse
%<*samplepart4>
%\fi
%    \begin{macrocode}
more text in part four
%    \end{macrocode}

%\iffalse
%</samplepart4>
%\fi
%
% %%%%%%%%%%%%%%%%%%%%%%%%%%%%%%%%%%%%%%
% \paragraph{Forwarding for a Complete Draft.}
%
% The following forwarding file |cdocsdrf.tex|
% compiles the main document in draft mode:
%\iffalse
%<*sampledraft>
%\fi
%    \begin{macrocode}
\def\version{draft}
\input{childdoc.def}
\childdocforward{cdocsamp}
%    \end{macrocode}

%\iffalse
%</sampledraft>
%\fi
%
% %%%%%%%%%%%%%%%%%%%%%%%%%%%%%%%%%%%%%%
% \paragraph{Forwarding for Final Version of the Chapters.}
%
% The following forwarding files |cdocsfn1.tex| and |cdocsfn2.tex|
% (with identical content)
% compile the final versions of the child documents
% |cdocsch1.tex| and |cdocsch2.tex|, respectively:
%\iffalse
%<*samplefinal>
%\fi
%    \begin{macrocode}
\def\version{final}
\input{childdoc.def}
\childdocforwardprefix[cdocsamp]{cdocsfn}{cdocsch}
%    \end{macrocode}

%\iffalse
%</samplefinal>
%\fi
%
% %%%%%%%%%%%%%%%%%%%%%%%%%%%%%%%%%%%%%%
% \paragraph{Command Line Processing.}
%
% The following three command lines generate the output files
% |cdocscld|, |cdocscl1| and |cdocscl2|
% which should be identical to
% |cdocsdrf|, |cdocsch1| and |cdocsfn2|, respectively:
% \begin{center}
% \begin{tabular}{l}
% |latex -jobname cdocscld \|\\
% |  "\def\version{draft}\input{childdoc.def}\childdocforward{cdocsamp}"|\\
% |latex -jobname cdocscl1 \|\\
% |  "\input{childdoc.def}\childdocforward[cdocsamp]{cdocsch1}"|\\
% |latex -jobname cdocscl2 \|\\
% |  "\def\version{final}\input{childdoc.def}\childdocforward{cdocsch2}"|
% \end{tabular}
% \end{center}
% Note that the trailing backslash on each first line
% merely continues the input to the second line
% (for convenient cut ant paste).
% Furthermore, the command |latex| can be replaced by any
% of its alternative versions such as |pdflatex|.
%
% %%%%%%%%%%%%%%%%%%%%%%%%%%%%%%%%%%%%%%%%%%%%%%%%%%%%%%%%%%%%%%%%%%%%%%%%%%%%%%
% %%%%%%%%%%%%%%%%%%%%%%%%%%%%%%%%%%%%%%%%%%%%%%%%%%%%%%%%%%%%%%%%%%%%%%%%%%%%%%
% \section{Implementation}
%\iffalse
%<*package>
%\fi
%
% This section describes the definitions file |childdoc.def|.

% The definitions cannot be loaded using |\usepackage| or |\RequirePackage|
% which has a mechanism to prevent loading a style file more than once.
% When loading the definitions by means of |\input|
% multiple instances have to be prevented manually:
%\iffalse
%This code needs to be before the `\ProvidesFile' directive
%which is defined at the beginning of this file.
%Therefore it is also placed there and commented out here.
%</package>
%<*discard>
%\fi
%    \begin{macrocode}
\ifdefined\childdocmain\endinput\fi
%    \end{macrocode}
%\iffalse
%</discard>
%<*package>
%\fi
%
% \macro{\ifchilddoc}
% \macro{\ifchilddocmanual}
% The conditional |\ifchilddoc| tells whether a
% child (true) or main (false) document is being compiled.
% The conditional |\ifchilddocmanual| tells whether
% the |\includeonly| mechanism is used (false) or
% the selection of child files must be performed manually (true).
% The definitions initialise to false:
%    \begin{macrocode}
\newif\ifchilddoc
\newif\ifchilddocmanual
%    \end{macrocode}

% \macro{\childdocname}
% \macro{\childdocjob}
% The macro |\childdocname| stores the name of the main document
% to be compiled. The macro |\childdocjob| stores the name of
% the document on which the \LaTeX{} compiler was originally invoked.
% The content of |\jobname| cannot be compared
% to filenames specified in the source due to different catcodes.
% The following code rescans |\jobname|, stores the result
% in |\childdocname| and saves a copy in |\childdocjob|:
%    \begin{macrocode}
\edef\childdocname{\scantokens\expandafter{\jobname\noexpand}}
\let\childdocjob\childdocname
%    \end{macrocode}

% \macro{\childdocdisable}
% The macro |\childdocdisable| prevents the main file
% from being processed more than once.
% At this stage, the main document command |\childdocmain|
% is assumed to be called once again where it should do nothing.
% Any subsequent call to it should prevent
% a secondary processing of the main document
% It overwrites the forwarding commands
% |\childdocof| and |\childdocforward|
% with empty macros to prevent further inclusions of the main document:
%    \begin{macrocode}
\newcommand{\childdocdisable}
{
  \renewcommand{\childdocmain}[1]{\renewcommand{\childdocmain}[1]{\endinput}}
  \renewcommand{\childdocof}[1]{}
  \renewcommand{\childdocby}[2][]{}
  \renewcommand{\childdocforward}[2][]{}
  \renewcommand{\childdocdisable}{}
}
%    \end{macrocode}

% \macro{\childdocmain}
% The macro |\childdocmain| is to be called at the top of the main file
% with nothing or the main filename (without extension) as argument.
% First, it breaks loops.
% If the argument is not empty and does not match |\childdocname|
% (which is set by the first inclusion of |childdoc.def|),
% |\ifchilddoc| is set to true, |\includeonly| is applied to the child file
% and |\jobname| is set to the main file
% (for proper handling of |.aux| files):
%    \begin{macrocode}
\newcommand{\childdocmain}[1]
{
  \childdocdisable\childdocmain{}
  \if?#1?\else
    \begingroup
      \def\childdoctmp{#1}
      \ifx\childdoctmp\childdocname
        \def\childdoctmp{}
      \else
        \def\childdoctmp
        {
          \childdoctrue
          \includeonly{\childdocname}
          \def\childdocjob{#1}
          \def\jobname{#1}
        }
      \fi
      \expandafter
    \endgroup
    \childdoctmp
  \fi
}
%    \end{macrocode}

% \macro{\childdocof}
% The command |\childdocof| redirects
% compilation to the main file |#1|.
%    \begin{macrocode}
\newcommand{\childdocof}[1]
{
  \childdocdisable
  \childdoctrue
  \includeonly{\childdocname}
  \def\jobname{#1}
  \def\childdocjob{#1}
  \input{#1}
}
%    \end{macrocode}

% \macro{\childdocby}
% The command |\childdocby| ....
%    \begin{macrocode}
\newcommand{\childdocby}[2][]
{
  \childdocdisable
  \childdoctrue
  \childdocmanualtrue
  \if?#1?\else
    \def\jobname{#2}
  \fi
  \def\childdocjob{#2}
  \input{#2}
  \endinput
}
%    \end{macrocode}

% \macro{\childdocforward}
% The command |\childdocforward| redirects
% compilation to the main file or
% (if the optional argument is given) a child file.
% Parameters are set as if the main file
% or a child file starting with |\childdocof| was compiled.
% Then compilation is handed over to the main file:
%    \begin{macrocode}
\newcommand{\childdocforward}[2][]
{
  \begingroup
    \if?#1?
      \def\childdoctmp
      {
        \def\childdocname{#2}
        \def\childdocjob{#2}
        \def\jobname{#2}
        \input{#2}
        \endinput
      }
    \else
      \def\childdoctmp
      {
        \childdocdisable
        \def\childdocname{#2}
        \childdoctrue
        \includeonly{#2}
        \def\childdocjob{#1}
        \def\jobname{#1}
        \input{#1}
        \endinput
      }
    \fi
    \expandafter
  \endgroup
  \childdoctmp
}
%    \end{macrocode}

% \macro{\childdocforwardprefix}
% The command |\childdocforwardprefix| redirects
% compilation to the main or a child file by means of a pattern.
% The prefix |#1| in the current filename is replaced by |#2|
% and the suffix of the current filename is kept
% (it is assumed that the filename does not contain the substring `|~~~|'
% which is used as a delimiter).
% Compilation is handed over to the new file by |\childdocforward|:
%    \begin{macrocode}
\newcommand{\childdocforwardprefix}[3][]
{
  \begingroup
    \def\childdocextract #2##1~~~{\def\childdoctmp{\childdocforward[#1]{#3##1}}}
    \expandafter\childdocextract\childdocname~~~
    \expandafter
  \endgroup
  \childdoctmp
}
%    \end{macrocode}

% \macro{\childdoc}
% The deprecated macro |\childdoc| is a legacy version of |\childdocmain|:
%    \begin{macrocode}
\newcommand{\childdoc}{\childdocmain}
%    \end{macrocode}

% \macro{\childdocredirect}
% The deprecated macro |\childdocredirect| is a legacy version
% of |\childdocforward| and |\childdocforwardprefix|:
%    \begin{macrocode}
\newcommand{\childdocredirect}[2][]
{
  \begingroup
    \if?#1?
      \def\childdoctmp{\childdocforward{#2}}
    \else
      \def\childdoctmp{\childdocforwardprefix{#1}{#2}}
    \fi
    \expandafter
  \endgroup
  \childdoctmp
}
%    \end{macrocode}

%\iffalse
%</package>
%\fi
%
\endinput

\childdocmain{}
%    \end{macrocode}

% Optional override for |\version| flag:
%    \begin{macrocode}
%%\ifchilddoc\else\providecommand{\version}{draft}\fi
%    \end{macrocode}

% Define the default values for the |\version| flag
% (|final| for the main file and |draft| for childs):
%    \begin{macrocode}
\ifchilddoc
\providecommand{\version}{draft}
\else
\providecommand{\version}{final}
\fi
%    \end{macrocode}

% Load the standard document class:
%    \begin{macrocode}
\documentclass[12pt]{article}
%    \end{macrocode}

% Start the document body:
%    \begin{macrocode}
\begin{document}
%    \end{macrocode}

% Declare a title page.
% Print title, part of document being processed and version flag:
%    \begin{macrocode}
\addtocounter{page}{-1}
\begin{center}
{\LARGE\bfseries{}childdoc example\par}
\vspace{1cm}
\ifchilddoc
\ifchilddocmanual part\else chapter\fi:
`\childdocname' of `\childdocjob'\par
\else
main document: `\childdocjob'\par
\fi
version: \version\par
\end{center}
\newpage
%    \end{macrocode}

% Manually include selected file,
% otherwise process as usual:
%    \begin{macrocode}
\ifchilddocmanual
\section*{part `\childdocname'}
\input{\childdocname}
\else
%    \end{macrocode}

% Include the two chapters:
%    \begin{macrocode}
\include{cdocsch1}
\include{cdocsch2}
%    \end{macrocode}

% Include the two parts unless only chapters should be displayed:
%    \begin{macrocode}
\ifchilddoc\else
\section{part three}
\input{cdocspt3}
\section{part four}
\input{cdocspt4}
\fi
%    \end{macrocode}

% Process as usual until here:
%    \begin{macrocode}
\fi
%    \end{macrocode}

% End of document body:
%    \begin{macrocode}
\end{document}
%    \end{macrocode}
%\iffalse
%</samplemain>
%\fi
%
% %%%%%%%%%%%%%%%%%%%%%%%%%%%%%%%%%%%%%%
% \paragraph{Chapter Include Files.}
%
% The include files are called |cdocsch1.tex| and |cdocsch2.tex|.
%
%\iffalse
%<*samplechap1|samplechap2>
%\fi

% Optional override for |\version| flag:
%    \begin{macrocode}
%%\providecommand{\version}{final}
%    \end{macrocode}

% Include the main document:
%    \begin{macrocode}
% \iffalse
%
% childdoc.dtx Copyright (C) 2017-2018 Niklas Beisert
%
% This work may be distributed and/or modified under the
% conditions of the LaTeX Project Public License, either version 1.3
% of this license or (at your option) any later version.
% The latest version of this license is in
%   http://www.latex-project.org/lppl.txt
% and version 1.3 or later is part of all distributions of LaTeX
% version 2005/12/01 or later.
%
% This work has the LPPL maintenance status `maintained'.
%
% The Current Maintainer of this work is Niklas Beisert.
%
% This work consists of the files childdoc.dtx and childdoc.ins
% and the derived files childdoc.def and cdocsamp.tex with
% cdocsch1.tex, cdocsch2.tex, cdocsdrf.tex, cdocsfn1.tex, cdocsfn2.tex.
%
%<package>\ifdefined\childdocmain\endinput\fi
%<package>\ProvidesFile{childdoc.def}[2018/12/30 v2.0 child document driver]
%<samplemain>\ProvidesFile{cdocsamp.tex}[2018/12/30 v2.0 sample for childdoc]
%<*driver>
%\ProvidesFile{childdoc.drv}[2018/12/30 v2.0 childdoc reference manual file]
\PassOptionsToClass{10pt,a4paper}{article}
\documentclass{ltxdoc}

\usepackage[margin=35mm]{geometry}
\usepackage{hyperref}
\usepackage{hyperxmp}
\usepackage[usenames]{color}

\hypersetup{colorlinks=true}
\hypersetup{pdfstartview=FitH}
\hypersetup{pdfpagemode=UseNone}
\hypersetup{pdfsource={}}
\hypersetup{pdflang={en-UK}}
\hypersetup{pdfcopyright={Copyright 2017-2018 Niklas Beisert.
  This work may be distributed and/or modified under the
  conditions of the LaTeX Project Public License, either version 1.3
  of this license or (at your option) any later version.}}
\hypersetup{pdflicenseurl={http://www.latex-project.org/lppl.txt}}
\hypersetup{pdfcontactaddress={ETH Zurich, ITP, HIT K,
  Wolfgang-Pauli-Strasse 27}}
\hypersetup{pdfcontactpostcode={8093}}
\hypersetup{pdfcontactcity={Zurich}}
\hypersetup{pdfcontactcountry={Switzerland}}
\hypersetup{pdfcontactemail={nbeisert@itp.phys.ethz.ch}}
\hypersetup{pdfcontacturl={http://people.phys.ethz.ch/\xmptilde nbeisert/}}

\newcommand{\secref}[1]{\hyperref[#1]{section \ref*{#1}}}

\parskip1ex
\parindent0pt
\let\olditemize\itemize
\def\itemize{\olditemize\parskip0pt}

\begin{document}

\title{The \textsf{childdoc} Package}
\hypersetup{pdftitle={The childdoc Package}}
\author{Niklas Beisert\\[2ex]
  Institut f\"ur Theoretische Physik\\
  Eidgen\"ossische Technische Hochschule Z\"urich\\
  Wolfgang-Pauli-Strasse 27, 8093 Z\"urich, Switzerland\\[1ex]
  \href{mailto:nbeisert@itp.phys.ethz.ch}
  {\texttt{nbeisert@itp.phys.ethz.ch}}}
\hypersetup{pdfauthor={Niklas Beisert}}
\hypersetup{pdfsubject={Manual for the LaTeX2e Package childdoc}}
\date{30 December 2018, \textsf{v2.0}}
\maketitle

\begin{abstract}\noindent
\textsf{childdoc} is a \LaTeXe{} package
that enables the direct compilation
of document sections included by |\include|
to individual files.
\end{abstract}

\begingroup
\parskip0ex
\tableofcontents
\endgroup

%%%%%%%%%%%%%%%%%%%%%%%%%%%%%%%%%%%%%%%%%%%%%%%%%%%%%%%%%%%%%%%%%%%%%%%%%%%%%%%%
%%%%%%%%%%%%%%%%%%%%%%%%%%%%%%%%%%%%%%%%%%%%%%%%%%%%%%%%%%%%%%%%%%%%%%%%%%%%%%%%
\section{Introduction}

\LaTeX{} provides a mechanism to structure a large document (such as a book)
into a main file and several child files (containing the chapters)
using the |\include| command.
This mechanism is beneficial for documents
which span hundreds of pages in order to
make the source file(s) more manageable.
Moreover, compilation can be restricted to
selected child files by means of the |\includeonly| command.
The latter feature can be used to reduce the compilation time while editing
(this was significantly more useful in the earlier days of \LaTeX{})
or to generate a smaller document which is easier to navigate.
Another application of |\includeonly| is to generate
documents consisting of selected parts of the complete document.

However, there are a few drawbacks of the plain |\include| mechanism:
\begin{itemize}
\item
The child files cannot be compiled on their own,
they can only be compiled via the main file.
A naive editing environment
(such as a text editor with an option
to have the current file processed by \LaTeX)
may require one to switch to the main file before compiling;
attempting to compile the child file produces errors.
\item
The main file must be modified (each time)
to adjust the |\includeonly| command
to the present needs. This easily leaves the main file in a messy state.
\item
The generated document will always carry the filename
of the main document. This is inconvenient if
several child files are to be compiled and
to be kept for distribution.
\end{itemize}

The present package provides a simple interface
to make child files individually compilable by \LaTeX{}.
Compiling a child file then has the same effect as compiling
the main file with an |\includeonly| command
to select the appropriate child.
Moreover the generated document will carry the name of the child
rather than the main file.
This resolves all three above issues.

This feature is meant to make the editing of books,
thesis documents and lecture notes somewhat more convenient.
However, the package can also be used efficiently for
composing a series of documents (such as exercise sheets)
which are typically distributed individually.
It then assists the author in generating the individual documents
(potentially in different versions)
as well as a document containing the collected series.
Another application is in developing style files
or other kinds of included material
where compilation of the style file could redirect
to a sample or test file.

%%%%%%%%%%%%%%%%%%%%%%%%%%%%%%%%%%%%%%%%%%%%%%%%%%%%%%%%%%%%%%%%%%%%%%%%%%%%%%%%
%%%%%%%%%%%%%%%%%%%%%%%%%%%%%%%%%%%%%%%%%%%%%%%%%%%%%%%%%%%%%%%%%%%%%%%%%%%%%%%%
\section{Usage}

First of all, the package \textsf{childdoc} is \emph{not} a standard
\LaTeXe{} |.sty| style file! Therefore it needs to be invoked in
a non-standard way.

%%%%%%%%%%%%%%%%%%%%%%%%%%%%%%%%%%%%%%%%%%%%%%%%%%%%%%%%%%%%%%%%%%%%%%%%%%%%%%%%
\subsection{Included Files}
\label{sec:include}

%%%%%%%%%%%%%%%%%%%%%%%%%%%%%%%%%%%%%%%%
\DescribeMacro{\childdocmain}
To use the package, add the commands
\begin{center}
\begin{tabular}{l}
|\input{childdoc.def}|\\
|\childdocmain{}|\\
\end{tabular}
\end{center}
at the very top of the main \LaTeX{} file,
in particular \emph{before} the |\documentclass| statement!
The argument of |\childdocmain| should be left empty
(but it must be present).

%%%%%%%%%%%%%%%%%%%%%%%%%%%%%%%%%%%%%%%%
\DescribeMacro{\childdocof}
Furthermore, add the commands
\begin{center}
\begin{tabular}{l}
|\input{childdoc.def}|\\
|\childdocof{|\textit{main}|}|\\
\end{tabular}
\end{center}
at the top of every child file \textit{child}
which is included by |\include{|\textit{child}|}|
from within the main file
(or at least for those files to be compiled individually).
The argument \textit{main} must be the filename of the main file.

There are a couple of
considerations in setting up the main and child documents:

%%%%%%%%%%%%%%%%%%%%%%%%%%%%%%%%%%%%%%%%
\paragraph{Restrictions.}

Please note the following restrictions:
\begin{itemize}
\item
|\childdocmain| must be called with one argument \textit{main}
to ensure compatibility with earlier version of the package.
It must either be empty (|\childdocmain{}|)
or precisely match the filename of the main file in which it is specified.
See \secref{sec:detection} for further information.
\item
The filename \textit{main} must be specified without the |.tex| extension.
\item
The filename \textit{main} is case sensitive
(even in case-insensitive file systems)
due to internal string comparison.
\item
The argument \textit{main} should be fully expanded, it cannot be a macro.
\item
Subdirectories and special characters should be avoided in filenames.
\item
The command |\childdocmain{|\textit{main}|}| must be followed by a whitespace.
It should not be followed immediately by another command
or by a comment mark `|%|'.
This is because the \TeX{} parser reads the token immediately following
the argument of |\childdocmain| and puts it
at the beginning of every child section;
however, a white\-space is ignored.
\end{itemize}

%%%%%%%%%%%%%%%%%%%%%%%%%%%%%%%%%%%%%%%%
\paragraph{Content of Main File.}

It is advisable to place all content in the child files included by |\include|.
Any output contained in the main file will appear in all child documents
unless suppressed manually;
it cannot be suppressed automatically by the |\includeonly| directive
and thus should normally be avoided.
A method to include some content in the main file
by means of conditional processing is described in \secref{sec:conditional}.

%%%%%%%%%%%%%%%%%%%%%%%%%%%%%%%%%%%%%%%%
\paragraph{Page Numbering.}

When only a part of the document is compiled,
the appropriate numbering of pages
(as well as other status parameters)
is determined from the |.aux| files.
The latter contain information from previous passes.
However this information needs to propagate through
all intermediate child documents.
Therefore the page numbering in child documents may well
be inconsistent until the complete document is compiled at least once.

A useful (if unconventional) way to always ensure a consistent
page numbering is to restart the numbering in each child document
and denote the pages by `\textit{child}|.|\textit{page}'
where \textit{child} represents the chapter/section number of the child file.
This can be achieved by the command
|\numberwithin{page}{|\textit{child}|}|
of the \textsf{amsmath} package
where \textit{child} can be |chapter| or |section|
depending on the chosen structuring.
Alternatively, one can modify the macro |\thepage| appropriately
and reset the counter |page| at the start of each child file.

%%%%%%%%%%%%%%%%%%%%%%%%%%%%%%%%%%%%%%%%%%%%%%%%%%%%%%%%%%%%%%%%%%%%%%%%%%%%%%%%
\subsection{Conditional Processing}
\label{sec:conditional}

The package provides a mechanism to compile different versions
of a document. To customise the versions further some conditional processing
can come in handy to distinguish which version is being compiled.
The package provides two macros to describe the compilation context:

%%%%%%%%%%%%%%%%%%%%%%%%%%%%%%%%%%%%%%%%
\DescribeMacro{\ifchilddoc}
The conditional |\ifchilddoc| distinguishes between the compilation of
child documents and the main document:
%
\begin{center}
|\ifchilddoc |\textit{child-code}| |[|\||else |\textit{main-code}]| \||fi|
\end{center}

%%%%%%%%%%%%%%%%%%%%%%%%%%%%%%%%%%%%%%%%
\DescribeMacro{\childdocname}
\DescribeMacro{\childdocjob}
The macro |\childdocname| contains the filename (without extension)
of the main or child file being processed.
Note that |\childdocjob| will always contain the name of the main file.

%%%%%%%%%%%%%%%%%%%%%%%%%%%%%%%%%%%%%%%%
\paragraph{Title Page.}

Conditional processing can be used to include a title or banner page
in the main document when proper precautions are taken.
Importantly, the code in the main file should ensure that the page counter
(as well as other status parameters which are stored in the |.aux| files)
takes the same value after the conditional processing.
Otherwise the page numbers may take divergent values
depending on which part is compiled.

For example, a title page could be declared by:
%
\begin{center}
\begin{tabular}{l}
|\ifchilddoc\||else|\\
|\addtocounter{page}{-1}|\\
\textit{code for title page}\\
|\newpage|\\
|\||fi|
\end{tabular}
\end{center}
%
A banner page for the child documents can be generated by:
%
\begin{center}
\begin{tabular}{l}
|\ifchilddoc|\\
|\addtocounter{page}{-1}|\\
\textit{code for banner page}\\
|\newpage|\\
|\||fi|
\end{tabular}
\end{center}
%
Here one could write a message such as:
\begin{center}
|This is the part \childdocname{} of \childdocjob{}.|
\end{center}

%%%%%%%%%%%%%%%%%%%%%%%%%%%%%%%%%%%%%%%%%%%%%%%%%%%%%%%%%%%%%%%%%%%%%%%%%%%%%%%%
\subsection{Flags}
\label{sec:flags}

The package makes it easy to generate different versions
of the main or child documents.
To this end compilation flags can be defined
and assigned different default values.
They will be particularly useful in conjunction
with the forwarding mechanism described in \secref{sec:forward}.

For example, it may be useful to have a flag |\version|
which can be set to |draft| or |final|.
The document source will contain some conditional code
depending on the value of |\version|.
Suppose further, the flag should default to |final| for the main file
and to |draft| for child files
which is a natural assignment for editing the document.
This is achieved by placing the following code
in the preamble of the main document
(below the |\childdocmain| directive):
%
\begin{center}
\begin{tabular}{l}
|\ifchilddoc|\\
|\providecommand{\version}{draft}|\\
|\||else|\\
|\providecommand{\version}{final}|\\
|\||fi|
\end{tabular}
\end{center}
%
The definition by |\providecommand| makes sure
that previous definitions are not overwritten.
Further statements |\providecommand{\version}{...}|
can thus be added before the above code to override it.

For the main file, one might add a line
(between |\childdocmain| and the above block)
%
\begin{center}
|%\ifchilddoc\||else\providecommand{\version}{draft}\||fi|
\end{center}
%
which can be uncommented to produce a draft version.
Likewise one can add a line to the very top of a child file
(above the |\childdocof{|\textit{main}|}| directive)
%
\begin{center}
|%\providecommand{\version}{final}|
\end{center}
%
which can be uncommented to produce the final version of this child document.

%%%%%%%%%%%%%%%%%%%%%%%%%%%%%%%%%%%%%%%%%%%%%%%%%%%%%%%%%%%%%%%%%%%%%%%%%%%%%%%%
\subsection{Forwarding}
\label{sec:forward}

Different versions of the main or child documents
using compilation flags as described in \secref{sec:flags}
can be (permanently) stored in different files
for convenient compilation, viewing and distribution.
To this end, the package defines a command
to pass on compilation to a different file:

%%%%%%%%%%%%%%%%%%%%%%%%%%%%%%%%%%%%%%%%
\DescribeMacro{\childdocforward}
The command |\childdocforward| redirects processing to
another source file:
%
\begin{center}
\begin{tabular}{l}
|\input{childdoc.def}|\\
|\childdocforward[|\textit{main}|]{|\textit{dest}|}|\\
\end{tabular}
\end{center}
%
The argument \textit{dest} is the destination file
(without extension).
It should be the main file or one of the child files.
Note that further \textsf{childdoc} directives
such as |\childdocof| and |\childdocforward|
in the indicated file will be processed in this form.
The optional argument \textit{main}
passes on directly to the main file \textit{main}
while pretending to compile the child \textit{dest}.
This form behaves as if \textit{dest}
issues |\childdocof{|\textit{main}|}| right away,
and no further \textsf{childdoc} directives will be processed.

%%%%%%%%%%%%%%%%%%%%%%%%%%%%%%%%%%%%%%%%
\DescribeMacro{\...prefix}
In the alternative form |\childdocforwardprefix|,
%
\begin{center}
\begin{tabular}{l}
|\input{childdoc.def}|\\
|\childdocforwardprefix[|\textit{main}|]{|\textit{prefix}|}{|\textit{dest}|}|
\end{tabular}
\end{center}
%
the destination file is determined by a pattern
depending on the current file:
To make this work, the current file must be called
`{\textit{prefix}\hspace{0.2em}\textit{suffix}}'
with \textit{prefix} matching precisely the argument.
Processing is then passed on to the file
`{\textit{dest}\hspace{0.2em}\textit{suffix}}'.
Surely, the same effect is achieved by
directly specifying the
argument `{\textit{dest}\hspace{0.2em}\textit{suffix}}'
in the first form.
However, that requires to set up a different file
for each child. With the alternative form of the command
all these files can have exactly the same content
which simplifies setting them up and maintaining them.

For example, the following file |draft.tex|
with a compilation flag |\version| as described in \secref{sec:flags}
compiles the main document as a draft:
%
\begin{center}
\begin{tabular}{l}
|\def\version{draft}|\\
|\input{childdoc.def}|\\
|\childdocforward{|\textit{main}|}|
\end{tabular}
\end{center}
%
Likewise, the following files |final|\textit{nn}|.tex|
compile the final version of the child document
|child|\textit{nn}|.tex|:
%
\begin{center}
\begin{tabular}{l}
|\def\version{final}|\\
|\input{childdoc.def}|\\
|\childdocforwardprefix{final}{child}|
\end{tabular}
\end{center}
%

Note that when several versions of a main file and/or of each child file
are to be generated, it may be convenient to set up a |Makefile| or
shell script to automatise the process.

%%%%%%%%%%%%%%%%%%%%%%%%%%%%%%%%%%%%%%%%%%%%%%%%%%%%%%%%%%%%%%%%%%%%%%%%%%%%%%%%
\subsection{Command Line Processing}
\label{sec:commandline}

The effect of redirection files can also be achieved by invoking
the \LaTeX{} compiler with a more elaborate command line.
Most conveniently this should be done as part
of a shell script or a |Makefile|.

When using \textsf{childdoc} in the main file, the following
command lines effectively perform a redirection
(note that depending on the shell being used,
backslashes may have to be doubled: `|\|' $\to$ `|\\|'):
%
\begin{center}
|... -jobname "|\textit{target}|" |\\|"|[\textit{flags}]%
|\input{childdoc.def}\childdocforward[|\textit{main}|]{|\textit{dest}|}"|
\end{center}
%
Here \textit{target} is the name of the output file,
\textit{main} is the name of the main file
and \textit{dest} is the name of the main or child file to be processed
(all filenames without extensions).
The optional argument \textit{main} can be omitted
if \textit{main} matches \textit{dest}.
Optionally, compilation \textit{flags} can be defined via |\def| commands.
This command line makes the \TeX{} engine believe
it is compiling the file \textit{target}
whose content is specified as the latter parameter.
The provided code then forwards the processing to
\textit{main} or \textit{dest} as described in \secref{sec:forward}.

%%%%%%%%%%%%%%%%%%%%%%%%%%%%%%%%%%%%%%%%%%%%%%%%%%%%%%%%%%%%%%%%%%%%%%%%%%%%%%%%
\subsection{Include by Input}
\label{sec:input}

Including child documents by |\include| has some restrictions by design.
Most notably, the content of a child document always occupies
its own set of pages; pages cannot be shared between child documents.
Usually, this behaviour makes perfect sense
because each child document contain an essential part of the document.
However, in some situations it may be desirable to compose
a document from a collection of parts
without having mandatory page breaks between then.
For this case, the package
provides a mechanism to include parts
by |\input| which can also be processed individually.
However, by construction this mechanism
requires manual handling of the content to be output.

%%%%%%%%%%%%%%%%%%%%%%%%%%%%%%%%%%%%%%%%
\DescribeMacro{\ifchilddocmanual}
The main file should be prepared as usual, see \secref{sec:include}.
However, the document body must make a distinction
between processing of an individual part and of the main document, e.g.:
%
\begin{center}
\begin{tabular}{l}
|\ifchilddocmanual|\\
|\input{\childdocname}|\\
|\||else|\\
\textit{document body with }|\input{|\textit{part}|}|\\
|\||fi|
\end{tabular}
\end{center}
%
The conditional |\ifchilddocmanual| is true whenever
a part to be included by |\input| is being compiled,
and the name of the part is stored in |\childdocname|.

%%%%%%%%%%%%%%%%%%%%%%%%%%%%%%%%%%%%%%%%
\DescribeMacro{\childdocby}
Each part to be included by |\input| should start with:
%
\begin{center}
\begin{tabular}{l}
|\input{childdoc.def}|\\
|\childdocby{|\textit{main}|}|\\
\end{tabular}
\end{center}
%
The directive |\childdocby| is similar to |\childdocof|
described in \secref{sec:include},
but the subsequent selection of content must be done manually.
To that end, both |\ifchilddoc| and |\ifchilddocmanual|
will be true upon processing of a part,
and the name of the part is stored in |\childdocname|.
Note that |\jobname| will be set to the filename of the current part
so that each part receives an individual |.aux| file
that does not interfere with the |.aux| file(s) of the main document.
This behaviour can be altered by the alternative form
|\childdocby[*]{|\textit{main}|}| (with a non-empty optional argument)
which uses the |.aux| file of the main document
by setting |\jobname| to \textit{main}.

%%%%%%%%%%%%%%%%%%%%%%%%%%%%%%%%%%%%%%%%%%%%%%%%%%%%%%%%%%%%%%%%%%%%%%%%%%%%%%%%
\subsection{Driver Development}
\label{sec:driver}

The \textsf{childdoc} mechanism can also be use for the development
of definition files such as \LaTeX{} styles or classes.
This case differs from the above setup with multiple parts
included by |\include| in that no |\includeonly| should be invoked.
This can be achieved by starting the include file
(before |\ProvidesPackage|) with:
%
\begin{center}
\begin{tabular}{l}
|\input{childdoc.def}|\\
|\childdocforward{|\textit{main}|}|\\
\end{tabular}
\end{center}
%
or alternatively with:
%
\begin{center}
\begin{tabular}{l}
|\input{childdoc.def}|\\
|\childdocby{|\textit{main}|}|\\
\end{tabular}
\end{center}
%
Both forms have slightly different effects as described above.
The main file is prepared as usual, see \secref{sec:include}.

%%%%%%%%%%%%%%%%%%%%%%%%%%%%%%%%%%%%%%%%%%%%%%%%%%%%%%%%%%%%%%%%%%%%%%%%%%%%%%%%
\subsection{Legacy Detection}
\label{sec:detection}

The directive |\childdocmain| in the main file can detect
whether the complete document or merely a child is to be compiled
even without using the directive |\childdocof|.
This method is deprecated because it is less robust
and there is no compelling reason to use it;
it is merely provided for backward compatibility
and it may be removed in future versions.

If the detection mechanism is to be used,
it is mandatory to correctly specify
the filename of the main file as the argument of |\childdocmain|:
%
\begin{center}
\begin{tabular}{l}
|\input{childdoc.def}|\\
|\childdocmain{|\textit{main}|}|\\
\end{tabular}
\end{center}
%
If |\jobname| does not match the argument \textit{main} of |\childdocmain|,
it is assumed that |\jobname| points to the child file to be compiled.
When using |\childdocmain| with the main file specified as argument,
it suffices to start a child file
with just |\input{|\textit{main}|}|
without loading of the package and using |\childdocof|.
If instead all processing is done
with the appropriate \textsf{childdoc} directives,
the argument of \textit{main} of |\childdocmain| can be empty.

An alternative version of the command line processing described
in \secref{sec:commandline} using the detection mechanism reads:
%
\begin{center}
|... -jobname "|\textit{target}|" "|[\textit{flags}]%
[|\def\jobname{|\textit{dest}|}|]|\input{|\textit{main}|}"|
\end{center}

%%%%%%%%%%%%%%%%%%%%%%%%%%%%%%%%%%%%%%%%%%%%%%%%%%%%%%%%%%%%%%%%%%%%%%%%%%%%%%%%
\subsection{Manual Code}
\label{sec:manual}

In case one cannot be certain whether the definitions file |childdoc.def|
is installed on the target \TeX{} distribution
and one prefers not to ship it,
it is conceivable to paste a few relevant commands into the sources.

To that end, drop all statements |\input{childdoc.def}|
and perform the replacements as outlined below.
Instead of |\childdocmain{|\textit{main}|}| add the following code
to the top of the main file:
%
\begin{center}
\begin{tabular}{l}
|\||ifdefined\childdocname\endinput\||fi\newif\ifchilddoc|\\
|\edef\childdocname{\scantokens\expandafter{\jobname\noexpand}}|\\
|\def\childdocmain{|\textit{main}|}\||ifx\childdocmain\childdocname\||else|\\
|\childdoctrue\includeonly{\childdocname}\let\jobname\childdocmain\||fi|\\
\end{tabular}
\end{center}
%
Instead of |\childdocof{|\textit{main}|}| just include the main file
at the top of each child file:
%
\begin{center}
|\input{|\textit{main}|}|
\end{center}
%
A simple redirection |\childdocforward{|\textit{dest}|}| is achieved by:
%
\begin{center}
|\def\jobname{|\textit{dest}|}\input{\jobname}|
\end{center}
%
The redirection with prefix
|\childdocforwardprefix[|\textit{prefix}|]{|\textit{dest}|}|
is accomplished by:
%
\begin{center}
\begin{tabular}{l}
|{\edef\jobname{\scantokens\expandafter{\jobname\noexpand}}|\\
|\def\redirectjob |\textit{prefix}|#1~~~{\gdef\jobname{|\textit{dest}|#1}}|\\
|\expandafter\redirectjob\jobname~~~}\input{\jobname}|
\end{tabular}
\end{center}

In an alternative approach,
child documents can be compiled by a specific command line
without additional code or specific definitions:
%
\begin{center}
|... -jobname "|\textit{target}|" "|[\textit{flags}]%
|\includeonly{|\textit{dest}|}\input{|\textit{main}|}"|
\end{center}
%

%%%%%%%%%%%%%%%%%%%%%%%%%%%%%%%%%%%%%%%%%%%%%%%%%%%%%%%%%%%%%%%%%%%%%%%%%%%%%%%%
%%%%%%%%%%%%%%%%%%%%%%%%%%%%%%%%%%%%%%%%%%%%%%%%%%%%%%%%%%%%%%%%%%%%%%%%%%%%%%%%
\section{Information}

%%%%%%%%%%%%%%%%%%%%%%%%%%%%%%%%%%%%%%%%%%%%%%%%%%%%%%%%%%%%%%%%%%%%%%%%%%%%%%%%
\subsection{Copyright}

Copyright \copyright{} 2017--2018 Niklas Beisert

This work may be distributed and/or modified under the
conditions of the \LaTeX{} Project Public License, either version 1.3
of this license or (at your option) any later version.
The latest version of this license is in
  \url{http://www.latex-project.org/lppl.txt}
and version 1.3 or later is part of all distributions of \LaTeX{}
version 2005/12/01 or later.

This work has the LPPL maintenance status `maintained'.

The Current Maintainer of this work is Niklas Beisert.

This work consists of the files |README.txt|, |childdoc.ins| and |childdoc.dtx|
as well as the derived files |childdoc.def|, |cdocsamp.tex|
with |cdocsch1.tex|, |cdocsch2.tex|, |cdocspt3.tex|, |cdocspt4.tex|,
|cdocsdrf.tex|, |cdocsfn1.tex|, |cdocsfn2.tex|
as well as |childdoc.pdf|.

%%%%%%%%%%%%%%%%%%%%%%%%%%%%%%%%%%%%%%%%%%%%%%%%%%%%%%%%%%%%%%%%%%%%%%%%%%%%%%%%
\subsection{Files and Installation}

The package consists of the files:
%
\begin{center}
\begin{tabular}{ll}
    |README.txt|   & readme file \\
    |childdoc.ins| & installation file \\
    |childdoc.dtx| & source file \\
    |childdoc.def| & definition file \\
    |cdocsamp.tex| & sample main file \\
    |cdocsch1.tex| & sample include file \\
    |cdocsch2.tex| & sample include file \\
    |cdocspt3.tex| & sample part file \\
    |cdocspt4.tex| & sample part file \\
    |cdocsdrf.tex| & sample redirection file \\
    |cdocsfn1.tex| & sample redirection file \\
    |cdocsfn2.tex| & sample redirection file \\
    |childdoc.pdf| & manual
\end{tabular}
\end{center}
%
The distribution consists of the files
|README.txt|, |childdoc.ins| and |childdoc.dtx|.
%
\begin{itemize}
\item
Run (pdf)\LaTeX{} on |childdoc.dtx|
to compile the manual |childdoc.pdf| (this file).
\item
Run \LaTeX{} on |childdoc.ins| to create the definitions file |childdoc.def|
and the sample |cdocsamp.tex| with include files
|cdocsch1.tex|, |cdocsch2.tex|, |cdocspt3.tex|, |cdocspt4.tex|,
|cdocsdrf.tex|, |cdocsfn1.tex|, |cdocsfn2.tex|.
Then copy the file |childdoc.def| to an appropriate directory of your \LaTeX{}
distribution, e.g.\ \textit{texmf-root}|/tex/latex/childdoc|.
\end{itemize}

%%%%%%%%%%%%%%%%%%%%%%%%%%%%%%%%%%%%%%%%%%%%%%%%%%%%%%%%%%%%%%%%%%%%%%%%%%%%%%%%
\subsection{Related CTAN Packages}

There are several other packages which offer a similar functionality:
%
\begin{itemize}
\item
The packages
\href{http://ctan.org/pkg/docmute}{\textsf{docmute}},
\href{http://ctan.org/pkg/includex}{\textsf{includex}} and
\href{http://ctan.org/pkg/standalone}{\textsf{standalone}}
provide commands to include only the document body of
a child file thus allowing both files to be compiled individually.
\item
The packages \href{http://ctan.org/pkg/subdocs}{\textsf{subdocs}}
and \href{http://ctan.org/pkg/subfiles}{\textsf{subfiles}}
provide structures in which the main and child documents can be
encapsulated and allowing them to be compiled individually.
The inclusion mechanism is different from the conventional |\include|.
\item
The package \href{http://ctan.org/pkg/combine}{\textsf{combine}}
is an elaborate solution to combine several documents into one.
\end{itemize}
%
See also the CTAN topic \href{http://ctan.org/topic/subdocs}{\textsf{subdocs}}
for further related packages.
The present package differs from the above solutions in that
a document structure constructed with the conventional |\include| mechanism
just needs two extra commands at the top of every file
such that all constituent files can be compiled individually.

%%%%%%%%%%%%%%%%%%%%%%%%%%%%%%%%%%%%%%%%%%%%%%%%%%%%%%%%%%%%%%%%%%%%%%%%%%%%%%%%
%\subsection{Feature Suggestions}
%
%The following is a list of features which may be useful for future
%versions of this package:
%%
%\begin{itemize}
%\item
%\ldots
%\end{itemize}

%%%%%%%%%%%%%%%%%%%%%%%%%%%%%%%%%%%%%%%%%%%%%%%%%%%%%%%%%%%%%%%%%%%%%%%%%%%%%%%%
\subsection{Revision History}

%%%%%%%%%%%%%%%%%%%%%%%%%%%%%%%%%%%%%%%%
\paragraph{v2.0:} 2018/12/30

\begin{itemize}
\item
immediate forward processing
\item
added |\childdocby| mechanism
\item
manual restructured
\end{itemize}

%%%%%%%%%%%%%%%%%%%%%%%%%%%%%%%%%%%%%%%%
\paragraph{v1.6:} 2018/01/17

\begin{itemize}
\item
application for development of include files
\item
corrections to manual
\end{itemize}

%%%%%%%%%%%%%%%%%%%%%%%%%%%%%%%%%%%%%%%%
\paragraph{v1.5:} 2017/05/21

\begin{itemize}
\item
more complete structuring introduced
\item
|\childdocof| introduced
\item
|\childdoc| renamed to |\childdocmain|
\item
|\childredirect| renamed to |\childdocforward| and |\childdocforwardprefix|
and functionality expanded
\end{itemize}

%%%%%%%%%%%%%%%%%%%%%%%%%%%%%%%%%%%%%%%%
\paragraph{v1.0:} 2017/04/27

\begin{itemize}
\item
manual and install package
\item
first version published on CTAN
\end{itemize}

%%%%%%%%%%%%%%%%%%%%%%%%%%%%%%%%%%%%%%%%
\paragraph{v0.6:} 2017/04/26

\begin{itemize}
\item
redirection mechanism added
\end{itemize}

%%%%%%%%%%%%%%%%%%%%%%%%%%%%%%%%%%%%%%%%
\paragraph{v0.5:} 2017/04/26

\begin{itemize}
\item
functionality in definition file
\end{itemize}


%%%%%%%%%%%%%%%%%%%%%%%%%%%%%%%%%%%%%%%%%%%%%%%%%%%%%%%%%%%%%%%%%%%%%%%%%%%%%%%%
%%%%%%%%%%%%%%%%%%%%%%%%%%%%%%%%%%%%%%%%%%%%%%%%%%%%%%%%%%%%%%%%%%%%%%%%%%%%%%%%
%%%%%%%%%%%%%%%%%%%%%%%%%%%%%%%%%%%%%%%%%%%%%%%%%%%%%%%%%%%%%%%%%%%%%%%%%%%%%%%%
\appendix

\settowidth\MacroIndent{\rmfamily\scriptsize 000\ }

 \DocInput{childdoc.dtx}

\end{document}
%</driver>
% \fi
%
% %%%%%%%%%%%%%%%%%%%%%%%%%%%%%%%%%%%%%%%%%%%%%%%%%%%%%%%%%%%%%%%%%%%%%%%%%%%%%%
% %%%%%%%%%%%%%%%%%%%%%%%%%%%%%%%%%%%%%%%%%%%%%%%%%%%%%%%%%%%%%%%%%%%%%%%%%%%%%%
% \section{Sample}
%\iffalse
%<*samplemain>
%\fi
%
% The following presents a sample document
% with two chapters, two parts, a title page,
% a compile flag as well as three forwarding files to set the flag.
% It consists of eight |.tex| files:
% \begin{center}
% \begin{tabular}{ll}
% |cdocsamp.tex|&main file\\
% |cdocsch1.tex|&include file for chapter 1\\
% |cdocsch2.tex|&include file for chapter 2\\
% |cdocspt3.tex|&include file for part 3\\
% |cdocspt4.tex|&include file for part 4\\
% |cdocsdrf.tex|&forwarding file for main file in draft mode\\
% |cdocsfi1.tex|&forwarding file for final version of chapter 1\\
% |cdocsfi2.tex|&forwarding file for final version of chapter 2\\
% \end{tabular}
% \end{center}
% Each of the eight files can be compiled directly by the \LaTeX{} compiler.
%
% %%%%%%%%%%%%%%%%%%%%%%%%%%%%%%%%%%%%%%
% \paragraph{Main File.}
%
% The main file is called |cdocsamp.tex|.
%
% Load the \textsf{childdoc} definitions and
% declare the filename for the main document:
%    \begin{macrocode}
\input{childdoc.def}
\childdocmain{}
%    \end{macrocode}

% Optional override for |\version| flag:
%    \begin{macrocode}
%%\ifchilddoc\else\providecommand{\version}{draft}\fi
%    \end{macrocode}

% Define the default values for the |\version| flag
% (|final| for the main file and |draft| for childs):
%    \begin{macrocode}
\ifchilddoc
\providecommand{\version}{draft}
\else
\providecommand{\version}{final}
\fi
%    \end{macrocode}

% Load the standard document class:
%    \begin{macrocode}
\documentclass[12pt]{article}
%    \end{macrocode}

% Start the document body:
%    \begin{macrocode}
\begin{document}
%    \end{macrocode}

% Declare a title page.
% Print title, part of document being processed and version flag:
%    \begin{macrocode}
\addtocounter{page}{-1}
\begin{center}
{\LARGE\bfseries{}childdoc example\par}
\vspace{1cm}
\ifchilddoc
\ifchilddocmanual part\else chapter\fi:
`\childdocname' of `\childdocjob'\par
\else
main document: `\childdocjob'\par
\fi
version: \version\par
\end{center}
\newpage
%    \end{macrocode}

% Manually include selected file,
% otherwise process as usual:
%    \begin{macrocode}
\ifchilddocmanual
\section*{part `\childdocname'}
\input{\childdocname}
\else
%    \end{macrocode}

% Include the two chapters:
%    \begin{macrocode}
\include{cdocsch1}
\include{cdocsch2}
%    \end{macrocode}

% Include the two parts unless only chapters should be displayed:
%    \begin{macrocode}
\ifchilddoc\else
\section{part three}
\input{cdocspt3}
\section{part four}
\input{cdocspt4}
\fi
%    \end{macrocode}

% Process as usual until here:
%    \begin{macrocode}
\fi
%    \end{macrocode}

% End of document body:
%    \begin{macrocode}
\end{document}
%    \end{macrocode}
%\iffalse
%</samplemain>
%\fi
%
% %%%%%%%%%%%%%%%%%%%%%%%%%%%%%%%%%%%%%%
% \paragraph{Chapter Include Files.}
%
% The include files are called |cdocsch1.tex| and |cdocsch2.tex|.
%
%\iffalse
%<*samplechap1|samplechap2>
%\fi

% Optional override for |\version| flag:
%    \begin{macrocode}
%%\providecommand{\version}{final}
%    \end{macrocode}

% Include the main document:
%    \begin{macrocode}
\input{childdoc.def}
\childdocof{cdocsamp}
%    \end{macrocode}

%\iffalse
%</samplechap1|samplechap2>
%\fi
%
%\iffalse
%<*samplechap1>
%\fi
% Some text for chapter 1:
%    \begin{macrocode}
\section{one}
some text in chapter one
%    \end{macrocode}

%\iffalse
%</samplechap1>
%\fi
% Some text for chapter 2:
%\iffalse
%<*samplechap2>
%\fi
%    \begin{macrocode}
\section{two}
more text in chapter two
%    \end{macrocode}

%\iffalse
%</samplechap2>
%\fi
%
% %%%%%%%%%%%%%%%%%%%%%%%%%%%%%%%%%%%%%%
% \paragraph{Part Include Files.}
%
% The include files are called |cdocspt3.tex| and |cdocspt4.tex|.
%
%\iffalse
%<*samplepart3|samplepart4>
%\fi

% Optional override for |\version| flag:
%    \begin{macrocode}
%%\providecommand{\version}{final}
%    \end{macrocode}

% Include the main document:
%    \begin{macrocode}
\input{childdoc.def}
\childdocby{cdocsamp}
%    \end{macrocode}

%\iffalse
%</samplepart3|samplepart4>
%\fi
%
%\iffalse
%<*samplepart3>
%\fi
% Some text for part 3:
%    \begin{macrocode}
some text in part three
%    \end{macrocode}

%\iffalse
%</samplepart3>
%\fi
% Some text for part 4:
%\iffalse
%<*samplepart4>
%\fi
%    \begin{macrocode}
more text in part four
%    \end{macrocode}

%\iffalse
%</samplepart4>
%\fi
%
% %%%%%%%%%%%%%%%%%%%%%%%%%%%%%%%%%%%%%%
% \paragraph{Forwarding for a Complete Draft.}
%
% The following forwarding file |cdocsdrf.tex|
% compiles the main document in draft mode:
%\iffalse
%<*sampledraft>
%\fi
%    \begin{macrocode}
\def\version{draft}
\input{childdoc.def}
\childdocforward{cdocsamp}
%    \end{macrocode}

%\iffalse
%</sampledraft>
%\fi
%
% %%%%%%%%%%%%%%%%%%%%%%%%%%%%%%%%%%%%%%
% \paragraph{Forwarding for Final Version of the Chapters.}
%
% The following forwarding files |cdocsfn1.tex| and |cdocsfn2.tex|
% (with identical content)
% compile the final versions of the child documents
% |cdocsch1.tex| and |cdocsch2.tex|, respectively:
%\iffalse
%<*samplefinal>
%\fi
%    \begin{macrocode}
\def\version{final}
\input{childdoc.def}
\childdocforwardprefix[cdocsamp]{cdocsfn}{cdocsch}
%    \end{macrocode}

%\iffalse
%</samplefinal>
%\fi
%
% %%%%%%%%%%%%%%%%%%%%%%%%%%%%%%%%%%%%%%
% \paragraph{Command Line Processing.}
%
% The following three command lines generate the output files
% |cdocscld|, |cdocscl1| and |cdocscl2|
% which should be identical to
% |cdocsdrf|, |cdocsch1| and |cdocsfn2|, respectively:
% \begin{center}
% \begin{tabular}{l}
% |latex -jobname cdocscld \|\\
% |  "\def\version{draft}\input{childdoc.def}\childdocforward{cdocsamp}"|\\
% |latex -jobname cdocscl1 \|\\
% |  "\input{childdoc.def}\childdocforward[cdocsamp]{cdocsch1}"|\\
% |latex -jobname cdocscl2 \|\\
% |  "\def\version{final}\input{childdoc.def}\childdocforward{cdocsch2}"|
% \end{tabular}
% \end{center}
% Note that the trailing backslash on each first line
% merely continues the input to the second line
% (for convenient cut ant paste).
% Furthermore, the command |latex| can be replaced by any
% of its alternative versions such as |pdflatex|.
%
% %%%%%%%%%%%%%%%%%%%%%%%%%%%%%%%%%%%%%%%%%%%%%%%%%%%%%%%%%%%%%%%%%%%%%%%%%%%%%%
% %%%%%%%%%%%%%%%%%%%%%%%%%%%%%%%%%%%%%%%%%%%%%%%%%%%%%%%%%%%%%%%%%%%%%%%%%%%%%%
% \section{Implementation}
%\iffalse
%<*package>
%\fi
%
% This section describes the definitions file |childdoc.def|.

% The definitions cannot be loaded using |\usepackage| or |\RequirePackage|
% which has a mechanism to prevent loading a style file more than once.
% When loading the definitions by means of |\input|
% multiple instances have to be prevented manually:
%\iffalse
%This code needs to be before the `\ProvidesFile' directive
%which is defined at the beginning of this file.
%Therefore it is also placed there and commented out here.
%</package>
%<*discard>
%\fi
%    \begin{macrocode}
\ifdefined\childdocmain\endinput\fi
%    \end{macrocode}
%\iffalse
%</discard>
%<*package>
%\fi
%
% \macro{\ifchilddoc}
% \macro{\ifchilddocmanual}
% The conditional |\ifchilddoc| tells whether a
% child (true) or main (false) document is being compiled.
% The conditional |\ifchilddocmanual| tells whether
% the |\includeonly| mechanism is used (false) or
% the selection of child files must be performed manually (true).
% The definitions initialise to false:
%    \begin{macrocode}
\newif\ifchilddoc
\newif\ifchilddocmanual
%    \end{macrocode}

% \macro{\childdocname}
% \macro{\childdocjob}
% The macro |\childdocname| stores the name of the main document
% to be compiled. The macro |\childdocjob| stores the name of
% the document on which the \LaTeX{} compiler was originally invoked.
% The content of |\jobname| cannot be compared
% to filenames specified in the source due to different catcodes.
% The following code rescans |\jobname|, stores the result
% in |\childdocname| and saves a copy in |\childdocjob|:
%    \begin{macrocode}
\edef\childdocname{\scantokens\expandafter{\jobname\noexpand}}
\let\childdocjob\childdocname
%    \end{macrocode}

% \macro{\childdocdisable}
% The macro |\childdocdisable| prevents the main file
% from being processed more than once.
% At this stage, the main document command |\childdocmain|
% is assumed to be called once again where it should do nothing.
% Any subsequent call to it should prevent
% a secondary processing of the main document
% It overwrites the forwarding commands
% |\childdocof| and |\childdocforward|
% with empty macros to prevent further inclusions of the main document:
%    \begin{macrocode}
\newcommand{\childdocdisable}
{
  \renewcommand{\childdocmain}[1]{\renewcommand{\childdocmain}[1]{\endinput}}
  \renewcommand{\childdocof}[1]{}
  \renewcommand{\childdocby}[2][]{}
  \renewcommand{\childdocforward}[2][]{}
  \renewcommand{\childdocdisable}{}
}
%    \end{macrocode}

% \macro{\childdocmain}
% The macro |\childdocmain| is to be called at the top of the main file
% with nothing or the main filename (without extension) as argument.
% First, it breaks loops.
% If the argument is not empty and does not match |\childdocname|
% (which is set by the first inclusion of |childdoc.def|),
% |\ifchilddoc| is set to true, |\includeonly| is applied to the child file
% and |\jobname| is set to the main file
% (for proper handling of |.aux| files):
%    \begin{macrocode}
\newcommand{\childdocmain}[1]
{
  \childdocdisable\childdocmain{}
  \if?#1?\else
    \begingroup
      \def\childdoctmp{#1}
      \ifx\childdoctmp\childdocname
        \def\childdoctmp{}
      \else
        \def\childdoctmp
        {
          \childdoctrue
          \includeonly{\childdocname}
          \def\childdocjob{#1}
          \def\jobname{#1}
        }
      \fi
      \expandafter
    \endgroup
    \childdoctmp
  \fi
}
%    \end{macrocode}

% \macro{\childdocof}
% The command |\childdocof| redirects
% compilation to the main file |#1|.
%    \begin{macrocode}
\newcommand{\childdocof}[1]
{
  \childdocdisable
  \childdoctrue
  \includeonly{\childdocname}
  \def\jobname{#1}
  \def\childdocjob{#1}
  \input{#1}
}
%    \end{macrocode}

% \macro{\childdocby}
% The command |\childdocby| ....
%    \begin{macrocode}
\newcommand{\childdocby}[2][]
{
  \childdocdisable
  \childdoctrue
  \childdocmanualtrue
  \if?#1?\else
    \def\jobname{#2}
  \fi
  \def\childdocjob{#2}
  \input{#2}
  \endinput
}
%    \end{macrocode}

% \macro{\childdocforward}
% The command |\childdocforward| redirects
% compilation to the main file or
% (if the optional argument is given) a child file.
% Parameters are set as if the main file
% or a child file starting with |\childdocof| was compiled.
% Then compilation is handed over to the main file:
%    \begin{macrocode}
\newcommand{\childdocforward}[2][]
{
  \begingroup
    \if?#1?
      \def\childdoctmp
      {
        \def\childdocname{#2}
        \def\childdocjob{#2}
        \def\jobname{#2}
        \input{#2}
        \endinput
      }
    \else
      \def\childdoctmp
      {
        \childdocdisable
        \def\childdocname{#2}
        \childdoctrue
        \includeonly{#2}
        \def\childdocjob{#1}
        \def\jobname{#1}
        \input{#1}
        \endinput
      }
    \fi
    \expandafter
  \endgroup
  \childdoctmp
}
%    \end{macrocode}

% \macro{\childdocforwardprefix}
% The command |\childdocforwardprefix| redirects
% compilation to the main or a child file by means of a pattern.
% The prefix |#1| in the current filename is replaced by |#2|
% and the suffix of the current filename is kept
% (it is assumed that the filename does not contain the substring `|~~~|'
% which is used as a delimiter).
% Compilation is handed over to the new file by |\childdocforward|:
%    \begin{macrocode}
\newcommand{\childdocforwardprefix}[3][]
{
  \begingroup
    \def\childdocextract #2##1~~~{\def\childdoctmp{\childdocforward[#1]{#3##1}}}
    \expandafter\childdocextract\childdocname~~~
    \expandafter
  \endgroup
  \childdoctmp
}
%    \end{macrocode}

% \macro{\childdoc}
% The deprecated macro |\childdoc| is a legacy version of |\childdocmain|:
%    \begin{macrocode}
\newcommand{\childdoc}{\childdocmain}
%    \end{macrocode}

% \macro{\childdocredirect}
% The deprecated macro |\childdocredirect| is a legacy version
% of |\childdocforward| and |\childdocforwardprefix|:
%    \begin{macrocode}
\newcommand{\childdocredirect}[2][]
{
  \begingroup
    \if?#1?
      \def\childdoctmp{\childdocforward{#2}}
    \else
      \def\childdoctmp{\childdocforwardprefix{#1}{#2}}
    \fi
    \expandafter
  \endgroup
  \childdoctmp
}
%    \end{macrocode}

%\iffalse
%</package>
%\fi
%
\endinput

\childdocof{cdocsamp}
%    \end{macrocode}

%\iffalse
%</samplechap1|samplechap2>
%\fi
%
%\iffalse
%<*samplechap1>
%\fi
% Some text for chapter 1:
%    \begin{macrocode}
\section{one}
some text in chapter one
%    \end{macrocode}

%\iffalse
%</samplechap1>
%\fi
% Some text for chapter 2:
%\iffalse
%<*samplechap2>
%\fi
%    \begin{macrocode}
\section{two}
more text in chapter two
%    \end{macrocode}

%\iffalse
%</samplechap2>
%\fi
%
% %%%%%%%%%%%%%%%%%%%%%%%%%%%%%%%%%%%%%%
% \paragraph{Part Include Files.}
%
% The include files are called |cdocspt3.tex| and |cdocspt4.tex|.
%
%\iffalse
%<*samplepart3|samplepart4>
%\fi

% Optional override for |\version| flag:
%    \begin{macrocode}
%%\providecommand{\version}{final}
%    \end{macrocode}

% Include the main document:
%    \begin{macrocode}
% \iffalse
%
% childdoc.dtx Copyright (C) 2017-2018 Niklas Beisert
%
% This work may be distributed and/or modified under the
% conditions of the LaTeX Project Public License, either version 1.3
% of this license or (at your option) any later version.
% The latest version of this license is in
%   http://www.latex-project.org/lppl.txt
% and version 1.3 or later is part of all distributions of LaTeX
% version 2005/12/01 or later.
%
% This work has the LPPL maintenance status `maintained'.
%
% The Current Maintainer of this work is Niklas Beisert.
%
% This work consists of the files childdoc.dtx and childdoc.ins
% and the derived files childdoc.def and cdocsamp.tex with
% cdocsch1.tex, cdocsch2.tex, cdocsdrf.tex, cdocsfn1.tex, cdocsfn2.tex.
%
%<package>\ifdefined\childdocmain\endinput\fi
%<package>\ProvidesFile{childdoc.def}[2018/12/30 v2.0 child document driver]
%<samplemain>\ProvidesFile{cdocsamp.tex}[2018/12/30 v2.0 sample for childdoc]
%<*driver>
%\ProvidesFile{childdoc.drv}[2018/12/30 v2.0 childdoc reference manual file]
\PassOptionsToClass{10pt,a4paper}{article}
\documentclass{ltxdoc}

\usepackage[margin=35mm]{geometry}
\usepackage{hyperref}
\usepackage{hyperxmp}
\usepackage[usenames]{color}

\hypersetup{colorlinks=true}
\hypersetup{pdfstartview=FitH}
\hypersetup{pdfpagemode=UseNone}
\hypersetup{pdfsource={}}
\hypersetup{pdflang={en-UK}}
\hypersetup{pdfcopyright={Copyright 2017-2018 Niklas Beisert.
  This work may be distributed and/or modified under the
  conditions of the LaTeX Project Public License, either version 1.3
  of this license or (at your option) any later version.}}
\hypersetup{pdflicenseurl={http://www.latex-project.org/lppl.txt}}
\hypersetup{pdfcontactaddress={ETH Zurich, ITP, HIT K,
  Wolfgang-Pauli-Strasse 27}}
\hypersetup{pdfcontactpostcode={8093}}
\hypersetup{pdfcontactcity={Zurich}}
\hypersetup{pdfcontactcountry={Switzerland}}
\hypersetup{pdfcontactemail={nbeisert@itp.phys.ethz.ch}}
\hypersetup{pdfcontacturl={http://people.phys.ethz.ch/\xmptilde nbeisert/}}

\newcommand{\secref}[1]{\hyperref[#1]{section \ref*{#1}}}

\parskip1ex
\parindent0pt
\let\olditemize\itemize
\def\itemize{\olditemize\parskip0pt}

\begin{document}

\title{The \textsf{childdoc} Package}
\hypersetup{pdftitle={The childdoc Package}}
\author{Niklas Beisert\\[2ex]
  Institut f\"ur Theoretische Physik\\
  Eidgen\"ossische Technische Hochschule Z\"urich\\
  Wolfgang-Pauli-Strasse 27, 8093 Z\"urich, Switzerland\\[1ex]
  \href{mailto:nbeisert@itp.phys.ethz.ch}
  {\texttt{nbeisert@itp.phys.ethz.ch}}}
\hypersetup{pdfauthor={Niklas Beisert}}
\hypersetup{pdfsubject={Manual for the LaTeX2e Package childdoc}}
\date{30 December 2018, \textsf{v2.0}}
\maketitle

\begin{abstract}\noindent
\textsf{childdoc} is a \LaTeXe{} package
that enables the direct compilation
of document sections included by |\include|
to individual files.
\end{abstract}

\begingroup
\parskip0ex
\tableofcontents
\endgroup

%%%%%%%%%%%%%%%%%%%%%%%%%%%%%%%%%%%%%%%%%%%%%%%%%%%%%%%%%%%%%%%%%%%%%%%%%%%%%%%%
%%%%%%%%%%%%%%%%%%%%%%%%%%%%%%%%%%%%%%%%%%%%%%%%%%%%%%%%%%%%%%%%%%%%%%%%%%%%%%%%
\section{Introduction}

\LaTeX{} provides a mechanism to structure a large document (such as a book)
into a main file and several child files (containing the chapters)
using the |\include| command.
This mechanism is beneficial for documents
which span hundreds of pages in order to
make the source file(s) more manageable.
Moreover, compilation can be restricted to
selected child files by means of the |\includeonly| command.
The latter feature can be used to reduce the compilation time while editing
(this was significantly more useful in the earlier days of \LaTeX{})
or to generate a smaller document which is easier to navigate.
Another application of |\includeonly| is to generate
documents consisting of selected parts of the complete document.

However, there are a few drawbacks of the plain |\include| mechanism:
\begin{itemize}
\item
The child files cannot be compiled on their own,
they can only be compiled via the main file.
A naive editing environment
(such as a text editor with an option
to have the current file processed by \LaTeX)
may require one to switch to the main file before compiling;
attempting to compile the child file produces errors.
\item
The main file must be modified (each time)
to adjust the |\includeonly| command
to the present needs. This easily leaves the main file in a messy state.
\item
The generated document will always carry the filename
of the main document. This is inconvenient if
several child files are to be compiled and
to be kept for distribution.
\end{itemize}

The present package provides a simple interface
to make child files individually compilable by \LaTeX{}.
Compiling a child file then has the same effect as compiling
the main file with an |\includeonly| command
to select the appropriate child.
Moreover the generated document will carry the name of the child
rather than the main file.
This resolves all three above issues.

This feature is meant to make the editing of books,
thesis documents and lecture notes somewhat more convenient.
However, the package can also be used efficiently for
composing a series of documents (such as exercise sheets)
which are typically distributed individually.
It then assists the author in generating the individual documents
(potentially in different versions)
as well as a document containing the collected series.
Another application is in developing style files
or other kinds of included material
where compilation of the style file could redirect
to a sample or test file.

%%%%%%%%%%%%%%%%%%%%%%%%%%%%%%%%%%%%%%%%%%%%%%%%%%%%%%%%%%%%%%%%%%%%%%%%%%%%%%%%
%%%%%%%%%%%%%%%%%%%%%%%%%%%%%%%%%%%%%%%%%%%%%%%%%%%%%%%%%%%%%%%%%%%%%%%%%%%%%%%%
\section{Usage}

First of all, the package \textsf{childdoc} is \emph{not} a standard
\LaTeXe{} |.sty| style file! Therefore it needs to be invoked in
a non-standard way.

%%%%%%%%%%%%%%%%%%%%%%%%%%%%%%%%%%%%%%%%%%%%%%%%%%%%%%%%%%%%%%%%%%%%%%%%%%%%%%%%
\subsection{Included Files}
\label{sec:include}

%%%%%%%%%%%%%%%%%%%%%%%%%%%%%%%%%%%%%%%%
\DescribeMacro{\childdocmain}
To use the package, add the commands
\begin{center}
\begin{tabular}{l}
|\input{childdoc.def}|\\
|\childdocmain{}|\\
\end{tabular}
\end{center}
at the very top of the main \LaTeX{} file,
in particular \emph{before} the |\documentclass| statement!
The argument of |\childdocmain| should be left empty
(but it must be present).

%%%%%%%%%%%%%%%%%%%%%%%%%%%%%%%%%%%%%%%%
\DescribeMacro{\childdocof}
Furthermore, add the commands
\begin{center}
\begin{tabular}{l}
|\input{childdoc.def}|\\
|\childdocof{|\textit{main}|}|\\
\end{tabular}
\end{center}
at the top of every child file \textit{child}
which is included by |\include{|\textit{child}|}|
from within the main file
(or at least for those files to be compiled individually).
The argument \textit{main} must be the filename of the main file.

There are a couple of
considerations in setting up the main and child documents:

%%%%%%%%%%%%%%%%%%%%%%%%%%%%%%%%%%%%%%%%
\paragraph{Restrictions.}

Please note the following restrictions:
\begin{itemize}
\item
|\childdocmain| must be called with one argument \textit{main}
to ensure compatibility with earlier version of the package.
It must either be empty (|\childdocmain{}|)
or precisely match the filename of the main file in which it is specified.
See \secref{sec:detection} for further information.
\item
The filename \textit{main} must be specified without the |.tex| extension.
\item
The filename \textit{main} is case sensitive
(even in case-insensitive file systems)
due to internal string comparison.
\item
The argument \textit{main} should be fully expanded, it cannot be a macro.
\item
Subdirectories and special characters should be avoided in filenames.
\item
The command |\childdocmain{|\textit{main}|}| must be followed by a whitespace.
It should not be followed immediately by another command
or by a comment mark `|%|'.
This is because the \TeX{} parser reads the token immediately following
the argument of |\childdocmain| and puts it
at the beginning of every child section;
however, a white\-space is ignored.
\end{itemize}

%%%%%%%%%%%%%%%%%%%%%%%%%%%%%%%%%%%%%%%%
\paragraph{Content of Main File.}

It is advisable to place all content in the child files included by |\include|.
Any output contained in the main file will appear in all child documents
unless suppressed manually;
it cannot be suppressed automatically by the |\includeonly| directive
and thus should normally be avoided.
A method to include some content in the main file
by means of conditional processing is described in \secref{sec:conditional}.

%%%%%%%%%%%%%%%%%%%%%%%%%%%%%%%%%%%%%%%%
\paragraph{Page Numbering.}

When only a part of the document is compiled,
the appropriate numbering of pages
(as well as other status parameters)
is determined from the |.aux| files.
The latter contain information from previous passes.
However this information needs to propagate through
all intermediate child documents.
Therefore the page numbering in child documents may well
be inconsistent until the complete document is compiled at least once.

A useful (if unconventional) way to always ensure a consistent
page numbering is to restart the numbering in each child document
and denote the pages by `\textit{child}|.|\textit{page}'
where \textit{child} represents the chapter/section number of the child file.
This can be achieved by the command
|\numberwithin{page}{|\textit{child}|}|
of the \textsf{amsmath} package
where \textit{child} can be |chapter| or |section|
depending on the chosen structuring.
Alternatively, one can modify the macro |\thepage| appropriately
and reset the counter |page| at the start of each child file.

%%%%%%%%%%%%%%%%%%%%%%%%%%%%%%%%%%%%%%%%%%%%%%%%%%%%%%%%%%%%%%%%%%%%%%%%%%%%%%%%
\subsection{Conditional Processing}
\label{sec:conditional}

The package provides a mechanism to compile different versions
of a document. To customise the versions further some conditional processing
can come in handy to distinguish which version is being compiled.
The package provides two macros to describe the compilation context:

%%%%%%%%%%%%%%%%%%%%%%%%%%%%%%%%%%%%%%%%
\DescribeMacro{\ifchilddoc}
The conditional |\ifchilddoc| distinguishes between the compilation of
child documents and the main document:
%
\begin{center}
|\ifchilddoc |\textit{child-code}| |[|\||else |\textit{main-code}]| \||fi|
\end{center}

%%%%%%%%%%%%%%%%%%%%%%%%%%%%%%%%%%%%%%%%
\DescribeMacro{\childdocname}
\DescribeMacro{\childdocjob}
The macro |\childdocname| contains the filename (without extension)
of the main or child file being processed.
Note that |\childdocjob| will always contain the name of the main file.

%%%%%%%%%%%%%%%%%%%%%%%%%%%%%%%%%%%%%%%%
\paragraph{Title Page.}

Conditional processing can be used to include a title or banner page
in the main document when proper precautions are taken.
Importantly, the code in the main file should ensure that the page counter
(as well as other status parameters which are stored in the |.aux| files)
takes the same value after the conditional processing.
Otherwise the page numbers may take divergent values
depending on which part is compiled.

For example, a title page could be declared by:
%
\begin{center}
\begin{tabular}{l}
|\ifchilddoc\||else|\\
|\addtocounter{page}{-1}|\\
\textit{code for title page}\\
|\newpage|\\
|\||fi|
\end{tabular}
\end{center}
%
A banner page for the child documents can be generated by:
%
\begin{center}
\begin{tabular}{l}
|\ifchilddoc|\\
|\addtocounter{page}{-1}|\\
\textit{code for banner page}\\
|\newpage|\\
|\||fi|
\end{tabular}
\end{center}
%
Here one could write a message such as:
\begin{center}
|This is the part \childdocname{} of \childdocjob{}.|
\end{center}

%%%%%%%%%%%%%%%%%%%%%%%%%%%%%%%%%%%%%%%%%%%%%%%%%%%%%%%%%%%%%%%%%%%%%%%%%%%%%%%%
\subsection{Flags}
\label{sec:flags}

The package makes it easy to generate different versions
of the main or child documents.
To this end compilation flags can be defined
and assigned different default values.
They will be particularly useful in conjunction
with the forwarding mechanism described in \secref{sec:forward}.

For example, it may be useful to have a flag |\version|
which can be set to |draft| or |final|.
The document source will contain some conditional code
depending on the value of |\version|.
Suppose further, the flag should default to |final| for the main file
and to |draft| for child files
which is a natural assignment for editing the document.
This is achieved by placing the following code
in the preamble of the main document
(below the |\childdocmain| directive):
%
\begin{center}
\begin{tabular}{l}
|\ifchilddoc|\\
|\providecommand{\version}{draft}|\\
|\||else|\\
|\providecommand{\version}{final}|\\
|\||fi|
\end{tabular}
\end{center}
%
The definition by |\providecommand| makes sure
that previous definitions are not overwritten.
Further statements |\providecommand{\version}{...}|
can thus be added before the above code to override it.

For the main file, one might add a line
(between |\childdocmain| and the above block)
%
\begin{center}
|%\ifchilddoc\||else\providecommand{\version}{draft}\||fi|
\end{center}
%
which can be uncommented to produce a draft version.
Likewise one can add a line to the very top of a child file
(above the |\childdocof{|\textit{main}|}| directive)
%
\begin{center}
|%\providecommand{\version}{final}|
\end{center}
%
which can be uncommented to produce the final version of this child document.

%%%%%%%%%%%%%%%%%%%%%%%%%%%%%%%%%%%%%%%%%%%%%%%%%%%%%%%%%%%%%%%%%%%%%%%%%%%%%%%%
\subsection{Forwarding}
\label{sec:forward}

Different versions of the main or child documents
using compilation flags as described in \secref{sec:flags}
can be (permanently) stored in different files
for convenient compilation, viewing and distribution.
To this end, the package defines a command
to pass on compilation to a different file:

%%%%%%%%%%%%%%%%%%%%%%%%%%%%%%%%%%%%%%%%
\DescribeMacro{\childdocforward}
The command |\childdocforward| redirects processing to
another source file:
%
\begin{center}
\begin{tabular}{l}
|\input{childdoc.def}|\\
|\childdocforward[|\textit{main}|]{|\textit{dest}|}|\\
\end{tabular}
\end{center}
%
The argument \textit{dest} is the destination file
(without extension).
It should be the main file or one of the child files.
Note that further \textsf{childdoc} directives
such as |\childdocof| and |\childdocforward|
in the indicated file will be processed in this form.
The optional argument \textit{main}
passes on directly to the main file \textit{main}
while pretending to compile the child \textit{dest}.
This form behaves as if \textit{dest}
issues |\childdocof{|\textit{main}|}| right away,
and no further \textsf{childdoc} directives will be processed.

%%%%%%%%%%%%%%%%%%%%%%%%%%%%%%%%%%%%%%%%
\DescribeMacro{\...prefix}
In the alternative form |\childdocforwardprefix|,
%
\begin{center}
\begin{tabular}{l}
|\input{childdoc.def}|\\
|\childdocforwardprefix[|\textit{main}|]{|\textit{prefix}|}{|\textit{dest}|}|
\end{tabular}
\end{center}
%
the destination file is determined by a pattern
depending on the current file:
To make this work, the current file must be called
`{\textit{prefix}\hspace{0.2em}\textit{suffix}}'
with \textit{prefix} matching precisely the argument.
Processing is then passed on to the file
`{\textit{dest}\hspace{0.2em}\textit{suffix}}'.
Surely, the same effect is achieved by
directly specifying the
argument `{\textit{dest}\hspace{0.2em}\textit{suffix}}'
in the first form.
However, that requires to set up a different file
for each child. With the alternative form of the command
all these files can have exactly the same content
which simplifies setting them up and maintaining them.

For example, the following file |draft.tex|
with a compilation flag |\version| as described in \secref{sec:flags}
compiles the main document as a draft:
%
\begin{center}
\begin{tabular}{l}
|\def\version{draft}|\\
|\input{childdoc.def}|\\
|\childdocforward{|\textit{main}|}|
\end{tabular}
\end{center}
%
Likewise, the following files |final|\textit{nn}|.tex|
compile the final version of the child document
|child|\textit{nn}|.tex|:
%
\begin{center}
\begin{tabular}{l}
|\def\version{final}|\\
|\input{childdoc.def}|\\
|\childdocforwardprefix{final}{child}|
\end{tabular}
\end{center}
%

Note that when several versions of a main file and/or of each child file
are to be generated, it may be convenient to set up a |Makefile| or
shell script to automatise the process.

%%%%%%%%%%%%%%%%%%%%%%%%%%%%%%%%%%%%%%%%%%%%%%%%%%%%%%%%%%%%%%%%%%%%%%%%%%%%%%%%
\subsection{Command Line Processing}
\label{sec:commandline}

The effect of redirection files can also be achieved by invoking
the \LaTeX{} compiler with a more elaborate command line.
Most conveniently this should be done as part
of a shell script or a |Makefile|.

When using \textsf{childdoc} in the main file, the following
command lines effectively perform a redirection
(note that depending on the shell being used,
backslashes may have to be doubled: `|\|' $\to$ `|\\|'):
%
\begin{center}
|... -jobname "|\textit{target}|" |\\|"|[\textit{flags}]%
|\input{childdoc.def}\childdocforward[|\textit{main}|]{|\textit{dest}|}"|
\end{center}
%
Here \textit{target} is the name of the output file,
\textit{main} is the name of the main file
and \textit{dest} is the name of the main or child file to be processed
(all filenames without extensions).
The optional argument \textit{main} can be omitted
if \textit{main} matches \textit{dest}.
Optionally, compilation \textit{flags} can be defined via |\def| commands.
This command line makes the \TeX{} engine believe
it is compiling the file \textit{target}
whose content is specified as the latter parameter.
The provided code then forwards the processing to
\textit{main} or \textit{dest} as described in \secref{sec:forward}.

%%%%%%%%%%%%%%%%%%%%%%%%%%%%%%%%%%%%%%%%%%%%%%%%%%%%%%%%%%%%%%%%%%%%%%%%%%%%%%%%
\subsection{Include by Input}
\label{sec:input}

Including child documents by |\include| has some restrictions by design.
Most notably, the content of a child document always occupies
its own set of pages; pages cannot be shared between child documents.
Usually, this behaviour makes perfect sense
because each child document contain an essential part of the document.
However, in some situations it may be desirable to compose
a document from a collection of parts
without having mandatory page breaks between then.
For this case, the package
provides a mechanism to include parts
by |\input| which can also be processed individually.
However, by construction this mechanism
requires manual handling of the content to be output.

%%%%%%%%%%%%%%%%%%%%%%%%%%%%%%%%%%%%%%%%
\DescribeMacro{\ifchilddocmanual}
The main file should be prepared as usual, see \secref{sec:include}.
However, the document body must make a distinction
between processing of an individual part and of the main document, e.g.:
%
\begin{center}
\begin{tabular}{l}
|\ifchilddocmanual|\\
|\input{\childdocname}|\\
|\||else|\\
\textit{document body with }|\input{|\textit{part}|}|\\
|\||fi|
\end{tabular}
\end{center}
%
The conditional |\ifchilddocmanual| is true whenever
a part to be included by |\input| is being compiled,
and the name of the part is stored in |\childdocname|.

%%%%%%%%%%%%%%%%%%%%%%%%%%%%%%%%%%%%%%%%
\DescribeMacro{\childdocby}
Each part to be included by |\input| should start with:
%
\begin{center}
\begin{tabular}{l}
|\input{childdoc.def}|\\
|\childdocby{|\textit{main}|}|\\
\end{tabular}
\end{center}
%
The directive |\childdocby| is similar to |\childdocof|
described in \secref{sec:include},
but the subsequent selection of content must be done manually.
To that end, both |\ifchilddoc| and |\ifchilddocmanual|
will be true upon processing of a part,
and the name of the part is stored in |\childdocname|.
Note that |\jobname| will be set to the filename of the current part
so that each part receives an individual |.aux| file
that does not interfere with the |.aux| file(s) of the main document.
This behaviour can be altered by the alternative form
|\childdocby[*]{|\textit{main}|}| (with a non-empty optional argument)
which uses the |.aux| file of the main document
by setting |\jobname| to \textit{main}.

%%%%%%%%%%%%%%%%%%%%%%%%%%%%%%%%%%%%%%%%%%%%%%%%%%%%%%%%%%%%%%%%%%%%%%%%%%%%%%%%
\subsection{Driver Development}
\label{sec:driver}

The \textsf{childdoc} mechanism can also be use for the development
of definition files such as \LaTeX{} styles or classes.
This case differs from the above setup with multiple parts
included by |\include| in that no |\includeonly| should be invoked.
This can be achieved by starting the include file
(before |\ProvidesPackage|) with:
%
\begin{center}
\begin{tabular}{l}
|\input{childdoc.def}|\\
|\childdocforward{|\textit{main}|}|\\
\end{tabular}
\end{center}
%
or alternatively with:
%
\begin{center}
\begin{tabular}{l}
|\input{childdoc.def}|\\
|\childdocby{|\textit{main}|}|\\
\end{tabular}
\end{center}
%
Both forms have slightly different effects as described above.
The main file is prepared as usual, see \secref{sec:include}.

%%%%%%%%%%%%%%%%%%%%%%%%%%%%%%%%%%%%%%%%%%%%%%%%%%%%%%%%%%%%%%%%%%%%%%%%%%%%%%%%
\subsection{Legacy Detection}
\label{sec:detection}

The directive |\childdocmain| in the main file can detect
whether the complete document or merely a child is to be compiled
even without using the directive |\childdocof|.
This method is deprecated because it is less robust
and there is no compelling reason to use it;
it is merely provided for backward compatibility
and it may be removed in future versions.

If the detection mechanism is to be used,
it is mandatory to correctly specify
the filename of the main file as the argument of |\childdocmain|:
%
\begin{center}
\begin{tabular}{l}
|\input{childdoc.def}|\\
|\childdocmain{|\textit{main}|}|\\
\end{tabular}
\end{center}
%
If |\jobname| does not match the argument \textit{main} of |\childdocmain|,
it is assumed that |\jobname| points to the child file to be compiled.
When using |\childdocmain| with the main file specified as argument,
it suffices to start a child file
with just |\input{|\textit{main}|}|
without loading of the package and using |\childdocof|.
If instead all processing is done
with the appropriate \textsf{childdoc} directives,
the argument of \textit{main} of |\childdocmain| can be empty.

An alternative version of the command line processing described
in \secref{sec:commandline} using the detection mechanism reads:
%
\begin{center}
|... -jobname "|\textit{target}|" "|[\textit{flags}]%
[|\def\jobname{|\textit{dest}|}|]|\input{|\textit{main}|}"|
\end{center}

%%%%%%%%%%%%%%%%%%%%%%%%%%%%%%%%%%%%%%%%%%%%%%%%%%%%%%%%%%%%%%%%%%%%%%%%%%%%%%%%
\subsection{Manual Code}
\label{sec:manual}

In case one cannot be certain whether the definitions file |childdoc.def|
is installed on the target \TeX{} distribution
and one prefers not to ship it,
it is conceivable to paste a few relevant commands into the sources.

To that end, drop all statements |\input{childdoc.def}|
and perform the replacements as outlined below.
Instead of |\childdocmain{|\textit{main}|}| add the following code
to the top of the main file:
%
\begin{center}
\begin{tabular}{l}
|\||ifdefined\childdocname\endinput\||fi\newif\ifchilddoc|\\
|\edef\childdocname{\scantokens\expandafter{\jobname\noexpand}}|\\
|\def\childdocmain{|\textit{main}|}\||ifx\childdocmain\childdocname\||else|\\
|\childdoctrue\includeonly{\childdocname}\let\jobname\childdocmain\||fi|\\
\end{tabular}
\end{center}
%
Instead of |\childdocof{|\textit{main}|}| just include the main file
at the top of each child file:
%
\begin{center}
|\input{|\textit{main}|}|
\end{center}
%
A simple redirection |\childdocforward{|\textit{dest}|}| is achieved by:
%
\begin{center}
|\def\jobname{|\textit{dest}|}\input{\jobname}|
\end{center}
%
The redirection with prefix
|\childdocforwardprefix[|\textit{prefix}|]{|\textit{dest}|}|
is accomplished by:
%
\begin{center}
\begin{tabular}{l}
|{\edef\jobname{\scantokens\expandafter{\jobname\noexpand}}|\\
|\def\redirectjob |\textit{prefix}|#1~~~{\gdef\jobname{|\textit{dest}|#1}}|\\
|\expandafter\redirectjob\jobname~~~}\input{\jobname}|
\end{tabular}
\end{center}

In an alternative approach,
child documents can be compiled by a specific command line
without additional code or specific definitions:
%
\begin{center}
|... -jobname "|\textit{target}|" "|[\textit{flags}]%
|\includeonly{|\textit{dest}|}\input{|\textit{main}|}"|
\end{center}
%

%%%%%%%%%%%%%%%%%%%%%%%%%%%%%%%%%%%%%%%%%%%%%%%%%%%%%%%%%%%%%%%%%%%%%%%%%%%%%%%%
%%%%%%%%%%%%%%%%%%%%%%%%%%%%%%%%%%%%%%%%%%%%%%%%%%%%%%%%%%%%%%%%%%%%%%%%%%%%%%%%
\section{Information}

%%%%%%%%%%%%%%%%%%%%%%%%%%%%%%%%%%%%%%%%%%%%%%%%%%%%%%%%%%%%%%%%%%%%%%%%%%%%%%%%
\subsection{Copyright}

Copyright \copyright{} 2017--2018 Niklas Beisert

This work may be distributed and/or modified under the
conditions of the \LaTeX{} Project Public License, either version 1.3
of this license or (at your option) any later version.
The latest version of this license is in
  \url{http://www.latex-project.org/lppl.txt}
and version 1.3 or later is part of all distributions of \LaTeX{}
version 2005/12/01 or later.

This work has the LPPL maintenance status `maintained'.

The Current Maintainer of this work is Niklas Beisert.

This work consists of the files |README.txt|, |childdoc.ins| and |childdoc.dtx|
as well as the derived files |childdoc.def|, |cdocsamp.tex|
with |cdocsch1.tex|, |cdocsch2.tex|, |cdocspt3.tex|, |cdocspt4.tex|,
|cdocsdrf.tex|, |cdocsfn1.tex|, |cdocsfn2.tex|
as well as |childdoc.pdf|.

%%%%%%%%%%%%%%%%%%%%%%%%%%%%%%%%%%%%%%%%%%%%%%%%%%%%%%%%%%%%%%%%%%%%%%%%%%%%%%%%
\subsection{Files and Installation}

The package consists of the files:
%
\begin{center}
\begin{tabular}{ll}
    |README.txt|   & readme file \\
    |childdoc.ins| & installation file \\
    |childdoc.dtx| & source file \\
    |childdoc.def| & definition file \\
    |cdocsamp.tex| & sample main file \\
    |cdocsch1.tex| & sample include file \\
    |cdocsch2.tex| & sample include file \\
    |cdocspt3.tex| & sample part file \\
    |cdocspt4.tex| & sample part file \\
    |cdocsdrf.tex| & sample redirection file \\
    |cdocsfn1.tex| & sample redirection file \\
    |cdocsfn2.tex| & sample redirection file \\
    |childdoc.pdf| & manual
\end{tabular}
\end{center}
%
The distribution consists of the files
|README.txt|, |childdoc.ins| and |childdoc.dtx|.
%
\begin{itemize}
\item
Run (pdf)\LaTeX{} on |childdoc.dtx|
to compile the manual |childdoc.pdf| (this file).
\item
Run \LaTeX{} on |childdoc.ins| to create the definitions file |childdoc.def|
and the sample |cdocsamp.tex| with include files
|cdocsch1.tex|, |cdocsch2.tex|, |cdocspt3.tex|, |cdocspt4.tex|,
|cdocsdrf.tex|, |cdocsfn1.tex|, |cdocsfn2.tex|.
Then copy the file |childdoc.def| to an appropriate directory of your \LaTeX{}
distribution, e.g.\ \textit{texmf-root}|/tex/latex/childdoc|.
\end{itemize}

%%%%%%%%%%%%%%%%%%%%%%%%%%%%%%%%%%%%%%%%%%%%%%%%%%%%%%%%%%%%%%%%%%%%%%%%%%%%%%%%
\subsection{Related CTAN Packages}

There are several other packages which offer a similar functionality:
%
\begin{itemize}
\item
The packages
\href{http://ctan.org/pkg/docmute}{\textsf{docmute}},
\href{http://ctan.org/pkg/includex}{\textsf{includex}} and
\href{http://ctan.org/pkg/standalone}{\textsf{standalone}}
provide commands to include only the document body of
a child file thus allowing both files to be compiled individually.
\item
The packages \href{http://ctan.org/pkg/subdocs}{\textsf{subdocs}}
and \href{http://ctan.org/pkg/subfiles}{\textsf{subfiles}}
provide structures in which the main and child documents can be
encapsulated and allowing them to be compiled individually.
The inclusion mechanism is different from the conventional |\include|.
\item
The package \href{http://ctan.org/pkg/combine}{\textsf{combine}}
is an elaborate solution to combine several documents into one.
\end{itemize}
%
See also the CTAN topic \href{http://ctan.org/topic/subdocs}{\textsf{subdocs}}
for further related packages.
The present package differs from the above solutions in that
a document structure constructed with the conventional |\include| mechanism
just needs two extra commands at the top of every file
such that all constituent files can be compiled individually.

%%%%%%%%%%%%%%%%%%%%%%%%%%%%%%%%%%%%%%%%%%%%%%%%%%%%%%%%%%%%%%%%%%%%%%%%%%%%%%%%
%\subsection{Feature Suggestions}
%
%The following is a list of features which may be useful for future
%versions of this package:
%%
%\begin{itemize}
%\item
%\ldots
%\end{itemize}

%%%%%%%%%%%%%%%%%%%%%%%%%%%%%%%%%%%%%%%%%%%%%%%%%%%%%%%%%%%%%%%%%%%%%%%%%%%%%%%%
\subsection{Revision History}

%%%%%%%%%%%%%%%%%%%%%%%%%%%%%%%%%%%%%%%%
\paragraph{v2.0:} 2018/12/30

\begin{itemize}
\item
immediate forward processing
\item
added |\childdocby| mechanism
\item
manual restructured
\end{itemize}

%%%%%%%%%%%%%%%%%%%%%%%%%%%%%%%%%%%%%%%%
\paragraph{v1.6:} 2018/01/17

\begin{itemize}
\item
application for development of include files
\item
corrections to manual
\end{itemize}

%%%%%%%%%%%%%%%%%%%%%%%%%%%%%%%%%%%%%%%%
\paragraph{v1.5:} 2017/05/21

\begin{itemize}
\item
more complete structuring introduced
\item
|\childdocof| introduced
\item
|\childdoc| renamed to |\childdocmain|
\item
|\childredirect| renamed to |\childdocforward| and |\childdocforwardprefix|
and functionality expanded
\end{itemize}

%%%%%%%%%%%%%%%%%%%%%%%%%%%%%%%%%%%%%%%%
\paragraph{v1.0:} 2017/04/27

\begin{itemize}
\item
manual and install package
\item
first version published on CTAN
\end{itemize}

%%%%%%%%%%%%%%%%%%%%%%%%%%%%%%%%%%%%%%%%
\paragraph{v0.6:} 2017/04/26

\begin{itemize}
\item
redirection mechanism added
\end{itemize}

%%%%%%%%%%%%%%%%%%%%%%%%%%%%%%%%%%%%%%%%
\paragraph{v0.5:} 2017/04/26

\begin{itemize}
\item
functionality in definition file
\end{itemize}


%%%%%%%%%%%%%%%%%%%%%%%%%%%%%%%%%%%%%%%%%%%%%%%%%%%%%%%%%%%%%%%%%%%%%%%%%%%%%%%%
%%%%%%%%%%%%%%%%%%%%%%%%%%%%%%%%%%%%%%%%%%%%%%%%%%%%%%%%%%%%%%%%%%%%%%%%%%%%%%%%
%%%%%%%%%%%%%%%%%%%%%%%%%%%%%%%%%%%%%%%%%%%%%%%%%%%%%%%%%%%%%%%%%%%%%%%%%%%%%%%%
\appendix

\settowidth\MacroIndent{\rmfamily\scriptsize 000\ }

 \DocInput{childdoc.dtx}

\end{document}
%</driver>
% \fi
%
% %%%%%%%%%%%%%%%%%%%%%%%%%%%%%%%%%%%%%%%%%%%%%%%%%%%%%%%%%%%%%%%%%%%%%%%%%%%%%%
% %%%%%%%%%%%%%%%%%%%%%%%%%%%%%%%%%%%%%%%%%%%%%%%%%%%%%%%%%%%%%%%%%%%%%%%%%%%%%%
% \section{Sample}
%\iffalse
%<*samplemain>
%\fi
%
% The following presents a sample document
% with two chapters, two parts, a title page,
% a compile flag as well as three forwarding files to set the flag.
% It consists of eight |.tex| files:
% \begin{center}
% \begin{tabular}{ll}
% |cdocsamp.tex|&main file\\
% |cdocsch1.tex|&include file for chapter 1\\
% |cdocsch2.tex|&include file for chapter 2\\
% |cdocspt3.tex|&include file for part 3\\
% |cdocspt4.tex|&include file for part 4\\
% |cdocsdrf.tex|&forwarding file for main file in draft mode\\
% |cdocsfi1.tex|&forwarding file for final version of chapter 1\\
% |cdocsfi2.tex|&forwarding file for final version of chapter 2\\
% \end{tabular}
% \end{center}
% Each of the eight files can be compiled directly by the \LaTeX{} compiler.
%
% %%%%%%%%%%%%%%%%%%%%%%%%%%%%%%%%%%%%%%
% \paragraph{Main File.}
%
% The main file is called |cdocsamp.tex|.
%
% Load the \textsf{childdoc} definitions and
% declare the filename for the main document:
%    \begin{macrocode}
\input{childdoc.def}
\childdocmain{}
%    \end{macrocode}

% Optional override for |\version| flag:
%    \begin{macrocode}
%%\ifchilddoc\else\providecommand{\version}{draft}\fi
%    \end{macrocode}

% Define the default values for the |\version| flag
% (|final| for the main file and |draft| for childs):
%    \begin{macrocode}
\ifchilddoc
\providecommand{\version}{draft}
\else
\providecommand{\version}{final}
\fi
%    \end{macrocode}

% Load the standard document class:
%    \begin{macrocode}
\documentclass[12pt]{article}
%    \end{macrocode}

% Start the document body:
%    \begin{macrocode}
\begin{document}
%    \end{macrocode}

% Declare a title page.
% Print title, part of document being processed and version flag:
%    \begin{macrocode}
\addtocounter{page}{-1}
\begin{center}
{\LARGE\bfseries{}childdoc example\par}
\vspace{1cm}
\ifchilddoc
\ifchilddocmanual part\else chapter\fi:
`\childdocname' of `\childdocjob'\par
\else
main document: `\childdocjob'\par
\fi
version: \version\par
\end{center}
\newpage
%    \end{macrocode}

% Manually include selected file,
% otherwise process as usual:
%    \begin{macrocode}
\ifchilddocmanual
\section*{part `\childdocname'}
\input{\childdocname}
\else
%    \end{macrocode}

% Include the two chapters:
%    \begin{macrocode}
\include{cdocsch1}
\include{cdocsch2}
%    \end{macrocode}

% Include the two parts unless only chapters should be displayed:
%    \begin{macrocode}
\ifchilddoc\else
\section{part three}
\input{cdocspt3}
\section{part four}
\input{cdocspt4}
\fi
%    \end{macrocode}

% Process as usual until here:
%    \begin{macrocode}
\fi
%    \end{macrocode}

% End of document body:
%    \begin{macrocode}
\end{document}
%    \end{macrocode}
%\iffalse
%</samplemain>
%\fi
%
% %%%%%%%%%%%%%%%%%%%%%%%%%%%%%%%%%%%%%%
% \paragraph{Chapter Include Files.}
%
% The include files are called |cdocsch1.tex| and |cdocsch2.tex|.
%
%\iffalse
%<*samplechap1|samplechap2>
%\fi

% Optional override for |\version| flag:
%    \begin{macrocode}
%%\providecommand{\version}{final}
%    \end{macrocode}

% Include the main document:
%    \begin{macrocode}
\input{childdoc.def}
\childdocof{cdocsamp}
%    \end{macrocode}

%\iffalse
%</samplechap1|samplechap2>
%\fi
%
%\iffalse
%<*samplechap1>
%\fi
% Some text for chapter 1:
%    \begin{macrocode}
\section{one}
some text in chapter one
%    \end{macrocode}

%\iffalse
%</samplechap1>
%\fi
% Some text for chapter 2:
%\iffalse
%<*samplechap2>
%\fi
%    \begin{macrocode}
\section{two}
more text in chapter two
%    \end{macrocode}

%\iffalse
%</samplechap2>
%\fi
%
% %%%%%%%%%%%%%%%%%%%%%%%%%%%%%%%%%%%%%%
% \paragraph{Part Include Files.}
%
% The include files are called |cdocspt3.tex| and |cdocspt4.tex|.
%
%\iffalse
%<*samplepart3|samplepart4>
%\fi

% Optional override for |\version| flag:
%    \begin{macrocode}
%%\providecommand{\version}{final}
%    \end{macrocode}

% Include the main document:
%    \begin{macrocode}
\input{childdoc.def}
\childdocby{cdocsamp}
%    \end{macrocode}

%\iffalse
%</samplepart3|samplepart4>
%\fi
%
%\iffalse
%<*samplepart3>
%\fi
% Some text for part 3:
%    \begin{macrocode}
some text in part three
%    \end{macrocode}

%\iffalse
%</samplepart3>
%\fi
% Some text for part 4:
%\iffalse
%<*samplepart4>
%\fi
%    \begin{macrocode}
more text in part four
%    \end{macrocode}

%\iffalse
%</samplepart4>
%\fi
%
% %%%%%%%%%%%%%%%%%%%%%%%%%%%%%%%%%%%%%%
% \paragraph{Forwarding for a Complete Draft.}
%
% The following forwarding file |cdocsdrf.tex|
% compiles the main document in draft mode:
%\iffalse
%<*sampledraft>
%\fi
%    \begin{macrocode}
\def\version{draft}
\input{childdoc.def}
\childdocforward{cdocsamp}
%    \end{macrocode}

%\iffalse
%</sampledraft>
%\fi
%
% %%%%%%%%%%%%%%%%%%%%%%%%%%%%%%%%%%%%%%
% \paragraph{Forwarding for Final Version of the Chapters.}
%
% The following forwarding files |cdocsfn1.tex| and |cdocsfn2.tex|
% (with identical content)
% compile the final versions of the child documents
% |cdocsch1.tex| and |cdocsch2.tex|, respectively:
%\iffalse
%<*samplefinal>
%\fi
%    \begin{macrocode}
\def\version{final}
\input{childdoc.def}
\childdocforwardprefix[cdocsamp]{cdocsfn}{cdocsch}
%    \end{macrocode}

%\iffalse
%</samplefinal>
%\fi
%
% %%%%%%%%%%%%%%%%%%%%%%%%%%%%%%%%%%%%%%
% \paragraph{Command Line Processing.}
%
% The following three command lines generate the output files
% |cdocscld|, |cdocscl1| and |cdocscl2|
% which should be identical to
% |cdocsdrf|, |cdocsch1| and |cdocsfn2|, respectively:
% \begin{center}
% \begin{tabular}{l}
% |latex -jobname cdocscld \|\\
% |  "\def\version{draft}\input{childdoc.def}\childdocforward{cdocsamp}"|\\
% |latex -jobname cdocscl1 \|\\
% |  "\input{childdoc.def}\childdocforward[cdocsamp]{cdocsch1}"|\\
% |latex -jobname cdocscl2 \|\\
% |  "\def\version{final}\input{childdoc.def}\childdocforward{cdocsch2}"|
% \end{tabular}
% \end{center}
% Note that the trailing backslash on each first line
% merely continues the input to the second line
% (for convenient cut ant paste).
% Furthermore, the command |latex| can be replaced by any
% of its alternative versions such as |pdflatex|.
%
% %%%%%%%%%%%%%%%%%%%%%%%%%%%%%%%%%%%%%%%%%%%%%%%%%%%%%%%%%%%%%%%%%%%%%%%%%%%%%%
% %%%%%%%%%%%%%%%%%%%%%%%%%%%%%%%%%%%%%%%%%%%%%%%%%%%%%%%%%%%%%%%%%%%%%%%%%%%%%%
% \section{Implementation}
%\iffalse
%<*package>
%\fi
%
% This section describes the definitions file |childdoc.def|.

% The definitions cannot be loaded using |\usepackage| or |\RequirePackage|
% which has a mechanism to prevent loading a style file more than once.
% When loading the definitions by means of |\input|
% multiple instances have to be prevented manually:
%\iffalse
%This code needs to be before the `\ProvidesFile' directive
%which is defined at the beginning of this file.
%Therefore it is also placed there and commented out here.
%</package>
%<*discard>
%\fi
%    \begin{macrocode}
\ifdefined\childdocmain\endinput\fi
%    \end{macrocode}
%\iffalse
%</discard>
%<*package>
%\fi
%
% \macro{\ifchilddoc}
% \macro{\ifchilddocmanual}
% The conditional |\ifchilddoc| tells whether a
% child (true) or main (false) document is being compiled.
% The conditional |\ifchilddocmanual| tells whether
% the |\includeonly| mechanism is used (false) or
% the selection of child files must be performed manually (true).
% The definitions initialise to false:
%    \begin{macrocode}
\newif\ifchilddoc
\newif\ifchilddocmanual
%    \end{macrocode}

% \macro{\childdocname}
% \macro{\childdocjob}
% The macro |\childdocname| stores the name of the main document
% to be compiled. The macro |\childdocjob| stores the name of
% the document on which the \LaTeX{} compiler was originally invoked.
% The content of |\jobname| cannot be compared
% to filenames specified in the source due to different catcodes.
% The following code rescans |\jobname|, stores the result
% in |\childdocname| and saves a copy in |\childdocjob|:
%    \begin{macrocode}
\edef\childdocname{\scantokens\expandafter{\jobname\noexpand}}
\let\childdocjob\childdocname
%    \end{macrocode}

% \macro{\childdocdisable}
% The macro |\childdocdisable| prevents the main file
% from being processed more than once.
% At this stage, the main document command |\childdocmain|
% is assumed to be called once again where it should do nothing.
% Any subsequent call to it should prevent
% a secondary processing of the main document
% It overwrites the forwarding commands
% |\childdocof| and |\childdocforward|
% with empty macros to prevent further inclusions of the main document:
%    \begin{macrocode}
\newcommand{\childdocdisable}
{
  \renewcommand{\childdocmain}[1]{\renewcommand{\childdocmain}[1]{\endinput}}
  \renewcommand{\childdocof}[1]{}
  \renewcommand{\childdocby}[2][]{}
  \renewcommand{\childdocforward}[2][]{}
  \renewcommand{\childdocdisable}{}
}
%    \end{macrocode}

% \macro{\childdocmain}
% The macro |\childdocmain| is to be called at the top of the main file
% with nothing or the main filename (without extension) as argument.
% First, it breaks loops.
% If the argument is not empty and does not match |\childdocname|
% (which is set by the first inclusion of |childdoc.def|),
% |\ifchilddoc| is set to true, |\includeonly| is applied to the child file
% and |\jobname| is set to the main file
% (for proper handling of |.aux| files):
%    \begin{macrocode}
\newcommand{\childdocmain}[1]
{
  \childdocdisable\childdocmain{}
  \if?#1?\else
    \begingroup
      \def\childdoctmp{#1}
      \ifx\childdoctmp\childdocname
        \def\childdoctmp{}
      \else
        \def\childdoctmp
        {
          \childdoctrue
          \includeonly{\childdocname}
          \def\childdocjob{#1}
          \def\jobname{#1}
        }
      \fi
      \expandafter
    \endgroup
    \childdoctmp
  \fi
}
%    \end{macrocode}

% \macro{\childdocof}
% The command |\childdocof| redirects
% compilation to the main file |#1|.
%    \begin{macrocode}
\newcommand{\childdocof}[1]
{
  \childdocdisable
  \childdoctrue
  \includeonly{\childdocname}
  \def\jobname{#1}
  \def\childdocjob{#1}
  \input{#1}
}
%    \end{macrocode}

% \macro{\childdocby}
% The command |\childdocby| ....
%    \begin{macrocode}
\newcommand{\childdocby}[2][]
{
  \childdocdisable
  \childdoctrue
  \childdocmanualtrue
  \if?#1?\else
    \def\jobname{#2}
  \fi
  \def\childdocjob{#2}
  \input{#2}
  \endinput
}
%    \end{macrocode}

% \macro{\childdocforward}
% The command |\childdocforward| redirects
% compilation to the main file or
% (if the optional argument is given) a child file.
% Parameters are set as if the main file
% or a child file starting with |\childdocof| was compiled.
% Then compilation is handed over to the main file:
%    \begin{macrocode}
\newcommand{\childdocforward}[2][]
{
  \begingroup
    \if?#1?
      \def\childdoctmp
      {
        \def\childdocname{#2}
        \def\childdocjob{#2}
        \def\jobname{#2}
        \input{#2}
        \endinput
      }
    \else
      \def\childdoctmp
      {
        \childdocdisable
        \def\childdocname{#2}
        \childdoctrue
        \includeonly{#2}
        \def\childdocjob{#1}
        \def\jobname{#1}
        \input{#1}
        \endinput
      }
    \fi
    \expandafter
  \endgroup
  \childdoctmp
}
%    \end{macrocode}

% \macro{\childdocforwardprefix}
% The command |\childdocforwardprefix| redirects
% compilation to the main or a child file by means of a pattern.
% The prefix |#1| in the current filename is replaced by |#2|
% and the suffix of the current filename is kept
% (it is assumed that the filename does not contain the substring `|~~~|'
% which is used as a delimiter).
% Compilation is handed over to the new file by |\childdocforward|:
%    \begin{macrocode}
\newcommand{\childdocforwardprefix}[3][]
{
  \begingroup
    \def\childdocextract #2##1~~~{\def\childdoctmp{\childdocforward[#1]{#3##1}}}
    \expandafter\childdocextract\childdocname~~~
    \expandafter
  \endgroup
  \childdoctmp
}
%    \end{macrocode}

% \macro{\childdoc}
% The deprecated macro |\childdoc| is a legacy version of |\childdocmain|:
%    \begin{macrocode}
\newcommand{\childdoc}{\childdocmain}
%    \end{macrocode}

% \macro{\childdocredirect}
% The deprecated macro |\childdocredirect| is a legacy version
% of |\childdocforward| and |\childdocforwardprefix|:
%    \begin{macrocode}
\newcommand{\childdocredirect}[2][]
{
  \begingroup
    \if?#1?
      \def\childdoctmp{\childdocforward{#2}}
    \else
      \def\childdoctmp{\childdocforwardprefix{#1}{#2}}
    \fi
    \expandafter
  \endgroup
  \childdoctmp
}
%    \end{macrocode}

%\iffalse
%</package>
%\fi
%
\endinput

\childdocby{cdocsamp}
%    \end{macrocode}

%\iffalse
%</samplepart3|samplepart4>
%\fi
%
%\iffalse
%<*samplepart3>
%\fi
% Some text for part 3:
%    \begin{macrocode}
some text in part three
%    \end{macrocode}

%\iffalse
%</samplepart3>
%\fi
% Some text for part 4:
%\iffalse
%<*samplepart4>
%\fi
%    \begin{macrocode}
more text in part four
%    \end{macrocode}

%\iffalse
%</samplepart4>
%\fi
%
% %%%%%%%%%%%%%%%%%%%%%%%%%%%%%%%%%%%%%%
% \paragraph{Forwarding for a Complete Draft.}
%
% The following forwarding file |cdocsdrf.tex|
% compiles the main document in draft mode:
%\iffalse
%<*sampledraft>
%\fi
%    \begin{macrocode}
\def\version{draft}
% \iffalse
%
% childdoc.dtx Copyright (C) 2017-2018 Niklas Beisert
%
% This work may be distributed and/or modified under the
% conditions of the LaTeX Project Public License, either version 1.3
% of this license or (at your option) any later version.
% The latest version of this license is in
%   http://www.latex-project.org/lppl.txt
% and version 1.3 or later is part of all distributions of LaTeX
% version 2005/12/01 or later.
%
% This work has the LPPL maintenance status `maintained'.
%
% The Current Maintainer of this work is Niklas Beisert.
%
% This work consists of the files childdoc.dtx and childdoc.ins
% and the derived files childdoc.def and cdocsamp.tex with
% cdocsch1.tex, cdocsch2.tex, cdocsdrf.tex, cdocsfn1.tex, cdocsfn2.tex.
%
%<package>\ifdefined\childdocmain\endinput\fi
%<package>\ProvidesFile{childdoc.def}[2018/12/30 v2.0 child document driver]
%<samplemain>\ProvidesFile{cdocsamp.tex}[2018/12/30 v2.0 sample for childdoc]
%<*driver>
%\ProvidesFile{childdoc.drv}[2018/12/30 v2.0 childdoc reference manual file]
\PassOptionsToClass{10pt,a4paper}{article}
\documentclass{ltxdoc}

\usepackage[margin=35mm]{geometry}
\usepackage{hyperref}
\usepackage{hyperxmp}
\usepackage[usenames]{color}

\hypersetup{colorlinks=true}
\hypersetup{pdfstartview=FitH}
\hypersetup{pdfpagemode=UseNone}
\hypersetup{pdfsource={}}
\hypersetup{pdflang={en-UK}}
\hypersetup{pdfcopyright={Copyright 2017-2018 Niklas Beisert.
  This work may be distributed and/or modified under the
  conditions of the LaTeX Project Public License, either version 1.3
  of this license or (at your option) any later version.}}
\hypersetup{pdflicenseurl={http://www.latex-project.org/lppl.txt}}
\hypersetup{pdfcontactaddress={ETH Zurich, ITP, HIT K,
  Wolfgang-Pauli-Strasse 27}}
\hypersetup{pdfcontactpostcode={8093}}
\hypersetup{pdfcontactcity={Zurich}}
\hypersetup{pdfcontactcountry={Switzerland}}
\hypersetup{pdfcontactemail={nbeisert@itp.phys.ethz.ch}}
\hypersetup{pdfcontacturl={http://people.phys.ethz.ch/\xmptilde nbeisert/}}

\newcommand{\secref}[1]{\hyperref[#1]{section \ref*{#1}}}

\parskip1ex
\parindent0pt
\let\olditemize\itemize
\def\itemize{\olditemize\parskip0pt}

\begin{document}

\title{The \textsf{childdoc} Package}
\hypersetup{pdftitle={The childdoc Package}}
\author{Niklas Beisert\\[2ex]
  Institut f\"ur Theoretische Physik\\
  Eidgen\"ossische Technische Hochschule Z\"urich\\
  Wolfgang-Pauli-Strasse 27, 8093 Z\"urich, Switzerland\\[1ex]
  \href{mailto:nbeisert@itp.phys.ethz.ch}
  {\texttt{nbeisert@itp.phys.ethz.ch}}}
\hypersetup{pdfauthor={Niklas Beisert}}
\hypersetup{pdfsubject={Manual for the LaTeX2e Package childdoc}}
\date{30 December 2018, \textsf{v2.0}}
\maketitle

\begin{abstract}\noindent
\textsf{childdoc} is a \LaTeXe{} package
that enables the direct compilation
of document sections included by |\include|
to individual files.
\end{abstract}

\begingroup
\parskip0ex
\tableofcontents
\endgroup

%%%%%%%%%%%%%%%%%%%%%%%%%%%%%%%%%%%%%%%%%%%%%%%%%%%%%%%%%%%%%%%%%%%%%%%%%%%%%%%%
%%%%%%%%%%%%%%%%%%%%%%%%%%%%%%%%%%%%%%%%%%%%%%%%%%%%%%%%%%%%%%%%%%%%%%%%%%%%%%%%
\section{Introduction}

\LaTeX{} provides a mechanism to structure a large document (such as a book)
into a main file and several child files (containing the chapters)
using the |\include| command.
This mechanism is beneficial for documents
which span hundreds of pages in order to
make the source file(s) more manageable.
Moreover, compilation can be restricted to
selected child files by means of the |\includeonly| command.
The latter feature can be used to reduce the compilation time while editing
(this was significantly more useful in the earlier days of \LaTeX{})
or to generate a smaller document which is easier to navigate.
Another application of |\includeonly| is to generate
documents consisting of selected parts of the complete document.

However, there are a few drawbacks of the plain |\include| mechanism:
\begin{itemize}
\item
The child files cannot be compiled on their own,
they can only be compiled via the main file.
A naive editing environment
(such as a text editor with an option
to have the current file processed by \LaTeX)
may require one to switch to the main file before compiling;
attempting to compile the child file produces errors.
\item
The main file must be modified (each time)
to adjust the |\includeonly| command
to the present needs. This easily leaves the main file in a messy state.
\item
The generated document will always carry the filename
of the main document. This is inconvenient if
several child files are to be compiled and
to be kept for distribution.
\end{itemize}

The present package provides a simple interface
to make child files individually compilable by \LaTeX{}.
Compiling a child file then has the same effect as compiling
the main file with an |\includeonly| command
to select the appropriate child.
Moreover the generated document will carry the name of the child
rather than the main file.
This resolves all three above issues.

This feature is meant to make the editing of books,
thesis documents and lecture notes somewhat more convenient.
However, the package can also be used efficiently for
composing a series of documents (such as exercise sheets)
which are typically distributed individually.
It then assists the author in generating the individual documents
(potentially in different versions)
as well as a document containing the collected series.
Another application is in developing style files
or other kinds of included material
where compilation of the style file could redirect
to a sample or test file.

%%%%%%%%%%%%%%%%%%%%%%%%%%%%%%%%%%%%%%%%%%%%%%%%%%%%%%%%%%%%%%%%%%%%%%%%%%%%%%%%
%%%%%%%%%%%%%%%%%%%%%%%%%%%%%%%%%%%%%%%%%%%%%%%%%%%%%%%%%%%%%%%%%%%%%%%%%%%%%%%%
\section{Usage}

First of all, the package \textsf{childdoc} is \emph{not} a standard
\LaTeXe{} |.sty| style file! Therefore it needs to be invoked in
a non-standard way.

%%%%%%%%%%%%%%%%%%%%%%%%%%%%%%%%%%%%%%%%%%%%%%%%%%%%%%%%%%%%%%%%%%%%%%%%%%%%%%%%
\subsection{Included Files}
\label{sec:include}

%%%%%%%%%%%%%%%%%%%%%%%%%%%%%%%%%%%%%%%%
\DescribeMacro{\childdocmain}
To use the package, add the commands
\begin{center}
\begin{tabular}{l}
|\input{childdoc.def}|\\
|\childdocmain{}|\\
\end{tabular}
\end{center}
at the very top of the main \LaTeX{} file,
in particular \emph{before} the |\documentclass| statement!
The argument of |\childdocmain| should be left empty
(but it must be present).

%%%%%%%%%%%%%%%%%%%%%%%%%%%%%%%%%%%%%%%%
\DescribeMacro{\childdocof}
Furthermore, add the commands
\begin{center}
\begin{tabular}{l}
|\input{childdoc.def}|\\
|\childdocof{|\textit{main}|}|\\
\end{tabular}
\end{center}
at the top of every child file \textit{child}
which is included by |\include{|\textit{child}|}|
from within the main file
(or at least for those files to be compiled individually).
The argument \textit{main} must be the filename of the main file.

There are a couple of
considerations in setting up the main and child documents:

%%%%%%%%%%%%%%%%%%%%%%%%%%%%%%%%%%%%%%%%
\paragraph{Restrictions.}

Please note the following restrictions:
\begin{itemize}
\item
|\childdocmain| must be called with one argument \textit{main}
to ensure compatibility with earlier version of the package.
It must either be empty (|\childdocmain{}|)
or precisely match the filename of the main file in which it is specified.
See \secref{sec:detection} for further information.
\item
The filename \textit{main} must be specified without the |.tex| extension.
\item
The filename \textit{main} is case sensitive
(even in case-insensitive file systems)
due to internal string comparison.
\item
The argument \textit{main} should be fully expanded, it cannot be a macro.
\item
Subdirectories and special characters should be avoided in filenames.
\item
The command |\childdocmain{|\textit{main}|}| must be followed by a whitespace.
It should not be followed immediately by another command
or by a comment mark `|%|'.
This is because the \TeX{} parser reads the token immediately following
the argument of |\childdocmain| and puts it
at the beginning of every child section;
however, a white\-space is ignored.
\end{itemize}

%%%%%%%%%%%%%%%%%%%%%%%%%%%%%%%%%%%%%%%%
\paragraph{Content of Main File.}

It is advisable to place all content in the child files included by |\include|.
Any output contained in the main file will appear in all child documents
unless suppressed manually;
it cannot be suppressed automatically by the |\includeonly| directive
and thus should normally be avoided.
A method to include some content in the main file
by means of conditional processing is described in \secref{sec:conditional}.

%%%%%%%%%%%%%%%%%%%%%%%%%%%%%%%%%%%%%%%%
\paragraph{Page Numbering.}

When only a part of the document is compiled,
the appropriate numbering of pages
(as well as other status parameters)
is determined from the |.aux| files.
The latter contain information from previous passes.
However this information needs to propagate through
all intermediate child documents.
Therefore the page numbering in child documents may well
be inconsistent until the complete document is compiled at least once.

A useful (if unconventional) way to always ensure a consistent
page numbering is to restart the numbering in each child document
and denote the pages by `\textit{child}|.|\textit{page}'
where \textit{child} represents the chapter/section number of the child file.
This can be achieved by the command
|\numberwithin{page}{|\textit{child}|}|
of the \textsf{amsmath} package
where \textit{child} can be |chapter| or |section|
depending on the chosen structuring.
Alternatively, one can modify the macro |\thepage| appropriately
and reset the counter |page| at the start of each child file.

%%%%%%%%%%%%%%%%%%%%%%%%%%%%%%%%%%%%%%%%%%%%%%%%%%%%%%%%%%%%%%%%%%%%%%%%%%%%%%%%
\subsection{Conditional Processing}
\label{sec:conditional}

The package provides a mechanism to compile different versions
of a document. To customise the versions further some conditional processing
can come in handy to distinguish which version is being compiled.
The package provides two macros to describe the compilation context:

%%%%%%%%%%%%%%%%%%%%%%%%%%%%%%%%%%%%%%%%
\DescribeMacro{\ifchilddoc}
The conditional |\ifchilddoc| distinguishes between the compilation of
child documents and the main document:
%
\begin{center}
|\ifchilddoc |\textit{child-code}| |[|\||else |\textit{main-code}]| \||fi|
\end{center}

%%%%%%%%%%%%%%%%%%%%%%%%%%%%%%%%%%%%%%%%
\DescribeMacro{\childdocname}
\DescribeMacro{\childdocjob}
The macro |\childdocname| contains the filename (without extension)
of the main or child file being processed.
Note that |\childdocjob| will always contain the name of the main file.

%%%%%%%%%%%%%%%%%%%%%%%%%%%%%%%%%%%%%%%%
\paragraph{Title Page.}

Conditional processing can be used to include a title or banner page
in the main document when proper precautions are taken.
Importantly, the code in the main file should ensure that the page counter
(as well as other status parameters which are stored in the |.aux| files)
takes the same value after the conditional processing.
Otherwise the page numbers may take divergent values
depending on which part is compiled.

For example, a title page could be declared by:
%
\begin{center}
\begin{tabular}{l}
|\ifchilddoc\||else|\\
|\addtocounter{page}{-1}|\\
\textit{code for title page}\\
|\newpage|\\
|\||fi|
\end{tabular}
\end{center}
%
A banner page for the child documents can be generated by:
%
\begin{center}
\begin{tabular}{l}
|\ifchilddoc|\\
|\addtocounter{page}{-1}|\\
\textit{code for banner page}\\
|\newpage|\\
|\||fi|
\end{tabular}
\end{center}
%
Here one could write a message such as:
\begin{center}
|This is the part \childdocname{} of \childdocjob{}.|
\end{center}

%%%%%%%%%%%%%%%%%%%%%%%%%%%%%%%%%%%%%%%%%%%%%%%%%%%%%%%%%%%%%%%%%%%%%%%%%%%%%%%%
\subsection{Flags}
\label{sec:flags}

The package makes it easy to generate different versions
of the main or child documents.
To this end compilation flags can be defined
and assigned different default values.
They will be particularly useful in conjunction
with the forwarding mechanism described in \secref{sec:forward}.

For example, it may be useful to have a flag |\version|
which can be set to |draft| or |final|.
The document source will contain some conditional code
depending on the value of |\version|.
Suppose further, the flag should default to |final| for the main file
and to |draft| for child files
which is a natural assignment for editing the document.
This is achieved by placing the following code
in the preamble of the main document
(below the |\childdocmain| directive):
%
\begin{center}
\begin{tabular}{l}
|\ifchilddoc|\\
|\providecommand{\version}{draft}|\\
|\||else|\\
|\providecommand{\version}{final}|\\
|\||fi|
\end{tabular}
\end{center}
%
The definition by |\providecommand| makes sure
that previous definitions are not overwritten.
Further statements |\providecommand{\version}{...}|
can thus be added before the above code to override it.

For the main file, one might add a line
(between |\childdocmain| and the above block)
%
\begin{center}
|%\ifchilddoc\||else\providecommand{\version}{draft}\||fi|
\end{center}
%
which can be uncommented to produce a draft version.
Likewise one can add a line to the very top of a child file
(above the |\childdocof{|\textit{main}|}| directive)
%
\begin{center}
|%\providecommand{\version}{final}|
\end{center}
%
which can be uncommented to produce the final version of this child document.

%%%%%%%%%%%%%%%%%%%%%%%%%%%%%%%%%%%%%%%%%%%%%%%%%%%%%%%%%%%%%%%%%%%%%%%%%%%%%%%%
\subsection{Forwarding}
\label{sec:forward}

Different versions of the main or child documents
using compilation flags as described in \secref{sec:flags}
can be (permanently) stored in different files
for convenient compilation, viewing and distribution.
To this end, the package defines a command
to pass on compilation to a different file:

%%%%%%%%%%%%%%%%%%%%%%%%%%%%%%%%%%%%%%%%
\DescribeMacro{\childdocforward}
The command |\childdocforward| redirects processing to
another source file:
%
\begin{center}
\begin{tabular}{l}
|\input{childdoc.def}|\\
|\childdocforward[|\textit{main}|]{|\textit{dest}|}|\\
\end{tabular}
\end{center}
%
The argument \textit{dest} is the destination file
(without extension).
It should be the main file or one of the child files.
Note that further \textsf{childdoc} directives
such as |\childdocof| and |\childdocforward|
in the indicated file will be processed in this form.
The optional argument \textit{main}
passes on directly to the main file \textit{main}
while pretending to compile the child \textit{dest}.
This form behaves as if \textit{dest}
issues |\childdocof{|\textit{main}|}| right away,
and no further \textsf{childdoc} directives will be processed.

%%%%%%%%%%%%%%%%%%%%%%%%%%%%%%%%%%%%%%%%
\DescribeMacro{\...prefix}
In the alternative form |\childdocforwardprefix|,
%
\begin{center}
\begin{tabular}{l}
|\input{childdoc.def}|\\
|\childdocforwardprefix[|\textit{main}|]{|\textit{prefix}|}{|\textit{dest}|}|
\end{tabular}
\end{center}
%
the destination file is determined by a pattern
depending on the current file:
To make this work, the current file must be called
`{\textit{prefix}\hspace{0.2em}\textit{suffix}}'
with \textit{prefix} matching precisely the argument.
Processing is then passed on to the file
`{\textit{dest}\hspace{0.2em}\textit{suffix}}'.
Surely, the same effect is achieved by
directly specifying the
argument `{\textit{dest}\hspace{0.2em}\textit{suffix}}'
in the first form.
However, that requires to set up a different file
for each child. With the alternative form of the command
all these files can have exactly the same content
which simplifies setting them up and maintaining them.

For example, the following file |draft.tex|
with a compilation flag |\version| as described in \secref{sec:flags}
compiles the main document as a draft:
%
\begin{center}
\begin{tabular}{l}
|\def\version{draft}|\\
|\input{childdoc.def}|\\
|\childdocforward{|\textit{main}|}|
\end{tabular}
\end{center}
%
Likewise, the following files |final|\textit{nn}|.tex|
compile the final version of the child document
|child|\textit{nn}|.tex|:
%
\begin{center}
\begin{tabular}{l}
|\def\version{final}|\\
|\input{childdoc.def}|\\
|\childdocforwardprefix{final}{child}|
\end{tabular}
\end{center}
%

Note that when several versions of a main file and/or of each child file
are to be generated, it may be convenient to set up a |Makefile| or
shell script to automatise the process.

%%%%%%%%%%%%%%%%%%%%%%%%%%%%%%%%%%%%%%%%%%%%%%%%%%%%%%%%%%%%%%%%%%%%%%%%%%%%%%%%
\subsection{Command Line Processing}
\label{sec:commandline}

The effect of redirection files can also be achieved by invoking
the \LaTeX{} compiler with a more elaborate command line.
Most conveniently this should be done as part
of a shell script or a |Makefile|.

When using \textsf{childdoc} in the main file, the following
command lines effectively perform a redirection
(note that depending on the shell being used,
backslashes may have to be doubled: `|\|' $\to$ `|\\|'):
%
\begin{center}
|... -jobname "|\textit{target}|" |\\|"|[\textit{flags}]%
|\input{childdoc.def}\childdocforward[|\textit{main}|]{|\textit{dest}|}"|
\end{center}
%
Here \textit{target} is the name of the output file,
\textit{main} is the name of the main file
and \textit{dest} is the name of the main or child file to be processed
(all filenames without extensions).
The optional argument \textit{main} can be omitted
if \textit{main} matches \textit{dest}.
Optionally, compilation \textit{flags} can be defined via |\def| commands.
This command line makes the \TeX{} engine believe
it is compiling the file \textit{target}
whose content is specified as the latter parameter.
The provided code then forwards the processing to
\textit{main} or \textit{dest} as described in \secref{sec:forward}.

%%%%%%%%%%%%%%%%%%%%%%%%%%%%%%%%%%%%%%%%%%%%%%%%%%%%%%%%%%%%%%%%%%%%%%%%%%%%%%%%
\subsection{Include by Input}
\label{sec:input}

Including child documents by |\include| has some restrictions by design.
Most notably, the content of a child document always occupies
its own set of pages; pages cannot be shared between child documents.
Usually, this behaviour makes perfect sense
because each child document contain an essential part of the document.
However, in some situations it may be desirable to compose
a document from a collection of parts
without having mandatory page breaks between then.
For this case, the package
provides a mechanism to include parts
by |\input| which can also be processed individually.
However, by construction this mechanism
requires manual handling of the content to be output.

%%%%%%%%%%%%%%%%%%%%%%%%%%%%%%%%%%%%%%%%
\DescribeMacro{\ifchilddocmanual}
The main file should be prepared as usual, see \secref{sec:include}.
However, the document body must make a distinction
between processing of an individual part and of the main document, e.g.:
%
\begin{center}
\begin{tabular}{l}
|\ifchilddocmanual|\\
|\input{\childdocname}|\\
|\||else|\\
\textit{document body with }|\input{|\textit{part}|}|\\
|\||fi|
\end{tabular}
\end{center}
%
The conditional |\ifchilddocmanual| is true whenever
a part to be included by |\input| is being compiled,
and the name of the part is stored in |\childdocname|.

%%%%%%%%%%%%%%%%%%%%%%%%%%%%%%%%%%%%%%%%
\DescribeMacro{\childdocby}
Each part to be included by |\input| should start with:
%
\begin{center}
\begin{tabular}{l}
|\input{childdoc.def}|\\
|\childdocby{|\textit{main}|}|\\
\end{tabular}
\end{center}
%
The directive |\childdocby| is similar to |\childdocof|
described in \secref{sec:include},
but the subsequent selection of content must be done manually.
To that end, both |\ifchilddoc| and |\ifchilddocmanual|
will be true upon processing of a part,
and the name of the part is stored in |\childdocname|.
Note that |\jobname| will be set to the filename of the current part
so that each part receives an individual |.aux| file
that does not interfere with the |.aux| file(s) of the main document.
This behaviour can be altered by the alternative form
|\childdocby[*]{|\textit{main}|}| (with a non-empty optional argument)
which uses the |.aux| file of the main document
by setting |\jobname| to \textit{main}.

%%%%%%%%%%%%%%%%%%%%%%%%%%%%%%%%%%%%%%%%%%%%%%%%%%%%%%%%%%%%%%%%%%%%%%%%%%%%%%%%
\subsection{Driver Development}
\label{sec:driver}

The \textsf{childdoc} mechanism can also be use for the development
of definition files such as \LaTeX{} styles or classes.
This case differs from the above setup with multiple parts
included by |\include| in that no |\includeonly| should be invoked.
This can be achieved by starting the include file
(before |\ProvidesPackage|) with:
%
\begin{center}
\begin{tabular}{l}
|\input{childdoc.def}|\\
|\childdocforward{|\textit{main}|}|\\
\end{tabular}
\end{center}
%
or alternatively with:
%
\begin{center}
\begin{tabular}{l}
|\input{childdoc.def}|\\
|\childdocby{|\textit{main}|}|\\
\end{tabular}
\end{center}
%
Both forms have slightly different effects as described above.
The main file is prepared as usual, see \secref{sec:include}.

%%%%%%%%%%%%%%%%%%%%%%%%%%%%%%%%%%%%%%%%%%%%%%%%%%%%%%%%%%%%%%%%%%%%%%%%%%%%%%%%
\subsection{Legacy Detection}
\label{sec:detection}

The directive |\childdocmain| in the main file can detect
whether the complete document or merely a child is to be compiled
even without using the directive |\childdocof|.
This method is deprecated because it is less robust
and there is no compelling reason to use it;
it is merely provided for backward compatibility
and it may be removed in future versions.

If the detection mechanism is to be used,
it is mandatory to correctly specify
the filename of the main file as the argument of |\childdocmain|:
%
\begin{center}
\begin{tabular}{l}
|\input{childdoc.def}|\\
|\childdocmain{|\textit{main}|}|\\
\end{tabular}
\end{center}
%
If |\jobname| does not match the argument \textit{main} of |\childdocmain|,
it is assumed that |\jobname| points to the child file to be compiled.
When using |\childdocmain| with the main file specified as argument,
it suffices to start a child file
with just |\input{|\textit{main}|}|
without loading of the package and using |\childdocof|.
If instead all processing is done
with the appropriate \textsf{childdoc} directives,
the argument of \textit{main} of |\childdocmain| can be empty.

An alternative version of the command line processing described
in \secref{sec:commandline} using the detection mechanism reads:
%
\begin{center}
|... -jobname "|\textit{target}|" "|[\textit{flags}]%
[|\def\jobname{|\textit{dest}|}|]|\input{|\textit{main}|}"|
\end{center}

%%%%%%%%%%%%%%%%%%%%%%%%%%%%%%%%%%%%%%%%%%%%%%%%%%%%%%%%%%%%%%%%%%%%%%%%%%%%%%%%
\subsection{Manual Code}
\label{sec:manual}

In case one cannot be certain whether the definitions file |childdoc.def|
is installed on the target \TeX{} distribution
and one prefers not to ship it,
it is conceivable to paste a few relevant commands into the sources.

To that end, drop all statements |\input{childdoc.def}|
and perform the replacements as outlined below.
Instead of |\childdocmain{|\textit{main}|}| add the following code
to the top of the main file:
%
\begin{center}
\begin{tabular}{l}
|\||ifdefined\childdocname\endinput\||fi\newif\ifchilddoc|\\
|\edef\childdocname{\scantokens\expandafter{\jobname\noexpand}}|\\
|\def\childdocmain{|\textit{main}|}\||ifx\childdocmain\childdocname\||else|\\
|\childdoctrue\includeonly{\childdocname}\let\jobname\childdocmain\||fi|\\
\end{tabular}
\end{center}
%
Instead of |\childdocof{|\textit{main}|}| just include the main file
at the top of each child file:
%
\begin{center}
|\input{|\textit{main}|}|
\end{center}
%
A simple redirection |\childdocforward{|\textit{dest}|}| is achieved by:
%
\begin{center}
|\def\jobname{|\textit{dest}|}\input{\jobname}|
\end{center}
%
The redirection with prefix
|\childdocforwardprefix[|\textit{prefix}|]{|\textit{dest}|}|
is accomplished by:
%
\begin{center}
\begin{tabular}{l}
|{\edef\jobname{\scantokens\expandafter{\jobname\noexpand}}|\\
|\def\redirectjob |\textit{prefix}|#1~~~{\gdef\jobname{|\textit{dest}|#1}}|\\
|\expandafter\redirectjob\jobname~~~}\input{\jobname}|
\end{tabular}
\end{center}

In an alternative approach,
child documents can be compiled by a specific command line
without additional code or specific definitions:
%
\begin{center}
|... -jobname "|\textit{target}|" "|[\textit{flags}]%
|\includeonly{|\textit{dest}|}\input{|\textit{main}|}"|
\end{center}
%

%%%%%%%%%%%%%%%%%%%%%%%%%%%%%%%%%%%%%%%%%%%%%%%%%%%%%%%%%%%%%%%%%%%%%%%%%%%%%%%%
%%%%%%%%%%%%%%%%%%%%%%%%%%%%%%%%%%%%%%%%%%%%%%%%%%%%%%%%%%%%%%%%%%%%%%%%%%%%%%%%
\section{Information}

%%%%%%%%%%%%%%%%%%%%%%%%%%%%%%%%%%%%%%%%%%%%%%%%%%%%%%%%%%%%%%%%%%%%%%%%%%%%%%%%
\subsection{Copyright}

Copyright \copyright{} 2017--2018 Niklas Beisert

This work may be distributed and/or modified under the
conditions of the \LaTeX{} Project Public License, either version 1.3
of this license or (at your option) any later version.
The latest version of this license is in
  \url{http://www.latex-project.org/lppl.txt}
and version 1.3 or later is part of all distributions of \LaTeX{}
version 2005/12/01 or later.

This work has the LPPL maintenance status `maintained'.

The Current Maintainer of this work is Niklas Beisert.

This work consists of the files |README.txt|, |childdoc.ins| and |childdoc.dtx|
as well as the derived files |childdoc.def|, |cdocsamp.tex|
with |cdocsch1.tex|, |cdocsch2.tex|, |cdocspt3.tex|, |cdocspt4.tex|,
|cdocsdrf.tex|, |cdocsfn1.tex|, |cdocsfn2.tex|
as well as |childdoc.pdf|.

%%%%%%%%%%%%%%%%%%%%%%%%%%%%%%%%%%%%%%%%%%%%%%%%%%%%%%%%%%%%%%%%%%%%%%%%%%%%%%%%
\subsection{Files and Installation}

The package consists of the files:
%
\begin{center}
\begin{tabular}{ll}
    |README.txt|   & readme file \\
    |childdoc.ins| & installation file \\
    |childdoc.dtx| & source file \\
    |childdoc.def| & definition file \\
    |cdocsamp.tex| & sample main file \\
    |cdocsch1.tex| & sample include file \\
    |cdocsch2.tex| & sample include file \\
    |cdocspt3.tex| & sample part file \\
    |cdocspt4.tex| & sample part file \\
    |cdocsdrf.tex| & sample redirection file \\
    |cdocsfn1.tex| & sample redirection file \\
    |cdocsfn2.tex| & sample redirection file \\
    |childdoc.pdf| & manual
\end{tabular}
\end{center}
%
The distribution consists of the files
|README.txt|, |childdoc.ins| and |childdoc.dtx|.
%
\begin{itemize}
\item
Run (pdf)\LaTeX{} on |childdoc.dtx|
to compile the manual |childdoc.pdf| (this file).
\item
Run \LaTeX{} on |childdoc.ins| to create the definitions file |childdoc.def|
and the sample |cdocsamp.tex| with include files
|cdocsch1.tex|, |cdocsch2.tex|, |cdocspt3.tex|, |cdocspt4.tex|,
|cdocsdrf.tex|, |cdocsfn1.tex|, |cdocsfn2.tex|.
Then copy the file |childdoc.def| to an appropriate directory of your \LaTeX{}
distribution, e.g.\ \textit{texmf-root}|/tex/latex/childdoc|.
\end{itemize}

%%%%%%%%%%%%%%%%%%%%%%%%%%%%%%%%%%%%%%%%%%%%%%%%%%%%%%%%%%%%%%%%%%%%%%%%%%%%%%%%
\subsection{Related CTAN Packages}

There are several other packages which offer a similar functionality:
%
\begin{itemize}
\item
The packages
\href{http://ctan.org/pkg/docmute}{\textsf{docmute}},
\href{http://ctan.org/pkg/includex}{\textsf{includex}} and
\href{http://ctan.org/pkg/standalone}{\textsf{standalone}}
provide commands to include only the document body of
a child file thus allowing both files to be compiled individually.
\item
The packages \href{http://ctan.org/pkg/subdocs}{\textsf{subdocs}}
and \href{http://ctan.org/pkg/subfiles}{\textsf{subfiles}}
provide structures in which the main and child documents can be
encapsulated and allowing them to be compiled individually.
The inclusion mechanism is different from the conventional |\include|.
\item
The package \href{http://ctan.org/pkg/combine}{\textsf{combine}}
is an elaborate solution to combine several documents into one.
\end{itemize}
%
See also the CTAN topic \href{http://ctan.org/topic/subdocs}{\textsf{subdocs}}
for further related packages.
The present package differs from the above solutions in that
a document structure constructed with the conventional |\include| mechanism
just needs two extra commands at the top of every file
such that all constituent files can be compiled individually.

%%%%%%%%%%%%%%%%%%%%%%%%%%%%%%%%%%%%%%%%%%%%%%%%%%%%%%%%%%%%%%%%%%%%%%%%%%%%%%%%
%\subsection{Feature Suggestions}
%
%The following is a list of features which may be useful for future
%versions of this package:
%%
%\begin{itemize}
%\item
%\ldots
%\end{itemize}

%%%%%%%%%%%%%%%%%%%%%%%%%%%%%%%%%%%%%%%%%%%%%%%%%%%%%%%%%%%%%%%%%%%%%%%%%%%%%%%%
\subsection{Revision History}

%%%%%%%%%%%%%%%%%%%%%%%%%%%%%%%%%%%%%%%%
\paragraph{v2.0:} 2018/12/30

\begin{itemize}
\item
immediate forward processing
\item
added |\childdocby| mechanism
\item
manual restructured
\end{itemize}

%%%%%%%%%%%%%%%%%%%%%%%%%%%%%%%%%%%%%%%%
\paragraph{v1.6:} 2018/01/17

\begin{itemize}
\item
application for development of include files
\item
corrections to manual
\end{itemize}

%%%%%%%%%%%%%%%%%%%%%%%%%%%%%%%%%%%%%%%%
\paragraph{v1.5:} 2017/05/21

\begin{itemize}
\item
more complete structuring introduced
\item
|\childdocof| introduced
\item
|\childdoc| renamed to |\childdocmain|
\item
|\childredirect| renamed to |\childdocforward| and |\childdocforwardprefix|
and functionality expanded
\end{itemize}

%%%%%%%%%%%%%%%%%%%%%%%%%%%%%%%%%%%%%%%%
\paragraph{v1.0:} 2017/04/27

\begin{itemize}
\item
manual and install package
\item
first version published on CTAN
\end{itemize}

%%%%%%%%%%%%%%%%%%%%%%%%%%%%%%%%%%%%%%%%
\paragraph{v0.6:} 2017/04/26

\begin{itemize}
\item
redirection mechanism added
\end{itemize}

%%%%%%%%%%%%%%%%%%%%%%%%%%%%%%%%%%%%%%%%
\paragraph{v0.5:} 2017/04/26

\begin{itemize}
\item
functionality in definition file
\end{itemize}


%%%%%%%%%%%%%%%%%%%%%%%%%%%%%%%%%%%%%%%%%%%%%%%%%%%%%%%%%%%%%%%%%%%%%%%%%%%%%%%%
%%%%%%%%%%%%%%%%%%%%%%%%%%%%%%%%%%%%%%%%%%%%%%%%%%%%%%%%%%%%%%%%%%%%%%%%%%%%%%%%
%%%%%%%%%%%%%%%%%%%%%%%%%%%%%%%%%%%%%%%%%%%%%%%%%%%%%%%%%%%%%%%%%%%%%%%%%%%%%%%%
\appendix

\settowidth\MacroIndent{\rmfamily\scriptsize 000\ }

 \DocInput{childdoc.dtx}

\end{document}
%</driver>
% \fi
%
% %%%%%%%%%%%%%%%%%%%%%%%%%%%%%%%%%%%%%%%%%%%%%%%%%%%%%%%%%%%%%%%%%%%%%%%%%%%%%%
% %%%%%%%%%%%%%%%%%%%%%%%%%%%%%%%%%%%%%%%%%%%%%%%%%%%%%%%%%%%%%%%%%%%%%%%%%%%%%%
% \section{Sample}
%\iffalse
%<*samplemain>
%\fi
%
% The following presents a sample document
% with two chapters, two parts, a title page,
% a compile flag as well as three forwarding files to set the flag.
% It consists of eight |.tex| files:
% \begin{center}
% \begin{tabular}{ll}
% |cdocsamp.tex|&main file\\
% |cdocsch1.tex|&include file for chapter 1\\
% |cdocsch2.tex|&include file for chapter 2\\
% |cdocspt3.tex|&include file for part 3\\
% |cdocspt4.tex|&include file for part 4\\
% |cdocsdrf.tex|&forwarding file for main file in draft mode\\
% |cdocsfi1.tex|&forwarding file for final version of chapter 1\\
% |cdocsfi2.tex|&forwarding file for final version of chapter 2\\
% \end{tabular}
% \end{center}
% Each of the eight files can be compiled directly by the \LaTeX{} compiler.
%
% %%%%%%%%%%%%%%%%%%%%%%%%%%%%%%%%%%%%%%
% \paragraph{Main File.}
%
% The main file is called |cdocsamp.tex|.
%
% Load the \textsf{childdoc} definitions and
% declare the filename for the main document:
%    \begin{macrocode}
\input{childdoc.def}
\childdocmain{}
%    \end{macrocode}

% Optional override for |\version| flag:
%    \begin{macrocode}
%%\ifchilddoc\else\providecommand{\version}{draft}\fi
%    \end{macrocode}

% Define the default values for the |\version| flag
% (|final| for the main file and |draft| for childs):
%    \begin{macrocode}
\ifchilddoc
\providecommand{\version}{draft}
\else
\providecommand{\version}{final}
\fi
%    \end{macrocode}

% Load the standard document class:
%    \begin{macrocode}
\documentclass[12pt]{article}
%    \end{macrocode}

% Start the document body:
%    \begin{macrocode}
\begin{document}
%    \end{macrocode}

% Declare a title page.
% Print title, part of document being processed and version flag:
%    \begin{macrocode}
\addtocounter{page}{-1}
\begin{center}
{\LARGE\bfseries{}childdoc example\par}
\vspace{1cm}
\ifchilddoc
\ifchilddocmanual part\else chapter\fi:
`\childdocname' of `\childdocjob'\par
\else
main document: `\childdocjob'\par
\fi
version: \version\par
\end{center}
\newpage
%    \end{macrocode}

% Manually include selected file,
% otherwise process as usual:
%    \begin{macrocode}
\ifchilddocmanual
\section*{part `\childdocname'}
\input{\childdocname}
\else
%    \end{macrocode}

% Include the two chapters:
%    \begin{macrocode}
\include{cdocsch1}
\include{cdocsch2}
%    \end{macrocode}

% Include the two parts unless only chapters should be displayed:
%    \begin{macrocode}
\ifchilddoc\else
\section{part three}
\input{cdocspt3}
\section{part four}
\input{cdocspt4}
\fi
%    \end{macrocode}

% Process as usual until here:
%    \begin{macrocode}
\fi
%    \end{macrocode}

% End of document body:
%    \begin{macrocode}
\end{document}
%    \end{macrocode}
%\iffalse
%</samplemain>
%\fi
%
% %%%%%%%%%%%%%%%%%%%%%%%%%%%%%%%%%%%%%%
% \paragraph{Chapter Include Files.}
%
% The include files are called |cdocsch1.tex| and |cdocsch2.tex|.
%
%\iffalse
%<*samplechap1|samplechap2>
%\fi

% Optional override for |\version| flag:
%    \begin{macrocode}
%%\providecommand{\version}{final}
%    \end{macrocode}

% Include the main document:
%    \begin{macrocode}
\input{childdoc.def}
\childdocof{cdocsamp}
%    \end{macrocode}

%\iffalse
%</samplechap1|samplechap2>
%\fi
%
%\iffalse
%<*samplechap1>
%\fi
% Some text for chapter 1:
%    \begin{macrocode}
\section{one}
some text in chapter one
%    \end{macrocode}

%\iffalse
%</samplechap1>
%\fi
% Some text for chapter 2:
%\iffalse
%<*samplechap2>
%\fi
%    \begin{macrocode}
\section{two}
more text in chapter two
%    \end{macrocode}

%\iffalse
%</samplechap2>
%\fi
%
% %%%%%%%%%%%%%%%%%%%%%%%%%%%%%%%%%%%%%%
% \paragraph{Part Include Files.}
%
% The include files are called |cdocspt3.tex| and |cdocspt4.tex|.
%
%\iffalse
%<*samplepart3|samplepart4>
%\fi

% Optional override for |\version| flag:
%    \begin{macrocode}
%%\providecommand{\version}{final}
%    \end{macrocode}

% Include the main document:
%    \begin{macrocode}
\input{childdoc.def}
\childdocby{cdocsamp}
%    \end{macrocode}

%\iffalse
%</samplepart3|samplepart4>
%\fi
%
%\iffalse
%<*samplepart3>
%\fi
% Some text for part 3:
%    \begin{macrocode}
some text in part three
%    \end{macrocode}

%\iffalse
%</samplepart3>
%\fi
% Some text for part 4:
%\iffalse
%<*samplepart4>
%\fi
%    \begin{macrocode}
more text in part four
%    \end{macrocode}

%\iffalse
%</samplepart4>
%\fi
%
% %%%%%%%%%%%%%%%%%%%%%%%%%%%%%%%%%%%%%%
% \paragraph{Forwarding for a Complete Draft.}
%
% The following forwarding file |cdocsdrf.tex|
% compiles the main document in draft mode:
%\iffalse
%<*sampledraft>
%\fi
%    \begin{macrocode}
\def\version{draft}
\input{childdoc.def}
\childdocforward{cdocsamp}
%    \end{macrocode}

%\iffalse
%</sampledraft>
%\fi
%
% %%%%%%%%%%%%%%%%%%%%%%%%%%%%%%%%%%%%%%
% \paragraph{Forwarding for Final Version of the Chapters.}
%
% The following forwarding files |cdocsfn1.tex| and |cdocsfn2.tex|
% (with identical content)
% compile the final versions of the child documents
% |cdocsch1.tex| and |cdocsch2.tex|, respectively:
%\iffalse
%<*samplefinal>
%\fi
%    \begin{macrocode}
\def\version{final}
\input{childdoc.def}
\childdocforwardprefix[cdocsamp]{cdocsfn}{cdocsch}
%    \end{macrocode}

%\iffalse
%</samplefinal>
%\fi
%
% %%%%%%%%%%%%%%%%%%%%%%%%%%%%%%%%%%%%%%
% \paragraph{Command Line Processing.}
%
% The following three command lines generate the output files
% |cdocscld|, |cdocscl1| and |cdocscl2|
% which should be identical to
% |cdocsdrf|, |cdocsch1| and |cdocsfn2|, respectively:
% \begin{center}
% \begin{tabular}{l}
% |latex -jobname cdocscld \|\\
% |  "\def\version{draft}\input{childdoc.def}\childdocforward{cdocsamp}"|\\
% |latex -jobname cdocscl1 \|\\
% |  "\input{childdoc.def}\childdocforward[cdocsamp]{cdocsch1}"|\\
% |latex -jobname cdocscl2 \|\\
% |  "\def\version{final}\input{childdoc.def}\childdocforward{cdocsch2}"|
% \end{tabular}
% \end{center}
% Note that the trailing backslash on each first line
% merely continues the input to the second line
% (for convenient cut ant paste).
% Furthermore, the command |latex| can be replaced by any
% of its alternative versions such as |pdflatex|.
%
% %%%%%%%%%%%%%%%%%%%%%%%%%%%%%%%%%%%%%%%%%%%%%%%%%%%%%%%%%%%%%%%%%%%%%%%%%%%%%%
% %%%%%%%%%%%%%%%%%%%%%%%%%%%%%%%%%%%%%%%%%%%%%%%%%%%%%%%%%%%%%%%%%%%%%%%%%%%%%%
% \section{Implementation}
%\iffalse
%<*package>
%\fi
%
% This section describes the definitions file |childdoc.def|.

% The definitions cannot be loaded using |\usepackage| or |\RequirePackage|
% which has a mechanism to prevent loading a style file more than once.
% When loading the definitions by means of |\input|
% multiple instances have to be prevented manually:
%\iffalse
%This code needs to be before the `\ProvidesFile' directive
%which is defined at the beginning of this file.
%Therefore it is also placed there and commented out here.
%</package>
%<*discard>
%\fi
%    \begin{macrocode}
\ifdefined\childdocmain\endinput\fi
%    \end{macrocode}
%\iffalse
%</discard>
%<*package>
%\fi
%
% \macro{\ifchilddoc}
% \macro{\ifchilddocmanual}
% The conditional |\ifchilddoc| tells whether a
% child (true) or main (false) document is being compiled.
% The conditional |\ifchilddocmanual| tells whether
% the |\includeonly| mechanism is used (false) or
% the selection of child files must be performed manually (true).
% The definitions initialise to false:
%    \begin{macrocode}
\newif\ifchilddoc
\newif\ifchilddocmanual
%    \end{macrocode}

% \macro{\childdocname}
% \macro{\childdocjob}
% The macro |\childdocname| stores the name of the main document
% to be compiled. The macro |\childdocjob| stores the name of
% the document on which the \LaTeX{} compiler was originally invoked.
% The content of |\jobname| cannot be compared
% to filenames specified in the source due to different catcodes.
% The following code rescans |\jobname|, stores the result
% in |\childdocname| and saves a copy in |\childdocjob|:
%    \begin{macrocode}
\edef\childdocname{\scantokens\expandafter{\jobname\noexpand}}
\let\childdocjob\childdocname
%    \end{macrocode}

% \macro{\childdocdisable}
% The macro |\childdocdisable| prevents the main file
% from being processed more than once.
% At this stage, the main document command |\childdocmain|
% is assumed to be called once again where it should do nothing.
% Any subsequent call to it should prevent
% a secondary processing of the main document
% It overwrites the forwarding commands
% |\childdocof| and |\childdocforward|
% with empty macros to prevent further inclusions of the main document:
%    \begin{macrocode}
\newcommand{\childdocdisable}
{
  \renewcommand{\childdocmain}[1]{\renewcommand{\childdocmain}[1]{\endinput}}
  \renewcommand{\childdocof}[1]{}
  \renewcommand{\childdocby}[2][]{}
  \renewcommand{\childdocforward}[2][]{}
  \renewcommand{\childdocdisable}{}
}
%    \end{macrocode}

% \macro{\childdocmain}
% The macro |\childdocmain| is to be called at the top of the main file
% with nothing or the main filename (without extension) as argument.
% First, it breaks loops.
% If the argument is not empty and does not match |\childdocname|
% (which is set by the first inclusion of |childdoc.def|),
% |\ifchilddoc| is set to true, |\includeonly| is applied to the child file
% and |\jobname| is set to the main file
% (for proper handling of |.aux| files):
%    \begin{macrocode}
\newcommand{\childdocmain}[1]
{
  \childdocdisable\childdocmain{}
  \if?#1?\else
    \begingroup
      \def\childdoctmp{#1}
      \ifx\childdoctmp\childdocname
        \def\childdoctmp{}
      \else
        \def\childdoctmp
        {
          \childdoctrue
          \includeonly{\childdocname}
          \def\childdocjob{#1}
          \def\jobname{#1}
        }
      \fi
      \expandafter
    \endgroup
    \childdoctmp
  \fi
}
%    \end{macrocode}

% \macro{\childdocof}
% The command |\childdocof| redirects
% compilation to the main file |#1|.
%    \begin{macrocode}
\newcommand{\childdocof}[1]
{
  \childdocdisable
  \childdoctrue
  \includeonly{\childdocname}
  \def\jobname{#1}
  \def\childdocjob{#1}
  \input{#1}
}
%    \end{macrocode}

% \macro{\childdocby}
% The command |\childdocby| ....
%    \begin{macrocode}
\newcommand{\childdocby}[2][]
{
  \childdocdisable
  \childdoctrue
  \childdocmanualtrue
  \if?#1?\else
    \def\jobname{#2}
  \fi
  \def\childdocjob{#2}
  \input{#2}
  \endinput
}
%    \end{macrocode}

% \macro{\childdocforward}
% The command |\childdocforward| redirects
% compilation to the main file or
% (if the optional argument is given) a child file.
% Parameters are set as if the main file
% or a child file starting with |\childdocof| was compiled.
% Then compilation is handed over to the main file:
%    \begin{macrocode}
\newcommand{\childdocforward}[2][]
{
  \begingroup
    \if?#1?
      \def\childdoctmp
      {
        \def\childdocname{#2}
        \def\childdocjob{#2}
        \def\jobname{#2}
        \input{#2}
        \endinput
      }
    \else
      \def\childdoctmp
      {
        \childdocdisable
        \def\childdocname{#2}
        \childdoctrue
        \includeonly{#2}
        \def\childdocjob{#1}
        \def\jobname{#1}
        \input{#1}
        \endinput
      }
    \fi
    \expandafter
  \endgroup
  \childdoctmp
}
%    \end{macrocode}

% \macro{\childdocforwardprefix}
% The command |\childdocforwardprefix| redirects
% compilation to the main or a child file by means of a pattern.
% The prefix |#1| in the current filename is replaced by |#2|
% and the suffix of the current filename is kept
% (it is assumed that the filename does not contain the substring `|~~~|'
% which is used as a delimiter).
% Compilation is handed over to the new file by |\childdocforward|:
%    \begin{macrocode}
\newcommand{\childdocforwardprefix}[3][]
{
  \begingroup
    \def\childdocextract #2##1~~~{\def\childdoctmp{\childdocforward[#1]{#3##1}}}
    \expandafter\childdocextract\childdocname~~~
    \expandafter
  \endgroup
  \childdoctmp
}
%    \end{macrocode}

% \macro{\childdoc}
% The deprecated macro |\childdoc| is a legacy version of |\childdocmain|:
%    \begin{macrocode}
\newcommand{\childdoc}{\childdocmain}
%    \end{macrocode}

% \macro{\childdocredirect}
% The deprecated macro |\childdocredirect| is a legacy version
% of |\childdocforward| and |\childdocforwardprefix|:
%    \begin{macrocode}
\newcommand{\childdocredirect}[2][]
{
  \begingroup
    \if?#1?
      \def\childdoctmp{\childdocforward{#2}}
    \else
      \def\childdoctmp{\childdocforwardprefix{#1}{#2}}
    \fi
    \expandafter
  \endgroup
  \childdoctmp
}
%    \end{macrocode}

%\iffalse
%</package>
%\fi
%
\endinput

\childdocforward{cdocsamp}
%    \end{macrocode}

%\iffalse
%</sampledraft>
%\fi
%
% %%%%%%%%%%%%%%%%%%%%%%%%%%%%%%%%%%%%%%
% \paragraph{Forwarding for Final Version of the Chapters.}
%
% The following forwarding files |cdocsfn1.tex| and |cdocsfn2.tex|
% (with identical content)
% compile the final versions of the child documents
% |cdocsch1.tex| and |cdocsch2.tex|, respectively:
%\iffalse
%<*samplefinal>
%\fi
%    \begin{macrocode}
\def\version{final}
% \iffalse
%
% childdoc.dtx Copyright (C) 2017-2018 Niklas Beisert
%
% This work may be distributed and/or modified under the
% conditions of the LaTeX Project Public License, either version 1.3
% of this license or (at your option) any later version.
% The latest version of this license is in
%   http://www.latex-project.org/lppl.txt
% and version 1.3 or later is part of all distributions of LaTeX
% version 2005/12/01 or later.
%
% This work has the LPPL maintenance status `maintained'.
%
% The Current Maintainer of this work is Niklas Beisert.
%
% This work consists of the files childdoc.dtx and childdoc.ins
% and the derived files childdoc.def and cdocsamp.tex with
% cdocsch1.tex, cdocsch2.tex, cdocsdrf.tex, cdocsfn1.tex, cdocsfn2.tex.
%
%<package>\ifdefined\childdocmain\endinput\fi
%<package>\ProvidesFile{childdoc.def}[2018/12/30 v2.0 child document driver]
%<samplemain>\ProvidesFile{cdocsamp.tex}[2018/12/30 v2.0 sample for childdoc]
%<*driver>
%\ProvidesFile{childdoc.drv}[2018/12/30 v2.0 childdoc reference manual file]
\PassOptionsToClass{10pt,a4paper}{article}
\documentclass{ltxdoc}

\usepackage[margin=35mm]{geometry}
\usepackage{hyperref}
\usepackage{hyperxmp}
\usepackage[usenames]{color}

\hypersetup{colorlinks=true}
\hypersetup{pdfstartview=FitH}
\hypersetup{pdfpagemode=UseNone}
\hypersetup{pdfsource={}}
\hypersetup{pdflang={en-UK}}
\hypersetup{pdfcopyright={Copyright 2017-2018 Niklas Beisert.
  This work may be distributed and/or modified under the
  conditions of the LaTeX Project Public License, either version 1.3
  of this license or (at your option) any later version.}}
\hypersetup{pdflicenseurl={http://www.latex-project.org/lppl.txt}}
\hypersetup{pdfcontactaddress={ETH Zurich, ITP, HIT K,
  Wolfgang-Pauli-Strasse 27}}
\hypersetup{pdfcontactpostcode={8093}}
\hypersetup{pdfcontactcity={Zurich}}
\hypersetup{pdfcontactcountry={Switzerland}}
\hypersetup{pdfcontactemail={nbeisert@itp.phys.ethz.ch}}
\hypersetup{pdfcontacturl={http://people.phys.ethz.ch/\xmptilde nbeisert/}}

\newcommand{\secref}[1]{\hyperref[#1]{section \ref*{#1}}}

\parskip1ex
\parindent0pt
\let\olditemize\itemize
\def\itemize{\olditemize\parskip0pt}

\begin{document}

\title{The \textsf{childdoc} Package}
\hypersetup{pdftitle={The childdoc Package}}
\author{Niklas Beisert\\[2ex]
  Institut f\"ur Theoretische Physik\\
  Eidgen\"ossische Technische Hochschule Z\"urich\\
  Wolfgang-Pauli-Strasse 27, 8093 Z\"urich, Switzerland\\[1ex]
  \href{mailto:nbeisert@itp.phys.ethz.ch}
  {\texttt{nbeisert@itp.phys.ethz.ch}}}
\hypersetup{pdfauthor={Niklas Beisert}}
\hypersetup{pdfsubject={Manual for the LaTeX2e Package childdoc}}
\date{30 December 2018, \textsf{v2.0}}
\maketitle

\begin{abstract}\noindent
\textsf{childdoc} is a \LaTeXe{} package
that enables the direct compilation
of document sections included by |\include|
to individual files.
\end{abstract}

\begingroup
\parskip0ex
\tableofcontents
\endgroup

%%%%%%%%%%%%%%%%%%%%%%%%%%%%%%%%%%%%%%%%%%%%%%%%%%%%%%%%%%%%%%%%%%%%%%%%%%%%%%%%
%%%%%%%%%%%%%%%%%%%%%%%%%%%%%%%%%%%%%%%%%%%%%%%%%%%%%%%%%%%%%%%%%%%%%%%%%%%%%%%%
\section{Introduction}

\LaTeX{} provides a mechanism to structure a large document (such as a book)
into a main file and several child files (containing the chapters)
using the |\include| command.
This mechanism is beneficial for documents
which span hundreds of pages in order to
make the source file(s) more manageable.
Moreover, compilation can be restricted to
selected child files by means of the |\includeonly| command.
The latter feature can be used to reduce the compilation time while editing
(this was significantly more useful in the earlier days of \LaTeX{})
or to generate a smaller document which is easier to navigate.
Another application of |\includeonly| is to generate
documents consisting of selected parts of the complete document.

However, there are a few drawbacks of the plain |\include| mechanism:
\begin{itemize}
\item
The child files cannot be compiled on their own,
they can only be compiled via the main file.
A naive editing environment
(such as a text editor with an option
to have the current file processed by \LaTeX)
may require one to switch to the main file before compiling;
attempting to compile the child file produces errors.
\item
The main file must be modified (each time)
to adjust the |\includeonly| command
to the present needs. This easily leaves the main file in a messy state.
\item
The generated document will always carry the filename
of the main document. This is inconvenient if
several child files are to be compiled and
to be kept for distribution.
\end{itemize}

The present package provides a simple interface
to make child files individually compilable by \LaTeX{}.
Compiling a child file then has the same effect as compiling
the main file with an |\includeonly| command
to select the appropriate child.
Moreover the generated document will carry the name of the child
rather than the main file.
This resolves all three above issues.

This feature is meant to make the editing of books,
thesis documents and lecture notes somewhat more convenient.
However, the package can also be used efficiently for
composing a series of documents (such as exercise sheets)
which are typically distributed individually.
It then assists the author in generating the individual documents
(potentially in different versions)
as well as a document containing the collected series.
Another application is in developing style files
or other kinds of included material
where compilation of the style file could redirect
to a sample or test file.

%%%%%%%%%%%%%%%%%%%%%%%%%%%%%%%%%%%%%%%%%%%%%%%%%%%%%%%%%%%%%%%%%%%%%%%%%%%%%%%%
%%%%%%%%%%%%%%%%%%%%%%%%%%%%%%%%%%%%%%%%%%%%%%%%%%%%%%%%%%%%%%%%%%%%%%%%%%%%%%%%
\section{Usage}

First of all, the package \textsf{childdoc} is \emph{not} a standard
\LaTeXe{} |.sty| style file! Therefore it needs to be invoked in
a non-standard way.

%%%%%%%%%%%%%%%%%%%%%%%%%%%%%%%%%%%%%%%%%%%%%%%%%%%%%%%%%%%%%%%%%%%%%%%%%%%%%%%%
\subsection{Included Files}
\label{sec:include}

%%%%%%%%%%%%%%%%%%%%%%%%%%%%%%%%%%%%%%%%
\DescribeMacro{\childdocmain}
To use the package, add the commands
\begin{center}
\begin{tabular}{l}
|\input{childdoc.def}|\\
|\childdocmain{}|\\
\end{tabular}
\end{center}
at the very top of the main \LaTeX{} file,
in particular \emph{before} the |\documentclass| statement!
The argument of |\childdocmain| should be left empty
(but it must be present).

%%%%%%%%%%%%%%%%%%%%%%%%%%%%%%%%%%%%%%%%
\DescribeMacro{\childdocof}
Furthermore, add the commands
\begin{center}
\begin{tabular}{l}
|\input{childdoc.def}|\\
|\childdocof{|\textit{main}|}|\\
\end{tabular}
\end{center}
at the top of every child file \textit{child}
which is included by |\include{|\textit{child}|}|
from within the main file
(or at least for those files to be compiled individually).
The argument \textit{main} must be the filename of the main file.

There are a couple of
considerations in setting up the main and child documents:

%%%%%%%%%%%%%%%%%%%%%%%%%%%%%%%%%%%%%%%%
\paragraph{Restrictions.}

Please note the following restrictions:
\begin{itemize}
\item
|\childdocmain| must be called with one argument \textit{main}
to ensure compatibility with earlier version of the package.
It must either be empty (|\childdocmain{}|)
or precisely match the filename of the main file in which it is specified.
See \secref{sec:detection} for further information.
\item
The filename \textit{main} must be specified without the |.tex| extension.
\item
The filename \textit{main} is case sensitive
(even in case-insensitive file systems)
due to internal string comparison.
\item
The argument \textit{main} should be fully expanded, it cannot be a macro.
\item
Subdirectories and special characters should be avoided in filenames.
\item
The command |\childdocmain{|\textit{main}|}| must be followed by a whitespace.
It should not be followed immediately by another command
or by a comment mark `|%|'.
This is because the \TeX{} parser reads the token immediately following
the argument of |\childdocmain| and puts it
at the beginning of every child section;
however, a white\-space is ignored.
\end{itemize}

%%%%%%%%%%%%%%%%%%%%%%%%%%%%%%%%%%%%%%%%
\paragraph{Content of Main File.}

It is advisable to place all content in the child files included by |\include|.
Any output contained in the main file will appear in all child documents
unless suppressed manually;
it cannot be suppressed automatically by the |\includeonly| directive
and thus should normally be avoided.
A method to include some content in the main file
by means of conditional processing is described in \secref{sec:conditional}.

%%%%%%%%%%%%%%%%%%%%%%%%%%%%%%%%%%%%%%%%
\paragraph{Page Numbering.}

When only a part of the document is compiled,
the appropriate numbering of pages
(as well as other status parameters)
is determined from the |.aux| files.
The latter contain information from previous passes.
However this information needs to propagate through
all intermediate child documents.
Therefore the page numbering in child documents may well
be inconsistent until the complete document is compiled at least once.

A useful (if unconventional) way to always ensure a consistent
page numbering is to restart the numbering in each child document
and denote the pages by `\textit{child}|.|\textit{page}'
where \textit{child} represents the chapter/section number of the child file.
This can be achieved by the command
|\numberwithin{page}{|\textit{child}|}|
of the \textsf{amsmath} package
where \textit{child} can be |chapter| or |section|
depending on the chosen structuring.
Alternatively, one can modify the macro |\thepage| appropriately
and reset the counter |page| at the start of each child file.

%%%%%%%%%%%%%%%%%%%%%%%%%%%%%%%%%%%%%%%%%%%%%%%%%%%%%%%%%%%%%%%%%%%%%%%%%%%%%%%%
\subsection{Conditional Processing}
\label{sec:conditional}

The package provides a mechanism to compile different versions
of a document. To customise the versions further some conditional processing
can come in handy to distinguish which version is being compiled.
The package provides two macros to describe the compilation context:

%%%%%%%%%%%%%%%%%%%%%%%%%%%%%%%%%%%%%%%%
\DescribeMacro{\ifchilddoc}
The conditional |\ifchilddoc| distinguishes between the compilation of
child documents and the main document:
%
\begin{center}
|\ifchilddoc |\textit{child-code}| |[|\||else |\textit{main-code}]| \||fi|
\end{center}

%%%%%%%%%%%%%%%%%%%%%%%%%%%%%%%%%%%%%%%%
\DescribeMacro{\childdocname}
\DescribeMacro{\childdocjob}
The macro |\childdocname| contains the filename (without extension)
of the main or child file being processed.
Note that |\childdocjob| will always contain the name of the main file.

%%%%%%%%%%%%%%%%%%%%%%%%%%%%%%%%%%%%%%%%
\paragraph{Title Page.}

Conditional processing can be used to include a title or banner page
in the main document when proper precautions are taken.
Importantly, the code in the main file should ensure that the page counter
(as well as other status parameters which are stored in the |.aux| files)
takes the same value after the conditional processing.
Otherwise the page numbers may take divergent values
depending on which part is compiled.

For example, a title page could be declared by:
%
\begin{center}
\begin{tabular}{l}
|\ifchilddoc\||else|\\
|\addtocounter{page}{-1}|\\
\textit{code for title page}\\
|\newpage|\\
|\||fi|
\end{tabular}
\end{center}
%
A banner page for the child documents can be generated by:
%
\begin{center}
\begin{tabular}{l}
|\ifchilddoc|\\
|\addtocounter{page}{-1}|\\
\textit{code for banner page}\\
|\newpage|\\
|\||fi|
\end{tabular}
\end{center}
%
Here one could write a message such as:
\begin{center}
|This is the part \childdocname{} of \childdocjob{}.|
\end{center}

%%%%%%%%%%%%%%%%%%%%%%%%%%%%%%%%%%%%%%%%%%%%%%%%%%%%%%%%%%%%%%%%%%%%%%%%%%%%%%%%
\subsection{Flags}
\label{sec:flags}

The package makes it easy to generate different versions
of the main or child documents.
To this end compilation flags can be defined
and assigned different default values.
They will be particularly useful in conjunction
with the forwarding mechanism described in \secref{sec:forward}.

For example, it may be useful to have a flag |\version|
which can be set to |draft| or |final|.
The document source will contain some conditional code
depending on the value of |\version|.
Suppose further, the flag should default to |final| for the main file
and to |draft| for child files
which is a natural assignment for editing the document.
This is achieved by placing the following code
in the preamble of the main document
(below the |\childdocmain| directive):
%
\begin{center}
\begin{tabular}{l}
|\ifchilddoc|\\
|\providecommand{\version}{draft}|\\
|\||else|\\
|\providecommand{\version}{final}|\\
|\||fi|
\end{tabular}
\end{center}
%
The definition by |\providecommand| makes sure
that previous definitions are not overwritten.
Further statements |\providecommand{\version}{...}|
can thus be added before the above code to override it.

For the main file, one might add a line
(between |\childdocmain| and the above block)
%
\begin{center}
|%\ifchilddoc\||else\providecommand{\version}{draft}\||fi|
\end{center}
%
which can be uncommented to produce a draft version.
Likewise one can add a line to the very top of a child file
(above the |\childdocof{|\textit{main}|}| directive)
%
\begin{center}
|%\providecommand{\version}{final}|
\end{center}
%
which can be uncommented to produce the final version of this child document.

%%%%%%%%%%%%%%%%%%%%%%%%%%%%%%%%%%%%%%%%%%%%%%%%%%%%%%%%%%%%%%%%%%%%%%%%%%%%%%%%
\subsection{Forwarding}
\label{sec:forward}

Different versions of the main or child documents
using compilation flags as described in \secref{sec:flags}
can be (permanently) stored in different files
for convenient compilation, viewing and distribution.
To this end, the package defines a command
to pass on compilation to a different file:

%%%%%%%%%%%%%%%%%%%%%%%%%%%%%%%%%%%%%%%%
\DescribeMacro{\childdocforward}
The command |\childdocforward| redirects processing to
another source file:
%
\begin{center}
\begin{tabular}{l}
|\input{childdoc.def}|\\
|\childdocforward[|\textit{main}|]{|\textit{dest}|}|\\
\end{tabular}
\end{center}
%
The argument \textit{dest} is the destination file
(without extension).
It should be the main file or one of the child files.
Note that further \textsf{childdoc} directives
such as |\childdocof| and |\childdocforward|
in the indicated file will be processed in this form.
The optional argument \textit{main}
passes on directly to the main file \textit{main}
while pretending to compile the child \textit{dest}.
This form behaves as if \textit{dest}
issues |\childdocof{|\textit{main}|}| right away,
and no further \textsf{childdoc} directives will be processed.

%%%%%%%%%%%%%%%%%%%%%%%%%%%%%%%%%%%%%%%%
\DescribeMacro{\...prefix}
In the alternative form |\childdocforwardprefix|,
%
\begin{center}
\begin{tabular}{l}
|\input{childdoc.def}|\\
|\childdocforwardprefix[|\textit{main}|]{|\textit{prefix}|}{|\textit{dest}|}|
\end{tabular}
\end{center}
%
the destination file is determined by a pattern
depending on the current file:
To make this work, the current file must be called
`{\textit{prefix}\hspace{0.2em}\textit{suffix}}'
with \textit{prefix} matching precisely the argument.
Processing is then passed on to the file
`{\textit{dest}\hspace{0.2em}\textit{suffix}}'.
Surely, the same effect is achieved by
directly specifying the
argument `{\textit{dest}\hspace{0.2em}\textit{suffix}}'
in the first form.
However, that requires to set up a different file
for each child. With the alternative form of the command
all these files can have exactly the same content
which simplifies setting them up and maintaining them.

For example, the following file |draft.tex|
with a compilation flag |\version| as described in \secref{sec:flags}
compiles the main document as a draft:
%
\begin{center}
\begin{tabular}{l}
|\def\version{draft}|\\
|\input{childdoc.def}|\\
|\childdocforward{|\textit{main}|}|
\end{tabular}
\end{center}
%
Likewise, the following files |final|\textit{nn}|.tex|
compile the final version of the child document
|child|\textit{nn}|.tex|:
%
\begin{center}
\begin{tabular}{l}
|\def\version{final}|\\
|\input{childdoc.def}|\\
|\childdocforwardprefix{final}{child}|
\end{tabular}
\end{center}
%

Note that when several versions of a main file and/or of each child file
are to be generated, it may be convenient to set up a |Makefile| or
shell script to automatise the process.

%%%%%%%%%%%%%%%%%%%%%%%%%%%%%%%%%%%%%%%%%%%%%%%%%%%%%%%%%%%%%%%%%%%%%%%%%%%%%%%%
\subsection{Command Line Processing}
\label{sec:commandline}

The effect of redirection files can also be achieved by invoking
the \LaTeX{} compiler with a more elaborate command line.
Most conveniently this should be done as part
of a shell script or a |Makefile|.

When using \textsf{childdoc} in the main file, the following
command lines effectively perform a redirection
(note that depending on the shell being used,
backslashes may have to be doubled: `|\|' $\to$ `|\\|'):
%
\begin{center}
|... -jobname "|\textit{target}|" |\\|"|[\textit{flags}]%
|\input{childdoc.def}\childdocforward[|\textit{main}|]{|\textit{dest}|}"|
\end{center}
%
Here \textit{target} is the name of the output file,
\textit{main} is the name of the main file
and \textit{dest} is the name of the main or child file to be processed
(all filenames without extensions).
The optional argument \textit{main} can be omitted
if \textit{main} matches \textit{dest}.
Optionally, compilation \textit{flags} can be defined via |\def| commands.
This command line makes the \TeX{} engine believe
it is compiling the file \textit{target}
whose content is specified as the latter parameter.
The provided code then forwards the processing to
\textit{main} or \textit{dest} as described in \secref{sec:forward}.

%%%%%%%%%%%%%%%%%%%%%%%%%%%%%%%%%%%%%%%%%%%%%%%%%%%%%%%%%%%%%%%%%%%%%%%%%%%%%%%%
\subsection{Include by Input}
\label{sec:input}

Including child documents by |\include| has some restrictions by design.
Most notably, the content of a child document always occupies
its own set of pages; pages cannot be shared between child documents.
Usually, this behaviour makes perfect sense
because each child document contain an essential part of the document.
However, in some situations it may be desirable to compose
a document from a collection of parts
without having mandatory page breaks between then.
For this case, the package
provides a mechanism to include parts
by |\input| which can also be processed individually.
However, by construction this mechanism
requires manual handling of the content to be output.

%%%%%%%%%%%%%%%%%%%%%%%%%%%%%%%%%%%%%%%%
\DescribeMacro{\ifchilddocmanual}
The main file should be prepared as usual, see \secref{sec:include}.
However, the document body must make a distinction
between processing of an individual part and of the main document, e.g.:
%
\begin{center}
\begin{tabular}{l}
|\ifchilddocmanual|\\
|\input{\childdocname}|\\
|\||else|\\
\textit{document body with }|\input{|\textit{part}|}|\\
|\||fi|
\end{tabular}
\end{center}
%
The conditional |\ifchilddocmanual| is true whenever
a part to be included by |\input| is being compiled,
and the name of the part is stored in |\childdocname|.

%%%%%%%%%%%%%%%%%%%%%%%%%%%%%%%%%%%%%%%%
\DescribeMacro{\childdocby}
Each part to be included by |\input| should start with:
%
\begin{center}
\begin{tabular}{l}
|\input{childdoc.def}|\\
|\childdocby{|\textit{main}|}|\\
\end{tabular}
\end{center}
%
The directive |\childdocby| is similar to |\childdocof|
described in \secref{sec:include},
but the subsequent selection of content must be done manually.
To that end, both |\ifchilddoc| and |\ifchilddocmanual|
will be true upon processing of a part,
and the name of the part is stored in |\childdocname|.
Note that |\jobname| will be set to the filename of the current part
so that each part receives an individual |.aux| file
that does not interfere with the |.aux| file(s) of the main document.
This behaviour can be altered by the alternative form
|\childdocby[*]{|\textit{main}|}| (with a non-empty optional argument)
which uses the |.aux| file of the main document
by setting |\jobname| to \textit{main}.

%%%%%%%%%%%%%%%%%%%%%%%%%%%%%%%%%%%%%%%%%%%%%%%%%%%%%%%%%%%%%%%%%%%%%%%%%%%%%%%%
\subsection{Driver Development}
\label{sec:driver}

The \textsf{childdoc} mechanism can also be use for the development
of definition files such as \LaTeX{} styles or classes.
This case differs from the above setup with multiple parts
included by |\include| in that no |\includeonly| should be invoked.
This can be achieved by starting the include file
(before |\ProvidesPackage|) with:
%
\begin{center}
\begin{tabular}{l}
|\input{childdoc.def}|\\
|\childdocforward{|\textit{main}|}|\\
\end{tabular}
\end{center}
%
or alternatively with:
%
\begin{center}
\begin{tabular}{l}
|\input{childdoc.def}|\\
|\childdocby{|\textit{main}|}|\\
\end{tabular}
\end{center}
%
Both forms have slightly different effects as described above.
The main file is prepared as usual, see \secref{sec:include}.

%%%%%%%%%%%%%%%%%%%%%%%%%%%%%%%%%%%%%%%%%%%%%%%%%%%%%%%%%%%%%%%%%%%%%%%%%%%%%%%%
\subsection{Legacy Detection}
\label{sec:detection}

The directive |\childdocmain| in the main file can detect
whether the complete document or merely a child is to be compiled
even without using the directive |\childdocof|.
This method is deprecated because it is less robust
and there is no compelling reason to use it;
it is merely provided for backward compatibility
and it may be removed in future versions.

If the detection mechanism is to be used,
it is mandatory to correctly specify
the filename of the main file as the argument of |\childdocmain|:
%
\begin{center}
\begin{tabular}{l}
|\input{childdoc.def}|\\
|\childdocmain{|\textit{main}|}|\\
\end{tabular}
\end{center}
%
If |\jobname| does not match the argument \textit{main} of |\childdocmain|,
it is assumed that |\jobname| points to the child file to be compiled.
When using |\childdocmain| with the main file specified as argument,
it suffices to start a child file
with just |\input{|\textit{main}|}|
without loading of the package and using |\childdocof|.
If instead all processing is done
with the appropriate \textsf{childdoc} directives,
the argument of \textit{main} of |\childdocmain| can be empty.

An alternative version of the command line processing described
in \secref{sec:commandline} using the detection mechanism reads:
%
\begin{center}
|... -jobname "|\textit{target}|" "|[\textit{flags}]%
[|\def\jobname{|\textit{dest}|}|]|\input{|\textit{main}|}"|
\end{center}

%%%%%%%%%%%%%%%%%%%%%%%%%%%%%%%%%%%%%%%%%%%%%%%%%%%%%%%%%%%%%%%%%%%%%%%%%%%%%%%%
\subsection{Manual Code}
\label{sec:manual}

In case one cannot be certain whether the definitions file |childdoc.def|
is installed on the target \TeX{} distribution
and one prefers not to ship it,
it is conceivable to paste a few relevant commands into the sources.

To that end, drop all statements |\input{childdoc.def}|
and perform the replacements as outlined below.
Instead of |\childdocmain{|\textit{main}|}| add the following code
to the top of the main file:
%
\begin{center}
\begin{tabular}{l}
|\||ifdefined\childdocname\endinput\||fi\newif\ifchilddoc|\\
|\edef\childdocname{\scantokens\expandafter{\jobname\noexpand}}|\\
|\def\childdocmain{|\textit{main}|}\||ifx\childdocmain\childdocname\||else|\\
|\childdoctrue\includeonly{\childdocname}\let\jobname\childdocmain\||fi|\\
\end{tabular}
\end{center}
%
Instead of |\childdocof{|\textit{main}|}| just include the main file
at the top of each child file:
%
\begin{center}
|\input{|\textit{main}|}|
\end{center}
%
A simple redirection |\childdocforward{|\textit{dest}|}| is achieved by:
%
\begin{center}
|\def\jobname{|\textit{dest}|}\input{\jobname}|
\end{center}
%
The redirection with prefix
|\childdocforwardprefix[|\textit{prefix}|]{|\textit{dest}|}|
is accomplished by:
%
\begin{center}
\begin{tabular}{l}
|{\edef\jobname{\scantokens\expandafter{\jobname\noexpand}}|\\
|\def\redirectjob |\textit{prefix}|#1~~~{\gdef\jobname{|\textit{dest}|#1}}|\\
|\expandafter\redirectjob\jobname~~~}\input{\jobname}|
\end{tabular}
\end{center}

In an alternative approach,
child documents can be compiled by a specific command line
without additional code or specific definitions:
%
\begin{center}
|... -jobname "|\textit{target}|" "|[\textit{flags}]%
|\includeonly{|\textit{dest}|}\input{|\textit{main}|}"|
\end{center}
%

%%%%%%%%%%%%%%%%%%%%%%%%%%%%%%%%%%%%%%%%%%%%%%%%%%%%%%%%%%%%%%%%%%%%%%%%%%%%%%%%
%%%%%%%%%%%%%%%%%%%%%%%%%%%%%%%%%%%%%%%%%%%%%%%%%%%%%%%%%%%%%%%%%%%%%%%%%%%%%%%%
\section{Information}

%%%%%%%%%%%%%%%%%%%%%%%%%%%%%%%%%%%%%%%%%%%%%%%%%%%%%%%%%%%%%%%%%%%%%%%%%%%%%%%%
\subsection{Copyright}

Copyright \copyright{} 2017--2018 Niklas Beisert

This work may be distributed and/or modified under the
conditions of the \LaTeX{} Project Public License, either version 1.3
of this license or (at your option) any later version.
The latest version of this license is in
  \url{http://www.latex-project.org/lppl.txt}
and version 1.3 or later is part of all distributions of \LaTeX{}
version 2005/12/01 or later.

This work has the LPPL maintenance status `maintained'.

The Current Maintainer of this work is Niklas Beisert.

This work consists of the files |README.txt|, |childdoc.ins| and |childdoc.dtx|
as well as the derived files |childdoc.def|, |cdocsamp.tex|
with |cdocsch1.tex|, |cdocsch2.tex|, |cdocspt3.tex|, |cdocspt4.tex|,
|cdocsdrf.tex|, |cdocsfn1.tex|, |cdocsfn2.tex|
as well as |childdoc.pdf|.

%%%%%%%%%%%%%%%%%%%%%%%%%%%%%%%%%%%%%%%%%%%%%%%%%%%%%%%%%%%%%%%%%%%%%%%%%%%%%%%%
\subsection{Files and Installation}

The package consists of the files:
%
\begin{center}
\begin{tabular}{ll}
    |README.txt|   & readme file \\
    |childdoc.ins| & installation file \\
    |childdoc.dtx| & source file \\
    |childdoc.def| & definition file \\
    |cdocsamp.tex| & sample main file \\
    |cdocsch1.tex| & sample include file \\
    |cdocsch2.tex| & sample include file \\
    |cdocspt3.tex| & sample part file \\
    |cdocspt4.tex| & sample part file \\
    |cdocsdrf.tex| & sample redirection file \\
    |cdocsfn1.tex| & sample redirection file \\
    |cdocsfn2.tex| & sample redirection file \\
    |childdoc.pdf| & manual
\end{tabular}
\end{center}
%
The distribution consists of the files
|README.txt|, |childdoc.ins| and |childdoc.dtx|.
%
\begin{itemize}
\item
Run (pdf)\LaTeX{} on |childdoc.dtx|
to compile the manual |childdoc.pdf| (this file).
\item
Run \LaTeX{} on |childdoc.ins| to create the definitions file |childdoc.def|
and the sample |cdocsamp.tex| with include files
|cdocsch1.tex|, |cdocsch2.tex|, |cdocspt3.tex|, |cdocspt4.tex|,
|cdocsdrf.tex|, |cdocsfn1.tex|, |cdocsfn2.tex|.
Then copy the file |childdoc.def| to an appropriate directory of your \LaTeX{}
distribution, e.g.\ \textit{texmf-root}|/tex/latex/childdoc|.
\end{itemize}

%%%%%%%%%%%%%%%%%%%%%%%%%%%%%%%%%%%%%%%%%%%%%%%%%%%%%%%%%%%%%%%%%%%%%%%%%%%%%%%%
\subsection{Related CTAN Packages}

There are several other packages which offer a similar functionality:
%
\begin{itemize}
\item
The packages
\href{http://ctan.org/pkg/docmute}{\textsf{docmute}},
\href{http://ctan.org/pkg/includex}{\textsf{includex}} and
\href{http://ctan.org/pkg/standalone}{\textsf{standalone}}
provide commands to include only the document body of
a child file thus allowing both files to be compiled individually.
\item
The packages \href{http://ctan.org/pkg/subdocs}{\textsf{subdocs}}
and \href{http://ctan.org/pkg/subfiles}{\textsf{subfiles}}
provide structures in which the main and child documents can be
encapsulated and allowing them to be compiled individually.
The inclusion mechanism is different from the conventional |\include|.
\item
The package \href{http://ctan.org/pkg/combine}{\textsf{combine}}
is an elaborate solution to combine several documents into one.
\end{itemize}
%
See also the CTAN topic \href{http://ctan.org/topic/subdocs}{\textsf{subdocs}}
for further related packages.
The present package differs from the above solutions in that
a document structure constructed with the conventional |\include| mechanism
just needs two extra commands at the top of every file
such that all constituent files can be compiled individually.

%%%%%%%%%%%%%%%%%%%%%%%%%%%%%%%%%%%%%%%%%%%%%%%%%%%%%%%%%%%%%%%%%%%%%%%%%%%%%%%%
%\subsection{Feature Suggestions}
%
%The following is a list of features which may be useful for future
%versions of this package:
%%
%\begin{itemize}
%\item
%\ldots
%\end{itemize}

%%%%%%%%%%%%%%%%%%%%%%%%%%%%%%%%%%%%%%%%%%%%%%%%%%%%%%%%%%%%%%%%%%%%%%%%%%%%%%%%
\subsection{Revision History}

%%%%%%%%%%%%%%%%%%%%%%%%%%%%%%%%%%%%%%%%
\paragraph{v2.0:} 2018/12/30

\begin{itemize}
\item
immediate forward processing
\item
added |\childdocby| mechanism
\item
manual restructured
\end{itemize}

%%%%%%%%%%%%%%%%%%%%%%%%%%%%%%%%%%%%%%%%
\paragraph{v1.6:} 2018/01/17

\begin{itemize}
\item
application for development of include files
\item
corrections to manual
\end{itemize}

%%%%%%%%%%%%%%%%%%%%%%%%%%%%%%%%%%%%%%%%
\paragraph{v1.5:} 2017/05/21

\begin{itemize}
\item
more complete structuring introduced
\item
|\childdocof| introduced
\item
|\childdoc| renamed to |\childdocmain|
\item
|\childredirect| renamed to |\childdocforward| and |\childdocforwardprefix|
and functionality expanded
\end{itemize}

%%%%%%%%%%%%%%%%%%%%%%%%%%%%%%%%%%%%%%%%
\paragraph{v1.0:} 2017/04/27

\begin{itemize}
\item
manual and install package
\item
first version published on CTAN
\end{itemize}

%%%%%%%%%%%%%%%%%%%%%%%%%%%%%%%%%%%%%%%%
\paragraph{v0.6:} 2017/04/26

\begin{itemize}
\item
redirection mechanism added
\end{itemize}

%%%%%%%%%%%%%%%%%%%%%%%%%%%%%%%%%%%%%%%%
\paragraph{v0.5:} 2017/04/26

\begin{itemize}
\item
functionality in definition file
\end{itemize}


%%%%%%%%%%%%%%%%%%%%%%%%%%%%%%%%%%%%%%%%%%%%%%%%%%%%%%%%%%%%%%%%%%%%%%%%%%%%%%%%
%%%%%%%%%%%%%%%%%%%%%%%%%%%%%%%%%%%%%%%%%%%%%%%%%%%%%%%%%%%%%%%%%%%%%%%%%%%%%%%%
%%%%%%%%%%%%%%%%%%%%%%%%%%%%%%%%%%%%%%%%%%%%%%%%%%%%%%%%%%%%%%%%%%%%%%%%%%%%%%%%
\appendix

\settowidth\MacroIndent{\rmfamily\scriptsize 000\ }

 \DocInput{childdoc.dtx}

\end{document}
%</driver>
% \fi
%
% %%%%%%%%%%%%%%%%%%%%%%%%%%%%%%%%%%%%%%%%%%%%%%%%%%%%%%%%%%%%%%%%%%%%%%%%%%%%%%
% %%%%%%%%%%%%%%%%%%%%%%%%%%%%%%%%%%%%%%%%%%%%%%%%%%%%%%%%%%%%%%%%%%%%%%%%%%%%%%
% \section{Sample}
%\iffalse
%<*samplemain>
%\fi
%
% The following presents a sample document
% with two chapters, two parts, a title page,
% a compile flag as well as three forwarding files to set the flag.
% It consists of eight |.tex| files:
% \begin{center}
% \begin{tabular}{ll}
% |cdocsamp.tex|&main file\\
% |cdocsch1.tex|&include file for chapter 1\\
% |cdocsch2.tex|&include file for chapter 2\\
% |cdocspt3.tex|&include file for part 3\\
% |cdocspt4.tex|&include file for part 4\\
% |cdocsdrf.tex|&forwarding file for main file in draft mode\\
% |cdocsfi1.tex|&forwarding file for final version of chapter 1\\
% |cdocsfi2.tex|&forwarding file for final version of chapter 2\\
% \end{tabular}
% \end{center}
% Each of the eight files can be compiled directly by the \LaTeX{} compiler.
%
% %%%%%%%%%%%%%%%%%%%%%%%%%%%%%%%%%%%%%%
% \paragraph{Main File.}
%
% The main file is called |cdocsamp.tex|.
%
% Load the \textsf{childdoc} definitions and
% declare the filename for the main document:
%    \begin{macrocode}
\input{childdoc.def}
\childdocmain{}
%    \end{macrocode}

% Optional override for |\version| flag:
%    \begin{macrocode}
%%\ifchilddoc\else\providecommand{\version}{draft}\fi
%    \end{macrocode}

% Define the default values for the |\version| flag
% (|final| for the main file and |draft| for childs):
%    \begin{macrocode}
\ifchilddoc
\providecommand{\version}{draft}
\else
\providecommand{\version}{final}
\fi
%    \end{macrocode}

% Load the standard document class:
%    \begin{macrocode}
\documentclass[12pt]{article}
%    \end{macrocode}

% Start the document body:
%    \begin{macrocode}
\begin{document}
%    \end{macrocode}

% Declare a title page.
% Print title, part of document being processed and version flag:
%    \begin{macrocode}
\addtocounter{page}{-1}
\begin{center}
{\LARGE\bfseries{}childdoc example\par}
\vspace{1cm}
\ifchilddoc
\ifchilddocmanual part\else chapter\fi:
`\childdocname' of `\childdocjob'\par
\else
main document: `\childdocjob'\par
\fi
version: \version\par
\end{center}
\newpage
%    \end{macrocode}

% Manually include selected file,
% otherwise process as usual:
%    \begin{macrocode}
\ifchilddocmanual
\section*{part `\childdocname'}
\input{\childdocname}
\else
%    \end{macrocode}

% Include the two chapters:
%    \begin{macrocode}
\include{cdocsch1}
\include{cdocsch2}
%    \end{macrocode}

% Include the two parts unless only chapters should be displayed:
%    \begin{macrocode}
\ifchilddoc\else
\section{part three}
\input{cdocspt3}
\section{part four}
\input{cdocspt4}
\fi
%    \end{macrocode}

% Process as usual until here:
%    \begin{macrocode}
\fi
%    \end{macrocode}

% End of document body:
%    \begin{macrocode}
\end{document}
%    \end{macrocode}
%\iffalse
%</samplemain>
%\fi
%
% %%%%%%%%%%%%%%%%%%%%%%%%%%%%%%%%%%%%%%
% \paragraph{Chapter Include Files.}
%
% The include files are called |cdocsch1.tex| and |cdocsch2.tex|.
%
%\iffalse
%<*samplechap1|samplechap2>
%\fi

% Optional override for |\version| flag:
%    \begin{macrocode}
%%\providecommand{\version}{final}
%    \end{macrocode}

% Include the main document:
%    \begin{macrocode}
\input{childdoc.def}
\childdocof{cdocsamp}
%    \end{macrocode}

%\iffalse
%</samplechap1|samplechap2>
%\fi
%
%\iffalse
%<*samplechap1>
%\fi
% Some text for chapter 1:
%    \begin{macrocode}
\section{one}
some text in chapter one
%    \end{macrocode}

%\iffalse
%</samplechap1>
%\fi
% Some text for chapter 2:
%\iffalse
%<*samplechap2>
%\fi
%    \begin{macrocode}
\section{two}
more text in chapter two
%    \end{macrocode}

%\iffalse
%</samplechap2>
%\fi
%
% %%%%%%%%%%%%%%%%%%%%%%%%%%%%%%%%%%%%%%
% \paragraph{Part Include Files.}
%
% The include files are called |cdocspt3.tex| and |cdocspt4.tex|.
%
%\iffalse
%<*samplepart3|samplepart4>
%\fi

% Optional override for |\version| flag:
%    \begin{macrocode}
%%\providecommand{\version}{final}
%    \end{macrocode}

% Include the main document:
%    \begin{macrocode}
\input{childdoc.def}
\childdocby{cdocsamp}
%    \end{macrocode}

%\iffalse
%</samplepart3|samplepart4>
%\fi
%
%\iffalse
%<*samplepart3>
%\fi
% Some text for part 3:
%    \begin{macrocode}
some text in part three
%    \end{macrocode}

%\iffalse
%</samplepart3>
%\fi
% Some text for part 4:
%\iffalse
%<*samplepart4>
%\fi
%    \begin{macrocode}
more text in part four
%    \end{macrocode}

%\iffalse
%</samplepart4>
%\fi
%
% %%%%%%%%%%%%%%%%%%%%%%%%%%%%%%%%%%%%%%
% \paragraph{Forwarding for a Complete Draft.}
%
% The following forwarding file |cdocsdrf.tex|
% compiles the main document in draft mode:
%\iffalse
%<*sampledraft>
%\fi
%    \begin{macrocode}
\def\version{draft}
\input{childdoc.def}
\childdocforward{cdocsamp}
%    \end{macrocode}

%\iffalse
%</sampledraft>
%\fi
%
% %%%%%%%%%%%%%%%%%%%%%%%%%%%%%%%%%%%%%%
% \paragraph{Forwarding for Final Version of the Chapters.}
%
% The following forwarding files |cdocsfn1.tex| and |cdocsfn2.tex|
% (with identical content)
% compile the final versions of the child documents
% |cdocsch1.tex| and |cdocsch2.tex|, respectively:
%\iffalse
%<*samplefinal>
%\fi
%    \begin{macrocode}
\def\version{final}
\input{childdoc.def}
\childdocforwardprefix[cdocsamp]{cdocsfn}{cdocsch}
%    \end{macrocode}

%\iffalse
%</samplefinal>
%\fi
%
% %%%%%%%%%%%%%%%%%%%%%%%%%%%%%%%%%%%%%%
% \paragraph{Command Line Processing.}
%
% The following three command lines generate the output files
% |cdocscld|, |cdocscl1| and |cdocscl2|
% which should be identical to
% |cdocsdrf|, |cdocsch1| and |cdocsfn2|, respectively:
% \begin{center}
% \begin{tabular}{l}
% |latex -jobname cdocscld \|\\
% |  "\def\version{draft}\input{childdoc.def}\childdocforward{cdocsamp}"|\\
% |latex -jobname cdocscl1 \|\\
% |  "\input{childdoc.def}\childdocforward[cdocsamp]{cdocsch1}"|\\
% |latex -jobname cdocscl2 \|\\
% |  "\def\version{final}\input{childdoc.def}\childdocforward{cdocsch2}"|
% \end{tabular}
% \end{center}
% Note that the trailing backslash on each first line
% merely continues the input to the second line
% (for convenient cut ant paste).
% Furthermore, the command |latex| can be replaced by any
% of its alternative versions such as |pdflatex|.
%
% %%%%%%%%%%%%%%%%%%%%%%%%%%%%%%%%%%%%%%%%%%%%%%%%%%%%%%%%%%%%%%%%%%%%%%%%%%%%%%
% %%%%%%%%%%%%%%%%%%%%%%%%%%%%%%%%%%%%%%%%%%%%%%%%%%%%%%%%%%%%%%%%%%%%%%%%%%%%%%
% \section{Implementation}
%\iffalse
%<*package>
%\fi
%
% This section describes the definitions file |childdoc.def|.

% The definitions cannot be loaded using |\usepackage| or |\RequirePackage|
% which has a mechanism to prevent loading a style file more than once.
% When loading the definitions by means of |\input|
% multiple instances have to be prevented manually:
%\iffalse
%This code needs to be before the `\ProvidesFile' directive
%which is defined at the beginning of this file.
%Therefore it is also placed there and commented out here.
%</package>
%<*discard>
%\fi
%    \begin{macrocode}
\ifdefined\childdocmain\endinput\fi
%    \end{macrocode}
%\iffalse
%</discard>
%<*package>
%\fi
%
% \macro{\ifchilddoc}
% \macro{\ifchilddocmanual}
% The conditional |\ifchilddoc| tells whether a
% child (true) or main (false) document is being compiled.
% The conditional |\ifchilddocmanual| tells whether
% the |\includeonly| mechanism is used (false) or
% the selection of child files must be performed manually (true).
% The definitions initialise to false:
%    \begin{macrocode}
\newif\ifchilddoc
\newif\ifchilddocmanual
%    \end{macrocode}

% \macro{\childdocname}
% \macro{\childdocjob}
% The macro |\childdocname| stores the name of the main document
% to be compiled. The macro |\childdocjob| stores the name of
% the document on which the \LaTeX{} compiler was originally invoked.
% The content of |\jobname| cannot be compared
% to filenames specified in the source due to different catcodes.
% The following code rescans |\jobname|, stores the result
% in |\childdocname| and saves a copy in |\childdocjob|:
%    \begin{macrocode}
\edef\childdocname{\scantokens\expandafter{\jobname\noexpand}}
\let\childdocjob\childdocname
%    \end{macrocode}

% \macro{\childdocdisable}
% The macro |\childdocdisable| prevents the main file
% from being processed more than once.
% At this stage, the main document command |\childdocmain|
% is assumed to be called once again where it should do nothing.
% Any subsequent call to it should prevent
% a secondary processing of the main document
% It overwrites the forwarding commands
% |\childdocof| and |\childdocforward|
% with empty macros to prevent further inclusions of the main document:
%    \begin{macrocode}
\newcommand{\childdocdisable}
{
  \renewcommand{\childdocmain}[1]{\renewcommand{\childdocmain}[1]{\endinput}}
  \renewcommand{\childdocof}[1]{}
  \renewcommand{\childdocby}[2][]{}
  \renewcommand{\childdocforward}[2][]{}
  \renewcommand{\childdocdisable}{}
}
%    \end{macrocode}

% \macro{\childdocmain}
% The macro |\childdocmain| is to be called at the top of the main file
% with nothing or the main filename (without extension) as argument.
% First, it breaks loops.
% If the argument is not empty and does not match |\childdocname|
% (which is set by the first inclusion of |childdoc.def|),
% |\ifchilddoc| is set to true, |\includeonly| is applied to the child file
% and |\jobname| is set to the main file
% (for proper handling of |.aux| files):
%    \begin{macrocode}
\newcommand{\childdocmain}[1]
{
  \childdocdisable\childdocmain{}
  \if?#1?\else
    \begingroup
      \def\childdoctmp{#1}
      \ifx\childdoctmp\childdocname
        \def\childdoctmp{}
      \else
        \def\childdoctmp
        {
          \childdoctrue
          \includeonly{\childdocname}
          \def\childdocjob{#1}
          \def\jobname{#1}
        }
      \fi
      \expandafter
    \endgroup
    \childdoctmp
  \fi
}
%    \end{macrocode}

% \macro{\childdocof}
% The command |\childdocof| redirects
% compilation to the main file |#1|.
%    \begin{macrocode}
\newcommand{\childdocof}[1]
{
  \childdocdisable
  \childdoctrue
  \includeonly{\childdocname}
  \def\jobname{#1}
  \def\childdocjob{#1}
  \input{#1}
}
%    \end{macrocode}

% \macro{\childdocby}
% The command |\childdocby| ....
%    \begin{macrocode}
\newcommand{\childdocby}[2][]
{
  \childdocdisable
  \childdoctrue
  \childdocmanualtrue
  \if?#1?\else
    \def\jobname{#2}
  \fi
  \def\childdocjob{#2}
  \input{#2}
  \endinput
}
%    \end{macrocode}

% \macro{\childdocforward}
% The command |\childdocforward| redirects
% compilation to the main file or
% (if the optional argument is given) a child file.
% Parameters are set as if the main file
% or a child file starting with |\childdocof| was compiled.
% Then compilation is handed over to the main file:
%    \begin{macrocode}
\newcommand{\childdocforward}[2][]
{
  \begingroup
    \if?#1?
      \def\childdoctmp
      {
        \def\childdocname{#2}
        \def\childdocjob{#2}
        \def\jobname{#2}
        \input{#2}
        \endinput
      }
    \else
      \def\childdoctmp
      {
        \childdocdisable
        \def\childdocname{#2}
        \childdoctrue
        \includeonly{#2}
        \def\childdocjob{#1}
        \def\jobname{#1}
        \input{#1}
        \endinput
      }
    \fi
    \expandafter
  \endgroup
  \childdoctmp
}
%    \end{macrocode}

% \macro{\childdocforwardprefix}
% The command |\childdocforwardprefix| redirects
% compilation to the main or a child file by means of a pattern.
% The prefix |#1| in the current filename is replaced by |#2|
% and the suffix of the current filename is kept
% (it is assumed that the filename does not contain the substring `|~~~|'
% which is used as a delimiter).
% Compilation is handed over to the new file by |\childdocforward|:
%    \begin{macrocode}
\newcommand{\childdocforwardprefix}[3][]
{
  \begingroup
    \def\childdocextract #2##1~~~{\def\childdoctmp{\childdocforward[#1]{#3##1}}}
    \expandafter\childdocextract\childdocname~~~
    \expandafter
  \endgroup
  \childdoctmp
}
%    \end{macrocode}

% \macro{\childdoc}
% The deprecated macro |\childdoc| is a legacy version of |\childdocmain|:
%    \begin{macrocode}
\newcommand{\childdoc}{\childdocmain}
%    \end{macrocode}

% \macro{\childdocredirect}
% The deprecated macro |\childdocredirect| is a legacy version
% of |\childdocforward| and |\childdocforwardprefix|:
%    \begin{macrocode}
\newcommand{\childdocredirect}[2][]
{
  \begingroup
    \if?#1?
      \def\childdoctmp{\childdocforward{#2}}
    \else
      \def\childdoctmp{\childdocforwardprefix{#1}{#2}}
    \fi
    \expandafter
  \endgroup
  \childdoctmp
}
%    \end{macrocode}

%\iffalse
%</package>
%\fi
%
\endinput

\childdocforwardprefix[cdocsamp]{cdocsfn}{cdocsch}
%    \end{macrocode}

%\iffalse
%</samplefinal>
%\fi
%
% %%%%%%%%%%%%%%%%%%%%%%%%%%%%%%%%%%%%%%
% \paragraph{Command Line Processing.}
%
% The following three command lines generate the output files
% |cdocscld|, |cdocscl1| and |cdocscl2|
% which should be identical to
% |cdocsdrf|, |cdocsch1| and |cdocsfn2|, respectively:
% \begin{center}
% \begin{tabular}{l}
% |latex -jobname cdocscld \|\\
% |  "\def\version{draft}% \iffalse
%
% childdoc.dtx Copyright (C) 2017-2018 Niklas Beisert
%
% This work may be distributed and/or modified under the
% conditions of the LaTeX Project Public License, either version 1.3
% of this license or (at your option) any later version.
% The latest version of this license is in
%   http://www.latex-project.org/lppl.txt
% and version 1.3 or later is part of all distributions of LaTeX
% version 2005/12/01 or later.
%
% This work has the LPPL maintenance status `maintained'.
%
% The Current Maintainer of this work is Niklas Beisert.
%
% This work consists of the files childdoc.dtx and childdoc.ins
% and the derived files childdoc.def and cdocsamp.tex with
% cdocsch1.tex, cdocsch2.tex, cdocsdrf.tex, cdocsfn1.tex, cdocsfn2.tex.
%
%<package>\ifdefined\childdocmain\endinput\fi
%<package>\ProvidesFile{childdoc.def}[2018/12/30 v2.0 child document driver]
%<samplemain>\ProvidesFile{cdocsamp.tex}[2018/12/30 v2.0 sample for childdoc]
%<*driver>
%\ProvidesFile{childdoc.drv}[2018/12/30 v2.0 childdoc reference manual file]
\PassOptionsToClass{10pt,a4paper}{article}
\documentclass{ltxdoc}

\usepackage[margin=35mm]{geometry}
\usepackage{hyperref}
\usepackage{hyperxmp}
\usepackage[usenames]{color}

\hypersetup{colorlinks=true}
\hypersetup{pdfstartview=FitH}
\hypersetup{pdfpagemode=UseNone}
\hypersetup{pdfsource={}}
\hypersetup{pdflang={en-UK}}
\hypersetup{pdfcopyright={Copyright 2017-2018 Niklas Beisert.
  This work may be distributed and/or modified under the
  conditions of the LaTeX Project Public License, either version 1.3
  of this license or (at your option) any later version.}}
\hypersetup{pdflicenseurl={http://www.latex-project.org/lppl.txt}}
\hypersetup{pdfcontactaddress={ETH Zurich, ITP, HIT K,
  Wolfgang-Pauli-Strasse 27}}
\hypersetup{pdfcontactpostcode={8093}}
\hypersetup{pdfcontactcity={Zurich}}
\hypersetup{pdfcontactcountry={Switzerland}}
\hypersetup{pdfcontactemail={nbeisert@itp.phys.ethz.ch}}
\hypersetup{pdfcontacturl={http://people.phys.ethz.ch/\xmptilde nbeisert/}}

\newcommand{\secref}[1]{\hyperref[#1]{section \ref*{#1}}}

\parskip1ex
\parindent0pt
\let\olditemize\itemize
\def\itemize{\olditemize\parskip0pt}

\begin{document}

\title{The \textsf{childdoc} Package}
\hypersetup{pdftitle={The childdoc Package}}
\author{Niklas Beisert\\[2ex]
  Institut f\"ur Theoretische Physik\\
  Eidgen\"ossische Technische Hochschule Z\"urich\\
  Wolfgang-Pauli-Strasse 27, 8093 Z\"urich, Switzerland\\[1ex]
  \href{mailto:nbeisert@itp.phys.ethz.ch}
  {\texttt{nbeisert@itp.phys.ethz.ch}}}
\hypersetup{pdfauthor={Niklas Beisert}}
\hypersetup{pdfsubject={Manual for the LaTeX2e Package childdoc}}
\date{30 December 2018, \textsf{v2.0}}
\maketitle

\begin{abstract}\noindent
\textsf{childdoc} is a \LaTeXe{} package
that enables the direct compilation
of document sections included by |\include|
to individual files.
\end{abstract}

\begingroup
\parskip0ex
\tableofcontents
\endgroup

%%%%%%%%%%%%%%%%%%%%%%%%%%%%%%%%%%%%%%%%%%%%%%%%%%%%%%%%%%%%%%%%%%%%%%%%%%%%%%%%
%%%%%%%%%%%%%%%%%%%%%%%%%%%%%%%%%%%%%%%%%%%%%%%%%%%%%%%%%%%%%%%%%%%%%%%%%%%%%%%%
\section{Introduction}

\LaTeX{} provides a mechanism to structure a large document (such as a book)
into a main file and several child files (containing the chapters)
using the |\include| command.
This mechanism is beneficial for documents
which span hundreds of pages in order to
make the source file(s) more manageable.
Moreover, compilation can be restricted to
selected child files by means of the |\includeonly| command.
The latter feature can be used to reduce the compilation time while editing
(this was significantly more useful in the earlier days of \LaTeX{})
or to generate a smaller document which is easier to navigate.
Another application of |\includeonly| is to generate
documents consisting of selected parts of the complete document.

However, there are a few drawbacks of the plain |\include| mechanism:
\begin{itemize}
\item
The child files cannot be compiled on their own,
they can only be compiled via the main file.
A naive editing environment
(such as a text editor with an option
to have the current file processed by \LaTeX)
may require one to switch to the main file before compiling;
attempting to compile the child file produces errors.
\item
The main file must be modified (each time)
to adjust the |\includeonly| command
to the present needs. This easily leaves the main file in a messy state.
\item
The generated document will always carry the filename
of the main document. This is inconvenient if
several child files are to be compiled and
to be kept for distribution.
\end{itemize}

The present package provides a simple interface
to make child files individually compilable by \LaTeX{}.
Compiling a child file then has the same effect as compiling
the main file with an |\includeonly| command
to select the appropriate child.
Moreover the generated document will carry the name of the child
rather than the main file.
This resolves all three above issues.

This feature is meant to make the editing of books,
thesis documents and lecture notes somewhat more convenient.
However, the package can also be used efficiently for
composing a series of documents (such as exercise sheets)
which are typically distributed individually.
It then assists the author in generating the individual documents
(potentially in different versions)
as well as a document containing the collected series.
Another application is in developing style files
or other kinds of included material
where compilation of the style file could redirect
to a sample or test file.

%%%%%%%%%%%%%%%%%%%%%%%%%%%%%%%%%%%%%%%%%%%%%%%%%%%%%%%%%%%%%%%%%%%%%%%%%%%%%%%%
%%%%%%%%%%%%%%%%%%%%%%%%%%%%%%%%%%%%%%%%%%%%%%%%%%%%%%%%%%%%%%%%%%%%%%%%%%%%%%%%
\section{Usage}

First of all, the package \textsf{childdoc} is \emph{not} a standard
\LaTeXe{} |.sty| style file! Therefore it needs to be invoked in
a non-standard way.

%%%%%%%%%%%%%%%%%%%%%%%%%%%%%%%%%%%%%%%%%%%%%%%%%%%%%%%%%%%%%%%%%%%%%%%%%%%%%%%%
\subsection{Included Files}
\label{sec:include}

%%%%%%%%%%%%%%%%%%%%%%%%%%%%%%%%%%%%%%%%
\DescribeMacro{\childdocmain}
To use the package, add the commands
\begin{center}
\begin{tabular}{l}
|\input{childdoc.def}|\\
|\childdocmain{}|\\
\end{tabular}
\end{center}
at the very top of the main \LaTeX{} file,
in particular \emph{before} the |\documentclass| statement!
The argument of |\childdocmain| should be left empty
(but it must be present).

%%%%%%%%%%%%%%%%%%%%%%%%%%%%%%%%%%%%%%%%
\DescribeMacro{\childdocof}
Furthermore, add the commands
\begin{center}
\begin{tabular}{l}
|\input{childdoc.def}|\\
|\childdocof{|\textit{main}|}|\\
\end{tabular}
\end{center}
at the top of every child file \textit{child}
which is included by |\include{|\textit{child}|}|
from within the main file
(or at least for those files to be compiled individually).
The argument \textit{main} must be the filename of the main file.

There are a couple of
considerations in setting up the main and child documents:

%%%%%%%%%%%%%%%%%%%%%%%%%%%%%%%%%%%%%%%%
\paragraph{Restrictions.}

Please note the following restrictions:
\begin{itemize}
\item
|\childdocmain| must be called with one argument \textit{main}
to ensure compatibility with earlier version of the package.
It must either be empty (|\childdocmain{}|)
or precisely match the filename of the main file in which it is specified.
See \secref{sec:detection} for further information.
\item
The filename \textit{main} must be specified without the |.tex| extension.
\item
The filename \textit{main} is case sensitive
(even in case-insensitive file systems)
due to internal string comparison.
\item
The argument \textit{main} should be fully expanded, it cannot be a macro.
\item
Subdirectories and special characters should be avoided in filenames.
\item
The command |\childdocmain{|\textit{main}|}| must be followed by a whitespace.
It should not be followed immediately by another command
or by a comment mark `|%|'.
This is because the \TeX{} parser reads the token immediately following
the argument of |\childdocmain| and puts it
at the beginning of every child section;
however, a white\-space is ignored.
\end{itemize}

%%%%%%%%%%%%%%%%%%%%%%%%%%%%%%%%%%%%%%%%
\paragraph{Content of Main File.}

It is advisable to place all content in the child files included by |\include|.
Any output contained in the main file will appear in all child documents
unless suppressed manually;
it cannot be suppressed automatically by the |\includeonly| directive
and thus should normally be avoided.
A method to include some content in the main file
by means of conditional processing is described in \secref{sec:conditional}.

%%%%%%%%%%%%%%%%%%%%%%%%%%%%%%%%%%%%%%%%
\paragraph{Page Numbering.}

When only a part of the document is compiled,
the appropriate numbering of pages
(as well as other status parameters)
is determined from the |.aux| files.
The latter contain information from previous passes.
However this information needs to propagate through
all intermediate child documents.
Therefore the page numbering in child documents may well
be inconsistent until the complete document is compiled at least once.

A useful (if unconventional) way to always ensure a consistent
page numbering is to restart the numbering in each child document
and denote the pages by `\textit{child}|.|\textit{page}'
where \textit{child} represents the chapter/section number of the child file.
This can be achieved by the command
|\numberwithin{page}{|\textit{child}|}|
of the \textsf{amsmath} package
where \textit{child} can be |chapter| or |section|
depending on the chosen structuring.
Alternatively, one can modify the macro |\thepage| appropriately
and reset the counter |page| at the start of each child file.

%%%%%%%%%%%%%%%%%%%%%%%%%%%%%%%%%%%%%%%%%%%%%%%%%%%%%%%%%%%%%%%%%%%%%%%%%%%%%%%%
\subsection{Conditional Processing}
\label{sec:conditional}

The package provides a mechanism to compile different versions
of a document. To customise the versions further some conditional processing
can come in handy to distinguish which version is being compiled.
The package provides two macros to describe the compilation context:

%%%%%%%%%%%%%%%%%%%%%%%%%%%%%%%%%%%%%%%%
\DescribeMacro{\ifchilddoc}
The conditional |\ifchilddoc| distinguishes between the compilation of
child documents and the main document:
%
\begin{center}
|\ifchilddoc |\textit{child-code}| |[|\||else |\textit{main-code}]| \||fi|
\end{center}

%%%%%%%%%%%%%%%%%%%%%%%%%%%%%%%%%%%%%%%%
\DescribeMacro{\childdocname}
\DescribeMacro{\childdocjob}
The macro |\childdocname| contains the filename (without extension)
of the main or child file being processed.
Note that |\childdocjob| will always contain the name of the main file.

%%%%%%%%%%%%%%%%%%%%%%%%%%%%%%%%%%%%%%%%
\paragraph{Title Page.}

Conditional processing can be used to include a title or banner page
in the main document when proper precautions are taken.
Importantly, the code in the main file should ensure that the page counter
(as well as other status parameters which are stored in the |.aux| files)
takes the same value after the conditional processing.
Otherwise the page numbers may take divergent values
depending on which part is compiled.

For example, a title page could be declared by:
%
\begin{center}
\begin{tabular}{l}
|\ifchilddoc\||else|\\
|\addtocounter{page}{-1}|\\
\textit{code for title page}\\
|\newpage|\\
|\||fi|
\end{tabular}
\end{center}
%
A banner page for the child documents can be generated by:
%
\begin{center}
\begin{tabular}{l}
|\ifchilddoc|\\
|\addtocounter{page}{-1}|\\
\textit{code for banner page}\\
|\newpage|\\
|\||fi|
\end{tabular}
\end{center}
%
Here one could write a message such as:
\begin{center}
|This is the part \childdocname{} of \childdocjob{}.|
\end{center}

%%%%%%%%%%%%%%%%%%%%%%%%%%%%%%%%%%%%%%%%%%%%%%%%%%%%%%%%%%%%%%%%%%%%%%%%%%%%%%%%
\subsection{Flags}
\label{sec:flags}

The package makes it easy to generate different versions
of the main or child documents.
To this end compilation flags can be defined
and assigned different default values.
They will be particularly useful in conjunction
with the forwarding mechanism described in \secref{sec:forward}.

For example, it may be useful to have a flag |\version|
which can be set to |draft| or |final|.
The document source will contain some conditional code
depending on the value of |\version|.
Suppose further, the flag should default to |final| for the main file
and to |draft| for child files
which is a natural assignment for editing the document.
This is achieved by placing the following code
in the preamble of the main document
(below the |\childdocmain| directive):
%
\begin{center}
\begin{tabular}{l}
|\ifchilddoc|\\
|\providecommand{\version}{draft}|\\
|\||else|\\
|\providecommand{\version}{final}|\\
|\||fi|
\end{tabular}
\end{center}
%
The definition by |\providecommand| makes sure
that previous definitions are not overwritten.
Further statements |\providecommand{\version}{...}|
can thus be added before the above code to override it.

For the main file, one might add a line
(between |\childdocmain| and the above block)
%
\begin{center}
|%\ifchilddoc\||else\providecommand{\version}{draft}\||fi|
\end{center}
%
which can be uncommented to produce a draft version.
Likewise one can add a line to the very top of a child file
(above the |\childdocof{|\textit{main}|}| directive)
%
\begin{center}
|%\providecommand{\version}{final}|
\end{center}
%
which can be uncommented to produce the final version of this child document.

%%%%%%%%%%%%%%%%%%%%%%%%%%%%%%%%%%%%%%%%%%%%%%%%%%%%%%%%%%%%%%%%%%%%%%%%%%%%%%%%
\subsection{Forwarding}
\label{sec:forward}

Different versions of the main or child documents
using compilation flags as described in \secref{sec:flags}
can be (permanently) stored in different files
for convenient compilation, viewing and distribution.
To this end, the package defines a command
to pass on compilation to a different file:

%%%%%%%%%%%%%%%%%%%%%%%%%%%%%%%%%%%%%%%%
\DescribeMacro{\childdocforward}
The command |\childdocforward| redirects processing to
another source file:
%
\begin{center}
\begin{tabular}{l}
|\input{childdoc.def}|\\
|\childdocforward[|\textit{main}|]{|\textit{dest}|}|\\
\end{tabular}
\end{center}
%
The argument \textit{dest} is the destination file
(without extension).
It should be the main file or one of the child files.
Note that further \textsf{childdoc} directives
such as |\childdocof| and |\childdocforward|
in the indicated file will be processed in this form.
The optional argument \textit{main}
passes on directly to the main file \textit{main}
while pretending to compile the child \textit{dest}.
This form behaves as if \textit{dest}
issues |\childdocof{|\textit{main}|}| right away,
and no further \textsf{childdoc} directives will be processed.

%%%%%%%%%%%%%%%%%%%%%%%%%%%%%%%%%%%%%%%%
\DescribeMacro{\...prefix}
In the alternative form |\childdocforwardprefix|,
%
\begin{center}
\begin{tabular}{l}
|\input{childdoc.def}|\\
|\childdocforwardprefix[|\textit{main}|]{|\textit{prefix}|}{|\textit{dest}|}|
\end{tabular}
\end{center}
%
the destination file is determined by a pattern
depending on the current file:
To make this work, the current file must be called
`{\textit{prefix}\hspace{0.2em}\textit{suffix}}'
with \textit{prefix} matching precisely the argument.
Processing is then passed on to the file
`{\textit{dest}\hspace{0.2em}\textit{suffix}}'.
Surely, the same effect is achieved by
directly specifying the
argument `{\textit{dest}\hspace{0.2em}\textit{suffix}}'
in the first form.
However, that requires to set up a different file
for each child. With the alternative form of the command
all these files can have exactly the same content
which simplifies setting them up and maintaining them.

For example, the following file |draft.tex|
with a compilation flag |\version| as described in \secref{sec:flags}
compiles the main document as a draft:
%
\begin{center}
\begin{tabular}{l}
|\def\version{draft}|\\
|\input{childdoc.def}|\\
|\childdocforward{|\textit{main}|}|
\end{tabular}
\end{center}
%
Likewise, the following files |final|\textit{nn}|.tex|
compile the final version of the child document
|child|\textit{nn}|.tex|:
%
\begin{center}
\begin{tabular}{l}
|\def\version{final}|\\
|\input{childdoc.def}|\\
|\childdocforwardprefix{final}{child}|
\end{tabular}
\end{center}
%

Note that when several versions of a main file and/or of each child file
are to be generated, it may be convenient to set up a |Makefile| or
shell script to automatise the process.

%%%%%%%%%%%%%%%%%%%%%%%%%%%%%%%%%%%%%%%%%%%%%%%%%%%%%%%%%%%%%%%%%%%%%%%%%%%%%%%%
\subsection{Command Line Processing}
\label{sec:commandline}

The effect of redirection files can also be achieved by invoking
the \LaTeX{} compiler with a more elaborate command line.
Most conveniently this should be done as part
of a shell script or a |Makefile|.

When using \textsf{childdoc} in the main file, the following
command lines effectively perform a redirection
(note that depending on the shell being used,
backslashes may have to be doubled: `|\|' $\to$ `|\\|'):
%
\begin{center}
|... -jobname "|\textit{target}|" |\\|"|[\textit{flags}]%
|\input{childdoc.def}\childdocforward[|\textit{main}|]{|\textit{dest}|}"|
\end{center}
%
Here \textit{target} is the name of the output file,
\textit{main} is the name of the main file
and \textit{dest} is the name of the main or child file to be processed
(all filenames without extensions).
The optional argument \textit{main} can be omitted
if \textit{main} matches \textit{dest}.
Optionally, compilation \textit{flags} can be defined via |\def| commands.
This command line makes the \TeX{} engine believe
it is compiling the file \textit{target}
whose content is specified as the latter parameter.
The provided code then forwards the processing to
\textit{main} or \textit{dest} as described in \secref{sec:forward}.

%%%%%%%%%%%%%%%%%%%%%%%%%%%%%%%%%%%%%%%%%%%%%%%%%%%%%%%%%%%%%%%%%%%%%%%%%%%%%%%%
\subsection{Include by Input}
\label{sec:input}

Including child documents by |\include| has some restrictions by design.
Most notably, the content of a child document always occupies
its own set of pages; pages cannot be shared between child documents.
Usually, this behaviour makes perfect sense
because each child document contain an essential part of the document.
However, in some situations it may be desirable to compose
a document from a collection of parts
without having mandatory page breaks between then.
For this case, the package
provides a mechanism to include parts
by |\input| which can also be processed individually.
However, by construction this mechanism
requires manual handling of the content to be output.

%%%%%%%%%%%%%%%%%%%%%%%%%%%%%%%%%%%%%%%%
\DescribeMacro{\ifchilddocmanual}
The main file should be prepared as usual, see \secref{sec:include}.
However, the document body must make a distinction
between processing of an individual part and of the main document, e.g.:
%
\begin{center}
\begin{tabular}{l}
|\ifchilddocmanual|\\
|\input{\childdocname}|\\
|\||else|\\
\textit{document body with }|\input{|\textit{part}|}|\\
|\||fi|
\end{tabular}
\end{center}
%
The conditional |\ifchilddocmanual| is true whenever
a part to be included by |\input| is being compiled,
and the name of the part is stored in |\childdocname|.

%%%%%%%%%%%%%%%%%%%%%%%%%%%%%%%%%%%%%%%%
\DescribeMacro{\childdocby}
Each part to be included by |\input| should start with:
%
\begin{center}
\begin{tabular}{l}
|\input{childdoc.def}|\\
|\childdocby{|\textit{main}|}|\\
\end{tabular}
\end{center}
%
The directive |\childdocby| is similar to |\childdocof|
described in \secref{sec:include},
but the subsequent selection of content must be done manually.
To that end, both |\ifchilddoc| and |\ifchilddocmanual|
will be true upon processing of a part,
and the name of the part is stored in |\childdocname|.
Note that |\jobname| will be set to the filename of the current part
so that each part receives an individual |.aux| file
that does not interfere with the |.aux| file(s) of the main document.
This behaviour can be altered by the alternative form
|\childdocby[*]{|\textit{main}|}| (with a non-empty optional argument)
which uses the |.aux| file of the main document
by setting |\jobname| to \textit{main}.

%%%%%%%%%%%%%%%%%%%%%%%%%%%%%%%%%%%%%%%%%%%%%%%%%%%%%%%%%%%%%%%%%%%%%%%%%%%%%%%%
\subsection{Driver Development}
\label{sec:driver}

The \textsf{childdoc} mechanism can also be use for the development
of definition files such as \LaTeX{} styles or classes.
This case differs from the above setup with multiple parts
included by |\include| in that no |\includeonly| should be invoked.
This can be achieved by starting the include file
(before |\ProvidesPackage|) with:
%
\begin{center}
\begin{tabular}{l}
|\input{childdoc.def}|\\
|\childdocforward{|\textit{main}|}|\\
\end{tabular}
\end{center}
%
or alternatively with:
%
\begin{center}
\begin{tabular}{l}
|\input{childdoc.def}|\\
|\childdocby{|\textit{main}|}|\\
\end{tabular}
\end{center}
%
Both forms have slightly different effects as described above.
The main file is prepared as usual, see \secref{sec:include}.

%%%%%%%%%%%%%%%%%%%%%%%%%%%%%%%%%%%%%%%%%%%%%%%%%%%%%%%%%%%%%%%%%%%%%%%%%%%%%%%%
\subsection{Legacy Detection}
\label{sec:detection}

The directive |\childdocmain| in the main file can detect
whether the complete document or merely a child is to be compiled
even without using the directive |\childdocof|.
This method is deprecated because it is less robust
and there is no compelling reason to use it;
it is merely provided for backward compatibility
and it may be removed in future versions.

If the detection mechanism is to be used,
it is mandatory to correctly specify
the filename of the main file as the argument of |\childdocmain|:
%
\begin{center}
\begin{tabular}{l}
|\input{childdoc.def}|\\
|\childdocmain{|\textit{main}|}|\\
\end{tabular}
\end{center}
%
If |\jobname| does not match the argument \textit{main} of |\childdocmain|,
it is assumed that |\jobname| points to the child file to be compiled.
When using |\childdocmain| with the main file specified as argument,
it suffices to start a child file
with just |\input{|\textit{main}|}|
without loading of the package and using |\childdocof|.
If instead all processing is done
with the appropriate \textsf{childdoc} directives,
the argument of \textit{main} of |\childdocmain| can be empty.

An alternative version of the command line processing described
in \secref{sec:commandline} using the detection mechanism reads:
%
\begin{center}
|... -jobname "|\textit{target}|" "|[\textit{flags}]%
[|\def\jobname{|\textit{dest}|}|]|\input{|\textit{main}|}"|
\end{center}

%%%%%%%%%%%%%%%%%%%%%%%%%%%%%%%%%%%%%%%%%%%%%%%%%%%%%%%%%%%%%%%%%%%%%%%%%%%%%%%%
\subsection{Manual Code}
\label{sec:manual}

In case one cannot be certain whether the definitions file |childdoc.def|
is installed on the target \TeX{} distribution
and one prefers not to ship it,
it is conceivable to paste a few relevant commands into the sources.

To that end, drop all statements |\input{childdoc.def}|
and perform the replacements as outlined below.
Instead of |\childdocmain{|\textit{main}|}| add the following code
to the top of the main file:
%
\begin{center}
\begin{tabular}{l}
|\||ifdefined\childdocname\endinput\||fi\newif\ifchilddoc|\\
|\edef\childdocname{\scantokens\expandafter{\jobname\noexpand}}|\\
|\def\childdocmain{|\textit{main}|}\||ifx\childdocmain\childdocname\||else|\\
|\childdoctrue\includeonly{\childdocname}\let\jobname\childdocmain\||fi|\\
\end{tabular}
\end{center}
%
Instead of |\childdocof{|\textit{main}|}| just include the main file
at the top of each child file:
%
\begin{center}
|\input{|\textit{main}|}|
\end{center}
%
A simple redirection |\childdocforward{|\textit{dest}|}| is achieved by:
%
\begin{center}
|\def\jobname{|\textit{dest}|}\input{\jobname}|
\end{center}
%
The redirection with prefix
|\childdocforwardprefix[|\textit{prefix}|]{|\textit{dest}|}|
is accomplished by:
%
\begin{center}
\begin{tabular}{l}
|{\edef\jobname{\scantokens\expandafter{\jobname\noexpand}}|\\
|\def\redirectjob |\textit{prefix}|#1~~~{\gdef\jobname{|\textit{dest}|#1}}|\\
|\expandafter\redirectjob\jobname~~~}\input{\jobname}|
\end{tabular}
\end{center}

In an alternative approach,
child documents can be compiled by a specific command line
without additional code or specific definitions:
%
\begin{center}
|... -jobname "|\textit{target}|" "|[\textit{flags}]%
|\includeonly{|\textit{dest}|}\input{|\textit{main}|}"|
\end{center}
%

%%%%%%%%%%%%%%%%%%%%%%%%%%%%%%%%%%%%%%%%%%%%%%%%%%%%%%%%%%%%%%%%%%%%%%%%%%%%%%%%
%%%%%%%%%%%%%%%%%%%%%%%%%%%%%%%%%%%%%%%%%%%%%%%%%%%%%%%%%%%%%%%%%%%%%%%%%%%%%%%%
\section{Information}

%%%%%%%%%%%%%%%%%%%%%%%%%%%%%%%%%%%%%%%%%%%%%%%%%%%%%%%%%%%%%%%%%%%%%%%%%%%%%%%%
\subsection{Copyright}

Copyright \copyright{} 2017--2018 Niklas Beisert

This work may be distributed and/or modified under the
conditions of the \LaTeX{} Project Public License, either version 1.3
of this license or (at your option) any later version.
The latest version of this license is in
  \url{http://www.latex-project.org/lppl.txt}
and version 1.3 or later is part of all distributions of \LaTeX{}
version 2005/12/01 or later.

This work has the LPPL maintenance status `maintained'.

The Current Maintainer of this work is Niklas Beisert.

This work consists of the files |README.txt|, |childdoc.ins| and |childdoc.dtx|
as well as the derived files |childdoc.def|, |cdocsamp.tex|
with |cdocsch1.tex|, |cdocsch2.tex|, |cdocspt3.tex|, |cdocspt4.tex|,
|cdocsdrf.tex|, |cdocsfn1.tex|, |cdocsfn2.tex|
as well as |childdoc.pdf|.

%%%%%%%%%%%%%%%%%%%%%%%%%%%%%%%%%%%%%%%%%%%%%%%%%%%%%%%%%%%%%%%%%%%%%%%%%%%%%%%%
\subsection{Files and Installation}

The package consists of the files:
%
\begin{center}
\begin{tabular}{ll}
    |README.txt|   & readme file \\
    |childdoc.ins| & installation file \\
    |childdoc.dtx| & source file \\
    |childdoc.def| & definition file \\
    |cdocsamp.tex| & sample main file \\
    |cdocsch1.tex| & sample include file \\
    |cdocsch2.tex| & sample include file \\
    |cdocspt3.tex| & sample part file \\
    |cdocspt4.tex| & sample part file \\
    |cdocsdrf.tex| & sample redirection file \\
    |cdocsfn1.tex| & sample redirection file \\
    |cdocsfn2.tex| & sample redirection file \\
    |childdoc.pdf| & manual
\end{tabular}
\end{center}
%
The distribution consists of the files
|README.txt|, |childdoc.ins| and |childdoc.dtx|.
%
\begin{itemize}
\item
Run (pdf)\LaTeX{} on |childdoc.dtx|
to compile the manual |childdoc.pdf| (this file).
\item
Run \LaTeX{} on |childdoc.ins| to create the definitions file |childdoc.def|
and the sample |cdocsamp.tex| with include files
|cdocsch1.tex|, |cdocsch2.tex|, |cdocspt3.tex|, |cdocspt4.tex|,
|cdocsdrf.tex|, |cdocsfn1.tex|, |cdocsfn2.tex|.
Then copy the file |childdoc.def| to an appropriate directory of your \LaTeX{}
distribution, e.g.\ \textit{texmf-root}|/tex/latex/childdoc|.
\end{itemize}

%%%%%%%%%%%%%%%%%%%%%%%%%%%%%%%%%%%%%%%%%%%%%%%%%%%%%%%%%%%%%%%%%%%%%%%%%%%%%%%%
\subsection{Related CTAN Packages}

There are several other packages which offer a similar functionality:
%
\begin{itemize}
\item
The packages
\href{http://ctan.org/pkg/docmute}{\textsf{docmute}},
\href{http://ctan.org/pkg/includex}{\textsf{includex}} and
\href{http://ctan.org/pkg/standalone}{\textsf{standalone}}
provide commands to include only the document body of
a child file thus allowing both files to be compiled individually.
\item
The packages \href{http://ctan.org/pkg/subdocs}{\textsf{subdocs}}
and \href{http://ctan.org/pkg/subfiles}{\textsf{subfiles}}
provide structures in which the main and child documents can be
encapsulated and allowing them to be compiled individually.
The inclusion mechanism is different from the conventional |\include|.
\item
The package \href{http://ctan.org/pkg/combine}{\textsf{combine}}
is an elaborate solution to combine several documents into one.
\end{itemize}
%
See also the CTAN topic \href{http://ctan.org/topic/subdocs}{\textsf{subdocs}}
for further related packages.
The present package differs from the above solutions in that
a document structure constructed with the conventional |\include| mechanism
just needs two extra commands at the top of every file
such that all constituent files can be compiled individually.

%%%%%%%%%%%%%%%%%%%%%%%%%%%%%%%%%%%%%%%%%%%%%%%%%%%%%%%%%%%%%%%%%%%%%%%%%%%%%%%%
%\subsection{Feature Suggestions}
%
%The following is a list of features which may be useful for future
%versions of this package:
%%
%\begin{itemize}
%\item
%\ldots
%\end{itemize}

%%%%%%%%%%%%%%%%%%%%%%%%%%%%%%%%%%%%%%%%%%%%%%%%%%%%%%%%%%%%%%%%%%%%%%%%%%%%%%%%
\subsection{Revision History}

%%%%%%%%%%%%%%%%%%%%%%%%%%%%%%%%%%%%%%%%
\paragraph{v2.0:} 2018/12/30

\begin{itemize}
\item
immediate forward processing
\item
added |\childdocby| mechanism
\item
manual restructured
\end{itemize}

%%%%%%%%%%%%%%%%%%%%%%%%%%%%%%%%%%%%%%%%
\paragraph{v1.6:} 2018/01/17

\begin{itemize}
\item
application for development of include files
\item
corrections to manual
\end{itemize}

%%%%%%%%%%%%%%%%%%%%%%%%%%%%%%%%%%%%%%%%
\paragraph{v1.5:} 2017/05/21

\begin{itemize}
\item
more complete structuring introduced
\item
|\childdocof| introduced
\item
|\childdoc| renamed to |\childdocmain|
\item
|\childredirect| renamed to |\childdocforward| and |\childdocforwardprefix|
and functionality expanded
\end{itemize}

%%%%%%%%%%%%%%%%%%%%%%%%%%%%%%%%%%%%%%%%
\paragraph{v1.0:} 2017/04/27

\begin{itemize}
\item
manual and install package
\item
first version published on CTAN
\end{itemize}

%%%%%%%%%%%%%%%%%%%%%%%%%%%%%%%%%%%%%%%%
\paragraph{v0.6:} 2017/04/26

\begin{itemize}
\item
redirection mechanism added
\end{itemize}

%%%%%%%%%%%%%%%%%%%%%%%%%%%%%%%%%%%%%%%%
\paragraph{v0.5:} 2017/04/26

\begin{itemize}
\item
functionality in definition file
\end{itemize}


%%%%%%%%%%%%%%%%%%%%%%%%%%%%%%%%%%%%%%%%%%%%%%%%%%%%%%%%%%%%%%%%%%%%%%%%%%%%%%%%
%%%%%%%%%%%%%%%%%%%%%%%%%%%%%%%%%%%%%%%%%%%%%%%%%%%%%%%%%%%%%%%%%%%%%%%%%%%%%%%%
%%%%%%%%%%%%%%%%%%%%%%%%%%%%%%%%%%%%%%%%%%%%%%%%%%%%%%%%%%%%%%%%%%%%%%%%%%%%%%%%
\appendix

\settowidth\MacroIndent{\rmfamily\scriptsize 000\ }

 \DocInput{childdoc.dtx}

\end{document}
%</driver>
% \fi
%
% %%%%%%%%%%%%%%%%%%%%%%%%%%%%%%%%%%%%%%%%%%%%%%%%%%%%%%%%%%%%%%%%%%%%%%%%%%%%%%
% %%%%%%%%%%%%%%%%%%%%%%%%%%%%%%%%%%%%%%%%%%%%%%%%%%%%%%%%%%%%%%%%%%%%%%%%%%%%%%
% \section{Sample}
%\iffalse
%<*samplemain>
%\fi
%
% The following presents a sample document
% with two chapters, two parts, a title page,
% a compile flag as well as three forwarding files to set the flag.
% It consists of eight |.tex| files:
% \begin{center}
% \begin{tabular}{ll}
% |cdocsamp.tex|&main file\\
% |cdocsch1.tex|&include file for chapter 1\\
% |cdocsch2.tex|&include file for chapter 2\\
% |cdocspt3.tex|&include file for part 3\\
% |cdocspt4.tex|&include file for part 4\\
% |cdocsdrf.tex|&forwarding file for main file in draft mode\\
% |cdocsfi1.tex|&forwarding file for final version of chapter 1\\
% |cdocsfi2.tex|&forwarding file for final version of chapter 2\\
% \end{tabular}
% \end{center}
% Each of the eight files can be compiled directly by the \LaTeX{} compiler.
%
% %%%%%%%%%%%%%%%%%%%%%%%%%%%%%%%%%%%%%%
% \paragraph{Main File.}
%
% The main file is called |cdocsamp.tex|.
%
% Load the \textsf{childdoc} definitions and
% declare the filename for the main document:
%    \begin{macrocode}
\input{childdoc.def}
\childdocmain{}
%    \end{macrocode}

% Optional override for |\version| flag:
%    \begin{macrocode}
%%\ifchilddoc\else\providecommand{\version}{draft}\fi
%    \end{macrocode}

% Define the default values for the |\version| flag
% (|final| for the main file and |draft| for childs):
%    \begin{macrocode}
\ifchilddoc
\providecommand{\version}{draft}
\else
\providecommand{\version}{final}
\fi
%    \end{macrocode}

% Load the standard document class:
%    \begin{macrocode}
\documentclass[12pt]{article}
%    \end{macrocode}

% Start the document body:
%    \begin{macrocode}
\begin{document}
%    \end{macrocode}

% Declare a title page.
% Print title, part of document being processed and version flag:
%    \begin{macrocode}
\addtocounter{page}{-1}
\begin{center}
{\LARGE\bfseries{}childdoc example\par}
\vspace{1cm}
\ifchilddoc
\ifchilddocmanual part\else chapter\fi:
`\childdocname' of `\childdocjob'\par
\else
main document: `\childdocjob'\par
\fi
version: \version\par
\end{center}
\newpage
%    \end{macrocode}

% Manually include selected file,
% otherwise process as usual:
%    \begin{macrocode}
\ifchilddocmanual
\section*{part `\childdocname'}
\input{\childdocname}
\else
%    \end{macrocode}

% Include the two chapters:
%    \begin{macrocode}
\include{cdocsch1}
\include{cdocsch2}
%    \end{macrocode}

% Include the two parts unless only chapters should be displayed:
%    \begin{macrocode}
\ifchilddoc\else
\section{part three}
\input{cdocspt3}
\section{part four}
\input{cdocspt4}
\fi
%    \end{macrocode}

% Process as usual until here:
%    \begin{macrocode}
\fi
%    \end{macrocode}

% End of document body:
%    \begin{macrocode}
\end{document}
%    \end{macrocode}
%\iffalse
%</samplemain>
%\fi
%
% %%%%%%%%%%%%%%%%%%%%%%%%%%%%%%%%%%%%%%
% \paragraph{Chapter Include Files.}
%
% The include files are called |cdocsch1.tex| and |cdocsch2.tex|.
%
%\iffalse
%<*samplechap1|samplechap2>
%\fi

% Optional override for |\version| flag:
%    \begin{macrocode}
%%\providecommand{\version}{final}
%    \end{macrocode}

% Include the main document:
%    \begin{macrocode}
\input{childdoc.def}
\childdocof{cdocsamp}
%    \end{macrocode}

%\iffalse
%</samplechap1|samplechap2>
%\fi
%
%\iffalse
%<*samplechap1>
%\fi
% Some text for chapter 1:
%    \begin{macrocode}
\section{one}
some text in chapter one
%    \end{macrocode}

%\iffalse
%</samplechap1>
%\fi
% Some text for chapter 2:
%\iffalse
%<*samplechap2>
%\fi
%    \begin{macrocode}
\section{two}
more text in chapter two
%    \end{macrocode}

%\iffalse
%</samplechap2>
%\fi
%
% %%%%%%%%%%%%%%%%%%%%%%%%%%%%%%%%%%%%%%
% \paragraph{Part Include Files.}
%
% The include files are called |cdocspt3.tex| and |cdocspt4.tex|.
%
%\iffalse
%<*samplepart3|samplepart4>
%\fi

% Optional override for |\version| flag:
%    \begin{macrocode}
%%\providecommand{\version}{final}
%    \end{macrocode}

% Include the main document:
%    \begin{macrocode}
\input{childdoc.def}
\childdocby{cdocsamp}
%    \end{macrocode}

%\iffalse
%</samplepart3|samplepart4>
%\fi
%
%\iffalse
%<*samplepart3>
%\fi
% Some text for part 3:
%    \begin{macrocode}
some text in part three
%    \end{macrocode}

%\iffalse
%</samplepart3>
%\fi
% Some text for part 4:
%\iffalse
%<*samplepart4>
%\fi
%    \begin{macrocode}
more text in part four
%    \end{macrocode}

%\iffalse
%</samplepart4>
%\fi
%
% %%%%%%%%%%%%%%%%%%%%%%%%%%%%%%%%%%%%%%
% \paragraph{Forwarding for a Complete Draft.}
%
% The following forwarding file |cdocsdrf.tex|
% compiles the main document in draft mode:
%\iffalse
%<*sampledraft>
%\fi
%    \begin{macrocode}
\def\version{draft}
\input{childdoc.def}
\childdocforward{cdocsamp}
%    \end{macrocode}

%\iffalse
%</sampledraft>
%\fi
%
% %%%%%%%%%%%%%%%%%%%%%%%%%%%%%%%%%%%%%%
% \paragraph{Forwarding for Final Version of the Chapters.}
%
% The following forwarding files |cdocsfn1.tex| and |cdocsfn2.tex|
% (with identical content)
% compile the final versions of the child documents
% |cdocsch1.tex| and |cdocsch2.tex|, respectively:
%\iffalse
%<*samplefinal>
%\fi
%    \begin{macrocode}
\def\version{final}
\input{childdoc.def}
\childdocforwardprefix[cdocsamp]{cdocsfn}{cdocsch}
%    \end{macrocode}

%\iffalse
%</samplefinal>
%\fi
%
% %%%%%%%%%%%%%%%%%%%%%%%%%%%%%%%%%%%%%%
% \paragraph{Command Line Processing.}
%
% The following three command lines generate the output files
% |cdocscld|, |cdocscl1| and |cdocscl2|
% which should be identical to
% |cdocsdrf|, |cdocsch1| and |cdocsfn2|, respectively:
% \begin{center}
% \begin{tabular}{l}
% |latex -jobname cdocscld \|\\
% |  "\def\version{draft}\input{childdoc.def}\childdocforward{cdocsamp}"|\\
% |latex -jobname cdocscl1 \|\\
% |  "\input{childdoc.def}\childdocforward[cdocsamp]{cdocsch1}"|\\
% |latex -jobname cdocscl2 \|\\
% |  "\def\version{final}\input{childdoc.def}\childdocforward{cdocsch2}"|
% \end{tabular}
% \end{center}
% Note that the trailing backslash on each first line
% merely continues the input to the second line
% (for convenient cut ant paste).
% Furthermore, the command |latex| can be replaced by any
% of its alternative versions such as |pdflatex|.
%
% %%%%%%%%%%%%%%%%%%%%%%%%%%%%%%%%%%%%%%%%%%%%%%%%%%%%%%%%%%%%%%%%%%%%%%%%%%%%%%
% %%%%%%%%%%%%%%%%%%%%%%%%%%%%%%%%%%%%%%%%%%%%%%%%%%%%%%%%%%%%%%%%%%%%%%%%%%%%%%
% \section{Implementation}
%\iffalse
%<*package>
%\fi
%
% This section describes the definitions file |childdoc.def|.

% The definitions cannot be loaded using |\usepackage| or |\RequirePackage|
% which has a mechanism to prevent loading a style file more than once.
% When loading the definitions by means of |\input|
% multiple instances have to be prevented manually:
%\iffalse
%This code needs to be before the `\ProvidesFile' directive
%which is defined at the beginning of this file.
%Therefore it is also placed there and commented out here.
%</package>
%<*discard>
%\fi
%    \begin{macrocode}
\ifdefined\childdocmain\endinput\fi
%    \end{macrocode}
%\iffalse
%</discard>
%<*package>
%\fi
%
% \macro{\ifchilddoc}
% \macro{\ifchilddocmanual}
% The conditional |\ifchilddoc| tells whether a
% child (true) or main (false) document is being compiled.
% The conditional |\ifchilddocmanual| tells whether
% the |\includeonly| mechanism is used (false) or
% the selection of child files must be performed manually (true).
% The definitions initialise to false:
%    \begin{macrocode}
\newif\ifchilddoc
\newif\ifchilddocmanual
%    \end{macrocode}

% \macro{\childdocname}
% \macro{\childdocjob}
% The macro |\childdocname| stores the name of the main document
% to be compiled. The macro |\childdocjob| stores the name of
% the document on which the \LaTeX{} compiler was originally invoked.
% The content of |\jobname| cannot be compared
% to filenames specified in the source due to different catcodes.
% The following code rescans |\jobname|, stores the result
% in |\childdocname| and saves a copy in |\childdocjob|:
%    \begin{macrocode}
\edef\childdocname{\scantokens\expandafter{\jobname\noexpand}}
\let\childdocjob\childdocname
%    \end{macrocode}

% \macro{\childdocdisable}
% The macro |\childdocdisable| prevents the main file
% from being processed more than once.
% At this stage, the main document command |\childdocmain|
% is assumed to be called once again where it should do nothing.
% Any subsequent call to it should prevent
% a secondary processing of the main document
% It overwrites the forwarding commands
% |\childdocof| and |\childdocforward|
% with empty macros to prevent further inclusions of the main document:
%    \begin{macrocode}
\newcommand{\childdocdisable}
{
  \renewcommand{\childdocmain}[1]{\renewcommand{\childdocmain}[1]{\endinput}}
  \renewcommand{\childdocof}[1]{}
  \renewcommand{\childdocby}[2][]{}
  \renewcommand{\childdocforward}[2][]{}
  \renewcommand{\childdocdisable}{}
}
%    \end{macrocode}

% \macro{\childdocmain}
% The macro |\childdocmain| is to be called at the top of the main file
% with nothing or the main filename (without extension) as argument.
% First, it breaks loops.
% If the argument is not empty and does not match |\childdocname|
% (which is set by the first inclusion of |childdoc.def|),
% |\ifchilddoc| is set to true, |\includeonly| is applied to the child file
% and |\jobname| is set to the main file
% (for proper handling of |.aux| files):
%    \begin{macrocode}
\newcommand{\childdocmain}[1]
{
  \childdocdisable\childdocmain{}
  \if?#1?\else
    \begingroup
      \def\childdoctmp{#1}
      \ifx\childdoctmp\childdocname
        \def\childdoctmp{}
      \else
        \def\childdoctmp
        {
          \childdoctrue
          \includeonly{\childdocname}
          \def\childdocjob{#1}
          \def\jobname{#1}
        }
      \fi
      \expandafter
    \endgroup
    \childdoctmp
  \fi
}
%    \end{macrocode}

% \macro{\childdocof}
% The command |\childdocof| redirects
% compilation to the main file |#1|.
%    \begin{macrocode}
\newcommand{\childdocof}[1]
{
  \childdocdisable
  \childdoctrue
  \includeonly{\childdocname}
  \def\jobname{#1}
  \def\childdocjob{#1}
  \input{#1}
}
%    \end{macrocode}

% \macro{\childdocby}
% The command |\childdocby| ....
%    \begin{macrocode}
\newcommand{\childdocby}[2][]
{
  \childdocdisable
  \childdoctrue
  \childdocmanualtrue
  \if?#1?\else
    \def\jobname{#2}
  \fi
  \def\childdocjob{#2}
  \input{#2}
  \endinput
}
%    \end{macrocode}

% \macro{\childdocforward}
% The command |\childdocforward| redirects
% compilation to the main file or
% (if the optional argument is given) a child file.
% Parameters are set as if the main file
% or a child file starting with |\childdocof| was compiled.
% Then compilation is handed over to the main file:
%    \begin{macrocode}
\newcommand{\childdocforward}[2][]
{
  \begingroup
    \if?#1?
      \def\childdoctmp
      {
        \def\childdocname{#2}
        \def\childdocjob{#2}
        \def\jobname{#2}
        \input{#2}
        \endinput
      }
    \else
      \def\childdoctmp
      {
        \childdocdisable
        \def\childdocname{#2}
        \childdoctrue
        \includeonly{#2}
        \def\childdocjob{#1}
        \def\jobname{#1}
        \input{#1}
        \endinput
      }
    \fi
    \expandafter
  \endgroup
  \childdoctmp
}
%    \end{macrocode}

% \macro{\childdocforwardprefix}
% The command |\childdocforwardprefix| redirects
% compilation to the main or a child file by means of a pattern.
% The prefix |#1| in the current filename is replaced by |#2|
% and the suffix of the current filename is kept
% (it is assumed that the filename does not contain the substring `|~~~|'
% which is used as a delimiter).
% Compilation is handed over to the new file by |\childdocforward|:
%    \begin{macrocode}
\newcommand{\childdocforwardprefix}[3][]
{
  \begingroup
    \def\childdocextract #2##1~~~{\def\childdoctmp{\childdocforward[#1]{#3##1}}}
    \expandafter\childdocextract\childdocname~~~
    \expandafter
  \endgroup
  \childdoctmp
}
%    \end{macrocode}

% \macro{\childdoc}
% The deprecated macro |\childdoc| is a legacy version of |\childdocmain|:
%    \begin{macrocode}
\newcommand{\childdoc}{\childdocmain}
%    \end{macrocode}

% \macro{\childdocredirect}
% The deprecated macro |\childdocredirect| is a legacy version
% of |\childdocforward| and |\childdocforwardprefix|:
%    \begin{macrocode}
\newcommand{\childdocredirect}[2][]
{
  \begingroup
    \if?#1?
      \def\childdoctmp{\childdocforward{#2}}
    \else
      \def\childdoctmp{\childdocforwardprefix{#1}{#2}}
    \fi
    \expandafter
  \endgroup
  \childdoctmp
}
%    \end{macrocode}

%\iffalse
%</package>
%\fi
%
\endinput
\childdocforward{cdocsamp}"|\\
% |latex -jobname cdocscl1 \|\\
% |  "% \iffalse
%
% childdoc.dtx Copyright (C) 2017-2018 Niklas Beisert
%
% This work may be distributed and/or modified under the
% conditions of the LaTeX Project Public License, either version 1.3
% of this license or (at your option) any later version.
% The latest version of this license is in
%   http://www.latex-project.org/lppl.txt
% and version 1.3 or later is part of all distributions of LaTeX
% version 2005/12/01 or later.
%
% This work has the LPPL maintenance status `maintained'.
%
% The Current Maintainer of this work is Niklas Beisert.
%
% This work consists of the files childdoc.dtx and childdoc.ins
% and the derived files childdoc.def and cdocsamp.tex with
% cdocsch1.tex, cdocsch2.tex, cdocsdrf.tex, cdocsfn1.tex, cdocsfn2.tex.
%
%<package>\ifdefined\childdocmain\endinput\fi
%<package>\ProvidesFile{childdoc.def}[2018/12/30 v2.0 child document driver]
%<samplemain>\ProvidesFile{cdocsamp.tex}[2018/12/30 v2.0 sample for childdoc]
%<*driver>
%\ProvidesFile{childdoc.drv}[2018/12/30 v2.0 childdoc reference manual file]
\PassOptionsToClass{10pt,a4paper}{article}
\documentclass{ltxdoc}

\usepackage[margin=35mm]{geometry}
\usepackage{hyperref}
\usepackage{hyperxmp}
\usepackage[usenames]{color}

\hypersetup{colorlinks=true}
\hypersetup{pdfstartview=FitH}
\hypersetup{pdfpagemode=UseNone}
\hypersetup{pdfsource={}}
\hypersetup{pdflang={en-UK}}
\hypersetup{pdfcopyright={Copyright 2017-2018 Niklas Beisert.
  This work may be distributed and/or modified under the
  conditions of the LaTeX Project Public License, either version 1.3
  of this license or (at your option) any later version.}}
\hypersetup{pdflicenseurl={http://www.latex-project.org/lppl.txt}}
\hypersetup{pdfcontactaddress={ETH Zurich, ITP, HIT K,
  Wolfgang-Pauli-Strasse 27}}
\hypersetup{pdfcontactpostcode={8093}}
\hypersetup{pdfcontactcity={Zurich}}
\hypersetup{pdfcontactcountry={Switzerland}}
\hypersetup{pdfcontactemail={nbeisert@itp.phys.ethz.ch}}
\hypersetup{pdfcontacturl={http://people.phys.ethz.ch/\xmptilde nbeisert/}}

\newcommand{\secref}[1]{\hyperref[#1]{section \ref*{#1}}}

\parskip1ex
\parindent0pt
\let\olditemize\itemize
\def\itemize{\olditemize\parskip0pt}

\begin{document}

\title{The \textsf{childdoc} Package}
\hypersetup{pdftitle={The childdoc Package}}
\author{Niklas Beisert\\[2ex]
  Institut f\"ur Theoretische Physik\\
  Eidgen\"ossische Technische Hochschule Z\"urich\\
  Wolfgang-Pauli-Strasse 27, 8093 Z\"urich, Switzerland\\[1ex]
  \href{mailto:nbeisert@itp.phys.ethz.ch}
  {\texttt{nbeisert@itp.phys.ethz.ch}}}
\hypersetup{pdfauthor={Niklas Beisert}}
\hypersetup{pdfsubject={Manual for the LaTeX2e Package childdoc}}
\date{30 December 2018, \textsf{v2.0}}
\maketitle

\begin{abstract}\noindent
\textsf{childdoc} is a \LaTeXe{} package
that enables the direct compilation
of document sections included by |\include|
to individual files.
\end{abstract}

\begingroup
\parskip0ex
\tableofcontents
\endgroup

%%%%%%%%%%%%%%%%%%%%%%%%%%%%%%%%%%%%%%%%%%%%%%%%%%%%%%%%%%%%%%%%%%%%%%%%%%%%%%%%
%%%%%%%%%%%%%%%%%%%%%%%%%%%%%%%%%%%%%%%%%%%%%%%%%%%%%%%%%%%%%%%%%%%%%%%%%%%%%%%%
\section{Introduction}

\LaTeX{} provides a mechanism to structure a large document (such as a book)
into a main file and several child files (containing the chapters)
using the |\include| command.
This mechanism is beneficial for documents
which span hundreds of pages in order to
make the source file(s) more manageable.
Moreover, compilation can be restricted to
selected child files by means of the |\includeonly| command.
The latter feature can be used to reduce the compilation time while editing
(this was significantly more useful in the earlier days of \LaTeX{})
or to generate a smaller document which is easier to navigate.
Another application of |\includeonly| is to generate
documents consisting of selected parts of the complete document.

However, there are a few drawbacks of the plain |\include| mechanism:
\begin{itemize}
\item
The child files cannot be compiled on their own,
they can only be compiled via the main file.
A naive editing environment
(such as a text editor with an option
to have the current file processed by \LaTeX)
may require one to switch to the main file before compiling;
attempting to compile the child file produces errors.
\item
The main file must be modified (each time)
to adjust the |\includeonly| command
to the present needs. This easily leaves the main file in a messy state.
\item
The generated document will always carry the filename
of the main document. This is inconvenient if
several child files are to be compiled and
to be kept for distribution.
\end{itemize}

The present package provides a simple interface
to make child files individually compilable by \LaTeX{}.
Compiling a child file then has the same effect as compiling
the main file with an |\includeonly| command
to select the appropriate child.
Moreover the generated document will carry the name of the child
rather than the main file.
This resolves all three above issues.

This feature is meant to make the editing of books,
thesis documents and lecture notes somewhat more convenient.
However, the package can also be used efficiently for
composing a series of documents (such as exercise sheets)
which are typically distributed individually.
It then assists the author in generating the individual documents
(potentially in different versions)
as well as a document containing the collected series.
Another application is in developing style files
or other kinds of included material
where compilation of the style file could redirect
to a sample or test file.

%%%%%%%%%%%%%%%%%%%%%%%%%%%%%%%%%%%%%%%%%%%%%%%%%%%%%%%%%%%%%%%%%%%%%%%%%%%%%%%%
%%%%%%%%%%%%%%%%%%%%%%%%%%%%%%%%%%%%%%%%%%%%%%%%%%%%%%%%%%%%%%%%%%%%%%%%%%%%%%%%
\section{Usage}

First of all, the package \textsf{childdoc} is \emph{not} a standard
\LaTeXe{} |.sty| style file! Therefore it needs to be invoked in
a non-standard way.

%%%%%%%%%%%%%%%%%%%%%%%%%%%%%%%%%%%%%%%%%%%%%%%%%%%%%%%%%%%%%%%%%%%%%%%%%%%%%%%%
\subsection{Included Files}
\label{sec:include}

%%%%%%%%%%%%%%%%%%%%%%%%%%%%%%%%%%%%%%%%
\DescribeMacro{\childdocmain}
To use the package, add the commands
\begin{center}
\begin{tabular}{l}
|\input{childdoc.def}|\\
|\childdocmain{}|\\
\end{tabular}
\end{center}
at the very top of the main \LaTeX{} file,
in particular \emph{before} the |\documentclass| statement!
The argument of |\childdocmain| should be left empty
(but it must be present).

%%%%%%%%%%%%%%%%%%%%%%%%%%%%%%%%%%%%%%%%
\DescribeMacro{\childdocof}
Furthermore, add the commands
\begin{center}
\begin{tabular}{l}
|\input{childdoc.def}|\\
|\childdocof{|\textit{main}|}|\\
\end{tabular}
\end{center}
at the top of every child file \textit{child}
which is included by |\include{|\textit{child}|}|
from within the main file
(or at least for those files to be compiled individually).
The argument \textit{main} must be the filename of the main file.

There are a couple of
considerations in setting up the main and child documents:

%%%%%%%%%%%%%%%%%%%%%%%%%%%%%%%%%%%%%%%%
\paragraph{Restrictions.}

Please note the following restrictions:
\begin{itemize}
\item
|\childdocmain| must be called with one argument \textit{main}
to ensure compatibility with earlier version of the package.
It must either be empty (|\childdocmain{}|)
or precisely match the filename of the main file in which it is specified.
See \secref{sec:detection} for further information.
\item
The filename \textit{main} must be specified without the |.tex| extension.
\item
The filename \textit{main} is case sensitive
(even in case-insensitive file systems)
due to internal string comparison.
\item
The argument \textit{main} should be fully expanded, it cannot be a macro.
\item
Subdirectories and special characters should be avoided in filenames.
\item
The command |\childdocmain{|\textit{main}|}| must be followed by a whitespace.
It should not be followed immediately by another command
or by a comment mark `|%|'.
This is because the \TeX{} parser reads the token immediately following
the argument of |\childdocmain| and puts it
at the beginning of every child section;
however, a white\-space is ignored.
\end{itemize}

%%%%%%%%%%%%%%%%%%%%%%%%%%%%%%%%%%%%%%%%
\paragraph{Content of Main File.}

It is advisable to place all content in the child files included by |\include|.
Any output contained in the main file will appear in all child documents
unless suppressed manually;
it cannot be suppressed automatically by the |\includeonly| directive
and thus should normally be avoided.
A method to include some content in the main file
by means of conditional processing is described in \secref{sec:conditional}.

%%%%%%%%%%%%%%%%%%%%%%%%%%%%%%%%%%%%%%%%
\paragraph{Page Numbering.}

When only a part of the document is compiled,
the appropriate numbering of pages
(as well as other status parameters)
is determined from the |.aux| files.
The latter contain information from previous passes.
However this information needs to propagate through
all intermediate child documents.
Therefore the page numbering in child documents may well
be inconsistent until the complete document is compiled at least once.

A useful (if unconventional) way to always ensure a consistent
page numbering is to restart the numbering in each child document
and denote the pages by `\textit{child}|.|\textit{page}'
where \textit{child} represents the chapter/section number of the child file.
This can be achieved by the command
|\numberwithin{page}{|\textit{child}|}|
of the \textsf{amsmath} package
where \textit{child} can be |chapter| or |section|
depending on the chosen structuring.
Alternatively, one can modify the macro |\thepage| appropriately
and reset the counter |page| at the start of each child file.

%%%%%%%%%%%%%%%%%%%%%%%%%%%%%%%%%%%%%%%%%%%%%%%%%%%%%%%%%%%%%%%%%%%%%%%%%%%%%%%%
\subsection{Conditional Processing}
\label{sec:conditional}

The package provides a mechanism to compile different versions
of a document. To customise the versions further some conditional processing
can come in handy to distinguish which version is being compiled.
The package provides two macros to describe the compilation context:

%%%%%%%%%%%%%%%%%%%%%%%%%%%%%%%%%%%%%%%%
\DescribeMacro{\ifchilddoc}
The conditional |\ifchilddoc| distinguishes between the compilation of
child documents and the main document:
%
\begin{center}
|\ifchilddoc |\textit{child-code}| |[|\||else |\textit{main-code}]| \||fi|
\end{center}

%%%%%%%%%%%%%%%%%%%%%%%%%%%%%%%%%%%%%%%%
\DescribeMacro{\childdocname}
\DescribeMacro{\childdocjob}
The macro |\childdocname| contains the filename (without extension)
of the main or child file being processed.
Note that |\childdocjob| will always contain the name of the main file.

%%%%%%%%%%%%%%%%%%%%%%%%%%%%%%%%%%%%%%%%
\paragraph{Title Page.}

Conditional processing can be used to include a title or banner page
in the main document when proper precautions are taken.
Importantly, the code in the main file should ensure that the page counter
(as well as other status parameters which are stored in the |.aux| files)
takes the same value after the conditional processing.
Otherwise the page numbers may take divergent values
depending on which part is compiled.

For example, a title page could be declared by:
%
\begin{center}
\begin{tabular}{l}
|\ifchilddoc\||else|\\
|\addtocounter{page}{-1}|\\
\textit{code for title page}\\
|\newpage|\\
|\||fi|
\end{tabular}
\end{center}
%
A banner page for the child documents can be generated by:
%
\begin{center}
\begin{tabular}{l}
|\ifchilddoc|\\
|\addtocounter{page}{-1}|\\
\textit{code for banner page}\\
|\newpage|\\
|\||fi|
\end{tabular}
\end{center}
%
Here one could write a message such as:
\begin{center}
|This is the part \childdocname{} of \childdocjob{}.|
\end{center}

%%%%%%%%%%%%%%%%%%%%%%%%%%%%%%%%%%%%%%%%%%%%%%%%%%%%%%%%%%%%%%%%%%%%%%%%%%%%%%%%
\subsection{Flags}
\label{sec:flags}

The package makes it easy to generate different versions
of the main or child documents.
To this end compilation flags can be defined
and assigned different default values.
They will be particularly useful in conjunction
with the forwarding mechanism described in \secref{sec:forward}.

For example, it may be useful to have a flag |\version|
which can be set to |draft| or |final|.
The document source will contain some conditional code
depending on the value of |\version|.
Suppose further, the flag should default to |final| for the main file
and to |draft| for child files
which is a natural assignment for editing the document.
This is achieved by placing the following code
in the preamble of the main document
(below the |\childdocmain| directive):
%
\begin{center}
\begin{tabular}{l}
|\ifchilddoc|\\
|\providecommand{\version}{draft}|\\
|\||else|\\
|\providecommand{\version}{final}|\\
|\||fi|
\end{tabular}
\end{center}
%
The definition by |\providecommand| makes sure
that previous definitions are not overwritten.
Further statements |\providecommand{\version}{...}|
can thus be added before the above code to override it.

For the main file, one might add a line
(between |\childdocmain| and the above block)
%
\begin{center}
|%\ifchilddoc\||else\providecommand{\version}{draft}\||fi|
\end{center}
%
which can be uncommented to produce a draft version.
Likewise one can add a line to the very top of a child file
(above the |\childdocof{|\textit{main}|}| directive)
%
\begin{center}
|%\providecommand{\version}{final}|
\end{center}
%
which can be uncommented to produce the final version of this child document.

%%%%%%%%%%%%%%%%%%%%%%%%%%%%%%%%%%%%%%%%%%%%%%%%%%%%%%%%%%%%%%%%%%%%%%%%%%%%%%%%
\subsection{Forwarding}
\label{sec:forward}

Different versions of the main or child documents
using compilation flags as described in \secref{sec:flags}
can be (permanently) stored in different files
for convenient compilation, viewing and distribution.
To this end, the package defines a command
to pass on compilation to a different file:

%%%%%%%%%%%%%%%%%%%%%%%%%%%%%%%%%%%%%%%%
\DescribeMacro{\childdocforward}
The command |\childdocforward| redirects processing to
another source file:
%
\begin{center}
\begin{tabular}{l}
|\input{childdoc.def}|\\
|\childdocforward[|\textit{main}|]{|\textit{dest}|}|\\
\end{tabular}
\end{center}
%
The argument \textit{dest} is the destination file
(without extension).
It should be the main file or one of the child files.
Note that further \textsf{childdoc} directives
such as |\childdocof| and |\childdocforward|
in the indicated file will be processed in this form.
The optional argument \textit{main}
passes on directly to the main file \textit{main}
while pretending to compile the child \textit{dest}.
This form behaves as if \textit{dest}
issues |\childdocof{|\textit{main}|}| right away,
and no further \textsf{childdoc} directives will be processed.

%%%%%%%%%%%%%%%%%%%%%%%%%%%%%%%%%%%%%%%%
\DescribeMacro{\...prefix}
In the alternative form |\childdocforwardprefix|,
%
\begin{center}
\begin{tabular}{l}
|\input{childdoc.def}|\\
|\childdocforwardprefix[|\textit{main}|]{|\textit{prefix}|}{|\textit{dest}|}|
\end{tabular}
\end{center}
%
the destination file is determined by a pattern
depending on the current file:
To make this work, the current file must be called
`{\textit{prefix}\hspace{0.2em}\textit{suffix}}'
with \textit{prefix} matching precisely the argument.
Processing is then passed on to the file
`{\textit{dest}\hspace{0.2em}\textit{suffix}}'.
Surely, the same effect is achieved by
directly specifying the
argument `{\textit{dest}\hspace{0.2em}\textit{suffix}}'
in the first form.
However, that requires to set up a different file
for each child. With the alternative form of the command
all these files can have exactly the same content
which simplifies setting them up and maintaining them.

For example, the following file |draft.tex|
with a compilation flag |\version| as described in \secref{sec:flags}
compiles the main document as a draft:
%
\begin{center}
\begin{tabular}{l}
|\def\version{draft}|\\
|\input{childdoc.def}|\\
|\childdocforward{|\textit{main}|}|
\end{tabular}
\end{center}
%
Likewise, the following files |final|\textit{nn}|.tex|
compile the final version of the child document
|child|\textit{nn}|.tex|:
%
\begin{center}
\begin{tabular}{l}
|\def\version{final}|\\
|\input{childdoc.def}|\\
|\childdocforwardprefix{final}{child}|
\end{tabular}
\end{center}
%

Note that when several versions of a main file and/or of each child file
are to be generated, it may be convenient to set up a |Makefile| or
shell script to automatise the process.

%%%%%%%%%%%%%%%%%%%%%%%%%%%%%%%%%%%%%%%%%%%%%%%%%%%%%%%%%%%%%%%%%%%%%%%%%%%%%%%%
\subsection{Command Line Processing}
\label{sec:commandline}

The effect of redirection files can also be achieved by invoking
the \LaTeX{} compiler with a more elaborate command line.
Most conveniently this should be done as part
of a shell script or a |Makefile|.

When using \textsf{childdoc} in the main file, the following
command lines effectively perform a redirection
(note that depending on the shell being used,
backslashes may have to be doubled: `|\|' $\to$ `|\\|'):
%
\begin{center}
|... -jobname "|\textit{target}|" |\\|"|[\textit{flags}]%
|\input{childdoc.def}\childdocforward[|\textit{main}|]{|\textit{dest}|}"|
\end{center}
%
Here \textit{target} is the name of the output file,
\textit{main} is the name of the main file
and \textit{dest} is the name of the main or child file to be processed
(all filenames without extensions).
The optional argument \textit{main} can be omitted
if \textit{main} matches \textit{dest}.
Optionally, compilation \textit{flags} can be defined via |\def| commands.
This command line makes the \TeX{} engine believe
it is compiling the file \textit{target}
whose content is specified as the latter parameter.
The provided code then forwards the processing to
\textit{main} or \textit{dest} as described in \secref{sec:forward}.

%%%%%%%%%%%%%%%%%%%%%%%%%%%%%%%%%%%%%%%%%%%%%%%%%%%%%%%%%%%%%%%%%%%%%%%%%%%%%%%%
\subsection{Include by Input}
\label{sec:input}

Including child documents by |\include| has some restrictions by design.
Most notably, the content of a child document always occupies
its own set of pages; pages cannot be shared between child documents.
Usually, this behaviour makes perfect sense
because each child document contain an essential part of the document.
However, in some situations it may be desirable to compose
a document from a collection of parts
without having mandatory page breaks between then.
For this case, the package
provides a mechanism to include parts
by |\input| which can also be processed individually.
However, by construction this mechanism
requires manual handling of the content to be output.

%%%%%%%%%%%%%%%%%%%%%%%%%%%%%%%%%%%%%%%%
\DescribeMacro{\ifchilddocmanual}
The main file should be prepared as usual, see \secref{sec:include}.
However, the document body must make a distinction
between processing of an individual part and of the main document, e.g.:
%
\begin{center}
\begin{tabular}{l}
|\ifchilddocmanual|\\
|\input{\childdocname}|\\
|\||else|\\
\textit{document body with }|\input{|\textit{part}|}|\\
|\||fi|
\end{tabular}
\end{center}
%
The conditional |\ifchilddocmanual| is true whenever
a part to be included by |\input| is being compiled,
and the name of the part is stored in |\childdocname|.

%%%%%%%%%%%%%%%%%%%%%%%%%%%%%%%%%%%%%%%%
\DescribeMacro{\childdocby}
Each part to be included by |\input| should start with:
%
\begin{center}
\begin{tabular}{l}
|\input{childdoc.def}|\\
|\childdocby{|\textit{main}|}|\\
\end{tabular}
\end{center}
%
The directive |\childdocby| is similar to |\childdocof|
described in \secref{sec:include},
but the subsequent selection of content must be done manually.
To that end, both |\ifchilddoc| and |\ifchilddocmanual|
will be true upon processing of a part,
and the name of the part is stored in |\childdocname|.
Note that |\jobname| will be set to the filename of the current part
so that each part receives an individual |.aux| file
that does not interfere with the |.aux| file(s) of the main document.
This behaviour can be altered by the alternative form
|\childdocby[*]{|\textit{main}|}| (with a non-empty optional argument)
which uses the |.aux| file of the main document
by setting |\jobname| to \textit{main}.

%%%%%%%%%%%%%%%%%%%%%%%%%%%%%%%%%%%%%%%%%%%%%%%%%%%%%%%%%%%%%%%%%%%%%%%%%%%%%%%%
\subsection{Driver Development}
\label{sec:driver}

The \textsf{childdoc} mechanism can also be use for the development
of definition files such as \LaTeX{} styles or classes.
This case differs from the above setup with multiple parts
included by |\include| in that no |\includeonly| should be invoked.
This can be achieved by starting the include file
(before |\ProvidesPackage|) with:
%
\begin{center}
\begin{tabular}{l}
|\input{childdoc.def}|\\
|\childdocforward{|\textit{main}|}|\\
\end{tabular}
\end{center}
%
or alternatively with:
%
\begin{center}
\begin{tabular}{l}
|\input{childdoc.def}|\\
|\childdocby{|\textit{main}|}|\\
\end{tabular}
\end{center}
%
Both forms have slightly different effects as described above.
The main file is prepared as usual, see \secref{sec:include}.

%%%%%%%%%%%%%%%%%%%%%%%%%%%%%%%%%%%%%%%%%%%%%%%%%%%%%%%%%%%%%%%%%%%%%%%%%%%%%%%%
\subsection{Legacy Detection}
\label{sec:detection}

The directive |\childdocmain| in the main file can detect
whether the complete document or merely a child is to be compiled
even without using the directive |\childdocof|.
This method is deprecated because it is less robust
and there is no compelling reason to use it;
it is merely provided for backward compatibility
and it may be removed in future versions.

If the detection mechanism is to be used,
it is mandatory to correctly specify
the filename of the main file as the argument of |\childdocmain|:
%
\begin{center}
\begin{tabular}{l}
|\input{childdoc.def}|\\
|\childdocmain{|\textit{main}|}|\\
\end{tabular}
\end{center}
%
If |\jobname| does not match the argument \textit{main} of |\childdocmain|,
it is assumed that |\jobname| points to the child file to be compiled.
When using |\childdocmain| with the main file specified as argument,
it suffices to start a child file
with just |\input{|\textit{main}|}|
without loading of the package and using |\childdocof|.
If instead all processing is done
with the appropriate \textsf{childdoc} directives,
the argument of \textit{main} of |\childdocmain| can be empty.

An alternative version of the command line processing described
in \secref{sec:commandline} using the detection mechanism reads:
%
\begin{center}
|... -jobname "|\textit{target}|" "|[\textit{flags}]%
[|\def\jobname{|\textit{dest}|}|]|\input{|\textit{main}|}"|
\end{center}

%%%%%%%%%%%%%%%%%%%%%%%%%%%%%%%%%%%%%%%%%%%%%%%%%%%%%%%%%%%%%%%%%%%%%%%%%%%%%%%%
\subsection{Manual Code}
\label{sec:manual}

In case one cannot be certain whether the definitions file |childdoc.def|
is installed on the target \TeX{} distribution
and one prefers not to ship it,
it is conceivable to paste a few relevant commands into the sources.

To that end, drop all statements |\input{childdoc.def}|
and perform the replacements as outlined below.
Instead of |\childdocmain{|\textit{main}|}| add the following code
to the top of the main file:
%
\begin{center}
\begin{tabular}{l}
|\||ifdefined\childdocname\endinput\||fi\newif\ifchilddoc|\\
|\edef\childdocname{\scantokens\expandafter{\jobname\noexpand}}|\\
|\def\childdocmain{|\textit{main}|}\||ifx\childdocmain\childdocname\||else|\\
|\childdoctrue\includeonly{\childdocname}\let\jobname\childdocmain\||fi|\\
\end{tabular}
\end{center}
%
Instead of |\childdocof{|\textit{main}|}| just include the main file
at the top of each child file:
%
\begin{center}
|\input{|\textit{main}|}|
\end{center}
%
A simple redirection |\childdocforward{|\textit{dest}|}| is achieved by:
%
\begin{center}
|\def\jobname{|\textit{dest}|}\input{\jobname}|
\end{center}
%
The redirection with prefix
|\childdocforwardprefix[|\textit{prefix}|]{|\textit{dest}|}|
is accomplished by:
%
\begin{center}
\begin{tabular}{l}
|{\edef\jobname{\scantokens\expandafter{\jobname\noexpand}}|\\
|\def\redirectjob |\textit{prefix}|#1~~~{\gdef\jobname{|\textit{dest}|#1}}|\\
|\expandafter\redirectjob\jobname~~~}\input{\jobname}|
\end{tabular}
\end{center}

In an alternative approach,
child documents can be compiled by a specific command line
without additional code or specific definitions:
%
\begin{center}
|... -jobname "|\textit{target}|" "|[\textit{flags}]%
|\includeonly{|\textit{dest}|}\input{|\textit{main}|}"|
\end{center}
%

%%%%%%%%%%%%%%%%%%%%%%%%%%%%%%%%%%%%%%%%%%%%%%%%%%%%%%%%%%%%%%%%%%%%%%%%%%%%%%%%
%%%%%%%%%%%%%%%%%%%%%%%%%%%%%%%%%%%%%%%%%%%%%%%%%%%%%%%%%%%%%%%%%%%%%%%%%%%%%%%%
\section{Information}

%%%%%%%%%%%%%%%%%%%%%%%%%%%%%%%%%%%%%%%%%%%%%%%%%%%%%%%%%%%%%%%%%%%%%%%%%%%%%%%%
\subsection{Copyright}

Copyright \copyright{} 2017--2018 Niklas Beisert

This work may be distributed and/or modified under the
conditions of the \LaTeX{} Project Public License, either version 1.3
of this license or (at your option) any later version.
The latest version of this license is in
  \url{http://www.latex-project.org/lppl.txt}
and version 1.3 or later is part of all distributions of \LaTeX{}
version 2005/12/01 or later.

This work has the LPPL maintenance status `maintained'.

The Current Maintainer of this work is Niklas Beisert.

This work consists of the files |README.txt|, |childdoc.ins| and |childdoc.dtx|
as well as the derived files |childdoc.def|, |cdocsamp.tex|
with |cdocsch1.tex|, |cdocsch2.tex|, |cdocspt3.tex|, |cdocspt4.tex|,
|cdocsdrf.tex|, |cdocsfn1.tex|, |cdocsfn2.tex|
as well as |childdoc.pdf|.

%%%%%%%%%%%%%%%%%%%%%%%%%%%%%%%%%%%%%%%%%%%%%%%%%%%%%%%%%%%%%%%%%%%%%%%%%%%%%%%%
\subsection{Files and Installation}

The package consists of the files:
%
\begin{center}
\begin{tabular}{ll}
    |README.txt|   & readme file \\
    |childdoc.ins| & installation file \\
    |childdoc.dtx| & source file \\
    |childdoc.def| & definition file \\
    |cdocsamp.tex| & sample main file \\
    |cdocsch1.tex| & sample include file \\
    |cdocsch2.tex| & sample include file \\
    |cdocspt3.tex| & sample part file \\
    |cdocspt4.tex| & sample part file \\
    |cdocsdrf.tex| & sample redirection file \\
    |cdocsfn1.tex| & sample redirection file \\
    |cdocsfn2.tex| & sample redirection file \\
    |childdoc.pdf| & manual
\end{tabular}
\end{center}
%
The distribution consists of the files
|README.txt|, |childdoc.ins| and |childdoc.dtx|.
%
\begin{itemize}
\item
Run (pdf)\LaTeX{} on |childdoc.dtx|
to compile the manual |childdoc.pdf| (this file).
\item
Run \LaTeX{} on |childdoc.ins| to create the definitions file |childdoc.def|
and the sample |cdocsamp.tex| with include files
|cdocsch1.tex|, |cdocsch2.tex|, |cdocspt3.tex|, |cdocspt4.tex|,
|cdocsdrf.tex|, |cdocsfn1.tex|, |cdocsfn2.tex|.
Then copy the file |childdoc.def| to an appropriate directory of your \LaTeX{}
distribution, e.g.\ \textit{texmf-root}|/tex/latex/childdoc|.
\end{itemize}

%%%%%%%%%%%%%%%%%%%%%%%%%%%%%%%%%%%%%%%%%%%%%%%%%%%%%%%%%%%%%%%%%%%%%%%%%%%%%%%%
\subsection{Related CTAN Packages}

There are several other packages which offer a similar functionality:
%
\begin{itemize}
\item
The packages
\href{http://ctan.org/pkg/docmute}{\textsf{docmute}},
\href{http://ctan.org/pkg/includex}{\textsf{includex}} and
\href{http://ctan.org/pkg/standalone}{\textsf{standalone}}
provide commands to include only the document body of
a child file thus allowing both files to be compiled individually.
\item
The packages \href{http://ctan.org/pkg/subdocs}{\textsf{subdocs}}
and \href{http://ctan.org/pkg/subfiles}{\textsf{subfiles}}
provide structures in which the main and child documents can be
encapsulated and allowing them to be compiled individually.
The inclusion mechanism is different from the conventional |\include|.
\item
The package \href{http://ctan.org/pkg/combine}{\textsf{combine}}
is an elaborate solution to combine several documents into one.
\end{itemize}
%
See also the CTAN topic \href{http://ctan.org/topic/subdocs}{\textsf{subdocs}}
for further related packages.
The present package differs from the above solutions in that
a document structure constructed with the conventional |\include| mechanism
just needs two extra commands at the top of every file
such that all constituent files can be compiled individually.

%%%%%%%%%%%%%%%%%%%%%%%%%%%%%%%%%%%%%%%%%%%%%%%%%%%%%%%%%%%%%%%%%%%%%%%%%%%%%%%%
%\subsection{Feature Suggestions}
%
%The following is a list of features which may be useful for future
%versions of this package:
%%
%\begin{itemize}
%\item
%\ldots
%\end{itemize}

%%%%%%%%%%%%%%%%%%%%%%%%%%%%%%%%%%%%%%%%%%%%%%%%%%%%%%%%%%%%%%%%%%%%%%%%%%%%%%%%
\subsection{Revision History}

%%%%%%%%%%%%%%%%%%%%%%%%%%%%%%%%%%%%%%%%
\paragraph{v2.0:} 2018/12/30

\begin{itemize}
\item
immediate forward processing
\item
added |\childdocby| mechanism
\item
manual restructured
\end{itemize}

%%%%%%%%%%%%%%%%%%%%%%%%%%%%%%%%%%%%%%%%
\paragraph{v1.6:} 2018/01/17

\begin{itemize}
\item
application for development of include files
\item
corrections to manual
\end{itemize}

%%%%%%%%%%%%%%%%%%%%%%%%%%%%%%%%%%%%%%%%
\paragraph{v1.5:} 2017/05/21

\begin{itemize}
\item
more complete structuring introduced
\item
|\childdocof| introduced
\item
|\childdoc| renamed to |\childdocmain|
\item
|\childredirect| renamed to |\childdocforward| and |\childdocforwardprefix|
and functionality expanded
\end{itemize}

%%%%%%%%%%%%%%%%%%%%%%%%%%%%%%%%%%%%%%%%
\paragraph{v1.0:} 2017/04/27

\begin{itemize}
\item
manual and install package
\item
first version published on CTAN
\end{itemize}

%%%%%%%%%%%%%%%%%%%%%%%%%%%%%%%%%%%%%%%%
\paragraph{v0.6:} 2017/04/26

\begin{itemize}
\item
redirection mechanism added
\end{itemize}

%%%%%%%%%%%%%%%%%%%%%%%%%%%%%%%%%%%%%%%%
\paragraph{v0.5:} 2017/04/26

\begin{itemize}
\item
functionality in definition file
\end{itemize}


%%%%%%%%%%%%%%%%%%%%%%%%%%%%%%%%%%%%%%%%%%%%%%%%%%%%%%%%%%%%%%%%%%%%%%%%%%%%%%%%
%%%%%%%%%%%%%%%%%%%%%%%%%%%%%%%%%%%%%%%%%%%%%%%%%%%%%%%%%%%%%%%%%%%%%%%%%%%%%%%%
%%%%%%%%%%%%%%%%%%%%%%%%%%%%%%%%%%%%%%%%%%%%%%%%%%%%%%%%%%%%%%%%%%%%%%%%%%%%%%%%
\appendix

\settowidth\MacroIndent{\rmfamily\scriptsize 000\ }

 \DocInput{childdoc.dtx}

\end{document}
%</driver>
% \fi
%
% %%%%%%%%%%%%%%%%%%%%%%%%%%%%%%%%%%%%%%%%%%%%%%%%%%%%%%%%%%%%%%%%%%%%%%%%%%%%%%
% %%%%%%%%%%%%%%%%%%%%%%%%%%%%%%%%%%%%%%%%%%%%%%%%%%%%%%%%%%%%%%%%%%%%%%%%%%%%%%
% \section{Sample}
%\iffalse
%<*samplemain>
%\fi
%
% The following presents a sample document
% with two chapters, two parts, a title page,
% a compile flag as well as three forwarding files to set the flag.
% It consists of eight |.tex| files:
% \begin{center}
% \begin{tabular}{ll}
% |cdocsamp.tex|&main file\\
% |cdocsch1.tex|&include file for chapter 1\\
% |cdocsch2.tex|&include file for chapter 2\\
% |cdocspt3.tex|&include file for part 3\\
% |cdocspt4.tex|&include file for part 4\\
% |cdocsdrf.tex|&forwarding file for main file in draft mode\\
% |cdocsfi1.tex|&forwarding file for final version of chapter 1\\
% |cdocsfi2.tex|&forwarding file for final version of chapter 2\\
% \end{tabular}
% \end{center}
% Each of the eight files can be compiled directly by the \LaTeX{} compiler.
%
% %%%%%%%%%%%%%%%%%%%%%%%%%%%%%%%%%%%%%%
% \paragraph{Main File.}
%
% The main file is called |cdocsamp.tex|.
%
% Load the \textsf{childdoc} definitions and
% declare the filename for the main document:
%    \begin{macrocode}
\input{childdoc.def}
\childdocmain{}
%    \end{macrocode}

% Optional override for |\version| flag:
%    \begin{macrocode}
%%\ifchilddoc\else\providecommand{\version}{draft}\fi
%    \end{macrocode}

% Define the default values for the |\version| flag
% (|final| for the main file and |draft| for childs):
%    \begin{macrocode}
\ifchilddoc
\providecommand{\version}{draft}
\else
\providecommand{\version}{final}
\fi
%    \end{macrocode}

% Load the standard document class:
%    \begin{macrocode}
\documentclass[12pt]{article}
%    \end{macrocode}

% Start the document body:
%    \begin{macrocode}
\begin{document}
%    \end{macrocode}

% Declare a title page.
% Print title, part of document being processed and version flag:
%    \begin{macrocode}
\addtocounter{page}{-1}
\begin{center}
{\LARGE\bfseries{}childdoc example\par}
\vspace{1cm}
\ifchilddoc
\ifchilddocmanual part\else chapter\fi:
`\childdocname' of `\childdocjob'\par
\else
main document: `\childdocjob'\par
\fi
version: \version\par
\end{center}
\newpage
%    \end{macrocode}

% Manually include selected file,
% otherwise process as usual:
%    \begin{macrocode}
\ifchilddocmanual
\section*{part `\childdocname'}
\input{\childdocname}
\else
%    \end{macrocode}

% Include the two chapters:
%    \begin{macrocode}
\include{cdocsch1}
\include{cdocsch2}
%    \end{macrocode}

% Include the two parts unless only chapters should be displayed:
%    \begin{macrocode}
\ifchilddoc\else
\section{part three}
\input{cdocspt3}
\section{part four}
\input{cdocspt4}
\fi
%    \end{macrocode}

% Process as usual until here:
%    \begin{macrocode}
\fi
%    \end{macrocode}

% End of document body:
%    \begin{macrocode}
\end{document}
%    \end{macrocode}
%\iffalse
%</samplemain>
%\fi
%
% %%%%%%%%%%%%%%%%%%%%%%%%%%%%%%%%%%%%%%
% \paragraph{Chapter Include Files.}
%
% The include files are called |cdocsch1.tex| and |cdocsch2.tex|.
%
%\iffalse
%<*samplechap1|samplechap2>
%\fi

% Optional override for |\version| flag:
%    \begin{macrocode}
%%\providecommand{\version}{final}
%    \end{macrocode}

% Include the main document:
%    \begin{macrocode}
\input{childdoc.def}
\childdocof{cdocsamp}
%    \end{macrocode}

%\iffalse
%</samplechap1|samplechap2>
%\fi
%
%\iffalse
%<*samplechap1>
%\fi
% Some text for chapter 1:
%    \begin{macrocode}
\section{one}
some text in chapter one
%    \end{macrocode}

%\iffalse
%</samplechap1>
%\fi
% Some text for chapter 2:
%\iffalse
%<*samplechap2>
%\fi
%    \begin{macrocode}
\section{two}
more text in chapter two
%    \end{macrocode}

%\iffalse
%</samplechap2>
%\fi
%
% %%%%%%%%%%%%%%%%%%%%%%%%%%%%%%%%%%%%%%
% \paragraph{Part Include Files.}
%
% The include files are called |cdocspt3.tex| and |cdocspt4.tex|.
%
%\iffalse
%<*samplepart3|samplepart4>
%\fi

% Optional override for |\version| flag:
%    \begin{macrocode}
%%\providecommand{\version}{final}
%    \end{macrocode}

% Include the main document:
%    \begin{macrocode}
\input{childdoc.def}
\childdocby{cdocsamp}
%    \end{macrocode}

%\iffalse
%</samplepart3|samplepart4>
%\fi
%
%\iffalse
%<*samplepart3>
%\fi
% Some text for part 3:
%    \begin{macrocode}
some text in part three
%    \end{macrocode}

%\iffalse
%</samplepart3>
%\fi
% Some text for part 4:
%\iffalse
%<*samplepart4>
%\fi
%    \begin{macrocode}
more text in part four
%    \end{macrocode}

%\iffalse
%</samplepart4>
%\fi
%
% %%%%%%%%%%%%%%%%%%%%%%%%%%%%%%%%%%%%%%
% \paragraph{Forwarding for a Complete Draft.}
%
% The following forwarding file |cdocsdrf.tex|
% compiles the main document in draft mode:
%\iffalse
%<*sampledraft>
%\fi
%    \begin{macrocode}
\def\version{draft}
\input{childdoc.def}
\childdocforward{cdocsamp}
%    \end{macrocode}

%\iffalse
%</sampledraft>
%\fi
%
% %%%%%%%%%%%%%%%%%%%%%%%%%%%%%%%%%%%%%%
% \paragraph{Forwarding for Final Version of the Chapters.}
%
% The following forwarding files |cdocsfn1.tex| and |cdocsfn2.tex|
% (with identical content)
% compile the final versions of the child documents
% |cdocsch1.tex| and |cdocsch2.tex|, respectively:
%\iffalse
%<*samplefinal>
%\fi
%    \begin{macrocode}
\def\version{final}
\input{childdoc.def}
\childdocforwardprefix[cdocsamp]{cdocsfn}{cdocsch}
%    \end{macrocode}

%\iffalse
%</samplefinal>
%\fi
%
% %%%%%%%%%%%%%%%%%%%%%%%%%%%%%%%%%%%%%%
% \paragraph{Command Line Processing.}
%
% The following three command lines generate the output files
% |cdocscld|, |cdocscl1| and |cdocscl2|
% which should be identical to
% |cdocsdrf|, |cdocsch1| and |cdocsfn2|, respectively:
% \begin{center}
% \begin{tabular}{l}
% |latex -jobname cdocscld \|\\
% |  "\def\version{draft}\input{childdoc.def}\childdocforward{cdocsamp}"|\\
% |latex -jobname cdocscl1 \|\\
% |  "\input{childdoc.def}\childdocforward[cdocsamp]{cdocsch1}"|\\
% |latex -jobname cdocscl2 \|\\
% |  "\def\version{final}\input{childdoc.def}\childdocforward{cdocsch2}"|
% \end{tabular}
% \end{center}
% Note that the trailing backslash on each first line
% merely continues the input to the second line
% (for convenient cut ant paste).
% Furthermore, the command |latex| can be replaced by any
% of its alternative versions such as |pdflatex|.
%
% %%%%%%%%%%%%%%%%%%%%%%%%%%%%%%%%%%%%%%%%%%%%%%%%%%%%%%%%%%%%%%%%%%%%%%%%%%%%%%
% %%%%%%%%%%%%%%%%%%%%%%%%%%%%%%%%%%%%%%%%%%%%%%%%%%%%%%%%%%%%%%%%%%%%%%%%%%%%%%
% \section{Implementation}
%\iffalse
%<*package>
%\fi
%
% This section describes the definitions file |childdoc.def|.

% The definitions cannot be loaded using |\usepackage| or |\RequirePackage|
% which has a mechanism to prevent loading a style file more than once.
% When loading the definitions by means of |\input|
% multiple instances have to be prevented manually:
%\iffalse
%This code needs to be before the `\ProvidesFile' directive
%which is defined at the beginning of this file.
%Therefore it is also placed there and commented out here.
%</package>
%<*discard>
%\fi
%    \begin{macrocode}
\ifdefined\childdocmain\endinput\fi
%    \end{macrocode}
%\iffalse
%</discard>
%<*package>
%\fi
%
% \macro{\ifchilddoc}
% \macro{\ifchilddocmanual}
% The conditional |\ifchilddoc| tells whether a
% child (true) or main (false) document is being compiled.
% The conditional |\ifchilddocmanual| tells whether
% the |\includeonly| mechanism is used (false) or
% the selection of child files must be performed manually (true).
% The definitions initialise to false:
%    \begin{macrocode}
\newif\ifchilddoc
\newif\ifchilddocmanual
%    \end{macrocode}

% \macro{\childdocname}
% \macro{\childdocjob}
% The macro |\childdocname| stores the name of the main document
% to be compiled. The macro |\childdocjob| stores the name of
% the document on which the \LaTeX{} compiler was originally invoked.
% The content of |\jobname| cannot be compared
% to filenames specified in the source due to different catcodes.
% The following code rescans |\jobname|, stores the result
% in |\childdocname| and saves a copy in |\childdocjob|:
%    \begin{macrocode}
\edef\childdocname{\scantokens\expandafter{\jobname\noexpand}}
\let\childdocjob\childdocname
%    \end{macrocode}

% \macro{\childdocdisable}
% The macro |\childdocdisable| prevents the main file
% from being processed more than once.
% At this stage, the main document command |\childdocmain|
% is assumed to be called once again where it should do nothing.
% Any subsequent call to it should prevent
% a secondary processing of the main document
% It overwrites the forwarding commands
% |\childdocof| and |\childdocforward|
% with empty macros to prevent further inclusions of the main document:
%    \begin{macrocode}
\newcommand{\childdocdisable}
{
  \renewcommand{\childdocmain}[1]{\renewcommand{\childdocmain}[1]{\endinput}}
  \renewcommand{\childdocof}[1]{}
  \renewcommand{\childdocby}[2][]{}
  \renewcommand{\childdocforward}[2][]{}
  \renewcommand{\childdocdisable}{}
}
%    \end{macrocode}

% \macro{\childdocmain}
% The macro |\childdocmain| is to be called at the top of the main file
% with nothing or the main filename (without extension) as argument.
% First, it breaks loops.
% If the argument is not empty and does not match |\childdocname|
% (which is set by the first inclusion of |childdoc.def|),
% |\ifchilddoc| is set to true, |\includeonly| is applied to the child file
% and |\jobname| is set to the main file
% (for proper handling of |.aux| files):
%    \begin{macrocode}
\newcommand{\childdocmain}[1]
{
  \childdocdisable\childdocmain{}
  \if?#1?\else
    \begingroup
      \def\childdoctmp{#1}
      \ifx\childdoctmp\childdocname
        \def\childdoctmp{}
      \else
        \def\childdoctmp
        {
          \childdoctrue
          \includeonly{\childdocname}
          \def\childdocjob{#1}
          \def\jobname{#1}
        }
      \fi
      \expandafter
    \endgroup
    \childdoctmp
  \fi
}
%    \end{macrocode}

% \macro{\childdocof}
% The command |\childdocof| redirects
% compilation to the main file |#1|.
%    \begin{macrocode}
\newcommand{\childdocof}[1]
{
  \childdocdisable
  \childdoctrue
  \includeonly{\childdocname}
  \def\jobname{#1}
  \def\childdocjob{#1}
  \input{#1}
}
%    \end{macrocode}

% \macro{\childdocby}
% The command |\childdocby| ....
%    \begin{macrocode}
\newcommand{\childdocby}[2][]
{
  \childdocdisable
  \childdoctrue
  \childdocmanualtrue
  \if?#1?\else
    \def\jobname{#2}
  \fi
  \def\childdocjob{#2}
  \input{#2}
  \endinput
}
%    \end{macrocode}

% \macro{\childdocforward}
% The command |\childdocforward| redirects
% compilation to the main file or
% (if the optional argument is given) a child file.
% Parameters are set as if the main file
% or a child file starting with |\childdocof| was compiled.
% Then compilation is handed over to the main file:
%    \begin{macrocode}
\newcommand{\childdocforward}[2][]
{
  \begingroup
    \if?#1?
      \def\childdoctmp
      {
        \def\childdocname{#2}
        \def\childdocjob{#2}
        \def\jobname{#2}
        \input{#2}
        \endinput
      }
    \else
      \def\childdoctmp
      {
        \childdocdisable
        \def\childdocname{#2}
        \childdoctrue
        \includeonly{#2}
        \def\childdocjob{#1}
        \def\jobname{#1}
        \input{#1}
        \endinput
      }
    \fi
    \expandafter
  \endgroup
  \childdoctmp
}
%    \end{macrocode}

% \macro{\childdocforwardprefix}
% The command |\childdocforwardprefix| redirects
% compilation to the main or a child file by means of a pattern.
% The prefix |#1| in the current filename is replaced by |#2|
% and the suffix of the current filename is kept
% (it is assumed that the filename does not contain the substring `|~~~|'
% which is used as a delimiter).
% Compilation is handed over to the new file by |\childdocforward|:
%    \begin{macrocode}
\newcommand{\childdocforwardprefix}[3][]
{
  \begingroup
    \def\childdocextract #2##1~~~{\def\childdoctmp{\childdocforward[#1]{#3##1}}}
    \expandafter\childdocextract\childdocname~~~
    \expandafter
  \endgroup
  \childdoctmp
}
%    \end{macrocode}

% \macro{\childdoc}
% The deprecated macro |\childdoc| is a legacy version of |\childdocmain|:
%    \begin{macrocode}
\newcommand{\childdoc}{\childdocmain}
%    \end{macrocode}

% \macro{\childdocredirect}
% The deprecated macro |\childdocredirect| is a legacy version
% of |\childdocforward| and |\childdocforwardprefix|:
%    \begin{macrocode}
\newcommand{\childdocredirect}[2][]
{
  \begingroup
    \if?#1?
      \def\childdoctmp{\childdocforward{#2}}
    \else
      \def\childdoctmp{\childdocforwardprefix{#1}{#2}}
    \fi
    \expandafter
  \endgroup
  \childdoctmp
}
%    \end{macrocode}

%\iffalse
%</package>
%\fi
%
\endinput
\childdocforward[cdocsamp]{cdocsch1}"|\\
% |latex -jobname cdocscl2 \|\\
% |  "\def\version{final}% \iffalse
%
% childdoc.dtx Copyright (C) 2017-2018 Niklas Beisert
%
% This work may be distributed and/or modified under the
% conditions of the LaTeX Project Public License, either version 1.3
% of this license or (at your option) any later version.
% The latest version of this license is in
%   http://www.latex-project.org/lppl.txt
% and version 1.3 or later is part of all distributions of LaTeX
% version 2005/12/01 or later.
%
% This work has the LPPL maintenance status `maintained'.
%
% The Current Maintainer of this work is Niklas Beisert.
%
% This work consists of the files childdoc.dtx and childdoc.ins
% and the derived files childdoc.def and cdocsamp.tex with
% cdocsch1.tex, cdocsch2.tex, cdocsdrf.tex, cdocsfn1.tex, cdocsfn2.tex.
%
%<package>\ifdefined\childdocmain\endinput\fi
%<package>\ProvidesFile{childdoc.def}[2018/12/30 v2.0 child document driver]
%<samplemain>\ProvidesFile{cdocsamp.tex}[2018/12/30 v2.0 sample for childdoc]
%<*driver>
%\ProvidesFile{childdoc.drv}[2018/12/30 v2.0 childdoc reference manual file]
\PassOptionsToClass{10pt,a4paper}{article}
\documentclass{ltxdoc}

\usepackage[margin=35mm]{geometry}
\usepackage{hyperref}
\usepackage{hyperxmp}
\usepackage[usenames]{color}

\hypersetup{colorlinks=true}
\hypersetup{pdfstartview=FitH}
\hypersetup{pdfpagemode=UseNone}
\hypersetup{pdfsource={}}
\hypersetup{pdflang={en-UK}}
\hypersetup{pdfcopyright={Copyright 2017-2018 Niklas Beisert.
  This work may be distributed and/or modified under the
  conditions of the LaTeX Project Public License, either version 1.3
  of this license or (at your option) any later version.}}
\hypersetup{pdflicenseurl={http://www.latex-project.org/lppl.txt}}
\hypersetup{pdfcontactaddress={ETH Zurich, ITP, HIT K,
  Wolfgang-Pauli-Strasse 27}}
\hypersetup{pdfcontactpostcode={8093}}
\hypersetup{pdfcontactcity={Zurich}}
\hypersetup{pdfcontactcountry={Switzerland}}
\hypersetup{pdfcontactemail={nbeisert@itp.phys.ethz.ch}}
\hypersetup{pdfcontacturl={http://people.phys.ethz.ch/\xmptilde nbeisert/}}

\newcommand{\secref}[1]{\hyperref[#1]{section \ref*{#1}}}

\parskip1ex
\parindent0pt
\let\olditemize\itemize
\def\itemize{\olditemize\parskip0pt}

\begin{document}

\title{The \textsf{childdoc} Package}
\hypersetup{pdftitle={The childdoc Package}}
\author{Niklas Beisert\\[2ex]
  Institut f\"ur Theoretische Physik\\
  Eidgen\"ossische Technische Hochschule Z\"urich\\
  Wolfgang-Pauli-Strasse 27, 8093 Z\"urich, Switzerland\\[1ex]
  \href{mailto:nbeisert@itp.phys.ethz.ch}
  {\texttt{nbeisert@itp.phys.ethz.ch}}}
\hypersetup{pdfauthor={Niklas Beisert}}
\hypersetup{pdfsubject={Manual for the LaTeX2e Package childdoc}}
\date{30 December 2018, \textsf{v2.0}}
\maketitle

\begin{abstract}\noindent
\textsf{childdoc} is a \LaTeXe{} package
that enables the direct compilation
of document sections included by |\include|
to individual files.
\end{abstract}

\begingroup
\parskip0ex
\tableofcontents
\endgroup

%%%%%%%%%%%%%%%%%%%%%%%%%%%%%%%%%%%%%%%%%%%%%%%%%%%%%%%%%%%%%%%%%%%%%%%%%%%%%%%%
%%%%%%%%%%%%%%%%%%%%%%%%%%%%%%%%%%%%%%%%%%%%%%%%%%%%%%%%%%%%%%%%%%%%%%%%%%%%%%%%
\section{Introduction}

\LaTeX{} provides a mechanism to structure a large document (such as a book)
into a main file and several child files (containing the chapters)
using the |\include| command.
This mechanism is beneficial for documents
which span hundreds of pages in order to
make the source file(s) more manageable.
Moreover, compilation can be restricted to
selected child files by means of the |\includeonly| command.
The latter feature can be used to reduce the compilation time while editing
(this was significantly more useful in the earlier days of \LaTeX{})
or to generate a smaller document which is easier to navigate.
Another application of |\includeonly| is to generate
documents consisting of selected parts of the complete document.

However, there are a few drawbacks of the plain |\include| mechanism:
\begin{itemize}
\item
The child files cannot be compiled on their own,
they can only be compiled via the main file.
A naive editing environment
(such as a text editor with an option
to have the current file processed by \LaTeX)
may require one to switch to the main file before compiling;
attempting to compile the child file produces errors.
\item
The main file must be modified (each time)
to adjust the |\includeonly| command
to the present needs. This easily leaves the main file in a messy state.
\item
The generated document will always carry the filename
of the main document. This is inconvenient if
several child files are to be compiled and
to be kept for distribution.
\end{itemize}

The present package provides a simple interface
to make child files individually compilable by \LaTeX{}.
Compiling a child file then has the same effect as compiling
the main file with an |\includeonly| command
to select the appropriate child.
Moreover the generated document will carry the name of the child
rather than the main file.
This resolves all three above issues.

This feature is meant to make the editing of books,
thesis documents and lecture notes somewhat more convenient.
However, the package can also be used efficiently for
composing a series of documents (such as exercise sheets)
which are typically distributed individually.
It then assists the author in generating the individual documents
(potentially in different versions)
as well as a document containing the collected series.
Another application is in developing style files
or other kinds of included material
where compilation of the style file could redirect
to a sample or test file.

%%%%%%%%%%%%%%%%%%%%%%%%%%%%%%%%%%%%%%%%%%%%%%%%%%%%%%%%%%%%%%%%%%%%%%%%%%%%%%%%
%%%%%%%%%%%%%%%%%%%%%%%%%%%%%%%%%%%%%%%%%%%%%%%%%%%%%%%%%%%%%%%%%%%%%%%%%%%%%%%%
\section{Usage}

First of all, the package \textsf{childdoc} is \emph{not} a standard
\LaTeXe{} |.sty| style file! Therefore it needs to be invoked in
a non-standard way.

%%%%%%%%%%%%%%%%%%%%%%%%%%%%%%%%%%%%%%%%%%%%%%%%%%%%%%%%%%%%%%%%%%%%%%%%%%%%%%%%
\subsection{Included Files}
\label{sec:include}

%%%%%%%%%%%%%%%%%%%%%%%%%%%%%%%%%%%%%%%%
\DescribeMacro{\childdocmain}
To use the package, add the commands
\begin{center}
\begin{tabular}{l}
|\input{childdoc.def}|\\
|\childdocmain{}|\\
\end{tabular}
\end{center}
at the very top of the main \LaTeX{} file,
in particular \emph{before} the |\documentclass| statement!
The argument of |\childdocmain| should be left empty
(but it must be present).

%%%%%%%%%%%%%%%%%%%%%%%%%%%%%%%%%%%%%%%%
\DescribeMacro{\childdocof}
Furthermore, add the commands
\begin{center}
\begin{tabular}{l}
|\input{childdoc.def}|\\
|\childdocof{|\textit{main}|}|\\
\end{tabular}
\end{center}
at the top of every child file \textit{child}
which is included by |\include{|\textit{child}|}|
from within the main file
(or at least for those files to be compiled individually).
The argument \textit{main} must be the filename of the main file.

There are a couple of
considerations in setting up the main and child documents:

%%%%%%%%%%%%%%%%%%%%%%%%%%%%%%%%%%%%%%%%
\paragraph{Restrictions.}

Please note the following restrictions:
\begin{itemize}
\item
|\childdocmain| must be called with one argument \textit{main}
to ensure compatibility with earlier version of the package.
It must either be empty (|\childdocmain{}|)
or precisely match the filename of the main file in which it is specified.
See \secref{sec:detection} for further information.
\item
The filename \textit{main} must be specified without the |.tex| extension.
\item
The filename \textit{main} is case sensitive
(even in case-insensitive file systems)
due to internal string comparison.
\item
The argument \textit{main} should be fully expanded, it cannot be a macro.
\item
Subdirectories and special characters should be avoided in filenames.
\item
The command |\childdocmain{|\textit{main}|}| must be followed by a whitespace.
It should not be followed immediately by another command
or by a comment mark `|%|'.
This is because the \TeX{} parser reads the token immediately following
the argument of |\childdocmain| and puts it
at the beginning of every child section;
however, a white\-space is ignored.
\end{itemize}

%%%%%%%%%%%%%%%%%%%%%%%%%%%%%%%%%%%%%%%%
\paragraph{Content of Main File.}

It is advisable to place all content in the child files included by |\include|.
Any output contained in the main file will appear in all child documents
unless suppressed manually;
it cannot be suppressed automatically by the |\includeonly| directive
and thus should normally be avoided.
A method to include some content in the main file
by means of conditional processing is described in \secref{sec:conditional}.

%%%%%%%%%%%%%%%%%%%%%%%%%%%%%%%%%%%%%%%%
\paragraph{Page Numbering.}

When only a part of the document is compiled,
the appropriate numbering of pages
(as well as other status parameters)
is determined from the |.aux| files.
The latter contain information from previous passes.
However this information needs to propagate through
all intermediate child documents.
Therefore the page numbering in child documents may well
be inconsistent until the complete document is compiled at least once.

A useful (if unconventional) way to always ensure a consistent
page numbering is to restart the numbering in each child document
and denote the pages by `\textit{child}|.|\textit{page}'
where \textit{child} represents the chapter/section number of the child file.
This can be achieved by the command
|\numberwithin{page}{|\textit{child}|}|
of the \textsf{amsmath} package
where \textit{child} can be |chapter| or |section|
depending on the chosen structuring.
Alternatively, one can modify the macro |\thepage| appropriately
and reset the counter |page| at the start of each child file.

%%%%%%%%%%%%%%%%%%%%%%%%%%%%%%%%%%%%%%%%%%%%%%%%%%%%%%%%%%%%%%%%%%%%%%%%%%%%%%%%
\subsection{Conditional Processing}
\label{sec:conditional}

The package provides a mechanism to compile different versions
of a document. To customise the versions further some conditional processing
can come in handy to distinguish which version is being compiled.
The package provides two macros to describe the compilation context:

%%%%%%%%%%%%%%%%%%%%%%%%%%%%%%%%%%%%%%%%
\DescribeMacro{\ifchilddoc}
The conditional |\ifchilddoc| distinguishes between the compilation of
child documents and the main document:
%
\begin{center}
|\ifchilddoc |\textit{child-code}| |[|\||else |\textit{main-code}]| \||fi|
\end{center}

%%%%%%%%%%%%%%%%%%%%%%%%%%%%%%%%%%%%%%%%
\DescribeMacro{\childdocname}
\DescribeMacro{\childdocjob}
The macro |\childdocname| contains the filename (without extension)
of the main or child file being processed.
Note that |\childdocjob| will always contain the name of the main file.

%%%%%%%%%%%%%%%%%%%%%%%%%%%%%%%%%%%%%%%%
\paragraph{Title Page.}

Conditional processing can be used to include a title or banner page
in the main document when proper precautions are taken.
Importantly, the code in the main file should ensure that the page counter
(as well as other status parameters which are stored in the |.aux| files)
takes the same value after the conditional processing.
Otherwise the page numbers may take divergent values
depending on which part is compiled.

For example, a title page could be declared by:
%
\begin{center}
\begin{tabular}{l}
|\ifchilddoc\||else|\\
|\addtocounter{page}{-1}|\\
\textit{code for title page}\\
|\newpage|\\
|\||fi|
\end{tabular}
\end{center}
%
A banner page for the child documents can be generated by:
%
\begin{center}
\begin{tabular}{l}
|\ifchilddoc|\\
|\addtocounter{page}{-1}|\\
\textit{code for banner page}\\
|\newpage|\\
|\||fi|
\end{tabular}
\end{center}
%
Here one could write a message such as:
\begin{center}
|This is the part \childdocname{} of \childdocjob{}.|
\end{center}

%%%%%%%%%%%%%%%%%%%%%%%%%%%%%%%%%%%%%%%%%%%%%%%%%%%%%%%%%%%%%%%%%%%%%%%%%%%%%%%%
\subsection{Flags}
\label{sec:flags}

The package makes it easy to generate different versions
of the main or child documents.
To this end compilation flags can be defined
and assigned different default values.
They will be particularly useful in conjunction
with the forwarding mechanism described in \secref{sec:forward}.

For example, it may be useful to have a flag |\version|
which can be set to |draft| or |final|.
The document source will contain some conditional code
depending on the value of |\version|.
Suppose further, the flag should default to |final| for the main file
and to |draft| for child files
which is a natural assignment for editing the document.
This is achieved by placing the following code
in the preamble of the main document
(below the |\childdocmain| directive):
%
\begin{center}
\begin{tabular}{l}
|\ifchilddoc|\\
|\providecommand{\version}{draft}|\\
|\||else|\\
|\providecommand{\version}{final}|\\
|\||fi|
\end{tabular}
\end{center}
%
The definition by |\providecommand| makes sure
that previous definitions are not overwritten.
Further statements |\providecommand{\version}{...}|
can thus be added before the above code to override it.

For the main file, one might add a line
(between |\childdocmain| and the above block)
%
\begin{center}
|%\ifchilddoc\||else\providecommand{\version}{draft}\||fi|
\end{center}
%
which can be uncommented to produce a draft version.
Likewise one can add a line to the very top of a child file
(above the |\childdocof{|\textit{main}|}| directive)
%
\begin{center}
|%\providecommand{\version}{final}|
\end{center}
%
which can be uncommented to produce the final version of this child document.

%%%%%%%%%%%%%%%%%%%%%%%%%%%%%%%%%%%%%%%%%%%%%%%%%%%%%%%%%%%%%%%%%%%%%%%%%%%%%%%%
\subsection{Forwarding}
\label{sec:forward}

Different versions of the main or child documents
using compilation flags as described in \secref{sec:flags}
can be (permanently) stored in different files
for convenient compilation, viewing and distribution.
To this end, the package defines a command
to pass on compilation to a different file:

%%%%%%%%%%%%%%%%%%%%%%%%%%%%%%%%%%%%%%%%
\DescribeMacro{\childdocforward}
The command |\childdocforward| redirects processing to
another source file:
%
\begin{center}
\begin{tabular}{l}
|\input{childdoc.def}|\\
|\childdocforward[|\textit{main}|]{|\textit{dest}|}|\\
\end{tabular}
\end{center}
%
The argument \textit{dest} is the destination file
(without extension).
It should be the main file or one of the child files.
Note that further \textsf{childdoc} directives
such as |\childdocof| and |\childdocforward|
in the indicated file will be processed in this form.
The optional argument \textit{main}
passes on directly to the main file \textit{main}
while pretending to compile the child \textit{dest}.
This form behaves as if \textit{dest}
issues |\childdocof{|\textit{main}|}| right away,
and no further \textsf{childdoc} directives will be processed.

%%%%%%%%%%%%%%%%%%%%%%%%%%%%%%%%%%%%%%%%
\DescribeMacro{\...prefix}
In the alternative form |\childdocforwardprefix|,
%
\begin{center}
\begin{tabular}{l}
|\input{childdoc.def}|\\
|\childdocforwardprefix[|\textit{main}|]{|\textit{prefix}|}{|\textit{dest}|}|
\end{tabular}
\end{center}
%
the destination file is determined by a pattern
depending on the current file:
To make this work, the current file must be called
`{\textit{prefix}\hspace{0.2em}\textit{suffix}}'
with \textit{prefix} matching precisely the argument.
Processing is then passed on to the file
`{\textit{dest}\hspace{0.2em}\textit{suffix}}'.
Surely, the same effect is achieved by
directly specifying the
argument `{\textit{dest}\hspace{0.2em}\textit{suffix}}'
in the first form.
However, that requires to set up a different file
for each child. With the alternative form of the command
all these files can have exactly the same content
which simplifies setting them up and maintaining them.

For example, the following file |draft.tex|
with a compilation flag |\version| as described in \secref{sec:flags}
compiles the main document as a draft:
%
\begin{center}
\begin{tabular}{l}
|\def\version{draft}|\\
|\input{childdoc.def}|\\
|\childdocforward{|\textit{main}|}|
\end{tabular}
\end{center}
%
Likewise, the following files |final|\textit{nn}|.tex|
compile the final version of the child document
|child|\textit{nn}|.tex|:
%
\begin{center}
\begin{tabular}{l}
|\def\version{final}|\\
|\input{childdoc.def}|\\
|\childdocforwardprefix{final}{child}|
\end{tabular}
\end{center}
%

Note that when several versions of a main file and/or of each child file
are to be generated, it may be convenient to set up a |Makefile| or
shell script to automatise the process.

%%%%%%%%%%%%%%%%%%%%%%%%%%%%%%%%%%%%%%%%%%%%%%%%%%%%%%%%%%%%%%%%%%%%%%%%%%%%%%%%
\subsection{Command Line Processing}
\label{sec:commandline}

The effect of redirection files can also be achieved by invoking
the \LaTeX{} compiler with a more elaborate command line.
Most conveniently this should be done as part
of a shell script or a |Makefile|.

When using \textsf{childdoc} in the main file, the following
command lines effectively perform a redirection
(note that depending on the shell being used,
backslashes may have to be doubled: `|\|' $\to$ `|\\|'):
%
\begin{center}
|... -jobname "|\textit{target}|" |\\|"|[\textit{flags}]%
|\input{childdoc.def}\childdocforward[|\textit{main}|]{|\textit{dest}|}"|
\end{center}
%
Here \textit{target} is the name of the output file,
\textit{main} is the name of the main file
and \textit{dest} is the name of the main or child file to be processed
(all filenames without extensions).
The optional argument \textit{main} can be omitted
if \textit{main} matches \textit{dest}.
Optionally, compilation \textit{flags} can be defined via |\def| commands.
This command line makes the \TeX{} engine believe
it is compiling the file \textit{target}
whose content is specified as the latter parameter.
The provided code then forwards the processing to
\textit{main} or \textit{dest} as described in \secref{sec:forward}.

%%%%%%%%%%%%%%%%%%%%%%%%%%%%%%%%%%%%%%%%%%%%%%%%%%%%%%%%%%%%%%%%%%%%%%%%%%%%%%%%
\subsection{Include by Input}
\label{sec:input}

Including child documents by |\include| has some restrictions by design.
Most notably, the content of a child document always occupies
its own set of pages; pages cannot be shared between child documents.
Usually, this behaviour makes perfect sense
because each child document contain an essential part of the document.
However, in some situations it may be desirable to compose
a document from a collection of parts
without having mandatory page breaks between then.
For this case, the package
provides a mechanism to include parts
by |\input| which can also be processed individually.
However, by construction this mechanism
requires manual handling of the content to be output.

%%%%%%%%%%%%%%%%%%%%%%%%%%%%%%%%%%%%%%%%
\DescribeMacro{\ifchilddocmanual}
The main file should be prepared as usual, see \secref{sec:include}.
However, the document body must make a distinction
between processing of an individual part and of the main document, e.g.:
%
\begin{center}
\begin{tabular}{l}
|\ifchilddocmanual|\\
|\input{\childdocname}|\\
|\||else|\\
\textit{document body with }|\input{|\textit{part}|}|\\
|\||fi|
\end{tabular}
\end{center}
%
The conditional |\ifchilddocmanual| is true whenever
a part to be included by |\input| is being compiled,
and the name of the part is stored in |\childdocname|.

%%%%%%%%%%%%%%%%%%%%%%%%%%%%%%%%%%%%%%%%
\DescribeMacro{\childdocby}
Each part to be included by |\input| should start with:
%
\begin{center}
\begin{tabular}{l}
|\input{childdoc.def}|\\
|\childdocby{|\textit{main}|}|\\
\end{tabular}
\end{center}
%
The directive |\childdocby| is similar to |\childdocof|
described in \secref{sec:include},
but the subsequent selection of content must be done manually.
To that end, both |\ifchilddoc| and |\ifchilddocmanual|
will be true upon processing of a part,
and the name of the part is stored in |\childdocname|.
Note that |\jobname| will be set to the filename of the current part
so that each part receives an individual |.aux| file
that does not interfere with the |.aux| file(s) of the main document.
This behaviour can be altered by the alternative form
|\childdocby[*]{|\textit{main}|}| (with a non-empty optional argument)
which uses the |.aux| file of the main document
by setting |\jobname| to \textit{main}.

%%%%%%%%%%%%%%%%%%%%%%%%%%%%%%%%%%%%%%%%%%%%%%%%%%%%%%%%%%%%%%%%%%%%%%%%%%%%%%%%
\subsection{Driver Development}
\label{sec:driver}

The \textsf{childdoc} mechanism can also be use for the development
of definition files such as \LaTeX{} styles or classes.
This case differs from the above setup with multiple parts
included by |\include| in that no |\includeonly| should be invoked.
This can be achieved by starting the include file
(before |\ProvidesPackage|) with:
%
\begin{center}
\begin{tabular}{l}
|\input{childdoc.def}|\\
|\childdocforward{|\textit{main}|}|\\
\end{tabular}
\end{center}
%
or alternatively with:
%
\begin{center}
\begin{tabular}{l}
|\input{childdoc.def}|\\
|\childdocby{|\textit{main}|}|\\
\end{tabular}
\end{center}
%
Both forms have slightly different effects as described above.
The main file is prepared as usual, see \secref{sec:include}.

%%%%%%%%%%%%%%%%%%%%%%%%%%%%%%%%%%%%%%%%%%%%%%%%%%%%%%%%%%%%%%%%%%%%%%%%%%%%%%%%
\subsection{Legacy Detection}
\label{sec:detection}

The directive |\childdocmain| in the main file can detect
whether the complete document or merely a child is to be compiled
even without using the directive |\childdocof|.
This method is deprecated because it is less robust
and there is no compelling reason to use it;
it is merely provided for backward compatibility
and it may be removed in future versions.

If the detection mechanism is to be used,
it is mandatory to correctly specify
the filename of the main file as the argument of |\childdocmain|:
%
\begin{center}
\begin{tabular}{l}
|\input{childdoc.def}|\\
|\childdocmain{|\textit{main}|}|\\
\end{tabular}
\end{center}
%
If |\jobname| does not match the argument \textit{main} of |\childdocmain|,
it is assumed that |\jobname| points to the child file to be compiled.
When using |\childdocmain| with the main file specified as argument,
it suffices to start a child file
with just |\input{|\textit{main}|}|
without loading of the package and using |\childdocof|.
If instead all processing is done
with the appropriate \textsf{childdoc} directives,
the argument of \textit{main} of |\childdocmain| can be empty.

An alternative version of the command line processing described
in \secref{sec:commandline} using the detection mechanism reads:
%
\begin{center}
|... -jobname "|\textit{target}|" "|[\textit{flags}]%
[|\def\jobname{|\textit{dest}|}|]|\input{|\textit{main}|}"|
\end{center}

%%%%%%%%%%%%%%%%%%%%%%%%%%%%%%%%%%%%%%%%%%%%%%%%%%%%%%%%%%%%%%%%%%%%%%%%%%%%%%%%
\subsection{Manual Code}
\label{sec:manual}

In case one cannot be certain whether the definitions file |childdoc.def|
is installed on the target \TeX{} distribution
and one prefers not to ship it,
it is conceivable to paste a few relevant commands into the sources.

To that end, drop all statements |\input{childdoc.def}|
and perform the replacements as outlined below.
Instead of |\childdocmain{|\textit{main}|}| add the following code
to the top of the main file:
%
\begin{center}
\begin{tabular}{l}
|\||ifdefined\childdocname\endinput\||fi\newif\ifchilddoc|\\
|\edef\childdocname{\scantokens\expandafter{\jobname\noexpand}}|\\
|\def\childdocmain{|\textit{main}|}\||ifx\childdocmain\childdocname\||else|\\
|\childdoctrue\includeonly{\childdocname}\let\jobname\childdocmain\||fi|\\
\end{tabular}
\end{center}
%
Instead of |\childdocof{|\textit{main}|}| just include the main file
at the top of each child file:
%
\begin{center}
|\input{|\textit{main}|}|
\end{center}
%
A simple redirection |\childdocforward{|\textit{dest}|}| is achieved by:
%
\begin{center}
|\def\jobname{|\textit{dest}|}\input{\jobname}|
\end{center}
%
The redirection with prefix
|\childdocforwardprefix[|\textit{prefix}|]{|\textit{dest}|}|
is accomplished by:
%
\begin{center}
\begin{tabular}{l}
|{\edef\jobname{\scantokens\expandafter{\jobname\noexpand}}|\\
|\def\redirectjob |\textit{prefix}|#1~~~{\gdef\jobname{|\textit{dest}|#1}}|\\
|\expandafter\redirectjob\jobname~~~}\input{\jobname}|
\end{tabular}
\end{center}

In an alternative approach,
child documents can be compiled by a specific command line
without additional code or specific definitions:
%
\begin{center}
|... -jobname "|\textit{target}|" "|[\textit{flags}]%
|\includeonly{|\textit{dest}|}\input{|\textit{main}|}"|
\end{center}
%

%%%%%%%%%%%%%%%%%%%%%%%%%%%%%%%%%%%%%%%%%%%%%%%%%%%%%%%%%%%%%%%%%%%%%%%%%%%%%%%%
%%%%%%%%%%%%%%%%%%%%%%%%%%%%%%%%%%%%%%%%%%%%%%%%%%%%%%%%%%%%%%%%%%%%%%%%%%%%%%%%
\section{Information}

%%%%%%%%%%%%%%%%%%%%%%%%%%%%%%%%%%%%%%%%%%%%%%%%%%%%%%%%%%%%%%%%%%%%%%%%%%%%%%%%
\subsection{Copyright}

Copyright \copyright{} 2017--2018 Niklas Beisert

This work may be distributed and/or modified under the
conditions of the \LaTeX{} Project Public License, either version 1.3
of this license or (at your option) any later version.
The latest version of this license is in
  \url{http://www.latex-project.org/lppl.txt}
and version 1.3 or later is part of all distributions of \LaTeX{}
version 2005/12/01 or later.

This work has the LPPL maintenance status `maintained'.

The Current Maintainer of this work is Niklas Beisert.

This work consists of the files |README.txt|, |childdoc.ins| and |childdoc.dtx|
as well as the derived files |childdoc.def|, |cdocsamp.tex|
with |cdocsch1.tex|, |cdocsch2.tex|, |cdocspt3.tex|, |cdocspt4.tex|,
|cdocsdrf.tex|, |cdocsfn1.tex|, |cdocsfn2.tex|
as well as |childdoc.pdf|.

%%%%%%%%%%%%%%%%%%%%%%%%%%%%%%%%%%%%%%%%%%%%%%%%%%%%%%%%%%%%%%%%%%%%%%%%%%%%%%%%
\subsection{Files and Installation}

The package consists of the files:
%
\begin{center}
\begin{tabular}{ll}
    |README.txt|   & readme file \\
    |childdoc.ins| & installation file \\
    |childdoc.dtx| & source file \\
    |childdoc.def| & definition file \\
    |cdocsamp.tex| & sample main file \\
    |cdocsch1.tex| & sample include file \\
    |cdocsch2.tex| & sample include file \\
    |cdocspt3.tex| & sample part file \\
    |cdocspt4.tex| & sample part file \\
    |cdocsdrf.tex| & sample redirection file \\
    |cdocsfn1.tex| & sample redirection file \\
    |cdocsfn2.tex| & sample redirection file \\
    |childdoc.pdf| & manual
\end{tabular}
\end{center}
%
The distribution consists of the files
|README.txt|, |childdoc.ins| and |childdoc.dtx|.
%
\begin{itemize}
\item
Run (pdf)\LaTeX{} on |childdoc.dtx|
to compile the manual |childdoc.pdf| (this file).
\item
Run \LaTeX{} on |childdoc.ins| to create the definitions file |childdoc.def|
and the sample |cdocsamp.tex| with include files
|cdocsch1.tex|, |cdocsch2.tex|, |cdocspt3.tex|, |cdocspt4.tex|,
|cdocsdrf.tex|, |cdocsfn1.tex|, |cdocsfn2.tex|.
Then copy the file |childdoc.def| to an appropriate directory of your \LaTeX{}
distribution, e.g.\ \textit{texmf-root}|/tex/latex/childdoc|.
\end{itemize}

%%%%%%%%%%%%%%%%%%%%%%%%%%%%%%%%%%%%%%%%%%%%%%%%%%%%%%%%%%%%%%%%%%%%%%%%%%%%%%%%
\subsection{Related CTAN Packages}

There are several other packages which offer a similar functionality:
%
\begin{itemize}
\item
The packages
\href{http://ctan.org/pkg/docmute}{\textsf{docmute}},
\href{http://ctan.org/pkg/includex}{\textsf{includex}} and
\href{http://ctan.org/pkg/standalone}{\textsf{standalone}}
provide commands to include only the document body of
a child file thus allowing both files to be compiled individually.
\item
The packages \href{http://ctan.org/pkg/subdocs}{\textsf{subdocs}}
and \href{http://ctan.org/pkg/subfiles}{\textsf{subfiles}}
provide structures in which the main and child documents can be
encapsulated and allowing them to be compiled individually.
The inclusion mechanism is different from the conventional |\include|.
\item
The package \href{http://ctan.org/pkg/combine}{\textsf{combine}}
is an elaborate solution to combine several documents into one.
\end{itemize}
%
See also the CTAN topic \href{http://ctan.org/topic/subdocs}{\textsf{subdocs}}
for further related packages.
The present package differs from the above solutions in that
a document structure constructed with the conventional |\include| mechanism
just needs two extra commands at the top of every file
such that all constituent files can be compiled individually.

%%%%%%%%%%%%%%%%%%%%%%%%%%%%%%%%%%%%%%%%%%%%%%%%%%%%%%%%%%%%%%%%%%%%%%%%%%%%%%%%
%\subsection{Feature Suggestions}
%
%The following is a list of features which may be useful for future
%versions of this package:
%%
%\begin{itemize}
%\item
%\ldots
%\end{itemize}

%%%%%%%%%%%%%%%%%%%%%%%%%%%%%%%%%%%%%%%%%%%%%%%%%%%%%%%%%%%%%%%%%%%%%%%%%%%%%%%%
\subsection{Revision History}

%%%%%%%%%%%%%%%%%%%%%%%%%%%%%%%%%%%%%%%%
\paragraph{v2.0:} 2018/12/30

\begin{itemize}
\item
immediate forward processing
\item
added |\childdocby| mechanism
\item
manual restructured
\end{itemize}

%%%%%%%%%%%%%%%%%%%%%%%%%%%%%%%%%%%%%%%%
\paragraph{v1.6:} 2018/01/17

\begin{itemize}
\item
application for development of include files
\item
corrections to manual
\end{itemize}

%%%%%%%%%%%%%%%%%%%%%%%%%%%%%%%%%%%%%%%%
\paragraph{v1.5:} 2017/05/21

\begin{itemize}
\item
more complete structuring introduced
\item
|\childdocof| introduced
\item
|\childdoc| renamed to |\childdocmain|
\item
|\childredirect| renamed to |\childdocforward| and |\childdocforwardprefix|
and functionality expanded
\end{itemize}

%%%%%%%%%%%%%%%%%%%%%%%%%%%%%%%%%%%%%%%%
\paragraph{v1.0:} 2017/04/27

\begin{itemize}
\item
manual and install package
\item
first version published on CTAN
\end{itemize}

%%%%%%%%%%%%%%%%%%%%%%%%%%%%%%%%%%%%%%%%
\paragraph{v0.6:} 2017/04/26

\begin{itemize}
\item
redirection mechanism added
\end{itemize}

%%%%%%%%%%%%%%%%%%%%%%%%%%%%%%%%%%%%%%%%
\paragraph{v0.5:} 2017/04/26

\begin{itemize}
\item
functionality in definition file
\end{itemize}


%%%%%%%%%%%%%%%%%%%%%%%%%%%%%%%%%%%%%%%%%%%%%%%%%%%%%%%%%%%%%%%%%%%%%%%%%%%%%%%%
%%%%%%%%%%%%%%%%%%%%%%%%%%%%%%%%%%%%%%%%%%%%%%%%%%%%%%%%%%%%%%%%%%%%%%%%%%%%%%%%
%%%%%%%%%%%%%%%%%%%%%%%%%%%%%%%%%%%%%%%%%%%%%%%%%%%%%%%%%%%%%%%%%%%%%%%%%%%%%%%%
\appendix

\settowidth\MacroIndent{\rmfamily\scriptsize 000\ }

 \DocInput{childdoc.dtx}

\end{document}
%</driver>
% \fi
%
% %%%%%%%%%%%%%%%%%%%%%%%%%%%%%%%%%%%%%%%%%%%%%%%%%%%%%%%%%%%%%%%%%%%%%%%%%%%%%%
% %%%%%%%%%%%%%%%%%%%%%%%%%%%%%%%%%%%%%%%%%%%%%%%%%%%%%%%%%%%%%%%%%%%%%%%%%%%%%%
% \section{Sample}
%\iffalse
%<*samplemain>
%\fi
%
% The following presents a sample document
% with two chapters, two parts, a title page,
% a compile flag as well as three forwarding files to set the flag.
% It consists of eight |.tex| files:
% \begin{center}
% \begin{tabular}{ll}
% |cdocsamp.tex|&main file\\
% |cdocsch1.tex|&include file for chapter 1\\
% |cdocsch2.tex|&include file for chapter 2\\
% |cdocspt3.tex|&include file for part 3\\
% |cdocspt4.tex|&include file for part 4\\
% |cdocsdrf.tex|&forwarding file for main file in draft mode\\
% |cdocsfi1.tex|&forwarding file for final version of chapter 1\\
% |cdocsfi2.tex|&forwarding file for final version of chapter 2\\
% \end{tabular}
% \end{center}
% Each of the eight files can be compiled directly by the \LaTeX{} compiler.
%
% %%%%%%%%%%%%%%%%%%%%%%%%%%%%%%%%%%%%%%
% \paragraph{Main File.}
%
% The main file is called |cdocsamp.tex|.
%
% Load the \textsf{childdoc} definitions and
% declare the filename for the main document:
%    \begin{macrocode}
\input{childdoc.def}
\childdocmain{}
%    \end{macrocode}

% Optional override for |\version| flag:
%    \begin{macrocode}
%%\ifchilddoc\else\providecommand{\version}{draft}\fi
%    \end{macrocode}

% Define the default values for the |\version| flag
% (|final| for the main file and |draft| for childs):
%    \begin{macrocode}
\ifchilddoc
\providecommand{\version}{draft}
\else
\providecommand{\version}{final}
\fi
%    \end{macrocode}

% Load the standard document class:
%    \begin{macrocode}
\documentclass[12pt]{article}
%    \end{macrocode}

% Start the document body:
%    \begin{macrocode}
\begin{document}
%    \end{macrocode}

% Declare a title page.
% Print title, part of document being processed and version flag:
%    \begin{macrocode}
\addtocounter{page}{-1}
\begin{center}
{\LARGE\bfseries{}childdoc example\par}
\vspace{1cm}
\ifchilddoc
\ifchilddocmanual part\else chapter\fi:
`\childdocname' of `\childdocjob'\par
\else
main document: `\childdocjob'\par
\fi
version: \version\par
\end{center}
\newpage
%    \end{macrocode}

% Manually include selected file,
% otherwise process as usual:
%    \begin{macrocode}
\ifchilddocmanual
\section*{part `\childdocname'}
\input{\childdocname}
\else
%    \end{macrocode}

% Include the two chapters:
%    \begin{macrocode}
\include{cdocsch1}
\include{cdocsch2}
%    \end{macrocode}

% Include the two parts unless only chapters should be displayed:
%    \begin{macrocode}
\ifchilddoc\else
\section{part three}
\input{cdocspt3}
\section{part four}
\input{cdocspt4}
\fi
%    \end{macrocode}

% Process as usual until here:
%    \begin{macrocode}
\fi
%    \end{macrocode}

% End of document body:
%    \begin{macrocode}
\end{document}
%    \end{macrocode}
%\iffalse
%</samplemain>
%\fi
%
% %%%%%%%%%%%%%%%%%%%%%%%%%%%%%%%%%%%%%%
% \paragraph{Chapter Include Files.}
%
% The include files are called |cdocsch1.tex| and |cdocsch2.tex|.
%
%\iffalse
%<*samplechap1|samplechap2>
%\fi

% Optional override for |\version| flag:
%    \begin{macrocode}
%%\providecommand{\version}{final}
%    \end{macrocode}

% Include the main document:
%    \begin{macrocode}
\input{childdoc.def}
\childdocof{cdocsamp}
%    \end{macrocode}

%\iffalse
%</samplechap1|samplechap2>
%\fi
%
%\iffalse
%<*samplechap1>
%\fi
% Some text for chapter 1:
%    \begin{macrocode}
\section{one}
some text in chapter one
%    \end{macrocode}

%\iffalse
%</samplechap1>
%\fi
% Some text for chapter 2:
%\iffalse
%<*samplechap2>
%\fi
%    \begin{macrocode}
\section{two}
more text in chapter two
%    \end{macrocode}

%\iffalse
%</samplechap2>
%\fi
%
% %%%%%%%%%%%%%%%%%%%%%%%%%%%%%%%%%%%%%%
% \paragraph{Part Include Files.}
%
% The include files are called |cdocspt3.tex| and |cdocspt4.tex|.
%
%\iffalse
%<*samplepart3|samplepart4>
%\fi

% Optional override for |\version| flag:
%    \begin{macrocode}
%%\providecommand{\version}{final}
%    \end{macrocode}

% Include the main document:
%    \begin{macrocode}
\input{childdoc.def}
\childdocby{cdocsamp}
%    \end{macrocode}

%\iffalse
%</samplepart3|samplepart4>
%\fi
%
%\iffalse
%<*samplepart3>
%\fi
% Some text for part 3:
%    \begin{macrocode}
some text in part three
%    \end{macrocode}

%\iffalse
%</samplepart3>
%\fi
% Some text for part 4:
%\iffalse
%<*samplepart4>
%\fi
%    \begin{macrocode}
more text in part four
%    \end{macrocode}

%\iffalse
%</samplepart4>
%\fi
%
% %%%%%%%%%%%%%%%%%%%%%%%%%%%%%%%%%%%%%%
% \paragraph{Forwarding for a Complete Draft.}
%
% The following forwarding file |cdocsdrf.tex|
% compiles the main document in draft mode:
%\iffalse
%<*sampledraft>
%\fi
%    \begin{macrocode}
\def\version{draft}
\input{childdoc.def}
\childdocforward{cdocsamp}
%    \end{macrocode}

%\iffalse
%</sampledraft>
%\fi
%
% %%%%%%%%%%%%%%%%%%%%%%%%%%%%%%%%%%%%%%
% \paragraph{Forwarding for Final Version of the Chapters.}
%
% The following forwarding files |cdocsfn1.tex| and |cdocsfn2.tex|
% (with identical content)
% compile the final versions of the child documents
% |cdocsch1.tex| and |cdocsch2.tex|, respectively:
%\iffalse
%<*samplefinal>
%\fi
%    \begin{macrocode}
\def\version{final}
\input{childdoc.def}
\childdocforwardprefix[cdocsamp]{cdocsfn}{cdocsch}
%    \end{macrocode}

%\iffalse
%</samplefinal>
%\fi
%
% %%%%%%%%%%%%%%%%%%%%%%%%%%%%%%%%%%%%%%
% \paragraph{Command Line Processing.}
%
% The following three command lines generate the output files
% |cdocscld|, |cdocscl1| and |cdocscl2|
% which should be identical to
% |cdocsdrf|, |cdocsch1| and |cdocsfn2|, respectively:
% \begin{center}
% \begin{tabular}{l}
% |latex -jobname cdocscld \|\\
% |  "\def\version{draft}\input{childdoc.def}\childdocforward{cdocsamp}"|\\
% |latex -jobname cdocscl1 \|\\
% |  "\input{childdoc.def}\childdocforward[cdocsamp]{cdocsch1}"|\\
% |latex -jobname cdocscl2 \|\\
% |  "\def\version{final}\input{childdoc.def}\childdocforward{cdocsch2}"|
% \end{tabular}
% \end{center}
% Note that the trailing backslash on each first line
% merely continues the input to the second line
% (for convenient cut ant paste).
% Furthermore, the command |latex| can be replaced by any
% of its alternative versions such as |pdflatex|.
%
% %%%%%%%%%%%%%%%%%%%%%%%%%%%%%%%%%%%%%%%%%%%%%%%%%%%%%%%%%%%%%%%%%%%%%%%%%%%%%%
% %%%%%%%%%%%%%%%%%%%%%%%%%%%%%%%%%%%%%%%%%%%%%%%%%%%%%%%%%%%%%%%%%%%%%%%%%%%%%%
% \section{Implementation}
%\iffalse
%<*package>
%\fi
%
% This section describes the definitions file |childdoc.def|.

% The definitions cannot be loaded using |\usepackage| or |\RequirePackage|
% which has a mechanism to prevent loading a style file more than once.
% When loading the definitions by means of |\input|
% multiple instances have to be prevented manually:
%\iffalse
%This code needs to be before the `\ProvidesFile' directive
%which is defined at the beginning of this file.
%Therefore it is also placed there and commented out here.
%</package>
%<*discard>
%\fi
%    \begin{macrocode}
\ifdefined\childdocmain\endinput\fi
%    \end{macrocode}
%\iffalse
%</discard>
%<*package>
%\fi
%
% \macro{\ifchilddoc}
% \macro{\ifchilddocmanual}
% The conditional |\ifchilddoc| tells whether a
% child (true) or main (false) document is being compiled.
% The conditional |\ifchilddocmanual| tells whether
% the |\includeonly| mechanism is used (false) or
% the selection of child files must be performed manually (true).
% The definitions initialise to false:
%    \begin{macrocode}
\newif\ifchilddoc
\newif\ifchilddocmanual
%    \end{macrocode}

% \macro{\childdocname}
% \macro{\childdocjob}
% The macro |\childdocname| stores the name of the main document
% to be compiled. The macro |\childdocjob| stores the name of
% the document on which the \LaTeX{} compiler was originally invoked.
% The content of |\jobname| cannot be compared
% to filenames specified in the source due to different catcodes.
% The following code rescans |\jobname|, stores the result
% in |\childdocname| and saves a copy in |\childdocjob|:
%    \begin{macrocode}
\edef\childdocname{\scantokens\expandafter{\jobname\noexpand}}
\let\childdocjob\childdocname
%    \end{macrocode}

% \macro{\childdocdisable}
% The macro |\childdocdisable| prevents the main file
% from being processed more than once.
% At this stage, the main document command |\childdocmain|
% is assumed to be called once again where it should do nothing.
% Any subsequent call to it should prevent
% a secondary processing of the main document
% It overwrites the forwarding commands
% |\childdocof| and |\childdocforward|
% with empty macros to prevent further inclusions of the main document:
%    \begin{macrocode}
\newcommand{\childdocdisable}
{
  \renewcommand{\childdocmain}[1]{\renewcommand{\childdocmain}[1]{\endinput}}
  \renewcommand{\childdocof}[1]{}
  \renewcommand{\childdocby}[2][]{}
  \renewcommand{\childdocforward}[2][]{}
  \renewcommand{\childdocdisable}{}
}
%    \end{macrocode}

% \macro{\childdocmain}
% The macro |\childdocmain| is to be called at the top of the main file
% with nothing or the main filename (without extension) as argument.
% First, it breaks loops.
% If the argument is not empty and does not match |\childdocname|
% (which is set by the first inclusion of |childdoc.def|),
% |\ifchilddoc| is set to true, |\includeonly| is applied to the child file
% and |\jobname| is set to the main file
% (for proper handling of |.aux| files):
%    \begin{macrocode}
\newcommand{\childdocmain}[1]
{
  \childdocdisable\childdocmain{}
  \if?#1?\else
    \begingroup
      \def\childdoctmp{#1}
      \ifx\childdoctmp\childdocname
        \def\childdoctmp{}
      \else
        \def\childdoctmp
        {
          \childdoctrue
          \includeonly{\childdocname}
          \def\childdocjob{#1}
          \def\jobname{#1}
        }
      \fi
      \expandafter
    \endgroup
    \childdoctmp
  \fi
}
%    \end{macrocode}

% \macro{\childdocof}
% The command |\childdocof| redirects
% compilation to the main file |#1|.
%    \begin{macrocode}
\newcommand{\childdocof}[1]
{
  \childdocdisable
  \childdoctrue
  \includeonly{\childdocname}
  \def\jobname{#1}
  \def\childdocjob{#1}
  \input{#1}
}
%    \end{macrocode}

% \macro{\childdocby}
% The command |\childdocby| ....
%    \begin{macrocode}
\newcommand{\childdocby}[2][]
{
  \childdocdisable
  \childdoctrue
  \childdocmanualtrue
  \if?#1?\else
    \def\jobname{#2}
  \fi
  \def\childdocjob{#2}
  \input{#2}
  \endinput
}
%    \end{macrocode}

% \macro{\childdocforward}
% The command |\childdocforward| redirects
% compilation to the main file or
% (if the optional argument is given) a child file.
% Parameters are set as if the main file
% or a child file starting with |\childdocof| was compiled.
% Then compilation is handed over to the main file:
%    \begin{macrocode}
\newcommand{\childdocforward}[2][]
{
  \begingroup
    \if?#1?
      \def\childdoctmp
      {
        \def\childdocname{#2}
        \def\childdocjob{#2}
        \def\jobname{#2}
        \input{#2}
        \endinput
      }
    \else
      \def\childdoctmp
      {
        \childdocdisable
        \def\childdocname{#2}
        \childdoctrue
        \includeonly{#2}
        \def\childdocjob{#1}
        \def\jobname{#1}
        \input{#1}
        \endinput
      }
    \fi
    \expandafter
  \endgroup
  \childdoctmp
}
%    \end{macrocode}

% \macro{\childdocforwardprefix}
% The command |\childdocforwardprefix| redirects
% compilation to the main or a child file by means of a pattern.
% The prefix |#1| in the current filename is replaced by |#2|
% and the suffix of the current filename is kept
% (it is assumed that the filename does not contain the substring `|~~~|'
% which is used as a delimiter).
% Compilation is handed over to the new file by |\childdocforward|:
%    \begin{macrocode}
\newcommand{\childdocforwardprefix}[3][]
{
  \begingroup
    \def\childdocextract #2##1~~~{\def\childdoctmp{\childdocforward[#1]{#3##1}}}
    \expandafter\childdocextract\childdocname~~~
    \expandafter
  \endgroup
  \childdoctmp
}
%    \end{macrocode}

% \macro{\childdoc}
% The deprecated macro |\childdoc| is a legacy version of |\childdocmain|:
%    \begin{macrocode}
\newcommand{\childdoc}{\childdocmain}
%    \end{macrocode}

% \macro{\childdocredirect}
% The deprecated macro |\childdocredirect| is a legacy version
% of |\childdocforward| and |\childdocforwardprefix|:
%    \begin{macrocode}
\newcommand{\childdocredirect}[2][]
{
  \begingroup
    \if?#1?
      \def\childdoctmp{\childdocforward{#2}}
    \else
      \def\childdoctmp{\childdocforwardprefix{#1}{#2}}
    \fi
    \expandafter
  \endgroup
  \childdoctmp
}
%    \end{macrocode}

%\iffalse
%</package>
%\fi
%
\endinput
\childdocforward{cdocsch2}"|
% \end{tabular}
% \end{center}
% Note that the trailing backslash on each first line
% merely continues the input to the second line
% (for convenient cut ant paste).
% Furthermore, the command |latex| can be replaced by any
% of its alternative versions such as |pdflatex|.
%
% %%%%%%%%%%%%%%%%%%%%%%%%%%%%%%%%%%%%%%%%%%%%%%%%%%%%%%%%%%%%%%%%%%%%%%%%%%%%%%
% %%%%%%%%%%%%%%%%%%%%%%%%%%%%%%%%%%%%%%%%%%%%%%%%%%%%%%%%%%%%%%%%%%%%%%%%%%%%%%
% \section{Implementation}
%\iffalse
%<*package>
%\fi
%
% This section describes the definitions file |childdoc.def|.

% The definitions cannot be loaded using |\usepackage| or |\RequirePackage|
% which has a mechanism to prevent loading a style file more than once.
% When loading the definitions by means of |\input|
% multiple instances have to be prevented manually:
%\iffalse
%This code needs to be before the `\ProvidesFile' directive
%which is defined at the beginning of this file.
%Therefore it is also placed there and commented out here.
%</package>
%<*discard>
%\fi
%    \begin{macrocode}
\ifdefined\childdocmain\endinput\fi
%    \end{macrocode}
%\iffalse
%</discard>
%<*package>
%\fi
%
% \macro{\ifchilddoc}
% \macro{\ifchilddocmanual}
% The conditional |\ifchilddoc| tells whether a
% child (true) or main (false) document is being compiled.
% The conditional |\ifchilddocmanual| tells whether
% the |\includeonly| mechanism is used (false) or
% the selection of child files must be performed manually (true).
% The definitions initialise to false:
%    \begin{macrocode}
\newif\ifchilddoc
\newif\ifchilddocmanual
%    \end{macrocode}

% \macro{\childdocname}
% \macro{\childdocjob}
% The macro |\childdocname| stores the name of the main document
% to be compiled. The macro |\childdocjob| stores the name of
% the document on which the \LaTeX{} compiler was originally invoked.
% The content of |\jobname| cannot be compared
% to filenames specified in the source due to different catcodes.
% The following code rescans |\jobname|, stores the result
% in |\childdocname| and saves a copy in |\childdocjob|:
%    \begin{macrocode}
\edef\childdocname{\scantokens\expandafter{\jobname\noexpand}}
\let\childdocjob\childdocname
%    \end{macrocode}

% \macro{\childdocdisable}
% The macro |\childdocdisable| prevents the main file
% from being processed more than once.
% At this stage, the main document command |\childdocmain|
% is assumed to be called once again where it should do nothing.
% Any subsequent call to it should prevent
% a secondary processing of the main document
% It overwrites the forwarding commands
% |\childdocof| and |\childdocforward|
% with empty macros to prevent further inclusions of the main document:
%    \begin{macrocode}
\newcommand{\childdocdisable}
{
  \renewcommand{\childdocmain}[1]{\renewcommand{\childdocmain}[1]{\endinput}}
  \renewcommand{\childdocof}[1]{}
  \renewcommand{\childdocby}[2][]{}
  \renewcommand{\childdocforward}[2][]{}
  \renewcommand{\childdocdisable}{}
}
%    \end{macrocode}

% \macro{\childdocmain}
% The macro |\childdocmain| is to be called at the top of the main file
% with nothing or the main filename (without extension) as argument.
% First, it breaks loops.
% If the argument is not empty and does not match |\childdocname|
% (which is set by the first inclusion of |childdoc.def|),
% |\ifchilddoc| is set to true, |\includeonly| is applied to the child file
% and |\jobname| is set to the main file
% (for proper handling of |.aux| files):
%    \begin{macrocode}
\newcommand{\childdocmain}[1]
{
  \childdocdisable\childdocmain{}
  \if?#1?\else
    \begingroup
      \def\childdoctmp{#1}
      \ifx\childdoctmp\childdocname
        \def\childdoctmp{}
      \else
        \def\childdoctmp
        {
          \childdoctrue
          \includeonly{\childdocname}
          \def\childdocjob{#1}
          \def\jobname{#1}
        }
      \fi
      \expandafter
    \endgroup
    \childdoctmp
  \fi
}
%    \end{macrocode}

% \macro{\childdocof}
% The command |\childdocof| redirects
% compilation to the main file |#1|.
%    \begin{macrocode}
\newcommand{\childdocof}[1]
{
  \childdocdisable
  \childdoctrue
  \includeonly{\childdocname}
  \def\jobname{#1}
  \def\childdocjob{#1}
  \input{#1}
}
%    \end{macrocode}

% \macro{\childdocby}
% The command |\childdocby| ....
%    \begin{macrocode}
\newcommand{\childdocby}[2][]
{
  \childdocdisable
  \childdoctrue
  \childdocmanualtrue
  \if?#1?\else
    \def\jobname{#2}
  \fi
  \def\childdocjob{#2}
  \input{#2}
  \endinput
}
%    \end{macrocode}

% \macro{\childdocforward}
% The command |\childdocforward| redirects
% compilation to the main file or
% (if the optional argument is given) a child file.
% Parameters are set as if the main file
% or a child file starting with |\childdocof| was compiled.
% Then compilation is handed over to the main file:
%    \begin{macrocode}
\newcommand{\childdocforward}[2][]
{
  \begingroup
    \if?#1?
      \def\childdoctmp
      {
        \def\childdocname{#2}
        \def\childdocjob{#2}
        \def\jobname{#2}
        \input{#2}
        \endinput
      }
    \else
      \def\childdoctmp
      {
        \childdocdisable
        \def\childdocname{#2}
        \childdoctrue
        \includeonly{#2}
        \def\childdocjob{#1}
        \def\jobname{#1}
        \input{#1}
        \endinput
      }
    \fi
    \expandafter
  \endgroup
  \childdoctmp
}
%    \end{macrocode}

% \macro{\childdocforwardprefix}
% The command |\childdocforwardprefix| redirects
% compilation to the main or a child file by means of a pattern.
% The prefix |#1| in the current filename is replaced by |#2|
% and the suffix of the current filename is kept
% (it is assumed that the filename does not contain the substring `|~~~|'
% which is used as a delimiter).
% Compilation is handed over to the new file by |\childdocforward|:
%    \begin{macrocode}
\newcommand{\childdocforwardprefix}[3][]
{
  \begingroup
    \def\childdocextract #2##1~~~{\def\childdoctmp{\childdocforward[#1]{#3##1}}}
    \expandafter\childdocextract\childdocname~~~
    \expandafter
  \endgroup
  \childdoctmp
}
%    \end{macrocode}

% \macro{\childdoc}
% The deprecated macro |\childdoc| is a legacy version of |\childdocmain|:
%    \begin{macrocode}
\newcommand{\childdoc}{\childdocmain}
%    \end{macrocode}

% \macro{\childdocredirect}
% The deprecated macro |\childdocredirect| is a legacy version
% of |\childdocforward| and |\childdocforwardprefix|:
%    \begin{macrocode}
\newcommand{\childdocredirect}[2][]
{
  \begingroup
    \if?#1?
      \def\childdoctmp{\childdocforward{#2}}
    \else
      \def\childdoctmp{\childdocforwardprefix{#1}{#2}}
    \fi
    \expandafter
  \endgroup
  \childdoctmp
}
%    \end{macrocode}

%\iffalse
%</package>
%\fi
%
\endinput
|\\
|\childdocby{|\textit{main}|}|\\
\end{tabular}
\end{center}
%
The directive |\childdocby| is similar to |\childdocof|
described in \secref{sec:include},
but the subsequent selection of content must be done manually.
To that end, both |\ifchilddoc| and |\ifchilddocmanual|
will be true upon processing of a part,
and the name of the part is stored in |\childdocname|.
Note that |\jobname| will be set to the filename of the current part
so that each part receives an individual |.aux| file
that does not interfere with the |.aux| file(s) of the main document.
This behaviour can be altered by the alternative form
|\childdocby[*]{|\textit{main}|}| (with a non-empty optional argument)
which uses the |.aux| file of the main document
by setting |\jobname| to \textit{main}.

%%%%%%%%%%%%%%%%%%%%%%%%%%%%%%%%%%%%%%%%%%%%%%%%%%%%%%%%%%%%%%%%%%%%%%%%%%%%%%%%
\subsection{Driver Development}
\label{sec:driver}

The \textsf{childdoc} mechanism can also be use for the development
of definition files such as \LaTeX{} styles or classes.
This case differs from the above setup with multiple parts
included by |\include| in that no |\includeonly| should be invoked.
This can be achieved by starting the include file
(before |\ProvidesPackage|) with:
%
\begin{center}
\begin{tabular}{l}
|% \iffalse
%
% childdoc.dtx Copyright (C) 2017-2018 Niklas Beisert
%
% This work may be distributed and/or modified under the
% conditions of the LaTeX Project Public License, either version 1.3
% of this license or (at your option) any later version.
% The latest version of this license is in
%   http://www.latex-project.org/lppl.txt
% and version 1.3 or later is part of all distributions of LaTeX
% version 2005/12/01 or later.
%
% This work has the LPPL maintenance status `maintained'.
%
% The Current Maintainer of this work is Niklas Beisert.
%
% This work consists of the files childdoc.dtx and childdoc.ins
% and the derived files childdoc.def and cdocsamp.tex with
% cdocsch1.tex, cdocsch2.tex, cdocsdrf.tex, cdocsfn1.tex, cdocsfn2.tex.
%
%<package>\ifdefined\childdocmain\endinput\fi
%<package>\ProvidesFile{childdoc.def}[2018/12/30 v2.0 child document driver]
%<samplemain>\ProvidesFile{cdocsamp.tex}[2018/12/30 v2.0 sample for childdoc]
%<*driver>
%\ProvidesFile{childdoc.drv}[2018/12/30 v2.0 childdoc reference manual file]
\PassOptionsToClass{10pt,a4paper}{article}
\documentclass{ltxdoc}

\usepackage[margin=35mm]{geometry}
\usepackage{hyperref}
\usepackage{hyperxmp}
\usepackage[usenames]{color}

\hypersetup{colorlinks=true}
\hypersetup{pdfstartview=FitH}
\hypersetup{pdfpagemode=UseNone}
\hypersetup{pdfsource={}}
\hypersetup{pdflang={en-UK}}
\hypersetup{pdfcopyright={Copyright 2017-2018 Niklas Beisert.
  This work may be distributed and/or modified under the
  conditions of the LaTeX Project Public License, either version 1.3
  of this license or (at your option) any later version.}}
\hypersetup{pdflicenseurl={http://www.latex-project.org/lppl.txt}}
\hypersetup{pdfcontactaddress={ETH Zurich, ITP, HIT K,
  Wolfgang-Pauli-Strasse 27}}
\hypersetup{pdfcontactpostcode={8093}}
\hypersetup{pdfcontactcity={Zurich}}
\hypersetup{pdfcontactcountry={Switzerland}}
\hypersetup{pdfcontactemail={nbeisert@itp.phys.ethz.ch}}
\hypersetup{pdfcontacturl={http://people.phys.ethz.ch/\xmptilde nbeisert/}}

\newcommand{\secref}[1]{\hyperref[#1]{section \ref*{#1}}}

\parskip1ex
\parindent0pt
\let\olditemize\itemize
\def\itemize{\olditemize\parskip0pt}

\begin{document}

\title{The \textsf{childdoc} Package}
\hypersetup{pdftitle={The childdoc Package}}
\author{Niklas Beisert\\[2ex]
  Institut f\"ur Theoretische Physik\\
  Eidgen\"ossische Technische Hochschule Z\"urich\\
  Wolfgang-Pauli-Strasse 27, 8093 Z\"urich, Switzerland\\[1ex]
  \href{mailto:nbeisert@itp.phys.ethz.ch}
  {\texttt{nbeisert@itp.phys.ethz.ch}}}
\hypersetup{pdfauthor={Niklas Beisert}}
\hypersetup{pdfsubject={Manual for the LaTeX2e Package childdoc}}
\date{30 December 2018, \textsf{v2.0}}
\maketitle

\begin{abstract}\noindent
\textsf{childdoc} is a \LaTeXe{} package
that enables the direct compilation
of document sections included by |\include|
to individual files.
\end{abstract}

\begingroup
\parskip0ex
\tableofcontents
\endgroup

%%%%%%%%%%%%%%%%%%%%%%%%%%%%%%%%%%%%%%%%%%%%%%%%%%%%%%%%%%%%%%%%%%%%%%%%%%%%%%%%
%%%%%%%%%%%%%%%%%%%%%%%%%%%%%%%%%%%%%%%%%%%%%%%%%%%%%%%%%%%%%%%%%%%%%%%%%%%%%%%%
\section{Introduction}

\LaTeX{} provides a mechanism to structure a large document (such as a book)
into a main file and several child files (containing the chapters)
using the |\include| command.
This mechanism is beneficial for documents
which span hundreds of pages in order to
make the source file(s) more manageable.
Moreover, compilation can be restricted to
selected child files by means of the |\includeonly| command.
The latter feature can be used to reduce the compilation time while editing
(this was significantly more useful in the earlier days of \LaTeX{})
or to generate a smaller document which is easier to navigate.
Another application of |\includeonly| is to generate
documents consisting of selected parts of the complete document.

However, there are a few drawbacks of the plain |\include| mechanism:
\begin{itemize}
\item
The child files cannot be compiled on their own,
they can only be compiled via the main file.
A naive editing environment
(such as a text editor with an option
to have the current file processed by \LaTeX)
may require one to switch to the main file before compiling;
attempting to compile the child file produces errors.
\item
The main file must be modified (each time)
to adjust the |\includeonly| command
to the present needs. This easily leaves the main file in a messy state.
\item
The generated document will always carry the filename
of the main document. This is inconvenient if
several child files are to be compiled and
to be kept for distribution.
\end{itemize}

The present package provides a simple interface
to make child files individually compilable by \LaTeX{}.
Compiling a child file then has the same effect as compiling
the main file with an |\includeonly| command
to select the appropriate child.
Moreover the generated document will carry the name of the child
rather than the main file.
This resolves all three above issues.

This feature is meant to make the editing of books,
thesis documents and lecture notes somewhat more convenient.
However, the package can also be used efficiently for
composing a series of documents (such as exercise sheets)
which are typically distributed individually.
It then assists the author in generating the individual documents
(potentially in different versions)
as well as a document containing the collected series.
Another application is in developing style files
or other kinds of included material
where compilation of the style file could redirect
to a sample or test file.

%%%%%%%%%%%%%%%%%%%%%%%%%%%%%%%%%%%%%%%%%%%%%%%%%%%%%%%%%%%%%%%%%%%%%%%%%%%%%%%%
%%%%%%%%%%%%%%%%%%%%%%%%%%%%%%%%%%%%%%%%%%%%%%%%%%%%%%%%%%%%%%%%%%%%%%%%%%%%%%%%
\section{Usage}

First of all, the package \textsf{childdoc} is \emph{not} a standard
\LaTeXe{} |.sty| style file! Therefore it needs to be invoked in
a non-standard way.

%%%%%%%%%%%%%%%%%%%%%%%%%%%%%%%%%%%%%%%%%%%%%%%%%%%%%%%%%%%%%%%%%%%%%%%%%%%%%%%%
\subsection{Included Files}
\label{sec:include}

%%%%%%%%%%%%%%%%%%%%%%%%%%%%%%%%%%%%%%%%
\DescribeMacro{\childdocmain}
To use the package, add the commands
\begin{center}
\begin{tabular}{l}
|% \iffalse
%
% childdoc.dtx Copyright (C) 2017-2018 Niklas Beisert
%
% This work may be distributed and/or modified under the
% conditions of the LaTeX Project Public License, either version 1.3
% of this license or (at your option) any later version.
% The latest version of this license is in
%   http://www.latex-project.org/lppl.txt
% and version 1.3 or later is part of all distributions of LaTeX
% version 2005/12/01 or later.
%
% This work has the LPPL maintenance status `maintained'.
%
% The Current Maintainer of this work is Niklas Beisert.
%
% This work consists of the files childdoc.dtx and childdoc.ins
% and the derived files childdoc.def and cdocsamp.tex with
% cdocsch1.tex, cdocsch2.tex, cdocsdrf.tex, cdocsfn1.tex, cdocsfn2.tex.
%
%<package>\ifdefined\childdocmain\endinput\fi
%<package>\ProvidesFile{childdoc.def}[2018/12/30 v2.0 child document driver]
%<samplemain>\ProvidesFile{cdocsamp.tex}[2018/12/30 v2.0 sample for childdoc]
%<*driver>
%\ProvidesFile{childdoc.drv}[2018/12/30 v2.0 childdoc reference manual file]
\PassOptionsToClass{10pt,a4paper}{article}
\documentclass{ltxdoc}

\usepackage[margin=35mm]{geometry}
\usepackage{hyperref}
\usepackage{hyperxmp}
\usepackage[usenames]{color}

\hypersetup{colorlinks=true}
\hypersetup{pdfstartview=FitH}
\hypersetup{pdfpagemode=UseNone}
\hypersetup{pdfsource={}}
\hypersetup{pdflang={en-UK}}
\hypersetup{pdfcopyright={Copyright 2017-2018 Niklas Beisert.
  This work may be distributed and/or modified under the
  conditions of the LaTeX Project Public License, either version 1.3
  of this license or (at your option) any later version.}}
\hypersetup{pdflicenseurl={http://www.latex-project.org/lppl.txt}}
\hypersetup{pdfcontactaddress={ETH Zurich, ITP, HIT K,
  Wolfgang-Pauli-Strasse 27}}
\hypersetup{pdfcontactpostcode={8093}}
\hypersetup{pdfcontactcity={Zurich}}
\hypersetup{pdfcontactcountry={Switzerland}}
\hypersetup{pdfcontactemail={nbeisert@itp.phys.ethz.ch}}
\hypersetup{pdfcontacturl={http://people.phys.ethz.ch/\xmptilde nbeisert/}}

\newcommand{\secref}[1]{\hyperref[#1]{section \ref*{#1}}}

\parskip1ex
\parindent0pt
\let\olditemize\itemize
\def\itemize{\olditemize\parskip0pt}

\begin{document}

\title{The \textsf{childdoc} Package}
\hypersetup{pdftitle={The childdoc Package}}
\author{Niklas Beisert\\[2ex]
  Institut f\"ur Theoretische Physik\\
  Eidgen\"ossische Technische Hochschule Z\"urich\\
  Wolfgang-Pauli-Strasse 27, 8093 Z\"urich, Switzerland\\[1ex]
  \href{mailto:nbeisert@itp.phys.ethz.ch}
  {\texttt{nbeisert@itp.phys.ethz.ch}}}
\hypersetup{pdfauthor={Niklas Beisert}}
\hypersetup{pdfsubject={Manual for the LaTeX2e Package childdoc}}
\date{30 December 2018, \textsf{v2.0}}
\maketitle

\begin{abstract}\noindent
\textsf{childdoc} is a \LaTeXe{} package
that enables the direct compilation
of document sections included by |\include|
to individual files.
\end{abstract}

\begingroup
\parskip0ex
\tableofcontents
\endgroup

%%%%%%%%%%%%%%%%%%%%%%%%%%%%%%%%%%%%%%%%%%%%%%%%%%%%%%%%%%%%%%%%%%%%%%%%%%%%%%%%
%%%%%%%%%%%%%%%%%%%%%%%%%%%%%%%%%%%%%%%%%%%%%%%%%%%%%%%%%%%%%%%%%%%%%%%%%%%%%%%%
\section{Introduction}

\LaTeX{} provides a mechanism to structure a large document (such as a book)
into a main file and several child files (containing the chapters)
using the |\include| command.
This mechanism is beneficial for documents
which span hundreds of pages in order to
make the source file(s) more manageable.
Moreover, compilation can be restricted to
selected child files by means of the |\includeonly| command.
The latter feature can be used to reduce the compilation time while editing
(this was significantly more useful in the earlier days of \LaTeX{})
or to generate a smaller document which is easier to navigate.
Another application of |\includeonly| is to generate
documents consisting of selected parts of the complete document.

However, there are a few drawbacks of the plain |\include| mechanism:
\begin{itemize}
\item
The child files cannot be compiled on their own,
they can only be compiled via the main file.
A naive editing environment
(such as a text editor with an option
to have the current file processed by \LaTeX)
may require one to switch to the main file before compiling;
attempting to compile the child file produces errors.
\item
The main file must be modified (each time)
to adjust the |\includeonly| command
to the present needs. This easily leaves the main file in a messy state.
\item
The generated document will always carry the filename
of the main document. This is inconvenient if
several child files are to be compiled and
to be kept for distribution.
\end{itemize}

The present package provides a simple interface
to make child files individually compilable by \LaTeX{}.
Compiling a child file then has the same effect as compiling
the main file with an |\includeonly| command
to select the appropriate child.
Moreover the generated document will carry the name of the child
rather than the main file.
This resolves all three above issues.

This feature is meant to make the editing of books,
thesis documents and lecture notes somewhat more convenient.
However, the package can also be used efficiently for
composing a series of documents (such as exercise sheets)
which are typically distributed individually.
It then assists the author in generating the individual documents
(potentially in different versions)
as well as a document containing the collected series.
Another application is in developing style files
or other kinds of included material
where compilation of the style file could redirect
to a sample or test file.

%%%%%%%%%%%%%%%%%%%%%%%%%%%%%%%%%%%%%%%%%%%%%%%%%%%%%%%%%%%%%%%%%%%%%%%%%%%%%%%%
%%%%%%%%%%%%%%%%%%%%%%%%%%%%%%%%%%%%%%%%%%%%%%%%%%%%%%%%%%%%%%%%%%%%%%%%%%%%%%%%
\section{Usage}

First of all, the package \textsf{childdoc} is \emph{not} a standard
\LaTeXe{} |.sty| style file! Therefore it needs to be invoked in
a non-standard way.

%%%%%%%%%%%%%%%%%%%%%%%%%%%%%%%%%%%%%%%%%%%%%%%%%%%%%%%%%%%%%%%%%%%%%%%%%%%%%%%%
\subsection{Included Files}
\label{sec:include}

%%%%%%%%%%%%%%%%%%%%%%%%%%%%%%%%%%%%%%%%
\DescribeMacro{\childdocmain}
To use the package, add the commands
\begin{center}
\begin{tabular}{l}
|\input{childdoc.def}|\\
|\childdocmain{}|\\
\end{tabular}
\end{center}
at the very top of the main \LaTeX{} file,
in particular \emph{before} the |\documentclass| statement!
The argument of |\childdocmain| should be left empty
(but it must be present).

%%%%%%%%%%%%%%%%%%%%%%%%%%%%%%%%%%%%%%%%
\DescribeMacro{\childdocof}
Furthermore, add the commands
\begin{center}
\begin{tabular}{l}
|\input{childdoc.def}|\\
|\childdocof{|\textit{main}|}|\\
\end{tabular}
\end{center}
at the top of every child file \textit{child}
which is included by |\include{|\textit{child}|}|
from within the main file
(or at least for those files to be compiled individually).
The argument \textit{main} must be the filename of the main file.

There are a couple of
considerations in setting up the main and child documents:

%%%%%%%%%%%%%%%%%%%%%%%%%%%%%%%%%%%%%%%%
\paragraph{Restrictions.}

Please note the following restrictions:
\begin{itemize}
\item
|\childdocmain| must be called with one argument \textit{main}
to ensure compatibility with earlier version of the package.
It must either be empty (|\childdocmain{}|)
or precisely match the filename of the main file in which it is specified.
See \secref{sec:detection} for further information.
\item
The filename \textit{main} must be specified without the |.tex| extension.
\item
The filename \textit{main} is case sensitive
(even in case-insensitive file systems)
due to internal string comparison.
\item
The argument \textit{main} should be fully expanded, it cannot be a macro.
\item
Subdirectories and special characters should be avoided in filenames.
\item
The command |\childdocmain{|\textit{main}|}| must be followed by a whitespace.
It should not be followed immediately by another command
or by a comment mark `|%|'.
This is because the \TeX{} parser reads the token immediately following
the argument of |\childdocmain| and puts it
at the beginning of every child section;
however, a white\-space is ignored.
\end{itemize}

%%%%%%%%%%%%%%%%%%%%%%%%%%%%%%%%%%%%%%%%
\paragraph{Content of Main File.}

It is advisable to place all content in the child files included by |\include|.
Any output contained in the main file will appear in all child documents
unless suppressed manually;
it cannot be suppressed automatically by the |\includeonly| directive
and thus should normally be avoided.
A method to include some content in the main file
by means of conditional processing is described in \secref{sec:conditional}.

%%%%%%%%%%%%%%%%%%%%%%%%%%%%%%%%%%%%%%%%
\paragraph{Page Numbering.}

When only a part of the document is compiled,
the appropriate numbering of pages
(as well as other status parameters)
is determined from the |.aux| files.
The latter contain information from previous passes.
However this information needs to propagate through
all intermediate child documents.
Therefore the page numbering in child documents may well
be inconsistent until the complete document is compiled at least once.

A useful (if unconventional) way to always ensure a consistent
page numbering is to restart the numbering in each child document
and denote the pages by `\textit{child}|.|\textit{page}'
where \textit{child} represents the chapter/section number of the child file.
This can be achieved by the command
|\numberwithin{page}{|\textit{child}|}|
of the \textsf{amsmath} package
where \textit{child} can be |chapter| or |section|
depending on the chosen structuring.
Alternatively, one can modify the macro |\thepage| appropriately
and reset the counter |page| at the start of each child file.

%%%%%%%%%%%%%%%%%%%%%%%%%%%%%%%%%%%%%%%%%%%%%%%%%%%%%%%%%%%%%%%%%%%%%%%%%%%%%%%%
\subsection{Conditional Processing}
\label{sec:conditional}

The package provides a mechanism to compile different versions
of a document. To customise the versions further some conditional processing
can come in handy to distinguish which version is being compiled.
The package provides two macros to describe the compilation context:

%%%%%%%%%%%%%%%%%%%%%%%%%%%%%%%%%%%%%%%%
\DescribeMacro{\ifchilddoc}
The conditional |\ifchilddoc| distinguishes between the compilation of
child documents and the main document:
%
\begin{center}
|\ifchilddoc |\textit{child-code}| |[|\||else |\textit{main-code}]| \||fi|
\end{center}

%%%%%%%%%%%%%%%%%%%%%%%%%%%%%%%%%%%%%%%%
\DescribeMacro{\childdocname}
\DescribeMacro{\childdocjob}
The macro |\childdocname| contains the filename (without extension)
of the main or child file being processed.
Note that |\childdocjob| will always contain the name of the main file.

%%%%%%%%%%%%%%%%%%%%%%%%%%%%%%%%%%%%%%%%
\paragraph{Title Page.}

Conditional processing can be used to include a title or banner page
in the main document when proper precautions are taken.
Importantly, the code in the main file should ensure that the page counter
(as well as other status parameters which are stored in the |.aux| files)
takes the same value after the conditional processing.
Otherwise the page numbers may take divergent values
depending on which part is compiled.

For example, a title page could be declared by:
%
\begin{center}
\begin{tabular}{l}
|\ifchilddoc\||else|\\
|\addtocounter{page}{-1}|\\
\textit{code for title page}\\
|\newpage|\\
|\||fi|
\end{tabular}
\end{center}
%
A banner page for the child documents can be generated by:
%
\begin{center}
\begin{tabular}{l}
|\ifchilddoc|\\
|\addtocounter{page}{-1}|\\
\textit{code for banner page}\\
|\newpage|\\
|\||fi|
\end{tabular}
\end{center}
%
Here one could write a message such as:
\begin{center}
|This is the part \childdocname{} of \childdocjob{}.|
\end{center}

%%%%%%%%%%%%%%%%%%%%%%%%%%%%%%%%%%%%%%%%%%%%%%%%%%%%%%%%%%%%%%%%%%%%%%%%%%%%%%%%
\subsection{Flags}
\label{sec:flags}

The package makes it easy to generate different versions
of the main or child documents.
To this end compilation flags can be defined
and assigned different default values.
They will be particularly useful in conjunction
with the forwarding mechanism described in \secref{sec:forward}.

For example, it may be useful to have a flag |\version|
which can be set to |draft| or |final|.
The document source will contain some conditional code
depending on the value of |\version|.
Suppose further, the flag should default to |final| for the main file
and to |draft| for child files
which is a natural assignment for editing the document.
This is achieved by placing the following code
in the preamble of the main document
(below the |\childdocmain| directive):
%
\begin{center}
\begin{tabular}{l}
|\ifchilddoc|\\
|\providecommand{\version}{draft}|\\
|\||else|\\
|\providecommand{\version}{final}|\\
|\||fi|
\end{tabular}
\end{center}
%
The definition by |\providecommand| makes sure
that previous definitions are not overwritten.
Further statements |\providecommand{\version}{...}|
can thus be added before the above code to override it.

For the main file, one might add a line
(between |\childdocmain| and the above block)
%
\begin{center}
|%\ifchilddoc\||else\providecommand{\version}{draft}\||fi|
\end{center}
%
which can be uncommented to produce a draft version.
Likewise one can add a line to the very top of a child file
(above the |\childdocof{|\textit{main}|}| directive)
%
\begin{center}
|%\providecommand{\version}{final}|
\end{center}
%
which can be uncommented to produce the final version of this child document.

%%%%%%%%%%%%%%%%%%%%%%%%%%%%%%%%%%%%%%%%%%%%%%%%%%%%%%%%%%%%%%%%%%%%%%%%%%%%%%%%
\subsection{Forwarding}
\label{sec:forward}

Different versions of the main or child documents
using compilation flags as described in \secref{sec:flags}
can be (permanently) stored in different files
for convenient compilation, viewing and distribution.
To this end, the package defines a command
to pass on compilation to a different file:

%%%%%%%%%%%%%%%%%%%%%%%%%%%%%%%%%%%%%%%%
\DescribeMacro{\childdocforward}
The command |\childdocforward| redirects processing to
another source file:
%
\begin{center}
\begin{tabular}{l}
|\input{childdoc.def}|\\
|\childdocforward[|\textit{main}|]{|\textit{dest}|}|\\
\end{tabular}
\end{center}
%
The argument \textit{dest} is the destination file
(without extension).
It should be the main file or one of the child files.
Note that further \textsf{childdoc} directives
such as |\childdocof| and |\childdocforward|
in the indicated file will be processed in this form.
The optional argument \textit{main}
passes on directly to the main file \textit{main}
while pretending to compile the child \textit{dest}.
This form behaves as if \textit{dest}
issues |\childdocof{|\textit{main}|}| right away,
and no further \textsf{childdoc} directives will be processed.

%%%%%%%%%%%%%%%%%%%%%%%%%%%%%%%%%%%%%%%%
\DescribeMacro{\...prefix}
In the alternative form |\childdocforwardprefix|,
%
\begin{center}
\begin{tabular}{l}
|\input{childdoc.def}|\\
|\childdocforwardprefix[|\textit{main}|]{|\textit{prefix}|}{|\textit{dest}|}|
\end{tabular}
\end{center}
%
the destination file is determined by a pattern
depending on the current file:
To make this work, the current file must be called
`{\textit{prefix}\hspace{0.2em}\textit{suffix}}'
with \textit{prefix} matching precisely the argument.
Processing is then passed on to the file
`{\textit{dest}\hspace{0.2em}\textit{suffix}}'.
Surely, the same effect is achieved by
directly specifying the
argument `{\textit{dest}\hspace{0.2em}\textit{suffix}}'
in the first form.
However, that requires to set up a different file
for each child. With the alternative form of the command
all these files can have exactly the same content
which simplifies setting them up and maintaining them.

For example, the following file |draft.tex|
with a compilation flag |\version| as described in \secref{sec:flags}
compiles the main document as a draft:
%
\begin{center}
\begin{tabular}{l}
|\def\version{draft}|\\
|\input{childdoc.def}|\\
|\childdocforward{|\textit{main}|}|
\end{tabular}
\end{center}
%
Likewise, the following files |final|\textit{nn}|.tex|
compile the final version of the child document
|child|\textit{nn}|.tex|:
%
\begin{center}
\begin{tabular}{l}
|\def\version{final}|\\
|\input{childdoc.def}|\\
|\childdocforwardprefix{final}{child}|
\end{tabular}
\end{center}
%

Note that when several versions of a main file and/or of each child file
are to be generated, it may be convenient to set up a |Makefile| or
shell script to automatise the process.

%%%%%%%%%%%%%%%%%%%%%%%%%%%%%%%%%%%%%%%%%%%%%%%%%%%%%%%%%%%%%%%%%%%%%%%%%%%%%%%%
\subsection{Command Line Processing}
\label{sec:commandline}

The effect of redirection files can also be achieved by invoking
the \LaTeX{} compiler with a more elaborate command line.
Most conveniently this should be done as part
of a shell script or a |Makefile|.

When using \textsf{childdoc} in the main file, the following
command lines effectively perform a redirection
(note that depending on the shell being used,
backslashes may have to be doubled: `|\|' $\to$ `|\\|'):
%
\begin{center}
|... -jobname "|\textit{target}|" |\\|"|[\textit{flags}]%
|\input{childdoc.def}\childdocforward[|\textit{main}|]{|\textit{dest}|}"|
\end{center}
%
Here \textit{target} is the name of the output file,
\textit{main} is the name of the main file
and \textit{dest} is the name of the main or child file to be processed
(all filenames without extensions).
The optional argument \textit{main} can be omitted
if \textit{main} matches \textit{dest}.
Optionally, compilation \textit{flags} can be defined via |\def| commands.
This command line makes the \TeX{} engine believe
it is compiling the file \textit{target}
whose content is specified as the latter parameter.
The provided code then forwards the processing to
\textit{main} or \textit{dest} as described in \secref{sec:forward}.

%%%%%%%%%%%%%%%%%%%%%%%%%%%%%%%%%%%%%%%%%%%%%%%%%%%%%%%%%%%%%%%%%%%%%%%%%%%%%%%%
\subsection{Include by Input}
\label{sec:input}

Including child documents by |\include| has some restrictions by design.
Most notably, the content of a child document always occupies
its own set of pages; pages cannot be shared between child documents.
Usually, this behaviour makes perfect sense
because each child document contain an essential part of the document.
However, in some situations it may be desirable to compose
a document from a collection of parts
without having mandatory page breaks between then.
For this case, the package
provides a mechanism to include parts
by |\input| which can also be processed individually.
However, by construction this mechanism
requires manual handling of the content to be output.

%%%%%%%%%%%%%%%%%%%%%%%%%%%%%%%%%%%%%%%%
\DescribeMacro{\ifchilddocmanual}
The main file should be prepared as usual, see \secref{sec:include}.
However, the document body must make a distinction
between processing of an individual part and of the main document, e.g.:
%
\begin{center}
\begin{tabular}{l}
|\ifchilddocmanual|\\
|\input{\childdocname}|\\
|\||else|\\
\textit{document body with }|\input{|\textit{part}|}|\\
|\||fi|
\end{tabular}
\end{center}
%
The conditional |\ifchilddocmanual| is true whenever
a part to be included by |\input| is being compiled,
and the name of the part is stored in |\childdocname|.

%%%%%%%%%%%%%%%%%%%%%%%%%%%%%%%%%%%%%%%%
\DescribeMacro{\childdocby}
Each part to be included by |\input| should start with:
%
\begin{center}
\begin{tabular}{l}
|\input{childdoc.def}|\\
|\childdocby{|\textit{main}|}|\\
\end{tabular}
\end{center}
%
The directive |\childdocby| is similar to |\childdocof|
described in \secref{sec:include},
but the subsequent selection of content must be done manually.
To that end, both |\ifchilddoc| and |\ifchilddocmanual|
will be true upon processing of a part,
and the name of the part is stored in |\childdocname|.
Note that |\jobname| will be set to the filename of the current part
so that each part receives an individual |.aux| file
that does not interfere with the |.aux| file(s) of the main document.
This behaviour can be altered by the alternative form
|\childdocby[*]{|\textit{main}|}| (with a non-empty optional argument)
which uses the |.aux| file of the main document
by setting |\jobname| to \textit{main}.

%%%%%%%%%%%%%%%%%%%%%%%%%%%%%%%%%%%%%%%%%%%%%%%%%%%%%%%%%%%%%%%%%%%%%%%%%%%%%%%%
\subsection{Driver Development}
\label{sec:driver}

The \textsf{childdoc} mechanism can also be use for the development
of definition files such as \LaTeX{} styles or classes.
This case differs from the above setup with multiple parts
included by |\include| in that no |\includeonly| should be invoked.
This can be achieved by starting the include file
(before |\ProvidesPackage|) with:
%
\begin{center}
\begin{tabular}{l}
|\input{childdoc.def}|\\
|\childdocforward{|\textit{main}|}|\\
\end{tabular}
\end{center}
%
or alternatively with:
%
\begin{center}
\begin{tabular}{l}
|\input{childdoc.def}|\\
|\childdocby{|\textit{main}|}|\\
\end{tabular}
\end{center}
%
Both forms have slightly different effects as described above.
The main file is prepared as usual, see \secref{sec:include}.

%%%%%%%%%%%%%%%%%%%%%%%%%%%%%%%%%%%%%%%%%%%%%%%%%%%%%%%%%%%%%%%%%%%%%%%%%%%%%%%%
\subsection{Legacy Detection}
\label{sec:detection}

The directive |\childdocmain| in the main file can detect
whether the complete document or merely a child is to be compiled
even without using the directive |\childdocof|.
This method is deprecated because it is less robust
and there is no compelling reason to use it;
it is merely provided for backward compatibility
and it may be removed in future versions.

If the detection mechanism is to be used,
it is mandatory to correctly specify
the filename of the main file as the argument of |\childdocmain|:
%
\begin{center}
\begin{tabular}{l}
|\input{childdoc.def}|\\
|\childdocmain{|\textit{main}|}|\\
\end{tabular}
\end{center}
%
If |\jobname| does not match the argument \textit{main} of |\childdocmain|,
it is assumed that |\jobname| points to the child file to be compiled.
When using |\childdocmain| with the main file specified as argument,
it suffices to start a child file
with just |\input{|\textit{main}|}|
without loading of the package and using |\childdocof|.
If instead all processing is done
with the appropriate \textsf{childdoc} directives,
the argument of \textit{main} of |\childdocmain| can be empty.

An alternative version of the command line processing described
in \secref{sec:commandline} using the detection mechanism reads:
%
\begin{center}
|... -jobname "|\textit{target}|" "|[\textit{flags}]%
[|\def\jobname{|\textit{dest}|}|]|\input{|\textit{main}|}"|
\end{center}

%%%%%%%%%%%%%%%%%%%%%%%%%%%%%%%%%%%%%%%%%%%%%%%%%%%%%%%%%%%%%%%%%%%%%%%%%%%%%%%%
\subsection{Manual Code}
\label{sec:manual}

In case one cannot be certain whether the definitions file |childdoc.def|
is installed on the target \TeX{} distribution
and one prefers not to ship it,
it is conceivable to paste a few relevant commands into the sources.

To that end, drop all statements |\input{childdoc.def}|
and perform the replacements as outlined below.
Instead of |\childdocmain{|\textit{main}|}| add the following code
to the top of the main file:
%
\begin{center}
\begin{tabular}{l}
|\||ifdefined\childdocname\endinput\||fi\newif\ifchilddoc|\\
|\edef\childdocname{\scantokens\expandafter{\jobname\noexpand}}|\\
|\def\childdocmain{|\textit{main}|}\||ifx\childdocmain\childdocname\||else|\\
|\childdoctrue\includeonly{\childdocname}\let\jobname\childdocmain\||fi|\\
\end{tabular}
\end{center}
%
Instead of |\childdocof{|\textit{main}|}| just include the main file
at the top of each child file:
%
\begin{center}
|\input{|\textit{main}|}|
\end{center}
%
A simple redirection |\childdocforward{|\textit{dest}|}| is achieved by:
%
\begin{center}
|\def\jobname{|\textit{dest}|}\input{\jobname}|
\end{center}
%
The redirection with prefix
|\childdocforwardprefix[|\textit{prefix}|]{|\textit{dest}|}|
is accomplished by:
%
\begin{center}
\begin{tabular}{l}
|{\edef\jobname{\scantokens\expandafter{\jobname\noexpand}}|\\
|\def\redirectjob |\textit{prefix}|#1~~~{\gdef\jobname{|\textit{dest}|#1}}|\\
|\expandafter\redirectjob\jobname~~~}\input{\jobname}|
\end{tabular}
\end{center}

In an alternative approach,
child documents can be compiled by a specific command line
without additional code or specific definitions:
%
\begin{center}
|... -jobname "|\textit{target}|" "|[\textit{flags}]%
|\includeonly{|\textit{dest}|}\input{|\textit{main}|}"|
\end{center}
%

%%%%%%%%%%%%%%%%%%%%%%%%%%%%%%%%%%%%%%%%%%%%%%%%%%%%%%%%%%%%%%%%%%%%%%%%%%%%%%%%
%%%%%%%%%%%%%%%%%%%%%%%%%%%%%%%%%%%%%%%%%%%%%%%%%%%%%%%%%%%%%%%%%%%%%%%%%%%%%%%%
\section{Information}

%%%%%%%%%%%%%%%%%%%%%%%%%%%%%%%%%%%%%%%%%%%%%%%%%%%%%%%%%%%%%%%%%%%%%%%%%%%%%%%%
\subsection{Copyright}

Copyright \copyright{} 2017--2018 Niklas Beisert

This work may be distributed and/or modified under the
conditions of the \LaTeX{} Project Public License, either version 1.3
of this license or (at your option) any later version.
The latest version of this license is in
  \url{http://www.latex-project.org/lppl.txt}
and version 1.3 or later is part of all distributions of \LaTeX{}
version 2005/12/01 or later.

This work has the LPPL maintenance status `maintained'.

The Current Maintainer of this work is Niklas Beisert.

This work consists of the files |README.txt|, |childdoc.ins| and |childdoc.dtx|
as well as the derived files |childdoc.def|, |cdocsamp.tex|
with |cdocsch1.tex|, |cdocsch2.tex|, |cdocspt3.tex|, |cdocspt4.tex|,
|cdocsdrf.tex|, |cdocsfn1.tex|, |cdocsfn2.tex|
as well as |childdoc.pdf|.

%%%%%%%%%%%%%%%%%%%%%%%%%%%%%%%%%%%%%%%%%%%%%%%%%%%%%%%%%%%%%%%%%%%%%%%%%%%%%%%%
\subsection{Files and Installation}

The package consists of the files:
%
\begin{center}
\begin{tabular}{ll}
    |README.txt|   & readme file \\
    |childdoc.ins| & installation file \\
    |childdoc.dtx| & source file \\
    |childdoc.def| & definition file \\
    |cdocsamp.tex| & sample main file \\
    |cdocsch1.tex| & sample include file \\
    |cdocsch2.tex| & sample include file \\
    |cdocspt3.tex| & sample part file \\
    |cdocspt4.tex| & sample part file \\
    |cdocsdrf.tex| & sample redirection file \\
    |cdocsfn1.tex| & sample redirection file \\
    |cdocsfn2.tex| & sample redirection file \\
    |childdoc.pdf| & manual
\end{tabular}
\end{center}
%
The distribution consists of the files
|README.txt|, |childdoc.ins| and |childdoc.dtx|.
%
\begin{itemize}
\item
Run (pdf)\LaTeX{} on |childdoc.dtx|
to compile the manual |childdoc.pdf| (this file).
\item
Run \LaTeX{} on |childdoc.ins| to create the definitions file |childdoc.def|
and the sample |cdocsamp.tex| with include files
|cdocsch1.tex|, |cdocsch2.tex|, |cdocspt3.tex|, |cdocspt4.tex|,
|cdocsdrf.tex|, |cdocsfn1.tex|, |cdocsfn2.tex|.
Then copy the file |childdoc.def| to an appropriate directory of your \LaTeX{}
distribution, e.g.\ \textit{texmf-root}|/tex/latex/childdoc|.
\end{itemize}

%%%%%%%%%%%%%%%%%%%%%%%%%%%%%%%%%%%%%%%%%%%%%%%%%%%%%%%%%%%%%%%%%%%%%%%%%%%%%%%%
\subsection{Related CTAN Packages}

There are several other packages which offer a similar functionality:
%
\begin{itemize}
\item
The packages
\href{http://ctan.org/pkg/docmute}{\textsf{docmute}},
\href{http://ctan.org/pkg/includex}{\textsf{includex}} and
\href{http://ctan.org/pkg/standalone}{\textsf{standalone}}
provide commands to include only the document body of
a child file thus allowing both files to be compiled individually.
\item
The packages \href{http://ctan.org/pkg/subdocs}{\textsf{subdocs}}
and \href{http://ctan.org/pkg/subfiles}{\textsf{subfiles}}
provide structures in which the main and child documents can be
encapsulated and allowing them to be compiled individually.
The inclusion mechanism is different from the conventional |\include|.
\item
The package \href{http://ctan.org/pkg/combine}{\textsf{combine}}
is an elaborate solution to combine several documents into one.
\end{itemize}
%
See also the CTAN topic \href{http://ctan.org/topic/subdocs}{\textsf{subdocs}}
for further related packages.
The present package differs from the above solutions in that
a document structure constructed with the conventional |\include| mechanism
just needs two extra commands at the top of every file
such that all constituent files can be compiled individually.

%%%%%%%%%%%%%%%%%%%%%%%%%%%%%%%%%%%%%%%%%%%%%%%%%%%%%%%%%%%%%%%%%%%%%%%%%%%%%%%%
%\subsection{Feature Suggestions}
%
%The following is a list of features which may be useful for future
%versions of this package:
%%
%\begin{itemize}
%\item
%\ldots
%\end{itemize}

%%%%%%%%%%%%%%%%%%%%%%%%%%%%%%%%%%%%%%%%%%%%%%%%%%%%%%%%%%%%%%%%%%%%%%%%%%%%%%%%
\subsection{Revision History}

%%%%%%%%%%%%%%%%%%%%%%%%%%%%%%%%%%%%%%%%
\paragraph{v2.0:} 2018/12/30

\begin{itemize}
\item
immediate forward processing
\item
added |\childdocby| mechanism
\item
manual restructured
\end{itemize}

%%%%%%%%%%%%%%%%%%%%%%%%%%%%%%%%%%%%%%%%
\paragraph{v1.6:} 2018/01/17

\begin{itemize}
\item
application for development of include files
\item
corrections to manual
\end{itemize}

%%%%%%%%%%%%%%%%%%%%%%%%%%%%%%%%%%%%%%%%
\paragraph{v1.5:} 2017/05/21

\begin{itemize}
\item
more complete structuring introduced
\item
|\childdocof| introduced
\item
|\childdoc| renamed to |\childdocmain|
\item
|\childredirect| renamed to |\childdocforward| and |\childdocforwardprefix|
and functionality expanded
\end{itemize}

%%%%%%%%%%%%%%%%%%%%%%%%%%%%%%%%%%%%%%%%
\paragraph{v1.0:} 2017/04/27

\begin{itemize}
\item
manual and install package
\item
first version published on CTAN
\end{itemize}

%%%%%%%%%%%%%%%%%%%%%%%%%%%%%%%%%%%%%%%%
\paragraph{v0.6:} 2017/04/26

\begin{itemize}
\item
redirection mechanism added
\end{itemize}

%%%%%%%%%%%%%%%%%%%%%%%%%%%%%%%%%%%%%%%%
\paragraph{v0.5:} 2017/04/26

\begin{itemize}
\item
functionality in definition file
\end{itemize}


%%%%%%%%%%%%%%%%%%%%%%%%%%%%%%%%%%%%%%%%%%%%%%%%%%%%%%%%%%%%%%%%%%%%%%%%%%%%%%%%
%%%%%%%%%%%%%%%%%%%%%%%%%%%%%%%%%%%%%%%%%%%%%%%%%%%%%%%%%%%%%%%%%%%%%%%%%%%%%%%%
%%%%%%%%%%%%%%%%%%%%%%%%%%%%%%%%%%%%%%%%%%%%%%%%%%%%%%%%%%%%%%%%%%%%%%%%%%%%%%%%
\appendix

\settowidth\MacroIndent{\rmfamily\scriptsize 000\ }

 \DocInput{childdoc.dtx}

\end{document}
%</driver>
% \fi
%
% %%%%%%%%%%%%%%%%%%%%%%%%%%%%%%%%%%%%%%%%%%%%%%%%%%%%%%%%%%%%%%%%%%%%%%%%%%%%%%
% %%%%%%%%%%%%%%%%%%%%%%%%%%%%%%%%%%%%%%%%%%%%%%%%%%%%%%%%%%%%%%%%%%%%%%%%%%%%%%
% \section{Sample}
%\iffalse
%<*samplemain>
%\fi
%
% The following presents a sample document
% with two chapters, two parts, a title page,
% a compile flag as well as three forwarding files to set the flag.
% It consists of eight |.tex| files:
% \begin{center}
% \begin{tabular}{ll}
% |cdocsamp.tex|&main file\\
% |cdocsch1.tex|&include file for chapter 1\\
% |cdocsch2.tex|&include file for chapter 2\\
% |cdocspt3.tex|&include file for part 3\\
% |cdocspt4.tex|&include file for part 4\\
% |cdocsdrf.tex|&forwarding file for main file in draft mode\\
% |cdocsfi1.tex|&forwarding file for final version of chapter 1\\
% |cdocsfi2.tex|&forwarding file for final version of chapter 2\\
% \end{tabular}
% \end{center}
% Each of the eight files can be compiled directly by the \LaTeX{} compiler.
%
% %%%%%%%%%%%%%%%%%%%%%%%%%%%%%%%%%%%%%%
% \paragraph{Main File.}
%
% The main file is called |cdocsamp.tex|.
%
% Load the \textsf{childdoc} definitions and
% declare the filename for the main document:
%    \begin{macrocode}
\input{childdoc.def}
\childdocmain{}
%    \end{macrocode}

% Optional override for |\version| flag:
%    \begin{macrocode}
%%\ifchilddoc\else\providecommand{\version}{draft}\fi
%    \end{macrocode}

% Define the default values for the |\version| flag
% (|final| for the main file and |draft| for childs):
%    \begin{macrocode}
\ifchilddoc
\providecommand{\version}{draft}
\else
\providecommand{\version}{final}
\fi
%    \end{macrocode}

% Load the standard document class:
%    \begin{macrocode}
\documentclass[12pt]{article}
%    \end{macrocode}

% Start the document body:
%    \begin{macrocode}
\begin{document}
%    \end{macrocode}

% Declare a title page.
% Print title, part of document being processed and version flag:
%    \begin{macrocode}
\addtocounter{page}{-1}
\begin{center}
{\LARGE\bfseries{}childdoc example\par}
\vspace{1cm}
\ifchilddoc
\ifchilddocmanual part\else chapter\fi:
`\childdocname' of `\childdocjob'\par
\else
main document: `\childdocjob'\par
\fi
version: \version\par
\end{center}
\newpage
%    \end{macrocode}

% Manually include selected file,
% otherwise process as usual:
%    \begin{macrocode}
\ifchilddocmanual
\section*{part `\childdocname'}
\input{\childdocname}
\else
%    \end{macrocode}

% Include the two chapters:
%    \begin{macrocode}
\include{cdocsch1}
\include{cdocsch2}
%    \end{macrocode}

% Include the two parts unless only chapters should be displayed:
%    \begin{macrocode}
\ifchilddoc\else
\section{part three}
\input{cdocspt3}
\section{part four}
\input{cdocspt4}
\fi
%    \end{macrocode}

% Process as usual until here:
%    \begin{macrocode}
\fi
%    \end{macrocode}

% End of document body:
%    \begin{macrocode}
\end{document}
%    \end{macrocode}
%\iffalse
%</samplemain>
%\fi
%
% %%%%%%%%%%%%%%%%%%%%%%%%%%%%%%%%%%%%%%
% \paragraph{Chapter Include Files.}
%
% The include files are called |cdocsch1.tex| and |cdocsch2.tex|.
%
%\iffalse
%<*samplechap1|samplechap2>
%\fi

% Optional override for |\version| flag:
%    \begin{macrocode}
%%\providecommand{\version}{final}
%    \end{macrocode}

% Include the main document:
%    \begin{macrocode}
\input{childdoc.def}
\childdocof{cdocsamp}
%    \end{macrocode}

%\iffalse
%</samplechap1|samplechap2>
%\fi
%
%\iffalse
%<*samplechap1>
%\fi
% Some text for chapter 1:
%    \begin{macrocode}
\section{one}
some text in chapter one
%    \end{macrocode}

%\iffalse
%</samplechap1>
%\fi
% Some text for chapter 2:
%\iffalse
%<*samplechap2>
%\fi
%    \begin{macrocode}
\section{two}
more text in chapter two
%    \end{macrocode}

%\iffalse
%</samplechap2>
%\fi
%
% %%%%%%%%%%%%%%%%%%%%%%%%%%%%%%%%%%%%%%
% \paragraph{Part Include Files.}
%
% The include files are called |cdocspt3.tex| and |cdocspt4.tex|.
%
%\iffalse
%<*samplepart3|samplepart4>
%\fi

% Optional override for |\version| flag:
%    \begin{macrocode}
%%\providecommand{\version}{final}
%    \end{macrocode}

% Include the main document:
%    \begin{macrocode}
\input{childdoc.def}
\childdocby{cdocsamp}
%    \end{macrocode}

%\iffalse
%</samplepart3|samplepart4>
%\fi
%
%\iffalse
%<*samplepart3>
%\fi
% Some text for part 3:
%    \begin{macrocode}
some text in part three
%    \end{macrocode}

%\iffalse
%</samplepart3>
%\fi
% Some text for part 4:
%\iffalse
%<*samplepart4>
%\fi
%    \begin{macrocode}
more text in part four
%    \end{macrocode}

%\iffalse
%</samplepart4>
%\fi
%
% %%%%%%%%%%%%%%%%%%%%%%%%%%%%%%%%%%%%%%
% \paragraph{Forwarding for a Complete Draft.}
%
% The following forwarding file |cdocsdrf.tex|
% compiles the main document in draft mode:
%\iffalse
%<*sampledraft>
%\fi
%    \begin{macrocode}
\def\version{draft}
\input{childdoc.def}
\childdocforward{cdocsamp}
%    \end{macrocode}

%\iffalse
%</sampledraft>
%\fi
%
% %%%%%%%%%%%%%%%%%%%%%%%%%%%%%%%%%%%%%%
% \paragraph{Forwarding for Final Version of the Chapters.}
%
% The following forwarding files |cdocsfn1.tex| and |cdocsfn2.tex|
% (with identical content)
% compile the final versions of the child documents
% |cdocsch1.tex| and |cdocsch2.tex|, respectively:
%\iffalse
%<*samplefinal>
%\fi
%    \begin{macrocode}
\def\version{final}
\input{childdoc.def}
\childdocforwardprefix[cdocsamp]{cdocsfn}{cdocsch}
%    \end{macrocode}

%\iffalse
%</samplefinal>
%\fi
%
% %%%%%%%%%%%%%%%%%%%%%%%%%%%%%%%%%%%%%%
% \paragraph{Command Line Processing.}
%
% The following three command lines generate the output files
% |cdocscld|, |cdocscl1| and |cdocscl2|
% which should be identical to
% |cdocsdrf|, |cdocsch1| and |cdocsfn2|, respectively:
% \begin{center}
% \begin{tabular}{l}
% |latex -jobname cdocscld \|\\
% |  "\def\version{draft}\input{childdoc.def}\childdocforward{cdocsamp}"|\\
% |latex -jobname cdocscl1 \|\\
% |  "\input{childdoc.def}\childdocforward[cdocsamp]{cdocsch1}"|\\
% |latex -jobname cdocscl2 \|\\
% |  "\def\version{final}\input{childdoc.def}\childdocforward{cdocsch2}"|
% \end{tabular}
% \end{center}
% Note that the trailing backslash on each first line
% merely continues the input to the second line
% (for convenient cut ant paste).
% Furthermore, the command |latex| can be replaced by any
% of its alternative versions such as |pdflatex|.
%
% %%%%%%%%%%%%%%%%%%%%%%%%%%%%%%%%%%%%%%%%%%%%%%%%%%%%%%%%%%%%%%%%%%%%%%%%%%%%%%
% %%%%%%%%%%%%%%%%%%%%%%%%%%%%%%%%%%%%%%%%%%%%%%%%%%%%%%%%%%%%%%%%%%%%%%%%%%%%%%
% \section{Implementation}
%\iffalse
%<*package>
%\fi
%
% This section describes the definitions file |childdoc.def|.

% The definitions cannot be loaded using |\usepackage| or |\RequirePackage|
% which has a mechanism to prevent loading a style file more than once.
% When loading the definitions by means of |\input|
% multiple instances have to be prevented manually:
%\iffalse
%This code needs to be before the `\ProvidesFile' directive
%which is defined at the beginning of this file.
%Therefore it is also placed there and commented out here.
%</package>
%<*discard>
%\fi
%    \begin{macrocode}
\ifdefined\childdocmain\endinput\fi
%    \end{macrocode}
%\iffalse
%</discard>
%<*package>
%\fi
%
% \macro{\ifchilddoc}
% \macro{\ifchilddocmanual}
% The conditional |\ifchilddoc| tells whether a
% child (true) or main (false) document is being compiled.
% The conditional |\ifchilddocmanual| tells whether
% the |\includeonly| mechanism is used (false) or
% the selection of child files must be performed manually (true).
% The definitions initialise to false:
%    \begin{macrocode}
\newif\ifchilddoc
\newif\ifchilddocmanual
%    \end{macrocode}

% \macro{\childdocname}
% \macro{\childdocjob}
% The macro |\childdocname| stores the name of the main document
% to be compiled. The macro |\childdocjob| stores the name of
% the document on which the \LaTeX{} compiler was originally invoked.
% The content of |\jobname| cannot be compared
% to filenames specified in the source due to different catcodes.
% The following code rescans |\jobname|, stores the result
% in |\childdocname| and saves a copy in |\childdocjob|:
%    \begin{macrocode}
\edef\childdocname{\scantokens\expandafter{\jobname\noexpand}}
\let\childdocjob\childdocname
%    \end{macrocode}

% \macro{\childdocdisable}
% The macro |\childdocdisable| prevents the main file
% from being processed more than once.
% At this stage, the main document command |\childdocmain|
% is assumed to be called once again where it should do nothing.
% Any subsequent call to it should prevent
% a secondary processing of the main document
% It overwrites the forwarding commands
% |\childdocof| and |\childdocforward|
% with empty macros to prevent further inclusions of the main document:
%    \begin{macrocode}
\newcommand{\childdocdisable}
{
  \renewcommand{\childdocmain}[1]{\renewcommand{\childdocmain}[1]{\endinput}}
  \renewcommand{\childdocof}[1]{}
  \renewcommand{\childdocby}[2][]{}
  \renewcommand{\childdocforward}[2][]{}
  \renewcommand{\childdocdisable}{}
}
%    \end{macrocode}

% \macro{\childdocmain}
% The macro |\childdocmain| is to be called at the top of the main file
% with nothing or the main filename (without extension) as argument.
% First, it breaks loops.
% If the argument is not empty and does not match |\childdocname|
% (which is set by the first inclusion of |childdoc.def|),
% |\ifchilddoc| is set to true, |\includeonly| is applied to the child file
% and |\jobname| is set to the main file
% (for proper handling of |.aux| files):
%    \begin{macrocode}
\newcommand{\childdocmain}[1]
{
  \childdocdisable\childdocmain{}
  \if?#1?\else
    \begingroup
      \def\childdoctmp{#1}
      \ifx\childdoctmp\childdocname
        \def\childdoctmp{}
      \else
        \def\childdoctmp
        {
          \childdoctrue
          \includeonly{\childdocname}
          \def\childdocjob{#1}
          \def\jobname{#1}
        }
      \fi
      \expandafter
    \endgroup
    \childdoctmp
  \fi
}
%    \end{macrocode}

% \macro{\childdocof}
% The command |\childdocof| redirects
% compilation to the main file |#1|.
%    \begin{macrocode}
\newcommand{\childdocof}[1]
{
  \childdocdisable
  \childdoctrue
  \includeonly{\childdocname}
  \def\jobname{#1}
  \def\childdocjob{#1}
  \input{#1}
}
%    \end{macrocode}

% \macro{\childdocby}
% The command |\childdocby| ....
%    \begin{macrocode}
\newcommand{\childdocby}[2][]
{
  \childdocdisable
  \childdoctrue
  \childdocmanualtrue
  \if?#1?\else
    \def\jobname{#2}
  \fi
  \def\childdocjob{#2}
  \input{#2}
  \endinput
}
%    \end{macrocode}

% \macro{\childdocforward}
% The command |\childdocforward| redirects
% compilation to the main file or
% (if the optional argument is given) a child file.
% Parameters are set as if the main file
% or a child file starting with |\childdocof| was compiled.
% Then compilation is handed over to the main file:
%    \begin{macrocode}
\newcommand{\childdocforward}[2][]
{
  \begingroup
    \if?#1?
      \def\childdoctmp
      {
        \def\childdocname{#2}
        \def\childdocjob{#2}
        \def\jobname{#2}
        \input{#2}
        \endinput
      }
    \else
      \def\childdoctmp
      {
        \childdocdisable
        \def\childdocname{#2}
        \childdoctrue
        \includeonly{#2}
        \def\childdocjob{#1}
        \def\jobname{#1}
        \input{#1}
        \endinput
      }
    \fi
    \expandafter
  \endgroup
  \childdoctmp
}
%    \end{macrocode}

% \macro{\childdocforwardprefix}
% The command |\childdocforwardprefix| redirects
% compilation to the main or a child file by means of a pattern.
% The prefix |#1| in the current filename is replaced by |#2|
% and the suffix of the current filename is kept
% (it is assumed that the filename does not contain the substring `|~~~|'
% which is used as a delimiter).
% Compilation is handed over to the new file by |\childdocforward|:
%    \begin{macrocode}
\newcommand{\childdocforwardprefix}[3][]
{
  \begingroup
    \def\childdocextract #2##1~~~{\def\childdoctmp{\childdocforward[#1]{#3##1}}}
    \expandafter\childdocextract\childdocname~~~
    \expandafter
  \endgroup
  \childdoctmp
}
%    \end{macrocode}

% \macro{\childdoc}
% The deprecated macro |\childdoc| is a legacy version of |\childdocmain|:
%    \begin{macrocode}
\newcommand{\childdoc}{\childdocmain}
%    \end{macrocode}

% \macro{\childdocredirect}
% The deprecated macro |\childdocredirect| is a legacy version
% of |\childdocforward| and |\childdocforwardprefix|:
%    \begin{macrocode}
\newcommand{\childdocredirect}[2][]
{
  \begingroup
    \if?#1?
      \def\childdoctmp{\childdocforward{#2}}
    \else
      \def\childdoctmp{\childdocforwardprefix{#1}{#2}}
    \fi
    \expandafter
  \endgroup
  \childdoctmp
}
%    \end{macrocode}

%\iffalse
%</package>
%\fi
%
\endinput
|\\
|\childdocmain{}|\\
\end{tabular}
\end{center}
at the very top of the main \LaTeX{} file,
in particular \emph{before} the |\documentclass| statement!
The argument of |\childdocmain| should be left empty
(but it must be present).

%%%%%%%%%%%%%%%%%%%%%%%%%%%%%%%%%%%%%%%%
\DescribeMacro{\childdocof}
Furthermore, add the commands
\begin{center}
\begin{tabular}{l}
|% \iffalse
%
% childdoc.dtx Copyright (C) 2017-2018 Niklas Beisert
%
% This work may be distributed and/or modified under the
% conditions of the LaTeX Project Public License, either version 1.3
% of this license or (at your option) any later version.
% The latest version of this license is in
%   http://www.latex-project.org/lppl.txt
% and version 1.3 or later is part of all distributions of LaTeX
% version 2005/12/01 or later.
%
% This work has the LPPL maintenance status `maintained'.
%
% The Current Maintainer of this work is Niklas Beisert.
%
% This work consists of the files childdoc.dtx and childdoc.ins
% and the derived files childdoc.def and cdocsamp.tex with
% cdocsch1.tex, cdocsch2.tex, cdocsdrf.tex, cdocsfn1.tex, cdocsfn2.tex.
%
%<package>\ifdefined\childdocmain\endinput\fi
%<package>\ProvidesFile{childdoc.def}[2018/12/30 v2.0 child document driver]
%<samplemain>\ProvidesFile{cdocsamp.tex}[2018/12/30 v2.0 sample for childdoc]
%<*driver>
%\ProvidesFile{childdoc.drv}[2018/12/30 v2.0 childdoc reference manual file]
\PassOptionsToClass{10pt,a4paper}{article}
\documentclass{ltxdoc}

\usepackage[margin=35mm]{geometry}
\usepackage{hyperref}
\usepackage{hyperxmp}
\usepackage[usenames]{color}

\hypersetup{colorlinks=true}
\hypersetup{pdfstartview=FitH}
\hypersetup{pdfpagemode=UseNone}
\hypersetup{pdfsource={}}
\hypersetup{pdflang={en-UK}}
\hypersetup{pdfcopyright={Copyright 2017-2018 Niklas Beisert.
  This work may be distributed and/or modified under the
  conditions of the LaTeX Project Public License, either version 1.3
  of this license or (at your option) any later version.}}
\hypersetup{pdflicenseurl={http://www.latex-project.org/lppl.txt}}
\hypersetup{pdfcontactaddress={ETH Zurich, ITP, HIT K,
  Wolfgang-Pauli-Strasse 27}}
\hypersetup{pdfcontactpostcode={8093}}
\hypersetup{pdfcontactcity={Zurich}}
\hypersetup{pdfcontactcountry={Switzerland}}
\hypersetup{pdfcontactemail={nbeisert@itp.phys.ethz.ch}}
\hypersetup{pdfcontacturl={http://people.phys.ethz.ch/\xmptilde nbeisert/}}

\newcommand{\secref}[1]{\hyperref[#1]{section \ref*{#1}}}

\parskip1ex
\parindent0pt
\let\olditemize\itemize
\def\itemize{\olditemize\parskip0pt}

\begin{document}

\title{The \textsf{childdoc} Package}
\hypersetup{pdftitle={The childdoc Package}}
\author{Niklas Beisert\\[2ex]
  Institut f\"ur Theoretische Physik\\
  Eidgen\"ossische Technische Hochschule Z\"urich\\
  Wolfgang-Pauli-Strasse 27, 8093 Z\"urich, Switzerland\\[1ex]
  \href{mailto:nbeisert@itp.phys.ethz.ch}
  {\texttt{nbeisert@itp.phys.ethz.ch}}}
\hypersetup{pdfauthor={Niklas Beisert}}
\hypersetup{pdfsubject={Manual for the LaTeX2e Package childdoc}}
\date{30 December 2018, \textsf{v2.0}}
\maketitle

\begin{abstract}\noindent
\textsf{childdoc} is a \LaTeXe{} package
that enables the direct compilation
of document sections included by |\include|
to individual files.
\end{abstract}

\begingroup
\parskip0ex
\tableofcontents
\endgroup

%%%%%%%%%%%%%%%%%%%%%%%%%%%%%%%%%%%%%%%%%%%%%%%%%%%%%%%%%%%%%%%%%%%%%%%%%%%%%%%%
%%%%%%%%%%%%%%%%%%%%%%%%%%%%%%%%%%%%%%%%%%%%%%%%%%%%%%%%%%%%%%%%%%%%%%%%%%%%%%%%
\section{Introduction}

\LaTeX{} provides a mechanism to structure a large document (such as a book)
into a main file and several child files (containing the chapters)
using the |\include| command.
This mechanism is beneficial for documents
which span hundreds of pages in order to
make the source file(s) more manageable.
Moreover, compilation can be restricted to
selected child files by means of the |\includeonly| command.
The latter feature can be used to reduce the compilation time while editing
(this was significantly more useful in the earlier days of \LaTeX{})
or to generate a smaller document which is easier to navigate.
Another application of |\includeonly| is to generate
documents consisting of selected parts of the complete document.

However, there are a few drawbacks of the plain |\include| mechanism:
\begin{itemize}
\item
The child files cannot be compiled on their own,
they can only be compiled via the main file.
A naive editing environment
(such as a text editor with an option
to have the current file processed by \LaTeX)
may require one to switch to the main file before compiling;
attempting to compile the child file produces errors.
\item
The main file must be modified (each time)
to adjust the |\includeonly| command
to the present needs. This easily leaves the main file in a messy state.
\item
The generated document will always carry the filename
of the main document. This is inconvenient if
several child files are to be compiled and
to be kept for distribution.
\end{itemize}

The present package provides a simple interface
to make child files individually compilable by \LaTeX{}.
Compiling a child file then has the same effect as compiling
the main file with an |\includeonly| command
to select the appropriate child.
Moreover the generated document will carry the name of the child
rather than the main file.
This resolves all three above issues.

This feature is meant to make the editing of books,
thesis documents and lecture notes somewhat more convenient.
However, the package can also be used efficiently for
composing a series of documents (such as exercise sheets)
which are typically distributed individually.
It then assists the author in generating the individual documents
(potentially in different versions)
as well as a document containing the collected series.
Another application is in developing style files
or other kinds of included material
where compilation of the style file could redirect
to a sample or test file.

%%%%%%%%%%%%%%%%%%%%%%%%%%%%%%%%%%%%%%%%%%%%%%%%%%%%%%%%%%%%%%%%%%%%%%%%%%%%%%%%
%%%%%%%%%%%%%%%%%%%%%%%%%%%%%%%%%%%%%%%%%%%%%%%%%%%%%%%%%%%%%%%%%%%%%%%%%%%%%%%%
\section{Usage}

First of all, the package \textsf{childdoc} is \emph{not} a standard
\LaTeXe{} |.sty| style file! Therefore it needs to be invoked in
a non-standard way.

%%%%%%%%%%%%%%%%%%%%%%%%%%%%%%%%%%%%%%%%%%%%%%%%%%%%%%%%%%%%%%%%%%%%%%%%%%%%%%%%
\subsection{Included Files}
\label{sec:include}

%%%%%%%%%%%%%%%%%%%%%%%%%%%%%%%%%%%%%%%%
\DescribeMacro{\childdocmain}
To use the package, add the commands
\begin{center}
\begin{tabular}{l}
|\input{childdoc.def}|\\
|\childdocmain{}|\\
\end{tabular}
\end{center}
at the very top of the main \LaTeX{} file,
in particular \emph{before} the |\documentclass| statement!
The argument of |\childdocmain| should be left empty
(but it must be present).

%%%%%%%%%%%%%%%%%%%%%%%%%%%%%%%%%%%%%%%%
\DescribeMacro{\childdocof}
Furthermore, add the commands
\begin{center}
\begin{tabular}{l}
|\input{childdoc.def}|\\
|\childdocof{|\textit{main}|}|\\
\end{tabular}
\end{center}
at the top of every child file \textit{child}
which is included by |\include{|\textit{child}|}|
from within the main file
(or at least for those files to be compiled individually).
The argument \textit{main} must be the filename of the main file.

There are a couple of
considerations in setting up the main and child documents:

%%%%%%%%%%%%%%%%%%%%%%%%%%%%%%%%%%%%%%%%
\paragraph{Restrictions.}

Please note the following restrictions:
\begin{itemize}
\item
|\childdocmain| must be called with one argument \textit{main}
to ensure compatibility with earlier version of the package.
It must either be empty (|\childdocmain{}|)
or precisely match the filename of the main file in which it is specified.
See \secref{sec:detection} for further information.
\item
The filename \textit{main} must be specified without the |.tex| extension.
\item
The filename \textit{main} is case sensitive
(even in case-insensitive file systems)
due to internal string comparison.
\item
The argument \textit{main} should be fully expanded, it cannot be a macro.
\item
Subdirectories and special characters should be avoided in filenames.
\item
The command |\childdocmain{|\textit{main}|}| must be followed by a whitespace.
It should not be followed immediately by another command
or by a comment mark `|%|'.
This is because the \TeX{} parser reads the token immediately following
the argument of |\childdocmain| and puts it
at the beginning of every child section;
however, a white\-space is ignored.
\end{itemize}

%%%%%%%%%%%%%%%%%%%%%%%%%%%%%%%%%%%%%%%%
\paragraph{Content of Main File.}

It is advisable to place all content in the child files included by |\include|.
Any output contained in the main file will appear in all child documents
unless suppressed manually;
it cannot be suppressed automatically by the |\includeonly| directive
and thus should normally be avoided.
A method to include some content in the main file
by means of conditional processing is described in \secref{sec:conditional}.

%%%%%%%%%%%%%%%%%%%%%%%%%%%%%%%%%%%%%%%%
\paragraph{Page Numbering.}

When only a part of the document is compiled,
the appropriate numbering of pages
(as well as other status parameters)
is determined from the |.aux| files.
The latter contain information from previous passes.
However this information needs to propagate through
all intermediate child documents.
Therefore the page numbering in child documents may well
be inconsistent until the complete document is compiled at least once.

A useful (if unconventional) way to always ensure a consistent
page numbering is to restart the numbering in each child document
and denote the pages by `\textit{child}|.|\textit{page}'
where \textit{child} represents the chapter/section number of the child file.
This can be achieved by the command
|\numberwithin{page}{|\textit{child}|}|
of the \textsf{amsmath} package
where \textit{child} can be |chapter| or |section|
depending on the chosen structuring.
Alternatively, one can modify the macro |\thepage| appropriately
and reset the counter |page| at the start of each child file.

%%%%%%%%%%%%%%%%%%%%%%%%%%%%%%%%%%%%%%%%%%%%%%%%%%%%%%%%%%%%%%%%%%%%%%%%%%%%%%%%
\subsection{Conditional Processing}
\label{sec:conditional}

The package provides a mechanism to compile different versions
of a document. To customise the versions further some conditional processing
can come in handy to distinguish which version is being compiled.
The package provides two macros to describe the compilation context:

%%%%%%%%%%%%%%%%%%%%%%%%%%%%%%%%%%%%%%%%
\DescribeMacro{\ifchilddoc}
The conditional |\ifchilddoc| distinguishes between the compilation of
child documents and the main document:
%
\begin{center}
|\ifchilddoc |\textit{child-code}| |[|\||else |\textit{main-code}]| \||fi|
\end{center}

%%%%%%%%%%%%%%%%%%%%%%%%%%%%%%%%%%%%%%%%
\DescribeMacro{\childdocname}
\DescribeMacro{\childdocjob}
The macro |\childdocname| contains the filename (without extension)
of the main or child file being processed.
Note that |\childdocjob| will always contain the name of the main file.

%%%%%%%%%%%%%%%%%%%%%%%%%%%%%%%%%%%%%%%%
\paragraph{Title Page.}

Conditional processing can be used to include a title or banner page
in the main document when proper precautions are taken.
Importantly, the code in the main file should ensure that the page counter
(as well as other status parameters which are stored in the |.aux| files)
takes the same value after the conditional processing.
Otherwise the page numbers may take divergent values
depending on which part is compiled.

For example, a title page could be declared by:
%
\begin{center}
\begin{tabular}{l}
|\ifchilddoc\||else|\\
|\addtocounter{page}{-1}|\\
\textit{code for title page}\\
|\newpage|\\
|\||fi|
\end{tabular}
\end{center}
%
A banner page for the child documents can be generated by:
%
\begin{center}
\begin{tabular}{l}
|\ifchilddoc|\\
|\addtocounter{page}{-1}|\\
\textit{code for banner page}\\
|\newpage|\\
|\||fi|
\end{tabular}
\end{center}
%
Here one could write a message such as:
\begin{center}
|This is the part \childdocname{} of \childdocjob{}.|
\end{center}

%%%%%%%%%%%%%%%%%%%%%%%%%%%%%%%%%%%%%%%%%%%%%%%%%%%%%%%%%%%%%%%%%%%%%%%%%%%%%%%%
\subsection{Flags}
\label{sec:flags}

The package makes it easy to generate different versions
of the main or child documents.
To this end compilation flags can be defined
and assigned different default values.
They will be particularly useful in conjunction
with the forwarding mechanism described in \secref{sec:forward}.

For example, it may be useful to have a flag |\version|
which can be set to |draft| or |final|.
The document source will contain some conditional code
depending on the value of |\version|.
Suppose further, the flag should default to |final| for the main file
and to |draft| for child files
which is a natural assignment for editing the document.
This is achieved by placing the following code
in the preamble of the main document
(below the |\childdocmain| directive):
%
\begin{center}
\begin{tabular}{l}
|\ifchilddoc|\\
|\providecommand{\version}{draft}|\\
|\||else|\\
|\providecommand{\version}{final}|\\
|\||fi|
\end{tabular}
\end{center}
%
The definition by |\providecommand| makes sure
that previous definitions are not overwritten.
Further statements |\providecommand{\version}{...}|
can thus be added before the above code to override it.

For the main file, one might add a line
(between |\childdocmain| and the above block)
%
\begin{center}
|%\ifchilddoc\||else\providecommand{\version}{draft}\||fi|
\end{center}
%
which can be uncommented to produce a draft version.
Likewise one can add a line to the very top of a child file
(above the |\childdocof{|\textit{main}|}| directive)
%
\begin{center}
|%\providecommand{\version}{final}|
\end{center}
%
which can be uncommented to produce the final version of this child document.

%%%%%%%%%%%%%%%%%%%%%%%%%%%%%%%%%%%%%%%%%%%%%%%%%%%%%%%%%%%%%%%%%%%%%%%%%%%%%%%%
\subsection{Forwarding}
\label{sec:forward}

Different versions of the main or child documents
using compilation flags as described in \secref{sec:flags}
can be (permanently) stored in different files
for convenient compilation, viewing and distribution.
To this end, the package defines a command
to pass on compilation to a different file:

%%%%%%%%%%%%%%%%%%%%%%%%%%%%%%%%%%%%%%%%
\DescribeMacro{\childdocforward}
The command |\childdocforward| redirects processing to
another source file:
%
\begin{center}
\begin{tabular}{l}
|\input{childdoc.def}|\\
|\childdocforward[|\textit{main}|]{|\textit{dest}|}|\\
\end{tabular}
\end{center}
%
The argument \textit{dest} is the destination file
(without extension).
It should be the main file or one of the child files.
Note that further \textsf{childdoc} directives
such as |\childdocof| and |\childdocforward|
in the indicated file will be processed in this form.
The optional argument \textit{main}
passes on directly to the main file \textit{main}
while pretending to compile the child \textit{dest}.
This form behaves as if \textit{dest}
issues |\childdocof{|\textit{main}|}| right away,
and no further \textsf{childdoc} directives will be processed.

%%%%%%%%%%%%%%%%%%%%%%%%%%%%%%%%%%%%%%%%
\DescribeMacro{\...prefix}
In the alternative form |\childdocforwardprefix|,
%
\begin{center}
\begin{tabular}{l}
|\input{childdoc.def}|\\
|\childdocforwardprefix[|\textit{main}|]{|\textit{prefix}|}{|\textit{dest}|}|
\end{tabular}
\end{center}
%
the destination file is determined by a pattern
depending on the current file:
To make this work, the current file must be called
`{\textit{prefix}\hspace{0.2em}\textit{suffix}}'
with \textit{prefix} matching precisely the argument.
Processing is then passed on to the file
`{\textit{dest}\hspace{0.2em}\textit{suffix}}'.
Surely, the same effect is achieved by
directly specifying the
argument `{\textit{dest}\hspace{0.2em}\textit{suffix}}'
in the first form.
However, that requires to set up a different file
for each child. With the alternative form of the command
all these files can have exactly the same content
which simplifies setting them up and maintaining them.

For example, the following file |draft.tex|
with a compilation flag |\version| as described in \secref{sec:flags}
compiles the main document as a draft:
%
\begin{center}
\begin{tabular}{l}
|\def\version{draft}|\\
|\input{childdoc.def}|\\
|\childdocforward{|\textit{main}|}|
\end{tabular}
\end{center}
%
Likewise, the following files |final|\textit{nn}|.tex|
compile the final version of the child document
|child|\textit{nn}|.tex|:
%
\begin{center}
\begin{tabular}{l}
|\def\version{final}|\\
|\input{childdoc.def}|\\
|\childdocforwardprefix{final}{child}|
\end{tabular}
\end{center}
%

Note that when several versions of a main file and/or of each child file
are to be generated, it may be convenient to set up a |Makefile| or
shell script to automatise the process.

%%%%%%%%%%%%%%%%%%%%%%%%%%%%%%%%%%%%%%%%%%%%%%%%%%%%%%%%%%%%%%%%%%%%%%%%%%%%%%%%
\subsection{Command Line Processing}
\label{sec:commandline}

The effect of redirection files can also be achieved by invoking
the \LaTeX{} compiler with a more elaborate command line.
Most conveniently this should be done as part
of a shell script or a |Makefile|.

When using \textsf{childdoc} in the main file, the following
command lines effectively perform a redirection
(note that depending on the shell being used,
backslashes may have to be doubled: `|\|' $\to$ `|\\|'):
%
\begin{center}
|... -jobname "|\textit{target}|" |\\|"|[\textit{flags}]%
|\input{childdoc.def}\childdocforward[|\textit{main}|]{|\textit{dest}|}"|
\end{center}
%
Here \textit{target} is the name of the output file,
\textit{main} is the name of the main file
and \textit{dest} is the name of the main or child file to be processed
(all filenames without extensions).
The optional argument \textit{main} can be omitted
if \textit{main} matches \textit{dest}.
Optionally, compilation \textit{flags} can be defined via |\def| commands.
This command line makes the \TeX{} engine believe
it is compiling the file \textit{target}
whose content is specified as the latter parameter.
The provided code then forwards the processing to
\textit{main} or \textit{dest} as described in \secref{sec:forward}.

%%%%%%%%%%%%%%%%%%%%%%%%%%%%%%%%%%%%%%%%%%%%%%%%%%%%%%%%%%%%%%%%%%%%%%%%%%%%%%%%
\subsection{Include by Input}
\label{sec:input}

Including child documents by |\include| has some restrictions by design.
Most notably, the content of a child document always occupies
its own set of pages; pages cannot be shared between child documents.
Usually, this behaviour makes perfect sense
because each child document contain an essential part of the document.
However, in some situations it may be desirable to compose
a document from a collection of parts
without having mandatory page breaks between then.
For this case, the package
provides a mechanism to include parts
by |\input| which can also be processed individually.
However, by construction this mechanism
requires manual handling of the content to be output.

%%%%%%%%%%%%%%%%%%%%%%%%%%%%%%%%%%%%%%%%
\DescribeMacro{\ifchilddocmanual}
The main file should be prepared as usual, see \secref{sec:include}.
However, the document body must make a distinction
between processing of an individual part and of the main document, e.g.:
%
\begin{center}
\begin{tabular}{l}
|\ifchilddocmanual|\\
|\input{\childdocname}|\\
|\||else|\\
\textit{document body with }|\input{|\textit{part}|}|\\
|\||fi|
\end{tabular}
\end{center}
%
The conditional |\ifchilddocmanual| is true whenever
a part to be included by |\input| is being compiled,
and the name of the part is stored in |\childdocname|.

%%%%%%%%%%%%%%%%%%%%%%%%%%%%%%%%%%%%%%%%
\DescribeMacro{\childdocby}
Each part to be included by |\input| should start with:
%
\begin{center}
\begin{tabular}{l}
|\input{childdoc.def}|\\
|\childdocby{|\textit{main}|}|\\
\end{tabular}
\end{center}
%
The directive |\childdocby| is similar to |\childdocof|
described in \secref{sec:include},
but the subsequent selection of content must be done manually.
To that end, both |\ifchilddoc| and |\ifchilddocmanual|
will be true upon processing of a part,
and the name of the part is stored in |\childdocname|.
Note that |\jobname| will be set to the filename of the current part
so that each part receives an individual |.aux| file
that does not interfere with the |.aux| file(s) of the main document.
This behaviour can be altered by the alternative form
|\childdocby[*]{|\textit{main}|}| (with a non-empty optional argument)
which uses the |.aux| file of the main document
by setting |\jobname| to \textit{main}.

%%%%%%%%%%%%%%%%%%%%%%%%%%%%%%%%%%%%%%%%%%%%%%%%%%%%%%%%%%%%%%%%%%%%%%%%%%%%%%%%
\subsection{Driver Development}
\label{sec:driver}

The \textsf{childdoc} mechanism can also be use for the development
of definition files such as \LaTeX{} styles or classes.
This case differs from the above setup with multiple parts
included by |\include| in that no |\includeonly| should be invoked.
This can be achieved by starting the include file
(before |\ProvidesPackage|) with:
%
\begin{center}
\begin{tabular}{l}
|\input{childdoc.def}|\\
|\childdocforward{|\textit{main}|}|\\
\end{tabular}
\end{center}
%
or alternatively with:
%
\begin{center}
\begin{tabular}{l}
|\input{childdoc.def}|\\
|\childdocby{|\textit{main}|}|\\
\end{tabular}
\end{center}
%
Both forms have slightly different effects as described above.
The main file is prepared as usual, see \secref{sec:include}.

%%%%%%%%%%%%%%%%%%%%%%%%%%%%%%%%%%%%%%%%%%%%%%%%%%%%%%%%%%%%%%%%%%%%%%%%%%%%%%%%
\subsection{Legacy Detection}
\label{sec:detection}

The directive |\childdocmain| in the main file can detect
whether the complete document or merely a child is to be compiled
even without using the directive |\childdocof|.
This method is deprecated because it is less robust
and there is no compelling reason to use it;
it is merely provided for backward compatibility
and it may be removed in future versions.

If the detection mechanism is to be used,
it is mandatory to correctly specify
the filename of the main file as the argument of |\childdocmain|:
%
\begin{center}
\begin{tabular}{l}
|\input{childdoc.def}|\\
|\childdocmain{|\textit{main}|}|\\
\end{tabular}
\end{center}
%
If |\jobname| does not match the argument \textit{main} of |\childdocmain|,
it is assumed that |\jobname| points to the child file to be compiled.
When using |\childdocmain| with the main file specified as argument,
it suffices to start a child file
with just |\input{|\textit{main}|}|
without loading of the package and using |\childdocof|.
If instead all processing is done
with the appropriate \textsf{childdoc} directives,
the argument of \textit{main} of |\childdocmain| can be empty.

An alternative version of the command line processing described
in \secref{sec:commandline} using the detection mechanism reads:
%
\begin{center}
|... -jobname "|\textit{target}|" "|[\textit{flags}]%
[|\def\jobname{|\textit{dest}|}|]|\input{|\textit{main}|}"|
\end{center}

%%%%%%%%%%%%%%%%%%%%%%%%%%%%%%%%%%%%%%%%%%%%%%%%%%%%%%%%%%%%%%%%%%%%%%%%%%%%%%%%
\subsection{Manual Code}
\label{sec:manual}

In case one cannot be certain whether the definitions file |childdoc.def|
is installed on the target \TeX{} distribution
and one prefers not to ship it,
it is conceivable to paste a few relevant commands into the sources.

To that end, drop all statements |\input{childdoc.def}|
and perform the replacements as outlined below.
Instead of |\childdocmain{|\textit{main}|}| add the following code
to the top of the main file:
%
\begin{center}
\begin{tabular}{l}
|\||ifdefined\childdocname\endinput\||fi\newif\ifchilddoc|\\
|\edef\childdocname{\scantokens\expandafter{\jobname\noexpand}}|\\
|\def\childdocmain{|\textit{main}|}\||ifx\childdocmain\childdocname\||else|\\
|\childdoctrue\includeonly{\childdocname}\let\jobname\childdocmain\||fi|\\
\end{tabular}
\end{center}
%
Instead of |\childdocof{|\textit{main}|}| just include the main file
at the top of each child file:
%
\begin{center}
|\input{|\textit{main}|}|
\end{center}
%
A simple redirection |\childdocforward{|\textit{dest}|}| is achieved by:
%
\begin{center}
|\def\jobname{|\textit{dest}|}\input{\jobname}|
\end{center}
%
The redirection with prefix
|\childdocforwardprefix[|\textit{prefix}|]{|\textit{dest}|}|
is accomplished by:
%
\begin{center}
\begin{tabular}{l}
|{\edef\jobname{\scantokens\expandafter{\jobname\noexpand}}|\\
|\def\redirectjob |\textit{prefix}|#1~~~{\gdef\jobname{|\textit{dest}|#1}}|\\
|\expandafter\redirectjob\jobname~~~}\input{\jobname}|
\end{tabular}
\end{center}

In an alternative approach,
child documents can be compiled by a specific command line
without additional code or specific definitions:
%
\begin{center}
|... -jobname "|\textit{target}|" "|[\textit{flags}]%
|\includeonly{|\textit{dest}|}\input{|\textit{main}|}"|
\end{center}
%

%%%%%%%%%%%%%%%%%%%%%%%%%%%%%%%%%%%%%%%%%%%%%%%%%%%%%%%%%%%%%%%%%%%%%%%%%%%%%%%%
%%%%%%%%%%%%%%%%%%%%%%%%%%%%%%%%%%%%%%%%%%%%%%%%%%%%%%%%%%%%%%%%%%%%%%%%%%%%%%%%
\section{Information}

%%%%%%%%%%%%%%%%%%%%%%%%%%%%%%%%%%%%%%%%%%%%%%%%%%%%%%%%%%%%%%%%%%%%%%%%%%%%%%%%
\subsection{Copyright}

Copyright \copyright{} 2017--2018 Niklas Beisert

This work may be distributed and/or modified under the
conditions of the \LaTeX{} Project Public License, either version 1.3
of this license or (at your option) any later version.
The latest version of this license is in
  \url{http://www.latex-project.org/lppl.txt}
and version 1.3 or later is part of all distributions of \LaTeX{}
version 2005/12/01 or later.

This work has the LPPL maintenance status `maintained'.

The Current Maintainer of this work is Niklas Beisert.

This work consists of the files |README.txt|, |childdoc.ins| and |childdoc.dtx|
as well as the derived files |childdoc.def|, |cdocsamp.tex|
with |cdocsch1.tex|, |cdocsch2.tex|, |cdocspt3.tex|, |cdocspt4.tex|,
|cdocsdrf.tex|, |cdocsfn1.tex|, |cdocsfn2.tex|
as well as |childdoc.pdf|.

%%%%%%%%%%%%%%%%%%%%%%%%%%%%%%%%%%%%%%%%%%%%%%%%%%%%%%%%%%%%%%%%%%%%%%%%%%%%%%%%
\subsection{Files and Installation}

The package consists of the files:
%
\begin{center}
\begin{tabular}{ll}
    |README.txt|   & readme file \\
    |childdoc.ins| & installation file \\
    |childdoc.dtx| & source file \\
    |childdoc.def| & definition file \\
    |cdocsamp.tex| & sample main file \\
    |cdocsch1.tex| & sample include file \\
    |cdocsch2.tex| & sample include file \\
    |cdocspt3.tex| & sample part file \\
    |cdocspt4.tex| & sample part file \\
    |cdocsdrf.tex| & sample redirection file \\
    |cdocsfn1.tex| & sample redirection file \\
    |cdocsfn2.tex| & sample redirection file \\
    |childdoc.pdf| & manual
\end{tabular}
\end{center}
%
The distribution consists of the files
|README.txt|, |childdoc.ins| and |childdoc.dtx|.
%
\begin{itemize}
\item
Run (pdf)\LaTeX{} on |childdoc.dtx|
to compile the manual |childdoc.pdf| (this file).
\item
Run \LaTeX{} on |childdoc.ins| to create the definitions file |childdoc.def|
and the sample |cdocsamp.tex| with include files
|cdocsch1.tex|, |cdocsch2.tex|, |cdocspt3.tex|, |cdocspt4.tex|,
|cdocsdrf.tex|, |cdocsfn1.tex|, |cdocsfn2.tex|.
Then copy the file |childdoc.def| to an appropriate directory of your \LaTeX{}
distribution, e.g.\ \textit{texmf-root}|/tex/latex/childdoc|.
\end{itemize}

%%%%%%%%%%%%%%%%%%%%%%%%%%%%%%%%%%%%%%%%%%%%%%%%%%%%%%%%%%%%%%%%%%%%%%%%%%%%%%%%
\subsection{Related CTAN Packages}

There are several other packages which offer a similar functionality:
%
\begin{itemize}
\item
The packages
\href{http://ctan.org/pkg/docmute}{\textsf{docmute}},
\href{http://ctan.org/pkg/includex}{\textsf{includex}} and
\href{http://ctan.org/pkg/standalone}{\textsf{standalone}}
provide commands to include only the document body of
a child file thus allowing both files to be compiled individually.
\item
The packages \href{http://ctan.org/pkg/subdocs}{\textsf{subdocs}}
and \href{http://ctan.org/pkg/subfiles}{\textsf{subfiles}}
provide structures in which the main and child documents can be
encapsulated and allowing them to be compiled individually.
The inclusion mechanism is different from the conventional |\include|.
\item
The package \href{http://ctan.org/pkg/combine}{\textsf{combine}}
is an elaborate solution to combine several documents into one.
\end{itemize}
%
See also the CTAN topic \href{http://ctan.org/topic/subdocs}{\textsf{subdocs}}
for further related packages.
The present package differs from the above solutions in that
a document structure constructed with the conventional |\include| mechanism
just needs two extra commands at the top of every file
such that all constituent files can be compiled individually.

%%%%%%%%%%%%%%%%%%%%%%%%%%%%%%%%%%%%%%%%%%%%%%%%%%%%%%%%%%%%%%%%%%%%%%%%%%%%%%%%
%\subsection{Feature Suggestions}
%
%The following is a list of features which may be useful for future
%versions of this package:
%%
%\begin{itemize}
%\item
%\ldots
%\end{itemize}

%%%%%%%%%%%%%%%%%%%%%%%%%%%%%%%%%%%%%%%%%%%%%%%%%%%%%%%%%%%%%%%%%%%%%%%%%%%%%%%%
\subsection{Revision History}

%%%%%%%%%%%%%%%%%%%%%%%%%%%%%%%%%%%%%%%%
\paragraph{v2.0:} 2018/12/30

\begin{itemize}
\item
immediate forward processing
\item
added |\childdocby| mechanism
\item
manual restructured
\end{itemize}

%%%%%%%%%%%%%%%%%%%%%%%%%%%%%%%%%%%%%%%%
\paragraph{v1.6:} 2018/01/17

\begin{itemize}
\item
application for development of include files
\item
corrections to manual
\end{itemize}

%%%%%%%%%%%%%%%%%%%%%%%%%%%%%%%%%%%%%%%%
\paragraph{v1.5:} 2017/05/21

\begin{itemize}
\item
more complete structuring introduced
\item
|\childdocof| introduced
\item
|\childdoc| renamed to |\childdocmain|
\item
|\childredirect| renamed to |\childdocforward| and |\childdocforwardprefix|
and functionality expanded
\end{itemize}

%%%%%%%%%%%%%%%%%%%%%%%%%%%%%%%%%%%%%%%%
\paragraph{v1.0:} 2017/04/27

\begin{itemize}
\item
manual and install package
\item
first version published on CTAN
\end{itemize}

%%%%%%%%%%%%%%%%%%%%%%%%%%%%%%%%%%%%%%%%
\paragraph{v0.6:} 2017/04/26

\begin{itemize}
\item
redirection mechanism added
\end{itemize}

%%%%%%%%%%%%%%%%%%%%%%%%%%%%%%%%%%%%%%%%
\paragraph{v0.5:} 2017/04/26

\begin{itemize}
\item
functionality in definition file
\end{itemize}


%%%%%%%%%%%%%%%%%%%%%%%%%%%%%%%%%%%%%%%%%%%%%%%%%%%%%%%%%%%%%%%%%%%%%%%%%%%%%%%%
%%%%%%%%%%%%%%%%%%%%%%%%%%%%%%%%%%%%%%%%%%%%%%%%%%%%%%%%%%%%%%%%%%%%%%%%%%%%%%%%
%%%%%%%%%%%%%%%%%%%%%%%%%%%%%%%%%%%%%%%%%%%%%%%%%%%%%%%%%%%%%%%%%%%%%%%%%%%%%%%%
\appendix

\settowidth\MacroIndent{\rmfamily\scriptsize 000\ }

 \DocInput{childdoc.dtx}

\end{document}
%</driver>
% \fi
%
% %%%%%%%%%%%%%%%%%%%%%%%%%%%%%%%%%%%%%%%%%%%%%%%%%%%%%%%%%%%%%%%%%%%%%%%%%%%%%%
% %%%%%%%%%%%%%%%%%%%%%%%%%%%%%%%%%%%%%%%%%%%%%%%%%%%%%%%%%%%%%%%%%%%%%%%%%%%%%%
% \section{Sample}
%\iffalse
%<*samplemain>
%\fi
%
% The following presents a sample document
% with two chapters, two parts, a title page,
% a compile flag as well as three forwarding files to set the flag.
% It consists of eight |.tex| files:
% \begin{center}
% \begin{tabular}{ll}
% |cdocsamp.tex|&main file\\
% |cdocsch1.tex|&include file for chapter 1\\
% |cdocsch2.tex|&include file for chapter 2\\
% |cdocspt3.tex|&include file for part 3\\
% |cdocspt4.tex|&include file for part 4\\
% |cdocsdrf.tex|&forwarding file for main file in draft mode\\
% |cdocsfi1.tex|&forwarding file for final version of chapter 1\\
% |cdocsfi2.tex|&forwarding file for final version of chapter 2\\
% \end{tabular}
% \end{center}
% Each of the eight files can be compiled directly by the \LaTeX{} compiler.
%
% %%%%%%%%%%%%%%%%%%%%%%%%%%%%%%%%%%%%%%
% \paragraph{Main File.}
%
% The main file is called |cdocsamp.tex|.
%
% Load the \textsf{childdoc} definitions and
% declare the filename for the main document:
%    \begin{macrocode}
\input{childdoc.def}
\childdocmain{}
%    \end{macrocode}

% Optional override for |\version| flag:
%    \begin{macrocode}
%%\ifchilddoc\else\providecommand{\version}{draft}\fi
%    \end{macrocode}

% Define the default values for the |\version| flag
% (|final| for the main file and |draft| for childs):
%    \begin{macrocode}
\ifchilddoc
\providecommand{\version}{draft}
\else
\providecommand{\version}{final}
\fi
%    \end{macrocode}

% Load the standard document class:
%    \begin{macrocode}
\documentclass[12pt]{article}
%    \end{macrocode}

% Start the document body:
%    \begin{macrocode}
\begin{document}
%    \end{macrocode}

% Declare a title page.
% Print title, part of document being processed and version flag:
%    \begin{macrocode}
\addtocounter{page}{-1}
\begin{center}
{\LARGE\bfseries{}childdoc example\par}
\vspace{1cm}
\ifchilddoc
\ifchilddocmanual part\else chapter\fi:
`\childdocname' of `\childdocjob'\par
\else
main document: `\childdocjob'\par
\fi
version: \version\par
\end{center}
\newpage
%    \end{macrocode}

% Manually include selected file,
% otherwise process as usual:
%    \begin{macrocode}
\ifchilddocmanual
\section*{part `\childdocname'}
\input{\childdocname}
\else
%    \end{macrocode}

% Include the two chapters:
%    \begin{macrocode}
\include{cdocsch1}
\include{cdocsch2}
%    \end{macrocode}

% Include the two parts unless only chapters should be displayed:
%    \begin{macrocode}
\ifchilddoc\else
\section{part three}
\input{cdocspt3}
\section{part four}
\input{cdocspt4}
\fi
%    \end{macrocode}

% Process as usual until here:
%    \begin{macrocode}
\fi
%    \end{macrocode}

% End of document body:
%    \begin{macrocode}
\end{document}
%    \end{macrocode}
%\iffalse
%</samplemain>
%\fi
%
% %%%%%%%%%%%%%%%%%%%%%%%%%%%%%%%%%%%%%%
% \paragraph{Chapter Include Files.}
%
% The include files are called |cdocsch1.tex| and |cdocsch2.tex|.
%
%\iffalse
%<*samplechap1|samplechap2>
%\fi

% Optional override for |\version| flag:
%    \begin{macrocode}
%%\providecommand{\version}{final}
%    \end{macrocode}

% Include the main document:
%    \begin{macrocode}
\input{childdoc.def}
\childdocof{cdocsamp}
%    \end{macrocode}

%\iffalse
%</samplechap1|samplechap2>
%\fi
%
%\iffalse
%<*samplechap1>
%\fi
% Some text for chapter 1:
%    \begin{macrocode}
\section{one}
some text in chapter one
%    \end{macrocode}

%\iffalse
%</samplechap1>
%\fi
% Some text for chapter 2:
%\iffalse
%<*samplechap2>
%\fi
%    \begin{macrocode}
\section{two}
more text in chapter two
%    \end{macrocode}

%\iffalse
%</samplechap2>
%\fi
%
% %%%%%%%%%%%%%%%%%%%%%%%%%%%%%%%%%%%%%%
% \paragraph{Part Include Files.}
%
% The include files are called |cdocspt3.tex| and |cdocspt4.tex|.
%
%\iffalse
%<*samplepart3|samplepart4>
%\fi

% Optional override for |\version| flag:
%    \begin{macrocode}
%%\providecommand{\version}{final}
%    \end{macrocode}

% Include the main document:
%    \begin{macrocode}
\input{childdoc.def}
\childdocby{cdocsamp}
%    \end{macrocode}

%\iffalse
%</samplepart3|samplepart4>
%\fi
%
%\iffalse
%<*samplepart3>
%\fi
% Some text for part 3:
%    \begin{macrocode}
some text in part three
%    \end{macrocode}

%\iffalse
%</samplepart3>
%\fi
% Some text for part 4:
%\iffalse
%<*samplepart4>
%\fi
%    \begin{macrocode}
more text in part four
%    \end{macrocode}

%\iffalse
%</samplepart4>
%\fi
%
% %%%%%%%%%%%%%%%%%%%%%%%%%%%%%%%%%%%%%%
% \paragraph{Forwarding for a Complete Draft.}
%
% The following forwarding file |cdocsdrf.tex|
% compiles the main document in draft mode:
%\iffalse
%<*sampledraft>
%\fi
%    \begin{macrocode}
\def\version{draft}
\input{childdoc.def}
\childdocforward{cdocsamp}
%    \end{macrocode}

%\iffalse
%</sampledraft>
%\fi
%
% %%%%%%%%%%%%%%%%%%%%%%%%%%%%%%%%%%%%%%
% \paragraph{Forwarding for Final Version of the Chapters.}
%
% The following forwarding files |cdocsfn1.tex| and |cdocsfn2.tex|
% (with identical content)
% compile the final versions of the child documents
% |cdocsch1.tex| and |cdocsch2.tex|, respectively:
%\iffalse
%<*samplefinal>
%\fi
%    \begin{macrocode}
\def\version{final}
\input{childdoc.def}
\childdocforwardprefix[cdocsamp]{cdocsfn}{cdocsch}
%    \end{macrocode}

%\iffalse
%</samplefinal>
%\fi
%
% %%%%%%%%%%%%%%%%%%%%%%%%%%%%%%%%%%%%%%
% \paragraph{Command Line Processing.}
%
% The following three command lines generate the output files
% |cdocscld|, |cdocscl1| and |cdocscl2|
% which should be identical to
% |cdocsdrf|, |cdocsch1| and |cdocsfn2|, respectively:
% \begin{center}
% \begin{tabular}{l}
% |latex -jobname cdocscld \|\\
% |  "\def\version{draft}\input{childdoc.def}\childdocforward{cdocsamp}"|\\
% |latex -jobname cdocscl1 \|\\
% |  "\input{childdoc.def}\childdocforward[cdocsamp]{cdocsch1}"|\\
% |latex -jobname cdocscl2 \|\\
% |  "\def\version{final}\input{childdoc.def}\childdocforward{cdocsch2}"|
% \end{tabular}
% \end{center}
% Note that the trailing backslash on each first line
% merely continues the input to the second line
% (for convenient cut ant paste).
% Furthermore, the command |latex| can be replaced by any
% of its alternative versions such as |pdflatex|.
%
% %%%%%%%%%%%%%%%%%%%%%%%%%%%%%%%%%%%%%%%%%%%%%%%%%%%%%%%%%%%%%%%%%%%%%%%%%%%%%%
% %%%%%%%%%%%%%%%%%%%%%%%%%%%%%%%%%%%%%%%%%%%%%%%%%%%%%%%%%%%%%%%%%%%%%%%%%%%%%%
% \section{Implementation}
%\iffalse
%<*package>
%\fi
%
% This section describes the definitions file |childdoc.def|.

% The definitions cannot be loaded using |\usepackage| or |\RequirePackage|
% which has a mechanism to prevent loading a style file more than once.
% When loading the definitions by means of |\input|
% multiple instances have to be prevented manually:
%\iffalse
%This code needs to be before the `\ProvidesFile' directive
%which is defined at the beginning of this file.
%Therefore it is also placed there and commented out here.
%</package>
%<*discard>
%\fi
%    \begin{macrocode}
\ifdefined\childdocmain\endinput\fi
%    \end{macrocode}
%\iffalse
%</discard>
%<*package>
%\fi
%
% \macro{\ifchilddoc}
% \macro{\ifchilddocmanual}
% The conditional |\ifchilddoc| tells whether a
% child (true) or main (false) document is being compiled.
% The conditional |\ifchilddocmanual| tells whether
% the |\includeonly| mechanism is used (false) or
% the selection of child files must be performed manually (true).
% The definitions initialise to false:
%    \begin{macrocode}
\newif\ifchilddoc
\newif\ifchilddocmanual
%    \end{macrocode}

% \macro{\childdocname}
% \macro{\childdocjob}
% The macro |\childdocname| stores the name of the main document
% to be compiled. The macro |\childdocjob| stores the name of
% the document on which the \LaTeX{} compiler was originally invoked.
% The content of |\jobname| cannot be compared
% to filenames specified in the source due to different catcodes.
% The following code rescans |\jobname|, stores the result
% in |\childdocname| and saves a copy in |\childdocjob|:
%    \begin{macrocode}
\edef\childdocname{\scantokens\expandafter{\jobname\noexpand}}
\let\childdocjob\childdocname
%    \end{macrocode}

% \macro{\childdocdisable}
% The macro |\childdocdisable| prevents the main file
% from being processed more than once.
% At this stage, the main document command |\childdocmain|
% is assumed to be called once again where it should do nothing.
% Any subsequent call to it should prevent
% a secondary processing of the main document
% It overwrites the forwarding commands
% |\childdocof| and |\childdocforward|
% with empty macros to prevent further inclusions of the main document:
%    \begin{macrocode}
\newcommand{\childdocdisable}
{
  \renewcommand{\childdocmain}[1]{\renewcommand{\childdocmain}[1]{\endinput}}
  \renewcommand{\childdocof}[1]{}
  \renewcommand{\childdocby}[2][]{}
  \renewcommand{\childdocforward}[2][]{}
  \renewcommand{\childdocdisable}{}
}
%    \end{macrocode}

% \macro{\childdocmain}
% The macro |\childdocmain| is to be called at the top of the main file
% with nothing or the main filename (without extension) as argument.
% First, it breaks loops.
% If the argument is not empty and does not match |\childdocname|
% (which is set by the first inclusion of |childdoc.def|),
% |\ifchilddoc| is set to true, |\includeonly| is applied to the child file
% and |\jobname| is set to the main file
% (for proper handling of |.aux| files):
%    \begin{macrocode}
\newcommand{\childdocmain}[1]
{
  \childdocdisable\childdocmain{}
  \if?#1?\else
    \begingroup
      \def\childdoctmp{#1}
      \ifx\childdoctmp\childdocname
        \def\childdoctmp{}
      \else
        \def\childdoctmp
        {
          \childdoctrue
          \includeonly{\childdocname}
          \def\childdocjob{#1}
          \def\jobname{#1}
        }
      \fi
      \expandafter
    \endgroup
    \childdoctmp
  \fi
}
%    \end{macrocode}

% \macro{\childdocof}
% The command |\childdocof| redirects
% compilation to the main file |#1|.
%    \begin{macrocode}
\newcommand{\childdocof}[1]
{
  \childdocdisable
  \childdoctrue
  \includeonly{\childdocname}
  \def\jobname{#1}
  \def\childdocjob{#1}
  \input{#1}
}
%    \end{macrocode}

% \macro{\childdocby}
% The command |\childdocby| ....
%    \begin{macrocode}
\newcommand{\childdocby}[2][]
{
  \childdocdisable
  \childdoctrue
  \childdocmanualtrue
  \if?#1?\else
    \def\jobname{#2}
  \fi
  \def\childdocjob{#2}
  \input{#2}
  \endinput
}
%    \end{macrocode}

% \macro{\childdocforward}
% The command |\childdocforward| redirects
% compilation to the main file or
% (if the optional argument is given) a child file.
% Parameters are set as if the main file
% or a child file starting with |\childdocof| was compiled.
% Then compilation is handed over to the main file:
%    \begin{macrocode}
\newcommand{\childdocforward}[2][]
{
  \begingroup
    \if?#1?
      \def\childdoctmp
      {
        \def\childdocname{#2}
        \def\childdocjob{#2}
        \def\jobname{#2}
        \input{#2}
        \endinput
      }
    \else
      \def\childdoctmp
      {
        \childdocdisable
        \def\childdocname{#2}
        \childdoctrue
        \includeonly{#2}
        \def\childdocjob{#1}
        \def\jobname{#1}
        \input{#1}
        \endinput
      }
    \fi
    \expandafter
  \endgroup
  \childdoctmp
}
%    \end{macrocode}

% \macro{\childdocforwardprefix}
% The command |\childdocforwardprefix| redirects
% compilation to the main or a child file by means of a pattern.
% The prefix |#1| in the current filename is replaced by |#2|
% and the suffix of the current filename is kept
% (it is assumed that the filename does not contain the substring `|~~~|'
% which is used as a delimiter).
% Compilation is handed over to the new file by |\childdocforward|:
%    \begin{macrocode}
\newcommand{\childdocforwardprefix}[3][]
{
  \begingroup
    \def\childdocextract #2##1~~~{\def\childdoctmp{\childdocforward[#1]{#3##1}}}
    \expandafter\childdocextract\childdocname~~~
    \expandafter
  \endgroup
  \childdoctmp
}
%    \end{macrocode}

% \macro{\childdoc}
% The deprecated macro |\childdoc| is a legacy version of |\childdocmain|:
%    \begin{macrocode}
\newcommand{\childdoc}{\childdocmain}
%    \end{macrocode}

% \macro{\childdocredirect}
% The deprecated macro |\childdocredirect| is a legacy version
% of |\childdocforward| and |\childdocforwardprefix|:
%    \begin{macrocode}
\newcommand{\childdocredirect}[2][]
{
  \begingroup
    \if?#1?
      \def\childdoctmp{\childdocforward{#2}}
    \else
      \def\childdoctmp{\childdocforwardprefix{#1}{#2}}
    \fi
    \expandafter
  \endgroup
  \childdoctmp
}
%    \end{macrocode}

%\iffalse
%</package>
%\fi
%
\endinput
|\\
|\childdocof{|\textit{main}|}|\\
\end{tabular}
\end{center}
at the top of every child file \textit{child}
which is included by |\include{|\textit{child}|}|
from within the main file
(or at least for those files to be compiled individually).
The argument \textit{main} must be the filename of the main file.

There are a couple of
considerations in setting up the main and child documents:

%%%%%%%%%%%%%%%%%%%%%%%%%%%%%%%%%%%%%%%%
\paragraph{Restrictions.}

Please note the following restrictions:
\begin{itemize}
\item
|\childdocmain| must be called with one argument \textit{main}
to ensure compatibility with earlier version of the package.
It must either be empty (|\childdocmain{}|)
or precisely match the filename of the main file in which it is specified.
See \secref{sec:detection} for further information.
\item
The filename \textit{main} must be specified without the |.tex| extension.
\item
The filename \textit{main} is case sensitive
(even in case-insensitive file systems)
due to internal string comparison.
\item
The argument \textit{main} should be fully expanded, it cannot be a macro.
\item
Subdirectories and special characters should be avoided in filenames.
\item
The command |\childdocmain{|\textit{main}|}| must be followed by a whitespace.
It should not be followed immediately by another command
or by a comment mark `|%|'.
This is because the \TeX{} parser reads the token immediately following
the argument of |\childdocmain| and puts it
at the beginning of every child section;
however, a white\-space is ignored.
\end{itemize}

%%%%%%%%%%%%%%%%%%%%%%%%%%%%%%%%%%%%%%%%
\paragraph{Content of Main File.}

It is advisable to place all content in the child files included by |\include|.
Any output contained in the main file will appear in all child documents
unless suppressed manually;
it cannot be suppressed automatically by the |\includeonly| directive
and thus should normally be avoided.
A method to include some content in the main file
by means of conditional processing is described in \secref{sec:conditional}.

%%%%%%%%%%%%%%%%%%%%%%%%%%%%%%%%%%%%%%%%
\paragraph{Page Numbering.}

When only a part of the document is compiled,
the appropriate numbering of pages
(as well as other status parameters)
is determined from the |.aux| files.
The latter contain information from previous passes.
However this information needs to propagate through
all intermediate child documents.
Therefore the page numbering in child documents may well
be inconsistent until the complete document is compiled at least once.

A useful (if unconventional) way to always ensure a consistent
page numbering is to restart the numbering in each child document
and denote the pages by `\textit{child}|.|\textit{page}'
where \textit{child} represents the chapter/section number of the child file.
This can be achieved by the command
|\numberwithin{page}{|\textit{child}|}|
of the \textsf{amsmath} package
where \textit{child} can be |chapter| or |section|
depending on the chosen structuring.
Alternatively, one can modify the macro |\thepage| appropriately
and reset the counter |page| at the start of each child file.

%%%%%%%%%%%%%%%%%%%%%%%%%%%%%%%%%%%%%%%%%%%%%%%%%%%%%%%%%%%%%%%%%%%%%%%%%%%%%%%%
\subsection{Conditional Processing}
\label{sec:conditional}

The package provides a mechanism to compile different versions
of a document. To customise the versions further some conditional processing
can come in handy to distinguish which version is being compiled.
The package provides two macros to describe the compilation context:

%%%%%%%%%%%%%%%%%%%%%%%%%%%%%%%%%%%%%%%%
\DescribeMacro{\ifchilddoc}
The conditional |\ifchilddoc| distinguishes between the compilation of
child documents and the main document:
%
\begin{center}
|\ifchilddoc |\textit{child-code}| |[|\||else |\textit{main-code}]| \||fi|
\end{center}

%%%%%%%%%%%%%%%%%%%%%%%%%%%%%%%%%%%%%%%%
\DescribeMacro{\childdocname}
\DescribeMacro{\childdocjob}
The macro |\childdocname| contains the filename (without extension)
of the main or child file being processed.
Note that |\childdocjob| will always contain the name of the main file.

%%%%%%%%%%%%%%%%%%%%%%%%%%%%%%%%%%%%%%%%
\paragraph{Title Page.}

Conditional processing can be used to include a title or banner page
in the main document when proper precautions are taken.
Importantly, the code in the main file should ensure that the page counter
(as well as other status parameters which are stored in the |.aux| files)
takes the same value after the conditional processing.
Otherwise the page numbers may take divergent values
depending on which part is compiled.

For example, a title page could be declared by:
%
\begin{center}
\begin{tabular}{l}
|\ifchilddoc\||else|\\
|\addtocounter{page}{-1}|\\
\textit{code for title page}\\
|\newpage|\\
|\||fi|
\end{tabular}
\end{center}
%
A banner page for the child documents can be generated by:
%
\begin{center}
\begin{tabular}{l}
|\ifchilddoc|\\
|\addtocounter{page}{-1}|\\
\textit{code for banner page}\\
|\newpage|\\
|\||fi|
\end{tabular}
\end{center}
%
Here one could write a message such as:
\begin{center}
|This is the part \childdocname{} of \childdocjob{}.|
\end{center}

%%%%%%%%%%%%%%%%%%%%%%%%%%%%%%%%%%%%%%%%%%%%%%%%%%%%%%%%%%%%%%%%%%%%%%%%%%%%%%%%
\subsection{Flags}
\label{sec:flags}

The package makes it easy to generate different versions
of the main or child documents.
To this end compilation flags can be defined
and assigned different default values.
They will be particularly useful in conjunction
with the forwarding mechanism described in \secref{sec:forward}.

For example, it may be useful to have a flag |\version|
which can be set to |draft| or |final|.
The document source will contain some conditional code
depending on the value of |\version|.
Suppose further, the flag should default to |final| for the main file
and to |draft| for child files
which is a natural assignment for editing the document.
This is achieved by placing the following code
in the preamble of the main document
(below the |\childdocmain| directive):
%
\begin{center}
\begin{tabular}{l}
|\ifchilddoc|\\
|\providecommand{\version}{draft}|\\
|\||else|\\
|\providecommand{\version}{final}|\\
|\||fi|
\end{tabular}
\end{center}
%
The definition by |\providecommand| makes sure
that previous definitions are not overwritten.
Further statements |\providecommand{\version}{...}|
can thus be added before the above code to override it.

For the main file, one might add a line
(between |\childdocmain| and the above block)
%
\begin{center}
|%\ifchilddoc\||else\providecommand{\version}{draft}\||fi|
\end{center}
%
which can be uncommented to produce a draft version.
Likewise one can add a line to the very top of a child file
(above the |\childdocof{|\textit{main}|}| directive)
%
\begin{center}
|%\providecommand{\version}{final}|
\end{center}
%
which can be uncommented to produce the final version of this child document.

%%%%%%%%%%%%%%%%%%%%%%%%%%%%%%%%%%%%%%%%%%%%%%%%%%%%%%%%%%%%%%%%%%%%%%%%%%%%%%%%
\subsection{Forwarding}
\label{sec:forward}

Different versions of the main or child documents
using compilation flags as described in \secref{sec:flags}
can be (permanently) stored in different files
for convenient compilation, viewing and distribution.
To this end, the package defines a command
to pass on compilation to a different file:

%%%%%%%%%%%%%%%%%%%%%%%%%%%%%%%%%%%%%%%%
\DescribeMacro{\childdocforward}
The command |\childdocforward| redirects processing to
another source file:
%
\begin{center}
\begin{tabular}{l}
|% \iffalse
%
% childdoc.dtx Copyright (C) 2017-2018 Niklas Beisert
%
% This work may be distributed and/or modified under the
% conditions of the LaTeX Project Public License, either version 1.3
% of this license or (at your option) any later version.
% The latest version of this license is in
%   http://www.latex-project.org/lppl.txt
% and version 1.3 or later is part of all distributions of LaTeX
% version 2005/12/01 or later.
%
% This work has the LPPL maintenance status `maintained'.
%
% The Current Maintainer of this work is Niklas Beisert.
%
% This work consists of the files childdoc.dtx and childdoc.ins
% and the derived files childdoc.def and cdocsamp.tex with
% cdocsch1.tex, cdocsch2.tex, cdocsdrf.tex, cdocsfn1.tex, cdocsfn2.tex.
%
%<package>\ifdefined\childdocmain\endinput\fi
%<package>\ProvidesFile{childdoc.def}[2018/12/30 v2.0 child document driver]
%<samplemain>\ProvidesFile{cdocsamp.tex}[2018/12/30 v2.0 sample for childdoc]
%<*driver>
%\ProvidesFile{childdoc.drv}[2018/12/30 v2.0 childdoc reference manual file]
\PassOptionsToClass{10pt,a4paper}{article}
\documentclass{ltxdoc}

\usepackage[margin=35mm]{geometry}
\usepackage{hyperref}
\usepackage{hyperxmp}
\usepackage[usenames]{color}

\hypersetup{colorlinks=true}
\hypersetup{pdfstartview=FitH}
\hypersetup{pdfpagemode=UseNone}
\hypersetup{pdfsource={}}
\hypersetup{pdflang={en-UK}}
\hypersetup{pdfcopyright={Copyright 2017-2018 Niklas Beisert.
  This work may be distributed and/or modified under the
  conditions of the LaTeX Project Public License, either version 1.3
  of this license or (at your option) any later version.}}
\hypersetup{pdflicenseurl={http://www.latex-project.org/lppl.txt}}
\hypersetup{pdfcontactaddress={ETH Zurich, ITP, HIT K,
  Wolfgang-Pauli-Strasse 27}}
\hypersetup{pdfcontactpostcode={8093}}
\hypersetup{pdfcontactcity={Zurich}}
\hypersetup{pdfcontactcountry={Switzerland}}
\hypersetup{pdfcontactemail={nbeisert@itp.phys.ethz.ch}}
\hypersetup{pdfcontacturl={http://people.phys.ethz.ch/\xmptilde nbeisert/}}

\newcommand{\secref}[1]{\hyperref[#1]{section \ref*{#1}}}

\parskip1ex
\parindent0pt
\let\olditemize\itemize
\def\itemize{\olditemize\parskip0pt}

\begin{document}

\title{The \textsf{childdoc} Package}
\hypersetup{pdftitle={The childdoc Package}}
\author{Niklas Beisert\\[2ex]
  Institut f\"ur Theoretische Physik\\
  Eidgen\"ossische Technische Hochschule Z\"urich\\
  Wolfgang-Pauli-Strasse 27, 8093 Z\"urich, Switzerland\\[1ex]
  \href{mailto:nbeisert@itp.phys.ethz.ch}
  {\texttt{nbeisert@itp.phys.ethz.ch}}}
\hypersetup{pdfauthor={Niklas Beisert}}
\hypersetup{pdfsubject={Manual for the LaTeX2e Package childdoc}}
\date{30 December 2018, \textsf{v2.0}}
\maketitle

\begin{abstract}\noindent
\textsf{childdoc} is a \LaTeXe{} package
that enables the direct compilation
of document sections included by |\include|
to individual files.
\end{abstract}

\begingroup
\parskip0ex
\tableofcontents
\endgroup

%%%%%%%%%%%%%%%%%%%%%%%%%%%%%%%%%%%%%%%%%%%%%%%%%%%%%%%%%%%%%%%%%%%%%%%%%%%%%%%%
%%%%%%%%%%%%%%%%%%%%%%%%%%%%%%%%%%%%%%%%%%%%%%%%%%%%%%%%%%%%%%%%%%%%%%%%%%%%%%%%
\section{Introduction}

\LaTeX{} provides a mechanism to structure a large document (such as a book)
into a main file and several child files (containing the chapters)
using the |\include| command.
This mechanism is beneficial for documents
which span hundreds of pages in order to
make the source file(s) more manageable.
Moreover, compilation can be restricted to
selected child files by means of the |\includeonly| command.
The latter feature can be used to reduce the compilation time while editing
(this was significantly more useful in the earlier days of \LaTeX{})
or to generate a smaller document which is easier to navigate.
Another application of |\includeonly| is to generate
documents consisting of selected parts of the complete document.

However, there are a few drawbacks of the plain |\include| mechanism:
\begin{itemize}
\item
The child files cannot be compiled on their own,
they can only be compiled via the main file.
A naive editing environment
(such as a text editor with an option
to have the current file processed by \LaTeX)
may require one to switch to the main file before compiling;
attempting to compile the child file produces errors.
\item
The main file must be modified (each time)
to adjust the |\includeonly| command
to the present needs. This easily leaves the main file in a messy state.
\item
The generated document will always carry the filename
of the main document. This is inconvenient if
several child files are to be compiled and
to be kept for distribution.
\end{itemize}

The present package provides a simple interface
to make child files individually compilable by \LaTeX{}.
Compiling a child file then has the same effect as compiling
the main file with an |\includeonly| command
to select the appropriate child.
Moreover the generated document will carry the name of the child
rather than the main file.
This resolves all three above issues.

This feature is meant to make the editing of books,
thesis documents and lecture notes somewhat more convenient.
However, the package can also be used efficiently for
composing a series of documents (such as exercise sheets)
which are typically distributed individually.
It then assists the author in generating the individual documents
(potentially in different versions)
as well as a document containing the collected series.
Another application is in developing style files
or other kinds of included material
where compilation of the style file could redirect
to a sample or test file.

%%%%%%%%%%%%%%%%%%%%%%%%%%%%%%%%%%%%%%%%%%%%%%%%%%%%%%%%%%%%%%%%%%%%%%%%%%%%%%%%
%%%%%%%%%%%%%%%%%%%%%%%%%%%%%%%%%%%%%%%%%%%%%%%%%%%%%%%%%%%%%%%%%%%%%%%%%%%%%%%%
\section{Usage}

First of all, the package \textsf{childdoc} is \emph{not} a standard
\LaTeXe{} |.sty| style file! Therefore it needs to be invoked in
a non-standard way.

%%%%%%%%%%%%%%%%%%%%%%%%%%%%%%%%%%%%%%%%%%%%%%%%%%%%%%%%%%%%%%%%%%%%%%%%%%%%%%%%
\subsection{Included Files}
\label{sec:include}

%%%%%%%%%%%%%%%%%%%%%%%%%%%%%%%%%%%%%%%%
\DescribeMacro{\childdocmain}
To use the package, add the commands
\begin{center}
\begin{tabular}{l}
|\input{childdoc.def}|\\
|\childdocmain{}|\\
\end{tabular}
\end{center}
at the very top of the main \LaTeX{} file,
in particular \emph{before} the |\documentclass| statement!
The argument of |\childdocmain| should be left empty
(but it must be present).

%%%%%%%%%%%%%%%%%%%%%%%%%%%%%%%%%%%%%%%%
\DescribeMacro{\childdocof}
Furthermore, add the commands
\begin{center}
\begin{tabular}{l}
|\input{childdoc.def}|\\
|\childdocof{|\textit{main}|}|\\
\end{tabular}
\end{center}
at the top of every child file \textit{child}
which is included by |\include{|\textit{child}|}|
from within the main file
(or at least for those files to be compiled individually).
The argument \textit{main} must be the filename of the main file.

There are a couple of
considerations in setting up the main and child documents:

%%%%%%%%%%%%%%%%%%%%%%%%%%%%%%%%%%%%%%%%
\paragraph{Restrictions.}

Please note the following restrictions:
\begin{itemize}
\item
|\childdocmain| must be called with one argument \textit{main}
to ensure compatibility with earlier version of the package.
It must either be empty (|\childdocmain{}|)
or precisely match the filename of the main file in which it is specified.
See \secref{sec:detection} for further information.
\item
The filename \textit{main} must be specified without the |.tex| extension.
\item
The filename \textit{main} is case sensitive
(even in case-insensitive file systems)
due to internal string comparison.
\item
The argument \textit{main} should be fully expanded, it cannot be a macro.
\item
Subdirectories and special characters should be avoided in filenames.
\item
The command |\childdocmain{|\textit{main}|}| must be followed by a whitespace.
It should not be followed immediately by another command
or by a comment mark `|%|'.
This is because the \TeX{} parser reads the token immediately following
the argument of |\childdocmain| and puts it
at the beginning of every child section;
however, a white\-space is ignored.
\end{itemize}

%%%%%%%%%%%%%%%%%%%%%%%%%%%%%%%%%%%%%%%%
\paragraph{Content of Main File.}

It is advisable to place all content in the child files included by |\include|.
Any output contained in the main file will appear in all child documents
unless suppressed manually;
it cannot be suppressed automatically by the |\includeonly| directive
and thus should normally be avoided.
A method to include some content in the main file
by means of conditional processing is described in \secref{sec:conditional}.

%%%%%%%%%%%%%%%%%%%%%%%%%%%%%%%%%%%%%%%%
\paragraph{Page Numbering.}

When only a part of the document is compiled,
the appropriate numbering of pages
(as well as other status parameters)
is determined from the |.aux| files.
The latter contain information from previous passes.
However this information needs to propagate through
all intermediate child documents.
Therefore the page numbering in child documents may well
be inconsistent until the complete document is compiled at least once.

A useful (if unconventional) way to always ensure a consistent
page numbering is to restart the numbering in each child document
and denote the pages by `\textit{child}|.|\textit{page}'
where \textit{child} represents the chapter/section number of the child file.
This can be achieved by the command
|\numberwithin{page}{|\textit{child}|}|
of the \textsf{amsmath} package
where \textit{child} can be |chapter| or |section|
depending on the chosen structuring.
Alternatively, one can modify the macro |\thepage| appropriately
and reset the counter |page| at the start of each child file.

%%%%%%%%%%%%%%%%%%%%%%%%%%%%%%%%%%%%%%%%%%%%%%%%%%%%%%%%%%%%%%%%%%%%%%%%%%%%%%%%
\subsection{Conditional Processing}
\label{sec:conditional}

The package provides a mechanism to compile different versions
of a document. To customise the versions further some conditional processing
can come in handy to distinguish which version is being compiled.
The package provides two macros to describe the compilation context:

%%%%%%%%%%%%%%%%%%%%%%%%%%%%%%%%%%%%%%%%
\DescribeMacro{\ifchilddoc}
The conditional |\ifchilddoc| distinguishes between the compilation of
child documents and the main document:
%
\begin{center}
|\ifchilddoc |\textit{child-code}| |[|\||else |\textit{main-code}]| \||fi|
\end{center}

%%%%%%%%%%%%%%%%%%%%%%%%%%%%%%%%%%%%%%%%
\DescribeMacro{\childdocname}
\DescribeMacro{\childdocjob}
The macro |\childdocname| contains the filename (without extension)
of the main or child file being processed.
Note that |\childdocjob| will always contain the name of the main file.

%%%%%%%%%%%%%%%%%%%%%%%%%%%%%%%%%%%%%%%%
\paragraph{Title Page.}

Conditional processing can be used to include a title or banner page
in the main document when proper precautions are taken.
Importantly, the code in the main file should ensure that the page counter
(as well as other status parameters which are stored in the |.aux| files)
takes the same value after the conditional processing.
Otherwise the page numbers may take divergent values
depending on which part is compiled.

For example, a title page could be declared by:
%
\begin{center}
\begin{tabular}{l}
|\ifchilddoc\||else|\\
|\addtocounter{page}{-1}|\\
\textit{code for title page}\\
|\newpage|\\
|\||fi|
\end{tabular}
\end{center}
%
A banner page for the child documents can be generated by:
%
\begin{center}
\begin{tabular}{l}
|\ifchilddoc|\\
|\addtocounter{page}{-1}|\\
\textit{code for banner page}\\
|\newpage|\\
|\||fi|
\end{tabular}
\end{center}
%
Here one could write a message such as:
\begin{center}
|This is the part \childdocname{} of \childdocjob{}.|
\end{center}

%%%%%%%%%%%%%%%%%%%%%%%%%%%%%%%%%%%%%%%%%%%%%%%%%%%%%%%%%%%%%%%%%%%%%%%%%%%%%%%%
\subsection{Flags}
\label{sec:flags}

The package makes it easy to generate different versions
of the main or child documents.
To this end compilation flags can be defined
and assigned different default values.
They will be particularly useful in conjunction
with the forwarding mechanism described in \secref{sec:forward}.

For example, it may be useful to have a flag |\version|
which can be set to |draft| or |final|.
The document source will contain some conditional code
depending on the value of |\version|.
Suppose further, the flag should default to |final| for the main file
and to |draft| for child files
which is a natural assignment for editing the document.
This is achieved by placing the following code
in the preamble of the main document
(below the |\childdocmain| directive):
%
\begin{center}
\begin{tabular}{l}
|\ifchilddoc|\\
|\providecommand{\version}{draft}|\\
|\||else|\\
|\providecommand{\version}{final}|\\
|\||fi|
\end{tabular}
\end{center}
%
The definition by |\providecommand| makes sure
that previous definitions are not overwritten.
Further statements |\providecommand{\version}{...}|
can thus be added before the above code to override it.

For the main file, one might add a line
(between |\childdocmain| and the above block)
%
\begin{center}
|%\ifchilddoc\||else\providecommand{\version}{draft}\||fi|
\end{center}
%
which can be uncommented to produce a draft version.
Likewise one can add a line to the very top of a child file
(above the |\childdocof{|\textit{main}|}| directive)
%
\begin{center}
|%\providecommand{\version}{final}|
\end{center}
%
which can be uncommented to produce the final version of this child document.

%%%%%%%%%%%%%%%%%%%%%%%%%%%%%%%%%%%%%%%%%%%%%%%%%%%%%%%%%%%%%%%%%%%%%%%%%%%%%%%%
\subsection{Forwarding}
\label{sec:forward}

Different versions of the main or child documents
using compilation flags as described in \secref{sec:flags}
can be (permanently) stored in different files
for convenient compilation, viewing and distribution.
To this end, the package defines a command
to pass on compilation to a different file:

%%%%%%%%%%%%%%%%%%%%%%%%%%%%%%%%%%%%%%%%
\DescribeMacro{\childdocforward}
The command |\childdocforward| redirects processing to
another source file:
%
\begin{center}
\begin{tabular}{l}
|\input{childdoc.def}|\\
|\childdocforward[|\textit{main}|]{|\textit{dest}|}|\\
\end{tabular}
\end{center}
%
The argument \textit{dest} is the destination file
(without extension).
It should be the main file or one of the child files.
Note that further \textsf{childdoc} directives
such as |\childdocof| and |\childdocforward|
in the indicated file will be processed in this form.
The optional argument \textit{main}
passes on directly to the main file \textit{main}
while pretending to compile the child \textit{dest}.
This form behaves as if \textit{dest}
issues |\childdocof{|\textit{main}|}| right away,
and no further \textsf{childdoc} directives will be processed.

%%%%%%%%%%%%%%%%%%%%%%%%%%%%%%%%%%%%%%%%
\DescribeMacro{\...prefix}
In the alternative form |\childdocforwardprefix|,
%
\begin{center}
\begin{tabular}{l}
|\input{childdoc.def}|\\
|\childdocforwardprefix[|\textit{main}|]{|\textit{prefix}|}{|\textit{dest}|}|
\end{tabular}
\end{center}
%
the destination file is determined by a pattern
depending on the current file:
To make this work, the current file must be called
`{\textit{prefix}\hspace{0.2em}\textit{suffix}}'
with \textit{prefix} matching precisely the argument.
Processing is then passed on to the file
`{\textit{dest}\hspace{0.2em}\textit{suffix}}'.
Surely, the same effect is achieved by
directly specifying the
argument `{\textit{dest}\hspace{0.2em}\textit{suffix}}'
in the first form.
However, that requires to set up a different file
for each child. With the alternative form of the command
all these files can have exactly the same content
which simplifies setting them up and maintaining them.

For example, the following file |draft.tex|
with a compilation flag |\version| as described in \secref{sec:flags}
compiles the main document as a draft:
%
\begin{center}
\begin{tabular}{l}
|\def\version{draft}|\\
|\input{childdoc.def}|\\
|\childdocforward{|\textit{main}|}|
\end{tabular}
\end{center}
%
Likewise, the following files |final|\textit{nn}|.tex|
compile the final version of the child document
|child|\textit{nn}|.tex|:
%
\begin{center}
\begin{tabular}{l}
|\def\version{final}|\\
|\input{childdoc.def}|\\
|\childdocforwardprefix{final}{child}|
\end{tabular}
\end{center}
%

Note that when several versions of a main file and/or of each child file
are to be generated, it may be convenient to set up a |Makefile| or
shell script to automatise the process.

%%%%%%%%%%%%%%%%%%%%%%%%%%%%%%%%%%%%%%%%%%%%%%%%%%%%%%%%%%%%%%%%%%%%%%%%%%%%%%%%
\subsection{Command Line Processing}
\label{sec:commandline}

The effect of redirection files can also be achieved by invoking
the \LaTeX{} compiler with a more elaborate command line.
Most conveniently this should be done as part
of a shell script or a |Makefile|.

When using \textsf{childdoc} in the main file, the following
command lines effectively perform a redirection
(note that depending on the shell being used,
backslashes may have to be doubled: `|\|' $\to$ `|\\|'):
%
\begin{center}
|... -jobname "|\textit{target}|" |\\|"|[\textit{flags}]%
|\input{childdoc.def}\childdocforward[|\textit{main}|]{|\textit{dest}|}"|
\end{center}
%
Here \textit{target} is the name of the output file,
\textit{main} is the name of the main file
and \textit{dest} is the name of the main or child file to be processed
(all filenames without extensions).
The optional argument \textit{main} can be omitted
if \textit{main} matches \textit{dest}.
Optionally, compilation \textit{flags} can be defined via |\def| commands.
This command line makes the \TeX{} engine believe
it is compiling the file \textit{target}
whose content is specified as the latter parameter.
The provided code then forwards the processing to
\textit{main} or \textit{dest} as described in \secref{sec:forward}.

%%%%%%%%%%%%%%%%%%%%%%%%%%%%%%%%%%%%%%%%%%%%%%%%%%%%%%%%%%%%%%%%%%%%%%%%%%%%%%%%
\subsection{Include by Input}
\label{sec:input}

Including child documents by |\include| has some restrictions by design.
Most notably, the content of a child document always occupies
its own set of pages; pages cannot be shared between child documents.
Usually, this behaviour makes perfect sense
because each child document contain an essential part of the document.
However, in some situations it may be desirable to compose
a document from a collection of parts
without having mandatory page breaks between then.
For this case, the package
provides a mechanism to include parts
by |\input| which can also be processed individually.
However, by construction this mechanism
requires manual handling of the content to be output.

%%%%%%%%%%%%%%%%%%%%%%%%%%%%%%%%%%%%%%%%
\DescribeMacro{\ifchilddocmanual}
The main file should be prepared as usual, see \secref{sec:include}.
However, the document body must make a distinction
between processing of an individual part and of the main document, e.g.:
%
\begin{center}
\begin{tabular}{l}
|\ifchilddocmanual|\\
|\input{\childdocname}|\\
|\||else|\\
\textit{document body with }|\input{|\textit{part}|}|\\
|\||fi|
\end{tabular}
\end{center}
%
The conditional |\ifchilddocmanual| is true whenever
a part to be included by |\input| is being compiled,
and the name of the part is stored in |\childdocname|.

%%%%%%%%%%%%%%%%%%%%%%%%%%%%%%%%%%%%%%%%
\DescribeMacro{\childdocby}
Each part to be included by |\input| should start with:
%
\begin{center}
\begin{tabular}{l}
|\input{childdoc.def}|\\
|\childdocby{|\textit{main}|}|\\
\end{tabular}
\end{center}
%
The directive |\childdocby| is similar to |\childdocof|
described in \secref{sec:include},
but the subsequent selection of content must be done manually.
To that end, both |\ifchilddoc| and |\ifchilddocmanual|
will be true upon processing of a part,
and the name of the part is stored in |\childdocname|.
Note that |\jobname| will be set to the filename of the current part
so that each part receives an individual |.aux| file
that does not interfere with the |.aux| file(s) of the main document.
This behaviour can be altered by the alternative form
|\childdocby[*]{|\textit{main}|}| (with a non-empty optional argument)
which uses the |.aux| file of the main document
by setting |\jobname| to \textit{main}.

%%%%%%%%%%%%%%%%%%%%%%%%%%%%%%%%%%%%%%%%%%%%%%%%%%%%%%%%%%%%%%%%%%%%%%%%%%%%%%%%
\subsection{Driver Development}
\label{sec:driver}

The \textsf{childdoc} mechanism can also be use for the development
of definition files such as \LaTeX{} styles or classes.
This case differs from the above setup with multiple parts
included by |\include| in that no |\includeonly| should be invoked.
This can be achieved by starting the include file
(before |\ProvidesPackage|) with:
%
\begin{center}
\begin{tabular}{l}
|\input{childdoc.def}|\\
|\childdocforward{|\textit{main}|}|\\
\end{tabular}
\end{center}
%
or alternatively with:
%
\begin{center}
\begin{tabular}{l}
|\input{childdoc.def}|\\
|\childdocby{|\textit{main}|}|\\
\end{tabular}
\end{center}
%
Both forms have slightly different effects as described above.
The main file is prepared as usual, see \secref{sec:include}.

%%%%%%%%%%%%%%%%%%%%%%%%%%%%%%%%%%%%%%%%%%%%%%%%%%%%%%%%%%%%%%%%%%%%%%%%%%%%%%%%
\subsection{Legacy Detection}
\label{sec:detection}

The directive |\childdocmain| in the main file can detect
whether the complete document or merely a child is to be compiled
even without using the directive |\childdocof|.
This method is deprecated because it is less robust
and there is no compelling reason to use it;
it is merely provided for backward compatibility
and it may be removed in future versions.

If the detection mechanism is to be used,
it is mandatory to correctly specify
the filename of the main file as the argument of |\childdocmain|:
%
\begin{center}
\begin{tabular}{l}
|\input{childdoc.def}|\\
|\childdocmain{|\textit{main}|}|\\
\end{tabular}
\end{center}
%
If |\jobname| does not match the argument \textit{main} of |\childdocmain|,
it is assumed that |\jobname| points to the child file to be compiled.
When using |\childdocmain| with the main file specified as argument,
it suffices to start a child file
with just |\input{|\textit{main}|}|
without loading of the package and using |\childdocof|.
If instead all processing is done
with the appropriate \textsf{childdoc} directives,
the argument of \textit{main} of |\childdocmain| can be empty.

An alternative version of the command line processing described
in \secref{sec:commandline} using the detection mechanism reads:
%
\begin{center}
|... -jobname "|\textit{target}|" "|[\textit{flags}]%
[|\def\jobname{|\textit{dest}|}|]|\input{|\textit{main}|}"|
\end{center}

%%%%%%%%%%%%%%%%%%%%%%%%%%%%%%%%%%%%%%%%%%%%%%%%%%%%%%%%%%%%%%%%%%%%%%%%%%%%%%%%
\subsection{Manual Code}
\label{sec:manual}

In case one cannot be certain whether the definitions file |childdoc.def|
is installed on the target \TeX{} distribution
and one prefers not to ship it,
it is conceivable to paste a few relevant commands into the sources.

To that end, drop all statements |\input{childdoc.def}|
and perform the replacements as outlined below.
Instead of |\childdocmain{|\textit{main}|}| add the following code
to the top of the main file:
%
\begin{center}
\begin{tabular}{l}
|\||ifdefined\childdocname\endinput\||fi\newif\ifchilddoc|\\
|\edef\childdocname{\scantokens\expandafter{\jobname\noexpand}}|\\
|\def\childdocmain{|\textit{main}|}\||ifx\childdocmain\childdocname\||else|\\
|\childdoctrue\includeonly{\childdocname}\let\jobname\childdocmain\||fi|\\
\end{tabular}
\end{center}
%
Instead of |\childdocof{|\textit{main}|}| just include the main file
at the top of each child file:
%
\begin{center}
|\input{|\textit{main}|}|
\end{center}
%
A simple redirection |\childdocforward{|\textit{dest}|}| is achieved by:
%
\begin{center}
|\def\jobname{|\textit{dest}|}\input{\jobname}|
\end{center}
%
The redirection with prefix
|\childdocforwardprefix[|\textit{prefix}|]{|\textit{dest}|}|
is accomplished by:
%
\begin{center}
\begin{tabular}{l}
|{\edef\jobname{\scantokens\expandafter{\jobname\noexpand}}|\\
|\def\redirectjob |\textit{prefix}|#1~~~{\gdef\jobname{|\textit{dest}|#1}}|\\
|\expandafter\redirectjob\jobname~~~}\input{\jobname}|
\end{tabular}
\end{center}

In an alternative approach,
child documents can be compiled by a specific command line
without additional code or specific definitions:
%
\begin{center}
|... -jobname "|\textit{target}|" "|[\textit{flags}]%
|\includeonly{|\textit{dest}|}\input{|\textit{main}|}"|
\end{center}
%

%%%%%%%%%%%%%%%%%%%%%%%%%%%%%%%%%%%%%%%%%%%%%%%%%%%%%%%%%%%%%%%%%%%%%%%%%%%%%%%%
%%%%%%%%%%%%%%%%%%%%%%%%%%%%%%%%%%%%%%%%%%%%%%%%%%%%%%%%%%%%%%%%%%%%%%%%%%%%%%%%
\section{Information}

%%%%%%%%%%%%%%%%%%%%%%%%%%%%%%%%%%%%%%%%%%%%%%%%%%%%%%%%%%%%%%%%%%%%%%%%%%%%%%%%
\subsection{Copyright}

Copyright \copyright{} 2017--2018 Niklas Beisert

This work may be distributed and/or modified under the
conditions of the \LaTeX{} Project Public License, either version 1.3
of this license or (at your option) any later version.
The latest version of this license is in
  \url{http://www.latex-project.org/lppl.txt}
and version 1.3 or later is part of all distributions of \LaTeX{}
version 2005/12/01 or later.

This work has the LPPL maintenance status `maintained'.

The Current Maintainer of this work is Niklas Beisert.

This work consists of the files |README.txt|, |childdoc.ins| and |childdoc.dtx|
as well as the derived files |childdoc.def|, |cdocsamp.tex|
with |cdocsch1.tex|, |cdocsch2.tex|, |cdocspt3.tex|, |cdocspt4.tex|,
|cdocsdrf.tex|, |cdocsfn1.tex|, |cdocsfn2.tex|
as well as |childdoc.pdf|.

%%%%%%%%%%%%%%%%%%%%%%%%%%%%%%%%%%%%%%%%%%%%%%%%%%%%%%%%%%%%%%%%%%%%%%%%%%%%%%%%
\subsection{Files and Installation}

The package consists of the files:
%
\begin{center}
\begin{tabular}{ll}
    |README.txt|   & readme file \\
    |childdoc.ins| & installation file \\
    |childdoc.dtx| & source file \\
    |childdoc.def| & definition file \\
    |cdocsamp.tex| & sample main file \\
    |cdocsch1.tex| & sample include file \\
    |cdocsch2.tex| & sample include file \\
    |cdocspt3.tex| & sample part file \\
    |cdocspt4.tex| & sample part file \\
    |cdocsdrf.tex| & sample redirection file \\
    |cdocsfn1.tex| & sample redirection file \\
    |cdocsfn2.tex| & sample redirection file \\
    |childdoc.pdf| & manual
\end{tabular}
\end{center}
%
The distribution consists of the files
|README.txt|, |childdoc.ins| and |childdoc.dtx|.
%
\begin{itemize}
\item
Run (pdf)\LaTeX{} on |childdoc.dtx|
to compile the manual |childdoc.pdf| (this file).
\item
Run \LaTeX{} on |childdoc.ins| to create the definitions file |childdoc.def|
and the sample |cdocsamp.tex| with include files
|cdocsch1.tex|, |cdocsch2.tex|, |cdocspt3.tex|, |cdocspt4.tex|,
|cdocsdrf.tex|, |cdocsfn1.tex|, |cdocsfn2.tex|.
Then copy the file |childdoc.def| to an appropriate directory of your \LaTeX{}
distribution, e.g.\ \textit{texmf-root}|/tex/latex/childdoc|.
\end{itemize}

%%%%%%%%%%%%%%%%%%%%%%%%%%%%%%%%%%%%%%%%%%%%%%%%%%%%%%%%%%%%%%%%%%%%%%%%%%%%%%%%
\subsection{Related CTAN Packages}

There are several other packages which offer a similar functionality:
%
\begin{itemize}
\item
The packages
\href{http://ctan.org/pkg/docmute}{\textsf{docmute}},
\href{http://ctan.org/pkg/includex}{\textsf{includex}} and
\href{http://ctan.org/pkg/standalone}{\textsf{standalone}}
provide commands to include only the document body of
a child file thus allowing both files to be compiled individually.
\item
The packages \href{http://ctan.org/pkg/subdocs}{\textsf{subdocs}}
and \href{http://ctan.org/pkg/subfiles}{\textsf{subfiles}}
provide structures in which the main and child documents can be
encapsulated and allowing them to be compiled individually.
The inclusion mechanism is different from the conventional |\include|.
\item
The package \href{http://ctan.org/pkg/combine}{\textsf{combine}}
is an elaborate solution to combine several documents into one.
\end{itemize}
%
See also the CTAN topic \href{http://ctan.org/topic/subdocs}{\textsf{subdocs}}
for further related packages.
The present package differs from the above solutions in that
a document structure constructed with the conventional |\include| mechanism
just needs two extra commands at the top of every file
such that all constituent files can be compiled individually.

%%%%%%%%%%%%%%%%%%%%%%%%%%%%%%%%%%%%%%%%%%%%%%%%%%%%%%%%%%%%%%%%%%%%%%%%%%%%%%%%
%\subsection{Feature Suggestions}
%
%The following is a list of features which may be useful for future
%versions of this package:
%%
%\begin{itemize}
%\item
%\ldots
%\end{itemize}

%%%%%%%%%%%%%%%%%%%%%%%%%%%%%%%%%%%%%%%%%%%%%%%%%%%%%%%%%%%%%%%%%%%%%%%%%%%%%%%%
\subsection{Revision History}

%%%%%%%%%%%%%%%%%%%%%%%%%%%%%%%%%%%%%%%%
\paragraph{v2.0:} 2018/12/30

\begin{itemize}
\item
immediate forward processing
\item
added |\childdocby| mechanism
\item
manual restructured
\end{itemize}

%%%%%%%%%%%%%%%%%%%%%%%%%%%%%%%%%%%%%%%%
\paragraph{v1.6:} 2018/01/17

\begin{itemize}
\item
application for development of include files
\item
corrections to manual
\end{itemize}

%%%%%%%%%%%%%%%%%%%%%%%%%%%%%%%%%%%%%%%%
\paragraph{v1.5:} 2017/05/21

\begin{itemize}
\item
more complete structuring introduced
\item
|\childdocof| introduced
\item
|\childdoc| renamed to |\childdocmain|
\item
|\childredirect| renamed to |\childdocforward| and |\childdocforwardprefix|
and functionality expanded
\end{itemize}

%%%%%%%%%%%%%%%%%%%%%%%%%%%%%%%%%%%%%%%%
\paragraph{v1.0:} 2017/04/27

\begin{itemize}
\item
manual and install package
\item
first version published on CTAN
\end{itemize}

%%%%%%%%%%%%%%%%%%%%%%%%%%%%%%%%%%%%%%%%
\paragraph{v0.6:} 2017/04/26

\begin{itemize}
\item
redirection mechanism added
\end{itemize}

%%%%%%%%%%%%%%%%%%%%%%%%%%%%%%%%%%%%%%%%
\paragraph{v0.5:} 2017/04/26

\begin{itemize}
\item
functionality in definition file
\end{itemize}


%%%%%%%%%%%%%%%%%%%%%%%%%%%%%%%%%%%%%%%%%%%%%%%%%%%%%%%%%%%%%%%%%%%%%%%%%%%%%%%%
%%%%%%%%%%%%%%%%%%%%%%%%%%%%%%%%%%%%%%%%%%%%%%%%%%%%%%%%%%%%%%%%%%%%%%%%%%%%%%%%
%%%%%%%%%%%%%%%%%%%%%%%%%%%%%%%%%%%%%%%%%%%%%%%%%%%%%%%%%%%%%%%%%%%%%%%%%%%%%%%%
\appendix

\settowidth\MacroIndent{\rmfamily\scriptsize 000\ }

 \DocInput{childdoc.dtx}

\end{document}
%</driver>
% \fi
%
% %%%%%%%%%%%%%%%%%%%%%%%%%%%%%%%%%%%%%%%%%%%%%%%%%%%%%%%%%%%%%%%%%%%%%%%%%%%%%%
% %%%%%%%%%%%%%%%%%%%%%%%%%%%%%%%%%%%%%%%%%%%%%%%%%%%%%%%%%%%%%%%%%%%%%%%%%%%%%%
% \section{Sample}
%\iffalse
%<*samplemain>
%\fi
%
% The following presents a sample document
% with two chapters, two parts, a title page,
% a compile flag as well as three forwarding files to set the flag.
% It consists of eight |.tex| files:
% \begin{center}
% \begin{tabular}{ll}
% |cdocsamp.tex|&main file\\
% |cdocsch1.tex|&include file for chapter 1\\
% |cdocsch2.tex|&include file for chapter 2\\
% |cdocspt3.tex|&include file for part 3\\
% |cdocspt4.tex|&include file for part 4\\
% |cdocsdrf.tex|&forwarding file for main file in draft mode\\
% |cdocsfi1.tex|&forwarding file for final version of chapter 1\\
% |cdocsfi2.tex|&forwarding file for final version of chapter 2\\
% \end{tabular}
% \end{center}
% Each of the eight files can be compiled directly by the \LaTeX{} compiler.
%
% %%%%%%%%%%%%%%%%%%%%%%%%%%%%%%%%%%%%%%
% \paragraph{Main File.}
%
% The main file is called |cdocsamp.tex|.
%
% Load the \textsf{childdoc} definitions and
% declare the filename for the main document:
%    \begin{macrocode}
\input{childdoc.def}
\childdocmain{}
%    \end{macrocode}

% Optional override for |\version| flag:
%    \begin{macrocode}
%%\ifchilddoc\else\providecommand{\version}{draft}\fi
%    \end{macrocode}

% Define the default values for the |\version| flag
% (|final| for the main file and |draft| for childs):
%    \begin{macrocode}
\ifchilddoc
\providecommand{\version}{draft}
\else
\providecommand{\version}{final}
\fi
%    \end{macrocode}

% Load the standard document class:
%    \begin{macrocode}
\documentclass[12pt]{article}
%    \end{macrocode}

% Start the document body:
%    \begin{macrocode}
\begin{document}
%    \end{macrocode}

% Declare a title page.
% Print title, part of document being processed and version flag:
%    \begin{macrocode}
\addtocounter{page}{-1}
\begin{center}
{\LARGE\bfseries{}childdoc example\par}
\vspace{1cm}
\ifchilddoc
\ifchilddocmanual part\else chapter\fi:
`\childdocname' of `\childdocjob'\par
\else
main document: `\childdocjob'\par
\fi
version: \version\par
\end{center}
\newpage
%    \end{macrocode}

% Manually include selected file,
% otherwise process as usual:
%    \begin{macrocode}
\ifchilddocmanual
\section*{part `\childdocname'}
\input{\childdocname}
\else
%    \end{macrocode}

% Include the two chapters:
%    \begin{macrocode}
\include{cdocsch1}
\include{cdocsch2}
%    \end{macrocode}

% Include the two parts unless only chapters should be displayed:
%    \begin{macrocode}
\ifchilddoc\else
\section{part three}
\input{cdocspt3}
\section{part four}
\input{cdocspt4}
\fi
%    \end{macrocode}

% Process as usual until here:
%    \begin{macrocode}
\fi
%    \end{macrocode}

% End of document body:
%    \begin{macrocode}
\end{document}
%    \end{macrocode}
%\iffalse
%</samplemain>
%\fi
%
% %%%%%%%%%%%%%%%%%%%%%%%%%%%%%%%%%%%%%%
% \paragraph{Chapter Include Files.}
%
% The include files are called |cdocsch1.tex| and |cdocsch2.tex|.
%
%\iffalse
%<*samplechap1|samplechap2>
%\fi

% Optional override for |\version| flag:
%    \begin{macrocode}
%%\providecommand{\version}{final}
%    \end{macrocode}

% Include the main document:
%    \begin{macrocode}
\input{childdoc.def}
\childdocof{cdocsamp}
%    \end{macrocode}

%\iffalse
%</samplechap1|samplechap2>
%\fi
%
%\iffalse
%<*samplechap1>
%\fi
% Some text for chapter 1:
%    \begin{macrocode}
\section{one}
some text in chapter one
%    \end{macrocode}

%\iffalse
%</samplechap1>
%\fi
% Some text for chapter 2:
%\iffalse
%<*samplechap2>
%\fi
%    \begin{macrocode}
\section{two}
more text in chapter two
%    \end{macrocode}

%\iffalse
%</samplechap2>
%\fi
%
% %%%%%%%%%%%%%%%%%%%%%%%%%%%%%%%%%%%%%%
% \paragraph{Part Include Files.}
%
% The include files are called |cdocspt3.tex| and |cdocspt4.tex|.
%
%\iffalse
%<*samplepart3|samplepart4>
%\fi

% Optional override for |\version| flag:
%    \begin{macrocode}
%%\providecommand{\version}{final}
%    \end{macrocode}

% Include the main document:
%    \begin{macrocode}
\input{childdoc.def}
\childdocby{cdocsamp}
%    \end{macrocode}

%\iffalse
%</samplepart3|samplepart4>
%\fi
%
%\iffalse
%<*samplepart3>
%\fi
% Some text for part 3:
%    \begin{macrocode}
some text in part three
%    \end{macrocode}

%\iffalse
%</samplepart3>
%\fi
% Some text for part 4:
%\iffalse
%<*samplepart4>
%\fi
%    \begin{macrocode}
more text in part four
%    \end{macrocode}

%\iffalse
%</samplepart4>
%\fi
%
% %%%%%%%%%%%%%%%%%%%%%%%%%%%%%%%%%%%%%%
% \paragraph{Forwarding for a Complete Draft.}
%
% The following forwarding file |cdocsdrf.tex|
% compiles the main document in draft mode:
%\iffalse
%<*sampledraft>
%\fi
%    \begin{macrocode}
\def\version{draft}
\input{childdoc.def}
\childdocforward{cdocsamp}
%    \end{macrocode}

%\iffalse
%</sampledraft>
%\fi
%
% %%%%%%%%%%%%%%%%%%%%%%%%%%%%%%%%%%%%%%
% \paragraph{Forwarding for Final Version of the Chapters.}
%
% The following forwarding files |cdocsfn1.tex| and |cdocsfn2.tex|
% (with identical content)
% compile the final versions of the child documents
% |cdocsch1.tex| and |cdocsch2.tex|, respectively:
%\iffalse
%<*samplefinal>
%\fi
%    \begin{macrocode}
\def\version{final}
\input{childdoc.def}
\childdocforwardprefix[cdocsamp]{cdocsfn}{cdocsch}
%    \end{macrocode}

%\iffalse
%</samplefinal>
%\fi
%
% %%%%%%%%%%%%%%%%%%%%%%%%%%%%%%%%%%%%%%
% \paragraph{Command Line Processing.}
%
% The following three command lines generate the output files
% |cdocscld|, |cdocscl1| and |cdocscl2|
% which should be identical to
% |cdocsdrf|, |cdocsch1| and |cdocsfn2|, respectively:
% \begin{center}
% \begin{tabular}{l}
% |latex -jobname cdocscld \|\\
% |  "\def\version{draft}\input{childdoc.def}\childdocforward{cdocsamp}"|\\
% |latex -jobname cdocscl1 \|\\
% |  "\input{childdoc.def}\childdocforward[cdocsamp]{cdocsch1}"|\\
% |latex -jobname cdocscl2 \|\\
% |  "\def\version{final}\input{childdoc.def}\childdocforward{cdocsch2}"|
% \end{tabular}
% \end{center}
% Note that the trailing backslash on each first line
% merely continues the input to the second line
% (for convenient cut ant paste).
% Furthermore, the command |latex| can be replaced by any
% of its alternative versions such as |pdflatex|.
%
% %%%%%%%%%%%%%%%%%%%%%%%%%%%%%%%%%%%%%%%%%%%%%%%%%%%%%%%%%%%%%%%%%%%%%%%%%%%%%%
% %%%%%%%%%%%%%%%%%%%%%%%%%%%%%%%%%%%%%%%%%%%%%%%%%%%%%%%%%%%%%%%%%%%%%%%%%%%%%%
% \section{Implementation}
%\iffalse
%<*package>
%\fi
%
% This section describes the definitions file |childdoc.def|.

% The definitions cannot be loaded using |\usepackage| or |\RequirePackage|
% which has a mechanism to prevent loading a style file more than once.
% When loading the definitions by means of |\input|
% multiple instances have to be prevented manually:
%\iffalse
%This code needs to be before the `\ProvidesFile' directive
%which is defined at the beginning of this file.
%Therefore it is also placed there and commented out here.
%</package>
%<*discard>
%\fi
%    \begin{macrocode}
\ifdefined\childdocmain\endinput\fi
%    \end{macrocode}
%\iffalse
%</discard>
%<*package>
%\fi
%
% \macro{\ifchilddoc}
% \macro{\ifchilddocmanual}
% The conditional |\ifchilddoc| tells whether a
% child (true) or main (false) document is being compiled.
% The conditional |\ifchilddocmanual| tells whether
% the |\includeonly| mechanism is used (false) or
% the selection of child files must be performed manually (true).
% The definitions initialise to false:
%    \begin{macrocode}
\newif\ifchilddoc
\newif\ifchilddocmanual
%    \end{macrocode}

% \macro{\childdocname}
% \macro{\childdocjob}
% The macro |\childdocname| stores the name of the main document
% to be compiled. The macro |\childdocjob| stores the name of
% the document on which the \LaTeX{} compiler was originally invoked.
% The content of |\jobname| cannot be compared
% to filenames specified in the source due to different catcodes.
% The following code rescans |\jobname|, stores the result
% in |\childdocname| and saves a copy in |\childdocjob|:
%    \begin{macrocode}
\edef\childdocname{\scantokens\expandafter{\jobname\noexpand}}
\let\childdocjob\childdocname
%    \end{macrocode}

% \macro{\childdocdisable}
% The macro |\childdocdisable| prevents the main file
% from being processed more than once.
% At this stage, the main document command |\childdocmain|
% is assumed to be called once again where it should do nothing.
% Any subsequent call to it should prevent
% a secondary processing of the main document
% It overwrites the forwarding commands
% |\childdocof| and |\childdocforward|
% with empty macros to prevent further inclusions of the main document:
%    \begin{macrocode}
\newcommand{\childdocdisable}
{
  \renewcommand{\childdocmain}[1]{\renewcommand{\childdocmain}[1]{\endinput}}
  \renewcommand{\childdocof}[1]{}
  \renewcommand{\childdocby}[2][]{}
  \renewcommand{\childdocforward}[2][]{}
  \renewcommand{\childdocdisable}{}
}
%    \end{macrocode}

% \macro{\childdocmain}
% The macro |\childdocmain| is to be called at the top of the main file
% with nothing or the main filename (without extension) as argument.
% First, it breaks loops.
% If the argument is not empty and does not match |\childdocname|
% (which is set by the first inclusion of |childdoc.def|),
% |\ifchilddoc| is set to true, |\includeonly| is applied to the child file
% and |\jobname| is set to the main file
% (for proper handling of |.aux| files):
%    \begin{macrocode}
\newcommand{\childdocmain}[1]
{
  \childdocdisable\childdocmain{}
  \if?#1?\else
    \begingroup
      \def\childdoctmp{#1}
      \ifx\childdoctmp\childdocname
        \def\childdoctmp{}
      \else
        \def\childdoctmp
        {
          \childdoctrue
          \includeonly{\childdocname}
          \def\childdocjob{#1}
          \def\jobname{#1}
        }
      \fi
      \expandafter
    \endgroup
    \childdoctmp
  \fi
}
%    \end{macrocode}

% \macro{\childdocof}
% The command |\childdocof| redirects
% compilation to the main file |#1|.
%    \begin{macrocode}
\newcommand{\childdocof}[1]
{
  \childdocdisable
  \childdoctrue
  \includeonly{\childdocname}
  \def\jobname{#1}
  \def\childdocjob{#1}
  \input{#1}
}
%    \end{macrocode}

% \macro{\childdocby}
% The command |\childdocby| ....
%    \begin{macrocode}
\newcommand{\childdocby}[2][]
{
  \childdocdisable
  \childdoctrue
  \childdocmanualtrue
  \if?#1?\else
    \def\jobname{#2}
  \fi
  \def\childdocjob{#2}
  \input{#2}
  \endinput
}
%    \end{macrocode}

% \macro{\childdocforward}
% The command |\childdocforward| redirects
% compilation to the main file or
% (if the optional argument is given) a child file.
% Parameters are set as if the main file
% or a child file starting with |\childdocof| was compiled.
% Then compilation is handed over to the main file:
%    \begin{macrocode}
\newcommand{\childdocforward}[2][]
{
  \begingroup
    \if?#1?
      \def\childdoctmp
      {
        \def\childdocname{#2}
        \def\childdocjob{#2}
        \def\jobname{#2}
        \input{#2}
        \endinput
      }
    \else
      \def\childdoctmp
      {
        \childdocdisable
        \def\childdocname{#2}
        \childdoctrue
        \includeonly{#2}
        \def\childdocjob{#1}
        \def\jobname{#1}
        \input{#1}
        \endinput
      }
    \fi
    \expandafter
  \endgroup
  \childdoctmp
}
%    \end{macrocode}

% \macro{\childdocforwardprefix}
% The command |\childdocforwardprefix| redirects
% compilation to the main or a child file by means of a pattern.
% The prefix |#1| in the current filename is replaced by |#2|
% and the suffix of the current filename is kept
% (it is assumed that the filename does not contain the substring `|~~~|'
% which is used as a delimiter).
% Compilation is handed over to the new file by |\childdocforward|:
%    \begin{macrocode}
\newcommand{\childdocforwardprefix}[3][]
{
  \begingroup
    \def\childdocextract #2##1~~~{\def\childdoctmp{\childdocforward[#1]{#3##1}}}
    \expandafter\childdocextract\childdocname~~~
    \expandafter
  \endgroup
  \childdoctmp
}
%    \end{macrocode}

% \macro{\childdoc}
% The deprecated macro |\childdoc| is a legacy version of |\childdocmain|:
%    \begin{macrocode}
\newcommand{\childdoc}{\childdocmain}
%    \end{macrocode}

% \macro{\childdocredirect}
% The deprecated macro |\childdocredirect| is a legacy version
% of |\childdocforward| and |\childdocforwardprefix|:
%    \begin{macrocode}
\newcommand{\childdocredirect}[2][]
{
  \begingroup
    \if?#1?
      \def\childdoctmp{\childdocforward{#2}}
    \else
      \def\childdoctmp{\childdocforwardprefix{#1}{#2}}
    \fi
    \expandafter
  \endgroup
  \childdoctmp
}
%    \end{macrocode}

%\iffalse
%</package>
%\fi
%
\endinput
|\\
|\childdocforward[|\textit{main}|]{|\textit{dest}|}|\\
\end{tabular}
\end{center}
%
The argument \textit{dest} is the destination file
(without extension).
It should be the main file or one of the child files.
Note that further \textsf{childdoc} directives
such as |\childdocof| and |\childdocforward|
in the indicated file will be processed in this form.
The optional argument \textit{main}
passes on directly to the main file \textit{main}
while pretending to compile the child \textit{dest}.
This form behaves as if \textit{dest}
issues |\childdocof{|\textit{main}|}| right away,
and no further \textsf{childdoc} directives will be processed.

%%%%%%%%%%%%%%%%%%%%%%%%%%%%%%%%%%%%%%%%
\DescribeMacro{\...prefix}
In the alternative form |\childdocforwardprefix|,
%
\begin{center}
\begin{tabular}{l}
|% \iffalse
%
% childdoc.dtx Copyright (C) 2017-2018 Niklas Beisert
%
% This work may be distributed and/or modified under the
% conditions of the LaTeX Project Public License, either version 1.3
% of this license or (at your option) any later version.
% The latest version of this license is in
%   http://www.latex-project.org/lppl.txt
% and version 1.3 or later is part of all distributions of LaTeX
% version 2005/12/01 or later.
%
% This work has the LPPL maintenance status `maintained'.
%
% The Current Maintainer of this work is Niklas Beisert.
%
% This work consists of the files childdoc.dtx and childdoc.ins
% and the derived files childdoc.def and cdocsamp.tex with
% cdocsch1.tex, cdocsch2.tex, cdocsdrf.tex, cdocsfn1.tex, cdocsfn2.tex.
%
%<package>\ifdefined\childdocmain\endinput\fi
%<package>\ProvidesFile{childdoc.def}[2018/12/30 v2.0 child document driver]
%<samplemain>\ProvidesFile{cdocsamp.tex}[2018/12/30 v2.0 sample for childdoc]
%<*driver>
%\ProvidesFile{childdoc.drv}[2018/12/30 v2.0 childdoc reference manual file]
\PassOptionsToClass{10pt,a4paper}{article}
\documentclass{ltxdoc}

\usepackage[margin=35mm]{geometry}
\usepackage{hyperref}
\usepackage{hyperxmp}
\usepackage[usenames]{color}

\hypersetup{colorlinks=true}
\hypersetup{pdfstartview=FitH}
\hypersetup{pdfpagemode=UseNone}
\hypersetup{pdfsource={}}
\hypersetup{pdflang={en-UK}}
\hypersetup{pdfcopyright={Copyright 2017-2018 Niklas Beisert.
  This work may be distributed and/or modified under the
  conditions of the LaTeX Project Public License, either version 1.3
  of this license or (at your option) any later version.}}
\hypersetup{pdflicenseurl={http://www.latex-project.org/lppl.txt}}
\hypersetup{pdfcontactaddress={ETH Zurich, ITP, HIT K,
  Wolfgang-Pauli-Strasse 27}}
\hypersetup{pdfcontactpostcode={8093}}
\hypersetup{pdfcontactcity={Zurich}}
\hypersetup{pdfcontactcountry={Switzerland}}
\hypersetup{pdfcontactemail={nbeisert@itp.phys.ethz.ch}}
\hypersetup{pdfcontacturl={http://people.phys.ethz.ch/\xmptilde nbeisert/}}

\newcommand{\secref}[1]{\hyperref[#1]{section \ref*{#1}}}

\parskip1ex
\parindent0pt
\let\olditemize\itemize
\def\itemize{\olditemize\parskip0pt}

\begin{document}

\title{The \textsf{childdoc} Package}
\hypersetup{pdftitle={The childdoc Package}}
\author{Niklas Beisert\\[2ex]
  Institut f\"ur Theoretische Physik\\
  Eidgen\"ossische Technische Hochschule Z\"urich\\
  Wolfgang-Pauli-Strasse 27, 8093 Z\"urich, Switzerland\\[1ex]
  \href{mailto:nbeisert@itp.phys.ethz.ch}
  {\texttt{nbeisert@itp.phys.ethz.ch}}}
\hypersetup{pdfauthor={Niklas Beisert}}
\hypersetup{pdfsubject={Manual for the LaTeX2e Package childdoc}}
\date{30 December 2018, \textsf{v2.0}}
\maketitle

\begin{abstract}\noindent
\textsf{childdoc} is a \LaTeXe{} package
that enables the direct compilation
of document sections included by |\include|
to individual files.
\end{abstract}

\begingroup
\parskip0ex
\tableofcontents
\endgroup

%%%%%%%%%%%%%%%%%%%%%%%%%%%%%%%%%%%%%%%%%%%%%%%%%%%%%%%%%%%%%%%%%%%%%%%%%%%%%%%%
%%%%%%%%%%%%%%%%%%%%%%%%%%%%%%%%%%%%%%%%%%%%%%%%%%%%%%%%%%%%%%%%%%%%%%%%%%%%%%%%
\section{Introduction}

\LaTeX{} provides a mechanism to structure a large document (such as a book)
into a main file and several child files (containing the chapters)
using the |\include| command.
This mechanism is beneficial for documents
which span hundreds of pages in order to
make the source file(s) more manageable.
Moreover, compilation can be restricted to
selected child files by means of the |\includeonly| command.
The latter feature can be used to reduce the compilation time while editing
(this was significantly more useful in the earlier days of \LaTeX{})
or to generate a smaller document which is easier to navigate.
Another application of |\includeonly| is to generate
documents consisting of selected parts of the complete document.

However, there are a few drawbacks of the plain |\include| mechanism:
\begin{itemize}
\item
The child files cannot be compiled on their own,
they can only be compiled via the main file.
A naive editing environment
(such as a text editor with an option
to have the current file processed by \LaTeX)
may require one to switch to the main file before compiling;
attempting to compile the child file produces errors.
\item
The main file must be modified (each time)
to adjust the |\includeonly| command
to the present needs. This easily leaves the main file in a messy state.
\item
The generated document will always carry the filename
of the main document. This is inconvenient if
several child files are to be compiled and
to be kept for distribution.
\end{itemize}

The present package provides a simple interface
to make child files individually compilable by \LaTeX{}.
Compiling a child file then has the same effect as compiling
the main file with an |\includeonly| command
to select the appropriate child.
Moreover the generated document will carry the name of the child
rather than the main file.
This resolves all three above issues.

This feature is meant to make the editing of books,
thesis documents and lecture notes somewhat more convenient.
However, the package can also be used efficiently for
composing a series of documents (such as exercise sheets)
which are typically distributed individually.
It then assists the author in generating the individual documents
(potentially in different versions)
as well as a document containing the collected series.
Another application is in developing style files
or other kinds of included material
where compilation of the style file could redirect
to a sample or test file.

%%%%%%%%%%%%%%%%%%%%%%%%%%%%%%%%%%%%%%%%%%%%%%%%%%%%%%%%%%%%%%%%%%%%%%%%%%%%%%%%
%%%%%%%%%%%%%%%%%%%%%%%%%%%%%%%%%%%%%%%%%%%%%%%%%%%%%%%%%%%%%%%%%%%%%%%%%%%%%%%%
\section{Usage}

First of all, the package \textsf{childdoc} is \emph{not} a standard
\LaTeXe{} |.sty| style file! Therefore it needs to be invoked in
a non-standard way.

%%%%%%%%%%%%%%%%%%%%%%%%%%%%%%%%%%%%%%%%%%%%%%%%%%%%%%%%%%%%%%%%%%%%%%%%%%%%%%%%
\subsection{Included Files}
\label{sec:include}

%%%%%%%%%%%%%%%%%%%%%%%%%%%%%%%%%%%%%%%%
\DescribeMacro{\childdocmain}
To use the package, add the commands
\begin{center}
\begin{tabular}{l}
|\input{childdoc.def}|\\
|\childdocmain{}|\\
\end{tabular}
\end{center}
at the very top of the main \LaTeX{} file,
in particular \emph{before} the |\documentclass| statement!
The argument of |\childdocmain| should be left empty
(but it must be present).

%%%%%%%%%%%%%%%%%%%%%%%%%%%%%%%%%%%%%%%%
\DescribeMacro{\childdocof}
Furthermore, add the commands
\begin{center}
\begin{tabular}{l}
|\input{childdoc.def}|\\
|\childdocof{|\textit{main}|}|\\
\end{tabular}
\end{center}
at the top of every child file \textit{child}
which is included by |\include{|\textit{child}|}|
from within the main file
(or at least for those files to be compiled individually).
The argument \textit{main} must be the filename of the main file.

There are a couple of
considerations in setting up the main and child documents:

%%%%%%%%%%%%%%%%%%%%%%%%%%%%%%%%%%%%%%%%
\paragraph{Restrictions.}

Please note the following restrictions:
\begin{itemize}
\item
|\childdocmain| must be called with one argument \textit{main}
to ensure compatibility with earlier version of the package.
It must either be empty (|\childdocmain{}|)
or precisely match the filename of the main file in which it is specified.
See \secref{sec:detection} for further information.
\item
The filename \textit{main} must be specified without the |.tex| extension.
\item
The filename \textit{main} is case sensitive
(even in case-insensitive file systems)
due to internal string comparison.
\item
The argument \textit{main} should be fully expanded, it cannot be a macro.
\item
Subdirectories and special characters should be avoided in filenames.
\item
The command |\childdocmain{|\textit{main}|}| must be followed by a whitespace.
It should not be followed immediately by another command
or by a comment mark `|%|'.
This is because the \TeX{} parser reads the token immediately following
the argument of |\childdocmain| and puts it
at the beginning of every child section;
however, a white\-space is ignored.
\end{itemize}

%%%%%%%%%%%%%%%%%%%%%%%%%%%%%%%%%%%%%%%%
\paragraph{Content of Main File.}

It is advisable to place all content in the child files included by |\include|.
Any output contained in the main file will appear in all child documents
unless suppressed manually;
it cannot be suppressed automatically by the |\includeonly| directive
and thus should normally be avoided.
A method to include some content in the main file
by means of conditional processing is described in \secref{sec:conditional}.

%%%%%%%%%%%%%%%%%%%%%%%%%%%%%%%%%%%%%%%%
\paragraph{Page Numbering.}

When only a part of the document is compiled,
the appropriate numbering of pages
(as well as other status parameters)
is determined from the |.aux| files.
The latter contain information from previous passes.
However this information needs to propagate through
all intermediate child documents.
Therefore the page numbering in child documents may well
be inconsistent until the complete document is compiled at least once.

A useful (if unconventional) way to always ensure a consistent
page numbering is to restart the numbering in each child document
and denote the pages by `\textit{child}|.|\textit{page}'
where \textit{child} represents the chapter/section number of the child file.
This can be achieved by the command
|\numberwithin{page}{|\textit{child}|}|
of the \textsf{amsmath} package
where \textit{child} can be |chapter| or |section|
depending on the chosen structuring.
Alternatively, one can modify the macro |\thepage| appropriately
and reset the counter |page| at the start of each child file.

%%%%%%%%%%%%%%%%%%%%%%%%%%%%%%%%%%%%%%%%%%%%%%%%%%%%%%%%%%%%%%%%%%%%%%%%%%%%%%%%
\subsection{Conditional Processing}
\label{sec:conditional}

The package provides a mechanism to compile different versions
of a document. To customise the versions further some conditional processing
can come in handy to distinguish which version is being compiled.
The package provides two macros to describe the compilation context:

%%%%%%%%%%%%%%%%%%%%%%%%%%%%%%%%%%%%%%%%
\DescribeMacro{\ifchilddoc}
The conditional |\ifchilddoc| distinguishes between the compilation of
child documents and the main document:
%
\begin{center}
|\ifchilddoc |\textit{child-code}| |[|\||else |\textit{main-code}]| \||fi|
\end{center}

%%%%%%%%%%%%%%%%%%%%%%%%%%%%%%%%%%%%%%%%
\DescribeMacro{\childdocname}
\DescribeMacro{\childdocjob}
The macro |\childdocname| contains the filename (without extension)
of the main or child file being processed.
Note that |\childdocjob| will always contain the name of the main file.

%%%%%%%%%%%%%%%%%%%%%%%%%%%%%%%%%%%%%%%%
\paragraph{Title Page.}

Conditional processing can be used to include a title or banner page
in the main document when proper precautions are taken.
Importantly, the code in the main file should ensure that the page counter
(as well as other status parameters which are stored in the |.aux| files)
takes the same value after the conditional processing.
Otherwise the page numbers may take divergent values
depending on which part is compiled.

For example, a title page could be declared by:
%
\begin{center}
\begin{tabular}{l}
|\ifchilddoc\||else|\\
|\addtocounter{page}{-1}|\\
\textit{code for title page}\\
|\newpage|\\
|\||fi|
\end{tabular}
\end{center}
%
A banner page for the child documents can be generated by:
%
\begin{center}
\begin{tabular}{l}
|\ifchilddoc|\\
|\addtocounter{page}{-1}|\\
\textit{code for banner page}\\
|\newpage|\\
|\||fi|
\end{tabular}
\end{center}
%
Here one could write a message such as:
\begin{center}
|This is the part \childdocname{} of \childdocjob{}.|
\end{center}

%%%%%%%%%%%%%%%%%%%%%%%%%%%%%%%%%%%%%%%%%%%%%%%%%%%%%%%%%%%%%%%%%%%%%%%%%%%%%%%%
\subsection{Flags}
\label{sec:flags}

The package makes it easy to generate different versions
of the main or child documents.
To this end compilation flags can be defined
and assigned different default values.
They will be particularly useful in conjunction
with the forwarding mechanism described in \secref{sec:forward}.

For example, it may be useful to have a flag |\version|
which can be set to |draft| or |final|.
The document source will contain some conditional code
depending on the value of |\version|.
Suppose further, the flag should default to |final| for the main file
and to |draft| for child files
which is a natural assignment for editing the document.
This is achieved by placing the following code
in the preamble of the main document
(below the |\childdocmain| directive):
%
\begin{center}
\begin{tabular}{l}
|\ifchilddoc|\\
|\providecommand{\version}{draft}|\\
|\||else|\\
|\providecommand{\version}{final}|\\
|\||fi|
\end{tabular}
\end{center}
%
The definition by |\providecommand| makes sure
that previous definitions are not overwritten.
Further statements |\providecommand{\version}{...}|
can thus be added before the above code to override it.

For the main file, one might add a line
(between |\childdocmain| and the above block)
%
\begin{center}
|%\ifchilddoc\||else\providecommand{\version}{draft}\||fi|
\end{center}
%
which can be uncommented to produce a draft version.
Likewise one can add a line to the very top of a child file
(above the |\childdocof{|\textit{main}|}| directive)
%
\begin{center}
|%\providecommand{\version}{final}|
\end{center}
%
which can be uncommented to produce the final version of this child document.

%%%%%%%%%%%%%%%%%%%%%%%%%%%%%%%%%%%%%%%%%%%%%%%%%%%%%%%%%%%%%%%%%%%%%%%%%%%%%%%%
\subsection{Forwarding}
\label{sec:forward}

Different versions of the main or child documents
using compilation flags as described in \secref{sec:flags}
can be (permanently) stored in different files
for convenient compilation, viewing and distribution.
To this end, the package defines a command
to pass on compilation to a different file:

%%%%%%%%%%%%%%%%%%%%%%%%%%%%%%%%%%%%%%%%
\DescribeMacro{\childdocforward}
The command |\childdocforward| redirects processing to
another source file:
%
\begin{center}
\begin{tabular}{l}
|\input{childdoc.def}|\\
|\childdocforward[|\textit{main}|]{|\textit{dest}|}|\\
\end{tabular}
\end{center}
%
The argument \textit{dest} is the destination file
(without extension).
It should be the main file or one of the child files.
Note that further \textsf{childdoc} directives
such as |\childdocof| and |\childdocforward|
in the indicated file will be processed in this form.
The optional argument \textit{main}
passes on directly to the main file \textit{main}
while pretending to compile the child \textit{dest}.
This form behaves as if \textit{dest}
issues |\childdocof{|\textit{main}|}| right away,
and no further \textsf{childdoc} directives will be processed.

%%%%%%%%%%%%%%%%%%%%%%%%%%%%%%%%%%%%%%%%
\DescribeMacro{\...prefix}
In the alternative form |\childdocforwardprefix|,
%
\begin{center}
\begin{tabular}{l}
|\input{childdoc.def}|\\
|\childdocforwardprefix[|\textit{main}|]{|\textit{prefix}|}{|\textit{dest}|}|
\end{tabular}
\end{center}
%
the destination file is determined by a pattern
depending on the current file:
To make this work, the current file must be called
`{\textit{prefix}\hspace{0.2em}\textit{suffix}}'
with \textit{prefix} matching precisely the argument.
Processing is then passed on to the file
`{\textit{dest}\hspace{0.2em}\textit{suffix}}'.
Surely, the same effect is achieved by
directly specifying the
argument `{\textit{dest}\hspace{0.2em}\textit{suffix}}'
in the first form.
However, that requires to set up a different file
for each child. With the alternative form of the command
all these files can have exactly the same content
which simplifies setting them up and maintaining them.

For example, the following file |draft.tex|
with a compilation flag |\version| as described in \secref{sec:flags}
compiles the main document as a draft:
%
\begin{center}
\begin{tabular}{l}
|\def\version{draft}|\\
|\input{childdoc.def}|\\
|\childdocforward{|\textit{main}|}|
\end{tabular}
\end{center}
%
Likewise, the following files |final|\textit{nn}|.tex|
compile the final version of the child document
|child|\textit{nn}|.tex|:
%
\begin{center}
\begin{tabular}{l}
|\def\version{final}|\\
|\input{childdoc.def}|\\
|\childdocforwardprefix{final}{child}|
\end{tabular}
\end{center}
%

Note that when several versions of a main file and/or of each child file
are to be generated, it may be convenient to set up a |Makefile| or
shell script to automatise the process.

%%%%%%%%%%%%%%%%%%%%%%%%%%%%%%%%%%%%%%%%%%%%%%%%%%%%%%%%%%%%%%%%%%%%%%%%%%%%%%%%
\subsection{Command Line Processing}
\label{sec:commandline}

The effect of redirection files can also be achieved by invoking
the \LaTeX{} compiler with a more elaborate command line.
Most conveniently this should be done as part
of a shell script or a |Makefile|.

When using \textsf{childdoc} in the main file, the following
command lines effectively perform a redirection
(note that depending on the shell being used,
backslashes may have to be doubled: `|\|' $\to$ `|\\|'):
%
\begin{center}
|... -jobname "|\textit{target}|" |\\|"|[\textit{flags}]%
|\input{childdoc.def}\childdocforward[|\textit{main}|]{|\textit{dest}|}"|
\end{center}
%
Here \textit{target} is the name of the output file,
\textit{main} is the name of the main file
and \textit{dest} is the name of the main or child file to be processed
(all filenames without extensions).
The optional argument \textit{main} can be omitted
if \textit{main} matches \textit{dest}.
Optionally, compilation \textit{flags} can be defined via |\def| commands.
This command line makes the \TeX{} engine believe
it is compiling the file \textit{target}
whose content is specified as the latter parameter.
The provided code then forwards the processing to
\textit{main} or \textit{dest} as described in \secref{sec:forward}.

%%%%%%%%%%%%%%%%%%%%%%%%%%%%%%%%%%%%%%%%%%%%%%%%%%%%%%%%%%%%%%%%%%%%%%%%%%%%%%%%
\subsection{Include by Input}
\label{sec:input}

Including child documents by |\include| has some restrictions by design.
Most notably, the content of a child document always occupies
its own set of pages; pages cannot be shared between child documents.
Usually, this behaviour makes perfect sense
because each child document contain an essential part of the document.
However, in some situations it may be desirable to compose
a document from a collection of parts
without having mandatory page breaks between then.
For this case, the package
provides a mechanism to include parts
by |\input| which can also be processed individually.
However, by construction this mechanism
requires manual handling of the content to be output.

%%%%%%%%%%%%%%%%%%%%%%%%%%%%%%%%%%%%%%%%
\DescribeMacro{\ifchilddocmanual}
The main file should be prepared as usual, see \secref{sec:include}.
However, the document body must make a distinction
between processing of an individual part and of the main document, e.g.:
%
\begin{center}
\begin{tabular}{l}
|\ifchilddocmanual|\\
|\input{\childdocname}|\\
|\||else|\\
\textit{document body with }|\input{|\textit{part}|}|\\
|\||fi|
\end{tabular}
\end{center}
%
The conditional |\ifchilddocmanual| is true whenever
a part to be included by |\input| is being compiled,
and the name of the part is stored in |\childdocname|.

%%%%%%%%%%%%%%%%%%%%%%%%%%%%%%%%%%%%%%%%
\DescribeMacro{\childdocby}
Each part to be included by |\input| should start with:
%
\begin{center}
\begin{tabular}{l}
|\input{childdoc.def}|\\
|\childdocby{|\textit{main}|}|\\
\end{tabular}
\end{center}
%
The directive |\childdocby| is similar to |\childdocof|
described in \secref{sec:include},
but the subsequent selection of content must be done manually.
To that end, both |\ifchilddoc| and |\ifchilddocmanual|
will be true upon processing of a part,
and the name of the part is stored in |\childdocname|.
Note that |\jobname| will be set to the filename of the current part
so that each part receives an individual |.aux| file
that does not interfere with the |.aux| file(s) of the main document.
This behaviour can be altered by the alternative form
|\childdocby[*]{|\textit{main}|}| (with a non-empty optional argument)
which uses the |.aux| file of the main document
by setting |\jobname| to \textit{main}.

%%%%%%%%%%%%%%%%%%%%%%%%%%%%%%%%%%%%%%%%%%%%%%%%%%%%%%%%%%%%%%%%%%%%%%%%%%%%%%%%
\subsection{Driver Development}
\label{sec:driver}

The \textsf{childdoc} mechanism can also be use for the development
of definition files such as \LaTeX{} styles or classes.
This case differs from the above setup with multiple parts
included by |\include| in that no |\includeonly| should be invoked.
This can be achieved by starting the include file
(before |\ProvidesPackage|) with:
%
\begin{center}
\begin{tabular}{l}
|\input{childdoc.def}|\\
|\childdocforward{|\textit{main}|}|\\
\end{tabular}
\end{center}
%
or alternatively with:
%
\begin{center}
\begin{tabular}{l}
|\input{childdoc.def}|\\
|\childdocby{|\textit{main}|}|\\
\end{tabular}
\end{center}
%
Both forms have slightly different effects as described above.
The main file is prepared as usual, see \secref{sec:include}.

%%%%%%%%%%%%%%%%%%%%%%%%%%%%%%%%%%%%%%%%%%%%%%%%%%%%%%%%%%%%%%%%%%%%%%%%%%%%%%%%
\subsection{Legacy Detection}
\label{sec:detection}

The directive |\childdocmain| in the main file can detect
whether the complete document or merely a child is to be compiled
even without using the directive |\childdocof|.
This method is deprecated because it is less robust
and there is no compelling reason to use it;
it is merely provided for backward compatibility
and it may be removed in future versions.

If the detection mechanism is to be used,
it is mandatory to correctly specify
the filename of the main file as the argument of |\childdocmain|:
%
\begin{center}
\begin{tabular}{l}
|\input{childdoc.def}|\\
|\childdocmain{|\textit{main}|}|\\
\end{tabular}
\end{center}
%
If |\jobname| does not match the argument \textit{main} of |\childdocmain|,
it is assumed that |\jobname| points to the child file to be compiled.
When using |\childdocmain| with the main file specified as argument,
it suffices to start a child file
with just |\input{|\textit{main}|}|
without loading of the package and using |\childdocof|.
If instead all processing is done
with the appropriate \textsf{childdoc} directives,
the argument of \textit{main} of |\childdocmain| can be empty.

An alternative version of the command line processing described
in \secref{sec:commandline} using the detection mechanism reads:
%
\begin{center}
|... -jobname "|\textit{target}|" "|[\textit{flags}]%
[|\def\jobname{|\textit{dest}|}|]|\input{|\textit{main}|}"|
\end{center}

%%%%%%%%%%%%%%%%%%%%%%%%%%%%%%%%%%%%%%%%%%%%%%%%%%%%%%%%%%%%%%%%%%%%%%%%%%%%%%%%
\subsection{Manual Code}
\label{sec:manual}

In case one cannot be certain whether the definitions file |childdoc.def|
is installed on the target \TeX{} distribution
and one prefers not to ship it,
it is conceivable to paste a few relevant commands into the sources.

To that end, drop all statements |\input{childdoc.def}|
and perform the replacements as outlined below.
Instead of |\childdocmain{|\textit{main}|}| add the following code
to the top of the main file:
%
\begin{center}
\begin{tabular}{l}
|\||ifdefined\childdocname\endinput\||fi\newif\ifchilddoc|\\
|\edef\childdocname{\scantokens\expandafter{\jobname\noexpand}}|\\
|\def\childdocmain{|\textit{main}|}\||ifx\childdocmain\childdocname\||else|\\
|\childdoctrue\includeonly{\childdocname}\let\jobname\childdocmain\||fi|\\
\end{tabular}
\end{center}
%
Instead of |\childdocof{|\textit{main}|}| just include the main file
at the top of each child file:
%
\begin{center}
|\input{|\textit{main}|}|
\end{center}
%
A simple redirection |\childdocforward{|\textit{dest}|}| is achieved by:
%
\begin{center}
|\def\jobname{|\textit{dest}|}\input{\jobname}|
\end{center}
%
The redirection with prefix
|\childdocforwardprefix[|\textit{prefix}|]{|\textit{dest}|}|
is accomplished by:
%
\begin{center}
\begin{tabular}{l}
|{\edef\jobname{\scantokens\expandafter{\jobname\noexpand}}|\\
|\def\redirectjob |\textit{prefix}|#1~~~{\gdef\jobname{|\textit{dest}|#1}}|\\
|\expandafter\redirectjob\jobname~~~}\input{\jobname}|
\end{tabular}
\end{center}

In an alternative approach,
child documents can be compiled by a specific command line
without additional code or specific definitions:
%
\begin{center}
|... -jobname "|\textit{target}|" "|[\textit{flags}]%
|\includeonly{|\textit{dest}|}\input{|\textit{main}|}"|
\end{center}
%

%%%%%%%%%%%%%%%%%%%%%%%%%%%%%%%%%%%%%%%%%%%%%%%%%%%%%%%%%%%%%%%%%%%%%%%%%%%%%%%%
%%%%%%%%%%%%%%%%%%%%%%%%%%%%%%%%%%%%%%%%%%%%%%%%%%%%%%%%%%%%%%%%%%%%%%%%%%%%%%%%
\section{Information}

%%%%%%%%%%%%%%%%%%%%%%%%%%%%%%%%%%%%%%%%%%%%%%%%%%%%%%%%%%%%%%%%%%%%%%%%%%%%%%%%
\subsection{Copyright}

Copyright \copyright{} 2017--2018 Niklas Beisert

This work may be distributed and/or modified under the
conditions of the \LaTeX{} Project Public License, either version 1.3
of this license or (at your option) any later version.
The latest version of this license is in
  \url{http://www.latex-project.org/lppl.txt}
and version 1.3 or later is part of all distributions of \LaTeX{}
version 2005/12/01 or later.

This work has the LPPL maintenance status `maintained'.

The Current Maintainer of this work is Niklas Beisert.

This work consists of the files |README.txt|, |childdoc.ins| and |childdoc.dtx|
as well as the derived files |childdoc.def|, |cdocsamp.tex|
with |cdocsch1.tex|, |cdocsch2.tex|, |cdocspt3.tex|, |cdocspt4.tex|,
|cdocsdrf.tex|, |cdocsfn1.tex|, |cdocsfn2.tex|
as well as |childdoc.pdf|.

%%%%%%%%%%%%%%%%%%%%%%%%%%%%%%%%%%%%%%%%%%%%%%%%%%%%%%%%%%%%%%%%%%%%%%%%%%%%%%%%
\subsection{Files and Installation}

The package consists of the files:
%
\begin{center}
\begin{tabular}{ll}
    |README.txt|   & readme file \\
    |childdoc.ins| & installation file \\
    |childdoc.dtx| & source file \\
    |childdoc.def| & definition file \\
    |cdocsamp.tex| & sample main file \\
    |cdocsch1.tex| & sample include file \\
    |cdocsch2.tex| & sample include file \\
    |cdocspt3.tex| & sample part file \\
    |cdocspt4.tex| & sample part file \\
    |cdocsdrf.tex| & sample redirection file \\
    |cdocsfn1.tex| & sample redirection file \\
    |cdocsfn2.tex| & sample redirection file \\
    |childdoc.pdf| & manual
\end{tabular}
\end{center}
%
The distribution consists of the files
|README.txt|, |childdoc.ins| and |childdoc.dtx|.
%
\begin{itemize}
\item
Run (pdf)\LaTeX{} on |childdoc.dtx|
to compile the manual |childdoc.pdf| (this file).
\item
Run \LaTeX{} on |childdoc.ins| to create the definitions file |childdoc.def|
and the sample |cdocsamp.tex| with include files
|cdocsch1.tex|, |cdocsch2.tex|, |cdocspt3.tex|, |cdocspt4.tex|,
|cdocsdrf.tex|, |cdocsfn1.tex|, |cdocsfn2.tex|.
Then copy the file |childdoc.def| to an appropriate directory of your \LaTeX{}
distribution, e.g.\ \textit{texmf-root}|/tex/latex/childdoc|.
\end{itemize}

%%%%%%%%%%%%%%%%%%%%%%%%%%%%%%%%%%%%%%%%%%%%%%%%%%%%%%%%%%%%%%%%%%%%%%%%%%%%%%%%
\subsection{Related CTAN Packages}

There are several other packages which offer a similar functionality:
%
\begin{itemize}
\item
The packages
\href{http://ctan.org/pkg/docmute}{\textsf{docmute}},
\href{http://ctan.org/pkg/includex}{\textsf{includex}} and
\href{http://ctan.org/pkg/standalone}{\textsf{standalone}}
provide commands to include only the document body of
a child file thus allowing both files to be compiled individually.
\item
The packages \href{http://ctan.org/pkg/subdocs}{\textsf{subdocs}}
and \href{http://ctan.org/pkg/subfiles}{\textsf{subfiles}}
provide structures in which the main and child documents can be
encapsulated and allowing them to be compiled individually.
The inclusion mechanism is different from the conventional |\include|.
\item
The package \href{http://ctan.org/pkg/combine}{\textsf{combine}}
is an elaborate solution to combine several documents into one.
\end{itemize}
%
See also the CTAN topic \href{http://ctan.org/topic/subdocs}{\textsf{subdocs}}
for further related packages.
The present package differs from the above solutions in that
a document structure constructed with the conventional |\include| mechanism
just needs two extra commands at the top of every file
such that all constituent files can be compiled individually.

%%%%%%%%%%%%%%%%%%%%%%%%%%%%%%%%%%%%%%%%%%%%%%%%%%%%%%%%%%%%%%%%%%%%%%%%%%%%%%%%
%\subsection{Feature Suggestions}
%
%The following is a list of features which may be useful for future
%versions of this package:
%%
%\begin{itemize}
%\item
%\ldots
%\end{itemize}

%%%%%%%%%%%%%%%%%%%%%%%%%%%%%%%%%%%%%%%%%%%%%%%%%%%%%%%%%%%%%%%%%%%%%%%%%%%%%%%%
\subsection{Revision History}

%%%%%%%%%%%%%%%%%%%%%%%%%%%%%%%%%%%%%%%%
\paragraph{v2.0:} 2018/12/30

\begin{itemize}
\item
immediate forward processing
\item
added |\childdocby| mechanism
\item
manual restructured
\end{itemize}

%%%%%%%%%%%%%%%%%%%%%%%%%%%%%%%%%%%%%%%%
\paragraph{v1.6:} 2018/01/17

\begin{itemize}
\item
application for development of include files
\item
corrections to manual
\end{itemize}

%%%%%%%%%%%%%%%%%%%%%%%%%%%%%%%%%%%%%%%%
\paragraph{v1.5:} 2017/05/21

\begin{itemize}
\item
more complete structuring introduced
\item
|\childdocof| introduced
\item
|\childdoc| renamed to |\childdocmain|
\item
|\childredirect| renamed to |\childdocforward| and |\childdocforwardprefix|
and functionality expanded
\end{itemize}

%%%%%%%%%%%%%%%%%%%%%%%%%%%%%%%%%%%%%%%%
\paragraph{v1.0:} 2017/04/27

\begin{itemize}
\item
manual and install package
\item
first version published on CTAN
\end{itemize}

%%%%%%%%%%%%%%%%%%%%%%%%%%%%%%%%%%%%%%%%
\paragraph{v0.6:} 2017/04/26

\begin{itemize}
\item
redirection mechanism added
\end{itemize}

%%%%%%%%%%%%%%%%%%%%%%%%%%%%%%%%%%%%%%%%
\paragraph{v0.5:} 2017/04/26

\begin{itemize}
\item
functionality in definition file
\end{itemize}


%%%%%%%%%%%%%%%%%%%%%%%%%%%%%%%%%%%%%%%%%%%%%%%%%%%%%%%%%%%%%%%%%%%%%%%%%%%%%%%%
%%%%%%%%%%%%%%%%%%%%%%%%%%%%%%%%%%%%%%%%%%%%%%%%%%%%%%%%%%%%%%%%%%%%%%%%%%%%%%%%
%%%%%%%%%%%%%%%%%%%%%%%%%%%%%%%%%%%%%%%%%%%%%%%%%%%%%%%%%%%%%%%%%%%%%%%%%%%%%%%%
\appendix

\settowidth\MacroIndent{\rmfamily\scriptsize 000\ }

 \DocInput{childdoc.dtx}

\end{document}
%</driver>
% \fi
%
% %%%%%%%%%%%%%%%%%%%%%%%%%%%%%%%%%%%%%%%%%%%%%%%%%%%%%%%%%%%%%%%%%%%%%%%%%%%%%%
% %%%%%%%%%%%%%%%%%%%%%%%%%%%%%%%%%%%%%%%%%%%%%%%%%%%%%%%%%%%%%%%%%%%%%%%%%%%%%%
% \section{Sample}
%\iffalse
%<*samplemain>
%\fi
%
% The following presents a sample document
% with two chapters, two parts, a title page,
% a compile flag as well as three forwarding files to set the flag.
% It consists of eight |.tex| files:
% \begin{center}
% \begin{tabular}{ll}
% |cdocsamp.tex|&main file\\
% |cdocsch1.tex|&include file for chapter 1\\
% |cdocsch2.tex|&include file for chapter 2\\
% |cdocspt3.tex|&include file for part 3\\
% |cdocspt4.tex|&include file for part 4\\
% |cdocsdrf.tex|&forwarding file for main file in draft mode\\
% |cdocsfi1.tex|&forwarding file for final version of chapter 1\\
% |cdocsfi2.tex|&forwarding file for final version of chapter 2\\
% \end{tabular}
% \end{center}
% Each of the eight files can be compiled directly by the \LaTeX{} compiler.
%
% %%%%%%%%%%%%%%%%%%%%%%%%%%%%%%%%%%%%%%
% \paragraph{Main File.}
%
% The main file is called |cdocsamp.tex|.
%
% Load the \textsf{childdoc} definitions and
% declare the filename for the main document:
%    \begin{macrocode}
\input{childdoc.def}
\childdocmain{}
%    \end{macrocode}

% Optional override for |\version| flag:
%    \begin{macrocode}
%%\ifchilddoc\else\providecommand{\version}{draft}\fi
%    \end{macrocode}

% Define the default values for the |\version| flag
% (|final| for the main file and |draft| for childs):
%    \begin{macrocode}
\ifchilddoc
\providecommand{\version}{draft}
\else
\providecommand{\version}{final}
\fi
%    \end{macrocode}

% Load the standard document class:
%    \begin{macrocode}
\documentclass[12pt]{article}
%    \end{macrocode}

% Start the document body:
%    \begin{macrocode}
\begin{document}
%    \end{macrocode}

% Declare a title page.
% Print title, part of document being processed and version flag:
%    \begin{macrocode}
\addtocounter{page}{-1}
\begin{center}
{\LARGE\bfseries{}childdoc example\par}
\vspace{1cm}
\ifchilddoc
\ifchilddocmanual part\else chapter\fi:
`\childdocname' of `\childdocjob'\par
\else
main document: `\childdocjob'\par
\fi
version: \version\par
\end{center}
\newpage
%    \end{macrocode}

% Manually include selected file,
% otherwise process as usual:
%    \begin{macrocode}
\ifchilddocmanual
\section*{part `\childdocname'}
\input{\childdocname}
\else
%    \end{macrocode}

% Include the two chapters:
%    \begin{macrocode}
\include{cdocsch1}
\include{cdocsch2}
%    \end{macrocode}

% Include the two parts unless only chapters should be displayed:
%    \begin{macrocode}
\ifchilddoc\else
\section{part three}
\input{cdocspt3}
\section{part four}
\input{cdocspt4}
\fi
%    \end{macrocode}

% Process as usual until here:
%    \begin{macrocode}
\fi
%    \end{macrocode}

% End of document body:
%    \begin{macrocode}
\end{document}
%    \end{macrocode}
%\iffalse
%</samplemain>
%\fi
%
% %%%%%%%%%%%%%%%%%%%%%%%%%%%%%%%%%%%%%%
% \paragraph{Chapter Include Files.}
%
% The include files are called |cdocsch1.tex| and |cdocsch2.tex|.
%
%\iffalse
%<*samplechap1|samplechap2>
%\fi

% Optional override for |\version| flag:
%    \begin{macrocode}
%%\providecommand{\version}{final}
%    \end{macrocode}

% Include the main document:
%    \begin{macrocode}
\input{childdoc.def}
\childdocof{cdocsamp}
%    \end{macrocode}

%\iffalse
%</samplechap1|samplechap2>
%\fi
%
%\iffalse
%<*samplechap1>
%\fi
% Some text for chapter 1:
%    \begin{macrocode}
\section{one}
some text in chapter one
%    \end{macrocode}

%\iffalse
%</samplechap1>
%\fi
% Some text for chapter 2:
%\iffalse
%<*samplechap2>
%\fi
%    \begin{macrocode}
\section{two}
more text in chapter two
%    \end{macrocode}

%\iffalse
%</samplechap2>
%\fi
%
% %%%%%%%%%%%%%%%%%%%%%%%%%%%%%%%%%%%%%%
% \paragraph{Part Include Files.}
%
% The include files are called |cdocspt3.tex| and |cdocspt4.tex|.
%
%\iffalse
%<*samplepart3|samplepart4>
%\fi

% Optional override for |\version| flag:
%    \begin{macrocode}
%%\providecommand{\version}{final}
%    \end{macrocode}

% Include the main document:
%    \begin{macrocode}
\input{childdoc.def}
\childdocby{cdocsamp}
%    \end{macrocode}

%\iffalse
%</samplepart3|samplepart4>
%\fi
%
%\iffalse
%<*samplepart3>
%\fi
% Some text for part 3:
%    \begin{macrocode}
some text in part three
%    \end{macrocode}

%\iffalse
%</samplepart3>
%\fi
% Some text for part 4:
%\iffalse
%<*samplepart4>
%\fi
%    \begin{macrocode}
more text in part four
%    \end{macrocode}

%\iffalse
%</samplepart4>
%\fi
%
% %%%%%%%%%%%%%%%%%%%%%%%%%%%%%%%%%%%%%%
% \paragraph{Forwarding for a Complete Draft.}
%
% The following forwarding file |cdocsdrf.tex|
% compiles the main document in draft mode:
%\iffalse
%<*sampledraft>
%\fi
%    \begin{macrocode}
\def\version{draft}
\input{childdoc.def}
\childdocforward{cdocsamp}
%    \end{macrocode}

%\iffalse
%</sampledraft>
%\fi
%
% %%%%%%%%%%%%%%%%%%%%%%%%%%%%%%%%%%%%%%
% \paragraph{Forwarding for Final Version of the Chapters.}
%
% The following forwarding files |cdocsfn1.tex| and |cdocsfn2.tex|
% (with identical content)
% compile the final versions of the child documents
% |cdocsch1.tex| and |cdocsch2.tex|, respectively:
%\iffalse
%<*samplefinal>
%\fi
%    \begin{macrocode}
\def\version{final}
\input{childdoc.def}
\childdocforwardprefix[cdocsamp]{cdocsfn}{cdocsch}
%    \end{macrocode}

%\iffalse
%</samplefinal>
%\fi
%
% %%%%%%%%%%%%%%%%%%%%%%%%%%%%%%%%%%%%%%
% \paragraph{Command Line Processing.}
%
% The following three command lines generate the output files
% |cdocscld|, |cdocscl1| and |cdocscl2|
% which should be identical to
% |cdocsdrf|, |cdocsch1| and |cdocsfn2|, respectively:
% \begin{center}
% \begin{tabular}{l}
% |latex -jobname cdocscld \|\\
% |  "\def\version{draft}\input{childdoc.def}\childdocforward{cdocsamp}"|\\
% |latex -jobname cdocscl1 \|\\
% |  "\input{childdoc.def}\childdocforward[cdocsamp]{cdocsch1}"|\\
% |latex -jobname cdocscl2 \|\\
% |  "\def\version{final}\input{childdoc.def}\childdocforward{cdocsch2}"|
% \end{tabular}
% \end{center}
% Note that the trailing backslash on each first line
% merely continues the input to the second line
% (for convenient cut ant paste).
% Furthermore, the command |latex| can be replaced by any
% of its alternative versions such as |pdflatex|.
%
% %%%%%%%%%%%%%%%%%%%%%%%%%%%%%%%%%%%%%%%%%%%%%%%%%%%%%%%%%%%%%%%%%%%%%%%%%%%%%%
% %%%%%%%%%%%%%%%%%%%%%%%%%%%%%%%%%%%%%%%%%%%%%%%%%%%%%%%%%%%%%%%%%%%%%%%%%%%%%%
% \section{Implementation}
%\iffalse
%<*package>
%\fi
%
% This section describes the definitions file |childdoc.def|.

% The definitions cannot be loaded using |\usepackage| or |\RequirePackage|
% which has a mechanism to prevent loading a style file more than once.
% When loading the definitions by means of |\input|
% multiple instances have to be prevented manually:
%\iffalse
%This code needs to be before the `\ProvidesFile' directive
%which is defined at the beginning of this file.
%Therefore it is also placed there and commented out here.
%</package>
%<*discard>
%\fi
%    \begin{macrocode}
\ifdefined\childdocmain\endinput\fi
%    \end{macrocode}
%\iffalse
%</discard>
%<*package>
%\fi
%
% \macro{\ifchilddoc}
% \macro{\ifchilddocmanual}
% The conditional |\ifchilddoc| tells whether a
% child (true) or main (false) document is being compiled.
% The conditional |\ifchilddocmanual| tells whether
% the |\includeonly| mechanism is used (false) or
% the selection of child files must be performed manually (true).
% The definitions initialise to false:
%    \begin{macrocode}
\newif\ifchilddoc
\newif\ifchilddocmanual
%    \end{macrocode}

% \macro{\childdocname}
% \macro{\childdocjob}
% The macro |\childdocname| stores the name of the main document
% to be compiled. The macro |\childdocjob| stores the name of
% the document on which the \LaTeX{} compiler was originally invoked.
% The content of |\jobname| cannot be compared
% to filenames specified in the source due to different catcodes.
% The following code rescans |\jobname|, stores the result
% in |\childdocname| and saves a copy in |\childdocjob|:
%    \begin{macrocode}
\edef\childdocname{\scantokens\expandafter{\jobname\noexpand}}
\let\childdocjob\childdocname
%    \end{macrocode}

% \macro{\childdocdisable}
% The macro |\childdocdisable| prevents the main file
% from being processed more than once.
% At this stage, the main document command |\childdocmain|
% is assumed to be called once again where it should do nothing.
% Any subsequent call to it should prevent
% a secondary processing of the main document
% It overwrites the forwarding commands
% |\childdocof| and |\childdocforward|
% with empty macros to prevent further inclusions of the main document:
%    \begin{macrocode}
\newcommand{\childdocdisable}
{
  \renewcommand{\childdocmain}[1]{\renewcommand{\childdocmain}[1]{\endinput}}
  \renewcommand{\childdocof}[1]{}
  \renewcommand{\childdocby}[2][]{}
  \renewcommand{\childdocforward}[2][]{}
  \renewcommand{\childdocdisable}{}
}
%    \end{macrocode}

% \macro{\childdocmain}
% The macro |\childdocmain| is to be called at the top of the main file
% with nothing or the main filename (without extension) as argument.
% First, it breaks loops.
% If the argument is not empty and does not match |\childdocname|
% (which is set by the first inclusion of |childdoc.def|),
% |\ifchilddoc| is set to true, |\includeonly| is applied to the child file
% and |\jobname| is set to the main file
% (for proper handling of |.aux| files):
%    \begin{macrocode}
\newcommand{\childdocmain}[1]
{
  \childdocdisable\childdocmain{}
  \if?#1?\else
    \begingroup
      \def\childdoctmp{#1}
      \ifx\childdoctmp\childdocname
        \def\childdoctmp{}
      \else
        \def\childdoctmp
        {
          \childdoctrue
          \includeonly{\childdocname}
          \def\childdocjob{#1}
          \def\jobname{#1}
        }
      \fi
      \expandafter
    \endgroup
    \childdoctmp
  \fi
}
%    \end{macrocode}

% \macro{\childdocof}
% The command |\childdocof| redirects
% compilation to the main file |#1|.
%    \begin{macrocode}
\newcommand{\childdocof}[1]
{
  \childdocdisable
  \childdoctrue
  \includeonly{\childdocname}
  \def\jobname{#1}
  \def\childdocjob{#1}
  \input{#1}
}
%    \end{macrocode}

% \macro{\childdocby}
% The command |\childdocby| ....
%    \begin{macrocode}
\newcommand{\childdocby}[2][]
{
  \childdocdisable
  \childdoctrue
  \childdocmanualtrue
  \if?#1?\else
    \def\jobname{#2}
  \fi
  \def\childdocjob{#2}
  \input{#2}
  \endinput
}
%    \end{macrocode}

% \macro{\childdocforward}
% The command |\childdocforward| redirects
% compilation to the main file or
% (if the optional argument is given) a child file.
% Parameters are set as if the main file
% or a child file starting with |\childdocof| was compiled.
% Then compilation is handed over to the main file:
%    \begin{macrocode}
\newcommand{\childdocforward}[2][]
{
  \begingroup
    \if?#1?
      \def\childdoctmp
      {
        \def\childdocname{#2}
        \def\childdocjob{#2}
        \def\jobname{#2}
        \input{#2}
        \endinput
      }
    \else
      \def\childdoctmp
      {
        \childdocdisable
        \def\childdocname{#2}
        \childdoctrue
        \includeonly{#2}
        \def\childdocjob{#1}
        \def\jobname{#1}
        \input{#1}
        \endinput
      }
    \fi
    \expandafter
  \endgroup
  \childdoctmp
}
%    \end{macrocode}

% \macro{\childdocforwardprefix}
% The command |\childdocforwardprefix| redirects
% compilation to the main or a child file by means of a pattern.
% The prefix |#1| in the current filename is replaced by |#2|
% and the suffix of the current filename is kept
% (it is assumed that the filename does not contain the substring `|~~~|'
% which is used as a delimiter).
% Compilation is handed over to the new file by |\childdocforward|:
%    \begin{macrocode}
\newcommand{\childdocforwardprefix}[3][]
{
  \begingroup
    \def\childdocextract #2##1~~~{\def\childdoctmp{\childdocforward[#1]{#3##1}}}
    \expandafter\childdocextract\childdocname~~~
    \expandafter
  \endgroup
  \childdoctmp
}
%    \end{macrocode}

% \macro{\childdoc}
% The deprecated macro |\childdoc| is a legacy version of |\childdocmain|:
%    \begin{macrocode}
\newcommand{\childdoc}{\childdocmain}
%    \end{macrocode}

% \macro{\childdocredirect}
% The deprecated macro |\childdocredirect| is a legacy version
% of |\childdocforward| and |\childdocforwardprefix|:
%    \begin{macrocode}
\newcommand{\childdocredirect}[2][]
{
  \begingroup
    \if?#1?
      \def\childdoctmp{\childdocforward{#2}}
    \else
      \def\childdoctmp{\childdocforwardprefix{#1}{#2}}
    \fi
    \expandafter
  \endgroup
  \childdoctmp
}
%    \end{macrocode}

%\iffalse
%</package>
%\fi
%
\endinput
|\\
|\childdocforwardprefix[|\textit{main}|]{|\textit{prefix}|}{|\textit{dest}|}|
\end{tabular}
\end{center}
%
the destination file is determined by a pattern
depending on the current file:
To make this work, the current file must be called
`{\textit{prefix}\hspace{0.2em}\textit{suffix}}'
with \textit{prefix} matching precisely the argument.
Processing is then passed on to the file
`{\textit{dest}\hspace{0.2em}\textit{suffix}}'.
Surely, the same effect is achieved by
directly specifying the
argument `{\textit{dest}\hspace{0.2em}\textit{suffix}}'
in the first form.
However, that requires to set up a different file
for each child. With the alternative form of the command
all these files can have exactly the same content
which simplifies setting them up and maintaining them.

For example, the following file |draft.tex|
with a compilation flag |\version| as described in \secref{sec:flags}
compiles the main document as a draft:
%
\begin{center}
\begin{tabular}{l}
|\def\version{draft}|\\
|% \iffalse
%
% childdoc.dtx Copyright (C) 2017-2018 Niklas Beisert
%
% This work may be distributed and/or modified under the
% conditions of the LaTeX Project Public License, either version 1.3
% of this license or (at your option) any later version.
% The latest version of this license is in
%   http://www.latex-project.org/lppl.txt
% and version 1.3 or later is part of all distributions of LaTeX
% version 2005/12/01 or later.
%
% This work has the LPPL maintenance status `maintained'.
%
% The Current Maintainer of this work is Niklas Beisert.
%
% This work consists of the files childdoc.dtx and childdoc.ins
% and the derived files childdoc.def and cdocsamp.tex with
% cdocsch1.tex, cdocsch2.tex, cdocsdrf.tex, cdocsfn1.tex, cdocsfn2.tex.
%
%<package>\ifdefined\childdocmain\endinput\fi
%<package>\ProvidesFile{childdoc.def}[2018/12/30 v2.0 child document driver]
%<samplemain>\ProvidesFile{cdocsamp.tex}[2018/12/30 v2.0 sample for childdoc]
%<*driver>
%\ProvidesFile{childdoc.drv}[2018/12/30 v2.0 childdoc reference manual file]
\PassOptionsToClass{10pt,a4paper}{article}
\documentclass{ltxdoc}

\usepackage[margin=35mm]{geometry}
\usepackage{hyperref}
\usepackage{hyperxmp}
\usepackage[usenames]{color}

\hypersetup{colorlinks=true}
\hypersetup{pdfstartview=FitH}
\hypersetup{pdfpagemode=UseNone}
\hypersetup{pdfsource={}}
\hypersetup{pdflang={en-UK}}
\hypersetup{pdfcopyright={Copyright 2017-2018 Niklas Beisert.
  This work may be distributed and/or modified under the
  conditions of the LaTeX Project Public License, either version 1.3
  of this license or (at your option) any later version.}}
\hypersetup{pdflicenseurl={http://www.latex-project.org/lppl.txt}}
\hypersetup{pdfcontactaddress={ETH Zurich, ITP, HIT K,
  Wolfgang-Pauli-Strasse 27}}
\hypersetup{pdfcontactpostcode={8093}}
\hypersetup{pdfcontactcity={Zurich}}
\hypersetup{pdfcontactcountry={Switzerland}}
\hypersetup{pdfcontactemail={nbeisert@itp.phys.ethz.ch}}
\hypersetup{pdfcontacturl={http://people.phys.ethz.ch/\xmptilde nbeisert/}}

\newcommand{\secref}[1]{\hyperref[#1]{section \ref*{#1}}}

\parskip1ex
\parindent0pt
\let\olditemize\itemize
\def\itemize{\olditemize\parskip0pt}

\begin{document}

\title{The \textsf{childdoc} Package}
\hypersetup{pdftitle={The childdoc Package}}
\author{Niklas Beisert\\[2ex]
  Institut f\"ur Theoretische Physik\\
  Eidgen\"ossische Technische Hochschule Z\"urich\\
  Wolfgang-Pauli-Strasse 27, 8093 Z\"urich, Switzerland\\[1ex]
  \href{mailto:nbeisert@itp.phys.ethz.ch}
  {\texttt{nbeisert@itp.phys.ethz.ch}}}
\hypersetup{pdfauthor={Niklas Beisert}}
\hypersetup{pdfsubject={Manual for the LaTeX2e Package childdoc}}
\date{30 December 2018, \textsf{v2.0}}
\maketitle

\begin{abstract}\noindent
\textsf{childdoc} is a \LaTeXe{} package
that enables the direct compilation
of document sections included by |\include|
to individual files.
\end{abstract}

\begingroup
\parskip0ex
\tableofcontents
\endgroup

%%%%%%%%%%%%%%%%%%%%%%%%%%%%%%%%%%%%%%%%%%%%%%%%%%%%%%%%%%%%%%%%%%%%%%%%%%%%%%%%
%%%%%%%%%%%%%%%%%%%%%%%%%%%%%%%%%%%%%%%%%%%%%%%%%%%%%%%%%%%%%%%%%%%%%%%%%%%%%%%%
\section{Introduction}

\LaTeX{} provides a mechanism to structure a large document (such as a book)
into a main file and several child files (containing the chapters)
using the |\include| command.
This mechanism is beneficial for documents
which span hundreds of pages in order to
make the source file(s) more manageable.
Moreover, compilation can be restricted to
selected child files by means of the |\includeonly| command.
The latter feature can be used to reduce the compilation time while editing
(this was significantly more useful in the earlier days of \LaTeX{})
or to generate a smaller document which is easier to navigate.
Another application of |\includeonly| is to generate
documents consisting of selected parts of the complete document.

However, there are a few drawbacks of the plain |\include| mechanism:
\begin{itemize}
\item
The child files cannot be compiled on their own,
they can only be compiled via the main file.
A naive editing environment
(such as a text editor with an option
to have the current file processed by \LaTeX)
may require one to switch to the main file before compiling;
attempting to compile the child file produces errors.
\item
The main file must be modified (each time)
to adjust the |\includeonly| command
to the present needs. This easily leaves the main file in a messy state.
\item
The generated document will always carry the filename
of the main document. This is inconvenient if
several child files are to be compiled and
to be kept for distribution.
\end{itemize}

The present package provides a simple interface
to make child files individually compilable by \LaTeX{}.
Compiling a child file then has the same effect as compiling
the main file with an |\includeonly| command
to select the appropriate child.
Moreover the generated document will carry the name of the child
rather than the main file.
This resolves all three above issues.

This feature is meant to make the editing of books,
thesis documents and lecture notes somewhat more convenient.
However, the package can also be used efficiently for
composing a series of documents (such as exercise sheets)
which are typically distributed individually.
It then assists the author in generating the individual documents
(potentially in different versions)
as well as a document containing the collected series.
Another application is in developing style files
or other kinds of included material
where compilation of the style file could redirect
to a sample or test file.

%%%%%%%%%%%%%%%%%%%%%%%%%%%%%%%%%%%%%%%%%%%%%%%%%%%%%%%%%%%%%%%%%%%%%%%%%%%%%%%%
%%%%%%%%%%%%%%%%%%%%%%%%%%%%%%%%%%%%%%%%%%%%%%%%%%%%%%%%%%%%%%%%%%%%%%%%%%%%%%%%
\section{Usage}

First of all, the package \textsf{childdoc} is \emph{not} a standard
\LaTeXe{} |.sty| style file! Therefore it needs to be invoked in
a non-standard way.

%%%%%%%%%%%%%%%%%%%%%%%%%%%%%%%%%%%%%%%%%%%%%%%%%%%%%%%%%%%%%%%%%%%%%%%%%%%%%%%%
\subsection{Included Files}
\label{sec:include}

%%%%%%%%%%%%%%%%%%%%%%%%%%%%%%%%%%%%%%%%
\DescribeMacro{\childdocmain}
To use the package, add the commands
\begin{center}
\begin{tabular}{l}
|\input{childdoc.def}|\\
|\childdocmain{}|\\
\end{tabular}
\end{center}
at the very top of the main \LaTeX{} file,
in particular \emph{before} the |\documentclass| statement!
The argument of |\childdocmain| should be left empty
(but it must be present).

%%%%%%%%%%%%%%%%%%%%%%%%%%%%%%%%%%%%%%%%
\DescribeMacro{\childdocof}
Furthermore, add the commands
\begin{center}
\begin{tabular}{l}
|\input{childdoc.def}|\\
|\childdocof{|\textit{main}|}|\\
\end{tabular}
\end{center}
at the top of every child file \textit{child}
which is included by |\include{|\textit{child}|}|
from within the main file
(or at least for those files to be compiled individually).
The argument \textit{main} must be the filename of the main file.

There are a couple of
considerations in setting up the main and child documents:

%%%%%%%%%%%%%%%%%%%%%%%%%%%%%%%%%%%%%%%%
\paragraph{Restrictions.}

Please note the following restrictions:
\begin{itemize}
\item
|\childdocmain| must be called with one argument \textit{main}
to ensure compatibility with earlier version of the package.
It must either be empty (|\childdocmain{}|)
or precisely match the filename of the main file in which it is specified.
See \secref{sec:detection} for further information.
\item
The filename \textit{main} must be specified without the |.tex| extension.
\item
The filename \textit{main} is case sensitive
(even in case-insensitive file systems)
due to internal string comparison.
\item
The argument \textit{main} should be fully expanded, it cannot be a macro.
\item
Subdirectories and special characters should be avoided in filenames.
\item
The command |\childdocmain{|\textit{main}|}| must be followed by a whitespace.
It should not be followed immediately by another command
or by a comment mark `|%|'.
This is because the \TeX{} parser reads the token immediately following
the argument of |\childdocmain| and puts it
at the beginning of every child section;
however, a white\-space is ignored.
\end{itemize}

%%%%%%%%%%%%%%%%%%%%%%%%%%%%%%%%%%%%%%%%
\paragraph{Content of Main File.}

It is advisable to place all content in the child files included by |\include|.
Any output contained in the main file will appear in all child documents
unless suppressed manually;
it cannot be suppressed automatically by the |\includeonly| directive
and thus should normally be avoided.
A method to include some content in the main file
by means of conditional processing is described in \secref{sec:conditional}.

%%%%%%%%%%%%%%%%%%%%%%%%%%%%%%%%%%%%%%%%
\paragraph{Page Numbering.}

When only a part of the document is compiled,
the appropriate numbering of pages
(as well as other status parameters)
is determined from the |.aux| files.
The latter contain information from previous passes.
However this information needs to propagate through
all intermediate child documents.
Therefore the page numbering in child documents may well
be inconsistent until the complete document is compiled at least once.

A useful (if unconventional) way to always ensure a consistent
page numbering is to restart the numbering in each child document
and denote the pages by `\textit{child}|.|\textit{page}'
where \textit{child} represents the chapter/section number of the child file.
This can be achieved by the command
|\numberwithin{page}{|\textit{child}|}|
of the \textsf{amsmath} package
where \textit{child} can be |chapter| or |section|
depending on the chosen structuring.
Alternatively, one can modify the macro |\thepage| appropriately
and reset the counter |page| at the start of each child file.

%%%%%%%%%%%%%%%%%%%%%%%%%%%%%%%%%%%%%%%%%%%%%%%%%%%%%%%%%%%%%%%%%%%%%%%%%%%%%%%%
\subsection{Conditional Processing}
\label{sec:conditional}

The package provides a mechanism to compile different versions
of a document. To customise the versions further some conditional processing
can come in handy to distinguish which version is being compiled.
The package provides two macros to describe the compilation context:

%%%%%%%%%%%%%%%%%%%%%%%%%%%%%%%%%%%%%%%%
\DescribeMacro{\ifchilddoc}
The conditional |\ifchilddoc| distinguishes between the compilation of
child documents and the main document:
%
\begin{center}
|\ifchilddoc |\textit{child-code}| |[|\||else |\textit{main-code}]| \||fi|
\end{center}

%%%%%%%%%%%%%%%%%%%%%%%%%%%%%%%%%%%%%%%%
\DescribeMacro{\childdocname}
\DescribeMacro{\childdocjob}
The macro |\childdocname| contains the filename (without extension)
of the main or child file being processed.
Note that |\childdocjob| will always contain the name of the main file.

%%%%%%%%%%%%%%%%%%%%%%%%%%%%%%%%%%%%%%%%
\paragraph{Title Page.}

Conditional processing can be used to include a title or banner page
in the main document when proper precautions are taken.
Importantly, the code in the main file should ensure that the page counter
(as well as other status parameters which are stored in the |.aux| files)
takes the same value after the conditional processing.
Otherwise the page numbers may take divergent values
depending on which part is compiled.

For example, a title page could be declared by:
%
\begin{center}
\begin{tabular}{l}
|\ifchilddoc\||else|\\
|\addtocounter{page}{-1}|\\
\textit{code for title page}\\
|\newpage|\\
|\||fi|
\end{tabular}
\end{center}
%
A banner page for the child documents can be generated by:
%
\begin{center}
\begin{tabular}{l}
|\ifchilddoc|\\
|\addtocounter{page}{-1}|\\
\textit{code for banner page}\\
|\newpage|\\
|\||fi|
\end{tabular}
\end{center}
%
Here one could write a message such as:
\begin{center}
|This is the part \childdocname{} of \childdocjob{}.|
\end{center}

%%%%%%%%%%%%%%%%%%%%%%%%%%%%%%%%%%%%%%%%%%%%%%%%%%%%%%%%%%%%%%%%%%%%%%%%%%%%%%%%
\subsection{Flags}
\label{sec:flags}

The package makes it easy to generate different versions
of the main or child documents.
To this end compilation flags can be defined
and assigned different default values.
They will be particularly useful in conjunction
with the forwarding mechanism described in \secref{sec:forward}.

For example, it may be useful to have a flag |\version|
which can be set to |draft| or |final|.
The document source will contain some conditional code
depending on the value of |\version|.
Suppose further, the flag should default to |final| for the main file
and to |draft| for child files
which is a natural assignment for editing the document.
This is achieved by placing the following code
in the preamble of the main document
(below the |\childdocmain| directive):
%
\begin{center}
\begin{tabular}{l}
|\ifchilddoc|\\
|\providecommand{\version}{draft}|\\
|\||else|\\
|\providecommand{\version}{final}|\\
|\||fi|
\end{tabular}
\end{center}
%
The definition by |\providecommand| makes sure
that previous definitions are not overwritten.
Further statements |\providecommand{\version}{...}|
can thus be added before the above code to override it.

For the main file, one might add a line
(between |\childdocmain| and the above block)
%
\begin{center}
|%\ifchilddoc\||else\providecommand{\version}{draft}\||fi|
\end{center}
%
which can be uncommented to produce a draft version.
Likewise one can add a line to the very top of a child file
(above the |\childdocof{|\textit{main}|}| directive)
%
\begin{center}
|%\providecommand{\version}{final}|
\end{center}
%
which can be uncommented to produce the final version of this child document.

%%%%%%%%%%%%%%%%%%%%%%%%%%%%%%%%%%%%%%%%%%%%%%%%%%%%%%%%%%%%%%%%%%%%%%%%%%%%%%%%
\subsection{Forwarding}
\label{sec:forward}

Different versions of the main or child documents
using compilation flags as described in \secref{sec:flags}
can be (permanently) stored in different files
for convenient compilation, viewing and distribution.
To this end, the package defines a command
to pass on compilation to a different file:

%%%%%%%%%%%%%%%%%%%%%%%%%%%%%%%%%%%%%%%%
\DescribeMacro{\childdocforward}
The command |\childdocforward| redirects processing to
another source file:
%
\begin{center}
\begin{tabular}{l}
|\input{childdoc.def}|\\
|\childdocforward[|\textit{main}|]{|\textit{dest}|}|\\
\end{tabular}
\end{center}
%
The argument \textit{dest} is the destination file
(without extension).
It should be the main file or one of the child files.
Note that further \textsf{childdoc} directives
such as |\childdocof| and |\childdocforward|
in the indicated file will be processed in this form.
The optional argument \textit{main}
passes on directly to the main file \textit{main}
while pretending to compile the child \textit{dest}.
This form behaves as if \textit{dest}
issues |\childdocof{|\textit{main}|}| right away,
and no further \textsf{childdoc} directives will be processed.

%%%%%%%%%%%%%%%%%%%%%%%%%%%%%%%%%%%%%%%%
\DescribeMacro{\...prefix}
In the alternative form |\childdocforwardprefix|,
%
\begin{center}
\begin{tabular}{l}
|\input{childdoc.def}|\\
|\childdocforwardprefix[|\textit{main}|]{|\textit{prefix}|}{|\textit{dest}|}|
\end{tabular}
\end{center}
%
the destination file is determined by a pattern
depending on the current file:
To make this work, the current file must be called
`{\textit{prefix}\hspace{0.2em}\textit{suffix}}'
with \textit{prefix} matching precisely the argument.
Processing is then passed on to the file
`{\textit{dest}\hspace{0.2em}\textit{suffix}}'.
Surely, the same effect is achieved by
directly specifying the
argument `{\textit{dest}\hspace{0.2em}\textit{suffix}}'
in the first form.
However, that requires to set up a different file
for each child. With the alternative form of the command
all these files can have exactly the same content
which simplifies setting them up and maintaining them.

For example, the following file |draft.tex|
with a compilation flag |\version| as described in \secref{sec:flags}
compiles the main document as a draft:
%
\begin{center}
\begin{tabular}{l}
|\def\version{draft}|\\
|\input{childdoc.def}|\\
|\childdocforward{|\textit{main}|}|
\end{tabular}
\end{center}
%
Likewise, the following files |final|\textit{nn}|.tex|
compile the final version of the child document
|child|\textit{nn}|.tex|:
%
\begin{center}
\begin{tabular}{l}
|\def\version{final}|\\
|\input{childdoc.def}|\\
|\childdocforwardprefix{final}{child}|
\end{tabular}
\end{center}
%

Note that when several versions of a main file and/or of each child file
are to be generated, it may be convenient to set up a |Makefile| or
shell script to automatise the process.

%%%%%%%%%%%%%%%%%%%%%%%%%%%%%%%%%%%%%%%%%%%%%%%%%%%%%%%%%%%%%%%%%%%%%%%%%%%%%%%%
\subsection{Command Line Processing}
\label{sec:commandline}

The effect of redirection files can also be achieved by invoking
the \LaTeX{} compiler with a more elaborate command line.
Most conveniently this should be done as part
of a shell script or a |Makefile|.

When using \textsf{childdoc} in the main file, the following
command lines effectively perform a redirection
(note that depending on the shell being used,
backslashes may have to be doubled: `|\|' $\to$ `|\\|'):
%
\begin{center}
|... -jobname "|\textit{target}|" |\\|"|[\textit{flags}]%
|\input{childdoc.def}\childdocforward[|\textit{main}|]{|\textit{dest}|}"|
\end{center}
%
Here \textit{target} is the name of the output file,
\textit{main} is the name of the main file
and \textit{dest} is the name of the main or child file to be processed
(all filenames without extensions).
The optional argument \textit{main} can be omitted
if \textit{main} matches \textit{dest}.
Optionally, compilation \textit{flags} can be defined via |\def| commands.
This command line makes the \TeX{} engine believe
it is compiling the file \textit{target}
whose content is specified as the latter parameter.
The provided code then forwards the processing to
\textit{main} or \textit{dest} as described in \secref{sec:forward}.

%%%%%%%%%%%%%%%%%%%%%%%%%%%%%%%%%%%%%%%%%%%%%%%%%%%%%%%%%%%%%%%%%%%%%%%%%%%%%%%%
\subsection{Include by Input}
\label{sec:input}

Including child documents by |\include| has some restrictions by design.
Most notably, the content of a child document always occupies
its own set of pages; pages cannot be shared between child documents.
Usually, this behaviour makes perfect sense
because each child document contain an essential part of the document.
However, in some situations it may be desirable to compose
a document from a collection of parts
without having mandatory page breaks between then.
For this case, the package
provides a mechanism to include parts
by |\input| which can also be processed individually.
However, by construction this mechanism
requires manual handling of the content to be output.

%%%%%%%%%%%%%%%%%%%%%%%%%%%%%%%%%%%%%%%%
\DescribeMacro{\ifchilddocmanual}
The main file should be prepared as usual, see \secref{sec:include}.
However, the document body must make a distinction
between processing of an individual part and of the main document, e.g.:
%
\begin{center}
\begin{tabular}{l}
|\ifchilddocmanual|\\
|\input{\childdocname}|\\
|\||else|\\
\textit{document body with }|\input{|\textit{part}|}|\\
|\||fi|
\end{tabular}
\end{center}
%
The conditional |\ifchilddocmanual| is true whenever
a part to be included by |\input| is being compiled,
and the name of the part is stored in |\childdocname|.

%%%%%%%%%%%%%%%%%%%%%%%%%%%%%%%%%%%%%%%%
\DescribeMacro{\childdocby}
Each part to be included by |\input| should start with:
%
\begin{center}
\begin{tabular}{l}
|\input{childdoc.def}|\\
|\childdocby{|\textit{main}|}|\\
\end{tabular}
\end{center}
%
The directive |\childdocby| is similar to |\childdocof|
described in \secref{sec:include},
but the subsequent selection of content must be done manually.
To that end, both |\ifchilddoc| and |\ifchilddocmanual|
will be true upon processing of a part,
and the name of the part is stored in |\childdocname|.
Note that |\jobname| will be set to the filename of the current part
so that each part receives an individual |.aux| file
that does not interfere with the |.aux| file(s) of the main document.
This behaviour can be altered by the alternative form
|\childdocby[*]{|\textit{main}|}| (with a non-empty optional argument)
which uses the |.aux| file of the main document
by setting |\jobname| to \textit{main}.

%%%%%%%%%%%%%%%%%%%%%%%%%%%%%%%%%%%%%%%%%%%%%%%%%%%%%%%%%%%%%%%%%%%%%%%%%%%%%%%%
\subsection{Driver Development}
\label{sec:driver}

The \textsf{childdoc} mechanism can also be use for the development
of definition files such as \LaTeX{} styles or classes.
This case differs from the above setup with multiple parts
included by |\include| in that no |\includeonly| should be invoked.
This can be achieved by starting the include file
(before |\ProvidesPackage|) with:
%
\begin{center}
\begin{tabular}{l}
|\input{childdoc.def}|\\
|\childdocforward{|\textit{main}|}|\\
\end{tabular}
\end{center}
%
or alternatively with:
%
\begin{center}
\begin{tabular}{l}
|\input{childdoc.def}|\\
|\childdocby{|\textit{main}|}|\\
\end{tabular}
\end{center}
%
Both forms have slightly different effects as described above.
The main file is prepared as usual, see \secref{sec:include}.

%%%%%%%%%%%%%%%%%%%%%%%%%%%%%%%%%%%%%%%%%%%%%%%%%%%%%%%%%%%%%%%%%%%%%%%%%%%%%%%%
\subsection{Legacy Detection}
\label{sec:detection}

The directive |\childdocmain| in the main file can detect
whether the complete document or merely a child is to be compiled
even without using the directive |\childdocof|.
This method is deprecated because it is less robust
and there is no compelling reason to use it;
it is merely provided for backward compatibility
and it may be removed in future versions.

If the detection mechanism is to be used,
it is mandatory to correctly specify
the filename of the main file as the argument of |\childdocmain|:
%
\begin{center}
\begin{tabular}{l}
|\input{childdoc.def}|\\
|\childdocmain{|\textit{main}|}|\\
\end{tabular}
\end{center}
%
If |\jobname| does not match the argument \textit{main} of |\childdocmain|,
it is assumed that |\jobname| points to the child file to be compiled.
When using |\childdocmain| with the main file specified as argument,
it suffices to start a child file
with just |\input{|\textit{main}|}|
without loading of the package and using |\childdocof|.
If instead all processing is done
with the appropriate \textsf{childdoc} directives,
the argument of \textit{main} of |\childdocmain| can be empty.

An alternative version of the command line processing described
in \secref{sec:commandline} using the detection mechanism reads:
%
\begin{center}
|... -jobname "|\textit{target}|" "|[\textit{flags}]%
[|\def\jobname{|\textit{dest}|}|]|\input{|\textit{main}|}"|
\end{center}

%%%%%%%%%%%%%%%%%%%%%%%%%%%%%%%%%%%%%%%%%%%%%%%%%%%%%%%%%%%%%%%%%%%%%%%%%%%%%%%%
\subsection{Manual Code}
\label{sec:manual}

In case one cannot be certain whether the definitions file |childdoc.def|
is installed on the target \TeX{} distribution
and one prefers not to ship it,
it is conceivable to paste a few relevant commands into the sources.

To that end, drop all statements |\input{childdoc.def}|
and perform the replacements as outlined below.
Instead of |\childdocmain{|\textit{main}|}| add the following code
to the top of the main file:
%
\begin{center}
\begin{tabular}{l}
|\||ifdefined\childdocname\endinput\||fi\newif\ifchilddoc|\\
|\edef\childdocname{\scantokens\expandafter{\jobname\noexpand}}|\\
|\def\childdocmain{|\textit{main}|}\||ifx\childdocmain\childdocname\||else|\\
|\childdoctrue\includeonly{\childdocname}\let\jobname\childdocmain\||fi|\\
\end{tabular}
\end{center}
%
Instead of |\childdocof{|\textit{main}|}| just include the main file
at the top of each child file:
%
\begin{center}
|\input{|\textit{main}|}|
\end{center}
%
A simple redirection |\childdocforward{|\textit{dest}|}| is achieved by:
%
\begin{center}
|\def\jobname{|\textit{dest}|}\input{\jobname}|
\end{center}
%
The redirection with prefix
|\childdocforwardprefix[|\textit{prefix}|]{|\textit{dest}|}|
is accomplished by:
%
\begin{center}
\begin{tabular}{l}
|{\edef\jobname{\scantokens\expandafter{\jobname\noexpand}}|\\
|\def\redirectjob |\textit{prefix}|#1~~~{\gdef\jobname{|\textit{dest}|#1}}|\\
|\expandafter\redirectjob\jobname~~~}\input{\jobname}|
\end{tabular}
\end{center}

In an alternative approach,
child documents can be compiled by a specific command line
without additional code or specific definitions:
%
\begin{center}
|... -jobname "|\textit{target}|" "|[\textit{flags}]%
|\includeonly{|\textit{dest}|}\input{|\textit{main}|}"|
\end{center}
%

%%%%%%%%%%%%%%%%%%%%%%%%%%%%%%%%%%%%%%%%%%%%%%%%%%%%%%%%%%%%%%%%%%%%%%%%%%%%%%%%
%%%%%%%%%%%%%%%%%%%%%%%%%%%%%%%%%%%%%%%%%%%%%%%%%%%%%%%%%%%%%%%%%%%%%%%%%%%%%%%%
\section{Information}

%%%%%%%%%%%%%%%%%%%%%%%%%%%%%%%%%%%%%%%%%%%%%%%%%%%%%%%%%%%%%%%%%%%%%%%%%%%%%%%%
\subsection{Copyright}

Copyright \copyright{} 2017--2018 Niklas Beisert

This work may be distributed and/or modified under the
conditions of the \LaTeX{} Project Public License, either version 1.3
of this license or (at your option) any later version.
The latest version of this license is in
  \url{http://www.latex-project.org/lppl.txt}
and version 1.3 or later is part of all distributions of \LaTeX{}
version 2005/12/01 or later.

This work has the LPPL maintenance status `maintained'.

The Current Maintainer of this work is Niklas Beisert.

This work consists of the files |README.txt|, |childdoc.ins| and |childdoc.dtx|
as well as the derived files |childdoc.def|, |cdocsamp.tex|
with |cdocsch1.tex|, |cdocsch2.tex|, |cdocspt3.tex|, |cdocspt4.tex|,
|cdocsdrf.tex|, |cdocsfn1.tex|, |cdocsfn2.tex|
as well as |childdoc.pdf|.

%%%%%%%%%%%%%%%%%%%%%%%%%%%%%%%%%%%%%%%%%%%%%%%%%%%%%%%%%%%%%%%%%%%%%%%%%%%%%%%%
\subsection{Files and Installation}

The package consists of the files:
%
\begin{center}
\begin{tabular}{ll}
    |README.txt|   & readme file \\
    |childdoc.ins| & installation file \\
    |childdoc.dtx| & source file \\
    |childdoc.def| & definition file \\
    |cdocsamp.tex| & sample main file \\
    |cdocsch1.tex| & sample include file \\
    |cdocsch2.tex| & sample include file \\
    |cdocspt3.tex| & sample part file \\
    |cdocspt4.tex| & sample part file \\
    |cdocsdrf.tex| & sample redirection file \\
    |cdocsfn1.tex| & sample redirection file \\
    |cdocsfn2.tex| & sample redirection file \\
    |childdoc.pdf| & manual
\end{tabular}
\end{center}
%
The distribution consists of the files
|README.txt|, |childdoc.ins| and |childdoc.dtx|.
%
\begin{itemize}
\item
Run (pdf)\LaTeX{} on |childdoc.dtx|
to compile the manual |childdoc.pdf| (this file).
\item
Run \LaTeX{} on |childdoc.ins| to create the definitions file |childdoc.def|
and the sample |cdocsamp.tex| with include files
|cdocsch1.tex|, |cdocsch2.tex|, |cdocspt3.tex|, |cdocspt4.tex|,
|cdocsdrf.tex|, |cdocsfn1.tex|, |cdocsfn2.tex|.
Then copy the file |childdoc.def| to an appropriate directory of your \LaTeX{}
distribution, e.g.\ \textit{texmf-root}|/tex/latex/childdoc|.
\end{itemize}

%%%%%%%%%%%%%%%%%%%%%%%%%%%%%%%%%%%%%%%%%%%%%%%%%%%%%%%%%%%%%%%%%%%%%%%%%%%%%%%%
\subsection{Related CTAN Packages}

There are several other packages which offer a similar functionality:
%
\begin{itemize}
\item
The packages
\href{http://ctan.org/pkg/docmute}{\textsf{docmute}},
\href{http://ctan.org/pkg/includex}{\textsf{includex}} and
\href{http://ctan.org/pkg/standalone}{\textsf{standalone}}
provide commands to include only the document body of
a child file thus allowing both files to be compiled individually.
\item
The packages \href{http://ctan.org/pkg/subdocs}{\textsf{subdocs}}
and \href{http://ctan.org/pkg/subfiles}{\textsf{subfiles}}
provide structures in which the main and child documents can be
encapsulated and allowing them to be compiled individually.
The inclusion mechanism is different from the conventional |\include|.
\item
The package \href{http://ctan.org/pkg/combine}{\textsf{combine}}
is an elaborate solution to combine several documents into one.
\end{itemize}
%
See also the CTAN topic \href{http://ctan.org/topic/subdocs}{\textsf{subdocs}}
for further related packages.
The present package differs from the above solutions in that
a document structure constructed with the conventional |\include| mechanism
just needs two extra commands at the top of every file
such that all constituent files can be compiled individually.

%%%%%%%%%%%%%%%%%%%%%%%%%%%%%%%%%%%%%%%%%%%%%%%%%%%%%%%%%%%%%%%%%%%%%%%%%%%%%%%%
%\subsection{Feature Suggestions}
%
%The following is a list of features which may be useful for future
%versions of this package:
%%
%\begin{itemize}
%\item
%\ldots
%\end{itemize}

%%%%%%%%%%%%%%%%%%%%%%%%%%%%%%%%%%%%%%%%%%%%%%%%%%%%%%%%%%%%%%%%%%%%%%%%%%%%%%%%
\subsection{Revision History}

%%%%%%%%%%%%%%%%%%%%%%%%%%%%%%%%%%%%%%%%
\paragraph{v2.0:} 2018/12/30

\begin{itemize}
\item
immediate forward processing
\item
added |\childdocby| mechanism
\item
manual restructured
\end{itemize}

%%%%%%%%%%%%%%%%%%%%%%%%%%%%%%%%%%%%%%%%
\paragraph{v1.6:} 2018/01/17

\begin{itemize}
\item
application for development of include files
\item
corrections to manual
\end{itemize}

%%%%%%%%%%%%%%%%%%%%%%%%%%%%%%%%%%%%%%%%
\paragraph{v1.5:} 2017/05/21

\begin{itemize}
\item
more complete structuring introduced
\item
|\childdocof| introduced
\item
|\childdoc| renamed to |\childdocmain|
\item
|\childredirect| renamed to |\childdocforward| and |\childdocforwardprefix|
and functionality expanded
\end{itemize}

%%%%%%%%%%%%%%%%%%%%%%%%%%%%%%%%%%%%%%%%
\paragraph{v1.0:} 2017/04/27

\begin{itemize}
\item
manual and install package
\item
first version published on CTAN
\end{itemize}

%%%%%%%%%%%%%%%%%%%%%%%%%%%%%%%%%%%%%%%%
\paragraph{v0.6:} 2017/04/26

\begin{itemize}
\item
redirection mechanism added
\end{itemize}

%%%%%%%%%%%%%%%%%%%%%%%%%%%%%%%%%%%%%%%%
\paragraph{v0.5:} 2017/04/26

\begin{itemize}
\item
functionality in definition file
\end{itemize}


%%%%%%%%%%%%%%%%%%%%%%%%%%%%%%%%%%%%%%%%%%%%%%%%%%%%%%%%%%%%%%%%%%%%%%%%%%%%%%%%
%%%%%%%%%%%%%%%%%%%%%%%%%%%%%%%%%%%%%%%%%%%%%%%%%%%%%%%%%%%%%%%%%%%%%%%%%%%%%%%%
%%%%%%%%%%%%%%%%%%%%%%%%%%%%%%%%%%%%%%%%%%%%%%%%%%%%%%%%%%%%%%%%%%%%%%%%%%%%%%%%
\appendix

\settowidth\MacroIndent{\rmfamily\scriptsize 000\ }

 \DocInput{childdoc.dtx}

\end{document}
%</driver>
% \fi
%
% %%%%%%%%%%%%%%%%%%%%%%%%%%%%%%%%%%%%%%%%%%%%%%%%%%%%%%%%%%%%%%%%%%%%%%%%%%%%%%
% %%%%%%%%%%%%%%%%%%%%%%%%%%%%%%%%%%%%%%%%%%%%%%%%%%%%%%%%%%%%%%%%%%%%%%%%%%%%%%
% \section{Sample}
%\iffalse
%<*samplemain>
%\fi
%
% The following presents a sample document
% with two chapters, two parts, a title page,
% a compile flag as well as three forwarding files to set the flag.
% It consists of eight |.tex| files:
% \begin{center}
% \begin{tabular}{ll}
% |cdocsamp.tex|&main file\\
% |cdocsch1.tex|&include file for chapter 1\\
% |cdocsch2.tex|&include file for chapter 2\\
% |cdocspt3.tex|&include file for part 3\\
% |cdocspt4.tex|&include file for part 4\\
% |cdocsdrf.tex|&forwarding file for main file in draft mode\\
% |cdocsfi1.tex|&forwarding file for final version of chapter 1\\
% |cdocsfi2.tex|&forwarding file for final version of chapter 2\\
% \end{tabular}
% \end{center}
% Each of the eight files can be compiled directly by the \LaTeX{} compiler.
%
% %%%%%%%%%%%%%%%%%%%%%%%%%%%%%%%%%%%%%%
% \paragraph{Main File.}
%
% The main file is called |cdocsamp.tex|.
%
% Load the \textsf{childdoc} definitions and
% declare the filename for the main document:
%    \begin{macrocode}
\input{childdoc.def}
\childdocmain{}
%    \end{macrocode}

% Optional override for |\version| flag:
%    \begin{macrocode}
%%\ifchilddoc\else\providecommand{\version}{draft}\fi
%    \end{macrocode}

% Define the default values for the |\version| flag
% (|final| for the main file and |draft| for childs):
%    \begin{macrocode}
\ifchilddoc
\providecommand{\version}{draft}
\else
\providecommand{\version}{final}
\fi
%    \end{macrocode}

% Load the standard document class:
%    \begin{macrocode}
\documentclass[12pt]{article}
%    \end{macrocode}

% Start the document body:
%    \begin{macrocode}
\begin{document}
%    \end{macrocode}

% Declare a title page.
% Print title, part of document being processed and version flag:
%    \begin{macrocode}
\addtocounter{page}{-1}
\begin{center}
{\LARGE\bfseries{}childdoc example\par}
\vspace{1cm}
\ifchilddoc
\ifchilddocmanual part\else chapter\fi:
`\childdocname' of `\childdocjob'\par
\else
main document: `\childdocjob'\par
\fi
version: \version\par
\end{center}
\newpage
%    \end{macrocode}

% Manually include selected file,
% otherwise process as usual:
%    \begin{macrocode}
\ifchilddocmanual
\section*{part `\childdocname'}
\input{\childdocname}
\else
%    \end{macrocode}

% Include the two chapters:
%    \begin{macrocode}
\include{cdocsch1}
\include{cdocsch2}
%    \end{macrocode}

% Include the two parts unless only chapters should be displayed:
%    \begin{macrocode}
\ifchilddoc\else
\section{part three}
\input{cdocspt3}
\section{part four}
\input{cdocspt4}
\fi
%    \end{macrocode}

% Process as usual until here:
%    \begin{macrocode}
\fi
%    \end{macrocode}

% End of document body:
%    \begin{macrocode}
\end{document}
%    \end{macrocode}
%\iffalse
%</samplemain>
%\fi
%
% %%%%%%%%%%%%%%%%%%%%%%%%%%%%%%%%%%%%%%
% \paragraph{Chapter Include Files.}
%
% The include files are called |cdocsch1.tex| and |cdocsch2.tex|.
%
%\iffalse
%<*samplechap1|samplechap2>
%\fi

% Optional override for |\version| flag:
%    \begin{macrocode}
%%\providecommand{\version}{final}
%    \end{macrocode}

% Include the main document:
%    \begin{macrocode}
\input{childdoc.def}
\childdocof{cdocsamp}
%    \end{macrocode}

%\iffalse
%</samplechap1|samplechap2>
%\fi
%
%\iffalse
%<*samplechap1>
%\fi
% Some text for chapter 1:
%    \begin{macrocode}
\section{one}
some text in chapter one
%    \end{macrocode}

%\iffalse
%</samplechap1>
%\fi
% Some text for chapter 2:
%\iffalse
%<*samplechap2>
%\fi
%    \begin{macrocode}
\section{two}
more text in chapter two
%    \end{macrocode}

%\iffalse
%</samplechap2>
%\fi
%
% %%%%%%%%%%%%%%%%%%%%%%%%%%%%%%%%%%%%%%
% \paragraph{Part Include Files.}
%
% The include files are called |cdocspt3.tex| and |cdocspt4.tex|.
%
%\iffalse
%<*samplepart3|samplepart4>
%\fi

% Optional override for |\version| flag:
%    \begin{macrocode}
%%\providecommand{\version}{final}
%    \end{macrocode}

% Include the main document:
%    \begin{macrocode}
\input{childdoc.def}
\childdocby{cdocsamp}
%    \end{macrocode}

%\iffalse
%</samplepart3|samplepart4>
%\fi
%
%\iffalse
%<*samplepart3>
%\fi
% Some text for part 3:
%    \begin{macrocode}
some text in part three
%    \end{macrocode}

%\iffalse
%</samplepart3>
%\fi
% Some text for part 4:
%\iffalse
%<*samplepart4>
%\fi
%    \begin{macrocode}
more text in part four
%    \end{macrocode}

%\iffalse
%</samplepart4>
%\fi
%
% %%%%%%%%%%%%%%%%%%%%%%%%%%%%%%%%%%%%%%
% \paragraph{Forwarding for a Complete Draft.}
%
% The following forwarding file |cdocsdrf.tex|
% compiles the main document in draft mode:
%\iffalse
%<*sampledraft>
%\fi
%    \begin{macrocode}
\def\version{draft}
\input{childdoc.def}
\childdocforward{cdocsamp}
%    \end{macrocode}

%\iffalse
%</sampledraft>
%\fi
%
% %%%%%%%%%%%%%%%%%%%%%%%%%%%%%%%%%%%%%%
% \paragraph{Forwarding for Final Version of the Chapters.}
%
% The following forwarding files |cdocsfn1.tex| and |cdocsfn2.tex|
% (with identical content)
% compile the final versions of the child documents
% |cdocsch1.tex| and |cdocsch2.tex|, respectively:
%\iffalse
%<*samplefinal>
%\fi
%    \begin{macrocode}
\def\version{final}
\input{childdoc.def}
\childdocforwardprefix[cdocsamp]{cdocsfn}{cdocsch}
%    \end{macrocode}

%\iffalse
%</samplefinal>
%\fi
%
% %%%%%%%%%%%%%%%%%%%%%%%%%%%%%%%%%%%%%%
% \paragraph{Command Line Processing.}
%
% The following three command lines generate the output files
% |cdocscld|, |cdocscl1| and |cdocscl2|
% which should be identical to
% |cdocsdrf|, |cdocsch1| and |cdocsfn2|, respectively:
% \begin{center}
% \begin{tabular}{l}
% |latex -jobname cdocscld \|\\
% |  "\def\version{draft}\input{childdoc.def}\childdocforward{cdocsamp}"|\\
% |latex -jobname cdocscl1 \|\\
% |  "\input{childdoc.def}\childdocforward[cdocsamp]{cdocsch1}"|\\
% |latex -jobname cdocscl2 \|\\
% |  "\def\version{final}\input{childdoc.def}\childdocforward{cdocsch2}"|
% \end{tabular}
% \end{center}
% Note that the trailing backslash on each first line
% merely continues the input to the second line
% (for convenient cut ant paste).
% Furthermore, the command |latex| can be replaced by any
% of its alternative versions such as |pdflatex|.
%
% %%%%%%%%%%%%%%%%%%%%%%%%%%%%%%%%%%%%%%%%%%%%%%%%%%%%%%%%%%%%%%%%%%%%%%%%%%%%%%
% %%%%%%%%%%%%%%%%%%%%%%%%%%%%%%%%%%%%%%%%%%%%%%%%%%%%%%%%%%%%%%%%%%%%%%%%%%%%%%
% \section{Implementation}
%\iffalse
%<*package>
%\fi
%
% This section describes the definitions file |childdoc.def|.

% The definitions cannot be loaded using |\usepackage| or |\RequirePackage|
% which has a mechanism to prevent loading a style file more than once.
% When loading the definitions by means of |\input|
% multiple instances have to be prevented manually:
%\iffalse
%This code needs to be before the `\ProvidesFile' directive
%which is defined at the beginning of this file.
%Therefore it is also placed there and commented out here.
%</package>
%<*discard>
%\fi
%    \begin{macrocode}
\ifdefined\childdocmain\endinput\fi
%    \end{macrocode}
%\iffalse
%</discard>
%<*package>
%\fi
%
% \macro{\ifchilddoc}
% \macro{\ifchilddocmanual}
% The conditional |\ifchilddoc| tells whether a
% child (true) or main (false) document is being compiled.
% The conditional |\ifchilddocmanual| tells whether
% the |\includeonly| mechanism is used (false) or
% the selection of child files must be performed manually (true).
% The definitions initialise to false:
%    \begin{macrocode}
\newif\ifchilddoc
\newif\ifchilddocmanual
%    \end{macrocode}

% \macro{\childdocname}
% \macro{\childdocjob}
% The macro |\childdocname| stores the name of the main document
% to be compiled. The macro |\childdocjob| stores the name of
% the document on which the \LaTeX{} compiler was originally invoked.
% The content of |\jobname| cannot be compared
% to filenames specified in the source due to different catcodes.
% The following code rescans |\jobname|, stores the result
% in |\childdocname| and saves a copy in |\childdocjob|:
%    \begin{macrocode}
\edef\childdocname{\scantokens\expandafter{\jobname\noexpand}}
\let\childdocjob\childdocname
%    \end{macrocode}

% \macro{\childdocdisable}
% The macro |\childdocdisable| prevents the main file
% from being processed more than once.
% At this stage, the main document command |\childdocmain|
% is assumed to be called once again where it should do nothing.
% Any subsequent call to it should prevent
% a secondary processing of the main document
% It overwrites the forwarding commands
% |\childdocof| and |\childdocforward|
% with empty macros to prevent further inclusions of the main document:
%    \begin{macrocode}
\newcommand{\childdocdisable}
{
  \renewcommand{\childdocmain}[1]{\renewcommand{\childdocmain}[1]{\endinput}}
  \renewcommand{\childdocof}[1]{}
  \renewcommand{\childdocby}[2][]{}
  \renewcommand{\childdocforward}[2][]{}
  \renewcommand{\childdocdisable}{}
}
%    \end{macrocode}

% \macro{\childdocmain}
% The macro |\childdocmain| is to be called at the top of the main file
% with nothing or the main filename (without extension) as argument.
% First, it breaks loops.
% If the argument is not empty and does not match |\childdocname|
% (which is set by the first inclusion of |childdoc.def|),
% |\ifchilddoc| is set to true, |\includeonly| is applied to the child file
% and |\jobname| is set to the main file
% (for proper handling of |.aux| files):
%    \begin{macrocode}
\newcommand{\childdocmain}[1]
{
  \childdocdisable\childdocmain{}
  \if?#1?\else
    \begingroup
      \def\childdoctmp{#1}
      \ifx\childdoctmp\childdocname
        \def\childdoctmp{}
      \else
        \def\childdoctmp
        {
          \childdoctrue
          \includeonly{\childdocname}
          \def\childdocjob{#1}
          \def\jobname{#1}
        }
      \fi
      \expandafter
    \endgroup
    \childdoctmp
  \fi
}
%    \end{macrocode}

% \macro{\childdocof}
% The command |\childdocof| redirects
% compilation to the main file |#1|.
%    \begin{macrocode}
\newcommand{\childdocof}[1]
{
  \childdocdisable
  \childdoctrue
  \includeonly{\childdocname}
  \def\jobname{#1}
  \def\childdocjob{#1}
  \input{#1}
}
%    \end{macrocode}

% \macro{\childdocby}
% The command |\childdocby| ....
%    \begin{macrocode}
\newcommand{\childdocby}[2][]
{
  \childdocdisable
  \childdoctrue
  \childdocmanualtrue
  \if?#1?\else
    \def\jobname{#2}
  \fi
  \def\childdocjob{#2}
  \input{#2}
  \endinput
}
%    \end{macrocode}

% \macro{\childdocforward}
% The command |\childdocforward| redirects
% compilation to the main file or
% (if the optional argument is given) a child file.
% Parameters are set as if the main file
% or a child file starting with |\childdocof| was compiled.
% Then compilation is handed over to the main file:
%    \begin{macrocode}
\newcommand{\childdocforward}[2][]
{
  \begingroup
    \if?#1?
      \def\childdoctmp
      {
        \def\childdocname{#2}
        \def\childdocjob{#2}
        \def\jobname{#2}
        \input{#2}
        \endinput
      }
    \else
      \def\childdoctmp
      {
        \childdocdisable
        \def\childdocname{#2}
        \childdoctrue
        \includeonly{#2}
        \def\childdocjob{#1}
        \def\jobname{#1}
        \input{#1}
        \endinput
      }
    \fi
    \expandafter
  \endgroup
  \childdoctmp
}
%    \end{macrocode}

% \macro{\childdocforwardprefix}
% The command |\childdocforwardprefix| redirects
% compilation to the main or a child file by means of a pattern.
% The prefix |#1| in the current filename is replaced by |#2|
% and the suffix of the current filename is kept
% (it is assumed that the filename does not contain the substring `|~~~|'
% which is used as a delimiter).
% Compilation is handed over to the new file by |\childdocforward|:
%    \begin{macrocode}
\newcommand{\childdocforwardprefix}[3][]
{
  \begingroup
    \def\childdocextract #2##1~~~{\def\childdoctmp{\childdocforward[#1]{#3##1}}}
    \expandafter\childdocextract\childdocname~~~
    \expandafter
  \endgroup
  \childdoctmp
}
%    \end{macrocode}

% \macro{\childdoc}
% The deprecated macro |\childdoc| is a legacy version of |\childdocmain|:
%    \begin{macrocode}
\newcommand{\childdoc}{\childdocmain}
%    \end{macrocode}

% \macro{\childdocredirect}
% The deprecated macro |\childdocredirect| is a legacy version
% of |\childdocforward| and |\childdocforwardprefix|:
%    \begin{macrocode}
\newcommand{\childdocredirect}[2][]
{
  \begingroup
    \if?#1?
      \def\childdoctmp{\childdocforward{#2}}
    \else
      \def\childdoctmp{\childdocforwardprefix{#1}{#2}}
    \fi
    \expandafter
  \endgroup
  \childdoctmp
}
%    \end{macrocode}

%\iffalse
%</package>
%\fi
%
\endinput
|\\
|\childdocforward{|\textit{main}|}|
\end{tabular}
\end{center}
%
Likewise, the following files |final|\textit{nn}|.tex|
compile the final version of the child document
|child|\textit{nn}|.tex|:
%
\begin{center}
\begin{tabular}{l}
|\def\version{final}|\\
|% \iffalse
%
% childdoc.dtx Copyright (C) 2017-2018 Niklas Beisert
%
% This work may be distributed and/or modified under the
% conditions of the LaTeX Project Public License, either version 1.3
% of this license or (at your option) any later version.
% The latest version of this license is in
%   http://www.latex-project.org/lppl.txt
% and version 1.3 or later is part of all distributions of LaTeX
% version 2005/12/01 or later.
%
% This work has the LPPL maintenance status `maintained'.
%
% The Current Maintainer of this work is Niklas Beisert.
%
% This work consists of the files childdoc.dtx and childdoc.ins
% and the derived files childdoc.def and cdocsamp.tex with
% cdocsch1.tex, cdocsch2.tex, cdocsdrf.tex, cdocsfn1.tex, cdocsfn2.tex.
%
%<package>\ifdefined\childdocmain\endinput\fi
%<package>\ProvidesFile{childdoc.def}[2018/12/30 v2.0 child document driver]
%<samplemain>\ProvidesFile{cdocsamp.tex}[2018/12/30 v2.0 sample for childdoc]
%<*driver>
%\ProvidesFile{childdoc.drv}[2018/12/30 v2.0 childdoc reference manual file]
\PassOptionsToClass{10pt,a4paper}{article}
\documentclass{ltxdoc}

\usepackage[margin=35mm]{geometry}
\usepackage{hyperref}
\usepackage{hyperxmp}
\usepackage[usenames]{color}

\hypersetup{colorlinks=true}
\hypersetup{pdfstartview=FitH}
\hypersetup{pdfpagemode=UseNone}
\hypersetup{pdfsource={}}
\hypersetup{pdflang={en-UK}}
\hypersetup{pdfcopyright={Copyright 2017-2018 Niklas Beisert.
  This work may be distributed and/or modified under the
  conditions of the LaTeX Project Public License, either version 1.3
  of this license or (at your option) any later version.}}
\hypersetup{pdflicenseurl={http://www.latex-project.org/lppl.txt}}
\hypersetup{pdfcontactaddress={ETH Zurich, ITP, HIT K,
  Wolfgang-Pauli-Strasse 27}}
\hypersetup{pdfcontactpostcode={8093}}
\hypersetup{pdfcontactcity={Zurich}}
\hypersetup{pdfcontactcountry={Switzerland}}
\hypersetup{pdfcontactemail={nbeisert@itp.phys.ethz.ch}}
\hypersetup{pdfcontacturl={http://people.phys.ethz.ch/\xmptilde nbeisert/}}

\newcommand{\secref}[1]{\hyperref[#1]{section \ref*{#1}}}

\parskip1ex
\parindent0pt
\let\olditemize\itemize
\def\itemize{\olditemize\parskip0pt}

\begin{document}

\title{The \textsf{childdoc} Package}
\hypersetup{pdftitle={The childdoc Package}}
\author{Niklas Beisert\\[2ex]
  Institut f\"ur Theoretische Physik\\
  Eidgen\"ossische Technische Hochschule Z\"urich\\
  Wolfgang-Pauli-Strasse 27, 8093 Z\"urich, Switzerland\\[1ex]
  \href{mailto:nbeisert@itp.phys.ethz.ch}
  {\texttt{nbeisert@itp.phys.ethz.ch}}}
\hypersetup{pdfauthor={Niklas Beisert}}
\hypersetup{pdfsubject={Manual for the LaTeX2e Package childdoc}}
\date{30 December 2018, \textsf{v2.0}}
\maketitle

\begin{abstract}\noindent
\textsf{childdoc} is a \LaTeXe{} package
that enables the direct compilation
of document sections included by |\include|
to individual files.
\end{abstract}

\begingroup
\parskip0ex
\tableofcontents
\endgroup

%%%%%%%%%%%%%%%%%%%%%%%%%%%%%%%%%%%%%%%%%%%%%%%%%%%%%%%%%%%%%%%%%%%%%%%%%%%%%%%%
%%%%%%%%%%%%%%%%%%%%%%%%%%%%%%%%%%%%%%%%%%%%%%%%%%%%%%%%%%%%%%%%%%%%%%%%%%%%%%%%
\section{Introduction}

\LaTeX{} provides a mechanism to structure a large document (such as a book)
into a main file and several child files (containing the chapters)
using the |\include| command.
This mechanism is beneficial for documents
which span hundreds of pages in order to
make the source file(s) more manageable.
Moreover, compilation can be restricted to
selected child files by means of the |\includeonly| command.
The latter feature can be used to reduce the compilation time while editing
(this was significantly more useful in the earlier days of \LaTeX{})
or to generate a smaller document which is easier to navigate.
Another application of |\includeonly| is to generate
documents consisting of selected parts of the complete document.

However, there are a few drawbacks of the plain |\include| mechanism:
\begin{itemize}
\item
The child files cannot be compiled on their own,
they can only be compiled via the main file.
A naive editing environment
(such as a text editor with an option
to have the current file processed by \LaTeX)
may require one to switch to the main file before compiling;
attempting to compile the child file produces errors.
\item
The main file must be modified (each time)
to adjust the |\includeonly| command
to the present needs. This easily leaves the main file in a messy state.
\item
The generated document will always carry the filename
of the main document. This is inconvenient if
several child files are to be compiled and
to be kept for distribution.
\end{itemize}

The present package provides a simple interface
to make child files individually compilable by \LaTeX{}.
Compiling a child file then has the same effect as compiling
the main file with an |\includeonly| command
to select the appropriate child.
Moreover the generated document will carry the name of the child
rather than the main file.
This resolves all three above issues.

This feature is meant to make the editing of books,
thesis documents and lecture notes somewhat more convenient.
However, the package can also be used efficiently for
composing a series of documents (such as exercise sheets)
which are typically distributed individually.
It then assists the author in generating the individual documents
(potentially in different versions)
as well as a document containing the collected series.
Another application is in developing style files
or other kinds of included material
where compilation of the style file could redirect
to a sample or test file.

%%%%%%%%%%%%%%%%%%%%%%%%%%%%%%%%%%%%%%%%%%%%%%%%%%%%%%%%%%%%%%%%%%%%%%%%%%%%%%%%
%%%%%%%%%%%%%%%%%%%%%%%%%%%%%%%%%%%%%%%%%%%%%%%%%%%%%%%%%%%%%%%%%%%%%%%%%%%%%%%%
\section{Usage}

First of all, the package \textsf{childdoc} is \emph{not} a standard
\LaTeXe{} |.sty| style file! Therefore it needs to be invoked in
a non-standard way.

%%%%%%%%%%%%%%%%%%%%%%%%%%%%%%%%%%%%%%%%%%%%%%%%%%%%%%%%%%%%%%%%%%%%%%%%%%%%%%%%
\subsection{Included Files}
\label{sec:include}

%%%%%%%%%%%%%%%%%%%%%%%%%%%%%%%%%%%%%%%%
\DescribeMacro{\childdocmain}
To use the package, add the commands
\begin{center}
\begin{tabular}{l}
|\input{childdoc.def}|\\
|\childdocmain{}|\\
\end{tabular}
\end{center}
at the very top of the main \LaTeX{} file,
in particular \emph{before} the |\documentclass| statement!
The argument of |\childdocmain| should be left empty
(but it must be present).

%%%%%%%%%%%%%%%%%%%%%%%%%%%%%%%%%%%%%%%%
\DescribeMacro{\childdocof}
Furthermore, add the commands
\begin{center}
\begin{tabular}{l}
|\input{childdoc.def}|\\
|\childdocof{|\textit{main}|}|\\
\end{tabular}
\end{center}
at the top of every child file \textit{child}
which is included by |\include{|\textit{child}|}|
from within the main file
(or at least for those files to be compiled individually).
The argument \textit{main} must be the filename of the main file.

There are a couple of
considerations in setting up the main and child documents:

%%%%%%%%%%%%%%%%%%%%%%%%%%%%%%%%%%%%%%%%
\paragraph{Restrictions.}

Please note the following restrictions:
\begin{itemize}
\item
|\childdocmain| must be called with one argument \textit{main}
to ensure compatibility with earlier version of the package.
It must either be empty (|\childdocmain{}|)
or precisely match the filename of the main file in which it is specified.
See \secref{sec:detection} for further information.
\item
The filename \textit{main} must be specified without the |.tex| extension.
\item
The filename \textit{main} is case sensitive
(even in case-insensitive file systems)
due to internal string comparison.
\item
The argument \textit{main} should be fully expanded, it cannot be a macro.
\item
Subdirectories and special characters should be avoided in filenames.
\item
The command |\childdocmain{|\textit{main}|}| must be followed by a whitespace.
It should not be followed immediately by another command
or by a comment mark `|%|'.
This is because the \TeX{} parser reads the token immediately following
the argument of |\childdocmain| and puts it
at the beginning of every child section;
however, a white\-space is ignored.
\end{itemize}

%%%%%%%%%%%%%%%%%%%%%%%%%%%%%%%%%%%%%%%%
\paragraph{Content of Main File.}

It is advisable to place all content in the child files included by |\include|.
Any output contained in the main file will appear in all child documents
unless suppressed manually;
it cannot be suppressed automatically by the |\includeonly| directive
and thus should normally be avoided.
A method to include some content in the main file
by means of conditional processing is described in \secref{sec:conditional}.

%%%%%%%%%%%%%%%%%%%%%%%%%%%%%%%%%%%%%%%%
\paragraph{Page Numbering.}

When only a part of the document is compiled,
the appropriate numbering of pages
(as well as other status parameters)
is determined from the |.aux| files.
The latter contain information from previous passes.
However this information needs to propagate through
all intermediate child documents.
Therefore the page numbering in child documents may well
be inconsistent until the complete document is compiled at least once.

A useful (if unconventional) way to always ensure a consistent
page numbering is to restart the numbering in each child document
and denote the pages by `\textit{child}|.|\textit{page}'
where \textit{child} represents the chapter/section number of the child file.
This can be achieved by the command
|\numberwithin{page}{|\textit{child}|}|
of the \textsf{amsmath} package
where \textit{child} can be |chapter| or |section|
depending on the chosen structuring.
Alternatively, one can modify the macro |\thepage| appropriately
and reset the counter |page| at the start of each child file.

%%%%%%%%%%%%%%%%%%%%%%%%%%%%%%%%%%%%%%%%%%%%%%%%%%%%%%%%%%%%%%%%%%%%%%%%%%%%%%%%
\subsection{Conditional Processing}
\label{sec:conditional}

The package provides a mechanism to compile different versions
of a document. To customise the versions further some conditional processing
can come in handy to distinguish which version is being compiled.
The package provides two macros to describe the compilation context:

%%%%%%%%%%%%%%%%%%%%%%%%%%%%%%%%%%%%%%%%
\DescribeMacro{\ifchilddoc}
The conditional |\ifchilddoc| distinguishes between the compilation of
child documents and the main document:
%
\begin{center}
|\ifchilddoc |\textit{child-code}| |[|\||else |\textit{main-code}]| \||fi|
\end{center}

%%%%%%%%%%%%%%%%%%%%%%%%%%%%%%%%%%%%%%%%
\DescribeMacro{\childdocname}
\DescribeMacro{\childdocjob}
The macro |\childdocname| contains the filename (without extension)
of the main or child file being processed.
Note that |\childdocjob| will always contain the name of the main file.

%%%%%%%%%%%%%%%%%%%%%%%%%%%%%%%%%%%%%%%%
\paragraph{Title Page.}

Conditional processing can be used to include a title or banner page
in the main document when proper precautions are taken.
Importantly, the code in the main file should ensure that the page counter
(as well as other status parameters which are stored in the |.aux| files)
takes the same value after the conditional processing.
Otherwise the page numbers may take divergent values
depending on which part is compiled.

For example, a title page could be declared by:
%
\begin{center}
\begin{tabular}{l}
|\ifchilddoc\||else|\\
|\addtocounter{page}{-1}|\\
\textit{code for title page}\\
|\newpage|\\
|\||fi|
\end{tabular}
\end{center}
%
A banner page for the child documents can be generated by:
%
\begin{center}
\begin{tabular}{l}
|\ifchilddoc|\\
|\addtocounter{page}{-1}|\\
\textit{code for banner page}\\
|\newpage|\\
|\||fi|
\end{tabular}
\end{center}
%
Here one could write a message such as:
\begin{center}
|This is the part \childdocname{} of \childdocjob{}.|
\end{center}

%%%%%%%%%%%%%%%%%%%%%%%%%%%%%%%%%%%%%%%%%%%%%%%%%%%%%%%%%%%%%%%%%%%%%%%%%%%%%%%%
\subsection{Flags}
\label{sec:flags}

The package makes it easy to generate different versions
of the main or child documents.
To this end compilation flags can be defined
and assigned different default values.
They will be particularly useful in conjunction
with the forwarding mechanism described in \secref{sec:forward}.

For example, it may be useful to have a flag |\version|
which can be set to |draft| or |final|.
The document source will contain some conditional code
depending on the value of |\version|.
Suppose further, the flag should default to |final| for the main file
and to |draft| for child files
which is a natural assignment for editing the document.
This is achieved by placing the following code
in the preamble of the main document
(below the |\childdocmain| directive):
%
\begin{center}
\begin{tabular}{l}
|\ifchilddoc|\\
|\providecommand{\version}{draft}|\\
|\||else|\\
|\providecommand{\version}{final}|\\
|\||fi|
\end{tabular}
\end{center}
%
The definition by |\providecommand| makes sure
that previous definitions are not overwritten.
Further statements |\providecommand{\version}{...}|
can thus be added before the above code to override it.

For the main file, one might add a line
(between |\childdocmain| and the above block)
%
\begin{center}
|%\ifchilddoc\||else\providecommand{\version}{draft}\||fi|
\end{center}
%
which can be uncommented to produce a draft version.
Likewise one can add a line to the very top of a child file
(above the |\childdocof{|\textit{main}|}| directive)
%
\begin{center}
|%\providecommand{\version}{final}|
\end{center}
%
which can be uncommented to produce the final version of this child document.

%%%%%%%%%%%%%%%%%%%%%%%%%%%%%%%%%%%%%%%%%%%%%%%%%%%%%%%%%%%%%%%%%%%%%%%%%%%%%%%%
\subsection{Forwarding}
\label{sec:forward}

Different versions of the main or child documents
using compilation flags as described in \secref{sec:flags}
can be (permanently) stored in different files
for convenient compilation, viewing and distribution.
To this end, the package defines a command
to pass on compilation to a different file:

%%%%%%%%%%%%%%%%%%%%%%%%%%%%%%%%%%%%%%%%
\DescribeMacro{\childdocforward}
The command |\childdocforward| redirects processing to
another source file:
%
\begin{center}
\begin{tabular}{l}
|\input{childdoc.def}|\\
|\childdocforward[|\textit{main}|]{|\textit{dest}|}|\\
\end{tabular}
\end{center}
%
The argument \textit{dest} is the destination file
(without extension).
It should be the main file or one of the child files.
Note that further \textsf{childdoc} directives
such as |\childdocof| and |\childdocforward|
in the indicated file will be processed in this form.
The optional argument \textit{main}
passes on directly to the main file \textit{main}
while pretending to compile the child \textit{dest}.
This form behaves as if \textit{dest}
issues |\childdocof{|\textit{main}|}| right away,
and no further \textsf{childdoc} directives will be processed.

%%%%%%%%%%%%%%%%%%%%%%%%%%%%%%%%%%%%%%%%
\DescribeMacro{\...prefix}
In the alternative form |\childdocforwardprefix|,
%
\begin{center}
\begin{tabular}{l}
|\input{childdoc.def}|\\
|\childdocforwardprefix[|\textit{main}|]{|\textit{prefix}|}{|\textit{dest}|}|
\end{tabular}
\end{center}
%
the destination file is determined by a pattern
depending on the current file:
To make this work, the current file must be called
`{\textit{prefix}\hspace{0.2em}\textit{suffix}}'
with \textit{prefix} matching precisely the argument.
Processing is then passed on to the file
`{\textit{dest}\hspace{0.2em}\textit{suffix}}'.
Surely, the same effect is achieved by
directly specifying the
argument `{\textit{dest}\hspace{0.2em}\textit{suffix}}'
in the first form.
However, that requires to set up a different file
for each child. With the alternative form of the command
all these files can have exactly the same content
which simplifies setting them up and maintaining them.

For example, the following file |draft.tex|
with a compilation flag |\version| as described in \secref{sec:flags}
compiles the main document as a draft:
%
\begin{center}
\begin{tabular}{l}
|\def\version{draft}|\\
|\input{childdoc.def}|\\
|\childdocforward{|\textit{main}|}|
\end{tabular}
\end{center}
%
Likewise, the following files |final|\textit{nn}|.tex|
compile the final version of the child document
|child|\textit{nn}|.tex|:
%
\begin{center}
\begin{tabular}{l}
|\def\version{final}|\\
|\input{childdoc.def}|\\
|\childdocforwardprefix{final}{child}|
\end{tabular}
\end{center}
%

Note that when several versions of a main file and/or of each child file
are to be generated, it may be convenient to set up a |Makefile| or
shell script to automatise the process.

%%%%%%%%%%%%%%%%%%%%%%%%%%%%%%%%%%%%%%%%%%%%%%%%%%%%%%%%%%%%%%%%%%%%%%%%%%%%%%%%
\subsection{Command Line Processing}
\label{sec:commandline}

The effect of redirection files can also be achieved by invoking
the \LaTeX{} compiler with a more elaborate command line.
Most conveniently this should be done as part
of a shell script or a |Makefile|.

When using \textsf{childdoc} in the main file, the following
command lines effectively perform a redirection
(note that depending on the shell being used,
backslashes may have to be doubled: `|\|' $\to$ `|\\|'):
%
\begin{center}
|... -jobname "|\textit{target}|" |\\|"|[\textit{flags}]%
|\input{childdoc.def}\childdocforward[|\textit{main}|]{|\textit{dest}|}"|
\end{center}
%
Here \textit{target} is the name of the output file,
\textit{main} is the name of the main file
and \textit{dest} is the name of the main or child file to be processed
(all filenames without extensions).
The optional argument \textit{main} can be omitted
if \textit{main} matches \textit{dest}.
Optionally, compilation \textit{flags} can be defined via |\def| commands.
This command line makes the \TeX{} engine believe
it is compiling the file \textit{target}
whose content is specified as the latter parameter.
The provided code then forwards the processing to
\textit{main} or \textit{dest} as described in \secref{sec:forward}.

%%%%%%%%%%%%%%%%%%%%%%%%%%%%%%%%%%%%%%%%%%%%%%%%%%%%%%%%%%%%%%%%%%%%%%%%%%%%%%%%
\subsection{Include by Input}
\label{sec:input}

Including child documents by |\include| has some restrictions by design.
Most notably, the content of a child document always occupies
its own set of pages; pages cannot be shared between child documents.
Usually, this behaviour makes perfect sense
because each child document contain an essential part of the document.
However, in some situations it may be desirable to compose
a document from a collection of parts
without having mandatory page breaks between then.
For this case, the package
provides a mechanism to include parts
by |\input| which can also be processed individually.
However, by construction this mechanism
requires manual handling of the content to be output.

%%%%%%%%%%%%%%%%%%%%%%%%%%%%%%%%%%%%%%%%
\DescribeMacro{\ifchilddocmanual}
The main file should be prepared as usual, see \secref{sec:include}.
However, the document body must make a distinction
between processing of an individual part and of the main document, e.g.:
%
\begin{center}
\begin{tabular}{l}
|\ifchilddocmanual|\\
|\input{\childdocname}|\\
|\||else|\\
\textit{document body with }|\input{|\textit{part}|}|\\
|\||fi|
\end{tabular}
\end{center}
%
The conditional |\ifchilddocmanual| is true whenever
a part to be included by |\input| is being compiled,
and the name of the part is stored in |\childdocname|.

%%%%%%%%%%%%%%%%%%%%%%%%%%%%%%%%%%%%%%%%
\DescribeMacro{\childdocby}
Each part to be included by |\input| should start with:
%
\begin{center}
\begin{tabular}{l}
|\input{childdoc.def}|\\
|\childdocby{|\textit{main}|}|\\
\end{tabular}
\end{center}
%
The directive |\childdocby| is similar to |\childdocof|
described in \secref{sec:include},
but the subsequent selection of content must be done manually.
To that end, both |\ifchilddoc| and |\ifchilddocmanual|
will be true upon processing of a part,
and the name of the part is stored in |\childdocname|.
Note that |\jobname| will be set to the filename of the current part
so that each part receives an individual |.aux| file
that does not interfere with the |.aux| file(s) of the main document.
This behaviour can be altered by the alternative form
|\childdocby[*]{|\textit{main}|}| (with a non-empty optional argument)
which uses the |.aux| file of the main document
by setting |\jobname| to \textit{main}.

%%%%%%%%%%%%%%%%%%%%%%%%%%%%%%%%%%%%%%%%%%%%%%%%%%%%%%%%%%%%%%%%%%%%%%%%%%%%%%%%
\subsection{Driver Development}
\label{sec:driver}

The \textsf{childdoc} mechanism can also be use for the development
of definition files such as \LaTeX{} styles or classes.
This case differs from the above setup with multiple parts
included by |\include| in that no |\includeonly| should be invoked.
This can be achieved by starting the include file
(before |\ProvidesPackage|) with:
%
\begin{center}
\begin{tabular}{l}
|\input{childdoc.def}|\\
|\childdocforward{|\textit{main}|}|\\
\end{tabular}
\end{center}
%
or alternatively with:
%
\begin{center}
\begin{tabular}{l}
|\input{childdoc.def}|\\
|\childdocby{|\textit{main}|}|\\
\end{tabular}
\end{center}
%
Both forms have slightly different effects as described above.
The main file is prepared as usual, see \secref{sec:include}.

%%%%%%%%%%%%%%%%%%%%%%%%%%%%%%%%%%%%%%%%%%%%%%%%%%%%%%%%%%%%%%%%%%%%%%%%%%%%%%%%
\subsection{Legacy Detection}
\label{sec:detection}

The directive |\childdocmain| in the main file can detect
whether the complete document or merely a child is to be compiled
even without using the directive |\childdocof|.
This method is deprecated because it is less robust
and there is no compelling reason to use it;
it is merely provided for backward compatibility
and it may be removed in future versions.

If the detection mechanism is to be used,
it is mandatory to correctly specify
the filename of the main file as the argument of |\childdocmain|:
%
\begin{center}
\begin{tabular}{l}
|\input{childdoc.def}|\\
|\childdocmain{|\textit{main}|}|\\
\end{tabular}
\end{center}
%
If |\jobname| does not match the argument \textit{main} of |\childdocmain|,
it is assumed that |\jobname| points to the child file to be compiled.
When using |\childdocmain| with the main file specified as argument,
it suffices to start a child file
with just |\input{|\textit{main}|}|
without loading of the package and using |\childdocof|.
If instead all processing is done
with the appropriate \textsf{childdoc} directives,
the argument of \textit{main} of |\childdocmain| can be empty.

An alternative version of the command line processing described
in \secref{sec:commandline} using the detection mechanism reads:
%
\begin{center}
|... -jobname "|\textit{target}|" "|[\textit{flags}]%
[|\def\jobname{|\textit{dest}|}|]|\input{|\textit{main}|}"|
\end{center}

%%%%%%%%%%%%%%%%%%%%%%%%%%%%%%%%%%%%%%%%%%%%%%%%%%%%%%%%%%%%%%%%%%%%%%%%%%%%%%%%
\subsection{Manual Code}
\label{sec:manual}

In case one cannot be certain whether the definitions file |childdoc.def|
is installed on the target \TeX{} distribution
and one prefers not to ship it,
it is conceivable to paste a few relevant commands into the sources.

To that end, drop all statements |\input{childdoc.def}|
and perform the replacements as outlined below.
Instead of |\childdocmain{|\textit{main}|}| add the following code
to the top of the main file:
%
\begin{center}
\begin{tabular}{l}
|\||ifdefined\childdocname\endinput\||fi\newif\ifchilddoc|\\
|\edef\childdocname{\scantokens\expandafter{\jobname\noexpand}}|\\
|\def\childdocmain{|\textit{main}|}\||ifx\childdocmain\childdocname\||else|\\
|\childdoctrue\includeonly{\childdocname}\let\jobname\childdocmain\||fi|\\
\end{tabular}
\end{center}
%
Instead of |\childdocof{|\textit{main}|}| just include the main file
at the top of each child file:
%
\begin{center}
|\input{|\textit{main}|}|
\end{center}
%
A simple redirection |\childdocforward{|\textit{dest}|}| is achieved by:
%
\begin{center}
|\def\jobname{|\textit{dest}|}\input{\jobname}|
\end{center}
%
The redirection with prefix
|\childdocforwardprefix[|\textit{prefix}|]{|\textit{dest}|}|
is accomplished by:
%
\begin{center}
\begin{tabular}{l}
|{\edef\jobname{\scantokens\expandafter{\jobname\noexpand}}|\\
|\def\redirectjob |\textit{prefix}|#1~~~{\gdef\jobname{|\textit{dest}|#1}}|\\
|\expandafter\redirectjob\jobname~~~}\input{\jobname}|
\end{tabular}
\end{center}

In an alternative approach,
child documents can be compiled by a specific command line
without additional code or specific definitions:
%
\begin{center}
|... -jobname "|\textit{target}|" "|[\textit{flags}]%
|\includeonly{|\textit{dest}|}\input{|\textit{main}|}"|
\end{center}
%

%%%%%%%%%%%%%%%%%%%%%%%%%%%%%%%%%%%%%%%%%%%%%%%%%%%%%%%%%%%%%%%%%%%%%%%%%%%%%%%%
%%%%%%%%%%%%%%%%%%%%%%%%%%%%%%%%%%%%%%%%%%%%%%%%%%%%%%%%%%%%%%%%%%%%%%%%%%%%%%%%
\section{Information}

%%%%%%%%%%%%%%%%%%%%%%%%%%%%%%%%%%%%%%%%%%%%%%%%%%%%%%%%%%%%%%%%%%%%%%%%%%%%%%%%
\subsection{Copyright}

Copyright \copyright{} 2017--2018 Niklas Beisert

This work may be distributed and/or modified under the
conditions of the \LaTeX{} Project Public License, either version 1.3
of this license or (at your option) any later version.
The latest version of this license is in
  \url{http://www.latex-project.org/lppl.txt}
and version 1.3 or later is part of all distributions of \LaTeX{}
version 2005/12/01 or later.

This work has the LPPL maintenance status `maintained'.

The Current Maintainer of this work is Niklas Beisert.

This work consists of the files |README.txt|, |childdoc.ins| and |childdoc.dtx|
as well as the derived files |childdoc.def|, |cdocsamp.tex|
with |cdocsch1.tex|, |cdocsch2.tex|, |cdocspt3.tex|, |cdocspt4.tex|,
|cdocsdrf.tex|, |cdocsfn1.tex|, |cdocsfn2.tex|
as well as |childdoc.pdf|.

%%%%%%%%%%%%%%%%%%%%%%%%%%%%%%%%%%%%%%%%%%%%%%%%%%%%%%%%%%%%%%%%%%%%%%%%%%%%%%%%
\subsection{Files and Installation}

The package consists of the files:
%
\begin{center}
\begin{tabular}{ll}
    |README.txt|   & readme file \\
    |childdoc.ins| & installation file \\
    |childdoc.dtx| & source file \\
    |childdoc.def| & definition file \\
    |cdocsamp.tex| & sample main file \\
    |cdocsch1.tex| & sample include file \\
    |cdocsch2.tex| & sample include file \\
    |cdocspt3.tex| & sample part file \\
    |cdocspt4.tex| & sample part file \\
    |cdocsdrf.tex| & sample redirection file \\
    |cdocsfn1.tex| & sample redirection file \\
    |cdocsfn2.tex| & sample redirection file \\
    |childdoc.pdf| & manual
\end{tabular}
\end{center}
%
The distribution consists of the files
|README.txt|, |childdoc.ins| and |childdoc.dtx|.
%
\begin{itemize}
\item
Run (pdf)\LaTeX{} on |childdoc.dtx|
to compile the manual |childdoc.pdf| (this file).
\item
Run \LaTeX{} on |childdoc.ins| to create the definitions file |childdoc.def|
and the sample |cdocsamp.tex| with include files
|cdocsch1.tex|, |cdocsch2.tex|, |cdocspt3.tex|, |cdocspt4.tex|,
|cdocsdrf.tex|, |cdocsfn1.tex|, |cdocsfn2.tex|.
Then copy the file |childdoc.def| to an appropriate directory of your \LaTeX{}
distribution, e.g.\ \textit{texmf-root}|/tex/latex/childdoc|.
\end{itemize}

%%%%%%%%%%%%%%%%%%%%%%%%%%%%%%%%%%%%%%%%%%%%%%%%%%%%%%%%%%%%%%%%%%%%%%%%%%%%%%%%
\subsection{Related CTAN Packages}

There are several other packages which offer a similar functionality:
%
\begin{itemize}
\item
The packages
\href{http://ctan.org/pkg/docmute}{\textsf{docmute}},
\href{http://ctan.org/pkg/includex}{\textsf{includex}} and
\href{http://ctan.org/pkg/standalone}{\textsf{standalone}}
provide commands to include only the document body of
a child file thus allowing both files to be compiled individually.
\item
The packages \href{http://ctan.org/pkg/subdocs}{\textsf{subdocs}}
and \href{http://ctan.org/pkg/subfiles}{\textsf{subfiles}}
provide structures in which the main and child documents can be
encapsulated and allowing them to be compiled individually.
The inclusion mechanism is different from the conventional |\include|.
\item
The package \href{http://ctan.org/pkg/combine}{\textsf{combine}}
is an elaborate solution to combine several documents into one.
\end{itemize}
%
See also the CTAN topic \href{http://ctan.org/topic/subdocs}{\textsf{subdocs}}
for further related packages.
The present package differs from the above solutions in that
a document structure constructed with the conventional |\include| mechanism
just needs two extra commands at the top of every file
such that all constituent files can be compiled individually.

%%%%%%%%%%%%%%%%%%%%%%%%%%%%%%%%%%%%%%%%%%%%%%%%%%%%%%%%%%%%%%%%%%%%%%%%%%%%%%%%
%\subsection{Feature Suggestions}
%
%The following is a list of features which may be useful for future
%versions of this package:
%%
%\begin{itemize}
%\item
%\ldots
%\end{itemize}

%%%%%%%%%%%%%%%%%%%%%%%%%%%%%%%%%%%%%%%%%%%%%%%%%%%%%%%%%%%%%%%%%%%%%%%%%%%%%%%%
\subsection{Revision History}

%%%%%%%%%%%%%%%%%%%%%%%%%%%%%%%%%%%%%%%%
\paragraph{v2.0:} 2018/12/30

\begin{itemize}
\item
immediate forward processing
\item
added |\childdocby| mechanism
\item
manual restructured
\end{itemize}

%%%%%%%%%%%%%%%%%%%%%%%%%%%%%%%%%%%%%%%%
\paragraph{v1.6:} 2018/01/17

\begin{itemize}
\item
application for development of include files
\item
corrections to manual
\end{itemize}

%%%%%%%%%%%%%%%%%%%%%%%%%%%%%%%%%%%%%%%%
\paragraph{v1.5:} 2017/05/21

\begin{itemize}
\item
more complete structuring introduced
\item
|\childdocof| introduced
\item
|\childdoc| renamed to |\childdocmain|
\item
|\childredirect| renamed to |\childdocforward| and |\childdocforwardprefix|
and functionality expanded
\end{itemize}

%%%%%%%%%%%%%%%%%%%%%%%%%%%%%%%%%%%%%%%%
\paragraph{v1.0:} 2017/04/27

\begin{itemize}
\item
manual and install package
\item
first version published on CTAN
\end{itemize}

%%%%%%%%%%%%%%%%%%%%%%%%%%%%%%%%%%%%%%%%
\paragraph{v0.6:} 2017/04/26

\begin{itemize}
\item
redirection mechanism added
\end{itemize}

%%%%%%%%%%%%%%%%%%%%%%%%%%%%%%%%%%%%%%%%
\paragraph{v0.5:} 2017/04/26

\begin{itemize}
\item
functionality in definition file
\end{itemize}


%%%%%%%%%%%%%%%%%%%%%%%%%%%%%%%%%%%%%%%%%%%%%%%%%%%%%%%%%%%%%%%%%%%%%%%%%%%%%%%%
%%%%%%%%%%%%%%%%%%%%%%%%%%%%%%%%%%%%%%%%%%%%%%%%%%%%%%%%%%%%%%%%%%%%%%%%%%%%%%%%
%%%%%%%%%%%%%%%%%%%%%%%%%%%%%%%%%%%%%%%%%%%%%%%%%%%%%%%%%%%%%%%%%%%%%%%%%%%%%%%%
\appendix

\settowidth\MacroIndent{\rmfamily\scriptsize 000\ }

 \DocInput{childdoc.dtx}

\end{document}
%</driver>
% \fi
%
% %%%%%%%%%%%%%%%%%%%%%%%%%%%%%%%%%%%%%%%%%%%%%%%%%%%%%%%%%%%%%%%%%%%%%%%%%%%%%%
% %%%%%%%%%%%%%%%%%%%%%%%%%%%%%%%%%%%%%%%%%%%%%%%%%%%%%%%%%%%%%%%%%%%%%%%%%%%%%%
% \section{Sample}
%\iffalse
%<*samplemain>
%\fi
%
% The following presents a sample document
% with two chapters, two parts, a title page,
% a compile flag as well as three forwarding files to set the flag.
% It consists of eight |.tex| files:
% \begin{center}
% \begin{tabular}{ll}
% |cdocsamp.tex|&main file\\
% |cdocsch1.tex|&include file for chapter 1\\
% |cdocsch2.tex|&include file for chapter 2\\
% |cdocspt3.tex|&include file for part 3\\
% |cdocspt4.tex|&include file for part 4\\
% |cdocsdrf.tex|&forwarding file for main file in draft mode\\
% |cdocsfi1.tex|&forwarding file for final version of chapter 1\\
% |cdocsfi2.tex|&forwarding file for final version of chapter 2\\
% \end{tabular}
% \end{center}
% Each of the eight files can be compiled directly by the \LaTeX{} compiler.
%
% %%%%%%%%%%%%%%%%%%%%%%%%%%%%%%%%%%%%%%
% \paragraph{Main File.}
%
% The main file is called |cdocsamp.tex|.
%
% Load the \textsf{childdoc} definitions and
% declare the filename for the main document:
%    \begin{macrocode}
\input{childdoc.def}
\childdocmain{}
%    \end{macrocode}

% Optional override for |\version| flag:
%    \begin{macrocode}
%%\ifchilddoc\else\providecommand{\version}{draft}\fi
%    \end{macrocode}

% Define the default values for the |\version| flag
% (|final| for the main file and |draft| for childs):
%    \begin{macrocode}
\ifchilddoc
\providecommand{\version}{draft}
\else
\providecommand{\version}{final}
\fi
%    \end{macrocode}

% Load the standard document class:
%    \begin{macrocode}
\documentclass[12pt]{article}
%    \end{macrocode}

% Start the document body:
%    \begin{macrocode}
\begin{document}
%    \end{macrocode}

% Declare a title page.
% Print title, part of document being processed and version flag:
%    \begin{macrocode}
\addtocounter{page}{-1}
\begin{center}
{\LARGE\bfseries{}childdoc example\par}
\vspace{1cm}
\ifchilddoc
\ifchilddocmanual part\else chapter\fi:
`\childdocname' of `\childdocjob'\par
\else
main document: `\childdocjob'\par
\fi
version: \version\par
\end{center}
\newpage
%    \end{macrocode}

% Manually include selected file,
% otherwise process as usual:
%    \begin{macrocode}
\ifchilddocmanual
\section*{part `\childdocname'}
\input{\childdocname}
\else
%    \end{macrocode}

% Include the two chapters:
%    \begin{macrocode}
\include{cdocsch1}
\include{cdocsch2}
%    \end{macrocode}

% Include the two parts unless only chapters should be displayed:
%    \begin{macrocode}
\ifchilddoc\else
\section{part three}
\input{cdocspt3}
\section{part four}
\input{cdocspt4}
\fi
%    \end{macrocode}

% Process as usual until here:
%    \begin{macrocode}
\fi
%    \end{macrocode}

% End of document body:
%    \begin{macrocode}
\end{document}
%    \end{macrocode}
%\iffalse
%</samplemain>
%\fi
%
% %%%%%%%%%%%%%%%%%%%%%%%%%%%%%%%%%%%%%%
% \paragraph{Chapter Include Files.}
%
% The include files are called |cdocsch1.tex| and |cdocsch2.tex|.
%
%\iffalse
%<*samplechap1|samplechap2>
%\fi

% Optional override for |\version| flag:
%    \begin{macrocode}
%%\providecommand{\version}{final}
%    \end{macrocode}

% Include the main document:
%    \begin{macrocode}
\input{childdoc.def}
\childdocof{cdocsamp}
%    \end{macrocode}

%\iffalse
%</samplechap1|samplechap2>
%\fi
%
%\iffalse
%<*samplechap1>
%\fi
% Some text for chapter 1:
%    \begin{macrocode}
\section{one}
some text in chapter one
%    \end{macrocode}

%\iffalse
%</samplechap1>
%\fi
% Some text for chapter 2:
%\iffalse
%<*samplechap2>
%\fi
%    \begin{macrocode}
\section{two}
more text in chapter two
%    \end{macrocode}

%\iffalse
%</samplechap2>
%\fi
%
% %%%%%%%%%%%%%%%%%%%%%%%%%%%%%%%%%%%%%%
% \paragraph{Part Include Files.}
%
% The include files are called |cdocspt3.tex| and |cdocspt4.tex|.
%
%\iffalse
%<*samplepart3|samplepart4>
%\fi

% Optional override for |\version| flag:
%    \begin{macrocode}
%%\providecommand{\version}{final}
%    \end{macrocode}

% Include the main document:
%    \begin{macrocode}
\input{childdoc.def}
\childdocby{cdocsamp}
%    \end{macrocode}

%\iffalse
%</samplepart3|samplepart4>
%\fi
%
%\iffalse
%<*samplepart3>
%\fi
% Some text for part 3:
%    \begin{macrocode}
some text in part three
%    \end{macrocode}

%\iffalse
%</samplepart3>
%\fi
% Some text for part 4:
%\iffalse
%<*samplepart4>
%\fi
%    \begin{macrocode}
more text in part four
%    \end{macrocode}

%\iffalse
%</samplepart4>
%\fi
%
% %%%%%%%%%%%%%%%%%%%%%%%%%%%%%%%%%%%%%%
% \paragraph{Forwarding for a Complete Draft.}
%
% The following forwarding file |cdocsdrf.tex|
% compiles the main document in draft mode:
%\iffalse
%<*sampledraft>
%\fi
%    \begin{macrocode}
\def\version{draft}
\input{childdoc.def}
\childdocforward{cdocsamp}
%    \end{macrocode}

%\iffalse
%</sampledraft>
%\fi
%
% %%%%%%%%%%%%%%%%%%%%%%%%%%%%%%%%%%%%%%
% \paragraph{Forwarding for Final Version of the Chapters.}
%
% The following forwarding files |cdocsfn1.tex| and |cdocsfn2.tex|
% (with identical content)
% compile the final versions of the child documents
% |cdocsch1.tex| and |cdocsch2.tex|, respectively:
%\iffalse
%<*samplefinal>
%\fi
%    \begin{macrocode}
\def\version{final}
\input{childdoc.def}
\childdocforwardprefix[cdocsamp]{cdocsfn}{cdocsch}
%    \end{macrocode}

%\iffalse
%</samplefinal>
%\fi
%
% %%%%%%%%%%%%%%%%%%%%%%%%%%%%%%%%%%%%%%
% \paragraph{Command Line Processing.}
%
% The following three command lines generate the output files
% |cdocscld|, |cdocscl1| and |cdocscl2|
% which should be identical to
% |cdocsdrf|, |cdocsch1| and |cdocsfn2|, respectively:
% \begin{center}
% \begin{tabular}{l}
% |latex -jobname cdocscld \|\\
% |  "\def\version{draft}\input{childdoc.def}\childdocforward{cdocsamp}"|\\
% |latex -jobname cdocscl1 \|\\
% |  "\input{childdoc.def}\childdocforward[cdocsamp]{cdocsch1}"|\\
% |latex -jobname cdocscl2 \|\\
% |  "\def\version{final}\input{childdoc.def}\childdocforward{cdocsch2}"|
% \end{tabular}
% \end{center}
% Note that the trailing backslash on each first line
% merely continues the input to the second line
% (for convenient cut ant paste).
% Furthermore, the command |latex| can be replaced by any
% of its alternative versions such as |pdflatex|.
%
% %%%%%%%%%%%%%%%%%%%%%%%%%%%%%%%%%%%%%%%%%%%%%%%%%%%%%%%%%%%%%%%%%%%%%%%%%%%%%%
% %%%%%%%%%%%%%%%%%%%%%%%%%%%%%%%%%%%%%%%%%%%%%%%%%%%%%%%%%%%%%%%%%%%%%%%%%%%%%%
% \section{Implementation}
%\iffalse
%<*package>
%\fi
%
% This section describes the definitions file |childdoc.def|.

% The definitions cannot be loaded using |\usepackage| or |\RequirePackage|
% which has a mechanism to prevent loading a style file more than once.
% When loading the definitions by means of |\input|
% multiple instances have to be prevented manually:
%\iffalse
%This code needs to be before the `\ProvidesFile' directive
%which is defined at the beginning of this file.
%Therefore it is also placed there and commented out here.
%</package>
%<*discard>
%\fi
%    \begin{macrocode}
\ifdefined\childdocmain\endinput\fi
%    \end{macrocode}
%\iffalse
%</discard>
%<*package>
%\fi
%
% \macro{\ifchilddoc}
% \macro{\ifchilddocmanual}
% The conditional |\ifchilddoc| tells whether a
% child (true) or main (false) document is being compiled.
% The conditional |\ifchilddocmanual| tells whether
% the |\includeonly| mechanism is used (false) or
% the selection of child files must be performed manually (true).
% The definitions initialise to false:
%    \begin{macrocode}
\newif\ifchilddoc
\newif\ifchilddocmanual
%    \end{macrocode}

% \macro{\childdocname}
% \macro{\childdocjob}
% The macro |\childdocname| stores the name of the main document
% to be compiled. The macro |\childdocjob| stores the name of
% the document on which the \LaTeX{} compiler was originally invoked.
% The content of |\jobname| cannot be compared
% to filenames specified in the source due to different catcodes.
% The following code rescans |\jobname|, stores the result
% in |\childdocname| and saves a copy in |\childdocjob|:
%    \begin{macrocode}
\edef\childdocname{\scantokens\expandafter{\jobname\noexpand}}
\let\childdocjob\childdocname
%    \end{macrocode}

% \macro{\childdocdisable}
% The macro |\childdocdisable| prevents the main file
% from being processed more than once.
% At this stage, the main document command |\childdocmain|
% is assumed to be called once again where it should do nothing.
% Any subsequent call to it should prevent
% a secondary processing of the main document
% It overwrites the forwarding commands
% |\childdocof| and |\childdocforward|
% with empty macros to prevent further inclusions of the main document:
%    \begin{macrocode}
\newcommand{\childdocdisable}
{
  \renewcommand{\childdocmain}[1]{\renewcommand{\childdocmain}[1]{\endinput}}
  \renewcommand{\childdocof}[1]{}
  \renewcommand{\childdocby}[2][]{}
  \renewcommand{\childdocforward}[2][]{}
  \renewcommand{\childdocdisable}{}
}
%    \end{macrocode}

% \macro{\childdocmain}
% The macro |\childdocmain| is to be called at the top of the main file
% with nothing or the main filename (without extension) as argument.
% First, it breaks loops.
% If the argument is not empty and does not match |\childdocname|
% (which is set by the first inclusion of |childdoc.def|),
% |\ifchilddoc| is set to true, |\includeonly| is applied to the child file
% and |\jobname| is set to the main file
% (for proper handling of |.aux| files):
%    \begin{macrocode}
\newcommand{\childdocmain}[1]
{
  \childdocdisable\childdocmain{}
  \if?#1?\else
    \begingroup
      \def\childdoctmp{#1}
      \ifx\childdoctmp\childdocname
        \def\childdoctmp{}
      \else
        \def\childdoctmp
        {
          \childdoctrue
          \includeonly{\childdocname}
          \def\childdocjob{#1}
          \def\jobname{#1}
        }
      \fi
      \expandafter
    \endgroup
    \childdoctmp
  \fi
}
%    \end{macrocode}

% \macro{\childdocof}
% The command |\childdocof| redirects
% compilation to the main file |#1|.
%    \begin{macrocode}
\newcommand{\childdocof}[1]
{
  \childdocdisable
  \childdoctrue
  \includeonly{\childdocname}
  \def\jobname{#1}
  \def\childdocjob{#1}
  \input{#1}
}
%    \end{macrocode}

% \macro{\childdocby}
% The command |\childdocby| ....
%    \begin{macrocode}
\newcommand{\childdocby}[2][]
{
  \childdocdisable
  \childdoctrue
  \childdocmanualtrue
  \if?#1?\else
    \def\jobname{#2}
  \fi
  \def\childdocjob{#2}
  \input{#2}
  \endinput
}
%    \end{macrocode}

% \macro{\childdocforward}
% The command |\childdocforward| redirects
% compilation to the main file or
% (if the optional argument is given) a child file.
% Parameters are set as if the main file
% or a child file starting with |\childdocof| was compiled.
% Then compilation is handed over to the main file:
%    \begin{macrocode}
\newcommand{\childdocforward}[2][]
{
  \begingroup
    \if?#1?
      \def\childdoctmp
      {
        \def\childdocname{#2}
        \def\childdocjob{#2}
        \def\jobname{#2}
        \input{#2}
        \endinput
      }
    \else
      \def\childdoctmp
      {
        \childdocdisable
        \def\childdocname{#2}
        \childdoctrue
        \includeonly{#2}
        \def\childdocjob{#1}
        \def\jobname{#1}
        \input{#1}
        \endinput
      }
    \fi
    \expandafter
  \endgroup
  \childdoctmp
}
%    \end{macrocode}

% \macro{\childdocforwardprefix}
% The command |\childdocforwardprefix| redirects
% compilation to the main or a child file by means of a pattern.
% The prefix |#1| in the current filename is replaced by |#2|
% and the suffix of the current filename is kept
% (it is assumed that the filename does not contain the substring `|~~~|'
% which is used as a delimiter).
% Compilation is handed over to the new file by |\childdocforward|:
%    \begin{macrocode}
\newcommand{\childdocforwardprefix}[3][]
{
  \begingroup
    \def\childdocextract #2##1~~~{\def\childdoctmp{\childdocforward[#1]{#3##1}}}
    \expandafter\childdocextract\childdocname~~~
    \expandafter
  \endgroup
  \childdoctmp
}
%    \end{macrocode}

% \macro{\childdoc}
% The deprecated macro |\childdoc| is a legacy version of |\childdocmain|:
%    \begin{macrocode}
\newcommand{\childdoc}{\childdocmain}
%    \end{macrocode}

% \macro{\childdocredirect}
% The deprecated macro |\childdocredirect| is a legacy version
% of |\childdocforward| and |\childdocforwardprefix|:
%    \begin{macrocode}
\newcommand{\childdocredirect}[2][]
{
  \begingroup
    \if?#1?
      \def\childdoctmp{\childdocforward{#2}}
    \else
      \def\childdoctmp{\childdocforwardprefix{#1}{#2}}
    \fi
    \expandafter
  \endgroup
  \childdoctmp
}
%    \end{macrocode}

%\iffalse
%</package>
%\fi
%
\endinput
|\\
|\childdocforwardprefix{final}{child}|
\end{tabular}
\end{center}
%

Note that when several versions of a main file and/or of each child file
are to be generated, it may be convenient to set up a |Makefile| or
shell script to automatise the process.

%%%%%%%%%%%%%%%%%%%%%%%%%%%%%%%%%%%%%%%%%%%%%%%%%%%%%%%%%%%%%%%%%%%%%%%%%%%%%%%%
\subsection{Command Line Processing}
\label{sec:commandline}

The effect of redirection files can also be achieved by invoking
the \LaTeX{} compiler with a more elaborate command line.
Most conveniently this should be done as part
of a shell script or a |Makefile|.

When using \textsf{childdoc} in the main file, the following
command lines effectively perform a redirection
(note that depending on the shell being used,
backslashes may have to be doubled: `|\|' $\to$ `|\\|'):
%
\begin{center}
|... -jobname "|\textit{target}|" |\\|"|[\textit{flags}]%
|% \iffalse
%
% childdoc.dtx Copyright (C) 2017-2018 Niklas Beisert
%
% This work may be distributed and/or modified under the
% conditions of the LaTeX Project Public License, either version 1.3
% of this license or (at your option) any later version.
% The latest version of this license is in
%   http://www.latex-project.org/lppl.txt
% and version 1.3 or later is part of all distributions of LaTeX
% version 2005/12/01 or later.
%
% This work has the LPPL maintenance status `maintained'.
%
% The Current Maintainer of this work is Niklas Beisert.
%
% This work consists of the files childdoc.dtx and childdoc.ins
% and the derived files childdoc.def and cdocsamp.tex with
% cdocsch1.tex, cdocsch2.tex, cdocsdrf.tex, cdocsfn1.tex, cdocsfn2.tex.
%
%<package>\ifdefined\childdocmain\endinput\fi
%<package>\ProvidesFile{childdoc.def}[2018/12/30 v2.0 child document driver]
%<samplemain>\ProvidesFile{cdocsamp.tex}[2018/12/30 v2.0 sample for childdoc]
%<*driver>
%\ProvidesFile{childdoc.drv}[2018/12/30 v2.0 childdoc reference manual file]
\PassOptionsToClass{10pt,a4paper}{article}
\documentclass{ltxdoc}

\usepackage[margin=35mm]{geometry}
\usepackage{hyperref}
\usepackage{hyperxmp}
\usepackage[usenames]{color}

\hypersetup{colorlinks=true}
\hypersetup{pdfstartview=FitH}
\hypersetup{pdfpagemode=UseNone}
\hypersetup{pdfsource={}}
\hypersetup{pdflang={en-UK}}
\hypersetup{pdfcopyright={Copyright 2017-2018 Niklas Beisert.
  This work may be distributed and/or modified under the
  conditions of the LaTeX Project Public License, either version 1.3
  of this license or (at your option) any later version.}}
\hypersetup{pdflicenseurl={http://www.latex-project.org/lppl.txt}}
\hypersetup{pdfcontactaddress={ETH Zurich, ITP, HIT K,
  Wolfgang-Pauli-Strasse 27}}
\hypersetup{pdfcontactpostcode={8093}}
\hypersetup{pdfcontactcity={Zurich}}
\hypersetup{pdfcontactcountry={Switzerland}}
\hypersetup{pdfcontactemail={nbeisert@itp.phys.ethz.ch}}
\hypersetup{pdfcontacturl={http://people.phys.ethz.ch/\xmptilde nbeisert/}}

\newcommand{\secref}[1]{\hyperref[#1]{section \ref*{#1}}}

\parskip1ex
\parindent0pt
\let\olditemize\itemize
\def\itemize{\olditemize\parskip0pt}

\begin{document}

\title{The \textsf{childdoc} Package}
\hypersetup{pdftitle={The childdoc Package}}
\author{Niklas Beisert\\[2ex]
  Institut f\"ur Theoretische Physik\\
  Eidgen\"ossische Technische Hochschule Z\"urich\\
  Wolfgang-Pauli-Strasse 27, 8093 Z\"urich, Switzerland\\[1ex]
  \href{mailto:nbeisert@itp.phys.ethz.ch}
  {\texttt{nbeisert@itp.phys.ethz.ch}}}
\hypersetup{pdfauthor={Niklas Beisert}}
\hypersetup{pdfsubject={Manual for the LaTeX2e Package childdoc}}
\date{30 December 2018, \textsf{v2.0}}
\maketitle

\begin{abstract}\noindent
\textsf{childdoc} is a \LaTeXe{} package
that enables the direct compilation
of document sections included by |\include|
to individual files.
\end{abstract}

\begingroup
\parskip0ex
\tableofcontents
\endgroup

%%%%%%%%%%%%%%%%%%%%%%%%%%%%%%%%%%%%%%%%%%%%%%%%%%%%%%%%%%%%%%%%%%%%%%%%%%%%%%%%
%%%%%%%%%%%%%%%%%%%%%%%%%%%%%%%%%%%%%%%%%%%%%%%%%%%%%%%%%%%%%%%%%%%%%%%%%%%%%%%%
\section{Introduction}

\LaTeX{} provides a mechanism to structure a large document (such as a book)
into a main file and several child files (containing the chapters)
using the |\include| command.
This mechanism is beneficial for documents
which span hundreds of pages in order to
make the source file(s) more manageable.
Moreover, compilation can be restricted to
selected child files by means of the |\includeonly| command.
The latter feature can be used to reduce the compilation time while editing
(this was significantly more useful in the earlier days of \LaTeX{})
or to generate a smaller document which is easier to navigate.
Another application of |\includeonly| is to generate
documents consisting of selected parts of the complete document.

However, there are a few drawbacks of the plain |\include| mechanism:
\begin{itemize}
\item
The child files cannot be compiled on their own,
they can only be compiled via the main file.
A naive editing environment
(such as a text editor with an option
to have the current file processed by \LaTeX)
may require one to switch to the main file before compiling;
attempting to compile the child file produces errors.
\item
The main file must be modified (each time)
to adjust the |\includeonly| command
to the present needs. This easily leaves the main file in a messy state.
\item
The generated document will always carry the filename
of the main document. This is inconvenient if
several child files are to be compiled and
to be kept for distribution.
\end{itemize}

The present package provides a simple interface
to make child files individually compilable by \LaTeX{}.
Compiling a child file then has the same effect as compiling
the main file with an |\includeonly| command
to select the appropriate child.
Moreover the generated document will carry the name of the child
rather than the main file.
This resolves all three above issues.

This feature is meant to make the editing of books,
thesis documents and lecture notes somewhat more convenient.
However, the package can also be used efficiently for
composing a series of documents (such as exercise sheets)
which are typically distributed individually.
It then assists the author in generating the individual documents
(potentially in different versions)
as well as a document containing the collected series.
Another application is in developing style files
or other kinds of included material
where compilation of the style file could redirect
to a sample or test file.

%%%%%%%%%%%%%%%%%%%%%%%%%%%%%%%%%%%%%%%%%%%%%%%%%%%%%%%%%%%%%%%%%%%%%%%%%%%%%%%%
%%%%%%%%%%%%%%%%%%%%%%%%%%%%%%%%%%%%%%%%%%%%%%%%%%%%%%%%%%%%%%%%%%%%%%%%%%%%%%%%
\section{Usage}

First of all, the package \textsf{childdoc} is \emph{not} a standard
\LaTeXe{} |.sty| style file! Therefore it needs to be invoked in
a non-standard way.

%%%%%%%%%%%%%%%%%%%%%%%%%%%%%%%%%%%%%%%%%%%%%%%%%%%%%%%%%%%%%%%%%%%%%%%%%%%%%%%%
\subsection{Included Files}
\label{sec:include}

%%%%%%%%%%%%%%%%%%%%%%%%%%%%%%%%%%%%%%%%
\DescribeMacro{\childdocmain}
To use the package, add the commands
\begin{center}
\begin{tabular}{l}
|\input{childdoc.def}|\\
|\childdocmain{}|\\
\end{tabular}
\end{center}
at the very top of the main \LaTeX{} file,
in particular \emph{before} the |\documentclass| statement!
The argument of |\childdocmain| should be left empty
(but it must be present).

%%%%%%%%%%%%%%%%%%%%%%%%%%%%%%%%%%%%%%%%
\DescribeMacro{\childdocof}
Furthermore, add the commands
\begin{center}
\begin{tabular}{l}
|\input{childdoc.def}|\\
|\childdocof{|\textit{main}|}|\\
\end{tabular}
\end{center}
at the top of every child file \textit{child}
which is included by |\include{|\textit{child}|}|
from within the main file
(or at least for those files to be compiled individually).
The argument \textit{main} must be the filename of the main file.

There are a couple of
considerations in setting up the main and child documents:

%%%%%%%%%%%%%%%%%%%%%%%%%%%%%%%%%%%%%%%%
\paragraph{Restrictions.}

Please note the following restrictions:
\begin{itemize}
\item
|\childdocmain| must be called with one argument \textit{main}
to ensure compatibility with earlier version of the package.
It must either be empty (|\childdocmain{}|)
or precisely match the filename of the main file in which it is specified.
See \secref{sec:detection} for further information.
\item
The filename \textit{main} must be specified without the |.tex| extension.
\item
The filename \textit{main} is case sensitive
(even in case-insensitive file systems)
due to internal string comparison.
\item
The argument \textit{main} should be fully expanded, it cannot be a macro.
\item
Subdirectories and special characters should be avoided in filenames.
\item
The command |\childdocmain{|\textit{main}|}| must be followed by a whitespace.
It should not be followed immediately by another command
or by a comment mark `|%|'.
This is because the \TeX{} parser reads the token immediately following
the argument of |\childdocmain| and puts it
at the beginning of every child section;
however, a white\-space is ignored.
\end{itemize}

%%%%%%%%%%%%%%%%%%%%%%%%%%%%%%%%%%%%%%%%
\paragraph{Content of Main File.}

It is advisable to place all content in the child files included by |\include|.
Any output contained in the main file will appear in all child documents
unless suppressed manually;
it cannot be suppressed automatically by the |\includeonly| directive
and thus should normally be avoided.
A method to include some content in the main file
by means of conditional processing is described in \secref{sec:conditional}.

%%%%%%%%%%%%%%%%%%%%%%%%%%%%%%%%%%%%%%%%
\paragraph{Page Numbering.}

When only a part of the document is compiled,
the appropriate numbering of pages
(as well as other status parameters)
is determined from the |.aux| files.
The latter contain information from previous passes.
However this information needs to propagate through
all intermediate child documents.
Therefore the page numbering in child documents may well
be inconsistent until the complete document is compiled at least once.

A useful (if unconventional) way to always ensure a consistent
page numbering is to restart the numbering in each child document
and denote the pages by `\textit{child}|.|\textit{page}'
where \textit{child} represents the chapter/section number of the child file.
This can be achieved by the command
|\numberwithin{page}{|\textit{child}|}|
of the \textsf{amsmath} package
where \textit{child} can be |chapter| or |section|
depending on the chosen structuring.
Alternatively, one can modify the macro |\thepage| appropriately
and reset the counter |page| at the start of each child file.

%%%%%%%%%%%%%%%%%%%%%%%%%%%%%%%%%%%%%%%%%%%%%%%%%%%%%%%%%%%%%%%%%%%%%%%%%%%%%%%%
\subsection{Conditional Processing}
\label{sec:conditional}

The package provides a mechanism to compile different versions
of a document. To customise the versions further some conditional processing
can come in handy to distinguish which version is being compiled.
The package provides two macros to describe the compilation context:

%%%%%%%%%%%%%%%%%%%%%%%%%%%%%%%%%%%%%%%%
\DescribeMacro{\ifchilddoc}
The conditional |\ifchilddoc| distinguishes between the compilation of
child documents and the main document:
%
\begin{center}
|\ifchilddoc |\textit{child-code}| |[|\||else |\textit{main-code}]| \||fi|
\end{center}

%%%%%%%%%%%%%%%%%%%%%%%%%%%%%%%%%%%%%%%%
\DescribeMacro{\childdocname}
\DescribeMacro{\childdocjob}
The macro |\childdocname| contains the filename (without extension)
of the main or child file being processed.
Note that |\childdocjob| will always contain the name of the main file.

%%%%%%%%%%%%%%%%%%%%%%%%%%%%%%%%%%%%%%%%
\paragraph{Title Page.}

Conditional processing can be used to include a title or banner page
in the main document when proper precautions are taken.
Importantly, the code in the main file should ensure that the page counter
(as well as other status parameters which are stored in the |.aux| files)
takes the same value after the conditional processing.
Otherwise the page numbers may take divergent values
depending on which part is compiled.

For example, a title page could be declared by:
%
\begin{center}
\begin{tabular}{l}
|\ifchilddoc\||else|\\
|\addtocounter{page}{-1}|\\
\textit{code for title page}\\
|\newpage|\\
|\||fi|
\end{tabular}
\end{center}
%
A banner page for the child documents can be generated by:
%
\begin{center}
\begin{tabular}{l}
|\ifchilddoc|\\
|\addtocounter{page}{-1}|\\
\textit{code for banner page}\\
|\newpage|\\
|\||fi|
\end{tabular}
\end{center}
%
Here one could write a message such as:
\begin{center}
|This is the part \childdocname{} of \childdocjob{}.|
\end{center}

%%%%%%%%%%%%%%%%%%%%%%%%%%%%%%%%%%%%%%%%%%%%%%%%%%%%%%%%%%%%%%%%%%%%%%%%%%%%%%%%
\subsection{Flags}
\label{sec:flags}

The package makes it easy to generate different versions
of the main or child documents.
To this end compilation flags can be defined
and assigned different default values.
They will be particularly useful in conjunction
with the forwarding mechanism described in \secref{sec:forward}.

For example, it may be useful to have a flag |\version|
which can be set to |draft| or |final|.
The document source will contain some conditional code
depending on the value of |\version|.
Suppose further, the flag should default to |final| for the main file
and to |draft| for child files
which is a natural assignment for editing the document.
This is achieved by placing the following code
in the preamble of the main document
(below the |\childdocmain| directive):
%
\begin{center}
\begin{tabular}{l}
|\ifchilddoc|\\
|\providecommand{\version}{draft}|\\
|\||else|\\
|\providecommand{\version}{final}|\\
|\||fi|
\end{tabular}
\end{center}
%
The definition by |\providecommand| makes sure
that previous definitions are not overwritten.
Further statements |\providecommand{\version}{...}|
can thus be added before the above code to override it.

For the main file, one might add a line
(between |\childdocmain| and the above block)
%
\begin{center}
|%\ifchilddoc\||else\providecommand{\version}{draft}\||fi|
\end{center}
%
which can be uncommented to produce a draft version.
Likewise one can add a line to the very top of a child file
(above the |\childdocof{|\textit{main}|}| directive)
%
\begin{center}
|%\providecommand{\version}{final}|
\end{center}
%
which can be uncommented to produce the final version of this child document.

%%%%%%%%%%%%%%%%%%%%%%%%%%%%%%%%%%%%%%%%%%%%%%%%%%%%%%%%%%%%%%%%%%%%%%%%%%%%%%%%
\subsection{Forwarding}
\label{sec:forward}

Different versions of the main or child documents
using compilation flags as described in \secref{sec:flags}
can be (permanently) stored in different files
for convenient compilation, viewing and distribution.
To this end, the package defines a command
to pass on compilation to a different file:

%%%%%%%%%%%%%%%%%%%%%%%%%%%%%%%%%%%%%%%%
\DescribeMacro{\childdocforward}
The command |\childdocforward| redirects processing to
another source file:
%
\begin{center}
\begin{tabular}{l}
|\input{childdoc.def}|\\
|\childdocforward[|\textit{main}|]{|\textit{dest}|}|\\
\end{tabular}
\end{center}
%
The argument \textit{dest} is the destination file
(without extension).
It should be the main file or one of the child files.
Note that further \textsf{childdoc} directives
such as |\childdocof| and |\childdocforward|
in the indicated file will be processed in this form.
The optional argument \textit{main}
passes on directly to the main file \textit{main}
while pretending to compile the child \textit{dest}.
This form behaves as if \textit{dest}
issues |\childdocof{|\textit{main}|}| right away,
and no further \textsf{childdoc} directives will be processed.

%%%%%%%%%%%%%%%%%%%%%%%%%%%%%%%%%%%%%%%%
\DescribeMacro{\...prefix}
In the alternative form |\childdocforwardprefix|,
%
\begin{center}
\begin{tabular}{l}
|\input{childdoc.def}|\\
|\childdocforwardprefix[|\textit{main}|]{|\textit{prefix}|}{|\textit{dest}|}|
\end{tabular}
\end{center}
%
the destination file is determined by a pattern
depending on the current file:
To make this work, the current file must be called
`{\textit{prefix}\hspace{0.2em}\textit{suffix}}'
with \textit{prefix} matching precisely the argument.
Processing is then passed on to the file
`{\textit{dest}\hspace{0.2em}\textit{suffix}}'.
Surely, the same effect is achieved by
directly specifying the
argument `{\textit{dest}\hspace{0.2em}\textit{suffix}}'
in the first form.
However, that requires to set up a different file
for each child. With the alternative form of the command
all these files can have exactly the same content
which simplifies setting them up and maintaining them.

For example, the following file |draft.tex|
with a compilation flag |\version| as described in \secref{sec:flags}
compiles the main document as a draft:
%
\begin{center}
\begin{tabular}{l}
|\def\version{draft}|\\
|\input{childdoc.def}|\\
|\childdocforward{|\textit{main}|}|
\end{tabular}
\end{center}
%
Likewise, the following files |final|\textit{nn}|.tex|
compile the final version of the child document
|child|\textit{nn}|.tex|:
%
\begin{center}
\begin{tabular}{l}
|\def\version{final}|\\
|\input{childdoc.def}|\\
|\childdocforwardprefix{final}{child}|
\end{tabular}
\end{center}
%

Note that when several versions of a main file and/or of each child file
are to be generated, it may be convenient to set up a |Makefile| or
shell script to automatise the process.

%%%%%%%%%%%%%%%%%%%%%%%%%%%%%%%%%%%%%%%%%%%%%%%%%%%%%%%%%%%%%%%%%%%%%%%%%%%%%%%%
\subsection{Command Line Processing}
\label{sec:commandline}

The effect of redirection files can also be achieved by invoking
the \LaTeX{} compiler with a more elaborate command line.
Most conveniently this should be done as part
of a shell script or a |Makefile|.

When using \textsf{childdoc} in the main file, the following
command lines effectively perform a redirection
(note that depending on the shell being used,
backslashes may have to be doubled: `|\|' $\to$ `|\\|'):
%
\begin{center}
|... -jobname "|\textit{target}|" |\\|"|[\textit{flags}]%
|\input{childdoc.def}\childdocforward[|\textit{main}|]{|\textit{dest}|}"|
\end{center}
%
Here \textit{target} is the name of the output file,
\textit{main} is the name of the main file
and \textit{dest} is the name of the main or child file to be processed
(all filenames without extensions).
The optional argument \textit{main} can be omitted
if \textit{main} matches \textit{dest}.
Optionally, compilation \textit{flags} can be defined via |\def| commands.
This command line makes the \TeX{} engine believe
it is compiling the file \textit{target}
whose content is specified as the latter parameter.
The provided code then forwards the processing to
\textit{main} or \textit{dest} as described in \secref{sec:forward}.

%%%%%%%%%%%%%%%%%%%%%%%%%%%%%%%%%%%%%%%%%%%%%%%%%%%%%%%%%%%%%%%%%%%%%%%%%%%%%%%%
\subsection{Include by Input}
\label{sec:input}

Including child documents by |\include| has some restrictions by design.
Most notably, the content of a child document always occupies
its own set of pages; pages cannot be shared between child documents.
Usually, this behaviour makes perfect sense
because each child document contain an essential part of the document.
However, in some situations it may be desirable to compose
a document from a collection of parts
without having mandatory page breaks between then.
For this case, the package
provides a mechanism to include parts
by |\input| which can also be processed individually.
However, by construction this mechanism
requires manual handling of the content to be output.

%%%%%%%%%%%%%%%%%%%%%%%%%%%%%%%%%%%%%%%%
\DescribeMacro{\ifchilddocmanual}
The main file should be prepared as usual, see \secref{sec:include}.
However, the document body must make a distinction
between processing of an individual part and of the main document, e.g.:
%
\begin{center}
\begin{tabular}{l}
|\ifchilddocmanual|\\
|\input{\childdocname}|\\
|\||else|\\
\textit{document body with }|\input{|\textit{part}|}|\\
|\||fi|
\end{tabular}
\end{center}
%
The conditional |\ifchilddocmanual| is true whenever
a part to be included by |\input| is being compiled,
and the name of the part is stored in |\childdocname|.

%%%%%%%%%%%%%%%%%%%%%%%%%%%%%%%%%%%%%%%%
\DescribeMacro{\childdocby}
Each part to be included by |\input| should start with:
%
\begin{center}
\begin{tabular}{l}
|\input{childdoc.def}|\\
|\childdocby{|\textit{main}|}|\\
\end{tabular}
\end{center}
%
The directive |\childdocby| is similar to |\childdocof|
described in \secref{sec:include},
but the subsequent selection of content must be done manually.
To that end, both |\ifchilddoc| and |\ifchilddocmanual|
will be true upon processing of a part,
and the name of the part is stored in |\childdocname|.
Note that |\jobname| will be set to the filename of the current part
so that each part receives an individual |.aux| file
that does not interfere with the |.aux| file(s) of the main document.
This behaviour can be altered by the alternative form
|\childdocby[*]{|\textit{main}|}| (with a non-empty optional argument)
which uses the |.aux| file of the main document
by setting |\jobname| to \textit{main}.

%%%%%%%%%%%%%%%%%%%%%%%%%%%%%%%%%%%%%%%%%%%%%%%%%%%%%%%%%%%%%%%%%%%%%%%%%%%%%%%%
\subsection{Driver Development}
\label{sec:driver}

The \textsf{childdoc} mechanism can also be use for the development
of definition files such as \LaTeX{} styles or classes.
This case differs from the above setup with multiple parts
included by |\include| in that no |\includeonly| should be invoked.
This can be achieved by starting the include file
(before |\ProvidesPackage|) with:
%
\begin{center}
\begin{tabular}{l}
|\input{childdoc.def}|\\
|\childdocforward{|\textit{main}|}|\\
\end{tabular}
\end{center}
%
or alternatively with:
%
\begin{center}
\begin{tabular}{l}
|\input{childdoc.def}|\\
|\childdocby{|\textit{main}|}|\\
\end{tabular}
\end{center}
%
Both forms have slightly different effects as described above.
The main file is prepared as usual, see \secref{sec:include}.

%%%%%%%%%%%%%%%%%%%%%%%%%%%%%%%%%%%%%%%%%%%%%%%%%%%%%%%%%%%%%%%%%%%%%%%%%%%%%%%%
\subsection{Legacy Detection}
\label{sec:detection}

The directive |\childdocmain| in the main file can detect
whether the complete document or merely a child is to be compiled
even without using the directive |\childdocof|.
This method is deprecated because it is less robust
and there is no compelling reason to use it;
it is merely provided for backward compatibility
and it may be removed in future versions.

If the detection mechanism is to be used,
it is mandatory to correctly specify
the filename of the main file as the argument of |\childdocmain|:
%
\begin{center}
\begin{tabular}{l}
|\input{childdoc.def}|\\
|\childdocmain{|\textit{main}|}|\\
\end{tabular}
\end{center}
%
If |\jobname| does not match the argument \textit{main} of |\childdocmain|,
it is assumed that |\jobname| points to the child file to be compiled.
When using |\childdocmain| with the main file specified as argument,
it suffices to start a child file
with just |\input{|\textit{main}|}|
without loading of the package and using |\childdocof|.
If instead all processing is done
with the appropriate \textsf{childdoc} directives,
the argument of \textit{main} of |\childdocmain| can be empty.

An alternative version of the command line processing described
in \secref{sec:commandline} using the detection mechanism reads:
%
\begin{center}
|... -jobname "|\textit{target}|" "|[\textit{flags}]%
[|\def\jobname{|\textit{dest}|}|]|\input{|\textit{main}|}"|
\end{center}

%%%%%%%%%%%%%%%%%%%%%%%%%%%%%%%%%%%%%%%%%%%%%%%%%%%%%%%%%%%%%%%%%%%%%%%%%%%%%%%%
\subsection{Manual Code}
\label{sec:manual}

In case one cannot be certain whether the definitions file |childdoc.def|
is installed on the target \TeX{} distribution
and one prefers not to ship it,
it is conceivable to paste a few relevant commands into the sources.

To that end, drop all statements |\input{childdoc.def}|
and perform the replacements as outlined below.
Instead of |\childdocmain{|\textit{main}|}| add the following code
to the top of the main file:
%
\begin{center}
\begin{tabular}{l}
|\||ifdefined\childdocname\endinput\||fi\newif\ifchilddoc|\\
|\edef\childdocname{\scantokens\expandafter{\jobname\noexpand}}|\\
|\def\childdocmain{|\textit{main}|}\||ifx\childdocmain\childdocname\||else|\\
|\childdoctrue\includeonly{\childdocname}\let\jobname\childdocmain\||fi|\\
\end{tabular}
\end{center}
%
Instead of |\childdocof{|\textit{main}|}| just include the main file
at the top of each child file:
%
\begin{center}
|\input{|\textit{main}|}|
\end{center}
%
A simple redirection |\childdocforward{|\textit{dest}|}| is achieved by:
%
\begin{center}
|\def\jobname{|\textit{dest}|}\input{\jobname}|
\end{center}
%
The redirection with prefix
|\childdocforwardprefix[|\textit{prefix}|]{|\textit{dest}|}|
is accomplished by:
%
\begin{center}
\begin{tabular}{l}
|{\edef\jobname{\scantokens\expandafter{\jobname\noexpand}}|\\
|\def\redirectjob |\textit{prefix}|#1~~~{\gdef\jobname{|\textit{dest}|#1}}|\\
|\expandafter\redirectjob\jobname~~~}\input{\jobname}|
\end{tabular}
\end{center}

In an alternative approach,
child documents can be compiled by a specific command line
without additional code or specific definitions:
%
\begin{center}
|... -jobname "|\textit{target}|" "|[\textit{flags}]%
|\includeonly{|\textit{dest}|}\input{|\textit{main}|}"|
\end{center}
%

%%%%%%%%%%%%%%%%%%%%%%%%%%%%%%%%%%%%%%%%%%%%%%%%%%%%%%%%%%%%%%%%%%%%%%%%%%%%%%%%
%%%%%%%%%%%%%%%%%%%%%%%%%%%%%%%%%%%%%%%%%%%%%%%%%%%%%%%%%%%%%%%%%%%%%%%%%%%%%%%%
\section{Information}

%%%%%%%%%%%%%%%%%%%%%%%%%%%%%%%%%%%%%%%%%%%%%%%%%%%%%%%%%%%%%%%%%%%%%%%%%%%%%%%%
\subsection{Copyright}

Copyright \copyright{} 2017--2018 Niklas Beisert

This work may be distributed and/or modified under the
conditions of the \LaTeX{} Project Public License, either version 1.3
of this license or (at your option) any later version.
The latest version of this license is in
  \url{http://www.latex-project.org/lppl.txt}
and version 1.3 or later is part of all distributions of \LaTeX{}
version 2005/12/01 or later.

This work has the LPPL maintenance status `maintained'.

The Current Maintainer of this work is Niklas Beisert.

This work consists of the files |README.txt|, |childdoc.ins| and |childdoc.dtx|
as well as the derived files |childdoc.def|, |cdocsamp.tex|
with |cdocsch1.tex|, |cdocsch2.tex|, |cdocspt3.tex|, |cdocspt4.tex|,
|cdocsdrf.tex|, |cdocsfn1.tex|, |cdocsfn2.tex|
as well as |childdoc.pdf|.

%%%%%%%%%%%%%%%%%%%%%%%%%%%%%%%%%%%%%%%%%%%%%%%%%%%%%%%%%%%%%%%%%%%%%%%%%%%%%%%%
\subsection{Files and Installation}

The package consists of the files:
%
\begin{center}
\begin{tabular}{ll}
    |README.txt|   & readme file \\
    |childdoc.ins| & installation file \\
    |childdoc.dtx| & source file \\
    |childdoc.def| & definition file \\
    |cdocsamp.tex| & sample main file \\
    |cdocsch1.tex| & sample include file \\
    |cdocsch2.tex| & sample include file \\
    |cdocspt3.tex| & sample part file \\
    |cdocspt4.tex| & sample part file \\
    |cdocsdrf.tex| & sample redirection file \\
    |cdocsfn1.tex| & sample redirection file \\
    |cdocsfn2.tex| & sample redirection file \\
    |childdoc.pdf| & manual
\end{tabular}
\end{center}
%
The distribution consists of the files
|README.txt|, |childdoc.ins| and |childdoc.dtx|.
%
\begin{itemize}
\item
Run (pdf)\LaTeX{} on |childdoc.dtx|
to compile the manual |childdoc.pdf| (this file).
\item
Run \LaTeX{} on |childdoc.ins| to create the definitions file |childdoc.def|
and the sample |cdocsamp.tex| with include files
|cdocsch1.tex|, |cdocsch2.tex|, |cdocspt3.tex|, |cdocspt4.tex|,
|cdocsdrf.tex|, |cdocsfn1.tex|, |cdocsfn2.tex|.
Then copy the file |childdoc.def| to an appropriate directory of your \LaTeX{}
distribution, e.g.\ \textit{texmf-root}|/tex/latex/childdoc|.
\end{itemize}

%%%%%%%%%%%%%%%%%%%%%%%%%%%%%%%%%%%%%%%%%%%%%%%%%%%%%%%%%%%%%%%%%%%%%%%%%%%%%%%%
\subsection{Related CTAN Packages}

There are several other packages which offer a similar functionality:
%
\begin{itemize}
\item
The packages
\href{http://ctan.org/pkg/docmute}{\textsf{docmute}},
\href{http://ctan.org/pkg/includex}{\textsf{includex}} and
\href{http://ctan.org/pkg/standalone}{\textsf{standalone}}
provide commands to include only the document body of
a child file thus allowing both files to be compiled individually.
\item
The packages \href{http://ctan.org/pkg/subdocs}{\textsf{subdocs}}
and \href{http://ctan.org/pkg/subfiles}{\textsf{subfiles}}
provide structures in which the main and child documents can be
encapsulated and allowing them to be compiled individually.
The inclusion mechanism is different from the conventional |\include|.
\item
The package \href{http://ctan.org/pkg/combine}{\textsf{combine}}
is an elaborate solution to combine several documents into one.
\end{itemize}
%
See also the CTAN topic \href{http://ctan.org/topic/subdocs}{\textsf{subdocs}}
for further related packages.
The present package differs from the above solutions in that
a document structure constructed with the conventional |\include| mechanism
just needs two extra commands at the top of every file
such that all constituent files can be compiled individually.

%%%%%%%%%%%%%%%%%%%%%%%%%%%%%%%%%%%%%%%%%%%%%%%%%%%%%%%%%%%%%%%%%%%%%%%%%%%%%%%%
%\subsection{Feature Suggestions}
%
%The following is a list of features which may be useful for future
%versions of this package:
%%
%\begin{itemize}
%\item
%\ldots
%\end{itemize}

%%%%%%%%%%%%%%%%%%%%%%%%%%%%%%%%%%%%%%%%%%%%%%%%%%%%%%%%%%%%%%%%%%%%%%%%%%%%%%%%
\subsection{Revision History}

%%%%%%%%%%%%%%%%%%%%%%%%%%%%%%%%%%%%%%%%
\paragraph{v2.0:} 2018/12/30

\begin{itemize}
\item
immediate forward processing
\item
added |\childdocby| mechanism
\item
manual restructured
\end{itemize}

%%%%%%%%%%%%%%%%%%%%%%%%%%%%%%%%%%%%%%%%
\paragraph{v1.6:} 2018/01/17

\begin{itemize}
\item
application for development of include files
\item
corrections to manual
\end{itemize}

%%%%%%%%%%%%%%%%%%%%%%%%%%%%%%%%%%%%%%%%
\paragraph{v1.5:} 2017/05/21

\begin{itemize}
\item
more complete structuring introduced
\item
|\childdocof| introduced
\item
|\childdoc| renamed to |\childdocmain|
\item
|\childredirect| renamed to |\childdocforward| and |\childdocforwardprefix|
and functionality expanded
\end{itemize}

%%%%%%%%%%%%%%%%%%%%%%%%%%%%%%%%%%%%%%%%
\paragraph{v1.0:} 2017/04/27

\begin{itemize}
\item
manual and install package
\item
first version published on CTAN
\end{itemize}

%%%%%%%%%%%%%%%%%%%%%%%%%%%%%%%%%%%%%%%%
\paragraph{v0.6:} 2017/04/26

\begin{itemize}
\item
redirection mechanism added
\end{itemize}

%%%%%%%%%%%%%%%%%%%%%%%%%%%%%%%%%%%%%%%%
\paragraph{v0.5:} 2017/04/26

\begin{itemize}
\item
functionality in definition file
\end{itemize}


%%%%%%%%%%%%%%%%%%%%%%%%%%%%%%%%%%%%%%%%%%%%%%%%%%%%%%%%%%%%%%%%%%%%%%%%%%%%%%%%
%%%%%%%%%%%%%%%%%%%%%%%%%%%%%%%%%%%%%%%%%%%%%%%%%%%%%%%%%%%%%%%%%%%%%%%%%%%%%%%%
%%%%%%%%%%%%%%%%%%%%%%%%%%%%%%%%%%%%%%%%%%%%%%%%%%%%%%%%%%%%%%%%%%%%%%%%%%%%%%%%
\appendix

\settowidth\MacroIndent{\rmfamily\scriptsize 000\ }

 \DocInput{childdoc.dtx}

\end{document}
%</driver>
% \fi
%
% %%%%%%%%%%%%%%%%%%%%%%%%%%%%%%%%%%%%%%%%%%%%%%%%%%%%%%%%%%%%%%%%%%%%%%%%%%%%%%
% %%%%%%%%%%%%%%%%%%%%%%%%%%%%%%%%%%%%%%%%%%%%%%%%%%%%%%%%%%%%%%%%%%%%%%%%%%%%%%
% \section{Sample}
%\iffalse
%<*samplemain>
%\fi
%
% The following presents a sample document
% with two chapters, two parts, a title page,
% a compile flag as well as three forwarding files to set the flag.
% It consists of eight |.tex| files:
% \begin{center}
% \begin{tabular}{ll}
% |cdocsamp.tex|&main file\\
% |cdocsch1.tex|&include file for chapter 1\\
% |cdocsch2.tex|&include file for chapter 2\\
% |cdocspt3.tex|&include file for part 3\\
% |cdocspt4.tex|&include file for part 4\\
% |cdocsdrf.tex|&forwarding file for main file in draft mode\\
% |cdocsfi1.tex|&forwarding file for final version of chapter 1\\
% |cdocsfi2.tex|&forwarding file for final version of chapter 2\\
% \end{tabular}
% \end{center}
% Each of the eight files can be compiled directly by the \LaTeX{} compiler.
%
% %%%%%%%%%%%%%%%%%%%%%%%%%%%%%%%%%%%%%%
% \paragraph{Main File.}
%
% The main file is called |cdocsamp.tex|.
%
% Load the \textsf{childdoc} definitions and
% declare the filename for the main document:
%    \begin{macrocode}
\input{childdoc.def}
\childdocmain{}
%    \end{macrocode}

% Optional override for |\version| flag:
%    \begin{macrocode}
%%\ifchilddoc\else\providecommand{\version}{draft}\fi
%    \end{macrocode}

% Define the default values for the |\version| flag
% (|final| for the main file and |draft| for childs):
%    \begin{macrocode}
\ifchilddoc
\providecommand{\version}{draft}
\else
\providecommand{\version}{final}
\fi
%    \end{macrocode}

% Load the standard document class:
%    \begin{macrocode}
\documentclass[12pt]{article}
%    \end{macrocode}

% Start the document body:
%    \begin{macrocode}
\begin{document}
%    \end{macrocode}

% Declare a title page.
% Print title, part of document being processed and version flag:
%    \begin{macrocode}
\addtocounter{page}{-1}
\begin{center}
{\LARGE\bfseries{}childdoc example\par}
\vspace{1cm}
\ifchilddoc
\ifchilddocmanual part\else chapter\fi:
`\childdocname' of `\childdocjob'\par
\else
main document: `\childdocjob'\par
\fi
version: \version\par
\end{center}
\newpage
%    \end{macrocode}

% Manually include selected file,
% otherwise process as usual:
%    \begin{macrocode}
\ifchilddocmanual
\section*{part `\childdocname'}
\input{\childdocname}
\else
%    \end{macrocode}

% Include the two chapters:
%    \begin{macrocode}
\include{cdocsch1}
\include{cdocsch2}
%    \end{macrocode}

% Include the two parts unless only chapters should be displayed:
%    \begin{macrocode}
\ifchilddoc\else
\section{part three}
\input{cdocspt3}
\section{part four}
\input{cdocspt4}
\fi
%    \end{macrocode}

% Process as usual until here:
%    \begin{macrocode}
\fi
%    \end{macrocode}

% End of document body:
%    \begin{macrocode}
\end{document}
%    \end{macrocode}
%\iffalse
%</samplemain>
%\fi
%
% %%%%%%%%%%%%%%%%%%%%%%%%%%%%%%%%%%%%%%
% \paragraph{Chapter Include Files.}
%
% The include files are called |cdocsch1.tex| and |cdocsch2.tex|.
%
%\iffalse
%<*samplechap1|samplechap2>
%\fi

% Optional override for |\version| flag:
%    \begin{macrocode}
%%\providecommand{\version}{final}
%    \end{macrocode}

% Include the main document:
%    \begin{macrocode}
\input{childdoc.def}
\childdocof{cdocsamp}
%    \end{macrocode}

%\iffalse
%</samplechap1|samplechap2>
%\fi
%
%\iffalse
%<*samplechap1>
%\fi
% Some text for chapter 1:
%    \begin{macrocode}
\section{one}
some text in chapter one
%    \end{macrocode}

%\iffalse
%</samplechap1>
%\fi
% Some text for chapter 2:
%\iffalse
%<*samplechap2>
%\fi
%    \begin{macrocode}
\section{two}
more text in chapter two
%    \end{macrocode}

%\iffalse
%</samplechap2>
%\fi
%
% %%%%%%%%%%%%%%%%%%%%%%%%%%%%%%%%%%%%%%
% \paragraph{Part Include Files.}
%
% The include files are called |cdocspt3.tex| and |cdocspt4.tex|.
%
%\iffalse
%<*samplepart3|samplepart4>
%\fi

% Optional override for |\version| flag:
%    \begin{macrocode}
%%\providecommand{\version}{final}
%    \end{macrocode}

% Include the main document:
%    \begin{macrocode}
\input{childdoc.def}
\childdocby{cdocsamp}
%    \end{macrocode}

%\iffalse
%</samplepart3|samplepart4>
%\fi
%
%\iffalse
%<*samplepart3>
%\fi
% Some text for part 3:
%    \begin{macrocode}
some text in part three
%    \end{macrocode}

%\iffalse
%</samplepart3>
%\fi
% Some text for part 4:
%\iffalse
%<*samplepart4>
%\fi
%    \begin{macrocode}
more text in part four
%    \end{macrocode}

%\iffalse
%</samplepart4>
%\fi
%
% %%%%%%%%%%%%%%%%%%%%%%%%%%%%%%%%%%%%%%
% \paragraph{Forwarding for a Complete Draft.}
%
% The following forwarding file |cdocsdrf.tex|
% compiles the main document in draft mode:
%\iffalse
%<*sampledraft>
%\fi
%    \begin{macrocode}
\def\version{draft}
\input{childdoc.def}
\childdocforward{cdocsamp}
%    \end{macrocode}

%\iffalse
%</sampledraft>
%\fi
%
% %%%%%%%%%%%%%%%%%%%%%%%%%%%%%%%%%%%%%%
% \paragraph{Forwarding for Final Version of the Chapters.}
%
% The following forwarding files |cdocsfn1.tex| and |cdocsfn2.tex|
% (with identical content)
% compile the final versions of the child documents
% |cdocsch1.tex| and |cdocsch2.tex|, respectively:
%\iffalse
%<*samplefinal>
%\fi
%    \begin{macrocode}
\def\version{final}
\input{childdoc.def}
\childdocforwardprefix[cdocsamp]{cdocsfn}{cdocsch}
%    \end{macrocode}

%\iffalse
%</samplefinal>
%\fi
%
% %%%%%%%%%%%%%%%%%%%%%%%%%%%%%%%%%%%%%%
% \paragraph{Command Line Processing.}
%
% The following three command lines generate the output files
% |cdocscld|, |cdocscl1| and |cdocscl2|
% which should be identical to
% |cdocsdrf|, |cdocsch1| and |cdocsfn2|, respectively:
% \begin{center}
% \begin{tabular}{l}
% |latex -jobname cdocscld \|\\
% |  "\def\version{draft}\input{childdoc.def}\childdocforward{cdocsamp}"|\\
% |latex -jobname cdocscl1 \|\\
% |  "\input{childdoc.def}\childdocforward[cdocsamp]{cdocsch1}"|\\
% |latex -jobname cdocscl2 \|\\
% |  "\def\version{final}\input{childdoc.def}\childdocforward{cdocsch2}"|
% \end{tabular}
% \end{center}
% Note that the trailing backslash on each first line
% merely continues the input to the second line
% (for convenient cut ant paste).
% Furthermore, the command |latex| can be replaced by any
% of its alternative versions such as |pdflatex|.
%
% %%%%%%%%%%%%%%%%%%%%%%%%%%%%%%%%%%%%%%%%%%%%%%%%%%%%%%%%%%%%%%%%%%%%%%%%%%%%%%
% %%%%%%%%%%%%%%%%%%%%%%%%%%%%%%%%%%%%%%%%%%%%%%%%%%%%%%%%%%%%%%%%%%%%%%%%%%%%%%
% \section{Implementation}
%\iffalse
%<*package>
%\fi
%
% This section describes the definitions file |childdoc.def|.

% The definitions cannot be loaded using |\usepackage| or |\RequirePackage|
% which has a mechanism to prevent loading a style file more than once.
% When loading the definitions by means of |\input|
% multiple instances have to be prevented manually:
%\iffalse
%This code needs to be before the `\ProvidesFile' directive
%which is defined at the beginning of this file.
%Therefore it is also placed there and commented out here.
%</package>
%<*discard>
%\fi
%    \begin{macrocode}
\ifdefined\childdocmain\endinput\fi
%    \end{macrocode}
%\iffalse
%</discard>
%<*package>
%\fi
%
% \macro{\ifchilddoc}
% \macro{\ifchilddocmanual}
% The conditional |\ifchilddoc| tells whether a
% child (true) or main (false) document is being compiled.
% The conditional |\ifchilddocmanual| tells whether
% the |\includeonly| mechanism is used (false) or
% the selection of child files must be performed manually (true).
% The definitions initialise to false:
%    \begin{macrocode}
\newif\ifchilddoc
\newif\ifchilddocmanual
%    \end{macrocode}

% \macro{\childdocname}
% \macro{\childdocjob}
% The macro |\childdocname| stores the name of the main document
% to be compiled. The macro |\childdocjob| stores the name of
% the document on which the \LaTeX{} compiler was originally invoked.
% The content of |\jobname| cannot be compared
% to filenames specified in the source due to different catcodes.
% The following code rescans |\jobname|, stores the result
% in |\childdocname| and saves a copy in |\childdocjob|:
%    \begin{macrocode}
\edef\childdocname{\scantokens\expandafter{\jobname\noexpand}}
\let\childdocjob\childdocname
%    \end{macrocode}

% \macro{\childdocdisable}
% The macro |\childdocdisable| prevents the main file
% from being processed more than once.
% At this stage, the main document command |\childdocmain|
% is assumed to be called once again where it should do nothing.
% Any subsequent call to it should prevent
% a secondary processing of the main document
% It overwrites the forwarding commands
% |\childdocof| and |\childdocforward|
% with empty macros to prevent further inclusions of the main document:
%    \begin{macrocode}
\newcommand{\childdocdisable}
{
  \renewcommand{\childdocmain}[1]{\renewcommand{\childdocmain}[1]{\endinput}}
  \renewcommand{\childdocof}[1]{}
  \renewcommand{\childdocby}[2][]{}
  \renewcommand{\childdocforward}[2][]{}
  \renewcommand{\childdocdisable}{}
}
%    \end{macrocode}

% \macro{\childdocmain}
% The macro |\childdocmain| is to be called at the top of the main file
% with nothing or the main filename (without extension) as argument.
% First, it breaks loops.
% If the argument is not empty and does not match |\childdocname|
% (which is set by the first inclusion of |childdoc.def|),
% |\ifchilddoc| is set to true, |\includeonly| is applied to the child file
% and |\jobname| is set to the main file
% (for proper handling of |.aux| files):
%    \begin{macrocode}
\newcommand{\childdocmain}[1]
{
  \childdocdisable\childdocmain{}
  \if?#1?\else
    \begingroup
      \def\childdoctmp{#1}
      \ifx\childdoctmp\childdocname
        \def\childdoctmp{}
      \else
        \def\childdoctmp
        {
          \childdoctrue
          \includeonly{\childdocname}
          \def\childdocjob{#1}
          \def\jobname{#1}
        }
      \fi
      \expandafter
    \endgroup
    \childdoctmp
  \fi
}
%    \end{macrocode}

% \macro{\childdocof}
% The command |\childdocof| redirects
% compilation to the main file |#1|.
%    \begin{macrocode}
\newcommand{\childdocof}[1]
{
  \childdocdisable
  \childdoctrue
  \includeonly{\childdocname}
  \def\jobname{#1}
  \def\childdocjob{#1}
  \input{#1}
}
%    \end{macrocode}

% \macro{\childdocby}
% The command |\childdocby| ....
%    \begin{macrocode}
\newcommand{\childdocby}[2][]
{
  \childdocdisable
  \childdoctrue
  \childdocmanualtrue
  \if?#1?\else
    \def\jobname{#2}
  \fi
  \def\childdocjob{#2}
  \input{#2}
  \endinput
}
%    \end{macrocode}

% \macro{\childdocforward}
% The command |\childdocforward| redirects
% compilation to the main file or
% (if the optional argument is given) a child file.
% Parameters are set as if the main file
% or a child file starting with |\childdocof| was compiled.
% Then compilation is handed over to the main file:
%    \begin{macrocode}
\newcommand{\childdocforward}[2][]
{
  \begingroup
    \if?#1?
      \def\childdoctmp
      {
        \def\childdocname{#2}
        \def\childdocjob{#2}
        \def\jobname{#2}
        \input{#2}
        \endinput
      }
    \else
      \def\childdoctmp
      {
        \childdocdisable
        \def\childdocname{#2}
        \childdoctrue
        \includeonly{#2}
        \def\childdocjob{#1}
        \def\jobname{#1}
        \input{#1}
        \endinput
      }
    \fi
    \expandafter
  \endgroup
  \childdoctmp
}
%    \end{macrocode}

% \macro{\childdocforwardprefix}
% The command |\childdocforwardprefix| redirects
% compilation to the main or a child file by means of a pattern.
% The prefix |#1| in the current filename is replaced by |#2|
% and the suffix of the current filename is kept
% (it is assumed that the filename does not contain the substring `|~~~|'
% which is used as a delimiter).
% Compilation is handed over to the new file by |\childdocforward|:
%    \begin{macrocode}
\newcommand{\childdocforwardprefix}[3][]
{
  \begingroup
    \def\childdocextract #2##1~~~{\def\childdoctmp{\childdocforward[#1]{#3##1}}}
    \expandafter\childdocextract\childdocname~~~
    \expandafter
  \endgroup
  \childdoctmp
}
%    \end{macrocode}

% \macro{\childdoc}
% The deprecated macro |\childdoc| is a legacy version of |\childdocmain|:
%    \begin{macrocode}
\newcommand{\childdoc}{\childdocmain}
%    \end{macrocode}

% \macro{\childdocredirect}
% The deprecated macro |\childdocredirect| is a legacy version
% of |\childdocforward| and |\childdocforwardprefix|:
%    \begin{macrocode}
\newcommand{\childdocredirect}[2][]
{
  \begingroup
    \if?#1?
      \def\childdoctmp{\childdocforward{#2}}
    \else
      \def\childdoctmp{\childdocforwardprefix{#1}{#2}}
    \fi
    \expandafter
  \endgroup
  \childdoctmp
}
%    \end{macrocode}

%\iffalse
%</package>
%\fi
%
\endinput
\childdocforward[|\textit{main}|]{|\textit{dest}|}"|
\end{center}
%
Here \textit{target} is the name of the output file,
\textit{main} is the name of the main file
and \textit{dest} is the name of the main or child file to be processed
(all filenames without extensions).
The optional argument \textit{main} can be omitted
if \textit{main} matches \textit{dest}.
Optionally, compilation \textit{flags} can be defined via |\def| commands.
This command line makes the \TeX{} engine believe
it is compiling the file \textit{target}
whose content is specified as the latter parameter.
The provided code then forwards the processing to
\textit{main} or \textit{dest} as described in \secref{sec:forward}.

%%%%%%%%%%%%%%%%%%%%%%%%%%%%%%%%%%%%%%%%%%%%%%%%%%%%%%%%%%%%%%%%%%%%%%%%%%%%%%%%
\subsection{Include by Input}
\label{sec:input}

Including child documents by |\include| has some restrictions by design.
Most notably, the content of a child document always occupies
its own set of pages; pages cannot be shared between child documents.
Usually, this behaviour makes perfect sense
because each child document contain an essential part of the document.
However, in some situations it may be desirable to compose
a document from a collection of parts
without having mandatory page breaks between then.
For this case, the package
provides a mechanism to include parts
by |\input| which can also be processed individually.
However, by construction this mechanism
requires manual handling of the content to be output.

%%%%%%%%%%%%%%%%%%%%%%%%%%%%%%%%%%%%%%%%
\DescribeMacro{\ifchilddocmanual}
The main file should be prepared as usual, see \secref{sec:include}.
However, the document body must make a distinction
between processing of an individual part and of the main document, e.g.:
%
\begin{center}
\begin{tabular}{l}
|\ifchilddocmanual|\\
|\input{\childdocname}|\\
|\||else|\\
\textit{document body with }|\input{|\textit{part}|}|\\
|\||fi|
\end{tabular}
\end{center}
%
The conditional |\ifchilddocmanual| is true whenever
a part to be included by |\input| is being compiled,
and the name of the part is stored in |\childdocname|.

%%%%%%%%%%%%%%%%%%%%%%%%%%%%%%%%%%%%%%%%
\DescribeMacro{\childdocby}
Each part to be included by |\input| should start with:
%
\begin{center}
\begin{tabular}{l}
|% \iffalse
%
% childdoc.dtx Copyright (C) 2017-2018 Niklas Beisert
%
% This work may be distributed and/or modified under the
% conditions of the LaTeX Project Public License, either version 1.3
% of this license or (at your option) any later version.
% The latest version of this license is in
%   http://www.latex-project.org/lppl.txt
% and version 1.3 or later is part of all distributions of LaTeX
% version 2005/12/01 or later.
%
% This work has the LPPL maintenance status `maintained'.
%
% The Current Maintainer of this work is Niklas Beisert.
%
% This work consists of the files childdoc.dtx and childdoc.ins
% and the derived files childdoc.def and cdocsamp.tex with
% cdocsch1.tex, cdocsch2.tex, cdocsdrf.tex, cdocsfn1.tex, cdocsfn2.tex.
%
%<package>\ifdefined\childdocmain\endinput\fi
%<package>\ProvidesFile{childdoc.def}[2018/12/30 v2.0 child document driver]
%<samplemain>\ProvidesFile{cdocsamp.tex}[2018/12/30 v2.0 sample for childdoc]
%<*driver>
%\ProvidesFile{childdoc.drv}[2018/12/30 v2.0 childdoc reference manual file]
\PassOptionsToClass{10pt,a4paper}{article}
\documentclass{ltxdoc}

\usepackage[margin=35mm]{geometry}
\usepackage{hyperref}
\usepackage{hyperxmp}
\usepackage[usenames]{color}

\hypersetup{colorlinks=true}
\hypersetup{pdfstartview=FitH}
\hypersetup{pdfpagemode=UseNone}
\hypersetup{pdfsource={}}
\hypersetup{pdflang={en-UK}}
\hypersetup{pdfcopyright={Copyright 2017-2018 Niklas Beisert.
  This work may be distributed and/or modified under the
  conditions of the LaTeX Project Public License, either version 1.3
  of this license or (at your option) any later version.}}
\hypersetup{pdflicenseurl={http://www.latex-project.org/lppl.txt}}
\hypersetup{pdfcontactaddress={ETH Zurich, ITP, HIT K,
  Wolfgang-Pauli-Strasse 27}}
\hypersetup{pdfcontactpostcode={8093}}
\hypersetup{pdfcontactcity={Zurich}}
\hypersetup{pdfcontactcountry={Switzerland}}
\hypersetup{pdfcontactemail={nbeisert@itp.phys.ethz.ch}}
\hypersetup{pdfcontacturl={http://people.phys.ethz.ch/\xmptilde nbeisert/}}

\newcommand{\secref}[1]{\hyperref[#1]{section \ref*{#1}}}

\parskip1ex
\parindent0pt
\let\olditemize\itemize
\def\itemize{\olditemize\parskip0pt}

\begin{document}

\title{The \textsf{childdoc} Package}
\hypersetup{pdftitle={The childdoc Package}}
\author{Niklas Beisert\\[2ex]
  Institut f\"ur Theoretische Physik\\
  Eidgen\"ossische Technische Hochschule Z\"urich\\
  Wolfgang-Pauli-Strasse 27, 8093 Z\"urich, Switzerland\\[1ex]
  \href{mailto:nbeisert@itp.phys.ethz.ch}
  {\texttt{nbeisert@itp.phys.ethz.ch}}}
\hypersetup{pdfauthor={Niklas Beisert}}
\hypersetup{pdfsubject={Manual for the LaTeX2e Package childdoc}}
\date{30 December 2018, \textsf{v2.0}}
\maketitle

\begin{abstract}\noindent
\textsf{childdoc} is a \LaTeXe{} package
that enables the direct compilation
of document sections included by |\include|
to individual files.
\end{abstract}

\begingroup
\parskip0ex
\tableofcontents
\endgroup

%%%%%%%%%%%%%%%%%%%%%%%%%%%%%%%%%%%%%%%%%%%%%%%%%%%%%%%%%%%%%%%%%%%%%%%%%%%%%%%%
%%%%%%%%%%%%%%%%%%%%%%%%%%%%%%%%%%%%%%%%%%%%%%%%%%%%%%%%%%%%%%%%%%%%%%%%%%%%%%%%
\section{Introduction}

\LaTeX{} provides a mechanism to structure a large document (such as a book)
into a main file and several child files (containing the chapters)
using the |\include| command.
This mechanism is beneficial for documents
which span hundreds of pages in order to
make the source file(s) more manageable.
Moreover, compilation can be restricted to
selected child files by means of the |\includeonly| command.
The latter feature can be used to reduce the compilation time while editing
(this was significantly more useful in the earlier days of \LaTeX{})
or to generate a smaller document which is easier to navigate.
Another application of |\includeonly| is to generate
documents consisting of selected parts of the complete document.

However, there are a few drawbacks of the plain |\include| mechanism:
\begin{itemize}
\item
The child files cannot be compiled on their own,
they can only be compiled via the main file.
A naive editing environment
(such as a text editor with an option
to have the current file processed by \LaTeX)
may require one to switch to the main file before compiling;
attempting to compile the child file produces errors.
\item
The main file must be modified (each time)
to adjust the |\includeonly| command
to the present needs. This easily leaves the main file in a messy state.
\item
The generated document will always carry the filename
of the main document. This is inconvenient if
several child files are to be compiled and
to be kept for distribution.
\end{itemize}

The present package provides a simple interface
to make child files individually compilable by \LaTeX{}.
Compiling a child file then has the same effect as compiling
the main file with an |\includeonly| command
to select the appropriate child.
Moreover the generated document will carry the name of the child
rather than the main file.
This resolves all three above issues.

This feature is meant to make the editing of books,
thesis documents and lecture notes somewhat more convenient.
However, the package can also be used efficiently for
composing a series of documents (such as exercise sheets)
which are typically distributed individually.
It then assists the author in generating the individual documents
(potentially in different versions)
as well as a document containing the collected series.
Another application is in developing style files
or other kinds of included material
where compilation of the style file could redirect
to a sample or test file.

%%%%%%%%%%%%%%%%%%%%%%%%%%%%%%%%%%%%%%%%%%%%%%%%%%%%%%%%%%%%%%%%%%%%%%%%%%%%%%%%
%%%%%%%%%%%%%%%%%%%%%%%%%%%%%%%%%%%%%%%%%%%%%%%%%%%%%%%%%%%%%%%%%%%%%%%%%%%%%%%%
\section{Usage}

First of all, the package \textsf{childdoc} is \emph{not} a standard
\LaTeXe{} |.sty| style file! Therefore it needs to be invoked in
a non-standard way.

%%%%%%%%%%%%%%%%%%%%%%%%%%%%%%%%%%%%%%%%%%%%%%%%%%%%%%%%%%%%%%%%%%%%%%%%%%%%%%%%
\subsection{Included Files}
\label{sec:include}

%%%%%%%%%%%%%%%%%%%%%%%%%%%%%%%%%%%%%%%%
\DescribeMacro{\childdocmain}
To use the package, add the commands
\begin{center}
\begin{tabular}{l}
|\input{childdoc.def}|\\
|\childdocmain{}|\\
\end{tabular}
\end{center}
at the very top of the main \LaTeX{} file,
in particular \emph{before} the |\documentclass| statement!
The argument of |\childdocmain| should be left empty
(but it must be present).

%%%%%%%%%%%%%%%%%%%%%%%%%%%%%%%%%%%%%%%%
\DescribeMacro{\childdocof}
Furthermore, add the commands
\begin{center}
\begin{tabular}{l}
|\input{childdoc.def}|\\
|\childdocof{|\textit{main}|}|\\
\end{tabular}
\end{center}
at the top of every child file \textit{child}
which is included by |\include{|\textit{child}|}|
from within the main file
(or at least for those files to be compiled individually).
The argument \textit{main} must be the filename of the main file.

There are a couple of
considerations in setting up the main and child documents:

%%%%%%%%%%%%%%%%%%%%%%%%%%%%%%%%%%%%%%%%
\paragraph{Restrictions.}

Please note the following restrictions:
\begin{itemize}
\item
|\childdocmain| must be called with one argument \textit{main}
to ensure compatibility with earlier version of the package.
It must either be empty (|\childdocmain{}|)
or precisely match the filename of the main file in which it is specified.
See \secref{sec:detection} for further information.
\item
The filename \textit{main} must be specified without the |.tex| extension.
\item
The filename \textit{main} is case sensitive
(even in case-insensitive file systems)
due to internal string comparison.
\item
The argument \textit{main} should be fully expanded, it cannot be a macro.
\item
Subdirectories and special characters should be avoided in filenames.
\item
The command |\childdocmain{|\textit{main}|}| must be followed by a whitespace.
It should not be followed immediately by another command
or by a comment mark `|%|'.
This is because the \TeX{} parser reads the token immediately following
the argument of |\childdocmain| and puts it
at the beginning of every child section;
however, a white\-space is ignored.
\end{itemize}

%%%%%%%%%%%%%%%%%%%%%%%%%%%%%%%%%%%%%%%%
\paragraph{Content of Main File.}

It is advisable to place all content in the child files included by |\include|.
Any output contained in the main file will appear in all child documents
unless suppressed manually;
it cannot be suppressed automatically by the |\includeonly| directive
and thus should normally be avoided.
A method to include some content in the main file
by means of conditional processing is described in \secref{sec:conditional}.

%%%%%%%%%%%%%%%%%%%%%%%%%%%%%%%%%%%%%%%%
\paragraph{Page Numbering.}

When only a part of the document is compiled,
the appropriate numbering of pages
(as well as other status parameters)
is determined from the |.aux| files.
The latter contain information from previous passes.
However this information needs to propagate through
all intermediate child documents.
Therefore the page numbering in child documents may well
be inconsistent until the complete document is compiled at least once.

A useful (if unconventional) way to always ensure a consistent
page numbering is to restart the numbering in each child document
and denote the pages by `\textit{child}|.|\textit{page}'
where \textit{child} represents the chapter/section number of the child file.
This can be achieved by the command
|\numberwithin{page}{|\textit{child}|}|
of the \textsf{amsmath} package
where \textit{child} can be |chapter| or |section|
depending on the chosen structuring.
Alternatively, one can modify the macro |\thepage| appropriately
and reset the counter |page| at the start of each child file.

%%%%%%%%%%%%%%%%%%%%%%%%%%%%%%%%%%%%%%%%%%%%%%%%%%%%%%%%%%%%%%%%%%%%%%%%%%%%%%%%
\subsection{Conditional Processing}
\label{sec:conditional}

The package provides a mechanism to compile different versions
of a document. To customise the versions further some conditional processing
can come in handy to distinguish which version is being compiled.
The package provides two macros to describe the compilation context:

%%%%%%%%%%%%%%%%%%%%%%%%%%%%%%%%%%%%%%%%
\DescribeMacro{\ifchilddoc}
The conditional |\ifchilddoc| distinguishes between the compilation of
child documents and the main document:
%
\begin{center}
|\ifchilddoc |\textit{child-code}| |[|\||else |\textit{main-code}]| \||fi|
\end{center}

%%%%%%%%%%%%%%%%%%%%%%%%%%%%%%%%%%%%%%%%
\DescribeMacro{\childdocname}
\DescribeMacro{\childdocjob}
The macro |\childdocname| contains the filename (without extension)
of the main or child file being processed.
Note that |\childdocjob| will always contain the name of the main file.

%%%%%%%%%%%%%%%%%%%%%%%%%%%%%%%%%%%%%%%%
\paragraph{Title Page.}

Conditional processing can be used to include a title or banner page
in the main document when proper precautions are taken.
Importantly, the code in the main file should ensure that the page counter
(as well as other status parameters which are stored in the |.aux| files)
takes the same value after the conditional processing.
Otherwise the page numbers may take divergent values
depending on which part is compiled.

For example, a title page could be declared by:
%
\begin{center}
\begin{tabular}{l}
|\ifchilddoc\||else|\\
|\addtocounter{page}{-1}|\\
\textit{code for title page}\\
|\newpage|\\
|\||fi|
\end{tabular}
\end{center}
%
A banner page for the child documents can be generated by:
%
\begin{center}
\begin{tabular}{l}
|\ifchilddoc|\\
|\addtocounter{page}{-1}|\\
\textit{code for banner page}\\
|\newpage|\\
|\||fi|
\end{tabular}
\end{center}
%
Here one could write a message such as:
\begin{center}
|This is the part \childdocname{} of \childdocjob{}.|
\end{center}

%%%%%%%%%%%%%%%%%%%%%%%%%%%%%%%%%%%%%%%%%%%%%%%%%%%%%%%%%%%%%%%%%%%%%%%%%%%%%%%%
\subsection{Flags}
\label{sec:flags}

The package makes it easy to generate different versions
of the main or child documents.
To this end compilation flags can be defined
and assigned different default values.
They will be particularly useful in conjunction
with the forwarding mechanism described in \secref{sec:forward}.

For example, it may be useful to have a flag |\version|
which can be set to |draft| or |final|.
The document source will contain some conditional code
depending on the value of |\version|.
Suppose further, the flag should default to |final| for the main file
and to |draft| for child files
which is a natural assignment for editing the document.
This is achieved by placing the following code
in the preamble of the main document
(below the |\childdocmain| directive):
%
\begin{center}
\begin{tabular}{l}
|\ifchilddoc|\\
|\providecommand{\version}{draft}|\\
|\||else|\\
|\providecommand{\version}{final}|\\
|\||fi|
\end{tabular}
\end{center}
%
The definition by |\providecommand| makes sure
that previous definitions are not overwritten.
Further statements |\providecommand{\version}{...}|
can thus be added before the above code to override it.

For the main file, one might add a line
(between |\childdocmain| and the above block)
%
\begin{center}
|%\ifchilddoc\||else\providecommand{\version}{draft}\||fi|
\end{center}
%
which can be uncommented to produce a draft version.
Likewise one can add a line to the very top of a child file
(above the |\childdocof{|\textit{main}|}| directive)
%
\begin{center}
|%\providecommand{\version}{final}|
\end{center}
%
which can be uncommented to produce the final version of this child document.

%%%%%%%%%%%%%%%%%%%%%%%%%%%%%%%%%%%%%%%%%%%%%%%%%%%%%%%%%%%%%%%%%%%%%%%%%%%%%%%%
\subsection{Forwarding}
\label{sec:forward}

Different versions of the main or child documents
using compilation flags as described in \secref{sec:flags}
can be (permanently) stored in different files
for convenient compilation, viewing and distribution.
To this end, the package defines a command
to pass on compilation to a different file:

%%%%%%%%%%%%%%%%%%%%%%%%%%%%%%%%%%%%%%%%
\DescribeMacro{\childdocforward}
The command |\childdocforward| redirects processing to
another source file:
%
\begin{center}
\begin{tabular}{l}
|\input{childdoc.def}|\\
|\childdocforward[|\textit{main}|]{|\textit{dest}|}|\\
\end{tabular}
\end{center}
%
The argument \textit{dest} is the destination file
(without extension).
It should be the main file or one of the child files.
Note that further \textsf{childdoc} directives
such as |\childdocof| and |\childdocforward|
in the indicated file will be processed in this form.
The optional argument \textit{main}
passes on directly to the main file \textit{main}
while pretending to compile the child \textit{dest}.
This form behaves as if \textit{dest}
issues |\childdocof{|\textit{main}|}| right away,
and no further \textsf{childdoc} directives will be processed.

%%%%%%%%%%%%%%%%%%%%%%%%%%%%%%%%%%%%%%%%
\DescribeMacro{\...prefix}
In the alternative form |\childdocforwardprefix|,
%
\begin{center}
\begin{tabular}{l}
|\input{childdoc.def}|\\
|\childdocforwardprefix[|\textit{main}|]{|\textit{prefix}|}{|\textit{dest}|}|
\end{tabular}
\end{center}
%
the destination file is determined by a pattern
depending on the current file:
To make this work, the current file must be called
`{\textit{prefix}\hspace{0.2em}\textit{suffix}}'
with \textit{prefix} matching precisely the argument.
Processing is then passed on to the file
`{\textit{dest}\hspace{0.2em}\textit{suffix}}'.
Surely, the same effect is achieved by
directly specifying the
argument `{\textit{dest}\hspace{0.2em}\textit{suffix}}'
in the first form.
However, that requires to set up a different file
for each child. With the alternative form of the command
all these files can have exactly the same content
which simplifies setting them up and maintaining them.

For example, the following file |draft.tex|
with a compilation flag |\version| as described in \secref{sec:flags}
compiles the main document as a draft:
%
\begin{center}
\begin{tabular}{l}
|\def\version{draft}|\\
|\input{childdoc.def}|\\
|\childdocforward{|\textit{main}|}|
\end{tabular}
\end{center}
%
Likewise, the following files |final|\textit{nn}|.tex|
compile the final version of the child document
|child|\textit{nn}|.tex|:
%
\begin{center}
\begin{tabular}{l}
|\def\version{final}|\\
|\input{childdoc.def}|\\
|\childdocforwardprefix{final}{child}|
\end{tabular}
\end{center}
%

Note that when several versions of a main file and/or of each child file
are to be generated, it may be convenient to set up a |Makefile| or
shell script to automatise the process.

%%%%%%%%%%%%%%%%%%%%%%%%%%%%%%%%%%%%%%%%%%%%%%%%%%%%%%%%%%%%%%%%%%%%%%%%%%%%%%%%
\subsection{Command Line Processing}
\label{sec:commandline}

The effect of redirection files can also be achieved by invoking
the \LaTeX{} compiler with a more elaborate command line.
Most conveniently this should be done as part
of a shell script or a |Makefile|.

When using \textsf{childdoc} in the main file, the following
command lines effectively perform a redirection
(note that depending on the shell being used,
backslashes may have to be doubled: `|\|' $\to$ `|\\|'):
%
\begin{center}
|... -jobname "|\textit{target}|" |\\|"|[\textit{flags}]%
|\input{childdoc.def}\childdocforward[|\textit{main}|]{|\textit{dest}|}"|
\end{center}
%
Here \textit{target} is the name of the output file,
\textit{main} is the name of the main file
and \textit{dest} is the name of the main or child file to be processed
(all filenames without extensions).
The optional argument \textit{main} can be omitted
if \textit{main} matches \textit{dest}.
Optionally, compilation \textit{flags} can be defined via |\def| commands.
This command line makes the \TeX{} engine believe
it is compiling the file \textit{target}
whose content is specified as the latter parameter.
The provided code then forwards the processing to
\textit{main} or \textit{dest} as described in \secref{sec:forward}.

%%%%%%%%%%%%%%%%%%%%%%%%%%%%%%%%%%%%%%%%%%%%%%%%%%%%%%%%%%%%%%%%%%%%%%%%%%%%%%%%
\subsection{Include by Input}
\label{sec:input}

Including child documents by |\include| has some restrictions by design.
Most notably, the content of a child document always occupies
its own set of pages; pages cannot be shared between child documents.
Usually, this behaviour makes perfect sense
because each child document contain an essential part of the document.
However, in some situations it may be desirable to compose
a document from a collection of parts
without having mandatory page breaks between then.
For this case, the package
provides a mechanism to include parts
by |\input| which can also be processed individually.
However, by construction this mechanism
requires manual handling of the content to be output.

%%%%%%%%%%%%%%%%%%%%%%%%%%%%%%%%%%%%%%%%
\DescribeMacro{\ifchilddocmanual}
The main file should be prepared as usual, see \secref{sec:include}.
However, the document body must make a distinction
between processing of an individual part and of the main document, e.g.:
%
\begin{center}
\begin{tabular}{l}
|\ifchilddocmanual|\\
|\input{\childdocname}|\\
|\||else|\\
\textit{document body with }|\input{|\textit{part}|}|\\
|\||fi|
\end{tabular}
\end{center}
%
The conditional |\ifchilddocmanual| is true whenever
a part to be included by |\input| is being compiled,
and the name of the part is stored in |\childdocname|.

%%%%%%%%%%%%%%%%%%%%%%%%%%%%%%%%%%%%%%%%
\DescribeMacro{\childdocby}
Each part to be included by |\input| should start with:
%
\begin{center}
\begin{tabular}{l}
|\input{childdoc.def}|\\
|\childdocby{|\textit{main}|}|\\
\end{tabular}
\end{center}
%
The directive |\childdocby| is similar to |\childdocof|
described in \secref{sec:include},
but the subsequent selection of content must be done manually.
To that end, both |\ifchilddoc| and |\ifchilddocmanual|
will be true upon processing of a part,
and the name of the part is stored in |\childdocname|.
Note that |\jobname| will be set to the filename of the current part
so that each part receives an individual |.aux| file
that does not interfere with the |.aux| file(s) of the main document.
This behaviour can be altered by the alternative form
|\childdocby[*]{|\textit{main}|}| (with a non-empty optional argument)
which uses the |.aux| file of the main document
by setting |\jobname| to \textit{main}.

%%%%%%%%%%%%%%%%%%%%%%%%%%%%%%%%%%%%%%%%%%%%%%%%%%%%%%%%%%%%%%%%%%%%%%%%%%%%%%%%
\subsection{Driver Development}
\label{sec:driver}

The \textsf{childdoc} mechanism can also be use for the development
of definition files such as \LaTeX{} styles or classes.
This case differs from the above setup with multiple parts
included by |\include| in that no |\includeonly| should be invoked.
This can be achieved by starting the include file
(before |\ProvidesPackage|) with:
%
\begin{center}
\begin{tabular}{l}
|\input{childdoc.def}|\\
|\childdocforward{|\textit{main}|}|\\
\end{tabular}
\end{center}
%
or alternatively with:
%
\begin{center}
\begin{tabular}{l}
|\input{childdoc.def}|\\
|\childdocby{|\textit{main}|}|\\
\end{tabular}
\end{center}
%
Both forms have slightly different effects as described above.
The main file is prepared as usual, see \secref{sec:include}.

%%%%%%%%%%%%%%%%%%%%%%%%%%%%%%%%%%%%%%%%%%%%%%%%%%%%%%%%%%%%%%%%%%%%%%%%%%%%%%%%
\subsection{Legacy Detection}
\label{sec:detection}

The directive |\childdocmain| in the main file can detect
whether the complete document or merely a child is to be compiled
even without using the directive |\childdocof|.
This method is deprecated because it is less robust
and there is no compelling reason to use it;
it is merely provided for backward compatibility
and it may be removed in future versions.

If the detection mechanism is to be used,
it is mandatory to correctly specify
the filename of the main file as the argument of |\childdocmain|:
%
\begin{center}
\begin{tabular}{l}
|\input{childdoc.def}|\\
|\childdocmain{|\textit{main}|}|\\
\end{tabular}
\end{center}
%
If |\jobname| does not match the argument \textit{main} of |\childdocmain|,
it is assumed that |\jobname| points to the child file to be compiled.
When using |\childdocmain| with the main file specified as argument,
it suffices to start a child file
with just |\input{|\textit{main}|}|
without loading of the package and using |\childdocof|.
If instead all processing is done
with the appropriate \textsf{childdoc} directives,
the argument of \textit{main} of |\childdocmain| can be empty.

An alternative version of the command line processing described
in \secref{sec:commandline} using the detection mechanism reads:
%
\begin{center}
|... -jobname "|\textit{target}|" "|[\textit{flags}]%
[|\def\jobname{|\textit{dest}|}|]|\input{|\textit{main}|}"|
\end{center}

%%%%%%%%%%%%%%%%%%%%%%%%%%%%%%%%%%%%%%%%%%%%%%%%%%%%%%%%%%%%%%%%%%%%%%%%%%%%%%%%
\subsection{Manual Code}
\label{sec:manual}

In case one cannot be certain whether the definitions file |childdoc.def|
is installed on the target \TeX{} distribution
and one prefers not to ship it,
it is conceivable to paste a few relevant commands into the sources.

To that end, drop all statements |\input{childdoc.def}|
and perform the replacements as outlined below.
Instead of |\childdocmain{|\textit{main}|}| add the following code
to the top of the main file:
%
\begin{center}
\begin{tabular}{l}
|\||ifdefined\childdocname\endinput\||fi\newif\ifchilddoc|\\
|\edef\childdocname{\scantokens\expandafter{\jobname\noexpand}}|\\
|\def\childdocmain{|\textit{main}|}\||ifx\childdocmain\childdocname\||else|\\
|\childdoctrue\includeonly{\childdocname}\let\jobname\childdocmain\||fi|\\
\end{tabular}
\end{center}
%
Instead of |\childdocof{|\textit{main}|}| just include the main file
at the top of each child file:
%
\begin{center}
|\input{|\textit{main}|}|
\end{center}
%
A simple redirection |\childdocforward{|\textit{dest}|}| is achieved by:
%
\begin{center}
|\def\jobname{|\textit{dest}|}\input{\jobname}|
\end{center}
%
The redirection with prefix
|\childdocforwardprefix[|\textit{prefix}|]{|\textit{dest}|}|
is accomplished by:
%
\begin{center}
\begin{tabular}{l}
|{\edef\jobname{\scantokens\expandafter{\jobname\noexpand}}|\\
|\def\redirectjob |\textit{prefix}|#1~~~{\gdef\jobname{|\textit{dest}|#1}}|\\
|\expandafter\redirectjob\jobname~~~}\input{\jobname}|
\end{tabular}
\end{center}

In an alternative approach,
child documents can be compiled by a specific command line
without additional code or specific definitions:
%
\begin{center}
|... -jobname "|\textit{target}|" "|[\textit{flags}]%
|\includeonly{|\textit{dest}|}\input{|\textit{main}|}"|
\end{center}
%

%%%%%%%%%%%%%%%%%%%%%%%%%%%%%%%%%%%%%%%%%%%%%%%%%%%%%%%%%%%%%%%%%%%%%%%%%%%%%%%%
%%%%%%%%%%%%%%%%%%%%%%%%%%%%%%%%%%%%%%%%%%%%%%%%%%%%%%%%%%%%%%%%%%%%%%%%%%%%%%%%
\section{Information}

%%%%%%%%%%%%%%%%%%%%%%%%%%%%%%%%%%%%%%%%%%%%%%%%%%%%%%%%%%%%%%%%%%%%%%%%%%%%%%%%
\subsection{Copyright}

Copyright \copyright{} 2017--2018 Niklas Beisert

This work may be distributed and/or modified under the
conditions of the \LaTeX{} Project Public License, either version 1.3
of this license or (at your option) any later version.
The latest version of this license is in
  \url{http://www.latex-project.org/lppl.txt}
and version 1.3 or later is part of all distributions of \LaTeX{}
version 2005/12/01 or later.

This work has the LPPL maintenance status `maintained'.

The Current Maintainer of this work is Niklas Beisert.

This work consists of the files |README.txt|, |childdoc.ins| and |childdoc.dtx|
as well as the derived files |childdoc.def|, |cdocsamp.tex|
with |cdocsch1.tex|, |cdocsch2.tex|, |cdocspt3.tex|, |cdocspt4.tex|,
|cdocsdrf.tex|, |cdocsfn1.tex|, |cdocsfn2.tex|
as well as |childdoc.pdf|.

%%%%%%%%%%%%%%%%%%%%%%%%%%%%%%%%%%%%%%%%%%%%%%%%%%%%%%%%%%%%%%%%%%%%%%%%%%%%%%%%
\subsection{Files and Installation}

The package consists of the files:
%
\begin{center}
\begin{tabular}{ll}
    |README.txt|   & readme file \\
    |childdoc.ins| & installation file \\
    |childdoc.dtx| & source file \\
    |childdoc.def| & definition file \\
    |cdocsamp.tex| & sample main file \\
    |cdocsch1.tex| & sample include file \\
    |cdocsch2.tex| & sample include file \\
    |cdocspt3.tex| & sample part file \\
    |cdocspt4.tex| & sample part file \\
    |cdocsdrf.tex| & sample redirection file \\
    |cdocsfn1.tex| & sample redirection file \\
    |cdocsfn2.tex| & sample redirection file \\
    |childdoc.pdf| & manual
\end{tabular}
\end{center}
%
The distribution consists of the files
|README.txt|, |childdoc.ins| and |childdoc.dtx|.
%
\begin{itemize}
\item
Run (pdf)\LaTeX{} on |childdoc.dtx|
to compile the manual |childdoc.pdf| (this file).
\item
Run \LaTeX{} on |childdoc.ins| to create the definitions file |childdoc.def|
and the sample |cdocsamp.tex| with include files
|cdocsch1.tex|, |cdocsch2.tex|, |cdocspt3.tex|, |cdocspt4.tex|,
|cdocsdrf.tex|, |cdocsfn1.tex|, |cdocsfn2.tex|.
Then copy the file |childdoc.def| to an appropriate directory of your \LaTeX{}
distribution, e.g.\ \textit{texmf-root}|/tex/latex/childdoc|.
\end{itemize}

%%%%%%%%%%%%%%%%%%%%%%%%%%%%%%%%%%%%%%%%%%%%%%%%%%%%%%%%%%%%%%%%%%%%%%%%%%%%%%%%
\subsection{Related CTAN Packages}

There are several other packages which offer a similar functionality:
%
\begin{itemize}
\item
The packages
\href{http://ctan.org/pkg/docmute}{\textsf{docmute}},
\href{http://ctan.org/pkg/includex}{\textsf{includex}} and
\href{http://ctan.org/pkg/standalone}{\textsf{standalone}}
provide commands to include only the document body of
a child file thus allowing both files to be compiled individually.
\item
The packages \href{http://ctan.org/pkg/subdocs}{\textsf{subdocs}}
and \href{http://ctan.org/pkg/subfiles}{\textsf{subfiles}}
provide structures in which the main and child documents can be
encapsulated and allowing them to be compiled individually.
The inclusion mechanism is different from the conventional |\include|.
\item
The package \href{http://ctan.org/pkg/combine}{\textsf{combine}}
is an elaborate solution to combine several documents into one.
\end{itemize}
%
See also the CTAN topic \href{http://ctan.org/topic/subdocs}{\textsf{subdocs}}
for further related packages.
The present package differs from the above solutions in that
a document structure constructed with the conventional |\include| mechanism
just needs two extra commands at the top of every file
such that all constituent files can be compiled individually.

%%%%%%%%%%%%%%%%%%%%%%%%%%%%%%%%%%%%%%%%%%%%%%%%%%%%%%%%%%%%%%%%%%%%%%%%%%%%%%%%
%\subsection{Feature Suggestions}
%
%The following is a list of features which may be useful for future
%versions of this package:
%%
%\begin{itemize}
%\item
%\ldots
%\end{itemize}

%%%%%%%%%%%%%%%%%%%%%%%%%%%%%%%%%%%%%%%%%%%%%%%%%%%%%%%%%%%%%%%%%%%%%%%%%%%%%%%%
\subsection{Revision History}

%%%%%%%%%%%%%%%%%%%%%%%%%%%%%%%%%%%%%%%%
\paragraph{v2.0:} 2018/12/30

\begin{itemize}
\item
immediate forward processing
\item
added |\childdocby| mechanism
\item
manual restructured
\end{itemize}

%%%%%%%%%%%%%%%%%%%%%%%%%%%%%%%%%%%%%%%%
\paragraph{v1.6:} 2018/01/17

\begin{itemize}
\item
application for development of include files
\item
corrections to manual
\end{itemize}

%%%%%%%%%%%%%%%%%%%%%%%%%%%%%%%%%%%%%%%%
\paragraph{v1.5:} 2017/05/21

\begin{itemize}
\item
more complete structuring introduced
\item
|\childdocof| introduced
\item
|\childdoc| renamed to |\childdocmain|
\item
|\childredirect| renamed to |\childdocforward| and |\childdocforwardprefix|
and functionality expanded
\end{itemize}

%%%%%%%%%%%%%%%%%%%%%%%%%%%%%%%%%%%%%%%%
\paragraph{v1.0:} 2017/04/27

\begin{itemize}
\item
manual and install package
\item
first version published on CTAN
\end{itemize}

%%%%%%%%%%%%%%%%%%%%%%%%%%%%%%%%%%%%%%%%
\paragraph{v0.6:} 2017/04/26

\begin{itemize}
\item
redirection mechanism added
\end{itemize}

%%%%%%%%%%%%%%%%%%%%%%%%%%%%%%%%%%%%%%%%
\paragraph{v0.5:} 2017/04/26

\begin{itemize}
\item
functionality in definition file
\end{itemize}


%%%%%%%%%%%%%%%%%%%%%%%%%%%%%%%%%%%%%%%%%%%%%%%%%%%%%%%%%%%%%%%%%%%%%%%%%%%%%%%%
%%%%%%%%%%%%%%%%%%%%%%%%%%%%%%%%%%%%%%%%%%%%%%%%%%%%%%%%%%%%%%%%%%%%%%%%%%%%%%%%
%%%%%%%%%%%%%%%%%%%%%%%%%%%%%%%%%%%%%%%%%%%%%%%%%%%%%%%%%%%%%%%%%%%%%%%%%%%%%%%%
\appendix

\settowidth\MacroIndent{\rmfamily\scriptsize 000\ }

 \DocInput{childdoc.dtx}

\end{document}
%</driver>
% \fi
%
% %%%%%%%%%%%%%%%%%%%%%%%%%%%%%%%%%%%%%%%%%%%%%%%%%%%%%%%%%%%%%%%%%%%%%%%%%%%%%%
% %%%%%%%%%%%%%%%%%%%%%%%%%%%%%%%%%%%%%%%%%%%%%%%%%%%%%%%%%%%%%%%%%%%%%%%%%%%%%%
% \section{Sample}
%\iffalse
%<*samplemain>
%\fi
%
% The following presents a sample document
% with two chapters, two parts, a title page,
% a compile flag as well as three forwarding files to set the flag.
% It consists of eight |.tex| files:
% \begin{center}
% \begin{tabular}{ll}
% |cdocsamp.tex|&main file\\
% |cdocsch1.tex|&include file for chapter 1\\
% |cdocsch2.tex|&include file for chapter 2\\
% |cdocspt3.tex|&include file for part 3\\
% |cdocspt4.tex|&include file for part 4\\
% |cdocsdrf.tex|&forwarding file for main file in draft mode\\
% |cdocsfi1.tex|&forwarding file for final version of chapter 1\\
% |cdocsfi2.tex|&forwarding file for final version of chapter 2\\
% \end{tabular}
% \end{center}
% Each of the eight files can be compiled directly by the \LaTeX{} compiler.
%
% %%%%%%%%%%%%%%%%%%%%%%%%%%%%%%%%%%%%%%
% \paragraph{Main File.}
%
% The main file is called |cdocsamp.tex|.
%
% Load the \textsf{childdoc} definitions and
% declare the filename for the main document:
%    \begin{macrocode}
\input{childdoc.def}
\childdocmain{}
%    \end{macrocode}

% Optional override for |\version| flag:
%    \begin{macrocode}
%%\ifchilddoc\else\providecommand{\version}{draft}\fi
%    \end{macrocode}

% Define the default values for the |\version| flag
% (|final| for the main file and |draft| for childs):
%    \begin{macrocode}
\ifchilddoc
\providecommand{\version}{draft}
\else
\providecommand{\version}{final}
\fi
%    \end{macrocode}

% Load the standard document class:
%    \begin{macrocode}
\documentclass[12pt]{article}
%    \end{macrocode}

% Start the document body:
%    \begin{macrocode}
\begin{document}
%    \end{macrocode}

% Declare a title page.
% Print title, part of document being processed and version flag:
%    \begin{macrocode}
\addtocounter{page}{-1}
\begin{center}
{\LARGE\bfseries{}childdoc example\par}
\vspace{1cm}
\ifchilddoc
\ifchilddocmanual part\else chapter\fi:
`\childdocname' of `\childdocjob'\par
\else
main document: `\childdocjob'\par
\fi
version: \version\par
\end{center}
\newpage
%    \end{macrocode}

% Manually include selected file,
% otherwise process as usual:
%    \begin{macrocode}
\ifchilddocmanual
\section*{part `\childdocname'}
\input{\childdocname}
\else
%    \end{macrocode}

% Include the two chapters:
%    \begin{macrocode}
\include{cdocsch1}
\include{cdocsch2}
%    \end{macrocode}

% Include the two parts unless only chapters should be displayed:
%    \begin{macrocode}
\ifchilddoc\else
\section{part three}
\input{cdocspt3}
\section{part four}
\input{cdocspt4}
\fi
%    \end{macrocode}

% Process as usual until here:
%    \begin{macrocode}
\fi
%    \end{macrocode}

% End of document body:
%    \begin{macrocode}
\end{document}
%    \end{macrocode}
%\iffalse
%</samplemain>
%\fi
%
% %%%%%%%%%%%%%%%%%%%%%%%%%%%%%%%%%%%%%%
% \paragraph{Chapter Include Files.}
%
% The include files are called |cdocsch1.tex| and |cdocsch2.tex|.
%
%\iffalse
%<*samplechap1|samplechap2>
%\fi

% Optional override for |\version| flag:
%    \begin{macrocode}
%%\providecommand{\version}{final}
%    \end{macrocode}

% Include the main document:
%    \begin{macrocode}
\input{childdoc.def}
\childdocof{cdocsamp}
%    \end{macrocode}

%\iffalse
%</samplechap1|samplechap2>
%\fi
%
%\iffalse
%<*samplechap1>
%\fi
% Some text for chapter 1:
%    \begin{macrocode}
\section{one}
some text in chapter one
%    \end{macrocode}

%\iffalse
%</samplechap1>
%\fi
% Some text for chapter 2:
%\iffalse
%<*samplechap2>
%\fi
%    \begin{macrocode}
\section{two}
more text in chapter two
%    \end{macrocode}

%\iffalse
%</samplechap2>
%\fi
%
% %%%%%%%%%%%%%%%%%%%%%%%%%%%%%%%%%%%%%%
% \paragraph{Part Include Files.}
%
% The include files are called |cdocspt3.tex| and |cdocspt4.tex|.
%
%\iffalse
%<*samplepart3|samplepart4>
%\fi

% Optional override for |\version| flag:
%    \begin{macrocode}
%%\providecommand{\version}{final}
%    \end{macrocode}

% Include the main document:
%    \begin{macrocode}
\input{childdoc.def}
\childdocby{cdocsamp}
%    \end{macrocode}

%\iffalse
%</samplepart3|samplepart4>
%\fi
%
%\iffalse
%<*samplepart3>
%\fi
% Some text for part 3:
%    \begin{macrocode}
some text in part three
%    \end{macrocode}

%\iffalse
%</samplepart3>
%\fi
% Some text for part 4:
%\iffalse
%<*samplepart4>
%\fi
%    \begin{macrocode}
more text in part four
%    \end{macrocode}

%\iffalse
%</samplepart4>
%\fi
%
% %%%%%%%%%%%%%%%%%%%%%%%%%%%%%%%%%%%%%%
% \paragraph{Forwarding for a Complete Draft.}
%
% The following forwarding file |cdocsdrf.tex|
% compiles the main document in draft mode:
%\iffalse
%<*sampledraft>
%\fi
%    \begin{macrocode}
\def\version{draft}
\input{childdoc.def}
\childdocforward{cdocsamp}
%    \end{macrocode}

%\iffalse
%</sampledraft>
%\fi
%
% %%%%%%%%%%%%%%%%%%%%%%%%%%%%%%%%%%%%%%
% \paragraph{Forwarding for Final Version of the Chapters.}
%
% The following forwarding files |cdocsfn1.tex| and |cdocsfn2.tex|
% (with identical content)
% compile the final versions of the child documents
% |cdocsch1.tex| and |cdocsch2.tex|, respectively:
%\iffalse
%<*samplefinal>
%\fi
%    \begin{macrocode}
\def\version{final}
\input{childdoc.def}
\childdocforwardprefix[cdocsamp]{cdocsfn}{cdocsch}
%    \end{macrocode}

%\iffalse
%</samplefinal>
%\fi
%
% %%%%%%%%%%%%%%%%%%%%%%%%%%%%%%%%%%%%%%
% \paragraph{Command Line Processing.}
%
% The following three command lines generate the output files
% |cdocscld|, |cdocscl1| and |cdocscl2|
% which should be identical to
% |cdocsdrf|, |cdocsch1| and |cdocsfn2|, respectively:
% \begin{center}
% \begin{tabular}{l}
% |latex -jobname cdocscld \|\\
% |  "\def\version{draft}\input{childdoc.def}\childdocforward{cdocsamp}"|\\
% |latex -jobname cdocscl1 \|\\
% |  "\input{childdoc.def}\childdocforward[cdocsamp]{cdocsch1}"|\\
% |latex -jobname cdocscl2 \|\\
% |  "\def\version{final}\input{childdoc.def}\childdocforward{cdocsch2}"|
% \end{tabular}
% \end{center}
% Note that the trailing backslash on each first line
% merely continues the input to the second line
% (for convenient cut ant paste).
% Furthermore, the command |latex| can be replaced by any
% of its alternative versions such as |pdflatex|.
%
% %%%%%%%%%%%%%%%%%%%%%%%%%%%%%%%%%%%%%%%%%%%%%%%%%%%%%%%%%%%%%%%%%%%%%%%%%%%%%%
% %%%%%%%%%%%%%%%%%%%%%%%%%%%%%%%%%%%%%%%%%%%%%%%%%%%%%%%%%%%%%%%%%%%%%%%%%%%%%%
% \section{Implementation}
%\iffalse
%<*package>
%\fi
%
% This section describes the definitions file |childdoc.def|.

% The definitions cannot be loaded using |\usepackage| or |\RequirePackage|
% which has a mechanism to prevent loading a style file more than once.
% When loading the definitions by means of |\input|
% multiple instances have to be prevented manually:
%\iffalse
%This code needs to be before the `\ProvidesFile' directive
%which is defined at the beginning of this file.
%Therefore it is also placed there and commented out here.
%</package>
%<*discard>
%\fi
%    \begin{macrocode}
\ifdefined\childdocmain\endinput\fi
%    \end{macrocode}
%\iffalse
%</discard>
%<*package>
%\fi
%
% \macro{\ifchilddoc}
% \macro{\ifchilddocmanual}
% The conditional |\ifchilddoc| tells whether a
% child (true) or main (false) document is being compiled.
% The conditional |\ifchilddocmanual| tells whether
% the |\includeonly| mechanism is used (false) or
% the selection of child files must be performed manually (true).
% The definitions initialise to false:
%    \begin{macrocode}
\newif\ifchilddoc
\newif\ifchilddocmanual
%    \end{macrocode}

% \macro{\childdocname}
% \macro{\childdocjob}
% The macro |\childdocname| stores the name of the main document
% to be compiled. The macro |\childdocjob| stores the name of
% the document on which the \LaTeX{} compiler was originally invoked.
% The content of |\jobname| cannot be compared
% to filenames specified in the source due to different catcodes.
% The following code rescans |\jobname|, stores the result
% in |\childdocname| and saves a copy in |\childdocjob|:
%    \begin{macrocode}
\edef\childdocname{\scantokens\expandafter{\jobname\noexpand}}
\let\childdocjob\childdocname
%    \end{macrocode}

% \macro{\childdocdisable}
% The macro |\childdocdisable| prevents the main file
% from being processed more than once.
% At this stage, the main document command |\childdocmain|
% is assumed to be called once again where it should do nothing.
% Any subsequent call to it should prevent
% a secondary processing of the main document
% It overwrites the forwarding commands
% |\childdocof| and |\childdocforward|
% with empty macros to prevent further inclusions of the main document:
%    \begin{macrocode}
\newcommand{\childdocdisable}
{
  \renewcommand{\childdocmain}[1]{\renewcommand{\childdocmain}[1]{\endinput}}
  \renewcommand{\childdocof}[1]{}
  \renewcommand{\childdocby}[2][]{}
  \renewcommand{\childdocforward}[2][]{}
  \renewcommand{\childdocdisable}{}
}
%    \end{macrocode}

% \macro{\childdocmain}
% The macro |\childdocmain| is to be called at the top of the main file
% with nothing or the main filename (without extension) as argument.
% First, it breaks loops.
% If the argument is not empty and does not match |\childdocname|
% (which is set by the first inclusion of |childdoc.def|),
% |\ifchilddoc| is set to true, |\includeonly| is applied to the child file
% and |\jobname| is set to the main file
% (for proper handling of |.aux| files):
%    \begin{macrocode}
\newcommand{\childdocmain}[1]
{
  \childdocdisable\childdocmain{}
  \if?#1?\else
    \begingroup
      \def\childdoctmp{#1}
      \ifx\childdoctmp\childdocname
        \def\childdoctmp{}
      \else
        \def\childdoctmp
        {
          \childdoctrue
          \includeonly{\childdocname}
          \def\childdocjob{#1}
          \def\jobname{#1}
        }
      \fi
      \expandafter
    \endgroup
    \childdoctmp
  \fi
}
%    \end{macrocode}

% \macro{\childdocof}
% The command |\childdocof| redirects
% compilation to the main file |#1|.
%    \begin{macrocode}
\newcommand{\childdocof}[1]
{
  \childdocdisable
  \childdoctrue
  \includeonly{\childdocname}
  \def\jobname{#1}
  \def\childdocjob{#1}
  \input{#1}
}
%    \end{macrocode}

% \macro{\childdocby}
% The command |\childdocby| ....
%    \begin{macrocode}
\newcommand{\childdocby}[2][]
{
  \childdocdisable
  \childdoctrue
  \childdocmanualtrue
  \if?#1?\else
    \def\jobname{#2}
  \fi
  \def\childdocjob{#2}
  \input{#2}
  \endinput
}
%    \end{macrocode}

% \macro{\childdocforward}
% The command |\childdocforward| redirects
% compilation to the main file or
% (if the optional argument is given) a child file.
% Parameters are set as if the main file
% or a child file starting with |\childdocof| was compiled.
% Then compilation is handed over to the main file:
%    \begin{macrocode}
\newcommand{\childdocforward}[2][]
{
  \begingroup
    \if?#1?
      \def\childdoctmp
      {
        \def\childdocname{#2}
        \def\childdocjob{#2}
        \def\jobname{#2}
        \input{#2}
        \endinput
      }
    \else
      \def\childdoctmp
      {
        \childdocdisable
        \def\childdocname{#2}
        \childdoctrue
        \includeonly{#2}
        \def\childdocjob{#1}
        \def\jobname{#1}
        \input{#1}
        \endinput
      }
    \fi
    \expandafter
  \endgroup
  \childdoctmp
}
%    \end{macrocode}

% \macro{\childdocforwardprefix}
% The command |\childdocforwardprefix| redirects
% compilation to the main or a child file by means of a pattern.
% The prefix |#1| in the current filename is replaced by |#2|
% and the suffix of the current filename is kept
% (it is assumed that the filename does not contain the substring `|~~~|'
% which is used as a delimiter).
% Compilation is handed over to the new file by |\childdocforward|:
%    \begin{macrocode}
\newcommand{\childdocforwardprefix}[3][]
{
  \begingroup
    \def\childdocextract #2##1~~~{\def\childdoctmp{\childdocforward[#1]{#3##1}}}
    \expandafter\childdocextract\childdocname~~~
    \expandafter
  \endgroup
  \childdoctmp
}
%    \end{macrocode}

% \macro{\childdoc}
% The deprecated macro |\childdoc| is a legacy version of |\childdocmain|:
%    \begin{macrocode}
\newcommand{\childdoc}{\childdocmain}
%    \end{macrocode}

% \macro{\childdocredirect}
% The deprecated macro |\childdocredirect| is a legacy version
% of |\childdocforward| and |\childdocforwardprefix|:
%    \begin{macrocode}
\newcommand{\childdocredirect}[2][]
{
  \begingroup
    \if?#1?
      \def\childdoctmp{\childdocforward{#2}}
    \else
      \def\childdoctmp{\childdocforwardprefix{#1}{#2}}
    \fi
    \expandafter
  \endgroup
  \childdoctmp
}
%    \end{macrocode}

%\iffalse
%</package>
%\fi
%
\endinput
|\\
|\childdocby{|\textit{main}|}|\\
\end{tabular}
\end{center}
%
The directive |\childdocby| is similar to |\childdocof|
described in \secref{sec:include},
but the subsequent selection of content must be done manually.
To that end, both |\ifchilddoc| and |\ifchilddocmanual|
will be true upon processing of a part,
and the name of the part is stored in |\childdocname|.
Note that |\jobname| will be set to the filename of the current part
so that each part receives an individual |.aux| file
that does not interfere with the |.aux| file(s) of the main document.
This behaviour can be altered by the alternative form
|\childdocby[*]{|\textit{main}|}| (with a non-empty optional argument)
which uses the |.aux| file of the main document
by setting |\jobname| to \textit{main}.

%%%%%%%%%%%%%%%%%%%%%%%%%%%%%%%%%%%%%%%%%%%%%%%%%%%%%%%%%%%%%%%%%%%%%%%%%%%%%%%%
\subsection{Driver Development}
\label{sec:driver}

The \textsf{childdoc} mechanism can also be use for the development
of definition files such as \LaTeX{} styles or classes.
This case differs from the above setup with multiple parts
included by |\include| in that no |\includeonly| should be invoked.
This can be achieved by starting the include file
(before |\ProvidesPackage|) with:
%
\begin{center}
\begin{tabular}{l}
|% \iffalse
%
% childdoc.dtx Copyright (C) 2017-2018 Niklas Beisert
%
% This work may be distributed and/or modified under the
% conditions of the LaTeX Project Public License, either version 1.3
% of this license or (at your option) any later version.
% The latest version of this license is in
%   http://www.latex-project.org/lppl.txt
% and version 1.3 or later is part of all distributions of LaTeX
% version 2005/12/01 or later.
%
% This work has the LPPL maintenance status `maintained'.
%
% The Current Maintainer of this work is Niklas Beisert.
%
% This work consists of the files childdoc.dtx and childdoc.ins
% and the derived files childdoc.def and cdocsamp.tex with
% cdocsch1.tex, cdocsch2.tex, cdocsdrf.tex, cdocsfn1.tex, cdocsfn2.tex.
%
%<package>\ifdefined\childdocmain\endinput\fi
%<package>\ProvidesFile{childdoc.def}[2018/12/30 v2.0 child document driver]
%<samplemain>\ProvidesFile{cdocsamp.tex}[2018/12/30 v2.0 sample for childdoc]
%<*driver>
%\ProvidesFile{childdoc.drv}[2018/12/30 v2.0 childdoc reference manual file]
\PassOptionsToClass{10pt,a4paper}{article}
\documentclass{ltxdoc}

\usepackage[margin=35mm]{geometry}
\usepackage{hyperref}
\usepackage{hyperxmp}
\usepackage[usenames]{color}

\hypersetup{colorlinks=true}
\hypersetup{pdfstartview=FitH}
\hypersetup{pdfpagemode=UseNone}
\hypersetup{pdfsource={}}
\hypersetup{pdflang={en-UK}}
\hypersetup{pdfcopyright={Copyright 2017-2018 Niklas Beisert.
  This work may be distributed and/or modified under the
  conditions of the LaTeX Project Public License, either version 1.3
  of this license or (at your option) any later version.}}
\hypersetup{pdflicenseurl={http://www.latex-project.org/lppl.txt}}
\hypersetup{pdfcontactaddress={ETH Zurich, ITP, HIT K,
  Wolfgang-Pauli-Strasse 27}}
\hypersetup{pdfcontactpostcode={8093}}
\hypersetup{pdfcontactcity={Zurich}}
\hypersetup{pdfcontactcountry={Switzerland}}
\hypersetup{pdfcontactemail={nbeisert@itp.phys.ethz.ch}}
\hypersetup{pdfcontacturl={http://people.phys.ethz.ch/\xmptilde nbeisert/}}

\newcommand{\secref}[1]{\hyperref[#1]{section \ref*{#1}}}

\parskip1ex
\parindent0pt
\let\olditemize\itemize
\def\itemize{\olditemize\parskip0pt}

\begin{document}

\title{The \textsf{childdoc} Package}
\hypersetup{pdftitle={The childdoc Package}}
\author{Niklas Beisert\\[2ex]
  Institut f\"ur Theoretische Physik\\
  Eidgen\"ossische Technische Hochschule Z\"urich\\
  Wolfgang-Pauli-Strasse 27, 8093 Z\"urich, Switzerland\\[1ex]
  \href{mailto:nbeisert@itp.phys.ethz.ch}
  {\texttt{nbeisert@itp.phys.ethz.ch}}}
\hypersetup{pdfauthor={Niklas Beisert}}
\hypersetup{pdfsubject={Manual for the LaTeX2e Package childdoc}}
\date{30 December 2018, \textsf{v2.0}}
\maketitle

\begin{abstract}\noindent
\textsf{childdoc} is a \LaTeXe{} package
that enables the direct compilation
of document sections included by |\include|
to individual files.
\end{abstract}

\begingroup
\parskip0ex
\tableofcontents
\endgroup

%%%%%%%%%%%%%%%%%%%%%%%%%%%%%%%%%%%%%%%%%%%%%%%%%%%%%%%%%%%%%%%%%%%%%%%%%%%%%%%%
%%%%%%%%%%%%%%%%%%%%%%%%%%%%%%%%%%%%%%%%%%%%%%%%%%%%%%%%%%%%%%%%%%%%%%%%%%%%%%%%
\section{Introduction}

\LaTeX{} provides a mechanism to structure a large document (such as a book)
into a main file and several child files (containing the chapters)
using the |\include| command.
This mechanism is beneficial for documents
which span hundreds of pages in order to
make the source file(s) more manageable.
Moreover, compilation can be restricted to
selected child files by means of the |\includeonly| command.
The latter feature can be used to reduce the compilation time while editing
(this was significantly more useful in the earlier days of \LaTeX{})
or to generate a smaller document which is easier to navigate.
Another application of |\includeonly| is to generate
documents consisting of selected parts of the complete document.

However, there are a few drawbacks of the plain |\include| mechanism:
\begin{itemize}
\item
The child files cannot be compiled on their own,
they can only be compiled via the main file.
A naive editing environment
(such as a text editor with an option
to have the current file processed by \LaTeX)
may require one to switch to the main file before compiling;
attempting to compile the child file produces errors.
\item
The main file must be modified (each time)
to adjust the |\includeonly| command
to the present needs. This easily leaves the main file in a messy state.
\item
The generated document will always carry the filename
of the main document. This is inconvenient if
several child files are to be compiled and
to be kept for distribution.
\end{itemize}

The present package provides a simple interface
to make child files individually compilable by \LaTeX{}.
Compiling a child file then has the same effect as compiling
the main file with an |\includeonly| command
to select the appropriate child.
Moreover the generated document will carry the name of the child
rather than the main file.
This resolves all three above issues.

This feature is meant to make the editing of books,
thesis documents and lecture notes somewhat more convenient.
However, the package can also be used efficiently for
composing a series of documents (such as exercise sheets)
which are typically distributed individually.
It then assists the author in generating the individual documents
(potentially in different versions)
as well as a document containing the collected series.
Another application is in developing style files
or other kinds of included material
where compilation of the style file could redirect
to a sample or test file.

%%%%%%%%%%%%%%%%%%%%%%%%%%%%%%%%%%%%%%%%%%%%%%%%%%%%%%%%%%%%%%%%%%%%%%%%%%%%%%%%
%%%%%%%%%%%%%%%%%%%%%%%%%%%%%%%%%%%%%%%%%%%%%%%%%%%%%%%%%%%%%%%%%%%%%%%%%%%%%%%%
\section{Usage}

First of all, the package \textsf{childdoc} is \emph{not} a standard
\LaTeXe{} |.sty| style file! Therefore it needs to be invoked in
a non-standard way.

%%%%%%%%%%%%%%%%%%%%%%%%%%%%%%%%%%%%%%%%%%%%%%%%%%%%%%%%%%%%%%%%%%%%%%%%%%%%%%%%
\subsection{Included Files}
\label{sec:include}

%%%%%%%%%%%%%%%%%%%%%%%%%%%%%%%%%%%%%%%%
\DescribeMacro{\childdocmain}
To use the package, add the commands
\begin{center}
\begin{tabular}{l}
|\input{childdoc.def}|\\
|\childdocmain{}|\\
\end{tabular}
\end{center}
at the very top of the main \LaTeX{} file,
in particular \emph{before} the |\documentclass| statement!
The argument of |\childdocmain| should be left empty
(but it must be present).

%%%%%%%%%%%%%%%%%%%%%%%%%%%%%%%%%%%%%%%%
\DescribeMacro{\childdocof}
Furthermore, add the commands
\begin{center}
\begin{tabular}{l}
|\input{childdoc.def}|\\
|\childdocof{|\textit{main}|}|\\
\end{tabular}
\end{center}
at the top of every child file \textit{child}
which is included by |\include{|\textit{child}|}|
from within the main file
(or at least for those files to be compiled individually).
The argument \textit{main} must be the filename of the main file.

There are a couple of
considerations in setting up the main and child documents:

%%%%%%%%%%%%%%%%%%%%%%%%%%%%%%%%%%%%%%%%
\paragraph{Restrictions.}

Please note the following restrictions:
\begin{itemize}
\item
|\childdocmain| must be called with one argument \textit{main}
to ensure compatibility with earlier version of the package.
It must either be empty (|\childdocmain{}|)
or precisely match the filename of the main file in which it is specified.
See \secref{sec:detection} for further information.
\item
The filename \textit{main} must be specified without the |.tex| extension.
\item
The filename \textit{main} is case sensitive
(even in case-insensitive file systems)
due to internal string comparison.
\item
The argument \textit{main} should be fully expanded, it cannot be a macro.
\item
Subdirectories and special characters should be avoided in filenames.
\item
The command |\childdocmain{|\textit{main}|}| must be followed by a whitespace.
It should not be followed immediately by another command
or by a comment mark `|%|'.
This is because the \TeX{} parser reads the token immediately following
the argument of |\childdocmain| and puts it
at the beginning of every child section;
however, a white\-space is ignored.
\end{itemize}

%%%%%%%%%%%%%%%%%%%%%%%%%%%%%%%%%%%%%%%%
\paragraph{Content of Main File.}

It is advisable to place all content in the child files included by |\include|.
Any output contained in the main file will appear in all child documents
unless suppressed manually;
it cannot be suppressed automatically by the |\includeonly| directive
and thus should normally be avoided.
A method to include some content in the main file
by means of conditional processing is described in \secref{sec:conditional}.

%%%%%%%%%%%%%%%%%%%%%%%%%%%%%%%%%%%%%%%%
\paragraph{Page Numbering.}

When only a part of the document is compiled,
the appropriate numbering of pages
(as well as other status parameters)
is determined from the |.aux| files.
The latter contain information from previous passes.
However this information needs to propagate through
all intermediate child documents.
Therefore the page numbering in child documents may well
be inconsistent until the complete document is compiled at least once.

A useful (if unconventional) way to always ensure a consistent
page numbering is to restart the numbering in each child document
and denote the pages by `\textit{child}|.|\textit{page}'
where \textit{child} represents the chapter/section number of the child file.
This can be achieved by the command
|\numberwithin{page}{|\textit{child}|}|
of the \textsf{amsmath} package
where \textit{child} can be |chapter| or |section|
depending on the chosen structuring.
Alternatively, one can modify the macro |\thepage| appropriately
and reset the counter |page| at the start of each child file.

%%%%%%%%%%%%%%%%%%%%%%%%%%%%%%%%%%%%%%%%%%%%%%%%%%%%%%%%%%%%%%%%%%%%%%%%%%%%%%%%
\subsection{Conditional Processing}
\label{sec:conditional}

The package provides a mechanism to compile different versions
of a document. To customise the versions further some conditional processing
can come in handy to distinguish which version is being compiled.
The package provides two macros to describe the compilation context:

%%%%%%%%%%%%%%%%%%%%%%%%%%%%%%%%%%%%%%%%
\DescribeMacro{\ifchilddoc}
The conditional |\ifchilddoc| distinguishes between the compilation of
child documents and the main document:
%
\begin{center}
|\ifchilddoc |\textit{child-code}| |[|\||else |\textit{main-code}]| \||fi|
\end{center}

%%%%%%%%%%%%%%%%%%%%%%%%%%%%%%%%%%%%%%%%
\DescribeMacro{\childdocname}
\DescribeMacro{\childdocjob}
The macro |\childdocname| contains the filename (without extension)
of the main or child file being processed.
Note that |\childdocjob| will always contain the name of the main file.

%%%%%%%%%%%%%%%%%%%%%%%%%%%%%%%%%%%%%%%%
\paragraph{Title Page.}

Conditional processing can be used to include a title or banner page
in the main document when proper precautions are taken.
Importantly, the code in the main file should ensure that the page counter
(as well as other status parameters which are stored in the |.aux| files)
takes the same value after the conditional processing.
Otherwise the page numbers may take divergent values
depending on which part is compiled.

For example, a title page could be declared by:
%
\begin{center}
\begin{tabular}{l}
|\ifchilddoc\||else|\\
|\addtocounter{page}{-1}|\\
\textit{code for title page}\\
|\newpage|\\
|\||fi|
\end{tabular}
\end{center}
%
A banner page for the child documents can be generated by:
%
\begin{center}
\begin{tabular}{l}
|\ifchilddoc|\\
|\addtocounter{page}{-1}|\\
\textit{code for banner page}\\
|\newpage|\\
|\||fi|
\end{tabular}
\end{center}
%
Here one could write a message such as:
\begin{center}
|This is the part \childdocname{} of \childdocjob{}.|
\end{center}

%%%%%%%%%%%%%%%%%%%%%%%%%%%%%%%%%%%%%%%%%%%%%%%%%%%%%%%%%%%%%%%%%%%%%%%%%%%%%%%%
\subsection{Flags}
\label{sec:flags}

The package makes it easy to generate different versions
of the main or child documents.
To this end compilation flags can be defined
and assigned different default values.
They will be particularly useful in conjunction
with the forwarding mechanism described in \secref{sec:forward}.

For example, it may be useful to have a flag |\version|
which can be set to |draft| or |final|.
The document source will contain some conditional code
depending on the value of |\version|.
Suppose further, the flag should default to |final| for the main file
and to |draft| for child files
which is a natural assignment for editing the document.
This is achieved by placing the following code
in the preamble of the main document
(below the |\childdocmain| directive):
%
\begin{center}
\begin{tabular}{l}
|\ifchilddoc|\\
|\providecommand{\version}{draft}|\\
|\||else|\\
|\providecommand{\version}{final}|\\
|\||fi|
\end{tabular}
\end{center}
%
The definition by |\providecommand| makes sure
that previous definitions are not overwritten.
Further statements |\providecommand{\version}{...}|
can thus be added before the above code to override it.

For the main file, one might add a line
(between |\childdocmain| and the above block)
%
\begin{center}
|%\ifchilddoc\||else\providecommand{\version}{draft}\||fi|
\end{center}
%
which can be uncommented to produce a draft version.
Likewise one can add a line to the very top of a child file
(above the |\childdocof{|\textit{main}|}| directive)
%
\begin{center}
|%\providecommand{\version}{final}|
\end{center}
%
which can be uncommented to produce the final version of this child document.

%%%%%%%%%%%%%%%%%%%%%%%%%%%%%%%%%%%%%%%%%%%%%%%%%%%%%%%%%%%%%%%%%%%%%%%%%%%%%%%%
\subsection{Forwarding}
\label{sec:forward}

Different versions of the main or child documents
using compilation flags as described in \secref{sec:flags}
can be (permanently) stored in different files
for convenient compilation, viewing and distribution.
To this end, the package defines a command
to pass on compilation to a different file:

%%%%%%%%%%%%%%%%%%%%%%%%%%%%%%%%%%%%%%%%
\DescribeMacro{\childdocforward}
The command |\childdocforward| redirects processing to
another source file:
%
\begin{center}
\begin{tabular}{l}
|\input{childdoc.def}|\\
|\childdocforward[|\textit{main}|]{|\textit{dest}|}|\\
\end{tabular}
\end{center}
%
The argument \textit{dest} is the destination file
(without extension).
It should be the main file or one of the child files.
Note that further \textsf{childdoc} directives
such as |\childdocof| and |\childdocforward|
in the indicated file will be processed in this form.
The optional argument \textit{main}
passes on directly to the main file \textit{main}
while pretending to compile the child \textit{dest}.
This form behaves as if \textit{dest}
issues |\childdocof{|\textit{main}|}| right away,
and no further \textsf{childdoc} directives will be processed.

%%%%%%%%%%%%%%%%%%%%%%%%%%%%%%%%%%%%%%%%
\DescribeMacro{\...prefix}
In the alternative form |\childdocforwardprefix|,
%
\begin{center}
\begin{tabular}{l}
|\input{childdoc.def}|\\
|\childdocforwardprefix[|\textit{main}|]{|\textit{prefix}|}{|\textit{dest}|}|
\end{tabular}
\end{center}
%
the destination file is determined by a pattern
depending on the current file:
To make this work, the current file must be called
`{\textit{prefix}\hspace{0.2em}\textit{suffix}}'
with \textit{prefix} matching precisely the argument.
Processing is then passed on to the file
`{\textit{dest}\hspace{0.2em}\textit{suffix}}'.
Surely, the same effect is achieved by
directly specifying the
argument `{\textit{dest}\hspace{0.2em}\textit{suffix}}'
in the first form.
However, that requires to set up a different file
for each child. With the alternative form of the command
all these files can have exactly the same content
which simplifies setting them up and maintaining them.

For example, the following file |draft.tex|
with a compilation flag |\version| as described in \secref{sec:flags}
compiles the main document as a draft:
%
\begin{center}
\begin{tabular}{l}
|\def\version{draft}|\\
|\input{childdoc.def}|\\
|\childdocforward{|\textit{main}|}|
\end{tabular}
\end{center}
%
Likewise, the following files |final|\textit{nn}|.tex|
compile the final version of the child document
|child|\textit{nn}|.tex|:
%
\begin{center}
\begin{tabular}{l}
|\def\version{final}|\\
|\input{childdoc.def}|\\
|\childdocforwardprefix{final}{child}|
\end{tabular}
\end{center}
%

Note that when several versions of a main file and/or of each child file
are to be generated, it may be convenient to set up a |Makefile| or
shell script to automatise the process.

%%%%%%%%%%%%%%%%%%%%%%%%%%%%%%%%%%%%%%%%%%%%%%%%%%%%%%%%%%%%%%%%%%%%%%%%%%%%%%%%
\subsection{Command Line Processing}
\label{sec:commandline}

The effect of redirection files can also be achieved by invoking
the \LaTeX{} compiler with a more elaborate command line.
Most conveniently this should be done as part
of a shell script or a |Makefile|.

When using \textsf{childdoc} in the main file, the following
command lines effectively perform a redirection
(note that depending on the shell being used,
backslashes may have to be doubled: `|\|' $\to$ `|\\|'):
%
\begin{center}
|... -jobname "|\textit{target}|" |\\|"|[\textit{flags}]%
|\input{childdoc.def}\childdocforward[|\textit{main}|]{|\textit{dest}|}"|
\end{center}
%
Here \textit{target} is the name of the output file,
\textit{main} is the name of the main file
and \textit{dest} is the name of the main or child file to be processed
(all filenames without extensions).
The optional argument \textit{main} can be omitted
if \textit{main} matches \textit{dest}.
Optionally, compilation \textit{flags} can be defined via |\def| commands.
This command line makes the \TeX{} engine believe
it is compiling the file \textit{target}
whose content is specified as the latter parameter.
The provided code then forwards the processing to
\textit{main} or \textit{dest} as described in \secref{sec:forward}.

%%%%%%%%%%%%%%%%%%%%%%%%%%%%%%%%%%%%%%%%%%%%%%%%%%%%%%%%%%%%%%%%%%%%%%%%%%%%%%%%
\subsection{Include by Input}
\label{sec:input}

Including child documents by |\include| has some restrictions by design.
Most notably, the content of a child document always occupies
its own set of pages; pages cannot be shared between child documents.
Usually, this behaviour makes perfect sense
because each child document contain an essential part of the document.
However, in some situations it may be desirable to compose
a document from a collection of parts
without having mandatory page breaks between then.
For this case, the package
provides a mechanism to include parts
by |\input| which can also be processed individually.
However, by construction this mechanism
requires manual handling of the content to be output.

%%%%%%%%%%%%%%%%%%%%%%%%%%%%%%%%%%%%%%%%
\DescribeMacro{\ifchilddocmanual}
The main file should be prepared as usual, see \secref{sec:include}.
However, the document body must make a distinction
between processing of an individual part and of the main document, e.g.:
%
\begin{center}
\begin{tabular}{l}
|\ifchilddocmanual|\\
|\input{\childdocname}|\\
|\||else|\\
\textit{document body with }|\input{|\textit{part}|}|\\
|\||fi|
\end{tabular}
\end{center}
%
The conditional |\ifchilddocmanual| is true whenever
a part to be included by |\input| is being compiled,
and the name of the part is stored in |\childdocname|.

%%%%%%%%%%%%%%%%%%%%%%%%%%%%%%%%%%%%%%%%
\DescribeMacro{\childdocby}
Each part to be included by |\input| should start with:
%
\begin{center}
\begin{tabular}{l}
|\input{childdoc.def}|\\
|\childdocby{|\textit{main}|}|\\
\end{tabular}
\end{center}
%
The directive |\childdocby| is similar to |\childdocof|
described in \secref{sec:include},
but the subsequent selection of content must be done manually.
To that end, both |\ifchilddoc| and |\ifchilddocmanual|
will be true upon processing of a part,
and the name of the part is stored in |\childdocname|.
Note that |\jobname| will be set to the filename of the current part
so that each part receives an individual |.aux| file
that does not interfere with the |.aux| file(s) of the main document.
This behaviour can be altered by the alternative form
|\childdocby[*]{|\textit{main}|}| (with a non-empty optional argument)
which uses the |.aux| file of the main document
by setting |\jobname| to \textit{main}.

%%%%%%%%%%%%%%%%%%%%%%%%%%%%%%%%%%%%%%%%%%%%%%%%%%%%%%%%%%%%%%%%%%%%%%%%%%%%%%%%
\subsection{Driver Development}
\label{sec:driver}

The \textsf{childdoc} mechanism can also be use for the development
of definition files such as \LaTeX{} styles or classes.
This case differs from the above setup with multiple parts
included by |\include| in that no |\includeonly| should be invoked.
This can be achieved by starting the include file
(before |\ProvidesPackage|) with:
%
\begin{center}
\begin{tabular}{l}
|\input{childdoc.def}|\\
|\childdocforward{|\textit{main}|}|\\
\end{tabular}
\end{center}
%
or alternatively with:
%
\begin{center}
\begin{tabular}{l}
|\input{childdoc.def}|\\
|\childdocby{|\textit{main}|}|\\
\end{tabular}
\end{center}
%
Both forms have slightly different effects as described above.
The main file is prepared as usual, see \secref{sec:include}.

%%%%%%%%%%%%%%%%%%%%%%%%%%%%%%%%%%%%%%%%%%%%%%%%%%%%%%%%%%%%%%%%%%%%%%%%%%%%%%%%
\subsection{Legacy Detection}
\label{sec:detection}

The directive |\childdocmain| in the main file can detect
whether the complete document or merely a child is to be compiled
even without using the directive |\childdocof|.
This method is deprecated because it is less robust
and there is no compelling reason to use it;
it is merely provided for backward compatibility
and it may be removed in future versions.

If the detection mechanism is to be used,
it is mandatory to correctly specify
the filename of the main file as the argument of |\childdocmain|:
%
\begin{center}
\begin{tabular}{l}
|\input{childdoc.def}|\\
|\childdocmain{|\textit{main}|}|\\
\end{tabular}
\end{center}
%
If |\jobname| does not match the argument \textit{main} of |\childdocmain|,
it is assumed that |\jobname| points to the child file to be compiled.
When using |\childdocmain| with the main file specified as argument,
it suffices to start a child file
with just |\input{|\textit{main}|}|
without loading of the package and using |\childdocof|.
If instead all processing is done
with the appropriate \textsf{childdoc} directives,
the argument of \textit{main} of |\childdocmain| can be empty.

An alternative version of the command line processing described
in \secref{sec:commandline} using the detection mechanism reads:
%
\begin{center}
|... -jobname "|\textit{target}|" "|[\textit{flags}]%
[|\def\jobname{|\textit{dest}|}|]|\input{|\textit{main}|}"|
\end{center}

%%%%%%%%%%%%%%%%%%%%%%%%%%%%%%%%%%%%%%%%%%%%%%%%%%%%%%%%%%%%%%%%%%%%%%%%%%%%%%%%
\subsection{Manual Code}
\label{sec:manual}

In case one cannot be certain whether the definitions file |childdoc.def|
is installed on the target \TeX{} distribution
and one prefers not to ship it,
it is conceivable to paste a few relevant commands into the sources.

To that end, drop all statements |\input{childdoc.def}|
and perform the replacements as outlined below.
Instead of |\childdocmain{|\textit{main}|}| add the following code
to the top of the main file:
%
\begin{center}
\begin{tabular}{l}
|\||ifdefined\childdocname\endinput\||fi\newif\ifchilddoc|\\
|\edef\childdocname{\scantokens\expandafter{\jobname\noexpand}}|\\
|\def\childdocmain{|\textit{main}|}\||ifx\childdocmain\childdocname\||else|\\
|\childdoctrue\includeonly{\childdocname}\let\jobname\childdocmain\||fi|\\
\end{tabular}
\end{center}
%
Instead of |\childdocof{|\textit{main}|}| just include the main file
at the top of each child file:
%
\begin{center}
|\input{|\textit{main}|}|
\end{center}
%
A simple redirection |\childdocforward{|\textit{dest}|}| is achieved by:
%
\begin{center}
|\def\jobname{|\textit{dest}|}\input{\jobname}|
\end{center}
%
The redirection with prefix
|\childdocforwardprefix[|\textit{prefix}|]{|\textit{dest}|}|
is accomplished by:
%
\begin{center}
\begin{tabular}{l}
|{\edef\jobname{\scantokens\expandafter{\jobname\noexpand}}|\\
|\def\redirectjob |\textit{prefix}|#1~~~{\gdef\jobname{|\textit{dest}|#1}}|\\
|\expandafter\redirectjob\jobname~~~}\input{\jobname}|
\end{tabular}
\end{center}

In an alternative approach,
child documents can be compiled by a specific command line
without additional code or specific definitions:
%
\begin{center}
|... -jobname "|\textit{target}|" "|[\textit{flags}]%
|\includeonly{|\textit{dest}|}\input{|\textit{main}|}"|
\end{center}
%

%%%%%%%%%%%%%%%%%%%%%%%%%%%%%%%%%%%%%%%%%%%%%%%%%%%%%%%%%%%%%%%%%%%%%%%%%%%%%%%%
%%%%%%%%%%%%%%%%%%%%%%%%%%%%%%%%%%%%%%%%%%%%%%%%%%%%%%%%%%%%%%%%%%%%%%%%%%%%%%%%
\section{Information}

%%%%%%%%%%%%%%%%%%%%%%%%%%%%%%%%%%%%%%%%%%%%%%%%%%%%%%%%%%%%%%%%%%%%%%%%%%%%%%%%
\subsection{Copyright}

Copyright \copyright{} 2017--2018 Niklas Beisert

This work may be distributed and/or modified under the
conditions of the \LaTeX{} Project Public License, either version 1.3
of this license or (at your option) any later version.
The latest version of this license is in
  \url{http://www.latex-project.org/lppl.txt}
and version 1.3 or later is part of all distributions of \LaTeX{}
version 2005/12/01 or later.

This work has the LPPL maintenance status `maintained'.

The Current Maintainer of this work is Niklas Beisert.

This work consists of the files |README.txt|, |childdoc.ins| and |childdoc.dtx|
as well as the derived files |childdoc.def|, |cdocsamp.tex|
with |cdocsch1.tex|, |cdocsch2.tex|, |cdocspt3.tex|, |cdocspt4.tex|,
|cdocsdrf.tex|, |cdocsfn1.tex|, |cdocsfn2.tex|
as well as |childdoc.pdf|.

%%%%%%%%%%%%%%%%%%%%%%%%%%%%%%%%%%%%%%%%%%%%%%%%%%%%%%%%%%%%%%%%%%%%%%%%%%%%%%%%
\subsection{Files and Installation}

The package consists of the files:
%
\begin{center}
\begin{tabular}{ll}
    |README.txt|   & readme file \\
    |childdoc.ins| & installation file \\
    |childdoc.dtx| & source file \\
    |childdoc.def| & definition file \\
    |cdocsamp.tex| & sample main file \\
    |cdocsch1.tex| & sample include file \\
    |cdocsch2.tex| & sample include file \\
    |cdocspt3.tex| & sample part file \\
    |cdocspt4.tex| & sample part file \\
    |cdocsdrf.tex| & sample redirection file \\
    |cdocsfn1.tex| & sample redirection file \\
    |cdocsfn2.tex| & sample redirection file \\
    |childdoc.pdf| & manual
\end{tabular}
\end{center}
%
The distribution consists of the files
|README.txt|, |childdoc.ins| and |childdoc.dtx|.
%
\begin{itemize}
\item
Run (pdf)\LaTeX{} on |childdoc.dtx|
to compile the manual |childdoc.pdf| (this file).
\item
Run \LaTeX{} on |childdoc.ins| to create the definitions file |childdoc.def|
and the sample |cdocsamp.tex| with include files
|cdocsch1.tex|, |cdocsch2.tex|, |cdocspt3.tex|, |cdocspt4.tex|,
|cdocsdrf.tex|, |cdocsfn1.tex|, |cdocsfn2.tex|.
Then copy the file |childdoc.def| to an appropriate directory of your \LaTeX{}
distribution, e.g.\ \textit{texmf-root}|/tex/latex/childdoc|.
\end{itemize}

%%%%%%%%%%%%%%%%%%%%%%%%%%%%%%%%%%%%%%%%%%%%%%%%%%%%%%%%%%%%%%%%%%%%%%%%%%%%%%%%
\subsection{Related CTAN Packages}

There are several other packages which offer a similar functionality:
%
\begin{itemize}
\item
The packages
\href{http://ctan.org/pkg/docmute}{\textsf{docmute}},
\href{http://ctan.org/pkg/includex}{\textsf{includex}} and
\href{http://ctan.org/pkg/standalone}{\textsf{standalone}}
provide commands to include only the document body of
a child file thus allowing both files to be compiled individually.
\item
The packages \href{http://ctan.org/pkg/subdocs}{\textsf{subdocs}}
and \href{http://ctan.org/pkg/subfiles}{\textsf{subfiles}}
provide structures in which the main and child documents can be
encapsulated and allowing them to be compiled individually.
The inclusion mechanism is different from the conventional |\include|.
\item
The package \href{http://ctan.org/pkg/combine}{\textsf{combine}}
is an elaborate solution to combine several documents into one.
\end{itemize}
%
See also the CTAN topic \href{http://ctan.org/topic/subdocs}{\textsf{subdocs}}
for further related packages.
The present package differs from the above solutions in that
a document structure constructed with the conventional |\include| mechanism
just needs two extra commands at the top of every file
such that all constituent files can be compiled individually.

%%%%%%%%%%%%%%%%%%%%%%%%%%%%%%%%%%%%%%%%%%%%%%%%%%%%%%%%%%%%%%%%%%%%%%%%%%%%%%%%
%\subsection{Feature Suggestions}
%
%The following is a list of features which may be useful for future
%versions of this package:
%%
%\begin{itemize}
%\item
%\ldots
%\end{itemize}

%%%%%%%%%%%%%%%%%%%%%%%%%%%%%%%%%%%%%%%%%%%%%%%%%%%%%%%%%%%%%%%%%%%%%%%%%%%%%%%%
\subsection{Revision History}

%%%%%%%%%%%%%%%%%%%%%%%%%%%%%%%%%%%%%%%%
\paragraph{v2.0:} 2018/12/30

\begin{itemize}
\item
immediate forward processing
\item
added |\childdocby| mechanism
\item
manual restructured
\end{itemize}

%%%%%%%%%%%%%%%%%%%%%%%%%%%%%%%%%%%%%%%%
\paragraph{v1.6:} 2018/01/17

\begin{itemize}
\item
application for development of include files
\item
corrections to manual
\end{itemize}

%%%%%%%%%%%%%%%%%%%%%%%%%%%%%%%%%%%%%%%%
\paragraph{v1.5:} 2017/05/21

\begin{itemize}
\item
more complete structuring introduced
\item
|\childdocof| introduced
\item
|\childdoc| renamed to |\childdocmain|
\item
|\childredirect| renamed to |\childdocforward| and |\childdocforwardprefix|
and functionality expanded
\end{itemize}

%%%%%%%%%%%%%%%%%%%%%%%%%%%%%%%%%%%%%%%%
\paragraph{v1.0:} 2017/04/27

\begin{itemize}
\item
manual and install package
\item
first version published on CTAN
\end{itemize}

%%%%%%%%%%%%%%%%%%%%%%%%%%%%%%%%%%%%%%%%
\paragraph{v0.6:} 2017/04/26

\begin{itemize}
\item
redirection mechanism added
\end{itemize}

%%%%%%%%%%%%%%%%%%%%%%%%%%%%%%%%%%%%%%%%
\paragraph{v0.5:} 2017/04/26

\begin{itemize}
\item
functionality in definition file
\end{itemize}


%%%%%%%%%%%%%%%%%%%%%%%%%%%%%%%%%%%%%%%%%%%%%%%%%%%%%%%%%%%%%%%%%%%%%%%%%%%%%%%%
%%%%%%%%%%%%%%%%%%%%%%%%%%%%%%%%%%%%%%%%%%%%%%%%%%%%%%%%%%%%%%%%%%%%%%%%%%%%%%%%
%%%%%%%%%%%%%%%%%%%%%%%%%%%%%%%%%%%%%%%%%%%%%%%%%%%%%%%%%%%%%%%%%%%%%%%%%%%%%%%%
\appendix

\settowidth\MacroIndent{\rmfamily\scriptsize 000\ }

 \DocInput{childdoc.dtx}

\end{document}
%</driver>
% \fi
%
% %%%%%%%%%%%%%%%%%%%%%%%%%%%%%%%%%%%%%%%%%%%%%%%%%%%%%%%%%%%%%%%%%%%%%%%%%%%%%%
% %%%%%%%%%%%%%%%%%%%%%%%%%%%%%%%%%%%%%%%%%%%%%%%%%%%%%%%%%%%%%%%%%%%%%%%%%%%%%%
% \section{Sample}
%\iffalse
%<*samplemain>
%\fi
%
% The following presents a sample document
% with two chapters, two parts, a title page,
% a compile flag as well as three forwarding files to set the flag.
% It consists of eight |.tex| files:
% \begin{center}
% \begin{tabular}{ll}
% |cdocsamp.tex|&main file\\
% |cdocsch1.tex|&include file for chapter 1\\
% |cdocsch2.tex|&include file for chapter 2\\
% |cdocspt3.tex|&include file for part 3\\
% |cdocspt4.tex|&include file for part 4\\
% |cdocsdrf.tex|&forwarding file for main file in draft mode\\
% |cdocsfi1.tex|&forwarding file for final version of chapter 1\\
% |cdocsfi2.tex|&forwarding file for final version of chapter 2\\
% \end{tabular}
% \end{center}
% Each of the eight files can be compiled directly by the \LaTeX{} compiler.
%
% %%%%%%%%%%%%%%%%%%%%%%%%%%%%%%%%%%%%%%
% \paragraph{Main File.}
%
% The main file is called |cdocsamp.tex|.
%
% Load the \textsf{childdoc} definitions and
% declare the filename for the main document:
%    \begin{macrocode}
\input{childdoc.def}
\childdocmain{}
%    \end{macrocode}

% Optional override for |\version| flag:
%    \begin{macrocode}
%%\ifchilddoc\else\providecommand{\version}{draft}\fi
%    \end{macrocode}

% Define the default values for the |\version| flag
% (|final| for the main file and |draft| for childs):
%    \begin{macrocode}
\ifchilddoc
\providecommand{\version}{draft}
\else
\providecommand{\version}{final}
\fi
%    \end{macrocode}

% Load the standard document class:
%    \begin{macrocode}
\documentclass[12pt]{article}
%    \end{macrocode}

% Start the document body:
%    \begin{macrocode}
\begin{document}
%    \end{macrocode}

% Declare a title page.
% Print title, part of document being processed and version flag:
%    \begin{macrocode}
\addtocounter{page}{-1}
\begin{center}
{\LARGE\bfseries{}childdoc example\par}
\vspace{1cm}
\ifchilddoc
\ifchilddocmanual part\else chapter\fi:
`\childdocname' of `\childdocjob'\par
\else
main document: `\childdocjob'\par
\fi
version: \version\par
\end{center}
\newpage
%    \end{macrocode}

% Manually include selected file,
% otherwise process as usual:
%    \begin{macrocode}
\ifchilddocmanual
\section*{part `\childdocname'}
\input{\childdocname}
\else
%    \end{macrocode}

% Include the two chapters:
%    \begin{macrocode}
\include{cdocsch1}
\include{cdocsch2}
%    \end{macrocode}

% Include the two parts unless only chapters should be displayed:
%    \begin{macrocode}
\ifchilddoc\else
\section{part three}
\input{cdocspt3}
\section{part four}
\input{cdocspt4}
\fi
%    \end{macrocode}

% Process as usual until here:
%    \begin{macrocode}
\fi
%    \end{macrocode}

% End of document body:
%    \begin{macrocode}
\end{document}
%    \end{macrocode}
%\iffalse
%</samplemain>
%\fi
%
% %%%%%%%%%%%%%%%%%%%%%%%%%%%%%%%%%%%%%%
% \paragraph{Chapter Include Files.}
%
% The include files are called |cdocsch1.tex| and |cdocsch2.tex|.
%
%\iffalse
%<*samplechap1|samplechap2>
%\fi

% Optional override for |\version| flag:
%    \begin{macrocode}
%%\providecommand{\version}{final}
%    \end{macrocode}

% Include the main document:
%    \begin{macrocode}
\input{childdoc.def}
\childdocof{cdocsamp}
%    \end{macrocode}

%\iffalse
%</samplechap1|samplechap2>
%\fi
%
%\iffalse
%<*samplechap1>
%\fi
% Some text for chapter 1:
%    \begin{macrocode}
\section{one}
some text in chapter one
%    \end{macrocode}

%\iffalse
%</samplechap1>
%\fi
% Some text for chapter 2:
%\iffalse
%<*samplechap2>
%\fi
%    \begin{macrocode}
\section{two}
more text in chapter two
%    \end{macrocode}

%\iffalse
%</samplechap2>
%\fi
%
% %%%%%%%%%%%%%%%%%%%%%%%%%%%%%%%%%%%%%%
% \paragraph{Part Include Files.}
%
% The include files are called |cdocspt3.tex| and |cdocspt4.tex|.
%
%\iffalse
%<*samplepart3|samplepart4>
%\fi

% Optional override for |\version| flag:
%    \begin{macrocode}
%%\providecommand{\version}{final}
%    \end{macrocode}

% Include the main document:
%    \begin{macrocode}
\input{childdoc.def}
\childdocby{cdocsamp}
%    \end{macrocode}

%\iffalse
%</samplepart3|samplepart4>
%\fi
%
%\iffalse
%<*samplepart3>
%\fi
% Some text for part 3:
%    \begin{macrocode}
some text in part three
%    \end{macrocode}

%\iffalse
%</samplepart3>
%\fi
% Some text for part 4:
%\iffalse
%<*samplepart4>
%\fi
%    \begin{macrocode}
more text in part four
%    \end{macrocode}

%\iffalse
%</samplepart4>
%\fi
%
% %%%%%%%%%%%%%%%%%%%%%%%%%%%%%%%%%%%%%%
% \paragraph{Forwarding for a Complete Draft.}
%
% The following forwarding file |cdocsdrf.tex|
% compiles the main document in draft mode:
%\iffalse
%<*sampledraft>
%\fi
%    \begin{macrocode}
\def\version{draft}
\input{childdoc.def}
\childdocforward{cdocsamp}
%    \end{macrocode}

%\iffalse
%</sampledraft>
%\fi
%
% %%%%%%%%%%%%%%%%%%%%%%%%%%%%%%%%%%%%%%
% \paragraph{Forwarding for Final Version of the Chapters.}
%
% The following forwarding files |cdocsfn1.tex| and |cdocsfn2.tex|
% (with identical content)
% compile the final versions of the child documents
% |cdocsch1.tex| and |cdocsch2.tex|, respectively:
%\iffalse
%<*samplefinal>
%\fi
%    \begin{macrocode}
\def\version{final}
\input{childdoc.def}
\childdocforwardprefix[cdocsamp]{cdocsfn}{cdocsch}
%    \end{macrocode}

%\iffalse
%</samplefinal>
%\fi
%
% %%%%%%%%%%%%%%%%%%%%%%%%%%%%%%%%%%%%%%
% \paragraph{Command Line Processing.}
%
% The following three command lines generate the output files
% |cdocscld|, |cdocscl1| and |cdocscl2|
% which should be identical to
% |cdocsdrf|, |cdocsch1| and |cdocsfn2|, respectively:
% \begin{center}
% \begin{tabular}{l}
% |latex -jobname cdocscld \|\\
% |  "\def\version{draft}\input{childdoc.def}\childdocforward{cdocsamp}"|\\
% |latex -jobname cdocscl1 \|\\
% |  "\input{childdoc.def}\childdocforward[cdocsamp]{cdocsch1}"|\\
% |latex -jobname cdocscl2 \|\\
% |  "\def\version{final}\input{childdoc.def}\childdocforward{cdocsch2}"|
% \end{tabular}
% \end{center}
% Note that the trailing backslash on each first line
% merely continues the input to the second line
% (for convenient cut ant paste).
% Furthermore, the command |latex| can be replaced by any
% of its alternative versions such as |pdflatex|.
%
% %%%%%%%%%%%%%%%%%%%%%%%%%%%%%%%%%%%%%%%%%%%%%%%%%%%%%%%%%%%%%%%%%%%%%%%%%%%%%%
% %%%%%%%%%%%%%%%%%%%%%%%%%%%%%%%%%%%%%%%%%%%%%%%%%%%%%%%%%%%%%%%%%%%%%%%%%%%%%%
% \section{Implementation}
%\iffalse
%<*package>
%\fi
%
% This section describes the definitions file |childdoc.def|.

% The definitions cannot be loaded using |\usepackage| or |\RequirePackage|
% which has a mechanism to prevent loading a style file more than once.
% When loading the definitions by means of |\input|
% multiple instances have to be prevented manually:
%\iffalse
%This code needs to be before the `\ProvidesFile' directive
%which is defined at the beginning of this file.
%Therefore it is also placed there and commented out here.
%</package>
%<*discard>
%\fi
%    \begin{macrocode}
\ifdefined\childdocmain\endinput\fi
%    \end{macrocode}
%\iffalse
%</discard>
%<*package>
%\fi
%
% \macro{\ifchilddoc}
% \macro{\ifchilddocmanual}
% The conditional |\ifchilddoc| tells whether a
% child (true) or main (false) document is being compiled.
% The conditional |\ifchilddocmanual| tells whether
% the |\includeonly| mechanism is used (false) or
% the selection of child files must be performed manually (true).
% The definitions initialise to false:
%    \begin{macrocode}
\newif\ifchilddoc
\newif\ifchilddocmanual
%    \end{macrocode}

% \macro{\childdocname}
% \macro{\childdocjob}
% The macro |\childdocname| stores the name of the main document
% to be compiled. The macro |\childdocjob| stores the name of
% the document on which the \LaTeX{} compiler was originally invoked.
% The content of |\jobname| cannot be compared
% to filenames specified in the source due to different catcodes.
% The following code rescans |\jobname|, stores the result
% in |\childdocname| and saves a copy in |\childdocjob|:
%    \begin{macrocode}
\edef\childdocname{\scantokens\expandafter{\jobname\noexpand}}
\let\childdocjob\childdocname
%    \end{macrocode}

% \macro{\childdocdisable}
% The macro |\childdocdisable| prevents the main file
% from being processed more than once.
% At this stage, the main document command |\childdocmain|
% is assumed to be called once again where it should do nothing.
% Any subsequent call to it should prevent
% a secondary processing of the main document
% It overwrites the forwarding commands
% |\childdocof| and |\childdocforward|
% with empty macros to prevent further inclusions of the main document:
%    \begin{macrocode}
\newcommand{\childdocdisable}
{
  \renewcommand{\childdocmain}[1]{\renewcommand{\childdocmain}[1]{\endinput}}
  \renewcommand{\childdocof}[1]{}
  \renewcommand{\childdocby}[2][]{}
  \renewcommand{\childdocforward}[2][]{}
  \renewcommand{\childdocdisable}{}
}
%    \end{macrocode}

% \macro{\childdocmain}
% The macro |\childdocmain| is to be called at the top of the main file
% with nothing or the main filename (without extension) as argument.
% First, it breaks loops.
% If the argument is not empty and does not match |\childdocname|
% (which is set by the first inclusion of |childdoc.def|),
% |\ifchilddoc| is set to true, |\includeonly| is applied to the child file
% and |\jobname| is set to the main file
% (for proper handling of |.aux| files):
%    \begin{macrocode}
\newcommand{\childdocmain}[1]
{
  \childdocdisable\childdocmain{}
  \if?#1?\else
    \begingroup
      \def\childdoctmp{#1}
      \ifx\childdoctmp\childdocname
        \def\childdoctmp{}
      \else
        \def\childdoctmp
        {
          \childdoctrue
          \includeonly{\childdocname}
          \def\childdocjob{#1}
          \def\jobname{#1}
        }
      \fi
      \expandafter
    \endgroup
    \childdoctmp
  \fi
}
%    \end{macrocode}

% \macro{\childdocof}
% The command |\childdocof| redirects
% compilation to the main file |#1|.
%    \begin{macrocode}
\newcommand{\childdocof}[1]
{
  \childdocdisable
  \childdoctrue
  \includeonly{\childdocname}
  \def\jobname{#1}
  \def\childdocjob{#1}
  \input{#1}
}
%    \end{macrocode}

% \macro{\childdocby}
% The command |\childdocby| ....
%    \begin{macrocode}
\newcommand{\childdocby}[2][]
{
  \childdocdisable
  \childdoctrue
  \childdocmanualtrue
  \if?#1?\else
    \def\jobname{#2}
  \fi
  \def\childdocjob{#2}
  \input{#2}
  \endinput
}
%    \end{macrocode}

% \macro{\childdocforward}
% The command |\childdocforward| redirects
% compilation to the main file or
% (if the optional argument is given) a child file.
% Parameters are set as if the main file
% or a child file starting with |\childdocof| was compiled.
% Then compilation is handed over to the main file:
%    \begin{macrocode}
\newcommand{\childdocforward}[2][]
{
  \begingroup
    \if?#1?
      \def\childdoctmp
      {
        \def\childdocname{#2}
        \def\childdocjob{#2}
        \def\jobname{#2}
        \input{#2}
        \endinput
      }
    \else
      \def\childdoctmp
      {
        \childdocdisable
        \def\childdocname{#2}
        \childdoctrue
        \includeonly{#2}
        \def\childdocjob{#1}
        \def\jobname{#1}
        \input{#1}
        \endinput
      }
    \fi
    \expandafter
  \endgroup
  \childdoctmp
}
%    \end{macrocode}

% \macro{\childdocforwardprefix}
% The command |\childdocforwardprefix| redirects
% compilation to the main or a child file by means of a pattern.
% The prefix |#1| in the current filename is replaced by |#2|
% and the suffix of the current filename is kept
% (it is assumed that the filename does not contain the substring `|~~~|'
% which is used as a delimiter).
% Compilation is handed over to the new file by |\childdocforward|:
%    \begin{macrocode}
\newcommand{\childdocforwardprefix}[3][]
{
  \begingroup
    \def\childdocextract #2##1~~~{\def\childdoctmp{\childdocforward[#1]{#3##1}}}
    \expandafter\childdocextract\childdocname~~~
    \expandafter
  \endgroup
  \childdoctmp
}
%    \end{macrocode}

% \macro{\childdoc}
% The deprecated macro |\childdoc| is a legacy version of |\childdocmain|:
%    \begin{macrocode}
\newcommand{\childdoc}{\childdocmain}
%    \end{macrocode}

% \macro{\childdocredirect}
% The deprecated macro |\childdocredirect| is a legacy version
% of |\childdocforward| and |\childdocforwardprefix|:
%    \begin{macrocode}
\newcommand{\childdocredirect}[2][]
{
  \begingroup
    \if?#1?
      \def\childdoctmp{\childdocforward{#2}}
    \else
      \def\childdoctmp{\childdocforwardprefix{#1}{#2}}
    \fi
    \expandafter
  \endgroup
  \childdoctmp
}
%    \end{macrocode}

%\iffalse
%</package>
%\fi
%
\endinput
|\\
|\childdocforward{|\textit{main}|}|\\
\end{tabular}
\end{center}
%
or alternatively with:
%
\begin{center}
\begin{tabular}{l}
|% \iffalse
%
% childdoc.dtx Copyright (C) 2017-2018 Niklas Beisert
%
% This work may be distributed and/or modified under the
% conditions of the LaTeX Project Public License, either version 1.3
% of this license or (at your option) any later version.
% The latest version of this license is in
%   http://www.latex-project.org/lppl.txt
% and version 1.3 or later is part of all distributions of LaTeX
% version 2005/12/01 or later.
%
% This work has the LPPL maintenance status `maintained'.
%
% The Current Maintainer of this work is Niklas Beisert.
%
% This work consists of the files childdoc.dtx and childdoc.ins
% and the derived files childdoc.def and cdocsamp.tex with
% cdocsch1.tex, cdocsch2.tex, cdocsdrf.tex, cdocsfn1.tex, cdocsfn2.tex.
%
%<package>\ifdefined\childdocmain\endinput\fi
%<package>\ProvidesFile{childdoc.def}[2018/12/30 v2.0 child document driver]
%<samplemain>\ProvidesFile{cdocsamp.tex}[2018/12/30 v2.0 sample for childdoc]
%<*driver>
%\ProvidesFile{childdoc.drv}[2018/12/30 v2.0 childdoc reference manual file]
\PassOptionsToClass{10pt,a4paper}{article}
\documentclass{ltxdoc}

\usepackage[margin=35mm]{geometry}
\usepackage{hyperref}
\usepackage{hyperxmp}
\usepackage[usenames]{color}

\hypersetup{colorlinks=true}
\hypersetup{pdfstartview=FitH}
\hypersetup{pdfpagemode=UseNone}
\hypersetup{pdfsource={}}
\hypersetup{pdflang={en-UK}}
\hypersetup{pdfcopyright={Copyright 2017-2018 Niklas Beisert.
  This work may be distributed and/or modified under the
  conditions of the LaTeX Project Public License, either version 1.3
  of this license or (at your option) any later version.}}
\hypersetup{pdflicenseurl={http://www.latex-project.org/lppl.txt}}
\hypersetup{pdfcontactaddress={ETH Zurich, ITP, HIT K,
  Wolfgang-Pauli-Strasse 27}}
\hypersetup{pdfcontactpostcode={8093}}
\hypersetup{pdfcontactcity={Zurich}}
\hypersetup{pdfcontactcountry={Switzerland}}
\hypersetup{pdfcontactemail={nbeisert@itp.phys.ethz.ch}}
\hypersetup{pdfcontacturl={http://people.phys.ethz.ch/\xmptilde nbeisert/}}

\newcommand{\secref}[1]{\hyperref[#1]{section \ref*{#1}}}

\parskip1ex
\parindent0pt
\let\olditemize\itemize
\def\itemize{\olditemize\parskip0pt}

\begin{document}

\title{The \textsf{childdoc} Package}
\hypersetup{pdftitle={The childdoc Package}}
\author{Niklas Beisert\\[2ex]
  Institut f\"ur Theoretische Physik\\
  Eidgen\"ossische Technische Hochschule Z\"urich\\
  Wolfgang-Pauli-Strasse 27, 8093 Z\"urich, Switzerland\\[1ex]
  \href{mailto:nbeisert@itp.phys.ethz.ch}
  {\texttt{nbeisert@itp.phys.ethz.ch}}}
\hypersetup{pdfauthor={Niklas Beisert}}
\hypersetup{pdfsubject={Manual for the LaTeX2e Package childdoc}}
\date{30 December 2018, \textsf{v2.0}}
\maketitle

\begin{abstract}\noindent
\textsf{childdoc} is a \LaTeXe{} package
that enables the direct compilation
of document sections included by |\include|
to individual files.
\end{abstract}

\begingroup
\parskip0ex
\tableofcontents
\endgroup

%%%%%%%%%%%%%%%%%%%%%%%%%%%%%%%%%%%%%%%%%%%%%%%%%%%%%%%%%%%%%%%%%%%%%%%%%%%%%%%%
%%%%%%%%%%%%%%%%%%%%%%%%%%%%%%%%%%%%%%%%%%%%%%%%%%%%%%%%%%%%%%%%%%%%%%%%%%%%%%%%
\section{Introduction}

\LaTeX{} provides a mechanism to structure a large document (such as a book)
into a main file and several child files (containing the chapters)
using the |\include| command.
This mechanism is beneficial for documents
which span hundreds of pages in order to
make the source file(s) more manageable.
Moreover, compilation can be restricted to
selected child files by means of the |\includeonly| command.
The latter feature can be used to reduce the compilation time while editing
(this was significantly more useful in the earlier days of \LaTeX{})
or to generate a smaller document which is easier to navigate.
Another application of |\includeonly| is to generate
documents consisting of selected parts of the complete document.

However, there are a few drawbacks of the plain |\include| mechanism:
\begin{itemize}
\item
The child files cannot be compiled on their own,
they can only be compiled via the main file.
A naive editing environment
(such as a text editor with an option
to have the current file processed by \LaTeX)
may require one to switch to the main file before compiling;
attempting to compile the child file produces errors.
\item
The main file must be modified (each time)
to adjust the |\includeonly| command
to the present needs. This easily leaves the main file in a messy state.
\item
The generated document will always carry the filename
of the main document. This is inconvenient if
several child files are to be compiled and
to be kept for distribution.
\end{itemize}

The present package provides a simple interface
to make child files individually compilable by \LaTeX{}.
Compiling a child file then has the same effect as compiling
the main file with an |\includeonly| command
to select the appropriate child.
Moreover the generated document will carry the name of the child
rather than the main file.
This resolves all three above issues.

This feature is meant to make the editing of books,
thesis documents and lecture notes somewhat more convenient.
However, the package can also be used efficiently for
composing a series of documents (such as exercise sheets)
which are typically distributed individually.
It then assists the author in generating the individual documents
(potentially in different versions)
as well as a document containing the collected series.
Another application is in developing style files
or other kinds of included material
where compilation of the style file could redirect
to a sample or test file.

%%%%%%%%%%%%%%%%%%%%%%%%%%%%%%%%%%%%%%%%%%%%%%%%%%%%%%%%%%%%%%%%%%%%%%%%%%%%%%%%
%%%%%%%%%%%%%%%%%%%%%%%%%%%%%%%%%%%%%%%%%%%%%%%%%%%%%%%%%%%%%%%%%%%%%%%%%%%%%%%%
\section{Usage}

First of all, the package \textsf{childdoc} is \emph{not} a standard
\LaTeXe{} |.sty| style file! Therefore it needs to be invoked in
a non-standard way.

%%%%%%%%%%%%%%%%%%%%%%%%%%%%%%%%%%%%%%%%%%%%%%%%%%%%%%%%%%%%%%%%%%%%%%%%%%%%%%%%
\subsection{Included Files}
\label{sec:include}

%%%%%%%%%%%%%%%%%%%%%%%%%%%%%%%%%%%%%%%%
\DescribeMacro{\childdocmain}
To use the package, add the commands
\begin{center}
\begin{tabular}{l}
|\input{childdoc.def}|\\
|\childdocmain{}|\\
\end{tabular}
\end{center}
at the very top of the main \LaTeX{} file,
in particular \emph{before} the |\documentclass| statement!
The argument of |\childdocmain| should be left empty
(but it must be present).

%%%%%%%%%%%%%%%%%%%%%%%%%%%%%%%%%%%%%%%%
\DescribeMacro{\childdocof}
Furthermore, add the commands
\begin{center}
\begin{tabular}{l}
|\input{childdoc.def}|\\
|\childdocof{|\textit{main}|}|\\
\end{tabular}
\end{center}
at the top of every child file \textit{child}
which is included by |\include{|\textit{child}|}|
from within the main file
(or at least for those files to be compiled individually).
The argument \textit{main} must be the filename of the main file.

There are a couple of
considerations in setting up the main and child documents:

%%%%%%%%%%%%%%%%%%%%%%%%%%%%%%%%%%%%%%%%
\paragraph{Restrictions.}

Please note the following restrictions:
\begin{itemize}
\item
|\childdocmain| must be called with one argument \textit{main}
to ensure compatibility with earlier version of the package.
It must either be empty (|\childdocmain{}|)
or precisely match the filename of the main file in which it is specified.
See \secref{sec:detection} for further information.
\item
The filename \textit{main} must be specified without the |.tex| extension.
\item
The filename \textit{main} is case sensitive
(even in case-insensitive file systems)
due to internal string comparison.
\item
The argument \textit{main} should be fully expanded, it cannot be a macro.
\item
Subdirectories and special characters should be avoided in filenames.
\item
The command |\childdocmain{|\textit{main}|}| must be followed by a whitespace.
It should not be followed immediately by another command
or by a comment mark `|%|'.
This is because the \TeX{} parser reads the token immediately following
the argument of |\childdocmain| and puts it
at the beginning of every child section;
however, a white\-space is ignored.
\end{itemize}

%%%%%%%%%%%%%%%%%%%%%%%%%%%%%%%%%%%%%%%%
\paragraph{Content of Main File.}

It is advisable to place all content in the child files included by |\include|.
Any output contained in the main file will appear in all child documents
unless suppressed manually;
it cannot be suppressed automatically by the |\includeonly| directive
and thus should normally be avoided.
A method to include some content in the main file
by means of conditional processing is described in \secref{sec:conditional}.

%%%%%%%%%%%%%%%%%%%%%%%%%%%%%%%%%%%%%%%%
\paragraph{Page Numbering.}

When only a part of the document is compiled,
the appropriate numbering of pages
(as well as other status parameters)
is determined from the |.aux| files.
The latter contain information from previous passes.
However this information needs to propagate through
all intermediate child documents.
Therefore the page numbering in child documents may well
be inconsistent until the complete document is compiled at least once.

A useful (if unconventional) way to always ensure a consistent
page numbering is to restart the numbering in each child document
and denote the pages by `\textit{child}|.|\textit{page}'
where \textit{child} represents the chapter/section number of the child file.
This can be achieved by the command
|\numberwithin{page}{|\textit{child}|}|
of the \textsf{amsmath} package
where \textit{child} can be |chapter| or |section|
depending on the chosen structuring.
Alternatively, one can modify the macro |\thepage| appropriately
and reset the counter |page| at the start of each child file.

%%%%%%%%%%%%%%%%%%%%%%%%%%%%%%%%%%%%%%%%%%%%%%%%%%%%%%%%%%%%%%%%%%%%%%%%%%%%%%%%
\subsection{Conditional Processing}
\label{sec:conditional}

The package provides a mechanism to compile different versions
of a document. To customise the versions further some conditional processing
can come in handy to distinguish which version is being compiled.
The package provides two macros to describe the compilation context:

%%%%%%%%%%%%%%%%%%%%%%%%%%%%%%%%%%%%%%%%
\DescribeMacro{\ifchilddoc}
The conditional |\ifchilddoc| distinguishes between the compilation of
child documents and the main document:
%
\begin{center}
|\ifchilddoc |\textit{child-code}| |[|\||else |\textit{main-code}]| \||fi|
\end{center}

%%%%%%%%%%%%%%%%%%%%%%%%%%%%%%%%%%%%%%%%
\DescribeMacro{\childdocname}
\DescribeMacro{\childdocjob}
The macro |\childdocname| contains the filename (without extension)
of the main or child file being processed.
Note that |\childdocjob| will always contain the name of the main file.

%%%%%%%%%%%%%%%%%%%%%%%%%%%%%%%%%%%%%%%%
\paragraph{Title Page.}

Conditional processing can be used to include a title or banner page
in the main document when proper precautions are taken.
Importantly, the code in the main file should ensure that the page counter
(as well as other status parameters which are stored in the |.aux| files)
takes the same value after the conditional processing.
Otherwise the page numbers may take divergent values
depending on which part is compiled.

For example, a title page could be declared by:
%
\begin{center}
\begin{tabular}{l}
|\ifchilddoc\||else|\\
|\addtocounter{page}{-1}|\\
\textit{code for title page}\\
|\newpage|\\
|\||fi|
\end{tabular}
\end{center}
%
A banner page for the child documents can be generated by:
%
\begin{center}
\begin{tabular}{l}
|\ifchilddoc|\\
|\addtocounter{page}{-1}|\\
\textit{code for banner page}\\
|\newpage|\\
|\||fi|
\end{tabular}
\end{center}
%
Here one could write a message such as:
\begin{center}
|This is the part \childdocname{} of \childdocjob{}.|
\end{center}

%%%%%%%%%%%%%%%%%%%%%%%%%%%%%%%%%%%%%%%%%%%%%%%%%%%%%%%%%%%%%%%%%%%%%%%%%%%%%%%%
\subsection{Flags}
\label{sec:flags}

The package makes it easy to generate different versions
of the main or child documents.
To this end compilation flags can be defined
and assigned different default values.
They will be particularly useful in conjunction
with the forwarding mechanism described in \secref{sec:forward}.

For example, it may be useful to have a flag |\version|
which can be set to |draft| or |final|.
The document source will contain some conditional code
depending on the value of |\version|.
Suppose further, the flag should default to |final| for the main file
and to |draft| for child files
which is a natural assignment for editing the document.
This is achieved by placing the following code
in the preamble of the main document
(below the |\childdocmain| directive):
%
\begin{center}
\begin{tabular}{l}
|\ifchilddoc|\\
|\providecommand{\version}{draft}|\\
|\||else|\\
|\providecommand{\version}{final}|\\
|\||fi|
\end{tabular}
\end{center}
%
The definition by |\providecommand| makes sure
that previous definitions are not overwritten.
Further statements |\providecommand{\version}{...}|
can thus be added before the above code to override it.

For the main file, one might add a line
(between |\childdocmain| and the above block)
%
\begin{center}
|%\ifchilddoc\||else\providecommand{\version}{draft}\||fi|
\end{center}
%
which can be uncommented to produce a draft version.
Likewise one can add a line to the very top of a child file
(above the |\childdocof{|\textit{main}|}| directive)
%
\begin{center}
|%\providecommand{\version}{final}|
\end{center}
%
which can be uncommented to produce the final version of this child document.

%%%%%%%%%%%%%%%%%%%%%%%%%%%%%%%%%%%%%%%%%%%%%%%%%%%%%%%%%%%%%%%%%%%%%%%%%%%%%%%%
\subsection{Forwarding}
\label{sec:forward}

Different versions of the main or child documents
using compilation flags as described in \secref{sec:flags}
can be (permanently) stored in different files
for convenient compilation, viewing and distribution.
To this end, the package defines a command
to pass on compilation to a different file:

%%%%%%%%%%%%%%%%%%%%%%%%%%%%%%%%%%%%%%%%
\DescribeMacro{\childdocforward}
The command |\childdocforward| redirects processing to
another source file:
%
\begin{center}
\begin{tabular}{l}
|\input{childdoc.def}|\\
|\childdocforward[|\textit{main}|]{|\textit{dest}|}|\\
\end{tabular}
\end{center}
%
The argument \textit{dest} is the destination file
(without extension).
It should be the main file or one of the child files.
Note that further \textsf{childdoc} directives
such as |\childdocof| and |\childdocforward|
in the indicated file will be processed in this form.
The optional argument \textit{main}
passes on directly to the main file \textit{main}
while pretending to compile the child \textit{dest}.
This form behaves as if \textit{dest}
issues |\childdocof{|\textit{main}|}| right away,
and no further \textsf{childdoc} directives will be processed.

%%%%%%%%%%%%%%%%%%%%%%%%%%%%%%%%%%%%%%%%
\DescribeMacro{\...prefix}
In the alternative form |\childdocforwardprefix|,
%
\begin{center}
\begin{tabular}{l}
|\input{childdoc.def}|\\
|\childdocforwardprefix[|\textit{main}|]{|\textit{prefix}|}{|\textit{dest}|}|
\end{tabular}
\end{center}
%
the destination file is determined by a pattern
depending on the current file:
To make this work, the current file must be called
`{\textit{prefix}\hspace{0.2em}\textit{suffix}}'
with \textit{prefix} matching precisely the argument.
Processing is then passed on to the file
`{\textit{dest}\hspace{0.2em}\textit{suffix}}'.
Surely, the same effect is achieved by
directly specifying the
argument `{\textit{dest}\hspace{0.2em}\textit{suffix}}'
in the first form.
However, that requires to set up a different file
for each child. With the alternative form of the command
all these files can have exactly the same content
which simplifies setting them up and maintaining them.

For example, the following file |draft.tex|
with a compilation flag |\version| as described in \secref{sec:flags}
compiles the main document as a draft:
%
\begin{center}
\begin{tabular}{l}
|\def\version{draft}|\\
|\input{childdoc.def}|\\
|\childdocforward{|\textit{main}|}|
\end{tabular}
\end{center}
%
Likewise, the following files |final|\textit{nn}|.tex|
compile the final version of the child document
|child|\textit{nn}|.tex|:
%
\begin{center}
\begin{tabular}{l}
|\def\version{final}|\\
|\input{childdoc.def}|\\
|\childdocforwardprefix{final}{child}|
\end{tabular}
\end{center}
%

Note that when several versions of a main file and/or of each child file
are to be generated, it may be convenient to set up a |Makefile| or
shell script to automatise the process.

%%%%%%%%%%%%%%%%%%%%%%%%%%%%%%%%%%%%%%%%%%%%%%%%%%%%%%%%%%%%%%%%%%%%%%%%%%%%%%%%
\subsection{Command Line Processing}
\label{sec:commandline}

The effect of redirection files can also be achieved by invoking
the \LaTeX{} compiler with a more elaborate command line.
Most conveniently this should be done as part
of a shell script or a |Makefile|.

When using \textsf{childdoc} in the main file, the following
command lines effectively perform a redirection
(note that depending on the shell being used,
backslashes may have to be doubled: `|\|' $\to$ `|\\|'):
%
\begin{center}
|... -jobname "|\textit{target}|" |\\|"|[\textit{flags}]%
|\input{childdoc.def}\childdocforward[|\textit{main}|]{|\textit{dest}|}"|
\end{center}
%
Here \textit{target} is the name of the output file,
\textit{main} is the name of the main file
and \textit{dest} is the name of the main or child file to be processed
(all filenames without extensions).
The optional argument \textit{main} can be omitted
if \textit{main} matches \textit{dest}.
Optionally, compilation \textit{flags} can be defined via |\def| commands.
This command line makes the \TeX{} engine believe
it is compiling the file \textit{target}
whose content is specified as the latter parameter.
The provided code then forwards the processing to
\textit{main} or \textit{dest} as described in \secref{sec:forward}.

%%%%%%%%%%%%%%%%%%%%%%%%%%%%%%%%%%%%%%%%%%%%%%%%%%%%%%%%%%%%%%%%%%%%%%%%%%%%%%%%
\subsection{Include by Input}
\label{sec:input}

Including child documents by |\include| has some restrictions by design.
Most notably, the content of a child document always occupies
its own set of pages; pages cannot be shared between child documents.
Usually, this behaviour makes perfect sense
because each child document contain an essential part of the document.
However, in some situations it may be desirable to compose
a document from a collection of parts
without having mandatory page breaks between then.
For this case, the package
provides a mechanism to include parts
by |\input| which can also be processed individually.
However, by construction this mechanism
requires manual handling of the content to be output.

%%%%%%%%%%%%%%%%%%%%%%%%%%%%%%%%%%%%%%%%
\DescribeMacro{\ifchilddocmanual}
The main file should be prepared as usual, see \secref{sec:include}.
However, the document body must make a distinction
between processing of an individual part and of the main document, e.g.:
%
\begin{center}
\begin{tabular}{l}
|\ifchilddocmanual|\\
|\input{\childdocname}|\\
|\||else|\\
\textit{document body with }|\input{|\textit{part}|}|\\
|\||fi|
\end{tabular}
\end{center}
%
The conditional |\ifchilddocmanual| is true whenever
a part to be included by |\input| is being compiled,
and the name of the part is stored in |\childdocname|.

%%%%%%%%%%%%%%%%%%%%%%%%%%%%%%%%%%%%%%%%
\DescribeMacro{\childdocby}
Each part to be included by |\input| should start with:
%
\begin{center}
\begin{tabular}{l}
|\input{childdoc.def}|\\
|\childdocby{|\textit{main}|}|\\
\end{tabular}
\end{center}
%
The directive |\childdocby| is similar to |\childdocof|
described in \secref{sec:include},
but the subsequent selection of content must be done manually.
To that end, both |\ifchilddoc| and |\ifchilddocmanual|
will be true upon processing of a part,
and the name of the part is stored in |\childdocname|.
Note that |\jobname| will be set to the filename of the current part
so that each part receives an individual |.aux| file
that does not interfere with the |.aux| file(s) of the main document.
This behaviour can be altered by the alternative form
|\childdocby[*]{|\textit{main}|}| (with a non-empty optional argument)
which uses the |.aux| file of the main document
by setting |\jobname| to \textit{main}.

%%%%%%%%%%%%%%%%%%%%%%%%%%%%%%%%%%%%%%%%%%%%%%%%%%%%%%%%%%%%%%%%%%%%%%%%%%%%%%%%
\subsection{Driver Development}
\label{sec:driver}

The \textsf{childdoc} mechanism can also be use for the development
of definition files such as \LaTeX{} styles or classes.
This case differs from the above setup with multiple parts
included by |\include| in that no |\includeonly| should be invoked.
This can be achieved by starting the include file
(before |\ProvidesPackage|) with:
%
\begin{center}
\begin{tabular}{l}
|\input{childdoc.def}|\\
|\childdocforward{|\textit{main}|}|\\
\end{tabular}
\end{center}
%
or alternatively with:
%
\begin{center}
\begin{tabular}{l}
|\input{childdoc.def}|\\
|\childdocby{|\textit{main}|}|\\
\end{tabular}
\end{center}
%
Both forms have slightly different effects as described above.
The main file is prepared as usual, see \secref{sec:include}.

%%%%%%%%%%%%%%%%%%%%%%%%%%%%%%%%%%%%%%%%%%%%%%%%%%%%%%%%%%%%%%%%%%%%%%%%%%%%%%%%
\subsection{Legacy Detection}
\label{sec:detection}

The directive |\childdocmain| in the main file can detect
whether the complete document or merely a child is to be compiled
even without using the directive |\childdocof|.
This method is deprecated because it is less robust
and there is no compelling reason to use it;
it is merely provided for backward compatibility
and it may be removed in future versions.

If the detection mechanism is to be used,
it is mandatory to correctly specify
the filename of the main file as the argument of |\childdocmain|:
%
\begin{center}
\begin{tabular}{l}
|\input{childdoc.def}|\\
|\childdocmain{|\textit{main}|}|\\
\end{tabular}
\end{center}
%
If |\jobname| does not match the argument \textit{main} of |\childdocmain|,
it is assumed that |\jobname| points to the child file to be compiled.
When using |\childdocmain| with the main file specified as argument,
it suffices to start a child file
with just |\input{|\textit{main}|}|
without loading of the package and using |\childdocof|.
If instead all processing is done
with the appropriate \textsf{childdoc} directives,
the argument of \textit{main} of |\childdocmain| can be empty.

An alternative version of the command line processing described
in \secref{sec:commandline} using the detection mechanism reads:
%
\begin{center}
|... -jobname "|\textit{target}|" "|[\textit{flags}]%
[|\def\jobname{|\textit{dest}|}|]|\input{|\textit{main}|}"|
\end{center}

%%%%%%%%%%%%%%%%%%%%%%%%%%%%%%%%%%%%%%%%%%%%%%%%%%%%%%%%%%%%%%%%%%%%%%%%%%%%%%%%
\subsection{Manual Code}
\label{sec:manual}

In case one cannot be certain whether the definitions file |childdoc.def|
is installed on the target \TeX{} distribution
and one prefers not to ship it,
it is conceivable to paste a few relevant commands into the sources.

To that end, drop all statements |\input{childdoc.def}|
and perform the replacements as outlined below.
Instead of |\childdocmain{|\textit{main}|}| add the following code
to the top of the main file:
%
\begin{center}
\begin{tabular}{l}
|\||ifdefined\childdocname\endinput\||fi\newif\ifchilddoc|\\
|\edef\childdocname{\scantokens\expandafter{\jobname\noexpand}}|\\
|\def\childdocmain{|\textit{main}|}\||ifx\childdocmain\childdocname\||else|\\
|\childdoctrue\includeonly{\childdocname}\let\jobname\childdocmain\||fi|\\
\end{tabular}
\end{center}
%
Instead of |\childdocof{|\textit{main}|}| just include the main file
at the top of each child file:
%
\begin{center}
|\input{|\textit{main}|}|
\end{center}
%
A simple redirection |\childdocforward{|\textit{dest}|}| is achieved by:
%
\begin{center}
|\def\jobname{|\textit{dest}|}\input{\jobname}|
\end{center}
%
The redirection with prefix
|\childdocforwardprefix[|\textit{prefix}|]{|\textit{dest}|}|
is accomplished by:
%
\begin{center}
\begin{tabular}{l}
|{\edef\jobname{\scantokens\expandafter{\jobname\noexpand}}|\\
|\def\redirectjob |\textit{prefix}|#1~~~{\gdef\jobname{|\textit{dest}|#1}}|\\
|\expandafter\redirectjob\jobname~~~}\input{\jobname}|
\end{tabular}
\end{center}

In an alternative approach,
child documents can be compiled by a specific command line
without additional code or specific definitions:
%
\begin{center}
|... -jobname "|\textit{target}|" "|[\textit{flags}]%
|\includeonly{|\textit{dest}|}\input{|\textit{main}|}"|
\end{center}
%

%%%%%%%%%%%%%%%%%%%%%%%%%%%%%%%%%%%%%%%%%%%%%%%%%%%%%%%%%%%%%%%%%%%%%%%%%%%%%%%%
%%%%%%%%%%%%%%%%%%%%%%%%%%%%%%%%%%%%%%%%%%%%%%%%%%%%%%%%%%%%%%%%%%%%%%%%%%%%%%%%
\section{Information}

%%%%%%%%%%%%%%%%%%%%%%%%%%%%%%%%%%%%%%%%%%%%%%%%%%%%%%%%%%%%%%%%%%%%%%%%%%%%%%%%
\subsection{Copyright}

Copyright \copyright{} 2017--2018 Niklas Beisert

This work may be distributed and/or modified under the
conditions of the \LaTeX{} Project Public License, either version 1.3
of this license or (at your option) any later version.
The latest version of this license is in
  \url{http://www.latex-project.org/lppl.txt}
and version 1.3 or later is part of all distributions of \LaTeX{}
version 2005/12/01 or later.

This work has the LPPL maintenance status `maintained'.

The Current Maintainer of this work is Niklas Beisert.

This work consists of the files |README.txt|, |childdoc.ins| and |childdoc.dtx|
as well as the derived files |childdoc.def|, |cdocsamp.tex|
with |cdocsch1.tex|, |cdocsch2.tex|, |cdocspt3.tex|, |cdocspt4.tex|,
|cdocsdrf.tex|, |cdocsfn1.tex|, |cdocsfn2.tex|
as well as |childdoc.pdf|.

%%%%%%%%%%%%%%%%%%%%%%%%%%%%%%%%%%%%%%%%%%%%%%%%%%%%%%%%%%%%%%%%%%%%%%%%%%%%%%%%
\subsection{Files and Installation}

The package consists of the files:
%
\begin{center}
\begin{tabular}{ll}
    |README.txt|   & readme file \\
    |childdoc.ins| & installation file \\
    |childdoc.dtx| & source file \\
    |childdoc.def| & definition file \\
    |cdocsamp.tex| & sample main file \\
    |cdocsch1.tex| & sample include file \\
    |cdocsch2.tex| & sample include file \\
    |cdocspt3.tex| & sample part file \\
    |cdocspt4.tex| & sample part file \\
    |cdocsdrf.tex| & sample redirection file \\
    |cdocsfn1.tex| & sample redirection file \\
    |cdocsfn2.tex| & sample redirection file \\
    |childdoc.pdf| & manual
\end{tabular}
\end{center}
%
The distribution consists of the files
|README.txt|, |childdoc.ins| and |childdoc.dtx|.
%
\begin{itemize}
\item
Run (pdf)\LaTeX{} on |childdoc.dtx|
to compile the manual |childdoc.pdf| (this file).
\item
Run \LaTeX{} on |childdoc.ins| to create the definitions file |childdoc.def|
and the sample |cdocsamp.tex| with include files
|cdocsch1.tex|, |cdocsch2.tex|, |cdocspt3.tex|, |cdocspt4.tex|,
|cdocsdrf.tex|, |cdocsfn1.tex|, |cdocsfn2.tex|.
Then copy the file |childdoc.def| to an appropriate directory of your \LaTeX{}
distribution, e.g.\ \textit{texmf-root}|/tex/latex/childdoc|.
\end{itemize}

%%%%%%%%%%%%%%%%%%%%%%%%%%%%%%%%%%%%%%%%%%%%%%%%%%%%%%%%%%%%%%%%%%%%%%%%%%%%%%%%
\subsection{Related CTAN Packages}

There are several other packages which offer a similar functionality:
%
\begin{itemize}
\item
The packages
\href{http://ctan.org/pkg/docmute}{\textsf{docmute}},
\href{http://ctan.org/pkg/includex}{\textsf{includex}} and
\href{http://ctan.org/pkg/standalone}{\textsf{standalone}}
provide commands to include only the document body of
a child file thus allowing both files to be compiled individually.
\item
The packages \href{http://ctan.org/pkg/subdocs}{\textsf{subdocs}}
and \href{http://ctan.org/pkg/subfiles}{\textsf{subfiles}}
provide structures in which the main and child documents can be
encapsulated and allowing them to be compiled individually.
The inclusion mechanism is different from the conventional |\include|.
\item
The package \href{http://ctan.org/pkg/combine}{\textsf{combine}}
is an elaborate solution to combine several documents into one.
\end{itemize}
%
See also the CTAN topic \href{http://ctan.org/topic/subdocs}{\textsf{subdocs}}
for further related packages.
The present package differs from the above solutions in that
a document structure constructed with the conventional |\include| mechanism
just needs two extra commands at the top of every file
such that all constituent files can be compiled individually.

%%%%%%%%%%%%%%%%%%%%%%%%%%%%%%%%%%%%%%%%%%%%%%%%%%%%%%%%%%%%%%%%%%%%%%%%%%%%%%%%
%\subsection{Feature Suggestions}
%
%The following is a list of features which may be useful for future
%versions of this package:
%%
%\begin{itemize}
%\item
%\ldots
%\end{itemize}

%%%%%%%%%%%%%%%%%%%%%%%%%%%%%%%%%%%%%%%%%%%%%%%%%%%%%%%%%%%%%%%%%%%%%%%%%%%%%%%%
\subsection{Revision History}

%%%%%%%%%%%%%%%%%%%%%%%%%%%%%%%%%%%%%%%%
\paragraph{v2.0:} 2018/12/30

\begin{itemize}
\item
immediate forward processing
\item
added |\childdocby| mechanism
\item
manual restructured
\end{itemize}

%%%%%%%%%%%%%%%%%%%%%%%%%%%%%%%%%%%%%%%%
\paragraph{v1.6:} 2018/01/17

\begin{itemize}
\item
application for development of include files
\item
corrections to manual
\end{itemize}

%%%%%%%%%%%%%%%%%%%%%%%%%%%%%%%%%%%%%%%%
\paragraph{v1.5:} 2017/05/21

\begin{itemize}
\item
more complete structuring introduced
\item
|\childdocof| introduced
\item
|\childdoc| renamed to |\childdocmain|
\item
|\childredirect| renamed to |\childdocforward| and |\childdocforwardprefix|
and functionality expanded
\end{itemize}

%%%%%%%%%%%%%%%%%%%%%%%%%%%%%%%%%%%%%%%%
\paragraph{v1.0:} 2017/04/27

\begin{itemize}
\item
manual and install package
\item
first version published on CTAN
\end{itemize}

%%%%%%%%%%%%%%%%%%%%%%%%%%%%%%%%%%%%%%%%
\paragraph{v0.6:} 2017/04/26

\begin{itemize}
\item
redirection mechanism added
\end{itemize}

%%%%%%%%%%%%%%%%%%%%%%%%%%%%%%%%%%%%%%%%
\paragraph{v0.5:} 2017/04/26

\begin{itemize}
\item
functionality in definition file
\end{itemize}


%%%%%%%%%%%%%%%%%%%%%%%%%%%%%%%%%%%%%%%%%%%%%%%%%%%%%%%%%%%%%%%%%%%%%%%%%%%%%%%%
%%%%%%%%%%%%%%%%%%%%%%%%%%%%%%%%%%%%%%%%%%%%%%%%%%%%%%%%%%%%%%%%%%%%%%%%%%%%%%%%
%%%%%%%%%%%%%%%%%%%%%%%%%%%%%%%%%%%%%%%%%%%%%%%%%%%%%%%%%%%%%%%%%%%%%%%%%%%%%%%%
\appendix

\settowidth\MacroIndent{\rmfamily\scriptsize 000\ }

 \DocInput{childdoc.dtx}

\end{document}
%</driver>
% \fi
%
% %%%%%%%%%%%%%%%%%%%%%%%%%%%%%%%%%%%%%%%%%%%%%%%%%%%%%%%%%%%%%%%%%%%%%%%%%%%%%%
% %%%%%%%%%%%%%%%%%%%%%%%%%%%%%%%%%%%%%%%%%%%%%%%%%%%%%%%%%%%%%%%%%%%%%%%%%%%%%%
% \section{Sample}
%\iffalse
%<*samplemain>
%\fi
%
% The following presents a sample document
% with two chapters, two parts, a title page,
% a compile flag as well as three forwarding files to set the flag.
% It consists of eight |.tex| files:
% \begin{center}
% \begin{tabular}{ll}
% |cdocsamp.tex|&main file\\
% |cdocsch1.tex|&include file for chapter 1\\
% |cdocsch2.tex|&include file for chapter 2\\
% |cdocspt3.tex|&include file for part 3\\
% |cdocspt4.tex|&include file for part 4\\
% |cdocsdrf.tex|&forwarding file for main file in draft mode\\
% |cdocsfi1.tex|&forwarding file for final version of chapter 1\\
% |cdocsfi2.tex|&forwarding file for final version of chapter 2\\
% \end{tabular}
% \end{center}
% Each of the eight files can be compiled directly by the \LaTeX{} compiler.
%
% %%%%%%%%%%%%%%%%%%%%%%%%%%%%%%%%%%%%%%
% \paragraph{Main File.}
%
% The main file is called |cdocsamp.tex|.
%
% Load the \textsf{childdoc} definitions and
% declare the filename for the main document:
%    \begin{macrocode}
\input{childdoc.def}
\childdocmain{}
%    \end{macrocode}

% Optional override for |\version| flag:
%    \begin{macrocode}
%%\ifchilddoc\else\providecommand{\version}{draft}\fi
%    \end{macrocode}

% Define the default values for the |\version| flag
% (|final| for the main file and |draft| for childs):
%    \begin{macrocode}
\ifchilddoc
\providecommand{\version}{draft}
\else
\providecommand{\version}{final}
\fi
%    \end{macrocode}

% Load the standard document class:
%    \begin{macrocode}
\documentclass[12pt]{article}
%    \end{macrocode}

% Start the document body:
%    \begin{macrocode}
\begin{document}
%    \end{macrocode}

% Declare a title page.
% Print title, part of document being processed and version flag:
%    \begin{macrocode}
\addtocounter{page}{-1}
\begin{center}
{\LARGE\bfseries{}childdoc example\par}
\vspace{1cm}
\ifchilddoc
\ifchilddocmanual part\else chapter\fi:
`\childdocname' of `\childdocjob'\par
\else
main document: `\childdocjob'\par
\fi
version: \version\par
\end{center}
\newpage
%    \end{macrocode}

% Manually include selected file,
% otherwise process as usual:
%    \begin{macrocode}
\ifchilddocmanual
\section*{part `\childdocname'}
\input{\childdocname}
\else
%    \end{macrocode}

% Include the two chapters:
%    \begin{macrocode}
\include{cdocsch1}
\include{cdocsch2}
%    \end{macrocode}

% Include the two parts unless only chapters should be displayed:
%    \begin{macrocode}
\ifchilddoc\else
\section{part three}
\input{cdocspt3}
\section{part four}
\input{cdocspt4}
\fi
%    \end{macrocode}

% Process as usual until here:
%    \begin{macrocode}
\fi
%    \end{macrocode}

% End of document body:
%    \begin{macrocode}
\end{document}
%    \end{macrocode}
%\iffalse
%</samplemain>
%\fi
%
% %%%%%%%%%%%%%%%%%%%%%%%%%%%%%%%%%%%%%%
% \paragraph{Chapter Include Files.}
%
% The include files are called |cdocsch1.tex| and |cdocsch2.tex|.
%
%\iffalse
%<*samplechap1|samplechap2>
%\fi

% Optional override for |\version| flag:
%    \begin{macrocode}
%%\providecommand{\version}{final}
%    \end{macrocode}

% Include the main document:
%    \begin{macrocode}
\input{childdoc.def}
\childdocof{cdocsamp}
%    \end{macrocode}

%\iffalse
%</samplechap1|samplechap2>
%\fi
%
%\iffalse
%<*samplechap1>
%\fi
% Some text for chapter 1:
%    \begin{macrocode}
\section{one}
some text in chapter one
%    \end{macrocode}

%\iffalse
%</samplechap1>
%\fi
% Some text for chapter 2:
%\iffalse
%<*samplechap2>
%\fi
%    \begin{macrocode}
\section{two}
more text in chapter two
%    \end{macrocode}

%\iffalse
%</samplechap2>
%\fi
%
% %%%%%%%%%%%%%%%%%%%%%%%%%%%%%%%%%%%%%%
% \paragraph{Part Include Files.}
%
% The include files are called |cdocspt3.tex| and |cdocspt4.tex|.
%
%\iffalse
%<*samplepart3|samplepart4>
%\fi

% Optional override for |\version| flag:
%    \begin{macrocode}
%%\providecommand{\version}{final}
%    \end{macrocode}

% Include the main document:
%    \begin{macrocode}
\input{childdoc.def}
\childdocby{cdocsamp}
%    \end{macrocode}

%\iffalse
%</samplepart3|samplepart4>
%\fi
%
%\iffalse
%<*samplepart3>
%\fi
% Some text for part 3:
%    \begin{macrocode}
some text in part three
%    \end{macrocode}

%\iffalse
%</samplepart3>
%\fi
% Some text for part 4:
%\iffalse
%<*samplepart4>
%\fi
%    \begin{macrocode}
more text in part four
%    \end{macrocode}

%\iffalse
%</samplepart4>
%\fi
%
% %%%%%%%%%%%%%%%%%%%%%%%%%%%%%%%%%%%%%%
% \paragraph{Forwarding for a Complete Draft.}
%
% The following forwarding file |cdocsdrf.tex|
% compiles the main document in draft mode:
%\iffalse
%<*sampledraft>
%\fi
%    \begin{macrocode}
\def\version{draft}
\input{childdoc.def}
\childdocforward{cdocsamp}
%    \end{macrocode}

%\iffalse
%</sampledraft>
%\fi
%
% %%%%%%%%%%%%%%%%%%%%%%%%%%%%%%%%%%%%%%
% \paragraph{Forwarding for Final Version of the Chapters.}
%
% The following forwarding files |cdocsfn1.tex| and |cdocsfn2.tex|
% (with identical content)
% compile the final versions of the child documents
% |cdocsch1.tex| and |cdocsch2.tex|, respectively:
%\iffalse
%<*samplefinal>
%\fi
%    \begin{macrocode}
\def\version{final}
\input{childdoc.def}
\childdocforwardprefix[cdocsamp]{cdocsfn}{cdocsch}
%    \end{macrocode}

%\iffalse
%</samplefinal>
%\fi
%
% %%%%%%%%%%%%%%%%%%%%%%%%%%%%%%%%%%%%%%
% \paragraph{Command Line Processing.}
%
% The following three command lines generate the output files
% |cdocscld|, |cdocscl1| and |cdocscl2|
% which should be identical to
% |cdocsdrf|, |cdocsch1| and |cdocsfn2|, respectively:
% \begin{center}
% \begin{tabular}{l}
% |latex -jobname cdocscld \|\\
% |  "\def\version{draft}\input{childdoc.def}\childdocforward{cdocsamp}"|\\
% |latex -jobname cdocscl1 \|\\
% |  "\input{childdoc.def}\childdocforward[cdocsamp]{cdocsch1}"|\\
% |latex -jobname cdocscl2 \|\\
% |  "\def\version{final}\input{childdoc.def}\childdocforward{cdocsch2}"|
% \end{tabular}
% \end{center}
% Note that the trailing backslash on each first line
% merely continues the input to the second line
% (for convenient cut ant paste).
% Furthermore, the command |latex| can be replaced by any
% of its alternative versions such as |pdflatex|.
%
% %%%%%%%%%%%%%%%%%%%%%%%%%%%%%%%%%%%%%%%%%%%%%%%%%%%%%%%%%%%%%%%%%%%%%%%%%%%%%%
% %%%%%%%%%%%%%%%%%%%%%%%%%%%%%%%%%%%%%%%%%%%%%%%%%%%%%%%%%%%%%%%%%%%%%%%%%%%%%%
% \section{Implementation}
%\iffalse
%<*package>
%\fi
%
% This section describes the definitions file |childdoc.def|.

% The definitions cannot be loaded using |\usepackage| or |\RequirePackage|
% which has a mechanism to prevent loading a style file more than once.
% When loading the definitions by means of |\input|
% multiple instances have to be prevented manually:
%\iffalse
%This code needs to be before the `\ProvidesFile' directive
%which is defined at the beginning of this file.
%Therefore it is also placed there and commented out here.
%</package>
%<*discard>
%\fi
%    \begin{macrocode}
\ifdefined\childdocmain\endinput\fi
%    \end{macrocode}
%\iffalse
%</discard>
%<*package>
%\fi
%
% \macro{\ifchilddoc}
% \macro{\ifchilddocmanual}
% The conditional |\ifchilddoc| tells whether a
% child (true) or main (false) document is being compiled.
% The conditional |\ifchilddocmanual| tells whether
% the |\includeonly| mechanism is used (false) or
% the selection of child files must be performed manually (true).
% The definitions initialise to false:
%    \begin{macrocode}
\newif\ifchilddoc
\newif\ifchilddocmanual
%    \end{macrocode}

% \macro{\childdocname}
% \macro{\childdocjob}
% The macro |\childdocname| stores the name of the main document
% to be compiled. The macro |\childdocjob| stores the name of
% the document on which the \LaTeX{} compiler was originally invoked.
% The content of |\jobname| cannot be compared
% to filenames specified in the source due to different catcodes.
% The following code rescans |\jobname|, stores the result
% in |\childdocname| and saves a copy in |\childdocjob|:
%    \begin{macrocode}
\edef\childdocname{\scantokens\expandafter{\jobname\noexpand}}
\let\childdocjob\childdocname
%    \end{macrocode}

% \macro{\childdocdisable}
% The macro |\childdocdisable| prevents the main file
% from being processed more than once.
% At this stage, the main document command |\childdocmain|
% is assumed to be called once again where it should do nothing.
% Any subsequent call to it should prevent
% a secondary processing of the main document
% It overwrites the forwarding commands
% |\childdocof| and |\childdocforward|
% with empty macros to prevent further inclusions of the main document:
%    \begin{macrocode}
\newcommand{\childdocdisable}
{
  \renewcommand{\childdocmain}[1]{\renewcommand{\childdocmain}[1]{\endinput}}
  \renewcommand{\childdocof}[1]{}
  \renewcommand{\childdocby}[2][]{}
  \renewcommand{\childdocforward}[2][]{}
  \renewcommand{\childdocdisable}{}
}
%    \end{macrocode}

% \macro{\childdocmain}
% The macro |\childdocmain| is to be called at the top of the main file
% with nothing or the main filename (without extension) as argument.
% First, it breaks loops.
% If the argument is not empty and does not match |\childdocname|
% (which is set by the first inclusion of |childdoc.def|),
% |\ifchilddoc| is set to true, |\includeonly| is applied to the child file
% and |\jobname| is set to the main file
% (for proper handling of |.aux| files):
%    \begin{macrocode}
\newcommand{\childdocmain}[1]
{
  \childdocdisable\childdocmain{}
  \if?#1?\else
    \begingroup
      \def\childdoctmp{#1}
      \ifx\childdoctmp\childdocname
        \def\childdoctmp{}
      \else
        \def\childdoctmp
        {
          \childdoctrue
          \includeonly{\childdocname}
          \def\childdocjob{#1}
          \def\jobname{#1}
        }
      \fi
      \expandafter
    \endgroup
    \childdoctmp
  \fi
}
%    \end{macrocode}

% \macro{\childdocof}
% The command |\childdocof| redirects
% compilation to the main file |#1|.
%    \begin{macrocode}
\newcommand{\childdocof}[1]
{
  \childdocdisable
  \childdoctrue
  \includeonly{\childdocname}
  \def\jobname{#1}
  \def\childdocjob{#1}
  \input{#1}
}
%    \end{macrocode}

% \macro{\childdocby}
% The command |\childdocby| ....
%    \begin{macrocode}
\newcommand{\childdocby}[2][]
{
  \childdocdisable
  \childdoctrue
  \childdocmanualtrue
  \if?#1?\else
    \def\jobname{#2}
  \fi
  \def\childdocjob{#2}
  \input{#2}
  \endinput
}
%    \end{macrocode}

% \macro{\childdocforward}
% The command |\childdocforward| redirects
% compilation to the main file or
% (if the optional argument is given) a child file.
% Parameters are set as if the main file
% or a child file starting with |\childdocof| was compiled.
% Then compilation is handed over to the main file:
%    \begin{macrocode}
\newcommand{\childdocforward}[2][]
{
  \begingroup
    \if?#1?
      \def\childdoctmp
      {
        \def\childdocname{#2}
        \def\childdocjob{#2}
        \def\jobname{#2}
        \input{#2}
        \endinput
      }
    \else
      \def\childdoctmp
      {
        \childdocdisable
        \def\childdocname{#2}
        \childdoctrue
        \includeonly{#2}
        \def\childdocjob{#1}
        \def\jobname{#1}
        \input{#1}
        \endinput
      }
    \fi
    \expandafter
  \endgroup
  \childdoctmp
}
%    \end{macrocode}

% \macro{\childdocforwardprefix}
% The command |\childdocforwardprefix| redirects
% compilation to the main or a child file by means of a pattern.
% The prefix |#1| in the current filename is replaced by |#2|
% and the suffix of the current filename is kept
% (it is assumed that the filename does not contain the substring `|~~~|'
% which is used as a delimiter).
% Compilation is handed over to the new file by |\childdocforward|:
%    \begin{macrocode}
\newcommand{\childdocforwardprefix}[3][]
{
  \begingroup
    \def\childdocextract #2##1~~~{\def\childdoctmp{\childdocforward[#1]{#3##1}}}
    \expandafter\childdocextract\childdocname~~~
    \expandafter
  \endgroup
  \childdoctmp
}
%    \end{macrocode}

% \macro{\childdoc}
% The deprecated macro |\childdoc| is a legacy version of |\childdocmain|:
%    \begin{macrocode}
\newcommand{\childdoc}{\childdocmain}
%    \end{macrocode}

% \macro{\childdocredirect}
% The deprecated macro |\childdocredirect| is a legacy version
% of |\childdocforward| and |\childdocforwardprefix|:
%    \begin{macrocode}
\newcommand{\childdocredirect}[2][]
{
  \begingroup
    \if?#1?
      \def\childdoctmp{\childdocforward{#2}}
    \else
      \def\childdoctmp{\childdocforwardprefix{#1}{#2}}
    \fi
    \expandafter
  \endgroup
  \childdoctmp
}
%    \end{macrocode}

%\iffalse
%</package>
%\fi
%
\endinput
|\\
|\childdocby{|\textit{main}|}|\\
\end{tabular}
\end{center}
%
Both forms have slightly different effects as described above.
The main file is prepared as usual, see \secref{sec:include}.

%%%%%%%%%%%%%%%%%%%%%%%%%%%%%%%%%%%%%%%%%%%%%%%%%%%%%%%%%%%%%%%%%%%%%%%%%%%%%%%%
\subsection{Legacy Detection}
\label{sec:detection}

The directive |\childdocmain| in the main file can detect
whether the complete document or merely a child is to be compiled
even without using the directive |\childdocof|.
This method is deprecated because it is less robust
and there is no compelling reason to use it;
it is merely provided for backward compatibility
and it may be removed in future versions.

If the detection mechanism is to be used,
it is mandatory to correctly specify
the filename of the main file as the argument of |\childdocmain|:
%
\begin{center}
\begin{tabular}{l}
|% \iffalse
%
% childdoc.dtx Copyright (C) 2017-2018 Niklas Beisert
%
% This work may be distributed and/or modified under the
% conditions of the LaTeX Project Public License, either version 1.3
% of this license or (at your option) any later version.
% The latest version of this license is in
%   http://www.latex-project.org/lppl.txt
% and version 1.3 or later is part of all distributions of LaTeX
% version 2005/12/01 or later.
%
% This work has the LPPL maintenance status `maintained'.
%
% The Current Maintainer of this work is Niklas Beisert.
%
% This work consists of the files childdoc.dtx and childdoc.ins
% and the derived files childdoc.def and cdocsamp.tex with
% cdocsch1.tex, cdocsch2.tex, cdocsdrf.tex, cdocsfn1.tex, cdocsfn2.tex.
%
%<package>\ifdefined\childdocmain\endinput\fi
%<package>\ProvidesFile{childdoc.def}[2018/12/30 v2.0 child document driver]
%<samplemain>\ProvidesFile{cdocsamp.tex}[2018/12/30 v2.0 sample for childdoc]
%<*driver>
%\ProvidesFile{childdoc.drv}[2018/12/30 v2.0 childdoc reference manual file]
\PassOptionsToClass{10pt,a4paper}{article}
\documentclass{ltxdoc}

\usepackage[margin=35mm]{geometry}
\usepackage{hyperref}
\usepackage{hyperxmp}
\usepackage[usenames]{color}

\hypersetup{colorlinks=true}
\hypersetup{pdfstartview=FitH}
\hypersetup{pdfpagemode=UseNone}
\hypersetup{pdfsource={}}
\hypersetup{pdflang={en-UK}}
\hypersetup{pdfcopyright={Copyright 2017-2018 Niklas Beisert.
  This work may be distributed and/or modified under the
  conditions of the LaTeX Project Public License, either version 1.3
  of this license or (at your option) any later version.}}
\hypersetup{pdflicenseurl={http://www.latex-project.org/lppl.txt}}
\hypersetup{pdfcontactaddress={ETH Zurich, ITP, HIT K,
  Wolfgang-Pauli-Strasse 27}}
\hypersetup{pdfcontactpostcode={8093}}
\hypersetup{pdfcontactcity={Zurich}}
\hypersetup{pdfcontactcountry={Switzerland}}
\hypersetup{pdfcontactemail={nbeisert@itp.phys.ethz.ch}}
\hypersetup{pdfcontacturl={http://people.phys.ethz.ch/\xmptilde nbeisert/}}

\newcommand{\secref}[1]{\hyperref[#1]{section \ref*{#1}}}

\parskip1ex
\parindent0pt
\let\olditemize\itemize
\def\itemize{\olditemize\parskip0pt}

\begin{document}

\title{The \textsf{childdoc} Package}
\hypersetup{pdftitle={The childdoc Package}}
\author{Niklas Beisert\\[2ex]
  Institut f\"ur Theoretische Physik\\
  Eidgen\"ossische Technische Hochschule Z\"urich\\
  Wolfgang-Pauli-Strasse 27, 8093 Z\"urich, Switzerland\\[1ex]
  \href{mailto:nbeisert@itp.phys.ethz.ch}
  {\texttt{nbeisert@itp.phys.ethz.ch}}}
\hypersetup{pdfauthor={Niklas Beisert}}
\hypersetup{pdfsubject={Manual for the LaTeX2e Package childdoc}}
\date{30 December 2018, \textsf{v2.0}}
\maketitle

\begin{abstract}\noindent
\textsf{childdoc} is a \LaTeXe{} package
that enables the direct compilation
of document sections included by |\include|
to individual files.
\end{abstract}

\begingroup
\parskip0ex
\tableofcontents
\endgroup

%%%%%%%%%%%%%%%%%%%%%%%%%%%%%%%%%%%%%%%%%%%%%%%%%%%%%%%%%%%%%%%%%%%%%%%%%%%%%%%%
%%%%%%%%%%%%%%%%%%%%%%%%%%%%%%%%%%%%%%%%%%%%%%%%%%%%%%%%%%%%%%%%%%%%%%%%%%%%%%%%
\section{Introduction}

\LaTeX{} provides a mechanism to structure a large document (such as a book)
into a main file and several child files (containing the chapters)
using the |\include| command.
This mechanism is beneficial for documents
which span hundreds of pages in order to
make the source file(s) more manageable.
Moreover, compilation can be restricted to
selected child files by means of the |\includeonly| command.
The latter feature can be used to reduce the compilation time while editing
(this was significantly more useful in the earlier days of \LaTeX{})
or to generate a smaller document which is easier to navigate.
Another application of |\includeonly| is to generate
documents consisting of selected parts of the complete document.

However, there are a few drawbacks of the plain |\include| mechanism:
\begin{itemize}
\item
The child files cannot be compiled on their own,
they can only be compiled via the main file.
A naive editing environment
(such as a text editor with an option
to have the current file processed by \LaTeX)
may require one to switch to the main file before compiling;
attempting to compile the child file produces errors.
\item
The main file must be modified (each time)
to adjust the |\includeonly| command
to the present needs. This easily leaves the main file in a messy state.
\item
The generated document will always carry the filename
of the main document. This is inconvenient if
several child files are to be compiled and
to be kept for distribution.
\end{itemize}

The present package provides a simple interface
to make child files individually compilable by \LaTeX{}.
Compiling a child file then has the same effect as compiling
the main file with an |\includeonly| command
to select the appropriate child.
Moreover the generated document will carry the name of the child
rather than the main file.
This resolves all three above issues.

This feature is meant to make the editing of books,
thesis documents and lecture notes somewhat more convenient.
However, the package can also be used efficiently for
composing a series of documents (such as exercise sheets)
which are typically distributed individually.
It then assists the author in generating the individual documents
(potentially in different versions)
as well as a document containing the collected series.
Another application is in developing style files
or other kinds of included material
where compilation of the style file could redirect
to a sample or test file.

%%%%%%%%%%%%%%%%%%%%%%%%%%%%%%%%%%%%%%%%%%%%%%%%%%%%%%%%%%%%%%%%%%%%%%%%%%%%%%%%
%%%%%%%%%%%%%%%%%%%%%%%%%%%%%%%%%%%%%%%%%%%%%%%%%%%%%%%%%%%%%%%%%%%%%%%%%%%%%%%%
\section{Usage}

First of all, the package \textsf{childdoc} is \emph{not} a standard
\LaTeXe{} |.sty| style file! Therefore it needs to be invoked in
a non-standard way.

%%%%%%%%%%%%%%%%%%%%%%%%%%%%%%%%%%%%%%%%%%%%%%%%%%%%%%%%%%%%%%%%%%%%%%%%%%%%%%%%
\subsection{Included Files}
\label{sec:include}

%%%%%%%%%%%%%%%%%%%%%%%%%%%%%%%%%%%%%%%%
\DescribeMacro{\childdocmain}
To use the package, add the commands
\begin{center}
\begin{tabular}{l}
|\input{childdoc.def}|\\
|\childdocmain{}|\\
\end{tabular}
\end{center}
at the very top of the main \LaTeX{} file,
in particular \emph{before} the |\documentclass| statement!
The argument of |\childdocmain| should be left empty
(but it must be present).

%%%%%%%%%%%%%%%%%%%%%%%%%%%%%%%%%%%%%%%%
\DescribeMacro{\childdocof}
Furthermore, add the commands
\begin{center}
\begin{tabular}{l}
|\input{childdoc.def}|\\
|\childdocof{|\textit{main}|}|\\
\end{tabular}
\end{center}
at the top of every child file \textit{child}
which is included by |\include{|\textit{child}|}|
from within the main file
(or at least for those files to be compiled individually).
The argument \textit{main} must be the filename of the main file.

There are a couple of
considerations in setting up the main and child documents:

%%%%%%%%%%%%%%%%%%%%%%%%%%%%%%%%%%%%%%%%
\paragraph{Restrictions.}

Please note the following restrictions:
\begin{itemize}
\item
|\childdocmain| must be called with one argument \textit{main}
to ensure compatibility with earlier version of the package.
It must either be empty (|\childdocmain{}|)
or precisely match the filename of the main file in which it is specified.
See \secref{sec:detection} for further information.
\item
The filename \textit{main} must be specified without the |.tex| extension.
\item
The filename \textit{main} is case sensitive
(even in case-insensitive file systems)
due to internal string comparison.
\item
The argument \textit{main} should be fully expanded, it cannot be a macro.
\item
Subdirectories and special characters should be avoided in filenames.
\item
The command |\childdocmain{|\textit{main}|}| must be followed by a whitespace.
It should not be followed immediately by another command
or by a comment mark `|%|'.
This is because the \TeX{} parser reads the token immediately following
the argument of |\childdocmain| and puts it
at the beginning of every child section;
however, a white\-space is ignored.
\end{itemize}

%%%%%%%%%%%%%%%%%%%%%%%%%%%%%%%%%%%%%%%%
\paragraph{Content of Main File.}

It is advisable to place all content in the child files included by |\include|.
Any output contained in the main file will appear in all child documents
unless suppressed manually;
it cannot be suppressed automatically by the |\includeonly| directive
and thus should normally be avoided.
A method to include some content in the main file
by means of conditional processing is described in \secref{sec:conditional}.

%%%%%%%%%%%%%%%%%%%%%%%%%%%%%%%%%%%%%%%%
\paragraph{Page Numbering.}

When only a part of the document is compiled,
the appropriate numbering of pages
(as well as other status parameters)
is determined from the |.aux| files.
The latter contain information from previous passes.
However this information needs to propagate through
all intermediate child documents.
Therefore the page numbering in child documents may well
be inconsistent until the complete document is compiled at least once.

A useful (if unconventional) way to always ensure a consistent
page numbering is to restart the numbering in each child document
and denote the pages by `\textit{child}|.|\textit{page}'
where \textit{child} represents the chapter/section number of the child file.
This can be achieved by the command
|\numberwithin{page}{|\textit{child}|}|
of the \textsf{amsmath} package
where \textit{child} can be |chapter| or |section|
depending on the chosen structuring.
Alternatively, one can modify the macro |\thepage| appropriately
and reset the counter |page| at the start of each child file.

%%%%%%%%%%%%%%%%%%%%%%%%%%%%%%%%%%%%%%%%%%%%%%%%%%%%%%%%%%%%%%%%%%%%%%%%%%%%%%%%
\subsection{Conditional Processing}
\label{sec:conditional}

The package provides a mechanism to compile different versions
of a document. To customise the versions further some conditional processing
can come in handy to distinguish which version is being compiled.
The package provides two macros to describe the compilation context:

%%%%%%%%%%%%%%%%%%%%%%%%%%%%%%%%%%%%%%%%
\DescribeMacro{\ifchilddoc}
The conditional |\ifchilddoc| distinguishes between the compilation of
child documents and the main document:
%
\begin{center}
|\ifchilddoc |\textit{child-code}| |[|\||else |\textit{main-code}]| \||fi|
\end{center}

%%%%%%%%%%%%%%%%%%%%%%%%%%%%%%%%%%%%%%%%
\DescribeMacro{\childdocname}
\DescribeMacro{\childdocjob}
The macro |\childdocname| contains the filename (without extension)
of the main or child file being processed.
Note that |\childdocjob| will always contain the name of the main file.

%%%%%%%%%%%%%%%%%%%%%%%%%%%%%%%%%%%%%%%%
\paragraph{Title Page.}

Conditional processing can be used to include a title or banner page
in the main document when proper precautions are taken.
Importantly, the code in the main file should ensure that the page counter
(as well as other status parameters which are stored in the |.aux| files)
takes the same value after the conditional processing.
Otherwise the page numbers may take divergent values
depending on which part is compiled.

For example, a title page could be declared by:
%
\begin{center}
\begin{tabular}{l}
|\ifchilddoc\||else|\\
|\addtocounter{page}{-1}|\\
\textit{code for title page}\\
|\newpage|\\
|\||fi|
\end{tabular}
\end{center}
%
A banner page for the child documents can be generated by:
%
\begin{center}
\begin{tabular}{l}
|\ifchilddoc|\\
|\addtocounter{page}{-1}|\\
\textit{code for banner page}\\
|\newpage|\\
|\||fi|
\end{tabular}
\end{center}
%
Here one could write a message such as:
\begin{center}
|This is the part \childdocname{} of \childdocjob{}.|
\end{center}

%%%%%%%%%%%%%%%%%%%%%%%%%%%%%%%%%%%%%%%%%%%%%%%%%%%%%%%%%%%%%%%%%%%%%%%%%%%%%%%%
\subsection{Flags}
\label{sec:flags}

The package makes it easy to generate different versions
of the main or child documents.
To this end compilation flags can be defined
and assigned different default values.
They will be particularly useful in conjunction
with the forwarding mechanism described in \secref{sec:forward}.

For example, it may be useful to have a flag |\version|
which can be set to |draft| or |final|.
The document source will contain some conditional code
depending on the value of |\version|.
Suppose further, the flag should default to |final| for the main file
and to |draft| for child files
which is a natural assignment for editing the document.
This is achieved by placing the following code
in the preamble of the main document
(below the |\childdocmain| directive):
%
\begin{center}
\begin{tabular}{l}
|\ifchilddoc|\\
|\providecommand{\version}{draft}|\\
|\||else|\\
|\providecommand{\version}{final}|\\
|\||fi|
\end{tabular}
\end{center}
%
The definition by |\providecommand| makes sure
that previous definitions are not overwritten.
Further statements |\providecommand{\version}{...}|
can thus be added before the above code to override it.

For the main file, one might add a line
(between |\childdocmain| and the above block)
%
\begin{center}
|%\ifchilddoc\||else\providecommand{\version}{draft}\||fi|
\end{center}
%
which can be uncommented to produce a draft version.
Likewise one can add a line to the very top of a child file
(above the |\childdocof{|\textit{main}|}| directive)
%
\begin{center}
|%\providecommand{\version}{final}|
\end{center}
%
which can be uncommented to produce the final version of this child document.

%%%%%%%%%%%%%%%%%%%%%%%%%%%%%%%%%%%%%%%%%%%%%%%%%%%%%%%%%%%%%%%%%%%%%%%%%%%%%%%%
\subsection{Forwarding}
\label{sec:forward}

Different versions of the main or child documents
using compilation flags as described in \secref{sec:flags}
can be (permanently) stored in different files
for convenient compilation, viewing and distribution.
To this end, the package defines a command
to pass on compilation to a different file:

%%%%%%%%%%%%%%%%%%%%%%%%%%%%%%%%%%%%%%%%
\DescribeMacro{\childdocforward}
The command |\childdocforward| redirects processing to
another source file:
%
\begin{center}
\begin{tabular}{l}
|\input{childdoc.def}|\\
|\childdocforward[|\textit{main}|]{|\textit{dest}|}|\\
\end{tabular}
\end{center}
%
The argument \textit{dest} is the destination file
(without extension).
It should be the main file or one of the child files.
Note that further \textsf{childdoc} directives
such as |\childdocof| and |\childdocforward|
in the indicated file will be processed in this form.
The optional argument \textit{main}
passes on directly to the main file \textit{main}
while pretending to compile the child \textit{dest}.
This form behaves as if \textit{dest}
issues |\childdocof{|\textit{main}|}| right away,
and no further \textsf{childdoc} directives will be processed.

%%%%%%%%%%%%%%%%%%%%%%%%%%%%%%%%%%%%%%%%
\DescribeMacro{\...prefix}
In the alternative form |\childdocforwardprefix|,
%
\begin{center}
\begin{tabular}{l}
|\input{childdoc.def}|\\
|\childdocforwardprefix[|\textit{main}|]{|\textit{prefix}|}{|\textit{dest}|}|
\end{tabular}
\end{center}
%
the destination file is determined by a pattern
depending on the current file:
To make this work, the current file must be called
`{\textit{prefix}\hspace{0.2em}\textit{suffix}}'
with \textit{prefix} matching precisely the argument.
Processing is then passed on to the file
`{\textit{dest}\hspace{0.2em}\textit{suffix}}'.
Surely, the same effect is achieved by
directly specifying the
argument `{\textit{dest}\hspace{0.2em}\textit{suffix}}'
in the first form.
However, that requires to set up a different file
for each child. With the alternative form of the command
all these files can have exactly the same content
which simplifies setting them up and maintaining them.

For example, the following file |draft.tex|
with a compilation flag |\version| as described in \secref{sec:flags}
compiles the main document as a draft:
%
\begin{center}
\begin{tabular}{l}
|\def\version{draft}|\\
|\input{childdoc.def}|\\
|\childdocforward{|\textit{main}|}|
\end{tabular}
\end{center}
%
Likewise, the following files |final|\textit{nn}|.tex|
compile the final version of the child document
|child|\textit{nn}|.tex|:
%
\begin{center}
\begin{tabular}{l}
|\def\version{final}|\\
|\input{childdoc.def}|\\
|\childdocforwardprefix{final}{child}|
\end{tabular}
\end{center}
%

Note that when several versions of a main file and/or of each child file
are to be generated, it may be convenient to set up a |Makefile| or
shell script to automatise the process.

%%%%%%%%%%%%%%%%%%%%%%%%%%%%%%%%%%%%%%%%%%%%%%%%%%%%%%%%%%%%%%%%%%%%%%%%%%%%%%%%
\subsection{Command Line Processing}
\label{sec:commandline}

The effect of redirection files can also be achieved by invoking
the \LaTeX{} compiler with a more elaborate command line.
Most conveniently this should be done as part
of a shell script or a |Makefile|.

When using \textsf{childdoc} in the main file, the following
command lines effectively perform a redirection
(note that depending on the shell being used,
backslashes may have to be doubled: `|\|' $\to$ `|\\|'):
%
\begin{center}
|... -jobname "|\textit{target}|" |\\|"|[\textit{flags}]%
|\input{childdoc.def}\childdocforward[|\textit{main}|]{|\textit{dest}|}"|
\end{center}
%
Here \textit{target} is the name of the output file,
\textit{main} is the name of the main file
and \textit{dest} is the name of the main or child file to be processed
(all filenames without extensions).
The optional argument \textit{main} can be omitted
if \textit{main} matches \textit{dest}.
Optionally, compilation \textit{flags} can be defined via |\def| commands.
This command line makes the \TeX{} engine believe
it is compiling the file \textit{target}
whose content is specified as the latter parameter.
The provided code then forwards the processing to
\textit{main} or \textit{dest} as described in \secref{sec:forward}.

%%%%%%%%%%%%%%%%%%%%%%%%%%%%%%%%%%%%%%%%%%%%%%%%%%%%%%%%%%%%%%%%%%%%%%%%%%%%%%%%
\subsection{Include by Input}
\label{sec:input}

Including child documents by |\include| has some restrictions by design.
Most notably, the content of a child document always occupies
its own set of pages; pages cannot be shared between child documents.
Usually, this behaviour makes perfect sense
because each child document contain an essential part of the document.
However, in some situations it may be desirable to compose
a document from a collection of parts
without having mandatory page breaks between then.
For this case, the package
provides a mechanism to include parts
by |\input| which can also be processed individually.
However, by construction this mechanism
requires manual handling of the content to be output.

%%%%%%%%%%%%%%%%%%%%%%%%%%%%%%%%%%%%%%%%
\DescribeMacro{\ifchilddocmanual}
The main file should be prepared as usual, see \secref{sec:include}.
However, the document body must make a distinction
between processing of an individual part and of the main document, e.g.:
%
\begin{center}
\begin{tabular}{l}
|\ifchilddocmanual|\\
|\input{\childdocname}|\\
|\||else|\\
\textit{document body with }|\input{|\textit{part}|}|\\
|\||fi|
\end{tabular}
\end{center}
%
The conditional |\ifchilddocmanual| is true whenever
a part to be included by |\input| is being compiled,
and the name of the part is stored in |\childdocname|.

%%%%%%%%%%%%%%%%%%%%%%%%%%%%%%%%%%%%%%%%
\DescribeMacro{\childdocby}
Each part to be included by |\input| should start with:
%
\begin{center}
\begin{tabular}{l}
|\input{childdoc.def}|\\
|\childdocby{|\textit{main}|}|\\
\end{tabular}
\end{center}
%
The directive |\childdocby| is similar to |\childdocof|
described in \secref{sec:include},
but the subsequent selection of content must be done manually.
To that end, both |\ifchilddoc| and |\ifchilddocmanual|
will be true upon processing of a part,
and the name of the part is stored in |\childdocname|.
Note that |\jobname| will be set to the filename of the current part
so that each part receives an individual |.aux| file
that does not interfere with the |.aux| file(s) of the main document.
This behaviour can be altered by the alternative form
|\childdocby[*]{|\textit{main}|}| (with a non-empty optional argument)
which uses the |.aux| file of the main document
by setting |\jobname| to \textit{main}.

%%%%%%%%%%%%%%%%%%%%%%%%%%%%%%%%%%%%%%%%%%%%%%%%%%%%%%%%%%%%%%%%%%%%%%%%%%%%%%%%
\subsection{Driver Development}
\label{sec:driver}

The \textsf{childdoc} mechanism can also be use for the development
of definition files such as \LaTeX{} styles or classes.
This case differs from the above setup with multiple parts
included by |\include| in that no |\includeonly| should be invoked.
This can be achieved by starting the include file
(before |\ProvidesPackage|) with:
%
\begin{center}
\begin{tabular}{l}
|\input{childdoc.def}|\\
|\childdocforward{|\textit{main}|}|\\
\end{tabular}
\end{center}
%
or alternatively with:
%
\begin{center}
\begin{tabular}{l}
|\input{childdoc.def}|\\
|\childdocby{|\textit{main}|}|\\
\end{tabular}
\end{center}
%
Both forms have slightly different effects as described above.
The main file is prepared as usual, see \secref{sec:include}.

%%%%%%%%%%%%%%%%%%%%%%%%%%%%%%%%%%%%%%%%%%%%%%%%%%%%%%%%%%%%%%%%%%%%%%%%%%%%%%%%
\subsection{Legacy Detection}
\label{sec:detection}

The directive |\childdocmain| in the main file can detect
whether the complete document or merely a child is to be compiled
even without using the directive |\childdocof|.
This method is deprecated because it is less robust
and there is no compelling reason to use it;
it is merely provided for backward compatibility
and it may be removed in future versions.

If the detection mechanism is to be used,
it is mandatory to correctly specify
the filename of the main file as the argument of |\childdocmain|:
%
\begin{center}
\begin{tabular}{l}
|\input{childdoc.def}|\\
|\childdocmain{|\textit{main}|}|\\
\end{tabular}
\end{center}
%
If |\jobname| does not match the argument \textit{main} of |\childdocmain|,
it is assumed that |\jobname| points to the child file to be compiled.
When using |\childdocmain| with the main file specified as argument,
it suffices to start a child file
with just |\input{|\textit{main}|}|
without loading of the package and using |\childdocof|.
If instead all processing is done
with the appropriate \textsf{childdoc} directives,
the argument of \textit{main} of |\childdocmain| can be empty.

An alternative version of the command line processing described
in \secref{sec:commandline} using the detection mechanism reads:
%
\begin{center}
|... -jobname "|\textit{target}|" "|[\textit{flags}]%
[|\def\jobname{|\textit{dest}|}|]|\input{|\textit{main}|}"|
\end{center}

%%%%%%%%%%%%%%%%%%%%%%%%%%%%%%%%%%%%%%%%%%%%%%%%%%%%%%%%%%%%%%%%%%%%%%%%%%%%%%%%
\subsection{Manual Code}
\label{sec:manual}

In case one cannot be certain whether the definitions file |childdoc.def|
is installed on the target \TeX{} distribution
and one prefers not to ship it,
it is conceivable to paste a few relevant commands into the sources.

To that end, drop all statements |\input{childdoc.def}|
and perform the replacements as outlined below.
Instead of |\childdocmain{|\textit{main}|}| add the following code
to the top of the main file:
%
\begin{center}
\begin{tabular}{l}
|\||ifdefined\childdocname\endinput\||fi\newif\ifchilddoc|\\
|\edef\childdocname{\scantokens\expandafter{\jobname\noexpand}}|\\
|\def\childdocmain{|\textit{main}|}\||ifx\childdocmain\childdocname\||else|\\
|\childdoctrue\includeonly{\childdocname}\let\jobname\childdocmain\||fi|\\
\end{tabular}
\end{center}
%
Instead of |\childdocof{|\textit{main}|}| just include the main file
at the top of each child file:
%
\begin{center}
|\input{|\textit{main}|}|
\end{center}
%
A simple redirection |\childdocforward{|\textit{dest}|}| is achieved by:
%
\begin{center}
|\def\jobname{|\textit{dest}|}\input{\jobname}|
\end{center}
%
The redirection with prefix
|\childdocforwardprefix[|\textit{prefix}|]{|\textit{dest}|}|
is accomplished by:
%
\begin{center}
\begin{tabular}{l}
|{\edef\jobname{\scantokens\expandafter{\jobname\noexpand}}|\\
|\def\redirectjob |\textit{prefix}|#1~~~{\gdef\jobname{|\textit{dest}|#1}}|\\
|\expandafter\redirectjob\jobname~~~}\input{\jobname}|
\end{tabular}
\end{center}

In an alternative approach,
child documents can be compiled by a specific command line
without additional code or specific definitions:
%
\begin{center}
|... -jobname "|\textit{target}|" "|[\textit{flags}]%
|\includeonly{|\textit{dest}|}\input{|\textit{main}|}"|
\end{center}
%

%%%%%%%%%%%%%%%%%%%%%%%%%%%%%%%%%%%%%%%%%%%%%%%%%%%%%%%%%%%%%%%%%%%%%%%%%%%%%%%%
%%%%%%%%%%%%%%%%%%%%%%%%%%%%%%%%%%%%%%%%%%%%%%%%%%%%%%%%%%%%%%%%%%%%%%%%%%%%%%%%
\section{Information}

%%%%%%%%%%%%%%%%%%%%%%%%%%%%%%%%%%%%%%%%%%%%%%%%%%%%%%%%%%%%%%%%%%%%%%%%%%%%%%%%
\subsection{Copyright}

Copyright \copyright{} 2017--2018 Niklas Beisert

This work may be distributed and/or modified under the
conditions of the \LaTeX{} Project Public License, either version 1.3
of this license or (at your option) any later version.
The latest version of this license is in
  \url{http://www.latex-project.org/lppl.txt}
and version 1.3 or later is part of all distributions of \LaTeX{}
version 2005/12/01 or later.

This work has the LPPL maintenance status `maintained'.

The Current Maintainer of this work is Niklas Beisert.

This work consists of the files |README.txt|, |childdoc.ins| and |childdoc.dtx|
as well as the derived files |childdoc.def|, |cdocsamp.tex|
with |cdocsch1.tex|, |cdocsch2.tex|, |cdocspt3.tex|, |cdocspt4.tex|,
|cdocsdrf.tex|, |cdocsfn1.tex|, |cdocsfn2.tex|
as well as |childdoc.pdf|.

%%%%%%%%%%%%%%%%%%%%%%%%%%%%%%%%%%%%%%%%%%%%%%%%%%%%%%%%%%%%%%%%%%%%%%%%%%%%%%%%
\subsection{Files and Installation}

The package consists of the files:
%
\begin{center}
\begin{tabular}{ll}
    |README.txt|   & readme file \\
    |childdoc.ins| & installation file \\
    |childdoc.dtx| & source file \\
    |childdoc.def| & definition file \\
    |cdocsamp.tex| & sample main file \\
    |cdocsch1.tex| & sample include file \\
    |cdocsch2.tex| & sample include file \\
    |cdocspt3.tex| & sample part file \\
    |cdocspt4.tex| & sample part file \\
    |cdocsdrf.tex| & sample redirection file \\
    |cdocsfn1.tex| & sample redirection file \\
    |cdocsfn2.tex| & sample redirection file \\
    |childdoc.pdf| & manual
\end{tabular}
\end{center}
%
The distribution consists of the files
|README.txt|, |childdoc.ins| and |childdoc.dtx|.
%
\begin{itemize}
\item
Run (pdf)\LaTeX{} on |childdoc.dtx|
to compile the manual |childdoc.pdf| (this file).
\item
Run \LaTeX{} on |childdoc.ins| to create the definitions file |childdoc.def|
and the sample |cdocsamp.tex| with include files
|cdocsch1.tex|, |cdocsch2.tex|, |cdocspt3.tex|, |cdocspt4.tex|,
|cdocsdrf.tex|, |cdocsfn1.tex|, |cdocsfn2.tex|.
Then copy the file |childdoc.def| to an appropriate directory of your \LaTeX{}
distribution, e.g.\ \textit{texmf-root}|/tex/latex/childdoc|.
\end{itemize}

%%%%%%%%%%%%%%%%%%%%%%%%%%%%%%%%%%%%%%%%%%%%%%%%%%%%%%%%%%%%%%%%%%%%%%%%%%%%%%%%
\subsection{Related CTAN Packages}

There are several other packages which offer a similar functionality:
%
\begin{itemize}
\item
The packages
\href{http://ctan.org/pkg/docmute}{\textsf{docmute}},
\href{http://ctan.org/pkg/includex}{\textsf{includex}} and
\href{http://ctan.org/pkg/standalone}{\textsf{standalone}}
provide commands to include only the document body of
a child file thus allowing both files to be compiled individually.
\item
The packages \href{http://ctan.org/pkg/subdocs}{\textsf{subdocs}}
and \href{http://ctan.org/pkg/subfiles}{\textsf{subfiles}}
provide structures in which the main and child documents can be
encapsulated and allowing them to be compiled individually.
The inclusion mechanism is different from the conventional |\include|.
\item
The package \href{http://ctan.org/pkg/combine}{\textsf{combine}}
is an elaborate solution to combine several documents into one.
\end{itemize}
%
See also the CTAN topic \href{http://ctan.org/topic/subdocs}{\textsf{subdocs}}
for further related packages.
The present package differs from the above solutions in that
a document structure constructed with the conventional |\include| mechanism
just needs two extra commands at the top of every file
such that all constituent files can be compiled individually.

%%%%%%%%%%%%%%%%%%%%%%%%%%%%%%%%%%%%%%%%%%%%%%%%%%%%%%%%%%%%%%%%%%%%%%%%%%%%%%%%
%\subsection{Feature Suggestions}
%
%The following is a list of features which may be useful for future
%versions of this package:
%%
%\begin{itemize}
%\item
%\ldots
%\end{itemize}

%%%%%%%%%%%%%%%%%%%%%%%%%%%%%%%%%%%%%%%%%%%%%%%%%%%%%%%%%%%%%%%%%%%%%%%%%%%%%%%%
\subsection{Revision History}

%%%%%%%%%%%%%%%%%%%%%%%%%%%%%%%%%%%%%%%%
\paragraph{v2.0:} 2018/12/30

\begin{itemize}
\item
immediate forward processing
\item
added |\childdocby| mechanism
\item
manual restructured
\end{itemize}

%%%%%%%%%%%%%%%%%%%%%%%%%%%%%%%%%%%%%%%%
\paragraph{v1.6:} 2018/01/17

\begin{itemize}
\item
application for development of include files
\item
corrections to manual
\end{itemize}

%%%%%%%%%%%%%%%%%%%%%%%%%%%%%%%%%%%%%%%%
\paragraph{v1.5:} 2017/05/21

\begin{itemize}
\item
more complete structuring introduced
\item
|\childdocof| introduced
\item
|\childdoc| renamed to |\childdocmain|
\item
|\childredirect| renamed to |\childdocforward| and |\childdocforwardprefix|
and functionality expanded
\end{itemize}

%%%%%%%%%%%%%%%%%%%%%%%%%%%%%%%%%%%%%%%%
\paragraph{v1.0:} 2017/04/27

\begin{itemize}
\item
manual and install package
\item
first version published on CTAN
\end{itemize}

%%%%%%%%%%%%%%%%%%%%%%%%%%%%%%%%%%%%%%%%
\paragraph{v0.6:} 2017/04/26

\begin{itemize}
\item
redirection mechanism added
\end{itemize}

%%%%%%%%%%%%%%%%%%%%%%%%%%%%%%%%%%%%%%%%
\paragraph{v0.5:} 2017/04/26

\begin{itemize}
\item
functionality in definition file
\end{itemize}


%%%%%%%%%%%%%%%%%%%%%%%%%%%%%%%%%%%%%%%%%%%%%%%%%%%%%%%%%%%%%%%%%%%%%%%%%%%%%%%%
%%%%%%%%%%%%%%%%%%%%%%%%%%%%%%%%%%%%%%%%%%%%%%%%%%%%%%%%%%%%%%%%%%%%%%%%%%%%%%%%
%%%%%%%%%%%%%%%%%%%%%%%%%%%%%%%%%%%%%%%%%%%%%%%%%%%%%%%%%%%%%%%%%%%%%%%%%%%%%%%%
\appendix

\settowidth\MacroIndent{\rmfamily\scriptsize 000\ }

 \DocInput{childdoc.dtx}

\end{document}
%</driver>
% \fi
%
% %%%%%%%%%%%%%%%%%%%%%%%%%%%%%%%%%%%%%%%%%%%%%%%%%%%%%%%%%%%%%%%%%%%%%%%%%%%%%%
% %%%%%%%%%%%%%%%%%%%%%%%%%%%%%%%%%%%%%%%%%%%%%%%%%%%%%%%%%%%%%%%%%%%%%%%%%%%%%%
% \section{Sample}
%\iffalse
%<*samplemain>
%\fi
%
% The following presents a sample document
% with two chapters, two parts, a title page,
% a compile flag as well as three forwarding files to set the flag.
% It consists of eight |.tex| files:
% \begin{center}
% \begin{tabular}{ll}
% |cdocsamp.tex|&main file\\
% |cdocsch1.tex|&include file for chapter 1\\
% |cdocsch2.tex|&include file for chapter 2\\
% |cdocspt3.tex|&include file for part 3\\
% |cdocspt4.tex|&include file for part 4\\
% |cdocsdrf.tex|&forwarding file for main file in draft mode\\
% |cdocsfi1.tex|&forwarding file for final version of chapter 1\\
% |cdocsfi2.tex|&forwarding file for final version of chapter 2\\
% \end{tabular}
% \end{center}
% Each of the eight files can be compiled directly by the \LaTeX{} compiler.
%
% %%%%%%%%%%%%%%%%%%%%%%%%%%%%%%%%%%%%%%
% \paragraph{Main File.}
%
% The main file is called |cdocsamp.tex|.
%
% Load the \textsf{childdoc} definitions and
% declare the filename for the main document:
%    \begin{macrocode}
\input{childdoc.def}
\childdocmain{}
%    \end{macrocode}

% Optional override for |\version| flag:
%    \begin{macrocode}
%%\ifchilddoc\else\providecommand{\version}{draft}\fi
%    \end{macrocode}

% Define the default values for the |\version| flag
% (|final| for the main file and |draft| for childs):
%    \begin{macrocode}
\ifchilddoc
\providecommand{\version}{draft}
\else
\providecommand{\version}{final}
\fi
%    \end{macrocode}

% Load the standard document class:
%    \begin{macrocode}
\documentclass[12pt]{article}
%    \end{macrocode}

% Start the document body:
%    \begin{macrocode}
\begin{document}
%    \end{macrocode}

% Declare a title page.
% Print title, part of document being processed and version flag:
%    \begin{macrocode}
\addtocounter{page}{-1}
\begin{center}
{\LARGE\bfseries{}childdoc example\par}
\vspace{1cm}
\ifchilddoc
\ifchilddocmanual part\else chapter\fi:
`\childdocname' of `\childdocjob'\par
\else
main document: `\childdocjob'\par
\fi
version: \version\par
\end{center}
\newpage
%    \end{macrocode}

% Manually include selected file,
% otherwise process as usual:
%    \begin{macrocode}
\ifchilddocmanual
\section*{part `\childdocname'}
\input{\childdocname}
\else
%    \end{macrocode}

% Include the two chapters:
%    \begin{macrocode}
\include{cdocsch1}
\include{cdocsch2}
%    \end{macrocode}

% Include the two parts unless only chapters should be displayed:
%    \begin{macrocode}
\ifchilddoc\else
\section{part three}
\input{cdocspt3}
\section{part four}
\input{cdocspt4}
\fi
%    \end{macrocode}

% Process as usual until here:
%    \begin{macrocode}
\fi
%    \end{macrocode}

% End of document body:
%    \begin{macrocode}
\end{document}
%    \end{macrocode}
%\iffalse
%</samplemain>
%\fi
%
% %%%%%%%%%%%%%%%%%%%%%%%%%%%%%%%%%%%%%%
% \paragraph{Chapter Include Files.}
%
% The include files are called |cdocsch1.tex| and |cdocsch2.tex|.
%
%\iffalse
%<*samplechap1|samplechap2>
%\fi

% Optional override for |\version| flag:
%    \begin{macrocode}
%%\providecommand{\version}{final}
%    \end{macrocode}

% Include the main document:
%    \begin{macrocode}
\input{childdoc.def}
\childdocof{cdocsamp}
%    \end{macrocode}

%\iffalse
%</samplechap1|samplechap2>
%\fi
%
%\iffalse
%<*samplechap1>
%\fi
% Some text for chapter 1:
%    \begin{macrocode}
\section{one}
some text in chapter one
%    \end{macrocode}

%\iffalse
%</samplechap1>
%\fi
% Some text for chapter 2:
%\iffalse
%<*samplechap2>
%\fi
%    \begin{macrocode}
\section{two}
more text in chapter two
%    \end{macrocode}

%\iffalse
%</samplechap2>
%\fi
%
% %%%%%%%%%%%%%%%%%%%%%%%%%%%%%%%%%%%%%%
% \paragraph{Part Include Files.}
%
% The include files are called |cdocspt3.tex| and |cdocspt4.tex|.
%
%\iffalse
%<*samplepart3|samplepart4>
%\fi

% Optional override for |\version| flag:
%    \begin{macrocode}
%%\providecommand{\version}{final}
%    \end{macrocode}

% Include the main document:
%    \begin{macrocode}
\input{childdoc.def}
\childdocby{cdocsamp}
%    \end{macrocode}

%\iffalse
%</samplepart3|samplepart4>
%\fi
%
%\iffalse
%<*samplepart3>
%\fi
% Some text for part 3:
%    \begin{macrocode}
some text in part three
%    \end{macrocode}

%\iffalse
%</samplepart3>
%\fi
% Some text for part 4:
%\iffalse
%<*samplepart4>
%\fi
%    \begin{macrocode}
more text in part four
%    \end{macrocode}

%\iffalse
%</samplepart4>
%\fi
%
% %%%%%%%%%%%%%%%%%%%%%%%%%%%%%%%%%%%%%%
% \paragraph{Forwarding for a Complete Draft.}
%
% The following forwarding file |cdocsdrf.tex|
% compiles the main document in draft mode:
%\iffalse
%<*sampledraft>
%\fi
%    \begin{macrocode}
\def\version{draft}
\input{childdoc.def}
\childdocforward{cdocsamp}
%    \end{macrocode}

%\iffalse
%</sampledraft>
%\fi
%
% %%%%%%%%%%%%%%%%%%%%%%%%%%%%%%%%%%%%%%
% \paragraph{Forwarding for Final Version of the Chapters.}
%
% The following forwarding files |cdocsfn1.tex| and |cdocsfn2.tex|
% (with identical content)
% compile the final versions of the child documents
% |cdocsch1.tex| and |cdocsch2.tex|, respectively:
%\iffalse
%<*samplefinal>
%\fi
%    \begin{macrocode}
\def\version{final}
\input{childdoc.def}
\childdocforwardprefix[cdocsamp]{cdocsfn}{cdocsch}
%    \end{macrocode}

%\iffalse
%</samplefinal>
%\fi
%
% %%%%%%%%%%%%%%%%%%%%%%%%%%%%%%%%%%%%%%
% \paragraph{Command Line Processing.}
%
% The following three command lines generate the output files
% |cdocscld|, |cdocscl1| and |cdocscl2|
% which should be identical to
% |cdocsdrf|, |cdocsch1| and |cdocsfn2|, respectively:
% \begin{center}
% \begin{tabular}{l}
% |latex -jobname cdocscld \|\\
% |  "\def\version{draft}\input{childdoc.def}\childdocforward{cdocsamp}"|\\
% |latex -jobname cdocscl1 \|\\
% |  "\input{childdoc.def}\childdocforward[cdocsamp]{cdocsch1}"|\\
% |latex -jobname cdocscl2 \|\\
% |  "\def\version{final}\input{childdoc.def}\childdocforward{cdocsch2}"|
% \end{tabular}
% \end{center}
% Note that the trailing backslash on each first line
% merely continues the input to the second line
% (for convenient cut ant paste).
% Furthermore, the command |latex| can be replaced by any
% of its alternative versions such as |pdflatex|.
%
% %%%%%%%%%%%%%%%%%%%%%%%%%%%%%%%%%%%%%%%%%%%%%%%%%%%%%%%%%%%%%%%%%%%%%%%%%%%%%%
% %%%%%%%%%%%%%%%%%%%%%%%%%%%%%%%%%%%%%%%%%%%%%%%%%%%%%%%%%%%%%%%%%%%%%%%%%%%%%%
% \section{Implementation}
%\iffalse
%<*package>
%\fi
%
% This section describes the definitions file |childdoc.def|.

% The definitions cannot be loaded using |\usepackage| or |\RequirePackage|
% which has a mechanism to prevent loading a style file more than once.
% When loading the definitions by means of |\input|
% multiple instances have to be prevented manually:
%\iffalse
%This code needs to be before the `\ProvidesFile' directive
%which is defined at the beginning of this file.
%Therefore it is also placed there and commented out here.
%</package>
%<*discard>
%\fi
%    \begin{macrocode}
\ifdefined\childdocmain\endinput\fi
%    \end{macrocode}
%\iffalse
%</discard>
%<*package>
%\fi
%
% \macro{\ifchilddoc}
% \macro{\ifchilddocmanual}
% The conditional |\ifchilddoc| tells whether a
% child (true) or main (false) document is being compiled.
% The conditional |\ifchilddocmanual| tells whether
% the |\includeonly| mechanism is used (false) or
% the selection of child files must be performed manually (true).
% The definitions initialise to false:
%    \begin{macrocode}
\newif\ifchilddoc
\newif\ifchilddocmanual
%    \end{macrocode}

% \macro{\childdocname}
% \macro{\childdocjob}
% The macro |\childdocname| stores the name of the main document
% to be compiled. The macro |\childdocjob| stores the name of
% the document on which the \LaTeX{} compiler was originally invoked.
% The content of |\jobname| cannot be compared
% to filenames specified in the source due to different catcodes.
% The following code rescans |\jobname|, stores the result
% in |\childdocname| and saves a copy in |\childdocjob|:
%    \begin{macrocode}
\edef\childdocname{\scantokens\expandafter{\jobname\noexpand}}
\let\childdocjob\childdocname
%    \end{macrocode}

% \macro{\childdocdisable}
% The macro |\childdocdisable| prevents the main file
% from being processed more than once.
% At this stage, the main document command |\childdocmain|
% is assumed to be called once again where it should do nothing.
% Any subsequent call to it should prevent
% a secondary processing of the main document
% It overwrites the forwarding commands
% |\childdocof| and |\childdocforward|
% with empty macros to prevent further inclusions of the main document:
%    \begin{macrocode}
\newcommand{\childdocdisable}
{
  \renewcommand{\childdocmain}[1]{\renewcommand{\childdocmain}[1]{\endinput}}
  \renewcommand{\childdocof}[1]{}
  \renewcommand{\childdocby}[2][]{}
  \renewcommand{\childdocforward}[2][]{}
  \renewcommand{\childdocdisable}{}
}
%    \end{macrocode}

% \macro{\childdocmain}
% The macro |\childdocmain| is to be called at the top of the main file
% with nothing or the main filename (without extension) as argument.
% First, it breaks loops.
% If the argument is not empty and does not match |\childdocname|
% (which is set by the first inclusion of |childdoc.def|),
% |\ifchilddoc| is set to true, |\includeonly| is applied to the child file
% and |\jobname| is set to the main file
% (for proper handling of |.aux| files):
%    \begin{macrocode}
\newcommand{\childdocmain}[1]
{
  \childdocdisable\childdocmain{}
  \if?#1?\else
    \begingroup
      \def\childdoctmp{#1}
      \ifx\childdoctmp\childdocname
        \def\childdoctmp{}
      \else
        \def\childdoctmp
        {
          \childdoctrue
          \includeonly{\childdocname}
          \def\childdocjob{#1}
          \def\jobname{#1}
        }
      \fi
      \expandafter
    \endgroup
    \childdoctmp
  \fi
}
%    \end{macrocode}

% \macro{\childdocof}
% The command |\childdocof| redirects
% compilation to the main file |#1|.
%    \begin{macrocode}
\newcommand{\childdocof}[1]
{
  \childdocdisable
  \childdoctrue
  \includeonly{\childdocname}
  \def\jobname{#1}
  \def\childdocjob{#1}
  \input{#1}
}
%    \end{macrocode}

% \macro{\childdocby}
% The command |\childdocby| ....
%    \begin{macrocode}
\newcommand{\childdocby}[2][]
{
  \childdocdisable
  \childdoctrue
  \childdocmanualtrue
  \if?#1?\else
    \def\jobname{#2}
  \fi
  \def\childdocjob{#2}
  \input{#2}
  \endinput
}
%    \end{macrocode}

% \macro{\childdocforward}
% The command |\childdocforward| redirects
% compilation to the main file or
% (if the optional argument is given) a child file.
% Parameters are set as if the main file
% or a child file starting with |\childdocof| was compiled.
% Then compilation is handed over to the main file:
%    \begin{macrocode}
\newcommand{\childdocforward}[2][]
{
  \begingroup
    \if?#1?
      \def\childdoctmp
      {
        \def\childdocname{#2}
        \def\childdocjob{#2}
        \def\jobname{#2}
        \input{#2}
        \endinput
      }
    \else
      \def\childdoctmp
      {
        \childdocdisable
        \def\childdocname{#2}
        \childdoctrue
        \includeonly{#2}
        \def\childdocjob{#1}
        \def\jobname{#1}
        \input{#1}
        \endinput
      }
    \fi
    \expandafter
  \endgroup
  \childdoctmp
}
%    \end{macrocode}

% \macro{\childdocforwardprefix}
% The command |\childdocforwardprefix| redirects
% compilation to the main or a child file by means of a pattern.
% The prefix |#1| in the current filename is replaced by |#2|
% and the suffix of the current filename is kept
% (it is assumed that the filename does not contain the substring `|~~~|'
% which is used as a delimiter).
% Compilation is handed over to the new file by |\childdocforward|:
%    \begin{macrocode}
\newcommand{\childdocforwardprefix}[3][]
{
  \begingroup
    \def\childdocextract #2##1~~~{\def\childdoctmp{\childdocforward[#1]{#3##1}}}
    \expandafter\childdocextract\childdocname~~~
    \expandafter
  \endgroup
  \childdoctmp
}
%    \end{macrocode}

% \macro{\childdoc}
% The deprecated macro |\childdoc| is a legacy version of |\childdocmain|:
%    \begin{macrocode}
\newcommand{\childdoc}{\childdocmain}
%    \end{macrocode}

% \macro{\childdocredirect}
% The deprecated macro |\childdocredirect| is a legacy version
% of |\childdocforward| and |\childdocforwardprefix|:
%    \begin{macrocode}
\newcommand{\childdocredirect}[2][]
{
  \begingroup
    \if?#1?
      \def\childdoctmp{\childdocforward{#2}}
    \else
      \def\childdoctmp{\childdocforwardprefix{#1}{#2}}
    \fi
    \expandafter
  \endgroup
  \childdoctmp
}
%    \end{macrocode}

%\iffalse
%</package>
%\fi
%
\endinput
|\\
|\childdocmain{|\textit{main}|}|\\
\end{tabular}
\end{center}
%
If |\jobname| does not match the argument \textit{main} of |\childdocmain|,
it is assumed that |\jobname| points to the child file to be compiled.
When using |\childdocmain| with the main file specified as argument,
it suffices to start a child file
with just |\input{|\textit{main}|}|
without loading of the package and using |\childdocof|.
If instead all processing is done
with the appropriate \textsf{childdoc} directives,
the argument of \textit{main} of |\childdocmain| can be empty.

An alternative version of the command line processing described
in \secref{sec:commandline} using the detection mechanism reads:
%
\begin{center}
|... -jobname "|\textit{target}|" "|[\textit{flags}]%
[|\def\jobname{|\textit{dest}|}|]|\input{|\textit{main}|}"|
\end{center}

%%%%%%%%%%%%%%%%%%%%%%%%%%%%%%%%%%%%%%%%%%%%%%%%%%%%%%%%%%%%%%%%%%%%%%%%%%%%%%%%
\subsection{Manual Code}
\label{sec:manual}

In case one cannot be certain whether the definitions file |childdoc.def|
is installed on the target \TeX{} distribution
and one prefers not to ship it,
it is conceivable to paste a few relevant commands into the sources.

To that end, drop all statements |% \iffalse
%
% childdoc.dtx Copyright (C) 2017-2018 Niklas Beisert
%
% This work may be distributed and/or modified under the
% conditions of the LaTeX Project Public License, either version 1.3
% of this license or (at your option) any later version.
% The latest version of this license is in
%   http://www.latex-project.org/lppl.txt
% and version 1.3 or later is part of all distributions of LaTeX
% version 2005/12/01 or later.
%
% This work has the LPPL maintenance status `maintained'.
%
% The Current Maintainer of this work is Niklas Beisert.
%
% This work consists of the files childdoc.dtx and childdoc.ins
% and the derived files childdoc.def and cdocsamp.tex with
% cdocsch1.tex, cdocsch2.tex, cdocsdrf.tex, cdocsfn1.tex, cdocsfn2.tex.
%
%<package>\ifdefined\childdocmain\endinput\fi
%<package>\ProvidesFile{childdoc.def}[2018/12/30 v2.0 child document driver]
%<samplemain>\ProvidesFile{cdocsamp.tex}[2018/12/30 v2.0 sample for childdoc]
%<*driver>
%\ProvidesFile{childdoc.drv}[2018/12/30 v2.0 childdoc reference manual file]
\PassOptionsToClass{10pt,a4paper}{article}
\documentclass{ltxdoc}

\usepackage[margin=35mm]{geometry}
\usepackage{hyperref}
\usepackage{hyperxmp}
\usepackage[usenames]{color}

\hypersetup{colorlinks=true}
\hypersetup{pdfstartview=FitH}
\hypersetup{pdfpagemode=UseNone}
\hypersetup{pdfsource={}}
\hypersetup{pdflang={en-UK}}
\hypersetup{pdfcopyright={Copyright 2017-2018 Niklas Beisert.
  This work may be distributed and/or modified under the
  conditions of the LaTeX Project Public License, either version 1.3
  of this license or (at your option) any later version.}}
\hypersetup{pdflicenseurl={http://www.latex-project.org/lppl.txt}}
\hypersetup{pdfcontactaddress={ETH Zurich, ITP, HIT K,
  Wolfgang-Pauli-Strasse 27}}
\hypersetup{pdfcontactpostcode={8093}}
\hypersetup{pdfcontactcity={Zurich}}
\hypersetup{pdfcontactcountry={Switzerland}}
\hypersetup{pdfcontactemail={nbeisert@itp.phys.ethz.ch}}
\hypersetup{pdfcontacturl={http://people.phys.ethz.ch/\xmptilde nbeisert/}}

\newcommand{\secref}[1]{\hyperref[#1]{section \ref*{#1}}}

\parskip1ex
\parindent0pt
\let\olditemize\itemize
\def\itemize{\olditemize\parskip0pt}

\begin{document}

\title{The \textsf{childdoc} Package}
\hypersetup{pdftitle={The childdoc Package}}
\author{Niklas Beisert\\[2ex]
  Institut f\"ur Theoretische Physik\\
  Eidgen\"ossische Technische Hochschule Z\"urich\\
  Wolfgang-Pauli-Strasse 27, 8093 Z\"urich, Switzerland\\[1ex]
  \href{mailto:nbeisert@itp.phys.ethz.ch}
  {\texttt{nbeisert@itp.phys.ethz.ch}}}
\hypersetup{pdfauthor={Niklas Beisert}}
\hypersetup{pdfsubject={Manual for the LaTeX2e Package childdoc}}
\date{30 December 2018, \textsf{v2.0}}
\maketitle

\begin{abstract}\noindent
\textsf{childdoc} is a \LaTeXe{} package
that enables the direct compilation
of document sections included by |\include|
to individual files.
\end{abstract}

\begingroup
\parskip0ex
\tableofcontents
\endgroup

%%%%%%%%%%%%%%%%%%%%%%%%%%%%%%%%%%%%%%%%%%%%%%%%%%%%%%%%%%%%%%%%%%%%%%%%%%%%%%%%
%%%%%%%%%%%%%%%%%%%%%%%%%%%%%%%%%%%%%%%%%%%%%%%%%%%%%%%%%%%%%%%%%%%%%%%%%%%%%%%%
\section{Introduction}

\LaTeX{} provides a mechanism to structure a large document (such as a book)
into a main file and several child files (containing the chapters)
using the |\include| command.
This mechanism is beneficial for documents
which span hundreds of pages in order to
make the source file(s) more manageable.
Moreover, compilation can be restricted to
selected child files by means of the |\includeonly| command.
The latter feature can be used to reduce the compilation time while editing
(this was significantly more useful in the earlier days of \LaTeX{})
or to generate a smaller document which is easier to navigate.
Another application of |\includeonly| is to generate
documents consisting of selected parts of the complete document.

However, there are a few drawbacks of the plain |\include| mechanism:
\begin{itemize}
\item
The child files cannot be compiled on their own,
they can only be compiled via the main file.
A naive editing environment
(such as a text editor with an option
to have the current file processed by \LaTeX)
may require one to switch to the main file before compiling;
attempting to compile the child file produces errors.
\item
The main file must be modified (each time)
to adjust the |\includeonly| command
to the present needs. This easily leaves the main file in a messy state.
\item
The generated document will always carry the filename
of the main document. This is inconvenient if
several child files are to be compiled and
to be kept for distribution.
\end{itemize}

The present package provides a simple interface
to make child files individually compilable by \LaTeX{}.
Compiling a child file then has the same effect as compiling
the main file with an |\includeonly| command
to select the appropriate child.
Moreover the generated document will carry the name of the child
rather than the main file.
This resolves all three above issues.

This feature is meant to make the editing of books,
thesis documents and lecture notes somewhat more convenient.
However, the package can also be used efficiently for
composing a series of documents (such as exercise sheets)
which are typically distributed individually.
It then assists the author in generating the individual documents
(potentially in different versions)
as well as a document containing the collected series.
Another application is in developing style files
or other kinds of included material
where compilation of the style file could redirect
to a sample or test file.

%%%%%%%%%%%%%%%%%%%%%%%%%%%%%%%%%%%%%%%%%%%%%%%%%%%%%%%%%%%%%%%%%%%%%%%%%%%%%%%%
%%%%%%%%%%%%%%%%%%%%%%%%%%%%%%%%%%%%%%%%%%%%%%%%%%%%%%%%%%%%%%%%%%%%%%%%%%%%%%%%
\section{Usage}

First of all, the package \textsf{childdoc} is \emph{not} a standard
\LaTeXe{} |.sty| style file! Therefore it needs to be invoked in
a non-standard way.

%%%%%%%%%%%%%%%%%%%%%%%%%%%%%%%%%%%%%%%%%%%%%%%%%%%%%%%%%%%%%%%%%%%%%%%%%%%%%%%%
\subsection{Included Files}
\label{sec:include}

%%%%%%%%%%%%%%%%%%%%%%%%%%%%%%%%%%%%%%%%
\DescribeMacro{\childdocmain}
To use the package, add the commands
\begin{center}
\begin{tabular}{l}
|\input{childdoc.def}|\\
|\childdocmain{}|\\
\end{tabular}
\end{center}
at the very top of the main \LaTeX{} file,
in particular \emph{before} the |\documentclass| statement!
The argument of |\childdocmain| should be left empty
(but it must be present).

%%%%%%%%%%%%%%%%%%%%%%%%%%%%%%%%%%%%%%%%
\DescribeMacro{\childdocof}
Furthermore, add the commands
\begin{center}
\begin{tabular}{l}
|\input{childdoc.def}|\\
|\childdocof{|\textit{main}|}|\\
\end{tabular}
\end{center}
at the top of every child file \textit{child}
which is included by |\include{|\textit{child}|}|
from within the main file
(or at least for those files to be compiled individually).
The argument \textit{main} must be the filename of the main file.

There are a couple of
considerations in setting up the main and child documents:

%%%%%%%%%%%%%%%%%%%%%%%%%%%%%%%%%%%%%%%%
\paragraph{Restrictions.}

Please note the following restrictions:
\begin{itemize}
\item
|\childdocmain| must be called with one argument \textit{main}
to ensure compatibility with earlier version of the package.
It must either be empty (|\childdocmain{}|)
or precisely match the filename of the main file in which it is specified.
See \secref{sec:detection} for further information.
\item
The filename \textit{main} must be specified without the |.tex| extension.
\item
The filename \textit{main} is case sensitive
(even in case-insensitive file systems)
due to internal string comparison.
\item
The argument \textit{main} should be fully expanded, it cannot be a macro.
\item
Subdirectories and special characters should be avoided in filenames.
\item
The command |\childdocmain{|\textit{main}|}| must be followed by a whitespace.
It should not be followed immediately by another command
or by a comment mark `|%|'.
This is because the \TeX{} parser reads the token immediately following
the argument of |\childdocmain| and puts it
at the beginning of every child section;
however, a white\-space is ignored.
\end{itemize}

%%%%%%%%%%%%%%%%%%%%%%%%%%%%%%%%%%%%%%%%
\paragraph{Content of Main File.}

It is advisable to place all content in the child files included by |\include|.
Any output contained in the main file will appear in all child documents
unless suppressed manually;
it cannot be suppressed automatically by the |\includeonly| directive
and thus should normally be avoided.
A method to include some content in the main file
by means of conditional processing is described in \secref{sec:conditional}.

%%%%%%%%%%%%%%%%%%%%%%%%%%%%%%%%%%%%%%%%
\paragraph{Page Numbering.}

When only a part of the document is compiled,
the appropriate numbering of pages
(as well as other status parameters)
is determined from the |.aux| files.
The latter contain information from previous passes.
However this information needs to propagate through
all intermediate child documents.
Therefore the page numbering in child documents may well
be inconsistent until the complete document is compiled at least once.

A useful (if unconventional) way to always ensure a consistent
page numbering is to restart the numbering in each child document
and denote the pages by `\textit{child}|.|\textit{page}'
where \textit{child} represents the chapter/section number of the child file.
This can be achieved by the command
|\numberwithin{page}{|\textit{child}|}|
of the \textsf{amsmath} package
where \textit{child} can be |chapter| or |section|
depending on the chosen structuring.
Alternatively, one can modify the macro |\thepage| appropriately
and reset the counter |page| at the start of each child file.

%%%%%%%%%%%%%%%%%%%%%%%%%%%%%%%%%%%%%%%%%%%%%%%%%%%%%%%%%%%%%%%%%%%%%%%%%%%%%%%%
\subsection{Conditional Processing}
\label{sec:conditional}

The package provides a mechanism to compile different versions
of a document. To customise the versions further some conditional processing
can come in handy to distinguish which version is being compiled.
The package provides two macros to describe the compilation context:

%%%%%%%%%%%%%%%%%%%%%%%%%%%%%%%%%%%%%%%%
\DescribeMacro{\ifchilddoc}
The conditional |\ifchilddoc| distinguishes between the compilation of
child documents and the main document:
%
\begin{center}
|\ifchilddoc |\textit{child-code}| |[|\||else |\textit{main-code}]| \||fi|
\end{center}

%%%%%%%%%%%%%%%%%%%%%%%%%%%%%%%%%%%%%%%%
\DescribeMacro{\childdocname}
\DescribeMacro{\childdocjob}
The macro |\childdocname| contains the filename (without extension)
of the main or child file being processed.
Note that |\childdocjob| will always contain the name of the main file.

%%%%%%%%%%%%%%%%%%%%%%%%%%%%%%%%%%%%%%%%
\paragraph{Title Page.}

Conditional processing can be used to include a title or banner page
in the main document when proper precautions are taken.
Importantly, the code in the main file should ensure that the page counter
(as well as other status parameters which are stored in the |.aux| files)
takes the same value after the conditional processing.
Otherwise the page numbers may take divergent values
depending on which part is compiled.

For example, a title page could be declared by:
%
\begin{center}
\begin{tabular}{l}
|\ifchilddoc\||else|\\
|\addtocounter{page}{-1}|\\
\textit{code for title page}\\
|\newpage|\\
|\||fi|
\end{tabular}
\end{center}
%
A banner page for the child documents can be generated by:
%
\begin{center}
\begin{tabular}{l}
|\ifchilddoc|\\
|\addtocounter{page}{-1}|\\
\textit{code for banner page}\\
|\newpage|\\
|\||fi|
\end{tabular}
\end{center}
%
Here one could write a message such as:
\begin{center}
|This is the part \childdocname{} of \childdocjob{}.|
\end{center}

%%%%%%%%%%%%%%%%%%%%%%%%%%%%%%%%%%%%%%%%%%%%%%%%%%%%%%%%%%%%%%%%%%%%%%%%%%%%%%%%
\subsection{Flags}
\label{sec:flags}

The package makes it easy to generate different versions
of the main or child documents.
To this end compilation flags can be defined
and assigned different default values.
They will be particularly useful in conjunction
with the forwarding mechanism described in \secref{sec:forward}.

For example, it may be useful to have a flag |\version|
which can be set to |draft| or |final|.
The document source will contain some conditional code
depending on the value of |\version|.
Suppose further, the flag should default to |final| for the main file
and to |draft| for child files
which is a natural assignment for editing the document.
This is achieved by placing the following code
in the preamble of the main document
(below the |\childdocmain| directive):
%
\begin{center}
\begin{tabular}{l}
|\ifchilddoc|\\
|\providecommand{\version}{draft}|\\
|\||else|\\
|\providecommand{\version}{final}|\\
|\||fi|
\end{tabular}
\end{center}
%
The definition by |\providecommand| makes sure
that previous definitions are not overwritten.
Further statements |\providecommand{\version}{...}|
can thus be added before the above code to override it.

For the main file, one might add a line
(between |\childdocmain| and the above block)
%
\begin{center}
|%\ifchilddoc\||else\providecommand{\version}{draft}\||fi|
\end{center}
%
which can be uncommented to produce a draft version.
Likewise one can add a line to the very top of a child file
(above the |\childdocof{|\textit{main}|}| directive)
%
\begin{center}
|%\providecommand{\version}{final}|
\end{center}
%
which can be uncommented to produce the final version of this child document.

%%%%%%%%%%%%%%%%%%%%%%%%%%%%%%%%%%%%%%%%%%%%%%%%%%%%%%%%%%%%%%%%%%%%%%%%%%%%%%%%
\subsection{Forwarding}
\label{sec:forward}

Different versions of the main or child documents
using compilation flags as described in \secref{sec:flags}
can be (permanently) stored in different files
for convenient compilation, viewing and distribution.
To this end, the package defines a command
to pass on compilation to a different file:

%%%%%%%%%%%%%%%%%%%%%%%%%%%%%%%%%%%%%%%%
\DescribeMacro{\childdocforward}
The command |\childdocforward| redirects processing to
another source file:
%
\begin{center}
\begin{tabular}{l}
|\input{childdoc.def}|\\
|\childdocforward[|\textit{main}|]{|\textit{dest}|}|\\
\end{tabular}
\end{center}
%
The argument \textit{dest} is the destination file
(without extension).
It should be the main file or one of the child files.
Note that further \textsf{childdoc} directives
such as |\childdocof| and |\childdocforward|
in the indicated file will be processed in this form.
The optional argument \textit{main}
passes on directly to the main file \textit{main}
while pretending to compile the child \textit{dest}.
This form behaves as if \textit{dest}
issues |\childdocof{|\textit{main}|}| right away,
and no further \textsf{childdoc} directives will be processed.

%%%%%%%%%%%%%%%%%%%%%%%%%%%%%%%%%%%%%%%%
\DescribeMacro{\...prefix}
In the alternative form |\childdocforwardprefix|,
%
\begin{center}
\begin{tabular}{l}
|\input{childdoc.def}|\\
|\childdocforwardprefix[|\textit{main}|]{|\textit{prefix}|}{|\textit{dest}|}|
\end{tabular}
\end{center}
%
the destination file is determined by a pattern
depending on the current file:
To make this work, the current file must be called
`{\textit{prefix}\hspace{0.2em}\textit{suffix}}'
with \textit{prefix} matching precisely the argument.
Processing is then passed on to the file
`{\textit{dest}\hspace{0.2em}\textit{suffix}}'.
Surely, the same effect is achieved by
directly specifying the
argument `{\textit{dest}\hspace{0.2em}\textit{suffix}}'
in the first form.
However, that requires to set up a different file
for each child. With the alternative form of the command
all these files can have exactly the same content
which simplifies setting them up and maintaining them.

For example, the following file |draft.tex|
with a compilation flag |\version| as described in \secref{sec:flags}
compiles the main document as a draft:
%
\begin{center}
\begin{tabular}{l}
|\def\version{draft}|\\
|\input{childdoc.def}|\\
|\childdocforward{|\textit{main}|}|
\end{tabular}
\end{center}
%
Likewise, the following files |final|\textit{nn}|.tex|
compile the final version of the child document
|child|\textit{nn}|.tex|:
%
\begin{center}
\begin{tabular}{l}
|\def\version{final}|\\
|\input{childdoc.def}|\\
|\childdocforwardprefix{final}{child}|
\end{tabular}
\end{center}
%

Note that when several versions of a main file and/or of each child file
are to be generated, it may be convenient to set up a |Makefile| or
shell script to automatise the process.

%%%%%%%%%%%%%%%%%%%%%%%%%%%%%%%%%%%%%%%%%%%%%%%%%%%%%%%%%%%%%%%%%%%%%%%%%%%%%%%%
\subsection{Command Line Processing}
\label{sec:commandline}

The effect of redirection files can also be achieved by invoking
the \LaTeX{} compiler with a more elaborate command line.
Most conveniently this should be done as part
of a shell script or a |Makefile|.

When using \textsf{childdoc} in the main file, the following
command lines effectively perform a redirection
(note that depending on the shell being used,
backslashes may have to be doubled: `|\|' $\to$ `|\\|'):
%
\begin{center}
|... -jobname "|\textit{target}|" |\\|"|[\textit{flags}]%
|\input{childdoc.def}\childdocforward[|\textit{main}|]{|\textit{dest}|}"|
\end{center}
%
Here \textit{target} is the name of the output file,
\textit{main} is the name of the main file
and \textit{dest} is the name of the main or child file to be processed
(all filenames without extensions).
The optional argument \textit{main} can be omitted
if \textit{main} matches \textit{dest}.
Optionally, compilation \textit{flags} can be defined via |\def| commands.
This command line makes the \TeX{} engine believe
it is compiling the file \textit{target}
whose content is specified as the latter parameter.
The provided code then forwards the processing to
\textit{main} or \textit{dest} as described in \secref{sec:forward}.

%%%%%%%%%%%%%%%%%%%%%%%%%%%%%%%%%%%%%%%%%%%%%%%%%%%%%%%%%%%%%%%%%%%%%%%%%%%%%%%%
\subsection{Include by Input}
\label{sec:input}

Including child documents by |\include| has some restrictions by design.
Most notably, the content of a child document always occupies
its own set of pages; pages cannot be shared between child documents.
Usually, this behaviour makes perfect sense
because each child document contain an essential part of the document.
However, in some situations it may be desirable to compose
a document from a collection of parts
without having mandatory page breaks between then.
For this case, the package
provides a mechanism to include parts
by |\input| which can also be processed individually.
However, by construction this mechanism
requires manual handling of the content to be output.

%%%%%%%%%%%%%%%%%%%%%%%%%%%%%%%%%%%%%%%%
\DescribeMacro{\ifchilddocmanual}
The main file should be prepared as usual, see \secref{sec:include}.
However, the document body must make a distinction
between processing of an individual part and of the main document, e.g.:
%
\begin{center}
\begin{tabular}{l}
|\ifchilddocmanual|\\
|\input{\childdocname}|\\
|\||else|\\
\textit{document body with }|\input{|\textit{part}|}|\\
|\||fi|
\end{tabular}
\end{center}
%
The conditional |\ifchilddocmanual| is true whenever
a part to be included by |\input| is being compiled,
and the name of the part is stored in |\childdocname|.

%%%%%%%%%%%%%%%%%%%%%%%%%%%%%%%%%%%%%%%%
\DescribeMacro{\childdocby}
Each part to be included by |\input| should start with:
%
\begin{center}
\begin{tabular}{l}
|\input{childdoc.def}|\\
|\childdocby{|\textit{main}|}|\\
\end{tabular}
\end{center}
%
The directive |\childdocby| is similar to |\childdocof|
described in \secref{sec:include},
but the subsequent selection of content must be done manually.
To that end, both |\ifchilddoc| and |\ifchilddocmanual|
will be true upon processing of a part,
and the name of the part is stored in |\childdocname|.
Note that |\jobname| will be set to the filename of the current part
so that each part receives an individual |.aux| file
that does not interfere with the |.aux| file(s) of the main document.
This behaviour can be altered by the alternative form
|\childdocby[*]{|\textit{main}|}| (with a non-empty optional argument)
which uses the |.aux| file of the main document
by setting |\jobname| to \textit{main}.

%%%%%%%%%%%%%%%%%%%%%%%%%%%%%%%%%%%%%%%%%%%%%%%%%%%%%%%%%%%%%%%%%%%%%%%%%%%%%%%%
\subsection{Driver Development}
\label{sec:driver}

The \textsf{childdoc} mechanism can also be use for the development
of definition files such as \LaTeX{} styles or classes.
This case differs from the above setup with multiple parts
included by |\include| in that no |\includeonly| should be invoked.
This can be achieved by starting the include file
(before |\ProvidesPackage|) with:
%
\begin{center}
\begin{tabular}{l}
|\input{childdoc.def}|\\
|\childdocforward{|\textit{main}|}|\\
\end{tabular}
\end{center}
%
or alternatively with:
%
\begin{center}
\begin{tabular}{l}
|\input{childdoc.def}|\\
|\childdocby{|\textit{main}|}|\\
\end{tabular}
\end{center}
%
Both forms have slightly different effects as described above.
The main file is prepared as usual, see \secref{sec:include}.

%%%%%%%%%%%%%%%%%%%%%%%%%%%%%%%%%%%%%%%%%%%%%%%%%%%%%%%%%%%%%%%%%%%%%%%%%%%%%%%%
\subsection{Legacy Detection}
\label{sec:detection}

The directive |\childdocmain| in the main file can detect
whether the complete document or merely a child is to be compiled
even without using the directive |\childdocof|.
This method is deprecated because it is less robust
and there is no compelling reason to use it;
it is merely provided for backward compatibility
and it may be removed in future versions.

If the detection mechanism is to be used,
it is mandatory to correctly specify
the filename of the main file as the argument of |\childdocmain|:
%
\begin{center}
\begin{tabular}{l}
|\input{childdoc.def}|\\
|\childdocmain{|\textit{main}|}|\\
\end{tabular}
\end{center}
%
If |\jobname| does not match the argument \textit{main} of |\childdocmain|,
it is assumed that |\jobname| points to the child file to be compiled.
When using |\childdocmain| with the main file specified as argument,
it suffices to start a child file
with just |\input{|\textit{main}|}|
without loading of the package and using |\childdocof|.
If instead all processing is done
with the appropriate \textsf{childdoc} directives,
the argument of \textit{main} of |\childdocmain| can be empty.

An alternative version of the command line processing described
in \secref{sec:commandline} using the detection mechanism reads:
%
\begin{center}
|... -jobname "|\textit{target}|" "|[\textit{flags}]%
[|\def\jobname{|\textit{dest}|}|]|\input{|\textit{main}|}"|
\end{center}

%%%%%%%%%%%%%%%%%%%%%%%%%%%%%%%%%%%%%%%%%%%%%%%%%%%%%%%%%%%%%%%%%%%%%%%%%%%%%%%%
\subsection{Manual Code}
\label{sec:manual}

In case one cannot be certain whether the definitions file |childdoc.def|
is installed on the target \TeX{} distribution
and one prefers not to ship it,
it is conceivable to paste a few relevant commands into the sources.

To that end, drop all statements |\input{childdoc.def}|
and perform the replacements as outlined below.
Instead of |\childdocmain{|\textit{main}|}| add the following code
to the top of the main file:
%
\begin{center}
\begin{tabular}{l}
|\||ifdefined\childdocname\endinput\||fi\newif\ifchilddoc|\\
|\edef\childdocname{\scantokens\expandafter{\jobname\noexpand}}|\\
|\def\childdocmain{|\textit{main}|}\||ifx\childdocmain\childdocname\||else|\\
|\childdoctrue\includeonly{\childdocname}\let\jobname\childdocmain\||fi|\\
\end{tabular}
\end{center}
%
Instead of |\childdocof{|\textit{main}|}| just include the main file
at the top of each child file:
%
\begin{center}
|\input{|\textit{main}|}|
\end{center}
%
A simple redirection |\childdocforward{|\textit{dest}|}| is achieved by:
%
\begin{center}
|\def\jobname{|\textit{dest}|}\input{\jobname}|
\end{center}
%
The redirection with prefix
|\childdocforwardprefix[|\textit{prefix}|]{|\textit{dest}|}|
is accomplished by:
%
\begin{center}
\begin{tabular}{l}
|{\edef\jobname{\scantokens\expandafter{\jobname\noexpand}}|\\
|\def\redirectjob |\textit{prefix}|#1~~~{\gdef\jobname{|\textit{dest}|#1}}|\\
|\expandafter\redirectjob\jobname~~~}\input{\jobname}|
\end{tabular}
\end{center}

In an alternative approach,
child documents can be compiled by a specific command line
without additional code or specific definitions:
%
\begin{center}
|... -jobname "|\textit{target}|" "|[\textit{flags}]%
|\includeonly{|\textit{dest}|}\input{|\textit{main}|}"|
\end{center}
%

%%%%%%%%%%%%%%%%%%%%%%%%%%%%%%%%%%%%%%%%%%%%%%%%%%%%%%%%%%%%%%%%%%%%%%%%%%%%%%%%
%%%%%%%%%%%%%%%%%%%%%%%%%%%%%%%%%%%%%%%%%%%%%%%%%%%%%%%%%%%%%%%%%%%%%%%%%%%%%%%%
\section{Information}

%%%%%%%%%%%%%%%%%%%%%%%%%%%%%%%%%%%%%%%%%%%%%%%%%%%%%%%%%%%%%%%%%%%%%%%%%%%%%%%%
\subsection{Copyright}

Copyright \copyright{} 2017--2018 Niklas Beisert

This work may be distributed and/or modified under the
conditions of the \LaTeX{} Project Public License, either version 1.3
of this license or (at your option) any later version.
The latest version of this license is in
  \url{http://www.latex-project.org/lppl.txt}
and version 1.3 or later is part of all distributions of \LaTeX{}
version 2005/12/01 or later.

This work has the LPPL maintenance status `maintained'.

The Current Maintainer of this work is Niklas Beisert.

This work consists of the files |README.txt|, |childdoc.ins| and |childdoc.dtx|
as well as the derived files |childdoc.def|, |cdocsamp.tex|
with |cdocsch1.tex|, |cdocsch2.tex|, |cdocspt3.tex|, |cdocspt4.tex|,
|cdocsdrf.tex|, |cdocsfn1.tex|, |cdocsfn2.tex|
as well as |childdoc.pdf|.

%%%%%%%%%%%%%%%%%%%%%%%%%%%%%%%%%%%%%%%%%%%%%%%%%%%%%%%%%%%%%%%%%%%%%%%%%%%%%%%%
\subsection{Files and Installation}

The package consists of the files:
%
\begin{center}
\begin{tabular}{ll}
    |README.txt|   & readme file \\
    |childdoc.ins| & installation file \\
    |childdoc.dtx| & source file \\
    |childdoc.def| & definition file \\
    |cdocsamp.tex| & sample main file \\
    |cdocsch1.tex| & sample include file \\
    |cdocsch2.tex| & sample include file \\
    |cdocspt3.tex| & sample part file \\
    |cdocspt4.tex| & sample part file \\
    |cdocsdrf.tex| & sample redirection file \\
    |cdocsfn1.tex| & sample redirection file \\
    |cdocsfn2.tex| & sample redirection file \\
    |childdoc.pdf| & manual
\end{tabular}
\end{center}
%
The distribution consists of the files
|README.txt|, |childdoc.ins| and |childdoc.dtx|.
%
\begin{itemize}
\item
Run (pdf)\LaTeX{} on |childdoc.dtx|
to compile the manual |childdoc.pdf| (this file).
\item
Run \LaTeX{} on |childdoc.ins| to create the definitions file |childdoc.def|
and the sample |cdocsamp.tex| with include files
|cdocsch1.tex|, |cdocsch2.tex|, |cdocspt3.tex|, |cdocspt4.tex|,
|cdocsdrf.tex|, |cdocsfn1.tex|, |cdocsfn2.tex|.
Then copy the file |childdoc.def| to an appropriate directory of your \LaTeX{}
distribution, e.g.\ \textit{texmf-root}|/tex/latex/childdoc|.
\end{itemize}

%%%%%%%%%%%%%%%%%%%%%%%%%%%%%%%%%%%%%%%%%%%%%%%%%%%%%%%%%%%%%%%%%%%%%%%%%%%%%%%%
\subsection{Related CTAN Packages}

There are several other packages which offer a similar functionality:
%
\begin{itemize}
\item
The packages
\href{http://ctan.org/pkg/docmute}{\textsf{docmute}},
\href{http://ctan.org/pkg/includex}{\textsf{includex}} and
\href{http://ctan.org/pkg/standalone}{\textsf{standalone}}
provide commands to include only the document body of
a child file thus allowing both files to be compiled individually.
\item
The packages \href{http://ctan.org/pkg/subdocs}{\textsf{subdocs}}
and \href{http://ctan.org/pkg/subfiles}{\textsf{subfiles}}
provide structures in which the main and child documents can be
encapsulated and allowing them to be compiled individually.
The inclusion mechanism is different from the conventional |\include|.
\item
The package \href{http://ctan.org/pkg/combine}{\textsf{combine}}
is an elaborate solution to combine several documents into one.
\end{itemize}
%
See also the CTAN topic \href{http://ctan.org/topic/subdocs}{\textsf{subdocs}}
for further related packages.
The present package differs from the above solutions in that
a document structure constructed with the conventional |\include| mechanism
just needs two extra commands at the top of every file
such that all constituent files can be compiled individually.

%%%%%%%%%%%%%%%%%%%%%%%%%%%%%%%%%%%%%%%%%%%%%%%%%%%%%%%%%%%%%%%%%%%%%%%%%%%%%%%%
%\subsection{Feature Suggestions}
%
%The following is a list of features which may be useful for future
%versions of this package:
%%
%\begin{itemize}
%\item
%\ldots
%\end{itemize}

%%%%%%%%%%%%%%%%%%%%%%%%%%%%%%%%%%%%%%%%%%%%%%%%%%%%%%%%%%%%%%%%%%%%%%%%%%%%%%%%
\subsection{Revision History}

%%%%%%%%%%%%%%%%%%%%%%%%%%%%%%%%%%%%%%%%
\paragraph{v2.0:} 2018/12/30

\begin{itemize}
\item
immediate forward processing
\item
added |\childdocby| mechanism
\item
manual restructured
\end{itemize}

%%%%%%%%%%%%%%%%%%%%%%%%%%%%%%%%%%%%%%%%
\paragraph{v1.6:} 2018/01/17

\begin{itemize}
\item
application for development of include files
\item
corrections to manual
\end{itemize}

%%%%%%%%%%%%%%%%%%%%%%%%%%%%%%%%%%%%%%%%
\paragraph{v1.5:} 2017/05/21

\begin{itemize}
\item
more complete structuring introduced
\item
|\childdocof| introduced
\item
|\childdoc| renamed to |\childdocmain|
\item
|\childredirect| renamed to |\childdocforward| and |\childdocforwardprefix|
and functionality expanded
\end{itemize}

%%%%%%%%%%%%%%%%%%%%%%%%%%%%%%%%%%%%%%%%
\paragraph{v1.0:} 2017/04/27

\begin{itemize}
\item
manual and install package
\item
first version published on CTAN
\end{itemize}

%%%%%%%%%%%%%%%%%%%%%%%%%%%%%%%%%%%%%%%%
\paragraph{v0.6:} 2017/04/26

\begin{itemize}
\item
redirection mechanism added
\end{itemize}

%%%%%%%%%%%%%%%%%%%%%%%%%%%%%%%%%%%%%%%%
\paragraph{v0.5:} 2017/04/26

\begin{itemize}
\item
functionality in definition file
\end{itemize}


%%%%%%%%%%%%%%%%%%%%%%%%%%%%%%%%%%%%%%%%%%%%%%%%%%%%%%%%%%%%%%%%%%%%%%%%%%%%%%%%
%%%%%%%%%%%%%%%%%%%%%%%%%%%%%%%%%%%%%%%%%%%%%%%%%%%%%%%%%%%%%%%%%%%%%%%%%%%%%%%%
%%%%%%%%%%%%%%%%%%%%%%%%%%%%%%%%%%%%%%%%%%%%%%%%%%%%%%%%%%%%%%%%%%%%%%%%%%%%%%%%
\appendix

\settowidth\MacroIndent{\rmfamily\scriptsize 000\ }

 \DocInput{childdoc.dtx}

\end{document}
%</driver>
% \fi
%
% %%%%%%%%%%%%%%%%%%%%%%%%%%%%%%%%%%%%%%%%%%%%%%%%%%%%%%%%%%%%%%%%%%%%%%%%%%%%%%
% %%%%%%%%%%%%%%%%%%%%%%%%%%%%%%%%%%%%%%%%%%%%%%%%%%%%%%%%%%%%%%%%%%%%%%%%%%%%%%
% \section{Sample}
%\iffalse
%<*samplemain>
%\fi
%
% The following presents a sample document
% with two chapters, two parts, a title page,
% a compile flag as well as three forwarding files to set the flag.
% It consists of eight |.tex| files:
% \begin{center}
% \begin{tabular}{ll}
% |cdocsamp.tex|&main file\\
% |cdocsch1.tex|&include file for chapter 1\\
% |cdocsch2.tex|&include file for chapter 2\\
% |cdocspt3.tex|&include file for part 3\\
% |cdocspt4.tex|&include file for part 4\\
% |cdocsdrf.tex|&forwarding file for main file in draft mode\\
% |cdocsfi1.tex|&forwarding file for final version of chapter 1\\
% |cdocsfi2.tex|&forwarding file for final version of chapter 2\\
% \end{tabular}
% \end{center}
% Each of the eight files can be compiled directly by the \LaTeX{} compiler.
%
% %%%%%%%%%%%%%%%%%%%%%%%%%%%%%%%%%%%%%%
% \paragraph{Main File.}
%
% The main file is called |cdocsamp.tex|.
%
% Load the \textsf{childdoc} definitions and
% declare the filename for the main document:
%    \begin{macrocode}
\input{childdoc.def}
\childdocmain{}
%    \end{macrocode}

% Optional override for |\version| flag:
%    \begin{macrocode}
%%\ifchilddoc\else\providecommand{\version}{draft}\fi
%    \end{macrocode}

% Define the default values for the |\version| flag
% (|final| for the main file and |draft| for childs):
%    \begin{macrocode}
\ifchilddoc
\providecommand{\version}{draft}
\else
\providecommand{\version}{final}
\fi
%    \end{macrocode}

% Load the standard document class:
%    \begin{macrocode}
\documentclass[12pt]{article}
%    \end{macrocode}

% Start the document body:
%    \begin{macrocode}
\begin{document}
%    \end{macrocode}

% Declare a title page.
% Print title, part of document being processed and version flag:
%    \begin{macrocode}
\addtocounter{page}{-1}
\begin{center}
{\LARGE\bfseries{}childdoc example\par}
\vspace{1cm}
\ifchilddoc
\ifchilddocmanual part\else chapter\fi:
`\childdocname' of `\childdocjob'\par
\else
main document: `\childdocjob'\par
\fi
version: \version\par
\end{center}
\newpage
%    \end{macrocode}

% Manually include selected file,
% otherwise process as usual:
%    \begin{macrocode}
\ifchilddocmanual
\section*{part `\childdocname'}
\input{\childdocname}
\else
%    \end{macrocode}

% Include the two chapters:
%    \begin{macrocode}
\include{cdocsch1}
\include{cdocsch2}
%    \end{macrocode}

% Include the two parts unless only chapters should be displayed:
%    \begin{macrocode}
\ifchilddoc\else
\section{part three}
\input{cdocspt3}
\section{part four}
\input{cdocspt4}
\fi
%    \end{macrocode}

% Process as usual until here:
%    \begin{macrocode}
\fi
%    \end{macrocode}

% End of document body:
%    \begin{macrocode}
\end{document}
%    \end{macrocode}
%\iffalse
%</samplemain>
%\fi
%
% %%%%%%%%%%%%%%%%%%%%%%%%%%%%%%%%%%%%%%
% \paragraph{Chapter Include Files.}
%
% The include files are called |cdocsch1.tex| and |cdocsch2.tex|.
%
%\iffalse
%<*samplechap1|samplechap2>
%\fi

% Optional override for |\version| flag:
%    \begin{macrocode}
%%\providecommand{\version}{final}
%    \end{macrocode}

% Include the main document:
%    \begin{macrocode}
\input{childdoc.def}
\childdocof{cdocsamp}
%    \end{macrocode}

%\iffalse
%</samplechap1|samplechap2>
%\fi
%
%\iffalse
%<*samplechap1>
%\fi
% Some text for chapter 1:
%    \begin{macrocode}
\section{one}
some text in chapter one
%    \end{macrocode}

%\iffalse
%</samplechap1>
%\fi
% Some text for chapter 2:
%\iffalse
%<*samplechap2>
%\fi
%    \begin{macrocode}
\section{two}
more text in chapter two
%    \end{macrocode}

%\iffalse
%</samplechap2>
%\fi
%
% %%%%%%%%%%%%%%%%%%%%%%%%%%%%%%%%%%%%%%
% \paragraph{Part Include Files.}
%
% The include files are called |cdocspt3.tex| and |cdocspt4.tex|.
%
%\iffalse
%<*samplepart3|samplepart4>
%\fi

% Optional override for |\version| flag:
%    \begin{macrocode}
%%\providecommand{\version}{final}
%    \end{macrocode}

% Include the main document:
%    \begin{macrocode}
\input{childdoc.def}
\childdocby{cdocsamp}
%    \end{macrocode}

%\iffalse
%</samplepart3|samplepart4>
%\fi
%
%\iffalse
%<*samplepart3>
%\fi
% Some text for part 3:
%    \begin{macrocode}
some text in part three
%    \end{macrocode}

%\iffalse
%</samplepart3>
%\fi
% Some text for part 4:
%\iffalse
%<*samplepart4>
%\fi
%    \begin{macrocode}
more text in part four
%    \end{macrocode}

%\iffalse
%</samplepart4>
%\fi
%
% %%%%%%%%%%%%%%%%%%%%%%%%%%%%%%%%%%%%%%
% \paragraph{Forwarding for a Complete Draft.}
%
% The following forwarding file |cdocsdrf.tex|
% compiles the main document in draft mode:
%\iffalse
%<*sampledraft>
%\fi
%    \begin{macrocode}
\def\version{draft}
\input{childdoc.def}
\childdocforward{cdocsamp}
%    \end{macrocode}

%\iffalse
%</sampledraft>
%\fi
%
% %%%%%%%%%%%%%%%%%%%%%%%%%%%%%%%%%%%%%%
% \paragraph{Forwarding for Final Version of the Chapters.}
%
% The following forwarding files |cdocsfn1.tex| and |cdocsfn2.tex|
% (with identical content)
% compile the final versions of the child documents
% |cdocsch1.tex| and |cdocsch2.tex|, respectively:
%\iffalse
%<*samplefinal>
%\fi
%    \begin{macrocode}
\def\version{final}
\input{childdoc.def}
\childdocforwardprefix[cdocsamp]{cdocsfn}{cdocsch}
%    \end{macrocode}

%\iffalse
%</samplefinal>
%\fi
%
% %%%%%%%%%%%%%%%%%%%%%%%%%%%%%%%%%%%%%%
% \paragraph{Command Line Processing.}
%
% The following three command lines generate the output files
% |cdocscld|, |cdocscl1| and |cdocscl2|
% which should be identical to
% |cdocsdrf|, |cdocsch1| and |cdocsfn2|, respectively:
% \begin{center}
% \begin{tabular}{l}
% |latex -jobname cdocscld \|\\
% |  "\def\version{draft}\input{childdoc.def}\childdocforward{cdocsamp}"|\\
% |latex -jobname cdocscl1 \|\\
% |  "\input{childdoc.def}\childdocforward[cdocsamp]{cdocsch1}"|\\
% |latex -jobname cdocscl2 \|\\
% |  "\def\version{final}\input{childdoc.def}\childdocforward{cdocsch2}"|
% \end{tabular}
% \end{center}
% Note that the trailing backslash on each first line
% merely continues the input to the second line
% (for convenient cut ant paste).
% Furthermore, the command |latex| can be replaced by any
% of its alternative versions such as |pdflatex|.
%
% %%%%%%%%%%%%%%%%%%%%%%%%%%%%%%%%%%%%%%%%%%%%%%%%%%%%%%%%%%%%%%%%%%%%%%%%%%%%%%
% %%%%%%%%%%%%%%%%%%%%%%%%%%%%%%%%%%%%%%%%%%%%%%%%%%%%%%%%%%%%%%%%%%%%%%%%%%%%%%
% \section{Implementation}
%\iffalse
%<*package>
%\fi
%
% This section describes the definitions file |childdoc.def|.

% The definitions cannot be loaded using |\usepackage| or |\RequirePackage|
% which has a mechanism to prevent loading a style file more than once.
% When loading the definitions by means of |\input|
% multiple instances have to be prevented manually:
%\iffalse
%This code needs to be before the `\ProvidesFile' directive
%which is defined at the beginning of this file.
%Therefore it is also placed there and commented out here.
%</package>
%<*discard>
%\fi
%    \begin{macrocode}
\ifdefined\childdocmain\endinput\fi
%    \end{macrocode}
%\iffalse
%</discard>
%<*package>
%\fi
%
% \macro{\ifchilddoc}
% \macro{\ifchilddocmanual}
% The conditional |\ifchilddoc| tells whether a
% child (true) or main (false) document is being compiled.
% The conditional |\ifchilddocmanual| tells whether
% the |\includeonly| mechanism is used (false) or
% the selection of child files must be performed manually (true).
% The definitions initialise to false:
%    \begin{macrocode}
\newif\ifchilddoc
\newif\ifchilddocmanual
%    \end{macrocode}

% \macro{\childdocname}
% \macro{\childdocjob}
% The macro |\childdocname| stores the name of the main document
% to be compiled. The macro |\childdocjob| stores the name of
% the document on which the \LaTeX{} compiler was originally invoked.
% The content of |\jobname| cannot be compared
% to filenames specified in the source due to different catcodes.
% The following code rescans |\jobname|, stores the result
% in |\childdocname| and saves a copy in |\childdocjob|:
%    \begin{macrocode}
\edef\childdocname{\scantokens\expandafter{\jobname\noexpand}}
\let\childdocjob\childdocname
%    \end{macrocode}

% \macro{\childdocdisable}
% The macro |\childdocdisable| prevents the main file
% from being processed more than once.
% At this stage, the main document command |\childdocmain|
% is assumed to be called once again where it should do nothing.
% Any subsequent call to it should prevent
% a secondary processing of the main document
% It overwrites the forwarding commands
% |\childdocof| and |\childdocforward|
% with empty macros to prevent further inclusions of the main document:
%    \begin{macrocode}
\newcommand{\childdocdisable}
{
  \renewcommand{\childdocmain}[1]{\renewcommand{\childdocmain}[1]{\endinput}}
  \renewcommand{\childdocof}[1]{}
  \renewcommand{\childdocby}[2][]{}
  \renewcommand{\childdocforward}[2][]{}
  \renewcommand{\childdocdisable}{}
}
%    \end{macrocode}

% \macro{\childdocmain}
% The macro |\childdocmain| is to be called at the top of the main file
% with nothing or the main filename (without extension) as argument.
% First, it breaks loops.
% If the argument is not empty and does not match |\childdocname|
% (which is set by the first inclusion of |childdoc.def|),
% |\ifchilddoc| is set to true, |\includeonly| is applied to the child file
% and |\jobname| is set to the main file
% (for proper handling of |.aux| files):
%    \begin{macrocode}
\newcommand{\childdocmain}[1]
{
  \childdocdisable\childdocmain{}
  \if?#1?\else
    \begingroup
      \def\childdoctmp{#1}
      \ifx\childdoctmp\childdocname
        \def\childdoctmp{}
      \else
        \def\childdoctmp
        {
          \childdoctrue
          \includeonly{\childdocname}
          \def\childdocjob{#1}
          \def\jobname{#1}
        }
      \fi
      \expandafter
    \endgroup
    \childdoctmp
  \fi
}
%    \end{macrocode}

% \macro{\childdocof}
% The command |\childdocof| redirects
% compilation to the main file |#1|.
%    \begin{macrocode}
\newcommand{\childdocof}[1]
{
  \childdocdisable
  \childdoctrue
  \includeonly{\childdocname}
  \def\jobname{#1}
  \def\childdocjob{#1}
  \input{#1}
}
%    \end{macrocode}

% \macro{\childdocby}
% The command |\childdocby| ....
%    \begin{macrocode}
\newcommand{\childdocby}[2][]
{
  \childdocdisable
  \childdoctrue
  \childdocmanualtrue
  \if?#1?\else
    \def\jobname{#2}
  \fi
  \def\childdocjob{#2}
  \input{#2}
  \endinput
}
%    \end{macrocode}

% \macro{\childdocforward}
% The command |\childdocforward| redirects
% compilation to the main file or
% (if the optional argument is given) a child file.
% Parameters are set as if the main file
% or a child file starting with |\childdocof| was compiled.
% Then compilation is handed over to the main file:
%    \begin{macrocode}
\newcommand{\childdocforward}[2][]
{
  \begingroup
    \if?#1?
      \def\childdoctmp
      {
        \def\childdocname{#2}
        \def\childdocjob{#2}
        \def\jobname{#2}
        \input{#2}
        \endinput
      }
    \else
      \def\childdoctmp
      {
        \childdocdisable
        \def\childdocname{#2}
        \childdoctrue
        \includeonly{#2}
        \def\childdocjob{#1}
        \def\jobname{#1}
        \input{#1}
        \endinput
      }
    \fi
    \expandafter
  \endgroup
  \childdoctmp
}
%    \end{macrocode}

% \macro{\childdocforwardprefix}
% The command |\childdocforwardprefix| redirects
% compilation to the main or a child file by means of a pattern.
% The prefix |#1| in the current filename is replaced by |#2|
% and the suffix of the current filename is kept
% (it is assumed that the filename does not contain the substring `|~~~|'
% which is used as a delimiter).
% Compilation is handed over to the new file by |\childdocforward|:
%    \begin{macrocode}
\newcommand{\childdocforwardprefix}[3][]
{
  \begingroup
    \def\childdocextract #2##1~~~{\def\childdoctmp{\childdocforward[#1]{#3##1}}}
    \expandafter\childdocextract\childdocname~~~
    \expandafter
  \endgroup
  \childdoctmp
}
%    \end{macrocode}

% \macro{\childdoc}
% The deprecated macro |\childdoc| is a legacy version of |\childdocmain|:
%    \begin{macrocode}
\newcommand{\childdoc}{\childdocmain}
%    \end{macrocode}

% \macro{\childdocredirect}
% The deprecated macro |\childdocredirect| is a legacy version
% of |\childdocforward| and |\childdocforwardprefix|:
%    \begin{macrocode}
\newcommand{\childdocredirect}[2][]
{
  \begingroup
    \if?#1?
      \def\childdoctmp{\childdocforward{#2}}
    \else
      \def\childdoctmp{\childdocforwardprefix{#1}{#2}}
    \fi
    \expandafter
  \endgroup
  \childdoctmp
}
%    \end{macrocode}

%\iffalse
%</package>
%\fi
%
\endinput
|
and perform the replacements as outlined below.
Instead of |\childdocmain{|\textit{main}|}| add the following code
to the top of the main file:
%
\begin{center}
\begin{tabular}{l}
|\||ifdefined\childdocname\endinput\||fi\newif\ifchilddoc|\\
|\edef\childdocname{\scantokens\expandafter{\jobname\noexpand}}|\\
|\def\childdocmain{|\textit{main}|}\||ifx\childdocmain\childdocname\||else|\\
|\childdoctrue\includeonly{\childdocname}\let\jobname\childdocmain\||fi|\\
\end{tabular}
\end{center}
%
Instead of |\childdocof{|\textit{main}|}| just include the main file
at the top of each child file:
%
\begin{center}
|\input{|\textit{main}|}|
\end{center}
%
A simple redirection |\childdocforward{|\textit{dest}|}| is achieved by:
%
\begin{center}
|\def\jobname{|\textit{dest}|}\input{\jobname}|
\end{center}
%
The redirection with prefix
|\childdocforwardprefix[|\textit{prefix}|]{|\textit{dest}|}|
is accomplished by:
%
\begin{center}
\begin{tabular}{l}
|{\edef\jobname{\scantokens\expandafter{\jobname\noexpand}}|\\
|\def\redirectjob |\textit{prefix}|#1~~~{\gdef\jobname{|\textit{dest}|#1}}|\\
|\expandafter\redirectjob\jobname~~~}\input{\jobname}|
\end{tabular}
\end{center}

In an alternative approach,
child documents can be compiled by a specific command line
without additional code or specific definitions:
%
\begin{center}
|... -jobname "|\textit{target}|" "|[\textit{flags}]%
|\includeonly{|\textit{dest}|}\input{|\textit{main}|}"|
\end{center}
%

%%%%%%%%%%%%%%%%%%%%%%%%%%%%%%%%%%%%%%%%%%%%%%%%%%%%%%%%%%%%%%%%%%%%%%%%%%%%%%%%
%%%%%%%%%%%%%%%%%%%%%%%%%%%%%%%%%%%%%%%%%%%%%%%%%%%%%%%%%%%%%%%%%%%%%%%%%%%%%%%%
\section{Information}

%%%%%%%%%%%%%%%%%%%%%%%%%%%%%%%%%%%%%%%%%%%%%%%%%%%%%%%%%%%%%%%%%%%%%%%%%%%%%%%%
\subsection{Copyright}

Copyright \copyright{} 2017--2018 Niklas Beisert

This work may be distributed and/or modified under the
conditions of the \LaTeX{} Project Public License, either version 1.3
of this license or (at your option) any later version.
The latest version of this license is in
  \url{http://www.latex-project.org/lppl.txt}
and version 1.3 or later is part of all distributions of \LaTeX{}
version 2005/12/01 or later.

This work has the LPPL maintenance status `maintained'.

The Current Maintainer of this work is Niklas Beisert.

This work consists of the files |README.txt|, |childdoc.ins| and |childdoc.dtx|
as well as the derived files |childdoc.def|, |cdocsamp.tex|
with |cdocsch1.tex|, |cdocsch2.tex|, |cdocspt3.tex|, |cdocspt4.tex|,
|cdocsdrf.tex|, |cdocsfn1.tex|, |cdocsfn2.tex|
as well as |childdoc.pdf|.

%%%%%%%%%%%%%%%%%%%%%%%%%%%%%%%%%%%%%%%%%%%%%%%%%%%%%%%%%%%%%%%%%%%%%%%%%%%%%%%%
\subsection{Files and Installation}

The package consists of the files:
%
\begin{center}
\begin{tabular}{ll}
    |README.txt|   & readme file \\
    |childdoc.ins| & installation file \\
    |childdoc.dtx| & source file \\
    |childdoc.def| & definition file \\
    |cdocsamp.tex| & sample main file \\
    |cdocsch1.tex| & sample include file \\
    |cdocsch2.tex| & sample include file \\
    |cdocspt3.tex| & sample part file \\
    |cdocspt4.tex| & sample part file \\
    |cdocsdrf.tex| & sample redirection file \\
    |cdocsfn1.tex| & sample redirection file \\
    |cdocsfn2.tex| & sample redirection file \\
    |childdoc.pdf| & manual
\end{tabular}
\end{center}
%
The distribution consists of the files
|README.txt|, |childdoc.ins| and |childdoc.dtx|.
%
\begin{itemize}
\item
Run (pdf)\LaTeX{} on |childdoc.dtx|
to compile the manual |childdoc.pdf| (this file).
\item
Run \LaTeX{} on |childdoc.ins| to create the definitions file |childdoc.def|
and the sample |cdocsamp.tex| with include files
|cdocsch1.tex|, |cdocsch2.tex|, |cdocspt3.tex|, |cdocspt4.tex|,
|cdocsdrf.tex|, |cdocsfn1.tex|, |cdocsfn2.tex|.
Then copy the file |childdoc.def| to an appropriate directory of your \LaTeX{}
distribution, e.g.\ \textit{texmf-root}|/tex/latex/childdoc|.
\end{itemize}

%%%%%%%%%%%%%%%%%%%%%%%%%%%%%%%%%%%%%%%%%%%%%%%%%%%%%%%%%%%%%%%%%%%%%%%%%%%%%%%%
\subsection{Related CTAN Packages}

There are several other packages which offer a similar functionality:
%
\begin{itemize}
\item
The packages
\href{http://ctan.org/pkg/docmute}{\textsf{docmute}},
\href{http://ctan.org/pkg/includex}{\textsf{includex}} and
\href{http://ctan.org/pkg/standalone}{\textsf{standalone}}
provide commands to include only the document body of
a child file thus allowing both files to be compiled individually.
\item
The packages \href{http://ctan.org/pkg/subdocs}{\textsf{subdocs}}
and \href{http://ctan.org/pkg/subfiles}{\textsf{subfiles}}
provide structures in which the main and child documents can be
encapsulated and allowing them to be compiled individually.
The inclusion mechanism is different from the conventional |\include|.
\item
The package \href{http://ctan.org/pkg/combine}{\textsf{combine}}
is an elaborate solution to combine several documents into one.
\end{itemize}
%
See also the CTAN topic \href{http://ctan.org/topic/subdocs}{\textsf{subdocs}}
for further related packages.
The present package differs from the above solutions in that
a document structure constructed with the conventional |\include| mechanism
just needs two extra commands at the top of every file
such that all constituent files can be compiled individually.

%%%%%%%%%%%%%%%%%%%%%%%%%%%%%%%%%%%%%%%%%%%%%%%%%%%%%%%%%%%%%%%%%%%%%%%%%%%%%%%%
%\subsection{Feature Suggestions}
%
%The following is a list of features which may be useful for future
%versions of this package:
%%
%\begin{itemize}
%\item
%\ldots
%\end{itemize}

%%%%%%%%%%%%%%%%%%%%%%%%%%%%%%%%%%%%%%%%%%%%%%%%%%%%%%%%%%%%%%%%%%%%%%%%%%%%%%%%
\subsection{Revision History}

%%%%%%%%%%%%%%%%%%%%%%%%%%%%%%%%%%%%%%%%
\paragraph{v2.0:} 2018/12/30

\begin{itemize}
\item
immediate forward processing
\item
added |\childdocby| mechanism
\item
manual restructured
\end{itemize}

%%%%%%%%%%%%%%%%%%%%%%%%%%%%%%%%%%%%%%%%
\paragraph{v1.6:} 2018/01/17

\begin{itemize}
\item
application for development of include files
\item
corrections to manual
\end{itemize}

%%%%%%%%%%%%%%%%%%%%%%%%%%%%%%%%%%%%%%%%
\paragraph{v1.5:} 2017/05/21

\begin{itemize}
\item
more complete structuring introduced
\item
|\childdocof| introduced
\item
|\childdoc| renamed to |\childdocmain|
\item
|\childredirect| renamed to |\childdocforward| and |\childdocforwardprefix|
and functionality expanded
\end{itemize}

%%%%%%%%%%%%%%%%%%%%%%%%%%%%%%%%%%%%%%%%
\paragraph{v1.0:} 2017/04/27

\begin{itemize}
\item
manual and install package
\item
first version published on CTAN
\end{itemize}

%%%%%%%%%%%%%%%%%%%%%%%%%%%%%%%%%%%%%%%%
\paragraph{v0.6:} 2017/04/26

\begin{itemize}
\item
redirection mechanism added
\end{itemize}

%%%%%%%%%%%%%%%%%%%%%%%%%%%%%%%%%%%%%%%%
\paragraph{v0.5:} 2017/04/26

\begin{itemize}
\item
functionality in definition file
\end{itemize}


%%%%%%%%%%%%%%%%%%%%%%%%%%%%%%%%%%%%%%%%%%%%%%%%%%%%%%%%%%%%%%%%%%%%%%%%%%%%%%%%
%%%%%%%%%%%%%%%%%%%%%%%%%%%%%%%%%%%%%%%%%%%%%%%%%%%%%%%%%%%%%%%%%%%%%%%%%%%%%%%%
%%%%%%%%%%%%%%%%%%%%%%%%%%%%%%%%%%%%%%%%%%%%%%%%%%%%%%%%%%%%%%%%%%%%%%%%%%%%%%%%
\appendix

\settowidth\MacroIndent{\rmfamily\scriptsize 000\ }

 \DocInput{childdoc.dtx}

\end{document}
%</driver>
% \fi
%
% %%%%%%%%%%%%%%%%%%%%%%%%%%%%%%%%%%%%%%%%%%%%%%%%%%%%%%%%%%%%%%%%%%%%%%%%%%%%%%
% %%%%%%%%%%%%%%%%%%%%%%%%%%%%%%%%%%%%%%%%%%%%%%%%%%%%%%%%%%%%%%%%%%%%%%%%%%%%%%
% \section{Sample}
%\iffalse
%<*samplemain>
%\fi
%
% The following presents a sample document
% with two chapters, two parts, a title page,
% a compile flag as well as three forwarding files to set the flag.
% It consists of eight |.tex| files:
% \begin{center}
% \begin{tabular}{ll}
% |cdocsamp.tex|&main file\\
% |cdocsch1.tex|&include file for chapter 1\\
% |cdocsch2.tex|&include file for chapter 2\\
% |cdocspt3.tex|&include file for part 3\\
% |cdocspt4.tex|&include file for part 4\\
% |cdocsdrf.tex|&forwarding file for main file in draft mode\\
% |cdocsfi1.tex|&forwarding file for final version of chapter 1\\
% |cdocsfi2.tex|&forwarding file for final version of chapter 2\\
% \end{tabular}
% \end{center}
% Each of the eight files can be compiled directly by the \LaTeX{} compiler.
%
% %%%%%%%%%%%%%%%%%%%%%%%%%%%%%%%%%%%%%%
% \paragraph{Main File.}
%
% The main file is called |cdocsamp.tex|.
%
% Load the \textsf{childdoc} definitions and
% declare the filename for the main document:
%    \begin{macrocode}
% \iffalse
%
% childdoc.dtx Copyright (C) 2017-2018 Niklas Beisert
%
% This work may be distributed and/or modified under the
% conditions of the LaTeX Project Public License, either version 1.3
% of this license or (at your option) any later version.
% The latest version of this license is in
%   http://www.latex-project.org/lppl.txt
% and version 1.3 or later is part of all distributions of LaTeX
% version 2005/12/01 or later.
%
% This work has the LPPL maintenance status `maintained'.
%
% The Current Maintainer of this work is Niklas Beisert.
%
% This work consists of the files childdoc.dtx and childdoc.ins
% and the derived files childdoc.def and cdocsamp.tex with
% cdocsch1.tex, cdocsch2.tex, cdocsdrf.tex, cdocsfn1.tex, cdocsfn2.tex.
%
%<package>\ifdefined\childdocmain\endinput\fi
%<package>\ProvidesFile{childdoc.def}[2018/12/30 v2.0 child document driver]
%<samplemain>\ProvidesFile{cdocsamp.tex}[2018/12/30 v2.0 sample for childdoc]
%<*driver>
%\ProvidesFile{childdoc.drv}[2018/12/30 v2.0 childdoc reference manual file]
\PassOptionsToClass{10pt,a4paper}{article}
\documentclass{ltxdoc}

\usepackage[margin=35mm]{geometry}
\usepackage{hyperref}
\usepackage{hyperxmp}
\usepackage[usenames]{color}

\hypersetup{colorlinks=true}
\hypersetup{pdfstartview=FitH}
\hypersetup{pdfpagemode=UseNone}
\hypersetup{pdfsource={}}
\hypersetup{pdflang={en-UK}}
\hypersetup{pdfcopyright={Copyright 2017-2018 Niklas Beisert.
  This work may be distributed and/or modified under the
  conditions of the LaTeX Project Public License, either version 1.3
  of this license or (at your option) any later version.}}
\hypersetup{pdflicenseurl={http://www.latex-project.org/lppl.txt}}
\hypersetup{pdfcontactaddress={ETH Zurich, ITP, HIT K,
  Wolfgang-Pauli-Strasse 27}}
\hypersetup{pdfcontactpostcode={8093}}
\hypersetup{pdfcontactcity={Zurich}}
\hypersetup{pdfcontactcountry={Switzerland}}
\hypersetup{pdfcontactemail={nbeisert@itp.phys.ethz.ch}}
\hypersetup{pdfcontacturl={http://people.phys.ethz.ch/\xmptilde nbeisert/}}

\newcommand{\secref}[1]{\hyperref[#1]{section \ref*{#1}}}

\parskip1ex
\parindent0pt
\let\olditemize\itemize
\def\itemize{\olditemize\parskip0pt}

\begin{document}

\title{The \textsf{childdoc} Package}
\hypersetup{pdftitle={The childdoc Package}}
\author{Niklas Beisert\\[2ex]
  Institut f\"ur Theoretische Physik\\
  Eidgen\"ossische Technische Hochschule Z\"urich\\
  Wolfgang-Pauli-Strasse 27, 8093 Z\"urich, Switzerland\\[1ex]
  \href{mailto:nbeisert@itp.phys.ethz.ch}
  {\texttt{nbeisert@itp.phys.ethz.ch}}}
\hypersetup{pdfauthor={Niklas Beisert}}
\hypersetup{pdfsubject={Manual for the LaTeX2e Package childdoc}}
\date{30 December 2018, \textsf{v2.0}}
\maketitle

\begin{abstract}\noindent
\textsf{childdoc} is a \LaTeXe{} package
that enables the direct compilation
of document sections included by |\include|
to individual files.
\end{abstract}

\begingroup
\parskip0ex
\tableofcontents
\endgroup

%%%%%%%%%%%%%%%%%%%%%%%%%%%%%%%%%%%%%%%%%%%%%%%%%%%%%%%%%%%%%%%%%%%%%%%%%%%%%%%%
%%%%%%%%%%%%%%%%%%%%%%%%%%%%%%%%%%%%%%%%%%%%%%%%%%%%%%%%%%%%%%%%%%%%%%%%%%%%%%%%
\section{Introduction}

\LaTeX{} provides a mechanism to structure a large document (such as a book)
into a main file and several child files (containing the chapters)
using the |\include| command.
This mechanism is beneficial for documents
which span hundreds of pages in order to
make the source file(s) more manageable.
Moreover, compilation can be restricted to
selected child files by means of the |\includeonly| command.
The latter feature can be used to reduce the compilation time while editing
(this was significantly more useful in the earlier days of \LaTeX{})
or to generate a smaller document which is easier to navigate.
Another application of |\includeonly| is to generate
documents consisting of selected parts of the complete document.

However, there are a few drawbacks of the plain |\include| mechanism:
\begin{itemize}
\item
The child files cannot be compiled on their own,
they can only be compiled via the main file.
A naive editing environment
(such as a text editor with an option
to have the current file processed by \LaTeX)
may require one to switch to the main file before compiling;
attempting to compile the child file produces errors.
\item
The main file must be modified (each time)
to adjust the |\includeonly| command
to the present needs. This easily leaves the main file in a messy state.
\item
The generated document will always carry the filename
of the main document. This is inconvenient if
several child files are to be compiled and
to be kept for distribution.
\end{itemize}

The present package provides a simple interface
to make child files individually compilable by \LaTeX{}.
Compiling a child file then has the same effect as compiling
the main file with an |\includeonly| command
to select the appropriate child.
Moreover the generated document will carry the name of the child
rather than the main file.
This resolves all three above issues.

This feature is meant to make the editing of books,
thesis documents and lecture notes somewhat more convenient.
However, the package can also be used efficiently for
composing a series of documents (such as exercise sheets)
which are typically distributed individually.
It then assists the author in generating the individual documents
(potentially in different versions)
as well as a document containing the collected series.
Another application is in developing style files
or other kinds of included material
where compilation of the style file could redirect
to a sample or test file.

%%%%%%%%%%%%%%%%%%%%%%%%%%%%%%%%%%%%%%%%%%%%%%%%%%%%%%%%%%%%%%%%%%%%%%%%%%%%%%%%
%%%%%%%%%%%%%%%%%%%%%%%%%%%%%%%%%%%%%%%%%%%%%%%%%%%%%%%%%%%%%%%%%%%%%%%%%%%%%%%%
\section{Usage}

First of all, the package \textsf{childdoc} is \emph{not} a standard
\LaTeXe{} |.sty| style file! Therefore it needs to be invoked in
a non-standard way.

%%%%%%%%%%%%%%%%%%%%%%%%%%%%%%%%%%%%%%%%%%%%%%%%%%%%%%%%%%%%%%%%%%%%%%%%%%%%%%%%
\subsection{Included Files}
\label{sec:include}

%%%%%%%%%%%%%%%%%%%%%%%%%%%%%%%%%%%%%%%%
\DescribeMacro{\childdocmain}
To use the package, add the commands
\begin{center}
\begin{tabular}{l}
|\input{childdoc.def}|\\
|\childdocmain{}|\\
\end{tabular}
\end{center}
at the very top of the main \LaTeX{} file,
in particular \emph{before} the |\documentclass| statement!
The argument of |\childdocmain| should be left empty
(but it must be present).

%%%%%%%%%%%%%%%%%%%%%%%%%%%%%%%%%%%%%%%%
\DescribeMacro{\childdocof}
Furthermore, add the commands
\begin{center}
\begin{tabular}{l}
|\input{childdoc.def}|\\
|\childdocof{|\textit{main}|}|\\
\end{tabular}
\end{center}
at the top of every child file \textit{child}
which is included by |\include{|\textit{child}|}|
from within the main file
(or at least for those files to be compiled individually).
The argument \textit{main} must be the filename of the main file.

There are a couple of
considerations in setting up the main and child documents:

%%%%%%%%%%%%%%%%%%%%%%%%%%%%%%%%%%%%%%%%
\paragraph{Restrictions.}

Please note the following restrictions:
\begin{itemize}
\item
|\childdocmain| must be called with one argument \textit{main}
to ensure compatibility with earlier version of the package.
It must either be empty (|\childdocmain{}|)
or precisely match the filename of the main file in which it is specified.
See \secref{sec:detection} for further information.
\item
The filename \textit{main} must be specified without the |.tex| extension.
\item
The filename \textit{main} is case sensitive
(even in case-insensitive file systems)
due to internal string comparison.
\item
The argument \textit{main} should be fully expanded, it cannot be a macro.
\item
Subdirectories and special characters should be avoided in filenames.
\item
The command |\childdocmain{|\textit{main}|}| must be followed by a whitespace.
It should not be followed immediately by another command
or by a comment mark `|%|'.
This is because the \TeX{} parser reads the token immediately following
the argument of |\childdocmain| and puts it
at the beginning of every child section;
however, a white\-space is ignored.
\end{itemize}

%%%%%%%%%%%%%%%%%%%%%%%%%%%%%%%%%%%%%%%%
\paragraph{Content of Main File.}

It is advisable to place all content in the child files included by |\include|.
Any output contained in the main file will appear in all child documents
unless suppressed manually;
it cannot be suppressed automatically by the |\includeonly| directive
and thus should normally be avoided.
A method to include some content in the main file
by means of conditional processing is described in \secref{sec:conditional}.

%%%%%%%%%%%%%%%%%%%%%%%%%%%%%%%%%%%%%%%%
\paragraph{Page Numbering.}

When only a part of the document is compiled,
the appropriate numbering of pages
(as well as other status parameters)
is determined from the |.aux| files.
The latter contain information from previous passes.
However this information needs to propagate through
all intermediate child documents.
Therefore the page numbering in child documents may well
be inconsistent until the complete document is compiled at least once.

A useful (if unconventional) way to always ensure a consistent
page numbering is to restart the numbering in each child document
and denote the pages by `\textit{child}|.|\textit{page}'
where \textit{child} represents the chapter/section number of the child file.
This can be achieved by the command
|\numberwithin{page}{|\textit{child}|}|
of the \textsf{amsmath} package
where \textit{child} can be |chapter| or |section|
depending on the chosen structuring.
Alternatively, one can modify the macro |\thepage| appropriately
and reset the counter |page| at the start of each child file.

%%%%%%%%%%%%%%%%%%%%%%%%%%%%%%%%%%%%%%%%%%%%%%%%%%%%%%%%%%%%%%%%%%%%%%%%%%%%%%%%
\subsection{Conditional Processing}
\label{sec:conditional}

The package provides a mechanism to compile different versions
of a document. To customise the versions further some conditional processing
can come in handy to distinguish which version is being compiled.
The package provides two macros to describe the compilation context:

%%%%%%%%%%%%%%%%%%%%%%%%%%%%%%%%%%%%%%%%
\DescribeMacro{\ifchilddoc}
The conditional |\ifchilddoc| distinguishes between the compilation of
child documents and the main document:
%
\begin{center}
|\ifchilddoc |\textit{child-code}| |[|\||else |\textit{main-code}]| \||fi|
\end{center}

%%%%%%%%%%%%%%%%%%%%%%%%%%%%%%%%%%%%%%%%
\DescribeMacro{\childdocname}
\DescribeMacro{\childdocjob}
The macro |\childdocname| contains the filename (without extension)
of the main or child file being processed.
Note that |\childdocjob| will always contain the name of the main file.

%%%%%%%%%%%%%%%%%%%%%%%%%%%%%%%%%%%%%%%%
\paragraph{Title Page.}

Conditional processing can be used to include a title or banner page
in the main document when proper precautions are taken.
Importantly, the code in the main file should ensure that the page counter
(as well as other status parameters which are stored in the |.aux| files)
takes the same value after the conditional processing.
Otherwise the page numbers may take divergent values
depending on which part is compiled.

For example, a title page could be declared by:
%
\begin{center}
\begin{tabular}{l}
|\ifchilddoc\||else|\\
|\addtocounter{page}{-1}|\\
\textit{code for title page}\\
|\newpage|\\
|\||fi|
\end{tabular}
\end{center}
%
A banner page for the child documents can be generated by:
%
\begin{center}
\begin{tabular}{l}
|\ifchilddoc|\\
|\addtocounter{page}{-1}|\\
\textit{code for banner page}\\
|\newpage|\\
|\||fi|
\end{tabular}
\end{center}
%
Here one could write a message such as:
\begin{center}
|This is the part \childdocname{} of \childdocjob{}.|
\end{center}

%%%%%%%%%%%%%%%%%%%%%%%%%%%%%%%%%%%%%%%%%%%%%%%%%%%%%%%%%%%%%%%%%%%%%%%%%%%%%%%%
\subsection{Flags}
\label{sec:flags}

The package makes it easy to generate different versions
of the main or child documents.
To this end compilation flags can be defined
and assigned different default values.
They will be particularly useful in conjunction
with the forwarding mechanism described in \secref{sec:forward}.

For example, it may be useful to have a flag |\version|
which can be set to |draft| or |final|.
The document source will contain some conditional code
depending on the value of |\version|.
Suppose further, the flag should default to |final| for the main file
and to |draft| for child files
which is a natural assignment for editing the document.
This is achieved by placing the following code
in the preamble of the main document
(below the |\childdocmain| directive):
%
\begin{center}
\begin{tabular}{l}
|\ifchilddoc|\\
|\providecommand{\version}{draft}|\\
|\||else|\\
|\providecommand{\version}{final}|\\
|\||fi|
\end{tabular}
\end{center}
%
The definition by |\providecommand| makes sure
that previous definitions are not overwritten.
Further statements |\providecommand{\version}{...}|
can thus be added before the above code to override it.

For the main file, one might add a line
(between |\childdocmain| and the above block)
%
\begin{center}
|%\ifchilddoc\||else\providecommand{\version}{draft}\||fi|
\end{center}
%
which can be uncommented to produce a draft version.
Likewise one can add a line to the very top of a child file
(above the |\childdocof{|\textit{main}|}| directive)
%
\begin{center}
|%\providecommand{\version}{final}|
\end{center}
%
which can be uncommented to produce the final version of this child document.

%%%%%%%%%%%%%%%%%%%%%%%%%%%%%%%%%%%%%%%%%%%%%%%%%%%%%%%%%%%%%%%%%%%%%%%%%%%%%%%%
\subsection{Forwarding}
\label{sec:forward}

Different versions of the main or child documents
using compilation flags as described in \secref{sec:flags}
can be (permanently) stored in different files
for convenient compilation, viewing and distribution.
To this end, the package defines a command
to pass on compilation to a different file:

%%%%%%%%%%%%%%%%%%%%%%%%%%%%%%%%%%%%%%%%
\DescribeMacro{\childdocforward}
The command |\childdocforward| redirects processing to
another source file:
%
\begin{center}
\begin{tabular}{l}
|\input{childdoc.def}|\\
|\childdocforward[|\textit{main}|]{|\textit{dest}|}|\\
\end{tabular}
\end{center}
%
The argument \textit{dest} is the destination file
(without extension).
It should be the main file or one of the child files.
Note that further \textsf{childdoc} directives
such as |\childdocof| and |\childdocforward|
in the indicated file will be processed in this form.
The optional argument \textit{main}
passes on directly to the main file \textit{main}
while pretending to compile the child \textit{dest}.
This form behaves as if \textit{dest}
issues |\childdocof{|\textit{main}|}| right away,
and no further \textsf{childdoc} directives will be processed.

%%%%%%%%%%%%%%%%%%%%%%%%%%%%%%%%%%%%%%%%
\DescribeMacro{\...prefix}
In the alternative form |\childdocforwardprefix|,
%
\begin{center}
\begin{tabular}{l}
|\input{childdoc.def}|\\
|\childdocforwardprefix[|\textit{main}|]{|\textit{prefix}|}{|\textit{dest}|}|
\end{tabular}
\end{center}
%
the destination file is determined by a pattern
depending on the current file:
To make this work, the current file must be called
`{\textit{prefix}\hspace{0.2em}\textit{suffix}}'
with \textit{prefix} matching precisely the argument.
Processing is then passed on to the file
`{\textit{dest}\hspace{0.2em}\textit{suffix}}'.
Surely, the same effect is achieved by
directly specifying the
argument `{\textit{dest}\hspace{0.2em}\textit{suffix}}'
in the first form.
However, that requires to set up a different file
for each child. With the alternative form of the command
all these files can have exactly the same content
which simplifies setting them up and maintaining them.

For example, the following file |draft.tex|
with a compilation flag |\version| as described in \secref{sec:flags}
compiles the main document as a draft:
%
\begin{center}
\begin{tabular}{l}
|\def\version{draft}|\\
|\input{childdoc.def}|\\
|\childdocforward{|\textit{main}|}|
\end{tabular}
\end{center}
%
Likewise, the following files |final|\textit{nn}|.tex|
compile the final version of the child document
|child|\textit{nn}|.tex|:
%
\begin{center}
\begin{tabular}{l}
|\def\version{final}|\\
|\input{childdoc.def}|\\
|\childdocforwardprefix{final}{child}|
\end{tabular}
\end{center}
%

Note that when several versions of a main file and/or of each child file
are to be generated, it may be convenient to set up a |Makefile| or
shell script to automatise the process.

%%%%%%%%%%%%%%%%%%%%%%%%%%%%%%%%%%%%%%%%%%%%%%%%%%%%%%%%%%%%%%%%%%%%%%%%%%%%%%%%
\subsection{Command Line Processing}
\label{sec:commandline}

The effect of redirection files can also be achieved by invoking
the \LaTeX{} compiler with a more elaborate command line.
Most conveniently this should be done as part
of a shell script or a |Makefile|.

When using \textsf{childdoc} in the main file, the following
command lines effectively perform a redirection
(note that depending on the shell being used,
backslashes may have to be doubled: `|\|' $\to$ `|\\|'):
%
\begin{center}
|... -jobname "|\textit{target}|" |\\|"|[\textit{flags}]%
|\input{childdoc.def}\childdocforward[|\textit{main}|]{|\textit{dest}|}"|
\end{center}
%
Here \textit{target} is the name of the output file,
\textit{main} is the name of the main file
and \textit{dest} is the name of the main or child file to be processed
(all filenames without extensions).
The optional argument \textit{main} can be omitted
if \textit{main} matches \textit{dest}.
Optionally, compilation \textit{flags} can be defined via |\def| commands.
This command line makes the \TeX{} engine believe
it is compiling the file \textit{target}
whose content is specified as the latter parameter.
The provided code then forwards the processing to
\textit{main} or \textit{dest} as described in \secref{sec:forward}.

%%%%%%%%%%%%%%%%%%%%%%%%%%%%%%%%%%%%%%%%%%%%%%%%%%%%%%%%%%%%%%%%%%%%%%%%%%%%%%%%
\subsection{Include by Input}
\label{sec:input}

Including child documents by |\include| has some restrictions by design.
Most notably, the content of a child document always occupies
its own set of pages; pages cannot be shared between child documents.
Usually, this behaviour makes perfect sense
because each child document contain an essential part of the document.
However, in some situations it may be desirable to compose
a document from a collection of parts
without having mandatory page breaks between then.
For this case, the package
provides a mechanism to include parts
by |\input| which can also be processed individually.
However, by construction this mechanism
requires manual handling of the content to be output.

%%%%%%%%%%%%%%%%%%%%%%%%%%%%%%%%%%%%%%%%
\DescribeMacro{\ifchilddocmanual}
The main file should be prepared as usual, see \secref{sec:include}.
However, the document body must make a distinction
between processing of an individual part and of the main document, e.g.:
%
\begin{center}
\begin{tabular}{l}
|\ifchilddocmanual|\\
|\input{\childdocname}|\\
|\||else|\\
\textit{document body with }|\input{|\textit{part}|}|\\
|\||fi|
\end{tabular}
\end{center}
%
The conditional |\ifchilddocmanual| is true whenever
a part to be included by |\input| is being compiled,
and the name of the part is stored in |\childdocname|.

%%%%%%%%%%%%%%%%%%%%%%%%%%%%%%%%%%%%%%%%
\DescribeMacro{\childdocby}
Each part to be included by |\input| should start with:
%
\begin{center}
\begin{tabular}{l}
|\input{childdoc.def}|\\
|\childdocby{|\textit{main}|}|\\
\end{tabular}
\end{center}
%
The directive |\childdocby| is similar to |\childdocof|
described in \secref{sec:include},
but the subsequent selection of content must be done manually.
To that end, both |\ifchilddoc| and |\ifchilddocmanual|
will be true upon processing of a part,
and the name of the part is stored in |\childdocname|.
Note that |\jobname| will be set to the filename of the current part
so that each part receives an individual |.aux| file
that does not interfere with the |.aux| file(s) of the main document.
This behaviour can be altered by the alternative form
|\childdocby[*]{|\textit{main}|}| (with a non-empty optional argument)
which uses the |.aux| file of the main document
by setting |\jobname| to \textit{main}.

%%%%%%%%%%%%%%%%%%%%%%%%%%%%%%%%%%%%%%%%%%%%%%%%%%%%%%%%%%%%%%%%%%%%%%%%%%%%%%%%
\subsection{Driver Development}
\label{sec:driver}

The \textsf{childdoc} mechanism can also be use for the development
of definition files such as \LaTeX{} styles or classes.
This case differs from the above setup with multiple parts
included by |\include| in that no |\includeonly| should be invoked.
This can be achieved by starting the include file
(before |\ProvidesPackage|) with:
%
\begin{center}
\begin{tabular}{l}
|\input{childdoc.def}|\\
|\childdocforward{|\textit{main}|}|\\
\end{tabular}
\end{center}
%
or alternatively with:
%
\begin{center}
\begin{tabular}{l}
|\input{childdoc.def}|\\
|\childdocby{|\textit{main}|}|\\
\end{tabular}
\end{center}
%
Both forms have slightly different effects as described above.
The main file is prepared as usual, see \secref{sec:include}.

%%%%%%%%%%%%%%%%%%%%%%%%%%%%%%%%%%%%%%%%%%%%%%%%%%%%%%%%%%%%%%%%%%%%%%%%%%%%%%%%
\subsection{Legacy Detection}
\label{sec:detection}

The directive |\childdocmain| in the main file can detect
whether the complete document or merely a child is to be compiled
even without using the directive |\childdocof|.
This method is deprecated because it is less robust
and there is no compelling reason to use it;
it is merely provided for backward compatibility
and it may be removed in future versions.

If the detection mechanism is to be used,
it is mandatory to correctly specify
the filename of the main file as the argument of |\childdocmain|:
%
\begin{center}
\begin{tabular}{l}
|\input{childdoc.def}|\\
|\childdocmain{|\textit{main}|}|\\
\end{tabular}
\end{center}
%
If |\jobname| does not match the argument \textit{main} of |\childdocmain|,
it is assumed that |\jobname| points to the child file to be compiled.
When using |\childdocmain| with the main file specified as argument,
it suffices to start a child file
with just |\input{|\textit{main}|}|
without loading of the package and using |\childdocof|.
If instead all processing is done
with the appropriate \textsf{childdoc} directives,
the argument of \textit{main} of |\childdocmain| can be empty.

An alternative version of the command line processing described
in \secref{sec:commandline} using the detection mechanism reads:
%
\begin{center}
|... -jobname "|\textit{target}|" "|[\textit{flags}]%
[|\def\jobname{|\textit{dest}|}|]|\input{|\textit{main}|}"|
\end{center}

%%%%%%%%%%%%%%%%%%%%%%%%%%%%%%%%%%%%%%%%%%%%%%%%%%%%%%%%%%%%%%%%%%%%%%%%%%%%%%%%
\subsection{Manual Code}
\label{sec:manual}

In case one cannot be certain whether the definitions file |childdoc.def|
is installed on the target \TeX{} distribution
and one prefers not to ship it,
it is conceivable to paste a few relevant commands into the sources.

To that end, drop all statements |\input{childdoc.def}|
and perform the replacements as outlined below.
Instead of |\childdocmain{|\textit{main}|}| add the following code
to the top of the main file:
%
\begin{center}
\begin{tabular}{l}
|\||ifdefined\childdocname\endinput\||fi\newif\ifchilddoc|\\
|\edef\childdocname{\scantokens\expandafter{\jobname\noexpand}}|\\
|\def\childdocmain{|\textit{main}|}\||ifx\childdocmain\childdocname\||else|\\
|\childdoctrue\includeonly{\childdocname}\let\jobname\childdocmain\||fi|\\
\end{tabular}
\end{center}
%
Instead of |\childdocof{|\textit{main}|}| just include the main file
at the top of each child file:
%
\begin{center}
|\input{|\textit{main}|}|
\end{center}
%
A simple redirection |\childdocforward{|\textit{dest}|}| is achieved by:
%
\begin{center}
|\def\jobname{|\textit{dest}|}\input{\jobname}|
\end{center}
%
The redirection with prefix
|\childdocforwardprefix[|\textit{prefix}|]{|\textit{dest}|}|
is accomplished by:
%
\begin{center}
\begin{tabular}{l}
|{\edef\jobname{\scantokens\expandafter{\jobname\noexpand}}|\\
|\def\redirectjob |\textit{prefix}|#1~~~{\gdef\jobname{|\textit{dest}|#1}}|\\
|\expandafter\redirectjob\jobname~~~}\input{\jobname}|
\end{tabular}
\end{center}

In an alternative approach,
child documents can be compiled by a specific command line
without additional code or specific definitions:
%
\begin{center}
|... -jobname "|\textit{target}|" "|[\textit{flags}]%
|\includeonly{|\textit{dest}|}\input{|\textit{main}|}"|
\end{center}
%

%%%%%%%%%%%%%%%%%%%%%%%%%%%%%%%%%%%%%%%%%%%%%%%%%%%%%%%%%%%%%%%%%%%%%%%%%%%%%%%%
%%%%%%%%%%%%%%%%%%%%%%%%%%%%%%%%%%%%%%%%%%%%%%%%%%%%%%%%%%%%%%%%%%%%%%%%%%%%%%%%
\section{Information}

%%%%%%%%%%%%%%%%%%%%%%%%%%%%%%%%%%%%%%%%%%%%%%%%%%%%%%%%%%%%%%%%%%%%%%%%%%%%%%%%
\subsection{Copyright}

Copyright \copyright{} 2017--2018 Niklas Beisert

This work may be distributed and/or modified under the
conditions of the \LaTeX{} Project Public License, either version 1.3
of this license or (at your option) any later version.
The latest version of this license is in
  \url{http://www.latex-project.org/lppl.txt}
and version 1.3 or later is part of all distributions of \LaTeX{}
version 2005/12/01 or later.

This work has the LPPL maintenance status `maintained'.

The Current Maintainer of this work is Niklas Beisert.

This work consists of the files |README.txt|, |childdoc.ins| and |childdoc.dtx|
as well as the derived files |childdoc.def|, |cdocsamp.tex|
with |cdocsch1.tex|, |cdocsch2.tex|, |cdocspt3.tex|, |cdocspt4.tex|,
|cdocsdrf.tex|, |cdocsfn1.tex|, |cdocsfn2.tex|
as well as |childdoc.pdf|.

%%%%%%%%%%%%%%%%%%%%%%%%%%%%%%%%%%%%%%%%%%%%%%%%%%%%%%%%%%%%%%%%%%%%%%%%%%%%%%%%
\subsection{Files and Installation}

The package consists of the files:
%
\begin{center}
\begin{tabular}{ll}
    |README.txt|   & readme file \\
    |childdoc.ins| & installation file \\
    |childdoc.dtx| & source file \\
    |childdoc.def| & definition file \\
    |cdocsamp.tex| & sample main file \\
    |cdocsch1.tex| & sample include file \\
    |cdocsch2.tex| & sample include file \\
    |cdocspt3.tex| & sample part file \\
    |cdocspt4.tex| & sample part file \\
    |cdocsdrf.tex| & sample redirection file \\
    |cdocsfn1.tex| & sample redirection file \\
    |cdocsfn2.tex| & sample redirection file \\
    |childdoc.pdf| & manual
\end{tabular}
\end{center}
%
The distribution consists of the files
|README.txt|, |childdoc.ins| and |childdoc.dtx|.
%
\begin{itemize}
\item
Run (pdf)\LaTeX{} on |childdoc.dtx|
to compile the manual |childdoc.pdf| (this file).
\item
Run \LaTeX{} on |childdoc.ins| to create the definitions file |childdoc.def|
and the sample |cdocsamp.tex| with include files
|cdocsch1.tex|, |cdocsch2.tex|, |cdocspt3.tex|, |cdocspt4.tex|,
|cdocsdrf.tex|, |cdocsfn1.tex|, |cdocsfn2.tex|.
Then copy the file |childdoc.def| to an appropriate directory of your \LaTeX{}
distribution, e.g.\ \textit{texmf-root}|/tex/latex/childdoc|.
\end{itemize}

%%%%%%%%%%%%%%%%%%%%%%%%%%%%%%%%%%%%%%%%%%%%%%%%%%%%%%%%%%%%%%%%%%%%%%%%%%%%%%%%
\subsection{Related CTAN Packages}

There are several other packages which offer a similar functionality:
%
\begin{itemize}
\item
The packages
\href{http://ctan.org/pkg/docmute}{\textsf{docmute}},
\href{http://ctan.org/pkg/includex}{\textsf{includex}} and
\href{http://ctan.org/pkg/standalone}{\textsf{standalone}}
provide commands to include only the document body of
a child file thus allowing both files to be compiled individually.
\item
The packages \href{http://ctan.org/pkg/subdocs}{\textsf{subdocs}}
and \href{http://ctan.org/pkg/subfiles}{\textsf{subfiles}}
provide structures in which the main and child documents can be
encapsulated and allowing them to be compiled individually.
The inclusion mechanism is different from the conventional |\include|.
\item
The package \href{http://ctan.org/pkg/combine}{\textsf{combine}}
is an elaborate solution to combine several documents into one.
\end{itemize}
%
See also the CTAN topic \href{http://ctan.org/topic/subdocs}{\textsf{subdocs}}
for further related packages.
The present package differs from the above solutions in that
a document structure constructed with the conventional |\include| mechanism
just needs two extra commands at the top of every file
such that all constituent files can be compiled individually.

%%%%%%%%%%%%%%%%%%%%%%%%%%%%%%%%%%%%%%%%%%%%%%%%%%%%%%%%%%%%%%%%%%%%%%%%%%%%%%%%
%\subsection{Feature Suggestions}
%
%The following is a list of features which may be useful for future
%versions of this package:
%%
%\begin{itemize}
%\item
%\ldots
%\end{itemize}

%%%%%%%%%%%%%%%%%%%%%%%%%%%%%%%%%%%%%%%%%%%%%%%%%%%%%%%%%%%%%%%%%%%%%%%%%%%%%%%%
\subsection{Revision History}

%%%%%%%%%%%%%%%%%%%%%%%%%%%%%%%%%%%%%%%%
\paragraph{v2.0:} 2018/12/30

\begin{itemize}
\item
immediate forward processing
\item
added |\childdocby| mechanism
\item
manual restructured
\end{itemize}

%%%%%%%%%%%%%%%%%%%%%%%%%%%%%%%%%%%%%%%%
\paragraph{v1.6:} 2018/01/17

\begin{itemize}
\item
application for development of include files
\item
corrections to manual
\end{itemize}

%%%%%%%%%%%%%%%%%%%%%%%%%%%%%%%%%%%%%%%%
\paragraph{v1.5:} 2017/05/21

\begin{itemize}
\item
more complete structuring introduced
\item
|\childdocof| introduced
\item
|\childdoc| renamed to |\childdocmain|
\item
|\childredirect| renamed to |\childdocforward| and |\childdocforwardprefix|
and functionality expanded
\end{itemize}

%%%%%%%%%%%%%%%%%%%%%%%%%%%%%%%%%%%%%%%%
\paragraph{v1.0:} 2017/04/27

\begin{itemize}
\item
manual and install package
\item
first version published on CTAN
\end{itemize}

%%%%%%%%%%%%%%%%%%%%%%%%%%%%%%%%%%%%%%%%
\paragraph{v0.6:} 2017/04/26

\begin{itemize}
\item
redirection mechanism added
\end{itemize}

%%%%%%%%%%%%%%%%%%%%%%%%%%%%%%%%%%%%%%%%
\paragraph{v0.5:} 2017/04/26

\begin{itemize}
\item
functionality in definition file
\end{itemize}


%%%%%%%%%%%%%%%%%%%%%%%%%%%%%%%%%%%%%%%%%%%%%%%%%%%%%%%%%%%%%%%%%%%%%%%%%%%%%%%%
%%%%%%%%%%%%%%%%%%%%%%%%%%%%%%%%%%%%%%%%%%%%%%%%%%%%%%%%%%%%%%%%%%%%%%%%%%%%%%%%
%%%%%%%%%%%%%%%%%%%%%%%%%%%%%%%%%%%%%%%%%%%%%%%%%%%%%%%%%%%%%%%%%%%%%%%%%%%%%%%%
\appendix

\settowidth\MacroIndent{\rmfamily\scriptsize 000\ }

 \DocInput{childdoc.dtx}

\end{document}
%</driver>
% \fi
%
% %%%%%%%%%%%%%%%%%%%%%%%%%%%%%%%%%%%%%%%%%%%%%%%%%%%%%%%%%%%%%%%%%%%%%%%%%%%%%%
% %%%%%%%%%%%%%%%%%%%%%%%%%%%%%%%%%%%%%%%%%%%%%%%%%%%%%%%%%%%%%%%%%%%%%%%%%%%%%%
% \section{Sample}
%\iffalse
%<*samplemain>
%\fi
%
% The following presents a sample document
% with two chapters, two parts, a title page,
% a compile flag as well as three forwarding files to set the flag.
% It consists of eight |.tex| files:
% \begin{center}
% \begin{tabular}{ll}
% |cdocsamp.tex|&main file\\
% |cdocsch1.tex|&include file for chapter 1\\
% |cdocsch2.tex|&include file for chapter 2\\
% |cdocspt3.tex|&include file for part 3\\
% |cdocspt4.tex|&include file for part 4\\
% |cdocsdrf.tex|&forwarding file for main file in draft mode\\
% |cdocsfi1.tex|&forwarding file for final version of chapter 1\\
% |cdocsfi2.tex|&forwarding file for final version of chapter 2\\
% \end{tabular}
% \end{center}
% Each of the eight files can be compiled directly by the \LaTeX{} compiler.
%
% %%%%%%%%%%%%%%%%%%%%%%%%%%%%%%%%%%%%%%
% \paragraph{Main File.}
%
% The main file is called |cdocsamp.tex|.
%
% Load the \textsf{childdoc} definitions and
% declare the filename for the main document:
%    \begin{macrocode}
\input{childdoc.def}
\childdocmain{}
%    \end{macrocode}

% Optional override for |\version| flag:
%    \begin{macrocode}
%%\ifchilddoc\else\providecommand{\version}{draft}\fi
%    \end{macrocode}

% Define the default values for the |\version| flag
% (|final| for the main file and |draft| for childs):
%    \begin{macrocode}
\ifchilddoc
\providecommand{\version}{draft}
\else
\providecommand{\version}{final}
\fi
%    \end{macrocode}

% Load the standard document class:
%    \begin{macrocode}
\documentclass[12pt]{article}
%    \end{macrocode}

% Start the document body:
%    \begin{macrocode}
\begin{document}
%    \end{macrocode}

% Declare a title page.
% Print title, part of document being processed and version flag:
%    \begin{macrocode}
\addtocounter{page}{-1}
\begin{center}
{\LARGE\bfseries{}childdoc example\par}
\vspace{1cm}
\ifchilddoc
\ifchilddocmanual part\else chapter\fi:
`\childdocname' of `\childdocjob'\par
\else
main document: `\childdocjob'\par
\fi
version: \version\par
\end{center}
\newpage
%    \end{macrocode}

% Manually include selected file,
% otherwise process as usual:
%    \begin{macrocode}
\ifchilddocmanual
\section*{part `\childdocname'}
\input{\childdocname}
\else
%    \end{macrocode}

% Include the two chapters:
%    \begin{macrocode}
\include{cdocsch1}
\include{cdocsch2}
%    \end{macrocode}

% Include the two parts unless only chapters should be displayed:
%    \begin{macrocode}
\ifchilddoc\else
\section{part three}
\input{cdocspt3}
\section{part four}
\input{cdocspt4}
\fi
%    \end{macrocode}

% Process as usual until here:
%    \begin{macrocode}
\fi
%    \end{macrocode}

% End of document body:
%    \begin{macrocode}
\end{document}
%    \end{macrocode}
%\iffalse
%</samplemain>
%\fi
%
% %%%%%%%%%%%%%%%%%%%%%%%%%%%%%%%%%%%%%%
% \paragraph{Chapter Include Files.}
%
% The include files are called |cdocsch1.tex| and |cdocsch2.tex|.
%
%\iffalse
%<*samplechap1|samplechap2>
%\fi

% Optional override for |\version| flag:
%    \begin{macrocode}
%%\providecommand{\version}{final}
%    \end{macrocode}

% Include the main document:
%    \begin{macrocode}
\input{childdoc.def}
\childdocof{cdocsamp}
%    \end{macrocode}

%\iffalse
%</samplechap1|samplechap2>
%\fi
%
%\iffalse
%<*samplechap1>
%\fi
% Some text for chapter 1:
%    \begin{macrocode}
\section{one}
some text in chapter one
%    \end{macrocode}

%\iffalse
%</samplechap1>
%\fi
% Some text for chapter 2:
%\iffalse
%<*samplechap2>
%\fi
%    \begin{macrocode}
\section{two}
more text in chapter two
%    \end{macrocode}

%\iffalse
%</samplechap2>
%\fi
%
% %%%%%%%%%%%%%%%%%%%%%%%%%%%%%%%%%%%%%%
% \paragraph{Part Include Files.}
%
% The include files are called |cdocspt3.tex| and |cdocspt4.tex|.
%
%\iffalse
%<*samplepart3|samplepart4>
%\fi

% Optional override for |\version| flag:
%    \begin{macrocode}
%%\providecommand{\version}{final}
%    \end{macrocode}

% Include the main document:
%    \begin{macrocode}
\input{childdoc.def}
\childdocby{cdocsamp}
%    \end{macrocode}

%\iffalse
%</samplepart3|samplepart4>
%\fi
%
%\iffalse
%<*samplepart3>
%\fi
% Some text for part 3:
%    \begin{macrocode}
some text in part three
%    \end{macrocode}

%\iffalse
%</samplepart3>
%\fi
% Some text for part 4:
%\iffalse
%<*samplepart4>
%\fi
%    \begin{macrocode}
more text in part four
%    \end{macrocode}

%\iffalse
%</samplepart4>
%\fi
%
% %%%%%%%%%%%%%%%%%%%%%%%%%%%%%%%%%%%%%%
% \paragraph{Forwarding for a Complete Draft.}
%
% The following forwarding file |cdocsdrf.tex|
% compiles the main document in draft mode:
%\iffalse
%<*sampledraft>
%\fi
%    \begin{macrocode}
\def\version{draft}
\input{childdoc.def}
\childdocforward{cdocsamp}
%    \end{macrocode}

%\iffalse
%</sampledraft>
%\fi
%
% %%%%%%%%%%%%%%%%%%%%%%%%%%%%%%%%%%%%%%
% \paragraph{Forwarding for Final Version of the Chapters.}
%
% The following forwarding files |cdocsfn1.tex| and |cdocsfn2.tex|
% (with identical content)
% compile the final versions of the child documents
% |cdocsch1.tex| and |cdocsch2.tex|, respectively:
%\iffalse
%<*samplefinal>
%\fi
%    \begin{macrocode}
\def\version{final}
\input{childdoc.def}
\childdocforwardprefix[cdocsamp]{cdocsfn}{cdocsch}
%    \end{macrocode}

%\iffalse
%</samplefinal>
%\fi
%
% %%%%%%%%%%%%%%%%%%%%%%%%%%%%%%%%%%%%%%
% \paragraph{Command Line Processing.}
%
% The following three command lines generate the output files
% |cdocscld|, |cdocscl1| and |cdocscl2|
% which should be identical to
% |cdocsdrf|, |cdocsch1| and |cdocsfn2|, respectively:
% \begin{center}
% \begin{tabular}{l}
% |latex -jobname cdocscld \|\\
% |  "\def\version{draft}\input{childdoc.def}\childdocforward{cdocsamp}"|\\
% |latex -jobname cdocscl1 \|\\
% |  "\input{childdoc.def}\childdocforward[cdocsamp]{cdocsch1}"|\\
% |latex -jobname cdocscl2 \|\\
% |  "\def\version{final}\input{childdoc.def}\childdocforward{cdocsch2}"|
% \end{tabular}
% \end{center}
% Note that the trailing backslash on each first line
% merely continues the input to the second line
% (for convenient cut ant paste).
% Furthermore, the command |latex| can be replaced by any
% of its alternative versions such as |pdflatex|.
%
% %%%%%%%%%%%%%%%%%%%%%%%%%%%%%%%%%%%%%%%%%%%%%%%%%%%%%%%%%%%%%%%%%%%%%%%%%%%%%%
% %%%%%%%%%%%%%%%%%%%%%%%%%%%%%%%%%%%%%%%%%%%%%%%%%%%%%%%%%%%%%%%%%%%%%%%%%%%%%%
% \section{Implementation}
%\iffalse
%<*package>
%\fi
%
% This section describes the definitions file |childdoc.def|.

% The definitions cannot be loaded using |\usepackage| or |\RequirePackage|
% which has a mechanism to prevent loading a style file more than once.
% When loading the definitions by means of |\input|
% multiple instances have to be prevented manually:
%\iffalse
%This code needs to be before the `\ProvidesFile' directive
%which is defined at the beginning of this file.
%Therefore it is also placed there and commented out here.
%</package>
%<*discard>
%\fi
%    \begin{macrocode}
\ifdefined\childdocmain\endinput\fi
%    \end{macrocode}
%\iffalse
%</discard>
%<*package>
%\fi
%
% \macro{\ifchilddoc}
% \macro{\ifchilddocmanual}
% The conditional |\ifchilddoc| tells whether a
% child (true) or main (false) document is being compiled.
% The conditional |\ifchilddocmanual| tells whether
% the |\includeonly| mechanism is used (false) or
% the selection of child files must be performed manually (true).
% The definitions initialise to false:
%    \begin{macrocode}
\newif\ifchilddoc
\newif\ifchilddocmanual
%    \end{macrocode}

% \macro{\childdocname}
% \macro{\childdocjob}
% The macro |\childdocname| stores the name of the main document
% to be compiled. The macro |\childdocjob| stores the name of
% the document on which the \LaTeX{} compiler was originally invoked.
% The content of |\jobname| cannot be compared
% to filenames specified in the source due to different catcodes.
% The following code rescans |\jobname|, stores the result
% in |\childdocname| and saves a copy in |\childdocjob|:
%    \begin{macrocode}
\edef\childdocname{\scantokens\expandafter{\jobname\noexpand}}
\let\childdocjob\childdocname
%    \end{macrocode}

% \macro{\childdocdisable}
% The macro |\childdocdisable| prevents the main file
% from being processed more than once.
% At this stage, the main document command |\childdocmain|
% is assumed to be called once again where it should do nothing.
% Any subsequent call to it should prevent
% a secondary processing of the main document
% It overwrites the forwarding commands
% |\childdocof| and |\childdocforward|
% with empty macros to prevent further inclusions of the main document:
%    \begin{macrocode}
\newcommand{\childdocdisable}
{
  \renewcommand{\childdocmain}[1]{\renewcommand{\childdocmain}[1]{\endinput}}
  \renewcommand{\childdocof}[1]{}
  \renewcommand{\childdocby}[2][]{}
  \renewcommand{\childdocforward}[2][]{}
  \renewcommand{\childdocdisable}{}
}
%    \end{macrocode}

% \macro{\childdocmain}
% The macro |\childdocmain| is to be called at the top of the main file
% with nothing or the main filename (without extension) as argument.
% First, it breaks loops.
% If the argument is not empty and does not match |\childdocname|
% (which is set by the first inclusion of |childdoc.def|),
% |\ifchilddoc| is set to true, |\includeonly| is applied to the child file
% and |\jobname| is set to the main file
% (for proper handling of |.aux| files):
%    \begin{macrocode}
\newcommand{\childdocmain}[1]
{
  \childdocdisable\childdocmain{}
  \if?#1?\else
    \begingroup
      \def\childdoctmp{#1}
      \ifx\childdoctmp\childdocname
        \def\childdoctmp{}
      \else
        \def\childdoctmp
        {
          \childdoctrue
          \includeonly{\childdocname}
          \def\childdocjob{#1}
          \def\jobname{#1}
        }
      \fi
      \expandafter
    \endgroup
    \childdoctmp
  \fi
}
%    \end{macrocode}

% \macro{\childdocof}
% The command |\childdocof| redirects
% compilation to the main file |#1|.
%    \begin{macrocode}
\newcommand{\childdocof}[1]
{
  \childdocdisable
  \childdoctrue
  \includeonly{\childdocname}
  \def\jobname{#1}
  \def\childdocjob{#1}
  \input{#1}
}
%    \end{macrocode}

% \macro{\childdocby}
% The command |\childdocby| ....
%    \begin{macrocode}
\newcommand{\childdocby}[2][]
{
  \childdocdisable
  \childdoctrue
  \childdocmanualtrue
  \if?#1?\else
    \def\jobname{#2}
  \fi
  \def\childdocjob{#2}
  \input{#2}
  \endinput
}
%    \end{macrocode}

% \macro{\childdocforward}
% The command |\childdocforward| redirects
% compilation to the main file or
% (if the optional argument is given) a child file.
% Parameters are set as if the main file
% or a child file starting with |\childdocof| was compiled.
% Then compilation is handed over to the main file:
%    \begin{macrocode}
\newcommand{\childdocforward}[2][]
{
  \begingroup
    \if?#1?
      \def\childdoctmp
      {
        \def\childdocname{#2}
        \def\childdocjob{#2}
        \def\jobname{#2}
        \input{#2}
        \endinput
      }
    \else
      \def\childdoctmp
      {
        \childdocdisable
        \def\childdocname{#2}
        \childdoctrue
        \includeonly{#2}
        \def\childdocjob{#1}
        \def\jobname{#1}
        \input{#1}
        \endinput
      }
    \fi
    \expandafter
  \endgroup
  \childdoctmp
}
%    \end{macrocode}

% \macro{\childdocforwardprefix}
% The command |\childdocforwardprefix| redirects
% compilation to the main or a child file by means of a pattern.
% The prefix |#1| in the current filename is replaced by |#2|
% and the suffix of the current filename is kept
% (it is assumed that the filename does not contain the substring `|~~~|'
% which is used as a delimiter).
% Compilation is handed over to the new file by |\childdocforward|:
%    \begin{macrocode}
\newcommand{\childdocforwardprefix}[3][]
{
  \begingroup
    \def\childdocextract #2##1~~~{\def\childdoctmp{\childdocforward[#1]{#3##1}}}
    \expandafter\childdocextract\childdocname~~~
    \expandafter
  \endgroup
  \childdoctmp
}
%    \end{macrocode}

% \macro{\childdoc}
% The deprecated macro |\childdoc| is a legacy version of |\childdocmain|:
%    \begin{macrocode}
\newcommand{\childdoc}{\childdocmain}
%    \end{macrocode}

% \macro{\childdocredirect}
% The deprecated macro |\childdocredirect| is a legacy version
% of |\childdocforward| and |\childdocforwardprefix|:
%    \begin{macrocode}
\newcommand{\childdocredirect}[2][]
{
  \begingroup
    \if?#1?
      \def\childdoctmp{\childdocforward{#2}}
    \else
      \def\childdoctmp{\childdocforwardprefix{#1}{#2}}
    \fi
    \expandafter
  \endgroup
  \childdoctmp
}
%    \end{macrocode}

%\iffalse
%</package>
%\fi
%
\endinput

\childdocmain{}
%    \end{macrocode}

% Optional override for |\version| flag:
%    \begin{macrocode}
%%\ifchilddoc\else\providecommand{\version}{draft}\fi
%    \end{macrocode}

% Define the default values for the |\version| flag
% (|final| for the main file and |draft| for childs):
%    \begin{macrocode}
\ifchilddoc
\providecommand{\version}{draft}
\else
\providecommand{\version}{final}
\fi
%    \end{macrocode}

% Load the standard document class:
%    \begin{macrocode}
\documentclass[12pt]{article}
%    \end{macrocode}

% Start the document body:
%    \begin{macrocode}
\begin{document}
%    \end{macrocode}

% Declare a title page.
% Print title, part of document being processed and version flag:
%    \begin{macrocode}
\addtocounter{page}{-1}
\begin{center}
{\LARGE\bfseries{}childdoc example\par}
\vspace{1cm}
\ifchilddoc
\ifchilddocmanual part\else chapter\fi:
`\childdocname' of `\childdocjob'\par
\else
main document: `\childdocjob'\par
\fi
version: \version\par
\end{center}
\newpage
%    \end{macrocode}

% Manually include selected file,
% otherwise process as usual:
%    \begin{macrocode}
\ifchilddocmanual
\section*{part `\childdocname'}
\input{\childdocname}
\else
%    \end{macrocode}

% Include the two chapters:
%    \begin{macrocode}
\include{cdocsch1}
\include{cdocsch2}
%    \end{macrocode}

% Include the two parts unless only chapters should be displayed:
%    \begin{macrocode}
\ifchilddoc\else
\section{part three}
\input{cdocspt3}
\section{part four}
\input{cdocspt4}
\fi
%    \end{macrocode}

% Process as usual until here:
%    \begin{macrocode}
\fi
%    \end{macrocode}

% End of document body:
%    \begin{macrocode}
\end{document}
%    \end{macrocode}
%\iffalse
%</samplemain>
%\fi
%
% %%%%%%%%%%%%%%%%%%%%%%%%%%%%%%%%%%%%%%
% \paragraph{Chapter Include Files.}
%
% The include files are called |cdocsch1.tex| and |cdocsch2.tex|.
%
%\iffalse
%<*samplechap1|samplechap2>
%\fi

% Optional override for |\version| flag:
%    \begin{macrocode}
%%\providecommand{\version}{final}
%    \end{macrocode}

% Include the main document:
%    \begin{macrocode}
% \iffalse
%
% childdoc.dtx Copyright (C) 2017-2018 Niklas Beisert
%
% This work may be distributed and/or modified under the
% conditions of the LaTeX Project Public License, either version 1.3
% of this license or (at your option) any later version.
% The latest version of this license is in
%   http://www.latex-project.org/lppl.txt
% and version 1.3 or later is part of all distributions of LaTeX
% version 2005/12/01 or later.
%
% This work has the LPPL maintenance status `maintained'.
%
% The Current Maintainer of this work is Niklas Beisert.
%
% This work consists of the files childdoc.dtx and childdoc.ins
% and the derived files childdoc.def and cdocsamp.tex with
% cdocsch1.tex, cdocsch2.tex, cdocsdrf.tex, cdocsfn1.tex, cdocsfn2.tex.
%
%<package>\ifdefined\childdocmain\endinput\fi
%<package>\ProvidesFile{childdoc.def}[2018/12/30 v2.0 child document driver]
%<samplemain>\ProvidesFile{cdocsamp.tex}[2018/12/30 v2.0 sample for childdoc]
%<*driver>
%\ProvidesFile{childdoc.drv}[2018/12/30 v2.0 childdoc reference manual file]
\PassOptionsToClass{10pt,a4paper}{article}
\documentclass{ltxdoc}

\usepackage[margin=35mm]{geometry}
\usepackage{hyperref}
\usepackage{hyperxmp}
\usepackage[usenames]{color}

\hypersetup{colorlinks=true}
\hypersetup{pdfstartview=FitH}
\hypersetup{pdfpagemode=UseNone}
\hypersetup{pdfsource={}}
\hypersetup{pdflang={en-UK}}
\hypersetup{pdfcopyright={Copyright 2017-2018 Niklas Beisert.
  This work may be distributed and/or modified under the
  conditions of the LaTeX Project Public License, either version 1.3
  of this license or (at your option) any later version.}}
\hypersetup{pdflicenseurl={http://www.latex-project.org/lppl.txt}}
\hypersetup{pdfcontactaddress={ETH Zurich, ITP, HIT K,
  Wolfgang-Pauli-Strasse 27}}
\hypersetup{pdfcontactpostcode={8093}}
\hypersetup{pdfcontactcity={Zurich}}
\hypersetup{pdfcontactcountry={Switzerland}}
\hypersetup{pdfcontactemail={nbeisert@itp.phys.ethz.ch}}
\hypersetup{pdfcontacturl={http://people.phys.ethz.ch/\xmptilde nbeisert/}}

\newcommand{\secref}[1]{\hyperref[#1]{section \ref*{#1}}}

\parskip1ex
\parindent0pt
\let\olditemize\itemize
\def\itemize{\olditemize\parskip0pt}

\begin{document}

\title{The \textsf{childdoc} Package}
\hypersetup{pdftitle={The childdoc Package}}
\author{Niklas Beisert\\[2ex]
  Institut f\"ur Theoretische Physik\\
  Eidgen\"ossische Technische Hochschule Z\"urich\\
  Wolfgang-Pauli-Strasse 27, 8093 Z\"urich, Switzerland\\[1ex]
  \href{mailto:nbeisert@itp.phys.ethz.ch}
  {\texttt{nbeisert@itp.phys.ethz.ch}}}
\hypersetup{pdfauthor={Niklas Beisert}}
\hypersetup{pdfsubject={Manual for the LaTeX2e Package childdoc}}
\date{30 December 2018, \textsf{v2.0}}
\maketitle

\begin{abstract}\noindent
\textsf{childdoc} is a \LaTeXe{} package
that enables the direct compilation
of document sections included by |\include|
to individual files.
\end{abstract}

\begingroup
\parskip0ex
\tableofcontents
\endgroup

%%%%%%%%%%%%%%%%%%%%%%%%%%%%%%%%%%%%%%%%%%%%%%%%%%%%%%%%%%%%%%%%%%%%%%%%%%%%%%%%
%%%%%%%%%%%%%%%%%%%%%%%%%%%%%%%%%%%%%%%%%%%%%%%%%%%%%%%%%%%%%%%%%%%%%%%%%%%%%%%%
\section{Introduction}

\LaTeX{} provides a mechanism to structure a large document (such as a book)
into a main file and several child files (containing the chapters)
using the |\include| command.
This mechanism is beneficial for documents
which span hundreds of pages in order to
make the source file(s) more manageable.
Moreover, compilation can be restricted to
selected child files by means of the |\includeonly| command.
The latter feature can be used to reduce the compilation time while editing
(this was significantly more useful in the earlier days of \LaTeX{})
or to generate a smaller document which is easier to navigate.
Another application of |\includeonly| is to generate
documents consisting of selected parts of the complete document.

However, there are a few drawbacks of the plain |\include| mechanism:
\begin{itemize}
\item
The child files cannot be compiled on their own,
they can only be compiled via the main file.
A naive editing environment
(such as a text editor with an option
to have the current file processed by \LaTeX)
may require one to switch to the main file before compiling;
attempting to compile the child file produces errors.
\item
The main file must be modified (each time)
to adjust the |\includeonly| command
to the present needs. This easily leaves the main file in a messy state.
\item
The generated document will always carry the filename
of the main document. This is inconvenient if
several child files are to be compiled and
to be kept for distribution.
\end{itemize}

The present package provides a simple interface
to make child files individually compilable by \LaTeX{}.
Compiling a child file then has the same effect as compiling
the main file with an |\includeonly| command
to select the appropriate child.
Moreover the generated document will carry the name of the child
rather than the main file.
This resolves all three above issues.

This feature is meant to make the editing of books,
thesis documents and lecture notes somewhat more convenient.
However, the package can also be used efficiently for
composing a series of documents (such as exercise sheets)
which are typically distributed individually.
It then assists the author in generating the individual documents
(potentially in different versions)
as well as a document containing the collected series.
Another application is in developing style files
or other kinds of included material
where compilation of the style file could redirect
to a sample or test file.

%%%%%%%%%%%%%%%%%%%%%%%%%%%%%%%%%%%%%%%%%%%%%%%%%%%%%%%%%%%%%%%%%%%%%%%%%%%%%%%%
%%%%%%%%%%%%%%%%%%%%%%%%%%%%%%%%%%%%%%%%%%%%%%%%%%%%%%%%%%%%%%%%%%%%%%%%%%%%%%%%
\section{Usage}

First of all, the package \textsf{childdoc} is \emph{not} a standard
\LaTeXe{} |.sty| style file! Therefore it needs to be invoked in
a non-standard way.

%%%%%%%%%%%%%%%%%%%%%%%%%%%%%%%%%%%%%%%%%%%%%%%%%%%%%%%%%%%%%%%%%%%%%%%%%%%%%%%%
\subsection{Included Files}
\label{sec:include}

%%%%%%%%%%%%%%%%%%%%%%%%%%%%%%%%%%%%%%%%
\DescribeMacro{\childdocmain}
To use the package, add the commands
\begin{center}
\begin{tabular}{l}
|\input{childdoc.def}|\\
|\childdocmain{}|\\
\end{tabular}
\end{center}
at the very top of the main \LaTeX{} file,
in particular \emph{before} the |\documentclass| statement!
The argument of |\childdocmain| should be left empty
(but it must be present).

%%%%%%%%%%%%%%%%%%%%%%%%%%%%%%%%%%%%%%%%
\DescribeMacro{\childdocof}
Furthermore, add the commands
\begin{center}
\begin{tabular}{l}
|\input{childdoc.def}|\\
|\childdocof{|\textit{main}|}|\\
\end{tabular}
\end{center}
at the top of every child file \textit{child}
which is included by |\include{|\textit{child}|}|
from within the main file
(or at least for those files to be compiled individually).
The argument \textit{main} must be the filename of the main file.

There are a couple of
considerations in setting up the main and child documents:

%%%%%%%%%%%%%%%%%%%%%%%%%%%%%%%%%%%%%%%%
\paragraph{Restrictions.}

Please note the following restrictions:
\begin{itemize}
\item
|\childdocmain| must be called with one argument \textit{main}
to ensure compatibility with earlier version of the package.
It must either be empty (|\childdocmain{}|)
or precisely match the filename of the main file in which it is specified.
See \secref{sec:detection} for further information.
\item
The filename \textit{main} must be specified without the |.tex| extension.
\item
The filename \textit{main} is case sensitive
(even in case-insensitive file systems)
due to internal string comparison.
\item
The argument \textit{main} should be fully expanded, it cannot be a macro.
\item
Subdirectories and special characters should be avoided in filenames.
\item
The command |\childdocmain{|\textit{main}|}| must be followed by a whitespace.
It should not be followed immediately by another command
or by a comment mark `|%|'.
This is because the \TeX{} parser reads the token immediately following
the argument of |\childdocmain| and puts it
at the beginning of every child section;
however, a white\-space is ignored.
\end{itemize}

%%%%%%%%%%%%%%%%%%%%%%%%%%%%%%%%%%%%%%%%
\paragraph{Content of Main File.}

It is advisable to place all content in the child files included by |\include|.
Any output contained in the main file will appear in all child documents
unless suppressed manually;
it cannot be suppressed automatically by the |\includeonly| directive
and thus should normally be avoided.
A method to include some content in the main file
by means of conditional processing is described in \secref{sec:conditional}.

%%%%%%%%%%%%%%%%%%%%%%%%%%%%%%%%%%%%%%%%
\paragraph{Page Numbering.}

When only a part of the document is compiled,
the appropriate numbering of pages
(as well as other status parameters)
is determined from the |.aux| files.
The latter contain information from previous passes.
However this information needs to propagate through
all intermediate child documents.
Therefore the page numbering in child documents may well
be inconsistent until the complete document is compiled at least once.

A useful (if unconventional) way to always ensure a consistent
page numbering is to restart the numbering in each child document
and denote the pages by `\textit{child}|.|\textit{page}'
where \textit{child} represents the chapter/section number of the child file.
This can be achieved by the command
|\numberwithin{page}{|\textit{child}|}|
of the \textsf{amsmath} package
where \textit{child} can be |chapter| or |section|
depending on the chosen structuring.
Alternatively, one can modify the macro |\thepage| appropriately
and reset the counter |page| at the start of each child file.

%%%%%%%%%%%%%%%%%%%%%%%%%%%%%%%%%%%%%%%%%%%%%%%%%%%%%%%%%%%%%%%%%%%%%%%%%%%%%%%%
\subsection{Conditional Processing}
\label{sec:conditional}

The package provides a mechanism to compile different versions
of a document. To customise the versions further some conditional processing
can come in handy to distinguish which version is being compiled.
The package provides two macros to describe the compilation context:

%%%%%%%%%%%%%%%%%%%%%%%%%%%%%%%%%%%%%%%%
\DescribeMacro{\ifchilddoc}
The conditional |\ifchilddoc| distinguishes between the compilation of
child documents and the main document:
%
\begin{center}
|\ifchilddoc |\textit{child-code}| |[|\||else |\textit{main-code}]| \||fi|
\end{center}

%%%%%%%%%%%%%%%%%%%%%%%%%%%%%%%%%%%%%%%%
\DescribeMacro{\childdocname}
\DescribeMacro{\childdocjob}
The macro |\childdocname| contains the filename (without extension)
of the main or child file being processed.
Note that |\childdocjob| will always contain the name of the main file.

%%%%%%%%%%%%%%%%%%%%%%%%%%%%%%%%%%%%%%%%
\paragraph{Title Page.}

Conditional processing can be used to include a title or banner page
in the main document when proper precautions are taken.
Importantly, the code in the main file should ensure that the page counter
(as well as other status parameters which are stored in the |.aux| files)
takes the same value after the conditional processing.
Otherwise the page numbers may take divergent values
depending on which part is compiled.

For example, a title page could be declared by:
%
\begin{center}
\begin{tabular}{l}
|\ifchilddoc\||else|\\
|\addtocounter{page}{-1}|\\
\textit{code for title page}\\
|\newpage|\\
|\||fi|
\end{tabular}
\end{center}
%
A banner page for the child documents can be generated by:
%
\begin{center}
\begin{tabular}{l}
|\ifchilddoc|\\
|\addtocounter{page}{-1}|\\
\textit{code for banner page}\\
|\newpage|\\
|\||fi|
\end{tabular}
\end{center}
%
Here one could write a message such as:
\begin{center}
|This is the part \childdocname{} of \childdocjob{}.|
\end{center}

%%%%%%%%%%%%%%%%%%%%%%%%%%%%%%%%%%%%%%%%%%%%%%%%%%%%%%%%%%%%%%%%%%%%%%%%%%%%%%%%
\subsection{Flags}
\label{sec:flags}

The package makes it easy to generate different versions
of the main or child documents.
To this end compilation flags can be defined
and assigned different default values.
They will be particularly useful in conjunction
with the forwarding mechanism described in \secref{sec:forward}.

For example, it may be useful to have a flag |\version|
which can be set to |draft| or |final|.
The document source will contain some conditional code
depending on the value of |\version|.
Suppose further, the flag should default to |final| for the main file
and to |draft| for child files
which is a natural assignment for editing the document.
This is achieved by placing the following code
in the preamble of the main document
(below the |\childdocmain| directive):
%
\begin{center}
\begin{tabular}{l}
|\ifchilddoc|\\
|\providecommand{\version}{draft}|\\
|\||else|\\
|\providecommand{\version}{final}|\\
|\||fi|
\end{tabular}
\end{center}
%
The definition by |\providecommand| makes sure
that previous definitions are not overwritten.
Further statements |\providecommand{\version}{...}|
can thus be added before the above code to override it.

For the main file, one might add a line
(between |\childdocmain| and the above block)
%
\begin{center}
|%\ifchilddoc\||else\providecommand{\version}{draft}\||fi|
\end{center}
%
which can be uncommented to produce a draft version.
Likewise one can add a line to the very top of a child file
(above the |\childdocof{|\textit{main}|}| directive)
%
\begin{center}
|%\providecommand{\version}{final}|
\end{center}
%
which can be uncommented to produce the final version of this child document.

%%%%%%%%%%%%%%%%%%%%%%%%%%%%%%%%%%%%%%%%%%%%%%%%%%%%%%%%%%%%%%%%%%%%%%%%%%%%%%%%
\subsection{Forwarding}
\label{sec:forward}

Different versions of the main or child documents
using compilation flags as described in \secref{sec:flags}
can be (permanently) stored in different files
for convenient compilation, viewing and distribution.
To this end, the package defines a command
to pass on compilation to a different file:

%%%%%%%%%%%%%%%%%%%%%%%%%%%%%%%%%%%%%%%%
\DescribeMacro{\childdocforward}
The command |\childdocforward| redirects processing to
another source file:
%
\begin{center}
\begin{tabular}{l}
|\input{childdoc.def}|\\
|\childdocforward[|\textit{main}|]{|\textit{dest}|}|\\
\end{tabular}
\end{center}
%
The argument \textit{dest} is the destination file
(without extension).
It should be the main file or one of the child files.
Note that further \textsf{childdoc} directives
such as |\childdocof| and |\childdocforward|
in the indicated file will be processed in this form.
The optional argument \textit{main}
passes on directly to the main file \textit{main}
while pretending to compile the child \textit{dest}.
This form behaves as if \textit{dest}
issues |\childdocof{|\textit{main}|}| right away,
and no further \textsf{childdoc} directives will be processed.

%%%%%%%%%%%%%%%%%%%%%%%%%%%%%%%%%%%%%%%%
\DescribeMacro{\...prefix}
In the alternative form |\childdocforwardprefix|,
%
\begin{center}
\begin{tabular}{l}
|\input{childdoc.def}|\\
|\childdocforwardprefix[|\textit{main}|]{|\textit{prefix}|}{|\textit{dest}|}|
\end{tabular}
\end{center}
%
the destination file is determined by a pattern
depending on the current file:
To make this work, the current file must be called
`{\textit{prefix}\hspace{0.2em}\textit{suffix}}'
with \textit{prefix} matching precisely the argument.
Processing is then passed on to the file
`{\textit{dest}\hspace{0.2em}\textit{suffix}}'.
Surely, the same effect is achieved by
directly specifying the
argument `{\textit{dest}\hspace{0.2em}\textit{suffix}}'
in the first form.
However, that requires to set up a different file
for each child. With the alternative form of the command
all these files can have exactly the same content
which simplifies setting them up and maintaining them.

For example, the following file |draft.tex|
with a compilation flag |\version| as described in \secref{sec:flags}
compiles the main document as a draft:
%
\begin{center}
\begin{tabular}{l}
|\def\version{draft}|\\
|\input{childdoc.def}|\\
|\childdocforward{|\textit{main}|}|
\end{tabular}
\end{center}
%
Likewise, the following files |final|\textit{nn}|.tex|
compile the final version of the child document
|child|\textit{nn}|.tex|:
%
\begin{center}
\begin{tabular}{l}
|\def\version{final}|\\
|\input{childdoc.def}|\\
|\childdocforwardprefix{final}{child}|
\end{tabular}
\end{center}
%

Note that when several versions of a main file and/or of each child file
are to be generated, it may be convenient to set up a |Makefile| or
shell script to automatise the process.

%%%%%%%%%%%%%%%%%%%%%%%%%%%%%%%%%%%%%%%%%%%%%%%%%%%%%%%%%%%%%%%%%%%%%%%%%%%%%%%%
\subsection{Command Line Processing}
\label{sec:commandline}

The effect of redirection files can also be achieved by invoking
the \LaTeX{} compiler with a more elaborate command line.
Most conveniently this should be done as part
of a shell script or a |Makefile|.

When using \textsf{childdoc} in the main file, the following
command lines effectively perform a redirection
(note that depending on the shell being used,
backslashes may have to be doubled: `|\|' $\to$ `|\\|'):
%
\begin{center}
|... -jobname "|\textit{target}|" |\\|"|[\textit{flags}]%
|\input{childdoc.def}\childdocforward[|\textit{main}|]{|\textit{dest}|}"|
\end{center}
%
Here \textit{target} is the name of the output file,
\textit{main} is the name of the main file
and \textit{dest} is the name of the main or child file to be processed
(all filenames without extensions).
The optional argument \textit{main} can be omitted
if \textit{main} matches \textit{dest}.
Optionally, compilation \textit{flags} can be defined via |\def| commands.
This command line makes the \TeX{} engine believe
it is compiling the file \textit{target}
whose content is specified as the latter parameter.
The provided code then forwards the processing to
\textit{main} or \textit{dest} as described in \secref{sec:forward}.

%%%%%%%%%%%%%%%%%%%%%%%%%%%%%%%%%%%%%%%%%%%%%%%%%%%%%%%%%%%%%%%%%%%%%%%%%%%%%%%%
\subsection{Include by Input}
\label{sec:input}

Including child documents by |\include| has some restrictions by design.
Most notably, the content of a child document always occupies
its own set of pages; pages cannot be shared between child documents.
Usually, this behaviour makes perfect sense
because each child document contain an essential part of the document.
However, in some situations it may be desirable to compose
a document from a collection of parts
without having mandatory page breaks between then.
For this case, the package
provides a mechanism to include parts
by |\input| which can also be processed individually.
However, by construction this mechanism
requires manual handling of the content to be output.

%%%%%%%%%%%%%%%%%%%%%%%%%%%%%%%%%%%%%%%%
\DescribeMacro{\ifchilddocmanual}
The main file should be prepared as usual, see \secref{sec:include}.
However, the document body must make a distinction
between processing of an individual part and of the main document, e.g.:
%
\begin{center}
\begin{tabular}{l}
|\ifchilddocmanual|\\
|\input{\childdocname}|\\
|\||else|\\
\textit{document body with }|\input{|\textit{part}|}|\\
|\||fi|
\end{tabular}
\end{center}
%
The conditional |\ifchilddocmanual| is true whenever
a part to be included by |\input| is being compiled,
and the name of the part is stored in |\childdocname|.

%%%%%%%%%%%%%%%%%%%%%%%%%%%%%%%%%%%%%%%%
\DescribeMacro{\childdocby}
Each part to be included by |\input| should start with:
%
\begin{center}
\begin{tabular}{l}
|\input{childdoc.def}|\\
|\childdocby{|\textit{main}|}|\\
\end{tabular}
\end{center}
%
The directive |\childdocby| is similar to |\childdocof|
described in \secref{sec:include},
but the subsequent selection of content must be done manually.
To that end, both |\ifchilddoc| and |\ifchilddocmanual|
will be true upon processing of a part,
and the name of the part is stored in |\childdocname|.
Note that |\jobname| will be set to the filename of the current part
so that each part receives an individual |.aux| file
that does not interfere with the |.aux| file(s) of the main document.
This behaviour can be altered by the alternative form
|\childdocby[*]{|\textit{main}|}| (with a non-empty optional argument)
which uses the |.aux| file of the main document
by setting |\jobname| to \textit{main}.

%%%%%%%%%%%%%%%%%%%%%%%%%%%%%%%%%%%%%%%%%%%%%%%%%%%%%%%%%%%%%%%%%%%%%%%%%%%%%%%%
\subsection{Driver Development}
\label{sec:driver}

The \textsf{childdoc} mechanism can also be use for the development
of definition files such as \LaTeX{} styles or classes.
This case differs from the above setup with multiple parts
included by |\include| in that no |\includeonly| should be invoked.
This can be achieved by starting the include file
(before |\ProvidesPackage|) with:
%
\begin{center}
\begin{tabular}{l}
|\input{childdoc.def}|\\
|\childdocforward{|\textit{main}|}|\\
\end{tabular}
\end{center}
%
or alternatively with:
%
\begin{center}
\begin{tabular}{l}
|\input{childdoc.def}|\\
|\childdocby{|\textit{main}|}|\\
\end{tabular}
\end{center}
%
Both forms have slightly different effects as described above.
The main file is prepared as usual, see \secref{sec:include}.

%%%%%%%%%%%%%%%%%%%%%%%%%%%%%%%%%%%%%%%%%%%%%%%%%%%%%%%%%%%%%%%%%%%%%%%%%%%%%%%%
\subsection{Legacy Detection}
\label{sec:detection}

The directive |\childdocmain| in the main file can detect
whether the complete document or merely a child is to be compiled
even without using the directive |\childdocof|.
This method is deprecated because it is less robust
and there is no compelling reason to use it;
it is merely provided for backward compatibility
and it may be removed in future versions.

If the detection mechanism is to be used,
it is mandatory to correctly specify
the filename of the main file as the argument of |\childdocmain|:
%
\begin{center}
\begin{tabular}{l}
|\input{childdoc.def}|\\
|\childdocmain{|\textit{main}|}|\\
\end{tabular}
\end{center}
%
If |\jobname| does not match the argument \textit{main} of |\childdocmain|,
it is assumed that |\jobname| points to the child file to be compiled.
When using |\childdocmain| with the main file specified as argument,
it suffices to start a child file
with just |\input{|\textit{main}|}|
without loading of the package and using |\childdocof|.
If instead all processing is done
with the appropriate \textsf{childdoc} directives,
the argument of \textit{main} of |\childdocmain| can be empty.

An alternative version of the command line processing described
in \secref{sec:commandline} using the detection mechanism reads:
%
\begin{center}
|... -jobname "|\textit{target}|" "|[\textit{flags}]%
[|\def\jobname{|\textit{dest}|}|]|\input{|\textit{main}|}"|
\end{center}

%%%%%%%%%%%%%%%%%%%%%%%%%%%%%%%%%%%%%%%%%%%%%%%%%%%%%%%%%%%%%%%%%%%%%%%%%%%%%%%%
\subsection{Manual Code}
\label{sec:manual}

In case one cannot be certain whether the definitions file |childdoc.def|
is installed on the target \TeX{} distribution
and one prefers not to ship it,
it is conceivable to paste a few relevant commands into the sources.

To that end, drop all statements |\input{childdoc.def}|
and perform the replacements as outlined below.
Instead of |\childdocmain{|\textit{main}|}| add the following code
to the top of the main file:
%
\begin{center}
\begin{tabular}{l}
|\||ifdefined\childdocname\endinput\||fi\newif\ifchilddoc|\\
|\edef\childdocname{\scantokens\expandafter{\jobname\noexpand}}|\\
|\def\childdocmain{|\textit{main}|}\||ifx\childdocmain\childdocname\||else|\\
|\childdoctrue\includeonly{\childdocname}\let\jobname\childdocmain\||fi|\\
\end{tabular}
\end{center}
%
Instead of |\childdocof{|\textit{main}|}| just include the main file
at the top of each child file:
%
\begin{center}
|\input{|\textit{main}|}|
\end{center}
%
A simple redirection |\childdocforward{|\textit{dest}|}| is achieved by:
%
\begin{center}
|\def\jobname{|\textit{dest}|}\input{\jobname}|
\end{center}
%
The redirection with prefix
|\childdocforwardprefix[|\textit{prefix}|]{|\textit{dest}|}|
is accomplished by:
%
\begin{center}
\begin{tabular}{l}
|{\edef\jobname{\scantokens\expandafter{\jobname\noexpand}}|\\
|\def\redirectjob |\textit{prefix}|#1~~~{\gdef\jobname{|\textit{dest}|#1}}|\\
|\expandafter\redirectjob\jobname~~~}\input{\jobname}|
\end{tabular}
\end{center}

In an alternative approach,
child documents can be compiled by a specific command line
without additional code or specific definitions:
%
\begin{center}
|... -jobname "|\textit{target}|" "|[\textit{flags}]%
|\includeonly{|\textit{dest}|}\input{|\textit{main}|}"|
\end{center}
%

%%%%%%%%%%%%%%%%%%%%%%%%%%%%%%%%%%%%%%%%%%%%%%%%%%%%%%%%%%%%%%%%%%%%%%%%%%%%%%%%
%%%%%%%%%%%%%%%%%%%%%%%%%%%%%%%%%%%%%%%%%%%%%%%%%%%%%%%%%%%%%%%%%%%%%%%%%%%%%%%%
\section{Information}

%%%%%%%%%%%%%%%%%%%%%%%%%%%%%%%%%%%%%%%%%%%%%%%%%%%%%%%%%%%%%%%%%%%%%%%%%%%%%%%%
\subsection{Copyright}

Copyright \copyright{} 2017--2018 Niklas Beisert

This work may be distributed and/or modified under the
conditions of the \LaTeX{} Project Public License, either version 1.3
of this license or (at your option) any later version.
The latest version of this license is in
  \url{http://www.latex-project.org/lppl.txt}
and version 1.3 or later is part of all distributions of \LaTeX{}
version 2005/12/01 or later.

This work has the LPPL maintenance status `maintained'.

The Current Maintainer of this work is Niklas Beisert.

This work consists of the files |README.txt|, |childdoc.ins| and |childdoc.dtx|
as well as the derived files |childdoc.def|, |cdocsamp.tex|
with |cdocsch1.tex|, |cdocsch2.tex|, |cdocspt3.tex|, |cdocspt4.tex|,
|cdocsdrf.tex|, |cdocsfn1.tex|, |cdocsfn2.tex|
as well as |childdoc.pdf|.

%%%%%%%%%%%%%%%%%%%%%%%%%%%%%%%%%%%%%%%%%%%%%%%%%%%%%%%%%%%%%%%%%%%%%%%%%%%%%%%%
\subsection{Files and Installation}

The package consists of the files:
%
\begin{center}
\begin{tabular}{ll}
    |README.txt|   & readme file \\
    |childdoc.ins| & installation file \\
    |childdoc.dtx| & source file \\
    |childdoc.def| & definition file \\
    |cdocsamp.tex| & sample main file \\
    |cdocsch1.tex| & sample include file \\
    |cdocsch2.tex| & sample include file \\
    |cdocspt3.tex| & sample part file \\
    |cdocspt4.tex| & sample part file \\
    |cdocsdrf.tex| & sample redirection file \\
    |cdocsfn1.tex| & sample redirection file \\
    |cdocsfn2.tex| & sample redirection file \\
    |childdoc.pdf| & manual
\end{tabular}
\end{center}
%
The distribution consists of the files
|README.txt|, |childdoc.ins| and |childdoc.dtx|.
%
\begin{itemize}
\item
Run (pdf)\LaTeX{} on |childdoc.dtx|
to compile the manual |childdoc.pdf| (this file).
\item
Run \LaTeX{} on |childdoc.ins| to create the definitions file |childdoc.def|
and the sample |cdocsamp.tex| with include files
|cdocsch1.tex|, |cdocsch2.tex|, |cdocspt3.tex|, |cdocspt4.tex|,
|cdocsdrf.tex|, |cdocsfn1.tex|, |cdocsfn2.tex|.
Then copy the file |childdoc.def| to an appropriate directory of your \LaTeX{}
distribution, e.g.\ \textit{texmf-root}|/tex/latex/childdoc|.
\end{itemize}

%%%%%%%%%%%%%%%%%%%%%%%%%%%%%%%%%%%%%%%%%%%%%%%%%%%%%%%%%%%%%%%%%%%%%%%%%%%%%%%%
\subsection{Related CTAN Packages}

There are several other packages which offer a similar functionality:
%
\begin{itemize}
\item
The packages
\href{http://ctan.org/pkg/docmute}{\textsf{docmute}},
\href{http://ctan.org/pkg/includex}{\textsf{includex}} and
\href{http://ctan.org/pkg/standalone}{\textsf{standalone}}
provide commands to include only the document body of
a child file thus allowing both files to be compiled individually.
\item
The packages \href{http://ctan.org/pkg/subdocs}{\textsf{subdocs}}
and \href{http://ctan.org/pkg/subfiles}{\textsf{subfiles}}
provide structures in which the main and child documents can be
encapsulated and allowing them to be compiled individually.
The inclusion mechanism is different from the conventional |\include|.
\item
The package \href{http://ctan.org/pkg/combine}{\textsf{combine}}
is an elaborate solution to combine several documents into one.
\end{itemize}
%
See also the CTAN topic \href{http://ctan.org/topic/subdocs}{\textsf{subdocs}}
for further related packages.
The present package differs from the above solutions in that
a document structure constructed with the conventional |\include| mechanism
just needs two extra commands at the top of every file
such that all constituent files can be compiled individually.

%%%%%%%%%%%%%%%%%%%%%%%%%%%%%%%%%%%%%%%%%%%%%%%%%%%%%%%%%%%%%%%%%%%%%%%%%%%%%%%%
%\subsection{Feature Suggestions}
%
%The following is a list of features which may be useful for future
%versions of this package:
%%
%\begin{itemize}
%\item
%\ldots
%\end{itemize}

%%%%%%%%%%%%%%%%%%%%%%%%%%%%%%%%%%%%%%%%%%%%%%%%%%%%%%%%%%%%%%%%%%%%%%%%%%%%%%%%
\subsection{Revision History}

%%%%%%%%%%%%%%%%%%%%%%%%%%%%%%%%%%%%%%%%
\paragraph{v2.0:} 2018/12/30

\begin{itemize}
\item
immediate forward processing
\item
added |\childdocby| mechanism
\item
manual restructured
\end{itemize}

%%%%%%%%%%%%%%%%%%%%%%%%%%%%%%%%%%%%%%%%
\paragraph{v1.6:} 2018/01/17

\begin{itemize}
\item
application for development of include files
\item
corrections to manual
\end{itemize}

%%%%%%%%%%%%%%%%%%%%%%%%%%%%%%%%%%%%%%%%
\paragraph{v1.5:} 2017/05/21

\begin{itemize}
\item
more complete structuring introduced
\item
|\childdocof| introduced
\item
|\childdoc| renamed to |\childdocmain|
\item
|\childredirect| renamed to |\childdocforward| and |\childdocforwardprefix|
and functionality expanded
\end{itemize}

%%%%%%%%%%%%%%%%%%%%%%%%%%%%%%%%%%%%%%%%
\paragraph{v1.0:} 2017/04/27

\begin{itemize}
\item
manual and install package
\item
first version published on CTAN
\end{itemize}

%%%%%%%%%%%%%%%%%%%%%%%%%%%%%%%%%%%%%%%%
\paragraph{v0.6:} 2017/04/26

\begin{itemize}
\item
redirection mechanism added
\end{itemize}

%%%%%%%%%%%%%%%%%%%%%%%%%%%%%%%%%%%%%%%%
\paragraph{v0.5:} 2017/04/26

\begin{itemize}
\item
functionality in definition file
\end{itemize}


%%%%%%%%%%%%%%%%%%%%%%%%%%%%%%%%%%%%%%%%%%%%%%%%%%%%%%%%%%%%%%%%%%%%%%%%%%%%%%%%
%%%%%%%%%%%%%%%%%%%%%%%%%%%%%%%%%%%%%%%%%%%%%%%%%%%%%%%%%%%%%%%%%%%%%%%%%%%%%%%%
%%%%%%%%%%%%%%%%%%%%%%%%%%%%%%%%%%%%%%%%%%%%%%%%%%%%%%%%%%%%%%%%%%%%%%%%%%%%%%%%
\appendix

\settowidth\MacroIndent{\rmfamily\scriptsize 000\ }

 \DocInput{childdoc.dtx}

\end{document}
%</driver>
% \fi
%
% %%%%%%%%%%%%%%%%%%%%%%%%%%%%%%%%%%%%%%%%%%%%%%%%%%%%%%%%%%%%%%%%%%%%%%%%%%%%%%
% %%%%%%%%%%%%%%%%%%%%%%%%%%%%%%%%%%%%%%%%%%%%%%%%%%%%%%%%%%%%%%%%%%%%%%%%%%%%%%
% \section{Sample}
%\iffalse
%<*samplemain>
%\fi
%
% The following presents a sample document
% with two chapters, two parts, a title page,
% a compile flag as well as three forwarding files to set the flag.
% It consists of eight |.tex| files:
% \begin{center}
% \begin{tabular}{ll}
% |cdocsamp.tex|&main file\\
% |cdocsch1.tex|&include file for chapter 1\\
% |cdocsch2.tex|&include file for chapter 2\\
% |cdocspt3.tex|&include file for part 3\\
% |cdocspt4.tex|&include file for part 4\\
% |cdocsdrf.tex|&forwarding file for main file in draft mode\\
% |cdocsfi1.tex|&forwarding file for final version of chapter 1\\
% |cdocsfi2.tex|&forwarding file for final version of chapter 2\\
% \end{tabular}
% \end{center}
% Each of the eight files can be compiled directly by the \LaTeX{} compiler.
%
% %%%%%%%%%%%%%%%%%%%%%%%%%%%%%%%%%%%%%%
% \paragraph{Main File.}
%
% The main file is called |cdocsamp.tex|.
%
% Load the \textsf{childdoc} definitions and
% declare the filename for the main document:
%    \begin{macrocode}
\input{childdoc.def}
\childdocmain{}
%    \end{macrocode}

% Optional override for |\version| flag:
%    \begin{macrocode}
%%\ifchilddoc\else\providecommand{\version}{draft}\fi
%    \end{macrocode}

% Define the default values for the |\version| flag
% (|final| for the main file and |draft| for childs):
%    \begin{macrocode}
\ifchilddoc
\providecommand{\version}{draft}
\else
\providecommand{\version}{final}
\fi
%    \end{macrocode}

% Load the standard document class:
%    \begin{macrocode}
\documentclass[12pt]{article}
%    \end{macrocode}

% Start the document body:
%    \begin{macrocode}
\begin{document}
%    \end{macrocode}

% Declare a title page.
% Print title, part of document being processed and version flag:
%    \begin{macrocode}
\addtocounter{page}{-1}
\begin{center}
{\LARGE\bfseries{}childdoc example\par}
\vspace{1cm}
\ifchilddoc
\ifchilddocmanual part\else chapter\fi:
`\childdocname' of `\childdocjob'\par
\else
main document: `\childdocjob'\par
\fi
version: \version\par
\end{center}
\newpage
%    \end{macrocode}

% Manually include selected file,
% otherwise process as usual:
%    \begin{macrocode}
\ifchilddocmanual
\section*{part `\childdocname'}
\input{\childdocname}
\else
%    \end{macrocode}

% Include the two chapters:
%    \begin{macrocode}
\include{cdocsch1}
\include{cdocsch2}
%    \end{macrocode}

% Include the two parts unless only chapters should be displayed:
%    \begin{macrocode}
\ifchilddoc\else
\section{part three}
\input{cdocspt3}
\section{part four}
\input{cdocspt4}
\fi
%    \end{macrocode}

% Process as usual until here:
%    \begin{macrocode}
\fi
%    \end{macrocode}

% End of document body:
%    \begin{macrocode}
\end{document}
%    \end{macrocode}
%\iffalse
%</samplemain>
%\fi
%
% %%%%%%%%%%%%%%%%%%%%%%%%%%%%%%%%%%%%%%
% \paragraph{Chapter Include Files.}
%
% The include files are called |cdocsch1.tex| and |cdocsch2.tex|.
%
%\iffalse
%<*samplechap1|samplechap2>
%\fi

% Optional override for |\version| flag:
%    \begin{macrocode}
%%\providecommand{\version}{final}
%    \end{macrocode}

% Include the main document:
%    \begin{macrocode}
\input{childdoc.def}
\childdocof{cdocsamp}
%    \end{macrocode}

%\iffalse
%</samplechap1|samplechap2>
%\fi
%
%\iffalse
%<*samplechap1>
%\fi
% Some text for chapter 1:
%    \begin{macrocode}
\section{one}
some text in chapter one
%    \end{macrocode}

%\iffalse
%</samplechap1>
%\fi
% Some text for chapter 2:
%\iffalse
%<*samplechap2>
%\fi
%    \begin{macrocode}
\section{two}
more text in chapter two
%    \end{macrocode}

%\iffalse
%</samplechap2>
%\fi
%
% %%%%%%%%%%%%%%%%%%%%%%%%%%%%%%%%%%%%%%
% \paragraph{Part Include Files.}
%
% The include files are called |cdocspt3.tex| and |cdocspt4.tex|.
%
%\iffalse
%<*samplepart3|samplepart4>
%\fi

% Optional override for |\version| flag:
%    \begin{macrocode}
%%\providecommand{\version}{final}
%    \end{macrocode}

% Include the main document:
%    \begin{macrocode}
\input{childdoc.def}
\childdocby{cdocsamp}
%    \end{macrocode}

%\iffalse
%</samplepart3|samplepart4>
%\fi
%
%\iffalse
%<*samplepart3>
%\fi
% Some text for part 3:
%    \begin{macrocode}
some text in part three
%    \end{macrocode}

%\iffalse
%</samplepart3>
%\fi
% Some text for part 4:
%\iffalse
%<*samplepart4>
%\fi
%    \begin{macrocode}
more text in part four
%    \end{macrocode}

%\iffalse
%</samplepart4>
%\fi
%
% %%%%%%%%%%%%%%%%%%%%%%%%%%%%%%%%%%%%%%
% \paragraph{Forwarding for a Complete Draft.}
%
% The following forwarding file |cdocsdrf.tex|
% compiles the main document in draft mode:
%\iffalse
%<*sampledraft>
%\fi
%    \begin{macrocode}
\def\version{draft}
\input{childdoc.def}
\childdocforward{cdocsamp}
%    \end{macrocode}

%\iffalse
%</sampledraft>
%\fi
%
% %%%%%%%%%%%%%%%%%%%%%%%%%%%%%%%%%%%%%%
% \paragraph{Forwarding for Final Version of the Chapters.}
%
% The following forwarding files |cdocsfn1.tex| and |cdocsfn2.tex|
% (with identical content)
% compile the final versions of the child documents
% |cdocsch1.tex| and |cdocsch2.tex|, respectively:
%\iffalse
%<*samplefinal>
%\fi
%    \begin{macrocode}
\def\version{final}
\input{childdoc.def}
\childdocforwardprefix[cdocsamp]{cdocsfn}{cdocsch}
%    \end{macrocode}

%\iffalse
%</samplefinal>
%\fi
%
% %%%%%%%%%%%%%%%%%%%%%%%%%%%%%%%%%%%%%%
% \paragraph{Command Line Processing.}
%
% The following three command lines generate the output files
% |cdocscld|, |cdocscl1| and |cdocscl2|
% which should be identical to
% |cdocsdrf|, |cdocsch1| and |cdocsfn2|, respectively:
% \begin{center}
% \begin{tabular}{l}
% |latex -jobname cdocscld \|\\
% |  "\def\version{draft}\input{childdoc.def}\childdocforward{cdocsamp}"|\\
% |latex -jobname cdocscl1 \|\\
% |  "\input{childdoc.def}\childdocforward[cdocsamp]{cdocsch1}"|\\
% |latex -jobname cdocscl2 \|\\
% |  "\def\version{final}\input{childdoc.def}\childdocforward{cdocsch2}"|
% \end{tabular}
% \end{center}
% Note that the trailing backslash on each first line
% merely continues the input to the second line
% (for convenient cut ant paste).
% Furthermore, the command |latex| can be replaced by any
% of its alternative versions such as |pdflatex|.
%
% %%%%%%%%%%%%%%%%%%%%%%%%%%%%%%%%%%%%%%%%%%%%%%%%%%%%%%%%%%%%%%%%%%%%%%%%%%%%%%
% %%%%%%%%%%%%%%%%%%%%%%%%%%%%%%%%%%%%%%%%%%%%%%%%%%%%%%%%%%%%%%%%%%%%%%%%%%%%%%
% \section{Implementation}
%\iffalse
%<*package>
%\fi
%
% This section describes the definitions file |childdoc.def|.

% The definitions cannot be loaded using |\usepackage| or |\RequirePackage|
% which has a mechanism to prevent loading a style file more than once.
% When loading the definitions by means of |\input|
% multiple instances have to be prevented manually:
%\iffalse
%This code needs to be before the `\ProvidesFile' directive
%which is defined at the beginning of this file.
%Therefore it is also placed there and commented out here.
%</package>
%<*discard>
%\fi
%    \begin{macrocode}
\ifdefined\childdocmain\endinput\fi
%    \end{macrocode}
%\iffalse
%</discard>
%<*package>
%\fi
%
% \macro{\ifchilddoc}
% \macro{\ifchilddocmanual}
% The conditional |\ifchilddoc| tells whether a
% child (true) or main (false) document is being compiled.
% The conditional |\ifchilddocmanual| tells whether
% the |\includeonly| mechanism is used (false) or
% the selection of child files must be performed manually (true).
% The definitions initialise to false:
%    \begin{macrocode}
\newif\ifchilddoc
\newif\ifchilddocmanual
%    \end{macrocode}

% \macro{\childdocname}
% \macro{\childdocjob}
% The macro |\childdocname| stores the name of the main document
% to be compiled. The macro |\childdocjob| stores the name of
% the document on which the \LaTeX{} compiler was originally invoked.
% The content of |\jobname| cannot be compared
% to filenames specified in the source due to different catcodes.
% The following code rescans |\jobname|, stores the result
% in |\childdocname| and saves a copy in |\childdocjob|:
%    \begin{macrocode}
\edef\childdocname{\scantokens\expandafter{\jobname\noexpand}}
\let\childdocjob\childdocname
%    \end{macrocode}

% \macro{\childdocdisable}
% The macro |\childdocdisable| prevents the main file
% from being processed more than once.
% At this stage, the main document command |\childdocmain|
% is assumed to be called once again where it should do nothing.
% Any subsequent call to it should prevent
% a secondary processing of the main document
% It overwrites the forwarding commands
% |\childdocof| and |\childdocforward|
% with empty macros to prevent further inclusions of the main document:
%    \begin{macrocode}
\newcommand{\childdocdisable}
{
  \renewcommand{\childdocmain}[1]{\renewcommand{\childdocmain}[1]{\endinput}}
  \renewcommand{\childdocof}[1]{}
  \renewcommand{\childdocby}[2][]{}
  \renewcommand{\childdocforward}[2][]{}
  \renewcommand{\childdocdisable}{}
}
%    \end{macrocode}

% \macro{\childdocmain}
% The macro |\childdocmain| is to be called at the top of the main file
% with nothing or the main filename (without extension) as argument.
% First, it breaks loops.
% If the argument is not empty and does not match |\childdocname|
% (which is set by the first inclusion of |childdoc.def|),
% |\ifchilddoc| is set to true, |\includeonly| is applied to the child file
% and |\jobname| is set to the main file
% (for proper handling of |.aux| files):
%    \begin{macrocode}
\newcommand{\childdocmain}[1]
{
  \childdocdisable\childdocmain{}
  \if?#1?\else
    \begingroup
      \def\childdoctmp{#1}
      \ifx\childdoctmp\childdocname
        \def\childdoctmp{}
      \else
        \def\childdoctmp
        {
          \childdoctrue
          \includeonly{\childdocname}
          \def\childdocjob{#1}
          \def\jobname{#1}
        }
      \fi
      \expandafter
    \endgroup
    \childdoctmp
  \fi
}
%    \end{macrocode}

% \macro{\childdocof}
% The command |\childdocof| redirects
% compilation to the main file |#1|.
%    \begin{macrocode}
\newcommand{\childdocof}[1]
{
  \childdocdisable
  \childdoctrue
  \includeonly{\childdocname}
  \def\jobname{#1}
  \def\childdocjob{#1}
  \input{#1}
}
%    \end{macrocode}

% \macro{\childdocby}
% The command |\childdocby| ....
%    \begin{macrocode}
\newcommand{\childdocby}[2][]
{
  \childdocdisable
  \childdoctrue
  \childdocmanualtrue
  \if?#1?\else
    \def\jobname{#2}
  \fi
  \def\childdocjob{#2}
  \input{#2}
  \endinput
}
%    \end{macrocode}

% \macro{\childdocforward}
% The command |\childdocforward| redirects
% compilation to the main file or
% (if the optional argument is given) a child file.
% Parameters are set as if the main file
% or a child file starting with |\childdocof| was compiled.
% Then compilation is handed over to the main file:
%    \begin{macrocode}
\newcommand{\childdocforward}[2][]
{
  \begingroup
    \if?#1?
      \def\childdoctmp
      {
        \def\childdocname{#2}
        \def\childdocjob{#2}
        \def\jobname{#2}
        \input{#2}
        \endinput
      }
    \else
      \def\childdoctmp
      {
        \childdocdisable
        \def\childdocname{#2}
        \childdoctrue
        \includeonly{#2}
        \def\childdocjob{#1}
        \def\jobname{#1}
        \input{#1}
        \endinput
      }
    \fi
    \expandafter
  \endgroup
  \childdoctmp
}
%    \end{macrocode}

% \macro{\childdocforwardprefix}
% The command |\childdocforwardprefix| redirects
% compilation to the main or a child file by means of a pattern.
% The prefix |#1| in the current filename is replaced by |#2|
% and the suffix of the current filename is kept
% (it is assumed that the filename does not contain the substring `|~~~|'
% which is used as a delimiter).
% Compilation is handed over to the new file by |\childdocforward|:
%    \begin{macrocode}
\newcommand{\childdocforwardprefix}[3][]
{
  \begingroup
    \def\childdocextract #2##1~~~{\def\childdoctmp{\childdocforward[#1]{#3##1}}}
    \expandafter\childdocextract\childdocname~~~
    \expandafter
  \endgroup
  \childdoctmp
}
%    \end{macrocode}

% \macro{\childdoc}
% The deprecated macro |\childdoc| is a legacy version of |\childdocmain|:
%    \begin{macrocode}
\newcommand{\childdoc}{\childdocmain}
%    \end{macrocode}

% \macro{\childdocredirect}
% The deprecated macro |\childdocredirect| is a legacy version
% of |\childdocforward| and |\childdocforwardprefix|:
%    \begin{macrocode}
\newcommand{\childdocredirect}[2][]
{
  \begingroup
    \if?#1?
      \def\childdoctmp{\childdocforward{#2}}
    \else
      \def\childdoctmp{\childdocforwardprefix{#1}{#2}}
    \fi
    \expandafter
  \endgroup
  \childdoctmp
}
%    \end{macrocode}

%\iffalse
%</package>
%\fi
%
\endinput

\childdocof{cdocsamp}
%    \end{macrocode}

%\iffalse
%</samplechap1|samplechap2>
%\fi
%
%\iffalse
%<*samplechap1>
%\fi
% Some text for chapter 1:
%    \begin{macrocode}
\section{one}
some text in chapter one
%    \end{macrocode}

%\iffalse
%</samplechap1>
%\fi
% Some text for chapter 2:
%\iffalse
%<*samplechap2>
%\fi
%    \begin{macrocode}
\section{two}
more text in chapter two
%    \end{macrocode}

%\iffalse
%</samplechap2>
%\fi
%
% %%%%%%%%%%%%%%%%%%%%%%%%%%%%%%%%%%%%%%
% \paragraph{Part Include Files.}
%
% The include files are called |cdocspt3.tex| and |cdocspt4.tex|.
%
%\iffalse
%<*samplepart3|samplepart4>
%\fi

% Optional override for |\version| flag:
%    \begin{macrocode}
%%\providecommand{\version}{final}
%    \end{macrocode}

% Include the main document:
%    \begin{macrocode}
% \iffalse
%
% childdoc.dtx Copyright (C) 2017-2018 Niklas Beisert
%
% This work may be distributed and/or modified under the
% conditions of the LaTeX Project Public License, either version 1.3
% of this license or (at your option) any later version.
% The latest version of this license is in
%   http://www.latex-project.org/lppl.txt
% and version 1.3 or later is part of all distributions of LaTeX
% version 2005/12/01 or later.
%
% This work has the LPPL maintenance status `maintained'.
%
% The Current Maintainer of this work is Niklas Beisert.
%
% This work consists of the files childdoc.dtx and childdoc.ins
% and the derived files childdoc.def and cdocsamp.tex with
% cdocsch1.tex, cdocsch2.tex, cdocsdrf.tex, cdocsfn1.tex, cdocsfn2.tex.
%
%<package>\ifdefined\childdocmain\endinput\fi
%<package>\ProvidesFile{childdoc.def}[2018/12/30 v2.0 child document driver]
%<samplemain>\ProvidesFile{cdocsamp.tex}[2018/12/30 v2.0 sample for childdoc]
%<*driver>
%\ProvidesFile{childdoc.drv}[2018/12/30 v2.0 childdoc reference manual file]
\PassOptionsToClass{10pt,a4paper}{article}
\documentclass{ltxdoc}

\usepackage[margin=35mm]{geometry}
\usepackage{hyperref}
\usepackage{hyperxmp}
\usepackage[usenames]{color}

\hypersetup{colorlinks=true}
\hypersetup{pdfstartview=FitH}
\hypersetup{pdfpagemode=UseNone}
\hypersetup{pdfsource={}}
\hypersetup{pdflang={en-UK}}
\hypersetup{pdfcopyright={Copyright 2017-2018 Niklas Beisert.
  This work may be distributed and/or modified under the
  conditions of the LaTeX Project Public License, either version 1.3
  of this license or (at your option) any later version.}}
\hypersetup{pdflicenseurl={http://www.latex-project.org/lppl.txt}}
\hypersetup{pdfcontactaddress={ETH Zurich, ITP, HIT K,
  Wolfgang-Pauli-Strasse 27}}
\hypersetup{pdfcontactpostcode={8093}}
\hypersetup{pdfcontactcity={Zurich}}
\hypersetup{pdfcontactcountry={Switzerland}}
\hypersetup{pdfcontactemail={nbeisert@itp.phys.ethz.ch}}
\hypersetup{pdfcontacturl={http://people.phys.ethz.ch/\xmptilde nbeisert/}}

\newcommand{\secref}[1]{\hyperref[#1]{section \ref*{#1}}}

\parskip1ex
\parindent0pt
\let\olditemize\itemize
\def\itemize{\olditemize\parskip0pt}

\begin{document}

\title{The \textsf{childdoc} Package}
\hypersetup{pdftitle={The childdoc Package}}
\author{Niklas Beisert\\[2ex]
  Institut f\"ur Theoretische Physik\\
  Eidgen\"ossische Technische Hochschule Z\"urich\\
  Wolfgang-Pauli-Strasse 27, 8093 Z\"urich, Switzerland\\[1ex]
  \href{mailto:nbeisert@itp.phys.ethz.ch}
  {\texttt{nbeisert@itp.phys.ethz.ch}}}
\hypersetup{pdfauthor={Niklas Beisert}}
\hypersetup{pdfsubject={Manual for the LaTeX2e Package childdoc}}
\date{30 December 2018, \textsf{v2.0}}
\maketitle

\begin{abstract}\noindent
\textsf{childdoc} is a \LaTeXe{} package
that enables the direct compilation
of document sections included by |\include|
to individual files.
\end{abstract}

\begingroup
\parskip0ex
\tableofcontents
\endgroup

%%%%%%%%%%%%%%%%%%%%%%%%%%%%%%%%%%%%%%%%%%%%%%%%%%%%%%%%%%%%%%%%%%%%%%%%%%%%%%%%
%%%%%%%%%%%%%%%%%%%%%%%%%%%%%%%%%%%%%%%%%%%%%%%%%%%%%%%%%%%%%%%%%%%%%%%%%%%%%%%%
\section{Introduction}

\LaTeX{} provides a mechanism to structure a large document (such as a book)
into a main file and several child files (containing the chapters)
using the |\include| command.
This mechanism is beneficial for documents
which span hundreds of pages in order to
make the source file(s) more manageable.
Moreover, compilation can be restricted to
selected child files by means of the |\includeonly| command.
The latter feature can be used to reduce the compilation time while editing
(this was significantly more useful in the earlier days of \LaTeX{})
or to generate a smaller document which is easier to navigate.
Another application of |\includeonly| is to generate
documents consisting of selected parts of the complete document.

However, there are a few drawbacks of the plain |\include| mechanism:
\begin{itemize}
\item
The child files cannot be compiled on their own,
they can only be compiled via the main file.
A naive editing environment
(such as a text editor with an option
to have the current file processed by \LaTeX)
may require one to switch to the main file before compiling;
attempting to compile the child file produces errors.
\item
The main file must be modified (each time)
to adjust the |\includeonly| command
to the present needs. This easily leaves the main file in a messy state.
\item
The generated document will always carry the filename
of the main document. This is inconvenient if
several child files are to be compiled and
to be kept for distribution.
\end{itemize}

The present package provides a simple interface
to make child files individually compilable by \LaTeX{}.
Compiling a child file then has the same effect as compiling
the main file with an |\includeonly| command
to select the appropriate child.
Moreover the generated document will carry the name of the child
rather than the main file.
This resolves all three above issues.

This feature is meant to make the editing of books,
thesis documents and lecture notes somewhat more convenient.
However, the package can also be used efficiently for
composing a series of documents (such as exercise sheets)
which are typically distributed individually.
It then assists the author in generating the individual documents
(potentially in different versions)
as well as a document containing the collected series.
Another application is in developing style files
or other kinds of included material
where compilation of the style file could redirect
to a sample or test file.

%%%%%%%%%%%%%%%%%%%%%%%%%%%%%%%%%%%%%%%%%%%%%%%%%%%%%%%%%%%%%%%%%%%%%%%%%%%%%%%%
%%%%%%%%%%%%%%%%%%%%%%%%%%%%%%%%%%%%%%%%%%%%%%%%%%%%%%%%%%%%%%%%%%%%%%%%%%%%%%%%
\section{Usage}

First of all, the package \textsf{childdoc} is \emph{not} a standard
\LaTeXe{} |.sty| style file! Therefore it needs to be invoked in
a non-standard way.

%%%%%%%%%%%%%%%%%%%%%%%%%%%%%%%%%%%%%%%%%%%%%%%%%%%%%%%%%%%%%%%%%%%%%%%%%%%%%%%%
\subsection{Included Files}
\label{sec:include}

%%%%%%%%%%%%%%%%%%%%%%%%%%%%%%%%%%%%%%%%
\DescribeMacro{\childdocmain}
To use the package, add the commands
\begin{center}
\begin{tabular}{l}
|\input{childdoc.def}|\\
|\childdocmain{}|\\
\end{tabular}
\end{center}
at the very top of the main \LaTeX{} file,
in particular \emph{before} the |\documentclass| statement!
The argument of |\childdocmain| should be left empty
(but it must be present).

%%%%%%%%%%%%%%%%%%%%%%%%%%%%%%%%%%%%%%%%
\DescribeMacro{\childdocof}
Furthermore, add the commands
\begin{center}
\begin{tabular}{l}
|\input{childdoc.def}|\\
|\childdocof{|\textit{main}|}|\\
\end{tabular}
\end{center}
at the top of every child file \textit{child}
which is included by |\include{|\textit{child}|}|
from within the main file
(or at least for those files to be compiled individually).
The argument \textit{main} must be the filename of the main file.

There are a couple of
considerations in setting up the main and child documents:

%%%%%%%%%%%%%%%%%%%%%%%%%%%%%%%%%%%%%%%%
\paragraph{Restrictions.}

Please note the following restrictions:
\begin{itemize}
\item
|\childdocmain| must be called with one argument \textit{main}
to ensure compatibility with earlier version of the package.
It must either be empty (|\childdocmain{}|)
or precisely match the filename of the main file in which it is specified.
See \secref{sec:detection} for further information.
\item
The filename \textit{main} must be specified without the |.tex| extension.
\item
The filename \textit{main} is case sensitive
(even in case-insensitive file systems)
due to internal string comparison.
\item
The argument \textit{main} should be fully expanded, it cannot be a macro.
\item
Subdirectories and special characters should be avoided in filenames.
\item
The command |\childdocmain{|\textit{main}|}| must be followed by a whitespace.
It should not be followed immediately by another command
or by a comment mark `|%|'.
This is because the \TeX{} parser reads the token immediately following
the argument of |\childdocmain| and puts it
at the beginning of every child section;
however, a white\-space is ignored.
\end{itemize}

%%%%%%%%%%%%%%%%%%%%%%%%%%%%%%%%%%%%%%%%
\paragraph{Content of Main File.}

It is advisable to place all content in the child files included by |\include|.
Any output contained in the main file will appear in all child documents
unless suppressed manually;
it cannot be suppressed automatically by the |\includeonly| directive
and thus should normally be avoided.
A method to include some content in the main file
by means of conditional processing is described in \secref{sec:conditional}.

%%%%%%%%%%%%%%%%%%%%%%%%%%%%%%%%%%%%%%%%
\paragraph{Page Numbering.}

When only a part of the document is compiled,
the appropriate numbering of pages
(as well as other status parameters)
is determined from the |.aux| files.
The latter contain information from previous passes.
However this information needs to propagate through
all intermediate child documents.
Therefore the page numbering in child documents may well
be inconsistent until the complete document is compiled at least once.

A useful (if unconventional) way to always ensure a consistent
page numbering is to restart the numbering in each child document
and denote the pages by `\textit{child}|.|\textit{page}'
where \textit{child} represents the chapter/section number of the child file.
This can be achieved by the command
|\numberwithin{page}{|\textit{child}|}|
of the \textsf{amsmath} package
where \textit{child} can be |chapter| or |section|
depending on the chosen structuring.
Alternatively, one can modify the macro |\thepage| appropriately
and reset the counter |page| at the start of each child file.

%%%%%%%%%%%%%%%%%%%%%%%%%%%%%%%%%%%%%%%%%%%%%%%%%%%%%%%%%%%%%%%%%%%%%%%%%%%%%%%%
\subsection{Conditional Processing}
\label{sec:conditional}

The package provides a mechanism to compile different versions
of a document. To customise the versions further some conditional processing
can come in handy to distinguish which version is being compiled.
The package provides two macros to describe the compilation context:

%%%%%%%%%%%%%%%%%%%%%%%%%%%%%%%%%%%%%%%%
\DescribeMacro{\ifchilddoc}
The conditional |\ifchilddoc| distinguishes between the compilation of
child documents and the main document:
%
\begin{center}
|\ifchilddoc |\textit{child-code}| |[|\||else |\textit{main-code}]| \||fi|
\end{center}

%%%%%%%%%%%%%%%%%%%%%%%%%%%%%%%%%%%%%%%%
\DescribeMacro{\childdocname}
\DescribeMacro{\childdocjob}
The macro |\childdocname| contains the filename (without extension)
of the main or child file being processed.
Note that |\childdocjob| will always contain the name of the main file.

%%%%%%%%%%%%%%%%%%%%%%%%%%%%%%%%%%%%%%%%
\paragraph{Title Page.}

Conditional processing can be used to include a title or banner page
in the main document when proper precautions are taken.
Importantly, the code in the main file should ensure that the page counter
(as well as other status parameters which are stored in the |.aux| files)
takes the same value after the conditional processing.
Otherwise the page numbers may take divergent values
depending on which part is compiled.

For example, a title page could be declared by:
%
\begin{center}
\begin{tabular}{l}
|\ifchilddoc\||else|\\
|\addtocounter{page}{-1}|\\
\textit{code for title page}\\
|\newpage|\\
|\||fi|
\end{tabular}
\end{center}
%
A banner page for the child documents can be generated by:
%
\begin{center}
\begin{tabular}{l}
|\ifchilddoc|\\
|\addtocounter{page}{-1}|\\
\textit{code for banner page}\\
|\newpage|\\
|\||fi|
\end{tabular}
\end{center}
%
Here one could write a message such as:
\begin{center}
|This is the part \childdocname{} of \childdocjob{}.|
\end{center}

%%%%%%%%%%%%%%%%%%%%%%%%%%%%%%%%%%%%%%%%%%%%%%%%%%%%%%%%%%%%%%%%%%%%%%%%%%%%%%%%
\subsection{Flags}
\label{sec:flags}

The package makes it easy to generate different versions
of the main or child documents.
To this end compilation flags can be defined
and assigned different default values.
They will be particularly useful in conjunction
with the forwarding mechanism described in \secref{sec:forward}.

For example, it may be useful to have a flag |\version|
which can be set to |draft| or |final|.
The document source will contain some conditional code
depending on the value of |\version|.
Suppose further, the flag should default to |final| for the main file
and to |draft| for child files
which is a natural assignment for editing the document.
This is achieved by placing the following code
in the preamble of the main document
(below the |\childdocmain| directive):
%
\begin{center}
\begin{tabular}{l}
|\ifchilddoc|\\
|\providecommand{\version}{draft}|\\
|\||else|\\
|\providecommand{\version}{final}|\\
|\||fi|
\end{tabular}
\end{center}
%
The definition by |\providecommand| makes sure
that previous definitions are not overwritten.
Further statements |\providecommand{\version}{...}|
can thus be added before the above code to override it.

For the main file, one might add a line
(between |\childdocmain| and the above block)
%
\begin{center}
|%\ifchilddoc\||else\providecommand{\version}{draft}\||fi|
\end{center}
%
which can be uncommented to produce a draft version.
Likewise one can add a line to the very top of a child file
(above the |\childdocof{|\textit{main}|}| directive)
%
\begin{center}
|%\providecommand{\version}{final}|
\end{center}
%
which can be uncommented to produce the final version of this child document.

%%%%%%%%%%%%%%%%%%%%%%%%%%%%%%%%%%%%%%%%%%%%%%%%%%%%%%%%%%%%%%%%%%%%%%%%%%%%%%%%
\subsection{Forwarding}
\label{sec:forward}

Different versions of the main or child documents
using compilation flags as described in \secref{sec:flags}
can be (permanently) stored in different files
for convenient compilation, viewing and distribution.
To this end, the package defines a command
to pass on compilation to a different file:

%%%%%%%%%%%%%%%%%%%%%%%%%%%%%%%%%%%%%%%%
\DescribeMacro{\childdocforward}
The command |\childdocforward| redirects processing to
another source file:
%
\begin{center}
\begin{tabular}{l}
|\input{childdoc.def}|\\
|\childdocforward[|\textit{main}|]{|\textit{dest}|}|\\
\end{tabular}
\end{center}
%
The argument \textit{dest} is the destination file
(without extension).
It should be the main file or one of the child files.
Note that further \textsf{childdoc} directives
such as |\childdocof| and |\childdocforward|
in the indicated file will be processed in this form.
The optional argument \textit{main}
passes on directly to the main file \textit{main}
while pretending to compile the child \textit{dest}.
This form behaves as if \textit{dest}
issues |\childdocof{|\textit{main}|}| right away,
and no further \textsf{childdoc} directives will be processed.

%%%%%%%%%%%%%%%%%%%%%%%%%%%%%%%%%%%%%%%%
\DescribeMacro{\...prefix}
In the alternative form |\childdocforwardprefix|,
%
\begin{center}
\begin{tabular}{l}
|\input{childdoc.def}|\\
|\childdocforwardprefix[|\textit{main}|]{|\textit{prefix}|}{|\textit{dest}|}|
\end{tabular}
\end{center}
%
the destination file is determined by a pattern
depending on the current file:
To make this work, the current file must be called
`{\textit{prefix}\hspace{0.2em}\textit{suffix}}'
with \textit{prefix} matching precisely the argument.
Processing is then passed on to the file
`{\textit{dest}\hspace{0.2em}\textit{suffix}}'.
Surely, the same effect is achieved by
directly specifying the
argument `{\textit{dest}\hspace{0.2em}\textit{suffix}}'
in the first form.
However, that requires to set up a different file
for each child. With the alternative form of the command
all these files can have exactly the same content
which simplifies setting them up and maintaining them.

For example, the following file |draft.tex|
with a compilation flag |\version| as described in \secref{sec:flags}
compiles the main document as a draft:
%
\begin{center}
\begin{tabular}{l}
|\def\version{draft}|\\
|\input{childdoc.def}|\\
|\childdocforward{|\textit{main}|}|
\end{tabular}
\end{center}
%
Likewise, the following files |final|\textit{nn}|.tex|
compile the final version of the child document
|child|\textit{nn}|.tex|:
%
\begin{center}
\begin{tabular}{l}
|\def\version{final}|\\
|\input{childdoc.def}|\\
|\childdocforwardprefix{final}{child}|
\end{tabular}
\end{center}
%

Note that when several versions of a main file and/or of each child file
are to be generated, it may be convenient to set up a |Makefile| or
shell script to automatise the process.

%%%%%%%%%%%%%%%%%%%%%%%%%%%%%%%%%%%%%%%%%%%%%%%%%%%%%%%%%%%%%%%%%%%%%%%%%%%%%%%%
\subsection{Command Line Processing}
\label{sec:commandline}

The effect of redirection files can also be achieved by invoking
the \LaTeX{} compiler with a more elaborate command line.
Most conveniently this should be done as part
of a shell script or a |Makefile|.

When using \textsf{childdoc} in the main file, the following
command lines effectively perform a redirection
(note that depending on the shell being used,
backslashes may have to be doubled: `|\|' $\to$ `|\\|'):
%
\begin{center}
|... -jobname "|\textit{target}|" |\\|"|[\textit{flags}]%
|\input{childdoc.def}\childdocforward[|\textit{main}|]{|\textit{dest}|}"|
\end{center}
%
Here \textit{target} is the name of the output file,
\textit{main} is the name of the main file
and \textit{dest} is the name of the main or child file to be processed
(all filenames without extensions).
The optional argument \textit{main} can be omitted
if \textit{main} matches \textit{dest}.
Optionally, compilation \textit{flags} can be defined via |\def| commands.
This command line makes the \TeX{} engine believe
it is compiling the file \textit{target}
whose content is specified as the latter parameter.
The provided code then forwards the processing to
\textit{main} or \textit{dest} as described in \secref{sec:forward}.

%%%%%%%%%%%%%%%%%%%%%%%%%%%%%%%%%%%%%%%%%%%%%%%%%%%%%%%%%%%%%%%%%%%%%%%%%%%%%%%%
\subsection{Include by Input}
\label{sec:input}

Including child documents by |\include| has some restrictions by design.
Most notably, the content of a child document always occupies
its own set of pages; pages cannot be shared between child documents.
Usually, this behaviour makes perfect sense
because each child document contain an essential part of the document.
However, in some situations it may be desirable to compose
a document from a collection of parts
without having mandatory page breaks between then.
For this case, the package
provides a mechanism to include parts
by |\input| which can also be processed individually.
However, by construction this mechanism
requires manual handling of the content to be output.

%%%%%%%%%%%%%%%%%%%%%%%%%%%%%%%%%%%%%%%%
\DescribeMacro{\ifchilddocmanual}
The main file should be prepared as usual, see \secref{sec:include}.
However, the document body must make a distinction
between processing of an individual part and of the main document, e.g.:
%
\begin{center}
\begin{tabular}{l}
|\ifchilddocmanual|\\
|\input{\childdocname}|\\
|\||else|\\
\textit{document body with }|\input{|\textit{part}|}|\\
|\||fi|
\end{tabular}
\end{center}
%
The conditional |\ifchilddocmanual| is true whenever
a part to be included by |\input| is being compiled,
and the name of the part is stored in |\childdocname|.

%%%%%%%%%%%%%%%%%%%%%%%%%%%%%%%%%%%%%%%%
\DescribeMacro{\childdocby}
Each part to be included by |\input| should start with:
%
\begin{center}
\begin{tabular}{l}
|\input{childdoc.def}|\\
|\childdocby{|\textit{main}|}|\\
\end{tabular}
\end{center}
%
The directive |\childdocby| is similar to |\childdocof|
described in \secref{sec:include},
but the subsequent selection of content must be done manually.
To that end, both |\ifchilddoc| and |\ifchilddocmanual|
will be true upon processing of a part,
and the name of the part is stored in |\childdocname|.
Note that |\jobname| will be set to the filename of the current part
so that each part receives an individual |.aux| file
that does not interfere with the |.aux| file(s) of the main document.
This behaviour can be altered by the alternative form
|\childdocby[*]{|\textit{main}|}| (with a non-empty optional argument)
which uses the |.aux| file of the main document
by setting |\jobname| to \textit{main}.

%%%%%%%%%%%%%%%%%%%%%%%%%%%%%%%%%%%%%%%%%%%%%%%%%%%%%%%%%%%%%%%%%%%%%%%%%%%%%%%%
\subsection{Driver Development}
\label{sec:driver}

The \textsf{childdoc} mechanism can also be use for the development
of definition files such as \LaTeX{} styles or classes.
This case differs from the above setup with multiple parts
included by |\include| in that no |\includeonly| should be invoked.
This can be achieved by starting the include file
(before |\ProvidesPackage|) with:
%
\begin{center}
\begin{tabular}{l}
|\input{childdoc.def}|\\
|\childdocforward{|\textit{main}|}|\\
\end{tabular}
\end{center}
%
or alternatively with:
%
\begin{center}
\begin{tabular}{l}
|\input{childdoc.def}|\\
|\childdocby{|\textit{main}|}|\\
\end{tabular}
\end{center}
%
Both forms have slightly different effects as described above.
The main file is prepared as usual, see \secref{sec:include}.

%%%%%%%%%%%%%%%%%%%%%%%%%%%%%%%%%%%%%%%%%%%%%%%%%%%%%%%%%%%%%%%%%%%%%%%%%%%%%%%%
\subsection{Legacy Detection}
\label{sec:detection}

The directive |\childdocmain| in the main file can detect
whether the complete document or merely a child is to be compiled
even without using the directive |\childdocof|.
This method is deprecated because it is less robust
and there is no compelling reason to use it;
it is merely provided for backward compatibility
and it may be removed in future versions.

If the detection mechanism is to be used,
it is mandatory to correctly specify
the filename of the main file as the argument of |\childdocmain|:
%
\begin{center}
\begin{tabular}{l}
|\input{childdoc.def}|\\
|\childdocmain{|\textit{main}|}|\\
\end{tabular}
\end{center}
%
If |\jobname| does not match the argument \textit{main} of |\childdocmain|,
it is assumed that |\jobname| points to the child file to be compiled.
When using |\childdocmain| with the main file specified as argument,
it suffices to start a child file
with just |\input{|\textit{main}|}|
without loading of the package and using |\childdocof|.
If instead all processing is done
with the appropriate \textsf{childdoc} directives,
the argument of \textit{main} of |\childdocmain| can be empty.

An alternative version of the command line processing described
in \secref{sec:commandline} using the detection mechanism reads:
%
\begin{center}
|... -jobname "|\textit{target}|" "|[\textit{flags}]%
[|\def\jobname{|\textit{dest}|}|]|\input{|\textit{main}|}"|
\end{center}

%%%%%%%%%%%%%%%%%%%%%%%%%%%%%%%%%%%%%%%%%%%%%%%%%%%%%%%%%%%%%%%%%%%%%%%%%%%%%%%%
\subsection{Manual Code}
\label{sec:manual}

In case one cannot be certain whether the definitions file |childdoc.def|
is installed on the target \TeX{} distribution
and one prefers not to ship it,
it is conceivable to paste a few relevant commands into the sources.

To that end, drop all statements |\input{childdoc.def}|
and perform the replacements as outlined below.
Instead of |\childdocmain{|\textit{main}|}| add the following code
to the top of the main file:
%
\begin{center}
\begin{tabular}{l}
|\||ifdefined\childdocname\endinput\||fi\newif\ifchilddoc|\\
|\edef\childdocname{\scantokens\expandafter{\jobname\noexpand}}|\\
|\def\childdocmain{|\textit{main}|}\||ifx\childdocmain\childdocname\||else|\\
|\childdoctrue\includeonly{\childdocname}\let\jobname\childdocmain\||fi|\\
\end{tabular}
\end{center}
%
Instead of |\childdocof{|\textit{main}|}| just include the main file
at the top of each child file:
%
\begin{center}
|\input{|\textit{main}|}|
\end{center}
%
A simple redirection |\childdocforward{|\textit{dest}|}| is achieved by:
%
\begin{center}
|\def\jobname{|\textit{dest}|}\input{\jobname}|
\end{center}
%
The redirection with prefix
|\childdocforwardprefix[|\textit{prefix}|]{|\textit{dest}|}|
is accomplished by:
%
\begin{center}
\begin{tabular}{l}
|{\edef\jobname{\scantokens\expandafter{\jobname\noexpand}}|\\
|\def\redirectjob |\textit{prefix}|#1~~~{\gdef\jobname{|\textit{dest}|#1}}|\\
|\expandafter\redirectjob\jobname~~~}\input{\jobname}|
\end{tabular}
\end{center}

In an alternative approach,
child documents can be compiled by a specific command line
without additional code or specific definitions:
%
\begin{center}
|... -jobname "|\textit{target}|" "|[\textit{flags}]%
|\includeonly{|\textit{dest}|}\input{|\textit{main}|}"|
\end{center}
%

%%%%%%%%%%%%%%%%%%%%%%%%%%%%%%%%%%%%%%%%%%%%%%%%%%%%%%%%%%%%%%%%%%%%%%%%%%%%%%%%
%%%%%%%%%%%%%%%%%%%%%%%%%%%%%%%%%%%%%%%%%%%%%%%%%%%%%%%%%%%%%%%%%%%%%%%%%%%%%%%%
\section{Information}

%%%%%%%%%%%%%%%%%%%%%%%%%%%%%%%%%%%%%%%%%%%%%%%%%%%%%%%%%%%%%%%%%%%%%%%%%%%%%%%%
\subsection{Copyright}

Copyright \copyright{} 2017--2018 Niklas Beisert

This work may be distributed and/or modified under the
conditions of the \LaTeX{} Project Public License, either version 1.3
of this license or (at your option) any later version.
The latest version of this license is in
  \url{http://www.latex-project.org/lppl.txt}
and version 1.3 or later is part of all distributions of \LaTeX{}
version 2005/12/01 or later.

This work has the LPPL maintenance status `maintained'.

The Current Maintainer of this work is Niklas Beisert.

This work consists of the files |README.txt|, |childdoc.ins| and |childdoc.dtx|
as well as the derived files |childdoc.def|, |cdocsamp.tex|
with |cdocsch1.tex|, |cdocsch2.tex|, |cdocspt3.tex|, |cdocspt4.tex|,
|cdocsdrf.tex|, |cdocsfn1.tex|, |cdocsfn2.tex|
as well as |childdoc.pdf|.

%%%%%%%%%%%%%%%%%%%%%%%%%%%%%%%%%%%%%%%%%%%%%%%%%%%%%%%%%%%%%%%%%%%%%%%%%%%%%%%%
\subsection{Files and Installation}

The package consists of the files:
%
\begin{center}
\begin{tabular}{ll}
    |README.txt|   & readme file \\
    |childdoc.ins| & installation file \\
    |childdoc.dtx| & source file \\
    |childdoc.def| & definition file \\
    |cdocsamp.tex| & sample main file \\
    |cdocsch1.tex| & sample include file \\
    |cdocsch2.tex| & sample include file \\
    |cdocspt3.tex| & sample part file \\
    |cdocspt4.tex| & sample part file \\
    |cdocsdrf.tex| & sample redirection file \\
    |cdocsfn1.tex| & sample redirection file \\
    |cdocsfn2.tex| & sample redirection file \\
    |childdoc.pdf| & manual
\end{tabular}
\end{center}
%
The distribution consists of the files
|README.txt|, |childdoc.ins| and |childdoc.dtx|.
%
\begin{itemize}
\item
Run (pdf)\LaTeX{} on |childdoc.dtx|
to compile the manual |childdoc.pdf| (this file).
\item
Run \LaTeX{} on |childdoc.ins| to create the definitions file |childdoc.def|
and the sample |cdocsamp.tex| with include files
|cdocsch1.tex|, |cdocsch2.tex|, |cdocspt3.tex|, |cdocspt4.tex|,
|cdocsdrf.tex|, |cdocsfn1.tex|, |cdocsfn2.tex|.
Then copy the file |childdoc.def| to an appropriate directory of your \LaTeX{}
distribution, e.g.\ \textit{texmf-root}|/tex/latex/childdoc|.
\end{itemize}

%%%%%%%%%%%%%%%%%%%%%%%%%%%%%%%%%%%%%%%%%%%%%%%%%%%%%%%%%%%%%%%%%%%%%%%%%%%%%%%%
\subsection{Related CTAN Packages}

There are several other packages which offer a similar functionality:
%
\begin{itemize}
\item
The packages
\href{http://ctan.org/pkg/docmute}{\textsf{docmute}},
\href{http://ctan.org/pkg/includex}{\textsf{includex}} and
\href{http://ctan.org/pkg/standalone}{\textsf{standalone}}
provide commands to include only the document body of
a child file thus allowing both files to be compiled individually.
\item
The packages \href{http://ctan.org/pkg/subdocs}{\textsf{subdocs}}
and \href{http://ctan.org/pkg/subfiles}{\textsf{subfiles}}
provide structures in which the main and child documents can be
encapsulated and allowing them to be compiled individually.
The inclusion mechanism is different from the conventional |\include|.
\item
The package \href{http://ctan.org/pkg/combine}{\textsf{combine}}
is an elaborate solution to combine several documents into one.
\end{itemize}
%
See also the CTAN topic \href{http://ctan.org/topic/subdocs}{\textsf{subdocs}}
for further related packages.
The present package differs from the above solutions in that
a document structure constructed with the conventional |\include| mechanism
just needs two extra commands at the top of every file
such that all constituent files can be compiled individually.

%%%%%%%%%%%%%%%%%%%%%%%%%%%%%%%%%%%%%%%%%%%%%%%%%%%%%%%%%%%%%%%%%%%%%%%%%%%%%%%%
%\subsection{Feature Suggestions}
%
%The following is a list of features which may be useful for future
%versions of this package:
%%
%\begin{itemize}
%\item
%\ldots
%\end{itemize}

%%%%%%%%%%%%%%%%%%%%%%%%%%%%%%%%%%%%%%%%%%%%%%%%%%%%%%%%%%%%%%%%%%%%%%%%%%%%%%%%
\subsection{Revision History}

%%%%%%%%%%%%%%%%%%%%%%%%%%%%%%%%%%%%%%%%
\paragraph{v2.0:} 2018/12/30

\begin{itemize}
\item
immediate forward processing
\item
added |\childdocby| mechanism
\item
manual restructured
\end{itemize}

%%%%%%%%%%%%%%%%%%%%%%%%%%%%%%%%%%%%%%%%
\paragraph{v1.6:} 2018/01/17

\begin{itemize}
\item
application for development of include files
\item
corrections to manual
\end{itemize}

%%%%%%%%%%%%%%%%%%%%%%%%%%%%%%%%%%%%%%%%
\paragraph{v1.5:} 2017/05/21

\begin{itemize}
\item
more complete structuring introduced
\item
|\childdocof| introduced
\item
|\childdoc| renamed to |\childdocmain|
\item
|\childredirect| renamed to |\childdocforward| and |\childdocforwardprefix|
and functionality expanded
\end{itemize}

%%%%%%%%%%%%%%%%%%%%%%%%%%%%%%%%%%%%%%%%
\paragraph{v1.0:} 2017/04/27

\begin{itemize}
\item
manual and install package
\item
first version published on CTAN
\end{itemize}

%%%%%%%%%%%%%%%%%%%%%%%%%%%%%%%%%%%%%%%%
\paragraph{v0.6:} 2017/04/26

\begin{itemize}
\item
redirection mechanism added
\end{itemize}

%%%%%%%%%%%%%%%%%%%%%%%%%%%%%%%%%%%%%%%%
\paragraph{v0.5:} 2017/04/26

\begin{itemize}
\item
functionality in definition file
\end{itemize}


%%%%%%%%%%%%%%%%%%%%%%%%%%%%%%%%%%%%%%%%%%%%%%%%%%%%%%%%%%%%%%%%%%%%%%%%%%%%%%%%
%%%%%%%%%%%%%%%%%%%%%%%%%%%%%%%%%%%%%%%%%%%%%%%%%%%%%%%%%%%%%%%%%%%%%%%%%%%%%%%%
%%%%%%%%%%%%%%%%%%%%%%%%%%%%%%%%%%%%%%%%%%%%%%%%%%%%%%%%%%%%%%%%%%%%%%%%%%%%%%%%
\appendix

\settowidth\MacroIndent{\rmfamily\scriptsize 000\ }

 \DocInput{childdoc.dtx}

\end{document}
%</driver>
% \fi
%
% %%%%%%%%%%%%%%%%%%%%%%%%%%%%%%%%%%%%%%%%%%%%%%%%%%%%%%%%%%%%%%%%%%%%%%%%%%%%%%
% %%%%%%%%%%%%%%%%%%%%%%%%%%%%%%%%%%%%%%%%%%%%%%%%%%%%%%%%%%%%%%%%%%%%%%%%%%%%%%
% \section{Sample}
%\iffalse
%<*samplemain>
%\fi
%
% The following presents a sample document
% with two chapters, two parts, a title page,
% a compile flag as well as three forwarding files to set the flag.
% It consists of eight |.tex| files:
% \begin{center}
% \begin{tabular}{ll}
% |cdocsamp.tex|&main file\\
% |cdocsch1.tex|&include file for chapter 1\\
% |cdocsch2.tex|&include file for chapter 2\\
% |cdocspt3.tex|&include file for part 3\\
% |cdocspt4.tex|&include file for part 4\\
% |cdocsdrf.tex|&forwarding file for main file in draft mode\\
% |cdocsfi1.tex|&forwarding file for final version of chapter 1\\
% |cdocsfi2.tex|&forwarding file for final version of chapter 2\\
% \end{tabular}
% \end{center}
% Each of the eight files can be compiled directly by the \LaTeX{} compiler.
%
% %%%%%%%%%%%%%%%%%%%%%%%%%%%%%%%%%%%%%%
% \paragraph{Main File.}
%
% The main file is called |cdocsamp.tex|.
%
% Load the \textsf{childdoc} definitions and
% declare the filename for the main document:
%    \begin{macrocode}
\input{childdoc.def}
\childdocmain{}
%    \end{macrocode}

% Optional override for |\version| flag:
%    \begin{macrocode}
%%\ifchilddoc\else\providecommand{\version}{draft}\fi
%    \end{macrocode}

% Define the default values for the |\version| flag
% (|final| for the main file and |draft| for childs):
%    \begin{macrocode}
\ifchilddoc
\providecommand{\version}{draft}
\else
\providecommand{\version}{final}
\fi
%    \end{macrocode}

% Load the standard document class:
%    \begin{macrocode}
\documentclass[12pt]{article}
%    \end{macrocode}

% Start the document body:
%    \begin{macrocode}
\begin{document}
%    \end{macrocode}

% Declare a title page.
% Print title, part of document being processed and version flag:
%    \begin{macrocode}
\addtocounter{page}{-1}
\begin{center}
{\LARGE\bfseries{}childdoc example\par}
\vspace{1cm}
\ifchilddoc
\ifchilddocmanual part\else chapter\fi:
`\childdocname' of `\childdocjob'\par
\else
main document: `\childdocjob'\par
\fi
version: \version\par
\end{center}
\newpage
%    \end{macrocode}

% Manually include selected file,
% otherwise process as usual:
%    \begin{macrocode}
\ifchilddocmanual
\section*{part `\childdocname'}
\input{\childdocname}
\else
%    \end{macrocode}

% Include the two chapters:
%    \begin{macrocode}
\include{cdocsch1}
\include{cdocsch2}
%    \end{macrocode}

% Include the two parts unless only chapters should be displayed:
%    \begin{macrocode}
\ifchilddoc\else
\section{part three}
\input{cdocspt3}
\section{part four}
\input{cdocspt4}
\fi
%    \end{macrocode}

% Process as usual until here:
%    \begin{macrocode}
\fi
%    \end{macrocode}

% End of document body:
%    \begin{macrocode}
\end{document}
%    \end{macrocode}
%\iffalse
%</samplemain>
%\fi
%
% %%%%%%%%%%%%%%%%%%%%%%%%%%%%%%%%%%%%%%
% \paragraph{Chapter Include Files.}
%
% The include files are called |cdocsch1.tex| and |cdocsch2.tex|.
%
%\iffalse
%<*samplechap1|samplechap2>
%\fi

% Optional override for |\version| flag:
%    \begin{macrocode}
%%\providecommand{\version}{final}
%    \end{macrocode}

% Include the main document:
%    \begin{macrocode}
\input{childdoc.def}
\childdocof{cdocsamp}
%    \end{macrocode}

%\iffalse
%</samplechap1|samplechap2>
%\fi
%
%\iffalse
%<*samplechap1>
%\fi
% Some text for chapter 1:
%    \begin{macrocode}
\section{one}
some text in chapter one
%    \end{macrocode}

%\iffalse
%</samplechap1>
%\fi
% Some text for chapter 2:
%\iffalse
%<*samplechap2>
%\fi
%    \begin{macrocode}
\section{two}
more text in chapter two
%    \end{macrocode}

%\iffalse
%</samplechap2>
%\fi
%
% %%%%%%%%%%%%%%%%%%%%%%%%%%%%%%%%%%%%%%
% \paragraph{Part Include Files.}
%
% The include files are called |cdocspt3.tex| and |cdocspt4.tex|.
%
%\iffalse
%<*samplepart3|samplepart4>
%\fi

% Optional override for |\version| flag:
%    \begin{macrocode}
%%\providecommand{\version}{final}
%    \end{macrocode}

% Include the main document:
%    \begin{macrocode}
\input{childdoc.def}
\childdocby{cdocsamp}
%    \end{macrocode}

%\iffalse
%</samplepart3|samplepart4>
%\fi
%
%\iffalse
%<*samplepart3>
%\fi
% Some text for part 3:
%    \begin{macrocode}
some text in part three
%    \end{macrocode}

%\iffalse
%</samplepart3>
%\fi
% Some text for part 4:
%\iffalse
%<*samplepart4>
%\fi
%    \begin{macrocode}
more text in part four
%    \end{macrocode}

%\iffalse
%</samplepart4>
%\fi
%
% %%%%%%%%%%%%%%%%%%%%%%%%%%%%%%%%%%%%%%
% \paragraph{Forwarding for a Complete Draft.}
%
% The following forwarding file |cdocsdrf.tex|
% compiles the main document in draft mode:
%\iffalse
%<*sampledraft>
%\fi
%    \begin{macrocode}
\def\version{draft}
\input{childdoc.def}
\childdocforward{cdocsamp}
%    \end{macrocode}

%\iffalse
%</sampledraft>
%\fi
%
% %%%%%%%%%%%%%%%%%%%%%%%%%%%%%%%%%%%%%%
% \paragraph{Forwarding for Final Version of the Chapters.}
%
% The following forwarding files |cdocsfn1.tex| and |cdocsfn2.tex|
% (with identical content)
% compile the final versions of the child documents
% |cdocsch1.tex| and |cdocsch2.tex|, respectively:
%\iffalse
%<*samplefinal>
%\fi
%    \begin{macrocode}
\def\version{final}
\input{childdoc.def}
\childdocforwardprefix[cdocsamp]{cdocsfn}{cdocsch}
%    \end{macrocode}

%\iffalse
%</samplefinal>
%\fi
%
% %%%%%%%%%%%%%%%%%%%%%%%%%%%%%%%%%%%%%%
% \paragraph{Command Line Processing.}
%
% The following three command lines generate the output files
% |cdocscld|, |cdocscl1| and |cdocscl2|
% which should be identical to
% |cdocsdrf|, |cdocsch1| and |cdocsfn2|, respectively:
% \begin{center}
% \begin{tabular}{l}
% |latex -jobname cdocscld \|\\
% |  "\def\version{draft}\input{childdoc.def}\childdocforward{cdocsamp}"|\\
% |latex -jobname cdocscl1 \|\\
% |  "\input{childdoc.def}\childdocforward[cdocsamp]{cdocsch1}"|\\
% |latex -jobname cdocscl2 \|\\
% |  "\def\version{final}\input{childdoc.def}\childdocforward{cdocsch2}"|
% \end{tabular}
% \end{center}
% Note that the trailing backslash on each first line
% merely continues the input to the second line
% (for convenient cut ant paste).
% Furthermore, the command |latex| can be replaced by any
% of its alternative versions such as |pdflatex|.
%
% %%%%%%%%%%%%%%%%%%%%%%%%%%%%%%%%%%%%%%%%%%%%%%%%%%%%%%%%%%%%%%%%%%%%%%%%%%%%%%
% %%%%%%%%%%%%%%%%%%%%%%%%%%%%%%%%%%%%%%%%%%%%%%%%%%%%%%%%%%%%%%%%%%%%%%%%%%%%%%
% \section{Implementation}
%\iffalse
%<*package>
%\fi
%
% This section describes the definitions file |childdoc.def|.

% The definitions cannot be loaded using |\usepackage| or |\RequirePackage|
% which has a mechanism to prevent loading a style file more than once.
% When loading the definitions by means of |\input|
% multiple instances have to be prevented manually:
%\iffalse
%This code needs to be before the `\ProvidesFile' directive
%which is defined at the beginning of this file.
%Therefore it is also placed there and commented out here.
%</package>
%<*discard>
%\fi
%    \begin{macrocode}
\ifdefined\childdocmain\endinput\fi
%    \end{macrocode}
%\iffalse
%</discard>
%<*package>
%\fi
%
% \macro{\ifchilddoc}
% \macro{\ifchilddocmanual}
% The conditional |\ifchilddoc| tells whether a
% child (true) or main (false) document is being compiled.
% The conditional |\ifchilddocmanual| tells whether
% the |\includeonly| mechanism is used (false) or
% the selection of child files must be performed manually (true).
% The definitions initialise to false:
%    \begin{macrocode}
\newif\ifchilddoc
\newif\ifchilddocmanual
%    \end{macrocode}

% \macro{\childdocname}
% \macro{\childdocjob}
% The macro |\childdocname| stores the name of the main document
% to be compiled. The macro |\childdocjob| stores the name of
% the document on which the \LaTeX{} compiler was originally invoked.
% The content of |\jobname| cannot be compared
% to filenames specified in the source due to different catcodes.
% The following code rescans |\jobname|, stores the result
% in |\childdocname| and saves a copy in |\childdocjob|:
%    \begin{macrocode}
\edef\childdocname{\scantokens\expandafter{\jobname\noexpand}}
\let\childdocjob\childdocname
%    \end{macrocode}

% \macro{\childdocdisable}
% The macro |\childdocdisable| prevents the main file
% from being processed more than once.
% At this stage, the main document command |\childdocmain|
% is assumed to be called once again where it should do nothing.
% Any subsequent call to it should prevent
% a secondary processing of the main document
% It overwrites the forwarding commands
% |\childdocof| and |\childdocforward|
% with empty macros to prevent further inclusions of the main document:
%    \begin{macrocode}
\newcommand{\childdocdisable}
{
  \renewcommand{\childdocmain}[1]{\renewcommand{\childdocmain}[1]{\endinput}}
  \renewcommand{\childdocof}[1]{}
  \renewcommand{\childdocby}[2][]{}
  \renewcommand{\childdocforward}[2][]{}
  \renewcommand{\childdocdisable}{}
}
%    \end{macrocode}

% \macro{\childdocmain}
% The macro |\childdocmain| is to be called at the top of the main file
% with nothing or the main filename (without extension) as argument.
% First, it breaks loops.
% If the argument is not empty and does not match |\childdocname|
% (which is set by the first inclusion of |childdoc.def|),
% |\ifchilddoc| is set to true, |\includeonly| is applied to the child file
% and |\jobname| is set to the main file
% (for proper handling of |.aux| files):
%    \begin{macrocode}
\newcommand{\childdocmain}[1]
{
  \childdocdisable\childdocmain{}
  \if?#1?\else
    \begingroup
      \def\childdoctmp{#1}
      \ifx\childdoctmp\childdocname
        \def\childdoctmp{}
      \else
        \def\childdoctmp
        {
          \childdoctrue
          \includeonly{\childdocname}
          \def\childdocjob{#1}
          \def\jobname{#1}
        }
      \fi
      \expandafter
    \endgroup
    \childdoctmp
  \fi
}
%    \end{macrocode}

% \macro{\childdocof}
% The command |\childdocof| redirects
% compilation to the main file |#1|.
%    \begin{macrocode}
\newcommand{\childdocof}[1]
{
  \childdocdisable
  \childdoctrue
  \includeonly{\childdocname}
  \def\jobname{#1}
  \def\childdocjob{#1}
  \input{#1}
}
%    \end{macrocode}

% \macro{\childdocby}
% The command |\childdocby| ....
%    \begin{macrocode}
\newcommand{\childdocby}[2][]
{
  \childdocdisable
  \childdoctrue
  \childdocmanualtrue
  \if?#1?\else
    \def\jobname{#2}
  \fi
  \def\childdocjob{#2}
  \input{#2}
  \endinput
}
%    \end{macrocode}

% \macro{\childdocforward}
% The command |\childdocforward| redirects
% compilation to the main file or
% (if the optional argument is given) a child file.
% Parameters are set as if the main file
% or a child file starting with |\childdocof| was compiled.
% Then compilation is handed over to the main file:
%    \begin{macrocode}
\newcommand{\childdocforward}[2][]
{
  \begingroup
    \if?#1?
      \def\childdoctmp
      {
        \def\childdocname{#2}
        \def\childdocjob{#2}
        \def\jobname{#2}
        \input{#2}
        \endinput
      }
    \else
      \def\childdoctmp
      {
        \childdocdisable
        \def\childdocname{#2}
        \childdoctrue
        \includeonly{#2}
        \def\childdocjob{#1}
        \def\jobname{#1}
        \input{#1}
        \endinput
      }
    \fi
    \expandafter
  \endgroup
  \childdoctmp
}
%    \end{macrocode}

% \macro{\childdocforwardprefix}
% The command |\childdocforwardprefix| redirects
% compilation to the main or a child file by means of a pattern.
% The prefix |#1| in the current filename is replaced by |#2|
% and the suffix of the current filename is kept
% (it is assumed that the filename does not contain the substring `|~~~|'
% which is used as a delimiter).
% Compilation is handed over to the new file by |\childdocforward|:
%    \begin{macrocode}
\newcommand{\childdocforwardprefix}[3][]
{
  \begingroup
    \def\childdocextract #2##1~~~{\def\childdoctmp{\childdocforward[#1]{#3##1}}}
    \expandafter\childdocextract\childdocname~~~
    \expandafter
  \endgroup
  \childdoctmp
}
%    \end{macrocode}

% \macro{\childdoc}
% The deprecated macro |\childdoc| is a legacy version of |\childdocmain|:
%    \begin{macrocode}
\newcommand{\childdoc}{\childdocmain}
%    \end{macrocode}

% \macro{\childdocredirect}
% The deprecated macro |\childdocredirect| is a legacy version
% of |\childdocforward| and |\childdocforwardprefix|:
%    \begin{macrocode}
\newcommand{\childdocredirect}[2][]
{
  \begingroup
    \if?#1?
      \def\childdoctmp{\childdocforward{#2}}
    \else
      \def\childdoctmp{\childdocforwardprefix{#1}{#2}}
    \fi
    \expandafter
  \endgroup
  \childdoctmp
}
%    \end{macrocode}

%\iffalse
%</package>
%\fi
%
\endinput

\childdocby{cdocsamp}
%    \end{macrocode}

%\iffalse
%</samplepart3|samplepart4>
%\fi
%
%\iffalse
%<*samplepart3>
%\fi
% Some text for part 3:
%    \begin{macrocode}
some text in part three
%    \end{macrocode}

%\iffalse
%</samplepart3>
%\fi
% Some text for part 4:
%\iffalse
%<*samplepart4>
%\fi
%    \begin{macrocode}
more text in part four
%    \end{macrocode}

%\iffalse
%</samplepart4>
%\fi
%
% %%%%%%%%%%%%%%%%%%%%%%%%%%%%%%%%%%%%%%
% \paragraph{Forwarding for a Complete Draft.}
%
% The following forwarding file |cdocsdrf.tex|
% compiles the main document in draft mode:
%\iffalse
%<*sampledraft>
%\fi
%    \begin{macrocode}
\def\version{draft}
% \iffalse
%
% childdoc.dtx Copyright (C) 2017-2018 Niklas Beisert
%
% This work may be distributed and/or modified under the
% conditions of the LaTeX Project Public License, either version 1.3
% of this license or (at your option) any later version.
% The latest version of this license is in
%   http://www.latex-project.org/lppl.txt
% and version 1.3 or later is part of all distributions of LaTeX
% version 2005/12/01 or later.
%
% This work has the LPPL maintenance status `maintained'.
%
% The Current Maintainer of this work is Niklas Beisert.
%
% This work consists of the files childdoc.dtx and childdoc.ins
% and the derived files childdoc.def and cdocsamp.tex with
% cdocsch1.tex, cdocsch2.tex, cdocsdrf.tex, cdocsfn1.tex, cdocsfn2.tex.
%
%<package>\ifdefined\childdocmain\endinput\fi
%<package>\ProvidesFile{childdoc.def}[2018/12/30 v2.0 child document driver]
%<samplemain>\ProvidesFile{cdocsamp.tex}[2018/12/30 v2.0 sample for childdoc]
%<*driver>
%\ProvidesFile{childdoc.drv}[2018/12/30 v2.0 childdoc reference manual file]
\PassOptionsToClass{10pt,a4paper}{article}
\documentclass{ltxdoc}

\usepackage[margin=35mm]{geometry}
\usepackage{hyperref}
\usepackage{hyperxmp}
\usepackage[usenames]{color}

\hypersetup{colorlinks=true}
\hypersetup{pdfstartview=FitH}
\hypersetup{pdfpagemode=UseNone}
\hypersetup{pdfsource={}}
\hypersetup{pdflang={en-UK}}
\hypersetup{pdfcopyright={Copyright 2017-2018 Niklas Beisert.
  This work may be distributed and/or modified under the
  conditions of the LaTeX Project Public License, either version 1.3
  of this license or (at your option) any later version.}}
\hypersetup{pdflicenseurl={http://www.latex-project.org/lppl.txt}}
\hypersetup{pdfcontactaddress={ETH Zurich, ITP, HIT K,
  Wolfgang-Pauli-Strasse 27}}
\hypersetup{pdfcontactpostcode={8093}}
\hypersetup{pdfcontactcity={Zurich}}
\hypersetup{pdfcontactcountry={Switzerland}}
\hypersetup{pdfcontactemail={nbeisert@itp.phys.ethz.ch}}
\hypersetup{pdfcontacturl={http://people.phys.ethz.ch/\xmptilde nbeisert/}}

\newcommand{\secref}[1]{\hyperref[#1]{section \ref*{#1}}}

\parskip1ex
\parindent0pt
\let\olditemize\itemize
\def\itemize{\olditemize\parskip0pt}

\begin{document}

\title{The \textsf{childdoc} Package}
\hypersetup{pdftitle={The childdoc Package}}
\author{Niklas Beisert\\[2ex]
  Institut f\"ur Theoretische Physik\\
  Eidgen\"ossische Technische Hochschule Z\"urich\\
  Wolfgang-Pauli-Strasse 27, 8093 Z\"urich, Switzerland\\[1ex]
  \href{mailto:nbeisert@itp.phys.ethz.ch}
  {\texttt{nbeisert@itp.phys.ethz.ch}}}
\hypersetup{pdfauthor={Niklas Beisert}}
\hypersetup{pdfsubject={Manual for the LaTeX2e Package childdoc}}
\date{30 December 2018, \textsf{v2.0}}
\maketitle

\begin{abstract}\noindent
\textsf{childdoc} is a \LaTeXe{} package
that enables the direct compilation
of document sections included by |\include|
to individual files.
\end{abstract}

\begingroup
\parskip0ex
\tableofcontents
\endgroup

%%%%%%%%%%%%%%%%%%%%%%%%%%%%%%%%%%%%%%%%%%%%%%%%%%%%%%%%%%%%%%%%%%%%%%%%%%%%%%%%
%%%%%%%%%%%%%%%%%%%%%%%%%%%%%%%%%%%%%%%%%%%%%%%%%%%%%%%%%%%%%%%%%%%%%%%%%%%%%%%%
\section{Introduction}

\LaTeX{} provides a mechanism to structure a large document (such as a book)
into a main file and several child files (containing the chapters)
using the |\include| command.
This mechanism is beneficial for documents
which span hundreds of pages in order to
make the source file(s) more manageable.
Moreover, compilation can be restricted to
selected child files by means of the |\includeonly| command.
The latter feature can be used to reduce the compilation time while editing
(this was significantly more useful in the earlier days of \LaTeX{})
or to generate a smaller document which is easier to navigate.
Another application of |\includeonly| is to generate
documents consisting of selected parts of the complete document.

However, there are a few drawbacks of the plain |\include| mechanism:
\begin{itemize}
\item
The child files cannot be compiled on their own,
they can only be compiled via the main file.
A naive editing environment
(such as a text editor with an option
to have the current file processed by \LaTeX)
may require one to switch to the main file before compiling;
attempting to compile the child file produces errors.
\item
The main file must be modified (each time)
to adjust the |\includeonly| command
to the present needs. This easily leaves the main file in a messy state.
\item
The generated document will always carry the filename
of the main document. This is inconvenient if
several child files are to be compiled and
to be kept for distribution.
\end{itemize}

The present package provides a simple interface
to make child files individually compilable by \LaTeX{}.
Compiling a child file then has the same effect as compiling
the main file with an |\includeonly| command
to select the appropriate child.
Moreover the generated document will carry the name of the child
rather than the main file.
This resolves all three above issues.

This feature is meant to make the editing of books,
thesis documents and lecture notes somewhat more convenient.
However, the package can also be used efficiently for
composing a series of documents (such as exercise sheets)
which are typically distributed individually.
It then assists the author in generating the individual documents
(potentially in different versions)
as well as a document containing the collected series.
Another application is in developing style files
or other kinds of included material
where compilation of the style file could redirect
to a sample or test file.

%%%%%%%%%%%%%%%%%%%%%%%%%%%%%%%%%%%%%%%%%%%%%%%%%%%%%%%%%%%%%%%%%%%%%%%%%%%%%%%%
%%%%%%%%%%%%%%%%%%%%%%%%%%%%%%%%%%%%%%%%%%%%%%%%%%%%%%%%%%%%%%%%%%%%%%%%%%%%%%%%
\section{Usage}

First of all, the package \textsf{childdoc} is \emph{not} a standard
\LaTeXe{} |.sty| style file! Therefore it needs to be invoked in
a non-standard way.

%%%%%%%%%%%%%%%%%%%%%%%%%%%%%%%%%%%%%%%%%%%%%%%%%%%%%%%%%%%%%%%%%%%%%%%%%%%%%%%%
\subsection{Included Files}
\label{sec:include}

%%%%%%%%%%%%%%%%%%%%%%%%%%%%%%%%%%%%%%%%
\DescribeMacro{\childdocmain}
To use the package, add the commands
\begin{center}
\begin{tabular}{l}
|\input{childdoc.def}|\\
|\childdocmain{}|\\
\end{tabular}
\end{center}
at the very top of the main \LaTeX{} file,
in particular \emph{before} the |\documentclass| statement!
The argument of |\childdocmain| should be left empty
(but it must be present).

%%%%%%%%%%%%%%%%%%%%%%%%%%%%%%%%%%%%%%%%
\DescribeMacro{\childdocof}
Furthermore, add the commands
\begin{center}
\begin{tabular}{l}
|\input{childdoc.def}|\\
|\childdocof{|\textit{main}|}|\\
\end{tabular}
\end{center}
at the top of every child file \textit{child}
which is included by |\include{|\textit{child}|}|
from within the main file
(or at least for those files to be compiled individually).
The argument \textit{main} must be the filename of the main file.

There are a couple of
considerations in setting up the main and child documents:

%%%%%%%%%%%%%%%%%%%%%%%%%%%%%%%%%%%%%%%%
\paragraph{Restrictions.}

Please note the following restrictions:
\begin{itemize}
\item
|\childdocmain| must be called with one argument \textit{main}
to ensure compatibility with earlier version of the package.
It must either be empty (|\childdocmain{}|)
or precisely match the filename of the main file in which it is specified.
See \secref{sec:detection} for further information.
\item
The filename \textit{main} must be specified without the |.tex| extension.
\item
The filename \textit{main} is case sensitive
(even in case-insensitive file systems)
due to internal string comparison.
\item
The argument \textit{main} should be fully expanded, it cannot be a macro.
\item
Subdirectories and special characters should be avoided in filenames.
\item
The command |\childdocmain{|\textit{main}|}| must be followed by a whitespace.
It should not be followed immediately by another command
or by a comment mark `|%|'.
This is because the \TeX{} parser reads the token immediately following
the argument of |\childdocmain| and puts it
at the beginning of every child section;
however, a white\-space is ignored.
\end{itemize}

%%%%%%%%%%%%%%%%%%%%%%%%%%%%%%%%%%%%%%%%
\paragraph{Content of Main File.}

It is advisable to place all content in the child files included by |\include|.
Any output contained in the main file will appear in all child documents
unless suppressed manually;
it cannot be suppressed automatically by the |\includeonly| directive
and thus should normally be avoided.
A method to include some content in the main file
by means of conditional processing is described in \secref{sec:conditional}.

%%%%%%%%%%%%%%%%%%%%%%%%%%%%%%%%%%%%%%%%
\paragraph{Page Numbering.}

When only a part of the document is compiled,
the appropriate numbering of pages
(as well as other status parameters)
is determined from the |.aux| files.
The latter contain information from previous passes.
However this information needs to propagate through
all intermediate child documents.
Therefore the page numbering in child documents may well
be inconsistent until the complete document is compiled at least once.

A useful (if unconventional) way to always ensure a consistent
page numbering is to restart the numbering in each child document
and denote the pages by `\textit{child}|.|\textit{page}'
where \textit{child} represents the chapter/section number of the child file.
This can be achieved by the command
|\numberwithin{page}{|\textit{child}|}|
of the \textsf{amsmath} package
where \textit{child} can be |chapter| or |section|
depending on the chosen structuring.
Alternatively, one can modify the macro |\thepage| appropriately
and reset the counter |page| at the start of each child file.

%%%%%%%%%%%%%%%%%%%%%%%%%%%%%%%%%%%%%%%%%%%%%%%%%%%%%%%%%%%%%%%%%%%%%%%%%%%%%%%%
\subsection{Conditional Processing}
\label{sec:conditional}

The package provides a mechanism to compile different versions
of a document. To customise the versions further some conditional processing
can come in handy to distinguish which version is being compiled.
The package provides two macros to describe the compilation context:

%%%%%%%%%%%%%%%%%%%%%%%%%%%%%%%%%%%%%%%%
\DescribeMacro{\ifchilddoc}
The conditional |\ifchilddoc| distinguishes between the compilation of
child documents and the main document:
%
\begin{center}
|\ifchilddoc |\textit{child-code}| |[|\||else |\textit{main-code}]| \||fi|
\end{center}

%%%%%%%%%%%%%%%%%%%%%%%%%%%%%%%%%%%%%%%%
\DescribeMacro{\childdocname}
\DescribeMacro{\childdocjob}
The macro |\childdocname| contains the filename (without extension)
of the main or child file being processed.
Note that |\childdocjob| will always contain the name of the main file.

%%%%%%%%%%%%%%%%%%%%%%%%%%%%%%%%%%%%%%%%
\paragraph{Title Page.}

Conditional processing can be used to include a title or banner page
in the main document when proper precautions are taken.
Importantly, the code in the main file should ensure that the page counter
(as well as other status parameters which are stored in the |.aux| files)
takes the same value after the conditional processing.
Otherwise the page numbers may take divergent values
depending on which part is compiled.

For example, a title page could be declared by:
%
\begin{center}
\begin{tabular}{l}
|\ifchilddoc\||else|\\
|\addtocounter{page}{-1}|\\
\textit{code for title page}\\
|\newpage|\\
|\||fi|
\end{tabular}
\end{center}
%
A banner page for the child documents can be generated by:
%
\begin{center}
\begin{tabular}{l}
|\ifchilddoc|\\
|\addtocounter{page}{-1}|\\
\textit{code for banner page}\\
|\newpage|\\
|\||fi|
\end{tabular}
\end{center}
%
Here one could write a message such as:
\begin{center}
|This is the part \childdocname{} of \childdocjob{}.|
\end{center}

%%%%%%%%%%%%%%%%%%%%%%%%%%%%%%%%%%%%%%%%%%%%%%%%%%%%%%%%%%%%%%%%%%%%%%%%%%%%%%%%
\subsection{Flags}
\label{sec:flags}

The package makes it easy to generate different versions
of the main or child documents.
To this end compilation flags can be defined
and assigned different default values.
They will be particularly useful in conjunction
with the forwarding mechanism described in \secref{sec:forward}.

For example, it may be useful to have a flag |\version|
which can be set to |draft| or |final|.
The document source will contain some conditional code
depending on the value of |\version|.
Suppose further, the flag should default to |final| for the main file
and to |draft| for child files
which is a natural assignment for editing the document.
This is achieved by placing the following code
in the preamble of the main document
(below the |\childdocmain| directive):
%
\begin{center}
\begin{tabular}{l}
|\ifchilddoc|\\
|\providecommand{\version}{draft}|\\
|\||else|\\
|\providecommand{\version}{final}|\\
|\||fi|
\end{tabular}
\end{center}
%
The definition by |\providecommand| makes sure
that previous definitions are not overwritten.
Further statements |\providecommand{\version}{...}|
can thus be added before the above code to override it.

For the main file, one might add a line
(between |\childdocmain| and the above block)
%
\begin{center}
|%\ifchilddoc\||else\providecommand{\version}{draft}\||fi|
\end{center}
%
which can be uncommented to produce a draft version.
Likewise one can add a line to the very top of a child file
(above the |\childdocof{|\textit{main}|}| directive)
%
\begin{center}
|%\providecommand{\version}{final}|
\end{center}
%
which can be uncommented to produce the final version of this child document.

%%%%%%%%%%%%%%%%%%%%%%%%%%%%%%%%%%%%%%%%%%%%%%%%%%%%%%%%%%%%%%%%%%%%%%%%%%%%%%%%
\subsection{Forwarding}
\label{sec:forward}

Different versions of the main or child documents
using compilation flags as described in \secref{sec:flags}
can be (permanently) stored in different files
for convenient compilation, viewing and distribution.
To this end, the package defines a command
to pass on compilation to a different file:

%%%%%%%%%%%%%%%%%%%%%%%%%%%%%%%%%%%%%%%%
\DescribeMacro{\childdocforward}
The command |\childdocforward| redirects processing to
another source file:
%
\begin{center}
\begin{tabular}{l}
|\input{childdoc.def}|\\
|\childdocforward[|\textit{main}|]{|\textit{dest}|}|\\
\end{tabular}
\end{center}
%
The argument \textit{dest} is the destination file
(without extension).
It should be the main file or one of the child files.
Note that further \textsf{childdoc} directives
such as |\childdocof| and |\childdocforward|
in the indicated file will be processed in this form.
The optional argument \textit{main}
passes on directly to the main file \textit{main}
while pretending to compile the child \textit{dest}.
This form behaves as if \textit{dest}
issues |\childdocof{|\textit{main}|}| right away,
and no further \textsf{childdoc} directives will be processed.

%%%%%%%%%%%%%%%%%%%%%%%%%%%%%%%%%%%%%%%%
\DescribeMacro{\...prefix}
In the alternative form |\childdocforwardprefix|,
%
\begin{center}
\begin{tabular}{l}
|\input{childdoc.def}|\\
|\childdocforwardprefix[|\textit{main}|]{|\textit{prefix}|}{|\textit{dest}|}|
\end{tabular}
\end{center}
%
the destination file is determined by a pattern
depending on the current file:
To make this work, the current file must be called
`{\textit{prefix}\hspace{0.2em}\textit{suffix}}'
with \textit{prefix} matching precisely the argument.
Processing is then passed on to the file
`{\textit{dest}\hspace{0.2em}\textit{suffix}}'.
Surely, the same effect is achieved by
directly specifying the
argument `{\textit{dest}\hspace{0.2em}\textit{suffix}}'
in the first form.
However, that requires to set up a different file
for each child. With the alternative form of the command
all these files can have exactly the same content
which simplifies setting them up and maintaining them.

For example, the following file |draft.tex|
with a compilation flag |\version| as described in \secref{sec:flags}
compiles the main document as a draft:
%
\begin{center}
\begin{tabular}{l}
|\def\version{draft}|\\
|\input{childdoc.def}|\\
|\childdocforward{|\textit{main}|}|
\end{tabular}
\end{center}
%
Likewise, the following files |final|\textit{nn}|.tex|
compile the final version of the child document
|child|\textit{nn}|.tex|:
%
\begin{center}
\begin{tabular}{l}
|\def\version{final}|\\
|\input{childdoc.def}|\\
|\childdocforwardprefix{final}{child}|
\end{tabular}
\end{center}
%

Note that when several versions of a main file and/or of each child file
are to be generated, it may be convenient to set up a |Makefile| or
shell script to automatise the process.

%%%%%%%%%%%%%%%%%%%%%%%%%%%%%%%%%%%%%%%%%%%%%%%%%%%%%%%%%%%%%%%%%%%%%%%%%%%%%%%%
\subsection{Command Line Processing}
\label{sec:commandline}

The effect of redirection files can also be achieved by invoking
the \LaTeX{} compiler with a more elaborate command line.
Most conveniently this should be done as part
of a shell script or a |Makefile|.

When using \textsf{childdoc} in the main file, the following
command lines effectively perform a redirection
(note that depending on the shell being used,
backslashes may have to be doubled: `|\|' $\to$ `|\\|'):
%
\begin{center}
|... -jobname "|\textit{target}|" |\\|"|[\textit{flags}]%
|\input{childdoc.def}\childdocforward[|\textit{main}|]{|\textit{dest}|}"|
\end{center}
%
Here \textit{target} is the name of the output file,
\textit{main} is the name of the main file
and \textit{dest} is the name of the main or child file to be processed
(all filenames without extensions).
The optional argument \textit{main} can be omitted
if \textit{main} matches \textit{dest}.
Optionally, compilation \textit{flags} can be defined via |\def| commands.
This command line makes the \TeX{} engine believe
it is compiling the file \textit{target}
whose content is specified as the latter parameter.
The provided code then forwards the processing to
\textit{main} or \textit{dest} as described in \secref{sec:forward}.

%%%%%%%%%%%%%%%%%%%%%%%%%%%%%%%%%%%%%%%%%%%%%%%%%%%%%%%%%%%%%%%%%%%%%%%%%%%%%%%%
\subsection{Include by Input}
\label{sec:input}

Including child documents by |\include| has some restrictions by design.
Most notably, the content of a child document always occupies
its own set of pages; pages cannot be shared between child documents.
Usually, this behaviour makes perfect sense
because each child document contain an essential part of the document.
However, in some situations it may be desirable to compose
a document from a collection of parts
without having mandatory page breaks between then.
For this case, the package
provides a mechanism to include parts
by |\input| which can also be processed individually.
However, by construction this mechanism
requires manual handling of the content to be output.

%%%%%%%%%%%%%%%%%%%%%%%%%%%%%%%%%%%%%%%%
\DescribeMacro{\ifchilddocmanual}
The main file should be prepared as usual, see \secref{sec:include}.
However, the document body must make a distinction
between processing of an individual part and of the main document, e.g.:
%
\begin{center}
\begin{tabular}{l}
|\ifchilddocmanual|\\
|\input{\childdocname}|\\
|\||else|\\
\textit{document body with }|\input{|\textit{part}|}|\\
|\||fi|
\end{tabular}
\end{center}
%
The conditional |\ifchilddocmanual| is true whenever
a part to be included by |\input| is being compiled,
and the name of the part is stored in |\childdocname|.

%%%%%%%%%%%%%%%%%%%%%%%%%%%%%%%%%%%%%%%%
\DescribeMacro{\childdocby}
Each part to be included by |\input| should start with:
%
\begin{center}
\begin{tabular}{l}
|\input{childdoc.def}|\\
|\childdocby{|\textit{main}|}|\\
\end{tabular}
\end{center}
%
The directive |\childdocby| is similar to |\childdocof|
described in \secref{sec:include},
but the subsequent selection of content must be done manually.
To that end, both |\ifchilddoc| and |\ifchilddocmanual|
will be true upon processing of a part,
and the name of the part is stored in |\childdocname|.
Note that |\jobname| will be set to the filename of the current part
so that each part receives an individual |.aux| file
that does not interfere with the |.aux| file(s) of the main document.
This behaviour can be altered by the alternative form
|\childdocby[*]{|\textit{main}|}| (with a non-empty optional argument)
which uses the |.aux| file of the main document
by setting |\jobname| to \textit{main}.

%%%%%%%%%%%%%%%%%%%%%%%%%%%%%%%%%%%%%%%%%%%%%%%%%%%%%%%%%%%%%%%%%%%%%%%%%%%%%%%%
\subsection{Driver Development}
\label{sec:driver}

The \textsf{childdoc} mechanism can also be use for the development
of definition files such as \LaTeX{} styles or classes.
This case differs from the above setup with multiple parts
included by |\include| in that no |\includeonly| should be invoked.
This can be achieved by starting the include file
(before |\ProvidesPackage|) with:
%
\begin{center}
\begin{tabular}{l}
|\input{childdoc.def}|\\
|\childdocforward{|\textit{main}|}|\\
\end{tabular}
\end{center}
%
or alternatively with:
%
\begin{center}
\begin{tabular}{l}
|\input{childdoc.def}|\\
|\childdocby{|\textit{main}|}|\\
\end{tabular}
\end{center}
%
Both forms have slightly different effects as described above.
The main file is prepared as usual, see \secref{sec:include}.

%%%%%%%%%%%%%%%%%%%%%%%%%%%%%%%%%%%%%%%%%%%%%%%%%%%%%%%%%%%%%%%%%%%%%%%%%%%%%%%%
\subsection{Legacy Detection}
\label{sec:detection}

The directive |\childdocmain| in the main file can detect
whether the complete document or merely a child is to be compiled
even without using the directive |\childdocof|.
This method is deprecated because it is less robust
and there is no compelling reason to use it;
it is merely provided for backward compatibility
and it may be removed in future versions.

If the detection mechanism is to be used,
it is mandatory to correctly specify
the filename of the main file as the argument of |\childdocmain|:
%
\begin{center}
\begin{tabular}{l}
|\input{childdoc.def}|\\
|\childdocmain{|\textit{main}|}|\\
\end{tabular}
\end{center}
%
If |\jobname| does not match the argument \textit{main} of |\childdocmain|,
it is assumed that |\jobname| points to the child file to be compiled.
When using |\childdocmain| with the main file specified as argument,
it suffices to start a child file
with just |\input{|\textit{main}|}|
without loading of the package and using |\childdocof|.
If instead all processing is done
with the appropriate \textsf{childdoc} directives,
the argument of \textit{main} of |\childdocmain| can be empty.

An alternative version of the command line processing described
in \secref{sec:commandline} using the detection mechanism reads:
%
\begin{center}
|... -jobname "|\textit{target}|" "|[\textit{flags}]%
[|\def\jobname{|\textit{dest}|}|]|\input{|\textit{main}|}"|
\end{center}

%%%%%%%%%%%%%%%%%%%%%%%%%%%%%%%%%%%%%%%%%%%%%%%%%%%%%%%%%%%%%%%%%%%%%%%%%%%%%%%%
\subsection{Manual Code}
\label{sec:manual}

In case one cannot be certain whether the definitions file |childdoc.def|
is installed on the target \TeX{} distribution
and one prefers not to ship it,
it is conceivable to paste a few relevant commands into the sources.

To that end, drop all statements |\input{childdoc.def}|
and perform the replacements as outlined below.
Instead of |\childdocmain{|\textit{main}|}| add the following code
to the top of the main file:
%
\begin{center}
\begin{tabular}{l}
|\||ifdefined\childdocname\endinput\||fi\newif\ifchilddoc|\\
|\edef\childdocname{\scantokens\expandafter{\jobname\noexpand}}|\\
|\def\childdocmain{|\textit{main}|}\||ifx\childdocmain\childdocname\||else|\\
|\childdoctrue\includeonly{\childdocname}\let\jobname\childdocmain\||fi|\\
\end{tabular}
\end{center}
%
Instead of |\childdocof{|\textit{main}|}| just include the main file
at the top of each child file:
%
\begin{center}
|\input{|\textit{main}|}|
\end{center}
%
A simple redirection |\childdocforward{|\textit{dest}|}| is achieved by:
%
\begin{center}
|\def\jobname{|\textit{dest}|}\input{\jobname}|
\end{center}
%
The redirection with prefix
|\childdocforwardprefix[|\textit{prefix}|]{|\textit{dest}|}|
is accomplished by:
%
\begin{center}
\begin{tabular}{l}
|{\edef\jobname{\scantokens\expandafter{\jobname\noexpand}}|\\
|\def\redirectjob |\textit{prefix}|#1~~~{\gdef\jobname{|\textit{dest}|#1}}|\\
|\expandafter\redirectjob\jobname~~~}\input{\jobname}|
\end{tabular}
\end{center}

In an alternative approach,
child documents can be compiled by a specific command line
without additional code or specific definitions:
%
\begin{center}
|... -jobname "|\textit{target}|" "|[\textit{flags}]%
|\includeonly{|\textit{dest}|}\input{|\textit{main}|}"|
\end{center}
%

%%%%%%%%%%%%%%%%%%%%%%%%%%%%%%%%%%%%%%%%%%%%%%%%%%%%%%%%%%%%%%%%%%%%%%%%%%%%%%%%
%%%%%%%%%%%%%%%%%%%%%%%%%%%%%%%%%%%%%%%%%%%%%%%%%%%%%%%%%%%%%%%%%%%%%%%%%%%%%%%%
\section{Information}

%%%%%%%%%%%%%%%%%%%%%%%%%%%%%%%%%%%%%%%%%%%%%%%%%%%%%%%%%%%%%%%%%%%%%%%%%%%%%%%%
\subsection{Copyright}

Copyright \copyright{} 2017--2018 Niklas Beisert

This work may be distributed and/or modified under the
conditions of the \LaTeX{} Project Public License, either version 1.3
of this license or (at your option) any later version.
The latest version of this license is in
  \url{http://www.latex-project.org/lppl.txt}
and version 1.3 or later is part of all distributions of \LaTeX{}
version 2005/12/01 or later.

This work has the LPPL maintenance status `maintained'.

The Current Maintainer of this work is Niklas Beisert.

This work consists of the files |README.txt|, |childdoc.ins| and |childdoc.dtx|
as well as the derived files |childdoc.def|, |cdocsamp.tex|
with |cdocsch1.tex|, |cdocsch2.tex|, |cdocspt3.tex|, |cdocspt4.tex|,
|cdocsdrf.tex|, |cdocsfn1.tex|, |cdocsfn2.tex|
as well as |childdoc.pdf|.

%%%%%%%%%%%%%%%%%%%%%%%%%%%%%%%%%%%%%%%%%%%%%%%%%%%%%%%%%%%%%%%%%%%%%%%%%%%%%%%%
\subsection{Files and Installation}

The package consists of the files:
%
\begin{center}
\begin{tabular}{ll}
    |README.txt|   & readme file \\
    |childdoc.ins| & installation file \\
    |childdoc.dtx| & source file \\
    |childdoc.def| & definition file \\
    |cdocsamp.tex| & sample main file \\
    |cdocsch1.tex| & sample include file \\
    |cdocsch2.tex| & sample include file \\
    |cdocspt3.tex| & sample part file \\
    |cdocspt4.tex| & sample part file \\
    |cdocsdrf.tex| & sample redirection file \\
    |cdocsfn1.tex| & sample redirection file \\
    |cdocsfn2.tex| & sample redirection file \\
    |childdoc.pdf| & manual
\end{tabular}
\end{center}
%
The distribution consists of the files
|README.txt|, |childdoc.ins| and |childdoc.dtx|.
%
\begin{itemize}
\item
Run (pdf)\LaTeX{} on |childdoc.dtx|
to compile the manual |childdoc.pdf| (this file).
\item
Run \LaTeX{} on |childdoc.ins| to create the definitions file |childdoc.def|
and the sample |cdocsamp.tex| with include files
|cdocsch1.tex|, |cdocsch2.tex|, |cdocspt3.tex|, |cdocspt4.tex|,
|cdocsdrf.tex|, |cdocsfn1.tex|, |cdocsfn2.tex|.
Then copy the file |childdoc.def| to an appropriate directory of your \LaTeX{}
distribution, e.g.\ \textit{texmf-root}|/tex/latex/childdoc|.
\end{itemize}

%%%%%%%%%%%%%%%%%%%%%%%%%%%%%%%%%%%%%%%%%%%%%%%%%%%%%%%%%%%%%%%%%%%%%%%%%%%%%%%%
\subsection{Related CTAN Packages}

There are several other packages which offer a similar functionality:
%
\begin{itemize}
\item
The packages
\href{http://ctan.org/pkg/docmute}{\textsf{docmute}},
\href{http://ctan.org/pkg/includex}{\textsf{includex}} and
\href{http://ctan.org/pkg/standalone}{\textsf{standalone}}
provide commands to include only the document body of
a child file thus allowing both files to be compiled individually.
\item
The packages \href{http://ctan.org/pkg/subdocs}{\textsf{subdocs}}
and \href{http://ctan.org/pkg/subfiles}{\textsf{subfiles}}
provide structures in which the main and child documents can be
encapsulated and allowing them to be compiled individually.
The inclusion mechanism is different from the conventional |\include|.
\item
The package \href{http://ctan.org/pkg/combine}{\textsf{combine}}
is an elaborate solution to combine several documents into one.
\end{itemize}
%
See also the CTAN topic \href{http://ctan.org/topic/subdocs}{\textsf{subdocs}}
for further related packages.
The present package differs from the above solutions in that
a document structure constructed with the conventional |\include| mechanism
just needs two extra commands at the top of every file
such that all constituent files can be compiled individually.

%%%%%%%%%%%%%%%%%%%%%%%%%%%%%%%%%%%%%%%%%%%%%%%%%%%%%%%%%%%%%%%%%%%%%%%%%%%%%%%%
%\subsection{Feature Suggestions}
%
%The following is a list of features which may be useful for future
%versions of this package:
%%
%\begin{itemize}
%\item
%\ldots
%\end{itemize}

%%%%%%%%%%%%%%%%%%%%%%%%%%%%%%%%%%%%%%%%%%%%%%%%%%%%%%%%%%%%%%%%%%%%%%%%%%%%%%%%
\subsection{Revision History}

%%%%%%%%%%%%%%%%%%%%%%%%%%%%%%%%%%%%%%%%
\paragraph{v2.0:} 2018/12/30

\begin{itemize}
\item
immediate forward processing
\item
added |\childdocby| mechanism
\item
manual restructured
\end{itemize}

%%%%%%%%%%%%%%%%%%%%%%%%%%%%%%%%%%%%%%%%
\paragraph{v1.6:} 2018/01/17

\begin{itemize}
\item
application for development of include files
\item
corrections to manual
\end{itemize}

%%%%%%%%%%%%%%%%%%%%%%%%%%%%%%%%%%%%%%%%
\paragraph{v1.5:} 2017/05/21

\begin{itemize}
\item
more complete structuring introduced
\item
|\childdocof| introduced
\item
|\childdoc| renamed to |\childdocmain|
\item
|\childredirect| renamed to |\childdocforward| and |\childdocforwardprefix|
and functionality expanded
\end{itemize}

%%%%%%%%%%%%%%%%%%%%%%%%%%%%%%%%%%%%%%%%
\paragraph{v1.0:} 2017/04/27

\begin{itemize}
\item
manual and install package
\item
first version published on CTAN
\end{itemize}

%%%%%%%%%%%%%%%%%%%%%%%%%%%%%%%%%%%%%%%%
\paragraph{v0.6:} 2017/04/26

\begin{itemize}
\item
redirection mechanism added
\end{itemize}

%%%%%%%%%%%%%%%%%%%%%%%%%%%%%%%%%%%%%%%%
\paragraph{v0.5:} 2017/04/26

\begin{itemize}
\item
functionality in definition file
\end{itemize}


%%%%%%%%%%%%%%%%%%%%%%%%%%%%%%%%%%%%%%%%%%%%%%%%%%%%%%%%%%%%%%%%%%%%%%%%%%%%%%%%
%%%%%%%%%%%%%%%%%%%%%%%%%%%%%%%%%%%%%%%%%%%%%%%%%%%%%%%%%%%%%%%%%%%%%%%%%%%%%%%%
%%%%%%%%%%%%%%%%%%%%%%%%%%%%%%%%%%%%%%%%%%%%%%%%%%%%%%%%%%%%%%%%%%%%%%%%%%%%%%%%
\appendix

\settowidth\MacroIndent{\rmfamily\scriptsize 000\ }

 \DocInput{childdoc.dtx}

\end{document}
%</driver>
% \fi
%
% %%%%%%%%%%%%%%%%%%%%%%%%%%%%%%%%%%%%%%%%%%%%%%%%%%%%%%%%%%%%%%%%%%%%%%%%%%%%%%
% %%%%%%%%%%%%%%%%%%%%%%%%%%%%%%%%%%%%%%%%%%%%%%%%%%%%%%%%%%%%%%%%%%%%%%%%%%%%%%
% \section{Sample}
%\iffalse
%<*samplemain>
%\fi
%
% The following presents a sample document
% with two chapters, two parts, a title page,
% a compile flag as well as three forwarding files to set the flag.
% It consists of eight |.tex| files:
% \begin{center}
% \begin{tabular}{ll}
% |cdocsamp.tex|&main file\\
% |cdocsch1.tex|&include file for chapter 1\\
% |cdocsch2.tex|&include file for chapter 2\\
% |cdocspt3.tex|&include file for part 3\\
% |cdocspt4.tex|&include file for part 4\\
% |cdocsdrf.tex|&forwarding file for main file in draft mode\\
% |cdocsfi1.tex|&forwarding file for final version of chapter 1\\
% |cdocsfi2.tex|&forwarding file for final version of chapter 2\\
% \end{tabular}
% \end{center}
% Each of the eight files can be compiled directly by the \LaTeX{} compiler.
%
% %%%%%%%%%%%%%%%%%%%%%%%%%%%%%%%%%%%%%%
% \paragraph{Main File.}
%
% The main file is called |cdocsamp.tex|.
%
% Load the \textsf{childdoc} definitions and
% declare the filename for the main document:
%    \begin{macrocode}
\input{childdoc.def}
\childdocmain{}
%    \end{macrocode}

% Optional override for |\version| flag:
%    \begin{macrocode}
%%\ifchilddoc\else\providecommand{\version}{draft}\fi
%    \end{macrocode}

% Define the default values for the |\version| flag
% (|final| for the main file and |draft| for childs):
%    \begin{macrocode}
\ifchilddoc
\providecommand{\version}{draft}
\else
\providecommand{\version}{final}
\fi
%    \end{macrocode}

% Load the standard document class:
%    \begin{macrocode}
\documentclass[12pt]{article}
%    \end{macrocode}

% Start the document body:
%    \begin{macrocode}
\begin{document}
%    \end{macrocode}

% Declare a title page.
% Print title, part of document being processed and version flag:
%    \begin{macrocode}
\addtocounter{page}{-1}
\begin{center}
{\LARGE\bfseries{}childdoc example\par}
\vspace{1cm}
\ifchilddoc
\ifchilddocmanual part\else chapter\fi:
`\childdocname' of `\childdocjob'\par
\else
main document: `\childdocjob'\par
\fi
version: \version\par
\end{center}
\newpage
%    \end{macrocode}

% Manually include selected file,
% otherwise process as usual:
%    \begin{macrocode}
\ifchilddocmanual
\section*{part `\childdocname'}
\input{\childdocname}
\else
%    \end{macrocode}

% Include the two chapters:
%    \begin{macrocode}
\include{cdocsch1}
\include{cdocsch2}
%    \end{macrocode}

% Include the two parts unless only chapters should be displayed:
%    \begin{macrocode}
\ifchilddoc\else
\section{part three}
\input{cdocspt3}
\section{part four}
\input{cdocspt4}
\fi
%    \end{macrocode}

% Process as usual until here:
%    \begin{macrocode}
\fi
%    \end{macrocode}

% End of document body:
%    \begin{macrocode}
\end{document}
%    \end{macrocode}
%\iffalse
%</samplemain>
%\fi
%
% %%%%%%%%%%%%%%%%%%%%%%%%%%%%%%%%%%%%%%
% \paragraph{Chapter Include Files.}
%
% The include files are called |cdocsch1.tex| and |cdocsch2.tex|.
%
%\iffalse
%<*samplechap1|samplechap2>
%\fi

% Optional override for |\version| flag:
%    \begin{macrocode}
%%\providecommand{\version}{final}
%    \end{macrocode}

% Include the main document:
%    \begin{macrocode}
\input{childdoc.def}
\childdocof{cdocsamp}
%    \end{macrocode}

%\iffalse
%</samplechap1|samplechap2>
%\fi
%
%\iffalse
%<*samplechap1>
%\fi
% Some text for chapter 1:
%    \begin{macrocode}
\section{one}
some text in chapter one
%    \end{macrocode}

%\iffalse
%</samplechap1>
%\fi
% Some text for chapter 2:
%\iffalse
%<*samplechap2>
%\fi
%    \begin{macrocode}
\section{two}
more text in chapter two
%    \end{macrocode}

%\iffalse
%</samplechap2>
%\fi
%
% %%%%%%%%%%%%%%%%%%%%%%%%%%%%%%%%%%%%%%
% \paragraph{Part Include Files.}
%
% The include files are called |cdocspt3.tex| and |cdocspt4.tex|.
%
%\iffalse
%<*samplepart3|samplepart4>
%\fi

% Optional override for |\version| flag:
%    \begin{macrocode}
%%\providecommand{\version}{final}
%    \end{macrocode}

% Include the main document:
%    \begin{macrocode}
\input{childdoc.def}
\childdocby{cdocsamp}
%    \end{macrocode}

%\iffalse
%</samplepart3|samplepart4>
%\fi
%
%\iffalse
%<*samplepart3>
%\fi
% Some text for part 3:
%    \begin{macrocode}
some text in part three
%    \end{macrocode}

%\iffalse
%</samplepart3>
%\fi
% Some text for part 4:
%\iffalse
%<*samplepart4>
%\fi
%    \begin{macrocode}
more text in part four
%    \end{macrocode}

%\iffalse
%</samplepart4>
%\fi
%
% %%%%%%%%%%%%%%%%%%%%%%%%%%%%%%%%%%%%%%
% \paragraph{Forwarding for a Complete Draft.}
%
% The following forwarding file |cdocsdrf.tex|
% compiles the main document in draft mode:
%\iffalse
%<*sampledraft>
%\fi
%    \begin{macrocode}
\def\version{draft}
\input{childdoc.def}
\childdocforward{cdocsamp}
%    \end{macrocode}

%\iffalse
%</sampledraft>
%\fi
%
% %%%%%%%%%%%%%%%%%%%%%%%%%%%%%%%%%%%%%%
% \paragraph{Forwarding for Final Version of the Chapters.}
%
% The following forwarding files |cdocsfn1.tex| and |cdocsfn2.tex|
% (with identical content)
% compile the final versions of the child documents
% |cdocsch1.tex| and |cdocsch2.tex|, respectively:
%\iffalse
%<*samplefinal>
%\fi
%    \begin{macrocode}
\def\version{final}
\input{childdoc.def}
\childdocforwardprefix[cdocsamp]{cdocsfn}{cdocsch}
%    \end{macrocode}

%\iffalse
%</samplefinal>
%\fi
%
% %%%%%%%%%%%%%%%%%%%%%%%%%%%%%%%%%%%%%%
% \paragraph{Command Line Processing.}
%
% The following three command lines generate the output files
% |cdocscld|, |cdocscl1| and |cdocscl2|
% which should be identical to
% |cdocsdrf|, |cdocsch1| and |cdocsfn2|, respectively:
% \begin{center}
% \begin{tabular}{l}
% |latex -jobname cdocscld \|\\
% |  "\def\version{draft}\input{childdoc.def}\childdocforward{cdocsamp}"|\\
% |latex -jobname cdocscl1 \|\\
% |  "\input{childdoc.def}\childdocforward[cdocsamp]{cdocsch1}"|\\
% |latex -jobname cdocscl2 \|\\
% |  "\def\version{final}\input{childdoc.def}\childdocforward{cdocsch2}"|
% \end{tabular}
% \end{center}
% Note that the trailing backslash on each first line
% merely continues the input to the second line
% (for convenient cut ant paste).
% Furthermore, the command |latex| can be replaced by any
% of its alternative versions such as |pdflatex|.
%
% %%%%%%%%%%%%%%%%%%%%%%%%%%%%%%%%%%%%%%%%%%%%%%%%%%%%%%%%%%%%%%%%%%%%%%%%%%%%%%
% %%%%%%%%%%%%%%%%%%%%%%%%%%%%%%%%%%%%%%%%%%%%%%%%%%%%%%%%%%%%%%%%%%%%%%%%%%%%%%
% \section{Implementation}
%\iffalse
%<*package>
%\fi
%
% This section describes the definitions file |childdoc.def|.

% The definitions cannot be loaded using |\usepackage| or |\RequirePackage|
% which has a mechanism to prevent loading a style file more than once.
% When loading the definitions by means of |\input|
% multiple instances have to be prevented manually:
%\iffalse
%This code needs to be before the `\ProvidesFile' directive
%which is defined at the beginning of this file.
%Therefore it is also placed there and commented out here.
%</package>
%<*discard>
%\fi
%    \begin{macrocode}
\ifdefined\childdocmain\endinput\fi
%    \end{macrocode}
%\iffalse
%</discard>
%<*package>
%\fi
%
% \macro{\ifchilddoc}
% \macro{\ifchilddocmanual}
% The conditional |\ifchilddoc| tells whether a
% child (true) or main (false) document is being compiled.
% The conditional |\ifchilddocmanual| tells whether
% the |\includeonly| mechanism is used (false) or
% the selection of child files must be performed manually (true).
% The definitions initialise to false:
%    \begin{macrocode}
\newif\ifchilddoc
\newif\ifchilddocmanual
%    \end{macrocode}

% \macro{\childdocname}
% \macro{\childdocjob}
% The macro |\childdocname| stores the name of the main document
% to be compiled. The macro |\childdocjob| stores the name of
% the document on which the \LaTeX{} compiler was originally invoked.
% The content of |\jobname| cannot be compared
% to filenames specified in the source due to different catcodes.
% The following code rescans |\jobname|, stores the result
% in |\childdocname| and saves a copy in |\childdocjob|:
%    \begin{macrocode}
\edef\childdocname{\scantokens\expandafter{\jobname\noexpand}}
\let\childdocjob\childdocname
%    \end{macrocode}

% \macro{\childdocdisable}
% The macro |\childdocdisable| prevents the main file
% from being processed more than once.
% At this stage, the main document command |\childdocmain|
% is assumed to be called once again where it should do nothing.
% Any subsequent call to it should prevent
% a secondary processing of the main document
% It overwrites the forwarding commands
% |\childdocof| and |\childdocforward|
% with empty macros to prevent further inclusions of the main document:
%    \begin{macrocode}
\newcommand{\childdocdisable}
{
  \renewcommand{\childdocmain}[1]{\renewcommand{\childdocmain}[1]{\endinput}}
  \renewcommand{\childdocof}[1]{}
  \renewcommand{\childdocby}[2][]{}
  \renewcommand{\childdocforward}[2][]{}
  \renewcommand{\childdocdisable}{}
}
%    \end{macrocode}

% \macro{\childdocmain}
% The macro |\childdocmain| is to be called at the top of the main file
% with nothing or the main filename (without extension) as argument.
% First, it breaks loops.
% If the argument is not empty and does not match |\childdocname|
% (which is set by the first inclusion of |childdoc.def|),
% |\ifchilddoc| is set to true, |\includeonly| is applied to the child file
% and |\jobname| is set to the main file
% (for proper handling of |.aux| files):
%    \begin{macrocode}
\newcommand{\childdocmain}[1]
{
  \childdocdisable\childdocmain{}
  \if?#1?\else
    \begingroup
      \def\childdoctmp{#1}
      \ifx\childdoctmp\childdocname
        \def\childdoctmp{}
      \else
        \def\childdoctmp
        {
          \childdoctrue
          \includeonly{\childdocname}
          \def\childdocjob{#1}
          \def\jobname{#1}
        }
      \fi
      \expandafter
    \endgroup
    \childdoctmp
  \fi
}
%    \end{macrocode}

% \macro{\childdocof}
% The command |\childdocof| redirects
% compilation to the main file |#1|.
%    \begin{macrocode}
\newcommand{\childdocof}[1]
{
  \childdocdisable
  \childdoctrue
  \includeonly{\childdocname}
  \def\jobname{#1}
  \def\childdocjob{#1}
  \input{#1}
}
%    \end{macrocode}

% \macro{\childdocby}
% The command |\childdocby| ....
%    \begin{macrocode}
\newcommand{\childdocby}[2][]
{
  \childdocdisable
  \childdoctrue
  \childdocmanualtrue
  \if?#1?\else
    \def\jobname{#2}
  \fi
  \def\childdocjob{#2}
  \input{#2}
  \endinput
}
%    \end{macrocode}

% \macro{\childdocforward}
% The command |\childdocforward| redirects
% compilation to the main file or
% (if the optional argument is given) a child file.
% Parameters are set as if the main file
% or a child file starting with |\childdocof| was compiled.
% Then compilation is handed over to the main file:
%    \begin{macrocode}
\newcommand{\childdocforward}[2][]
{
  \begingroup
    \if?#1?
      \def\childdoctmp
      {
        \def\childdocname{#2}
        \def\childdocjob{#2}
        \def\jobname{#2}
        \input{#2}
        \endinput
      }
    \else
      \def\childdoctmp
      {
        \childdocdisable
        \def\childdocname{#2}
        \childdoctrue
        \includeonly{#2}
        \def\childdocjob{#1}
        \def\jobname{#1}
        \input{#1}
        \endinput
      }
    \fi
    \expandafter
  \endgroup
  \childdoctmp
}
%    \end{macrocode}

% \macro{\childdocforwardprefix}
% The command |\childdocforwardprefix| redirects
% compilation to the main or a child file by means of a pattern.
% The prefix |#1| in the current filename is replaced by |#2|
% and the suffix of the current filename is kept
% (it is assumed that the filename does not contain the substring `|~~~|'
% which is used as a delimiter).
% Compilation is handed over to the new file by |\childdocforward|:
%    \begin{macrocode}
\newcommand{\childdocforwardprefix}[3][]
{
  \begingroup
    \def\childdocextract #2##1~~~{\def\childdoctmp{\childdocforward[#1]{#3##1}}}
    \expandafter\childdocextract\childdocname~~~
    \expandafter
  \endgroup
  \childdoctmp
}
%    \end{macrocode}

% \macro{\childdoc}
% The deprecated macro |\childdoc| is a legacy version of |\childdocmain|:
%    \begin{macrocode}
\newcommand{\childdoc}{\childdocmain}
%    \end{macrocode}

% \macro{\childdocredirect}
% The deprecated macro |\childdocredirect| is a legacy version
% of |\childdocforward| and |\childdocforwardprefix|:
%    \begin{macrocode}
\newcommand{\childdocredirect}[2][]
{
  \begingroup
    \if?#1?
      \def\childdoctmp{\childdocforward{#2}}
    \else
      \def\childdoctmp{\childdocforwardprefix{#1}{#2}}
    \fi
    \expandafter
  \endgroup
  \childdoctmp
}
%    \end{macrocode}

%\iffalse
%</package>
%\fi
%
\endinput

\childdocforward{cdocsamp}
%    \end{macrocode}

%\iffalse
%</sampledraft>
%\fi
%
% %%%%%%%%%%%%%%%%%%%%%%%%%%%%%%%%%%%%%%
% \paragraph{Forwarding for Final Version of the Chapters.}
%
% The following forwarding files |cdocsfn1.tex| and |cdocsfn2.tex|
% (with identical content)
% compile the final versions of the child documents
% |cdocsch1.tex| and |cdocsch2.tex|, respectively:
%\iffalse
%<*samplefinal>
%\fi
%    \begin{macrocode}
\def\version{final}
% \iffalse
%
% childdoc.dtx Copyright (C) 2017-2018 Niklas Beisert
%
% This work may be distributed and/or modified under the
% conditions of the LaTeX Project Public License, either version 1.3
% of this license or (at your option) any later version.
% The latest version of this license is in
%   http://www.latex-project.org/lppl.txt
% and version 1.3 or later is part of all distributions of LaTeX
% version 2005/12/01 or later.
%
% This work has the LPPL maintenance status `maintained'.
%
% The Current Maintainer of this work is Niklas Beisert.
%
% This work consists of the files childdoc.dtx and childdoc.ins
% and the derived files childdoc.def and cdocsamp.tex with
% cdocsch1.tex, cdocsch2.tex, cdocsdrf.tex, cdocsfn1.tex, cdocsfn2.tex.
%
%<package>\ifdefined\childdocmain\endinput\fi
%<package>\ProvidesFile{childdoc.def}[2018/12/30 v2.0 child document driver]
%<samplemain>\ProvidesFile{cdocsamp.tex}[2018/12/30 v2.0 sample for childdoc]
%<*driver>
%\ProvidesFile{childdoc.drv}[2018/12/30 v2.0 childdoc reference manual file]
\PassOptionsToClass{10pt,a4paper}{article}
\documentclass{ltxdoc}

\usepackage[margin=35mm]{geometry}
\usepackage{hyperref}
\usepackage{hyperxmp}
\usepackage[usenames]{color}

\hypersetup{colorlinks=true}
\hypersetup{pdfstartview=FitH}
\hypersetup{pdfpagemode=UseNone}
\hypersetup{pdfsource={}}
\hypersetup{pdflang={en-UK}}
\hypersetup{pdfcopyright={Copyright 2017-2018 Niklas Beisert.
  This work may be distributed and/or modified under the
  conditions of the LaTeX Project Public License, either version 1.3
  of this license or (at your option) any later version.}}
\hypersetup{pdflicenseurl={http://www.latex-project.org/lppl.txt}}
\hypersetup{pdfcontactaddress={ETH Zurich, ITP, HIT K,
  Wolfgang-Pauli-Strasse 27}}
\hypersetup{pdfcontactpostcode={8093}}
\hypersetup{pdfcontactcity={Zurich}}
\hypersetup{pdfcontactcountry={Switzerland}}
\hypersetup{pdfcontactemail={nbeisert@itp.phys.ethz.ch}}
\hypersetup{pdfcontacturl={http://people.phys.ethz.ch/\xmptilde nbeisert/}}

\newcommand{\secref}[1]{\hyperref[#1]{section \ref*{#1}}}

\parskip1ex
\parindent0pt
\let\olditemize\itemize
\def\itemize{\olditemize\parskip0pt}

\begin{document}

\title{The \textsf{childdoc} Package}
\hypersetup{pdftitle={The childdoc Package}}
\author{Niklas Beisert\\[2ex]
  Institut f\"ur Theoretische Physik\\
  Eidgen\"ossische Technische Hochschule Z\"urich\\
  Wolfgang-Pauli-Strasse 27, 8093 Z\"urich, Switzerland\\[1ex]
  \href{mailto:nbeisert@itp.phys.ethz.ch}
  {\texttt{nbeisert@itp.phys.ethz.ch}}}
\hypersetup{pdfauthor={Niklas Beisert}}
\hypersetup{pdfsubject={Manual for the LaTeX2e Package childdoc}}
\date{30 December 2018, \textsf{v2.0}}
\maketitle

\begin{abstract}\noindent
\textsf{childdoc} is a \LaTeXe{} package
that enables the direct compilation
of document sections included by |\include|
to individual files.
\end{abstract}

\begingroup
\parskip0ex
\tableofcontents
\endgroup

%%%%%%%%%%%%%%%%%%%%%%%%%%%%%%%%%%%%%%%%%%%%%%%%%%%%%%%%%%%%%%%%%%%%%%%%%%%%%%%%
%%%%%%%%%%%%%%%%%%%%%%%%%%%%%%%%%%%%%%%%%%%%%%%%%%%%%%%%%%%%%%%%%%%%%%%%%%%%%%%%
\section{Introduction}

\LaTeX{} provides a mechanism to structure a large document (such as a book)
into a main file and several child files (containing the chapters)
using the |\include| command.
This mechanism is beneficial for documents
which span hundreds of pages in order to
make the source file(s) more manageable.
Moreover, compilation can be restricted to
selected child files by means of the |\includeonly| command.
The latter feature can be used to reduce the compilation time while editing
(this was significantly more useful in the earlier days of \LaTeX{})
or to generate a smaller document which is easier to navigate.
Another application of |\includeonly| is to generate
documents consisting of selected parts of the complete document.

However, there are a few drawbacks of the plain |\include| mechanism:
\begin{itemize}
\item
The child files cannot be compiled on their own,
they can only be compiled via the main file.
A naive editing environment
(such as a text editor with an option
to have the current file processed by \LaTeX)
may require one to switch to the main file before compiling;
attempting to compile the child file produces errors.
\item
The main file must be modified (each time)
to adjust the |\includeonly| command
to the present needs. This easily leaves the main file in a messy state.
\item
The generated document will always carry the filename
of the main document. This is inconvenient if
several child files are to be compiled and
to be kept for distribution.
\end{itemize}

The present package provides a simple interface
to make child files individually compilable by \LaTeX{}.
Compiling a child file then has the same effect as compiling
the main file with an |\includeonly| command
to select the appropriate child.
Moreover the generated document will carry the name of the child
rather than the main file.
This resolves all three above issues.

This feature is meant to make the editing of books,
thesis documents and lecture notes somewhat more convenient.
However, the package can also be used efficiently for
composing a series of documents (such as exercise sheets)
which are typically distributed individually.
It then assists the author in generating the individual documents
(potentially in different versions)
as well as a document containing the collected series.
Another application is in developing style files
or other kinds of included material
where compilation of the style file could redirect
to a sample or test file.

%%%%%%%%%%%%%%%%%%%%%%%%%%%%%%%%%%%%%%%%%%%%%%%%%%%%%%%%%%%%%%%%%%%%%%%%%%%%%%%%
%%%%%%%%%%%%%%%%%%%%%%%%%%%%%%%%%%%%%%%%%%%%%%%%%%%%%%%%%%%%%%%%%%%%%%%%%%%%%%%%
\section{Usage}

First of all, the package \textsf{childdoc} is \emph{not} a standard
\LaTeXe{} |.sty| style file! Therefore it needs to be invoked in
a non-standard way.

%%%%%%%%%%%%%%%%%%%%%%%%%%%%%%%%%%%%%%%%%%%%%%%%%%%%%%%%%%%%%%%%%%%%%%%%%%%%%%%%
\subsection{Included Files}
\label{sec:include}

%%%%%%%%%%%%%%%%%%%%%%%%%%%%%%%%%%%%%%%%
\DescribeMacro{\childdocmain}
To use the package, add the commands
\begin{center}
\begin{tabular}{l}
|\input{childdoc.def}|\\
|\childdocmain{}|\\
\end{tabular}
\end{center}
at the very top of the main \LaTeX{} file,
in particular \emph{before} the |\documentclass| statement!
The argument of |\childdocmain| should be left empty
(but it must be present).

%%%%%%%%%%%%%%%%%%%%%%%%%%%%%%%%%%%%%%%%
\DescribeMacro{\childdocof}
Furthermore, add the commands
\begin{center}
\begin{tabular}{l}
|\input{childdoc.def}|\\
|\childdocof{|\textit{main}|}|\\
\end{tabular}
\end{center}
at the top of every child file \textit{child}
which is included by |\include{|\textit{child}|}|
from within the main file
(or at least for those files to be compiled individually).
The argument \textit{main} must be the filename of the main file.

There are a couple of
considerations in setting up the main and child documents:

%%%%%%%%%%%%%%%%%%%%%%%%%%%%%%%%%%%%%%%%
\paragraph{Restrictions.}

Please note the following restrictions:
\begin{itemize}
\item
|\childdocmain| must be called with one argument \textit{main}
to ensure compatibility with earlier version of the package.
It must either be empty (|\childdocmain{}|)
or precisely match the filename of the main file in which it is specified.
See \secref{sec:detection} for further information.
\item
The filename \textit{main} must be specified without the |.tex| extension.
\item
The filename \textit{main} is case sensitive
(even in case-insensitive file systems)
due to internal string comparison.
\item
The argument \textit{main} should be fully expanded, it cannot be a macro.
\item
Subdirectories and special characters should be avoided in filenames.
\item
The command |\childdocmain{|\textit{main}|}| must be followed by a whitespace.
It should not be followed immediately by another command
or by a comment mark `|%|'.
This is because the \TeX{} parser reads the token immediately following
the argument of |\childdocmain| and puts it
at the beginning of every child section;
however, a white\-space is ignored.
\end{itemize}

%%%%%%%%%%%%%%%%%%%%%%%%%%%%%%%%%%%%%%%%
\paragraph{Content of Main File.}

It is advisable to place all content in the child files included by |\include|.
Any output contained in the main file will appear in all child documents
unless suppressed manually;
it cannot be suppressed automatically by the |\includeonly| directive
and thus should normally be avoided.
A method to include some content in the main file
by means of conditional processing is described in \secref{sec:conditional}.

%%%%%%%%%%%%%%%%%%%%%%%%%%%%%%%%%%%%%%%%
\paragraph{Page Numbering.}

When only a part of the document is compiled,
the appropriate numbering of pages
(as well as other status parameters)
is determined from the |.aux| files.
The latter contain information from previous passes.
However this information needs to propagate through
all intermediate child documents.
Therefore the page numbering in child documents may well
be inconsistent until the complete document is compiled at least once.

A useful (if unconventional) way to always ensure a consistent
page numbering is to restart the numbering in each child document
and denote the pages by `\textit{child}|.|\textit{page}'
where \textit{child} represents the chapter/section number of the child file.
This can be achieved by the command
|\numberwithin{page}{|\textit{child}|}|
of the \textsf{amsmath} package
where \textit{child} can be |chapter| or |section|
depending on the chosen structuring.
Alternatively, one can modify the macro |\thepage| appropriately
and reset the counter |page| at the start of each child file.

%%%%%%%%%%%%%%%%%%%%%%%%%%%%%%%%%%%%%%%%%%%%%%%%%%%%%%%%%%%%%%%%%%%%%%%%%%%%%%%%
\subsection{Conditional Processing}
\label{sec:conditional}

The package provides a mechanism to compile different versions
of a document. To customise the versions further some conditional processing
can come in handy to distinguish which version is being compiled.
The package provides two macros to describe the compilation context:

%%%%%%%%%%%%%%%%%%%%%%%%%%%%%%%%%%%%%%%%
\DescribeMacro{\ifchilddoc}
The conditional |\ifchilddoc| distinguishes between the compilation of
child documents and the main document:
%
\begin{center}
|\ifchilddoc |\textit{child-code}| |[|\||else |\textit{main-code}]| \||fi|
\end{center}

%%%%%%%%%%%%%%%%%%%%%%%%%%%%%%%%%%%%%%%%
\DescribeMacro{\childdocname}
\DescribeMacro{\childdocjob}
The macro |\childdocname| contains the filename (without extension)
of the main or child file being processed.
Note that |\childdocjob| will always contain the name of the main file.

%%%%%%%%%%%%%%%%%%%%%%%%%%%%%%%%%%%%%%%%
\paragraph{Title Page.}

Conditional processing can be used to include a title or banner page
in the main document when proper precautions are taken.
Importantly, the code in the main file should ensure that the page counter
(as well as other status parameters which are stored in the |.aux| files)
takes the same value after the conditional processing.
Otherwise the page numbers may take divergent values
depending on which part is compiled.

For example, a title page could be declared by:
%
\begin{center}
\begin{tabular}{l}
|\ifchilddoc\||else|\\
|\addtocounter{page}{-1}|\\
\textit{code for title page}\\
|\newpage|\\
|\||fi|
\end{tabular}
\end{center}
%
A banner page for the child documents can be generated by:
%
\begin{center}
\begin{tabular}{l}
|\ifchilddoc|\\
|\addtocounter{page}{-1}|\\
\textit{code for banner page}\\
|\newpage|\\
|\||fi|
\end{tabular}
\end{center}
%
Here one could write a message such as:
\begin{center}
|This is the part \childdocname{} of \childdocjob{}.|
\end{center}

%%%%%%%%%%%%%%%%%%%%%%%%%%%%%%%%%%%%%%%%%%%%%%%%%%%%%%%%%%%%%%%%%%%%%%%%%%%%%%%%
\subsection{Flags}
\label{sec:flags}

The package makes it easy to generate different versions
of the main or child documents.
To this end compilation flags can be defined
and assigned different default values.
They will be particularly useful in conjunction
with the forwarding mechanism described in \secref{sec:forward}.

For example, it may be useful to have a flag |\version|
which can be set to |draft| or |final|.
The document source will contain some conditional code
depending on the value of |\version|.
Suppose further, the flag should default to |final| for the main file
and to |draft| for child files
which is a natural assignment for editing the document.
This is achieved by placing the following code
in the preamble of the main document
(below the |\childdocmain| directive):
%
\begin{center}
\begin{tabular}{l}
|\ifchilddoc|\\
|\providecommand{\version}{draft}|\\
|\||else|\\
|\providecommand{\version}{final}|\\
|\||fi|
\end{tabular}
\end{center}
%
The definition by |\providecommand| makes sure
that previous definitions are not overwritten.
Further statements |\providecommand{\version}{...}|
can thus be added before the above code to override it.

For the main file, one might add a line
(between |\childdocmain| and the above block)
%
\begin{center}
|%\ifchilddoc\||else\providecommand{\version}{draft}\||fi|
\end{center}
%
which can be uncommented to produce a draft version.
Likewise one can add a line to the very top of a child file
(above the |\childdocof{|\textit{main}|}| directive)
%
\begin{center}
|%\providecommand{\version}{final}|
\end{center}
%
which can be uncommented to produce the final version of this child document.

%%%%%%%%%%%%%%%%%%%%%%%%%%%%%%%%%%%%%%%%%%%%%%%%%%%%%%%%%%%%%%%%%%%%%%%%%%%%%%%%
\subsection{Forwarding}
\label{sec:forward}

Different versions of the main or child documents
using compilation flags as described in \secref{sec:flags}
can be (permanently) stored in different files
for convenient compilation, viewing and distribution.
To this end, the package defines a command
to pass on compilation to a different file:

%%%%%%%%%%%%%%%%%%%%%%%%%%%%%%%%%%%%%%%%
\DescribeMacro{\childdocforward}
The command |\childdocforward| redirects processing to
another source file:
%
\begin{center}
\begin{tabular}{l}
|\input{childdoc.def}|\\
|\childdocforward[|\textit{main}|]{|\textit{dest}|}|\\
\end{tabular}
\end{center}
%
The argument \textit{dest} is the destination file
(without extension).
It should be the main file or one of the child files.
Note that further \textsf{childdoc} directives
such as |\childdocof| and |\childdocforward|
in the indicated file will be processed in this form.
The optional argument \textit{main}
passes on directly to the main file \textit{main}
while pretending to compile the child \textit{dest}.
This form behaves as if \textit{dest}
issues |\childdocof{|\textit{main}|}| right away,
and no further \textsf{childdoc} directives will be processed.

%%%%%%%%%%%%%%%%%%%%%%%%%%%%%%%%%%%%%%%%
\DescribeMacro{\...prefix}
In the alternative form |\childdocforwardprefix|,
%
\begin{center}
\begin{tabular}{l}
|\input{childdoc.def}|\\
|\childdocforwardprefix[|\textit{main}|]{|\textit{prefix}|}{|\textit{dest}|}|
\end{tabular}
\end{center}
%
the destination file is determined by a pattern
depending on the current file:
To make this work, the current file must be called
`{\textit{prefix}\hspace{0.2em}\textit{suffix}}'
with \textit{prefix} matching precisely the argument.
Processing is then passed on to the file
`{\textit{dest}\hspace{0.2em}\textit{suffix}}'.
Surely, the same effect is achieved by
directly specifying the
argument `{\textit{dest}\hspace{0.2em}\textit{suffix}}'
in the first form.
However, that requires to set up a different file
for each child. With the alternative form of the command
all these files can have exactly the same content
which simplifies setting them up and maintaining them.

For example, the following file |draft.tex|
with a compilation flag |\version| as described in \secref{sec:flags}
compiles the main document as a draft:
%
\begin{center}
\begin{tabular}{l}
|\def\version{draft}|\\
|\input{childdoc.def}|\\
|\childdocforward{|\textit{main}|}|
\end{tabular}
\end{center}
%
Likewise, the following files |final|\textit{nn}|.tex|
compile the final version of the child document
|child|\textit{nn}|.tex|:
%
\begin{center}
\begin{tabular}{l}
|\def\version{final}|\\
|\input{childdoc.def}|\\
|\childdocforwardprefix{final}{child}|
\end{tabular}
\end{center}
%

Note that when several versions of a main file and/or of each child file
are to be generated, it may be convenient to set up a |Makefile| or
shell script to automatise the process.

%%%%%%%%%%%%%%%%%%%%%%%%%%%%%%%%%%%%%%%%%%%%%%%%%%%%%%%%%%%%%%%%%%%%%%%%%%%%%%%%
\subsection{Command Line Processing}
\label{sec:commandline}

The effect of redirection files can also be achieved by invoking
the \LaTeX{} compiler with a more elaborate command line.
Most conveniently this should be done as part
of a shell script or a |Makefile|.

When using \textsf{childdoc} in the main file, the following
command lines effectively perform a redirection
(note that depending on the shell being used,
backslashes may have to be doubled: `|\|' $\to$ `|\\|'):
%
\begin{center}
|... -jobname "|\textit{target}|" |\\|"|[\textit{flags}]%
|\input{childdoc.def}\childdocforward[|\textit{main}|]{|\textit{dest}|}"|
\end{center}
%
Here \textit{target} is the name of the output file,
\textit{main} is the name of the main file
and \textit{dest} is the name of the main or child file to be processed
(all filenames without extensions).
The optional argument \textit{main} can be omitted
if \textit{main} matches \textit{dest}.
Optionally, compilation \textit{flags} can be defined via |\def| commands.
This command line makes the \TeX{} engine believe
it is compiling the file \textit{target}
whose content is specified as the latter parameter.
The provided code then forwards the processing to
\textit{main} or \textit{dest} as described in \secref{sec:forward}.

%%%%%%%%%%%%%%%%%%%%%%%%%%%%%%%%%%%%%%%%%%%%%%%%%%%%%%%%%%%%%%%%%%%%%%%%%%%%%%%%
\subsection{Include by Input}
\label{sec:input}

Including child documents by |\include| has some restrictions by design.
Most notably, the content of a child document always occupies
its own set of pages; pages cannot be shared between child documents.
Usually, this behaviour makes perfect sense
because each child document contain an essential part of the document.
However, in some situations it may be desirable to compose
a document from a collection of parts
without having mandatory page breaks between then.
For this case, the package
provides a mechanism to include parts
by |\input| which can also be processed individually.
However, by construction this mechanism
requires manual handling of the content to be output.

%%%%%%%%%%%%%%%%%%%%%%%%%%%%%%%%%%%%%%%%
\DescribeMacro{\ifchilddocmanual}
The main file should be prepared as usual, see \secref{sec:include}.
However, the document body must make a distinction
between processing of an individual part and of the main document, e.g.:
%
\begin{center}
\begin{tabular}{l}
|\ifchilddocmanual|\\
|\input{\childdocname}|\\
|\||else|\\
\textit{document body with }|\input{|\textit{part}|}|\\
|\||fi|
\end{tabular}
\end{center}
%
The conditional |\ifchilddocmanual| is true whenever
a part to be included by |\input| is being compiled,
and the name of the part is stored in |\childdocname|.

%%%%%%%%%%%%%%%%%%%%%%%%%%%%%%%%%%%%%%%%
\DescribeMacro{\childdocby}
Each part to be included by |\input| should start with:
%
\begin{center}
\begin{tabular}{l}
|\input{childdoc.def}|\\
|\childdocby{|\textit{main}|}|\\
\end{tabular}
\end{center}
%
The directive |\childdocby| is similar to |\childdocof|
described in \secref{sec:include},
but the subsequent selection of content must be done manually.
To that end, both |\ifchilddoc| and |\ifchilddocmanual|
will be true upon processing of a part,
and the name of the part is stored in |\childdocname|.
Note that |\jobname| will be set to the filename of the current part
so that each part receives an individual |.aux| file
that does not interfere with the |.aux| file(s) of the main document.
This behaviour can be altered by the alternative form
|\childdocby[*]{|\textit{main}|}| (with a non-empty optional argument)
which uses the |.aux| file of the main document
by setting |\jobname| to \textit{main}.

%%%%%%%%%%%%%%%%%%%%%%%%%%%%%%%%%%%%%%%%%%%%%%%%%%%%%%%%%%%%%%%%%%%%%%%%%%%%%%%%
\subsection{Driver Development}
\label{sec:driver}

The \textsf{childdoc} mechanism can also be use for the development
of definition files such as \LaTeX{} styles or classes.
This case differs from the above setup with multiple parts
included by |\include| in that no |\includeonly| should be invoked.
This can be achieved by starting the include file
(before |\ProvidesPackage|) with:
%
\begin{center}
\begin{tabular}{l}
|\input{childdoc.def}|\\
|\childdocforward{|\textit{main}|}|\\
\end{tabular}
\end{center}
%
or alternatively with:
%
\begin{center}
\begin{tabular}{l}
|\input{childdoc.def}|\\
|\childdocby{|\textit{main}|}|\\
\end{tabular}
\end{center}
%
Both forms have slightly different effects as described above.
The main file is prepared as usual, see \secref{sec:include}.

%%%%%%%%%%%%%%%%%%%%%%%%%%%%%%%%%%%%%%%%%%%%%%%%%%%%%%%%%%%%%%%%%%%%%%%%%%%%%%%%
\subsection{Legacy Detection}
\label{sec:detection}

The directive |\childdocmain| in the main file can detect
whether the complete document or merely a child is to be compiled
even without using the directive |\childdocof|.
This method is deprecated because it is less robust
and there is no compelling reason to use it;
it is merely provided for backward compatibility
and it may be removed in future versions.

If the detection mechanism is to be used,
it is mandatory to correctly specify
the filename of the main file as the argument of |\childdocmain|:
%
\begin{center}
\begin{tabular}{l}
|\input{childdoc.def}|\\
|\childdocmain{|\textit{main}|}|\\
\end{tabular}
\end{center}
%
If |\jobname| does not match the argument \textit{main} of |\childdocmain|,
it is assumed that |\jobname| points to the child file to be compiled.
When using |\childdocmain| with the main file specified as argument,
it suffices to start a child file
with just |\input{|\textit{main}|}|
without loading of the package and using |\childdocof|.
If instead all processing is done
with the appropriate \textsf{childdoc} directives,
the argument of \textit{main} of |\childdocmain| can be empty.

An alternative version of the command line processing described
in \secref{sec:commandline} using the detection mechanism reads:
%
\begin{center}
|... -jobname "|\textit{target}|" "|[\textit{flags}]%
[|\def\jobname{|\textit{dest}|}|]|\input{|\textit{main}|}"|
\end{center}

%%%%%%%%%%%%%%%%%%%%%%%%%%%%%%%%%%%%%%%%%%%%%%%%%%%%%%%%%%%%%%%%%%%%%%%%%%%%%%%%
\subsection{Manual Code}
\label{sec:manual}

In case one cannot be certain whether the definitions file |childdoc.def|
is installed on the target \TeX{} distribution
and one prefers not to ship it,
it is conceivable to paste a few relevant commands into the sources.

To that end, drop all statements |\input{childdoc.def}|
and perform the replacements as outlined below.
Instead of |\childdocmain{|\textit{main}|}| add the following code
to the top of the main file:
%
\begin{center}
\begin{tabular}{l}
|\||ifdefined\childdocname\endinput\||fi\newif\ifchilddoc|\\
|\edef\childdocname{\scantokens\expandafter{\jobname\noexpand}}|\\
|\def\childdocmain{|\textit{main}|}\||ifx\childdocmain\childdocname\||else|\\
|\childdoctrue\includeonly{\childdocname}\let\jobname\childdocmain\||fi|\\
\end{tabular}
\end{center}
%
Instead of |\childdocof{|\textit{main}|}| just include the main file
at the top of each child file:
%
\begin{center}
|\input{|\textit{main}|}|
\end{center}
%
A simple redirection |\childdocforward{|\textit{dest}|}| is achieved by:
%
\begin{center}
|\def\jobname{|\textit{dest}|}\input{\jobname}|
\end{center}
%
The redirection with prefix
|\childdocforwardprefix[|\textit{prefix}|]{|\textit{dest}|}|
is accomplished by:
%
\begin{center}
\begin{tabular}{l}
|{\edef\jobname{\scantokens\expandafter{\jobname\noexpand}}|\\
|\def\redirectjob |\textit{prefix}|#1~~~{\gdef\jobname{|\textit{dest}|#1}}|\\
|\expandafter\redirectjob\jobname~~~}\input{\jobname}|
\end{tabular}
\end{center}

In an alternative approach,
child documents can be compiled by a specific command line
without additional code or specific definitions:
%
\begin{center}
|... -jobname "|\textit{target}|" "|[\textit{flags}]%
|\includeonly{|\textit{dest}|}\input{|\textit{main}|}"|
\end{center}
%

%%%%%%%%%%%%%%%%%%%%%%%%%%%%%%%%%%%%%%%%%%%%%%%%%%%%%%%%%%%%%%%%%%%%%%%%%%%%%%%%
%%%%%%%%%%%%%%%%%%%%%%%%%%%%%%%%%%%%%%%%%%%%%%%%%%%%%%%%%%%%%%%%%%%%%%%%%%%%%%%%
\section{Information}

%%%%%%%%%%%%%%%%%%%%%%%%%%%%%%%%%%%%%%%%%%%%%%%%%%%%%%%%%%%%%%%%%%%%%%%%%%%%%%%%
\subsection{Copyright}

Copyright \copyright{} 2017--2018 Niklas Beisert

This work may be distributed and/or modified under the
conditions of the \LaTeX{} Project Public License, either version 1.3
of this license or (at your option) any later version.
The latest version of this license is in
  \url{http://www.latex-project.org/lppl.txt}
and version 1.3 or later is part of all distributions of \LaTeX{}
version 2005/12/01 or later.

This work has the LPPL maintenance status `maintained'.

The Current Maintainer of this work is Niklas Beisert.

This work consists of the files |README.txt|, |childdoc.ins| and |childdoc.dtx|
as well as the derived files |childdoc.def|, |cdocsamp.tex|
with |cdocsch1.tex|, |cdocsch2.tex|, |cdocspt3.tex|, |cdocspt4.tex|,
|cdocsdrf.tex|, |cdocsfn1.tex|, |cdocsfn2.tex|
as well as |childdoc.pdf|.

%%%%%%%%%%%%%%%%%%%%%%%%%%%%%%%%%%%%%%%%%%%%%%%%%%%%%%%%%%%%%%%%%%%%%%%%%%%%%%%%
\subsection{Files and Installation}

The package consists of the files:
%
\begin{center}
\begin{tabular}{ll}
    |README.txt|   & readme file \\
    |childdoc.ins| & installation file \\
    |childdoc.dtx| & source file \\
    |childdoc.def| & definition file \\
    |cdocsamp.tex| & sample main file \\
    |cdocsch1.tex| & sample include file \\
    |cdocsch2.tex| & sample include file \\
    |cdocspt3.tex| & sample part file \\
    |cdocspt4.tex| & sample part file \\
    |cdocsdrf.tex| & sample redirection file \\
    |cdocsfn1.tex| & sample redirection file \\
    |cdocsfn2.tex| & sample redirection file \\
    |childdoc.pdf| & manual
\end{tabular}
\end{center}
%
The distribution consists of the files
|README.txt|, |childdoc.ins| and |childdoc.dtx|.
%
\begin{itemize}
\item
Run (pdf)\LaTeX{} on |childdoc.dtx|
to compile the manual |childdoc.pdf| (this file).
\item
Run \LaTeX{} on |childdoc.ins| to create the definitions file |childdoc.def|
and the sample |cdocsamp.tex| with include files
|cdocsch1.tex|, |cdocsch2.tex|, |cdocspt3.tex|, |cdocspt4.tex|,
|cdocsdrf.tex|, |cdocsfn1.tex|, |cdocsfn2.tex|.
Then copy the file |childdoc.def| to an appropriate directory of your \LaTeX{}
distribution, e.g.\ \textit{texmf-root}|/tex/latex/childdoc|.
\end{itemize}

%%%%%%%%%%%%%%%%%%%%%%%%%%%%%%%%%%%%%%%%%%%%%%%%%%%%%%%%%%%%%%%%%%%%%%%%%%%%%%%%
\subsection{Related CTAN Packages}

There are several other packages which offer a similar functionality:
%
\begin{itemize}
\item
The packages
\href{http://ctan.org/pkg/docmute}{\textsf{docmute}},
\href{http://ctan.org/pkg/includex}{\textsf{includex}} and
\href{http://ctan.org/pkg/standalone}{\textsf{standalone}}
provide commands to include only the document body of
a child file thus allowing both files to be compiled individually.
\item
The packages \href{http://ctan.org/pkg/subdocs}{\textsf{subdocs}}
and \href{http://ctan.org/pkg/subfiles}{\textsf{subfiles}}
provide structures in which the main and child documents can be
encapsulated and allowing them to be compiled individually.
The inclusion mechanism is different from the conventional |\include|.
\item
The package \href{http://ctan.org/pkg/combine}{\textsf{combine}}
is an elaborate solution to combine several documents into one.
\end{itemize}
%
See also the CTAN topic \href{http://ctan.org/topic/subdocs}{\textsf{subdocs}}
for further related packages.
The present package differs from the above solutions in that
a document structure constructed with the conventional |\include| mechanism
just needs two extra commands at the top of every file
such that all constituent files can be compiled individually.

%%%%%%%%%%%%%%%%%%%%%%%%%%%%%%%%%%%%%%%%%%%%%%%%%%%%%%%%%%%%%%%%%%%%%%%%%%%%%%%%
%\subsection{Feature Suggestions}
%
%The following is a list of features which may be useful for future
%versions of this package:
%%
%\begin{itemize}
%\item
%\ldots
%\end{itemize}

%%%%%%%%%%%%%%%%%%%%%%%%%%%%%%%%%%%%%%%%%%%%%%%%%%%%%%%%%%%%%%%%%%%%%%%%%%%%%%%%
\subsection{Revision History}

%%%%%%%%%%%%%%%%%%%%%%%%%%%%%%%%%%%%%%%%
\paragraph{v2.0:} 2018/12/30

\begin{itemize}
\item
immediate forward processing
\item
added |\childdocby| mechanism
\item
manual restructured
\end{itemize}

%%%%%%%%%%%%%%%%%%%%%%%%%%%%%%%%%%%%%%%%
\paragraph{v1.6:} 2018/01/17

\begin{itemize}
\item
application for development of include files
\item
corrections to manual
\end{itemize}

%%%%%%%%%%%%%%%%%%%%%%%%%%%%%%%%%%%%%%%%
\paragraph{v1.5:} 2017/05/21

\begin{itemize}
\item
more complete structuring introduced
\item
|\childdocof| introduced
\item
|\childdoc| renamed to |\childdocmain|
\item
|\childredirect| renamed to |\childdocforward| and |\childdocforwardprefix|
and functionality expanded
\end{itemize}

%%%%%%%%%%%%%%%%%%%%%%%%%%%%%%%%%%%%%%%%
\paragraph{v1.0:} 2017/04/27

\begin{itemize}
\item
manual and install package
\item
first version published on CTAN
\end{itemize}

%%%%%%%%%%%%%%%%%%%%%%%%%%%%%%%%%%%%%%%%
\paragraph{v0.6:} 2017/04/26

\begin{itemize}
\item
redirection mechanism added
\end{itemize}

%%%%%%%%%%%%%%%%%%%%%%%%%%%%%%%%%%%%%%%%
\paragraph{v0.5:} 2017/04/26

\begin{itemize}
\item
functionality in definition file
\end{itemize}


%%%%%%%%%%%%%%%%%%%%%%%%%%%%%%%%%%%%%%%%%%%%%%%%%%%%%%%%%%%%%%%%%%%%%%%%%%%%%%%%
%%%%%%%%%%%%%%%%%%%%%%%%%%%%%%%%%%%%%%%%%%%%%%%%%%%%%%%%%%%%%%%%%%%%%%%%%%%%%%%%
%%%%%%%%%%%%%%%%%%%%%%%%%%%%%%%%%%%%%%%%%%%%%%%%%%%%%%%%%%%%%%%%%%%%%%%%%%%%%%%%
\appendix

\settowidth\MacroIndent{\rmfamily\scriptsize 000\ }

 \DocInput{childdoc.dtx}

\end{document}
%</driver>
% \fi
%
% %%%%%%%%%%%%%%%%%%%%%%%%%%%%%%%%%%%%%%%%%%%%%%%%%%%%%%%%%%%%%%%%%%%%%%%%%%%%%%
% %%%%%%%%%%%%%%%%%%%%%%%%%%%%%%%%%%%%%%%%%%%%%%%%%%%%%%%%%%%%%%%%%%%%%%%%%%%%%%
% \section{Sample}
%\iffalse
%<*samplemain>
%\fi
%
% The following presents a sample document
% with two chapters, two parts, a title page,
% a compile flag as well as three forwarding files to set the flag.
% It consists of eight |.tex| files:
% \begin{center}
% \begin{tabular}{ll}
% |cdocsamp.tex|&main file\\
% |cdocsch1.tex|&include file for chapter 1\\
% |cdocsch2.tex|&include file for chapter 2\\
% |cdocspt3.tex|&include file for part 3\\
% |cdocspt4.tex|&include file for part 4\\
% |cdocsdrf.tex|&forwarding file for main file in draft mode\\
% |cdocsfi1.tex|&forwarding file for final version of chapter 1\\
% |cdocsfi2.tex|&forwarding file for final version of chapter 2\\
% \end{tabular}
% \end{center}
% Each of the eight files can be compiled directly by the \LaTeX{} compiler.
%
% %%%%%%%%%%%%%%%%%%%%%%%%%%%%%%%%%%%%%%
% \paragraph{Main File.}
%
% The main file is called |cdocsamp.tex|.
%
% Load the \textsf{childdoc} definitions and
% declare the filename for the main document:
%    \begin{macrocode}
\input{childdoc.def}
\childdocmain{}
%    \end{macrocode}

% Optional override for |\version| flag:
%    \begin{macrocode}
%%\ifchilddoc\else\providecommand{\version}{draft}\fi
%    \end{macrocode}

% Define the default values for the |\version| flag
% (|final| for the main file and |draft| for childs):
%    \begin{macrocode}
\ifchilddoc
\providecommand{\version}{draft}
\else
\providecommand{\version}{final}
\fi
%    \end{macrocode}

% Load the standard document class:
%    \begin{macrocode}
\documentclass[12pt]{article}
%    \end{macrocode}

% Start the document body:
%    \begin{macrocode}
\begin{document}
%    \end{macrocode}

% Declare a title page.
% Print title, part of document being processed and version flag:
%    \begin{macrocode}
\addtocounter{page}{-1}
\begin{center}
{\LARGE\bfseries{}childdoc example\par}
\vspace{1cm}
\ifchilddoc
\ifchilddocmanual part\else chapter\fi:
`\childdocname' of `\childdocjob'\par
\else
main document: `\childdocjob'\par
\fi
version: \version\par
\end{center}
\newpage
%    \end{macrocode}

% Manually include selected file,
% otherwise process as usual:
%    \begin{macrocode}
\ifchilddocmanual
\section*{part `\childdocname'}
\input{\childdocname}
\else
%    \end{macrocode}

% Include the two chapters:
%    \begin{macrocode}
\include{cdocsch1}
\include{cdocsch2}
%    \end{macrocode}

% Include the two parts unless only chapters should be displayed:
%    \begin{macrocode}
\ifchilddoc\else
\section{part three}
\input{cdocspt3}
\section{part four}
\input{cdocspt4}
\fi
%    \end{macrocode}

% Process as usual until here:
%    \begin{macrocode}
\fi
%    \end{macrocode}

% End of document body:
%    \begin{macrocode}
\end{document}
%    \end{macrocode}
%\iffalse
%</samplemain>
%\fi
%
% %%%%%%%%%%%%%%%%%%%%%%%%%%%%%%%%%%%%%%
% \paragraph{Chapter Include Files.}
%
% The include files are called |cdocsch1.tex| and |cdocsch2.tex|.
%
%\iffalse
%<*samplechap1|samplechap2>
%\fi

% Optional override for |\version| flag:
%    \begin{macrocode}
%%\providecommand{\version}{final}
%    \end{macrocode}

% Include the main document:
%    \begin{macrocode}
\input{childdoc.def}
\childdocof{cdocsamp}
%    \end{macrocode}

%\iffalse
%</samplechap1|samplechap2>
%\fi
%
%\iffalse
%<*samplechap1>
%\fi
% Some text for chapter 1:
%    \begin{macrocode}
\section{one}
some text in chapter one
%    \end{macrocode}

%\iffalse
%</samplechap1>
%\fi
% Some text for chapter 2:
%\iffalse
%<*samplechap2>
%\fi
%    \begin{macrocode}
\section{two}
more text in chapter two
%    \end{macrocode}

%\iffalse
%</samplechap2>
%\fi
%
% %%%%%%%%%%%%%%%%%%%%%%%%%%%%%%%%%%%%%%
% \paragraph{Part Include Files.}
%
% The include files are called |cdocspt3.tex| and |cdocspt4.tex|.
%
%\iffalse
%<*samplepart3|samplepart4>
%\fi

% Optional override for |\version| flag:
%    \begin{macrocode}
%%\providecommand{\version}{final}
%    \end{macrocode}

% Include the main document:
%    \begin{macrocode}
\input{childdoc.def}
\childdocby{cdocsamp}
%    \end{macrocode}

%\iffalse
%</samplepart3|samplepart4>
%\fi
%
%\iffalse
%<*samplepart3>
%\fi
% Some text for part 3:
%    \begin{macrocode}
some text in part three
%    \end{macrocode}

%\iffalse
%</samplepart3>
%\fi
% Some text for part 4:
%\iffalse
%<*samplepart4>
%\fi
%    \begin{macrocode}
more text in part four
%    \end{macrocode}

%\iffalse
%</samplepart4>
%\fi
%
% %%%%%%%%%%%%%%%%%%%%%%%%%%%%%%%%%%%%%%
% \paragraph{Forwarding for a Complete Draft.}
%
% The following forwarding file |cdocsdrf.tex|
% compiles the main document in draft mode:
%\iffalse
%<*sampledraft>
%\fi
%    \begin{macrocode}
\def\version{draft}
\input{childdoc.def}
\childdocforward{cdocsamp}
%    \end{macrocode}

%\iffalse
%</sampledraft>
%\fi
%
% %%%%%%%%%%%%%%%%%%%%%%%%%%%%%%%%%%%%%%
% \paragraph{Forwarding for Final Version of the Chapters.}
%
% The following forwarding files |cdocsfn1.tex| and |cdocsfn2.tex|
% (with identical content)
% compile the final versions of the child documents
% |cdocsch1.tex| and |cdocsch2.tex|, respectively:
%\iffalse
%<*samplefinal>
%\fi
%    \begin{macrocode}
\def\version{final}
\input{childdoc.def}
\childdocforwardprefix[cdocsamp]{cdocsfn}{cdocsch}
%    \end{macrocode}

%\iffalse
%</samplefinal>
%\fi
%
% %%%%%%%%%%%%%%%%%%%%%%%%%%%%%%%%%%%%%%
% \paragraph{Command Line Processing.}
%
% The following three command lines generate the output files
% |cdocscld|, |cdocscl1| and |cdocscl2|
% which should be identical to
% |cdocsdrf|, |cdocsch1| and |cdocsfn2|, respectively:
% \begin{center}
% \begin{tabular}{l}
% |latex -jobname cdocscld \|\\
% |  "\def\version{draft}\input{childdoc.def}\childdocforward{cdocsamp}"|\\
% |latex -jobname cdocscl1 \|\\
% |  "\input{childdoc.def}\childdocforward[cdocsamp]{cdocsch1}"|\\
% |latex -jobname cdocscl2 \|\\
% |  "\def\version{final}\input{childdoc.def}\childdocforward{cdocsch2}"|
% \end{tabular}
% \end{center}
% Note that the trailing backslash on each first line
% merely continues the input to the second line
% (for convenient cut ant paste).
% Furthermore, the command |latex| can be replaced by any
% of its alternative versions such as |pdflatex|.
%
% %%%%%%%%%%%%%%%%%%%%%%%%%%%%%%%%%%%%%%%%%%%%%%%%%%%%%%%%%%%%%%%%%%%%%%%%%%%%%%
% %%%%%%%%%%%%%%%%%%%%%%%%%%%%%%%%%%%%%%%%%%%%%%%%%%%%%%%%%%%%%%%%%%%%%%%%%%%%%%
% \section{Implementation}
%\iffalse
%<*package>
%\fi
%
% This section describes the definitions file |childdoc.def|.

% The definitions cannot be loaded using |\usepackage| or |\RequirePackage|
% which has a mechanism to prevent loading a style file more than once.
% When loading the definitions by means of |\input|
% multiple instances have to be prevented manually:
%\iffalse
%This code needs to be before the `\ProvidesFile' directive
%which is defined at the beginning of this file.
%Therefore it is also placed there and commented out here.
%</package>
%<*discard>
%\fi
%    \begin{macrocode}
\ifdefined\childdocmain\endinput\fi
%    \end{macrocode}
%\iffalse
%</discard>
%<*package>
%\fi
%
% \macro{\ifchilddoc}
% \macro{\ifchilddocmanual}
% The conditional |\ifchilddoc| tells whether a
% child (true) or main (false) document is being compiled.
% The conditional |\ifchilddocmanual| tells whether
% the |\includeonly| mechanism is used (false) or
% the selection of child files must be performed manually (true).
% The definitions initialise to false:
%    \begin{macrocode}
\newif\ifchilddoc
\newif\ifchilddocmanual
%    \end{macrocode}

% \macro{\childdocname}
% \macro{\childdocjob}
% The macro |\childdocname| stores the name of the main document
% to be compiled. The macro |\childdocjob| stores the name of
% the document on which the \LaTeX{} compiler was originally invoked.
% The content of |\jobname| cannot be compared
% to filenames specified in the source due to different catcodes.
% The following code rescans |\jobname|, stores the result
% in |\childdocname| and saves a copy in |\childdocjob|:
%    \begin{macrocode}
\edef\childdocname{\scantokens\expandafter{\jobname\noexpand}}
\let\childdocjob\childdocname
%    \end{macrocode}

% \macro{\childdocdisable}
% The macro |\childdocdisable| prevents the main file
% from being processed more than once.
% At this stage, the main document command |\childdocmain|
% is assumed to be called once again where it should do nothing.
% Any subsequent call to it should prevent
% a secondary processing of the main document
% It overwrites the forwarding commands
% |\childdocof| and |\childdocforward|
% with empty macros to prevent further inclusions of the main document:
%    \begin{macrocode}
\newcommand{\childdocdisable}
{
  \renewcommand{\childdocmain}[1]{\renewcommand{\childdocmain}[1]{\endinput}}
  \renewcommand{\childdocof}[1]{}
  \renewcommand{\childdocby}[2][]{}
  \renewcommand{\childdocforward}[2][]{}
  \renewcommand{\childdocdisable}{}
}
%    \end{macrocode}

% \macro{\childdocmain}
% The macro |\childdocmain| is to be called at the top of the main file
% with nothing or the main filename (without extension) as argument.
% First, it breaks loops.
% If the argument is not empty and does not match |\childdocname|
% (which is set by the first inclusion of |childdoc.def|),
% |\ifchilddoc| is set to true, |\includeonly| is applied to the child file
% and |\jobname| is set to the main file
% (for proper handling of |.aux| files):
%    \begin{macrocode}
\newcommand{\childdocmain}[1]
{
  \childdocdisable\childdocmain{}
  \if?#1?\else
    \begingroup
      \def\childdoctmp{#1}
      \ifx\childdoctmp\childdocname
        \def\childdoctmp{}
      \else
        \def\childdoctmp
        {
          \childdoctrue
          \includeonly{\childdocname}
          \def\childdocjob{#1}
          \def\jobname{#1}
        }
      \fi
      \expandafter
    \endgroup
    \childdoctmp
  \fi
}
%    \end{macrocode}

% \macro{\childdocof}
% The command |\childdocof| redirects
% compilation to the main file |#1|.
%    \begin{macrocode}
\newcommand{\childdocof}[1]
{
  \childdocdisable
  \childdoctrue
  \includeonly{\childdocname}
  \def\jobname{#1}
  \def\childdocjob{#1}
  \input{#1}
}
%    \end{macrocode}

% \macro{\childdocby}
% The command |\childdocby| ....
%    \begin{macrocode}
\newcommand{\childdocby}[2][]
{
  \childdocdisable
  \childdoctrue
  \childdocmanualtrue
  \if?#1?\else
    \def\jobname{#2}
  \fi
  \def\childdocjob{#2}
  \input{#2}
  \endinput
}
%    \end{macrocode}

% \macro{\childdocforward}
% The command |\childdocforward| redirects
% compilation to the main file or
% (if the optional argument is given) a child file.
% Parameters are set as if the main file
% or a child file starting with |\childdocof| was compiled.
% Then compilation is handed over to the main file:
%    \begin{macrocode}
\newcommand{\childdocforward}[2][]
{
  \begingroup
    \if?#1?
      \def\childdoctmp
      {
        \def\childdocname{#2}
        \def\childdocjob{#2}
        \def\jobname{#2}
        \input{#2}
        \endinput
      }
    \else
      \def\childdoctmp
      {
        \childdocdisable
        \def\childdocname{#2}
        \childdoctrue
        \includeonly{#2}
        \def\childdocjob{#1}
        \def\jobname{#1}
        \input{#1}
        \endinput
      }
    \fi
    \expandafter
  \endgroup
  \childdoctmp
}
%    \end{macrocode}

% \macro{\childdocforwardprefix}
% The command |\childdocforwardprefix| redirects
% compilation to the main or a child file by means of a pattern.
% The prefix |#1| in the current filename is replaced by |#2|
% and the suffix of the current filename is kept
% (it is assumed that the filename does not contain the substring `|~~~|'
% which is used as a delimiter).
% Compilation is handed over to the new file by |\childdocforward|:
%    \begin{macrocode}
\newcommand{\childdocforwardprefix}[3][]
{
  \begingroup
    \def\childdocextract #2##1~~~{\def\childdoctmp{\childdocforward[#1]{#3##1}}}
    \expandafter\childdocextract\childdocname~~~
    \expandafter
  \endgroup
  \childdoctmp
}
%    \end{macrocode}

% \macro{\childdoc}
% The deprecated macro |\childdoc| is a legacy version of |\childdocmain|:
%    \begin{macrocode}
\newcommand{\childdoc}{\childdocmain}
%    \end{macrocode}

% \macro{\childdocredirect}
% The deprecated macro |\childdocredirect| is a legacy version
% of |\childdocforward| and |\childdocforwardprefix|:
%    \begin{macrocode}
\newcommand{\childdocredirect}[2][]
{
  \begingroup
    \if?#1?
      \def\childdoctmp{\childdocforward{#2}}
    \else
      \def\childdoctmp{\childdocforwardprefix{#1}{#2}}
    \fi
    \expandafter
  \endgroup
  \childdoctmp
}
%    \end{macrocode}

%\iffalse
%</package>
%\fi
%
\endinput

\childdocforwardprefix[cdocsamp]{cdocsfn}{cdocsch}
%    \end{macrocode}

%\iffalse
%</samplefinal>
%\fi
%
% %%%%%%%%%%%%%%%%%%%%%%%%%%%%%%%%%%%%%%
% \paragraph{Command Line Processing.}
%
% The following three command lines generate the output files
% |cdocscld|, |cdocscl1| and |cdocscl2|
% which should be identical to
% |cdocsdrf|, |cdocsch1| and |cdocsfn2|, respectively:
% \begin{center}
% \begin{tabular}{l}
% |latex -jobname cdocscld \|\\
% |  "\def\version{draft}% \iffalse
%
% childdoc.dtx Copyright (C) 2017-2018 Niklas Beisert
%
% This work may be distributed and/or modified under the
% conditions of the LaTeX Project Public License, either version 1.3
% of this license or (at your option) any later version.
% The latest version of this license is in
%   http://www.latex-project.org/lppl.txt
% and version 1.3 or later is part of all distributions of LaTeX
% version 2005/12/01 or later.
%
% This work has the LPPL maintenance status `maintained'.
%
% The Current Maintainer of this work is Niklas Beisert.
%
% This work consists of the files childdoc.dtx and childdoc.ins
% and the derived files childdoc.def and cdocsamp.tex with
% cdocsch1.tex, cdocsch2.tex, cdocsdrf.tex, cdocsfn1.tex, cdocsfn2.tex.
%
%<package>\ifdefined\childdocmain\endinput\fi
%<package>\ProvidesFile{childdoc.def}[2018/12/30 v2.0 child document driver]
%<samplemain>\ProvidesFile{cdocsamp.tex}[2018/12/30 v2.0 sample for childdoc]
%<*driver>
%\ProvidesFile{childdoc.drv}[2018/12/30 v2.0 childdoc reference manual file]
\PassOptionsToClass{10pt,a4paper}{article}
\documentclass{ltxdoc}

\usepackage[margin=35mm]{geometry}
\usepackage{hyperref}
\usepackage{hyperxmp}
\usepackage[usenames]{color}

\hypersetup{colorlinks=true}
\hypersetup{pdfstartview=FitH}
\hypersetup{pdfpagemode=UseNone}
\hypersetup{pdfsource={}}
\hypersetup{pdflang={en-UK}}
\hypersetup{pdfcopyright={Copyright 2017-2018 Niklas Beisert.
  This work may be distributed and/or modified under the
  conditions of the LaTeX Project Public License, either version 1.3
  of this license or (at your option) any later version.}}
\hypersetup{pdflicenseurl={http://www.latex-project.org/lppl.txt}}
\hypersetup{pdfcontactaddress={ETH Zurich, ITP, HIT K,
  Wolfgang-Pauli-Strasse 27}}
\hypersetup{pdfcontactpostcode={8093}}
\hypersetup{pdfcontactcity={Zurich}}
\hypersetup{pdfcontactcountry={Switzerland}}
\hypersetup{pdfcontactemail={nbeisert@itp.phys.ethz.ch}}
\hypersetup{pdfcontacturl={http://people.phys.ethz.ch/\xmptilde nbeisert/}}

\newcommand{\secref}[1]{\hyperref[#1]{section \ref*{#1}}}

\parskip1ex
\parindent0pt
\let\olditemize\itemize
\def\itemize{\olditemize\parskip0pt}

\begin{document}

\title{The \textsf{childdoc} Package}
\hypersetup{pdftitle={The childdoc Package}}
\author{Niklas Beisert\\[2ex]
  Institut f\"ur Theoretische Physik\\
  Eidgen\"ossische Technische Hochschule Z\"urich\\
  Wolfgang-Pauli-Strasse 27, 8093 Z\"urich, Switzerland\\[1ex]
  \href{mailto:nbeisert@itp.phys.ethz.ch}
  {\texttt{nbeisert@itp.phys.ethz.ch}}}
\hypersetup{pdfauthor={Niklas Beisert}}
\hypersetup{pdfsubject={Manual for the LaTeX2e Package childdoc}}
\date{30 December 2018, \textsf{v2.0}}
\maketitle

\begin{abstract}\noindent
\textsf{childdoc} is a \LaTeXe{} package
that enables the direct compilation
of document sections included by |\include|
to individual files.
\end{abstract}

\begingroup
\parskip0ex
\tableofcontents
\endgroup

%%%%%%%%%%%%%%%%%%%%%%%%%%%%%%%%%%%%%%%%%%%%%%%%%%%%%%%%%%%%%%%%%%%%%%%%%%%%%%%%
%%%%%%%%%%%%%%%%%%%%%%%%%%%%%%%%%%%%%%%%%%%%%%%%%%%%%%%%%%%%%%%%%%%%%%%%%%%%%%%%
\section{Introduction}

\LaTeX{} provides a mechanism to structure a large document (such as a book)
into a main file and several child files (containing the chapters)
using the |\include| command.
This mechanism is beneficial for documents
which span hundreds of pages in order to
make the source file(s) more manageable.
Moreover, compilation can be restricted to
selected child files by means of the |\includeonly| command.
The latter feature can be used to reduce the compilation time while editing
(this was significantly more useful in the earlier days of \LaTeX{})
or to generate a smaller document which is easier to navigate.
Another application of |\includeonly| is to generate
documents consisting of selected parts of the complete document.

However, there are a few drawbacks of the plain |\include| mechanism:
\begin{itemize}
\item
The child files cannot be compiled on their own,
they can only be compiled via the main file.
A naive editing environment
(such as a text editor with an option
to have the current file processed by \LaTeX)
may require one to switch to the main file before compiling;
attempting to compile the child file produces errors.
\item
The main file must be modified (each time)
to adjust the |\includeonly| command
to the present needs. This easily leaves the main file in a messy state.
\item
The generated document will always carry the filename
of the main document. This is inconvenient if
several child files are to be compiled and
to be kept for distribution.
\end{itemize}

The present package provides a simple interface
to make child files individually compilable by \LaTeX{}.
Compiling a child file then has the same effect as compiling
the main file with an |\includeonly| command
to select the appropriate child.
Moreover the generated document will carry the name of the child
rather than the main file.
This resolves all three above issues.

This feature is meant to make the editing of books,
thesis documents and lecture notes somewhat more convenient.
However, the package can also be used efficiently for
composing a series of documents (such as exercise sheets)
which are typically distributed individually.
It then assists the author in generating the individual documents
(potentially in different versions)
as well as a document containing the collected series.
Another application is in developing style files
or other kinds of included material
where compilation of the style file could redirect
to a sample or test file.

%%%%%%%%%%%%%%%%%%%%%%%%%%%%%%%%%%%%%%%%%%%%%%%%%%%%%%%%%%%%%%%%%%%%%%%%%%%%%%%%
%%%%%%%%%%%%%%%%%%%%%%%%%%%%%%%%%%%%%%%%%%%%%%%%%%%%%%%%%%%%%%%%%%%%%%%%%%%%%%%%
\section{Usage}

First of all, the package \textsf{childdoc} is \emph{not} a standard
\LaTeXe{} |.sty| style file! Therefore it needs to be invoked in
a non-standard way.

%%%%%%%%%%%%%%%%%%%%%%%%%%%%%%%%%%%%%%%%%%%%%%%%%%%%%%%%%%%%%%%%%%%%%%%%%%%%%%%%
\subsection{Included Files}
\label{sec:include}

%%%%%%%%%%%%%%%%%%%%%%%%%%%%%%%%%%%%%%%%
\DescribeMacro{\childdocmain}
To use the package, add the commands
\begin{center}
\begin{tabular}{l}
|\input{childdoc.def}|\\
|\childdocmain{}|\\
\end{tabular}
\end{center}
at the very top of the main \LaTeX{} file,
in particular \emph{before} the |\documentclass| statement!
The argument of |\childdocmain| should be left empty
(but it must be present).

%%%%%%%%%%%%%%%%%%%%%%%%%%%%%%%%%%%%%%%%
\DescribeMacro{\childdocof}
Furthermore, add the commands
\begin{center}
\begin{tabular}{l}
|\input{childdoc.def}|\\
|\childdocof{|\textit{main}|}|\\
\end{tabular}
\end{center}
at the top of every child file \textit{child}
which is included by |\include{|\textit{child}|}|
from within the main file
(or at least for those files to be compiled individually).
The argument \textit{main} must be the filename of the main file.

There are a couple of
considerations in setting up the main and child documents:

%%%%%%%%%%%%%%%%%%%%%%%%%%%%%%%%%%%%%%%%
\paragraph{Restrictions.}

Please note the following restrictions:
\begin{itemize}
\item
|\childdocmain| must be called with one argument \textit{main}
to ensure compatibility with earlier version of the package.
It must either be empty (|\childdocmain{}|)
or precisely match the filename of the main file in which it is specified.
See \secref{sec:detection} for further information.
\item
The filename \textit{main} must be specified without the |.tex| extension.
\item
The filename \textit{main} is case sensitive
(even in case-insensitive file systems)
due to internal string comparison.
\item
The argument \textit{main} should be fully expanded, it cannot be a macro.
\item
Subdirectories and special characters should be avoided in filenames.
\item
The command |\childdocmain{|\textit{main}|}| must be followed by a whitespace.
It should not be followed immediately by another command
or by a comment mark `|%|'.
This is because the \TeX{} parser reads the token immediately following
the argument of |\childdocmain| and puts it
at the beginning of every child section;
however, a white\-space is ignored.
\end{itemize}

%%%%%%%%%%%%%%%%%%%%%%%%%%%%%%%%%%%%%%%%
\paragraph{Content of Main File.}

It is advisable to place all content in the child files included by |\include|.
Any output contained in the main file will appear in all child documents
unless suppressed manually;
it cannot be suppressed automatically by the |\includeonly| directive
and thus should normally be avoided.
A method to include some content in the main file
by means of conditional processing is described in \secref{sec:conditional}.

%%%%%%%%%%%%%%%%%%%%%%%%%%%%%%%%%%%%%%%%
\paragraph{Page Numbering.}

When only a part of the document is compiled,
the appropriate numbering of pages
(as well as other status parameters)
is determined from the |.aux| files.
The latter contain information from previous passes.
However this information needs to propagate through
all intermediate child documents.
Therefore the page numbering in child documents may well
be inconsistent until the complete document is compiled at least once.

A useful (if unconventional) way to always ensure a consistent
page numbering is to restart the numbering in each child document
and denote the pages by `\textit{child}|.|\textit{page}'
where \textit{child} represents the chapter/section number of the child file.
This can be achieved by the command
|\numberwithin{page}{|\textit{child}|}|
of the \textsf{amsmath} package
where \textit{child} can be |chapter| or |section|
depending on the chosen structuring.
Alternatively, one can modify the macro |\thepage| appropriately
and reset the counter |page| at the start of each child file.

%%%%%%%%%%%%%%%%%%%%%%%%%%%%%%%%%%%%%%%%%%%%%%%%%%%%%%%%%%%%%%%%%%%%%%%%%%%%%%%%
\subsection{Conditional Processing}
\label{sec:conditional}

The package provides a mechanism to compile different versions
of a document. To customise the versions further some conditional processing
can come in handy to distinguish which version is being compiled.
The package provides two macros to describe the compilation context:

%%%%%%%%%%%%%%%%%%%%%%%%%%%%%%%%%%%%%%%%
\DescribeMacro{\ifchilddoc}
The conditional |\ifchilddoc| distinguishes between the compilation of
child documents and the main document:
%
\begin{center}
|\ifchilddoc |\textit{child-code}| |[|\||else |\textit{main-code}]| \||fi|
\end{center}

%%%%%%%%%%%%%%%%%%%%%%%%%%%%%%%%%%%%%%%%
\DescribeMacro{\childdocname}
\DescribeMacro{\childdocjob}
The macro |\childdocname| contains the filename (without extension)
of the main or child file being processed.
Note that |\childdocjob| will always contain the name of the main file.

%%%%%%%%%%%%%%%%%%%%%%%%%%%%%%%%%%%%%%%%
\paragraph{Title Page.}

Conditional processing can be used to include a title or banner page
in the main document when proper precautions are taken.
Importantly, the code in the main file should ensure that the page counter
(as well as other status parameters which are stored in the |.aux| files)
takes the same value after the conditional processing.
Otherwise the page numbers may take divergent values
depending on which part is compiled.

For example, a title page could be declared by:
%
\begin{center}
\begin{tabular}{l}
|\ifchilddoc\||else|\\
|\addtocounter{page}{-1}|\\
\textit{code for title page}\\
|\newpage|\\
|\||fi|
\end{tabular}
\end{center}
%
A banner page for the child documents can be generated by:
%
\begin{center}
\begin{tabular}{l}
|\ifchilddoc|\\
|\addtocounter{page}{-1}|\\
\textit{code for banner page}\\
|\newpage|\\
|\||fi|
\end{tabular}
\end{center}
%
Here one could write a message such as:
\begin{center}
|This is the part \childdocname{} of \childdocjob{}.|
\end{center}

%%%%%%%%%%%%%%%%%%%%%%%%%%%%%%%%%%%%%%%%%%%%%%%%%%%%%%%%%%%%%%%%%%%%%%%%%%%%%%%%
\subsection{Flags}
\label{sec:flags}

The package makes it easy to generate different versions
of the main or child documents.
To this end compilation flags can be defined
and assigned different default values.
They will be particularly useful in conjunction
with the forwarding mechanism described in \secref{sec:forward}.

For example, it may be useful to have a flag |\version|
which can be set to |draft| or |final|.
The document source will contain some conditional code
depending on the value of |\version|.
Suppose further, the flag should default to |final| for the main file
and to |draft| for child files
which is a natural assignment for editing the document.
This is achieved by placing the following code
in the preamble of the main document
(below the |\childdocmain| directive):
%
\begin{center}
\begin{tabular}{l}
|\ifchilddoc|\\
|\providecommand{\version}{draft}|\\
|\||else|\\
|\providecommand{\version}{final}|\\
|\||fi|
\end{tabular}
\end{center}
%
The definition by |\providecommand| makes sure
that previous definitions are not overwritten.
Further statements |\providecommand{\version}{...}|
can thus be added before the above code to override it.

For the main file, one might add a line
(between |\childdocmain| and the above block)
%
\begin{center}
|%\ifchilddoc\||else\providecommand{\version}{draft}\||fi|
\end{center}
%
which can be uncommented to produce a draft version.
Likewise one can add a line to the very top of a child file
(above the |\childdocof{|\textit{main}|}| directive)
%
\begin{center}
|%\providecommand{\version}{final}|
\end{center}
%
which can be uncommented to produce the final version of this child document.

%%%%%%%%%%%%%%%%%%%%%%%%%%%%%%%%%%%%%%%%%%%%%%%%%%%%%%%%%%%%%%%%%%%%%%%%%%%%%%%%
\subsection{Forwarding}
\label{sec:forward}

Different versions of the main or child documents
using compilation flags as described in \secref{sec:flags}
can be (permanently) stored in different files
for convenient compilation, viewing and distribution.
To this end, the package defines a command
to pass on compilation to a different file:

%%%%%%%%%%%%%%%%%%%%%%%%%%%%%%%%%%%%%%%%
\DescribeMacro{\childdocforward}
The command |\childdocforward| redirects processing to
another source file:
%
\begin{center}
\begin{tabular}{l}
|\input{childdoc.def}|\\
|\childdocforward[|\textit{main}|]{|\textit{dest}|}|\\
\end{tabular}
\end{center}
%
The argument \textit{dest} is the destination file
(without extension).
It should be the main file or one of the child files.
Note that further \textsf{childdoc} directives
such as |\childdocof| and |\childdocforward|
in the indicated file will be processed in this form.
The optional argument \textit{main}
passes on directly to the main file \textit{main}
while pretending to compile the child \textit{dest}.
This form behaves as if \textit{dest}
issues |\childdocof{|\textit{main}|}| right away,
and no further \textsf{childdoc} directives will be processed.

%%%%%%%%%%%%%%%%%%%%%%%%%%%%%%%%%%%%%%%%
\DescribeMacro{\...prefix}
In the alternative form |\childdocforwardprefix|,
%
\begin{center}
\begin{tabular}{l}
|\input{childdoc.def}|\\
|\childdocforwardprefix[|\textit{main}|]{|\textit{prefix}|}{|\textit{dest}|}|
\end{tabular}
\end{center}
%
the destination file is determined by a pattern
depending on the current file:
To make this work, the current file must be called
`{\textit{prefix}\hspace{0.2em}\textit{suffix}}'
with \textit{prefix} matching precisely the argument.
Processing is then passed on to the file
`{\textit{dest}\hspace{0.2em}\textit{suffix}}'.
Surely, the same effect is achieved by
directly specifying the
argument `{\textit{dest}\hspace{0.2em}\textit{suffix}}'
in the first form.
However, that requires to set up a different file
for each child. With the alternative form of the command
all these files can have exactly the same content
which simplifies setting them up and maintaining them.

For example, the following file |draft.tex|
with a compilation flag |\version| as described in \secref{sec:flags}
compiles the main document as a draft:
%
\begin{center}
\begin{tabular}{l}
|\def\version{draft}|\\
|\input{childdoc.def}|\\
|\childdocforward{|\textit{main}|}|
\end{tabular}
\end{center}
%
Likewise, the following files |final|\textit{nn}|.tex|
compile the final version of the child document
|child|\textit{nn}|.tex|:
%
\begin{center}
\begin{tabular}{l}
|\def\version{final}|\\
|\input{childdoc.def}|\\
|\childdocforwardprefix{final}{child}|
\end{tabular}
\end{center}
%

Note that when several versions of a main file and/or of each child file
are to be generated, it may be convenient to set up a |Makefile| or
shell script to automatise the process.

%%%%%%%%%%%%%%%%%%%%%%%%%%%%%%%%%%%%%%%%%%%%%%%%%%%%%%%%%%%%%%%%%%%%%%%%%%%%%%%%
\subsection{Command Line Processing}
\label{sec:commandline}

The effect of redirection files can also be achieved by invoking
the \LaTeX{} compiler with a more elaborate command line.
Most conveniently this should be done as part
of a shell script or a |Makefile|.

When using \textsf{childdoc} in the main file, the following
command lines effectively perform a redirection
(note that depending on the shell being used,
backslashes may have to be doubled: `|\|' $\to$ `|\\|'):
%
\begin{center}
|... -jobname "|\textit{target}|" |\\|"|[\textit{flags}]%
|\input{childdoc.def}\childdocforward[|\textit{main}|]{|\textit{dest}|}"|
\end{center}
%
Here \textit{target} is the name of the output file,
\textit{main} is the name of the main file
and \textit{dest} is the name of the main or child file to be processed
(all filenames without extensions).
The optional argument \textit{main} can be omitted
if \textit{main} matches \textit{dest}.
Optionally, compilation \textit{flags} can be defined via |\def| commands.
This command line makes the \TeX{} engine believe
it is compiling the file \textit{target}
whose content is specified as the latter parameter.
The provided code then forwards the processing to
\textit{main} or \textit{dest} as described in \secref{sec:forward}.

%%%%%%%%%%%%%%%%%%%%%%%%%%%%%%%%%%%%%%%%%%%%%%%%%%%%%%%%%%%%%%%%%%%%%%%%%%%%%%%%
\subsection{Include by Input}
\label{sec:input}

Including child documents by |\include| has some restrictions by design.
Most notably, the content of a child document always occupies
its own set of pages; pages cannot be shared between child documents.
Usually, this behaviour makes perfect sense
because each child document contain an essential part of the document.
However, in some situations it may be desirable to compose
a document from a collection of parts
without having mandatory page breaks between then.
For this case, the package
provides a mechanism to include parts
by |\input| which can also be processed individually.
However, by construction this mechanism
requires manual handling of the content to be output.

%%%%%%%%%%%%%%%%%%%%%%%%%%%%%%%%%%%%%%%%
\DescribeMacro{\ifchilddocmanual}
The main file should be prepared as usual, see \secref{sec:include}.
However, the document body must make a distinction
between processing of an individual part and of the main document, e.g.:
%
\begin{center}
\begin{tabular}{l}
|\ifchilddocmanual|\\
|\input{\childdocname}|\\
|\||else|\\
\textit{document body with }|\input{|\textit{part}|}|\\
|\||fi|
\end{tabular}
\end{center}
%
The conditional |\ifchilddocmanual| is true whenever
a part to be included by |\input| is being compiled,
and the name of the part is stored in |\childdocname|.

%%%%%%%%%%%%%%%%%%%%%%%%%%%%%%%%%%%%%%%%
\DescribeMacro{\childdocby}
Each part to be included by |\input| should start with:
%
\begin{center}
\begin{tabular}{l}
|\input{childdoc.def}|\\
|\childdocby{|\textit{main}|}|\\
\end{tabular}
\end{center}
%
The directive |\childdocby| is similar to |\childdocof|
described in \secref{sec:include},
but the subsequent selection of content must be done manually.
To that end, both |\ifchilddoc| and |\ifchilddocmanual|
will be true upon processing of a part,
and the name of the part is stored in |\childdocname|.
Note that |\jobname| will be set to the filename of the current part
so that each part receives an individual |.aux| file
that does not interfere with the |.aux| file(s) of the main document.
This behaviour can be altered by the alternative form
|\childdocby[*]{|\textit{main}|}| (with a non-empty optional argument)
which uses the |.aux| file of the main document
by setting |\jobname| to \textit{main}.

%%%%%%%%%%%%%%%%%%%%%%%%%%%%%%%%%%%%%%%%%%%%%%%%%%%%%%%%%%%%%%%%%%%%%%%%%%%%%%%%
\subsection{Driver Development}
\label{sec:driver}

The \textsf{childdoc} mechanism can also be use for the development
of definition files such as \LaTeX{} styles or classes.
This case differs from the above setup with multiple parts
included by |\include| in that no |\includeonly| should be invoked.
This can be achieved by starting the include file
(before |\ProvidesPackage|) with:
%
\begin{center}
\begin{tabular}{l}
|\input{childdoc.def}|\\
|\childdocforward{|\textit{main}|}|\\
\end{tabular}
\end{center}
%
or alternatively with:
%
\begin{center}
\begin{tabular}{l}
|\input{childdoc.def}|\\
|\childdocby{|\textit{main}|}|\\
\end{tabular}
\end{center}
%
Both forms have slightly different effects as described above.
The main file is prepared as usual, see \secref{sec:include}.

%%%%%%%%%%%%%%%%%%%%%%%%%%%%%%%%%%%%%%%%%%%%%%%%%%%%%%%%%%%%%%%%%%%%%%%%%%%%%%%%
\subsection{Legacy Detection}
\label{sec:detection}

The directive |\childdocmain| in the main file can detect
whether the complete document or merely a child is to be compiled
even without using the directive |\childdocof|.
This method is deprecated because it is less robust
and there is no compelling reason to use it;
it is merely provided for backward compatibility
and it may be removed in future versions.

If the detection mechanism is to be used,
it is mandatory to correctly specify
the filename of the main file as the argument of |\childdocmain|:
%
\begin{center}
\begin{tabular}{l}
|\input{childdoc.def}|\\
|\childdocmain{|\textit{main}|}|\\
\end{tabular}
\end{center}
%
If |\jobname| does not match the argument \textit{main} of |\childdocmain|,
it is assumed that |\jobname| points to the child file to be compiled.
When using |\childdocmain| with the main file specified as argument,
it suffices to start a child file
with just |\input{|\textit{main}|}|
without loading of the package and using |\childdocof|.
If instead all processing is done
with the appropriate \textsf{childdoc} directives,
the argument of \textit{main} of |\childdocmain| can be empty.

An alternative version of the command line processing described
in \secref{sec:commandline} using the detection mechanism reads:
%
\begin{center}
|... -jobname "|\textit{target}|" "|[\textit{flags}]%
[|\def\jobname{|\textit{dest}|}|]|\input{|\textit{main}|}"|
\end{center}

%%%%%%%%%%%%%%%%%%%%%%%%%%%%%%%%%%%%%%%%%%%%%%%%%%%%%%%%%%%%%%%%%%%%%%%%%%%%%%%%
\subsection{Manual Code}
\label{sec:manual}

In case one cannot be certain whether the definitions file |childdoc.def|
is installed on the target \TeX{} distribution
and one prefers not to ship it,
it is conceivable to paste a few relevant commands into the sources.

To that end, drop all statements |\input{childdoc.def}|
and perform the replacements as outlined below.
Instead of |\childdocmain{|\textit{main}|}| add the following code
to the top of the main file:
%
\begin{center}
\begin{tabular}{l}
|\||ifdefined\childdocname\endinput\||fi\newif\ifchilddoc|\\
|\edef\childdocname{\scantokens\expandafter{\jobname\noexpand}}|\\
|\def\childdocmain{|\textit{main}|}\||ifx\childdocmain\childdocname\||else|\\
|\childdoctrue\includeonly{\childdocname}\let\jobname\childdocmain\||fi|\\
\end{tabular}
\end{center}
%
Instead of |\childdocof{|\textit{main}|}| just include the main file
at the top of each child file:
%
\begin{center}
|\input{|\textit{main}|}|
\end{center}
%
A simple redirection |\childdocforward{|\textit{dest}|}| is achieved by:
%
\begin{center}
|\def\jobname{|\textit{dest}|}\input{\jobname}|
\end{center}
%
The redirection with prefix
|\childdocforwardprefix[|\textit{prefix}|]{|\textit{dest}|}|
is accomplished by:
%
\begin{center}
\begin{tabular}{l}
|{\edef\jobname{\scantokens\expandafter{\jobname\noexpand}}|\\
|\def\redirectjob |\textit{prefix}|#1~~~{\gdef\jobname{|\textit{dest}|#1}}|\\
|\expandafter\redirectjob\jobname~~~}\input{\jobname}|
\end{tabular}
\end{center}

In an alternative approach,
child documents can be compiled by a specific command line
without additional code or specific definitions:
%
\begin{center}
|... -jobname "|\textit{target}|" "|[\textit{flags}]%
|\includeonly{|\textit{dest}|}\input{|\textit{main}|}"|
\end{center}
%

%%%%%%%%%%%%%%%%%%%%%%%%%%%%%%%%%%%%%%%%%%%%%%%%%%%%%%%%%%%%%%%%%%%%%%%%%%%%%%%%
%%%%%%%%%%%%%%%%%%%%%%%%%%%%%%%%%%%%%%%%%%%%%%%%%%%%%%%%%%%%%%%%%%%%%%%%%%%%%%%%
\section{Information}

%%%%%%%%%%%%%%%%%%%%%%%%%%%%%%%%%%%%%%%%%%%%%%%%%%%%%%%%%%%%%%%%%%%%%%%%%%%%%%%%
\subsection{Copyright}

Copyright \copyright{} 2017--2018 Niklas Beisert

This work may be distributed and/or modified under the
conditions of the \LaTeX{} Project Public License, either version 1.3
of this license or (at your option) any later version.
The latest version of this license is in
  \url{http://www.latex-project.org/lppl.txt}
and version 1.3 or later is part of all distributions of \LaTeX{}
version 2005/12/01 or later.

This work has the LPPL maintenance status `maintained'.

The Current Maintainer of this work is Niklas Beisert.

This work consists of the files |README.txt|, |childdoc.ins| and |childdoc.dtx|
as well as the derived files |childdoc.def|, |cdocsamp.tex|
with |cdocsch1.tex|, |cdocsch2.tex|, |cdocspt3.tex|, |cdocspt4.tex|,
|cdocsdrf.tex|, |cdocsfn1.tex|, |cdocsfn2.tex|
as well as |childdoc.pdf|.

%%%%%%%%%%%%%%%%%%%%%%%%%%%%%%%%%%%%%%%%%%%%%%%%%%%%%%%%%%%%%%%%%%%%%%%%%%%%%%%%
\subsection{Files and Installation}

The package consists of the files:
%
\begin{center}
\begin{tabular}{ll}
    |README.txt|   & readme file \\
    |childdoc.ins| & installation file \\
    |childdoc.dtx| & source file \\
    |childdoc.def| & definition file \\
    |cdocsamp.tex| & sample main file \\
    |cdocsch1.tex| & sample include file \\
    |cdocsch2.tex| & sample include file \\
    |cdocspt3.tex| & sample part file \\
    |cdocspt4.tex| & sample part file \\
    |cdocsdrf.tex| & sample redirection file \\
    |cdocsfn1.tex| & sample redirection file \\
    |cdocsfn2.tex| & sample redirection file \\
    |childdoc.pdf| & manual
\end{tabular}
\end{center}
%
The distribution consists of the files
|README.txt|, |childdoc.ins| and |childdoc.dtx|.
%
\begin{itemize}
\item
Run (pdf)\LaTeX{} on |childdoc.dtx|
to compile the manual |childdoc.pdf| (this file).
\item
Run \LaTeX{} on |childdoc.ins| to create the definitions file |childdoc.def|
and the sample |cdocsamp.tex| with include files
|cdocsch1.tex|, |cdocsch2.tex|, |cdocspt3.tex|, |cdocspt4.tex|,
|cdocsdrf.tex|, |cdocsfn1.tex|, |cdocsfn2.tex|.
Then copy the file |childdoc.def| to an appropriate directory of your \LaTeX{}
distribution, e.g.\ \textit{texmf-root}|/tex/latex/childdoc|.
\end{itemize}

%%%%%%%%%%%%%%%%%%%%%%%%%%%%%%%%%%%%%%%%%%%%%%%%%%%%%%%%%%%%%%%%%%%%%%%%%%%%%%%%
\subsection{Related CTAN Packages}

There are several other packages which offer a similar functionality:
%
\begin{itemize}
\item
The packages
\href{http://ctan.org/pkg/docmute}{\textsf{docmute}},
\href{http://ctan.org/pkg/includex}{\textsf{includex}} and
\href{http://ctan.org/pkg/standalone}{\textsf{standalone}}
provide commands to include only the document body of
a child file thus allowing both files to be compiled individually.
\item
The packages \href{http://ctan.org/pkg/subdocs}{\textsf{subdocs}}
and \href{http://ctan.org/pkg/subfiles}{\textsf{subfiles}}
provide structures in which the main and child documents can be
encapsulated and allowing them to be compiled individually.
The inclusion mechanism is different from the conventional |\include|.
\item
The package \href{http://ctan.org/pkg/combine}{\textsf{combine}}
is an elaborate solution to combine several documents into one.
\end{itemize}
%
See also the CTAN topic \href{http://ctan.org/topic/subdocs}{\textsf{subdocs}}
for further related packages.
The present package differs from the above solutions in that
a document structure constructed with the conventional |\include| mechanism
just needs two extra commands at the top of every file
such that all constituent files can be compiled individually.

%%%%%%%%%%%%%%%%%%%%%%%%%%%%%%%%%%%%%%%%%%%%%%%%%%%%%%%%%%%%%%%%%%%%%%%%%%%%%%%%
%\subsection{Feature Suggestions}
%
%The following is a list of features which may be useful for future
%versions of this package:
%%
%\begin{itemize}
%\item
%\ldots
%\end{itemize}

%%%%%%%%%%%%%%%%%%%%%%%%%%%%%%%%%%%%%%%%%%%%%%%%%%%%%%%%%%%%%%%%%%%%%%%%%%%%%%%%
\subsection{Revision History}

%%%%%%%%%%%%%%%%%%%%%%%%%%%%%%%%%%%%%%%%
\paragraph{v2.0:} 2018/12/30

\begin{itemize}
\item
immediate forward processing
\item
added |\childdocby| mechanism
\item
manual restructured
\end{itemize}

%%%%%%%%%%%%%%%%%%%%%%%%%%%%%%%%%%%%%%%%
\paragraph{v1.6:} 2018/01/17

\begin{itemize}
\item
application for development of include files
\item
corrections to manual
\end{itemize}

%%%%%%%%%%%%%%%%%%%%%%%%%%%%%%%%%%%%%%%%
\paragraph{v1.5:} 2017/05/21

\begin{itemize}
\item
more complete structuring introduced
\item
|\childdocof| introduced
\item
|\childdoc| renamed to |\childdocmain|
\item
|\childredirect| renamed to |\childdocforward| and |\childdocforwardprefix|
and functionality expanded
\end{itemize}

%%%%%%%%%%%%%%%%%%%%%%%%%%%%%%%%%%%%%%%%
\paragraph{v1.0:} 2017/04/27

\begin{itemize}
\item
manual and install package
\item
first version published on CTAN
\end{itemize}

%%%%%%%%%%%%%%%%%%%%%%%%%%%%%%%%%%%%%%%%
\paragraph{v0.6:} 2017/04/26

\begin{itemize}
\item
redirection mechanism added
\end{itemize}

%%%%%%%%%%%%%%%%%%%%%%%%%%%%%%%%%%%%%%%%
\paragraph{v0.5:} 2017/04/26

\begin{itemize}
\item
functionality in definition file
\end{itemize}


%%%%%%%%%%%%%%%%%%%%%%%%%%%%%%%%%%%%%%%%%%%%%%%%%%%%%%%%%%%%%%%%%%%%%%%%%%%%%%%%
%%%%%%%%%%%%%%%%%%%%%%%%%%%%%%%%%%%%%%%%%%%%%%%%%%%%%%%%%%%%%%%%%%%%%%%%%%%%%%%%
%%%%%%%%%%%%%%%%%%%%%%%%%%%%%%%%%%%%%%%%%%%%%%%%%%%%%%%%%%%%%%%%%%%%%%%%%%%%%%%%
\appendix

\settowidth\MacroIndent{\rmfamily\scriptsize 000\ }

 \DocInput{childdoc.dtx}

\end{document}
%</driver>
% \fi
%
% %%%%%%%%%%%%%%%%%%%%%%%%%%%%%%%%%%%%%%%%%%%%%%%%%%%%%%%%%%%%%%%%%%%%%%%%%%%%%%
% %%%%%%%%%%%%%%%%%%%%%%%%%%%%%%%%%%%%%%%%%%%%%%%%%%%%%%%%%%%%%%%%%%%%%%%%%%%%%%
% \section{Sample}
%\iffalse
%<*samplemain>
%\fi
%
% The following presents a sample document
% with two chapters, two parts, a title page,
% a compile flag as well as three forwarding files to set the flag.
% It consists of eight |.tex| files:
% \begin{center}
% \begin{tabular}{ll}
% |cdocsamp.tex|&main file\\
% |cdocsch1.tex|&include file for chapter 1\\
% |cdocsch2.tex|&include file for chapter 2\\
% |cdocspt3.tex|&include file for part 3\\
% |cdocspt4.tex|&include file for part 4\\
% |cdocsdrf.tex|&forwarding file for main file in draft mode\\
% |cdocsfi1.tex|&forwarding file for final version of chapter 1\\
% |cdocsfi2.tex|&forwarding file for final version of chapter 2\\
% \end{tabular}
% \end{center}
% Each of the eight files can be compiled directly by the \LaTeX{} compiler.
%
% %%%%%%%%%%%%%%%%%%%%%%%%%%%%%%%%%%%%%%
% \paragraph{Main File.}
%
% The main file is called |cdocsamp.tex|.
%
% Load the \textsf{childdoc} definitions and
% declare the filename for the main document:
%    \begin{macrocode}
\input{childdoc.def}
\childdocmain{}
%    \end{macrocode}

% Optional override for |\version| flag:
%    \begin{macrocode}
%%\ifchilddoc\else\providecommand{\version}{draft}\fi
%    \end{macrocode}

% Define the default values for the |\version| flag
% (|final| for the main file and |draft| for childs):
%    \begin{macrocode}
\ifchilddoc
\providecommand{\version}{draft}
\else
\providecommand{\version}{final}
\fi
%    \end{macrocode}

% Load the standard document class:
%    \begin{macrocode}
\documentclass[12pt]{article}
%    \end{macrocode}

% Start the document body:
%    \begin{macrocode}
\begin{document}
%    \end{macrocode}

% Declare a title page.
% Print title, part of document being processed and version flag:
%    \begin{macrocode}
\addtocounter{page}{-1}
\begin{center}
{\LARGE\bfseries{}childdoc example\par}
\vspace{1cm}
\ifchilddoc
\ifchilddocmanual part\else chapter\fi:
`\childdocname' of `\childdocjob'\par
\else
main document: `\childdocjob'\par
\fi
version: \version\par
\end{center}
\newpage
%    \end{macrocode}

% Manually include selected file,
% otherwise process as usual:
%    \begin{macrocode}
\ifchilddocmanual
\section*{part `\childdocname'}
\input{\childdocname}
\else
%    \end{macrocode}

% Include the two chapters:
%    \begin{macrocode}
\include{cdocsch1}
\include{cdocsch2}
%    \end{macrocode}

% Include the two parts unless only chapters should be displayed:
%    \begin{macrocode}
\ifchilddoc\else
\section{part three}
\input{cdocspt3}
\section{part four}
\input{cdocspt4}
\fi
%    \end{macrocode}

% Process as usual until here:
%    \begin{macrocode}
\fi
%    \end{macrocode}

% End of document body:
%    \begin{macrocode}
\end{document}
%    \end{macrocode}
%\iffalse
%</samplemain>
%\fi
%
% %%%%%%%%%%%%%%%%%%%%%%%%%%%%%%%%%%%%%%
% \paragraph{Chapter Include Files.}
%
% The include files are called |cdocsch1.tex| and |cdocsch2.tex|.
%
%\iffalse
%<*samplechap1|samplechap2>
%\fi

% Optional override for |\version| flag:
%    \begin{macrocode}
%%\providecommand{\version}{final}
%    \end{macrocode}

% Include the main document:
%    \begin{macrocode}
\input{childdoc.def}
\childdocof{cdocsamp}
%    \end{macrocode}

%\iffalse
%</samplechap1|samplechap2>
%\fi
%
%\iffalse
%<*samplechap1>
%\fi
% Some text for chapter 1:
%    \begin{macrocode}
\section{one}
some text in chapter one
%    \end{macrocode}

%\iffalse
%</samplechap1>
%\fi
% Some text for chapter 2:
%\iffalse
%<*samplechap2>
%\fi
%    \begin{macrocode}
\section{two}
more text in chapter two
%    \end{macrocode}

%\iffalse
%</samplechap2>
%\fi
%
% %%%%%%%%%%%%%%%%%%%%%%%%%%%%%%%%%%%%%%
% \paragraph{Part Include Files.}
%
% The include files are called |cdocspt3.tex| and |cdocspt4.tex|.
%
%\iffalse
%<*samplepart3|samplepart4>
%\fi

% Optional override for |\version| flag:
%    \begin{macrocode}
%%\providecommand{\version}{final}
%    \end{macrocode}

% Include the main document:
%    \begin{macrocode}
\input{childdoc.def}
\childdocby{cdocsamp}
%    \end{macrocode}

%\iffalse
%</samplepart3|samplepart4>
%\fi
%
%\iffalse
%<*samplepart3>
%\fi
% Some text for part 3:
%    \begin{macrocode}
some text in part three
%    \end{macrocode}

%\iffalse
%</samplepart3>
%\fi
% Some text for part 4:
%\iffalse
%<*samplepart4>
%\fi
%    \begin{macrocode}
more text in part four
%    \end{macrocode}

%\iffalse
%</samplepart4>
%\fi
%
% %%%%%%%%%%%%%%%%%%%%%%%%%%%%%%%%%%%%%%
% \paragraph{Forwarding for a Complete Draft.}
%
% The following forwarding file |cdocsdrf.tex|
% compiles the main document in draft mode:
%\iffalse
%<*sampledraft>
%\fi
%    \begin{macrocode}
\def\version{draft}
\input{childdoc.def}
\childdocforward{cdocsamp}
%    \end{macrocode}

%\iffalse
%</sampledraft>
%\fi
%
% %%%%%%%%%%%%%%%%%%%%%%%%%%%%%%%%%%%%%%
% \paragraph{Forwarding for Final Version of the Chapters.}
%
% The following forwarding files |cdocsfn1.tex| and |cdocsfn2.tex|
% (with identical content)
% compile the final versions of the child documents
% |cdocsch1.tex| and |cdocsch2.tex|, respectively:
%\iffalse
%<*samplefinal>
%\fi
%    \begin{macrocode}
\def\version{final}
\input{childdoc.def}
\childdocforwardprefix[cdocsamp]{cdocsfn}{cdocsch}
%    \end{macrocode}

%\iffalse
%</samplefinal>
%\fi
%
% %%%%%%%%%%%%%%%%%%%%%%%%%%%%%%%%%%%%%%
% \paragraph{Command Line Processing.}
%
% The following three command lines generate the output files
% |cdocscld|, |cdocscl1| and |cdocscl2|
% which should be identical to
% |cdocsdrf|, |cdocsch1| and |cdocsfn2|, respectively:
% \begin{center}
% \begin{tabular}{l}
% |latex -jobname cdocscld \|\\
% |  "\def\version{draft}\input{childdoc.def}\childdocforward{cdocsamp}"|\\
% |latex -jobname cdocscl1 \|\\
% |  "\input{childdoc.def}\childdocforward[cdocsamp]{cdocsch1}"|\\
% |latex -jobname cdocscl2 \|\\
% |  "\def\version{final}\input{childdoc.def}\childdocforward{cdocsch2}"|
% \end{tabular}
% \end{center}
% Note that the trailing backslash on each first line
% merely continues the input to the second line
% (for convenient cut ant paste).
% Furthermore, the command |latex| can be replaced by any
% of its alternative versions such as |pdflatex|.
%
% %%%%%%%%%%%%%%%%%%%%%%%%%%%%%%%%%%%%%%%%%%%%%%%%%%%%%%%%%%%%%%%%%%%%%%%%%%%%%%
% %%%%%%%%%%%%%%%%%%%%%%%%%%%%%%%%%%%%%%%%%%%%%%%%%%%%%%%%%%%%%%%%%%%%%%%%%%%%%%
% \section{Implementation}
%\iffalse
%<*package>
%\fi
%
% This section describes the definitions file |childdoc.def|.

% The definitions cannot be loaded using |\usepackage| or |\RequirePackage|
% which has a mechanism to prevent loading a style file more than once.
% When loading the definitions by means of |\input|
% multiple instances have to be prevented manually:
%\iffalse
%This code needs to be before the `\ProvidesFile' directive
%which is defined at the beginning of this file.
%Therefore it is also placed there and commented out here.
%</package>
%<*discard>
%\fi
%    \begin{macrocode}
\ifdefined\childdocmain\endinput\fi
%    \end{macrocode}
%\iffalse
%</discard>
%<*package>
%\fi
%
% \macro{\ifchilddoc}
% \macro{\ifchilddocmanual}
% The conditional |\ifchilddoc| tells whether a
% child (true) or main (false) document is being compiled.
% The conditional |\ifchilddocmanual| tells whether
% the |\includeonly| mechanism is used (false) or
% the selection of child files must be performed manually (true).
% The definitions initialise to false:
%    \begin{macrocode}
\newif\ifchilddoc
\newif\ifchilddocmanual
%    \end{macrocode}

% \macro{\childdocname}
% \macro{\childdocjob}
% The macro |\childdocname| stores the name of the main document
% to be compiled. The macro |\childdocjob| stores the name of
% the document on which the \LaTeX{} compiler was originally invoked.
% The content of |\jobname| cannot be compared
% to filenames specified in the source due to different catcodes.
% The following code rescans |\jobname|, stores the result
% in |\childdocname| and saves a copy in |\childdocjob|:
%    \begin{macrocode}
\edef\childdocname{\scantokens\expandafter{\jobname\noexpand}}
\let\childdocjob\childdocname
%    \end{macrocode}

% \macro{\childdocdisable}
% The macro |\childdocdisable| prevents the main file
% from being processed more than once.
% At this stage, the main document command |\childdocmain|
% is assumed to be called once again where it should do nothing.
% Any subsequent call to it should prevent
% a secondary processing of the main document
% It overwrites the forwarding commands
% |\childdocof| and |\childdocforward|
% with empty macros to prevent further inclusions of the main document:
%    \begin{macrocode}
\newcommand{\childdocdisable}
{
  \renewcommand{\childdocmain}[1]{\renewcommand{\childdocmain}[1]{\endinput}}
  \renewcommand{\childdocof}[1]{}
  \renewcommand{\childdocby}[2][]{}
  \renewcommand{\childdocforward}[2][]{}
  \renewcommand{\childdocdisable}{}
}
%    \end{macrocode}

% \macro{\childdocmain}
% The macro |\childdocmain| is to be called at the top of the main file
% with nothing or the main filename (without extension) as argument.
% First, it breaks loops.
% If the argument is not empty and does not match |\childdocname|
% (which is set by the first inclusion of |childdoc.def|),
% |\ifchilddoc| is set to true, |\includeonly| is applied to the child file
% and |\jobname| is set to the main file
% (for proper handling of |.aux| files):
%    \begin{macrocode}
\newcommand{\childdocmain}[1]
{
  \childdocdisable\childdocmain{}
  \if?#1?\else
    \begingroup
      \def\childdoctmp{#1}
      \ifx\childdoctmp\childdocname
        \def\childdoctmp{}
      \else
        \def\childdoctmp
        {
          \childdoctrue
          \includeonly{\childdocname}
          \def\childdocjob{#1}
          \def\jobname{#1}
        }
      \fi
      \expandafter
    \endgroup
    \childdoctmp
  \fi
}
%    \end{macrocode}

% \macro{\childdocof}
% The command |\childdocof| redirects
% compilation to the main file |#1|.
%    \begin{macrocode}
\newcommand{\childdocof}[1]
{
  \childdocdisable
  \childdoctrue
  \includeonly{\childdocname}
  \def\jobname{#1}
  \def\childdocjob{#1}
  \input{#1}
}
%    \end{macrocode}

% \macro{\childdocby}
% The command |\childdocby| ....
%    \begin{macrocode}
\newcommand{\childdocby}[2][]
{
  \childdocdisable
  \childdoctrue
  \childdocmanualtrue
  \if?#1?\else
    \def\jobname{#2}
  \fi
  \def\childdocjob{#2}
  \input{#2}
  \endinput
}
%    \end{macrocode}

% \macro{\childdocforward}
% The command |\childdocforward| redirects
% compilation to the main file or
% (if the optional argument is given) a child file.
% Parameters are set as if the main file
% or a child file starting with |\childdocof| was compiled.
% Then compilation is handed over to the main file:
%    \begin{macrocode}
\newcommand{\childdocforward}[2][]
{
  \begingroup
    \if?#1?
      \def\childdoctmp
      {
        \def\childdocname{#2}
        \def\childdocjob{#2}
        \def\jobname{#2}
        \input{#2}
        \endinput
      }
    \else
      \def\childdoctmp
      {
        \childdocdisable
        \def\childdocname{#2}
        \childdoctrue
        \includeonly{#2}
        \def\childdocjob{#1}
        \def\jobname{#1}
        \input{#1}
        \endinput
      }
    \fi
    \expandafter
  \endgroup
  \childdoctmp
}
%    \end{macrocode}

% \macro{\childdocforwardprefix}
% The command |\childdocforwardprefix| redirects
% compilation to the main or a child file by means of a pattern.
% The prefix |#1| in the current filename is replaced by |#2|
% and the suffix of the current filename is kept
% (it is assumed that the filename does not contain the substring `|~~~|'
% which is used as a delimiter).
% Compilation is handed over to the new file by |\childdocforward|:
%    \begin{macrocode}
\newcommand{\childdocforwardprefix}[3][]
{
  \begingroup
    \def\childdocextract #2##1~~~{\def\childdoctmp{\childdocforward[#1]{#3##1}}}
    \expandafter\childdocextract\childdocname~~~
    \expandafter
  \endgroup
  \childdoctmp
}
%    \end{macrocode}

% \macro{\childdoc}
% The deprecated macro |\childdoc| is a legacy version of |\childdocmain|:
%    \begin{macrocode}
\newcommand{\childdoc}{\childdocmain}
%    \end{macrocode}

% \macro{\childdocredirect}
% The deprecated macro |\childdocredirect| is a legacy version
% of |\childdocforward| and |\childdocforwardprefix|:
%    \begin{macrocode}
\newcommand{\childdocredirect}[2][]
{
  \begingroup
    \if?#1?
      \def\childdoctmp{\childdocforward{#2}}
    \else
      \def\childdoctmp{\childdocforwardprefix{#1}{#2}}
    \fi
    \expandafter
  \endgroup
  \childdoctmp
}
%    \end{macrocode}

%\iffalse
%</package>
%\fi
%
\endinput
\childdocforward{cdocsamp}"|\\
% |latex -jobname cdocscl1 \|\\
% |  "% \iffalse
%
% childdoc.dtx Copyright (C) 2017-2018 Niklas Beisert
%
% This work may be distributed and/or modified under the
% conditions of the LaTeX Project Public License, either version 1.3
% of this license or (at your option) any later version.
% The latest version of this license is in
%   http://www.latex-project.org/lppl.txt
% and version 1.3 or later is part of all distributions of LaTeX
% version 2005/12/01 or later.
%
% This work has the LPPL maintenance status `maintained'.
%
% The Current Maintainer of this work is Niklas Beisert.
%
% This work consists of the files childdoc.dtx and childdoc.ins
% and the derived files childdoc.def and cdocsamp.tex with
% cdocsch1.tex, cdocsch2.tex, cdocsdrf.tex, cdocsfn1.tex, cdocsfn2.tex.
%
%<package>\ifdefined\childdocmain\endinput\fi
%<package>\ProvidesFile{childdoc.def}[2018/12/30 v2.0 child document driver]
%<samplemain>\ProvidesFile{cdocsamp.tex}[2018/12/30 v2.0 sample for childdoc]
%<*driver>
%\ProvidesFile{childdoc.drv}[2018/12/30 v2.0 childdoc reference manual file]
\PassOptionsToClass{10pt,a4paper}{article}
\documentclass{ltxdoc}

\usepackage[margin=35mm]{geometry}
\usepackage{hyperref}
\usepackage{hyperxmp}
\usepackage[usenames]{color}

\hypersetup{colorlinks=true}
\hypersetup{pdfstartview=FitH}
\hypersetup{pdfpagemode=UseNone}
\hypersetup{pdfsource={}}
\hypersetup{pdflang={en-UK}}
\hypersetup{pdfcopyright={Copyright 2017-2018 Niklas Beisert.
  This work may be distributed and/or modified under the
  conditions of the LaTeX Project Public License, either version 1.3
  of this license or (at your option) any later version.}}
\hypersetup{pdflicenseurl={http://www.latex-project.org/lppl.txt}}
\hypersetup{pdfcontactaddress={ETH Zurich, ITP, HIT K,
  Wolfgang-Pauli-Strasse 27}}
\hypersetup{pdfcontactpostcode={8093}}
\hypersetup{pdfcontactcity={Zurich}}
\hypersetup{pdfcontactcountry={Switzerland}}
\hypersetup{pdfcontactemail={nbeisert@itp.phys.ethz.ch}}
\hypersetup{pdfcontacturl={http://people.phys.ethz.ch/\xmptilde nbeisert/}}

\newcommand{\secref}[1]{\hyperref[#1]{section \ref*{#1}}}

\parskip1ex
\parindent0pt
\let\olditemize\itemize
\def\itemize{\olditemize\parskip0pt}

\begin{document}

\title{The \textsf{childdoc} Package}
\hypersetup{pdftitle={The childdoc Package}}
\author{Niklas Beisert\\[2ex]
  Institut f\"ur Theoretische Physik\\
  Eidgen\"ossische Technische Hochschule Z\"urich\\
  Wolfgang-Pauli-Strasse 27, 8093 Z\"urich, Switzerland\\[1ex]
  \href{mailto:nbeisert@itp.phys.ethz.ch}
  {\texttt{nbeisert@itp.phys.ethz.ch}}}
\hypersetup{pdfauthor={Niklas Beisert}}
\hypersetup{pdfsubject={Manual for the LaTeX2e Package childdoc}}
\date{30 December 2018, \textsf{v2.0}}
\maketitle

\begin{abstract}\noindent
\textsf{childdoc} is a \LaTeXe{} package
that enables the direct compilation
of document sections included by |\include|
to individual files.
\end{abstract}

\begingroup
\parskip0ex
\tableofcontents
\endgroup

%%%%%%%%%%%%%%%%%%%%%%%%%%%%%%%%%%%%%%%%%%%%%%%%%%%%%%%%%%%%%%%%%%%%%%%%%%%%%%%%
%%%%%%%%%%%%%%%%%%%%%%%%%%%%%%%%%%%%%%%%%%%%%%%%%%%%%%%%%%%%%%%%%%%%%%%%%%%%%%%%
\section{Introduction}

\LaTeX{} provides a mechanism to structure a large document (such as a book)
into a main file and several child files (containing the chapters)
using the |\include| command.
This mechanism is beneficial for documents
which span hundreds of pages in order to
make the source file(s) more manageable.
Moreover, compilation can be restricted to
selected child files by means of the |\includeonly| command.
The latter feature can be used to reduce the compilation time while editing
(this was significantly more useful in the earlier days of \LaTeX{})
or to generate a smaller document which is easier to navigate.
Another application of |\includeonly| is to generate
documents consisting of selected parts of the complete document.

However, there are a few drawbacks of the plain |\include| mechanism:
\begin{itemize}
\item
The child files cannot be compiled on their own,
they can only be compiled via the main file.
A naive editing environment
(such as a text editor with an option
to have the current file processed by \LaTeX)
may require one to switch to the main file before compiling;
attempting to compile the child file produces errors.
\item
The main file must be modified (each time)
to adjust the |\includeonly| command
to the present needs. This easily leaves the main file in a messy state.
\item
The generated document will always carry the filename
of the main document. This is inconvenient if
several child files are to be compiled and
to be kept for distribution.
\end{itemize}

The present package provides a simple interface
to make child files individually compilable by \LaTeX{}.
Compiling a child file then has the same effect as compiling
the main file with an |\includeonly| command
to select the appropriate child.
Moreover the generated document will carry the name of the child
rather than the main file.
This resolves all three above issues.

This feature is meant to make the editing of books,
thesis documents and lecture notes somewhat more convenient.
However, the package can also be used efficiently for
composing a series of documents (such as exercise sheets)
which are typically distributed individually.
It then assists the author in generating the individual documents
(potentially in different versions)
as well as a document containing the collected series.
Another application is in developing style files
or other kinds of included material
where compilation of the style file could redirect
to a sample or test file.

%%%%%%%%%%%%%%%%%%%%%%%%%%%%%%%%%%%%%%%%%%%%%%%%%%%%%%%%%%%%%%%%%%%%%%%%%%%%%%%%
%%%%%%%%%%%%%%%%%%%%%%%%%%%%%%%%%%%%%%%%%%%%%%%%%%%%%%%%%%%%%%%%%%%%%%%%%%%%%%%%
\section{Usage}

First of all, the package \textsf{childdoc} is \emph{not} a standard
\LaTeXe{} |.sty| style file! Therefore it needs to be invoked in
a non-standard way.

%%%%%%%%%%%%%%%%%%%%%%%%%%%%%%%%%%%%%%%%%%%%%%%%%%%%%%%%%%%%%%%%%%%%%%%%%%%%%%%%
\subsection{Included Files}
\label{sec:include}

%%%%%%%%%%%%%%%%%%%%%%%%%%%%%%%%%%%%%%%%
\DescribeMacro{\childdocmain}
To use the package, add the commands
\begin{center}
\begin{tabular}{l}
|\input{childdoc.def}|\\
|\childdocmain{}|\\
\end{tabular}
\end{center}
at the very top of the main \LaTeX{} file,
in particular \emph{before} the |\documentclass| statement!
The argument of |\childdocmain| should be left empty
(but it must be present).

%%%%%%%%%%%%%%%%%%%%%%%%%%%%%%%%%%%%%%%%
\DescribeMacro{\childdocof}
Furthermore, add the commands
\begin{center}
\begin{tabular}{l}
|\input{childdoc.def}|\\
|\childdocof{|\textit{main}|}|\\
\end{tabular}
\end{center}
at the top of every child file \textit{child}
which is included by |\include{|\textit{child}|}|
from within the main file
(or at least for those files to be compiled individually).
The argument \textit{main} must be the filename of the main file.

There are a couple of
considerations in setting up the main and child documents:

%%%%%%%%%%%%%%%%%%%%%%%%%%%%%%%%%%%%%%%%
\paragraph{Restrictions.}

Please note the following restrictions:
\begin{itemize}
\item
|\childdocmain| must be called with one argument \textit{main}
to ensure compatibility with earlier version of the package.
It must either be empty (|\childdocmain{}|)
or precisely match the filename of the main file in which it is specified.
See \secref{sec:detection} for further information.
\item
The filename \textit{main} must be specified without the |.tex| extension.
\item
The filename \textit{main} is case sensitive
(even in case-insensitive file systems)
due to internal string comparison.
\item
The argument \textit{main} should be fully expanded, it cannot be a macro.
\item
Subdirectories and special characters should be avoided in filenames.
\item
The command |\childdocmain{|\textit{main}|}| must be followed by a whitespace.
It should not be followed immediately by another command
or by a comment mark `|%|'.
This is because the \TeX{} parser reads the token immediately following
the argument of |\childdocmain| and puts it
at the beginning of every child section;
however, a white\-space is ignored.
\end{itemize}

%%%%%%%%%%%%%%%%%%%%%%%%%%%%%%%%%%%%%%%%
\paragraph{Content of Main File.}

It is advisable to place all content in the child files included by |\include|.
Any output contained in the main file will appear in all child documents
unless suppressed manually;
it cannot be suppressed automatically by the |\includeonly| directive
and thus should normally be avoided.
A method to include some content in the main file
by means of conditional processing is described in \secref{sec:conditional}.

%%%%%%%%%%%%%%%%%%%%%%%%%%%%%%%%%%%%%%%%
\paragraph{Page Numbering.}

When only a part of the document is compiled,
the appropriate numbering of pages
(as well as other status parameters)
is determined from the |.aux| files.
The latter contain information from previous passes.
However this information needs to propagate through
all intermediate child documents.
Therefore the page numbering in child documents may well
be inconsistent until the complete document is compiled at least once.

A useful (if unconventional) way to always ensure a consistent
page numbering is to restart the numbering in each child document
and denote the pages by `\textit{child}|.|\textit{page}'
where \textit{child} represents the chapter/section number of the child file.
This can be achieved by the command
|\numberwithin{page}{|\textit{child}|}|
of the \textsf{amsmath} package
where \textit{child} can be |chapter| or |section|
depending on the chosen structuring.
Alternatively, one can modify the macro |\thepage| appropriately
and reset the counter |page| at the start of each child file.

%%%%%%%%%%%%%%%%%%%%%%%%%%%%%%%%%%%%%%%%%%%%%%%%%%%%%%%%%%%%%%%%%%%%%%%%%%%%%%%%
\subsection{Conditional Processing}
\label{sec:conditional}

The package provides a mechanism to compile different versions
of a document. To customise the versions further some conditional processing
can come in handy to distinguish which version is being compiled.
The package provides two macros to describe the compilation context:

%%%%%%%%%%%%%%%%%%%%%%%%%%%%%%%%%%%%%%%%
\DescribeMacro{\ifchilddoc}
The conditional |\ifchilddoc| distinguishes between the compilation of
child documents and the main document:
%
\begin{center}
|\ifchilddoc |\textit{child-code}| |[|\||else |\textit{main-code}]| \||fi|
\end{center}

%%%%%%%%%%%%%%%%%%%%%%%%%%%%%%%%%%%%%%%%
\DescribeMacro{\childdocname}
\DescribeMacro{\childdocjob}
The macro |\childdocname| contains the filename (without extension)
of the main or child file being processed.
Note that |\childdocjob| will always contain the name of the main file.

%%%%%%%%%%%%%%%%%%%%%%%%%%%%%%%%%%%%%%%%
\paragraph{Title Page.}

Conditional processing can be used to include a title or banner page
in the main document when proper precautions are taken.
Importantly, the code in the main file should ensure that the page counter
(as well as other status parameters which are stored in the |.aux| files)
takes the same value after the conditional processing.
Otherwise the page numbers may take divergent values
depending on which part is compiled.

For example, a title page could be declared by:
%
\begin{center}
\begin{tabular}{l}
|\ifchilddoc\||else|\\
|\addtocounter{page}{-1}|\\
\textit{code for title page}\\
|\newpage|\\
|\||fi|
\end{tabular}
\end{center}
%
A banner page for the child documents can be generated by:
%
\begin{center}
\begin{tabular}{l}
|\ifchilddoc|\\
|\addtocounter{page}{-1}|\\
\textit{code for banner page}\\
|\newpage|\\
|\||fi|
\end{tabular}
\end{center}
%
Here one could write a message such as:
\begin{center}
|This is the part \childdocname{} of \childdocjob{}.|
\end{center}

%%%%%%%%%%%%%%%%%%%%%%%%%%%%%%%%%%%%%%%%%%%%%%%%%%%%%%%%%%%%%%%%%%%%%%%%%%%%%%%%
\subsection{Flags}
\label{sec:flags}

The package makes it easy to generate different versions
of the main or child documents.
To this end compilation flags can be defined
and assigned different default values.
They will be particularly useful in conjunction
with the forwarding mechanism described in \secref{sec:forward}.

For example, it may be useful to have a flag |\version|
which can be set to |draft| or |final|.
The document source will contain some conditional code
depending on the value of |\version|.
Suppose further, the flag should default to |final| for the main file
and to |draft| for child files
which is a natural assignment for editing the document.
This is achieved by placing the following code
in the preamble of the main document
(below the |\childdocmain| directive):
%
\begin{center}
\begin{tabular}{l}
|\ifchilddoc|\\
|\providecommand{\version}{draft}|\\
|\||else|\\
|\providecommand{\version}{final}|\\
|\||fi|
\end{tabular}
\end{center}
%
The definition by |\providecommand| makes sure
that previous definitions are not overwritten.
Further statements |\providecommand{\version}{...}|
can thus be added before the above code to override it.

For the main file, one might add a line
(between |\childdocmain| and the above block)
%
\begin{center}
|%\ifchilddoc\||else\providecommand{\version}{draft}\||fi|
\end{center}
%
which can be uncommented to produce a draft version.
Likewise one can add a line to the very top of a child file
(above the |\childdocof{|\textit{main}|}| directive)
%
\begin{center}
|%\providecommand{\version}{final}|
\end{center}
%
which can be uncommented to produce the final version of this child document.

%%%%%%%%%%%%%%%%%%%%%%%%%%%%%%%%%%%%%%%%%%%%%%%%%%%%%%%%%%%%%%%%%%%%%%%%%%%%%%%%
\subsection{Forwarding}
\label{sec:forward}

Different versions of the main or child documents
using compilation flags as described in \secref{sec:flags}
can be (permanently) stored in different files
for convenient compilation, viewing and distribution.
To this end, the package defines a command
to pass on compilation to a different file:

%%%%%%%%%%%%%%%%%%%%%%%%%%%%%%%%%%%%%%%%
\DescribeMacro{\childdocforward}
The command |\childdocforward| redirects processing to
another source file:
%
\begin{center}
\begin{tabular}{l}
|\input{childdoc.def}|\\
|\childdocforward[|\textit{main}|]{|\textit{dest}|}|\\
\end{tabular}
\end{center}
%
The argument \textit{dest} is the destination file
(without extension).
It should be the main file or one of the child files.
Note that further \textsf{childdoc} directives
such as |\childdocof| and |\childdocforward|
in the indicated file will be processed in this form.
The optional argument \textit{main}
passes on directly to the main file \textit{main}
while pretending to compile the child \textit{dest}.
This form behaves as if \textit{dest}
issues |\childdocof{|\textit{main}|}| right away,
and no further \textsf{childdoc} directives will be processed.

%%%%%%%%%%%%%%%%%%%%%%%%%%%%%%%%%%%%%%%%
\DescribeMacro{\...prefix}
In the alternative form |\childdocforwardprefix|,
%
\begin{center}
\begin{tabular}{l}
|\input{childdoc.def}|\\
|\childdocforwardprefix[|\textit{main}|]{|\textit{prefix}|}{|\textit{dest}|}|
\end{tabular}
\end{center}
%
the destination file is determined by a pattern
depending on the current file:
To make this work, the current file must be called
`{\textit{prefix}\hspace{0.2em}\textit{suffix}}'
with \textit{prefix} matching precisely the argument.
Processing is then passed on to the file
`{\textit{dest}\hspace{0.2em}\textit{suffix}}'.
Surely, the same effect is achieved by
directly specifying the
argument `{\textit{dest}\hspace{0.2em}\textit{suffix}}'
in the first form.
However, that requires to set up a different file
for each child. With the alternative form of the command
all these files can have exactly the same content
which simplifies setting them up and maintaining them.

For example, the following file |draft.tex|
with a compilation flag |\version| as described in \secref{sec:flags}
compiles the main document as a draft:
%
\begin{center}
\begin{tabular}{l}
|\def\version{draft}|\\
|\input{childdoc.def}|\\
|\childdocforward{|\textit{main}|}|
\end{tabular}
\end{center}
%
Likewise, the following files |final|\textit{nn}|.tex|
compile the final version of the child document
|child|\textit{nn}|.tex|:
%
\begin{center}
\begin{tabular}{l}
|\def\version{final}|\\
|\input{childdoc.def}|\\
|\childdocforwardprefix{final}{child}|
\end{tabular}
\end{center}
%

Note that when several versions of a main file and/or of each child file
are to be generated, it may be convenient to set up a |Makefile| or
shell script to automatise the process.

%%%%%%%%%%%%%%%%%%%%%%%%%%%%%%%%%%%%%%%%%%%%%%%%%%%%%%%%%%%%%%%%%%%%%%%%%%%%%%%%
\subsection{Command Line Processing}
\label{sec:commandline}

The effect of redirection files can also be achieved by invoking
the \LaTeX{} compiler with a more elaborate command line.
Most conveniently this should be done as part
of a shell script or a |Makefile|.

When using \textsf{childdoc} in the main file, the following
command lines effectively perform a redirection
(note that depending on the shell being used,
backslashes may have to be doubled: `|\|' $\to$ `|\\|'):
%
\begin{center}
|... -jobname "|\textit{target}|" |\\|"|[\textit{flags}]%
|\input{childdoc.def}\childdocforward[|\textit{main}|]{|\textit{dest}|}"|
\end{center}
%
Here \textit{target} is the name of the output file,
\textit{main} is the name of the main file
and \textit{dest} is the name of the main or child file to be processed
(all filenames without extensions).
The optional argument \textit{main} can be omitted
if \textit{main} matches \textit{dest}.
Optionally, compilation \textit{flags} can be defined via |\def| commands.
This command line makes the \TeX{} engine believe
it is compiling the file \textit{target}
whose content is specified as the latter parameter.
The provided code then forwards the processing to
\textit{main} or \textit{dest} as described in \secref{sec:forward}.

%%%%%%%%%%%%%%%%%%%%%%%%%%%%%%%%%%%%%%%%%%%%%%%%%%%%%%%%%%%%%%%%%%%%%%%%%%%%%%%%
\subsection{Include by Input}
\label{sec:input}

Including child documents by |\include| has some restrictions by design.
Most notably, the content of a child document always occupies
its own set of pages; pages cannot be shared between child documents.
Usually, this behaviour makes perfect sense
because each child document contain an essential part of the document.
However, in some situations it may be desirable to compose
a document from a collection of parts
without having mandatory page breaks between then.
For this case, the package
provides a mechanism to include parts
by |\input| which can also be processed individually.
However, by construction this mechanism
requires manual handling of the content to be output.

%%%%%%%%%%%%%%%%%%%%%%%%%%%%%%%%%%%%%%%%
\DescribeMacro{\ifchilddocmanual}
The main file should be prepared as usual, see \secref{sec:include}.
However, the document body must make a distinction
between processing of an individual part and of the main document, e.g.:
%
\begin{center}
\begin{tabular}{l}
|\ifchilddocmanual|\\
|\input{\childdocname}|\\
|\||else|\\
\textit{document body with }|\input{|\textit{part}|}|\\
|\||fi|
\end{tabular}
\end{center}
%
The conditional |\ifchilddocmanual| is true whenever
a part to be included by |\input| is being compiled,
and the name of the part is stored in |\childdocname|.

%%%%%%%%%%%%%%%%%%%%%%%%%%%%%%%%%%%%%%%%
\DescribeMacro{\childdocby}
Each part to be included by |\input| should start with:
%
\begin{center}
\begin{tabular}{l}
|\input{childdoc.def}|\\
|\childdocby{|\textit{main}|}|\\
\end{tabular}
\end{center}
%
The directive |\childdocby| is similar to |\childdocof|
described in \secref{sec:include},
but the subsequent selection of content must be done manually.
To that end, both |\ifchilddoc| and |\ifchilddocmanual|
will be true upon processing of a part,
and the name of the part is stored in |\childdocname|.
Note that |\jobname| will be set to the filename of the current part
so that each part receives an individual |.aux| file
that does not interfere with the |.aux| file(s) of the main document.
This behaviour can be altered by the alternative form
|\childdocby[*]{|\textit{main}|}| (with a non-empty optional argument)
which uses the |.aux| file of the main document
by setting |\jobname| to \textit{main}.

%%%%%%%%%%%%%%%%%%%%%%%%%%%%%%%%%%%%%%%%%%%%%%%%%%%%%%%%%%%%%%%%%%%%%%%%%%%%%%%%
\subsection{Driver Development}
\label{sec:driver}

The \textsf{childdoc} mechanism can also be use for the development
of definition files such as \LaTeX{} styles or classes.
This case differs from the above setup with multiple parts
included by |\include| in that no |\includeonly| should be invoked.
This can be achieved by starting the include file
(before |\ProvidesPackage|) with:
%
\begin{center}
\begin{tabular}{l}
|\input{childdoc.def}|\\
|\childdocforward{|\textit{main}|}|\\
\end{tabular}
\end{center}
%
or alternatively with:
%
\begin{center}
\begin{tabular}{l}
|\input{childdoc.def}|\\
|\childdocby{|\textit{main}|}|\\
\end{tabular}
\end{center}
%
Both forms have slightly different effects as described above.
The main file is prepared as usual, see \secref{sec:include}.

%%%%%%%%%%%%%%%%%%%%%%%%%%%%%%%%%%%%%%%%%%%%%%%%%%%%%%%%%%%%%%%%%%%%%%%%%%%%%%%%
\subsection{Legacy Detection}
\label{sec:detection}

The directive |\childdocmain| in the main file can detect
whether the complete document or merely a child is to be compiled
even without using the directive |\childdocof|.
This method is deprecated because it is less robust
and there is no compelling reason to use it;
it is merely provided for backward compatibility
and it may be removed in future versions.

If the detection mechanism is to be used,
it is mandatory to correctly specify
the filename of the main file as the argument of |\childdocmain|:
%
\begin{center}
\begin{tabular}{l}
|\input{childdoc.def}|\\
|\childdocmain{|\textit{main}|}|\\
\end{tabular}
\end{center}
%
If |\jobname| does not match the argument \textit{main} of |\childdocmain|,
it is assumed that |\jobname| points to the child file to be compiled.
When using |\childdocmain| with the main file specified as argument,
it suffices to start a child file
with just |\input{|\textit{main}|}|
without loading of the package and using |\childdocof|.
If instead all processing is done
with the appropriate \textsf{childdoc} directives,
the argument of \textit{main} of |\childdocmain| can be empty.

An alternative version of the command line processing described
in \secref{sec:commandline} using the detection mechanism reads:
%
\begin{center}
|... -jobname "|\textit{target}|" "|[\textit{flags}]%
[|\def\jobname{|\textit{dest}|}|]|\input{|\textit{main}|}"|
\end{center}

%%%%%%%%%%%%%%%%%%%%%%%%%%%%%%%%%%%%%%%%%%%%%%%%%%%%%%%%%%%%%%%%%%%%%%%%%%%%%%%%
\subsection{Manual Code}
\label{sec:manual}

In case one cannot be certain whether the definitions file |childdoc.def|
is installed on the target \TeX{} distribution
and one prefers not to ship it,
it is conceivable to paste a few relevant commands into the sources.

To that end, drop all statements |\input{childdoc.def}|
and perform the replacements as outlined below.
Instead of |\childdocmain{|\textit{main}|}| add the following code
to the top of the main file:
%
\begin{center}
\begin{tabular}{l}
|\||ifdefined\childdocname\endinput\||fi\newif\ifchilddoc|\\
|\edef\childdocname{\scantokens\expandafter{\jobname\noexpand}}|\\
|\def\childdocmain{|\textit{main}|}\||ifx\childdocmain\childdocname\||else|\\
|\childdoctrue\includeonly{\childdocname}\let\jobname\childdocmain\||fi|\\
\end{tabular}
\end{center}
%
Instead of |\childdocof{|\textit{main}|}| just include the main file
at the top of each child file:
%
\begin{center}
|\input{|\textit{main}|}|
\end{center}
%
A simple redirection |\childdocforward{|\textit{dest}|}| is achieved by:
%
\begin{center}
|\def\jobname{|\textit{dest}|}\input{\jobname}|
\end{center}
%
The redirection with prefix
|\childdocforwardprefix[|\textit{prefix}|]{|\textit{dest}|}|
is accomplished by:
%
\begin{center}
\begin{tabular}{l}
|{\edef\jobname{\scantokens\expandafter{\jobname\noexpand}}|\\
|\def\redirectjob |\textit{prefix}|#1~~~{\gdef\jobname{|\textit{dest}|#1}}|\\
|\expandafter\redirectjob\jobname~~~}\input{\jobname}|
\end{tabular}
\end{center}

In an alternative approach,
child documents can be compiled by a specific command line
without additional code or specific definitions:
%
\begin{center}
|... -jobname "|\textit{target}|" "|[\textit{flags}]%
|\includeonly{|\textit{dest}|}\input{|\textit{main}|}"|
\end{center}
%

%%%%%%%%%%%%%%%%%%%%%%%%%%%%%%%%%%%%%%%%%%%%%%%%%%%%%%%%%%%%%%%%%%%%%%%%%%%%%%%%
%%%%%%%%%%%%%%%%%%%%%%%%%%%%%%%%%%%%%%%%%%%%%%%%%%%%%%%%%%%%%%%%%%%%%%%%%%%%%%%%
\section{Information}

%%%%%%%%%%%%%%%%%%%%%%%%%%%%%%%%%%%%%%%%%%%%%%%%%%%%%%%%%%%%%%%%%%%%%%%%%%%%%%%%
\subsection{Copyright}

Copyright \copyright{} 2017--2018 Niklas Beisert

This work may be distributed and/or modified under the
conditions of the \LaTeX{} Project Public License, either version 1.3
of this license or (at your option) any later version.
The latest version of this license is in
  \url{http://www.latex-project.org/lppl.txt}
and version 1.3 or later is part of all distributions of \LaTeX{}
version 2005/12/01 or later.

This work has the LPPL maintenance status `maintained'.

The Current Maintainer of this work is Niklas Beisert.

This work consists of the files |README.txt|, |childdoc.ins| and |childdoc.dtx|
as well as the derived files |childdoc.def|, |cdocsamp.tex|
with |cdocsch1.tex|, |cdocsch2.tex|, |cdocspt3.tex|, |cdocspt4.tex|,
|cdocsdrf.tex|, |cdocsfn1.tex|, |cdocsfn2.tex|
as well as |childdoc.pdf|.

%%%%%%%%%%%%%%%%%%%%%%%%%%%%%%%%%%%%%%%%%%%%%%%%%%%%%%%%%%%%%%%%%%%%%%%%%%%%%%%%
\subsection{Files and Installation}

The package consists of the files:
%
\begin{center}
\begin{tabular}{ll}
    |README.txt|   & readme file \\
    |childdoc.ins| & installation file \\
    |childdoc.dtx| & source file \\
    |childdoc.def| & definition file \\
    |cdocsamp.tex| & sample main file \\
    |cdocsch1.tex| & sample include file \\
    |cdocsch2.tex| & sample include file \\
    |cdocspt3.tex| & sample part file \\
    |cdocspt4.tex| & sample part file \\
    |cdocsdrf.tex| & sample redirection file \\
    |cdocsfn1.tex| & sample redirection file \\
    |cdocsfn2.tex| & sample redirection file \\
    |childdoc.pdf| & manual
\end{tabular}
\end{center}
%
The distribution consists of the files
|README.txt|, |childdoc.ins| and |childdoc.dtx|.
%
\begin{itemize}
\item
Run (pdf)\LaTeX{} on |childdoc.dtx|
to compile the manual |childdoc.pdf| (this file).
\item
Run \LaTeX{} on |childdoc.ins| to create the definitions file |childdoc.def|
and the sample |cdocsamp.tex| with include files
|cdocsch1.tex|, |cdocsch2.tex|, |cdocspt3.tex|, |cdocspt4.tex|,
|cdocsdrf.tex|, |cdocsfn1.tex|, |cdocsfn2.tex|.
Then copy the file |childdoc.def| to an appropriate directory of your \LaTeX{}
distribution, e.g.\ \textit{texmf-root}|/tex/latex/childdoc|.
\end{itemize}

%%%%%%%%%%%%%%%%%%%%%%%%%%%%%%%%%%%%%%%%%%%%%%%%%%%%%%%%%%%%%%%%%%%%%%%%%%%%%%%%
\subsection{Related CTAN Packages}

There are several other packages which offer a similar functionality:
%
\begin{itemize}
\item
The packages
\href{http://ctan.org/pkg/docmute}{\textsf{docmute}},
\href{http://ctan.org/pkg/includex}{\textsf{includex}} and
\href{http://ctan.org/pkg/standalone}{\textsf{standalone}}
provide commands to include only the document body of
a child file thus allowing both files to be compiled individually.
\item
The packages \href{http://ctan.org/pkg/subdocs}{\textsf{subdocs}}
and \href{http://ctan.org/pkg/subfiles}{\textsf{subfiles}}
provide structures in which the main and child documents can be
encapsulated and allowing them to be compiled individually.
The inclusion mechanism is different from the conventional |\include|.
\item
The package \href{http://ctan.org/pkg/combine}{\textsf{combine}}
is an elaborate solution to combine several documents into one.
\end{itemize}
%
See also the CTAN topic \href{http://ctan.org/topic/subdocs}{\textsf{subdocs}}
for further related packages.
The present package differs from the above solutions in that
a document structure constructed with the conventional |\include| mechanism
just needs two extra commands at the top of every file
such that all constituent files can be compiled individually.

%%%%%%%%%%%%%%%%%%%%%%%%%%%%%%%%%%%%%%%%%%%%%%%%%%%%%%%%%%%%%%%%%%%%%%%%%%%%%%%%
%\subsection{Feature Suggestions}
%
%The following is a list of features which may be useful for future
%versions of this package:
%%
%\begin{itemize}
%\item
%\ldots
%\end{itemize}

%%%%%%%%%%%%%%%%%%%%%%%%%%%%%%%%%%%%%%%%%%%%%%%%%%%%%%%%%%%%%%%%%%%%%%%%%%%%%%%%
\subsection{Revision History}

%%%%%%%%%%%%%%%%%%%%%%%%%%%%%%%%%%%%%%%%
\paragraph{v2.0:} 2018/12/30

\begin{itemize}
\item
immediate forward processing
\item
added |\childdocby| mechanism
\item
manual restructured
\end{itemize}

%%%%%%%%%%%%%%%%%%%%%%%%%%%%%%%%%%%%%%%%
\paragraph{v1.6:} 2018/01/17

\begin{itemize}
\item
application for development of include files
\item
corrections to manual
\end{itemize}

%%%%%%%%%%%%%%%%%%%%%%%%%%%%%%%%%%%%%%%%
\paragraph{v1.5:} 2017/05/21

\begin{itemize}
\item
more complete structuring introduced
\item
|\childdocof| introduced
\item
|\childdoc| renamed to |\childdocmain|
\item
|\childredirect| renamed to |\childdocforward| and |\childdocforwardprefix|
and functionality expanded
\end{itemize}

%%%%%%%%%%%%%%%%%%%%%%%%%%%%%%%%%%%%%%%%
\paragraph{v1.0:} 2017/04/27

\begin{itemize}
\item
manual and install package
\item
first version published on CTAN
\end{itemize}

%%%%%%%%%%%%%%%%%%%%%%%%%%%%%%%%%%%%%%%%
\paragraph{v0.6:} 2017/04/26

\begin{itemize}
\item
redirection mechanism added
\end{itemize}

%%%%%%%%%%%%%%%%%%%%%%%%%%%%%%%%%%%%%%%%
\paragraph{v0.5:} 2017/04/26

\begin{itemize}
\item
functionality in definition file
\end{itemize}


%%%%%%%%%%%%%%%%%%%%%%%%%%%%%%%%%%%%%%%%%%%%%%%%%%%%%%%%%%%%%%%%%%%%%%%%%%%%%%%%
%%%%%%%%%%%%%%%%%%%%%%%%%%%%%%%%%%%%%%%%%%%%%%%%%%%%%%%%%%%%%%%%%%%%%%%%%%%%%%%%
%%%%%%%%%%%%%%%%%%%%%%%%%%%%%%%%%%%%%%%%%%%%%%%%%%%%%%%%%%%%%%%%%%%%%%%%%%%%%%%%
\appendix

\settowidth\MacroIndent{\rmfamily\scriptsize 000\ }

 \DocInput{childdoc.dtx}

\end{document}
%</driver>
% \fi
%
% %%%%%%%%%%%%%%%%%%%%%%%%%%%%%%%%%%%%%%%%%%%%%%%%%%%%%%%%%%%%%%%%%%%%%%%%%%%%%%
% %%%%%%%%%%%%%%%%%%%%%%%%%%%%%%%%%%%%%%%%%%%%%%%%%%%%%%%%%%%%%%%%%%%%%%%%%%%%%%
% \section{Sample}
%\iffalse
%<*samplemain>
%\fi
%
% The following presents a sample document
% with two chapters, two parts, a title page,
% a compile flag as well as three forwarding files to set the flag.
% It consists of eight |.tex| files:
% \begin{center}
% \begin{tabular}{ll}
% |cdocsamp.tex|&main file\\
% |cdocsch1.tex|&include file for chapter 1\\
% |cdocsch2.tex|&include file for chapter 2\\
% |cdocspt3.tex|&include file for part 3\\
% |cdocspt4.tex|&include file for part 4\\
% |cdocsdrf.tex|&forwarding file for main file in draft mode\\
% |cdocsfi1.tex|&forwarding file for final version of chapter 1\\
% |cdocsfi2.tex|&forwarding file for final version of chapter 2\\
% \end{tabular}
% \end{center}
% Each of the eight files can be compiled directly by the \LaTeX{} compiler.
%
% %%%%%%%%%%%%%%%%%%%%%%%%%%%%%%%%%%%%%%
% \paragraph{Main File.}
%
% The main file is called |cdocsamp.tex|.
%
% Load the \textsf{childdoc} definitions and
% declare the filename for the main document:
%    \begin{macrocode}
\input{childdoc.def}
\childdocmain{}
%    \end{macrocode}

% Optional override for |\version| flag:
%    \begin{macrocode}
%%\ifchilddoc\else\providecommand{\version}{draft}\fi
%    \end{macrocode}

% Define the default values for the |\version| flag
% (|final| for the main file and |draft| for childs):
%    \begin{macrocode}
\ifchilddoc
\providecommand{\version}{draft}
\else
\providecommand{\version}{final}
\fi
%    \end{macrocode}

% Load the standard document class:
%    \begin{macrocode}
\documentclass[12pt]{article}
%    \end{macrocode}

% Start the document body:
%    \begin{macrocode}
\begin{document}
%    \end{macrocode}

% Declare a title page.
% Print title, part of document being processed and version flag:
%    \begin{macrocode}
\addtocounter{page}{-1}
\begin{center}
{\LARGE\bfseries{}childdoc example\par}
\vspace{1cm}
\ifchilddoc
\ifchilddocmanual part\else chapter\fi:
`\childdocname' of `\childdocjob'\par
\else
main document: `\childdocjob'\par
\fi
version: \version\par
\end{center}
\newpage
%    \end{macrocode}

% Manually include selected file,
% otherwise process as usual:
%    \begin{macrocode}
\ifchilddocmanual
\section*{part `\childdocname'}
\input{\childdocname}
\else
%    \end{macrocode}

% Include the two chapters:
%    \begin{macrocode}
\include{cdocsch1}
\include{cdocsch2}
%    \end{macrocode}

% Include the two parts unless only chapters should be displayed:
%    \begin{macrocode}
\ifchilddoc\else
\section{part three}
\input{cdocspt3}
\section{part four}
\input{cdocspt4}
\fi
%    \end{macrocode}

% Process as usual until here:
%    \begin{macrocode}
\fi
%    \end{macrocode}

% End of document body:
%    \begin{macrocode}
\end{document}
%    \end{macrocode}
%\iffalse
%</samplemain>
%\fi
%
% %%%%%%%%%%%%%%%%%%%%%%%%%%%%%%%%%%%%%%
% \paragraph{Chapter Include Files.}
%
% The include files are called |cdocsch1.tex| and |cdocsch2.tex|.
%
%\iffalse
%<*samplechap1|samplechap2>
%\fi

% Optional override for |\version| flag:
%    \begin{macrocode}
%%\providecommand{\version}{final}
%    \end{macrocode}

% Include the main document:
%    \begin{macrocode}
\input{childdoc.def}
\childdocof{cdocsamp}
%    \end{macrocode}

%\iffalse
%</samplechap1|samplechap2>
%\fi
%
%\iffalse
%<*samplechap1>
%\fi
% Some text for chapter 1:
%    \begin{macrocode}
\section{one}
some text in chapter one
%    \end{macrocode}

%\iffalse
%</samplechap1>
%\fi
% Some text for chapter 2:
%\iffalse
%<*samplechap2>
%\fi
%    \begin{macrocode}
\section{two}
more text in chapter two
%    \end{macrocode}

%\iffalse
%</samplechap2>
%\fi
%
% %%%%%%%%%%%%%%%%%%%%%%%%%%%%%%%%%%%%%%
% \paragraph{Part Include Files.}
%
% The include files are called |cdocspt3.tex| and |cdocspt4.tex|.
%
%\iffalse
%<*samplepart3|samplepart4>
%\fi

% Optional override for |\version| flag:
%    \begin{macrocode}
%%\providecommand{\version}{final}
%    \end{macrocode}

% Include the main document:
%    \begin{macrocode}
\input{childdoc.def}
\childdocby{cdocsamp}
%    \end{macrocode}

%\iffalse
%</samplepart3|samplepart4>
%\fi
%
%\iffalse
%<*samplepart3>
%\fi
% Some text for part 3:
%    \begin{macrocode}
some text in part three
%    \end{macrocode}

%\iffalse
%</samplepart3>
%\fi
% Some text for part 4:
%\iffalse
%<*samplepart4>
%\fi
%    \begin{macrocode}
more text in part four
%    \end{macrocode}

%\iffalse
%</samplepart4>
%\fi
%
% %%%%%%%%%%%%%%%%%%%%%%%%%%%%%%%%%%%%%%
% \paragraph{Forwarding for a Complete Draft.}
%
% The following forwarding file |cdocsdrf.tex|
% compiles the main document in draft mode:
%\iffalse
%<*sampledraft>
%\fi
%    \begin{macrocode}
\def\version{draft}
\input{childdoc.def}
\childdocforward{cdocsamp}
%    \end{macrocode}

%\iffalse
%</sampledraft>
%\fi
%
% %%%%%%%%%%%%%%%%%%%%%%%%%%%%%%%%%%%%%%
% \paragraph{Forwarding for Final Version of the Chapters.}
%
% The following forwarding files |cdocsfn1.tex| and |cdocsfn2.tex|
% (with identical content)
% compile the final versions of the child documents
% |cdocsch1.tex| and |cdocsch2.tex|, respectively:
%\iffalse
%<*samplefinal>
%\fi
%    \begin{macrocode}
\def\version{final}
\input{childdoc.def}
\childdocforwardprefix[cdocsamp]{cdocsfn}{cdocsch}
%    \end{macrocode}

%\iffalse
%</samplefinal>
%\fi
%
% %%%%%%%%%%%%%%%%%%%%%%%%%%%%%%%%%%%%%%
% \paragraph{Command Line Processing.}
%
% The following three command lines generate the output files
% |cdocscld|, |cdocscl1| and |cdocscl2|
% which should be identical to
% |cdocsdrf|, |cdocsch1| and |cdocsfn2|, respectively:
% \begin{center}
% \begin{tabular}{l}
% |latex -jobname cdocscld \|\\
% |  "\def\version{draft}\input{childdoc.def}\childdocforward{cdocsamp}"|\\
% |latex -jobname cdocscl1 \|\\
% |  "\input{childdoc.def}\childdocforward[cdocsamp]{cdocsch1}"|\\
% |latex -jobname cdocscl2 \|\\
% |  "\def\version{final}\input{childdoc.def}\childdocforward{cdocsch2}"|
% \end{tabular}
% \end{center}
% Note that the trailing backslash on each first line
% merely continues the input to the second line
% (for convenient cut ant paste).
% Furthermore, the command |latex| can be replaced by any
% of its alternative versions such as |pdflatex|.
%
% %%%%%%%%%%%%%%%%%%%%%%%%%%%%%%%%%%%%%%%%%%%%%%%%%%%%%%%%%%%%%%%%%%%%%%%%%%%%%%
% %%%%%%%%%%%%%%%%%%%%%%%%%%%%%%%%%%%%%%%%%%%%%%%%%%%%%%%%%%%%%%%%%%%%%%%%%%%%%%
% \section{Implementation}
%\iffalse
%<*package>
%\fi
%
% This section describes the definitions file |childdoc.def|.

% The definitions cannot be loaded using |\usepackage| or |\RequirePackage|
% which has a mechanism to prevent loading a style file more than once.
% When loading the definitions by means of |\input|
% multiple instances have to be prevented manually:
%\iffalse
%This code needs to be before the `\ProvidesFile' directive
%which is defined at the beginning of this file.
%Therefore it is also placed there and commented out here.
%</package>
%<*discard>
%\fi
%    \begin{macrocode}
\ifdefined\childdocmain\endinput\fi
%    \end{macrocode}
%\iffalse
%</discard>
%<*package>
%\fi
%
% \macro{\ifchilddoc}
% \macro{\ifchilddocmanual}
% The conditional |\ifchilddoc| tells whether a
% child (true) or main (false) document is being compiled.
% The conditional |\ifchilddocmanual| tells whether
% the |\includeonly| mechanism is used (false) or
% the selection of child files must be performed manually (true).
% The definitions initialise to false:
%    \begin{macrocode}
\newif\ifchilddoc
\newif\ifchilddocmanual
%    \end{macrocode}

% \macro{\childdocname}
% \macro{\childdocjob}
% The macro |\childdocname| stores the name of the main document
% to be compiled. The macro |\childdocjob| stores the name of
% the document on which the \LaTeX{} compiler was originally invoked.
% The content of |\jobname| cannot be compared
% to filenames specified in the source due to different catcodes.
% The following code rescans |\jobname|, stores the result
% in |\childdocname| and saves a copy in |\childdocjob|:
%    \begin{macrocode}
\edef\childdocname{\scantokens\expandafter{\jobname\noexpand}}
\let\childdocjob\childdocname
%    \end{macrocode}

% \macro{\childdocdisable}
% The macro |\childdocdisable| prevents the main file
% from being processed more than once.
% At this stage, the main document command |\childdocmain|
% is assumed to be called once again where it should do nothing.
% Any subsequent call to it should prevent
% a secondary processing of the main document
% It overwrites the forwarding commands
% |\childdocof| and |\childdocforward|
% with empty macros to prevent further inclusions of the main document:
%    \begin{macrocode}
\newcommand{\childdocdisable}
{
  \renewcommand{\childdocmain}[1]{\renewcommand{\childdocmain}[1]{\endinput}}
  \renewcommand{\childdocof}[1]{}
  \renewcommand{\childdocby}[2][]{}
  \renewcommand{\childdocforward}[2][]{}
  \renewcommand{\childdocdisable}{}
}
%    \end{macrocode}

% \macro{\childdocmain}
% The macro |\childdocmain| is to be called at the top of the main file
% with nothing or the main filename (without extension) as argument.
% First, it breaks loops.
% If the argument is not empty and does not match |\childdocname|
% (which is set by the first inclusion of |childdoc.def|),
% |\ifchilddoc| is set to true, |\includeonly| is applied to the child file
% and |\jobname| is set to the main file
% (for proper handling of |.aux| files):
%    \begin{macrocode}
\newcommand{\childdocmain}[1]
{
  \childdocdisable\childdocmain{}
  \if?#1?\else
    \begingroup
      \def\childdoctmp{#1}
      \ifx\childdoctmp\childdocname
        \def\childdoctmp{}
      \else
        \def\childdoctmp
        {
          \childdoctrue
          \includeonly{\childdocname}
          \def\childdocjob{#1}
          \def\jobname{#1}
        }
      \fi
      \expandafter
    \endgroup
    \childdoctmp
  \fi
}
%    \end{macrocode}

% \macro{\childdocof}
% The command |\childdocof| redirects
% compilation to the main file |#1|.
%    \begin{macrocode}
\newcommand{\childdocof}[1]
{
  \childdocdisable
  \childdoctrue
  \includeonly{\childdocname}
  \def\jobname{#1}
  \def\childdocjob{#1}
  \input{#1}
}
%    \end{macrocode}

% \macro{\childdocby}
% The command |\childdocby| ....
%    \begin{macrocode}
\newcommand{\childdocby}[2][]
{
  \childdocdisable
  \childdoctrue
  \childdocmanualtrue
  \if?#1?\else
    \def\jobname{#2}
  \fi
  \def\childdocjob{#2}
  \input{#2}
  \endinput
}
%    \end{macrocode}

% \macro{\childdocforward}
% The command |\childdocforward| redirects
% compilation to the main file or
% (if the optional argument is given) a child file.
% Parameters are set as if the main file
% or a child file starting with |\childdocof| was compiled.
% Then compilation is handed over to the main file:
%    \begin{macrocode}
\newcommand{\childdocforward}[2][]
{
  \begingroup
    \if?#1?
      \def\childdoctmp
      {
        \def\childdocname{#2}
        \def\childdocjob{#2}
        \def\jobname{#2}
        \input{#2}
        \endinput
      }
    \else
      \def\childdoctmp
      {
        \childdocdisable
        \def\childdocname{#2}
        \childdoctrue
        \includeonly{#2}
        \def\childdocjob{#1}
        \def\jobname{#1}
        \input{#1}
        \endinput
      }
    \fi
    \expandafter
  \endgroup
  \childdoctmp
}
%    \end{macrocode}

% \macro{\childdocforwardprefix}
% The command |\childdocforwardprefix| redirects
% compilation to the main or a child file by means of a pattern.
% The prefix |#1| in the current filename is replaced by |#2|
% and the suffix of the current filename is kept
% (it is assumed that the filename does not contain the substring `|~~~|'
% which is used as a delimiter).
% Compilation is handed over to the new file by |\childdocforward|:
%    \begin{macrocode}
\newcommand{\childdocforwardprefix}[3][]
{
  \begingroup
    \def\childdocextract #2##1~~~{\def\childdoctmp{\childdocforward[#1]{#3##1}}}
    \expandafter\childdocextract\childdocname~~~
    \expandafter
  \endgroup
  \childdoctmp
}
%    \end{macrocode}

% \macro{\childdoc}
% The deprecated macro |\childdoc| is a legacy version of |\childdocmain|:
%    \begin{macrocode}
\newcommand{\childdoc}{\childdocmain}
%    \end{macrocode}

% \macro{\childdocredirect}
% The deprecated macro |\childdocredirect| is a legacy version
% of |\childdocforward| and |\childdocforwardprefix|:
%    \begin{macrocode}
\newcommand{\childdocredirect}[2][]
{
  \begingroup
    \if?#1?
      \def\childdoctmp{\childdocforward{#2}}
    \else
      \def\childdoctmp{\childdocforwardprefix{#1}{#2}}
    \fi
    \expandafter
  \endgroup
  \childdoctmp
}
%    \end{macrocode}

%\iffalse
%</package>
%\fi
%
\endinput
\childdocforward[cdocsamp]{cdocsch1}"|\\
% |latex -jobname cdocscl2 \|\\
% |  "\def\version{final}% \iffalse
%
% childdoc.dtx Copyright (C) 2017-2018 Niklas Beisert
%
% This work may be distributed and/or modified under the
% conditions of the LaTeX Project Public License, either version 1.3
% of this license or (at your option) any later version.
% The latest version of this license is in
%   http://www.latex-project.org/lppl.txt
% and version 1.3 or later is part of all distributions of LaTeX
% version 2005/12/01 or later.
%
% This work has the LPPL maintenance status `maintained'.
%
% The Current Maintainer of this work is Niklas Beisert.
%
% This work consists of the files childdoc.dtx and childdoc.ins
% and the derived files childdoc.def and cdocsamp.tex with
% cdocsch1.tex, cdocsch2.tex, cdocsdrf.tex, cdocsfn1.tex, cdocsfn2.tex.
%
%<package>\ifdefined\childdocmain\endinput\fi
%<package>\ProvidesFile{childdoc.def}[2018/12/30 v2.0 child document driver]
%<samplemain>\ProvidesFile{cdocsamp.tex}[2018/12/30 v2.0 sample for childdoc]
%<*driver>
%\ProvidesFile{childdoc.drv}[2018/12/30 v2.0 childdoc reference manual file]
\PassOptionsToClass{10pt,a4paper}{article}
\documentclass{ltxdoc}

\usepackage[margin=35mm]{geometry}
\usepackage{hyperref}
\usepackage{hyperxmp}
\usepackage[usenames]{color}

\hypersetup{colorlinks=true}
\hypersetup{pdfstartview=FitH}
\hypersetup{pdfpagemode=UseNone}
\hypersetup{pdfsource={}}
\hypersetup{pdflang={en-UK}}
\hypersetup{pdfcopyright={Copyright 2017-2018 Niklas Beisert.
  This work may be distributed and/or modified under the
  conditions of the LaTeX Project Public License, either version 1.3
  of this license or (at your option) any later version.}}
\hypersetup{pdflicenseurl={http://www.latex-project.org/lppl.txt}}
\hypersetup{pdfcontactaddress={ETH Zurich, ITP, HIT K,
  Wolfgang-Pauli-Strasse 27}}
\hypersetup{pdfcontactpostcode={8093}}
\hypersetup{pdfcontactcity={Zurich}}
\hypersetup{pdfcontactcountry={Switzerland}}
\hypersetup{pdfcontactemail={nbeisert@itp.phys.ethz.ch}}
\hypersetup{pdfcontacturl={http://people.phys.ethz.ch/\xmptilde nbeisert/}}

\newcommand{\secref}[1]{\hyperref[#1]{section \ref*{#1}}}

\parskip1ex
\parindent0pt
\let\olditemize\itemize
\def\itemize{\olditemize\parskip0pt}

\begin{document}

\title{The \textsf{childdoc} Package}
\hypersetup{pdftitle={The childdoc Package}}
\author{Niklas Beisert\\[2ex]
  Institut f\"ur Theoretische Physik\\
  Eidgen\"ossische Technische Hochschule Z\"urich\\
  Wolfgang-Pauli-Strasse 27, 8093 Z\"urich, Switzerland\\[1ex]
  \href{mailto:nbeisert@itp.phys.ethz.ch}
  {\texttt{nbeisert@itp.phys.ethz.ch}}}
\hypersetup{pdfauthor={Niklas Beisert}}
\hypersetup{pdfsubject={Manual for the LaTeX2e Package childdoc}}
\date{30 December 2018, \textsf{v2.0}}
\maketitle

\begin{abstract}\noindent
\textsf{childdoc} is a \LaTeXe{} package
that enables the direct compilation
of document sections included by |\include|
to individual files.
\end{abstract}

\begingroup
\parskip0ex
\tableofcontents
\endgroup

%%%%%%%%%%%%%%%%%%%%%%%%%%%%%%%%%%%%%%%%%%%%%%%%%%%%%%%%%%%%%%%%%%%%%%%%%%%%%%%%
%%%%%%%%%%%%%%%%%%%%%%%%%%%%%%%%%%%%%%%%%%%%%%%%%%%%%%%%%%%%%%%%%%%%%%%%%%%%%%%%
\section{Introduction}

\LaTeX{} provides a mechanism to structure a large document (such as a book)
into a main file and several child files (containing the chapters)
using the |\include| command.
This mechanism is beneficial for documents
which span hundreds of pages in order to
make the source file(s) more manageable.
Moreover, compilation can be restricted to
selected child files by means of the |\includeonly| command.
The latter feature can be used to reduce the compilation time while editing
(this was significantly more useful in the earlier days of \LaTeX{})
or to generate a smaller document which is easier to navigate.
Another application of |\includeonly| is to generate
documents consisting of selected parts of the complete document.

However, there are a few drawbacks of the plain |\include| mechanism:
\begin{itemize}
\item
The child files cannot be compiled on their own,
they can only be compiled via the main file.
A naive editing environment
(such as a text editor with an option
to have the current file processed by \LaTeX)
may require one to switch to the main file before compiling;
attempting to compile the child file produces errors.
\item
The main file must be modified (each time)
to adjust the |\includeonly| command
to the present needs. This easily leaves the main file in a messy state.
\item
The generated document will always carry the filename
of the main document. This is inconvenient if
several child files are to be compiled and
to be kept for distribution.
\end{itemize}

The present package provides a simple interface
to make child files individually compilable by \LaTeX{}.
Compiling a child file then has the same effect as compiling
the main file with an |\includeonly| command
to select the appropriate child.
Moreover the generated document will carry the name of the child
rather than the main file.
This resolves all three above issues.

This feature is meant to make the editing of books,
thesis documents and lecture notes somewhat more convenient.
However, the package can also be used efficiently for
composing a series of documents (such as exercise sheets)
which are typically distributed individually.
It then assists the author in generating the individual documents
(potentially in different versions)
as well as a document containing the collected series.
Another application is in developing style files
or other kinds of included material
where compilation of the style file could redirect
to a sample or test file.

%%%%%%%%%%%%%%%%%%%%%%%%%%%%%%%%%%%%%%%%%%%%%%%%%%%%%%%%%%%%%%%%%%%%%%%%%%%%%%%%
%%%%%%%%%%%%%%%%%%%%%%%%%%%%%%%%%%%%%%%%%%%%%%%%%%%%%%%%%%%%%%%%%%%%%%%%%%%%%%%%
\section{Usage}

First of all, the package \textsf{childdoc} is \emph{not} a standard
\LaTeXe{} |.sty| style file! Therefore it needs to be invoked in
a non-standard way.

%%%%%%%%%%%%%%%%%%%%%%%%%%%%%%%%%%%%%%%%%%%%%%%%%%%%%%%%%%%%%%%%%%%%%%%%%%%%%%%%
\subsection{Included Files}
\label{sec:include}

%%%%%%%%%%%%%%%%%%%%%%%%%%%%%%%%%%%%%%%%
\DescribeMacro{\childdocmain}
To use the package, add the commands
\begin{center}
\begin{tabular}{l}
|\input{childdoc.def}|\\
|\childdocmain{}|\\
\end{tabular}
\end{center}
at the very top of the main \LaTeX{} file,
in particular \emph{before} the |\documentclass| statement!
The argument of |\childdocmain| should be left empty
(but it must be present).

%%%%%%%%%%%%%%%%%%%%%%%%%%%%%%%%%%%%%%%%
\DescribeMacro{\childdocof}
Furthermore, add the commands
\begin{center}
\begin{tabular}{l}
|\input{childdoc.def}|\\
|\childdocof{|\textit{main}|}|\\
\end{tabular}
\end{center}
at the top of every child file \textit{child}
which is included by |\include{|\textit{child}|}|
from within the main file
(or at least for those files to be compiled individually).
The argument \textit{main} must be the filename of the main file.

There are a couple of
considerations in setting up the main and child documents:

%%%%%%%%%%%%%%%%%%%%%%%%%%%%%%%%%%%%%%%%
\paragraph{Restrictions.}

Please note the following restrictions:
\begin{itemize}
\item
|\childdocmain| must be called with one argument \textit{main}
to ensure compatibility with earlier version of the package.
It must either be empty (|\childdocmain{}|)
or precisely match the filename of the main file in which it is specified.
See \secref{sec:detection} for further information.
\item
The filename \textit{main} must be specified without the |.tex| extension.
\item
The filename \textit{main} is case sensitive
(even in case-insensitive file systems)
due to internal string comparison.
\item
The argument \textit{main} should be fully expanded, it cannot be a macro.
\item
Subdirectories and special characters should be avoided in filenames.
\item
The command |\childdocmain{|\textit{main}|}| must be followed by a whitespace.
It should not be followed immediately by another command
or by a comment mark `|%|'.
This is because the \TeX{} parser reads the token immediately following
the argument of |\childdocmain| and puts it
at the beginning of every child section;
however, a white\-space is ignored.
\end{itemize}

%%%%%%%%%%%%%%%%%%%%%%%%%%%%%%%%%%%%%%%%
\paragraph{Content of Main File.}

It is advisable to place all content in the child files included by |\include|.
Any output contained in the main file will appear in all child documents
unless suppressed manually;
it cannot be suppressed automatically by the |\includeonly| directive
and thus should normally be avoided.
A method to include some content in the main file
by means of conditional processing is described in \secref{sec:conditional}.

%%%%%%%%%%%%%%%%%%%%%%%%%%%%%%%%%%%%%%%%
\paragraph{Page Numbering.}

When only a part of the document is compiled,
the appropriate numbering of pages
(as well as other status parameters)
is determined from the |.aux| files.
The latter contain information from previous passes.
However this information needs to propagate through
all intermediate child documents.
Therefore the page numbering in child documents may well
be inconsistent until the complete document is compiled at least once.

A useful (if unconventional) way to always ensure a consistent
page numbering is to restart the numbering in each child document
and denote the pages by `\textit{child}|.|\textit{page}'
where \textit{child} represents the chapter/section number of the child file.
This can be achieved by the command
|\numberwithin{page}{|\textit{child}|}|
of the \textsf{amsmath} package
where \textit{child} can be |chapter| or |section|
depending on the chosen structuring.
Alternatively, one can modify the macro |\thepage| appropriately
and reset the counter |page| at the start of each child file.

%%%%%%%%%%%%%%%%%%%%%%%%%%%%%%%%%%%%%%%%%%%%%%%%%%%%%%%%%%%%%%%%%%%%%%%%%%%%%%%%
\subsection{Conditional Processing}
\label{sec:conditional}

The package provides a mechanism to compile different versions
of a document. To customise the versions further some conditional processing
can come in handy to distinguish which version is being compiled.
The package provides two macros to describe the compilation context:

%%%%%%%%%%%%%%%%%%%%%%%%%%%%%%%%%%%%%%%%
\DescribeMacro{\ifchilddoc}
The conditional |\ifchilddoc| distinguishes between the compilation of
child documents and the main document:
%
\begin{center}
|\ifchilddoc |\textit{child-code}| |[|\||else |\textit{main-code}]| \||fi|
\end{center}

%%%%%%%%%%%%%%%%%%%%%%%%%%%%%%%%%%%%%%%%
\DescribeMacro{\childdocname}
\DescribeMacro{\childdocjob}
The macro |\childdocname| contains the filename (without extension)
of the main or child file being processed.
Note that |\childdocjob| will always contain the name of the main file.

%%%%%%%%%%%%%%%%%%%%%%%%%%%%%%%%%%%%%%%%
\paragraph{Title Page.}

Conditional processing can be used to include a title or banner page
in the main document when proper precautions are taken.
Importantly, the code in the main file should ensure that the page counter
(as well as other status parameters which are stored in the |.aux| files)
takes the same value after the conditional processing.
Otherwise the page numbers may take divergent values
depending on which part is compiled.

For example, a title page could be declared by:
%
\begin{center}
\begin{tabular}{l}
|\ifchilddoc\||else|\\
|\addtocounter{page}{-1}|\\
\textit{code for title page}\\
|\newpage|\\
|\||fi|
\end{tabular}
\end{center}
%
A banner page for the child documents can be generated by:
%
\begin{center}
\begin{tabular}{l}
|\ifchilddoc|\\
|\addtocounter{page}{-1}|\\
\textit{code for banner page}\\
|\newpage|\\
|\||fi|
\end{tabular}
\end{center}
%
Here one could write a message such as:
\begin{center}
|This is the part \childdocname{} of \childdocjob{}.|
\end{center}

%%%%%%%%%%%%%%%%%%%%%%%%%%%%%%%%%%%%%%%%%%%%%%%%%%%%%%%%%%%%%%%%%%%%%%%%%%%%%%%%
\subsection{Flags}
\label{sec:flags}

The package makes it easy to generate different versions
of the main or child documents.
To this end compilation flags can be defined
and assigned different default values.
They will be particularly useful in conjunction
with the forwarding mechanism described in \secref{sec:forward}.

For example, it may be useful to have a flag |\version|
which can be set to |draft| or |final|.
The document source will contain some conditional code
depending on the value of |\version|.
Suppose further, the flag should default to |final| for the main file
and to |draft| for child files
which is a natural assignment for editing the document.
This is achieved by placing the following code
in the preamble of the main document
(below the |\childdocmain| directive):
%
\begin{center}
\begin{tabular}{l}
|\ifchilddoc|\\
|\providecommand{\version}{draft}|\\
|\||else|\\
|\providecommand{\version}{final}|\\
|\||fi|
\end{tabular}
\end{center}
%
The definition by |\providecommand| makes sure
that previous definitions are not overwritten.
Further statements |\providecommand{\version}{...}|
can thus be added before the above code to override it.

For the main file, one might add a line
(between |\childdocmain| and the above block)
%
\begin{center}
|%\ifchilddoc\||else\providecommand{\version}{draft}\||fi|
\end{center}
%
which can be uncommented to produce a draft version.
Likewise one can add a line to the very top of a child file
(above the |\childdocof{|\textit{main}|}| directive)
%
\begin{center}
|%\providecommand{\version}{final}|
\end{center}
%
which can be uncommented to produce the final version of this child document.

%%%%%%%%%%%%%%%%%%%%%%%%%%%%%%%%%%%%%%%%%%%%%%%%%%%%%%%%%%%%%%%%%%%%%%%%%%%%%%%%
\subsection{Forwarding}
\label{sec:forward}

Different versions of the main or child documents
using compilation flags as described in \secref{sec:flags}
can be (permanently) stored in different files
for convenient compilation, viewing and distribution.
To this end, the package defines a command
to pass on compilation to a different file:

%%%%%%%%%%%%%%%%%%%%%%%%%%%%%%%%%%%%%%%%
\DescribeMacro{\childdocforward}
The command |\childdocforward| redirects processing to
another source file:
%
\begin{center}
\begin{tabular}{l}
|\input{childdoc.def}|\\
|\childdocforward[|\textit{main}|]{|\textit{dest}|}|\\
\end{tabular}
\end{center}
%
The argument \textit{dest} is the destination file
(without extension).
It should be the main file or one of the child files.
Note that further \textsf{childdoc} directives
such as |\childdocof| and |\childdocforward|
in the indicated file will be processed in this form.
The optional argument \textit{main}
passes on directly to the main file \textit{main}
while pretending to compile the child \textit{dest}.
This form behaves as if \textit{dest}
issues |\childdocof{|\textit{main}|}| right away,
and no further \textsf{childdoc} directives will be processed.

%%%%%%%%%%%%%%%%%%%%%%%%%%%%%%%%%%%%%%%%
\DescribeMacro{\...prefix}
In the alternative form |\childdocforwardprefix|,
%
\begin{center}
\begin{tabular}{l}
|\input{childdoc.def}|\\
|\childdocforwardprefix[|\textit{main}|]{|\textit{prefix}|}{|\textit{dest}|}|
\end{tabular}
\end{center}
%
the destination file is determined by a pattern
depending on the current file:
To make this work, the current file must be called
`{\textit{prefix}\hspace{0.2em}\textit{suffix}}'
with \textit{prefix} matching precisely the argument.
Processing is then passed on to the file
`{\textit{dest}\hspace{0.2em}\textit{suffix}}'.
Surely, the same effect is achieved by
directly specifying the
argument `{\textit{dest}\hspace{0.2em}\textit{suffix}}'
in the first form.
However, that requires to set up a different file
for each child. With the alternative form of the command
all these files can have exactly the same content
which simplifies setting them up and maintaining them.

For example, the following file |draft.tex|
with a compilation flag |\version| as described in \secref{sec:flags}
compiles the main document as a draft:
%
\begin{center}
\begin{tabular}{l}
|\def\version{draft}|\\
|\input{childdoc.def}|\\
|\childdocforward{|\textit{main}|}|
\end{tabular}
\end{center}
%
Likewise, the following files |final|\textit{nn}|.tex|
compile the final version of the child document
|child|\textit{nn}|.tex|:
%
\begin{center}
\begin{tabular}{l}
|\def\version{final}|\\
|\input{childdoc.def}|\\
|\childdocforwardprefix{final}{child}|
\end{tabular}
\end{center}
%

Note that when several versions of a main file and/or of each child file
are to be generated, it may be convenient to set up a |Makefile| or
shell script to automatise the process.

%%%%%%%%%%%%%%%%%%%%%%%%%%%%%%%%%%%%%%%%%%%%%%%%%%%%%%%%%%%%%%%%%%%%%%%%%%%%%%%%
\subsection{Command Line Processing}
\label{sec:commandline}

The effect of redirection files can also be achieved by invoking
the \LaTeX{} compiler with a more elaborate command line.
Most conveniently this should be done as part
of a shell script or a |Makefile|.

When using \textsf{childdoc} in the main file, the following
command lines effectively perform a redirection
(note that depending on the shell being used,
backslashes may have to be doubled: `|\|' $\to$ `|\\|'):
%
\begin{center}
|... -jobname "|\textit{target}|" |\\|"|[\textit{flags}]%
|\input{childdoc.def}\childdocforward[|\textit{main}|]{|\textit{dest}|}"|
\end{center}
%
Here \textit{target} is the name of the output file,
\textit{main} is the name of the main file
and \textit{dest} is the name of the main or child file to be processed
(all filenames without extensions).
The optional argument \textit{main} can be omitted
if \textit{main} matches \textit{dest}.
Optionally, compilation \textit{flags} can be defined via |\def| commands.
This command line makes the \TeX{} engine believe
it is compiling the file \textit{target}
whose content is specified as the latter parameter.
The provided code then forwards the processing to
\textit{main} or \textit{dest} as described in \secref{sec:forward}.

%%%%%%%%%%%%%%%%%%%%%%%%%%%%%%%%%%%%%%%%%%%%%%%%%%%%%%%%%%%%%%%%%%%%%%%%%%%%%%%%
\subsection{Include by Input}
\label{sec:input}

Including child documents by |\include| has some restrictions by design.
Most notably, the content of a child document always occupies
its own set of pages; pages cannot be shared between child documents.
Usually, this behaviour makes perfect sense
because each child document contain an essential part of the document.
However, in some situations it may be desirable to compose
a document from a collection of parts
without having mandatory page breaks between then.
For this case, the package
provides a mechanism to include parts
by |\input| which can also be processed individually.
However, by construction this mechanism
requires manual handling of the content to be output.

%%%%%%%%%%%%%%%%%%%%%%%%%%%%%%%%%%%%%%%%
\DescribeMacro{\ifchilddocmanual}
The main file should be prepared as usual, see \secref{sec:include}.
However, the document body must make a distinction
between processing of an individual part and of the main document, e.g.:
%
\begin{center}
\begin{tabular}{l}
|\ifchilddocmanual|\\
|\input{\childdocname}|\\
|\||else|\\
\textit{document body with }|\input{|\textit{part}|}|\\
|\||fi|
\end{tabular}
\end{center}
%
The conditional |\ifchilddocmanual| is true whenever
a part to be included by |\input| is being compiled,
and the name of the part is stored in |\childdocname|.

%%%%%%%%%%%%%%%%%%%%%%%%%%%%%%%%%%%%%%%%
\DescribeMacro{\childdocby}
Each part to be included by |\input| should start with:
%
\begin{center}
\begin{tabular}{l}
|\input{childdoc.def}|\\
|\childdocby{|\textit{main}|}|\\
\end{tabular}
\end{center}
%
The directive |\childdocby| is similar to |\childdocof|
described in \secref{sec:include},
but the subsequent selection of content must be done manually.
To that end, both |\ifchilddoc| and |\ifchilddocmanual|
will be true upon processing of a part,
and the name of the part is stored in |\childdocname|.
Note that |\jobname| will be set to the filename of the current part
so that each part receives an individual |.aux| file
that does not interfere with the |.aux| file(s) of the main document.
This behaviour can be altered by the alternative form
|\childdocby[*]{|\textit{main}|}| (with a non-empty optional argument)
which uses the |.aux| file of the main document
by setting |\jobname| to \textit{main}.

%%%%%%%%%%%%%%%%%%%%%%%%%%%%%%%%%%%%%%%%%%%%%%%%%%%%%%%%%%%%%%%%%%%%%%%%%%%%%%%%
\subsection{Driver Development}
\label{sec:driver}

The \textsf{childdoc} mechanism can also be use for the development
of definition files such as \LaTeX{} styles or classes.
This case differs from the above setup with multiple parts
included by |\include| in that no |\includeonly| should be invoked.
This can be achieved by starting the include file
(before |\ProvidesPackage|) with:
%
\begin{center}
\begin{tabular}{l}
|\input{childdoc.def}|\\
|\childdocforward{|\textit{main}|}|\\
\end{tabular}
\end{center}
%
or alternatively with:
%
\begin{center}
\begin{tabular}{l}
|\input{childdoc.def}|\\
|\childdocby{|\textit{main}|}|\\
\end{tabular}
\end{center}
%
Both forms have slightly different effects as described above.
The main file is prepared as usual, see \secref{sec:include}.

%%%%%%%%%%%%%%%%%%%%%%%%%%%%%%%%%%%%%%%%%%%%%%%%%%%%%%%%%%%%%%%%%%%%%%%%%%%%%%%%
\subsection{Legacy Detection}
\label{sec:detection}

The directive |\childdocmain| in the main file can detect
whether the complete document or merely a child is to be compiled
even without using the directive |\childdocof|.
This method is deprecated because it is less robust
and there is no compelling reason to use it;
it is merely provided for backward compatibility
and it may be removed in future versions.

If the detection mechanism is to be used,
it is mandatory to correctly specify
the filename of the main file as the argument of |\childdocmain|:
%
\begin{center}
\begin{tabular}{l}
|\input{childdoc.def}|\\
|\childdocmain{|\textit{main}|}|\\
\end{tabular}
\end{center}
%
If |\jobname| does not match the argument \textit{main} of |\childdocmain|,
it is assumed that |\jobname| points to the child file to be compiled.
When using |\childdocmain| with the main file specified as argument,
it suffices to start a child file
with just |\input{|\textit{main}|}|
without loading of the package and using |\childdocof|.
If instead all processing is done
with the appropriate \textsf{childdoc} directives,
the argument of \textit{main} of |\childdocmain| can be empty.

An alternative version of the command line processing described
in \secref{sec:commandline} using the detection mechanism reads:
%
\begin{center}
|... -jobname "|\textit{target}|" "|[\textit{flags}]%
[|\def\jobname{|\textit{dest}|}|]|\input{|\textit{main}|}"|
\end{center}

%%%%%%%%%%%%%%%%%%%%%%%%%%%%%%%%%%%%%%%%%%%%%%%%%%%%%%%%%%%%%%%%%%%%%%%%%%%%%%%%
\subsection{Manual Code}
\label{sec:manual}

In case one cannot be certain whether the definitions file |childdoc.def|
is installed on the target \TeX{} distribution
and one prefers not to ship it,
it is conceivable to paste a few relevant commands into the sources.

To that end, drop all statements |\input{childdoc.def}|
and perform the replacements as outlined below.
Instead of |\childdocmain{|\textit{main}|}| add the following code
to the top of the main file:
%
\begin{center}
\begin{tabular}{l}
|\||ifdefined\childdocname\endinput\||fi\newif\ifchilddoc|\\
|\edef\childdocname{\scantokens\expandafter{\jobname\noexpand}}|\\
|\def\childdocmain{|\textit{main}|}\||ifx\childdocmain\childdocname\||else|\\
|\childdoctrue\includeonly{\childdocname}\let\jobname\childdocmain\||fi|\\
\end{tabular}
\end{center}
%
Instead of |\childdocof{|\textit{main}|}| just include the main file
at the top of each child file:
%
\begin{center}
|\input{|\textit{main}|}|
\end{center}
%
A simple redirection |\childdocforward{|\textit{dest}|}| is achieved by:
%
\begin{center}
|\def\jobname{|\textit{dest}|}\input{\jobname}|
\end{center}
%
The redirection with prefix
|\childdocforwardprefix[|\textit{prefix}|]{|\textit{dest}|}|
is accomplished by:
%
\begin{center}
\begin{tabular}{l}
|{\edef\jobname{\scantokens\expandafter{\jobname\noexpand}}|\\
|\def\redirectjob |\textit{prefix}|#1~~~{\gdef\jobname{|\textit{dest}|#1}}|\\
|\expandafter\redirectjob\jobname~~~}\input{\jobname}|
\end{tabular}
\end{center}

In an alternative approach,
child documents can be compiled by a specific command line
without additional code or specific definitions:
%
\begin{center}
|... -jobname "|\textit{target}|" "|[\textit{flags}]%
|\includeonly{|\textit{dest}|}\input{|\textit{main}|}"|
\end{center}
%

%%%%%%%%%%%%%%%%%%%%%%%%%%%%%%%%%%%%%%%%%%%%%%%%%%%%%%%%%%%%%%%%%%%%%%%%%%%%%%%%
%%%%%%%%%%%%%%%%%%%%%%%%%%%%%%%%%%%%%%%%%%%%%%%%%%%%%%%%%%%%%%%%%%%%%%%%%%%%%%%%
\section{Information}

%%%%%%%%%%%%%%%%%%%%%%%%%%%%%%%%%%%%%%%%%%%%%%%%%%%%%%%%%%%%%%%%%%%%%%%%%%%%%%%%
\subsection{Copyright}

Copyright \copyright{} 2017--2018 Niklas Beisert

This work may be distributed and/or modified under the
conditions of the \LaTeX{} Project Public License, either version 1.3
of this license or (at your option) any later version.
The latest version of this license is in
  \url{http://www.latex-project.org/lppl.txt}
and version 1.3 or later is part of all distributions of \LaTeX{}
version 2005/12/01 or later.

This work has the LPPL maintenance status `maintained'.

The Current Maintainer of this work is Niklas Beisert.

This work consists of the files |README.txt|, |childdoc.ins| and |childdoc.dtx|
as well as the derived files |childdoc.def|, |cdocsamp.tex|
with |cdocsch1.tex|, |cdocsch2.tex|, |cdocspt3.tex|, |cdocspt4.tex|,
|cdocsdrf.tex|, |cdocsfn1.tex|, |cdocsfn2.tex|
as well as |childdoc.pdf|.

%%%%%%%%%%%%%%%%%%%%%%%%%%%%%%%%%%%%%%%%%%%%%%%%%%%%%%%%%%%%%%%%%%%%%%%%%%%%%%%%
\subsection{Files and Installation}

The package consists of the files:
%
\begin{center}
\begin{tabular}{ll}
    |README.txt|   & readme file \\
    |childdoc.ins| & installation file \\
    |childdoc.dtx| & source file \\
    |childdoc.def| & definition file \\
    |cdocsamp.tex| & sample main file \\
    |cdocsch1.tex| & sample include file \\
    |cdocsch2.tex| & sample include file \\
    |cdocspt3.tex| & sample part file \\
    |cdocspt4.tex| & sample part file \\
    |cdocsdrf.tex| & sample redirection file \\
    |cdocsfn1.tex| & sample redirection file \\
    |cdocsfn2.tex| & sample redirection file \\
    |childdoc.pdf| & manual
\end{tabular}
\end{center}
%
The distribution consists of the files
|README.txt|, |childdoc.ins| and |childdoc.dtx|.
%
\begin{itemize}
\item
Run (pdf)\LaTeX{} on |childdoc.dtx|
to compile the manual |childdoc.pdf| (this file).
\item
Run \LaTeX{} on |childdoc.ins| to create the definitions file |childdoc.def|
and the sample |cdocsamp.tex| with include files
|cdocsch1.tex|, |cdocsch2.tex|, |cdocspt3.tex|, |cdocspt4.tex|,
|cdocsdrf.tex|, |cdocsfn1.tex|, |cdocsfn2.tex|.
Then copy the file |childdoc.def| to an appropriate directory of your \LaTeX{}
distribution, e.g.\ \textit{texmf-root}|/tex/latex/childdoc|.
\end{itemize}

%%%%%%%%%%%%%%%%%%%%%%%%%%%%%%%%%%%%%%%%%%%%%%%%%%%%%%%%%%%%%%%%%%%%%%%%%%%%%%%%
\subsection{Related CTAN Packages}

There are several other packages which offer a similar functionality:
%
\begin{itemize}
\item
The packages
\href{http://ctan.org/pkg/docmute}{\textsf{docmute}},
\href{http://ctan.org/pkg/includex}{\textsf{includex}} and
\href{http://ctan.org/pkg/standalone}{\textsf{standalone}}
provide commands to include only the document body of
a child file thus allowing both files to be compiled individually.
\item
The packages \href{http://ctan.org/pkg/subdocs}{\textsf{subdocs}}
and \href{http://ctan.org/pkg/subfiles}{\textsf{subfiles}}
provide structures in which the main and child documents can be
encapsulated and allowing them to be compiled individually.
The inclusion mechanism is different from the conventional |\include|.
\item
The package \href{http://ctan.org/pkg/combine}{\textsf{combine}}
is an elaborate solution to combine several documents into one.
\end{itemize}
%
See also the CTAN topic \href{http://ctan.org/topic/subdocs}{\textsf{subdocs}}
for further related packages.
The present package differs from the above solutions in that
a document structure constructed with the conventional |\include| mechanism
just needs two extra commands at the top of every file
such that all constituent files can be compiled individually.

%%%%%%%%%%%%%%%%%%%%%%%%%%%%%%%%%%%%%%%%%%%%%%%%%%%%%%%%%%%%%%%%%%%%%%%%%%%%%%%%
%\subsection{Feature Suggestions}
%
%The following is a list of features which may be useful for future
%versions of this package:
%%
%\begin{itemize}
%\item
%\ldots
%\end{itemize}

%%%%%%%%%%%%%%%%%%%%%%%%%%%%%%%%%%%%%%%%%%%%%%%%%%%%%%%%%%%%%%%%%%%%%%%%%%%%%%%%
\subsection{Revision History}

%%%%%%%%%%%%%%%%%%%%%%%%%%%%%%%%%%%%%%%%
\paragraph{v2.0:} 2018/12/30

\begin{itemize}
\item
immediate forward processing
\item
added |\childdocby| mechanism
\item
manual restructured
\end{itemize}

%%%%%%%%%%%%%%%%%%%%%%%%%%%%%%%%%%%%%%%%
\paragraph{v1.6:} 2018/01/17

\begin{itemize}
\item
application for development of include files
\item
corrections to manual
\end{itemize}

%%%%%%%%%%%%%%%%%%%%%%%%%%%%%%%%%%%%%%%%
\paragraph{v1.5:} 2017/05/21

\begin{itemize}
\item
more complete structuring introduced
\item
|\childdocof| introduced
\item
|\childdoc| renamed to |\childdocmain|
\item
|\childredirect| renamed to |\childdocforward| and |\childdocforwardprefix|
and functionality expanded
\end{itemize}

%%%%%%%%%%%%%%%%%%%%%%%%%%%%%%%%%%%%%%%%
\paragraph{v1.0:} 2017/04/27

\begin{itemize}
\item
manual and install package
\item
first version published on CTAN
\end{itemize}

%%%%%%%%%%%%%%%%%%%%%%%%%%%%%%%%%%%%%%%%
\paragraph{v0.6:} 2017/04/26

\begin{itemize}
\item
redirection mechanism added
\end{itemize}

%%%%%%%%%%%%%%%%%%%%%%%%%%%%%%%%%%%%%%%%
\paragraph{v0.5:} 2017/04/26

\begin{itemize}
\item
functionality in definition file
\end{itemize}


%%%%%%%%%%%%%%%%%%%%%%%%%%%%%%%%%%%%%%%%%%%%%%%%%%%%%%%%%%%%%%%%%%%%%%%%%%%%%%%%
%%%%%%%%%%%%%%%%%%%%%%%%%%%%%%%%%%%%%%%%%%%%%%%%%%%%%%%%%%%%%%%%%%%%%%%%%%%%%%%%
%%%%%%%%%%%%%%%%%%%%%%%%%%%%%%%%%%%%%%%%%%%%%%%%%%%%%%%%%%%%%%%%%%%%%%%%%%%%%%%%
\appendix

\settowidth\MacroIndent{\rmfamily\scriptsize 000\ }

 \DocInput{childdoc.dtx}

\end{document}
%</driver>
% \fi
%
% %%%%%%%%%%%%%%%%%%%%%%%%%%%%%%%%%%%%%%%%%%%%%%%%%%%%%%%%%%%%%%%%%%%%%%%%%%%%%%
% %%%%%%%%%%%%%%%%%%%%%%%%%%%%%%%%%%%%%%%%%%%%%%%%%%%%%%%%%%%%%%%%%%%%%%%%%%%%%%
% \section{Sample}
%\iffalse
%<*samplemain>
%\fi
%
% The following presents a sample document
% with two chapters, two parts, a title page,
% a compile flag as well as three forwarding files to set the flag.
% It consists of eight |.tex| files:
% \begin{center}
% \begin{tabular}{ll}
% |cdocsamp.tex|&main file\\
% |cdocsch1.tex|&include file for chapter 1\\
% |cdocsch2.tex|&include file for chapter 2\\
% |cdocspt3.tex|&include file for part 3\\
% |cdocspt4.tex|&include file for part 4\\
% |cdocsdrf.tex|&forwarding file for main file in draft mode\\
% |cdocsfi1.tex|&forwarding file for final version of chapter 1\\
% |cdocsfi2.tex|&forwarding file for final version of chapter 2\\
% \end{tabular}
% \end{center}
% Each of the eight files can be compiled directly by the \LaTeX{} compiler.
%
% %%%%%%%%%%%%%%%%%%%%%%%%%%%%%%%%%%%%%%
% \paragraph{Main File.}
%
% The main file is called |cdocsamp.tex|.
%
% Load the \textsf{childdoc} definitions and
% declare the filename for the main document:
%    \begin{macrocode}
\input{childdoc.def}
\childdocmain{}
%    \end{macrocode}

% Optional override for |\version| flag:
%    \begin{macrocode}
%%\ifchilddoc\else\providecommand{\version}{draft}\fi
%    \end{macrocode}

% Define the default values for the |\version| flag
% (|final| for the main file and |draft| for childs):
%    \begin{macrocode}
\ifchilddoc
\providecommand{\version}{draft}
\else
\providecommand{\version}{final}
\fi
%    \end{macrocode}

% Load the standard document class:
%    \begin{macrocode}
\documentclass[12pt]{article}
%    \end{macrocode}

% Start the document body:
%    \begin{macrocode}
\begin{document}
%    \end{macrocode}

% Declare a title page.
% Print title, part of document being processed and version flag:
%    \begin{macrocode}
\addtocounter{page}{-1}
\begin{center}
{\LARGE\bfseries{}childdoc example\par}
\vspace{1cm}
\ifchilddoc
\ifchilddocmanual part\else chapter\fi:
`\childdocname' of `\childdocjob'\par
\else
main document: `\childdocjob'\par
\fi
version: \version\par
\end{center}
\newpage
%    \end{macrocode}

% Manually include selected file,
% otherwise process as usual:
%    \begin{macrocode}
\ifchilddocmanual
\section*{part `\childdocname'}
\input{\childdocname}
\else
%    \end{macrocode}

% Include the two chapters:
%    \begin{macrocode}
\include{cdocsch1}
\include{cdocsch2}
%    \end{macrocode}

% Include the two parts unless only chapters should be displayed:
%    \begin{macrocode}
\ifchilddoc\else
\section{part three}
\input{cdocspt3}
\section{part four}
\input{cdocspt4}
\fi
%    \end{macrocode}

% Process as usual until here:
%    \begin{macrocode}
\fi
%    \end{macrocode}

% End of document body:
%    \begin{macrocode}
\end{document}
%    \end{macrocode}
%\iffalse
%</samplemain>
%\fi
%
% %%%%%%%%%%%%%%%%%%%%%%%%%%%%%%%%%%%%%%
% \paragraph{Chapter Include Files.}
%
% The include files are called |cdocsch1.tex| and |cdocsch2.tex|.
%
%\iffalse
%<*samplechap1|samplechap2>
%\fi

% Optional override for |\version| flag:
%    \begin{macrocode}
%%\providecommand{\version}{final}
%    \end{macrocode}

% Include the main document:
%    \begin{macrocode}
\input{childdoc.def}
\childdocof{cdocsamp}
%    \end{macrocode}

%\iffalse
%</samplechap1|samplechap2>
%\fi
%
%\iffalse
%<*samplechap1>
%\fi
% Some text for chapter 1:
%    \begin{macrocode}
\section{one}
some text in chapter one
%    \end{macrocode}

%\iffalse
%</samplechap1>
%\fi
% Some text for chapter 2:
%\iffalse
%<*samplechap2>
%\fi
%    \begin{macrocode}
\section{two}
more text in chapter two
%    \end{macrocode}

%\iffalse
%</samplechap2>
%\fi
%
% %%%%%%%%%%%%%%%%%%%%%%%%%%%%%%%%%%%%%%
% \paragraph{Part Include Files.}
%
% The include files are called |cdocspt3.tex| and |cdocspt4.tex|.
%
%\iffalse
%<*samplepart3|samplepart4>
%\fi

% Optional override for |\version| flag:
%    \begin{macrocode}
%%\providecommand{\version}{final}
%    \end{macrocode}

% Include the main document:
%    \begin{macrocode}
\input{childdoc.def}
\childdocby{cdocsamp}
%    \end{macrocode}

%\iffalse
%</samplepart3|samplepart4>
%\fi
%
%\iffalse
%<*samplepart3>
%\fi
% Some text for part 3:
%    \begin{macrocode}
some text in part three
%    \end{macrocode}

%\iffalse
%</samplepart3>
%\fi
% Some text for part 4:
%\iffalse
%<*samplepart4>
%\fi
%    \begin{macrocode}
more text in part four
%    \end{macrocode}

%\iffalse
%</samplepart4>
%\fi
%
% %%%%%%%%%%%%%%%%%%%%%%%%%%%%%%%%%%%%%%
% \paragraph{Forwarding for a Complete Draft.}
%
% The following forwarding file |cdocsdrf.tex|
% compiles the main document in draft mode:
%\iffalse
%<*sampledraft>
%\fi
%    \begin{macrocode}
\def\version{draft}
\input{childdoc.def}
\childdocforward{cdocsamp}
%    \end{macrocode}

%\iffalse
%</sampledraft>
%\fi
%
% %%%%%%%%%%%%%%%%%%%%%%%%%%%%%%%%%%%%%%
% \paragraph{Forwarding for Final Version of the Chapters.}
%
% The following forwarding files |cdocsfn1.tex| and |cdocsfn2.tex|
% (with identical content)
% compile the final versions of the child documents
% |cdocsch1.tex| and |cdocsch2.tex|, respectively:
%\iffalse
%<*samplefinal>
%\fi
%    \begin{macrocode}
\def\version{final}
\input{childdoc.def}
\childdocforwardprefix[cdocsamp]{cdocsfn}{cdocsch}
%    \end{macrocode}

%\iffalse
%</samplefinal>
%\fi
%
% %%%%%%%%%%%%%%%%%%%%%%%%%%%%%%%%%%%%%%
% \paragraph{Command Line Processing.}
%
% The following three command lines generate the output files
% |cdocscld|, |cdocscl1| and |cdocscl2|
% which should be identical to
% |cdocsdrf|, |cdocsch1| and |cdocsfn2|, respectively:
% \begin{center}
% \begin{tabular}{l}
% |latex -jobname cdocscld \|\\
% |  "\def\version{draft}\input{childdoc.def}\childdocforward{cdocsamp}"|\\
% |latex -jobname cdocscl1 \|\\
% |  "\input{childdoc.def}\childdocforward[cdocsamp]{cdocsch1}"|\\
% |latex -jobname cdocscl2 \|\\
% |  "\def\version{final}\input{childdoc.def}\childdocforward{cdocsch2}"|
% \end{tabular}
% \end{center}
% Note that the trailing backslash on each first line
% merely continues the input to the second line
% (for convenient cut ant paste).
% Furthermore, the command |latex| can be replaced by any
% of its alternative versions such as |pdflatex|.
%
% %%%%%%%%%%%%%%%%%%%%%%%%%%%%%%%%%%%%%%%%%%%%%%%%%%%%%%%%%%%%%%%%%%%%%%%%%%%%%%
% %%%%%%%%%%%%%%%%%%%%%%%%%%%%%%%%%%%%%%%%%%%%%%%%%%%%%%%%%%%%%%%%%%%%%%%%%%%%%%
% \section{Implementation}
%\iffalse
%<*package>
%\fi
%
% This section describes the definitions file |childdoc.def|.

% The definitions cannot be loaded using |\usepackage| or |\RequirePackage|
% which has a mechanism to prevent loading a style file more than once.
% When loading the definitions by means of |\input|
% multiple instances have to be prevented manually:
%\iffalse
%This code needs to be before the `\ProvidesFile' directive
%which is defined at the beginning of this file.
%Therefore it is also placed there and commented out here.
%</package>
%<*discard>
%\fi
%    \begin{macrocode}
\ifdefined\childdocmain\endinput\fi
%    \end{macrocode}
%\iffalse
%</discard>
%<*package>
%\fi
%
% \macro{\ifchilddoc}
% \macro{\ifchilddocmanual}
% The conditional |\ifchilddoc| tells whether a
% child (true) or main (false) document is being compiled.
% The conditional |\ifchilddocmanual| tells whether
% the |\includeonly| mechanism is used (false) or
% the selection of child files must be performed manually (true).
% The definitions initialise to false:
%    \begin{macrocode}
\newif\ifchilddoc
\newif\ifchilddocmanual
%    \end{macrocode}

% \macro{\childdocname}
% \macro{\childdocjob}
% The macro |\childdocname| stores the name of the main document
% to be compiled. The macro |\childdocjob| stores the name of
% the document on which the \LaTeX{} compiler was originally invoked.
% The content of |\jobname| cannot be compared
% to filenames specified in the source due to different catcodes.
% The following code rescans |\jobname|, stores the result
% in |\childdocname| and saves a copy in |\childdocjob|:
%    \begin{macrocode}
\edef\childdocname{\scantokens\expandafter{\jobname\noexpand}}
\let\childdocjob\childdocname
%    \end{macrocode}

% \macro{\childdocdisable}
% The macro |\childdocdisable| prevents the main file
% from being processed more than once.
% At this stage, the main document command |\childdocmain|
% is assumed to be called once again where it should do nothing.
% Any subsequent call to it should prevent
% a secondary processing of the main document
% It overwrites the forwarding commands
% |\childdocof| and |\childdocforward|
% with empty macros to prevent further inclusions of the main document:
%    \begin{macrocode}
\newcommand{\childdocdisable}
{
  \renewcommand{\childdocmain}[1]{\renewcommand{\childdocmain}[1]{\endinput}}
  \renewcommand{\childdocof}[1]{}
  \renewcommand{\childdocby}[2][]{}
  \renewcommand{\childdocforward}[2][]{}
  \renewcommand{\childdocdisable}{}
}
%    \end{macrocode}

% \macro{\childdocmain}
% The macro |\childdocmain| is to be called at the top of the main file
% with nothing or the main filename (without extension) as argument.
% First, it breaks loops.
% If the argument is not empty and does not match |\childdocname|
% (which is set by the first inclusion of |childdoc.def|),
% |\ifchilddoc| is set to true, |\includeonly| is applied to the child file
% and |\jobname| is set to the main file
% (for proper handling of |.aux| files):
%    \begin{macrocode}
\newcommand{\childdocmain}[1]
{
  \childdocdisable\childdocmain{}
  \if?#1?\else
    \begingroup
      \def\childdoctmp{#1}
      \ifx\childdoctmp\childdocname
        \def\childdoctmp{}
      \else
        \def\childdoctmp
        {
          \childdoctrue
          \includeonly{\childdocname}
          \def\childdocjob{#1}
          \def\jobname{#1}
        }
      \fi
      \expandafter
    \endgroup
    \childdoctmp
  \fi
}
%    \end{macrocode}

% \macro{\childdocof}
% The command |\childdocof| redirects
% compilation to the main file |#1|.
%    \begin{macrocode}
\newcommand{\childdocof}[1]
{
  \childdocdisable
  \childdoctrue
  \includeonly{\childdocname}
  \def\jobname{#1}
  \def\childdocjob{#1}
  \input{#1}
}
%    \end{macrocode}

% \macro{\childdocby}
% The command |\childdocby| ....
%    \begin{macrocode}
\newcommand{\childdocby}[2][]
{
  \childdocdisable
  \childdoctrue
  \childdocmanualtrue
  \if?#1?\else
    \def\jobname{#2}
  \fi
  \def\childdocjob{#2}
  \input{#2}
  \endinput
}
%    \end{macrocode}

% \macro{\childdocforward}
% The command |\childdocforward| redirects
% compilation to the main file or
% (if the optional argument is given) a child file.
% Parameters are set as if the main file
% or a child file starting with |\childdocof| was compiled.
% Then compilation is handed over to the main file:
%    \begin{macrocode}
\newcommand{\childdocforward}[2][]
{
  \begingroup
    \if?#1?
      \def\childdoctmp
      {
        \def\childdocname{#2}
        \def\childdocjob{#2}
        \def\jobname{#2}
        \input{#2}
        \endinput
      }
    \else
      \def\childdoctmp
      {
        \childdocdisable
        \def\childdocname{#2}
        \childdoctrue
        \includeonly{#2}
        \def\childdocjob{#1}
        \def\jobname{#1}
        \input{#1}
        \endinput
      }
    \fi
    \expandafter
  \endgroup
  \childdoctmp
}
%    \end{macrocode}

% \macro{\childdocforwardprefix}
% The command |\childdocforwardprefix| redirects
% compilation to the main or a child file by means of a pattern.
% The prefix |#1| in the current filename is replaced by |#2|
% and the suffix of the current filename is kept
% (it is assumed that the filename does not contain the substring `|~~~|'
% which is used as a delimiter).
% Compilation is handed over to the new file by |\childdocforward|:
%    \begin{macrocode}
\newcommand{\childdocforwardprefix}[3][]
{
  \begingroup
    \def\childdocextract #2##1~~~{\def\childdoctmp{\childdocforward[#1]{#3##1}}}
    \expandafter\childdocextract\childdocname~~~
    \expandafter
  \endgroup
  \childdoctmp
}
%    \end{macrocode}

% \macro{\childdoc}
% The deprecated macro |\childdoc| is a legacy version of |\childdocmain|:
%    \begin{macrocode}
\newcommand{\childdoc}{\childdocmain}
%    \end{macrocode}

% \macro{\childdocredirect}
% The deprecated macro |\childdocredirect| is a legacy version
% of |\childdocforward| and |\childdocforwardprefix|:
%    \begin{macrocode}
\newcommand{\childdocredirect}[2][]
{
  \begingroup
    \if?#1?
      \def\childdoctmp{\childdocforward{#2}}
    \else
      \def\childdoctmp{\childdocforwardprefix{#1}{#2}}
    \fi
    \expandafter
  \endgroup
  \childdoctmp
}
%    \end{macrocode}

%\iffalse
%</package>
%\fi
%
\endinput
\childdocforward{cdocsch2}"|
% \end{tabular}
% \end{center}
% Note that the trailing backslash on each first line
% merely continues the input to the second line
% (for convenient cut ant paste).
% Furthermore, the command |latex| can be replaced by any
% of its alternative versions such as |pdflatex|.
%
% %%%%%%%%%%%%%%%%%%%%%%%%%%%%%%%%%%%%%%%%%%%%%%%%%%%%%%%%%%%%%%%%%%%%%%%%%%%%%%
% %%%%%%%%%%%%%%%%%%%%%%%%%%%%%%%%%%%%%%%%%%%%%%%%%%%%%%%%%%%%%%%%%%%%%%%%%%%%%%
% \section{Implementation}
%\iffalse
%<*package>
%\fi
%
% This section describes the definitions file |childdoc.def|.

% The definitions cannot be loaded using |\usepackage| or |\RequirePackage|
% which has a mechanism to prevent loading a style file more than once.
% When loading the definitions by means of |\input|
% multiple instances have to be prevented manually:
%\iffalse
%This code needs to be before the `\ProvidesFile' directive
%which is defined at the beginning of this file.
%Therefore it is also placed there and commented out here.
%</package>
%<*discard>
%\fi
%    \begin{macrocode}
\ifdefined\childdocmain\endinput\fi
%    \end{macrocode}
%\iffalse
%</discard>
%<*package>
%\fi
%
% \macro{\ifchilddoc}
% \macro{\ifchilddocmanual}
% The conditional |\ifchilddoc| tells whether a
% child (true) or main (false) document is being compiled.
% The conditional |\ifchilddocmanual| tells whether
% the |\includeonly| mechanism is used (false) or
% the selection of child files must be performed manually (true).
% The definitions initialise to false:
%    \begin{macrocode}
\newif\ifchilddoc
\newif\ifchilddocmanual
%    \end{macrocode}

% \macro{\childdocname}
% \macro{\childdocjob}
% The macro |\childdocname| stores the name of the main document
% to be compiled. The macro |\childdocjob| stores the name of
% the document on which the \LaTeX{} compiler was originally invoked.
% The content of |\jobname| cannot be compared
% to filenames specified in the source due to different catcodes.
% The following code rescans |\jobname|, stores the result
% in |\childdocname| and saves a copy in |\childdocjob|:
%    \begin{macrocode}
\edef\childdocname{\scantokens\expandafter{\jobname\noexpand}}
\let\childdocjob\childdocname
%    \end{macrocode}

% \macro{\childdocdisable}
% The macro |\childdocdisable| prevents the main file
% from being processed more than once.
% At this stage, the main document command |\childdocmain|
% is assumed to be called once again where it should do nothing.
% Any subsequent call to it should prevent
% a secondary processing of the main document
% It overwrites the forwarding commands
% |\childdocof| and |\childdocforward|
% with empty macros to prevent further inclusions of the main document:
%    \begin{macrocode}
\newcommand{\childdocdisable}
{
  \renewcommand{\childdocmain}[1]{\renewcommand{\childdocmain}[1]{\endinput}}
  \renewcommand{\childdocof}[1]{}
  \renewcommand{\childdocby}[2][]{}
  \renewcommand{\childdocforward}[2][]{}
  \renewcommand{\childdocdisable}{}
}
%    \end{macrocode}

% \macro{\childdocmain}
% The macro |\childdocmain| is to be called at the top of the main file
% with nothing or the main filename (without extension) as argument.
% First, it breaks loops.
% If the argument is not empty and does not match |\childdocname|
% (which is set by the first inclusion of |childdoc.def|),
% |\ifchilddoc| is set to true, |\includeonly| is applied to the child file
% and |\jobname| is set to the main file
% (for proper handling of |.aux| files):
%    \begin{macrocode}
\newcommand{\childdocmain}[1]
{
  \childdocdisable\childdocmain{}
  \if?#1?\else
    \begingroup
      \def\childdoctmp{#1}
      \ifx\childdoctmp\childdocname
        \def\childdoctmp{}
      \else
        \def\childdoctmp
        {
          \childdoctrue
          \includeonly{\childdocname}
          \def\childdocjob{#1}
          \def\jobname{#1}
        }
      \fi
      \expandafter
    \endgroup
    \childdoctmp
  \fi
}
%    \end{macrocode}

% \macro{\childdocof}
% The command |\childdocof| redirects
% compilation to the main file |#1|.
%    \begin{macrocode}
\newcommand{\childdocof}[1]
{
  \childdocdisable
  \childdoctrue
  \includeonly{\childdocname}
  \def\jobname{#1}
  \def\childdocjob{#1}
  \input{#1}
}
%    \end{macrocode}

% \macro{\childdocby}
% The command |\childdocby| ....
%    \begin{macrocode}
\newcommand{\childdocby}[2][]
{
  \childdocdisable
  \childdoctrue
  \childdocmanualtrue
  \if?#1?\else
    \def\jobname{#2}
  \fi
  \def\childdocjob{#2}
  \input{#2}
  \endinput
}
%    \end{macrocode}

% \macro{\childdocforward}
% The command |\childdocforward| redirects
% compilation to the main file or
% (if the optional argument is given) a child file.
% Parameters are set as if the main file
% or a child file starting with |\childdocof| was compiled.
% Then compilation is handed over to the main file:
%    \begin{macrocode}
\newcommand{\childdocforward}[2][]
{
  \begingroup
    \if?#1?
      \def\childdoctmp
      {
        \def\childdocname{#2}
        \def\childdocjob{#2}
        \def\jobname{#2}
        \input{#2}
        \endinput
      }
    \else
      \def\childdoctmp
      {
        \childdocdisable
        \def\childdocname{#2}
        \childdoctrue
        \includeonly{#2}
        \def\childdocjob{#1}
        \def\jobname{#1}
        \input{#1}
        \endinput
      }
    \fi
    \expandafter
  \endgroup
  \childdoctmp
}
%    \end{macrocode}

% \macro{\childdocforwardprefix}
% The command |\childdocforwardprefix| redirects
% compilation to the main or a child file by means of a pattern.
% The prefix |#1| in the current filename is replaced by |#2|
% and the suffix of the current filename is kept
% (it is assumed that the filename does not contain the substring `|~~~|'
% which is used as a delimiter).
% Compilation is handed over to the new file by |\childdocforward|:
%    \begin{macrocode}
\newcommand{\childdocforwardprefix}[3][]
{
  \begingroup
    \def\childdocextract #2##1~~~{\def\childdoctmp{\childdocforward[#1]{#3##1}}}
    \expandafter\childdocextract\childdocname~~~
    \expandafter
  \endgroup
  \childdoctmp
}
%    \end{macrocode}

% \macro{\childdoc}
% The deprecated macro |\childdoc| is a legacy version of |\childdocmain|:
%    \begin{macrocode}
\newcommand{\childdoc}{\childdocmain}
%    \end{macrocode}

% \macro{\childdocredirect}
% The deprecated macro |\childdocredirect| is a legacy version
% of |\childdocforward| and |\childdocforwardprefix|:
%    \begin{macrocode}
\newcommand{\childdocredirect}[2][]
{
  \begingroup
    \if?#1?
      \def\childdoctmp{\childdocforward{#2}}
    \else
      \def\childdoctmp{\childdocforwardprefix{#1}{#2}}
    \fi
    \expandafter
  \endgroup
  \childdoctmp
}
%    \end{macrocode}

%\iffalse
%</package>
%\fi
%
\endinput
|\\
|\childdocforward{|\textit{main}|}|\\
\end{tabular}
\end{center}
%
or alternatively with:
%
\begin{center}
\begin{tabular}{l}
|% \iffalse
%
% childdoc.dtx Copyright (C) 2017-2018 Niklas Beisert
%
% This work may be distributed and/or modified under the
% conditions of the LaTeX Project Public License, either version 1.3
% of this license or (at your option) any later version.
% The latest version of this license is in
%   http://www.latex-project.org/lppl.txt
% and version 1.3 or later is part of all distributions of LaTeX
% version 2005/12/01 or later.
%
% This work has the LPPL maintenance status `maintained'.
%
% The Current Maintainer of this work is Niklas Beisert.
%
% This work consists of the files childdoc.dtx and childdoc.ins
% and the derived files childdoc.def and cdocsamp.tex with
% cdocsch1.tex, cdocsch2.tex, cdocsdrf.tex, cdocsfn1.tex, cdocsfn2.tex.
%
%<package>\ifdefined\childdocmain\endinput\fi
%<package>\ProvidesFile{childdoc.def}[2018/12/30 v2.0 child document driver]
%<samplemain>\ProvidesFile{cdocsamp.tex}[2018/12/30 v2.0 sample for childdoc]
%<*driver>
%\ProvidesFile{childdoc.drv}[2018/12/30 v2.0 childdoc reference manual file]
\PassOptionsToClass{10pt,a4paper}{article}
\documentclass{ltxdoc}

\usepackage[margin=35mm]{geometry}
\usepackage{hyperref}
\usepackage{hyperxmp}
\usepackage[usenames]{color}

\hypersetup{colorlinks=true}
\hypersetup{pdfstartview=FitH}
\hypersetup{pdfpagemode=UseNone}
\hypersetup{pdfsource={}}
\hypersetup{pdflang={en-UK}}
\hypersetup{pdfcopyright={Copyright 2017-2018 Niklas Beisert.
  This work may be distributed and/or modified under the
  conditions of the LaTeX Project Public License, either version 1.3
  of this license or (at your option) any later version.}}
\hypersetup{pdflicenseurl={http://www.latex-project.org/lppl.txt}}
\hypersetup{pdfcontactaddress={ETH Zurich, ITP, HIT K,
  Wolfgang-Pauli-Strasse 27}}
\hypersetup{pdfcontactpostcode={8093}}
\hypersetup{pdfcontactcity={Zurich}}
\hypersetup{pdfcontactcountry={Switzerland}}
\hypersetup{pdfcontactemail={nbeisert@itp.phys.ethz.ch}}
\hypersetup{pdfcontacturl={http://people.phys.ethz.ch/\xmptilde nbeisert/}}

\newcommand{\secref}[1]{\hyperref[#1]{section \ref*{#1}}}

\parskip1ex
\parindent0pt
\let\olditemize\itemize
\def\itemize{\olditemize\parskip0pt}

\begin{document}

\title{The \textsf{childdoc} Package}
\hypersetup{pdftitle={The childdoc Package}}
\author{Niklas Beisert\\[2ex]
  Institut f\"ur Theoretische Physik\\
  Eidgen\"ossische Technische Hochschule Z\"urich\\
  Wolfgang-Pauli-Strasse 27, 8093 Z\"urich, Switzerland\\[1ex]
  \href{mailto:nbeisert@itp.phys.ethz.ch}
  {\texttt{nbeisert@itp.phys.ethz.ch}}}
\hypersetup{pdfauthor={Niklas Beisert}}
\hypersetup{pdfsubject={Manual for the LaTeX2e Package childdoc}}
\date{30 December 2018, \textsf{v2.0}}
\maketitle

\begin{abstract}\noindent
\textsf{childdoc} is a \LaTeXe{} package
that enables the direct compilation
of document sections included by |\include|
to individual files.
\end{abstract}

\begingroup
\parskip0ex
\tableofcontents
\endgroup

%%%%%%%%%%%%%%%%%%%%%%%%%%%%%%%%%%%%%%%%%%%%%%%%%%%%%%%%%%%%%%%%%%%%%%%%%%%%%%%%
%%%%%%%%%%%%%%%%%%%%%%%%%%%%%%%%%%%%%%%%%%%%%%%%%%%%%%%%%%%%%%%%%%%%%%%%%%%%%%%%
\section{Introduction}

\LaTeX{} provides a mechanism to structure a large document (such as a book)
into a main file and several child files (containing the chapters)
using the |\include| command.
This mechanism is beneficial for documents
which span hundreds of pages in order to
make the source file(s) more manageable.
Moreover, compilation can be restricted to
selected child files by means of the |\includeonly| command.
The latter feature can be used to reduce the compilation time while editing
(this was significantly more useful in the earlier days of \LaTeX{})
or to generate a smaller document which is easier to navigate.
Another application of |\includeonly| is to generate
documents consisting of selected parts of the complete document.

However, there are a few drawbacks of the plain |\include| mechanism:
\begin{itemize}
\item
The child files cannot be compiled on their own,
they can only be compiled via the main file.
A naive editing environment
(such as a text editor with an option
to have the current file processed by \LaTeX)
may require one to switch to the main file before compiling;
attempting to compile the child file produces errors.
\item
The main file must be modified (each time)
to adjust the |\includeonly| command
to the present needs. This easily leaves the main file in a messy state.
\item
The generated document will always carry the filename
of the main document. This is inconvenient if
several child files are to be compiled and
to be kept for distribution.
\end{itemize}

The present package provides a simple interface
to make child files individually compilable by \LaTeX{}.
Compiling a child file then has the same effect as compiling
the main file with an |\includeonly| command
to select the appropriate child.
Moreover the generated document will carry the name of the child
rather than the main file.
This resolves all three above issues.

This feature is meant to make the editing of books,
thesis documents and lecture notes somewhat more convenient.
However, the package can also be used efficiently for
composing a series of documents (such as exercise sheets)
which are typically distributed individually.
It then assists the author in generating the individual documents
(potentially in different versions)
as well as a document containing the collected series.
Another application is in developing style files
or other kinds of included material
where compilation of the style file could redirect
to a sample or test file.

%%%%%%%%%%%%%%%%%%%%%%%%%%%%%%%%%%%%%%%%%%%%%%%%%%%%%%%%%%%%%%%%%%%%%%%%%%%%%%%%
%%%%%%%%%%%%%%%%%%%%%%%%%%%%%%%%%%%%%%%%%%%%%%%%%%%%%%%%%%%%%%%%%%%%%%%%%%%%%%%%
\section{Usage}

First of all, the package \textsf{childdoc} is \emph{not} a standard
\LaTeXe{} |.sty| style file! Therefore it needs to be invoked in
a non-standard way.

%%%%%%%%%%%%%%%%%%%%%%%%%%%%%%%%%%%%%%%%%%%%%%%%%%%%%%%%%%%%%%%%%%%%%%%%%%%%%%%%
\subsection{Included Files}
\label{sec:include}

%%%%%%%%%%%%%%%%%%%%%%%%%%%%%%%%%%%%%%%%
\DescribeMacro{\childdocmain}
To use the package, add the commands
\begin{center}
\begin{tabular}{l}
|% \iffalse
%
% childdoc.dtx Copyright (C) 2017-2018 Niklas Beisert
%
% This work may be distributed and/or modified under the
% conditions of the LaTeX Project Public License, either version 1.3
% of this license or (at your option) any later version.
% The latest version of this license is in
%   http://www.latex-project.org/lppl.txt
% and version 1.3 or later is part of all distributions of LaTeX
% version 2005/12/01 or later.
%
% This work has the LPPL maintenance status `maintained'.
%
% The Current Maintainer of this work is Niklas Beisert.
%
% This work consists of the files childdoc.dtx and childdoc.ins
% and the derived files childdoc.def and cdocsamp.tex with
% cdocsch1.tex, cdocsch2.tex, cdocsdrf.tex, cdocsfn1.tex, cdocsfn2.tex.
%
%<package>\ifdefined\childdocmain\endinput\fi
%<package>\ProvidesFile{childdoc.def}[2018/12/30 v2.0 child document driver]
%<samplemain>\ProvidesFile{cdocsamp.tex}[2018/12/30 v2.0 sample for childdoc]
%<*driver>
%\ProvidesFile{childdoc.drv}[2018/12/30 v2.0 childdoc reference manual file]
\PassOptionsToClass{10pt,a4paper}{article}
\documentclass{ltxdoc}

\usepackage[margin=35mm]{geometry}
\usepackage{hyperref}
\usepackage{hyperxmp}
\usepackage[usenames]{color}

\hypersetup{colorlinks=true}
\hypersetup{pdfstartview=FitH}
\hypersetup{pdfpagemode=UseNone}
\hypersetup{pdfsource={}}
\hypersetup{pdflang={en-UK}}
\hypersetup{pdfcopyright={Copyright 2017-2018 Niklas Beisert.
  This work may be distributed and/or modified under the
  conditions of the LaTeX Project Public License, either version 1.3
  of this license or (at your option) any later version.}}
\hypersetup{pdflicenseurl={http://www.latex-project.org/lppl.txt}}
\hypersetup{pdfcontactaddress={ETH Zurich, ITP, HIT K,
  Wolfgang-Pauli-Strasse 27}}
\hypersetup{pdfcontactpostcode={8093}}
\hypersetup{pdfcontactcity={Zurich}}
\hypersetup{pdfcontactcountry={Switzerland}}
\hypersetup{pdfcontactemail={nbeisert@itp.phys.ethz.ch}}
\hypersetup{pdfcontacturl={http://people.phys.ethz.ch/\xmptilde nbeisert/}}

\newcommand{\secref}[1]{\hyperref[#1]{section \ref*{#1}}}

\parskip1ex
\parindent0pt
\let\olditemize\itemize
\def\itemize{\olditemize\parskip0pt}

\begin{document}

\title{The \textsf{childdoc} Package}
\hypersetup{pdftitle={The childdoc Package}}
\author{Niklas Beisert\\[2ex]
  Institut f\"ur Theoretische Physik\\
  Eidgen\"ossische Technische Hochschule Z\"urich\\
  Wolfgang-Pauli-Strasse 27, 8093 Z\"urich, Switzerland\\[1ex]
  \href{mailto:nbeisert@itp.phys.ethz.ch}
  {\texttt{nbeisert@itp.phys.ethz.ch}}}
\hypersetup{pdfauthor={Niklas Beisert}}
\hypersetup{pdfsubject={Manual for the LaTeX2e Package childdoc}}
\date{30 December 2018, \textsf{v2.0}}
\maketitle

\begin{abstract}\noindent
\textsf{childdoc} is a \LaTeXe{} package
that enables the direct compilation
of document sections included by |\include|
to individual files.
\end{abstract}

\begingroup
\parskip0ex
\tableofcontents
\endgroup

%%%%%%%%%%%%%%%%%%%%%%%%%%%%%%%%%%%%%%%%%%%%%%%%%%%%%%%%%%%%%%%%%%%%%%%%%%%%%%%%
%%%%%%%%%%%%%%%%%%%%%%%%%%%%%%%%%%%%%%%%%%%%%%%%%%%%%%%%%%%%%%%%%%%%%%%%%%%%%%%%
\section{Introduction}

\LaTeX{} provides a mechanism to structure a large document (such as a book)
into a main file and several child files (containing the chapters)
using the |\include| command.
This mechanism is beneficial for documents
which span hundreds of pages in order to
make the source file(s) more manageable.
Moreover, compilation can be restricted to
selected child files by means of the |\includeonly| command.
The latter feature can be used to reduce the compilation time while editing
(this was significantly more useful in the earlier days of \LaTeX{})
or to generate a smaller document which is easier to navigate.
Another application of |\includeonly| is to generate
documents consisting of selected parts of the complete document.

However, there are a few drawbacks of the plain |\include| mechanism:
\begin{itemize}
\item
The child files cannot be compiled on their own,
they can only be compiled via the main file.
A naive editing environment
(such as a text editor with an option
to have the current file processed by \LaTeX)
may require one to switch to the main file before compiling;
attempting to compile the child file produces errors.
\item
The main file must be modified (each time)
to adjust the |\includeonly| command
to the present needs. This easily leaves the main file in a messy state.
\item
The generated document will always carry the filename
of the main document. This is inconvenient if
several child files are to be compiled and
to be kept for distribution.
\end{itemize}

The present package provides a simple interface
to make child files individually compilable by \LaTeX{}.
Compiling a child file then has the same effect as compiling
the main file with an |\includeonly| command
to select the appropriate child.
Moreover the generated document will carry the name of the child
rather than the main file.
This resolves all three above issues.

This feature is meant to make the editing of books,
thesis documents and lecture notes somewhat more convenient.
However, the package can also be used efficiently for
composing a series of documents (such as exercise sheets)
which are typically distributed individually.
It then assists the author in generating the individual documents
(potentially in different versions)
as well as a document containing the collected series.
Another application is in developing style files
or other kinds of included material
where compilation of the style file could redirect
to a sample or test file.

%%%%%%%%%%%%%%%%%%%%%%%%%%%%%%%%%%%%%%%%%%%%%%%%%%%%%%%%%%%%%%%%%%%%%%%%%%%%%%%%
%%%%%%%%%%%%%%%%%%%%%%%%%%%%%%%%%%%%%%%%%%%%%%%%%%%%%%%%%%%%%%%%%%%%%%%%%%%%%%%%
\section{Usage}

First of all, the package \textsf{childdoc} is \emph{not} a standard
\LaTeXe{} |.sty| style file! Therefore it needs to be invoked in
a non-standard way.

%%%%%%%%%%%%%%%%%%%%%%%%%%%%%%%%%%%%%%%%%%%%%%%%%%%%%%%%%%%%%%%%%%%%%%%%%%%%%%%%
\subsection{Included Files}
\label{sec:include}

%%%%%%%%%%%%%%%%%%%%%%%%%%%%%%%%%%%%%%%%
\DescribeMacro{\childdocmain}
To use the package, add the commands
\begin{center}
\begin{tabular}{l}
|\input{childdoc.def}|\\
|\childdocmain{}|\\
\end{tabular}
\end{center}
at the very top of the main \LaTeX{} file,
in particular \emph{before} the |\documentclass| statement!
The argument of |\childdocmain| should be left empty
(but it must be present).

%%%%%%%%%%%%%%%%%%%%%%%%%%%%%%%%%%%%%%%%
\DescribeMacro{\childdocof}
Furthermore, add the commands
\begin{center}
\begin{tabular}{l}
|\input{childdoc.def}|\\
|\childdocof{|\textit{main}|}|\\
\end{tabular}
\end{center}
at the top of every child file \textit{child}
which is included by |\include{|\textit{child}|}|
from within the main file
(or at least for those files to be compiled individually).
The argument \textit{main} must be the filename of the main file.

There are a couple of
considerations in setting up the main and child documents:

%%%%%%%%%%%%%%%%%%%%%%%%%%%%%%%%%%%%%%%%
\paragraph{Restrictions.}

Please note the following restrictions:
\begin{itemize}
\item
|\childdocmain| must be called with one argument \textit{main}
to ensure compatibility with earlier version of the package.
It must either be empty (|\childdocmain{}|)
or precisely match the filename of the main file in which it is specified.
See \secref{sec:detection} for further information.
\item
The filename \textit{main} must be specified without the |.tex| extension.
\item
The filename \textit{main} is case sensitive
(even in case-insensitive file systems)
due to internal string comparison.
\item
The argument \textit{main} should be fully expanded, it cannot be a macro.
\item
Subdirectories and special characters should be avoided in filenames.
\item
The command |\childdocmain{|\textit{main}|}| must be followed by a whitespace.
It should not be followed immediately by another command
or by a comment mark `|%|'.
This is because the \TeX{} parser reads the token immediately following
the argument of |\childdocmain| and puts it
at the beginning of every child section;
however, a white\-space is ignored.
\end{itemize}

%%%%%%%%%%%%%%%%%%%%%%%%%%%%%%%%%%%%%%%%
\paragraph{Content of Main File.}

It is advisable to place all content in the child files included by |\include|.
Any output contained in the main file will appear in all child documents
unless suppressed manually;
it cannot be suppressed automatically by the |\includeonly| directive
and thus should normally be avoided.
A method to include some content in the main file
by means of conditional processing is described in \secref{sec:conditional}.

%%%%%%%%%%%%%%%%%%%%%%%%%%%%%%%%%%%%%%%%
\paragraph{Page Numbering.}

When only a part of the document is compiled,
the appropriate numbering of pages
(as well as other status parameters)
is determined from the |.aux| files.
The latter contain information from previous passes.
However this information needs to propagate through
all intermediate child documents.
Therefore the page numbering in child documents may well
be inconsistent until the complete document is compiled at least once.

A useful (if unconventional) way to always ensure a consistent
page numbering is to restart the numbering in each child document
and denote the pages by `\textit{child}|.|\textit{page}'
where \textit{child} represents the chapter/section number of the child file.
This can be achieved by the command
|\numberwithin{page}{|\textit{child}|}|
of the \textsf{amsmath} package
where \textit{child} can be |chapter| or |section|
depending on the chosen structuring.
Alternatively, one can modify the macro |\thepage| appropriately
and reset the counter |page| at the start of each child file.

%%%%%%%%%%%%%%%%%%%%%%%%%%%%%%%%%%%%%%%%%%%%%%%%%%%%%%%%%%%%%%%%%%%%%%%%%%%%%%%%
\subsection{Conditional Processing}
\label{sec:conditional}

The package provides a mechanism to compile different versions
of a document. To customise the versions further some conditional processing
can come in handy to distinguish which version is being compiled.
The package provides two macros to describe the compilation context:

%%%%%%%%%%%%%%%%%%%%%%%%%%%%%%%%%%%%%%%%
\DescribeMacro{\ifchilddoc}
The conditional |\ifchilddoc| distinguishes between the compilation of
child documents and the main document:
%
\begin{center}
|\ifchilddoc |\textit{child-code}| |[|\||else |\textit{main-code}]| \||fi|
\end{center}

%%%%%%%%%%%%%%%%%%%%%%%%%%%%%%%%%%%%%%%%
\DescribeMacro{\childdocname}
\DescribeMacro{\childdocjob}
The macro |\childdocname| contains the filename (without extension)
of the main or child file being processed.
Note that |\childdocjob| will always contain the name of the main file.

%%%%%%%%%%%%%%%%%%%%%%%%%%%%%%%%%%%%%%%%
\paragraph{Title Page.}

Conditional processing can be used to include a title or banner page
in the main document when proper precautions are taken.
Importantly, the code in the main file should ensure that the page counter
(as well as other status parameters which are stored in the |.aux| files)
takes the same value after the conditional processing.
Otherwise the page numbers may take divergent values
depending on which part is compiled.

For example, a title page could be declared by:
%
\begin{center}
\begin{tabular}{l}
|\ifchilddoc\||else|\\
|\addtocounter{page}{-1}|\\
\textit{code for title page}\\
|\newpage|\\
|\||fi|
\end{tabular}
\end{center}
%
A banner page for the child documents can be generated by:
%
\begin{center}
\begin{tabular}{l}
|\ifchilddoc|\\
|\addtocounter{page}{-1}|\\
\textit{code for banner page}\\
|\newpage|\\
|\||fi|
\end{tabular}
\end{center}
%
Here one could write a message such as:
\begin{center}
|This is the part \childdocname{} of \childdocjob{}.|
\end{center}

%%%%%%%%%%%%%%%%%%%%%%%%%%%%%%%%%%%%%%%%%%%%%%%%%%%%%%%%%%%%%%%%%%%%%%%%%%%%%%%%
\subsection{Flags}
\label{sec:flags}

The package makes it easy to generate different versions
of the main or child documents.
To this end compilation flags can be defined
and assigned different default values.
They will be particularly useful in conjunction
with the forwarding mechanism described in \secref{sec:forward}.

For example, it may be useful to have a flag |\version|
which can be set to |draft| or |final|.
The document source will contain some conditional code
depending on the value of |\version|.
Suppose further, the flag should default to |final| for the main file
and to |draft| for child files
which is a natural assignment for editing the document.
This is achieved by placing the following code
in the preamble of the main document
(below the |\childdocmain| directive):
%
\begin{center}
\begin{tabular}{l}
|\ifchilddoc|\\
|\providecommand{\version}{draft}|\\
|\||else|\\
|\providecommand{\version}{final}|\\
|\||fi|
\end{tabular}
\end{center}
%
The definition by |\providecommand| makes sure
that previous definitions are not overwritten.
Further statements |\providecommand{\version}{...}|
can thus be added before the above code to override it.

For the main file, one might add a line
(between |\childdocmain| and the above block)
%
\begin{center}
|%\ifchilddoc\||else\providecommand{\version}{draft}\||fi|
\end{center}
%
which can be uncommented to produce a draft version.
Likewise one can add a line to the very top of a child file
(above the |\childdocof{|\textit{main}|}| directive)
%
\begin{center}
|%\providecommand{\version}{final}|
\end{center}
%
which can be uncommented to produce the final version of this child document.

%%%%%%%%%%%%%%%%%%%%%%%%%%%%%%%%%%%%%%%%%%%%%%%%%%%%%%%%%%%%%%%%%%%%%%%%%%%%%%%%
\subsection{Forwarding}
\label{sec:forward}

Different versions of the main or child documents
using compilation flags as described in \secref{sec:flags}
can be (permanently) stored in different files
for convenient compilation, viewing and distribution.
To this end, the package defines a command
to pass on compilation to a different file:

%%%%%%%%%%%%%%%%%%%%%%%%%%%%%%%%%%%%%%%%
\DescribeMacro{\childdocforward}
The command |\childdocforward| redirects processing to
another source file:
%
\begin{center}
\begin{tabular}{l}
|\input{childdoc.def}|\\
|\childdocforward[|\textit{main}|]{|\textit{dest}|}|\\
\end{tabular}
\end{center}
%
The argument \textit{dest} is the destination file
(without extension).
It should be the main file or one of the child files.
Note that further \textsf{childdoc} directives
such as |\childdocof| and |\childdocforward|
in the indicated file will be processed in this form.
The optional argument \textit{main}
passes on directly to the main file \textit{main}
while pretending to compile the child \textit{dest}.
This form behaves as if \textit{dest}
issues |\childdocof{|\textit{main}|}| right away,
and no further \textsf{childdoc} directives will be processed.

%%%%%%%%%%%%%%%%%%%%%%%%%%%%%%%%%%%%%%%%
\DescribeMacro{\...prefix}
In the alternative form |\childdocforwardprefix|,
%
\begin{center}
\begin{tabular}{l}
|\input{childdoc.def}|\\
|\childdocforwardprefix[|\textit{main}|]{|\textit{prefix}|}{|\textit{dest}|}|
\end{tabular}
\end{center}
%
the destination file is determined by a pattern
depending on the current file:
To make this work, the current file must be called
`{\textit{prefix}\hspace{0.2em}\textit{suffix}}'
with \textit{prefix} matching precisely the argument.
Processing is then passed on to the file
`{\textit{dest}\hspace{0.2em}\textit{suffix}}'.
Surely, the same effect is achieved by
directly specifying the
argument `{\textit{dest}\hspace{0.2em}\textit{suffix}}'
in the first form.
However, that requires to set up a different file
for each child. With the alternative form of the command
all these files can have exactly the same content
which simplifies setting them up and maintaining them.

For example, the following file |draft.tex|
with a compilation flag |\version| as described in \secref{sec:flags}
compiles the main document as a draft:
%
\begin{center}
\begin{tabular}{l}
|\def\version{draft}|\\
|\input{childdoc.def}|\\
|\childdocforward{|\textit{main}|}|
\end{tabular}
\end{center}
%
Likewise, the following files |final|\textit{nn}|.tex|
compile the final version of the child document
|child|\textit{nn}|.tex|:
%
\begin{center}
\begin{tabular}{l}
|\def\version{final}|\\
|\input{childdoc.def}|\\
|\childdocforwardprefix{final}{child}|
\end{tabular}
\end{center}
%

Note that when several versions of a main file and/or of each child file
are to be generated, it may be convenient to set up a |Makefile| or
shell script to automatise the process.

%%%%%%%%%%%%%%%%%%%%%%%%%%%%%%%%%%%%%%%%%%%%%%%%%%%%%%%%%%%%%%%%%%%%%%%%%%%%%%%%
\subsection{Command Line Processing}
\label{sec:commandline}

The effect of redirection files can also be achieved by invoking
the \LaTeX{} compiler with a more elaborate command line.
Most conveniently this should be done as part
of a shell script or a |Makefile|.

When using \textsf{childdoc} in the main file, the following
command lines effectively perform a redirection
(note that depending on the shell being used,
backslashes may have to be doubled: `|\|' $\to$ `|\\|'):
%
\begin{center}
|... -jobname "|\textit{target}|" |\\|"|[\textit{flags}]%
|\input{childdoc.def}\childdocforward[|\textit{main}|]{|\textit{dest}|}"|
\end{center}
%
Here \textit{target} is the name of the output file,
\textit{main} is the name of the main file
and \textit{dest} is the name of the main or child file to be processed
(all filenames without extensions).
The optional argument \textit{main} can be omitted
if \textit{main} matches \textit{dest}.
Optionally, compilation \textit{flags} can be defined via |\def| commands.
This command line makes the \TeX{} engine believe
it is compiling the file \textit{target}
whose content is specified as the latter parameter.
The provided code then forwards the processing to
\textit{main} or \textit{dest} as described in \secref{sec:forward}.

%%%%%%%%%%%%%%%%%%%%%%%%%%%%%%%%%%%%%%%%%%%%%%%%%%%%%%%%%%%%%%%%%%%%%%%%%%%%%%%%
\subsection{Include by Input}
\label{sec:input}

Including child documents by |\include| has some restrictions by design.
Most notably, the content of a child document always occupies
its own set of pages; pages cannot be shared between child documents.
Usually, this behaviour makes perfect sense
because each child document contain an essential part of the document.
However, in some situations it may be desirable to compose
a document from a collection of parts
without having mandatory page breaks between then.
For this case, the package
provides a mechanism to include parts
by |\input| which can also be processed individually.
However, by construction this mechanism
requires manual handling of the content to be output.

%%%%%%%%%%%%%%%%%%%%%%%%%%%%%%%%%%%%%%%%
\DescribeMacro{\ifchilddocmanual}
The main file should be prepared as usual, see \secref{sec:include}.
However, the document body must make a distinction
between processing of an individual part and of the main document, e.g.:
%
\begin{center}
\begin{tabular}{l}
|\ifchilddocmanual|\\
|\input{\childdocname}|\\
|\||else|\\
\textit{document body with }|\input{|\textit{part}|}|\\
|\||fi|
\end{tabular}
\end{center}
%
The conditional |\ifchilddocmanual| is true whenever
a part to be included by |\input| is being compiled,
and the name of the part is stored in |\childdocname|.

%%%%%%%%%%%%%%%%%%%%%%%%%%%%%%%%%%%%%%%%
\DescribeMacro{\childdocby}
Each part to be included by |\input| should start with:
%
\begin{center}
\begin{tabular}{l}
|\input{childdoc.def}|\\
|\childdocby{|\textit{main}|}|\\
\end{tabular}
\end{center}
%
The directive |\childdocby| is similar to |\childdocof|
described in \secref{sec:include},
but the subsequent selection of content must be done manually.
To that end, both |\ifchilddoc| and |\ifchilddocmanual|
will be true upon processing of a part,
and the name of the part is stored in |\childdocname|.
Note that |\jobname| will be set to the filename of the current part
so that each part receives an individual |.aux| file
that does not interfere with the |.aux| file(s) of the main document.
This behaviour can be altered by the alternative form
|\childdocby[*]{|\textit{main}|}| (with a non-empty optional argument)
which uses the |.aux| file of the main document
by setting |\jobname| to \textit{main}.

%%%%%%%%%%%%%%%%%%%%%%%%%%%%%%%%%%%%%%%%%%%%%%%%%%%%%%%%%%%%%%%%%%%%%%%%%%%%%%%%
\subsection{Driver Development}
\label{sec:driver}

The \textsf{childdoc} mechanism can also be use for the development
of definition files such as \LaTeX{} styles or classes.
This case differs from the above setup with multiple parts
included by |\include| in that no |\includeonly| should be invoked.
This can be achieved by starting the include file
(before |\ProvidesPackage|) with:
%
\begin{center}
\begin{tabular}{l}
|\input{childdoc.def}|\\
|\childdocforward{|\textit{main}|}|\\
\end{tabular}
\end{center}
%
or alternatively with:
%
\begin{center}
\begin{tabular}{l}
|\input{childdoc.def}|\\
|\childdocby{|\textit{main}|}|\\
\end{tabular}
\end{center}
%
Both forms have slightly different effects as described above.
The main file is prepared as usual, see \secref{sec:include}.

%%%%%%%%%%%%%%%%%%%%%%%%%%%%%%%%%%%%%%%%%%%%%%%%%%%%%%%%%%%%%%%%%%%%%%%%%%%%%%%%
\subsection{Legacy Detection}
\label{sec:detection}

The directive |\childdocmain| in the main file can detect
whether the complete document or merely a child is to be compiled
even without using the directive |\childdocof|.
This method is deprecated because it is less robust
and there is no compelling reason to use it;
it is merely provided for backward compatibility
and it may be removed in future versions.

If the detection mechanism is to be used,
it is mandatory to correctly specify
the filename of the main file as the argument of |\childdocmain|:
%
\begin{center}
\begin{tabular}{l}
|\input{childdoc.def}|\\
|\childdocmain{|\textit{main}|}|\\
\end{tabular}
\end{center}
%
If |\jobname| does not match the argument \textit{main} of |\childdocmain|,
it is assumed that |\jobname| points to the child file to be compiled.
When using |\childdocmain| with the main file specified as argument,
it suffices to start a child file
with just |\input{|\textit{main}|}|
without loading of the package and using |\childdocof|.
If instead all processing is done
with the appropriate \textsf{childdoc} directives,
the argument of \textit{main} of |\childdocmain| can be empty.

An alternative version of the command line processing described
in \secref{sec:commandline} using the detection mechanism reads:
%
\begin{center}
|... -jobname "|\textit{target}|" "|[\textit{flags}]%
[|\def\jobname{|\textit{dest}|}|]|\input{|\textit{main}|}"|
\end{center}

%%%%%%%%%%%%%%%%%%%%%%%%%%%%%%%%%%%%%%%%%%%%%%%%%%%%%%%%%%%%%%%%%%%%%%%%%%%%%%%%
\subsection{Manual Code}
\label{sec:manual}

In case one cannot be certain whether the definitions file |childdoc.def|
is installed on the target \TeX{} distribution
and one prefers not to ship it,
it is conceivable to paste a few relevant commands into the sources.

To that end, drop all statements |\input{childdoc.def}|
and perform the replacements as outlined below.
Instead of |\childdocmain{|\textit{main}|}| add the following code
to the top of the main file:
%
\begin{center}
\begin{tabular}{l}
|\||ifdefined\childdocname\endinput\||fi\newif\ifchilddoc|\\
|\edef\childdocname{\scantokens\expandafter{\jobname\noexpand}}|\\
|\def\childdocmain{|\textit{main}|}\||ifx\childdocmain\childdocname\||else|\\
|\childdoctrue\includeonly{\childdocname}\let\jobname\childdocmain\||fi|\\
\end{tabular}
\end{center}
%
Instead of |\childdocof{|\textit{main}|}| just include the main file
at the top of each child file:
%
\begin{center}
|\input{|\textit{main}|}|
\end{center}
%
A simple redirection |\childdocforward{|\textit{dest}|}| is achieved by:
%
\begin{center}
|\def\jobname{|\textit{dest}|}\input{\jobname}|
\end{center}
%
The redirection with prefix
|\childdocforwardprefix[|\textit{prefix}|]{|\textit{dest}|}|
is accomplished by:
%
\begin{center}
\begin{tabular}{l}
|{\edef\jobname{\scantokens\expandafter{\jobname\noexpand}}|\\
|\def\redirectjob |\textit{prefix}|#1~~~{\gdef\jobname{|\textit{dest}|#1}}|\\
|\expandafter\redirectjob\jobname~~~}\input{\jobname}|
\end{tabular}
\end{center}

In an alternative approach,
child documents can be compiled by a specific command line
without additional code or specific definitions:
%
\begin{center}
|... -jobname "|\textit{target}|" "|[\textit{flags}]%
|\includeonly{|\textit{dest}|}\input{|\textit{main}|}"|
\end{center}
%

%%%%%%%%%%%%%%%%%%%%%%%%%%%%%%%%%%%%%%%%%%%%%%%%%%%%%%%%%%%%%%%%%%%%%%%%%%%%%%%%
%%%%%%%%%%%%%%%%%%%%%%%%%%%%%%%%%%%%%%%%%%%%%%%%%%%%%%%%%%%%%%%%%%%%%%%%%%%%%%%%
\section{Information}

%%%%%%%%%%%%%%%%%%%%%%%%%%%%%%%%%%%%%%%%%%%%%%%%%%%%%%%%%%%%%%%%%%%%%%%%%%%%%%%%
\subsection{Copyright}

Copyright \copyright{} 2017--2018 Niklas Beisert

This work may be distributed and/or modified under the
conditions of the \LaTeX{} Project Public License, either version 1.3
of this license or (at your option) any later version.
The latest version of this license is in
  \url{http://www.latex-project.org/lppl.txt}
and version 1.3 or later is part of all distributions of \LaTeX{}
version 2005/12/01 or later.

This work has the LPPL maintenance status `maintained'.

The Current Maintainer of this work is Niklas Beisert.

This work consists of the files |README.txt|, |childdoc.ins| and |childdoc.dtx|
as well as the derived files |childdoc.def|, |cdocsamp.tex|
with |cdocsch1.tex|, |cdocsch2.tex|, |cdocspt3.tex|, |cdocspt4.tex|,
|cdocsdrf.tex|, |cdocsfn1.tex|, |cdocsfn2.tex|
as well as |childdoc.pdf|.

%%%%%%%%%%%%%%%%%%%%%%%%%%%%%%%%%%%%%%%%%%%%%%%%%%%%%%%%%%%%%%%%%%%%%%%%%%%%%%%%
\subsection{Files and Installation}

The package consists of the files:
%
\begin{center}
\begin{tabular}{ll}
    |README.txt|   & readme file \\
    |childdoc.ins| & installation file \\
    |childdoc.dtx| & source file \\
    |childdoc.def| & definition file \\
    |cdocsamp.tex| & sample main file \\
    |cdocsch1.tex| & sample include file \\
    |cdocsch2.tex| & sample include file \\
    |cdocspt3.tex| & sample part file \\
    |cdocspt4.tex| & sample part file \\
    |cdocsdrf.tex| & sample redirection file \\
    |cdocsfn1.tex| & sample redirection file \\
    |cdocsfn2.tex| & sample redirection file \\
    |childdoc.pdf| & manual
\end{tabular}
\end{center}
%
The distribution consists of the files
|README.txt|, |childdoc.ins| and |childdoc.dtx|.
%
\begin{itemize}
\item
Run (pdf)\LaTeX{} on |childdoc.dtx|
to compile the manual |childdoc.pdf| (this file).
\item
Run \LaTeX{} on |childdoc.ins| to create the definitions file |childdoc.def|
and the sample |cdocsamp.tex| with include files
|cdocsch1.tex|, |cdocsch2.tex|, |cdocspt3.tex|, |cdocspt4.tex|,
|cdocsdrf.tex|, |cdocsfn1.tex|, |cdocsfn2.tex|.
Then copy the file |childdoc.def| to an appropriate directory of your \LaTeX{}
distribution, e.g.\ \textit{texmf-root}|/tex/latex/childdoc|.
\end{itemize}

%%%%%%%%%%%%%%%%%%%%%%%%%%%%%%%%%%%%%%%%%%%%%%%%%%%%%%%%%%%%%%%%%%%%%%%%%%%%%%%%
\subsection{Related CTAN Packages}

There are several other packages which offer a similar functionality:
%
\begin{itemize}
\item
The packages
\href{http://ctan.org/pkg/docmute}{\textsf{docmute}},
\href{http://ctan.org/pkg/includex}{\textsf{includex}} and
\href{http://ctan.org/pkg/standalone}{\textsf{standalone}}
provide commands to include only the document body of
a child file thus allowing both files to be compiled individually.
\item
The packages \href{http://ctan.org/pkg/subdocs}{\textsf{subdocs}}
and \href{http://ctan.org/pkg/subfiles}{\textsf{subfiles}}
provide structures in which the main and child documents can be
encapsulated and allowing them to be compiled individually.
The inclusion mechanism is different from the conventional |\include|.
\item
The package \href{http://ctan.org/pkg/combine}{\textsf{combine}}
is an elaborate solution to combine several documents into one.
\end{itemize}
%
See also the CTAN topic \href{http://ctan.org/topic/subdocs}{\textsf{subdocs}}
for further related packages.
The present package differs from the above solutions in that
a document structure constructed with the conventional |\include| mechanism
just needs two extra commands at the top of every file
such that all constituent files can be compiled individually.

%%%%%%%%%%%%%%%%%%%%%%%%%%%%%%%%%%%%%%%%%%%%%%%%%%%%%%%%%%%%%%%%%%%%%%%%%%%%%%%%
%\subsection{Feature Suggestions}
%
%The following is a list of features which may be useful for future
%versions of this package:
%%
%\begin{itemize}
%\item
%\ldots
%\end{itemize}

%%%%%%%%%%%%%%%%%%%%%%%%%%%%%%%%%%%%%%%%%%%%%%%%%%%%%%%%%%%%%%%%%%%%%%%%%%%%%%%%
\subsection{Revision History}

%%%%%%%%%%%%%%%%%%%%%%%%%%%%%%%%%%%%%%%%
\paragraph{v2.0:} 2018/12/30

\begin{itemize}
\item
immediate forward processing
\item
added |\childdocby| mechanism
\item
manual restructured
\end{itemize}

%%%%%%%%%%%%%%%%%%%%%%%%%%%%%%%%%%%%%%%%
\paragraph{v1.6:} 2018/01/17

\begin{itemize}
\item
application for development of include files
\item
corrections to manual
\end{itemize}

%%%%%%%%%%%%%%%%%%%%%%%%%%%%%%%%%%%%%%%%
\paragraph{v1.5:} 2017/05/21

\begin{itemize}
\item
more complete structuring introduced
\item
|\childdocof| introduced
\item
|\childdoc| renamed to |\childdocmain|
\item
|\childredirect| renamed to |\childdocforward| and |\childdocforwardprefix|
and functionality expanded
\end{itemize}

%%%%%%%%%%%%%%%%%%%%%%%%%%%%%%%%%%%%%%%%
\paragraph{v1.0:} 2017/04/27

\begin{itemize}
\item
manual and install package
\item
first version published on CTAN
\end{itemize}

%%%%%%%%%%%%%%%%%%%%%%%%%%%%%%%%%%%%%%%%
\paragraph{v0.6:} 2017/04/26

\begin{itemize}
\item
redirection mechanism added
\end{itemize}

%%%%%%%%%%%%%%%%%%%%%%%%%%%%%%%%%%%%%%%%
\paragraph{v0.5:} 2017/04/26

\begin{itemize}
\item
functionality in definition file
\end{itemize}


%%%%%%%%%%%%%%%%%%%%%%%%%%%%%%%%%%%%%%%%%%%%%%%%%%%%%%%%%%%%%%%%%%%%%%%%%%%%%%%%
%%%%%%%%%%%%%%%%%%%%%%%%%%%%%%%%%%%%%%%%%%%%%%%%%%%%%%%%%%%%%%%%%%%%%%%%%%%%%%%%
%%%%%%%%%%%%%%%%%%%%%%%%%%%%%%%%%%%%%%%%%%%%%%%%%%%%%%%%%%%%%%%%%%%%%%%%%%%%%%%%
\appendix

\settowidth\MacroIndent{\rmfamily\scriptsize 000\ }

 \DocInput{childdoc.dtx}

\end{document}
%</driver>
% \fi
%
% %%%%%%%%%%%%%%%%%%%%%%%%%%%%%%%%%%%%%%%%%%%%%%%%%%%%%%%%%%%%%%%%%%%%%%%%%%%%%%
% %%%%%%%%%%%%%%%%%%%%%%%%%%%%%%%%%%%%%%%%%%%%%%%%%%%%%%%%%%%%%%%%%%%%%%%%%%%%%%
% \section{Sample}
%\iffalse
%<*samplemain>
%\fi
%
% The following presents a sample document
% with two chapters, two parts, a title page,
% a compile flag as well as three forwarding files to set the flag.
% It consists of eight |.tex| files:
% \begin{center}
% \begin{tabular}{ll}
% |cdocsamp.tex|&main file\\
% |cdocsch1.tex|&include file for chapter 1\\
% |cdocsch2.tex|&include file for chapter 2\\
% |cdocspt3.tex|&include file for part 3\\
% |cdocspt4.tex|&include file for part 4\\
% |cdocsdrf.tex|&forwarding file for main file in draft mode\\
% |cdocsfi1.tex|&forwarding file for final version of chapter 1\\
% |cdocsfi2.tex|&forwarding file for final version of chapter 2\\
% \end{tabular}
% \end{center}
% Each of the eight files can be compiled directly by the \LaTeX{} compiler.
%
% %%%%%%%%%%%%%%%%%%%%%%%%%%%%%%%%%%%%%%
% \paragraph{Main File.}
%
% The main file is called |cdocsamp.tex|.
%
% Load the \textsf{childdoc} definitions and
% declare the filename for the main document:
%    \begin{macrocode}
\input{childdoc.def}
\childdocmain{}
%    \end{macrocode}

% Optional override for |\version| flag:
%    \begin{macrocode}
%%\ifchilddoc\else\providecommand{\version}{draft}\fi
%    \end{macrocode}

% Define the default values for the |\version| flag
% (|final| for the main file and |draft| for childs):
%    \begin{macrocode}
\ifchilddoc
\providecommand{\version}{draft}
\else
\providecommand{\version}{final}
\fi
%    \end{macrocode}

% Load the standard document class:
%    \begin{macrocode}
\documentclass[12pt]{article}
%    \end{macrocode}

% Start the document body:
%    \begin{macrocode}
\begin{document}
%    \end{macrocode}

% Declare a title page.
% Print title, part of document being processed and version flag:
%    \begin{macrocode}
\addtocounter{page}{-1}
\begin{center}
{\LARGE\bfseries{}childdoc example\par}
\vspace{1cm}
\ifchilddoc
\ifchilddocmanual part\else chapter\fi:
`\childdocname' of `\childdocjob'\par
\else
main document: `\childdocjob'\par
\fi
version: \version\par
\end{center}
\newpage
%    \end{macrocode}

% Manually include selected file,
% otherwise process as usual:
%    \begin{macrocode}
\ifchilddocmanual
\section*{part `\childdocname'}
\input{\childdocname}
\else
%    \end{macrocode}

% Include the two chapters:
%    \begin{macrocode}
\include{cdocsch1}
\include{cdocsch2}
%    \end{macrocode}

% Include the two parts unless only chapters should be displayed:
%    \begin{macrocode}
\ifchilddoc\else
\section{part three}
\input{cdocspt3}
\section{part four}
\input{cdocspt4}
\fi
%    \end{macrocode}

% Process as usual until here:
%    \begin{macrocode}
\fi
%    \end{macrocode}

% End of document body:
%    \begin{macrocode}
\end{document}
%    \end{macrocode}
%\iffalse
%</samplemain>
%\fi
%
% %%%%%%%%%%%%%%%%%%%%%%%%%%%%%%%%%%%%%%
% \paragraph{Chapter Include Files.}
%
% The include files are called |cdocsch1.tex| and |cdocsch2.tex|.
%
%\iffalse
%<*samplechap1|samplechap2>
%\fi

% Optional override for |\version| flag:
%    \begin{macrocode}
%%\providecommand{\version}{final}
%    \end{macrocode}

% Include the main document:
%    \begin{macrocode}
\input{childdoc.def}
\childdocof{cdocsamp}
%    \end{macrocode}

%\iffalse
%</samplechap1|samplechap2>
%\fi
%
%\iffalse
%<*samplechap1>
%\fi
% Some text for chapter 1:
%    \begin{macrocode}
\section{one}
some text in chapter one
%    \end{macrocode}

%\iffalse
%</samplechap1>
%\fi
% Some text for chapter 2:
%\iffalse
%<*samplechap2>
%\fi
%    \begin{macrocode}
\section{two}
more text in chapter two
%    \end{macrocode}

%\iffalse
%</samplechap2>
%\fi
%
% %%%%%%%%%%%%%%%%%%%%%%%%%%%%%%%%%%%%%%
% \paragraph{Part Include Files.}
%
% The include files are called |cdocspt3.tex| and |cdocspt4.tex|.
%
%\iffalse
%<*samplepart3|samplepart4>
%\fi

% Optional override for |\version| flag:
%    \begin{macrocode}
%%\providecommand{\version}{final}
%    \end{macrocode}

% Include the main document:
%    \begin{macrocode}
\input{childdoc.def}
\childdocby{cdocsamp}
%    \end{macrocode}

%\iffalse
%</samplepart3|samplepart4>
%\fi
%
%\iffalse
%<*samplepart3>
%\fi
% Some text for part 3:
%    \begin{macrocode}
some text in part three
%    \end{macrocode}

%\iffalse
%</samplepart3>
%\fi
% Some text for part 4:
%\iffalse
%<*samplepart4>
%\fi
%    \begin{macrocode}
more text in part four
%    \end{macrocode}

%\iffalse
%</samplepart4>
%\fi
%
% %%%%%%%%%%%%%%%%%%%%%%%%%%%%%%%%%%%%%%
% \paragraph{Forwarding for a Complete Draft.}
%
% The following forwarding file |cdocsdrf.tex|
% compiles the main document in draft mode:
%\iffalse
%<*sampledraft>
%\fi
%    \begin{macrocode}
\def\version{draft}
\input{childdoc.def}
\childdocforward{cdocsamp}
%    \end{macrocode}

%\iffalse
%</sampledraft>
%\fi
%
% %%%%%%%%%%%%%%%%%%%%%%%%%%%%%%%%%%%%%%
% \paragraph{Forwarding for Final Version of the Chapters.}
%
% The following forwarding files |cdocsfn1.tex| and |cdocsfn2.tex|
% (with identical content)
% compile the final versions of the child documents
% |cdocsch1.tex| and |cdocsch2.tex|, respectively:
%\iffalse
%<*samplefinal>
%\fi
%    \begin{macrocode}
\def\version{final}
\input{childdoc.def}
\childdocforwardprefix[cdocsamp]{cdocsfn}{cdocsch}
%    \end{macrocode}

%\iffalse
%</samplefinal>
%\fi
%
% %%%%%%%%%%%%%%%%%%%%%%%%%%%%%%%%%%%%%%
% \paragraph{Command Line Processing.}
%
% The following three command lines generate the output files
% |cdocscld|, |cdocscl1| and |cdocscl2|
% which should be identical to
% |cdocsdrf|, |cdocsch1| and |cdocsfn2|, respectively:
% \begin{center}
% \begin{tabular}{l}
% |latex -jobname cdocscld \|\\
% |  "\def\version{draft}\input{childdoc.def}\childdocforward{cdocsamp}"|\\
% |latex -jobname cdocscl1 \|\\
% |  "\input{childdoc.def}\childdocforward[cdocsamp]{cdocsch1}"|\\
% |latex -jobname cdocscl2 \|\\
% |  "\def\version{final}\input{childdoc.def}\childdocforward{cdocsch2}"|
% \end{tabular}
% \end{center}
% Note that the trailing backslash on each first line
% merely continues the input to the second line
% (for convenient cut ant paste).
% Furthermore, the command |latex| can be replaced by any
% of its alternative versions such as |pdflatex|.
%
% %%%%%%%%%%%%%%%%%%%%%%%%%%%%%%%%%%%%%%%%%%%%%%%%%%%%%%%%%%%%%%%%%%%%%%%%%%%%%%
% %%%%%%%%%%%%%%%%%%%%%%%%%%%%%%%%%%%%%%%%%%%%%%%%%%%%%%%%%%%%%%%%%%%%%%%%%%%%%%
% \section{Implementation}
%\iffalse
%<*package>
%\fi
%
% This section describes the definitions file |childdoc.def|.

% The definitions cannot be loaded using |\usepackage| or |\RequirePackage|
% which has a mechanism to prevent loading a style file more than once.
% When loading the definitions by means of |\input|
% multiple instances have to be prevented manually:
%\iffalse
%This code needs to be before the `\ProvidesFile' directive
%which is defined at the beginning of this file.
%Therefore it is also placed there and commented out here.
%</package>
%<*discard>
%\fi
%    \begin{macrocode}
\ifdefined\childdocmain\endinput\fi
%    \end{macrocode}
%\iffalse
%</discard>
%<*package>
%\fi
%
% \macro{\ifchilddoc}
% \macro{\ifchilddocmanual}
% The conditional |\ifchilddoc| tells whether a
% child (true) or main (false) document is being compiled.
% The conditional |\ifchilddocmanual| tells whether
% the |\includeonly| mechanism is used (false) or
% the selection of child files must be performed manually (true).
% The definitions initialise to false:
%    \begin{macrocode}
\newif\ifchilddoc
\newif\ifchilddocmanual
%    \end{macrocode}

% \macro{\childdocname}
% \macro{\childdocjob}
% The macro |\childdocname| stores the name of the main document
% to be compiled. The macro |\childdocjob| stores the name of
% the document on which the \LaTeX{} compiler was originally invoked.
% The content of |\jobname| cannot be compared
% to filenames specified in the source due to different catcodes.
% The following code rescans |\jobname|, stores the result
% in |\childdocname| and saves a copy in |\childdocjob|:
%    \begin{macrocode}
\edef\childdocname{\scantokens\expandafter{\jobname\noexpand}}
\let\childdocjob\childdocname
%    \end{macrocode}

% \macro{\childdocdisable}
% The macro |\childdocdisable| prevents the main file
% from being processed more than once.
% At this stage, the main document command |\childdocmain|
% is assumed to be called once again where it should do nothing.
% Any subsequent call to it should prevent
% a secondary processing of the main document
% It overwrites the forwarding commands
% |\childdocof| and |\childdocforward|
% with empty macros to prevent further inclusions of the main document:
%    \begin{macrocode}
\newcommand{\childdocdisable}
{
  \renewcommand{\childdocmain}[1]{\renewcommand{\childdocmain}[1]{\endinput}}
  \renewcommand{\childdocof}[1]{}
  \renewcommand{\childdocby}[2][]{}
  \renewcommand{\childdocforward}[2][]{}
  \renewcommand{\childdocdisable}{}
}
%    \end{macrocode}

% \macro{\childdocmain}
% The macro |\childdocmain| is to be called at the top of the main file
% with nothing or the main filename (without extension) as argument.
% First, it breaks loops.
% If the argument is not empty and does not match |\childdocname|
% (which is set by the first inclusion of |childdoc.def|),
% |\ifchilddoc| is set to true, |\includeonly| is applied to the child file
% and |\jobname| is set to the main file
% (for proper handling of |.aux| files):
%    \begin{macrocode}
\newcommand{\childdocmain}[1]
{
  \childdocdisable\childdocmain{}
  \if?#1?\else
    \begingroup
      \def\childdoctmp{#1}
      \ifx\childdoctmp\childdocname
        \def\childdoctmp{}
      \else
        \def\childdoctmp
        {
          \childdoctrue
          \includeonly{\childdocname}
          \def\childdocjob{#1}
          \def\jobname{#1}
        }
      \fi
      \expandafter
    \endgroup
    \childdoctmp
  \fi
}
%    \end{macrocode}

% \macro{\childdocof}
% The command |\childdocof| redirects
% compilation to the main file |#1|.
%    \begin{macrocode}
\newcommand{\childdocof}[1]
{
  \childdocdisable
  \childdoctrue
  \includeonly{\childdocname}
  \def\jobname{#1}
  \def\childdocjob{#1}
  \input{#1}
}
%    \end{macrocode}

% \macro{\childdocby}
% The command |\childdocby| ....
%    \begin{macrocode}
\newcommand{\childdocby}[2][]
{
  \childdocdisable
  \childdoctrue
  \childdocmanualtrue
  \if?#1?\else
    \def\jobname{#2}
  \fi
  \def\childdocjob{#2}
  \input{#2}
  \endinput
}
%    \end{macrocode}

% \macro{\childdocforward}
% The command |\childdocforward| redirects
% compilation to the main file or
% (if the optional argument is given) a child file.
% Parameters are set as if the main file
% or a child file starting with |\childdocof| was compiled.
% Then compilation is handed over to the main file:
%    \begin{macrocode}
\newcommand{\childdocforward}[2][]
{
  \begingroup
    \if?#1?
      \def\childdoctmp
      {
        \def\childdocname{#2}
        \def\childdocjob{#2}
        \def\jobname{#2}
        \input{#2}
        \endinput
      }
    \else
      \def\childdoctmp
      {
        \childdocdisable
        \def\childdocname{#2}
        \childdoctrue
        \includeonly{#2}
        \def\childdocjob{#1}
        \def\jobname{#1}
        \input{#1}
        \endinput
      }
    \fi
    \expandafter
  \endgroup
  \childdoctmp
}
%    \end{macrocode}

% \macro{\childdocforwardprefix}
% The command |\childdocforwardprefix| redirects
% compilation to the main or a child file by means of a pattern.
% The prefix |#1| in the current filename is replaced by |#2|
% and the suffix of the current filename is kept
% (it is assumed that the filename does not contain the substring `|~~~|'
% which is used as a delimiter).
% Compilation is handed over to the new file by |\childdocforward|:
%    \begin{macrocode}
\newcommand{\childdocforwardprefix}[3][]
{
  \begingroup
    \def\childdocextract #2##1~~~{\def\childdoctmp{\childdocforward[#1]{#3##1}}}
    \expandafter\childdocextract\childdocname~~~
    \expandafter
  \endgroup
  \childdoctmp
}
%    \end{macrocode}

% \macro{\childdoc}
% The deprecated macro |\childdoc| is a legacy version of |\childdocmain|:
%    \begin{macrocode}
\newcommand{\childdoc}{\childdocmain}
%    \end{macrocode}

% \macro{\childdocredirect}
% The deprecated macro |\childdocredirect| is a legacy version
% of |\childdocforward| and |\childdocforwardprefix|:
%    \begin{macrocode}
\newcommand{\childdocredirect}[2][]
{
  \begingroup
    \if?#1?
      \def\childdoctmp{\childdocforward{#2}}
    \else
      \def\childdoctmp{\childdocforwardprefix{#1}{#2}}
    \fi
    \expandafter
  \endgroup
  \childdoctmp
}
%    \end{macrocode}

%\iffalse
%</package>
%\fi
%
\endinput
|\\
|\childdocmain{}|\\
\end{tabular}
\end{center}
at the very top of the main \LaTeX{} file,
in particular \emph{before} the |\documentclass| statement!
The argument of |\childdocmain| should be left empty
(but it must be present).

%%%%%%%%%%%%%%%%%%%%%%%%%%%%%%%%%%%%%%%%
\DescribeMacro{\childdocof}
Furthermore, add the commands
\begin{center}
\begin{tabular}{l}
|% \iffalse
%
% childdoc.dtx Copyright (C) 2017-2018 Niklas Beisert
%
% This work may be distributed and/or modified under the
% conditions of the LaTeX Project Public License, either version 1.3
% of this license or (at your option) any later version.
% The latest version of this license is in
%   http://www.latex-project.org/lppl.txt
% and version 1.3 or later is part of all distributions of LaTeX
% version 2005/12/01 or later.
%
% This work has the LPPL maintenance status `maintained'.
%
% The Current Maintainer of this work is Niklas Beisert.
%
% This work consists of the files childdoc.dtx and childdoc.ins
% and the derived files childdoc.def and cdocsamp.tex with
% cdocsch1.tex, cdocsch2.tex, cdocsdrf.tex, cdocsfn1.tex, cdocsfn2.tex.
%
%<package>\ifdefined\childdocmain\endinput\fi
%<package>\ProvidesFile{childdoc.def}[2018/12/30 v2.0 child document driver]
%<samplemain>\ProvidesFile{cdocsamp.tex}[2018/12/30 v2.0 sample for childdoc]
%<*driver>
%\ProvidesFile{childdoc.drv}[2018/12/30 v2.0 childdoc reference manual file]
\PassOptionsToClass{10pt,a4paper}{article}
\documentclass{ltxdoc}

\usepackage[margin=35mm]{geometry}
\usepackage{hyperref}
\usepackage{hyperxmp}
\usepackage[usenames]{color}

\hypersetup{colorlinks=true}
\hypersetup{pdfstartview=FitH}
\hypersetup{pdfpagemode=UseNone}
\hypersetup{pdfsource={}}
\hypersetup{pdflang={en-UK}}
\hypersetup{pdfcopyright={Copyright 2017-2018 Niklas Beisert.
  This work may be distributed and/or modified under the
  conditions of the LaTeX Project Public License, either version 1.3
  of this license or (at your option) any later version.}}
\hypersetup{pdflicenseurl={http://www.latex-project.org/lppl.txt}}
\hypersetup{pdfcontactaddress={ETH Zurich, ITP, HIT K,
  Wolfgang-Pauli-Strasse 27}}
\hypersetup{pdfcontactpostcode={8093}}
\hypersetup{pdfcontactcity={Zurich}}
\hypersetup{pdfcontactcountry={Switzerland}}
\hypersetup{pdfcontactemail={nbeisert@itp.phys.ethz.ch}}
\hypersetup{pdfcontacturl={http://people.phys.ethz.ch/\xmptilde nbeisert/}}

\newcommand{\secref}[1]{\hyperref[#1]{section \ref*{#1}}}

\parskip1ex
\parindent0pt
\let\olditemize\itemize
\def\itemize{\olditemize\parskip0pt}

\begin{document}

\title{The \textsf{childdoc} Package}
\hypersetup{pdftitle={The childdoc Package}}
\author{Niklas Beisert\\[2ex]
  Institut f\"ur Theoretische Physik\\
  Eidgen\"ossische Technische Hochschule Z\"urich\\
  Wolfgang-Pauli-Strasse 27, 8093 Z\"urich, Switzerland\\[1ex]
  \href{mailto:nbeisert@itp.phys.ethz.ch}
  {\texttt{nbeisert@itp.phys.ethz.ch}}}
\hypersetup{pdfauthor={Niklas Beisert}}
\hypersetup{pdfsubject={Manual for the LaTeX2e Package childdoc}}
\date{30 December 2018, \textsf{v2.0}}
\maketitle

\begin{abstract}\noindent
\textsf{childdoc} is a \LaTeXe{} package
that enables the direct compilation
of document sections included by |\include|
to individual files.
\end{abstract}

\begingroup
\parskip0ex
\tableofcontents
\endgroup

%%%%%%%%%%%%%%%%%%%%%%%%%%%%%%%%%%%%%%%%%%%%%%%%%%%%%%%%%%%%%%%%%%%%%%%%%%%%%%%%
%%%%%%%%%%%%%%%%%%%%%%%%%%%%%%%%%%%%%%%%%%%%%%%%%%%%%%%%%%%%%%%%%%%%%%%%%%%%%%%%
\section{Introduction}

\LaTeX{} provides a mechanism to structure a large document (such as a book)
into a main file and several child files (containing the chapters)
using the |\include| command.
This mechanism is beneficial for documents
which span hundreds of pages in order to
make the source file(s) more manageable.
Moreover, compilation can be restricted to
selected child files by means of the |\includeonly| command.
The latter feature can be used to reduce the compilation time while editing
(this was significantly more useful in the earlier days of \LaTeX{})
or to generate a smaller document which is easier to navigate.
Another application of |\includeonly| is to generate
documents consisting of selected parts of the complete document.

However, there are a few drawbacks of the plain |\include| mechanism:
\begin{itemize}
\item
The child files cannot be compiled on their own,
they can only be compiled via the main file.
A naive editing environment
(such as a text editor with an option
to have the current file processed by \LaTeX)
may require one to switch to the main file before compiling;
attempting to compile the child file produces errors.
\item
The main file must be modified (each time)
to adjust the |\includeonly| command
to the present needs. This easily leaves the main file in a messy state.
\item
The generated document will always carry the filename
of the main document. This is inconvenient if
several child files are to be compiled and
to be kept for distribution.
\end{itemize}

The present package provides a simple interface
to make child files individually compilable by \LaTeX{}.
Compiling a child file then has the same effect as compiling
the main file with an |\includeonly| command
to select the appropriate child.
Moreover the generated document will carry the name of the child
rather than the main file.
This resolves all three above issues.

This feature is meant to make the editing of books,
thesis documents and lecture notes somewhat more convenient.
However, the package can also be used efficiently for
composing a series of documents (such as exercise sheets)
which are typically distributed individually.
It then assists the author in generating the individual documents
(potentially in different versions)
as well as a document containing the collected series.
Another application is in developing style files
or other kinds of included material
where compilation of the style file could redirect
to a sample or test file.

%%%%%%%%%%%%%%%%%%%%%%%%%%%%%%%%%%%%%%%%%%%%%%%%%%%%%%%%%%%%%%%%%%%%%%%%%%%%%%%%
%%%%%%%%%%%%%%%%%%%%%%%%%%%%%%%%%%%%%%%%%%%%%%%%%%%%%%%%%%%%%%%%%%%%%%%%%%%%%%%%
\section{Usage}

First of all, the package \textsf{childdoc} is \emph{not} a standard
\LaTeXe{} |.sty| style file! Therefore it needs to be invoked in
a non-standard way.

%%%%%%%%%%%%%%%%%%%%%%%%%%%%%%%%%%%%%%%%%%%%%%%%%%%%%%%%%%%%%%%%%%%%%%%%%%%%%%%%
\subsection{Included Files}
\label{sec:include}

%%%%%%%%%%%%%%%%%%%%%%%%%%%%%%%%%%%%%%%%
\DescribeMacro{\childdocmain}
To use the package, add the commands
\begin{center}
\begin{tabular}{l}
|\input{childdoc.def}|\\
|\childdocmain{}|\\
\end{tabular}
\end{center}
at the very top of the main \LaTeX{} file,
in particular \emph{before} the |\documentclass| statement!
The argument of |\childdocmain| should be left empty
(but it must be present).

%%%%%%%%%%%%%%%%%%%%%%%%%%%%%%%%%%%%%%%%
\DescribeMacro{\childdocof}
Furthermore, add the commands
\begin{center}
\begin{tabular}{l}
|\input{childdoc.def}|\\
|\childdocof{|\textit{main}|}|\\
\end{tabular}
\end{center}
at the top of every child file \textit{child}
which is included by |\include{|\textit{child}|}|
from within the main file
(or at least for those files to be compiled individually).
The argument \textit{main} must be the filename of the main file.

There are a couple of
considerations in setting up the main and child documents:

%%%%%%%%%%%%%%%%%%%%%%%%%%%%%%%%%%%%%%%%
\paragraph{Restrictions.}

Please note the following restrictions:
\begin{itemize}
\item
|\childdocmain| must be called with one argument \textit{main}
to ensure compatibility with earlier version of the package.
It must either be empty (|\childdocmain{}|)
or precisely match the filename of the main file in which it is specified.
See \secref{sec:detection} for further information.
\item
The filename \textit{main} must be specified without the |.tex| extension.
\item
The filename \textit{main} is case sensitive
(even in case-insensitive file systems)
due to internal string comparison.
\item
The argument \textit{main} should be fully expanded, it cannot be a macro.
\item
Subdirectories and special characters should be avoided in filenames.
\item
The command |\childdocmain{|\textit{main}|}| must be followed by a whitespace.
It should not be followed immediately by another command
or by a comment mark `|%|'.
This is because the \TeX{} parser reads the token immediately following
the argument of |\childdocmain| and puts it
at the beginning of every child section;
however, a white\-space is ignored.
\end{itemize}

%%%%%%%%%%%%%%%%%%%%%%%%%%%%%%%%%%%%%%%%
\paragraph{Content of Main File.}

It is advisable to place all content in the child files included by |\include|.
Any output contained in the main file will appear in all child documents
unless suppressed manually;
it cannot be suppressed automatically by the |\includeonly| directive
and thus should normally be avoided.
A method to include some content in the main file
by means of conditional processing is described in \secref{sec:conditional}.

%%%%%%%%%%%%%%%%%%%%%%%%%%%%%%%%%%%%%%%%
\paragraph{Page Numbering.}

When only a part of the document is compiled,
the appropriate numbering of pages
(as well as other status parameters)
is determined from the |.aux| files.
The latter contain information from previous passes.
However this information needs to propagate through
all intermediate child documents.
Therefore the page numbering in child documents may well
be inconsistent until the complete document is compiled at least once.

A useful (if unconventional) way to always ensure a consistent
page numbering is to restart the numbering in each child document
and denote the pages by `\textit{child}|.|\textit{page}'
where \textit{child} represents the chapter/section number of the child file.
This can be achieved by the command
|\numberwithin{page}{|\textit{child}|}|
of the \textsf{amsmath} package
where \textit{child} can be |chapter| or |section|
depending on the chosen structuring.
Alternatively, one can modify the macro |\thepage| appropriately
and reset the counter |page| at the start of each child file.

%%%%%%%%%%%%%%%%%%%%%%%%%%%%%%%%%%%%%%%%%%%%%%%%%%%%%%%%%%%%%%%%%%%%%%%%%%%%%%%%
\subsection{Conditional Processing}
\label{sec:conditional}

The package provides a mechanism to compile different versions
of a document. To customise the versions further some conditional processing
can come in handy to distinguish which version is being compiled.
The package provides two macros to describe the compilation context:

%%%%%%%%%%%%%%%%%%%%%%%%%%%%%%%%%%%%%%%%
\DescribeMacro{\ifchilddoc}
The conditional |\ifchilddoc| distinguishes between the compilation of
child documents and the main document:
%
\begin{center}
|\ifchilddoc |\textit{child-code}| |[|\||else |\textit{main-code}]| \||fi|
\end{center}

%%%%%%%%%%%%%%%%%%%%%%%%%%%%%%%%%%%%%%%%
\DescribeMacro{\childdocname}
\DescribeMacro{\childdocjob}
The macro |\childdocname| contains the filename (without extension)
of the main or child file being processed.
Note that |\childdocjob| will always contain the name of the main file.

%%%%%%%%%%%%%%%%%%%%%%%%%%%%%%%%%%%%%%%%
\paragraph{Title Page.}

Conditional processing can be used to include a title or banner page
in the main document when proper precautions are taken.
Importantly, the code in the main file should ensure that the page counter
(as well as other status parameters which are stored in the |.aux| files)
takes the same value after the conditional processing.
Otherwise the page numbers may take divergent values
depending on which part is compiled.

For example, a title page could be declared by:
%
\begin{center}
\begin{tabular}{l}
|\ifchilddoc\||else|\\
|\addtocounter{page}{-1}|\\
\textit{code for title page}\\
|\newpage|\\
|\||fi|
\end{tabular}
\end{center}
%
A banner page for the child documents can be generated by:
%
\begin{center}
\begin{tabular}{l}
|\ifchilddoc|\\
|\addtocounter{page}{-1}|\\
\textit{code for banner page}\\
|\newpage|\\
|\||fi|
\end{tabular}
\end{center}
%
Here one could write a message such as:
\begin{center}
|This is the part \childdocname{} of \childdocjob{}.|
\end{center}

%%%%%%%%%%%%%%%%%%%%%%%%%%%%%%%%%%%%%%%%%%%%%%%%%%%%%%%%%%%%%%%%%%%%%%%%%%%%%%%%
\subsection{Flags}
\label{sec:flags}

The package makes it easy to generate different versions
of the main or child documents.
To this end compilation flags can be defined
and assigned different default values.
They will be particularly useful in conjunction
with the forwarding mechanism described in \secref{sec:forward}.

For example, it may be useful to have a flag |\version|
which can be set to |draft| or |final|.
The document source will contain some conditional code
depending on the value of |\version|.
Suppose further, the flag should default to |final| for the main file
and to |draft| for child files
which is a natural assignment for editing the document.
This is achieved by placing the following code
in the preamble of the main document
(below the |\childdocmain| directive):
%
\begin{center}
\begin{tabular}{l}
|\ifchilddoc|\\
|\providecommand{\version}{draft}|\\
|\||else|\\
|\providecommand{\version}{final}|\\
|\||fi|
\end{tabular}
\end{center}
%
The definition by |\providecommand| makes sure
that previous definitions are not overwritten.
Further statements |\providecommand{\version}{...}|
can thus be added before the above code to override it.

For the main file, one might add a line
(between |\childdocmain| and the above block)
%
\begin{center}
|%\ifchilddoc\||else\providecommand{\version}{draft}\||fi|
\end{center}
%
which can be uncommented to produce a draft version.
Likewise one can add a line to the very top of a child file
(above the |\childdocof{|\textit{main}|}| directive)
%
\begin{center}
|%\providecommand{\version}{final}|
\end{center}
%
which can be uncommented to produce the final version of this child document.

%%%%%%%%%%%%%%%%%%%%%%%%%%%%%%%%%%%%%%%%%%%%%%%%%%%%%%%%%%%%%%%%%%%%%%%%%%%%%%%%
\subsection{Forwarding}
\label{sec:forward}

Different versions of the main or child documents
using compilation flags as described in \secref{sec:flags}
can be (permanently) stored in different files
for convenient compilation, viewing and distribution.
To this end, the package defines a command
to pass on compilation to a different file:

%%%%%%%%%%%%%%%%%%%%%%%%%%%%%%%%%%%%%%%%
\DescribeMacro{\childdocforward}
The command |\childdocforward| redirects processing to
another source file:
%
\begin{center}
\begin{tabular}{l}
|\input{childdoc.def}|\\
|\childdocforward[|\textit{main}|]{|\textit{dest}|}|\\
\end{tabular}
\end{center}
%
The argument \textit{dest} is the destination file
(without extension).
It should be the main file or one of the child files.
Note that further \textsf{childdoc} directives
such as |\childdocof| and |\childdocforward|
in the indicated file will be processed in this form.
The optional argument \textit{main}
passes on directly to the main file \textit{main}
while pretending to compile the child \textit{dest}.
This form behaves as if \textit{dest}
issues |\childdocof{|\textit{main}|}| right away,
and no further \textsf{childdoc} directives will be processed.

%%%%%%%%%%%%%%%%%%%%%%%%%%%%%%%%%%%%%%%%
\DescribeMacro{\...prefix}
In the alternative form |\childdocforwardprefix|,
%
\begin{center}
\begin{tabular}{l}
|\input{childdoc.def}|\\
|\childdocforwardprefix[|\textit{main}|]{|\textit{prefix}|}{|\textit{dest}|}|
\end{tabular}
\end{center}
%
the destination file is determined by a pattern
depending on the current file:
To make this work, the current file must be called
`{\textit{prefix}\hspace{0.2em}\textit{suffix}}'
with \textit{prefix} matching precisely the argument.
Processing is then passed on to the file
`{\textit{dest}\hspace{0.2em}\textit{suffix}}'.
Surely, the same effect is achieved by
directly specifying the
argument `{\textit{dest}\hspace{0.2em}\textit{suffix}}'
in the first form.
However, that requires to set up a different file
for each child. With the alternative form of the command
all these files can have exactly the same content
which simplifies setting them up and maintaining them.

For example, the following file |draft.tex|
with a compilation flag |\version| as described in \secref{sec:flags}
compiles the main document as a draft:
%
\begin{center}
\begin{tabular}{l}
|\def\version{draft}|\\
|\input{childdoc.def}|\\
|\childdocforward{|\textit{main}|}|
\end{tabular}
\end{center}
%
Likewise, the following files |final|\textit{nn}|.tex|
compile the final version of the child document
|child|\textit{nn}|.tex|:
%
\begin{center}
\begin{tabular}{l}
|\def\version{final}|\\
|\input{childdoc.def}|\\
|\childdocforwardprefix{final}{child}|
\end{tabular}
\end{center}
%

Note that when several versions of a main file and/or of each child file
are to be generated, it may be convenient to set up a |Makefile| or
shell script to automatise the process.

%%%%%%%%%%%%%%%%%%%%%%%%%%%%%%%%%%%%%%%%%%%%%%%%%%%%%%%%%%%%%%%%%%%%%%%%%%%%%%%%
\subsection{Command Line Processing}
\label{sec:commandline}

The effect of redirection files can also be achieved by invoking
the \LaTeX{} compiler with a more elaborate command line.
Most conveniently this should be done as part
of a shell script or a |Makefile|.

When using \textsf{childdoc} in the main file, the following
command lines effectively perform a redirection
(note that depending on the shell being used,
backslashes may have to be doubled: `|\|' $\to$ `|\\|'):
%
\begin{center}
|... -jobname "|\textit{target}|" |\\|"|[\textit{flags}]%
|\input{childdoc.def}\childdocforward[|\textit{main}|]{|\textit{dest}|}"|
\end{center}
%
Here \textit{target} is the name of the output file,
\textit{main} is the name of the main file
and \textit{dest} is the name of the main or child file to be processed
(all filenames without extensions).
The optional argument \textit{main} can be omitted
if \textit{main} matches \textit{dest}.
Optionally, compilation \textit{flags} can be defined via |\def| commands.
This command line makes the \TeX{} engine believe
it is compiling the file \textit{target}
whose content is specified as the latter parameter.
The provided code then forwards the processing to
\textit{main} or \textit{dest} as described in \secref{sec:forward}.

%%%%%%%%%%%%%%%%%%%%%%%%%%%%%%%%%%%%%%%%%%%%%%%%%%%%%%%%%%%%%%%%%%%%%%%%%%%%%%%%
\subsection{Include by Input}
\label{sec:input}

Including child documents by |\include| has some restrictions by design.
Most notably, the content of a child document always occupies
its own set of pages; pages cannot be shared between child documents.
Usually, this behaviour makes perfect sense
because each child document contain an essential part of the document.
However, in some situations it may be desirable to compose
a document from a collection of parts
without having mandatory page breaks between then.
For this case, the package
provides a mechanism to include parts
by |\input| which can also be processed individually.
However, by construction this mechanism
requires manual handling of the content to be output.

%%%%%%%%%%%%%%%%%%%%%%%%%%%%%%%%%%%%%%%%
\DescribeMacro{\ifchilddocmanual}
The main file should be prepared as usual, see \secref{sec:include}.
However, the document body must make a distinction
between processing of an individual part and of the main document, e.g.:
%
\begin{center}
\begin{tabular}{l}
|\ifchilddocmanual|\\
|\input{\childdocname}|\\
|\||else|\\
\textit{document body with }|\input{|\textit{part}|}|\\
|\||fi|
\end{tabular}
\end{center}
%
The conditional |\ifchilddocmanual| is true whenever
a part to be included by |\input| is being compiled,
and the name of the part is stored in |\childdocname|.

%%%%%%%%%%%%%%%%%%%%%%%%%%%%%%%%%%%%%%%%
\DescribeMacro{\childdocby}
Each part to be included by |\input| should start with:
%
\begin{center}
\begin{tabular}{l}
|\input{childdoc.def}|\\
|\childdocby{|\textit{main}|}|\\
\end{tabular}
\end{center}
%
The directive |\childdocby| is similar to |\childdocof|
described in \secref{sec:include},
but the subsequent selection of content must be done manually.
To that end, both |\ifchilddoc| and |\ifchilddocmanual|
will be true upon processing of a part,
and the name of the part is stored in |\childdocname|.
Note that |\jobname| will be set to the filename of the current part
so that each part receives an individual |.aux| file
that does not interfere with the |.aux| file(s) of the main document.
This behaviour can be altered by the alternative form
|\childdocby[*]{|\textit{main}|}| (with a non-empty optional argument)
which uses the |.aux| file of the main document
by setting |\jobname| to \textit{main}.

%%%%%%%%%%%%%%%%%%%%%%%%%%%%%%%%%%%%%%%%%%%%%%%%%%%%%%%%%%%%%%%%%%%%%%%%%%%%%%%%
\subsection{Driver Development}
\label{sec:driver}

The \textsf{childdoc} mechanism can also be use for the development
of definition files such as \LaTeX{} styles or classes.
This case differs from the above setup with multiple parts
included by |\include| in that no |\includeonly| should be invoked.
This can be achieved by starting the include file
(before |\ProvidesPackage|) with:
%
\begin{center}
\begin{tabular}{l}
|\input{childdoc.def}|\\
|\childdocforward{|\textit{main}|}|\\
\end{tabular}
\end{center}
%
or alternatively with:
%
\begin{center}
\begin{tabular}{l}
|\input{childdoc.def}|\\
|\childdocby{|\textit{main}|}|\\
\end{tabular}
\end{center}
%
Both forms have slightly different effects as described above.
The main file is prepared as usual, see \secref{sec:include}.

%%%%%%%%%%%%%%%%%%%%%%%%%%%%%%%%%%%%%%%%%%%%%%%%%%%%%%%%%%%%%%%%%%%%%%%%%%%%%%%%
\subsection{Legacy Detection}
\label{sec:detection}

The directive |\childdocmain| in the main file can detect
whether the complete document or merely a child is to be compiled
even without using the directive |\childdocof|.
This method is deprecated because it is less robust
and there is no compelling reason to use it;
it is merely provided for backward compatibility
and it may be removed in future versions.

If the detection mechanism is to be used,
it is mandatory to correctly specify
the filename of the main file as the argument of |\childdocmain|:
%
\begin{center}
\begin{tabular}{l}
|\input{childdoc.def}|\\
|\childdocmain{|\textit{main}|}|\\
\end{tabular}
\end{center}
%
If |\jobname| does not match the argument \textit{main} of |\childdocmain|,
it is assumed that |\jobname| points to the child file to be compiled.
When using |\childdocmain| with the main file specified as argument,
it suffices to start a child file
with just |\input{|\textit{main}|}|
without loading of the package and using |\childdocof|.
If instead all processing is done
with the appropriate \textsf{childdoc} directives,
the argument of \textit{main} of |\childdocmain| can be empty.

An alternative version of the command line processing described
in \secref{sec:commandline} using the detection mechanism reads:
%
\begin{center}
|... -jobname "|\textit{target}|" "|[\textit{flags}]%
[|\def\jobname{|\textit{dest}|}|]|\input{|\textit{main}|}"|
\end{center}

%%%%%%%%%%%%%%%%%%%%%%%%%%%%%%%%%%%%%%%%%%%%%%%%%%%%%%%%%%%%%%%%%%%%%%%%%%%%%%%%
\subsection{Manual Code}
\label{sec:manual}

In case one cannot be certain whether the definitions file |childdoc.def|
is installed on the target \TeX{} distribution
and one prefers not to ship it,
it is conceivable to paste a few relevant commands into the sources.

To that end, drop all statements |\input{childdoc.def}|
and perform the replacements as outlined below.
Instead of |\childdocmain{|\textit{main}|}| add the following code
to the top of the main file:
%
\begin{center}
\begin{tabular}{l}
|\||ifdefined\childdocname\endinput\||fi\newif\ifchilddoc|\\
|\edef\childdocname{\scantokens\expandafter{\jobname\noexpand}}|\\
|\def\childdocmain{|\textit{main}|}\||ifx\childdocmain\childdocname\||else|\\
|\childdoctrue\includeonly{\childdocname}\let\jobname\childdocmain\||fi|\\
\end{tabular}
\end{center}
%
Instead of |\childdocof{|\textit{main}|}| just include the main file
at the top of each child file:
%
\begin{center}
|\input{|\textit{main}|}|
\end{center}
%
A simple redirection |\childdocforward{|\textit{dest}|}| is achieved by:
%
\begin{center}
|\def\jobname{|\textit{dest}|}\input{\jobname}|
\end{center}
%
The redirection with prefix
|\childdocforwardprefix[|\textit{prefix}|]{|\textit{dest}|}|
is accomplished by:
%
\begin{center}
\begin{tabular}{l}
|{\edef\jobname{\scantokens\expandafter{\jobname\noexpand}}|\\
|\def\redirectjob |\textit{prefix}|#1~~~{\gdef\jobname{|\textit{dest}|#1}}|\\
|\expandafter\redirectjob\jobname~~~}\input{\jobname}|
\end{tabular}
\end{center}

In an alternative approach,
child documents can be compiled by a specific command line
without additional code or specific definitions:
%
\begin{center}
|... -jobname "|\textit{target}|" "|[\textit{flags}]%
|\includeonly{|\textit{dest}|}\input{|\textit{main}|}"|
\end{center}
%

%%%%%%%%%%%%%%%%%%%%%%%%%%%%%%%%%%%%%%%%%%%%%%%%%%%%%%%%%%%%%%%%%%%%%%%%%%%%%%%%
%%%%%%%%%%%%%%%%%%%%%%%%%%%%%%%%%%%%%%%%%%%%%%%%%%%%%%%%%%%%%%%%%%%%%%%%%%%%%%%%
\section{Information}

%%%%%%%%%%%%%%%%%%%%%%%%%%%%%%%%%%%%%%%%%%%%%%%%%%%%%%%%%%%%%%%%%%%%%%%%%%%%%%%%
\subsection{Copyright}

Copyright \copyright{} 2017--2018 Niklas Beisert

This work may be distributed and/or modified under the
conditions of the \LaTeX{} Project Public License, either version 1.3
of this license or (at your option) any later version.
The latest version of this license is in
  \url{http://www.latex-project.org/lppl.txt}
and version 1.3 or later is part of all distributions of \LaTeX{}
version 2005/12/01 or later.

This work has the LPPL maintenance status `maintained'.

The Current Maintainer of this work is Niklas Beisert.

This work consists of the files |README.txt|, |childdoc.ins| and |childdoc.dtx|
as well as the derived files |childdoc.def|, |cdocsamp.tex|
with |cdocsch1.tex|, |cdocsch2.tex|, |cdocspt3.tex|, |cdocspt4.tex|,
|cdocsdrf.tex|, |cdocsfn1.tex|, |cdocsfn2.tex|
as well as |childdoc.pdf|.

%%%%%%%%%%%%%%%%%%%%%%%%%%%%%%%%%%%%%%%%%%%%%%%%%%%%%%%%%%%%%%%%%%%%%%%%%%%%%%%%
\subsection{Files and Installation}

The package consists of the files:
%
\begin{center}
\begin{tabular}{ll}
    |README.txt|   & readme file \\
    |childdoc.ins| & installation file \\
    |childdoc.dtx| & source file \\
    |childdoc.def| & definition file \\
    |cdocsamp.tex| & sample main file \\
    |cdocsch1.tex| & sample include file \\
    |cdocsch2.tex| & sample include file \\
    |cdocspt3.tex| & sample part file \\
    |cdocspt4.tex| & sample part file \\
    |cdocsdrf.tex| & sample redirection file \\
    |cdocsfn1.tex| & sample redirection file \\
    |cdocsfn2.tex| & sample redirection file \\
    |childdoc.pdf| & manual
\end{tabular}
\end{center}
%
The distribution consists of the files
|README.txt|, |childdoc.ins| and |childdoc.dtx|.
%
\begin{itemize}
\item
Run (pdf)\LaTeX{} on |childdoc.dtx|
to compile the manual |childdoc.pdf| (this file).
\item
Run \LaTeX{} on |childdoc.ins| to create the definitions file |childdoc.def|
and the sample |cdocsamp.tex| with include files
|cdocsch1.tex|, |cdocsch2.tex|, |cdocspt3.tex|, |cdocspt4.tex|,
|cdocsdrf.tex|, |cdocsfn1.tex|, |cdocsfn2.tex|.
Then copy the file |childdoc.def| to an appropriate directory of your \LaTeX{}
distribution, e.g.\ \textit{texmf-root}|/tex/latex/childdoc|.
\end{itemize}

%%%%%%%%%%%%%%%%%%%%%%%%%%%%%%%%%%%%%%%%%%%%%%%%%%%%%%%%%%%%%%%%%%%%%%%%%%%%%%%%
\subsection{Related CTAN Packages}

There are several other packages which offer a similar functionality:
%
\begin{itemize}
\item
The packages
\href{http://ctan.org/pkg/docmute}{\textsf{docmute}},
\href{http://ctan.org/pkg/includex}{\textsf{includex}} and
\href{http://ctan.org/pkg/standalone}{\textsf{standalone}}
provide commands to include only the document body of
a child file thus allowing both files to be compiled individually.
\item
The packages \href{http://ctan.org/pkg/subdocs}{\textsf{subdocs}}
and \href{http://ctan.org/pkg/subfiles}{\textsf{subfiles}}
provide structures in which the main and child documents can be
encapsulated and allowing them to be compiled individually.
The inclusion mechanism is different from the conventional |\include|.
\item
The package \href{http://ctan.org/pkg/combine}{\textsf{combine}}
is an elaborate solution to combine several documents into one.
\end{itemize}
%
See also the CTAN topic \href{http://ctan.org/topic/subdocs}{\textsf{subdocs}}
for further related packages.
The present package differs from the above solutions in that
a document structure constructed with the conventional |\include| mechanism
just needs two extra commands at the top of every file
such that all constituent files can be compiled individually.

%%%%%%%%%%%%%%%%%%%%%%%%%%%%%%%%%%%%%%%%%%%%%%%%%%%%%%%%%%%%%%%%%%%%%%%%%%%%%%%%
%\subsection{Feature Suggestions}
%
%The following is a list of features which may be useful for future
%versions of this package:
%%
%\begin{itemize}
%\item
%\ldots
%\end{itemize}

%%%%%%%%%%%%%%%%%%%%%%%%%%%%%%%%%%%%%%%%%%%%%%%%%%%%%%%%%%%%%%%%%%%%%%%%%%%%%%%%
\subsection{Revision History}

%%%%%%%%%%%%%%%%%%%%%%%%%%%%%%%%%%%%%%%%
\paragraph{v2.0:} 2018/12/30

\begin{itemize}
\item
immediate forward processing
\item
added |\childdocby| mechanism
\item
manual restructured
\end{itemize}

%%%%%%%%%%%%%%%%%%%%%%%%%%%%%%%%%%%%%%%%
\paragraph{v1.6:} 2018/01/17

\begin{itemize}
\item
application for development of include files
\item
corrections to manual
\end{itemize}

%%%%%%%%%%%%%%%%%%%%%%%%%%%%%%%%%%%%%%%%
\paragraph{v1.5:} 2017/05/21

\begin{itemize}
\item
more complete structuring introduced
\item
|\childdocof| introduced
\item
|\childdoc| renamed to |\childdocmain|
\item
|\childredirect| renamed to |\childdocforward| and |\childdocforwardprefix|
and functionality expanded
\end{itemize}

%%%%%%%%%%%%%%%%%%%%%%%%%%%%%%%%%%%%%%%%
\paragraph{v1.0:} 2017/04/27

\begin{itemize}
\item
manual and install package
\item
first version published on CTAN
\end{itemize}

%%%%%%%%%%%%%%%%%%%%%%%%%%%%%%%%%%%%%%%%
\paragraph{v0.6:} 2017/04/26

\begin{itemize}
\item
redirection mechanism added
\end{itemize}

%%%%%%%%%%%%%%%%%%%%%%%%%%%%%%%%%%%%%%%%
\paragraph{v0.5:} 2017/04/26

\begin{itemize}
\item
functionality in definition file
\end{itemize}


%%%%%%%%%%%%%%%%%%%%%%%%%%%%%%%%%%%%%%%%%%%%%%%%%%%%%%%%%%%%%%%%%%%%%%%%%%%%%%%%
%%%%%%%%%%%%%%%%%%%%%%%%%%%%%%%%%%%%%%%%%%%%%%%%%%%%%%%%%%%%%%%%%%%%%%%%%%%%%%%%
%%%%%%%%%%%%%%%%%%%%%%%%%%%%%%%%%%%%%%%%%%%%%%%%%%%%%%%%%%%%%%%%%%%%%%%%%%%%%%%%
\appendix

\settowidth\MacroIndent{\rmfamily\scriptsize 000\ }

 \DocInput{childdoc.dtx}

\end{document}
%</driver>
% \fi
%
% %%%%%%%%%%%%%%%%%%%%%%%%%%%%%%%%%%%%%%%%%%%%%%%%%%%%%%%%%%%%%%%%%%%%%%%%%%%%%%
% %%%%%%%%%%%%%%%%%%%%%%%%%%%%%%%%%%%%%%%%%%%%%%%%%%%%%%%%%%%%%%%%%%%%%%%%%%%%%%
% \section{Sample}
%\iffalse
%<*samplemain>
%\fi
%
% The following presents a sample document
% with two chapters, two parts, a title page,
% a compile flag as well as three forwarding files to set the flag.
% It consists of eight |.tex| files:
% \begin{center}
% \begin{tabular}{ll}
% |cdocsamp.tex|&main file\\
% |cdocsch1.tex|&include file for chapter 1\\
% |cdocsch2.tex|&include file for chapter 2\\
% |cdocspt3.tex|&include file for part 3\\
% |cdocspt4.tex|&include file for part 4\\
% |cdocsdrf.tex|&forwarding file for main file in draft mode\\
% |cdocsfi1.tex|&forwarding file for final version of chapter 1\\
% |cdocsfi2.tex|&forwarding file for final version of chapter 2\\
% \end{tabular}
% \end{center}
% Each of the eight files can be compiled directly by the \LaTeX{} compiler.
%
% %%%%%%%%%%%%%%%%%%%%%%%%%%%%%%%%%%%%%%
% \paragraph{Main File.}
%
% The main file is called |cdocsamp.tex|.
%
% Load the \textsf{childdoc} definitions and
% declare the filename for the main document:
%    \begin{macrocode}
\input{childdoc.def}
\childdocmain{}
%    \end{macrocode}

% Optional override for |\version| flag:
%    \begin{macrocode}
%%\ifchilddoc\else\providecommand{\version}{draft}\fi
%    \end{macrocode}

% Define the default values for the |\version| flag
% (|final| for the main file and |draft| for childs):
%    \begin{macrocode}
\ifchilddoc
\providecommand{\version}{draft}
\else
\providecommand{\version}{final}
\fi
%    \end{macrocode}

% Load the standard document class:
%    \begin{macrocode}
\documentclass[12pt]{article}
%    \end{macrocode}

% Start the document body:
%    \begin{macrocode}
\begin{document}
%    \end{macrocode}

% Declare a title page.
% Print title, part of document being processed and version flag:
%    \begin{macrocode}
\addtocounter{page}{-1}
\begin{center}
{\LARGE\bfseries{}childdoc example\par}
\vspace{1cm}
\ifchilddoc
\ifchilddocmanual part\else chapter\fi:
`\childdocname' of `\childdocjob'\par
\else
main document: `\childdocjob'\par
\fi
version: \version\par
\end{center}
\newpage
%    \end{macrocode}

% Manually include selected file,
% otherwise process as usual:
%    \begin{macrocode}
\ifchilddocmanual
\section*{part `\childdocname'}
\input{\childdocname}
\else
%    \end{macrocode}

% Include the two chapters:
%    \begin{macrocode}
\include{cdocsch1}
\include{cdocsch2}
%    \end{macrocode}

% Include the two parts unless only chapters should be displayed:
%    \begin{macrocode}
\ifchilddoc\else
\section{part three}
\input{cdocspt3}
\section{part four}
\input{cdocspt4}
\fi
%    \end{macrocode}

% Process as usual until here:
%    \begin{macrocode}
\fi
%    \end{macrocode}

% End of document body:
%    \begin{macrocode}
\end{document}
%    \end{macrocode}
%\iffalse
%</samplemain>
%\fi
%
% %%%%%%%%%%%%%%%%%%%%%%%%%%%%%%%%%%%%%%
% \paragraph{Chapter Include Files.}
%
% The include files are called |cdocsch1.tex| and |cdocsch2.tex|.
%
%\iffalse
%<*samplechap1|samplechap2>
%\fi

% Optional override for |\version| flag:
%    \begin{macrocode}
%%\providecommand{\version}{final}
%    \end{macrocode}

% Include the main document:
%    \begin{macrocode}
\input{childdoc.def}
\childdocof{cdocsamp}
%    \end{macrocode}

%\iffalse
%</samplechap1|samplechap2>
%\fi
%
%\iffalse
%<*samplechap1>
%\fi
% Some text for chapter 1:
%    \begin{macrocode}
\section{one}
some text in chapter one
%    \end{macrocode}

%\iffalse
%</samplechap1>
%\fi
% Some text for chapter 2:
%\iffalse
%<*samplechap2>
%\fi
%    \begin{macrocode}
\section{two}
more text in chapter two
%    \end{macrocode}

%\iffalse
%</samplechap2>
%\fi
%
% %%%%%%%%%%%%%%%%%%%%%%%%%%%%%%%%%%%%%%
% \paragraph{Part Include Files.}
%
% The include files are called |cdocspt3.tex| and |cdocspt4.tex|.
%
%\iffalse
%<*samplepart3|samplepart4>
%\fi

% Optional override for |\version| flag:
%    \begin{macrocode}
%%\providecommand{\version}{final}
%    \end{macrocode}

% Include the main document:
%    \begin{macrocode}
\input{childdoc.def}
\childdocby{cdocsamp}
%    \end{macrocode}

%\iffalse
%</samplepart3|samplepart4>
%\fi
%
%\iffalse
%<*samplepart3>
%\fi
% Some text for part 3:
%    \begin{macrocode}
some text in part three
%    \end{macrocode}

%\iffalse
%</samplepart3>
%\fi
% Some text for part 4:
%\iffalse
%<*samplepart4>
%\fi
%    \begin{macrocode}
more text in part four
%    \end{macrocode}

%\iffalse
%</samplepart4>
%\fi
%
% %%%%%%%%%%%%%%%%%%%%%%%%%%%%%%%%%%%%%%
% \paragraph{Forwarding for a Complete Draft.}
%
% The following forwarding file |cdocsdrf.tex|
% compiles the main document in draft mode:
%\iffalse
%<*sampledraft>
%\fi
%    \begin{macrocode}
\def\version{draft}
\input{childdoc.def}
\childdocforward{cdocsamp}
%    \end{macrocode}

%\iffalse
%</sampledraft>
%\fi
%
% %%%%%%%%%%%%%%%%%%%%%%%%%%%%%%%%%%%%%%
% \paragraph{Forwarding for Final Version of the Chapters.}
%
% The following forwarding files |cdocsfn1.tex| and |cdocsfn2.tex|
% (with identical content)
% compile the final versions of the child documents
% |cdocsch1.tex| and |cdocsch2.tex|, respectively:
%\iffalse
%<*samplefinal>
%\fi
%    \begin{macrocode}
\def\version{final}
\input{childdoc.def}
\childdocforwardprefix[cdocsamp]{cdocsfn}{cdocsch}
%    \end{macrocode}

%\iffalse
%</samplefinal>
%\fi
%
% %%%%%%%%%%%%%%%%%%%%%%%%%%%%%%%%%%%%%%
% \paragraph{Command Line Processing.}
%
% The following three command lines generate the output files
% |cdocscld|, |cdocscl1| and |cdocscl2|
% which should be identical to
% |cdocsdrf|, |cdocsch1| and |cdocsfn2|, respectively:
% \begin{center}
% \begin{tabular}{l}
% |latex -jobname cdocscld \|\\
% |  "\def\version{draft}\input{childdoc.def}\childdocforward{cdocsamp}"|\\
% |latex -jobname cdocscl1 \|\\
% |  "\input{childdoc.def}\childdocforward[cdocsamp]{cdocsch1}"|\\
% |latex -jobname cdocscl2 \|\\
% |  "\def\version{final}\input{childdoc.def}\childdocforward{cdocsch2}"|
% \end{tabular}
% \end{center}
% Note that the trailing backslash on each first line
% merely continues the input to the second line
% (for convenient cut ant paste).
% Furthermore, the command |latex| can be replaced by any
% of its alternative versions such as |pdflatex|.
%
% %%%%%%%%%%%%%%%%%%%%%%%%%%%%%%%%%%%%%%%%%%%%%%%%%%%%%%%%%%%%%%%%%%%%%%%%%%%%%%
% %%%%%%%%%%%%%%%%%%%%%%%%%%%%%%%%%%%%%%%%%%%%%%%%%%%%%%%%%%%%%%%%%%%%%%%%%%%%%%
% \section{Implementation}
%\iffalse
%<*package>
%\fi
%
% This section describes the definitions file |childdoc.def|.

% The definitions cannot be loaded using |\usepackage| or |\RequirePackage|
% which has a mechanism to prevent loading a style file more than once.
% When loading the definitions by means of |\input|
% multiple instances have to be prevented manually:
%\iffalse
%This code needs to be before the `\ProvidesFile' directive
%which is defined at the beginning of this file.
%Therefore it is also placed there and commented out here.
%</package>
%<*discard>
%\fi
%    \begin{macrocode}
\ifdefined\childdocmain\endinput\fi
%    \end{macrocode}
%\iffalse
%</discard>
%<*package>
%\fi
%
% \macro{\ifchilddoc}
% \macro{\ifchilddocmanual}
% The conditional |\ifchilddoc| tells whether a
% child (true) or main (false) document is being compiled.
% The conditional |\ifchilddocmanual| tells whether
% the |\includeonly| mechanism is used (false) or
% the selection of child files must be performed manually (true).
% The definitions initialise to false:
%    \begin{macrocode}
\newif\ifchilddoc
\newif\ifchilddocmanual
%    \end{macrocode}

% \macro{\childdocname}
% \macro{\childdocjob}
% The macro |\childdocname| stores the name of the main document
% to be compiled. The macro |\childdocjob| stores the name of
% the document on which the \LaTeX{} compiler was originally invoked.
% The content of |\jobname| cannot be compared
% to filenames specified in the source due to different catcodes.
% The following code rescans |\jobname|, stores the result
% in |\childdocname| and saves a copy in |\childdocjob|:
%    \begin{macrocode}
\edef\childdocname{\scantokens\expandafter{\jobname\noexpand}}
\let\childdocjob\childdocname
%    \end{macrocode}

% \macro{\childdocdisable}
% The macro |\childdocdisable| prevents the main file
% from being processed more than once.
% At this stage, the main document command |\childdocmain|
% is assumed to be called once again where it should do nothing.
% Any subsequent call to it should prevent
% a secondary processing of the main document
% It overwrites the forwarding commands
% |\childdocof| and |\childdocforward|
% with empty macros to prevent further inclusions of the main document:
%    \begin{macrocode}
\newcommand{\childdocdisable}
{
  \renewcommand{\childdocmain}[1]{\renewcommand{\childdocmain}[1]{\endinput}}
  \renewcommand{\childdocof}[1]{}
  \renewcommand{\childdocby}[2][]{}
  \renewcommand{\childdocforward}[2][]{}
  \renewcommand{\childdocdisable}{}
}
%    \end{macrocode}

% \macro{\childdocmain}
% The macro |\childdocmain| is to be called at the top of the main file
% with nothing or the main filename (without extension) as argument.
% First, it breaks loops.
% If the argument is not empty and does not match |\childdocname|
% (which is set by the first inclusion of |childdoc.def|),
% |\ifchilddoc| is set to true, |\includeonly| is applied to the child file
% and |\jobname| is set to the main file
% (for proper handling of |.aux| files):
%    \begin{macrocode}
\newcommand{\childdocmain}[1]
{
  \childdocdisable\childdocmain{}
  \if?#1?\else
    \begingroup
      \def\childdoctmp{#1}
      \ifx\childdoctmp\childdocname
        \def\childdoctmp{}
      \else
        \def\childdoctmp
        {
          \childdoctrue
          \includeonly{\childdocname}
          \def\childdocjob{#1}
          \def\jobname{#1}
        }
      \fi
      \expandafter
    \endgroup
    \childdoctmp
  \fi
}
%    \end{macrocode}

% \macro{\childdocof}
% The command |\childdocof| redirects
% compilation to the main file |#1|.
%    \begin{macrocode}
\newcommand{\childdocof}[1]
{
  \childdocdisable
  \childdoctrue
  \includeonly{\childdocname}
  \def\jobname{#1}
  \def\childdocjob{#1}
  \input{#1}
}
%    \end{macrocode}

% \macro{\childdocby}
% The command |\childdocby| ....
%    \begin{macrocode}
\newcommand{\childdocby}[2][]
{
  \childdocdisable
  \childdoctrue
  \childdocmanualtrue
  \if?#1?\else
    \def\jobname{#2}
  \fi
  \def\childdocjob{#2}
  \input{#2}
  \endinput
}
%    \end{macrocode}

% \macro{\childdocforward}
% The command |\childdocforward| redirects
% compilation to the main file or
% (if the optional argument is given) a child file.
% Parameters are set as if the main file
% or a child file starting with |\childdocof| was compiled.
% Then compilation is handed over to the main file:
%    \begin{macrocode}
\newcommand{\childdocforward}[2][]
{
  \begingroup
    \if?#1?
      \def\childdoctmp
      {
        \def\childdocname{#2}
        \def\childdocjob{#2}
        \def\jobname{#2}
        \input{#2}
        \endinput
      }
    \else
      \def\childdoctmp
      {
        \childdocdisable
        \def\childdocname{#2}
        \childdoctrue
        \includeonly{#2}
        \def\childdocjob{#1}
        \def\jobname{#1}
        \input{#1}
        \endinput
      }
    \fi
    \expandafter
  \endgroup
  \childdoctmp
}
%    \end{macrocode}

% \macro{\childdocforwardprefix}
% The command |\childdocforwardprefix| redirects
% compilation to the main or a child file by means of a pattern.
% The prefix |#1| in the current filename is replaced by |#2|
% and the suffix of the current filename is kept
% (it is assumed that the filename does not contain the substring `|~~~|'
% which is used as a delimiter).
% Compilation is handed over to the new file by |\childdocforward|:
%    \begin{macrocode}
\newcommand{\childdocforwardprefix}[3][]
{
  \begingroup
    \def\childdocextract #2##1~~~{\def\childdoctmp{\childdocforward[#1]{#3##1}}}
    \expandafter\childdocextract\childdocname~~~
    \expandafter
  \endgroup
  \childdoctmp
}
%    \end{macrocode}

% \macro{\childdoc}
% The deprecated macro |\childdoc| is a legacy version of |\childdocmain|:
%    \begin{macrocode}
\newcommand{\childdoc}{\childdocmain}
%    \end{macrocode}

% \macro{\childdocredirect}
% The deprecated macro |\childdocredirect| is a legacy version
% of |\childdocforward| and |\childdocforwardprefix|:
%    \begin{macrocode}
\newcommand{\childdocredirect}[2][]
{
  \begingroup
    \if?#1?
      \def\childdoctmp{\childdocforward{#2}}
    \else
      \def\childdoctmp{\childdocforwardprefix{#1}{#2}}
    \fi
    \expandafter
  \endgroup
  \childdoctmp
}
%    \end{macrocode}

%\iffalse
%</package>
%\fi
%
\endinput
|\\
|\childdocof{|\textit{main}|}|\\
\end{tabular}
\end{center}
at the top of every child file \textit{child}
which is included by |\include{|\textit{child}|}|
from within the main file
(or at least for those files to be compiled individually).
The argument \textit{main} must be the filename of the main file.

There are a couple of
considerations in setting up the main and child documents:

%%%%%%%%%%%%%%%%%%%%%%%%%%%%%%%%%%%%%%%%
\paragraph{Restrictions.}

Please note the following restrictions:
\begin{itemize}
\item
|\childdocmain| must be called with one argument \textit{main}
to ensure compatibility with earlier version of the package.
It must either be empty (|\childdocmain{}|)
or precisely match the filename of the main file in which it is specified.
See \secref{sec:detection} for further information.
\item
The filename \textit{main} must be specified without the |.tex| extension.
\item
The filename \textit{main} is case sensitive
(even in case-insensitive file systems)
due to internal string comparison.
\item
The argument \textit{main} should be fully expanded, it cannot be a macro.
\item
Subdirectories and special characters should be avoided in filenames.
\item
The command |\childdocmain{|\textit{main}|}| must be followed by a whitespace.
It should not be followed immediately by another command
or by a comment mark `|%|'.
This is because the \TeX{} parser reads the token immediately following
the argument of |\childdocmain| and puts it
at the beginning of every child section;
however, a white\-space is ignored.
\end{itemize}

%%%%%%%%%%%%%%%%%%%%%%%%%%%%%%%%%%%%%%%%
\paragraph{Content of Main File.}

It is advisable to place all content in the child files included by |\include|.
Any output contained in the main file will appear in all child documents
unless suppressed manually;
it cannot be suppressed automatically by the |\includeonly| directive
and thus should normally be avoided.
A method to include some content in the main file
by means of conditional processing is described in \secref{sec:conditional}.

%%%%%%%%%%%%%%%%%%%%%%%%%%%%%%%%%%%%%%%%
\paragraph{Page Numbering.}

When only a part of the document is compiled,
the appropriate numbering of pages
(as well as other status parameters)
is determined from the |.aux| files.
The latter contain information from previous passes.
However this information needs to propagate through
all intermediate child documents.
Therefore the page numbering in child documents may well
be inconsistent until the complete document is compiled at least once.

A useful (if unconventional) way to always ensure a consistent
page numbering is to restart the numbering in each child document
and denote the pages by `\textit{child}|.|\textit{page}'
where \textit{child} represents the chapter/section number of the child file.
This can be achieved by the command
|\numberwithin{page}{|\textit{child}|}|
of the \textsf{amsmath} package
where \textit{child} can be |chapter| or |section|
depending on the chosen structuring.
Alternatively, one can modify the macro |\thepage| appropriately
and reset the counter |page| at the start of each child file.

%%%%%%%%%%%%%%%%%%%%%%%%%%%%%%%%%%%%%%%%%%%%%%%%%%%%%%%%%%%%%%%%%%%%%%%%%%%%%%%%
\subsection{Conditional Processing}
\label{sec:conditional}

The package provides a mechanism to compile different versions
of a document. To customise the versions further some conditional processing
can come in handy to distinguish which version is being compiled.
The package provides two macros to describe the compilation context:

%%%%%%%%%%%%%%%%%%%%%%%%%%%%%%%%%%%%%%%%
\DescribeMacro{\ifchilddoc}
The conditional |\ifchilddoc| distinguishes between the compilation of
child documents and the main document:
%
\begin{center}
|\ifchilddoc |\textit{child-code}| |[|\||else |\textit{main-code}]| \||fi|
\end{center}

%%%%%%%%%%%%%%%%%%%%%%%%%%%%%%%%%%%%%%%%
\DescribeMacro{\childdocname}
\DescribeMacro{\childdocjob}
The macro |\childdocname| contains the filename (without extension)
of the main or child file being processed.
Note that |\childdocjob| will always contain the name of the main file.

%%%%%%%%%%%%%%%%%%%%%%%%%%%%%%%%%%%%%%%%
\paragraph{Title Page.}

Conditional processing can be used to include a title or banner page
in the main document when proper precautions are taken.
Importantly, the code in the main file should ensure that the page counter
(as well as other status parameters which are stored in the |.aux| files)
takes the same value after the conditional processing.
Otherwise the page numbers may take divergent values
depending on which part is compiled.

For example, a title page could be declared by:
%
\begin{center}
\begin{tabular}{l}
|\ifchilddoc\||else|\\
|\addtocounter{page}{-1}|\\
\textit{code for title page}\\
|\newpage|\\
|\||fi|
\end{tabular}
\end{center}
%
A banner page for the child documents can be generated by:
%
\begin{center}
\begin{tabular}{l}
|\ifchilddoc|\\
|\addtocounter{page}{-1}|\\
\textit{code for banner page}\\
|\newpage|\\
|\||fi|
\end{tabular}
\end{center}
%
Here one could write a message such as:
\begin{center}
|This is the part \childdocname{} of \childdocjob{}.|
\end{center}

%%%%%%%%%%%%%%%%%%%%%%%%%%%%%%%%%%%%%%%%%%%%%%%%%%%%%%%%%%%%%%%%%%%%%%%%%%%%%%%%
\subsection{Flags}
\label{sec:flags}

The package makes it easy to generate different versions
of the main or child documents.
To this end compilation flags can be defined
and assigned different default values.
They will be particularly useful in conjunction
with the forwarding mechanism described in \secref{sec:forward}.

For example, it may be useful to have a flag |\version|
which can be set to |draft| or |final|.
The document source will contain some conditional code
depending on the value of |\version|.
Suppose further, the flag should default to |final| for the main file
and to |draft| for child files
which is a natural assignment for editing the document.
This is achieved by placing the following code
in the preamble of the main document
(below the |\childdocmain| directive):
%
\begin{center}
\begin{tabular}{l}
|\ifchilddoc|\\
|\providecommand{\version}{draft}|\\
|\||else|\\
|\providecommand{\version}{final}|\\
|\||fi|
\end{tabular}
\end{center}
%
The definition by |\providecommand| makes sure
that previous definitions are not overwritten.
Further statements |\providecommand{\version}{...}|
can thus be added before the above code to override it.

For the main file, one might add a line
(between |\childdocmain| and the above block)
%
\begin{center}
|%\ifchilddoc\||else\providecommand{\version}{draft}\||fi|
\end{center}
%
which can be uncommented to produce a draft version.
Likewise one can add a line to the very top of a child file
(above the |\childdocof{|\textit{main}|}| directive)
%
\begin{center}
|%\providecommand{\version}{final}|
\end{center}
%
which can be uncommented to produce the final version of this child document.

%%%%%%%%%%%%%%%%%%%%%%%%%%%%%%%%%%%%%%%%%%%%%%%%%%%%%%%%%%%%%%%%%%%%%%%%%%%%%%%%
\subsection{Forwarding}
\label{sec:forward}

Different versions of the main or child documents
using compilation flags as described in \secref{sec:flags}
can be (permanently) stored in different files
for convenient compilation, viewing and distribution.
To this end, the package defines a command
to pass on compilation to a different file:

%%%%%%%%%%%%%%%%%%%%%%%%%%%%%%%%%%%%%%%%
\DescribeMacro{\childdocforward}
The command |\childdocforward| redirects processing to
another source file:
%
\begin{center}
\begin{tabular}{l}
|% \iffalse
%
% childdoc.dtx Copyright (C) 2017-2018 Niklas Beisert
%
% This work may be distributed and/or modified under the
% conditions of the LaTeX Project Public License, either version 1.3
% of this license or (at your option) any later version.
% The latest version of this license is in
%   http://www.latex-project.org/lppl.txt
% and version 1.3 or later is part of all distributions of LaTeX
% version 2005/12/01 or later.
%
% This work has the LPPL maintenance status `maintained'.
%
% The Current Maintainer of this work is Niklas Beisert.
%
% This work consists of the files childdoc.dtx and childdoc.ins
% and the derived files childdoc.def and cdocsamp.tex with
% cdocsch1.tex, cdocsch2.tex, cdocsdrf.tex, cdocsfn1.tex, cdocsfn2.tex.
%
%<package>\ifdefined\childdocmain\endinput\fi
%<package>\ProvidesFile{childdoc.def}[2018/12/30 v2.0 child document driver]
%<samplemain>\ProvidesFile{cdocsamp.tex}[2018/12/30 v2.0 sample for childdoc]
%<*driver>
%\ProvidesFile{childdoc.drv}[2018/12/30 v2.0 childdoc reference manual file]
\PassOptionsToClass{10pt,a4paper}{article}
\documentclass{ltxdoc}

\usepackage[margin=35mm]{geometry}
\usepackage{hyperref}
\usepackage{hyperxmp}
\usepackage[usenames]{color}

\hypersetup{colorlinks=true}
\hypersetup{pdfstartview=FitH}
\hypersetup{pdfpagemode=UseNone}
\hypersetup{pdfsource={}}
\hypersetup{pdflang={en-UK}}
\hypersetup{pdfcopyright={Copyright 2017-2018 Niklas Beisert.
  This work may be distributed and/or modified under the
  conditions of the LaTeX Project Public License, either version 1.3
  of this license or (at your option) any later version.}}
\hypersetup{pdflicenseurl={http://www.latex-project.org/lppl.txt}}
\hypersetup{pdfcontactaddress={ETH Zurich, ITP, HIT K,
  Wolfgang-Pauli-Strasse 27}}
\hypersetup{pdfcontactpostcode={8093}}
\hypersetup{pdfcontactcity={Zurich}}
\hypersetup{pdfcontactcountry={Switzerland}}
\hypersetup{pdfcontactemail={nbeisert@itp.phys.ethz.ch}}
\hypersetup{pdfcontacturl={http://people.phys.ethz.ch/\xmptilde nbeisert/}}

\newcommand{\secref}[1]{\hyperref[#1]{section \ref*{#1}}}

\parskip1ex
\parindent0pt
\let\olditemize\itemize
\def\itemize{\olditemize\parskip0pt}

\begin{document}

\title{The \textsf{childdoc} Package}
\hypersetup{pdftitle={The childdoc Package}}
\author{Niklas Beisert\\[2ex]
  Institut f\"ur Theoretische Physik\\
  Eidgen\"ossische Technische Hochschule Z\"urich\\
  Wolfgang-Pauli-Strasse 27, 8093 Z\"urich, Switzerland\\[1ex]
  \href{mailto:nbeisert@itp.phys.ethz.ch}
  {\texttt{nbeisert@itp.phys.ethz.ch}}}
\hypersetup{pdfauthor={Niklas Beisert}}
\hypersetup{pdfsubject={Manual for the LaTeX2e Package childdoc}}
\date{30 December 2018, \textsf{v2.0}}
\maketitle

\begin{abstract}\noindent
\textsf{childdoc} is a \LaTeXe{} package
that enables the direct compilation
of document sections included by |\include|
to individual files.
\end{abstract}

\begingroup
\parskip0ex
\tableofcontents
\endgroup

%%%%%%%%%%%%%%%%%%%%%%%%%%%%%%%%%%%%%%%%%%%%%%%%%%%%%%%%%%%%%%%%%%%%%%%%%%%%%%%%
%%%%%%%%%%%%%%%%%%%%%%%%%%%%%%%%%%%%%%%%%%%%%%%%%%%%%%%%%%%%%%%%%%%%%%%%%%%%%%%%
\section{Introduction}

\LaTeX{} provides a mechanism to structure a large document (such as a book)
into a main file and several child files (containing the chapters)
using the |\include| command.
This mechanism is beneficial for documents
which span hundreds of pages in order to
make the source file(s) more manageable.
Moreover, compilation can be restricted to
selected child files by means of the |\includeonly| command.
The latter feature can be used to reduce the compilation time while editing
(this was significantly more useful in the earlier days of \LaTeX{})
or to generate a smaller document which is easier to navigate.
Another application of |\includeonly| is to generate
documents consisting of selected parts of the complete document.

However, there are a few drawbacks of the plain |\include| mechanism:
\begin{itemize}
\item
The child files cannot be compiled on their own,
they can only be compiled via the main file.
A naive editing environment
(such as a text editor with an option
to have the current file processed by \LaTeX)
may require one to switch to the main file before compiling;
attempting to compile the child file produces errors.
\item
The main file must be modified (each time)
to adjust the |\includeonly| command
to the present needs. This easily leaves the main file in a messy state.
\item
The generated document will always carry the filename
of the main document. This is inconvenient if
several child files are to be compiled and
to be kept for distribution.
\end{itemize}

The present package provides a simple interface
to make child files individually compilable by \LaTeX{}.
Compiling a child file then has the same effect as compiling
the main file with an |\includeonly| command
to select the appropriate child.
Moreover the generated document will carry the name of the child
rather than the main file.
This resolves all three above issues.

This feature is meant to make the editing of books,
thesis documents and lecture notes somewhat more convenient.
However, the package can also be used efficiently for
composing a series of documents (such as exercise sheets)
which are typically distributed individually.
It then assists the author in generating the individual documents
(potentially in different versions)
as well as a document containing the collected series.
Another application is in developing style files
or other kinds of included material
where compilation of the style file could redirect
to a sample or test file.

%%%%%%%%%%%%%%%%%%%%%%%%%%%%%%%%%%%%%%%%%%%%%%%%%%%%%%%%%%%%%%%%%%%%%%%%%%%%%%%%
%%%%%%%%%%%%%%%%%%%%%%%%%%%%%%%%%%%%%%%%%%%%%%%%%%%%%%%%%%%%%%%%%%%%%%%%%%%%%%%%
\section{Usage}

First of all, the package \textsf{childdoc} is \emph{not} a standard
\LaTeXe{} |.sty| style file! Therefore it needs to be invoked in
a non-standard way.

%%%%%%%%%%%%%%%%%%%%%%%%%%%%%%%%%%%%%%%%%%%%%%%%%%%%%%%%%%%%%%%%%%%%%%%%%%%%%%%%
\subsection{Included Files}
\label{sec:include}

%%%%%%%%%%%%%%%%%%%%%%%%%%%%%%%%%%%%%%%%
\DescribeMacro{\childdocmain}
To use the package, add the commands
\begin{center}
\begin{tabular}{l}
|\input{childdoc.def}|\\
|\childdocmain{}|\\
\end{tabular}
\end{center}
at the very top of the main \LaTeX{} file,
in particular \emph{before} the |\documentclass| statement!
The argument of |\childdocmain| should be left empty
(but it must be present).

%%%%%%%%%%%%%%%%%%%%%%%%%%%%%%%%%%%%%%%%
\DescribeMacro{\childdocof}
Furthermore, add the commands
\begin{center}
\begin{tabular}{l}
|\input{childdoc.def}|\\
|\childdocof{|\textit{main}|}|\\
\end{tabular}
\end{center}
at the top of every child file \textit{child}
which is included by |\include{|\textit{child}|}|
from within the main file
(or at least for those files to be compiled individually).
The argument \textit{main} must be the filename of the main file.

There are a couple of
considerations in setting up the main and child documents:

%%%%%%%%%%%%%%%%%%%%%%%%%%%%%%%%%%%%%%%%
\paragraph{Restrictions.}

Please note the following restrictions:
\begin{itemize}
\item
|\childdocmain| must be called with one argument \textit{main}
to ensure compatibility with earlier version of the package.
It must either be empty (|\childdocmain{}|)
or precisely match the filename of the main file in which it is specified.
See \secref{sec:detection} for further information.
\item
The filename \textit{main} must be specified without the |.tex| extension.
\item
The filename \textit{main} is case sensitive
(even in case-insensitive file systems)
due to internal string comparison.
\item
The argument \textit{main} should be fully expanded, it cannot be a macro.
\item
Subdirectories and special characters should be avoided in filenames.
\item
The command |\childdocmain{|\textit{main}|}| must be followed by a whitespace.
It should not be followed immediately by another command
or by a comment mark `|%|'.
This is because the \TeX{} parser reads the token immediately following
the argument of |\childdocmain| and puts it
at the beginning of every child section;
however, a white\-space is ignored.
\end{itemize}

%%%%%%%%%%%%%%%%%%%%%%%%%%%%%%%%%%%%%%%%
\paragraph{Content of Main File.}

It is advisable to place all content in the child files included by |\include|.
Any output contained in the main file will appear in all child documents
unless suppressed manually;
it cannot be suppressed automatically by the |\includeonly| directive
and thus should normally be avoided.
A method to include some content in the main file
by means of conditional processing is described in \secref{sec:conditional}.

%%%%%%%%%%%%%%%%%%%%%%%%%%%%%%%%%%%%%%%%
\paragraph{Page Numbering.}

When only a part of the document is compiled,
the appropriate numbering of pages
(as well as other status parameters)
is determined from the |.aux| files.
The latter contain information from previous passes.
However this information needs to propagate through
all intermediate child documents.
Therefore the page numbering in child documents may well
be inconsistent until the complete document is compiled at least once.

A useful (if unconventional) way to always ensure a consistent
page numbering is to restart the numbering in each child document
and denote the pages by `\textit{child}|.|\textit{page}'
where \textit{child} represents the chapter/section number of the child file.
This can be achieved by the command
|\numberwithin{page}{|\textit{child}|}|
of the \textsf{amsmath} package
where \textit{child} can be |chapter| or |section|
depending on the chosen structuring.
Alternatively, one can modify the macro |\thepage| appropriately
and reset the counter |page| at the start of each child file.

%%%%%%%%%%%%%%%%%%%%%%%%%%%%%%%%%%%%%%%%%%%%%%%%%%%%%%%%%%%%%%%%%%%%%%%%%%%%%%%%
\subsection{Conditional Processing}
\label{sec:conditional}

The package provides a mechanism to compile different versions
of a document. To customise the versions further some conditional processing
can come in handy to distinguish which version is being compiled.
The package provides two macros to describe the compilation context:

%%%%%%%%%%%%%%%%%%%%%%%%%%%%%%%%%%%%%%%%
\DescribeMacro{\ifchilddoc}
The conditional |\ifchilddoc| distinguishes between the compilation of
child documents and the main document:
%
\begin{center}
|\ifchilddoc |\textit{child-code}| |[|\||else |\textit{main-code}]| \||fi|
\end{center}

%%%%%%%%%%%%%%%%%%%%%%%%%%%%%%%%%%%%%%%%
\DescribeMacro{\childdocname}
\DescribeMacro{\childdocjob}
The macro |\childdocname| contains the filename (without extension)
of the main or child file being processed.
Note that |\childdocjob| will always contain the name of the main file.

%%%%%%%%%%%%%%%%%%%%%%%%%%%%%%%%%%%%%%%%
\paragraph{Title Page.}

Conditional processing can be used to include a title or banner page
in the main document when proper precautions are taken.
Importantly, the code in the main file should ensure that the page counter
(as well as other status parameters which are stored in the |.aux| files)
takes the same value after the conditional processing.
Otherwise the page numbers may take divergent values
depending on which part is compiled.

For example, a title page could be declared by:
%
\begin{center}
\begin{tabular}{l}
|\ifchilddoc\||else|\\
|\addtocounter{page}{-1}|\\
\textit{code for title page}\\
|\newpage|\\
|\||fi|
\end{tabular}
\end{center}
%
A banner page for the child documents can be generated by:
%
\begin{center}
\begin{tabular}{l}
|\ifchilddoc|\\
|\addtocounter{page}{-1}|\\
\textit{code for banner page}\\
|\newpage|\\
|\||fi|
\end{tabular}
\end{center}
%
Here one could write a message such as:
\begin{center}
|This is the part \childdocname{} of \childdocjob{}.|
\end{center}

%%%%%%%%%%%%%%%%%%%%%%%%%%%%%%%%%%%%%%%%%%%%%%%%%%%%%%%%%%%%%%%%%%%%%%%%%%%%%%%%
\subsection{Flags}
\label{sec:flags}

The package makes it easy to generate different versions
of the main or child documents.
To this end compilation flags can be defined
and assigned different default values.
They will be particularly useful in conjunction
with the forwarding mechanism described in \secref{sec:forward}.

For example, it may be useful to have a flag |\version|
which can be set to |draft| or |final|.
The document source will contain some conditional code
depending on the value of |\version|.
Suppose further, the flag should default to |final| for the main file
and to |draft| for child files
which is a natural assignment for editing the document.
This is achieved by placing the following code
in the preamble of the main document
(below the |\childdocmain| directive):
%
\begin{center}
\begin{tabular}{l}
|\ifchilddoc|\\
|\providecommand{\version}{draft}|\\
|\||else|\\
|\providecommand{\version}{final}|\\
|\||fi|
\end{tabular}
\end{center}
%
The definition by |\providecommand| makes sure
that previous definitions are not overwritten.
Further statements |\providecommand{\version}{...}|
can thus be added before the above code to override it.

For the main file, one might add a line
(between |\childdocmain| and the above block)
%
\begin{center}
|%\ifchilddoc\||else\providecommand{\version}{draft}\||fi|
\end{center}
%
which can be uncommented to produce a draft version.
Likewise one can add a line to the very top of a child file
(above the |\childdocof{|\textit{main}|}| directive)
%
\begin{center}
|%\providecommand{\version}{final}|
\end{center}
%
which can be uncommented to produce the final version of this child document.

%%%%%%%%%%%%%%%%%%%%%%%%%%%%%%%%%%%%%%%%%%%%%%%%%%%%%%%%%%%%%%%%%%%%%%%%%%%%%%%%
\subsection{Forwarding}
\label{sec:forward}

Different versions of the main or child documents
using compilation flags as described in \secref{sec:flags}
can be (permanently) stored in different files
for convenient compilation, viewing and distribution.
To this end, the package defines a command
to pass on compilation to a different file:

%%%%%%%%%%%%%%%%%%%%%%%%%%%%%%%%%%%%%%%%
\DescribeMacro{\childdocforward}
The command |\childdocforward| redirects processing to
another source file:
%
\begin{center}
\begin{tabular}{l}
|\input{childdoc.def}|\\
|\childdocforward[|\textit{main}|]{|\textit{dest}|}|\\
\end{tabular}
\end{center}
%
The argument \textit{dest} is the destination file
(without extension).
It should be the main file or one of the child files.
Note that further \textsf{childdoc} directives
such as |\childdocof| and |\childdocforward|
in the indicated file will be processed in this form.
The optional argument \textit{main}
passes on directly to the main file \textit{main}
while pretending to compile the child \textit{dest}.
This form behaves as if \textit{dest}
issues |\childdocof{|\textit{main}|}| right away,
and no further \textsf{childdoc} directives will be processed.

%%%%%%%%%%%%%%%%%%%%%%%%%%%%%%%%%%%%%%%%
\DescribeMacro{\...prefix}
In the alternative form |\childdocforwardprefix|,
%
\begin{center}
\begin{tabular}{l}
|\input{childdoc.def}|\\
|\childdocforwardprefix[|\textit{main}|]{|\textit{prefix}|}{|\textit{dest}|}|
\end{tabular}
\end{center}
%
the destination file is determined by a pattern
depending on the current file:
To make this work, the current file must be called
`{\textit{prefix}\hspace{0.2em}\textit{suffix}}'
with \textit{prefix} matching precisely the argument.
Processing is then passed on to the file
`{\textit{dest}\hspace{0.2em}\textit{suffix}}'.
Surely, the same effect is achieved by
directly specifying the
argument `{\textit{dest}\hspace{0.2em}\textit{suffix}}'
in the first form.
However, that requires to set up a different file
for each child. With the alternative form of the command
all these files can have exactly the same content
which simplifies setting them up and maintaining them.

For example, the following file |draft.tex|
with a compilation flag |\version| as described in \secref{sec:flags}
compiles the main document as a draft:
%
\begin{center}
\begin{tabular}{l}
|\def\version{draft}|\\
|\input{childdoc.def}|\\
|\childdocforward{|\textit{main}|}|
\end{tabular}
\end{center}
%
Likewise, the following files |final|\textit{nn}|.tex|
compile the final version of the child document
|child|\textit{nn}|.tex|:
%
\begin{center}
\begin{tabular}{l}
|\def\version{final}|\\
|\input{childdoc.def}|\\
|\childdocforwardprefix{final}{child}|
\end{tabular}
\end{center}
%

Note that when several versions of a main file and/or of each child file
are to be generated, it may be convenient to set up a |Makefile| or
shell script to automatise the process.

%%%%%%%%%%%%%%%%%%%%%%%%%%%%%%%%%%%%%%%%%%%%%%%%%%%%%%%%%%%%%%%%%%%%%%%%%%%%%%%%
\subsection{Command Line Processing}
\label{sec:commandline}

The effect of redirection files can also be achieved by invoking
the \LaTeX{} compiler with a more elaborate command line.
Most conveniently this should be done as part
of a shell script or a |Makefile|.

When using \textsf{childdoc} in the main file, the following
command lines effectively perform a redirection
(note that depending on the shell being used,
backslashes may have to be doubled: `|\|' $\to$ `|\\|'):
%
\begin{center}
|... -jobname "|\textit{target}|" |\\|"|[\textit{flags}]%
|\input{childdoc.def}\childdocforward[|\textit{main}|]{|\textit{dest}|}"|
\end{center}
%
Here \textit{target} is the name of the output file,
\textit{main} is the name of the main file
and \textit{dest} is the name of the main or child file to be processed
(all filenames without extensions).
The optional argument \textit{main} can be omitted
if \textit{main} matches \textit{dest}.
Optionally, compilation \textit{flags} can be defined via |\def| commands.
This command line makes the \TeX{} engine believe
it is compiling the file \textit{target}
whose content is specified as the latter parameter.
The provided code then forwards the processing to
\textit{main} or \textit{dest} as described in \secref{sec:forward}.

%%%%%%%%%%%%%%%%%%%%%%%%%%%%%%%%%%%%%%%%%%%%%%%%%%%%%%%%%%%%%%%%%%%%%%%%%%%%%%%%
\subsection{Include by Input}
\label{sec:input}

Including child documents by |\include| has some restrictions by design.
Most notably, the content of a child document always occupies
its own set of pages; pages cannot be shared between child documents.
Usually, this behaviour makes perfect sense
because each child document contain an essential part of the document.
However, in some situations it may be desirable to compose
a document from a collection of parts
without having mandatory page breaks between then.
For this case, the package
provides a mechanism to include parts
by |\input| which can also be processed individually.
However, by construction this mechanism
requires manual handling of the content to be output.

%%%%%%%%%%%%%%%%%%%%%%%%%%%%%%%%%%%%%%%%
\DescribeMacro{\ifchilddocmanual}
The main file should be prepared as usual, see \secref{sec:include}.
However, the document body must make a distinction
between processing of an individual part and of the main document, e.g.:
%
\begin{center}
\begin{tabular}{l}
|\ifchilddocmanual|\\
|\input{\childdocname}|\\
|\||else|\\
\textit{document body with }|\input{|\textit{part}|}|\\
|\||fi|
\end{tabular}
\end{center}
%
The conditional |\ifchilddocmanual| is true whenever
a part to be included by |\input| is being compiled,
and the name of the part is stored in |\childdocname|.

%%%%%%%%%%%%%%%%%%%%%%%%%%%%%%%%%%%%%%%%
\DescribeMacro{\childdocby}
Each part to be included by |\input| should start with:
%
\begin{center}
\begin{tabular}{l}
|\input{childdoc.def}|\\
|\childdocby{|\textit{main}|}|\\
\end{tabular}
\end{center}
%
The directive |\childdocby| is similar to |\childdocof|
described in \secref{sec:include},
but the subsequent selection of content must be done manually.
To that end, both |\ifchilddoc| and |\ifchilddocmanual|
will be true upon processing of a part,
and the name of the part is stored in |\childdocname|.
Note that |\jobname| will be set to the filename of the current part
so that each part receives an individual |.aux| file
that does not interfere with the |.aux| file(s) of the main document.
This behaviour can be altered by the alternative form
|\childdocby[*]{|\textit{main}|}| (with a non-empty optional argument)
which uses the |.aux| file of the main document
by setting |\jobname| to \textit{main}.

%%%%%%%%%%%%%%%%%%%%%%%%%%%%%%%%%%%%%%%%%%%%%%%%%%%%%%%%%%%%%%%%%%%%%%%%%%%%%%%%
\subsection{Driver Development}
\label{sec:driver}

The \textsf{childdoc} mechanism can also be use for the development
of definition files such as \LaTeX{} styles or classes.
This case differs from the above setup with multiple parts
included by |\include| in that no |\includeonly| should be invoked.
This can be achieved by starting the include file
(before |\ProvidesPackage|) with:
%
\begin{center}
\begin{tabular}{l}
|\input{childdoc.def}|\\
|\childdocforward{|\textit{main}|}|\\
\end{tabular}
\end{center}
%
or alternatively with:
%
\begin{center}
\begin{tabular}{l}
|\input{childdoc.def}|\\
|\childdocby{|\textit{main}|}|\\
\end{tabular}
\end{center}
%
Both forms have slightly different effects as described above.
The main file is prepared as usual, see \secref{sec:include}.

%%%%%%%%%%%%%%%%%%%%%%%%%%%%%%%%%%%%%%%%%%%%%%%%%%%%%%%%%%%%%%%%%%%%%%%%%%%%%%%%
\subsection{Legacy Detection}
\label{sec:detection}

The directive |\childdocmain| in the main file can detect
whether the complete document or merely a child is to be compiled
even without using the directive |\childdocof|.
This method is deprecated because it is less robust
and there is no compelling reason to use it;
it is merely provided for backward compatibility
and it may be removed in future versions.

If the detection mechanism is to be used,
it is mandatory to correctly specify
the filename of the main file as the argument of |\childdocmain|:
%
\begin{center}
\begin{tabular}{l}
|\input{childdoc.def}|\\
|\childdocmain{|\textit{main}|}|\\
\end{tabular}
\end{center}
%
If |\jobname| does not match the argument \textit{main} of |\childdocmain|,
it is assumed that |\jobname| points to the child file to be compiled.
When using |\childdocmain| with the main file specified as argument,
it suffices to start a child file
with just |\input{|\textit{main}|}|
without loading of the package and using |\childdocof|.
If instead all processing is done
with the appropriate \textsf{childdoc} directives,
the argument of \textit{main} of |\childdocmain| can be empty.

An alternative version of the command line processing described
in \secref{sec:commandline} using the detection mechanism reads:
%
\begin{center}
|... -jobname "|\textit{target}|" "|[\textit{flags}]%
[|\def\jobname{|\textit{dest}|}|]|\input{|\textit{main}|}"|
\end{center}

%%%%%%%%%%%%%%%%%%%%%%%%%%%%%%%%%%%%%%%%%%%%%%%%%%%%%%%%%%%%%%%%%%%%%%%%%%%%%%%%
\subsection{Manual Code}
\label{sec:manual}

In case one cannot be certain whether the definitions file |childdoc.def|
is installed on the target \TeX{} distribution
and one prefers not to ship it,
it is conceivable to paste a few relevant commands into the sources.

To that end, drop all statements |\input{childdoc.def}|
and perform the replacements as outlined below.
Instead of |\childdocmain{|\textit{main}|}| add the following code
to the top of the main file:
%
\begin{center}
\begin{tabular}{l}
|\||ifdefined\childdocname\endinput\||fi\newif\ifchilddoc|\\
|\edef\childdocname{\scantokens\expandafter{\jobname\noexpand}}|\\
|\def\childdocmain{|\textit{main}|}\||ifx\childdocmain\childdocname\||else|\\
|\childdoctrue\includeonly{\childdocname}\let\jobname\childdocmain\||fi|\\
\end{tabular}
\end{center}
%
Instead of |\childdocof{|\textit{main}|}| just include the main file
at the top of each child file:
%
\begin{center}
|\input{|\textit{main}|}|
\end{center}
%
A simple redirection |\childdocforward{|\textit{dest}|}| is achieved by:
%
\begin{center}
|\def\jobname{|\textit{dest}|}\input{\jobname}|
\end{center}
%
The redirection with prefix
|\childdocforwardprefix[|\textit{prefix}|]{|\textit{dest}|}|
is accomplished by:
%
\begin{center}
\begin{tabular}{l}
|{\edef\jobname{\scantokens\expandafter{\jobname\noexpand}}|\\
|\def\redirectjob |\textit{prefix}|#1~~~{\gdef\jobname{|\textit{dest}|#1}}|\\
|\expandafter\redirectjob\jobname~~~}\input{\jobname}|
\end{tabular}
\end{center}

In an alternative approach,
child documents can be compiled by a specific command line
without additional code or specific definitions:
%
\begin{center}
|... -jobname "|\textit{target}|" "|[\textit{flags}]%
|\includeonly{|\textit{dest}|}\input{|\textit{main}|}"|
\end{center}
%

%%%%%%%%%%%%%%%%%%%%%%%%%%%%%%%%%%%%%%%%%%%%%%%%%%%%%%%%%%%%%%%%%%%%%%%%%%%%%%%%
%%%%%%%%%%%%%%%%%%%%%%%%%%%%%%%%%%%%%%%%%%%%%%%%%%%%%%%%%%%%%%%%%%%%%%%%%%%%%%%%
\section{Information}

%%%%%%%%%%%%%%%%%%%%%%%%%%%%%%%%%%%%%%%%%%%%%%%%%%%%%%%%%%%%%%%%%%%%%%%%%%%%%%%%
\subsection{Copyright}

Copyright \copyright{} 2017--2018 Niklas Beisert

This work may be distributed and/or modified under the
conditions of the \LaTeX{} Project Public License, either version 1.3
of this license or (at your option) any later version.
The latest version of this license is in
  \url{http://www.latex-project.org/lppl.txt}
and version 1.3 or later is part of all distributions of \LaTeX{}
version 2005/12/01 or later.

This work has the LPPL maintenance status `maintained'.

The Current Maintainer of this work is Niklas Beisert.

This work consists of the files |README.txt|, |childdoc.ins| and |childdoc.dtx|
as well as the derived files |childdoc.def|, |cdocsamp.tex|
with |cdocsch1.tex|, |cdocsch2.tex|, |cdocspt3.tex|, |cdocspt4.tex|,
|cdocsdrf.tex|, |cdocsfn1.tex|, |cdocsfn2.tex|
as well as |childdoc.pdf|.

%%%%%%%%%%%%%%%%%%%%%%%%%%%%%%%%%%%%%%%%%%%%%%%%%%%%%%%%%%%%%%%%%%%%%%%%%%%%%%%%
\subsection{Files and Installation}

The package consists of the files:
%
\begin{center}
\begin{tabular}{ll}
    |README.txt|   & readme file \\
    |childdoc.ins| & installation file \\
    |childdoc.dtx| & source file \\
    |childdoc.def| & definition file \\
    |cdocsamp.tex| & sample main file \\
    |cdocsch1.tex| & sample include file \\
    |cdocsch2.tex| & sample include file \\
    |cdocspt3.tex| & sample part file \\
    |cdocspt4.tex| & sample part file \\
    |cdocsdrf.tex| & sample redirection file \\
    |cdocsfn1.tex| & sample redirection file \\
    |cdocsfn2.tex| & sample redirection file \\
    |childdoc.pdf| & manual
\end{tabular}
\end{center}
%
The distribution consists of the files
|README.txt|, |childdoc.ins| and |childdoc.dtx|.
%
\begin{itemize}
\item
Run (pdf)\LaTeX{} on |childdoc.dtx|
to compile the manual |childdoc.pdf| (this file).
\item
Run \LaTeX{} on |childdoc.ins| to create the definitions file |childdoc.def|
and the sample |cdocsamp.tex| with include files
|cdocsch1.tex|, |cdocsch2.tex|, |cdocspt3.tex|, |cdocspt4.tex|,
|cdocsdrf.tex|, |cdocsfn1.tex|, |cdocsfn2.tex|.
Then copy the file |childdoc.def| to an appropriate directory of your \LaTeX{}
distribution, e.g.\ \textit{texmf-root}|/tex/latex/childdoc|.
\end{itemize}

%%%%%%%%%%%%%%%%%%%%%%%%%%%%%%%%%%%%%%%%%%%%%%%%%%%%%%%%%%%%%%%%%%%%%%%%%%%%%%%%
\subsection{Related CTAN Packages}

There are several other packages which offer a similar functionality:
%
\begin{itemize}
\item
The packages
\href{http://ctan.org/pkg/docmute}{\textsf{docmute}},
\href{http://ctan.org/pkg/includex}{\textsf{includex}} and
\href{http://ctan.org/pkg/standalone}{\textsf{standalone}}
provide commands to include only the document body of
a child file thus allowing both files to be compiled individually.
\item
The packages \href{http://ctan.org/pkg/subdocs}{\textsf{subdocs}}
and \href{http://ctan.org/pkg/subfiles}{\textsf{subfiles}}
provide structures in which the main and child documents can be
encapsulated and allowing them to be compiled individually.
The inclusion mechanism is different from the conventional |\include|.
\item
The package \href{http://ctan.org/pkg/combine}{\textsf{combine}}
is an elaborate solution to combine several documents into one.
\end{itemize}
%
See also the CTAN topic \href{http://ctan.org/topic/subdocs}{\textsf{subdocs}}
for further related packages.
The present package differs from the above solutions in that
a document structure constructed with the conventional |\include| mechanism
just needs two extra commands at the top of every file
such that all constituent files can be compiled individually.

%%%%%%%%%%%%%%%%%%%%%%%%%%%%%%%%%%%%%%%%%%%%%%%%%%%%%%%%%%%%%%%%%%%%%%%%%%%%%%%%
%\subsection{Feature Suggestions}
%
%The following is a list of features which may be useful for future
%versions of this package:
%%
%\begin{itemize}
%\item
%\ldots
%\end{itemize}

%%%%%%%%%%%%%%%%%%%%%%%%%%%%%%%%%%%%%%%%%%%%%%%%%%%%%%%%%%%%%%%%%%%%%%%%%%%%%%%%
\subsection{Revision History}

%%%%%%%%%%%%%%%%%%%%%%%%%%%%%%%%%%%%%%%%
\paragraph{v2.0:} 2018/12/30

\begin{itemize}
\item
immediate forward processing
\item
added |\childdocby| mechanism
\item
manual restructured
\end{itemize}

%%%%%%%%%%%%%%%%%%%%%%%%%%%%%%%%%%%%%%%%
\paragraph{v1.6:} 2018/01/17

\begin{itemize}
\item
application for development of include files
\item
corrections to manual
\end{itemize}

%%%%%%%%%%%%%%%%%%%%%%%%%%%%%%%%%%%%%%%%
\paragraph{v1.5:} 2017/05/21

\begin{itemize}
\item
more complete structuring introduced
\item
|\childdocof| introduced
\item
|\childdoc| renamed to |\childdocmain|
\item
|\childredirect| renamed to |\childdocforward| and |\childdocforwardprefix|
and functionality expanded
\end{itemize}

%%%%%%%%%%%%%%%%%%%%%%%%%%%%%%%%%%%%%%%%
\paragraph{v1.0:} 2017/04/27

\begin{itemize}
\item
manual and install package
\item
first version published on CTAN
\end{itemize}

%%%%%%%%%%%%%%%%%%%%%%%%%%%%%%%%%%%%%%%%
\paragraph{v0.6:} 2017/04/26

\begin{itemize}
\item
redirection mechanism added
\end{itemize}

%%%%%%%%%%%%%%%%%%%%%%%%%%%%%%%%%%%%%%%%
\paragraph{v0.5:} 2017/04/26

\begin{itemize}
\item
functionality in definition file
\end{itemize}


%%%%%%%%%%%%%%%%%%%%%%%%%%%%%%%%%%%%%%%%%%%%%%%%%%%%%%%%%%%%%%%%%%%%%%%%%%%%%%%%
%%%%%%%%%%%%%%%%%%%%%%%%%%%%%%%%%%%%%%%%%%%%%%%%%%%%%%%%%%%%%%%%%%%%%%%%%%%%%%%%
%%%%%%%%%%%%%%%%%%%%%%%%%%%%%%%%%%%%%%%%%%%%%%%%%%%%%%%%%%%%%%%%%%%%%%%%%%%%%%%%
\appendix

\settowidth\MacroIndent{\rmfamily\scriptsize 000\ }

 \DocInput{childdoc.dtx}

\end{document}
%</driver>
% \fi
%
% %%%%%%%%%%%%%%%%%%%%%%%%%%%%%%%%%%%%%%%%%%%%%%%%%%%%%%%%%%%%%%%%%%%%%%%%%%%%%%
% %%%%%%%%%%%%%%%%%%%%%%%%%%%%%%%%%%%%%%%%%%%%%%%%%%%%%%%%%%%%%%%%%%%%%%%%%%%%%%
% \section{Sample}
%\iffalse
%<*samplemain>
%\fi
%
% The following presents a sample document
% with two chapters, two parts, a title page,
% a compile flag as well as three forwarding files to set the flag.
% It consists of eight |.tex| files:
% \begin{center}
% \begin{tabular}{ll}
% |cdocsamp.tex|&main file\\
% |cdocsch1.tex|&include file for chapter 1\\
% |cdocsch2.tex|&include file for chapter 2\\
% |cdocspt3.tex|&include file for part 3\\
% |cdocspt4.tex|&include file for part 4\\
% |cdocsdrf.tex|&forwarding file for main file in draft mode\\
% |cdocsfi1.tex|&forwarding file for final version of chapter 1\\
% |cdocsfi2.tex|&forwarding file for final version of chapter 2\\
% \end{tabular}
% \end{center}
% Each of the eight files can be compiled directly by the \LaTeX{} compiler.
%
% %%%%%%%%%%%%%%%%%%%%%%%%%%%%%%%%%%%%%%
% \paragraph{Main File.}
%
% The main file is called |cdocsamp.tex|.
%
% Load the \textsf{childdoc} definitions and
% declare the filename for the main document:
%    \begin{macrocode}
\input{childdoc.def}
\childdocmain{}
%    \end{macrocode}

% Optional override for |\version| flag:
%    \begin{macrocode}
%%\ifchilddoc\else\providecommand{\version}{draft}\fi
%    \end{macrocode}

% Define the default values for the |\version| flag
% (|final| for the main file and |draft| for childs):
%    \begin{macrocode}
\ifchilddoc
\providecommand{\version}{draft}
\else
\providecommand{\version}{final}
\fi
%    \end{macrocode}

% Load the standard document class:
%    \begin{macrocode}
\documentclass[12pt]{article}
%    \end{macrocode}

% Start the document body:
%    \begin{macrocode}
\begin{document}
%    \end{macrocode}

% Declare a title page.
% Print title, part of document being processed and version flag:
%    \begin{macrocode}
\addtocounter{page}{-1}
\begin{center}
{\LARGE\bfseries{}childdoc example\par}
\vspace{1cm}
\ifchilddoc
\ifchilddocmanual part\else chapter\fi:
`\childdocname' of `\childdocjob'\par
\else
main document: `\childdocjob'\par
\fi
version: \version\par
\end{center}
\newpage
%    \end{macrocode}

% Manually include selected file,
% otherwise process as usual:
%    \begin{macrocode}
\ifchilddocmanual
\section*{part `\childdocname'}
\input{\childdocname}
\else
%    \end{macrocode}

% Include the two chapters:
%    \begin{macrocode}
\include{cdocsch1}
\include{cdocsch2}
%    \end{macrocode}

% Include the two parts unless only chapters should be displayed:
%    \begin{macrocode}
\ifchilddoc\else
\section{part three}
\input{cdocspt3}
\section{part four}
\input{cdocspt4}
\fi
%    \end{macrocode}

% Process as usual until here:
%    \begin{macrocode}
\fi
%    \end{macrocode}

% End of document body:
%    \begin{macrocode}
\end{document}
%    \end{macrocode}
%\iffalse
%</samplemain>
%\fi
%
% %%%%%%%%%%%%%%%%%%%%%%%%%%%%%%%%%%%%%%
% \paragraph{Chapter Include Files.}
%
% The include files are called |cdocsch1.tex| and |cdocsch2.tex|.
%
%\iffalse
%<*samplechap1|samplechap2>
%\fi

% Optional override for |\version| flag:
%    \begin{macrocode}
%%\providecommand{\version}{final}
%    \end{macrocode}

% Include the main document:
%    \begin{macrocode}
\input{childdoc.def}
\childdocof{cdocsamp}
%    \end{macrocode}

%\iffalse
%</samplechap1|samplechap2>
%\fi
%
%\iffalse
%<*samplechap1>
%\fi
% Some text for chapter 1:
%    \begin{macrocode}
\section{one}
some text in chapter one
%    \end{macrocode}

%\iffalse
%</samplechap1>
%\fi
% Some text for chapter 2:
%\iffalse
%<*samplechap2>
%\fi
%    \begin{macrocode}
\section{two}
more text in chapter two
%    \end{macrocode}

%\iffalse
%</samplechap2>
%\fi
%
% %%%%%%%%%%%%%%%%%%%%%%%%%%%%%%%%%%%%%%
% \paragraph{Part Include Files.}
%
% The include files are called |cdocspt3.tex| and |cdocspt4.tex|.
%
%\iffalse
%<*samplepart3|samplepart4>
%\fi

% Optional override for |\version| flag:
%    \begin{macrocode}
%%\providecommand{\version}{final}
%    \end{macrocode}

% Include the main document:
%    \begin{macrocode}
\input{childdoc.def}
\childdocby{cdocsamp}
%    \end{macrocode}

%\iffalse
%</samplepart3|samplepart4>
%\fi
%
%\iffalse
%<*samplepart3>
%\fi
% Some text for part 3:
%    \begin{macrocode}
some text in part three
%    \end{macrocode}

%\iffalse
%</samplepart3>
%\fi
% Some text for part 4:
%\iffalse
%<*samplepart4>
%\fi
%    \begin{macrocode}
more text in part four
%    \end{macrocode}

%\iffalse
%</samplepart4>
%\fi
%
% %%%%%%%%%%%%%%%%%%%%%%%%%%%%%%%%%%%%%%
% \paragraph{Forwarding for a Complete Draft.}
%
% The following forwarding file |cdocsdrf.tex|
% compiles the main document in draft mode:
%\iffalse
%<*sampledraft>
%\fi
%    \begin{macrocode}
\def\version{draft}
\input{childdoc.def}
\childdocforward{cdocsamp}
%    \end{macrocode}

%\iffalse
%</sampledraft>
%\fi
%
% %%%%%%%%%%%%%%%%%%%%%%%%%%%%%%%%%%%%%%
% \paragraph{Forwarding for Final Version of the Chapters.}
%
% The following forwarding files |cdocsfn1.tex| and |cdocsfn2.tex|
% (with identical content)
% compile the final versions of the child documents
% |cdocsch1.tex| and |cdocsch2.tex|, respectively:
%\iffalse
%<*samplefinal>
%\fi
%    \begin{macrocode}
\def\version{final}
\input{childdoc.def}
\childdocforwardprefix[cdocsamp]{cdocsfn}{cdocsch}
%    \end{macrocode}

%\iffalse
%</samplefinal>
%\fi
%
% %%%%%%%%%%%%%%%%%%%%%%%%%%%%%%%%%%%%%%
% \paragraph{Command Line Processing.}
%
% The following three command lines generate the output files
% |cdocscld|, |cdocscl1| and |cdocscl2|
% which should be identical to
% |cdocsdrf|, |cdocsch1| and |cdocsfn2|, respectively:
% \begin{center}
% \begin{tabular}{l}
% |latex -jobname cdocscld \|\\
% |  "\def\version{draft}\input{childdoc.def}\childdocforward{cdocsamp}"|\\
% |latex -jobname cdocscl1 \|\\
% |  "\input{childdoc.def}\childdocforward[cdocsamp]{cdocsch1}"|\\
% |latex -jobname cdocscl2 \|\\
% |  "\def\version{final}\input{childdoc.def}\childdocforward{cdocsch2}"|
% \end{tabular}
% \end{center}
% Note that the trailing backslash on each first line
% merely continues the input to the second line
% (for convenient cut ant paste).
% Furthermore, the command |latex| can be replaced by any
% of its alternative versions such as |pdflatex|.
%
% %%%%%%%%%%%%%%%%%%%%%%%%%%%%%%%%%%%%%%%%%%%%%%%%%%%%%%%%%%%%%%%%%%%%%%%%%%%%%%
% %%%%%%%%%%%%%%%%%%%%%%%%%%%%%%%%%%%%%%%%%%%%%%%%%%%%%%%%%%%%%%%%%%%%%%%%%%%%%%
% \section{Implementation}
%\iffalse
%<*package>
%\fi
%
% This section describes the definitions file |childdoc.def|.

% The definitions cannot be loaded using |\usepackage| or |\RequirePackage|
% which has a mechanism to prevent loading a style file more than once.
% When loading the definitions by means of |\input|
% multiple instances have to be prevented manually:
%\iffalse
%This code needs to be before the `\ProvidesFile' directive
%which is defined at the beginning of this file.
%Therefore it is also placed there and commented out here.
%</package>
%<*discard>
%\fi
%    \begin{macrocode}
\ifdefined\childdocmain\endinput\fi
%    \end{macrocode}
%\iffalse
%</discard>
%<*package>
%\fi
%
% \macro{\ifchilddoc}
% \macro{\ifchilddocmanual}
% The conditional |\ifchilddoc| tells whether a
% child (true) or main (false) document is being compiled.
% The conditional |\ifchilddocmanual| tells whether
% the |\includeonly| mechanism is used (false) or
% the selection of child files must be performed manually (true).
% The definitions initialise to false:
%    \begin{macrocode}
\newif\ifchilddoc
\newif\ifchilddocmanual
%    \end{macrocode}

% \macro{\childdocname}
% \macro{\childdocjob}
% The macro |\childdocname| stores the name of the main document
% to be compiled. The macro |\childdocjob| stores the name of
% the document on which the \LaTeX{} compiler was originally invoked.
% The content of |\jobname| cannot be compared
% to filenames specified in the source due to different catcodes.
% The following code rescans |\jobname|, stores the result
% in |\childdocname| and saves a copy in |\childdocjob|:
%    \begin{macrocode}
\edef\childdocname{\scantokens\expandafter{\jobname\noexpand}}
\let\childdocjob\childdocname
%    \end{macrocode}

% \macro{\childdocdisable}
% The macro |\childdocdisable| prevents the main file
% from being processed more than once.
% At this stage, the main document command |\childdocmain|
% is assumed to be called once again where it should do nothing.
% Any subsequent call to it should prevent
% a secondary processing of the main document
% It overwrites the forwarding commands
% |\childdocof| and |\childdocforward|
% with empty macros to prevent further inclusions of the main document:
%    \begin{macrocode}
\newcommand{\childdocdisable}
{
  \renewcommand{\childdocmain}[1]{\renewcommand{\childdocmain}[1]{\endinput}}
  \renewcommand{\childdocof}[1]{}
  \renewcommand{\childdocby}[2][]{}
  \renewcommand{\childdocforward}[2][]{}
  \renewcommand{\childdocdisable}{}
}
%    \end{macrocode}

% \macro{\childdocmain}
% The macro |\childdocmain| is to be called at the top of the main file
% with nothing or the main filename (without extension) as argument.
% First, it breaks loops.
% If the argument is not empty and does not match |\childdocname|
% (which is set by the first inclusion of |childdoc.def|),
% |\ifchilddoc| is set to true, |\includeonly| is applied to the child file
% and |\jobname| is set to the main file
% (for proper handling of |.aux| files):
%    \begin{macrocode}
\newcommand{\childdocmain}[1]
{
  \childdocdisable\childdocmain{}
  \if?#1?\else
    \begingroup
      \def\childdoctmp{#1}
      \ifx\childdoctmp\childdocname
        \def\childdoctmp{}
      \else
        \def\childdoctmp
        {
          \childdoctrue
          \includeonly{\childdocname}
          \def\childdocjob{#1}
          \def\jobname{#1}
        }
      \fi
      \expandafter
    \endgroup
    \childdoctmp
  \fi
}
%    \end{macrocode}

% \macro{\childdocof}
% The command |\childdocof| redirects
% compilation to the main file |#1|.
%    \begin{macrocode}
\newcommand{\childdocof}[1]
{
  \childdocdisable
  \childdoctrue
  \includeonly{\childdocname}
  \def\jobname{#1}
  \def\childdocjob{#1}
  \input{#1}
}
%    \end{macrocode}

% \macro{\childdocby}
% The command |\childdocby| ....
%    \begin{macrocode}
\newcommand{\childdocby}[2][]
{
  \childdocdisable
  \childdoctrue
  \childdocmanualtrue
  \if?#1?\else
    \def\jobname{#2}
  \fi
  \def\childdocjob{#2}
  \input{#2}
  \endinput
}
%    \end{macrocode}

% \macro{\childdocforward}
% The command |\childdocforward| redirects
% compilation to the main file or
% (if the optional argument is given) a child file.
% Parameters are set as if the main file
% or a child file starting with |\childdocof| was compiled.
% Then compilation is handed over to the main file:
%    \begin{macrocode}
\newcommand{\childdocforward}[2][]
{
  \begingroup
    \if?#1?
      \def\childdoctmp
      {
        \def\childdocname{#2}
        \def\childdocjob{#2}
        \def\jobname{#2}
        \input{#2}
        \endinput
      }
    \else
      \def\childdoctmp
      {
        \childdocdisable
        \def\childdocname{#2}
        \childdoctrue
        \includeonly{#2}
        \def\childdocjob{#1}
        \def\jobname{#1}
        \input{#1}
        \endinput
      }
    \fi
    \expandafter
  \endgroup
  \childdoctmp
}
%    \end{macrocode}

% \macro{\childdocforwardprefix}
% The command |\childdocforwardprefix| redirects
% compilation to the main or a child file by means of a pattern.
% The prefix |#1| in the current filename is replaced by |#2|
% and the suffix of the current filename is kept
% (it is assumed that the filename does not contain the substring `|~~~|'
% which is used as a delimiter).
% Compilation is handed over to the new file by |\childdocforward|:
%    \begin{macrocode}
\newcommand{\childdocforwardprefix}[3][]
{
  \begingroup
    \def\childdocextract #2##1~~~{\def\childdoctmp{\childdocforward[#1]{#3##1}}}
    \expandafter\childdocextract\childdocname~~~
    \expandafter
  \endgroup
  \childdoctmp
}
%    \end{macrocode}

% \macro{\childdoc}
% The deprecated macro |\childdoc| is a legacy version of |\childdocmain|:
%    \begin{macrocode}
\newcommand{\childdoc}{\childdocmain}
%    \end{macrocode}

% \macro{\childdocredirect}
% The deprecated macro |\childdocredirect| is a legacy version
% of |\childdocforward| and |\childdocforwardprefix|:
%    \begin{macrocode}
\newcommand{\childdocredirect}[2][]
{
  \begingroup
    \if?#1?
      \def\childdoctmp{\childdocforward{#2}}
    \else
      \def\childdoctmp{\childdocforwardprefix{#1}{#2}}
    \fi
    \expandafter
  \endgroup
  \childdoctmp
}
%    \end{macrocode}

%\iffalse
%</package>
%\fi
%
\endinput
|\\
|\childdocforward[|\textit{main}|]{|\textit{dest}|}|\\
\end{tabular}
\end{center}
%
The argument \textit{dest} is the destination file
(without extension).
It should be the main file or one of the child files.
Note that further \textsf{childdoc} directives
such as |\childdocof| and |\childdocforward|
in the indicated file will be processed in this form.
The optional argument \textit{main}
passes on directly to the main file \textit{main}
while pretending to compile the child \textit{dest}.
This form behaves as if \textit{dest}
issues |\childdocof{|\textit{main}|}| right away,
and no further \textsf{childdoc} directives will be processed.

%%%%%%%%%%%%%%%%%%%%%%%%%%%%%%%%%%%%%%%%
\DescribeMacro{\...prefix}
In the alternative form |\childdocforwardprefix|,
%
\begin{center}
\begin{tabular}{l}
|% \iffalse
%
% childdoc.dtx Copyright (C) 2017-2018 Niklas Beisert
%
% This work may be distributed and/or modified under the
% conditions of the LaTeX Project Public License, either version 1.3
% of this license or (at your option) any later version.
% The latest version of this license is in
%   http://www.latex-project.org/lppl.txt
% and version 1.3 or later is part of all distributions of LaTeX
% version 2005/12/01 or later.
%
% This work has the LPPL maintenance status `maintained'.
%
% The Current Maintainer of this work is Niklas Beisert.
%
% This work consists of the files childdoc.dtx and childdoc.ins
% and the derived files childdoc.def and cdocsamp.tex with
% cdocsch1.tex, cdocsch2.tex, cdocsdrf.tex, cdocsfn1.tex, cdocsfn2.tex.
%
%<package>\ifdefined\childdocmain\endinput\fi
%<package>\ProvidesFile{childdoc.def}[2018/12/30 v2.0 child document driver]
%<samplemain>\ProvidesFile{cdocsamp.tex}[2018/12/30 v2.0 sample for childdoc]
%<*driver>
%\ProvidesFile{childdoc.drv}[2018/12/30 v2.0 childdoc reference manual file]
\PassOptionsToClass{10pt,a4paper}{article}
\documentclass{ltxdoc}

\usepackage[margin=35mm]{geometry}
\usepackage{hyperref}
\usepackage{hyperxmp}
\usepackage[usenames]{color}

\hypersetup{colorlinks=true}
\hypersetup{pdfstartview=FitH}
\hypersetup{pdfpagemode=UseNone}
\hypersetup{pdfsource={}}
\hypersetup{pdflang={en-UK}}
\hypersetup{pdfcopyright={Copyright 2017-2018 Niklas Beisert.
  This work may be distributed and/or modified under the
  conditions of the LaTeX Project Public License, either version 1.3
  of this license or (at your option) any later version.}}
\hypersetup{pdflicenseurl={http://www.latex-project.org/lppl.txt}}
\hypersetup{pdfcontactaddress={ETH Zurich, ITP, HIT K,
  Wolfgang-Pauli-Strasse 27}}
\hypersetup{pdfcontactpostcode={8093}}
\hypersetup{pdfcontactcity={Zurich}}
\hypersetup{pdfcontactcountry={Switzerland}}
\hypersetup{pdfcontactemail={nbeisert@itp.phys.ethz.ch}}
\hypersetup{pdfcontacturl={http://people.phys.ethz.ch/\xmptilde nbeisert/}}

\newcommand{\secref}[1]{\hyperref[#1]{section \ref*{#1}}}

\parskip1ex
\parindent0pt
\let\olditemize\itemize
\def\itemize{\olditemize\parskip0pt}

\begin{document}

\title{The \textsf{childdoc} Package}
\hypersetup{pdftitle={The childdoc Package}}
\author{Niklas Beisert\\[2ex]
  Institut f\"ur Theoretische Physik\\
  Eidgen\"ossische Technische Hochschule Z\"urich\\
  Wolfgang-Pauli-Strasse 27, 8093 Z\"urich, Switzerland\\[1ex]
  \href{mailto:nbeisert@itp.phys.ethz.ch}
  {\texttt{nbeisert@itp.phys.ethz.ch}}}
\hypersetup{pdfauthor={Niklas Beisert}}
\hypersetup{pdfsubject={Manual for the LaTeX2e Package childdoc}}
\date{30 December 2018, \textsf{v2.0}}
\maketitle

\begin{abstract}\noindent
\textsf{childdoc} is a \LaTeXe{} package
that enables the direct compilation
of document sections included by |\include|
to individual files.
\end{abstract}

\begingroup
\parskip0ex
\tableofcontents
\endgroup

%%%%%%%%%%%%%%%%%%%%%%%%%%%%%%%%%%%%%%%%%%%%%%%%%%%%%%%%%%%%%%%%%%%%%%%%%%%%%%%%
%%%%%%%%%%%%%%%%%%%%%%%%%%%%%%%%%%%%%%%%%%%%%%%%%%%%%%%%%%%%%%%%%%%%%%%%%%%%%%%%
\section{Introduction}

\LaTeX{} provides a mechanism to structure a large document (such as a book)
into a main file and several child files (containing the chapters)
using the |\include| command.
This mechanism is beneficial for documents
which span hundreds of pages in order to
make the source file(s) more manageable.
Moreover, compilation can be restricted to
selected child files by means of the |\includeonly| command.
The latter feature can be used to reduce the compilation time while editing
(this was significantly more useful in the earlier days of \LaTeX{})
or to generate a smaller document which is easier to navigate.
Another application of |\includeonly| is to generate
documents consisting of selected parts of the complete document.

However, there are a few drawbacks of the plain |\include| mechanism:
\begin{itemize}
\item
The child files cannot be compiled on their own,
they can only be compiled via the main file.
A naive editing environment
(such as a text editor with an option
to have the current file processed by \LaTeX)
may require one to switch to the main file before compiling;
attempting to compile the child file produces errors.
\item
The main file must be modified (each time)
to adjust the |\includeonly| command
to the present needs. This easily leaves the main file in a messy state.
\item
The generated document will always carry the filename
of the main document. This is inconvenient if
several child files are to be compiled and
to be kept for distribution.
\end{itemize}

The present package provides a simple interface
to make child files individually compilable by \LaTeX{}.
Compiling a child file then has the same effect as compiling
the main file with an |\includeonly| command
to select the appropriate child.
Moreover the generated document will carry the name of the child
rather than the main file.
This resolves all three above issues.

This feature is meant to make the editing of books,
thesis documents and lecture notes somewhat more convenient.
However, the package can also be used efficiently for
composing a series of documents (such as exercise sheets)
which are typically distributed individually.
It then assists the author in generating the individual documents
(potentially in different versions)
as well as a document containing the collected series.
Another application is in developing style files
or other kinds of included material
where compilation of the style file could redirect
to a sample or test file.

%%%%%%%%%%%%%%%%%%%%%%%%%%%%%%%%%%%%%%%%%%%%%%%%%%%%%%%%%%%%%%%%%%%%%%%%%%%%%%%%
%%%%%%%%%%%%%%%%%%%%%%%%%%%%%%%%%%%%%%%%%%%%%%%%%%%%%%%%%%%%%%%%%%%%%%%%%%%%%%%%
\section{Usage}

First of all, the package \textsf{childdoc} is \emph{not} a standard
\LaTeXe{} |.sty| style file! Therefore it needs to be invoked in
a non-standard way.

%%%%%%%%%%%%%%%%%%%%%%%%%%%%%%%%%%%%%%%%%%%%%%%%%%%%%%%%%%%%%%%%%%%%%%%%%%%%%%%%
\subsection{Included Files}
\label{sec:include}

%%%%%%%%%%%%%%%%%%%%%%%%%%%%%%%%%%%%%%%%
\DescribeMacro{\childdocmain}
To use the package, add the commands
\begin{center}
\begin{tabular}{l}
|\input{childdoc.def}|\\
|\childdocmain{}|\\
\end{tabular}
\end{center}
at the very top of the main \LaTeX{} file,
in particular \emph{before} the |\documentclass| statement!
The argument of |\childdocmain| should be left empty
(but it must be present).

%%%%%%%%%%%%%%%%%%%%%%%%%%%%%%%%%%%%%%%%
\DescribeMacro{\childdocof}
Furthermore, add the commands
\begin{center}
\begin{tabular}{l}
|\input{childdoc.def}|\\
|\childdocof{|\textit{main}|}|\\
\end{tabular}
\end{center}
at the top of every child file \textit{child}
which is included by |\include{|\textit{child}|}|
from within the main file
(or at least for those files to be compiled individually).
The argument \textit{main} must be the filename of the main file.

There are a couple of
considerations in setting up the main and child documents:

%%%%%%%%%%%%%%%%%%%%%%%%%%%%%%%%%%%%%%%%
\paragraph{Restrictions.}

Please note the following restrictions:
\begin{itemize}
\item
|\childdocmain| must be called with one argument \textit{main}
to ensure compatibility with earlier version of the package.
It must either be empty (|\childdocmain{}|)
or precisely match the filename of the main file in which it is specified.
See \secref{sec:detection} for further information.
\item
The filename \textit{main} must be specified without the |.tex| extension.
\item
The filename \textit{main} is case sensitive
(even in case-insensitive file systems)
due to internal string comparison.
\item
The argument \textit{main} should be fully expanded, it cannot be a macro.
\item
Subdirectories and special characters should be avoided in filenames.
\item
The command |\childdocmain{|\textit{main}|}| must be followed by a whitespace.
It should not be followed immediately by another command
or by a comment mark `|%|'.
This is because the \TeX{} parser reads the token immediately following
the argument of |\childdocmain| and puts it
at the beginning of every child section;
however, a white\-space is ignored.
\end{itemize}

%%%%%%%%%%%%%%%%%%%%%%%%%%%%%%%%%%%%%%%%
\paragraph{Content of Main File.}

It is advisable to place all content in the child files included by |\include|.
Any output contained in the main file will appear in all child documents
unless suppressed manually;
it cannot be suppressed automatically by the |\includeonly| directive
and thus should normally be avoided.
A method to include some content in the main file
by means of conditional processing is described in \secref{sec:conditional}.

%%%%%%%%%%%%%%%%%%%%%%%%%%%%%%%%%%%%%%%%
\paragraph{Page Numbering.}

When only a part of the document is compiled,
the appropriate numbering of pages
(as well as other status parameters)
is determined from the |.aux| files.
The latter contain information from previous passes.
However this information needs to propagate through
all intermediate child documents.
Therefore the page numbering in child documents may well
be inconsistent until the complete document is compiled at least once.

A useful (if unconventional) way to always ensure a consistent
page numbering is to restart the numbering in each child document
and denote the pages by `\textit{child}|.|\textit{page}'
where \textit{child} represents the chapter/section number of the child file.
This can be achieved by the command
|\numberwithin{page}{|\textit{child}|}|
of the \textsf{amsmath} package
where \textit{child} can be |chapter| or |section|
depending on the chosen structuring.
Alternatively, one can modify the macro |\thepage| appropriately
and reset the counter |page| at the start of each child file.

%%%%%%%%%%%%%%%%%%%%%%%%%%%%%%%%%%%%%%%%%%%%%%%%%%%%%%%%%%%%%%%%%%%%%%%%%%%%%%%%
\subsection{Conditional Processing}
\label{sec:conditional}

The package provides a mechanism to compile different versions
of a document. To customise the versions further some conditional processing
can come in handy to distinguish which version is being compiled.
The package provides two macros to describe the compilation context:

%%%%%%%%%%%%%%%%%%%%%%%%%%%%%%%%%%%%%%%%
\DescribeMacro{\ifchilddoc}
The conditional |\ifchilddoc| distinguishes between the compilation of
child documents and the main document:
%
\begin{center}
|\ifchilddoc |\textit{child-code}| |[|\||else |\textit{main-code}]| \||fi|
\end{center}

%%%%%%%%%%%%%%%%%%%%%%%%%%%%%%%%%%%%%%%%
\DescribeMacro{\childdocname}
\DescribeMacro{\childdocjob}
The macro |\childdocname| contains the filename (without extension)
of the main or child file being processed.
Note that |\childdocjob| will always contain the name of the main file.

%%%%%%%%%%%%%%%%%%%%%%%%%%%%%%%%%%%%%%%%
\paragraph{Title Page.}

Conditional processing can be used to include a title or banner page
in the main document when proper precautions are taken.
Importantly, the code in the main file should ensure that the page counter
(as well as other status parameters which are stored in the |.aux| files)
takes the same value after the conditional processing.
Otherwise the page numbers may take divergent values
depending on which part is compiled.

For example, a title page could be declared by:
%
\begin{center}
\begin{tabular}{l}
|\ifchilddoc\||else|\\
|\addtocounter{page}{-1}|\\
\textit{code for title page}\\
|\newpage|\\
|\||fi|
\end{tabular}
\end{center}
%
A banner page for the child documents can be generated by:
%
\begin{center}
\begin{tabular}{l}
|\ifchilddoc|\\
|\addtocounter{page}{-1}|\\
\textit{code for banner page}\\
|\newpage|\\
|\||fi|
\end{tabular}
\end{center}
%
Here one could write a message such as:
\begin{center}
|This is the part \childdocname{} of \childdocjob{}.|
\end{center}

%%%%%%%%%%%%%%%%%%%%%%%%%%%%%%%%%%%%%%%%%%%%%%%%%%%%%%%%%%%%%%%%%%%%%%%%%%%%%%%%
\subsection{Flags}
\label{sec:flags}

The package makes it easy to generate different versions
of the main or child documents.
To this end compilation flags can be defined
and assigned different default values.
They will be particularly useful in conjunction
with the forwarding mechanism described in \secref{sec:forward}.

For example, it may be useful to have a flag |\version|
which can be set to |draft| or |final|.
The document source will contain some conditional code
depending on the value of |\version|.
Suppose further, the flag should default to |final| for the main file
and to |draft| for child files
which is a natural assignment for editing the document.
This is achieved by placing the following code
in the preamble of the main document
(below the |\childdocmain| directive):
%
\begin{center}
\begin{tabular}{l}
|\ifchilddoc|\\
|\providecommand{\version}{draft}|\\
|\||else|\\
|\providecommand{\version}{final}|\\
|\||fi|
\end{tabular}
\end{center}
%
The definition by |\providecommand| makes sure
that previous definitions are not overwritten.
Further statements |\providecommand{\version}{...}|
can thus be added before the above code to override it.

For the main file, one might add a line
(between |\childdocmain| and the above block)
%
\begin{center}
|%\ifchilddoc\||else\providecommand{\version}{draft}\||fi|
\end{center}
%
which can be uncommented to produce a draft version.
Likewise one can add a line to the very top of a child file
(above the |\childdocof{|\textit{main}|}| directive)
%
\begin{center}
|%\providecommand{\version}{final}|
\end{center}
%
which can be uncommented to produce the final version of this child document.

%%%%%%%%%%%%%%%%%%%%%%%%%%%%%%%%%%%%%%%%%%%%%%%%%%%%%%%%%%%%%%%%%%%%%%%%%%%%%%%%
\subsection{Forwarding}
\label{sec:forward}

Different versions of the main or child documents
using compilation flags as described in \secref{sec:flags}
can be (permanently) stored in different files
for convenient compilation, viewing and distribution.
To this end, the package defines a command
to pass on compilation to a different file:

%%%%%%%%%%%%%%%%%%%%%%%%%%%%%%%%%%%%%%%%
\DescribeMacro{\childdocforward}
The command |\childdocforward| redirects processing to
another source file:
%
\begin{center}
\begin{tabular}{l}
|\input{childdoc.def}|\\
|\childdocforward[|\textit{main}|]{|\textit{dest}|}|\\
\end{tabular}
\end{center}
%
The argument \textit{dest} is the destination file
(without extension).
It should be the main file or one of the child files.
Note that further \textsf{childdoc} directives
such as |\childdocof| and |\childdocforward|
in the indicated file will be processed in this form.
The optional argument \textit{main}
passes on directly to the main file \textit{main}
while pretending to compile the child \textit{dest}.
This form behaves as if \textit{dest}
issues |\childdocof{|\textit{main}|}| right away,
and no further \textsf{childdoc} directives will be processed.

%%%%%%%%%%%%%%%%%%%%%%%%%%%%%%%%%%%%%%%%
\DescribeMacro{\...prefix}
In the alternative form |\childdocforwardprefix|,
%
\begin{center}
\begin{tabular}{l}
|\input{childdoc.def}|\\
|\childdocforwardprefix[|\textit{main}|]{|\textit{prefix}|}{|\textit{dest}|}|
\end{tabular}
\end{center}
%
the destination file is determined by a pattern
depending on the current file:
To make this work, the current file must be called
`{\textit{prefix}\hspace{0.2em}\textit{suffix}}'
with \textit{prefix} matching precisely the argument.
Processing is then passed on to the file
`{\textit{dest}\hspace{0.2em}\textit{suffix}}'.
Surely, the same effect is achieved by
directly specifying the
argument `{\textit{dest}\hspace{0.2em}\textit{suffix}}'
in the first form.
However, that requires to set up a different file
for each child. With the alternative form of the command
all these files can have exactly the same content
which simplifies setting them up and maintaining them.

For example, the following file |draft.tex|
with a compilation flag |\version| as described in \secref{sec:flags}
compiles the main document as a draft:
%
\begin{center}
\begin{tabular}{l}
|\def\version{draft}|\\
|\input{childdoc.def}|\\
|\childdocforward{|\textit{main}|}|
\end{tabular}
\end{center}
%
Likewise, the following files |final|\textit{nn}|.tex|
compile the final version of the child document
|child|\textit{nn}|.tex|:
%
\begin{center}
\begin{tabular}{l}
|\def\version{final}|\\
|\input{childdoc.def}|\\
|\childdocforwardprefix{final}{child}|
\end{tabular}
\end{center}
%

Note that when several versions of a main file and/or of each child file
are to be generated, it may be convenient to set up a |Makefile| or
shell script to automatise the process.

%%%%%%%%%%%%%%%%%%%%%%%%%%%%%%%%%%%%%%%%%%%%%%%%%%%%%%%%%%%%%%%%%%%%%%%%%%%%%%%%
\subsection{Command Line Processing}
\label{sec:commandline}

The effect of redirection files can also be achieved by invoking
the \LaTeX{} compiler with a more elaborate command line.
Most conveniently this should be done as part
of a shell script or a |Makefile|.

When using \textsf{childdoc} in the main file, the following
command lines effectively perform a redirection
(note that depending on the shell being used,
backslashes may have to be doubled: `|\|' $\to$ `|\\|'):
%
\begin{center}
|... -jobname "|\textit{target}|" |\\|"|[\textit{flags}]%
|\input{childdoc.def}\childdocforward[|\textit{main}|]{|\textit{dest}|}"|
\end{center}
%
Here \textit{target} is the name of the output file,
\textit{main} is the name of the main file
and \textit{dest} is the name of the main or child file to be processed
(all filenames without extensions).
The optional argument \textit{main} can be omitted
if \textit{main} matches \textit{dest}.
Optionally, compilation \textit{flags} can be defined via |\def| commands.
This command line makes the \TeX{} engine believe
it is compiling the file \textit{target}
whose content is specified as the latter parameter.
The provided code then forwards the processing to
\textit{main} or \textit{dest} as described in \secref{sec:forward}.

%%%%%%%%%%%%%%%%%%%%%%%%%%%%%%%%%%%%%%%%%%%%%%%%%%%%%%%%%%%%%%%%%%%%%%%%%%%%%%%%
\subsection{Include by Input}
\label{sec:input}

Including child documents by |\include| has some restrictions by design.
Most notably, the content of a child document always occupies
its own set of pages; pages cannot be shared between child documents.
Usually, this behaviour makes perfect sense
because each child document contain an essential part of the document.
However, in some situations it may be desirable to compose
a document from a collection of parts
without having mandatory page breaks between then.
For this case, the package
provides a mechanism to include parts
by |\input| which can also be processed individually.
However, by construction this mechanism
requires manual handling of the content to be output.

%%%%%%%%%%%%%%%%%%%%%%%%%%%%%%%%%%%%%%%%
\DescribeMacro{\ifchilddocmanual}
The main file should be prepared as usual, see \secref{sec:include}.
However, the document body must make a distinction
between processing of an individual part and of the main document, e.g.:
%
\begin{center}
\begin{tabular}{l}
|\ifchilddocmanual|\\
|\input{\childdocname}|\\
|\||else|\\
\textit{document body with }|\input{|\textit{part}|}|\\
|\||fi|
\end{tabular}
\end{center}
%
The conditional |\ifchilddocmanual| is true whenever
a part to be included by |\input| is being compiled,
and the name of the part is stored in |\childdocname|.

%%%%%%%%%%%%%%%%%%%%%%%%%%%%%%%%%%%%%%%%
\DescribeMacro{\childdocby}
Each part to be included by |\input| should start with:
%
\begin{center}
\begin{tabular}{l}
|\input{childdoc.def}|\\
|\childdocby{|\textit{main}|}|\\
\end{tabular}
\end{center}
%
The directive |\childdocby| is similar to |\childdocof|
described in \secref{sec:include},
but the subsequent selection of content must be done manually.
To that end, both |\ifchilddoc| and |\ifchilddocmanual|
will be true upon processing of a part,
and the name of the part is stored in |\childdocname|.
Note that |\jobname| will be set to the filename of the current part
so that each part receives an individual |.aux| file
that does not interfere with the |.aux| file(s) of the main document.
This behaviour can be altered by the alternative form
|\childdocby[*]{|\textit{main}|}| (with a non-empty optional argument)
which uses the |.aux| file of the main document
by setting |\jobname| to \textit{main}.

%%%%%%%%%%%%%%%%%%%%%%%%%%%%%%%%%%%%%%%%%%%%%%%%%%%%%%%%%%%%%%%%%%%%%%%%%%%%%%%%
\subsection{Driver Development}
\label{sec:driver}

The \textsf{childdoc} mechanism can also be use for the development
of definition files such as \LaTeX{} styles or classes.
This case differs from the above setup with multiple parts
included by |\include| in that no |\includeonly| should be invoked.
This can be achieved by starting the include file
(before |\ProvidesPackage|) with:
%
\begin{center}
\begin{tabular}{l}
|\input{childdoc.def}|\\
|\childdocforward{|\textit{main}|}|\\
\end{tabular}
\end{center}
%
or alternatively with:
%
\begin{center}
\begin{tabular}{l}
|\input{childdoc.def}|\\
|\childdocby{|\textit{main}|}|\\
\end{tabular}
\end{center}
%
Both forms have slightly different effects as described above.
The main file is prepared as usual, see \secref{sec:include}.

%%%%%%%%%%%%%%%%%%%%%%%%%%%%%%%%%%%%%%%%%%%%%%%%%%%%%%%%%%%%%%%%%%%%%%%%%%%%%%%%
\subsection{Legacy Detection}
\label{sec:detection}

The directive |\childdocmain| in the main file can detect
whether the complete document or merely a child is to be compiled
even without using the directive |\childdocof|.
This method is deprecated because it is less robust
and there is no compelling reason to use it;
it is merely provided for backward compatibility
and it may be removed in future versions.

If the detection mechanism is to be used,
it is mandatory to correctly specify
the filename of the main file as the argument of |\childdocmain|:
%
\begin{center}
\begin{tabular}{l}
|\input{childdoc.def}|\\
|\childdocmain{|\textit{main}|}|\\
\end{tabular}
\end{center}
%
If |\jobname| does not match the argument \textit{main} of |\childdocmain|,
it is assumed that |\jobname| points to the child file to be compiled.
When using |\childdocmain| with the main file specified as argument,
it suffices to start a child file
with just |\input{|\textit{main}|}|
without loading of the package and using |\childdocof|.
If instead all processing is done
with the appropriate \textsf{childdoc} directives,
the argument of \textit{main} of |\childdocmain| can be empty.

An alternative version of the command line processing described
in \secref{sec:commandline} using the detection mechanism reads:
%
\begin{center}
|... -jobname "|\textit{target}|" "|[\textit{flags}]%
[|\def\jobname{|\textit{dest}|}|]|\input{|\textit{main}|}"|
\end{center}

%%%%%%%%%%%%%%%%%%%%%%%%%%%%%%%%%%%%%%%%%%%%%%%%%%%%%%%%%%%%%%%%%%%%%%%%%%%%%%%%
\subsection{Manual Code}
\label{sec:manual}

In case one cannot be certain whether the definitions file |childdoc.def|
is installed on the target \TeX{} distribution
and one prefers not to ship it,
it is conceivable to paste a few relevant commands into the sources.

To that end, drop all statements |\input{childdoc.def}|
and perform the replacements as outlined below.
Instead of |\childdocmain{|\textit{main}|}| add the following code
to the top of the main file:
%
\begin{center}
\begin{tabular}{l}
|\||ifdefined\childdocname\endinput\||fi\newif\ifchilddoc|\\
|\edef\childdocname{\scantokens\expandafter{\jobname\noexpand}}|\\
|\def\childdocmain{|\textit{main}|}\||ifx\childdocmain\childdocname\||else|\\
|\childdoctrue\includeonly{\childdocname}\let\jobname\childdocmain\||fi|\\
\end{tabular}
\end{center}
%
Instead of |\childdocof{|\textit{main}|}| just include the main file
at the top of each child file:
%
\begin{center}
|\input{|\textit{main}|}|
\end{center}
%
A simple redirection |\childdocforward{|\textit{dest}|}| is achieved by:
%
\begin{center}
|\def\jobname{|\textit{dest}|}\input{\jobname}|
\end{center}
%
The redirection with prefix
|\childdocforwardprefix[|\textit{prefix}|]{|\textit{dest}|}|
is accomplished by:
%
\begin{center}
\begin{tabular}{l}
|{\edef\jobname{\scantokens\expandafter{\jobname\noexpand}}|\\
|\def\redirectjob |\textit{prefix}|#1~~~{\gdef\jobname{|\textit{dest}|#1}}|\\
|\expandafter\redirectjob\jobname~~~}\input{\jobname}|
\end{tabular}
\end{center}

In an alternative approach,
child documents can be compiled by a specific command line
without additional code or specific definitions:
%
\begin{center}
|... -jobname "|\textit{target}|" "|[\textit{flags}]%
|\includeonly{|\textit{dest}|}\input{|\textit{main}|}"|
\end{center}
%

%%%%%%%%%%%%%%%%%%%%%%%%%%%%%%%%%%%%%%%%%%%%%%%%%%%%%%%%%%%%%%%%%%%%%%%%%%%%%%%%
%%%%%%%%%%%%%%%%%%%%%%%%%%%%%%%%%%%%%%%%%%%%%%%%%%%%%%%%%%%%%%%%%%%%%%%%%%%%%%%%
\section{Information}

%%%%%%%%%%%%%%%%%%%%%%%%%%%%%%%%%%%%%%%%%%%%%%%%%%%%%%%%%%%%%%%%%%%%%%%%%%%%%%%%
\subsection{Copyright}

Copyright \copyright{} 2017--2018 Niklas Beisert

This work may be distributed and/or modified under the
conditions of the \LaTeX{} Project Public License, either version 1.3
of this license or (at your option) any later version.
The latest version of this license is in
  \url{http://www.latex-project.org/lppl.txt}
and version 1.3 or later is part of all distributions of \LaTeX{}
version 2005/12/01 or later.

This work has the LPPL maintenance status `maintained'.

The Current Maintainer of this work is Niklas Beisert.

This work consists of the files |README.txt|, |childdoc.ins| and |childdoc.dtx|
as well as the derived files |childdoc.def|, |cdocsamp.tex|
with |cdocsch1.tex|, |cdocsch2.tex|, |cdocspt3.tex|, |cdocspt4.tex|,
|cdocsdrf.tex|, |cdocsfn1.tex|, |cdocsfn2.tex|
as well as |childdoc.pdf|.

%%%%%%%%%%%%%%%%%%%%%%%%%%%%%%%%%%%%%%%%%%%%%%%%%%%%%%%%%%%%%%%%%%%%%%%%%%%%%%%%
\subsection{Files and Installation}

The package consists of the files:
%
\begin{center}
\begin{tabular}{ll}
    |README.txt|   & readme file \\
    |childdoc.ins| & installation file \\
    |childdoc.dtx| & source file \\
    |childdoc.def| & definition file \\
    |cdocsamp.tex| & sample main file \\
    |cdocsch1.tex| & sample include file \\
    |cdocsch2.tex| & sample include file \\
    |cdocspt3.tex| & sample part file \\
    |cdocspt4.tex| & sample part file \\
    |cdocsdrf.tex| & sample redirection file \\
    |cdocsfn1.tex| & sample redirection file \\
    |cdocsfn2.tex| & sample redirection file \\
    |childdoc.pdf| & manual
\end{tabular}
\end{center}
%
The distribution consists of the files
|README.txt|, |childdoc.ins| and |childdoc.dtx|.
%
\begin{itemize}
\item
Run (pdf)\LaTeX{} on |childdoc.dtx|
to compile the manual |childdoc.pdf| (this file).
\item
Run \LaTeX{} on |childdoc.ins| to create the definitions file |childdoc.def|
and the sample |cdocsamp.tex| with include files
|cdocsch1.tex|, |cdocsch2.tex|, |cdocspt3.tex|, |cdocspt4.tex|,
|cdocsdrf.tex|, |cdocsfn1.tex|, |cdocsfn2.tex|.
Then copy the file |childdoc.def| to an appropriate directory of your \LaTeX{}
distribution, e.g.\ \textit{texmf-root}|/tex/latex/childdoc|.
\end{itemize}

%%%%%%%%%%%%%%%%%%%%%%%%%%%%%%%%%%%%%%%%%%%%%%%%%%%%%%%%%%%%%%%%%%%%%%%%%%%%%%%%
\subsection{Related CTAN Packages}

There are several other packages which offer a similar functionality:
%
\begin{itemize}
\item
The packages
\href{http://ctan.org/pkg/docmute}{\textsf{docmute}},
\href{http://ctan.org/pkg/includex}{\textsf{includex}} and
\href{http://ctan.org/pkg/standalone}{\textsf{standalone}}
provide commands to include only the document body of
a child file thus allowing both files to be compiled individually.
\item
The packages \href{http://ctan.org/pkg/subdocs}{\textsf{subdocs}}
and \href{http://ctan.org/pkg/subfiles}{\textsf{subfiles}}
provide structures in which the main and child documents can be
encapsulated and allowing them to be compiled individually.
The inclusion mechanism is different from the conventional |\include|.
\item
The package \href{http://ctan.org/pkg/combine}{\textsf{combine}}
is an elaborate solution to combine several documents into one.
\end{itemize}
%
See also the CTAN topic \href{http://ctan.org/topic/subdocs}{\textsf{subdocs}}
for further related packages.
The present package differs from the above solutions in that
a document structure constructed with the conventional |\include| mechanism
just needs two extra commands at the top of every file
such that all constituent files can be compiled individually.

%%%%%%%%%%%%%%%%%%%%%%%%%%%%%%%%%%%%%%%%%%%%%%%%%%%%%%%%%%%%%%%%%%%%%%%%%%%%%%%%
%\subsection{Feature Suggestions}
%
%The following is a list of features which may be useful for future
%versions of this package:
%%
%\begin{itemize}
%\item
%\ldots
%\end{itemize}

%%%%%%%%%%%%%%%%%%%%%%%%%%%%%%%%%%%%%%%%%%%%%%%%%%%%%%%%%%%%%%%%%%%%%%%%%%%%%%%%
\subsection{Revision History}

%%%%%%%%%%%%%%%%%%%%%%%%%%%%%%%%%%%%%%%%
\paragraph{v2.0:} 2018/12/30

\begin{itemize}
\item
immediate forward processing
\item
added |\childdocby| mechanism
\item
manual restructured
\end{itemize}

%%%%%%%%%%%%%%%%%%%%%%%%%%%%%%%%%%%%%%%%
\paragraph{v1.6:} 2018/01/17

\begin{itemize}
\item
application for development of include files
\item
corrections to manual
\end{itemize}

%%%%%%%%%%%%%%%%%%%%%%%%%%%%%%%%%%%%%%%%
\paragraph{v1.5:} 2017/05/21

\begin{itemize}
\item
more complete structuring introduced
\item
|\childdocof| introduced
\item
|\childdoc| renamed to |\childdocmain|
\item
|\childredirect| renamed to |\childdocforward| and |\childdocforwardprefix|
and functionality expanded
\end{itemize}

%%%%%%%%%%%%%%%%%%%%%%%%%%%%%%%%%%%%%%%%
\paragraph{v1.0:} 2017/04/27

\begin{itemize}
\item
manual and install package
\item
first version published on CTAN
\end{itemize}

%%%%%%%%%%%%%%%%%%%%%%%%%%%%%%%%%%%%%%%%
\paragraph{v0.6:} 2017/04/26

\begin{itemize}
\item
redirection mechanism added
\end{itemize}

%%%%%%%%%%%%%%%%%%%%%%%%%%%%%%%%%%%%%%%%
\paragraph{v0.5:} 2017/04/26

\begin{itemize}
\item
functionality in definition file
\end{itemize}


%%%%%%%%%%%%%%%%%%%%%%%%%%%%%%%%%%%%%%%%%%%%%%%%%%%%%%%%%%%%%%%%%%%%%%%%%%%%%%%%
%%%%%%%%%%%%%%%%%%%%%%%%%%%%%%%%%%%%%%%%%%%%%%%%%%%%%%%%%%%%%%%%%%%%%%%%%%%%%%%%
%%%%%%%%%%%%%%%%%%%%%%%%%%%%%%%%%%%%%%%%%%%%%%%%%%%%%%%%%%%%%%%%%%%%%%%%%%%%%%%%
\appendix

\settowidth\MacroIndent{\rmfamily\scriptsize 000\ }

 \DocInput{childdoc.dtx}

\end{document}
%</driver>
% \fi
%
% %%%%%%%%%%%%%%%%%%%%%%%%%%%%%%%%%%%%%%%%%%%%%%%%%%%%%%%%%%%%%%%%%%%%%%%%%%%%%%
% %%%%%%%%%%%%%%%%%%%%%%%%%%%%%%%%%%%%%%%%%%%%%%%%%%%%%%%%%%%%%%%%%%%%%%%%%%%%%%
% \section{Sample}
%\iffalse
%<*samplemain>
%\fi
%
% The following presents a sample document
% with two chapters, two parts, a title page,
% a compile flag as well as three forwarding files to set the flag.
% It consists of eight |.tex| files:
% \begin{center}
% \begin{tabular}{ll}
% |cdocsamp.tex|&main file\\
% |cdocsch1.tex|&include file for chapter 1\\
% |cdocsch2.tex|&include file for chapter 2\\
% |cdocspt3.tex|&include file for part 3\\
% |cdocspt4.tex|&include file for part 4\\
% |cdocsdrf.tex|&forwarding file for main file in draft mode\\
% |cdocsfi1.tex|&forwarding file for final version of chapter 1\\
% |cdocsfi2.tex|&forwarding file for final version of chapter 2\\
% \end{tabular}
% \end{center}
% Each of the eight files can be compiled directly by the \LaTeX{} compiler.
%
% %%%%%%%%%%%%%%%%%%%%%%%%%%%%%%%%%%%%%%
% \paragraph{Main File.}
%
% The main file is called |cdocsamp.tex|.
%
% Load the \textsf{childdoc} definitions and
% declare the filename for the main document:
%    \begin{macrocode}
\input{childdoc.def}
\childdocmain{}
%    \end{macrocode}

% Optional override for |\version| flag:
%    \begin{macrocode}
%%\ifchilddoc\else\providecommand{\version}{draft}\fi
%    \end{macrocode}

% Define the default values for the |\version| flag
% (|final| for the main file and |draft| for childs):
%    \begin{macrocode}
\ifchilddoc
\providecommand{\version}{draft}
\else
\providecommand{\version}{final}
\fi
%    \end{macrocode}

% Load the standard document class:
%    \begin{macrocode}
\documentclass[12pt]{article}
%    \end{macrocode}

% Start the document body:
%    \begin{macrocode}
\begin{document}
%    \end{macrocode}

% Declare a title page.
% Print title, part of document being processed and version flag:
%    \begin{macrocode}
\addtocounter{page}{-1}
\begin{center}
{\LARGE\bfseries{}childdoc example\par}
\vspace{1cm}
\ifchilddoc
\ifchilddocmanual part\else chapter\fi:
`\childdocname' of `\childdocjob'\par
\else
main document: `\childdocjob'\par
\fi
version: \version\par
\end{center}
\newpage
%    \end{macrocode}

% Manually include selected file,
% otherwise process as usual:
%    \begin{macrocode}
\ifchilddocmanual
\section*{part `\childdocname'}
\input{\childdocname}
\else
%    \end{macrocode}

% Include the two chapters:
%    \begin{macrocode}
\include{cdocsch1}
\include{cdocsch2}
%    \end{macrocode}

% Include the two parts unless only chapters should be displayed:
%    \begin{macrocode}
\ifchilddoc\else
\section{part three}
\input{cdocspt3}
\section{part four}
\input{cdocspt4}
\fi
%    \end{macrocode}

% Process as usual until here:
%    \begin{macrocode}
\fi
%    \end{macrocode}

% End of document body:
%    \begin{macrocode}
\end{document}
%    \end{macrocode}
%\iffalse
%</samplemain>
%\fi
%
% %%%%%%%%%%%%%%%%%%%%%%%%%%%%%%%%%%%%%%
% \paragraph{Chapter Include Files.}
%
% The include files are called |cdocsch1.tex| and |cdocsch2.tex|.
%
%\iffalse
%<*samplechap1|samplechap2>
%\fi

% Optional override for |\version| flag:
%    \begin{macrocode}
%%\providecommand{\version}{final}
%    \end{macrocode}

% Include the main document:
%    \begin{macrocode}
\input{childdoc.def}
\childdocof{cdocsamp}
%    \end{macrocode}

%\iffalse
%</samplechap1|samplechap2>
%\fi
%
%\iffalse
%<*samplechap1>
%\fi
% Some text for chapter 1:
%    \begin{macrocode}
\section{one}
some text in chapter one
%    \end{macrocode}

%\iffalse
%</samplechap1>
%\fi
% Some text for chapter 2:
%\iffalse
%<*samplechap2>
%\fi
%    \begin{macrocode}
\section{two}
more text in chapter two
%    \end{macrocode}

%\iffalse
%</samplechap2>
%\fi
%
% %%%%%%%%%%%%%%%%%%%%%%%%%%%%%%%%%%%%%%
% \paragraph{Part Include Files.}
%
% The include files are called |cdocspt3.tex| and |cdocspt4.tex|.
%
%\iffalse
%<*samplepart3|samplepart4>
%\fi

% Optional override for |\version| flag:
%    \begin{macrocode}
%%\providecommand{\version}{final}
%    \end{macrocode}

% Include the main document:
%    \begin{macrocode}
\input{childdoc.def}
\childdocby{cdocsamp}
%    \end{macrocode}

%\iffalse
%</samplepart3|samplepart4>
%\fi
%
%\iffalse
%<*samplepart3>
%\fi
% Some text for part 3:
%    \begin{macrocode}
some text in part three
%    \end{macrocode}

%\iffalse
%</samplepart3>
%\fi
% Some text for part 4:
%\iffalse
%<*samplepart4>
%\fi
%    \begin{macrocode}
more text in part four
%    \end{macrocode}

%\iffalse
%</samplepart4>
%\fi
%
% %%%%%%%%%%%%%%%%%%%%%%%%%%%%%%%%%%%%%%
% \paragraph{Forwarding for a Complete Draft.}
%
% The following forwarding file |cdocsdrf.tex|
% compiles the main document in draft mode:
%\iffalse
%<*sampledraft>
%\fi
%    \begin{macrocode}
\def\version{draft}
\input{childdoc.def}
\childdocforward{cdocsamp}
%    \end{macrocode}

%\iffalse
%</sampledraft>
%\fi
%
% %%%%%%%%%%%%%%%%%%%%%%%%%%%%%%%%%%%%%%
% \paragraph{Forwarding for Final Version of the Chapters.}
%
% The following forwarding files |cdocsfn1.tex| and |cdocsfn2.tex|
% (with identical content)
% compile the final versions of the child documents
% |cdocsch1.tex| and |cdocsch2.tex|, respectively:
%\iffalse
%<*samplefinal>
%\fi
%    \begin{macrocode}
\def\version{final}
\input{childdoc.def}
\childdocforwardprefix[cdocsamp]{cdocsfn}{cdocsch}
%    \end{macrocode}

%\iffalse
%</samplefinal>
%\fi
%
% %%%%%%%%%%%%%%%%%%%%%%%%%%%%%%%%%%%%%%
% \paragraph{Command Line Processing.}
%
% The following three command lines generate the output files
% |cdocscld|, |cdocscl1| and |cdocscl2|
% which should be identical to
% |cdocsdrf|, |cdocsch1| and |cdocsfn2|, respectively:
% \begin{center}
% \begin{tabular}{l}
% |latex -jobname cdocscld \|\\
% |  "\def\version{draft}\input{childdoc.def}\childdocforward{cdocsamp}"|\\
% |latex -jobname cdocscl1 \|\\
% |  "\input{childdoc.def}\childdocforward[cdocsamp]{cdocsch1}"|\\
% |latex -jobname cdocscl2 \|\\
% |  "\def\version{final}\input{childdoc.def}\childdocforward{cdocsch2}"|
% \end{tabular}
% \end{center}
% Note that the trailing backslash on each first line
% merely continues the input to the second line
% (for convenient cut ant paste).
% Furthermore, the command |latex| can be replaced by any
% of its alternative versions such as |pdflatex|.
%
% %%%%%%%%%%%%%%%%%%%%%%%%%%%%%%%%%%%%%%%%%%%%%%%%%%%%%%%%%%%%%%%%%%%%%%%%%%%%%%
% %%%%%%%%%%%%%%%%%%%%%%%%%%%%%%%%%%%%%%%%%%%%%%%%%%%%%%%%%%%%%%%%%%%%%%%%%%%%%%
% \section{Implementation}
%\iffalse
%<*package>
%\fi
%
% This section describes the definitions file |childdoc.def|.

% The definitions cannot be loaded using |\usepackage| or |\RequirePackage|
% which has a mechanism to prevent loading a style file more than once.
% When loading the definitions by means of |\input|
% multiple instances have to be prevented manually:
%\iffalse
%This code needs to be before the `\ProvidesFile' directive
%which is defined at the beginning of this file.
%Therefore it is also placed there and commented out here.
%</package>
%<*discard>
%\fi
%    \begin{macrocode}
\ifdefined\childdocmain\endinput\fi
%    \end{macrocode}
%\iffalse
%</discard>
%<*package>
%\fi
%
% \macro{\ifchilddoc}
% \macro{\ifchilddocmanual}
% The conditional |\ifchilddoc| tells whether a
% child (true) or main (false) document is being compiled.
% The conditional |\ifchilddocmanual| tells whether
% the |\includeonly| mechanism is used (false) or
% the selection of child files must be performed manually (true).
% The definitions initialise to false:
%    \begin{macrocode}
\newif\ifchilddoc
\newif\ifchilddocmanual
%    \end{macrocode}

% \macro{\childdocname}
% \macro{\childdocjob}
% The macro |\childdocname| stores the name of the main document
% to be compiled. The macro |\childdocjob| stores the name of
% the document on which the \LaTeX{} compiler was originally invoked.
% The content of |\jobname| cannot be compared
% to filenames specified in the source due to different catcodes.
% The following code rescans |\jobname|, stores the result
% in |\childdocname| and saves a copy in |\childdocjob|:
%    \begin{macrocode}
\edef\childdocname{\scantokens\expandafter{\jobname\noexpand}}
\let\childdocjob\childdocname
%    \end{macrocode}

% \macro{\childdocdisable}
% The macro |\childdocdisable| prevents the main file
% from being processed more than once.
% At this stage, the main document command |\childdocmain|
% is assumed to be called once again where it should do nothing.
% Any subsequent call to it should prevent
% a secondary processing of the main document
% It overwrites the forwarding commands
% |\childdocof| and |\childdocforward|
% with empty macros to prevent further inclusions of the main document:
%    \begin{macrocode}
\newcommand{\childdocdisable}
{
  \renewcommand{\childdocmain}[1]{\renewcommand{\childdocmain}[1]{\endinput}}
  \renewcommand{\childdocof}[1]{}
  \renewcommand{\childdocby}[2][]{}
  \renewcommand{\childdocforward}[2][]{}
  \renewcommand{\childdocdisable}{}
}
%    \end{macrocode}

% \macro{\childdocmain}
% The macro |\childdocmain| is to be called at the top of the main file
% with nothing or the main filename (without extension) as argument.
% First, it breaks loops.
% If the argument is not empty and does not match |\childdocname|
% (which is set by the first inclusion of |childdoc.def|),
% |\ifchilddoc| is set to true, |\includeonly| is applied to the child file
% and |\jobname| is set to the main file
% (for proper handling of |.aux| files):
%    \begin{macrocode}
\newcommand{\childdocmain}[1]
{
  \childdocdisable\childdocmain{}
  \if?#1?\else
    \begingroup
      \def\childdoctmp{#1}
      \ifx\childdoctmp\childdocname
        \def\childdoctmp{}
      \else
        \def\childdoctmp
        {
          \childdoctrue
          \includeonly{\childdocname}
          \def\childdocjob{#1}
          \def\jobname{#1}
        }
      \fi
      \expandafter
    \endgroup
    \childdoctmp
  \fi
}
%    \end{macrocode}

% \macro{\childdocof}
% The command |\childdocof| redirects
% compilation to the main file |#1|.
%    \begin{macrocode}
\newcommand{\childdocof}[1]
{
  \childdocdisable
  \childdoctrue
  \includeonly{\childdocname}
  \def\jobname{#1}
  \def\childdocjob{#1}
  \input{#1}
}
%    \end{macrocode}

% \macro{\childdocby}
% The command |\childdocby| ....
%    \begin{macrocode}
\newcommand{\childdocby}[2][]
{
  \childdocdisable
  \childdoctrue
  \childdocmanualtrue
  \if?#1?\else
    \def\jobname{#2}
  \fi
  \def\childdocjob{#2}
  \input{#2}
  \endinput
}
%    \end{macrocode}

% \macro{\childdocforward}
% The command |\childdocforward| redirects
% compilation to the main file or
% (if the optional argument is given) a child file.
% Parameters are set as if the main file
% or a child file starting with |\childdocof| was compiled.
% Then compilation is handed over to the main file:
%    \begin{macrocode}
\newcommand{\childdocforward}[2][]
{
  \begingroup
    \if?#1?
      \def\childdoctmp
      {
        \def\childdocname{#2}
        \def\childdocjob{#2}
        \def\jobname{#2}
        \input{#2}
        \endinput
      }
    \else
      \def\childdoctmp
      {
        \childdocdisable
        \def\childdocname{#2}
        \childdoctrue
        \includeonly{#2}
        \def\childdocjob{#1}
        \def\jobname{#1}
        \input{#1}
        \endinput
      }
    \fi
    \expandafter
  \endgroup
  \childdoctmp
}
%    \end{macrocode}

% \macro{\childdocforwardprefix}
% The command |\childdocforwardprefix| redirects
% compilation to the main or a child file by means of a pattern.
% The prefix |#1| in the current filename is replaced by |#2|
% and the suffix of the current filename is kept
% (it is assumed that the filename does not contain the substring `|~~~|'
% which is used as a delimiter).
% Compilation is handed over to the new file by |\childdocforward|:
%    \begin{macrocode}
\newcommand{\childdocforwardprefix}[3][]
{
  \begingroup
    \def\childdocextract #2##1~~~{\def\childdoctmp{\childdocforward[#1]{#3##1}}}
    \expandafter\childdocextract\childdocname~~~
    \expandafter
  \endgroup
  \childdoctmp
}
%    \end{macrocode}

% \macro{\childdoc}
% The deprecated macro |\childdoc| is a legacy version of |\childdocmain|:
%    \begin{macrocode}
\newcommand{\childdoc}{\childdocmain}
%    \end{macrocode}

% \macro{\childdocredirect}
% The deprecated macro |\childdocredirect| is a legacy version
% of |\childdocforward| and |\childdocforwardprefix|:
%    \begin{macrocode}
\newcommand{\childdocredirect}[2][]
{
  \begingroup
    \if?#1?
      \def\childdoctmp{\childdocforward{#2}}
    \else
      \def\childdoctmp{\childdocforwardprefix{#1}{#2}}
    \fi
    \expandafter
  \endgroup
  \childdoctmp
}
%    \end{macrocode}

%\iffalse
%</package>
%\fi
%
\endinput
|\\
|\childdocforwardprefix[|\textit{main}|]{|\textit{prefix}|}{|\textit{dest}|}|
\end{tabular}
\end{center}
%
the destination file is determined by a pattern
depending on the current file:
To make this work, the current file must be called
`{\textit{prefix}\hspace{0.2em}\textit{suffix}}'
with \textit{prefix} matching precisely the argument.
Processing is then passed on to the file
`{\textit{dest}\hspace{0.2em}\textit{suffix}}'.
Surely, the same effect is achieved by
directly specifying the
argument `{\textit{dest}\hspace{0.2em}\textit{suffix}}'
in the first form.
However, that requires to set up a different file
for each child. With the alternative form of the command
all these files can have exactly the same content
which simplifies setting them up and maintaining them.

For example, the following file |draft.tex|
with a compilation flag |\version| as described in \secref{sec:flags}
compiles the main document as a draft:
%
\begin{center}
\begin{tabular}{l}
|\def\version{draft}|\\
|% \iffalse
%
% childdoc.dtx Copyright (C) 2017-2018 Niklas Beisert
%
% This work may be distributed and/or modified under the
% conditions of the LaTeX Project Public License, either version 1.3
% of this license or (at your option) any later version.
% The latest version of this license is in
%   http://www.latex-project.org/lppl.txt
% and version 1.3 or later is part of all distributions of LaTeX
% version 2005/12/01 or later.
%
% This work has the LPPL maintenance status `maintained'.
%
% The Current Maintainer of this work is Niklas Beisert.
%
% This work consists of the files childdoc.dtx and childdoc.ins
% and the derived files childdoc.def and cdocsamp.tex with
% cdocsch1.tex, cdocsch2.tex, cdocsdrf.tex, cdocsfn1.tex, cdocsfn2.tex.
%
%<package>\ifdefined\childdocmain\endinput\fi
%<package>\ProvidesFile{childdoc.def}[2018/12/30 v2.0 child document driver]
%<samplemain>\ProvidesFile{cdocsamp.tex}[2018/12/30 v2.0 sample for childdoc]
%<*driver>
%\ProvidesFile{childdoc.drv}[2018/12/30 v2.0 childdoc reference manual file]
\PassOptionsToClass{10pt,a4paper}{article}
\documentclass{ltxdoc}

\usepackage[margin=35mm]{geometry}
\usepackage{hyperref}
\usepackage{hyperxmp}
\usepackage[usenames]{color}

\hypersetup{colorlinks=true}
\hypersetup{pdfstartview=FitH}
\hypersetup{pdfpagemode=UseNone}
\hypersetup{pdfsource={}}
\hypersetup{pdflang={en-UK}}
\hypersetup{pdfcopyright={Copyright 2017-2018 Niklas Beisert.
  This work may be distributed and/or modified under the
  conditions of the LaTeX Project Public License, either version 1.3
  of this license or (at your option) any later version.}}
\hypersetup{pdflicenseurl={http://www.latex-project.org/lppl.txt}}
\hypersetup{pdfcontactaddress={ETH Zurich, ITP, HIT K,
  Wolfgang-Pauli-Strasse 27}}
\hypersetup{pdfcontactpostcode={8093}}
\hypersetup{pdfcontactcity={Zurich}}
\hypersetup{pdfcontactcountry={Switzerland}}
\hypersetup{pdfcontactemail={nbeisert@itp.phys.ethz.ch}}
\hypersetup{pdfcontacturl={http://people.phys.ethz.ch/\xmptilde nbeisert/}}

\newcommand{\secref}[1]{\hyperref[#1]{section \ref*{#1}}}

\parskip1ex
\parindent0pt
\let\olditemize\itemize
\def\itemize{\olditemize\parskip0pt}

\begin{document}

\title{The \textsf{childdoc} Package}
\hypersetup{pdftitle={The childdoc Package}}
\author{Niklas Beisert\\[2ex]
  Institut f\"ur Theoretische Physik\\
  Eidgen\"ossische Technische Hochschule Z\"urich\\
  Wolfgang-Pauli-Strasse 27, 8093 Z\"urich, Switzerland\\[1ex]
  \href{mailto:nbeisert@itp.phys.ethz.ch}
  {\texttt{nbeisert@itp.phys.ethz.ch}}}
\hypersetup{pdfauthor={Niklas Beisert}}
\hypersetup{pdfsubject={Manual for the LaTeX2e Package childdoc}}
\date{30 December 2018, \textsf{v2.0}}
\maketitle

\begin{abstract}\noindent
\textsf{childdoc} is a \LaTeXe{} package
that enables the direct compilation
of document sections included by |\include|
to individual files.
\end{abstract}

\begingroup
\parskip0ex
\tableofcontents
\endgroup

%%%%%%%%%%%%%%%%%%%%%%%%%%%%%%%%%%%%%%%%%%%%%%%%%%%%%%%%%%%%%%%%%%%%%%%%%%%%%%%%
%%%%%%%%%%%%%%%%%%%%%%%%%%%%%%%%%%%%%%%%%%%%%%%%%%%%%%%%%%%%%%%%%%%%%%%%%%%%%%%%
\section{Introduction}

\LaTeX{} provides a mechanism to structure a large document (such as a book)
into a main file and several child files (containing the chapters)
using the |\include| command.
This mechanism is beneficial for documents
which span hundreds of pages in order to
make the source file(s) more manageable.
Moreover, compilation can be restricted to
selected child files by means of the |\includeonly| command.
The latter feature can be used to reduce the compilation time while editing
(this was significantly more useful in the earlier days of \LaTeX{})
or to generate a smaller document which is easier to navigate.
Another application of |\includeonly| is to generate
documents consisting of selected parts of the complete document.

However, there are a few drawbacks of the plain |\include| mechanism:
\begin{itemize}
\item
The child files cannot be compiled on their own,
they can only be compiled via the main file.
A naive editing environment
(such as a text editor with an option
to have the current file processed by \LaTeX)
may require one to switch to the main file before compiling;
attempting to compile the child file produces errors.
\item
The main file must be modified (each time)
to adjust the |\includeonly| command
to the present needs. This easily leaves the main file in a messy state.
\item
The generated document will always carry the filename
of the main document. This is inconvenient if
several child files are to be compiled and
to be kept for distribution.
\end{itemize}

The present package provides a simple interface
to make child files individually compilable by \LaTeX{}.
Compiling a child file then has the same effect as compiling
the main file with an |\includeonly| command
to select the appropriate child.
Moreover the generated document will carry the name of the child
rather than the main file.
This resolves all three above issues.

This feature is meant to make the editing of books,
thesis documents and lecture notes somewhat more convenient.
However, the package can also be used efficiently for
composing a series of documents (such as exercise sheets)
which are typically distributed individually.
It then assists the author in generating the individual documents
(potentially in different versions)
as well as a document containing the collected series.
Another application is in developing style files
or other kinds of included material
where compilation of the style file could redirect
to a sample or test file.

%%%%%%%%%%%%%%%%%%%%%%%%%%%%%%%%%%%%%%%%%%%%%%%%%%%%%%%%%%%%%%%%%%%%%%%%%%%%%%%%
%%%%%%%%%%%%%%%%%%%%%%%%%%%%%%%%%%%%%%%%%%%%%%%%%%%%%%%%%%%%%%%%%%%%%%%%%%%%%%%%
\section{Usage}

First of all, the package \textsf{childdoc} is \emph{not} a standard
\LaTeXe{} |.sty| style file! Therefore it needs to be invoked in
a non-standard way.

%%%%%%%%%%%%%%%%%%%%%%%%%%%%%%%%%%%%%%%%%%%%%%%%%%%%%%%%%%%%%%%%%%%%%%%%%%%%%%%%
\subsection{Included Files}
\label{sec:include}

%%%%%%%%%%%%%%%%%%%%%%%%%%%%%%%%%%%%%%%%
\DescribeMacro{\childdocmain}
To use the package, add the commands
\begin{center}
\begin{tabular}{l}
|\input{childdoc.def}|\\
|\childdocmain{}|\\
\end{tabular}
\end{center}
at the very top of the main \LaTeX{} file,
in particular \emph{before} the |\documentclass| statement!
The argument of |\childdocmain| should be left empty
(but it must be present).

%%%%%%%%%%%%%%%%%%%%%%%%%%%%%%%%%%%%%%%%
\DescribeMacro{\childdocof}
Furthermore, add the commands
\begin{center}
\begin{tabular}{l}
|\input{childdoc.def}|\\
|\childdocof{|\textit{main}|}|\\
\end{tabular}
\end{center}
at the top of every child file \textit{child}
which is included by |\include{|\textit{child}|}|
from within the main file
(or at least for those files to be compiled individually).
The argument \textit{main} must be the filename of the main file.

There are a couple of
considerations in setting up the main and child documents:

%%%%%%%%%%%%%%%%%%%%%%%%%%%%%%%%%%%%%%%%
\paragraph{Restrictions.}

Please note the following restrictions:
\begin{itemize}
\item
|\childdocmain| must be called with one argument \textit{main}
to ensure compatibility with earlier version of the package.
It must either be empty (|\childdocmain{}|)
or precisely match the filename of the main file in which it is specified.
See \secref{sec:detection} for further information.
\item
The filename \textit{main} must be specified without the |.tex| extension.
\item
The filename \textit{main} is case sensitive
(even in case-insensitive file systems)
due to internal string comparison.
\item
The argument \textit{main} should be fully expanded, it cannot be a macro.
\item
Subdirectories and special characters should be avoided in filenames.
\item
The command |\childdocmain{|\textit{main}|}| must be followed by a whitespace.
It should not be followed immediately by another command
or by a comment mark `|%|'.
This is because the \TeX{} parser reads the token immediately following
the argument of |\childdocmain| and puts it
at the beginning of every child section;
however, a white\-space is ignored.
\end{itemize}

%%%%%%%%%%%%%%%%%%%%%%%%%%%%%%%%%%%%%%%%
\paragraph{Content of Main File.}

It is advisable to place all content in the child files included by |\include|.
Any output contained in the main file will appear in all child documents
unless suppressed manually;
it cannot be suppressed automatically by the |\includeonly| directive
and thus should normally be avoided.
A method to include some content in the main file
by means of conditional processing is described in \secref{sec:conditional}.

%%%%%%%%%%%%%%%%%%%%%%%%%%%%%%%%%%%%%%%%
\paragraph{Page Numbering.}

When only a part of the document is compiled,
the appropriate numbering of pages
(as well as other status parameters)
is determined from the |.aux| files.
The latter contain information from previous passes.
However this information needs to propagate through
all intermediate child documents.
Therefore the page numbering in child documents may well
be inconsistent until the complete document is compiled at least once.

A useful (if unconventional) way to always ensure a consistent
page numbering is to restart the numbering in each child document
and denote the pages by `\textit{child}|.|\textit{page}'
where \textit{child} represents the chapter/section number of the child file.
This can be achieved by the command
|\numberwithin{page}{|\textit{child}|}|
of the \textsf{amsmath} package
where \textit{child} can be |chapter| or |section|
depending on the chosen structuring.
Alternatively, one can modify the macro |\thepage| appropriately
and reset the counter |page| at the start of each child file.

%%%%%%%%%%%%%%%%%%%%%%%%%%%%%%%%%%%%%%%%%%%%%%%%%%%%%%%%%%%%%%%%%%%%%%%%%%%%%%%%
\subsection{Conditional Processing}
\label{sec:conditional}

The package provides a mechanism to compile different versions
of a document. To customise the versions further some conditional processing
can come in handy to distinguish which version is being compiled.
The package provides two macros to describe the compilation context:

%%%%%%%%%%%%%%%%%%%%%%%%%%%%%%%%%%%%%%%%
\DescribeMacro{\ifchilddoc}
The conditional |\ifchilddoc| distinguishes between the compilation of
child documents and the main document:
%
\begin{center}
|\ifchilddoc |\textit{child-code}| |[|\||else |\textit{main-code}]| \||fi|
\end{center}

%%%%%%%%%%%%%%%%%%%%%%%%%%%%%%%%%%%%%%%%
\DescribeMacro{\childdocname}
\DescribeMacro{\childdocjob}
The macro |\childdocname| contains the filename (without extension)
of the main or child file being processed.
Note that |\childdocjob| will always contain the name of the main file.

%%%%%%%%%%%%%%%%%%%%%%%%%%%%%%%%%%%%%%%%
\paragraph{Title Page.}

Conditional processing can be used to include a title or banner page
in the main document when proper precautions are taken.
Importantly, the code in the main file should ensure that the page counter
(as well as other status parameters which are stored in the |.aux| files)
takes the same value after the conditional processing.
Otherwise the page numbers may take divergent values
depending on which part is compiled.

For example, a title page could be declared by:
%
\begin{center}
\begin{tabular}{l}
|\ifchilddoc\||else|\\
|\addtocounter{page}{-1}|\\
\textit{code for title page}\\
|\newpage|\\
|\||fi|
\end{tabular}
\end{center}
%
A banner page for the child documents can be generated by:
%
\begin{center}
\begin{tabular}{l}
|\ifchilddoc|\\
|\addtocounter{page}{-1}|\\
\textit{code for banner page}\\
|\newpage|\\
|\||fi|
\end{tabular}
\end{center}
%
Here one could write a message such as:
\begin{center}
|This is the part \childdocname{} of \childdocjob{}.|
\end{center}

%%%%%%%%%%%%%%%%%%%%%%%%%%%%%%%%%%%%%%%%%%%%%%%%%%%%%%%%%%%%%%%%%%%%%%%%%%%%%%%%
\subsection{Flags}
\label{sec:flags}

The package makes it easy to generate different versions
of the main or child documents.
To this end compilation flags can be defined
and assigned different default values.
They will be particularly useful in conjunction
with the forwarding mechanism described in \secref{sec:forward}.

For example, it may be useful to have a flag |\version|
which can be set to |draft| or |final|.
The document source will contain some conditional code
depending on the value of |\version|.
Suppose further, the flag should default to |final| for the main file
and to |draft| for child files
which is a natural assignment for editing the document.
This is achieved by placing the following code
in the preamble of the main document
(below the |\childdocmain| directive):
%
\begin{center}
\begin{tabular}{l}
|\ifchilddoc|\\
|\providecommand{\version}{draft}|\\
|\||else|\\
|\providecommand{\version}{final}|\\
|\||fi|
\end{tabular}
\end{center}
%
The definition by |\providecommand| makes sure
that previous definitions are not overwritten.
Further statements |\providecommand{\version}{...}|
can thus be added before the above code to override it.

For the main file, one might add a line
(between |\childdocmain| and the above block)
%
\begin{center}
|%\ifchilddoc\||else\providecommand{\version}{draft}\||fi|
\end{center}
%
which can be uncommented to produce a draft version.
Likewise one can add a line to the very top of a child file
(above the |\childdocof{|\textit{main}|}| directive)
%
\begin{center}
|%\providecommand{\version}{final}|
\end{center}
%
which can be uncommented to produce the final version of this child document.

%%%%%%%%%%%%%%%%%%%%%%%%%%%%%%%%%%%%%%%%%%%%%%%%%%%%%%%%%%%%%%%%%%%%%%%%%%%%%%%%
\subsection{Forwarding}
\label{sec:forward}

Different versions of the main or child documents
using compilation flags as described in \secref{sec:flags}
can be (permanently) stored in different files
for convenient compilation, viewing and distribution.
To this end, the package defines a command
to pass on compilation to a different file:

%%%%%%%%%%%%%%%%%%%%%%%%%%%%%%%%%%%%%%%%
\DescribeMacro{\childdocforward}
The command |\childdocforward| redirects processing to
another source file:
%
\begin{center}
\begin{tabular}{l}
|\input{childdoc.def}|\\
|\childdocforward[|\textit{main}|]{|\textit{dest}|}|\\
\end{tabular}
\end{center}
%
The argument \textit{dest} is the destination file
(without extension).
It should be the main file or one of the child files.
Note that further \textsf{childdoc} directives
such as |\childdocof| and |\childdocforward|
in the indicated file will be processed in this form.
The optional argument \textit{main}
passes on directly to the main file \textit{main}
while pretending to compile the child \textit{dest}.
This form behaves as if \textit{dest}
issues |\childdocof{|\textit{main}|}| right away,
and no further \textsf{childdoc} directives will be processed.

%%%%%%%%%%%%%%%%%%%%%%%%%%%%%%%%%%%%%%%%
\DescribeMacro{\...prefix}
In the alternative form |\childdocforwardprefix|,
%
\begin{center}
\begin{tabular}{l}
|\input{childdoc.def}|\\
|\childdocforwardprefix[|\textit{main}|]{|\textit{prefix}|}{|\textit{dest}|}|
\end{tabular}
\end{center}
%
the destination file is determined by a pattern
depending on the current file:
To make this work, the current file must be called
`{\textit{prefix}\hspace{0.2em}\textit{suffix}}'
with \textit{prefix} matching precisely the argument.
Processing is then passed on to the file
`{\textit{dest}\hspace{0.2em}\textit{suffix}}'.
Surely, the same effect is achieved by
directly specifying the
argument `{\textit{dest}\hspace{0.2em}\textit{suffix}}'
in the first form.
However, that requires to set up a different file
for each child. With the alternative form of the command
all these files can have exactly the same content
which simplifies setting them up and maintaining them.

For example, the following file |draft.tex|
with a compilation flag |\version| as described in \secref{sec:flags}
compiles the main document as a draft:
%
\begin{center}
\begin{tabular}{l}
|\def\version{draft}|\\
|\input{childdoc.def}|\\
|\childdocforward{|\textit{main}|}|
\end{tabular}
\end{center}
%
Likewise, the following files |final|\textit{nn}|.tex|
compile the final version of the child document
|child|\textit{nn}|.tex|:
%
\begin{center}
\begin{tabular}{l}
|\def\version{final}|\\
|\input{childdoc.def}|\\
|\childdocforwardprefix{final}{child}|
\end{tabular}
\end{center}
%

Note that when several versions of a main file and/or of each child file
are to be generated, it may be convenient to set up a |Makefile| or
shell script to automatise the process.

%%%%%%%%%%%%%%%%%%%%%%%%%%%%%%%%%%%%%%%%%%%%%%%%%%%%%%%%%%%%%%%%%%%%%%%%%%%%%%%%
\subsection{Command Line Processing}
\label{sec:commandline}

The effect of redirection files can also be achieved by invoking
the \LaTeX{} compiler with a more elaborate command line.
Most conveniently this should be done as part
of a shell script or a |Makefile|.

When using \textsf{childdoc} in the main file, the following
command lines effectively perform a redirection
(note that depending on the shell being used,
backslashes may have to be doubled: `|\|' $\to$ `|\\|'):
%
\begin{center}
|... -jobname "|\textit{target}|" |\\|"|[\textit{flags}]%
|\input{childdoc.def}\childdocforward[|\textit{main}|]{|\textit{dest}|}"|
\end{center}
%
Here \textit{target} is the name of the output file,
\textit{main} is the name of the main file
and \textit{dest} is the name of the main or child file to be processed
(all filenames without extensions).
The optional argument \textit{main} can be omitted
if \textit{main} matches \textit{dest}.
Optionally, compilation \textit{flags} can be defined via |\def| commands.
This command line makes the \TeX{} engine believe
it is compiling the file \textit{target}
whose content is specified as the latter parameter.
The provided code then forwards the processing to
\textit{main} or \textit{dest} as described in \secref{sec:forward}.

%%%%%%%%%%%%%%%%%%%%%%%%%%%%%%%%%%%%%%%%%%%%%%%%%%%%%%%%%%%%%%%%%%%%%%%%%%%%%%%%
\subsection{Include by Input}
\label{sec:input}

Including child documents by |\include| has some restrictions by design.
Most notably, the content of a child document always occupies
its own set of pages; pages cannot be shared between child documents.
Usually, this behaviour makes perfect sense
because each child document contain an essential part of the document.
However, in some situations it may be desirable to compose
a document from a collection of parts
without having mandatory page breaks between then.
For this case, the package
provides a mechanism to include parts
by |\input| which can also be processed individually.
However, by construction this mechanism
requires manual handling of the content to be output.

%%%%%%%%%%%%%%%%%%%%%%%%%%%%%%%%%%%%%%%%
\DescribeMacro{\ifchilddocmanual}
The main file should be prepared as usual, see \secref{sec:include}.
However, the document body must make a distinction
between processing of an individual part and of the main document, e.g.:
%
\begin{center}
\begin{tabular}{l}
|\ifchilddocmanual|\\
|\input{\childdocname}|\\
|\||else|\\
\textit{document body with }|\input{|\textit{part}|}|\\
|\||fi|
\end{tabular}
\end{center}
%
The conditional |\ifchilddocmanual| is true whenever
a part to be included by |\input| is being compiled,
and the name of the part is stored in |\childdocname|.

%%%%%%%%%%%%%%%%%%%%%%%%%%%%%%%%%%%%%%%%
\DescribeMacro{\childdocby}
Each part to be included by |\input| should start with:
%
\begin{center}
\begin{tabular}{l}
|\input{childdoc.def}|\\
|\childdocby{|\textit{main}|}|\\
\end{tabular}
\end{center}
%
The directive |\childdocby| is similar to |\childdocof|
described in \secref{sec:include},
but the subsequent selection of content must be done manually.
To that end, both |\ifchilddoc| and |\ifchilddocmanual|
will be true upon processing of a part,
and the name of the part is stored in |\childdocname|.
Note that |\jobname| will be set to the filename of the current part
so that each part receives an individual |.aux| file
that does not interfere with the |.aux| file(s) of the main document.
This behaviour can be altered by the alternative form
|\childdocby[*]{|\textit{main}|}| (with a non-empty optional argument)
which uses the |.aux| file of the main document
by setting |\jobname| to \textit{main}.

%%%%%%%%%%%%%%%%%%%%%%%%%%%%%%%%%%%%%%%%%%%%%%%%%%%%%%%%%%%%%%%%%%%%%%%%%%%%%%%%
\subsection{Driver Development}
\label{sec:driver}

The \textsf{childdoc} mechanism can also be use for the development
of definition files such as \LaTeX{} styles or classes.
This case differs from the above setup with multiple parts
included by |\include| in that no |\includeonly| should be invoked.
This can be achieved by starting the include file
(before |\ProvidesPackage|) with:
%
\begin{center}
\begin{tabular}{l}
|\input{childdoc.def}|\\
|\childdocforward{|\textit{main}|}|\\
\end{tabular}
\end{center}
%
or alternatively with:
%
\begin{center}
\begin{tabular}{l}
|\input{childdoc.def}|\\
|\childdocby{|\textit{main}|}|\\
\end{tabular}
\end{center}
%
Both forms have slightly different effects as described above.
The main file is prepared as usual, see \secref{sec:include}.

%%%%%%%%%%%%%%%%%%%%%%%%%%%%%%%%%%%%%%%%%%%%%%%%%%%%%%%%%%%%%%%%%%%%%%%%%%%%%%%%
\subsection{Legacy Detection}
\label{sec:detection}

The directive |\childdocmain| in the main file can detect
whether the complete document or merely a child is to be compiled
even without using the directive |\childdocof|.
This method is deprecated because it is less robust
and there is no compelling reason to use it;
it is merely provided for backward compatibility
and it may be removed in future versions.

If the detection mechanism is to be used,
it is mandatory to correctly specify
the filename of the main file as the argument of |\childdocmain|:
%
\begin{center}
\begin{tabular}{l}
|\input{childdoc.def}|\\
|\childdocmain{|\textit{main}|}|\\
\end{tabular}
\end{center}
%
If |\jobname| does not match the argument \textit{main} of |\childdocmain|,
it is assumed that |\jobname| points to the child file to be compiled.
When using |\childdocmain| with the main file specified as argument,
it suffices to start a child file
with just |\input{|\textit{main}|}|
without loading of the package and using |\childdocof|.
If instead all processing is done
with the appropriate \textsf{childdoc} directives,
the argument of \textit{main} of |\childdocmain| can be empty.

An alternative version of the command line processing described
in \secref{sec:commandline} using the detection mechanism reads:
%
\begin{center}
|... -jobname "|\textit{target}|" "|[\textit{flags}]%
[|\def\jobname{|\textit{dest}|}|]|\input{|\textit{main}|}"|
\end{center}

%%%%%%%%%%%%%%%%%%%%%%%%%%%%%%%%%%%%%%%%%%%%%%%%%%%%%%%%%%%%%%%%%%%%%%%%%%%%%%%%
\subsection{Manual Code}
\label{sec:manual}

In case one cannot be certain whether the definitions file |childdoc.def|
is installed on the target \TeX{} distribution
and one prefers not to ship it,
it is conceivable to paste a few relevant commands into the sources.

To that end, drop all statements |\input{childdoc.def}|
and perform the replacements as outlined below.
Instead of |\childdocmain{|\textit{main}|}| add the following code
to the top of the main file:
%
\begin{center}
\begin{tabular}{l}
|\||ifdefined\childdocname\endinput\||fi\newif\ifchilddoc|\\
|\edef\childdocname{\scantokens\expandafter{\jobname\noexpand}}|\\
|\def\childdocmain{|\textit{main}|}\||ifx\childdocmain\childdocname\||else|\\
|\childdoctrue\includeonly{\childdocname}\let\jobname\childdocmain\||fi|\\
\end{tabular}
\end{center}
%
Instead of |\childdocof{|\textit{main}|}| just include the main file
at the top of each child file:
%
\begin{center}
|\input{|\textit{main}|}|
\end{center}
%
A simple redirection |\childdocforward{|\textit{dest}|}| is achieved by:
%
\begin{center}
|\def\jobname{|\textit{dest}|}\input{\jobname}|
\end{center}
%
The redirection with prefix
|\childdocforwardprefix[|\textit{prefix}|]{|\textit{dest}|}|
is accomplished by:
%
\begin{center}
\begin{tabular}{l}
|{\edef\jobname{\scantokens\expandafter{\jobname\noexpand}}|\\
|\def\redirectjob |\textit{prefix}|#1~~~{\gdef\jobname{|\textit{dest}|#1}}|\\
|\expandafter\redirectjob\jobname~~~}\input{\jobname}|
\end{tabular}
\end{center}

In an alternative approach,
child documents can be compiled by a specific command line
without additional code or specific definitions:
%
\begin{center}
|... -jobname "|\textit{target}|" "|[\textit{flags}]%
|\includeonly{|\textit{dest}|}\input{|\textit{main}|}"|
\end{center}
%

%%%%%%%%%%%%%%%%%%%%%%%%%%%%%%%%%%%%%%%%%%%%%%%%%%%%%%%%%%%%%%%%%%%%%%%%%%%%%%%%
%%%%%%%%%%%%%%%%%%%%%%%%%%%%%%%%%%%%%%%%%%%%%%%%%%%%%%%%%%%%%%%%%%%%%%%%%%%%%%%%
\section{Information}

%%%%%%%%%%%%%%%%%%%%%%%%%%%%%%%%%%%%%%%%%%%%%%%%%%%%%%%%%%%%%%%%%%%%%%%%%%%%%%%%
\subsection{Copyright}

Copyright \copyright{} 2017--2018 Niklas Beisert

This work may be distributed and/or modified under the
conditions of the \LaTeX{} Project Public License, either version 1.3
of this license or (at your option) any later version.
The latest version of this license is in
  \url{http://www.latex-project.org/lppl.txt}
and version 1.3 or later is part of all distributions of \LaTeX{}
version 2005/12/01 or later.

This work has the LPPL maintenance status `maintained'.

The Current Maintainer of this work is Niklas Beisert.

This work consists of the files |README.txt|, |childdoc.ins| and |childdoc.dtx|
as well as the derived files |childdoc.def|, |cdocsamp.tex|
with |cdocsch1.tex|, |cdocsch2.tex|, |cdocspt3.tex|, |cdocspt4.tex|,
|cdocsdrf.tex|, |cdocsfn1.tex|, |cdocsfn2.tex|
as well as |childdoc.pdf|.

%%%%%%%%%%%%%%%%%%%%%%%%%%%%%%%%%%%%%%%%%%%%%%%%%%%%%%%%%%%%%%%%%%%%%%%%%%%%%%%%
\subsection{Files and Installation}

The package consists of the files:
%
\begin{center}
\begin{tabular}{ll}
    |README.txt|   & readme file \\
    |childdoc.ins| & installation file \\
    |childdoc.dtx| & source file \\
    |childdoc.def| & definition file \\
    |cdocsamp.tex| & sample main file \\
    |cdocsch1.tex| & sample include file \\
    |cdocsch2.tex| & sample include file \\
    |cdocspt3.tex| & sample part file \\
    |cdocspt4.tex| & sample part file \\
    |cdocsdrf.tex| & sample redirection file \\
    |cdocsfn1.tex| & sample redirection file \\
    |cdocsfn2.tex| & sample redirection file \\
    |childdoc.pdf| & manual
\end{tabular}
\end{center}
%
The distribution consists of the files
|README.txt|, |childdoc.ins| and |childdoc.dtx|.
%
\begin{itemize}
\item
Run (pdf)\LaTeX{} on |childdoc.dtx|
to compile the manual |childdoc.pdf| (this file).
\item
Run \LaTeX{} on |childdoc.ins| to create the definitions file |childdoc.def|
and the sample |cdocsamp.tex| with include files
|cdocsch1.tex|, |cdocsch2.tex|, |cdocspt3.tex|, |cdocspt4.tex|,
|cdocsdrf.tex|, |cdocsfn1.tex|, |cdocsfn2.tex|.
Then copy the file |childdoc.def| to an appropriate directory of your \LaTeX{}
distribution, e.g.\ \textit{texmf-root}|/tex/latex/childdoc|.
\end{itemize}

%%%%%%%%%%%%%%%%%%%%%%%%%%%%%%%%%%%%%%%%%%%%%%%%%%%%%%%%%%%%%%%%%%%%%%%%%%%%%%%%
\subsection{Related CTAN Packages}

There are several other packages which offer a similar functionality:
%
\begin{itemize}
\item
The packages
\href{http://ctan.org/pkg/docmute}{\textsf{docmute}},
\href{http://ctan.org/pkg/includex}{\textsf{includex}} and
\href{http://ctan.org/pkg/standalone}{\textsf{standalone}}
provide commands to include only the document body of
a child file thus allowing both files to be compiled individually.
\item
The packages \href{http://ctan.org/pkg/subdocs}{\textsf{subdocs}}
and \href{http://ctan.org/pkg/subfiles}{\textsf{subfiles}}
provide structures in which the main and child documents can be
encapsulated and allowing them to be compiled individually.
The inclusion mechanism is different from the conventional |\include|.
\item
The package \href{http://ctan.org/pkg/combine}{\textsf{combine}}
is an elaborate solution to combine several documents into one.
\end{itemize}
%
See also the CTAN topic \href{http://ctan.org/topic/subdocs}{\textsf{subdocs}}
for further related packages.
The present package differs from the above solutions in that
a document structure constructed with the conventional |\include| mechanism
just needs two extra commands at the top of every file
such that all constituent files can be compiled individually.

%%%%%%%%%%%%%%%%%%%%%%%%%%%%%%%%%%%%%%%%%%%%%%%%%%%%%%%%%%%%%%%%%%%%%%%%%%%%%%%%
%\subsection{Feature Suggestions}
%
%The following is a list of features which may be useful for future
%versions of this package:
%%
%\begin{itemize}
%\item
%\ldots
%\end{itemize}

%%%%%%%%%%%%%%%%%%%%%%%%%%%%%%%%%%%%%%%%%%%%%%%%%%%%%%%%%%%%%%%%%%%%%%%%%%%%%%%%
\subsection{Revision History}

%%%%%%%%%%%%%%%%%%%%%%%%%%%%%%%%%%%%%%%%
\paragraph{v2.0:} 2018/12/30

\begin{itemize}
\item
immediate forward processing
\item
added |\childdocby| mechanism
\item
manual restructured
\end{itemize}

%%%%%%%%%%%%%%%%%%%%%%%%%%%%%%%%%%%%%%%%
\paragraph{v1.6:} 2018/01/17

\begin{itemize}
\item
application for development of include files
\item
corrections to manual
\end{itemize}

%%%%%%%%%%%%%%%%%%%%%%%%%%%%%%%%%%%%%%%%
\paragraph{v1.5:} 2017/05/21

\begin{itemize}
\item
more complete structuring introduced
\item
|\childdocof| introduced
\item
|\childdoc| renamed to |\childdocmain|
\item
|\childredirect| renamed to |\childdocforward| and |\childdocforwardprefix|
and functionality expanded
\end{itemize}

%%%%%%%%%%%%%%%%%%%%%%%%%%%%%%%%%%%%%%%%
\paragraph{v1.0:} 2017/04/27

\begin{itemize}
\item
manual and install package
\item
first version published on CTAN
\end{itemize}

%%%%%%%%%%%%%%%%%%%%%%%%%%%%%%%%%%%%%%%%
\paragraph{v0.6:} 2017/04/26

\begin{itemize}
\item
redirection mechanism added
\end{itemize}

%%%%%%%%%%%%%%%%%%%%%%%%%%%%%%%%%%%%%%%%
\paragraph{v0.5:} 2017/04/26

\begin{itemize}
\item
functionality in definition file
\end{itemize}


%%%%%%%%%%%%%%%%%%%%%%%%%%%%%%%%%%%%%%%%%%%%%%%%%%%%%%%%%%%%%%%%%%%%%%%%%%%%%%%%
%%%%%%%%%%%%%%%%%%%%%%%%%%%%%%%%%%%%%%%%%%%%%%%%%%%%%%%%%%%%%%%%%%%%%%%%%%%%%%%%
%%%%%%%%%%%%%%%%%%%%%%%%%%%%%%%%%%%%%%%%%%%%%%%%%%%%%%%%%%%%%%%%%%%%%%%%%%%%%%%%
\appendix

\settowidth\MacroIndent{\rmfamily\scriptsize 000\ }

 \DocInput{childdoc.dtx}

\end{document}
%</driver>
% \fi
%
% %%%%%%%%%%%%%%%%%%%%%%%%%%%%%%%%%%%%%%%%%%%%%%%%%%%%%%%%%%%%%%%%%%%%%%%%%%%%%%
% %%%%%%%%%%%%%%%%%%%%%%%%%%%%%%%%%%%%%%%%%%%%%%%%%%%%%%%%%%%%%%%%%%%%%%%%%%%%%%
% \section{Sample}
%\iffalse
%<*samplemain>
%\fi
%
% The following presents a sample document
% with two chapters, two parts, a title page,
% a compile flag as well as three forwarding files to set the flag.
% It consists of eight |.tex| files:
% \begin{center}
% \begin{tabular}{ll}
% |cdocsamp.tex|&main file\\
% |cdocsch1.tex|&include file for chapter 1\\
% |cdocsch2.tex|&include file for chapter 2\\
% |cdocspt3.tex|&include file for part 3\\
% |cdocspt4.tex|&include file for part 4\\
% |cdocsdrf.tex|&forwarding file for main file in draft mode\\
% |cdocsfi1.tex|&forwarding file for final version of chapter 1\\
% |cdocsfi2.tex|&forwarding file for final version of chapter 2\\
% \end{tabular}
% \end{center}
% Each of the eight files can be compiled directly by the \LaTeX{} compiler.
%
% %%%%%%%%%%%%%%%%%%%%%%%%%%%%%%%%%%%%%%
% \paragraph{Main File.}
%
% The main file is called |cdocsamp.tex|.
%
% Load the \textsf{childdoc} definitions and
% declare the filename for the main document:
%    \begin{macrocode}
\input{childdoc.def}
\childdocmain{}
%    \end{macrocode}

% Optional override for |\version| flag:
%    \begin{macrocode}
%%\ifchilddoc\else\providecommand{\version}{draft}\fi
%    \end{macrocode}

% Define the default values for the |\version| flag
% (|final| for the main file and |draft| for childs):
%    \begin{macrocode}
\ifchilddoc
\providecommand{\version}{draft}
\else
\providecommand{\version}{final}
\fi
%    \end{macrocode}

% Load the standard document class:
%    \begin{macrocode}
\documentclass[12pt]{article}
%    \end{macrocode}

% Start the document body:
%    \begin{macrocode}
\begin{document}
%    \end{macrocode}

% Declare a title page.
% Print title, part of document being processed and version flag:
%    \begin{macrocode}
\addtocounter{page}{-1}
\begin{center}
{\LARGE\bfseries{}childdoc example\par}
\vspace{1cm}
\ifchilddoc
\ifchilddocmanual part\else chapter\fi:
`\childdocname' of `\childdocjob'\par
\else
main document: `\childdocjob'\par
\fi
version: \version\par
\end{center}
\newpage
%    \end{macrocode}

% Manually include selected file,
% otherwise process as usual:
%    \begin{macrocode}
\ifchilddocmanual
\section*{part `\childdocname'}
\input{\childdocname}
\else
%    \end{macrocode}

% Include the two chapters:
%    \begin{macrocode}
\include{cdocsch1}
\include{cdocsch2}
%    \end{macrocode}

% Include the two parts unless only chapters should be displayed:
%    \begin{macrocode}
\ifchilddoc\else
\section{part three}
\input{cdocspt3}
\section{part four}
\input{cdocspt4}
\fi
%    \end{macrocode}

% Process as usual until here:
%    \begin{macrocode}
\fi
%    \end{macrocode}

% End of document body:
%    \begin{macrocode}
\end{document}
%    \end{macrocode}
%\iffalse
%</samplemain>
%\fi
%
% %%%%%%%%%%%%%%%%%%%%%%%%%%%%%%%%%%%%%%
% \paragraph{Chapter Include Files.}
%
% The include files are called |cdocsch1.tex| and |cdocsch2.tex|.
%
%\iffalse
%<*samplechap1|samplechap2>
%\fi

% Optional override for |\version| flag:
%    \begin{macrocode}
%%\providecommand{\version}{final}
%    \end{macrocode}

% Include the main document:
%    \begin{macrocode}
\input{childdoc.def}
\childdocof{cdocsamp}
%    \end{macrocode}

%\iffalse
%</samplechap1|samplechap2>
%\fi
%
%\iffalse
%<*samplechap1>
%\fi
% Some text for chapter 1:
%    \begin{macrocode}
\section{one}
some text in chapter one
%    \end{macrocode}

%\iffalse
%</samplechap1>
%\fi
% Some text for chapter 2:
%\iffalse
%<*samplechap2>
%\fi
%    \begin{macrocode}
\section{two}
more text in chapter two
%    \end{macrocode}

%\iffalse
%</samplechap2>
%\fi
%
% %%%%%%%%%%%%%%%%%%%%%%%%%%%%%%%%%%%%%%
% \paragraph{Part Include Files.}
%
% The include files are called |cdocspt3.tex| and |cdocspt4.tex|.
%
%\iffalse
%<*samplepart3|samplepart4>
%\fi

% Optional override for |\version| flag:
%    \begin{macrocode}
%%\providecommand{\version}{final}
%    \end{macrocode}

% Include the main document:
%    \begin{macrocode}
\input{childdoc.def}
\childdocby{cdocsamp}
%    \end{macrocode}

%\iffalse
%</samplepart3|samplepart4>
%\fi
%
%\iffalse
%<*samplepart3>
%\fi
% Some text for part 3:
%    \begin{macrocode}
some text in part three
%    \end{macrocode}

%\iffalse
%</samplepart3>
%\fi
% Some text for part 4:
%\iffalse
%<*samplepart4>
%\fi
%    \begin{macrocode}
more text in part four
%    \end{macrocode}

%\iffalse
%</samplepart4>
%\fi
%
% %%%%%%%%%%%%%%%%%%%%%%%%%%%%%%%%%%%%%%
% \paragraph{Forwarding for a Complete Draft.}
%
% The following forwarding file |cdocsdrf.tex|
% compiles the main document in draft mode:
%\iffalse
%<*sampledraft>
%\fi
%    \begin{macrocode}
\def\version{draft}
\input{childdoc.def}
\childdocforward{cdocsamp}
%    \end{macrocode}

%\iffalse
%</sampledraft>
%\fi
%
% %%%%%%%%%%%%%%%%%%%%%%%%%%%%%%%%%%%%%%
% \paragraph{Forwarding for Final Version of the Chapters.}
%
% The following forwarding files |cdocsfn1.tex| and |cdocsfn2.tex|
% (with identical content)
% compile the final versions of the child documents
% |cdocsch1.tex| and |cdocsch2.tex|, respectively:
%\iffalse
%<*samplefinal>
%\fi
%    \begin{macrocode}
\def\version{final}
\input{childdoc.def}
\childdocforwardprefix[cdocsamp]{cdocsfn}{cdocsch}
%    \end{macrocode}

%\iffalse
%</samplefinal>
%\fi
%
% %%%%%%%%%%%%%%%%%%%%%%%%%%%%%%%%%%%%%%
% \paragraph{Command Line Processing.}
%
% The following three command lines generate the output files
% |cdocscld|, |cdocscl1| and |cdocscl2|
% which should be identical to
% |cdocsdrf|, |cdocsch1| and |cdocsfn2|, respectively:
% \begin{center}
% \begin{tabular}{l}
% |latex -jobname cdocscld \|\\
% |  "\def\version{draft}\input{childdoc.def}\childdocforward{cdocsamp}"|\\
% |latex -jobname cdocscl1 \|\\
% |  "\input{childdoc.def}\childdocforward[cdocsamp]{cdocsch1}"|\\
% |latex -jobname cdocscl2 \|\\
% |  "\def\version{final}\input{childdoc.def}\childdocforward{cdocsch2}"|
% \end{tabular}
% \end{center}
% Note that the trailing backslash on each first line
% merely continues the input to the second line
% (for convenient cut ant paste).
% Furthermore, the command |latex| can be replaced by any
% of its alternative versions such as |pdflatex|.
%
% %%%%%%%%%%%%%%%%%%%%%%%%%%%%%%%%%%%%%%%%%%%%%%%%%%%%%%%%%%%%%%%%%%%%%%%%%%%%%%
% %%%%%%%%%%%%%%%%%%%%%%%%%%%%%%%%%%%%%%%%%%%%%%%%%%%%%%%%%%%%%%%%%%%%%%%%%%%%%%
% \section{Implementation}
%\iffalse
%<*package>
%\fi
%
% This section describes the definitions file |childdoc.def|.

% The definitions cannot be loaded using |\usepackage| or |\RequirePackage|
% which has a mechanism to prevent loading a style file more than once.
% When loading the definitions by means of |\input|
% multiple instances have to be prevented manually:
%\iffalse
%This code needs to be before the `\ProvidesFile' directive
%which is defined at the beginning of this file.
%Therefore it is also placed there and commented out here.
%</package>
%<*discard>
%\fi
%    \begin{macrocode}
\ifdefined\childdocmain\endinput\fi
%    \end{macrocode}
%\iffalse
%</discard>
%<*package>
%\fi
%
% \macro{\ifchilddoc}
% \macro{\ifchilddocmanual}
% The conditional |\ifchilddoc| tells whether a
% child (true) or main (false) document is being compiled.
% The conditional |\ifchilddocmanual| tells whether
% the |\includeonly| mechanism is used (false) or
% the selection of child files must be performed manually (true).
% The definitions initialise to false:
%    \begin{macrocode}
\newif\ifchilddoc
\newif\ifchilddocmanual
%    \end{macrocode}

% \macro{\childdocname}
% \macro{\childdocjob}
% The macro |\childdocname| stores the name of the main document
% to be compiled. The macro |\childdocjob| stores the name of
% the document on which the \LaTeX{} compiler was originally invoked.
% The content of |\jobname| cannot be compared
% to filenames specified in the source due to different catcodes.
% The following code rescans |\jobname|, stores the result
% in |\childdocname| and saves a copy in |\childdocjob|:
%    \begin{macrocode}
\edef\childdocname{\scantokens\expandafter{\jobname\noexpand}}
\let\childdocjob\childdocname
%    \end{macrocode}

% \macro{\childdocdisable}
% The macro |\childdocdisable| prevents the main file
% from being processed more than once.
% At this stage, the main document command |\childdocmain|
% is assumed to be called once again where it should do nothing.
% Any subsequent call to it should prevent
% a secondary processing of the main document
% It overwrites the forwarding commands
% |\childdocof| and |\childdocforward|
% with empty macros to prevent further inclusions of the main document:
%    \begin{macrocode}
\newcommand{\childdocdisable}
{
  \renewcommand{\childdocmain}[1]{\renewcommand{\childdocmain}[1]{\endinput}}
  \renewcommand{\childdocof}[1]{}
  \renewcommand{\childdocby}[2][]{}
  \renewcommand{\childdocforward}[2][]{}
  \renewcommand{\childdocdisable}{}
}
%    \end{macrocode}

% \macro{\childdocmain}
% The macro |\childdocmain| is to be called at the top of the main file
% with nothing or the main filename (without extension) as argument.
% First, it breaks loops.
% If the argument is not empty and does not match |\childdocname|
% (which is set by the first inclusion of |childdoc.def|),
% |\ifchilddoc| is set to true, |\includeonly| is applied to the child file
% and |\jobname| is set to the main file
% (for proper handling of |.aux| files):
%    \begin{macrocode}
\newcommand{\childdocmain}[1]
{
  \childdocdisable\childdocmain{}
  \if?#1?\else
    \begingroup
      \def\childdoctmp{#1}
      \ifx\childdoctmp\childdocname
        \def\childdoctmp{}
      \else
        \def\childdoctmp
        {
          \childdoctrue
          \includeonly{\childdocname}
          \def\childdocjob{#1}
          \def\jobname{#1}
        }
      \fi
      \expandafter
    \endgroup
    \childdoctmp
  \fi
}
%    \end{macrocode}

% \macro{\childdocof}
% The command |\childdocof| redirects
% compilation to the main file |#1|.
%    \begin{macrocode}
\newcommand{\childdocof}[1]
{
  \childdocdisable
  \childdoctrue
  \includeonly{\childdocname}
  \def\jobname{#1}
  \def\childdocjob{#1}
  \input{#1}
}
%    \end{macrocode}

% \macro{\childdocby}
% The command |\childdocby| ....
%    \begin{macrocode}
\newcommand{\childdocby}[2][]
{
  \childdocdisable
  \childdoctrue
  \childdocmanualtrue
  \if?#1?\else
    \def\jobname{#2}
  \fi
  \def\childdocjob{#2}
  \input{#2}
  \endinput
}
%    \end{macrocode}

% \macro{\childdocforward}
% The command |\childdocforward| redirects
% compilation to the main file or
% (if the optional argument is given) a child file.
% Parameters are set as if the main file
% or a child file starting with |\childdocof| was compiled.
% Then compilation is handed over to the main file:
%    \begin{macrocode}
\newcommand{\childdocforward}[2][]
{
  \begingroup
    \if?#1?
      \def\childdoctmp
      {
        \def\childdocname{#2}
        \def\childdocjob{#2}
        \def\jobname{#2}
        \input{#2}
        \endinput
      }
    \else
      \def\childdoctmp
      {
        \childdocdisable
        \def\childdocname{#2}
        \childdoctrue
        \includeonly{#2}
        \def\childdocjob{#1}
        \def\jobname{#1}
        \input{#1}
        \endinput
      }
    \fi
    \expandafter
  \endgroup
  \childdoctmp
}
%    \end{macrocode}

% \macro{\childdocforwardprefix}
% The command |\childdocforwardprefix| redirects
% compilation to the main or a child file by means of a pattern.
% The prefix |#1| in the current filename is replaced by |#2|
% and the suffix of the current filename is kept
% (it is assumed that the filename does not contain the substring `|~~~|'
% which is used as a delimiter).
% Compilation is handed over to the new file by |\childdocforward|:
%    \begin{macrocode}
\newcommand{\childdocforwardprefix}[3][]
{
  \begingroup
    \def\childdocextract #2##1~~~{\def\childdoctmp{\childdocforward[#1]{#3##1}}}
    \expandafter\childdocextract\childdocname~~~
    \expandafter
  \endgroup
  \childdoctmp
}
%    \end{macrocode}

% \macro{\childdoc}
% The deprecated macro |\childdoc| is a legacy version of |\childdocmain|:
%    \begin{macrocode}
\newcommand{\childdoc}{\childdocmain}
%    \end{macrocode}

% \macro{\childdocredirect}
% The deprecated macro |\childdocredirect| is a legacy version
% of |\childdocforward| and |\childdocforwardprefix|:
%    \begin{macrocode}
\newcommand{\childdocredirect}[2][]
{
  \begingroup
    \if?#1?
      \def\childdoctmp{\childdocforward{#2}}
    \else
      \def\childdoctmp{\childdocforwardprefix{#1}{#2}}
    \fi
    \expandafter
  \endgroup
  \childdoctmp
}
%    \end{macrocode}

%\iffalse
%</package>
%\fi
%
\endinput
|\\
|\childdocforward{|\textit{main}|}|
\end{tabular}
\end{center}
%
Likewise, the following files |final|\textit{nn}|.tex|
compile the final version of the child document
|child|\textit{nn}|.tex|:
%
\begin{center}
\begin{tabular}{l}
|\def\version{final}|\\
|% \iffalse
%
% childdoc.dtx Copyright (C) 2017-2018 Niklas Beisert
%
% This work may be distributed and/or modified under the
% conditions of the LaTeX Project Public License, either version 1.3
% of this license or (at your option) any later version.
% The latest version of this license is in
%   http://www.latex-project.org/lppl.txt
% and version 1.3 or later is part of all distributions of LaTeX
% version 2005/12/01 or later.
%
% This work has the LPPL maintenance status `maintained'.
%
% The Current Maintainer of this work is Niklas Beisert.
%
% This work consists of the files childdoc.dtx and childdoc.ins
% and the derived files childdoc.def and cdocsamp.tex with
% cdocsch1.tex, cdocsch2.tex, cdocsdrf.tex, cdocsfn1.tex, cdocsfn2.tex.
%
%<package>\ifdefined\childdocmain\endinput\fi
%<package>\ProvidesFile{childdoc.def}[2018/12/30 v2.0 child document driver]
%<samplemain>\ProvidesFile{cdocsamp.tex}[2018/12/30 v2.0 sample for childdoc]
%<*driver>
%\ProvidesFile{childdoc.drv}[2018/12/30 v2.0 childdoc reference manual file]
\PassOptionsToClass{10pt,a4paper}{article}
\documentclass{ltxdoc}

\usepackage[margin=35mm]{geometry}
\usepackage{hyperref}
\usepackage{hyperxmp}
\usepackage[usenames]{color}

\hypersetup{colorlinks=true}
\hypersetup{pdfstartview=FitH}
\hypersetup{pdfpagemode=UseNone}
\hypersetup{pdfsource={}}
\hypersetup{pdflang={en-UK}}
\hypersetup{pdfcopyright={Copyright 2017-2018 Niklas Beisert.
  This work may be distributed and/or modified under the
  conditions of the LaTeX Project Public License, either version 1.3
  of this license or (at your option) any later version.}}
\hypersetup{pdflicenseurl={http://www.latex-project.org/lppl.txt}}
\hypersetup{pdfcontactaddress={ETH Zurich, ITP, HIT K,
  Wolfgang-Pauli-Strasse 27}}
\hypersetup{pdfcontactpostcode={8093}}
\hypersetup{pdfcontactcity={Zurich}}
\hypersetup{pdfcontactcountry={Switzerland}}
\hypersetup{pdfcontactemail={nbeisert@itp.phys.ethz.ch}}
\hypersetup{pdfcontacturl={http://people.phys.ethz.ch/\xmptilde nbeisert/}}

\newcommand{\secref}[1]{\hyperref[#1]{section \ref*{#1}}}

\parskip1ex
\parindent0pt
\let\olditemize\itemize
\def\itemize{\olditemize\parskip0pt}

\begin{document}

\title{The \textsf{childdoc} Package}
\hypersetup{pdftitle={The childdoc Package}}
\author{Niklas Beisert\\[2ex]
  Institut f\"ur Theoretische Physik\\
  Eidgen\"ossische Technische Hochschule Z\"urich\\
  Wolfgang-Pauli-Strasse 27, 8093 Z\"urich, Switzerland\\[1ex]
  \href{mailto:nbeisert@itp.phys.ethz.ch}
  {\texttt{nbeisert@itp.phys.ethz.ch}}}
\hypersetup{pdfauthor={Niklas Beisert}}
\hypersetup{pdfsubject={Manual for the LaTeX2e Package childdoc}}
\date{30 December 2018, \textsf{v2.0}}
\maketitle

\begin{abstract}\noindent
\textsf{childdoc} is a \LaTeXe{} package
that enables the direct compilation
of document sections included by |\include|
to individual files.
\end{abstract}

\begingroup
\parskip0ex
\tableofcontents
\endgroup

%%%%%%%%%%%%%%%%%%%%%%%%%%%%%%%%%%%%%%%%%%%%%%%%%%%%%%%%%%%%%%%%%%%%%%%%%%%%%%%%
%%%%%%%%%%%%%%%%%%%%%%%%%%%%%%%%%%%%%%%%%%%%%%%%%%%%%%%%%%%%%%%%%%%%%%%%%%%%%%%%
\section{Introduction}

\LaTeX{} provides a mechanism to structure a large document (such as a book)
into a main file and several child files (containing the chapters)
using the |\include| command.
This mechanism is beneficial for documents
which span hundreds of pages in order to
make the source file(s) more manageable.
Moreover, compilation can be restricted to
selected child files by means of the |\includeonly| command.
The latter feature can be used to reduce the compilation time while editing
(this was significantly more useful in the earlier days of \LaTeX{})
or to generate a smaller document which is easier to navigate.
Another application of |\includeonly| is to generate
documents consisting of selected parts of the complete document.

However, there are a few drawbacks of the plain |\include| mechanism:
\begin{itemize}
\item
The child files cannot be compiled on their own,
they can only be compiled via the main file.
A naive editing environment
(such as a text editor with an option
to have the current file processed by \LaTeX)
may require one to switch to the main file before compiling;
attempting to compile the child file produces errors.
\item
The main file must be modified (each time)
to adjust the |\includeonly| command
to the present needs. This easily leaves the main file in a messy state.
\item
The generated document will always carry the filename
of the main document. This is inconvenient if
several child files are to be compiled and
to be kept for distribution.
\end{itemize}

The present package provides a simple interface
to make child files individually compilable by \LaTeX{}.
Compiling a child file then has the same effect as compiling
the main file with an |\includeonly| command
to select the appropriate child.
Moreover the generated document will carry the name of the child
rather than the main file.
This resolves all three above issues.

This feature is meant to make the editing of books,
thesis documents and lecture notes somewhat more convenient.
However, the package can also be used efficiently for
composing a series of documents (such as exercise sheets)
which are typically distributed individually.
It then assists the author in generating the individual documents
(potentially in different versions)
as well as a document containing the collected series.
Another application is in developing style files
or other kinds of included material
where compilation of the style file could redirect
to a sample or test file.

%%%%%%%%%%%%%%%%%%%%%%%%%%%%%%%%%%%%%%%%%%%%%%%%%%%%%%%%%%%%%%%%%%%%%%%%%%%%%%%%
%%%%%%%%%%%%%%%%%%%%%%%%%%%%%%%%%%%%%%%%%%%%%%%%%%%%%%%%%%%%%%%%%%%%%%%%%%%%%%%%
\section{Usage}

First of all, the package \textsf{childdoc} is \emph{not} a standard
\LaTeXe{} |.sty| style file! Therefore it needs to be invoked in
a non-standard way.

%%%%%%%%%%%%%%%%%%%%%%%%%%%%%%%%%%%%%%%%%%%%%%%%%%%%%%%%%%%%%%%%%%%%%%%%%%%%%%%%
\subsection{Included Files}
\label{sec:include}

%%%%%%%%%%%%%%%%%%%%%%%%%%%%%%%%%%%%%%%%
\DescribeMacro{\childdocmain}
To use the package, add the commands
\begin{center}
\begin{tabular}{l}
|\input{childdoc.def}|\\
|\childdocmain{}|\\
\end{tabular}
\end{center}
at the very top of the main \LaTeX{} file,
in particular \emph{before} the |\documentclass| statement!
The argument of |\childdocmain| should be left empty
(but it must be present).

%%%%%%%%%%%%%%%%%%%%%%%%%%%%%%%%%%%%%%%%
\DescribeMacro{\childdocof}
Furthermore, add the commands
\begin{center}
\begin{tabular}{l}
|\input{childdoc.def}|\\
|\childdocof{|\textit{main}|}|\\
\end{tabular}
\end{center}
at the top of every child file \textit{child}
which is included by |\include{|\textit{child}|}|
from within the main file
(or at least for those files to be compiled individually).
The argument \textit{main} must be the filename of the main file.

There are a couple of
considerations in setting up the main and child documents:

%%%%%%%%%%%%%%%%%%%%%%%%%%%%%%%%%%%%%%%%
\paragraph{Restrictions.}

Please note the following restrictions:
\begin{itemize}
\item
|\childdocmain| must be called with one argument \textit{main}
to ensure compatibility with earlier version of the package.
It must either be empty (|\childdocmain{}|)
or precisely match the filename of the main file in which it is specified.
See \secref{sec:detection} for further information.
\item
The filename \textit{main} must be specified without the |.tex| extension.
\item
The filename \textit{main} is case sensitive
(even in case-insensitive file systems)
due to internal string comparison.
\item
The argument \textit{main} should be fully expanded, it cannot be a macro.
\item
Subdirectories and special characters should be avoided in filenames.
\item
The command |\childdocmain{|\textit{main}|}| must be followed by a whitespace.
It should not be followed immediately by another command
or by a comment mark `|%|'.
This is because the \TeX{} parser reads the token immediately following
the argument of |\childdocmain| and puts it
at the beginning of every child section;
however, a white\-space is ignored.
\end{itemize}

%%%%%%%%%%%%%%%%%%%%%%%%%%%%%%%%%%%%%%%%
\paragraph{Content of Main File.}

It is advisable to place all content in the child files included by |\include|.
Any output contained in the main file will appear in all child documents
unless suppressed manually;
it cannot be suppressed automatically by the |\includeonly| directive
and thus should normally be avoided.
A method to include some content in the main file
by means of conditional processing is described in \secref{sec:conditional}.

%%%%%%%%%%%%%%%%%%%%%%%%%%%%%%%%%%%%%%%%
\paragraph{Page Numbering.}

When only a part of the document is compiled,
the appropriate numbering of pages
(as well as other status parameters)
is determined from the |.aux| files.
The latter contain information from previous passes.
However this information needs to propagate through
all intermediate child documents.
Therefore the page numbering in child documents may well
be inconsistent until the complete document is compiled at least once.

A useful (if unconventional) way to always ensure a consistent
page numbering is to restart the numbering in each child document
and denote the pages by `\textit{child}|.|\textit{page}'
where \textit{child} represents the chapter/section number of the child file.
This can be achieved by the command
|\numberwithin{page}{|\textit{child}|}|
of the \textsf{amsmath} package
where \textit{child} can be |chapter| or |section|
depending on the chosen structuring.
Alternatively, one can modify the macro |\thepage| appropriately
and reset the counter |page| at the start of each child file.

%%%%%%%%%%%%%%%%%%%%%%%%%%%%%%%%%%%%%%%%%%%%%%%%%%%%%%%%%%%%%%%%%%%%%%%%%%%%%%%%
\subsection{Conditional Processing}
\label{sec:conditional}

The package provides a mechanism to compile different versions
of a document. To customise the versions further some conditional processing
can come in handy to distinguish which version is being compiled.
The package provides two macros to describe the compilation context:

%%%%%%%%%%%%%%%%%%%%%%%%%%%%%%%%%%%%%%%%
\DescribeMacro{\ifchilddoc}
The conditional |\ifchilddoc| distinguishes between the compilation of
child documents and the main document:
%
\begin{center}
|\ifchilddoc |\textit{child-code}| |[|\||else |\textit{main-code}]| \||fi|
\end{center}

%%%%%%%%%%%%%%%%%%%%%%%%%%%%%%%%%%%%%%%%
\DescribeMacro{\childdocname}
\DescribeMacro{\childdocjob}
The macro |\childdocname| contains the filename (without extension)
of the main or child file being processed.
Note that |\childdocjob| will always contain the name of the main file.

%%%%%%%%%%%%%%%%%%%%%%%%%%%%%%%%%%%%%%%%
\paragraph{Title Page.}

Conditional processing can be used to include a title or banner page
in the main document when proper precautions are taken.
Importantly, the code in the main file should ensure that the page counter
(as well as other status parameters which are stored in the |.aux| files)
takes the same value after the conditional processing.
Otherwise the page numbers may take divergent values
depending on which part is compiled.

For example, a title page could be declared by:
%
\begin{center}
\begin{tabular}{l}
|\ifchilddoc\||else|\\
|\addtocounter{page}{-1}|\\
\textit{code for title page}\\
|\newpage|\\
|\||fi|
\end{tabular}
\end{center}
%
A banner page for the child documents can be generated by:
%
\begin{center}
\begin{tabular}{l}
|\ifchilddoc|\\
|\addtocounter{page}{-1}|\\
\textit{code for banner page}\\
|\newpage|\\
|\||fi|
\end{tabular}
\end{center}
%
Here one could write a message such as:
\begin{center}
|This is the part \childdocname{} of \childdocjob{}.|
\end{center}

%%%%%%%%%%%%%%%%%%%%%%%%%%%%%%%%%%%%%%%%%%%%%%%%%%%%%%%%%%%%%%%%%%%%%%%%%%%%%%%%
\subsection{Flags}
\label{sec:flags}

The package makes it easy to generate different versions
of the main or child documents.
To this end compilation flags can be defined
and assigned different default values.
They will be particularly useful in conjunction
with the forwarding mechanism described in \secref{sec:forward}.

For example, it may be useful to have a flag |\version|
which can be set to |draft| or |final|.
The document source will contain some conditional code
depending on the value of |\version|.
Suppose further, the flag should default to |final| for the main file
and to |draft| for child files
which is a natural assignment for editing the document.
This is achieved by placing the following code
in the preamble of the main document
(below the |\childdocmain| directive):
%
\begin{center}
\begin{tabular}{l}
|\ifchilddoc|\\
|\providecommand{\version}{draft}|\\
|\||else|\\
|\providecommand{\version}{final}|\\
|\||fi|
\end{tabular}
\end{center}
%
The definition by |\providecommand| makes sure
that previous definitions are not overwritten.
Further statements |\providecommand{\version}{...}|
can thus be added before the above code to override it.

For the main file, one might add a line
(between |\childdocmain| and the above block)
%
\begin{center}
|%\ifchilddoc\||else\providecommand{\version}{draft}\||fi|
\end{center}
%
which can be uncommented to produce a draft version.
Likewise one can add a line to the very top of a child file
(above the |\childdocof{|\textit{main}|}| directive)
%
\begin{center}
|%\providecommand{\version}{final}|
\end{center}
%
which can be uncommented to produce the final version of this child document.

%%%%%%%%%%%%%%%%%%%%%%%%%%%%%%%%%%%%%%%%%%%%%%%%%%%%%%%%%%%%%%%%%%%%%%%%%%%%%%%%
\subsection{Forwarding}
\label{sec:forward}

Different versions of the main or child documents
using compilation flags as described in \secref{sec:flags}
can be (permanently) stored in different files
for convenient compilation, viewing and distribution.
To this end, the package defines a command
to pass on compilation to a different file:

%%%%%%%%%%%%%%%%%%%%%%%%%%%%%%%%%%%%%%%%
\DescribeMacro{\childdocforward}
The command |\childdocforward| redirects processing to
another source file:
%
\begin{center}
\begin{tabular}{l}
|\input{childdoc.def}|\\
|\childdocforward[|\textit{main}|]{|\textit{dest}|}|\\
\end{tabular}
\end{center}
%
The argument \textit{dest} is the destination file
(without extension).
It should be the main file or one of the child files.
Note that further \textsf{childdoc} directives
such as |\childdocof| and |\childdocforward|
in the indicated file will be processed in this form.
The optional argument \textit{main}
passes on directly to the main file \textit{main}
while pretending to compile the child \textit{dest}.
This form behaves as if \textit{dest}
issues |\childdocof{|\textit{main}|}| right away,
and no further \textsf{childdoc} directives will be processed.

%%%%%%%%%%%%%%%%%%%%%%%%%%%%%%%%%%%%%%%%
\DescribeMacro{\...prefix}
In the alternative form |\childdocforwardprefix|,
%
\begin{center}
\begin{tabular}{l}
|\input{childdoc.def}|\\
|\childdocforwardprefix[|\textit{main}|]{|\textit{prefix}|}{|\textit{dest}|}|
\end{tabular}
\end{center}
%
the destination file is determined by a pattern
depending on the current file:
To make this work, the current file must be called
`{\textit{prefix}\hspace{0.2em}\textit{suffix}}'
with \textit{prefix} matching precisely the argument.
Processing is then passed on to the file
`{\textit{dest}\hspace{0.2em}\textit{suffix}}'.
Surely, the same effect is achieved by
directly specifying the
argument `{\textit{dest}\hspace{0.2em}\textit{suffix}}'
in the first form.
However, that requires to set up a different file
for each child. With the alternative form of the command
all these files can have exactly the same content
which simplifies setting them up and maintaining them.

For example, the following file |draft.tex|
with a compilation flag |\version| as described in \secref{sec:flags}
compiles the main document as a draft:
%
\begin{center}
\begin{tabular}{l}
|\def\version{draft}|\\
|\input{childdoc.def}|\\
|\childdocforward{|\textit{main}|}|
\end{tabular}
\end{center}
%
Likewise, the following files |final|\textit{nn}|.tex|
compile the final version of the child document
|child|\textit{nn}|.tex|:
%
\begin{center}
\begin{tabular}{l}
|\def\version{final}|\\
|\input{childdoc.def}|\\
|\childdocforwardprefix{final}{child}|
\end{tabular}
\end{center}
%

Note that when several versions of a main file and/or of each child file
are to be generated, it may be convenient to set up a |Makefile| or
shell script to automatise the process.

%%%%%%%%%%%%%%%%%%%%%%%%%%%%%%%%%%%%%%%%%%%%%%%%%%%%%%%%%%%%%%%%%%%%%%%%%%%%%%%%
\subsection{Command Line Processing}
\label{sec:commandline}

The effect of redirection files can also be achieved by invoking
the \LaTeX{} compiler with a more elaborate command line.
Most conveniently this should be done as part
of a shell script or a |Makefile|.

When using \textsf{childdoc} in the main file, the following
command lines effectively perform a redirection
(note that depending on the shell being used,
backslashes may have to be doubled: `|\|' $\to$ `|\\|'):
%
\begin{center}
|... -jobname "|\textit{target}|" |\\|"|[\textit{flags}]%
|\input{childdoc.def}\childdocforward[|\textit{main}|]{|\textit{dest}|}"|
\end{center}
%
Here \textit{target} is the name of the output file,
\textit{main} is the name of the main file
and \textit{dest} is the name of the main or child file to be processed
(all filenames without extensions).
The optional argument \textit{main} can be omitted
if \textit{main} matches \textit{dest}.
Optionally, compilation \textit{flags} can be defined via |\def| commands.
This command line makes the \TeX{} engine believe
it is compiling the file \textit{target}
whose content is specified as the latter parameter.
The provided code then forwards the processing to
\textit{main} or \textit{dest} as described in \secref{sec:forward}.

%%%%%%%%%%%%%%%%%%%%%%%%%%%%%%%%%%%%%%%%%%%%%%%%%%%%%%%%%%%%%%%%%%%%%%%%%%%%%%%%
\subsection{Include by Input}
\label{sec:input}

Including child documents by |\include| has some restrictions by design.
Most notably, the content of a child document always occupies
its own set of pages; pages cannot be shared between child documents.
Usually, this behaviour makes perfect sense
because each child document contain an essential part of the document.
However, in some situations it may be desirable to compose
a document from a collection of parts
without having mandatory page breaks between then.
For this case, the package
provides a mechanism to include parts
by |\input| which can also be processed individually.
However, by construction this mechanism
requires manual handling of the content to be output.

%%%%%%%%%%%%%%%%%%%%%%%%%%%%%%%%%%%%%%%%
\DescribeMacro{\ifchilddocmanual}
The main file should be prepared as usual, see \secref{sec:include}.
However, the document body must make a distinction
between processing of an individual part and of the main document, e.g.:
%
\begin{center}
\begin{tabular}{l}
|\ifchilddocmanual|\\
|\input{\childdocname}|\\
|\||else|\\
\textit{document body with }|\input{|\textit{part}|}|\\
|\||fi|
\end{tabular}
\end{center}
%
The conditional |\ifchilddocmanual| is true whenever
a part to be included by |\input| is being compiled,
and the name of the part is stored in |\childdocname|.

%%%%%%%%%%%%%%%%%%%%%%%%%%%%%%%%%%%%%%%%
\DescribeMacro{\childdocby}
Each part to be included by |\input| should start with:
%
\begin{center}
\begin{tabular}{l}
|\input{childdoc.def}|\\
|\childdocby{|\textit{main}|}|\\
\end{tabular}
\end{center}
%
The directive |\childdocby| is similar to |\childdocof|
described in \secref{sec:include},
but the subsequent selection of content must be done manually.
To that end, both |\ifchilddoc| and |\ifchilddocmanual|
will be true upon processing of a part,
and the name of the part is stored in |\childdocname|.
Note that |\jobname| will be set to the filename of the current part
so that each part receives an individual |.aux| file
that does not interfere with the |.aux| file(s) of the main document.
This behaviour can be altered by the alternative form
|\childdocby[*]{|\textit{main}|}| (with a non-empty optional argument)
which uses the |.aux| file of the main document
by setting |\jobname| to \textit{main}.

%%%%%%%%%%%%%%%%%%%%%%%%%%%%%%%%%%%%%%%%%%%%%%%%%%%%%%%%%%%%%%%%%%%%%%%%%%%%%%%%
\subsection{Driver Development}
\label{sec:driver}

The \textsf{childdoc} mechanism can also be use for the development
of definition files such as \LaTeX{} styles or classes.
This case differs from the above setup with multiple parts
included by |\include| in that no |\includeonly| should be invoked.
This can be achieved by starting the include file
(before |\ProvidesPackage|) with:
%
\begin{center}
\begin{tabular}{l}
|\input{childdoc.def}|\\
|\childdocforward{|\textit{main}|}|\\
\end{tabular}
\end{center}
%
or alternatively with:
%
\begin{center}
\begin{tabular}{l}
|\input{childdoc.def}|\\
|\childdocby{|\textit{main}|}|\\
\end{tabular}
\end{center}
%
Both forms have slightly different effects as described above.
The main file is prepared as usual, see \secref{sec:include}.

%%%%%%%%%%%%%%%%%%%%%%%%%%%%%%%%%%%%%%%%%%%%%%%%%%%%%%%%%%%%%%%%%%%%%%%%%%%%%%%%
\subsection{Legacy Detection}
\label{sec:detection}

The directive |\childdocmain| in the main file can detect
whether the complete document or merely a child is to be compiled
even without using the directive |\childdocof|.
This method is deprecated because it is less robust
and there is no compelling reason to use it;
it is merely provided for backward compatibility
and it may be removed in future versions.

If the detection mechanism is to be used,
it is mandatory to correctly specify
the filename of the main file as the argument of |\childdocmain|:
%
\begin{center}
\begin{tabular}{l}
|\input{childdoc.def}|\\
|\childdocmain{|\textit{main}|}|\\
\end{tabular}
\end{center}
%
If |\jobname| does not match the argument \textit{main} of |\childdocmain|,
it is assumed that |\jobname| points to the child file to be compiled.
When using |\childdocmain| with the main file specified as argument,
it suffices to start a child file
with just |\input{|\textit{main}|}|
without loading of the package and using |\childdocof|.
If instead all processing is done
with the appropriate \textsf{childdoc} directives,
the argument of \textit{main} of |\childdocmain| can be empty.

An alternative version of the command line processing described
in \secref{sec:commandline} using the detection mechanism reads:
%
\begin{center}
|... -jobname "|\textit{target}|" "|[\textit{flags}]%
[|\def\jobname{|\textit{dest}|}|]|\input{|\textit{main}|}"|
\end{center}

%%%%%%%%%%%%%%%%%%%%%%%%%%%%%%%%%%%%%%%%%%%%%%%%%%%%%%%%%%%%%%%%%%%%%%%%%%%%%%%%
\subsection{Manual Code}
\label{sec:manual}

In case one cannot be certain whether the definitions file |childdoc.def|
is installed on the target \TeX{} distribution
and one prefers not to ship it,
it is conceivable to paste a few relevant commands into the sources.

To that end, drop all statements |\input{childdoc.def}|
and perform the replacements as outlined below.
Instead of |\childdocmain{|\textit{main}|}| add the following code
to the top of the main file:
%
\begin{center}
\begin{tabular}{l}
|\||ifdefined\childdocname\endinput\||fi\newif\ifchilddoc|\\
|\edef\childdocname{\scantokens\expandafter{\jobname\noexpand}}|\\
|\def\childdocmain{|\textit{main}|}\||ifx\childdocmain\childdocname\||else|\\
|\childdoctrue\includeonly{\childdocname}\let\jobname\childdocmain\||fi|\\
\end{tabular}
\end{center}
%
Instead of |\childdocof{|\textit{main}|}| just include the main file
at the top of each child file:
%
\begin{center}
|\input{|\textit{main}|}|
\end{center}
%
A simple redirection |\childdocforward{|\textit{dest}|}| is achieved by:
%
\begin{center}
|\def\jobname{|\textit{dest}|}\input{\jobname}|
\end{center}
%
The redirection with prefix
|\childdocforwardprefix[|\textit{prefix}|]{|\textit{dest}|}|
is accomplished by:
%
\begin{center}
\begin{tabular}{l}
|{\edef\jobname{\scantokens\expandafter{\jobname\noexpand}}|\\
|\def\redirectjob |\textit{prefix}|#1~~~{\gdef\jobname{|\textit{dest}|#1}}|\\
|\expandafter\redirectjob\jobname~~~}\input{\jobname}|
\end{tabular}
\end{center}

In an alternative approach,
child documents can be compiled by a specific command line
without additional code or specific definitions:
%
\begin{center}
|... -jobname "|\textit{target}|" "|[\textit{flags}]%
|\includeonly{|\textit{dest}|}\input{|\textit{main}|}"|
\end{center}
%

%%%%%%%%%%%%%%%%%%%%%%%%%%%%%%%%%%%%%%%%%%%%%%%%%%%%%%%%%%%%%%%%%%%%%%%%%%%%%%%%
%%%%%%%%%%%%%%%%%%%%%%%%%%%%%%%%%%%%%%%%%%%%%%%%%%%%%%%%%%%%%%%%%%%%%%%%%%%%%%%%
\section{Information}

%%%%%%%%%%%%%%%%%%%%%%%%%%%%%%%%%%%%%%%%%%%%%%%%%%%%%%%%%%%%%%%%%%%%%%%%%%%%%%%%
\subsection{Copyright}

Copyright \copyright{} 2017--2018 Niklas Beisert

This work may be distributed and/or modified under the
conditions of the \LaTeX{} Project Public License, either version 1.3
of this license or (at your option) any later version.
The latest version of this license is in
  \url{http://www.latex-project.org/lppl.txt}
and version 1.3 or later is part of all distributions of \LaTeX{}
version 2005/12/01 or later.

This work has the LPPL maintenance status `maintained'.

The Current Maintainer of this work is Niklas Beisert.

This work consists of the files |README.txt|, |childdoc.ins| and |childdoc.dtx|
as well as the derived files |childdoc.def|, |cdocsamp.tex|
with |cdocsch1.tex|, |cdocsch2.tex|, |cdocspt3.tex|, |cdocspt4.tex|,
|cdocsdrf.tex|, |cdocsfn1.tex|, |cdocsfn2.tex|
as well as |childdoc.pdf|.

%%%%%%%%%%%%%%%%%%%%%%%%%%%%%%%%%%%%%%%%%%%%%%%%%%%%%%%%%%%%%%%%%%%%%%%%%%%%%%%%
\subsection{Files and Installation}

The package consists of the files:
%
\begin{center}
\begin{tabular}{ll}
    |README.txt|   & readme file \\
    |childdoc.ins| & installation file \\
    |childdoc.dtx| & source file \\
    |childdoc.def| & definition file \\
    |cdocsamp.tex| & sample main file \\
    |cdocsch1.tex| & sample include file \\
    |cdocsch2.tex| & sample include file \\
    |cdocspt3.tex| & sample part file \\
    |cdocspt4.tex| & sample part file \\
    |cdocsdrf.tex| & sample redirection file \\
    |cdocsfn1.tex| & sample redirection file \\
    |cdocsfn2.tex| & sample redirection file \\
    |childdoc.pdf| & manual
\end{tabular}
\end{center}
%
The distribution consists of the files
|README.txt|, |childdoc.ins| and |childdoc.dtx|.
%
\begin{itemize}
\item
Run (pdf)\LaTeX{} on |childdoc.dtx|
to compile the manual |childdoc.pdf| (this file).
\item
Run \LaTeX{} on |childdoc.ins| to create the definitions file |childdoc.def|
and the sample |cdocsamp.tex| with include files
|cdocsch1.tex|, |cdocsch2.tex|, |cdocspt3.tex|, |cdocspt4.tex|,
|cdocsdrf.tex|, |cdocsfn1.tex|, |cdocsfn2.tex|.
Then copy the file |childdoc.def| to an appropriate directory of your \LaTeX{}
distribution, e.g.\ \textit{texmf-root}|/tex/latex/childdoc|.
\end{itemize}

%%%%%%%%%%%%%%%%%%%%%%%%%%%%%%%%%%%%%%%%%%%%%%%%%%%%%%%%%%%%%%%%%%%%%%%%%%%%%%%%
\subsection{Related CTAN Packages}

There are several other packages which offer a similar functionality:
%
\begin{itemize}
\item
The packages
\href{http://ctan.org/pkg/docmute}{\textsf{docmute}},
\href{http://ctan.org/pkg/includex}{\textsf{includex}} and
\href{http://ctan.org/pkg/standalone}{\textsf{standalone}}
provide commands to include only the document body of
a child file thus allowing both files to be compiled individually.
\item
The packages \href{http://ctan.org/pkg/subdocs}{\textsf{subdocs}}
and \href{http://ctan.org/pkg/subfiles}{\textsf{subfiles}}
provide structures in which the main and child documents can be
encapsulated and allowing them to be compiled individually.
The inclusion mechanism is different from the conventional |\include|.
\item
The package \href{http://ctan.org/pkg/combine}{\textsf{combine}}
is an elaborate solution to combine several documents into one.
\end{itemize}
%
See also the CTAN topic \href{http://ctan.org/topic/subdocs}{\textsf{subdocs}}
for further related packages.
The present package differs from the above solutions in that
a document structure constructed with the conventional |\include| mechanism
just needs two extra commands at the top of every file
such that all constituent files can be compiled individually.

%%%%%%%%%%%%%%%%%%%%%%%%%%%%%%%%%%%%%%%%%%%%%%%%%%%%%%%%%%%%%%%%%%%%%%%%%%%%%%%%
%\subsection{Feature Suggestions}
%
%The following is a list of features which may be useful for future
%versions of this package:
%%
%\begin{itemize}
%\item
%\ldots
%\end{itemize}

%%%%%%%%%%%%%%%%%%%%%%%%%%%%%%%%%%%%%%%%%%%%%%%%%%%%%%%%%%%%%%%%%%%%%%%%%%%%%%%%
\subsection{Revision History}

%%%%%%%%%%%%%%%%%%%%%%%%%%%%%%%%%%%%%%%%
\paragraph{v2.0:} 2018/12/30

\begin{itemize}
\item
immediate forward processing
\item
added |\childdocby| mechanism
\item
manual restructured
\end{itemize}

%%%%%%%%%%%%%%%%%%%%%%%%%%%%%%%%%%%%%%%%
\paragraph{v1.6:} 2018/01/17

\begin{itemize}
\item
application for development of include files
\item
corrections to manual
\end{itemize}

%%%%%%%%%%%%%%%%%%%%%%%%%%%%%%%%%%%%%%%%
\paragraph{v1.5:} 2017/05/21

\begin{itemize}
\item
more complete structuring introduced
\item
|\childdocof| introduced
\item
|\childdoc| renamed to |\childdocmain|
\item
|\childredirect| renamed to |\childdocforward| and |\childdocforwardprefix|
and functionality expanded
\end{itemize}

%%%%%%%%%%%%%%%%%%%%%%%%%%%%%%%%%%%%%%%%
\paragraph{v1.0:} 2017/04/27

\begin{itemize}
\item
manual and install package
\item
first version published on CTAN
\end{itemize}

%%%%%%%%%%%%%%%%%%%%%%%%%%%%%%%%%%%%%%%%
\paragraph{v0.6:} 2017/04/26

\begin{itemize}
\item
redirection mechanism added
\end{itemize}

%%%%%%%%%%%%%%%%%%%%%%%%%%%%%%%%%%%%%%%%
\paragraph{v0.5:} 2017/04/26

\begin{itemize}
\item
functionality in definition file
\end{itemize}


%%%%%%%%%%%%%%%%%%%%%%%%%%%%%%%%%%%%%%%%%%%%%%%%%%%%%%%%%%%%%%%%%%%%%%%%%%%%%%%%
%%%%%%%%%%%%%%%%%%%%%%%%%%%%%%%%%%%%%%%%%%%%%%%%%%%%%%%%%%%%%%%%%%%%%%%%%%%%%%%%
%%%%%%%%%%%%%%%%%%%%%%%%%%%%%%%%%%%%%%%%%%%%%%%%%%%%%%%%%%%%%%%%%%%%%%%%%%%%%%%%
\appendix

\settowidth\MacroIndent{\rmfamily\scriptsize 000\ }

 \DocInput{childdoc.dtx}

\end{document}
%</driver>
% \fi
%
% %%%%%%%%%%%%%%%%%%%%%%%%%%%%%%%%%%%%%%%%%%%%%%%%%%%%%%%%%%%%%%%%%%%%%%%%%%%%%%
% %%%%%%%%%%%%%%%%%%%%%%%%%%%%%%%%%%%%%%%%%%%%%%%%%%%%%%%%%%%%%%%%%%%%%%%%%%%%%%
% \section{Sample}
%\iffalse
%<*samplemain>
%\fi
%
% The following presents a sample document
% with two chapters, two parts, a title page,
% a compile flag as well as three forwarding files to set the flag.
% It consists of eight |.tex| files:
% \begin{center}
% \begin{tabular}{ll}
% |cdocsamp.tex|&main file\\
% |cdocsch1.tex|&include file for chapter 1\\
% |cdocsch2.tex|&include file for chapter 2\\
% |cdocspt3.tex|&include file for part 3\\
% |cdocspt4.tex|&include file for part 4\\
% |cdocsdrf.tex|&forwarding file for main file in draft mode\\
% |cdocsfi1.tex|&forwarding file for final version of chapter 1\\
% |cdocsfi2.tex|&forwarding file for final version of chapter 2\\
% \end{tabular}
% \end{center}
% Each of the eight files can be compiled directly by the \LaTeX{} compiler.
%
% %%%%%%%%%%%%%%%%%%%%%%%%%%%%%%%%%%%%%%
% \paragraph{Main File.}
%
% The main file is called |cdocsamp.tex|.
%
% Load the \textsf{childdoc} definitions and
% declare the filename for the main document:
%    \begin{macrocode}
\input{childdoc.def}
\childdocmain{}
%    \end{macrocode}

% Optional override for |\version| flag:
%    \begin{macrocode}
%%\ifchilddoc\else\providecommand{\version}{draft}\fi
%    \end{macrocode}

% Define the default values for the |\version| flag
% (|final| for the main file and |draft| for childs):
%    \begin{macrocode}
\ifchilddoc
\providecommand{\version}{draft}
\else
\providecommand{\version}{final}
\fi
%    \end{macrocode}

% Load the standard document class:
%    \begin{macrocode}
\documentclass[12pt]{article}
%    \end{macrocode}

% Start the document body:
%    \begin{macrocode}
\begin{document}
%    \end{macrocode}

% Declare a title page.
% Print title, part of document being processed and version flag:
%    \begin{macrocode}
\addtocounter{page}{-1}
\begin{center}
{\LARGE\bfseries{}childdoc example\par}
\vspace{1cm}
\ifchilddoc
\ifchilddocmanual part\else chapter\fi:
`\childdocname' of `\childdocjob'\par
\else
main document: `\childdocjob'\par
\fi
version: \version\par
\end{center}
\newpage
%    \end{macrocode}

% Manually include selected file,
% otherwise process as usual:
%    \begin{macrocode}
\ifchilddocmanual
\section*{part `\childdocname'}
\input{\childdocname}
\else
%    \end{macrocode}

% Include the two chapters:
%    \begin{macrocode}
\include{cdocsch1}
\include{cdocsch2}
%    \end{macrocode}

% Include the two parts unless only chapters should be displayed:
%    \begin{macrocode}
\ifchilddoc\else
\section{part three}
\input{cdocspt3}
\section{part four}
\input{cdocspt4}
\fi
%    \end{macrocode}

% Process as usual until here:
%    \begin{macrocode}
\fi
%    \end{macrocode}

% End of document body:
%    \begin{macrocode}
\end{document}
%    \end{macrocode}
%\iffalse
%</samplemain>
%\fi
%
% %%%%%%%%%%%%%%%%%%%%%%%%%%%%%%%%%%%%%%
% \paragraph{Chapter Include Files.}
%
% The include files are called |cdocsch1.tex| and |cdocsch2.tex|.
%
%\iffalse
%<*samplechap1|samplechap2>
%\fi

% Optional override for |\version| flag:
%    \begin{macrocode}
%%\providecommand{\version}{final}
%    \end{macrocode}

% Include the main document:
%    \begin{macrocode}
\input{childdoc.def}
\childdocof{cdocsamp}
%    \end{macrocode}

%\iffalse
%</samplechap1|samplechap2>
%\fi
%
%\iffalse
%<*samplechap1>
%\fi
% Some text for chapter 1:
%    \begin{macrocode}
\section{one}
some text in chapter one
%    \end{macrocode}

%\iffalse
%</samplechap1>
%\fi
% Some text for chapter 2:
%\iffalse
%<*samplechap2>
%\fi
%    \begin{macrocode}
\section{two}
more text in chapter two
%    \end{macrocode}

%\iffalse
%</samplechap2>
%\fi
%
% %%%%%%%%%%%%%%%%%%%%%%%%%%%%%%%%%%%%%%
% \paragraph{Part Include Files.}
%
% The include files are called |cdocspt3.tex| and |cdocspt4.tex|.
%
%\iffalse
%<*samplepart3|samplepart4>
%\fi

% Optional override for |\version| flag:
%    \begin{macrocode}
%%\providecommand{\version}{final}
%    \end{macrocode}

% Include the main document:
%    \begin{macrocode}
\input{childdoc.def}
\childdocby{cdocsamp}
%    \end{macrocode}

%\iffalse
%</samplepart3|samplepart4>
%\fi
%
%\iffalse
%<*samplepart3>
%\fi
% Some text for part 3:
%    \begin{macrocode}
some text in part three
%    \end{macrocode}

%\iffalse
%</samplepart3>
%\fi
% Some text for part 4:
%\iffalse
%<*samplepart4>
%\fi
%    \begin{macrocode}
more text in part four
%    \end{macrocode}

%\iffalse
%</samplepart4>
%\fi
%
% %%%%%%%%%%%%%%%%%%%%%%%%%%%%%%%%%%%%%%
% \paragraph{Forwarding for a Complete Draft.}
%
% The following forwarding file |cdocsdrf.tex|
% compiles the main document in draft mode:
%\iffalse
%<*sampledraft>
%\fi
%    \begin{macrocode}
\def\version{draft}
\input{childdoc.def}
\childdocforward{cdocsamp}
%    \end{macrocode}

%\iffalse
%</sampledraft>
%\fi
%
% %%%%%%%%%%%%%%%%%%%%%%%%%%%%%%%%%%%%%%
% \paragraph{Forwarding for Final Version of the Chapters.}
%
% The following forwarding files |cdocsfn1.tex| and |cdocsfn2.tex|
% (with identical content)
% compile the final versions of the child documents
% |cdocsch1.tex| and |cdocsch2.tex|, respectively:
%\iffalse
%<*samplefinal>
%\fi
%    \begin{macrocode}
\def\version{final}
\input{childdoc.def}
\childdocforwardprefix[cdocsamp]{cdocsfn}{cdocsch}
%    \end{macrocode}

%\iffalse
%</samplefinal>
%\fi
%
% %%%%%%%%%%%%%%%%%%%%%%%%%%%%%%%%%%%%%%
% \paragraph{Command Line Processing.}
%
% The following three command lines generate the output files
% |cdocscld|, |cdocscl1| and |cdocscl2|
% which should be identical to
% |cdocsdrf|, |cdocsch1| and |cdocsfn2|, respectively:
% \begin{center}
% \begin{tabular}{l}
% |latex -jobname cdocscld \|\\
% |  "\def\version{draft}\input{childdoc.def}\childdocforward{cdocsamp}"|\\
% |latex -jobname cdocscl1 \|\\
% |  "\input{childdoc.def}\childdocforward[cdocsamp]{cdocsch1}"|\\
% |latex -jobname cdocscl2 \|\\
% |  "\def\version{final}\input{childdoc.def}\childdocforward{cdocsch2}"|
% \end{tabular}
% \end{center}
% Note that the trailing backslash on each first line
% merely continues the input to the second line
% (for convenient cut ant paste).
% Furthermore, the command |latex| can be replaced by any
% of its alternative versions such as |pdflatex|.
%
% %%%%%%%%%%%%%%%%%%%%%%%%%%%%%%%%%%%%%%%%%%%%%%%%%%%%%%%%%%%%%%%%%%%%%%%%%%%%%%
% %%%%%%%%%%%%%%%%%%%%%%%%%%%%%%%%%%%%%%%%%%%%%%%%%%%%%%%%%%%%%%%%%%%%%%%%%%%%%%
% \section{Implementation}
%\iffalse
%<*package>
%\fi
%
% This section describes the definitions file |childdoc.def|.

% The definitions cannot be loaded using |\usepackage| or |\RequirePackage|
% which has a mechanism to prevent loading a style file more than once.
% When loading the definitions by means of |\input|
% multiple instances have to be prevented manually:
%\iffalse
%This code needs to be before the `\ProvidesFile' directive
%which is defined at the beginning of this file.
%Therefore it is also placed there and commented out here.
%</package>
%<*discard>
%\fi
%    \begin{macrocode}
\ifdefined\childdocmain\endinput\fi
%    \end{macrocode}
%\iffalse
%</discard>
%<*package>
%\fi
%
% \macro{\ifchilddoc}
% \macro{\ifchilddocmanual}
% The conditional |\ifchilddoc| tells whether a
% child (true) or main (false) document is being compiled.
% The conditional |\ifchilddocmanual| tells whether
% the |\includeonly| mechanism is used (false) or
% the selection of child files must be performed manually (true).
% The definitions initialise to false:
%    \begin{macrocode}
\newif\ifchilddoc
\newif\ifchilddocmanual
%    \end{macrocode}

% \macro{\childdocname}
% \macro{\childdocjob}
% The macro |\childdocname| stores the name of the main document
% to be compiled. The macro |\childdocjob| stores the name of
% the document on which the \LaTeX{} compiler was originally invoked.
% The content of |\jobname| cannot be compared
% to filenames specified in the source due to different catcodes.
% The following code rescans |\jobname|, stores the result
% in |\childdocname| and saves a copy in |\childdocjob|:
%    \begin{macrocode}
\edef\childdocname{\scantokens\expandafter{\jobname\noexpand}}
\let\childdocjob\childdocname
%    \end{macrocode}

% \macro{\childdocdisable}
% The macro |\childdocdisable| prevents the main file
% from being processed more than once.
% At this stage, the main document command |\childdocmain|
% is assumed to be called once again where it should do nothing.
% Any subsequent call to it should prevent
% a secondary processing of the main document
% It overwrites the forwarding commands
% |\childdocof| and |\childdocforward|
% with empty macros to prevent further inclusions of the main document:
%    \begin{macrocode}
\newcommand{\childdocdisable}
{
  \renewcommand{\childdocmain}[1]{\renewcommand{\childdocmain}[1]{\endinput}}
  \renewcommand{\childdocof}[1]{}
  \renewcommand{\childdocby}[2][]{}
  \renewcommand{\childdocforward}[2][]{}
  \renewcommand{\childdocdisable}{}
}
%    \end{macrocode}

% \macro{\childdocmain}
% The macro |\childdocmain| is to be called at the top of the main file
% with nothing or the main filename (without extension) as argument.
% First, it breaks loops.
% If the argument is not empty and does not match |\childdocname|
% (which is set by the first inclusion of |childdoc.def|),
% |\ifchilddoc| is set to true, |\includeonly| is applied to the child file
% and |\jobname| is set to the main file
% (for proper handling of |.aux| files):
%    \begin{macrocode}
\newcommand{\childdocmain}[1]
{
  \childdocdisable\childdocmain{}
  \if?#1?\else
    \begingroup
      \def\childdoctmp{#1}
      \ifx\childdoctmp\childdocname
        \def\childdoctmp{}
      \else
        \def\childdoctmp
        {
          \childdoctrue
          \includeonly{\childdocname}
          \def\childdocjob{#1}
          \def\jobname{#1}
        }
      \fi
      \expandafter
    \endgroup
    \childdoctmp
  \fi
}
%    \end{macrocode}

% \macro{\childdocof}
% The command |\childdocof| redirects
% compilation to the main file |#1|.
%    \begin{macrocode}
\newcommand{\childdocof}[1]
{
  \childdocdisable
  \childdoctrue
  \includeonly{\childdocname}
  \def\jobname{#1}
  \def\childdocjob{#1}
  \input{#1}
}
%    \end{macrocode}

% \macro{\childdocby}
% The command |\childdocby| ....
%    \begin{macrocode}
\newcommand{\childdocby}[2][]
{
  \childdocdisable
  \childdoctrue
  \childdocmanualtrue
  \if?#1?\else
    \def\jobname{#2}
  \fi
  \def\childdocjob{#2}
  \input{#2}
  \endinput
}
%    \end{macrocode}

% \macro{\childdocforward}
% The command |\childdocforward| redirects
% compilation to the main file or
% (if the optional argument is given) a child file.
% Parameters are set as if the main file
% or a child file starting with |\childdocof| was compiled.
% Then compilation is handed over to the main file:
%    \begin{macrocode}
\newcommand{\childdocforward}[2][]
{
  \begingroup
    \if?#1?
      \def\childdoctmp
      {
        \def\childdocname{#2}
        \def\childdocjob{#2}
        \def\jobname{#2}
        \input{#2}
        \endinput
      }
    \else
      \def\childdoctmp
      {
        \childdocdisable
        \def\childdocname{#2}
        \childdoctrue
        \includeonly{#2}
        \def\childdocjob{#1}
        \def\jobname{#1}
        \input{#1}
        \endinput
      }
    \fi
    \expandafter
  \endgroup
  \childdoctmp
}
%    \end{macrocode}

% \macro{\childdocforwardprefix}
% The command |\childdocforwardprefix| redirects
% compilation to the main or a child file by means of a pattern.
% The prefix |#1| in the current filename is replaced by |#2|
% and the suffix of the current filename is kept
% (it is assumed that the filename does not contain the substring `|~~~|'
% which is used as a delimiter).
% Compilation is handed over to the new file by |\childdocforward|:
%    \begin{macrocode}
\newcommand{\childdocforwardprefix}[3][]
{
  \begingroup
    \def\childdocextract #2##1~~~{\def\childdoctmp{\childdocforward[#1]{#3##1}}}
    \expandafter\childdocextract\childdocname~~~
    \expandafter
  \endgroup
  \childdoctmp
}
%    \end{macrocode}

% \macro{\childdoc}
% The deprecated macro |\childdoc| is a legacy version of |\childdocmain|:
%    \begin{macrocode}
\newcommand{\childdoc}{\childdocmain}
%    \end{macrocode}

% \macro{\childdocredirect}
% The deprecated macro |\childdocredirect| is a legacy version
% of |\childdocforward| and |\childdocforwardprefix|:
%    \begin{macrocode}
\newcommand{\childdocredirect}[2][]
{
  \begingroup
    \if?#1?
      \def\childdoctmp{\childdocforward{#2}}
    \else
      \def\childdoctmp{\childdocforwardprefix{#1}{#2}}
    \fi
    \expandafter
  \endgroup
  \childdoctmp
}
%    \end{macrocode}

%\iffalse
%</package>
%\fi
%
\endinput
|\\
|\childdocforwardprefix{final}{child}|
\end{tabular}
\end{center}
%

Note that when several versions of a main file and/or of each child file
are to be generated, it may be convenient to set up a |Makefile| or
shell script to automatise the process.

%%%%%%%%%%%%%%%%%%%%%%%%%%%%%%%%%%%%%%%%%%%%%%%%%%%%%%%%%%%%%%%%%%%%%%%%%%%%%%%%
\subsection{Command Line Processing}
\label{sec:commandline}

The effect of redirection files can also be achieved by invoking
the \LaTeX{} compiler with a more elaborate command line.
Most conveniently this should be done as part
of a shell script or a |Makefile|.

When using \textsf{childdoc} in the main file, the following
command lines effectively perform a redirection
(note that depending on the shell being used,
backslashes may have to be doubled: `|\|' $\to$ `|\\|'):
%
\begin{center}
|... -jobname "|\textit{target}|" |\\|"|[\textit{flags}]%
|% \iffalse
%
% childdoc.dtx Copyright (C) 2017-2018 Niklas Beisert
%
% This work may be distributed and/or modified under the
% conditions of the LaTeX Project Public License, either version 1.3
% of this license or (at your option) any later version.
% The latest version of this license is in
%   http://www.latex-project.org/lppl.txt
% and version 1.3 or later is part of all distributions of LaTeX
% version 2005/12/01 or later.
%
% This work has the LPPL maintenance status `maintained'.
%
% The Current Maintainer of this work is Niklas Beisert.
%
% This work consists of the files childdoc.dtx and childdoc.ins
% and the derived files childdoc.def and cdocsamp.tex with
% cdocsch1.tex, cdocsch2.tex, cdocsdrf.tex, cdocsfn1.tex, cdocsfn2.tex.
%
%<package>\ifdefined\childdocmain\endinput\fi
%<package>\ProvidesFile{childdoc.def}[2018/12/30 v2.0 child document driver]
%<samplemain>\ProvidesFile{cdocsamp.tex}[2018/12/30 v2.0 sample for childdoc]
%<*driver>
%\ProvidesFile{childdoc.drv}[2018/12/30 v2.0 childdoc reference manual file]
\PassOptionsToClass{10pt,a4paper}{article}
\documentclass{ltxdoc}

\usepackage[margin=35mm]{geometry}
\usepackage{hyperref}
\usepackage{hyperxmp}
\usepackage[usenames]{color}

\hypersetup{colorlinks=true}
\hypersetup{pdfstartview=FitH}
\hypersetup{pdfpagemode=UseNone}
\hypersetup{pdfsource={}}
\hypersetup{pdflang={en-UK}}
\hypersetup{pdfcopyright={Copyright 2017-2018 Niklas Beisert.
  This work may be distributed and/or modified under the
  conditions of the LaTeX Project Public License, either version 1.3
  of this license or (at your option) any later version.}}
\hypersetup{pdflicenseurl={http://www.latex-project.org/lppl.txt}}
\hypersetup{pdfcontactaddress={ETH Zurich, ITP, HIT K,
  Wolfgang-Pauli-Strasse 27}}
\hypersetup{pdfcontactpostcode={8093}}
\hypersetup{pdfcontactcity={Zurich}}
\hypersetup{pdfcontactcountry={Switzerland}}
\hypersetup{pdfcontactemail={nbeisert@itp.phys.ethz.ch}}
\hypersetup{pdfcontacturl={http://people.phys.ethz.ch/\xmptilde nbeisert/}}

\newcommand{\secref}[1]{\hyperref[#1]{section \ref*{#1}}}

\parskip1ex
\parindent0pt
\let\olditemize\itemize
\def\itemize{\olditemize\parskip0pt}

\begin{document}

\title{The \textsf{childdoc} Package}
\hypersetup{pdftitle={The childdoc Package}}
\author{Niklas Beisert\\[2ex]
  Institut f\"ur Theoretische Physik\\
  Eidgen\"ossische Technische Hochschule Z\"urich\\
  Wolfgang-Pauli-Strasse 27, 8093 Z\"urich, Switzerland\\[1ex]
  \href{mailto:nbeisert@itp.phys.ethz.ch}
  {\texttt{nbeisert@itp.phys.ethz.ch}}}
\hypersetup{pdfauthor={Niklas Beisert}}
\hypersetup{pdfsubject={Manual for the LaTeX2e Package childdoc}}
\date{30 December 2018, \textsf{v2.0}}
\maketitle

\begin{abstract}\noindent
\textsf{childdoc} is a \LaTeXe{} package
that enables the direct compilation
of document sections included by |\include|
to individual files.
\end{abstract}

\begingroup
\parskip0ex
\tableofcontents
\endgroup

%%%%%%%%%%%%%%%%%%%%%%%%%%%%%%%%%%%%%%%%%%%%%%%%%%%%%%%%%%%%%%%%%%%%%%%%%%%%%%%%
%%%%%%%%%%%%%%%%%%%%%%%%%%%%%%%%%%%%%%%%%%%%%%%%%%%%%%%%%%%%%%%%%%%%%%%%%%%%%%%%
\section{Introduction}

\LaTeX{} provides a mechanism to structure a large document (such as a book)
into a main file and several child files (containing the chapters)
using the |\include| command.
This mechanism is beneficial for documents
which span hundreds of pages in order to
make the source file(s) more manageable.
Moreover, compilation can be restricted to
selected child files by means of the |\includeonly| command.
The latter feature can be used to reduce the compilation time while editing
(this was significantly more useful in the earlier days of \LaTeX{})
or to generate a smaller document which is easier to navigate.
Another application of |\includeonly| is to generate
documents consisting of selected parts of the complete document.

However, there are a few drawbacks of the plain |\include| mechanism:
\begin{itemize}
\item
The child files cannot be compiled on their own,
they can only be compiled via the main file.
A naive editing environment
(such as a text editor with an option
to have the current file processed by \LaTeX)
may require one to switch to the main file before compiling;
attempting to compile the child file produces errors.
\item
The main file must be modified (each time)
to adjust the |\includeonly| command
to the present needs. This easily leaves the main file in a messy state.
\item
The generated document will always carry the filename
of the main document. This is inconvenient if
several child files are to be compiled and
to be kept for distribution.
\end{itemize}

The present package provides a simple interface
to make child files individually compilable by \LaTeX{}.
Compiling a child file then has the same effect as compiling
the main file with an |\includeonly| command
to select the appropriate child.
Moreover the generated document will carry the name of the child
rather than the main file.
This resolves all three above issues.

This feature is meant to make the editing of books,
thesis documents and lecture notes somewhat more convenient.
However, the package can also be used efficiently for
composing a series of documents (such as exercise sheets)
which are typically distributed individually.
It then assists the author in generating the individual documents
(potentially in different versions)
as well as a document containing the collected series.
Another application is in developing style files
or other kinds of included material
where compilation of the style file could redirect
to a sample or test file.

%%%%%%%%%%%%%%%%%%%%%%%%%%%%%%%%%%%%%%%%%%%%%%%%%%%%%%%%%%%%%%%%%%%%%%%%%%%%%%%%
%%%%%%%%%%%%%%%%%%%%%%%%%%%%%%%%%%%%%%%%%%%%%%%%%%%%%%%%%%%%%%%%%%%%%%%%%%%%%%%%
\section{Usage}

First of all, the package \textsf{childdoc} is \emph{not} a standard
\LaTeXe{} |.sty| style file! Therefore it needs to be invoked in
a non-standard way.

%%%%%%%%%%%%%%%%%%%%%%%%%%%%%%%%%%%%%%%%%%%%%%%%%%%%%%%%%%%%%%%%%%%%%%%%%%%%%%%%
\subsection{Included Files}
\label{sec:include}

%%%%%%%%%%%%%%%%%%%%%%%%%%%%%%%%%%%%%%%%
\DescribeMacro{\childdocmain}
To use the package, add the commands
\begin{center}
\begin{tabular}{l}
|\input{childdoc.def}|\\
|\childdocmain{}|\\
\end{tabular}
\end{center}
at the very top of the main \LaTeX{} file,
in particular \emph{before} the |\documentclass| statement!
The argument of |\childdocmain| should be left empty
(but it must be present).

%%%%%%%%%%%%%%%%%%%%%%%%%%%%%%%%%%%%%%%%
\DescribeMacro{\childdocof}
Furthermore, add the commands
\begin{center}
\begin{tabular}{l}
|\input{childdoc.def}|\\
|\childdocof{|\textit{main}|}|\\
\end{tabular}
\end{center}
at the top of every child file \textit{child}
which is included by |\include{|\textit{child}|}|
from within the main file
(or at least for those files to be compiled individually).
The argument \textit{main} must be the filename of the main file.

There are a couple of
considerations in setting up the main and child documents:

%%%%%%%%%%%%%%%%%%%%%%%%%%%%%%%%%%%%%%%%
\paragraph{Restrictions.}

Please note the following restrictions:
\begin{itemize}
\item
|\childdocmain| must be called with one argument \textit{main}
to ensure compatibility with earlier version of the package.
It must either be empty (|\childdocmain{}|)
or precisely match the filename of the main file in which it is specified.
See \secref{sec:detection} for further information.
\item
The filename \textit{main} must be specified without the |.tex| extension.
\item
The filename \textit{main} is case sensitive
(even in case-insensitive file systems)
due to internal string comparison.
\item
The argument \textit{main} should be fully expanded, it cannot be a macro.
\item
Subdirectories and special characters should be avoided in filenames.
\item
The command |\childdocmain{|\textit{main}|}| must be followed by a whitespace.
It should not be followed immediately by another command
or by a comment mark `|%|'.
This is because the \TeX{} parser reads the token immediately following
the argument of |\childdocmain| and puts it
at the beginning of every child section;
however, a white\-space is ignored.
\end{itemize}

%%%%%%%%%%%%%%%%%%%%%%%%%%%%%%%%%%%%%%%%
\paragraph{Content of Main File.}

It is advisable to place all content in the child files included by |\include|.
Any output contained in the main file will appear in all child documents
unless suppressed manually;
it cannot be suppressed automatically by the |\includeonly| directive
and thus should normally be avoided.
A method to include some content in the main file
by means of conditional processing is described in \secref{sec:conditional}.

%%%%%%%%%%%%%%%%%%%%%%%%%%%%%%%%%%%%%%%%
\paragraph{Page Numbering.}

When only a part of the document is compiled,
the appropriate numbering of pages
(as well as other status parameters)
is determined from the |.aux| files.
The latter contain information from previous passes.
However this information needs to propagate through
all intermediate child documents.
Therefore the page numbering in child documents may well
be inconsistent until the complete document is compiled at least once.

A useful (if unconventional) way to always ensure a consistent
page numbering is to restart the numbering in each child document
and denote the pages by `\textit{child}|.|\textit{page}'
where \textit{child} represents the chapter/section number of the child file.
This can be achieved by the command
|\numberwithin{page}{|\textit{child}|}|
of the \textsf{amsmath} package
where \textit{child} can be |chapter| or |section|
depending on the chosen structuring.
Alternatively, one can modify the macro |\thepage| appropriately
and reset the counter |page| at the start of each child file.

%%%%%%%%%%%%%%%%%%%%%%%%%%%%%%%%%%%%%%%%%%%%%%%%%%%%%%%%%%%%%%%%%%%%%%%%%%%%%%%%
\subsection{Conditional Processing}
\label{sec:conditional}

The package provides a mechanism to compile different versions
of a document. To customise the versions further some conditional processing
can come in handy to distinguish which version is being compiled.
The package provides two macros to describe the compilation context:

%%%%%%%%%%%%%%%%%%%%%%%%%%%%%%%%%%%%%%%%
\DescribeMacro{\ifchilddoc}
The conditional |\ifchilddoc| distinguishes between the compilation of
child documents and the main document:
%
\begin{center}
|\ifchilddoc |\textit{child-code}| |[|\||else |\textit{main-code}]| \||fi|
\end{center}

%%%%%%%%%%%%%%%%%%%%%%%%%%%%%%%%%%%%%%%%
\DescribeMacro{\childdocname}
\DescribeMacro{\childdocjob}
The macro |\childdocname| contains the filename (without extension)
of the main or child file being processed.
Note that |\childdocjob| will always contain the name of the main file.

%%%%%%%%%%%%%%%%%%%%%%%%%%%%%%%%%%%%%%%%
\paragraph{Title Page.}

Conditional processing can be used to include a title or banner page
in the main document when proper precautions are taken.
Importantly, the code in the main file should ensure that the page counter
(as well as other status parameters which are stored in the |.aux| files)
takes the same value after the conditional processing.
Otherwise the page numbers may take divergent values
depending on which part is compiled.

For example, a title page could be declared by:
%
\begin{center}
\begin{tabular}{l}
|\ifchilddoc\||else|\\
|\addtocounter{page}{-1}|\\
\textit{code for title page}\\
|\newpage|\\
|\||fi|
\end{tabular}
\end{center}
%
A banner page for the child documents can be generated by:
%
\begin{center}
\begin{tabular}{l}
|\ifchilddoc|\\
|\addtocounter{page}{-1}|\\
\textit{code for banner page}\\
|\newpage|\\
|\||fi|
\end{tabular}
\end{center}
%
Here one could write a message such as:
\begin{center}
|This is the part \childdocname{} of \childdocjob{}.|
\end{center}

%%%%%%%%%%%%%%%%%%%%%%%%%%%%%%%%%%%%%%%%%%%%%%%%%%%%%%%%%%%%%%%%%%%%%%%%%%%%%%%%
\subsection{Flags}
\label{sec:flags}

The package makes it easy to generate different versions
of the main or child documents.
To this end compilation flags can be defined
and assigned different default values.
They will be particularly useful in conjunction
with the forwarding mechanism described in \secref{sec:forward}.

For example, it may be useful to have a flag |\version|
which can be set to |draft| or |final|.
The document source will contain some conditional code
depending on the value of |\version|.
Suppose further, the flag should default to |final| for the main file
and to |draft| for child files
which is a natural assignment for editing the document.
This is achieved by placing the following code
in the preamble of the main document
(below the |\childdocmain| directive):
%
\begin{center}
\begin{tabular}{l}
|\ifchilddoc|\\
|\providecommand{\version}{draft}|\\
|\||else|\\
|\providecommand{\version}{final}|\\
|\||fi|
\end{tabular}
\end{center}
%
The definition by |\providecommand| makes sure
that previous definitions are not overwritten.
Further statements |\providecommand{\version}{...}|
can thus be added before the above code to override it.

For the main file, one might add a line
(between |\childdocmain| and the above block)
%
\begin{center}
|%\ifchilddoc\||else\providecommand{\version}{draft}\||fi|
\end{center}
%
which can be uncommented to produce a draft version.
Likewise one can add a line to the very top of a child file
(above the |\childdocof{|\textit{main}|}| directive)
%
\begin{center}
|%\providecommand{\version}{final}|
\end{center}
%
which can be uncommented to produce the final version of this child document.

%%%%%%%%%%%%%%%%%%%%%%%%%%%%%%%%%%%%%%%%%%%%%%%%%%%%%%%%%%%%%%%%%%%%%%%%%%%%%%%%
\subsection{Forwarding}
\label{sec:forward}

Different versions of the main or child documents
using compilation flags as described in \secref{sec:flags}
can be (permanently) stored in different files
for convenient compilation, viewing and distribution.
To this end, the package defines a command
to pass on compilation to a different file:

%%%%%%%%%%%%%%%%%%%%%%%%%%%%%%%%%%%%%%%%
\DescribeMacro{\childdocforward}
The command |\childdocforward| redirects processing to
another source file:
%
\begin{center}
\begin{tabular}{l}
|\input{childdoc.def}|\\
|\childdocforward[|\textit{main}|]{|\textit{dest}|}|\\
\end{tabular}
\end{center}
%
The argument \textit{dest} is the destination file
(without extension).
It should be the main file or one of the child files.
Note that further \textsf{childdoc} directives
such as |\childdocof| and |\childdocforward|
in the indicated file will be processed in this form.
The optional argument \textit{main}
passes on directly to the main file \textit{main}
while pretending to compile the child \textit{dest}.
This form behaves as if \textit{dest}
issues |\childdocof{|\textit{main}|}| right away,
and no further \textsf{childdoc} directives will be processed.

%%%%%%%%%%%%%%%%%%%%%%%%%%%%%%%%%%%%%%%%
\DescribeMacro{\...prefix}
In the alternative form |\childdocforwardprefix|,
%
\begin{center}
\begin{tabular}{l}
|\input{childdoc.def}|\\
|\childdocforwardprefix[|\textit{main}|]{|\textit{prefix}|}{|\textit{dest}|}|
\end{tabular}
\end{center}
%
the destination file is determined by a pattern
depending on the current file:
To make this work, the current file must be called
`{\textit{prefix}\hspace{0.2em}\textit{suffix}}'
with \textit{prefix} matching precisely the argument.
Processing is then passed on to the file
`{\textit{dest}\hspace{0.2em}\textit{suffix}}'.
Surely, the same effect is achieved by
directly specifying the
argument `{\textit{dest}\hspace{0.2em}\textit{suffix}}'
in the first form.
However, that requires to set up a different file
for each child. With the alternative form of the command
all these files can have exactly the same content
which simplifies setting them up and maintaining them.

For example, the following file |draft.tex|
with a compilation flag |\version| as described in \secref{sec:flags}
compiles the main document as a draft:
%
\begin{center}
\begin{tabular}{l}
|\def\version{draft}|\\
|\input{childdoc.def}|\\
|\childdocforward{|\textit{main}|}|
\end{tabular}
\end{center}
%
Likewise, the following files |final|\textit{nn}|.tex|
compile the final version of the child document
|child|\textit{nn}|.tex|:
%
\begin{center}
\begin{tabular}{l}
|\def\version{final}|\\
|\input{childdoc.def}|\\
|\childdocforwardprefix{final}{child}|
\end{tabular}
\end{center}
%

Note that when several versions of a main file and/or of each child file
are to be generated, it may be convenient to set up a |Makefile| or
shell script to automatise the process.

%%%%%%%%%%%%%%%%%%%%%%%%%%%%%%%%%%%%%%%%%%%%%%%%%%%%%%%%%%%%%%%%%%%%%%%%%%%%%%%%
\subsection{Command Line Processing}
\label{sec:commandline}

The effect of redirection files can also be achieved by invoking
the \LaTeX{} compiler with a more elaborate command line.
Most conveniently this should be done as part
of a shell script or a |Makefile|.

When using \textsf{childdoc} in the main file, the following
command lines effectively perform a redirection
(note that depending on the shell being used,
backslashes may have to be doubled: `|\|' $\to$ `|\\|'):
%
\begin{center}
|... -jobname "|\textit{target}|" |\\|"|[\textit{flags}]%
|\input{childdoc.def}\childdocforward[|\textit{main}|]{|\textit{dest}|}"|
\end{center}
%
Here \textit{target} is the name of the output file,
\textit{main} is the name of the main file
and \textit{dest} is the name of the main or child file to be processed
(all filenames without extensions).
The optional argument \textit{main} can be omitted
if \textit{main} matches \textit{dest}.
Optionally, compilation \textit{flags} can be defined via |\def| commands.
This command line makes the \TeX{} engine believe
it is compiling the file \textit{target}
whose content is specified as the latter parameter.
The provided code then forwards the processing to
\textit{main} or \textit{dest} as described in \secref{sec:forward}.

%%%%%%%%%%%%%%%%%%%%%%%%%%%%%%%%%%%%%%%%%%%%%%%%%%%%%%%%%%%%%%%%%%%%%%%%%%%%%%%%
\subsection{Include by Input}
\label{sec:input}

Including child documents by |\include| has some restrictions by design.
Most notably, the content of a child document always occupies
its own set of pages; pages cannot be shared between child documents.
Usually, this behaviour makes perfect sense
because each child document contain an essential part of the document.
However, in some situations it may be desirable to compose
a document from a collection of parts
without having mandatory page breaks between then.
For this case, the package
provides a mechanism to include parts
by |\input| which can also be processed individually.
However, by construction this mechanism
requires manual handling of the content to be output.

%%%%%%%%%%%%%%%%%%%%%%%%%%%%%%%%%%%%%%%%
\DescribeMacro{\ifchilddocmanual}
The main file should be prepared as usual, see \secref{sec:include}.
However, the document body must make a distinction
between processing of an individual part and of the main document, e.g.:
%
\begin{center}
\begin{tabular}{l}
|\ifchilddocmanual|\\
|\input{\childdocname}|\\
|\||else|\\
\textit{document body with }|\input{|\textit{part}|}|\\
|\||fi|
\end{tabular}
\end{center}
%
The conditional |\ifchilddocmanual| is true whenever
a part to be included by |\input| is being compiled,
and the name of the part is stored in |\childdocname|.

%%%%%%%%%%%%%%%%%%%%%%%%%%%%%%%%%%%%%%%%
\DescribeMacro{\childdocby}
Each part to be included by |\input| should start with:
%
\begin{center}
\begin{tabular}{l}
|\input{childdoc.def}|\\
|\childdocby{|\textit{main}|}|\\
\end{tabular}
\end{center}
%
The directive |\childdocby| is similar to |\childdocof|
described in \secref{sec:include},
but the subsequent selection of content must be done manually.
To that end, both |\ifchilddoc| and |\ifchilddocmanual|
will be true upon processing of a part,
and the name of the part is stored in |\childdocname|.
Note that |\jobname| will be set to the filename of the current part
so that each part receives an individual |.aux| file
that does not interfere with the |.aux| file(s) of the main document.
This behaviour can be altered by the alternative form
|\childdocby[*]{|\textit{main}|}| (with a non-empty optional argument)
which uses the |.aux| file of the main document
by setting |\jobname| to \textit{main}.

%%%%%%%%%%%%%%%%%%%%%%%%%%%%%%%%%%%%%%%%%%%%%%%%%%%%%%%%%%%%%%%%%%%%%%%%%%%%%%%%
\subsection{Driver Development}
\label{sec:driver}

The \textsf{childdoc} mechanism can also be use for the development
of definition files such as \LaTeX{} styles or classes.
This case differs from the above setup with multiple parts
included by |\include| in that no |\includeonly| should be invoked.
This can be achieved by starting the include file
(before |\ProvidesPackage|) with:
%
\begin{center}
\begin{tabular}{l}
|\input{childdoc.def}|\\
|\childdocforward{|\textit{main}|}|\\
\end{tabular}
\end{center}
%
or alternatively with:
%
\begin{center}
\begin{tabular}{l}
|\input{childdoc.def}|\\
|\childdocby{|\textit{main}|}|\\
\end{tabular}
\end{center}
%
Both forms have slightly different effects as described above.
The main file is prepared as usual, see \secref{sec:include}.

%%%%%%%%%%%%%%%%%%%%%%%%%%%%%%%%%%%%%%%%%%%%%%%%%%%%%%%%%%%%%%%%%%%%%%%%%%%%%%%%
\subsection{Legacy Detection}
\label{sec:detection}

The directive |\childdocmain| in the main file can detect
whether the complete document or merely a child is to be compiled
even without using the directive |\childdocof|.
This method is deprecated because it is less robust
and there is no compelling reason to use it;
it is merely provided for backward compatibility
and it may be removed in future versions.

If the detection mechanism is to be used,
it is mandatory to correctly specify
the filename of the main file as the argument of |\childdocmain|:
%
\begin{center}
\begin{tabular}{l}
|\input{childdoc.def}|\\
|\childdocmain{|\textit{main}|}|\\
\end{tabular}
\end{center}
%
If |\jobname| does not match the argument \textit{main} of |\childdocmain|,
it is assumed that |\jobname| points to the child file to be compiled.
When using |\childdocmain| with the main file specified as argument,
it suffices to start a child file
with just |\input{|\textit{main}|}|
without loading of the package and using |\childdocof|.
If instead all processing is done
with the appropriate \textsf{childdoc} directives,
the argument of \textit{main} of |\childdocmain| can be empty.

An alternative version of the command line processing described
in \secref{sec:commandline} using the detection mechanism reads:
%
\begin{center}
|... -jobname "|\textit{target}|" "|[\textit{flags}]%
[|\def\jobname{|\textit{dest}|}|]|\input{|\textit{main}|}"|
\end{center}

%%%%%%%%%%%%%%%%%%%%%%%%%%%%%%%%%%%%%%%%%%%%%%%%%%%%%%%%%%%%%%%%%%%%%%%%%%%%%%%%
\subsection{Manual Code}
\label{sec:manual}

In case one cannot be certain whether the definitions file |childdoc.def|
is installed on the target \TeX{} distribution
and one prefers not to ship it,
it is conceivable to paste a few relevant commands into the sources.

To that end, drop all statements |\input{childdoc.def}|
and perform the replacements as outlined below.
Instead of |\childdocmain{|\textit{main}|}| add the following code
to the top of the main file:
%
\begin{center}
\begin{tabular}{l}
|\||ifdefined\childdocname\endinput\||fi\newif\ifchilddoc|\\
|\edef\childdocname{\scantokens\expandafter{\jobname\noexpand}}|\\
|\def\childdocmain{|\textit{main}|}\||ifx\childdocmain\childdocname\||else|\\
|\childdoctrue\includeonly{\childdocname}\let\jobname\childdocmain\||fi|\\
\end{tabular}
\end{center}
%
Instead of |\childdocof{|\textit{main}|}| just include the main file
at the top of each child file:
%
\begin{center}
|\input{|\textit{main}|}|
\end{center}
%
A simple redirection |\childdocforward{|\textit{dest}|}| is achieved by:
%
\begin{center}
|\def\jobname{|\textit{dest}|}\input{\jobname}|
\end{center}
%
The redirection with prefix
|\childdocforwardprefix[|\textit{prefix}|]{|\textit{dest}|}|
is accomplished by:
%
\begin{center}
\begin{tabular}{l}
|{\edef\jobname{\scantokens\expandafter{\jobname\noexpand}}|\\
|\def\redirectjob |\textit{prefix}|#1~~~{\gdef\jobname{|\textit{dest}|#1}}|\\
|\expandafter\redirectjob\jobname~~~}\input{\jobname}|
\end{tabular}
\end{center}

In an alternative approach,
child documents can be compiled by a specific command line
without additional code or specific definitions:
%
\begin{center}
|... -jobname "|\textit{target}|" "|[\textit{flags}]%
|\includeonly{|\textit{dest}|}\input{|\textit{main}|}"|
\end{center}
%

%%%%%%%%%%%%%%%%%%%%%%%%%%%%%%%%%%%%%%%%%%%%%%%%%%%%%%%%%%%%%%%%%%%%%%%%%%%%%%%%
%%%%%%%%%%%%%%%%%%%%%%%%%%%%%%%%%%%%%%%%%%%%%%%%%%%%%%%%%%%%%%%%%%%%%%%%%%%%%%%%
\section{Information}

%%%%%%%%%%%%%%%%%%%%%%%%%%%%%%%%%%%%%%%%%%%%%%%%%%%%%%%%%%%%%%%%%%%%%%%%%%%%%%%%
\subsection{Copyright}

Copyright \copyright{} 2017--2018 Niklas Beisert

This work may be distributed and/or modified under the
conditions of the \LaTeX{} Project Public License, either version 1.3
of this license or (at your option) any later version.
The latest version of this license is in
  \url{http://www.latex-project.org/lppl.txt}
and version 1.3 or later is part of all distributions of \LaTeX{}
version 2005/12/01 or later.

This work has the LPPL maintenance status `maintained'.

The Current Maintainer of this work is Niklas Beisert.

This work consists of the files |README.txt|, |childdoc.ins| and |childdoc.dtx|
as well as the derived files |childdoc.def|, |cdocsamp.tex|
with |cdocsch1.tex|, |cdocsch2.tex|, |cdocspt3.tex|, |cdocspt4.tex|,
|cdocsdrf.tex|, |cdocsfn1.tex|, |cdocsfn2.tex|
as well as |childdoc.pdf|.

%%%%%%%%%%%%%%%%%%%%%%%%%%%%%%%%%%%%%%%%%%%%%%%%%%%%%%%%%%%%%%%%%%%%%%%%%%%%%%%%
\subsection{Files and Installation}

The package consists of the files:
%
\begin{center}
\begin{tabular}{ll}
    |README.txt|   & readme file \\
    |childdoc.ins| & installation file \\
    |childdoc.dtx| & source file \\
    |childdoc.def| & definition file \\
    |cdocsamp.tex| & sample main file \\
    |cdocsch1.tex| & sample include file \\
    |cdocsch2.tex| & sample include file \\
    |cdocspt3.tex| & sample part file \\
    |cdocspt4.tex| & sample part file \\
    |cdocsdrf.tex| & sample redirection file \\
    |cdocsfn1.tex| & sample redirection file \\
    |cdocsfn2.tex| & sample redirection file \\
    |childdoc.pdf| & manual
\end{tabular}
\end{center}
%
The distribution consists of the files
|README.txt|, |childdoc.ins| and |childdoc.dtx|.
%
\begin{itemize}
\item
Run (pdf)\LaTeX{} on |childdoc.dtx|
to compile the manual |childdoc.pdf| (this file).
\item
Run \LaTeX{} on |childdoc.ins| to create the definitions file |childdoc.def|
and the sample |cdocsamp.tex| with include files
|cdocsch1.tex|, |cdocsch2.tex|, |cdocspt3.tex|, |cdocspt4.tex|,
|cdocsdrf.tex|, |cdocsfn1.tex|, |cdocsfn2.tex|.
Then copy the file |childdoc.def| to an appropriate directory of your \LaTeX{}
distribution, e.g.\ \textit{texmf-root}|/tex/latex/childdoc|.
\end{itemize}

%%%%%%%%%%%%%%%%%%%%%%%%%%%%%%%%%%%%%%%%%%%%%%%%%%%%%%%%%%%%%%%%%%%%%%%%%%%%%%%%
\subsection{Related CTAN Packages}

There are several other packages which offer a similar functionality:
%
\begin{itemize}
\item
The packages
\href{http://ctan.org/pkg/docmute}{\textsf{docmute}},
\href{http://ctan.org/pkg/includex}{\textsf{includex}} and
\href{http://ctan.org/pkg/standalone}{\textsf{standalone}}
provide commands to include only the document body of
a child file thus allowing both files to be compiled individually.
\item
The packages \href{http://ctan.org/pkg/subdocs}{\textsf{subdocs}}
and \href{http://ctan.org/pkg/subfiles}{\textsf{subfiles}}
provide structures in which the main and child documents can be
encapsulated and allowing them to be compiled individually.
The inclusion mechanism is different from the conventional |\include|.
\item
The package \href{http://ctan.org/pkg/combine}{\textsf{combine}}
is an elaborate solution to combine several documents into one.
\end{itemize}
%
See also the CTAN topic \href{http://ctan.org/topic/subdocs}{\textsf{subdocs}}
for further related packages.
The present package differs from the above solutions in that
a document structure constructed with the conventional |\include| mechanism
just needs two extra commands at the top of every file
such that all constituent files can be compiled individually.

%%%%%%%%%%%%%%%%%%%%%%%%%%%%%%%%%%%%%%%%%%%%%%%%%%%%%%%%%%%%%%%%%%%%%%%%%%%%%%%%
%\subsection{Feature Suggestions}
%
%The following is a list of features which may be useful for future
%versions of this package:
%%
%\begin{itemize}
%\item
%\ldots
%\end{itemize}

%%%%%%%%%%%%%%%%%%%%%%%%%%%%%%%%%%%%%%%%%%%%%%%%%%%%%%%%%%%%%%%%%%%%%%%%%%%%%%%%
\subsection{Revision History}

%%%%%%%%%%%%%%%%%%%%%%%%%%%%%%%%%%%%%%%%
\paragraph{v2.0:} 2018/12/30

\begin{itemize}
\item
immediate forward processing
\item
added |\childdocby| mechanism
\item
manual restructured
\end{itemize}

%%%%%%%%%%%%%%%%%%%%%%%%%%%%%%%%%%%%%%%%
\paragraph{v1.6:} 2018/01/17

\begin{itemize}
\item
application for development of include files
\item
corrections to manual
\end{itemize}

%%%%%%%%%%%%%%%%%%%%%%%%%%%%%%%%%%%%%%%%
\paragraph{v1.5:} 2017/05/21

\begin{itemize}
\item
more complete structuring introduced
\item
|\childdocof| introduced
\item
|\childdoc| renamed to |\childdocmain|
\item
|\childredirect| renamed to |\childdocforward| and |\childdocforwardprefix|
and functionality expanded
\end{itemize}

%%%%%%%%%%%%%%%%%%%%%%%%%%%%%%%%%%%%%%%%
\paragraph{v1.0:} 2017/04/27

\begin{itemize}
\item
manual and install package
\item
first version published on CTAN
\end{itemize}

%%%%%%%%%%%%%%%%%%%%%%%%%%%%%%%%%%%%%%%%
\paragraph{v0.6:} 2017/04/26

\begin{itemize}
\item
redirection mechanism added
\end{itemize}

%%%%%%%%%%%%%%%%%%%%%%%%%%%%%%%%%%%%%%%%
\paragraph{v0.5:} 2017/04/26

\begin{itemize}
\item
functionality in definition file
\end{itemize}


%%%%%%%%%%%%%%%%%%%%%%%%%%%%%%%%%%%%%%%%%%%%%%%%%%%%%%%%%%%%%%%%%%%%%%%%%%%%%%%%
%%%%%%%%%%%%%%%%%%%%%%%%%%%%%%%%%%%%%%%%%%%%%%%%%%%%%%%%%%%%%%%%%%%%%%%%%%%%%%%%
%%%%%%%%%%%%%%%%%%%%%%%%%%%%%%%%%%%%%%%%%%%%%%%%%%%%%%%%%%%%%%%%%%%%%%%%%%%%%%%%
\appendix

\settowidth\MacroIndent{\rmfamily\scriptsize 000\ }

 \DocInput{childdoc.dtx}

\end{document}
%</driver>
% \fi
%
% %%%%%%%%%%%%%%%%%%%%%%%%%%%%%%%%%%%%%%%%%%%%%%%%%%%%%%%%%%%%%%%%%%%%%%%%%%%%%%
% %%%%%%%%%%%%%%%%%%%%%%%%%%%%%%%%%%%%%%%%%%%%%%%%%%%%%%%%%%%%%%%%%%%%%%%%%%%%%%
% \section{Sample}
%\iffalse
%<*samplemain>
%\fi
%
% The following presents a sample document
% with two chapters, two parts, a title page,
% a compile flag as well as three forwarding files to set the flag.
% It consists of eight |.tex| files:
% \begin{center}
% \begin{tabular}{ll}
% |cdocsamp.tex|&main file\\
% |cdocsch1.tex|&include file for chapter 1\\
% |cdocsch2.tex|&include file for chapter 2\\
% |cdocspt3.tex|&include file for part 3\\
% |cdocspt4.tex|&include file for part 4\\
% |cdocsdrf.tex|&forwarding file for main file in draft mode\\
% |cdocsfi1.tex|&forwarding file for final version of chapter 1\\
% |cdocsfi2.tex|&forwarding file for final version of chapter 2\\
% \end{tabular}
% \end{center}
% Each of the eight files can be compiled directly by the \LaTeX{} compiler.
%
% %%%%%%%%%%%%%%%%%%%%%%%%%%%%%%%%%%%%%%
% \paragraph{Main File.}
%
% The main file is called |cdocsamp.tex|.
%
% Load the \textsf{childdoc} definitions and
% declare the filename for the main document:
%    \begin{macrocode}
\input{childdoc.def}
\childdocmain{}
%    \end{macrocode}

% Optional override for |\version| flag:
%    \begin{macrocode}
%%\ifchilddoc\else\providecommand{\version}{draft}\fi
%    \end{macrocode}

% Define the default values for the |\version| flag
% (|final| for the main file and |draft| for childs):
%    \begin{macrocode}
\ifchilddoc
\providecommand{\version}{draft}
\else
\providecommand{\version}{final}
\fi
%    \end{macrocode}

% Load the standard document class:
%    \begin{macrocode}
\documentclass[12pt]{article}
%    \end{macrocode}

% Start the document body:
%    \begin{macrocode}
\begin{document}
%    \end{macrocode}

% Declare a title page.
% Print title, part of document being processed and version flag:
%    \begin{macrocode}
\addtocounter{page}{-1}
\begin{center}
{\LARGE\bfseries{}childdoc example\par}
\vspace{1cm}
\ifchilddoc
\ifchilddocmanual part\else chapter\fi:
`\childdocname' of `\childdocjob'\par
\else
main document: `\childdocjob'\par
\fi
version: \version\par
\end{center}
\newpage
%    \end{macrocode}

% Manually include selected file,
% otherwise process as usual:
%    \begin{macrocode}
\ifchilddocmanual
\section*{part `\childdocname'}
\input{\childdocname}
\else
%    \end{macrocode}

% Include the two chapters:
%    \begin{macrocode}
\include{cdocsch1}
\include{cdocsch2}
%    \end{macrocode}

% Include the two parts unless only chapters should be displayed:
%    \begin{macrocode}
\ifchilddoc\else
\section{part three}
\input{cdocspt3}
\section{part four}
\input{cdocspt4}
\fi
%    \end{macrocode}

% Process as usual until here:
%    \begin{macrocode}
\fi
%    \end{macrocode}

% End of document body:
%    \begin{macrocode}
\end{document}
%    \end{macrocode}
%\iffalse
%</samplemain>
%\fi
%
% %%%%%%%%%%%%%%%%%%%%%%%%%%%%%%%%%%%%%%
% \paragraph{Chapter Include Files.}
%
% The include files are called |cdocsch1.tex| and |cdocsch2.tex|.
%
%\iffalse
%<*samplechap1|samplechap2>
%\fi

% Optional override for |\version| flag:
%    \begin{macrocode}
%%\providecommand{\version}{final}
%    \end{macrocode}

% Include the main document:
%    \begin{macrocode}
\input{childdoc.def}
\childdocof{cdocsamp}
%    \end{macrocode}

%\iffalse
%</samplechap1|samplechap2>
%\fi
%
%\iffalse
%<*samplechap1>
%\fi
% Some text for chapter 1:
%    \begin{macrocode}
\section{one}
some text in chapter one
%    \end{macrocode}

%\iffalse
%</samplechap1>
%\fi
% Some text for chapter 2:
%\iffalse
%<*samplechap2>
%\fi
%    \begin{macrocode}
\section{two}
more text in chapter two
%    \end{macrocode}

%\iffalse
%</samplechap2>
%\fi
%
% %%%%%%%%%%%%%%%%%%%%%%%%%%%%%%%%%%%%%%
% \paragraph{Part Include Files.}
%
% The include files are called |cdocspt3.tex| and |cdocspt4.tex|.
%
%\iffalse
%<*samplepart3|samplepart4>
%\fi

% Optional override for |\version| flag:
%    \begin{macrocode}
%%\providecommand{\version}{final}
%    \end{macrocode}

% Include the main document:
%    \begin{macrocode}
\input{childdoc.def}
\childdocby{cdocsamp}
%    \end{macrocode}

%\iffalse
%</samplepart3|samplepart4>
%\fi
%
%\iffalse
%<*samplepart3>
%\fi
% Some text for part 3:
%    \begin{macrocode}
some text in part three
%    \end{macrocode}

%\iffalse
%</samplepart3>
%\fi
% Some text for part 4:
%\iffalse
%<*samplepart4>
%\fi
%    \begin{macrocode}
more text in part four
%    \end{macrocode}

%\iffalse
%</samplepart4>
%\fi
%
% %%%%%%%%%%%%%%%%%%%%%%%%%%%%%%%%%%%%%%
% \paragraph{Forwarding for a Complete Draft.}
%
% The following forwarding file |cdocsdrf.tex|
% compiles the main document in draft mode:
%\iffalse
%<*sampledraft>
%\fi
%    \begin{macrocode}
\def\version{draft}
\input{childdoc.def}
\childdocforward{cdocsamp}
%    \end{macrocode}

%\iffalse
%</sampledraft>
%\fi
%
% %%%%%%%%%%%%%%%%%%%%%%%%%%%%%%%%%%%%%%
% \paragraph{Forwarding for Final Version of the Chapters.}
%
% The following forwarding files |cdocsfn1.tex| and |cdocsfn2.tex|
% (with identical content)
% compile the final versions of the child documents
% |cdocsch1.tex| and |cdocsch2.tex|, respectively:
%\iffalse
%<*samplefinal>
%\fi
%    \begin{macrocode}
\def\version{final}
\input{childdoc.def}
\childdocforwardprefix[cdocsamp]{cdocsfn}{cdocsch}
%    \end{macrocode}

%\iffalse
%</samplefinal>
%\fi
%
% %%%%%%%%%%%%%%%%%%%%%%%%%%%%%%%%%%%%%%
% \paragraph{Command Line Processing.}
%
% The following three command lines generate the output files
% |cdocscld|, |cdocscl1| and |cdocscl2|
% which should be identical to
% |cdocsdrf|, |cdocsch1| and |cdocsfn2|, respectively:
% \begin{center}
% \begin{tabular}{l}
% |latex -jobname cdocscld \|\\
% |  "\def\version{draft}\input{childdoc.def}\childdocforward{cdocsamp}"|\\
% |latex -jobname cdocscl1 \|\\
% |  "\input{childdoc.def}\childdocforward[cdocsamp]{cdocsch1}"|\\
% |latex -jobname cdocscl2 \|\\
% |  "\def\version{final}\input{childdoc.def}\childdocforward{cdocsch2}"|
% \end{tabular}
% \end{center}
% Note that the trailing backslash on each first line
% merely continues the input to the second line
% (for convenient cut ant paste).
% Furthermore, the command |latex| can be replaced by any
% of its alternative versions such as |pdflatex|.
%
% %%%%%%%%%%%%%%%%%%%%%%%%%%%%%%%%%%%%%%%%%%%%%%%%%%%%%%%%%%%%%%%%%%%%%%%%%%%%%%
% %%%%%%%%%%%%%%%%%%%%%%%%%%%%%%%%%%%%%%%%%%%%%%%%%%%%%%%%%%%%%%%%%%%%%%%%%%%%%%
% \section{Implementation}
%\iffalse
%<*package>
%\fi
%
% This section describes the definitions file |childdoc.def|.

% The definitions cannot be loaded using |\usepackage| or |\RequirePackage|
% which has a mechanism to prevent loading a style file more than once.
% When loading the definitions by means of |\input|
% multiple instances have to be prevented manually:
%\iffalse
%This code needs to be before the `\ProvidesFile' directive
%which is defined at the beginning of this file.
%Therefore it is also placed there and commented out here.
%</package>
%<*discard>
%\fi
%    \begin{macrocode}
\ifdefined\childdocmain\endinput\fi
%    \end{macrocode}
%\iffalse
%</discard>
%<*package>
%\fi
%
% \macro{\ifchilddoc}
% \macro{\ifchilddocmanual}
% The conditional |\ifchilddoc| tells whether a
% child (true) or main (false) document is being compiled.
% The conditional |\ifchilddocmanual| tells whether
% the |\includeonly| mechanism is used (false) or
% the selection of child files must be performed manually (true).
% The definitions initialise to false:
%    \begin{macrocode}
\newif\ifchilddoc
\newif\ifchilddocmanual
%    \end{macrocode}

% \macro{\childdocname}
% \macro{\childdocjob}
% The macro |\childdocname| stores the name of the main document
% to be compiled. The macro |\childdocjob| stores the name of
% the document on which the \LaTeX{} compiler was originally invoked.
% The content of |\jobname| cannot be compared
% to filenames specified in the source due to different catcodes.
% The following code rescans |\jobname|, stores the result
% in |\childdocname| and saves a copy in |\childdocjob|:
%    \begin{macrocode}
\edef\childdocname{\scantokens\expandafter{\jobname\noexpand}}
\let\childdocjob\childdocname
%    \end{macrocode}

% \macro{\childdocdisable}
% The macro |\childdocdisable| prevents the main file
% from being processed more than once.
% At this stage, the main document command |\childdocmain|
% is assumed to be called once again where it should do nothing.
% Any subsequent call to it should prevent
% a secondary processing of the main document
% It overwrites the forwarding commands
% |\childdocof| and |\childdocforward|
% with empty macros to prevent further inclusions of the main document:
%    \begin{macrocode}
\newcommand{\childdocdisable}
{
  \renewcommand{\childdocmain}[1]{\renewcommand{\childdocmain}[1]{\endinput}}
  \renewcommand{\childdocof}[1]{}
  \renewcommand{\childdocby}[2][]{}
  \renewcommand{\childdocforward}[2][]{}
  \renewcommand{\childdocdisable}{}
}
%    \end{macrocode}

% \macro{\childdocmain}
% The macro |\childdocmain| is to be called at the top of the main file
% with nothing or the main filename (without extension) as argument.
% First, it breaks loops.
% If the argument is not empty and does not match |\childdocname|
% (which is set by the first inclusion of |childdoc.def|),
% |\ifchilddoc| is set to true, |\includeonly| is applied to the child file
% and |\jobname| is set to the main file
% (for proper handling of |.aux| files):
%    \begin{macrocode}
\newcommand{\childdocmain}[1]
{
  \childdocdisable\childdocmain{}
  \if?#1?\else
    \begingroup
      \def\childdoctmp{#1}
      \ifx\childdoctmp\childdocname
        \def\childdoctmp{}
      \else
        \def\childdoctmp
        {
          \childdoctrue
          \includeonly{\childdocname}
          \def\childdocjob{#1}
          \def\jobname{#1}
        }
      \fi
      \expandafter
    \endgroup
    \childdoctmp
  \fi
}
%    \end{macrocode}

% \macro{\childdocof}
% The command |\childdocof| redirects
% compilation to the main file |#1|.
%    \begin{macrocode}
\newcommand{\childdocof}[1]
{
  \childdocdisable
  \childdoctrue
  \includeonly{\childdocname}
  \def\jobname{#1}
  \def\childdocjob{#1}
  \input{#1}
}
%    \end{macrocode}

% \macro{\childdocby}
% The command |\childdocby| ....
%    \begin{macrocode}
\newcommand{\childdocby}[2][]
{
  \childdocdisable
  \childdoctrue
  \childdocmanualtrue
  \if?#1?\else
    \def\jobname{#2}
  \fi
  \def\childdocjob{#2}
  \input{#2}
  \endinput
}
%    \end{macrocode}

% \macro{\childdocforward}
% The command |\childdocforward| redirects
% compilation to the main file or
% (if the optional argument is given) a child file.
% Parameters are set as if the main file
% or a child file starting with |\childdocof| was compiled.
% Then compilation is handed over to the main file:
%    \begin{macrocode}
\newcommand{\childdocforward}[2][]
{
  \begingroup
    \if?#1?
      \def\childdoctmp
      {
        \def\childdocname{#2}
        \def\childdocjob{#2}
        \def\jobname{#2}
        \input{#2}
        \endinput
      }
    \else
      \def\childdoctmp
      {
        \childdocdisable
        \def\childdocname{#2}
        \childdoctrue
        \includeonly{#2}
        \def\childdocjob{#1}
        \def\jobname{#1}
        \input{#1}
        \endinput
      }
    \fi
    \expandafter
  \endgroup
  \childdoctmp
}
%    \end{macrocode}

% \macro{\childdocforwardprefix}
% The command |\childdocforwardprefix| redirects
% compilation to the main or a child file by means of a pattern.
% The prefix |#1| in the current filename is replaced by |#2|
% and the suffix of the current filename is kept
% (it is assumed that the filename does not contain the substring `|~~~|'
% which is used as a delimiter).
% Compilation is handed over to the new file by |\childdocforward|:
%    \begin{macrocode}
\newcommand{\childdocforwardprefix}[3][]
{
  \begingroup
    \def\childdocextract #2##1~~~{\def\childdoctmp{\childdocforward[#1]{#3##1}}}
    \expandafter\childdocextract\childdocname~~~
    \expandafter
  \endgroup
  \childdoctmp
}
%    \end{macrocode}

% \macro{\childdoc}
% The deprecated macro |\childdoc| is a legacy version of |\childdocmain|:
%    \begin{macrocode}
\newcommand{\childdoc}{\childdocmain}
%    \end{macrocode}

% \macro{\childdocredirect}
% The deprecated macro |\childdocredirect| is a legacy version
% of |\childdocforward| and |\childdocforwardprefix|:
%    \begin{macrocode}
\newcommand{\childdocredirect}[2][]
{
  \begingroup
    \if?#1?
      \def\childdoctmp{\childdocforward{#2}}
    \else
      \def\childdoctmp{\childdocforwardprefix{#1}{#2}}
    \fi
    \expandafter
  \endgroup
  \childdoctmp
}
%    \end{macrocode}

%\iffalse
%</package>
%\fi
%
\endinput
\childdocforward[|\textit{main}|]{|\textit{dest}|}"|
\end{center}
%
Here \textit{target} is the name of the output file,
\textit{main} is the name of the main file
and \textit{dest} is the name of the main or child file to be processed
(all filenames without extensions).
The optional argument \textit{main} can be omitted
if \textit{main} matches \textit{dest}.
Optionally, compilation \textit{flags} can be defined via |\def| commands.
This command line makes the \TeX{} engine believe
it is compiling the file \textit{target}
whose content is specified as the latter parameter.
The provided code then forwards the processing to
\textit{main} or \textit{dest} as described in \secref{sec:forward}.

%%%%%%%%%%%%%%%%%%%%%%%%%%%%%%%%%%%%%%%%%%%%%%%%%%%%%%%%%%%%%%%%%%%%%%%%%%%%%%%%
\subsection{Include by Input}
\label{sec:input}

Including child documents by |\include| has some restrictions by design.
Most notably, the content of a child document always occupies
its own set of pages; pages cannot be shared between child documents.
Usually, this behaviour makes perfect sense
because each child document contain an essential part of the document.
However, in some situations it may be desirable to compose
a document from a collection of parts
without having mandatory page breaks between then.
For this case, the package
provides a mechanism to include parts
by |\input| which can also be processed individually.
However, by construction this mechanism
requires manual handling of the content to be output.

%%%%%%%%%%%%%%%%%%%%%%%%%%%%%%%%%%%%%%%%
\DescribeMacro{\ifchilddocmanual}
The main file should be prepared as usual, see \secref{sec:include}.
However, the document body must make a distinction
between processing of an individual part and of the main document, e.g.:
%
\begin{center}
\begin{tabular}{l}
|\ifchilddocmanual|\\
|\input{\childdocname}|\\
|\||else|\\
\textit{document body with }|\input{|\textit{part}|}|\\
|\||fi|
\end{tabular}
\end{center}
%
The conditional |\ifchilddocmanual| is true whenever
a part to be included by |\input| is being compiled,
and the name of the part is stored in |\childdocname|.

%%%%%%%%%%%%%%%%%%%%%%%%%%%%%%%%%%%%%%%%
\DescribeMacro{\childdocby}
Each part to be included by |\input| should start with:
%
\begin{center}
\begin{tabular}{l}
|% \iffalse
%
% childdoc.dtx Copyright (C) 2017-2018 Niklas Beisert
%
% This work may be distributed and/or modified under the
% conditions of the LaTeX Project Public License, either version 1.3
% of this license or (at your option) any later version.
% The latest version of this license is in
%   http://www.latex-project.org/lppl.txt
% and version 1.3 or later is part of all distributions of LaTeX
% version 2005/12/01 or later.
%
% This work has the LPPL maintenance status `maintained'.
%
% The Current Maintainer of this work is Niklas Beisert.
%
% This work consists of the files childdoc.dtx and childdoc.ins
% and the derived files childdoc.def and cdocsamp.tex with
% cdocsch1.tex, cdocsch2.tex, cdocsdrf.tex, cdocsfn1.tex, cdocsfn2.tex.
%
%<package>\ifdefined\childdocmain\endinput\fi
%<package>\ProvidesFile{childdoc.def}[2018/12/30 v2.0 child document driver]
%<samplemain>\ProvidesFile{cdocsamp.tex}[2018/12/30 v2.0 sample for childdoc]
%<*driver>
%\ProvidesFile{childdoc.drv}[2018/12/30 v2.0 childdoc reference manual file]
\PassOptionsToClass{10pt,a4paper}{article}
\documentclass{ltxdoc}

\usepackage[margin=35mm]{geometry}
\usepackage{hyperref}
\usepackage{hyperxmp}
\usepackage[usenames]{color}

\hypersetup{colorlinks=true}
\hypersetup{pdfstartview=FitH}
\hypersetup{pdfpagemode=UseNone}
\hypersetup{pdfsource={}}
\hypersetup{pdflang={en-UK}}
\hypersetup{pdfcopyright={Copyright 2017-2018 Niklas Beisert.
  This work may be distributed and/or modified under the
  conditions of the LaTeX Project Public License, either version 1.3
  of this license or (at your option) any later version.}}
\hypersetup{pdflicenseurl={http://www.latex-project.org/lppl.txt}}
\hypersetup{pdfcontactaddress={ETH Zurich, ITP, HIT K,
  Wolfgang-Pauli-Strasse 27}}
\hypersetup{pdfcontactpostcode={8093}}
\hypersetup{pdfcontactcity={Zurich}}
\hypersetup{pdfcontactcountry={Switzerland}}
\hypersetup{pdfcontactemail={nbeisert@itp.phys.ethz.ch}}
\hypersetup{pdfcontacturl={http://people.phys.ethz.ch/\xmptilde nbeisert/}}

\newcommand{\secref}[1]{\hyperref[#1]{section \ref*{#1}}}

\parskip1ex
\parindent0pt
\let\olditemize\itemize
\def\itemize{\olditemize\parskip0pt}

\begin{document}

\title{The \textsf{childdoc} Package}
\hypersetup{pdftitle={The childdoc Package}}
\author{Niklas Beisert\\[2ex]
  Institut f\"ur Theoretische Physik\\
  Eidgen\"ossische Technische Hochschule Z\"urich\\
  Wolfgang-Pauli-Strasse 27, 8093 Z\"urich, Switzerland\\[1ex]
  \href{mailto:nbeisert@itp.phys.ethz.ch}
  {\texttt{nbeisert@itp.phys.ethz.ch}}}
\hypersetup{pdfauthor={Niklas Beisert}}
\hypersetup{pdfsubject={Manual for the LaTeX2e Package childdoc}}
\date{30 December 2018, \textsf{v2.0}}
\maketitle

\begin{abstract}\noindent
\textsf{childdoc} is a \LaTeXe{} package
that enables the direct compilation
of document sections included by |\include|
to individual files.
\end{abstract}

\begingroup
\parskip0ex
\tableofcontents
\endgroup

%%%%%%%%%%%%%%%%%%%%%%%%%%%%%%%%%%%%%%%%%%%%%%%%%%%%%%%%%%%%%%%%%%%%%%%%%%%%%%%%
%%%%%%%%%%%%%%%%%%%%%%%%%%%%%%%%%%%%%%%%%%%%%%%%%%%%%%%%%%%%%%%%%%%%%%%%%%%%%%%%
\section{Introduction}

\LaTeX{} provides a mechanism to structure a large document (such as a book)
into a main file and several child files (containing the chapters)
using the |\include| command.
This mechanism is beneficial for documents
which span hundreds of pages in order to
make the source file(s) more manageable.
Moreover, compilation can be restricted to
selected child files by means of the |\includeonly| command.
The latter feature can be used to reduce the compilation time while editing
(this was significantly more useful in the earlier days of \LaTeX{})
or to generate a smaller document which is easier to navigate.
Another application of |\includeonly| is to generate
documents consisting of selected parts of the complete document.

However, there are a few drawbacks of the plain |\include| mechanism:
\begin{itemize}
\item
The child files cannot be compiled on their own,
they can only be compiled via the main file.
A naive editing environment
(such as a text editor with an option
to have the current file processed by \LaTeX)
may require one to switch to the main file before compiling;
attempting to compile the child file produces errors.
\item
The main file must be modified (each time)
to adjust the |\includeonly| command
to the present needs. This easily leaves the main file in a messy state.
\item
The generated document will always carry the filename
of the main document. This is inconvenient if
several child files are to be compiled and
to be kept for distribution.
\end{itemize}

The present package provides a simple interface
to make child files individually compilable by \LaTeX{}.
Compiling a child file then has the same effect as compiling
the main file with an |\includeonly| command
to select the appropriate child.
Moreover the generated document will carry the name of the child
rather than the main file.
This resolves all three above issues.

This feature is meant to make the editing of books,
thesis documents and lecture notes somewhat more convenient.
However, the package can also be used efficiently for
composing a series of documents (such as exercise sheets)
which are typically distributed individually.
It then assists the author in generating the individual documents
(potentially in different versions)
as well as a document containing the collected series.
Another application is in developing style files
or other kinds of included material
where compilation of the style file could redirect
to a sample or test file.

%%%%%%%%%%%%%%%%%%%%%%%%%%%%%%%%%%%%%%%%%%%%%%%%%%%%%%%%%%%%%%%%%%%%%%%%%%%%%%%%
%%%%%%%%%%%%%%%%%%%%%%%%%%%%%%%%%%%%%%%%%%%%%%%%%%%%%%%%%%%%%%%%%%%%%%%%%%%%%%%%
\section{Usage}

First of all, the package \textsf{childdoc} is \emph{not} a standard
\LaTeXe{} |.sty| style file! Therefore it needs to be invoked in
a non-standard way.

%%%%%%%%%%%%%%%%%%%%%%%%%%%%%%%%%%%%%%%%%%%%%%%%%%%%%%%%%%%%%%%%%%%%%%%%%%%%%%%%
\subsection{Included Files}
\label{sec:include}

%%%%%%%%%%%%%%%%%%%%%%%%%%%%%%%%%%%%%%%%
\DescribeMacro{\childdocmain}
To use the package, add the commands
\begin{center}
\begin{tabular}{l}
|\input{childdoc.def}|\\
|\childdocmain{}|\\
\end{tabular}
\end{center}
at the very top of the main \LaTeX{} file,
in particular \emph{before} the |\documentclass| statement!
The argument of |\childdocmain| should be left empty
(but it must be present).

%%%%%%%%%%%%%%%%%%%%%%%%%%%%%%%%%%%%%%%%
\DescribeMacro{\childdocof}
Furthermore, add the commands
\begin{center}
\begin{tabular}{l}
|\input{childdoc.def}|\\
|\childdocof{|\textit{main}|}|\\
\end{tabular}
\end{center}
at the top of every child file \textit{child}
which is included by |\include{|\textit{child}|}|
from within the main file
(or at least for those files to be compiled individually).
The argument \textit{main} must be the filename of the main file.

There are a couple of
considerations in setting up the main and child documents:

%%%%%%%%%%%%%%%%%%%%%%%%%%%%%%%%%%%%%%%%
\paragraph{Restrictions.}

Please note the following restrictions:
\begin{itemize}
\item
|\childdocmain| must be called with one argument \textit{main}
to ensure compatibility with earlier version of the package.
It must either be empty (|\childdocmain{}|)
or precisely match the filename of the main file in which it is specified.
See \secref{sec:detection} for further information.
\item
The filename \textit{main} must be specified without the |.tex| extension.
\item
The filename \textit{main} is case sensitive
(even in case-insensitive file systems)
due to internal string comparison.
\item
The argument \textit{main} should be fully expanded, it cannot be a macro.
\item
Subdirectories and special characters should be avoided in filenames.
\item
The command |\childdocmain{|\textit{main}|}| must be followed by a whitespace.
It should not be followed immediately by another command
or by a comment mark `|%|'.
This is because the \TeX{} parser reads the token immediately following
the argument of |\childdocmain| and puts it
at the beginning of every child section;
however, a white\-space is ignored.
\end{itemize}

%%%%%%%%%%%%%%%%%%%%%%%%%%%%%%%%%%%%%%%%
\paragraph{Content of Main File.}

It is advisable to place all content in the child files included by |\include|.
Any output contained in the main file will appear in all child documents
unless suppressed manually;
it cannot be suppressed automatically by the |\includeonly| directive
and thus should normally be avoided.
A method to include some content in the main file
by means of conditional processing is described in \secref{sec:conditional}.

%%%%%%%%%%%%%%%%%%%%%%%%%%%%%%%%%%%%%%%%
\paragraph{Page Numbering.}

When only a part of the document is compiled,
the appropriate numbering of pages
(as well as other status parameters)
is determined from the |.aux| files.
The latter contain information from previous passes.
However this information needs to propagate through
all intermediate child documents.
Therefore the page numbering in child documents may well
be inconsistent until the complete document is compiled at least once.

A useful (if unconventional) way to always ensure a consistent
page numbering is to restart the numbering in each child document
and denote the pages by `\textit{child}|.|\textit{page}'
where \textit{child} represents the chapter/section number of the child file.
This can be achieved by the command
|\numberwithin{page}{|\textit{child}|}|
of the \textsf{amsmath} package
where \textit{child} can be |chapter| or |section|
depending on the chosen structuring.
Alternatively, one can modify the macro |\thepage| appropriately
and reset the counter |page| at the start of each child file.

%%%%%%%%%%%%%%%%%%%%%%%%%%%%%%%%%%%%%%%%%%%%%%%%%%%%%%%%%%%%%%%%%%%%%%%%%%%%%%%%
\subsection{Conditional Processing}
\label{sec:conditional}

The package provides a mechanism to compile different versions
of a document. To customise the versions further some conditional processing
can come in handy to distinguish which version is being compiled.
The package provides two macros to describe the compilation context:

%%%%%%%%%%%%%%%%%%%%%%%%%%%%%%%%%%%%%%%%
\DescribeMacro{\ifchilddoc}
The conditional |\ifchilddoc| distinguishes between the compilation of
child documents and the main document:
%
\begin{center}
|\ifchilddoc |\textit{child-code}| |[|\||else |\textit{main-code}]| \||fi|
\end{center}

%%%%%%%%%%%%%%%%%%%%%%%%%%%%%%%%%%%%%%%%
\DescribeMacro{\childdocname}
\DescribeMacro{\childdocjob}
The macro |\childdocname| contains the filename (without extension)
of the main or child file being processed.
Note that |\childdocjob| will always contain the name of the main file.

%%%%%%%%%%%%%%%%%%%%%%%%%%%%%%%%%%%%%%%%
\paragraph{Title Page.}

Conditional processing can be used to include a title or banner page
in the main document when proper precautions are taken.
Importantly, the code in the main file should ensure that the page counter
(as well as other status parameters which are stored in the |.aux| files)
takes the same value after the conditional processing.
Otherwise the page numbers may take divergent values
depending on which part is compiled.

For example, a title page could be declared by:
%
\begin{center}
\begin{tabular}{l}
|\ifchilddoc\||else|\\
|\addtocounter{page}{-1}|\\
\textit{code for title page}\\
|\newpage|\\
|\||fi|
\end{tabular}
\end{center}
%
A banner page for the child documents can be generated by:
%
\begin{center}
\begin{tabular}{l}
|\ifchilddoc|\\
|\addtocounter{page}{-1}|\\
\textit{code for banner page}\\
|\newpage|\\
|\||fi|
\end{tabular}
\end{center}
%
Here one could write a message such as:
\begin{center}
|This is the part \childdocname{} of \childdocjob{}.|
\end{center}

%%%%%%%%%%%%%%%%%%%%%%%%%%%%%%%%%%%%%%%%%%%%%%%%%%%%%%%%%%%%%%%%%%%%%%%%%%%%%%%%
\subsection{Flags}
\label{sec:flags}

The package makes it easy to generate different versions
of the main or child documents.
To this end compilation flags can be defined
and assigned different default values.
They will be particularly useful in conjunction
with the forwarding mechanism described in \secref{sec:forward}.

For example, it may be useful to have a flag |\version|
which can be set to |draft| or |final|.
The document source will contain some conditional code
depending on the value of |\version|.
Suppose further, the flag should default to |final| for the main file
and to |draft| for child files
which is a natural assignment for editing the document.
This is achieved by placing the following code
in the preamble of the main document
(below the |\childdocmain| directive):
%
\begin{center}
\begin{tabular}{l}
|\ifchilddoc|\\
|\providecommand{\version}{draft}|\\
|\||else|\\
|\providecommand{\version}{final}|\\
|\||fi|
\end{tabular}
\end{center}
%
The definition by |\providecommand| makes sure
that previous definitions are not overwritten.
Further statements |\providecommand{\version}{...}|
can thus be added before the above code to override it.

For the main file, one might add a line
(between |\childdocmain| and the above block)
%
\begin{center}
|%\ifchilddoc\||else\providecommand{\version}{draft}\||fi|
\end{center}
%
which can be uncommented to produce a draft version.
Likewise one can add a line to the very top of a child file
(above the |\childdocof{|\textit{main}|}| directive)
%
\begin{center}
|%\providecommand{\version}{final}|
\end{center}
%
which can be uncommented to produce the final version of this child document.

%%%%%%%%%%%%%%%%%%%%%%%%%%%%%%%%%%%%%%%%%%%%%%%%%%%%%%%%%%%%%%%%%%%%%%%%%%%%%%%%
\subsection{Forwarding}
\label{sec:forward}

Different versions of the main or child documents
using compilation flags as described in \secref{sec:flags}
can be (permanently) stored in different files
for convenient compilation, viewing and distribution.
To this end, the package defines a command
to pass on compilation to a different file:

%%%%%%%%%%%%%%%%%%%%%%%%%%%%%%%%%%%%%%%%
\DescribeMacro{\childdocforward}
The command |\childdocforward| redirects processing to
another source file:
%
\begin{center}
\begin{tabular}{l}
|\input{childdoc.def}|\\
|\childdocforward[|\textit{main}|]{|\textit{dest}|}|\\
\end{tabular}
\end{center}
%
The argument \textit{dest} is the destination file
(without extension).
It should be the main file or one of the child files.
Note that further \textsf{childdoc} directives
such as |\childdocof| and |\childdocforward|
in the indicated file will be processed in this form.
The optional argument \textit{main}
passes on directly to the main file \textit{main}
while pretending to compile the child \textit{dest}.
This form behaves as if \textit{dest}
issues |\childdocof{|\textit{main}|}| right away,
and no further \textsf{childdoc} directives will be processed.

%%%%%%%%%%%%%%%%%%%%%%%%%%%%%%%%%%%%%%%%
\DescribeMacro{\...prefix}
In the alternative form |\childdocforwardprefix|,
%
\begin{center}
\begin{tabular}{l}
|\input{childdoc.def}|\\
|\childdocforwardprefix[|\textit{main}|]{|\textit{prefix}|}{|\textit{dest}|}|
\end{tabular}
\end{center}
%
the destination file is determined by a pattern
depending on the current file:
To make this work, the current file must be called
`{\textit{prefix}\hspace{0.2em}\textit{suffix}}'
with \textit{prefix} matching precisely the argument.
Processing is then passed on to the file
`{\textit{dest}\hspace{0.2em}\textit{suffix}}'.
Surely, the same effect is achieved by
directly specifying the
argument `{\textit{dest}\hspace{0.2em}\textit{suffix}}'
in the first form.
However, that requires to set up a different file
for each child. With the alternative form of the command
all these files can have exactly the same content
which simplifies setting them up and maintaining them.

For example, the following file |draft.tex|
with a compilation flag |\version| as described in \secref{sec:flags}
compiles the main document as a draft:
%
\begin{center}
\begin{tabular}{l}
|\def\version{draft}|\\
|\input{childdoc.def}|\\
|\childdocforward{|\textit{main}|}|
\end{tabular}
\end{center}
%
Likewise, the following files |final|\textit{nn}|.tex|
compile the final version of the child document
|child|\textit{nn}|.tex|:
%
\begin{center}
\begin{tabular}{l}
|\def\version{final}|\\
|\input{childdoc.def}|\\
|\childdocforwardprefix{final}{child}|
\end{tabular}
\end{center}
%

Note that when several versions of a main file and/or of each child file
are to be generated, it may be convenient to set up a |Makefile| or
shell script to automatise the process.

%%%%%%%%%%%%%%%%%%%%%%%%%%%%%%%%%%%%%%%%%%%%%%%%%%%%%%%%%%%%%%%%%%%%%%%%%%%%%%%%
\subsection{Command Line Processing}
\label{sec:commandline}

The effect of redirection files can also be achieved by invoking
the \LaTeX{} compiler with a more elaborate command line.
Most conveniently this should be done as part
of a shell script or a |Makefile|.

When using \textsf{childdoc} in the main file, the following
command lines effectively perform a redirection
(note that depending on the shell being used,
backslashes may have to be doubled: `|\|' $\to$ `|\\|'):
%
\begin{center}
|... -jobname "|\textit{target}|" |\\|"|[\textit{flags}]%
|\input{childdoc.def}\childdocforward[|\textit{main}|]{|\textit{dest}|}"|
\end{center}
%
Here \textit{target} is the name of the output file,
\textit{main} is the name of the main file
and \textit{dest} is the name of the main or child file to be processed
(all filenames without extensions).
The optional argument \textit{main} can be omitted
if \textit{main} matches \textit{dest}.
Optionally, compilation \textit{flags} can be defined via |\def| commands.
This command line makes the \TeX{} engine believe
it is compiling the file \textit{target}
whose content is specified as the latter parameter.
The provided code then forwards the processing to
\textit{main} or \textit{dest} as described in \secref{sec:forward}.

%%%%%%%%%%%%%%%%%%%%%%%%%%%%%%%%%%%%%%%%%%%%%%%%%%%%%%%%%%%%%%%%%%%%%%%%%%%%%%%%
\subsection{Include by Input}
\label{sec:input}

Including child documents by |\include| has some restrictions by design.
Most notably, the content of a child document always occupies
its own set of pages; pages cannot be shared between child documents.
Usually, this behaviour makes perfect sense
because each child document contain an essential part of the document.
However, in some situations it may be desirable to compose
a document from a collection of parts
without having mandatory page breaks between then.
For this case, the package
provides a mechanism to include parts
by |\input| which can also be processed individually.
However, by construction this mechanism
requires manual handling of the content to be output.

%%%%%%%%%%%%%%%%%%%%%%%%%%%%%%%%%%%%%%%%
\DescribeMacro{\ifchilddocmanual}
The main file should be prepared as usual, see \secref{sec:include}.
However, the document body must make a distinction
between processing of an individual part and of the main document, e.g.:
%
\begin{center}
\begin{tabular}{l}
|\ifchilddocmanual|\\
|\input{\childdocname}|\\
|\||else|\\
\textit{document body with }|\input{|\textit{part}|}|\\
|\||fi|
\end{tabular}
\end{center}
%
The conditional |\ifchilddocmanual| is true whenever
a part to be included by |\input| is being compiled,
and the name of the part is stored in |\childdocname|.

%%%%%%%%%%%%%%%%%%%%%%%%%%%%%%%%%%%%%%%%
\DescribeMacro{\childdocby}
Each part to be included by |\input| should start with:
%
\begin{center}
\begin{tabular}{l}
|\input{childdoc.def}|\\
|\childdocby{|\textit{main}|}|\\
\end{tabular}
\end{center}
%
The directive |\childdocby| is similar to |\childdocof|
described in \secref{sec:include},
but the subsequent selection of content must be done manually.
To that end, both |\ifchilddoc| and |\ifchilddocmanual|
will be true upon processing of a part,
and the name of the part is stored in |\childdocname|.
Note that |\jobname| will be set to the filename of the current part
so that each part receives an individual |.aux| file
that does not interfere with the |.aux| file(s) of the main document.
This behaviour can be altered by the alternative form
|\childdocby[*]{|\textit{main}|}| (with a non-empty optional argument)
which uses the |.aux| file of the main document
by setting |\jobname| to \textit{main}.

%%%%%%%%%%%%%%%%%%%%%%%%%%%%%%%%%%%%%%%%%%%%%%%%%%%%%%%%%%%%%%%%%%%%%%%%%%%%%%%%
\subsection{Driver Development}
\label{sec:driver}

The \textsf{childdoc} mechanism can also be use for the development
of definition files such as \LaTeX{} styles or classes.
This case differs from the above setup with multiple parts
included by |\include| in that no |\includeonly| should be invoked.
This can be achieved by starting the include file
(before |\ProvidesPackage|) with:
%
\begin{center}
\begin{tabular}{l}
|\input{childdoc.def}|\\
|\childdocforward{|\textit{main}|}|\\
\end{tabular}
\end{center}
%
or alternatively with:
%
\begin{center}
\begin{tabular}{l}
|\input{childdoc.def}|\\
|\childdocby{|\textit{main}|}|\\
\end{tabular}
\end{center}
%
Both forms have slightly different effects as described above.
The main file is prepared as usual, see \secref{sec:include}.

%%%%%%%%%%%%%%%%%%%%%%%%%%%%%%%%%%%%%%%%%%%%%%%%%%%%%%%%%%%%%%%%%%%%%%%%%%%%%%%%
\subsection{Legacy Detection}
\label{sec:detection}

The directive |\childdocmain| in the main file can detect
whether the complete document or merely a child is to be compiled
even without using the directive |\childdocof|.
This method is deprecated because it is less robust
and there is no compelling reason to use it;
it is merely provided for backward compatibility
and it may be removed in future versions.

If the detection mechanism is to be used,
it is mandatory to correctly specify
the filename of the main file as the argument of |\childdocmain|:
%
\begin{center}
\begin{tabular}{l}
|\input{childdoc.def}|\\
|\childdocmain{|\textit{main}|}|\\
\end{tabular}
\end{center}
%
If |\jobname| does not match the argument \textit{main} of |\childdocmain|,
it is assumed that |\jobname| points to the child file to be compiled.
When using |\childdocmain| with the main file specified as argument,
it suffices to start a child file
with just |\input{|\textit{main}|}|
without loading of the package and using |\childdocof|.
If instead all processing is done
with the appropriate \textsf{childdoc} directives,
the argument of \textit{main} of |\childdocmain| can be empty.

An alternative version of the command line processing described
in \secref{sec:commandline} using the detection mechanism reads:
%
\begin{center}
|... -jobname "|\textit{target}|" "|[\textit{flags}]%
[|\def\jobname{|\textit{dest}|}|]|\input{|\textit{main}|}"|
\end{center}

%%%%%%%%%%%%%%%%%%%%%%%%%%%%%%%%%%%%%%%%%%%%%%%%%%%%%%%%%%%%%%%%%%%%%%%%%%%%%%%%
\subsection{Manual Code}
\label{sec:manual}

In case one cannot be certain whether the definitions file |childdoc.def|
is installed on the target \TeX{} distribution
and one prefers not to ship it,
it is conceivable to paste a few relevant commands into the sources.

To that end, drop all statements |\input{childdoc.def}|
and perform the replacements as outlined below.
Instead of |\childdocmain{|\textit{main}|}| add the following code
to the top of the main file:
%
\begin{center}
\begin{tabular}{l}
|\||ifdefined\childdocname\endinput\||fi\newif\ifchilddoc|\\
|\edef\childdocname{\scantokens\expandafter{\jobname\noexpand}}|\\
|\def\childdocmain{|\textit{main}|}\||ifx\childdocmain\childdocname\||else|\\
|\childdoctrue\includeonly{\childdocname}\let\jobname\childdocmain\||fi|\\
\end{tabular}
\end{center}
%
Instead of |\childdocof{|\textit{main}|}| just include the main file
at the top of each child file:
%
\begin{center}
|\input{|\textit{main}|}|
\end{center}
%
A simple redirection |\childdocforward{|\textit{dest}|}| is achieved by:
%
\begin{center}
|\def\jobname{|\textit{dest}|}\input{\jobname}|
\end{center}
%
The redirection with prefix
|\childdocforwardprefix[|\textit{prefix}|]{|\textit{dest}|}|
is accomplished by:
%
\begin{center}
\begin{tabular}{l}
|{\edef\jobname{\scantokens\expandafter{\jobname\noexpand}}|\\
|\def\redirectjob |\textit{prefix}|#1~~~{\gdef\jobname{|\textit{dest}|#1}}|\\
|\expandafter\redirectjob\jobname~~~}\input{\jobname}|
\end{tabular}
\end{center}

In an alternative approach,
child documents can be compiled by a specific command line
without additional code or specific definitions:
%
\begin{center}
|... -jobname "|\textit{target}|" "|[\textit{flags}]%
|\includeonly{|\textit{dest}|}\input{|\textit{main}|}"|
\end{center}
%

%%%%%%%%%%%%%%%%%%%%%%%%%%%%%%%%%%%%%%%%%%%%%%%%%%%%%%%%%%%%%%%%%%%%%%%%%%%%%%%%
%%%%%%%%%%%%%%%%%%%%%%%%%%%%%%%%%%%%%%%%%%%%%%%%%%%%%%%%%%%%%%%%%%%%%%%%%%%%%%%%
\section{Information}

%%%%%%%%%%%%%%%%%%%%%%%%%%%%%%%%%%%%%%%%%%%%%%%%%%%%%%%%%%%%%%%%%%%%%%%%%%%%%%%%
\subsection{Copyright}

Copyright \copyright{} 2017--2018 Niklas Beisert

This work may be distributed and/or modified under the
conditions of the \LaTeX{} Project Public License, either version 1.3
of this license or (at your option) any later version.
The latest version of this license is in
  \url{http://www.latex-project.org/lppl.txt}
and version 1.3 or later is part of all distributions of \LaTeX{}
version 2005/12/01 or later.

This work has the LPPL maintenance status `maintained'.

The Current Maintainer of this work is Niklas Beisert.

This work consists of the files |README.txt|, |childdoc.ins| and |childdoc.dtx|
as well as the derived files |childdoc.def|, |cdocsamp.tex|
with |cdocsch1.tex|, |cdocsch2.tex|, |cdocspt3.tex|, |cdocspt4.tex|,
|cdocsdrf.tex|, |cdocsfn1.tex|, |cdocsfn2.tex|
as well as |childdoc.pdf|.

%%%%%%%%%%%%%%%%%%%%%%%%%%%%%%%%%%%%%%%%%%%%%%%%%%%%%%%%%%%%%%%%%%%%%%%%%%%%%%%%
\subsection{Files and Installation}

The package consists of the files:
%
\begin{center}
\begin{tabular}{ll}
    |README.txt|   & readme file \\
    |childdoc.ins| & installation file \\
    |childdoc.dtx| & source file \\
    |childdoc.def| & definition file \\
    |cdocsamp.tex| & sample main file \\
    |cdocsch1.tex| & sample include file \\
    |cdocsch2.tex| & sample include file \\
    |cdocspt3.tex| & sample part file \\
    |cdocspt4.tex| & sample part file \\
    |cdocsdrf.tex| & sample redirection file \\
    |cdocsfn1.tex| & sample redirection file \\
    |cdocsfn2.tex| & sample redirection file \\
    |childdoc.pdf| & manual
\end{tabular}
\end{center}
%
The distribution consists of the files
|README.txt|, |childdoc.ins| and |childdoc.dtx|.
%
\begin{itemize}
\item
Run (pdf)\LaTeX{} on |childdoc.dtx|
to compile the manual |childdoc.pdf| (this file).
\item
Run \LaTeX{} on |childdoc.ins| to create the definitions file |childdoc.def|
and the sample |cdocsamp.tex| with include files
|cdocsch1.tex|, |cdocsch2.tex|, |cdocspt3.tex|, |cdocspt4.tex|,
|cdocsdrf.tex|, |cdocsfn1.tex|, |cdocsfn2.tex|.
Then copy the file |childdoc.def| to an appropriate directory of your \LaTeX{}
distribution, e.g.\ \textit{texmf-root}|/tex/latex/childdoc|.
\end{itemize}

%%%%%%%%%%%%%%%%%%%%%%%%%%%%%%%%%%%%%%%%%%%%%%%%%%%%%%%%%%%%%%%%%%%%%%%%%%%%%%%%
\subsection{Related CTAN Packages}

There are several other packages which offer a similar functionality:
%
\begin{itemize}
\item
The packages
\href{http://ctan.org/pkg/docmute}{\textsf{docmute}},
\href{http://ctan.org/pkg/includex}{\textsf{includex}} and
\href{http://ctan.org/pkg/standalone}{\textsf{standalone}}
provide commands to include only the document body of
a child file thus allowing both files to be compiled individually.
\item
The packages \href{http://ctan.org/pkg/subdocs}{\textsf{subdocs}}
and \href{http://ctan.org/pkg/subfiles}{\textsf{subfiles}}
provide structures in which the main and child documents can be
encapsulated and allowing them to be compiled individually.
The inclusion mechanism is different from the conventional |\include|.
\item
The package \href{http://ctan.org/pkg/combine}{\textsf{combine}}
is an elaborate solution to combine several documents into one.
\end{itemize}
%
See also the CTAN topic \href{http://ctan.org/topic/subdocs}{\textsf{subdocs}}
for further related packages.
The present package differs from the above solutions in that
a document structure constructed with the conventional |\include| mechanism
just needs two extra commands at the top of every file
such that all constituent files can be compiled individually.

%%%%%%%%%%%%%%%%%%%%%%%%%%%%%%%%%%%%%%%%%%%%%%%%%%%%%%%%%%%%%%%%%%%%%%%%%%%%%%%%
%\subsection{Feature Suggestions}
%
%The following is a list of features which may be useful for future
%versions of this package:
%%
%\begin{itemize}
%\item
%\ldots
%\end{itemize}

%%%%%%%%%%%%%%%%%%%%%%%%%%%%%%%%%%%%%%%%%%%%%%%%%%%%%%%%%%%%%%%%%%%%%%%%%%%%%%%%
\subsection{Revision History}

%%%%%%%%%%%%%%%%%%%%%%%%%%%%%%%%%%%%%%%%
\paragraph{v2.0:} 2018/12/30

\begin{itemize}
\item
immediate forward processing
\item
added |\childdocby| mechanism
\item
manual restructured
\end{itemize}

%%%%%%%%%%%%%%%%%%%%%%%%%%%%%%%%%%%%%%%%
\paragraph{v1.6:} 2018/01/17

\begin{itemize}
\item
application for development of include files
\item
corrections to manual
\end{itemize}

%%%%%%%%%%%%%%%%%%%%%%%%%%%%%%%%%%%%%%%%
\paragraph{v1.5:} 2017/05/21

\begin{itemize}
\item
more complete structuring introduced
\item
|\childdocof| introduced
\item
|\childdoc| renamed to |\childdocmain|
\item
|\childredirect| renamed to |\childdocforward| and |\childdocforwardprefix|
and functionality expanded
\end{itemize}

%%%%%%%%%%%%%%%%%%%%%%%%%%%%%%%%%%%%%%%%
\paragraph{v1.0:} 2017/04/27

\begin{itemize}
\item
manual and install package
\item
first version published on CTAN
\end{itemize}

%%%%%%%%%%%%%%%%%%%%%%%%%%%%%%%%%%%%%%%%
\paragraph{v0.6:} 2017/04/26

\begin{itemize}
\item
redirection mechanism added
\end{itemize}

%%%%%%%%%%%%%%%%%%%%%%%%%%%%%%%%%%%%%%%%
\paragraph{v0.5:} 2017/04/26

\begin{itemize}
\item
functionality in definition file
\end{itemize}


%%%%%%%%%%%%%%%%%%%%%%%%%%%%%%%%%%%%%%%%%%%%%%%%%%%%%%%%%%%%%%%%%%%%%%%%%%%%%%%%
%%%%%%%%%%%%%%%%%%%%%%%%%%%%%%%%%%%%%%%%%%%%%%%%%%%%%%%%%%%%%%%%%%%%%%%%%%%%%%%%
%%%%%%%%%%%%%%%%%%%%%%%%%%%%%%%%%%%%%%%%%%%%%%%%%%%%%%%%%%%%%%%%%%%%%%%%%%%%%%%%
\appendix

\settowidth\MacroIndent{\rmfamily\scriptsize 000\ }

 \DocInput{childdoc.dtx}

\end{document}
%</driver>
% \fi
%
% %%%%%%%%%%%%%%%%%%%%%%%%%%%%%%%%%%%%%%%%%%%%%%%%%%%%%%%%%%%%%%%%%%%%%%%%%%%%%%
% %%%%%%%%%%%%%%%%%%%%%%%%%%%%%%%%%%%%%%%%%%%%%%%%%%%%%%%%%%%%%%%%%%%%%%%%%%%%%%
% \section{Sample}
%\iffalse
%<*samplemain>
%\fi
%
% The following presents a sample document
% with two chapters, two parts, a title page,
% a compile flag as well as three forwarding files to set the flag.
% It consists of eight |.tex| files:
% \begin{center}
% \begin{tabular}{ll}
% |cdocsamp.tex|&main file\\
% |cdocsch1.tex|&include file for chapter 1\\
% |cdocsch2.tex|&include file for chapter 2\\
% |cdocspt3.tex|&include file for part 3\\
% |cdocspt4.tex|&include file for part 4\\
% |cdocsdrf.tex|&forwarding file for main file in draft mode\\
% |cdocsfi1.tex|&forwarding file for final version of chapter 1\\
% |cdocsfi2.tex|&forwarding file for final version of chapter 2\\
% \end{tabular}
% \end{center}
% Each of the eight files can be compiled directly by the \LaTeX{} compiler.
%
% %%%%%%%%%%%%%%%%%%%%%%%%%%%%%%%%%%%%%%
% \paragraph{Main File.}
%
% The main file is called |cdocsamp.tex|.
%
% Load the \textsf{childdoc} definitions and
% declare the filename for the main document:
%    \begin{macrocode}
\input{childdoc.def}
\childdocmain{}
%    \end{macrocode}

% Optional override for |\version| flag:
%    \begin{macrocode}
%%\ifchilddoc\else\providecommand{\version}{draft}\fi
%    \end{macrocode}

% Define the default values for the |\version| flag
% (|final| for the main file and |draft| for childs):
%    \begin{macrocode}
\ifchilddoc
\providecommand{\version}{draft}
\else
\providecommand{\version}{final}
\fi
%    \end{macrocode}

% Load the standard document class:
%    \begin{macrocode}
\documentclass[12pt]{article}
%    \end{macrocode}

% Start the document body:
%    \begin{macrocode}
\begin{document}
%    \end{macrocode}

% Declare a title page.
% Print title, part of document being processed and version flag:
%    \begin{macrocode}
\addtocounter{page}{-1}
\begin{center}
{\LARGE\bfseries{}childdoc example\par}
\vspace{1cm}
\ifchilddoc
\ifchilddocmanual part\else chapter\fi:
`\childdocname' of `\childdocjob'\par
\else
main document: `\childdocjob'\par
\fi
version: \version\par
\end{center}
\newpage
%    \end{macrocode}

% Manually include selected file,
% otherwise process as usual:
%    \begin{macrocode}
\ifchilddocmanual
\section*{part `\childdocname'}
\input{\childdocname}
\else
%    \end{macrocode}

% Include the two chapters:
%    \begin{macrocode}
\include{cdocsch1}
\include{cdocsch2}
%    \end{macrocode}

% Include the two parts unless only chapters should be displayed:
%    \begin{macrocode}
\ifchilddoc\else
\section{part three}
\input{cdocspt3}
\section{part four}
\input{cdocspt4}
\fi
%    \end{macrocode}

% Process as usual until here:
%    \begin{macrocode}
\fi
%    \end{macrocode}

% End of document body:
%    \begin{macrocode}
\end{document}
%    \end{macrocode}
%\iffalse
%</samplemain>
%\fi
%
% %%%%%%%%%%%%%%%%%%%%%%%%%%%%%%%%%%%%%%
% \paragraph{Chapter Include Files.}
%
% The include files are called |cdocsch1.tex| and |cdocsch2.tex|.
%
%\iffalse
%<*samplechap1|samplechap2>
%\fi

% Optional override for |\version| flag:
%    \begin{macrocode}
%%\providecommand{\version}{final}
%    \end{macrocode}

% Include the main document:
%    \begin{macrocode}
\input{childdoc.def}
\childdocof{cdocsamp}
%    \end{macrocode}

%\iffalse
%</samplechap1|samplechap2>
%\fi
%
%\iffalse
%<*samplechap1>
%\fi
% Some text for chapter 1:
%    \begin{macrocode}
\section{one}
some text in chapter one
%    \end{macrocode}

%\iffalse
%</samplechap1>
%\fi
% Some text for chapter 2:
%\iffalse
%<*samplechap2>
%\fi
%    \begin{macrocode}
\section{two}
more text in chapter two
%    \end{macrocode}

%\iffalse
%</samplechap2>
%\fi
%
% %%%%%%%%%%%%%%%%%%%%%%%%%%%%%%%%%%%%%%
% \paragraph{Part Include Files.}
%
% The include files are called |cdocspt3.tex| and |cdocspt4.tex|.
%
%\iffalse
%<*samplepart3|samplepart4>
%\fi

% Optional override for |\version| flag:
%    \begin{macrocode}
%%\providecommand{\version}{final}
%    \end{macrocode}

% Include the main document:
%    \begin{macrocode}
\input{childdoc.def}
\childdocby{cdocsamp}
%    \end{macrocode}

%\iffalse
%</samplepart3|samplepart4>
%\fi
%
%\iffalse
%<*samplepart3>
%\fi
% Some text for part 3:
%    \begin{macrocode}
some text in part three
%    \end{macrocode}

%\iffalse
%</samplepart3>
%\fi
% Some text for part 4:
%\iffalse
%<*samplepart4>
%\fi
%    \begin{macrocode}
more text in part four
%    \end{macrocode}

%\iffalse
%</samplepart4>
%\fi
%
% %%%%%%%%%%%%%%%%%%%%%%%%%%%%%%%%%%%%%%
% \paragraph{Forwarding for a Complete Draft.}
%
% The following forwarding file |cdocsdrf.tex|
% compiles the main document in draft mode:
%\iffalse
%<*sampledraft>
%\fi
%    \begin{macrocode}
\def\version{draft}
\input{childdoc.def}
\childdocforward{cdocsamp}
%    \end{macrocode}

%\iffalse
%</sampledraft>
%\fi
%
% %%%%%%%%%%%%%%%%%%%%%%%%%%%%%%%%%%%%%%
% \paragraph{Forwarding for Final Version of the Chapters.}
%
% The following forwarding files |cdocsfn1.tex| and |cdocsfn2.tex|
% (with identical content)
% compile the final versions of the child documents
% |cdocsch1.tex| and |cdocsch2.tex|, respectively:
%\iffalse
%<*samplefinal>
%\fi
%    \begin{macrocode}
\def\version{final}
\input{childdoc.def}
\childdocforwardprefix[cdocsamp]{cdocsfn}{cdocsch}
%    \end{macrocode}

%\iffalse
%</samplefinal>
%\fi
%
% %%%%%%%%%%%%%%%%%%%%%%%%%%%%%%%%%%%%%%
% \paragraph{Command Line Processing.}
%
% The following three command lines generate the output files
% |cdocscld|, |cdocscl1| and |cdocscl2|
% which should be identical to
% |cdocsdrf|, |cdocsch1| and |cdocsfn2|, respectively:
% \begin{center}
% \begin{tabular}{l}
% |latex -jobname cdocscld \|\\
% |  "\def\version{draft}\input{childdoc.def}\childdocforward{cdocsamp}"|\\
% |latex -jobname cdocscl1 \|\\
% |  "\input{childdoc.def}\childdocforward[cdocsamp]{cdocsch1}"|\\
% |latex -jobname cdocscl2 \|\\
% |  "\def\version{final}\input{childdoc.def}\childdocforward{cdocsch2}"|
% \end{tabular}
% \end{center}
% Note that the trailing backslash on each first line
% merely continues the input to the second line
% (for convenient cut ant paste).
% Furthermore, the command |latex| can be replaced by any
% of its alternative versions such as |pdflatex|.
%
% %%%%%%%%%%%%%%%%%%%%%%%%%%%%%%%%%%%%%%%%%%%%%%%%%%%%%%%%%%%%%%%%%%%%%%%%%%%%%%
% %%%%%%%%%%%%%%%%%%%%%%%%%%%%%%%%%%%%%%%%%%%%%%%%%%%%%%%%%%%%%%%%%%%%%%%%%%%%%%
% \section{Implementation}
%\iffalse
%<*package>
%\fi
%
% This section describes the definitions file |childdoc.def|.

% The definitions cannot be loaded using |\usepackage| or |\RequirePackage|
% which has a mechanism to prevent loading a style file more than once.
% When loading the definitions by means of |\input|
% multiple instances have to be prevented manually:
%\iffalse
%This code needs to be before the `\ProvidesFile' directive
%which is defined at the beginning of this file.
%Therefore it is also placed there and commented out here.
%</package>
%<*discard>
%\fi
%    \begin{macrocode}
\ifdefined\childdocmain\endinput\fi
%    \end{macrocode}
%\iffalse
%</discard>
%<*package>
%\fi
%
% \macro{\ifchilddoc}
% \macro{\ifchilddocmanual}
% The conditional |\ifchilddoc| tells whether a
% child (true) or main (false) document is being compiled.
% The conditional |\ifchilddocmanual| tells whether
% the |\includeonly| mechanism is used (false) or
% the selection of child files must be performed manually (true).
% The definitions initialise to false:
%    \begin{macrocode}
\newif\ifchilddoc
\newif\ifchilddocmanual
%    \end{macrocode}

% \macro{\childdocname}
% \macro{\childdocjob}
% The macro |\childdocname| stores the name of the main document
% to be compiled. The macro |\childdocjob| stores the name of
% the document on which the \LaTeX{} compiler was originally invoked.
% The content of |\jobname| cannot be compared
% to filenames specified in the source due to different catcodes.
% The following code rescans |\jobname|, stores the result
% in |\childdocname| and saves a copy in |\childdocjob|:
%    \begin{macrocode}
\edef\childdocname{\scantokens\expandafter{\jobname\noexpand}}
\let\childdocjob\childdocname
%    \end{macrocode}

% \macro{\childdocdisable}
% The macro |\childdocdisable| prevents the main file
% from being processed more than once.
% At this stage, the main document command |\childdocmain|
% is assumed to be called once again where it should do nothing.
% Any subsequent call to it should prevent
% a secondary processing of the main document
% It overwrites the forwarding commands
% |\childdocof| and |\childdocforward|
% with empty macros to prevent further inclusions of the main document:
%    \begin{macrocode}
\newcommand{\childdocdisable}
{
  \renewcommand{\childdocmain}[1]{\renewcommand{\childdocmain}[1]{\endinput}}
  \renewcommand{\childdocof}[1]{}
  \renewcommand{\childdocby}[2][]{}
  \renewcommand{\childdocforward}[2][]{}
  \renewcommand{\childdocdisable}{}
}
%    \end{macrocode}

% \macro{\childdocmain}
% The macro |\childdocmain| is to be called at the top of the main file
% with nothing or the main filename (without extension) as argument.
% First, it breaks loops.
% If the argument is not empty and does not match |\childdocname|
% (which is set by the first inclusion of |childdoc.def|),
% |\ifchilddoc| is set to true, |\includeonly| is applied to the child file
% and |\jobname| is set to the main file
% (for proper handling of |.aux| files):
%    \begin{macrocode}
\newcommand{\childdocmain}[1]
{
  \childdocdisable\childdocmain{}
  \if?#1?\else
    \begingroup
      \def\childdoctmp{#1}
      \ifx\childdoctmp\childdocname
        \def\childdoctmp{}
      \else
        \def\childdoctmp
        {
          \childdoctrue
          \includeonly{\childdocname}
          \def\childdocjob{#1}
          \def\jobname{#1}
        }
      \fi
      \expandafter
    \endgroup
    \childdoctmp
  \fi
}
%    \end{macrocode}

% \macro{\childdocof}
% The command |\childdocof| redirects
% compilation to the main file |#1|.
%    \begin{macrocode}
\newcommand{\childdocof}[1]
{
  \childdocdisable
  \childdoctrue
  \includeonly{\childdocname}
  \def\jobname{#1}
  \def\childdocjob{#1}
  \input{#1}
}
%    \end{macrocode}

% \macro{\childdocby}
% The command |\childdocby| ....
%    \begin{macrocode}
\newcommand{\childdocby}[2][]
{
  \childdocdisable
  \childdoctrue
  \childdocmanualtrue
  \if?#1?\else
    \def\jobname{#2}
  \fi
  \def\childdocjob{#2}
  \input{#2}
  \endinput
}
%    \end{macrocode}

% \macro{\childdocforward}
% The command |\childdocforward| redirects
% compilation to the main file or
% (if the optional argument is given) a child file.
% Parameters are set as if the main file
% or a child file starting with |\childdocof| was compiled.
% Then compilation is handed over to the main file:
%    \begin{macrocode}
\newcommand{\childdocforward}[2][]
{
  \begingroup
    \if?#1?
      \def\childdoctmp
      {
        \def\childdocname{#2}
        \def\childdocjob{#2}
        \def\jobname{#2}
        \input{#2}
        \endinput
      }
    \else
      \def\childdoctmp
      {
        \childdocdisable
        \def\childdocname{#2}
        \childdoctrue
        \includeonly{#2}
        \def\childdocjob{#1}
        \def\jobname{#1}
        \input{#1}
        \endinput
      }
    \fi
    \expandafter
  \endgroup
  \childdoctmp
}
%    \end{macrocode}

% \macro{\childdocforwardprefix}
% The command |\childdocforwardprefix| redirects
% compilation to the main or a child file by means of a pattern.
% The prefix |#1| in the current filename is replaced by |#2|
% and the suffix of the current filename is kept
% (it is assumed that the filename does not contain the substring `|~~~|'
% which is used as a delimiter).
% Compilation is handed over to the new file by |\childdocforward|:
%    \begin{macrocode}
\newcommand{\childdocforwardprefix}[3][]
{
  \begingroup
    \def\childdocextract #2##1~~~{\def\childdoctmp{\childdocforward[#1]{#3##1}}}
    \expandafter\childdocextract\childdocname~~~
    \expandafter
  \endgroup
  \childdoctmp
}
%    \end{macrocode}

% \macro{\childdoc}
% The deprecated macro |\childdoc| is a legacy version of |\childdocmain|:
%    \begin{macrocode}
\newcommand{\childdoc}{\childdocmain}
%    \end{macrocode}

% \macro{\childdocredirect}
% The deprecated macro |\childdocredirect| is a legacy version
% of |\childdocforward| and |\childdocforwardprefix|:
%    \begin{macrocode}
\newcommand{\childdocredirect}[2][]
{
  \begingroup
    \if?#1?
      \def\childdoctmp{\childdocforward{#2}}
    \else
      \def\childdoctmp{\childdocforwardprefix{#1}{#2}}
    \fi
    \expandafter
  \endgroup
  \childdoctmp
}
%    \end{macrocode}

%\iffalse
%</package>
%\fi
%
\endinput
|\\
|\childdocby{|\textit{main}|}|\\
\end{tabular}
\end{center}
%
The directive |\childdocby| is similar to |\childdocof|
described in \secref{sec:include},
but the subsequent selection of content must be done manually.
To that end, both |\ifchilddoc| and |\ifchilddocmanual|
will be true upon processing of a part,
and the name of the part is stored in |\childdocname|.
Note that |\jobname| will be set to the filename of the current part
so that each part receives an individual |.aux| file
that does not interfere with the |.aux| file(s) of the main document.
This behaviour can be altered by the alternative form
|\childdocby[*]{|\textit{main}|}| (with a non-empty optional argument)
which uses the |.aux| file of the main document
by setting |\jobname| to \textit{main}.

%%%%%%%%%%%%%%%%%%%%%%%%%%%%%%%%%%%%%%%%%%%%%%%%%%%%%%%%%%%%%%%%%%%%%%%%%%%%%%%%
\subsection{Driver Development}
\label{sec:driver}

The \textsf{childdoc} mechanism can also be use for the development
of definition files such as \LaTeX{} styles or classes.
This case differs from the above setup with multiple parts
included by |\include| in that no |\includeonly| should be invoked.
This can be achieved by starting the include file
(before |\ProvidesPackage|) with:
%
\begin{center}
\begin{tabular}{l}
|% \iffalse
%
% childdoc.dtx Copyright (C) 2017-2018 Niklas Beisert
%
% This work may be distributed and/or modified under the
% conditions of the LaTeX Project Public License, either version 1.3
% of this license or (at your option) any later version.
% The latest version of this license is in
%   http://www.latex-project.org/lppl.txt
% and version 1.3 or later is part of all distributions of LaTeX
% version 2005/12/01 or later.
%
% This work has the LPPL maintenance status `maintained'.
%
% The Current Maintainer of this work is Niklas Beisert.
%
% This work consists of the files childdoc.dtx and childdoc.ins
% and the derived files childdoc.def and cdocsamp.tex with
% cdocsch1.tex, cdocsch2.tex, cdocsdrf.tex, cdocsfn1.tex, cdocsfn2.tex.
%
%<package>\ifdefined\childdocmain\endinput\fi
%<package>\ProvidesFile{childdoc.def}[2018/12/30 v2.0 child document driver]
%<samplemain>\ProvidesFile{cdocsamp.tex}[2018/12/30 v2.0 sample for childdoc]
%<*driver>
%\ProvidesFile{childdoc.drv}[2018/12/30 v2.0 childdoc reference manual file]
\PassOptionsToClass{10pt,a4paper}{article}
\documentclass{ltxdoc}

\usepackage[margin=35mm]{geometry}
\usepackage{hyperref}
\usepackage{hyperxmp}
\usepackage[usenames]{color}

\hypersetup{colorlinks=true}
\hypersetup{pdfstartview=FitH}
\hypersetup{pdfpagemode=UseNone}
\hypersetup{pdfsource={}}
\hypersetup{pdflang={en-UK}}
\hypersetup{pdfcopyright={Copyright 2017-2018 Niklas Beisert.
  This work may be distributed and/or modified under the
  conditions of the LaTeX Project Public License, either version 1.3
  of this license or (at your option) any later version.}}
\hypersetup{pdflicenseurl={http://www.latex-project.org/lppl.txt}}
\hypersetup{pdfcontactaddress={ETH Zurich, ITP, HIT K,
  Wolfgang-Pauli-Strasse 27}}
\hypersetup{pdfcontactpostcode={8093}}
\hypersetup{pdfcontactcity={Zurich}}
\hypersetup{pdfcontactcountry={Switzerland}}
\hypersetup{pdfcontactemail={nbeisert@itp.phys.ethz.ch}}
\hypersetup{pdfcontacturl={http://people.phys.ethz.ch/\xmptilde nbeisert/}}

\newcommand{\secref}[1]{\hyperref[#1]{section \ref*{#1}}}

\parskip1ex
\parindent0pt
\let\olditemize\itemize
\def\itemize{\olditemize\parskip0pt}

\begin{document}

\title{The \textsf{childdoc} Package}
\hypersetup{pdftitle={The childdoc Package}}
\author{Niklas Beisert\\[2ex]
  Institut f\"ur Theoretische Physik\\
  Eidgen\"ossische Technische Hochschule Z\"urich\\
  Wolfgang-Pauli-Strasse 27, 8093 Z\"urich, Switzerland\\[1ex]
  \href{mailto:nbeisert@itp.phys.ethz.ch}
  {\texttt{nbeisert@itp.phys.ethz.ch}}}
\hypersetup{pdfauthor={Niklas Beisert}}
\hypersetup{pdfsubject={Manual for the LaTeX2e Package childdoc}}
\date{30 December 2018, \textsf{v2.0}}
\maketitle

\begin{abstract}\noindent
\textsf{childdoc} is a \LaTeXe{} package
that enables the direct compilation
of document sections included by |\include|
to individual files.
\end{abstract}

\begingroup
\parskip0ex
\tableofcontents
\endgroup

%%%%%%%%%%%%%%%%%%%%%%%%%%%%%%%%%%%%%%%%%%%%%%%%%%%%%%%%%%%%%%%%%%%%%%%%%%%%%%%%
%%%%%%%%%%%%%%%%%%%%%%%%%%%%%%%%%%%%%%%%%%%%%%%%%%%%%%%%%%%%%%%%%%%%%%%%%%%%%%%%
\section{Introduction}

\LaTeX{} provides a mechanism to structure a large document (such as a book)
into a main file and several child files (containing the chapters)
using the |\include| command.
This mechanism is beneficial for documents
which span hundreds of pages in order to
make the source file(s) more manageable.
Moreover, compilation can be restricted to
selected child files by means of the |\includeonly| command.
The latter feature can be used to reduce the compilation time while editing
(this was significantly more useful in the earlier days of \LaTeX{})
or to generate a smaller document which is easier to navigate.
Another application of |\includeonly| is to generate
documents consisting of selected parts of the complete document.

However, there are a few drawbacks of the plain |\include| mechanism:
\begin{itemize}
\item
The child files cannot be compiled on their own,
they can only be compiled via the main file.
A naive editing environment
(such as a text editor with an option
to have the current file processed by \LaTeX)
may require one to switch to the main file before compiling;
attempting to compile the child file produces errors.
\item
The main file must be modified (each time)
to adjust the |\includeonly| command
to the present needs. This easily leaves the main file in a messy state.
\item
The generated document will always carry the filename
of the main document. This is inconvenient if
several child files are to be compiled and
to be kept for distribution.
\end{itemize}

The present package provides a simple interface
to make child files individually compilable by \LaTeX{}.
Compiling a child file then has the same effect as compiling
the main file with an |\includeonly| command
to select the appropriate child.
Moreover the generated document will carry the name of the child
rather than the main file.
This resolves all three above issues.

This feature is meant to make the editing of books,
thesis documents and lecture notes somewhat more convenient.
However, the package can also be used efficiently for
composing a series of documents (such as exercise sheets)
which are typically distributed individually.
It then assists the author in generating the individual documents
(potentially in different versions)
as well as a document containing the collected series.
Another application is in developing style files
or other kinds of included material
where compilation of the style file could redirect
to a sample or test file.

%%%%%%%%%%%%%%%%%%%%%%%%%%%%%%%%%%%%%%%%%%%%%%%%%%%%%%%%%%%%%%%%%%%%%%%%%%%%%%%%
%%%%%%%%%%%%%%%%%%%%%%%%%%%%%%%%%%%%%%%%%%%%%%%%%%%%%%%%%%%%%%%%%%%%%%%%%%%%%%%%
\section{Usage}

First of all, the package \textsf{childdoc} is \emph{not} a standard
\LaTeXe{} |.sty| style file! Therefore it needs to be invoked in
a non-standard way.

%%%%%%%%%%%%%%%%%%%%%%%%%%%%%%%%%%%%%%%%%%%%%%%%%%%%%%%%%%%%%%%%%%%%%%%%%%%%%%%%
\subsection{Included Files}
\label{sec:include}

%%%%%%%%%%%%%%%%%%%%%%%%%%%%%%%%%%%%%%%%
\DescribeMacro{\childdocmain}
To use the package, add the commands
\begin{center}
\begin{tabular}{l}
|\input{childdoc.def}|\\
|\childdocmain{}|\\
\end{tabular}
\end{center}
at the very top of the main \LaTeX{} file,
in particular \emph{before} the |\documentclass| statement!
The argument of |\childdocmain| should be left empty
(but it must be present).

%%%%%%%%%%%%%%%%%%%%%%%%%%%%%%%%%%%%%%%%
\DescribeMacro{\childdocof}
Furthermore, add the commands
\begin{center}
\begin{tabular}{l}
|\input{childdoc.def}|\\
|\childdocof{|\textit{main}|}|\\
\end{tabular}
\end{center}
at the top of every child file \textit{child}
which is included by |\include{|\textit{child}|}|
from within the main file
(or at least for those files to be compiled individually).
The argument \textit{main} must be the filename of the main file.

There are a couple of
considerations in setting up the main and child documents:

%%%%%%%%%%%%%%%%%%%%%%%%%%%%%%%%%%%%%%%%
\paragraph{Restrictions.}

Please note the following restrictions:
\begin{itemize}
\item
|\childdocmain| must be called with one argument \textit{main}
to ensure compatibility with earlier version of the package.
It must either be empty (|\childdocmain{}|)
or precisely match the filename of the main file in which it is specified.
See \secref{sec:detection} for further information.
\item
The filename \textit{main} must be specified without the |.tex| extension.
\item
The filename \textit{main} is case sensitive
(even in case-insensitive file systems)
due to internal string comparison.
\item
The argument \textit{main} should be fully expanded, it cannot be a macro.
\item
Subdirectories and special characters should be avoided in filenames.
\item
The command |\childdocmain{|\textit{main}|}| must be followed by a whitespace.
It should not be followed immediately by another command
or by a comment mark `|%|'.
This is because the \TeX{} parser reads the token immediately following
the argument of |\childdocmain| and puts it
at the beginning of every child section;
however, a white\-space is ignored.
\end{itemize}

%%%%%%%%%%%%%%%%%%%%%%%%%%%%%%%%%%%%%%%%
\paragraph{Content of Main File.}

It is advisable to place all content in the child files included by |\include|.
Any output contained in the main file will appear in all child documents
unless suppressed manually;
it cannot be suppressed automatically by the |\includeonly| directive
and thus should normally be avoided.
A method to include some content in the main file
by means of conditional processing is described in \secref{sec:conditional}.

%%%%%%%%%%%%%%%%%%%%%%%%%%%%%%%%%%%%%%%%
\paragraph{Page Numbering.}

When only a part of the document is compiled,
the appropriate numbering of pages
(as well as other status parameters)
is determined from the |.aux| files.
The latter contain information from previous passes.
However this information needs to propagate through
all intermediate child documents.
Therefore the page numbering in child documents may well
be inconsistent until the complete document is compiled at least once.

A useful (if unconventional) way to always ensure a consistent
page numbering is to restart the numbering in each child document
and denote the pages by `\textit{child}|.|\textit{page}'
where \textit{child} represents the chapter/section number of the child file.
This can be achieved by the command
|\numberwithin{page}{|\textit{child}|}|
of the \textsf{amsmath} package
where \textit{child} can be |chapter| or |section|
depending on the chosen structuring.
Alternatively, one can modify the macro |\thepage| appropriately
and reset the counter |page| at the start of each child file.

%%%%%%%%%%%%%%%%%%%%%%%%%%%%%%%%%%%%%%%%%%%%%%%%%%%%%%%%%%%%%%%%%%%%%%%%%%%%%%%%
\subsection{Conditional Processing}
\label{sec:conditional}

The package provides a mechanism to compile different versions
of a document. To customise the versions further some conditional processing
can come in handy to distinguish which version is being compiled.
The package provides two macros to describe the compilation context:

%%%%%%%%%%%%%%%%%%%%%%%%%%%%%%%%%%%%%%%%
\DescribeMacro{\ifchilddoc}
The conditional |\ifchilddoc| distinguishes between the compilation of
child documents and the main document:
%
\begin{center}
|\ifchilddoc |\textit{child-code}| |[|\||else |\textit{main-code}]| \||fi|
\end{center}

%%%%%%%%%%%%%%%%%%%%%%%%%%%%%%%%%%%%%%%%
\DescribeMacro{\childdocname}
\DescribeMacro{\childdocjob}
The macro |\childdocname| contains the filename (without extension)
of the main or child file being processed.
Note that |\childdocjob| will always contain the name of the main file.

%%%%%%%%%%%%%%%%%%%%%%%%%%%%%%%%%%%%%%%%
\paragraph{Title Page.}

Conditional processing can be used to include a title or banner page
in the main document when proper precautions are taken.
Importantly, the code in the main file should ensure that the page counter
(as well as other status parameters which are stored in the |.aux| files)
takes the same value after the conditional processing.
Otherwise the page numbers may take divergent values
depending on which part is compiled.

For example, a title page could be declared by:
%
\begin{center}
\begin{tabular}{l}
|\ifchilddoc\||else|\\
|\addtocounter{page}{-1}|\\
\textit{code for title page}\\
|\newpage|\\
|\||fi|
\end{tabular}
\end{center}
%
A banner page for the child documents can be generated by:
%
\begin{center}
\begin{tabular}{l}
|\ifchilddoc|\\
|\addtocounter{page}{-1}|\\
\textit{code for banner page}\\
|\newpage|\\
|\||fi|
\end{tabular}
\end{center}
%
Here one could write a message such as:
\begin{center}
|This is the part \childdocname{} of \childdocjob{}.|
\end{center}

%%%%%%%%%%%%%%%%%%%%%%%%%%%%%%%%%%%%%%%%%%%%%%%%%%%%%%%%%%%%%%%%%%%%%%%%%%%%%%%%
\subsection{Flags}
\label{sec:flags}

The package makes it easy to generate different versions
of the main or child documents.
To this end compilation flags can be defined
and assigned different default values.
They will be particularly useful in conjunction
with the forwarding mechanism described in \secref{sec:forward}.

For example, it may be useful to have a flag |\version|
which can be set to |draft| or |final|.
The document source will contain some conditional code
depending on the value of |\version|.
Suppose further, the flag should default to |final| for the main file
and to |draft| for child files
which is a natural assignment for editing the document.
This is achieved by placing the following code
in the preamble of the main document
(below the |\childdocmain| directive):
%
\begin{center}
\begin{tabular}{l}
|\ifchilddoc|\\
|\providecommand{\version}{draft}|\\
|\||else|\\
|\providecommand{\version}{final}|\\
|\||fi|
\end{tabular}
\end{center}
%
The definition by |\providecommand| makes sure
that previous definitions are not overwritten.
Further statements |\providecommand{\version}{...}|
can thus be added before the above code to override it.

For the main file, one might add a line
(between |\childdocmain| and the above block)
%
\begin{center}
|%\ifchilddoc\||else\providecommand{\version}{draft}\||fi|
\end{center}
%
which can be uncommented to produce a draft version.
Likewise one can add a line to the very top of a child file
(above the |\childdocof{|\textit{main}|}| directive)
%
\begin{center}
|%\providecommand{\version}{final}|
\end{center}
%
which can be uncommented to produce the final version of this child document.

%%%%%%%%%%%%%%%%%%%%%%%%%%%%%%%%%%%%%%%%%%%%%%%%%%%%%%%%%%%%%%%%%%%%%%%%%%%%%%%%
\subsection{Forwarding}
\label{sec:forward}

Different versions of the main or child documents
using compilation flags as described in \secref{sec:flags}
can be (permanently) stored in different files
for convenient compilation, viewing and distribution.
To this end, the package defines a command
to pass on compilation to a different file:

%%%%%%%%%%%%%%%%%%%%%%%%%%%%%%%%%%%%%%%%
\DescribeMacro{\childdocforward}
The command |\childdocforward| redirects processing to
another source file:
%
\begin{center}
\begin{tabular}{l}
|\input{childdoc.def}|\\
|\childdocforward[|\textit{main}|]{|\textit{dest}|}|\\
\end{tabular}
\end{center}
%
The argument \textit{dest} is the destination file
(without extension).
It should be the main file or one of the child files.
Note that further \textsf{childdoc} directives
such as |\childdocof| and |\childdocforward|
in the indicated file will be processed in this form.
The optional argument \textit{main}
passes on directly to the main file \textit{main}
while pretending to compile the child \textit{dest}.
This form behaves as if \textit{dest}
issues |\childdocof{|\textit{main}|}| right away,
and no further \textsf{childdoc} directives will be processed.

%%%%%%%%%%%%%%%%%%%%%%%%%%%%%%%%%%%%%%%%
\DescribeMacro{\...prefix}
In the alternative form |\childdocforwardprefix|,
%
\begin{center}
\begin{tabular}{l}
|\input{childdoc.def}|\\
|\childdocforwardprefix[|\textit{main}|]{|\textit{prefix}|}{|\textit{dest}|}|
\end{tabular}
\end{center}
%
the destination file is determined by a pattern
depending on the current file:
To make this work, the current file must be called
`{\textit{prefix}\hspace{0.2em}\textit{suffix}}'
with \textit{prefix} matching precisely the argument.
Processing is then passed on to the file
`{\textit{dest}\hspace{0.2em}\textit{suffix}}'.
Surely, the same effect is achieved by
directly specifying the
argument `{\textit{dest}\hspace{0.2em}\textit{suffix}}'
in the first form.
However, that requires to set up a different file
for each child. With the alternative form of the command
all these files can have exactly the same content
which simplifies setting them up and maintaining them.

For example, the following file |draft.tex|
with a compilation flag |\version| as described in \secref{sec:flags}
compiles the main document as a draft:
%
\begin{center}
\begin{tabular}{l}
|\def\version{draft}|\\
|\input{childdoc.def}|\\
|\childdocforward{|\textit{main}|}|
\end{tabular}
\end{center}
%
Likewise, the following files |final|\textit{nn}|.tex|
compile the final version of the child document
|child|\textit{nn}|.tex|:
%
\begin{center}
\begin{tabular}{l}
|\def\version{final}|\\
|\input{childdoc.def}|\\
|\childdocforwardprefix{final}{child}|
\end{tabular}
\end{center}
%

Note that when several versions of a main file and/or of each child file
are to be generated, it may be convenient to set up a |Makefile| or
shell script to automatise the process.

%%%%%%%%%%%%%%%%%%%%%%%%%%%%%%%%%%%%%%%%%%%%%%%%%%%%%%%%%%%%%%%%%%%%%%%%%%%%%%%%
\subsection{Command Line Processing}
\label{sec:commandline}

The effect of redirection files can also be achieved by invoking
the \LaTeX{} compiler with a more elaborate command line.
Most conveniently this should be done as part
of a shell script or a |Makefile|.

When using \textsf{childdoc} in the main file, the following
command lines effectively perform a redirection
(note that depending on the shell being used,
backslashes may have to be doubled: `|\|' $\to$ `|\\|'):
%
\begin{center}
|... -jobname "|\textit{target}|" |\\|"|[\textit{flags}]%
|\input{childdoc.def}\childdocforward[|\textit{main}|]{|\textit{dest}|}"|
\end{center}
%
Here \textit{target} is the name of the output file,
\textit{main} is the name of the main file
and \textit{dest} is the name of the main or child file to be processed
(all filenames without extensions).
The optional argument \textit{main} can be omitted
if \textit{main} matches \textit{dest}.
Optionally, compilation \textit{flags} can be defined via |\def| commands.
This command line makes the \TeX{} engine believe
it is compiling the file \textit{target}
whose content is specified as the latter parameter.
The provided code then forwards the processing to
\textit{main} or \textit{dest} as described in \secref{sec:forward}.

%%%%%%%%%%%%%%%%%%%%%%%%%%%%%%%%%%%%%%%%%%%%%%%%%%%%%%%%%%%%%%%%%%%%%%%%%%%%%%%%
\subsection{Include by Input}
\label{sec:input}

Including child documents by |\include| has some restrictions by design.
Most notably, the content of a child document always occupies
its own set of pages; pages cannot be shared between child documents.
Usually, this behaviour makes perfect sense
because each child document contain an essential part of the document.
However, in some situations it may be desirable to compose
a document from a collection of parts
without having mandatory page breaks between then.
For this case, the package
provides a mechanism to include parts
by |\input| which can also be processed individually.
However, by construction this mechanism
requires manual handling of the content to be output.

%%%%%%%%%%%%%%%%%%%%%%%%%%%%%%%%%%%%%%%%
\DescribeMacro{\ifchilddocmanual}
The main file should be prepared as usual, see \secref{sec:include}.
However, the document body must make a distinction
between processing of an individual part and of the main document, e.g.:
%
\begin{center}
\begin{tabular}{l}
|\ifchilddocmanual|\\
|\input{\childdocname}|\\
|\||else|\\
\textit{document body with }|\input{|\textit{part}|}|\\
|\||fi|
\end{tabular}
\end{center}
%
The conditional |\ifchilddocmanual| is true whenever
a part to be included by |\input| is being compiled,
and the name of the part is stored in |\childdocname|.

%%%%%%%%%%%%%%%%%%%%%%%%%%%%%%%%%%%%%%%%
\DescribeMacro{\childdocby}
Each part to be included by |\input| should start with:
%
\begin{center}
\begin{tabular}{l}
|\input{childdoc.def}|\\
|\childdocby{|\textit{main}|}|\\
\end{tabular}
\end{center}
%
The directive |\childdocby| is similar to |\childdocof|
described in \secref{sec:include},
but the subsequent selection of content must be done manually.
To that end, both |\ifchilddoc| and |\ifchilddocmanual|
will be true upon processing of a part,
and the name of the part is stored in |\childdocname|.
Note that |\jobname| will be set to the filename of the current part
so that each part receives an individual |.aux| file
that does not interfere with the |.aux| file(s) of the main document.
This behaviour can be altered by the alternative form
|\childdocby[*]{|\textit{main}|}| (with a non-empty optional argument)
which uses the |.aux| file of the main document
by setting |\jobname| to \textit{main}.

%%%%%%%%%%%%%%%%%%%%%%%%%%%%%%%%%%%%%%%%%%%%%%%%%%%%%%%%%%%%%%%%%%%%%%%%%%%%%%%%
\subsection{Driver Development}
\label{sec:driver}

The \textsf{childdoc} mechanism can also be use for the development
of definition files such as \LaTeX{} styles or classes.
This case differs from the above setup with multiple parts
included by |\include| in that no |\includeonly| should be invoked.
This can be achieved by starting the include file
(before |\ProvidesPackage|) with:
%
\begin{center}
\begin{tabular}{l}
|\input{childdoc.def}|\\
|\childdocforward{|\textit{main}|}|\\
\end{tabular}
\end{center}
%
or alternatively with:
%
\begin{center}
\begin{tabular}{l}
|\input{childdoc.def}|\\
|\childdocby{|\textit{main}|}|\\
\end{tabular}
\end{center}
%
Both forms have slightly different effects as described above.
The main file is prepared as usual, see \secref{sec:include}.

%%%%%%%%%%%%%%%%%%%%%%%%%%%%%%%%%%%%%%%%%%%%%%%%%%%%%%%%%%%%%%%%%%%%%%%%%%%%%%%%
\subsection{Legacy Detection}
\label{sec:detection}

The directive |\childdocmain| in the main file can detect
whether the complete document or merely a child is to be compiled
even without using the directive |\childdocof|.
This method is deprecated because it is less robust
and there is no compelling reason to use it;
it is merely provided for backward compatibility
and it may be removed in future versions.

If the detection mechanism is to be used,
it is mandatory to correctly specify
the filename of the main file as the argument of |\childdocmain|:
%
\begin{center}
\begin{tabular}{l}
|\input{childdoc.def}|\\
|\childdocmain{|\textit{main}|}|\\
\end{tabular}
\end{center}
%
If |\jobname| does not match the argument \textit{main} of |\childdocmain|,
it is assumed that |\jobname| points to the child file to be compiled.
When using |\childdocmain| with the main file specified as argument,
it suffices to start a child file
with just |\input{|\textit{main}|}|
without loading of the package and using |\childdocof|.
If instead all processing is done
with the appropriate \textsf{childdoc} directives,
the argument of \textit{main} of |\childdocmain| can be empty.

An alternative version of the command line processing described
in \secref{sec:commandline} using the detection mechanism reads:
%
\begin{center}
|... -jobname "|\textit{target}|" "|[\textit{flags}]%
[|\def\jobname{|\textit{dest}|}|]|\input{|\textit{main}|}"|
\end{center}

%%%%%%%%%%%%%%%%%%%%%%%%%%%%%%%%%%%%%%%%%%%%%%%%%%%%%%%%%%%%%%%%%%%%%%%%%%%%%%%%
\subsection{Manual Code}
\label{sec:manual}

In case one cannot be certain whether the definitions file |childdoc.def|
is installed on the target \TeX{} distribution
and one prefers not to ship it,
it is conceivable to paste a few relevant commands into the sources.

To that end, drop all statements |\input{childdoc.def}|
and perform the replacements as outlined below.
Instead of |\childdocmain{|\textit{main}|}| add the following code
to the top of the main file:
%
\begin{center}
\begin{tabular}{l}
|\||ifdefined\childdocname\endinput\||fi\newif\ifchilddoc|\\
|\edef\childdocname{\scantokens\expandafter{\jobname\noexpand}}|\\
|\def\childdocmain{|\textit{main}|}\||ifx\childdocmain\childdocname\||else|\\
|\childdoctrue\includeonly{\childdocname}\let\jobname\childdocmain\||fi|\\
\end{tabular}
\end{center}
%
Instead of |\childdocof{|\textit{main}|}| just include the main file
at the top of each child file:
%
\begin{center}
|\input{|\textit{main}|}|
\end{center}
%
A simple redirection |\childdocforward{|\textit{dest}|}| is achieved by:
%
\begin{center}
|\def\jobname{|\textit{dest}|}\input{\jobname}|
\end{center}
%
The redirection with prefix
|\childdocforwardprefix[|\textit{prefix}|]{|\textit{dest}|}|
is accomplished by:
%
\begin{center}
\begin{tabular}{l}
|{\edef\jobname{\scantokens\expandafter{\jobname\noexpand}}|\\
|\def\redirectjob |\textit{prefix}|#1~~~{\gdef\jobname{|\textit{dest}|#1}}|\\
|\expandafter\redirectjob\jobname~~~}\input{\jobname}|
\end{tabular}
\end{center}

In an alternative approach,
child documents can be compiled by a specific command line
without additional code or specific definitions:
%
\begin{center}
|... -jobname "|\textit{target}|" "|[\textit{flags}]%
|\includeonly{|\textit{dest}|}\input{|\textit{main}|}"|
\end{center}
%

%%%%%%%%%%%%%%%%%%%%%%%%%%%%%%%%%%%%%%%%%%%%%%%%%%%%%%%%%%%%%%%%%%%%%%%%%%%%%%%%
%%%%%%%%%%%%%%%%%%%%%%%%%%%%%%%%%%%%%%%%%%%%%%%%%%%%%%%%%%%%%%%%%%%%%%%%%%%%%%%%
\section{Information}

%%%%%%%%%%%%%%%%%%%%%%%%%%%%%%%%%%%%%%%%%%%%%%%%%%%%%%%%%%%%%%%%%%%%%%%%%%%%%%%%
\subsection{Copyright}

Copyright \copyright{} 2017--2018 Niklas Beisert

This work may be distributed and/or modified under the
conditions of the \LaTeX{} Project Public License, either version 1.3
of this license or (at your option) any later version.
The latest version of this license is in
  \url{http://www.latex-project.org/lppl.txt}
and version 1.3 or later is part of all distributions of \LaTeX{}
version 2005/12/01 or later.

This work has the LPPL maintenance status `maintained'.

The Current Maintainer of this work is Niklas Beisert.

This work consists of the files |README.txt|, |childdoc.ins| and |childdoc.dtx|
as well as the derived files |childdoc.def|, |cdocsamp.tex|
with |cdocsch1.tex|, |cdocsch2.tex|, |cdocspt3.tex|, |cdocspt4.tex|,
|cdocsdrf.tex|, |cdocsfn1.tex|, |cdocsfn2.tex|
as well as |childdoc.pdf|.

%%%%%%%%%%%%%%%%%%%%%%%%%%%%%%%%%%%%%%%%%%%%%%%%%%%%%%%%%%%%%%%%%%%%%%%%%%%%%%%%
\subsection{Files and Installation}

The package consists of the files:
%
\begin{center}
\begin{tabular}{ll}
    |README.txt|   & readme file \\
    |childdoc.ins| & installation file \\
    |childdoc.dtx| & source file \\
    |childdoc.def| & definition file \\
    |cdocsamp.tex| & sample main file \\
    |cdocsch1.tex| & sample include file \\
    |cdocsch2.tex| & sample include file \\
    |cdocspt3.tex| & sample part file \\
    |cdocspt4.tex| & sample part file \\
    |cdocsdrf.tex| & sample redirection file \\
    |cdocsfn1.tex| & sample redirection file \\
    |cdocsfn2.tex| & sample redirection file \\
    |childdoc.pdf| & manual
\end{tabular}
\end{center}
%
The distribution consists of the files
|README.txt|, |childdoc.ins| and |childdoc.dtx|.
%
\begin{itemize}
\item
Run (pdf)\LaTeX{} on |childdoc.dtx|
to compile the manual |childdoc.pdf| (this file).
\item
Run \LaTeX{} on |childdoc.ins| to create the definitions file |childdoc.def|
and the sample |cdocsamp.tex| with include files
|cdocsch1.tex|, |cdocsch2.tex|, |cdocspt3.tex|, |cdocspt4.tex|,
|cdocsdrf.tex|, |cdocsfn1.tex|, |cdocsfn2.tex|.
Then copy the file |childdoc.def| to an appropriate directory of your \LaTeX{}
distribution, e.g.\ \textit{texmf-root}|/tex/latex/childdoc|.
\end{itemize}

%%%%%%%%%%%%%%%%%%%%%%%%%%%%%%%%%%%%%%%%%%%%%%%%%%%%%%%%%%%%%%%%%%%%%%%%%%%%%%%%
\subsection{Related CTAN Packages}

There are several other packages which offer a similar functionality:
%
\begin{itemize}
\item
The packages
\href{http://ctan.org/pkg/docmute}{\textsf{docmute}},
\href{http://ctan.org/pkg/includex}{\textsf{includex}} and
\href{http://ctan.org/pkg/standalone}{\textsf{standalone}}
provide commands to include only the document body of
a child file thus allowing both files to be compiled individually.
\item
The packages \href{http://ctan.org/pkg/subdocs}{\textsf{subdocs}}
and \href{http://ctan.org/pkg/subfiles}{\textsf{subfiles}}
provide structures in which the main and child documents can be
encapsulated and allowing them to be compiled individually.
The inclusion mechanism is different from the conventional |\include|.
\item
The package \href{http://ctan.org/pkg/combine}{\textsf{combine}}
is an elaborate solution to combine several documents into one.
\end{itemize}
%
See also the CTAN topic \href{http://ctan.org/topic/subdocs}{\textsf{subdocs}}
for further related packages.
The present package differs from the above solutions in that
a document structure constructed with the conventional |\include| mechanism
just needs two extra commands at the top of every file
such that all constituent files can be compiled individually.

%%%%%%%%%%%%%%%%%%%%%%%%%%%%%%%%%%%%%%%%%%%%%%%%%%%%%%%%%%%%%%%%%%%%%%%%%%%%%%%%
%\subsection{Feature Suggestions}
%
%The following is a list of features which may be useful for future
%versions of this package:
%%
%\begin{itemize}
%\item
%\ldots
%\end{itemize}

%%%%%%%%%%%%%%%%%%%%%%%%%%%%%%%%%%%%%%%%%%%%%%%%%%%%%%%%%%%%%%%%%%%%%%%%%%%%%%%%
\subsection{Revision History}

%%%%%%%%%%%%%%%%%%%%%%%%%%%%%%%%%%%%%%%%
\paragraph{v2.0:} 2018/12/30

\begin{itemize}
\item
immediate forward processing
\item
added |\childdocby| mechanism
\item
manual restructured
\end{itemize}

%%%%%%%%%%%%%%%%%%%%%%%%%%%%%%%%%%%%%%%%
\paragraph{v1.6:} 2018/01/17

\begin{itemize}
\item
application for development of include files
\item
corrections to manual
\end{itemize}

%%%%%%%%%%%%%%%%%%%%%%%%%%%%%%%%%%%%%%%%
\paragraph{v1.5:} 2017/05/21

\begin{itemize}
\item
more complete structuring introduced
\item
|\childdocof| introduced
\item
|\childdoc| renamed to |\childdocmain|
\item
|\childredirect| renamed to |\childdocforward| and |\childdocforwardprefix|
and functionality expanded
\end{itemize}

%%%%%%%%%%%%%%%%%%%%%%%%%%%%%%%%%%%%%%%%
\paragraph{v1.0:} 2017/04/27

\begin{itemize}
\item
manual and install package
\item
first version published on CTAN
\end{itemize}

%%%%%%%%%%%%%%%%%%%%%%%%%%%%%%%%%%%%%%%%
\paragraph{v0.6:} 2017/04/26

\begin{itemize}
\item
redirection mechanism added
\end{itemize}

%%%%%%%%%%%%%%%%%%%%%%%%%%%%%%%%%%%%%%%%
\paragraph{v0.5:} 2017/04/26

\begin{itemize}
\item
functionality in definition file
\end{itemize}


%%%%%%%%%%%%%%%%%%%%%%%%%%%%%%%%%%%%%%%%%%%%%%%%%%%%%%%%%%%%%%%%%%%%%%%%%%%%%%%%
%%%%%%%%%%%%%%%%%%%%%%%%%%%%%%%%%%%%%%%%%%%%%%%%%%%%%%%%%%%%%%%%%%%%%%%%%%%%%%%%
%%%%%%%%%%%%%%%%%%%%%%%%%%%%%%%%%%%%%%%%%%%%%%%%%%%%%%%%%%%%%%%%%%%%%%%%%%%%%%%%
\appendix

\settowidth\MacroIndent{\rmfamily\scriptsize 000\ }

 \DocInput{childdoc.dtx}

\end{document}
%</driver>
% \fi
%
% %%%%%%%%%%%%%%%%%%%%%%%%%%%%%%%%%%%%%%%%%%%%%%%%%%%%%%%%%%%%%%%%%%%%%%%%%%%%%%
% %%%%%%%%%%%%%%%%%%%%%%%%%%%%%%%%%%%%%%%%%%%%%%%%%%%%%%%%%%%%%%%%%%%%%%%%%%%%%%
% \section{Sample}
%\iffalse
%<*samplemain>
%\fi
%
% The following presents a sample document
% with two chapters, two parts, a title page,
% a compile flag as well as three forwarding files to set the flag.
% It consists of eight |.tex| files:
% \begin{center}
% \begin{tabular}{ll}
% |cdocsamp.tex|&main file\\
% |cdocsch1.tex|&include file for chapter 1\\
% |cdocsch2.tex|&include file for chapter 2\\
% |cdocspt3.tex|&include file for part 3\\
% |cdocspt4.tex|&include file for part 4\\
% |cdocsdrf.tex|&forwarding file for main file in draft mode\\
% |cdocsfi1.tex|&forwarding file for final version of chapter 1\\
% |cdocsfi2.tex|&forwarding file for final version of chapter 2\\
% \end{tabular}
% \end{center}
% Each of the eight files can be compiled directly by the \LaTeX{} compiler.
%
% %%%%%%%%%%%%%%%%%%%%%%%%%%%%%%%%%%%%%%
% \paragraph{Main File.}
%
% The main file is called |cdocsamp.tex|.
%
% Load the \textsf{childdoc} definitions and
% declare the filename for the main document:
%    \begin{macrocode}
\input{childdoc.def}
\childdocmain{}
%    \end{macrocode}

% Optional override for |\version| flag:
%    \begin{macrocode}
%%\ifchilddoc\else\providecommand{\version}{draft}\fi
%    \end{macrocode}

% Define the default values for the |\version| flag
% (|final| for the main file and |draft| for childs):
%    \begin{macrocode}
\ifchilddoc
\providecommand{\version}{draft}
\else
\providecommand{\version}{final}
\fi
%    \end{macrocode}

% Load the standard document class:
%    \begin{macrocode}
\documentclass[12pt]{article}
%    \end{macrocode}

% Start the document body:
%    \begin{macrocode}
\begin{document}
%    \end{macrocode}

% Declare a title page.
% Print title, part of document being processed and version flag:
%    \begin{macrocode}
\addtocounter{page}{-1}
\begin{center}
{\LARGE\bfseries{}childdoc example\par}
\vspace{1cm}
\ifchilddoc
\ifchilddocmanual part\else chapter\fi:
`\childdocname' of `\childdocjob'\par
\else
main document: `\childdocjob'\par
\fi
version: \version\par
\end{center}
\newpage
%    \end{macrocode}

% Manually include selected file,
% otherwise process as usual:
%    \begin{macrocode}
\ifchilddocmanual
\section*{part `\childdocname'}
\input{\childdocname}
\else
%    \end{macrocode}

% Include the two chapters:
%    \begin{macrocode}
\include{cdocsch1}
\include{cdocsch2}
%    \end{macrocode}

% Include the two parts unless only chapters should be displayed:
%    \begin{macrocode}
\ifchilddoc\else
\section{part three}
\input{cdocspt3}
\section{part four}
\input{cdocspt4}
\fi
%    \end{macrocode}

% Process as usual until here:
%    \begin{macrocode}
\fi
%    \end{macrocode}

% End of document body:
%    \begin{macrocode}
\end{document}
%    \end{macrocode}
%\iffalse
%</samplemain>
%\fi
%
% %%%%%%%%%%%%%%%%%%%%%%%%%%%%%%%%%%%%%%
% \paragraph{Chapter Include Files.}
%
% The include files are called |cdocsch1.tex| and |cdocsch2.tex|.
%
%\iffalse
%<*samplechap1|samplechap2>
%\fi

% Optional override for |\version| flag:
%    \begin{macrocode}
%%\providecommand{\version}{final}
%    \end{macrocode}

% Include the main document:
%    \begin{macrocode}
\input{childdoc.def}
\childdocof{cdocsamp}
%    \end{macrocode}

%\iffalse
%</samplechap1|samplechap2>
%\fi
%
%\iffalse
%<*samplechap1>
%\fi
% Some text for chapter 1:
%    \begin{macrocode}
\section{one}
some text in chapter one
%    \end{macrocode}

%\iffalse
%</samplechap1>
%\fi
% Some text for chapter 2:
%\iffalse
%<*samplechap2>
%\fi
%    \begin{macrocode}
\section{two}
more text in chapter two
%    \end{macrocode}

%\iffalse
%</samplechap2>
%\fi
%
% %%%%%%%%%%%%%%%%%%%%%%%%%%%%%%%%%%%%%%
% \paragraph{Part Include Files.}
%
% The include files are called |cdocspt3.tex| and |cdocspt4.tex|.
%
%\iffalse
%<*samplepart3|samplepart4>
%\fi

% Optional override for |\version| flag:
%    \begin{macrocode}
%%\providecommand{\version}{final}
%    \end{macrocode}

% Include the main document:
%    \begin{macrocode}
\input{childdoc.def}
\childdocby{cdocsamp}
%    \end{macrocode}

%\iffalse
%</samplepart3|samplepart4>
%\fi
%
%\iffalse
%<*samplepart3>
%\fi
% Some text for part 3:
%    \begin{macrocode}
some text in part three
%    \end{macrocode}

%\iffalse
%</samplepart3>
%\fi
% Some text for part 4:
%\iffalse
%<*samplepart4>
%\fi
%    \begin{macrocode}
more text in part four
%    \end{macrocode}

%\iffalse
%</samplepart4>
%\fi
%
% %%%%%%%%%%%%%%%%%%%%%%%%%%%%%%%%%%%%%%
% \paragraph{Forwarding for a Complete Draft.}
%
% The following forwarding file |cdocsdrf.tex|
% compiles the main document in draft mode:
%\iffalse
%<*sampledraft>
%\fi
%    \begin{macrocode}
\def\version{draft}
\input{childdoc.def}
\childdocforward{cdocsamp}
%    \end{macrocode}

%\iffalse
%</sampledraft>
%\fi
%
% %%%%%%%%%%%%%%%%%%%%%%%%%%%%%%%%%%%%%%
% \paragraph{Forwarding for Final Version of the Chapters.}
%
% The following forwarding files |cdocsfn1.tex| and |cdocsfn2.tex|
% (with identical content)
% compile the final versions of the child documents
% |cdocsch1.tex| and |cdocsch2.tex|, respectively:
%\iffalse
%<*samplefinal>
%\fi
%    \begin{macrocode}
\def\version{final}
\input{childdoc.def}
\childdocforwardprefix[cdocsamp]{cdocsfn}{cdocsch}
%    \end{macrocode}

%\iffalse
%</samplefinal>
%\fi
%
% %%%%%%%%%%%%%%%%%%%%%%%%%%%%%%%%%%%%%%
% \paragraph{Command Line Processing.}
%
% The following three command lines generate the output files
% |cdocscld|, |cdocscl1| and |cdocscl2|
% which should be identical to
% |cdocsdrf|, |cdocsch1| and |cdocsfn2|, respectively:
% \begin{center}
% \begin{tabular}{l}
% |latex -jobname cdocscld \|\\
% |  "\def\version{draft}\input{childdoc.def}\childdocforward{cdocsamp}"|\\
% |latex -jobname cdocscl1 \|\\
% |  "\input{childdoc.def}\childdocforward[cdocsamp]{cdocsch1}"|\\
% |latex -jobname cdocscl2 \|\\
% |  "\def\version{final}\input{childdoc.def}\childdocforward{cdocsch2}"|
% \end{tabular}
% \end{center}
% Note that the trailing backslash on each first line
% merely continues the input to the second line
% (for convenient cut ant paste).
% Furthermore, the command |latex| can be replaced by any
% of its alternative versions such as |pdflatex|.
%
% %%%%%%%%%%%%%%%%%%%%%%%%%%%%%%%%%%%%%%%%%%%%%%%%%%%%%%%%%%%%%%%%%%%%%%%%%%%%%%
% %%%%%%%%%%%%%%%%%%%%%%%%%%%%%%%%%%%%%%%%%%%%%%%%%%%%%%%%%%%%%%%%%%%%%%%%%%%%%%
% \section{Implementation}
%\iffalse
%<*package>
%\fi
%
% This section describes the definitions file |childdoc.def|.

% The definitions cannot be loaded using |\usepackage| or |\RequirePackage|
% which has a mechanism to prevent loading a style file more than once.
% When loading the definitions by means of |\input|
% multiple instances have to be prevented manually:
%\iffalse
%This code needs to be before the `\ProvidesFile' directive
%which is defined at the beginning of this file.
%Therefore it is also placed there and commented out here.
%</package>
%<*discard>
%\fi
%    \begin{macrocode}
\ifdefined\childdocmain\endinput\fi
%    \end{macrocode}
%\iffalse
%</discard>
%<*package>
%\fi
%
% \macro{\ifchilddoc}
% \macro{\ifchilddocmanual}
% The conditional |\ifchilddoc| tells whether a
% child (true) or main (false) document is being compiled.
% The conditional |\ifchilddocmanual| tells whether
% the |\includeonly| mechanism is used (false) or
% the selection of child files must be performed manually (true).
% The definitions initialise to false:
%    \begin{macrocode}
\newif\ifchilddoc
\newif\ifchilddocmanual
%    \end{macrocode}

% \macro{\childdocname}
% \macro{\childdocjob}
% The macro |\childdocname| stores the name of the main document
% to be compiled. The macro |\childdocjob| stores the name of
% the document on which the \LaTeX{} compiler was originally invoked.
% The content of |\jobname| cannot be compared
% to filenames specified in the source due to different catcodes.
% The following code rescans |\jobname|, stores the result
% in |\childdocname| and saves a copy in |\childdocjob|:
%    \begin{macrocode}
\edef\childdocname{\scantokens\expandafter{\jobname\noexpand}}
\let\childdocjob\childdocname
%    \end{macrocode}

% \macro{\childdocdisable}
% The macro |\childdocdisable| prevents the main file
% from being processed more than once.
% At this stage, the main document command |\childdocmain|
% is assumed to be called once again where it should do nothing.
% Any subsequent call to it should prevent
% a secondary processing of the main document
% It overwrites the forwarding commands
% |\childdocof| and |\childdocforward|
% with empty macros to prevent further inclusions of the main document:
%    \begin{macrocode}
\newcommand{\childdocdisable}
{
  \renewcommand{\childdocmain}[1]{\renewcommand{\childdocmain}[1]{\endinput}}
  \renewcommand{\childdocof}[1]{}
  \renewcommand{\childdocby}[2][]{}
  \renewcommand{\childdocforward}[2][]{}
  \renewcommand{\childdocdisable}{}
}
%    \end{macrocode}

% \macro{\childdocmain}
% The macro |\childdocmain| is to be called at the top of the main file
% with nothing or the main filename (without extension) as argument.
% First, it breaks loops.
% If the argument is not empty and does not match |\childdocname|
% (which is set by the first inclusion of |childdoc.def|),
% |\ifchilddoc| is set to true, |\includeonly| is applied to the child file
% and |\jobname| is set to the main file
% (for proper handling of |.aux| files):
%    \begin{macrocode}
\newcommand{\childdocmain}[1]
{
  \childdocdisable\childdocmain{}
  \if?#1?\else
    \begingroup
      \def\childdoctmp{#1}
      \ifx\childdoctmp\childdocname
        \def\childdoctmp{}
      \else
        \def\childdoctmp
        {
          \childdoctrue
          \includeonly{\childdocname}
          \def\childdocjob{#1}
          \def\jobname{#1}
        }
      \fi
      \expandafter
    \endgroup
    \childdoctmp
  \fi
}
%    \end{macrocode}

% \macro{\childdocof}
% The command |\childdocof| redirects
% compilation to the main file |#1|.
%    \begin{macrocode}
\newcommand{\childdocof}[1]
{
  \childdocdisable
  \childdoctrue
  \includeonly{\childdocname}
  \def\jobname{#1}
  \def\childdocjob{#1}
  \input{#1}
}
%    \end{macrocode}

% \macro{\childdocby}
% The command |\childdocby| ....
%    \begin{macrocode}
\newcommand{\childdocby}[2][]
{
  \childdocdisable
  \childdoctrue
  \childdocmanualtrue
  \if?#1?\else
    \def\jobname{#2}
  \fi
  \def\childdocjob{#2}
  \input{#2}
  \endinput
}
%    \end{macrocode}

% \macro{\childdocforward}
% The command |\childdocforward| redirects
% compilation to the main file or
% (if the optional argument is given) a child file.
% Parameters are set as if the main file
% or a child file starting with |\childdocof| was compiled.
% Then compilation is handed over to the main file:
%    \begin{macrocode}
\newcommand{\childdocforward}[2][]
{
  \begingroup
    \if?#1?
      \def\childdoctmp
      {
        \def\childdocname{#2}
        \def\childdocjob{#2}
        \def\jobname{#2}
        \input{#2}
        \endinput
      }
    \else
      \def\childdoctmp
      {
        \childdocdisable
        \def\childdocname{#2}
        \childdoctrue
        \includeonly{#2}
        \def\childdocjob{#1}
        \def\jobname{#1}
        \input{#1}
        \endinput
      }
    \fi
    \expandafter
  \endgroup
  \childdoctmp
}
%    \end{macrocode}

% \macro{\childdocforwardprefix}
% The command |\childdocforwardprefix| redirects
% compilation to the main or a child file by means of a pattern.
% The prefix |#1| in the current filename is replaced by |#2|
% and the suffix of the current filename is kept
% (it is assumed that the filename does not contain the substring `|~~~|'
% which is used as a delimiter).
% Compilation is handed over to the new file by |\childdocforward|:
%    \begin{macrocode}
\newcommand{\childdocforwardprefix}[3][]
{
  \begingroup
    \def\childdocextract #2##1~~~{\def\childdoctmp{\childdocforward[#1]{#3##1}}}
    \expandafter\childdocextract\childdocname~~~
    \expandafter
  \endgroup
  \childdoctmp
}
%    \end{macrocode}

% \macro{\childdoc}
% The deprecated macro |\childdoc| is a legacy version of |\childdocmain|:
%    \begin{macrocode}
\newcommand{\childdoc}{\childdocmain}
%    \end{macrocode}

% \macro{\childdocredirect}
% The deprecated macro |\childdocredirect| is a legacy version
% of |\childdocforward| and |\childdocforwardprefix|:
%    \begin{macrocode}
\newcommand{\childdocredirect}[2][]
{
  \begingroup
    \if?#1?
      \def\childdoctmp{\childdocforward{#2}}
    \else
      \def\childdoctmp{\childdocforwardprefix{#1}{#2}}
    \fi
    \expandafter
  \endgroup
  \childdoctmp
}
%    \end{macrocode}

%\iffalse
%</package>
%\fi
%
\endinput
|\\
|\childdocforward{|\textit{main}|}|\\
\end{tabular}
\end{center}
%
or alternatively with:
%
\begin{center}
\begin{tabular}{l}
|% \iffalse
%
% childdoc.dtx Copyright (C) 2017-2018 Niklas Beisert
%
% This work may be distributed and/or modified under the
% conditions of the LaTeX Project Public License, either version 1.3
% of this license or (at your option) any later version.
% The latest version of this license is in
%   http://www.latex-project.org/lppl.txt
% and version 1.3 or later is part of all distributions of LaTeX
% version 2005/12/01 or later.
%
% This work has the LPPL maintenance status `maintained'.
%
% The Current Maintainer of this work is Niklas Beisert.
%
% This work consists of the files childdoc.dtx and childdoc.ins
% and the derived files childdoc.def and cdocsamp.tex with
% cdocsch1.tex, cdocsch2.tex, cdocsdrf.tex, cdocsfn1.tex, cdocsfn2.tex.
%
%<package>\ifdefined\childdocmain\endinput\fi
%<package>\ProvidesFile{childdoc.def}[2018/12/30 v2.0 child document driver]
%<samplemain>\ProvidesFile{cdocsamp.tex}[2018/12/30 v2.0 sample for childdoc]
%<*driver>
%\ProvidesFile{childdoc.drv}[2018/12/30 v2.0 childdoc reference manual file]
\PassOptionsToClass{10pt,a4paper}{article}
\documentclass{ltxdoc}

\usepackage[margin=35mm]{geometry}
\usepackage{hyperref}
\usepackage{hyperxmp}
\usepackage[usenames]{color}

\hypersetup{colorlinks=true}
\hypersetup{pdfstartview=FitH}
\hypersetup{pdfpagemode=UseNone}
\hypersetup{pdfsource={}}
\hypersetup{pdflang={en-UK}}
\hypersetup{pdfcopyright={Copyright 2017-2018 Niklas Beisert.
  This work may be distributed and/or modified under the
  conditions of the LaTeX Project Public License, either version 1.3
  of this license or (at your option) any later version.}}
\hypersetup{pdflicenseurl={http://www.latex-project.org/lppl.txt}}
\hypersetup{pdfcontactaddress={ETH Zurich, ITP, HIT K,
  Wolfgang-Pauli-Strasse 27}}
\hypersetup{pdfcontactpostcode={8093}}
\hypersetup{pdfcontactcity={Zurich}}
\hypersetup{pdfcontactcountry={Switzerland}}
\hypersetup{pdfcontactemail={nbeisert@itp.phys.ethz.ch}}
\hypersetup{pdfcontacturl={http://people.phys.ethz.ch/\xmptilde nbeisert/}}

\newcommand{\secref}[1]{\hyperref[#1]{section \ref*{#1}}}

\parskip1ex
\parindent0pt
\let\olditemize\itemize
\def\itemize{\olditemize\parskip0pt}

\begin{document}

\title{The \textsf{childdoc} Package}
\hypersetup{pdftitle={The childdoc Package}}
\author{Niklas Beisert\\[2ex]
  Institut f\"ur Theoretische Physik\\
  Eidgen\"ossische Technische Hochschule Z\"urich\\
  Wolfgang-Pauli-Strasse 27, 8093 Z\"urich, Switzerland\\[1ex]
  \href{mailto:nbeisert@itp.phys.ethz.ch}
  {\texttt{nbeisert@itp.phys.ethz.ch}}}
\hypersetup{pdfauthor={Niklas Beisert}}
\hypersetup{pdfsubject={Manual for the LaTeX2e Package childdoc}}
\date{30 December 2018, \textsf{v2.0}}
\maketitle

\begin{abstract}\noindent
\textsf{childdoc} is a \LaTeXe{} package
that enables the direct compilation
of document sections included by |\include|
to individual files.
\end{abstract}

\begingroup
\parskip0ex
\tableofcontents
\endgroup

%%%%%%%%%%%%%%%%%%%%%%%%%%%%%%%%%%%%%%%%%%%%%%%%%%%%%%%%%%%%%%%%%%%%%%%%%%%%%%%%
%%%%%%%%%%%%%%%%%%%%%%%%%%%%%%%%%%%%%%%%%%%%%%%%%%%%%%%%%%%%%%%%%%%%%%%%%%%%%%%%
\section{Introduction}

\LaTeX{} provides a mechanism to structure a large document (such as a book)
into a main file and several child files (containing the chapters)
using the |\include| command.
This mechanism is beneficial for documents
which span hundreds of pages in order to
make the source file(s) more manageable.
Moreover, compilation can be restricted to
selected child files by means of the |\includeonly| command.
The latter feature can be used to reduce the compilation time while editing
(this was significantly more useful in the earlier days of \LaTeX{})
or to generate a smaller document which is easier to navigate.
Another application of |\includeonly| is to generate
documents consisting of selected parts of the complete document.

However, there are a few drawbacks of the plain |\include| mechanism:
\begin{itemize}
\item
The child files cannot be compiled on their own,
they can only be compiled via the main file.
A naive editing environment
(such as a text editor with an option
to have the current file processed by \LaTeX)
may require one to switch to the main file before compiling;
attempting to compile the child file produces errors.
\item
The main file must be modified (each time)
to adjust the |\includeonly| command
to the present needs. This easily leaves the main file in a messy state.
\item
The generated document will always carry the filename
of the main document. This is inconvenient if
several child files are to be compiled and
to be kept for distribution.
\end{itemize}

The present package provides a simple interface
to make child files individually compilable by \LaTeX{}.
Compiling a child file then has the same effect as compiling
the main file with an |\includeonly| command
to select the appropriate child.
Moreover the generated document will carry the name of the child
rather than the main file.
This resolves all three above issues.

This feature is meant to make the editing of books,
thesis documents and lecture notes somewhat more convenient.
However, the package can also be used efficiently for
composing a series of documents (such as exercise sheets)
which are typically distributed individually.
It then assists the author in generating the individual documents
(potentially in different versions)
as well as a document containing the collected series.
Another application is in developing style files
or other kinds of included material
where compilation of the style file could redirect
to a sample or test file.

%%%%%%%%%%%%%%%%%%%%%%%%%%%%%%%%%%%%%%%%%%%%%%%%%%%%%%%%%%%%%%%%%%%%%%%%%%%%%%%%
%%%%%%%%%%%%%%%%%%%%%%%%%%%%%%%%%%%%%%%%%%%%%%%%%%%%%%%%%%%%%%%%%%%%%%%%%%%%%%%%
\section{Usage}

First of all, the package \textsf{childdoc} is \emph{not} a standard
\LaTeXe{} |.sty| style file! Therefore it needs to be invoked in
a non-standard way.

%%%%%%%%%%%%%%%%%%%%%%%%%%%%%%%%%%%%%%%%%%%%%%%%%%%%%%%%%%%%%%%%%%%%%%%%%%%%%%%%
\subsection{Included Files}
\label{sec:include}

%%%%%%%%%%%%%%%%%%%%%%%%%%%%%%%%%%%%%%%%
\DescribeMacro{\childdocmain}
To use the package, add the commands
\begin{center}
\begin{tabular}{l}
|\input{childdoc.def}|\\
|\childdocmain{}|\\
\end{tabular}
\end{center}
at the very top of the main \LaTeX{} file,
in particular \emph{before} the |\documentclass| statement!
The argument of |\childdocmain| should be left empty
(but it must be present).

%%%%%%%%%%%%%%%%%%%%%%%%%%%%%%%%%%%%%%%%
\DescribeMacro{\childdocof}
Furthermore, add the commands
\begin{center}
\begin{tabular}{l}
|\input{childdoc.def}|\\
|\childdocof{|\textit{main}|}|\\
\end{tabular}
\end{center}
at the top of every child file \textit{child}
which is included by |\include{|\textit{child}|}|
from within the main file
(or at least for those files to be compiled individually).
The argument \textit{main} must be the filename of the main file.

There are a couple of
considerations in setting up the main and child documents:

%%%%%%%%%%%%%%%%%%%%%%%%%%%%%%%%%%%%%%%%
\paragraph{Restrictions.}

Please note the following restrictions:
\begin{itemize}
\item
|\childdocmain| must be called with one argument \textit{main}
to ensure compatibility with earlier version of the package.
It must either be empty (|\childdocmain{}|)
or precisely match the filename of the main file in which it is specified.
See \secref{sec:detection} for further information.
\item
The filename \textit{main} must be specified without the |.tex| extension.
\item
The filename \textit{main} is case sensitive
(even in case-insensitive file systems)
due to internal string comparison.
\item
The argument \textit{main} should be fully expanded, it cannot be a macro.
\item
Subdirectories and special characters should be avoided in filenames.
\item
The command |\childdocmain{|\textit{main}|}| must be followed by a whitespace.
It should not be followed immediately by another command
or by a comment mark `|%|'.
This is because the \TeX{} parser reads the token immediately following
the argument of |\childdocmain| and puts it
at the beginning of every child section;
however, a white\-space is ignored.
\end{itemize}

%%%%%%%%%%%%%%%%%%%%%%%%%%%%%%%%%%%%%%%%
\paragraph{Content of Main File.}

It is advisable to place all content in the child files included by |\include|.
Any output contained in the main file will appear in all child documents
unless suppressed manually;
it cannot be suppressed automatically by the |\includeonly| directive
and thus should normally be avoided.
A method to include some content in the main file
by means of conditional processing is described in \secref{sec:conditional}.

%%%%%%%%%%%%%%%%%%%%%%%%%%%%%%%%%%%%%%%%
\paragraph{Page Numbering.}

When only a part of the document is compiled,
the appropriate numbering of pages
(as well as other status parameters)
is determined from the |.aux| files.
The latter contain information from previous passes.
However this information needs to propagate through
all intermediate child documents.
Therefore the page numbering in child documents may well
be inconsistent until the complete document is compiled at least once.

A useful (if unconventional) way to always ensure a consistent
page numbering is to restart the numbering in each child document
and denote the pages by `\textit{child}|.|\textit{page}'
where \textit{child} represents the chapter/section number of the child file.
This can be achieved by the command
|\numberwithin{page}{|\textit{child}|}|
of the \textsf{amsmath} package
where \textit{child} can be |chapter| or |section|
depending on the chosen structuring.
Alternatively, one can modify the macro |\thepage| appropriately
and reset the counter |page| at the start of each child file.

%%%%%%%%%%%%%%%%%%%%%%%%%%%%%%%%%%%%%%%%%%%%%%%%%%%%%%%%%%%%%%%%%%%%%%%%%%%%%%%%
\subsection{Conditional Processing}
\label{sec:conditional}

The package provides a mechanism to compile different versions
of a document. To customise the versions further some conditional processing
can come in handy to distinguish which version is being compiled.
The package provides two macros to describe the compilation context:

%%%%%%%%%%%%%%%%%%%%%%%%%%%%%%%%%%%%%%%%
\DescribeMacro{\ifchilddoc}
The conditional |\ifchilddoc| distinguishes between the compilation of
child documents and the main document:
%
\begin{center}
|\ifchilddoc |\textit{child-code}| |[|\||else |\textit{main-code}]| \||fi|
\end{center}

%%%%%%%%%%%%%%%%%%%%%%%%%%%%%%%%%%%%%%%%
\DescribeMacro{\childdocname}
\DescribeMacro{\childdocjob}
The macro |\childdocname| contains the filename (without extension)
of the main or child file being processed.
Note that |\childdocjob| will always contain the name of the main file.

%%%%%%%%%%%%%%%%%%%%%%%%%%%%%%%%%%%%%%%%
\paragraph{Title Page.}

Conditional processing can be used to include a title or banner page
in the main document when proper precautions are taken.
Importantly, the code in the main file should ensure that the page counter
(as well as other status parameters which are stored in the |.aux| files)
takes the same value after the conditional processing.
Otherwise the page numbers may take divergent values
depending on which part is compiled.

For example, a title page could be declared by:
%
\begin{center}
\begin{tabular}{l}
|\ifchilddoc\||else|\\
|\addtocounter{page}{-1}|\\
\textit{code for title page}\\
|\newpage|\\
|\||fi|
\end{tabular}
\end{center}
%
A banner page for the child documents can be generated by:
%
\begin{center}
\begin{tabular}{l}
|\ifchilddoc|\\
|\addtocounter{page}{-1}|\\
\textit{code for banner page}\\
|\newpage|\\
|\||fi|
\end{tabular}
\end{center}
%
Here one could write a message such as:
\begin{center}
|This is the part \childdocname{} of \childdocjob{}.|
\end{center}

%%%%%%%%%%%%%%%%%%%%%%%%%%%%%%%%%%%%%%%%%%%%%%%%%%%%%%%%%%%%%%%%%%%%%%%%%%%%%%%%
\subsection{Flags}
\label{sec:flags}

The package makes it easy to generate different versions
of the main or child documents.
To this end compilation flags can be defined
and assigned different default values.
They will be particularly useful in conjunction
with the forwarding mechanism described in \secref{sec:forward}.

For example, it may be useful to have a flag |\version|
which can be set to |draft| or |final|.
The document source will contain some conditional code
depending on the value of |\version|.
Suppose further, the flag should default to |final| for the main file
and to |draft| for child files
which is a natural assignment for editing the document.
This is achieved by placing the following code
in the preamble of the main document
(below the |\childdocmain| directive):
%
\begin{center}
\begin{tabular}{l}
|\ifchilddoc|\\
|\providecommand{\version}{draft}|\\
|\||else|\\
|\providecommand{\version}{final}|\\
|\||fi|
\end{tabular}
\end{center}
%
The definition by |\providecommand| makes sure
that previous definitions are not overwritten.
Further statements |\providecommand{\version}{...}|
can thus be added before the above code to override it.

For the main file, one might add a line
(between |\childdocmain| and the above block)
%
\begin{center}
|%\ifchilddoc\||else\providecommand{\version}{draft}\||fi|
\end{center}
%
which can be uncommented to produce a draft version.
Likewise one can add a line to the very top of a child file
(above the |\childdocof{|\textit{main}|}| directive)
%
\begin{center}
|%\providecommand{\version}{final}|
\end{center}
%
which can be uncommented to produce the final version of this child document.

%%%%%%%%%%%%%%%%%%%%%%%%%%%%%%%%%%%%%%%%%%%%%%%%%%%%%%%%%%%%%%%%%%%%%%%%%%%%%%%%
\subsection{Forwarding}
\label{sec:forward}

Different versions of the main or child documents
using compilation flags as described in \secref{sec:flags}
can be (permanently) stored in different files
for convenient compilation, viewing and distribution.
To this end, the package defines a command
to pass on compilation to a different file:

%%%%%%%%%%%%%%%%%%%%%%%%%%%%%%%%%%%%%%%%
\DescribeMacro{\childdocforward}
The command |\childdocforward| redirects processing to
another source file:
%
\begin{center}
\begin{tabular}{l}
|\input{childdoc.def}|\\
|\childdocforward[|\textit{main}|]{|\textit{dest}|}|\\
\end{tabular}
\end{center}
%
The argument \textit{dest} is the destination file
(without extension).
It should be the main file or one of the child files.
Note that further \textsf{childdoc} directives
such as |\childdocof| and |\childdocforward|
in the indicated file will be processed in this form.
The optional argument \textit{main}
passes on directly to the main file \textit{main}
while pretending to compile the child \textit{dest}.
This form behaves as if \textit{dest}
issues |\childdocof{|\textit{main}|}| right away,
and no further \textsf{childdoc} directives will be processed.

%%%%%%%%%%%%%%%%%%%%%%%%%%%%%%%%%%%%%%%%
\DescribeMacro{\...prefix}
In the alternative form |\childdocforwardprefix|,
%
\begin{center}
\begin{tabular}{l}
|\input{childdoc.def}|\\
|\childdocforwardprefix[|\textit{main}|]{|\textit{prefix}|}{|\textit{dest}|}|
\end{tabular}
\end{center}
%
the destination file is determined by a pattern
depending on the current file:
To make this work, the current file must be called
`{\textit{prefix}\hspace{0.2em}\textit{suffix}}'
with \textit{prefix} matching precisely the argument.
Processing is then passed on to the file
`{\textit{dest}\hspace{0.2em}\textit{suffix}}'.
Surely, the same effect is achieved by
directly specifying the
argument `{\textit{dest}\hspace{0.2em}\textit{suffix}}'
in the first form.
However, that requires to set up a different file
for each child. With the alternative form of the command
all these files can have exactly the same content
which simplifies setting them up and maintaining them.

For example, the following file |draft.tex|
with a compilation flag |\version| as described in \secref{sec:flags}
compiles the main document as a draft:
%
\begin{center}
\begin{tabular}{l}
|\def\version{draft}|\\
|\input{childdoc.def}|\\
|\childdocforward{|\textit{main}|}|
\end{tabular}
\end{center}
%
Likewise, the following files |final|\textit{nn}|.tex|
compile the final version of the child document
|child|\textit{nn}|.tex|:
%
\begin{center}
\begin{tabular}{l}
|\def\version{final}|\\
|\input{childdoc.def}|\\
|\childdocforwardprefix{final}{child}|
\end{tabular}
\end{center}
%

Note that when several versions of a main file and/or of each child file
are to be generated, it may be convenient to set up a |Makefile| or
shell script to automatise the process.

%%%%%%%%%%%%%%%%%%%%%%%%%%%%%%%%%%%%%%%%%%%%%%%%%%%%%%%%%%%%%%%%%%%%%%%%%%%%%%%%
\subsection{Command Line Processing}
\label{sec:commandline}

The effect of redirection files can also be achieved by invoking
the \LaTeX{} compiler with a more elaborate command line.
Most conveniently this should be done as part
of a shell script or a |Makefile|.

When using \textsf{childdoc} in the main file, the following
command lines effectively perform a redirection
(note that depending on the shell being used,
backslashes may have to be doubled: `|\|' $\to$ `|\\|'):
%
\begin{center}
|... -jobname "|\textit{target}|" |\\|"|[\textit{flags}]%
|\input{childdoc.def}\childdocforward[|\textit{main}|]{|\textit{dest}|}"|
\end{center}
%
Here \textit{target} is the name of the output file,
\textit{main} is the name of the main file
and \textit{dest} is the name of the main or child file to be processed
(all filenames without extensions).
The optional argument \textit{main} can be omitted
if \textit{main} matches \textit{dest}.
Optionally, compilation \textit{flags} can be defined via |\def| commands.
This command line makes the \TeX{} engine believe
it is compiling the file \textit{target}
whose content is specified as the latter parameter.
The provided code then forwards the processing to
\textit{main} or \textit{dest} as described in \secref{sec:forward}.

%%%%%%%%%%%%%%%%%%%%%%%%%%%%%%%%%%%%%%%%%%%%%%%%%%%%%%%%%%%%%%%%%%%%%%%%%%%%%%%%
\subsection{Include by Input}
\label{sec:input}

Including child documents by |\include| has some restrictions by design.
Most notably, the content of a child document always occupies
its own set of pages; pages cannot be shared between child documents.
Usually, this behaviour makes perfect sense
because each child document contain an essential part of the document.
However, in some situations it may be desirable to compose
a document from a collection of parts
without having mandatory page breaks between then.
For this case, the package
provides a mechanism to include parts
by |\input| which can also be processed individually.
However, by construction this mechanism
requires manual handling of the content to be output.

%%%%%%%%%%%%%%%%%%%%%%%%%%%%%%%%%%%%%%%%
\DescribeMacro{\ifchilddocmanual}
The main file should be prepared as usual, see \secref{sec:include}.
However, the document body must make a distinction
between processing of an individual part and of the main document, e.g.:
%
\begin{center}
\begin{tabular}{l}
|\ifchilddocmanual|\\
|\input{\childdocname}|\\
|\||else|\\
\textit{document body with }|\input{|\textit{part}|}|\\
|\||fi|
\end{tabular}
\end{center}
%
The conditional |\ifchilddocmanual| is true whenever
a part to be included by |\input| is being compiled,
and the name of the part is stored in |\childdocname|.

%%%%%%%%%%%%%%%%%%%%%%%%%%%%%%%%%%%%%%%%
\DescribeMacro{\childdocby}
Each part to be included by |\input| should start with:
%
\begin{center}
\begin{tabular}{l}
|\input{childdoc.def}|\\
|\childdocby{|\textit{main}|}|\\
\end{tabular}
\end{center}
%
The directive |\childdocby| is similar to |\childdocof|
described in \secref{sec:include},
but the subsequent selection of content must be done manually.
To that end, both |\ifchilddoc| and |\ifchilddocmanual|
will be true upon processing of a part,
and the name of the part is stored in |\childdocname|.
Note that |\jobname| will be set to the filename of the current part
so that each part receives an individual |.aux| file
that does not interfere with the |.aux| file(s) of the main document.
This behaviour can be altered by the alternative form
|\childdocby[*]{|\textit{main}|}| (with a non-empty optional argument)
which uses the |.aux| file of the main document
by setting |\jobname| to \textit{main}.

%%%%%%%%%%%%%%%%%%%%%%%%%%%%%%%%%%%%%%%%%%%%%%%%%%%%%%%%%%%%%%%%%%%%%%%%%%%%%%%%
\subsection{Driver Development}
\label{sec:driver}

The \textsf{childdoc} mechanism can also be use for the development
of definition files such as \LaTeX{} styles or classes.
This case differs from the above setup with multiple parts
included by |\include| in that no |\includeonly| should be invoked.
This can be achieved by starting the include file
(before |\ProvidesPackage|) with:
%
\begin{center}
\begin{tabular}{l}
|\input{childdoc.def}|\\
|\childdocforward{|\textit{main}|}|\\
\end{tabular}
\end{center}
%
or alternatively with:
%
\begin{center}
\begin{tabular}{l}
|\input{childdoc.def}|\\
|\childdocby{|\textit{main}|}|\\
\end{tabular}
\end{center}
%
Both forms have slightly different effects as described above.
The main file is prepared as usual, see \secref{sec:include}.

%%%%%%%%%%%%%%%%%%%%%%%%%%%%%%%%%%%%%%%%%%%%%%%%%%%%%%%%%%%%%%%%%%%%%%%%%%%%%%%%
\subsection{Legacy Detection}
\label{sec:detection}

The directive |\childdocmain| in the main file can detect
whether the complete document or merely a child is to be compiled
even without using the directive |\childdocof|.
This method is deprecated because it is less robust
and there is no compelling reason to use it;
it is merely provided for backward compatibility
and it may be removed in future versions.

If the detection mechanism is to be used,
it is mandatory to correctly specify
the filename of the main file as the argument of |\childdocmain|:
%
\begin{center}
\begin{tabular}{l}
|\input{childdoc.def}|\\
|\childdocmain{|\textit{main}|}|\\
\end{tabular}
\end{center}
%
If |\jobname| does not match the argument \textit{main} of |\childdocmain|,
it is assumed that |\jobname| points to the child file to be compiled.
When using |\childdocmain| with the main file specified as argument,
it suffices to start a child file
with just |\input{|\textit{main}|}|
without loading of the package and using |\childdocof|.
If instead all processing is done
with the appropriate \textsf{childdoc} directives,
the argument of \textit{main} of |\childdocmain| can be empty.

An alternative version of the command line processing described
in \secref{sec:commandline} using the detection mechanism reads:
%
\begin{center}
|... -jobname "|\textit{target}|" "|[\textit{flags}]%
[|\def\jobname{|\textit{dest}|}|]|\input{|\textit{main}|}"|
\end{center}

%%%%%%%%%%%%%%%%%%%%%%%%%%%%%%%%%%%%%%%%%%%%%%%%%%%%%%%%%%%%%%%%%%%%%%%%%%%%%%%%
\subsection{Manual Code}
\label{sec:manual}

In case one cannot be certain whether the definitions file |childdoc.def|
is installed on the target \TeX{} distribution
and one prefers not to ship it,
it is conceivable to paste a few relevant commands into the sources.

To that end, drop all statements |\input{childdoc.def}|
and perform the replacements as outlined below.
Instead of |\childdocmain{|\textit{main}|}| add the following code
to the top of the main file:
%
\begin{center}
\begin{tabular}{l}
|\||ifdefined\childdocname\endinput\||fi\newif\ifchilddoc|\\
|\edef\childdocname{\scantokens\expandafter{\jobname\noexpand}}|\\
|\def\childdocmain{|\textit{main}|}\||ifx\childdocmain\childdocname\||else|\\
|\childdoctrue\includeonly{\childdocname}\let\jobname\childdocmain\||fi|\\
\end{tabular}
\end{center}
%
Instead of |\childdocof{|\textit{main}|}| just include the main file
at the top of each child file:
%
\begin{center}
|\input{|\textit{main}|}|
\end{center}
%
A simple redirection |\childdocforward{|\textit{dest}|}| is achieved by:
%
\begin{center}
|\def\jobname{|\textit{dest}|}\input{\jobname}|
\end{center}
%
The redirection with prefix
|\childdocforwardprefix[|\textit{prefix}|]{|\textit{dest}|}|
is accomplished by:
%
\begin{center}
\begin{tabular}{l}
|{\edef\jobname{\scantokens\expandafter{\jobname\noexpand}}|\\
|\def\redirectjob |\textit{prefix}|#1~~~{\gdef\jobname{|\textit{dest}|#1}}|\\
|\expandafter\redirectjob\jobname~~~}\input{\jobname}|
\end{tabular}
\end{center}

In an alternative approach,
child documents can be compiled by a specific command line
without additional code or specific definitions:
%
\begin{center}
|... -jobname "|\textit{target}|" "|[\textit{flags}]%
|\includeonly{|\textit{dest}|}\input{|\textit{main}|}"|
\end{center}
%

%%%%%%%%%%%%%%%%%%%%%%%%%%%%%%%%%%%%%%%%%%%%%%%%%%%%%%%%%%%%%%%%%%%%%%%%%%%%%%%%
%%%%%%%%%%%%%%%%%%%%%%%%%%%%%%%%%%%%%%%%%%%%%%%%%%%%%%%%%%%%%%%%%%%%%%%%%%%%%%%%
\section{Information}

%%%%%%%%%%%%%%%%%%%%%%%%%%%%%%%%%%%%%%%%%%%%%%%%%%%%%%%%%%%%%%%%%%%%%%%%%%%%%%%%
\subsection{Copyright}

Copyright \copyright{} 2017--2018 Niklas Beisert

This work may be distributed and/or modified under the
conditions of the \LaTeX{} Project Public License, either version 1.3
of this license or (at your option) any later version.
The latest version of this license is in
  \url{http://www.latex-project.org/lppl.txt}
and version 1.3 or later is part of all distributions of \LaTeX{}
version 2005/12/01 or later.

This work has the LPPL maintenance status `maintained'.

The Current Maintainer of this work is Niklas Beisert.

This work consists of the files |README.txt|, |childdoc.ins| and |childdoc.dtx|
as well as the derived files |childdoc.def|, |cdocsamp.tex|
with |cdocsch1.tex|, |cdocsch2.tex|, |cdocspt3.tex|, |cdocspt4.tex|,
|cdocsdrf.tex|, |cdocsfn1.tex|, |cdocsfn2.tex|
as well as |childdoc.pdf|.

%%%%%%%%%%%%%%%%%%%%%%%%%%%%%%%%%%%%%%%%%%%%%%%%%%%%%%%%%%%%%%%%%%%%%%%%%%%%%%%%
\subsection{Files and Installation}

The package consists of the files:
%
\begin{center}
\begin{tabular}{ll}
    |README.txt|   & readme file \\
    |childdoc.ins| & installation file \\
    |childdoc.dtx| & source file \\
    |childdoc.def| & definition file \\
    |cdocsamp.tex| & sample main file \\
    |cdocsch1.tex| & sample include file \\
    |cdocsch2.tex| & sample include file \\
    |cdocspt3.tex| & sample part file \\
    |cdocspt4.tex| & sample part file \\
    |cdocsdrf.tex| & sample redirection file \\
    |cdocsfn1.tex| & sample redirection file \\
    |cdocsfn2.tex| & sample redirection file \\
    |childdoc.pdf| & manual
\end{tabular}
\end{center}
%
The distribution consists of the files
|README.txt|, |childdoc.ins| and |childdoc.dtx|.
%
\begin{itemize}
\item
Run (pdf)\LaTeX{} on |childdoc.dtx|
to compile the manual |childdoc.pdf| (this file).
\item
Run \LaTeX{} on |childdoc.ins| to create the definitions file |childdoc.def|
and the sample |cdocsamp.tex| with include files
|cdocsch1.tex|, |cdocsch2.tex|, |cdocspt3.tex|, |cdocspt4.tex|,
|cdocsdrf.tex|, |cdocsfn1.tex|, |cdocsfn2.tex|.
Then copy the file |childdoc.def| to an appropriate directory of your \LaTeX{}
distribution, e.g.\ \textit{texmf-root}|/tex/latex/childdoc|.
\end{itemize}

%%%%%%%%%%%%%%%%%%%%%%%%%%%%%%%%%%%%%%%%%%%%%%%%%%%%%%%%%%%%%%%%%%%%%%%%%%%%%%%%
\subsection{Related CTAN Packages}

There are several other packages which offer a similar functionality:
%
\begin{itemize}
\item
The packages
\href{http://ctan.org/pkg/docmute}{\textsf{docmute}},
\href{http://ctan.org/pkg/includex}{\textsf{includex}} and
\href{http://ctan.org/pkg/standalone}{\textsf{standalone}}
provide commands to include only the document body of
a child file thus allowing both files to be compiled individually.
\item
The packages \href{http://ctan.org/pkg/subdocs}{\textsf{subdocs}}
and \href{http://ctan.org/pkg/subfiles}{\textsf{subfiles}}
provide structures in which the main and child documents can be
encapsulated and allowing them to be compiled individually.
The inclusion mechanism is different from the conventional |\include|.
\item
The package \href{http://ctan.org/pkg/combine}{\textsf{combine}}
is an elaborate solution to combine several documents into one.
\end{itemize}
%
See also the CTAN topic \href{http://ctan.org/topic/subdocs}{\textsf{subdocs}}
for further related packages.
The present package differs from the above solutions in that
a document structure constructed with the conventional |\include| mechanism
just needs two extra commands at the top of every file
such that all constituent files can be compiled individually.

%%%%%%%%%%%%%%%%%%%%%%%%%%%%%%%%%%%%%%%%%%%%%%%%%%%%%%%%%%%%%%%%%%%%%%%%%%%%%%%%
%\subsection{Feature Suggestions}
%
%The following is a list of features which may be useful for future
%versions of this package:
%%
%\begin{itemize}
%\item
%\ldots
%\end{itemize}

%%%%%%%%%%%%%%%%%%%%%%%%%%%%%%%%%%%%%%%%%%%%%%%%%%%%%%%%%%%%%%%%%%%%%%%%%%%%%%%%
\subsection{Revision History}

%%%%%%%%%%%%%%%%%%%%%%%%%%%%%%%%%%%%%%%%
\paragraph{v2.0:} 2018/12/30

\begin{itemize}
\item
immediate forward processing
\item
added |\childdocby| mechanism
\item
manual restructured
\end{itemize}

%%%%%%%%%%%%%%%%%%%%%%%%%%%%%%%%%%%%%%%%
\paragraph{v1.6:} 2018/01/17

\begin{itemize}
\item
application for development of include files
\item
corrections to manual
\end{itemize}

%%%%%%%%%%%%%%%%%%%%%%%%%%%%%%%%%%%%%%%%
\paragraph{v1.5:} 2017/05/21

\begin{itemize}
\item
more complete structuring introduced
\item
|\childdocof| introduced
\item
|\childdoc| renamed to |\childdocmain|
\item
|\childredirect| renamed to |\childdocforward| and |\childdocforwardprefix|
and functionality expanded
\end{itemize}

%%%%%%%%%%%%%%%%%%%%%%%%%%%%%%%%%%%%%%%%
\paragraph{v1.0:} 2017/04/27

\begin{itemize}
\item
manual and install package
\item
first version published on CTAN
\end{itemize}

%%%%%%%%%%%%%%%%%%%%%%%%%%%%%%%%%%%%%%%%
\paragraph{v0.6:} 2017/04/26

\begin{itemize}
\item
redirection mechanism added
\end{itemize}

%%%%%%%%%%%%%%%%%%%%%%%%%%%%%%%%%%%%%%%%
\paragraph{v0.5:} 2017/04/26

\begin{itemize}
\item
functionality in definition file
\end{itemize}


%%%%%%%%%%%%%%%%%%%%%%%%%%%%%%%%%%%%%%%%%%%%%%%%%%%%%%%%%%%%%%%%%%%%%%%%%%%%%%%%
%%%%%%%%%%%%%%%%%%%%%%%%%%%%%%%%%%%%%%%%%%%%%%%%%%%%%%%%%%%%%%%%%%%%%%%%%%%%%%%%
%%%%%%%%%%%%%%%%%%%%%%%%%%%%%%%%%%%%%%%%%%%%%%%%%%%%%%%%%%%%%%%%%%%%%%%%%%%%%%%%
\appendix

\settowidth\MacroIndent{\rmfamily\scriptsize 000\ }

 \DocInput{childdoc.dtx}

\end{document}
%</driver>
% \fi
%
% %%%%%%%%%%%%%%%%%%%%%%%%%%%%%%%%%%%%%%%%%%%%%%%%%%%%%%%%%%%%%%%%%%%%%%%%%%%%%%
% %%%%%%%%%%%%%%%%%%%%%%%%%%%%%%%%%%%%%%%%%%%%%%%%%%%%%%%%%%%%%%%%%%%%%%%%%%%%%%
% \section{Sample}
%\iffalse
%<*samplemain>
%\fi
%
% The following presents a sample document
% with two chapters, two parts, a title page,
% a compile flag as well as three forwarding files to set the flag.
% It consists of eight |.tex| files:
% \begin{center}
% \begin{tabular}{ll}
% |cdocsamp.tex|&main file\\
% |cdocsch1.tex|&include file for chapter 1\\
% |cdocsch2.tex|&include file for chapter 2\\
% |cdocspt3.tex|&include file for part 3\\
% |cdocspt4.tex|&include file for part 4\\
% |cdocsdrf.tex|&forwarding file for main file in draft mode\\
% |cdocsfi1.tex|&forwarding file for final version of chapter 1\\
% |cdocsfi2.tex|&forwarding file for final version of chapter 2\\
% \end{tabular}
% \end{center}
% Each of the eight files can be compiled directly by the \LaTeX{} compiler.
%
% %%%%%%%%%%%%%%%%%%%%%%%%%%%%%%%%%%%%%%
% \paragraph{Main File.}
%
% The main file is called |cdocsamp.tex|.
%
% Load the \textsf{childdoc} definitions and
% declare the filename for the main document:
%    \begin{macrocode}
\input{childdoc.def}
\childdocmain{}
%    \end{macrocode}

% Optional override for |\version| flag:
%    \begin{macrocode}
%%\ifchilddoc\else\providecommand{\version}{draft}\fi
%    \end{macrocode}

% Define the default values for the |\version| flag
% (|final| for the main file and |draft| for childs):
%    \begin{macrocode}
\ifchilddoc
\providecommand{\version}{draft}
\else
\providecommand{\version}{final}
\fi
%    \end{macrocode}

% Load the standard document class:
%    \begin{macrocode}
\documentclass[12pt]{article}
%    \end{macrocode}

% Start the document body:
%    \begin{macrocode}
\begin{document}
%    \end{macrocode}

% Declare a title page.
% Print title, part of document being processed and version flag:
%    \begin{macrocode}
\addtocounter{page}{-1}
\begin{center}
{\LARGE\bfseries{}childdoc example\par}
\vspace{1cm}
\ifchilddoc
\ifchilddocmanual part\else chapter\fi:
`\childdocname' of `\childdocjob'\par
\else
main document: `\childdocjob'\par
\fi
version: \version\par
\end{center}
\newpage
%    \end{macrocode}

% Manually include selected file,
% otherwise process as usual:
%    \begin{macrocode}
\ifchilddocmanual
\section*{part `\childdocname'}
\input{\childdocname}
\else
%    \end{macrocode}

% Include the two chapters:
%    \begin{macrocode}
\include{cdocsch1}
\include{cdocsch2}
%    \end{macrocode}

% Include the two parts unless only chapters should be displayed:
%    \begin{macrocode}
\ifchilddoc\else
\section{part three}
\input{cdocspt3}
\section{part four}
\input{cdocspt4}
\fi
%    \end{macrocode}

% Process as usual until here:
%    \begin{macrocode}
\fi
%    \end{macrocode}

% End of document body:
%    \begin{macrocode}
\end{document}
%    \end{macrocode}
%\iffalse
%</samplemain>
%\fi
%
% %%%%%%%%%%%%%%%%%%%%%%%%%%%%%%%%%%%%%%
% \paragraph{Chapter Include Files.}
%
% The include files are called |cdocsch1.tex| and |cdocsch2.tex|.
%
%\iffalse
%<*samplechap1|samplechap2>
%\fi

% Optional override for |\version| flag:
%    \begin{macrocode}
%%\providecommand{\version}{final}
%    \end{macrocode}

% Include the main document:
%    \begin{macrocode}
\input{childdoc.def}
\childdocof{cdocsamp}
%    \end{macrocode}

%\iffalse
%</samplechap1|samplechap2>
%\fi
%
%\iffalse
%<*samplechap1>
%\fi
% Some text for chapter 1:
%    \begin{macrocode}
\section{one}
some text in chapter one
%    \end{macrocode}

%\iffalse
%</samplechap1>
%\fi
% Some text for chapter 2:
%\iffalse
%<*samplechap2>
%\fi
%    \begin{macrocode}
\section{two}
more text in chapter two
%    \end{macrocode}

%\iffalse
%</samplechap2>
%\fi
%
% %%%%%%%%%%%%%%%%%%%%%%%%%%%%%%%%%%%%%%
% \paragraph{Part Include Files.}
%
% The include files are called |cdocspt3.tex| and |cdocspt4.tex|.
%
%\iffalse
%<*samplepart3|samplepart4>
%\fi

% Optional override for |\version| flag:
%    \begin{macrocode}
%%\providecommand{\version}{final}
%    \end{macrocode}

% Include the main document:
%    \begin{macrocode}
\input{childdoc.def}
\childdocby{cdocsamp}
%    \end{macrocode}

%\iffalse
%</samplepart3|samplepart4>
%\fi
%
%\iffalse
%<*samplepart3>
%\fi
% Some text for part 3:
%    \begin{macrocode}
some text in part three
%    \end{macrocode}

%\iffalse
%</samplepart3>
%\fi
% Some text for part 4:
%\iffalse
%<*samplepart4>
%\fi
%    \begin{macrocode}
more text in part four
%    \end{macrocode}

%\iffalse
%</samplepart4>
%\fi
%
% %%%%%%%%%%%%%%%%%%%%%%%%%%%%%%%%%%%%%%
% \paragraph{Forwarding for a Complete Draft.}
%
% The following forwarding file |cdocsdrf.tex|
% compiles the main document in draft mode:
%\iffalse
%<*sampledraft>
%\fi
%    \begin{macrocode}
\def\version{draft}
\input{childdoc.def}
\childdocforward{cdocsamp}
%    \end{macrocode}

%\iffalse
%</sampledraft>
%\fi
%
% %%%%%%%%%%%%%%%%%%%%%%%%%%%%%%%%%%%%%%
% \paragraph{Forwarding for Final Version of the Chapters.}
%
% The following forwarding files |cdocsfn1.tex| and |cdocsfn2.tex|
% (with identical content)
% compile the final versions of the child documents
% |cdocsch1.tex| and |cdocsch2.tex|, respectively:
%\iffalse
%<*samplefinal>
%\fi
%    \begin{macrocode}
\def\version{final}
\input{childdoc.def}
\childdocforwardprefix[cdocsamp]{cdocsfn}{cdocsch}
%    \end{macrocode}

%\iffalse
%</samplefinal>
%\fi
%
% %%%%%%%%%%%%%%%%%%%%%%%%%%%%%%%%%%%%%%
% \paragraph{Command Line Processing.}
%
% The following three command lines generate the output files
% |cdocscld|, |cdocscl1| and |cdocscl2|
% which should be identical to
% |cdocsdrf|, |cdocsch1| and |cdocsfn2|, respectively:
% \begin{center}
% \begin{tabular}{l}
% |latex -jobname cdocscld \|\\
% |  "\def\version{draft}\input{childdoc.def}\childdocforward{cdocsamp}"|\\
% |latex -jobname cdocscl1 \|\\
% |  "\input{childdoc.def}\childdocforward[cdocsamp]{cdocsch1}"|\\
% |latex -jobname cdocscl2 \|\\
% |  "\def\version{final}\input{childdoc.def}\childdocforward{cdocsch2}"|
% \end{tabular}
% \end{center}
% Note that the trailing backslash on each first line
% merely continues the input to the second line
% (for convenient cut ant paste).
% Furthermore, the command |latex| can be replaced by any
% of its alternative versions such as |pdflatex|.
%
% %%%%%%%%%%%%%%%%%%%%%%%%%%%%%%%%%%%%%%%%%%%%%%%%%%%%%%%%%%%%%%%%%%%%%%%%%%%%%%
% %%%%%%%%%%%%%%%%%%%%%%%%%%%%%%%%%%%%%%%%%%%%%%%%%%%%%%%%%%%%%%%%%%%%%%%%%%%%%%
% \section{Implementation}
%\iffalse
%<*package>
%\fi
%
% This section describes the definitions file |childdoc.def|.

% The definitions cannot be loaded using |\usepackage| or |\RequirePackage|
% which has a mechanism to prevent loading a style file more than once.
% When loading the definitions by means of |\input|
% multiple instances have to be prevented manually:
%\iffalse
%This code needs to be before the `\ProvidesFile' directive
%which is defined at the beginning of this file.
%Therefore it is also placed there and commented out here.
%</package>
%<*discard>
%\fi
%    \begin{macrocode}
\ifdefined\childdocmain\endinput\fi
%    \end{macrocode}
%\iffalse
%</discard>
%<*package>
%\fi
%
% \macro{\ifchilddoc}
% \macro{\ifchilddocmanual}
% The conditional |\ifchilddoc| tells whether a
% child (true) or main (false) document is being compiled.
% The conditional |\ifchilddocmanual| tells whether
% the |\includeonly| mechanism is used (false) or
% the selection of child files must be performed manually (true).
% The definitions initialise to false:
%    \begin{macrocode}
\newif\ifchilddoc
\newif\ifchilddocmanual
%    \end{macrocode}

% \macro{\childdocname}
% \macro{\childdocjob}
% The macro |\childdocname| stores the name of the main document
% to be compiled. The macro |\childdocjob| stores the name of
% the document on which the \LaTeX{} compiler was originally invoked.
% The content of |\jobname| cannot be compared
% to filenames specified in the source due to different catcodes.
% The following code rescans |\jobname|, stores the result
% in |\childdocname| and saves a copy in |\childdocjob|:
%    \begin{macrocode}
\edef\childdocname{\scantokens\expandafter{\jobname\noexpand}}
\let\childdocjob\childdocname
%    \end{macrocode}

% \macro{\childdocdisable}
% The macro |\childdocdisable| prevents the main file
% from being processed more than once.
% At this stage, the main document command |\childdocmain|
% is assumed to be called once again where it should do nothing.
% Any subsequent call to it should prevent
% a secondary processing of the main document
% It overwrites the forwarding commands
% |\childdocof| and |\childdocforward|
% with empty macros to prevent further inclusions of the main document:
%    \begin{macrocode}
\newcommand{\childdocdisable}
{
  \renewcommand{\childdocmain}[1]{\renewcommand{\childdocmain}[1]{\endinput}}
  \renewcommand{\childdocof}[1]{}
  \renewcommand{\childdocby}[2][]{}
  \renewcommand{\childdocforward}[2][]{}
  \renewcommand{\childdocdisable}{}
}
%    \end{macrocode}

% \macro{\childdocmain}
% The macro |\childdocmain| is to be called at the top of the main file
% with nothing or the main filename (without extension) as argument.
% First, it breaks loops.
% If the argument is not empty and does not match |\childdocname|
% (which is set by the first inclusion of |childdoc.def|),
% |\ifchilddoc| is set to true, |\includeonly| is applied to the child file
% and |\jobname| is set to the main file
% (for proper handling of |.aux| files):
%    \begin{macrocode}
\newcommand{\childdocmain}[1]
{
  \childdocdisable\childdocmain{}
  \if?#1?\else
    \begingroup
      \def\childdoctmp{#1}
      \ifx\childdoctmp\childdocname
        \def\childdoctmp{}
      \else
        \def\childdoctmp
        {
          \childdoctrue
          \includeonly{\childdocname}
          \def\childdocjob{#1}
          \def\jobname{#1}
        }
      \fi
      \expandafter
    \endgroup
    \childdoctmp
  \fi
}
%    \end{macrocode}

% \macro{\childdocof}
% The command |\childdocof| redirects
% compilation to the main file |#1|.
%    \begin{macrocode}
\newcommand{\childdocof}[1]
{
  \childdocdisable
  \childdoctrue
  \includeonly{\childdocname}
  \def\jobname{#1}
  \def\childdocjob{#1}
  \input{#1}
}
%    \end{macrocode}

% \macro{\childdocby}
% The command |\childdocby| ....
%    \begin{macrocode}
\newcommand{\childdocby}[2][]
{
  \childdocdisable
  \childdoctrue
  \childdocmanualtrue
  \if?#1?\else
    \def\jobname{#2}
  \fi
  \def\childdocjob{#2}
  \input{#2}
  \endinput
}
%    \end{macrocode}

% \macro{\childdocforward}
% The command |\childdocforward| redirects
% compilation to the main file or
% (if the optional argument is given) a child file.
% Parameters are set as if the main file
% or a child file starting with |\childdocof| was compiled.
% Then compilation is handed over to the main file:
%    \begin{macrocode}
\newcommand{\childdocforward}[2][]
{
  \begingroup
    \if?#1?
      \def\childdoctmp
      {
        \def\childdocname{#2}
        \def\childdocjob{#2}
        \def\jobname{#2}
        \input{#2}
        \endinput
      }
    \else
      \def\childdoctmp
      {
        \childdocdisable
        \def\childdocname{#2}
        \childdoctrue
        \includeonly{#2}
        \def\childdocjob{#1}
        \def\jobname{#1}
        \input{#1}
        \endinput
      }
    \fi
    \expandafter
  \endgroup
  \childdoctmp
}
%    \end{macrocode}

% \macro{\childdocforwardprefix}
% The command |\childdocforwardprefix| redirects
% compilation to the main or a child file by means of a pattern.
% The prefix |#1| in the current filename is replaced by |#2|
% and the suffix of the current filename is kept
% (it is assumed that the filename does not contain the substring `|~~~|'
% which is used as a delimiter).
% Compilation is handed over to the new file by |\childdocforward|:
%    \begin{macrocode}
\newcommand{\childdocforwardprefix}[3][]
{
  \begingroup
    \def\childdocextract #2##1~~~{\def\childdoctmp{\childdocforward[#1]{#3##1}}}
    \expandafter\childdocextract\childdocname~~~
    \expandafter
  \endgroup
  \childdoctmp
}
%    \end{macrocode}

% \macro{\childdoc}
% The deprecated macro |\childdoc| is a legacy version of |\childdocmain|:
%    \begin{macrocode}
\newcommand{\childdoc}{\childdocmain}
%    \end{macrocode}

% \macro{\childdocredirect}
% The deprecated macro |\childdocredirect| is a legacy version
% of |\childdocforward| and |\childdocforwardprefix|:
%    \begin{macrocode}
\newcommand{\childdocredirect}[2][]
{
  \begingroup
    \if?#1?
      \def\childdoctmp{\childdocforward{#2}}
    \else
      \def\childdoctmp{\childdocforwardprefix{#1}{#2}}
    \fi
    \expandafter
  \endgroup
  \childdoctmp
}
%    \end{macrocode}

%\iffalse
%</package>
%\fi
%
\endinput
|\\
|\childdocby{|\textit{main}|}|\\
\end{tabular}
\end{center}
%
Both forms have slightly different effects as described above.
The main file is prepared as usual, see \secref{sec:include}.

%%%%%%%%%%%%%%%%%%%%%%%%%%%%%%%%%%%%%%%%%%%%%%%%%%%%%%%%%%%%%%%%%%%%%%%%%%%%%%%%
\subsection{Legacy Detection}
\label{sec:detection}

The directive |\childdocmain| in the main file can detect
whether the complete document or merely a child is to be compiled
even without using the directive |\childdocof|.
This method is deprecated because it is less robust
and there is no compelling reason to use it;
it is merely provided for backward compatibility
and it may be removed in future versions.

If the detection mechanism is to be used,
it is mandatory to correctly specify
the filename of the main file as the argument of |\childdocmain|:
%
\begin{center}
\begin{tabular}{l}
|% \iffalse
%
% childdoc.dtx Copyright (C) 2017-2018 Niklas Beisert
%
% This work may be distributed and/or modified under the
% conditions of the LaTeX Project Public License, either version 1.3
% of this license or (at your option) any later version.
% The latest version of this license is in
%   http://www.latex-project.org/lppl.txt
% and version 1.3 or later is part of all distributions of LaTeX
% version 2005/12/01 or later.
%
% This work has the LPPL maintenance status `maintained'.
%
% The Current Maintainer of this work is Niklas Beisert.
%
% This work consists of the files childdoc.dtx and childdoc.ins
% and the derived files childdoc.def and cdocsamp.tex with
% cdocsch1.tex, cdocsch2.tex, cdocsdrf.tex, cdocsfn1.tex, cdocsfn2.tex.
%
%<package>\ifdefined\childdocmain\endinput\fi
%<package>\ProvidesFile{childdoc.def}[2018/12/30 v2.0 child document driver]
%<samplemain>\ProvidesFile{cdocsamp.tex}[2018/12/30 v2.0 sample for childdoc]
%<*driver>
%\ProvidesFile{childdoc.drv}[2018/12/30 v2.0 childdoc reference manual file]
\PassOptionsToClass{10pt,a4paper}{article}
\documentclass{ltxdoc}

\usepackage[margin=35mm]{geometry}
\usepackage{hyperref}
\usepackage{hyperxmp}
\usepackage[usenames]{color}

\hypersetup{colorlinks=true}
\hypersetup{pdfstartview=FitH}
\hypersetup{pdfpagemode=UseNone}
\hypersetup{pdfsource={}}
\hypersetup{pdflang={en-UK}}
\hypersetup{pdfcopyright={Copyright 2017-2018 Niklas Beisert.
  This work may be distributed and/or modified under the
  conditions of the LaTeX Project Public License, either version 1.3
  of this license or (at your option) any later version.}}
\hypersetup{pdflicenseurl={http://www.latex-project.org/lppl.txt}}
\hypersetup{pdfcontactaddress={ETH Zurich, ITP, HIT K,
  Wolfgang-Pauli-Strasse 27}}
\hypersetup{pdfcontactpostcode={8093}}
\hypersetup{pdfcontactcity={Zurich}}
\hypersetup{pdfcontactcountry={Switzerland}}
\hypersetup{pdfcontactemail={nbeisert@itp.phys.ethz.ch}}
\hypersetup{pdfcontacturl={http://people.phys.ethz.ch/\xmptilde nbeisert/}}

\newcommand{\secref}[1]{\hyperref[#1]{section \ref*{#1}}}

\parskip1ex
\parindent0pt
\let\olditemize\itemize
\def\itemize{\olditemize\parskip0pt}

\begin{document}

\title{The \textsf{childdoc} Package}
\hypersetup{pdftitle={The childdoc Package}}
\author{Niklas Beisert\\[2ex]
  Institut f\"ur Theoretische Physik\\
  Eidgen\"ossische Technische Hochschule Z\"urich\\
  Wolfgang-Pauli-Strasse 27, 8093 Z\"urich, Switzerland\\[1ex]
  \href{mailto:nbeisert@itp.phys.ethz.ch}
  {\texttt{nbeisert@itp.phys.ethz.ch}}}
\hypersetup{pdfauthor={Niklas Beisert}}
\hypersetup{pdfsubject={Manual for the LaTeX2e Package childdoc}}
\date{30 December 2018, \textsf{v2.0}}
\maketitle

\begin{abstract}\noindent
\textsf{childdoc} is a \LaTeXe{} package
that enables the direct compilation
of document sections included by |\include|
to individual files.
\end{abstract}

\begingroup
\parskip0ex
\tableofcontents
\endgroup

%%%%%%%%%%%%%%%%%%%%%%%%%%%%%%%%%%%%%%%%%%%%%%%%%%%%%%%%%%%%%%%%%%%%%%%%%%%%%%%%
%%%%%%%%%%%%%%%%%%%%%%%%%%%%%%%%%%%%%%%%%%%%%%%%%%%%%%%%%%%%%%%%%%%%%%%%%%%%%%%%
\section{Introduction}

\LaTeX{} provides a mechanism to structure a large document (such as a book)
into a main file and several child files (containing the chapters)
using the |\include| command.
This mechanism is beneficial for documents
which span hundreds of pages in order to
make the source file(s) more manageable.
Moreover, compilation can be restricted to
selected child files by means of the |\includeonly| command.
The latter feature can be used to reduce the compilation time while editing
(this was significantly more useful in the earlier days of \LaTeX{})
or to generate a smaller document which is easier to navigate.
Another application of |\includeonly| is to generate
documents consisting of selected parts of the complete document.

However, there are a few drawbacks of the plain |\include| mechanism:
\begin{itemize}
\item
The child files cannot be compiled on their own,
they can only be compiled via the main file.
A naive editing environment
(such as a text editor with an option
to have the current file processed by \LaTeX)
may require one to switch to the main file before compiling;
attempting to compile the child file produces errors.
\item
The main file must be modified (each time)
to adjust the |\includeonly| command
to the present needs. This easily leaves the main file in a messy state.
\item
The generated document will always carry the filename
of the main document. This is inconvenient if
several child files are to be compiled and
to be kept for distribution.
\end{itemize}

The present package provides a simple interface
to make child files individually compilable by \LaTeX{}.
Compiling a child file then has the same effect as compiling
the main file with an |\includeonly| command
to select the appropriate child.
Moreover the generated document will carry the name of the child
rather than the main file.
This resolves all three above issues.

This feature is meant to make the editing of books,
thesis documents and lecture notes somewhat more convenient.
However, the package can also be used efficiently for
composing a series of documents (such as exercise sheets)
which are typically distributed individually.
It then assists the author in generating the individual documents
(potentially in different versions)
as well as a document containing the collected series.
Another application is in developing style files
or other kinds of included material
where compilation of the style file could redirect
to a sample or test file.

%%%%%%%%%%%%%%%%%%%%%%%%%%%%%%%%%%%%%%%%%%%%%%%%%%%%%%%%%%%%%%%%%%%%%%%%%%%%%%%%
%%%%%%%%%%%%%%%%%%%%%%%%%%%%%%%%%%%%%%%%%%%%%%%%%%%%%%%%%%%%%%%%%%%%%%%%%%%%%%%%
\section{Usage}

First of all, the package \textsf{childdoc} is \emph{not} a standard
\LaTeXe{} |.sty| style file! Therefore it needs to be invoked in
a non-standard way.

%%%%%%%%%%%%%%%%%%%%%%%%%%%%%%%%%%%%%%%%%%%%%%%%%%%%%%%%%%%%%%%%%%%%%%%%%%%%%%%%
\subsection{Included Files}
\label{sec:include}

%%%%%%%%%%%%%%%%%%%%%%%%%%%%%%%%%%%%%%%%
\DescribeMacro{\childdocmain}
To use the package, add the commands
\begin{center}
\begin{tabular}{l}
|\input{childdoc.def}|\\
|\childdocmain{}|\\
\end{tabular}
\end{center}
at the very top of the main \LaTeX{} file,
in particular \emph{before} the |\documentclass| statement!
The argument of |\childdocmain| should be left empty
(but it must be present).

%%%%%%%%%%%%%%%%%%%%%%%%%%%%%%%%%%%%%%%%
\DescribeMacro{\childdocof}
Furthermore, add the commands
\begin{center}
\begin{tabular}{l}
|\input{childdoc.def}|\\
|\childdocof{|\textit{main}|}|\\
\end{tabular}
\end{center}
at the top of every child file \textit{child}
which is included by |\include{|\textit{child}|}|
from within the main file
(or at least for those files to be compiled individually).
The argument \textit{main} must be the filename of the main file.

There are a couple of
considerations in setting up the main and child documents:

%%%%%%%%%%%%%%%%%%%%%%%%%%%%%%%%%%%%%%%%
\paragraph{Restrictions.}

Please note the following restrictions:
\begin{itemize}
\item
|\childdocmain| must be called with one argument \textit{main}
to ensure compatibility with earlier version of the package.
It must either be empty (|\childdocmain{}|)
or precisely match the filename of the main file in which it is specified.
See \secref{sec:detection} for further information.
\item
The filename \textit{main} must be specified without the |.tex| extension.
\item
The filename \textit{main} is case sensitive
(even in case-insensitive file systems)
due to internal string comparison.
\item
The argument \textit{main} should be fully expanded, it cannot be a macro.
\item
Subdirectories and special characters should be avoided in filenames.
\item
The command |\childdocmain{|\textit{main}|}| must be followed by a whitespace.
It should not be followed immediately by another command
or by a comment mark `|%|'.
This is because the \TeX{} parser reads the token immediately following
the argument of |\childdocmain| and puts it
at the beginning of every child section;
however, a white\-space is ignored.
\end{itemize}

%%%%%%%%%%%%%%%%%%%%%%%%%%%%%%%%%%%%%%%%
\paragraph{Content of Main File.}

It is advisable to place all content in the child files included by |\include|.
Any output contained in the main file will appear in all child documents
unless suppressed manually;
it cannot be suppressed automatically by the |\includeonly| directive
and thus should normally be avoided.
A method to include some content in the main file
by means of conditional processing is described in \secref{sec:conditional}.

%%%%%%%%%%%%%%%%%%%%%%%%%%%%%%%%%%%%%%%%
\paragraph{Page Numbering.}

When only a part of the document is compiled,
the appropriate numbering of pages
(as well as other status parameters)
is determined from the |.aux| files.
The latter contain information from previous passes.
However this information needs to propagate through
all intermediate child documents.
Therefore the page numbering in child documents may well
be inconsistent until the complete document is compiled at least once.

A useful (if unconventional) way to always ensure a consistent
page numbering is to restart the numbering in each child document
and denote the pages by `\textit{child}|.|\textit{page}'
where \textit{child} represents the chapter/section number of the child file.
This can be achieved by the command
|\numberwithin{page}{|\textit{child}|}|
of the \textsf{amsmath} package
where \textit{child} can be |chapter| or |section|
depending on the chosen structuring.
Alternatively, one can modify the macro |\thepage| appropriately
and reset the counter |page| at the start of each child file.

%%%%%%%%%%%%%%%%%%%%%%%%%%%%%%%%%%%%%%%%%%%%%%%%%%%%%%%%%%%%%%%%%%%%%%%%%%%%%%%%
\subsection{Conditional Processing}
\label{sec:conditional}

The package provides a mechanism to compile different versions
of a document. To customise the versions further some conditional processing
can come in handy to distinguish which version is being compiled.
The package provides two macros to describe the compilation context:

%%%%%%%%%%%%%%%%%%%%%%%%%%%%%%%%%%%%%%%%
\DescribeMacro{\ifchilddoc}
The conditional |\ifchilddoc| distinguishes between the compilation of
child documents and the main document:
%
\begin{center}
|\ifchilddoc |\textit{child-code}| |[|\||else |\textit{main-code}]| \||fi|
\end{center}

%%%%%%%%%%%%%%%%%%%%%%%%%%%%%%%%%%%%%%%%
\DescribeMacro{\childdocname}
\DescribeMacro{\childdocjob}
The macro |\childdocname| contains the filename (without extension)
of the main or child file being processed.
Note that |\childdocjob| will always contain the name of the main file.

%%%%%%%%%%%%%%%%%%%%%%%%%%%%%%%%%%%%%%%%
\paragraph{Title Page.}

Conditional processing can be used to include a title or banner page
in the main document when proper precautions are taken.
Importantly, the code in the main file should ensure that the page counter
(as well as other status parameters which are stored in the |.aux| files)
takes the same value after the conditional processing.
Otherwise the page numbers may take divergent values
depending on which part is compiled.

For example, a title page could be declared by:
%
\begin{center}
\begin{tabular}{l}
|\ifchilddoc\||else|\\
|\addtocounter{page}{-1}|\\
\textit{code for title page}\\
|\newpage|\\
|\||fi|
\end{tabular}
\end{center}
%
A banner page for the child documents can be generated by:
%
\begin{center}
\begin{tabular}{l}
|\ifchilddoc|\\
|\addtocounter{page}{-1}|\\
\textit{code for banner page}\\
|\newpage|\\
|\||fi|
\end{tabular}
\end{center}
%
Here one could write a message such as:
\begin{center}
|This is the part \childdocname{} of \childdocjob{}.|
\end{center}

%%%%%%%%%%%%%%%%%%%%%%%%%%%%%%%%%%%%%%%%%%%%%%%%%%%%%%%%%%%%%%%%%%%%%%%%%%%%%%%%
\subsection{Flags}
\label{sec:flags}

The package makes it easy to generate different versions
of the main or child documents.
To this end compilation flags can be defined
and assigned different default values.
They will be particularly useful in conjunction
with the forwarding mechanism described in \secref{sec:forward}.

For example, it may be useful to have a flag |\version|
which can be set to |draft| or |final|.
The document source will contain some conditional code
depending on the value of |\version|.
Suppose further, the flag should default to |final| for the main file
and to |draft| for child files
which is a natural assignment for editing the document.
This is achieved by placing the following code
in the preamble of the main document
(below the |\childdocmain| directive):
%
\begin{center}
\begin{tabular}{l}
|\ifchilddoc|\\
|\providecommand{\version}{draft}|\\
|\||else|\\
|\providecommand{\version}{final}|\\
|\||fi|
\end{tabular}
\end{center}
%
The definition by |\providecommand| makes sure
that previous definitions are not overwritten.
Further statements |\providecommand{\version}{...}|
can thus be added before the above code to override it.

For the main file, one might add a line
(between |\childdocmain| and the above block)
%
\begin{center}
|%\ifchilddoc\||else\providecommand{\version}{draft}\||fi|
\end{center}
%
which can be uncommented to produce a draft version.
Likewise one can add a line to the very top of a child file
(above the |\childdocof{|\textit{main}|}| directive)
%
\begin{center}
|%\providecommand{\version}{final}|
\end{center}
%
which can be uncommented to produce the final version of this child document.

%%%%%%%%%%%%%%%%%%%%%%%%%%%%%%%%%%%%%%%%%%%%%%%%%%%%%%%%%%%%%%%%%%%%%%%%%%%%%%%%
\subsection{Forwarding}
\label{sec:forward}

Different versions of the main or child documents
using compilation flags as described in \secref{sec:flags}
can be (permanently) stored in different files
for convenient compilation, viewing and distribution.
To this end, the package defines a command
to pass on compilation to a different file:

%%%%%%%%%%%%%%%%%%%%%%%%%%%%%%%%%%%%%%%%
\DescribeMacro{\childdocforward}
The command |\childdocforward| redirects processing to
another source file:
%
\begin{center}
\begin{tabular}{l}
|\input{childdoc.def}|\\
|\childdocforward[|\textit{main}|]{|\textit{dest}|}|\\
\end{tabular}
\end{center}
%
The argument \textit{dest} is the destination file
(without extension).
It should be the main file or one of the child files.
Note that further \textsf{childdoc} directives
such as |\childdocof| and |\childdocforward|
in the indicated file will be processed in this form.
The optional argument \textit{main}
passes on directly to the main file \textit{main}
while pretending to compile the child \textit{dest}.
This form behaves as if \textit{dest}
issues |\childdocof{|\textit{main}|}| right away,
and no further \textsf{childdoc} directives will be processed.

%%%%%%%%%%%%%%%%%%%%%%%%%%%%%%%%%%%%%%%%
\DescribeMacro{\...prefix}
In the alternative form |\childdocforwardprefix|,
%
\begin{center}
\begin{tabular}{l}
|\input{childdoc.def}|\\
|\childdocforwardprefix[|\textit{main}|]{|\textit{prefix}|}{|\textit{dest}|}|
\end{tabular}
\end{center}
%
the destination file is determined by a pattern
depending on the current file:
To make this work, the current file must be called
`{\textit{prefix}\hspace{0.2em}\textit{suffix}}'
with \textit{prefix} matching precisely the argument.
Processing is then passed on to the file
`{\textit{dest}\hspace{0.2em}\textit{suffix}}'.
Surely, the same effect is achieved by
directly specifying the
argument `{\textit{dest}\hspace{0.2em}\textit{suffix}}'
in the first form.
However, that requires to set up a different file
for each child. With the alternative form of the command
all these files can have exactly the same content
which simplifies setting them up and maintaining them.

For example, the following file |draft.tex|
with a compilation flag |\version| as described in \secref{sec:flags}
compiles the main document as a draft:
%
\begin{center}
\begin{tabular}{l}
|\def\version{draft}|\\
|\input{childdoc.def}|\\
|\childdocforward{|\textit{main}|}|
\end{tabular}
\end{center}
%
Likewise, the following files |final|\textit{nn}|.tex|
compile the final version of the child document
|child|\textit{nn}|.tex|:
%
\begin{center}
\begin{tabular}{l}
|\def\version{final}|\\
|\input{childdoc.def}|\\
|\childdocforwardprefix{final}{child}|
\end{tabular}
\end{center}
%

Note that when several versions of a main file and/or of each child file
are to be generated, it may be convenient to set up a |Makefile| or
shell script to automatise the process.

%%%%%%%%%%%%%%%%%%%%%%%%%%%%%%%%%%%%%%%%%%%%%%%%%%%%%%%%%%%%%%%%%%%%%%%%%%%%%%%%
\subsection{Command Line Processing}
\label{sec:commandline}

The effect of redirection files can also be achieved by invoking
the \LaTeX{} compiler with a more elaborate command line.
Most conveniently this should be done as part
of a shell script or a |Makefile|.

When using \textsf{childdoc} in the main file, the following
command lines effectively perform a redirection
(note that depending on the shell being used,
backslashes may have to be doubled: `|\|' $\to$ `|\\|'):
%
\begin{center}
|... -jobname "|\textit{target}|" |\\|"|[\textit{flags}]%
|\input{childdoc.def}\childdocforward[|\textit{main}|]{|\textit{dest}|}"|
\end{center}
%
Here \textit{target} is the name of the output file,
\textit{main} is the name of the main file
and \textit{dest} is the name of the main or child file to be processed
(all filenames without extensions).
The optional argument \textit{main} can be omitted
if \textit{main} matches \textit{dest}.
Optionally, compilation \textit{flags} can be defined via |\def| commands.
This command line makes the \TeX{} engine believe
it is compiling the file \textit{target}
whose content is specified as the latter parameter.
The provided code then forwards the processing to
\textit{main} or \textit{dest} as described in \secref{sec:forward}.

%%%%%%%%%%%%%%%%%%%%%%%%%%%%%%%%%%%%%%%%%%%%%%%%%%%%%%%%%%%%%%%%%%%%%%%%%%%%%%%%
\subsection{Include by Input}
\label{sec:input}

Including child documents by |\include| has some restrictions by design.
Most notably, the content of a child document always occupies
its own set of pages; pages cannot be shared between child documents.
Usually, this behaviour makes perfect sense
because each child document contain an essential part of the document.
However, in some situations it may be desirable to compose
a document from a collection of parts
without having mandatory page breaks between then.
For this case, the package
provides a mechanism to include parts
by |\input| which can also be processed individually.
However, by construction this mechanism
requires manual handling of the content to be output.

%%%%%%%%%%%%%%%%%%%%%%%%%%%%%%%%%%%%%%%%
\DescribeMacro{\ifchilddocmanual}
The main file should be prepared as usual, see \secref{sec:include}.
However, the document body must make a distinction
between processing of an individual part and of the main document, e.g.:
%
\begin{center}
\begin{tabular}{l}
|\ifchilddocmanual|\\
|\input{\childdocname}|\\
|\||else|\\
\textit{document body with }|\input{|\textit{part}|}|\\
|\||fi|
\end{tabular}
\end{center}
%
The conditional |\ifchilddocmanual| is true whenever
a part to be included by |\input| is being compiled,
and the name of the part is stored in |\childdocname|.

%%%%%%%%%%%%%%%%%%%%%%%%%%%%%%%%%%%%%%%%
\DescribeMacro{\childdocby}
Each part to be included by |\input| should start with:
%
\begin{center}
\begin{tabular}{l}
|\input{childdoc.def}|\\
|\childdocby{|\textit{main}|}|\\
\end{tabular}
\end{center}
%
The directive |\childdocby| is similar to |\childdocof|
described in \secref{sec:include},
but the subsequent selection of content must be done manually.
To that end, both |\ifchilddoc| and |\ifchilddocmanual|
will be true upon processing of a part,
and the name of the part is stored in |\childdocname|.
Note that |\jobname| will be set to the filename of the current part
so that each part receives an individual |.aux| file
that does not interfere with the |.aux| file(s) of the main document.
This behaviour can be altered by the alternative form
|\childdocby[*]{|\textit{main}|}| (with a non-empty optional argument)
which uses the |.aux| file of the main document
by setting |\jobname| to \textit{main}.

%%%%%%%%%%%%%%%%%%%%%%%%%%%%%%%%%%%%%%%%%%%%%%%%%%%%%%%%%%%%%%%%%%%%%%%%%%%%%%%%
\subsection{Driver Development}
\label{sec:driver}

The \textsf{childdoc} mechanism can also be use for the development
of definition files such as \LaTeX{} styles or classes.
This case differs from the above setup with multiple parts
included by |\include| in that no |\includeonly| should be invoked.
This can be achieved by starting the include file
(before |\ProvidesPackage|) with:
%
\begin{center}
\begin{tabular}{l}
|\input{childdoc.def}|\\
|\childdocforward{|\textit{main}|}|\\
\end{tabular}
\end{center}
%
or alternatively with:
%
\begin{center}
\begin{tabular}{l}
|\input{childdoc.def}|\\
|\childdocby{|\textit{main}|}|\\
\end{tabular}
\end{center}
%
Both forms have slightly different effects as described above.
The main file is prepared as usual, see \secref{sec:include}.

%%%%%%%%%%%%%%%%%%%%%%%%%%%%%%%%%%%%%%%%%%%%%%%%%%%%%%%%%%%%%%%%%%%%%%%%%%%%%%%%
\subsection{Legacy Detection}
\label{sec:detection}

The directive |\childdocmain| in the main file can detect
whether the complete document or merely a child is to be compiled
even without using the directive |\childdocof|.
This method is deprecated because it is less robust
and there is no compelling reason to use it;
it is merely provided for backward compatibility
and it may be removed in future versions.

If the detection mechanism is to be used,
it is mandatory to correctly specify
the filename of the main file as the argument of |\childdocmain|:
%
\begin{center}
\begin{tabular}{l}
|\input{childdoc.def}|\\
|\childdocmain{|\textit{main}|}|\\
\end{tabular}
\end{center}
%
If |\jobname| does not match the argument \textit{main} of |\childdocmain|,
it is assumed that |\jobname| points to the child file to be compiled.
When using |\childdocmain| with the main file specified as argument,
it suffices to start a child file
with just |\input{|\textit{main}|}|
without loading of the package and using |\childdocof|.
If instead all processing is done
with the appropriate \textsf{childdoc} directives,
the argument of \textit{main} of |\childdocmain| can be empty.

An alternative version of the command line processing described
in \secref{sec:commandline} using the detection mechanism reads:
%
\begin{center}
|... -jobname "|\textit{target}|" "|[\textit{flags}]%
[|\def\jobname{|\textit{dest}|}|]|\input{|\textit{main}|}"|
\end{center}

%%%%%%%%%%%%%%%%%%%%%%%%%%%%%%%%%%%%%%%%%%%%%%%%%%%%%%%%%%%%%%%%%%%%%%%%%%%%%%%%
\subsection{Manual Code}
\label{sec:manual}

In case one cannot be certain whether the definitions file |childdoc.def|
is installed on the target \TeX{} distribution
and one prefers not to ship it,
it is conceivable to paste a few relevant commands into the sources.

To that end, drop all statements |\input{childdoc.def}|
and perform the replacements as outlined below.
Instead of |\childdocmain{|\textit{main}|}| add the following code
to the top of the main file:
%
\begin{center}
\begin{tabular}{l}
|\||ifdefined\childdocname\endinput\||fi\newif\ifchilddoc|\\
|\edef\childdocname{\scantokens\expandafter{\jobname\noexpand}}|\\
|\def\childdocmain{|\textit{main}|}\||ifx\childdocmain\childdocname\||else|\\
|\childdoctrue\includeonly{\childdocname}\let\jobname\childdocmain\||fi|\\
\end{tabular}
\end{center}
%
Instead of |\childdocof{|\textit{main}|}| just include the main file
at the top of each child file:
%
\begin{center}
|\input{|\textit{main}|}|
\end{center}
%
A simple redirection |\childdocforward{|\textit{dest}|}| is achieved by:
%
\begin{center}
|\def\jobname{|\textit{dest}|}\input{\jobname}|
\end{center}
%
The redirection with prefix
|\childdocforwardprefix[|\textit{prefix}|]{|\textit{dest}|}|
is accomplished by:
%
\begin{center}
\begin{tabular}{l}
|{\edef\jobname{\scantokens\expandafter{\jobname\noexpand}}|\\
|\def\redirectjob |\textit{prefix}|#1~~~{\gdef\jobname{|\textit{dest}|#1}}|\\
|\expandafter\redirectjob\jobname~~~}\input{\jobname}|
\end{tabular}
\end{center}

In an alternative approach,
child documents can be compiled by a specific command line
without additional code or specific definitions:
%
\begin{center}
|... -jobname "|\textit{target}|" "|[\textit{flags}]%
|\includeonly{|\textit{dest}|}\input{|\textit{main}|}"|
\end{center}
%

%%%%%%%%%%%%%%%%%%%%%%%%%%%%%%%%%%%%%%%%%%%%%%%%%%%%%%%%%%%%%%%%%%%%%%%%%%%%%%%%
%%%%%%%%%%%%%%%%%%%%%%%%%%%%%%%%%%%%%%%%%%%%%%%%%%%%%%%%%%%%%%%%%%%%%%%%%%%%%%%%
\section{Information}

%%%%%%%%%%%%%%%%%%%%%%%%%%%%%%%%%%%%%%%%%%%%%%%%%%%%%%%%%%%%%%%%%%%%%%%%%%%%%%%%
\subsection{Copyright}

Copyright \copyright{} 2017--2018 Niklas Beisert

This work may be distributed and/or modified under the
conditions of the \LaTeX{} Project Public License, either version 1.3
of this license or (at your option) any later version.
The latest version of this license is in
  \url{http://www.latex-project.org/lppl.txt}
and version 1.3 or later is part of all distributions of \LaTeX{}
version 2005/12/01 or later.

This work has the LPPL maintenance status `maintained'.

The Current Maintainer of this work is Niklas Beisert.

This work consists of the files |README.txt|, |childdoc.ins| and |childdoc.dtx|
as well as the derived files |childdoc.def|, |cdocsamp.tex|
with |cdocsch1.tex|, |cdocsch2.tex|, |cdocspt3.tex|, |cdocspt4.tex|,
|cdocsdrf.tex|, |cdocsfn1.tex|, |cdocsfn2.tex|
as well as |childdoc.pdf|.

%%%%%%%%%%%%%%%%%%%%%%%%%%%%%%%%%%%%%%%%%%%%%%%%%%%%%%%%%%%%%%%%%%%%%%%%%%%%%%%%
\subsection{Files and Installation}

The package consists of the files:
%
\begin{center}
\begin{tabular}{ll}
    |README.txt|   & readme file \\
    |childdoc.ins| & installation file \\
    |childdoc.dtx| & source file \\
    |childdoc.def| & definition file \\
    |cdocsamp.tex| & sample main file \\
    |cdocsch1.tex| & sample include file \\
    |cdocsch2.tex| & sample include file \\
    |cdocspt3.tex| & sample part file \\
    |cdocspt4.tex| & sample part file \\
    |cdocsdrf.tex| & sample redirection file \\
    |cdocsfn1.tex| & sample redirection file \\
    |cdocsfn2.tex| & sample redirection file \\
    |childdoc.pdf| & manual
\end{tabular}
\end{center}
%
The distribution consists of the files
|README.txt|, |childdoc.ins| and |childdoc.dtx|.
%
\begin{itemize}
\item
Run (pdf)\LaTeX{} on |childdoc.dtx|
to compile the manual |childdoc.pdf| (this file).
\item
Run \LaTeX{} on |childdoc.ins| to create the definitions file |childdoc.def|
and the sample |cdocsamp.tex| with include files
|cdocsch1.tex|, |cdocsch2.tex|, |cdocspt3.tex|, |cdocspt4.tex|,
|cdocsdrf.tex|, |cdocsfn1.tex|, |cdocsfn2.tex|.
Then copy the file |childdoc.def| to an appropriate directory of your \LaTeX{}
distribution, e.g.\ \textit{texmf-root}|/tex/latex/childdoc|.
\end{itemize}

%%%%%%%%%%%%%%%%%%%%%%%%%%%%%%%%%%%%%%%%%%%%%%%%%%%%%%%%%%%%%%%%%%%%%%%%%%%%%%%%
\subsection{Related CTAN Packages}

There are several other packages which offer a similar functionality:
%
\begin{itemize}
\item
The packages
\href{http://ctan.org/pkg/docmute}{\textsf{docmute}},
\href{http://ctan.org/pkg/includex}{\textsf{includex}} and
\href{http://ctan.org/pkg/standalone}{\textsf{standalone}}
provide commands to include only the document body of
a child file thus allowing both files to be compiled individually.
\item
The packages \href{http://ctan.org/pkg/subdocs}{\textsf{subdocs}}
and \href{http://ctan.org/pkg/subfiles}{\textsf{subfiles}}
provide structures in which the main and child documents can be
encapsulated and allowing them to be compiled individually.
The inclusion mechanism is different from the conventional |\include|.
\item
The package \href{http://ctan.org/pkg/combine}{\textsf{combine}}
is an elaborate solution to combine several documents into one.
\end{itemize}
%
See also the CTAN topic \href{http://ctan.org/topic/subdocs}{\textsf{subdocs}}
for further related packages.
The present package differs from the above solutions in that
a document structure constructed with the conventional |\include| mechanism
just needs two extra commands at the top of every file
such that all constituent files can be compiled individually.

%%%%%%%%%%%%%%%%%%%%%%%%%%%%%%%%%%%%%%%%%%%%%%%%%%%%%%%%%%%%%%%%%%%%%%%%%%%%%%%%
%\subsection{Feature Suggestions}
%
%The following is a list of features which may be useful for future
%versions of this package:
%%
%\begin{itemize}
%\item
%\ldots
%\end{itemize}

%%%%%%%%%%%%%%%%%%%%%%%%%%%%%%%%%%%%%%%%%%%%%%%%%%%%%%%%%%%%%%%%%%%%%%%%%%%%%%%%
\subsection{Revision History}

%%%%%%%%%%%%%%%%%%%%%%%%%%%%%%%%%%%%%%%%
\paragraph{v2.0:} 2018/12/30

\begin{itemize}
\item
immediate forward processing
\item
added |\childdocby| mechanism
\item
manual restructured
\end{itemize}

%%%%%%%%%%%%%%%%%%%%%%%%%%%%%%%%%%%%%%%%
\paragraph{v1.6:} 2018/01/17

\begin{itemize}
\item
application for development of include files
\item
corrections to manual
\end{itemize}

%%%%%%%%%%%%%%%%%%%%%%%%%%%%%%%%%%%%%%%%
\paragraph{v1.5:} 2017/05/21

\begin{itemize}
\item
more complete structuring introduced
\item
|\childdocof| introduced
\item
|\childdoc| renamed to |\childdocmain|
\item
|\childredirect| renamed to |\childdocforward| and |\childdocforwardprefix|
and functionality expanded
\end{itemize}

%%%%%%%%%%%%%%%%%%%%%%%%%%%%%%%%%%%%%%%%
\paragraph{v1.0:} 2017/04/27

\begin{itemize}
\item
manual and install package
\item
first version published on CTAN
\end{itemize}

%%%%%%%%%%%%%%%%%%%%%%%%%%%%%%%%%%%%%%%%
\paragraph{v0.6:} 2017/04/26

\begin{itemize}
\item
redirection mechanism added
\end{itemize}

%%%%%%%%%%%%%%%%%%%%%%%%%%%%%%%%%%%%%%%%
\paragraph{v0.5:} 2017/04/26

\begin{itemize}
\item
functionality in definition file
\end{itemize}


%%%%%%%%%%%%%%%%%%%%%%%%%%%%%%%%%%%%%%%%%%%%%%%%%%%%%%%%%%%%%%%%%%%%%%%%%%%%%%%%
%%%%%%%%%%%%%%%%%%%%%%%%%%%%%%%%%%%%%%%%%%%%%%%%%%%%%%%%%%%%%%%%%%%%%%%%%%%%%%%%
%%%%%%%%%%%%%%%%%%%%%%%%%%%%%%%%%%%%%%%%%%%%%%%%%%%%%%%%%%%%%%%%%%%%%%%%%%%%%%%%
\appendix

\settowidth\MacroIndent{\rmfamily\scriptsize 000\ }

 \DocInput{childdoc.dtx}

\end{document}
%</driver>
% \fi
%
% %%%%%%%%%%%%%%%%%%%%%%%%%%%%%%%%%%%%%%%%%%%%%%%%%%%%%%%%%%%%%%%%%%%%%%%%%%%%%%
% %%%%%%%%%%%%%%%%%%%%%%%%%%%%%%%%%%%%%%%%%%%%%%%%%%%%%%%%%%%%%%%%%%%%%%%%%%%%%%
% \section{Sample}
%\iffalse
%<*samplemain>
%\fi
%
% The following presents a sample document
% with two chapters, two parts, a title page,
% a compile flag as well as three forwarding files to set the flag.
% It consists of eight |.tex| files:
% \begin{center}
% \begin{tabular}{ll}
% |cdocsamp.tex|&main file\\
% |cdocsch1.tex|&include file for chapter 1\\
% |cdocsch2.tex|&include file for chapter 2\\
% |cdocspt3.tex|&include file for part 3\\
% |cdocspt4.tex|&include file for part 4\\
% |cdocsdrf.tex|&forwarding file for main file in draft mode\\
% |cdocsfi1.tex|&forwarding file for final version of chapter 1\\
% |cdocsfi2.tex|&forwarding file for final version of chapter 2\\
% \end{tabular}
% \end{center}
% Each of the eight files can be compiled directly by the \LaTeX{} compiler.
%
% %%%%%%%%%%%%%%%%%%%%%%%%%%%%%%%%%%%%%%
% \paragraph{Main File.}
%
% The main file is called |cdocsamp.tex|.
%
% Load the \textsf{childdoc} definitions and
% declare the filename for the main document:
%    \begin{macrocode}
\input{childdoc.def}
\childdocmain{}
%    \end{macrocode}

% Optional override for |\version| flag:
%    \begin{macrocode}
%%\ifchilddoc\else\providecommand{\version}{draft}\fi
%    \end{macrocode}

% Define the default values for the |\version| flag
% (|final| for the main file and |draft| for childs):
%    \begin{macrocode}
\ifchilddoc
\providecommand{\version}{draft}
\else
\providecommand{\version}{final}
\fi
%    \end{macrocode}

% Load the standard document class:
%    \begin{macrocode}
\documentclass[12pt]{article}
%    \end{macrocode}

% Start the document body:
%    \begin{macrocode}
\begin{document}
%    \end{macrocode}

% Declare a title page.
% Print title, part of document being processed and version flag:
%    \begin{macrocode}
\addtocounter{page}{-1}
\begin{center}
{\LARGE\bfseries{}childdoc example\par}
\vspace{1cm}
\ifchilddoc
\ifchilddocmanual part\else chapter\fi:
`\childdocname' of `\childdocjob'\par
\else
main document: `\childdocjob'\par
\fi
version: \version\par
\end{center}
\newpage
%    \end{macrocode}

% Manually include selected file,
% otherwise process as usual:
%    \begin{macrocode}
\ifchilddocmanual
\section*{part `\childdocname'}
\input{\childdocname}
\else
%    \end{macrocode}

% Include the two chapters:
%    \begin{macrocode}
\include{cdocsch1}
\include{cdocsch2}
%    \end{macrocode}

% Include the two parts unless only chapters should be displayed:
%    \begin{macrocode}
\ifchilddoc\else
\section{part three}
\input{cdocspt3}
\section{part four}
\input{cdocspt4}
\fi
%    \end{macrocode}

% Process as usual until here:
%    \begin{macrocode}
\fi
%    \end{macrocode}

% End of document body:
%    \begin{macrocode}
\end{document}
%    \end{macrocode}
%\iffalse
%</samplemain>
%\fi
%
% %%%%%%%%%%%%%%%%%%%%%%%%%%%%%%%%%%%%%%
% \paragraph{Chapter Include Files.}
%
% The include files are called |cdocsch1.tex| and |cdocsch2.tex|.
%
%\iffalse
%<*samplechap1|samplechap2>
%\fi

% Optional override for |\version| flag:
%    \begin{macrocode}
%%\providecommand{\version}{final}
%    \end{macrocode}

% Include the main document:
%    \begin{macrocode}
\input{childdoc.def}
\childdocof{cdocsamp}
%    \end{macrocode}

%\iffalse
%</samplechap1|samplechap2>
%\fi
%
%\iffalse
%<*samplechap1>
%\fi
% Some text for chapter 1:
%    \begin{macrocode}
\section{one}
some text in chapter one
%    \end{macrocode}

%\iffalse
%</samplechap1>
%\fi
% Some text for chapter 2:
%\iffalse
%<*samplechap2>
%\fi
%    \begin{macrocode}
\section{two}
more text in chapter two
%    \end{macrocode}

%\iffalse
%</samplechap2>
%\fi
%
% %%%%%%%%%%%%%%%%%%%%%%%%%%%%%%%%%%%%%%
% \paragraph{Part Include Files.}
%
% The include files are called |cdocspt3.tex| and |cdocspt4.tex|.
%
%\iffalse
%<*samplepart3|samplepart4>
%\fi

% Optional override for |\version| flag:
%    \begin{macrocode}
%%\providecommand{\version}{final}
%    \end{macrocode}

% Include the main document:
%    \begin{macrocode}
\input{childdoc.def}
\childdocby{cdocsamp}
%    \end{macrocode}

%\iffalse
%</samplepart3|samplepart4>
%\fi
%
%\iffalse
%<*samplepart3>
%\fi
% Some text for part 3:
%    \begin{macrocode}
some text in part three
%    \end{macrocode}

%\iffalse
%</samplepart3>
%\fi
% Some text for part 4:
%\iffalse
%<*samplepart4>
%\fi
%    \begin{macrocode}
more text in part four
%    \end{macrocode}

%\iffalse
%</samplepart4>
%\fi
%
% %%%%%%%%%%%%%%%%%%%%%%%%%%%%%%%%%%%%%%
% \paragraph{Forwarding for a Complete Draft.}
%
% The following forwarding file |cdocsdrf.tex|
% compiles the main document in draft mode:
%\iffalse
%<*sampledraft>
%\fi
%    \begin{macrocode}
\def\version{draft}
\input{childdoc.def}
\childdocforward{cdocsamp}
%    \end{macrocode}

%\iffalse
%</sampledraft>
%\fi
%
% %%%%%%%%%%%%%%%%%%%%%%%%%%%%%%%%%%%%%%
% \paragraph{Forwarding for Final Version of the Chapters.}
%
% The following forwarding files |cdocsfn1.tex| and |cdocsfn2.tex|
% (with identical content)
% compile the final versions of the child documents
% |cdocsch1.tex| and |cdocsch2.tex|, respectively:
%\iffalse
%<*samplefinal>
%\fi
%    \begin{macrocode}
\def\version{final}
\input{childdoc.def}
\childdocforwardprefix[cdocsamp]{cdocsfn}{cdocsch}
%    \end{macrocode}

%\iffalse
%</samplefinal>
%\fi
%
% %%%%%%%%%%%%%%%%%%%%%%%%%%%%%%%%%%%%%%
% \paragraph{Command Line Processing.}
%
% The following three command lines generate the output files
% |cdocscld|, |cdocscl1| and |cdocscl2|
% which should be identical to
% |cdocsdrf|, |cdocsch1| and |cdocsfn2|, respectively:
% \begin{center}
% \begin{tabular}{l}
% |latex -jobname cdocscld \|\\
% |  "\def\version{draft}\input{childdoc.def}\childdocforward{cdocsamp}"|\\
% |latex -jobname cdocscl1 \|\\
% |  "\input{childdoc.def}\childdocforward[cdocsamp]{cdocsch1}"|\\
% |latex -jobname cdocscl2 \|\\
% |  "\def\version{final}\input{childdoc.def}\childdocforward{cdocsch2}"|
% \end{tabular}
% \end{center}
% Note that the trailing backslash on each first line
% merely continues the input to the second line
% (for convenient cut ant paste).
% Furthermore, the command |latex| can be replaced by any
% of its alternative versions such as |pdflatex|.
%
% %%%%%%%%%%%%%%%%%%%%%%%%%%%%%%%%%%%%%%%%%%%%%%%%%%%%%%%%%%%%%%%%%%%%%%%%%%%%%%
% %%%%%%%%%%%%%%%%%%%%%%%%%%%%%%%%%%%%%%%%%%%%%%%%%%%%%%%%%%%%%%%%%%%%%%%%%%%%%%
% \section{Implementation}
%\iffalse
%<*package>
%\fi
%
% This section describes the definitions file |childdoc.def|.

% The definitions cannot be loaded using |\usepackage| or |\RequirePackage|
% which has a mechanism to prevent loading a style file more than once.
% When loading the definitions by means of |\input|
% multiple instances have to be prevented manually:
%\iffalse
%This code needs to be before the `\ProvidesFile' directive
%which is defined at the beginning of this file.
%Therefore it is also placed there and commented out here.
%</package>
%<*discard>
%\fi
%    \begin{macrocode}
\ifdefined\childdocmain\endinput\fi
%    \end{macrocode}
%\iffalse
%</discard>
%<*package>
%\fi
%
% \macro{\ifchilddoc}
% \macro{\ifchilddocmanual}
% The conditional |\ifchilddoc| tells whether a
% child (true) or main (false) document is being compiled.
% The conditional |\ifchilddocmanual| tells whether
% the |\includeonly| mechanism is used (false) or
% the selection of child files must be performed manually (true).
% The definitions initialise to false:
%    \begin{macrocode}
\newif\ifchilddoc
\newif\ifchilddocmanual
%    \end{macrocode}

% \macro{\childdocname}
% \macro{\childdocjob}
% The macro |\childdocname| stores the name of the main document
% to be compiled. The macro |\childdocjob| stores the name of
% the document on which the \LaTeX{} compiler was originally invoked.
% The content of |\jobname| cannot be compared
% to filenames specified in the source due to different catcodes.
% The following code rescans |\jobname|, stores the result
% in |\childdocname| and saves a copy in |\childdocjob|:
%    \begin{macrocode}
\edef\childdocname{\scantokens\expandafter{\jobname\noexpand}}
\let\childdocjob\childdocname
%    \end{macrocode}

% \macro{\childdocdisable}
% The macro |\childdocdisable| prevents the main file
% from being processed more than once.
% At this stage, the main document command |\childdocmain|
% is assumed to be called once again where it should do nothing.
% Any subsequent call to it should prevent
% a secondary processing of the main document
% It overwrites the forwarding commands
% |\childdocof| and |\childdocforward|
% with empty macros to prevent further inclusions of the main document:
%    \begin{macrocode}
\newcommand{\childdocdisable}
{
  \renewcommand{\childdocmain}[1]{\renewcommand{\childdocmain}[1]{\endinput}}
  \renewcommand{\childdocof}[1]{}
  \renewcommand{\childdocby}[2][]{}
  \renewcommand{\childdocforward}[2][]{}
  \renewcommand{\childdocdisable}{}
}
%    \end{macrocode}

% \macro{\childdocmain}
% The macro |\childdocmain| is to be called at the top of the main file
% with nothing or the main filename (without extension) as argument.
% First, it breaks loops.
% If the argument is not empty and does not match |\childdocname|
% (which is set by the first inclusion of |childdoc.def|),
% |\ifchilddoc| is set to true, |\includeonly| is applied to the child file
% and |\jobname| is set to the main file
% (for proper handling of |.aux| files):
%    \begin{macrocode}
\newcommand{\childdocmain}[1]
{
  \childdocdisable\childdocmain{}
  \if?#1?\else
    \begingroup
      \def\childdoctmp{#1}
      \ifx\childdoctmp\childdocname
        \def\childdoctmp{}
      \else
        \def\childdoctmp
        {
          \childdoctrue
          \includeonly{\childdocname}
          \def\childdocjob{#1}
          \def\jobname{#1}
        }
      \fi
      \expandafter
    \endgroup
    \childdoctmp
  \fi
}
%    \end{macrocode}

% \macro{\childdocof}
% The command |\childdocof| redirects
% compilation to the main file |#1|.
%    \begin{macrocode}
\newcommand{\childdocof}[1]
{
  \childdocdisable
  \childdoctrue
  \includeonly{\childdocname}
  \def\jobname{#1}
  \def\childdocjob{#1}
  \input{#1}
}
%    \end{macrocode}

% \macro{\childdocby}
% The command |\childdocby| ....
%    \begin{macrocode}
\newcommand{\childdocby}[2][]
{
  \childdocdisable
  \childdoctrue
  \childdocmanualtrue
  \if?#1?\else
    \def\jobname{#2}
  \fi
  \def\childdocjob{#2}
  \input{#2}
  \endinput
}
%    \end{macrocode}

% \macro{\childdocforward}
% The command |\childdocforward| redirects
% compilation to the main file or
% (if the optional argument is given) a child file.
% Parameters are set as if the main file
% or a child file starting with |\childdocof| was compiled.
% Then compilation is handed over to the main file:
%    \begin{macrocode}
\newcommand{\childdocforward}[2][]
{
  \begingroup
    \if?#1?
      \def\childdoctmp
      {
        \def\childdocname{#2}
        \def\childdocjob{#2}
        \def\jobname{#2}
        \input{#2}
        \endinput
      }
    \else
      \def\childdoctmp
      {
        \childdocdisable
        \def\childdocname{#2}
        \childdoctrue
        \includeonly{#2}
        \def\childdocjob{#1}
        \def\jobname{#1}
        \input{#1}
        \endinput
      }
    \fi
    \expandafter
  \endgroup
  \childdoctmp
}
%    \end{macrocode}

% \macro{\childdocforwardprefix}
% The command |\childdocforwardprefix| redirects
% compilation to the main or a child file by means of a pattern.
% The prefix |#1| in the current filename is replaced by |#2|
% and the suffix of the current filename is kept
% (it is assumed that the filename does not contain the substring `|~~~|'
% which is used as a delimiter).
% Compilation is handed over to the new file by |\childdocforward|:
%    \begin{macrocode}
\newcommand{\childdocforwardprefix}[3][]
{
  \begingroup
    \def\childdocextract #2##1~~~{\def\childdoctmp{\childdocforward[#1]{#3##1}}}
    \expandafter\childdocextract\childdocname~~~
    \expandafter
  \endgroup
  \childdoctmp
}
%    \end{macrocode}

% \macro{\childdoc}
% The deprecated macro |\childdoc| is a legacy version of |\childdocmain|:
%    \begin{macrocode}
\newcommand{\childdoc}{\childdocmain}
%    \end{macrocode}

% \macro{\childdocredirect}
% The deprecated macro |\childdocredirect| is a legacy version
% of |\childdocforward| and |\childdocforwardprefix|:
%    \begin{macrocode}
\newcommand{\childdocredirect}[2][]
{
  \begingroup
    \if?#1?
      \def\childdoctmp{\childdocforward{#2}}
    \else
      \def\childdoctmp{\childdocforwardprefix{#1}{#2}}
    \fi
    \expandafter
  \endgroup
  \childdoctmp
}
%    \end{macrocode}

%\iffalse
%</package>
%\fi
%
\endinput
|\\
|\childdocmain{|\textit{main}|}|\\
\end{tabular}
\end{center}
%
If |\jobname| does not match the argument \textit{main} of |\childdocmain|,
it is assumed that |\jobname| points to the child file to be compiled.
When using |\childdocmain| with the main file specified as argument,
it suffices to start a child file
with just |\input{|\textit{main}|}|
without loading of the package and using |\childdocof|.
If instead all processing is done
with the appropriate \textsf{childdoc} directives,
the argument of \textit{main} of |\childdocmain| can be empty.

An alternative version of the command line processing described
in \secref{sec:commandline} using the detection mechanism reads:
%
\begin{center}
|... -jobname "|\textit{target}|" "|[\textit{flags}]%
[|\def\jobname{|\textit{dest}|}|]|\input{|\textit{main}|}"|
\end{center}

%%%%%%%%%%%%%%%%%%%%%%%%%%%%%%%%%%%%%%%%%%%%%%%%%%%%%%%%%%%%%%%%%%%%%%%%%%%%%%%%
\subsection{Manual Code}
\label{sec:manual}

In case one cannot be certain whether the definitions file |childdoc.def|
is installed on the target \TeX{} distribution
and one prefers not to ship it,
it is conceivable to paste a few relevant commands into the sources.

To that end, drop all statements |% \iffalse
%
% childdoc.dtx Copyright (C) 2017-2018 Niklas Beisert
%
% This work may be distributed and/or modified under the
% conditions of the LaTeX Project Public License, either version 1.3
% of this license or (at your option) any later version.
% The latest version of this license is in
%   http://www.latex-project.org/lppl.txt
% and version 1.3 or later is part of all distributions of LaTeX
% version 2005/12/01 or later.
%
% This work has the LPPL maintenance status `maintained'.
%
% The Current Maintainer of this work is Niklas Beisert.
%
% This work consists of the files childdoc.dtx and childdoc.ins
% and the derived files childdoc.def and cdocsamp.tex with
% cdocsch1.tex, cdocsch2.tex, cdocsdrf.tex, cdocsfn1.tex, cdocsfn2.tex.
%
%<package>\ifdefined\childdocmain\endinput\fi
%<package>\ProvidesFile{childdoc.def}[2018/12/30 v2.0 child document driver]
%<samplemain>\ProvidesFile{cdocsamp.tex}[2018/12/30 v2.0 sample for childdoc]
%<*driver>
%\ProvidesFile{childdoc.drv}[2018/12/30 v2.0 childdoc reference manual file]
\PassOptionsToClass{10pt,a4paper}{article}
\documentclass{ltxdoc}

\usepackage[margin=35mm]{geometry}
\usepackage{hyperref}
\usepackage{hyperxmp}
\usepackage[usenames]{color}

\hypersetup{colorlinks=true}
\hypersetup{pdfstartview=FitH}
\hypersetup{pdfpagemode=UseNone}
\hypersetup{pdfsource={}}
\hypersetup{pdflang={en-UK}}
\hypersetup{pdfcopyright={Copyright 2017-2018 Niklas Beisert.
  This work may be distributed and/or modified under the
  conditions of the LaTeX Project Public License, either version 1.3
  of this license or (at your option) any later version.}}
\hypersetup{pdflicenseurl={http://www.latex-project.org/lppl.txt}}
\hypersetup{pdfcontactaddress={ETH Zurich, ITP, HIT K,
  Wolfgang-Pauli-Strasse 27}}
\hypersetup{pdfcontactpostcode={8093}}
\hypersetup{pdfcontactcity={Zurich}}
\hypersetup{pdfcontactcountry={Switzerland}}
\hypersetup{pdfcontactemail={nbeisert@itp.phys.ethz.ch}}
\hypersetup{pdfcontacturl={http://people.phys.ethz.ch/\xmptilde nbeisert/}}

\newcommand{\secref}[1]{\hyperref[#1]{section \ref*{#1}}}

\parskip1ex
\parindent0pt
\let\olditemize\itemize
\def\itemize{\olditemize\parskip0pt}

\begin{document}

\title{The \textsf{childdoc} Package}
\hypersetup{pdftitle={The childdoc Package}}
\author{Niklas Beisert\\[2ex]
  Institut f\"ur Theoretische Physik\\
  Eidgen\"ossische Technische Hochschule Z\"urich\\
  Wolfgang-Pauli-Strasse 27, 8093 Z\"urich, Switzerland\\[1ex]
  \href{mailto:nbeisert@itp.phys.ethz.ch}
  {\texttt{nbeisert@itp.phys.ethz.ch}}}
\hypersetup{pdfauthor={Niklas Beisert}}
\hypersetup{pdfsubject={Manual for the LaTeX2e Package childdoc}}
\date{30 December 2018, \textsf{v2.0}}
\maketitle

\begin{abstract}\noindent
\textsf{childdoc} is a \LaTeXe{} package
that enables the direct compilation
of document sections included by |\include|
to individual files.
\end{abstract}

\begingroup
\parskip0ex
\tableofcontents
\endgroup

%%%%%%%%%%%%%%%%%%%%%%%%%%%%%%%%%%%%%%%%%%%%%%%%%%%%%%%%%%%%%%%%%%%%%%%%%%%%%%%%
%%%%%%%%%%%%%%%%%%%%%%%%%%%%%%%%%%%%%%%%%%%%%%%%%%%%%%%%%%%%%%%%%%%%%%%%%%%%%%%%
\section{Introduction}

\LaTeX{} provides a mechanism to structure a large document (such as a book)
into a main file and several child files (containing the chapters)
using the |\include| command.
This mechanism is beneficial for documents
which span hundreds of pages in order to
make the source file(s) more manageable.
Moreover, compilation can be restricted to
selected child files by means of the |\includeonly| command.
The latter feature can be used to reduce the compilation time while editing
(this was significantly more useful in the earlier days of \LaTeX{})
or to generate a smaller document which is easier to navigate.
Another application of |\includeonly| is to generate
documents consisting of selected parts of the complete document.

However, there are a few drawbacks of the plain |\include| mechanism:
\begin{itemize}
\item
The child files cannot be compiled on their own,
they can only be compiled via the main file.
A naive editing environment
(such as a text editor with an option
to have the current file processed by \LaTeX)
may require one to switch to the main file before compiling;
attempting to compile the child file produces errors.
\item
The main file must be modified (each time)
to adjust the |\includeonly| command
to the present needs. This easily leaves the main file in a messy state.
\item
The generated document will always carry the filename
of the main document. This is inconvenient if
several child files are to be compiled and
to be kept for distribution.
\end{itemize}

The present package provides a simple interface
to make child files individually compilable by \LaTeX{}.
Compiling a child file then has the same effect as compiling
the main file with an |\includeonly| command
to select the appropriate child.
Moreover the generated document will carry the name of the child
rather than the main file.
This resolves all three above issues.

This feature is meant to make the editing of books,
thesis documents and lecture notes somewhat more convenient.
However, the package can also be used efficiently for
composing a series of documents (such as exercise sheets)
which are typically distributed individually.
It then assists the author in generating the individual documents
(potentially in different versions)
as well as a document containing the collected series.
Another application is in developing style files
or other kinds of included material
where compilation of the style file could redirect
to a sample or test file.

%%%%%%%%%%%%%%%%%%%%%%%%%%%%%%%%%%%%%%%%%%%%%%%%%%%%%%%%%%%%%%%%%%%%%%%%%%%%%%%%
%%%%%%%%%%%%%%%%%%%%%%%%%%%%%%%%%%%%%%%%%%%%%%%%%%%%%%%%%%%%%%%%%%%%%%%%%%%%%%%%
\section{Usage}

First of all, the package \textsf{childdoc} is \emph{not} a standard
\LaTeXe{} |.sty| style file! Therefore it needs to be invoked in
a non-standard way.

%%%%%%%%%%%%%%%%%%%%%%%%%%%%%%%%%%%%%%%%%%%%%%%%%%%%%%%%%%%%%%%%%%%%%%%%%%%%%%%%
\subsection{Included Files}
\label{sec:include}

%%%%%%%%%%%%%%%%%%%%%%%%%%%%%%%%%%%%%%%%
\DescribeMacro{\childdocmain}
To use the package, add the commands
\begin{center}
\begin{tabular}{l}
|\input{childdoc.def}|\\
|\childdocmain{}|\\
\end{tabular}
\end{center}
at the very top of the main \LaTeX{} file,
in particular \emph{before} the |\documentclass| statement!
The argument of |\childdocmain| should be left empty
(but it must be present).

%%%%%%%%%%%%%%%%%%%%%%%%%%%%%%%%%%%%%%%%
\DescribeMacro{\childdocof}
Furthermore, add the commands
\begin{center}
\begin{tabular}{l}
|\input{childdoc.def}|\\
|\childdocof{|\textit{main}|}|\\
\end{tabular}
\end{center}
at the top of every child file \textit{child}
which is included by |\include{|\textit{child}|}|
from within the main file
(or at least for those files to be compiled individually).
The argument \textit{main} must be the filename of the main file.

There are a couple of
considerations in setting up the main and child documents:

%%%%%%%%%%%%%%%%%%%%%%%%%%%%%%%%%%%%%%%%
\paragraph{Restrictions.}

Please note the following restrictions:
\begin{itemize}
\item
|\childdocmain| must be called with one argument \textit{main}
to ensure compatibility with earlier version of the package.
It must either be empty (|\childdocmain{}|)
or precisely match the filename of the main file in which it is specified.
See \secref{sec:detection} for further information.
\item
The filename \textit{main} must be specified without the |.tex| extension.
\item
The filename \textit{main} is case sensitive
(even in case-insensitive file systems)
due to internal string comparison.
\item
The argument \textit{main} should be fully expanded, it cannot be a macro.
\item
Subdirectories and special characters should be avoided in filenames.
\item
The command |\childdocmain{|\textit{main}|}| must be followed by a whitespace.
It should not be followed immediately by another command
or by a comment mark `|%|'.
This is because the \TeX{} parser reads the token immediately following
the argument of |\childdocmain| and puts it
at the beginning of every child section;
however, a white\-space is ignored.
\end{itemize}

%%%%%%%%%%%%%%%%%%%%%%%%%%%%%%%%%%%%%%%%
\paragraph{Content of Main File.}

It is advisable to place all content in the child files included by |\include|.
Any output contained in the main file will appear in all child documents
unless suppressed manually;
it cannot be suppressed automatically by the |\includeonly| directive
and thus should normally be avoided.
A method to include some content in the main file
by means of conditional processing is described in \secref{sec:conditional}.

%%%%%%%%%%%%%%%%%%%%%%%%%%%%%%%%%%%%%%%%
\paragraph{Page Numbering.}

When only a part of the document is compiled,
the appropriate numbering of pages
(as well as other status parameters)
is determined from the |.aux| files.
The latter contain information from previous passes.
However this information needs to propagate through
all intermediate child documents.
Therefore the page numbering in child documents may well
be inconsistent until the complete document is compiled at least once.

A useful (if unconventional) way to always ensure a consistent
page numbering is to restart the numbering in each child document
and denote the pages by `\textit{child}|.|\textit{page}'
where \textit{child} represents the chapter/section number of the child file.
This can be achieved by the command
|\numberwithin{page}{|\textit{child}|}|
of the \textsf{amsmath} package
where \textit{child} can be |chapter| or |section|
depending on the chosen structuring.
Alternatively, one can modify the macro |\thepage| appropriately
and reset the counter |page| at the start of each child file.

%%%%%%%%%%%%%%%%%%%%%%%%%%%%%%%%%%%%%%%%%%%%%%%%%%%%%%%%%%%%%%%%%%%%%%%%%%%%%%%%
\subsection{Conditional Processing}
\label{sec:conditional}

The package provides a mechanism to compile different versions
of a document. To customise the versions further some conditional processing
can come in handy to distinguish which version is being compiled.
The package provides two macros to describe the compilation context:

%%%%%%%%%%%%%%%%%%%%%%%%%%%%%%%%%%%%%%%%
\DescribeMacro{\ifchilddoc}
The conditional |\ifchilddoc| distinguishes between the compilation of
child documents and the main document:
%
\begin{center}
|\ifchilddoc |\textit{child-code}| |[|\||else |\textit{main-code}]| \||fi|
\end{center}

%%%%%%%%%%%%%%%%%%%%%%%%%%%%%%%%%%%%%%%%
\DescribeMacro{\childdocname}
\DescribeMacro{\childdocjob}
The macro |\childdocname| contains the filename (without extension)
of the main or child file being processed.
Note that |\childdocjob| will always contain the name of the main file.

%%%%%%%%%%%%%%%%%%%%%%%%%%%%%%%%%%%%%%%%
\paragraph{Title Page.}

Conditional processing can be used to include a title or banner page
in the main document when proper precautions are taken.
Importantly, the code in the main file should ensure that the page counter
(as well as other status parameters which are stored in the |.aux| files)
takes the same value after the conditional processing.
Otherwise the page numbers may take divergent values
depending on which part is compiled.

For example, a title page could be declared by:
%
\begin{center}
\begin{tabular}{l}
|\ifchilddoc\||else|\\
|\addtocounter{page}{-1}|\\
\textit{code for title page}\\
|\newpage|\\
|\||fi|
\end{tabular}
\end{center}
%
A banner page for the child documents can be generated by:
%
\begin{center}
\begin{tabular}{l}
|\ifchilddoc|\\
|\addtocounter{page}{-1}|\\
\textit{code for banner page}\\
|\newpage|\\
|\||fi|
\end{tabular}
\end{center}
%
Here one could write a message such as:
\begin{center}
|This is the part \childdocname{} of \childdocjob{}.|
\end{center}

%%%%%%%%%%%%%%%%%%%%%%%%%%%%%%%%%%%%%%%%%%%%%%%%%%%%%%%%%%%%%%%%%%%%%%%%%%%%%%%%
\subsection{Flags}
\label{sec:flags}

The package makes it easy to generate different versions
of the main or child documents.
To this end compilation flags can be defined
and assigned different default values.
They will be particularly useful in conjunction
with the forwarding mechanism described in \secref{sec:forward}.

For example, it may be useful to have a flag |\version|
which can be set to |draft| or |final|.
The document source will contain some conditional code
depending on the value of |\version|.
Suppose further, the flag should default to |final| for the main file
and to |draft| for child files
which is a natural assignment for editing the document.
This is achieved by placing the following code
in the preamble of the main document
(below the |\childdocmain| directive):
%
\begin{center}
\begin{tabular}{l}
|\ifchilddoc|\\
|\providecommand{\version}{draft}|\\
|\||else|\\
|\providecommand{\version}{final}|\\
|\||fi|
\end{tabular}
\end{center}
%
The definition by |\providecommand| makes sure
that previous definitions are not overwritten.
Further statements |\providecommand{\version}{...}|
can thus be added before the above code to override it.

For the main file, one might add a line
(between |\childdocmain| and the above block)
%
\begin{center}
|%\ifchilddoc\||else\providecommand{\version}{draft}\||fi|
\end{center}
%
which can be uncommented to produce a draft version.
Likewise one can add a line to the very top of a child file
(above the |\childdocof{|\textit{main}|}| directive)
%
\begin{center}
|%\providecommand{\version}{final}|
\end{center}
%
which can be uncommented to produce the final version of this child document.

%%%%%%%%%%%%%%%%%%%%%%%%%%%%%%%%%%%%%%%%%%%%%%%%%%%%%%%%%%%%%%%%%%%%%%%%%%%%%%%%
\subsection{Forwarding}
\label{sec:forward}

Different versions of the main or child documents
using compilation flags as described in \secref{sec:flags}
can be (permanently) stored in different files
for convenient compilation, viewing and distribution.
To this end, the package defines a command
to pass on compilation to a different file:

%%%%%%%%%%%%%%%%%%%%%%%%%%%%%%%%%%%%%%%%
\DescribeMacro{\childdocforward}
The command |\childdocforward| redirects processing to
another source file:
%
\begin{center}
\begin{tabular}{l}
|\input{childdoc.def}|\\
|\childdocforward[|\textit{main}|]{|\textit{dest}|}|\\
\end{tabular}
\end{center}
%
The argument \textit{dest} is the destination file
(without extension).
It should be the main file or one of the child files.
Note that further \textsf{childdoc} directives
such as |\childdocof| and |\childdocforward|
in the indicated file will be processed in this form.
The optional argument \textit{main}
passes on directly to the main file \textit{main}
while pretending to compile the child \textit{dest}.
This form behaves as if \textit{dest}
issues |\childdocof{|\textit{main}|}| right away,
and no further \textsf{childdoc} directives will be processed.

%%%%%%%%%%%%%%%%%%%%%%%%%%%%%%%%%%%%%%%%
\DescribeMacro{\...prefix}
In the alternative form |\childdocforwardprefix|,
%
\begin{center}
\begin{tabular}{l}
|\input{childdoc.def}|\\
|\childdocforwardprefix[|\textit{main}|]{|\textit{prefix}|}{|\textit{dest}|}|
\end{tabular}
\end{center}
%
the destination file is determined by a pattern
depending on the current file:
To make this work, the current file must be called
`{\textit{prefix}\hspace{0.2em}\textit{suffix}}'
with \textit{prefix} matching precisely the argument.
Processing is then passed on to the file
`{\textit{dest}\hspace{0.2em}\textit{suffix}}'.
Surely, the same effect is achieved by
directly specifying the
argument `{\textit{dest}\hspace{0.2em}\textit{suffix}}'
in the first form.
However, that requires to set up a different file
for each child. With the alternative form of the command
all these files can have exactly the same content
which simplifies setting them up and maintaining them.

For example, the following file |draft.tex|
with a compilation flag |\version| as described in \secref{sec:flags}
compiles the main document as a draft:
%
\begin{center}
\begin{tabular}{l}
|\def\version{draft}|\\
|\input{childdoc.def}|\\
|\childdocforward{|\textit{main}|}|
\end{tabular}
\end{center}
%
Likewise, the following files |final|\textit{nn}|.tex|
compile the final version of the child document
|child|\textit{nn}|.tex|:
%
\begin{center}
\begin{tabular}{l}
|\def\version{final}|\\
|\input{childdoc.def}|\\
|\childdocforwardprefix{final}{child}|
\end{tabular}
\end{center}
%

Note that when several versions of a main file and/or of each child file
are to be generated, it may be convenient to set up a |Makefile| or
shell script to automatise the process.

%%%%%%%%%%%%%%%%%%%%%%%%%%%%%%%%%%%%%%%%%%%%%%%%%%%%%%%%%%%%%%%%%%%%%%%%%%%%%%%%
\subsection{Command Line Processing}
\label{sec:commandline}

The effect of redirection files can also be achieved by invoking
the \LaTeX{} compiler with a more elaborate command line.
Most conveniently this should be done as part
of a shell script or a |Makefile|.

When using \textsf{childdoc} in the main file, the following
command lines effectively perform a redirection
(note that depending on the shell being used,
backslashes may have to be doubled: `|\|' $\to$ `|\\|'):
%
\begin{center}
|... -jobname "|\textit{target}|" |\\|"|[\textit{flags}]%
|\input{childdoc.def}\childdocforward[|\textit{main}|]{|\textit{dest}|}"|
\end{center}
%
Here \textit{target} is the name of the output file,
\textit{main} is the name of the main file
and \textit{dest} is the name of the main or child file to be processed
(all filenames without extensions).
The optional argument \textit{main} can be omitted
if \textit{main} matches \textit{dest}.
Optionally, compilation \textit{flags} can be defined via |\def| commands.
This command line makes the \TeX{} engine believe
it is compiling the file \textit{target}
whose content is specified as the latter parameter.
The provided code then forwards the processing to
\textit{main} or \textit{dest} as described in \secref{sec:forward}.

%%%%%%%%%%%%%%%%%%%%%%%%%%%%%%%%%%%%%%%%%%%%%%%%%%%%%%%%%%%%%%%%%%%%%%%%%%%%%%%%
\subsection{Include by Input}
\label{sec:input}

Including child documents by |\include| has some restrictions by design.
Most notably, the content of a child document always occupies
its own set of pages; pages cannot be shared between child documents.
Usually, this behaviour makes perfect sense
because each child document contain an essential part of the document.
However, in some situations it may be desirable to compose
a document from a collection of parts
without having mandatory page breaks between then.
For this case, the package
provides a mechanism to include parts
by |\input| which can also be processed individually.
However, by construction this mechanism
requires manual handling of the content to be output.

%%%%%%%%%%%%%%%%%%%%%%%%%%%%%%%%%%%%%%%%
\DescribeMacro{\ifchilddocmanual}
The main file should be prepared as usual, see \secref{sec:include}.
However, the document body must make a distinction
between processing of an individual part and of the main document, e.g.:
%
\begin{center}
\begin{tabular}{l}
|\ifchilddocmanual|\\
|\input{\childdocname}|\\
|\||else|\\
\textit{document body with }|\input{|\textit{part}|}|\\
|\||fi|
\end{tabular}
\end{center}
%
The conditional |\ifchilddocmanual| is true whenever
a part to be included by |\input| is being compiled,
and the name of the part is stored in |\childdocname|.

%%%%%%%%%%%%%%%%%%%%%%%%%%%%%%%%%%%%%%%%
\DescribeMacro{\childdocby}
Each part to be included by |\input| should start with:
%
\begin{center}
\begin{tabular}{l}
|\input{childdoc.def}|\\
|\childdocby{|\textit{main}|}|\\
\end{tabular}
\end{center}
%
The directive |\childdocby| is similar to |\childdocof|
described in \secref{sec:include},
but the subsequent selection of content must be done manually.
To that end, both |\ifchilddoc| and |\ifchilddocmanual|
will be true upon processing of a part,
and the name of the part is stored in |\childdocname|.
Note that |\jobname| will be set to the filename of the current part
so that each part receives an individual |.aux| file
that does not interfere with the |.aux| file(s) of the main document.
This behaviour can be altered by the alternative form
|\childdocby[*]{|\textit{main}|}| (with a non-empty optional argument)
which uses the |.aux| file of the main document
by setting |\jobname| to \textit{main}.

%%%%%%%%%%%%%%%%%%%%%%%%%%%%%%%%%%%%%%%%%%%%%%%%%%%%%%%%%%%%%%%%%%%%%%%%%%%%%%%%
\subsection{Driver Development}
\label{sec:driver}

The \textsf{childdoc} mechanism can also be use for the development
of definition files such as \LaTeX{} styles or classes.
This case differs from the above setup with multiple parts
included by |\include| in that no |\includeonly| should be invoked.
This can be achieved by starting the include file
(before |\ProvidesPackage|) with:
%
\begin{center}
\begin{tabular}{l}
|\input{childdoc.def}|\\
|\childdocforward{|\textit{main}|}|\\
\end{tabular}
\end{center}
%
or alternatively with:
%
\begin{center}
\begin{tabular}{l}
|\input{childdoc.def}|\\
|\childdocby{|\textit{main}|}|\\
\end{tabular}
\end{center}
%
Both forms have slightly different effects as described above.
The main file is prepared as usual, see \secref{sec:include}.

%%%%%%%%%%%%%%%%%%%%%%%%%%%%%%%%%%%%%%%%%%%%%%%%%%%%%%%%%%%%%%%%%%%%%%%%%%%%%%%%
\subsection{Legacy Detection}
\label{sec:detection}

The directive |\childdocmain| in the main file can detect
whether the complete document or merely a child is to be compiled
even without using the directive |\childdocof|.
This method is deprecated because it is less robust
and there is no compelling reason to use it;
it is merely provided for backward compatibility
and it may be removed in future versions.

If the detection mechanism is to be used,
it is mandatory to correctly specify
the filename of the main file as the argument of |\childdocmain|:
%
\begin{center}
\begin{tabular}{l}
|\input{childdoc.def}|\\
|\childdocmain{|\textit{main}|}|\\
\end{tabular}
\end{center}
%
If |\jobname| does not match the argument \textit{main} of |\childdocmain|,
it is assumed that |\jobname| points to the child file to be compiled.
When using |\childdocmain| with the main file specified as argument,
it suffices to start a child file
with just |\input{|\textit{main}|}|
without loading of the package and using |\childdocof|.
If instead all processing is done
with the appropriate \textsf{childdoc} directives,
the argument of \textit{main} of |\childdocmain| can be empty.

An alternative version of the command line processing described
in \secref{sec:commandline} using the detection mechanism reads:
%
\begin{center}
|... -jobname "|\textit{target}|" "|[\textit{flags}]%
[|\def\jobname{|\textit{dest}|}|]|\input{|\textit{main}|}"|
\end{center}

%%%%%%%%%%%%%%%%%%%%%%%%%%%%%%%%%%%%%%%%%%%%%%%%%%%%%%%%%%%%%%%%%%%%%%%%%%%%%%%%
\subsection{Manual Code}
\label{sec:manual}

In case one cannot be certain whether the definitions file |childdoc.def|
is installed on the target \TeX{} distribution
and one prefers not to ship it,
it is conceivable to paste a few relevant commands into the sources.

To that end, drop all statements |\input{childdoc.def}|
and perform the replacements as outlined below.
Instead of |\childdocmain{|\textit{main}|}| add the following code
to the top of the main file:
%
\begin{center}
\begin{tabular}{l}
|\||ifdefined\childdocname\endinput\||fi\newif\ifchilddoc|\\
|\edef\childdocname{\scantokens\expandafter{\jobname\noexpand}}|\\
|\def\childdocmain{|\textit{main}|}\||ifx\childdocmain\childdocname\||else|\\
|\childdoctrue\includeonly{\childdocname}\let\jobname\childdocmain\||fi|\\
\end{tabular}
\end{center}
%
Instead of |\childdocof{|\textit{main}|}| just include the main file
at the top of each child file:
%
\begin{center}
|\input{|\textit{main}|}|
\end{center}
%
A simple redirection |\childdocforward{|\textit{dest}|}| is achieved by:
%
\begin{center}
|\def\jobname{|\textit{dest}|}\input{\jobname}|
\end{center}
%
The redirection with prefix
|\childdocforwardprefix[|\textit{prefix}|]{|\textit{dest}|}|
is accomplished by:
%
\begin{center}
\begin{tabular}{l}
|{\edef\jobname{\scantokens\expandafter{\jobname\noexpand}}|\\
|\def\redirectjob |\textit{prefix}|#1~~~{\gdef\jobname{|\textit{dest}|#1}}|\\
|\expandafter\redirectjob\jobname~~~}\input{\jobname}|
\end{tabular}
\end{center}

In an alternative approach,
child documents can be compiled by a specific command line
without additional code or specific definitions:
%
\begin{center}
|... -jobname "|\textit{target}|" "|[\textit{flags}]%
|\includeonly{|\textit{dest}|}\input{|\textit{main}|}"|
\end{center}
%

%%%%%%%%%%%%%%%%%%%%%%%%%%%%%%%%%%%%%%%%%%%%%%%%%%%%%%%%%%%%%%%%%%%%%%%%%%%%%%%%
%%%%%%%%%%%%%%%%%%%%%%%%%%%%%%%%%%%%%%%%%%%%%%%%%%%%%%%%%%%%%%%%%%%%%%%%%%%%%%%%
\section{Information}

%%%%%%%%%%%%%%%%%%%%%%%%%%%%%%%%%%%%%%%%%%%%%%%%%%%%%%%%%%%%%%%%%%%%%%%%%%%%%%%%
\subsection{Copyright}

Copyright \copyright{} 2017--2018 Niklas Beisert

This work may be distributed and/or modified under the
conditions of the \LaTeX{} Project Public License, either version 1.3
of this license or (at your option) any later version.
The latest version of this license is in
  \url{http://www.latex-project.org/lppl.txt}
and version 1.3 or later is part of all distributions of \LaTeX{}
version 2005/12/01 or later.

This work has the LPPL maintenance status `maintained'.

The Current Maintainer of this work is Niklas Beisert.

This work consists of the files |README.txt|, |childdoc.ins| and |childdoc.dtx|
as well as the derived files |childdoc.def|, |cdocsamp.tex|
with |cdocsch1.tex|, |cdocsch2.tex|, |cdocspt3.tex|, |cdocspt4.tex|,
|cdocsdrf.tex|, |cdocsfn1.tex|, |cdocsfn2.tex|
as well as |childdoc.pdf|.

%%%%%%%%%%%%%%%%%%%%%%%%%%%%%%%%%%%%%%%%%%%%%%%%%%%%%%%%%%%%%%%%%%%%%%%%%%%%%%%%
\subsection{Files and Installation}

The package consists of the files:
%
\begin{center}
\begin{tabular}{ll}
    |README.txt|   & readme file \\
    |childdoc.ins| & installation file \\
    |childdoc.dtx| & source file \\
    |childdoc.def| & definition file \\
    |cdocsamp.tex| & sample main file \\
    |cdocsch1.tex| & sample include file \\
    |cdocsch2.tex| & sample include file \\
    |cdocspt3.tex| & sample part file \\
    |cdocspt4.tex| & sample part file \\
    |cdocsdrf.tex| & sample redirection file \\
    |cdocsfn1.tex| & sample redirection file \\
    |cdocsfn2.tex| & sample redirection file \\
    |childdoc.pdf| & manual
\end{tabular}
\end{center}
%
The distribution consists of the files
|README.txt|, |childdoc.ins| and |childdoc.dtx|.
%
\begin{itemize}
\item
Run (pdf)\LaTeX{} on |childdoc.dtx|
to compile the manual |childdoc.pdf| (this file).
\item
Run \LaTeX{} on |childdoc.ins| to create the definitions file |childdoc.def|
and the sample |cdocsamp.tex| with include files
|cdocsch1.tex|, |cdocsch2.tex|, |cdocspt3.tex|, |cdocspt4.tex|,
|cdocsdrf.tex|, |cdocsfn1.tex|, |cdocsfn2.tex|.
Then copy the file |childdoc.def| to an appropriate directory of your \LaTeX{}
distribution, e.g.\ \textit{texmf-root}|/tex/latex/childdoc|.
\end{itemize}

%%%%%%%%%%%%%%%%%%%%%%%%%%%%%%%%%%%%%%%%%%%%%%%%%%%%%%%%%%%%%%%%%%%%%%%%%%%%%%%%
\subsection{Related CTAN Packages}

There are several other packages which offer a similar functionality:
%
\begin{itemize}
\item
The packages
\href{http://ctan.org/pkg/docmute}{\textsf{docmute}},
\href{http://ctan.org/pkg/includex}{\textsf{includex}} and
\href{http://ctan.org/pkg/standalone}{\textsf{standalone}}
provide commands to include only the document body of
a child file thus allowing both files to be compiled individually.
\item
The packages \href{http://ctan.org/pkg/subdocs}{\textsf{subdocs}}
and \href{http://ctan.org/pkg/subfiles}{\textsf{subfiles}}
provide structures in which the main and child documents can be
encapsulated and allowing them to be compiled individually.
The inclusion mechanism is different from the conventional |\include|.
\item
The package \href{http://ctan.org/pkg/combine}{\textsf{combine}}
is an elaborate solution to combine several documents into one.
\end{itemize}
%
See also the CTAN topic \href{http://ctan.org/topic/subdocs}{\textsf{subdocs}}
for further related packages.
The present package differs from the above solutions in that
a document structure constructed with the conventional |\include| mechanism
just needs two extra commands at the top of every file
such that all constituent files can be compiled individually.

%%%%%%%%%%%%%%%%%%%%%%%%%%%%%%%%%%%%%%%%%%%%%%%%%%%%%%%%%%%%%%%%%%%%%%%%%%%%%%%%
%\subsection{Feature Suggestions}
%
%The following is a list of features which may be useful for future
%versions of this package:
%%
%\begin{itemize}
%\item
%\ldots
%\end{itemize}

%%%%%%%%%%%%%%%%%%%%%%%%%%%%%%%%%%%%%%%%%%%%%%%%%%%%%%%%%%%%%%%%%%%%%%%%%%%%%%%%
\subsection{Revision History}

%%%%%%%%%%%%%%%%%%%%%%%%%%%%%%%%%%%%%%%%
\paragraph{v2.0:} 2018/12/30

\begin{itemize}
\item
immediate forward processing
\item
added |\childdocby| mechanism
\item
manual restructured
\end{itemize}

%%%%%%%%%%%%%%%%%%%%%%%%%%%%%%%%%%%%%%%%
\paragraph{v1.6:} 2018/01/17

\begin{itemize}
\item
application for development of include files
\item
corrections to manual
\end{itemize}

%%%%%%%%%%%%%%%%%%%%%%%%%%%%%%%%%%%%%%%%
\paragraph{v1.5:} 2017/05/21

\begin{itemize}
\item
more complete structuring introduced
\item
|\childdocof| introduced
\item
|\childdoc| renamed to |\childdocmain|
\item
|\childredirect| renamed to |\childdocforward| and |\childdocforwardprefix|
and functionality expanded
\end{itemize}

%%%%%%%%%%%%%%%%%%%%%%%%%%%%%%%%%%%%%%%%
\paragraph{v1.0:} 2017/04/27

\begin{itemize}
\item
manual and install package
\item
first version published on CTAN
\end{itemize}

%%%%%%%%%%%%%%%%%%%%%%%%%%%%%%%%%%%%%%%%
\paragraph{v0.6:} 2017/04/26

\begin{itemize}
\item
redirection mechanism added
\end{itemize}

%%%%%%%%%%%%%%%%%%%%%%%%%%%%%%%%%%%%%%%%
\paragraph{v0.5:} 2017/04/26

\begin{itemize}
\item
functionality in definition file
\end{itemize}


%%%%%%%%%%%%%%%%%%%%%%%%%%%%%%%%%%%%%%%%%%%%%%%%%%%%%%%%%%%%%%%%%%%%%%%%%%%%%%%%
%%%%%%%%%%%%%%%%%%%%%%%%%%%%%%%%%%%%%%%%%%%%%%%%%%%%%%%%%%%%%%%%%%%%%%%%%%%%%%%%
%%%%%%%%%%%%%%%%%%%%%%%%%%%%%%%%%%%%%%%%%%%%%%%%%%%%%%%%%%%%%%%%%%%%%%%%%%%%%%%%
\appendix

\settowidth\MacroIndent{\rmfamily\scriptsize 000\ }

 \DocInput{childdoc.dtx}

\end{document}
%</driver>
% \fi
%
% %%%%%%%%%%%%%%%%%%%%%%%%%%%%%%%%%%%%%%%%%%%%%%%%%%%%%%%%%%%%%%%%%%%%%%%%%%%%%%
% %%%%%%%%%%%%%%%%%%%%%%%%%%%%%%%%%%%%%%%%%%%%%%%%%%%%%%%%%%%%%%%%%%%%%%%%%%%%%%
% \section{Sample}
%\iffalse
%<*samplemain>
%\fi
%
% The following presents a sample document
% with two chapters, two parts, a title page,
% a compile flag as well as three forwarding files to set the flag.
% It consists of eight |.tex| files:
% \begin{center}
% \begin{tabular}{ll}
% |cdocsamp.tex|&main file\\
% |cdocsch1.tex|&include file for chapter 1\\
% |cdocsch2.tex|&include file for chapter 2\\
% |cdocspt3.tex|&include file for part 3\\
% |cdocspt4.tex|&include file for part 4\\
% |cdocsdrf.tex|&forwarding file for main file in draft mode\\
% |cdocsfi1.tex|&forwarding file for final version of chapter 1\\
% |cdocsfi2.tex|&forwarding file for final version of chapter 2\\
% \end{tabular}
% \end{center}
% Each of the eight files can be compiled directly by the \LaTeX{} compiler.
%
% %%%%%%%%%%%%%%%%%%%%%%%%%%%%%%%%%%%%%%
% \paragraph{Main File.}
%
% The main file is called |cdocsamp.tex|.
%
% Load the \textsf{childdoc} definitions and
% declare the filename for the main document:
%    \begin{macrocode}
\input{childdoc.def}
\childdocmain{}
%    \end{macrocode}

% Optional override for |\version| flag:
%    \begin{macrocode}
%%\ifchilddoc\else\providecommand{\version}{draft}\fi
%    \end{macrocode}

% Define the default values for the |\version| flag
% (|final| for the main file and |draft| for childs):
%    \begin{macrocode}
\ifchilddoc
\providecommand{\version}{draft}
\else
\providecommand{\version}{final}
\fi
%    \end{macrocode}

% Load the standard document class:
%    \begin{macrocode}
\documentclass[12pt]{article}
%    \end{macrocode}

% Start the document body:
%    \begin{macrocode}
\begin{document}
%    \end{macrocode}

% Declare a title page.
% Print title, part of document being processed and version flag:
%    \begin{macrocode}
\addtocounter{page}{-1}
\begin{center}
{\LARGE\bfseries{}childdoc example\par}
\vspace{1cm}
\ifchilddoc
\ifchilddocmanual part\else chapter\fi:
`\childdocname' of `\childdocjob'\par
\else
main document: `\childdocjob'\par
\fi
version: \version\par
\end{center}
\newpage
%    \end{macrocode}

% Manually include selected file,
% otherwise process as usual:
%    \begin{macrocode}
\ifchilddocmanual
\section*{part `\childdocname'}
\input{\childdocname}
\else
%    \end{macrocode}

% Include the two chapters:
%    \begin{macrocode}
\include{cdocsch1}
\include{cdocsch2}
%    \end{macrocode}

% Include the two parts unless only chapters should be displayed:
%    \begin{macrocode}
\ifchilddoc\else
\section{part three}
\input{cdocspt3}
\section{part four}
\input{cdocspt4}
\fi
%    \end{macrocode}

% Process as usual until here:
%    \begin{macrocode}
\fi
%    \end{macrocode}

% End of document body:
%    \begin{macrocode}
\end{document}
%    \end{macrocode}
%\iffalse
%</samplemain>
%\fi
%
% %%%%%%%%%%%%%%%%%%%%%%%%%%%%%%%%%%%%%%
% \paragraph{Chapter Include Files.}
%
% The include files are called |cdocsch1.tex| and |cdocsch2.tex|.
%
%\iffalse
%<*samplechap1|samplechap2>
%\fi

% Optional override for |\version| flag:
%    \begin{macrocode}
%%\providecommand{\version}{final}
%    \end{macrocode}

% Include the main document:
%    \begin{macrocode}
\input{childdoc.def}
\childdocof{cdocsamp}
%    \end{macrocode}

%\iffalse
%</samplechap1|samplechap2>
%\fi
%
%\iffalse
%<*samplechap1>
%\fi
% Some text for chapter 1:
%    \begin{macrocode}
\section{one}
some text in chapter one
%    \end{macrocode}

%\iffalse
%</samplechap1>
%\fi
% Some text for chapter 2:
%\iffalse
%<*samplechap2>
%\fi
%    \begin{macrocode}
\section{two}
more text in chapter two
%    \end{macrocode}

%\iffalse
%</samplechap2>
%\fi
%
% %%%%%%%%%%%%%%%%%%%%%%%%%%%%%%%%%%%%%%
% \paragraph{Part Include Files.}
%
% The include files are called |cdocspt3.tex| and |cdocspt4.tex|.
%
%\iffalse
%<*samplepart3|samplepart4>
%\fi

% Optional override for |\version| flag:
%    \begin{macrocode}
%%\providecommand{\version}{final}
%    \end{macrocode}

% Include the main document:
%    \begin{macrocode}
\input{childdoc.def}
\childdocby{cdocsamp}
%    \end{macrocode}

%\iffalse
%</samplepart3|samplepart4>
%\fi
%
%\iffalse
%<*samplepart3>
%\fi
% Some text for part 3:
%    \begin{macrocode}
some text in part three
%    \end{macrocode}

%\iffalse
%</samplepart3>
%\fi
% Some text for part 4:
%\iffalse
%<*samplepart4>
%\fi
%    \begin{macrocode}
more text in part four
%    \end{macrocode}

%\iffalse
%</samplepart4>
%\fi
%
% %%%%%%%%%%%%%%%%%%%%%%%%%%%%%%%%%%%%%%
% \paragraph{Forwarding for a Complete Draft.}
%
% The following forwarding file |cdocsdrf.tex|
% compiles the main document in draft mode:
%\iffalse
%<*sampledraft>
%\fi
%    \begin{macrocode}
\def\version{draft}
\input{childdoc.def}
\childdocforward{cdocsamp}
%    \end{macrocode}

%\iffalse
%</sampledraft>
%\fi
%
% %%%%%%%%%%%%%%%%%%%%%%%%%%%%%%%%%%%%%%
% \paragraph{Forwarding for Final Version of the Chapters.}
%
% The following forwarding files |cdocsfn1.tex| and |cdocsfn2.tex|
% (with identical content)
% compile the final versions of the child documents
% |cdocsch1.tex| and |cdocsch2.tex|, respectively:
%\iffalse
%<*samplefinal>
%\fi
%    \begin{macrocode}
\def\version{final}
\input{childdoc.def}
\childdocforwardprefix[cdocsamp]{cdocsfn}{cdocsch}
%    \end{macrocode}

%\iffalse
%</samplefinal>
%\fi
%
% %%%%%%%%%%%%%%%%%%%%%%%%%%%%%%%%%%%%%%
% \paragraph{Command Line Processing.}
%
% The following three command lines generate the output files
% |cdocscld|, |cdocscl1| and |cdocscl2|
% which should be identical to
% |cdocsdrf|, |cdocsch1| and |cdocsfn2|, respectively:
% \begin{center}
% \begin{tabular}{l}
% |latex -jobname cdocscld \|\\
% |  "\def\version{draft}\input{childdoc.def}\childdocforward{cdocsamp}"|\\
% |latex -jobname cdocscl1 \|\\
% |  "\input{childdoc.def}\childdocforward[cdocsamp]{cdocsch1}"|\\
% |latex -jobname cdocscl2 \|\\
% |  "\def\version{final}\input{childdoc.def}\childdocforward{cdocsch2}"|
% \end{tabular}
% \end{center}
% Note that the trailing backslash on each first line
% merely continues the input to the second line
% (for convenient cut ant paste).
% Furthermore, the command |latex| can be replaced by any
% of its alternative versions such as |pdflatex|.
%
% %%%%%%%%%%%%%%%%%%%%%%%%%%%%%%%%%%%%%%%%%%%%%%%%%%%%%%%%%%%%%%%%%%%%%%%%%%%%%%
% %%%%%%%%%%%%%%%%%%%%%%%%%%%%%%%%%%%%%%%%%%%%%%%%%%%%%%%%%%%%%%%%%%%%%%%%%%%%%%
% \section{Implementation}
%\iffalse
%<*package>
%\fi
%
% This section describes the definitions file |childdoc.def|.

% The definitions cannot be loaded using |\usepackage| or |\RequirePackage|
% which has a mechanism to prevent loading a style file more than once.
% When loading the definitions by means of |\input|
% multiple instances have to be prevented manually:
%\iffalse
%This code needs to be before the `\ProvidesFile' directive
%which is defined at the beginning of this file.
%Therefore it is also placed there and commented out here.
%</package>
%<*discard>
%\fi
%    \begin{macrocode}
\ifdefined\childdocmain\endinput\fi
%    \end{macrocode}
%\iffalse
%</discard>
%<*package>
%\fi
%
% \macro{\ifchilddoc}
% \macro{\ifchilddocmanual}
% The conditional |\ifchilddoc| tells whether a
% child (true) or main (false) document is being compiled.
% The conditional |\ifchilddocmanual| tells whether
% the |\includeonly| mechanism is used (false) or
% the selection of child files must be performed manually (true).
% The definitions initialise to false:
%    \begin{macrocode}
\newif\ifchilddoc
\newif\ifchilddocmanual
%    \end{macrocode}

% \macro{\childdocname}
% \macro{\childdocjob}
% The macro |\childdocname| stores the name of the main document
% to be compiled. The macro |\childdocjob| stores the name of
% the document on which the \LaTeX{} compiler was originally invoked.
% The content of |\jobname| cannot be compared
% to filenames specified in the source due to different catcodes.
% The following code rescans |\jobname|, stores the result
% in |\childdocname| and saves a copy in |\childdocjob|:
%    \begin{macrocode}
\edef\childdocname{\scantokens\expandafter{\jobname\noexpand}}
\let\childdocjob\childdocname
%    \end{macrocode}

% \macro{\childdocdisable}
% The macro |\childdocdisable| prevents the main file
% from being processed more than once.
% At this stage, the main document command |\childdocmain|
% is assumed to be called once again where it should do nothing.
% Any subsequent call to it should prevent
% a secondary processing of the main document
% It overwrites the forwarding commands
% |\childdocof| and |\childdocforward|
% with empty macros to prevent further inclusions of the main document:
%    \begin{macrocode}
\newcommand{\childdocdisable}
{
  \renewcommand{\childdocmain}[1]{\renewcommand{\childdocmain}[1]{\endinput}}
  \renewcommand{\childdocof}[1]{}
  \renewcommand{\childdocby}[2][]{}
  \renewcommand{\childdocforward}[2][]{}
  \renewcommand{\childdocdisable}{}
}
%    \end{macrocode}

% \macro{\childdocmain}
% The macro |\childdocmain| is to be called at the top of the main file
% with nothing or the main filename (without extension) as argument.
% First, it breaks loops.
% If the argument is not empty and does not match |\childdocname|
% (which is set by the first inclusion of |childdoc.def|),
% |\ifchilddoc| is set to true, |\includeonly| is applied to the child file
% and |\jobname| is set to the main file
% (for proper handling of |.aux| files):
%    \begin{macrocode}
\newcommand{\childdocmain}[1]
{
  \childdocdisable\childdocmain{}
  \if?#1?\else
    \begingroup
      \def\childdoctmp{#1}
      \ifx\childdoctmp\childdocname
        \def\childdoctmp{}
      \else
        \def\childdoctmp
        {
          \childdoctrue
          \includeonly{\childdocname}
          \def\childdocjob{#1}
          \def\jobname{#1}
        }
      \fi
      \expandafter
    \endgroup
    \childdoctmp
  \fi
}
%    \end{macrocode}

% \macro{\childdocof}
% The command |\childdocof| redirects
% compilation to the main file |#1|.
%    \begin{macrocode}
\newcommand{\childdocof}[1]
{
  \childdocdisable
  \childdoctrue
  \includeonly{\childdocname}
  \def\jobname{#1}
  \def\childdocjob{#1}
  \input{#1}
}
%    \end{macrocode}

% \macro{\childdocby}
% The command |\childdocby| ....
%    \begin{macrocode}
\newcommand{\childdocby}[2][]
{
  \childdocdisable
  \childdoctrue
  \childdocmanualtrue
  \if?#1?\else
    \def\jobname{#2}
  \fi
  \def\childdocjob{#2}
  \input{#2}
  \endinput
}
%    \end{macrocode}

% \macro{\childdocforward}
% The command |\childdocforward| redirects
% compilation to the main file or
% (if the optional argument is given) a child file.
% Parameters are set as if the main file
% or a child file starting with |\childdocof| was compiled.
% Then compilation is handed over to the main file:
%    \begin{macrocode}
\newcommand{\childdocforward}[2][]
{
  \begingroup
    \if?#1?
      \def\childdoctmp
      {
        \def\childdocname{#2}
        \def\childdocjob{#2}
        \def\jobname{#2}
        \input{#2}
        \endinput
      }
    \else
      \def\childdoctmp
      {
        \childdocdisable
        \def\childdocname{#2}
        \childdoctrue
        \includeonly{#2}
        \def\childdocjob{#1}
        \def\jobname{#1}
        \input{#1}
        \endinput
      }
    \fi
    \expandafter
  \endgroup
  \childdoctmp
}
%    \end{macrocode}

% \macro{\childdocforwardprefix}
% The command |\childdocforwardprefix| redirects
% compilation to the main or a child file by means of a pattern.
% The prefix |#1| in the current filename is replaced by |#2|
% and the suffix of the current filename is kept
% (it is assumed that the filename does not contain the substring `|~~~|'
% which is used as a delimiter).
% Compilation is handed over to the new file by |\childdocforward|:
%    \begin{macrocode}
\newcommand{\childdocforwardprefix}[3][]
{
  \begingroup
    \def\childdocextract #2##1~~~{\def\childdoctmp{\childdocforward[#1]{#3##1}}}
    \expandafter\childdocextract\childdocname~~~
    \expandafter
  \endgroup
  \childdoctmp
}
%    \end{macrocode}

% \macro{\childdoc}
% The deprecated macro |\childdoc| is a legacy version of |\childdocmain|:
%    \begin{macrocode}
\newcommand{\childdoc}{\childdocmain}
%    \end{macrocode}

% \macro{\childdocredirect}
% The deprecated macro |\childdocredirect| is a legacy version
% of |\childdocforward| and |\childdocforwardprefix|:
%    \begin{macrocode}
\newcommand{\childdocredirect}[2][]
{
  \begingroup
    \if?#1?
      \def\childdoctmp{\childdocforward{#2}}
    \else
      \def\childdoctmp{\childdocforwardprefix{#1}{#2}}
    \fi
    \expandafter
  \endgroup
  \childdoctmp
}
%    \end{macrocode}

%\iffalse
%</package>
%\fi
%
\endinput
|
and perform the replacements as outlined below.
Instead of |\childdocmain{|\textit{main}|}| add the following code
to the top of the main file:
%
\begin{center}
\begin{tabular}{l}
|\||ifdefined\childdocname\endinput\||fi\newif\ifchilddoc|\\
|\edef\childdocname{\scantokens\expandafter{\jobname\noexpand}}|\\
|\def\childdocmain{|\textit{main}|}\||ifx\childdocmain\childdocname\||else|\\
|\childdoctrue\includeonly{\childdocname}\let\jobname\childdocmain\||fi|\\
\end{tabular}
\end{center}
%
Instead of |\childdocof{|\textit{main}|}| just include the main file
at the top of each child file:
%
\begin{center}
|\input{|\textit{main}|}|
\end{center}
%
A simple redirection |\childdocforward{|\textit{dest}|}| is achieved by:
%
\begin{center}
|\def\jobname{|\textit{dest}|}\input{\jobname}|
\end{center}
%
The redirection with prefix
|\childdocforwardprefix[|\textit{prefix}|]{|\textit{dest}|}|
is accomplished by:
%
\begin{center}
\begin{tabular}{l}
|{\edef\jobname{\scantokens\expandafter{\jobname\noexpand}}|\\
|\def\redirectjob |\textit{prefix}|#1~~~{\gdef\jobname{|\textit{dest}|#1}}|\\
|\expandafter\redirectjob\jobname~~~}\input{\jobname}|
\end{tabular}
\end{center}

In an alternative approach,
child documents can be compiled by a specific command line
without additional code or specific definitions:
%
\begin{center}
|... -jobname "|\textit{target}|" "|[\textit{flags}]%
|\includeonly{|\textit{dest}|}\input{|\textit{main}|}"|
\end{center}
%

%%%%%%%%%%%%%%%%%%%%%%%%%%%%%%%%%%%%%%%%%%%%%%%%%%%%%%%%%%%%%%%%%%%%%%%%%%%%%%%%
%%%%%%%%%%%%%%%%%%%%%%%%%%%%%%%%%%%%%%%%%%%%%%%%%%%%%%%%%%%%%%%%%%%%%%%%%%%%%%%%
\section{Information}

%%%%%%%%%%%%%%%%%%%%%%%%%%%%%%%%%%%%%%%%%%%%%%%%%%%%%%%%%%%%%%%%%%%%%%%%%%%%%%%%
\subsection{Copyright}

Copyright \copyright{} 2017--2018 Niklas Beisert

This work may be distributed and/or modified under the
conditions of the \LaTeX{} Project Public License, either version 1.3
of this license or (at your option) any later version.
The latest version of this license is in
  \url{http://www.latex-project.org/lppl.txt}
and version 1.3 or later is part of all distributions of \LaTeX{}
version 2005/12/01 or later.

This work has the LPPL maintenance status `maintained'.

The Current Maintainer of this work is Niklas Beisert.

This work consists of the files |README.txt|, |childdoc.ins| and |childdoc.dtx|
as well as the derived files |childdoc.def|, |cdocsamp.tex|
with |cdocsch1.tex|, |cdocsch2.tex|, |cdocspt3.tex|, |cdocspt4.tex|,
|cdocsdrf.tex|, |cdocsfn1.tex|, |cdocsfn2.tex|
as well as |childdoc.pdf|.

%%%%%%%%%%%%%%%%%%%%%%%%%%%%%%%%%%%%%%%%%%%%%%%%%%%%%%%%%%%%%%%%%%%%%%%%%%%%%%%%
\subsection{Files and Installation}

The package consists of the files:
%
\begin{center}
\begin{tabular}{ll}
    |README.txt|   & readme file \\
    |childdoc.ins| & installation file \\
    |childdoc.dtx| & source file \\
    |childdoc.def| & definition file \\
    |cdocsamp.tex| & sample main file \\
    |cdocsch1.tex| & sample include file \\
    |cdocsch2.tex| & sample include file \\
    |cdocspt3.tex| & sample part file \\
    |cdocspt4.tex| & sample part file \\
    |cdocsdrf.tex| & sample redirection file \\
    |cdocsfn1.tex| & sample redirection file \\
    |cdocsfn2.tex| & sample redirection file \\
    |childdoc.pdf| & manual
\end{tabular}
\end{center}
%
The distribution consists of the files
|README.txt|, |childdoc.ins| and |childdoc.dtx|.
%
\begin{itemize}
\item
Run (pdf)\LaTeX{} on |childdoc.dtx|
to compile the manual |childdoc.pdf| (this file).
\item
Run \LaTeX{} on |childdoc.ins| to create the definitions file |childdoc.def|
and the sample |cdocsamp.tex| with include files
|cdocsch1.tex|, |cdocsch2.tex|, |cdocspt3.tex|, |cdocspt4.tex|,
|cdocsdrf.tex|, |cdocsfn1.tex|, |cdocsfn2.tex|.
Then copy the file |childdoc.def| to an appropriate directory of your \LaTeX{}
distribution, e.g.\ \textit{texmf-root}|/tex/latex/childdoc|.
\end{itemize}

%%%%%%%%%%%%%%%%%%%%%%%%%%%%%%%%%%%%%%%%%%%%%%%%%%%%%%%%%%%%%%%%%%%%%%%%%%%%%%%%
\subsection{Related CTAN Packages}

There are several other packages which offer a similar functionality:
%
\begin{itemize}
\item
The packages
\href{http://ctan.org/pkg/docmute}{\textsf{docmute}},
\href{http://ctan.org/pkg/includex}{\textsf{includex}} and
\href{http://ctan.org/pkg/standalone}{\textsf{standalone}}
provide commands to include only the document body of
a child file thus allowing both files to be compiled individually.
\item
The packages \href{http://ctan.org/pkg/subdocs}{\textsf{subdocs}}
and \href{http://ctan.org/pkg/subfiles}{\textsf{subfiles}}
provide structures in which the main and child documents can be
encapsulated and allowing them to be compiled individually.
The inclusion mechanism is different from the conventional |\include|.
\item
The package \href{http://ctan.org/pkg/combine}{\textsf{combine}}
is an elaborate solution to combine several documents into one.
\end{itemize}
%
See also the CTAN topic \href{http://ctan.org/topic/subdocs}{\textsf{subdocs}}
for further related packages.
The present package differs from the above solutions in that
a document structure constructed with the conventional |\include| mechanism
just needs two extra commands at the top of every file
such that all constituent files can be compiled individually.

%%%%%%%%%%%%%%%%%%%%%%%%%%%%%%%%%%%%%%%%%%%%%%%%%%%%%%%%%%%%%%%%%%%%%%%%%%%%%%%%
%\subsection{Feature Suggestions}
%
%The following is a list of features which may be useful for future
%versions of this package:
%%
%\begin{itemize}
%\item
%\ldots
%\end{itemize}

%%%%%%%%%%%%%%%%%%%%%%%%%%%%%%%%%%%%%%%%%%%%%%%%%%%%%%%%%%%%%%%%%%%%%%%%%%%%%%%%
\subsection{Revision History}

%%%%%%%%%%%%%%%%%%%%%%%%%%%%%%%%%%%%%%%%
\paragraph{v2.0:} 2018/12/30

\begin{itemize}
\item
immediate forward processing
\item
added |\childdocby| mechanism
\item
manual restructured
\end{itemize}

%%%%%%%%%%%%%%%%%%%%%%%%%%%%%%%%%%%%%%%%
\paragraph{v1.6:} 2018/01/17

\begin{itemize}
\item
application for development of include files
\item
corrections to manual
\end{itemize}

%%%%%%%%%%%%%%%%%%%%%%%%%%%%%%%%%%%%%%%%
\paragraph{v1.5:} 2017/05/21

\begin{itemize}
\item
more complete structuring introduced
\item
|\childdocof| introduced
\item
|\childdoc| renamed to |\childdocmain|
\item
|\childredirect| renamed to |\childdocforward| and |\childdocforwardprefix|
and functionality expanded
\end{itemize}

%%%%%%%%%%%%%%%%%%%%%%%%%%%%%%%%%%%%%%%%
\paragraph{v1.0:} 2017/04/27

\begin{itemize}
\item
manual and install package
\item
first version published on CTAN
\end{itemize}

%%%%%%%%%%%%%%%%%%%%%%%%%%%%%%%%%%%%%%%%
\paragraph{v0.6:} 2017/04/26

\begin{itemize}
\item
redirection mechanism added
\end{itemize}

%%%%%%%%%%%%%%%%%%%%%%%%%%%%%%%%%%%%%%%%
\paragraph{v0.5:} 2017/04/26

\begin{itemize}
\item
functionality in definition file
\end{itemize}


%%%%%%%%%%%%%%%%%%%%%%%%%%%%%%%%%%%%%%%%%%%%%%%%%%%%%%%%%%%%%%%%%%%%%%%%%%%%%%%%
%%%%%%%%%%%%%%%%%%%%%%%%%%%%%%%%%%%%%%%%%%%%%%%%%%%%%%%%%%%%%%%%%%%%%%%%%%%%%%%%
%%%%%%%%%%%%%%%%%%%%%%%%%%%%%%%%%%%%%%%%%%%%%%%%%%%%%%%%%%%%%%%%%%%%%%%%%%%%%%%%
\appendix

\settowidth\MacroIndent{\rmfamily\scriptsize 000\ }

 \DocInput{childdoc.dtx}

\end{document}
%</driver>
% \fi
%
% %%%%%%%%%%%%%%%%%%%%%%%%%%%%%%%%%%%%%%%%%%%%%%%%%%%%%%%%%%%%%%%%%%%%%%%%%%%%%%
% %%%%%%%%%%%%%%%%%%%%%%%%%%%%%%%%%%%%%%%%%%%%%%%%%%%%%%%%%%%%%%%%%%%%%%%%%%%%%%
% \section{Sample}
%\iffalse
%<*samplemain>
%\fi
%
% The following presents a sample document
% with two chapters, two parts, a title page,
% a compile flag as well as three forwarding files to set the flag.
% It consists of eight |.tex| files:
% \begin{center}
% \begin{tabular}{ll}
% |cdocsamp.tex|&main file\\
% |cdocsch1.tex|&include file for chapter 1\\
% |cdocsch2.tex|&include file for chapter 2\\
% |cdocspt3.tex|&include file for part 3\\
% |cdocspt4.tex|&include file for part 4\\
% |cdocsdrf.tex|&forwarding file for main file in draft mode\\
% |cdocsfi1.tex|&forwarding file for final version of chapter 1\\
% |cdocsfi2.tex|&forwarding file for final version of chapter 2\\
% \end{tabular}
% \end{center}
% Each of the eight files can be compiled directly by the \LaTeX{} compiler.
%
% %%%%%%%%%%%%%%%%%%%%%%%%%%%%%%%%%%%%%%
% \paragraph{Main File.}
%
% The main file is called |cdocsamp.tex|.
%
% Load the \textsf{childdoc} definitions and
% declare the filename for the main document:
%    \begin{macrocode}
% \iffalse
%
% childdoc.dtx Copyright (C) 2017-2018 Niklas Beisert
%
% This work may be distributed and/or modified under the
% conditions of the LaTeX Project Public License, either version 1.3
% of this license or (at your option) any later version.
% The latest version of this license is in
%   http://www.latex-project.org/lppl.txt
% and version 1.3 or later is part of all distributions of LaTeX
% version 2005/12/01 or later.
%
% This work has the LPPL maintenance status `maintained'.
%
% The Current Maintainer of this work is Niklas Beisert.
%
% This work consists of the files childdoc.dtx and childdoc.ins
% and the derived files childdoc.def and cdocsamp.tex with
% cdocsch1.tex, cdocsch2.tex, cdocsdrf.tex, cdocsfn1.tex, cdocsfn2.tex.
%
%<package>\ifdefined\childdocmain\endinput\fi
%<package>\ProvidesFile{childdoc.def}[2018/12/30 v2.0 child document driver]
%<samplemain>\ProvidesFile{cdocsamp.tex}[2018/12/30 v2.0 sample for childdoc]
%<*driver>
%\ProvidesFile{childdoc.drv}[2018/12/30 v2.0 childdoc reference manual file]
\PassOptionsToClass{10pt,a4paper}{article}
\documentclass{ltxdoc}

\usepackage[margin=35mm]{geometry}
\usepackage{hyperref}
\usepackage{hyperxmp}
\usepackage[usenames]{color}

\hypersetup{colorlinks=true}
\hypersetup{pdfstartview=FitH}
\hypersetup{pdfpagemode=UseNone}
\hypersetup{pdfsource={}}
\hypersetup{pdflang={en-UK}}
\hypersetup{pdfcopyright={Copyright 2017-2018 Niklas Beisert.
  This work may be distributed and/or modified under the
  conditions of the LaTeX Project Public License, either version 1.3
  of this license or (at your option) any later version.}}
\hypersetup{pdflicenseurl={http://www.latex-project.org/lppl.txt}}
\hypersetup{pdfcontactaddress={ETH Zurich, ITP, HIT K,
  Wolfgang-Pauli-Strasse 27}}
\hypersetup{pdfcontactpostcode={8093}}
\hypersetup{pdfcontactcity={Zurich}}
\hypersetup{pdfcontactcountry={Switzerland}}
\hypersetup{pdfcontactemail={nbeisert@itp.phys.ethz.ch}}
\hypersetup{pdfcontacturl={http://people.phys.ethz.ch/\xmptilde nbeisert/}}

\newcommand{\secref}[1]{\hyperref[#1]{section \ref*{#1}}}

\parskip1ex
\parindent0pt
\let\olditemize\itemize
\def\itemize{\olditemize\parskip0pt}

\begin{document}

\title{The \textsf{childdoc} Package}
\hypersetup{pdftitle={The childdoc Package}}
\author{Niklas Beisert\\[2ex]
  Institut f\"ur Theoretische Physik\\
  Eidgen\"ossische Technische Hochschule Z\"urich\\
  Wolfgang-Pauli-Strasse 27, 8093 Z\"urich, Switzerland\\[1ex]
  \href{mailto:nbeisert@itp.phys.ethz.ch}
  {\texttt{nbeisert@itp.phys.ethz.ch}}}
\hypersetup{pdfauthor={Niklas Beisert}}
\hypersetup{pdfsubject={Manual for the LaTeX2e Package childdoc}}
\date{30 December 2018, \textsf{v2.0}}
\maketitle

\begin{abstract}\noindent
\textsf{childdoc} is a \LaTeXe{} package
that enables the direct compilation
of document sections included by |\include|
to individual files.
\end{abstract}

\begingroup
\parskip0ex
\tableofcontents
\endgroup

%%%%%%%%%%%%%%%%%%%%%%%%%%%%%%%%%%%%%%%%%%%%%%%%%%%%%%%%%%%%%%%%%%%%%%%%%%%%%%%%
%%%%%%%%%%%%%%%%%%%%%%%%%%%%%%%%%%%%%%%%%%%%%%%%%%%%%%%%%%%%%%%%%%%%%%%%%%%%%%%%
\section{Introduction}

\LaTeX{} provides a mechanism to structure a large document (such as a book)
into a main file and several child files (containing the chapters)
using the |\include| command.
This mechanism is beneficial for documents
which span hundreds of pages in order to
make the source file(s) more manageable.
Moreover, compilation can be restricted to
selected child files by means of the |\includeonly| command.
The latter feature can be used to reduce the compilation time while editing
(this was significantly more useful in the earlier days of \LaTeX{})
or to generate a smaller document which is easier to navigate.
Another application of |\includeonly| is to generate
documents consisting of selected parts of the complete document.

However, there are a few drawbacks of the plain |\include| mechanism:
\begin{itemize}
\item
The child files cannot be compiled on their own,
they can only be compiled via the main file.
A naive editing environment
(such as a text editor with an option
to have the current file processed by \LaTeX)
may require one to switch to the main file before compiling;
attempting to compile the child file produces errors.
\item
The main file must be modified (each time)
to adjust the |\includeonly| command
to the present needs. This easily leaves the main file in a messy state.
\item
The generated document will always carry the filename
of the main document. This is inconvenient if
several child files are to be compiled and
to be kept for distribution.
\end{itemize}

The present package provides a simple interface
to make child files individually compilable by \LaTeX{}.
Compiling a child file then has the same effect as compiling
the main file with an |\includeonly| command
to select the appropriate child.
Moreover the generated document will carry the name of the child
rather than the main file.
This resolves all three above issues.

This feature is meant to make the editing of books,
thesis documents and lecture notes somewhat more convenient.
However, the package can also be used efficiently for
composing a series of documents (such as exercise sheets)
which are typically distributed individually.
It then assists the author in generating the individual documents
(potentially in different versions)
as well as a document containing the collected series.
Another application is in developing style files
or other kinds of included material
where compilation of the style file could redirect
to a sample or test file.

%%%%%%%%%%%%%%%%%%%%%%%%%%%%%%%%%%%%%%%%%%%%%%%%%%%%%%%%%%%%%%%%%%%%%%%%%%%%%%%%
%%%%%%%%%%%%%%%%%%%%%%%%%%%%%%%%%%%%%%%%%%%%%%%%%%%%%%%%%%%%%%%%%%%%%%%%%%%%%%%%
\section{Usage}

First of all, the package \textsf{childdoc} is \emph{not} a standard
\LaTeXe{} |.sty| style file! Therefore it needs to be invoked in
a non-standard way.

%%%%%%%%%%%%%%%%%%%%%%%%%%%%%%%%%%%%%%%%%%%%%%%%%%%%%%%%%%%%%%%%%%%%%%%%%%%%%%%%
\subsection{Included Files}
\label{sec:include}

%%%%%%%%%%%%%%%%%%%%%%%%%%%%%%%%%%%%%%%%
\DescribeMacro{\childdocmain}
To use the package, add the commands
\begin{center}
\begin{tabular}{l}
|\input{childdoc.def}|\\
|\childdocmain{}|\\
\end{tabular}
\end{center}
at the very top of the main \LaTeX{} file,
in particular \emph{before} the |\documentclass| statement!
The argument of |\childdocmain| should be left empty
(but it must be present).

%%%%%%%%%%%%%%%%%%%%%%%%%%%%%%%%%%%%%%%%
\DescribeMacro{\childdocof}
Furthermore, add the commands
\begin{center}
\begin{tabular}{l}
|\input{childdoc.def}|\\
|\childdocof{|\textit{main}|}|\\
\end{tabular}
\end{center}
at the top of every child file \textit{child}
which is included by |\include{|\textit{child}|}|
from within the main file
(or at least for those files to be compiled individually).
The argument \textit{main} must be the filename of the main file.

There are a couple of
considerations in setting up the main and child documents:

%%%%%%%%%%%%%%%%%%%%%%%%%%%%%%%%%%%%%%%%
\paragraph{Restrictions.}

Please note the following restrictions:
\begin{itemize}
\item
|\childdocmain| must be called with one argument \textit{main}
to ensure compatibility with earlier version of the package.
It must either be empty (|\childdocmain{}|)
or precisely match the filename of the main file in which it is specified.
See \secref{sec:detection} for further information.
\item
The filename \textit{main} must be specified without the |.tex| extension.
\item
The filename \textit{main} is case sensitive
(even in case-insensitive file systems)
due to internal string comparison.
\item
The argument \textit{main} should be fully expanded, it cannot be a macro.
\item
Subdirectories and special characters should be avoided in filenames.
\item
The command |\childdocmain{|\textit{main}|}| must be followed by a whitespace.
It should not be followed immediately by another command
or by a comment mark `|%|'.
This is because the \TeX{} parser reads the token immediately following
the argument of |\childdocmain| and puts it
at the beginning of every child section;
however, a white\-space is ignored.
\end{itemize}

%%%%%%%%%%%%%%%%%%%%%%%%%%%%%%%%%%%%%%%%
\paragraph{Content of Main File.}

It is advisable to place all content in the child files included by |\include|.
Any output contained in the main file will appear in all child documents
unless suppressed manually;
it cannot be suppressed automatically by the |\includeonly| directive
and thus should normally be avoided.
A method to include some content in the main file
by means of conditional processing is described in \secref{sec:conditional}.

%%%%%%%%%%%%%%%%%%%%%%%%%%%%%%%%%%%%%%%%
\paragraph{Page Numbering.}

When only a part of the document is compiled,
the appropriate numbering of pages
(as well as other status parameters)
is determined from the |.aux| files.
The latter contain information from previous passes.
However this information needs to propagate through
all intermediate child documents.
Therefore the page numbering in child documents may well
be inconsistent until the complete document is compiled at least once.

A useful (if unconventional) way to always ensure a consistent
page numbering is to restart the numbering in each child document
and denote the pages by `\textit{child}|.|\textit{page}'
where \textit{child} represents the chapter/section number of the child file.
This can be achieved by the command
|\numberwithin{page}{|\textit{child}|}|
of the \textsf{amsmath} package
where \textit{child} can be |chapter| or |section|
depending on the chosen structuring.
Alternatively, one can modify the macro |\thepage| appropriately
and reset the counter |page| at the start of each child file.

%%%%%%%%%%%%%%%%%%%%%%%%%%%%%%%%%%%%%%%%%%%%%%%%%%%%%%%%%%%%%%%%%%%%%%%%%%%%%%%%
\subsection{Conditional Processing}
\label{sec:conditional}

The package provides a mechanism to compile different versions
of a document. To customise the versions further some conditional processing
can come in handy to distinguish which version is being compiled.
The package provides two macros to describe the compilation context:

%%%%%%%%%%%%%%%%%%%%%%%%%%%%%%%%%%%%%%%%
\DescribeMacro{\ifchilddoc}
The conditional |\ifchilddoc| distinguishes between the compilation of
child documents and the main document:
%
\begin{center}
|\ifchilddoc |\textit{child-code}| |[|\||else |\textit{main-code}]| \||fi|
\end{center}

%%%%%%%%%%%%%%%%%%%%%%%%%%%%%%%%%%%%%%%%
\DescribeMacro{\childdocname}
\DescribeMacro{\childdocjob}
The macro |\childdocname| contains the filename (without extension)
of the main or child file being processed.
Note that |\childdocjob| will always contain the name of the main file.

%%%%%%%%%%%%%%%%%%%%%%%%%%%%%%%%%%%%%%%%
\paragraph{Title Page.}

Conditional processing can be used to include a title or banner page
in the main document when proper precautions are taken.
Importantly, the code in the main file should ensure that the page counter
(as well as other status parameters which are stored in the |.aux| files)
takes the same value after the conditional processing.
Otherwise the page numbers may take divergent values
depending on which part is compiled.

For example, a title page could be declared by:
%
\begin{center}
\begin{tabular}{l}
|\ifchilddoc\||else|\\
|\addtocounter{page}{-1}|\\
\textit{code for title page}\\
|\newpage|\\
|\||fi|
\end{tabular}
\end{center}
%
A banner page for the child documents can be generated by:
%
\begin{center}
\begin{tabular}{l}
|\ifchilddoc|\\
|\addtocounter{page}{-1}|\\
\textit{code for banner page}\\
|\newpage|\\
|\||fi|
\end{tabular}
\end{center}
%
Here one could write a message such as:
\begin{center}
|This is the part \childdocname{} of \childdocjob{}.|
\end{center}

%%%%%%%%%%%%%%%%%%%%%%%%%%%%%%%%%%%%%%%%%%%%%%%%%%%%%%%%%%%%%%%%%%%%%%%%%%%%%%%%
\subsection{Flags}
\label{sec:flags}

The package makes it easy to generate different versions
of the main or child documents.
To this end compilation flags can be defined
and assigned different default values.
They will be particularly useful in conjunction
with the forwarding mechanism described in \secref{sec:forward}.

For example, it may be useful to have a flag |\version|
which can be set to |draft| or |final|.
The document source will contain some conditional code
depending on the value of |\version|.
Suppose further, the flag should default to |final| for the main file
and to |draft| for child files
which is a natural assignment for editing the document.
This is achieved by placing the following code
in the preamble of the main document
(below the |\childdocmain| directive):
%
\begin{center}
\begin{tabular}{l}
|\ifchilddoc|\\
|\providecommand{\version}{draft}|\\
|\||else|\\
|\providecommand{\version}{final}|\\
|\||fi|
\end{tabular}
\end{center}
%
The definition by |\providecommand| makes sure
that previous definitions are not overwritten.
Further statements |\providecommand{\version}{...}|
can thus be added before the above code to override it.

For the main file, one might add a line
(between |\childdocmain| and the above block)
%
\begin{center}
|%\ifchilddoc\||else\providecommand{\version}{draft}\||fi|
\end{center}
%
which can be uncommented to produce a draft version.
Likewise one can add a line to the very top of a child file
(above the |\childdocof{|\textit{main}|}| directive)
%
\begin{center}
|%\providecommand{\version}{final}|
\end{center}
%
which can be uncommented to produce the final version of this child document.

%%%%%%%%%%%%%%%%%%%%%%%%%%%%%%%%%%%%%%%%%%%%%%%%%%%%%%%%%%%%%%%%%%%%%%%%%%%%%%%%
\subsection{Forwarding}
\label{sec:forward}

Different versions of the main or child documents
using compilation flags as described in \secref{sec:flags}
can be (permanently) stored in different files
for convenient compilation, viewing and distribution.
To this end, the package defines a command
to pass on compilation to a different file:

%%%%%%%%%%%%%%%%%%%%%%%%%%%%%%%%%%%%%%%%
\DescribeMacro{\childdocforward}
The command |\childdocforward| redirects processing to
another source file:
%
\begin{center}
\begin{tabular}{l}
|\input{childdoc.def}|\\
|\childdocforward[|\textit{main}|]{|\textit{dest}|}|\\
\end{tabular}
\end{center}
%
The argument \textit{dest} is the destination file
(without extension).
It should be the main file or one of the child files.
Note that further \textsf{childdoc} directives
such as |\childdocof| and |\childdocforward|
in the indicated file will be processed in this form.
The optional argument \textit{main}
passes on directly to the main file \textit{main}
while pretending to compile the child \textit{dest}.
This form behaves as if \textit{dest}
issues |\childdocof{|\textit{main}|}| right away,
and no further \textsf{childdoc} directives will be processed.

%%%%%%%%%%%%%%%%%%%%%%%%%%%%%%%%%%%%%%%%
\DescribeMacro{\...prefix}
In the alternative form |\childdocforwardprefix|,
%
\begin{center}
\begin{tabular}{l}
|\input{childdoc.def}|\\
|\childdocforwardprefix[|\textit{main}|]{|\textit{prefix}|}{|\textit{dest}|}|
\end{tabular}
\end{center}
%
the destination file is determined by a pattern
depending on the current file:
To make this work, the current file must be called
`{\textit{prefix}\hspace{0.2em}\textit{suffix}}'
with \textit{prefix} matching precisely the argument.
Processing is then passed on to the file
`{\textit{dest}\hspace{0.2em}\textit{suffix}}'.
Surely, the same effect is achieved by
directly specifying the
argument `{\textit{dest}\hspace{0.2em}\textit{suffix}}'
in the first form.
However, that requires to set up a different file
for each child. With the alternative form of the command
all these files can have exactly the same content
which simplifies setting them up and maintaining them.

For example, the following file |draft.tex|
with a compilation flag |\version| as described in \secref{sec:flags}
compiles the main document as a draft:
%
\begin{center}
\begin{tabular}{l}
|\def\version{draft}|\\
|\input{childdoc.def}|\\
|\childdocforward{|\textit{main}|}|
\end{tabular}
\end{center}
%
Likewise, the following files |final|\textit{nn}|.tex|
compile the final version of the child document
|child|\textit{nn}|.tex|:
%
\begin{center}
\begin{tabular}{l}
|\def\version{final}|\\
|\input{childdoc.def}|\\
|\childdocforwardprefix{final}{child}|
\end{tabular}
\end{center}
%

Note that when several versions of a main file and/or of each child file
are to be generated, it may be convenient to set up a |Makefile| or
shell script to automatise the process.

%%%%%%%%%%%%%%%%%%%%%%%%%%%%%%%%%%%%%%%%%%%%%%%%%%%%%%%%%%%%%%%%%%%%%%%%%%%%%%%%
\subsection{Command Line Processing}
\label{sec:commandline}

The effect of redirection files can also be achieved by invoking
the \LaTeX{} compiler with a more elaborate command line.
Most conveniently this should be done as part
of a shell script or a |Makefile|.

When using \textsf{childdoc} in the main file, the following
command lines effectively perform a redirection
(note that depending on the shell being used,
backslashes may have to be doubled: `|\|' $\to$ `|\\|'):
%
\begin{center}
|... -jobname "|\textit{target}|" |\\|"|[\textit{flags}]%
|\input{childdoc.def}\childdocforward[|\textit{main}|]{|\textit{dest}|}"|
\end{center}
%
Here \textit{target} is the name of the output file,
\textit{main} is the name of the main file
and \textit{dest} is the name of the main or child file to be processed
(all filenames without extensions).
The optional argument \textit{main} can be omitted
if \textit{main} matches \textit{dest}.
Optionally, compilation \textit{flags} can be defined via |\def| commands.
This command line makes the \TeX{} engine believe
it is compiling the file \textit{target}
whose content is specified as the latter parameter.
The provided code then forwards the processing to
\textit{main} or \textit{dest} as described in \secref{sec:forward}.

%%%%%%%%%%%%%%%%%%%%%%%%%%%%%%%%%%%%%%%%%%%%%%%%%%%%%%%%%%%%%%%%%%%%%%%%%%%%%%%%
\subsection{Include by Input}
\label{sec:input}

Including child documents by |\include| has some restrictions by design.
Most notably, the content of a child document always occupies
its own set of pages; pages cannot be shared between child documents.
Usually, this behaviour makes perfect sense
because each child document contain an essential part of the document.
However, in some situations it may be desirable to compose
a document from a collection of parts
without having mandatory page breaks between then.
For this case, the package
provides a mechanism to include parts
by |\input| which can also be processed individually.
However, by construction this mechanism
requires manual handling of the content to be output.

%%%%%%%%%%%%%%%%%%%%%%%%%%%%%%%%%%%%%%%%
\DescribeMacro{\ifchilddocmanual}
The main file should be prepared as usual, see \secref{sec:include}.
However, the document body must make a distinction
between processing of an individual part and of the main document, e.g.:
%
\begin{center}
\begin{tabular}{l}
|\ifchilddocmanual|\\
|\input{\childdocname}|\\
|\||else|\\
\textit{document body with }|\input{|\textit{part}|}|\\
|\||fi|
\end{tabular}
\end{center}
%
The conditional |\ifchilddocmanual| is true whenever
a part to be included by |\input| is being compiled,
and the name of the part is stored in |\childdocname|.

%%%%%%%%%%%%%%%%%%%%%%%%%%%%%%%%%%%%%%%%
\DescribeMacro{\childdocby}
Each part to be included by |\input| should start with:
%
\begin{center}
\begin{tabular}{l}
|\input{childdoc.def}|\\
|\childdocby{|\textit{main}|}|\\
\end{tabular}
\end{center}
%
The directive |\childdocby| is similar to |\childdocof|
described in \secref{sec:include},
but the subsequent selection of content must be done manually.
To that end, both |\ifchilddoc| and |\ifchilddocmanual|
will be true upon processing of a part,
and the name of the part is stored in |\childdocname|.
Note that |\jobname| will be set to the filename of the current part
so that each part receives an individual |.aux| file
that does not interfere with the |.aux| file(s) of the main document.
This behaviour can be altered by the alternative form
|\childdocby[*]{|\textit{main}|}| (with a non-empty optional argument)
which uses the |.aux| file of the main document
by setting |\jobname| to \textit{main}.

%%%%%%%%%%%%%%%%%%%%%%%%%%%%%%%%%%%%%%%%%%%%%%%%%%%%%%%%%%%%%%%%%%%%%%%%%%%%%%%%
\subsection{Driver Development}
\label{sec:driver}

The \textsf{childdoc} mechanism can also be use for the development
of definition files such as \LaTeX{} styles or classes.
This case differs from the above setup with multiple parts
included by |\include| in that no |\includeonly| should be invoked.
This can be achieved by starting the include file
(before |\ProvidesPackage|) with:
%
\begin{center}
\begin{tabular}{l}
|\input{childdoc.def}|\\
|\childdocforward{|\textit{main}|}|\\
\end{tabular}
\end{center}
%
or alternatively with:
%
\begin{center}
\begin{tabular}{l}
|\input{childdoc.def}|\\
|\childdocby{|\textit{main}|}|\\
\end{tabular}
\end{center}
%
Both forms have slightly different effects as described above.
The main file is prepared as usual, see \secref{sec:include}.

%%%%%%%%%%%%%%%%%%%%%%%%%%%%%%%%%%%%%%%%%%%%%%%%%%%%%%%%%%%%%%%%%%%%%%%%%%%%%%%%
\subsection{Legacy Detection}
\label{sec:detection}

The directive |\childdocmain| in the main file can detect
whether the complete document or merely a child is to be compiled
even without using the directive |\childdocof|.
This method is deprecated because it is less robust
and there is no compelling reason to use it;
it is merely provided for backward compatibility
and it may be removed in future versions.

If the detection mechanism is to be used,
it is mandatory to correctly specify
the filename of the main file as the argument of |\childdocmain|:
%
\begin{center}
\begin{tabular}{l}
|\input{childdoc.def}|\\
|\childdocmain{|\textit{main}|}|\\
\end{tabular}
\end{center}
%
If |\jobname| does not match the argument \textit{main} of |\childdocmain|,
it is assumed that |\jobname| points to the child file to be compiled.
When using |\childdocmain| with the main file specified as argument,
it suffices to start a child file
with just |\input{|\textit{main}|}|
without loading of the package and using |\childdocof|.
If instead all processing is done
with the appropriate \textsf{childdoc} directives,
the argument of \textit{main} of |\childdocmain| can be empty.

An alternative version of the command line processing described
in \secref{sec:commandline} using the detection mechanism reads:
%
\begin{center}
|... -jobname "|\textit{target}|" "|[\textit{flags}]%
[|\def\jobname{|\textit{dest}|}|]|\input{|\textit{main}|}"|
\end{center}

%%%%%%%%%%%%%%%%%%%%%%%%%%%%%%%%%%%%%%%%%%%%%%%%%%%%%%%%%%%%%%%%%%%%%%%%%%%%%%%%
\subsection{Manual Code}
\label{sec:manual}

In case one cannot be certain whether the definitions file |childdoc.def|
is installed on the target \TeX{} distribution
and one prefers not to ship it,
it is conceivable to paste a few relevant commands into the sources.

To that end, drop all statements |\input{childdoc.def}|
and perform the replacements as outlined below.
Instead of |\childdocmain{|\textit{main}|}| add the following code
to the top of the main file:
%
\begin{center}
\begin{tabular}{l}
|\||ifdefined\childdocname\endinput\||fi\newif\ifchilddoc|\\
|\edef\childdocname{\scantokens\expandafter{\jobname\noexpand}}|\\
|\def\childdocmain{|\textit{main}|}\||ifx\childdocmain\childdocname\||else|\\
|\childdoctrue\includeonly{\childdocname}\let\jobname\childdocmain\||fi|\\
\end{tabular}
\end{center}
%
Instead of |\childdocof{|\textit{main}|}| just include the main file
at the top of each child file:
%
\begin{center}
|\input{|\textit{main}|}|
\end{center}
%
A simple redirection |\childdocforward{|\textit{dest}|}| is achieved by:
%
\begin{center}
|\def\jobname{|\textit{dest}|}\input{\jobname}|
\end{center}
%
The redirection with prefix
|\childdocforwardprefix[|\textit{prefix}|]{|\textit{dest}|}|
is accomplished by:
%
\begin{center}
\begin{tabular}{l}
|{\edef\jobname{\scantokens\expandafter{\jobname\noexpand}}|\\
|\def\redirectjob |\textit{prefix}|#1~~~{\gdef\jobname{|\textit{dest}|#1}}|\\
|\expandafter\redirectjob\jobname~~~}\input{\jobname}|
\end{tabular}
\end{center}

In an alternative approach,
child documents can be compiled by a specific command line
without additional code or specific definitions:
%
\begin{center}
|... -jobname "|\textit{target}|" "|[\textit{flags}]%
|\includeonly{|\textit{dest}|}\input{|\textit{main}|}"|
\end{center}
%

%%%%%%%%%%%%%%%%%%%%%%%%%%%%%%%%%%%%%%%%%%%%%%%%%%%%%%%%%%%%%%%%%%%%%%%%%%%%%%%%
%%%%%%%%%%%%%%%%%%%%%%%%%%%%%%%%%%%%%%%%%%%%%%%%%%%%%%%%%%%%%%%%%%%%%%%%%%%%%%%%
\section{Information}

%%%%%%%%%%%%%%%%%%%%%%%%%%%%%%%%%%%%%%%%%%%%%%%%%%%%%%%%%%%%%%%%%%%%%%%%%%%%%%%%
\subsection{Copyright}

Copyright \copyright{} 2017--2018 Niklas Beisert

This work may be distributed and/or modified under the
conditions of the \LaTeX{} Project Public License, either version 1.3
of this license or (at your option) any later version.
The latest version of this license is in
  \url{http://www.latex-project.org/lppl.txt}
and version 1.3 or later is part of all distributions of \LaTeX{}
version 2005/12/01 or later.

This work has the LPPL maintenance status `maintained'.

The Current Maintainer of this work is Niklas Beisert.

This work consists of the files |README.txt|, |childdoc.ins| and |childdoc.dtx|
as well as the derived files |childdoc.def|, |cdocsamp.tex|
with |cdocsch1.tex|, |cdocsch2.tex|, |cdocspt3.tex|, |cdocspt4.tex|,
|cdocsdrf.tex|, |cdocsfn1.tex|, |cdocsfn2.tex|
as well as |childdoc.pdf|.

%%%%%%%%%%%%%%%%%%%%%%%%%%%%%%%%%%%%%%%%%%%%%%%%%%%%%%%%%%%%%%%%%%%%%%%%%%%%%%%%
\subsection{Files and Installation}

The package consists of the files:
%
\begin{center}
\begin{tabular}{ll}
    |README.txt|   & readme file \\
    |childdoc.ins| & installation file \\
    |childdoc.dtx| & source file \\
    |childdoc.def| & definition file \\
    |cdocsamp.tex| & sample main file \\
    |cdocsch1.tex| & sample include file \\
    |cdocsch2.tex| & sample include file \\
    |cdocspt3.tex| & sample part file \\
    |cdocspt4.tex| & sample part file \\
    |cdocsdrf.tex| & sample redirection file \\
    |cdocsfn1.tex| & sample redirection file \\
    |cdocsfn2.tex| & sample redirection file \\
    |childdoc.pdf| & manual
\end{tabular}
\end{center}
%
The distribution consists of the files
|README.txt|, |childdoc.ins| and |childdoc.dtx|.
%
\begin{itemize}
\item
Run (pdf)\LaTeX{} on |childdoc.dtx|
to compile the manual |childdoc.pdf| (this file).
\item
Run \LaTeX{} on |childdoc.ins| to create the definitions file |childdoc.def|
and the sample |cdocsamp.tex| with include files
|cdocsch1.tex|, |cdocsch2.tex|, |cdocspt3.tex|, |cdocspt4.tex|,
|cdocsdrf.tex|, |cdocsfn1.tex|, |cdocsfn2.tex|.
Then copy the file |childdoc.def| to an appropriate directory of your \LaTeX{}
distribution, e.g.\ \textit{texmf-root}|/tex/latex/childdoc|.
\end{itemize}

%%%%%%%%%%%%%%%%%%%%%%%%%%%%%%%%%%%%%%%%%%%%%%%%%%%%%%%%%%%%%%%%%%%%%%%%%%%%%%%%
\subsection{Related CTAN Packages}

There are several other packages which offer a similar functionality:
%
\begin{itemize}
\item
The packages
\href{http://ctan.org/pkg/docmute}{\textsf{docmute}},
\href{http://ctan.org/pkg/includex}{\textsf{includex}} and
\href{http://ctan.org/pkg/standalone}{\textsf{standalone}}
provide commands to include only the document body of
a child file thus allowing both files to be compiled individually.
\item
The packages \href{http://ctan.org/pkg/subdocs}{\textsf{subdocs}}
and \href{http://ctan.org/pkg/subfiles}{\textsf{subfiles}}
provide structures in which the main and child documents can be
encapsulated and allowing them to be compiled individually.
The inclusion mechanism is different from the conventional |\include|.
\item
The package \href{http://ctan.org/pkg/combine}{\textsf{combine}}
is an elaborate solution to combine several documents into one.
\end{itemize}
%
See also the CTAN topic \href{http://ctan.org/topic/subdocs}{\textsf{subdocs}}
for further related packages.
The present package differs from the above solutions in that
a document structure constructed with the conventional |\include| mechanism
just needs two extra commands at the top of every file
such that all constituent files can be compiled individually.

%%%%%%%%%%%%%%%%%%%%%%%%%%%%%%%%%%%%%%%%%%%%%%%%%%%%%%%%%%%%%%%%%%%%%%%%%%%%%%%%
%\subsection{Feature Suggestions}
%
%The following is a list of features which may be useful for future
%versions of this package:
%%
%\begin{itemize}
%\item
%\ldots
%\end{itemize}

%%%%%%%%%%%%%%%%%%%%%%%%%%%%%%%%%%%%%%%%%%%%%%%%%%%%%%%%%%%%%%%%%%%%%%%%%%%%%%%%
\subsection{Revision History}

%%%%%%%%%%%%%%%%%%%%%%%%%%%%%%%%%%%%%%%%
\paragraph{v2.0:} 2018/12/30

\begin{itemize}
\item
immediate forward processing
\item
added |\childdocby| mechanism
\item
manual restructured
\end{itemize}

%%%%%%%%%%%%%%%%%%%%%%%%%%%%%%%%%%%%%%%%
\paragraph{v1.6:} 2018/01/17

\begin{itemize}
\item
application for development of include files
\item
corrections to manual
\end{itemize}

%%%%%%%%%%%%%%%%%%%%%%%%%%%%%%%%%%%%%%%%
\paragraph{v1.5:} 2017/05/21

\begin{itemize}
\item
more complete structuring introduced
\item
|\childdocof| introduced
\item
|\childdoc| renamed to |\childdocmain|
\item
|\childredirect| renamed to |\childdocforward| and |\childdocforwardprefix|
and functionality expanded
\end{itemize}

%%%%%%%%%%%%%%%%%%%%%%%%%%%%%%%%%%%%%%%%
\paragraph{v1.0:} 2017/04/27

\begin{itemize}
\item
manual and install package
\item
first version published on CTAN
\end{itemize}

%%%%%%%%%%%%%%%%%%%%%%%%%%%%%%%%%%%%%%%%
\paragraph{v0.6:} 2017/04/26

\begin{itemize}
\item
redirection mechanism added
\end{itemize}

%%%%%%%%%%%%%%%%%%%%%%%%%%%%%%%%%%%%%%%%
\paragraph{v0.5:} 2017/04/26

\begin{itemize}
\item
functionality in definition file
\end{itemize}


%%%%%%%%%%%%%%%%%%%%%%%%%%%%%%%%%%%%%%%%%%%%%%%%%%%%%%%%%%%%%%%%%%%%%%%%%%%%%%%%
%%%%%%%%%%%%%%%%%%%%%%%%%%%%%%%%%%%%%%%%%%%%%%%%%%%%%%%%%%%%%%%%%%%%%%%%%%%%%%%%
%%%%%%%%%%%%%%%%%%%%%%%%%%%%%%%%%%%%%%%%%%%%%%%%%%%%%%%%%%%%%%%%%%%%%%%%%%%%%%%%
\appendix

\settowidth\MacroIndent{\rmfamily\scriptsize 000\ }

 \DocInput{childdoc.dtx}

\end{document}
%</driver>
% \fi
%
% %%%%%%%%%%%%%%%%%%%%%%%%%%%%%%%%%%%%%%%%%%%%%%%%%%%%%%%%%%%%%%%%%%%%%%%%%%%%%%
% %%%%%%%%%%%%%%%%%%%%%%%%%%%%%%%%%%%%%%%%%%%%%%%%%%%%%%%%%%%%%%%%%%%%%%%%%%%%%%
% \section{Sample}
%\iffalse
%<*samplemain>
%\fi
%
% The following presents a sample document
% with two chapters, two parts, a title page,
% a compile flag as well as three forwarding files to set the flag.
% It consists of eight |.tex| files:
% \begin{center}
% \begin{tabular}{ll}
% |cdocsamp.tex|&main file\\
% |cdocsch1.tex|&include file for chapter 1\\
% |cdocsch2.tex|&include file for chapter 2\\
% |cdocspt3.tex|&include file for part 3\\
% |cdocspt4.tex|&include file for part 4\\
% |cdocsdrf.tex|&forwarding file for main file in draft mode\\
% |cdocsfi1.tex|&forwarding file for final version of chapter 1\\
% |cdocsfi2.tex|&forwarding file for final version of chapter 2\\
% \end{tabular}
% \end{center}
% Each of the eight files can be compiled directly by the \LaTeX{} compiler.
%
% %%%%%%%%%%%%%%%%%%%%%%%%%%%%%%%%%%%%%%
% \paragraph{Main File.}
%
% The main file is called |cdocsamp.tex|.
%
% Load the \textsf{childdoc} definitions and
% declare the filename for the main document:
%    \begin{macrocode}
\input{childdoc.def}
\childdocmain{}
%    \end{macrocode}

% Optional override for |\version| flag:
%    \begin{macrocode}
%%\ifchilddoc\else\providecommand{\version}{draft}\fi
%    \end{macrocode}

% Define the default values for the |\version| flag
% (|final| for the main file and |draft| for childs):
%    \begin{macrocode}
\ifchilddoc
\providecommand{\version}{draft}
\else
\providecommand{\version}{final}
\fi
%    \end{macrocode}

% Load the standard document class:
%    \begin{macrocode}
\documentclass[12pt]{article}
%    \end{macrocode}

% Start the document body:
%    \begin{macrocode}
\begin{document}
%    \end{macrocode}

% Declare a title page.
% Print title, part of document being processed and version flag:
%    \begin{macrocode}
\addtocounter{page}{-1}
\begin{center}
{\LARGE\bfseries{}childdoc example\par}
\vspace{1cm}
\ifchilddoc
\ifchilddocmanual part\else chapter\fi:
`\childdocname' of `\childdocjob'\par
\else
main document: `\childdocjob'\par
\fi
version: \version\par
\end{center}
\newpage
%    \end{macrocode}

% Manually include selected file,
% otherwise process as usual:
%    \begin{macrocode}
\ifchilddocmanual
\section*{part `\childdocname'}
\input{\childdocname}
\else
%    \end{macrocode}

% Include the two chapters:
%    \begin{macrocode}
\include{cdocsch1}
\include{cdocsch2}
%    \end{macrocode}

% Include the two parts unless only chapters should be displayed:
%    \begin{macrocode}
\ifchilddoc\else
\section{part three}
\input{cdocspt3}
\section{part four}
\input{cdocspt4}
\fi
%    \end{macrocode}

% Process as usual until here:
%    \begin{macrocode}
\fi
%    \end{macrocode}

% End of document body:
%    \begin{macrocode}
\end{document}
%    \end{macrocode}
%\iffalse
%</samplemain>
%\fi
%
% %%%%%%%%%%%%%%%%%%%%%%%%%%%%%%%%%%%%%%
% \paragraph{Chapter Include Files.}
%
% The include files are called |cdocsch1.tex| and |cdocsch2.tex|.
%
%\iffalse
%<*samplechap1|samplechap2>
%\fi

% Optional override for |\version| flag:
%    \begin{macrocode}
%%\providecommand{\version}{final}
%    \end{macrocode}

% Include the main document:
%    \begin{macrocode}
\input{childdoc.def}
\childdocof{cdocsamp}
%    \end{macrocode}

%\iffalse
%</samplechap1|samplechap2>
%\fi
%
%\iffalse
%<*samplechap1>
%\fi
% Some text for chapter 1:
%    \begin{macrocode}
\section{one}
some text in chapter one
%    \end{macrocode}

%\iffalse
%</samplechap1>
%\fi
% Some text for chapter 2:
%\iffalse
%<*samplechap2>
%\fi
%    \begin{macrocode}
\section{two}
more text in chapter two
%    \end{macrocode}

%\iffalse
%</samplechap2>
%\fi
%
% %%%%%%%%%%%%%%%%%%%%%%%%%%%%%%%%%%%%%%
% \paragraph{Part Include Files.}
%
% The include files are called |cdocspt3.tex| and |cdocspt4.tex|.
%
%\iffalse
%<*samplepart3|samplepart4>
%\fi

% Optional override for |\version| flag:
%    \begin{macrocode}
%%\providecommand{\version}{final}
%    \end{macrocode}

% Include the main document:
%    \begin{macrocode}
\input{childdoc.def}
\childdocby{cdocsamp}
%    \end{macrocode}

%\iffalse
%</samplepart3|samplepart4>
%\fi
%
%\iffalse
%<*samplepart3>
%\fi
% Some text for part 3:
%    \begin{macrocode}
some text in part three
%    \end{macrocode}

%\iffalse
%</samplepart3>
%\fi
% Some text for part 4:
%\iffalse
%<*samplepart4>
%\fi
%    \begin{macrocode}
more text in part four
%    \end{macrocode}

%\iffalse
%</samplepart4>
%\fi
%
% %%%%%%%%%%%%%%%%%%%%%%%%%%%%%%%%%%%%%%
% \paragraph{Forwarding for a Complete Draft.}
%
% The following forwarding file |cdocsdrf.tex|
% compiles the main document in draft mode:
%\iffalse
%<*sampledraft>
%\fi
%    \begin{macrocode}
\def\version{draft}
\input{childdoc.def}
\childdocforward{cdocsamp}
%    \end{macrocode}

%\iffalse
%</sampledraft>
%\fi
%
% %%%%%%%%%%%%%%%%%%%%%%%%%%%%%%%%%%%%%%
% \paragraph{Forwarding for Final Version of the Chapters.}
%
% The following forwarding files |cdocsfn1.tex| and |cdocsfn2.tex|
% (with identical content)
% compile the final versions of the child documents
% |cdocsch1.tex| and |cdocsch2.tex|, respectively:
%\iffalse
%<*samplefinal>
%\fi
%    \begin{macrocode}
\def\version{final}
\input{childdoc.def}
\childdocforwardprefix[cdocsamp]{cdocsfn}{cdocsch}
%    \end{macrocode}

%\iffalse
%</samplefinal>
%\fi
%
% %%%%%%%%%%%%%%%%%%%%%%%%%%%%%%%%%%%%%%
% \paragraph{Command Line Processing.}
%
% The following three command lines generate the output files
% |cdocscld|, |cdocscl1| and |cdocscl2|
% which should be identical to
% |cdocsdrf|, |cdocsch1| and |cdocsfn2|, respectively:
% \begin{center}
% \begin{tabular}{l}
% |latex -jobname cdocscld \|\\
% |  "\def\version{draft}\input{childdoc.def}\childdocforward{cdocsamp}"|\\
% |latex -jobname cdocscl1 \|\\
% |  "\input{childdoc.def}\childdocforward[cdocsamp]{cdocsch1}"|\\
% |latex -jobname cdocscl2 \|\\
% |  "\def\version{final}\input{childdoc.def}\childdocforward{cdocsch2}"|
% \end{tabular}
% \end{center}
% Note that the trailing backslash on each first line
% merely continues the input to the second line
% (for convenient cut ant paste).
% Furthermore, the command |latex| can be replaced by any
% of its alternative versions such as |pdflatex|.
%
% %%%%%%%%%%%%%%%%%%%%%%%%%%%%%%%%%%%%%%%%%%%%%%%%%%%%%%%%%%%%%%%%%%%%%%%%%%%%%%
% %%%%%%%%%%%%%%%%%%%%%%%%%%%%%%%%%%%%%%%%%%%%%%%%%%%%%%%%%%%%%%%%%%%%%%%%%%%%%%
% \section{Implementation}
%\iffalse
%<*package>
%\fi
%
% This section describes the definitions file |childdoc.def|.

% The definitions cannot be loaded using |\usepackage| or |\RequirePackage|
% which has a mechanism to prevent loading a style file more than once.
% When loading the definitions by means of |\input|
% multiple instances have to be prevented manually:
%\iffalse
%This code needs to be before the `\ProvidesFile' directive
%which is defined at the beginning of this file.
%Therefore it is also placed there and commented out here.
%</package>
%<*discard>
%\fi
%    \begin{macrocode}
\ifdefined\childdocmain\endinput\fi
%    \end{macrocode}
%\iffalse
%</discard>
%<*package>
%\fi
%
% \macro{\ifchilddoc}
% \macro{\ifchilddocmanual}
% The conditional |\ifchilddoc| tells whether a
% child (true) or main (false) document is being compiled.
% The conditional |\ifchilddocmanual| tells whether
% the |\includeonly| mechanism is used (false) or
% the selection of child files must be performed manually (true).
% The definitions initialise to false:
%    \begin{macrocode}
\newif\ifchilddoc
\newif\ifchilddocmanual
%    \end{macrocode}

% \macro{\childdocname}
% \macro{\childdocjob}
% The macro |\childdocname| stores the name of the main document
% to be compiled. The macro |\childdocjob| stores the name of
% the document on which the \LaTeX{} compiler was originally invoked.
% The content of |\jobname| cannot be compared
% to filenames specified in the source due to different catcodes.
% The following code rescans |\jobname|, stores the result
% in |\childdocname| and saves a copy in |\childdocjob|:
%    \begin{macrocode}
\edef\childdocname{\scantokens\expandafter{\jobname\noexpand}}
\let\childdocjob\childdocname
%    \end{macrocode}

% \macro{\childdocdisable}
% The macro |\childdocdisable| prevents the main file
% from being processed more than once.
% At this stage, the main document command |\childdocmain|
% is assumed to be called once again where it should do nothing.
% Any subsequent call to it should prevent
% a secondary processing of the main document
% It overwrites the forwarding commands
% |\childdocof| and |\childdocforward|
% with empty macros to prevent further inclusions of the main document:
%    \begin{macrocode}
\newcommand{\childdocdisable}
{
  \renewcommand{\childdocmain}[1]{\renewcommand{\childdocmain}[1]{\endinput}}
  \renewcommand{\childdocof}[1]{}
  \renewcommand{\childdocby}[2][]{}
  \renewcommand{\childdocforward}[2][]{}
  \renewcommand{\childdocdisable}{}
}
%    \end{macrocode}

% \macro{\childdocmain}
% The macro |\childdocmain| is to be called at the top of the main file
% with nothing or the main filename (without extension) as argument.
% First, it breaks loops.
% If the argument is not empty and does not match |\childdocname|
% (which is set by the first inclusion of |childdoc.def|),
% |\ifchilddoc| is set to true, |\includeonly| is applied to the child file
% and |\jobname| is set to the main file
% (for proper handling of |.aux| files):
%    \begin{macrocode}
\newcommand{\childdocmain}[1]
{
  \childdocdisable\childdocmain{}
  \if?#1?\else
    \begingroup
      \def\childdoctmp{#1}
      \ifx\childdoctmp\childdocname
        \def\childdoctmp{}
      \else
        \def\childdoctmp
        {
          \childdoctrue
          \includeonly{\childdocname}
          \def\childdocjob{#1}
          \def\jobname{#1}
        }
      \fi
      \expandafter
    \endgroup
    \childdoctmp
  \fi
}
%    \end{macrocode}

% \macro{\childdocof}
% The command |\childdocof| redirects
% compilation to the main file |#1|.
%    \begin{macrocode}
\newcommand{\childdocof}[1]
{
  \childdocdisable
  \childdoctrue
  \includeonly{\childdocname}
  \def\jobname{#1}
  \def\childdocjob{#1}
  \input{#1}
}
%    \end{macrocode}

% \macro{\childdocby}
% The command |\childdocby| ....
%    \begin{macrocode}
\newcommand{\childdocby}[2][]
{
  \childdocdisable
  \childdoctrue
  \childdocmanualtrue
  \if?#1?\else
    \def\jobname{#2}
  \fi
  \def\childdocjob{#2}
  \input{#2}
  \endinput
}
%    \end{macrocode}

% \macro{\childdocforward}
% The command |\childdocforward| redirects
% compilation to the main file or
% (if the optional argument is given) a child file.
% Parameters are set as if the main file
% or a child file starting with |\childdocof| was compiled.
% Then compilation is handed over to the main file:
%    \begin{macrocode}
\newcommand{\childdocforward}[2][]
{
  \begingroup
    \if?#1?
      \def\childdoctmp
      {
        \def\childdocname{#2}
        \def\childdocjob{#2}
        \def\jobname{#2}
        \input{#2}
        \endinput
      }
    \else
      \def\childdoctmp
      {
        \childdocdisable
        \def\childdocname{#2}
        \childdoctrue
        \includeonly{#2}
        \def\childdocjob{#1}
        \def\jobname{#1}
        \input{#1}
        \endinput
      }
    \fi
    \expandafter
  \endgroup
  \childdoctmp
}
%    \end{macrocode}

% \macro{\childdocforwardprefix}
% The command |\childdocforwardprefix| redirects
% compilation to the main or a child file by means of a pattern.
% The prefix |#1| in the current filename is replaced by |#2|
% and the suffix of the current filename is kept
% (it is assumed that the filename does not contain the substring `|~~~|'
% which is used as a delimiter).
% Compilation is handed over to the new file by |\childdocforward|:
%    \begin{macrocode}
\newcommand{\childdocforwardprefix}[3][]
{
  \begingroup
    \def\childdocextract #2##1~~~{\def\childdoctmp{\childdocforward[#1]{#3##1}}}
    \expandafter\childdocextract\childdocname~~~
    \expandafter
  \endgroup
  \childdoctmp
}
%    \end{macrocode}

% \macro{\childdoc}
% The deprecated macro |\childdoc| is a legacy version of |\childdocmain|:
%    \begin{macrocode}
\newcommand{\childdoc}{\childdocmain}
%    \end{macrocode}

% \macro{\childdocredirect}
% The deprecated macro |\childdocredirect| is a legacy version
% of |\childdocforward| and |\childdocforwardprefix|:
%    \begin{macrocode}
\newcommand{\childdocredirect}[2][]
{
  \begingroup
    \if?#1?
      \def\childdoctmp{\childdocforward{#2}}
    \else
      \def\childdoctmp{\childdocforwardprefix{#1}{#2}}
    \fi
    \expandafter
  \endgroup
  \childdoctmp
}
%    \end{macrocode}

%\iffalse
%</package>
%\fi
%
\endinput

\childdocmain{}
%    \end{macrocode}

% Optional override for |\version| flag:
%    \begin{macrocode}
%%\ifchilddoc\else\providecommand{\version}{draft}\fi
%    \end{macrocode}

% Define the default values for the |\version| flag
% (|final| for the main file and |draft| for childs):
%    \begin{macrocode}
\ifchilddoc
\providecommand{\version}{draft}
\else
\providecommand{\version}{final}
\fi
%    \end{macrocode}

% Load the standard document class:
%    \begin{macrocode}
\documentclass[12pt]{article}
%    \end{macrocode}

% Start the document body:
%    \begin{macrocode}
\begin{document}
%    \end{macrocode}

% Declare a title page.
% Print title, part of document being processed and version flag:
%    \begin{macrocode}
\addtocounter{page}{-1}
\begin{center}
{\LARGE\bfseries{}childdoc example\par}
\vspace{1cm}
\ifchilddoc
\ifchilddocmanual part\else chapter\fi:
`\childdocname' of `\childdocjob'\par
\else
main document: `\childdocjob'\par
\fi
version: \version\par
\end{center}
\newpage
%    \end{macrocode}

% Manually include selected file,
% otherwise process as usual:
%    \begin{macrocode}
\ifchilddocmanual
\section*{part `\childdocname'}
\input{\childdocname}
\else
%    \end{macrocode}

% Include the two chapters:
%    \begin{macrocode}
\include{cdocsch1}
\include{cdocsch2}
%    \end{macrocode}

% Include the two parts unless only chapters should be displayed:
%    \begin{macrocode}
\ifchilddoc\else
\section{part three}
\input{cdocspt3}
\section{part four}
\input{cdocspt4}
\fi
%    \end{macrocode}

% Process as usual until here:
%    \begin{macrocode}
\fi
%    \end{macrocode}

% End of document body:
%    \begin{macrocode}
\end{document}
%    \end{macrocode}
%\iffalse
%</samplemain>
%\fi
%
% %%%%%%%%%%%%%%%%%%%%%%%%%%%%%%%%%%%%%%
% \paragraph{Chapter Include Files.}
%
% The include files are called |cdocsch1.tex| and |cdocsch2.tex|.
%
%\iffalse
%<*samplechap1|samplechap2>
%\fi

% Optional override for |\version| flag:
%    \begin{macrocode}
%%\providecommand{\version}{final}
%    \end{macrocode}

% Include the main document:
%    \begin{macrocode}
% \iffalse
%
% childdoc.dtx Copyright (C) 2017-2018 Niklas Beisert
%
% This work may be distributed and/or modified under the
% conditions of the LaTeX Project Public License, either version 1.3
% of this license or (at your option) any later version.
% The latest version of this license is in
%   http://www.latex-project.org/lppl.txt
% and version 1.3 or later is part of all distributions of LaTeX
% version 2005/12/01 or later.
%
% This work has the LPPL maintenance status `maintained'.
%
% The Current Maintainer of this work is Niklas Beisert.
%
% This work consists of the files childdoc.dtx and childdoc.ins
% and the derived files childdoc.def and cdocsamp.tex with
% cdocsch1.tex, cdocsch2.tex, cdocsdrf.tex, cdocsfn1.tex, cdocsfn2.tex.
%
%<package>\ifdefined\childdocmain\endinput\fi
%<package>\ProvidesFile{childdoc.def}[2018/12/30 v2.0 child document driver]
%<samplemain>\ProvidesFile{cdocsamp.tex}[2018/12/30 v2.0 sample for childdoc]
%<*driver>
%\ProvidesFile{childdoc.drv}[2018/12/30 v2.0 childdoc reference manual file]
\PassOptionsToClass{10pt,a4paper}{article}
\documentclass{ltxdoc}

\usepackage[margin=35mm]{geometry}
\usepackage{hyperref}
\usepackage{hyperxmp}
\usepackage[usenames]{color}

\hypersetup{colorlinks=true}
\hypersetup{pdfstartview=FitH}
\hypersetup{pdfpagemode=UseNone}
\hypersetup{pdfsource={}}
\hypersetup{pdflang={en-UK}}
\hypersetup{pdfcopyright={Copyright 2017-2018 Niklas Beisert.
  This work may be distributed and/or modified under the
  conditions of the LaTeX Project Public License, either version 1.3
  of this license or (at your option) any later version.}}
\hypersetup{pdflicenseurl={http://www.latex-project.org/lppl.txt}}
\hypersetup{pdfcontactaddress={ETH Zurich, ITP, HIT K,
  Wolfgang-Pauli-Strasse 27}}
\hypersetup{pdfcontactpostcode={8093}}
\hypersetup{pdfcontactcity={Zurich}}
\hypersetup{pdfcontactcountry={Switzerland}}
\hypersetup{pdfcontactemail={nbeisert@itp.phys.ethz.ch}}
\hypersetup{pdfcontacturl={http://people.phys.ethz.ch/\xmptilde nbeisert/}}

\newcommand{\secref}[1]{\hyperref[#1]{section \ref*{#1}}}

\parskip1ex
\parindent0pt
\let\olditemize\itemize
\def\itemize{\olditemize\parskip0pt}

\begin{document}

\title{The \textsf{childdoc} Package}
\hypersetup{pdftitle={The childdoc Package}}
\author{Niklas Beisert\\[2ex]
  Institut f\"ur Theoretische Physik\\
  Eidgen\"ossische Technische Hochschule Z\"urich\\
  Wolfgang-Pauli-Strasse 27, 8093 Z\"urich, Switzerland\\[1ex]
  \href{mailto:nbeisert@itp.phys.ethz.ch}
  {\texttt{nbeisert@itp.phys.ethz.ch}}}
\hypersetup{pdfauthor={Niklas Beisert}}
\hypersetup{pdfsubject={Manual for the LaTeX2e Package childdoc}}
\date{30 December 2018, \textsf{v2.0}}
\maketitle

\begin{abstract}\noindent
\textsf{childdoc} is a \LaTeXe{} package
that enables the direct compilation
of document sections included by |\include|
to individual files.
\end{abstract}

\begingroup
\parskip0ex
\tableofcontents
\endgroup

%%%%%%%%%%%%%%%%%%%%%%%%%%%%%%%%%%%%%%%%%%%%%%%%%%%%%%%%%%%%%%%%%%%%%%%%%%%%%%%%
%%%%%%%%%%%%%%%%%%%%%%%%%%%%%%%%%%%%%%%%%%%%%%%%%%%%%%%%%%%%%%%%%%%%%%%%%%%%%%%%
\section{Introduction}

\LaTeX{} provides a mechanism to structure a large document (such as a book)
into a main file and several child files (containing the chapters)
using the |\include| command.
This mechanism is beneficial for documents
which span hundreds of pages in order to
make the source file(s) more manageable.
Moreover, compilation can be restricted to
selected child files by means of the |\includeonly| command.
The latter feature can be used to reduce the compilation time while editing
(this was significantly more useful in the earlier days of \LaTeX{})
or to generate a smaller document which is easier to navigate.
Another application of |\includeonly| is to generate
documents consisting of selected parts of the complete document.

However, there are a few drawbacks of the plain |\include| mechanism:
\begin{itemize}
\item
The child files cannot be compiled on their own,
they can only be compiled via the main file.
A naive editing environment
(such as a text editor with an option
to have the current file processed by \LaTeX)
may require one to switch to the main file before compiling;
attempting to compile the child file produces errors.
\item
The main file must be modified (each time)
to adjust the |\includeonly| command
to the present needs. This easily leaves the main file in a messy state.
\item
The generated document will always carry the filename
of the main document. This is inconvenient if
several child files are to be compiled and
to be kept for distribution.
\end{itemize}

The present package provides a simple interface
to make child files individually compilable by \LaTeX{}.
Compiling a child file then has the same effect as compiling
the main file with an |\includeonly| command
to select the appropriate child.
Moreover the generated document will carry the name of the child
rather than the main file.
This resolves all three above issues.

This feature is meant to make the editing of books,
thesis documents and lecture notes somewhat more convenient.
However, the package can also be used efficiently for
composing a series of documents (such as exercise sheets)
which are typically distributed individually.
It then assists the author in generating the individual documents
(potentially in different versions)
as well as a document containing the collected series.
Another application is in developing style files
or other kinds of included material
where compilation of the style file could redirect
to a sample or test file.

%%%%%%%%%%%%%%%%%%%%%%%%%%%%%%%%%%%%%%%%%%%%%%%%%%%%%%%%%%%%%%%%%%%%%%%%%%%%%%%%
%%%%%%%%%%%%%%%%%%%%%%%%%%%%%%%%%%%%%%%%%%%%%%%%%%%%%%%%%%%%%%%%%%%%%%%%%%%%%%%%
\section{Usage}

First of all, the package \textsf{childdoc} is \emph{not} a standard
\LaTeXe{} |.sty| style file! Therefore it needs to be invoked in
a non-standard way.

%%%%%%%%%%%%%%%%%%%%%%%%%%%%%%%%%%%%%%%%%%%%%%%%%%%%%%%%%%%%%%%%%%%%%%%%%%%%%%%%
\subsection{Included Files}
\label{sec:include}

%%%%%%%%%%%%%%%%%%%%%%%%%%%%%%%%%%%%%%%%
\DescribeMacro{\childdocmain}
To use the package, add the commands
\begin{center}
\begin{tabular}{l}
|\input{childdoc.def}|\\
|\childdocmain{}|\\
\end{tabular}
\end{center}
at the very top of the main \LaTeX{} file,
in particular \emph{before} the |\documentclass| statement!
The argument of |\childdocmain| should be left empty
(but it must be present).

%%%%%%%%%%%%%%%%%%%%%%%%%%%%%%%%%%%%%%%%
\DescribeMacro{\childdocof}
Furthermore, add the commands
\begin{center}
\begin{tabular}{l}
|\input{childdoc.def}|\\
|\childdocof{|\textit{main}|}|\\
\end{tabular}
\end{center}
at the top of every child file \textit{child}
which is included by |\include{|\textit{child}|}|
from within the main file
(or at least for those files to be compiled individually).
The argument \textit{main} must be the filename of the main file.

There are a couple of
considerations in setting up the main and child documents:

%%%%%%%%%%%%%%%%%%%%%%%%%%%%%%%%%%%%%%%%
\paragraph{Restrictions.}

Please note the following restrictions:
\begin{itemize}
\item
|\childdocmain| must be called with one argument \textit{main}
to ensure compatibility with earlier version of the package.
It must either be empty (|\childdocmain{}|)
or precisely match the filename of the main file in which it is specified.
See \secref{sec:detection} for further information.
\item
The filename \textit{main} must be specified without the |.tex| extension.
\item
The filename \textit{main} is case sensitive
(even in case-insensitive file systems)
due to internal string comparison.
\item
The argument \textit{main} should be fully expanded, it cannot be a macro.
\item
Subdirectories and special characters should be avoided in filenames.
\item
The command |\childdocmain{|\textit{main}|}| must be followed by a whitespace.
It should not be followed immediately by another command
or by a comment mark `|%|'.
This is because the \TeX{} parser reads the token immediately following
the argument of |\childdocmain| and puts it
at the beginning of every child section;
however, a white\-space is ignored.
\end{itemize}

%%%%%%%%%%%%%%%%%%%%%%%%%%%%%%%%%%%%%%%%
\paragraph{Content of Main File.}

It is advisable to place all content in the child files included by |\include|.
Any output contained in the main file will appear in all child documents
unless suppressed manually;
it cannot be suppressed automatically by the |\includeonly| directive
and thus should normally be avoided.
A method to include some content in the main file
by means of conditional processing is described in \secref{sec:conditional}.

%%%%%%%%%%%%%%%%%%%%%%%%%%%%%%%%%%%%%%%%
\paragraph{Page Numbering.}

When only a part of the document is compiled,
the appropriate numbering of pages
(as well as other status parameters)
is determined from the |.aux| files.
The latter contain information from previous passes.
However this information needs to propagate through
all intermediate child documents.
Therefore the page numbering in child documents may well
be inconsistent until the complete document is compiled at least once.

A useful (if unconventional) way to always ensure a consistent
page numbering is to restart the numbering in each child document
and denote the pages by `\textit{child}|.|\textit{page}'
where \textit{child} represents the chapter/section number of the child file.
This can be achieved by the command
|\numberwithin{page}{|\textit{child}|}|
of the \textsf{amsmath} package
where \textit{child} can be |chapter| or |section|
depending on the chosen structuring.
Alternatively, one can modify the macro |\thepage| appropriately
and reset the counter |page| at the start of each child file.

%%%%%%%%%%%%%%%%%%%%%%%%%%%%%%%%%%%%%%%%%%%%%%%%%%%%%%%%%%%%%%%%%%%%%%%%%%%%%%%%
\subsection{Conditional Processing}
\label{sec:conditional}

The package provides a mechanism to compile different versions
of a document. To customise the versions further some conditional processing
can come in handy to distinguish which version is being compiled.
The package provides two macros to describe the compilation context:

%%%%%%%%%%%%%%%%%%%%%%%%%%%%%%%%%%%%%%%%
\DescribeMacro{\ifchilddoc}
The conditional |\ifchilddoc| distinguishes between the compilation of
child documents and the main document:
%
\begin{center}
|\ifchilddoc |\textit{child-code}| |[|\||else |\textit{main-code}]| \||fi|
\end{center}

%%%%%%%%%%%%%%%%%%%%%%%%%%%%%%%%%%%%%%%%
\DescribeMacro{\childdocname}
\DescribeMacro{\childdocjob}
The macro |\childdocname| contains the filename (without extension)
of the main or child file being processed.
Note that |\childdocjob| will always contain the name of the main file.

%%%%%%%%%%%%%%%%%%%%%%%%%%%%%%%%%%%%%%%%
\paragraph{Title Page.}

Conditional processing can be used to include a title or banner page
in the main document when proper precautions are taken.
Importantly, the code in the main file should ensure that the page counter
(as well as other status parameters which are stored in the |.aux| files)
takes the same value after the conditional processing.
Otherwise the page numbers may take divergent values
depending on which part is compiled.

For example, a title page could be declared by:
%
\begin{center}
\begin{tabular}{l}
|\ifchilddoc\||else|\\
|\addtocounter{page}{-1}|\\
\textit{code for title page}\\
|\newpage|\\
|\||fi|
\end{tabular}
\end{center}
%
A banner page for the child documents can be generated by:
%
\begin{center}
\begin{tabular}{l}
|\ifchilddoc|\\
|\addtocounter{page}{-1}|\\
\textit{code for banner page}\\
|\newpage|\\
|\||fi|
\end{tabular}
\end{center}
%
Here one could write a message such as:
\begin{center}
|This is the part \childdocname{} of \childdocjob{}.|
\end{center}

%%%%%%%%%%%%%%%%%%%%%%%%%%%%%%%%%%%%%%%%%%%%%%%%%%%%%%%%%%%%%%%%%%%%%%%%%%%%%%%%
\subsection{Flags}
\label{sec:flags}

The package makes it easy to generate different versions
of the main or child documents.
To this end compilation flags can be defined
and assigned different default values.
They will be particularly useful in conjunction
with the forwarding mechanism described in \secref{sec:forward}.

For example, it may be useful to have a flag |\version|
which can be set to |draft| or |final|.
The document source will contain some conditional code
depending on the value of |\version|.
Suppose further, the flag should default to |final| for the main file
and to |draft| for child files
which is a natural assignment for editing the document.
This is achieved by placing the following code
in the preamble of the main document
(below the |\childdocmain| directive):
%
\begin{center}
\begin{tabular}{l}
|\ifchilddoc|\\
|\providecommand{\version}{draft}|\\
|\||else|\\
|\providecommand{\version}{final}|\\
|\||fi|
\end{tabular}
\end{center}
%
The definition by |\providecommand| makes sure
that previous definitions are not overwritten.
Further statements |\providecommand{\version}{...}|
can thus be added before the above code to override it.

For the main file, one might add a line
(between |\childdocmain| and the above block)
%
\begin{center}
|%\ifchilddoc\||else\providecommand{\version}{draft}\||fi|
\end{center}
%
which can be uncommented to produce a draft version.
Likewise one can add a line to the very top of a child file
(above the |\childdocof{|\textit{main}|}| directive)
%
\begin{center}
|%\providecommand{\version}{final}|
\end{center}
%
which can be uncommented to produce the final version of this child document.

%%%%%%%%%%%%%%%%%%%%%%%%%%%%%%%%%%%%%%%%%%%%%%%%%%%%%%%%%%%%%%%%%%%%%%%%%%%%%%%%
\subsection{Forwarding}
\label{sec:forward}

Different versions of the main or child documents
using compilation flags as described in \secref{sec:flags}
can be (permanently) stored in different files
for convenient compilation, viewing and distribution.
To this end, the package defines a command
to pass on compilation to a different file:

%%%%%%%%%%%%%%%%%%%%%%%%%%%%%%%%%%%%%%%%
\DescribeMacro{\childdocforward}
The command |\childdocforward| redirects processing to
another source file:
%
\begin{center}
\begin{tabular}{l}
|\input{childdoc.def}|\\
|\childdocforward[|\textit{main}|]{|\textit{dest}|}|\\
\end{tabular}
\end{center}
%
The argument \textit{dest} is the destination file
(without extension).
It should be the main file or one of the child files.
Note that further \textsf{childdoc} directives
such as |\childdocof| and |\childdocforward|
in the indicated file will be processed in this form.
The optional argument \textit{main}
passes on directly to the main file \textit{main}
while pretending to compile the child \textit{dest}.
This form behaves as if \textit{dest}
issues |\childdocof{|\textit{main}|}| right away,
and no further \textsf{childdoc} directives will be processed.

%%%%%%%%%%%%%%%%%%%%%%%%%%%%%%%%%%%%%%%%
\DescribeMacro{\...prefix}
In the alternative form |\childdocforwardprefix|,
%
\begin{center}
\begin{tabular}{l}
|\input{childdoc.def}|\\
|\childdocforwardprefix[|\textit{main}|]{|\textit{prefix}|}{|\textit{dest}|}|
\end{tabular}
\end{center}
%
the destination file is determined by a pattern
depending on the current file:
To make this work, the current file must be called
`{\textit{prefix}\hspace{0.2em}\textit{suffix}}'
with \textit{prefix} matching precisely the argument.
Processing is then passed on to the file
`{\textit{dest}\hspace{0.2em}\textit{suffix}}'.
Surely, the same effect is achieved by
directly specifying the
argument `{\textit{dest}\hspace{0.2em}\textit{suffix}}'
in the first form.
However, that requires to set up a different file
for each child. With the alternative form of the command
all these files can have exactly the same content
which simplifies setting them up and maintaining them.

For example, the following file |draft.tex|
with a compilation flag |\version| as described in \secref{sec:flags}
compiles the main document as a draft:
%
\begin{center}
\begin{tabular}{l}
|\def\version{draft}|\\
|\input{childdoc.def}|\\
|\childdocforward{|\textit{main}|}|
\end{tabular}
\end{center}
%
Likewise, the following files |final|\textit{nn}|.tex|
compile the final version of the child document
|child|\textit{nn}|.tex|:
%
\begin{center}
\begin{tabular}{l}
|\def\version{final}|\\
|\input{childdoc.def}|\\
|\childdocforwardprefix{final}{child}|
\end{tabular}
\end{center}
%

Note that when several versions of a main file and/or of each child file
are to be generated, it may be convenient to set up a |Makefile| or
shell script to automatise the process.

%%%%%%%%%%%%%%%%%%%%%%%%%%%%%%%%%%%%%%%%%%%%%%%%%%%%%%%%%%%%%%%%%%%%%%%%%%%%%%%%
\subsection{Command Line Processing}
\label{sec:commandline}

The effect of redirection files can also be achieved by invoking
the \LaTeX{} compiler with a more elaborate command line.
Most conveniently this should be done as part
of a shell script or a |Makefile|.

When using \textsf{childdoc} in the main file, the following
command lines effectively perform a redirection
(note that depending on the shell being used,
backslashes may have to be doubled: `|\|' $\to$ `|\\|'):
%
\begin{center}
|... -jobname "|\textit{target}|" |\\|"|[\textit{flags}]%
|\input{childdoc.def}\childdocforward[|\textit{main}|]{|\textit{dest}|}"|
\end{center}
%
Here \textit{target} is the name of the output file,
\textit{main} is the name of the main file
and \textit{dest} is the name of the main or child file to be processed
(all filenames without extensions).
The optional argument \textit{main} can be omitted
if \textit{main} matches \textit{dest}.
Optionally, compilation \textit{flags} can be defined via |\def| commands.
This command line makes the \TeX{} engine believe
it is compiling the file \textit{target}
whose content is specified as the latter parameter.
The provided code then forwards the processing to
\textit{main} or \textit{dest} as described in \secref{sec:forward}.

%%%%%%%%%%%%%%%%%%%%%%%%%%%%%%%%%%%%%%%%%%%%%%%%%%%%%%%%%%%%%%%%%%%%%%%%%%%%%%%%
\subsection{Include by Input}
\label{sec:input}

Including child documents by |\include| has some restrictions by design.
Most notably, the content of a child document always occupies
its own set of pages; pages cannot be shared between child documents.
Usually, this behaviour makes perfect sense
because each child document contain an essential part of the document.
However, in some situations it may be desirable to compose
a document from a collection of parts
without having mandatory page breaks between then.
For this case, the package
provides a mechanism to include parts
by |\input| which can also be processed individually.
However, by construction this mechanism
requires manual handling of the content to be output.

%%%%%%%%%%%%%%%%%%%%%%%%%%%%%%%%%%%%%%%%
\DescribeMacro{\ifchilddocmanual}
The main file should be prepared as usual, see \secref{sec:include}.
However, the document body must make a distinction
between processing of an individual part and of the main document, e.g.:
%
\begin{center}
\begin{tabular}{l}
|\ifchilddocmanual|\\
|\input{\childdocname}|\\
|\||else|\\
\textit{document body with }|\input{|\textit{part}|}|\\
|\||fi|
\end{tabular}
\end{center}
%
The conditional |\ifchilddocmanual| is true whenever
a part to be included by |\input| is being compiled,
and the name of the part is stored in |\childdocname|.

%%%%%%%%%%%%%%%%%%%%%%%%%%%%%%%%%%%%%%%%
\DescribeMacro{\childdocby}
Each part to be included by |\input| should start with:
%
\begin{center}
\begin{tabular}{l}
|\input{childdoc.def}|\\
|\childdocby{|\textit{main}|}|\\
\end{tabular}
\end{center}
%
The directive |\childdocby| is similar to |\childdocof|
described in \secref{sec:include},
but the subsequent selection of content must be done manually.
To that end, both |\ifchilddoc| and |\ifchilddocmanual|
will be true upon processing of a part,
and the name of the part is stored in |\childdocname|.
Note that |\jobname| will be set to the filename of the current part
so that each part receives an individual |.aux| file
that does not interfere with the |.aux| file(s) of the main document.
This behaviour can be altered by the alternative form
|\childdocby[*]{|\textit{main}|}| (with a non-empty optional argument)
which uses the |.aux| file of the main document
by setting |\jobname| to \textit{main}.

%%%%%%%%%%%%%%%%%%%%%%%%%%%%%%%%%%%%%%%%%%%%%%%%%%%%%%%%%%%%%%%%%%%%%%%%%%%%%%%%
\subsection{Driver Development}
\label{sec:driver}

The \textsf{childdoc} mechanism can also be use for the development
of definition files such as \LaTeX{} styles or classes.
This case differs from the above setup with multiple parts
included by |\include| in that no |\includeonly| should be invoked.
This can be achieved by starting the include file
(before |\ProvidesPackage|) with:
%
\begin{center}
\begin{tabular}{l}
|\input{childdoc.def}|\\
|\childdocforward{|\textit{main}|}|\\
\end{tabular}
\end{center}
%
or alternatively with:
%
\begin{center}
\begin{tabular}{l}
|\input{childdoc.def}|\\
|\childdocby{|\textit{main}|}|\\
\end{tabular}
\end{center}
%
Both forms have slightly different effects as described above.
The main file is prepared as usual, see \secref{sec:include}.

%%%%%%%%%%%%%%%%%%%%%%%%%%%%%%%%%%%%%%%%%%%%%%%%%%%%%%%%%%%%%%%%%%%%%%%%%%%%%%%%
\subsection{Legacy Detection}
\label{sec:detection}

The directive |\childdocmain| in the main file can detect
whether the complete document or merely a child is to be compiled
even without using the directive |\childdocof|.
This method is deprecated because it is less robust
and there is no compelling reason to use it;
it is merely provided for backward compatibility
and it may be removed in future versions.

If the detection mechanism is to be used,
it is mandatory to correctly specify
the filename of the main file as the argument of |\childdocmain|:
%
\begin{center}
\begin{tabular}{l}
|\input{childdoc.def}|\\
|\childdocmain{|\textit{main}|}|\\
\end{tabular}
\end{center}
%
If |\jobname| does not match the argument \textit{main} of |\childdocmain|,
it is assumed that |\jobname| points to the child file to be compiled.
When using |\childdocmain| with the main file specified as argument,
it suffices to start a child file
with just |\input{|\textit{main}|}|
without loading of the package and using |\childdocof|.
If instead all processing is done
with the appropriate \textsf{childdoc} directives,
the argument of \textit{main} of |\childdocmain| can be empty.

An alternative version of the command line processing described
in \secref{sec:commandline} using the detection mechanism reads:
%
\begin{center}
|... -jobname "|\textit{target}|" "|[\textit{flags}]%
[|\def\jobname{|\textit{dest}|}|]|\input{|\textit{main}|}"|
\end{center}

%%%%%%%%%%%%%%%%%%%%%%%%%%%%%%%%%%%%%%%%%%%%%%%%%%%%%%%%%%%%%%%%%%%%%%%%%%%%%%%%
\subsection{Manual Code}
\label{sec:manual}

In case one cannot be certain whether the definitions file |childdoc.def|
is installed on the target \TeX{} distribution
and one prefers not to ship it,
it is conceivable to paste a few relevant commands into the sources.

To that end, drop all statements |\input{childdoc.def}|
and perform the replacements as outlined below.
Instead of |\childdocmain{|\textit{main}|}| add the following code
to the top of the main file:
%
\begin{center}
\begin{tabular}{l}
|\||ifdefined\childdocname\endinput\||fi\newif\ifchilddoc|\\
|\edef\childdocname{\scantokens\expandafter{\jobname\noexpand}}|\\
|\def\childdocmain{|\textit{main}|}\||ifx\childdocmain\childdocname\||else|\\
|\childdoctrue\includeonly{\childdocname}\let\jobname\childdocmain\||fi|\\
\end{tabular}
\end{center}
%
Instead of |\childdocof{|\textit{main}|}| just include the main file
at the top of each child file:
%
\begin{center}
|\input{|\textit{main}|}|
\end{center}
%
A simple redirection |\childdocforward{|\textit{dest}|}| is achieved by:
%
\begin{center}
|\def\jobname{|\textit{dest}|}\input{\jobname}|
\end{center}
%
The redirection with prefix
|\childdocforwardprefix[|\textit{prefix}|]{|\textit{dest}|}|
is accomplished by:
%
\begin{center}
\begin{tabular}{l}
|{\edef\jobname{\scantokens\expandafter{\jobname\noexpand}}|\\
|\def\redirectjob |\textit{prefix}|#1~~~{\gdef\jobname{|\textit{dest}|#1}}|\\
|\expandafter\redirectjob\jobname~~~}\input{\jobname}|
\end{tabular}
\end{center}

In an alternative approach,
child documents can be compiled by a specific command line
without additional code or specific definitions:
%
\begin{center}
|... -jobname "|\textit{target}|" "|[\textit{flags}]%
|\includeonly{|\textit{dest}|}\input{|\textit{main}|}"|
\end{center}
%

%%%%%%%%%%%%%%%%%%%%%%%%%%%%%%%%%%%%%%%%%%%%%%%%%%%%%%%%%%%%%%%%%%%%%%%%%%%%%%%%
%%%%%%%%%%%%%%%%%%%%%%%%%%%%%%%%%%%%%%%%%%%%%%%%%%%%%%%%%%%%%%%%%%%%%%%%%%%%%%%%
\section{Information}

%%%%%%%%%%%%%%%%%%%%%%%%%%%%%%%%%%%%%%%%%%%%%%%%%%%%%%%%%%%%%%%%%%%%%%%%%%%%%%%%
\subsection{Copyright}

Copyright \copyright{} 2017--2018 Niklas Beisert

This work may be distributed and/or modified under the
conditions of the \LaTeX{} Project Public License, either version 1.3
of this license or (at your option) any later version.
The latest version of this license is in
  \url{http://www.latex-project.org/lppl.txt}
and version 1.3 or later is part of all distributions of \LaTeX{}
version 2005/12/01 or later.

This work has the LPPL maintenance status `maintained'.

The Current Maintainer of this work is Niklas Beisert.

This work consists of the files |README.txt|, |childdoc.ins| and |childdoc.dtx|
as well as the derived files |childdoc.def|, |cdocsamp.tex|
with |cdocsch1.tex|, |cdocsch2.tex|, |cdocspt3.tex|, |cdocspt4.tex|,
|cdocsdrf.tex|, |cdocsfn1.tex|, |cdocsfn2.tex|
as well as |childdoc.pdf|.

%%%%%%%%%%%%%%%%%%%%%%%%%%%%%%%%%%%%%%%%%%%%%%%%%%%%%%%%%%%%%%%%%%%%%%%%%%%%%%%%
\subsection{Files and Installation}

The package consists of the files:
%
\begin{center}
\begin{tabular}{ll}
    |README.txt|   & readme file \\
    |childdoc.ins| & installation file \\
    |childdoc.dtx| & source file \\
    |childdoc.def| & definition file \\
    |cdocsamp.tex| & sample main file \\
    |cdocsch1.tex| & sample include file \\
    |cdocsch2.tex| & sample include file \\
    |cdocspt3.tex| & sample part file \\
    |cdocspt4.tex| & sample part file \\
    |cdocsdrf.tex| & sample redirection file \\
    |cdocsfn1.tex| & sample redirection file \\
    |cdocsfn2.tex| & sample redirection file \\
    |childdoc.pdf| & manual
\end{tabular}
\end{center}
%
The distribution consists of the files
|README.txt|, |childdoc.ins| and |childdoc.dtx|.
%
\begin{itemize}
\item
Run (pdf)\LaTeX{} on |childdoc.dtx|
to compile the manual |childdoc.pdf| (this file).
\item
Run \LaTeX{} on |childdoc.ins| to create the definitions file |childdoc.def|
and the sample |cdocsamp.tex| with include files
|cdocsch1.tex|, |cdocsch2.tex|, |cdocspt3.tex|, |cdocspt4.tex|,
|cdocsdrf.tex|, |cdocsfn1.tex|, |cdocsfn2.tex|.
Then copy the file |childdoc.def| to an appropriate directory of your \LaTeX{}
distribution, e.g.\ \textit{texmf-root}|/tex/latex/childdoc|.
\end{itemize}

%%%%%%%%%%%%%%%%%%%%%%%%%%%%%%%%%%%%%%%%%%%%%%%%%%%%%%%%%%%%%%%%%%%%%%%%%%%%%%%%
\subsection{Related CTAN Packages}

There are several other packages which offer a similar functionality:
%
\begin{itemize}
\item
The packages
\href{http://ctan.org/pkg/docmute}{\textsf{docmute}},
\href{http://ctan.org/pkg/includex}{\textsf{includex}} and
\href{http://ctan.org/pkg/standalone}{\textsf{standalone}}
provide commands to include only the document body of
a child file thus allowing both files to be compiled individually.
\item
The packages \href{http://ctan.org/pkg/subdocs}{\textsf{subdocs}}
and \href{http://ctan.org/pkg/subfiles}{\textsf{subfiles}}
provide structures in which the main and child documents can be
encapsulated and allowing them to be compiled individually.
The inclusion mechanism is different from the conventional |\include|.
\item
The package \href{http://ctan.org/pkg/combine}{\textsf{combine}}
is an elaborate solution to combine several documents into one.
\end{itemize}
%
See also the CTAN topic \href{http://ctan.org/topic/subdocs}{\textsf{subdocs}}
for further related packages.
The present package differs from the above solutions in that
a document structure constructed with the conventional |\include| mechanism
just needs two extra commands at the top of every file
such that all constituent files can be compiled individually.

%%%%%%%%%%%%%%%%%%%%%%%%%%%%%%%%%%%%%%%%%%%%%%%%%%%%%%%%%%%%%%%%%%%%%%%%%%%%%%%%
%\subsection{Feature Suggestions}
%
%The following is a list of features which may be useful for future
%versions of this package:
%%
%\begin{itemize}
%\item
%\ldots
%\end{itemize}

%%%%%%%%%%%%%%%%%%%%%%%%%%%%%%%%%%%%%%%%%%%%%%%%%%%%%%%%%%%%%%%%%%%%%%%%%%%%%%%%
\subsection{Revision History}

%%%%%%%%%%%%%%%%%%%%%%%%%%%%%%%%%%%%%%%%
\paragraph{v2.0:} 2018/12/30

\begin{itemize}
\item
immediate forward processing
\item
added |\childdocby| mechanism
\item
manual restructured
\end{itemize}

%%%%%%%%%%%%%%%%%%%%%%%%%%%%%%%%%%%%%%%%
\paragraph{v1.6:} 2018/01/17

\begin{itemize}
\item
application for development of include files
\item
corrections to manual
\end{itemize}

%%%%%%%%%%%%%%%%%%%%%%%%%%%%%%%%%%%%%%%%
\paragraph{v1.5:} 2017/05/21

\begin{itemize}
\item
more complete structuring introduced
\item
|\childdocof| introduced
\item
|\childdoc| renamed to |\childdocmain|
\item
|\childredirect| renamed to |\childdocforward| and |\childdocforwardprefix|
and functionality expanded
\end{itemize}

%%%%%%%%%%%%%%%%%%%%%%%%%%%%%%%%%%%%%%%%
\paragraph{v1.0:} 2017/04/27

\begin{itemize}
\item
manual and install package
\item
first version published on CTAN
\end{itemize}

%%%%%%%%%%%%%%%%%%%%%%%%%%%%%%%%%%%%%%%%
\paragraph{v0.6:} 2017/04/26

\begin{itemize}
\item
redirection mechanism added
\end{itemize}

%%%%%%%%%%%%%%%%%%%%%%%%%%%%%%%%%%%%%%%%
\paragraph{v0.5:} 2017/04/26

\begin{itemize}
\item
functionality in definition file
\end{itemize}


%%%%%%%%%%%%%%%%%%%%%%%%%%%%%%%%%%%%%%%%%%%%%%%%%%%%%%%%%%%%%%%%%%%%%%%%%%%%%%%%
%%%%%%%%%%%%%%%%%%%%%%%%%%%%%%%%%%%%%%%%%%%%%%%%%%%%%%%%%%%%%%%%%%%%%%%%%%%%%%%%
%%%%%%%%%%%%%%%%%%%%%%%%%%%%%%%%%%%%%%%%%%%%%%%%%%%%%%%%%%%%%%%%%%%%%%%%%%%%%%%%
\appendix

\settowidth\MacroIndent{\rmfamily\scriptsize 000\ }

 \DocInput{childdoc.dtx}

\end{document}
%</driver>
% \fi
%
% %%%%%%%%%%%%%%%%%%%%%%%%%%%%%%%%%%%%%%%%%%%%%%%%%%%%%%%%%%%%%%%%%%%%%%%%%%%%%%
% %%%%%%%%%%%%%%%%%%%%%%%%%%%%%%%%%%%%%%%%%%%%%%%%%%%%%%%%%%%%%%%%%%%%%%%%%%%%%%
% \section{Sample}
%\iffalse
%<*samplemain>
%\fi
%
% The following presents a sample document
% with two chapters, two parts, a title page,
% a compile flag as well as three forwarding files to set the flag.
% It consists of eight |.tex| files:
% \begin{center}
% \begin{tabular}{ll}
% |cdocsamp.tex|&main file\\
% |cdocsch1.tex|&include file for chapter 1\\
% |cdocsch2.tex|&include file for chapter 2\\
% |cdocspt3.tex|&include file for part 3\\
% |cdocspt4.tex|&include file for part 4\\
% |cdocsdrf.tex|&forwarding file for main file in draft mode\\
% |cdocsfi1.tex|&forwarding file for final version of chapter 1\\
% |cdocsfi2.tex|&forwarding file for final version of chapter 2\\
% \end{tabular}
% \end{center}
% Each of the eight files can be compiled directly by the \LaTeX{} compiler.
%
% %%%%%%%%%%%%%%%%%%%%%%%%%%%%%%%%%%%%%%
% \paragraph{Main File.}
%
% The main file is called |cdocsamp.tex|.
%
% Load the \textsf{childdoc} definitions and
% declare the filename for the main document:
%    \begin{macrocode}
\input{childdoc.def}
\childdocmain{}
%    \end{macrocode}

% Optional override for |\version| flag:
%    \begin{macrocode}
%%\ifchilddoc\else\providecommand{\version}{draft}\fi
%    \end{macrocode}

% Define the default values for the |\version| flag
% (|final| for the main file and |draft| for childs):
%    \begin{macrocode}
\ifchilddoc
\providecommand{\version}{draft}
\else
\providecommand{\version}{final}
\fi
%    \end{macrocode}

% Load the standard document class:
%    \begin{macrocode}
\documentclass[12pt]{article}
%    \end{macrocode}

% Start the document body:
%    \begin{macrocode}
\begin{document}
%    \end{macrocode}

% Declare a title page.
% Print title, part of document being processed and version flag:
%    \begin{macrocode}
\addtocounter{page}{-1}
\begin{center}
{\LARGE\bfseries{}childdoc example\par}
\vspace{1cm}
\ifchilddoc
\ifchilddocmanual part\else chapter\fi:
`\childdocname' of `\childdocjob'\par
\else
main document: `\childdocjob'\par
\fi
version: \version\par
\end{center}
\newpage
%    \end{macrocode}

% Manually include selected file,
% otherwise process as usual:
%    \begin{macrocode}
\ifchilddocmanual
\section*{part `\childdocname'}
\input{\childdocname}
\else
%    \end{macrocode}

% Include the two chapters:
%    \begin{macrocode}
\include{cdocsch1}
\include{cdocsch2}
%    \end{macrocode}

% Include the two parts unless only chapters should be displayed:
%    \begin{macrocode}
\ifchilddoc\else
\section{part three}
\input{cdocspt3}
\section{part four}
\input{cdocspt4}
\fi
%    \end{macrocode}

% Process as usual until here:
%    \begin{macrocode}
\fi
%    \end{macrocode}

% End of document body:
%    \begin{macrocode}
\end{document}
%    \end{macrocode}
%\iffalse
%</samplemain>
%\fi
%
% %%%%%%%%%%%%%%%%%%%%%%%%%%%%%%%%%%%%%%
% \paragraph{Chapter Include Files.}
%
% The include files are called |cdocsch1.tex| and |cdocsch2.tex|.
%
%\iffalse
%<*samplechap1|samplechap2>
%\fi

% Optional override for |\version| flag:
%    \begin{macrocode}
%%\providecommand{\version}{final}
%    \end{macrocode}

% Include the main document:
%    \begin{macrocode}
\input{childdoc.def}
\childdocof{cdocsamp}
%    \end{macrocode}

%\iffalse
%</samplechap1|samplechap2>
%\fi
%
%\iffalse
%<*samplechap1>
%\fi
% Some text for chapter 1:
%    \begin{macrocode}
\section{one}
some text in chapter one
%    \end{macrocode}

%\iffalse
%</samplechap1>
%\fi
% Some text for chapter 2:
%\iffalse
%<*samplechap2>
%\fi
%    \begin{macrocode}
\section{two}
more text in chapter two
%    \end{macrocode}

%\iffalse
%</samplechap2>
%\fi
%
% %%%%%%%%%%%%%%%%%%%%%%%%%%%%%%%%%%%%%%
% \paragraph{Part Include Files.}
%
% The include files are called |cdocspt3.tex| and |cdocspt4.tex|.
%
%\iffalse
%<*samplepart3|samplepart4>
%\fi

% Optional override for |\version| flag:
%    \begin{macrocode}
%%\providecommand{\version}{final}
%    \end{macrocode}

% Include the main document:
%    \begin{macrocode}
\input{childdoc.def}
\childdocby{cdocsamp}
%    \end{macrocode}

%\iffalse
%</samplepart3|samplepart4>
%\fi
%
%\iffalse
%<*samplepart3>
%\fi
% Some text for part 3:
%    \begin{macrocode}
some text in part three
%    \end{macrocode}

%\iffalse
%</samplepart3>
%\fi
% Some text for part 4:
%\iffalse
%<*samplepart4>
%\fi
%    \begin{macrocode}
more text in part four
%    \end{macrocode}

%\iffalse
%</samplepart4>
%\fi
%
% %%%%%%%%%%%%%%%%%%%%%%%%%%%%%%%%%%%%%%
% \paragraph{Forwarding for a Complete Draft.}
%
% The following forwarding file |cdocsdrf.tex|
% compiles the main document in draft mode:
%\iffalse
%<*sampledraft>
%\fi
%    \begin{macrocode}
\def\version{draft}
\input{childdoc.def}
\childdocforward{cdocsamp}
%    \end{macrocode}

%\iffalse
%</sampledraft>
%\fi
%
% %%%%%%%%%%%%%%%%%%%%%%%%%%%%%%%%%%%%%%
% \paragraph{Forwarding for Final Version of the Chapters.}
%
% The following forwarding files |cdocsfn1.tex| and |cdocsfn2.tex|
% (with identical content)
% compile the final versions of the child documents
% |cdocsch1.tex| and |cdocsch2.tex|, respectively:
%\iffalse
%<*samplefinal>
%\fi
%    \begin{macrocode}
\def\version{final}
\input{childdoc.def}
\childdocforwardprefix[cdocsamp]{cdocsfn}{cdocsch}
%    \end{macrocode}

%\iffalse
%</samplefinal>
%\fi
%
% %%%%%%%%%%%%%%%%%%%%%%%%%%%%%%%%%%%%%%
% \paragraph{Command Line Processing.}
%
% The following three command lines generate the output files
% |cdocscld|, |cdocscl1| and |cdocscl2|
% which should be identical to
% |cdocsdrf|, |cdocsch1| and |cdocsfn2|, respectively:
% \begin{center}
% \begin{tabular}{l}
% |latex -jobname cdocscld \|\\
% |  "\def\version{draft}\input{childdoc.def}\childdocforward{cdocsamp}"|\\
% |latex -jobname cdocscl1 \|\\
% |  "\input{childdoc.def}\childdocforward[cdocsamp]{cdocsch1}"|\\
% |latex -jobname cdocscl2 \|\\
% |  "\def\version{final}\input{childdoc.def}\childdocforward{cdocsch2}"|
% \end{tabular}
% \end{center}
% Note that the trailing backslash on each first line
% merely continues the input to the second line
% (for convenient cut ant paste).
% Furthermore, the command |latex| can be replaced by any
% of its alternative versions such as |pdflatex|.
%
% %%%%%%%%%%%%%%%%%%%%%%%%%%%%%%%%%%%%%%%%%%%%%%%%%%%%%%%%%%%%%%%%%%%%%%%%%%%%%%
% %%%%%%%%%%%%%%%%%%%%%%%%%%%%%%%%%%%%%%%%%%%%%%%%%%%%%%%%%%%%%%%%%%%%%%%%%%%%%%
% \section{Implementation}
%\iffalse
%<*package>
%\fi
%
% This section describes the definitions file |childdoc.def|.

% The definitions cannot be loaded using |\usepackage| or |\RequirePackage|
% which has a mechanism to prevent loading a style file more than once.
% When loading the definitions by means of |\input|
% multiple instances have to be prevented manually:
%\iffalse
%This code needs to be before the `\ProvidesFile' directive
%which is defined at the beginning of this file.
%Therefore it is also placed there and commented out here.
%</package>
%<*discard>
%\fi
%    \begin{macrocode}
\ifdefined\childdocmain\endinput\fi
%    \end{macrocode}
%\iffalse
%</discard>
%<*package>
%\fi
%
% \macro{\ifchilddoc}
% \macro{\ifchilddocmanual}
% The conditional |\ifchilddoc| tells whether a
% child (true) or main (false) document is being compiled.
% The conditional |\ifchilddocmanual| tells whether
% the |\includeonly| mechanism is used (false) or
% the selection of child files must be performed manually (true).
% The definitions initialise to false:
%    \begin{macrocode}
\newif\ifchilddoc
\newif\ifchilddocmanual
%    \end{macrocode}

% \macro{\childdocname}
% \macro{\childdocjob}
% The macro |\childdocname| stores the name of the main document
% to be compiled. The macro |\childdocjob| stores the name of
% the document on which the \LaTeX{} compiler was originally invoked.
% The content of |\jobname| cannot be compared
% to filenames specified in the source due to different catcodes.
% The following code rescans |\jobname|, stores the result
% in |\childdocname| and saves a copy in |\childdocjob|:
%    \begin{macrocode}
\edef\childdocname{\scantokens\expandafter{\jobname\noexpand}}
\let\childdocjob\childdocname
%    \end{macrocode}

% \macro{\childdocdisable}
% The macro |\childdocdisable| prevents the main file
% from being processed more than once.
% At this stage, the main document command |\childdocmain|
% is assumed to be called once again where it should do nothing.
% Any subsequent call to it should prevent
% a secondary processing of the main document
% It overwrites the forwarding commands
% |\childdocof| and |\childdocforward|
% with empty macros to prevent further inclusions of the main document:
%    \begin{macrocode}
\newcommand{\childdocdisable}
{
  \renewcommand{\childdocmain}[1]{\renewcommand{\childdocmain}[1]{\endinput}}
  \renewcommand{\childdocof}[1]{}
  \renewcommand{\childdocby}[2][]{}
  \renewcommand{\childdocforward}[2][]{}
  \renewcommand{\childdocdisable}{}
}
%    \end{macrocode}

% \macro{\childdocmain}
% The macro |\childdocmain| is to be called at the top of the main file
% with nothing or the main filename (without extension) as argument.
% First, it breaks loops.
% If the argument is not empty and does not match |\childdocname|
% (which is set by the first inclusion of |childdoc.def|),
% |\ifchilddoc| is set to true, |\includeonly| is applied to the child file
% and |\jobname| is set to the main file
% (for proper handling of |.aux| files):
%    \begin{macrocode}
\newcommand{\childdocmain}[1]
{
  \childdocdisable\childdocmain{}
  \if?#1?\else
    \begingroup
      \def\childdoctmp{#1}
      \ifx\childdoctmp\childdocname
        \def\childdoctmp{}
      \else
        \def\childdoctmp
        {
          \childdoctrue
          \includeonly{\childdocname}
          \def\childdocjob{#1}
          \def\jobname{#1}
        }
      \fi
      \expandafter
    \endgroup
    \childdoctmp
  \fi
}
%    \end{macrocode}

% \macro{\childdocof}
% The command |\childdocof| redirects
% compilation to the main file |#1|.
%    \begin{macrocode}
\newcommand{\childdocof}[1]
{
  \childdocdisable
  \childdoctrue
  \includeonly{\childdocname}
  \def\jobname{#1}
  \def\childdocjob{#1}
  \input{#1}
}
%    \end{macrocode}

% \macro{\childdocby}
% The command |\childdocby| ....
%    \begin{macrocode}
\newcommand{\childdocby}[2][]
{
  \childdocdisable
  \childdoctrue
  \childdocmanualtrue
  \if?#1?\else
    \def\jobname{#2}
  \fi
  \def\childdocjob{#2}
  \input{#2}
  \endinput
}
%    \end{macrocode}

% \macro{\childdocforward}
% The command |\childdocforward| redirects
% compilation to the main file or
% (if the optional argument is given) a child file.
% Parameters are set as if the main file
% or a child file starting with |\childdocof| was compiled.
% Then compilation is handed over to the main file:
%    \begin{macrocode}
\newcommand{\childdocforward}[2][]
{
  \begingroup
    \if?#1?
      \def\childdoctmp
      {
        \def\childdocname{#2}
        \def\childdocjob{#2}
        \def\jobname{#2}
        \input{#2}
        \endinput
      }
    \else
      \def\childdoctmp
      {
        \childdocdisable
        \def\childdocname{#2}
        \childdoctrue
        \includeonly{#2}
        \def\childdocjob{#1}
        \def\jobname{#1}
        \input{#1}
        \endinput
      }
    \fi
    \expandafter
  \endgroup
  \childdoctmp
}
%    \end{macrocode}

% \macro{\childdocforwardprefix}
% The command |\childdocforwardprefix| redirects
% compilation to the main or a child file by means of a pattern.
% The prefix |#1| in the current filename is replaced by |#2|
% and the suffix of the current filename is kept
% (it is assumed that the filename does not contain the substring `|~~~|'
% which is used as a delimiter).
% Compilation is handed over to the new file by |\childdocforward|:
%    \begin{macrocode}
\newcommand{\childdocforwardprefix}[3][]
{
  \begingroup
    \def\childdocextract #2##1~~~{\def\childdoctmp{\childdocforward[#1]{#3##1}}}
    \expandafter\childdocextract\childdocname~~~
    \expandafter
  \endgroup
  \childdoctmp
}
%    \end{macrocode}

% \macro{\childdoc}
% The deprecated macro |\childdoc| is a legacy version of |\childdocmain|:
%    \begin{macrocode}
\newcommand{\childdoc}{\childdocmain}
%    \end{macrocode}

% \macro{\childdocredirect}
% The deprecated macro |\childdocredirect| is a legacy version
% of |\childdocforward| and |\childdocforwardprefix|:
%    \begin{macrocode}
\newcommand{\childdocredirect}[2][]
{
  \begingroup
    \if?#1?
      \def\childdoctmp{\childdocforward{#2}}
    \else
      \def\childdoctmp{\childdocforwardprefix{#1}{#2}}
    \fi
    \expandafter
  \endgroup
  \childdoctmp
}
%    \end{macrocode}

%\iffalse
%</package>
%\fi
%
\endinput

\childdocof{cdocsamp}
%    \end{macrocode}

%\iffalse
%</samplechap1|samplechap2>
%\fi
%
%\iffalse
%<*samplechap1>
%\fi
% Some text for chapter 1:
%    \begin{macrocode}
\section{one}
some text in chapter one
%    \end{macrocode}

%\iffalse
%</samplechap1>
%\fi
% Some text for chapter 2:
%\iffalse
%<*samplechap2>
%\fi
%    \begin{macrocode}
\section{two}
more text in chapter two
%    \end{macrocode}

%\iffalse
%</samplechap2>
%\fi
%
% %%%%%%%%%%%%%%%%%%%%%%%%%%%%%%%%%%%%%%
% \paragraph{Part Include Files.}
%
% The include files are called |cdocspt3.tex| and |cdocspt4.tex|.
%
%\iffalse
%<*samplepart3|samplepart4>
%\fi

% Optional override for |\version| flag:
%    \begin{macrocode}
%%\providecommand{\version}{final}
%    \end{macrocode}

% Include the main document:
%    \begin{macrocode}
% \iffalse
%
% childdoc.dtx Copyright (C) 2017-2018 Niklas Beisert
%
% This work may be distributed and/or modified under the
% conditions of the LaTeX Project Public License, either version 1.3
% of this license or (at your option) any later version.
% The latest version of this license is in
%   http://www.latex-project.org/lppl.txt
% and version 1.3 or later is part of all distributions of LaTeX
% version 2005/12/01 or later.
%
% This work has the LPPL maintenance status `maintained'.
%
% The Current Maintainer of this work is Niklas Beisert.
%
% This work consists of the files childdoc.dtx and childdoc.ins
% and the derived files childdoc.def and cdocsamp.tex with
% cdocsch1.tex, cdocsch2.tex, cdocsdrf.tex, cdocsfn1.tex, cdocsfn2.tex.
%
%<package>\ifdefined\childdocmain\endinput\fi
%<package>\ProvidesFile{childdoc.def}[2018/12/30 v2.0 child document driver]
%<samplemain>\ProvidesFile{cdocsamp.tex}[2018/12/30 v2.0 sample for childdoc]
%<*driver>
%\ProvidesFile{childdoc.drv}[2018/12/30 v2.0 childdoc reference manual file]
\PassOptionsToClass{10pt,a4paper}{article}
\documentclass{ltxdoc}

\usepackage[margin=35mm]{geometry}
\usepackage{hyperref}
\usepackage{hyperxmp}
\usepackage[usenames]{color}

\hypersetup{colorlinks=true}
\hypersetup{pdfstartview=FitH}
\hypersetup{pdfpagemode=UseNone}
\hypersetup{pdfsource={}}
\hypersetup{pdflang={en-UK}}
\hypersetup{pdfcopyright={Copyright 2017-2018 Niklas Beisert.
  This work may be distributed and/or modified under the
  conditions of the LaTeX Project Public License, either version 1.3
  of this license or (at your option) any later version.}}
\hypersetup{pdflicenseurl={http://www.latex-project.org/lppl.txt}}
\hypersetup{pdfcontactaddress={ETH Zurich, ITP, HIT K,
  Wolfgang-Pauli-Strasse 27}}
\hypersetup{pdfcontactpostcode={8093}}
\hypersetup{pdfcontactcity={Zurich}}
\hypersetup{pdfcontactcountry={Switzerland}}
\hypersetup{pdfcontactemail={nbeisert@itp.phys.ethz.ch}}
\hypersetup{pdfcontacturl={http://people.phys.ethz.ch/\xmptilde nbeisert/}}

\newcommand{\secref}[1]{\hyperref[#1]{section \ref*{#1}}}

\parskip1ex
\parindent0pt
\let\olditemize\itemize
\def\itemize{\olditemize\parskip0pt}

\begin{document}

\title{The \textsf{childdoc} Package}
\hypersetup{pdftitle={The childdoc Package}}
\author{Niklas Beisert\\[2ex]
  Institut f\"ur Theoretische Physik\\
  Eidgen\"ossische Technische Hochschule Z\"urich\\
  Wolfgang-Pauli-Strasse 27, 8093 Z\"urich, Switzerland\\[1ex]
  \href{mailto:nbeisert@itp.phys.ethz.ch}
  {\texttt{nbeisert@itp.phys.ethz.ch}}}
\hypersetup{pdfauthor={Niklas Beisert}}
\hypersetup{pdfsubject={Manual for the LaTeX2e Package childdoc}}
\date{30 December 2018, \textsf{v2.0}}
\maketitle

\begin{abstract}\noindent
\textsf{childdoc} is a \LaTeXe{} package
that enables the direct compilation
of document sections included by |\include|
to individual files.
\end{abstract}

\begingroup
\parskip0ex
\tableofcontents
\endgroup

%%%%%%%%%%%%%%%%%%%%%%%%%%%%%%%%%%%%%%%%%%%%%%%%%%%%%%%%%%%%%%%%%%%%%%%%%%%%%%%%
%%%%%%%%%%%%%%%%%%%%%%%%%%%%%%%%%%%%%%%%%%%%%%%%%%%%%%%%%%%%%%%%%%%%%%%%%%%%%%%%
\section{Introduction}

\LaTeX{} provides a mechanism to structure a large document (such as a book)
into a main file and several child files (containing the chapters)
using the |\include| command.
This mechanism is beneficial for documents
which span hundreds of pages in order to
make the source file(s) more manageable.
Moreover, compilation can be restricted to
selected child files by means of the |\includeonly| command.
The latter feature can be used to reduce the compilation time while editing
(this was significantly more useful in the earlier days of \LaTeX{})
or to generate a smaller document which is easier to navigate.
Another application of |\includeonly| is to generate
documents consisting of selected parts of the complete document.

However, there are a few drawbacks of the plain |\include| mechanism:
\begin{itemize}
\item
The child files cannot be compiled on their own,
they can only be compiled via the main file.
A naive editing environment
(such as a text editor with an option
to have the current file processed by \LaTeX)
may require one to switch to the main file before compiling;
attempting to compile the child file produces errors.
\item
The main file must be modified (each time)
to adjust the |\includeonly| command
to the present needs. This easily leaves the main file in a messy state.
\item
The generated document will always carry the filename
of the main document. This is inconvenient if
several child files are to be compiled and
to be kept for distribution.
\end{itemize}

The present package provides a simple interface
to make child files individually compilable by \LaTeX{}.
Compiling a child file then has the same effect as compiling
the main file with an |\includeonly| command
to select the appropriate child.
Moreover the generated document will carry the name of the child
rather than the main file.
This resolves all three above issues.

This feature is meant to make the editing of books,
thesis documents and lecture notes somewhat more convenient.
However, the package can also be used efficiently for
composing a series of documents (such as exercise sheets)
which are typically distributed individually.
It then assists the author in generating the individual documents
(potentially in different versions)
as well as a document containing the collected series.
Another application is in developing style files
or other kinds of included material
where compilation of the style file could redirect
to a sample or test file.

%%%%%%%%%%%%%%%%%%%%%%%%%%%%%%%%%%%%%%%%%%%%%%%%%%%%%%%%%%%%%%%%%%%%%%%%%%%%%%%%
%%%%%%%%%%%%%%%%%%%%%%%%%%%%%%%%%%%%%%%%%%%%%%%%%%%%%%%%%%%%%%%%%%%%%%%%%%%%%%%%
\section{Usage}

First of all, the package \textsf{childdoc} is \emph{not} a standard
\LaTeXe{} |.sty| style file! Therefore it needs to be invoked in
a non-standard way.

%%%%%%%%%%%%%%%%%%%%%%%%%%%%%%%%%%%%%%%%%%%%%%%%%%%%%%%%%%%%%%%%%%%%%%%%%%%%%%%%
\subsection{Included Files}
\label{sec:include}

%%%%%%%%%%%%%%%%%%%%%%%%%%%%%%%%%%%%%%%%
\DescribeMacro{\childdocmain}
To use the package, add the commands
\begin{center}
\begin{tabular}{l}
|\input{childdoc.def}|\\
|\childdocmain{}|\\
\end{tabular}
\end{center}
at the very top of the main \LaTeX{} file,
in particular \emph{before} the |\documentclass| statement!
The argument of |\childdocmain| should be left empty
(but it must be present).

%%%%%%%%%%%%%%%%%%%%%%%%%%%%%%%%%%%%%%%%
\DescribeMacro{\childdocof}
Furthermore, add the commands
\begin{center}
\begin{tabular}{l}
|\input{childdoc.def}|\\
|\childdocof{|\textit{main}|}|\\
\end{tabular}
\end{center}
at the top of every child file \textit{child}
which is included by |\include{|\textit{child}|}|
from within the main file
(or at least for those files to be compiled individually).
The argument \textit{main} must be the filename of the main file.

There are a couple of
considerations in setting up the main and child documents:

%%%%%%%%%%%%%%%%%%%%%%%%%%%%%%%%%%%%%%%%
\paragraph{Restrictions.}

Please note the following restrictions:
\begin{itemize}
\item
|\childdocmain| must be called with one argument \textit{main}
to ensure compatibility with earlier version of the package.
It must either be empty (|\childdocmain{}|)
or precisely match the filename of the main file in which it is specified.
See \secref{sec:detection} for further information.
\item
The filename \textit{main} must be specified without the |.tex| extension.
\item
The filename \textit{main} is case sensitive
(even in case-insensitive file systems)
due to internal string comparison.
\item
The argument \textit{main} should be fully expanded, it cannot be a macro.
\item
Subdirectories and special characters should be avoided in filenames.
\item
The command |\childdocmain{|\textit{main}|}| must be followed by a whitespace.
It should not be followed immediately by another command
or by a comment mark `|%|'.
This is because the \TeX{} parser reads the token immediately following
the argument of |\childdocmain| and puts it
at the beginning of every child section;
however, a white\-space is ignored.
\end{itemize}

%%%%%%%%%%%%%%%%%%%%%%%%%%%%%%%%%%%%%%%%
\paragraph{Content of Main File.}

It is advisable to place all content in the child files included by |\include|.
Any output contained in the main file will appear in all child documents
unless suppressed manually;
it cannot be suppressed automatically by the |\includeonly| directive
and thus should normally be avoided.
A method to include some content in the main file
by means of conditional processing is described in \secref{sec:conditional}.

%%%%%%%%%%%%%%%%%%%%%%%%%%%%%%%%%%%%%%%%
\paragraph{Page Numbering.}

When only a part of the document is compiled,
the appropriate numbering of pages
(as well as other status parameters)
is determined from the |.aux| files.
The latter contain information from previous passes.
However this information needs to propagate through
all intermediate child documents.
Therefore the page numbering in child documents may well
be inconsistent until the complete document is compiled at least once.

A useful (if unconventional) way to always ensure a consistent
page numbering is to restart the numbering in each child document
and denote the pages by `\textit{child}|.|\textit{page}'
where \textit{child} represents the chapter/section number of the child file.
This can be achieved by the command
|\numberwithin{page}{|\textit{child}|}|
of the \textsf{amsmath} package
where \textit{child} can be |chapter| or |section|
depending on the chosen structuring.
Alternatively, one can modify the macro |\thepage| appropriately
and reset the counter |page| at the start of each child file.

%%%%%%%%%%%%%%%%%%%%%%%%%%%%%%%%%%%%%%%%%%%%%%%%%%%%%%%%%%%%%%%%%%%%%%%%%%%%%%%%
\subsection{Conditional Processing}
\label{sec:conditional}

The package provides a mechanism to compile different versions
of a document. To customise the versions further some conditional processing
can come in handy to distinguish which version is being compiled.
The package provides two macros to describe the compilation context:

%%%%%%%%%%%%%%%%%%%%%%%%%%%%%%%%%%%%%%%%
\DescribeMacro{\ifchilddoc}
The conditional |\ifchilddoc| distinguishes between the compilation of
child documents and the main document:
%
\begin{center}
|\ifchilddoc |\textit{child-code}| |[|\||else |\textit{main-code}]| \||fi|
\end{center}

%%%%%%%%%%%%%%%%%%%%%%%%%%%%%%%%%%%%%%%%
\DescribeMacro{\childdocname}
\DescribeMacro{\childdocjob}
The macro |\childdocname| contains the filename (without extension)
of the main or child file being processed.
Note that |\childdocjob| will always contain the name of the main file.

%%%%%%%%%%%%%%%%%%%%%%%%%%%%%%%%%%%%%%%%
\paragraph{Title Page.}

Conditional processing can be used to include a title or banner page
in the main document when proper precautions are taken.
Importantly, the code in the main file should ensure that the page counter
(as well as other status parameters which are stored in the |.aux| files)
takes the same value after the conditional processing.
Otherwise the page numbers may take divergent values
depending on which part is compiled.

For example, a title page could be declared by:
%
\begin{center}
\begin{tabular}{l}
|\ifchilddoc\||else|\\
|\addtocounter{page}{-1}|\\
\textit{code for title page}\\
|\newpage|\\
|\||fi|
\end{tabular}
\end{center}
%
A banner page for the child documents can be generated by:
%
\begin{center}
\begin{tabular}{l}
|\ifchilddoc|\\
|\addtocounter{page}{-1}|\\
\textit{code for banner page}\\
|\newpage|\\
|\||fi|
\end{tabular}
\end{center}
%
Here one could write a message such as:
\begin{center}
|This is the part \childdocname{} of \childdocjob{}.|
\end{center}

%%%%%%%%%%%%%%%%%%%%%%%%%%%%%%%%%%%%%%%%%%%%%%%%%%%%%%%%%%%%%%%%%%%%%%%%%%%%%%%%
\subsection{Flags}
\label{sec:flags}

The package makes it easy to generate different versions
of the main or child documents.
To this end compilation flags can be defined
and assigned different default values.
They will be particularly useful in conjunction
with the forwarding mechanism described in \secref{sec:forward}.

For example, it may be useful to have a flag |\version|
which can be set to |draft| or |final|.
The document source will contain some conditional code
depending on the value of |\version|.
Suppose further, the flag should default to |final| for the main file
and to |draft| for child files
which is a natural assignment for editing the document.
This is achieved by placing the following code
in the preamble of the main document
(below the |\childdocmain| directive):
%
\begin{center}
\begin{tabular}{l}
|\ifchilddoc|\\
|\providecommand{\version}{draft}|\\
|\||else|\\
|\providecommand{\version}{final}|\\
|\||fi|
\end{tabular}
\end{center}
%
The definition by |\providecommand| makes sure
that previous definitions are not overwritten.
Further statements |\providecommand{\version}{...}|
can thus be added before the above code to override it.

For the main file, one might add a line
(between |\childdocmain| and the above block)
%
\begin{center}
|%\ifchilddoc\||else\providecommand{\version}{draft}\||fi|
\end{center}
%
which can be uncommented to produce a draft version.
Likewise one can add a line to the very top of a child file
(above the |\childdocof{|\textit{main}|}| directive)
%
\begin{center}
|%\providecommand{\version}{final}|
\end{center}
%
which can be uncommented to produce the final version of this child document.

%%%%%%%%%%%%%%%%%%%%%%%%%%%%%%%%%%%%%%%%%%%%%%%%%%%%%%%%%%%%%%%%%%%%%%%%%%%%%%%%
\subsection{Forwarding}
\label{sec:forward}

Different versions of the main or child documents
using compilation flags as described in \secref{sec:flags}
can be (permanently) stored in different files
for convenient compilation, viewing and distribution.
To this end, the package defines a command
to pass on compilation to a different file:

%%%%%%%%%%%%%%%%%%%%%%%%%%%%%%%%%%%%%%%%
\DescribeMacro{\childdocforward}
The command |\childdocforward| redirects processing to
another source file:
%
\begin{center}
\begin{tabular}{l}
|\input{childdoc.def}|\\
|\childdocforward[|\textit{main}|]{|\textit{dest}|}|\\
\end{tabular}
\end{center}
%
The argument \textit{dest} is the destination file
(without extension).
It should be the main file or one of the child files.
Note that further \textsf{childdoc} directives
such as |\childdocof| and |\childdocforward|
in the indicated file will be processed in this form.
The optional argument \textit{main}
passes on directly to the main file \textit{main}
while pretending to compile the child \textit{dest}.
This form behaves as if \textit{dest}
issues |\childdocof{|\textit{main}|}| right away,
and no further \textsf{childdoc} directives will be processed.

%%%%%%%%%%%%%%%%%%%%%%%%%%%%%%%%%%%%%%%%
\DescribeMacro{\...prefix}
In the alternative form |\childdocforwardprefix|,
%
\begin{center}
\begin{tabular}{l}
|\input{childdoc.def}|\\
|\childdocforwardprefix[|\textit{main}|]{|\textit{prefix}|}{|\textit{dest}|}|
\end{tabular}
\end{center}
%
the destination file is determined by a pattern
depending on the current file:
To make this work, the current file must be called
`{\textit{prefix}\hspace{0.2em}\textit{suffix}}'
with \textit{prefix} matching precisely the argument.
Processing is then passed on to the file
`{\textit{dest}\hspace{0.2em}\textit{suffix}}'.
Surely, the same effect is achieved by
directly specifying the
argument `{\textit{dest}\hspace{0.2em}\textit{suffix}}'
in the first form.
However, that requires to set up a different file
for each child. With the alternative form of the command
all these files can have exactly the same content
which simplifies setting them up and maintaining them.

For example, the following file |draft.tex|
with a compilation flag |\version| as described in \secref{sec:flags}
compiles the main document as a draft:
%
\begin{center}
\begin{tabular}{l}
|\def\version{draft}|\\
|\input{childdoc.def}|\\
|\childdocforward{|\textit{main}|}|
\end{tabular}
\end{center}
%
Likewise, the following files |final|\textit{nn}|.tex|
compile the final version of the child document
|child|\textit{nn}|.tex|:
%
\begin{center}
\begin{tabular}{l}
|\def\version{final}|\\
|\input{childdoc.def}|\\
|\childdocforwardprefix{final}{child}|
\end{tabular}
\end{center}
%

Note that when several versions of a main file and/or of each child file
are to be generated, it may be convenient to set up a |Makefile| or
shell script to automatise the process.

%%%%%%%%%%%%%%%%%%%%%%%%%%%%%%%%%%%%%%%%%%%%%%%%%%%%%%%%%%%%%%%%%%%%%%%%%%%%%%%%
\subsection{Command Line Processing}
\label{sec:commandline}

The effect of redirection files can also be achieved by invoking
the \LaTeX{} compiler with a more elaborate command line.
Most conveniently this should be done as part
of a shell script or a |Makefile|.

When using \textsf{childdoc} in the main file, the following
command lines effectively perform a redirection
(note that depending on the shell being used,
backslashes may have to be doubled: `|\|' $\to$ `|\\|'):
%
\begin{center}
|... -jobname "|\textit{target}|" |\\|"|[\textit{flags}]%
|\input{childdoc.def}\childdocforward[|\textit{main}|]{|\textit{dest}|}"|
\end{center}
%
Here \textit{target} is the name of the output file,
\textit{main} is the name of the main file
and \textit{dest} is the name of the main or child file to be processed
(all filenames without extensions).
The optional argument \textit{main} can be omitted
if \textit{main} matches \textit{dest}.
Optionally, compilation \textit{flags} can be defined via |\def| commands.
This command line makes the \TeX{} engine believe
it is compiling the file \textit{target}
whose content is specified as the latter parameter.
The provided code then forwards the processing to
\textit{main} or \textit{dest} as described in \secref{sec:forward}.

%%%%%%%%%%%%%%%%%%%%%%%%%%%%%%%%%%%%%%%%%%%%%%%%%%%%%%%%%%%%%%%%%%%%%%%%%%%%%%%%
\subsection{Include by Input}
\label{sec:input}

Including child documents by |\include| has some restrictions by design.
Most notably, the content of a child document always occupies
its own set of pages; pages cannot be shared between child documents.
Usually, this behaviour makes perfect sense
because each child document contain an essential part of the document.
However, in some situations it may be desirable to compose
a document from a collection of parts
without having mandatory page breaks between then.
For this case, the package
provides a mechanism to include parts
by |\input| which can also be processed individually.
However, by construction this mechanism
requires manual handling of the content to be output.

%%%%%%%%%%%%%%%%%%%%%%%%%%%%%%%%%%%%%%%%
\DescribeMacro{\ifchilddocmanual}
The main file should be prepared as usual, see \secref{sec:include}.
However, the document body must make a distinction
between processing of an individual part and of the main document, e.g.:
%
\begin{center}
\begin{tabular}{l}
|\ifchilddocmanual|\\
|\input{\childdocname}|\\
|\||else|\\
\textit{document body with }|\input{|\textit{part}|}|\\
|\||fi|
\end{tabular}
\end{center}
%
The conditional |\ifchilddocmanual| is true whenever
a part to be included by |\input| is being compiled,
and the name of the part is stored in |\childdocname|.

%%%%%%%%%%%%%%%%%%%%%%%%%%%%%%%%%%%%%%%%
\DescribeMacro{\childdocby}
Each part to be included by |\input| should start with:
%
\begin{center}
\begin{tabular}{l}
|\input{childdoc.def}|\\
|\childdocby{|\textit{main}|}|\\
\end{tabular}
\end{center}
%
The directive |\childdocby| is similar to |\childdocof|
described in \secref{sec:include},
but the subsequent selection of content must be done manually.
To that end, both |\ifchilddoc| and |\ifchilddocmanual|
will be true upon processing of a part,
and the name of the part is stored in |\childdocname|.
Note that |\jobname| will be set to the filename of the current part
so that each part receives an individual |.aux| file
that does not interfere with the |.aux| file(s) of the main document.
This behaviour can be altered by the alternative form
|\childdocby[*]{|\textit{main}|}| (with a non-empty optional argument)
which uses the |.aux| file of the main document
by setting |\jobname| to \textit{main}.

%%%%%%%%%%%%%%%%%%%%%%%%%%%%%%%%%%%%%%%%%%%%%%%%%%%%%%%%%%%%%%%%%%%%%%%%%%%%%%%%
\subsection{Driver Development}
\label{sec:driver}

The \textsf{childdoc} mechanism can also be use for the development
of definition files such as \LaTeX{} styles or classes.
This case differs from the above setup with multiple parts
included by |\include| in that no |\includeonly| should be invoked.
This can be achieved by starting the include file
(before |\ProvidesPackage|) with:
%
\begin{center}
\begin{tabular}{l}
|\input{childdoc.def}|\\
|\childdocforward{|\textit{main}|}|\\
\end{tabular}
\end{center}
%
or alternatively with:
%
\begin{center}
\begin{tabular}{l}
|\input{childdoc.def}|\\
|\childdocby{|\textit{main}|}|\\
\end{tabular}
\end{center}
%
Both forms have slightly different effects as described above.
The main file is prepared as usual, see \secref{sec:include}.

%%%%%%%%%%%%%%%%%%%%%%%%%%%%%%%%%%%%%%%%%%%%%%%%%%%%%%%%%%%%%%%%%%%%%%%%%%%%%%%%
\subsection{Legacy Detection}
\label{sec:detection}

The directive |\childdocmain| in the main file can detect
whether the complete document or merely a child is to be compiled
even without using the directive |\childdocof|.
This method is deprecated because it is less robust
and there is no compelling reason to use it;
it is merely provided for backward compatibility
and it may be removed in future versions.

If the detection mechanism is to be used,
it is mandatory to correctly specify
the filename of the main file as the argument of |\childdocmain|:
%
\begin{center}
\begin{tabular}{l}
|\input{childdoc.def}|\\
|\childdocmain{|\textit{main}|}|\\
\end{tabular}
\end{center}
%
If |\jobname| does not match the argument \textit{main} of |\childdocmain|,
it is assumed that |\jobname| points to the child file to be compiled.
When using |\childdocmain| with the main file specified as argument,
it suffices to start a child file
with just |\input{|\textit{main}|}|
without loading of the package and using |\childdocof|.
If instead all processing is done
with the appropriate \textsf{childdoc} directives,
the argument of \textit{main} of |\childdocmain| can be empty.

An alternative version of the command line processing described
in \secref{sec:commandline} using the detection mechanism reads:
%
\begin{center}
|... -jobname "|\textit{target}|" "|[\textit{flags}]%
[|\def\jobname{|\textit{dest}|}|]|\input{|\textit{main}|}"|
\end{center}

%%%%%%%%%%%%%%%%%%%%%%%%%%%%%%%%%%%%%%%%%%%%%%%%%%%%%%%%%%%%%%%%%%%%%%%%%%%%%%%%
\subsection{Manual Code}
\label{sec:manual}

In case one cannot be certain whether the definitions file |childdoc.def|
is installed on the target \TeX{} distribution
and one prefers not to ship it,
it is conceivable to paste a few relevant commands into the sources.

To that end, drop all statements |\input{childdoc.def}|
and perform the replacements as outlined below.
Instead of |\childdocmain{|\textit{main}|}| add the following code
to the top of the main file:
%
\begin{center}
\begin{tabular}{l}
|\||ifdefined\childdocname\endinput\||fi\newif\ifchilddoc|\\
|\edef\childdocname{\scantokens\expandafter{\jobname\noexpand}}|\\
|\def\childdocmain{|\textit{main}|}\||ifx\childdocmain\childdocname\||else|\\
|\childdoctrue\includeonly{\childdocname}\let\jobname\childdocmain\||fi|\\
\end{tabular}
\end{center}
%
Instead of |\childdocof{|\textit{main}|}| just include the main file
at the top of each child file:
%
\begin{center}
|\input{|\textit{main}|}|
\end{center}
%
A simple redirection |\childdocforward{|\textit{dest}|}| is achieved by:
%
\begin{center}
|\def\jobname{|\textit{dest}|}\input{\jobname}|
\end{center}
%
The redirection with prefix
|\childdocforwardprefix[|\textit{prefix}|]{|\textit{dest}|}|
is accomplished by:
%
\begin{center}
\begin{tabular}{l}
|{\edef\jobname{\scantokens\expandafter{\jobname\noexpand}}|\\
|\def\redirectjob |\textit{prefix}|#1~~~{\gdef\jobname{|\textit{dest}|#1}}|\\
|\expandafter\redirectjob\jobname~~~}\input{\jobname}|
\end{tabular}
\end{center}

In an alternative approach,
child documents can be compiled by a specific command line
without additional code or specific definitions:
%
\begin{center}
|... -jobname "|\textit{target}|" "|[\textit{flags}]%
|\includeonly{|\textit{dest}|}\input{|\textit{main}|}"|
\end{center}
%

%%%%%%%%%%%%%%%%%%%%%%%%%%%%%%%%%%%%%%%%%%%%%%%%%%%%%%%%%%%%%%%%%%%%%%%%%%%%%%%%
%%%%%%%%%%%%%%%%%%%%%%%%%%%%%%%%%%%%%%%%%%%%%%%%%%%%%%%%%%%%%%%%%%%%%%%%%%%%%%%%
\section{Information}

%%%%%%%%%%%%%%%%%%%%%%%%%%%%%%%%%%%%%%%%%%%%%%%%%%%%%%%%%%%%%%%%%%%%%%%%%%%%%%%%
\subsection{Copyright}

Copyright \copyright{} 2017--2018 Niklas Beisert

This work may be distributed and/or modified under the
conditions of the \LaTeX{} Project Public License, either version 1.3
of this license or (at your option) any later version.
The latest version of this license is in
  \url{http://www.latex-project.org/lppl.txt}
and version 1.3 or later is part of all distributions of \LaTeX{}
version 2005/12/01 or later.

This work has the LPPL maintenance status `maintained'.

The Current Maintainer of this work is Niklas Beisert.

This work consists of the files |README.txt|, |childdoc.ins| and |childdoc.dtx|
as well as the derived files |childdoc.def|, |cdocsamp.tex|
with |cdocsch1.tex|, |cdocsch2.tex|, |cdocspt3.tex|, |cdocspt4.tex|,
|cdocsdrf.tex|, |cdocsfn1.tex|, |cdocsfn2.tex|
as well as |childdoc.pdf|.

%%%%%%%%%%%%%%%%%%%%%%%%%%%%%%%%%%%%%%%%%%%%%%%%%%%%%%%%%%%%%%%%%%%%%%%%%%%%%%%%
\subsection{Files and Installation}

The package consists of the files:
%
\begin{center}
\begin{tabular}{ll}
    |README.txt|   & readme file \\
    |childdoc.ins| & installation file \\
    |childdoc.dtx| & source file \\
    |childdoc.def| & definition file \\
    |cdocsamp.tex| & sample main file \\
    |cdocsch1.tex| & sample include file \\
    |cdocsch2.tex| & sample include file \\
    |cdocspt3.tex| & sample part file \\
    |cdocspt4.tex| & sample part file \\
    |cdocsdrf.tex| & sample redirection file \\
    |cdocsfn1.tex| & sample redirection file \\
    |cdocsfn2.tex| & sample redirection file \\
    |childdoc.pdf| & manual
\end{tabular}
\end{center}
%
The distribution consists of the files
|README.txt|, |childdoc.ins| and |childdoc.dtx|.
%
\begin{itemize}
\item
Run (pdf)\LaTeX{} on |childdoc.dtx|
to compile the manual |childdoc.pdf| (this file).
\item
Run \LaTeX{} on |childdoc.ins| to create the definitions file |childdoc.def|
and the sample |cdocsamp.tex| with include files
|cdocsch1.tex|, |cdocsch2.tex|, |cdocspt3.tex|, |cdocspt4.tex|,
|cdocsdrf.tex|, |cdocsfn1.tex|, |cdocsfn2.tex|.
Then copy the file |childdoc.def| to an appropriate directory of your \LaTeX{}
distribution, e.g.\ \textit{texmf-root}|/tex/latex/childdoc|.
\end{itemize}

%%%%%%%%%%%%%%%%%%%%%%%%%%%%%%%%%%%%%%%%%%%%%%%%%%%%%%%%%%%%%%%%%%%%%%%%%%%%%%%%
\subsection{Related CTAN Packages}

There are several other packages which offer a similar functionality:
%
\begin{itemize}
\item
The packages
\href{http://ctan.org/pkg/docmute}{\textsf{docmute}},
\href{http://ctan.org/pkg/includex}{\textsf{includex}} and
\href{http://ctan.org/pkg/standalone}{\textsf{standalone}}
provide commands to include only the document body of
a child file thus allowing both files to be compiled individually.
\item
The packages \href{http://ctan.org/pkg/subdocs}{\textsf{subdocs}}
and \href{http://ctan.org/pkg/subfiles}{\textsf{subfiles}}
provide structures in which the main and child documents can be
encapsulated and allowing them to be compiled individually.
The inclusion mechanism is different from the conventional |\include|.
\item
The package \href{http://ctan.org/pkg/combine}{\textsf{combine}}
is an elaborate solution to combine several documents into one.
\end{itemize}
%
See also the CTAN topic \href{http://ctan.org/topic/subdocs}{\textsf{subdocs}}
for further related packages.
The present package differs from the above solutions in that
a document structure constructed with the conventional |\include| mechanism
just needs two extra commands at the top of every file
such that all constituent files can be compiled individually.

%%%%%%%%%%%%%%%%%%%%%%%%%%%%%%%%%%%%%%%%%%%%%%%%%%%%%%%%%%%%%%%%%%%%%%%%%%%%%%%%
%\subsection{Feature Suggestions}
%
%The following is a list of features which may be useful for future
%versions of this package:
%%
%\begin{itemize}
%\item
%\ldots
%\end{itemize}

%%%%%%%%%%%%%%%%%%%%%%%%%%%%%%%%%%%%%%%%%%%%%%%%%%%%%%%%%%%%%%%%%%%%%%%%%%%%%%%%
\subsection{Revision History}

%%%%%%%%%%%%%%%%%%%%%%%%%%%%%%%%%%%%%%%%
\paragraph{v2.0:} 2018/12/30

\begin{itemize}
\item
immediate forward processing
\item
added |\childdocby| mechanism
\item
manual restructured
\end{itemize}

%%%%%%%%%%%%%%%%%%%%%%%%%%%%%%%%%%%%%%%%
\paragraph{v1.6:} 2018/01/17

\begin{itemize}
\item
application for development of include files
\item
corrections to manual
\end{itemize}

%%%%%%%%%%%%%%%%%%%%%%%%%%%%%%%%%%%%%%%%
\paragraph{v1.5:} 2017/05/21

\begin{itemize}
\item
more complete structuring introduced
\item
|\childdocof| introduced
\item
|\childdoc| renamed to |\childdocmain|
\item
|\childredirect| renamed to |\childdocforward| and |\childdocforwardprefix|
and functionality expanded
\end{itemize}

%%%%%%%%%%%%%%%%%%%%%%%%%%%%%%%%%%%%%%%%
\paragraph{v1.0:} 2017/04/27

\begin{itemize}
\item
manual and install package
\item
first version published on CTAN
\end{itemize}

%%%%%%%%%%%%%%%%%%%%%%%%%%%%%%%%%%%%%%%%
\paragraph{v0.6:} 2017/04/26

\begin{itemize}
\item
redirection mechanism added
\end{itemize}

%%%%%%%%%%%%%%%%%%%%%%%%%%%%%%%%%%%%%%%%
\paragraph{v0.5:} 2017/04/26

\begin{itemize}
\item
functionality in definition file
\end{itemize}


%%%%%%%%%%%%%%%%%%%%%%%%%%%%%%%%%%%%%%%%%%%%%%%%%%%%%%%%%%%%%%%%%%%%%%%%%%%%%%%%
%%%%%%%%%%%%%%%%%%%%%%%%%%%%%%%%%%%%%%%%%%%%%%%%%%%%%%%%%%%%%%%%%%%%%%%%%%%%%%%%
%%%%%%%%%%%%%%%%%%%%%%%%%%%%%%%%%%%%%%%%%%%%%%%%%%%%%%%%%%%%%%%%%%%%%%%%%%%%%%%%
\appendix

\settowidth\MacroIndent{\rmfamily\scriptsize 000\ }

 \DocInput{childdoc.dtx}

\end{document}
%</driver>
% \fi
%
% %%%%%%%%%%%%%%%%%%%%%%%%%%%%%%%%%%%%%%%%%%%%%%%%%%%%%%%%%%%%%%%%%%%%%%%%%%%%%%
% %%%%%%%%%%%%%%%%%%%%%%%%%%%%%%%%%%%%%%%%%%%%%%%%%%%%%%%%%%%%%%%%%%%%%%%%%%%%%%
% \section{Sample}
%\iffalse
%<*samplemain>
%\fi
%
% The following presents a sample document
% with two chapters, two parts, a title page,
% a compile flag as well as three forwarding files to set the flag.
% It consists of eight |.tex| files:
% \begin{center}
% \begin{tabular}{ll}
% |cdocsamp.tex|&main file\\
% |cdocsch1.tex|&include file for chapter 1\\
% |cdocsch2.tex|&include file for chapter 2\\
% |cdocspt3.tex|&include file for part 3\\
% |cdocspt4.tex|&include file for part 4\\
% |cdocsdrf.tex|&forwarding file for main file in draft mode\\
% |cdocsfi1.tex|&forwarding file for final version of chapter 1\\
% |cdocsfi2.tex|&forwarding file for final version of chapter 2\\
% \end{tabular}
% \end{center}
% Each of the eight files can be compiled directly by the \LaTeX{} compiler.
%
% %%%%%%%%%%%%%%%%%%%%%%%%%%%%%%%%%%%%%%
% \paragraph{Main File.}
%
% The main file is called |cdocsamp.tex|.
%
% Load the \textsf{childdoc} definitions and
% declare the filename for the main document:
%    \begin{macrocode}
\input{childdoc.def}
\childdocmain{}
%    \end{macrocode}

% Optional override for |\version| flag:
%    \begin{macrocode}
%%\ifchilddoc\else\providecommand{\version}{draft}\fi
%    \end{macrocode}

% Define the default values for the |\version| flag
% (|final| for the main file and |draft| for childs):
%    \begin{macrocode}
\ifchilddoc
\providecommand{\version}{draft}
\else
\providecommand{\version}{final}
\fi
%    \end{macrocode}

% Load the standard document class:
%    \begin{macrocode}
\documentclass[12pt]{article}
%    \end{macrocode}

% Start the document body:
%    \begin{macrocode}
\begin{document}
%    \end{macrocode}

% Declare a title page.
% Print title, part of document being processed and version flag:
%    \begin{macrocode}
\addtocounter{page}{-1}
\begin{center}
{\LARGE\bfseries{}childdoc example\par}
\vspace{1cm}
\ifchilddoc
\ifchilddocmanual part\else chapter\fi:
`\childdocname' of `\childdocjob'\par
\else
main document: `\childdocjob'\par
\fi
version: \version\par
\end{center}
\newpage
%    \end{macrocode}

% Manually include selected file,
% otherwise process as usual:
%    \begin{macrocode}
\ifchilddocmanual
\section*{part `\childdocname'}
\input{\childdocname}
\else
%    \end{macrocode}

% Include the two chapters:
%    \begin{macrocode}
\include{cdocsch1}
\include{cdocsch2}
%    \end{macrocode}

% Include the two parts unless only chapters should be displayed:
%    \begin{macrocode}
\ifchilddoc\else
\section{part three}
\input{cdocspt3}
\section{part four}
\input{cdocspt4}
\fi
%    \end{macrocode}

% Process as usual until here:
%    \begin{macrocode}
\fi
%    \end{macrocode}

% End of document body:
%    \begin{macrocode}
\end{document}
%    \end{macrocode}
%\iffalse
%</samplemain>
%\fi
%
% %%%%%%%%%%%%%%%%%%%%%%%%%%%%%%%%%%%%%%
% \paragraph{Chapter Include Files.}
%
% The include files are called |cdocsch1.tex| and |cdocsch2.tex|.
%
%\iffalse
%<*samplechap1|samplechap2>
%\fi

% Optional override for |\version| flag:
%    \begin{macrocode}
%%\providecommand{\version}{final}
%    \end{macrocode}

% Include the main document:
%    \begin{macrocode}
\input{childdoc.def}
\childdocof{cdocsamp}
%    \end{macrocode}

%\iffalse
%</samplechap1|samplechap2>
%\fi
%
%\iffalse
%<*samplechap1>
%\fi
% Some text for chapter 1:
%    \begin{macrocode}
\section{one}
some text in chapter one
%    \end{macrocode}

%\iffalse
%</samplechap1>
%\fi
% Some text for chapter 2:
%\iffalse
%<*samplechap2>
%\fi
%    \begin{macrocode}
\section{two}
more text in chapter two
%    \end{macrocode}

%\iffalse
%</samplechap2>
%\fi
%
% %%%%%%%%%%%%%%%%%%%%%%%%%%%%%%%%%%%%%%
% \paragraph{Part Include Files.}
%
% The include files are called |cdocspt3.tex| and |cdocspt4.tex|.
%
%\iffalse
%<*samplepart3|samplepart4>
%\fi

% Optional override for |\version| flag:
%    \begin{macrocode}
%%\providecommand{\version}{final}
%    \end{macrocode}

% Include the main document:
%    \begin{macrocode}
\input{childdoc.def}
\childdocby{cdocsamp}
%    \end{macrocode}

%\iffalse
%</samplepart3|samplepart4>
%\fi
%
%\iffalse
%<*samplepart3>
%\fi
% Some text for part 3:
%    \begin{macrocode}
some text in part three
%    \end{macrocode}

%\iffalse
%</samplepart3>
%\fi
% Some text for part 4:
%\iffalse
%<*samplepart4>
%\fi
%    \begin{macrocode}
more text in part four
%    \end{macrocode}

%\iffalse
%</samplepart4>
%\fi
%
% %%%%%%%%%%%%%%%%%%%%%%%%%%%%%%%%%%%%%%
% \paragraph{Forwarding for a Complete Draft.}
%
% The following forwarding file |cdocsdrf.tex|
% compiles the main document in draft mode:
%\iffalse
%<*sampledraft>
%\fi
%    \begin{macrocode}
\def\version{draft}
\input{childdoc.def}
\childdocforward{cdocsamp}
%    \end{macrocode}

%\iffalse
%</sampledraft>
%\fi
%
% %%%%%%%%%%%%%%%%%%%%%%%%%%%%%%%%%%%%%%
% \paragraph{Forwarding for Final Version of the Chapters.}
%
% The following forwarding files |cdocsfn1.tex| and |cdocsfn2.tex|
% (with identical content)
% compile the final versions of the child documents
% |cdocsch1.tex| and |cdocsch2.tex|, respectively:
%\iffalse
%<*samplefinal>
%\fi
%    \begin{macrocode}
\def\version{final}
\input{childdoc.def}
\childdocforwardprefix[cdocsamp]{cdocsfn}{cdocsch}
%    \end{macrocode}

%\iffalse
%</samplefinal>
%\fi
%
% %%%%%%%%%%%%%%%%%%%%%%%%%%%%%%%%%%%%%%
% \paragraph{Command Line Processing.}
%
% The following three command lines generate the output files
% |cdocscld|, |cdocscl1| and |cdocscl2|
% which should be identical to
% |cdocsdrf|, |cdocsch1| and |cdocsfn2|, respectively:
% \begin{center}
% \begin{tabular}{l}
% |latex -jobname cdocscld \|\\
% |  "\def\version{draft}\input{childdoc.def}\childdocforward{cdocsamp}"|\\
% |latex -jobname cdocscl1 \|\\
% |  "\input{childdoc.def}\childdocforward[cdocsamp]{cdocsch1}"|\\
% |latex -jobname cdocscl2 \|\\
% |  "\def\version{final}\input{childdoc.def}\childdocforward{cdocsch2}"|
% \end{tabular}
% \end{center}
% Note that the trailing backslash on each first line
% merely continues the input to the second line
% (for convenient cut ant paste).
% Furthermore, the command |latex| can be replaced by any
% of its alternative versions such as |pdflatex|.
%
% %%%%%%%%%%%%%%%%%%%%%%%%%%%%%%%%%%%%%%%%%%%%%%%%%%%%%%%%%%%%%%%%%%%%%%%%%%%%%%
% %%%%%%%%%%%%%%%%%%%%%%%%%%%%%%%%%%%%%%%%%%%%%%%%%%%%%%%%%%%%%%%%%%%%%%%%%%%%%%
% \section{Implementation}
%\iffalse
%<*package>
%\fi
%
% This section describes the definitions file |childdoc.def|.

% The definitions cannot be loaded using |\usepackage| or |\RequirePackage|
% which has a mechanism to prevent loading a style file more than once.
% When loading the definitions by means of |\input|
% multiple instances have to be prevented manually:
%\iffalse
%This code needs to be before the `\ProvidesFile' directive
%which is defined at the beginning of this file.
%Therefore it is also placed there and commented out here.
%</package>
%<*discard>
%\fi
%    \begin{macrocode}
\ifdefined\childdocmain\endinput\fi
%    \end{macrocode}
%\iffalse
%</discard>
%<*package>
%\fi
%
% \macro{\ifchilddoc}
% \macro{\ifchilddocmanual}
% The conditional |\ifchilddoc| tells whether a
% child (true) or main (false) document is being compiled.
% The conditional |\ifchilddocmanual| tells whether
% the |\includeonly| mechanism is used (false) or
% the selection of child files must be performed manually (true).
% The definitions initialise to false:
%    \begin{macrocode}
\newif\ifchilddoc
\newif\ifchilddocmanual
%    \end{macrocode}

% \macro{\childdocname}
% \macro{\childdocjob}
% The macro |\childdocname| stores the name of the main document
% to be compiled. The macro |\childdocjob| stores the name of
% the document on which the \LaTeX{} compiler was originally invoked.
% The content of |\jobname| cannot be compared
% to filenames specified in the source due to different catcodes.
% The following code rescans |\jobname|, stores the result
% in |\childdocname| and saves a copy in |\childdocjob|:
%    \begin{macrocode}
\edef\childdocname{\scantokens\expandafter{\jobname\noexpand}}
\let\childdocjob\childdocname
%    \end{macrocode}

% \macro{\childdocdisable}
% The macro |\childdocdisable| prevents the main file
% from being processed more than once.
% At this stage, the main document command |\childdocmain|
% is assumed to be called once again where it should do nothing.
% Any subsequent call to it should prevent
% a secondary processing of the main document
% It overwrites the forwarding commands
% |\childdocof| and |\childdocforward|
% with empty macros to prevent further inclusions of the main document:
%    \begin{macrocode}
\newcommand{\childdocdisable}
{
  \renewcommand{\childdocmain}[1]{\renewcommand{\childdocmain}[1]{\endinput}}
  \renewcommand{\childdocof}[1]{}
  \renewcommand{\childdocby}[2][]{}
  \renewcommand{\childdocforward}[2][]{}
  \renewcommand{\childdocdisable}{}
}
%    \end{macrocode}

% \macro{\childdocmain}
% The macro |\childdocmain| is to be called at the top of the main file
% with nothing or the main filename (without extension) as argument.
% First, it breaks loops.
% If the argument is not empty and does not match |\childdocname|
% (which is set by the first inclusion of |childdoc.def|),
% |\ifchilddoc| is set to true, |\includeonly| is applied to the child file
% and |\jobname| is set to the main file
% (for proper handling of |.aux| files):
%    \begin{macrocode}
\newcommand{\childdocmain}[1]
{
  \childdocdisable\childdocmain{}
  \if?#1?\else
    \begingroup
      \def\childdoctmp{#1}
      \ifx\childdoctmp\childdocname
        \def\childdoctmp{}
      \else
        \def\childdoctmp
        {
          \childdoctrue
          \includeonly{\childdocname}
          \def\childdocjob{#1}
          \def\jobname{#1}
        }
      \fi
      \expandafter
    \endgroup
    \childdoctmp
  \fi
}
%    \end{macrocode}

% \macro{\childdocof}
% The command |\childdocof| redirects
% compilation to the main file |#1|.
%    \begin{macrocode}
\newcommand{\childdocof}[1]
{
  \childdocdisable
  \childdoctrue
  \includeonly{\childdocname}
  \def\jobname{#1}
  \def\childdocjob{#1}
  \input{#1}
}
%    \end{macrocode}

% \macro{\childdocby}
% The command |\childdocby| ....
%    \begin{macrocode}
\newcommand{\childdocby}[2][]
{
  \childdocdisable
  \childdoctrue
  \childdocmanualtrue
  \if?#1?\else
    \def\jobname{#2}
  \fi
  \def\childdocjob{#2}
  \input{#2}
  \endinput
}
%    \end{macrocode}

% \macro{\childdocforward}
% The command |\childdocforward| redirects
% compilation to the main file or
% (if the optional argument is given) a child file.
% Parameters are set as if the main file
% or a child file starting with |\childdocof| was compiled.
% Then compilation is handed over to the main file:
%    \begin{macrocode}
\newcommand{\childdocforward}[2][]
{
  \begingroup
    \if?#1?
      \def\childdoctmp
      {
        \def\childdocname{#2}
        \def\childdocjob{#2}
        \def\jobname{#2}
        \input{#2}
        \endinput
      }
    \else
      \def\childdoctmp
      {
        \childdocdisable
        \def\childdocname{#2}
        \childdoctrue
        \includeonly{#2}
        \def\childdocjob{#1}
        \def\jobname{#1}
        \input{#1}
        \endinput
      }
    \fi
    \expandafter
  \endgroup
  \childdoctmp
}
%    \end{macrocode}

% \macro{\childdocforwardprefix}
% The command |\childdocforwardprefix| redirects
% compilation to the main or a child file by means of a pattern.
% The prefix |#1| in the current filename is replaced by |#2|
% and the suffix of the current filename is kept
% (it is assumed that the filename does not contain the substring `|~~~|'
% which is used as a delimiter).
% Compilation is handed over to the new file by |\childdocforward|:
%    \begin{macrocode}
\newcommand{\childdocforwardprefix}[3][]
{
  \begingroup
    \def\childdocextract #2##1~~~{\def\childdoctmp{\childdocforward[#1]{#3##1}}}
    \expandafter\childdocextract\childdocname~~~
    \expandafter
  \endgroup
  \childdoctmp
}
%    \end{macrocode}

% \macro{\childdoc}
% The deprecated macro |\childdoc| is a legacy version of |\childdocmain|:
%    \begin{macrocode}
\newcommand{\childdoc}{\childdocmain}
%    \end{macrocode}

% \macro{\childdocredirect}
% The deprecated macro |\childdocredirect| is a legacy version
% of |\childdocforward| and |\childdocforwardprefix|:
%    \begin{macrocode}
\newcommand{\childdocredirect}[2][]
{
  \begingroup
    \if?#1?
      \def\childdoctmp{\childdocforward{#2}}
    \else
      \def\childdoctmp{\childdocforwardprefix{#1}{#2}}
    \fi
    \expandafter
  \endgroup
  \childdoctmp
}
%    \end{macrocode}

%\iffalse
%</package>
%\fi
%
\endinput

\childdocby{cdocsamp}
%    \end{macrocode}

%\iffalse
%</samplepart3|samplepart4>
%\fi
%
%\iffalse
%<*samplepart3>
%\fi
% Some text for part 3:
%    \begin{macrocode}
some text in part three
%    \end{macrocode}

%\iffalse
%</samplepart3>
%\fi
% Some text for part 4:
%\iffalse
%<*samplepart4>
%\fi
%    \begin{macrocode}
more text in part four
%    \end{macrocode}

%\iffalse
%</samplepart4>
%\fi
%
% %%%%%%%%%%%%%%%%%%%%%%%%%%%%%%%%%%%%%%
% \paragraph{Forwarding for a Complete Draft.}
%
% The following forwarding file |cdocsdrf.tex|
% compiles the main document in draft mode:
%\iffalse
%<*sampledraft>
%\fi
%    \begin{macrocode}
\def\version{draft}
% \iffalse
%
% childdoc.dtx Copyright (C) 2017-2018 Niklas Beisert
%
% This work may be distributed and/or modified under the
% conditions of the LaTeX Project Public License, either version 1.3
% of this license or (at your option) any later version.
% The latest version of this license is in
%   http://www.latex-project.org/lppl.txt
% and version 1.3 or later is part of all distributions of LaTeX
% version 2005/12/01 or later.
%
% This work has the LPPL maintenance status `maintained'.
%
% The Current Maintainer of this work is Niklas Beisert.
%
% This work consists of the files childdoc.dtx and childdoc.ins
% and the derived files childdoc.def and cdocsamp.tex with
% cdocsch1.tex, cdocsch2.tex, cdocsdrf.tex, cdocsfn1.tex, cdocsfn2.tex.
%
%<package>\ifdefined\childdocmain\endinput\fi
%<package>\ProvidesFile{childdoc.def}[2018/12/30 v2.0 child document driver]
%<samplemain>\ProvidesFile{cdocsamp.tex}[2018/12/30 v2.0 sample for childdoc]
%<*driver>
%\ProvidesFile{childdoc.drv}[2018/12/30 v2.0 childdoc reference manual file]
\PassOptionsToClass{10pt,a4paper}{article}
\documentclass{ltxdoc}

\usepackage[margin=35mm]{geometry}
\usepackage{hyperref}
\usepackage{hyperxmp}
\usepackage[usenames]{color}

\hypersetup{colorlinks=true}
\hypersetup{pdfstartview=FitH}
\hypersetup{pdfpagemode=UseNone}
\hypersetup{pdfsource={}}
\hypersetup{pdflang={en-UK}}
\hypersetup{pdfcopyright={Copyright 2017-2018 Niklas Beisert.
  This work may be distributed and/or modified under the
  conditions of the LaTeX Project Public License, either version 1.3
  of this license or (at your option) any later version.}}
\hypersetup{pdflicenseurl={http://www.latex-project.org/lppl.txt}}
\hypersetup{pdfcontactaddress={ETH Zurich, ITP, HIT K,
  Wolfgang-Pauli-Strasse 27}}
\hypersetup{pdfcontactpostcode={8093}}
\hypersetup{pdfcontactcity={Zurich}}
\hypersetup{pdfcontactcountry={Switzerland}}
\hypersetup{pdfcontactemail={nbeisert@itp.phys.ethz.ch}}
\hypersetup{pdfcontacturl={http://people.phys.ethz.ch/\xmptilde nbeisert/}}

\newcommand{\secref}[1]{\hyperref[#1]{section \ref*{#1}}}

\parskip1ex
\parindent0pt
\let\olditemize\itemize
\def\itemize{\olditemize\parskip0pt}

\begin{document}

\title{The \textsf{childdoc} Package}
\hypersetup{pdftitle={The childdoc Package}}
\author{Niklas Beisert\\[2ex]
  Institut f\"ur Theoretische Physik\\
  Eidgen\"ossische Technische Hochschule Z\"urich\\
  Wolfgang-Pauli-Strasse 27, 8093 Z\"urich, Switzerland\\[1ex]
  \href{mailto:nbeisert@itp.phys.ethz.ch}
  {\texttt{nbeisert@itp.phys.ethz.ch}}}
\hypersetup{pdfauthor={Niklas Beisert}}
\hypersetup{pdfsubject={Manual for the LaTeX2e Package childdoc}}
\date{30 December 2018, \textsf{v2.0}}
\maketitle

\begin{abstract}\noindent
\textsf{childdoc} is a \LaTeXe{} package
that enables the direct compilation
of document sections included by |\include|
to individual files.
\end{abstract}

\begingroup
\parskip0ex
\tableofcontents
\endgroup

%%%%%%%%%%%%%%%%%%%%%%%%%%%%%%%%%%%%%%%%%%%%%%%%%%%%%%%%%%%%%%%%%%%%%%%%%%%%%%%%
%%%%%%%%%%%%%%%%%%%%%%%%%%%%%%%%%%%%%%%%%%%%%%%%%%%%%%%%%%%%%%%%%%%%%%%%%%%%%%%%
\section{Introduction}

\LaTeX{} provides a mechanism to structure a large document (such as a book)
into a main file and several child files (containing the chapters)
using the |\include| command.
This mechanism is beneficial for documents
which span hundreds of pages in order to
make the source file(s) more manageable.
Moreover, compilation can be restricted to
selected child files by means of the |\includeonly| command.
The latter feature can be used to reduce the compilation time while editing
(this was significantly more useful in the earlier days of \LaTeX{})
or to generate a smaller document which is easier to navigate.
Another application of |\includeonly| is to generate
documents consisting of selected parts of the complete document.

However, there are a few drawbacks of the plain |\include| mechanism:
\begin{itemize}
\item
The child files cannot be compiled on their own,
they can only be compiled via the main file.
A naive editing environment
(such as a text editor with an option
to have the current file processed by \LaTeX)
may require one to switch to the main file before compiling;
attempting to compile the child file produces errors.
\item
The main file must be modified (each time)
to adjust the |\includeonly| command
to the present needs. This easily leaves the main file in a messy state.
\item
The generated document will always carry the filename
of the main document. This is inconvenient if
several child files are to be compiled and
to be kept for distribution.
\end{itemize}

The present package provides a simple interface
to make child files individually compilable by \LaTeX{}.
Compiling a child file then has the same effect as compiling
the main file with an |\includeonly| command
to select the appropriate child.
Moreover the generated document will carry the name of the child
rather than the main file.
This resolves all three above issues.

This feature is meant to make the editing of books,
thesis documents and lecture notes somewhat more convenient.
However, the package can also be used efficiently for
composing a series of documents (such as exercise sheets)
which are typically distributed individually.
It then assists the author in generating the individual documents
(potentially in different versions)
as well as a document containing the collected series.
Another application is in developing style files
or other kinds of included material
where compilation of the style file could redirect
to a sample or test file.

%%%%%%%%%%%%%%%%%%%%%%%%%%%%%%%%%%%%%%%%%%%%%%%%%%%%%%%%%%%%%%%%%%%%%%%%%%%%%%%%
%%%%%%%%%%%%%%%%%%%%%%%%%%%%%%%%%%%%%%%%%%%%%%%%%%%%%%%%%%%%%%%%%%%%%%%%%%%%%%%%
\section{Usage}

First of all, the package \textsf{childdoc} is \emph{not} a standard
\LaTeXe{} |.sty| style file! Therefore it needs to be invoked in
a non-standard way.

%%%%%%%%%%%%%%%%%%%%%%%%%%%%%%%%%%%%%%%%%%%%%%%%%%%%%%%%%%%%%%%%%%%%%%%%%%%%%%%%
\subsection{Included Files}
\label{sec:include}

%%%%%%%%%%%%%%%%%%%%%%%%%%%%%%%%%%%%%%%%
\DescribeMacro{\childdocmain}
To use the package, add the commands
\begin{center}
\begin{tabular}{l}
|\input{childdoc.def}|\\
|\childdocmain{}|\\
\end{tabular}
\end{center}
at the very top of the main \LaTeX{} file,
in particular \emph{before} the |\documentclass| statement!
The argument of |\childdocmain| should be left empty
(but it must be present).

%%%%%%%%%%%%%%%%%%%%%%%%%%%%%%%%%%%%%%%%
\DescribeMacro{\childdocof}
Furthermore, add the commands
\begin{center}
\begin{tabular}{l}
|\input{childdoc.def}|\\
|\childdocof{|\textit{main}|}|\\
\end{tabular}
\end{center}
at the top of every child file \textit{child}
which is included by |\include{|\textit{child}|}|
from within the main file
(or at least for those files to be compiled individually).
The argument \textit{main} must be the filename of the main file.

There are a couple of
considerations in setting up the main and child documents:

%%%%%%%%%%%%%%%%%%%%%%%%%%%%%%%%%%%%%%%%
\paragraph{Restrictions.}

Please note the following restrictions:
\begin{itemize}
\item
|\childdocmain| must be called with one argument \textit{main}
to ensure compatibility with earlier version of the package.
It must either be empty (|\childdocmain{}|)
or precisely match the filename of the main file in which it is specified.
See \secref{sec:detection} for further information.
\item
The filename \textit{main} must be specified without the |.tex| extension.
\item
The filename \textit{main} is case sensitive
(even in case-insensitive file systems)
due to internal string comparison.
\item
The argument \textit{main} should be fully expanded, it cannot be a macro.
\item
Subdirectories and special characters should be avoided in filenames.
\item
The command |\childdocmain{|\textit{main}|}| must be followed by a whitespace.
It should not be followed immediately by another command
or by a comment mark `|%|'.
This is because the \TeX{} parser reads the token immediately following
the argument of |\childdocmain| and puts it
at the beginning of every child section;
however, a white\-space is ignored.
\end{itemize}

%%%%%%%%%%%%%%%%%%%%%%%%%%%%%%%%%%%%%%%%
\paragraph{Content of Main File.}

It is advisable to place all content in the child files included by |\include|.
Any output contained in the main file will appear in all child documents
unless suppressed manually;
it cannot be suppressed automatically by the |\includeonly| directive
and thus should normally be avoided.
A method to include some content in the main file
by means of conditional processing is described in \secref{sec:conditional}.

%%%%%%%%%%%%%%%%%%%%%%%%%%%%%%%%%%%%%%%%
\paragraph{Page Numbering.}

When only a part of the document is compiled,
the appropriate numbering of pages
(as well as other status parameters)
is determined from the |.aux| files.
The latter contain information from previous passes.
However this information needs to propagate through
all intermediate child documents.
Therefore the page numbering in child documents may well
be inconsistent until the complete document is compiled at least once.

A useful (if unconventional) way to always ensure a consistent
page numbering is to restart the numbering in each child document
and denote the pages by `\textit{child}|.|\textit{page}'
where \textit{child} represents the chapter/section number of the child file.
This can be achieved by the command
|\numberwithin{page}{|\textit{child}|}|
of the \textsf{amsmath} package
where \textit{child} can be |chapter| or |section|
depending on the chosen structuring.
Alternatively, one can modify the macro |\thepage| appropriately
and reset the counter |page| at the start of each child file.

%%%%%%%%%%%%%%%%%%%%%%%%%%%%%%%%%%%%%%%%%%%%%%%%%%%%%%%%%%%%%%%%%%%%%%%%%%%%%%%%
\subsection{Conditional Processing}
\label{sec:conditional}

The package provides a mechanism to compile different versions
of a document. To customise the versions further some conditional processing
can come in handy to distinguish which version is being compiled.
The package provides two macros to describe the compilation context:

%%%%%%%%%%%%%%%%%%%%%%%%%%%%%%%%%%%%%%%%
\DescribeMacro{\ifchilddoc}
The conditional |\ifchilddoc| distinguishes between the compilation of
child documents and the main document:
%
\begin{center}
|\ifchilddoc |\textit{child-code}| |[|\||else |\textit{main-code}]| \||fi|
\end{center}

%%%%%%%%%%%%%%%%%%%%%%%%%%%%%%%%%%%%%%%%
\DescribeMacro{\childdocname}
\DescribeMacro{\childdocjob}
The macro |\childdocname| contains the filename (without extension)
of the main or child file being processed.
Note that |\childdocjob| will always contain the name of the main file.

%%%%%%%%%%%%%%%%%%%%%%%%%%%%%%%%%%%%%%%%
\paragraph{Title Page.}

Conditional processing can be used to include a title or banner page
in the main document when proper precautions are taken.
Importantly, the code in the main file should ensure that the page counter
(as well as other status parameters which are stored in the |.aux| files)
takes the same value after the conditional processing.
Otherwise the page numbers may take divergent values
depending on which part is compiled.

For example, a title page could be declared by:
%
\begin{center}
\begin{tabular}{l}
|\ifchilddoc\||else|\\
|\addtocounter{page}{-1}|\\
\textit{code for title page}\\
|\newpage|\\
|\||fi|
\end{tabular}
\end{center}
%
A banner page for the child documents can be generated by:
%
\begin{center}
\begin{tabular}{l}
|\ifchilddoc|\\
|\addtocounter{page}{-1}|\\
\textit{code for banner page}\\
|\newpage|\\
|\||fi|
\end{tabular}
\end{center}
%
Here one could write a message such as:
\begin{center}
|This is the part \childdocname{} of \childdocjob{}.|
\end{center}

%%%%%%%%%%%%%%%%%%%%%%%%%%%%%%%%%%%%%%%%%%%%%%%%%%%%%%%%%%%%%%%%%%%%%%%%%%%%%%%%
\subsection{Flags}
\label{sec:flags}

The package makes it easy to generate different versions
of the main or child documents.
To this end compilation flags can be defined
and assigned different default values.
They will be particularly useful in conjunction
with the forwarding mechanism described in \secref{sec:forward}.

For example, it may be useful to have a flag |\version|
which can be set to |draft| or |final|.
The document source will contain some conditional code
depending on the value of |\version|.
Suppose further, the flag should default to |final| for the main file
and to |draft| for child files
which is a natural assignment for editing the document.
This is achieved by placing the following code
in the preamble of the main document
(below the |\childdocmain| directive):
%
\begin{center}
\begin{tabular}{l}
|\ifchilddoc|\\
|\providecommand{\version}{draft}|\\
|\||else|\\
|\providecommand{\version}{final}|\\
|\||fi|
\end{tabular}
\end{center}
%
The definition by |\providecommand| makes sure
that previous definitions are not overwritten.
Further statements |\providecommand{\version}{...}|
can thus be added before the above code to override it.

For the main file, one might add a line
(between |\childdocmain| and the above block)
%
\begin{center}
|%\ifchilddoc\||else\providecommand{\version}{draft}\||fi|
\end{center}
%
which can be uncommented to produce a draft version.
Likewise one can add a line to the very top of a child file
(above the |\childdocof{|\textit{main}|}| directive)
%
\begin{center}
|%\providecommand{\version}{final}|
\end{center}
%
which can be uncommented to produce the final version of this child document.

%%%%%%%%%%%%%%%%%%%%%%%%%%%%%%%%%%%%%%%%%%%%%%%%%%%%%%%%%%%%%%%%%%%%%%%%%%%%%%%%
\subsection{Forwarding}
\label{sec:forward}

Different versions of the main or child documents
using compilation flags as described in \secref{sec:flags}
can be (permanently) stored in different files
for convenient compilation, viewing and distribution.
To this end, the package defines a command
to pass on compilation to a different file:

%%%%%%%%%%%%%%%%%%%%%%%%%%%%%%%%%%%%%%%%
\DescribeMacro{\childdocforward}
The command |\childdocforward| redirects processing to
another source file:
%
\begin{center}
\begin{tabular}{l}
|\input{childdoc.def}|\\
|\childdocforward[|\textit{main}|]{|\textit{dest}|}|\\
\end{tabular}
\end{center}
%
The argument \textit{dest} is the destination file
(without extension).
It should be the main file or one of the child files.
Note that further \textsf{childdoc} directives
such as |\childdocof| and |\childdocforward|
in the indicated file will be processed in this form.
The optional argument \textit{main}
passes on directly to the main file \textit{main}
while pretending to compile the child \textit{dest}.
This form behaves as if \textit{dest}
issues |\childdocof{|\textit{main}|}| right away,
and no further \textsf{childdoc} directives will be processed.

%%%%%%%%%%%%%%%%%%%%%%%%%%%%%%%%%%%%%%%%
\DescribeMacro{\...prefix}
In the alternative form |\childdocforwardprefix|,
%
\begin{center}
\begin{tabular}{l}
|\input{childdoc.def}|\\
|\childdocforwardprefix[|\textit{main}|]{|\textit{prefix}|}{|\textit{dest}|}|
\end{tabular}
\end{center}
%
the destination file is determined by a pattern
depending on the current file:
To make this work, the current file must be called
`{\textit{prefix}\hspace{0.2em}\textit{suffix}}'
with \textit{prefix} matching precisely the argument.
Processing is then passed on to the file
`{\textit{dest}\hspace{0.2em}\textit{suffix}}'.
Surely, the same effect is achieved by
directly specifying the
argument `{\textit{dest}\hspace{0.2em}\textit{suffix}}'
in the first form.
However, that requires to set up a different file
for each child. With the alternative form of the command
all these files can have exactly the same content
which simplifies setting them up and maintaining them.

For example, the following file |draft.tex|
with a compilation flag |\version| as described in \secref{sec:flags}
compiles the main document as a draft:
%
\begin{center}
\begin{tabular}{l}
|\def\version{draft}|\\
|\input{childdoc.def}|\\
|\childdocforward{|\textit{main}|}|
\end{tabular}
\end{center}
%
Likewise, the following files |final|\textit{nn}|.tex|
compile the final version of the child document
|child|\textit{nn}|.tex|:
%
\begin{center}
\begin{tabular}{l}
|\def\version{final}|\\
|\input{childdoc.def}|\\
|\childdocforwardprefix{final}{child}|
\end{tabular}
\end{center}
%

Note that when several versions of a main file and/or of each child file
are to be generated, it may be convenient to set up a |Makefile| or
shell script to automatise the process.

%%%%%%%%%%%%%%%%%%%%%%%%%%%%%%%%%%%%%%%%%%%%%%%%%%%%%%%%%%%%%%%%%%%%%%%%%%%%%%%%
\subsection{Command Line Processing}
\label{sec:commandline}

The effect of redirection files can also be achieved by invoking
the \LaTeX{} compiler with a more elaborate command line.
Most conveniently this should be done as part
of a shell script or a |Makefile|.

When using \textsf{childdoc} in the main file, the following
command lines effectively perform a redirection
(note that depending on the shell being used,
backslashes may have to be doubled: `|\|' $\to$ `|\\|'):
%
\begin{center}
|... -jobname "|\textit{target}|" |\\|"|[\textit{flags}]%
|\input{childdoc.def}\childdocforward[|\textit{main}|]{|\textit{dest}|}"|
\end{center}
%
Here \textit{target} is the name of the output file,
\textit{main} is the name of the main file
and \textit{dest} is the name of the main or child file to be processed
(all filenames without extensions).
The optional argument \textit{main} can be omitted
if \textit{main} matches \textit{dest}.
Optionally, compilation \textit{flags} can be defined via |\def| commands.
This command line makes the \TeX{} engine believe
it is compiling the file \textit{target}
whose content is specified as the latter parameter.
The provided code then forwards the processing to
\textit{main} or \textit{dest} as described in \secref{sec:forward}.

%%%%%%%%%%%%%%%%%%%%%%%%%%%%%%%%%%%%%%%%%%%%%%%%%%%%%%%%%%%%%%%%%%%%%%%%%%%%%%%%
\subsection{Include by Input}
\label{sec:input}

Including child documents by |\include| has some restrictions by design.
Most notably, the content of a child document always occupies
its own set of pages; pages cannot be shared between child documents.
Usually, this behaviour makes perfect sense
because each child document contain an essential part of the document.
However, in some situations it may be desirable to compose
a document from a collection of parts
without having mandatory page breaks between then.
For this case, the package
provides a mechanism to include parts
by |\input| which can also be processed individually.
However, by construction this mechanism
requires manual handling of the content to be output.

%%%%%%%%%%%%%%%%%%%%%%%%%%%%%%%%%%%%%%%%
\DescribeMacro{\ifchilddocmanual}
The main file should be prepared as usual, see \secref{sec:include}.
However, the document body must make a distinction
between processing of an individual part and of the main document, e.g.:
%
\begin{center}
\begin{tabular}{l}
|\ifchilddocmanual|\\
|\input{\childdocname}|\\
|\||else|\\
\textit{document body with }|\input{|\textit{part}|}|\\
|\||fi|
\end{tabular}
\end{center}
%
The conditional |\ifchilddocmanual| is true whenever
a part to be included by |\input| is being compiled,
and the name of the part is stored in |\childdocname|.

%%%%%%%%%%%%%%%%%%%%%%%%%%%%%%%%%%%%%%%%
\DescribeMacro{\childdocby}
Each part to be included by |\input| should start with:
%
\begin{center}
\begin{tabular}{l}
|\input{childdoc.def}|\\
|\childdocby{|\textit{main}|}|\\
\end{tabular}
\end{center}
%
The directive |\childdocby| is similar to |\childdocof|
described in \secref{sec:include},
but the subsequent selection of content must be done manually.
To that end, both |\ifchilddoc| and |\ifchilddocmanual|
will be true upon processing of a part,
and the name of the part is stored in |\childdocname|.
Note that |\jobname| will be set to the filename of the current part
so that each part receives an individual |.aux| file
that does not interfere with the |.aux| file(s) of the main document.
This behaviour can be altered by the alternative form
|\childdocby[*]{|\textit{main}|}| (with a non-empty optional argument)
which uses the |.aux| file of the main document
by setting |\jobname| to \textit{main}.

%%%%%%%%%%%%%%%%%%%%%%%%%%%%%%%%%%%%%%%%%%%%%%%%%%%%%%%%%%%%%%%%%%%%%%%%%%%%%%%%
\subsection{Driver Development}
\label{sec:driver}

The \textsf{childdoc} mechanism can also be use for the development
of definition files such as \LaTeX{} styles or classes.
This case differs from the above setup with multiple parts
included by |\include| in that no |\includeonly| should be invoked.
This can be achieved by starting the include file
(before |\ProvidesPackage|) with:
%
\begin{center}
\begin{tabular}{l}
|\input{childdoc.def}|\\
|\childdocforward{|\textit{main}|}|\\
\end{tabular}
\end{center}
%
or alternatively with:
%
\begin{center}
\begin{tabular}{l}
|\input{childdoc.def}|\\
|\childdocby{|\textit{main}|}|\\
\end{tabular}
\end{center}
%
Both forms have slightly different effects as described above.
The main file is prepared as usual, see \secref{sec:include}.

%%%%%%%%%%%%%%%%%%%%%%%%%%%%%%%%%%%%%%%%%%%%%%%%%%%%%%%%%%%%%%%%%%%%%%%%%%%%%%%%
\subsection{Legacy Detection}
\label{sec:detection}

The directive |\childdocmain| in the main file can detect
whether the complete document or merely a child is to be compiled
even without using the directive |\childdocof|.
This method is deprecated because it is less robust
and there is no compelling reason to use it;
it is merely provided for backward compatibility
and it may be removed in future versions.

If the detection mechanism is to be used,
it is mandatory to correctly specify
the filename of the main file as the argument of |\childdocmain|:
%
\begin{center}
\begin{tabular}{l}
|\input{childdoc.def}|\\
|\childdocmain{|\textit{main}|}|\\
\end{tabular}
\end{center}
%
If |\jobname| does not match the argument \textit{main} of |\childdocmain|,
it is assumed that |\jobname| points to the child file to be compiled.
When using |\childdocmain| with the main file specified as argument,
it suffices to start a child file
with just |\input{|\textit{main}|}|
without loading of the package and using |\childdocof|.
If instead all processing is done
with the appropriate \textsf{childdoc} directives,
the argument of \textit{main} of |\childdocmain| can be empty.

An alternative version of the command line processing described
in \secref{sec:commandline} using the detection mechanism reads:
%
\begin{center}
|... -jobname "|\textit{target}|" "|[\textit{flags}]%
[|\def\jobname{|\textit{dest}|}|]|\input{|\textit{main}|}"|
\end{center}

%%%%%%%%%%%%%%%%%%%%%%%%%%%%%%%%%%%%%%%%%%%%%%%%%%%%%%%%%%%%%%%%%%%%%%%%%%%%%%%%
\subsection{Manual Code}
\label{sec:manual}

In case one cannot be certain whether the definitions file |childdoc.def|
is installed on the target \TeX{} distribution
and one prefers not to ship it,
it is conceivable to paste a few relevant commands into the sources.

To that end, drop all statements |\input{childdoc.def}|
and perform the replacements as outlined below.
Instead of |\childdocmain{|\textit{main}|}| add the following code
to the top of the main file:
%
\begin{center}
\begin{tabular}{l}
|\||ifdefined\childdocname\endinput\||fi\newif\ifchilddoc|\\
|\edef\childdocname{\scantokens\expandafter{\jobname\noexpand}}|\\
|\def\childdocmain{|\textit{main}|}\||ifx\childdocmain\childdocname\||else|\\
|\childdoctrue\includeonly{\childdocname}\let\jobname\childdocmain\||fi|\\
\end{tabular}
\end{center}
%
Instead of |\childdocof{|\textit{main}|}| just include the main file
at the top of each child file:
%
\begin{center}
|\input{|\textit{main}|}|
\end{center}
%
A simple redirection |\childdocforward{|\textit{dest}|}| is achieved by:
%
\begin{center}
|\def\jobname{|\textit{dest}|}\input{\jobname}|
\end{center}
%
The redirection with prefix
|\childdocforwardprefix[|\textit{prefix}|]{|\textit{dest}|}|
is accomplished by:
%
\begin{center}
\begin{tabular}{l}
|{\edef\jobname{\scantokens\expandafter{\jobname\noexpand}}|\\
|\def\redirectjob |\textit{prefix}|#1~~~{\gdef\jobname{|\textit{dest}|#1}}|\\
|\expandafter\redirectjob\jobname~~~}\input{\jobname}|
\end{tabular}
\end{center}

In an alternative approach,
child documents can be compiled by a specific command line
without additional code or specific definitions:
%
\begin{center}
|... -jobname "|\textit{target}|" "|[\textit{flags}]%
|\includeonly{|\textit{dest}|}\input{|\textit{main}|}"|
\end{center}
%

%%%%%%%%%%%%%%%%%%%%%%%%%%%%%%%%%%%%%%%%%%%%%%%%%%%%%%%%%%%%%%%%%%%%%%%%%%%%%%%%
%%%%%%%%%%%%%%%%%%%%%%%%%%%%%%%%%%%%%%%%%%%%%%%%%%%%%%%%%%%%%%%%%%%%%%%%%%%%%%%%
\section{Information}

%%%%%%%%%%%%%%%%%%%%%%%%%%%%%%%%%%%%%%%%%%%%%%%%%%%%%%%%%%%%%%%%%%%%%%%%%%%%%%%%
\subsection{Copyright}

Copyright \copyright{} 2017--2018 Niklas Beisert

This work may be distributed and/or modified under the
conditions of the \LaTeX{} Project Public License, either version 1.3
of this license or (at your option) any later version.
The latest version of this license is in
  \url{http://www.latex-project.org/lppl.txt}
and version 1.3 or later is part of all distributions of \LaTeX{}
version 2005/12/01 or later.

This work has the LPPL maintenance status `maintained'.

The Current Maintainer of this work is Niklas Beisert.

This work consists of the files |README.txt|, |childdoc.ins| and |childdoc.dtx|
as well as the derived files |childdoc.def|, |cdocsamp.tex|
with |cdocsch1.tex|, |cdocsch2.tex|, |cdocspt3.tex|, |cdocspt4.tex|,
|cdocsdrf.tex|, |cdocsfn1.tex|, |cdocsfn2.tex|
as well as |childdoc.pdf|.

%%%%%%%%%%%%%%%%%%%%%%%%%%%%%%%%%%%%%%%%%%%%%%%%%%%%%%%%%%%%%%%%%%%%%%%%%%%%%%%%
\subsection{Files and Installation}

The package consists of the files:
%
\begin{center}
\begin{tabular}{ll}
    |README.txt|   & readme file \\
    |childdoc.ins| & installation file \\
    |childdoc.dtx| & source file \\
    |childdoc.def| & definition file \\
    |cdocsamp.tex| & sample main file \\
    |cdocsch1.tex| & sample include file \\
    |cdocsch2.tex| & sample include file \\
    |cdocspt3.tex| & sample part file \\
    |cdocspt4.tex| & sample part file \\
    |cdocsdrf.tex| & sample redirection file \\
    |cdocsfn1.tex| & sample redirection file \\
    |cdocsfn2.tex| & sample redirection file \\
    |childdoc.pdf| & manual
\end{tabular}
\end{center}
%
The distribution consists of the files
|README.txt|, |childdoc.ins| and |childdoc.dtx|.
%
\begin{itemize}
\item
Run (pdf)\LaTeX{} on |childdoc.dtx|
to compile the manual |childdoc.pdf| (this file).
\item
Run \LaTeX{} on |childdoc.ins| to create the definitions file |childdoc.def|
and the sample |cdocsamp.tex| with include files
|cdocsch1.tex|, |cdocsch2.tex|, |cdocspt3.tex|, |cdocspt4.tex|,
|cdocsdrf.tex|, |cdocsfn1.tex|, |cdocsfn2.tex|.
Then copy the file |childdoc.def| to an appropriate directory of your \LaTeX{}
distribution, e.g.\ \textit{texmf-root}|/tex/latex/childdoc|.
\end{itemize}

%%%%%%%%%%%%%%%%%%%%%%%%%%%%%%%%%%%%%%%%%%%%%%%%%%%%%%%%%%%%%%%%%%%%%%%%%%%%%%%%
\subsection{Related CTAN Packages}

There are several other packages which offer a similar functionality:
%
\begin{itemize}
\item
The packages
\href{http://ctan.org/pkg/docmute}{\textsf{docmute}},
\href{http://ctan.org/pkg/includex}{\textsf{includex}} and
\href{http://ctan.org/pkg/standalone}{\textsf{standalone}}
provide commands to include only the document body of
a child file thus allowing both files to be compiled individually.
\item
The packages \href{http://ctan.org/pkg/subdocs}{\textsf{subdocs}}
and \href{http://ctan.org/pkg/subfiles}{\textsf{subfiles}}
provide structures in which the main and child documents can be
encapsulated and allowing them to be compiled individually.
The inclusion mechanism is different from the conventional |\include|.
\item
The package \href{http://ctan.org/pkg/combine}{\textsf{combine}}
is an elaborate solution to combine several documents into one.
\end{itemize}
%
See also the CTAN topic \href{http://ctan.org/topic/subdocs}{\textsf{subdocs}}
for further related packages.
The present package differs from the above solutions in that
a document structure constructed with the conventional |\include| mechanism
just needs two extra commands at the top of every file
such that all constituent files can be compiled individually.

%%%%%%%%%%%%%%%%%%%%%%%%%%%%%%%%%%%%%%%%%%%%%%%%%%%%%%%%%%%%%%%%%%%%%%%%%%%%%%%%
%\subsection{Feature Suggestions}
%
%The following is a list of features which may be useful for future
%versions of this package:
%%
%\begin{itemize}
%\item
%\ldots
%\end{itemize}

%%%%%%%%%%%%%%%%%%%%%%%%%%%%%%%%%%%%%%%%%%%%%%%%%%%%%%%%%%%%%%%%%%%%%%%%%%%%%%%%
\subsection{Revision History}

%%%%%%%%%%%%%%%%%%%%%%%%%%%%%%%%%%%%%%%%
\paragraph{v2.0:} 2018/12/30

\begin{itemize}
\item
immediate forward processing
\item
added |\childdocby| mechanism
\item
manual restructured
\end{itemize}

%%%%%%%%%%%%%%%%%%%%%%%%%%%%%%%%%%%%%%%%
\paragraph{v1.6:} 2018/01/17

\begin{itemize}
\item
application for development of include files
\item
corrections to manual
\end{itemize}

%%%%%%%%%%%%%%%%%%%%%%%%%%%%%%%%%%%%%%%%
\paragraph{v1.5:} 2017/05/21

\begin{itemize}
\item
more complete structuring introduced
\item
|\childdocof| introduced
\item
|\childdoc| renamed to |\childdocmain|
\item
|\childredirect| renamed to |\childdocforward| and |\childdocforwardprefix|
and functionality expanded
\end{itemize}

%%%%%%%%%%%%%%%%%%%%%%%%%%%%%%%%%%%%%%%%
\paragraph{v1.0:} 2017/04/27

\begin{itemize}
\item
manual and install package
\item
first version published on CTAN
\end{itemize}

%%%%%%%%%%%%%%%%%%%%%%%%%%%%%%%%%%%%%%%%
\paragraph{v0.6:} 2017/04/26

\begin{itemize}
\item
redirection mechanism added
\end{itemize}

%%%%%%%%%%%%%%%%%%%%%%%%%%%%%%%%%%%%%%%%
\paragraph{v0.5:} 2017/04/26

\begin{itemize}
\item
functionality in definition file
\end{itemize}


%%%%%%%%%%%%%%%%%%%%%%%%%%%%%%%%%%%%%%%%%%%%%%%%%%%%%%%%%%%%%%%%%%%%%%%%%%%%%%%%
%%%%%%%%%%%%%%%%%%%%%%%%%%%%%%%%%%%%%%%%%%%%%%%%%%%%%%%%%%%%%%%%%%%%%%%%%%%%%%%%
%%%%%%%%%%%%%%%%%%%%%%%%%%%%%%%%%%%%%%%%%%%%%%%%%%%%%%%%%%%%%%%%%%%%%%%%%%%%%%%%
\appendix

\settowidth\MacroIndent{\rmfamily\scriptsize 000\ }

 \DocInput{childdoc.dtx}

\end{document}
%</driver>
% \fi
%
% %%%%%%%%%%%%%%%%%%%%%%%%%%%%%%%%%%%%%%%%%%%%%%%%%%%%%%%%%%%%%%%%%%%%%%%%%%%%%%
% %%%%%%%%%%%%%%%%%%%%%%%%%%%%%%%%%%%%%%%%%%%%%%%%%%%%%%%%%%%%%%%%%%%%%%%%%%%%%%
% \section{Sample}
%\iffalse
%<*samplemain>
%\fi
%
% The following presents a sample document
% with two chapters, two parts, a title page,
% a compile flag as well as three forwarding files to set the flag.
% It consists of eight |.tex| files:
% \begin{center}
% \begin{tabular}{ll}
% |cdocsamp.tex|&main file\\
% |cdocsch1.tex|&include file for chapter 1\\
% |cdocsch2.tex|&include file for chapter 2\\
% |cdocspt3.tex|&include file for part 3\\
% |cdocspt4.tex|&include file for part 4\\
% |cdocsdrf.tex|&forwarding file for main file in draft mode\\
% |cdocsfi1.tex|&forwarding file for final version of chapter 1\\
% |cdocsfi2.tex|&forwarding file for final version of chapter 2\\
% \end{tabular}
% \end{center}
% Each of the eight files can be compiled directly by the \LaTeX{} compiler.
%
% %%%%%%%%%%%%%%%%%%%%%%%%%%%%%%%%%%%%%%
% \paragraph{Main File.}
%
% The main file is called |cdocsamp.tex|.
%
% Load the \textsf{childdoc} definitions and
% declare the filename for the main document:
%    \begin{macrocode}
\input{childdoc.def}
\childdocmain{}
%    \end{macrocode}

% Optional override for |\version| flag:
%    \begin{macrocode}
%%\ifchilddoc\else\providecommand{\version}{draft}\fi
%    \end{macrocode}

% Define the default values for the |\version| flag
% (|final| for the main file and |draft| for childs):
%    \begin{macrocode}
\ifchilddoc
\providecommand{\version}{draft}
\else
\providecommand{\version}{final}
\fi
%    \end{macrocode}

% Load the standard document class:
%    \begin{macrocode}
\documentclass[12pt]{article}
%    \end{macrocode}

% Start the document body:
%    \begin{macrocode}
\begin{document}
%    \end{macrocode}

% Declare a title page.
% Print title, part of document being processed and version flag:
%    \begin{macrocode}
\addtocounter{page}{-1}
\begin{center}
{\LARGE\bfseries{}childdoc example\par}
\vspace{1cm}
\ifchilddoc
\ifchilddocmanual part\else chapter\fi:
`\childdocname' of `\childdocjob'\par
\else
main document: `\childdocjob'\par
\fi
version: \version\par
\end{center}
\newpage
%    \end{macrocode}

% Manually include selected file,
% otherwise process as usual:
%    \begin{macrocode}
\ifchilddocmanual
\section*{part `\childdocname'}
\input{\childdocname}
\else
%    \end{macrocode}

% Include the two chapters:
%    \begin{macrocode}
\include{cdocsch1}
\include{cdocsch2}
%    \end{macrocode}

% Include the two parts unless only chapters should be displayed:
%    \begin{macrocode}
\ifchilddoc\else
\section{part three}
\input{cdocspt3}
\section{part four}
\input{cdocspt4}
\fi
%    \end{macrocode}

% Process as usual until here:
%    \begin{macrocode}
\fi
%    \end{macrocode}

% End of document body:
%    \begin{macrocode}
\end{document}
%    \end{macrocode}
%\iffalse
%</samplemain>
%\fi
%
% %%%%%%%%%%%%%%%%%%%%%%%%%%%%%%%%%%%%%%
% \paragraph{Chapter Include Files.}
%
% The include files are called |cdocsch1.tex| and |cdocsch2.tex|.
%
%\iffalse
%<*samplechap1|samplechap2>
%\fi

% Optional override for |\version| flag:
%    \begin{macrocode}
%%\providecommand{\version}{final}
%    \end{macrocode}

% Include the main document:
%    \begin{macrocode}
\input{childdoc.def}
\childdocof{cdocsamp}
%    \end{macrocode}

%\iffalse
%</samplechap1|samplechap2>
%\fi
%
%\iffalse
%<*samplechap1>
%\fi
% Some text for chapter 1:
%    \begin{macrocode}
\section{one}
some text in chapter one
%    \end{macrocode}

%\iffalse
%</samplechap1>
%\fi
% Some text for chapter 2:
%\iffalse
%<*samplechap2>
%\fi
%    \begin{macrocode}
\section{two}
more text in chapter two
%    \end{macrocode}

%\iffalse
%</samplechap2>
%\fi
%
% %%%%%%%%%%%%%%%%%%%%%%%%%%%%%%%%%%%%%%
% \paragraph{Part Include Files.}
%
% The include files are called |cdocspt3.tex| and |cdocspt4.tex|.
%
%\iffalse
%<*samplepart3|samplepart4>
%\fi

% Optional override for |\version| flag:
%    \begin{macrocode}
%%\providecommand{\version}{final}
%    \end{macrocode}

% Include the main document:
%    \begin{macrocode}
\input{childdoc.def}
\childdocby{cdocsamp}
%    \end{macrocode}

%\iffalse
%</samplepart3|samplepart4>
%\fi
%
%\iffalse
%<*samplepart3>
%\fi
% Some text for part 3:
%    \begin{macrocode}
some text in part three
%    \end{macrocode}

%\iffalse
%</samplepart3>
%\fi
% Some text for part 4:
%\iffalse
%<*samplepart4>
%\fi
%    \begin{macrocode}
more text in part four
%    \end{macrocode}

%\iffalse
%</samplepart4>
%\fi
%
% %%%%%%%%%%%%%%%%%%%%%%%%%%%%%%%%%%%%%%
% \paragraph{Forwarding for a Complete Draft.}
%
% The following forwarding file |cdocsdrf.tex|
% compiles the main document in draft mode:
%\iffalse
%<*sampledraft>
%\fi
%    \begin{macrocode}
\def\version{draft}
\input{childdoc.def}
\childdocforward{cdocsamp}
%    \end{macrocode}

%\iffalse
%</sampledraft>
%\fi
%
% %%%%%%%%%%%%%%%%%%%%%%%%%%%%%%%%%%%%%%
% \paragraph{Forwarding for Final Version of the Chapters.}
%
% The following forwarding files |cdocsfn1.tex| and |cdocsfn2.tex|
% (with identical content)
% compile the final versions of the child documents
% |cdocsch1.tex| and |cdocsch2.tex|, respectively:
%\iffalse
%<*samplefinal>
%\fi
%    \begin{macrocode}
\def\version{final}
\input{childdoc.def}
\childdocforwardprefix[cdocsamp]{cdocsfn}{cdocsch}
%    \end{macrocode}

%\iffalse
%</samplefinal>
%\fi
%
% %%%%%%%%%%%%%%%%%%%%%%%%%%%%%%%%%%%%%%
% \paragraph{Command Line Processing.}
%
% The following three command lines generate the output files
% |cdocscld|, |cdocscl1| and |cdocscl2|
% which should be identical to
% |cdocsdrf|, |cdocsch1| and |cdocsfn2|, respectively:
% \begin{center}
% \begin{tabular}{l}
% |latex -jobname cdocscld \|\\
% |  "\def\version{draft}\input{childdoc.def}\childdocforward{cdocsamp}"|\\
% |latex -jobname cdocscl1 \|\\
% |  "\input{childdoc.def}\childdocforward[cdocsamp]{cdocsch1}"|\\
% |latex -jobname cdocscl2 \|\\
% |  "\def\version{final}\input{childdoc.def}\childdocforward{cdocsch2}"|
% \end{tabular}
% \end{center}
% Note that the trailing backslash on each first line
% merely continues the input to the second line
% (for convenient cut ant paste).
% Furthermore, the command |latex| can be replaced by any
% of its alternative versions such as |pdflatex|.
%
% %%%%%%%%%%%%%%%%%%%%%%%%%%%%%%%%%%%%%%%%%%%%%%%%%%%%%%%%%%%%%%%%%%%%%%%%%%%%%%
% %%%%%%%%%%%%%%%%%%%%%%%%%%%%%%%%%%%%%%%%%%%%%%%%%%%%%%%%%%%%%%%%%%%%%%%%%%%%%%
% \section{Implementation}
%\iffalse
%<*package>
%\fi
%
% This section describes the definitions file |childdoc.def|.

% The definitions cannot be loaded using |\usepackage| or |\RequirePackage|
% which has a mechanism to prevent loading a style file more than once.
% When loading the definitions by means of |\input|
% multiple instances have to be prevented manually:
%\iffalse
%This code needs to be before the `\ProvidesFile' directive
%which is defined at the beginning of this file.
%Therefore it is also placed there and commented out here.
%</package>
%<*discard>
%\fi
%    \begin{macrocode}
\ifdefined\childdocmain\endinput\fi
%    \end{macrocode}
%\iffalse
%</discard>
%<*package>
%\fi
%
% \macro{\ifchilddoc}
% \macro{\ifchilddocmanual}
% The conditional |\ifchilddoc| tells whether a
% child (true) or main (false) document is being compiled.
% The conditional |\ifchilddocmanual| tells whether
% the |\includeonly| mechanism is used (false) or
% the selection of child files must be performed manually (true).
% The definitions initialise to false:
%    \begin{macrocode}
\newif\ifchilddoc
\newif\ifchilddocmanual
%    \end{macrocode}

% \macro{\childdocname}
% \macro{\childdocjob}
% The macro |\childdocname| stores the name of the main document
% to be compiled. The macro |\childdocjob| stores the name of
% the document on which the \LaTeX{} compiler was originally invoked.
% The content of |\jobname| cannot be compared
% to filenames specified in the source due to different catcodes.
% The following code rescans |\jobname|, stores the result
% in |\childdocname| and saves a copy in |\childdocjob|:
%    \begin{macrocode}
\edef\childdocname{\scantokens\expandafter{\jobname\noexpand}}
\let\childdocjob\childdocname
%    \end{macrocode}

% \macro{\childdocdisable}
% The macro |\childdocdisable| prevents the main file
% from being processed more than once.
% At this stage, the main document command |\childdocmain|
% is assumed to be called once again where it should do nothing.
% Any subsequent call to it should prevent
% a secondary processing of the main document
% It overwrites the forwarding commands
% |\childdocof| and |\childdocforward|
% with empty macros to prevent further inclusions of the main document:
%    \begin{macrocode}
\newcommand{\childdocdisable}
{
  \renewcommand{\childdocmain}[1]{\renewcommand{\childdocmain}[1]{\endinput}}
  \renewcommand{\childdocof}[1]{}
  \renewcommand{\childdocby}[2][]{}
  \renewcommand{\childdocforward}[2][]{}
  \renewcommand{\childdocdisable}{}
}
%    \end{macrocode}

% \macro{\childdocmain}
% The macro |\childdocmain| is to be called at the top of the main file
% with nothing or the main filename (without extension) as argument.
% First, it breaks loops.
% If the argument is not empty and does not match |\childdocname|
% (which is set by the first inclusion of |childdoc.def|),
% |\ifchilddoc| is set to true, |\includeonly| is applied to the child file
% and |\jobname| is set to the main file
% (for proper handling of |.aux| files):
%    \begin{macrocode}
\newcommand{\childdocmain}[1]
{
  \childdocdisable\childdocmain{}
  \if?#1?\else
    \begingroup
      \def\childdoctmp{#1}
      \ifx\childdoctmp\childdocname
        \def\childdoctmp{}
      \else
        \def\childdoctmp
        {
          \childdoctrue
          \includeonly{\childdocname}
          \def\childdocjob{#1}
          \def\jobname{#1}
        }
      \fi
      \expandafter
    \endgroup
    \childdoctmp
  \fi
}
%    \end{macrocode}

% \macro{\childdocof}
% The command |\childdocof| redirects
% compilation to the main file |#1|.
%    \begin{macrocode}
\newcommand{\childdocof}[1]
{
  \childdocdisable
  \childdoctrue
  \includeonly{\childdocname}
  \def\jobname{#1}
  \def\childdocjob{#1}
  \input{#1}
}
%    \end{macrocode}

% \macro{\childdocby}
% The command |\childdocby| ....
%    \begin{macrocode}
\newcommand{\childdocby}[2][]
{
  \childdocdisable
  \childdoctrue
  \childdocmanualtrue
  \if?#1?\else
    \def\jobname{#2}
  \fi
  \def\childdocjob{#2}
  \input{#2}
  \endinput
}
%    \end{macrocode}

% \macro{\childdocforward}
% The command |\childdocforward| redirects
% compilation to the main file or
% (if the optional argument is given) a child file.
% Parameters are set as if the main file
% or a child file starting with |\childdocof| was compiled.
% Then compilation is handed over to the main file:
%    \begin{macrocode}
\newcommand{\childdocforward}[2][]
{
  \begingroup
    \if?#1?
      \def\childdoctmp
      {
        \def\childdocname{#2}
        \def\childdocjob{#2}
        \def\jobname{#2}
        \input{#2}
        \endinput
      }
    \else
      \def\childdoctmp
      {
        \childdocdisable
        \def\childdocname{#2}
        \childdoctrue
        \includeonly{#2}
        \def\childdocjob{#1}
        \def\jobname{#1}
        \input{#1}
        \endinput
      }
    \fi
    \expandafter
  \endgroup
  \childdoctmp
}
%    \end{macrocode}

% \macro{\childdocforwardprefix}
% The command |\childdocforwardprefix| redirects
% compilation to the main or a child file by means of a pattern.
% The prefix |#1| in the current filename is replaced by |#2|
% and the suffix of the current filename is kept
% (it is assumed that the filename does not contain the substring `|~~~|'
% which is used as a delimiter).
% Compilation is handed over to the new file by |\childdocforward|:
%    \begin{macrocode}
\newcommand{\childdocforwardprefix}[3][]
{
  \begingroup
    \def\childdocextract #2##1~~~{\def\childdoctmp{\childdocforward[#1]{#3##1}}}
    \expandafter\childdocextract\childdocname~~~
    \expandafter
  \endgroup
  \childdoctmp
}
%    \end{macrocode}

% \macro{\childdoc}
% The deprecated macro |\childdoc| is a legacy version of |\childdocmain|:
%    \begin{macrocode}
\newcommand{\childdoc}{\childdocmain}
%    \end{macrocode}

% \macro{\childdocredirect}
% The deprecated macro |\childdocredirect| is a legacy version
% of |\childdocforward| and |\childdocforwardprefix|:
%    \begin{macrocode}
\newcommand{\childdocredirect}[2][]
{
  \begingroup
    \if?#1?
      \def\childdoctmp{\childdocforward{#2}}
    \else
      \def\childdoctmp{\childdocforwardprefix{#1}{#2}}
    \fi
    \expandafter
  \endgroup
  \childdoctmp
}
%    \end{macrocode}

%\iffalse
%</package>
%\fi
%
\endinput

\childdocforward{cdocsamp}
%    \end{macrocode}

%\iffalse
%</sampledraft>
%\fi
%
% %%%%%%%%%%%%%%%%%%%%%%%%%%%%%%%%%%%%%%
% \paragraph{Forwarding for Final Version of the Chapters.}
%
% The following forwarding files |cdocsfn1.tex| and |cdocsfn2.tex|
% (with identical content)
% compile the final versions of the child documents
% |cdocsch1.tex| and |cdocsch2.tex|, respectively:
%\iffalse
%<*samplefinal>
%\fi
%    \begin{macrocode}
\def\version{final}
% \iffalse
%
% childdoc.dtx Copyright (C) 2017-2018 Niklas Beisert
%
% This work may be distributed and/or modified under the
% conditions of the LaTeX Project Public License, either version 1.3
% of this license or (at your option) any later version.
% The latest version of this license is in
%   http://www.latex-project.org/lppl.txt
% and version 1.3 or later is part of all distributions of LaTeX
% version 2005/12/01 or later.
%
% This work has the LPPL maintenance status `maintained'.
%
% The Current Maintainer of this work is Niklas Beisert.
%
% This work consists of the files childdoc.dtx and childdoc.ins
% and the derived files childdoc.def and cdocsamp.tex with
% cdocsch1.tex, cdocsch2.tex, cdocsdrf.tex, cdocsfn1.tex, cdocsfn2.tex.
%
%<package>\ifdefined\childdocmain\endinput\fi
%<package>\ProvidesFile{childdoc.def}[2018/12/30 v2.0 child document driver]
%<samplemain>\ProvidesFile{cdocsamp.tex}[2018/12/30 v2.0 sample for childdoc]
%<*driver>
%\ProvidesFile{childdoc.drv}[2018/12/30 v2.0 childdoc reference manual file]
\PassOptionsToClass{10pt,a4paper}{article}
\documentclass{ltxdoc}

\usepackage[margin=35mm]{geometry}
\usepackage{hyperref}
\usepackage{hyperxmp}
\usepackage[usenames]{color}

\hypersetup{colorlinks=true}
\hypersetup{pdfstartview=FitH}
\hypersetup{pdfpagemode=UseNone}
\hypersetup{pdfsource={}}
\hypersetup{pdflang={en-UK}}
\hypersetup{pdfcopyright={Copyright 2017-2018 Niklas Beisert.
  This work may be distributed and/or modified under the
  conditions of the LaTeX Project Public License, either version 1.3
  of this license or (at your option) any later version.}}
\hypersetup{pdflicenseurl={http://www.latex-project.org/lppl.txt}}
\hypersetup{pdfcontactaddress={ETH Zurich, ITP, HIT K,
  Wolfgang-Pauli-Strasse 27}}
\hypersetup{pdfcontactpostcode={8093}}
\hypersetup{pdfcontactcity={Zurich}}
\hypersetup{pdfcontactcountry={Switzerland}}
\hypersetup{pdfcontactemail={nbeisert@itp.phys.ethz.ch}}
\hypersetup{pdfcontacturl={http://people.phys.ethz.ch/\xmptilde nbeisert/}}

\newcommand{\secref}[1]{\hyperref[#1]{section \ref*{#1}}}

\parskip1ex
\parindent0pt
\let\olditemize\itemize
\def\itemize{\olditemize\parskip0pt}

\begin{document}

\title{The \textsf{childdoc} Package}
\hypersetup{pdftitle={The childdoc Package}}
\author{Niklas Beisert\\[2ex]
  Institut f\"ur Theoretische Physik\\
  Eidgen\"ossische Technische Hochschule Z\"urich\\
  Wolfgang-Pauli-Strasse 27, 8093 Z\"urich, Switzerland\\[1ex]
  \href{mailto:nbeisert@itp.phys.ethz.ch}
  {\texttt{nbeisert@itp.phys.ethz.ch}}}
\hypersetup{pdfauthor={Niklas Beisert}}
\hypersetup{pdfsubject={Manual for the LaTeX2e Package childdoc}}
\date{30 December 2018, \textsf{v2.0}}
\maketitle

\begin{abstract}\noindent
\textsf{childdoc} is a \LaTeXe{} package
that enables the direct compilation
of document sections included by |\include|
to individual files.
\end{abstract}

\begingroup
\parskip0ex
\tableofcontents
\endgroup

%%%%%%%%%%%%%%%%%%%%%%%%%%%%%%%%%%%%%%%%%%%%%%%%%%%%%%%%%%%%%%%%%%%%%%%%%%%%%%%%
%%%%%%%%%%%%%%%%%%%%%%%%%%%%%%%%%%%%%%%%%%%%%%%%%%%%%%%%%%%%%%%%%%%%%%%%%%%%%%%%
\section{Introduction}

\LaTeX{} provides a mechanism to structure a large document (such as a book)
into a main file and several child files (containing the chapters)
using the |\include| command.
This mechanism is beneficial for documents
which span hundreds of pages in order to
make the source file(s) more manageable.
Moreover, compilation can be restricted to
selected child files by means of the |\includeonly| command.
The latter feature can be used to reduce the compilation time while editing
(this was significantly more useful in the earlier days of \LaTeX{})
or to generate a smaller document which is easier to navigate.
Another application of |\includeonly| is to generate
documents consisting of selected parts of the complete document.

However, there are a few drawbacks of the plain |\include| mechanism:
\begin{itemize}
\item
The child files cannot be compiled on their own,
they can only be compiled via the main file.
A naive editing environment
(such as a text editor with an option
to have the current file processed by \LaTeX)
may require one to switch to the main file before compiling;
attempting to compile the child file produces errors.
\item
The main file must be modified (each time)
to adjust the |\includeonly| command
to the present needs. This easily leaves the main file in a messy state.
\item
The generated document will always carry the filename
of the main document. This is inconvenient if
several child files are to be compiled and
to be kept for distribution.
\end{itemize}

The present package provides a simple interface
to make child files individually compilable by \LaTeX{}.
Compiling a child file then has the same effect as compiling
the main file with an |\includeonly| command
to select the appropriate child.
Moreover the generated document will carry the name of the child
rather than the main file.
This resolves all three above issues.

This feature is meant to make the editing of books,
thesis documents and lecture notes somewhat more convenient.
However, the package can also be used efficiently for
composing a series of documents (such as exercise sheets)
which are typically distributed individually.
It then assists the author in generating the individual documents
(potentially in different versions)
as well as a document containing the collected series.
Another application is in developing style files
or other kinds of included material
where compilation of the style file could redirect
to a sample or test file.

%%%%%%%%%%%%%%%%%%%%%%%%%%%%%%%%%%%%%%%%%%%%%%%%%%%%%%%%%%%%%%%%%%%%%%%%%%%%%%%%
%%%%%%%%%%%%%%%%%%%%%%%%%%%%%%%%%%%%%%%%%%%%%%%%%%%%%%%%%%%%%%%%%%%%%%%%%%%%%%%%
\section{Usage}

First of all, the package \textsf{childdoc} is \emph{not} a standard
\LaTeXe{} |.sty| style file! Therefore it needs to be invoked in
a non-standard way.

%%%%%%%%%%%%%%%%%%%%%%%%%%%%%%%%%%%%%%%%%%%%%%%%%%%%%%%%%%%%%%%%%%%%%%%%%%%%%%%%
\subsection{Included Files}
\label{sec:include}

%%%%%%%%%%%%%%%%%%%%%%%%%%%%%%%%%%%%%%%%
\DescribeMacro{\childdocmain}
To use the package, add the commands
\begin{center}
\begin{tabular}{l}
|\input{childdoc.def}|\\
|\childdocmain{}|\\
\end{tabular}
\end{center}
at the very top of the main \LaTeX{} file,
in particular \emph{before} the |\documentclass| statement!
The argument of |\childdocmain| should be left empty
(but it must be present).

%%%%%%%%%%%%%%%%%%%%%%%%%%%%%%%%%%%%%%%%
\DescribeMacro{\childdocof}
Furthermore, add the commands
\begin{center}
\begin{tabular}{l}
|\input{childdoc.def}|\\
|\childdocof{|\textit{main}|}|\\
\end{tabular}
\end{center}
at the top of every child file \textit{child}
which is included by |\include{|\textit{child}|}|
from within the main file
(or at least for those files to be compiled individually).
The argument \textit{main} must be the filename of the main file.

There are a couple of
considerations in setting up the main and child documents:

%%%%%%%%%%%%%%%%%%%%%%%%%%%%%%%%%%%%%%%%
\paragraph{Restrictions.}

Please note the following restrictions:
\begin{itemize}
\item
|\childdocmain| must be called with one argument \textit{main}
to ensure compatibility with earlier version of the package.
It must either be empty (|\childdocmain{}|)
or precisely match the filename of the main file in which it is specified.
See \secref{sec:detection} for further information.
\item
The filename \textit{main} must be specified without the |.tex| extension.
\item
The filename \textit{main} is case sensitive
(even in case-insensitive file systems)
due to internal string comparison.
\item
The argument \textit{main} should be fully expanded, it cannot be a macro.
\item
Subdirectories and special characters should be avoided in filenames.
\item
The command |\childdocmain{|\textit{main}|}| must be followed by a whitespace.
It should not be followed immediately by another command
or by a comment mark `|%|'.
This is because the \TeX{} parser reads the token immediately following
the argument of |\childdocmain| and puts it
at the beginning of every child section;
however, a white\-space is ignored.
\end{itemize}

%%%%%%%%%%%%%%%%%%%%%%%%%%%%%%%%%%%%%%%%
\paragraph{Content of Main File.}

It is advisable to place all content in the child files included by |\include|.
Any output contained in the main file will appear in all child documents
unless suppressed manually;
it cannot be suppressed automatically by the |\includeonly| directive
and thus should normally be avoided.
A method to include some content in the main file
by means of conditional processing is described in \secref{sec:conditional}.

%%%%%%%%%%%%%%%%%%%%%%%%%%%%%%%%%%%%%%%%
\paragraph{Page Numbering.}

When only a part of the document is compiled,
the appropriate numbering of pages
(as well as other status parameters)
is determined from the |.aux| files.
The latter contain information from previous passes.
However this information needs to propagate through
all intermediate child documents.
Therefore the page numbering in child documents may well
be inconsistent until the complete document is compiled at least once.

A useful (if unconventional) way to always ensure a consistent
page numbering is to restart the numbering in each child document
and denote the pages by `\textit{child}|.|\textit{page}'
where \textit{child} represents the chapter/section number of the child file.
This can be achieved by the command
|\numberwithin{page}{|\textit{child}|}|
of the \textsf{amsmath} package
where \textit{child} can be |chapter| or |section|
depending on the chosen structuring.
Alternatively, one can modify the macro |\thepage| appropriately
and reset the counter |page| at the start of each child file.

%%%%%%%%%%%%%%%%%%%%%%%%%%%%%%%%%%%%%%%%%%%%%%%%%%%%%%%%%%%%%%%%%%%%%%%%%%%%%%%%
\subsection{Conditional Processing}
\label{sec:conditional}

The package provides a mechanism to compile different versions
of a document. To customise the versions further some conditional processing
can come in handy to distinguish which version is being compiled.
The package provides two macros to describe the compilation context:

%%%%%%%%%%%%%%%%%%%%%%%%%%%%%%%%%%%%%%%%
\DescribeMacro{\ifchilddoc}
The conditional |\ifchilddoc| distinguishes between the compilation of
child documents and the main document:
%
\begin{center}
|\ifchilddoc |\textit{child-code}| |[|\||else |\textit{main-code}]| \||fi|
\end{center}

%%%%%%%%%%%%%%%%%%%%%%%%%%%%%%%%%%%%%%%%
\DescribeMacro{\childdocname}
\DescribeMacro{\childdocjob}
The macro |\childdocname| contains the filename (without extension)
of the main or child file being processed.
Note that |\childdocjob| will always contain the name of the main file.

%%%%%%%%%%%%%%%%%%%%%%%%%%%%%%%%%%%%%%%%
\paragraph{Title Page.}

Conditional processing can be used to include a title or banner page
in the main document when proper precautions are taken.
Importantly, the code in the main file should ensure that the page counter
(as well as other status parameters which are stored in the |.aux| files)
takes the same value after the conditional processing.
Otherwise the page numbers may take divergent values
depending on which part is compiled.

For example, a title page could be declared by:
%
\begin{center}
\begin{tabular}{l}
|\ifchilddoc\||else|\\
|\addtocounter{page}{-1}|\\
\textit{code for title page}\\
|\newpage|\\
|\||fi|
\end{tabular}
\end{center}
%
A banner page for the child documents can be generated by:
%
\begin{center}
\begin{tabular}{l}
|\ifchilddoc|\\
|\addtocounter{page}{-1}|\\
\textit{code for banner page}\\
|\newpage|\\
|\||fi|
\end{tabular}
\end{center}
%
Here one could write a message such as:
\begin{center}
|This is the part \childdocname{} of \childdocjob{}.|
\end{center}

%%%%%%%%%%%%%%%%%%%%%%%%%%%%%%%%%%%%%%%%%%%%%%%%%%%%%%%%%%%%%%%%%%%%%%%%%%%%%%%%
\subsection{Flags}
\label{sec:flags}

The package makes it easy to generate different versions
of the main or child documents.
To this end compilation flags can be defined
and assigned different default values.
They will be particularly useful in conjunction
with the forwarding mechanism described in \secref{sec:forward}.

For example, it may be useful to have a flag |\version|
which can be set to |draft| or |final|.
The document source will contain some conditional code
depending on the value of |\version|.
Suppose further, the flag should default to |final| for the main file
and to |draft| for child files
which is a natural assignment for editing the document.
This is achieved by placing the following code
in the preamble of the main document
(below the |\childdocmain| directive):
%
\begin{center}
\begin{tabular}{l}
|\ifchilddoc|\\
|\providecommand{\version}{draft}|\\
|\||else|\\
|\providecommand{\version}{final}|\\
|\||fi|
\end{tabular}
\end{center}
%
The definition by |\providecommand| makes sure
that previous definitions are not overwritten.
Further statements |\providecommand{\version}{...}|
can thus be added before the above code to override it.

For the main file, one might add a line
(between |\childdocmain| and the above block)
%
\begin{center}
|%\ifchilddoc\||else\providecommand{\version}{draft}\||fi|
\end{center}
%
which can be uncommented to produce a draft version.
Likewise one can add a line to the very top of a child file
(above the |\childdocof{|\textit{main}|}| directive)
%
\begin{center}
|%\providecommand{\version}{final}|
\end{center}
%
which can be uncommented to produce the final version of this child document.

%%%%%%%%%%%%%%%%%%%%%%%%%%%%%%%%%%%%%%%%%%%%%%%%%%%%%%%%%%%%%%%%%%%%%%%%%%%%%%%%
\subsection{Forwarding}
\label{sec:forward}

Different versions of the main or child documents
using compilation flags as described in \secref{sec:flags}
can be (permanently) stored in different files
for convenient compilation, viewing and distribution.
To this end, the package defines a command
to pass on compilation to a different file:

%%%%%%%%%%%%%%%%%%%%%%%%%%%%%%%%%%%%%%%%
\DescribeMacro{\childdocforward}
The command |\childdocforward| redirects processing to
another source file:
%
\begin{center}
\begin{tabular}{l}
|\input{childdoc.def}|\\
|\childdocforward[|\textit{main}|]{|\textit{dest}|}|\\
\end{tabular}
\end{center}
%
The argument \textit{dest} is the destination file
(without extension).
It should be the main file or one of the child files.
Note that further \textsf{childdoc} directives
such as |\childdocof| and |\childdocforward|
in the indicated file will be processed in this form.
The optional argument \textit{main}
passes on directly to the main file \textit{main}
while pretending to compile the child \textit{dest}.
This form behaves as if \textit{dest}
issues |\childdocof{|\textit{main}|}| right away,
and no further \textsf{childdoc} directives will be processed.

%%%%%%%%%%%%%%%%%%%%%%%%%%%%%%%%%%%%%%%%
\DescribeMacro{\...prefix}
In the alternative form |\childdocforwardprefix|,
%
\begin{center}
\begin{tabular}{l}
|\input{childdoc.def}|\\
|\childdocforwardprefix[|\textit{main}|]{|\textit{prefix}|}{|\textit{dest}|}|
\end{tabular}
\end{center}
%
the destination file is determined by a pattern
depending on the current file:
To make this work, the current file must be called
`{\textit{prefix}\hspace{0.2em}\textit{suffix}}'
with \textit{prefix} matching precisely the argument.
Processing is then passed on to the file
`{\textit{dest}\hspace{0.2em}\textit{suffix}}'.
Surely, the same effect is achieved by
directly specifying the
argument `{\textit{dest}\hspace{0.2em}\textit{suffix}}'
in the first form.
However, that requires to set up a different file
for each child. With the alternative form of the command
all these files can have exactly the same content
which simplifies setting them up and maintaining them.

For example, the following file |draft.tex|
with a compilation flag |\version| as described in \secref{sec:flags}
compiles the main document as a draft:
%
\begin{center}
\begin{tabular}{l}
|\def\version{draft}|\\
|\input{childdoc.def}|\\
|\childdocforward{|\textit{main}|}|
\end{tabular}
\end{center}
%
Likewise, the following files |final|\textit{nn}|.tex|
compile the final version of the child document
|child|\textit{nn}|.tex|:
%
\begin{center}
\begin{tabular}{l}
|\def\version{final}|\\
|\input{childdoc.def}|\\
|\childdocforwardprefix{final}{child}|
\end{tabular}
\end{center}
%

Note that when several versions of a main file and/or of each child file
are to be generated, it may be convenient to set up a |Makefile| or
shell script to automatise the process.

%%%%%%%%%%%%%%%%%%%%%%%%%%%%%%%%%%%%%%%%%%%%%%%%%%%%%%%%%%%%%%%%%%%%%%%%%%%%%%%%
\subsection{Command Line Processing}
\label{sec:commandline}

The effect of redirection files can also be achieved by invoking
the \LaTeX{} compiler with a more elaborate command line.
Most conveniently this should be done as part
of a shell script or a |Makefile|.

When using \textsf{childdoc} in the main file, the following
command lines effectively perform a redirection
(note that depending on the shell being used,
backslashes may have to be doubled: `|\|' $\to$ `|\\|'):
%
\begin{center}
|... -jobname "|\textit{target}|" |\\|"|[\textit{flags}]%
|\input{childdoc.def}\childdocforward[|\textit{main}|]{|\textit{dest}|}"|
\end{center}
%
Here \textit{target} is the name of the output file,
\textit{main} is the name of the main file
and \textit{dest} is the name of the main or child file to be processed
(all filenames without extensions).
The optional argument \textit{main} can be omitted
if \textit{main} matches \textit{dest}.
Optionally, compilation \textit{flags} can be defined via |\def| commands.
This command line makes the \TeX{} engine believe
it is compiling the file \textit{target}
whose content is specified as the latter parameter.
The provided code then forwards the processing to
\textit{main} or \textit{dest} as described in \secref{sec:forward}.

%%%%%%%%%%%%%%%%%%%%%%%%%%%%%%%%%%%%%%%%%%%%%%%%%%%%%%%%%%%%%%%%%%%%%%%%%%%%%%%%
\subsection{Include by Input}
\label{sec:input}

Including child documents by |\include| has some restrictions by design.
Most notably, the content of a child document always occupies
its own set of pages; pages cannot be shared between child documents.
Usually, this behaviour makes perfect sense
because each child document contain an essential part of the document.
However, in some situations it may be desirable to compose
a document from a collection of parts
without having mandatory page breaks between then.
For this case, the package
provides a mechanism to include parts
by |\input| which can also be processed individually.
However, by construction this mechanism
requires manual handling of the content to be output.

%%%%%%%%%%%%%%%%%%%%%%%%%%%%%%%%%%%%%%%%
\DescribeMacro{\ifchilddocmanual}
The main file should be prepared as usual, see \secref{sec:include}.
However, the document body must make a distinction
between processing of an individual part and of the main document, e.g.:
%
\begin{center}
\begin{tabular}{l}
|\ifchilddocmanual|\\
|\input{\childdocname}|\\
|\||else|\\
\textit{document body with }|\input{|\textit{part}|}|\\
|\||fi|
\end{tabular}
\end{center}
%
The conditional |\ifchilddocmanual| is true whenever
a part to be included by |\input| is being compiled,
and the name of the part is stored in |\childdocname|.

%%%%%%%%%%%%%%%%%%%%%%%%%%%%%%%%%%%%%%%%
\DescribeMacro{\childdocby}
Each part to be included by |\input| should start with:
%
\begin{center}
\begin{tabular}{l}
|\input{childdoc.def}|\\
|\childdocby{|\textit{main}|}|\\
\end{tabular}
\end{center}
%
The directive |\childdocby| is similar to |\childdocof|
described in \secref{sec:include},
but the subsequent selection of content must be done manually.
To that end, both |\ifchilddoc| and |\ifchilddocmanual|
will be true upon processing of a part,
and the name of the part is stored in |\childdocname|.
Note that |\jobname| will be set to the filename of the current part
so that each part receives an individual |.aux| file
that does not interfere with the |.aux| file(s) of the main document.
This behaviour can be altered by the alternative form
|\childdocby[*]{|\textit{main}|}| (with a non-empty optional argument)
which uses the |.aux| file of the main document
by setting |\jobname| to \textit{main}.

%%%%%%%%%%%%%%%%%%%%%%%%%%%%%%%%%%%%%%%%%%%%%%%%%%%%%%%%%%%%%%%%%%%%%%%%%%%%%%%%
\subsection{Driver Development}
\label{sec:driver}

The \textsf{childdoc} mechanism can also be use for the development
of definition files such as \LaTeX{} styles or classes.
This case differs from the above setup with multiple parts
included by |\include| in that no |\includeonly| should be invoked.
This can be achieved by starting the include file
(before |\ProvidesPackage|) with:
%
\begin{center}
\begin{tabular}{l}
|\input{childdoc.def}|\\
|\childdocforward{|\textit{main}|}|\\
\end{tabular}
\end{center}
%
or alternatively with:
%
\begin{center}
\begin{tabular}{l}
|\input{childdoc.def}|\\
|\childdocby{|\textit{main}|}|\\
\end{tabular}
\end{center}
%
Both forms have slightly different effects as described above.
The main file is prepared as usual, see \secref{sec:include}.

%%%%%%%%%%%%%%%%%%%%%%%%%%%%%%%%%%%%%%%%%%%%%%%%%%%%%%%%%%%%%%%%%%%%%%%%%%%%%%%%
\subsection{Legacy Detection}
\label{sec:detection}

The directive |\childdocmain| in the main file can detect
whether the complete document or merely a child is to be compiled
even without using the directive |\childdocof|.
This method is deprecated because it is less robust
and there is no compelling reason to use it;
it is merely provided for backward compatibility
and it may be removed in future versions.

If the detection mechanism is to be used,
it is mandatory to correctly specify
the filename of the main file as the argument of |\childdocmain|:
%
\begin{center}
\begin{tabular}{l}
|\input{childdoc.def}|\\
|\childdocmain{|\textit{main}|}|\\
\end{tabular}
\end{center}
%
If |\jobname| does not match the argument \textit{main} of |\childdocmain|,
it is assumed that |\jobname| points to the child file to be compiled.
When using |\childdocmain| with the main file specified as argument,
it suffices to start a child file
with just |\input{|\textit{main}|}|
without loading of the package and using |\childdocof|.
If instead all processing is done
with the appropriate \textsf{childdoc} directives,
the argument of \textit{main} of |\childdocmain| can be empty.

An alternative version of the command line processing described
in \secref{sec:commandline} using the detection mechanism reads:
%
\begin{center}
|... -jobname "|\textit{target}|" "|[\textit{flags}]%
[|\def\jobname{|\textit{dest}|}|]|\input{|\textit{main}|}"|
\end{center}

%%%%%%%%%%%%%%%%%%%%%%%%%%%%%%%%%%%%%%%%%%%%%%%%%%%%%%%%%%%%%%%%%%%%%%%%%%%%%%%%
\subsection{Manual Code}
\label{sec:manual}

In case one cannot be certain whether the definitions file |childdoc.def|
is installed on the target \TeX{} distribution
and one prefers not to ship it,
it is conceivable to paste a few relevant commands into the sources.

To that end, drop all statements |\input{childdoc.def}|
and perform the replacements as outlined below.
Instead of |\childdocmain{|\textit{main}|}| add the following code
to the top of the main file:
%
\begin{center}
\begin{tabular}{l}
|\||ifdefined\childdocname\endinput\||fi\newif\ifchilddoc|\\
|\edef\childdocname{\scantokens\expandafter{\jobname\noexpand}}|\\
|\def\childdocmain{|\textit{main}|}\||ifx\childdocmain\childdocname\||else|\\
|\childdoctrue\includeonly{\childdocname}\let\jobname\childdocmain\||fi|\\
\end{tabular}
\end{center}
%
Instead of |\childdocof{|\textit{main}|}| just include the main file
at the top of each child file:
%
\begin{center}
|\input{|\textit{main}|}|
\end{center}
%
A simple redirection |\childdocforward{|\textit{dest}|}| is achieved by:
%
\begin{center}
|\def\jobname{|\textit{dest}|}\input{\jobname}|
\end{center}
%
The redirection with prefix
|\childdocforwardprefix[|\textit{prefix}|]{|\textit{dest}|}|
is accomplished by:
%
\begin{center}
\begin{tabular}{l}
|{\edef\jobname{\scantokens\expandafter{\jobname\noexpand}}|\\
|\def\redirectjob |\textit{prefix}|#1~~~{\gdef\jobname{|\textit{dest}|#1}}|\\
|\expandafter\redirectjob\jobname~~~}\input{\jobname}|
\end{tabular}
\end{center}

In an alternative approach,
child documents can be compiled by a specific command line
without additional code or specific definitions:
%
\begin{center}
|... -jobname "|\textit{target}|" "|[\textit{flags}]%
|\includeonly{|\textit{dest}|}\input{|\textit{main}|}"|
\end{center}
%

%%%%%%%%%%%%%%%%%%%%%%%%%%%%%%%%%%%%%%%%%%%%%%%%%%%%%%%%%%%%%%%%%%%%%%%%%%%%%%%%
%%%%%%%%%%%%%%%%%%%%%%%%%%%%%%%%%%%%%%%%%%%%%%%%%%%%%%%%%%%%%%%%%%%%%%%%%%%%%%%%
\section{Information}

%%%%%%%%%%%%%%%%%%%%%%%%%%%%%%%%%%%%%%%%%%%%%%%%%%%%%%%%%%%%%%%%%%%%%%%%%%%%%%%%
\subsection{Copyright}

Copyright \copyright{} 2017--2018 Niklas Beisert

This work may be distributed and/or modified under the
conditions of the \LaTeX{} Project Public License, either version 1.3
of this license or (at your option) any later version.
The latest version of this license is in
  \url{http://www.latex-project.org/lppl.txt}
and version 1.3 or later is part of all distributions of \LaTeX{}
version 2005/12/01 or later.

This work has the LPPL maintenance status `maintained'.

The Current Maintainer of this work is Niklas Beisert.

This work consists of the files |README.txt|, |childdoc.ins| and |childdoc.dtx|
as well as the derived files |childdoc.def|, |cdocsamp.tex|
with |cdocsch1.tex|, |cdocsch2.tex|, |cdocspt3.tex|, |cdocspt4.tex|,
|cdocsdrf.tex|, |cdocsfn1.tex|, |cdocsfn2.tex|
as well as |childdoc.pdf|.

%%%%%%%%%%%%%%%%%%%%%%%%%%%%%%%%%%%%%%%%%%%%%%%%%%%%%%%%%%%%%%%%%%%%%%%%%%%%%%%%
\subsection{Files and Installation}

The package consists of the files:
%
\begin{center}
\begin{tabular}{ll}
    |README.txt|   & readme file \\
    |childdoc.ins| & installation file \\
    |childdoc.dtx| & source file \\
    |childdoc.def| & definition file \\
    |cdocsamp.tex| & sample main file \\
    |cdocsch1.tex| & sample include file \\
    |cdocsch2.tex| & sample include file \\
    |cdocspt3.tex| & sample part file \\
    |cdocspt4.tex| & sample part file \\
    |cdocsdrf.tex| & sample redirection file \\
    |cdocsfn1.tex| & sample redirection file \\
    |cdocsfn2.tex| & sample redirection file \\
    |childdoc.pdf| & manual
\end{tabular}
\end{center}
%
The distribution consists of the files
|README.txt|, |childdoc.ins| and |childdoc.dtx|.
%
\begin{itemize}
\item
Run (pdf)\LaTeX{} on |childdoc.dtx|
to compile the manual |childdoc.pdf| (this file).
\item
Run \LaTeX{} on |childdoc.ins| to create the definitions file |childdoc.def|
and the sample |cdocsamp.tex| with include files
|cdocsch1.tex|, |cdocsch2.tex|, |cdocspt3.tex|, |cdocspt4.tex|,
|cdocsdrf.tex|, |cdocsfn1.tex|, |cdocsfn2.tex|.
Then copy the file |childdoc.def| to an appropriate directory of your \LaTeX{}
distribution, e.g.\ \textit{texmf-root}|/tex/latex/childdoc|.
\end{itemize}

%%%%%%%%%%%%%%%%%%%%%%%%%%%%%%%%%%%%%%%%%%%%%%%%%%%%%%%%%%%%%%%%%%%%%%%%%%%%%%%%
\subsection{Related CTAN Packages}

There are several other packages which offer a similar functionality:
%
\begin{itemize}
\item
The packages
\href{http://ctan.org/pkg/docmute}{\textsf{docmute}},
\href{http://ctan.org/pkg/includex}{\textsf{includex}} and
\href{http://ctan.org/pkg/standalone}{\textsf{standalone}}
provide commands to include only the document body of
a child file thus allowing both files to be compiled individually.
\item
The packages \href{http://ctan.org/pkg/subdocs}{\textsf{subdocs}}
and \href{http://ctan.org/pkg/subfiles}{\textsf{subfiles}}
provide structures in which the main and child documents can be
encapsulated and allowing them to be compiled individually.
The inclusion mechanism is different from the conventional |\include|.
\item
The package \href{http://ctan.org/pkg/combine}{\textsf{combine}}
is an elaborate solution to combine several documents into one.
\end{itemize}
%
See also the CTAN topic \href{http://ctan.org/topic/subdocs}{\textsf{subdocs}}
for further related packages.
The present package differs from the above solutions in that
a document structure constructed with the conventional |\include| mechanism
just needs two extra commands at the top of every file
such that all constituent files can be compiled individually.

%%%%%%%%%%%%%%%%%%%%%%%%%%%%%%%%%%%%%%%%%%%%%%%%%%%%%%%%%%%%%%%%%%%%%%%%%%%%%%%%
%\subsection{Feature Suggestions}
%
%The following is a list of features which may be useful for future
%versions of this package:
%%
%\begin{itemize}
%\item
%\ldots
%\end{itemize}

%%%%%%%%%%%%%%%%%%%%%%%%%%%%%%%%%%%%%%%%%%%%%%%%%%%%%%%%%%%%%%%%%%%%%%%%%%%%%%%%
\subsection{Revision History}

%%%%%%%%%%%%%%%%%%%%%%%%%%%%%%%%%%%%%%%%
\paragraph{v2.0:} 2018/12/30

\begin{itemize}
\item
immediate forward processing
\item
added |\childdocby| mechanism
\item
manual restructured
\end{itemize}

%%%%%%%%%%%%%%%%%%%%%%%%%%%%%%%%%%%%%%%%
\paragraph{v1.6:} 2018/01/17

\begin{itemize}
\item
application for development of include files
\item
corrections to manual
\end{itemize}

%%%%%%%%%%%%%%%%%%%%%%%%%%%%%%%%%%%%%%%%
\paragraph{v1.5:} 2017/05/21

\begin{itemize}
\item
more complete structuring introduced
\item
|\childdocof| introduced
\item
|\childdoc| renamed to |\childdocmain|
\item
|\childredirect| renamed to |\childdocforward| and |\childdocforwardprefix|
and functionality expanded
\end{itemize}

%%%%%%%%%%%%%%%%%%%%%%%%%%%%%%%%%%%%%%%%
\paragraph{v1.0:} 2017/04/27

\begin{itemize}
\item
manual and install package
\item
first version published on CTAN
\end{itemize}

%%%%%%%%%%%%%%%%%%%%%%%%%%%%%%%%%%%%%%%%
\paragraph{v0.6:} 2017/04/26

\begin{itemize}
\item
redirection mechanism added
\end{itemize}

%%%%%%%%%%%%%%%%%%%%%%%%%%%%%%%%%%%%%%%%
\paragraph{v0.5:} 2017/04/26

\begin{itemize}
\item
functionality in definition file
\end{itemize}


%%%%%%%%%%%%%%%%%%%%%%%%%%%%%%%%%%%%%%%%%%%%%%%%%%%%%%%%%%%%%%%%%%%%%%%%%%%%%%%%
%%%%%%%%%%%%%%%%%%%%%%%%%%%%%%%%%%%%%%%%%%%%%%%%%%%%%%%%%%%%%%%%%%%%%%%%%%%%%%%%
%%%%%%%%%%%%%%%%%%%%%%%%%%%%%%%%%%%%%%%%%%%%%%%%%%%%%%%%%%%%%%%%%%%%%%%%%%%%%%%%
\appendix

\settowidth\MacroIndent{\rmfamily\scriptsize 000\ }

 \DocInput{childdoc.dtx}

\end{document}
%</driver>
% \fi
%
% %%%%%%%%%%%%%%%%%%%%%%%%%%%%%%%%%%%%%%%%%%%%%%%%%%%%%%%%%%%%%%%%%%%%%%%%%%%%%%
% %%%%%%%%%%%%%%%%%%%%%%%%%%%%%%%%%%%%%%%%%%%%%%%%%%%%%%%%%%%%%%%%%%%%%%%%%%%%%%
% \section{Sample}
%\iffalse
%<*samplemain>
%\fi
%
% The following presents a sample document
% with two chapters, two parts, a title page,
% a compile flag as well as three forwarding files to set the flag.
% It consists of eight |.tex| files:
% \begin{center}
% \begin{tabular}{ll}
% |cdocsamp.tex|&main file\\
% |cdocsch1.tex|&include file for chapter 1\\
% |cdocsch2.tex|&include file for chapter 2\\
% |cdocspt3.tex|&include file for part 3\\
% |cdocspt4.tex|&include file for part 4\\
% |cdocsdrf.tex|&forwarding file for main file in draft mode\\
% |cdocsfi1.tex|&forwarding file for final version of chapter 1\\
% |cdocsfi2.tex|&forwarding file for final version of chapter 2\\
% \end{tabular}
% \end{center}
% Each of the eight files can be compiled directly by the \LaTeX{} compiler.
%
% %%%%%%%%%%%%%%%%%%%%%%%%%%%%%%%%%%%%%%
% \paragraph{Main File.}
%
% The main file is called |cdocsamp.tex|.
%
% Load the \textsf{childdoc} definitions and
% declare the filename for the main document:
%    \begin{macrocode}
\input{childdoc.def}
\childdocmain{}
%    \end{macrocode}

% Optional override for |\version| flag:
%    \begin{macrocode}
%%\ifchilddoc\else\providecommand{\version}{draft}\fi
%    \end{macrocode}

% Define the default values for the |\version| flag
% (|final| for the main file and |draft| for childs):
%    \begin{macrocode}
\ifchilddoc
\providecommand{\version}{draft}
\else
\providecommand{\version}{final}
\fi
%    \end{macrocode}

% Load the standard document class:
%    \begin{macrocode}
\documentclass[12pt]{article}
%    \end{macrocode}

% Start the document body:
%    \begin{macrocode}
\begin{document}
%    \end{macrocode}

% Declare a title page.
% Print title, part of document being processed and version flag:
%    \begin{macrocode}
\addtocounter{page}{-1}
\begin{center}
{\LARGE\bfseries{}childdoc example\par}
\vspace{1cm}
\ifchilddoc
\ifchilddocmanual part\else chapter\fi:
`\childdocname' of `\childdocjob'\par
\else
main document: `\childdocjob'\par
\fi
version: \version\par
\end{center}
\newpage
%    \end{macrocode}

% Manually include selected file,
% otherwise process as usual:
%    \begin{macrocode}
\ifchilddocmanual
\section*{part `\childdocname'}
\input{\childdocname}
\else
%    \end{macrocode}

% Include the two chapters:
%    \begin{macrocode}
\include{cdocsch1}
\include{cdocsch2}
%    \end{macrocode}

% Include the two parts unless only chapters should be displayed:
%    \begin{macrocode}
\ifchilddoc\else
\section{part three}
\input{cdocspt3}
\section{part four}
\input{cdocspt4}
\fi
%    \end{macrocode}

% Process as usual until here:
%    \begin{macrocode}
\fi
%    \end{macrocode}

% End of document body:
%    \begin{macrocode}
\end{document}
%    \end{macrocode}
%\iffalse
%</samplemain>
%\fi
%
% %%%%%%%%%%%%%%%%%%%%%%%%%%%%%%%%%%%%%%
% \paragraph{Chapter Include Files.}
%
% The include files are called |cdocsch1.tex| and |cdocsch2.tex|.
%
%\iffalse
%<*samplechap1|samplechap2>
%\fi

% Optional override for |\version| flag:
%    \begin{macrocode}
%%\providecommand{\version}{final}
%    \end{macrocode}

% Include the main document:
%    \begin{macrocode}
\input{childdoc.def}
\childdocof{cdocsamp}
%    \end{macrocode}

%\iffalse
%</samplechap1|samplechap2>
%\fi
%
%\iffalse
%<*samplechap1>
%\fi
% Some text for chapter 1:
%    \begin{macrocode}
\section{one}
some text in chapter one
%    \end{macrocode}

%\iffalse
%</samplechap1>
%\fi
% Some text for chapter 2:
%\iffalse
%<*samplechap2>
%\fi
%    \begin{macrocode}
\section{two}
more text in chapter two
%    \end{macrocode}

%\iffalse
%</samplechap2>
%\fi
%
% %%%%%%%%%%%%%%%%%%%%%%%%%%%%%%%%%%%%%%
% \paragraph{Part Include Files.}
%
% The include files are called |cdocspt3.tex| and |cdocspt4.tex|.
%
%\iffalse
%<*samplepart3|samplepart4>
%\fi

% Optional override for |\version| flag:
%    \begin{macrocode}
%%\providecommand{\version}{final}
%    \end{macrocode}

% Include the main document:
%    \begin{macrocode}
\input{childdoc.def}
\childdocby{cdocsamp}
%    \end{macrocode}

%\iffalse
%</samplepart3|samplepart4>
%\fi
%
%\iffalse
%<*samplepart3>
%\fi
% Some text for part 3:
%    \begin{macrocode}
some text in part three
%    \end{macrocode}

%\iffalse
%</samplepart3>
%\fi
% Some text for part 4:
%\iffalse
%<*samplepart4>
%\fi
%    \begin{macrocode}
more text in part four
%    \end{macrocode}

%\iffalse
%</samplepart4>
%\fi
%
% %%%%%%%%%%%%%%%%%%%%%%%%%%%%%%%%%%%%%%
% \paragraph{Forwarding for a Complete Draft.}
%
% The following forwarding file |cdocsdrf.tex|
% compiles the main document in draft mode:
%\iffalse
%<*sampledraft>
%\fi
%    \begin{macrocode}
\def\version{draft}
\input{childdoc.def}
\childdocforward{cdocsamp}
%    \end{macrocode}

%\iffalse
%</sampledraft>
%\fi
%
% %%%%%%%%%%%%%%%%%%%%%%%%%%%%%%%%%%%%%%
% \paragraph{Forwarding for Final Version of the Chapters.}
%
% The following forwarding files |cdocsfn1.tex| and |cdocsfn2.tex|
% (with identical content)
% compile the final versions of the child documents
% |cdocsch1.tex| and |cdocsch2.tex|, respectively:
%\iffalse
%<*samplefinal>
%\fi
%    \begin{macrocode}
\def\version{final}
\input{childdoc.def}
\childdocforwardprefix[cdocsamp]{cdocsfn}{cdocsch}
%    \end{macrocode}

%\iffalse
%</samplefinal>
%\fi
%
% %%%%%%%%%%%%%%%%%%%%%%%%%%%%%%%%%%%%%%
% \paragraph{Command Line Processing.}
%
% The following three command lines generate the output files
% |cdocscld|, |cdocscl1| and |cdocscl2|
% which should be identical to
% |cdocsdrf|, |cdocsch1| and |cdocsfn2|, respectively:
% \begin{center}
% \begin{tabular}{l}
% |latex -jobname cdocscld \|\\
% |  "\def\version{draft}\input{childdoc.def}\childdocforward{cdocsamp}"|\\
% |latex -jobname cdocscl1 \|\\
% |  "\input{childdoc.def}\childdocforward[cdocsamp]{cdocsch1}"|\\
% |latex -jobname cdocscl2 \|\\
% |  "\def\version{final}\input{childdoc.def}\childdocforward{cdocsch2}"|
% \end{tabular}
% \end{center}
% Note that the trailing backslash on each first line
% merely continues the input to the second line
% (for convenient cut ant paste).
% Furthermore, the command |latex| can be replaced by any
% of its alternative versions such as |pdflatex|.
%
% %%%%%%%%%%%%%%%%%%%%%%%%%%%%%%%%%%%%%%%%%%%%%%%%%%%%%%%%%%%%%%%%%%%%%%%%%%%%%%
% %%%%%%%%%%%%%%%%%%%%%%%%%%%%%%%%%%%%%%%%%%%%%%%%%%%%%%%%%%%%%%%%%%%%%%%%%%%%%%
% \section{Implementation}
%\iffalse
%<*package>
%\fi
%
% This section describes the definitions file |childdoc.def|.

% The definitions cannot be loaded using |\usepackage| or |\RequirePackage|
% which has a mechanism to prevent loading a style file more than once.
% When loading the definitions by means of |\input|
% multiple instances have to be prevented manually:
%\iffalse
%This code needs to be before the `\ProvidesFile' directive
%which is defined at the beginning of this file.
%Therefore it is also placed there and commented out here.
%</package>
%<*discard>
%\fi
%    \begin{macrocode}
\ifdefined\childdocmain\endinput\fi
%    \end{macrocode}
%\iffalse
%</discard>
%<*package>
%\fi
%
% \macro{\ifchilddoc}
% \macro{\ifchilddocmanual}
% The conditional |\ifchilddoc| tells whether a
% child (true) or main (false) document is being compiled.
% The conditional |\ifchilddocmanual| tells whether
% the |\includeonly| mechanism is used (false) or
% the selection of child files must be performed manually (true).
% The definitions initialise to false:
%    \begin{macrocode}
\newif\ifchilddoc
\newif\ifchilddocmanual
%    \end{macrocode}

% \macro{\childdocname}
% \macro{\childdocjob}
% The macro |\childdocname| stores the name of the main document
% to be compiled. The macro |\childdocjob| stores the name of
% the document on which the \LaTeX{} compiler was originally invoked.
% The content of |\jobname| cannot be compared
% to filenames specified in the source due to different catcodes.
% The following code rescans |\jobname|, stores the result
% in |\childdocname| and saves a copy in |\childdocjob|:
%    \begin{macrocode}
\edef\childdocname{\scantokens\expandafter{\jobname\noexpand}}
\let\childdocjob\childdocname
%    \end{macrocode}

% \macro{\childdocdisable}
% The macro |\childdocdisable| prevents the main file
% from being processed more than once.
% At this stage, the main document command |\childdocmain|
% is assumed to be called once again where it should do nothing.
% Any subsequent call to it should prevent
% a secondary processing of the main document
% It overwrites the forwarding commands
% |\childdocof| and |\childdocforward|
% with empty macros to prevent further inclusions of the main document:
%    \begin{macrocode}
\newcommand{\childdocdisable}
{
  \renewcommand{\childdocmain}[1]{\renewcommand{\childdocmain}[1]{\endinput}}
  \renewcommand{\childdocof}[1]{}
  \renewcommand{\childdocby}[2][]{}
  \renewcommand{\childdocforward}[2][]{}
  \renewcommand{\childdocdisable}{}
}
%    \end{macrocode}

% \macro{\childdocmain}
% The macro |\childdocmain| is to be called at the top of the main file
% with nothing or the main filename (without extension) as argument.
% First, it breaks loops.
% If the argument is not empty and does not match |\childdocname|
% (which is set by the first inclusion of |childdoc.def|),
% |\ifchilddoc| is set to true, |\includeonly| is applied to the child file
% and |\jobname| is set to the main file
% (for proper handling of |.aux| files):
%    \begin{macrocode}
\newcommand{\childdocmain}[1]
{
  \childdocdisable\childdocmain{}
  \if?#1?\else
    \begingroup
      \def\childdoctmp{#1}
      \ifx\childdoctmp\childdocname
        \def\childdoctmp{}
      \else
        \def\childdoctmp
        {
          \childdoctrue
          \includeonly{\childdocname}
          \def\childdocjob{#1}
          \def\jobname{#1}
        }
      \fi
      \expandafter
    \endgroup
    \childdoctmp
  \fi
}
%    \end{macrocode}

% \macro{\childdocof}
% The command |\childdocof| redirects
% compilation to the main file |#1|.
%    \begin{macrocode}
\newcommand{\childdocof}[1]
{
  \childdocdisable
  \childdoctrue
  \includeonly{\childdocname}
  \def\jobname{#1}
  \def\childdocjob{#1}
  \input{#1}
}
%    \end{macrocode}

% \macro{\childdocby}
% The command |\childdocby| ....
%    \begin{macrocode}
\newcommand{\childdocby}[2][]
{
  \childdocdisable
  \childdoctrue
  \childdocmanualtrue
  \if?#1?\else
    \def\jobname{#2}
  \fi
  \def\childdocjob{#2}
  \input{#2}
  \endinput
}
%    \end{macrocode}

% \macro{\childdocforward}
% The command |\childdocforward| redirects
% compilation to the main file or
% (if the optional argument is given) a child file.
% Parameters are set as if the main file
% or a child file starting with |\childdocof| was compiled.
% Then compilation is handed over to the main file:
%    \begin{macrocode}
\newcommand{\childdocforward}[2][]
{
  \begingroup
    \if?#1?
      \def\childdoctmp
      {
        \def\childdocname{#2}
        \def\childdocjob{#2}
        \def\jobname{#2}
        \input{#2}
        \endinput
      }
    \else
      \def\childdoctmp
      {
        \childdocdisable
        \def\childdocname{#2}
        \childdoctrue
        \includeonly{#2}
        \def\childdocjob{#1}
        \def\jobname{#1}
        \input{#1}
        \endinput
      }
    \fi
    \expandafter
  \endgroup
  \childdoctmp
}
%    \end{macrocode}

% \macro{\childdocforwardprefix}
% The command |\childdocforwardprefix| redirects
% compilation to the main or a child file by means of a pattern.
% The prefix |#1| in the current filename is replaced by |#2|
% and the suffix of the current filename is kept
% (it is assumed that the filename does not contain the substring `|~~~|'
% which is used as a delimiter).
% Compilation is handed over to the new file by |\childdocforward|:
%    \begin{macrocode}
\newcommand{\childdocforwardprefix}[3][]
{
  \begingroup
    \def\childdocextract #2##1~~~{\def\childdoctmp{\childdocforward[#1]{#3##1}}}
    \expandafter\childdocextract\childdocname~~~
    \expandafter
  \endgroup
  \childdoctmp
}
%    \end{macrocode}

% \macro{\childdoc}
% The deprecated macro |\childdoc| is a legacy version of |\childdocmain|:
%    \begin{macrocode}
\newcommand{\childdoc}{\childdocmain}
%    \end{macrocode}

% \macro{\childdocredirect}
% The deprecated macro |\childdocredirect| is a legacy version
% of |\childdocforward| and |\childdocforwardprefix|:
%    \begin{macrocode}
\newcommand{\childdocredirect}[2][]
{
  \begingroup
    \if?#1?
      \def\childdoctmp{\childdocforward{#2}}
    \else
      \def\childdoctmp{\childdocforwardprefix{#1}{#2}}
    \fi
    \expandafter
  \endgroup
  \childdoctmp
}
%    \end{macrocode}

%\iffalse
%</package>
%\fi
%
\endinput

\childdocforwardprefix[cdocsamp]{cdocsfn}{cdocsch}
%    \end{macrocode}

%\iffalse
%</samplefinal>
%\fi
%
% %%%%%%%%%%%%%%%%%%%%%%%%%%%%%%%%%%%%%%
% \paragraph{Command Line Processing.}
%
% The following three command lines generate the output files
% |cdocscld|, |cdocscl1| and |cdocscl2|
% which should be identical to
% |cdocsdrf|, |cdocsch1| and |cdocsfn2|, respectively:
% \begin{center}
% \begin{tabular}{l}
% |latex -jobname cdocscld \|\\
% |  "\def\version{draft}% \iffalse
%
% childdoc.dtx Copyright (C) 2017-2018 Niklas Beisert
%
% This work may be distributed and/or modified under the
% conditions of the LaTeX Project Public License, either version 1.3
% of this license or (at your option) any later version.
% The latest version of this license is in
%   http://www.latex-project.org/lppl.txt
% and version 1.3 or later is part of all distributions of LaTeX
% version 2005/12/01 or later.
%
% This work has the LPPL maintenance status `maintained'.
%
% The Current Maintainer of this work is Niklas Beisert.
%
% This work consists of the files childdoc.dtx and childdoc.ins
% and the derived files childdoc.def and cdocsamp.tex with
% cdocsch1.tex, cdocsch2.tex, cdocsdrf.tex, cdocsfn1.tex, cdocsfn2.tex.
%
%<package>\ifdefined\childdocmain\endinput\fi
%<package>\ProvidesFile{childdoc.def}[2018/12/30 v2.0 child document driver]
%<samplemain>\ProvidesFile{cdocsamp.tex}[2018/12/30 v2.0 sample for childdoc]
%<*driver>
%\ProvidesFile{childdoc.drv}[2018/12/30 v2.0 childdoc reference manual file]
\PassOptionsToClass{10pt,a4paper}{article}
\documentclass{ltxdoc}

\usepackage[margin=35mm]{geometry}
\usepackage{hyperref}
\usepackage{hyperxmp}
\usepackage[usenames]{color}

\hypersetup{colorlinks=true}
\hypersetup{pdfstartview=FitH}
\hypersetup{pdfpagemode=UseNone}
\hypersetup{pdfsource={}}
\hypersetup{pdflang={en-UK}}
\hypersetup{pdfcopyright={Copyright 2017-2018 Niklas Beisert.
  This work may be distributed and/or modified under the
  conditions of the LaTeX Project Public License, either version 1.3
  of this license or (at your option) any later version.}}
\hypersetup{pdflicenseurl={http://www.latex-project.org/lppl.txt}}
\hypersetup{pdfcontactaddress={ETH Zurich, ITP, HIT K,
  Wolfgang-Pauli-Strasse 27}}
\hypersetup{pdfcontactpostcode={8093}}
\hypersetup{pdfcontactcity={Zurich}}
\hypersetup{pdfcontactcountry={Switzerland}}
\hypersetup{pdfcontactemail={nbeisert@itp.phys.ethz.ch}}
\hypersetup{pdfcontacturl={http://people.phys.ethz.ch/\xmptilde nbeisert/}}

\newcommand{\secref}[1]{\hyperref[#1]{section \ref*{#1}}}

\parskip1ex
\parindent0pt
\let\olditemize\itemize
\def\itemize{\olditemize\parskip0pt}

\begin{document}

\title{The \textsf{childdoc} Package}
\hypersetup{pdftitle={The childdoc Package}}
\author{Niklas Beisert\\[2ex]
  Institut f\"ur Theoretische Physik\\
  Eidgen\"ossische Technische Hochschule Z\"urich\\
  Wolfgang-Pauli-Strasse 27, 8093 Z\"urich, Switzerland\\[1ex]
  \href{mailto:nbeisert@itp.phys.ethz.ch}
  {\texttt{nbeisert@itp.phys.ethz.ch}}}
\hypersetup{pdfauthor={Niklas Beisert}}
\hypersetup{pdfsubject={Manual for the LaTeX2e Package childdoc}}
\date{30 December 2018, \textsf{v2.0}}
\maketitle

\begin{abstract}\noindent
\textsf{childdoc} is a \LaTeXe{} package
that enables the direct compilation
of document sections included by |\include|
to individual files.
\end{abstract}

\begingroup
\parskip0ex
\tableofcontents
\endgroup

%%%%%%%%%%%%%%%%%%%%%%%%%%%%%%%%%%%%%%%%%%%%%%%%%%%%%%%%%%%%%%%%%%%%%%%%%%%%%%%%
%%%%%%%%%%%%%%%%%%%%%%%%%%%%%%%%%%%%%%%%%%%%%%%%%%%%%%%%%%%%%%%%%%%%%%%%%%%%%%%%
\section{Introduction}

\LaTeX{} provides a mechanism to structure a large document (such as a book)
into a main file and several child files (containing the chapters)
using the |\include| command.
This mechanism is beneficial for documents
which span hundreds of pages in order to
make the source file(s) more manageable.
Moreover, compilation can be restricted to
selected child files by means of the |\includeonly| command.
The latter feature can be used to reduce the compilation time while editing
(this was significantly more useful in the earlier days of \LaTeX{})
or to generate a smaller document which is easier to navigate.
Another application of |\includeonly| is to generate
documents consisting of selected parts of the complete document.

However, there are a few drawbacks of the plain |\include| mechanism:
\begin{itemize}
\item
The child files cannot be compiled on their own,
they can only be compiled via the main file.
A naive editing environment
(such as a text editor with an option
to have the current file processed by \LaTeX)
may require one to switch to the main file before compiling;
attempting to compile the child file produces errors.
\item
The main file must be modified (each time)
to adjust the |\includeonly| command
to the present needs. This easily leaves the main file in a messy state.
\item
The generated document will always carry the filename
of the main document. This is inconvenient if
several child files are to be compiled and
to be kept for distribution.
\end{itemize}

The present package provides a simple interface
to make child files individually compilable by \LaTeX{}.
Compiling a child file then has the same effect as compiling
the main file with an |\includeonly| command
to select the appropriate child.
Moreover the generated document will carry the name of the child
rather than the main file.
This resolves all three above issues.

This feature is meant to make the editing of books,
thesis documents and lecture notes somewhat more convenient.
However, the package can also be used efficiently for
composing a series of documents (such as exercise sheets)
which are typically distributed individually.
It then assists the author in generating the individual documents
(potentially in different versions)
as well as a document containing the collected series.
Another application is in developing style files
or other kinds of included material
where compilation of the style file could redirect
to a sample or test file.

%%%%%%%%%%%%%%%%%%%%%%%%%%%%%%%%%%%%%%%%%%%%%%%%%%%%%%%%%%%%%%%%%%%%%%%%%%%%%%%%
%%%%%%%%%%%%%%%%%%%%%%%%%%%%%%%%%%%%%%%%%%%%%%%%%%%%%%%%%%%%%%%%%%%%%%%%%%%%%%%%
\section{Usage}

First of all, the package \textsf{childdoc} is \emph{not} a standard
\LaTeXe{} |.sty| style file! Therefore it needs to be invoked in
a non-standard way.

%%%%%%%%%%%%%%%%%%%%%%%%%%%%%%%%%%%%%%%%%%%%%%%%%%%%%%%%%%%%%%%%%%%%%%%%%%%%%%%%
\subsection{Included Files}
\label{sec:include}

%%%%%%%%%%%%%%%%%%%%%%%%%%%%%%%%%%%%%%%%
\DescribeMacro{\childdocmain}
To use the package, add the commands
\begin{center}
\begin{tabular}{l}
|\input{childdoc.def}|\\
|\childdocmain{}|\\
\end{tabular}
\end{center}
at the very top of the main \LaTeX{} file,
in particular \emph{before} the |\documentclass| statement!
The argument of |\childdocmain| should be left empty
(but it must be present).

%%%%%%%%%%%%%%%%%%%%%%%%%%%%%%%%%%%%%%%%
\DescribeMacro{\childdocof}
Furthermore, add the commands
\begin{center}
\begin{tabular}{l}
|\input{childdoc.def}|\\
|\childdocof{|\textit{main}|}|\\
\end{tabular}
\end{center}
at the top of every child file \textit{child}
which is included by |\include{|\textit{child}|}|
from within the main file
(or at least for those files to be compiled individually).
The argument \textit{main} must be the filename of the main file.

There are a couple of
considerations in setting up the main and child documents:

%%%%%%%%%%%%%%%%%%%%%%%%%%%%%%%%%%%%%%%%
\paragraph{Restrictions.}

Please note the following restrictions:
\begin{itemize}
\item
|\childdocmain| must be called with one argument \textit{main}
to ensure compatibility with earlier version of the package.
It must either be empty (|\childdocmain{}|)
or precisely match the filename of the main file in which it is specified.
See \secref{sec:detection} for further information.
\item
The filename \textit{main} must be specified without the |.tex| extension.
\item
The filename \textit{main} is case sensitive
(even in case-insensitive file systems)
due to internal string comparison.
\item
The argument \textit{main} should be fully expanded, it cannot be a macro.
\item
Subdirectories and special characters should be avoided in filenames.
\item
The command |\childdocmain{|\textit{main}|}| must be followed by a whitespace.
It should not be followed immediately by another command
or by a comment mark `|%|'.
This is because the \TeX{} parser reads the token immediately following
the argument of |\childdocmain| and puts it
at the beginning of every child section;
however, a white\-space is ignored.
\end{itemize}

%%%%%%%%%%%%%%%%%%%%%%%%%%%%%%%%%%%%%%%%
\paragraph{Content of Main File.}

It is advisable to place all content in the child files included by |\include|.
Any output contained in the main file will appear in all child documents
unless suppressed manually;
it cannot be suppressed automatically by the |\includeonly| directive
and thus should normally be avoided.
A method to include some content in the main file
by means of conditional processing is described in \secref{sec:conditional}.

%%%%%%%%%%%%%%%%%%%%%%%%%%%%%%%%%%%%%%%%
\paragraph{Page Numbering.}

When only a part of the document is compiled,
the appropriate numbering of pages
(as well as other status parameters)
is determined from the |.aux| files.
The latter contain information from previous passes.
However this information needs to propagate through
all intermediate child documents.
Therefore the page numbering in child documents may well
be inconsistent until the complete document is compiled at least once.

A useful (if unconventional) way to always ensure a consistent
page numbering is to restart the numbering in each child document
and denote the pages by `\textit{child}|.|\textit{page}'
where \textit{child} represents the chapter/section number of the child file.
This can be achieved by the command
|\numberwithin{page}{|\textit{child}|}|
of the \textsf{amsmath} package
where \textit{child} can be |chapter| or |section|
depending on the chosen structuring.
Alternatively, one can modify the macro |\thepage| appropriately
and reset the counter |page| at the start of each child file.

%%%%%%%%%%%%%%%%%%%%%%%%%%%%%%%%%%%%%%%%%%%%%%%%%%%%%%%%%%%%%%%%%%%%%%%%%%%%%%%%
\subsection{Conditional Processing}
\label{sec:conditional}

The package provides a mechanism to compile different versions
of a document. To customise the versions further some conditional processing
can come in handy to distinguish which version is being compiled.
The package provides two macros to describe the compilation context:

%%%%%%%%%%%%%%%%%%%%%%%%%%%%%%%%%%%%%%%%
\DescribeMacro{\ifchilddoc}
The conditional |\ifchilddoc| distinguishes between the compilation of
child documents and the main document:
%
\begin{center}
|\ifchilddoc |\textit{child-code}| |[|\||else |\textit{main-code}]| \||fi|
\end{center}

%%%%%%%%%%%%%%%%%%%%%%%%%%%%%%%%%%%%%%%%
\DescribeMacro{\childdocname}
\DescribeMacro{\childdocjob}
The macro |\childdocname| contains the filename (without extension)
of the main or child file being processed.
Note that |\childdocjob| will always contain the name of the main file.

%%%%%%%%%%%%%%%%%%%%%%%%%%%%%%%%%%%%%%%%
\paragraph{Title Page.}

Conditional processing can be used to include a title or banner page
in the main document when proper precautions are taken.
Importantly, the code in the main file should ensure that the page counter
(as well as other status parameters which are stored in the |.aux| files)
takes the same value after the conditional processing.
Otherwise the page numbers may take divergent values
depending on which part is compiled.

For example, a title page could be declared by:
%
\begin{center}
\begin{tabular}{l}
|\ifchilddoc\||else|\\
|\addtocounter{page}{-1}|\\
\textit{code for title page}\\
|\newpage|\\
|\||fi|
\end{tabular}
\end{center}
%
A banner page for the child documents can be generated by:
%
\begin{center}
\begin{tabular}{l}
|\ifchilddoc|\\
|\addtocounter{page}{-1}|\\
\textit{code for banner page}\\
|\newpage|\\
|\||fi|
\end{tabular}
\end{center}
%
Here one could write a message such as:
\begin{center}
|This is the part \childdocname{} of \childdocjob{}.|
\end{center}

%%%%%%%%%%%%%%%%%%%%%%%%%%%%%%%%%%%%%%%%%%%%%%%%%%%%%%%%%%%%%%%%%%%%%%%%%%%%%%%%
\subsection{Flags}
\label{sec:flags}

The package makes it easy to generate different versions
of the main or child documents.
To this end compilation flags can be defined
and assigned different default values.
They will be particularly useful in conjunction
with the forwarding mechanism described in \secref{sec:forward}.

For example, it may be useful to have a flag |\version|
which can be set to |draft| or |final|.
The document source will contain some conditional code
depending on the value of |\version|.
Suppose further, the flag should default to |final| for the main file
and to |draft| for child files
which is a natural assignment for editing the document.
This is achieved by placing the following code
in the preamble of the main document
(below the |\childdocmain| directive):
%
\begin{center}
\begin{tabular}{l}
|\ifchilddoc|\\
|\providecommand{\version}{draft}|\\
|\||else|\\
|\providecommand{\version}{final}|\\
|\||fi|
\end{tabular}
\end{center}
%
The definition by |\providecommand| makes sure
that previous definitions are not overwritten.
Further statements |\providecommand{\version}{...}|
can thus be added before the above code to override it.

For the main file, one might add a line
(between |\childdocmain| and the above block)
%
\begin{center}
|%\ifchilddoc\||else\providecommand{\version}{draft}\||fi|
\end{center}
%
which can be uncommented to produce a draft version.
Likewise one can add a line to the very top of a child file
(above the |\childdocof{|\textit{main}|}| directive)
%
\begin{center}
|%\providecommand{\version}{final}|
\end{center}
%
which can be uncommented to produce the final version of this child document.

%%%%%%%%%%%%%%%%%%%%%%%%%%%%%%%%%%%%%%%%%%%%%%%%%%%%%%%%%%%%%%%%%%%%%%%%%%%%%%%%
\subsection{Forwarding}
\label{sec:forward}

Different versions of the main or child documents
using compilation flags as described in \secref{sec:flags}
can be (permanently) stored in different files
for convenient compilation, viewing and distribution.
To this end, the package defines a command
to pass on compilation to a different file:

%%%%%%%%%%%%%%%%%%%%%%%%%%%%%%%%%%%%%%%%
\DescribeMacro{\childdocforward}
The command |\childdocforward| redirects processing to
another source file:
%
\begin{center}
\begin{tabular}{l}
|\input{childdoc.def}|\\
|\childdocforward[|\textit{main}|]{|\textit{dest}|}|\\
\end{tabular}
\end{center}
%
The argument \textit{dest} is the destination file
(without extension).
It should be the main file or one of the child files.
Note that further \textsf{childdoc} directives
such as |\childdocof| and |\childdocforward|
in the indicated file will be processed in this form.
The optional argument \textit{main}
passes on directly to the main file \textit{main}
while pretending to compile the child \textit{dest}.
This form behaves as if \textit{dest}
issues |\childdocof{|\textit{main}|}| right away,
and no further \textsf{childdoc} directives will be processed.

%%%%%%%%%%%%%%%%%%%%%%%%%%%%%%%%%%%%%%%%
\DescribeMacro{\...prefix}
In the alternative form |\childdocforwardprefix|,
%
\begin{center}
\begin{tabular}{l}
|\input{childdoc.def}|\\
|\childdocforwardprefix[|\textit{main}|]{|\textit{prefix}|}{|\textit{dest}|}|
\end{tabular}
\end{center}
%
the destination file is determined by a pattern
depending on the current file:
To make this work, the current file must be called
`{\textit{prefix}\hspace{0.2em}\textit{suffix}}'
with \textit{prefix} matching precisely the argument.
Processing is then passed on to the file
`{\textit{dest}\hspace{0.2em}\textit{suffix}}'.
Surely, the same effect is achieved by
directly specifying the
argument `{\textit{dest}\hspace{0.2em}\textit{suffix}}'
in the first form.
However, that requires to set up a different file
for each child. With the alternative form of the command
all these files can have exactly the same content
which simplifies setting them up and maintaining them.

For example, the following file |draft.tex|
with a compilation flag |\version| as described in \secref{sec:flags}
compiles the main document as a draft:
%
\begin{center}
\begin{tabular}{l}
|\def\version{draft}|\\
|\input{childdoc.def}|\\
|\childdocforward{|\textit{main}|}|
\end{tabular}
\end{center}
%
Likewise, the following files |final|\textit{nn}|.tex|
compile the final version of the child document
|child|\textit{nn}|.tex|:
%
\begin{center}
\begin{tabular}{l}
|\def\version{final}|\\
|\input{childdoc.def}|\\
|\childdocforwardprefix{final}{child}|
\end{tabular}
\end{center}
%

Note that when several versions of a main file and/or of each child file
are to be generated, it may be convenient to set up a |Makefile| or
shell script to automatise the process.

%%%%%%%%%%%%%%%%%%%%%%%%%%%%%%%%%%%%%%%%%%%%%%%%%%%%%%%%%%%%%%%%%%%%%%%%%%%%%%%%
\subsection{Command Line Processing}
\label{sec:commandline}

The effect of redirection files can also be achieved by invoking
the \LaTeX{} compiler with a more elaborate command line.
Most conveniently this should be done as part
of a shell script or a |Makefile|.

When using \textsf{childdoc} in the main file, the following
command lines effectively perform a redirection
(note that depending on the shell being used,
backslashes may have to be doubled: `|\|' $\to$ `|\\|'):
%
\begin{center}
|... -jobname "|\textit{target}|" |\\|"|[\textit{flags}]%
|\input{childdoc.def}\childdocforward[|\textit{main}|]{|\textit{dest}|}"|
\end{center}
%
Here \textit{target} is the name of the output file,
\textit{main} is the name of the main file
and \textit{dest} is the name of the main or child file to be processed
(all filenames without extensions).
The optional argument \textit{main} can be omitted
if \textit{main} matches \textit{dest}.
Optionally, compilation \textit{flags} can be defined via |\def| commands.
This command line makes the \TeX{} engine believe
it is compiling the file \textit{target}
whose content is specified as the latter parameter.
The provided code then forwards the processing to
\textit{main} or \textit{dest} as described in \secref{sec:forward}.

%%%%%%%%%%%%%%%%%%%%%%%%%%%%%%%%%%%%%%%%%%%%%%%%%%%%%%%%%%%%%%%%%%%%%%%%%%%%%%%%
\subsection{Include by Input}
\label{sec:input}

Including child documents by |\include| has some restrictions by design.
Most notably, the content of a child document always occupies
its own set of pages; pages cannot be shared between child documents.
Usually, this behaviour makes perfect sense
because each child document contain an essential part of the document.
However, in some situations it may be desirable to compose
a document from a collection of parts
without having mandatory page breaks between then.
For this case, the package
provides a mechanism to include parts
by |\input| which can also be processed individually.
However, by construction this mechanism
requires manual handling of the content to be output.

%%%%%%%%%%%%%%%%%%%%%%%%%%%%%%%%%%%%%%%%
\DescribeMacro{\ifchilddocmanual}
The main file should be prepared as usual, see \secref{sec:include}.
However, the document body must make a distinction
between processing of an individual part and of the main document, e.g.:
%
\begin{center}
\begin{tabular}{l}
|\ifchilddocmanual|\\
|\input{\childdocname}|\\
|\||else|\\
\textit{document body with }|\input{|\textit{part}|}|\\
|\||fi|
\end{tabular}
\end{center}
%
The conditional |\ifchilddocmanual| is true whenever
a part to be included by |\input| is being compiled,
and the name of the part is stored in |\childdocname|.

%%%%%%%%%%%%%%%%%%%%%%%%%%%%%%%%%%%%%%%%
\DescribeMacro{\childdocby}
Each part to be included by |\input| should start with:
%
\begin{center}
\begin{tabular}{l}
|\input{childdoc.def}|\\
|\childdocby{|\textit{main}|}|\\
\end{tabular}
\end{center}
%
The directive |\childdocby| is similar to |\childdocof|
described in \secref{sec:include},
but the subsequent selection of content must be done manually.
To that end, both |\ifchilddoc| and |\ifchilddocmanual|
will be true upon processing of a part,
and the name of the part is stored in |\childdocname|.
Note that |\jobname| will be set to the filename of the current part
so that each part receives an individual |.aux| file
that does not interfere with the |.aux| file(s) of the main document.
This behaviour can be altered by the alternative form
|\childdocby[*]{|\textit{main}|}| (with a non-empty optional argument)
which uses the |.aux| file of the main document
by setting |\jobname| to \textit{main}.

%%%%%%%%%%%%%%%%%%%%%%%%%%%%%%%%%%%%%%%%%%%%%%%%%%%%%%%%%%%%%%%%%%%%%%%%%%%%%%%%
\subsection{Driver Development}
\label{sec:driver}

The \textsf{childdoc} mechanism can also be use for the development
of definition files such as \LaTeX{} styles or classes.
This case differs from the above setup with multiple parts
included by |\include| in that no |\includeonly| should be invoked.
This can be achieved by starting the include file
(before |\ProvidesPackage|) with:
%
\begin{center}
\begin{tabular}{l}
|\input{childdoc.def}|\\
|\childdocforward{|\textit{main}|}|\\
\end{tabular}
\end{center}
%
or alternatively with:
%
\begin{center}
\begin{tabular}{l}
|\input{childdoc.def}|\\
|\childdocby{|\textit{main}|}|\\
\end{tabular}
\end{center}
%
Both forms have slightly different effects as described above.
The main file is prepared as usual, see \secref{sec:include}.

%%%%%%%%%%%%%%%%%%%%%%%%%%%%%%%%%%%%%%%%%%%%%%%%%%%%%%%%%%%%%%%%%%%%%%%%%%%%%%%%
\subsection{Legacy Detection}
\label{sec:detection}

The directive |\childdocmain| in the main file can detect
whether the complete document or merely a child is to be compiled
even without using the directive |\childdocof|.
This method is deprecated because it is less robust
and there is no compelling reason to use it;
it is merely provided for backward compatibility
and it may be removed in future versions.

If the detection mechanism is to be used,
it is mandatory to correctly specify
the filename of the main file as the argument of |\childdocmain|:
%
\begin{center}
\begin{tabular}{l}
|\input{childdoc.def}|\\
|\childdocmain{|\textit{main}|}|\\
\end{tabular}
\end{center}
%
If |\jobname| does not match the argument \textit{main} of |\childdocmain|,
it is assumed that |\jobname| points to the child file to be compiled.
When using |\childdocmain| with the main file specified as argument,
it suffices to start a child file
with just |\input{|\textit{main}|}|
without loading of the package and using |\childdocof|.
If instead all processing is done
with the appropriate \textsf{childdoc} directives,
the argument of \textit{main} of |\childdocmain| can be empty.

An alternative version of the command line processing described
in \secref{sec:commandline} using the detection mechanism reads:
%
\begin{center}
|... -jobname "|\textit{target}|" "|[\textit{flags}]%
[|\def\jobname{|\textit{dest}|}|]|\input{|\textit{main}|}"|
\end{center}

%%%%%%%%%%%%%%%%%%%%%%%%%%%%%%%%%%%%%%%%%%%%%%%%%%%%%%%%%%%%%%%%%%%%%%%%%%%%%%%%
\subsection{Manual Code}
\label{sec:manual}

In case one cannot be certain whether the definitions file |childdoc.def|
is installed on the target \TeX{} distribution
and one prefers not to ship it,
it is conceivable to paste a few relevant commands into the sources.

To that end, drop all statements |\input{childdoc.def}|
and perform the replacements as outlined below.
Instead of |\childdocmain{|\textit{main}|}| add the following code
to the top of the main file:
%
\begin{center}
\begin{tabular}{l}
|\||ifdefined\childdocname\endinput\||fi\newif\ifchilddoc|\\
|\edef\childdocname{\scantokens\expandafter{\jobname\noexpand}}|\\
|\def\childdocmain{|\textit{main}|}\||ifx\childdocmain\childdocname\||else|\\
|\childdoctrue\includeonly{\childdocname}\let\jobname\childdocmain\||fi|\\
\end{tabular}
\end{center}
%
Instead of |\childdocof{|\textit{main}|}| just include the main file
at the top of each child file:
%
\begin{center}
|\input{|\textit{main}|}|
\end{center}
%
A simple redirection |\childdocforward{|\textit{dest}|}| is achieved by:
%
\begin{center}
|\def\jobname{|\textit{dest}|}\input{\jobname}|
\end{center}
%
The redirection with prefix
|\childdocforwardprefix[|\textit{prefix}|]{|\textit{dest}|}|
is accomplished by:
%
\begin{center}
\begin{tabular}{l}
|{\edef\jobname{\scantokens\expandafter{\jobname\noexpand}}|\\
|\def\redirectjob |\textit{prefix}|#1~~~{\gdef\jobname{|\textit{dest}|#1}}|\\
|\expandafter\redirectjob\jobname~~~}\input{\jobname}|
\end{tabular}
\end{center}

In an alternative approach,
child documents can be compiled by a specific command line
without additional code or specific definitions:
%
\begin{center}
|... -jobname "|\textit{target}|" "|[\textit{flags}]%
|\includeonly{|\textit{dest}|}\input{|\textit{main}|}"|
\end{center}
%

%%%%%%%%%%%%%%%%%%%%%%%%%%%%%%%%%%%%%%%%%%%%%%%%%%%%%%%%%%%%%%%%%%%%%%%%%%%%%%%%
%%%%%%%%%%%%%%%%%%%%%%%%%%%%%%%%%%%%%%%%%%%%%%%%%%%%%%%%%%%%%%%%%%%%%%%%%%%%%%%%
\section{Information}

%%%%%%%%%%%%%%%%%%%%%%%%%%%%%%%%%%%%%%%%%%%%%%%%%%%%%%%%%%%%%%%%%%%%%%%%%%%%%%%%
\subsection{Copyright}

Copyright \copyright{} 2017--2018 Niklas Beisert

This work may be distributed and/or modified under the
conditions of the \LaTeX{} Project Public License, either version 1.3
of this license or (at your option) any later version.
The latest version of this license is in
  \url{http://www.latex-project.org/lppl.txt}
and version 1.3 or later is part of all distributions of \LaTeX{}
version 2005/12/01 or later.

This work has the LPPL maintenance status `maintained'.

The Current Maintainer of this work is Niklas Beisert.

This work consists of the files |README.txt|, |childdoc.ins| and |childdoc.dtx|
as well as the derived files |childdoc.def|, |cdocsamp.tex|
with |cdocsch1.tex|, |cdocsch2.tex|, |cdocspt3.tex|, |cdocspt4.tex|,
|cdocsdrf.tex|, |cdocsfn1.tex|, |cdocsfn2.tex|
as well as |childdoc.pdf|.

%%%%%%%%%%%%%%%%%%%%%%%%%%%%%%%%%%%%%%%%%%%%%%%%%%%%%%%%%%%%%%%%%%%%%%%%%%%%%%%%
\subsection{Files and Installation}

The package consists of the files:
%
\begin{center}
\begin{tabular}{ll}
    |README.txt|   & readme file \\
    |childdoc.ins| & installation file \\
    |childdoc.dtx| & source file \\
    |childdoc.def| & definition file \\
    |cdocsamp.tex| & sample main file \\
    |cdocsch1.tex| & sample include file \\
    |cdocsch2.tex| & sample include file \\
    |cdocspt3.tex| & sample part file \\
    |cdocspt4.tex| & sample part file \\
    |cdocsdrf.tex| & sample redirection file \\
    |cdocsfn1.tex| & sample redirection file \\
    |cdocsfn2.tex| & sample redirection file \\
    |childdoc.pdf| & manual
\end{tabular}
\end{center}
%
The distribution consists of the files
|README.txt|, |childdoc.ins| and |childdoc.dtx|.
%
\begin{itemize}
\item
Run (pdf)\LaTeX{} on |childdoc.dtx|
to compile the manual |childdoc.pdf| (this file).
\item
Run \LaTeX{} on |childdoc.ins| to create the definitions file |childdoc.def|
and the sample |cdocsamp.tex| with include files
|cdocsch1.tex|, |cdocsch2.tex|, |cdocspt3.tex|, |cdocspt4.tex|,
|cdocsdrf.tex|, |cdocsfn1.tex|, |cdocsfn2.tex|.
Then copy the file |childdoc.def| to an appropriate directory of your \LaTeX{}
distribution, e.g.\ \textit{texmf-root}|/tex/latex/childdoc|.
\end{itemize}

%%%%%%%%%%%%%%%%%%%%%%%%%%%%%%%%%%%%%%%%%%%%%%%%%%%%%%%%%%%%%%%%%%%%%%%%%%%%%%%%
\subsection{Related CTAN Packages}

There are several other packages which offer a similar functionality:
%
\begin{itemize}
\item
The packages
\href{http://ctan.org/pkg/docmute}{\textsf{docmute}},
\href{http://ctan.org/pkg/includex}{\textsf{includex}} and
\href{http://ctan.org/pkg/standalone}{\textsf{standalone}}
provide commands to include only the document body of
a child file thus allowing both files to be compiled individually.
\item
The packages \href{http://ctan.org/pkg/subdocs}{\textsf{subdocs}}
and \href{http://ctan.org/pkg/subfiles}{\textsf{subfiles}}
provide structures in which the main and child documents can be
encapsulated and allowing them to be compiled individually.
The inclusion mechanism is different from the conventional |\include|.
\item
The package \href{http://ctan.org/pkg/combine}{\textsf{combine}}
is an elaborate solution to combine several documents into one.
\end{itemize}
%
See also the CTAN topic \href{http://ctan.org/topic/subdocs}{\textsf{subdocs}}
for further related packages.
The present package differs from the above solutions in that
a document structure constructed with the conventional |\include| mechanism
just needs two extra commands at the top of every file
such that all constituent files can be compiled individually.

%%%%%%%%%%%%%%%%%%%%%%%%%%%%%%%%%%%%%%%%%%%%%%%%%%%%%%%%%%%%%%%%%%%%%%%%%%%%%%%%
%\subsection{Feature Suggestions}
%
%The following is a list of features which may be useful for future
%versions of this package:
%%
%\begin{itemize}
%\item
%\ldots
%\end{itemize}

%%%%%%%%%%%%%%%%%%%%%%%%%%%%%%%%%%%%%%%%%%%%%%%%%%%%%%%%%%%%%%%%%%%%%%%%%%%%%%%%
\subsection{Revision History}

%%%%%%%%%%%%%%%%%%%%%%%%%%%%%%%%%%%%%%%%
\paragraph{v2.0:} 2018/12/30

\begin{itemize}
\item
immediate forward processing
\item
added |\childdocby| mechanism
\item
manual restructured
\end{itemize}

%%%%%%%%%%%%%%%%%%%%%%%%%%%%%%%%%%%%%%%%
\paragraph{v1.6:} 2018/01/17

\begin{itemize}
\item
application for development of include files
\item
corrections to manual
\end{itemize}

%%%%%%%%%%%%%%%%%%%%%%%%%%%%%%%%%%%%%%%%
\paragraph{v1.5:} 2017/05/21

\begin{itemize}
\item
more complete structuring introduced
\item
|\childdocof| introduced
\item
|\childdoc| renamed to |\childdocmain|
\item
|\childredirect| renamed to |\childdocforward| and |\childdocforwardprefix|
and functionality expanded
\end{itemize}

%%%%%%%%%%%%%%%%%%%%%%%%%%%%%%%%%%%%%%%%
\paragraph{v1.0:} 2017/04/27

\begin{itemize}
\item
manual and install package
\item
first version published on CTAN
\end{itemize}

%%%%%%%%%%%%%%%%%%%%%%%%%%%%%%%%%%%%%%%%
\paragraph{v0.6:} 2017/04/26

\begin{itemize}
\item
redirection mechanism added
\end{itemize}

%%%%%%%%%%%%%%%%%%%%%%%%%%%%%%%%%%%%%%%%
\paragraph{v0.5:} 2017/04/26

\begin{itemize}
\item
functionality in definition file
\end{itemize}


%%%%%%%%%%%%%%%%%%%%%%%%%%%%%%%%%%%%%%%%%%%%%%%%%%%%%%%%%%%%%%%%%%%%%%%%%%%%%%%%
%%%%%%%%%%%%%%%%%%%%%%%%%%%%%%%%%%%%%%%%%%%%%%%%%%%%%%%%%%%%%%%%%%%%%%%%%%%%%%%%
%%%%%%%%%%%%%%%%%%%%%%%%%%%%%%%%%%%%%%%%%%%%%%%%%%%%%%%%%%%%%%%%%%%%%%%%%%%%%%%%
\appendix

\settowidth\MacroIndent{\rmfamily\scriptsize 000\ }

 \DocInput{childdoc.dtx}

\end{document}
%</driver>
% \fi
%
% %%%%%%%%%%%%%%%%%%%%%%%%%%%%%%%%%%%%%%%%%%%%%%%%%%%%%%%%%%%%%%%%%%%%%%%%%%%%%%
% %%%%%%%%%%%%%%%%%%%%%%%%%%%%%%%%%%%%%%%%%%%%%%%%%%%%%%%%%%%%%%%%%%%%%%%%%%%%%%
% \section{Sample}
%\iffalse
%<*samplemain>
%\fi
%
% The following presents a sample document
% with two chapters, two parts, a title page,
% a compile flag as well as three forwarding files to set the flag.
% It consists of eight |.tex| files:
% \begin{center}
% \begin{tabular}{ll}
% |cdocsamp.tex|&main file\\
% |cdocsch1.tex|&include file for chapter 1\\
% |cdocsch2.tex|&include file for chapter 2\\
% |cdocspt3.tex|&include file for part 3\\
% |cdocspt4.tex|&include file for part 4\\
% |cdocsdrf.tex|&forwarding file for main file in draft mode\\
% |cdocsfi1.tex|&forwarding file for final version of chapter 1\\
% |cdocsfi2.tex|&forwarding file for final version of chapter 2\\
% \end{tabular}
% \end{center}
% Each of the eight files can be compiled directly by the \LaTeX{} compiler.
%
% %%%%%%%%%%%%%%%%%%%%%%%%%%%%%%%%%%%%%%
% \paragraph{Main File.}
%
% The main file is called |cdocsamp.tex|.
%
% Load the \textsf{childdoc} definitions and
% declare the filename for the main document:
%    \begin{macrocode}
\input{childdoc.def}
\childdocmain{}
%    \end{macrocode}

% Optional override for |\version| flag:
%    \begin{macrocode}
%%\ifchilddoc\else\providecommand{\version}{draft}\fi
%    \end{macrocode}

% Define the default values for the |\version| flag
% (|final| for the main file and |draft| for childs):
%    \begin{macrocode}
\ifchilddoc
\providecommand{\version}{draft}
\else
\providecommand{\version}{final}
\fi
%    \end{macrocode}

% Load the standard document class:
%    \begin{macrocode}
\documentclass[12pt]{article}
%    \end{macrocode}

% Start the document body:
%    \begin{macrocode}
\begin{document}
%    \end{macrocode}

% Declare a title page.
% Print title, part of document being processed and version flag:
%    \begin{macrocode}
\addtocounter{page}{-1}
\begin{center}
{\LARGE\bfseries{}childdoc example\par}
\vspace{1cm}
\ifchilddoc
\ifchilddocmanual part\else chapter\fi:
`\childdocname' of `\childdocjob'\par
\else
main document: `\childdocjob'\par
\fi
version: \version\par
\end{center}
\newpage
%    \end{macrocode}

% Manually include selected file,
% otherwise process as usual:
%    \begin{macrocode}
\ifchilddocmanual
\section*{part `\childdocname'}
\input{\childdocname}
\else
%    \end{macrocode}

% Include the two chapters:
%    \begin{macrocode}
\include{cdocsch1}
\include{cdocsch2}
%    \end{macrocode}

% Include the two parts unless only chapters should be displayed:
%    \begin{macrocode}
\ifchilddoc\else
\section{part three}
\input{cdocspt3}
\section{part four}
\input{cdocspt4}
\fi
%    \end{macrocode}

% Process as usual until here:
%    \begin{macrocode}
\fi
%    \end{macrocode}

% End of document body:
%    \begin{macrocode}
\end{document}
%    \end{macrocode}
%\iffalse
%</samplemain>
%\fi
%
% %%%%%%%%%%%%%%%%%%%%%%%%%%%%%%%%%%%%%%
% \paragraph{Chapter Include Files.}
%
% The include files are called |cdocsch1.tex| and |cdocsch2.tex|.
%
%\iffalse
%<*samplechap1|samplechap2>
%\fi

% Optional override for |\version| flag:
%    \begin{macrocode}
%%\providecommand{\version}{final}
%    \end{macrocode}

% Include the main document:
%    \begin{macrocode}
\input{childdoc.def}
\childdocof{cdocsamp}
%    \end{macrocode}

%\iffalse
%</samplechap1|samplechap2>
%\fi
%
%\iffalse
%<*samplechap1>
%\fi
% Some text for chapter 1:
%    \begin{macrocode}
\section{one}
some text in chapter one
%    \end{macrocode}

%\iffalse
%</samplechap1>
%\fi
% Some text for chapter 2:
%\iffalse
%<*samplechap2>
%\fi
%    \begin{macrocode}
\section{two}
more text in chapter two
%    \end{macrocode}

%\iffalse
%</samplechap2>
%\fi
%
% %%%%%%%%%%%%%%%%%%%%%%%%%%%%%%%%%%%%%%
% \paragraph{Part Include Files.}
%
% The include files are called |cdocspt3.tex| and |cdocspt4.tex|.
%
%\iffalse
%<*samplepart3|samplepart4>
%\fi

% Optional override for |\version| flag:
%    \begin{macrocode}
%%\providecommand{\version}{final}
%    \end{macrocode}

% Include the main document:
%    \begin{macrocode}
\input{childdoc.def}
\childdocby{cdocsamp}
%    \end{macrocode}

%\iffalse
%</samplepart3|samplepart4>
%\fi
%
%\iffalse
%<*samplepart3>
%\fi
% Some text for part 3:
%    \begin{macrocode}
some text in part three
%    \end{macrocode}

%\iffalse
%</samplepart3>
%\fi
% Some text for part 4:
%\iffalse
%<*samplepart4>
%\fi
%    \begin{macrocode}
more text in part four
%    \end{macrocode}

%\iffalse
%</samplepart4>
%\fi
%
% %%%%%%%%%%%%%%%%%%%%%%%%%%%%%%%%%%%%%%
% \paragraph{Forwarding for a Complete Draft.}
%
% The following forwarding file |cdocsdrf.tex|
% compiles the main document in draft mode:
%\iffalse
%<*sampledraft>
%\fi
%    \begin{macrocode}
\def\version{draft}
\input{childdoc.def}
\childdocforward{cdocsamp}
%    \end{macrocode}

%\iffalse
%</sampledraft>
%\fi
%
% %%%%%%%%%%%%%%%%%%%%%%%%%%%%%%%%%%%%%%
% \paragraph{Forwarding for Final Version of the Chapters.}
%
% The following forwarding files |cdocsfn1.tex| and |cdocsfn2.tex|
% (with identical content)
% compile the final versions of the child documents
% |cdocsch1.tex| and |cdocsch2.tex|, respectively:
%\iffalse
%<*samplefinal>
%\fi
%    \begin{macrocode}
\def\version{final}
\input{childdoc.def}
\childdocforwardprefix[cdocsamp]{cdocsfn}{cdocsch}
%    \end{macrocode}

%\iffalse
%</samplefinal>
%\fi
%
% %%%%%%%%%%%%%%%%%%%%%%%%%%%%%%%%%%%%%%
% \paragraph{Command Line Processing.}
%
% The following three command lines generate the output files
% |cdocscld|, |cdocscl1| and |cdocscl2|
% which should be identical to
% |cdocsdrf|, |cdocsch1| and |cdocsfn2|, respectively:
% \begin{center}
% \begin{tabular}{l}
% |latex -jobname cdocscld \|\\
% |  "\def\version{draft}\input{childdoc.def}\childdocforward{cdocsamp}"|\\
% |latex -jobname cdocscl1 \|\\
% |  "\input{childdoc.def}\childdocforward[cdocsamp]{cdocsch1}"|\\
% |latex -jobname cdocscl2 \|\\
% |  "\def\version{final}\input{childdoc.def}\childdocforward{cdocsch2}"|
% \end{tabular}
% \end{center}
% Note that the trailing backslash on each first line
% merely continues the input to the second line
% (for convenient cut ant paste).
% Furthermore, the command |latex| can be replaced by any
% of its alternative versions such as |pdflatex|.
%
% %%%%%%%%%%%%%%%%%%%%%%%%%%%%%%%%%%%%%%%%%%%%%%%%%%%%%%%%%%%%%%%%%%%%%%%%%%%%%%
% %%%%%%%%%%%%%%%%%%%%%%%%%%%%%%%%%%%%%%%%%%%%%%%%%%%%%%%%%%%%%%%%%%%%%%%%%%%%%%
% \section{Implementation}
%\iffalse
%<*package>
%\fi
%
% This section describes the definitions file |childdoc.def|.

% The definitions cannot be loaded using |\usepackage| or |\RequirePackage|
% which has a mechanism to prevent loading a style file more than once.
% When loading the definitions by means of |\input|
% multiple instances have to be prevented manually:
%\iffalse
%This code needs to be before the `\ProvidesFile' directive
%which is defined at the beginning of this file.
%Therefore it is also placed there and commented out here.
%</package>
%<*discard>
%\fi
%    \begin{macrocode}
\ifdefined\childdocmain\endinput\fi
%    \end{macrocode}
%\iffalse
%</discard>
%<*package>
%\fi
%
% \macro{\ifchilddoc}
% \macro{\ifchilddocmanual}
% The conditional |\ifchilddoc| tells whether a
% child (true) or main (false) document is being compiled.
% The conditional |\ifchilddocmanual| tells whether
% the |\includeonly| mechanism is used (false) or
% the selection of child files must be performed manually (true).
% The definitions initialise to false:
%    \begin{macrocode}
\newif\ifchilddoc
\newif\ifchilddocmanual
%    \end{macrocode}

% \macro{\childdocname}
% \macro{\childdocjob}
% The macro |\childdocname| stores the name of the main document
% to be compiled. The macro |\childdocjob| stores the name of
% the document on which the \LaTeX{} compiler was originally invoked.
% The content of |\jobname| cannot be compared
% to filenames specified in the source due to different catcodes.
% The following code rescans |\jobname|, stores the result
% in |\childdocname| and saves a copy in |\childdocjob|:
%    \begin{macrocode}
\edef\childdocname{\scantokens\expandafter{\jobname\noexpand}}
\let\childdocjob\childdocname
%    \end{macrocode}

% \macro{\childdocdisable}
% The macro |\childdocdisable| prevents the main file
% from being processed more than once.
% At this stage, the main document command |\childdocmain|
% is assumed to be called once again where it should do nothing.
% Any subsequent call to it should prevent
% a secondary processing of the main document
% It overwrites the forwarding commands
% |\childdocof| and |\childdocforward|
% with empty macros to prevent further inclusions of the main document:
%    \begin{macrocode}
\newcommand{\childdocdisable}
{
  \renewcommand{\childdocmain}[1]{\renewcommand{\childdocmain}[1]{\endinput}}
  \renewcommand{\childdocof}[1]{}
  \renewcommand{\childdocby}[2][]{}
  \renewcommand{\childdocforward}[2][]{}
  \renewcommand{\childdocdisable}{}
}
%    \end{macrocode}

% \macro{\childdocmain}
% The macro |\childdocmain| is to be called at the top of the main file
% with nothing or the main filename (without extension) as argument.
% First, it breaks loops.
% If the argument is not empty and does not match |\childdocname|
% (which is set by the first inclusion of |childdoc.def|),
% |\ifchilddoc| is set to true, |\includeonly| is applied to the child file
% and |\jobname| is set to the main file
% (for proper handling of |.aux| files):
%    \begin{macrocode}
\newcommand{\childdocmain}[1]
{
  \childdocdisable\childdocmain{}
  \if?#1?\else
    \begingroup
      \def\childdoctmp{#1}
      \ifx\childdoctmp\childdocname
        \def\childdoctmp{}
      \else
        \def\childdoctmp
        {
          \childdoctrue
          \includeonly{\childdocname}
          \def\childdocjob{#1}
          \def\jobname{#1}
        }
      \fi
      \expandafter
    \endgroup
    \childdoctmp
  \fi
}
%    \end{macrocode}

% \macro{\childdocof}
% The command |\childdocof| redirects
% compilation to the main file |#1|.
%    \begin{macrocode}
\newcommand{\childdocof}[1]
{
  \childdocdisable
  \childdoctrue
  \includeonly{\childdocname}
  \def\jobname{#1}
  \def\childdocjob{#1}
  \input{#1}
}
%    \end{macrocode}

% \macro{\childdocby}
% The command |\childdocby| ....
%    \begin{macrocode}
\newcommand{\childdocby}[2][]
{
  \childdocdisable
  \childdoctrue
  \childdocmanualtrue
  \if?#1?\else
    \def\jobname{#2}
  \fi
  \def\childdocjob{#2}
  \input{#2}
  \endinput
}
%    \end{macrocode}

% \macro{\childdocforward}
% The command |\childdocforward| redirects
% compilation to the main file or
% (if the optional argument is given) a child file.
% Parameters are set as if the main file
% or a child file starting with |\childdocof| was compiled.
% Then compilation is handed over to the main file:
%    \begin{macrocode}
\newcommand{\childdocforward}[2][]
{
  \begingroup
    \if?#1?
      \def\childdoctmp
      {
        \def\childdocname{#2}
        \def\childdocjob{#2}
        \def\jobname{#2}
        \input{#2}
        \endinput
      }
    \else
      \def\childdoctmp
      {
        \childdocdisable
        \def\childdocname{#2}
        \childdoctrue
        \includeonly{#2}
        \def\childdocjob{#1}
        \def\jobname{#1}
        \input{#1}
        \endinput
      }
    \fi
    \expandafter
  \endgroup
  \childdoctmp
}
%    \end{macrocode}

% \macro{\childdocforwardprefix}
% The command |\childdocforwardprefix| redirects
% compilation to the main or a child file by means of a pattern.
% The prefix |#1| in the current filename is replaced by |#2|
% and the suffix of the current filename is kept
% (it is assumed that the filename does not contain the substring `|~~~|'
% which is used as a delimiter).
% Compilation is handed over to the new file by |\childdocforward|:
%    \begin{macrocode}
\newcommand{\childdocforwardprefix}[3][]
{
  \begingroup
    \def\childdocextract #2##1~~~{\def\childdoctmp{\childdocforward[#1]{#3##1}}}
    \expandafter\childdocextract\childdocname~~~
    \expandafter
  \endgroup
  \childdoctmp
}
%    \end{macrocode}

% \macro{\childdoc}
% The deprecated macro |\childdoc| is a legacy version of |\childdocmain|:
%    \begin{macrocode}
\newcommand{\childdoc}{\childdocmain}
%    \end{macrocode}

% \macro{\childdocredirect}
% The deprecated macro |\childdocredirect| is a legacy version
% of |\childdocforward| and |\childdocforwardprefix|:
%    \begin{macrocode}
\newcommand{\childdocredirect}[2][]
{
  \begingroup
    \if?#1?
      \def\childdoctmp{\childdocforward{#2}}
    \else
      \def\childdoctmp{\childdocforwardprefix{#1}{#2}}
    \fi
    \expandafter
  \endgroup
  \childdoctmp
}
%    \end{macrocode}

%\iffalse
%</package>
%\fi
%
\endinput
\childdocforward{cdocsamp}"|\\
% |latex -jobname cdocscl1 \|\\
% |  "% \iffalse
%
% childdoc.dtx Copyright (C) 2017-2018 Niklas Beisert
%
% This work may be distributed and/or modified under the
% conditions of the LaTeX Project Public License, either version 1.3
% of this license or (at your option) any later version.
% The latest version of this license is in
%   http://www.latex-project.org/lppl.txt
% and version 1.3 or later is part of all distributions of LaTeX
% version 2005/12/01 or later.
%
% This work has the LPPL maintenance status `maintained'.
%
% The Current Maintainer of this work is Niklas Beisert.
%
% This work consists of the files childdoc.dtx and childdoc.ins
% and the derived files childdoc.def and cdocsamp.tex with
% cdocsch1.tex, cdocsch2.tex, cdocsdrf.tex, cdocsfn1.tex, cdocsfn2.tex.
%
%<package>\ifdefined\childdocmain\endinput\fi
%<package>\ProvidesFile{childdoc.def}[2018/12/30 v2.0 child document driver]
%<samplemain>\ProvidesFile{cdocsamp.tex}[2018/12/30 v2.0 sample for childdoc]
%<*driver>
%\ProvidesFile{childdoc.drv}[2018/12/30 v2.0 childdoc reference manual file]
\PassOptionsToClass{10pt,a4paper}{article}
\documentclass{ltxdoc}

\usepackage[margin=35mm]{geometry}
\usepackage{hyperref}
\usepackage{hyperxmp}
\usepackage[usenames]{color}

\hypersetup{colorlinks=true}
\hypersetup{pdfstartview=FitH}
\hypersetup{pdfpagemode=UseNone}
\hypersetup{pdfsource={}}
\hypersetup{pdflang={en-UK}}
\hypersetup{pdfcopyright={Copyright 2017-2018 Niklas Beisert.
  This work may be distributed and/or modified under the
  conditions of the LaTeX Project Public License, either version 1.3
  of this license or (at your option) any later version.}}
\hypersetup{pdflicenseurl={http://www.latex-project.org/lppl.txt}}
\hypersetup{pdfcontactaddress={ETH Zurich, ITP, HIT K,
  Wolfgang-Pauli-Strasse 27}}
\hypersetup{pdfcontactpostcode={8093}}
\hypersetup{pdfcontactcity={Zurich}}
\hypersetup{pdfcontactcountry={Switzerland}}
\hypersetup{pdfcontactemail={nbeisert@itp.phys.ethz.ch}}
\hypersetup{pdfcontacturl={http://people.phys.ethz.ch/\xmptilde nbeisert/}}

\newcommand{\secref}[1]{\hyperref[#1]{section \ref*{#1}}}

\parskip1ex
\parindent0pt
\let\olditemize\itemize
\def\itemize{\olditemize\parskip0pt}

\begin{document}

\title{The \textsf{childdoc} Package}
\hypersetup{pdftitle={The childdoc Package}}
\author{Niklas Beisert\\[2ex]
  Institut f\"ur Theoretische Physik\\
  Eidgen\"ossische Technische Hochschule Z\"urich\\
  Wolfgang-Pauli-Strasse 27, 8093 Z\"urich, Switzerland\\[1ex]
  \href{mailto:nbeisert@itp.phys.ethz.ch}
  {\texttt{nbeisert@itp.phys.ethz.ch}}}
\hypersetup{pdfauthor={Niklas Beisert}}
\hypersetup{pdfsubject={Manual for the LaTeX2e Package childdoc}}
\date{30 December 2018, \textsf{v2.0}}
\maketitle

\begin{abstract}\noindent
\textsf{childdoc} is a \LaTeXe{} package
that enables the direct compilation
of document sections included by |\include|
to individual files.
\end{abstract}

\begingroup
\parskip0ex
\tableofcontents
\endgroup

%%%%%%%%%%%%%%%%%%%%%%%%%%%%%%%%%%%%%%%%%%%%%%%%%%%%%%%%%%%%%%%%%%%%%%%%%%%%%%%%
%%%%%%%%%%%%%%%%%%%%%%%%%%%%%%%%%%%%%%%%%%%%%%%%%%%%%%%%%%%%%%%%%%%%%%%%%%%%%%%%
\section{Introduction}

\LaTeX{} provides a mechanism to structure a large document (such as a book)
into a main file and several child files (containing the chapters)
using the |\include| command.
This mechanism is beneficial for documents
which span hundreds of pages in order to
make the source file(s) more manageable.
Moreover, compilation can be restricted to
selected child files by means of the |\includeonly| command.
The latter feature can be used to reduce the compilation time while editing
(this was significantly more useful in the earlier days of \LaTeX{})
or to generate a smaller document which is easier to navigate.
Another application of |\includeonly| is to generate
documents consisting of selected parts of the complete document.

However, there are a few drawbacks of the plain |\include| mechanism:
\begin{itemize}
\item
The child files cannot be compiled on their own,
they can only be compiled via the main file.
A naive editing environment
(such as a text editor with an option
to have the current file processed by \LaTeX)
may require one to switch to the main file before compiling;
attempting to compile the child file produces errors.
\item
The main file must be modified (each time)
to adjust the |\includeonly| command
to the present needs. This easily leaves the main file in a messy state.
\item
The generated document will always carry the filename
of the main document. This is inconvenient if
several child files are to be compiled and
to be kept for distribution.
\end{itemize}

The present package provides a simple interface
to make child files individually compilable by \LaTeX{}.
Compiling a child file then has the same effect as compiling
the main file with an |\includeonly| command
to select the appropriate child.
Moreover the generated document will carry the name of the child
rather than the main file.
This resolves all three above issues.

This feature is meant to make the editing of books,
thesis documents and lecture notes somewhat more convenient.
However, the package can also be used efficiently for
composing a series of documents (such as exercise sheets)
which are typically distributed individually.
It then assists the author in generating the individual documents
(potentially in different versions)
as well as a document containing the collected series.
Another application is in developing style files
or other kinds of included material
where compilation of the style file could redirect
to a sample or test file.

%%%%%%%%%%%%%%%%%%%%%%%%%%%%%%%%%%%%%%%%%%%%%%%%%%%%%%%%%%%%%%%%%%%%%%%%%%%%%%%%
%%%%%%%%%%%%%%%%%%%%%%%%%%%%%%%%%%%%%%%%%%%%%%%%%%%%%%%%%%%%%%%%%%%%%%%%%%%%%%%%
\section{Usage}

First of all, the package \textsf{childdoc} is \emph{not} a standard
\LaTeXe{} |.sty| style file! Therefore it needs to be invoked in
a non-standard way.

%%%%%%%%%%%%%%%%%%%%%%%%%%%%%%%%%%%%%%%%%%%%%%%%%%%%%%%%%%%%%%%%%%%%%%%%%%%%%%%%
\subsection{Included Files}
\label{sec:include}

%%%%%%%%%%%%%%%%%%%%%%%%%%%%%%%%%%%%%%%%
\DescribeMacro{\childdocmain}
To use the package, add the commands
\begin{center}
\begin{tabular}{l}
|\input{childdoc.def}|\\
|\childdocmain{}|\\
\end{tabular}
\end{center}
at the very top of the main \LaTeX{} file,
in particular \emph{before} the |\documentclass| statement!
The argument of |\childdocmain| should be left empty
(but it must be present).

%%%%%%%%%%%%%%%%%%%%%%%%%%%%%%%%%%%%%%%%
\DescribeMacro{\childdocof}
Furthermore, add the commands
\begin{center}
\begin{tabular}{l}
|\input{childdoc.def}|\\
|\childdocof{|\textit{main}|}|\\
\end{tabular}
\end{center}
at the top of every child file \textit{child}
which is included by |\include{|\textit{child}|}|
from within the main file
(or at least for those files to be compiled individually).
The argument \textit{main} must be the filename of the main file.

There are a couple of
considerations in setting up the main and child documents:

%%%%%%%%%%%%%%%%%%%%%%%%%%%%%%%%%%%%%%%%
\paragraph{Restrictions.}

Please note the following restrictions:
\begin{itemize}
\item
|\childdocmain| must be called with one argument \textit{main}
to ensure compatibility with earlier version of the package.
It must either be empty (|\childdocmain{}|)
or precisely match the filename of the main file in which it is specified.
See \secref{sec:detection} for further information.
\item
The filename \textit{main} must be specified without the |.tex| extension.
\item
The filename \textit{main} is case sensitive
(even in case-insensitive file systems)
due to internal string comparison.
\item
The argument \textit{main} should be fully expanded, it cannot be a macro.
\item
Subdirectories and special characters should be avoided in filenames.
\item
The command |\childdocmain{|\textit{main}|}| must be followed by a whitespace.
It should not be followed immediately by another command
or by a comment mark `|%|'.
This is because the \TeX{} parser reads the token immediately following
the argument of |\childdocmain| and puts it
at the beginning of every child section;
however, a white\-space is ignored.
\end{itemize}

%%%%%%%%%%%%%%%%%%%%%%%%%%%%%%%%%%%%%%%%
\paragraph{Content of Main File.}

It is advisable to place all content in the child files included by |\include|.
Any output contained in the main file will appear in all child documents
unless suppressed manually;
it cannot be suppressed automatically by the |\includeonly| directive
and thus should normally be avoided.
A method to include some content in the main file
by means of conditional processing is described in \secref{sec:conditional}.

%%%%%%%%%%%%%%%%%%%%%%%%%%%%%%%%%%%%%%%%
\paragraph{Page Numbering.}

When only a part of the document is compiled,
the appropriate numbering of pages
(as well as other status parameters)
is determined from the |.aux| files.
The latter contain information from previous passes.
However this information needs to propagate through
all intermediate child documents.
Therefore the page numbering in child documents may well
be inconsistent until the complete document is compiled at least once.

A useful (if unconventional) way to always ensure a consistent
page numbering is to restart the numbering in each child document
and denote the pages by `\textit{child}|.|\textit{page}'
where \textit{child} represents the chapter/section number of the child file.
This can be achieved by the command
|\numberwithin{page}{|\textit{child}|}|
of the \textsf{amsmath} package
where \textit{child} can be |chapter| or |section|
depending on the chosen structuring.
Alternatively, one can modify the macro |\thepage| appropriately
and reset the counter |page| at the start of each child file.

%%%%%%%%%%%%%%%%%%%%%%%%%%%%%%%%%%%%%%%%%%%%%%%%%%%%%%%%%%%%%%%%%%%%%%%%%%%%%%%%
\subsection{Conditional Processing}
\label{sec:conditional}

The package provides a mechanism to compile different versions
of a document. To customise the versions further some conditional processing
can come in handy to distinguish which version is being compiled.
The package provides two macros to describe the compilation context:

%%%%%%%%%%%%%%%%%%%%%%%%%%%%%%%%%%%%%%%%
\DescribeMacro{\ifchilddoc}
The conditional |\ifchilddoc| distinguishes between the compilation of
child documents and the main document:
%
\begin{center}
|\ifchilddoc |\textit{child-code}| |[|\||else |\textit{main-code}]| \||fi|
\end{center}

%%%%%%%%%%%%%%%%%%%%%%%%%%%%%%%%%%%%%%%%
\DescribeMacro{\childdocname}
\DescribeMacro{\childdocjob}
The macro |\childdocname| contains the filename (without extension)
of the main or child file being processed.
Note that |\childdocjob| will always contain the name of the main file.

%%%%%%%%%%%%%%%%%%%%%%%%%%%%%%%%%%%%%%%%
\paragraph{Title Page.}

Conditional processing can be used to include a title or banner page
in the main document when proper precautions are taken.
Importantly, the code in the main file should ensure that the page counter
(as well as other status parameters which are stored in the |.aux| files)
takes the same value after the conditional processing.
Otherwise the page numbers may take divergent values
depending on which part is compiled.

For example, a title page could be declared by:
%
\begin{center}
\begin{tabular}{l}
|\ifchilddoc\||else|\\
|\addtocounter{page}{-1}|\\
\textit{code for title page}\\
|\newpage|\\
|\||fi|
\end{tabular}
\end{center}
%
A banner page for the child documents can be generated by:
%
\begin{center}
\begin{tabular}{l}
|\ifchilddoc|\\
|\addtocounter{page}{-1}|\\
\textit{code for banner page}\\
|\newpage|\\
|\||fi|
\end{tabular}
\end{center}
%
Here one could write a message such as:
\begin{center}
|This is the part \childdocname{} of \childdocjob{}.|
\end{center}

%%%%%%%%%%%%%%%%%%%%%%%%%%%%%%%%%%%%%%%%%%%%%%%%%%%%%%%%%%%%%%%%%%%%%%%%%%%%%%%%
\subsection{Flags}
\label{sec:flags}

The package makes it easy to generate different versions
of the main or child documents.
To this end compilation flags can be defined
and assigned different default values.
They will be particularly useful in conjunction
with the forwarding mechanism described in \secref{sec:forward}.

For example, it may be useful to have a flag |\version|
which can be set to |draft| or |final|.
The document source will contain some conditional code
depending on the value of |\version|.
Suppose further, the flag should default to |final| for the main file
and to |draft| for child files
which is a natural assignment for editing the document.
This is achieved by placing the following code
in the preamble of the main document
(below the |\childdocmain| directive):
%
\begin{center}
\begin{tabular}{l}
|\ifchilddoc|\\
|\providecommand{\version}{draft}|\\
|\||else|\\
|\providecommand{\version}{final}|\\
|\||fi|
\end{tabular}
\end{center}
%
The definition by |\providecommand| makes sure
that previous definitions are not overwritten.
Further statements |\providecommand{\version}{...}|
can thus be added before the above code to override it.

For the main file, one might add a line
(between |\childdocmain| and the above block)
%
\begin{center}
|%\ifchilddoc\||else\providecommand{\version}{draft}\||fi|
\end{center}
%
which can be uncommented to produce a draft version.
Likewise one can add a line to the very top of a child file
(above the |\childdocof{|\textit{main}|}| directive)
%
\begin{center}
|%\providecommand{\version}{final}|
\end{center}
%
which can be uncommented to produce the final version of this child document.

%%%%%%%%%%%%%%%%%%%%%%%%%%%%%%%%%%%%%%%%%%%%%%%%%%%%%%%%%%%%%%%%%%%%%%%%%%%%%%%%
\subsection{Forwarding}
\label{sec:forward}

Different versions of the main or child documents
using compilation flags as described in \secref{sec:flags}
can be (permanently) stored in different files
for convenient compilation, viewing and distribution.
To this end, the package defines a command
to pass on compilation to a different file:

%%%%%%%%%%%%%%%%%%%%%%%%%%%%%%%%%%%%%%%%
\DescribeMacro{\childdocforward}
The command |\childdocforward| redirects processing to
another source file:
%
\begin{center}
\begin{tabular}{l}
|\input{childdoc.def}|\\
|\childdocforward[|\textit{main}|]{|\textit{dest}|}|\\
\end{tabular}
\end{center}
%
The argument \textit{dest} is the destination file
(without extension).
It should be the main file or one of the child files.
Note that further \textsf{childdoc} directives
such as |\childdocof| and |\childdocforward|
in the indicated file will be processed in this form.
The optional argument \textit{main}
passes on directly to the main file \textit{main}
while pretending to compile the child \textit{dest}.
This form behaves as if \textit{dest}
issues |\childdocof{|\textit{main}|}| right away,
and no further \textsf{childdoc} directives will be processed.

%%%%%%%%%%%%%%%%%%%%%%%%%%%%%%%%%%%%%%%%
\DescribeMacro{\...prefix}
In the alternative form |\childdocforwardprefix|,
%
\begin{center}
\begin{tabular}{l}
|\input{childdoc.def}|\\
|\childdocforwardprefix[|\textit{main}|]{|\textit{prefix}|}{|\textit{dest}|}|
\end{tabular}
\end{center}
%
the destination file is determined by a pattern
depending on the current file:
To make this work, the current file must be called
`{\textit{prefix}\hspace{0.2em}\textit{suffix}}'
with \textit{prefix} matching precisely the argument.
Processing is then passed on to the file
`{\textit{dest}\hspace{0.2em}\textit{suffix}}'.
Surely, the same effect is achieved by
directly specifying the
argument `{\textit{dest}\hspace{0.2em}\textit{suffix}}'
in the first form.
However, that requires to set up a different file
for each child. With the alternative form of the command
all these files can have exactly the same content
which simplifies setting them up and maintaining them.

For example, the following file |draft.tex|
with a compilation flag |\version| as described in \secref{sec:flags}
compiles the main document as a draft:
%
\begin{center}
\begin{tabular}{l}
|\def\version{draft}|\\
|\input{childdoc.def}|\\
|\childdocforward{|\textit{main}|}|
\end{tabular}
\end{center}
%
Likewise, the following files |final|\textit{nn}|.tex|
compile the final version of the child document
|child|\textit{nn}|.tex|:
%
\begin{center}
\begin{tabular}{l}
|\def\version{final}|\\
|\input{childdoc.def}|\\
|\childdocforwardprefix{final}{child}|
\end{tabular}
\end{center}
%

Note that when several versions of a main file and/or of each child file
are to be generated, it may be convenient to set up a |Makefile| or
shell script to automatise the process.

%%%%%%%%%%%%%%%%%%%%%%%%%%%%%%%%%%%%%%%%%%%%%%%%%%%%%%%%%%%%%%%%%%%%%%%%%%%%%%%%
\subsection{Command Line Processing}
\label{sec:commandline}

The effect of redirection files can also be achieved by invoking
the \LaTeX{} compiler with a more elaborate command line.
Most conveniently this should be done as part
of a shell script or a |Makefile|.

When using \textsf{childdoc} in the main file, the following
command lines effectively perform a redirection
(note that depending on the shell being used,
backslashes may have to be doubled: `|\|' $\to$ `|\\|'):
%
\begin{center}
|... -jobname "|\textit{target}|" |\\|"|[\textit{flags}]%
|\input{childdoc.def}\childdocforward[|\textit{main}|]{|\textit{dest}|}"|
\end{center}
%
Here \textit{target} is the name of the output file,
\textit{main} is the name of the main file
and \textit{dest} is the name of the main or child file to be processed
(all filenames without extensions).
The optional argument \textit{main} can be omitted
if \textit{main} matches \textit{dest}.
Optionally, compilation \textit{flags} can be defined via |\def| commands.
This command line makes the \TeX{} engine believe
it is compiling the file \textit{target}
whose content is specified as the latter parameter.
The provided code then forwards the processing to
\textit{main} or \textit{dest} as described in \secref{sec:forward}.

%%%%%%%%%%%%%%%%%%%%%%%%%%%%%%%%%%%%%%%%%%%%%%%%%%%%%%%%%%%%%%%%%%%%%%%%%%%%%%%%
\subsection{Include by Input}
\label{sec:input}

Including child documents by |\include| has some restrictions by design.
Most notably, the content of a child document always occupies
its own set of pages; pages cannot be shared between child documents.
Usually, this behaviour makes perfect sense
because each child document contain an essential part of the document.
However, in some situations it may be desirable to compose
a document from a collection of parts
without having mandatory page breaks between then.
For this case, the package
provides a mechanism to include parts
by |\input| which can also be processed individually.
However, by construction this mechanism
requires manual handling of the content to be output.

%%%%%%%%%%%%%%%%%%%%%%%%%%%%%%%%%%%%%%%%
\DescribeMacro{\ifchilddocmanual}
The main file should be prepared as usual, see \secref{sec:include}.
However, the document body must make a distinction
between processing of an individual part and of the main document, e.g.:
%
\begin{center}
\begin{tabular}{l}
|\ifchilddocmanual|\\
|\input{\childdocname}|\\
|\||else|\\
\textit{document body with }|\input{|\textit{part}|}|\\
|\||fi|
\end{tabular}
\end{center}
%
The conditional |\ifchilddocmanual| is true whenever
a part to be included by |\input| is being compiled,
and the name of the part is stored in |\childdocname|.

%%%%%%%%%%%%%%%%%%%%%%%%%%%%%%%%%%%%%%%%
\DescribeMacro{\childdocby}
Each part to be included by |\input| should start with:
%
\begin{center}
\begin{tabular}{l}
|\input{childdoc.def}|\\
|\childdocby{|\textit{main}|}|\\
\end{tabular}
\end{center}
%
The directive |\childdocby| is similar to |\childdocof|
described in \secref{sec:include},
but the subsequent selection of content must be done manually.
To that end, both |\ifchilddoc| and |\ifchilddocmanual|
will be true upon processing of a part,
and the name of the part is stored in |\childdocname|.
Note that |\jobname| will be set to the filename of the current part
so that each part receives an individual |.aux| file
that does not interfere with the |.aux| file(s) of the main document.
This behaviour can be altered by the alternative form
|\childdocby[*]{|\textit{main}|}| (with a non-empty optional argument)
which uses the |.aux| file of the main document
by setting |\jobname| to \textit{main}.

%%%%%%%%%%%%%%%%%%%%%%%%%%%%%%%%%%%%%%%%%%%%%%%%%%%%%%%%%%%%%%%%%%%%%%%%%%%%%%%%
\subsection{Driver Development}
\label{sec:driver}

The \textsf{childdoc} mechanism can also be use for the development
of definition files such as \LaTeX{} styles or classes.
This case differs from the above setup with multiple parts
included by |\include| in that no |\includeonly| should be invoked.
This can be achieved by starting the include file
(before |\ProvidesPackage|) with:
%
\begin{center}
\begin{tabular}{l}
|\input{childdoc.def}|\\
|\childdocforward{|\textit{main}|}|\\
\end{tabular}
\end{center}
%
or alternatively with:
%
\begin{center}
\begin{tabular}{l}
|\input{childdoc.def}|\\
|\childdocby{|\textit{main}|}|\\
\end{tabular}
\end{center}
%
Both forms have slightly different effects as described above.
The main file is prepared as usual, see \secref{sec:include}.

%%%%%%%%%%%%%%%%%%%%%%%%%%%%%%%%%%%%%%%%%%%%%%%%%%%%%%%%%%%%%%%%%%%%%%%%%%%%%%%%
\subsection{Legacy Detection}
\label{sec:detection}

The directive |\childdocmain| in the main file can detect
whether the complete document or merely a child is to be compiled
even without using the directive |\childdocof|.
This method is deprecated because it is less robust
and there is no compelling reason to use it;
it is merely provided for backward compatibility
and it may be removed in future versions.

If the detection mechanism is to be used,
it is mandatory to correctly specify
the filename of the main file as the argument of |\childdocmain|:
%
\begin{center}
\begin{tabular}{l}
|\input{childdoc.def}|\\
|\childdocmain{|\textit{main}|}|\\
\end{tabular}
\end{center}
%
If |\jobname| does not match the argument \textit{main} of |\childdocmain|,
it is assumed that |\jobname| points to the child file to be compiled.
When using |\childdocmain| with the main file specified as argument,
it suffices to start a child file
with just |\input{|\textit{main}|}|
without loading of the package and using |\childdocof|.
If instead all processing is done
with the appropriate \textsf{childdoc} directives,
the argument of \textit{main} of |\childdocmain| can be empty.

An alternative version of the command line processing described
in \secref{sec:commandline} using the detection mechanism reads:
%
\begin{center}
|... -jobname "|\textit{target}|" "|[\textit{flags}]%
[|\def\jobname{|\textit{dest}|}|]|\input{|\textit{main}|}"|
\end{center}

%%%%%%%%%%%%%%%%%%%%%%%%%%%%%%%%%%%%%%%%%%%%%%%%%%%%%%%%%%%%%%%%%%%%%%%%%%%%%%%%
\subsection{Manual Code}
\label{sec:manual}

In case one cannot be certain whether the definitions file |childdoc.def|
is installed on the target \TeX{} distribution
and one prefers not to ship it,
it is conceivable to paste a few relevant commands into the sources.

To that end, drop all statements |\input{childdoc.def}|
and perform the replacements as outlined below.
Instead of |\childdocmain{|\textit{main}|}| add the following code
to the top of the main file:
%
\begin{center}
\begin{tabular}{l}
|\||ifdefined\childdocname\endinput\||fi\newif\ifchilddoc|\\
|\edef\childdocname{\scantokens\expandafter{\jobname\noexpand}}|\\
|\def\childdocmain{|\textit{main}|}\||ifx\childdocmain\childdocname\||else|\\
|\childdoctrue\includeonly{\childdocname}\let\jobname\childdocmain\||fi|\\
\end{tabular}
\end{center}
%
Instead of |\childdocof{|\textit{main}|}| just include the main file
at the top of each child file:
%
\begin{center}
|\input{|\textit{main}|}|
\end{center}
%
A simple redirection |\childdocforward{|\textit{dest}|}| is achieved by:
%
\begin{center}
|\def\jobname{|\textit{dest}|}\input{\jobname}|
\end{center}
%
The redirection with prefix
|\childdocforwardprefix[|\textit{prefix}|]{|\textit{dest}|}|
is accomplished by:
%
\begin{center}
\begin{tabular}{l}
|{\edef\jobname{\scantokens\expandafter{\jobname\noexpand}}|\\
|\def\redirectjob |\textit{prefix}|#1~~~{\gdef\jobname{|\textit{dest}|#1}}|\\
|\expandafter\redirectjob\jobname~~~}\input{\jobname}|
\end{tabular}
\end{center}

In an alternative approach,
child documents can be compiled by a specific command line
without additional code or specific definitions:
%
\begin{center}
|... -jobname "|\textit{target}|" "|[\textit{flags}]%
|\includeonly{|\textit{dest}|}\input{|\textit{main}|}"|
\end{center}
%

%%%%%%%%%%%%%%%%%%%%%%%%%%%%%%%%%%%%%%%%%%%%%%%%%%%%%%%%%%%%%%%%%%%%%%%%%%%%%%%%
%%%%%%%%%%%%%%%%%%%%%%%%%%%%%%%%%%%%%%%%%%%%%%%%%%%%%%%%%%%%%%%%%%%%%%%%%%%%%%%%
\section{Information}

%%%%%%%%%%%%%%%%%%%%%%%%%%%%%%%%%%%%%%%%%%%%%%%%%%%%%%%%%%%%%%%%%%%%%%%%%%%%%%%%
\subsection{Copyright}

Copyright \copyright{} 2017--2018 Niklas Beisert

This work may be distributed and/or modified under the
conditions of the \LaTeX{} Project Public License, either version 1.3
of this license or (at your option) any later version.
The latest version of this license is in
  \url{http://www.latex-project.org/lppl.txt}
and version 1.3 or later is part of all distributions of \LaTeX{}
version 2005/12/01 or later.

This work has the LPPL maintenance status `maintained'.

The Current Maintainer of this work is Niklas Beisert.

This work consists of the files |README.txt|, |childdoc.ins| and |childdoc.dtx|
as well as the derived files |childdoc.def|, |cdocsamp.tex|
with |cdocsch1.tex|, |cdocsch2.tex|, |cdocspt3.tex|, |cdocspt4.tex|,
|cdocsdrf.tex|, |cdocsfn1.tex|, |cdocsfn2.tex|
as well as |childdoc.pdf|.

%%%%%%%%%%%%%%%%%%%%%%%%%%%%%%%%%%%%%%%%%%%%%%%%%%%%%%%%%%%%%%%%%%%%%%%%%%%%%%%%
\subsection{Files and Installation}

The package consists of the files:
%
\begin{center}
\begin{tabular}{ll}
    |README.txt|   & readme file \\
    |childdoc.ins| & installation file \\
    |childdoc.dtx| & source file \\
    |childdoc.def| & definition file \\
    |cdocsamp.tex| & sample main file \\
    |cdocsch1.tex| & sample include file \\
    |cdocsch2.tex| & sample include file \\
    |cdocspt3.tex| & sample part file \\
    |cdocspt4.tex| & sample part file \\
    |cdocsdrf.tex| & sample redirection file \\
    |cdocsfn1.tex| & sample redirection file \\
    |cdocsfn2.tex| & sample redirection file \\
    |childdoc.pdf| & manual
\end{tabular}
\end{center}
%
The distribution consists of the files
|README.txt|, |childdoc.ins| and |childdoc.dtx|.
%
\begin{itemize}
\item
Run (pdf)\LaTeX{} on |childdoc.dtx|
to compile the manual |childdoc.pdf| (this file).
\item
Run \LaTeX{} on |childdoc.ins| to create the definitions file |childdoc.def|
and the sample |cdocsamp.tex| with include files
|cdocsch1.tex|, |cdocsch2.tex|, |cdocspt3.tex|, |cdocspt4.tex|,
|cdocsdrf.tex|, |cdocsfn1.tex|, |cdocsfn2.tex|.
Then copy the file |childdoc.def| to an appropriate directory of your \LaTeX{}
distribution, e.g.\ \textit{texmf-root}|/tex/latex/childdoc|.
\end{itemize}

%%%%%%%%%%%%%%%%%%%%%%%%%%%%%%%%%%%%%%%%%%%%%%%%%%%%%%%%%%%%%%%%%%%%%%%%%%%%%%%%
\subsection{Related CTAN Packages}

There are several other packages which offer a similar functionality:
%
\begin{itemize}
\item
The packages
\href{http://ctan.org/pkg/docmute}{\textsf{docmute}},
\href{http://ctan.org/pkg/includex}{\textsf{includex}} and
\href{http://ctan.org/pkg/standalone}{\textsf{standalone}}
provide commands to include only the document body of
a child file thus allowing both files to be compiled individually.
\item
The packages \href{http://ctan.org/pkg/subdocs}{\textsf{subdocs}}
and \href{http://ctan.org/pkg/subfiles}{\textsf{subfiles}}
provide structures in which the main and child documents can be
encapsulated and allowing them to be compiled individually.
The inclusion mechanism is different from the conventional |\include|.
\item
The package \href{http://ctan.org/pkg/combine}{\textsf{combine}}
is an elaborate solution to combine several documents into one.
\end{itemize}
%
See also the CTAN topic \href{http://ctan.org/topic/subdocs}{\textsf{subdocs}}
for further related packages.
The present package differs from the above solutions in that
a document structure constructed with the conventional |\include| mechanism
just needs two extra commands at the top of every file
such that all constituent files can be compiled individually.

%%%%%%%%%%%%%%%%%%%%%%%%%%%%%%%%%%%%%%%%%%%%%%%%%%%%%%%%%%%%%%%%%%%%%%%%%%%%%%%%
%\subsection{Feature Suggestions}
%
%The following is a list of features which may be useful for future
%versions of this package:
%%
%\begin{itemize}
%\item
%\ldots
%\end{itemize}

%%%%%%%%%%%%%%%%%%%%%%%%%%%%%%%%%%%%%%%%%%%%%%%%%%%%%%%%%%%%%%%%%%%%%%%%%%%%%%%%
\subsection{Revision History}

%%%%%%%%%%%%%%%%%%%%%%%%%%%%%%%%%%%%%%%%
\paragraph{v2.0:} 2018/12/30

\begin{itemize}
\item
immediate forward processing
\item
added |\childdocby| mechanism
\item
manual restructured
\end{itemize}

%%%%%%%%%%%%%%%%%%%%%%%%%%%%%%%%%%%%%%%%
\paragraph{v1.6:} 2018/01/17

\begin{itemize}
\item
application for development of include files
\item
corrections to manual
\end{itemize}

%%%%%%%%%%%%%%%%%%%%%%%%%%%%%%%%%%%%%%%%
\paragraph{v1.5:} 2017/05/21

\begin{itemize}
\item
more complete structuring introduced
\item
|\childdocof| introduced
\item
|\childdoc| renamed to |\childdocmain|
\item
|\childredirect| renamed to |\childdocforward| and |\childdocforwardprefix|
and functionality expanded
\end{itemize}

%%%%%%%%%%%%%%%%%%%%%%%%%%%%%%%%%%%%%%%%
\paragraph{v1.0:} 2017/04/27

\begin{itemize}
\item
manual and install package
\item
first version published on CTAN
\end{itemize}

%%%%%%%%%%%%%%%%%%%%%%%%%%%%%%%%%%%%%%%%
\paragraph{v0.6:} 2017/04/26

\begin{itemize}
\item
redirection mechanism added
\end{itemize}

%%%%%%%%%%%%%%%%%%%%%%%%%%%%%%%%%%%%%%%%
\paragraph{v0.5:} 2017/04/26

\begin{itemize}
\item
functionality in definition file
\end{itemize}


%%%%%%%%%%%%%%%%%%%%%%%%%%%%%%%%%%%%%%%%%%%%%%%%%%%%%%%%%%%%%%%%%%%%%%%%%%%%%%%%
%%%%%%%%%%%%%%%%%%%%%%%%%%%%%%%%%%%%%%%%%%%%%%%%%%%%%%%%%%%%%%%%%%%%%%%%%%%%%%%%
%%%%%%%%%%%%%%%%%%%%%%%%%%%%%%%%%%%%%%%%%%%%%%%%%%%%%%%%%%%%%%%%%%%%%%%%%%%%%%%%
\appendix

\settowidth\MacroIndent{\rmfamily\scriptsize 000\ }

 \DocInput{childdoc.dtx}

\end{document}
%</driver>
% \fi
%
% %%%%%%%%%%%%%%%%%%%%%%%%%%%%%%%%%%%%%%%%%%%%%%%%%%%%%%%%%%%%%%%%%%%%%%%%%%%%%%
% %%%%%%%%%%%%%%%%%%%%%%%%%%%%%%%%%%%%%%%%%%%%%%%%%%%%%%%%%%%%%%%%%%%%%%%%%%%%%%
% \section{Sample}
%\iffalse
%<*samplemain>
%\fi
%
% The following presents a sample document
% with two chapters, two parts, a title page,
% a compile flag as well as three forwarding files to set the flag.
% It consists of eight |.tex| files:
% \begin{center}
% \begin{tabular}{ll}
% |cdocsamp.tex|&main file\\
% |cdocsch1.tex|&include file for chapter 1\\
% |cdocsch2.tex|&include file for chapter 2\\
% |cdocspt3.tex|&include file for part 3\\
% |cdocspt4.tex|&include file for part 4\\
% |cdocsdrf.tex|&forwarding file for main file in draft mode\\
% |cdocsfi1.tex|&forwarding file for final version of chapter 1\\
% |cdocsfi2.tex|&forwarding file for final version of chapter 2\\
% \end{tabular}
% \end{center}
% Each of the eight files can be compiled directly by the \LaTeX{} compiler.
%
% %%%%%%%%%%%%%%%%%%%%%%%%%%%%%%%%%%%%%%
% \paragraph{Main File.}
%
% The main file is called |cdocsamp.tex|.
%
% Load the \textsf{childdoc} definitions and
% declare the filename for the main document:
%    \begin{macrocode}
\input{childdoc.def}
\childdocmain{}
%    \end{macrocode}

% Optional override for |\version| flag:
%    \begin{macrocode}
%%\ifchilddoc\else\providecommand{\version}{draft}\fi
%    \end{macrocode}

% Define the default values for the |\version| flag
% (|final| for the main file and |draft| for childs):
%    \begin{macrocode}
\ifchilddoc
\providecommand{\version}{draft}
\else
\providecommand{\version}{final}
\fi
%    \end{macrocode}

% Load the standard document class:
%    \begin{macrocode}
\documentclass[12pt]{article}
%    \end{macrocode}

% Start the document body:
%    \begin{macrocode}
\begin{document}
%    \end{macrocode}

% Declare a title page.
% Print title, part of document being processed and version flag:
%    \begin{macrocode}
\addtocounter{page}{-1}
\begin{center}
{\LARGE\bfseries{}childdoc example\par}
\vspace{1cm}
\ifchilddoc
\ifchilddocmanual part\else chapter\fi:
`\childdocname' of `\childdocjob'\par
\else
main document: `\childdocjob'\par
\fi
version: \version\par
\end{center}
\newpage
%    \end{macrocode}

% Manually include selected file,
% otherwise process as usual:
%    \begin{macrocode}
\ifchilddocmanual
\section*{part `\childdocname'}
\input{\childdocname}
\else
%    \end{macrocode}

% Include the two chapters:
%    \begin{macrocode}
\include{cdocsch1}
\include{cdocsch2}
%    \end{macrocode}

% Include the two parts unless only chapters should be displayed:
%    \begin{macrocode}
\ifchilddoc\else
\section{part three}
\input{cdocspt3}
\section{part four}
\input{cdocspt4}
\fi
%    \end{macrocode}

% Process as usual until here:
%    \begin{macrocode}
\fi
%    \end{macrocode}

% End of document body:
%    \begin{macrocode}
\end{document}
%    \end{macrocode}
%\iffalse
%</samplemain>
%\fi
%
% %%%%%%%%%%%%%%%%%%%%%%%%%%%%%%%%%%%%%%
% \paragraph{Chapter Include Files.}
%
% The include files are called |cdocsch1.tex| and |cdocsch2.tex|.
%
%\iffalse
%<*samplechap1|samplechap2>
%\fi

% Optional override for |\version| flag:
%    \begin{macrocode}
%%\providecommand{\version}{final}
%    \end{macrocode}

% Include the main document:
%    \begin{macrocode}
\input{childdoc.def}
\childdocof{cdocsamp}
%    \end{macrocode}

%\iffalse
%</samplechap1|samplechap2>
%\fi
%
%\iffalse
%<*samplechap1>
%\fi
% Some text for chapter 1:
%    \begin{macrocode}
\section{one}
some text in chapter one
%    \end{macrocode}

%\iffalse
%</samplechap1>
%\fi
% Some text for chapter 2:
%\iffalse
%<*samplechap2>
%\fi
%    \begin{macrocode}
\section{two}
more text in chapter two
%    \end{macrocode}

%\iffalse
%</samplechap2>
%\fi
%
% %%%%%%%%%%%%%%%%%%%%%%%%%%%%%%%%%%%%%%
% \paragraph{Part Include Files.}
%
% The include files are called |cdocspt3.tex| and |cdocspt4.tex|.
%
%\iffalse
%<*samplepart3|samplepart4>
%\fi

% Optional override for |\version| flag:
%    \begin{macrocode}
%%\providecommand{\version}{final}
%    \end{macrocode}

% Include the main document:
%    \begin{macrocode}
\input{childdoc.def}
\childdocby{cdocsamp}
%    \end{macrocode}

%\iffalse
%</samplepart3|samplepart4>
%\fi
%
%\iffalse
%<*samplepart3>
%\fi
% Some text for part 3:
%    \begin{macrocode}
some text in part three
%    \end{macrocode}

%\iffalse
%</samplepart3>
%\fi
% Some text for part 4:
%\iffalse
%<*samplepart4>
%\fi
%    \begin{macrocode}
more text in part four
%    \end{macrocode}

%\iffalse
%</samplepart4>
%\fi
%
% %%%%%%%%%%%%%%%%%%%%%%%%%%%%%%%%%%%%%%
% \paragraph{Forwarding for a Complete Draft.}
%
% The following forwarding file |cdocsdrf.tex|
% compiles the main document in draft mode:
%\iffalse
%<*sampledraft>
%\fi
%    \begin{macrocode}
\def\version{draft}
\input{childdoc.def}
\childdocforward{cdocsamp}
%    \end{macrocode}

%\iffalse
%</sampledraft>
%\fi
%
% %%%%%%%%%%%%%%%%%%%%%%%%%%%%%%%%%%%%%%
% \paragraph{Forwarding for Final Version of the Chapters.}
%
% The following forwarding files |cdocsfn1.tex| and |cdocsfn2.tex|
% (with identical content)
% compile the final versions of the child documents
% |cdocsch1.tex| and |cdocsch2.tex|, respectively:
%\iffalse
%<*samplefinal>
%\fi
%    \begin{macrocode}
\def\version{final}
\input{childdoc.def}
\childdocforwardprefix[cdocsamp]{cdocsfn}{cdocsch}
%    \end{macrocode}

%\iffalse
%</samplefinal>
%\fi
%
% %%%%%%%%%%%%%%%%%%%%%%%%%%%%%%%%%%%%%%
% \paragraph{Command Line Processing.}
%
% The following three command lines generate the output files
% |cdocscld|, |cdocscl1| and |cdocscl2|
% which should be identical to
% |cdocsdrf|, |cdocsch1| and |cdocsfn2|, respectively:
% \begin{center}
% \begin{tabular}{l}
% |latex -jobname cdocscld \|\\
% |  "\def\version{draft}\input{childdoc.def}\childdocforward{cdocsamp}"|\\
% |latex -jobname cdocscl1 \|\\
% |  "\input{childdoc.def}\childdocforward[cdocsamp]{cdocsch1}"|\\
% |latex -jobname cdocscl2 \|\\
% |  "\def\version{final}\input{childdoc.def}\childdocforward{cdocsch2}"|
% \end{tabular}
% \end{center}
% Note that the trailing backslash on each first line
% merely continues the input to the second line
% (for convenient cut ant paste).
% Furthermore, the command |latex| can be replaced by any
% of its alternative versions such as |pdflatex|.
%
% %%%%%%%%%%%%%%%%%%%%%%%%%%%%%%%%%%%%%%%%%%%%%%%%%%%%%%%%%%%%%%%%%%%%%%%%%%%%%%
% %%%%%%%%%%%%%%%%%%%%%%%%%%%%%%%%%%%%%%%%%%%%%%%%%%%%%%%%%%%%%%%%%%%%%%%%%%%%%%
% \section{Implementation}
%\iffalse
%<*package>
%\fi
%
% This section describes the definitions file |childdoc.def|.

% The definitions cannot be loaded using |\usepackage| or |\RequirePackage|
% which has a mechanism to prevent loading a style file more than once.
% When loading the definitions by means of |\input|
% multiple instances have to be prevented manually:
%\iffalse
%This code needs to be before the `\ProvidesFile' directive
%which is defined at the beginning of this file.
%Therefore it is also placed there and commented out here.
%</package>
%<*discard>
%\fi
%    \begin{macrocode}
\ifdefined\childdocmain\endinput\fi
%    \end{macrocode}
%\iffalse
%</discard>
%<*package>
%\fi
%
% \macro{\ifchilddoc}
% \macro{\ifchilddocmanual}
% The conditional |\ifchilddoc| tells whether a
% child (true) or main (false) document is being compiled.
% The conditional |\ifchilddocmanual| tells whether
% the |\includeonly| mechanism is used (false) or
% the selection of child files must be performed manually (true).
% The definitions initialise to false:
%    \begin{macrocode}
\newif\ifchilddoc
\newif\ifchilddocmanual
%    \end{macrocode}

% \macro{\childdocname}
% \macro{\childdocjob}
% The macro |\childdocname| stores the name of the main document
% to be compiled. The macro |\childdocjob| stores the name of
% the document on which the \LaTeX{} compiler was originally invoked.
% The content of |\jobname| cannot be compared
% to filenames specified in the source due to different catcodes.
% The following code rescans |\jobname|, stores the result
% in |\childdocname| and saves a copy in |\childdocjob|:
%    \begin{macrocode}
\edef\childdocname{\scantokens\expandafter{\jobname\noexpand}}
\let\childdocjob\childdocname
%    \end{macrocode}

% \macro{\childdocdisable}
% The macro |\childdocdisable| prevents the main file
% from being processed more than once.
% At this stage, the main document command |\childdocmain|
% is assumed to be called once again where it should do nothing.
% Any subsequent call to it should prevent
% a secondary processing of the main document
% It overwrites the forwarding commands
% |\childdocof| and |\childdocforward|
% with empty macros to prevent further inclusions of the main document:
%    \begin{macrocode}
\newcommand{\childdocdisable}
{
  \renewcommand{\childdocmain}[1]{\renewcommand{\childdocmain}[1]{\endinput}}
  \renewcommand{\childdocof}[1]{}
  \renewcommand{\childdocby}[2][]{}
  \renewcommand{\childdocforward}[2][]{}
  \renewcommand{\childdocdisable}{}
}
%    \end{macrocode}

% \macro{\childdocmain}
% The macro |\childdocmain| is to be called at the top of the main file
% with nothing or the main filename (without extension) as argument.
% First, it breaks loops.
% If the argument is not empty and does not match |\childdocname|
% (which is set by the first inclusion of |childdoc.def|),
% |\ifchilddoc| is set to true, |\includeonly| is applied to the child file
% and |\jobname| is set to the main file
% (for proper handling of |.aux| files):
%    \begin{macrocode}
\newcommand{\childdocmain}[1]
{
  \childdocdisable\childdocmain{}
  \if?#1?\else
    \begingroup
      \def\childdoctmp{#1}
      \ifx\childdoctmp\childdocname
        \def\childdoctmp{}
      \else
        \def\childdoctmp
        {
          \childdoctrue
          \includeonly{\childdocname}
          \def\childdocjob{#1}
          \def\jobname{#1}
        }
      \fi
      \expandafter
    \endgroup
    \childdoctmp
  \fi
}
%    \end{macrocode}

% \macro{\childdocof}
% The command |\childdocof| redirects
% compilation to the main file |#1|.
%    \begin{macrocode}
\newcommand{\childdocof}[1]
{
  \childdocdisable
  \childdoctrue
  \includeonly{\childdocname}
  \def\jobname{#1}
  \def\childdocjob{#1}
  \input{#1}
}
%    \end{macrocode}

% \macro{\childdocby}
% The command |\childdocby| ....
%    \begin{macrocode}
\newcommand{\childdocby}[2][]
{
  \childdocdisable
  \childdoctrue
  \childdocmanualtrue
  \if?#1?\else
    \def\jobname{#2}
  \fi
  \def\childdocjob{#2}
  \input{#2}
  \endinput
}
%    \end{macrocode}

% \macro{\childdocforward}
% The command |\childdocforward| redirects
% compilation to the main file or
% (if the optional argument is given) a child file.
% Parameters are set as if the main file
% or a child file starting with |\childdocof| was compiled.
% Then compilation is handed over to the main file:
%    \begin{macrocode}
\newcommand{\childdocforward}[2][]
{
  \begingroup
    \if?#1?
      \def\childdoctmp
      {
        \def\childdocname{#2}
        \def\childdocjob{#2}
        \def\jobname{#2}
        \input{#2}
        \endinput
      }
    \else
      \def\childdoctmp
      {
        \childdocdisable
        \def\childdocname{#2}
        \childdoctrue
        \includeonly{#2}
        \def\childdocjob{#1}
        \def\jobname{#1}
        \input{#1}
        \endinput
      }
    \fi
    \expandafter
  \endgroup
  \childdoctmp
}
%    \end{macrocode}

% \macro{\childdocforwardprefix}
% The command |\childdocforwardprefix| redirects
% compilation to the main or a child file by means of a pattern.
% The prefix |#1| in the current filename is replaced by |#2|
% and the suffix of the current filename is kept
% (it is assumed that the filename does not contain the substring `|~~~|'
% which is used as a delimiter).
% Compilation is handed over to the new file by |\childdocforward|:
%    \begin{macrocode}
\newcommand{\childdocforwardprefix}[3][]
{
  \begingroup
    \def\childdocextract #2##1~~~{\def\childdoctmp{\childdocforward[#1]{#3##1}}}
    \expandafter\childdocextract\childdocname~~~
    \expandafter
  \endgroup
  \childdoctmp
}
%    \end{macrocode}

% \macro{\childdoc}
% The deprecated macro |\childdoc| is a legacy version of |\childdocmain|:
%    \begin{macrocode}
\newcommand{\childdoc}{\childdocmain}
%    \end{macrocode}

% \macro{\childdocredirect}
% The deprecated macro |\childdocredirect| is a legacy version
% of |\childdocforward| and |\childdocforwardprefix|:
%    \begin{macrocode}
\newcommand{\childdocredirect}[2][]
{
  \begingroup
    \if?#1?
      \def\childdoctmp{\childdocforward{#2}}
    \else
      \def\childdoctmp{\childdocforwardprefix{#1}{#2}}
    \fi
    \expandafter
  \endgroup
  \childdoctmp
}
%    \end{macrocode}

%\iffalse
%</package>
%\fi
%
\endinput
\childdocforward[cdocsamp]{cdocsch1}"|\\
% |latex -jobname cdocscl2 \|\\
% |  "\def\version{final}% \iffalse
%
% childdoc.dtx Copyright (C) 2017-2018 Niklas Beisert
%
% This work may be distributed and/or modified under the
% conditions of the LaTeX Project Public License, either version 1.3
% of this license or (at your option) any later version.
% The latest version of this license is in
%   http://www.latex-project.org/lppl.txt
% and version 1.3 or later is part of all distributions of LaTeX
% version 2005/12/01 or later.
%
% This work has the LPPL maintenance status `maintained'.
%
% The Current Maintainer of this work is Niklas Beisert.
%
% This work consists of the files childdoc.dtx and childdoc.ins
% and the derived files childdoc.def and cdocsamp.tex with
% cdocsch1.tex, cdocsch2.tex, cdocsdrf.tex, cdocsfn1.tex, cdocsfn2.tex.
%
%<package>\ifdefined\childdocmain\endinput\fi
%<package>\ProvidesFile{childdoc.def}[2018/12/30 v2.0 child document driver]
%<samplemain>\ProvidesFile{cdocsamp.tex}[2018/12/30 v2.0 sample for childdoc]
%<*driver>
%\ProvidesFile{childdoc.drv}[2018/12/30 v2.0 childdoc reference manual file]
\PassOptionsToClass{10pt,a4paper}{article}
\documentclass{ltxdoc}

\usepackage[margin=35mm]{geometry}
\usepackage{hyperref}
\usepackage{hyperxmp}
\usepackage[usenames]{color}

\hypersetup{colorlinks=true}
\hypersetup{pdfstartview=FitH}
\hypersetup{pdfpagemode=UseNone}
\hypersetup{pdfsource={}}
\hypersetup{pdflang={en-UK}}
\hypersetup{pdfcopyright={Copyright 2017-2018 Niklas Beisert.
  This work may be distributed and/or modified under the
  conditions of the LaTeX Project Public License, either version 1.3
  of this license or (at your option) any later version.}}
\hypersetup{pdflicenseurl={http://www.latex-project.org/lppl.txt}}
\hypersetup{pdfcontactaddress={ETH Zurich, ITP, HIT K,
  Wolfgang-Pauli-Strasse 27}}
\hypersetup{pdfcontactpostcode={8093}}
\hypersetup{pdfcontactcity={Zurich}}
\hypersetup{pdfcontactcountry={Switzerland}}
\hypersetup{pdfcontactemail={nbeisert@itp.phys.ethz.ch}}
\hypersetup{pdfcontacturl={http://people.phys.ethz.ch/\xmptilde nbeisert/}}

\newcommand{\secref}[1]{\hyperref[#1]{section \ref*{#1}}}

\parskip1ex
\parindent0pt
\let\olditemize\itemize
\def\itemize{\olditemize\parskip0pt}

\begin{document}

\title{The \textsf{childdoc} Package}
\hypersetup{pdftitle={The childdoc Package}}
\author{Niklas Beisert\\[2ex]
  Institut f\"ur Theoretische Physik\\
  Eidgen\"ossische Technische Hochschule Z\"urich\\
  Wolfgang-Pauli-Strasse 27, 8093 Z\"urich, Switzerland\\[1ex]
  \href{mailto:nbeisert@itp.phys.ethz.ch}
  {\texttt{nbeisert@itp.phys.ethz.ch}}}
\hypersetup{pdfauthor={Niklas Beisert}}
\hypersetup{pdfsubject={Manual for the LaTeX2e Package childdoc}}
\date{30 December 2018, \textsf{v2.0}}
\maketitle

\begin{abstract}\noindent
\textsf{childdoc} is a \LaTeXe{} package
that enables the direct compilation
of document sections included by |\include|
to individual files.
\end{abstract}

\begingroup
\parskip0ex
\tableofcontents
\endgroup

%%%%%%%%%%%%%%%%%%%%%%%%%%%%%%%%%%%%%%%%%%%%%%%%%%%%%%%%%%%%%%%%%%%%%%%%%%%%%%%%
%%%%%%%%%%%%%%%%%%%%%%%%%%%%%%%%%%%%%%%%%%%%%%%%%%%%%%%%%%%%%%%%%%%%%%%%%%%%%%%%
\section{Introduction}

\LaTeX{} provides a mechanism to structure a large document (such as a book)
into a main file and several child files (containing the chapters)
using the |\include| command.
This mechanism is beneficial for documents
which span hundreds of pages in order to
make the source file(s) more manageable.
Moreover, compilation can be restricted to
selected child files by means of the |\includeonly| command.
The latter feature can be used to reduce the compilation time while editing
(this was significantly more useful in the earlier days of \LaTeX{})
or to generate a smaller document which is easier to navigate.
Another application of |\includeonly| is to generate
documents consisting of selected parts of the complete document.

However, there are a few drawbacks of the plain |\include| mechanism:
\begin{itemize}
\item
The child files cannot be compiled on their own,
they can only be compiled via the main file.
A naive editing environment
(such as a text editor with an option
to have the current file processed by \LaTeX)
may require one to switch to the main file before compiling;
attempting to compile the child file produces errors.
\item
The main file must be modified (each time)
to adjust the |\includeonly| command
to the present needs. This easily leaves the main file in a messy state.
\item
The generated document will always carry the filename
of the main document. This is inconvenient if
several child files are to be compiled and
to be kept for distribution.
\end{itemize}

The present package provides a simple interface
to make child files individually compilable by \LaTeX{}.
Compiling a child file then has the same effect as compiling
the main file with an |\includeonly| command
to select the appropriate child.
Moreover the generated document will carry the name of the child
rather than the main file.
This resolves all three above issues.

This feature is meant to make the editing of books,
thesis documents and lecture notes somewhat more convenient.
However, the package can also be used efficiently for
composing a series of documents (such as exercise sheets)
which are typically distributed individually.
It then assists the author in generating the individual documents
(potentially in different versions)
as well as a document containing the collected series.
Another application is in developing style files
or other kinds of included material
where compilation of the style file could redirect
to a sample or test file.

%%%%%%%%%%%%%%%%%%%%%%%%%%%%%%%%%%%%%%%%%%%%%%%%%%%%%%%%%%%%%%%%%%%%%%%%%%%%%%%%
%%%%%%%%%%%%%%%%%%%%%%%%%%%%%%%%%%%%%%%%%%%%%%%%%%%%%%%%%%%%%%%%%%%%%%%%%%%%%%%%
\section{Usage}

First of all, the package \textsf{childdoc} is \emph{not} a standard
\LaTeXe{} |.sty| style file! Therefore it needs to be invoked in
a non-standard way.

%%%%%%%%%%%%%%%%%%%%%%%%%%%%%%%%%%%%%%%%%%%%%%%%%%%%%%%%%%%%%%%%%%%%%%%%%%%%%%%%
\subsection{Included Files}
\label{sec:include}

%%%%%%%%%%%%%%%%%%%%%%%%%%%%%%%%%%%%%%%%
\DescribeMacro{\childdocmain}
To use the package, add the commands
\begin{center}
\begin{tabular}{l}
|\input{childdoc.def}|\\
|\childdocmain{}|\\
\end{tabular}
\end{center}
at the very top of the main \LaTeX{} file,
in particular \emph{before} the |\documentclass| statement!
The argument of |\childdocmain| should be left empty
(but it must be present).

%%%%%%%%%%%%%%%%%%%%%%%%%%%%%%%%%%%%%%%%
\DescribeMacro{\childdocof}
Furthermore, add the commands
\begin{center}
\begin{tabular}{l}
|\input{childdoc.def}|\\
|\childdocof{|\textit{main}|}|\\
\end{tabular}
\end{center}
at the top of every child file \textit{child}
which is included by |\include{|\textit{child}|}|
from within the main file
(or at least for those files to be compiled individually).
The argument \textit{main} must be the filename of the main file.

There are a couple of
considerations in setting up the main and child documents:

%%%%%%%%%%%%%%%%%%%%%%%%%%%%%%%%%%%%%%%%
\paragraph{Restrictions.}

Please note the following restrictions:
\begin{itemize}
\item
|\childdocmain| must be called with one argument \textit{main}
to ensure compatibility with earlier version of the package.
It must either be empty (|\childdocmain{}|)
or precisely match the filename of the main file in which it is specified.
See \secref{sec:detection} for further information.
\item
The filename \textit{main} must be specified without the |.tex| extension.
\item
The filename \textit{main} is case sensitive
(even in case-insensitive file systems)
due to internal string comparison.
\item
The argument \textit{main} should be fully expanded, it cannot be a macro.
\item
Subdirectories and special characters should be avoided in filenames.
\item
The command |\childdocmain{|\textit{main}|}| must be followed by a whitespace.
It should not be followed immediately by another command
or by a comment mark `|%|'.
This is because the \TeX{} parser reads the token immediately following
the argument of |\childdocmain| and puts it
at the beginning of every child section;
however, a white\-space is ignored.
\end{itemize}

%%%%%%%%%%%%%%%%%%%%%%%%%%%%%%%%%%%%%%%%
\paragraph{Content of Main File.}

It is advisable to place all content in the child files included by |\include|.
Any output contained in the main file will appear in all child documents
unless suppressed manually;
it cannot be suppressed automatically by the |\includeonly| directive
and thus should normally be avoided.
A method to include some content in the main file
by means of conditional processing is described in \secref{sec:conditional}.

%%%%%%%%%%%%%%%%%%%%%%%%%%%%%%%%%%%%%%%%
\paragraph{Page Numbering.}

When only a part of the document is compiled,
the appropriate numbering of pages
(as well as other status parameters)
is determined from the |.aux| files.
The latter contain information from previous passes.
However this information needs to propagate through
all intermediate child documents.
Therefore the page numbering in child documents may well
be inconsistent until the complete document is compiled at least once.

A useful (if unconventional) way to always ensure a consistent
page numbering is to restart the numbering in each child document
and denote the pages by `\textit{child}|.|\textit{page}'
where \textit{child} represents the chapter/section number of the child file.
This can be achieved by the command
|\numberwithin{page}{|\textit{child}|}|
of the \textsf{amsmath} package
where \textit{child} can be |chapter| or |section|
depending on the chosen structuring.
Alternatively, one can modify the macro |\thepage| appropriately
and reset the counter |page| at the start of each child file.

%%%%%%%%%%%%%%%%%%%%%%%%%%%%%%%%%%%%%%%%%%%%%%%%%%%%%%%%%%%%%%%%%%%%%%%%%%%%%%%%
\subsection{Conditional Processing}
\label{sec:conditional}

The package provides a mechanism to compile different versions
of a document. To customise the versions further some conditional processing
can come in handy to distinguish which version is being compiled.
The package provides two macros to describe the compilation context:

%%%%%%%%%%%%%%%%%%%%%%%%%%%%%%%%%%%%%%%%
\DescribeMacro{\ifchilddoc}
The conditional |\ifchilddoc| distinguishes between the compilation of
child documents and the main document:
%
\begin{center}
|\ifchilddoc |\textit{child-code}| |[|\||else |\textit{main-code}]| \||fi|
\end{center}

%%%%%%%%%%%%%%%%%%%%%%%%%%%%%%%%%%%%%%%%
\DescribeMacro{\childdocname}
\DescribeMacro{\childdocjob}
The macro |\childdocname| contains the filename (without extension)
of the main or child file being processed.
Note that |\childdocjob| will always contain the name of the main file.

%%%%%%%%%%%%%%%%%%%%%%%%%%%%%%%%%%%%%%%%
\paragraph{Title Page.}

Conditional processing can be used to include a title or banner page
in the main document when proper precautions are taken.
Importantly, the code in the main file should ensure that the page counter
(as well as other status parameters which are stored in the |.aux| files)
takes the same value after the conditional processing.
Otherwise the page numbers may take divergent values
depending on which part is compiled.

For example, a title page could be declared by:
%
\begin{center}
\begin{tabular}{l}
|\ifchilddoc\||else|\\
|\addtocounter{page}{-1}|\\
\textit{code for title page}\\
|\newpage|\\
|\||fi|
\end{tabular}
\end{center}
%
A banner page for the child documents can be generated by:
%
\begin{center}
\begin{tabular}{l}
|\ifchilddoc|\\
|\addtocounter{page}{-1}|\\
\textit{code for banner page}\\
|\newpage|\\
|\||fi|
\end{tabular}
\end{center}
%
Here one could write a message such as:
\begin{center}
|This is the part \childdocname{} of \childdocjob{}.|
\end{center}

%%%%%%%%%%%%%%%%%%%%%%%%%%%%%%%%%%%%%%%%%%%%%%%%%%%%%%%%%%%%%%%%%%%%%%%%%%%%%%%%
\subsection{Flags}
\label{sec:flags}

The package makes it easy to generate different versions
of the main or child documents.
To this end compilation flags can be defined
and assigned different default values.
They will be particularly useful in conjunction
with the forwarding mechanism described in \secref{sec:forward}.

For example, it may be useful to have a flag |\version|
which can be set to |draft| or |final|.
The document source will contain some conditional code
depending on the value of |\version|.
Suppose further, the flag should default to |final| for the main file
and to |draft| for child files
which is a natural assignment for editing the document.
This is achieved by placing the following code
in the preamble of the main document
(below the |\childdocmain| directive):
%
\begin{center}
\begin{tabular}{l}
|\ifchilddoc|\\
|\providecommand{\version}{draft}|\\
|\||else|\\
|\providecommand{\version}{final}|\\
|\||fi|
\end{tabular}
\end{center}
%
The definition by |\providecommand| makes sure
that previous definitions are not overwritten.
Further statements |\providecommand{\version}{...}|
can thus be added before the above code to override it.

For the main file, one might add a line
(between |\childdocmain| and the above block)
%
\begin{center}
|%\ifchilddoc\||else\providecommand{\version}{draft}\||fi|
\end{center}
%
which can be uncommented to produce a draft version.
Likewise one can add a line to the very top of a child file
(above the |\childdocof{|\textit{main}|}| directive)
%
\begin{center}
|%\providecommand{\version}{final}|
\end{center}
%
which can be uncommented to produce the final version of this child document.

%%%%%%%%%%%%%%%%%%%%%%%%%%%%%%%%%%%%%%%%%%%%%%%%%%%%%%%%%%%%%%%%%%%%%%%%%%%%%%%%
\subsection{Forwarding}
\label{sec:forward}

Different versions of the main or child documents
using compilation flags as described in \secref{sec:flags}
can be (permanently) stored in different files
for convenient compilation, viewing and distribution.
To this end, the package defines a command
to pass on compilation to a different file:

%%%%%%%%%%%%%%%%%%%%%%%%%%%%%%%%%%%%%%%%
\DescribeMacro{\childdocforward}
The command |\childdocforward| redirects processing to
another source file:
%
\begin{center}
\begin{tabular}{l}
|\input{childdoc.def}|\\
|\childdocforward[|\textit{main}|]{|\textit{dest}|}|\\
\end{tabular}
\end{center}
%
The argument \textit{dest} is the destination file
(without extension).
It should be the main file or one of the child files.
Note that further \textsf{childdoc} directives
such as |\childdocof| and |\childdocforward|
in the indicated file will be processed in this form.
The optional argument \textit{main}
passes on directly to the main file \textit{main}
while pretending to compile the child \textit{dest}.
This form behaves as if \textit{dest}
issues |\childdocof{|\textit{main}|}| right away,
and no further \textsf{childdoc} directives will be processed.

%%%%%%%%%%%%%%%%%%%%%%%%%%%%%%%%%%%%%%%%
\DescribeMacro{\...prefix}
In the alternative form |\childdocforwardprefix|,
%
\begin{center}
\begin{tabular}{l}
|\input{childdoc.def}|\\
|\childdocforwardprefix[|\textit{main}|]{|\textit{prefix}|}{|\textit{dest}|}|
\end{tabular}
\end{center}
%
the destination file is determined by a pattern
depending on the current file:
To make this work, the current file must be called
`{\textit{prefix}\hspace{0.2em}\textit{suffix}}'
with \textit{prefix} matching precisely the argument.
Processing is then passed on to the file
`{\textit{dest}\hspace{0.2em}\textit{suffix}}'.
Surely, the same effect is achieved by
directly specifying the
argument `{\textit{dest}\hspace{0.2em}\textit{suffix}}'
in the first form.
However, that requires to set up a different file
for each child. With the alternative form of the command
all these files can have exactly the same content
which simplifies setting them up and maintaining them.

For example, the following file |draft.tex|
with a compilation flag |\version| as described in \secref{sec:flags}
compiles the main document as a draft:
%
\begin{center}
\begin{tabular}{l}
|\def\version{draft}|\\
|\input{childdoc.def}|\\
|\childdocforward{|\textit{main}|}|
\end{tabular}
\end{center}
%
Likewise, the following files |final|\textit{nn}|.tex|
compile the final version of the child document
|child|\textit{nn}|.tex|:
%
\begin{center}
\begin{tabular}{l}
|\def\version{final}|\\
|\input{childdoc.def}|\\
|\childdocforwardprefix{final}{child}|
\end{tabular}
\end{center}
%

Note that when several versions of a main file and/or of each child file
are to be generated, it may be convenient to set up a |Makefile| or
shell script to automatise the process.

%%%%%%%%%%%%%%%%%%%%%%%%%%%%%%%%%%%%%%%%%%%%%%%%%%%%%%%%%%%%%%%%%%%%%%%%%%%%%%%%
\subsection{Command Line Processing}
\label{sec:commandline}

The effect of redirection files can also be achieved by invoking
the \LaTeX{} compiler with a more elaborate command line.
Most conveniently this should be done as part
of a shell script or a |Makefile|.

When using \textsf{childdoc} in the main file, the following
command lines effectively perform a redirection
(note that depending on the shell being used,
backslashes may have to be doubled: `|\|' $\to$ `|\\|'):
%
\begin{center}
|... -jobname "|\textit{target}|" |\\|"|[\textit{flags}]%
|\input{childdoc.def}\childdocforward[|\textit{main}|]{|\textit{dest}|}"|
\end{center}
%
Here \textit{target} is the name of the output file,
\textit{main} is the name of the main file
and \textit{dest} is the name of the main or child file to be processed
(all filenames without extensions).
The optional argument \textit{main} can be omitted
if \textit{main} matches \textit{dest}.
Optionally, compilation \textit{flags} can be defined via |\def| commands.
This command line makes the \TeX{} engine believe
it is compiling the file \textit{target}
whose content is specified as the latter parameter.
The provided code then forwards the processing to
\textit{main} or \textit{dest} as described in \secref{sec:forward}.

%%%%%%%%%%%%%%%%%%%%%%%%%%%%%%%%%%%%%%%%%%%%%%%%%%%%%%%%%%%%%%%%%%%%%%%%%%%%%%%%
\subsection{Include by Input}
\label{sec:input}

Including child documents by |\include| has some restrictions by design.
Most notably, the content of a child document always occupies
its own set of pages; pages cannot be shared between child documents.
Usually, this behaviour makes perfect sense
because each child document contain an essential part of the document.
However, in some situations it may be desirable to compose
a document from a collection of parts
without having mandatory page breaks between then.
For this case, the package
provides a mechanism to include parts
by |\input| which can also be processed individually.
However, by construction this mechanism
requires manual handling of the content to be output.

%%%%%%%%%%%%%%%%%%%%%%%%%%%%%%%%%%%%%%%%
\DescribeMacro{\ifchilddocmanual}
The main file should be prepared as usual, see \secref{sec:include}.
However, the document body must make a distinction
between processing of an individual part and of the main document, e.g.:
%
\begin{center}
\begin{tabular}{l}
|\ifchilddocmanual|\\
|\input{\childdocname}|\\
|\||else|\\
\textit{document body with }|\input{|\textit{part}|}|\\
|\||fi|
\end{tabular}
\end{center}
%
The conditional |\ifchilddocmanual| is true whenever
a part to be included by |\input| is being compiled,
and the name of the part is stored in |\childdocname|.

%%%%%%%%%%%%%%%%%%%%%%%%%%%%%%%%%%%%%%%%
\DescribeMacro{\childdocby}
Each part to be included by |\input| should start with:
%
\begin{center}
\begin{tabular}{l}
|\input{childdoc.def}|\\
|\childdocby{|\textit{main}|}|\\
\end{tabular}
\end{center}
%
The directive |\childdocby| is similar to |\childdocof|
described in \secref{sec:include},
but the subsequent selection of content must be done manually.
To that end, both |\ifchilddoc| and |\ifchilddocmanual|
will be true upon processing of a part,
and the name of the part is stored in |\childdocname|.
Note that |\jobname| will be set to the filename of the current part
so that each part receives an individual |.aux| file
that does not interfere with the |.aux| file(s) of the main document.
This behaviour can be altered by the alternative form
|\childdocby[*]{|\textit{main}|}| (with a non-empty optional argument)
which uses the |.aux| file of the main document
by setting |\jobname| to \textit{main}.

%%%%%%%%%%%%%%%%%%%%%%%%%%%%%%%%%%%%%%%%%%%%%%%%%%%%%%%%%%%%%%%%%%%%%%%%%%%%%%%%
\subsection{Driver Development}
\label{sec:driver}

The \textsf{childdoc} mechanism can also be use for the development
of definition files such as \LaTeX{} styles or classes.
This case differs from the above setup with multiple parts
included by |\include| in that no |\includeonly| should be invoked.
This can be achieved by starting the include file
(before |\ProvidesPackage|) with:
%
\begin{center}
\begin{tabular}{l}
|\input{childdoc.def}|\\
|\childdocforward{|\textit{main}|}|\\
\end{tabular}
\end{center}
%
or alternatively with:
%
\begin{center}
\begin{tabular}{l}
|\input{childdoc.def}|\\
|\childdocby{|\textit{main}|}|\\
\end{tabular}
\end{center}
%
Both forms have slightly different effects as described above.
The main file is prepared as usual, see \secref{sec:include}.

%%%%%%%%%%%%%%%%%%%%%%%%%%%%%%%%%%%%%%%%%%%%%%%%%%%%%%%%%%%%%%%%%%%%%%%%%%%%%%%%
\subsection{Legacy Detection}
\label{sec:detection}

The directive |\childdocmain| in the main file can detect
whether the complete document or merely a child is to be compiled
even without using the directive |\childdocof|.
This method is deprecated because it is less robust
and there is no compelling reason to use it;
it is merely provided for backward compatibility
and it may be removed in future versions.

If the detection mechanism is to be used,
it is mandatory to correctly specify
the filename of the main file as the argument of |\childdocmain|:
%
\begin{center}
\begin{tabular}{l}
|\input{childdoc.def}|\\
|\childdocmain{|\textit{main}|}|\\
\end{tabular}
\end{center}
%
If |\jobname| does not match the argument \textit{main} of |\childdocmain|,
it is assumed that |\jobname| points to the child file to be compiled.
When using |\childdocmain| with the main file specified as argument,
it suffices to start a child file
with just |\input{|\textit{main}|}|
without loading of the package and using |\childdocof|.
If instead all processing is done
with the appropriate \textsf{childdoc} directives,
the argument of \textit{main} of |\childdocmain| can be empty.

An alternative version of the command line processing described
in \secref{sec:commandline} using the detection mechanism reads:
%
\begin{center}
|... -jobname "|\textit{target}|" "|[\textit{flags}]%
[|\def\jobname{|\textit{dest}|}|]|\input{|\textit{main}|}"|
\end{center}

%%%%%%%%%%%%%%%%%%%%%%%%%%%%%%%%%%%%%%%%%%%%%%%%%%%%%%%%%%%%%%%%%%%%%%%%%%%%%%%%
\subsection{Manual Code}
\label{sec:manual}

In case one cannot be certain whether the definitions file |childdoc.def|
is installed on the target \TeX{} distribution
and one prefers not to ship it,
it is conceivable to paste a few relevant commands into the sources.

To that end, drop all statements |\input{childdoc.def}|
and perform the replacements as outlined below.
Instead of |\childdocmain{|\textit{main}|}| add the following code
to the top of the main file:
%
\begin{center}
\begin{tabular}{l}
|\||ifdefined\childdocname\endinput\||fi\newif\ifchilddoc|\\
|\edef\childdocname{\scantokens\expandafter{\jobname\noexpand}}|\\
|\def\childdocmain{|\textit{main}|}\||ifx\childdocmain\childdocname\||else|\\
|\childdoctrue\includeonly{\childdocname}\let\jobname\childdocmain\||fi|\\
\end{tabular}
\end{center}
%
Instead of |\childdocof{|\textit{main}|}| just include the main file
at the top of each child file:
%
\begin{center}
|\input{|\textit{main}|}|
\end{center}
%
A simple redirection |\childdocforward{|\textit{dest}|}| is achieved by:
%
\begin{center}
|\def\jobname{|\textit{dest}|}\input{\jobname}|
\end{center}
%
The redirection with prefix
|\childdocforwardprefix[|\textit{prefix}|]{|\textit{dest}|}|
is accomplished by:
%
\begin{center}
\begin{tabular}{l}
|{\edef\jobname{\scantokens\expandafter{\jobname\noexpand}}|\\
|\def\redirectjob |\textit{prefix}|#1~~~{\gdef\jobname{|\textit{dest}|#1}}|\\
|\expandafter\redirectjob\jobname~~~}\input{\jobname}|
\end{tabular}
\end{center}

In an alternative approach,
child documents can be compiled by a specific command line
without additional code or specific definitions:
%
\begin{center}
|... -jobname "|\textit{target}|" "|[\textit{flags}]%
|\includeonly{|\textit{dest}|}\input{|\textit{main}|}"|
\end{center}
%

%%%%%%%%%%%%%%%%%%%%%%%%%%%%%%%%%%%%%%%%%%%%%%%%%%%%%%%%%%%%%%%%%%%%%%%%%%%%%%%%
%%%%%%%%%%%%%%%%%%%%%%%%%%%%%%%%%%%%%%%%%%%%%%%%%%%%%%%%%%%%%%%%%%%%%%%%%%%%%%%%
\section{Information}

%%%%%%%%%%%%%%%%%%%%%%%%%%%%%%%%%%%%%%%%%%%%%%%%%%%%%%%%%%%%%%%%%%%%%%%%%%%%%%%%
\subsection{Copyright}

Copyright \copyright{} 2017--2018 Niklas Beisert

This work may be distributed and/or modified under the
conditions of the \LaTeX{} Project Public License, either version 1.3
of this license or (at your option) any later version.
The latest version of this license is in
  \url{http://www.latex-project.org/lppl.txt}
and version 1.3 or later is part of all distributions of \LaTeX{}
version 2005/12/01 or later.

This work has the LPPL maintenance status `maintained'.

The Current Maintainer of this work is Niklas Beisert.

This work consists of the files |README.txt|, |childdoc.ins| and |childdoc.dtx|
as well as the derived files |childdoc.def|, |cdocsamp.tex|
with |cdocsch1.tex|, |cdocsch2.tex|, |cdocspt3.tex|, |cdocspt4.tex|,
|cdocsdrf.tex|, |cdocsfn1.tex|, |cdocsfn2.tex|
as well as |childdoc.pdf|.

%%%%%%%%%%%%%%%%%%%%%%%%%%%%%%%%%%%%%%%%%%%%%%%%%%%%%%%%%%%%%%%%%%%%%%%%%%%%%%%%
\subsection{Files and Installation}

The package consists of the files:
%
\begin{center}
\begin{tabular}{ll}
    |README.txt|   & readme file \\
    |childdoc.ins| & installation file \\
    |childdoc.dtx| & source file \\
    |childdoc.def| & definition file \\
    |cdocsamp.tex| & sample main file \\
    |cdocsch1.tex| & sample include file \\
    |cdocsch2.tex| & sample include file \\
    |cdocspt3.tex| & sample part file \\
    |cdocspt4.tex| & sample part file \\
    |cdocsdrf.tex| & sample redirection file \\
    |cdocsfn1.tex| & sample redirection file \\
    |cdocsfn2.tex| & sample redirection file \\
    |childdoc.pdf| & manual
\end{tabular}
\end{center}
%
The distribution consists of the files
|README.txt|, |childdoc.ins| and |childdoc.dtx|.
%
\begin{itemize}
\item
Run (pdf)\LaTeX{} on |childdoc.dtx|
to compile the manual |childdoc.pdf| (this file).
\item
Run \LaTeX{} on |childdoc.ins| to create the definitions file |childdoc.def|
and the sample |cdocsamp.tex| with include files
|cdocsch1.tex|, |cdocsch2.tex|, |cdocspt3.tex|, |cdocspt4.tex|,
|cdocsdrf.tex|, |cdocsfn1.tex|, |cdocsfn2.tex|.
Then copy the file |childdoc.def| to an appropriate directory of your \LaTeX{}
distribution, e.g.\ \textit{texmf-root}|/tex/latex/childdoc|.
\end{itemize}

%%%%%%%%%%%%%%%%%%%%%%%%%%%%%%%%%%%%%%%%%%%%%%%%%%%%%%%%%%%%%%%%%%%%%%%%%%%%%%%%
\subsection{Related CTAN Packages}

There are several other packages which offer a similar functionality:
%
\begin{itemize}
\item
The packages
\href{http://ctan.org/pkg/docmute}{\textsf{docmute}},
\href{http://ctan.org/pkg/includex}{\textsf{includex}} and
\href{http://ctan.org/pkg/standalone}{\textsf{standalone}}
provide commands to include only the document body of
a child file thus allowing both files to be compiled individually.
\item
The packages \href{http://ctan.org/pkg/subdocs}{\textsf{subdocs}}
and \href{http://ctan.org/pkg/subfiles}{\textsf{subfiles}}
provide structures in which the main and child documents can be
encapsulated and allowing them to be compiled individually.
The inclusion mechanism is different from the conventional |\include|.
\item
The package \href{http://ctan.org/pkg/combine}{\textsf{combine}}
is an elaborate solution to combine several documents into one.
\end{itemize}
%
See also the CTAN topic \href{http://ctan.org/topic/subdocs}{\textsf{subdocs}}
for further related packages.
The present package differs from the above solutions in that
a document structure constructed with the conventional |\include| mechanism
just needs two extra commands at the top of every file
such that all constituent files can be compiled individually.

%%%%%%%%%%%%%%%%%%%%%%%%%%%%%%%%%%%%%%%%%%%%%%%%%%%%%%%%%%%%%%%%%%%%%%%%%%%%%%%%
%\subsection{Feature Suggestions}
%
%The following is a list of features which may be useful for future
%versions of this package:
%%
%\begin{itemize}
%\item
%\ldots
%\end{itemize}

%%%%%%%%%%%%%%%%%%%%%%%%%%%%%%%%%%%%%%%%%%%%%%%%%%%%%%%%%%%%%%%%%%%%%%%%%%%%%%%%
\subsection{Revision History}

%%%%%%%%%%%%%%%%%%%%%%%%%%%%%%%%%%%%%%%%
\paragraph{v2.0:} 2018/12/30

\begin{itemize}
\item
immediate forward processing
\item
added |\childdocby| mechanism
\item
manual restructured
\end{itemize}

%%%%%%%%%%%%%%%%%%%%%%%%%%%%%%%%%%%%%%%%
\paragraph{v1.6:} 2018/01/17

\begin{itemize}
\item
application for development of include files
\item
corrections to manual
\end{itemize}

%%%%%%%%%%%%%%%%%%%%%%%%%%%%%%%%%%%%%%%%
\paragraph{v1.5:} 2017/05/21

\begin{itemize}
\item
more complete structuring introduced
\item
|\childdocof| introduced
\item
|\childdoc| renamed to |\childdocmain|
\item
|\childredirect| renamed to |\childdocforward| and |\childdocforwardprefix|
and functionality expanded
\end{itemize}

%%%%%%%%%%%%%%%%%%%%%%%%%%%%%%%%%%%%%%%%
\paragraph{v1.0:} 2017/04/27

\begin{itemize}
\item
manual and install package
\item
first version published on CTAN
\end{itemize}

%%%%%%%%%%%%%%%%%%%%%%%%%%%%%%%%%%%%%%%%
\paragraph{v0.6:} 2017/04/26

\begin{itemize}
\item
redirection mechanism added
\end{itemize}

%%%%%%%%%%%%%%%%%%%%%%%%%%%%%%%%%%%%%%%%
\paragraph{v0.5:} 2017/04/26

\begin{itemize}
\item
functionality in definition file
\end{itemize}


%%%%%%%%%%%%%%%%%%%%%%%%%%%%%%%%%%%%%%%%%%%%%%%%%%%%%%%%%%%%%%%%%%%%%%%%%%%%%%%%
%%%%%%%%%%%%%%%%%%%%%%%%%%%%%%%%%%%%%%%%%%%%%%%%%%%%%%%%%%%%%%%%%%%%%%%%%%%%%%%%
%%%%%%%%%%%%%%%%%%%%%%%%%%%%%%%%%%%%%%%%%%%%%%%%%%%%%%%%%%%%%%%%%%%%%%%%%%%%%%%%
\appendix

\settowidth\MacroIndent{\rmfamily\scriptsize 000\ }

 \DocInput{childdoc.dtx}

\end{document}
%</driver>
% \fi
%
% %%%%%%%%%%%%%%%%%%%%%%%%%%%%%%%%%%%%%%%%%%%%%%%%%%%%%%%%%%%%%%%%%%%%%%%%%%%%%%
% %%%%%%%%%%%%%%%%%%%%%%%%%%%%%%%%%%%%%%%%%%%%%%%%%%%%%%%%%%%%%%%%%%%%%%%%%%%%%%
% \section{Sample}
%\iffalse
%<*samplemain>
%\fi
%
% The following presents a sample document
% with two chapters, two parts, a title page,
% a compile flag as well as three forwarding files to set the flag.
% It consists of eight |.tex| files:
% \begin{center}
% \begin{tabular}{ll}
% |cdocsamp.tex|&main file\\
% |cdocsch1.tex|&include file for chapter 1\\
% |cdocsch2.tex|&include file for chapter 2\\
% |cdocspt3.tex|&include file for part 3\\
% |cdocspt4.tex|&include file for part 4\\
% |cdocsdrf.tex|&forwarding file for main file in draft mode\\
% |cdocsfi1.tex|&forwarding file for final version of chapter 1\\
% |cdocsfi2.tex|&forwarding file for final version of chapter 2\\
% \end{tabular}
% \end{center}
% Each of the eight files can be compiled directly by the \LaTeX{} compiler.
%
% %%%%%%%%%%%%%%%%%%%%%%%%%%%%%%%%%%%%%%
% \paragraph{Main File.}
%
% The main file is called |cdocsamp.tex|.
%
% Load the \textsf{childdoc} definitions and
% declare the filename for the main document:
%    \begin{macrocode}
\input{childdoc.def}
\childdocmain{}
%    \end{macrocode}

% Optional override for |\version| flag:
%    \begin{macrocode}
%%\ifchilddoc\else\providecommand{\version}{draft}\fi
%    \end{macrocode}

% Define the default values for the |\version| flag
% (|final| for the main file and |draft| for childs):
%    \begin{macrocode}
\ifchilddoc
\providecommand{\version}{draft}
\else
\providecommand{\version}{final}
\fi
%    \end{macrocode}

% Load the standard document class:
%    \begin{macrocode}
\documentclass[12pt]{article}
%    \end{macrocode}

% Start the document body:
%    \begin{macrocode}
\begin{document}
%    \end{macrocode}

% Declare a title page.
% Print title, part of document being processed and version flag:
%    \begin{macrocode}
\addtocounter{page}{-1}
\begin{center}
{\LARGE\bfseries{}childdoc example\par}
\vspace{1cm}
\ifchilddoc
\ifchilddocmanual part\else chapter\fi:
`\childdocname' of `\childdocjob'\par
\else
main document: `\childdocjob'\par
\fi
version: \version\par
\end{center}
\newpage
%    \end{macrocode}

% Manually include selected file,
% otherwise process as usual:
%    \begin{macrocode}
\ifchilddocmanual
\section*{part `\childdocname'}
\input{\childdocname}
\else
%    \end{macrocode}

% Include the two chapters:
%    \begin{macrocode}
\include{cdocsch1}
\include{cdocsch2}
%    \end{macrocode}

% Include the two parts unless only chapters should be displayed:
%    \begin{macrocode}
\ifchilddoc\else
\section{part three}
\input{cdocspt3}
\section{part four}
\input{cdocspt4}
\fi
%    \end{macrocode}

% Process as usual until here:
%    \begin{macrocode}
\fi
%    \end{macrocode}

% End of document body:
%    \begin{macrocode}
\end{document}
%    \end{macrocode}
%\iffalse
%</samplemain>
%\fi
%
% %%%%%%%%%%%%%%%%%%%%%%%%%%%%%%%%%%%%%%
% \paragraph{Chapter Include Files.}
%
% The include files are called |cdocsch1.tex| and |cdocsch2.tex|.
%
%\iffalse
%<*samplechap1|samplechap2>
%\fi

% Optional override for |\version| flag:
%    \begin{macrocode}
%%\providecommand{\version}{final}
%    \end{macrocode}

% Include the main document:
%    \begin{macrocode}
\input{childdoc.def}
\childdocof{cdocsamp}
%    \end{macrocode}

%\iffalse
%</samplechap1|samplechap2>
%\fi
%
%\iffalse
%<*samplechap1>
%\fi
% Some text for chapter 1:
%    \begin{macrocode}
\section{one}
some text in chapter one
%    \end{macrocode}

%\iffalse
%</samplechap1>
%\fi
% Some text for chapter 2:
%\iffalse
%<*samplechap2>
%\fi
%    \begin{macrocode}
\section{two}
more text in chapter two
%    \end{macrocode}

%\iffalse
%</samplechap2>
%\fi
%
% %%%%%%%%%%%%%%%%%%%%%%%%%%%%%%%%%%%%%%
% \paragraph{Part Include Files.}
%
% The include files are called |cdocspt3.tex| and |cdocspt4.tex|.
%
%\iffalse
%<*samplepart3|samplepart4>
%\fi

% Optional override for |\version| flag:
%    \begin{macrocode}
%%\providecommand{\version}{final}
%    \end{macrocode}

% Include the main document:
%    \begin{macrocode}
\input{childdoc.def}
\childdocby{cdocsamp}
%    \end{macrocode}

%\iffalse
%</samplepart3|samplepart4>
%\fi
%
%\iffalse
%<*samplepart3>
%\fi
% Some text for part 3:
%    \begin{macrocode}
some text in part three
%    \end{macrocode}

%\iffalse
%</samplepart3>
%\fi
% Some text for part 4:
%\iffalse
%<*samplepart4>
%\fi
%    \begin{macrocode}
more text in part four
%    \end{macrocode}

%\iffalse
%</samplepart4>
%\fi
%
% %%%%%%%%%%%%%%%%%%%%%%%%%%%%%%%%%%%%%%
% \paragraph{Forwarding for a Complete Draft.}
%
% The following forwarding file |cdocsdrf.tex|
% compiles the main document in draft mode:
%\iffalse
%<*sampledraft>
%\fi
%    \begin{macrocode}
\def\version{draft}
\input{childdoc.def}
\childdocforward{cdocsamp}
%    \end{macrocode}

%\iffalse
%</sampledraft>
%\fi
%
% %%%%%%%%%%%%%%%%%%%%%%%%%%%%%%%%%%%%%%
% \paragraph{Forwarding for Final Version of the Chapters.}
%
% The following forwarding files |cdocsfn1.tex| and |cdocsfn2.tex|
% (with identical content)
% compile the final versions of the child documents
% |cdocsch1.tex| and |cdocsch2.tex|, respectively:
%\iffalse
%<*samplefinal>
%\fi
%    \begin{macrocode}
\def\version{final}
\input{childdoc.def}
\childdocforwardprefix[cdocsamp]{cdocsfn}{cdocsch}
%    \end{macrocode}

%\iffalse
%</samplefinal>
%\fi
%
% %%%%%%%%%%%%%%%%%%%%%%%%%%%%%%%%%%%%%%
% \paragraph{Command Line Processing.}
%
% The following three command lines generate the output files
% |cdocscld|, |cdocscl1| and |cdocscl2|
% which should be identical to
% |cdocsdrf|, |cdocsch1| and |cdocsfn2|, respectively:
% \begin{center}
% \begin{tabular}{l}
% |latex -jobname cdocscld \|\\
% |  "\def\version{draft}\input{childdoc.def}\childdocforward{cdocsamp}"|\\
% |latex -jobname cdocscl1 \|\\
% |  "\input{childdoc.def}\childdocforward[cdocsamp]{cdocsch1}"|\\
% |latex -jobname cdocscl2 \|\\
% |  "\def\version{final}\input{childdoc.def}\childdocforward{cdocsch2}"|
% \end{tabular}
% \end{center}
% Note that the trailing backslash on each first line
% merely continues the input to the second line
% (for convenient cut ant paste).
% Furthermore, the command |latex| can be replaced by any
% of its alternative versions such as |pdflatex|.
%
% %%%%%%%%%%%%%%%%%%%%%%%%%%%%%%%%%%%%%%%%%%%%%%%%%%%%%%%%%%%%%%%%%%%%%%%%%%%%%%
% %%%%%%%%%%%%%%%%%%%%%%%%%%%%%%%%%%%%%%%%%%%%%%%%%%%%%%%%%%%%%%%%%%%%%%%%%%%%%%
% \section{Implementation}
%\iffalse
%<*package>
%\fi
%
% This section describes the definitions file |childdoc.def|.

% The definitions cannot be loaded using |\usepackage| or |\RequirePackage|
% which has a mechanism to prevent loading a style file more than once.
% When loading the definitions by means of |\input|
% multiple instances have to be prevented manually:
%\iffalse
%This code needs to be before the `\ProvidesFile' directive
%which is defined at the beginning of this file.
%Therefore it is also placed there and commented out here.
%</package>
%<*discard>
%\fi
%    \begin{macrocode}
\ifdefined\childdocmain\endinput\fi
%    \end{macrocode}
%\iffalse
%</discard>
%<*package>
%\fi
%
% \macro{\ifchilddoc}
% \macro{\ifchilddocmanual}
% The conditional |\ifchilddoc| tells whether a
% child (true) or main (false) document is being compiled.
% The conditional |\ifchilddocmanual| tells whether
% the |\includeonly| mechanism is used (false) or
% the selection of child files must be performed manually (true).
% The definitions initialise to false:
%    \begin{macrocode}
\newif\ifchilddoc
\newif\ifchilddocmanual
%    \end{macrocode}

% \macro{\childdocname}
% \macro{\childdocjob}
% The macro |\childdocname| stores the name of the main document
% to be compiled. The macro |\childdocjob| stores the name of
% the document on which the \LaTeX{} compiler was originally invoked.
% The content of |\jobname| cannot be compared
% to filenames specified in the source due to different catcodes.
% The following code rescans |\jobname|, stores the result
% in |\childdocname| and saves a copy in |\childdocjob|:
%    \begin{macrocode}
\edef\childdocname{\scantokens\expandafter{\jobname\noexpand}}
\let\childdocjob\childdocname
%    \end{macrocode}

% \macro{\childdocdisable}
% The macro |\childdocdisable| prevents the main file
% from being processed more than once.
% At this stage, the main document command |\childdocmain|
% is assumed to be called once again where it should do nothing.
% Any subsequent call to it should prevent
% a secondary processing of the main document
% It overwrites the forwarding commands
% |\childdocof| and |\childdocforward|
% with empty macros to prevent further inclusions of the main document:
%    \begin{macrocode}
\newcommand{\childdocdisable}
{
  \renewcommand{\childdocmain}[1]{\renewcommand{\childdocmain}[1]{\endinput}}
  \renewcommand{\childdocof}[1]{}
  \renewcommand{\childdocby}[2][]{}
  \renewcommand{\childdocforward}[2][]{}
  \renewcommand{\childdocdisable}{}
}
%    \end{macrocode}

% \macro{\childdocmain}
% The macro |\childdocmain| is to be called at the top of the main file
% with nothing or the main filename (without extension) as argument.
% First, it breaks loops.
% If the argument is not empty and does not match |\childdocname|
% (which is set by the first inclusion of |childdoc.def|),
% |\ifchilddoc| is set to true, |\includeonly| is applied to the child file
% and |\jobname| is set to the main file
% (for proper handling of |.aux| files):
%    \begin{macrocode}
\newcommand{\childdocmain}[1]
{
  \childdocdisable\childdocmain{}
  \if?#1?\else
    \begingroup
      \def\childdoctmp{#1}
      \ifx\childdoctmp\childdocname
        \def\childdoctmp{}
      \else
        \def\childdoctmp
        {
          \childdoctrue
          \includeonly{\childdocname}
          \def\childdocjob{#1}
          \def\jobname{#1}
        }
      \fi
      \expandafter
    \endgroup
    \childdoctmp
  \fi
}
%    \end{macrocode}

% \macro{\childdocof}
% The command |\childdocof| redirects
% compilation to the main file |#1|.
%    \begin{macrocode}
\newcommand{\childdocof}[1]
{
  \childdocdisable
  \childdoctrue
  \includeonly{\childdocname}
  \def\jobname{#1}
  \def\childdocjob{#1}
  \input{#1}
}
%    \end{macrocode}

% \macro{\childdocby}
% The command |\childdocby| ....
%    \begin{macrocode}
\newcommand{\childdocby}[2][]
{
  \childdocdisable
  \childdoctrue
  \childdocmanualtrue
  \if?#1?\else
    \def\jobname{#2}
  \fi
  \def\childdocjob{#2}
  \input{#2}
  \endinput
}
%    \end{macrocode}

% \macro{\childdocforward}
% The command |\childdocforward| redirects
% compilation to the main file or
% (if the optional argument is given) a child file.
% Parameters are set as if the main file
% or a child file starting with |\childdocof| was compiled.
% Then compilation is handed over to the main file:
%    \begin{macrocode}
\newcommand{\childdocforward}[2][]
{
  \begingroup
    \if?#1?
      \def\childdoctmp
      {
        \def\childdocname{#2}
        \def\childdocjob{#2}
        \def\jobname{#2}
        \input{#2}
        \endinput
      }
    \else
      \def\childdoctmp
      {
        \childdocdisable
        \def\childdocname{#2}
        \childdoctrue
        \includeonly{#2}
        \def\childdocjob{#1}
        \def\jobname{#1}
        \input{#1}
        \endinput
      }
    \fi
    \expandafter
  \endgroup
  \childdoctmp
}
%    \end{macrocode}

% \macro{\childdocforwardprefix}
% The command |\childdocforwardprefix| redirects
% compilation to the main or a child file by means of a pattern.
% The prefix |#1| in the current filename is replaced by |#2|
% and the suffix of the current filename is kept
% (it is assumed that the filename does not contain the substring `|~~~|'
% which is used as a delimiter).
% Compilation is handed over to the new file by |\childdocforward|:
%    \begin{macrocode}
\newcommand{\childdocforwardprefix}[3][]
{
  \begingroup
    \def\childdocextract #2##1~~~{\def\childdoctmp{\childdocforward[#1]{#3##1}}}
    \expandafter\childdocextract\childdocname~~~
    \expandafter
  \endgroup
  \childdoctmp
}
%    \end{macrocode}

% \macro{\childdoc}
% The deprecated macro |\childdoc| is a legacy version of |\childdocmain|:
%    \begin{macrocode}
\newcommand{\childdoc}{\childdocmain}
%    \end{macrocode}

% \macro{\childdocredirect}
% The deprecated macro |\childdocredirect| is a legacy version
% of |\childdocforward| and |\childdocforwardprefix|:
%    \begin{macrocode}
\newcommand{\childdocredirect}[2][]
{
  \begingroup
    \if?#1?
      \def\childdoctmp{\childdocforward{#2}}
    \else
      \def\childdoctmp{\childdocforwardprefix{#1}{#2}}
    \fi
    \expandafter
  \endgroup
  \childdoctmp
}
%    \end{macrocode}

%\iffalse
%</package>
%\fi
%
\endinput
\childdocforward{cdocsch2}"|
% \end{tabular}
% \end{center}
% Note that the trailing backslash on each first line
% merely continues the input to the second line
% (for convenient cut ant paste).
% Furthermore, the command |latex| can be replaced by any
% of its alternative versions such as |pdflatex|.
%
% %%%%%%%%%%%%%%%%%%%%%%%%%%%%%%%%%%%%%%%%%%%%%%%%%%%%%%%%%%%%%%%%%%%%%%%%%%%%%%
% %%%%%%%%%%%%%%%%%%%%%%%%%%%%%%%%%%%%%%%%%%%%%%%%%%%%%%%%%%%%%%%%%%%%%%%%%%%%%%
% \section{Implementation}
%\iffalse
%<*package>
%\fi
%
% This section describes the definitions file |childdoc.def|.

% The definitions cannot be loaded using |\usepackage| or |\RequirePackage|
% which has a mechanism to prevent loading a style file more than once.
% When loading the definitions by means of |\input|
% multiple instances have to be prevented manually:
%\iffalse
%This code needs to be before the `\ProvidesFile' directive
%which is defined at the beginning of this file.
%Therefore it is also placed there and commented out here.
%</package>
%<*discard>
%\fi
%    \begin{macrocode}
\ifdefined\childdocmain\endinput\fi
%    \end{macrocode}
%\iffalse
%</discard>
%<*package>
%\fi
%
% \macro{\ifchilddoc}
% \macro{\ifchilddocmanual}
% The conditional |\ifchilddoc| tells whether a
% child (true) or main (false) document is being compiled.
% The conditional |\ifchilddocmanual| tells whether
% the |\includeonly| mechanism is used (false) or
% the selection of child files must be performed manually (true).
% The definitions initialise to false:
%    \begin{macrocode}
\newif\ifchilddoc
\newif\ifchilddocmanual
%    \end{macrocode}

% \macro{\childdocname}
% \macro{\childdocjob}
% The macro |\childdocname| stores the name of the main document
% to be compiled. The macro |\childdocjob| stores the name of
% the document on which the \LaTeX{} compiler was originally invoked.
% The content of |\jobname| cannot be compared
% to filenames specified in the source due to different catcodes.
% The following code rescans |\jobname|, stores the result
% in |\childdocname| and saves a copy in |\childdocjob|:
%    \begin{macrocode}
\edef\childdocname{\scantokens\expandafter{\jobname\noexpand}}
\let\childdocjob\childdocname
%    \end{macrocode}

% \macro{\childdocdisable}
% The macro |\childdocdisable| prevents the main file
% from being processed more than once.
% At this stage, the main document command |\childdocmain|
% is assumed to be called once again where it should do nothing.
% Any subsequent call to it should prevent
% a secondary processing of the main document
% It overwrites the forwarding commands
% |\childdocof| and |\childdocforward|
% with empty macros to prevent further inclusions of the main document:
%    \begin{macrocode}
\newcommand{\childdocdisable}
{
  \renewcommand{\childdocmain}[1]{\renewcommand{\childdocmain}[1]{\endinput}}
  \renewcommand{\childdocof}[1]{}
  \renewcommand{\childdocby}[2][]{}
  \renewcommand{\childdocforward}[2][]{}
  \renewcommand{\childdocdisable}{}
}
%    \end{macrocode}

% \macro{\childdocmain}
% The macro |\childdocmain| is to be called at the top of the main file
% with nothing or the main filename (without extension) as argument.
% First, it breaks loops.
% If the argument is not empty and does not match |\childdocname|
% (which is set by the first inclusion of |childdoc.def|),
% |\ifchilddoc| is set to true, |\includeonly| is applied to the child file
% and |\jobname| is set to the main file
% (for proper handling of |.aux| files):
%    \begin{macrocode}
\newcommand{\childdocmain}[1]
{
  \childdocdisable\childdocmain{}
  \if?#1?\else
    \begingroup
      \def\childdoctmp{#1}
      \ifx\childdoctmp\childdocname
        \def\childdoctmp{}
      \else
        \def\childdoctmp
        {
          \childdoctrue
          \includeonly{\childdocname}
          \def\childdocjob{#1}
          \def\jobname{#1}
        }
      \fi
      \expandafter
    \endgroup
    \childdoctmp
  \fi
}
%    \end{macrocode}

% \macro{\childdocof}
% The command |\childdocof| redirects
% compilation to the main file |#1|.
%    \begin{macrocode}
\newcommand{\childdocof}[1]
{
  \childdocdisable
  \childdoctrue
  \includeonly{\childdocname}
  \def\jobname{#1}
  \def\childdocjob{#1}
  \input{#1}
}
%    \end{macrocode}

% \macro{\childdocby}
% The command |\childdocby| ....
%    \begin{macrocode}
\newcommand{\childdocby}[2][]
{
  \childdocdisable
  \childdoctrue
  \childdocmanualtrue
  \if?#1?\else
    \def\jobname{#2}
  \fi
  \def\childdocjob{#2}
  \input{#2}
  \endinput
}
%    \end{macrocode}

% \macro{\childdocforward}
% The command |\childdocforward| redirects
% compilation to the main file or
% (if the optional argument is given) a child file.
% Parameters are set as if the main file
% or a child file starting with |\childdocof| was compiled.
% Then compilation is handed over to the main file:
%    \begin{macrocode}
\newcommand{\childdocforward}[2][]
{
  \begingroup
    \if?#1?
      \def\childdoctmp
      {
        \def\childdocname{#2}
        \def\childdocjob{#2}
        \def\jobname{#2}
        \input{#2}
        \endinput
      }
    \else
      \def\childdoctmp
      {
        \childdocdisable
        \def\childdocname{#2}
        \childdoctrue
        \includeonly{#2}
        \def\childdocjob{#1}
        \def\jobname{#1}
        \input{#1}
        \endinput
      }
    \fi
    \expandafter
  \endgroup
  \childdoctmp
}
%    \end{macrocode}

% \macro{\childdocforwardprefix}
% The command |\childdocforwardprefix| redirects
% compilation to the main or a child file by means of a pattern.
% The prefix |#1| in the current filename is replaced by |#2|
% and the suffix of the current filename is kept
% (it is assumed that the filename does not contain the substring `|~~~|'
% which is used as a delimiter).
% Compilation is handed over to the new file by |\childdocforward|:
%    \begin{macrocode}
\newcommand{\childdocforwardprefix}[3][]
{
  \begingroup
    \def\childdocextract #2##1~~~{\def\childdoctmp{\childdocforward[#1]{#3##1}}}
    \expandafter\childdocextract\childdocname~~~
    \expandafter
  \endgroup
  \childdoctmp
}
%    \end{macrocode}

% \macro{\childdoc}
% The deprecated macro |\childdoc| is a legacy version of |\childdocmain|:
%    \begin{macrocode}
\newcommand{\childdoc}{\childdocmain}
%    \end{macrocode}

% \macro{\childdocredirect}
% The deprecated macro |\childdocredirect| is a legacy version
% of |\childdocforward| and |\childdocforwardprefix|:
%    \begin{macrocode}
\newcommand{\childdocredirect}[2][]
{
  \begingroup
    \if?#1?
      \def\childdoctmp{\childdocforward{#2}}
    \else
      \def\childdoctmp{\childdocforwardprefix{#1}{#2}}
    \fi
    \expandafter
  \endgroup
  \childdoctmp
}
%    \end{macrocode}

%\iffalse
%</package>
%\fi
%
\endinput
|\\
|\childdocby{|\textit{main}|}|\\
\end{tabular}
\end{center}
%
Both forms have slightly different effects as described above.
The main file is prepared as usual, see \secref{sec:include}.

%%%%%%%%%%%%%%%%%%%%%%%%%%%%%%%%%%%%%%%%%%%%%%%%%%%%%%%%%%%%%%%%%%%%%%%%%%%%%%%%
\subsection{Legacy Detection}
\label{sec:detection}

The directive |\childdocmain| in the main file can detect
whether the complete document or merely a child is to be compiled
even without using the directive |\childdocof|.
This method is deprecated because it is less robust
and there is no compelling reason to use it;
it is merely provided for backward compatibility
and it may be removed in future versions.

If the detection mechanism is to be used,
it is mandatory to correctly specify
the filename of the main file as the argument of |\childdocmain|:
%
\begin{center}
\begin{tabular}{l}
|% \iffalse
%
% childdoc.dtx Copyright (C) 2017-2018 Niklas Beisert
%
% This work may be distributed and/or modified under the
% conditions of the LaTeX Project Public License, either version 1.3
% of this license or (at your option) any later version.
% The latest version of this license is in
%   http://www.latex-project.org/lppl.txt
% and version 1.3 or later is part of all distributions of LaTeX
% version 2005/12/01 or later.
%
% This work has the LPPL maintenance status `maintained'.
%
% The Current Maintainer of this work is Niklas Beisert.
%
% This work consists of the files childdoc.dtx and childdoc.ins
% and the derived files childdoc.def and cdocsamp.tex with
% cdocsch1.tex, cdocsch2.tex, cdocsdrf.tex, cdocsfn1.tex, cdocsfn2.tex.
%
%<package>\ifdefined\childdocmain\endinput\fi
%<package>\ProvidesFile{childdoc.def}[2018/12/30 v2.0 child document driver]
%<samplemain>\ProvidesFile{cdocsamp.tex}[2018/12/30 v2.0 sample for childdoc]
%<*driver>
%\ProvidesFile{childdoc.drv}[2018/12/30 v2.0 childdoc reference manual file]
\PassOptionsToClass{10pt,a4paper}{article}
\documentclass{ltxdoc}

\usepackage[margin=35mm]{geometry}
\usepackage{hyperref}
\usepackage{hyperxmp}
\usepackage[usenames]{color}

\hypersetup{colorlinks=true}
\hypersetup{pdfstartview=FitH}
\hypersetup{pdfpagemode=UseNone}
\hypersetup{pdfsource={}}
\hypersetup{pdflang={en-UK}}
\hypersetup{pdfcopyright={Copyright 2017-2018 Niklas Beisert.
  This work may be distributed and/or modified under the
  conditions of the LaTeX Project Public License, either version 1.3
  of this license or (at your option) any later version.}}
\hypersetup{pdflicenseurl={http://www.latex-project.org/lppl.txt}}
\hypersetup{pdfcontactaddress={ETH Zurich, ITP, HIT K,
  Wolfgang-Pauli-Strasse 27}}
\hypersetup{pdfcontactpostcode={8093}}
\hypersetup{pdfcontactcity={Zurich}}
\hypersetup{pdfcontactcountry={Switzerland}}
\hypersetup{pdfcontactemail={nbeisert@itp.phys.ethz.ch}}
\hypersetup{pdfcontacturl={http://people.phys.ethz.ch/\xmptilde nbeisert/}}

\newcommand{\secref}[1]{\hyperref[#1]{section \ref*{#1}}}

\parskip1ex
\parindent0pt
\let\olditemize\itemize
\def\itemize{\olditemize\parskip0pt}

\begin{document}

\title{The \textsf{childdoc} Package}
\hypersetup{pdftitle={The childdoc Package}}
\author{Niklas Beisert\\[2ex]
  Institut f\"ur Theoretische Physik\\
  Eidgen\"ossische Technische Hochschule Z\"urich\\
  Wolfgang-Pauli-Strasse 27, 8093 Z\"urich, Switzerland\\[1ex]
  \href{mailto:nbeisert@itp.phys.ethz.ch}
  {\texttt{nbeisert@itp.phys.ethz.ch}}}
\hypersetup{pdfauthor={Niklas Beisert}}
\hypersetup{pdfsubject={Manual for the LaTeX2e Package childdoc}}
\date{30 December 2018, \textsf{v2.0}}
\maketitle

\begin{abstract}\noindent
\textsf{childdoc} is a \LaTeXe{} package
that enables the direct compilation
of document sections included by |\include|
to individual files.
\end{abstract}

\begingroup
\parskip0ex
\tableofcontents
\endgroup

%%%%%%%%%%%%%%%%%%%%%%%%%%%%%%%%%%%%%%%%%%%%%%%%%%%%%%%%%%%%%%%%%%%%%%%%%%%%%%%%
%%%%%%%%%%%%%%%%%%%%%%%%%%%%%%%%%%%%%%%%%%%%%%%%%%%%%%%%%%%%%%%%%%%%%%%%%%%%%%%%
\section{Introduction}

\LaTeX{} provides a mechanism to structure a large document (such as a book)
into a main file and several child files (containing the chapters)
using the |\include| command.
This mechanism is beneficial for documents
which span hundreds of pages in order to
make the source file(s) more manageable.
Moreover, compilation can be restricted to
selected child files by means of the |\includeonly| command.
The latter feature can be used to reduce the compilation time while editing
(this was significantly more useful in the earlier days of \LaTeX{})
or to generate a smaller document which is easier to navigate.
Another application of |\includeonly| is to generate
documents consisting of selected parts of the complete document.

However, there are a few drawbacks of the plain |\include| mechanism:
\begin{itemize}
\item
The child files cannot be compiled on their own,
they can only be compiled via the main file.
A naive editing environment
(such as a text editor with an option
to have the current file processed by \LaTeX)
may require one to switch to the main file before compiling;
attempting to compile the child file produces errors.
\item
The main file must be modified (each time)
to adjust the |\includeonly| command
to the present needs. This easily leaves the main file in a messy state.
\item
The generated document will always carry the filename
of the main document. This is inconvenient if
several child files are to be compiled and
to be kept for distribution.
\end{itemize}

The present package provides a simple interface
to make child files individually compilable by \LaTeX{}.
Compiling a child file then has the same effect as compiling
the main file with an |\includeonly| command
to select the appropriate child.
Moreover the generated document will carry the name of the child
rather than the main file.
This resolves all three above issues.

This feature is meant to make the editing of books,
thesis documents and lecture notes somewhat more convenient.
However, the package can also be used efficiently for
composing a series of documents (such as exercise sheets)
which are typically distributed individually.
It then assists the author in generating the individual documents
(potentially in different versions)
as well as a document containing the collected series.
Another application is in developing style files
or other kinds of included material
where compilation of the style file could redirect
to a sample or test file.

%%%%%%%%%%%%%%%%%%%%%%%%%%%%%%%%%%%%%%%%%%%%%%%%%%%%%%%%%%%%%%%%%%%%%%%%%%%%%%%%
%%%%%%%%%%%%%%%%%%%%%%%%%%%%%%%%%%%%%%%%%%%%%%%%%%%%%%%%%%%%%%%%%%%%%%%%%%%%%%%%
\section{Usage}

First of all, the package \textsf{childdoc} is \emph{not} a standard
\LaTeXe{} |.sty| style file! Therefore it needs to be invoked in
a non-standard way.

%%%%%%%%%%%%%%%%%%%%%%%%%%%%%%%%%%%%%%%%%%%%%%%%%%%%%%%%%%%%%%%%%%%%%%%%%%%%%%%%
\subsection{Included Files}
\label{sec:include}

%%%%%%%%%%%%%%%%%%%%%%%%%%%%%%%%%%%%%%%%
\DescribeMacro{\childdocmain}
To use the package, add the commands
\begin{center}
\begin{tabular}{l}
|% \iffalse
%
% childdoc.dtx Copyright (C) 2017-2018 Niklas Beisert
%
% This work may be distributed and/or modified under the
% conditions of the LaTeX Project Public License, either version 1.3
% of this license or (at your option) any later version.
% The latest version of this license is in
%   http://www.latex-project.org/lppl.txt
% and version 1.3 or later is part of all distributions of LaTeX
% version 2005/12/01 or later.
%
% This work has the LPPL maintenance status `maintained'.
%
% The Current Maintainer of this work is Niklas Beisert.
%
% This work consists of the files childdoc.dtx and childdoc.ins
% and the derived files childdoc.def and cdocsamp.tex with
% cdocsch1.tex, cdocsch2.tex, cdocsdrf.tex, cdocsfn1.tex, cdocsfn2.tex.
%
%<package>\ifdefined\childdocmain\endinput\fi
%<package>\ProvidesFile{childdoc.def}[2018/12/30 v2.0 child document driver]
%<samplemain>\ProvidesFile{cdocsamp.tex}[2018/12/30 v2.0 sample for childdoc]
%<*driver>
%\ProvidesFile{childdoc.drv}[2018/12/30 v2.0 childdoc reference manual file]
\PassOptionsToClass{10pt,a4paper}{article}
\documentclass{ltxdoc}

\usepackage[margin=35mm]{geometry}
\usepackage{hyperref}
\usepackage{hyperxmp}
\usepackage[usenames]{color}

\hypersetup{colorlinks=true}
\hypersetup{pdfstartview=FitH}
\hypersetup{pdfpagemode=UseNone}
\hypersetup{pdfsource={}}
\hypersetup{pdflang={en-UK}}
\hypersetup{pdfcopyright={Copyright 2017-2018 Niklas Beisert.
  This work may be distributed and/or modified under the
  conditions of the LaTeX Project Public License, either version 1.3
  of this license or (at your option) any later version.}}
\hypersetup{pdflicenseurl={http://www.latex-project.org/lppl.txt}}
\hypersetup{pdfcontactaddress={ETH Zurich, ITP, HIT K,
  Wolfgang-Pauli-Strasse 27}}
\hypersetup{pdfcontactpostcode={8093}}
\hypersetup{pdfcontactcity={Zurich}}
\hypersetup{pdfcontactcountry={Switzerland}}
\hypersetup{pdfcontactemail={nbeisert@itp.phys.ethz.ch}}
\hypersetup{pdfcontacturl={http://people.phys.ethz.ch/\xmptilde nbeisert/}}

\newcommand{\secref}[1]{\hyperref[#1]{section \ref*{#1}}}

\parskip1ex
\parindent0pt
\let\olditemize\itemize
\def\itemize{\olditemize\parskip0pt}

\begin{document}

\title{The \textsf{childdoc} Package}
\hypersetup{pdftitle={The childdoc Package}}
\author{Niklas Beisert\\[2ex]
  Institut f\"ur Theoretische Physik\\
  Eidgen\"ossische Technische Hochschule Z\"urich\\
  Wolfgang-Pauli-Strasse 27, 8093 Z\"urich, Switzerland\\[1ex]
  \href{mailto:nbeisert@itp.phys.ethz.ch}
  {\texttt{nbeisert@itp.phys.ethz.ch}}}
\hypersetup{pdfauthor={Niklas Beisert}}
\hypersetup{pdfsubject={Manual for the LaTeX2e Package childdoc}}
\date{30 December 2018, \textsf{v2.0}}
\maketitle

\begin{abstract}\noindent
\textsf{childdoc} is a \LaTeXe{} package
that enables the direct compilation
of document sections included by |\include|
to individual files.
\end{abstract}

\begingroup
\parskip0ex
\tableofcontents
\endgroup

%%%%%%%%%%%%%%%%%%%%%%%%%%%%%%%%%%%%%%%%%%%%%%%%%%%%%%%%%%%%%%%%%%%%%%%%%%%%%%%%
%%%%%%%%%%%%%%%%%%%%%%%%%%%%%%%%%%%%%%%%%%%%%%%%%%%%%%%%%%%%%%%%%%%%%%%%%%%%%%%%
\section{Introduction}

\LaTeX{} provides a mechanism to structure a large document (such as a book)
into a main file and several child files (containing the chapters)
using the |\include| command.
This mechanism is beneficial for documents
which span hundreds of pages in order to
make the source file(s) more manageable.
Moreover, compilation can be restricted to
selected child files by means of the |\includeonly| command.
The latter feature can be used to reduce the compilation time while editing
(this was significantly more useful in the earlier days of \LaTeX{})
or to generate a smaller document which is easier to navigate.
Another application of |\includeonly| is to generate
documents consisting of selected parts of the complete document.

However, there are a few drawbacks of the plain |\include| mechanism:
\begin{itemize}
\item
The child files cannot be compiled on their own,
they can only be compiled via the main file.
A naive editing environment
(such as a text editor with an option
to have the current file processed by \LaTeX)
may require one to switch to the main file before compiling;
attempting to compile the child file produces errors.
\item
The main file must be modified (each time)
to adjust the |\includeonly| command
to the present needs. This easily leaves the main file in a messy state.
\item
The generated document will always carry the filename
of the main document. This is inconvenient if
several child files are to be compiled and
to be kept for distribution.
\end{itemize}

The present package provides a simple interface
to make child files individually compilable by \LaTeX{}.
Compiling a child file then has the same effect as compiling
the main file with an |\includeonly| command
to select the appropriate child.
Moreover the generated document will carry the name of the child
rather than the main file.
This resolves all three above issues.

This feature is meant to make the editing of books,
thesis documents and lecture notes somewhat more convenient.
However, the package can also be used efficiently for
composing a series of documents (such as exercise sheets)
which are typically distributed individually.
It then assists the author in generating the individual documents
(potentially in different versions)
as well as a document containing the collected series.
Another application is in developing style files
or other kinds of included material
where compilation of the style file could redirect
to a sample or test file.

%%%%%%%%%%%%%%%%%%%%%%%%%%%%%%%%%%%%%%%%%%%%%%%%%%%%%%%%%%%%%%%%%%%%%%%%%%%%%%%%
%%%%%%%%%%%%%%%%%%%%%%%%%%%%%%%%%%%%%%%%%%%%%%%%%%%%%%%%%%%%%%%%%%%%%%%%%%%%%%%%
\section{Usage}

First of all, the package \textsf{childdoc} is \emph{not} a standard
\LaTeXe{} |.sty| style file! Therefore it needs to be invoked in
a non-standard way.

%%%%%%%%%%%%%%%%%%%%%%%%%%%%%%%%%%%%%%%%%%%%%%%%%%%%%%%%%%%%%%%%%%%%%%%%%%%%%%%%
\subsection{Included Files}
\label{sec:include}

%%%%%%%%%%%%%%%%%%%%%%%%%%%%%%%%%%%%%%%%
\DescribeMacro{\childdocmain}
To use the package, add the commands
\begin{center}
\begin{tabular}{l}
|\input{childdoc.def}|\\
|\childdocmain{}|\\
\end{tabular}
\end{center}
at the very top of the main \LaTeX{} file,
in particular \emph{before} the |\documentclass| statement!
The argument of |\childdocmain| should be left empty
(but it must be present).

%%%%%%%%%%%%%%%%%%%%%%%%%%%%%%%%%%%%%%%%
\DescribeMacro{\childdocof}
Furthermore, add the commands
\begin{center}
\begin{tabular}{l}
|\input{childdoc.def}|\\
|\childdocof{|\textit{main}|}|\\
\end{tabular}
\end{center}
at the top of every child file \textit{child}
which is included by |\include{|\textit{child}|}|
from within the main file
(or at least for those files to be compiled individually).
The argument \textit{main} must be the filename of the main file.

There are a couple of
considerations in setting up the main and child documents:

%%%%%%%%%%%%%%%%%%%%%%%%%%%%%%%%%%%%%%%%
\paragraph{Restrictions.}

Please note the following restrictions:
\begin{itemize}
\item
|\childdocmain| must be called with one argument \textit{main}
to ensure compatibility with earlier version of the package.
It must either be empty (|\childdocmain{}|)
or precisely match the filename of the main file in which it is specified.
See \secref{sec:detection} for further information.
\item
The filename \textit{main} must be specified without the |.tex| extension.
\item
The filename \textit{main} is case sensitive
(even in case-insensitive file systems)
due to internal string comparison.
\item
The argument \textit{main} should be fully expanded, it cannot be a macro.
\item
Subdirectories and special characters should be avoided in filenames.
\item
The command |\childdocmain{|\textit{main}|}| must be followed by a whitespace.
It should not be followed immediately by another command
or by a comment mark `|%|'.
This is because the \TeX{} parser reads the token immediately following
the argument of |\childdocmain| and puts it
at the beginning of every child section;
however, a white\-space is ignored.
\end{itemize}

%%%%%%%%%%%%%%%%%%%%%%%%%%%%%%%%%%%%%%%%
\paragraph{Content of Main File.}

It is advisable to place all content in the child files included by |\include|.
Any output contained in the main file will appear in all child documents
unless suppressed manually;
it cannot be suppressed automatically by the |\includeonly| directive
and thus should normally be avoided.
A method to include some content in the main file
by means of conditional processing is described in \secref{sec:conditional}.

%%%%%%%%%%%%%%%%%%%%%%%%%%%%%%%%%%%%%%%%
\paragraph{Page Numbering.}

When only a part of the document is compiled,
the appropriate numbering of pages
(as well as other status parameters)
is determined from the |.aux| files.
The latter contain information from previous passes.
However this information needs to propagate through
all intermediate child documents.
Therefore the page numbering in child documents may well
be inconsistent until the complete document is compiled at least once.

A useful (if unconventional) way to always ensure a consistent
page numbering is to restart the numbering in each child document
and denote the pages by `\textit{child}|.|\textit{page}'
where \textit{child} represents the chapter/section number of the child file.
This can be achieved by the command
|\numberwithin{page}{|\textit{child}|}|
of the \textsf{amsmath} package
where \textit{child} can be |chapter| or |section|
depending on the chosen structuring.
Alternatively, one can modify the macro |\thepage| appropriately
and reset the counter |page| at the start of each child file.

%%%%%%%%%%%%%%%%%%%%%%%%%%%%%%%%%%%%%%%%%%%%%%%%%%%%%%%%%%%%%%%%%%%%%%%%%%%%%%%%
\subsection{Conditional Processing}
\label{sec:conditional}

The package provides a mechanism to compile different versions
of a document. To customise the versions further some conditional processing
can come in handy to distinguish which version is being compiled.
The package provides two macros to describe the compilation context:

%%%%%%%%%%%%%%%%%%%%%%%%%%%%%%%%%%%%%%%%
\DescribeMacro{\ifchilddoc}
The conditional |\ifchilddoc| distinguishes between the compilation of
child documents and the main document:
%
\begin{center}
|\ifchilddoc |\textit{child-code}| |[|\||else |\textit{main-code}]| \||fi|
\end{center}

%%%%%%%%%%%%%%%%%%%%%%%%%%%%%%%%%%%%%%%%
\DescribeMacro{\childdocname}
\DescribeMacro{\childdocjob}
The macro |\childdocname| contains the filename (without extension)
of the main or child file being processed.
Note that |\childdocjob| will always contain the name of the main file.

%%%%%%%%%%%%%%%%%%%%%%%%%%%%%%%%%%%%%%%%
\paragraph{Title Page.}

Conditional processing can be used to include a title or banner page
in the main document when proper precautions are taken.
Importantly, the code in the main file should ensure that the page counter
(as well as other status parameters which are stored in the |.aux| files)
takes the same value after the conditional processing.
Otherwise the page numbers may take divergent values
depending on which part is compiled.

For example, a title page could be declared by:
%
\begin{center}
\begin{tabular}{l}
|\ifchilddoc\||else|\\
|\addtocounter{page}{-1}|\\
\textit{code for title page}\\
|\newpage|\\
|\||fi|
\end{tabular}
\end{center}
%
A banner page for the child documents can be generated by:
%
\begin{center}
\begin{tabular}{l}
|\ifchilddoc|\\
|\addtocounter{page}{-1}|\\
\textit{code for banner page}\\
|\newpage|\\
|\||fi|
\end{tabular}
\end{center}
%
Here one could write a message such as:
\begin{center}
|This is the part \childdocname{} of \childdocjob{}.|
\end{center}

%%%%%%%%%%%%%%%%%%%%%%%%%%%%%%%%%%%%%%%%%%%%%%%%%%%%%%%%%%%%%%%%%%%%%%%%%%%%%%%%
\subsection{Flags}
\label{sec:flags}

The package makes it easy to generate different versions
of the main or child documents.
To this end compilation flags can be defined
and assigned different default values.
They will be particularly useful in conjunction
with the forwarding mechanism described in \secref{sec:forward}.

For example, it may be useful to have a flag |\version|
which can be set to |draft| or |final|.
The document source will contain some conditional code
depending on the value of |\version|.
Suppose further, the flag should default to |final| for the main file
and to |draft| for child files
which is a natural assignment for editing the document.
This is achieved by placing the following code
in the preamble of the main document
(below the |\childdocmain| directive):
%
\begin{center}
\begin{tabular}{l}
|\ifchilddoc|\\
|\providecommand{\version}{draft}|\\
|\||else|\\
|\providecommand{\version}{final}|\\
|\||fi|
\end{tabular}
\end{center}
%
The definition by |\providecommand| makes sure
that previous definitions are not overwritten.
Further statements |\providecommand{\version}{...}|
can thus be added before the above code to override it.

For the main file, one might add a line
(between |\childdocmain| and the above block)
%
\begin{center}
|%\ifchilddoc\||else\providecommand{\version}{draft}\||fi|
\end{center}
%
which can be uncommented to produce a draft version.
Likewise one can add a line to the very top of a child file
(above the |\childdocof{|\textit{main}|}| directive)
%
\begin{center}
|%\providecommand{\version}{final}|
\end{center}
%
which can be uncommented to produce the final version of this child document.

%%%%%%%%%%%%%%%%%%%%%%%%%%%%%%%%%%%%%%%%%%%%%%%%%%%%%%%%%%%%%%%%%%%%%%%%%%%%%%%%
\subsection{Forwarding}
\label{sec:forward}

Different versions of the main or child documents
using compilation flags as described in \secref{sec:flags}
can be (permanently) stored in different files
for convenient compilation, viewing and distribution.
To this end, the package defines a command
to pass on compilation to a different file:

%%%%%%%%%%%%%%%%%%%%%%%%%%%%%%%%%%%%%%%%
\DescribeMacro{\childdocforward}
The command |\childdocforward| redirects processing to
another source file:
%
\begin{center}
\begin{tabular}{l}
|\input{childdoc.def}|\\
|\childdocforward[|\textit{main}|]{|\textit{dest}|}|\\
\end{tabular}
\end{center}
%
The argument \textit{dest} is the destination file
(without extension).
It should be the main file or one of the child files.
Note that further \textsf{childdoc} directives
such as |\childdocof| and |\childdocforward|
in the indicated file will be processed in this form.
The optional argument \textit{main}
passes on directly to the main file \textit{main}
while pretending to compile the child \textit{dest}.
This form behaves as if \textit{dest}
issues |\childdocof{|\textit{main}|}| right away,
and no further \textsf{childdoc} directives will be processed.

%%%%%%%%%%%%%%%%%%%%%%%%%%%%%%%%%%%%%%%%
\DescribeMacro{\...prefix}
In the alternative form |\childdocforwardprefix|,
%
\begin{center}
\begin{tabular}{l}
|\input{childdoc.def}|\\
|\childdocforwardprefix[|\textit{main}|]{|\textit{prefix}|}{|\textit{dest}|}|
\end{tabular}
\end{center}
%
the destination file is determined by a pattern
depending on the current file:
To make this work, the current file must be called
`{\textit{prefix}\hspace{0.2em}\textit{suffix}}'
with \textit{prefix} matching precisely the argument.
Processing is then passed on to the file
`{\textit{dest}\hspace{0.2em}\textit{suffix}}'.
Surely, the same effect is achieved by
directly specifying the
argument `{\textit{dest}\hspace{0.2em}\textit{suffix}}'
in the first form.
However, that requires to set up a different file
for each child. With the alternative form of the command
all these files can have exactly the same content
which simplifies setting them up and maintaining them.

For example, the following file |draft.tex|
with a compilation flag |\version| as described in \secref{sec:flags}
compiles the main document as a draft:
%
\begin{center}
\begin{tabular}{l}
|\def\version{draft}|\\
|\input{childdoc.def}|\\
|\childdocforward{|\textit{main}|}|
\end{tabular}
\end{center}
%
Likewise, the following files |final|\textit{nn}|.tex|
compile the final version of the child document
|child|\textit{nn}|.tex|:
%
\begin{center}
\begin{tabular}{l}
|\def\version{final}|\\
|\input{childdoc.def}|\\
|\childdocforwardprefix{final}{child}|
\end{tabular}
\end{center}
%

Note that when several versions of a main file and/or of each child file
are to be generated, it may be convenient to set up a |Makefile| or
shell script to automatise the process.

%%%%%%%%%%%%%%%%%%%%%%%%%%%%%%%%%%%%%%%%%%%%%%%%%%%%%%%%%%%%%%%%%%%%%%%%%%%%%%%%
\subsection{Command Line Processing}
\label{sec:commandline}

The effect of redirection files can also be achieved by invoking
the \LaTeX{} compiler with a more elaborate command line.
Most conveniently this should be done as part
of a shell script or a |Makefile|.

When using \textsf{childdoc} in the main file, the following
command lines effectively perform a redirection
(note that depending on the shell being used,
backslashes may have to be doubled: `|\|' $\to$ `|\\|'):
%
\begin{center}
|... -jobname "|\textit{target}|" |\\|"|[\textit{flags}]%
|\input{childdoc.def}\childdocforward[|\textit{main}|]{|\textit{dest}|}"|
\end{center}
%
Here \textit{target} is the name of the output file,
\textit{main} is the name of the main file
and \textit{dest} is the name of the main or child file to be processed
(all filenames without extensions).
The optional argument \textit{main} can be omitted
if \textit{main} matches \textit{dest}.
Optionally, compilation \textit{flags} can be defined via |\def| commands.
This command line makes the \TeX{} engine believe
it is compiling the file \textit{target}
whose content is specified as the latter parameter.
The provided code then forwards the processing to
\textit{main} or \textit{dest} as described in \secref{sec:forward}.

%%%%%%%%%%%%%%%%%%%%%%%%%%%%%%%%%%%%%%%%%%%%%%%%%%%%%%%%%%%%%%%%%%%%%%%%%%%%%%%%
\subsection{Include by Input}
\label{sec:input}

Including child documents by |\include| has some restrictions by design.
Most notably, the content of a child document always occupies
its own set of pages; pages cannot be shared between child documents.
Usually, this behaviour makes perfect sense
because each child document contain an essential part of the document.
However, in some situations it may be desirable to compose
a document from a collection of parts
without having mandatory page breaks between then.
For this case, the package
provides a mechanism to include parts
by |\input| which can also be processed individually.
However, by construction this mechanism
requires manual handling of the content to be output.

%%%%%%%%%%%%%%%%%%%%%%%%%%%%%%%%%%%%%%%%
\DescribeMacro{\ifchilddocmanual}
The main file should be prepared as usual, see \secref{sec:include}.
However, the document body must make a distinction
between processing of an individual part and of the main document, e.g.:
%
\begin{center}
\begin{tabular}{l}
|\ifchilddocmanual|\\
|\input{\childdocname}|\\
|\||else|\\
\textit{document body with }|\input{|\textit{part}|}|\\
|\||fi|
\end{tabular}
\end{center}
%
The conditional |\ifchilddocmanual| is true whenever
a part to be included by |\input| is being compiled,
and the name of the part is stored in |\childdocname|.

%%%%%%%%%%%%%%%%%%%%%%%%%%%%%%%%%%%%%%%%
\DescribeMacro{\childdocby}
Each part to be included by |\input| should start with:
%
\begin{center}
\begin{tabular}{l}
|\input{childdoc.def}|\\
|\childdocby{|\textit{main}|}|\\
\end{tabular}
\end{center}
%
The directive |\childdocby| is similar to |\childdocof|
described in \secref{sec:include},
but the subsequent selection of content must be done manually.
To that end, both |\ifchilddoc| and |\ifchilddocmanual|
will be true upon processing of a part,
and the name of the part is stored in |\childdocname|.
Note that |\jobname| will be set to the filename of the current part
so that each part receives an individual |.aux| file
that does not interfere with the |.aux| file(s) of the main document.
This behaviour can be altered by the alternative form
|\childdocby[*]{|\textit{main}|}| (with a non-empty optional argument)
which uses the |.aux| file of the main document
by setting |\jobname| to \textit{main}.

%%%%%%%%%%%%%%%%%%%%%%%%%%%%%%%%%%%%%%%%%%%%%%%%%%%%%%%%%%%%%%%%%%%%%%%%%%%%%%%%
\subsection{Driver Development}
\label{sec:driver}

The \textsf{childdoc} mechanism can also be use for the development
of definition files such as \LaTeX{} styles or classes.
This case differs from the above setup with multiple parts
included by |\include| in that no |\includeonly| should be invoked.
This can be achieved by starting the include file
(before |\ProvidesPackage|) with:
%
\begin{center}
\begin{tabular}{l}
|\input{childdoc.def}|\\
|\childdocforward{|\textit{main}|}|\\
\end{tabular}
\end{center}
%
or alternatively with:
%
\begin{center}
\begin{tabular}{l}
|\input{childdoc.def}|\\
|\childdocby{|\textit{main}|}|\\
\end{tabular}
\end{center}
%
Both forms have slightly different effects as described above.
The main file is prepared as usual, see \secref{sec:include}.

%%%%%%%%%%%%%%%%%%%%%%%%%%%%%%%%%%%%%%%%%%%%%%%%%%%%%%%%%%%%%%%%%%%%%%%%%%%%%%%%
\subsection{Legacy Detection}
\label{sec:detection}

The directive |\childdocmain| in the main file can detect
whether the complete document or merely a child is to be compiled
even without using the directive |\childdocof|.
This method is deprecated because it is less robust
and there is no compelling reason to use it;
it is merely provided for backward compatibility
and it may be removed in future versions.

If the detection mechanism is to be used,
it is mandatory to correctly specify
the filename of the main file as the argument of |\childdocmain|:
%
\begin{center}
\begin{tabular}{l}
|\input{childdoc.def}|\\
|\childdocmain{|\textit{main}|}|\\
\end{tabular}
\end{center}
%
If |\jobname| does not match the argument \textit{main} of |\childdocmain|,
it is assumed that |\jobname| points to the child file to be compiled.
When using |\childdocmain| with the main file specified as argument,
it suffices to start a child file
with just |\input{|\textit{main}|}|
without loading of the package and using |\childdocof|.
If instead all processing is done
with the appropriate \textsf{childdoc} directives,
the argument of \textit{main} of |\childdocmain| can be empty.

An alternative version of the command line processing described
in \secref{sec:commandline} using the detection mechanism reads:
%
\begin{center}
|... -jobname "|\textit{target}|" "|[\textit{flags}]%
[|\def\jobname{|\textit{dest}|}|]|\input{|\textit{main}|}"|
\end{center}

%%%%%%%%%%%%%%%%%%%%%%%%%%%%%%%%%%%%%%%%%%%%%%%%%%%%%%%%%%%%%%%%%%%%%%%%%%%%%%%%
\subsection{Manual Code}
\label{sec:manual}

In case one cannot be certain whether the definitions file |childdoc.def|
is installed on the target \TeX{} distribution
and one prefers not to ship it,
it is conceivable to paste a few relevant commands into the sources.

To that end, drop all statements |\input{childdoc.def}|
and perform the replacements as outlined below.
Instead of |\childdocmain{|\textit{main}|}| add the following code
to the top of the main file:
%
\begin{center}
\begin{tabular}{l}
|\||ifdefined\childdocname\endinput\||fi\newif\ifchilddoc|\\
|\edef\childdocname{\scantokens\expandafter{\jobname\noexpand}}|\\
|\def\childdocmain{|\textit{main}|}\||ifx\childdocmain\childdocname\||else|\\
|\childdoctrue\includeonly{\childdocname}\let\jobname\childdocmain\||fi|\\
\end{tabular}
\end{center}
%
Instead of |\childdocof{|\textit{main}|}| just include the main file
at the top of each child file:
%
\begin{center}
|\input{|\textit{main}|}|
\end{center}
%
A simple redirection |\childdocforward{|\textit{dest}|}| is achieved by:
%
\begin{center}
|\def\jobname{|\textit{dest}|}\input{\jobname}|
\end{center}
%
The redirection with prefix
|\childdocforwardprefix[|\textit{prefix}|]{|\textit{dest}|}|
is accomplished by:
%
\begin{center}
\begin{tabular}{l}
|{\edef\jobname{\scantokens\expandafter{\jobname\noexpand}}|\\
|\def\redirectjob |\textit{prefix}|#1~~~{\gdef\jobname{|\textit{dest}|#1}}|\\
|\expandafter\redirectjob\jobname~~~}\input{\jobname}|
\end{tabular}
\end{center}

In an alternative approach,
child documents can be compiled by a specific command line
without additional code or specific definitions:
%
\begin{center}
|... -jobname "|\textit{target}|" "|[\textit{flags}]%
|\includeonly{|\textit{dest}|}\input{|\textit{main}|}"|
\end{center}
%

%%%%%%%%%%%%%%%%%%%%%%%%%%%%%%%%%%%%%%%%%%%%%%%%%%%%%%%%%%%%%%%%%%%%%%%%%%%%%%%%
%%%%%%%%%%%%%%%%%%%%%%%%%%%%%%%%%%%%%%%%%%%%%%%%%%%%%%%%%%%%%%%%%%%%%%%%%%%%%%%%
\section{Information}

%%%%%%%%%%%%%%%%%%%%%%%%%%%%%%%%%%%%%%%%%%%%%%%%%%%%%%%%%%%%%%%%%%%%%%%%%%%%%%%%
\subsection{Copyright}

Copyright \copyright{} 2017--2018 Niklas Beisert

This work may be distributed and/or modified under the
conditions of the \LaTeX{} Project Public License, either version 1.3
of this license or (at your option) any later version.
The latest version of this license is in
  \url{http://www.latex-project.org/lppl.txt}
and version 1.3 or later is part of all distributions of \LaTeX{}
version 2005/12/01 or later.

This work has the LPPL maintenance status `maintained'.

The Current Maintainer of this work is Niklas Beisert.

This work consists of the files |README.txt|, |childdoc.ins| and |childdoc.dtx|
as well as the derived files |childdoc.def|, |cdocsamp.tex|
with |cdocsch1.tex|, |cdocsch2.tex|, |cdocspt3.tex|, |cdocspt4.tex|,
|cdocsdrf.tex|, |cdocsfn1.tex|, |cdocsfn2.tex|
as well as |childdoc.pdf|.

%%%%%%%%%%%%%%%%%%%%%%%%%%%%%%%%%%%%%%%%%%%%%%%%%%%%%%%%%%%%%%%%%%%%%%%%%%%%%%%%
\subsection{Files and Installation}

The package consists of the files:
%
\begin{center}
\begin{tabular}{ll}
    |README.txt|   & readme file \\
    |childdoc.ins| & installation file \\
    |childdoc.dtx| & source file \\
    |childdoc.def| & definition file \\
    |cdocsamp.tex| & sample main file \\
    |cdocsch1.tex| & sample include file \\
    |cdocsch2.tex| & sample include file \\
    |cdocspt3.tex| & sample part file \\
    |cdocspt4.tex| & sample part file \\
    |cdocsdrf.tex| & sample redirection file \\
    |cdocsfn1.tex| & sample redirection file \\
    |cdocsfn2.tex| & sample redirection file \\
    |childdoc.pdf| & manual
\end{tabular}
\end{center}
%
The distribution consists of the files
|README.txt|, |childdoc.ins| and |childdoc.dtx|.
%
\begin{itemize}
\item
Run (pdf)\LaTeX{} on |childdoc.dtx|
to compile the manual |childdoc.pdf| (this file).
\item
Run \LaTeX{} on |childdoc.ins| to create the definitions file |childdoc.def|
and the sample |cdocsamp.tex| with include files
|cdocsch1.tex|, |cdocsch2.tex|, |cdocspt3.tex|, |cdocspt4.tex|,
|cdocsdrf.tex|, |cdocsfn1.tex|, |cdocsfn2.tex|.
Then copy the file |childdoc.def| to an appropriate directory of your \LaTeX{}
distribution, e.g.\ \textit{texmf-root}|/tex/latex/childdoc|.
\end{itemize}

%%%%%%%%%%%%%%%%%%%%%%%%%%%%%%%%%%%%%%%%%%%%%%%%%%%%%%%%%%%%%%%%%%%%%%%%%%%%%%%%
\subsection{Related CTAN Packages}

There are several other packages which offer a similar functionality:
%
\begin{itemize}
\item
The packages
\href{http://ctan.org/pkg/docmute}{\textsf{docmute}},
\href{http://ctan.org/pkg/includex}{\textsf{includex}} and
\href{http://ctan.org/pkg/standalone}{\textsf{standalone}}
provide commands to include only the document body of
a child file thus allowing both files to be compiled individually.
\item
The packages \href{http://ctan.org/pkg/subdocs}{\textsf{subdocs}}
and \href{http://ctan.org/pkg/subfiles}{\textsf{subfiles}}
provide structures in which the main and child documents can be
encapsulated and allowing them to be compiled individually.
The inclusion mechanism is different from the conventional |\include|.
\item
The package \href{http://ctan.org/pkg/combine}{\textsf{combine}}
is an elaborate solution to combine several documents into one.
\end{itemize}
%
See also the CTAN topic \href{http://ctan.org/topic/subdocs}{\textsf{subdocs}}
for further related packages.
The present package differs from the above solutions in that
a document structure constructed with the conventional |\include| mechanism
just needs two extra commands at the top of every file
such that all constituent files can be compiled individually.

%%%%%%%%%%%%%%%%%%%%%%%%%%%%%%%%%%%%%%%%%%%%%%%%%%%%%%%%%%%%%%%%%%%%%%%%%%%%%%%%
%\subsection{Feature Suggestions}
%
%The following is a list of features which may be useful for future
%versions of this package:
%%
%\begin{itemize}
%\item
%\ldots
%\end{itemize}

%%%%%%%%%%%%%%%%%%%%%%%%%%%%%%%%%%%%%%%%%%%%%%%%%%%%%%%%%%%%%%%%%%%%%%%%%%%%%%%%
\subsection{Revision History}

%%%%%%%%%%%%%%%%%%%%%%%%%%%%%%%%%%%%%%%%
\paragraph{v2.0:} 2018/12/30

\begin{itemize}
\item
immediate forward processing
\item
added |\childdocby| mechanism
\item
manual restructured
\end{itemize}

%%%%%%%%%%%%%%%%%%%%%%%%%%%%%%%%%%%%%%%%
\paragraph{v1.6:} 2018/01/17

\begin{itemize}
\item
application for development of include files
\item
corrections to manual
\end{itemize}

%%%%%%%%%%%%%%%%%%%%%%%%%%%%%%%%%%%%%%%%
\paragraph{v1.5:} 2017/05/21

\begin{itemize}
\item
more complete structuring introduced
\item
|\childdocof| introduced
\item
|\childdoc| renamed to |\childdocmain|
\item
|\childredirect| renamed to |\childdocforward| and |\childdocforwardprefix|
and functionality expanded
\end{itemize}

%%%%%%%%%%%%%%%%%%%%%%%%%%%%%%%%%%%%%%%%
\paragraph{v1.0:} 2017/04/27

\begin{itemize}
\item
manual and install package
\item
first version published on CTAN
\end{itemize}

%%%%%%%%%%%%%%%%%%%%%%%%%%%%%%%%%%%%%%%%
\paragraph{v0.6:} 2017/04/26

\begin{itemize}
\item
redirection mechanism added
\end{itemize}

%%%%%%%%%%%%%%%%%%%%%%%%%%%%%%%%%%%%%%%%
\paragraph{v0.5:} 2017/04/26

\begin{itemize}
\item
functionality in definition file
\end{itemize}


%%%%%%%%%%%%%%%%%%%%%%%%%%%%%%%%%%%%%%%%%%%%%%%%%%%%%%%%%%%%%%%%%%%%%%%%%%%%%%%%
%%%%%%%%%%%%%%%%%%%%%%%%%%%%%%%%%%%%%%%%%%%%%%%%%%%%%%%%%%%%%%%%%%%%%%%%%%%%%%%%
%%%%%%%%%%%%%%%%%%%%%%%%%%%%%%%%%%%%%%%%%%%%%%%%%%%%%%%%%%%%%%%%%%%%%%%%%%%%%%%%
\appendix

\settowidth\MacroIndent{\rmfamily\scriptsize 000\ }

 \DocInput{childdoc.dtx}

\end{document}
%</driver>
% \fi
%
% %%%%%%%%%%%%%%%%%%%%%%%%%%%%%%%%%%%%%%%%%%%%%%%%%%%%%%%%%%%%%%%%%%%%%%%%%%%%%%
% %%%%%%%%%%%%%%%%%%%%%%%%%%%%%%%%%%%%%%%%%%%%%%%%%%%%%%%%%%%%%%%%%%%%%%%%%%%%%%
% \section{Sample}
%\iffalse
%<*samplemain>
%\fi
%
% The following presents a sample document
% with two chapters, two parts, a title page,
% a compile flag as well as three forwarding files to set the flag.
% It consists of eight |.tex| files:
% \begin{center}
% \begin{tabular}{ll}
% |cdocsamp.tex|&main file\\
% |cdocsch1.tex|&include file for chapter 1\\
% |cdocsch2.tex|&include file for chapter 2\\
% |cdocspt3.tex|&include file for part 3\\
% |cdocspt4.tex|&include file for part 4\\
% |cdocsdrf.tex|&forwarding file for main file in draft mode\\
% |cdocsfi1.tex|&forwarding file for final version of chapter 1\\
% |cdocsfi2.tex|&forwarding file for final version of chapter 2\\
% \end{tabular}
% \end{center}
% Each of the eight files can be compiled directly by the \LaTeX{} compiler.
%
% %%%%%%%%%%%%%%%%%%%%%%%%%%%%%%%%%%%%%%
% \paragraph{Main File.}
%
% The main file is called |cdocsamp.tex|.
%
% Load the \textsf{childdoc} definitions and
% declare the filename for the main document:
%    \begin{macrocode}
\input{childdoc.def}
\childdocmain{}
%    \end{macrocode}

% Optional override for |\version| flag:
%    \begin{macrocode}
%%\ifchilddoc\else\providecommand{\version}{draft}\fi
%    \end{macrocode}

% Define the default values for the |\version| flag
% (|final| for the main file and |draft| for childs):
%    \begin{macrocode}
\ifchilddoc
\providecommand{\version}{draft}
\else
\providecommand{\version}{final}
\fi
%    \end{macrocode}

% Load the standard document class:
%    \begin{macrocode}
\documentclass[12pt]{article}
%    \end{macrocode}

% Start the document body:
%    \begin{macrocode}
\begin{document}
%    \end{macrocode}

% Declare a title page.
% Print title, part of document being processed and version flag:
%    \begin{macrocode}
\addtocounter{page}{-1}
\begin{center}
{\LARGE\bfseries{}childdoc example\par}
\vspace{1cm}
\ifchilddoc
\ifchilddocmanual part\else chapter\fi:
`\childdocname' of `\childdocjob'\par
\else
main document: `\childdocjob'\par
\fi
version: \version\par
\end{center}
\newpage
%    \end{macrocode}

% Manually include selected file,
% otherwise process as usual:
%    \begin{macrocode}
\ifchilddocmanual
\section*{part `\childdocname'}
\input{\childdocname}
\else
%    \end{macrocode}

% Include the two chapters:
%    \begin{macrocode}
\include{cdocsch1}
\include{cdocsch2}
%    \end{macrocode}

% Include the two parts unless only chapters should be displayed:
%    \begin{macrocode}
\ifchilddoc\else
\section{part three}
\input{cdocspt3}
\section{part four}
\input{cdocspt4}
\fi
%    \end{macrocode}

% Process as usual until here:
%    \begin{macrocode}
\fi
%    \end{macrocode}

% End of document body:
%    \begin{macrocode}
\end{document}
%    \end{macrocode}
%\iffalse
%</samplemain>
%\fi
%
% %%%%%%%%%%%%%%%%%%%%%%%%%%%%%%%%%%%%%%
% \paragraph{Chapter Include Files.}
%
% The include files are called |cdocsch1.tex| and |cdocsch2.tex|.
%
%\iffalse
%<*samplechap1|samplechap2>
%\fi

% Optional override for |\version| flag:
%    \begin{macrocode}
%%\providecommand{\version}{final}
%    \end{macrocode}

% Include the main document:
%    \begin{macrocode}
\input{childdoc.def}
\childdocof{cdocsamp}
%    \end{macrocode}

%\iffalse
%</samplechap1|samplechap2>
%\fi
%
%\iffalse
%<*samplechap1>
%\fi
% Some text for chapter 1:
%    \begin{macrocode}
\section{one}
some text in chapter one
%    \end{macrocode}

%\iffalse
%</samplechap1>
%\fi
% Some text for chapter 2:
%\iffalse
%<*samplechap2>
%\fi
%    \begin{macrocode}
\section{two}
more text in chapter two
%    \end{macrocode}

%\iffalse
%</samplechap2>
%\fi
%
% %%%%%%%%%%%%%%%%%%%%%%%%%%%%%%%%%%%%%%
% \paragraph{Part Include Files.}
%
% The include files are called |cdocspt3.tex| and |cdocspt4.tex|.
%
%\iffalse
%<*samplepart3|samplepart4>
%\fi

% Optional override for |\version| flag:
%    \begin{macrocode}
%%\providecommand{\version}{final}
%    \end{macrocode}

% Include the main document:
%    \begin{macrocode}
\input{childdoc.def}
\childdocby{cdocsamp}
%    \end{macrocode}

%\iffalse
%</samplepart3|samplepart4>
%\fi
%
%\iffalse
%<*samplepart3>
%\fi
% Some text for part 3:
%    \begin{macrocode}
some text in part three
%    \end{macrocode}

%\iffalse
%</samplepart3>
%\fi
% Some text for part 4:
%\iffalse
%<*samplepart4>
%\fi
%    \begin{macrocode}
more text in part four
%    \end{macrocode}

%\iffalse
%</samplepart4>
%\fi
%
% %%%%%%%%%%%%%%%%%%%%%%%%%%%%%%%%%%%%%%
% \paragraph{Forwarding for a Complete Draft.}
%
% The following forwarding file |cdocsdrf.tex|
% compiles the main document in draft mode:
%\iffalse
%<*sampledraft>
%\fi
%    \begin{macrocode}
\def\version{draft}
\input{childdoc.def}
\childdocforward{cdocsamp}
%    \end{macrocode}

%\iffalse
%</sampledraft>
%\fi
%
% %%%%%%%%%%%%%%%%%%%%%%%%%%%%%%%%%%%%%%
% \paragraph{Forwarding for Final Version of the Chapters.}
%
% The following forwarding files |cdocsfn1.tex| and |cdocsfn2.tex|
% (with identical content)
% compile the final versions of the child documents
% |cdocsch1.tex| and |cdocsch2.tex|, respectively:
%\iffalse
%<*samplefinal>
%\fi
%    \begin{macrocode}
\def\version{final}
\input{childdoc.def}
\childdocforwardprefix[cdocsamp]{cdocsfn}{cdocsch}
%    \end{macrocode}

%\iffalse
%</samplefinal>
%\fi
%
% %%%%%%%%%%%%%%%%%%%%%%%%%%%%%%%%%%%%%%
% \paragraph{Command Line Processing.}
%
% The following three command lines generate the output files
% |cdocscld|, |cdocscl1| and |cdocscl2|
% which should be identical to
% |cdocsdrf|, |cdocsch1| and |cdocsfn2|, respectively:
% \begin{center}
% \begin{tabular}{l}
% |latex -jobname cdocscld \|\\
% |  "\def\version{draft}\input{childdoc.def}\childdocforward{cdocsamp}"|\\
% |latex -jobname cdocscl1 \|\\
% |  "\input{childdoc.def}\childdocforward[cdocsamp]{cdocsch1}"|\\
% |latex -jobname cdocscl2 \|\\
% |  "\def\version{final}\input{childdoc.def}\childdocforward{cdocsch2}"|
% \end{tabular}
% \end{center}
% Note that the trailing backslash on each first line
% merely continues the input to the second line
% (for convenient cut ant paste).
% Furthermore, the command |latex| can be replaced by any
% of its alternative versions such as |pdflatex|.
%
% %%%%%%%%%%%%%%%%%%%%%%%%%%%%%%%%%%%%%%%%%%%%%%%%%%%%%%%%%%%%%%%%%%%%%%%%%%%%%%
% %%%%%%%%%%%%%%%%%%%%%%%%%%%%%%%%%%%%%%%%%%%%%%%%%%%%%%%%%%%%%%%%%%%%%%%%%%%%%%
% \section{Implementation}
%\iffalse
%<*package>
%\fi
%
% This section describes the definitions file |childdoc.def|.

% The definitions cannot be loaded using |\usepackage| or |\RequirePackage|
% which has a mechanism to prevent loading a style file more than once.
% When loading the definitions by means of |\input|
% multiple instances have to be prevented manually:
%\iffalse
%This code needs to be before the `\ProvidesFile' directive
%which is defined at the beginning of this file.
%Therefore it is also placed there and commented out here.
%</package>
%<*discard>
%\fi
%    \begin{macrocode}
\ifdefined\childdocmain\endinput\fi
%    \end{macrocode}
%\iffalse
%</discard>
%<*package>
%\fi
%
% \macro{\ifchilddoc}
% \macro{\ifchilddocmanual}
% The conditional |\ifchilddoc| tells whether a
% child (true) or main (false) document is being compiled.
% The conditional |\ifchilddocmanual| tells whether
% the |\includeonly| mechanism is used (false) or
% the selection of child files must be performed manually (true).
% The definitions initialise to false:
%    \begin{macrocode}
\newif\ifchilddoc
\newif\ifchilddocmanual
%    \end{macrocode}

% \macro{\childdocname}
% \macro{\childdocjob}
% The macro |\childdocname| stores the name of the main document
% to be compiled. The macro |\childdocjob| stores the name of
% the document on which the \LaTeX{} compiler was originally invoked.
% The content of |\jobname| cannot be compared
% to filenames specified in the source due to different catcodes.
% The following code rescans |\jobname|, stores the result
% in |\childdocname| and saves a copy in |\childdocjob|:
%    \begin{macrocode}
\edef\childdocname{\scantokens\expandafter{\jobname\noexpand}}
\let\childdocjob\childdocname
%    \end{macrocode}

% \macro{\childdocdisable}
% The macro |\childdocdisable| prevents the main file
% from being processed more than once.
% At this stage, the main document command |\childdocmain|
% is assumed to be called once again where it should do nothing.
% Any subsequent call to it should prevent
% a secondary processing of the main document
% It overwrites the forwarding commands
% |\childdocof| and |\childdocforward|
% with empty macros to prevent further inclusions of the main document:
%    \begin{macrocode}
\newcommand{\childdocdisable}
{
  \renewcommand{\childdocmain}[1]{\renewcommand{\childdocmain}[1]{\endinput}}
  \renewcommand{\childdocof}[1]{}
  \renewcommand{\childdocby}[2][]{}
  \renewcommand{\childdocforward}[2][]{}
  \renewcommand{\childdocdisable}{}
}
%    \end{macrocode}

% \macro{\childdocmain}
% The macro |\childdocmain| is to be called at the top of the main file
% with nothing or the main filename (without extension) as argument.
% First, it breaks loops.
% If the argument is not empty and does not match |\childdocname|
% (which is set by the first inclusion of |childdoc.def|),
% |\ifchilddoc| is set to true, |\includeonly| is applied to the child file
% and |\jobname| is set to the main file
% (for proper handling of |.aux| files):
%    \begin{macrocode}
\newcommand{\childdocmain}[1]
{
  \childdocdisable\childdocmain{}
  \if?#1?\else
    \begingroup
      \def\childdoctmp{#1}
      \ifx\childdoctmp\childdocname
        \def\childdoctmp{}
      \else
        \def\childdoctmp
        {
          \childdoctrue
          \includeonly{\childdocname}
          \def\childdocjob{#1}
          \def\jobname{#1}
        }
      \fi
      \expandafter
    \endgroup
    \childdoctmp
  \fi
}
%    \end{macrocode}

% \macro{\childdocof}
% The command |\childdocof| redirects
% compilation to the main file |#1|.
%    \begin{macrocode}
\newcommand{\childdocof}[1]
{
  \childdocdisable
  \childdoctrue
  \includeonly{\childdocname}
  \def\jobname{#1}
  \def\childdocjob{#1}
  \input{#1}
}
%    \end{macrocode}

% \macro{\childdocby}
% The command |\childdocby| ....
%    \begin{macrocode}
\newcommand{\childdocby}[2][]
{
  \childdocdisable
  \childdoctrue
  \childdocmanualtrue
  \if?#1?\else
    \def\jobname{#2}
  \fi
  \def\childdocjob{#2}
  \input{#2}
  \endinput
}
%    \end{macrocode}

% \macro{\childdocforward}
% The command |\childdocforward| redirects
% compilation to the main file or
% (if the optional argument is given) a child file.
% Parameters are set as if the main file
% or a child file starting with |\childdocof| was compiled.
% Then compilation is handed over to the main file:
%    \begin{macrocode}
\newcommand{\childdocforward}[2][]
{
  \begingroup
    \if?#1?
      \def\childdoctmp
      {
        \def\childdocname{#2}
        \def\childdocjob{#2}
        \def\jobname{#2}
        \input{#2}
        \endinput
      }
    \else
      \def\childdoctmp
      {
        \childdocdisable
        \def\childdocname{#2}
        \childdoctrue
        \includeonly{#2}
        \def\childdocjob{#1}
        \def\jobname{#1}
        \input{#1}
        \endinput
      }
    \fi
    \expandafter
  \endgroup
  \childdoctmp
}
%    \end{macrocode}

% \macro{\childdocforwardprefix}
% The command |\childdocforwardprefix| redirects
% compilation to the main or a child file by means of a pattern.
% The prefix |#1| in the current filename is replaced by |#2|
% and the suffix of the current filename is kept
% (it is assumed that the filename does not contain the substring `|~~~|'
% which is used as a delimiter).
% Compilation is handed over to the new file by |\childdocforward|:
%    \begin{macrocode}
\newcommand{\childdocforwardprefix}[3][]
{
  \begingroup
    \def\childdocextract #2##1~~~{\def\childdoctmp{\childdocforward[#1]{#3##1}}}
    \expandafter\childdocextract\childdocname~~~
    \expandafter
  \endgroup
  \childdoctmp
}
%    \end{macrocode}

% \macro{\childdoc}
% The deprecated macro |\childdoc| is a legacy version of |\childdocmain|:
%    \begin{macrocode}
\newcommand{\childdoc}{\childdocmain}
%    \end{macrocode}

% \macro{\childdocredirect}
% The deprecated macro |\childdocredirect| is a legacy version
% of |\childdocforward| and |\childdocforwardprefix|:
%    \begin{macrocode}
\newcommand{\childdocredirect}[2][]
{
  \begingroup
    \if?#1?
      \def\childdoctmp{\childdocforward{#2}}
    \else
      \def\childdoctmp{\childdocforwardprefix{#1}{#2}}
    \fi
    \expandafter
  \endgroup
  \childdoctmp
}
%    \end{macrocode}

%\iffalse
%</package>
%\fi
%
\endinput
|\\
|\childdocmain{}|\\
\end{tabular}
\end{center}
at the very top of the main \LaTeX{} file,
in particular \emph{before} the |\documentclass| statement!
The argument of |\childdocmain| should be left empty
(but it must be present).

%%%%%%%%%%%%%%%%%%%%%%%%%%%%%%%%%%%%%%%%
\DescribeMacro{\childdocof}
Furthermore, add the commands
\begin{center}
\begin{tabular}{l}
|% \iffalse
%
% childdoc.dtx Copyright (C) 2017-2018 Niklas Beisert
%
% This work may be distributed and/or modified under the
% conditions of the LaTeX Project Public License, either version 1.3
% of this license or (at your option) any later version.
% The latest version of this license is in
%   http://www.latex-project.org/lppl.txt
% and version 1.3 or later is part of all distributions of LaTeX
% version 2005/12/01 or later.
%
% This work has the LPPL maintenance status `maintained'.
%
% The Current Maintainer of this work is Niklas Beisert.
%
% This work consists of the files childdoc.dtx and childdoc.ins
% and the derived files childdoc.def and cdocsamp.tex with
% cdocsch1.tex, cdocsch2.tex, cdocsdrf.tex, cdocsfn1.tex, cdocsfn2.tex.
%
%<package>\ifdefined\childdocmain\endinput\fi
%<package>\ProvidesFile{childdoc.def}[2018/12/30 v2.0 child document driver]
%<samplemain>\ProvidesFile{cdocsamp.tex}[2018/12/30 v2.0 sample for childdoc]
%<*driver>
%\ProvidesFile{childdoc.drv}[2018/12/30 v2.0 childdoc reference manual file]
\PassOptionsToClass{10pt,a4paper}{article}
\documentclass{ltxdoc}

\usepackage[margin=35mm]{geometry}
\usepackage{hyperref}
\usepackage{hyperxmp}
\usepackage[usenames]{color}

\hypersetup{colorlinks=true}
\hypersetup{pdfstartview=FitH}
\hypersetup{pdfpagemode=UseNone}
\hypersetup{pdfsource={}}
\hypersetup{pdflang={en-UK}}
\hypersetup{pdfcopyright={Copyright 2017-2018 Niklas Beisert.
  This work may be distributed and/or modified under the
  conditions of the LaTeX Project Public License, either version 1.3
  of this license or (at your option) any later version.}}
\hypersetup{pdflicenseurl={http://www.latex-project.org/lppl.txt}}
\hypersetup{pdfcontactaddress={ETH Zurich, ITP, HIT K,
  Wolfgang-Pauli-Strasse 27}}
\hypersetup{pdfcontactpostcode={8093}}
\hypersetup{pdfcontactcity={Zurich}}
\hypersetup{pdfcontactcountry={Switzerland}}
\hypersetup{pdfcontactemail={nbeisert@itp.phys.ethz.ch}}
\hypersetup{pdfcontacturl={http://people.phys.ethz.ch/\xmptilde nbeisert/}}

\newcommand{\secref}[1]{\hyperref[#1]{section \ref*{#1}}}

\parskip1ex
\parindent0pt
\let\olditemize\itemize
\def\itemize{\olditemize\parskip0pt}

\begin{document}

\title{The \textsf{childdoc} Package}
\hypersetup{pdftitle={The childdoc Package}}
\author{Niklas Beisert\\[2ex]
  Institut f\"ur Theoretische Physik\\
  Eidgen\"ossische Technische Hochschule Z\"urich\\
  Wolfgang-Pauli-Strasse 27, 8093 Z\"urich, Switzerland\\[1ex]
  \href{mailto:nbeisert@itp.phys.ethz.ch}
  {\texttt{nbeisert@itp.phys.ethz.ch}}}
\hypersetup{pdfauthor={Niklas Beisert}}
\hypersetup{pdfsubject={Manual for the LaTeX2e Package childdoc}}
\date{30 December 2018, \textsf{v2.0}}
\maketitle

\begin{abstract}\noindent
\textsf{childdoc} is a \LaTeXe{} package
that enables the direct compilation
of document sections included by |\include|
to individual files.
\end{abstract}

\begingroup
\parskip0ex
\tableofcontents
\endgroup

%%%%%%%%%%%%%%%%%%%%%%%%%%%%%%%%%%%%%%%%%%%%%%%%%%%%%%%%%%%%%%%%%%%%%%%%%%%%%%%%
%%%%%%%%%%%%%%%%%%%%%%%%%%%%%%%%%%%%%%%%%%%%%%%%%%%%%%%%%%%%%%%%%%%%%%%%%%%%%%%%
\section{Introduction}

\LaTeX{} provides a mechanism to structure a large document (such as a book)
into a main file and several child files (containing the chapters)
using the |\include| command.
This mechanism is beneficial for documents
which span hundreds of pages in order to
make the source file(s) more manageable.
Moreover, compilation can be restricted to
selected child files by means of the |\includeonly| command.
The latter feature can be used to reduce the compilation time while editing
(this was significantly more useful in the earlier days of \LaTeX{})
or to generate a smaller document which is easier to navigate.
Another application of |\includeonly| is to generate
documents consisting of selected parts of the complete document.

However, there are a few drawbacks of the plain |\include| mechanism:
\begin{itemize}
\item
The child files cannot be compiled on their own,
they can only be compiled via the main file.
A naive editing environment
(such as a text editor with an option
to have the current file processed by \LaTeX)
may require one to switch to the main file before compiling;
attempting to compile the child file produces errors.
\item
The main file must be modified (each time)
to adjust the |\includeonly| command
to the present needs. This easily leaves the main file in a messy state.
\item
The generated document will always carry the filename
of the main document. This is inconvenient if
several child files are to be compiled and
to be kept for distribution.
\end{itemize}

The present package provides a simple interface
to make child files individually compilable by \LaTeX{}.
Compiling a child file then has the same effect as compiling
the main file with an |\includeonly| command
to select the appropriate child.
Moreover the generated document will carry the name of the child
rather than the main file.
This resolves all three above issues.

This feature is meant to make the editing of books,
thesis documents and lecture notes somewhat more convenient.
However, the package can also be used efficiently for
composing a series of documents (such as exercise sheets)
which are typically distributed individually.
It then assists the author in generating the individual documents
(potentially in different versions)
as well as a document containing the collected series.
Another application is in developing style files
or other kinds of included material
where compilation of the style file could redirect
to a sample or test file.

%%%%%%%%%%%%%%%%%%%%%%%%%%%%%%%%%%%%%%%%%%%%%%%%%%%%%%%%%%%%%%%%%%%%%%%%%%%%%%%%
%%%%%%%%%%%%%%%%%%%%%%%%%%%%%%%%%%%%%%%%%%%%%%%%%%%%%%%%%%%%%%%%%%%%%%%%%%%%%%%%
\section{Usage}

First of all, the package \textsf{childdoc} is \emph{not} a standard
\LaTeXe{} |.sty| style file! Therefore it needs to be invoked in
a non-standard way.

%%%%%%%%%%%%%%%%%%%%%%%%%%%%%%%%%%%%%%%%%%%%%%%%%%%%%%%%%%%%%%%%%%%%%%%%%%%%%%%%
\subsection{Included Files}
\label{sec:include}

%%%%%%%%%%%%%%%%%%%%%%%%%%%%%%%%%%%%%%%%
\DescribeMacro{\childdocmain}
To use the package, add the commands
\begin{center}
\begin{tabular}{l}
|\input{childdoc.def}|\\
|\childdocmain{}|\\
\end{tabular}
\end{center}
at the very top of the main \LaTeX{} file,
in particular \emph{before} the |\documentclass| statement!
The argument of |\childdocmain| should be left empty
(but it must be present).

%%%%%%%%%%%%%%%%%%%%%%%%%%%%%%%%%%%%%%%%
\DescribeMacro{\childdocof}
Furthermore, add the commands
\begin{center}
\begin{tabular}{l}
|\input{childdoc.def}|\\
|\childdocof{|\textit{main}|}|\\
\end{tabular}
\end{center}
at the top of every child file \textit{child}
which is included by |\include{|\textit{child}|}|
from within the main file
(or at least for those files to be compiled individually).
The argument \textit{main} must be the filename of the main file.

There are a couple of
considerations in setting up the main and child documents:

%%%%%%%%%%%%%%%%%%%%%%%%%%%%%%%%%%%%%%%%
\paragraph{Restrictions.}

Please note the following restrictions:
\begin{itemize}
\item
|\childdocmain| must be called with one argument \textit{main}
to ensure compatibility with earlier version of the package.
It must either be empty (|\childdocmain{}|)
or precisely match the filename of the main file in which it is specified.
See \secref{sec:detection} for further information.
\item
The filename \textit{main} must be specified without the |.tex| extension.
\item
The filename \textit{main} is case sensitive
(even in case-insensitive file systems)
due to internal string comparison.
\item
The argument \textit{main} should be fully expanded, it cannot be a macro.
\item
Subdirectories and special characters should be avoided in filenames.
\item
The command |\childdocmain{|\textit{main}|}| must be followed by a whitespace.
It should not be followed immediately by another command
or by a comment mark `|%|'.
This is because the \TeX{} parser reads the token immediately following
the argument of |\childdocmain| and puts it
at the beginning of every child section;
however, a white\-space is ignored.
\end{itemize}

%%%%%%%%%%%%%%%%%%%%%%%%%%%%%%%%%%%%%%%%
\paragraph{Content of Main File.}

It is advisable to place all content in the child files included by |\include|.
Any output contained in the main file will appear in all child documents
unless suppressed manually;
it cannot be suppressed automatically by the |\includeonly| directive
and thus should normally be avoided.
A method to include some content in the main file
by means of conditional processing is described in \secref{sec:conditional}.

%%%%%%%%%%%%%%%%%%%%%%%%%%%%%%%%%%%%%%%%
\paragraph{Page Numbering.}

When only a part of the document is compiled,
the appropriate numbering of pages
(as well as other status parameters)
is determined from the |.aux| files.
The latter contain information from previous passes.
However this information needs to propagate through
all intermediate child documents.
Therefore the page numbering in child documents may well
be inconsistent until the complete document is compiled at least once.

A useful (if unconventional) way to always ensure a consistent
page numbering is to restart the numbering in each child document
and denote the pages by `\textit{child}|.|\textit{page}'
where \textit{child} represents the chapter/section number of the child file.
This can be achieved by the command
|\numberwithin{page}{|\textit{child}|}|
of the \textsf{amsmath} package
where \textit{child} can be |chapter| or |section|
depending on the chosen structuring.
Alternatively, one can modify the macro |\thepage| appropriately
and reset the counter |page| at the start of each child file.

%%%%%%%%%%%%%%%%%%%%%%%%%%%%%%%%%%%%%%%%%%%%%%%%%%%%%%%%%%%%%%%%%%%%%%%%%%%%%%%%
\subsection{Conditional Processing}
\label{sec:conditional}

The package provides a mechanism to compile different versions
of a document. To customise the versions further some conditional processing
can come in handy to distinguish which version is being compiled.
The package provides two macros to describe the compilation context:

%%%%%%%%%%%%%%%%%%%%%%%%%%%%%%%%%%%%%%%%
\DescribeMacro{\ifchilddoc}
The conditional |\ifchilddoc| distinguishes between the compilation of
child documents and the main document:
%
\begin{center}
|\ifchilddoc |\textit{child-code}| |[|\||else |\textit{main-code}]| \||fi|
\end{center}

%%%%%%%%%%%%%%%%%%%%%%%%%%%%%%%%%%%%%%%%
\DescribeMacro{\childdocname}
\DescribeMacro{\childdocjob}
The macro |\childdocname| contains the filename (without extension)
of the main or child file being processed.
Note that |\childdocjob| will always contain the name of the main file.

%%%%%%%%%%%%%%%%%%%%%%%%%%%%%%%%%%%%%%%%
\paragraph{Title Page.}

Conditional processing can be used to include a title or banner page
in the main document when proper precautions are taken.
Importantly, the code in the main file should ensure that the page counter
(as well as other status parameters which are stored in the |.aux| files)
takes the same value after the conditional processing.
Otherwise the page numbers may take divergent values
depending on which part is compiled.

For example, a title page could be declared by:
%
\begin{center}
\begin{tabular}{l}
|\ifchilddoc\||else|\\
|\addtocounter{page}{-1}|\\
\textit{code for title page}\\
|\newpage|\\
|\||fi|
\end{tabular}
\end{center}
%
A banner page for the child documents can be generated by:
%
\begin{center}
\begin{tabular}{l}
|\ifchilddoc|\\
|\addtocounter{page}{-1}|\\
\textit{code for banner page}\\
|\newpage|\\
|\||fi|
\end{tabular}
\end{center}
%
Here one could write a message such as:
\begin{center}
|This is the part \childdocname{} of \childdocjob{}.|
\end{center}

%%%%%%%%%%%%%%%%%%%%%%%%%%%%%%%%%%%%%%%%%%%%%%%%%%%%%%%%%%%%%%%%%%%%%%%%%%%%%%%%
\subsection{Flags}
\label{sec:flags}

The package makes it easy to generate different versions
of the main or child documents.
To this end compilation flags can be defined
and assigned different default values.
They will be particularly useful in conjunction
with the forwarding mechanism described in \secref{sec:forward}.

For example, it may be useful to have a flag |\version|
which can be set to |draft| or |final|.
The document source will contain some conditional code
depending on the value of |\version|.
Suppose further, the flag should default to |final| for the main file
and to |draft| for child files
which is a natural assignment for editing the document.
This is achieved by placing the following code
in the preamble of the main document
(below the |\childdocmain| directive):
%
\begin{center}
\begin{tabular}{l}
|\ifchilddoc|\\
|\providecommand{\version}{draft}|\\
|\||else|\\
|\providecommand{\version}{final}|\\
|\||fi|
\end{tabular}
\end{center}
%
The definition by |\providecommand| makes sure
that previous definitions are not overwritten.
Further statements |\providecommand{\version}{...}|
can thus be added before the above code to override it.

For the main file, one might add a line
(between |\childdocmain| and the above block)
%
\begin{center}
|%\ifchilddoc\||else\providecommand{\version}{draft}\||fi|
\end{center}
%
which can be uncommented to produce a draft version.
Likewise one can add a line to the very top of a child file
(above the |\childdocof{|\textit{main}|}| directive)
%
\begin{center}
|%\providecommand{\version}{final}|
\end{center}
%
which can be uncommented to produce the final version of this child document.

%%%%%%%%%%%%%%%%%%%%%%%%%%%%%%%%%%%%%%%%%%%%%%%%%%%%%%%%%%%%%%%%%%%%%%%%%%%%%%%%
\subsection{Forwarding}
\label{sec:forward}

Different versions of the main or child documents
using compilation flags as described in \secref{sec:flags}
can be (permanently) stored in different files
for convenient compilation, viewing and distribution.
To this end, the package defines a command
to pass on compilation to a different file:

%%%%%%%%%%%%%%%%%%%%%%%%%%%%%%%%%%%%%%%%
\DescribeMacro{\childdocforward}
The command |\childdocforward| redirects processing to
another source file:
%
\begin{center}
\begin{tabular}{l}
|\input{childdoc.def}|\\
|\childdocforward[|\textit{main}|]{|\textit{dest}|}|\\
\end{tabular}
\end{center}
%
The argument \textit{dest} is the destination file
(without extension).
It should be the main file or one of the child files.
Note that further \textsf{childdoc} directives
such as |\childdocof| and |\childdocforward|
in the indicated file will be processed in this form.
The optional argument \textit{main}
passes on directly to the main file \textit{main}
while pretending to compile the child \textit{dest}.
This form behaves as if \textit{dest}
issues |\childdocof{|\textit{main}|}| right away,
and no further \textsf{childdoc} directives will be processed.

%%%%%%%%%%%%%%%%%%%%%%%%%%%%%%%%%%%%%%%%
\DescribeMacro{\...prefix}
In the alternative form |\childdocforwardprefix|,
%
\begin{center}
\begin{tabular}{l}
|\input{childdoc.def}|\\
|\childdocforwardprefix[|\textit{main}|]{|\textit{prefix}|}{|\textit{dest}|}|
\end{tabular}
\end{center}
%
the destination file is determined by a pattern
depending on the current file:
To make this work, the current file must be called
`{\textit{prefix}\hspace{0.2em}\textit{suffix}}'
with \textit{prefix} matching precisely the argument.
Processing is then passed on to the file
`{\textit{dest}\hspace{0.2em}\textit{suffix}}'.
Surely, the same effect is achieved by
directly specifying the
argument `{\textit{dest}\hspace{0.2em}\textit{suffix}}'
in the first form.
However, that requires to set up a different file
for each child. With the alternative form of the command
all these files can have exactly the same content
which simplifies setting them up and maintaining them.

For example, the following file |draft.tex|
with a compilation flag |\version| as described in \secref{sec:flags}
compiles the main document as a draft:
%
\begin{center}
\begin{tabular}{l}
|\def\version{draft}|\\
|\input{childdoc.def}|\\
|\childdocforward{|\textit{main}|}|
\end{tabular}
\end{center}
%
Likewise, the following files |final|\textit{nn}|.tex|
compile the final version of the child document
|child|\textit{nn}|.tex|:
%
\begin{center}
\begin{tabular}{l}
|\def\version{final}|\\
|\input{childdoc.def}|\\
|\childdocforwardprefix{final}{child}|
\end{tabular}
\end{center}
%

Note that when several versions of a main file and/or of each child file
are to be generated, it may be convenient to set up a |Makefile| or
shell script to automatise the process.

%%%%%%%%%%%%%%%%%%%%%%%%%%%%%%%%%%%%%%%%%%%%%%%%%%%%%%%%%%%%%%%%%%%%%%%%%%%%%%%%
\subsection{Command Line Processing}
\label{sec:commandline}

The effect of redirection files can also be achieved by invoking
the \LaTeX{} compiler with a more elaborate command line.
Most conveniently this should be done as part
of a shell script or a |Makefile|.

When using \textsf{childdoc} in the main file, the following
command lines effectively perform a redirection
(note that depending on the shell being used,
backslashes may have to be doubled: `|\|' $\to$ `|\\|'):
%
\begin{center}
|... -jobname "|\textit{target}|" |\\|"|[\textit{flags}]%
|\input{childdoc.def}\childdocforward[|\textit{main}|]{|\textit{dest}|}"|
\end{center}
%
Here \textit{target} is the name of the output file,
\textit{main} is the name of the main file
and \textit{dest} is the name of the main or child file to be processed
(all filenames without extensions).
The optional argument \textit{main} can be omitted
if \textit{main} matches \textit{dest}.
Optionally, compilation \textit{flags} can be defined via |\def| commands.
This command line makes the \TeX{} engine believe
it is compiling the file \textit{target}
whose content is specified as the latter parameter.
The provided code then forwards the processing to
\textit{main} or \textit{dest} as described in \secref{sec:forward}.

%%%%%%%%%%%%%%%%%%%%%%%%%%%%%%%%%%%%%%%%%%%%%%%%%%%%%%%%%%%%%%%%%%%%%%%%%%%%%%%%
\subsection{Include by Input}
\label{sec:input}

Including child documents by |\include| has some restrictions by design.
Most notably, the content of a child document always occupies
its own set of pages; pages cannot be shared between child documents.
Usually, this behaviour makes perfect sense
because each child document contain an essential part of the document.
However, in some situations it may be desirable to compose
a document from a collection of parts
without having mandatory page breaks between then.
For this case, the package
provides a mechanism to include parts
by |\input| which can also be processed individually.
However, by construction this mechanism
requires manual handling of the content to be output.

%%%%%%%%%%%%%%%%%%%%%%%%%%%%%%%%%%%%%%%%
\DescribeMacro{\ifchilddocmanual}
The main file should be prepared as usual, see \secref{sec:include}.
However, the document body must make a distinction
between processing of an individual part and of the main document, e.g.:
%
\begin{center}
\begin{tabular}{l}
|\ifchilddocmanual|\\
|\input{\childdocname}|\\
|\||else|\\
\textit{document body with }|\input{|\textit{part}|}|\\
|\||fi|
\end{tabular}
\end{center}
%
The conditional |\ifchilddocmanual| is true whenever
a part to be included by |\input| is being compiled,
and the name of the part is stored in |\childdocname|.

%%%%%%%%%%%%%%%%%%%%%%%%%%%%%%%%%%%%%%%%
\DescribeMacro{\childdocby}
Each part to be included by |\input| should start with:
%
\begin{center}
\begin{tabular}{l}
|\input{childdoc.def}|\\
|\childdocby{|\textit{main}|}|\\
\end{tabular}
\end{center}
%
The directive |\childdocby| is similar to |\childdocof|
described in \secref{sec:include},
but the subsequent selection of content must be done manually.
To that end, both |\ifchilddoc| and |\ifchilddocmanual|
will be true upon processing of a part,
and the name of the part is stored in |\childdocname|.
Note that |\jobname| will be set to the filename of the current part
so that each part receives an individual |.aux| file
that does not interfere with the |.aux| file(s) of the main document.
This behaviour can be altered by the alternative form
|\childdocby[*]{|\textit{main}|}| (with a non-empty optional argument)
which uses the |.aux| file of the main document
by setting |\jobname| to \textit{main}.

%%%%%%%%%%%%%%%%%%%%%%%%%%%%%%%%%%%%%%%%%%%%%%%%%%%%%%%%%%%%%%%%%%%%%%%%%%%%%%%%
\subsection{Driver Development}
\label{sec:driver}

The \textsf{childdoc} mechanism can also be use for the development
of definition files such as \LaTeX{} styles or classes.
This case differs from the above setup with multiple parts
included by |\include| in that no |\includeonly| should be invoked.
This can be achieved by starting the include file
(before |\ProvidesPackage|) with:
%
\begin{center}
\begin{tabular}{l}
|\input{childdoc.def}|\\
|\childdocforward{|\textit{main}|}|\\
\end{tabular}
\end{center}
%
or alternatively with:
%
\begin{center}
\begin{tabular}{l}
|\input{childdoc.def}|\\
|\childdocby{|\textit{main}|}|\\
\end{tabular}
\end{center}
%
Both forms have slightly different effects as described above.
The main file is prepared as usual, see \secref{sec:include}.

%%%%%%%%%%%%%%%%%%%%%%%%%%%%%%%%%%%%%%%%%%%%%%%%%%%%%%%%%%%%%%%%%%%%%%%%%%%%%%%%
\subsection{Legacy Detection}
\label{sec:detection}

The directive |\childdocmain| in the main file can detect
whether the complete document or merely a child is to be compiled
even without using the directive |\childdocof|.
This method is deprecated because it is less robust
and there is no compelling reason to use it;
it is merely provided for backward compatibility
and it may be removed in future versions.

If the detection mechanism is to be used,
it is mandatory to correctly specify
the filename of the main file as the argument of |\childdocmain|:
%
\begin{center}
\begin{tabular}{l}
|\input{childdoc.def}|\\
|\childdocmain{|\textit{main}|}|\\
\end{tabular}
\end{center}
%
If |\jobname| does not match the argument \textit{main} of |\childdocmain|,
it is assumed that |\jobname| points to the child file to be compiled.
When using |\childdocmain| with the main file specified as argument,
it suffices to start a child file
with just |\input{|\textit{main}|}|
without loading of the package and using |\childdocof|.
If instead all processing is done
with the appropriate \textsf{childdoc} directives,
the argument of \textit{main} of |\childdocmain| can be empty.

An alternative version of the command line processing described
in \secref{sec:commandline} using the detection mechanism reads:
%
\begin{center}
|... -jobname "|\textit{target}|" "|[\textit{flags}]%
[|\def\jobname{|\textit{dest}|}|]|\input{|\textit{main}|}"|
\end{center}

%%%%%%%%%%%%%%%%%%%%%%%%%%%%%%%%%%%%%%%%%%%%%%%%%%%%%%%%%%%%%%%%%%%%%%%%%%%%%%%%
\subsection{Manual Code}
\label{sec:manual}

In case one cannot be certain whether the definitions file |childdoc.def|
is installed on the target \TeX{} distribution
and one prefers not to ship it,
it is conceivable to paste a few relevant commands into the sources.

To that end, drop all statements |\input{childdoc.def}|
and perform the replacements as outlined below.
Instead of |\childdocmain{|\textit{main}|}| add the following code
to the top of the main file:
%
\begin{center}
\begin{tabular}{l}
|\||ifdefined\childdocname\endinput\||fi\newif\ifchilddoc|\\
|\edef\childdocname{\scantokens\expandafter{\jobname\noexpand}}|\\
|\def\childdocmain{|\textit{main}|}\||ifx\childdocmain\childdocname\||else|\\
|\childdoctrue\includeonly{\childdocname}\let\jobname\childdocmain\||fi|\\
\end{tabular}
\end{center}
%
Instead of |\childdocof{|\textit{main}|}| just include the main file
at the top of each child file:
%
\begin{center}
|\input{|\textit{main}|}|
\end{center}
%
A simple redirection |\childdocforward{|\textit{dest}|}| is achieved by:
%
\begin{center}
|\def\jobname{|\textit{dest}|}\input{\jobname}|
\end{center}
%
The redirection with prefix
|\childdocforwardprefix[|\textit{prefix}|]{|\textit{dest}|}|
is accomplished by:
%
\begin{center}
\begin{tabular}{l}
|{\edef\jobname{\scantokens\expandafter{\jobname\noexpand}}|\\
|\def\redirectjob |\textit{prefix}|#1~~~{\gdef\jobname{|\textit{dest}|#1}}|\\
|\expandafter\redirectjob\jobname~~~}\input{\jobname}|
\end{tabular}
\end{center}

In an alternative approach,
child documents can be compiled by a specific command line
without additional code or specific definitions:
%
\begin{center}
|... -jobname "|\textit{target}|" "|[\textit{flags}]%
|\includeonly{|\textit{dest}|}\input{|\textit{main}|}"|
\end{center}
%

%%%%%%%%%%%%%%%%%%%%%%%%%%%%%%%%%%%%%%%%%%%%%%%%%%%%%%%%%%%%%%%%%%%%%%%%%%%%%%%%
%%%%%%%%%%%%%%%%%%%%%%%%%%%%%%%%%%%%%%%%%%%%%%%%%%%%%%%%%%%%%%%%%%%%%%%%%%%%%%%%
\section{Information}

%%%%%%%%%%%%%%%%%%%%%%%%%%%%%%%%%%%%%%%%%%%%%%%%%%%%%%%%%%%%%%%%%%%%%%%%%%%%%%%%
\subsection{Copyright}

Copyright \copyright{} 2017--2018 Niklas Beisert

This work may be distributed and/or modified under the
conditions of the \LaTeX{} Project Public License, either version 1.3
of this license or (at your option) any later version.
The latest version of this license is in
  \url{http://www.latex-project.org/lppl.txt}
and version 1.3 or later is part of all distributions of \LaTeX{}
version 2005/12/01 or later.

This work has the LPPL maintenance status `maintained'.

The Current Maintainer of this work is Niklas Beisert.

This work consists of the files |README.txt|, |childdoc.ins| and |childdoc.dtx|
as well as the derived files |childdoc.def|, |cdocsamp.tex|
with |cdocsch1.tex|, |cdocsch2.tex|, |cdocspt3.tex|, |cdocspt4.tex|,
|cdocsdrf.tex|, |cdocsfn1.tex|, |cdocsfn2.tex|
as well as |childdoc.pdf|.

%%%%%%%%%%%%%%%%%%%%%%%%%%%%%%%%%%%%%%%%%%%%%%%%%%%%%%%%%%%%%%%%%%%%%%%%%%%%%%%%
\subsection{Files and Installation}

The package consists of the files:
%
\begin{center}
\begin{tabular}{ll}
    |README.txt|   & readme file \\
    |childdoc.ins| & installation file \\
    |childdoc.dtx| & source file \\
    |childdoc.def| & definition file \\
    |cdocsamp.tex| & sample main file \\
    |cdocsch1.tex| & sample include file \\
    |cdocsch2.tex| & sample include file \\
    |cdocspt3.tex| & sample part file \\
    |cdocspt4.tex| & sample part file \\
    |cdocsdrf.tex| & sample redirection file \\
    |cdocsfn1.tex| & sample redirection file \\
    |cdocsfn2.tex| & sample redirection file \\
    |childdoc.pdf| & manual
\end{tabular}
\end{center}
%
The distribution consists of the files
|README.txt|, |childdoc.ins| and |childdoc.dtx|.
%
\begin{itemize}
\item
Run (pdf)\LaTeX{} on |childdoc.dtx|
to compile the manual |childdoc.pdf| (this file).
\item
Run \LaTeX{} on |childdoc.ins| to create the definitions file |childdoc.def|
and the sample |cdocsamp.tex| with include files
|cdocsch1.tex|, |cdocsch2.tex|, |cdocspt3.tex|, |cdocspt4.tex|,
|cdocsdrf.tex|, |cdocsfn1.tex|, |cdocsfn2.tex|.
Then copy the file |childdoc.def| to an appropriate directory of your \LaTeX{}
distribution, e.g.\ \textit{texmf-root}|/tex/latex/childdoc|.
\end{itemize}

%%%%%%%%%%%%%%%%%%%%%%%%%%%%%%%%%%%%%%%%%%%%%%%%%%%%%%%%%%%%%%%%%%%%%%%%%%%%%%%%
\subsection{Related CTAN Packages}

There are several other packages which offer a similar functionality:
%
\begin{itemize}
\item
The packages
\href{http://ctan.org/pkg/docmute}{\textsf{docmute}},
\href{http://ctan.org/pkg/includex}{\textsf{includex}} and
\href{http://ctan.org/pkg/standalone}{\textsf{standalone}}
provide commands to include only the document body of
a child file thus allowing both files to be compiled individually.
\item
The packages \href{http://ctan.org/pkg/subdocs}{\textsf{subdocs}}
and \href{http://ctan.org/pkg/subfiles}{\textsf{subfiles}}
provide structures in which the main and child documents can be
encapsulated and allowing them to be compiled individually.
The inclusion mechanism is different from the conventional |\include|.
\item
The package \href{http://ctan.org/pkg/combine}{\textsf{combine}}
is an elaborate solution to combine several documents into one.
\end{itemize}
%
See also the CTAN topic \href{http://ctan.org/topic/subdocs}{\textsf{subdocs}}
for further related packages.
The present package differs from the above solutions in that
a document structure constructed with the conventional |\include| mechanism
just needs two extra commands at the top of every file
such that all constituent files can be compiled individually.

%%%%%%%%%%%%%%%%%%%%%%%%%%%%%%%%%%%%%%%%%%%%%%%%%%%%%%%%%%%%%%%%%%%%%%%%%%%%%%%%
%\subsection{Feature Suggestions}
%
%The following is a list of features which may be useful for future
%versions of this package:
%%
%\begin{itemize}
%\item
%\ldots
%\end{itemize}

%%%%%%%%%%%%%%%%%%%%%%%%%%%%%%%%%%%%%%%%%%%%%%%%%%%%%%%%%%%%%%%%%%%%%%%%%%%%%%%%
\subsection{Revision History}

%%%%%%%%%%%%%%%%%%%%%%%%%%%%%%%%%%%%%%%%
\paragraph{v2.0:} 2018/12/30

\begin{itemize}
\item
immediate forward processing
\item
added |\childdocby| mechanism
\item
manual restructured
\end{itemize}

%%%%%%%%%%%%%%%%%%%%%%%%%%%%%%%%%%%%%%%%
\paragraph{v1.6:} 2018/01/17

\begin{itemize}
\item
application for development of include files
\item
corrections to manual
\end{itemize}

%%%%%%%%%%%%%%%%%%%%%%%%%%%%%%%%%%%%%%%%
\paragraph{v1.5:} 2017/05/21

\begin{itemize}
\item
more complete structuring introduced
\item
|\childdocof| introduced
\item
|\childdoc| renamed to |\childdocmain|
\item
|\childredirect| renamed to |\childdocforward| and |\childdocforwardprefix|
and functionality expanded
\end{itemize}

%%%%%%%%%%%%%%%%%%%%%%%%%%%%%%%%%%%%%%%%
\paragraph{v1.0:} 2017/04/27

\begin{itemize}
\item
manual and install package
\item
first version published on CTAN
\end{itemize}

%%%%%%%%%%%%%%%%%%%%%%%%%%%%%%%%%%%%%%%%
\paragraph{v0.6:} 2017/04/26

\begin{itemize}
\item
redirection mechanism added
\end{itemize}

%%%%%%%%%%%%%%%%%%%%%%%%%%%%%%%%%%%%%%%%
\paragraph{v0.5:} 2017/04/26

\begin{itemize}
\item
functionality in definition file
\end{itemize}


%%%%%%%%%%%%%%%%%%%%%%%%%%%%%%%%%%%%%%%%%%%%%%%%%%%%%%%%%%%%%%%%%%%%%%%%%%%%%%%%
%%%%%%%%%%%%%%%%%%%%%%%%%%%%%%%%%%%%%%%%%%%%%%%%%%%%%%%%%%%%%%%%%%%%%%%%%%%%%%%%
%%%%%%%%%%%%%%%%%%%%%%%%%%%%%%%%%%%%%%%%%%%%%%%%%%%%%%%%%%%%%%%%%%%%%%%%%%%%%%%%
\appendix

\settowidth\MacroIndent{\rmfamily\scriptsize 000\ }

 \DocInput{childdoc.dtx}

\end{document}
%</driver>
% \fi
%
% %%%%%%%%%%%%%%%%%%%%%%%%%%%%%%%%%%%%%%%%%%%%%%%%%%%%%%%%%%%%%%%%%%%%%%%%%%%%%%
% %%%%%%%%%%%%%%%%%%%%%%%%%%%%%%%%%%%%%%%%%%%%%%%%%%%%%%%%%%%%%%%%%%%%%%%%%%%%%%
% \section{Sample}
%\iffalse
%<*samplemain>
%\fi
%
% The following presents a sample document
% with two chapters, two parts, a title page,
% a compile flag as well as three forwarding files to set the flag.
% It consists of eight |.tex| files:
% \begin{center}
% \begin{tabular}{ll}
% |cdocsamp.tex|&main file\\
% |cdocsch1.tex|&include file for chapter 1\\
% |cdocsch2.tex|&include file for chapter 2\\
% |cdocspt3.tex|&include file for part 3\\
% |cdocspt4.tex|&include file for part 4\\
% |cdocsdrf.tex|&forwarding file for main file in draft mode\\
% |cdocsfi1.tex|&forwarding file for final version of chapter 1\\
% |cdocsfi2.tex|&forwarding file for final version of chapter 2\\
% \end{tabular}
% \end{center}
% Each of the eight files can be compiled directly by the \LaTeX{} compiler.
%
% %%%%%%%%%%%%%%%%%%%%%%%%%%%%%%%%%%%%%%
% \paragraph{Main File.}
%
% The main file is called |cdocsamp.tex|.
%
% Load the \textsf{childdoc} definitions and
% declare the filename for the main document:
%    \begin{macrocode}
\input{childdoc.def}
\childdocmain{}
%    \end{macrocode}

% Optional override for |\version| flag:
%    \begin{macrocode}
%%\ifchilddoc\else\providecommand{\version}{draft}\fi
%    \end{macrocode}

% Define the default values for the |\version| flag
% (|final| for the main file and |draft| for childs):
%    \begin{macrocode}
\ifchilddoc
\providecommand{\version}{draft}
\else
\providecommand{\version}{final}
\fi
%    \end{macrocode}

% Load the standard document class:
%    \begin{macrocode}
\documentclass[12pt]{article}
%    \end{macrocode}

% Start the document body:
%    \begin{macrocode}
\begin{document}
%    \end{macrocode}

% Declare a title page.
% Print title, part of document being processed and version flag:
%    \begin{macrocode}
\addtocounter{page}{-1}
\begin{center}
{\LARGE\bfseries{}childdoc example\par}
\vspace{1cm}
\ifchilddoc
\ifchilddocmanual part\else chapter\fi:
`\childdocname' of `\childdocjob'\par
\else
main document: `\childdocjob'\par
\fi
version: \version\par
\end{center}
\newpage
%    \end{macrocode}

% Manually include selected file,
% otherwise process as usual:
%    \begin{macrocode}
\ifchilddocmanual
\section*{part `\childdocname'}
\input{\childdocname}
\else
%    \end{macrocode}

% Include the two chapters:
%    \begin{macrocode}
\include{cdocsch1}
\include{cdocsch2}
%    \end{macrocode}

% Include the two parts unless only chapters should be displayed:
%    \begin{macrocode}
\ifchilddoc\else
\section{part three}
\input{cdocspt3}
\section{part four}
\input{cdocspt4}
\fi
%    \end{macrocode}

% Process as usual until here:
%    \begin{macrocode}
\fi
%    \end{macrocode}

% End of document body:
%    \begin{macrocode}
\end{document}
%    \end{macrocode}
%\iffalse
%</samplemain>
%\fi
%
% %%%%%%%%%%%%%%%%%%%%%%%%%%%%%%%%%%%%%%
% \paragraph{Chapter Include Files.}
%
% The include files are called |cdocsch1.tex| and |cdocsch2.tex|.
%
%\iffalse
%<*samplechap1|samplechap2>
%\fi

% Optional override for |\version| flag:
%    \begin{macrocode}
%%\providecommand{\version}{final}
%    \end{macrocode}

% Include the main document:
%    \begin{macrocode}
\input{childdoc.def}
\childdocof{cdocsamp}
%    \end{macrocode}

%\iffalse
%</samplechap1|samplechap2>
%\fi
%
%\iffalse
%<*samplechap1>
%\fi
% Some text for chapter 1:
%    \begin{macrocode}
\section{one}
some text in chapter one
%    \end{macrocode}

%\iffalse
%</samplechap1>
%\fi
% Some text for chapter 2:
%\iffalse
%<*samplechap2>
%\fi
%    \begin{macrocode}
\section{two}
more text in chapter two
%    \end{macrocode}

%\iffalse
%</samplechap2>
%\fi
%
% %%%%%%%%%%%%%%%%%%%%%%%%%%%%%%%%%%%%%%
% \paragraph{Part Include Files.}
%
% The include files are called |cdocspt3.tex| and |cdocspt4.tex|.
%
%\iffalse
%<*samplepart3|samplepart4>
%\fi

% Optional override for |\version| flag:
%    \begin{macrocode}
%%\providecommand{\version}{final}
%    \end{macrocode}

% Include the main document:
%    \begin{macrocode}
\input{childdoc.def}
\childdocby{cdocsamp}
%    \end{macrocode}

%\iffalse
%</samplepart3|samplepart4>
%\fi
%
%\iffalse
%<*samplepart3>
%\fi
% Some text for part 3:
%    \begin{macrocode}
some text in part three
%    \end{macrocode}

%\iffalse
%</samplepart3>
%\fi
% Some text for part 4:
%\iffalse
%<*samplepart4>
%\fi
%    \begin{macrocode}
more text in part four
%    \end{macrocode}

%\iffalse
%</samplepart4>
%\fi
%
% %%%%%%%%%%%%%%%%%%%%%%%%%%%%%%%%%%%%%%
% \paragraph{Forwarding for a Complete Draft.}
%
% The following forwarding file |cdocsdrf.tex|
% compiles the main document in draft mode:
%\iffalse
%<*sampledraft>
%\fi
%    \begin{macrocode}
\def\version{draft}
\input{childdoc.def}
\childdocforward{cdocsamp}
%    \end{macrocode}

%\iffalse
%</sampledraft>
%\fi
%
% %%%%%%%%%%%%%%%%%%%%%%%%%%%%%%%%%%%%%%
% \paragraph{Forwarding for Final Version of the Chapters.}
%
% The following forwarding files |cdocsfn1.tex| and |cdocsfn2.tex|
% (with identical content)
% compile the final versions of the child documents
% |cdocsch1.tex| and |cdocsch2.tex|, respectively:
%\iffalse
%<*samplefinal>
%\fi
%    \begin{macrocode}
\def\version{final}
\input{childdoc.def}
\childdocforwardprefix[cdocsamp]{cdocsfn}{cdocsch}
%    \end{macrocode}

%\iffalse
%</samplefinal>
%\fi
%
% %%%%%%%%%%%%%%%%%%%%%%%%%%%%%%%%%%%%%%
% \paragraph{Command Line Processing.}
%
% The following three command lines generate the output files
% |cdocscld|, |cdocscl1| and |cdocscl2|
% which should be identical to
% |cdocsdrf|, |cdocsch1| and |cdocsfn2|, respectively:
% \begin{center}
% \begin{tabular}{l}
% |latex -jobname cdocscld \|\\
% |  "\def\version{draft}\input{childdoc.def}\childdocforward{cdocsamp}"|\\
% |latex -jobname cdocscl1 \|\\
% |  "\input{childdoc.def}\childdocforward[cdocsamp]{cdocsch1}"|\\
% |latex -jobname cdocscl2 \|\\
% |  "\def\version{final}\input{childdoc.def}\childdocforward{cdocsch2}"|
% \end{tabular}
% \end{center}
% Note that the trailing backslash on each first line
% merely continues the input to the second line
% (for convenient cut ant paste).
% Furthermore, the command |latex| can be replaced by any
% of its alternative versions such as |pdflatex|.
%
% %%%%%%%%%%%%%%%%%%%%%%%%%%%%%%%%%%%%%%%%%%%%%%%%%%%%%%%%%%%%%%%%%%%%%%%%%%%%%%
% %%%%%%%%%%%%%%%%%%%%%%%%%%%%%%%%%%%%%%%%%%%%%%%%%%%%%%%%%%%%%%%%%%%%%%%%%%%%%%
% \section{Implementation}
%\iffalse
%<*package>
%\fi
%
% This section describes the definitions file |childdoc.def|.

% The definitions cannot be loaded using |\usepackage| or |\RequirePackage|
% which has a mechanism to prevent loading a style file more than once.
% When loading the definitions by means of |\input|
% multiple instances have to be prevented manually:
%\iffalse
%This code needs to be before the `\ProvidesFile' directive
%which is defined at the beginning of this file.
%Therefore it is also placed there and commented out here.
%</package>
%<*discard>
%\fi
%    \begin{macrocode}
\ifdefined\childdocmain\endinput\fi
%    \end{macrocode}
%\iffalse
%</discard>
%<*package>
%\fi
%
% \macro{\ifchilddoc}
% \macro{\ifchilddocmanual}
% The conditional |\ifchilddoc| tells whether a
% child (true) or main (false) document is being compiled.
% The conditional |\ifchilddocmanual| tells whether
% the |\includeonly| mechanism is used (false) or
% the selection of child files must be performed manually (true).
% The definitions initialise to false:
%    \begin{macrocode}
\newif\ifchilddoc
\newif\ifchilddocmanual
%    \end{macrocode}

% \macro{\childdocname}
% \macro{\childdocjob}
% The macro |\childdocname| stores the name of the main document
% to be compiled. The macro |\childdocjob| stores the name of
% the document on which the \LaTeX{} compiler was originally invoked.
% The content of |\jobname| cannot be compared
% to filenames specified in the source due to different catcodes.
% The following code rescans |\jobname|, stores the result
% in |\childdocname| and saves a copy in |\childdocjob|:
%    \begin{macrocode}
\edef\childdocname{\scantokens\expandafter{\jobname\noexpand}}
\let\childdocjob\childdocname
%    \end{macrocode}

% \macro{\childdocdisable}
% The macro |\childdocdisable| prevents the main file
% from being processed more than once.
% At this stage, the main document command |\childdocmain|
% is assumed to be called once again where it should do nothing.
% Any subsequent call to it should prevent
% a secondary processing of the main document
% It overwrites the forwarding commands
% |\childdocof| and |\childdocforward|
% with empty macros to prevent further inclusions of the main document:
%    \begin{macrocode}
\newcommand{\childdocdisable}
{
  \renewcommand{\childdocmain}[1]{\renewcommand{\childdocmain}[1]{\endinput}}
  \renewcommand{\childdocof}[1]{}
  \renewcommand{\childdocby}[2][]{}
  \renewcommand{\childdocforward}[2][]{}
  \renewcommand{\childdocdisable}{}
}
%    \end{macrocode}

% \macro{\childdocmain}
% The macro |\childdocmain| is to be called at the top of the main file
% with nothing or the main filename (without extension) as argument.
% First, it breaks loops.
% If the argument is not empty and does not match |\childdocname|
% (which is set by the first inclusion of |childdoc.def|),
% |\ifchilddoc| is set to true, |\includeonly| is applied to the child file
% and |\jobname| is set to the main file
% (for proper handling of |.aux| files):
%    \begin{macrocode}
\newcommand{\childdocmain}[1]
{
  \childdocdisable\childdocmain{}
  \if?#1?\else
    \begingroup
      \def\childdoctmp{#1}
      \ifx\childdoctmp\childdocname
        \def\childdoctmp{}
      \else
        \def\childdoctmp
        {
          \childdoctrue
          \includeonly{\childdocname}
          \def\childdocjob{#1}
          \def\jobname{#1}
        }
      \fi
      \expandafter
    \endgroup
    \childdoctmp
  \fi
}
%    \end{macrocode}

% \macro{\childdocof}
% The command |\childdocof| redirects
% compilation to the main file |#1|.
%    \begin{macrocode}
\newcommand{\childdocof}[1]
{
  \childdocdisable
  \childdoctrue
  \includeonly{\childdocname}
  \def\jobname{#1}
  \def\childdocjob{#1}
  \input{#1}
}
%    \end{macrocode}

% \macro{\childdocby}
% The command |\childdocby| ....
%    \begin{macrocode}
\newcommand{\childdocby}[2][]
{
  \childdocdisable
  \childdoctrue
  \childdocmanualtrue
  \if?#1?\else
    \def\jobname{#2}
  \fi
  \def\childdocjob{#2}
  \input{#2}
  \endinput
}
%    \end{macrocode}

% \macro{\childdocforward}
% The command |\childdocforward| redirects
% compilation to the main file or
% (if the optional argument is given) a child file.
% Parameters are set as if the main file
% or a child file starting with |\childdocof| was compiled.
% Then compilation is handed over to the main file:
%    \begin{macrocode}
\newcommand{\childdocforward}[2][]
{
  \begingroup
    \if?#1?
      \def\childdoctmp
      {
        \def\childdocname{#2}
        \def\childdocjob{#2}
        \def\jobname{#2}
        \input{#2}
        \endinput
      }
    \else
      \def\childdoctmp
      {
        \childdocdisable
        \def\childdocname{#2}
        \childdoctrue
        \includeonly{#2}
        \def\childdocjob{#1}
        \def\jobname{#1}
        \input{#1}
        \endinput
      }
    \fi
    \expandafter
  \endgroup
  \childdoctmp
}
%    \end{macrocode}

% \macro{\childdocforwardprefix}
% The command |\childdocforwardprefix| redirects
% compilation to the main or a child file by means of a pattern.
% The prefix |#1| in the current filename is replaced by |#2|
% and the suffix of the current filename is kept
% (it is assumed that the filename does not contain the substring `|~~~|'
% which is used as a delimiter).
% Compilation is handed over to the new file by |\childdocforward|:
%    \begin{macrocode}
\newcommand{\childdocforwardprefix}[3][]
{
  \begingroup
    \def\childdocextract #2##1~~~{\def\childdoctmp{\childdocforward[#1]{#3##1}}}
    \expandafter\childdocextract\childdocname~~~
    \expandafter
  \endgroup
  \childdoctmp
}
%    \end{macrocode}

% \macro{\childdoc}
% The deprecated macro |\childdoc| is a legacy version of |\childdocmain|:
%    \begin{macrocode}
\newcommand{\childdoc}{\childdocmain}
%    \end{macrocode}

% \macro{\childdocredirect}
% The deprecated macro |\childdocredirect| is a legacy version
% of |\childdocforward| and |\childdocforwardprefix|:
%    \begin{macrocode}
\newcommand{\childdocredirect}[2][]
{
  \begingroup
    \if?#1?
      \def\childdoctmp{\childdocforward{#2}}
    \else
      \def\childdoctmp{\childdocforwardprefix{#1}{#2}}
    \fi
    \expandafter
  \endgroup
  \childdoctmp
}
%    \end{macrocode}

%\iffalse
%</package>
%\fi
%
\endinput
|\\
|\childdocof{|\textit{main}|}|\\
\end{tabular}
\end{center}
at the top of every child file \textit{child}
which is included by |\include{|\textit{child}|}|
from within the main file
(or at least for those files to be compiled individually).
The argument \textit{main} must be the filename of the main file.

There are a couple of
considerations in setting up the main and child documents:

%%%%%%%%%%%%%%%%%%%%%%%%%%%%%%%%%%%%%%%%
\paragraph{Restrictions.}

Please note the following restrictions:
\begin{itemize}
\item
|\childdocmain| must be called with one argument \textit{main}
to ensure compatibility with earlier version of the package.
It must either be empty (|\childdocmain{}|)
or precisely match the filename of the main file in which it is specified.
See \secref{sec:detection} for further information.
\item
The filename \textit{main} must be specified without the |.tex| extension.
\item
The filename \textit{main} is case sensitive
(even in case-insensitive file systems)
due to internal string comparison.
\item
The argument \textit{main} should be fully expanded, it cannot be a macro.
\item
Subdirectories and special characters should be avoided in filenames.
\item
The command |\childdocmain{|\textit{main}|}| must be followed by a whitespace.
It should not be followed immediately by another command
or by a comment mark `|%|'.
This is because the \TeX{} parser reads the token immediately following
the argument of |\childdocmain| and puts it
at the beginning of every child section;
however, a white\-space is ignored.
\end{itemize}

%%%%%%%%%%%%%%%%%%%%%%%%%%%%%%%%%%%%%%%%
\paragraph{Content of Main File.}

It is advisable to place all content in the child files included by |\include|.
Any output contained in the main file will appear in all child documents
unless suppressed manually;
it cannot be suppressed automatically by the |\includeonly| directive
and thus should normally be avoided.
A method to include some content in the main file
by means of conditional processing is described in \secref{sec:conditional}.

%%%%%%%%%%%%%%%%%%%%%%%%%%%%%%%%%%%%%%%%
\paragraph{Page Numbering.}

When only a part of the document is compiled,
the appropriate numbering of pages
(as well as other status parameters)
is determined from the |.aux| files.
The latter contain information from previous passes.
However this information needs to propagate through
all intermediate child documents.
Therefore the page numbering in child documents may well
be inconsistent until the complete document is compiled at least once.

A useful (if unconventional) way to always ensure a consistent
page numbering is to restart the numbering in each child document
and denote the pages by `\textit{child}|.|\textit{page}'
where \textit{child} represents the chapter/section number of the child file.
This can be achieved by the command
|\numberwithin{page}{|\textit{child}|}|
of the \textsf{amsmath} package
where \textit{child} can be |chapter| or |section|
depending on the chosen structuring.
Alternatively, one can modify the macro |\thepage| appropriately
and reset the counter |page| at the start of each child file.

%%%%%%%%%%%%%%%%%%%%%%%%%%%%%%%%%%%%%%%%%%%%%%%%%%%%%%%%%%%%%%%%%%%%%%%%%%%%%%%%
\subsection{Conditional Processing}
\label{sec:conditional}

The package provides a mechanism to compile different versions
of a document. To customise the versions further some conditional processing
can come in handy to distinguish which version is being compiled.
The package provides two macros to describe the compilation context:

%%%%%%%%%%%%%%%%%%%%%%%%%%%%%%%%%%%%%%%%
\DescribeMacro{\ifchilddoc}
The conditional |\ifchilddoc| distinguishes between the compilation of
child documents and the main document:
%
\begin{center}
|\ifchilddoc |\textit{child-code}| |[|\||else |\textit{main-code}]| \||fi|
\end{center}

%%%%%%%%%%%%%%%%%%%%%%%%%%%%%%%%%%%%%%%%
\DescribeMacro{\childdocname}
\DescribeMacro{\childdocjob}
The macro |\childdocname| contains the filename (without extension)
of the main or child file being processed.
Note that |\childdocjob| will always contain the name of the main file.

%%%%%%%%%%%%%%%%%%%%%%%%%%%%%%%%%%%%%%%%
\paragraph{Title Page.}

Conditional processing can be used to include a title or banner page
in the main document when proper precautions are taken.
Importantly, the code in the main file should ensure that the page counter
(as well as other status parameters which are stored in the |.aux| files)
takes the same value after the conditional processing.
Otherwise the page numbers may take divergent values
depending on which part is compiled.

For example, a title page could be declared by:
%
\begin{center}
\begin{tabular}{l}
|\ifchilddoc\||else|\\
|\addtocounter{page}{-1}|\\
\textit{code for title page}\\
|\newpage|\\
|\||fi|
\end{tabular}
\end{center}
%
A banner page for the child documents can be generated by:
%
\begin{center}
\begin{tabular}{l}
|\ifchilddoc|\\
|\addtocounter{page}{-1}|\\
\textit{code for banner page}\\
|\newpage|\\
|\||fi|
\end{tabular}
\end{center}
%
Here one could write a message such as:
\begin{center}
|This is the part \childdocname{} of \childdocjob{}.|
\end{center}

%%%%%%%%%%%%%%%%%%%%%%%%%%%%%%%%%%%%%%%%%%%%%%%%%%%%%%%%%%%%%%%%%%%%%%%%%%%%%%%%
\subsection{Flags}
\label{sec:flags}

The package makes it easy to generate different versions
of the main or child documents.
To this end compilation flags can be defined
and assigned different default values.
They will be particularly useful in conjunction
with the forwarding mechanism described in \secref{sec:forward}.

For example, it may be useful to have a flag |\version|
which can be set to |draft| or |final|.
The document source will contain some conditional code
depending on the value of |\version|.
Suppose further, the flag should default to |final| for the main file
and to |draft| for child files
which is a natural assignment for editing the document.
This is achieved by placing the following code
in the preamble of the main document
(below the |\childdocmain| directive):
%
\begin{center}
\begin{tabular}{l}
|\ifchilddoc|\\
|\providecommand{\version}{draft}|\\
|\||else|\\
|\providecommand{\version}{final}|\\
|\||fi|
\end{tabular}
\end{center}
%
The definition by |\providecommand| makes sure
that previous definitions are not overwritten.
Further statements |\providecommand{\version}{...}|
can thus be added before the above code to override it.

For the main file, one might add a line
(between |\childdocmain| and the above block)
%
\begin{center}
|%\ifchilddoc\||else\providecommand{\version}{draft}\||fi|
\end{center}
%
which can be uncommented to produce a draft version.
Likewise one can add a line to the very top of a child file
(above the |\childdocof{|\textit{main}|}| directive)
%
\begin{center}
|%\providecommand{\version}{final}|
\end{center}
%
which can be uncommented to produce the final version of this child document.

%%%%%%%%%%%%%%%%%%%%%%%%%%%%%%%%%%%%%%%%%%%%%%%%%%%%%%%%%%%%%%%%%%%%%%%%%%%%%%%%
\subsection{Forwarding}
\label{sec:forward}

Different versions of the main or child documents
using compilation flags as described in \secref{sec:flags}
can be (permanently) stored in different files
for convenient compilation, viewing and distribution.
To this end, the package defines a command
to pass on compilation to a different file:

%%%%%%%%%%%%%%%%%%%%%%%%%%%%%%%%%%%%%%%%
\DescribeMacro{\childdocforward}
The command |\childdocforward| redirects processing to
another source file:
%
\begin{center}
\begin{tabular}{l}
|% \iffalse
%
% childdoc.dtx Copyright (C) 2017-2018 Niklas Beisert
%
% This work may be distributed and/or modified under the
% conditions of the LaTeX Project Public License, either version 1.3
% of this license or (at your option) any later version.
% The latest version of this license is in
%   http://www.latex-project.org/lppl.txt
% and version 1.3 or later is part of all distributions of LaTeX
% version 2005/12/01 or later.
%
% This work has the LPPL maintenance status `maintained'.
%
% The Current Maintainer of this work is Niklas Beisert.
%
% This work consists of the files childdoc.dtx and childdoc.ins
% and the derived files childdoc.def and cdocsamp.tex with
% cdocsch1.tex, cdocsch2.tex, cdocsdrf.tex, cdocsfn1.tex, cdocsfn2.tex.
%
%<package>\ifdefined\childdocmain\endinput\fi
%<package>\ProvidesFile{childdoc.def}[2018/12/30 v2.0 child document driver]
%<samplemain>\ProvidesFile{cdocsamp.tex}[2018/12/30 v2.0 sample for childdoc]
%<*driver>
%\ProvidesFile{childdoc.drv}[2018/12/30 v2.0 childdoc reference manual file]
\PassOptionsToClass{10pt,a4paper}{article}
\documentclass{ltxdoc}

\usepackage[margin=35mm]{geometry}
\usepackage{hyperref}
\usepackage{hyperxmp}
\usepackage[usenames]{color}

\hypersetup{colorlinks=true}
\hypersetup{pdfstartview=FitH}
\hypersetup{pdfpagemode=UseNone}
\hypersetup{pdfsource={}}
\hypersetup{pdflang={en-UK}}
\hypersetup{pdfcopyright={Copyright 2017-2018 Niklas Beisert.
  This work may be distributed and/or modified under the
  conditions of the LaTeX Project Public License, either version 1.3
  of this license or (at your option) any later version.}}
\hypersetup{pdflicenseurl={http://www.latex-project.org/lppl.txt}}
\hypersetup{pdfcontactaddress={ETH Zurich, ITP, HIT K,
  Wolfgang-Pauli-Strasse 27}}
\hypersetup{pdfcontactpostcode={8093}}
\hypersetup{pdfcontactcity={Zurich}}
\hypersetup{pdfcontactcountry={Switzerland}}
\hypersetup{pdfcontactemail={nbeisert@itp.phys.ethz.ch}}
\hypersetup{pdfcontacturl={http://people.phys.ethz.ch/\xmptilde nbeisert/}}

\newcommand{\secref}[1]{\hyperref[#1]{section \ref*{#1}}}

\parskip1ex
\parindent0pt
\let\olditemize\itemize
\def\itemize{\olditemize\parskip0pt}

\begin{document}

\title{The \textsf{childdoc} Package}
\hypersetup{pdftitle={The childdoc Package}}
\author{Niklas Beisert\\[2ex]
  Institut f\"ur Theoretische Physik\\
  Eidgen\"ossische Technische Hochschule Z\"urich\\
  Wolfgang-Pauli-Strasse 27, 8093 Z\"urich, Switzerland\\[1ex]
  \href{mailto:nbeisert@itp.phys.ethz.ch}
  {\texttt{nbeisert@itp.phys.ethz.ch}}}
\hypersetup{pdfauthor={Niklas Beisert}}
\hypersetup{pdfsubject={Manual for the LaTeX2e Package childdoc}}
\date{30 December 2018, \textsf{v2.0}}
\maketitle

\begin{abstract}\noindent
\textsf{childdoc} is a \LaTeXe{} package
that enables the direct compilation
of document sections included by |\include|
to individual files.
\end{abstract}

\begingroup
\parskip0ex
\tableofcontents
\endgroup

%%%%%%%%%%%%%%%%%%%%%%%%%%%%%%%%%%%%%%%%%%%%%%%%%%%%%%%%%%%%%%%%%%%%%%%%%%%%%%%%
%%%%%%%%%%%%%%%%%%%%%%%%%%%%%%%%%%%%%%%%%%%%%%%%%%%%%%%%%%%%%%%%%%%%%%%%%%%%%%%%
\section{Introduction}

\LaTeX{} provides a mechanism to structure a large document (such as a book)
into a main file and several child files (containing the chapters)
using the |\include| command.
This mechanism is beneficial for documents
which span hundreds of pages in order to
make the source file(s) more manageable.
Moreover, compilation can be restricted to
selected child files by means of the |\includeonly| command.
The latter feature can be used to reduce the compilation time while editing
(this was significantly more useful in the earlier days of \LaTeX{})
or to generate a smaller document which is easier to navigate.
Another application of |\includeonly| is to generate
documents consisting of selected parts of the complete document.

However, there are a few drawbacks of the plain |\include| mechanism:
\begin{itemize}
\item
The child files cannot be compiled on their own,
they can only be compiled via the main file.
A naive editing environment
(such as a text editor with an option
to have the current file processed by \LaTeX)
may require one to switch to the main file before compiling;
attempting to compile the child file produces errors.
\item
The main file must be modified (each time)
to adjust the |\includeonly| command
to the present needs. This easily leaves the main file in a messy state.
\item
The generated document will always carry the filename
of the main document. This is inconvenient if
several child files are to be compiled and
to be kept for distribution.
\end{itemize}

The present package provides a simple interface
to make child files individually compilable by \LaTeX{}.
Compiling a child file then has the same effect as compiling
the main file with an |\includeonly| command
to select the appropriate child.
Moreover the generated document will carry the name of the child
rather than the main file.
This resolves all three above issues.

This feature is meant to make the editing of books,
thesis documents and lecture notes somewhat more convenient.
However, the package can also be used efficiently for
composing a series of documents (such as exercise sheets)
which are typically distributed individually.
It then assists the author in generating the individual documents
(potentially in different versions)
as well as a document containing the collected series.
Another application is in developing style files
or other kinds of included material
where compilation of the style file could redirect
to a sample or test file.

%%%%%%%%%%%%%%%%%%%%%%%%%%%%%%%%%%%%%%%%%%%%%%%%%%%%%%%%%%%%%%%%%%%%%%%%%%%%%%%%
%%%%%%%%%%%%%%%%%%%%%%%%%%%%%%%%%%%%%%%%%%%%%%%%%%%%%%%%%%%%%%%%%%%%%%%%%%%%%%%%
\section{Usage}

First of all, the package \textsf{childdoc} is \emph{not} a standard
\LaTeXe{} |.sty| style file! Therefore it needs to be invoked in
a non-standard way.

%%%%%%%%%%%%%%%%%%%%%%%%%%%%%%%%%%%%%%%%%%%%%%%%%%%%%%%%%%%%%%%%%%%%%%%%%%%%%%%%
\subsection{Included Files}
\label{sec:include}

%%%%%%%%%%%%%%%%%%%%%%%%%%%%%%%%%%%%%%%%
\DescribeMacro{\childdocmain}
To use the package, add the commands
\begin{center}
\begin{tabular}{l}
|\input{childdoc.def}|\\
|\childdocmain{}|\\
\end{tabular}
\end{center}
at the very top of the main \LaTeX{} file,
in particular \emph{before} the |\documentclass| statement!
The argument of |\childdocmain| should be left empty
(but it must be present).

%%%%%%%%%%%%%%%%%%%%%%%%%%%%%%%%%%%%%%%%
\DescribeMacro{\childdocof}
Furthermore, add the commands
\begin{center}
\begin{tabular}{l}
|\input{childdoc.def}|\\
|\childdocof{|\textit{main}|}|\\
\end{tabular}
\end{center}
at the top of every child file \textit{child}
which is included by |\include{|\textit{child}|}|
from within the main file
(or at least for those files to be compiled individually).
The argument \textit{main} must be the filename of the main file.

There are a couple of
considerations in setting up the main and child documents:

%%%%%%%%%%%%%%%%%%%%%%%%%%%%%%%%%%%%%%%%
\paragraph{Restrictions.}

Please note the following restrictions:
\begin{itemize}
\item
|\childdocmain| must be called with one argument \textit{main}
to ensure compatibility with earlier version of the package.
It must either be empty (|\childdocmain{}|)
or precisely match the filename of the main file in which it is specified.
See \secref{sec:detection} for further information.
\item
The filename \textit{main} must be specified without the |.tex| extension.
\item
The filename \textit{main} is case sensitive
(even in case-insensitive file systems)
due to internal string comparison.
\item
The argument \textit{main} should be fully expanded, it cannot be a macro.
\item
Subdirectories and special characters should be avoided in filenames.
\item
The command |\childdocmain{|\textit{main}|}| must be followed by a whitespace.
It should not be followed immediately by another command
or by a comment mark `|%|'.
This is because the \TeX{} parser reads the token immediately following
the argument of |\childdocmain| and puts it
at the beginning of every child section;
however, a white\-space is ignored.
\end{itemize}

%%%%%%%%%%%%%%%%%%%%%%%%%%%%%%%%%%%%%%%%
\paragraph{Content of Main File.}

It is advisable to place all content in the child files included by |\include|.
Any output contained in the main file will appear in all child documents
unless suppressed manually;
it cannot be suppressed automatically by the |\includeonly| directive
and thus should normally be avoided.
A method to include some content in the main file
by means of conditional processing is described in \secref{sec:conditional}.

%%%%%%%%%%%%%%%%%%%%%%%%%%%%%%%%%%%%%%%%
\paragraph{Page Numbering.}

When only a part of the document is compiled,
the appropriate numbering of pages
(as well as other status parameters)
is determined from the |.aux| files.
The latter contain information from previous passes.
However this information needs to propagate through
all intermediate child documents.
Therefore the page numbering in child documents may well
be inconsistent until the complete document is compiled at least once.

A useful (if unconventional) way to always ensure a consistent
page numbering is to restart the numbering in each child document
and denote the pages by `\textit{child}|.|\textit{page}'
where \textit{child} represents the chapter/section number of the child file.
This can be achieved by the command
|\numberwithin{page}{|\textit{child}|}|
of the \textsf{amsmath} package
where \textit{child} can be |chapter| or |section|
depending on the chosen structuring.
Alternatively, one can modify the macro |\thepage| appropriately
and reset the counter |page| at the start of each child file.

%%%%%%%%%%%%%%%%%%%%%%%%%%%%%%%%%%%%%%%%%%%%%%%%%%%%%%%%%%%%%%%%%%%%%%%%%%%%%%%%
\subsection{Conditional Processing}
\label{sec:conditional}

The package provides a mechanism to compile different versions
of a document. To customise the versions further some conditional processing
can come in handy to distinguish which version is being compiled.
The package provides two macros to describe the compilation context:

%%%%%%%%%%%%%%%%%%%%%%%%%%%%%%%%%%%%%%%%
\DescribeMacro{\ifchilddoc}
The conditional |\ifchilddoc| distinguishes between the compilation of
child documents and the main document:
%
\begin{center}
|\ifchilddoc |\textit{child-code}| |[|\||else |\textit{main-code}]| \||fi|
\end{center}

%%%%%%%%%%%%%%%%%%%%%%%%%%%%%%%%%%%%%%%%
\DescribeMacro{\childdocname}
\DescribeMacro{\childdocjob}
The macro |\childdocname| contains the filename (without extension)
of the main or child file being processed.
Note that |\childdocjob| will always contain the name of the main file.

%%%%%%%%%%%%%%%%%%%%%%%%%%%%%%%%%%%%%%%%
\paragraph{Title Page.}

Conditional processing can be used to include a title or banner page
in the main document when proper precautions are taken.
Importantly, the code in the main file should ensure that the page counter
(as well as other status parameters which are stored in the |.aux| files)
takes the same value after the conditional processing.
Otherwise the page numbers may take divergent values
depending on which part is compiled.

For example, a title page could be declared by:
%
\begin{center}
\begin{tabular}{l}
|\ifchilddoc\||else|\\
|\addtocounter{page}{-1}|\\
\textit{code for title page}\\
|\newpage|\\
|\||fi|
\end{tabular}
\end{center}
%
A banner page for the child documents can be generated by:
%
\begin{center}
\begin{tabular}{l}
|\ifchilddoc|\\
|\addtocounter{page}{-1}|\\
\textit{code for banner page}\\
|\newpage|\\
|\||fi|
\end{tabular}
\end{center}
%
Here one could write a message such as:
\begin{center}
|This is the part \childdocname{} of \childdocjob{}.|
\end{center}

%%%%%%%%%%%%%%%%%%%%%%%%%%%%%%%%%%%%%%%%%%%%%%%%%%%%%%%%%%%%%%%%%%%%%%%%%%%%%%%%
\subsection{Flags}
\label{sec:flags}

The package makes it easy to generate different versions
of the main or child documents.
To this end compilation flags can be defined
and assigned different default values.
They will be particularly useful in conjunction
with the forwarding mechanism described in \secref{sec:forward}.

For example, it may be useful to have a flag |\version|
which can be set to |draft| or |final|.
The document source will contain some conditional code
depending on the value of |\version|.
Suppose further, the flag should default to |final| for the main file
and to |draft| for child files
which is a natural assignment for editing the document.
This is achieved by placing the following code
in the preamble of the main document
(below the |\childdocmain| directive):
%
\begin{center}
\begin{tabular}{l}
|\ifchilddoc|\\
|\providecommand{\version}{draft}|\\
|\||else|\\
|\providecommand{\version}{final}|\\
|\||fi|
\end{tabular}
\end{center}
%
The definition by |\providecommand| makes sure
that previous definitions are not overwritten.
Further statements |\providecommand{\version}{...}|
can thus be added before the above code to override it.

For the main file, one might add a line
(between |\childdocmain| and the above block)
%
\begin{center}
|%\ifchilddoc\||else\providecommand{\version}{draft}\||fi|
\end{center}
%
which can be uncommented to produce a draft version.
Likewise one can add a line to the very top of a child file
(above the |\childdocof{|\textit{main}|}| directive)
%
\begin{center}
|%\providecommand{\version}{final}|
\end{center}
%
which can be uncommented to produce the final version of this child document.

%%%%%%%%%%%%%%%%%%%%%%%%%%%%%%%%%%%%%%%%%%%%%%%%%%%%%%%%%%%%%%%%%%%%%%%%%%%%%%%%
\subsection{Forwarding}
\label{sec:forward}

Different versions of the main or child documents
using compilation flags as described in \secref{sec:flags}
can be (permanently) stored in different files
for convenient compilation, viewing and distribution.
To this end, the package defines a command
to pass on compilation to a different file:

%%%%%%%%%%%%%%%%%%%%%%%%%%%%%%%%%%%%%%%%
\DescribeMacro{\childdocforward}
The command |\childdocforward| redirects processing to
another source file:
%
\begin{center}
\begin{tabular}{l}
|\input{childdoc.def}|\\
|\childdocforward[|\textit{main}|]{|\textit{dest}|}|\\
\end{tabular}
\end{center}
%
The argument \textit{dest} is the destination file
(without extension).
It should be the main file or one of the child files.
Note that further \textsf{childdoc} directives
such as |\childdocof| and |\childdocforward|
in the indicated file will be processed in this form.
The optional argument \textit{main}
passes on directly to the main file \textit{main}
while pretending to compile the child \textit{dest}.
This form behaves as if \textit{dest}
issues |\childdocof{|\textit{main}|}| right away,
and no further \textsf{childdoc} directives will be processed.

%%%%%%%%%%%%%%%%%%%%%%%%%%%%%%%%%%%%%%%%
\DescribeMacro{\...prefix}
In the alternative form |\childdocforwardprefix|,
%
\begin{center}
\begin{tabular}{l}
|\input{childdoc.def}|\\
|\childdocforwardprefix[|\textit{main}|]{|\textit{prefix}|}{|\textit{dest}|}|
\end{tabular}
\end{center}
%
the destination file is determined by a pattern
depending on the current file:
To make this work, the current file must be called
`{\textit{prefix}\hspace{0.2em}\textit{suffix}}'
with \textit{prefix} matching precisely the argument.
Processing is then passed on to the file
`{\textit{dest}\hspace{0.2em}\textit{suffix}}'.
Surely, the same effect is achieved by
directly specifying the
argument `{\textit{dest}\hspace{0.2em}\textit{suffix}}'
in the first form.
However, that requires to set up a different file
for each child. With the alternative form of the command
all these files can have exactly the same content
which simplifies setting them up and maintaining them.

For example, the following file |draft.tex|
with a compilation flag |\version| as described in \secref{sec:flags}
compiles the main document as a draft:
%
\begin{center}
\begin{tabular}{l}
|\def\version{draft}|\\
|\input{childdoc.def}|\\
|\childdocforward{|\textit{main}|}|
\end{tabular}
\end{center}
%
Likewise, the following files |final|\textit{nn}|.tex|
compile the final version of the child document
|child|\textit{nn}|.tex|:
%
\begin{center}
\begin{tabular}{l}
|\def\version{final}|\\
|\input{childdoc.def}|\\
|\childdocforwardprefix{final}{child}|
\end{tabular}
\end{center}
%

Note that when several versions of a main file and/or of each child file
are to be generated, it may be convenient to set up a |Makefile| or
shell script to automatise the process.

%%%%%%%%%%%%%%%%%%%%%%%%%%%%%%%%%%%%%%%%%%%%%%%%%%%%%%%%%%%%%%%%%%%%%%%%%%%%%%%%
\subsection{Command Line Processing}
\label{sec:commandline}

The effect of redirection files can also be achieved by invoking
the \LaTeX{} compiler with a more elaborate command line.
Most conveniently this should be done as part
of a shell script or a |Makefile|.

When using \textsf{childdoc} in the main file, the following
command lines effectively perform a redirection
(note that depending on the shell being used,
backslashes may have to be doubled: `|\|' $\to$ `|\\|'):
%
\begin{center}
|... -jobname "|\textit{target}|" |\\|"|[\textit{flags}]%
|\input{childdoc.def}\childdocforward[|\textit{main}|]{|\textit{dest}|}"|
\end{center}
%
Here \textit{target} is the name of the output file,
\textit{main} is the name of the main file
and \textit{dest} is the name of the main or child file to be processed
(all filenames without extensions).
The optional argument \textit{main} can be omitted
if \textit{main} matches \textit{dest}.
Optionally, compilation \textit{flags} can be defined via |\def| commands.
This command line makes the \TeX{} engine believe
it is compiling the file \textit{target}
whose content is specified as the latter parameter.
The provided code then forwards the processing to
\textit{main} or \textit{dest} as described in \secref{sec:forward}.

%%%%%%%%%%%%%%%%%%%%%%%%%%%%%%%%%%%%%%%%%%%%%%%%%%%%%%%%%%%%%%%%%%%%%%%%%%%%%%%%
\subsection{Include by Input}
\label{sec:input}

Including child documents by |\include| has some restrictions by design.
Most notably, the content of a child document always occupies
its own set of pages; pages cannot be shared between child documents.
Usually, this behaviour makes perfect sense
because each child document contain an essential part of the document.
However, in some situations it may be desirable to compose
a document from a collection of parts
without having mandatory page breaks between then.
For this case, the package
provides a mechanism to include parts
by |\input| which can also be processed individually.
However, by construction this mechanism
requires manual handling of the content to be output.

%%%%%%%%%%%%%%%%%%%%%%%%%%%%%%%%%%%%%%%%
\DescribeMacro{\ifchilddocmanual}
The main file should be prepared as usual, see \secref{sec:include}.
However, the document body must make a distinction
between processing of an individual part and of the main document, e.g.:
%
\begin{center}
\begin{tabular}{l}
|\ifchilddocmanual|\\
|\input{\childdocname}|\\
|\||else|\\
\textit{document body with }|\input{|\textit{part}|}|\\
|\||fi|
\end{tabular}
\end{center}
%
The conditional |\ifchilddocmanual| is true whenever
a part to be included by |\input| is being compiled,
and the name of the part is stored in |\childdocname|.

%%%%%%%%%%%%%%%%%%%%%%%%%%%%%%%%%%%%%%%%
\DescribeMacro{\childdocby}
Each part to be included by |\input| should start with:
%
\begin{center}
\begin{tabular}{l}
|\input{childdoc.def}|\\
|\childdocby{|\textit{main}|}|\\
\end{tabular}
\end{center}
%
The directive |\childdocby| is similar to |\childdocof|
described in \secref{sec:include},
but the subsequent selection of content must be done manually.
To that end, both |\ifchilddoc| and |\ifchilddocmanual|
will be true upon processing of a part,
and the name of the part is stored in |\childdocname|.
Note that |\jobname| will be set to the filename of the current part
so that each part receives an individual |.aux| file
that does not interfere with the |.aux| file(s) of the main document.
This behaviour can be altered by the alternative form
|\childdocby[*]{|\textit{main}|}| (with a non-empty optional argument)
which uses the |.aux| file of the main document
by setting |\jobname| to \textit{main}.

%%%%%%%%%%%%%%%%%%%%%%%%%%%%%%%%%%%%%%%%%%%%%%%%%%%%%%%%%%%%%%%%%%%%%%%%%%%%%%%%
\subsection{Driver Development}
\label{sec:driver}

The \textsf{childdoc} mechanism can also be use for the development
of definition files such as \LaTeX{} styles or classes.
This case differs from the above setup with multiple parts
included by |\include| in that no |\includeonly| should be invoked.
This can be achieved by starting the include file
(before |\ProvidesPackage|) with:
%
\begin{center}
\begin{tabular}{l}
|\input{childdoc.def}|\\
|\childdocforward{|\textit{main}|}|\\
\end{tabular}
\end{center}
%
or alternatively with:
%
\begin{center}
\begin{tabular}{l}
|\input{childdoc.def}|\\
|\childdocby{|\textit{main}|}|\\
\end{tabular}
\end{center}
%
Both forms have slightly different effects as described above.
The main file is prepared as usual, see \secref{sec:include}.

%%%%%%%%%%%%%%%%%%%%%%%%%%%%%%%%%%%%%%%%%%%%%%%%%%%%%%%%%%%%%%%%%%%%%%%%%%%%%%%%
\subsection{Legacy Detection}
\label{sec:detection}

The directive |\childdocmain| in the main file can detect
whether the complete document or merely a child is to be compiled
even without using the directive |\childdocof|.
This method is deprecated because it is less robust
and there is no compelling reason to use it;
it is merely provided for backward compatibility
and it may be removed in future versions.

If the detection mechanism is to be used,
it is mandatory to correctly specify
the filename of the main file as the argument of |\childdocmain|:
%
\begin{center}
\begin{tabular}{l}
|\input{childdoc.def}|\\
|\childdocmain{|\textit{main}|}|\\
\end{tabular}
\end{center}
%
If |\jobname| does not match the argument \textit{main} of |\childdocmain|,
it is assumed that |\jobname| points to the child file to be compiled.
When using |\childdocmain| with the main file specified as argument,
it suffices to start a child file
with just |\input{|\textit{main}|}|
without loading of the package and using |\childdocof|.
If instead all processing is done
with the appropriate \textsf{childdoc} directives,
the argument of \textit{main} of |\childdocmain| can be empty.

An alternative version of the command line processing described
in \secref{sec:commandline} using the detection mechanism reads:
%
\begin{center}
|... -jobname "|\textit{target}|" "|[\textit{flags}]%
[|\def\jobname{|\textit{dest}|}|]|\input{|\textit{main}|}"|
\end{center}

%%%%%%%%%%%%%%%%%%%%%%%%%%%%%%%%%%%%%%%%%%%%%%%%%%%%%%%%%%%%%%%%%%%%%%%%%%%%%%%%
\subsection{Manual Code}
\label{sec:manual}

In case one cannot be certain whether the definitions file |childdoc.def|
is installed on the target \TeX{} distribution
and one prefers not to ship it,
it is conceivable to paste a few relevant commands into the sources.

To that end, drop all statements |\input{childdoc.def}|
and perform the replacements as outlined below.
Instead of |\childdocmain{|\textit{main}|}| add the following code
to the top of the main file:
%
\begin{center}
\begin{tabular}{l}
|\||ifdefined\childdocname\endinput\||fi\newif\ifchilddoc|\\
|\edef\childdocname{\scantokens\expandafter{\jobname\noexpand}}|\\
|\def\childdocmain{|\textit{main}|}\||ifx\childdocmain\childdocname\||else|\\
|\childdoctrue\includeonly{\childdocname}\let\jobname\childdocmain\||fi|\\
\end{tabular}
\end{center}
%
Instead of |\childdocof{|\textit{main}|}| just include the main file
at the top of each child file:
%
\begin{center}
|\input{|\textit{main}|}|
\end{center}
%
A simple redirection |\childdocforward{|\textit{dest}|}| is achieved by:
%
\begin{center}
|\def\jobname{|\textit{dest}|}\input{\jobname}|
\end{center}
%
The redirection with prefix
|\childdocforwardprefix[|\textit{prefix}|]{|\textit{dest}|}|
is accomplished by:
%
\begin{center}
\begin{tabular}{l}
|{\edef\jobname{\scantokens\expandafter{\jobname\noexpand}}|\\
|\def\redirectjob |\textit{prefix}|#1~~~{\gdef\jobname{|\textit{dest}|#1}}|\\
|\expandafter\redirectjob\jobname~~~}\input{\jobname}|
\end{tabular}
\end{center}

In an alternative approach,
child documents can be compiled by a specific command line
without additional code or specific definitions:
%
\begin{center}
|... -jobname "|\textit{target}|" "|[\textit{flags}]%
|\includeonly{|\textit{dest}|}\input{|\textit{main}|}"|
\end{center}
%

%%%%%%%%%%%%%%%%%%%%%%%%%%%%%%%%%%%%%%%%%%%%%%%%%%%%%%%%%%%%%%%%%%%%%%%%%%%%%%%%
%%%%%%%%%%%%%%%%%%%%%%%%%%%%%%%%%%%%%%%%%%%%%%%%%%%%%%%%%%%%%%%%%%%%%%%%%%%%%%%%
\section{Information}

%%%%%%%%%%%%%%%%%%%%%%%%%%%%%%%%%%%%%%%%%%%%%%%%%%%%%%%%%%%%%%%%%%%%%%%%%%%%%%%%
\subsection{Copyright}

Copyright \copyright{} 2017--2018 Niklas Beisert

This work may be distributed and/or modified under the
conditions of the \LaTeX{} Project Public License, either version 1.3
of this license or (at your option) any later version.
The latest version of this license is in
  \url{http://www.latex-project.org/lppl.txt}
and version 1.3 or later is part of all distributions of \LaTeX{}
version 2005/12/01 or later.

This work has the LPPL maintenance status `maintained'.

The Current Maintainer of this work is Niklas Beisert.

This work consists of the files |README.txt|, |childdoc.ins| and |childdoc.dtx|
as well as the derived files |childdoc.def|, |cdocsamp.tex|
with |cdocsch1.tex|, |cdocsch2.tex|, |cdocspt3.tex|, |cdocspt4.tex|,
|cdocsdrf.tex|, |cdocsfn1.tex|, |cdocsfn2.tex|
as well as |childdoc.pdf|.

%%%%%%%%%%%%%%%%%%%%%%%%%%%%%%%%%%%%%%%%%%%%%%%%%%%%%%%%%%%%%%%%%%%%%%%%%%%%%%%%
\subsection{Files and Installation}

The package consists of the files:
%
\begin{center}
\begin{tabular}{ll}
    |README.txt|   & readme file \\
    |childdoc.ins| & installation file \\
    |childdoc.dtx| & source file \\
    |childdoc.def| & definition file \\
    |cdocsamp.tex| & sample main file \\
    |cdocsch1.tex| & sample include file \\
    |cdocsch2.tex| & sample include file \\
    |cdocspt3.tex| & sample part file \\
    |cdocspt4.tex| & sample part file \\
    |cdocsdrf.tex| & sample redirection file \\
    |cdocsfn1.tex| & sample redirection file \\
    |cdocsfn2.tex| & sample redirection file \\
    |childdoc.pdf| & manual
\end{tabular}
\end{center}
%
The distribution consists of the files
|README.txt|, |childdoc.ins| and |childdoc.dtx|.
%
\begin{itemize}
\item
Run (pdf)\LaTeX{} on |childdoc.dtx|
to compile the manual |childdoc.pdf| (this file).
\item
Run \LaTeX{} on |childdoc.ins| to create the definitions file |childdoc.def|
and the sample |cdocsamp.tex| with include files
|cdocsch1.tex|, |cdocsch2.tex|, |cdocspt3.tex|, |cdocspt4.tex|,
|cdocsdrf.tex|, |cdocsfn1.tex|, |cdocsfn2.tex|.
Then copy the file |childdoc.def| to an appropriate directory of your \LaTeX{}
distribution, e.g.\ \textit{texmf-root}|/tex/latex/childdoc|.
\end{itemize}

%%%%%%%%%%%%%%%%%%%%%%%%%%%%%%%%%%%%%%%%%%%%%%%%%%%%%%%%%%%%%%%%%%%%%%%%%%%%%%%%
\subsection{Related CTAN Packages}

There are several other packages which offer a similar functionality:
%
\begin{itemize}
\item
The packages
\href{http://ctan.org/pkg/docmute}{\textsf{docmute}},
\href{http://ctan.org/pkg/includex}{\textsf{includex}} and
\href{http://ctan.org/pkg/standalone}{\textsf{standalone}}
provide commands to include only the document body of
a child file thus allowing both files to be compiled individually.
\item
The packages \href{http://ctan.org/pkg/subdocs}{\textsf{subdocs}}
and \href{http://ctan.org/pkg/subfiles}{\textsf{subfiles}}
provide structures in which the main and child documents can be
encapsulated and allowing them to be compiled individually.
The inclusion mechanism is different from the conventional |\include|.
\item
The package \href{http://ctan.org/pkg/combine}{\textsf{combine}}
is an elaborate solution to combine several documents into one.
\end{itemize}
%
See also the CTAN topic \href{http://ctan.org/topic/subdocs}{\textsf{subdocs}}
for further related packages.
The present package differs from the above solutions in that
a document structure constructed with the conventional |\include| mechanism
just needs two extra commands at the top of every file
such that all constituent files can be compiled individually.

%%%%%%%%%%%%%%%%%%%%%%%%%%%%%%%%%%%%%%%%%%%%%%%%%%%%%%%%%%%%%%%%%%%%%%%%%%%%%%%%
%\subsection{Feature Suggestions}
%
%The following is a list of features which may be useful for future
%versions of this package:
%%
%\begin{itemize}
%\item
%\ldots
%\end{itemize}

%%%%%%%%%%%%%%%%%%%%%%%%%%%%%%%%%%%%%%%%%%%%%%%%%%%%%%%%%%%%%%%%%%%%%%%%%%%%%%%%
\subsection{Revision History}

%%%%%%%%%%%%%%%%%%%%%%%%%%%%%%%%%%%%%%%%
\paragraph{v2.0:} 2018/12/30

\begin{itemize}
\item
immediate forward processing
\item
added |\childdocby| mechanism
\item
manual restructured
\end{itemize}

%%%%%%%%%%%%%%%%%%%%%%%%%%%%%%%%%%%%%%%%
\paragraph{v1.6:} 2018/01/17

\begin{itemize}
\item
application for development of include files
\item
corrections to manual
\end{itemize}

%%%%%%%%%%%%%%%%%%%%%%%%%%%%%%%%%%%%%%%%
\paragraph{v1.5:} 2017/05/21

\begin{itemize}
\item
more complete structuring introduced
\item
|\childdocof| introduced
\item
|\childdoc| renamed to |\childdocmain|
\item
|\childredirect| renamed to |\childdocforward| and |\childdocforwardprefix|
and functionality expanded
\end{itemize}

%%%%%%%%%%%%%%%%%%%%%%%%%%%%%%%%%%%%%%%%
\paragraph{v1.0:} 2017/04/27

\begin{itemize}
\item
manual and install package
\item
first version published on CTAN
\end{itemize}

%%%%%%%%%%%%%%%%%%%%%%%%%%%%%%%%%%%%%%%%
\paragraph{v0.6:} 2017/04/26

\begin{itemize}
\item
redirection mechanism added
\end{itemize}

%%%%%%%%%%%%%%%%%%%%%%%%%%%%%%%%%%%%%%%%
\paragraph{v0.5:} 2017/04/26

\begin{itemize}
\item
functionality in definition file
\end{itemize}


%%%%%%%%%%%%%%%%%%%%%%%%%%%%%%%%%%%%%%%%%%%%%%%%%%%%%%%%%%%%%%%%%%%%%%%%%%%%%%%%
%%%%%%%%%%%%%%%%%%%%%%%%%%%%%%%%%%%%%%%%%%%%%%%%%%%%%%%%%%%%%%%%%%%%%%%%%%%%%%%%
%%%%%%%%%%%%%%%%%%%%%%%%%%%%%%%%%%%%%%%%%%%%%%%%%%%%%%%%%%%%%%%%%%%%%%%%%%%%%%%%
\appendix

\settowidth\MacroIndent{\rmfamily\scriptsize 000\ }

 \DocInput{childdoc.dtx}

\end{document}
%</driver>
% \fi
%
% %%%%%%%%%%%%%%%%%%%%%%%%%%%%%%%%%%%%%%%%%%%%%%%%%%%%%%%%%%%%%%%%%%%%%%%%%%%%%%
% %%%%%%%%%%%%%%%%%%%%%%%%%%%%%%%%%%%%%%%%%%%%%%%%%%%%%%%%%%%%%%%%%%%%%%%%%%%%%%
% \section{Sample}
%\iffalse
%<*samplemain>
%\fi
%
% The following presents a sample document
% with two chapters, two parts, a title page,
% a compile flag as well as three forwarding files to set the flag.
% It consists of eight |.tex| files:
% \begin{center}
% \begin{tabular}{ll}
% |cdocsamp.tex|&main file\\
% |cdocsch1.tex|&include file for chapter 1\\
% |cdocsch2.tex|&include file for chapter 2\\
% |cdocspt3.tex|&include file for part 3\\
% |cdocspt4.tex|&include file for part 4\\
% |cdocsdrf.tex|&forwarding file for main file in draft mode\\
% |cdocsfi1.tex|&forwarding file for final version of chapter 1\\
% |cdocsfi2.tex|&forwarding file for final version of chapter 2\\
% \end{tabular}
% \end{center}
% Each of the eight files can be compiled directly by the \LaTeX{} compiler.
%
% %%%%%%%%%%%%%%%%%%%%%%%%%%%%%%%%%%%%%%
% \paragraph{Main File.}
%
% The main file is called |cdocsamp.tex|.
%
% Load the \textsf{childdoc} definitions and
% declare the filename for the main document:
%    \begin{macrocode}
\input{childdoc.def}
\childdocmain{}
%    \end{macrocode}

% Optional override for |\version| flag:
%    \begin{macrocode}
%%\ifchilddoc\else\providecommand{\version}{draft}\fi
%    \end{macrocode}

% Define the default values for the |\version| flag
% (|final| for the main file and |draft| for childs):
%    \begin{macrocode}
\ifchilddoc
\providecommand{\version}{draft}
\else
\providecommand{\version}{final}
\fi
%    \end{macrocode}

% Load the standard document class:
%    \begin{macrocode}
\documentclass[12pt]{article}
%    \end{macrocode}

% Start the document body:
%    \begin{macrocode}
\begin{document}
%    \end{macrocode}

% Declare a title page.
% Print title, part of document being processed and version flag:
%    \begin{macrocode}
\addtocounter{page}{-1}
\begin{center}
{\LARGE\bfseries{}childdoc example\par}
\vspace{1cm}
\ifchilddoc
\ifchilddocmanual part\else chapter\fi:
`\childdocname' of `\childdocjob'\par
\else
main document: `\childdocjob'\par
\fi
version: \version\par
\end{center}
\newpage
%    \end{macrocode}

% Manually include selected file,
% otherwise process as usual:
%    \begin{macrocode}
\ifchilddocmanual
\section*{part `\childdocname'}
\input{\childdocname}
\else
%    \end{macrocode}

% Include the two chapters:
%    \begin{macrocode}
\include{cdocsch1}
\include{cdocsch2}
%    \end{macrocode}

% Include the two parts unless only chapters should be displayed:
%    \begin{macrocode}
\ifchilddoc\else
\section{part three}
\input{cdocspt3}
\section{part four}
\input{cdocspt4}
\fi
%    \end{macrocode}

% Process as usual until here:
%    \begin{macrocode}
\fi
%    \end{macrocode}

% End of document body:
%    \begin{macrocode}
\end{document}
%    \end{macrocode}
%\iffalse
%</samplemain>
%\fi
%
% %%%%%%%%%%%%%%%%%%%%%%%%%%%%%%%%%%%%%%
% \paragraph{Chapter Include Files.}
%
% The include files are called |cdocsch1.tex| and |cdocsch2.tex|.
%
%\iffalse
%<*samplechap1|samplechap2>
%\fi

% Optional override for |\version| flag:
%    \begin{macrocode}
%%\providecommand{\version}{final}
%    \end{macrocode}

% Include the main document:
%    \begin{macrocode}
\input{childdoc.def}
\childdocof{cdocsamp}
%    \end{macrocode}

%\iffalse
%</samplechap1|samplechap2>
%\fi
%
%\iffalse
%<*samplechap1>
%\fi
% Some text for chapter 1:
%    \begin{macrocode}
\section{one}
some text in chapter one
%    \end{macrocode}

%\iffalse
%</samplechap1>
%\fi
% Some text for chapter 2:
%\iffalse
%<*samplechap2>
%\fi
%    \begin{macrocode}
\section{two}
more text in chapter two
%    \end{macrocode}

%\iffalse
%</samplechap2>
%\fi
%
% %%%%%%%%%%%%%%%%%%%%%%%%%%%%%%%%%%%%%%
% \paragraph{Part Include Files.}
%
% The include files are called |cdocspt3.tex| and |cdocspt4.tex|.
%
%\iffalse
%<*samplepart3|samplepart4>
%\fi

% Optional override for |\version| flag:
%    \begin{macrocode}
%%\providecommand{\version}{final}
%    \end{macrocode}

% Include the main document:
%    \begin{macrocode}
\input{childdoc.def}
\childdocby{cdocsamp}
%    \end{macrocode}

%\iffalse
%</samplepart3|samplepart4>
%\fi
%
%\iffalse
%<*samplepart3>
%\fi
% Some text for part 3:
%    \begin{macrocode}
some text in part three
%    \end{macrocode}

%\iffalse
%</samplepart3>
%\fi
% Some text for part 4:
%\iffalse
%<*samplepart4>
%\fi
%    \begin{macrocode}
more text in part four
%    \end{macrocode}

%\iffalse
%</samplepart4>
%\fi
%
% %%%%%%%%%%%%%%%%%%%%%%%%%%%%%%%%%%%%%%
% \paragraph{Forwarding for a Complete Draft.}
%
% The following forwarding file |cdocsdrf.tex|
% compiles the main document in draft mode:
%\iffalse
%<*sampledraft>
%\fi
%    \begin{macrocode}
\def\version{draft}
\input{childdoc.def}
\childdocforward{cdocsamp}
%    \end{macrocode}

%\iffalse
%</sampledraft>
%\fi
%
% %%%%%%%%%%%%%%%%%%%%%%%%%%%%%%%%%%%%%%
% \paragraph{Forwarding for Final Version of the Chapters.}
%
% The following forwarding files |cdocsfn1.tex| and |cdocsfn2.tex|
% (with identical content)
% compile the final versions of the child documents
% |cdocsch1.tex| and |cdocsch2.tex|, respectively:
%\iffalse
%<*samplefinal>
%\fi
%    \begin{macrocode}
\def\version{final}
\input{childdoc.def}
\childdocforwardprefix[cdocsamp]{cdocsfn}{cdocsch}
%    \end{macrocode}

%\iffalse
%</samplefinal>
%\fi
%
% %%%%%%%%%%%%%%%%%%%%%%%%%%%%%%%%%%%%%%
% \paragraph{Command Line Processing.}
%
% The following three command lines generate the output files
% |cdocscld|, |cdocscl1| and |cdocscl2|
% which should be identical to
% |cdocsdrf|, |cdocsch1| and |cdocsfn2|, respectively:
% \begin{center}
% \begin{tabular}{l}
% |latex -jobname cdocscld \|\\
% |  "\def\version{draft}\input{childdoc.def}\childdocforward{cdocsamp}"|\\
% |latex -jobname cdocscl1 \|\\
% |  "\input{childdoc.def}\childdocforward[cdocsamp]{cdocsch1}"|\\
% |latex -jobname cdocscl2 \|\\
% |  "\def\version{final}\input{childdoc.def}\childdocforward{cdocsch2}"|
% \end{tabular}
% \end{center}
% Note that the trailing backslash on each first line
% merely continues the input to the second line
% (for convenient cut ant paste).
% Furthermore, the command |latex| can be replaced by any
% of its alternative versions such as |pdflatex|.
%
% %%%%%%%%%%%%%%%%%%%%%%%%%%%%%%%%%%%%%%%%%%%%%%%%%%%%%%%%%%%%%%%%%%%%%%%%%%%%%%
% %%%%%%%%%%%%%%%%%%%%%%%%%%%%%%%%%%%%%%%%%%%%%%%%%%%%%%%%%%%%%%%%%%%%%%%%%%%%%%
% \section{Implementation}
%\iffalse
%<*package>
%\fi
%
% This section describes the definitions file |childdoc.def|.

% The definitions cannot be loaded using |\usepackage| or |\RequirePackage|
% which has a mechanism to prevent loading a style file more than once.
% When loading the definitions by means of |\input|
% multiple instances have to be prevented manually:
%\iffalse
%This code needs to be before the `\ProvidesFile' directive
%which is defined at the beginning of this file.
%Therefore it is also placed there and commented out here.
%</package>
%<*discard>
%\fi
%    \begin{macrocode}
\ifdefined\childdocmain\endinput\fi
%    \end{macrocode}
%\iffalse
%</discard>
%<*package>
%\fi
%
% \macro{\ifchilddoc}
% \macro{\ifchilddocmanual}
% The conditional |\ifchilddoc| tells whether a
% child (true) or main (false) document is being compiled.
% The conditional |\ifchilddocmanual| tells whether
% the |\includeonly| mechanism is used (false) or
% the selection of child files must be performed manually (true).
% The definitions initialise to false:
%    \begin{macrocode}
\newif\ifchilddoc
\newif\ifchilddocmanual
%    \end{macrocode}

% \macro{\childdocname}
% \macro{\childdocjob}
% The macro |\childdocname| stores the name of the main document
% to be compiled. The macro |\childdocjob| stores the name of
% the document on which the \LaTeX{} compiler was originally invoked.
% The content of |\jobname| cannot be compared
% to filenames specified in the source due to different catcodes.
% The following code rescans |\jobname|, stores the result
% in |\childdocname| and saves a copy in |\childdocjob|:
%    \begin{macrocode}
\edef\childdocname{\scantokens\expandafter{\jobname\noexpand}}
\let\childdocjob\childdocname
%    \end{macrocode}

% \macro{\childdocdisable}
% The macro |\childdocdisable| prevents the main file
% from being processed more than once.
% At this stage, the main document command |\childdocmain|
% is assumed to be called once again where it should do nothing.
% Any subsequent call to it should prevent
% a secondary processing of the main document
% It overwrites the forwarding commands
% |\childdocof| and |\childdocforward|
% with empty macros to prevent further inclusions of the main document:
%    \begin{macrocode}
\newcommand{\childdocdisable}
{
  \renewcommand{\childdocmain}[1]{\renewcommand{\childdocmain}[1]{\endinput}}
  \renewcommand{\childdocof}[1]{}
  \renewcommand{\childdocby}[2][]{}
  \renewcommand{\childdocforward}[2][]{}
  \renewcommand{\childdocdisable}{}
}
%    \end{macrocode}

% \macro{\childdocmain}
% The macro |\childdocmain| is to be called at the top of the main file
% with nothing or the main filename (without extension) as argument.
% First, it breaks loops.
% If the argument is not empty and does not match |\childdocname|
% (which is set by the first inclusion of |childdoc.def|),
% |\ifchilddoc| is set to true, |\includeonly| is applied to the child file
% and |\jobname| is set to the main file
% (for proper handling of |.aux| files):
%    \begin{macrocode}
\newcommand{\childdocmain}[1]
{
  \childdocdisable\childdocmain{}
  \if?#1?\else
    \begingroup
      \def\childdoctmp{#1}
      \ifx\childdoctmp\childdocname
        \def\childdoctmp{}
      \else
        \def\childdoctmp
        {
          \childdoctrue
          \includeonly{\childdocname}
          \def\childdocjob{#1}
          \def\jobname{#1}
        }
      \fi
      \expandafter
    \endgroup
    \childdoctmp
  \fi
}
%    \end{macrocode}

% \macro{\childdocof}
% The command |\childdocof| redirects
% compilation to the main file |#1|.
%    \begin{macrocode}
\newcommand{\childdocof}[1]
{
  \childdocdisable
  \childdoctrue
  \includeonly{\childdocname}
  \def\jobname{#1}
  \def\childdocjob{#1}
  \input{#1}
}
%    \end{macrocode}

% \macro{\childdocby}
% The command |\childdocby| ....
%    \begin{macrocode}
\newcommand{\childdocby}[2][]
{
  \childdocdisable
  \childdoctrue
  \childdocmanualtrue
  \if?#1?\else
    \def\jobname{#2}
  \fi
  \def\childdocjob{#2}
  \input{#2}
  \endinput
}
%    \end{macrocode}

% \macro{\childdocforward}
% The command |\childdocforward| redirects
% compilation to the main file or
% (if the optional argument is given) a child file.
% Parameters are set as if the main file
% or a child file starting with |\childdocof| was compiled.
% Then compilation is handed over to the main file:
%    \begin{macrocode}
\newcommand{\childdocforward}[2][]
{
  \begingroup
    \if?#1?
      \def\childdoctmp
      {
        \def\childdocname{#2}
        \def\childdocjob{#2}
        \def\jobname{#2}
        \input{#2}
        \endinput
      }
    \else
      \def\childdoctmp
      {
        \childdocdisable
        \def\childdocname{#2}
        \childdoctrue
        \includeonly{#2}
        \def\childdocjob{#1}
        \def\jobname{#1}
        \input{#1}
        \endinput
      }
    \fi
    \expandafter
  \endgroup
  \childdoctmp
}
%    \end{macrocode}

% \macro{\childdocforwardprefix}
% The command |\childdocforwardprefix| redirects
% compilation to the main or a child file by means of a pattern.
% The prefix |#1| in the current filename is replaced by |#2|
% and the suffix of the current filename is kept
% (it is assumed that the filename does not contain the substring `|~~~|'
% which is used as a delimiter).
% Compilation is handed over to the new file by |\childdocforward|:
%    \begin{macrocode}
\newcommand{\childdocforwardprefix}[3][]
{
  \begingroup
    \def\childdocextract #2##1~~~{\def\childdoctmp{\childdocforward[#1]{#3##1}}}
    \expandafter\childdocextract\childdocname~~~
    \expandafter
  \endgroup
  \childdoctmp
}
%    \end{macrocode}

% \macro{\childdoc}
% The deprecated macro |\childdoc| is a legacy version of |\childdocmain|:
%    \begin{macrocode}
\newcommand{\childdoc}{\childdocmain}
%    \end{macrocode}

% \macro{\childdocredirect}
% The deprecated macro |\childdocredirect| is a legacy version
% of |\childdocforward| and |\childdocforwardprefix|:
%    \begin{macrocode}
\newcommand{\childdocredirect}[2][]
{
  \begingroup
    \if?#1?
      \def\childdoctmp{\childdocforward{#2}}
    \else
      \def\childdoctmp{\childdocforwardprefix{#1}{#2}}
    \fi
    \expandafter
  \endgroup
  \childdoctmp
}
%    \end{macrocode}

%\iffalse
%</package>
%\fi
%
\endinput
|\\
|\childdocforward[|\textit{main}|]{|\textit{dest}|}|\\
\end{tabular}
\end{center}
%
The argument \textit{dest} is the destination file
(without extension).
It should be the main file or one of the child files.
Note that further \textsf{childdoc} directives
such as |\childdocof| and |\childdocforward|
in the indicated file will be processed in this form.
The optional argument \textit{main}
passes on directly to the main file \textit{main}
while pretending to compile the child \textit{dest}.
This form behaves as if \textit{dest}
issues |\childdocof{|\textit{main}|}| right away,
and no further \textsf{childdoc} directives will be processed.

%%%%%%%%%%%%%%%%%%%%%%%%%%%%%%%%%%%%%%%%
\DescribeMacro{\...prefix}
In the alternative form |\childdocforwardprefix|,
%
\begin{center}
\begin{tabular}{l}
|% \iffalse
%
% childdoc.dtx Copyright (C) 2017-2018 Niklas Beisert
%
% This work may be distributed and/or modified under the
% conditions of the LaTeX Project Public License, either version 1.3
% of this license or (at your option) any later version.
% The latest version of this license is in
%   http://www.latex-project.org/lppl.txt
% and version 1.3 or later is part of all distributions of LaTeX
% version 2005/12/01 or later.
%
% This work has the LPPL maintenance status `maintained'.
%
% The Current Maintainer of this work is Niklas Beisert.
%
% This work consists of the files childdoc.dtx and childdoc.ins
% and the derived files childdoc.def and cdocsamp.tex with
% cdocsch1.tex, cdocsch2.tex, cdocsdrf.tex, cdocsfn1.tex, cdocsfn2.tex.
%
%<package>\ifdefined\childdocmain\endinput\fi
%<package>\ProvidesFile{childdoc.def}[2018/12/30 v2.0 child document driver]
%<samplemain>\ProvidesFile{cdocsamp.tex}[2018/12/30 v2.0 sample for childdoc]
%<*driver>
%\ProvidesFile{childdoc.drv}[2018/12/30 v2.0 childdoc reference manual file]
\PassOptionsToClass{10pt,a4paper}{article}
\documentclass{ltxdoc}

\usepackage[margin=35mm]{geometry}
\usepackage{hyperref}
\usepackage{hyperxmp}
\usepackage[usenames]{color}

\hypersetup{colorlinks=true}
\hypersetup{pdfstartview=FitH}
\hypersetup{pdfpagemode=UseNone}
\hypersetup{pdfsource={}}
\hypersetup{pdflang={en-UK}}
\hypersetup{pdfcopyright={Copyright 2017-2018 Niklas Beisert.
  This work may be distributed and/or modified under the
  conditions of the LaTeX Project Public License, either version 1.3
  of this license or (at your option) any later version.}}
\hypersetup{pdflicenseurl={http://www.latex-project.org/lppl.txt}}
\hypersetup{pdfcontactaddress={ETH Zurich, ITP, HIT K,
  Wolfgang-Pauli-Strasse 27}}
\hypersetup{pdfcontactpostcode={8093}}
\hypersetup{pdfcontactcity={Zurich}}
\hypersetup{pdfcontactcountry={Switzerland}}
\hypersetup{pdfcontactemail={nbeisert@itp.phys.ethz.ch}}
\hypersetup{pdfcontacturl={http://people.phys.ethz.ch/\xmptilde nbeisert/}}

\newcommand{\secref}[1]{\hyperref[#1]{section \ref*{#1}}}

\parskip1ex
\parindent0pt
\let\olditemize\itemize
\def\itemize{\olditemize\parskip0pt}

\begin{document}

\title{The \textsf{childdoc} Package}
\hypersetup{pdftitle={The childdoc Package}}
\author{Niklas Beisert\\[2ex]
  Institut f\"ur Theoretische Physik\\
  Eidgen\"ossische Technische Hochschule Z\"urich\\
  Wolfgang-Pauli-Strasse 27, 8093 Z\"urich, Switzerland\\[1ex]
  \href{mailto:nbeisert@itp.phys.ethz.ch}
  {\texttt{nbeisert@itp.phys.ethz.ch}}}
\hypersetup{pdfauthor={Niklas Beisert}}
\hypersetup{pdfsubject={Manual for the LaTeX2e Package childdoc}}
\date{30 December 2018, \textsf{v2.0}}
\maketitle

\begin{abstract}\noindent
\textsf{childdoc} is a \LaTeXe{} package
that enables the direct compilation
of document sections included by |\include|
to individual files.
\end{abstract}

\begingroup
\parskip0ex
\tableofcontents
\endgroup

%%%%%%%%%%%%%%%%%%%%%%%%%%%%%%%%%%%%%%%%%%%%%%%%%%%%%%%%%%%%%%%%%%%%%%%%%%%%%%%%
%%%%%%%%%%%%%%%%%%%%%%%%%%%%%%%%%%%%%%%%%%%%%%%%%%%%%%%%%%%%%%%%%%%%%%%%%%%%%%%%
\section{Introduction}

\LaTeX{} provides a mechanism to structure a large document (such as a book)
into a main file and several child files (containing the chapters)
using the |\include| command.
This mechanism is beneficial for documents
which span hundreds of pages in order to
make the source file(s) more manageable.
Moreover, compilation can be restricted to
selected child files by means of the |\includeonly| command.
The latter feature can be used to reduce the compilation time while editing
(this was significantly more useful in the earlier days of \LaTeX{})
or to generate a smaller document which is easier to navigate.
Another application of |\includeonly| is to generate
documents consisting of selected parts of the complete document.

However, there are a few drawbacks of the plain |\include| mechanism:
\begin{itemize}
\item
The child files cannot be compiled on their own,
they can only be compiled via the main file.
A naive editing environment
(such as a text editor with an option
to have the current file processed by \LaTeX)
may require one to switch to the main file before compiling;
attempting to compile the child file produces errors.
\item
The main file must be modified (each time)
to adjust the |\includeonly| command
to the present needs. This easily leaves the main file in a messy state.
\item
The generated document will always carry the filename
of the main document. This is inconvenient if
several child files are to be compiled and
to be kept for distribution.
\end{itemize}

The present package provides a simple interface
to make child files individually compilable by \LaTeX{}.
Compiling a child file then has the same effect as compiling
the main file with an |\includeonly| command
to select the appropriate child.
Moreover the generated document will carry the name of the child
rather than the main file.
This resolves all three above issues.

This feature is meant to make the editing of books,
thesis documents and lecture notes somewhat more convenient.
However, the package can also be used efficiently for
composing a series of documents (such as exercise sheets)
which are typically distributed individually.
It then assists the author in generating the individual documents
(potentially in different versions)
as well as a document containing the collected series.
Another application is in developing style files
or other kinds of included material
where compilation of the style file could redirect
to a sample or test file.

%%%%%%%%%%%%%%%%%%%%%%%%%%%%%%%%%%%%%%%%%%%%%%%%%%%%%%%%%%%%%%%%%%%%%%%%%%%%%%%%
%%%%%%%%%%%%%%%%%%%%%%%%%%%%%%%%%%%%%%%%%%%%%%%%%%%%%%%%%%%%%%%%%%%%%%%%%%%%%%%%
\section{Usage}

First of all, the package \textsf{childdoc} is \emph{not} a standard
\LaTeXe{} |.sty| style file! Therefore it needs to be invoked in
a non-standard way.

%%%%%%%%%%%%%%%%%%%%%%%%%%%%%%%%%%%%%%%%%%%%%%%%%%%%%%%%%%%%%%%%%%%%%%%%%%%%%%%%
\subsection{Included Files}
\label{sec:include}

%%%%%%%%%%%%%%%%%%%%%%%%%%%%%%%%%%%%%%%%
\DescribeMacro{\childdocmain}
To use the package, add the commands
\begin{center}
\begin{tabular}{l}
|\input{childdoc.def}|\\
|\childdocmain{}|\\
\end{tabular}
\end{center}
at the very top of the main \LaTeX{} file,
in particular \emph{before} the |\documentclass| statement!
The argument of |\childdocmain| should be left empty
(but it must be present).

%%%%%%%%%%%%%%%%%%%%%%%%%%%%%%%%%%%%%%%%
\DescribeMacro{\childdocof}
Furthermore, add the commands
\begin{center}
\begin{tabular}{l}
|\input{childdoc.def}|\\
|\childdocof{|\textit{main}|}|\\
\end{tabular}
\end{center}
at the top of every child file \textit{child}
which is included by |\include{|\textit{child}|}|
from within the main file
(or at least for those files to be compiled individually).
The argument \textit{main} must be the filename of the main file.

There are a couple of
considerations in setting up the main and child documents:

%%%%%%%%%%%%%%%%%%%%%%%%%%%%%%%%%%%%%%%%
\paragraph{Restrictions.}

Please note the following restrictions:
\begin{itemize}
\item
|\childdocmain| must be called with one argument \textit{main}
to ensure compatibility with earlier version of the package.
It must either be empty (|\childdocmain{}|)
or precisely match the filename of the main file in which it is specified.
See \secref{sec:detection} for further information.
\item
The filename \textit{main} must be specified without the |.tex| extension.
\item
The filename \textit{main} is case sensitive
(even in case-insensitive file systems)
due to internal string comparison.
\item
The argument \textit{main} should be fully expanded, it cannot be a macro.
\item
Subdirectories and special characters should be avoided in filenames.
\item
The command |\childdocmain{|\textit{main}|}| must be followed by a whitespace.
It should not be followed immediately by another command
or by a comment mark `|%|'.
This is because the \TeX{} parser reads the token immediately following
the argument of |\childdocmain| and puts it
at the beginning of every child section;
however, a white\-space is ignored.
\end{itemize}

%%%%%%%%%%%%%%%%%%%%%%%%%%%%%%%%%%%%%%%%
\paragraph{Content of Main File.}

It is advisable to place all content in the child files included by |\include|.
Any output contained in the main file will appear in all child documents
unless suppressed manually;
it cannot be suppressed automatically by the |\includeonly| directive
and thus should normally be avoided.
A method to include some content in the main file
by means of conditional processing is described in \secref{sec:conditional}.

%%%%%%%%%%%%%%%%%%%%%%%%%%%%%%%%%%%%%%%%
\paragraph{Page Numbering.}

When only a part of the document is compiled,
the appropriate numbering of pages
(as well as other status parameters)
is determined from the |.aux| files.
The latter contain information from previous passes.
However this information needs to propagate through
all intermediate child documents.
Therefore the page numbering in child documents may well
be inconsistent until the complete document is compiled at least once.

A useful (if unconventional) way to always ensure a consistent
page numbering is to restart the numbering in each child document
and denote the pages by `\textit{child}|.|\textit{page}'
where \textit{child} represents the chapter/section number of the child file.
This can be achieved by the command
|\numberwithin{page}{|\textit{child}|}|
of the \textsf{amsmath} package
where \textit{child} can be |chapter| or |section|
depending on the chosen structuring.
Alternatively, one can modify the macro |\thepage| appropriately
and reset the counter |page| at the start of each child file.

%%%%%%%%%%%%%%%%%%%%%%%%%%%%%%%%%%%%%%%%%%%%%%%%%%%%%%%%%%%%%%%%%%%%%%%%%%%%%%%%
\subsection{Conditional Processing}
\label{sec:conditional}

The package provides a mechanism to compile different versions
of a document. To customise the versions further some conditional processing
can come in handy to distinguish which version is being compiled.
The package provides two macros to describe the compilation context:

%%%%%%%%%%%%%%%%%%%%%%%%%%%%%%%%%%%%%%%%
\DescribeMacro{\ifchilddoc}
The conditional |\ifchilddoc| distinguishes between the compilation of
child documents and the main document:
%
\begin{center}
|\ifchilddoc |\textit{child-code}| |[|\||else |\textit{main-code}]| \||fi|
\end{center}

%%%%%%%%%%%%%%%%%%%%%%%%%%%%%%%%%%%%%%%%
\DescribeMacro{\childdocname}
\DescribeMacro{\childdocjob}
The macro |\childdocname| contains the filename (without extension)
of the main or child file being processed.
Note that |\childdocjob| will always contain the name of the main file.

%%%%%%%%%%%%%%%%%%%%%%%%%%%%%%%%%%%%%%%%
\paragraph{Title Page.}

Conditional processing can be used to include a title or banner page
in the main document when proper precautions are taken.
Importantly, the code in the main file should ensure that the page counter
(as well as other status parameters which are stored in the |.aux| files)
takes the same value after the conditional processing.
Otherwise the page numbers may take divergent values
depending on which part is compiled.

For example, a title page could be declared by:
%
\begin{center}
\begin{tabular}{l}
|\ifchilddoc\||else|\\
|\addtocounter{page}{-1}|\\
\textit{code for title page}\\
|\newpage|\\
|\||fi|
\end{tabular}
\end{center}
%
A banner page for the child documents can be generated by:
%
\begin{center}
\begin{tabular}{l}
|\ifchilddoc|\\
|\addtocounter{page}{-1}|\\
\textit{code for banner page}\\
|\newpage|\\
|\||fi|
\end{tabular}
\end{center}
%
Here one could write a message such as:
\begin{center}
|This is the part \childdocname{} of \childdocjob{}.|
\end{center}

%%%%%%%%%%%%%%%%%%%%%%%%%%%%%%%%%%%%%%%%%%%%%%%%%%%%%%%%%%%%%%%%%%%%%%%%%%%%%%%%
\subsection{Flags}
\label{sec:flags}

The package makes it easy to generate different versions
of the main or child documents.
To this end compilation flags can be defined
and assigned different default values.
They will be particularly useful in conjunction
with the forwarding mechanism described in \secref{sec:forward}.

For example, it may be useful to have a flag |\version|
which can be set to |draft| or |final|.
The document source will contain some conditional code
depending on the value of |\version|.
Suppose further, the flag should default to |final| for the main file
and to |draft| for child files
which is a natural assignment for editing the document.
This is achieved by placing the following code
in the preamble of the main document
(below the |\childdocmain| directive):
%
\begin{center}
\begin{tabular}{l}
|\ifchilddoc|\\
|\providecommand{\version}{draft}|\\
|\||else|\\
|\providecommand{\version}{final}|\\
|\||fi|
\end{tabular}
\end{center}
%
The definition by |\providecommand| makes sure
that previous definitions are not overwritten.
Further statements |\providecommand{\version}{...}|
can thus be added before the above code to override it.

For the main file, one might add a line
(between |\childdocmain| and the above block)
%
\begin{center}
|%\ifchilddoc\||else\providecommand{\version}{draft}\||fi|
\end{center}
%
which can be uncommented to produce a draft version.
Likewise one can add a line to the very top of a child file
(above the |\childdocof{|\textit{main}|}| directive)
%
\begin{center}
|%\providecommand{\version}{final}|
\end{center}
%
which can be uncommented to produce the final version of this child document.

%%%%%%%%%%%%%%%%%%%%%%%%%%%%%%%%%%%%%%%%%%%%%%%%%%%%%%%%%%%%%%%%%%%%%%%%%%%%%%%%
\subsection{Forwarding}
\label{sec:forward}

Different versions of the main or child documents
using compilation flags as described in \secref{sec:flags}
can be (permanently) stored in different files
for convenient compilation, viewing and distribution.
To this end, the package defines a command
to pass on compilation to a different file:

%%%%%%%%%%%%%%%%%%%%%%%%%%%%%%%%%%%%%%%%
\DescribeMacro{\childdocforward}
The command |\childdocforward| redirects processing to
another source file:
%
\begin{center}
\begin{tabular}{l}
|\input{childdoc.def}|\\
|\childdocforward[|\textit{main}|]{|\textit{dest}|}|\\
\end{tabular}
\end{center}
%
The argument \textit{dest} is the destination file
(without extension).
It should be the main file or one of the child files.
Note that further \textsf{childdoc} directives
such as |\childdocof| and |\childdocforward|
in the indicated file will be processed in this form.
The optional argument \textit{main}
passes on directly to the main file \textit{main}
while pretending to compile the child \textit{dest}.
This form behaves as if \textit{dest}
issues |\childdocof{|\textit{main}|}| right away,
and no further \textsf{childdoc} directives will be processed.

%%%%%%%%%%%%%%%%%%%%%%%%%%%%%%%%%%%%%%%%
\DescribeMacro{\...prefix}
In the alternative form |\childdocforwardprefix|,
%
\begin{center}
\begin{tabular}{l}
|\input{childdoc.def}|\\
|\childdocforwardprefix[|\textit{main}|]{|\textit{prefix}|}{|\textit{dest}|}|
\end{tabular}
\end{center}
%
the destination file is determined by a pattern
depending on the current file:
To make this work, the current file must be called
`{\textit{prefix}\hspace{0.2em}\textit{suffix}}'
with \textit{prefix} matching precisely the argument.
Processing is then passed on to the file
`{\textit{dest}\hspace{0.2em}\textit{suffix}}'.
Surely, the same effect is achieved by
directly specifying the
argument `{\textit{dest}\hspace{0.2em}\textit{suffix}}'
in the first form.
However, that requires to set up a different file
for each child. With the alternative form of the command
all these files can have exactly the same content
which simplifies setting them up and maintaining them.

For example, the following file |draft.tex|
with a compilation flag |\version| as described in \secref{sec:flags}
compiles the main document as a draft:
%
\begin{center}
\begin{tabular}{l}
|\def\version{draft}|\\
|\input{childdoc.def}|\\
|\childdocforward{|\textit{main}|}|
\end{tabular}
\end{center}
%
Likewise, the following files |final|\textit{nn}|.tex|
compile the final version of the child document
|child|\textit{nn}|.tex|:
%
\begin{center}
\begin{tabular}{l}
|\def\version{final}|\\
|\input{childdoc.def}|\\
|\childdocforwardprefix{final}{child}|
\end{tabular}
\end{center}
%

Note that when several versions of a main file and/or of each child file
are to be generated, it may be convenient to set up a |Makefile| or
shell script to automatise the process.

%%%%%%%%%%%%%%%%%%%%%%%%%%%%%%%%%%%%%%%%%%%%%%%%%%%%%%%%%%%%%%%%%%%%%%%%%%%%%%%%
\subsection{Command Line Processing}
\label{sec:commandline}

The effect of redirection files can also be achieved by invoking
the \LaTeX{} compiler with a more elaborate command line.
Most conveniently this should be done as part
of a shell script or a |Makefile|.

When using \textsf{childdoc} in the main file, the following
command lines effectively perform a redirection
(note that depending on the shell being used,
backslashes may have to be doubled: `|\|' $\to$ `|\\|'):
%
\begin{center}
|... -jobname "|\textit{target}|" |\\|"|[\textit{flags}]%
|\input{childdoc.def}\childdocforward[|\textit{main}|]{|\textit{dest}|}"|
\end{center}
%
Here \textit{target} is the name of the output file,
\textit{main} is the name of the main file
and \textit{dest} is the name of the main or child file to be processed
(all filenames without extensions).
The optional argument \textit{main} can be omitted
if \textit{main} matches \textit{dest}.
Optionally, compilation \textit{flags} can be defined via |\def| commands.
This command line makes the \TeX{} engine believe
it is compiling the file \textit{target}
whose content is specified as the latter parameter.
The provided code then forwards the processing to
\textit{main} or \textit{dest} as described in \secref{sec:forward}.

%%%%%%%%%%%%%%%%%%%%%%%%%%%%%%%%%%%%%%%%%%%%%%%%%%%%%%%%%%%%%%%%%%%%%%%%%%%%%%%%
\subsection{Include by Input}
\label{sec:input}

Including child documents by |\include| has some restrictions by design.
Most notably, the content of a child document always occupies
its own set of pages; pages cannot be shared between child documents.
Usually, this behaviour makes perfect sense
because each child document contain an essential part of the document.
However, in some situations it may be desirable to compose
a document from a collection of parts
without having mandatory page breaks between then.
For this case, the package
provides a mechanism to include parts
by |\input| which can also be processed individually.
However, by construction this mechanism
requires manual handling of the content to be output.

%%%%%%%%%%%%%%%%%%%%%%%%%%%%%%%%%%%%%%%%
\DescribeMacro{\ifchilddocmanual}
The main file should be prepared as usual, see \secref{sec:include}.
However, the document body must make a distinction
between processing of an individual part and of the main document, e.g.:
%
\begin{center}
\begin{tabular}{l}
|\ifchilddocmanual|\\
|\input{\childdocname}|\\
|\||else|\\
\textit{document body with }|\input{|\textit{part}|}|\\
|\||fi|
\end{tabular}
\end{center}
%
The conditional |\ifchilddocmanual| is true whenever
a part to be included by |\input| is being compiled,
and the name of the part is stored in |\childdocname|.

%%%%%%%%%%%%%%%%%%%%%%%%%%%%%%%%%%%%%%%%
\DescribeMacro{\childdocby}
Each part to be included by |\input| should start with:
%
\begin{center}
\begin{tabular}{l}
|\input{childdoc.def}|\\
|\childdocby{|\textit{main}|}|\\
\end{tabular}
\end{center}
%
The directive |\childdocby| is similar to |\childdocof|
described in \secref{sec:include},
but the subsequent selection of content must be done manually.
To that end, both |\ifchilddoc| and |\ifchilddocmanual|
will be true upon processing of a part,
and the name of the part is stored in |\childdocname|.
Note that |\jobname| will be set to the filename of the current part
so that each part receives an individual |.aux| file
that does not interfere with the |.aux| file(s) of the main document.
This behaviour can be altered by the alternative form
|\childdocby[*]{|\textit{main}|}| (with a non-empty optional argument)
which uses the |.aux| file of the main document
by setting |\jobname| to \textit{main}.

%%%%%%%%%%%%%%%%%%%%%%%%%%%%%%%%%%%%%%%%%%%%%%%%%%%%%%%%%%%%%%%%%%%%%%%%%%%%%%%%
\subsection{Driver Development}
\label{sec:driver}

The \textsf{childdoc} mechanism can also be use for the development
of definition files such as \LaTeX{} styles or classes.
This case differs from the above setup with multiple parts
included by |\include| in that no |\includeonly| should be invoked.
This can be achieved by starting the include file
(before |\ProvidesPackage|) with:
%
\begin{center}
\begin{tabular}{l}
|\input{childdoc.def}|\\
|\childdocforward{|\textit{main}|}|\\
\end{tabular}
\end{center}
%
or alternatively with:
%
\begin{center}
\begin{tabular}{l}
|\input{childdoc.def}|\\
|\childdocby{|\textit{main}|}|\\
\end{tabular}
\end{center}
%
Both forms have slightly different effects as described above.
The main file is prepared as usual, see \secref{sec:include}.

%%%%%%%%%%%%%%%%%%%%%%%%%%%%%%%%%%%%%%%%%%%%%%%%%%%%%%%%%%%%%%%%%%%%%%%%%%%%%%%%
\subsection{Legacy Detection}
\label{sec:detection}

The directive |\childdocmain| in the main file can detect
whether the complete document or merely a child is to be compiled
even without using the directive |\childdocof|.
This method is deprecated because it is less robust
and there is no compelling reason to use it;
it is merely provided for backward compatibility
and it may be removed in future versions.

If the detection mechanism is to be used,
it is mandatory to correctly specify
the filename of the main file as the argument of |\childdocmain|:
%
\begin{center}
\begin{tabular}{l}
|\input{childdoc.def}|\\
|\childdocmain{|\textit{main}|}|\\
\end{tabular}
\end{center}
%
If |\jobname| does not match the argument \textit{main} of |\childdocmain|,
it is assumed that |\jobname| points to the child file to be compiled.
When using |\childdocmain| with the main file specified as argument,
it suffices to start a child file
with just |\input{|\textit{main}|}|
without loading of the package and using |\childdocof|.
If instead all processing is done
with the appropriate \textsf{childdoc} directives,
the argument of \textit{main} of |\childdocmain| can be empty.

An alternative version of the command line processing described
in \secref{sec:commandline} using the detection mechanism reads:
%
\begin{center}
|... -jobname "|\textit{target}|" "|[\textit{flags}]%
[|\def\jobname{|\textit{dest}|}|]|\input{|\textit{main}|}"|
\end{center}

%%%%%%%%%%%%%%%%%%%%%%%%%%%%%%%%%%%%%%%%%%%%%%%%%%%%%%%%%%%%%%%%%%%%%%%%%%%%%%%%
\subsection{Manual Code}
\label{sec:manual}

In case one cannot be certain whether the definitions file |childdoc.def|
is installed on the target \TeX{} distribution
and one prefers not to ship it,
it is conceivable to paste a few relevant commands into the sources.

To that end, drop all statements |\input{childdoc.def}|
and perform the replacements as outlined below.
Instead of |\childdocmain{|\textit{main}|}| add the following code
to the top of the main file:
%
\begin{center}
\begin{tabular}{l}
|\||ifdefined\childdocname\endinput\||fi\newif\ifchilddoc|\\
|\edef\childdocname{\scantokens\expandafter{\jobname\noexpand}}|\\
|\def\childdocmain{|\textit{main}|}\||ifx\childdocmain\childdocname\||else|\\
|\childdoctrue\includeonly{\childdocname}\let\jobname\childdocmain\||fi|\\
\end{tabular}
\end{center}
%
Instead of |\childdocof{|\textit{main}|}| just include the main file
at the top of each child file:
%
\begin{center}
|\input{|\textit{main}|}|
\end{center}
%
A simple redirection |\childdocforward{|\textit{dest}|}| is achieved by:
%
\begin{center}
|\def\jobname{|\textit{dest}|}\input{\jobname}|
\end{center}
%
The redirection with prefix
|\childdocforwardprefix[|\textit{prefix}|]{|\textit{dest}|}|
is accomplished by:
%
\begin{center}
\begin{tabular}{l}
|{\edef\jobname{\scantokens\expandafter{\jobname\noexpand}}|\\
|\def\redirectjob |\textit{prefix}|#1~~~{\gdef\jobname{|\textit{dest}|#1}}|\\
|\expandafter\redirectjob\jobname~~~}\input{\jobname}|
\end{tabular}
\end{center}

In an alternative approach,
child documents can be compiled by a specific command line
without additional code or specific definitions:
%
\begin{center}
|... -jobname "|\textit{target}|" "|[\textit{flags}]%
|\includeonly{|\textit{dest}|}\input{|\textit{main}|}"|
\end{center}
%

%%%%%%%%%%%%%%%%%%%%%%%%%%%%%%%%%%%%%%%%%%%%%%%%%%%%%%%%%%%%%%%%%%%%%%%%%%%%%%%%
%%%%%%%%%%%%%%%%%%%%%%%%%%%%%%%%%%%%%%%%%%%%%%%%%%%%%%%%%%%%%%%%%%%%%%%%%%%%%%%%
\section{Information}

%%%%%%%%%%%%%%%%%%%%%%%%%%%%%%%%%%%%%%%%%%%%%%%%%%%%%%%%%%%%%%%%%%%%%%%%%%%%%%%%
\subsection{Copyright}

Copyright \copyright{} 2017--2018 Niklas Beisert

This work may be distributed and/or modified under the
conditions of the \LaTeX{} Project Public License, either version 1.3
of this license or (at your option) any later version.
The latest version of this license is in
  \url{http://www.latex-project.org/lppl.txt}
and version 1.3 or later is part of all distributions of \LaTeX{}
version 2005/12/01 or later.

This work has the LPPL maintenance status `maintained'.

The Current Maintainer of this work is Niklas Beisert.

This work consists of the files |README.txt|, |childdoc.ins| and |childdoc.dtx|
as well as the derived files |childdoc.def|, |cdocsamp.tex|
with |cdocsch1.tex|, |cdocsch2.tex|, |cdocspt3.tex|, |cdocspt4.tex|,
|cdocsdrf.tex|, |cdocsfn1.tex|, |cdocsfn2.tex|
as well as |childdoc.pdf|.

%%%%%%%%%%%%%%%%%%%%%%%%%%%%%%%%%%%%%%%%%%%%%%%%%%%%%%%%%%%%%%%%%%%%%%%%%%%%%%%%
\subsection{Files and Installation}

The package consists of the files:
%
\begin{center}
\begin{tabular}{ll}
    |README.txt|   & readme file \\
    |childdoc.ins| & installation file \\
    |childdoc.dtx| & source file \\
    |childdoc.def| & definition file \\
    |cdocsamp.tex| & sample main file \\
    |cdocsch1.tex| & sample include file \\
    |cdocsch2.tex| & sample include file \\
    |cdocspt3.tex| & sample part file \\
    |cdocspt4.tex| & sample part file \\
    |cdocsdrf.tex| & sample redirection file \\
    |cdocsfn1.tex| & sample redirection file \\
    |cdocsfn2.tex| & sample redirection file \\
    |childdoc.pdf| & manual
\end{tabular}
\end{center}
%
The distribution consists of the files
|README.txt|, |childdoc.ins| and |childdoc.dtx|.
%
\begin{itemize}
\item
Run (pdf)\LaTeX{} on |childdoc.dtx|
to compile the manual |childdoc.pdf| (this file).
\item
Run \LaTeX{} on |childdoc.ins| to create the definitions file |childdoc.def|
and the sample |cdocsamp.tex| with include files
|cdocsch1.tex|, |cdocsch2.tex|, |cdocspt3.tex|, |cdocspt4.tex|,
|cdocsdrf.tex|, |cdocsfn1.tex|, |cdocsfn2.tex|.
Then copy the file |childdoc.def| to an appropriate directory of your \LaTeX{}
distribution, e.g.\ \textit{texmf-root}|/tex/latex/childdoc|.
\end{itemize}

%%%%%%%%%%%%%%%%%%%%%%%%%%%%%%%%%%%%%%%%%%%%%%%%%%%%%%%%%%%%%%%%%%%%%%%%%%%%%%%%
\subsection{Related CTAN Packages}

There are several other packages which offer a similar functionality:
%
\begin{itemize}
\item
The packages
\href{http://ctan.org/pkg/docmute}{\textsf{docmute}},
\href{http://ctan.org/pkg/includex}{\textsf{includex}} and
\href{http://ctan.org/pkg/standalone}{\textsf{standalone}}
provide commands to include only the document body of
a child file thus allowing both files to be compiled individually.
\item
The packages \href{http://ctan.org/pkg/subdocs}{\textsf{subdocs}}
and \href{http://ctan.org/pkg/subfiles}{\textsf{subfiles}}
provide structures in which the main and child documents can be
encapsulated and allowing them to be compiled individually.
The inclusion mechanism is different from the conventional |\include|.
\item
The package \href{http://ctan.org/pkg/combine}{\textsf{combine}}
is an elaborate solution to combine several documents into one.
\end{itemize}
%
See also the CTAN topic \href{http://ctan.org/topic/subdocs}{\textsf{subdocs}}
for further related packages.
The present package differs from the above solutions in that
a document structure constructed with the conventional |\include| mechanism
just needs two extra commands at the top of every file
such that all constituent files can be compiled individually.

%%%%%%%%%%%%%%%%%%%%%%%%%%%%%%%%%%%%%%%%%%%%%%%%%%%%%%%%%%%%%%%%%%%%%%%%%%%%%%%%
%\subsection{Feature Suggestions}
%
%The following is a list of features which may be useful for future
%versions of this package:
%%
%\begin{itemize}
%\item
%\ldots
%\end{itemize}

%%%%%%%%%%%%%%%%%%%%%%%%%%%%%%%%%%%%%%%%%%%%%%%%%%%%%%%%%%%%%%%%%%%%%%%%%%%%%%%%
\subsection{Revision History}

%%%%%%%%%%%%%%%%%%%%%%%%%%%%%%%%%%%%%%%%
\paragraph{v2.0:} 2018/12/30

\begin{itemize}
\item
immediate forward processing
\item
added |\childdocby| mechanism
\item
manual restructured
\end{itemize}

%%%%%%%%%%%%%%%%%%%%%%%%%%%%%%%%%%%%%%%%
\paragraph{v1.6:} 2018/01/17

\begin{itemize}
\item
application for development of include files
\item
corrections to manual
\end{itemize}

%%%%%%%%%%%%%%%%%%%%%%%%%%%%%%%%%%%%%%%%
\paragraph{v1.5:} 2017/05/21

\begin{itemize}
\item
more complete structuring introduced
\item
|\childdocof| introduced
\item
|\childdoc| renamed to |\childdocmain|
\item
|\childredirect| renamed to |\childdocforward| and |\childdocforwardprefix|
and functionality expanded
\end{itemize}

%%%%%%%%%%%%%%%%%%%%%%%%%%%%%%%%%%%%%%%%
\paragraph{v1.0:} 2017/04/27

\begin{itemize}
\item
manual and install package
\item
first version published on CTAN
\end{itemize}

%%%%%%%%%%%%%%%%%%%%%%%%%%%%%%%%%%%%%%%%
\paragraph{v0.6:} 2017/04/26

\begin{itemize}
\item
redirection mechanism added
\end{itemize}

%%%%%%%%%%%%%%%%%%%%%%%%%%%%%%%%%%%%%%%%
\paragraph{v0.5:} 2017/04/26

\begin{itemize}
\item
functionality in definition file
\end{itemize}


%%%%%%%%%%%%%%%%%%%%%%%%%%%%%%%%%%%%%%%%%%%%%%%%%%%%%%%%%%%%%%%%%%%%%%%%%%%%%%%%
%%%%%%%%%%%%%%%%%%%%%%%%%%%%%%%%%%%%%%%%%%%%%%%%%%%%%%%%%%%%%%%%%%%%%%%%%%%%%%%%
%%%%%%%%%%%%%%%%%%%%%%%%%%%%%%%%%%%%%%%%%%%%%%%%%%%%%%%%%%%%%%%%%%%%%%%%%%%%%%%%
\appendix

\settowidth\MacroIndent{\rmfamily\scriptsize 000\ }

 \DocInput{childdoc.dtx}

\end{document}
%</driver>
% \fi
%
% %%%%%%%%%%%%%%%%%%%%%%%%%%%%%%%%%%%%%%%%%%%%%%%%%%%%%%%%%%%%%%%%%%%%%%%%%%%%%%
% %%%%%%%%%%%%%%%%%%%%%%%%%%%%%%%%%%%%%%%%%%%%%%%%%%%%%%%%%%%%%%%%%%%%%%%%%%%%%%
% \section{Sample}
%\iffalse
%<*samplemain>
%\fi
%
% The following presents a sample document
% with two chapters, two parts, a title page,
% a compile flag as well as three forwarding files to set the flag.
% It consists of eight |.tex| files:
% \begin{center}
% \begin{tabular}{ll}
% |cdocsamp.tex|&main file\\
% |cdocsch1.tex|&include file for chapter 1\\
% |cdocsch2.tex|&include file for chapter 2\\
% |cdocspt3.tex|&include file for part 3\\
% |cdocspt4.tex|&include file for part 4\\
% |cdocsdrf.tex|&forwarding file for main file in draft mode\\
% |cdocsfi1.tex|&forwarding file for final version of chapter 1\\
% |cdocsfi2.tex|&forwarding file for final version of chapter 2\\
% \end{tabular}
% \end{center}
% Each of the eight files can be compiled directly by the \LaTeX{} compiler.
%
% %%%%%%%%%%%%%%%%%%%%%%%%%%%%%%%%%%%%%%
% \paragraph{Main File.}
%
% The main file is called |cdocsamp.tex|.
%
% Load the \textsf{childdoc} definitions and
% declare the filename for the main document:
%    \begin{macrocode}
\input{childdoc.def}
\childdocmain{}
%    \end{macrocode}

% Optional override for |\version| flag:
%    \begin{macrocode}
%%\ifchilddoc\else\providecommand{\version}{draft}\fi
%    \end{macrocode}

% Define the default values for the |\version| flag
% (|final| for the main file and |draft| for childs):
%    \begin{macrocode}
\ifchilddoc
\providecommand{\version}{draft}
\else
\providecommand{\version}{final}
\fi
%    \end{macrocode}

% Load the standard document class:
%    \begin{macrocode}
\documentclass[12pt]{article}
%    \end{macrocode}

% Start the document body:
%    \begin{macrocode}
\begin{document}
%    \end{macrocode}

% Declare a title page.
% Print title, part of document being processed and version flag:
%    \begin{macrocode}
\addtocounter{page}{-1}
\begin{center}
{\LARGE\bfseries{}childdoc example\par}
\vspace{1cm}
\ifchilddoc
\ifchilddocmanual part\else chapter\fi:
`\childdocname' of `\childdocjob'\par
\else
main document: `\childdocjob'\par
\fi
version: \version\par
\end{center}
\newpage
%    \end{macrocode}

% Manually include selected file,
% otherwise process as usual:
%    \begin{macrocode}
\ifchilddocmanual
\section*{part `\childdocname'}
\input{\childdocname}
\else
%    \end{macrocode}

% Include the two chapters:
%    \begin{macrocode}
\include{cdocsch1}
\include{cdocsch2}
%    \end{macrocode}

% Include the two parts unless only chapters should be displayed:
%    \begin{macrocode}
\ifchilddoc\else
\section{part three}
\input{cdocspt3}
\section{part four}
\input{cdocspt4}
\fi
%    \end{macrocode}

% Process as usual until here:
%    \begin{macrocode}
\fi
%    \end{macrocode}

% End of document body:
%    \begin{macrocode}
\end{document}
%    \end{macrocode}
%\iffalse
%</samplemain>
%\fi
%
% %%%%%%%%%%%%%%%%%%%%%%%%%%%%%%%%%%%%%%
% \paragraph{Chapter Include Files.}
%
% The include files are called |cdocsch1.tex| and |cdocsch2.tex|.
%
%\iffalse
%<*samplechap1|samplechap2>
%\fi

% Optional override for |\version| flag:
%    \begin{macrocode}
%%\providecommand{\version}{final}
%    \end{macrocode}

% Include the main document:
%    \begin{macrocode}
\input{childdoc.def}
\childdocof{cdocsamp}
%    \end{macrocode}

%\iffalse
%</samplechap1|samplechap2>
%\fi
%
%\iffalse
%<*samplechap1>
%\fi
% Some text for chapter 1:
%    \begin{macrocode}
\section{one}
some text in chapter one
%    \end{macrocode}

%\iffalse
%</samplechap1>
%\fi
% Some text for chapter 2:
%\iffalse
%<*samplechap2>
%\fi
%    \begin{macrocode}
\section{two}
more text in chapter two
%    \end{macrocode}

%\iffalse
%</samplechap2>
%\fi
%
% %%%%%%%%%%%%%%%%%%%%%%%%%%%%%%%%%%%%%%
% \paragraph{Part Include Files.}
%
% The include files are called |cdocspt3.tex| and |cdocspt4.tex|.
%
%\iffalse
%<*samplepart3|samplepart4>
%\fi

% Optional override for |\version| flag:
%    \begin{macrocode}
%%\providecommand{\version}{final}
%    \end{macrocode}

% Include the main document:
%    \begin{macrocode}
\input{childdoc.def}
\childdocby{cdocsamp}
%    \end{macrocode}

%\iffalse
%</samplepart3|samplepart4>
%\fi
%
%\iffalse
%<*samplepart3>
%\fi
% Some text for part 3:
%    \begin{macrocode}
some text in part three
%    \end{macrocode}

%\iffalse
%</samplepart3>
%\fi
% Some text for part 4:
%\iffalse
%<*samplepart4>
%\fi
%    \begin{macrocode}
more text in part four
%    \end{macrocode}

%\iffalse
%</samplepart4>
%\fi
%
% %%%%%%%%%%%%%%%%%%%%%%%%%%%%%%%%%%%%%%
% \paragraph{Forwarding for a Complete Draft.}
%
% The following forwarding file |cdocsdrf.tex|
% compiles the main document in draft mode:
%\iffalse
%<*sampledraft>
%\fi
%    \begin{macrocode}
\def\version{draft}
\input{childdoc.def}
\childdocforward{cdocsamp}
%    \end{macrocode}

%\iffalse
%</sampledraft>
%\fi
%
% %%%%%%%%%%%%%%%%%%%%%%%%%%%%%%%%%%%%%%
% \paragraph{Forwarding for Final Version of the Chapters.}
%
% The following forwarding files |cdocsfn1.tex| and |cdocsfn2.tex|
% (with identical content)
% compile the final versions of the child documents
% |cdocsch1.tex| and |cdocsch2.tex|, respectively:
%\iffalse
%<*samplefinal>
%\fi
%    \begin{macrocode}
\def\version{final}
\input{childdoc.def}
\childdocforwardprefix[cdocsamp]{cdocsfn}{cdocsch}
%    \end{macrocode}

%\iffalse
%</samplefinal>
%\fi
%
% %%%%%%%%%%%%%%%%%%%%%%%%%%%%%%%%%%%%%%
% \paragraph{Command Line Processing.}
%
% The following three command lines generate the output files
% |cdocscld|, |cdocscl1| and |cdocscl2|
% which should be identical to
% |cdocsdrf|, |cdocsch1| and |cdocsfn2|, respectively:
% \begin{center}
% \begin{tabular}{l}
% |latex -jobname cdocscld \|\\
% |  "\def\version{draft}\input{childdoc.def}\childdocforward{cdocsamp}"|\\
% |latex -jobname cdocscl1 \|\\
% |  "\input{childdoc.def}\childdocforward[cdocsamp]{cdocsch1}"|\\
% |latex -jobname cdocscl2 \|\\
% |  "\def\version{final}\input{childdoc.def}\childdocforward{cdocsch2}"|
% \end{tabular}
% \end{center}
% Note that the trailing backslash on each first line
% merely continues the input to the second line
% (for convenient cut ant paste).
% Furthermore, the command |latex| can be replaced by any
% of its alternative versions such as |pdflatex|.
%
% %%%%%%%%%%%%%%%%%%%%%%%%%%%%%%%%%%%%%%%%%%%%%%%%%%%%%%%%%%%%%%%%%%%%%%%%%%%%%%
% %%%%%%%%%%%%%%%%%%%%%%%%%%%%%%%%%%%%%%%%%%%%%%%%%%%%%%%%%%%%%%%%%%%%%%%%%%%%%%
% \section{Implementation}
%\iffalse
%<*package>
%\fi
%
% This section describes the definitions file |childdoc.def|.

% The definitions cannot be loaded using |\usepackage| or |\RequirePackage|
% which has a mechanism to prevent loading a style file more than once.
% When loading the definitions by means of |\input|
% multiple instances have to be prevented manually:
%\iffalse
%This code needs to be before the `\ProvidesFile' directive
%which is defined at the beginning of this file.
%Therefore it is also placed there and commented out here.
%</package>
%<*discard>
%\fi
%    \begin{macrocode}
\ifdefined\childdocmain\endinput\fi
%    \end{macrocode}
%\iffalse
%</discard>
%<*package>
%\fi
%
% \macro{\ifchilddoc}
% \macro{\ifchilddocmanual}
% The conditional |\ifchilddoc| tells whether a
% child (true) or main (false) document is being compiled.
% The conditional |\ifchilddocmanual| tells whether
% the |\includeonly| mechanism is used (false) or
% the selection of child files must be performed manually (true).
% The definitions initialise to false:
%    \begin{macrocode}
\newif\ifchilddoc
\newif\ifchilddocmanual
%    \end{macrocode}

% \macro{\childdocname}
% \macro{\childdocjob}
% The macro |\childdocname| stores the name of the main document
% to be compiled. The macro |\childdocjob| stores the name of
% the document on which the \LaTeX{} compiler was originally invoked.
% The content of |\jobname| cannot be compared
% to filenames specified in the source due to different catcodes.
% The following code rescans |\jobname|, stores the result
% in |\childdocname| and saves a copy in |\childdocjob|:
%    \begin{macrocode}
\edef\childdocname{\scantokens\expandafter{\jobname\noexpand}}
\let\childdocjob\childdocname
%    \end{macrocode}

% \macro{\childdocdisable}
% The macro |\childdocdisable| prevents the main file
% from being processed more than once.
% At this stage, the main document command |\childdocmain|
% is assumed to be called once again where it should do nothing.
% Any subsequent call to it should prevent
% a secondary processing of the main document
% It overwrites the forwarding commands
% |\childdocof| and |\childdocforward|
% with empty macros to prevent further inclusions of the main document:
%    \begin{macrocode}
\newcommand{\childdocdisable}
{
  \renewcommand{\childdocmain}[1]{\renewcommand{\childdocmain}[1]{\endinput}}
  \renewcommand{\childdocof}[1]{}
  \renewcommand{\childdocby}[2][]{}
  \renewcommand{\childdocforward}[2][]{}
  \renewcommand{\childdocdisable}{}
}
%    \end{macrocode}

% \macro{\childdocmain}
% The macro |\childdocmain| is to be called at the top of the main file
% with nothing or the main filename (without extension) as argument.
% First, it breaks loops.
% If the argument is not empty and does not match |\childdocname|
% (which is set by the first inclusion of |childdoc.def|),
% |\ifchilddoc| is set to true, |\includeonly| is applied to the child file
% and |\jobname| is set to the main file
% (for proper handling of |.aux| files):
%    \begin{macrocode}
\newcommand{\childdocmain}[1]
{
  \childdocdisable\childdocmain{}
  \if?#1?\else
    \begingroup
      \def\childdoctmp{#1}
      \ifx\childdoctmp\childdocname
        \def\childdoctmp{}
      \else
        \def\childdoctmp
        {
          \childdoctrue
          \includeonly{\childdocname}
          \def\childdocjob{#1}
          \def\jobname{#1}
        }
      \fi
      \expandafter
    \endgroup
    \childdoctmp
  \fi
}
%    \end{macrocode}

% \macro{\childdocof}
% The command |\childdocof| redirects
% compilation to the main file |#1|.
%    \begin{macrocode}
\newcommand{\childdocof}[1]
{
  \childdocdisable
  \childdoctrue
  \includeonly{\childdocname}
  \def\jobname{#1}
  \def\childdocjob{#1}
  \input{#1}
}
%    \end{macrocode}

% \macro{\childdocby}
% The command |\childdocby| ....
%    \begin{macrocode}
\newcommand{\childdocby}[2][]
{
  \childdocdisable
  \childdoctrue
  \childdocmanualtrue
  \if?#1?\else
    \def\jobname{#2}
  \fi
  \def\childdocjob{#2}
  \input{#2}
  \endinput
}
%    \end{macrocode}

% \macro{\childdocforward}
% The command |\childdocforward| redirects
% compilation to the main file or
% (if the optional argument is given) a child file.
% Parameters are set as if the main file
% or a child file starting with |\childdocof| was compiled.
% Then compilation is handed over to the main file:
%    \begin{macrocode}
\newcommand{\childdocforward}[2][]
{
  \begingroup
    \if?#1?
      \def\childdoctmp
      {
        \def\childdocname{#2}
        \def\childdocjob{#2}
        \def\jobname{#2}
        \input{#2}
        \endinput
      }
    \else
      \def\childdoctmp
      {
        \childdocdisable
        \def\childdocname{#2}
        \childdoctrue
        \includeonly{#2}
        \def\childdocjob{#1}
        \def\jobname{#1}
        \input{#1}
        \endinput
      }
    \fi
    \expandafter
  \endgroup
  \childdoctmp
}
%    \end{macrocode}

% \macro{\childdocforwardprefix}
% The command |\childdocforwardprefix| redirects
% compilation to the main or a child file by means of a pattern.
% The prefix |#1| in the current filename is replaced by |#2|
% and the suffix of the current filename is kept
% (it is assumed that the filename does not contain the substring `|~~~|'
% which is used as a delimiter).
% Compilation is handed over to the new file by |\childdocforward|:
%    \begin{macrocode}
\newcommand{\childdocforwardprefix}[3][]
{
  \begingroup
    \def\childdocextract #2##1~~~{\def\childdoctmp{\childdocforward[#1]{#3##1}}}
    \expandafter\childdocextract\childdocname~~~
    \expandafter
  \endgroup
  \childdoctmp
}
%    \end{macrocode}

% \macro{\childdoc}
% The deprecated macro |\childdoc| is a legacy version of |\childdocmain|:
%    \begin{macrocode}
\newcommand{\childdoc}{\childdocmain}
%    \end{macrocode}

% \macro{\childdocredirect}
% The deprecated macro |\childdocredirect| is a legacy version
% of |\childdocforward| and |\childdocforwardprefix|:
%    \begin{macrocode}
\newcommand{\childdocredirect}[2][]
{
  \begingroup
    \if?#1?
      \def\childdoctmp{\childdocforward{#2}}
    \else
      \def\childdoctmp{\childdocforwardprefix{#1}{#2}}
    \fi
    \expandafter
  \endgroup
  \childdoctmp
}
%    \end{macrocode}

%\iffalse
%</package>
%\fi
%
\endinput
|\\
|\childdocforwardprefix[|\textit{main}|]{|\textit{prefix}|}{|\textit{dest}|}|
\end{tabular}
\end{center}
%
the destination file is determined by a pattern
depending on the current file:
To make this work, the current file must be called
`{\textit{prefix}\hspace{0.2em}\textit{suffix}}'
with \textit{prefix} matching precisely the argument.
Processing is then passed on to the file
`{\textit{dest}\hspace{0.2em}\textit{suffix}}'.
Surely, the same effect is achieved by
directly specifying the
argument `{\textit{dest}\hspace{0.2em}\textit{suffix}}'
in the first form.
However, that requires to set up a different file
for each child. With the alternative form of the command
all these files can have exactly the same content
which simplifies setting them up and maintaining them.

For example, the following file |draft.tex|
with a compilation flag |\version| as described in \secref{sec:flags}
compiles the main document as a draft:
%
\begin{center}
\begin{tabular}{l}
|\def\version{draft}|\\
|% \iffalse
%
% childdoc.dtx Copyright (C) 2017-2018 Niklas Beisert
%
% This work may be distributed and/or modified under the
% conditions of the LaTeX Project Public License, either version 1.3
% of this license or (at your option) any later version.
% The latest version of this license is in
%   http://www.latex-project.org/lppl.txt
% and version 1.3 or later is part of all distributions of LaTeX
% version 2005/12/01 or later.
%
% This work has the LPPL maintenance status `maintained'.
%
% The Current Maintainer of this work is Niklas Beisert.
%
% This work consists of the files childdoc.dtx and childdoc.ins
% and the derived files childdoc.def and cdocsamp.tex with
% cdocsch1.tex, cdocsch2.tex, cdocsdrf.tex, cdocsfn1.tex, cdocsfn2.tex.
%
%<package>\ifdefined\childdocmain\endinput\fi
%<package>\ProvidesFile{childdoc.def}[2018/12/30 v2.0 child document driver]
%<samplemain>\ProvidesFile{cdocsamp.tex}[2018/12/30 v2.0 sample for childdoc]
%<*driver>
%\ProvidesFile{childdoc.drv}[2018/12/30 v2.0 childdoc reference manual file]
\PassOptionsToClass{10pt,a4paper}{article}
\documentclass{ltxdoc}

\usepackage[margin=35mm]{geometry}
\usepackage{hyperref}
\usepackage{hyperxmp}
\usepackage[usenames]{color}

\hypersetup{colorlinks=true}
\hypersetup{pdfstartview=FitH}
\hypersetup{pdfpagemode=UseNone}
\hypersetup{pdfsource={}}
\hypersetup{pdflang={en-UK}}
\hypersetup{pdfcopyright={Copyright 2017-2018 Niklas Beisert.
  This work may be distributed and/or modified under the
  conditions of the LaTeX Project Public License, either version 1.3
  of this license or (at your option) any later version.}}
\hypersetup{pdflicenseurl={http://www.latex-project.org/lppl.txt}}
\hypersetup{pdfcontactaddress={ETH Zurich, ITP, HIT K,
  Wolfgang-Pauli-Strasse 27}}
\hypersetup{pdfcontactpostcode={8093}}
\hypersetup{pdfcontactcity={Zurich}}
\hypersetup{pdfcontactcountry={Switzerland}}
\hypersetup{pdfcontactemail={nbeisert@itp.phys.ethz.ch}}
\hypersetup{pdfcontacturl={http://people.phys.ethz.ch/\xmptilde nbeisert/}}

\newcommand{\secref}[1]{\hyperref[#1]{section \ref*{#1}}}

\parskip1ex
\parindent0pt
\let\olditemize\itemize
\def\itemize{\olditemize\parskip0pt}

\begin{document}

\title{The \textsf{childdoc} Package}
\hypersetup{pdftitle={The childdoc Package}}
\author{Niklas Beisert\\[2ex]
  Institut f\"ur Theoretische Physik\\
  Eidgen\"ossische Technische Hochschule Z\"urich\\
  Wolfgang-Pauli-Strasse 27, 8093 Z\"urich, Switzerland\\[1ex]
  \href{mailto:nbeisert@itp.phys.ethz.ch}
  {\texttt{nbeisert@itp.phys.ethz.ch}}}
\hypersetup{pdfauthor={Niklas Beisert}}
\hypersetup{pdfsubject={Manual for the LaTeX2e Package childdoc}}
\date{30 December 2018, \textsf{v2.0}}
\maketitle

\begin{abstract}\noindent
\textsf{childdoc} is a \LaTeXe{} package
that enables the direct compilation
of document sections included by |\include|
to individual files.
\end{abstract}

\begingroup
\parskip0ex
\tableofcontents
\endgroup

%%%%%%%%%%%%%%%%%%%%%%%%%%%%%%%%%%%%%%%%%%%%%%%%%%%%%%%%%%%%%%%%%%%%%%%%%%%%%%%%
%%%%%%%%%%%%%%%%%%%%%%%%%%%%%%%%%%%%%%%%%%%%%%%%%%%%%%%%%%%%%%%%%%%%%%%%%%%%%%%%
\section{Introduction}

\LaTeX{} provides a mechanism to structure a large document (such as a book)
into a main file and several child files (containing the chapters)
using the |\include| command.
This mechanism is beneficial for documents
which span hundreds of pages in order to
make the source file(s) more manageable.
Moreover, compilation can be restricted to
selected child files by means of the |\includeonly| command.
The latter feature can be used to reduce the compilation time while editing
(this was significantly more useful in the earlier days of \LaTeX{})
or to generate a smaller document which is easier to navigate.
Another application of |\includeonly| is to generate
documents consisting of selected parts of the complete document.

However, there are a few drawbacks of the plain |\include| mechanism:
\begin{itemize}
\item
The child files cannot be compiled on their own,
they can only be compiled via the main file.
A naive editing environment
(such as a text editor with an option
to have the current file processed by \LaTeX)
may require one to switch to the main file before compiling;
attempting to compile the child file produces errors.
\item
The main file must be modified (each time)
to adjust the |\includeonly| command
to the present needs. This easily leaves the main file in a messy state.
\item
The generated document will always carry the filename
of the main document. This is inconvenient if
several child files are to be compiled and
to be kept for distribution.
\end{itemize}

The present package provides a simple interface
to make child files individually compilable by \LaTeX{}.
Compiling a child file then has the same effect as compiling
the main file with an |\includeonly| command
to select the appropriate child.
Moreover the generated document will carry the name of the child
rather than the main file.
This resolves all three above issues.

This feature is meant to make the editing of books,
thesis documents and lecture notes somewhat more convenient.
However, the package can also be used efficiently for
composing a series of documents (such as exercise sheets)
which are typically distributed individually.
It then assists the author in generating the individual documents
(potentially in different versions)
as well as a document containing the collected series.
Another application is in developing style files
or other kinds of included material
where compilation of the style file could redirect
to a sample or test file.

%%%%%%%%%%%%%%%%%%%%%%%%%%%%%%%%%%%%%%%%%%%%%%%%%%%%%%%%%%%%%%%%%%%%%%%%%%%%%%%%
%%%%%%%%%%%%%%%%%%%%%%%%%%%%%%%%%%%%%%%%%%%%%%%%%%%%%%%%%%%%%%%%%%%%%%%%%%%%%%%%
\section{Usage}

First of all, the package \textsf{childdoc} is \emph{not} a standard
\LaTeXe{} |.sty| style file! Therefore it needs to be invoked in
a non-standard way.

%%%%%%%%%%%%%%%%%%%%%%%%%%%%%%%%%%%%%%%%%%%%%%%%%%%%%%%%%%%%%%%%%%%%%%%%%%%%%%%%
\subsection{Included Files}
\label{sec:include}

%%%%%%%%%%%%%%%%%%%%%%%%%%%%%%%%%%%%%%%%
\DescribeMacro{\childdocmain}
To use the package, add the commands
\begin{center}
\begin{tabular}{l}
|\input{childdoc.def}|\\
|\childdocmain{}|\\
\end{tabular}
\end{center}
at the very top of the main \LaTeX{} file,
in particular \emph{before} the |\documentclass| statement!
The argument of |\childdocmain| should be left empty
(but it must be present).

%%%%%%%%%%%%%%%%%%%%%%%%%%%%%%%%%%%%%%%%
\DescribeMacro{\childdocof}
Furthermore, add the commands
\begin{center}
\begin{tabular}{l}
|\input{childdoc.def}|\\
|\childdocof{|\textit{main}|}|\\
\end{tabular}
\end{center}
at the top of every child file \textit{child}
which is included by |\include{|\textit{child}|}|
from within the main file
(or at least for those files to be compiled individually).
The argument \textit{main} must be the filename of the main file.

There are a couple of
considerations in setting up the main and child documents:

%%%%%%%%%%%%%%%%%%%%%%%%%%%%%%%%%%%%%%%%
\paragraph{Restrictions.}

Please note the following restrictions:
\begin{itemize}
\item
|\childdocmain| must be called with one argument \textit{main}
to ensure compatibility with earlier version of the package.
It must either be empty (|\childdocmain{}|)
or precisely match the filename of the main file in which it is specified.
See \secref{sec:detection} for further information.
\item
The filename \textit{main} must be specified without the |.tex| extension.
\item
The filename \textit{main} is case sensitive
(even in case-insensitive file systems)
due to internal string comparison.
\item
The argument \textit{main} should be fully expanded, it cannot be a macro.
\item
Subdirectories and special characters should be avoided in filenames.
\item
The command |\childdocmain{|\textit{main}|}| must be followed by a whitespace.
It should not be followed immediately by another command
or by a comment mark `|%|'.
This is because the \TeX{} parser reads the token immediately following
the argument of |\childdocmain| and puts it
at the beginning of every child section;
however, a white\-space is ignored.
\end{itemize}

%%%%%%%%%%%%%%%%%%%%%%%%%%%%%%%%%%%%%%%%
\paragraph{Content of Main File.}

It is advisable to place all content in the child files included by |\include|.
Any output contained in the main file will appear in all child documents
unless suppressed manually;
it cannot be suppressed automatically by the |\includeonly| directive
and thus should normally be avoided.
A method to include some content in the main file
by means of conditional processing is described in \secref{sec:conditional}.

%%%%%%%%%%%%%%%%%%%%%%%%%%%%%%%%%%%%%%%%
\paragraph{Page Numbering.}

When only a part of the document is compiled,
the appropriate numbering of pages
(as well as other status parameters)
is determined from the |.aux| files.
The latter contain information from previous passes.
However this information needs to propagate through
all intermediate child documents.
Therefore the page numbering in child documents may well
be inconsistent until the complete document is compiled at least once.

A useful (if unconventional) way to always ensure a consistent
page numbering is to restart the numbering in each child document
and denote the pages by `\textit{child}|.|\textit{page}'
where \textit{child} represents the chapter/section number of the child file.
This can be achieved by the command
|\numberwithin{page}{|\textit{child}|}|
of the \textsf{amsmath} package
where \textit{child} can be |chapter| or |section|
depending on the chosen structuring.
Alternatively, one can modify the macro |\thepage| appropriately
and reset the counter |page| at the start of each child file.

%%%%%%%%%%%%%%%%%%%%%%%%%%%%%%%%%%%%%%%%%%%%%%%%%%%%%%%%%%%%%%%%%%%%%%%%%%%%%%%%
\subsection{Conditional Processing}
\label{sec:conditional}

The package provides a mechanism to compile different versions
of a document. To customise the versions further some conditional processing
can come in handy to distinguish which version is being compiled.
The package provides two macros to describe the compilation context:

%%%%%%%%%%%%%%%%%%%%%%%%%%%%%%%%%%%%%%%%
\DescribeMacro{\ifchilddoc}
The conditional |\ifchilddoc| distinguishes between the compilation of
child documents and the main document:
%
\begin{center}
|\ifchilddoc |\textit{child-code}| |[|\||else |\textit{main-code}]| \||fi|
\end{center}

%%%%%%%%%%%%%%%%%%%%%%%%%%%%%%%%%%%%%%%%
\DescribeMacro{\childdocname}
\DescribeMacro{\childdocjob}
The macro |\childdocname| contains the filename (without extension)
of the main or child file being processed.
Note that |\childdocjob| will always contain the name of the main file.

%%%%%%%%%%%%%%%%%%%%%%%%%%%%%%%%%%%%%%%%
\paragraph{Title Page.}

Conditional processing can be used to include a title or banner page
in the main document when proper precautions are taken.
Importantly, the code in the main file should ensure that the page counter
(as well as other status parameters which are stored in the |.aux| files)
takes the same value after the conditional processing.
Otherwise the page numbers may take divergent values
depending on which part is compiled.

For example, a title page could be declared by:
%
\begin{center}
\begin{tabular}{l}
|\ifchilddoc\||else|\\
|\addtocounter{page}{-1}|\\
\textit{code for title page}\\
|\newpage|\\
|\||fi|
\end{tabular}
\end{center}
%
A banner page for the child documents can be generated by:
%
\begin{center}
\begin{tabular}{l}
|\ifchilddoc|\\
|\addtocounter{page}{-1}|\\
\textit{code for banner page}\\
|\newpage|\\
|\||fi|
\end{tabular}
\end{center}
%
Here one could write a message such as:
\begin{center}
|This is the part \childdocname{} of \childdocjob{}.|
\end{center}

%%%%%%%%%%%%%%%%%%%%%%%%%%%%%%%%%%%%%%%%%%%%%%%%%%%%%%%%%%%%%%%%%%%%%%%%%%%%%%%%
\subsection{Flags}
\label{sec:flags}

The package makes it easy to generate different versions
of the main or child documents.
To this end compilation flags can be defined
and assigned different default values.
They will be particularly useful in conjunction
with the forwarding mechanism described in \secref{sec:forward}.

For example, it may be useful to have a flag |\version|
which can be set to |draft| or |final|.
The document source will contain some conditional code
depending on the value of |\version|.
Suppose further, the flag should default to |final| for the main file
and to |draft| for child files
which is a natural assignment for editing the document.
This is achieved by placing the following code
in the preamble of the main document
(below the |\childdocmain| directive):
%
\begin{center}
\begin{tabular}{l}
|\ifchilddoc|\\
|\providecommand{\version}{draft}|\\
|\||else|\\
|\providecommand{\version}{final}|\\
|\||fi|
\end{tabular}
\end{center}
%
The definition by |\providecommand| makes sure
that previous definitions are not overwritten.
Further statements |\providecommand{\version}{...}|
can thus be added before the above code to override it.

For the main file, one might add a line
(between |\childdocmain| and the above block)
%
\begin{center}
|%\ifchilddoc\||else\providecommand{\version}{draft}\||fi|
\end{center}
%
which can be uncommented to produce a draft version.
Likewise one can add a line to the very top of a child file
(above the |\childdocof{|\textit{main}|}| directive)
%
\begin{center}
|%\providecommand{\version}{final}|
\end{center}
%
which can be uncommented to produce the final version of this child document.

%%%%%%%%%%%%%%%%%%%%%%%%%%%%%%%%%%%%%%%%%%%%%%%%%%%%%%%%%%%%%%%%%%%%%%%%%%%%%%%%
\subsection{Forwarding}
\label{sec:forward}

Different versions of the main or child documents
using compilation flags as described in \secref{sec:flags}
can be (permanently) stored in different files
for convenient compilation, viewing and distribution.
To this end, the package defines a command
to pass on compilation to a different file:

%%%%%%%%%%%%%%%%%%%%%%%%%%%%%%%%%%%%%%%%
\DescribeMacro{\childdocforward}
The command |\childdocforward| redirects processing to
another source file:
%
\begin{center}
\begin{tabular}{l}
|\input{childdoc.def}|\\
|\childdocforward[|\textit{main}|]{|\textit{dest}|}|\\
\end{tabular}
\end{center}
%
The argument \textit{dest} is the destination file
(without extension).
It should be the main file or one of the child files.
Note that further \textsf{childdoc} directives
such as |\childdocof| and |\childdocforward|
in the indicated file will be processed in this form.
The optional argument \textit{main}
passes on directly to the main file \textit{main}
while pretending to compile the child \textit{dest}.
This form behaves as if \textit{dest}
issues |\childdocof{|\textit{main}|}| right away,
and no further \textsf{childdoc} directives will be processed.

%%%%%%%%%%%%%%%%%%%%%%%%%%%%%%%%%%%%%%%%
\DescribeMacro{\...prefix}
In the alternative form |\childdocforwardprefix|,
%
\begin{center}
\begin{tabular}{l}
|\input{childdoc.def}|\\
|\childdocforwardprefix[|\textit{main}|]{|\textit{prefix}|}{|\textit{dest}|}|
\end{tabular}
\end{center}
%
the destination file is determined by a pattern
depending on the current file:
To make this work, the current file must be called
`{\textit{prefix}\hspace{0.2em}\textit{suffix}}'
with \textit{prefix} matching precisely the argument.
Processing is then passed on to the file
`{\textit{dest}\hspace{0.2em}\textit{suffix}}'.
Surely, the same effect is achieved by
directly specifying the
argument `{\textit{dest}\hspace{0.2em}\textit{suffix}}'
in the first form.
However, that requires to set up a different file
for each child. With the alternative form of the command
all these files can have exactly the same content
which simplifies setting them up and maintaining them.

For example, the following file |draft.tex|
with a compilation flag |\version| as described in \secref{sec:flags}
compiles the main document as a draft:
%
\begin{center}
\begin{tabular}{l}
|\def\version{draft}|\\
|\input{childdoc.def}|\\
|\childdocforward{|\textit{main}|}|
\end{tabular}
\end{center}
%
Likewise, the following files |final|\textit{nn}|.tex|
compile the final version of the child document
|child|\textit{nn}|.tex|:
%
\begin{center}
\begin{tabular}{l}
|\def\version{final}|\\
|\input{childdoc.def}|\\
|\childdocforwardprefix{final}{child}|
\end{tabular}
\end{center}
%

Note that when several versions of a main file and/or of each child file
are to be generated, it may be convenient to set up a |Makefile| or
shell script to automatise the process.

%%%%%%%%%%%%%%%%%%%%%%%%%%%%%%%%%%%%%%%%%%%%%%%%%%%%%%%%%%%%%%%%%%%%%%%%%%%%%%%%
\subsection{Command Line Processing}
\label{sec:commandline}

The effect of redirection files can also be achieved by invoking
the \LaTeX{} compiler with a more elaborate command line.
Most conveniently this should be done as part
of a shell script or a |Makefile|.

When using \textsf{childdoc} in the main file, the following
command lines effectively perform a redirection
(note that depending on the shell being used,
backslashes may have to be doubled: `|\|' $\to$ `|\\|'):
%
\begin{center}
|... -jobname "|\textit{target}|" |\\|"|[\textit{flags}]%
|\input{childdoc.def}\childdocforward[|\textit{main}|]{|\textit{dest}|}"|
\end{center}
%
Here \textit{target} is the name of the output file,
\textit{main} is the name of the main file
and \textit{dest} is the name of the main or child file to be processed
(all filenames without extensions).
The optional argument \textit{main} can be omitted
if \textit{main} matches \textit{dest}.
Optionally, compilation \textit{flags} can be defined via |\def| commands.
This command line makes the \TeX{} engine believe
it is compiling the file \textit{target}
whose content is specified as the latter parameter.
The provided code then forwards the processing to
\textit{main} or \textit{dest} as described in \secref{sec:forward}.

%%%%%%%%%%%%%%%%%%%%%%%%%%%%%%%%%%%%%%%%%%%%%%%%%%%%%%%%%%%%%%%%%%%%%%%%%%%%%%%%
\subsection{Include by Input}
\label{sec:input}

Including child documents by |\include| has some restrictions by design.
Most notably, the content of a child document always occupies
its own set of pages; pages cannot be shared between child documents.
Usually, this behaviour makes perfect sense
because each child document contain an essential part of the document.
However, in some situations it may be desirable to compose
a document from a collection of parts
without having mandatory page breaks between then.
For this case, the package
provides a mechanism to include parts
by |\input| which can also be processed individually.
However, by construction this mechanism
requires manual handling of the content to be output.

%%%%%%%%%%%%%%%%%%%%%%%%%%%%%%%%%%%%%%%%
\DescribeMacro{\ifchilddocmanual}
The main file should be prepared as usual, see \secref{sec:include}.
However, the document body must make a distinction
between processing of an individual part and of the main document, e.g.:
%
\begin{center}
\begin{tabular}{l}
|\ifchilddocmanual|\\
|\input{\childdocname}|\\
|\||else|\\
\textit{document body with }|\input{|\textit{part}|}|\\
|\||fi|
\end{tabular}
\end{center}
%
The conditional |\ifchilddocmanual| is true whenever
a part to be included by |\input| is being compiled,
and the name of the part is stored in |\childdocname|.

%%%%%%%%%%%%%%%%%%%%%%%%%%%%%%%%%%%%%%%%
\DescribeMacro{\childdocby}
Each part to be included by |\input| should start with:
%
\begin{center}
\begin{tabular}{l}
|\input{childdoc.def}|\\
|\childdocby{|\textit{main}|}|\\
\end{tabular}
\end{center}
%
The directive |\childdocby| is similar to |\childdocof|
described in \secref{sec:include},
but the subsequent selection of content must be done manually.
To that end, both |\ifchilddoc| and |\ifchilddocmanual|
will be true upon processing of a part,
and the name of the part is stored in |\childdocname|.
Note that |\jobname| will be set to the filename of the current part
so that each part receives an individual |.aux| file
that does not interfere with the |.aux| file(s) of the main document.
This behaviour can be altered by the alternative form
|\childdocby[*]{|\textit{main}|}| (with a non-empty optional argument)
which uses the |.aux| file of the main document
by setting |\jobname| to \textit{main}.

%%%%%%%%%%%%%%%%%%%%%%%%%%%%%%%%%%%%%%%%%%%%%%%%%%%%%%%%%%%%%%%%%%%%%%%%%%%%%%%%
\subsection{Driver Development}
\label{sec:driver}

The \textsf{childdoc} mechanism can also be use for the development
of definition files such as \LaTeX{} styles or classes.
This case differs from the above setup with multiple parts
included by |\include| in that no |\includeonly| should be invoked.
This can be achieved by starting the include file
(before |\ProvidesPackage|) with:
%
\begin{center}
\begin{tabular}{l}
|\input{childdoc.def}|\\
|\childdocforward{|\textit{main}|}|\\
\end{tabular}
\end{center}
%
or alternatively with:
%
\begin{center}
\begin{tabular}{l}
|\input{childdoc.def}|\\
|\childdocby{|\textit{main}|}|\\
\end{tabular}
\end{center}
%
Both forms have slightly different effects as described above.
The main file is prepared as usual, see \secref{sec:include}.

%%%%%%%%%%%%%%%%%%%%%%%%%%%%%%%%%%%%%%%%%%%%%%%%%%%%%%%%%%%%%%%%%%%%%%%%%%%%%%%%
\subsection{Legacy Detection}
\label{sec:detection}

The directive |\childdocmain| in the main file can detect
whether the complete document or merely a child is to be compiled
even without using the directive |\childdocof|.
This method is deprecated because it is less robust
and there is no compelling reason to use it;
it is merely provided for backward compatibility
and it may be removed in future versions.

If the detection mechanism is to be used,
it is mandatory to correctly specify
the filename of the main file as the argument of |\childdocmain|:
%
\begin{center}
\begin{tabular}{l}
|\input{childdoc.def}|\\
|\childdocmain{|\textit{main}|}|\\
\end{tabular}
\end{center}
%
If |\jobname| does not match the argument \textit{main} of |\childdocmain|,
it is assumed that |\jobname| points to the child file to be compiled.
When using |\childdocmain| with the main file specified as argument,
it suffices to start a child file
with just |\input{|\textit{main}|}|
without loading of the package and using |\childdocof|.
If instead all processing is done
with the appropriate \textsf{childdoc} directives,
the argument of \textit{main} of |\childdocmain| can be empty.

An alternative version of the command line processing described
in \secref{sec:commandline} using the detection mechanism reads:
%
\begin{center}
|... -jobname "|\textit{target}|" "|[\textit{flags}]%
[|\def\jobname{|\textit{dest}|}|]|\input{|\textit{main}|}"|
\end{center}

%%%%%%%%%%%%%%%%%%%%%%%%%%%%%%%%%%%%%%%%%%%%%%%%%%%%%%%%%%%%%%%%%%%%%%%%%%%%%%%%
\subsection{Manual Code}
\label{sec:manual}

In case one cannot be certain whether the definitions file |childdoc.def|
is installed on the target \TeX{} distribution
and one prefers not to ship it,
it is conceivable to paste a few relevant commands into the sources.

To that end, drop all statements |\input{childdoc.def}|
and perform the replacements as outlined below.
Instead of |\childdocmain{|\textit{main}|}| add the following code
to the top of the main file:
%
\begin{center}
\begin{tabular}{l}
|\||ifdefined\childdocname\endinput\||fi\newif\ifchilddoc|\\
|\edef\childdocname{\scantokens\expandafter{\jobname\noexpand}}|\\
|\def\childdocmain{|\textit{main}|}\||ifx\childdocmain\childdocname\||else|\\
|\childdoctrue\includeonly{\childdocname}\let\jobname\childdocmain\||fi|\\
\end{tabular}
\end{center}
%
Instead of |\childdocof{|\textit{main}|}| just include the main file
at the top of each child file:
%
\begin{center}
|\input{|\textit{main}|}|
\end{center}
%
A simple redirection |\childdocforward{|\textit{dest}|}| is achieved by:
%
\begin{center}
|\def\jobname{|\textit{dest}|}\input{\jobname}|
\end{center}
%
The redirection with prefix
|\childdocforwardprefix[|\textit{prefix}|]{|\textit{dest}|}|
is accomplished by:
%
\begin{center}
\begin{tabular}{l}
|{\edef\jobname{\scantokens\expandafter{\jobname\noexpand}}|\\
|\def\redirectjob |\textit{prefix}|#1~~~{\gdef\jobname{|\textit{dest}|#1}}|\\
|\expandafter\redirectjob\jobname~~~}\input{\jobname}|
\end{tabular}
\end{center}

In an alternative approach,
child documents can be compiled by a specific command line
without additional code or specific definitions:
%
\begin{center}
|... -jobname "|\textit{target}|" "|[\textit{flags}]%
|\includeonly{|\textit{dest}|}\input{|\textit{main}|}"|
\end{center}
%

%%%%%%%%%%%%%%%%%%%%%%%%%%%%%%%%%%%%%%%%%%%%%%%%%%%%%%%%%%%%%%%%%%%%%%%%%%%%%%%%
%%%%%%%%%%%%%%%%%%%%%%%%%%%%%%%%%%%%%%%%%%%%%%%%%%%%%%%%%%%%%%%%%%%%%%%%%%%%%%%%
\section{Information}

%%%%%%%%%%%%%%%%%%%%%%%%%%%%%%%%%%%%%%%%%%%%%%%%%%%%%%%%%%%%%%%%%%%%%%%%%%%%%%%%
\subsection{Copyright}

Copyright \copyright{} 2017--2018 Niklas Beisert

This work may be distributed and/or modified under the
conditions of the \LaTeX{} Project Public License, either version 1.3
of this license or (at your option) any later version.
The latest version of this license is in
  \url{http://www.latex-project.org/lppl.txt}
and version 1.3 or later is part of all distributions of \LaTeX{}
version 2005/12/01 or later.

This work has the LPPL maintenance status `maintained'.

The Current Maintainer of this work is Niklas Beisert.

This work consists of the files |README.txt|, |childdoc.ins| and |childdoc.dtx|
as well as the derived files |childdoc.def|, |cdocsamp.tex|
with |cdocsch1.tex|, |cdocsch2.tex|, |cdocspt3.tex|, |cdocspt4.tex|,
|cdocsdrf.tex|, |cdocsfn1.tex|, |cdocsfn2.tex|
as well as |childdoc.pdf|.

%%%%%%%%%%%%%%%%%%%%%%%%%%%%%%%%%%%%%%%%%%%%%%%%%%%%%%%%%%%%%%%%%%%%%%%%%%%%%%%%
\subsection{Files and Installation}

The package consists of the files:
%
\begin{center}
\begin{tabular}{ll}
    |README.txt|   & readme file \\
    |childdoc.ins| & installation file \\
    |childdoc.dtx| & source file \\
    |childdoc.def| & definition file \\
    |cdocsamp.tex| & sample main file \\
    |cdocsch1.tex| & sample include file \\
    |cdocsch2.tex| & sample include file \\
    |cdocspt3.tex| & sample part file \\
    |cdocspt4.tex| & sample part file \\
    |cdocsdrf.tex| & sample redirection file \\
    |cdocsfn1.tex| & sample redirection file \\
    |cdocsfn2.tex| & sample redirection file \\
    |childdoc.pdf| & manual
\end{tabular}
\end{center}
%
The distribution consists of the files
|README.txt|, |childdoc.ins| and |childdoc.dtx|.
%
\begin{itemize}
\item
Run (pdf)\LaTeX{} on |childdoc.dtx|
to compile the manual |childdoc.pdf| (this file).
\item
Run \LaTeX{} on |childdoc.ins| to create the definitions file |childdoc.def|
and the sample |cdocsamp.tex| with include files
|cdocsch1.tex|, |cdocsch2.tex|, |cdocspt3.tex|, |cdocspt4.tex|,
|cdocsdrf.tex|, |cdocsfn1.tex|, |cdocsfn2.tex|.
Then copy the file |childdoc.def| to an appropriate directory of your \LaTeX{}
distribution, e.g.\ \textit{texmf-root}|/tex/latex/childdoc|.
\end{itemize}

%%%%%%%%%%%%%%%%%%%%%%%%%%%%%%%%%%%%%%%%%%%%%%%%%%%%%%%%%%%%%%%%%%%%%%%%%%%%%%%%
\subsection{Related CTAN Packages}

There are several other packages which offer a similar functionality:
%
\begin{itemize}
\item
The packages
\href{http://ctan.org/pkg/docmute}{\textsf{docmute}},
\href{http://ctan.org/pkg/includex}{\textsf{includex}} and
\href{http://ctan.org/pkg/standalone}{\textsf{standalone}}
provide commands to include only the document body of
a child file thus allowing both files to be compiled individually.
\item
The packages \href{http://ctan.org/pkg/subdocs}{\textsf{subdocs}}
and \href{http://ctan.org/pkg/subfiles}{\textsf{subfiles}}
provide structures in which the main and child documents can be
encapsulated and allowing them to be compiled individually.
The inclusion mechanism is different from the conventional |\include|.
\item
The package \href{http://ctan.org/pkg/combine}{\textsf{combine}}
is an elaborate solution to combine several documents into one.
\end{itemize}
%
See also the CTAN topic \href{http://ctan.org/topic/subdocs}{\textsf{subdocs}}
for further related packages.
The present package differs from the above solutions in that
a document structure constructed with the conventional |\include| mechanism
just needs two extra commands at the top of every file
such that all constituent files can be compiled individually.

%%%%%%%%%%%%%%%%%%%%%%%%%%%%%%%%%%%%%%%%%%%%%%%%%%%%%%%%%%%%%%%%%%%%%%%%%%%%%%%%
%\subsection{Feature Suggestions}
%
%The following is a list of features which may be useful for future
%versions of this package:
%%
%\begin{itemize}
%\item
%\ldots
%\end{itemize}

%%%%%%%%%%%%%%%%%%%%%%%%%%%%%%%%%%%%%%%%%%%%%%%%%%%%%%%%%%%%%%%%%%%%%%%%%%%%%%%%
\subsection{Revision History}

%%%%%%%%%%%%%%%%%%%%%%%%%%%%%%%%%%%%%%%%
\paragraph{v2.0:} 2018/12/30

\begin{itemize}
\item
immediate forward processing
\item
added |\childdocby| mechanism
\item
manual restructured
\end{itemize}

%%%%%%%%%%%%%%%%%%%%%%%%%%%%%%%%%%%%%%%%
\paragraph{v1.6:} 2018/01/17

\begin{itemize}
\item
application for development of include files
\item
corrections to manual
\end{itemize}

%%%%%%%%%%%%%%%%%%%%%%%%%%%%%%%%%%%%%%%%
\paragraph{v1.5:} 2017/05/21

\begin{itemize}
\item
more complete structuring introduced
\item
|\childdocof| introduced
\item
|\childdoc| renamed to |\childdocmain|
\item
|\childredirect| renamed to |\childdocforward| and |\childdocforwardprefix|
and functionality expanded
\end{itemize}

%%%%%%%%%%%%%%%%%%%%%%%%%%%%%%%%%%%%%%%%
\paragraph{v1.0:} 2017/04/27

\begin{itemize}
\item
manual and install package
\item
first version published on CTAN
\end{itemize}

%%%%%%%%%%%%%%%%%%%%%%%%%%%%%%%%%%%%%%%%
\paragraph{v0.6:} 2017/04/26

\begin{itemize}
\item
redirection mechanism added
\end{itemize}

%%%%%%%%%%%%%%%%%%%%%%%%%%%%%%%%%%%%%%%%
\paragraph{v0.5:} 2017/04/26

\begin{itemize}
\item
functionality in definition file
\end{itemize}


%%%%%%%%%%%%%%%%%%%%%%%%%%%%%%%%%%%%%%%%%%%%%%%%%%%%%%%%%%%%%%%%%%%%%%%%%%%%%%%%
%%%%%%%%%%%%%%%%%%%%%%%%%%%%%%%%%%%%%%%%%%%%%%%%%%%%%%%%%%%%%%%%%%%%%%%%%%%%%%%%
%%%%%%%%%%%%%%%%%%%%%%%%%%%%%%%%%%%%%%%%%%%%%%%%%%%%%%%%%%%%%%%%%%%%%%%%%%%%%%%%
\appendix

\settowidth\MacroIndent{\rmfamily\scriptsize 000\ }

 \DocInput{childdoc.dtx}

\end{document}
%</driver>
% \fi
%
% %%%%%%%%%%%%%%%%%%%%%%%%%%%%%%%%%%%%%%%%%%%%%%%%%%%%%%%%%%%%%%%%%%%%%%%%%%%%%%
% %%%%%%%%%%%%%%%%%%%%%%%%%%%%%%%%%%%%%%%%%%%%%%%%%%%%%%%%%%%%%%%%%%%%%%%%%%%%%%
% \section{Sample}
%\iffalse
%<*samplemain>
%\fi
%
% The following presents a sample document
% with two chapters, two parts, a title page,
% a compile flag as well as three forwarding files to set the flag.
% It consists of eight |.tex| files:
% \begin{center}
% \begin{tabular}{ll}
% |cdocsamp.tex|&main file\\
% |cdocsch1.tex|&include file for chapter 1\\
% |cdocsch2.tex|&include file for chapter 2\\
% |cdocspt3.tex|&include file for part 3\\
% |cdocspt4.tex|&include file for part 4\\
% |cdocsdrf.tex|&forwarding file for main file in draft mode\\
% |cdocsfi1.tex|&forwarding file for final version of chapter 1\\
% |cdocsfi2.tex|&forwarding file for final version of chapter 2\\
% \end{tabular}
% \end{center}
% Each of the eight files can be compiled directly by the \LaTeX{} compiler.
%
% %%%%%%%%%%%%%%%%%%%%%%%%%%%%%%%%%%%%%%
% \paragraph{Main File.}
%
% The main file is called |cdocsamp.tex|.
%
% Load the \textsf{childdoc} definitions and
% declare the filename for the main document:
%    \begin{macrocode}
\input{childdoc.def}
\childdocmain{}
%    \end{macrocode}

% Optional override for |\version| flag:
%    \begin{macrocode}
%%\ifchilddoc\else\providecommand{\version}{draft}\fi
%    \end{macrocode}

% Define the default values for the |\version| flag
% (|final| for the main file and |draft| for childs):
%    \begin{macrocode}
\ifchilddoc
\providecommand{\version}{draft}
\else
\providecommand{\version}{final}
\fi
%    \end{macrocode}

% Load the standard document class:
%    \begin{macrocode}
\documentclass[12pt]{article}
%    \end{macrocode}

% Start the document body:
%    \begin{macrocode}
\begin{document}
%    \end{macrocode}

% Declare a title page.
% Print title, part of document being processed and version flag:
%    \begin{macrocode}
\addtocounter{page}{-1}
\begin{center}
{\LARGE\bfseries{}childdoc example\par}
\vspace{1cm}
\ifchilddoc
\ifchilddocmanual part\else chapter\fi:
`\childdocname' of `\childdocjob'\par
\else
main document: `\childdocjob'\par
\fi
version: \version\par
\end{center}
\newpage
%    \end{macrocode}

% Manually include selected file,
% otherwise process as usual:
%    \begin{macrocode}
\ifchilddocmanual
\section*{part `\childdocname'}
\input{\childdocname}
\else
%    \end{macrocode}

% Include the two chapters:
%    \begin{macrocode}
\include{cdocsch1}
\include{cdocsch2}
%    \end{macrocode}

% Include the two parts unless only chapters should be displayed:
%    \begin{macrocode}
\ifchilddoc\else
\section{part three}
\input{cdocspt3}
\section{part four}
\input{cdocspt4}
\fi
%    \end{macrocode}

% Process as usual until here:
%    \begin{macrocode}
\fi
%    \end{macrocode}

% End of document body:
%    \begin{macrocode}
\end{document}
%    \end{macrocode}
%\iffalse
%</samplemain>
%\fi
%
% %%%%%%%%%%%%%%%%%%%%%%%%%%%%%%%%%%%%%%
% \paragraph{Chapter Include Files.}
%
% The include files are called |cdocsch1.tex| and |cdocsch2.tex|.
%
%\iffalse
%<*samplechap1|samplechap2>
%\fi

% Optional override for |\version| flag:
%    \begin{macrocode}
%%\providecommand{\version}{final}
%    \end{macrocode}

% Include the main document:
%    \begin{macrocode}
\input{childdoc.def}
\childdocof{cdocsamp}
%    \end{macrocode}

%\iffalse
%</samplechap1|samplechap2>
%\fi
%
%\iffalse
%<*samplechap1>
%\fi
% Some text for chapter 1:
%    \begin{macrocode}
\section{one}
some text in chapter one
%    \end{macrocode}

%\iffalse
%</samplechap1>
%\fi
% Some text for chapter 2:
%\iffalse
%<*samplechap2>
%\fi
%    \begin{macrocode}
\section{two}
more text in chapter two
%    \end{macrocode}

%\iffalse
%</samplechap2>
%\fi
%
% %%%%%%%%%%%%%%%%%%%%%%%%%%%%%%%%%%%%%%
% \paragraph{Part Include Files.}
%
% The include files are called |cdocspt3.tex| and |cdocspt4.tex|.
%
%\iffalse
%<*samplepart3|samplepart4>
%\fi

% Optional override for |\version| flag:
%    \begin{macrocode}
%%\providecommand{\version}{final}
%    \end{macrocode}

% Include the main document:
%    \begin{macrocode}
\input{childdoc.def}
\childdocby{cdocsamp}
%    \end{macrocode}

%\iffalse
%</samplepart3|samplepart4>
%\fi
%
%\iffalse
%<*samplepart3>
%\fi
% Some text for part 3:
%    \begin{macrocode}
some text in part three
%    \end{macrocode}

%\iffalse
%</samplepart3>
%\fi
% Some text for part 4:
%\iffalse
%<*samplepart4>
%\fi
%    \begin{macrocode}
more text in part four
%    \end{macrocode}

%\iffalse
%</samplepart4>
%\fi
%
% %%%%%%%%%%%%%%%%%%%%%%%%%%%%%%%%%%%%%%
% \paragraph{Forwarding for a Complete Draft.}
%
% The following forwarding file |cdocsdrf.tex|
% compiles the main document in draft mode:
%\iffalse
%<*sampledraft>
%\fi
%    \begin{macrocode}
\def\version{draft}
\input{childdoc.def}
\childdocforward{cdocsamp}
%    \end{macrocode}

%\iffalse
%</sampledraft>
%\fi
%
% %%%%%%%%%%%%%%%%%%%%%%%%%%%%%%%%%%%%%%
% \paragraph{Forwarding for Final Version of the Chapters.}
%
% The following forwarding files |cdocsfn1.tex| and |cdocsfn2.tex|
% (with identical content)
% compile the final versions of the child documents
% |cdocsch1.tex| and |cdocsch2.tex|, respectively:
%\iffalse
%<*samplefinal>
%\fi
%    \begin{macrocode}
\def\version{final}
\input{childdoc.def}
\childdocforwardprefix[cdocsamp]{cdocsfn}{cdocsch}
%    \end{macrocode}

%\iffalse
%</samplefinal>
%\fi
%
% %%%%%%%%%%%%%%%%%%%%%%%%%%%%%%%%%%%%%%
% \paragraph{Command Line Processing.}
%
% The following three command lines generate the output files
% |cdocscld|, |cdocscl1| and |cdocscl2|
% which should be identical to
% |cdocsdrf|, |cdocsch1| and |cdocsfn2|, respectively:
% \begin{center}
% \begin{tabular}{l}
% |latex -jobname cdocscld \|\\
% |  "\def\version{draft}\input{childdoc.def}\childdocforward{cdocsamp}"|\\
% |latex -jobname cdocscl1 \|\\
% |  "\input{childdoc.def}\childdocforward[cdocsamp]{cdocsch1}"|\\
% |latex -jobname cdocscl2 \|\\
% |  "\def\version{final}\input{childdoc.def}\childdocforward{cdocsch2}"|
% \end{tabular}
% \end{center}
% Note that the trailing backslash on each first line
% merely continues the input to the second line
% (for convenient cut ant paste).
% Furthermore, the command |latex| can be replaced by any
% of its alternative versions such as |pdflatex|.
%
% %%%%%%%%%%%%%%%%%%%%%%%%%%%%%%%%%%%%%%%%%%%%%%%%%%%%%%%%%%%%%%%%%%%%%%%%%%%%%%
% %%%%%%%%%%%%%%%%%%%%%%%%%%%%%%%%%%%%%%%%%%%%%%%%%%%%%%%%%%%%%%%%%%%%%%%%%%%%%%
% \section{Implementation}
%\iffalse
%<*package>
%\fi
%
% This section describes the definitions file |childdoc.def|.

% The definitions cannot be loaded using |\usepackage| or |\RequirePackage|
% which has a mechanism to prevent loading a style file more than once.
% When loading the definitions by means of |\input|
% multiple instances have to be prevented manually:
%\iffalse
%This code needs to be before the `\ProvidesFile' directive
%which is defined at the beginning of this file.
%Therefore it is also placed there and commented out here.
%</package>
%<*discard>
%\fi
%    \begin{macrocode}
\ifdefined\childdocmain\endinput\fi
%    \end{macrocode}
%\iffalse
%</discard>
%<*package>
%\fi
%
% \macro{\ifchilddoc}
% \macro{\ifchilddocmanual}
% The conditional |\ifchilddoc| tells whether a
% child (true) or main (false) document is being compiled.
% The conditional |\ifchilddocmanual| tells whether
% the |\includeonly| mechanism is used (false) or
% the selection of child files must be performed manually (true).
% The definitions initialise to false:
%    \begin{macrocode}
\newif\ifchilddoc
\newif\ifchilddocmanual
%    \end{macrocode}

% \macro{\childdocname}
% \macro{\childdocjob}
% The macro |\childdocname| stores the name of the main document
% to be compiled. The macro |\childdocjob| stores the name of
% the document on which the \LaTeX{} compiler was originally invoked.
% The content of |\jobname| cannot be compared
% to filenames specified in the source due to different catcodes.
% The following code rescans |\jobname|, stores the result
% in |\childdocname| and saves a copy in |\childdocjob|:
%    \begin{macrocode}
\edef\childdocname{\scantokens\expandafter{\jobname\noexpand}}
\let\childdocjob\childdocname
%    \end{macrocode}

% \macro{\childdocdisable}
% The macro |\childdocdisable| prevents the main file
% from being processed more than once.
% At this stage, the main document command |\childdocmain|
% is assumed to be called once again where it should do nothing.
% Any subsequent call to it should prevent
% a secondary processing of the main document
% It overwrites the forwarding commands
% |\childdocof| and |\childdocforward|
% with empty macros to prevent further inclusions of the main document:
%    \begin{macrocode}
\newcommand{\childdocdisable}
{
  \renewcommand{\childdocmain}[1]{\renewcommand{\childdocmain}[1]{\endinput}}
  \renewcommand{\childdocof}[1]{}
  \renewcommand{\childdocby}[2][]{}
  \renewcommand{\childdocforward}[2][]{}
  \renewcommand{\childdocdisable}{}
}
%    \end{macrocode}

% \macro{\childdocmain}
% The macro |\childdocmain| is to be called at the top of the main file
% with nothing or the main filename (without extension) as argument.
% First, it breaks loops.
% If the argument is not empty and does not match |\childdocname|
% (which is set by the first inclusion of |childdoc.def|),
% |\ifchilddoc| is set to true, |\includeonly| is applied to the child file
% and |\jobname| is set to the main file
% (for proper handling of |.aux| files):
%    \begin{macrocode}
\newcommand{\childdocmain}[1]
{
  \childdocdisable\childdocmain{}
  \if?#1?\else
    \begingroup
      \def\childdoctmp{#1}
      \ifx\childdoctmp\childdocname
        \def\childdoctmp{}
      \else
        \def\childdoctmp
        {
          \childdoctrue
          \includeonly{\childdocname}
          \def\childdocjob{#1}
          \def\jobname{#1}
        }
      \fi
      \expandafter
    \endgroup
    \childdoctmp
  \fi
}
%    \end{macrocode}

% \macro{\childdocof}
% The command |\childdocof| redirects
% compilation to the main file |#1|.
%    \begin{macrocode}
\newcommand{\childdocof}[1]
{
  \childdocdisable
  \childdoctrue
  \includeonly{\childdocname}
  \def\jobname{#1}
  \def\childdocjob{#1}
  \input{#1}
}
%    \end{macrocode}

% \macro{\childdocby}
% The command |\childdocby| ....
%    \begin{macrocode}
\newcommand{\childdocby}[2][]
{
  \childdocdisable
  \childdoctrue
  \childdocmanualtrue
  \if?#1?\else
    \def\jobname{#2}
  \fi
  \def\childdocjob{#2}
  \input{#2}
  \endinput
}
%    \end{macrocode}

% \macro{\childdocforward}
% The command |\childdocforward| redirects
% compilation to the main file or
% (if the optional argument is given) a child file.
% Parameters are set as if the main file
% or a child file starting with |\childdocof| was compiled.
% Then compilation is handed over to the main file:
%    \begin{macrocode}
\newcommand{\childdocforward}[2][]
{
  \begingroup
    \if?#1?
      \def\childdoctmp
      {
        \def\childdocname{#2}
        \def\childdocjob{#2}
        \def\jobname{#2}
        \input{#2}
        \endinput
      }
    \else
      \def\childdoctmp
      {
        \childdocdisable
        \def\childdocname{#2}
        \childdoctrue
        \includeonly{#2}
        \def\childdocjob{#1}
        \def\jobname{#1}
        \input{#1}
        \endinput
      }
    \fi
    \expandafter
  \endgroup
  \childdoctmp
}
%    \end{macrocode}

% \macro{\childdocforwardprefix}
% The command |\childdocforwardprefix| redirects
% compilation to the main or a child file by means of a pattern.
% The prefix |#1| in the current filename is replaced by |#2|
% and the suffix of the current filename is kept
% (it is assumed that the filename does not contain the substring `|~~~|'
% which is used as a delimiter).
% Compilation is handed over to the new file by |\childdocforward|:
%    \begin{macrocode}
\newcommand{\childdocforwardprefix}[3][]
{
  \begingroup
    \def\childdocextract #2##1~~~{\def\childdoctmp{\childdocforward[#1]{#3##1}}}
    \expandafter\childdocextract\childdocname~~~
    \expandafter
  \endgroup
  \childdoctmp
}
%    \end{macrocode}

% \macro{\childdoc}
% The deprecated macro |\childdoc| is a legacy version of |\childdocmain|:
%    \begin{macrocode}
\newcommand{\childdoc}{\childdocmain}
%    \end{macrocode}

% \macro{\childdocredirect}
% The deprecated macro |\childdocredirect| is a legacy version
% of |\childdocforward| and |\childdocforwardprefix|:
%    \begin{macrocode}
\newcommand{\childdocredirect}[2][]
{
  \begingroup
    \if?#1?
      \def\childdoctmp{\childdocforward{#2}}
    \else
      \def\childdoctmp{\childdocforwardprefix{#1}{#2}}
    \fi
    \expandafter
  \endgroup
  \childdoctmp
}
%    \end{macrocode}

%\iffalse
%</package>
%\fi
%
\endinput
|\\
|\childdocforward{|\textit{main}|}|
\end{tabular}
\end{center}
%
Likewise, the following files |final|\textit{nn}|.tex|
compile the final version of the child document
|child|\textit{nn}|.tex|:
%
\begin{center}
\begin{tabular}{l}
|\def\version{final}|\\
|% \iffalse
%
% childdoc.dtx Copyright (C) 2017-2018 Niklas Beisert
%
% This work may be distributed and/or modified under the
% conditions of the LaTeX Project Public License, either version 1.3
% of this license or (at your option) any later version.
% The latest version of this license is in
%   http://www.latex-project.org/lppl.txt
% and version 1.3 or later is part of all distributions of LaTeX
% version 2005/12/01 or later.
%
% This work has the LPPL maintenance status `maintained'.
%
% The Current Maintainer of this work is Niklas Beisert.
%
% This work consists of the files childdoc.dtx and childdoc.ins
% and the derived files childdoc.def and cdocsamp.tex with
% cdocsch1.tex, cdocsch2.tex, cdocsdrf.tex, cdocsfn1.tex, cdocsfn2.tex.
%
%<package>\ifdefined\childdocmain\endinput\fi
%<package>\ProvidesFile{childdoc.def}[2018/12/30 v2.0 child document driver]
%<samplemain>\ProvidesFile{cdocsamp.tex}[2018/12/30 v2.0 sample for childdoc]
%<*driver>
%\ProvidesFile{childdoc.drv}[2018/12/30 v2.0 childdoc reference manual file]
\PassOptionsToClass{10pt,a4paper}{article}
\documentclass{ltxdoc}

\usepackage[margin=35mm]{geometry}
\usepackage{hyperref}
\usepackage{hyperxmp}
\usepackage[usenames]{color}

\hypersetup{colorlinks=true}
\hypersetup{pdfstartview=FitH}
\hypersetup{pdfpagemode=UseNone}
\hypersetup{pdfsource={}}
\hypersetup{pdflang={en-UK}}
\hypersetup{pdfcopyright={Copyright 2017-2018 Niklas Beisert.
  This work may be distributed and/or modified under the
  conditions of the LaTeX Project Public License, either version 1.3
  of this license or (at your option) any later version.}}
\hypersetup{pdflicenseurl={http://www.latex-project.org/lppl.txt}}
\hypersetup{pdfcontactaddress={ETH Zurich, ITP, HIT K,
  Wolfgang-Pauli-Strasse 27}}
\hypersetup{pdfcontactpostcode={8093}}
\hypersetup{pdfcontactcity={Zurich}}
\hypersetup{pdfcontactcountry={Switzerland}}
\hypersetup{pdfcontactemail={nbeisert@itp.phys.ethz.ch}}
\hypersetup{pdfcontacturl={http://people.phys.ethz.ch/\xmptilde nbeisert/}}

\newcommand{\secref}[1]{\hyperref[#1]{section \ref*{#1}}}

\parskip1ex
\parindent0pt
\let\olditemize\itemize
\def\itemize{\olditemize\parskip0pt}

\begin{document}

\title{The \textsf{childdoc} Package}
\hypersetup{pdftitle={The childdoc Package}}
\author{Niklas Beisert\\[2ex]
  Institut f\"ur Theoretische Physik\\
  Eidgen\"ossische Technische Hochschule Z\"urich\\
  Wolfgang-Pauli-Strasse 27, 8093 Z\"urich, Switzerland\\[1ex]
  \href{mailto:nbeisert@itp.phys.ethz.ch}
  {\texttt{nbeisert@itp.phys.ethz.ch}}}
\hypersetup{pdfauthor={Niklas Beisert}}
\hypersetup{pdfsubject={Manual for the LaTeX2e Package childdoc}}
\date{30 December 2018, \textsf{v2.0}}
\maketitle

\begin{abstract}\noindent
\textsf{childdoc} is a \LaTeXe{} package
that enables the direct compilation
of document sections included by |\include|
to individual files.
\end{abstract}

\begingroup
\parskip0ex
\tableofcontents
\endgroup

%%%%%%%%%%%%%%%%%%%%%%%%%%%%%%%%%%%%%%%%%%%%%%%%%%%%%%%%%%%%%%%%%%%%%%%%%%%%%%%%
%%%%%%%%%%%%%%%%%%%%%%%%%%%%%%%%%%%%%%%%%%%%%%%%%%%%%%%%%%%%%%%%%%%%%%%%%%%%%%%%
\section{Introduction}

\LaTeX{} provides a mechanism to structure a large document (such as a book)
into a main file and several child files (containing the chapters)
using the |\include| command.
This mechanism is beneficial for documents
which span hundreds of pages in order to
make the source file(s) more manageable.
Moreover, compilation can be restricted to
selected child files by means of the |\includeonly| command.
The latter feature can be used to reduce the compilation time while editing
(this was significantly more useful in the earlier days of \LaTeX{})
or to generate a smaller document which is easier to navigate.
Another application of |\includeonly| is to generate
documents consisting of selected parts of the complete document.

However, there are a few drawbacks of the plain |\include| mechanism:
\begin{itemize}
\item
The child files cannot be compiled on their own,
they can only be compiled via the main file.
A naive editing environment
(such as a text editor with an option
to have the current file processed by \LaTeX)
may require one to switch to the main file before compiling;
attempting to compile the child file produces errors.
\item
The main file must be modified (each time)
to adjust the |\includeonly| command
to the present needs. This easily leaves the main file in a messy state.
\item
The generated document will always carry the filename
of the main document. This is inconvenient if
several child files are to be compiled and
to be kept for distribution.
\end{itemize}

The present package provides a simple interface
to make child files individually compilable by \LaTeX{}.
Compiling a child file then has the same effect as compiling
the main file with an |\includeonly| command
to select the appropriate child.
Moreover the generated document will carry the name of the child
rather than the main file.
This resolves all three above issues.

This feature is meant to make the editing of books,
thesis documents and lecture notes somewhat more convenient.
However, the package can also be used efficiently for
composing a series of documents (such as exercise sheets)
which are typically distributed individually.
It then assists the author in generating the individual documents
(potentially in different versions)
as well as a document containing the collected series.
Another application is in developing style files
or other kinds of included material
where compilation of the style file could redirect
to a sample or test file.

%%%%%%%%%%%%%%%%%%%%%%%%%%%%%%%%%%%%%%%%%%%%%%%%%%%%%%%%%%%%%%%%%%%%%%%%%%%%%%%%
%%%%%%%%%%%%%%%%%%%%%%%%%%%%%%%%%%%%%%%%%%%%%%%%%%%%%%%%%%%%%%%%%%%%%%%%%%%%%%%%
\section{Usage}

First of all, the package \textsf{childdoc} is \emph{not} a standard
\LaTeXe{} |.sty| style file! Therefore it needs to be invoked in
a non-standard way.

%%%%%%%%%%%%%%%%%%%%%%%%%%%%%%%%%%%%%%%%%%%%%%%%%%%%%%%%%%%%%%%%%%%%%%%%%%%%%%%%
\subsection{Included Files}
\label{sec:include}

%%%%%%%%%%%%%%%%%%%%%%%%%%%%%%%%%%%%%%%%
\DescribeMacro{\childdocmain}
To use the package, add the commands
\begin{center}
\begin{tabular}{l}
|\input{childdoc.def}|\\
|\childdocmain{}|\\
\end{tabular}
\end{center}
at the very top of the main \LaTeX{} file,
in particular \emph{before} the |\documentclass| statement!
The argument of |\childdocmain| should be left empty
(but it must be present).

%%%%%%%%%%%%%%%%%%%%%%%%%%%%%%%%%%%%%%%%
\DescribeMacro{\childdocof}
Furthermore, add the commands
\begin{center}
\begin{tabular}{l}
|\input{childdoc.def}|\\
|\childdocof{|\textit{main}|}|\\
\end{tabular}
\end{center}
at the top of every child file \textit{child}
which is included by |\include{|\textit{child}|}|
from within the main file
(or at least for those files to be compiled individually).
The argument \textit{main} must be the filename of the main file.

There are a couple of
considerations in setting up the main and child documents:

%%%%%%%%%%%%%%%%%%%%%%%%%%%%%%%%%%%%%%%%
\paragraph{Restrictions.}

Please note the following restrictions:
\begin{itemize}
\item
|\childdocmain| must be called with one argument \textit{main}
to ensure compatibility with earlier version of the package.
It must either be empty (|\childdocmain{}|)
or precisely match the filename of the main file in which it is specified.
See \secref{sec:detection} for further information.
\item
The filename \textit{main} must be specified without the |.tex| extension.
\item
The filename \textit{main} is case sensitive
(even in case-insensitive file systems)
due to internal string comparison.
\item
The argument \textit{main} should be fully expanded, it cannot be a macro.
\item
Subdirectories and special characters should be avoided in filenames.
\item
The command |\childdocmain{|\textit{main}|}| must be followed by a whitespace.
It should not be followed immediately by another command
or by a comment mark `|%|'.
This is because the \TeX{} parser reads the token immediately following
the argument of |\childdocmain| and puts it
at the beginning of every child section;
however, a white\-space is ignored.
\end{itemize}

%%%%%%%%%%%%%%%%%%%%%%%%%%%%%%%%%%%%%%%%
\paragraph{Content of Main File.}

It is advisable to place all content in the child files included by |\include|.
Any output contained in the main file will appear in all child documents
unless suppressed manually;
it cannot be suppressed automatically by the |\includeonly| directive
and thus should normally be avoided.
A method to include some content in the main file
by means of conditional processing is described in \secref{sec:conditional}.

%%%%%%%%%%%%%%%%%%%%%%%%%%%%%%%%%%%%%%%%
\paragraph{Page Numbering.}

When only a part of the document is compiled,
the appropriate numbering of pages
(as well as other status parameters)
is determined from the |.aux| files.
The latter contain information from previous passes.
However this information needs to propagate through
all intermediate child documents.
Therefore the page numbering in child documents may well
be inconsistent until the complete document is compiled at least once.

A useful (if unconventional) way to always ensure a consistent
page numbering is to restart the numbering in each child document
and denote the pages by `\textit{child}|.|\textit{page}'
where \textit{child} represents the chapter/section number of the child file.
This can be achieved by the command
|\numberwithin{page}{|\textit{child}|}|
of the \textsf{amsmath} package
where \textit{child} can be |chapter| or |section|
depending on the chosen structuring.
Alternatively, one can modify the macro |\thepage| appropriately
and reset the counter |page| at the start of each child file.

%%%%%%%%%%%%%%%%%%%%%%%%%%%%%%%%%%%%%%%%%%%%%%%%%%%%%%%%%%%%%%%%%%%%%%%%%%%%%%%%
\subsection{Conditional Processing}
\label{sec:conditional}

The package provides a mechanism to compile different versions
of a document. To customise the versions further some conditional processing
can come in handy to distinguish which version is being compiled.
The package provides two macros to describe the compilation context:

%%%%%%%%%%%%%%%%%%%%%%%%%%%%%%%%%%%%%%%%
\DescribeMacro{\ifchilddoc}
The conditional |\ifchilddoc| distinguishes between the compilation of
child documents and the main document:
%
\begin{center}
|\ifchilddoc |\textit{child-code}| |[|\||else |\textit{main-code}]| \||fi|
\end{center}

%%%%%%%%%%%%%%%%%%%%%%%%%%%%%%%%%%%%%%%%
\DescribeMacro{\childdocname}
\DescribeMacro{\childdocjob}
The macro |\childdocname| contains the filename (without extension)
of the main or child file being processed.
Note that |\childdocjob| will always contain the name of the main file.

%%%%%%%%%%%%%%%%%%%%%%%%%%%%%%%%%%%%%%%%
\paragraph{Title Page.}

Conditional processing can be used to include a title or banner page
in the main document when proper precautions are taken.
Importantly, the code in the main file should ensure that the page counter
(as well as other status parameters which are stored in the |.aux| files)
takes the same value after the conditional processing.
Otherwise the page numbers may take divergent values
depending on which part is compiled.

For example, a title page could be declared by:
%
\begin{center}
\begin{tabular}{l}
|\ifchilddoc\||else|\\
|\addtocounter{page}{-1}|\\
\textit{code for title page}\\
|\newpage|\\
|\||fi|
\end{tabular}
\end{center}
%
A banner page for the child documents can be generated by:
%
\begin{center}
\begin{tabular}{l}
|\ifchilddoc|\\
|\addtocounter{page}{-1}|\\
\textit{code for banner page}\\
|\newpage|\\
|\||fi|
\end{tabular}
\end{center}
%
Here one could write a message such as:
\begin{center}
|This is the part \childdocname{} of \childdocjob{}.|
\end{center}

%%%%%%%%%%%%%%%%%%%%%%%%%%%%%%%%%%%%%%%%%%%%%%%%%%%%%%%%%%%%%%%%%%%%%%%%%%%%%%%%
\subsection{Flags}
\label{sec:flags}

The package makes it easy to generate different versions
of the main or child documents.
To this end compilation flags can be defined
and assigned different default values.
They will be particularly useful in conjunction
with the forwarding mechanism described in \secref{sec:forward}.

For example, it may be useful to have a flag |\version|
which can be set to |draft| or |final|.
The document source will contain some conditional code
depending on the value of |\version|.
Suppose further, the flag should default to |final| for the main file
and to |draft| for child files
which is a natural assignment for editing the document.
This is achieved by placing the following code
in the preamble of the main document
(below the |\childdocmain| directive):
%
\begin{center}
\begin{tabular}{l}
|\ifchilddoc|\\
|\providecommand{\version}{draft}|\\
|\||else|\\
|\providecommand{\version}{final}|\\
|\||fi|
\end{tabular}
\end{center}
%
The definition by |\providecommand| makes sure
that previous definitions are not overwritten.
Further statements |\providecommand{\version}{...}|
can thus be added before the above code to override it.

For the main file, one might add a line
(between |\childdocmain| and the above block)
%
\begin{center}
|%\ifchilddoc\||else\providecommand{\version}{draft}\||fi|
\end{center}
%
which can be uncommented to produce a draft version.
Likewise one can add a line to the very top of a child file
(above the |\childdocof{|\textit{main}|}| directive)
%
\begin{center}
|%\providecommand{\version}{final}|
\end{center}
%
which can be uncommented to produce the final version of this child document.

%%%%%%%%%%%%%%%%%%%%%%%%%%%%%%%%%%%%%%%%%%%%%%%%%%%%%%%%%%%%%%%%%%%%%%%%%%%%%%%%
\subsection{Forwarding}
\label{sec:forward}

Different versions of the main or child documents
using compilation flags as described in \secref{sec:flags}
can be (permanently) stored in different files
for convenient compilation, viewing and distribution.
To this end, the package defines a command
to pass on compilation to a different file:

%%%%%%%%%%%%%%%%%%%%%%%%%%%%%%%%%%%%%%%%
\DescribeMacro{\childdocforward}
The command |\childdocforward| redirects processing to
another source file:
%
\begin{center}
\begin{tabular}{l}
|\input{childdoc.def}|\\
|\childdocforward[|\textit{main}|]{|\textit{dest}|}|\\
\end{tabular}
\end{center}
%
The argument \textit{dest} is the destination file
(without extension).
It should be the main file or one of the child files.
Note that further \textsf{childdoc} directives
such as |\childdocof| and |\childdocforward|
in the indicated file will be processed in this form.
The optional argument \textit{main}
passes on directly to the main file \textit{main}
while pretending to compile the child \textit{dest}.
This form behaves as if \textit{dest}
issues |\childdocof{|\textit{main}|}| right away,
and no further \textsf{childdoc} directives will be processed.

%%%%%%%%%%%%%%%%%%%%%%%%%%%%%%%%%%%%%%%%
\DescribeMacro{\...prefix}
In the alternative form |\childdocforwardprefix|,
%
\begin{center}
\begin{tabular}{l}
|\input{childdoc.def}|\\
|\childdocforwardprefix[|\textit{main}|]{|\textit{prefix}|}{|\textit{dest}|}|
\end{tabular}
\end{center}
%
the destination file is determined by a pattern
depending on the current file:
To make this work, the current file must be called
`{\textit{prefix}\hspace{0.2em}\textit{suffix}}'
with \textit{prefix} matching precisely the argument.
Processing is then passed on to the file
`{\textit{dest}\hspace{0.2em}\textit{suffix}}'.
Surely, the same effect is achieved by
directly specifying the
argument `{\textit{dest}\hspace{0.2em}\textit{suffix}}'
in the first form.
However, that requires to set up a different file
for each child. With the alternative form of the command
all these files can have exactly the same content
which simplifies setting them up and maintaining them.

For example, the following file |draft.tex|
with a compilation flag |\version| as described in \secref{sec:flags}
compiles the main document as a draft:
%
\begin{center}
\begin{tabular}{l}
|\def\version{draft}|\\
|\input{childdoc.def}|\\
|\childdocforward{|\textit{main}|}|
\end{tabular}
\end{center}
%
Likewise, the following files |final|\textit{nn}|.tex|
compile the final version of the child document
|child|\textit{nn}|.tex|:
%
\begin{center}
\begin{tabular}{l}
|\def\version{final}|\\
|\input{childdoc.def}|\\
|\childdocforwardprefix{final}{child}|
\end{tabular}
\end{center}
%

Note that when several versions of a main file and/or of each child file
are to be generated, it may be convenient to set up a |Makefile| or
shell script to automatise the process.

%%%%%%%%%%%%%%%%%%%%%%%%%%%%%%%%%%%%%%%%%%%%%%%%%%%%%%%%%%%%%%%%%%%%%%%%%%%%%%%%
\subsection{Command Line Processing}
\label{sec:commandline}

The effect of redirection files can also be achieved by invoking
the \LaTeX{} compiler with a more elaborate command line.
Most conveniently this should be done as part
of a shell script or a |Makefile|.

When using \textsf{childdoc} in the main file, the following
command lines effectively perform a redirection
(note that depending on the shell being used,
backslashes may have to be doubled: `|\|' $\to$ `|\\|'):
%
\begin{center}
|... -jobname "|\textit{target}|" |\\|"|[\textit{flags}]%
|\input{childdoc.def}\childdocforward[|\textit{main}|]{|\textit{dest}|}"|
\end{center}
%
Here \textit{target} is the name of the output file,
\textit{main} is the name of the main file
and \textit{dest} is the name of the main or child file to be processed
(all filenames without extensions).
The optional argument \textit{main} can be omitted
if \textit{main} matches \textit{dest}.
Optionally, compilation \textit{flags} can be defined via |\def| commands.
This command line makes the \TeX{} engine believe
it is compiling the file \textit{target}
whose content is specified as the latter parameter.
The provided code then forwards the processing to
\textit{main} or \textit{dest} as described in \secref{sec:forward}.

%%%%%%%%%%%%%%%%%%%%%%%%%%%%%%%%%%%%%%%%%%%%%%%%%%%%%%%%%%%%%%%%%%%%%%%%%%%%%%%%
\subsection{Include by Input}
\label{sec:input}

Including child documents by |\include| has some restrictions by design.
Most notably, the content of a child document always occupies
its own set of pages; pages cannot be shared between child documents.
Usually, this behaviour makes perfect sense
because each child document contain an essential part of the document.
However, in some situations it may be desirable to compose
a document from a collection of parts
without having mandatory page breaks between then.
For this case, the package
provides a mechanism to include parts
by |\input| which can also be processed individually.
However, by construction this mechanism
requires manual handling of the content to be output.

%%%%%%%%%%%%%%%%%%%%%%%%%%%%%%%%%%%%%%%%
\DescribeMacro{\ifchilddocmanual}
The main file should be prepared as usual, see \secref{sec:include}.
However, the document body must make a distinction
between processing of an individual part and of the main document, e.g.:
%
\begin{center}
\begin{tabular}{l}
|\ifchilddocmanual|\\
|\input{\childdocname}|\\
|\||else|\\
\textit{document body with }|\input{|\textit{part}|}|\\
|\||fi|
\end{tabular}
\end{center}
%
The conditional |\ifchilddocmanual| is true whenever
a part to be included by |\input| is being compiled,
and the name of the part is stored in |\childdocname|.

%%%%%%%%%%%%%%%%%%%%%%%%%%%%%%%%%%%%%%%%
\DescribeMacro{\childdocby}
Each part to be included by |\input| should start with:
%
\begin{center}
\begin{tabular}{l}
|\input{childdoc.def}|\\
|\childdocby{|\textit{main}|}|\\
\end{tabular}
\end{center}
%
The directive |\childdocby| is similar to |\childdocof|
described in \secref{sec:include},
but the subsequent selection of content must be done manually.
To that end, both |\ifchilddoc| and |\ifchilddocmanual|
will be true upon processing of a part,
and the name of the part is stored in |\childdocname|.
Note that |\jobname| will be set to the filename of the current part
so that each part receives an individual |.aux| file
that does not interfere with the |.aux| file(s) of the main document.
This behaviour can be altered by the alternative form
|\childdocby[*]{|\textit{main}|}| (with a non-empty optional argument)
which uses the |.aux| file of the main document
by setting |\jobname| to \textit{main}.

%%%%%%%%%%%%%%%%%%%%%%%%%%%%%%%%%%%%%%%%%%%%%%%%%%%%%%%%%%%%%%%%%%%%%%%%%%%%%%%%
\subsection{Driver Development}
\label{sec:driver}

The \textsf{childdoc} mechanism can also be use for the development
of definition files such as \LaTeX{} styles or classes.
This case differs from the above setup with multiple parts
included by |\include| in that no |\includeonly| should be invoked.
This can be achieved by starting the include file
(before |\ProvidesPackage|) with:
%
\begin{center}
\begin{tabular}{l}
|\input{childdoc.def}|\\
|\childdocforward{|\textit{main}|}|\\
\end{tabular}
\end{center}
%
or alternatively with:
%
\begin{center}
\begin{tabular}{l}
|\input{childdoc.def}|\\
|\childdocby{|\textit{main}|}|\\
\end{tabular}
\end{center}
%
Both forms have slightly different effects as described above.
The main file is prepared as usual, see \secref{sec:include}.

%%%%%%%%%%%%%%%%%%%%%%%%%%%%%%%%%%%%%%%%%%%%%%%%%%%%%%%%%%%%%%%%%%%%%%%%%%%%%%%%
\subsection{Legacy Detection}
\label{sec:detection}

The directive |\childdocmain| in the main file can detect
whether the complete document or merely a child is to be compiled
even without using the directive |\childdocof|.
This method is deprecated because it is less robust
and there is no compelling reason to use it;
it is merely provided for backward compatibility
and it may be removed in future versions.

If the detection mechanism is to be used,
it is mandatory to correctly specify
the filename of the main file as the argument of |\childdocmain|:
%
\begin{center}
\begin{tabular}{l}
|\input{childdoc.def}|\\
|\childdocmain{|\textit{main}|}|\\
\end{tabular}
\end{center}
%
If |\jobname| does not match the argument \textit{main} of |\childdocmain|,
it is assumed that |\jobname| points to the child file to be compiled.
When using |\childdocmain| with the main file specified as argument,
it suffices to start a child file
with just |\input{|\textit{main}|}|
without loading of the package and using |\childdocof|.
If instead all processing is done
with the appropriate \textsf{childdoc} directives,
the argument of \textit{main} of |\childdocmain| can be empty.

An alternative version of the command line processing described
in \secref{sec:commandline} using the detection mechanism reads:
%
\begin{center}
|... -jobname "|\textit{target}|" "|[\textit{flags}]%
[|\def\jobname{|\textit{dest}|}|]|\input{|\textit{main}|}"|
\end{center}

%%%%%%%%%%%%%%%%%%%%%%%%%%%%%%%%%%%%%%%%%%%%%%%%%%%%%%%%%%%%%%%%%%%%%%%%%%%%%%%%
\subsection{Manual Code}
\label{sec:manual}

In case one cannot be certain whether the definitions file |childdoc.def|
is installed on the target \TeX{} distribution
and one prefers not to ship it,
it is conceivable to paste a few relevant commands into the sources.

To that end, drop all statements |\input{childdoc.def}|
and perform the replacements as outlined below.
Instead of |\childdocmain{|\textit{main}|}| add the following code
to the top of the main file:
%
\begin{center}
\begin{tabular}{l}
|\||ifdefined\childdocname\endinput\||fi\newif\ifchilddoc|\\
|\edef\childdocname{\scantokens\expandafter{\jobname\noexpand}}|\\
|\def\childdocmain{|\textit{main}|}\||ifx\childdocmain\childdocname\||else|\\
|\childdoctrue\includeonly{\childdocname}\let\jobname\childdocmain\||fi|\\
\end{tabular}
\end{center}
%
Instead of |\childdocof{|\textit{main}|}| just include the main file
at the top of each child file:
%
\begin{center}
|\input{|\textit{main}|}|
\end{center}
%
A simple redirection |\childdocforward{|\textit{dest}|}| is achieved by:
%
\begin{center}
|\def\jobname{|\textit{dest}|}\input{\jobname}|
\end{center}
%
The redirection with prefix
|\childdocforwardprefix[|\textit{prefix}|]{|\textit{dest}|}|
is accomplished by:
%
\begin{center}
\begin{tabular}{l}
|{\edef\jobname{\scantokens\expandafter{\jobname\noexpand}}|\\
|\def\redirectjob |\textit{prefix}|#1~~~{\gdef\jobname{|\textit{dest}|#1}}|\\
|\expandafter\redirectjob\jobname~~~}\input{\jobname}|
\end{tabular}
\end{center}

In an alternative approach,
child documents can be compiled by a specific command line
without additional code or specific definitions:
%
\begin{center}
|... -jobname "|\textit{target}|" "|[\textit{flags}]%
|\includeonly{|\textit{dest}|}\input{|\textit{main}|}"|
\end{center}
%

%%%%%%%%%%%%%%%%%%%%%%%%%%%%%%%%%%%%%%%%%%%%%%%%%%%%%%%%%%%%%%%%%%%%%%%%%%%%%%%%
%%%%%%%%%%%%%%%%%%%%%%%%%%%%%%%%%%%%%%%%%%%%%%%%%%%%%%%%%%%%%%%%%%%%%%%%%%%%%%%%
\section{Information}

%%%%%%%%%%%%%%%%%%%%%%%%%%%%%%%%%%%%%%%%%%%%%%%%%%%%%%%%%%%%%%%%%%%%%%%%%%%%%%%%
\subsection{Copyright}

Copyright \copyright{} 2017--2018 Niklas Beisert

This work may be distributed and/or modified under the
conditions of the \LaTeX{} Project Public License, either version 1.3
of this license or (at your option) any later version.
The latest version of this license is in
  \url{http://www.latex-project.org/lppl.txt}
and version 1.3 or later is part of all distributions of \LaTeX{}
version 2005/12/01 or later.

This work has the LPPL maintenance status `maintained'.

The Current Maintainer of this work is Niklas Beisert.

This work consists of the files |README.txt|, |childdoc.ins| and |childdoc.dtx|
as well as the derived files |childdoc.def|, |cdocsamp.tex|
with |cdocsch1.tex|, |cdocsch2.tex|, |cdocspt3.tex|, |cdocspt4.tex|,
|cdocsdrf.tex|, |cdocsfn1.tex|, |cdocsfn2.tex|
as well as |childdoc.pdf|.

%%%%%%%%%%%%%%%%%%%%%%%%%%%%%%%%%%%%%%%%%%%%%%%%%%%%%%%%%%%%%%%%%%%%%%%%%%%%%%%%
\subsection{Files and Installation}

The package consists of the files:
%
\begin{center}
\begin{tabular}{ll}
    |README.txt|   & readme file \\
    |childdoc.ins| & installation file \\
    |childdoc.dtx| & source file \\
    |childdoc.def| & definition file \\
    |cdocsamp.tex| & sample main file \\
    |cdocsch1.tex| & sample include file \\
    |cdocsch2.tex| & sample include file \\
    |cdocspt3.tex| & sample part file \\
    |cdocspt4.tex| & sample part file \\
    |cdocsdrf.tex| & sample redirection file \\
    |cdocsfn1.tex| & sample redirection file \\
    |cdocsfn2.tex| & sample redirection file \\
    |childdoc.pdf| & manual
\end{tabular}
\end{center}
%
The distribution consists of the files
|README.txt|, |childdoc.ins| and |childdoc.dtx|.
%
\begin{itemize}
\item
Run (pdf)\LaTeX{} on |childdoc.dtx|
to compile the manual |childdoc.pdf| (this file).
\item
Run \LaTeX{} on |childdoc.ins| to create the definitions file |childdoc.def|
and the sample |cdocsamp.tex| with include files
|cdocsch1.tex|, |cdocsch2.tex|, |cdocspt3.tex|, |cdocspt4.tex|,
|cdocsdrf.tex|, |cdocsfn1.tex|, |cdocsfn2.tex|.
Then copy the file |childdoc.def| to an appropriate directory of your \LaTeX{}
distribution, e.g.\ \textit{texmf-root}|/tex/latex/childdoc|.
\end{itemize}

%%%%%%%%%%%%%%%%%%%%%%%%%%%%%%%%%%%%%%%%%%%%%%%%%%%%%%%%%%%%%%%%%%%%%%%%%%%%%%%%
\subsection{Related CTAN Packages}

There are several other packages which offer a similar functionality:
%
\begin{itemize}
\item
The packages
\href{http://ctan.org/pkg/docmute}{\textsf{docmute}},
\href{http://ctan.org/pkg/includex}{\textsf{includex}} and
\href{http://ctan.org/pkg/standalone}{\textsf{standalone}}
provide commands to include only the document body of
a child file thus allowing both files to be compiled individually.
\item
The packages \href{http://ctan.org/pkg/subdocs}{\textsf{subdocs}}
and \href{http://ctan.org/pkg/subfiles}{\textsf{subfiles}}
provide structures in which the main and child documents can be
encapsulated and allowing them to be compiled individually.
The inclusion mechanism is different from the conventional |\include|.
\item
The package \href{http://ctan.org/pkg/combine}{\textsf{combine}}
is an elaborate solution to combine several documents into one.
\end{itemize}
%
See also the CTAN topic \href{http://ctan.org/topic/subdocs}{\textsf{subdocs}}
for further related packages.
The present package differs from the above solutions in that
a document structure constructed with the conventional |\include| mechanism
just needs two extra commands at the top of every file
such that all constituent files can be compiled individually.

%%%%%%%%%%%%%%%%%%%%%%%%%%%%%%%%%%%%%%%%%%%%%%%%%%%%%%%%%%%%%%%%%%%%%%%%%%%%%%%%
%\subsection{Feature Suggestions}
%
%The following is a list of features which may be useful for future
%versions of this package:
%%
%\begin{itemize}
%\item
%\ldots
%\end{itemize}

%%%%%%%%%%%%%%%%%%%%%%%%%%%%%%%%%%%%%%%%%%%%%%%%%%%%%%%%%%%%%%%%%%%%%%%%%%%%%%%%
\subsection{Revision History}

%%%%%%%%%%%%%%%%%%%%%%%%%%%%%%%%%%%%%%%%
\paragraph{v2.0:} 2018/12/30

\begin{itemize}
\item
immediate forward processing
\item
added |\childdocby| mechanism
\item
manual restructured
\end{itemize}

%%%%%%%%%%%%%%%%%%%%%%%%%%%%%%%%%%%%%%%%
\paragraph{v1.6:} 2018/01/17

\begin{itemize}
\item
application for development of include files
\item
corrections to manual
\end{itemize}

%%%%%%%%%%%%%%%%%%%%%%%%%%%%%%%%%%%%%%%%
\paragraph{v1.5:} 2017/05/21

\begin{itemize}
\item
more complete structuring introduced
\item
|\childdocof| introduced
\item
|\childdoc| renamed to |\childdocmain|
\item
|\childredirect| renamed to |\childdocforward| and |\childdocforwardprefix|
and functionality expanded
\end{itemize}

%%%%%%%%%%%%%%%%%%%%%%%%%%%%%%%%%%%%%%%%
\paragraph{v1.0:} 2017/04/27

\begin{itemize}
\item
manual and install package
\item
first version published on CTAN
\end{itemize}

%%%%%%%%%%%%%%%%%%%%%%%%%%%%%%%%%%%%%%%%
\paragraph{v0.6:} 2017/04/26

\begin{itemize}
\item
redirection mechanism added
\end{itemize}

%%%%%%%%%%%%%%%%%%%%%%%%%%%%%%%%%%%%%%%%
\paragraph{v0.5:} 2017/04/26

\begin{itemize}
\item
functionality in definition file
\end{itemize}


%%%%%%%%%%%%%%%%%%%%%%%%%%%%%%%%%%%%%%%%%%%%%%%%%%%%%%%%%%%%%%%%%%%%%%%%%%%%%%%%
%%%%%%%%%%%%%%%%%%%%%%%%%%%%%%%%%%%%%%%%%%%%%%%%%%%%%%%%%%%%%%%%%%%%%%%%%%%%%%%%
%%%%%%%%%%%%%%%%%%%%%%%%%%%%%%%%%%%%%%%%%%%%%%%%%%%%%%%%%%%%%%%%%%%%%%%%%%%%%%%%
\appendix

\settowidth\MacroIndent{\rmfamily\scriptsize 000\ }

 \DocInput{childdoc.dtx}

\end{document}
%</driver>
% \fi
%
% %%%%%%%%%%%%%%%%%%%%%%%%%%%%%%%%%%%%%%%%%%%%%%%%%%%%%%%%%%%%%%%%%%%%%%%%%%%%%%
% %%%%%%%%%%%%%%%%%%%%%%%%%%%%%%%%%%%%%%%%%%%%%%%%%%%%%%%%%%%%%%%%%%%%%%%%%%%%%%
% \section{Sample}
%\iffalse
%<*samplemain>
%\fi
%
% The following presents a sample document
% with two chapters, two parts, a title page,
% a compile flag as well as three forwarding files to set the flag.
% It consists of eight |.tex| files:
% \begin{center}
% \begin{tabular}{ll}
% |cdocsamp.tex|&main file\\
% |cdocsch1.tex|&include file for chapter 1\\
% |cdocsch2.tex|&include file for chapter 2\\
% |cdocspt3.tex|&include file for part 3\\
% |cdocspt4.tex|&include file for part 4\\
% |cdocsdrf.tex|&forwarding file for main file in draft mode\\
% |cdocsfi1.tex|&forwarding file for final version of chapter 1\\
% |cdocsfi2.tex|&forwarding file for final version of chapter 2\\
% \end{tabular}
% \end{center}
% Each of the eight files can be compiled directly by the \LaTeX{} compiler.
%
% %%%%%%%%%%%%%%%%%%%%%%%%%%%%%%%%%%%%%%
% \paragraph{Main File.}
%
% The main file is called |cdocsamp.tex|.
%
% Load the \textsf{childdoc} definitions and
% declare the filename for the main document:
%    \begin{macrocode}
\input{childdoc.def}
\childdocmain{}
%    \end{macrocode}

% Optional override for |\version| flag:
%    \begin{macrocode}
%%\ifchilddoc\else\providecommand{\version}{draft}\fi
%    \end{macrocode}

% Define the default values for the |\version| flag
% (|final| for the main file and |draft| for childs):
%    \begin{macrocode}
\ifchilddoc
\providecommand{\version}{draft}
\else
\providecommand{\version}{final}
\fi
%    \end{macrocode}

% Load the standard document class:
%    \begin{macrocode}
\documentclass[12pt]{article}
%    \end{macrocode}

% Start the document body:
%    \begin{macrocode}
\begin{document}
%    \end{macrocode}

% Declare a title page.
% Print title, part of document being processed and version flag:
%    \begin{macrocode}
\addtocounter{page}{-1}
\begin{center}
{\LARGE\bfseries{}childdoc example\par}
\vspace{1cm}
\ifchilddoc
\ifchilddocmanual part\else chapter\fi:
`\childdocname' of `\childdocjob'\par
\else
main document: `\childdocjob'\par
\fi
version: \version\par
\end{center}
\newpage
%    \end{macrocode}

% Manually include selected file,
% otherwise process as usual:
%    \begin{macrocode}
\ifchilddocmanual
\section*{part `\childdocname'}
\input{\childdocname}
\else
%    \end{macrocode}

% Include the two chapters:
%    \begin{macrocode}
\include{cdocsch1}
\include{cdocsch2}
%    \end{macrocode}

% Include the two parts unless only chapters should be displayed:
%    \begin{macrocode}
\ifchilddoc\else
\section{part three}
\input{cdocspt3}
\section{part four}
\input{cdocspt4}
\fi
%    \end{macrocode}

% Process as usual until here:
%    \begin{macrocode}
\fi
%    \end{macrocode}

% End of document body:
%    \begin{macrocode}
\end{document}
%    \end{macrocode}
%\iffalse
%</samplemain>
%\fi
%
% %%%%%%%%%%%%%%%%%%%%%%%%%%%%%%%%%%%%%%
% \paragraph{Chapter Include Files.}
%
% The include files are called |cdocsch1.tex| and |cdocsch2.tex|.
%
%\iffalse
%<*samplechap1|samplechap2>
%\fi

% Optional override for |\version| flag:
%    \begin{macrocode}
%%\providecommand{\version}{final}
%    \end{macrocode}

% Include the main document:
%    \begin{macrocode}
\input{childdoc.def}
\childdocof{cdocsamp}
%    \end{macrocode}

%\iffalse
%</samplechap1|samplechap2>
%\fi
%
%\iffalse
%<*samplechap1>
%\fi
% Some text for chapter 1:
%    \begin{macrocode}
\section{one}
some text in chapter one
%    \end{macrocode}

%\iffalse
%</samplechap1>
%\fi
% Some text for chapter 2:
%\iffalse
%<*samplechap2>
%\fi
%    \begin{macrocode}
\section{two}
more text in chapter two
%    \end{macrocode}

%\iffalse
%</samplechap2>
%\fi
%
% %%%%%%%%%%%%%%%%%%%%%%%%%%%%%%%%%%%%%%
% \paragraph{Part Include Files.}
%
% The include files are called |cdocspt3.tex| and |cdocspt4.tex|.
%
%\iffalse
%<*samplepart3|samplepart4>
%\fi

% Optional override for |\version| flag:
%    \begin{macrocode}
%%\providecommand{\version}{final}
%    \end{macrocode}

% Include the main document:
%    \begin{macrocode}
\input{childdoc.def}
\childdocby{cdocsamp}
%    \end{macrocode}

%\iffalse
%</samplepart3|samplepart4>
%\fi
%
%\iffalse
%<*samplepart3>
%\fi
% Some text for part 3:
%    \begin{macrocode}
some text in part three
%    \end{macrocode}

%\iffalse
%</samplepart3>
%\fi
% Some text for part 4:
%\iffalse
%<*samplepart4>
%\fi
%    \begin{macrocode}
more text in part four
%    \end{macrocode}

%\iffalse
%</samplepart4>
%\fi
%
% %%%%%%%%%%%%%%%%%%%%%%%%%%%%%%%%%%%%%%
% \paragraph{Forwarding for a Complete Draft.}
%
% The following forwarding file |cdocsdrf.tex|
% compiles the main document in draft mode:
%\iffalse
%<*sampledraft>
%\fi
%    \begin{macrocode}
\def\version{draft}
\input{childdoc.def}
\childdocforward{cdocsamp}
%    \end{macrocode}

%\iffalse
%</sampledraft>
%\fi
%
% %%%%%%%%%%%%%%%%%%%%%%%%%%%%%%%%%%%%%%
% \paragraph{Forwarding for Final Version of the Chapters.}
%
% The following forwarding files |cdocsfn1.tex| and |cdocsfn2.tex|
% (with identical content)
% compile the final versions of the child documents
% |cdocsch1.tex| and |cdocsch2.tex|, respectively:
%\iffalse
%<*samplefinal>
%\fi
%    \begin{macrocode}
\def\version{final}
\input{childdoc.def}
\childdocforwardprefix[cdocsamp]{cdocsfn}{cdocsch}
%    \end{macrocode}

%\iffalse
%</samplefinal>
%\fi
%
% %%%%%%%%%%%%%%%%%%%%%%%%%%%%%%%%%%%%%%
% \paragraph{Command Line Processing.}
%
% The following three command lines generate the output files
% |cdocscld|, |cdocscl1| and |cdocscl2|
% which should be identical to
% |cdocsdrf|, |cdocsch1| and |cdocsfn2|, respectively:
% \begin{center}
% \begin{tabular}{l}
% |latex -jobname cdocscld \|\\
% |  "\def\version{draft}\input{childdoc.def}\childdocforward{cdocsamp}"|\\
% |latex -jobname cdocscl1 \|\\
% |  "\input{childdoc.def}\childdocforward[cdocsamp]{cdocsch1}"|\\
% |latex -jobname cdocscl2 \|\\
% |  "\def\version{final}\input{childdoc.def}\childdocforward{cdocsch2}"|
% \end{tabular}
% \end{center}
% Note that the trailing backslash on each first line
% merely continues the input to the second line
% (for convenient cut ant paste).
% Furthermore, the command |latex| can be replaced by any
% of its alternative versions such as |pdflatex|.
%
% %%%%%%%%%%%%%%%%%%%%%%%%%%%%%%%%%%%%%%%%%%%%%%%%%%%%%%%%%%%%%%%%%%%%%%%%%%%%%%
% %%%%%%%%%%%%%%%%%%%%%%%%%%%%%%%%%%%%%%%%%%%%%%%%%%%%%%%%%%%%%%%%%%%%%%%%%%%%%%
% \section{Implementation}
%\iffalse
%<*package>
%\fi
%
% This section describes the definitions file |childdoc.def|.

% The definitions cannot be loaded using |\usepackage| or |\RequirePackage|
% which has a mechanism to prevent loading a style file more than once.
% When loading the definitions by means of |\input|
% multiple instances have to be prevented manually:
%\iffalse
%This code needs to be before the `\ProvidesFile' directive
%which is defined at the beginning of this file.
%Therefore it is also placed there and commented out here.
%</package>
%<*discard>
%\fi
%    \begin{macrocode}
\ifdefined\childdocmain\endinput\fi
%    \end{macrocode}
%\iffalse
%</discard>
%<*package>
%\fi
%
% \macro{\ifchilddoc}
% \macro{\ifchilddocmanual}
% The conditional |\ifchilddoc| tells whether a
% child (true) or main (false) document is being compiled.
% The conditional |\ifchilddocmanual| tells whether
% the |\includeonly| mechanism is used (false) or
% the selection of child files must be performed manually (true).
% The definitions initialise to false:
%    \begin{macrocode}
\newif\ifchilddoc
\newif\ifchilddocmanual
%    \end{macrocode}

% \macro{\childdocname}
% \macro{\childdocjob}
% The macro |\childdocname| stores the name of the main document
% to be compiled. The macro |\childdocjob| stores the name of
% the document on which the \LaTeX{} compiler was originally invoked.
% The content of |\jobname| cannot be compared
% to filenames specified in the source due to different catcodes.
% The following code rescans |\jobname|, stores the result
% in |\childdocname| and saves a copy in |\childdocjob|:
%    \begin{macrocode}
\edef\childdocname{\scantokens\expandafter{\jobname\noexpand}}
\let\childdocjob\childdocname
%    \end{macrocode}

% \macro{\childdocdisable}
% The macro |\childdocdisable| prevents the main file
% from being processed more than once.
% At this stage, the main document command |\childdocmain|
% is assumed to be called once again where it should do nothing.
% Any subsequent call to it should prevent
% a secondary processing of the main document
% It overwrites the forwarding commands
% |\childdocof| and |\childdocforward|
% with empty macros to prevent further inclusions of the main document:
%    \begin{macrocode}
\newcommand{\childdocdisable}
{
  \renewcommand{\childdocmain}[1]{\renewcommand{\childdocmain}[1]{\endinput}}
  \renewcommand{\childdocof}[1]{}
  \renewcommand{\childdocby}[2][]{}
  \renewcommand{\childdocforward}[2][]{}
  \renewcommand{\childdocdisable}{}
}
%    \end{macrocode}

% \macro{\childdocmain}
% The macro |\childdocmain| is to be called at the top of the main file
% with nothing or the main filename (without extension) as argument.
% First, it breaks loops.
% If the argument is not empty and does not match |\childdocname|
% (which is set by the first inclusion of |childdoc.def|),
% |\ifchilddoc| is set to true, |\includeonly| is applied to the child file
% and |\jobname| is set to the main file
% (for proper handling of |.aux| files):
%    \begin{macrocode}
\newcommand{\childdocmain}[1]
{
  \childdocdisable\childdocmain{}
  \if?#1?\else
    \begingroup
      \def\childdoctmp{#1}
      \ifx\childdoctmp\childdocname
        \def\childdoctmp{}
      \else
        \def\childdoctmp
        {
          \childdoctrue
          \includeonly{\childdocname}
          \def\childdocjob{#1}
          \def\jobname{#1}
        }
      \fi
      \expandafter
    \endgroup
    \childdoctmp
  \fi
}
%    \end{macrocode}

% \macro{\childdocof}
% The command |\childdocof| redirects
% compilation to the main file |#1|.
%    \begin{macrocode}
\newcommand{\childdocof}[1]
{
  \childdocdisable
  \childdoctrue
  \includeonly{\childdocname}
  \def\jobname{#1}
  \def\childdocjob{#1}
  \input{#1}
}
%    \end{macrocode}

% \macro{\childdocby}
% The command |\childdocby| ....
%    \begin{macrocode}
\newcommand{\childdocby}[2][]
{
  \childdocdisable
  \childdoctrue
  \childdocmanualtrue
  \if?#1?\else
    \def\jobname{#2}
  \fi
  \def\childdocjob{#2}
  \input{#2}
  \endinput
}
%    \end{macrocode}

% \macro{\childdocforward}
% The command |\childdocforward| redirects
% compilation to the main file or
% (if the optional argument is given) a child file.
% Parameters are set as if the main file
% or a child file starting with |\childdocof| was compiled.
% Then compilation is handed over to the main file:
%    \begin{macrocode}
\newcommand{\childdocforward}[2][]
{
  \begingroup
    \if?#1?
      \def\childdoctmp
      {
        \def\childdocname{#2}
        \def\childdocjob{#2}
        \def\jobname{#2}
        \input{#2}
        \endinput
      }
    \else
      \def\childdoctmp
      {
        \childdocdisable
        \def\childdocname{#2}
        \childdoctrue
        \includeonly{#2}
        \def\childdocjob{#1}
        \def\jobname{#1}
        \input{#1}
        \endinput
      }
    \fi
    \expandafter
  \endgroup
  \childdoctmp
}
%    \end{macrocode}

% \macro{\childdocforwardprefix}
% The command |\childdocforwardprefix| redirects
% compilation to the main or a child file by means of a pattern.
% The prefix |#1| in the current filename is replaced by |#2|
% and the suffix of the current filename is kept
% (it is assumed that the filename does not contain the substring `|~~~|'
% which is used as a delimiter).
% Compilation is handed over to the new file by |\childdocforward|:
%    \begin{macrocode}
\newcommand{\childdocforwardprefix}[3][]
{
  \begingroup
    \def\childdocextract #2##1~~~{\def\childdoctmp{\childdocforward[#1]{#3##1}}}
    \expandafter\childdocextract\childdocname~~~
    \expandafter
  \endgroup
  \childdoctmp
}
%    \end{macrocode}

% \macro{\childdoc}
% The deprecated macro |\childdoc| is a legacy version of |\childdocmain|:
%    \begin{macrocode}
\newcommand{\childdoc}{\childdocmain}
%    \end{macrocode}

% \macro{\childdocredirect}
% The deprecated macro |\childdocredirect| is a legacy version
% of |\childdocforward| and |\childdocforwardprefix|:
%    \begin{macrocode}
\newcommand{\childdocredirect}[2][]
{
  \begingroup
    \if?#1?
      \def\childdoctmp{\childdocforward{#2}}
    \else
      \def\childdoctmp{\childdocforwardprefix{#1}{#2}}
    \fi
    \expandafter
  \endgroup
  \childdoctmp
}
%    \end{macrocode}

%\iffalse
%</package>
%\fi
%
\endinput
|\\
|\childdocforwardprefix{final}{child}|
\end{tabular}
\end{center}
%

Note that when several versions of a main file and/or of each child file
are to be generated, it may be convenient to set up a |Makefile| or
shell script to automatise the process.

%%%%%%%%%%%%%%%%%%%%%%%%%%%%%%%%%%%%%%%%%%%%%%%%%%%%%%%%%%%%%%%%%%%%%%%%%%%%%%%%
\subsection{Command Line Processing}
\label{sec:commandline}

The effect of redirection files can also be achieved by invoking
the \LaTeX{} compiler with a more elaborate command line.
Most conveniently this should be done as part
of a shell script or a |Makefile|.

When using \textsf{childdoc} in the main file, the following
command lines effectively perform a redirection
(note that depending on the shell being used,
backslashes may have to be doubled: `|\|' $\to$ `|\\|'):
%
\begin{center}
|... -jobname "|\textit{target}|" |\\|"|[\textit{flags}]%
|% \iffalse
%
% childdoc.dtx Copyright (C) 2017-2018 Niklas Beisert
%
% This work may be distributed and/or modified under the
% conditions of the LaTeX Project Public License, either version 1.3
% of this license or (at your option) any later version.
% The latest version of this license is in
%   http://www.latex-project.org/lppl.txt
% and version 1.3 or later is part of all distributions of LaTeX
% version 2005/12/01 or later.
%
% This work has the LPPL maintenance status `maintained'.
%
% The Current Maintainer of this work is Niklas Beisert.
%
% This work consists of the files childdoc.dtx and childdoc.ins
% and the derived files childdoc.def and cdocsamp.tex with
% cdocsch1.tex, cdocsch2.tex, cdocsdrf.tex, cdocsfn1.tex, cdocsfn2.tex.
%
%<package>\ifdefined\childdocmain\endinput\fi
%<package>\ProvidesFile{childdoc.def}[2018/12/30 v2.0 child document driver]
%<samplemain>\ProvidesFile{cdocsamp.tex}[2018/12/30 v2.0 sample for childdoc]
%<*driver>
%\ProvidesFile{childdoc.drv}[2018/12/30 v2.0 childdoc reference manual file]
\PassOptionsToClass{10pt,a4paper}{article}
\documentclass{ltxdoc}

\usepackage[margin=35mm]{geometry}
\usepackage{hyperref}
\usepackage{hyperxmp}
\usepackage[usenames]{color}

\hypersetup{colorlinks=true}
\hypersetup{pdfstartview=FitH}
\hypersetup{pdfpagemode=UseNone}
\hypersetup{pdfsource={}}
\hypersetup{pdflang={en-UK}}
\hypersetup{pdfcopyright={Copyright 2017-2018 Niklas Beisert.
  This work may be distributed and/or modified under the
  conditions of the LaTeX Project Public License, either version 1.3
  of this license or (at your option) any later version.}}
\hypersetup{pdflicenseurl={http://www.latex-project.org/lppl.txt}}
\hypersetup{pdfcontactaddress={ETH Zurich, ITP, HIT K,
  Wolfgang-Pauli-Strasse 27}}
\hypersetup{pdfcontactpostcode={8093}}
\hypersetup{pdfcontactcity={Zurich}}
\hypersetup{pdfcontactcountry={Switzerland}}
\hypersetup{pdfcontactemail={nbeisert@itp.phys.ethz.ch}}
\hypersetup{pdfcontacturl={http://people.phys.ethz.ch/\xmptilde nbeisert/}}

\newcommand{\secref}[1]{\hyperref[#1]{section \ref*{#1}}}

\parskip1ex
\parindent0pt
\let\olditemize\itemize
\def\itemize{\olditemize\parskip0pt}

\begin{document}

\title{The \textsf{childdoc} Package}
\hypersetup{pdftitle={The childdoc Package}}
\author{Niklas Beisert\\[2ex]
  Institut f\"ur Theoretische Physik\\
  Eidgen\"ossische Technische Hochschule Z\"urich\\
  Wolfgang-Pauli-Strasse 27, 8093 Z\"urich, Switzerland\\[1ex]
  \href{mailto:nbeisert@itp.phys.ethz.ch}
  {\texttt{nbeisert@itp.phys.ethz.ch}}}
\hypersetup{pdfauthor={Niklas Beisert}}
\hypersetup{pdfsubject={Manual for the LaTeX2e Package childdoc}}
\date{30 December 2018, \textsf{v2.0}}
\maketitle

\begin{abstract}\noindent
\textsf{childdoc} is a \LaTeXe{} package
that enables the direct compilation
of document sections included by |\include|
to individual files.
\end{abstract}

\begingroup
\parskip0ex
\tableofcontents
\endgroup

%%%%%%%%%%%%%%%%%%%%%%%%%%%%%%%%%%%%%%%%%%%%%%%%%%%%%%%%%%%%%%%%%%%%%%%%%%%%%%%%
%%%%%%%%%%%%%%%%%%%%%%%%%%%%%%%%%%%%%%%%%%%%%%%%%%%%%%%%%%%%%%%%%%%%%%%%%%%%%%%%
\section{Introduction}

\LaTeX{} provides a mechanism to structure a large document (such as a book)
into a main file and several child files (containing the chapters)
using the |\include| command.
This mechanism is beneficial for documents
which span hundreds of pages in order to
make the source file(s) more manageable.
Moreover, compilation can be restricted to
selected child files by means of the |\includeonly| command.
The latter feature can be used to reduce the compilation time while editing
(this was significantly more useful in the earlier days of \LaTeX{})
or to generate a smaller document which is easier to navigate.
Another application of |\includeonly| is to generate
documents consisting of selected parts of the complete document.

However, there are a few drawbacks of the plain |\include| mechanism:
\begin{itemize}
\item
The child files cannot be compiled on their own,
they can only be compiled via the main file.
A naive editing environment
(such as a text editor with an option
to have the current file processed by \LaTeX)
may require one to switch to the main file before compiling;
attempting to compile the child file produces errors.
\item
The main file must be modified (each time)
to adjust the |\includeonly| command
to the present needs. This easily leaves the main file in a messy state.
\item
The generated document will always carry the filename
of the main document. This is inconvenient if
several child files are to be compiled and
to be kept for distribution.
\end{itemize}

The present package provides a simple interface
to make child files individually compilable by \LaTeX{}.
Compiling a child file then has the same effect as compiling
the main file with an |\includeonly| command
to select the appropriate child.
Moreover the generated document will carry the name of the child
rather than the main file.
This resolves all three above issues.

This feature is meant to make the editing of books,
thesis documents and lecture notes somewhat more convenient.
However, the package can also be used efficiently for
composing a series of documents (such as exercise sheets)
which are typically distributed individually.
It then assists the author in generating the individual documents
(potentially in different versions)
as well as a document containing the collected series.
Another application is in developing style files
or other kinds of included material
where compilation of the style file could redirect
to a sample or test file.

%%%%%%%%%%%%%%%%%%%%%%%%%%%%%%%%%%%%%%%%%%%%%%%%%%%%%%%%%%%%%%%%%%%%%%%%%%%%%%%%
%%%%%%%%%%%%%%%%%%%%%%%%%%%%%%%%%%%%%%%%%%%%%%%%%%%%%%%%%%%%%%%%%%%%%%%%%%%%%%%%
\section{Usage}

First of all, the package \textsf{childdoc} is \emph{not} a standard
\LaTeXe{} |.sty| style file! Therefore it needs to be invoked in
a non-standard way.

%%%%%%%%%%%%%%%%%%%%%%%%%%%%%%%%%%%%%%%%%%%%%%%%%%%%%%%%%%%%%%%%%%%%%%%%%%%%%%%%
\subsection{Included Files}
\label{sec:include}

%%%%%%%%%%%%%%%%%%%%%%%%%%%%%%%%%%%%%%%%
\DescribeMacro{\childdocmain}
To use the package, add the commands
\begin{center}
\begin{tabular}{l}
|\input{childdoc.def}|\\
|\childdocmain{}|\\
\end{tabular}
\end{center}
at the very top of the main \LaTeX{} file,
in particular \emph{before} the |\documentclass| statement!
The argument of |\childdocmain| should be left empty
(but it must be present).

%%%%%%%%%%%%%%%%%%%%%%%%%%%%%%%%%%%%%%%%
\DescribeMacro{\childdocof}
Furthermore, add the commands
\begin{center}
\begin{tabular}{l}
|\input{childdoc.def}|\\
|\childdocof{|\textit{main}|}|\\
\end{tabular}
\end{center}
at the top of every child file \textit{child}
which is included by |\include{|\textit{child}|}|
from within the main file
(or at least for those files to be compiled individually).
The argument \textit{main} must be the filename of the main file.

There are a couple of
considerations in setting up the main and child documents:

%%%%%%%%%%%%%%%%%%%%%%%%%%%%%%%%%%%%%%%%
\paragraph{Restrictions.}

Please note the following restrictions:
\begin{itemize}
\item
|\childdocmain| must be called with one argument \textit{main}
to ensure compatibility with earlier version of the package.
It must either be empty (|\childdocmain{}|)
or precisely match the filename of the main file in which it is specified.
See \secref{sec:detection} for further information.
\item
The filename \textit{main} must be specified without the |.tex| extension.
\item
The filename \textit{main} is case sensitive
(even in case-insensitive file systems)
due to internal string comparison.
\item
The argument \textit{main} should be fully expanded, it cannot be a macro.
\item
Subdirectories and special characters should be avoided in filenames.
\item
The command |\childdocmain{|\textit{main}|}| must be followed by a whitespace.
It should not be followed immediately by another command
or by a comment mark `|%|'.
This is because the \TeX{} parser reads the token immediately following
the argument of |\childdocmain| and puts it
at the beginning of every child section;
however, a white\-space is ignored.
\end{itemize}

%%%%%%%%%%%%%%%%%%%%%%%%%%%%%%%%%%%%%%%%
\paragraph{Content of Main File.}

It is advisable to place all content in the child files included by |\include|.
Any output contained in the main file will appear in all child documents
unless suppressed manually;
it cannot be suppressed automatically by the |\includeonly| directive
and thus should normally be avoided.
A method to include some content in the main file
by means of conditional processing is described in \secref{sec:conditional}.

%%%%%%%%%%%%%%%%%%%%%%%%%%%%%%%%%%%%%%%%
\paragraph{Page Numbering.}

When only a part of the document is compiled,
the appropriate numbering of pages
(as well as other status parameters)
is determined from the |.aux| files.
The latter contain information from previous passes.
However this information needs to propagate through
all intermediate child documents.
Therefore the page numbering in child documents may well
be inconsistent until the complete document is compiled at least once.

A useful (if unconventional) way to always ensure a consistent
page numbering is to restart the numbering in each child document
and denote the pages by `\textit{child}|.|\textit{page}'
where \textit{child} represents the chapter/section number of the child file.
This can be achieved by the command
|\numberwithin{page}{|\textit{child}|}|
of the \textsf{amsmath} package
where \textit{child} can be |chapter| or |section|
depending on the chosen structuring.
Alternatively, one can modify the macro |\thepage| appropriately
and reset the counter |page| at the start of each child file.

%%%%%%%%%%%%%%%%%%%%%%%%%%%%%%%%%%%%%%%%%%%%%%%%%%%%%%%%%%%%%%%%%%%%%%%%%%%%%%%%
\subsection{Conditional Processing}
\label{sec:conditional}

The package provides a mechanism to compile different versions
of a document. To customise the versions further some conditional processing
can come in handy to distinguish which version is being compiled.
The package provides two macros to describe the compilation context:

%%%%%%%%%%%%%%%%%%%%%%%%%%%%%%%%%%%%%%%%
\DescribeMacro{\ifchilddoc}
The conditional |\ifchilddoc| distinguishes between the compilation of
child documents and the main document:
%
\begin{center}
|\ifchilddoc |\textit{child-code}| |[|\||else |\textit{main-code}]| \||fi|
\end{center}

%%%%%%%%%%%%%%%%%%%%%%%%%%%%%%%%%%%%%%%%
\DescribeMacro{\childdocname}
\DescribeMacro{\childdocjob}
The macro |\childdocname| contains the filename (without extension)
of the main or child file being processed.
Note that |\childdocjob| will always contain the name of the main file.

%%%%%%%%%%%%%%%%%%%%%%%%%%%%%%%%%%%%%%%%
\paragraph{Title Page.}

Conditional processing can be used to include a title or banner page
in the main document when proper precautions are taken.
Importantly, the code in the main file should ensure that the page counter
(as well as other status parameters which are stored in the |.aux| files)
takes the same value after the conditional processing.
Otherwise the page numbers may take divergent values
depending on which part is compiled.

For example, a title page could be declared by:
%
\begin{center}
\begin{tabular}{l}
|\ifchilddoc\||else|\\
|\addtocounter{page}{-1}|\\
\textit{code for title page}\\
|\newpage|\\
|\||fi|
\end{tabular}
\end{center}
%
A banner page for the child documents can be generated by:
%
\begin{center}
\begin{tabular}{l}
|\ifchilddoc|\\
|\addtocounter{page}{-1}|\\
\textit{code for banner page}\\
|\newpage|\\
|\||fi|
\end{tabular}
\end{center}
%
Here one could write a message such as:
\begin{center}
|This is the part \childdocname{} of \childdocjob{}.|
\end{center}

%%%%%%%%%%%%%%%%%%%%%%%%%%%%%%%%%%%%%%%%%%%%%%%%%%%%%%%%%%%%%%%%%%%%%%%%%%%%%%%%
\subsection{Flags}
\label{sec:flags}

The package makes it easy to generate different versions
of the main or child documents.
To this end compilation flags can be defined
and assigned different default values.
They will be particularly useful in conjunction
with the forwarding mechanism described in \secref{sec:forward}.

For example, it may be useful to have a flag |\version|
which can be set to |draft| or |final|.
The document source will contain some conditional code
depending on the value of |\version|.
Suppose further, the flag should default to |final| for the main file
and to |draft| for child files
which is a natural assignment for editing the document.
This is achieved by placing the following code
in the preamble of the main document
(below the |\childdocmain| directive):
%
\begin{center}
\begin{tabular}{l}
|\ifchilddoc|\\
|\providecommand{\version}{draft}|\\
|\||else|\\
|\providecommand{\version}{final}|\\
|\||fi|
\end{tabular}
\end{center}
%
The definition by |\providecommand| makes sure
that previous definitions are not overwritten.
Further statements |\providecommand{\version}{...}|
can thus be added before the above code to override it.

For the main file, one might add a line
(between |\childdocmain| and the above block)
%
\begin{center}
|%\ifchilddoc\||else\providecommand{\version}{draft}\||fi|
\end{center}
%
which can be uncommented to produce a draft version.
Likewise one can add a line to the very top of a child file
(above the |\childdocof{|\textit{main}|}| directive)
%
\begin{center}
|%\providecommand{\version}{final}|
\end{center}
%
which can be uncommented to produce the final version of this child document.

%%%%%%%%%%%%%%%%%%%%%%%%%%%%%%%%%%%%%%%%%%%%%%%%%%%%%%%%%%%%%%%%%%%%%%%%%%%%%%%%
\subsection{Forwarding}
\label{sec:forward}

Different versions of the main or child documents
using compilation flags as described in \secref{sec:flags}
can be (permanently) stored in different files
for convenient compilation, viewing and distribution.
To this end, the package defines a command
to pass on compilation to a different file:

%%%%%%%%%%%%%%%%%%%%%%%%%%%%%%%%%%%%%%%%
\DescribeMacro{\childdocforward}
The command |\childdocforward| redirects processing to
another source file:
%
\begin{center}
\begin{tabular}{l}
|\input{childdoc.def}|\\
|\childdocforward[|\textit{main}|]{|\textit{dest}|}|\\
\end{tabular}
\end{center}
%
The argument \textit{dest} is the destination file
(without extension).
It should be the main file or one of the child files.
Note that further \textsf{childdoc} directives
such as |\childdocof| and |\childdocforward|
in the indicated file will be processed in this form.
The optional argument \textit{main}
passes on directly to the main file \textit{main}
while pretending to compile the child \textit{dest}.
This form behaves as if \textit{dest}
issues |\childdocof{|\textit{main}|}| right away,
and no further \textsf{childdoc} directives will be processed.

%%%%%%%%%%%%%%%%%%%%%%%%%%%%%%%%%%%%%%%%
\DescribeMacro{\...prefix}
In the alternative form |\childdocforwardprefix|,
%
\begin{center}
\begin{tabular}{l}
|\input{childdoc.def}|\\
|\childdocforwardprefix[|\textit{main}|]{|\textit{prefix}|}{|\textit{dest}|}|
\end{tabular}
\end{center}
%
the destination file is determined by a pattern
depending on the current file:
To make this work, the current file must be called
`{\textit{prefix}\hspace{0.2em}\textit{suffix}}'
with \textit{prefix} matching precisely the argument.
Processing is then passed on to the file
`{\textit{dest}\hspace{0.2em}\textit{suffix}}'.
Surely, the same effect is achieved by
directly specifying the
argument `{\textit{dest}\hspace{0.2em}\textit{suffix}}'
in the first form.
However, that requires to set up a different file
for each child. With the alternative form of the command
all these files can have exactly the same content
which simplifies setting them up and maintaining them.

For example, the following file |draft.tex|
with a compilation flag |\version| as described in \secref{sec:flags}
compiles the main document as a draft:
%
\begin{center}
\begin{tabular}{l}
|\def\version{draft}|\\
|\input{childdoc.def}|\\
|\childdocforward{|\textit{main}|}|
\end{tabular}
\end{center}
%
Likewise, the following files |final|\textit{nn}|.tex|
compile the final version of the child document
|child|\textit{nn}|.tex|:
%
\begin{center}
\begin{tabular}{l}
|\def\version{final}|\\
|\input{childdoc.def}|\\
|\childdocforwardprefix{final}{child}|
\end{tabular}
\end{center}
%

Note that when several versions of a main file and/or of each child file
are to be generated, it may be convenient to set up a |Makefile| or
shell script to automatise the process.

%%%%%%%%%%%%%%%%%%%%%%%%%%%%%%%%%%%%%%%%%%%%%%%%%%%%%%%%%%%%%%%%%%%%%%%%%%%%%%%%
\subsection{Command Line Processing}
\label{sec:commandline}

The effect of redirection files can also be achieved by invoking
the \LaTeX{} compiler with a more elaborate command line.
Most conveniently this should be done as part
of a shell script or a |Makefile|.

When using \textsf{childdoc} in the main file, the following
command lines effectively perform a redirection
(note that depending on the shell being used,
backslashes may have to be doubled: `|\|' $\to$ `|\\|'):
%
\begin{center}
|... -jobname "|\textit{target}|" |\\|"|[\textit{flags}]%
|\input{childdoc.def}\childdocforward[|\textit{main}|]{|\textit{dest}|}"|
\end{center}
%
Here \textit{target} is the name of the output file,
\textit{main} is the name of the main file
and \textit{dest} is the name of the main or child file to be processed
(all filenames without extensions).
The optional argument \textit{main} can be omitted
if \textit{main} matches \textit{dest}.
Optionally, compilation \textit{flags} can be defined via |\def| commands.
This command line makes the \TeX{} engine believe
it is compiling the file \textit{target}
whose content is specified as the latter parameter.
The provided code then forwards the processing to
\textit{main} or \textit{dest} as described in \secref{sec:forward}.

%%%%%%%%%%%%%%%%%%%%%%%%%%%%%%%%%%%%%%%%%%%%%%%%%%%%%%%%%%%%%%%%%%%%%%%%%%%%%%%%
\subsection{Include by Input}
\label{sec:input}

Including child documents by |\include| has some restrictions by design.
Most notably, the content of a child document always occupies
its own set of pages; pages cannot be shared between child documents.
Usually, this behaviour makes perfect sense
because each child document contain an essential part of the document.
However, in some situations it may be desirable to compose
a document from a collection of parts
without having mandatory page breaks between then.
For this case, the package
provides a mechanism to include parts
by |\input| which can also be processed individually.
However, by construction this mechanism
requires manual handling of the content to be output.

%%%%%%%%%%%%%%%%%%%%%%%%%%%%%%%%%%%%%%%%
\DescribeMacro{\ifchilddocmanual}
The main file should be prepared as usual, see \secref{sec:include}.
However, the document body must make a distinction
between processing of an individual part and of the main document, e.g.:
%
\begin{center}
\begin{tabular}{l}
|\ifchilddocmanual|\\
|\input{\childdocname}|\\
|\||else|\\
\textit{document body with }|\input{|\textit{part}|}|\\
|\||fi|
\end{tabular}
\end{center}
%
The conditional |\ifchilddocmanual| is true whenever
a part to be included by |\input| is being compiled,
and the name of the part is stored in |\childdocname|.

%%%%%%%%%%%%%%%%%%%%%%%%%%%%%%%%%%%%%%%%
\DescribeMacro{\childdocby}
Each part to be included by |\input| should start with:
%
\begin{center}
\begin{tabular}{l}
|\input{childdoc.def}|\\
|\childdocby{|\textit{main}|}|\\
\end{tabular}
\end{center}
%
The directive |\childdocby| is similar to |\childdocof|
described in \secref{sec:include},
but the subsequent selection of content must be done manually.
To that end, both |\ifchilddoc| and |\ifchilddocmanual|
will be true upon processing of a part,
and the name of the part is stored in |\childdocname|.
Note that |\jobname| will be set to the filename of the current part
so that each part receives an individual |.aux| file
that does not interfere with the |.aux| file(s) of the main document.
This behaviour can be altered by the alternative form
|\childdocby[*]{|\textit{main}|}| (with a non-empty optional argument)
which uses the |.aux| file of the main document
by setting |\jobname| to \textit{main}.

%%%%%%%%%%%%%%%%%%%%%%%%%%%%%%%%%%%%%%%%%%%%%%%%%%%%%%%%%%%%%%%%%%%%%%%%%%%%%%%%
\subsection{Driver Development}
\label{sec:driver}

The \textsf{childdoc} mechanism can also be use for the development
of definition files such as \LaTeX{} styles or classes.
This case differs from the above setup with multiple parts
included by |\include| in that no |\includeonly| should be invoked.
This can be achieved by starting the include file
(before |\ProvidesPackage|) with:
%
\begin{center}
\begin{tabular}{l}
|\input{childdoc.def}|\\
|\childdocforward{|\textit{main}|}|\\
\end{tabular}
\end{center}
%
or alternatively with:
%
\begin{center}
\begin{tabular}{l}
|\input{childdoc.def}|\\
|\childdocby{|\textit{main}|}|\\
\end{tabular}
\end{center}
%
Both forms have slightly different effects as described above.
The main file is prepared as usual, see \secref{sec:include}.

%%%%%%%%%%%%%%%%%%%%%%%%%%%%%%%%%%%%%%%%%%%%%%%%%%%%%%%%%%%%%%%%%%%%%%%%%%%%%%%%
\subsection{Legacy Detection}
\label{sec:detection}

The directive |\childdocmain| in the main file can detect
whether the complete document or merely a child is to be compiled
even without using the directive |\childdocof|.
This method is deprecated because it is less robust
and there is no compelling reason to use it;
it is merely provided for backward compatibility
and it may be removed in future versions.

If the detection mechanism is to be used,
it is mandatory to correctly specify
the filename of the main file as the argument of |\childdocmain|:
%
\begin{center}
\begin{tabular}{l}
|\input{childdoc.def}|\\
|\childdocmain{|\textit{main}|}|\\
\end{tabular}
\end{center}
%
If |\jobname| does not match the argument \textit{main} of |\childdocmain|,
it is assumed that |\jobname| points to the child file to be compiled.
When using |\childdocmain| with the main file specified as argument,
it suffices to start a child file
with just |\input{|\textit{main}|}|
without loading of the package and using |\childdocof|.
If instead all processing is done
with the appropriate \textsf{childdoc} directives,
the argument of \textit{main} of |\childdocmain| can be empty.

An alternative version of the command line processing described
in \secref{sec:commandline} using the detection mechanism reads:
%
\begin{center}
|... -jobname "|\textit{target}|" "|[\textit{flags}]%
[|\def\jobname{|\textit{dest}|}|]|\input{|\textit{main}|}"|
\end{center}

%%%%%%%%%%%%%%%%%%%%%%%%%%%%%%%%%%%%%%%%%%%%%%%%%%%%%%%%%%%%%%%%%%%%%%%%%%%%%%%%
\subsection{Manual Code}
\label{sec:manual}

In case one cannot be certain whether the definitions file |childdoc.def|
is installed on the target \TeX{} distribution
and one prefers not to ship it,
it is conceivable to paste a few relevant commands into the sources.

To that end, drop all statements |\input{childdoc.def}|
and perform the replacements as outlined below.
Instead of |\childdocmain{|\textit{main}|}| add the following code
to the top of the main file:
%
\begin{center}
\begin{tabular}{l}
|\||ifdefined\childdocname\endinput\||fi\newif\ifchilddoc|\\
|\edef\childdocname{\scantokens\expandafter{\jobname\noexpand}}|\\
|\def\childdocmain{|\textit{main}|}\||ifx\childdocmain\childdocname\||else|\\
|\childdoctrue\includeonly{\childdocname}\let\jobname\childdocmain\||fi|\\
\end{tabular}
\end{center}
%
Instead of |\childdocof{|\textit{main}|}| just include the main file
at the top of each child file:
%
\begin{center}
|\input{|\textit{main}|}|
\end{center}
%
A simple redirection |\childdocforward{|\textit{dest}|}| is achieved by:
%
\begin{center}
|\def\jobname{|\textit{dest}|}\input{\jobname}|
\end{center}
%
The redirection with prefix
|\childdocforwardprefix[|\textit{prefix}|]{|\textit{dest}|}|
is accomplished by:
%
\begin{center}
\begin{tabular}{l}
|{\edef\jobname{\scantokens\expandafter{\jobname\noexpand}}|\\
|\def\redirectjob |\textit{prefix}|#1~~~{\gdef\jobname{|\textit{dest}|#1}}|\\
|\expandafter\redirectjob\jobname~~~}\input{\jobname}|
\end{tabular}
\end{center}

In an alternative approach,
child documents can be compiled by a specific command line
without additional code or specific definitions:
%
\begin{center}
|... -jobname "|\textit{target}|" "|[\textit{flags}]%
|\includeonly{|\textit{dest}|}\input{|\textit{main}|}"|
\end{center}
%

%%%%%%%%%%%%%%%%%%%%%%%%%%%%%%%%%%%%%%%%%%%%%%%%%%%%%%%%%%%%%%%%%%%%%%%%%%%%%%%%
%%%%%%%%%%%%%%%%%%%%%%%%%%%%%%%%%%%%%%%%%%%%%%%%%%%%%%%%%%%%%%%%%%%%%%%%%%%%%%%%
\section{Information}

%%%%%%%%%%%%%%%%%%%%%%%%%%%%%%%%%%%%%%%%%%%%%%%%%%%%%%%%%%%%%%%%%%%%%%%%%%%%%%%%
\subsection{Copyright}

Copyright \copyright{} 2017--2018 Niklas Beisert

This work may be distributed and/or modified under the
conditions of the \LaTeX{} Project Public License, either version 1.3
of this license or (at your option) any later version.
The latest version of this license is in
  \url{http://www.latex-project.org/lppl.txt}
and version 1.3 or later is part of all distributions of \LaTeX{}
version 2005/12/01 or later.

This work has the LPPL maintenance status `maintained'.

The Current Maintainer of this work is Niklas Beisert.

This work consists of the files |README.txt|, |childdoc.ins| and |childdoc.dtx|
as well as the derived files |childdoc.def|, |cdocsamp.tex|
with |cdocsch1.tex|, |cdocsch2.tex|, |cdocspt3.tex|, |cdocspt4.tex|,
|cdocsdrf.tex|, |cdocsfn1.tex|, |cdocsfn2.tex|
as well as |childdoc.pdf|.

%%%%%%%%%%%%%%%%%%%%%%%%%%%%%%%%%%%%%%%%%%%%%%%%%%%%%%%%%%%%%%%%%%%%%%%%%%%%%%%%
\subsection{Files and Installation}

The package consists of the files:
%
\begin{center}
\begin{tabular}{ll}
    |README.txt|   & readme file \\
    |childdoc.ins| & installation file \\
    |childdoc.dtx| & source file \\
    |childdoc.def| & definition file \\
    |cdocsamp.tex| & sample main file \\
    |cdocsch1.tex| & sample include file \\
    |cdocsch2.tex| & sample include file \\
    |cdocspt3.tex| & sample part file \\
    |cdocspt4.tex| & sample part file \\
    |cdocsdrf.tex| & sample redirection file \\
    |cdocsfn1.tex| & sample redirection file \\
    |cdocsfn2.tex| & sample redirection file \\
    |childdoc.pdf| & manual
\end{tabular}
\end{center}
%
The distribution consists of the files
|README.txt|, |childdoc.ins| and |childdoc.dtx|.
%
\begin{itemize}
\item
Run (pdf)\LaTeX{} on |childdoc.dtx|
to compile the manual |childdoc.pdf| (this file).
\item
Run \LaTeX{} on |childdoc.ins| to create the definitions file |childdoc.def|
and the sample |cdocsamp.tex| with include files
|cdocsch1.tex|, |cdocsch2.tex|, |cdocspt3.tex|, |cdocspt4.tex|,
|cdocsdrf.tex|, |cdocsfn1.tex|, |cdocsfn2.tex|.
Then copy the file |childdoc.def| to an appropriate directory of your \LaTeX{}
distribution, e.g.\ \textit{texmf-root}|/tex/latex/childdoc|.
\end{itemize}

%%%%%%%%%%%%%%%%%%%%%%%%%%%%%%%%%%%%%%%%%%%%%%%%%%%%%%%%%%%%%%%%%%%%%%%%%%%%%%%%
\subsection{Related CTAN Packages}

There are several other packages which offer a similar functionality:
%
\begin{itemize}
\item
The packages
\href{http://ctan.org/pkg/docmute}{\textsf{docmute}},
\href{http://ctan.org/pkg/includex}{\textsf{includex}} and
\href{http://ctan.org/pkg/standalone}{\textsf{standalone}}
provide commands to include only the document body of
a child file thus allowing both files to be compiled individually.
\item
The packages \href{http://ctan.org/pkg/subdocs}{\textsf{subdocs}}
and \href{http://ctan.org/pkg/subfiles}{\textsf{subfiles}}
provide structures in which the main and child documents can be
encapsulated and allowing them to be compiled individually.
The inclusion mechanism is different from the conventional |\include|.
\item
The package \href{http://ctan.org/pkg/combine}{\textsf{combine}}
is an elaborate solution to combine several documents into one.
\end{itemize}
%
See also the CTAN topic \href{http://ctan.org/topic/subdocs}{\textsf{subdocs}}
for further related packages.
The present package differs from the above solutions in that
a document structure constructed with the conventional |\include| mechanism
just needs two extra commands at the top of every file
such that all constituent files can be compiled individually.

%%%%%%%%%%%%%%%%%%%%%%%%%%%%%%%%%%%%%%%%%%%%%%%%%%%%%%%%%%%%%%%%%%%%%%%%%%%%%%%%
%\subsection{Feature Suggestions}
%
%The following is a list of features which may be useful for future
%versions of this package:
%%
%\begin{itemize}
%\item
%\ldots
%\end{itemize}

%%%%%%%%%%%%%%%%%%%%%%%%%%%%%%%%%%%%%%%%%%%%%%%%%%%%%%%%%%%%%%%%%%%%%%%%%%%%%%%%
\subsection{Revision History}

%%%%%%%%%%%%%%%%%%%%%%%%%%%%%%%%%%%%%%%%
\paragraph{v2.0:} 2018/12/30

\begin{itemize}
\item
immediate forward processing
\item
added |\childdocby| mechanism
\item
manual restructured
\end{itemize}

%%%%%%%%%%%%%%%%%%%%%%%%%%%%%%%%%%%%%%%%
\paragraph{v1.6:} 2018/01/17

\begin{itemize}
\item
application for development of include files
\item
corrections to manual
\end{itemize}

%%%%%%%%%%%%%%%%%%%%%%%%%%%%%%%%%%%%%%%%
\paragraph{v1.5:} 2017/05/21

\begin{itemize}
\item
more complete structuring introduced
\item
|\childdocof| introduced
\item
|\childdoc| renamed to |\childdocmain|
\item
|\childredirect| renamed to |\childdocforward| and |\childdocforwardprefix|
and functionality expanded
\end{itemize}

%%%%%%%%%%%%%%%%%%%%%%%%%%%%%%%%%%%%%%%%
\paragraph{v1.0:} 2017/04/27

\begin{itemize}
\item
manual and install package
\item
first version published on CTAN
\end{itemize}

%%%%%%%%%%%%%%%%%%%%%%%%%%%%%%%%%%%%%%%%
\paragraph{v0.6:} 2017/04/26

\begin{itemize}
\item
redirection mechanism added
\end{itemize}

%%%%%%%%%%%%%%%%%%%%%%%%%%%%%%%%%%%%%%%%
\paragraph{v0.5:} 2017/04/26

\begin{itemize}
\item
functionality in definition file
\end{itemize}


%%%%%%%%%%%%%%%%%%%%%%%%%%%%%%%%%%%%%%%%%%%%%%%%%%%%%%%%%%%%%%%%%%%%%%%%%%%%%%%%
%%%%%%%%%%%%%%%%%%%%%%%%%%%%%%%%%%%%%%%%%%%%%%%%%%%%%%%%%%%%%%%%%%%%%%%%%%%%%%%%
%%%%%%%%%%%%%%%%%%%%%%%%%%%%%%%%%%%%%%%%%%%%%%%%%%%%%%%%%%%%%%%%%%%%%%%%%%%%%%%%
\appendix

\settowidth\MacroIndent{\rmfamily\scriptsize 000\ }

 \DocInput{childdoc.dtx}

\end{document}
%</driver>
% \fi
%
% %%%%%%%%%%%%%%%%%%%%%%%%%%%%%%%%%%%%%%%%%%%%%%%%%%%%%%%%%%%%%%%%%%%%%%%%%%%%%%
% %%%%%%%%%%%%%%%%%%%%%%%%%%%%%%%%%%%%%%%%%%%%%%%%%%%%%%%%%%%%%%%%%%%%%%%%%%%%%%
% \section{Sample}
%\iffalse
%<*samplemain>
%\fi
%
% The following presents a sample document
% with two chapters, two parts, a title page,
% a compile flag as well as three forwarding files to set the flag.
% It consists of eight |.tex| files:
% \begin{center}
% \begin{tabular}{ll}
% |cdocsamp.tex|&main file\\
% |cdocsch1.tex|&include file for chapter 1\\
% |cdocsch2.tex|&include file for chapter 2\\
% |cdocspt3.tex|&include file for part 3\\
% |cdocspt4.tex|&include file for part 4\\
% |cdocsdrf.tex|&forwarding file for main file in draft mode\\
% |cdocsfi1.tex|&forwarding file for final version of chapter 1\\
% |cdocsfi2.tex|&forwarding file for final version of chapter 2\\
% \end{tabular}
% \end{center}
% Each of the eight files can be compiled directly by the \LaTeX{} compiler.
%
% %%%%%%%%%%%%%%%%%%%%%%%%%%%%%%%%%%%%%%
% \paragraph{Main File.}
%
% The main file is called |cdocsamp.tex|.
%
% Load the \textsf{childdoc} definitions and
% declare the filename for the main document:
%    \begin{macrocode}
\input{childdoc.def}
\childdocmain{}
%    \end{macrocode}

% Optional override for |\version| flag:
%    \begin{macrocode}
%%\ifchilddoc\else\providecommand{\version}{draft}\fi
%    \end{macrocode}

% Define the default values for the |\version| flag
% (|final| for the main file and |draft| for childs):
%    \begin{macrocode}
\ifchilddoc
\providecommand{\version}{draft}
\else
\providecommand{\version}{final}
\fi
%    \end{macrocode}

% Load the standard document class:
%    \begin{macrocode}
\documentclass[12pt]{article}
%    \end{macrocode}

% Start the document body:
%    \begin{macrocode}
\begin{document}
%    \end{macrocode}

% Declare a title page.
% Print title, part of document being processed and version flag:
%    \begin{macrocode}
\addtocounter{page}{-1}
\begin{center}
{\LARGE\bfseries{}childdoc example\par}
\vspace{1cm}
\ifchilddoc
\ifchilddocmanual part\else chapter\fi:
`\childdocname' of `\childdocjob'\par
\else
main document: `\childdocjob'\par
\fi
version: \version\par
\end{center}
\newpage
%    \end{macrocode}

% Manually include selected file,
% otherwise process as usual:
%    \begin{macrocode}
\ifchilddocmanual
\section*{part `\childdocname'}
\input{\childdocname}
\else
%    \end{macrocode}

% Include the two chapters:
%    \begin{macrocode}
\include{cdocsch1}
\include{cdocsch2}
%    \end{macrocode}

% Include the two parts unless only chapters should be displayed:
%    \begin{macrocode}
\ifchilddoc\else
\section{part three}
\input{cdocspt3}
\section{part four}
\input{cdocspt4}
\fi
%    \end{macrocode}

% Process as usual until here:
%    \begin{macrocode}
\fi
%    \end{macrocode}

% End of document body:
%    \begin{macrocode}
\end{document}
%    \end{macrocode}
%\iffalse
%</samplemain>
%\fi
%
% %%%%%%%%%%%%%%%%%%%%%%%%%%%%%%%%%%%%%%
% \paragraph{Chapter Include Files.}
%
% The include files are called |cdocsch1.tex| and |cdocsch2.tex|.
%
%\iffalse
%<*samplechap1|samplechap2>
%\fi

% Optional override for |\version| flag:
%    \begin{macrocode}
%%\providecommand{\version}{final}
%    \end{macrocode}

% Include the main document:
%    \begin{macrocode}
\input{childdoc.def}
\childdocof{cdocsamp}
%    \end{macrocode}

%\iffalse
%</samplechap1|samplechap2>
%\fi
%
%\iffalse
%<*samplechap1>
%\fi
% Some text for chapter 1:
%    \begin{macrocode}
\section{one}
some text in chapter one
%    \end{macrocode}

%\iffalse
%</samplechap1>
%\fi
% Some text for chapter 2:
%\iffalse
%<*samplechap2>
%\fi
%    \begin{macrocode}
\section{two}
more text in chapter two
%    \end{macrocode}

%\iffalse
%</samplechap2>
%\fi
%
% %%%%%%%%%%%%%%%%%%%%%%%%%%%%%%%%%%%%%%
% \paragraph{Part Include Files.}
%
% The include files are called |cdocspt3.tex| and |cdocspt4.tex|.
%
%\iffalse
%<*samplepart3|samplepart4>
%\fi

% Optional override for |\version| flag:
%    \begin{macrocode}
%%\providecommand{\version}{final}
%    \end{macrocode}

% Include the main document:
%    \begin{macrocode}
\input{childdoc.def}
\childdocby{cdocsamp}
%    \end{macrocode}

%\iffalse
%</samplepart3|samplepart4>
%\fi
%
%\iffalse
%<*samplepart3>
%\fi
% Some text for part 3:
%    \begin{macrocode}
some text in part three
%    \end{macrocode}

%\iffalse
%</samplepart3>
%\fi
% Some text for part 4:
%\iffalse
%<*samplepart4>
%\fi
%    \begin{macrocode}
more text in part four
%    \end{macrocode}

%\iffalse
%</samplepart4>
%\fi
%
% %%%%%%%%%%%%%%%%%%%%%%%%%%%%%%%%%%%%%%
% \paragraph{Forwarding for a Complete Draft.}
%
% The following forwarding file |cdocsdrf.tex|
% compiles the main document in draft mode:
%\iffalse
%<*sampledraft>
%\fi
%    \begin{macrocode}
\def\version{draft}
\input{childdoc.def}
\childdocforward{cdocsamp}
%    \end{macrocode}

%\iffalse
%</sampledraft>
%\fi
%
% %%%%%%%%%%%%%%%%%%%%%%%%%%%%%%%%%%%%%%
% \paragraph{Forwarding for Final Version of the Chapters.}
%
% The following forwarding files |cdocsfn1.tex| and |cdocsfn2.tex|
% (with identical content)
% compile the final versions of the child documents
% |cdocsch1.tex| and |cdocsch2.tex|, respectively:
%\iffalse
%<*samplefinal>
%\fi
%    \begin{macrocode}
\def\version{final}
\input{childdoc.def}
\childdocforwardprefix[cdocsamp]{cdocsfn}{cdocsch}
%    \end{macrocode}

%\iffalse
%</samplefinal>
%\fi
%
% %%%%%%%%%%%%%%%%%%%%%%%%%%%%%%%%%%%%%%
% \paragraph{Command Line Processing.}
%
% The following three command lines generate the output files
% |cdocscld|, |cdocscl1| and |cdocscl2|
% which should be identical to
% |cdocsdrf|, |cdocsch1| and |cdocsfn2|, respectively:
% \begin{center}
% \begin{tabular}{l}
% |latex -jobname cdocscld \|\\
% |  "\def\version{draft}\input{childdoc.def}\childdocforward{cdocsamp}"|\\
% |latex -jobname cdocscl1 \|\\
% |  "\input{childdoc.def}\childdocforward[cdocsamp]{cdocsch1}"|\\
% |latex -jobname cdocscl2 \|\\
% |  "\def\version{final}\input{childdoc.def}\childdocforward{cdocsch2}"|
% \end{tabular}
% \end{center}
% Note that the trailing backslash on each first line
% merely continues the input to the second line
% (for convenient cut ant paste).
% Furthermore, the command |latex| can be replaced by any
% of its alternative versions such as |pdflatex|.
%
% %%%%%%%%%%%%%%%%%%%%%%%%%%%%%%%%%%%%%%%%%%%%%%%%%%%%%%%%%%%%%%%%%%%%%%%%%%%%%%
% %%%%%%%%%%%%%%%%%%%%%%%%%%%%%%%%%%%%%%%%%%%%%%%%%%%%%%%%%%%%%%%%%%%%%%%%%%%%%%
% \section{Implementation}
%\iffalse
%<*package>
%\fi
%
% This section describes the definitions file |childdoc.def|.

% The definitions cannot be loaded using |\usepackage| or |\RequirePackage|
% which has a mechanism to prevent loading a style file more than once.
% When loading the definitions by means of |\input|
% multiple instances have to be prevented manually:
%\iffalse
%This code needs to be before the `\ProvidesFile' directive
%which is defined at the beginning of this file.
%Therefore it is also placed there and commented out here.
%</package>
%<*discard>
%\fi
%    \begin{macrocode}
\ifdefined\childdocmain\endinput\fi
%    \end{macrocode}
%\iffalse
%</discard>
%<*package>
%\fi
%
% \macro{\ifchilddoc}
% \macro{\ifchilddocmanual}
% The conditional |\ifchilddoc| tells whether a
% child (true) or main (false) document is being compiled.
% The conditional |\ifchilddocmanual| tells whether
% the |\includeonly| mechanism is used (false) or
% the selection of child files must be performed manually (true).
% The definitions initialise to false:
%    \begin{macrocode}
\newif\ifchilddoc
\newif\ifchilddocmanual
%    \end{macrocode}

% \macro{\childdocname}
% \macro{\childdocjob}
% The macro |\childdocname| stores the name of the main document
% to be compiled. The macro |\childdocjob| stores the name of
% the document on which the \LaTeX{} compiler was originally invoked.
% The content of |\jobname| cannot be compared
% to filenames specified in the source due to different catcodes.
% The following code rescans |\jobname|, stores the result
% in |\childdocname| and saves a copy in |\childdocjob|:
%    \begin{macrocode}
\edef\childdocname{\scantokens\expandafter{\jobname\noexpand}}
\let\childdocjob\childdocname
%    \end{macrocode}

% \macro{\childdocdisable}
% The macro |\childdocdisable| prevents the main file
% from being processed more than once.
% At this stage, the main document command |\childdocmain|
% is assumed to be called once again where it should do nothing.
% Any subsequent call to it should prevent
% a secondary processing of the main document
% It overwrites the forwarding commands
% |\childdocof| and |\childdocforward|
% with empty macros to prevent further inclusions of the main document:
%    \begin{macrocode}
\newcommand{\childdocdisable}
{
  \renewcommand{\childdocmain}[1]{\renewcommand{\childdocmain}[1]{\endinput}}
  \renewcommand{\childdocof}[1]{}
  \renewcommand{\childdocby}[2][]{}
  \renewcommand{\childdocforward}[2][]{}
  \renewcommand{\childdocdisable}{}
}
%    \end{macrocode}

% \macro{\childdocmain}
% The macro |\childdocmain| is to be called at the top of the main file
% with nothing or the main filename (without extension) as argument.
% First, it breaks loops.
% If the argument is not empty and does not match |\childdocname|
% (which is set by the first inclusion of |childdoc.def|),
% |\ifchilddoc| is set to true, |\includeonly| is applied to the child file
% and |\jobname| is set to the main file
% (for proper handling of |.aux| files):
%    \begin{macrocode}
\newcommand{\childdocmain}[1]
{
  \childdocdisable\childdocmain{}
  \if?#1?\else
    \begingroup
      \def\childdoctmp{#1}
      \ifx\childdoctmp\childdocname
        \def\childdoctmp{}
      \else
        \def\childdoctmp
        {
          \childdoctrue
          \includeonly{\childdocname}
          \def\childdocjob{#1}
          \def\jobname{#1}
        }
      \fi
      \expandafter
    \endgroup
    \childdoctmp
  \fi
}
%    \end{macrocode}

% \macro{\childdocof}
% The command |\childdocof| redirects
% compilation to the main file |#1|.
%    \begin{macrocode}
\newcommand{\childdocof}[1]
{
  \childdocdisable
  \childdoctrue
  \includeonly{\childdocname}
  \def\jobname{#1}
  \def\childdocjob{#1}
  \input{#1}
}
%    \end{macrocode}

% \macro{\childdocby}
% The command |\childdocby| ....
%    \begin{macrocode}
\newcommand{\childdocby}[2][]
{
  \childdocdisable
  \childdoctrue
  \childdocmanualtrue
  \if?#1?\else
    \def\jobname{#2}
  \fi
  \def\childdocjob{#2}
  \input{#2}
  \endinput
}
%    \end{macrocode}

% \macro{\childdocforward}
% The command |\childdocforward| redirects
% compilation to the main file or
% (if the optional argument is given) a child file.
% Parameters are set as if the main file
% or a child file starting with |\childdocof| was compiled.
% Then compilation is handed over to the main file:
%    \begin{macrocode}
\newcommand{\childdocforward}[2][]
{
  \begingroup
    \if?#1?
      \def\childdoctmp
      {
        \def\childdocname{#2}
        \def\childdocjob{#2}
        \def\jobname{#2}
        \input{#2}
        \endinput
      }
    \else
      \def\childdoctmp
      {
        \childdocdisable
        \def\childdocname{#2}
        \childdoctrue
        \includeonly{#2}
        \def\childdocjob{#1}
        \def\jobname{#1}
        \input{#1}
        \endinput
      }
    \fi
    \expandafter
  \endgroup
  \childdoctmp
}
%    \end{macrocode}

% \macro{\childdocforwardprefix}
% The command |\childdocforwardprefix| redirects
% compilation to the main or a child file by means of a pattern.
% The prefix |#1| in the current filename is replaced by |#2|
% and the suffix of the current filename is kept
% (it is assumed that the filename does not contain the substring `|~~~|'
% which is used as a delimiter).
% Compilation is handed over to the new file by |\childdocforward|:
%    \begin{macrocode}
\newcommand{\childdocforwardprefix}[3][]
{
  \begingroup
    \def\childdocextract #2##1~~~{\def\childdoctmp{\childdocforward[#1]{#3##1}}}
    \expandafter\childdocextract\childdocname~~~
    \expandafter
  \endgroup
  \childdoctmp
}
%    \end{macrocode}

% \macro{\childdoc}
% The deprecated macro |\childdoc| is a legacy version of |\childdocmain|:
%    \begin{macrocode}
\newcommand{\childdoc}{\childdocmain}
%    \end{macrocode}

% \macro{\childdocredirect}
% The deprecated macro |\childdocredirect| is a legacy version
% of |\childdocforward| and |\childdocforwardprefix|:
%    \begin{macrocode}
\newcommand{\childdocredirect}[2][]
{
  \begingroup
    \if?#1?
      \def\childdoctmp{\childdocforward{#2}}
    \else
      \def\childdoctmp{\childdocforwardprefix{#1}{#2}}
    \fi
    \expandafter
  \endgroup
  \childdoctmp
}
%    \end{macrocode}

%\iffalse
%</package>
%\fi
%
\endinput
\childdocforward[|\textit{main}|]{|\textit{dest}|}"|
\end{center}
%
Here \textit{target} is the name of the output file,
\textit{main} is the name of the main file
and \textit{dest} is the name of the main or child file to be processed
(all filenames without extensions).
The optional argument \textit{main} can be omitted
if \textit{main} matches \textit{dest}.
Optionally, compilation \textit{flags} can be defined via |\def| commands.
This command line makes the \TeX{} engine believe
it is compiling the file \textit{target}
whose content is specified as the latter parameter.
The provided code then forwards the processing to
\textit{main} or \textit{dest} as described in \secref{sec:forward}.

%%%%%%%%%%%%%%%%%%%%%%%%%%%%%%%%%%%%%%%%%%%%%%%%%%%%%%%%%%%%%%%%%%%%%%%%%%%%%%%%
\subsection{Include by Input}
\label{sec:input}

Including child documents by |\include| has some restrictions by design.
Most notably, the content of a child document always occupies
its own set of pages; pages cannot be shared between child documents.
Usually, this behaviour makes perfect sense
because each child document contain an essential part of the document.
However, in some situations it may be desirable to compose
a document from a collection of parts
without having mandatory page breaks between then.
For this case, the package
provides a mechanism to include parts
by |\input| which can also be processed individually.
However, by construction this mechanism
requires manual handling of the content to be output.

%%%%%%%%%%%%%%%%%%%%%%%%%%%%%%%%%%%%%%%%
\DescribeMacro{\ifchilddocmanual}
The main file should be prepared as usual, see \secref{sec:include}.
However, the document body must make a distinction
between processing of an individual part and of the main document, e.g.:
%
\begin{center}
\begin{tabular}{l}
|\ifchilddocmanual|\\
|\input{\childdocname}|\\
|\||else|\\
\textit{document body with }|\input{|\textit{part}|}|\\
|\||fi|
\end{tabular}
\end{center}
%
The conditional |\ifchilddocmanual| is true whenever
a part to be included by |\input| is being compiled,
and the name of the part is stored in |\childdocname|.

%%%%%%%%%%%%%%%%%%%%%%%%%%%%%%%%%%%%%%%%
\DescribeMacro{\childdocby}
Each part to be included by |\input| should start with:
%
\begin{center}
\begin{tabular}{l}
|% \iffalse
%
% childdoc.dtx Copyright (C) 2017-2018 Niklas Beisert
%
% This work may be distributed and/or modified under the
% conditions of the LaTeX Project Public License, either version 1.3
% of this license or (at your option) any later version.
% The latest version of this license is in
%   http://www.latex-project.org/lppl.txt
% and version 1.3 or later is part of all distributions of LaTeX
% version 2005/12/01 or later.
%
% This work has the LPPL maintenance status `maintained'.
%
% The Current Maintainer of this work is Niklas Beisert.
%
% This work consists of the files childdoc.dtx and childdoc.ins
% and the derived files childdoc.def and cdocsamp.tex with
% cdocsch1.tex, cdocsch2.tex, cdocsdrf.tex, cdocsfn1.tex, cdocsfn2.tex.
%
%<package>\ifdefined\childdocmain\endinput\fi
%<package>\ProvidesFile{childdoc.def}[2018/12/30 v2.0 child document driver]
%<samplemain>\ProvidesFile{cdocsamp.tex}[2018/12/30 v2.0 sample for childdoc]
%<*driver>
%\ProvidesFile{childdoc.drv}[2018/12/30 v2.0 childdoc reference manual file]
\PassOptionsToClass{10pt,a4paper}{article}
\documentclass{ltxdoc}

\usepackage[margin=35mm]{geometry}
\usepackage{hyperref}
\usepackage{hyperxmp}
\usepackage[usenames]{color}

\hypersetup{colorlinks=true}
\hypersetup{pdfstartview=FitH}
\hypersetup{pdfpagemode=UseNone}
\hypersetup{pdfsource={}}
\hypersetup{pdflang={en-UK}}
\hypersetup{pdfcopyright={Copyright 2017-2018 Niklas Beisert.
  This work may be distributed and/or modified under the
  conditions of the LaTeX Project Public License, either version 1.3
  of this license or (at your option) any later version.}}
\hypersetup{pdflicenseurl={http://www.latex-project.org/lppl.txt}}
\hypersetup{pdfcontactaddress={ETH Zurich, ITP, HIT K,
  Wolfgang-Pauli-Strasse 27}}
\hypersetup{pdfcontactpostcode={8093}}
\hypersetup{pdfcontactcity={Zurich}}
\hypersetup{pdfcontactcountry={Switzerland}}
\hypersetup{pdfcontactemail={nbeisert@itp.phys.ethz.ch}}
\hypersetup{pdfcontacturl={http://people.phys.ethz.ch/\xmptilde nbeisert/}}

\newcommand{\secref}[1]{\hyperref[#1]{section \ref*{#1}}}

\parskip1ex
\parindent0pt
\let\olditemize\itemize
\def\itemize{\olditemize\parskip0pt}

\begin{document}

\title{The \textsf{childdoc} Package}
\hypersetup{pdftitle={The childdoc Package}}
\author{Niklas Beisert\\[2ex]
  Institut f\"ur Theoretische Physik\\
  Eidgen\"ossische Technische Hochschule Z\"urich\\
  Wolfgang-Pauli-Strasse 27, 8093 Z\"urich, Switzerland\\[1ex]
  \href{mailto:nbeisert@itp.phys.ethz.ch}
  {\texttt{nbeisert@itp.phys.ethz.ch}}}
\hypersetup{pdfauthor={Niklas Beisert}}
\hypersetup{pdfsubject={Manual for the LaTeX2e Package childdoc}}
\date{30 December 2018, \textsf{v2.0}}
\maketitle

\begin{abstract}\noindent
\textsf{childdoc} is a \LaTeXe{} package
that enables the direct compilation
of document sections included by |\include|
to individual files.
\end{abstract}

\begingroup
\parskip0ex
\tableofcontents
\endgroup

%%%%%%%%%%%%%%%%%%%%%%%%%%%%%%%%%%%%%%%%%%%%%%%%%%%%%%%%%%%%%%%%%%%%%%%%%%%%%%%%
%%%%%%%%%%%%%%%%%%%%%%%%%%%%%%%%%%%%%%%%%%%%%%%%%%%%%%%%%%%%%%%%%%%%%%%%%%%%%%%%
\section{Introduction}

\LaTeX{} provides a mechanism to structure a large document (such as a book)
into a main file and several child files (containing the chapters)
using the |\include| command.
This mechanism is beneficial for documents
which span hundreds of pages in order to
make the source file(s) more manageable.
Moreover, compilation can be restricted to
selected child files by means of the |\includeonly| command.
The latter feature can be used to reduce the compilation time while editing
(this was significantly more useful in the earlier days of \LaTeX{})
or to generate a smaller document which is easier to navigate.
Another application of |\includeonly| is to generate
documents consisting of selected parts of the complete document.

However, there are a few drawbacks of the plain |\include| mechanism:
\begin{itemize}
\item
The child files cannot be compiled on their own,
they can only be compiled via the main file.
A naive editing environment
(such as a text editor with an option
to have the current file processed by \LaTeX)
may require one to switch to the main file before compiling;
attempting to compile the child file produces errors.
\item
The main file must be modified (each time)
to adjust the |\includeonly| command
to the present needs. This easily leaves the main file in a messy state.
\item
The generated document will always carry the filename
of the main document. This is inconvenient if
several child files are to be compiled and
to be kept for distribution.
\end{itemize}

The present package provides a simple interface
to make child files individually compilable by \LaTeX{}.
Compiling a child file then has the same effect as compiling
the main file with an |\includeonly| command
to select the appropriate child.
Moreover the generated document will carry the name of the child
rather than the main file.
This resolves all three above issues.

This feature is meant to make the editing of books,
thesis documents and lecture notes somewhat more convenient.
However, the package can also be used efficiently for
composing a series of documents (such as exercise sheets)
which are typically distributed individually.
It then assists the author in generating the individual documents
(potentially in different versions)
as well as a document containing the collected series.
Another application is in developing style files
or other kinds of included material
where compilation of the style file could redirect
to a sample or test file.

%%%%%%%%%%%%%%%%%%%%%%%%%%%%%%%%%%%%%%%%%%%%%%%%%%%%%%%%%%%%%%%%%%%%%%%%%%%%%%%%
%%%%%%%%%%%%%%%%%%%%%%%%%%%%%%%%%%%%%%%%%%%%%%%%%%%%%%%%%%%%%%%%%%%%%%%%%%%%%%%%
\section{Usage}

First of all, the package \textsf{childdoc} is \emph{not} a standard
\LaTeXe{} |.sty| style file! Therefore it needs to be invoked in
a non-standard way.

%%%%%%%%%%%%%%%%%%%%%%%%%%%%%%%%%%%%%%%%%%%%%%%%%%%%%%%%%%%%%%%%%%%%%%%%%%%%%%%%
\subsection{Included Files}
\label{sec:include}

%%%%%%%%%%%%%%%%%%%%%%%%%%%%%%%%%%%%%%%%
\DescribeMacro{\childdocmain}
To use the package, add the commands
\begin{center}
\begin{tabular}{l}
|\input{childdoc.def}|\\
|\childdocmain{}|\\
\end{tabular}
\end{center}
at the very top of the main \LaTeX{} file,
in particular \emph{before} the |\documentclass| statement!
The argument of |\childdocmain| should be left empty
(but it must be present).

%%%%%%%%%%%%%%%%%%%%%%%%%%%%%%%%%%%%%%%%
\DescribeMacro{\childdocof}
Furthermore, add the commands
\begin{center}
\begin{tabular}{l}
|\input{childdoc.def}|\\
|\childdocof{|\textit{main}|}|\\
\end{tabular}
\end{center}
at the top of every child file \textit{child}
which is included by |\include{|\textit{child}|}|
from within the main file
(or at least for those files to be compiled individually).
The argument \textit{main} must be the filename of the main file.

There are a couple of
considerations in setting up the main and child documents:

%%%%%%%%%%%%%%%%%%%%%%%%%%%%%%%%%%%%%%%%
\paragraph{Restrictions.}

Please note the following restrictions:
\begin{itemize}
\item
|\childdocmain| must be called with one argument \textit{main}
to ensure compatibility with earlier version of the package.
It must either be empty (|\childdocmain{}|)
or precisely match the filename of the main file in which it is specified.
See \secref{sec:detection} for further information.
\item
The filename \textit{main} must be specified without the |.tex| extension.
\item
The filename \textit{main} is case sensitive
(even in case-insensitive file systems)
due to internal string comparison.
\item
The argument \textit{main} should be fully expanded, it cannot be a macro.
\item
Subdirectories and special characters should be avoided in filenames.
\item
The command |\childdocmain{|\textit{main}|}| must be followed by a whitespace.
It should not be followed immediately by another command
or by a comment mark `|%|'.
This is because the \TeX{} parser reads the token immediately following
the argument of |\childdocmain| and puts it
at the beginning of every child section;
however, a white\-space is ignored.
\end{itemize}

%%%%%%%%%%%%%%%%%%%%%%%%%%%%%%%%%%%%%%%%
\paragraph{Content of Main File.}

It is advisable to place all content in the child files included by |\include|.
Any output contained in the main file will appear in all child documents
unless suppressed manually;
it cannot be suppressed automatically by the |\includeonly| directive
and thus should normally be avoided.
A method to include some content in the main file
by means of conditional processing is described in \secref{sec:conditional}.

%%%%%%%%%%%%%%%%%%%%%%%%%%%%%%%%%%%%%%%%
\paragraph{Page Numbering.}

When only a part of the document is compiled,
the appropriate numbering of pages
(as well as other status parameters)
is determined from the |.aux| files.
The latter contain information from previous passes.
However this information needs to propagate through
all intermediate child documents.
Therefore the page numbering in child documents may well
be inconsistent until the complete document is compiled at least once.

A useful (if unconventional) way to always ensure a consistent
page numbering is to restart the numbering in each child document
and denote the pages by `\textit{child}|.|\textit{page}'
where \textit{child} represents the chapter/section number of the child file.
This can be achieved by the command
|\numberwithin{page}{|\textit{child}|}|
of the \textsf{amsmath} package
where \textit{child} can be |chapter| or |section|
depending on the chosen structuring.
Alternatively, one can modify the macro |\thepage| appropriately
and reset the counter |page| at the start of each child file.

%%%%%%%%%%%%%%%%%%%%%%%%%%%%%%%%%%%%%%%%%%%%%%%%%%%%%%%%%%%%%%%%%%%%%%%%%%%%%%%%
\subsection{Conditional Processing}
\label{sec:conditional}

The package provides a mechanism to compile different versions
of a document. To customise the versions further some conditional processing
can come in handy to distinguish which version is being compiled.
The package provides two macros to describe the compilation context:

%%%%%%%%%%%%%%%%%%%%%%%%%%%%%%%%%%%%%%%%
\DescribeMacro{\ifchilddoc}
The conditional |\ifchilddoc| distinguishes between the compilation of
child documents and the main document:
%
\begin{center}
|\ifchilddoc |\textit{child-code}| |[|\||else |\textit{main-code}]| \||fi|
\end{center}

%%%%%%%%%%%%%%%%%%%%%%%%%%%%%%%%%%%%%%%%
\DescribeMacro{\childdocname}
\DescribeMacro{\childdocjob}
The macro |\childdocname| contains the filename (without extension)
of the main or child file being processed.
Note that |\childdocjob| will always contain the name of the main file.

%%%%%%%%%%%%%%%%%%%%%%%%%%%%%%%%%%%%%%%%
\paragraph{Title Page.}

Conditional processing can be used to include a title or banner page
in the main document when proper precautions are taken.
Importantly, the code in the main file should ensure that the page counter
(as well as other status parameters which are stored in the |.aux| files)
takes the same value after the conditional processing.
Otherwise the page numbers may take divergent values
depending on which part is compiled.

For example, a title page could be declared by:
%
\begin{center}
\begin{tabular}{l}
|\ifchilddoc\||else|\\
|\addtocounter{page}{-1}|\\
\textit{code for title page}\\
|\newpage|\\
|\||fi|
\end{tabular}
\end{center}
%
A banner page for the child documents can be generated by:
%
\begin{center}
\begin{tabular}{l}
|\ifchilddoc|\\
|\addtocounter{page}{-1}|\\
\textit{code for banner page}\\
|\newpage|\\
|\||fi|
\end{tabular}
\end{center}
%
Here one could write a message such as:
\begin{center}
|This is the part \childdocname{} of \childdocjob{}.|
\end{center}

%%%%%%%%%%%%%%%%%%%%%%%%%%%%%%%%%%%%%%%%%%%%%%%%%%%%%%%%%%%%%%%%%%%%%%%%%%%%%%%%
\subsection{Flags}
\label{sec:flags}

The package makes it easy to generate different versions
of the main or child documents.
To this end compilation flags can be defined
and assigned different default values.
They will be particularly useful in conjunction
with the forwarding mechanism described in \secref{sec:forward}.

For example, it may be useful to have a flag |\version|
which can be set to |draft| or |final|.
The document source will contain some conditional code
depending on the value of |\version|.
Suppose further, the flag should default to |final| for the main file
and to |draft| for child files
which is a natural assignment for editing the document.
This is achieved by placing the following code
in the preamble of the main document
(below the |\childdocmain| directive):
%
\begin{center}
\begin{tabular}{l}
|\ifchilddoc|\\
|\providecommand{\version}{draft}|\\
|\||else|\\
|\providecommand{\version}{final}|\\
|\||fi|
\end{tabular}
\end{center}
%
The definition by |\providecommand| makes sure
that previous definitions are not overwritten.
Further statements |\providecommand{\version}{...}|
can thus be added before the above code to override it.

For the main file, one might add a line
(between |\childdocmain| and the above block)
%
\begin{center}
|%\ifchilddoc\||else\providecommand{\version}{draft}\||fi|
\end{center}
%
which can be uncommented to produce a draft version.
Likewise one can add a line to the very top of a child file
(above the |\childdocof{|\textit{main}|}| directive)
%
\begin{center}
|%\providecommand{\version}{final}|
\end{center}
%
which can be uncommented to produce the final version of this child document.

%%%%%%%%%%%%%%%%%%%%%%%%%%%%%%%%%%%%%%%%%%%%%%%%%%%%%%%%%%%%%%%%%%%%%%%%%%%%%%%%
\subsection{Forwarding}
\label{sec:forward}

Different versions of the main or child documents
using compilation flags as described in \secref{sec:flags}
can be (permanently) stored in different files
for convenient compilation, viewing and distribution.
To this end, the package defines a command
to pass on compilation to a different file:

%%%%%%%%%%%%%%%%%%%%%%%%%%%%%%%%%%%%%%%%
\DescribeMacro{\childdocforward}
The command |\childdocforward| redirects processing to
another source file:
%
\begin{center}
\begin{tabular}{l}
|\input{childdoc.def}|\\
|\childdocforward[|\textit{main}|]{|\textit{dest}|}|\\
\end{tabular}
\end{center}
%
The argument \textit{dest} is the destination file
(without extension).
It should be the main file or one of the child files.
Note that further \textsf{childdoc} directives
such as |\childdocof| and |\childdocforward|
in the indicated file will be processed in this form.
The optional argument \textit{main}
passes on directly to the main file \textit{main}
while pretending to compile the child \textit{dest}.
This form behaves as if \textit{dest}
issues |\childdocof{|\textit{main}|}| right away,
and no further \textsf{childdoc} directives will be processed.

%%%%%%%%%%%%%%%%%%%%%%%%%%%%%%%%%%%%%%%%
\DescribeMacro{\...prefix}
In the alternative form |\childdocforwardprefix|,
%
\begin{center}
\begin{tabular}{l}
|\input{childdoc.def}|\\
|\childdocforwardprefix[|\textit{main}|]{|\textit{prefix}|}{|\textit{dest}|}|
\end{tabular}
\end{center}
%
the destination file is determined by a pattern
depending on the current file:
To make this work, the current file must be called
`{\textit{prefix}\hspace{0.2em}\textit{suffix}}'
with \textit{prefix} matching precisely the argument.
Processing is then passed on to the file
`{\textit{dest}\hspace{0.2em}\textit{suffix}}'.
Surely, the same effect is achieved by
directly specifying the
argument `{\textit{dest}\hspace{0.2em}\textit{suffix}}'
in the first form.
However, that requires to set up a different file
for each child. With the alternative form of the command
all these files can have exactly the same content
which simplifies setting them up and maintaining them.

For example, the following file |draft.tex|
with a compilation flag |\version| as described in \secref{sec:flags}
compiles the main document as a draft:
%
\begin{center}
\begin{tabular}{l}
|\def\version{draft}|\\
|\input{childdoc.def}|\\
|\childdocforward{|\textit{main}|}|
\end{tabular}
\end{center}
%
Likewise, the following files |final|\textit{nn}|.tex|
compile the final version of the child document
|child|\textit{nn}|.tex|:
%
\begin{center}
\begin{tabular}{l}
|\def\version{final}|\\
|\input{childdoc.def}|\\
|\childdocforwardprefix{final}{child}|
\end{tabular}
\end{center}
%

Note that when several versions of a main file and/or of each child file
are to be generated, it may be convenient to set up a |Makefile| or
shell script to automatise the process.

%%%%%%%%%%%%%%%%%%%%%%%%%%%%%%%%%%%%%%%%%%%%%%%%%%%%%%%%%%%%%%%%%%%%%%%%%%%%%%%%
\subsection{Command Line Processing}
\label{sec:commandline}

The effect of redirection files can also be achieved by invoking
the \LaTeX{} compiler with a more elaborate command line.
Most conveniently this should be done as part
of a shell script or a |Makefile|.

When using \textsf{childdoc} in the main file, the following
command lines effectively perform a redirection
(note that depending on the shell being used,
backslashes may have to be doubled: `|\|' $\to$ `|\\|'):
%
\begin{center}
|... -jobname "|\textit{target}|" |\\|"|[\textit{flags}]%
|\input{childdoc.def}\childdocforward[|\textit{main}|]{|\textit{dest}|}"|
\end{center}
%
Here \textit{target} is the name of the output file,
\textit{main} is the name of the main file
and \textit{dest} is the name of the main or child file to be processed
(all filenames without extensions).
The optional argument \textit{main} can be omitted
if \textit{main} matches \textit{dest}.
Optionally, compilation \textit{flags} can be defined via |\def| commands.
This command line makes the \TeX{} engine believe
it is compiling the file \textit{target}
whose content is specified as the latter parameter.
The provided code then forwards the processing to
\textit{main} or \textit{dest} as described in \secref{sec:forward}.

%%%%%%%%%%%%%%%%%%%%%%%%%%%%%%%%%%%%%%%%%%%%%%%%%%%%%%%%%%%%%%%%%%%%%%%%%%%%%%%%
\subsection{Include by Input}
\label{sec:input}

Including child documents by |\include| has some restrictions by design.
Most notably, the content of a child document always occupies
its own set of pages; pages cannot be shared between child documents.
Usually, this behaviour makes perfect sense
because each child document contain an essential part of the document.
However, in some situations it may be desirable to compose
a document from a collection of parts
without having mandatory page breaks between then.
For this case, the package
provides a mechanism to include parts
by |\input| which can also be processed individually.
However, by construction this mechanism
requires manual handling of the content to be output.

%%%%%%%%%%%%%%%%%%%%%%%%%%%%%%%%%%%%%%%%
\DescribeMacro{\ifchilddocmanual}
The main file should be prepared as usual, see \secref{sec:include}.
However, the document body must make a distinction
between processing of an individual part and of the main document, e.g.:
%
\begin{center}
\begin{tabular}{l}
|\ifchilddocmanual|\\
|\input{\childdocname}|\\
|\||else|\\
\textit{document body with }|\input{|\textit{part}|}|\\
|\||fi|
\end{tabular}
\end{center}
%
The conditional |\ifchilddocmanual| is true whenever
a part to be included by |\input| is being compiled,
and the name of the part is stored in |\childdocname|.

%%%%%%%%%%%%%%%%%%%%%%%%%%%%%%%%%%%%%%%%
\DescribeMacro{\childdocby}
Each part to be included by |\input| should start with:
%
\begin{center}
\begin{tabular}{l}
|\input{childdoc.def}|\\
|\childdocby{|\textit{main}|}|\\
\end{tabular}
\end{center}
%
The directive |\childdocby| is similar to |\childdocof|
described in \secref{sec:include},
but the subsequent selection of content must be done manually.
To that end, both |\ifchilddoc| and |\ifchilddocmanual|
will be true upon processing of a part,
and the name of the part is stored in |\childdocname|.
Note that |\jobname| will be set to the filename of the current part
so that each part receives an individual |.aux| file
that does not interfere with the |.aux| file(s) of the main document.
This behaviour can be altered by the alternative form
|\childdocby[*]{|\textit{main}|}| (with a non-empty optional argument)
which uses the |.aux| file of the main document
by setting |\jobname| to \textit{main}.

%%%%%%%%%%%%%%%%%%%%%%%%%%%%%%%%%%%%%%%%%%%%%%%%%%%%%%%%%%%%%%%%%%%%%%%%%%%%%%%%
\subsection{Driver Development}
\label{sec:driver}

The \textsf{childdoc} mechanism can also be use for the development
of definition files such as \LaTeX{} styles or classes.
This case differs from the above setup with multiple parts
included by |\include| in that no |\includeonly| should be invoked.
This can be achieved by starting the include file
(before |\ProvidesPackage|) with:
%
\begin{center}
\begin{tabular}{l}
|\input{childdoc.def}|\\
|\childdocforward{|\textit{main}|}|\\
\end{tabular}
\end{center}
%
or alternatively with:
%
\begin{center}
\begin{tabular}{l}
|\input{childdoc.def}|\\
|\childdocby{|\textit{main}|}|\\
\end{tabular}
\end{center}
%
Both forms have slightly different effects as described above.
The main file is prepared as usual, see \secref{sec:include}.

%%%%%%%%%%%%%%%%%%%%%%%%%%%%%%%%%%%%%%%%%%%%%%%%%%%%%%%%%%%%%%%%%%%%%%%%%%%%%%%%
\subsection{Legacy Detection}
\label{sec:detection}

The directive |\childdocmain| in the main file can detect
whether the complete document or merely a child is to be compiled
even without using the directive |\childdocof|.
This method is deprecated because it is less robust
and there is no compelling reason to use it;
it is merely provided for backward compatibility
and it may be removed in future versions.

If the detection mechanism is to be used,
it is mandatory to correctly specify
the filename of the main file as the argument of |\childdocmain|:
%
\begin{center}
\begin{tabular}{l}
|\input{childdoc.def}|\\
|\childdocmain{|\textit{main}|}|\\
\end{tabular}
\end{center}
%
If |\jobname| does not match the argument \textit{main} of |\childdocmain|,
it is assumed that |\jobname| points to the child file to be compiled.
When using |\childdocmain| with the main file specified as argument,
it suffices to start a child file
with just |\input{|\textit{main}|}|
without loading of the package and using |\childdocof|.
If instead all processing is done
with the appropriate \textsf{childdoc} directives,
the argument of \textit{main} of |\childdocmain| can be empty.

An alternative version of the command line processing described
in \secref{sec:commandline} using the detection mechanism reads:
%
\begin{center}
|... -jobname "|\textit{target}|" "|[\textit{flags}]%
[|\def\jobname{|\textit{dest}|}|]|\input{|\textit{main}|}"|
\end{center}

%%%%%%%%%%%%%%%%%%%%%%%%%%%%%%%%%%%%%%%%%%%%%%%%%%%%%%%%%%%%%%%%%%%%%%%%%%%%%%%%
\subsection{Manual Code}
\label{sec:manual}

In case one cannot be certain whether the definitions file |childdoc.def|
is installed on the target \TeX{} distribution
and one prefers not to ship it,
it is conceivable to paste a few relevant commands into the sources.

To that end, drop all statements |\input{childdoc.def}|
and perform the replacements as outlined below.
Instead of |\childdocmain{|\textit{main}|}| add the following code
to the top of the main file:
%
\begin{center}
\begin{tabular}{l}
|\||ifdefined\childdocname\endinput\||fi\newif\ifchilddoc|\\
|\edef\childdocname{\scantokens\expandafter{\jobname\noexpand}}|\\
|\def\childdocmain{|\textit{main}|}\||ifx\childdocmain\childdocname\||else|\\
|\childdoctrue\includeonly{\childdocname}\let\jobname\childdocmain\||fi|\\
\end{tabular}
\end{center}
%
Instead of |\childdocof{|\textit{main}|}| just include the main file
at the top of each child file:
%
\begin{center}
|\input{|\textit{main}|}|
\end{center}
%
A simple redirection |\childdocforward{|\textit{dest}|}| is achieved by:
%
\begin{center}
|\def\jobname{|\textit{dest}|}\input{\jobname}|
\end{center}
%
The redirection with prefix
|\childdocforwardprefix[|\textit{prefix}|]{|\textit{dest}|}|
is accomplished by:
%
\begin{center}
\begin{tabular}{l}
|{\edef\jobname{\scantokens\expandafter{\jobname\noexpand}}|\\
|\def\redirectjob |\textit{prefix}|#1~~~{\gdef\jobname{|\textit{dest}|#1}}|\\
|\expandafter\redirectjob\jobname~~~}\input{\jobname}|
\end{tabular}
\end{center}

In an alternative approach,
child documents can be compiled by a specific command line
without additional code or specific definitions:
%
\begin{center}
|... -jobname "|\textit{target}|" "|[\textit{flags}]%
|\includeonly{|\textit{dest}|}\input{|\textit{main}|}"|
\end{center}
%

%%%%%%%%%%%%%%%%%%%%%%%%%%%%%%%%%%%%%%%%%%%%%%%%%%%%%%%%%%%%%%%%%%%%%%%%%%%%%%%%
%%%%%%%%%%%%%%%%%%%%%%%%%%%%%%%%%%%%%%%%%%%%%%%%%%%%%%%%%%%%%%%%%%%%%%%%%%%%%%%%
\section{Information}

%%%%%%%%%%%%%%%%%%%%%%%%%%%%%%%%%%%%%%%%%%%%%%%%%%%%%%%%%%%%%%%%%%%%%%%%%%%%%%%%
\subsection{Copyright}

Copyright \copyright{} 2017--2018 Niklas Beisert

This work may be distributed and/or modified under the
conditions of the \LaTeX{} Project Public License, either version 1.3
of this license or (at your option) any later version.
The latest version of this license is in
  \url{http://www.latex-project.org/lppl.txt}
and version 1.3 or later is part of all distributions of \LaTeX{}
version 2005/12/01 or later.

This work has the LPPL maintenance status `maintained'.

The Current Maintainer of this work is Niklas Beisert.

This work consists of the files |README.txt|, |childdoc.ins| and |childdoc.dtx|
as well as the derived files |childdoc.def|, |cdocsamp.tex|
with |cdocsch1.tex|, |cdocsch2.tex|, |cdocspt3.tex|, |cdocspt4.tex|,
|cdocsdrf.tex|, |cdocsfn1.tex|, |cdocsfn2.tex|
as well as |childdoc.pdf|.

%%%%%%%%%%%%%%%%%%%%%%%%%%%%%%%%%%%%%%%%%%%%%%%%%%%%%%%%%%%%%%%%%%%%%%%%%%%%%%%%
\subsection{Files and Installation}

The package consists of the files:
%
\begin{center}
\begin{tabular}{ll}
    |README.txt|   & readme file \\
    |childdoc.ins| & installation file \\
    |childdoc.dtx| & source file \\
    |childdoc.def| & definition file \\
    |cdocsamp.tex| & sample main file \\
    |cdocsch1.tex| & sample include file \\
    |cdocsch2.tex| & sample include file \\
    |cdocspt3.tex| & sample part file \\
    |cdocspt4.tex| & sample part file \\
    |cdocsdrf.tex| & sample redirection file \\
    |cdocsfn1.tex| & sample redirection file \\
    |cdocsfn2.tex| & sample redirection file \\
    |childdoc.pdf| & manual
\end{tabular}
\end{center}
%
The distribution consists of the files
|README.txt|, |childdoc.ins| and |childdoc.dtx|.
%
\begin{itemize}
\item
Run (pdf)\LaTeX{} on |childdoc.dtx|
to compile the manual |childdoc.pdf| (this file).
\item
Run \LaTeX{} on |childdoc.ins| to create the definitions file |childdoc.def|
and the sample |cdocsamp.tex| with include files
|cdocsch1.tex|, |cdocsch2.tex|, |cdocspt3.tex|, |cdocspt4.tex|,
|cdocsdrf.tex|, |cdocsfn1.tex|, |cdocsfn2.tex|.
Then copy the file |childdoc.def| to an appropriate directory of your \LaTeX{}
distribution, e.g.\ \textit{texmf-root}|/tex/latex/childdoc|.
\end{itemize}

%%%%%%%%%%%%%%%%%%%%%%%%%%%%%%%%%%%%%%%%%%%%%%%%%%%%%%%%%%%%%%%%%%%%%%%%%%%%%%%%
\subsection{Related CTAN Packages}

There are several other packages which offer a similar functionality:
%
\begin{itemize}
\item
The packages
\href{http://ctan.org/pkg/docmute}{\textsf{docmute}},
\href{http://ctan.org/pkg/includex}{\textsf{includex}} and
\href{http://ctan.org/pkg/standalone}{\textsf{standalone}}
provide commands to include only the document body of
a child file thus allowing both files to be compiled individually.
\item
The packages \href{http://ctan.org/pkg/subdocs}{\textsf{subdocs}}
and \href{http://ctan.org/pkg/subfiles}{\textsf{subfiles}}
provide structures in which the main and child documents can be
encapsulated and allowing them to be compiled individually.
The inclusion mechanism is different from the conventional |\include|.
\item
The package \href{http://ctan.org/pkg/combine}{\textsf{combine}}
is an elaborate solution to combine several documents into one.
\end{itemize}
%
See also the CTAN topic \href{http://ctan.org/topic/subdocs}{\textsf{subdocs}}
for further related packages.
The present package differs from the above solutions in that
a document structure constructed with the conventional |\include| mechanism
just needs two extra commands at the top of every file
such that all constituent files can be compiled individually.

%%%%%%%%%%%%%%%%%%%%%%%%%%%%%%%%%%%%%%%%%%%%%%%%%%%%%%%%%%%%%%%%%%%%%%%%%%%%%%%%
%\subsection{Feature Suggestions}
%
%The following is a list of features which may be useful for future
%versions of this package:
%%
%\begin{itemize}
%\item
%\ldots
%\end{itemize}

%%%%%%%%%%%%%%%%%%%%%%%%%%%%%%%%%%%%%%%%%%%%%%%%%%%%%%%%%%%%%%%%%%%%%%%%%%%%%%%%
\subsection{Revision History}

%%%%%%%%%%%%%%%%%%%%%%%%%%%%%%%%%%%%%%%%
\paragraph{v2.0:} 2018/12/30

\begin{itemize}
\item
immediate forward processing
\item
added |\childdocby| mechanism
\item
manual restructured
\end{itemize}

%%%%%%%%%%%%%%%%%%%%%%%%%%%%%%%%%%%%%%%%
\paragraph{v1.6:} 2018/01/17

\begin{itemize}
\item
application for development of include files
\item
corrections to manual
\end{itemize}

%%%%%%%%%%%%%%%%%%%%%%%%%%%%%%%%%%%%%%%%
\paragraph{v1.5:} 2017/05/21

\begin{itemize}
\item
more complete structuring introduced
\item
|\childdocof| introduced
\item
|\childdoc| renamed to |\childdocmain|
\item
|\childredirect| renamed to |\childdocforward| and |\childdocforwardprefix|
and functionality expanded
\end{itemize}

%%%%%%%%%%%%%%%%%%%%%%%%%%%%%%%%%%%%%%%%
\paragraph{v1.0:} 2017/04/27

\begin{itemize}
\item
manual and install package
\item
first version published on CTAN
\end{itemize}

%%%%%%%%%%%%%%%%%%%%%%%%%%%%%%%%%%%%%%%%
\paragraph{v0.6:} 2017/04/26

\begin{itemize}
\item
redirection mechanism added
\end{itemize}

%%%%%%%%%%%%%%%%%%%%%%%%%%%%%%%%%%%%%%%%
\paragraph{v0.5:} 2017/04/26

\begin{itemize}
\item
functionality in definition file
\end{itemize}


%%%%%%%%%%%%%%%%%%%%%%%%%%%%%%%%%%%%%%%%%%%%%%%%%%%%%%%%%%%%%%%%%%%%%%%%%%%%%%%%
%%%%%%%%%%%%%%%%%%%%%%%%%%%%%%%%%%%%%%%%%%%%%%%%%%%%%%%%%%%%%%%%%%%%%%%%%%%%%%%%
%%%%%%%%%%%%%%%%%%%%%%%%%%%%%%%%%%%%%%%%%%%%%%%%%%%%%%%%%%%%%%%%%%%%%%%%%%%%%%%%
\appendix

\settowidth\MacroIndent{\rmfamily\scriptsize 000\ }

 \DocInput{childdoc.dtx}

\end{document}
%</driver>
% \fi
%
% %%%%%%%%%%%%%%%%%%%%%%%%%%%%%%%%%%%%%%%%%%%%%%%%%%%%%%%%%%%%%%%%%%%%%%%%%%%%%%
% %%%%%%%%%%%%%%%%%%%%%%%%%%%%%%%%%%%%%%%%%%%%%%%%%%%%%%%%%%%%%%%%%%%%%%%%%%%%%%
% \section{Sample}
%\iffalse
%<*samplemain>
%\fi
%
% The following presents a sample document
% with two chapters, two parts, a title page,
% a compile flag as well as three forwarding files to set the flag.
% It consists of eight |.tex| files:
% \begin{center}
% \begin{tabular}{ll}
% |cdocsamp.tex|&main file\\
% |cdocsch1.tex|&include file for chapter 1\\
% |cdocsch2.tex|&include file for chapter 2\\
% |cdocspt3.tex|&include file for part 3\\
% |cdocspt4.tex|&include file for part 4\\
% |cdocsdrf.tex|&forwarding file for main file in draft mode\\
% |cdocsfi1.tex|&forwarding file for final version of chapter 1\\
% |cdocsfi2.tex|&forwarding file for final version of chapter 2\\
% \end{tabular}
% \end{center}
% Each of the eight files can be compiled directly by the \LaTeX{} compiler.
%
% %%%%%%%%%%%%%%%%%%%%%%%%%%%%%%%%%%%%%%
% \paragraph{Main File.}
%
% The main file is called |cdocsamp.tex|.
%
% Load the \textsf{childdoc} definitions and
% declare the filename for the main document:
%    \begin{macrocode}
\input{childdoc.def}
\childdocmain{}
%    \end{macrocode}

% Optional override for |\version| flag:
%    \begin{macrocode}
%%\ifchilddoc\else\providecommand{\version}{draft}\fi
%    \end{macrocode}

% Define the default values for the |\version| flag
% (|final| for the main file and |draft| for childs):
%    \begin{macrocode}
\ifchilddoc
\providecommand{\version}{draft}
\else
\providecommand{\version}{final}
\fi
%    \end{macrocode}

% Load the standard document class:
%    \begin{macrocode}
\documentclass[12pt]{article}
%    \end{macrocode}

% Start the document body:
%    \begin{macrocode}
\begin{document}
%    \end{macrocode}

% Declare a title page.
% Print title, part of document being processed and version flag:
%    \begin{macrocode}
\addtocounter{page}{-1}
\begin{center}
{\LARGE\bfseries{}childdoc example\par}
\vspace{1cm}
\ifchilddoc
\ifchilddocmanual part\else chapter\fi:
`\childdocname' of `\childdocjob'\par
\else
main document: `\childdocjob'\par
\fi
version: \version\par
\end{center}
\newpage
%    \end{macrocode}

% Manually include selected file,
% otherwise process as usual:
%    \begin{macrocode}
\ifchilddocmanual
\section*{part `\childdocname'}
\input{\childdocname}
\else
%    \end{macrocode}

% Include the two chapters:
%    \begin{macrocode}
\include{cdocsch1}
\include{cdocsch2}
%    \end{macrocode}

% Include the two parts unless only chapters should be displayed:
%    \begin{macrocode}
\ifchilddoc\else
\section{part three}
\input{cdocspt3}
\section{part four}
\input{cdocspt4}
\fi
%    \end{macrocode}

% Process as usual until here:
%    \begin{macrocode}
\fi
%    \end{macrocode}

% End of document body:
%    \begin{macrocode}
\end{document}
%    \end{macrocode}
%\iffalse
%</samplemain>
%\fi
%
% %%%%%%%%%%%%%%%%%%%%%%%%%%%%%%%%%%%%%%
% \paragraph{Chapter Include Files.}
%
% The include files are called |cdocsch1.tex| and |cdocsch2.tex|.
%
%\iffalse
%<*samplechap1|samplechap2>
%\fi

% Optional override for |\version| flag:
%    \begin{macrocode}
%%\providecommand{\version}{final}
%    \end{macrocode}

% Include the main document:
%    \begin{macrocode}
\input{childdoc.def}
\childdocof{cdocsamp}
%    \end{macrocode}

%\iffalse
%</samplechap1|samplechap2>
%\fi
%
%\iffalse
%<*samplechap1>
%\fi
% Some text for chapter 1:
%    \begin{macrocode}
\section{one}
some text in chapter one
%    \end{macrocode}

%\iffalse
%</samplechap1>
%\fi
% Some text for chapter 2:
%\iffalse
%<*samplechap2>
%\fi
%    \begin{macrocode}
\section{two}
more text in chapter two
%    \end{macrocode}

%\iffalse
%</samplechap2>
%\fi
%
% %%%%%%%%%%%%%%%%%%%%%%%%%%%%%%%%%%%%%%
% \paragraph{Part Include Files.}
%
% The include files are called |cdocspt3.tex| and |cdocspt4.tex|.
%
%\iffalse
%<*samplepart3|samplepart4>
%\fi

% Optional override for |\version| flag:
%    \begin{macrocode}
%%\providecommand{\version}{final}
%    \end{macrocode}

% Include the main document:
%    \begin{macrocode}
\input{childdoc.def}
\childdocby{cdocsamp}
%    \end{macrocode}

%\iffalse
%</samplepart3|samplepart4>
%\fi
%
%\iffalse
%<*samplepart3>
%\fi
% Some text for part 3:
%    \begin{macrocode}
some text in part three
%    \end{macrocode}

%\iffalse
%</samplepart3>
%\fi
% Some text for part 4:
%\iffalse
%<*samplepart4>
%\fi
%    \begin{macrocode}
more text in part four
%    \end{macrocode}

%\iffalse
%</samplepart4>
%\fi
%
% %%%%%%%%%%%%%%%%%%%%%%%%%%%%%%%%%%%%%%
% \paragraph{Forwarding for a Complete Draft.}
%
% The following forwarding file |cdocsdrf.tex|
% compiles the main document in draft mode:
%\iffalse
%<*sampledraft>
%\fi
%    \begin{macrocode}
\def\version{draft}
\input{childdoc.def}
\childdocforward{cdocsamp}
%    \end{macrocode}

%\iffalse
%</sampledraft>
%\fi
%
% %%%%%%%%%%%%%%%%%%%%%%%%%%%%%%%%%%%%%%
% \paragraph{Forwarding for Final Version of the Chapters.}
%
% The following forwarding files |cdocsfn1.tex| and |cdocsfn2.tex|
% (with identical content)
% compile the final versions of the child documents
% |cdocsch1.tex| and |cdocsch2.tex|, respectively:
%\iffalse
%<*samplefinal>
%\fi
%    \begin{macrocode}
\def\version{final}
\input{childdoc.def}
\childdocforwardprefix[cdocsamp]{cdocsfn}{cdocsch}
%    \end{macrocode}

%\iffalse
%</samplefinal>
%\fi
%
% %%%%%%%%%%%%%%%%%%%%%%%%%%%%%%%%%%%%%%
% \paragraph{Command Line Processing.}
%
% The following three command lines generate the output files
% |cdocscld|, |cdocscl1| and |cdocscl2|
% which should be identical to
% |cdocsdrf|, |cdocsch1| and |cdocsfn2|, respectively:
% \begin{center}
% \begin{tabular}{l}
% |latex -jobname cdocscld \|\\
% |  "\def\version{draft}\input{childdoc.def}\childdocforward{cdocsamp}"|\\
% |latex -jobname cdocscl1 \|\\
% |  "\input{childdoc.def}\childdocforward[cdocsamp]{cdocsch1}"|\\
% |latex -jobname cdocscl2 \|\\
% |  "\def\version{final}\input{childdoc.def}\childdocforward{cdocsch2}"|
% \end{tabular}
% \end{center}
% Note that the trailing backslash on each first line
% merely continues the input to the second line
% (for convenient cut ant paste).
% Furthermore, the command |latex| can be replaced by any
% of its alternative versions such as |pdflatex|.
%
% %%%%%%%%%%%%%%%%%%%%%%%%%%%%%%%%%%%%%%%%%%%%%%%%%%%%%%%%%%%%%%%%%%%%%%%%%%%%%%
% %%%%%%%%%%%%%%%%%%%%%%%%%%%%%%%%%%%%%%%%%%%%%%%%%%%%%%%%%%%%%%%%%%%%%%%%%%%%%%
% \section{Implementation}
%\iffalse
%<*package>
%\fi
%
% This section describes the definitions file |childdoc.def|.

% The definitions cannot be loaded using |\usepackage| or |\RequirePackage|
% which has a mechanism to prevent loading a style file more than once.
% When loading the definitions by means of |\input|
% multiple instances have to be prevented manually:
%\iffalse
%This code needs to be before the `\ProvidesFile' directive
%which is defined at the beginning of this file.
%Therefore it is also placed there and commented out here.
%</package>
%<*discard>
%\fi
%    \begin{macrocode}
\ifdefined\childdocmain\endinput\fi
%    \end{macrocode}
%\iffalse
%</discard>
%<*package>
%\fi
%
% \macro{\ifchilddoc}
% \macro{\ifchilddocmanual}
% The conditional |\ifchilddoc| tells whether a
% child (true) or main (false) document is being compiled.
% The conditional |\ifchilddocmanual| tells whether
% the |\includeonly| mechanism is used (false) or
% the selection of child files must be performed manually (true).
% The definitions initialise to false:
%    \begin{macrocode}
\newif\ifchilddoc
\newif\ifchilddocmanual
%    \end{macrocode}

% \macro{\childdocname}
% \macro{\childdocjob}
% The macro |\childdocname| stores the name of the main document
% to be compiled. The macro |\childdocjob| stores the name of
% the document on which the \LaTeX{} compiler was originally invoked.
% The content of |\jobname| cannot be compared
% to filenames specified in the source due to different catcodes.
% The following code rescans |\jobname|, stores the result
% in |\childdocname| and saves a copy in |\childdocjob|:
%    \begin{macrocode}
\edef\childdocname{\scantokens\expandafter{\jobname\noexpand}}
\let\childdocjob\childdocname
%    \end{macrocode}

% \macro{\childdocdisable}
% The macro |\childdocdisable| prevents the main file
% from being processed more than once.
% At this stage, the main document command |\childdocmain|
% is assumed to be called once again where it should do nothing.
% Any subsequent call to it should prevent
% a secondary processing of the main document
% It overwrites the forwarding commands
% |\childdocof| and |\childdocforward|
% with empty macros to prevent further inclusions of the main document:
%    \begin{macrocode}
\newcommand{\childdocdisable}
{
  \renewcommand{\childdocmain}[1]{\renewcommand{\childdocmain}[1]{\endinput}}
  \renewcommand{\childdocof}[1]{}
  \renewcommand{\childdocby}[2][]{}
  \renewcommand{\childdocforward}[2][]{}
  \renewcommand{\childdocdisable}{}
}
%    \end{macrocode}

% \macro{\childdocmain}
% The macro |\childdocmain| is to be called at the top of the main file
% with nothing or the main filename (without extension) as argument.
% First, it breaks loops.
% If the argument is not empty and does not match |\childdocname|
% (which is set by the first inclusion of |childdoc.def|),
% |\ifchilddoc| is set to true, |\includeonly| is applied to the child file
% and |\jobname| is set to the main file
% (for proper handling of |.aux| files):
%    \begin{macrocode}
\newcommand{\childdocmain}[1]
{
  \childdocdisable\childdocmain{}
  \if?#1?\else
    \begingroup
      \def\childdoctmp{#1}
      \ifx\childdoctmp\childdocname
        \def\childdoctmp{}
      \else
        \def\childdoctmp
        {
          \childdoctrue
          \includeonly{\childdocname}
          \def\childdocjob{#1}
          \def\jobname{#1}
        }
      \fi
      \expandafter
    \endgroup
    \childdoctmp
  \fi
}
%    \end{macrocode}

% \macro{\childdocof}
% The command |\childdocof| redirects
% compilation to the main file |#1|.
%    \begin{macrocode}
\newcommand{\childdocof}[1]
{
  \childdocdisable
  \childdoctrue
  \includeonly{\childdocname}
  \def\jobname{#1}
  \def\childdocjob{#1}
  \input{#1}
}
%    \end{macrocode}

% \macro{\childdocby}
% The command |\childdocby| ....
%    \begin{macrocode}
\newcommand{\childdocby}[2][]
{
  \childdocdisable
  \childdoctrue
  \childdocmanualtrue
  \if?#1?\else
    \def\jobname{#2}
  \fi
  \def\childdocjob{#2}
  \input{#2}
  \endinput
}
%    \end{macrocode}

% \macro{\childdocforward}
% The command |\childdocforward| redirects
% compilation to the main file or
% (if the optional argument is given) a child file.
% Parameters are set as if the main file
% or a child file starting with |\childdocof| was compiled.
% Then compilation is handed over to the main file:
%    \begin{macrocode}
\newcommand{\childdocforward}[2][]
{
  \begingroup
    \if?#1?
      \def\childdoctmp
      {
        \def\childdocname{#2}
        \def\childdocjob{#2}
        \def\jobname{#2}
        \input{#2}
        \endinput
      }
    \else
      \def\childdoctmp
      {
        \childdocdisable
        \def\childdocname{#2}
        \childdoctrue
        \includeonly{#2}
        \def\childdocjob{#1}
        \def\jobname{#1}
        \input{#1}
        \endinput
      }
    \fi
    \expandafter
  \endgroup
  \childdoctmp
}
%    \end{macrocode}

% \macro{\childdocforwardprefix}
% The command |\childdocforwardprefix| redirects
% compilation to the main or a child file by means of a pattern.
% The prefix |#1| in the current filename is replaced by |#2|
% and the suffix of the current filename is kept
% (it is assumed that the filename does not contain the substring `|~~~|'
% which is used as a delimiter).
% Compilation is handed over to the new file by |\childdocforward|:
%    \begin{macrocode}
\newcommand{\childdocforwardprefix}[3][]
{
  \begingroup
    \def\childdocextract #2##1~~~{\def\childdoctmp{\childdocforward[#1]{#3##1}}}
    \expandafter\childdocextract\childdocname~~~
    \expandafter
  \endgroup
  \childdoctmp
}
%    \end{macrocode}

% \macro{\childdoc}
% The deprecated macro |\childdoc| is a legacy version of |\childdocmain|:
%    \begin{macrocode}
\newcommand{\childdoc}{\childdocmain}
%    \end{macrocode}

% \macro{\childdocredirect}
% The deprecated macro |\childdocredirect| is a legacy version
% of |\childdocforward| and |\childdocforwardprefix|:
%    \begin{macrocode}
\newcommand{\childdocredirect}[2][]
{
  \begingroup
    \if?#1?
      \def\childdoctmp{\childdocforward{#2}}
    \else
      \def\childdoctmp{\childdocforwardprefix{#1}{#2}}
    \fi
    \expandafter
  \endgroup
  \childdoctmp
}
%    \end{macrocode}

%\iffalse
%</package>
%\fi
%
\endinput
|\\
|\childdocby{|\textit{main}|}|\\
\end{tabular}
\end{center}
%
The directive |\childdocby| is similar to |\childdocof|
described in \secref{sec:include},
but the subsequent selection of content must be done manually.
To that end, both |\ifchilddoc| and |\ifchilddocmanual|
will be true upon processing of a part,
and the name of the part is stored in |\childdocname|.
Note that |\jobname| will be set to the filename of the current part
so that each part receives an individual |.aux| file
that does not interfere with the |.aux| file(s) of the main document.
This behaviour can be altered by the alternative form
|\childdocby[*]{|\textit{main}|}| (with a non-empty optional argument)
which uses the |.aux| file of the main document
by setting |\jobname| to \textit{main}.

%%%%%%%%%%%%%%%%%%%%%%%%%%%%%%%%%%%%%%%%%%%%%%%%%%%%%%%%%%%%%%%%%%%%%%%%%%%%%%%%
\subsection{Driver Development}
\label{sec:driver}

The \textsf{childdoc} mechanism can also be use for the development
of definition files such as \LaTeX{} styles or classes.
This case differs from the above setup with multiple parts
included by |\include| in that no |\includeonly| should be invoked.
This can be achieved by starting the include file
(before |\ProvidesPackage|) with:
%
\begin{center}
\begin{tabular}{l}
|% \iffalse
%
% childdoc.dtx Copyright (C) 2017-2018 Niklas Beisert
%
% This work may be distributed and/or modified under the
% conditions of the LaTeX Project Public License, either version 1.3
% of this license or (at your option) any later version.
% The latest version of this license is in
%   http://www.latex-project.org/lppl.txt
% and version 1.3 or later is part of all distributions of LaTeX
% version 2005/12/01 or later.
%
% This work has the LPPL maintenance status `maintained'.
%
% The Current Maintainer of this work is Niklas Beisert.
%
% This work consists of the files childdoc.dtx and childdoc.ins
% and the derived files childdoc.def and cdocsamp.tex with
% cdocsch1.tex, cdocsch2.tex, cdocsdrf.tex, cdocsfn1.tex, cdocsfn2.tex.
%
%<package>\ifdefined\childdocmain\endinput\fi
%<package>\ProvidesFile{childdoc.def}[2018/12/30 v2.0 child document driver]
%<samplemain>\ProvidesFile{cdocsamp.tex}[2018/12/30 v2.0 sample for childdoc]
%<*driver>
%\ProvidesFile{childdoc.drv}[2018/12/30 v2.0 childdoc reference manual file]
\PassOptionsToClass{10pt,a4paper}{article}
\documentclass{ltxdoc}

\usepackage[margin=35mm]{geometry}
\usepackage{hyperref}
\usepackage{hyperxmp}
\usepackage[usenames]{color}

\hypersetup{colorlinks=true}
\hypersetup{pdfstartview=FitH}
\hypersetup{pdfpagemode=UseNone}
\hypersetup{pdfsource={}}
\hypersetup{pdflang={en-UK}}
\hypersetup{pdfcopyright={Copyright 2017-2018 Niklas Beisert.
  This work may be distributed and/or modified under the
  conditions of the LaTeX Project Public License, either version 1.3
  of this license or (at your option) any later version.}}
\hypersetup{pdflicenseurl={http://www.latex-project.org/lppl.txt}}
\hypersetup{pdfcontactaddress={ETH Zurich, ITP, HIT K,
  Wolfgang-Pauli-Strasse 27}}
\hypersetup{pdfcontactpostcode={8093}}
\hypersetup{pdfcontactcity={Zurich}}
\hypersetup{pdfcontactcountry={Switzerland}}
\hypersetup{pdfcontactemail={nbeisert@itp.phys.ethz.ch}}
\hypersetup{pdfcontacturl={http://people.phys.ethz.ch/\xmptilde nbeisert/}}

\newcommand{\secref}[1]{\hyperref[#1]{section \ref*{#1}}}

\parskip1ex
\parindent0pt
\let\olditemize\itemize
\def\itemize{\olditemize\parskip0pt}

\begin{document}

\title{The \textsf{childdoc} Package}
\hypersetup{pdftitle={The childdoc Package}}
\author{Niklas Beisert\\[2ex]
  Institut f\"ur Theoretische Physik\\
  Eidgen\"ossische Technische Hochschule Z\"urich\\
  Wolfgang-Pauli-Strasse 27, 8093 Z\"urich, Switzerland\\[1ex]
  \href{mailto:nbeisert@itp.phys.ethz.ch}
  {\texttt{nbeisert@itp.phys.ethz.ch}}}
\hypersetup{pdfauthor={Niklas Beisert}}
\hypersetup{pdfsubject={Manual for the LaTeX2e Package childdoc}}
\date{30 December 2018, \textsf{v2.0}}
\maketitle

\begin{abstract}\noindent
\textsf{childdoc} is a \LaTeXe{} package
that enables the direct compilation
of document sections included by |\include|
to individual files.
\end{abstract}

\begingroup
\parskip0ex
\tableofcontents
\endgroup

%%%%%%%%%%%%%%%%%%%%%%%%%%%%%%%%%%%%%%%%%%%%%%%%%%%%%%%%%%%%%%%%%%%%%%%%%%%%%%%%
%%%%%%%%%%%%%%%%%%%%%%%%%%%%%%%%%%%%%%%%%%%%%%%%%%%%%%%%%%%%%%%%%%%%%%%%%%%%%%%%
\section{Introduction}

\LaTeX{} provides a mechanism to structure a large document (such as a book)
into a main file and several child files (containing the chapters)
using the |\include| command.
This mechanism is beneficial for documents
which span hundreds of pages in order to
make the source file(s) more manageable.
Moreover, compilation can be restricted to
selected child files by means of the |\includeonly| command.
The latter feature can be used to reduce the compilation time while editing
(this was significantly more useful in the earlier days of \LaTeX{})
or to generate a smaller document which is easier to navigate.
Another application of |\includeonly| is to generate
documents consisting of selected parts of the complete document.

However, there are a few drawbacks of the plain |\include| mechanism:
\begin{itemize}
\item
The child files cannot be compiled on their own,
they can only be compiled via the main file.
A naive editing environment
(such as a text editor with an option
to have the current file processed by \LaTeX)
may require one to switch to the main file before compiling;
attempting to compile the child file produces errors.
\item
The main file must be modified (each time)
to adjust the |\includeonly| command
to the present needs. This easily leaves the main file in a messy state.
\item
The generated document will always carry the filename
of the main document. This is inconvenient if
several child files are to be compiled and
to be kept for distribution.
\end{itemize}

The present package provides a simple interface
to make child files individually compilable by \LaTeX{}.
Compiling a child file then has the same effect as compiling
the main file with an |\includeonly| command
to select the appropriate child.
Moreover the generated document will carry the name of the child
rather than the main file.
This resolves all three above issues.

This feature is meant to make the editing of books,
thesis documents and lecture notes somewhat more convenient.
However, the package can also be used efficiently for
composing a series of documents (such as exercise sheets)
which are typically distributed individually.
It then assists the author in generating the individual documents
(potentially in different versions)
as well as a document containing the collected series.
Another application is in developing style files
or other kinds of included material
where compilation of the style file could redirect
to a sample or test file.

%%%%%%%%%%%%%%%%%%%%%%%%%%%%%%%%%%%%%%%%%%%%%%%%%%%%%%%%%%%%%%%%%%%%%%%%%%%%%%%%
%%%%%%%%%%%%%%%%%%%%%%%%%%%%%%%%%%%%%%%%%%%%%%%%%%%%%%%%%%%%%%%%%%%%%%%%%%%%%%%%
\section{Usage}

First of all, the package \textsf{childdoc} is \emph{not} a standard
\LaTeXe{} |.sty| style file! Therefore it needs to be invoked in
a non-standard way.

%%%%%%%%%%%%%%%%%%%%%%%%%%%%%%%%%%%%%%%%%%%%%%%%%%%%%%%%%%%%%%%%%%%%%%%%%%%%%%%%
\subsection{Included Files}
\label{sec:include}

%%%%%%%%%%%%%%%%%%%%%%%%%%%%%%%%%%%%%%%%
\DescribeMacro{\childdocmain}
To use the package, add the commands
\begin{center}
\begin{tabular}{l}
|\input{childdoc.def}|\\
|\childdocmain{}|\\
\end{tabular}
\end{center}
at the very top of the main \LaTeX{} file,
in particular \emph{before} the |\documentclass| statement!
The argument of |\childdocmain| should be left empty
(but it must be present).

%%%%%%%%%%%%%%%%%%%%%%%%%%%%%%%%%%%%%%%%
\DescribeMacro{\childdocof}
Furthermore, add the commands
\begin{center}
\begin{tabular}{l}
|\input{childdoc.def}|\\
|\childdocof{|\textit{main}|}|\\
\end{tabular}
\end{center}
at the top of every child file \textit{child}
which is included by |\include{|\textit{child}|}|
from within the main file
(or at least for those files to be compiled individually).
The argument \textit{main} must be the filename of the main file.

There are a couple of
considerations in setting up the main and child documents:

%%%%%%%%%%%%%%%%%%%%%%%%%%%%%%%%%%%%%%%%
\paragraph{Restrictions.}

Please note the following restrictions:
\begin{itemize}
\item
|\childdocmain| must be called with one argument \textit{main}
to ensure compatibility with earlier version of the package.
It must either be empty (|\childdocmain{}|)
or precisely match the filename of the main file in which it is specified.
See \secref{sec:detection} for further information.
\item
The filename \textit{main} must be specified without the |.tex| extension.
\item
The filename \textit{main} is case sensitive
(even in case-insensitive file systems)
due to internal string comparison.
\item
The argument \textit{main} should be fully expanded, it cannot be a macro.
\item
Subdirectories and special characters should be avoided in filenames.
\item
The command |\childdocmain{|\textit{main}|}| must be followed by a whitespace.
It should not be followed immediately by another command
or by a comment mark `|%|'.
This is because the \TeX{} parser reads the token immediately following
the argument of |\childdocmain| and puts it
at the beginning of every child section;
however, a white\-space is ignored.
\end{itemize}

%%%%%%%%%%%%%%%%%%%%%%%%%%%%%%%%%%%%%%%%
\paragraph{Content of Main File.}

It is advisable to place all content in the child files included by |\include|.
Any output contained in the main file will appear in all child documents
unless suppressed manually;
it cannot be suppressed automatically by the |\includeonly| directive
and thus should normally be avoided.
A method to include some content in the main file
by means of conditional processing is described in \secref{sec:conditional}.

%%%%%%%%%%%%%%%%%%%%%%%%%%%%%%%%%%%%%%%%
\paragraph{Page Numbering.}

When only a part of the document is compiled,
the appropriate numbering of pages
(as well as other status parameters)
is determined from the |.aux| files.
The latter contain information from previous passes.
However this information needs to propagate through
all intermediate child documents.
Therefore the page numbering in child documents may well
be inconsistent until the complete document is compiled at least once.

A useful (if unconventional) way to always ensure a consistent
page numbering is to restart the numbering in each child document
and denote the pages by `\textit{child}|.|\textit{page}'
where \textit{child} represents the chapter/section number of the child file.
This can be achieved by the command
|\numberwithin{page}{|\textit{child}|}|
of the \textsf{amsmath} package
where \textit{child} can be |chapter| or |section|
depending on the chosen structuring.
Alternatively, one can modify the macro |\thepage| appropriately
and reset the counter |page| at the start of each child file.

%%%%%%%%%%%%%%%%%%%%%%%%%%%%%%%%%%%%%%%%%%%%%%%%%%%%%%%%%%%%%%%%%%%%%%%%%%%%%%%%
\subsection{Conditional Processing}
\label{sec:conditional}

The package provides a mechanism to compile different versions
of a document. To customise the versions further some conditional processing
can come in handy to distinguish which version is being compiled.
The package provides two macros to describe the compilation context:

%%%%%%%%%%%%%%%%%%%%%%%%%%%%%%%%%%%%%%%%
\DescribeMacro{\ifchilddoc}
The conditional |\ifchilddoc| distinguishes between the compilation of
child documents and the main document:
%
\begin{center}
|\ifchilddoc |\textit{child-code}| |[|\||else |\textit{main-code}]| \||fi|
\end{center}

%%%%%%%%%%%%%%%%%%%%%%%%%%%%%%%%%%%%%%%%
\DescribeMacro{\childdocname}
\DescribeMacro{\childdocjob}
The macro |\childdocname| contains the filename (without extension)
of the main or child file being processed.
Note that |\childdocjob| will always contain the name of the main file.

%%%%%%%%%%%%%%%%%%%%%%%%%%%%%%%%%%%%%%%%
\paragraph{Title Page.}

Conditional processing can be used to include a title or banner page
in the main document when proper precautions are taken.
Importantly, the code in the main file should ensure that the page counter
(as well as other status parameters which are stored in the |.aux| files)
takes the same value after the conditional processing.
Otherwise the page numbers may take divergent values
depending on which part is compiled.

For example, a title page could be declared by:
%
\begin{center}
\begin{tabular}{l}
|\ifchilddoc\||else|\\
|\addtocounter{page}{-1}|\\
\textit{code for title page}\\
|\newpage|\\
|\||fi|
\end{tabular}
\end{center}
%
A banner page for the child documents can be generated by:
%
\begin{center}
\begin{tabular}{l}
|\ifchilddoc|\\
|\addtocounter{page}{-1}|\\
\textit{code for banner page}\\
|\newpage|\\
|\||fi|
\end{tabular}
\end{center}
%
Here one could write a message such as:
\begin{center}
|This is the part \childdocname{} of \childdocjob{}.|
\end{center}

%%%%%%%%%%%%%%%%%%%%%%%%%%%%%%%%%%%%%%%%%%%%%%%%%%%%%%%%%%%%%%%%%%%%%%%%%%%%%%%%
\subsection{Flags}
\label{sec:flags}

The package makes it easy to generate different versions
of the main or child documents.
To this end compilation flags can be defined
and assigned different default values.
They will be particularly useful in conjunction
with the forwarding mechanism described in \secref{sec:forward}.

For example, it may be useful to have a flag |\version|
which can be set to |draft| or |final|.
The document source will contain some conditional code
depending on the value of |\version|.
Suppose further, the flag should default to |final| for the main file
and to |draft| for child files
which is a natural assignment for editing the document.
This is achieved by placing the following code
in the preamble of the main document
(below the |\childdocmain| directive):
%
\begin{center}
\begin{tabular}{l}
|\ifchilddoc|\\
|\providecommand{\version}{draft}|\\
|\||else|\\
|\providecommand{\version}{final}|\\
|\||fi|
\end{tabular}
\end{center}
%
The definition by |\providecommand| makes sure
that previous definitions are not overwritten.
Further statements |\providecommand{\version}{...}|
can thus be added before the above code to override it.

For the main file, one might add a line
(between |\childdocmain| and the above block)
%
\begin{center}
|%\ifchilddoc\||else\providecommand{\version}{draft}\||fi|
\end{center}
%
which can be uncommented to produce a draft version.
Likewise one can add a line to the very top of a child file
(above the |\childdocof{|\textit{main}|}| directive)
%
\begin{center}
|%\providecommand{\version}{final}|
\end{center}
%
which can be uncommented to produce the final version of this child document.

%%%%%%%%%%%%%%%%%%%%%%%%%%%%%%%%%%%%%%%%%%%%%%%%%%%%%%%%%%%%%%%%%%%%%%%%%%%%%%%%
\subsection{Forwarding}
\label{sec:forward}

Different versions of the main or child documents
using compilation flags as described in \secref{sec:flags}
can be (permanently) stored in different files
for convenient compilation, viewing and distribution.
To this end, the package defines a command
to pass on compilation to a different file:

%%%%%%%%%%%%%%%%%%%%%%%%%%%%%%%%%%%%%%%%
\DescribeMacro{\childdocforward}
The command |\childdocforward| redirects processing to
another source file:
%
\begin{center}
\begin{tabular}{l}
|\input{childdoc.def}|\\
|\childdocforward[|\textit{main}|]{|\textit{dest}|}|\\
\end{tabular}
\end{center}
%
The argument \textit{dest} is the destination file
(without extension).
It should be the main file or one of the child files.
Note that further \textsf{childdoc} directives
such as |\childdocof| and |\childdocforward|
in the indicated file will be processed in this form.
The optional argument \textit{main}
passes on directly to the main file \textit{main}
while pretending to compile the child \textit{dest}.
This form behaves as if \textit{dest}
issues |\childdocof{|\textit{main}|}| right away,
and no further \textsf{childdoc} directives will be processed.

%%%%%%%%%%%%%%%%%%%%%%%%%%%%%%%%%%%%%%%%
\DescribeMacro{\...prefix}
In the alternative form |\childdocforwardprefix|,
%
\begin{center}
\begin{tabular}{l}
|\input{childdoc.def}|\\
|\childdocforwardprefix[|\textit{main}|]{|\textit{prefix}|}{|\textit{dest}|}|
\end{tabular}
\end{center}
%
the destination file is determined by a pattern
depending on the current file:
To make this work, the current file must be called
`{\textit{prefix}\hspace{0.2em}\textit{suffix}}'
with \textit{prefix} matching precisely the argument.
Processing is then passed on to the file
`{\textit{dest}\hspace{0.2em}\textit{suffix}}'.
Surely, the same effect is achieved by
directly specifying the
argument `{\textit{dest}\hspace{0.2em}\textit{suffix}}'
in the first form.
However, that requires to set up a different file
for each child. With the alternative form of the command
all these files can have exactly the same content
which simplifies setting them up and maintaining them.

For example, the following file |draft.tex|
with a compilation flag |\version| as described in \secref{sec:flags}
compiles the main document as a draft:
%
\begin{center}
\begin{tabular}{l}
|\def\version{draft}|\\
|\input{childdoc.def}|\\
|\childdocforward{|\textit{main}|}|
\end{tabular}
\end{center}
%
Likewise, the following files |final|\textit{nn}|.tex|
compile the final version of the child document
|child|\textit{nn}|.tex|:
%
\begin{center}
\begin{tabular}{l}
|\def\version{final}|\\
|\input{childdoc.def}|\\
|\childdocforwardprefix{final}{child}|
\end{tabular}
\end{center}
%

Note that when several versions of a main file and/or of each child file
are to be generated, it may be convenient to set up a |Makefile| or
shell script to automatise the process.

%%%%%%%%%%%%%%%%%%%%%%%%%%%%%%%%%%%%%%%%%%%%%%%%%%%%%%%%%%%%%%%%%%%%%%%%%%%%%%%%
\subsection{Command Line Processing}
\label{sec:commandline}

The effect of redirection files can also be achieved by invoking
the \LaTeX{} compiler with a more elaborate command line.
Most conveniently this should be done as part
of a shell script or a |Makefile|.

When using \textsf{childdoc} in the main file, the following
command lines effectively perform a redirection
(note that depending on the shell being used,
backslashes may have to be doubled: `|\|' $\to$ `|\\|'):
%
\begin{center}
|... -jobname "|\textit{target}|" |\\|"|[\textit{flags}]%
|\input{childdoc.def}\childdocforward[|\textit{main}|]{|\textit{dest}|}"|
\end{center}
%
Here \textit{target} is the name of the output file,
\textit{main} is the name of the main file
and \textit{dest} is the name of the main or child file to be processed
(all filenames without extensions).
The optional argument \textit{main} can be omitted
if \textit{main} matches \textit{dest}.
Optionally, compilation \textit{flags} can be defined via |\def| commands.
This command line makes the \TeX{} engine believe
it is compiling the file \textit{target}
whose content is specified as the latter parameter.
The provided code then forwards the processing to
\textit{main} or \textit{dest} as described in \secref{sec:forward}.

%%%%%%%%%%%%%%%%%%%%%%%%%%%%%%%%%%%%%%%%%%%%%%%%%%%%%%%%%%%%%%%%%%%%%%%%%%%%%%%%
\subsection{Include by Input}
\label{sec:input}

Including child documents by |\include| has some restrictions by design.
Most notably, the content of a child document always occupies
its own set of pages; pages cannot be shared between child documents.
Usually, this behaviour makes perfect sense
because each child document contain an essential part of the document.
However, in some situations it may be desirable to compose
a document from a collection of parts
without having mandatory page breaks between then.
For this case, the package
provides a mechanism to include parts
by |\input| which can also be processed individually.
However, by construction this mechanism
requires manual handling of the content to be output.

%%%%%%%%%%%%%%%%%%%%%%%%%%%%%%%%%%%%%%%%
\DescribeMacro{\ifchilddocmanual}
The main file should be prepared as usual, see \secref{sec:include}.
However, the document body must make a distinction
between processing of an individual part and of the main document, e.g.:
%
\begin{center}
\begin{tabular}{l}
|\ifchilddocmanual|\\
|\input{\childdocname}|\\
|\||else|\\
\textit{document body with }|\input{|\textit{part}|}|\\
|\||fi|
\end{tabular}
\end{center}
%
The conditional |\ifchilddocmanual| is true whenever
a part to be included by |\input| is being compiled,
and the name of the part is stored in |\childdocname|.

%%%%%%%%%%%%%%%%%%%%%%%%%%%%%%%%%%%%%%%%
\DescribeMacro{\childdocby}
Each part to be included by |\input| should start with:
%
\begin{center}
\begin{tabular}{l}
|\input{childdoc.def}|\\
|\childdocby{|\textit{main}|}|\\
\end{tabular}
\end{center}
%
The directive |\childdocby| is similar to |\childdocof|
described in \secref{sec:include},
but the subsequent selection of content must be done manually.
To that end, both |\ifchilddoc| and |\ifchilddocmanual|
will be true upon processing of a part,
and the name of the part is stored in |\childdocname|.
Note that |\jobname| will be set to the filename of the current part
so that each part receives an individual |.aux| file
that does not interfere with the |.aux| file(s) of the main document.
This behaviour can be altered by the alternative form
|\childdocby[*]{|\textit{main}|}| (with a non-empty optional argument)
which uses the |.aux| file of the main document
by setting |\jobname| to \textit{main}.

%%%%%%%%%%%%%%%%%%%%%%%%%%%%%%%%%%%%%%%%%%%%%%%%%%%%%%%%%%%%%%%%%%%%%%%%%%%%%%%%
\subsection{Driver Development}
\label{sec:driver}

The \textsf{childdoc} mechanism can also be use for the development
of definition files such as \LaTeX{} styles or classes.
This case differs from the above setup with multiple parts
included by |\include| in that no |\includeonly| should be invoked.
This can be achieved by starting the include file
(before |\ProvidesPackage|) with:
%
\begin{center}
\begin{tabular}{l}
|\input{childdoc.def}|\\
|\childdocforward{|\textit{main}|}|\\
\end{tabular}
\end{center}
%
or alternatively with:
%
\begin{center}
\begin{tabular}{l}
|\input{childdoc.def}|\\
|\childdocby{|\textit{main}|}|\\
\end{tabular}
\end{center}
%
Both forms have slightly different effects as described above.
The main file is prepared as usual, see \secref{sec:include}.

%%%%%%%%%%%%%%%%%%%%%%%%%%%%%%%%%%%%%%%%%%%%%%%%%%%%%%%%%%%%%%%%%%%%%%%%%%%%%%%%
\subsection{Legacy Detection}
\label{sec:detection}

The directive |\childdocmain| in the main file can detect
whether the complete document or merely a child is to be compiled
even without using the directive |\childdocof|.
This method is deprecated because it is less robust
and there is no compelling reason to use it;
it is merely provided for backward compatibility
and it may be removed in future versions.

If the detection mechanism is to be used,
it is mandatory to correctly specify
the filename of the main file as the argument of |\childdocmain|:
%
\begin{center}
\begin{tabular}{l}
|\input{childdoc.def}|\\
|\childdocmain{|\textit{main}|}|\\
\end{tabular}
\end{center}
%
If |\jobname| does not match the argument \textit{main} of |\childdocmain|,
it is assumed that |\jobname| points to the child file to be compiled.
When using |\childdocmain| with the main file specified as argument,
it suffices to start a child file
with just |\input{|\textit{main}|}|
without loading of the package and using |\childdocof|.
If instead all processing is done
with the appropriate \textsf{childdoc} directives,
the argument of \textit{main} of |\childdocmain| can be empty.

An alternative version of the command line processing described
in \secref{sec:commandline} using the detection mechanism reads:
%
\begin{center}
|... -jobname "|\textit{target}|" "|[\textit{flags}]%
[|\def\jobname{|\textit{dest}|}|]|\input{|\textit{main}|}"|
\end{center}

%%%%%%%%%%%%%%%%%%%%%%%%%%%%%%%%%%%%%%%%%%%%%%%%%%%%%%%%%%%%%%%%%%%%%%%%%%%%%%%%
\subsection{Manual Code}
\label{sec:manual}

In case one cannot be certain whether the definitions file |childdoc.def|
is installed on the target \TeX{} distribution
and one prefers not to ship it,
it is conceivable to paste a few relevant commands into the sources.

To that end, drop all statements |\input{childdoc.def}|
and perform the replacements as outlined below.
Instead of |\childdocmain{|\textit{main}|}| add the following code
to the top of the main file:
%
\begin{center}
\begin{tabular}{l}
|\||ifdefined\childdocname\endinput\||fi\newif\ifchilddoc|\\
|\edef\childdocname{\scantokens\expandafter{\jobname\noexpand}}|\\
|\def\childdocmain{|\textit{main}|}\||ifx\childdocmain\childdocname\||else|\\
|\childdoctrue\includeonly{\childdocname}\let\jobname\childdocmain\||fi|\\
\end{tabular}
\end{center}
%
Instead of |\childdocof{|\textit{main}|}| just include the main file
at the top of each child file:
%
\begin{center}
|\input{|\textit{main}|}|
\end{center}
%
A simple redirection |\childdocforward{|\textit{dest}|}| is achieved by:
%
\begin{center}
|\def\jobname{|\textit{dest}|}\input{\jobname}|
\end{center}
%
The redirection with prefix
|\childdocforwardprefix[|\textit{prefix}|]{|\textit{dest}|}|
is accomplished by:
%
\begin{center}
\begin{tabular}{l}
|{\edef\jobname{\scantokens\expandafter{\jobname\noexpand}}|\\
|\def\redirectjob |\textit{prefix}|#1~~~{\gdef\jobname{|\textit{dest}|#1}}|\\
|\expandafter\redirectjob\jobname~~~}\input{\jobname}|
\end{tabular}
\end{center}

In an alternative approach,
child documents can be compiled by a specific command line
without additional code or specific definitions:
%
\begin{center}
|... -jobname "|\textit{target}|" "|[\textit{flags}]%
|\includeonly{|\textit{dest}|}\input{|\textit{main}|}"|
\end{center}
%

%%%%%%%%%%%%%%%%%%%%%%%%%%%%%%%%%%%%%%%%%%%%%%%%%%%%%%%%%%%%%%%%%%%%%%%%%%%%%%%%
%%%%%%%%%%%%%%%%%%%%%%%%%%%%%%%%%%%%%%%%%%%%%%%%%%%%%%%%%%%%%%%%%%%%%%%%%%%%%%%%
\section{Information}

%%%%%%%%%%%%%%%%%%%%%%%%%%%%%%%%%%%%%%%%%%%%%%%%%%%%%%%%%%%%%%%%%%%%%%%%%%%%%%%%
\subsection{Copyright}

Copyright \copyright{} 2017--2018 Niklas Beisert

This work may be distributed and/or modified under the
conditions of the \LaTeX{} Project Public License, either version 1.3
of this license or (at your option) any later version.
The latest version of this license is in
  \url{http://www.latex-project.org/lppl.txt}
and version 1.3 or later is part of all distributions of \LaTeX{}
version 2005/12/01 or later.

This work has the LPPL maintenance status `maintained'.

The Current Maintainer of this work is Niklas Beisert.

This work consists of the files |README.txt|, |childdoc.ins| and |childdoc.dtx|
as well as the derived files |childdoc.def|, |cdocsamp.tex|
with |cdocsch1.tex|, |cdocsch2.tex|, |cdocspt3.tex|, |cdocspt4.tex|,
|cdocsdrf.tex|, |cdocsfn1.tex|, |cdocsfn2.tex|
as well as |childdoc.pdf|.

%%%%%%%%%%%%%%%%%%%%%%%%%%%%%%%%%%%%%%%%%%%%%%%%%%%%%%%%%%%%%%%%%%%%%%%%%%%%%%%%
\subsection{Files and Installation}

The package consists of the files:
%
\begin{center}
\begin{tabular}{ll}
    |README.txt|   & readme file \\
    |childdoc.ins| & installation file \\
    |childdoc.dtx| & source file \\
    |childdoc.def| & definition file \\
    |cdocsamp.tex| & sample main file \\
    |cdocsch1.tex| & sample include file \\
    |cdocsch2.tex| & sample include file \\
    |cdocspt3.tex| & sample part file \\
    |cdocspt4.tex| & sample part file \\
    |cdocsdrf.tex| & sample redirection file \\
    |cdocsfn1.tex| & sample redirection file \\
    |cdocsfn2.tex| & sample redirection file \\
    |childdoc.pdf| & manual
\end{tabular}
\end{center}
%
The distribution consists of the files
|README.txt|, |childdoc.ins| and |childdoc.dtx|.
%
\begin{itemize}
\item
Run (pdf)\LaTeX{} on |childdoc.dtx|
to compile the manual |childdoc.pdf| (this file).
\item
Run \LaTeX{} on |childdoc.ins| to create the definitions file |childdoc.def|
and the sample |cdocsamp.tex| with include files
|cdocsch1.tex|, |cdocsch2.tex|, |cdocspt3.tex|, |cdocspt4.tex|,
|cdocsdrf.tex|, |cdocsfn1.tex|, |cdocsfn2.tex|.
Then copy the file |childdoc.def| to an appropriate directory of your \LaTeX{}
distribution, e.g.\ \textit{texmf-root}|/tex/latex/childdoc|.
\end{itemize}

%%%%%%%%%%%%%%%%%%%%%%%%%%%%%%%%%%%%%%%%%%%%%%%%%%%%%%%%%%%%%%%%%%%%%%%%%%%%%%%%
\subsection{Related CTAN Packages}

There are several other packages which offer a similar functionality:
%
\begin{itemize}
\item
The packages
\href{http://ctan.org/pkg/docmute}{\textsf{docmute}},
\href{http://ctan.org/pkg/includex}{\textsf{includex}} and
\href{http://ctan.org/pkg/standalone}{\textsf{standalone}}
provide commands to include only the document body of
a child file thus allowing both files to be compiled individually.
\item
The packages \href{http://ctan.org/pkg/subdocs}{\textsf{subdocs}}
and \href{http://ctan.org/pkg/subfiles}{\textsf{subfiles}}
provide structures in which the main and child documents can be
encapsulated and allowing them to be compiled individually.
The inclusion mechanism is different from the conventional |\include|.
\item
The package \href{http://ctan.org/pkg/combine}{\textsf{combine}}
is an elaborate solution to combine several documents into one.
\end{itemize}
%
See also the CTAN topic \href{http://ctan.org/topic/subdocs}{\textsf{subdocs}}
for further related packages.
The present package differs from the above solutions in that
a document structure constructed with the conventional |\include| mechanism
just needs two extra commands at the top of every file
such that all constituent files can be compiled individually.

%%%%%%%%%%%%%%%%%%%%%%%%%%%%%%%%%%%%%%%%%%%%%%%%%%%%%%%%%%%%%%%%%%%%%%%%%%%%%%%%
%\subsection{Feature Suggestions}
%
%The following is a list of features which may be useful for future
%versions of this package:
%%
%\begin{itemize}
%\item
%\ldots
%\end{itemize}

%%%%%%%%%%%%%%%%%%%%%%%%%%%%%%%%%%%%%%%%%%%%%%%%%%%%%%%%%%%%%%%%%%%%%%%%%%%%%%%%
\subsection{Revision History}

%%%%%%%%%%%%%%%%%%%%%%%%%%%%%%%%%%%%%%%%
\paragraph{v2.0:} 2018/12/30

\begin{itemize}
\item
immediate forward processing
\item
added |\childdocby| mechanism
\item
manual restructured
\end{itemize}

%%%%%%%%%%%%%%%%%%%%%%%%%%%%%%%%%%%%%%%%
\paragraph{v1.6:} 2018/01/17

\begin{itemize}
\item
application for development of include files
\item
corrections to manual
\end{itemize}

%%%%%%%%%%%%%%%%%%%%%%%%%%%%%%%%%%%%%%%%
\paragraph{v1.5:} 2017/05/21

\begin{itemize}
\item
more complete structuring introduced
\item
|\childdocof| introduced
\item
|\childdoc| renamed to |\childdocmain|
\item
|\childredirect| renamed to |\childdocforward| and |\childdocforwardprefix|
and functionality expanded
\end{itemize}

%%%%%%%%%%%%%%%%%%%%%%%%%%%%%%%%%%%%%%%%
\paragraph{v1.0:} 2017/04/27

\begin{itemize}
\item
manual and install package
\item
first version published on CTAN
\end{itemize}

%%%%%%%%%%%%%%%%%%%%%%%%%%%%%%%%%%%%%%%%
\paragraph{v0.6:} 2017/04/26

\begin{itemize}
\item
redirection mechanism added
\end{itemize}

%%%%%%%%%%%%%%%%%%%%%%%%%%%%%%%%%%%%%%%%
\paragraph{v0.5:} 2017/04/26

\begin{itemize}
\item
functionality in definition file
\end{itemize}


%%%%%%%%%%%%%%%%%%%%%%%%%%%%%%%%%%%%%%%%%%%%%%%%%%%%%%%%%%%%%%%%%%%%%%%%%%%%%%%%
%%%%%%%%%%%%%%%%%%%%%%%%%%%%%%%%%%%%%%%%%%%%%%%%%%%%%%%%%%%%%%%%%%%%%%%%%%%%%%%%
%%%%%%%%%%%%%%%%%%%%%%%%%%%%%%%%%%%%%%%%%%%%%%%%%%%%%%%%%%%%%%%%%%%%%%%%%%%%%%%%
\appendix

\settowidth\MacroIndent{\rmfamily\scriptsize 000\ }

 \DocInput{childdoc.dtx}

\end{document}
%</driver>
% \fi
%
% %%%%%%%%%%%%%%%%%%%%%%%%%%%%%%%%%%%%%%%%%%%%%%%%%%%%%%%%%%%%%%%%%%%%%%%%%%%%%%
% %%%%%%%%%%%%%%%%%%%%%%%%%%%%%%%%%%%%%%%%%%%%%%%%%%%%%%%%%%%%%%%%%%%%%%%%%%%%%%
% \section{Sample}
%\iffalse
%<*samplemain>
%\fi
%
% The following presents a sample document
% with two chapters, two parts, a title page,
% a compile flag as well as three forwarding files to set the flag.
% It consists of eight |.tex| files:
% \begin{center}
% \begin{tabular}{ll}
% |cdocsamp.tex|&main file\\
% |cdocsch1.tex|&include file for chapter 1\\
% |cdocsch2.tex|&include file for chapter 2\\
% |cdocspt3.tex|&include file for part 3\\
% |cdocspt4.tex|&include file for part 4\\
% |cdocsdrf.tex|&forwarding file for main file in draft mode\\
% |cdocsfi1.tex|&forwarding file for final version of chapter 1\\
% |cdocsfi2.tex|&forwarding file for final version of chapter 2\\
% \end{tabular}
% \end{center}
% Each of the eight files can be compiled directly by the \LaTeX{} compiler.
%
% %%%%%%%%%%%%%%%%%%%%%%%%%%%%%%%%%%%%%%
% \paragraph{Main File.}
%
% The main file is called |cdocsamp.tex|.
%
% Load the \textsf{childdoc} definitions and
% declare the filename for the main document:
%    \begin{macrocode}
\input{childdoc.def}
\childdocmain{}
%    \end{macrocode}

% Optional override for |\version| flag:
%    \begin{macrocode}
%%\ifchilddoc\else\providecommand{\version}{draft}\fi
%    \end{macrocode}

% Define the default values for the |\version| flag
% (|final| for the main file and |draft| for childs):
%    \begin{macrocode}
\ifchilddoc
\providecommand{\version}{draft}
\else
\providecommand{\version}{final}
\fi
%    \end{macrocode}

% Load the standard document class:
%    \begin{macrocode}
\documentclass[12pt]{article}
%    \end{macrocode}

% Start the document body:
%    \begin{macrocode}
\begin{document}
%    \end{macrocode}

% Declare a title page.
% Print title, part of document being processed and version flag:
%    \begin{macrocode}
\addtocounter{page}{-1}
\begin{center}
{\LARGE\bfseries{}childdoc example\par}
\vspace{1cm}
\ifchilddoc
\ifchilddocmanual part\else chapter\fi:
`\childdocname' of `\childdocjob'\par
\else
main document: `\childdocjob'\par
\fi
version: \version\par
\end{center}
\newpage
%    \end{macrocode}

% Manually include selected file,
% otherwise process as usual:
%    \begin{macrocode}
\ifchilddocmanual
\section*{part `\childdocname'}
\input{\childdocname}
\else
%    \end{macrocode}

% Include the two chapters:
%    \begin{macrocode}
\include{cdocsch1}
\include{cdocsch2}
%    \end{macrocode}

% Include the two parts unless only chapters should be displayed:
%    \begin{macrocode}
\ifchilddoc\else
\section{part three}
\input{cdocspt3}
\section{part four}
\input{cdocspt4}
\fi
%    \end{macrocode}

% Process as usual until here:
%    \begin{macrocode}
\fi
%    \end{macrocode}

% End of document body:
%    \begin{macrocode}
\end{document}
%    \end{macrocode}
%\iffalse
%</samplemain>
%\fi
%
% %%%%%%%%%%%%%%%%%%%%%%%%%%%%%%%%%%%%%%
% \paragraph{Chapter Include Files.}
%
% The include files are called |cdocsch1.tex| and |cdocsch2.tex|.
%
%\iffalse
%<*samplechap1|samplechap2>
%\fi

% Optional override for |\version| flag:
%    \begin{macrocode}
%%\providecommand{\version}{final}
%    \end{macrocode}

% Include the main document:
%    \begin{macrocode}
\input{childdoc.def}
\childdocof{cdocsamp}
%    \end{macrocode}

%\iffalse
%</samplechap1|samplechap2>
%\fi
%
%\iffalse
%<*samplechap1>
%\fi
% Some text for chapter 1:
%    \begin{macrocode}
\section{one}
some text in chapter one
%    \end{macrocode}

%\iffalse
%</samplechap1>
%\fi
% Some text for chapter 2:
%\iffalse
%<*samplechap2>
%\fi
%    \begin{macrocode}
\section{two}
more text in chapter two
%    \end{macrocode}

%\iffalse
%</samplechap2>
%\fi
%
% %%%%%%%%%%%%%%%%%%%%%%%%%%%%%%%%%%%%%%
% \paragraph{Part Include Files.}
%
% The include files are called |cdocspt3.tex| and |cdocspt4.tex|.
%
%\iffalse
%<*samplepart3|samplepart4>
%\fi

% Optional override for |\version| flag:
%    \begin{macrocode}
%%\providecommand{\version}{final}
%    \end{macrocode}

% Include the main document:
%    \begin{macrocode}
\input{childdoc.def}
\childdocby{cdocsamp}
%    \end{macrocode}

%\iffalse
%</samplepart3|samplepart4>
%\fi
%
%\iffalse
%<*samplepart3>
%\fi
% Some text for part 3:
%    \begin{macrocode}
some text in part three
%    \end{macrocode}

%\iffalse
%</samplepart3>
%\fi
% Some text for part 4:
%\iffalse
%<*samplepart4>
%\fi
%    \begin{macrocode}
more text in part four
%    \end{macrocode}

%\iffalse
%</samplepart4>
%\fi
%
% %%%%%%%%%%%%%%%%%%%%%%%%%%%%%%%%%%%%%%
% \paragraph{Forwarding for a Complete Draft.}
%
% The following forwarding file |cdocsdrf.tex|
% compiles the main document in draft mode:
%\iffalse
%<*sampledraft>
%\fi
%    \begin{macrocode}
\def\version{draft}
\input{childdoc.def}
\childdocforward{cdocsamp}
%    \end{macrocode}

%\iffalse
%</sampledraft>
%\fi
%
% %%%%%%%%%%%%%%%%%%%%%%%%%%%%%%%%%%%%%%
% \paragraph{Forwarding for Final Version of the Chapters.}
%
% The following forwarding files |cdocsfn1.tex| and |cdocsfn2.tex|
% (with identical content)
% compile the final versions of the child documents
% |cdocsch1.tex| and |cdocsch2.tex|, respectively:
%\iffalse
%<*samplefinal>
%\fi
%    \begin{macrocode}
\def\version{final}
\input{childdoc.def}
\childdocforwardprefix[cdocsamp]{cdocsfn}{cdocsch}
%    \end{macrocode}

%\iffalse
%</samplefinal>
%\fi
%
% %%%%%%%%%%%%%%%%%%%%%%%%%%%%%%%%%%%%%%
% \paragraph{Command Line Processing.}
%
% The following three command lines generate the output files
% |cdocscld|, |cdocscl1| and |cdocscl2|
% which should be identical to
% |cdocsdrf|, |cdocsch1| and |cdocsfn2|, respectively:
% \begin{center}
% \begin{tabular}{l}
% |latex -jobname cdocscld \|\\
% |  "\def\version{draft}\input{childdoc.def}\childdocforward{cdocsamp}"|\\
% |latex -jobname cdocscl1 \|\\
% |  "\input{childdoc.def}\childdocforward[cdocsamp]{cdocsch1}"|\\
% |latex -jobname cdocscl2 \|\\
% |  "\def\version{final}\input{childdoc.def}\childdocforward{cdocsch2}"|
% \end{tabular}
% \end{center}
% Note that the trailing backslash on each first line
% merely continues the input to the second line
% (for convenient cut ant paste).
% Furthermore, the command |latex| can be replaced by any
% of its alternative versions such as |pdflatex|.
%
% %%%%%%%%%%%%%%%%%%%%%%%%%%%%%%%%%%%%%%%%%%%%%%%%%%%%%%%%%%%%%%%%%%%%%%%%%%%%%%
% %%%%%%%%%%%%%%%%%%%%%%%%%%%%%%%%%%%%%%%%%%%%%%%%%%%%%%%%%%%%%%%%%%%%%%%%%%%%%%
% \section{Implementation}
%\iffalse
%<*package>
%\fi
%
% This section describes the definitions file |childdoc.def|.

% The definitions cannot be loaded using |\usepackage| or |\RequirePackage|
% which has a mechanism to prevent loading a style file more than once.
% When loading the definitions by means of |\input|
% multiple instances have to be prevented manually:
%\iffalse
%This code needs to be before the `\ProvidesFile' directive
%which is defined at the beginning of this file.
%Therefore it is also placed there and commented out here.
%</package>
%<*discard>
%\fi
%    \begin{macrocode}
\ifdefined\childdocmain\endinput\fi
%    \end{macrocode}
%\iffalse
%</discard>
%<*package>
%\fi
%
% \macro{\ifchilddoc}
% \macro{\ifchilddocmanual}
% The conditional |\ifchilddoc| tells whether a
% child (true) or main (false) document is being compiled.
% The conditional |\ifchilddocmanual| tells whether
% the |\includeonly| mechanism is used (false) or
% the selection of child files must be performed manually (true).
% The definitions initialise to false:
%    \begin{macrocode}
\newif\ifchilddoc
\newif\ifchilddocmanual
%    \end{macrocode}

% \macro{\childdocname}
% \macro{\childdocjob}
% The macro |\childdocname| stores the name of the main document
% to be compiled. The macro |\childdocjob| stores the name of
% the document on which the \LaTeX{} compiler was originally invoked.
% The content of |\jobname| cannot be compared
% to filenames specified in the source due to different catcodes.
% The following code rescans |\jobname|, stores the result
% in |\childdocname| and saves a copy in |\childdocjob|:
%    \begin{macrocode}
\edef\childdocname{\scantokens\expandafter{\jobname\noexpand}}
\let\childdocjob\childdocname
%    \end{macrocode}

% \macro{\childdocdisable}
% The macro |\childdocdisable| prevents the main file
% from being processed more than once.
% At this stage, the main document command |\childdocmain|
% is assumed to be called once again where it should do nothing.
% Any subsequent call to it should prevent
% a secondary processing of the main document
% It overwrites the forwarding commands
% |\childdocof| and |\childdocforward|
% with empty macros to prevent further inclusions of the main document:
%    \begin{macrocode}
\newcommand{\childdocdisable}
{
  \renewcommand{\childdocmain}[1]{\renewcommand{\childdocmain}[1]{\endinput}}
  \renewcommand{\childdocof}[1]{}
  \renewcommand{\childdocby}[2][]{}
  \renewcommand{\childdocforward}[2][]{}
  \renewcommand{\childdocdisable}{}
}
%    \end{macrocode}

% \macro{\childdocmain}
% The macro |\childdocmain| is to be called at the top of the main file
% with nothing or the main filename (without extension) as argument.
% First, it breaks loops.
% If the argument is not empty and does not match |\childdocname|
% (which is set by the first inclusion of |childdoc.def|),
% |\ifchilddoc| is set to true, |\includeonly| is applied to the child file
% and |\jobname| is set to the main file
% (for proper handling of |.aux| files):
%    \begin{macrocode}
\newcommand{\childdocmain}[1]
{
  \childdocdisable\childdocmain{}
  \if?#1?\else
    \begingroup
      \def\childdoctmp{#1}
      \ifx\childdoctmp\childdocname
        \def\childdoctmp{}
      \else
        \def\childdoctmp
        {
          \childdoctrue
          \includeonly{\childdocname}
          \def\childdocjob{#1}
          \def\jobname{#1}
        }
      \fi
      \expandafter
    \endgroup
    \childdoctmp
  \fi
}
%    \end{macrocode}

% \macro{\childdocof}
% The command |\childdocof| redirects
% compilation to the main file |#1|.
%    \begin{macrocode}
\newcommand{\childdocof}[1]
{
  \childdocdisable
  \childdoctrue
  \includeonly{\childdocname}
  \def\jobname{#1}
  \def\childdocjob{#1}
  \input{#1}
}
%    \end{macrocode}

% \macro{\childdocby}
% The command |\childdocby| ....
%    \begin{macrocode}
\newcommand{\childdocby}[2][]
{
  \childdocdisable
  \childdoctrue
  \childdocmanualtrue
  \if?#1?\else
    \def\jobname{#2}
  \fi
  \def\childdocjob{#2}
  \input{#2}
  \endinput
}
%    \end{macrocode}

% \macro{\childdocforward}
% The command |\childdocforward| redirects
% compilation to the main file or
% (if the optional argument is given) a child file.
% Parameters are set as if the main file
% or a child file starting with |\childdocof| was compiled.
% Then compilation is handed over to the main file:
%    \begin{macrocode}
\newcommand{\childdocforward}[2][]
{
  \begingroup
    \if?#1?
      \def\childdoctmp
      {
        \def\childdocname{#2}
        \def\childdocjob{#2}
        \def\jobname{#2}
        \input{#2}
        \endinput
      }
    \else
      \def\childdoctmp
      {
        \childdocdisable
        \def\childdocname{#2}
        \childdoctrue
        \includeonly{#2}
        \def\childdocjob{#1}
        \def\jobname{#1}
        \input{#1}
        \endinput
      }
    \fi
    \expandafter
  \endgroup
  \childdoctmp
}
%    \end{macrocode}

% \macro{\childdocforwardprefix}
% The command |\childdocforwardprefix| redirects
% compilation to the main or a child file by means of a pattern.
% The prefix |#1| in the current filename is replaced by |#2|
% and the suffix of the current filename is kept
% (it is assumed that the filename does not contain the substring `|~~~|'
% which is used as a delimiter).
% Compilation is handed over to the new file by |\childdocforward|:
%    \begin{macrocode}
\newcommand{\childdocforwardprefix}[3][]
{
  \begingroup
    \def\childdocextract #2##1~~~{\def\childdoctmp{\childdocforward[#1]{#3##1}}}
    \expandafter\childdocextract\childdocname~~~
    \expandafter
  \endgroup
  \childdoctmp
}
%    \end{macrocode}

% \macro{\childdoc}
% The deprecated macro |\childdoc| is a legacy version of |\childdocmain|:
%    \begin{macrocode}
\newcommand{\childdoc}{\childdocmain}
%    \end{macrocode}

% \macro{\childdocredirect}
% The deprecated macro |\childdocredirect| is a legacy version
% of |\childdocforward| and |\childdocforwardprefix|:
%    \begin{macrocode}
\newcommand{\childdocredirect}[2][]
{
  \begingroup
    \if?#1?
      \def\childdoctmp{\childdocforward{#2}}
    \else
      \def\childdoctmp{\childdocforwardprefix{#1}{#2}}
    \fi
    \expandafter
  \endgroup
  \childdoctmp
}
%    \end{macrocode}

%\iffalse
%</package>
%\fi
%
\endinput
|\\
|\childdocforward{|\textit{main}|}|\\
\end{tabular}
\end{center}
%
or alternatively with:
%
\begin{center}
\begin{tabular}{l}
|% \iffalse
%
% childdoc.dtx Copyright (C) 2017-2018 Niklas Beisert
%
% This work may be distributed and/or modified under the
% conditions of the LaTeX Project Public License, either version 1.3
% of this license or (at your option) any later version.
% The latest version of this license is in
%   http://www.latex-project.org/lppl.txt
% and version 1.3 or later is part of all distributions of LaTeX
% version 2005/12/01 or later.
%
% This work has the LPPL maintenance status `maintained'.
%
% The Current Maintainer of this work is Niklas Beisert.
%
% This work consists of the files childdoc.dtx and childdoc.ins
% and the derived files childdoc.def and cdocsamp.tex with
% cdocsch1.tex, cdocsch2.tex, cdocsdrf.tex, cdocsfn1.tex, cdocsfn2.tex.
%
%<package>\ifdefined\childdocmain\endinput\fi
%<package>\ProvidesFile{childdoc.def}[2018/12/30 v2.0 child document driver]
%<samplemain>\ProvidesFile{cdocsamp.tex}[2018/12/30 v2.0 sample for childdoc]
%<*driver>
%\ProvidesFile{childdoc.drv}[2018/12/30 v2.0 childdoc reference manual file]
\PassOptionsToClass{10pt,a4paper}{article}
\documentclass{ltxdoc}

\usepackage[margin=35mm]{geometry}
\usepackage{hyperref}
\usepackage{hyperxmp}
\usepackage[usenames]{color}

\hypersetup{colorlinks=true}
\hypersetup{pdfstartview=FitH}
\hypersetup{pdfpagemode=UseNone}
\hypersetup{pdfsource={}}
\hypersetup{pdflang={en-UK}}
\hypersetup{pdfcopyright={Copyright 2017-2018 Niklas Beisert.
  This work may be distributed and/or modified under the
  conditions of the LaTeX Project Public License, either version 1.3
  of this license or (at your option) any later version.}}
\hypersetup{pdflicenseurl={http://www.latex-project.org/lppl.txt}}
\hypersetup{pdfcontactaddress={ETH Zurich, ITP, HIT K,
  Wolfgang-Pauli-Strasse 27}}
\hypersetup{pdfcontactpostcode={8093}}
\hypersetup{pdfcontactcity={Zurich}}
\hypersetup{pdfcontactcountry={Switzerland}}
\hypersetup{pdfcontactemail={nbeisert@itp.phys.ethz.ch}}
\hypersetup{pdfcontacturl={http://people.phys.ethz.ch/\xmptilde nbeisert/}}

\newcommand{\secref}[1]{\hyperref[#1]{section \ref*{#1}}}

\parskip1ex
\parindent0pt
\let\olditemize\itemize
\def\itemize{\olditemize\parskip0pt}

\begin{document}

\title{The \textsf{childdoc} Package}
\hypersetup{pdftitle={The childdoc Package}}
\author{Niklas Beisert\\[2ex]
  Institut f\"ur Theoretische Physik\\
  Eidgen\"ossische Technische Hochschule Z\"urich\\
  Wolfgang-Pauli-Strasse 27, 8093 Z\"urich, Switzerland\\[1ex]
  \href{mailto:nbeisert@itp.phys.ethz.ch}
  {\texttt{nbeisert@itp.phys.ethz.ch}}}
\hypersetup{pdfauthor={Niklas Beisert}}
\hypersetup{pdfsubject={Manual for the LaTeX2e Package childdoc}}
\date{30 December 2018, \textsf{v2.0}}
\maketitle

\begin{abstract}\noindent
\textsf{childdoc} is a \LaTeXe{} package
that enables the direct compilation
of document sections included by |\include|
to individual files.
\end{abstract}

\begingroup
\parskip0ex
\tableofcontents
\endgroup

%%%%%%%%%%%%%%%%%%%%%%%%%%%%%%%%%%%%%%%%%%%%%%%%%%%%%%%%%%%%%%%%%%%%%%%%%%%%%%%%
%%%%%%%%%%%%%%%%%%%%%%%%%%%%%%%%%%%%%%%%%%%%%%%%%%%%%%%%%%%%%%%%%%%%%%%%%%%%%%%%
\section{Introduction}

\LaTeX{} provides a mechanism to structure a large document (such as a book)
into a main file and several child files (containing the chapters)
using the |\include| command.
This mechanism is beneficial for documents
which span hundreds of pages in order to
make the source file(s) more manageable.
Moreover, compilation can be restricted to
selected child files by means of the |\includeonly| command.
The latter feature can be used to reduce the compilation time while editing
(this was significantly more useful in the earlier days of \LaTeX{})
or to generate a smaller document which is easier to navigate.
Another application of |\includeonly| is to generate
documents consisting of selected parts of the complete document.

However, there are a few drawbacks of the plain |\include| mechanism:
\begin{itemize}
\item
The child files cannot be compiled on their own,
they can only be compiled via the main file.
A naive editing environment
(such as a text editor with an option
to have the current file processed by \LaTeX)
may require one to switch to the main file before compiling;
attempting to compile the child file produces errors.
\item
The main file must be modified (each time)
to adjust the |\includeonly| command
to the present needs. This easily leaves the main file in a messy state.
\item
The generated document will always carry the filename
of the main document. This is inconvenient if
several child files are to be compiled and
to be kept for distribution.
\end{itemize}

The present package provides a simple interface
to make child files individually compilable by \LaTeX{}.
Compiling a child file then has the same effect as compiling
the main file with an |\includeonly| command
to select the appropriate child.
Moreover the generated document will carry the name of the child
rather than the main file.
This resolves all three above issues.

This feature is meant to make the editing of books,
thesis documents and lecture notes somewhat more convenient.
However, the package can also be used efficiently for
composing a series of documents (such as exercise sheets)
which are typically distributed individually.
It then assists the author in generating the individual documents
(potentially in different versions)
as well as a document containing the collected series.
Another application is in developing style files
or other kinds of included material
where compilation of the style file could redirect
to a sample or test file.

%%%%%%%%%%%%%%%%%%%%%%%%%%%%%%%%%%%%%%%%%%%%%%%%%%%%%%%%%%%%%%%%%%%%%%%%%%%%%%%%
%%%%%%%%%%%%%%%%%%%%%%%%%%%%%%%%%%%%%%%%%%%%%%%%%%%%%%%%%%%%%%%%%%%%%%%%%%%%%%%%
\section{Usage}

First of all, the package \textsf{childdoc} is \emph{not} a standard
\LaTeXe{} |.sty| style file! Therefore it needs to be invoked in
a non-standard way.

%%%%%%%%%%%%%%%%%%%%%%%%%%%%%%%%%%%%%%%%%%%%%%%%%%%%%%%%%%%%%%%%%%%%%%%%%%%%%%%%
\subsection{Included Files}
\label{sec:include}

%%%%%%%%%%%%%%%%%%%%%%%%%%%%%%%%%%%%%%%%
\DescribeMacro{\childdocmain}
To use the package, add the commands
\begin{center}
\begin{tabular}{l}
|\input{childdoc.def}|\\
|\childdocmain{}|\\
\end{tabular}
\end{center}
at the very top of the main \LaTeX{} file,
in particular \emph{before} the |\documentclass| statement!
The argument of |\childdocmain| should be left empty
(but it must be present).

%%%%%%%%%%%%%%%%%%%%%%%%%%%%%%%%%%%%%%%%
\DescribeMacro{\childdocof}
Furthermore, add the commands
\begin{center}
\begin{tabular}{l}
|\input{childdoc.def}|\\
|\childdocof{|\textit{main}|}|\\
\end{tabular}
\end{center}
at the top of every child file \textit{child}
which is included by |\include{|\textit{child}|}|
from within the main file
(or at least for those files to be compiled individually).
The argument \textit{main} must be the filename of the main file.

There are a couple of
considerations in setting up the main and child documents:

%%%%%%%%%%%%%%%%%%%%%%%%%%%%%%%%%%%%%%%%
\paragraph{Restrictions.}

Please note the following restrictions:
\begin{itemize}
\item
|\childdocmain| must be called with one argument \textit{main}
to ensure compatibility with earlier version of the package.
It must either be empty (|\childdocmain{}|)
or precisely match the filename of the main file in which it is specified.
See \secref{sec:detection} for further information.
\item
The filename \textit{main} must be specified without the |.tex| extension.
\item
The filename \textit{main} is case sensitive
(even in case-insensitive file systems)
due to internal string comparison.
\item
The argument \textit{main} should be fully expanded, it cannot be a macro.
\item
Subdirectories and special characters should be avoided in filenames.
\item
The command |\childdocmain{|\textit{main}|}| must be followed by a whitespace.
It should not be followed immediately by another command
or by a comment mark `|%|'.
This is because the \TeX{} parser reads the token immediately following
the argument of |\childdocmain| and puts it
at the beginning of every child section;
however, a white\-space is ignored.
\end{itemize}

%%%%%%%%%%%%%%%%%%%%%%%%%%%%%%%%%%%%%%%%
\paragraph{Content of Main File.}

It is advisable to place all content in the child files included by |\include|.
Any output contained in the main file will appear in all child documents
unless suppressed manually;
it cannot be suppressed automatically by the |\includeonly| directive
and thus should normally be avoided.
A method to include some content in the main file
by means of conditional processing is described in \secref{sec:conditional}.

%%%%%%%%%%%%%%%%%%%%%%%%%%%%%%%%%%%%%%%%
\paragraph{Page Numbering.}

When only a part of the document is compiled,
the appropriate numbering of pages
(as well as other status parameters)
is determined from the |.aux| files.
The latter contain information from previous passes.
However this information needs to propagate through
all intermediate child documents.
Therefore the page numbering in child documents may well
be inconsistent until the complete document is compiled at least once.

A useful (if unconventional) way to always ensure a consistent
page numbering is to restart the numbering in each child document
and denote the pages by `\textit{child}|.|\textit{page}'
where \textit{child} represents the chapter/section number of the child file.
This can be achieved by the command
|\numberwithin{page}{|\textit{child}|}|
of the \textsf{amsmath} package
where \textit{child} can be |chapter| or |section|
depending on the chosen structuring.
Alternatively, one can modify the macro |\thepage| appropriately
and reset the counter |page| at the start of each child file.

%%%%%%%%%%%%%%%%%%%%%%%%%%%%%%%%%%%%%%%%%%%%%%%%%%%%%%%%%%%%%%%%%%%%%%%%%%%%%%%%
\subsection{Conditional Processing}
\label{sec:conditional}

The package provides a mechanism to compile different versions
of a document. To customise the versions further some conditional processing
can come in handy to distinguish which version is being compiled.
The package provides two macros to describe the compilation context:

%%%%%%%%%%%%%%%%%%%%%%%%%%%%%%%%%%%%%%%%
\DescribeMacro{\ifchilddoc}
The conditional |\ifchilddoc| distinguishes between the compilation of
child documents and the main document:
%
\begin{center}
|\ifchilddoc |\textit{child-code}| |[|\||else |\textit{main-code}]| \||fi|
\end{center}

%%%%%%%%%%%%%%%%%%%%%%%%%%%%%%%%%%%%%%%%
\DescribeMacro{\childdocname}
\DescribeMacro{\childdocjob}
The macro |\childdocname| contains the filename (without extension)
of the main or child file being processed.
Note that |\childdocjob| will always contain the name of the main file.

%%%%%%%%%%%%%%%%%%%%%%%%%%%%%%%%%%%%%%%%
\paragraph{Title Page.}

Conditional processing can be used to include a title or banner page
in the main document when proper precautions are taken.
Importantly, the code in the main file should ensure that the page counter
(as well as other status parameters which are stored in the |.aux| files)
takes the same value after the conditional processing.
Otherwise the page numbers may take divergent values
depending on which part is compiled.

For example, a title page could be declared by:
%
\begin{center}
\begin{tabular}{l}
|\ifchilddoc\||else|\\
|\addtocounter{page}{-1}|\\
\textit{code for title page}\\
|\newpage|\\
|\||fi|
\end{tabular}
\end{center}
%
A banner page for the child documents can be generated by:
%
\begin{center}
\begin{tabular}{l}
|\ifchilddoc|\\
|\addtocounter{page}{-1}|\\
\textit{code for banner page}\\
|\newpage|\\
|\||fi|
\end{tabular}
\end{center}
%
Here one could write a message such as:
\begin{center}
|This is the part \childdocname{} of \childdocjob{}.|
\end{center}

%%%%%%%%%%%%%%%%%%%%%%%%%%%%%%%%%%%%%%%%%%%%%%%%%%%%%%%%%%%%%%%%%%%%%%%%%%%%%%%%
\subsection{Flags}
\label{sec:flags}

The package makes it easy to generate different versions
of the main or child documents.
To this end compilation flags can be defined
and assigned different default values.
They will be particularly useful in conjunction
with the forwarding mechanism described in \secref{sec:forward}.

For example, it may be useful to have a flag |\version|
which can be set to |draft| or |final|.
The document source will contain some conditional code
depending on the value of |\version|.
Suppose further, the flag should default to |final| for the main file
and to |draft| for child files
which is a natural assignment for editing the document.
This is achieved by placing the following code
in the preamble of the main document
(below the |\childdocmain| directive):
%
\begin{center}
\begin{tabular}{l}
|\ifchilddoc|\\
|\providecommand{\version}{draft}|\\
|\||else|\\
|\providecommand{\version}{final}|\\
|\||fi|
\end{tabular}
\end{center}
%
The definition by |\providecommand| makes sure
that previous definitions are not overwritten.
Further statements |\providecommand{\version}{...}|
can thus be added before the above code to override it.

For the main file, one might add a line
(between |\childdocmain| and the above block)
%
\begin{center}
|%\ifchilddoc\||else\providecommand{\version}{draft}\||fi|
\end{center}
%
which can be uncommented to produce a draft version.
Likewise one can add a line to the very top of a child file
(above the |\childdocof{|\textit{main}|}| directive)
%
\begin{center}
|%\providecommand{\version}{final}|
\end{center}
%
which can be uncommented to produce the final version of this child document.

%%%%%%%%%%%%%%%%%%%%%%%%%%%%%%%%%%%%%%%%%%%%%%%%%%%%%%%%%%%%%%%%%%%%%%%%%%%%%%%%
\subsection{Forwarding}
\label{sec:forward}

Different versions of the main or child documents
using compilation flags as described in \secref{sec:flags}
can be (permanently) stored in different files
for convenient compilation, viewing and distribution.
To this end, the package defines a command
to pass on compilation to a different file:

%%%%%%%%%%%%%%%%%%%%%%%%%%%%%%%%%%%%%%%%
\DescribeMacro{\childdocforward}
The command |\childdocforward| redirects processing to
another source file:
%
\begin{center}
\begin{tabular}{l}
|\input{childdoc.def}|\\
|\childdocforward[|\textit{main}|]{|\textit{dest}|}|\\
\end{tabular}
\end{center}
%
The argument \textit{dest} is the destination file
(without extension).
It should be the main file or one of the child files.
Note that further \textsf{childdoc} directives
such as |\childdocof| and |\childdocforward|
in the indicated file will be processed in this form.
The optional argument \textit{main}
passes on directly to the main file \textit{main}
while pretending to compile the child \textit{dest}.
This form behaves as if \textit{dest}
issues |\childdocof{|\textit{main}|}| right away,
and no further \textsf{childdoc} directives will be processed.

%%%%%%%%%%%%%%%%%%%%%%%%%%%%%%%%%%%%%%%%
\DescribeMacro{\...prefix}
In the alternative form |\childdocforwardprefix|,
%
\begin{center}
\begin{tabular}{l}
|\input{childdoc.def}|\\
|\childdocforwardprefix[|\textit{main}|]{|\textit{prefix}|}{|\textit{dest}|}|
\end{tabular}
\end{center}
%
the destination file is determined by a pattern
depending on the current file:
To make this work, the current file must be called
`{\textit{prefix}\hspace{0.2em}\textit{suffix}}'
with \textit{prefix} matching precisely the argument.
Processing is then passed on to the file
`{\textit{dest}\hspace{0.2em}\textit{suffix}}'.
Surely, the same effect is achieved by
directly specifying the
argument `{\textit{dest}\hspace{0.2em}\textit{suffix}}'
in the first form.
However, that requires to set up a different file
for each child. With the alternative form of the command
all these files can have exactly the same content
which simplifies setting them up and maintaining them.

For example, the following file |draft.tex|
with a compilation flag |\version| as described in \secref{sec:flags}
compiles the main document as a draft:
%
\begin{center}
\begin{tabular}{l}
|\def\version{draft}|\\
|\input{childdoc.def}|\\
|\childdocforward{|\textit{main}|}|
\end{tabular}
\end{center}
%
Likewise, the following files |final|\textit{nn}|.tex|
compile the final version of the child document
|child|\textit{nn}|.tex|:
%
\begin{center}
\begin{tabular}{l}
|\def\version{final}|\\
|\input{childdoc.def}|\\
|\childdocforwardprefix{final}{child}|
\end{tabular}
\end{center}
%

Note that when several versions of a main file and/or of each child file
are to be generated, it may be convenient to set up a |Makefile| or
shell script to automatise the process.

%%%%%%%%%%%%%%%%%%%%%%%%%%%%%%%%%%%%%%%%%%%%%%%%%%%%%%%%%%%%%%%%%%%%%%%%%%%%%%%%
\subsection{Command Line Processing}
\label{sec:commandline}

The effect of redirection files can also be achieved by invoking
the \LaTeX{} compiler with a more elaborate command line.
Most conveniently this should be done as part
of a shell script or a |Makefile|.

When using \textsf{childdoc} in the main file, the following
command lines effectively perform a redirection
(note that depending on the shell being used,
backslashes may have to be doubled: `|\|' $\to$ `|\\|'):
%
\begin{center}
|... -jobname "|\textit{target}|" |\\|"|[\textit{flags}]%
|\input{childdoc.def}\childdocforward[|\textit{main}|]{|\textit{dest}|}"|
\end{center}
%
Here \textit{target} is the name of the output file,
\textit{main} is the name of the main file
and \textit{dest} is the name of the main or child file to be processed
(all filenames without extensions).
The optional argument \textit{main} can be omitted
if \textit{main} matches \textit{dest}.
Optionally, compilation \textit{flags} can be defined via |\def| commands.
This command line makes the \TeX{} engine believe
it is compiling the file \textit{target}
whose content is specified as the latter parameter.
The provided code then forwards the processing to
\textit{main} or \textit{dest} as described in \secref{sec:forward}.

%%%%%%%%%%%%%%%%%%%%%%%%%%%%%%%%%%%%%%%%%%%%%%%%%%%%%%%%%%%%%%%%%%%%%%%%%%%%%%%%
\subsection{Include by Input}
\label{sec:input}

Including child documents by |\include| has some restrictions by design.
Most notably, the content of a child document always occupies
its own set of pages; pages cannot be shared between child documents.
Usually, this behaviour makes perfect sense
because each child document contain an essential part of the document.
However, in some situations it may be desirable to compose
a document from a collection of parts
without having mandatory page breaks between then.
For this case, the package
provides a mechanism to include parts
by |\input| which can also be processed individually.
However, by construction this mechanism
requires manual handling of the content to be output.

%%%%%%%%%%%%%%%%%%%%%%%%%%%%%%%%%%%%%%%%
\DescribeMacro{\ifchilddocmanual}
The main file should be prepared as usual, see \secref{sec:include}.
However, the document body must make a distinction
between processing of an individual part and of the main document, e.g.:
%
\begin{center}
\begin{tabular}{l}
|\ifchilddocmanual|\\
|\input{\childdocname}|\\
|\||else|\\
\textit{document body with }|\input{|\textit{part}|}|\\
|\||fi|
\end{tabular}
\end{center}
%
The conditional |\ifchilddocmanual| is true whenever
a part to be included by |\input| is being compiled,
and the name of the part is stored in |\childdocname|.

%%%%%%%%%%%%%%%%%%%%%%%%%%%%%%%%%%%%%%%%
\DescribeMacro{\childdocby}
Each part to be included by |\input| should start with:
%
\begin{center}
\begin{tabular}{l}
|\input{childdoc.def}|\\
|\childdocby{|\textit{main}|}|\\
\end{tabular}
\end{center}
%
The directive |\childdocby| is similar to |\childdocof|
described in \secref{sec:include},
but the subsequent selection of content must be done manually.
To that end, both |\ifchilddoc| and |\ifchilddocmanual|
will be true upon processing of a part,
and the name of the part is stored in |\childdocname|.
Note that |\jobname| will be set to the filename of the current part
so that each part receives an individual |.aux| file
that does not interfere with the |.aux| file(s) of the main document.
This behaviour can be altered by the alternative form
|\childdocby[*]{|\textit{main}|}| (with a non-empty optional argument)
which uses the |.aux| file of the main document
by setting |\jobname| to \textit{main}.

%%%%%%%%%%%%%%%%%%%%%%%%%%%%%%%%%%%%%%%%%%%%%%%%%%%%%%%%%%%%%%%%%%%%%%%%%%%%%%%%
\subsection{Driver Development}
\label{sec:driver}

The \textsf{childdoc} mechanism can also be use for the development
of definition files such as \LaTeX{} styles or classes.
This case differs from the above setup with multiple parts
included by |\include| in that no |\includeonly| should be invoked.
This can be achieved by starting the include file
(before |\ProvidesPackage|) with:
%
\begin{center}
\begin{tabular}{l}
|\input{childdoc.def}|\\
|\childdocforward{|\textit{main}|}|\\
\end{tabular}
\end{center}
%
or alternatively with:
%
\begin{center}
\begin{tabular}{l}
|\input{childdoc.def}|\\
|\childdocby{|\textit{main}|}|\\
\end{tabular}
\end{center}
%
Both forms have slightly different effects as described above.
The main file is prepared as usual, see \secref{sec:include}.

%%%%%%%%%%%%%%%%%%%%%%%%%%%%%%%%%%%%%%%%%%%%%%%%%%%%%%%%%%%%%%%%%%%%%%%%%%%%%%%%
\subsection{Legacy Detection}
\label{sec:detection}

The directive |\childdocmain| in the main file can detect
whether the complete document or merely a child is to be compiled
even without using the directive |\childdocof|.
This method is deprecated because it is less robust
and there is no compelling reason to use it;
it is merely provided for backward compatibility
and it may be removed in future versions.

If the detection mechanism is to be used,
it is mandatory to correctly specify
the filename of the main file as the argument of |\childdocmain|:
%
\begin{center}
\begin{tabular}{l}
|\input{childdoc.def}|\\
|\childdocmain{|\textit{main}|}|\\
\end{tabular}
\end{center}
%
If |\jobname| does not match the argument \textit{main} of |\childdocmain|,
it is assumed that |\jobname| points to the child file to be compiled.
When using |\childdocmain| with the main file specified as argument,
it suffices to start a child file
with just |\input{|\textit{main}|}|
without loading of the package and using |\childdocof|.
If instead all processing is done
with the appropriate \textsf{childdoc} directives,
the argument of \textit{main} of |\childdocmain| can be empty.

An alternative version of the command line processing described
in \secref{sec:commandline} using the detection mechanism reads:
%
\begin{center}
|... -jobname "|\textit{target}|" "|[\textit{flags}]%
[|\def\jobname{|\textit{dest}|}|]|\input{|\textit{main}|}"|
\end{center}

%%%%%%%%%%%%%%%%%%%%%%%%%%%%%%%%%%%%%%%%%%%%%%%%%%%%%%%%%%%%%%%%%%%%%%%%%%%%%%%%
\subsection{Manual Code}
\label{sec:manual}

In case one cannot be certain whether the definitions file |childdoc.def|
is installed on the target \TeX{} distribution
and one prefers not to ship it,
it is conceivable to paste a few relevant commands into the sources.

To that end, drop all statements |\input{childdoc.def}|
and perform the replacements as outlined below.
Instead of |\childdocmain{|\textit{main}|}| add the following code
to the top of the main file:
%
\begin{center}
\begin{tabular}{l}
|\||ifdefined\childdocname\endinput\||fi\newif\ifchilddoc|\\
|\edef\childdocname{\scantokens\expandafter{\jobname\noexpand}}|\\
|\def\childdocmain{|\textit{main}|}\||ifx\childdocmain\childdocname\||else|\\
|\childdoctrue\includeonly{\childdocname}\let\jobname\childdocmain\||fi|\\
\end{tabular}
\end{center}
%
Instead of |\childdocof{|\textit{main}|}| just include the main file
at the top of each child file:
%
\begin{center}
|\input{|\textit{main}|}|
\end{center}
%
A simple redirection |\childdocforward{|\textit{dest}|}| is achieved by:
%
\begin{center}
|\def\jobname{|\textit{dest}|}\input{\jobname}|
\end{center}
%
The redirection with prefix
|\childdocforwardprefix[|\textit{prefix}|]{|\textit{dest}|}|
is accomplished by:
%
\begin{center}
\begin{tabular}{l}
|{\edef\jobname{\scantokens\expandafter{\jobname\noexpand}}|\\
|\def\redirectjob |\textit{prefix}|#1~~~{\gdef\jobname{|\textit{dest}|#1}}|\\
|\expandafter\redirectjob\jobname~~~}\input{\jobname}|
\end{tabular}
\end{center}

In an alternative approach,
child documents can be compiled by a specific command line
without additional code or specific definitions:
%
\begin{center}
|... -jobname "|\textit{target}|" "|[\textit{flags}]%
|\includeonly{|\textit{dest}|}\input{|\textit{main}|}"|
\end{center}
%

%%%%%%%%%%%%%%%%%%%%%%%%%%%%%%%%%%%%%%%%%%%%%%%%%%%%%%%%%%%%%%%%%%%%%%%%%%%%%%%%
%%%%%%%%%%%%%%%%%%%%%%%%%%%%%%%%%%%%%%%%%%%%%%%%%%%%%%%%%%%%%%%%%%%%%%%%%%%%%%%%
\section{Information}

%%%%%%%%%%%%%%%%%%%%%%%%%%%%%%%%%%%%%%%%%%%%%%%%%%%%%%%%%%%%%%%%%%%%%%%%%%%%%%%%
\subsection{Copyright}

Copyright \copyright{} 2017--2018 Niklas Beisert

This work may be distributed and/or modified under the
conditions of the \LaTeX{} Project Public License, either version 1.3
of this license or (at your option) any later version.
The latest version of this license is in
  \url{http://www.latex-project.org/lppl.txt}
and version 1.3 or later is part of all distributions of \LaTeX{}
version 2005/12/01 or later.

This work has the LPPL maintenance status `maintained'.

The Current Maintainer of this work is Niklas Beisert.

This work consists of the files |README.txt|, |childdoc.ins| and |childdoc.dtx|
as well as the derived files |childdoc.def|, |cdocsamp.tex|
with |cdocsch1.tex|, |cdocsch2.tex|, |cdocspt3.tex|, |cdocspt4.tex|,
|cdocsdrf.tex|, |cdocsfn1.tex|, |cdocsfn2.tex|
as well as |childdoc.pdf|.

%%%%%%%%%%%%%%%%%%%%%%%%%%%%%%%%%%%%%%%%%%%%%%%%%%%%%%%%%%%%%%%%%%%%%%%%%%%%%%%%
\subsection{Files and Installation}

The package consists of the files:
%
\begin{center}
\begin{tabular}{ll}
    |README.txt|   & readme file \\
    |childdoc.ins| & installation file \\
    |childdoc.dtx| & source file \\
    |childdoc.def| & definition file \\
    |cdocsamp.tex| & sample main file \\
    |cdocsch1.tex| & sample include file \\
    |cdocsch2.tex| & sample include file \\
    |cdocspt3.tex| & sample part file \\
    |cdocspt4.tex| & sample part file \\
    |cdocsdrf.tex| & sample redirection file \\
    |cdocsfn1.tex| & sample redirection file \\
    |cdocsfn2.tex| & sample redirection file \\
    |childdoc.pdf| & manual
\end{tabular}
\end{center}
%
The distribution consists of the files
|README.txt|, |childdoc.ins| and |childdoc.dtx|.
%
\begin{itemize}
\item
Run (pdf)\LaTeX{} on |childdoc.dtx|
to compile the manual |childdoc.pdf| (this file).
\item
Run \LaTeX{} on |childdoc.ins| to create the definitions file |childdoc.def|
and the sample |cdocsamp.tex| with include files
|cdocsch1.tex|, |cdocsch2.tex|, |cdocspt3.tex|, |cdocspt4.tex|,
|cdocsdrf.tex|, |cdocsfn1.tex|, |cdocsfn2.tex|.
Then copy the file |childdoc.def| to an appropriate directory of your \LaTeX{}
distribution, e.g.\ \textit{texmf-root}|/tex/latex/childdoc|.
\end{itemize}

%%%%%%%%%%%%%%%%%%%%%%%%%%%%%%%%%%%%%%%%%%%%%%%%%%%%%%%%%%%%%%%%%%%%%%%%%%%%%%%%
\subsection{Related CTAN Packages}

There are several other packages which offer a similar functionality:
%
\begin{itemize}
\item
The packages
\href{http://ctan.org/pkg/docmute}{\textsf{docmute}},
\href{http://ctan.org/pkg/includex}{\textsf{includex}} and
\href{http://ctan.org/pkg/standalone}{\textsf{standalone}}
provide commands to include only the document body of
a child file thus allowing both files to be compiled individually.
\item
The packages \href{http://ctan.org/pkg/subdocs}{\textsf{subdocs}}
and \href{http://ctan.org/pkg/subfiles}{\textsf{subfiles}}
provide structures in which the main and child documents can be
encapsulated and allowing them to be compiled individually.
The inclusion mechanism is different from the conventional |\include|.
\item
The package \href{http://ctan.org/pkg/combine}{\textsf{combine}}
is an elaborate solution to combine several documents into one.
\end{itemize}
%
See also the CTAN topic \href{http://ctan.org/topic/subdocs}{\textsf{subdocs}}
for further related packages.
The present package differs from the above solutions in that
a document structure constructed with the conventional |\include| mechanism
just needs two extra commands at the top of every file
such that all constituent files can be compiled individually.

%%%%%%%%%%%%%%%%%%%%%%%%%%%%%%%%%%%%%%%%%%%%%%%%%%%%%%%%%%%%%%%%%%%%%%%%%%%%%%%%
%\subsection{Feature Suggestions}
%
%The following is a list of features which may be useful for future
%versions of this package:
%%
%\begin{itemize}
%\item
%\ldots
%\end{itemize}

%%%%%%%%%%%%%%%%%%%%%%%%%%%%%%%%%%%%%%%%%%%%%%%%%%%%%%%%%%%%%%%%%%%%%%%%%%%%%%%%
\subsection{Revision History}

%%%%%%%%%%%%%%%%%%%%%%%%%%%%%%%%%%%%%%%%
\paragraph{v2.0:} 2018/12/30

\begin{itemize}
\item
immediate forward processing
\item
added |\childdocby| mechanism
\item
manual restructured
\end{itemize}

%%%%%%%%%%%%%%%%%%%%%%%%%%%%%%%%%%%%%%%%
\paragraph{v1.6:} 2018/01/17

\begin{itemize}
\item
application for development of include files
\item
corrections to manual
\end{itemize}

%%%%%%%%%%%%%%%%%%%%%%%%%%%%%%%%%%%%%%%%
\paragraph{v1.5:} 2017/05/21

\begin{itemize}
\item
more complete structuring introduced
\item
|\childdocof| introduced
\item
|\childdoc| renamed to |\childdocmain|
\item
|\childredirect| renamed to |\childdocforward| and |\childdocforwardprefix|
and functionality expanded
\end{itemize}

%%%%%%%%%%%%%%%%%%%%%%%%%%%%%%%%%%%%%%%%
\paragraph{v1.0:} 2017/04/27

\begin{itemize}
\item
manual and install package
\item
first version published on CTAN
\end{itemize}

%%%%%%%%%%%%%%%%%%%%%%%%%%%%%%%%%%%%%%%%
\paragraph{v0.6:} 2017/04/26

\begin{itemize}
\item
redirection mechanism added
\end{itemize}

%%%%%%%%%%%%%%%%%%%%%%%%%%%%%%%%%%%%%%%%
\paragraph{v0.5:} 2017/04/26

\begin{itemize}
\item
functionality in definition file
\end{itemize}


%%%%%%%%%%%%%%%%%%%%%%%%%%%%%%%%%%%%%%%%%%%%%%%%%%%%%%%%%%%%%%%%%%%%%%%%%%%%%%%%
%%%%%%%%%%%%%%%%%%%%%%%%%%%%%%%%%%%%%%%%%%%%%%%%%%%%%%%%%%%%%%%%%%%%%%%%%%%%%%%%
%%%%%%%%%%%%%%%%%%%%%%%%%%%%%%%%%%%%%%%%%%%%%%%%%%%%%%%%%%%%%%%%%%%%%%%%%%%%%%%%
\appendix

\settowidth\MacroIndent{\rmfamily\scriptsize 000\ }

 \DocInput{childdoc.dtx}

\end{document}
%</driver>
% \fi
%
% %%%%%%%%%%%%%%%%%%%%%%%%%%%%%%%%%%%%%%%%%%%%%%%%%%%%%%%%%%%%%%%%%%%%%%%%%%%%%%
% %%%%%%%%%%%%%%%%%%%%%%%%%%%%%%%%%%%%%%%%%%%%%%%%%%%%%%%%%%%%%%%%%%%%%%%%%%%%%%
% \section{Sample}
%\iffalse
%<*samplemain>
%\fi
%
% The following presents a sample document
% with two chapters, two parts, a title page,
% a compile flag as well as three forwarding files to set the flag.
% It consists of eight |.tex| files:
% \begin{center}
% \begin{tabular}{ll}
% |cdocsamp.tex|&main file\\
% |cdocsch1.tex|&include file for chapter 1\\
% |cdocsch2.tex|&include file for chapter 2\\
% |cdocspt3.tex|&include file for part 3\\
% |cdocspt4.tex|&include file for part 4\\
% |cdocsdrf.tex|&forwarding file for main file in draft mode\\
% |cdocsfi1.tex|&forwarding file for final version of chapter 1\\
% |cdocsfi2.tex|&forwarding file for final version of chapter 2\\
% \end{tabular}
% \end{center}
% Each of the eight files can be compiled directly by the \LaTeX{} compiler.
%
% %%%%%%%%%%%%%%%%%%%%%%%%%%%%%%%%%%%%%%
% \paragraph{Main File.}
%
% The main file is called |cdocsamp.tex|.
%
% Load the \textsf{childdoc} definitions and
% declare the filename for the main document:
%    \begin{macrocode}
\input{childdoc.def}
\childdocmain{}
%    \end{macrocode}

% Optional override for |\version| flag:
%    \begin{macrocode}
%%\ifchilddoc\else\providecommand{\version}{draft}\fi
%    \end{macrocode}

% Define the default values for the |\version| flag
% (|final| for the main file and |draft| for childs):
%    \begin{macrocode}
\ifchilddoc
\providecommand{\version}{draft}
\else
\providecommand{\version}{final}
\fi
%    \end{macrocode}

% Load the standard document class:
%    \begin{macrocode}
\documentclass[12pt]{article}
%    \end{macrocode}

% Start the document body:
%    \begin{macrocode}
\begin{document}
%    \end{macrocode}

% Declare a title page.
% Print title, part of document being processed and version flag:
%    \begin{macrocode}
\addtocounter{page}{-1}
\begin{center}
{\LARGE\bfseries{}childdoc example\par}
\vspace{1cm}
\ifchilddoc
\ifchilddocmanual part\else chapter\fi:
`\childdocname' of `\childdocjob'\par
\else
main document: `\childdocjob'\par
\fi
version: \version\par
\end{center}
\newpage
%    \end{macrocode}

% Manually include selected file,
% otherwise process as usual:
%    \begin{macrocode}
\ifchilddocmanual
\section*{part `\childdocname'}
\input{\childdocname}
\else
%    \end{macrocode}

% Include the two chapters:
%    \begin{macrocode}
\include{cdocsch1}
\include{cdocsch2}
%    \end{macrocode}

% Include the two parts unless only chapters should be displayed:
%    \begin{macrocode}
\ifchilddoc\else
\section{part three}
\input{cdocspt3}
\section{part four}
\input{cdocspt4}
\fi
%    \end{macrocode}

% Process as usual until here:
%    \begin{macrocode}
\fi
%    \end{macrocode}

% End of document body:
%    \begin{macrocode}
\end{document}
%    \end{macrocode}
%\iffalse
%</samplemain>
%\fi
%
% %%%%%%%%%%%%%%%%%%%%%%%%%%%%%%%%%%%%%%
% \paragraph{Chapter Include Files.}
%
% The include files are called |cdocsch1.tex| and |cdocsch2.tex|.
%
%\iffalse
%<*samplechap1|samplechap2>
%\fi

% Optional override for |\version| flag:
%    \begin{macrocode}
%%\providecommand{\version}{final}
%    \end{macrocode}

% Include the main document:
%    \begin{macrocode}
\input{childdoc.def}
\childdocof{cdocsamp}
%    \end{macrocode}

%\iffalse
%</samplechap1|samplechap2>
%\fi
%
%\iffalse
%<*samplechap1>
%\fi
% Some text for chapter 1:
%    \begin{macrocode}
\section{one}
some text in chapter one
%    \end{macrocode}

%\iffalse
%</samplechap1>
%\fi
% Some text for chapter 2:
%\iffalse
%<*samplechap2>
%\fi
%    \begin{macrocode}
\section{two}
more text in chapter two
%    \end{macrocode}

%\iffalse
%</samplechap2>
%\fi
%
% %%%%%%%%%%%%%%%%%%%%%%%%%%%%%%%%%%%%%%
% \paragraph{Part Include Files.}
%
% The include files are called |cdocspt3.tex| and |cdocspt4.tex|.
%
%\iffalse
%<*samplepart3|samplepart4>
%\fi

% Optional override for |\version| flag:
%    \begin{macrocode}
%%\providecommand{\version}{final}
%    \end{macrocode}

% Include the main document:
%    \begin{macrocode}
\input{childdoc.def}
\childdocby{cdocsamp}
%    \end{macrocode}

%\iffalse
%</samplepart3|samplepart4>
%\fi
%
%\iffalse
%<*samplepart3>
%\fi
% Some text for part 3:
%    \begin{macrocode}
some text in part three
%    \end{macrocode}

%\iffalse
%</samplepart3>
%\fi
% Some text for part 4:
%\iffalse
%<*samplepart4>
%\fi
%    \begin{macrocode}
more text in part four
%    \end{macrocode}

%\iffalse
%</samplepart4>
%\fi
%
% %%%%%%%%%%%%%%%%%%%%%%%%%%%%%%%%%%%%%%
% \paragraph{Forwarding for a Complete Draft.}
%
% The following forwarding file |cdocsdrf.tex|
% compiles the main document in draft mode:
%\iffalse
%<*sampledraft>
%\fi
%    \begin{macrocode}
\def\version{draft}
\input{childdoc.def}
\childdocforward{cdocsamp}
%    \end{macrocode}

%\iffalse
%</sampledraft>
%\fi
%
% %%%%%%%%%%%%%%%%%%%%%%%%%%%%%%%%%%%%%%
% \paragraph{Forwarding for Final Version of the Chapters.}
%
% The following forwarding files |cdocsfn1.tex| and |cdocsfn2.tex|
% (with identical content)
% compile the final versions of the child documents
% |cdocsch1.tex| and |cdocsch2.tex|, respectively:
%\iffalse
%<*samplefinal>
%\fi
%    \begin{macrocode}
\def\version{final}
\input{childdoc.def}
\childdocforwardprefix[cdocsamp]{cdocsfn}{cdocsch}
%    \end{macrocode}

%\iffalse
%</samplefinal>
%\fi
%
% %%%%%%%%%%%%%%%%%%%%%%%%%%%%%%%%%%%%%%
% \paragraph{Command Line Processing.}
%
% The following three command lines generate the output files
% |cdocscld|, |cdocscl1| and |cdocscl2|
% which should be identical to
% |cdocsdrf|, |cdocsch1| and |cdocsfn2|, respectively:
% \begin{center}
% \begin{tabular}{l}
% |latex -jobname cdocscld \|\\
% |  "\def\version{draft}\input{childdoc.def}\childdocforward{cdocsamp}"|\\
% |latex -jobname cdocscl1 \|\\
% |  "\input{childdoc.def}\childdocforward[cdocsamp]{cdocsch1}"|\\
% |latex -jobname cdocscl2 \|\\
% |  "\def\version{final}\input{childdoc.def}\childdocforward{cdocsch2}"|
% \end{tabular}
% \end{center}
% Note that the trailing backslash on each first line
% merely continues the input to the second line
% (for convenient cut ant paste).
% Furthermore, the command |latex| can be replaced by any
% of its alternative versions such as |pdflatex|.
%
% %%%%%%%%%%%%%%%%%%%%%%%%%%%%%%%%%%%%%%%%%%%%%%%%%%%%%%%%%%%%%%%%%%%%%%%%%%%%%%
% %%%%%%%%%%%%%%%%%%%%%%%%%%%%%%%%%%%%%%%%%%%%%%%%%%%%%%%%%%%%%%%%%%%%%%%%%%%%%%
% \section{Implementation}
%\iffalse
%<*package>
%\fi
%
% This section describes the definitions file |childdoc.def|.

% The definitions cannot be loaded using |\usepackage| or |\RequirePackage|
% which has a mechanism to prevent loading a style file more than once.
% When loading the definitions by means of |\input|
% multiple instances have to be prevented manually:
%\iffalse
%This code needs to be before the `\ProvidesFile' directive
%which is defined at the beginning of this file.
%Therefore it is also placed there and commented out here.
%</package>
%<*discard>
%\fi
%    \begin{macrocode}
\ifdefined\childdocmain\endinput\fi
%    \end{macrocode}
%\iffalse
%</discard>
%<*package>
%\fi
%
% \macro{\ifchilddoc}
% \macro{\ifchilddocmanual}
% The conditional |\ifchilddoc| tells whether a
% child (true) or main (false) document is being compiled.
% The conditional |\ifchilddocmanual| tells whether
% the |\includeonly| mechanism is used (false) or
% the selection of child files must be performed manually (true).
% The definitions initialise to false:
%    \begin{macrocode}
\newif\ifchilddoc
\newif\ifchilddocmanual
%    \end{macrocode}

% \macro{\childdocname}
% \macro{\childdocjob}
% The macro |\childdocname| stores the name of the main document
% to be compiled. The macro |\childdocjob| stores the name of
% the document on which the \LaTeX{} compiler was originally invoked.
% The content of |\jobname| cannot be compared
% to filenames specified in the source due to different catcodes.
% The following code rescans |\jobname|, stores the result
% in |\childdocname| and saves a copy in |\childdocjob|:
%    \begin{macrocode}
\edef\childdocname{\scantokens\expandafter{\jobname\noexpand}}
\let\childdocjob\childdocname
%    \end{macrocode}

% \macro{\childdocdisable}
% The macro |\childdocdisable| prevents the main file
% from being processed more than once.
% At this stage, the main document command |\childdocmain|
% is assumed to be called once again where it should do nothing.
% Any subsequent call to it should prevent
% a secondary processing of the main document
% It overwrites the forwarding commands
% |\childdocof| and |\childdocforward|
% with empty macros to prevent further inclusions of the main document:
%    \begin{macrocode}
\newcommand{\childdocdisable}
{
  \renewcommand{\childdocmain}[1]{\renewcommand{\childdocmain}[1]{\endinput}}
  \renewcommand{\childdocof}[1]{}
  \renewcommand{\childdocby}[2][]{}
  \renewcommand{\childdocforward}[2][]{}
  \renewcommand{\childdocdisable}{}
}
%    \end{macrocode}

% \macro{\childdocmain}
% The macro |\childdocmain| is to be called at the top of the main file
% with nothing or the main filename (without extension) as argument.
% First, it breaks loops.
% If the argument is not empty and does not match |\childdocname|
% (which is set by the first inclusion of |childdoc.def|),
% |\ifchilddoc| is set to true, |\includeonly| is applied to the child file
% and |\jobname| is set to the main file
% (for proper handling of |.aux| files):
%    \begin{macrocode}
\newcommand{\childdocmain}[1]
{
  \childdocdisable\childdocmain{}
  \if?#1?\else
    \begingroup
      \def\childdoctmp{#1}
      \ifx\childdoctmp\childdocname
        \def\childdoctmp{}
      \else
        \def\childdoctmp
        {
          \childdoctrue
          \includeonly{\childdocname}
          \def\childdocjob{#1}
          \def\jobname{#1}
        }
      \fi
      \expandafter
    \endgroup
    \childdoctmp
  \fi
}
%    \end{macrocode}

% \macro{\childdocof}
% The command |\childdocof| redirects
% compilation to the main file |#1|.
%    \begin{macrocode}
\newcommand{\childdocof}[1]
{
  \childdocdisable
  \childdoctrue
  \includeonly{\childdocname}
  \def\jobname{#1}
  \def\childdocjob{#1}
  \input{#1}
}
%    \end{macrocode}

% \macro{\childdocby}
% The command |\childdocby| ....
%    \begin{macrocode}
\newcommand{\childdocby}[2][]
{
  \childdocdisable
  \childdoctrue
  \childdocmanualtrue
  \if?#1?\else
    \def\jobname{#2}
  \fi
  \def\childdocjob{#2}
  \input{#2}
  \endinput
}
%    \end{macrocode}

% \macro{\childdocforward}
% The command |\childdocforward| redirects
% compilation to the main file or
% (if the optional argument is given) a child file.
% Parameters are set as if the main file
% or a child file starting with |\childdocof| was compiled.
% Then compilation is handed over to the main file:
%    \begin{macrocode}
\newcommand{\childdocforward}[2][]
{
  \begingroup
    \if?#1?
      \def\childdoctmp
      {
        \def\childdocname{#2}
        \def\childdocjob{#2}
        \def\jobname{#2}
        \input{#2}
        \endinput
      }
    \else
      \def\childdoctmp
      {
        \childdocdisable
        \def\childdocname{#2}
        \childdoctrue
        \includeonly{#2}
        \def\childdocjob{#1}
        \def\jobname{#1}
        \input{#1}
        \endinput
      }
    \fi
    \expandafter
  \endgroup
  \childdoctmp
}
%    \end{macrocode}

% \macro{\childdocforwardprefix}
% The command |\childdocforwardprefix| redirects
% compilation to the main or a child file by means of a pattern.
% The prefix |#1| in the current filename is replaced by |#2|
% and the suffix of the current filename is kept
% (it is assumed that the filename does not contain the substring `|~~~|'
% which is used as a delimiter).
% Compilation is handed over to the new file by |\childdocforward|:
%    \begin{macrocode}
\newcommand{\childdocforwardprefix}[3][]
{
  \begingroup
    \def\childdocextract #2##1~~~{\def\childdoctmp{\childdocforward[#1]{#3##1}}}
    \expandafter\childdocextract\childdocname~~~
    \expandafter
  \endgroup
  \childdoctmp
}
%    \end{macrocode}

% \macro{\childdoc}
% The deprecated macro |\childdoc| is a legacy version of |\childdocmain|:
%    \begin{macrocode}
\newcommand{\childdoc}{\childdocmain}
%    \end{macrocode}

% \macro{\childdocredirect}
% The deprecated macro |\childdocredirect| is a legacy version
% of |\childdocforward| and |\childdocforwardprefix|:
%    \begin{macrocode}
\newcommand{\childdocredirect}[2][]
{
  \begingroup
    \if?#1?
      \def\childdoctmp{\childdocforward{#2}}
    \else
      \def\childdoctmp{\childdocforwardprefix{#1}{#2}}
    \fi
    \expandafter
  \endgroup
  \childdoctmp
}
%    \end{macrocode}

%\iffalse
%</package>
%\fi
%
\endinput
|\\
|\childdocby{|\textit{main}|}|\\
\end{tabular}
\end{center}
%
Both forms have slightly different effects as described above.
The main file is prepared as usual, see \secref{sec:include}.

%%%%%%%%%%%%%%%%%%%%%%%%%%%%%%%%%%%%%%%%%%%%%%%%%%%%%%%%%%%%%%%%%%%%%%%%%%%%%%%%
\subsection{Legacy Detection}
\label{sec:detection}

The directive |\childdocmain| in the main file can detect
whether the complete document or merely a child is to be compiled
even without using the directive |\childdocof|.
This method is deprecated because it is less robust
and there is no compelling reason to use it;
it is merely provided for backward compatibility
and it may be removed in future versions.

If the detection mechanism is to be used,
it is mandatory to correctly specify
the filename of the main file as the argument of |\childdocmain|:
%
\begin{center}
\begin{tabular}{l}
|% \iffalse
%
% childdoc.dtx Copyright (C) 2017-2018 Niklas Beisert
%
% This work may be distributed and/or modified under the
% conditions of the LaTeX Project Public License, either version 1.3
% of this license or (at your option) any later version.
% The latest version of this license is in
%   http://www.latex-project.org/lppl.txt
% and version 1.3 or later is part of all distributions of LaTeX
% version 2005/12/01 or later.
%
% This work has the LPPL maintenance status `maintained'.
%
% The Current Maintainer of this work is Niklas Beisert.
%
% This work consists of the files childdoc.dtx and childdoc.ins
% and the derived files childdoc.def and cdocsamp.tex with
% cdocsch1.tex, cdocsch2.tex, cdocsdrf.tex, cdocsfn1.tex, cdocsfn2.tex.
%
%<package>\ifdefined\childdocmain\endinput\fi
%<package>\ProvidesFile{childdoc.def}[2018/12/30 v2.0 child document driver]
%<samplemain>\ProvidesFile{cdocsamp.tex}[2018/12/30 v2.0 sample for childdoc]
%<*driver>
%\ProvidesFile{childdoc.drv}[2018/12/30 v2.0 childdoc reference manual file]
\PassOptionsToClass{10pt,a4paper}{article}
\documentclass{ltxdoc}

\usepackage[margin=35mm]{geometry}
\usepackage{hyperref}
\usepackage{hyperxmp}
\usepackage[usenames]{color}

\hypersetup{colorlinks=true}
\hypersetup{pdfstartview=FitH}
\hypersetup{pdfpagemode=UseNone}
\hypersetup{pdfsource={}}
\hypersetup{pdflang={en-UK}}
\hypersetup{pdfcopyright={Copyright 2017-2018 Niklas Beisert.
  This work may be distributed and/or modified under the
  conditions of the LaTeX Project Public License, either version 1.3
  of this license or (at your option) any later version.}}
\hypersetup{pdflicenseurl={http://www.latex-project.org/lppl.txt}}
\hypersetup{pdfcontactaddress={ETH Zurich, ITP, HIT K,
  Wolfgang-Pauli-Strasse 27}}
\hypersetup{pdfcontactpostcode={8093}}
\hypersetup{pdfcontactcity={Zurich}}
\hypersetup{pdfcontactcountry={Switzerland}}
\hypersetup{pdfcontactemail={nbeisert@itp.phys.ethz.ch}}
\hypersetup{pdfcontacturl={http://people.phys.ethz.ch/\xmptilde nbeisert/}}

\newcommand{\secref}[1]{\hyperref[#1]{section \ref*{#1}}}

\parskip1ex
\parindent0pt
\let\olditemize\itemize
\def\itemize{\olditemize\parskip0pt}

\begin{document}

\title{The \textsf{childdoc} Package}
\hypersetup{pdftitle={The childdoc Package}}
\author{Niklas Beisert\\[2ex]
  Institut f\"ur Theoretische Physik\\
  Eidgen\"ossische Technische Hochschule Z\"urich\\
  Wolfgang-Pauli-Strasse 27, 8093 Z\"urich, Switzerland\\[1ex]
  \href{mailto:nbeisert@itp.phys.ethz.ch}
  {\texttt{nbeisert@itp.phys.ethz.ch}}}
\hypersetup{pdfauthor={Niklas Beisert}}
\hypersetup{pdfsubject={Manual for the LaTeX2e Package childdoc}}
\date{30 December 2018, \textsf{v2.0}}
\maketitle

\begin{abstract}\noindent
\textsf{childdoc} is a \LaTeXe{} package
that enables the direct compilation
of document sections included by |\include|
to individual files.
\end{abstract}

\begingroup
\parskip0ex
\tableofcontents
\endgroup

%%%%%%%%%%%%%%%%%%%%%%%%%%%%%%%%%%%%%%%%%%%%%%%%%%%%%%%%%%%%%%%%%%%%%%%%%%%%%%%%
%%%%%%%%%%%%%%%%%%%%%%%%%%%%%%%%%%%%%%%%%%%%%%%%%%%%%%%%%%%%%%%%%%%%%%%%%%%%%%%%
\section{Introduction}

\LaTeX{} provides a mechanism to structure a large document (such as a book)
into a main file and several child files (containing the chapters)
using the |\include| command.
This mechanism is beneficial for documents
which span hundreds of pages in order to
make the source file(s) more manageable.
Moreover, compilation can be restricted to
selected child files by means of the |\includeonly| command.
The latter feature can be used to reduce the compilation time while editing
(this was significantly more useful in the earlier days of \LaTeX{})
or to generate a smaller document which is easier to navigate.
Another application of |\includeonly| is to generate
documents consisting of selected parts of the complete document.

However, there are a few drawbacks of the plain |\include| mechanism:
\begin{itemize}
\item
The child files cannot be compiled on their own,
they can only be compiled via the main file.
A naive editing environment
(such as a text editor with an option
to have the current file processed by \LaTeX)
may require one to switch to the main file before compiling;
attempting to compile the child file produces errors.
\item
The main file must be modified (each time)
to adjust the |\includeonly| command
to the present needs. This easily leaves the main file in a messy state.
\item
The generated document will always carry the filename
of the main document. This is inconvenient if
several child files are to be compiled and
to be kept for distribution.
\end{itemize}

The present package provides a simple interface
to make child files individually compilable by \LaTeX{}.
Compiling a child file then has the same effect as compiling
the main file with an |\includeonly| command
to select the appropriate child.
Moreover the generated document will carry the name of the child
rather than the main file.
This resolves all three above issues.

This feature is meant to make the editing of books,
thesis documents and lecture notes somewhat more convenient.
However, the package can also be used efficiently for
composing a series of documents (such as exercise sheets)
which are typically distributed individually.
It then assists the author in generating the individual documents
(potentially in different versions)
as well as a document containing the collected series.
Another application is in developing style files
or other kinds of included material
where compilation of the style file could redirect
to a sample or test file.

%%%%%%%%%%%%%%%%%%%%%%%%%%%%%%%%%%%%%%%%%%%%%%%%%%%%%%%%%%%%%%%%%%%%%%%%%%%%%%%%
%%%%%%%%%%%%%%%%%%%%%%%%%%%%%%%%%%%%%%%%%%%%%%%%%%%%%%%%%%%%%%%%%%%%%%%%%%%%%%%%
\section{Usage}

First of all, the package \textsf{childdoc} is \emph{not} a standard
\LaTeXe{} |.sty| style file! Therefore it needs to be invoked in
a non-standard way.

%%%%%%%%%%%%%%%%%%%%%%%%%%%%%%%%%%%%%%%%%%%%%%%%%%%%%%%%%%%%%%%%%%%%%%%%%%%%%%%%
\subsection{Included Files}
\label{sec:include}

%%%%%%%%%%%%%%%%%%%%%%%%%%%%%%%%%%%%%%%%
\DescribeMacro{\childdocmain}
To use the package, add the commands
\begin{center}
\begin{tabular}{l}
|\input{childdoc.def}|\\
|\childdocmain{}|\\
\end{tabular}
\end{center}
at the very top of the main \LaTeX{} file,
in particular \emph{before} the |\documentclass| statement!
The argument of |\childdocmain| should be left empty
(but it must be present).

%%%%%%%%%%%%%%%%%%%%%%%%%%%%%%%%%%%%%%%%
\DescribeMacro{\childdocof}
Furthermore, add the commands
\begin{center}
\begin{tabular}{l}
|\input{childdoc.def}|\\
|\childdocof{|\textit{main}|}|\\
\end{tabular}
\end{center}
at the top of every child file \textit{child}
which is included by |\include{|\textit{child}|}|
from within the main file
(or at least for those files to be compiled individually).
The argument \textit{main} must be the filename of the main file.

There are a couple of
considerations in setting up the main and child documents:

%%%%%%%%%%%%%%%%%%%%%%%%%%%%%%%%%%%%%%%%
\paragraph{Restrictions.}

Please note the following restrictions:
\begin{itemize}
\item
|\childdocmain| must be called with one argument \textit{main}
to ensure compatibility with earlier version of the package.
It must either be empty (|\childdocmain{}|)
or precisely match the filename of the main file in which it is specified.
See \secref{sec:detection} for further information.
\item
The filename \textit{main} must be specified without the |.tex| extension.
\item
The filename \textit{main} is case sensitive
(even in case-insensitive file systems)
due to internal string comparison.
\item
The argument \textit{main} should be fully expanded, it cannot be a macro.
\item
Subdirectories and special characters should be avoided in filenames.
\item
The command |\childdocmain{|\textit{main}|}| must be followed by a whitespace.
It should not be followed immediately by another command
or by a comment mark `|%|'.
This is because the \TeX{} parser reads the token immediately following
the argument of |\childdocmain| and puts it
at the beginning of every child section;
however, a white\-space is ignored.
\end{itemize}

%%%%%%%%%%%%%%%%%%%%%%%%%%%%%%%%%%%%%%%%
\paragraph{Content of Main File.}

It is advisable to place all content in the child files included by |\include|.
Any output contained in the main file will appear in all child documents
unless suppressed manually;
it cannot be suppressed automatically by the |\includeonly| directive
and thus should normally be avoided.
A method to include some content in the main file
by means of conditional processing is described in \secref{sec:conditional}.

%%%%%%%%%%%%%%%%%%%%%%%%%%%%%%%%%%%%%%%%
\paragraph{Page Numbering.}

When only a part of the document is compiled,
the appropriate numbering of pages
(as well as other status parameters)
is determined from the |.aux| files.
The latter contain information from previous passes.
However this information needs to propagate through
all intermediate child documents.
Therefore the page numbering in child documents may well
be inconsistent until the complete document is compiled at least once.

A useful (if unconventional) way to always ensure a consistent
page numbering is to restart the numbering in each child document
and denote the pages by `\textit{child}|.|\textit{page}'
where \textit{child} represents the chapter/section number of the child file.
This can be achieved by the command
|\numberwithin{page}{|\textit{child}|}|
of the \textsf{amsmath} package
where \textit{child} can be |chapter| or |section|
depending on the chosen structuring.
Alternatively, one can modify the macro |\thepage| appropriately
and reset the counter |page| at the start of each child file.

%%%%%%%%%%%%%%%%%%%%%%%%%%%%%%%%%%%%%%%%%%%%%%%%%%%%%%%%%%%%%%%%%%%%%%%%%%%%%%%%
\subsection{Conditional Processing}
\label{sec:conditional}

The package provides a mechanism to compile different versions
of a document. To customise the versions further some conditional processing
can come in handy to distinguish which version is being compiled.
The package provides two macros to describe the compilation context:

%%%%%%%%%%%%%%%%%%%%%%%%%%%%%%%%%%%%%%%%
\DescribeMacro{\ifchilddoc}
The conditional |\ifchilddoc| distinguishes between the compilation of
child documents and the main document:
%
\begin{center}
|\ifchilddoc |\textit{child-code}| |[|\||else |\textit{main-code}]| \||fi|
\end{center}

%%%%%%%%%%%%%%%%%%%%%%%%%%%%%%%%%%%%%%%%
\DescribeMacro{\childdocname}
\DescribeMacro{\childdocjob}
The macro |\childdocname| contains the filename (without extension)
of the main or child file being processed.
Note that |\childdocjob| will always contain the name of the main file.

%%%%%%%%%%%%%%%%%%%%%%%%%%%%%%%%%%%%%%%%
\paragraph{Title Page.}

Conditional processing can be used to include a title or banner page
in the main document when proper precautions are taken.
Importantly, the code in the main file should ensure that the page counter
(as well as other status parameters which are stored in the |.aux| files)
takes the same value after the conditional processing.
Otherwise the page numbers may take divergent values
depending on which part is compiled.

For example, a title page could be declared by:
%
\begin{center}
\begin{tabular}{l}
|\ifchilddoc\||else|\\
|\addtocounter{page}{-1}|\\
\textit{code for title page}\\
|\newpage|\\
|\||fi|
\end{tabular}
\end{center}
%
A banner page for the child documents can be generated by:
%
\begin{center}
\begin{tabular}{l}
|\ifchilddoc|\\
|\addtocounter{page}{-1}|\\
\textit{code for banner page}\\
|\newpage|\\
|\||fi|
\end{tabular}
\end{center}
%
Here one could write a message such as:
\begin{center}
|This is the part \childdocname{} of \childdocjob{}.|
\end{center}

%%%%%%%%%%%%%%%%%%%%%%%%%%%%%%%%%%%%%%%%%%%%%%%%%%%%%%%%%%%%%%%%%%%%%%%%%%%%%%%%
\subsection{Flags}
\label{sec:flags}

The package makes it easy to generate different versions
of the main or child documents.
To this end compilation flags can be defined
and assigned different default values.
They will be particularly useful in conjunction
with the forwarding mechanism described in \secref{sec:forward}.

For example, it may be useful to have a flag |\version|
which can be set to |draft| or |final|.
The document source will contain some conditional code
depending on the value of |\version|.
Suppose further, the flag should default to |final| for the main file
and to |draft| for child files
which is a natural assignment for editing the document.
This is achieved by placing the following code
in the preamble of the main document
(below the |\childdocmain| directive):
%
\begin{center}
\begin{tabular}{l}
|\ifchilddoc|\\
|\providecommand{\version}{draft}|\\
|\||else|\\
|\providecommand{\version}{final}|\\
|\||fi|
\end{tabular}
\end{center}
%
The definition by |\providecommand| makes sure
that previous definitions are not overwritten.
Further statements |\providecommand{\version}{...}|
can thus be added before the above code to override it.

For the main file, one might add a line
(between |\childdocmain| and the above block)
%
\begin{center}
|%\ifchilddoc\||else\providecommand{\version}{draft}\||fi|
\end{center}
%
which can be uncommented to produce a draft version.
Likewise one can add a line to the very top of a child file
(above the |\childdocof{|\textit{main}|}| directive)
%
\begin{center}
|%\providecommand{\version}{final}|
\end{center}
%
which can be uncommented to produce the final version of this child document.

%%%%%%%%%%%%%%%%%%%%%%%%%%%%%%%%%%%%%%%%%%%%%%%%%%%%%%%%%%%%%%%%%%%%%%%%%%%%%%%%
\subsection{Forwarding}
\label{sec:forward}

Different versions of the main or child documents
using compilation flags as described in \secref{sec:flags}
can be (permanently) stored in different files
for convenient compilation, viewing and distribution.
To this end, the package defines a command
to pass on compilation to a different file:

%%%%%%%%%%%%%%%%%%%%%%%%%%%%%%%%%%%%%%%%
\DescribeMacro{\childdocforward}
The command |\childdocforward| redirects processing to
another source file:
%
\begin{center}
\begin{tabular}{l}
|\input{childdoc.def}|\\
|\childdocforward[|\textit{main}|]{|\textit{dest}|}|\\
\end{tabular}
\end{center}
%
The argument \textit{dest} is the destination file
(without extension).
It should be the main file or one of the child files.
Note that further \textsf{childdoc} directives
such as |\childdocof| and |\childdocforward|
in the indicated file will be processed in this form.
The optional argument \textit{main}
passes on directly to the main file \textit{main}
while pretending to compile the child \textit{dest}.
This form behaves as if \textit{dest}
issues |\childdocof{|\textit{main}|}| right away,
and no further \textsf{childdoc} directives will be processed.

%%%%%%%%%%%%%%%%%%%%%%%%%%%%%%%%%%%%%%%%
\DescribeMacro{\...prefix}
In the alternative form |\childdocforwardprefix|,
%
\begin{center}
\begin{tabular}{l}
|\input{childdoc.def}|\\
|\childdocforwardprefix[|\textit{main}|]{|\textit{prefix}|}{|\textit{dest}|}|
\end{tabular}
\end{center}
%
the destination file is determined by a pattern
depending on the current file:
To make this work, the current file must be called
`{\textit{prefix}\hspace{0.2em}\textit{suffix}}'
with \textit{prefix} matching precisely the argument.
Processing is then passed on to the file
`{\textit{dest}\hspace{0.2em}\textit{suffix}}'.
Surely, the same effect is achieved by
directly specifying the
argument `{\textit{dest}\hspace{0.2em}\textit{suffix}}'
in the first form.
However, that requires to set up a different file
for each child. With the alternative form of the command
all these files can have exactly the same content
which simplifies setting them up and maintaining them.

For example, the following file |draft.tex|
with a compilation flag |\version| as described in \secref{sec:flags}
compiles the main document as a draft:
%
\begin{center}
\begin{tabular}{l}
|\def\version{draft}|\\
|\input{childdoc.def}|\\
|\childdocforward{|\textit{main}|}|
\end{tabular}
\end{center}
%
Likewise, the following files |final|\textit{nn}|.tex|
compile the final version of the child document
|child|\textit{nn}|.tex|:
%
\begin{center}
\begin{tabular}{l}
|\def\version{final}|\\
|\input{childdoc.def}|\\
|\childdocforwardprefix{final}{child}|
\end{tabular}
\end{center}
%

Note that when several versions of a main file and/or of each child file
are to be generated, it may be convenient to set up a |Makefile| or
shell script to automatise the process.

%%%%%%%%%%%%%%%%%%%%%%%%%%%%%%%%%%%%%%%%%%%%%%%%%%%%%%%%%%%%%%%%%%%%%%%%%%%%%%%%
\subsection{Command Line Processing}
\label{sec:commandline}

The effect of redirection files can also be achieved by invoking
the \LaTeX{} compiler with a more elaborate command line.
Most conveniently this should be done as part
of a shell script or a |Makefile|.

When using \textsf{childdoc} in the main file, the following
command lines effectively perform a redirection
(note that depending on the shell being used,
backslashes may have to be doubled: `|\|' $\to$ `|\\|'):
%
\begin{center}
|... -jobname "|\textit{target}|" |\\|"|[\textit{flags}]%
|\input{childdoc.def}\childdocforward[|\textit{main}|]{|\textit{dest}|}"|
\end{center}
%
Here \textit{target} is the name of the output file,
\textit{main} is the name of the main file
and \textit{dest} is the name of the main or child file to be processed
(all filenames without extensions).
The optional argument \textit{main} can be omitted
if \textit{main} matches \textit{dest}.
Optionally, compilation \textit{flags} can be defined via |\def| commands.
This command line makes the \TeX{} engine believe
it is compiling the file \textit{target}
whose content is specified as the latter parameter.
The provided code then forwards the processing to
\textit{main} or \textit{dest} as described in \secref{sec:forward}.

%%%%%%%%%%%%%%%%%%%%%%%%%%%%%%%%%%%%%%%%%%%%%%%%%%%%%%%%%%%%%%%%%%%%%%%%%%%%%%%%
\subsection{Include by Input}
\label{sec:input}

Including child documents by |\include| has some restrictions by design.
Most notably, the content of a child document always occupies
its own set of pages; pages cannot be shared between child documents.
Usually, this behaviour makes perfect sense
because each child document contain an essential part of the document.
However, in some situations it may be desirable to compose
a document from a collection of parts
without having mandatory page breaks between then.
For this case, the package
provides a mechanism to include parts
by |\input| which can also be processed individually.
However, by construction this mechanism
requires manual handling of the content to be output.

%%%%%%%%%%%%%%%%%%%%%%%%%%%%%%%%%%%%%%%%
\DescribeMacro{\ifchilddocmanual}
The main file should be prepared as usual, see \secref{sec:include}.
However, the document body must make a distinction
between processing of an individual part and of the main document, e.g.:
%
\begin{center}
\begin{tabular}{l}
|\ifchilddocmanual|\\
|\input{\childdocname}|\\
|\||else|\\
\textit{document body with }|\input{|\textit{part}|}|\\
|\||fi|
\end{tabular}
\end{center}
%
The conditional |\ifchilddocmanual| is true whenever
a part to be included by |\input| is being compiled,
and the name of the part is stored in |\childdocname|.

%%%%%%%%%%%%%%%%%%%%%%%%%%%%%%%%%%%%%%%%
\DescribeMacro{\childdocby}
Each part to be included by |\input| should start with:
%
\begin{center}
\begin{tabular}{l}
|\input{childdoc.def}|\\
|\childdocby{|\textit{main}|}|\\
\end{tabular}
\end{center}
%
The directive |\childdocby| is similar to |\childdocof|
described in \secref{sec:include},
but the subsequent selection of content must be done manually.
To that end, both |\ifchilddoc| and |\ifchilddocmanual|
will be true upon processing of a part,
and the name of the part is stored in |\childdocname|.
Note that |\jobname| will be set to the filename of the current part
so that each part receives an individual |.aux| file
that does not interfere with the |.aux| file(s) of the main document.
This behaviour can be altered by the alternative form
|\childdocby[*]{|\textit{main}|}| (with a non-empty optional argument)
which uses the |.aux| file of the main document
by setting |\jobname| to \textit{main}.

%%%%%%%%%%%%%%%%%%%%%%%%%%%%%%%%%%%%%%%%%%%%%%%%%%%%%%%%%%%%%%%%%%%%%%%%%%%%%%%%
\subsection{Driver Development}
\label{sec:driver}

The \textsf{childdoc} mechanism can also be use for the development
of definition files such as \LaTeX{} styles or classes.
This case differs from the above setup with multiple parts
included by |\include| in that no |\includeonly| should be invoked.
This can be achieved by starting the include file
(before |\ProvidesPackage|) with:
%
\begin{center}
\begin{tabular}{l}
|\input{childdoc.def}|\\
|\childdocforward{|\textit{main}|}|\\
\end{tabular}
\end{center}
%
or alternatively with:
%
\begin{center}
\begin{tabular}{l}
|\input{childdoc.def}|\\
|\childdocby{|\textit{main}|}|\\
\end{tabular}
\end{center}
%
Both forms have slightly different effects as described above.
The main file is prepared as usual, see \secref{sec:include}.

%%%%%%%%%%%%%%%%%%%%%%%%%%%%%%%%%%%%%%%%%%%%%%%%%%%%%%%%%%%%%%%%%%%%%%%%%%%%%%%%
\subsection{Legacy Detection}
\label{sec:detection}

The directive |\childdocmain| in the main file can detect
whether the complete document or merely a child is to be compiled
even without using the directive |\childdocof|.
This method is deprecated because it is less robust
and there is no compelling reason to use it;
it is merely provided for backward compatibility
and it may be removed in future versions.

If the detection mechanism is to be used,
it is mandatory to correctly specify
the filename of the main file as the argument of |\childdocmain|:
%
\begin{center}
\begin{tabular}{l}
|\input{childdoc.def}|\\
|\childdocmain{|\textit{main}|}|\\
\end{tabular}
\end{center}
%
If |\jobname| does not match the argument \textit{main} of |\childdocmain|,
it is assumed that |\jobname| points to the child file to be compiled.
When using |\childdocmain| with the main file specified as argument,
it suffices to start a child file
with just |\input{|\textit{main}|}|
without loading of the package and using |\childdocof|.
If instead all processing is done
with the appropriate \textsf{childdoc} directives,
the argument of \textit{main} of |\childdocmain| can be empty.

An alternative version of the command line processing described
in \secref{sec:commandline} using the detection mechanism reads:
%
\begin{center}
|... -jobname "|\textit{target}|" "|[\textit{flags}]%
[|\def\jobname{|\textit{dest}|}|]|\input{|\textit{main}|}"|
\end{center}

%%%%%%%%%%%%%%%%%%%%%%%%%%%%%%%%%%%%%%%%%%%%%%%%%%%%%%%%%%%%%%%%%%%%%%%%%%%%%%%%
\subsection{Manual Code}
\label{sec:manual}

In case one cannot be certain whether the definitions file |childdoc.def|
is installed on the target \TeX{} distribution
and one prefers not to ship it,
it is conceivable to paste a few relevant commands into the sources.

To that end, drop all statements |\input{childdoc.def}|
and perform the replacements as outlined below.
Instead of |\childdocmain{|\textit{main}|}| add the following code
to the top of the main file:
%
\begin{center}
\begin{tabular}{l}
|\||ifdefined\childdocname\endinput\||fi\newif\ifchilddoc|\\
|\edef\childdocname{\scantokens\expandafter{\jobname\noexpand}}|\\
|\def\childdocmain{|\textit{main}|}\||ifx\childdocmain\childdocname\||else|\\
|\childdoctrue\includeonly{\childdocname}\let\jobname\childdocmain\||fi|\\
\end{tabular}
\end{center}
%
Instead of |\childdocof{|\textit{main}|}| just include the main file
at the top of each child file:
%
\begin{center}
|\input{|\textit{main}|}|
\end{center}
%
A simple redirection |\childdocforward{|\textit{dest}|}| is achieved by:
%
\begin{center}
|\def\jobname{|\textit{dest}|}\input{\jobname}|
\end{center}
%
The redirection with prefix
|\childdocforwardprefix[|\textit{prefix}|]{|\textit{dest}|}|
is accomplished by:
%
\begin{center}
\begin{tabular}{l}
|{\edef\jobname{\scantokens\expandafter{\jobname\noexpand}}|\\
|\def\redirectjob |\textit{prefix}|#1~~~{\gdef\jobname{|\textit{dest}|#1}}|\\
|\expandafter\redirectjob\jobname~~~}\input{\jobname}|
\end{tabular}
\end{center}

In an alternative approach,
child documents can be compiled by a specific command line
without additional code or specific definitions:
%
\begin{center}
|... -jobname "|\textit{target}|" "|[\textit{flags}]%
|\includeonly{|\textit{dest}|}\input{|\textit{main}|}"|
\end{center}
%

%%%%%%%%%%%%%%%%%%%%%%%%%%%%%%%%%%%%%%%%%%%%%%%%%%%%%%%%%%%%%%%%%%%%%%%%%%%%%%%%
%%%%%%%%%%%%%%%%%%%%%%%%%%%%%%%%%%%%%%%%%%%%%%%%%%%%%%%%%%%%%%%%%%%%%%%%%%%%%%%%
\section{Information}

%%%%%%%%%%%%%%%%%%%%%%%%%%%%%%%%%%%%%%%%%%%%%%%%%%%%%%%%%%%%%%%%%%%%%%%%%%%%%%%%
\subsection{Copyright}

Copyright \copyright{} 2017--2018 Niklas Beisert

This work may be distributed and/or modified under the
conditions of the \LaTeX{} Project Public License, either version 1.3
of this license or (at your option) any later version.
The latest version of this license is in
  \url{http://www.latex-project.org/lppl.txt}
and version 1.3 or later is part of all distributions of \LaTeX{}
version 2005/12/01 or later.

This work has the LPPL maintenance status `maintained'.

The Current Maintainer of this work is Niklas Beisert.

This work consists of the files |README.txt|, |childdoc.ins| and |childdoc.dtx|
as well as the derived files |childdoc.def|, |cdocsamp.tex|
with |cdocsch1.tex|, |cdocsch2.tex|, |cdocspt3.tex|, |cdocspt4.tex|,
|cdocsdrf.tex|, |cdocsfn1.tex|, |cdocsfn2.tex|
as well as |childdoc.pdf|.

%%%%%%%%%%%%%%%%%%%%%%%%%%%%%%%%%%%%%%%%%%%%%%%%%%%%%%%%%%%%%%%%%%%%%%%%%%%%%%%%
\subsection{Files and Installation}

The package consists of the files:
%
\begin{center}
\begin{tabular}{ll}
    |README.txt|   & readme file \\
    |childdoc.ins| & installation file \\
    |childdoc.dtx| & source file \\
    |childdoc.def| & definition file \\
    |cdocsamp.tex| & sample main file \\
    |cdocsch1.tex| & sample include file \\
    |cdocsch2.tex| & sample include file \\
    |cdocspt3.tex| & sample part file \\
    |cdocspt4.tex| & sample part file \\
    |cdocsdrf.tex| & sample redirection file \\
    |cdocsfn1.tex| & sample redirection file \\
    |cdocsfn2.tex| & sample redirection file \\
    |childdoc.pdf| & manual
\end{tabular}
\end{center}
%
The distribution consists of the files
|README.txt|, |childdoc.ins| and |childdoc.dtx|.
%
\begin{itemize}
\item
Run (pdf)\LaTeX{} on |childdoc.dtx|
to compile the manual |childdoc.pdf| (this file).
\item
Run \LaTeX{} on |childdoc.ins| to create the definitions file |childdoc.def|
and the sample |cdocsamp.tex| with include files
|cdocsch1.tex|, |cdocsch2.tex|, |cdocspt3.tex|, |cdocspt4.tex|,
|cdocsdrf.tex|, |cdocsfn1.tex|, |cdocsfn2.tex|.
Then copy the file |childdoc.def| to an appropriate directory of your \LaTeX{}
distribution, e.g.\ \textit{texmf-root}|/tex/latex/childdoc|.
\end{itemize}

%%%%%%%%%%%%%%%%%%%%%%%%%%%%%%%%%%%%%%%%%%%%%%%%%%%%%%%%%%%%%%%%%%%%%%%%%%%%%%%%
\subsection{Related CTAN Packages}

There are several other packages which offer a similar functionality:
%
\begin{itemize}
\item
The packages
\href{http://ctan.org/pkg/docmute}{\textsf{docmute}},
\href{http://ctan.org/pkg/includex}{\textsf{includex}} and
\href{http://ctan.org/pkg/standalone}{\textsf{standalone}}
provide commands to include only the document body of
a child file thus allowing both files to be compiled individually.
\item
The packages \href{http://ctan.org/pkg/subdocs}{\textsf{subdocs}}
and \href{http://ctan.org/pkg/subfiles}{\textsf{subfiles}}
provide structures in which the main and child documents can be
encapsulated and allowing them to be compiled individually.
The inclusion mechanism is different from the conventional |\include|.
\item
The package \href{http://ctan.org/pkg/combine}{\textsf{combine}}
is an elaborate solution to combine several documents into one.
\end{itemize}
%
See also the CTAN topic \href{http://ctan.org/topic/subdocs}{\textsf{subdocs}}
for further related packages.
The present package differs from the above solutions in that
a document structure constructed with the conventional |\include| mechanism
just needs two extra commands at the top of every file
such that all constituent files can be compiled individually.

%%%%%%%%%%%%%%%%%%%%%%%%%%%%%%%%%%%%%%%%%%%%%%%%%%%%%%%%%%%%%%%%%%%%%%%%%%%%%%%%
%\subsection{Feature Suggestions}
%
%The following is a list of features which may be useful for future
%versions of this package:
%%
%\begin{itemize}
%\item
%\ldots
%\end{itemize}

%%%%%%%%%%%%%%%%%%%%%%%%%%%%%%%%%%%%%%%%%%%%%%%%%%%%%%%%%%%%%%%%%%%%%%%%%%%%%%%%
\subsection{Revision History}

%%%%%%%%%%%%%%%%%%%%%%%%%%%%%%%%%%%%%%%%
\paragraph{v2.0:} 2018/12/30

\begin{itemize}
\item
immediate forward processing
\item
added |\childdocby| mechanism
\item
manual restructured
\end{itemize}

%%%%%%%%%%%%%%%%%%%%%%%%%%%%%%%%%%%%%%%%
\paragraph{v1.6:} 2018/01/17

\begin{itemize}
\item
application for development of include files
\item
corrections to manual
\end{itemize}

%%%%%%%%%%%%%%%%%%%%%%%%%%%%%%%%%%%%%%%%
\paragraph{v1.5:} 2017/05/21

\begin{itemize}
\item
more complete structuring introduced
\item
|\childdocof| introduced
\item
|\childdoc| renamed to |\childdocmain|
\item
|\childredirect| renamed to |\childdocforward| and |\childdocforwardprefix|
and functionality expanded
\end{itemize}

%%%%%%%%%%%%%%%%%%%%%%%%%%%%%%%%%%%%%%%%
\paragraph{v1.0:} 2017/04/27

\begin{itemize}
\item
manual and install package
\item
first version published on CTAN
\end{itemize}

%%%%%%%%%%%%%%%%%%%%%%%%%%%%%%%%%%%%%%%%
\paragraph{v0.6:} 2017/04/26

\begin{itemize}
\item
redirection mechanism added
\end{itemize}

%%%%%%%%%%%%%%%%%%%%%%%%%%%%%%%%%%%%%%%%
\paragraph{v0.5:} 2017/04/26

\begin{itemize}
\item
functionality in definition file
\end{itemize}


%%%%%%%%%%%%%%%%%%%%%%%%%%%%%%%%%%%%%%%%%%%%%%%%%%%%%%%%%%%%%%%%%%%%%%%%%%%%%%%%
%%%%%%%%%%%%%%%%%%%%%%%%%%%%%%%%%%%%%%%%%%%%%%%%%%%%%%%%%%%%%%%%%%%%%%%%%%%%%%%%
%%%%%%%%%%%%%%%%%%%%%%%%%%%%%%%%%%%%%%%%%%%%%%%%%%%%%%%%%%%%%%%%%%%%%%%%%%%%%%%%
\appendix

\settowidth\MacroIndent{\rmfamily\scriptsize 000\ }

 \DocInput{childdoc.dtx}

\end{document}
%</driver>
% \fi
%
% %%%%%%%%%%%%%%%%%%%%%%%%%%%%%%%%%%%%%%%%%%%%%%%%%%%%%%%%%%%%%%%%%%%%%%%%%%%%%%
% %%%%%%%%%%%%%%%%%%%%%%%%%%%%%%%%%%%%%%%%%%%%%%%%%%%%%%%%%%%%%%%%%%%%%%%%%%%%%%
% \section{Sample}
%\iffalse
%<*samplemain>
%\fi
%
% The following presents a sample document
% with two chapters, two parts, a title page,
% a compile flag as well as three forwarding files to set the flag.
% It consists of eight |.tex| files:
% \begin{center}
% \begin{tabular}{ll}
% |cdocsamp.tex|&main file\\
% |cdocsch1.tex|&include file for chapter 1\\
% |cdocsch2.tex|&include file for chapter 2\\
% |cdocspt3.tex|&include file for part 3\\
% |cdocspt4.tex|&include file for part 4\\
% |cdocsdrf.tex|&forwarding file for main file in draft mode\\
% |cdocsfi1.tex|&forwarding file for final version of chapter 1\\
% |cdocsfi2.tex|&forwarding file for final version of chapter 2\\
% \end{tabular}
% \end{center}
% Each of the eight files can be compiled directly by the \LaTeX{} compiler.
%
% %%%%%%%%%%%%%%%%%%%%%%%%%%%%%%%%%%%%%%
% \paragraph{Main File.}
%
% The main file is called |cdocsamp.tex|.
%
% Load the \textsf{childdoc} definitions and
% declare the filename for the main document:
%    \begin{macrocode}
\input{childdoc.def}
\childdocmain{}
%    \end{macrocode}

% Optional override for |\version| flag:
%    \begin{macrocode}
%%\ifchilddoc\else\providecommand{\version}{draft}\fi
%    \end{macrocode}

% Define the default values for the |\version| flag
% (|final| for the main file and |draft| for childs):
%    \begin{macrocode}
\ifchilddoc
\providecommand{\version}{draft}
\else
\providecommand{\version}{final}
\fi
%    \end{macrocode}

% Load the standard document class:
%    \begin{macrocode}
\documentclass[12pt]{article}
%    \end{macrocode}

% Start the document body:
%    \begin{macrocode}
\begin{document}
%    \end{macrocode}

% Declare a title page.
% Print title, part of document being processed and version flag:
%    \begin{macrocode}
\addtocounter{page}{-1}
\begin{center}
{\LARGE\bfseries{}childdoc example\par}
\vspace{1cm}
\ifchilddoc
\ifchilddocmanual part\else chapter\fi:
`\childdocname' of `\childdocjob'\par
\else
main document: `\childdocjob'\par
\fi
version: \version\par
\end{center}
\newpage
%    \end{macrocode}

% Manually include selected file,
% otherwise process as usual:
%    \begin{macrocode}
\ifchilddocmanual
\section*{part `\childdocname'}
\input{\childdocname}
\else
%    \end{macrocode}

% Include the two chapters:
%    \begin{macrocode}
\include{cdocsch1}
\include{cdocsch2}
%    \end{macrocode}

% Include the two parts unless only chapters should be displayed:
%    \begin{macrocode}
\ifchilddoc\else
\section{part three}
\input{cdocspt3}
\section{part four}
\input{cdocspt4}
\fi
%    \end{macrocode}

% Process as usual until here:
%    \begin{macrocode}
\fi
%    \end{macrocode}

% End of document body:
%    \begin{macrocode}
\end{document}
%    \end{macrocode}
%\iffalse
%</samplemain>
%\fi
%
% %%%%%%%%%%%%%%%%%%%%%%%%%%%%%%%%%%%%%%
% \paragraph{Chapter Include Files.}
%
% The include files are called |cdocsch1.tex| and |cdocsch2.tex|.
%
%\iffalse
%<*samplechap1|samplechap2>
%\fi

% Optional override for |\version| flag:
%    \begin{macrocode}
%%\providecommand{\version}{final}
%    \end{macrocode}

% Include the main document:
%    \begin{macrocode}
\input{childdoc.def}
\childdocof{cdocsamp}
%    \end{macrocode}

%\iffalse
%</samplechap1|samplechap2>
%\fi
%
%\iffalse
%<*samplechap1>
%\fi
% Some text for chapter 1:
%    \begin{macrocode}
\section{one}
some text in chapter one
%    \end{macrocode}

%\iffalse
%</samplechap1>
%\fi
% Some text for chapter 2:
%\iffalse
%<*samplechap2>
%\fi
%    \begin{macrocode}
\section{two}
more text in chapter two
%    \end{macrocode}

%\iffalse
%</samplechap2>
%\fi
%
% %%%%%%%%%%%%%%%%%%%%%%%%%%%%%%%%%%%%%%
% \paragraph{Part Include Files.}
%
% The include files are called |cdocspt3.tex| and |cdocspt4.tex|.
%
%\iffalse
%<*samplepart3|samplepart4>
%\fi

% Optional override for |\version| flag:
%    \begin{macrocode}
%%\providecommand{\version}{final}
%    \end{macrocode}

% Include the main document:
%    \begin{macrocode}
\input{childdoc.def}
\childdocby{cdocsamp}
%    \end{macrocode}

%\iffalse
%</samplepart3|samplepart4>
%\fi
%
%\iffalse
%<*samplepart3>
%\fi
% Some text for part 3:
%    \begin{macrocode}
some text in part three
%    \end{macrocode}

%\iffalse
%</samplepart3>
%\fi
% Some text for part 4:
%\iffalse
%<*samplepart4>
%\fi
%    \begin{macrocode}
more text in part four
%    \end{macrocode}

%\iffalse
%</samplepart4>
%\fi
%
% %%%%%%%%%%%%%%%%%%%%%%%%%%%%%%%%%%%%%%
% \paragraph{Forwarding for a Complete Draft.}
%
% The following forwarding file |cdocsdrf.tex|
% compiles the main document in draft mode:
%\iffalse
%<*sampledraft>
%\fi
%    \begin{macrocode}
\def\version{draft}
\input{childdoc.def}
\childdocforward{cdocsamp}
%    \end{macrocode}

%\iffalse
%</sampledraft>
%\fi
%
% %%%%%%%%%%%%%%%%%%%%%%%%%%%%%%%%%%%%%%
% \paragraph{Forwarding for Final Version of the Chapters.}
%
% The following forwarding files |cdocsfn1.tex| and |cdocsfn2.tex|
% (with identical content)
% compile the final versions of the child documents
% |cdocsch1.tex| and |cdocsch2.tex|, respectively:
%\iffalse
%<*samplefinal>
%\fi
%    \begin{macrocode}
\def\version{final}
\input{childdoc.def}
\childdocforwardprefix[cdocsamp]{cdocsfn}{cdocsch}
%    \end{macrocode}

%\iffalse
%</samplefinal>
%\fi
%
% %%%%%%%%%%%%%%%%%%%%%%%%%%%%%%%%%%%%%%
% \paragraph{Command Line Processing.}
%
% The following three command lines generate the output files
% |cdocscld|, |cdocscl1| and |cdocscl2|
% which should be identical to
% |cdocsdrf|, |cdocsch1| and |cdocsfn2|, respectively:
% \begin{center}
% \begin{tabular}{l}
% |latex -jobname cdocscld \|\\
% |  "\def\version{draft}\input{childdoc.def}\childdocforward{cdocsamp}"|\\
% |latex -jobname cdocscl1 \|\\
% |  "\input{childdoc.def}\childdocforward[cdocsamp]{cdocsch1}"|\\
% |latex -jobname cdocscl2 \|\\
% |  "\def\version{final}\input{childdoc.def}\childdocforward{cdocsch2}"|
% \end{tabular}
% \end{center}
% Note that the trailing backslash on each first line
% merely continues the input to the second line
% (for convenient cut ant paste).
% Furthermore, the command |latex| can be replaced by any
% of its alternative versions such as |pdflatex|.
%
% %%%%%%%%%%%%%%%%%%%%%%%%%%%%%%%%%%%%%%%%%%%%%%%%%%%%%%%%%%%%%%%%%%%%%%%%%%%%%%
% %%%%%%%%%%%%%%%%%%%%%%%%%%%%%%%%%%%%%%%%%%%%%%%%%%%%%%%%%%%%%%%%%%%%%%%%%%%%%%
% \section{Implementation}
%\iffalse
%<*package>
%\fi
%
% This section describes the definitions file |childdoc.def|.

% The definitions cannot be loaded using |\usepackage| or |\RequirePackage|
% which has a mechanism to prevent loading a style file more than once.
% When loading the definitions by means of |\input|
% multiple instances have to be prevented manually:
%\iffalse
%This code needs to be before the `\ProvidesFile' directive
%which is defined at the beginning of this file.
%Therefore it is also placed there and commented out here.
%</package>
%<*discard>
%\fi
%    \begin{macrocode}
\ifdefined\childdocmain\endinput\fi
%    \end{macrocode}
%\iffalse
%</discard>
%<*package>
%\fi
%
% \macro{\ifchilddoc}
% \macro{\ifchilddocmanual}
% The conditional |\ifchilddoc| tells whether a
% child (true) or main (false) document is being compiled.
% The conditional |\ifchilddocmanual| tells whether
% the |\includeonly| mechanism is used (false) or
% the selection of child files must be performed manually (true).
% The definitions initialise to false:
%    \begin{macrocode}
\newif\ifchilddoc
\newif\ifchilddocmanual
%    \end{macrocode}

% \macro{\childdocname}
% \macro{\childdocjob}
% The macro |\childdocname| stores the name of the main document
% to be compiled. The macro |\childdocjob| stores the name of
% the document on which the \LaTeX{} compiler was originally invoked.
% The content of |\jobname| cannot be compared
% to filenames specified in the source due to different catcodes.
% The following code rescans |\jobname|, stores the result
% in |\childdocname| and saves a copy in |\childdocjob|:
%    \begin{macrocode}
\edef\childdocname{\scantokens\expandafter{\jobname\noexpand}}
\let\childdocjob\childdocname
%    \end{macrocode}

% \macro{\childdocdisable}
% The macro |\childdocdisable| prevents the main file
% from being processed more than once.
% At this stage, the main document command |\childdocmain|
% is assumed to be called once again where it should do nothing.
% Any subsequent call to it should prevent
% a secondary processing of the main document
% It overwrites the forwarding commands
% |\childdocof| and |\childdocforward|
% with empty macros to prevent further inclusions of the main document:
%    \begin{macrocode}
\newcommand{\childdocdisable}
{
  \renewcommand{\childdocmain}[1]{\renewcommand{\childdocmain}[1]{\endinput}}
  \renewcommand{\childdocof}[1]{}
  \renewcommand{\childdocby}[2][]{}
  \renewcommand{\childdocforward}[2][]{}
  \renewcommand{\childdocdisable}{}
}
%    \end{macrocode}

% \macro{\childdocmain}
% The macro |\childdocmain| is to be called at the top of the main file
% with nothing or the main filename (without extension) as argument.
% First, it breaks loops.
% If the argument is not empty and does not match |\childdocname|
% (which is set by the first inclusion of |childdoc.def|),
% |\ifchilddoc| is set to true, |\includeonly| is applied to the child file
% and |\jobname| is set to the main file
% (for proper handling of |.aux| files):
%    \begin{macrocode}
\newcommand{\childdocmain}[1]
{
  \childdocdisable\childdocmain{}
  \if?#1?\else
    \begingroup
      \def\childdoctmp{#1}
      \ifx\childdoctmp\childdocname
        \def\childdoctmp{}
      \else
        \def\childdoctmp
        {
          \childdoctrue
          \includeonly{\childdocname}
          \def\childdocjob{#1}
          \def\jobname{#1}
        }
      \fi
      \expandafter
    \endgroup
    \childdoctmp
  \fi
}
%    \end{macrocode}

% \macro{\childdocof}
% The command |\childdocof| redirects
% compilation to the main file |#1|.
%    \begin{macrocode}
\newcommand{\childdocof}[1]
{
  \childdocdisable
  \childdoctrue
  \includeonly{\childdocname}
  \def\jobname{#1}
  \def\childdocjob{#1}
  \input{#1}
}
%    \end{macrocode}

% \macro{\childdocby}
% The command |\childdocby| ....
%    \begin{macrocode}
\newcommand{\childdocby}[2][]
{
  \childdocdisable
  \childdoctrue
  \childdocmanualtrue
  \if?#1?\else
    \def\jobname{#2}
  \fi
  \def\childdocjob{#2}
  \input{#2}
  \endinput
}
%    \end{macrocode}

% \macro{\childdocforward}
% The command |\childdocforward| redirects
% compilation to the main file or
% (if the optional argument is given) a child file.
% Parameters are set as if the main file
% or a child file starting with |\childdocof| was compiled.
% Then compilation is handed over to the main file:
%    \begin{macrocode}
\newcommand{\childdocforward}[2][]
{
  \begingroup
    \if?#1?
      \def\childdoctmp
      {
        \def\childdocname{#2}
        \def\childdocjob{#2}
        \def\jobname{#2}
        \input{#2}
        \endinput
      }
    \else
      \def\childdoctmp
      {
        \childdocdisable
        \def\childdocname{#2}
        \childdoctrue
        \includeonly{#2}
        \def\childdocjob{#1}
        \def\jobname{#1}
        \input{#1}
        \endinput
      }
    \fi
    \expandafter
  \endgroup
  \childdoctmp
}
%    \end{macrocode}

% \macro{\childdocforwardprefix}
% The command |\childdocforwardprefix| redirects
% compilation to the main or a child file by means of a pattern.
% The prefix |#1| in the current filename is replaced by |#2|
% and the suffix of the current filename is kept
% (it is assumed that the filename does not contain the substring `|~~~|'
% which is used as a delimiter).
% Compilation is handed over to the new file by |\childdocforward|:
%    \begin{macrocode}
\newcommand{\childdocforwardprefix}[3][]
{
  \begingroup
    \def\childdocextract #2##1~~~{\def\childdoctmp{\childdocforward[#1]{#3##1}}}
    \expandafter\childdocextract\childdocname~~~
    \expandafter
  \endgroup
  \childdoctmp
}
%    \end{macrocode}

% \macro{\childdoc}
% The deprecated macro |\childdoc| is a legacy version of |\childdocmain|:
%    \begin{macrocode}
\newcommand{\childdoc}{\childdocmain}
%    \end{macrocode}

% \macro{\childdocredirect}
% The deprecated macro |\childdocredirect| is a legacy version
% of |\childdocforward| and |\childdocforwardprefix|:
%    \begin{macrocode}
\newcommand{\childdocredirect}[2][]
{
  \begingroup
    \if?#1?
      \def\childdoctmp{\childdocforward{#2}}
    \else
      \def\childdoctmp{\childdocforwardprefix{#1}{#2}}
    \fi
    \expandafter
  \endgroup
  \childdoctmp
}
%    \end{macrocode}

%\iffalse
%</package>
%\fi
%
\endinput
|\\
|\childdocmain{|\textit{main}|}|\\
\end{tabular}
\end{center}
%
If |\jobname| does not match the argument \textit{main} of |\childdocmain|,
it is assumed that |\jobname| points to the child file to be compiled.
When using |\childdocmain| with the main file specified as argument,
it suffices to start a child file
with just |\input{|\textit{main}|}|
without loading of the package and using |\childdocof|.
If instead all processing is done
with the appropriate \textsf{childdoc} directives,
the argument of \textit{main} of |\childdocmain| can be empty.

An alternative version of the command line processing described
in \secref{sec:commandline} using the detection mechanism reads:
%
\begin{center}
|... -jobname "|\textit{target}|" "|[\textit{flags}]%
[|\def\jobname{|\textit{dest}|}|]|\input{|\textit{main}|}"|
\end{center}

%%%%%%%%%%%%%%%%%%%%%%%%%%%%%%%%%%%%%%%%%%%%%%%%%%%%%%%%%%%%%%%%%%%%%%%%%%%%%%%%
\subsection{Manual Code}
\label{sec:manual}

In case one cannot be certain whether the definitions file |childdoc.def|
is installed on the target \TeX{} distribution
and one prefers not to ship it,
it is conceivable to paste a few relevant commands into the sources.

To that end, drop all statements |% \iffalse
%
% childdoc.dtx Copyright (C) 2017-2018 Niklas Beisert
%
% This work may be distributed and/or modified under the
% conditions of the LaTeX Project Public License, either version 1.3
% of this license or (at your option) any later version.
% The latest version of this license is in
%   http://www.latex-project.org/lppl.txt
% and version 1.3 or later is part of all distributions of LaTeX
% version 2005/12/01 or later.
%
% This work has the LPPL maintenance status `maintained'.
%
% The Current Maintainer of this work is Niklas Beisert.
%
% This work consists of the files childdoc.dtx and childdoc.ins
% and the derived files childdoc.def and cdocsamp.tex with
% cdocsch1.tex, cdocsch2.tex, cdocsdrf.tex, cdocsfn1.tex, cdocsfn2.tex.
%
%<package>\ifdefined\childdocmain\endinput\fi
%<package>\ProvidesFile{childdoc.def}[2018/12/30 v2.0 child document driver]
%<samplemain>\ProvidesFile{cdocsamp.tex}[2018/12/30 v2.0 sample for childdoc]
%<*driver>
%\ProvidesFile{childdoc.drv}[2018/12/30 v2.0 childdoc reference manual file]
\PassOptionsToClass{10pt,a4paper}{article}
\documentclass{ltxdoc}

\usepackage[margin=35mm]{geometry}
\usepackage{hyperref}
\usepackage{hyperxmp}
\usepackage[usenames]{color}

\hypersetup{colorlinks=true}
\hypersetup{pdfstartview=FitH}
\hypersetup{pdfpagemode=UseNone}
\hypersetup{pdfsource={}}
\hypersetup{pdflang={en-UK}}
\hypersetup{pdfcopyright={Copyright 2017-2018 Niklas Beisert.
  This work may be distributed and/or modified under the
  conditions of the LaTeX Project Public License, either version 1.3
  of this license or (at your option) any later version.}}
\hypersetup{pdflicenseurl={http://www.latex-project.org/lppl.txt}}
\hypersetup{pdfcontactaddress={ETH Zurich, ITP, HIT K,
  Wolfgang-Pauli-Strasse 27}}
\hypersetup{pdfcontactpostcode={8093}}
\hypersetup{pdfcontactcity={Zurich}}
\hypersetup{pdfcontactcountry={Switzerland}}
\hypersetup{pdfcontactemail={nbeisert@itp.phys.ethz.ch}}
\hypersetup{pdfcontacturl={http://people.phys.ethz.ch/\xmptilde nbeisert/}}

\newcommand{\secref}[1]{\hyperref[#1]{section \ref*{#1}}}

\parskip1ex
\parindent0pt
\let\olditemize\itemize
\def\itemize{\olditemize\parskip0pt}

\begin{document}

\title{The \textsf{childdoc} Package}
\hypersetup{pdftitle={The childdoc Package}}
\author{Niklas Beisert\\[2ex]
  Institut f\"ur Theoretische Physik\\
  Eidgen\"ossische Technische Hochschule Z\"urich\\
  Wolfgang-Pauli-Strasse 27, 8093 Z\"urich, Switzerland\\[1ex]
  \href{mailto:nbeisert@itp.phys.ethz.ch}
  {\texttt{nbeisert@itp.phys.ethz.ch}}}
\hypersetup{pdfauthor={Niklas Beisert}}
\hypersetup{pdfsubject={Manual for the LaTeX2e Package childdoc}}
\date{30 December 2018, \textsf{v2.0}}
\maketitle

\begin{abstract}\noindent
\textsf{childdoc} is a \LaTeXe{} package
that enables the direct compilation
of document sections included by |\include|
to individual files.
\end{abstract}

\begingroup
\parskip0ex
\tableofcontents
\endgroup

%%%%%%%%%%%%%%%%%%%%%%%%%%%%%%%%%%%%%%%%%%%%%%%%%%%%%%%%%%%%%%%%%%%%%%%%%%%%%%%%
%%%%%%%%%%%%%%%%%%%%%%%%%%%%%%%%%%%%%%%%%%%%%%%%%%%%%%%%%%%%%%%%%%%%%%%%%%%%%%%%
\section{Introduction}

\LaTeX{} provides a mechanism to structure a large document (such as a book)
into a main file and several child files (containing the chapters)
using the |\include| command.
This mechanism is beneficial for documents
which span hundreds of pages in order to
make the source file(s) more manageable.
Moreover, compilation can be restricted to
selected child files by means of the |\includeonly| command.
The latter feature can be used to reduce the compilation time while editing
(this was significantly more useful in the earlier days of \LaTeX{})
or to generate a smaller document which is easier to navigate.
Another application of |\includeonly| is to generate
documents consisting of selected parts of the complete document.

However, there are a few drawbacks of the plain |\include| mechanism:
\begin{itemize}
\item
The child files cannot be compiled on their own,
they can only be compiled via the main file.
A naive editing environment
(such as a text editor with an option
to have the current file processed by \LaTeX)
may require one to switch to the main file before compiling;
attempting to compile the child file produces errors.
\item
The main file must be modified (each time)
to adjust the |\includeonly| command
to the present needs. This easily leaves the main file in a messy state.
\item
The generated document will always carry the filename
of the main document. This is inconvenient if
several child files are to be compiled and
to be kept for distribution.
\end{itemize}

The present package provides a simple interface
to make child files individually compilable by \LaTeX{}.
Compiling a child file then has the same effect as compiling
the main file with an |\includeonly| command
to select the appropriate child.
Moreover the generated document will carry the name of the child
rather than the main file.
This resolves all three above issues.

This feature is meant to make the editing of books,
thesis documents and lecture notes somewhat more convenient.
However, the package can also be used efficiently for
composing a series of documents (such as exercise sheets)
which are typically distributed individually.
It then assists the author in generating the individual documents
(potentially in different versions)
as well as a document containing the collected series.
Another application is in developing style files
or other kinds of included material
where compilation of the style file could redirect
to a sample or test file.

%%%%%%%%%%%%%%%%%%%%%%%%%%%%%%%%%%%%%%%%%%%%%%%%%%%%%%%%%%%%%%%%%%%%%%%%%%%%%%%%
%%%%%%%%%%%%%%%%%%%%%%%%%%%%%%%%%%%%%%%%%%%%%%%%%%%%%%%%%%%%%%%%%%%%%%%%%%%%%%%%
\section{Usage}

First of all, the package \textsf{childdoc} is \emph{not} a standard
\LaTeXe{} |.sty| style file! Therefore it needs to be invoked in
a non-standard way.

%%%%%%%%%%%%%%%%%%%%%%%%%%%%%%%%%%%%%%%%%%%%%%%%%%%%%%%%%%%%%%%%%%%%%%%%%%%%%%%%
\subsection{Included Files}
\label{sec:include}

%%%%%%%%%%%%%%%%%%%%%%%%%%%%%%%%%%%%%%%%
\DescribeMacro{\childdocmain}
To use the package, add the commands
\begin{center}
\begin{tabular}{l}
|\input{childdoc.def}|\\
|\childdocmain{}|\\
\end{tabular}
\end{center}
at the very top of the main \LaTeX{} file,
in particular \emph{before} the |\documentclass| statement!
The argument of |\childdocmain| should be left empty
(but it must be present).

%%%%%%%%%%%%%%%%%%%%%%%%%%%%%%%%%%%%%%%%
\DescribeMacro{\childdocof}
Furthermore, add the commands
\begin{center}
\begin{tabular}{l}
|\input{childdoc.def}|\\
|\childdocof{|\textit{main}|}|\\
\end{tabular}
\end{center}
at the top of every child file \textit{child}
which is included by |\include{|\textit{child}|}|
from within the main file
(or at least for those files to be compiled individually).
The argument \textit{main} must be the filename of the main file.

There are a couple of
considerations in setting up the main and child documents:

%%%%%%%%%%%%%%%%%%%%%%%%%%%%%%%%%%%%%%%%
\paragraph{Restrictions.}

Please note the following restrictions:
\begin{itemize}
\item
|\childdocmain| must be called with one argument \textit{main}
to ensure compatibility with earlier version of the package.
It must either be empty (|\childdocmain{}|)
or precisely match the filename of the main file in which it is specified.
See \secref{sec:detection} for further information.
\item
The filename \textit{main} must be specified without the |.tex| extension.
\item
The filename \textit{main} is case sensitive
(even in case-insensitive file systems)
due to internal string comparison.
\item
The argument \textit{main} should be fully expanded, it cannot be a macro.
\item
Subdirectories and special characters should be avoided in filenames.
\item
The command |\childdocmain{|\textit{main}|}| must be followed by a whitespace.
It should not be followed immediately by another command
or by a comment mark `|%|'.
This is because the \TeX{} parser reads the token immediately following
the argument of |\childdocmain| and puts it
at the beginning of every child section;
however, a white\-space is ignored.
\end{itemize}

%%%%%%%%%%%%%%%%%%%%%%%%%%%%%%%%%%%%%%%%
\paragraph{Content of Main File.}

It is advisable to place all content in the child files included by |\include|.
Any output contained in the main file will appear in all child documents
unless suppressed manually;
it cannot be suppressed automatically by the |\includeonly| directive
and thus should normally be avoided.
A method to include some content in the main file
by means of conditional processing is described in \secref{sec:conditional}.

%%%%%%%%%%%%%%%%%%%%%%%%%%%%%%%%%%%%%%%%
\paragraph{Page Numbering.}

When only a part of the document is compiled,
the appropriate numbering of pages
(as well as other status parameters)
is determined from the |.aux| files.
The latter contain information from previous passes.
However this information needs to propagate through
all intermediate child documents.
Therefore the page numbering in child documents may well
be inconsistent until the complete document is compiled at least once.

A useful (if unconventional) way to always ensure a consistent
page numbering is to restart the numbering in each child document
and denote the pages by `\textit{child}|.|\textit{page}'
where \textit{child} represents the chapter/section number of the child file.
This can be achieved by the command
|\numberwithin{page}{|\textit{child}|}|
of the \textsf{amsmath} package
where \textit{child} can be |chapter| or |section|
depending on the chosen structuring.
Alternatively, one can modify the macro |\thepage| appropriately
and reset the counter |page| at the start of each child file.

%%%%%%%%%%%%%%%%%%%%%%%%%%%%%%%%%%%%%%%%%%%%%%%%%%%%%%%%%%%%%%%%%%%%%%%%%%%%%%%%
\subsection{Conditional Processing}
\label{sec:conditional}

The package provides a mechanism to compile different versions
of a document. To customise the versions further some conditional processing
can come in handy to distinguish which version is being compiled.
The package provides two macros to describe the compilation context:

%%%%%%%%%%%%%%%%%%%%%%%%%%%%%%%%%%%%%%%%
\DescribeMacro{\ifchilddoc}
The conditional |\ifchilddoc| distinguishes between the compilation of
child documents and the main document:
%
\begin{center}
|\ifchilddoc |\textit{child-code}| |[|\||else |\textit{main-code}]| \||fi|
\end{center}

%%%%%%%%%%%%%%%%%%%%%%%%%%%%%%%%%%%%%%%%
\DescribeMacro{\childdocname}
\DescribeMacro{\childdocjob}
The macro |\childdocname| contains the filename (without extension)
of the main or child file being processed.
Note that |\childdocjob| will always contain the name of the main file.

%%%%%%%%%%%%%%%%%%%%%%%%%%%%%%%%%%%%%%%%
\paragraph{Title Page.}

Conditional processing can be used to include a title or banner page
in the main document when proper precautions are taken.
Importantly, the code in the main file should ensure that the page counter
(as well as other status parameters which are stored in the |.aux| files)
takes the same value after the conditional processing.
Otherwise the page numbers may take divergent values
depending on which part is compiled.

For example, a title page could be declared by:
%
\begin{center}
\begin{tabular}{l}
|\ifchilddoc\||else|\\
|\addtocounter{page}{-1}|\\
\textit{code for title page}\\
|\newpage|\\
|\||fi|
\end{tabular}
\end{center}
%
A banner page for the child documents can be generated by:
%
\begin{center}
\begin{tabular}{l}
|\ifchilddoc|\\
|\addtocounter{page}{-1}|\\
\textit{code for banner page}\\
|\newpage|\\
|\||fi|
\end{tabular}
\end{center}
%
Here one could write a message such as:
\begin{center}
|This is the part \childdocname{} of \childdocjob{}.|
\end{center}

%%%%%%%%%%%%%%%%%%%%%%%%%%%%%%%%%%%%%%%%%%%%%%%%%%%%%%%%%%%%%%%%%%%%%%%%%%%%%%%%
\subsection{Flags}
\label{sec:flags}

The package makes it easy to generate different versions
of the main or child documents.
To this end compilation flags can be defined
and assigned different default values.
They will be particularly useful in conjunction
with the forwarding mechanism described in \secref{sec:forward}.

For example, it may be useful to have a flag |\version|
which can be set to |draft| or |final|.
The document source will contain some conditional code
depending on the value of |\version|.
Suppose further, the flag should default to |final| for the main file
and to |draft| for child files
which is a natural assignment for editing the document.
This is achieved by placing the following code
in the preamble of the main document
(below the |\childdocmain| directive):
%
\begin{center}
\begin{tabular}{l}
|\ifchilddoc|\\
|\providecommand{\version}{draft}|\\
|\||else|\\
|\providecommand{\version}{final}|\\
|\||fi|
\end{tabular}
\end{center}
%
The definition by |\providecommand| makes sure
that previous definitions are not overwritten.
Further statements |\providecommand{\version}{...}|
can thus be added before the above code to override it.

For the main file, one might add a line
(between |\childdocmain| and the above block)
%
\begin{center}
|%\ifchilddoc\||else\providecommand{\version}{draft}\||fi|
\end{center}
%
which can be uncommented to produce a draft version.
Likewise one can add a line to the very top of a child file
(above the |\childdocof{|\textit{main}|}| directive)
%
\begin{center}
|%\providecommand{\version}{final}|
\end{center}
%
which can be uncommented to produce the final version of this child document.

%%%%%%%%%%%%%%%%%%%%%%%%%%%%%%%%%%%%%%%%%%%%%%%%%%%%%%%%%%%%%%%%%%%%%%%%%%%%%%%%
\subsection{Forwarding}
\label{sec:forward}

Different versions of the main or child documents
using compilation flags as described in \secref{sec:flags}
can be (permanently) stored in different files
for convenient compilation, viewing and distribution.
To this end, the package defines a command
to pass on compilation to a different file:

%%%%%%%%%%%%%%%%%%%%%%%%%%%%%%%%%%%%%%%%
\DescribeMacro{\childdocforward}
The command |\childdocforward| redirects processing to
another source file:
%
\begin{center}
\begin{tabular}{l}
|\input{childdoc.def}|\\
|\childdocforward[|\textit{main}|]{|\textit{dest}|}|\\
\end{tabular}
\end{center}
%
The argument \textit{dest} is the destination file
(without extension).
It should be the main file or one of the child files.
Note that further \textsf{childdoc} directives
such as |\childdocof| and |\childdocforward|
in the indicated file will be processed in this form.
The optional argument \textit{main}
passes on directly to the main file \textit{main}
while pretending to compile the child \textit{dest}.
This form behaves as if \textit{dest}
issues |\childdocof{|\textit{main}|}| right away,
and no further \textsf{childdoc} directives will be processed.

%%%%%%%%%%%%%%%%%%%%%%%%%%%%%%%%%%%%%%%%
\DescribeMacro{\...prefix}
In the alternative form |\childdocforwardprefix|,
%
\begin{center}
\begin{tabular}{l}
|\input{childdoc.def}|\\
|\childdocforwardprefix[|\textit{main}|]{|\textit{prefix}|}{|\textit{dest}|}|
\end{tabular}
\end{center}
%
the destination file is determined by a pattern
depending on the current file:
To make this work, the current file must be called
`{\textit{prefix}\hspace{0.2em}\textit{suffix}}'
with \textit{prefix} matching precisely the argument.
Processing is then passed on to the file
`{\textit{dest}\hspace{0.2em}\textit{suffix}}'.
Surely, the same effect is achieved by
directly specifying the
argument `{\textit{dest}\hspace{0.2em}\textit{suffix}}'
in the first form.
However, that requires to set up a different file
for each child. With the alternative form of the command
all these files can have exactly the same content
which simplifies setting them up and maintaining them.

For example, the following file |draft.tex|
with a compilation flag |\version| as described in \secref{sec:flags}
compiles the main document as a draft:
%
\begin{center}
\begin{tabular}{l}
|\def\version{draft}|\\
|\input{childdoc.def}|\\
|\childdocforward{|\textit{main}|}|
\end{tabular}
\end{center}
%
Likewise, the following files |final|\textit{nn}|.tex|
compile the final version of the child document
|child|\textit{nn}|.tex|:
%
\begin{center}
\begin{tabular}{l}
|\def\version{final}|\\
|\input{childdoc.def}|\\
|\childdocforwardprefix{final}{child}|
\end{tabular}
\end{center}
%

Note that when several versions of a main file and/or of each child file
are to be generated, it may be convenient to set up a |Makefile| or
shell script to automatise the process.

%%%%%%%%%%%%%%%%%%%%%%%%%%%%%%%%%%%%%%%%%%%%%%%%%%%%%%%%%%%%%%%%%%%%%%%%%%%%%%%%
\subsection{Command Line Processing}
\label{sec:commandline}

The effect of redirection files can also be achieved by invoking
the \LaTeX{} compiler with a more elaborate command line.
Most conveniently this should be done as part
of a shell script or a |Makefile|.

When using \textsf{childdoc} in the main file, the following
command lines effectively perform a redirection
(note that depending on the shell being used,
backslashes may have to be doubled: `|\|' $\to$ `|\\|'):
%
\begin{center}
|... -jobname "|\textit{target}|" |\\|"|[\textit{flags}]%
|\input{childdoc.def}\childdocforward[|\textit{main}|]{|\textit{dest}|}"|
\end{center}
%
Here \textit{target} is the name of the output file,
\textit{main} is the name of the main file
and \textit{dest} is the name of the main or child file to be processed
(all filenames without extensions).
The optional argument \textit{main} can be omitted
if \textit{main} matches \textit{dest}.
Optionally, compilation \textit{flags} can be defined via |\def| commands.
This command line makes the \TeX{} engine believe
it is compiling the file \textit{target}
whose content is specified as the latter parameter.
The provided code then forwards the processing to
\textit{main} or \textit{dest} as described in \secref{sec:forward}.

%%%%%%%%%%%%%%%%%%%%%%%%%%%%%%%%%%%%%%%%%%%%%%%%%%%%%%%%%%%%%%%%%%%%%%%%%%%%%%%%
\subsection{Include by Input}
\label{sec:input}

Including child documents by |\include| has some restrictions by design.
Most notably, the content of a child document always occupies
its own set of pages; pages cannot be shared between child documents.
Usually, this behaviour makes perfect sense
because each child document contain an essential part of the document.
However, in some situations it may be desirable to compose
a document from a collection of parts
without having mandatory page breaks between then.
For this case, the package
provides a mechanism to include parts
by |\input| which can also be processed individually.
However, by construction this mechanism
requires manual handling of the content to be output.

%%%%%%%%%%%%%%%%%%%%%%%%%%%%%%%%%%%%%%%%
\DescribeMacro{\ifchilddocmanual}
The main file should be prepared as usual, see \secref{sec:include}.
However, the document body must make a distinction
between processing of an individual part and of the main document, e.g.:
%
\begin{center}
\begin{tabular}{l}
|\ifchilddocmanual|\\
|\input{\childdocname}|\\
|\||else|\\
\textit{document body with }|\input{|\textit{part}|}|\\
|\||fi|
\end{tabular}
\end{center}
%
The conditional |\ifchilddocmanual| is true whenever
a part to be included by |\input| is being compiled,
and the name of the part is stored in |\childdocname|.

%%%%%%%%%%%%%%%%%%%%%%%%%%%%%%%%%%%%%%%%
\DescribeMacro{\childdocby}
Each part to be included by |\input| should start with:
%
\begin{center}
\begin{tabular}{l}
|\input{childdoc.def}|\\
|\childdocby{|\textit{main}|}|\\
\end{tabular}
\end{center}
%
The directive |\childdocby| is similar to |\childdocof|
described in \secref{sec:include},
but the subsequent selection of content must be done manually.
To that end, both |\ifchilddoc| and |\ifchilddocmanual|
will be true upon processing of a part,
and the name of the part is stored in |\childdocname|.
Note that |\jobname| will be set to the filename of the current part
so that each part receives an individual |.aux| file
that does not interfere with the |.aux| file(s) of the main document.
This behaviour can be altered by the alternative form
|\childdocby[*]{|\textit{main}|}| (with a non-empty optional argument)
which uses the |.aux| file of the main document
by setting |\jobname| to \textit{main}.

%%%%%%%%%%%%%%%%%%%%%%%%%%%%%%%%%%%%%%%%%%%%%%%%%%%%%%%%%%%%%%%%%%%%%%%%%%%%%%%%
\subsection{Driver Development}
\label{sec:driver}

The \textsf{childdoc} mechanism can also be use for the development
of definition files such as \LaTeX{} styles or classes.
This case differs from the above setup with multiple parts
included by |\include| in that no |\includeonly| should be invoked.
This can be achieved by starting the include file
(before |\ProvidesPackage|) with:
%
\begin{center}
\begin{tabular}{l}
|\input{childdoc.def}|\\
|\childdocforward{|\textit{main}|}|\\
\end{tabular}
\end{center}
%
or alternatively with:
%
\begin{center}
\begin{tabular}{l}
|\input{childdoc.def}|\\
|\childdocby{|\textit{main}|}|\\
\end{tabular}
\end{center}
%
Both forms have slightly different effects as described above.
The main file is prepared as usual, see \secref{sec:include}.

%%%%%%%%%%%%%%%%%%%%%%%%%%%%%%%%%%%%%%%%%%%%%%%%%%%%%%%%%%%%%%%%%%%%%%%%%%%%%%%%
\subsection{Legacy Detection}
\label{sec:detection}

The directive |\childdocmain| in the main file can detect
whether the complete document or merely a child is to be compiled
even without using the directive |\childdocof|.
This method is deprecated because it is less robust
and there is no compelling reason to use it;
it is merely provided for backward compatibility
and it may be removed in future versions.

If the detection mechanism is to be used,
it is mandatory to correctly specify
the filename of the main file as the argument of |\childdocmain|:
%
\begin{center}
\begin{tabular}{l}
|\input{childdoc.def}|\\
|\childdocmain{|\textit{main}|}|\\
\end{tabular}
\end{center}
%
If |\jobname| does not match the argument \textit{main} of |\childdocmain|,
it is assumed that |\jobname| points to the child file to be compiled.
When using |\childdocmain| with the main file specified as argument,
it suffices to start a child file
with just |\input{|\textit{main}|}|
without loading of the package and using |\childdocof|.
If instead all processing is done
with the appropriate \textsf{childdoc} directives,
the argument of \textit{main} of |\childdocmain| can be empty.

An alternative version of the command line processing described
in \secref{sec:commandline} using the detection mechanism reads:
%
\begin{center}
|... -jobname "|\textit{target}|" "|[\textit{flags}]%
[|\def\jobname{|\textit{dest}|}|]|\input{|\textit{main}|}"|
\end{center}

%%%%%%%%%%%%%%%%%%%%%%%%%%%%%%%%%%%%%%%%%%%%%%%%%%%%%%%%%%%%%%%%%%%%%%%%%%%%%%%%
\subsection{Manual Code}
\label{sec:manual}

In case one cannot be certain whether the definitions file |childdoc.def|
is installed on the target \TeX{} distribution
and one prefers not to ship it,
it is conceivable to paste a few relevant commands into the sources.

To that end, drop all statements |\input{childdoc.def}|
and perform the replacements as outlined below.
Instead of |\childdocmain{|\textit{main}|}| add the following code
to the top of the main file:
%
\begin{center}
\begin{tabular}{l}
|\||ifdefined\childdocname\endinput\||fi\newif\ifchilddoc|\\
|\edef\childdocname{\scantokens\expandafter{\jobname\noexpand}}|\\
|\def\childdocmain{|\textit{main}|}\||ifx\childdocmain\childdocname\||else|\\
|\childdoctrue\includeonly{\childdocname}\let\jobname\childdocmain\||fi|\\
\end{tabular}
\end{center}
%
Instead of |\childdocof{|\textit{main}|}| just include the main file
at the top of each child file:
%
\begin{center}
|\input{|\textit{main}|}|
\end{center}
%
A simple redirection |\childdocforward{|\textit{dest}|}| is achieved by:
%
\begin{center}
|\def\jobname{|\textit{dest}|}\input{\jobname}|
\end{center}
%
The redirection with prefix
|\childdocforwardprefix[|\textit{prefix}|]{|\textit{dest}|}|
is accomplished by:
%
\begin{center}
\begin{tabular}{l}
|{\edef\jobname{\scantokens\expandafter{\jobname\noexpand}}|\\
|\def\redirectjob |\textit{prefix}|#1~~~{\gdef\jobname{|\textit{dest}|#1}}|\\
|\expandafter\redirectjob\jobname~~~}\input{\jobname}|
\end{tabular}
\end{center}

In an alternative approach,
child documents can be compiled by a specific command line
without additional code or specific definitions:
%
\begin{center}
|... -jobname "|\textit{target}|" "|[\textit{flags}]%
|\includeonly{|\textit{dest}|}\input{|\textit{main}|}"|
\end{center}
%

%%%%%%%%%%%%%%%%%%%%%%%%%%%%%%%%%%%%%%%%%%%%%%%%%%%%%%%%%%%%%%%%%%%%%%%%%%%%%%%%
%%%%%%%%%%%%%%%%%%%%%%%%%%%%%%%%%%%%%%%%%%%%%%%%%%%%%%%%%%%%%%%%%%%%%%%%%%%%%%%%
\section{Information}

%%%%%%%%%%%%%%%%%%%%%%%%%%%%%%%%%%%%%%%%%%%%%%%%%%%%%%%%%%%%%%%%%%%%%%%%%%%%%%%%
\subsection{Copyright}

Copyright \copyright{} 2017--2018 Niklas Beisert

This work may be distributed and/or modified under the
conditions of the \LaTeX{} Project Public License, either version 1.3
of this license or (at your option) any later version.
The latest version of this license is in
  \url{http://www.latex-project.org/lppl.txt}
and version 1.3 or later is part of all distributions of \LaTeX{}
version 2005/12/01 or later.

This work has the LPPL maintenance status `maintained'.

The Current Maintainer of this work is Niklas Beisert.

This work consists of the files |README.txt|, |childdoc.ins| and |childdoc.dtx|
as well as the derived files |childdoc.def|, |cdocsamp.tex|
with |cdocsch1.tex|, |cdocsch2.tex|, |cdocspt3.tex|, |cdocspt4.tex|,
|cdocsdrf.tex|, |cdocsfn1.tex|, |cdocsfn2.tex|
as well as |childdoc.pdf|.

%%%%%%%%%%%%%%%%%%%%%%%%%%%%%%%%%%%%%%%%%%%%%%%%%%%%%%%%%%%%%%%%%%%%%%%%%%%%%%%%
\subsection{Files and Installation}

The package consists of the files:
%
\begin{center}
\begin{tabular}{ll}
    |README.txt|   & readme file \\
    |childdoc.ins| & installation file \\
    |childdoc.dtx| & source file \\
    |childdoc.def| & definition file \\
    |cdocsamp.tex| & sample main file \\
    |cdocsch1.tex| & sample include file \\
    |cdocsch2.tex| & sample include file \\
    |cdocspt3.tex| & sample part file \\
    |cdocspt4.tex| & sample part file \\
    |cdocsdrf.tex| & sample redirection file \\
    |cdocsfn1.tex| & sample redirection file \\
    |cdocsfn2.tex| & sample redirection file \\
    |childdoc.pdf| & manual
\end{tabular}
\end{center}
%
The distribution consists of the files
|README.txt|, |childdoc.ins| and |childdoc.dtx|.
%
\begin{itemize}
\item
Run (pdf)\LaTeX{} on |childdoc.dtx|
to compile the manual |childdoc.pdf| (this file).
\item
Run \LaTeX{} on |childdoc.ins| to create the definitions file |childdoc.def|
and the sample |cdocsamp.tex| with include files
|cdocsch1.tex|, |cdocsch2.tex|, |cdocspt3.tex|, |cdocspt4.tex|,
|cdocsdrf.tex|, |cdocsfn1.tex|, |cdocsfn2.tex|.
Then copy the file |childdoc.def| to an appropriate directory of your \LaTeX{}
distribution, e.g.\ \textit{texmf-root}|/tex/latex/childdoc|.
\end{itemize}

%%%%%%%%%%%%%%%%%%%%%%%%%%%%%%%%%%%%%%%%%%%%%%%%%%%%%%%%%%%%%%%%%%%%%%%%%%%%%%%%
\subsection{Related CTAN Packages}

There are several other packages which offer a similar functionality:
%
\begin{itemize}
\item
The packages
\href{http://ctan.org/pkg/docmute}{\textsf{docmute}},
\href{http://ctan.org/pkg/includex}{\textsf{includex}} and
\href{http://ctan.org/pkg/standalone}{\textsf{standalone}}
provide commands to include only the document body of
a child file thus allowing both files to be compiled individually.
\item
The packages \href{http://ctan.org/pkg/subdocs}{\textsf{subdocs}}
and \href{http://ctan.org/pkg/subfiles}{\textsf{subfiles}}
provide structures in which the main and child documents can be
encapsulated and allowing them to be compiled individually.
The inclusion mechanism is different from the conventional |\include|.
\item
The package \href{http://ctan.org/pkg/combine}{\textsf{combine}}
is an elaborate solution to combine several documents into one.
\end{itemize}
%
See also the CTAN topic \href{http://ctan.org/topic/subdocs}{\textsf{subdocs}}
for further related packages.
The present package differs from the above solutions in that
a document structure constructed with the conventional |\include| mechanism
just needs two extra commands at the top of every file
such that all constituent files can be compiled individually.

%%%%%%%%%%%%%%%%%%%%%%%%%%%%%%%%%%%%%%%%%%%%%%%%%%%%%%%%%%%%%%%%%%%%%%%%%%%%%%%%
%\subsection{Feature Suggestions}
%
%The following is a list of features which may be useful for future
%versions of this package:
%%
%\begin{itemize}
%\item
%\ldots
%\end{itemize}

%%%%%%%%%%%%%%%%%%%%%%%%%%%%%%%%%%%%%%%%%%%%%%%%%%%%%%%%%%%%%%%%%%%%%%%%%%%%%%%%
\subsection{Revision History}

%%%%%%%%%%%%%%%%%%%%%%%%%%%%%%%%%%%%%%%%
\paragraph{v2.0:} 2018/12/30

\begin{itemize}
\item
immediate forward processing
\item
added |\childdocby| mechanism
\item
manual restructured
\end{itemize}

%%%%%%%%%%%%%%%%%%%%%%%%%%%%%%%%%%%%%%%%
\paragraph{v1.6:} 2018/01/17

\begin{itemize}
\item
application for development of include files
\item
corrections to manual
\end{itemize}

%%%%%%%%%%%%%%%%%%%%%%%%%%%%%%%%%%%%%%%%
\paragraph{v1.5:} 2017/05/21

\begin{itemize}
\item
more complete structuring introduced
\item
|\childdocof| introduced
\item
|\childdoc| renamed to |\childdocmain|
\item
|\childredirect| renamed to |\childdocforward| and |\childdocforwardprefix|
and functionality expanded
\end{itemize}

%%%%%%%%%%%%%%%%%%%%%%%%%%%%%%%%%%%%%%%%
\paragraph{v1.0:} 2017/04/27

\begin{itemize}
\item
manual and install package
\item
first version published on CTAN
\end{itemize}

%%%%%%%%%%%%%%%%%%%%%%%%%%%%%%%%%%%%%%%%
\paragraph{v0.6:} 2017/04/26

\begin{itemize}
\item
redirection mechanism added
\end{itemize}

%%%%%%%%%%%%%%%%%%%%%%%%%%%%%%%%%%%%%%%%
\paragraph{v0.5:} 2017/04/26

\begin{itemize}
\item
functionality in definition file
\end{itemize}


%%%%%%%%%%%%%%%%%%%%%%%%%%%%%%%%%%%%%%%%%%%%%%%%%%%%%%%%%%%%%%%%%%%%%%%%%%%%%%%%
%%%%%%%%%%%%%%%%%%%%%%%%%%%%%%%%%%%%%%%%%%%%%%%%%%%%%%%%%%%%%%%%%%%%%%%%%%%%%%%%
%%%%%%%%%%%%%%%%%%%%%%%%%%%%%%%%%%%%%%%%%%%%%%%%%%%%%%%%%%%%%%%%%%%%%%%%%%%%%%%%
\appendix

\settowidth\MacroIndent{\rmfamily\scriptsize 000\ }

 \DocInput{childdoc.dtx}

\end{document}
%</driver>
% \fi
%
% %%%%%%%%%%%%%%%%%%%%%%%%%%%%%%%%%%%%%%%%%%%%%%%%%%%%%%%%%%%%%%%%%%%%%%%%%%%%%%
% %%%%%%%%%%%%%%%%%%%%%%%%%%%%%%%%%%%%%%%%%%%%%%%%%%%%%%%%%%%%%%%%%%%%%%%%%%%%%%
% \section{Sample}
%\iffalse
%<*samplemain>
%\fi
%
% The following presents a sample document
% with two chapters, two parts, a title page,
% a compile flag as well as three forwarding files to set the flag.
% It consists of eight |.tex| files:
% \begin{center}
% \begin{tabular}{ll}
% |cdocsamp.tex|&main file\\
% |cdocsch1.tex|&include file for chapter 1\\
% |cdocsch2.tex|&include file for chapter 2\\
% |cdocspt3.tex|&include file for part 3\\
% |cdocspt4.tex|&include file for part 4\\
% |cdocsdrf.tex|&forwarding file for main file in draft mode\\
% |cdocsfi1.tex|&forwarding file for final version of chapter 1\\
% |cdocsfi2.tex|&forwarding file for final version of chapter 2\\
% \end{tabular}
% \end{center}
% Each of the eight files can be compiled directly by the \LaTeX{} compiler.
%
% %%%%%%%%%%%%%%%%%%%%%%%%%%%%%%%%%%%%%%
% \paragraph{Main File.}
%
% The main file is called |cdocsamp.tex|.
%
% Load the \textsf{childdoc} definitions and
% declare the filename for the main document:
%    \begin{macrocode}
\input{childdoc.def}
\childdocmain{}
%    \end{macrocode}

% Optional override for |\version| flag:
%    \begin{macrocode}
%%\ifchilddoc\else\providecommand{\version}{draft}\fi
%    \end{macrocode}

% Define the default values for the |\version| flag
% (|final| for the main file and |draft| for childs):
%    \begin{macrocode}
\ifchilddoc
\providecommand{\version}{draft}
\else
\providecommand{\version}{final}
\fi
%    \end{macrocode}

% Load the standard document class:
%    \begin{macrocode}
\documentclass[12pt]{article}
%    \end{macrocode}

% Start the document body:
%    \begin{macrocode}
\begin{document}
%    \end{macrocode}

% Declare a title page.
% Print title, part of document being processed and version flag:
%    \begin{macrocode}
\addtocounter{page}{-1}
\begin{center}
{\LARGE\bfseries{}childdoc example\par}
\vspace{1cm}
\ifchilddoc
\ifchilddocmanual part\else chapter\fi:
`\childdocname' of `\childdocjob'\par
\else
main document: `\childdocjob'\par
\fi
version: \version\par
\end{center}
\newpage
%    \end{macrocode}

% Manually include selected file,
% otherwise process as usual:
%    \begin{macrocode}
\ifchilddocmanual
\section*{part `\childdocname'}
\input{\childdocname}
\else
%    \end{macrocode}

% Include the two chapters:
%    \begin{macrocode}
\include{cdocsch1}
\include{cdocsch2}
%    \end{macrocode}

% Include the two parts unless only chapters should be displayed:
%    \begin{macrocode}
\ifchilddoc\else
\section{part three}
\input{cdocspt3}
\section{part four}
\input{cdocspt4}
\fi
%    \end{macrocode}

% Process as usual until here:
%    \begin{macrocode}
\fi
%    \end{macrocode}

% End of document body:
%    \begin{macrocode}
\end{document}
%    \end{macrocode}
%\iffalse
%</samplemain>
%\fi
%
% %%%%%%%%%%%%%%%%%%%%%%%%%%%%%%%%%%%%%%
% \paragraph{Chapter Include Files.}
%
% The include files are called |cdocsch1.tex| and |cdocsch2.tex|.
%
%\iffalse
%<*samplechap1|samplechap2>
%\fi

% Optional override for |\version| flag:
%    \begin{macrocode}
%%\providecommand{\version}{final}
%    \end{macrocode}

% Include the main document:
%    \begin{macrocode}
\input{childdoc.def}
\childdocof{cdocsamp}
%    \end{macrocode}

%\iffalse
%</samplechap1|samplechap2>
%\fi
%
%\iffalse
%<*samplechap1>
%\fi
% Some text for chapter 1:
%    \begin{macrocode}
\section{one}
some text in chapter one
%    \end{macrocode}

%\iffalse
%</samplechap1>
%\fi
% Some text for chapter 2:
%\iffalse
%<*samplechap2>
%\fi
%    \begin{macrocode}
\section{two}
more text in chapter two
%    \end{macrocode}

%\iffalse
%</samplechap2>
%\fi
%
% %%%%%%%%%%%%%%%%%%%%%%%%%%%%%%%%%%%%%%
% \paragraph{Part Include Files.}
%
% The include files are called |cdocspt3.tex| and |cdocspt4.tex|.
%
%\iffalse
%<*samplepart3|samplepart4>
%\fi

% Optional override for |\version| flag:
%    \begin{macrocode}
%%\providecommand{\version}{final}
%    \end{macrocode}

% Include the main document:
%    \begin{macrocode}
\input{childdoc.def}
\childdocby{cdocsamp}
%    \end{macrocode}

%\iffalse
%</samplepart3|samplepart4>
%\fi
%
%\iffalse
%<*samplepart3>
%\fi
% Some text for part 3:
%    \begin{macrocode}
some text in part three
%    \end{macrocode}

%\iffalse
%</samplepart3>
%\fi
% Some text for part 4:
%\iffalse
%<*samplepart4>
%\fi
%    \begin{macrocode}
more text in part four
%    \end{macrocode}

%\iffalse
%</samplepart4>
%\fi
%
% %%%%%%%%%%%%%%%%%%%%%%%%%%%%%%%%%%%%%%
% \paragraph{Forwarding for a Complete Draft.}
%
% The following forwarding file |cdocsdrf.tex|
% compiles the main document in draft mode:
%\iffalse
%<*sampledraft>
%\fi
%    \begin{macrocode}
\def\version{draft}
\input{childdoc.def}
\childdocforward{cdocsamp}
%    \end{macrocode}

%\iffalse
%</sampledraft>
%\fi
%
% %%%%%%%%%%%%%%%%%%%%%%%%%%%%%%%%%%%%%%
% \paragraph{Forwarding for Final Version of the Chapters.}
%
% The following forwarding files |cdocsfn1.tex| and |cdocsfn2.tex|
% (with identical content)
% compile the final versions of the child documents
% |cdocsch1.tex| and |cdocsch2.tex|, respectively:
%\iffalse
%<*samplefinal>
%\fi
%    \begin{macrocode}
\def\version{final}
\input{childdoc.def}
\childdocforwardprefix[cdocsamp]{cdocsfn}{cdocsch}
%    \end{macrocode}

%\iffalse
%</samplefinal>
%\fi
%
% %%%%%%%%%%%%%%%%%%%%%%%%%%%%%%%%%%%%%%
% \paragraph{Command Line Processing.}
%
% The following three command lines generate the output files
% |cdocscld|, |cdocscl1| and |cdocscl2|
% which should be identical to
% |cdocsdrf|, |cdocsch1| and |cdocsfn2|, respectively:
% \begin{center}
% \begin{tabular}{l}
% |latex -jobname cdocscld \|\\
% |  "\def\version{draft}\input{childdoc.def}\childdocforward{cdocsamp}"|\\
% |latex -jobname cdocscl1 \|\\
% |  "\input{childdoc.def}\childdocforward[cdocsamp]{cdocsch1}"|\\
% |latex -jobname cdocscl2 \|\\
% |  "\def\version{final}\input{childdoc.def}\childdocforward{cdocsch2}"|
% \end{tabular}
% \end{center}
% Note that the trailing backslash on each first line
% merely continues the input to the second line
% (for convenient cut ant paste).
% Furthermore, the command |latex| can be replaced by any
% of its alternative versions such as |pdflatex|.
%
% %%%%%%%%%%%%%%%%%%%%%%%%%%%%%%%%%%%%%%%%%%%%%%%%%%%%%%%%%%%%%%%%%%%%%%%%%%%%%%
% %%%%%%%%%%%%%%%%%%%%%%%%%%%%%%%%%%%%%%%%%%%%%%%%%%%%%%%%%%%%%%%%%%%%%%%%%%%%%%
% \section{Implementation}
%\iffalse
%<*package>
%\fi
%
% This section describes the definitions file |childdoc.def|.

% The definitions cannot be loaded using |\usepackage| or |\RequirePackage|
% which has a mechanism to prevent loading a style file more than once.
% When loading the definitions by means of |\input|
% multiple instances have to be prevented manually:
%\iffalse
%This code needs to be before the `\ProvidesFile' directive
%which is defined at the beginning of this file.
%Therefore it is also placed there and commented out here.
%</package>
%<*discard>
%\fi
%    \begin{macrocode}
\ifdefined\childdocmain\endinput\fi
%    \end{macrocode}
%\iffalse
%</discard>
%<*package>
%\fi
%
% \macro{\ifchilddoc}
% \macro{\ifchilddocmanual}
% The conditional |\ifchilddoc| tells whether a
% child (true) or main (false) document is being compiled.
% The conditional |\ifchilddocmanual| tells whether
% the |\includeonly| mechanism is used (false) or
% the selection of child files must be performed manually (true).
% The definitions initialise to false:
%    \begin{macrocode}
\newif\ifchilddoc
\newif\ifchilddocmanual
%    \end{macrocode}

% \macro{\childdocname}
% \macro{\childdocjob}
% The macro |\childdocname| stores the name of the main document
% to be compiled. The macro |\childdocjob| stores the name of
% the document on which the \LaTeX{} compiler was originally invoked.
% The content of |\jobname| cannot be compared
% to filenames specified in the source due to different catcodes.
% The following code rescans |\jobname|, stores the result
% in |\childdocname| and saves a copy in |\childdocjob|:
%    \begin{macrocode}
\edef\childdocname{\scantokens\expandafter{\jobname\noexpand}}
\let\childdocjob\childdocname
%    \end{macrocode}

% \macro{\childdocdisable}
% The macro |\childdocdisable| prevents the main file
% from being processed more than once.
% At this stage, the main document command |\childdocmain|
% is assumed to be called once again where it should do nothing.
% Any subsequent call to it should prevent
% a secondary processing of the main document
% It overwrites the forwarding commands
% |\childdocof| and |\childdocforward|
% with empty macros to prevent further inclusions of the main document:
%    \begin{macrocode}
\newcommand{\childdocdisable}
{
  \renewcommand{\childdocmain}[1]{\renewcommand{\childdocmain}[1]{\endinput}}
  \renewcommand{\childdocof}[1]{}
  \renewcommand{\childdocby}[2][]{}
  \renewcommand{\childdocforward}[2][]{}
  \renewcommand{\childdocdisable}{}
}
%    \end{macrocode}

% \macro{\childdocmain}
% The macro |\childdocmain| is to be called at the top of the main file
% with nothing or the main filename (without extension) as argument.
% First, it breaks loops.
% If the argument is not empty and does not match |\childdocname|
% (which is set by the first inclusion of |childdoc.def|),
% |\ifchilddoc| is set to true, |\includeonly| is applied to the child file
% and |\jobname| is set to the main file
% (for proper handling of |.aux| files):
%    \begin{macrocode}
\newcommand{\childdocmain}[1]
{
  \childdocdisable\childdocmain{}
  \if?#1?\else
    \begingroup
      \def\childdoctmp{#1}
      \ifx\childdoctmp\childdocname
        \def\childdoctmp{}
      \else
        \def\childdoctmp
        {
          \childdoctrue
          \includeonly{\childdocname}
          \def\childdocjob{#1}
          \def\jobname{#1}
        }
      \fi
      \expandafter
    \endgroup
    \childdoctmp
  \fi
}
%    \end{macrocode}

% \macro{\childdocof}
% The command |\childdocof| redirects
% compilation to the main file |#1|.
%    \begin{macrocode}
\newcommand{\childdocof}[1]
{
  \childdocdisable
  \childdoctrue
  \includeonly{\childdocname}
  \def\jobname{#1}
  \def\childdocjob{#1}
  \input{#1}
}
%    \end{macrocode}

% \macro{\childdocby}
% The command |\childdocby| ....
%    \begin{macrocode}
\newcommand{\childdocby}[2][]
{
  \childdocdisable
  \childdoctrue
  \childdocmanualtrue
  \if?#1?\else
    \def\jobname{#2}
  \fi
  \def\childdocjob{#2}
  \input{#2}
  \endinput
}
%    \end{macrocode}

% \macro{\childdocforward}
% The command |\childdocforward| redirects
% compilation to the main file or
% (if the optional argument is given) a child file.
% Parameters are set as if the main file
% or a child file starting with |\childdocof| was compiled.
% Then compilation is handed over to the main file:
%    \begin{macrocode}
\newcommand{\childdocforward}[2][]
{
  \begingroup
    \if?#1?
      \def\childdoctmp
      {
        \def\childdocname{#2}
        \def\childdocjob{#2}
        \def\jobname{#2}
        \input{#2}
        \endinput
      }
    \else
      \def\childdoctmp
      {
        \childdocdisable
        \def\childdocname{#2}
        \childdoctrue
        \includeonly{#2}
        \def\childdocjob{#1}
        \def\jobname{#1}
        \input{#1}
        \endinput
      }
    \fi
    \expandafter
  \endgroup
  \childdoctmp
}
%    \end{macrocode}

% \macro{\childdocforwardprefix}
% The command |\childdocforwardprefix| redirects
% compilation to the main or a child file by means of a pattern.
% The prefix |#1| in the current filename is replaced by |#2|
% and the suffix of the current filename is kept
% (it is assumed that the filename does not contain the substring `|~~~|'
% which is used as a delimiter).
% Compilation is handed over to the new file by |\childdocforward|:
%    \begin{macrocode}
\newcommand{\childdocforwardprefix}[3][]
{
  \begingroup
    \def\childdocextract #2##1~~~{\def\childdoctmp{\childdocforward[#1]{#3##1}}}
    \expandafter\childdocextract\childdocname~~~
    \expandafter
  \endgroup
  \childdoctmp
}
%    \end{macrocode}

% \macro{\childdoc}
% The deprecated macro |\childdoc| is a legacy version of |\childdocmain|:
%    \begin{macrocode}
\newcommand{\childdoc}{\childdocmain}
%    \end{macrocode}

% \macro{\childdocredirect}
% The deprecated macro |\childdocredirect| is a legacy version
% of |\childdocforward| and |\childdocforwardprefix|:
%    \begin{macrocode}
\newcommand{\childdocredirect}[2][]
{
  \begingroup
    \if?#1?
      \def\childdoctmp{\childdocforward{#2}}
    \else
      \def\childdoctmp{\childdocforwardprefix{#1}{#2}}
    \fi
    \expandafter
  \endgroup
  \childdoctmp
}
%    \end{macrocode}

%\iffalse
%</package>
%\fi
%
\endinput
|
and perform the replacements as outlined below.
Instead of |\childdocmain{|\textit{main}|}| add the following code
to the top of the main file:
%
\begin{center}
\begin{tabular}{l}
|\||ifdefined\childdocname\endinput\||fi\newif\ifchilddoc|\\
|\edef\childdocname{\scantokens\expandafter{\jobname\noexpand}}|\\
|\def\childdocmain{|\textit{main}|}\||ifx\childdocmain\childdocname\||else|\\
|\childdoctrue\includeonly{\childdocname}\let\jobname\childdocmain\||fi|\\
\end{tabular}
\end{center}
%
Instead of |\childdocof{|\textit{main}|}| just include the main file
at the top of each child file:
%
\begin{center}
|\input{|\textit{main}|}|
\end{center}
%
A simple redirection |\childdocforward{|\textit{dest}|}| is achieved by:
%
\begin{center}
|\def\jobname{|\textit{dest}|}\input{\jobname}|
\end{center}
%
The redirection with prefix
|\childdocforwardprefix[|\textit{prefix}|]{|\textit{dest}|}|
is accomplished by:
%
\begin{center}
\begin{tabular}{l}
|{\edef\jobname{\scantokens\expandafter{\jobname\noexpand}}|\\
|\def\redirectjob |\textit{prefix}|#1~~~{\gdef\jobname{|\textit{dest}|#1}}|\\
|\expandafter\redirectjob\jobname~~~}\input{\jobname}|
\end{tabular}
\end{center}

In an alternative approach,
child documents can be compiled by a specific command line
without additional code or specific definitions:
%
\begin{center}
|... -jobname "|\textit{target}|" "|[\textit{flags}]%
|\includeonly{|\textit{dest}|}\input{|\textit{main}|}"|
\end{center}
%

%%%%%%%%%%%%%%%%%%%%%%%%%%%%%%%%%%%%%%%%%%%%%%%%%%%%%%%%%%%%%%%%%%%%%%%%%%%%%%%%
%%%%%%%%%%%%%%%%%%%%%%%%%%%%%%%%%%%%%%%%%%%%%%%%%%%%%%%%%%%%%%%%%%%%%%%%%%%%%%%%
\section{Information}

%%%%%%%%%%%%%%%%%%%%%%%%%%%%%%%%%%%%%%%%%%%%%%%%%%%%%%%%%%%%%%%%%%%%%%%%%%%%%%%%
\subsection{Copyright}

Copyright \copyright{} 2017--2018 Niklas Beisert

This work may be distributed and/or modified under the
conditions of the \LaTeX{} Project Public License, either version 1.3
of this license or (at your option) any later version.
The latest version of this license is in
  \url{http://www.latex-project.org/lppl.txt}
and version 1.3 or later is part of all distributions of \LaTeX{}
version 2005/12/01 or later.

This work has the LPPL maintenance status `maintained'.

The Current Maintainer of this work is Niklas Beisert.

This work consists of the files |README.txt|, |childdoc.ins| and |childdoc.dtx|
as well as the derived files |childdoc.def|, |cdocsamp.tex|
with |cdocsch1.tex|, |cdocsch2.tex|, |cdocspt3.tex|, |cdocspt4.tex|,
|cdocsdrf.tex|, |cdocsfn1.tex|, |cdocsfn2.tex|
as well as |childdoc.pdf|.

%%%%%%%%%%%%%%%%%%%%%%%%%%%%%%%%%%%%%%%%%%%%%%%%%%%%%%%%%%%%%%%%%%%%%%%%%%%%%%%%
\subsection{Files and Installation}

The package consists of the files:
%
\begin{center}
\begin{tabular}{ll}
    |README.txt|   & readme file \\
    |childdoc.ins| & installation file \\
    |childdoc.dtx| & source file \\
    |childdoc.def| & definition file \\
    |cdocsamp.tex| & sample main file \\
    |cdocsch1.tex| & sample include file \\
    |cdocsch2.tex| & sample include file \\
    |cdocspt3.tex| & sample part file \\
    |cdocspt4.tex| & sample part file \\
    |cdocsdrf.tex| & sample redirection file \\
    |cdocsfn1.tex| & sample redirection file \\
    |cdocsfn2.tex| & sample redirection file \\
    |childdoc.pdf| & manual
\end{tabular}
\end{center}
%
The distribution consists of the files
|README.txt|, |childdoc.ins| and |childdoc.dtx|.
%
\begin{itemize}
\item
Run (pdf)\LaTeX{} on |childdoc.dtx|
to compile the manual |childdoc.pdf| (this file).
\item
Run \LaTeX{} on |childdoc.ins| to create the definitions file |childdoc.def|
and the sample |cdocsamp.tex| with include files
|cdocsch1.tex|, |cdocsch2.tex|, |cdocspt3.tex|, |cdocspt4.tex|,
|cdocsdrf.tex|, |cdocsfn1.tex|, |cdocsfn2.tex|.
Then copy the file |childdoc.def| to an appropriate directory of your \LaTeX{}
distribution, e.g.\ \textit{texmf-root}|/tex/latex/childdoc|.
\end{itemize}

%%%%%%%%%%%%%%%%%%%%%%%%%%%%%%%%%%%%%%%%%%%%%%%%%%%%%%%%%%%%%%%%%%%%%%%%%%%%%%%%
\subsection{Related CTAN Packages}

There are several other packages which offer a similar functionality:
%
\begin{itemize}
\item
The packages
\href{http://ctan.org/pkg/docmute}{\textsf{docmute}},
\href{http://ctan.org/pkg/includex}{\textsf{includex}} and
\href{http://ctan.org/pkg/standalone}{\textsf{standalone}}
provide commands to include only the document body of
a child file thus allowing both files to be compiled individually.
\item
The packages \href{http://ctan.org/pkg/subdocs}{\textsf{subdocs}}
and \href{http://ctan.org/pkg/subfiles}{\textsf{subfiles}}
provide structures in which the main and child documents can be
encapsulated and allowing them to be compiled individually.
The inclusion mechanism is different from the conventional |\include|.
\item
The package \href{http://ctan.org/pkg/combine}{\textsf{combine}}
is an elaborate solution to combine several documents into one.
\end{itemize}
%
See also the CTAN topic \href{http://ctan.org/topic/subdocs}{\textsf{subdocs}}
for further related packages.
The present package differs from the above solutions in that
a document structure constructed with the conventional |\include| mechanism
just needs two extra commands at the top of every file
such that all constituent files can be compiled individually.

%%%%%%%%%%%%%%%%%%%%%%%%%%%%%%%%%%%%%%%%%%%%%%%%%%%%%%%%%%%%%%%%%%%%%%%%%%%%%%%%
%\subsection{Feature Suggestions}
%
%The following is a list of features which may be useful for future
%versions of this package:
%%
%\begin{itemize}
%\item
%\ldots
%\end{itemize}

%%%%%%%%%%%%%%%%%%%%%%%%%%%%%%%%%%%%%%%%%%%%%%%%%%%%%%%%%%%%%%%%%%%%%%%%%%%%%%%%
\subsection{Revision History}

%%%%%%%%%%%%%%%%%%%%%%%%%%%%%%%%%%%%%%%%
\paragraph{v2.0:} 2018/12/30

\begin{itemize}
\item
immediate forward processing
\item
added |\childdocby| mechanism
\item
manual restructured
\end{itemize}

%%%%%%%%%%%%%%%%%%%%%%%%%%%%%%%%%%%%%%%%
\paragraph{v1.6:} 2018/01/17

\begin{itemize}
\item
application for development of include files
\item
corrections to manual
\end{itemize}

%%%%%%%%%%%%%%%%%%%%%%%%%%%%%%%%%%%%%%%%
\paragraph{v1.5:} 2017/05/21

\begin{itemize}
\item
more complete structuring introduced
\item
|\childdocof| introduced
\item
|\childdoc| renamed to |\childdocmain|
\item
|\childredirect| renamed to |\childdocforward| and |\childdocforwardprefix|
and functionality expanded
\end{itemize}

%%%%%%%%%%%%%%%%%%%%%%%%%%%%%%%%%%%%%%%%
\paragraph{v1.0:} 2017/04/27

\begin{itemize}
\item
manual and install package
\item
first version published on CTAN
\end{itemize}

%%%%%%%%%%%%%%%%%%%%%%%%%%%%%%%%%%%%%%%%
\paragraph{v0.6:} 2017/04/26

\begin{itemize}
\item
redirection mechanism added
\end{itemize}

%%%%%%%%%%%%%%%%%%%%%%%%%%%%%%%%%%%%%%%%
\paragraph{v0.5:} 2017/04/26

\begin{itemize}
\item
functionality in definition file
\end{itemize}


%%%%%%%%%%%%%%%%%%%%%%%%%%%%%%%%%%%%%%%%%%%%%%%%%%%%%%%%%%%%%%%%%%%%%%%%%%%%%%%%
%%%%%%%%%%%%%%%%%%%%%%%%%%%%%%%%%%%%%%%%%%%%%%%%%%%%%%%%%%%%%%%%%%%%%%%%%%%%%%%%
%%%%%%%%%%%%%%%%%%%%%%%%%%%%%%%%%%%%%%%%%%%%%%%%%%%%%%%%%%%%%%%%%%%%%%%%%%%%%%%%
\appendix

\settowidth\MacroIndent{\rmfamily\scriptsize 000\ }

 \DocInput{childdoc.dtx}

\end{document}
%</driver>
% \fi
%
% %%%%%%%%%%%%%%%%%%%%%%%%%%%%%%%%%%%%%%%%%%%%%%%%%%%%%%%%%%%%%%%%%%%%%%%%%%%%%%
% %%%%%%%%%%%%%%%%%%%%%%%%%%%%%%%%%%%%%%%%%%%%%%%%%%%%%%%%%%%%%%%%%%%%%%%%%%%%%%
% \section{Sample}
%\iffalse
%<*samplemain>
%\fi
%
% The following presents a sample document
% with two chapters, two parts, a title page,
% a compile flag as well as three forwarding files to set the flag.
% It consists of eight |.tex| files:
% \begin{center}
% \begin{tabular}{ll}
% |cdocsamp.tex|&main file\\
% |cdocsch1.tex|&include file for chapter 1\\
% |cdocsch2.tex|&include file for chapter 2\\
% |cdocspt3.tex|&include file for part 3\\
% |cdocspt4.tex|&include file for part 4\\
% |cdocsdrf.tex|&forwarding file for main file in draft mode\\
% |cdocsfi1.tex|&forwarding file for final version of chapter 1\\
% |cdocsfi2.tex|&forwarding file for final version of chapter 2\\
% \end{tabular}
% \end{center}
% Each of the eight files can be compiled directly by the \LaTeX{} compiler.
%
% %%%%%%%%%%%%%%%%%%%%%%%%%%%%%%%%%%%%%%
% \paragraph{Main File.}
%
% The main file is called |cdocsamp.tex|.
%
% Load the \textsf{childdoc} definitions and
% declare the filename for the main document:
%    \begin{macrocode}
% \iffalse
%
% childdoc.dtx Copyright (C) 2017-2018 Niklas Beisert
%
% This work may be distributed and/or modified under the
% conditions of the LaTeX Project Public License, either version 1.3
% of this license or (at your option) any later version.
% The latest version of this license is in
%   http://www.latex-project.org/lppl.txt
% and version 1.3 or later is part of all distributions of LaTeX
% version 2005/12/01 or later.
%
% This work has the LPPL maintenance status `maintained'.
%
% The Current Maintainer of this work is Niklas Beisert.
%
% This work consists of the files childdoc.dtx and childdoc.ins
% and the derived files childdoc.def and cdocsamp.tex with
% cdocsch1.tex, cdocsch2.tex, cdocsdrf.tex, cdocsfn1.tex, cdocsfn2.tex.
%
%<package>\ifdefined\childdocmain\endinput\fi
%<package>\ProvidesFile{childdoc.def}[2018/12/30 v2.0 child document driver]
%<samplemain>\ProvidesFile{cdocsamp.tex}[2018/12/30 v2.0 sample for childdoc]
%<*driver>
%\ProvidesFile{childdoc.drv}[2018/12/30 v2.0 childdoc reference manual file]
\PassOptionsToClass{10pt,a4paper}{article}
\documentclass{ltxdoc}

\usepackage[margin=35mm]{geometry}
\usepackage{hyperref}
\usepackage{hyperxmp}
\usepackage[usenames]{color}

\hypersetup{colorlinks=true}
\hypersetup{pdfstartview=FitH}
\hypersetup{pdfpagemode=UseNone}
\hypersetup{pdfsource={}}
\hypersetup{pdflang={en-UK}}
\hypersetup{pdfcopyright={Copyright 2017-2018 Niklas Beisert.
  This work may be distributed and/or modified under the
  conditions of the LaTeX Project Public License, either version 1.3
  of this license or (at your option) any later version.}}
\hypersetup{pdflicenseurl={http://www.latex-project.org/lppl.txt}}
\hypersetup{pdfcontactaddress={ETH Zurich, ITP, HIT K,
  Wolfgang-Pauli-Strasse 27}}
\hypersetup{pdfcontactpostcode={8093}}
\hypersetup{pdfcontactcity={Zurich}}
\hypersetup{pdfcontactcountry={Switzerland}}
\hypersetup{pdfcontactemail={nbeisert@itp.phys.ethz.ch}}
\hypersetup{pdfcontacturl={http://people.phys.ethz.ch/\xmptilde nbeisert/}}

\newcommand{\secref}[1]{\hyperref[#1]{section \ref*{#1}}}

\parskip1ex
\parindent0pt
\let\olditemize\itemize
\def\itemize{\olditemize\parskip0pt}

\begin{document}

\title{The \textsf{childdoc} Package}
\hypersetup{pdftitle={The childdoc Package}}
\author{Niklas Beisert\\[2ex]
  Institut f\"ur Theoretische Physik\\
  Eidgen\"ossische Technische Hochschule Z\"urich\\
  Wolfgang-Pauli-Strasse 27, 8093 Z\"urich, Switzerland\\[1ex]
  \href{mailto:nbeisert@itp.phys.ethz.ch}
  {\texttt{nbeisert@itp.phys.ethz.ch}}}
\hypersetup{pdfauthor={Niklas Beisert}}
\hypersetup{pdfsubject={Manual for the LaTeX2e Package childdoc}}
\date{30 December 2018, \textsf{v2.0}}
\maketitle

\begin{abstract}\noindent
\textsf{childdoc} is a \LaTeXe{} package
that enables the direct compilation
of document sections included by |\include|
to individual files.
\end{abstract}

\begingroup
\parskip0ex
\tableofcontents
\endgroup

%%%%%%%%%%%%%%%%%%%%%%%%%%%%%%%%%%%%%%%%%%%%%%%%%%%%%%%%%%%%%%%%%%%%%%%%%%%%%%%%
%%%%%%%%%%%%%%%%%%%%%%%%%%%%%%%%%%%%%%%%%%%%%%%%%%%%%%%%%%%%%%%%%%%%%%%%%%%%%%%%
\section{Introduction}

\LaTeX{} provides a mechanism to structure a large document (such as a book)
into a main file and several child files (containing the chapters)
using the |\include| command.
This mechanism is beneficial for documents
which span hundreds of pages in order to
make the source file(s) more manageable.
Moreover, compilation can be restricted to
selected child files by means of the |\includeonly| command.
The latter feature can be used to reduce the compilation time while editing
(this was significantly more useful in the earlier days of \LaTeX{})
or to generate a smaller document which is easier to navigate.
Another application of |\includeonly| is to generate
documents consisting of selected parts of the complete document.

However, there are a few drawbacks of the plain |\include| mechanism:
\begin{itemize}
\item
The child files cannot be compiled on their own,
they can only be compiled via the main file.
A naive editing environment
(such as a text editor with an option
to have the current file processed by \LaTeX)
may require one to switch to the main file before compiling;
attempting to compile the child file produces errors.
\item
The main file must be modified (each time)
to adjust the |\includeonly| command
to the present needs. This easily leaves the main file in a messy state.
\item
The generated document will always carry the filename
of the main document. This is inconvenient if
several child files are to be compiled and
to be kept for distribution.
\end{itemize}

The present package provides a simple interface
to make child files individually compilable by \LaTeX{}.
Compiling a child file then has the same effect as compiling
the main file with an |\includeonly| command
to select the appropriate child.
Moreover the generated document will carry the name of the child
rather than the main file.
This resolves all three above issues.

This feature is meant to make the editing of books,
thesis documents and lecture notes somewhat more convenient.
However, the package can also be used efficiently for
composing a series of documents (such as exercise sheets)
which are typically distributed individually.
It then assists the author in generating the individual documents
(potentially in different versions)
as well as a document containing the collected series.
Another application is in developing style files
or other kinds of included material
where compilation of the style file could redirect
to a sample or test file.

%%%%%%%%%%%%%%%%%%%%%%%%%%%%%%%%%%%%%%%%%%%%%%%%%%%%%%%%%%%%%%%%%%%%%%%%%%%%%%%%
%%%%%%%%%%%%%%%%%%%%%%%%%%%%%%%%%%%%%%%%%%%%%%%%%%%%%%%%%%%%%%%%%%%%%%%%%%%%%%%%
\section{Usage}

First of all, the package \textsf{childdoc} is \emph{not} a standard
\LaTeXe{} |.sty| style file! Therefore it needs to be invoked in
a non-standard way.

%%%%%%%%%%%%%%%%%%%%%%%%%%%%%%%%%%%%%%%%%%%%%%%%%%%%%%%%%%%%%%%%%%%%%%%%%%%%%%%%
\subsection{Included Files}
\label{sec:include}

%%%%%%%%%%%%%%%%%%%%%%%%%%%%%%%%%%%%%%%%
\DescribeMacro{\childdocmain}
To use the package, add the commands
\begin{center}
\begin{tabular}{l}
|\input{childdoc.def}|\\
|\childdocmain{}|\\
\end{tabular}
\end{center}
at the very top of the main \LaTeX{} file,
in particular \emph{before} the |\documentclass| statement!
The argument of |\childdocmain| should be left empty
(but it must be present).

%%%%%%%%%%%%%%%%%%%%%%%%%%%%%%%%%%%%%%%%
\DescribeMacro{\childdocof}
Furthermore, add the commands
\begin{center}
\begin{tabular}{l}
|\input{childdoc.def}|\\
|\childdocof{|\textit{main}|}|\\
\end{tabular}
\end{center}
at the top of every child file \textit{child}
which is included by |\include{|\textit{child}|}|
from within the main file
(or at least for those files to be compiled individually).
The argument \textit{main} must be the filename of the main file.

There are a couple of
considerations in setting up the main and child documents:

%%%%%%%%%%%%%%%%%%%%%%%%%%%%%%%%%%%%%%%%
\paragraph{Restrictions.}

Please note the following restrictions:
\begin{itemize}
\item
|\childdocmain| must be called with one argument \textit{main}
to ensure compatibility with earlier version of the package.
It must either be empty (|\childdocmain{}|)
or precisely match the filename of the main file in which it is specified.
See \secref{sec:detection} for further information.
\item
The filename \textit{main} must be specified without the |.tex| extension.
\item
The filename \textit{main} is case sensitive
(even in case-insensitive file systems)
due to internal string comparison.
\item
The argument \textit{main} should be fully expanded, it cannot be a macro.
\item
Subdirectories and special characters should be avoided in filenames.
\item
The command |\childdocmain{|\textit{main}|}| must be followed by a whitespace.
It should not be followed immediately by another command
or by a comment mark `|%|'.
This is because the \TeX{} parser reads the token immediately following
the argument of |\childdocmain| and puts it
at the beginning of every child section;
however, a white\-space is ignored.
\end{itemize}

%%%%%%%%%%%%%%%%%%%%%%%%%%%%%%%%%%%%%%%%
\paragraph{Content of Main File.}

It is advisable to place all content in the child files included by |\include|.
Any output contained in the main file will appear in all child documents
unless suppressed manually;
it cannot be suppressed automatically by the |\includeonly| directive
and thus should normally be avoided.
A method to include some content in the main file
by means of conditional processing is described in \secref{sec:conditional}.

%%%%%%%%%%%%%%%%%%%%%%%%%%%%%%%%%%%%%%%%
\paragraph{Page Numbering.}

When only a part of the document is compiled,
the appropriate numbering of pages
(as well as other status parameters)
is determined from the |.aux| files.
The latter contain information from previous passes.
However this information needs to propagate through
all intermediate child documents.
Therefore the page numbering in child documents may well
be inconsistent until the complete document is compiled at least once.

A useful (if unconventional) way to always ensure a consistent
page numbering is to restart the numbering in each child document
and denote the pages by `\textit{child}|.|\textit{page}'
where \textit{child} represents the chapter/section number of the child file.
This can be achieved by the command
|\numberwithin{page}{|\textit{child}|}|
of the \textsf{amsmath} package
where \textit{child} can be |chapter| or |section|
depending on the chosen structuring.
Alternatively, one can modify the macro |\thepage| appropriately
and reset the counter |page| at the start of each child file.

%%%%%%%%%%%%%%%%%%%%%%%%%%%%%%%%%%%%%%%%%%%%%%%%%%%%%%%%%%%%%%%%%%%%%%%%%%%%%%%%
\subsection{Conditional Processing}
\label{sec:conditional}

The package provides a mechanism to compile different versions
of a document. To customise the versions further some conditional processing
can come in handy to distinguish which version is being compiled.
The package provides two macros to describe the compilation context:

%%%%%%%%%%%%%%%%%%%%%%%%%%%%%%%%%%%%%%%%
\DescribeMacro{\ifchilddoc}
The conditional |\ifchilddoc| distinguishes between the compilation of
child documents and the main document:
%
\begin{center}
|\ifchilddoc |\textit{child-code}| |[|\||else |\textit{main-code}]| \||fi|
\end{center}

%%%%%%%%%%%%%%%%%%%%%%%%%%%%%%%%%%%%%%%%
\DescribeMacro{\childdocname}
\DescribeMacro{\childdocjob}
The macro |\childdocname| contains the filename (without extension)
of the main or child file being processed.
Note that |\childdocjob| will always contain the name of the main file.

%%%%%%%%%%%%%%%%%%%%%%%%%%%%%%%%%%%%%%%%
\paragraph{Title Page.}

Conditional processing can be used to include a title or banner page
in the main document when proper precautions are taken.
Importantly, the code in the main file should ensure that the page counter
(as well as other status parameters which are stored in the |.aux| files)
takes the same value after the conditional processing.
Otherwise the page numbers may take divergent values
depending on which part is compiled.

For example, a title page could be declared by:
%
\begin{center}
\begin{tabular}{l}
|\ifchilddoc\||else|\\
|\addtocounter{page}{-1}|\\
\textit{code for title page}\\
|\newpage|\\
|\||fi|
\end{tabular}
\end{center}
%
A banner page for the child documents can be generated by:
%
\begin{center}
\begin{tabular}{l}
|\ifchilddoc|\\
|\addtocounter{page}{-1}|\\
\textit{code for banner page}\\
|\newpage|\\
|\||fi|
\end{tabular}
\end{center}
%
Here one could write a message such as:
\begin{center}
|This is the part \childdocname{} of \childdocjob{}.|
\end{center}

%%%%%%%%%%%%%%%%%%%%%%%%%%%%%%%%%%%%%%%%%%%%%%%%%%%%%%%%%%%%%%%%%%%%%%%%%%%%%%%%
\subsection{Flags}
\label{sec:flags}

The package makes it easy to generate different versions
of the main or child documents.
To this end compilation flags can be defined
and assigned different default values.
They will be particularly useful in conjunction
with the forwarding mechanism described in \secref{sec:forward}.

For example, it may be useful to have a flag |\version|
which can be set to |draft| or |final|.
The document source will contain some conditional code
depending on the value of |\version|.
Suppose further, the flag should default to |final| for the main file
and to |draft| for child files
which is a natural assignment for editing the document.
This is achieved by placing the following code
in the preamble of the main document
(below the |\childdocmain| directive):
%
\begin{center}
\begin{tabular}{l}
|\ifchilddoc|\\
|\providecommand{\version}{draft}|\\
|\||else|\\
|\providecommand{\version}{final}|\\
|\||fi|
\end{tabular}
\end{center}
%
The definition by |\providecommand| makes sure
that previous definitions are not overwritten.
Further statements |\providecommand{\version}{...}|
can thus be added before the above code to override it.

For the main file, one might add a line
(between |\childdocmain| and the above block)
%
\begin{center}
|%\ifchilddoc\||else\providecommand{\version}{draft}\||fi|
\end{center}
%
which can be uncommented to produce a draft version.
Likewise one can add a line to the very top of a child file
(above the |\childdocof{|\textit{main}|}| directive)
%
\begin{center}
|%\providecommand{\version}{final}|
\end{center}
%
which can be uncommented to produce the final version of this child document.

%%%%%%%%%%%%%%%%%%%%%%%%%%%%%%%%%%%%%%%%%%%%%%%%%%%%%%%%%%%%%%%%%%%%%%%%%%%%%%%%
\subsection{Forwarding}
\label{sec:forward}

Different versions of the main or child documents
using compilation flags as described in \secref{sec:flags}
can be (permanently) stored in different files
for convenient compilation, viewing and distribution.
To this end, the package defines a command
to pass on compilation to a different file:

%%%%%%%%%%%%%%%%%%%%%%%%%%%%%%%%%%%%%%%%
\DescribeMacro{\childdocforward}
The command |\childdocforward| redirects processing to
another source file:
%
\begin{center}
\begin{tabular}{l}
|\input{childdoc.def}|\\
|\childdocforward[|\textit{main}|]{|\textit{dest}|}|\\
\end{tabular}
\end{center}
%
The argument \textit{dest} is the destination file
(without extension).
It should be the main file or one of the child files.
Note that further \textsf{childdoc} directives
such as |\childdocof| and |\childdocforward|
in the indicated file will be processed in this form.
The optional argument \textit{main}
passes on directly to the main file \textit{main}
while pretending to compile the child \textit{dest}.
This form behaves as if \textit{dest}
issues |\childdocof{|\textit{main}|}| right away,
and no further \textsf{childdoc} directives will be processed.

%%%%%%%%%%%%%%%%%%%%%%%%%%%%%%%%%%%%%%%%
\DescribeMacro{\...prefix}
In the alternative form |\childdocforwardprefix|,
%
\begin{center}
\begin{tabular}{l}
|\input{childdoc.def}|\\
|\childdocforwardprefix[|\textit{main}|]{|\textit{prefix}|}{|\textit{dest}|}|
\end{tabular}
\end{center}
%
the destination file is determined by a pattern
depending on the current file:
To make this work, the current file must be called
`{\textit{prefix}\hspace{0.2em}\textit{suffix}}'
with \textit{prefix} matching precisely the argument.
Processing is then passed on to the file
`{\textit{dest}\hspace{0.2em}\textit{suffix}}'.
Surely, the same effect is achieved by
directly specifying the
argument `{\textit{dest}\hspace{0.2em}\textit{suffix}}'
in the first form.
However, that requires to set up a different file
for each child. With the alternative form of the command
all these files can have exactly the same content
which simplifies setting them up and maintaining them.

For example, the following file |draft.tex|
with a compilation flag |\version| as described in \secref{sec:flags}
compiles the main document as a draft:
%
\begin{center}
\begin{tabular}{l}
|\def\version{draft}|\\
|\input{childdoc.def}|\\
|\childdocforward{|\textit{main}|}|
\end{tabular}
\end{center}
%
Likewise, the following files |final|\textit{nn}|.tex|
compile the final version of the child document
|child|\textit{nn}|.tex|:
%
\begin{center}
\begin{tabular}{l}
|\def\version{final}|\\
|\input{childdoc.def}|\\
|\childdocforwardprefix{final}{child}|
\end{tabular}
\end{center}
%

Note that when several versions of a main file and/or of each child file
are to be generated, it may be convenient to set up a |Makefile| or
shell script to automatise the process.

%%%%%%%%%%%%%%%%%%%%%%%%%%%%%%%%%%%%%%%%%%%%%%%%%%%%%%%%%%%%%%%%%%%%%%%%%%%%%%%%
\subsection{Command Line Processing}
\label{sec:commandline}

The effect of redirection files can also be achieved by invoking
the \LaTeX{} compiler with a more elaborate command line.
Most conveniently this should be done as part
of a shell script or a |Makefile|.

When using \textsf{childdoc} in the main file, the following
command lines effectively perform a redirection
(note that depending on the shell being used,
backslashes may have to be doubled: `|\|' $\to$ `|\\|'):
%
\begin{center}
|... -jobname "|\textit{target}|" |\\|"|[\textit{flags}]%
|\input{childdoc.def}\childdocforward[|\textit{main}|]{|\textit{dest}|}"|
\end{center}
%
Here \textit{target} is the name of the output file,
\textit{main} is the name of the main file
and \textit{dest} is the name of the main or child file to be processed
(all filenames without extensions).
The optional argument \textit{main} can be omitted
if \textit{main} matches \textit{dest}.
Optionally, compilation \textit{flags} can be defined via |\def| commands.
This command line makes the \TeX{} engine believe
it is compiling the file \textit{target}
whose content is specified as the latter parameter.
The provided code then forwards the processing to
\textit{main} or \textit{dest} as described in \secref{sec:forward}.

%%%%%%%%%%%%%%%%%%%%%%%%%%%%%%%%%%%%%%%%%%%%%%%%%%%%%%%%%%%%%%%%%%%%%%%%%%%%%%%%
\subsection{Include by Input}
\label{sec:input}

Including child documents by |\include| has some restrictions by design.
Most notably, the content of a child document always occupies
its own set of pages; pages cannot be shared between child documents.
Usually, this behaviour makes perfect sense
because each child document contain an essential part of the document.
However, in some situations it may be desirable to compose
a document from a collection of parts
without having mandatory page breaks between then.
For this case, the package
provides a mechanism to include parts
by |\input| which can also be processed individually.
However, by construction this mechanism
requires manual handling of the content to be output.

%%%%%%%%%%%%%%%%%%%%%%%%%%%%%%%%%%%%%%%%
\DescribeMacro{\ifchilddocmanual}
The main file should be prepared as usual, see \secref{sec:include}.
However, the document body must make a distinction
between processing of an individual part and of the main document, e.g.:
%
\begin{center}
\begin{tabular}{l}
|\ifchilddocmanual|\\
|\input{\childdocname}|\\
|\||else|\\
\textit{document body with }|\input{|\textit{part}|}|\\
|\||fi|
\end{tabular}
\end{center}
%
The conditional |\ifchilddocmanual| is true whenever
a part to be included by |\input| is being compiled,
and the name of the part is stored in |\childdocname|.

%%%%%%%%%%%%%%%%%%%%%%%%%%%%%%%%%%%%%%%%
\DescribeMacro{\childdocby}
Each part to be included by |\input| should start with:
%
\begin{center}
\begin{tabular}{l}
|\input{childdoc.def}|\\
|\childdocby{|\textit{main}|}|\\
\end{tabular}
\end{center}
%
The directive |\childdocby| is similar to |\childdocof|
described in \secref{sec:include},
but the subsequent selection of content must be done manually.
To that end, both |\ifchilddoc| and |\ifchilddocmanual|
will be true upon processing of a part,
and the name of the part is stored in |\childdocname|.
Note that |\jobname| will be set to the filename of the current part
so that each part receives an individual |.aux| file
that does not interfere with the |.aux| file(s) of the main document.
This behaviour can be altered by the alternative form
|\childdocby[*]{|\textit{main}|}| (with a non-empty optional argument)
which uses the |.aux| file of the main document
by setting |\jobname| to \textit{main}.

%%%%%%%%%%%%%%%%%%%%%%%%%%%%%%%%%%%%%%%%%%%%%%%%%%%%%%%%%%%%%%%%%%%%%%%%%%%%%%%%
\subsection{Driver Development}
\label{sec:driver}

The \textsf{childdoc} mechanism can also be use for the development
of definition files such as \LaTeX{} styles or classes.
This case differs from the above setup with multiple parts
included by |\include| in that no |\includeonly| should be invoked.
This can be achieved by starting the include file
(before |\ProvidesPackage|) with:
%
\begin{center}
\begin{tabular}{l}
|\input{childdoc.def}|\\
|\childdocforward{|\textit{main}|}|\\
\end{tabular}
\end{center}
%
or alternatively with:
%
\begin{center}
\begin{tabular}{l}
|\input{childdoc.def}|\\
|\childdocby{|\textit{main}|}|\\
\end{tabular}
\end{center}
%
Both forms have slightly different effects as described above.
The main file is prepared as usual, see \secref{sec:include}.

%%%%%%%%%%%%%%%%%%%%%%%%%%%%%%%%%%%%%%%%%%%%%%%%%%%%%%%%%%%%%%%%%%%%%%%%%%%%%%%%
\subsection{Legacy Detection}
\label{sec:detection}

The directive |\childdocmain| in the main file can detect
whether the complete document or merely a child is to be compiled
even without using the directive |\childdocof|.
This method is deprecated because it is less robust
and there is no compelling reason to use it;
it is merely provided for backward compatibility
and it may be removed in future versions.

If the detection mechanism is to be used,
it is mandatory to correctly specify
the filename of the main file as the argument of |\childdocmain|:
%
\begin{center}
\begin{tabular}{l}
|\input{childdoc.def}|\\
|\childdocmain{|\textit{main}|}|\\
\end{tabular}
\end{center}
%
If |\jobname| does not match the argument \textit{main} of |\childdocmain|,
it is assumed that |\jobname| points to the child file to be compiled.
When using |\childdocmain| with the main file specified as argument,
it suffices to start a child file
with just |\input{|\textit{main}|}|
without loading of the package and using |\childdocof|.
If instead all processing is done
with the appropriate \textsf{childdoc} directives,
the argument of \textit{main} of |\childdocmain| can be empty.

An alternative version of the command line processing described
in \secref{sec:commandline} using the detection mechanism reads:
%
\begin{center}
|... -jobname "|\textit{target}|" "|[\textit{flags}]%
[|\def\jobname{|\textit{dest}|}|]|\input{|\textit{main}|}"|
\end{center}

%%%%%%%%%%%%%%%%%%%%%%%%%%%%%%%%%%%%%%%%%%%%%%%%%%%%%%%%%%%%%%%%%%%%%%%%%%%%%%%%
\subsection{Manual Code}
\label{sec:manual}

In case one cannot be certain whether the definitions file |childdoc.def|
is installed on the target \TeX{} distribution
and one prefers not to ship it,
it is conceivable to paste a few relevant commands into the sources.

To that end, drop all statements |\input{childdoc.def}|
and perform the replacements as outlined below.
Instead of |\childdocmain{|\textit{main}|}| add the following code
to the top of the main file:
%
\begin{center}
\begin{tabular}{l}
|\||ifdefined\childdocname\endinput\||fi\newif\ifchilddoc|\\
|\edef\childdocname{\scantokens\expandafter{\jobname\noexpand}}|\\
|\def\childdocmain{|\textit{main}|}\||ifx\childdocmain\childdocname\||else|\\
|\childdoctrue\includeonly{\childdocname}\let\jobname\childdocmain\||fi|\\
\end{tabular}
\end{center}
%
Instead of |\childdocof{|\textit{main}|}| just include the main file
at the top of each child file:
%
\begin{center}
|\input{|\textit{main}|}|
\end{center}
%
A simple redirection |\childdocforward{|\textit{dest}|}| is achieved by:
%
\begin{center}
|\def\jobname{|\textit{dest}|}\input{\jobname}|
\end{center}
%
The redirection with prefix
|\childdocforwardprefix[|\textit{prefix}|]{|\textit{dest}|}|
is accomplished by:
%
\begin{center}
\begin{tabular}{l}
|{\edef\jobname{\scantokens\expandafter{\jobname\noexpand}}|\\
|\def\redirectjob |\textit{prefix}|#1~~~{\gdef\jobname{|\textit{dest}|#1}}|\\
|\expandafter\redirectjob\jobname~~~}\input{\jobname}|
\end{tabular}
\end{center}

In an alternative approach,
child documents can be compiled by a specific command line
without additional code or specific definitions:
%
\begin{center}
|... -jobname "|\textit{target}|" "|[\textit{flags}]%
|\includeonly{|\textit{dest}|}\input{|\textit{main}|}"|
\end{center}
%

%%%%%%%%%%%%%%%%%%%%%%%%%%%%%%%%%%%%%%%%%%%%%%%%%%%%%%%%%%%%%%%%%%%%%%%%%%%%%%%%
%%%%%%%%%%%%%%%%%%%%%%%%%%%%%%%%%%%%%%%%%%%%%%%%%%%%%%%%%%%%%%%%%%%%%%%%%%%%%%%%
\section{Information}

%%%%%%%%%%%%%%%%%%%%%%%%%%%%%%%%%%%%%%%%%%%%%%%%%%%%%%%%%%%%%%%%%%%%%%%%%%%%%%%%
\subsection{Copyright}

Copyright \copyright{} 2017--2018 Niklas Beisert

This work may be distributed and/or modified under the
conditions of the \LaTeX{} Project Public License, either version 1.3
of this license or (at your option) any later version.
The latest version of this license is in
  \url{http://www.latex-project.org/lppl.txt}
and version 1.3 or later is part of all distributions of \LaTeX{}
version 2005/12/01 or later.

This work has the LPPL maintenance status `maintained'.

The Current Maintainer of this work is Niklas Beisert.

This work consists of the files |README.txt|, |childdoc.ins| and |childdoc.dtx|
as well as the derived files |childdoc.def|, |cdocsamp.tex|
with |cdocsch1.tex|, |cdocsch2.tex|, |cdocspt3.tex|, |cdocspt4.tex|,
|cdocsdrf.tex|, |cdocsfn1.tex|, |cdocsfn2.tex|
as well as |childdoc.pdf|.

%%%%%%%%%%%%%%%%%%%%%%%%%%%%%%%%%%%%%%%%%%%%%%%%%%%%%%%%%%%%%%%%%%%%%%%%%%%%%%%%
\subsection{Files and Installation}

The package consists of the files:
%
\begin{center}
\begin{tabular}{ll}
    |README.txt|   & readme file \\
    |childdoc.ins| & installation file \\
    |childdoc.dtx| & source file \\
    |childdoc.def| & definition file \\
    |cdocsamp.tex| & sample main file \\
    |cdocsch1.tex| & sample include file \\
    |cdocsch2.tex| & sample include file \\
    |cdocspt3.tex| & sample part file \\
    |cdocspt4.tex| & sample part file \\
    |cdocsdrf.tex| & sample redirection file \\
    |cdocsfn1.tex| & sample redirection file \\
    |cdocsfn2.tex| & sample redirection file \\
    |childdoc.pdf| & manual
\end{tabular}
\end{center}
%
The distribution consists of the files
|README.txt|, |childdoc.ins| and |childdoc.dtx|.
%
\begin{itemize}
\item
Run (pdf)\LaTeX{} on |childdoc.dtx|
to compile the manual |childdoc.pdf| (this file).
\item
Run \LaTeX{} on |childdoc.ins| to create the definitions file |childdoc.def|
and the sample |cdocsamp.tex| with include files
|cdocsch1.tex|, |cdocsch2.tex|, |cdocspt3.tex|, |cdocspt4.tex|,
|cdocsdrf.tex|, |cdocsfn1.tex|, |cdocsfn2.tex|.
Then copy the file |childdoc.def| to an appropriate directory of your \LaTeX{}
distribution, e.g.\ \textit{texmf-root}|/tex/latex/childdoc|.
\end{itemize}

%%%%%%%%%%%%%%%%%%%%%%%%%%%%%%%%%%%%%%%%%%%%%%%%%%%%%%%%%%%%%%%%%%%%%%%%%%%%%%%%
\subsection{Related CTAN Packages}

There are several other packages which offer a similar functionality:
%
\begin{itemize}
\item
The packages
\href{http://ctan.org/pkg/docmute}{\textsf{docmute}},
\href{http://ctan.org/pkg/includex}{\textsf{includex}} and
\href{http://ctan.org/pkg/standalone}{\textsf{standalone}}
provide commands to include only the document body of
a child file thus allowing both files to be compiled individually.
\item
The packages \href{http://ctan.org/pkg/subdocs}{\textsf{subdocs}}
and \href{http://ctan.org/pkg/subfiles}{\textsf{subfiles}}
provide structures in which the main and child documents can be
encapsulated and allowing them to be compiled individually.
The inclusion mechanism is different from the conventional |\include|.
\item
The package \href{http://ctan.org/pkg/combine}{\textsf{combine}}
is an elaborate solution to combine several documents into one.
\end{itemize}
%
See also the CTAN topic \href{http://ctan.org/topic/subdocs}{\textsf{subdocs}}
for further related packages.
The present package differs from the above solutions in that
a document structure constructed with the conventional |\include| mechanism
just needs two extra commands at the top of every file
such that all constituent files can be compiled individually.

%%%%%%%%%%%%%%%%%%%%%%%%%%%%%%%%%%%%%%%%%%%%%%%%%%%%%%%%%%%%%%%%%%%%%%%%%%%%%%%%
%\subsection{Feature Suggestions}
%
%The following is a list of features which may be useful for future
%versions of this package:
%%
%\begin{itemize}
%\item
%\ldots
%\end{itemize}

%%%%%%%%%%%%%%%%%%%%%%%%%%%%%%%%%%%%%%%%%%%%%%%%%%%%%%%%%%%%%%%%%%%%%%%%%%%%%%%%
\subsection{Revision History}

%%%%%%%%%%%%%%%%%%%%%%%%%%%%%%%%%%%%%%%%
\paragraph{v2.0:} 2018/12/30

\begin{itemize}
\item
immediate forward processing
\item
added |\childdocby| mechanism
\item
manual restructured
\end{itemize}

%%%%%%%%%%%%%%%%%%%%%%%%%%%%%%%%%%%%%%%%
\paragraph{v1.6:} 2018/01/17

\begin{itemize}
\item
application for development of include files
\item
corrections to manual
\end{itemize}

%%%%%%%%%%%%%%%%%%%%%%%%%%%%%%%%%%%%%%%%
\paragraph{v1.5:} 2017/05/21

\begin{itemize}
\item
more complete structuring introduced
\item
|\childdocof| introduced
\item
|\childdoc| renamed to |\childdocmain|
\item
|\childredirect| renamed to |\childdocforward| and |\childdocforwardprefix|
and functionality expanded
\end{itemize}

%%%%%%%%%%%%%%%%%%%%%%%%%%%%%%%%%%%%%%%%
\paragraph{v1.0:} 2017/04/27

\begin{itemize}
\item
manual and install package
\item
first version published on CTAN
\end{itemize}

%%%%%%%%%%%%%%%%%%%%%%%%%%%%%%%%%%%%%%%%
\paragraph{v0.6:} 2017/04/26

\begin{itemize}
\item
redirection mechanism added
\end{itemize}

%%%%%%%%%%%%%%%%%%%%%%%%%%%%%%%%%%%%%%%%
\paragraph{v0.5:} 2017/04/26

\begin{itemize}
\item
functionality in definition file
\end{itemize}


%%%%%%%%%%%%%%%%%%%%%%%%%%%%%%%%%%%%%%%%%%%%%%%%%%%%%%%%%%%%%%%%%%%%%%%%%%%%%%%%
%%%%%%%%%%%%%%%%%%%%%%%%%%%%%%%%%%%%%%%%%%%%%%%%%%%%%%%%%%%%%%%%%%%%%%%%%%%%%%%%
%%%%%%%%%%%%%%%%%%%%%%%%%%%%%%%%%%%%%%%%%%%%%%%%%%%%%%%%%%%%%%%%%%%%%%%%%%%%%%%%
\appendix

\settowidth\MacroIndent{\rmfamily\scriptsize 000\ }

 \DocInput{childdoc.dtx}

\end{document}
%</driver>
% \fi
%
% %%%%%%%%%%%%%%%%%%%%%%%%%%%%%%%%%%%%%%%%%%%%%%%%%%%%%%%%%%%%%%%%%%%%%%%%%%%%%%
% %%%%%%%%%%%%%%%%%%%%%%%%%%%%%%%%%%%%%%%%%%%%%%%%%%%%%%%%%%%%%%%%%%%%%%%%%%%%%%
% \section{Sample}
%\iffalse
%<*samplemain>
%\fi
%
% The following presents a sample document
% with two chapters, two parts, a title page,
% a compile flag as well as three forwarding files to set the flag.
% It consists of eight |.tex| files:
% \begin{center}
% \begin{tabular}{ll}
% |cdocsamp.tex|&main file\\
% |cdocsch1.tex|&include file for chapter 1\\
% |cdocsch2.tex|&include file for chapter 2\\
% |cdocspt3.tex|&include file for part 3\\
% |cdocspt4.tex|&include file for part 4\\
% |cdocsdrf.tex|&forwarding file for main file in draft mode\\
% |cdocsfi1.tex|&forwarding file for final version of chapter 1\\
% |cdocsfi2.tex|&forwarding file for final version of chapter 2\\
% \end{tabular}
% \end{center}
% Each of the eight files can be compiled directly by the \LaTeX{} compiler.
%
% %%%%%%%%%%%%%%%%%%%%%%%%%%%%%%%%%%%%%%
% \paragraph{Main File.}
%
% The main file is called |cdocsamp.tex|.
%
% Load the \textsf{childdoc} definitions and
% declare the filename for the main document:
%    \begin{macrocode}
\input{childdoc.def}
\childdocmain{}
%    \end{macrocode}

% Optional override for |\version| flag:
%    \begin{macrocode}
%%\ifchilddoc\else\providecommand{\version}{draft}\fi
%    \end{macrocode}

% Define the default values for the |\version| flag
% (|final| for the main file and |draft| for childs):
%    \begin{macrocode}
\ifchilddoc
\providecommand{\version}{draft}
\else
\providecommand{\version}{final}
\fi
%    \end{macrocode}

% Load the standard document class:
%    \begin{macrocode}
\documentclass[12pt]{article}
%    \end{macrocode}

% Start the document body:
%    \begin{macrocode}
\begin{document}
%    \end{macrocode}

% Declare a title page.
% Print title, part of document being processed and version flag:
%    \begin{macrocode}
\addtocounter{page}{-1}
\begin{center}
{\LARGE\bfseries{}childdoc example\par}
\vspace{1cm}
\ifchilddoc
\ifchilddocmanual part\else chapter\fi:
`\childdocname' of `\childdocjob'\par
\else
main document: `\childdocjob'\par
\fi
version: \version\par
\end{center}
\newpage
%    \end{macrocode}

% Manually include selected file,
% otherwise process as usual:
%    \begin{macrocode}
\ifchilddocmanual
\section*{part `\childdocname'}
\input{\childdocname}
\else
%    \end{macrocode}

% Include the two chapters:
%    \begin{macrocode}
\include{cdocsch1}
\include{cdocsch2}
%    \end{macrocode}

% Include the two parts unless only chapters should be displayed:
%    \begin{macrocode}
\ifchilddoc\else
\section{part three}
\input{cdocspt3}
\section{part four}
\input{cdocspt4}
\fi
%    \end{macrocode}

% Process as usual until here:
%    \begin{macrocode}
\fi
%    \end{macrocode}

% End of document body:
%    \begin{macrocode}
\end{document}
%    \end{macrocode}
%\iffalse
%</samplemain>
%\fi
%
% %%%%%%%%%%%%%%%%%%%%%%%%%%%%%%%%%%%%%%
% \paragraph{Chapter Include Files.}
%
% The include files are called |cdocsch1.tex| and |cdocsch2.tex|.
%
%\iffalse
%<*samplechap1|samplechap2>
%\fi

% Optional override for |\version| flag:
%    \begin{macrocode}
%%\providecommand{\version}{final}
%    \end{macrocode}

% Include the main document:
%    \begin{macrocode}
\input{childdoc.def}
\childdocof{cdocsamp}
%    \end{macrocode}

%\iffalse
%</samplechap1|samplechap2>
%\fi
%
%\iffalse
%<*samplechap1>
%\fi
% Some text for chapter 1:
%    \begin{macrocode}
\section{one}
some text in chapter one
%    \end{macrocode}

%\iffalse
%</samplechap1>
%\fi
% Some text for chapter 2:
%\iffalse
%<*samplechap2>
%\fi
%    \begin{macrocode}
\section{two}
more text in chapter two
%    \end{macrocode}

%\iffalse
%</samplechap2>
%\fi
%
% %%%%%%%%%%%%%%%%%%%%%%%%%%%%%%%%%%%%%%
% \paragraph{Part Include Files.}
%
% The include files are called |cdocspt3.tex| and |cdocspt4.tex|.
%
%\iffalse
%<*samplepart3|samplepart4>
%\fi

% Optional override for |\version| flag:
%    \begin{macrocode}
%%\providecommand{\version}{final}
%    \end{macrocode}

% Include the main document:
%    \begin{macrocode}
\input{childdoc.def}
\childdocby{cdocsamp}
%    \end{macrocode}

%\iffalse
%</samplepart3|samplepart4>
%\fi
%
%\iffalse
%<*samplepart3>
%\fi
% Some text for part 3:
%    \begin{macrocode}
some text in part three
%    \end{macrocode}

%\iffalse
%</samplepart3>
%\fi
% Some text for part 4:
%\iffalse
%<*samplepart4>
%\fi
%    \begin{macrocode}
more text in part four
%    \end{macrocode}

%\iffalse
%</samplepart4>
%\fi
%
% %%%%%%%%%%%%%%%%%%%%%%%%%%%%%%%%%%%%%%
% \paragraph{Forwarding for a Complete Draft.}
%
% The following forwarding file |cdocsdrf.tex|
% compiles the main document in draft mode:
%\iffalse
%<*sampledraft>
%\fi
%    \begin{macrocode}
\def\version{draft}
\input{childdoc.def}
\childdocforward{cdocsamp}
%    \end{macrocode}

%\iffalse
%</sampledraft>
%\fi
%
% %%%%%%%%%%%%%%%%%%%%%%%%%%%%%%%%%%%%%%
% \paragraph{Forwarding for Final Version of the Chapters.}
%
% The following forwarding files |cdocsfn1.tex| and |cdocsfn2.tex|
% (with identical content)
% compile the final versions of the child documents
% |cdocsch1.tex| and |cdocsch2.tex|, respectively:
%\iffalse
%<*samplefinal>
%\fi
%    \begin{macrocode}
\def\version{final}
\input{childdoc.def}
\childdocforwardprefix[cdocsamp]{cdocsfn}{cdocsch}
%    \end{macrocode}

%\iffalse
%</samplefinal>
%\fi
%
% %%%%%%%%%%%%%%%%%%%%%%%%%%%%%%%%%%%%%%
% \paragraph{Command Line Processing.}
%
% The following three command lines generate the output files
% |cdocscld|, |cdocscl1| and |cdocscl2|
% which should be identical to
% |cdocsdrf|, |cdocsch1| and |cdocsfn2|, respectively:
% \begin{center}
% \begin{tabular}{l}
% |latex -jobname cdocscld \|\\
% |  "\def\version{draft}\input{childdoc.def}\childdocforward{cdocsamp}"|\\
% |latex -jobname cdocscl1 \|\\
% |  "\input{childdoc.def}\childdocforward[cdocsamp]{cdocsch1}"|\\
% |latex -jobname cdocscl2 \|\\
% |  "\def\version{final}\input{childdoc.def}\childdocforward{cdocsch2}"|
% \end{tabular}
% \end{center}
% Note that the trailing backslash on each first line
% merely continues the input to the second line
% (for convenient cut ant paste).
% Furthermore, the command |latex| can be replaced by any
% of its alternative versions such as |pdflatex|.
%
% %%%%%%%%%%%%%%%%%%%%%%%%%%%%%%%%%%%%%%%%%%%%%%%%%%%%%%%%%%%%%%%%%%%%%%%%%%%%%%
% %%%%%%%%%%%%%%%%%%%%%%%%%%%%%%%%%%%%%%%%%%%%%%%%%%%%%%%%%%%%%%%%%%%%%%%%%%%%%%
% \section{Implementation}
%\iffalse
%<*package>
%\fi
%
% This section describes the definitions file |childdoc.def|.

% The definitions cannot be loaded using |\usepackage| or |\RequirePackage|
% which has a mechanism to prevent loading a style file more than once.
% When loading the definitions by means of |\input|
% multiple instances have to be prevented manually:
%\iffalse
%This code needs to be before the `\ProvidesFile' directive
%which is defined at the beginning of this file.
%Therefore it is also placed there and commented out here.
%</package>
%<*discard>
%\fi
%    \begin{macrocode}
\ifdefined\childdocmain\endinput\fi
%    \end{macrocode}
%\iffalse
%</discard>
%<*package>
%\fi
%
% \macro{\ifchilddoc}
% \macro{\ifchilddocmanual}
% The conditional |\ifchilddoc| tells whether a
% child (true) or main (false) document is being compiled.
% The conditional |\ifchilddocmanual| tells whether
% the |\includeonly| mechanism is used (false) or
% the selection of child files must be performed manually (true).
% The definitions initialise to false:
%    \begin{macrocode}
\newif\ifchilddoc
\newif\ifchilddocmanual
%    \end{macrocode}

% \macro{\childdocname}
% \macro{\childdocjob}
% The macro |\childdocname| stores the name of the main document
% to be compiled. The macro |\childdocjob| stores the name of
% the document on which the \LaTeX{} compiler was originally invoked.
% The content of |\jobname| cannot be compared
% to filenames specified in the source due to different catcodes.
% The following code rescans |\jobname|, stores the result
% in |\childdocname| and saves a copy in |\childdocjob|:
%    \begin{macrocode}
\edef\childdocname{\scantokens\expandafter{\jobname\noexpand}}
\let\childdocjob\childdocname
%    \end{macrocode}

% \macro{\childdocdisable}
% The macro |\childdocdisable| prevents the main file
% from being processed more than once.
% At this stage, the main document command |\childdocmain|
% is assumed to be called once again where it should do nothing.
% Any subsequent call to it should prevent
% a secondary processing of the main document
% It overwrites the forwarding commands
% |\childdocof| and |\childdocforward|
% with empty macros to prevent further inclusions of the main document:
%    \begin{macrocode}
\newcommand{\childdocdisable}
{
  \renewcommand{\childdocmain}[1]{\renewcommand{\childdocmain}[1]{\endinput}}
  \renewcommand{\childdocof}[1]{}
  \renewcommand{\childdocby}[2][]{}
  \renewcommand{\childdocforward}[2][]{}
  \renewcommand{\childdocdisable}{}
}
%    \end{macrocode}

% \macro{\childdocmain}
% The macro |\childdocmain| is to be called at the top of the main file
% with nothing or the main filename (without extension) as argument.
% First, it breaks loops.
% If the argument is not empty and does not match |\childdocname|
% (which is set by the first inclusion of |childdoc.def|),
% |\ifchilddoc| is set to true, |\includeonly| is applied to the child file
% and |\jobname| is set to the main file
% (for proper handling of |.aux| files):
%    \begin{macrocode}
\newcommand{\childdocmain}[1]
{
  \childdocdisable\childdocmain{}
  \if?#1?\else
    \begingroup
      \def\childdoctmp{#1}
      \ifx\childdoctmp\childdocname
        \def\childdoctmp{}
      \else
        \def\childdoctmp
        {
          \childdoctrue
          \includeonly{\childdocname}
          \def\childdocjob{#1}
          \def\jobname{#1}
        }
      \fi
      \expandafter
    \endgroup
    \childdoctmp
  \fi
}
%    \end{macrocode}

% \macro{\childdocof}
% The command |\childdocof| redirects
% compilation to the main file |#1|.
%    \begin{macrocode}
\newcommand{\childdocof}[1]
{
  \childdocdisable
  \childdoctrue
  \includeonly{\childdocname}
  \def\jobname{#1}
  \def\childdocjob{#1}
  \input{#1}
}
%    \end{macrocode}

% \macro{\childdocby}
% The command |\childdocby| ....
%    \begin{macrocode}
\newcommand{\childdocby}[2][]
{
  \childdocdisable
  \childdoctrue
  \childdocmanualtrue
  \if?#1?\else
    \def\jobname{#2}
  \fi
  \def\childdocjob{#2}
  \input{#2}
  \endinput
}
%    \end{macrocode}

% \macro{\childdocforward}
% The command |\childdocforward| redirects
% compilation to the main file or
% (if the optional argument is given) a child file.
% Parameters are set as if the main file
% or a child file starting with |\childdocof| was compiled.
% Then compilation is handed over to the main file:
%    \begin{macrocode}
\newcommand{\childdocforward}[2][]
{
  \begingroup
    \if?#1?
      \def\childdoctmp
      {
        \def\childdocname{#2}
        \def\childdocjob{#2}
        \def\jobname{#2}
        \input{#2}
        \endinput
      }
    \else
      \def\childdoctmp
      {
        \childdocdisable
        \def\childdocname{#2}
        \childdoctrue
        \includeonly{#2}
        \def\childdocjob{#1}
        \def\jobname{#1}
        \input{#1}
        \endinput
      }
    \fi
    \expandafter
  \endgroup
  \childdoctmp
}
%    \end{macrocode}

% \macro{\childdocforwardprefix}
% The command |\childdocforwardprefix| redirects
% compilation to the main or a child file by means of a pattern.
% The prefix |#1| in the current filename is replaced by |#2|
% and the suffix of the current filename is kept
% (it is assumed that the filename does not contain the substring `|~~~|'
% which is used as a delimiter).
% Compilation is handed over to the new file by |\childdocforward|:
%    \begin{macrocode}
\newcommand{\childdocforwardprefix}[3][]
{
  \begingroup
    \def\childdocextract #2##1~~~{\def\childdoctmp{\childdocforward[#1]{#3##1}}}
    \expandafter\childdocextract\childdocname~~~
    \expandafter
  \endgroup
  \childdoctmp
}
%    \end{macrocode}

% \macro{\childdoc}
% The deprecated macro |\childdoc| is a legacy version of |\childdocmain|:
%    \begin{macrocode}
\newcommand{\childdoc}{\childdocmain}
%    \end{macrocode}

% \macro{\childdocredirect}
% The deprecated macro |\childdocredirect| is a legacy version
% of |\childdocforward| and |\childdocforwardprefix|:
%    \begin{macrocode}
\newcommand{\childdocredirect}[2][]
{
  \begingroup
    \if?#1?
      \def\childdoctmp{\childdocforward{#2}}
    \else
      \def\childdoctmp{\childdocforwardprefix{#1}{#2}}
    \fi
    \expandafter
  \endgroup
  \childdoctmp
}
%    \end{macrocode}

%\iffalse
%</package>
%\fi
%
\endinput

\childdocmain{}
%    \end{macrocode}

% Optional override for |\version| flag:
%    \begin{macrocode}
%%\ifchilddoc\else\providecommand{\version}{draft}\fi
%    \end{macrocode}

% Define the default values for the |\version| flag
% (|final| for the main file and |draft| for childs):
%    \begin{macrocode}
\ifchilddoc
\providecommand{\version}{draft}
\else
\providecommand{\version}{final}
\fi
%    \end{macrocode}

% Load the standard document class:
%    \begin{macrocode}
\documentclass[12pt]{article}
%    \end{macrocode}

% Start the document body:
%    \begin{macrocode}
\begin{document}
%    \end{macrocode}

% Declare a title page.
% Print title, part of document being processed and version flag:
%    \begin{macrocode}
\addtocounter{page}{-1}
\begin{center}
{\LARGE\bfseries{}childdoc example\par}
\vspace{1cm}
\ifchilddoc
\ifchilddocmanual part\else chapter\fi:
`\childdocname' of `\childdocjob'\par
\else
main document: `\childdocjob'\par
\fi
version: \version\par
\end{center}
\newpage
%    \end{macrocode}

% Manually include selected file,
% otherwise process as usual:
%    \begin{macrocode}
\ifchilddocmanual
\section*{part `\childdocname'}
\input{\childdocname}
\else
%    \end{macrocode}

% Include the two chapters:
%    \begin{macrocode}
\include{cdocsch1}
\include{cdocsch2}
%    \end{macrocode}

% Include the two parts unless only chapters should be displayed:
%    \begin{macrocode}
\ifchilddoc\else
\section{part three}
\input{cdocspt3}
\section{part four}
\input{cdocspt4}
\fi
%    \end{macrocode}

% Process as usual until here:
%    \begin{macrocode}
\fi
%    \end{macrocode}

% End of document body:
%    \begin{macrocode}
\end{document}
%    \end{macrocode}
%\iffalse
%</samplemain>
%\fi
%
% %%%%%%%%%%%%%%%%%%%%%%%%%%%%%%%%%%%%%%
% \paragraph{Chapter Include Files.}
%
% The include files are called |cdocsch1.tex| and |cdocsch2.tex|.
%
%\iffalse
%<*samplechap1|samplechap2>
%\fi

% Optional override for |\version| flag:
%    \begin{macrocode}
%%\providecommand{\version}{final}
%    \end{macrocode}

% Include the main document:
%    \begin{macrocode}
% \iffalse
%
% childdoc.dtx Copyright (C) 2017-2018 Niklas Beisert
%
% This work may be distributed and/or modified under the
% conditions of the LaTeX Project Public License, either version 1.3
% of this license or (at your option) any later version.
% The latest version of this license is in
%   http://www.latex-project.org/lppl.txt
% and version 1.3 or later is part of all distributions of LaTeX
% version 2005/12/01 or later.
%
% This work has the LPPL maintenance status `maintained'.
%
% The Current Maintainer of this work is Niklas Beisert.
%
% This work consists of the files childdoc.dtx and childdoc.ins
% and the derived files childdoc.def and cdocsamp.tex with
% cdocsch1.tex, cdocsch2.tex, cdocsdrf.tex, cdocsfn1.tex, cdocsfn2.tex.
%
%<package>\ifdefined\childdocmain\endinput\fi
%<package>\ProvidesFile{childdoc.def}[2018/12/30 v2.0 child document driver]
%<samplemain>\ProvidesFile{cdocsamp.tex}[2018/12/30 v2.0 sample for childdoc]
%<*driver>
%\ProvidesFile{childdoc.drv}[2018/12/30 v2.0 childdoc reference manual file]
\PassOptionsToClass{10pt,a4paper}{article}
\documentclass{ltxdoc}

\usepackage[margin=35mm]{geometry}
\usepackage{hyperref}
\usepackage{hyperxmp}
\usepackage[usenames]{color}

\hypersetup{colorlinks=true}
\hypersetup{pdfstartview=FitH}
\hypersetup{pdfpagemode=UseNone}
\hypersetup{pdfsource={}}
\hypersetup{pdflang={en-UK}}
\hypersetup{pdfcopyright={Copyright 2017-2018 Niklas Beisert.
  This work may be distributed and/or modified under the
  conditions of the LaTeX Project Public License, either version 1.3
  of this license or (at your option) any later version.}}
\hypersetup{pdflicenseurl={http://www.latex-project.org/lppl.txt}}
\hypersetup{pdfcontactaddress={ETH Zurich, ITP, HIT K,
  Wolfgang-Pauli-Strasse 27}}
\hypersetup{pdfcontactpostcode={8093}}
\hypersetup{pdfcontactcity={Zurich}}
\hypersetup{pdfcontactcountry={Switzerland}}
\hypersetup{pdfcontactemail={nbeisert@itp.phys.ethz.ch}}
\hypersetup{pdfcontacturl={http://people.phys.ethz.ch/\xmptilde nbeisert/}}

\newcommand{\secref}[1]{\hyperref[#1]{section \ref*{#1}}}

\parskip1ex
\parindent0pt
\let\olditemize\itemize
\def\itemize{\olditemize\parskip0pt}

\begin{document}

\title{The \textsf{childdoc} Package}
\hypersetup{pdftitle={The childdoc Package}}
\author{Niklas Beisert\\[2ex]
  Institut f\"ur Theoretische Physik\\
  Eidgen\"ossische Technische Hochschule Z\"urich\\
  Wolfgang-Pauli-Strasse 27, 8093 Z\"urich, Switzerland\\[1ex]
  \href{mailto:nbeisert@itp.phys.ethz.ch}
  {\texttt{nbeisert@itp.phys.ethz.ch}}}
\hypersetup{pdfauthor={Niklas Beisert}}
\hypersetup{pdfsubject={Manual for the LaTeX2e Package childdoc}}
\date{30 December 2018, \textsf{v2.0}}
\maketitle

\begin{abstract}\noindent
\textsf{childdoc} is a \LaTeXe{} package
that enables the direct compilation
of document sections included by |\include|
to individual files.
\end{abstract}

\begingroup
\parskip0ex
\tableofcontents
\endgroup

%%%%%%%%%%%%%%%%%%%%%%%%%%%%%%%%%%%%%%%%%%%%%%%%%%%%%%%%%%%%%%%%%%%%%%%%%%%%%%%%
%%%%%%%%%%%%%%%%%%%%%%%%%%%%%%%%%%%%%%%%%%%%%%%%%%%%%%%%%%%%%%%%%%%%%%%%%%%%%%%%
\section{Introduction}

\LaTeX{} provides a mechanism to structure a large document (such as a book)
into a main file and several child files (containing the chapters)
using the |\include| command.
This mechanism is beneficial for documents
which span hundreds of pages in order to
make the source file(s) more manageable.
Moreover, compilation can be restricted to
selected child files by means of the |\includeonly| command.
The latter feature can be used to reduce the compilation time while editing
(this was significantly more useful in the earlier days of \LaTeX{})
or to generate a smaller document which is easier to navigate.
Another application of |\includeonly| is to generate
documents consisting of selected parts of the complete document.

However, there are a few drawbacks of the plain |\include| mechanism:
\begin{itemize}
\item
The child files cannot be compiled on their own,
they can only be compiled via the main file.
A naive editing environment
(such as a text editor with an option
to have the current file processed by \LaTeX)
may require one to switch to the main file before compiling;
attempting to compile the child file produces errors.
\item
The main file must be modified (each time)
to adjust the |\includeonly| command
to the present needs. This easily leaves the main file in a messy state.
\item
The generated document will always carry the filename
of the main document. This is inconvenient if
several child files are to be compiled and
to be kept for distribution.
\end{itemize}

The present package provides a simple interface
to make child files individually compilable by \LaTeX{}.
Compiling a child file then has the same effect as compiling
the main file with an |\includeonly| command
to select the appropriate child.
Moreover the generated document will carry the name of the child
rather than the main file.
This resolves all three above issues.

This feature is meant to make the editing of books,
thesis documents and lecture notes somewhat more convenient.
However, the package can also be used efficiently for
composing a series of documents (such as exercise sheets)
which are typically distributed individually.
It then assists the author in generating the individual documents
(potentially in different versions)
as well as a document containing the collected series.
Another application is in developing style files
or other kinds of included material
where compilation of the style file could redirect
to a sample or test file.

%%%%%%%%%%%%%%%%%%%%%%%%%%%%%%%%%%%%%%%%%%%%%%%%%%%%%%%%%%%%%%%%%%%%%%%%%%%%%%%%
%%%%%%%%%%%%%%%%%%%%%%%%%%%%%%%%%%%%%%%%%%%%%%%%%%%%%%%%%%%%%%%%%%%%%%%%%%%%%%%%
\section{Usage}

First of all, the package \textsf{childdoc} is \emph{not} a standard
\LaTeXe{} |.sty| style file! Therefore it needs to be invoked in
a non-standard way.

%%%%%%%%%%%%%%%%%%%%%%%%%%%%%%%%%%%%%%%%%%%%%%%%%%%%%%%%%%%%%%%%%%%%%%%%%%%%%%%%
\subsection{Included Files}
\label{sec:include}

%%%%%%%%%%%%%%%%%%%%%%%%%%%%%%%%%%%%%%%%
\DescribeMacro{\childdocmain}
To use the package, add the commands
\begin{center}
\begin{tabular}{l}
|\input{childdoc.def}|\\
|\childdocmain{}|\\
\end{tabular}
\end{center}
at the very top of the main \LaTeX{} file,
in particular \emph{before} the |\documentclass| statement!
The argument of |\childdocmain| should be left empty
(but it must be present).

%%%%%%%%%%%%%%%%%%%%%%%%%%%%%%%%%%%%%%%%
\DescribeMacro{\childdocof}
Furthermore, add the commands
\begin{center}
\begin{tabular}{l}
|\input{childdoc.def}|\\
|\childdocof{|\textit{main}|}|\\
\end{tabular}
\end{center}
at the top of every child file \textit{child}
which is included by |\include{|\textit{child}|}|
from within the main file
(or at least for those files to be compiled individually).
The argument \textit{main} must be the filename of the main file.

There are a couple of
considerations in setting up the main and child documents:

%%%%%%%%%%%%%%%%%%%%%%%%%%%%%%%%%%%%%%%%
\paragraph{Restrictions.}

Please note the following restrictions:
\begin{itemize}
\item
|\childdocmain| must be called with one argument \textit{main}
to ensure compatibility with earlier version of the package.
It must either be empty (|\childdocmain{}|)
or precisely match the filename of the main file in which it is specified.
See \secref{sec:detection} for further information.
\item
The filename \textit{main} must be specified without the |.tex| extension.
\item
The filename \textit{main} is case sensitive
(even in case-insensitive file systems)
due to internal string comparison.
\item
The argument \textit{main} should be fully expanded, it cannot be a macro.
\item
Subdirectories and special characters should be avoided in filenames.
\item
The command |\childdocmain{|\textit{main}|}| must be followed by a whitespace.
It should not be followed immediately by another command
or by a comment mark `|%|'.
This is because the \TeX{} parser reads the token immediately following
the argument of |\childdocmain| and puts it
at the beginning of every child section;
however, a white\-space is ignored.
\end{itemize}

%%%%%%%%%%%%%%%%%%%%%%%%%%%%%%%%%%%%%%%%
\paragraph{Content of Main File.}

It is advisable to place all content in the child files included by |\include|.
Any output contained in the main file will appear in all child documents
unless suppressed manually;
it cannot be suppressed automatically by the |\includeonly| directive
and thus should normally be avoided.
A method to include some content in the main file
by means of conditional processing is described in \secref{sec:conditional}.

%%%%%%%%%%%%%%%%%%%%%%%%%%%%%%%%%%%%%%%%
\paragraph{Page Numbering.}

When only a part of the document is compiled,
the appropriate numbering of pages
(as well as other status parameters)
is determined from the |.aux| files.
The latter contain information from previous passes.
However this information needs to propagate through
all intermediate child documents.
Therefore the page numbering in child documents may well
be inconsistent until the complete document is compiled at least once.

A useful (if unconventional) way to always ensure a consistent
page numbering is to restart the numbering in each child document
and denote the pages by `\textit{child}|.|\textit{page}'
where \textit{child} represents the chapter/section number of the child file.
This can be achieved by the command
|\numberwithin{page}{|\textit{child}|}|
of the \textsf{amsmath} package
where \textit{child} can be |chapter| or |section|
depending on the chosen structuring.
Alternatively, one can modify the macro |\thepage| appropriately
and reset the counter |page| at the start of each child file.

%%%%%%%%%%%%%%%%%%%%%%%%%%%%%%%%%%%%%%%%%%%%%%%%%%%%%%%%%%%%%%%%%%%%%%%%%%%%%%%%
\subsection{Conditional Processing}
\label{sec:conditional}

The package provides a mechanism to compile different versions
of a document. To customise the versions further some conditional processing
can come in handy to distinguish which version is being compiled.
The package provides two macros to describe the compilation context:

%%%%%%%%%%%%%%%%%%%%%%%%%%%%%%%%%%%%%%%%
\DescribeMacro{\ifchilddoc}
The conditional |\ifchilddoc| distinguishes between the compilation of
child documents and the main document:
%
\begin{center}
|\ifchilddoc |\textit{child-code}| |[|\||else |\textit{main-code}]| \||fi|
\end{center}

%%%%%%%%%%%%%%%%%%%%%%%%%%%%%%%%%%%%%%%%
\DescribeMacro{\childdocname}
\DescribeMacro{\childdocjob}
The macro |\childdocname| contains the filename (without extension)
of the main or child file being processed.
Note that |\childdocjob| will always contain the name of the main file.

%%%%%%%%%%%%%%%%%%%%%%%%%%%%%%%%%%%%%%%%
\paragraph{Title Page.}

Conditional processing can be used to include a title or banner page
in the main document when proper precautions are taken.
Importantly, the code in the main file should ensure that the page counter
(as well as other status parameters which are stored in the |.aux| files)
takes the same value after the conditional processing.
Otherwise the page numbers may take divergent values
depending on which part is compiled.

For example, a title page could be declared by:
%
\begin{center}
\begin{tabular}{l}
|\ifchilddoc\||else|\\
|\addtocounter{page}{-1}|\\
\textit{code for title page}\\
|\newpage|\\
|\||fi|
\end{tabular}
\end{center}
%
A banner page for the child documents can be generated by:
%
\begin{center}
\begin{tabular}{l}
|\ifchilddoc|\\
|\addtocounter{page}{-1}|\\
\textit{code for banner page}\\
|\newpage|\\
|\||fi|
\end{tabular}
\end{center}
%
Here one could write a message such as:
\begin{center}
|This is the part \childdocname{} of \childdocjob{}.|
\end{center}

%%%%%%%%%%%%%%%%%%%%%%%%%%%%%%%%%%%%%%%%%%%%%%%%%%%%%%%%%%%%%%%%%%%%%%%%%%%%%%%%
\subsection{Flags}
\label{sec:flags}

The package makes it easy to generate different versions
of the main or child documents.
To this end compilation flags can be defined
and assigned different default values.
They will be particularly useful in conjunction
with the forwarding mechanism described in \secref{sec:forward}.

For example, it may be useful to have a flag |\version|
which can be set to |draft| or |final|.
The document source will contain some conditional code
depending on the value of |\version|.
Suppose further, the flag should default to |final| for the main file
and to |draft| for child files
which is a natural assignment for editing the document.
This is achieved by placing the following code
in the preamble of the main document
(below the |\childdocmain| directive):
%
\begin{center}
\begin{tabular}{l}
|\ifchilddoc|\\
|\providecommand{\version}{draft}|\\
|\||else|\\
|\providecommand{\version}{final}|\\
|\||fi|
\end{tabular}
\end{center}
%
The definition by |\providecommand| makes sure
that previous definitions are not overwritten.
Further statements |\providecommand{\version}{...}|
can thus be added before the above code to override it.

For the main file, one might add a line
(between |\childdocmain| and the above block)
%
\begin{center}
|%\ifchilddoc\||else\providecommand{\version}{draft}\||fi|
\end{center}
%
which can be uncommented to produce a draft version.
Likewise one can add a line to the very top of a child file
(above the |\childdocof{|\textit{main}|}| directive)
%
\begin{center}
|%\providecommand{\version}{final}|
\end{center}
%
which can be uncommented to produce the final version of this child document.

%%%%%%%%%%%%%%%%%%%%%%%%%%%%%%%%%%%%%%%%%%%%%%%%%%%%%%%%%%%%%%%%%%%%%%%%%%%%%%%%
\subsection{Forwarding}
\label{sec:forward}

Different versions of the main or child documents
using compilation flags as described in \secref{sec:flags}
can be (permanently) stored in different files
for convenient compilation, viewing and distribution.
To this end, the package defines a command
to pass on compilation to a different file:

%%%%%%%%%%%%%%%%%%%%%%%%%%%%%%%%%%%%%%%%
\DescribeMacro{\childdocforward}
The command |\childdocforward| redirects processing to
another source file:
%
\begin{center}
\begin{tabular}{l}
|\input{childdoc.def}|\\
|\childdocforward[|\textit{main}|]{|\textit{dest}|}|\\
\end{tabular}
\end{center}
%
The argument \textit{dest} is the destination file
(without extension).
It should be the main file or one of the child files.
Note that further \textsf{childdoc} directives
such as |\childdocof| and |\childdocforward|
in the indicated file will be processed in this form.
The optional argument \textit{main}
passes on directly to the main file \textit{main}
while pretending to compile the child \textit{dest}.
This form behaves as if \textit{dest}
issues |\childdocof{|\textit{main}|}| right away,
and no further \textsf{childdoc} directives will be processed.

%%%%%%%%%%%%%%%%%%%%%%%%%%%%%%%%%%%%%%%%
\DescribeMacro{\...prefix}
In the alternative form |\childdocforwardprefix|,
%
\begin{center}
\begin{tabular}{l}
|\input{childdoc.def}|\\
|\childdocforwardprefix[|\textit{main}|]{|\textit{prefix}|}{|\textit{dest}|}|
\end{tabular}
\end{center}
%
the destination file is determined by a pattern
depending on the current file:
To make this work, the current file must be called
`{\textit{prefix}\hspace{0.2em}\textit{suffix}}'
with \textit{prefix} matching precisely the argument.
Processing is then passed on to the file
`{\textit{dest}\hspace{0.2em}\textit{suffix}}'.
Surely, the same effect is achieved by
directly specifying the
argument `{\textit{dest}\hspace{0.2em}\textit{suffix}}'
in the first form.
However, that requires to set up a different file
for each child. With the alternative form of the command
all these files can have exactly the same content
which simplifies setting them up and maintaining them.

For example, the following file |draft.tex|
with a compilation flag |\version| as described in \secref{sec:flags}
compiles the main document as a draft:
%
\begin{center}
\begin{tabular}{l}
|\def\version{draft}|\\
|\input{childdoc.def}|\\
|\childdocforward{|\textit{main}|}|
\end{tabular}
\end{center}
%
Likewise, the following files |final|\textit{nn}|.tex|
compile the final version of the child document
|child|\textit{nn}|.tex|:
%
\begin{center}
\begin{tabular}{l}
|\def\version{final}|\\
|\input{childdoc.def}|\\
|\childdocforwardprefix{final}{child}|
\end{tabular}
\end{center}
%

Note that when several versions of a main file and/or of each child file
are to be generated, it may be convenient to set up a |Makefile| or
shell script to automatise the process.

%%%%%%%%%%%%%%%%%%%%%%%%%%%%%%%%%%%%%%%%%%%%%%%%%%%%%%%%%%%%%%%%%%%%%%%%%%%%%%%%
\subsection{Command Line Processing}
\label{sec:commandline}

The effect of redirection files can also be achieved by invoking
the \LaTeX{} compiler with a more elaborate command line.
Most conveniently this should be done as part
of a shell script or a |Makefile|.

When using \textsf{childdoc} in the main file, the following
command lines effectively perform a redirection
(note that depending on the shell being used,
backslashes may have to be doubled: `|\|' $\to$ `|\\|'):
%
\begin{center}
|... -jobname "|\textit{target}|" |\\|"|[\textit{flags}]%
|\input{childdoc.def}\childdocforward[|\textit{main}|]{|\textit{dest}|}"|
\end{center}
%
Here \textit{target} is the name of the output file,
\textit{main} is the name of the main file
and \textit{dest} is the name of the main or child file to be processed
(all filenames without extensions).
The optional argument \textit{main} can be omitted
if \textit{main} matches \textit{dest}.
Optionally, compilation \textit{flags} can be defined via |\def| commands.
This command line makes the \TeX{} engine believe
it is compiling the file \textit{target}
whose content is specified as the latter parameter.
The provided code then forwards the processing to
\textit{main} or \textit{dest} as described in \secref{sec:forward}.

%%%%%%%%%%%%%%%%%%%%%%%%%%%%%%%%%%%%%%%%%%%%%%%%%%%%%%%%%%%%%%%%%%%%%%%%%%%%%%%%
\subsection{Include by Input}
\label{sec:input}

Including child documents by |\include| has some restrictions by design.
Most notably, the content of a child document always occupies
its own set of pages; pages cannot be shared between child documents.
Usually, this behaviour makes perfect sense
because each child document contain an essential part of the document.
However, in some situations it may be desirable to compose
a document from a collection of parts
without having mandatory page breaks between then.
For this case, the package
provides a mechanism to include parts
by |\input| which can also be processed individually.
However, by construction this mechanism
requires manual handling of the content to be output.

%%%%%%%%%%%%%%%%%%%%%%%%%%%%%%%%%%%%%%%%
\DescribeMacro{\ifchilddocmanual}
The main file should be prepared as usual, see \secref{sec:include}.
However, the document body must make a distinction
between processing of an individual part and of the main document, e.g.:
%
\begin{center}
\begin{tabular}{l}
|\ifchilddocmanual|\\
|\input{\childdocname}|\\
|\||else|\\
\textit{document body with }|\input{|\textit{part}|}|\\
|\||fi|
\end{tabular}
\end{center}
%
The conditional |\ifchilddocmanual| is true whenever
a part to be included by |\input| is being compiled,
and the name of the part is stored in |\childdocname|.

%%%%%%%%%%%%%%%%%%%%%%%%%%%%%%%%%%%%%%%%
\DescribeMacro{\childdocby}
Each part to be included by |\input| should start with:
%
\begin{center}
\begin{tabular}{l}
|\input{childdoc.def}|\\
|\childdocby{|\textit{main}|}|\\
\end{tabular}
\end{center}
%
The directive |\childdocby| is similar to |\childdocof|
described in \secref{sec:include},
but the subsequent selection of content must be done manually.
To that end, both |\ifchilddoc| and |\ifchilddocmanual|
will be true upon processing of a part,
and the name of the part is stored in |\childdocname|.
Note that |\jobname| will be set to the filename of the current part
so that each part receives an individual |.aux| file
that does not interfere with the |.aux| file(s) of the main document.
This behaviour can be altered by the alternative form
|\childdocby[*]{|\textit{main}|}| (with a non-empty optional argument)
which uses the |.aux| file of the main document
by setting |\jobname| to \textit{main}.

%%%%%%%%%%%%%%%%%%%%%%%%%%%%%%%%%%%%%%%%%%%%%%%%%%%%%%%%%%%%%%%%%%%%%%%%%%%%%%%%
\subsection{Driver Development}
\label{sec:driver}

The \textsf{childdoc} mechanism can also be use for the development
of definition files such as \LaTeX{} styles or classes.
This case differs from the above setup with multiple parts
included by |\include| in that no |\includeonly| should be invoked.
This can be achieved by starting the include file
(before |\ProvidesPackage|) with:
%
\begin{center}
\begin{tabular}{l}
|\input{childdoc.def}|\\
|\childdocforward{|\textit{main}|}|\\
\end{tabular}
\end{center}
%
or alternatively with:
%
\begin{center}
\begin{tabular}{l}
|\input{childdoc.def}|\\
|\childdocby{|\textit{main}|}|\\
\end{tabular}
\end{center}
%
Both forms have slightly different effects as described above.
The main file is prepared as usual, see \secref{sec:include}.

%%%%%%%%%%%%%%%%%%%%%%%%%%%%%%%%%%%%%%%%%%%%%%%%%%%%%%%%%%%%%%%%%%%%%%%%%%%%%%%%
\subsection{Legacy Detection}
\label{sec:detection}

The directive |\childdocmain| in the main file can detect
whether the complete document or merely a child is to be compiled
even without using the directive |\childdocof|.
This method is deprecated because it is less robust
and there is no compelling reason to use it;
it is merely provided for backward compatibility
and it may be removed in future versions.

If the detection mechanism is to be used,
it is mandatory to correctly specify
the filename of the main file as the argument of |\childdocmain|:
%
\begin{center}
\begin{tabular}{l}
|\input{childdoc.def}|\\
|\childdocmain{|\textit{main}|}|\\
\end{tabular}
\end{center}
%
If |\jobname| does not match the argument \textit{main} of |\childdocmain|,
it is assumed that |\jobname| points to the child file to be compiled.
When using |\childdocmain| with the main file specified as argument,
it suffices to start a child file
with just |\input{|\textit{main}|}|
without loading of the package and using |\childdocof|.
If instead all processing is done
with the appropriate \textsf{childdoc} directives,
the argument of \textit{main} of |\childdocmain| can be empty.

An alternative version of the command line processing described
in \secref{sec:commandline} using the detection mechanism reads:
%
\begin{center}
|... -jobname "|\textit{target}|" "|[\textit{flags}]%
[|\def\jobname{|\textit{dest}|}|]|\input{|\textit{main}|}"|
\end{center}

%%%%%%%%%%%%%%%%%%%%%%%%%%%%%%%%%%%%%%%%%%%%%%%%%%%%%%%%%%%%%%%%%%%%%%%%%%%%%%%%
\subsection{Manual Code}
\label{sec:manual}

In case one cannot be certain whether the definitions file |childdoc.def|
is installed on the target \TeX{} distribution
and one prefers not to ship it,
it is conceivable to paste a few relevant commands into the sources.

To that end, drop all statements |\input{childdoc.def}|
and perform the replacements as outlined below.
Instead of |\childdocmain{|\textit{main}|}| add the following code
to the top of the main file:
%
\begin{center}
\begin{tabular}{l}
|\||ifdefined\childdocname\endinput\||fi\newif\ifchilddoc|\\
|\edef\childdocname{\scantokens\expandafter{\jobname\noexpand}}|\\
|\def\childdocmain{|\textit{main}|}\||ifx\childdocmain\childdocname\||else|\\
|\childdoctrue\includeonly{\childdocname}\let\jobname\childdocmain\||fi|\\
\end{tabular}
\end{center}
%
Instead of |\childdocof{|\textit{main}|}| just include the main file
at the top of each child file:
%
\begin{center}
|\input{|\textit{main}|}|
\end{center}
%
A simple redirection |\childdocforward{|\textit{dest}|}| is achieved by:
%
\begin{center}
|\def\jobname{|\textit{dest}|}\input{\jobname}|
\end{center}
%
The redirection with prefix
|\childdocforwardprefix[|\textit{prefix}|]{|\textit{dest}|}|
is accomplished by:
%
\begin{center}
\begin{tabular}{l}
|{\edef\jobname{\scantokens\expandafter{\jobname\noexpand}}|\\
|\def\redirectjob |\textit{prefix}|#1~~~{\gdef\jobname{|\textit{dest}|#1}}|\\
|\expandafter\redirectjob\jobname~~~}\input{\jobname}|
\end{tabular}
\end{center}

In an alternative approach,
child documents can be compiled by a specific command line
without additional code or specific definitions:
%
\begin{center}
|... -jobname "|\textit{target}|" "|[\textit{flags}]%
|\includeonly{|\textit{dest}|}\input{|\textit{main}|}"|
\end{center}
%

%%%%%%%%%%%%%%%%%%%%%%%%%%%%%%%%%%%%%%%%%%%%%%%%%%%%%%%%%%%%%%%%%%%%%%%%%%%%%%%%
%%%%%%%%%%%%%%%%%%%%%%%%%%%%%%%%%%%%%%%%%%%%%%%%%%%%%%%%%%%%%%%%%%%%%%%%%%%%%%%%
\section{Information}

%%%%%%%%%%%%%%%%%%%%%%%%%%%%%%%%%%%%%%%%%%%%%%%%%%%%%%%%%%%%%%%%%%%%%%%%%%%%%%%%
\subsection{Copyright}

Copyright \copyright{} 2017--2018 Niklas Beisert

This work may be distributed and/or modified under the
conditions of the \LaTeX{} Project Public License, either version 1.3
of this license or (at your option) any later version.
The latest version of this license is in
  \url{http://www.latex-project.org/lppl.txt}
and version 1.3 or later is part of all distributions of \LaTeX{}
version 2005/12/01 or later.

This work has the LPPL maintenance status `maintained'.

The Current Maintainer of this work is Niklas Beisert.

This work consists of the files |README.txt|, |childdoc.ins| and |childdoc.dtx|
as well as the derived files |childdoc.def|, |cdocsamp.tex|
with |cdocsch1.tex|, |cdocsch2.tex|, |cdocspt3.tex|, |cdocspt4.tex|,
|cdocsdrf.tex|, |cdocsfn1.tex|, |cdocsfn2.tex|
as well as |childdoc.pdf|.

%%%%%%%%%%%%%%%%%%%%%%%%%%%%%%%%%%%%%%%%%%%%%%%%%%%%%%%%%%%%%%%%%%%%%%%%%%%%%%%%
\subsection{Files and Installation}

The package consists of the files:
%
\begin{center}
\begin{tabular}{ll}
    |README.txt|   & readme file \\
    |childdoc.ins| & installation file \\
    |childdoc.dtx| & source file \\
    |childdoc.def| & definition file \\
    |cdocsamp.tex| & sample main file \\
    |cdocsch1.tex| & sample include file \\
    |cdocsch2.tex| & sample include file \\
    |cdocspt3.tex| & sample part file \\
    |cdocspt4.tex| & sample part file \\
    |cdocsdrf.tex| & sample redirection file \\
    |cdocsfn1.tex| & sample redirection file \\
    |cdocsfn2.tex| & sample redirection file \\
    |childdoc.pdf| & manual
\end{tabular}
\end{center}
%
The distribution consists of the files
|README.txt|, |childdoc.ins| and |childdoc.dtx|.
%
\begin{itemize}
\item
Run (pdf)\LaTeX{} on |childdoc.dtx|
to compile the manual |childdoc.pdf| (this file).
\item
Run \LaTeX{} on |childdoc.ins| to create the definitions file |childdoc.def|
and the sample |cdocsamp.tex| with include files
|cdocsch1.tex|, |cdocsch2.tex|, |cdocspt3.tex|, |cdocspt4.tex|,
|cdocsdrf.tex|, |cdocsfn1.tex|, |cdocsfn2.tex|.
Then copy the file |childdoc.def| to an appropriate directory of your \LaTeX{}
distribution, e.g.\ \textit{texmf-root}|/tex/latex/childdoc|.
\end{itemize}

%%%%%%%%%%%%%%%%%%%%%%%%%%%%%%%%%%%%%%%%%%%%%%%%%%%%%%%%%%%%%%%%%%%%%%%%%%%%%%%%
\subsection{Related CTAN Packages}

There are several other packages which offer a similar functionality:
%
\begin{itemize}
\item
The packages
\href{http://ctan.org/pkg/docmute}{\textsf{docmute}},
\href{http://ctan.org/pkg/includex}{\textsf{includex}} and
\href{http://ctan.org/pkg/standalone}{\textsf{standalone}}
provide commands to include only the document body of
a child file thus allowing both files to be compiled individually.
\item
The packages \href{http://ctan.org/pkg/subdocs}{\textsf{subdocs}}
and \href{http://ctan.org/pkg/subfiles}{\textsf{subfiles}}
provide structures in which the main and child documents can be
encapsulated and allowing them to be compiled individually.
The inclusion mechanism is different from the conventional |\include|.
\item
The package \href{http://ctan.org/pkg/combine}{\textsf{combine}}
is an elaborate solution to combine several documents into one.
\end{itemize}
%
See also the CTAN topic \href{http://ctan.org/topic/subdocs}{\textsf{subdocs}}
for further related packages.
The present package differs from the above solutions in that
a document structure constructed with the conventional |\include| mechanism
just needs two extra commands at the top of every file
such that all constituent files can be compiled individually.

%%%%%%%%%%%%%%%%%%%%%%%%%%%%%%%%%%%%%%%%%%%%%%%%%%%%%%%%%%%%%%%%%%%%%%%%%%%%%%%%
%\subsection{Feature Suggestions}
%
%The following is a list of features which may be useful for future
%versions of this package:
%%
%\begin{itemize}
%\item
%\ldots
%\end{itemize}

%%%%%%%%%%%%%%%%%%%%%%%%%%%%%%%%%%%%%%%%%%%%%%%%%%%%%%%%%%%%%%%%%%%%%%%%%%%%%%%%
\subsection{Revision History}

%%%%%%%%%%%%%%%%%%%%%%%%%%%%%%%%%%%%%%%%
\paragraph{v2.0:} 2018/12/30

\begin{itemize}
\item
immediate forward processing
\item
added |\childdocby| mechanism
\item
manual restructured
\end{itemize}

%%%%%%%%%%%%%%%%%%%%%%%%%%%%%%%%%%%%%%%%
\paragraph{v1.6:} 2018/01/17

\begin{itemize}
\item
application for development of include files
\item
corrections to manual
\end{itemize}

%%%%%%%%%%%%%%%%%%%%%%%%%%%%%%%%%%%%%%%%
\paragraph{v1.5:} 2017/05/21

\begin{itemize}
\item
more complete structuring introduced
\item
|\childdocof| introduced
\item
|\childdoc| renamed to |\childdocmain|
\item
|\childredirect| renamed to |\childdocforward| and |\childdocforwardprefix|
and functionality expanded
\end{itemize}

%%%%%%%%%%%%%%%%%%%%%%%%%%%%%%%%%%%%%%%%
\paragraph{v1.0:} 2017/04/27

\begin{itemize}
\item
manual and install package
\item
first version published on CTAN
\end{itemize}

%%%%%%%%%%%%%%%%%%%%%%%%%%%%%%%%%%%%%%%%
\paragraph{v0.6:} 2017/04/26

\begin{itemize}
\item
redirection mechanism added
\end{itemize}

%%%%%%%%%%%%%%%%%%%%%%%%%%%%%%%%%%%%%%%%
\paragraph{v0.5:} 2017/04/26

\begin{itemize}
\item
functionality in definition file
\end{itemize}


%%%%%%%%%%%%%%%%%%%%%%%%%%%%%%%%%%%%%%%%%%%%%%%%%%%%%%%%%%%%%%%%%%%%%%%%%%%%%%%%
%%%%%%%%%%%%%%%%%%%%%%%%%%%%%%%%%%%%%%%%%%%%%%%%%%%%%%%%%%%%%%%%%%%%%%%%%%%%%%%%
%%%%%%%%%%%%%%%%%%%%%%%%%%%%%%%%%%%%%%%%%%%%%%%%%%%%%%%%%%%%%%%%%%%%%%%%%%%%%%%%
\appendix

\settowidth\MacroIndent{\rmfamily\scriptsize 000\ }

 \DocInput{childdoc.dtx}

\end{document}
%</driver>
% \fi
%
% %%%%%%%%%%%%%%%%%%%%%%%%%%%%%%%%%%%%%%%%%%%%%%%%%%%%%%%%%%%%%%%%%%%%%%%%%%%%%%
% %%%%%%%%%%%%%%%%%%%%%%%%%%%%%%%%%%%%%%%%%%%%%%%%%%%%%%%%%%%%%%%%%%%%%%%%%%%%%%
% \section{Sample}
%\iffalse
%<*samplemain>
%\fi
%
% The following presents a sample document
% with two chapters, two parts, a title page,
% a compile flag as well as three forwarding files to set the flag.
% It consists of eight |.tex| files:
% \begin{center}
% \begin{tabular}{ll}
% |cdocsamp.tex|&main file\\
% |cdocsch1.tex|&include file for chapter 1\\
% |cdocsch2.tex|&include file for chapter 2\\
% |cdocspt3.tex|&include file for part 3\\
% |cdocspt4.tex|&include file for part 4\\
% |cdocsdrf.tex|&forwarding file for main file in draft mode\\
% |cdocsfi1.tex|&forwarding file for final version of chapter 1\\
% |cdocsfi2.tex|&forwarding file for final version of chapter 2\\
% \end{tabular}
% \end{center}
% Each of the eight files can be compiled directly by the \LaTeX{} compiler.
%
% %%%%%%%%%%%%%%%%%%%%%%%%%%%%%%%%%%%%%%
% \paragraph{Main File.}
%
% The main file is called |cdocsamp.tex|.
%
% Load the \textsf{childdoc} definitions and
% declare the filename for the main document:
%    \begin{macrocode}
\input{childdoc.def}
\childdocmain{}
%    \end{macrocode}

% Optional override for |\version| flag:
%    \begin{macrocode}
%%\ifchilddoc\else\providecommand{\version}{draft}\fi
%    \end{macrocode}

% Define the default values for the |\version| flag
% (|final| for the main file and |draft| for childs):
%    \begin{macrocode}
\ifchilddoc
\providecommand{\version}{draft}
\else
\providecommand{\version}{final}
\fi
%    \end{macrocode}

% Load the standard document class:
%    \begin{macrocode}
\documentclass[12pt]{article}
%    \end{macrocode}

% Start the document body:
%    \begin{macrocode}
\begin{document}
%    \end{macrocode}

% Declare a title page.
% Print title, part of document being processed and version flag:
%    \begin{macrocode}
\addtocounter{page}{-1}
\begin{center}
{\LARGE\bfseries{}childdoc example\par}
\vspace{1cm}
\ifchilddoc
\ifchilddocmanual part\else chapter\fi:
`\childdocname' of `\childdocjob'\par
\else
main document: `\childdocjob'\par
\fi
version: \version\par
\end{center}
\newpage
%    \end{macrocode}

% Manually include selected file,
% otherwise process as usual:
%    \begin{macrocode}
\ifchilddocmanual
\section*{part `\childdocname'}
\input{\childdocname}
\else
%    \end{macrocode}

% Include the two chapters:
%    \begin{macrocode}
\include{cdocsch1}
\include{cdocsch2}
%    \end{macrocode}

% Include the two parts unless only chapters should be displayed:
%    \begin{macrocode}
\ifchilddoc\else
\section{part three}
\input{cdocspt3}
\section{part four}
\input{cdocspt4}
\fi
%    \end{macrocode}

% Process as usual until here:
%    \begin{macrocode}
\fi
%    \end{macrocode}

% End of document body:
%    \begin{macrocode}
\end{document}
%    \end{macrocode}
%\iffalse
%</samplemain>
%\fi
%
% %%%%%%%%%%%%%%%%%%%%%%%%%%%%%%%%%%%%%%
% \paragraph{Chapter Include Files.}
%
% The include files are called |cdocsch1.tex| and |cdocsch2.tex|.
%
%\iffalse
%<*samplechap1|samplechap2>
%\fi

% Optional override for |\version| flag:
%    \begin{macrocode}
%%\providecommand{\version}{final}
%    \end{macrocode}

% Include the main document:
%    \begin{macrocode}
\input{childdoc.def}
\childdocof{cdocsamp}
%    \end{macrocode}

%\iffalse
%</samplechap1|samplechap2>
%\fi
%
%\iffalse
%<*samplechap1>
%\fi
% Some text for chapter 1:
%    \begin{macrocode}
\section{one}
some text in chapter one
%    \end{macrocode}

%\iffalse
%</samplechap1>
%\fi
% Some text for chapter 2:
%\iffalse
%<*samplechap2>
%\fi
%    \begin{macrocode}
\section{two}
more text in chapter two
%    \end{macrocode}

%\iffalse
%</samplechap2>
%\fi
%
% %%%%%%%%%%%%%%%%%%%%%%%%%%%%%%%%%%%%%%
% \paragraph{Part Include Files.}
%
% The include files are called |cdocspt3.tex| and |cdocspt4.tex|.
%
%\iffalse
%<*samplepart3|samplepart4>
%\fi

% Optional override for |\version| flag:
%    \begin{macrocode}
%%\providecommand{\version}{final}
%    \end{macrocode}

% Include the main document:
%    \begin{macrocode}
\input{childdoc.def}
\childdocby{cdocsamp}
%    \end{macrocode}

%\iffalse
%</samplepart3|samplepart4>
%\fi
%
%\iffalse
%<*samplepart3>
%\fi
% Some text for part 3:
%    \begin{macrocode}
some text in part three
%    \end{macrocode}

%\iffalse
%</samplepart3>
%\fi
% Some text for part 4:
%\iffalse
%<*samplepart4>
%\fi
%    \begin{macrocode}
more text in part four
%    \end{macrocode}

%\iffalse
%</samplepart4>
%\fi
%
% %%%%%%%%%%%%%%%%%%%%%%%%%%%%%%%%%%%%%%
% \paragraph{Forwarding for a Complete Draft.}
%
% The following forwarding file |cdocsdrf.tex|
% compiles the main document in draft mode:
%\iffalse
%<*sampledraft>
%\fi
%    \begin{macrocode}
\def\version{draft}
\input{childdoc.def}
\childdocforward{cdocsamp}
%    \end{macrocode}

%\iffalse
%</sampledraft>
%\fi
%
% %%%%%%%%%%%%%%%%%%%%%%%%%%%%%%%%%%%%%%
% \paragraph{Forwarding for Final Version of the Chapters.}
%
% The following forwarding files |cdocsfn1.tex| and |cdocsfn2.tex|
% (with identical content)
% compile the final versions of the child documents
% |cdocsch1.tex| and |cdocsch2.tex|, respectively:
%\iffalse
%<*samplefinal>
%\fi
%    \begin{macrocode}
\def\version{final}
\input{childdoc.def}
\childdocforwardprefix[cdocsamp]{cdocsfn}{cdocsch}
%    \end{macrocode}

%\iffalse
%</samplefinal>
%\fi
%
% %%%%%%%%%%%%%%%%%%%%%%%%%%%%%%%%%%%%%%
% \paragraph{Command Line Processing.}
%
% The following three command lines generate the output files
% |cdocscld|, |cdocscl1| and |cdocscl2|
% which should be identical to
% |cdocsdrf|, |cdocsch1| and |cdocsfn2|, respectively:
% \begin{center}
% \begin{tabular}{l}
% |latex -jobname cdocscld \|\\
% |  "\def\version{draft}\input{childdoc.def}\childdocforward{cdocsamp}"|\\
% |latex -jobname cdocscl1 \|\\
% |  "\input{childdoc.def}\childdocforward[cdocsamp]{cdocsch1}"|\\
% |latex -jobname cdocscl2 \|\\
% |  "\def\version{final}\input{childdoc.def}\childdocforward{cdocsch2}"|
% \end{tabular}
% \end{center}
% Note that the trailing backslash on each first line
% merely continues the input to the second line
% (for convenient cut ant paste).
% Furthermore, the command |latex| can be replaced by any
% of its alternative versions such as |pdflatex|.
%
% %%%%%%%%%%%%%%%%%%%%%%%%%%%%%%%%%%%%%%%%%%%%%%%%%%%%%%%%%%%%%%%%%%%%%%%%%%%%%%
% %%%%%%%%%%%%%%%%%%%%%%%%%%%%%%%%%%%%%%%%%%%%%%%%%%%%%%%%%%%%%%%%%%%%%%%%%%%%%%
% \section{Implementation}
%\iffalse
%<*package>
%\fi
%
% This section describes the definitions file |childdoc.def|.

% The definitions cannot be loaded using |\usepackage| or |\RequirePackage|
% which has a mechanism to prevent loading a style file more than once.
% When loading the definitions by means of |\input|
% multiple instances have to be prevented manually:
%\iffalse
%This code needs to be before the `\ProvidesFile' directive
%which is defined at the beginning of this file.
%Therefore it is also placed there and commented out here.
%</package>
%<*discard>
%\fi
%    \begin{macrocode}
\ifdefined\childdocmain\endinput\fi
%    \end{macrocode}
%\iffalse
%</discard>
%<*package>
%\fi
%
% \macro{\ifchilddoc}
% \macro{\ifchilddocmanual}
% The conditional |\ifchilddoc| tells whether a
% child (true) or main (false) document is being compiled.
% The conditional |\ifchilddocmanual| tells whether
% the |\includeonly| mechanism is used (false) or
% the selection of child files must be performed manually (true).
% The definitions initialise to false:
%    \begin{macrocode}
\newif\ifchilddoc
\newif\ifchilddocmanual
%    \end{macrocode}

% \macro{\childdocname}
% \macro{\childdocjob}
% The macro |\childdocname| stores the name of the main document
% to be compiled. The macro |\childdocjob| stores the name of
% the document on which the \LaTeX{} compiler was originally invoked.
% The content of |\jobname| cannot be compared
% to filenames specified in the source due to different catcodes.
% The following code rescans |\jobname|, stores the result
% in |\childdocname| and saves a copy in |\childdocjob|:
%    \begin{macrocode}
\edef\childdocname{\scantokens\expandafter{\jobname\noexpand}}
\let\childdocjob\childdocname
%    \end{macrocode}

% \macro{\childdocdisable}
% The macro |\childdocdisable| prevents the main file
% from being processed more than once.
% At this stage, the main document command |\childdocmain|
% is assumed to be called once again where it should do nothing.
% Any subsequent call to it should prevent
% a secondary processing of the main document
% It overwrites the forwarding commands
% |\childdocof| and |\childdocforward|
% with empty macros to prevent further inclusions of the main document:
%    \begin{macrocode}
\newcommand{\childdocdisable}
{
  \renewcommand{\childdocmain}[1]{\renewcommand{\childdocmain}[1]{\endinput}}
  \renewcommand{\childdocof}[1]{}
  \renewcommand{\childdocby}[2][]{}
  \renewcommand{\childdocforward}[2][]{}
  \renewcommand{\childdocdisable}{}
}
%    \end{macrocode}

% \macro{\childdocmain}
% The macro |\childdocmain| is to be called at the top of the main file
% with nothing or the main filename (without extension) as argument.
% First, it breaks loops.
% If the argument is not empty and does not match |\childdocname|
% (which is set by the first inclusion of |childdoc.def|),
% |\ifchilddoc| is set to true, |\includeonly| is applied to the child file
% and |\jobname| is set to the main file
% (for proper handling of |.aux| files):
%    \begin{macrocode}
\newcommand{\childdocmain}[1]
{
  \childdocdisable\childdocmain{}
  \if?#1?\else
    \begingroup
      \def\childdoctmp{#1}
      \ifx\childdoctmp\childdocname
        \def\childdoctmp{}
      \else
        \def\childdoctmp
        {
          \childdoctrue
          \includeonly{\childdocname}
          \def\childdocjob{#1}
          \def\jobname{#1}
        }
      \fi
      \expandafter
    \endgroup
    \childdoctmp
  \fi
}
%    \end{macrocode}

% \macro{\childdocof}
% The command |\childdocof| redirects
% compilation to the main file |#1|.
%    \begin{macrocode}
\newcommand{\childdocof}[1]
{
  \childdocdisable
  \childdoctrue
  \includeonly{\childdocname}
  \def\jobname{#1}
  \def\childdocjob{#1}
  \input{#1}
}
%    \end{macrocode}

% \macro{\childdocby}
% The command |\childdocby| ....
%    \begin{macrocode}
\newcommand{\childdocby}[2][]
{
  \childdocdisable
  \childdoctrue
  \childdocmanualtrue
  \if?#1?\else
    \def\jobname{#2}
  \fi
  \def\childdocjob{#2}
  \input{#2}
  \endinput
}
%    \end{macrocode}

% \macro{\childdocforward}
% The command |\childdocforward| redirects
% compilation to the main file or
% (if the optional argument is given) a child file.
% Parameters are set as if the main file
% or a child file starting with |\childdocof| was compiled.
% Then compilation is handed over to the main file:
%    \begin{macrocode}
\newcommand{\childdocforward}[2][]
{
  \begingroup
    \if?#1?
      \def\childdoctmp
      {
        \def\childdocname{#2}
        \def\childdocjob{#2}
        \def\jobname{#2}
        \input{#2}
        \endinput
      }
    \else
      \def\childdoctmp
      {
        \childdocdisable
        \def\childdocname{#2}
        \childdoctrue
        \includeonly{#2}
        \def\childdocjob{#1}
        \def\jobname{#1}
        \input{#1}
        \endinput
      }
    \fi
    \expandafter
  \endgroup
  \childdoctmp
}
%    \end{macrocode}

% \macro{\childdocforwardprefix}
% The command |\childdocforwardprefix| redirects
% compilation to the main or a child file by means of a pattern.
% The prefix |#1| in the current filename is replaced by |#2|
% and the suffix of the current filename is kept
% (it is assumed that the filename does not contain the substring `|~~~|'
% which is used as a delimiter).
% Compilation is handed over to the new file by |\childdocforward|:
%    \begin{macrocode}
\newcommand{\childdocforwardprefix}[3][]
{
  \begingroup
    \def\childdocextract #2##1~~~{\def\childdoctmp{\childdocforward[#1]{#3##1}}}
    \expandafter\childdocextract\childdocname~~~
    \expandafter
  \endgroup
  \childdoctmp
}
%    \end{macrocode}

% \macro{\childdoc}
% The deprecated macro |\childdoc| is a legacy version of |\childdocmain|:
%    \begin{macrocode}
\newcommand{\childdoc}{\childdocmain}
%    \end{macrocode}

% \macro{\childdocredirect}
% The deprecated macro |\childdocredirect| is a legacy version
% of |\childdocforward| and |\childdocforwardprefix|:
%    \begin{macrocode}
\newcommand{\childdocredirect}[2][]
{
  \begingroup
    \if?#1?
      \def\childdoctmp{\childdocforward{#2}}
    \else
      \def\childdoctmp{\childdocforwardprefix{#1}{#2}}
    \fi
    \expandafter
  \endgroup
  \childdoctmp
}
%    \end{macrocode}

%\iffalse
%</package>
%\fi
%
\endinput

\childdocof{cdocsamp}
%    \end{macrocode}

%\iffalse
%</samplechap1|samplechap2>
%\fi
%
%\iffalse
%<*samplechap1>
%\fi
% Some text for chapter 1:
%    \begin{macrocode}
\section{one}
some text in chapter one
%    \end{macrocode}

%\iffalse
%</samplechap1>
%\fi
% Some text for chapter 2:
%\iffalse
%<*samplechap2>
%\fi
%    \begin{macrocode}
\section{two}
more text in chapter two
%    \end{macrocode}

%\iffalse
%</samplechap2>
%\fi
%
% %%%%%%%%%%%%%%%%%%%%%%%%%%%%%%%%%%%%%%
% \paragraph{Part Include Files.}
%
% The include files are called |cdocspt3.tex| and |cdocspt4.tex|.
%
%\iffalse
%<*samplepart3|samplepart4>
%\fi

% Optional override for |\version| flag:
%    \begin{macrocode}
%%\providecommand{\version}{final}
%    \end{macrocode}

% Include the main document:
%    \begin{macrocode}
% \iffalse
%
% childdoc.dtx Copyright (C) 2017-2018 Niklas Beisert
%
% This work may be distributed and/or modified under the
% conditions of the LaTeX Project Public License, either version 1.3
% of this license or (at your option) any later version.
% The latest version of this license is in
%   http://www.latex-project.org/lppl.txt
% and version 1.3 or later is part of all distributions of LaTeX
% version 2005/12/01 or later.
%
% This work has the LPPL maintenance status `maintained'.
%
% The Current Maintainer of this work is Niklas Beisert.
%
% This work consists of the files childdoc.dtx and childdoc.ins
% and the derived files childdoc.def and cdocsamp.tex with
% cdocsch1.tex, cdocsch2.tex, cdocsdrf.tex, cdocsfn1.tex, cdocsfn2.tex.
%
%<package>\ifdefined\childdocmain\endinput\fi
%<package>\ProvidesFile{childdoc.def}[2018/12/30 v2.0 child document driver]
%<samplemain>\ProvidesFile{cdocsamp.tex}[2018/12/30 v2.0 sample for childdoc]
%<*driver>
%\ProvidesFile{childdoc.drv}[2018/12/30 v2.0 childdoc reference manual file]
\PassOptionsToClass{10pt,a4paper}{article}
\documentclass{ltxdoc}

\usepackage[margin=35mm]{geometry}
\usepackage{hyperref}
\usepackage{hyperxmp}
\usepackage[usenames]{color}

\hypersetup{colorlinks=true}
\hypersetup{pdfstartview=FitH}
\hypersetup{pdfpagemode=UseNone}
\hypersetup{pdfsource={}}
\hypersetup{pdflang={en-UK}}
\hypersetup{pdfcopyright={Copyright 2017-2018 Niklas Beisert.
  This work may be distributed and/or modified under the
  conditions of the LaTeX Project Public License, either version 1.3
  of this license or (at your option) any later version.}}
\hypersetup{pdflicenseurl={http://www.latex-project.org/lppl.txt}}
\hypersetup{pdfcontactaddress={ETH Zurich, ITP, HIT K,
  Wolfgang-Pauli-Strasse 27}}
\hypersetup{pdfcontactpostcode={8093}}
\hypersetup{pdfcontactcity={Zurich}}
\hypersetup{pdfcontactcountry={Switzerland}}
\hypersetup{pdfcontactemail={nbeisert@itp.phys.ethz.ch}}
\hypersetup{pdfcontacturl={http://people.phys.ethz.ch/\xmptilde nbeisert/}}

\newcommand{\secref}[1]{\hyperref[#1]{section \ref*{#1}}}

\parskip1ex
\parindent0pt
\let\olditemize\itemize
\def\itemize{\olditemize\parskip0pt}

\begin{document}

\title{The \textsf{childdoc} Package}
\hypersetup{pdftitle={The childdoc Package}}
\author{Niklas Beisert\\[2ex]
  Institut f\"ur Theoretische Physik\\
  Eidgen\"ossische Technische Hochschule Z\"urich\\
  Wolfgang-Pauli-Strasse 27, 8093 Z\"urich, Switzerland\\[1ex]
  \href{mailto:nbeisert@itp.phys.ethz.ch}
  {\texttt{nbeisert@itp.phys.ethz.ch}}}
\hypersetup{pdfauthor={Niklas Beisert}}
\hypersetup{pdfsubject={Manual for the LaTeX2e Package childdoc}}
\date{30 December 2018, \textsf{v2.0}}
\maketitle

\begin{abstract}\noindent
\textsf{childdoc} is a \LaTeXe{} package
that enables the direct compilation
of document sections included by |\include|
to individual files.
\end{abstract}

\begingroup
\parskip0ex
\tableofcontents
\endgroup

%%%%%%%%%%%%%%%%%%%%%%%%%%%%%%%%%%%%%%%%%%%%%%%%%%%%%%%%%%%%%%%%%%%%%%%%%%%%%%%%
%%%%%%%%%%%%%%%%%%%%%%%%%%%%%%%%%%%%%%%%%%%%%%%%%%%%%%%%%%%%%%%%%%%%%%%%%%%%%%%%
\section{Introduction}

\LaTeX{} provides a mechanism to structure a large document (such as a book)
into a main file and several child files (containing the chapters)
using the |\include| command.
This mechanism is beneficial for documents
which span hundreds of pages in order to
make the source file(s) more manageable.
Moreover, compilation can be restricted to
selected child files by means of the |\includeonly| command.
The latter feature can be used to reduce the compilation time while editing
(this was significantly more useful in the earlier days of \LaTeX{})
or to generate a smaller document which is easier to navigate.
Another application of |\includeonly| is to generate
documents consisting of selected parts of the complete document.

However, there are a few drawbacks of the plain |\include| mechanism:
\begin{itemize}
\item
The child files cannot be compiled on their own,
they can only be compiled via the main file.
A naive editing environment
(such as a text editor with an option
to have the current file processed by \LaTeX)
may require one to switch to the main file before compiling;
attempting to compile the child file produces errors.
\item
The main file must be modified (each time)
to adjust the |\includeonly| command
to the present needs. This easily leaves the main file in a messy state.
\item
The generated document will always carry the filename
of the main document. This is inconvenient if
several child files are to be compiled and
to be kept for distribution.
\end{itemize}

The present package provides a simple interface
to make child files individually compilable by \LaTeX{}.
Compiling a child file then has the same effect as compiling
the main file with an |\includeonly| command
to select the appropriate child.
Moreover the generated document will carry the name of the child
rather than the main file.
This resolves all three above issues.

This feature is meant to make the editing of books,
thesis documents and lecture notes somewhat more convenient.
However, the package can also be used efficiently for
composing a series of documents (such as exercise sheets)
which are typically distributed individually.
It then assists the author in generating the individual documents
(potentially in different versions)
as well as a document containing the collected series.
Another application is in developing style files
or other kinds of included material
where compilation of the style file could redirect
to a sample or test file.

%%%%%%%%%%%%%%%%%%%%%%%%%%%%%%%%%%%%%%%%%%%%%%%%%%%%%%%%%%%%%%%%%%%%%%%%%%%%%%%%
%%%%%%%%%%%%%%%%%%%%%%%%%%%%%%%%%%%%%%%%%%%%%%%%%%%%%%%%%%%%%%%%%%%%%%%%%%%%%%%%
\section{Usage}

First of all, the package \textsf{childdoc} is \emph{not} a standard
\LaTeXe{} |.sty| style file! Therefore it needs to be invoked in
a non-standard way.

%%%%%%%%%%%%%%%%%%%%%%%%%%%%%%%%%%%%%%%%%%%%%%%%%%%%%%%%%%%%%%%%%%%%%%%%%%%%%%%%
\subsection{Included Files}
\label{sec:include}

%%%%%%%%%%%%%%%%%%%%%%%%%%%%%%%%%%%%%%%%
\DescribeMacro{\childdocmain}
To use the package, add the commands
\begin{center}
\begin{tabular}{l}
|\input{childdoc.def}|\\
|\childdocmain{}|\\
\end{tabular}
\end{center}
at the very top of the main \LaTeX{} file,
in particular \emph{before} the |\documentclass| statement!
The argument of |\childdocmain| should be left empty
(but it must be present).

%%%%%%%%%%%%%%%%%%%%%%%%%%%%%%%%%%%%%%%%
\DescribeMacro{\childdocof}
Furthermore, add the commands
\begin{center}
\begin{tabular}{l}
|\input{childdoc.def}|\\
|\childdocof{|\textit{main}|}|\\
\end{tabular}
\end{center}
at the top of every child file \textit{child}
which is included by |\include{|\textit{child}|}|
from within the main file
(or at least for those files to be compiled individually).
The argument \textit{main} must be the filename of the main file.

There are a couple of
considerations in setting up the main and child documents:

%%%%%%%%%%%%%%%%%%%%%%%%%%%%%%%%%%%%%%%%
\paragraph{Restrictions.}

Please note the following restrictions:
\begin{itemize}
\item
|\childdocmain| must be called with one argument \textit{main}
to ensure compatibility with earlier version of the package.
It must either be empty (|\childdocmain{}|)
or precisely match the filename of the main file in which it is specified.
See \secref{sec:detection} for further information.
\item
The filename \textit{main} must be specified without the |.tex| extension.
\item
The filename \textit{main} is case sensitive
(even in case-insensitive file systems)
due to internal string comparison.
\item
The argument \textit{main} should be fully expanded, it cannot be a macro.
\item
Subdirectories and special characters should be avoided in filenames.
\item
The command |\childdocmain{|\textit{main}|}| must be followed by a whitespace.
It should not be followed immediately by another command
or by a comment mark `|%|'.
This is because the \TeX{} parser reads the token immediately following
the argument of |\childdocmain| and puts it
at the beginning of every child section;
however, a white\-space is ignored.
\end{itemize}

%%%%%%%%%%%%%%%%%%%%%%%%%%%%%%%%%%%%%%%%
\paragraph{Content of Main File.}

It is advisable to place all content in the child files included by |\include|.
Any output contained in the main file will appear in all child documents
unless suppressed manually;
it cannot be suppressed automatically by the |\includeonly| directive
and thus should normally be avoided.
A method to include some content in the main file
by means of conditional processing is described in \secref{sec:conditional}.

%%%%%%%%%%%%%%%%%%%%%%%%%%%%%%%%%%%%%%%%
\paragraph{Page Numbering.}

When only a part of the document is compiled,
the appropriate numbering of pages
(as well as other status parameters)
is determined from the |.aux| files.
The latter contain information from previous passes.
However this information needs to propagate through
all intermediate child documents.
Therefore the page numbering in child documents may well
be inconsistent until the complete document is compiled at least once.

A useful (if unconventional) way to always ensure a consistent
page numbering is to restart the numbering in each child document
and denote the pages by `\textit{child}|.|\textit{page}'
where \textit{child} represents the chapter/section number of the child file.
This can be achieved by the command
|\numberwithin{page}{|\textit{child}|}|
of the \textsf{amsmath} package
where \textit{child} can be |chapter| or |section|
depending on the chosen structuring.
Alternatively, one can modify the macro |\thepage| appropriately
and reset the counter |page| at the start of each child file.

%%%%%%%%%%%%%%%%%%%%%%%%%%%%%%%%%%%%%%%%%%%%%%%%%%%%%%%%%%%%%%%%%%%%%%%%%%%%%%%%
\subsection{Conditional Processing}
\label{sec:conditional}

The package provides a mechanism to compile different versions
of a document. To customise the versions further some conditional processing
can come in handy to distinguish which version is being compiled.
The package provides two macros to describe the compilation context:

%%%%%%%%%%%%%%%%%%%%%%%%%%%%%%%%%%%%%%%%
\DescribeMacro{\ifchilddoc}
The conditional |\ifchilddoc| distinguishes between the compilation of
child documents and the main document:
%
\begin{center}
|\ifchilddoc |\textit{child-code}| |[|\||else |\textit{main-code}]| \||fi|
\end{center}

%%%%%%%%%%%%%%%%%%%%%%%%%%%%%%%%%%%%%%%%
\DescribeMacro{\childdocname}
\DescribeMacro{\childdocjob}
The macro |\childdocname| contains the filename (without extension)
of the main or child file being processed.
Note that |\childdocjob| will always contain the name of the main file.

%%%%%%%%%%%%%%%%%%%%%%%%%%%%%%%%%%%%%%%%
\paragraph{Title Page.}

Conditional processing can be used to include a title or banner page
in the main document when proper precautions are taken.
Importantly, the code in the main file should ensure that the page counter
(as well as other status parameters which are stored in the |.aux| files)
takes the same value after the conditional processing.
Otherwise the page numbers may take divergent values
depending on which part is compiled.

For example, a title page could be declared by:
%
\begin{center}
\begin{tabular}{l}
|\ifchilddoc\||else|\\
|\addtocounter{page}{-1}|\\
\textit{code for title page}\\
|\newpage|\\
|\||fi|
\end{tabular}
\end{center}
%
A banner page for the child documents can be generated by:
%
\begin{center}
\begin{tabular}{l}
|\ifchilddoc|\\
|\addtocounter{page}{-1}|\\
\textit{code for banner page}\\
|\newpage|\\
|\||fi|
\end{tabular}
\end{center}
%
Here one could write a message such as:
\begin{center}
|This is the part \childdocname{} of \childdocjob{}.|
\end{center}

%%%%%%%%%%%%%%%%%%%%%%%%%%%%%%%%%%%%%%%%%%%%%%%%%%%%%%%%%%%%%%%%%%%%%%%%%%%%%%%%
\subsection{Flags}
\label{sec:flags}

The package makes it easy to generate different versions
of the main or child documents.
To this end compilation flags can be defined
and assigned different default values.
They will be particularly useful in conjunction
with the forwarding mechanism described in \secref{sec:forward}.

For example, it may be useful to have a flag |\version|
which can be set to |draft| or |final|.
The document source will contain some conditional code
depending on the value of |\version|.
Suppose further, the flag should default to |final| for the main file
and to |draft| for child files
which is a natural assignment for editing the document.
This is achieved by placing the following code
in the preamble of the main document
(below the |\childdocmain| directive):
%
\begin{center}
\begin{tabular}{l}
|\ifchilddoc|\\
|\providecommand{\version}{draft}|\\
|\||else|\\
|\providecommand{\version}{final}|\\
|\||fi|
\end{tabular}
\end{center}
%
The definition by |\providecommand| makes sure
that previous definitions are not overwritten.
Further statements |\providecommand{\version}{...}|
can thus be added before the above code to override it.

For the main file, one might add a line
(between |\childdocmain| and the above block)
%
\begin{center}
|%\ifchilddoc\||else\providecommand{\version}{draft}\||fi|
\end{center}
%
which can be uncommented to produce a draft version.
Likewise one can add a line to the very top of a child file
(above the |\childdocof{|\textit{main}|}| directive)
%
\begin{center}
|%\providecommand{\version}{final}|
\end{center}
%
which can be uncommented to produce the final version of this child document.

%%%%%%%%%%%%%%%%%%%%%%%%%%%%%%%%%%%%%%%%%%%%%%%%%%%%%%%%%%%%%%%%%%%%%%%%%%%%%%%%
\subsection{Forwarding}
\label{sec:forward}

Different versions of the main or child documents
using compilation flags as described in \secref{sec:flags}
can be (permanently) stored in different files
for convenient compilation, viewing and distribution.
To this end, the package defines a command
to pass on compilation to a different file:

%%%%%%%%%%%%%%%%%%%%%%%%%%%%%%%%%%%%%%%%
\DescribeMacro{\childdocforward}
The command |\childdocforward| redirects processing to
another source file:
%
\begin{center}
\begin{tabular}{l}
|\input{childdoc.def}|\\
|\childdocforward[|\textit{main}|]{|\textit{dest}|}|\\
\end{tabular}
\end{center}
%
The argument \textit{dest} is the destination file
(without extension).
It should be the main file or one of the child files.
Note that further \textsf{childdoc} directives
such as |\childdocof| and |\childdocforward|
in the indicated file will be processed in this form.
The optional argument \textit{main}
passes on directly to the main file \textit{main}
while pretending to compile the child \textit{dest}.
This form behaves as if \textit{dest}
issues |\childdocof{|\textit{main}|}| right away,
and no further \textsf{childdoc} directives will be processed.

%%%%%%%%%%%%%%%%%%%%%%%%%%%%%%%%%%%%%%%%
\DescribeMacro{\...prefix}
In the alternative form |\childdocforwardprefix|,
%
\begin{center}
\begin{tabular}{l}
|\input{childdoc.def}|\\
|\childdocforwardprefix[|\textit{main}|]{|\textit{prefix}|}{|\textit{dest}|}|
\end{tabular}
\end{center}
%
the destination file is determined by a pattern
depending on the current file:
To make this work, the current file must be called
`{\textit{prefix}\hspace{0.2em}\textit{suffix}}'
with \textit{prefix} matching precisely the argument.
Processing is then passed on to the file
`{\textit{dest}\hspace{0.2em}\textit{suffix}}'.
Surely, the same effect is achieved by
directly specifying the
argument `{\textit{dest}\hspace{0.2em}\textit{suffix}}'
in the first form.
However, that requires to set up a different file
for each child. With the alternative form of the command
all these files can have exactly the same content
which simplifies setting them up and maintaining them.

For example, the following file |draft.tex|
with a compilation flag |\version| as described in \secref{sec:flags}
compiles the main document as a draft:
%
\begin{center}
\begin{tabular}{l}
|\def\version{draft}|\\
|\input{childdoc.def}|\\
|\childdocforward{|\textit{main}|}|
\end{tabular}
\end{center}
%
Likewise, the following files |final|\textit{nn}|.tex|
compile the final version of the child document
|child|\textit{nn}|.tex|:
%
\begin{center}
\begin{tabular}{l}
|\def\version{final}|\\
|\input{childdoc.def}|\\
|\childdocforwardprefix{final}{child}|
\end{tabular}
\end{center}
%

Note that when several versions of a main file and/or of each child file
are to be generated, it may be convenient to set up a |Makefile| or
shell script to automatise the process.

%%%%%%%%%%%%%%%%%%%%%%%%%%%%%%%%%%%%%%%%%%%%%%%%%%%%%%%%%%%%%%%%%%%%%%%%%%%%%%%%
\subsection{Command Line Processing}
\label{sec:commandline}

The effect of redirection files can also be achieved by invoking
the \LaTeX{} compiler with a more elaborate command line.
Most conveniently this should be done as part
of a shell script or a |Makefile|.

When using \textsf{childdoc} in the main file, the following
command lines effectively perform a redirection
(note that depending on the shell being used,
backslashes may have to be doubled: `|\|' $\to$ `|\\|'):
%
\begin{center}
|... -jobname "|\textit{target}|" |\\|"|[\textit{flags}]%
|\input{childdoc.def}\childdocforward[|\textit{main}|]{|\textit{dest}|}"|
\end{center}
%
Here \textit{target} is the name of the output file,
\textit{main} is the name of the main file
and \textit{dest} is the name of the main or child file to be processed
(all filenames without extensions).
The optional argument \textit{main} can be omitted
if \textit{main} matches \textit{dest}.
Optionally, compilation \textit{flags} can be defined via |\def| commands.
This command line makes the \TeX{} engine believe
it is compiling the file \textit{target}
whose content is specified as the latter parameter.
The provided code then forwards the processing to
\textit{main} or \textit{dest} as described in \secref{sec:forward}.

%%%%%%%%%%%%%%%%%%%%%%%%%%%%%%%%%%%%%%%%%%%%%%%%%%%%%%%%%%%%%%%%%%%%%%%%%%%%%%%%
\subsection{Include by Input}
\label{sec:input}

Including child documents by |\include| has some restrictions by design.
Most notably, the content of a child document always occupies
its own set of pages; pages cannot be shared between child documents.
Usually, this behaviour makes perfect sense
because each child document contain an essential part of the document.
However, in some situations it may be desirable to compose
a document from a collection of parts
without having mandatory page breaks between then.
For this case, the package
provides a mechanism to include parts
by |\input| which can also be processed individually.
However, by construction this mechanism
requires manual handling of the content to be output.

%%%%%%%%%%%%%%%%%%%%%%%%%%%%%%%%%%%%%%%%
\DescribeMacro{\ifchilddocmanual}
The main file should be prepared as usual, see \secref{sec:include}.
However, the document body must make a distinction
between processing of an individual part and of the main document, e.g.:
%
\begin{center}
\begin{tabular}{l}
|\ifchilddocmanual|\\
|\input{\childdocname}|\\
|\||else|\\
\textit{document body with }|\input{|\textit{part}|}|\\
|\||fi|
\end{tabular}
\end{center}
%
The conditional |\ifchilddocmanual| is true whenever
a part to be included by |\input| is being compiled,
and the name of the part is stored in |\childdocname|.

%%%%%%%%%%%%%%%%%%%%%%%%%%%%%%%%%%%%%%%%
\DescribeMacro{\childdocby}
Each part to be included by |\input| should start with:
%
\begin{center}
\begin{tabular}{l}
|\input{childdoc.def}|\\
|\childdocby{|\textit{main}|}|\\
\end{tabular}
\end{center}
%
The directive |\childdocby| is similar to |\childdocof|
described in \secref{sec:include},
but the subsequent selection of content must be done manually.
To that end, both |\ifchilddoc| and |\ifchilddocmanual|
will be true upon processing of a part,
and the name of the part is stored in |\childdocname|.
Note that |\jobname| will be set to the filename of the current part
so that each part receives an individual |.aux| file
that does not interfere with the |.aux| file(s) of the main document.
This behaviour can be altered by the alternative form
|\childdocby[*]{|\textit{main}|}| (with a non-empty optional argument)
which uses the |.aux| file of the main document
by setting |\jobname| to \textit{main}.

%%%%%%%%%%%%%%%%%%%%%%%%%%%%%%%%%%%%%%%%%%%%%%%%%%%%%%%%%%%%%%%%%%%%%%%%%%%%%%%%
\subsection{Driver Development}
\label{sec:driver}

The \textsf{childdoc} mechanism can also be use for the development
of definition files such as \LaTeX{} styles or classes.
This case differs from the above setup with multiple parts
included by |\include| in that no |\includeonly| should be invoked.
This can be achieved by starting the include file
(before |\ProvidesPackage|) with:
%
\begin{center}
\begin{tabular}{l}
|\input{childdoc.def}|\\
|\childdocforward{|\textit{main}|}|\\
\end{tabular}
\end{center}
%
or alternatively with:
%
\begin{center}
\begin{tabular}{l}
|\input{childdoc.def}|\\
|\childdocby{|\textit{main}|}|\\
\end{tabular}
\end{center}
%
Both forms have slightly different effects as described above.
The main file is prepared as usual, see \secref{sec:include}.

%%%%%%%%%%%%%%%%%%%%%%%%%%%%%%%%%%%%%%%%%%%%%%%%%%%%%%%%%%%%%%%%%%%%%%%%%%%%%%%%
\subsection{Legacy Detection}
\label{sec:detection}

The directive |\childdocmain| in the main file can detect
whether the complete document or merely a child is to be compiled
even without using the directive |\childdocof|.
This method is deprecated because it is less robust
and there is no compelling reason to use it;
it is merely provided for backward compatibility
and it may be removed in future versions.

If the detection mechanism is to be used,
it is mandatory to correctly specify
the filename of the main file as the argument of |\childdocmain|:
%
\begin{center}
\begin{tabular}{l}
|\input{childdoc.def}|\\
|\childdocmain{|\textit{main}|}|\\
\end{tabular}
\end{center}
%
If |\jobname| does not match the argument \textit{main} of |\childdocmain|,
it is assumed that |\jobname| points to the child file to be compiled.
When using |\childdocmain| with the main file specified as argument,
it suffices to start a child file
with just |\input{|\textit{main}|}|
without loading of the package and using |\childdocof|.
If instead all processing is done
with the appropriate \textsf{childdoc} directives,
the argument of \textit{main} of |\childdocmain| can be empty.

An alternative version of the command line processing described
in \secref{sec:commandline} using the detection mechanism reads:
%
\begin{center}
|... -jobname "|\textit{target}|" "|[\textit{flags}]%
[|\def\jobname{|\textit{dest}|}|]|\input{|\textit{main}|}"|
\end{center}

%%%%%%%%%%%%%%%%%%%%%%%%%%%%%%%%%%%%%%%%%%%%%%%%%%%%%%%%%%%%%%%%%%%%%%%%%%%%%%%%
\subsection{Manual Code}
\label{sec:manual}

In case one cannot be certain whether the definitions file |childdoc.def|
is installed on the target \TeX{} distribution
and one prefers not to ship it,
it is conceivable to paste a few relevant commands into the sources.

To that end, drop all statements |\input{childdoc.def}|
and perform the replacements as outlined below.
Instead of |\childdocmain{|\textit{main}|}| add the following code
to the top of the main file:
%
\begin{center}
\begin{tabular}{l}
|\||ifdefined\childdocname\endinput\||fi\newif\ifchilddoc|\\
|\edef\childdocname{\scantokens\expandafter{\jobname\noexpand}}|\\
|\def\childdocmain{|\textit{main}|}\||ifx\childdocmain\childdocname\||else|\\
|\childdoctrue\includeonly{\childdocname}\let\jobname\childdocmain\||fi|\\
\end{tabular}
\end{center}
%
Instead of |\childdocof{|\textit{main}|}| just include the main file
at the top of each child file:
%
\begin{center}
|\input{|\textit{main}|}|
\end{center}
%
A simple redirection |\childdocforward{|\textit{dest}|}| is achieved by:
%
\begin{center}
|\def\jobname{|\textit{dest}|}\input{\jobname}|
\end{center}
%
The redirection with prefix
|\childdocforwardprefix[|\textit{prefix}|]{|\textit{dest}|}|
is accomplished by:
%
\begin{center}
\begin{tabular}{l}
|{\edef\jobname{\scantokens\expandafter{\jobname\noexpand}}|\\
|\def\redirectjob |\textit{prefix}|#1~~~{\gdef\jobname{|\textit{dest}|#1}}|\\
|\expandafter\redirectjob\jobname~~~}\input{\jobname}|
\end{tabular}
\end{center}

In an alternative approach,
child documents can be compiled by a specific command line
without additional code or specific definitions:
%
\begin{center}
|... -jobname "|\textit{target}|" "|[\textit{flags}]%
|\includeonly{|\textit{dest}|}\input{|\textit{main}|}"|
\end{center}
%

%%%%%%%%%%%%%%%%%%%%%%%%%%%%%%%%%%%%%%%%%%%%%%%%%%%%%%%%%%%%%%%%%%%%%%%%%%%%%%%%
%%%%%%%%%%%%%%%%%%%%%%%%%%%%%%%%%%%%%%%%%%%%%%%%%%%%%%%%%%%%%%%%%%%%%%%%%%%%%%%%
\section{Information}

%%%%%%%%%%%%%%%%%%%%%%%%%%%%%%%%%%%%%%%%%%%%%%%%%%%%%%%%%%%%%%%%%%%%%%%%%%%%%%%%
\subsection{Copyright}

Copyright \copyright{} 2017--2018 Niklas Beisert

This work may be distributed and/or modified under the
conditions of the \LaTeX{} Project Public License, either version 1.3
of this license or (at your option) any later version.
The latest version of this license is in
  \url{http://www.latex-project.org/lppl.txt}
and version 1.3 or later is part of all distributions of \LaTeX{}
version 2005/12/01 or later.

This work has the LPPL maintenance status `maintained'.

The Current Maintainer of this work is Niklas Beisert.

This work consists of the files |README.txt|, |childdoc.ins| and |childdoc.dtx|
as well as the derived files |childdoc.def|, |cdocsamp.tex|
with |cdocsch1.tex|, |cdocsch2.tex|, |cdocspt3.tex|, |cdocspt4.tex|,
|cdocsdrf.tex|, |cdocsfn1.tex|, |cdocsfn2.tex|
as well as |childdoc.pdf|.

%%%%%%%%%%%%%%%%%%%%%%%%%%%%%%%%%%%%%%%%%%%%%%%%%%%%%%%%%%%%%%%%%%%%%%%%%%%%%%%%
\subsection{Files and Installation}

The package consists of the files:
%
\begin{center}
\begin{tabular}{ll}
    |README.txt|   & readme file \\
    |childdoc.ins| & installation file \\
    |childdoc.dtx| & source file \\
    |childdoc.def| & definition file \\
    |cdocsamp.tex| & sample main file \\
    |cdocsch1.tex| & sample include file \\
    |cdocsch2.tex| & sample include file \\
    |cdocspt3.tex| & sample part file \\
    |cdocspt4.tex| & sample part file \\
    |cdocsdrf.tex| & sample redirection file \\
    |cdocsfn1.tex| & sample redirection file \\
    |cdocsfn2.tex| & sample redirection file \\
    |childdoc.pdf| & manual
\end{tabular}
\end{center}
%
The distribution consists of the files
|README.txt|, |childdoc.ins| and |childdoc.dtx|.
%
\begin{itemize}
\item
Run (pdf)\LaTeX{} on |childdoc.dtx|
to compile the manual |childdoc.pdf| (this file).
\item
Run \LaTeX{} on |childdoc.ins| to create the definitions file |childdoc.def|
and the sample |cdocsamp.tex| with include files
|cdocsch1.tex|, |cdocsch2.tex|, |cdocspt3.tex|, |cdocspt4.tex|,
|cdocsdrf.tex|, |cdocsfn1.tex|, |cdocsfn2.tex|.
Then copy the file |childdoc.def| to an appropriate directory of your \LaTeX{}
distribution, e.g.\ \textit{texmf-root}|/tex/latex/childdoc|.
\end{itemize}

%%%%%%%%%%%%%%%%%%%%%%%%%%%%%%%%%%%%%%%%%%%%%%%%%%%%%%%%%%%%%%%%%%%%%%%%%%%%%%%%
\subsection{Related CTAN Packages}

There are several other packages which offer a similar functionality:
%
\begin{itemize}
\item
The packages
\href{http://ctan.org/pkg/docmute}{\textsf{docmute}},
\href{http://ctan.org/pkg/includex}{\textsf{includex}} and
\href{http://ctan.org/pkg/standalone}{\textsf{standalone}}
provide commands to include only the document body of
a child file thus allowing both files to be compiled individually.
\item
The packages \href{http://ctan.org/pkg/subdocs}{\textsf{subdocs}}
and \href{http://ctan.org/pkg/subfiles}{\textsf{subfiles}}
provide structures in which the main and child documents can be
encapsulated and allowing them to be compiled individually.
The inclusion mechanism is different from the conventional |\include|.
\item
The package \href{http://ctan.org/pkg/combine}{\textsf{combine}}
is an elaborate solution to combine several documents into one.
\end{itemize}
%
See also the CTAN topic \href{http://ctan.org/topic/subdocs}{\textsf{subdocs}}
for further related packages.
The present package differs from the above solutions in that
a document structure constructed with the conventional |\include| mechanism
just needs two extra commands at the top of every file
such that all constituent files can be compiled individually.

%%%%%%%%%%%%%%%%%%%%%%%%%%%%%%%%%%%%%%%%%%%%%%%%%%%%%%%%%%%%%%%%%%%%%%%%%%%%%%%%
%\subsection{Feature Suggestions}
%
%The following is a list of features which may be useful for future
%versions of this package:
%%
%\begin{itemize}
%\item
%\ldots
%\end{itemize}

%%%%%%%%%%%%%%%%%%%%%%%%%%%%%%%%%%%%%%%%%%%%%%%%%%%%%%%%%%%%%%%%%%%%%%%%%%%%%%%%
\subsection{Revision History}

%%%%%%%%%%%%%%%%%%%%%%%%%%%%%%%%%%%%%%%%
\paragraph{v2.0:} 2018/12/30

\begin{itemize}
\item
immediate forward processing
\item
added |\childdocby| mechanism
\item
manual restructured
\end{itemize}

%%%%%%%%%%%%%%%%%%%%%%%%%%%%%%%%%%%%%%%%
\paragraph{v1.6:} 2018/01/17

\begin{itemize}
\item
application for development of include files
\item
corrections to manual
\end{itemize}

%%%%%%%%%%%%%%%%%%%%%%%%%%%%%%%%%%%%%%%%
\paragraph{v1.5:} 2017/05/21

\begin{itemize}
\item
more complete structuring introduced
\item
|\childdocof| introduced
\item
|\childdoc| renamed to |\childdocmain|
\item
|\childredirect| renamed to |\childdocforward| and |\childdocforwardprefix|
and functionality expanded
\end{itemize}

%%%%%%%%%%%%%%%%%%%%%%%%%%%%%%%%%%%%%%%%
\paragraph{v1.0:} 2017/04/27

\begin{itemize}
\item
manual and install package
\item
first version published on CTAN
\end{itemize}

%%%%%%%%%%%%%%%%%%%%%%%%%%%%%%%%%%%%%%%%
\paragraph{v0.6:} 2017/04/26

\begin{itemize}
\item
redirection mechanism added
\end{itemize}

%%%%%%%%%%%%%%%%%%%%%%%%%%%%%%%%%%%%%%%%
\paragraph{v0.5:} 2017/04/26

\begin{itemize}
\item
functionality in definition file
\end{itemize}


%%%%%%%%%%%%%%%%%%%%%%%%%%%%%%%%%%%%%%%%%%%%%%%%%%%%%%%%%%%%%%%%%%%%%%%%%%%%%%%%
%%%%%%%%%%%%%%%%%%%%%%%%%%%%%%%%%%%%%%%%%%%%%%%%%%%%%%%%%%%%%%%%%%%%%%%%%%%%%%%%
%%%%%%%%%%%%%%%%%%%%%%%%%%%%%%%%%%%%%%%%%%%%%%%%%%%%%%%%%%%%%%%%%%%%%%%%%%%%%%%%
\appendix

\settowidth\MacroIndent{\rmfamily\scriptsize 000\ }

 \DocInput{childdoc.dtx}

\end{document}
%</driver>
% \fi
%
% %%%%%%%%%%%%%%%%%%%%%%%%%%%%%%%%%%%%%%%%%%%%%%%%%%%%%%%%%%%%%%%%%%%%%%%%%%%%%%
% %%%%%%%%%%%%%%%%%%%%%%%%%%%%%%%%%%%%%%%%%%%%%%%%%%%%%%%%%%%%%%%%%%%%%%%%%%%%%%
% \section{Sample}
%\iffalse
%<*samplemain>
%\fi
%
% The following presents a sample document
% with two chapters, two parts, a title page,
% a compile flag as well as three forwarding files to set the flag.
% It consists of eight |.tex| files:
% \begin{center}
% \begin{tabular}{ll}
% |cdocsamp.tex|&main file\\
% |cdocsch1.tex|&include file for chapter 1\\
% |cdocsch2.tex|&include file for chapter 2\\
% |cdocspt3.tex|&include file for part 3\\
% |cdocspt4.tex|&include file for part 4\\
% |cdocsdrf.tex|&forwarding file for main file in draft mode\\
% |cdocsfi1.tex|&forwarding file for final version of chapter 1\\
% |cdocsfi2.tex|&forwarding file for final version of chapter 2\\
% \end{tabular}
% \end{center}
% Each of the eight files can be compiled directly by the \LaTeX{} compiler.
%
% %%%%%%%%%%%%%%%%%%%%%%%%%%%%%%%%%%%%%%
% \paragraph{Main File.}
%
% The main file is called |cdocsamp.tex|.
%
% Load the \textsf{childdoc} definitions and
% declare the filename for the main document:
%    \begin{macrocode}
\input{childdoc.def}
\childdocmain{}
%    \end{macrocode}

% Optional override for |\version| flag:
%    \begin{macrocode}
%%\ifchilddoc\else\providecommand{\version}{draft}\fi
%    \end{macrocode}

% Define the default values for the |\version| flag
% (|final| for the main file and |draft| for childs):
%    \begin{macrocode}
\ifchilddoc
\providecommand{\version}{draft}
\else
\providecommand{\version}{final}
\fi
%    \end{macrocode}

% Load the standard document class:
%    \begin{macrocode}
\documentclass[12pt]{article}
%    \end{macrocode}

% Start the document body:
%    \begin{macrocode}
\begin{document}
%    \end{macrocode}

% Declare a title page.
% Print title, part of document being processed and version flag:
%    \begin{macrocode}
\addtocounter{page}{-1}
\begin{center}
{\LARGE\bfseries{}childdoc example\par}
\vspace{1cm}
\ifchilddoc
\ifchilddocmanual part\else chapter\fi:
`\childdocname' of `\childdocjob'\par
\else
main document: `\childdocjob'\par
\fi
version: \version\par
\end{center}
\newpage
%    \end{macrocode}

% Manually include selected file,
% otherwise process as usual:
%    \begin{macrocode}
\ifchilddocmanual
\section*{part `\childdocname'}
\input{\childdocname}
\else
%    \end{macrocode}

% Include the two chapters:
%    \begin{macrocode}
\include{cdocsch1}
\include{cdocsch2}
%    \end{macrocode}

% Include the two parts unless only chapters should be displayed:
%    \begin{macrocode}
\ifchilddoc\else
\section{part three}
\input{cdocspt3}
\section{part four}
\input{cdocspt4}
\fi
%    \end{macrocode}

% Process as usual until here:
%    \begin{macrocode}
\fi
%    \end{macrocode}

% End of document body:
%    \begin{macrocode}
\end{document}
%    \end{macrocode}
%\iffalse
%</samplemain>
%\fi
%
% %%%%%%%%%%%%%%%%%%%%%%%%%%%%%%%%%%%%%%
% \paragraph{Chapter Include Files.}
%
% The include files are called |cdocsch1.tex| and |cdocsch2.tex|.
%
%\iffalse
%<*samplechap1|samplechap2>
%\fi

% Optional override for |\version| flag:
%    \begin{macrocode}
%%\providecommand{\version}{final}
%    \end{macrocode}

% Include the main document:
%    \begin{macrocode}
\input{childdoc.def}
\childdocof{cdocsamp}
%    \end{macrocode}

%\iffalse
%</samplechap1|samplechap2>
%\fi
%
%\iffalse
%<*samplechap1>
%\fi
% Some text for chapter 1:
%    \begin{macrocode}
\section{one}
some text in chapter one
%    \end{macrocode}

%\iffalse
%</samplechap1>
%\fi
% Some text for chapter 2:
%\iffalse
%<*samplechap2>
%\fi
%    \begin{macrocode}
\section{two}
more text in chapter two
%    \end{macrocode}

%\iffalse
%</samplechap2>
%\fi
%
% %%%%%%%%%%%%%%%%%%%%%%%%%%%%%%%%%%%%%%
% \paragraph{Part Include Files.}
%
% The include files are called |cdocspt3.tex| and |cdocspt4.tex|.
%
%\iffalse
%<*samplepart3|samplepart4>
%\fi

% Optional override for |\version| flag:
%    \begin{macrocode}
%%\providecommand{\version}{final}
%    \end{macrocode}

% Include the main document:
%    \begin{macrocode}
\input{childdoc.def}
\childdocby{cdocsamp}
%    \end{macrocode}

%\iffalse
%</samplepart3|samplepart4>
%\fi
%
%\iffalse
%<*samplepart3>
%\fi
% Some text for part 3:
%    \begin{macrocode}
some text in part three
%    \end{macrocode}

%\iffalse
%</samplepart3>
%\fi
% Some text for part 4:
%\iffalse
%<*samplepart4>
%\fi
%    \begin{macrocode}
more text in part four
%    \end{macrocode}

%\iffalse
%</samplepart4>
%\fi
%
% %%%%%%%%%%%%%%%%%%%%%%%%%%%%%%%%%%%%%%
% \paragraph{Forwarding for a Complete Draft.}
%
% The following forwarding file |cdocsdrf.tex|
% compiles the main document in draft mode:
%\iffalse
%<*sampledraft>
%\fi
%    \begin{macrocode}
\def\version{draft}
\input{childdoc.def}
\childdocforward{cdocsamp}
%    \end{macrocode}

%\iffalse
%</sampledraft>
%\fi
%
% %%%%%%%%%%%%%%%%%%%%%%%%%%%%%%%%%%%%%%
% \paragraph{Forwarding for Final Version of the Chapters.}
%
% The following forwarding files |cdocsfn1.tex| and |cdocsfn2.tex|
% (with identical content)
% compile the final versions of the child documents
% |cdocsch1.tex| and |cdocsch2.tex|, respectively:
%\iffalse
%<*samplefinal>
%\fi
%    \begin{macrocode}
\def\version{final}
\input{childdoc.def}
\childdocforwardprefix[cdocsamp]{cdocsfn}{cdocsch}
%    \end{macrocode}

%\iffalse
%</samplefinal>
%\fi
%
% %%%%%%%%%%%%%%%%%%%%%%%%%%%%%%%%%%%%%%
% \paragraph{Command Line Processing.}
%
% The following three command lines generate the output files
% |cdocscld|, |cdocscl1| and |cdocscl2|
% which should be identical to
% |cdocsdrf|, |cdocsch1| and |cdocsfn2|, respectively:
% \begin{center}
% \begin{tabular}{l}
% |latex -jobname cdocscld \|\\
% |  "\def\version{draft}\input{childdoc.def}\childdocforward{cdocsamp}"|\\
% |latex -jobname cdocscl1 \|\\
% |  "\input{childdoc.def}\childdocforward[cdocsamp]{cdocsch1}"|\\
% |latex -jobname cdocscl2 \|\\
% |  "\def\version{final}\input{childdoc.def}\childdocforward{cdocsch2}"|
% \end{tabular}
% \end{center}
% Note that the trailing backslash on each first line
% merely continues the input to the second line
% (for convenient cut ant paste).
% Furthermore, the command |latex| can be replaced by any
% of its alternative versions such as |pdflatex|.
%
% %%%%%%%%%%%%%%%%%%%%%%%%%%%%%%%%%%%%%%%%%%%%%%%%%%%%%%%%%%%%%%%%%%%%%%%%%%%%%%
% %%%%%%%%%%%%%%%%%%%%%%%%%%%%%%%%%%%%%%%%%%%%%%%%%%%%%%%%%%%%%%%%%%%%%%%%%%%%%%
% \section{Implementation}
%\iffalse
%<*package>
%\fi
%
% This section describes the definitions file |childdoc.def|.

% The definitions cannot be loaded using |\usepackage| or |\RequirePackage|
% which has a mechanism to prevent loading a style file more than once.
% When loading the definitions by means of |\input|
% multiple instances have to be prevented manually:
%\iffalse
%This code needs to be before the `\ProvidesFile' directive
%which is defined at the beginning of this file.
%Therefore it is also placed there and commented out here.
%</package>
%<*discard>
%\fi
%    \begin{macrocode}
\ifdefined\childdocmain\endinput\fi
%    \end{macrocode}
%\iffalse
%</discard>
%<*package>
%\fi
%
% \macro{\ifchilddoc}
% \macro{\ifchilddocmanual}
% The conditional |\ifchilddoc| tells whether a
% child (true) or main (false) document is being compiled.
% The conditional |\ifchilddocmanual| tells whether
% the |\includeonly| mechanism is used (false) or
% the selection of child files must be performed manually (true).
% The definitions initialise to false:
%    \begin{macrocode}
\newif\ifchilddoc
\newif\ifchilddocmanual
%    \end{macrocode}

% \macro{\childdocname}
% \macro{\childdocjob}
% The macro |\childdocname| stores the name of the main document
% to be compiled. The macro |\childdocjob| stores the name of
% the document on which the \LaTeX{} compiler was originally invoked.
% The content of |\jobname| cannot be compared
% to filenames specified in the source due to different catcodes.
% The following code rescans |\jobname|, stores the result
% in |\childdocname| and saves a copy in |\childdocjob|:
%    \begin{macrocode}
\edef\childdocname{\scantokens\expandafter{\jobname\noexpand}}
\let\childdocjob\childdocname
%    \end{macrocode}

% \macro{\childdocdisable}
% The macro |\childdocdisable| prevents the main file
% from being processed more than once.
% At this stage, the main document command |\childdocmain|
% is assumed to be called once again where it should do nothing.
% Any subsequent call to it should prevent
% a secondary processing of the main document
% It overwrites the forwarding commands
% |\childdocof| and |\childdocforward|
% with empty macros to prevent further inclusions of the main document:
%    \begin{macrocode}
\newcommand{\childdocdisable}
{
  \renewcommand{\childdocmain}[1]{\renewcommand{\childdocmain}[1]{\endinput}}
  \renewcommand{\childdocof}[1]{}
  \renewcommand{\childdocby}[2][]{}
  \renewcommand{\childdocforward}[2][]{}
  \renewcommand{\childdocdisable}{}
}
%    \end{macrocode}

% \macro{\childdocmain}
% The macro |\childdocmain| is to be called at the top of the main file
% with nothing or the main filename (without extension) as argument.
% First, it breaks loops.
% If the argument is not empty and does not match |\childdocname|
% (which is set by the first inclusion of |childdoc.def|),
% |\ifchilddoc| is set to true, |\includeonly| is applied to the child file
% and |\jobname| is set to the main file
% (for proper handling of |.aux| files):
%    \begin{macrocode}
\newcommand{\childdocmain}[1]
{
  \childdocdisable\childdocmain{}
  \if?#1?\else
    \begingroup
      \def\childdoctmp{#1}
      \ifx\childdoctmp\childdocname
        \def\childdoctmp{}
      \else
        \def\childdoctmp
        {
          \childdoctrue
          \includeonly{\childdocname}
          \def\childdocjob{#1}
          \def\jobname{#1}
        }
      \fi
      \expandafter
    \endgroup
    \childdoctmp
  \fi
}
%    \end{macrocode}

% \macro{\childdocof}
% The command |\childdocof| redirects
% compilation to the main file |#1|.
%    \begin{macrocode}
\newcommand{\childdocof}[1]
{
  \childdocdisable
  \childdoctrue
  \includeonly{\childdocname}
  \def\jobname{#1}
  \def\childdocjob{#1}
  \input{#1}
}
%    \end{macrocode}

% \macro{\childdocby}
% The command |\childdocby| ....
%    \begin{macrocode}
\newcommand{\childdocby}[2][]
{
  \childdocdisable
  \childdoctrue
  \childdocmanualtrue
  \if?#1?\else
    \def\jobname{#2}
  \fi
  \def\childdocjob{#2}
  \input{#2}
  \endinput
}
%    \end{macrocode}

% \macro{\childdocforward}
% The command |\childdocforward| redirects
% compilation to the main file or
% (if the optional argument is given) a child file.
% Parameters are set as if the main file
% or a child file starting with |\childdocof| was compiled.
% Then compilation is handed over to the main file:
%    \begin{macrocode}
\newcommand{\childdocforward}[2][]
{
  \begingroup
    \if?#1?
      \def\childdoctmp
      {
        \def\childdocname{#2}
        \def\childdocjob{#2}
        \def\jobname{#2}
        \input{#2}
        \endinput
      }
    \else
      \def\childdoctmp
      {
        \childdocdisable
        \def\childdocname{#2}
        \childdoctrue
        \includeonly{#2}
        \def\childdocjob{#1}
        \def\jobname{#1}
        \input{#1}
        \endinput
      }
    \fi
    \expandafter
  \endgroup
  \childdoctmp
}
%    \end{macrocode}

% \macro{\childdocforwardprefix}
% The command |\childdocforwardprefix| redirects
% compilation to the main or a child file by means of a pattern.
% The prefix |#1| in the current filename is replaced by |#2|
% and the suffix of the current filename is kept
% (it is assumed that the filename does not contain the substring `|~~~|'
% which is used as a delimiter).
% Compilation is handed over to the new file by |\childdocforward|:
%    \begin{macrocode}
\newcommand{\childdocforwardprefix}[3][]
{
  \begingroup
    \def\childdocextract #2##1~~~{\def\childdoctmp{\childdocforward[#1]{#3##1}}}
    \expandafter\childdocextract\childdocname~~~
    \expandafter
  \endgroup
  \childdoctmp
}
%    \end{macrocode}

% \macro{\childdoc}
% The deprecated macro |\childdoc| is a legacy version of |\childdocmain|:
%    \begin{macrocode}
\newcommand{\childdoc}{\childdocmain}
%    \end{macrocode}

% \macro{\childdocredirect}
% The deprecated macro |\childdocredirect| is a legacy version
% of |\childdocforward| and |\childdocforwardprefix|:
%    \begin{macrocode}
\newcommand{\childdocredirect}[2][]
{
  \begingroup
    \if?#1?
      \def\childdoctmp{\childdocforward{#2}}
    \else
      \def\childdoctmp{\childdocforwardprefix{#1}{#2}}
    \fi
    \expandafter
  \endgroup
  \childdoctmp
}
%    \end{macrocode}

%\iffalse
%</package>
%\fi
%
\endinput

\childdocby{cdocsamp}
%    \end{macrocode}

%\iffalse
%</samplepart3|samplepart4>
%\fi
%
%\iffalse
%<*samplepart3>
%\fi
% Some text for part 3:
%    \begin{macrocode}
some text in part three
%    \end{macrocode}

%\iffalse
%</samplepart3>
%\fi
% Some text for part 4:
%\iffalse
%<*samplepart4>
%\fi
%    \begin{macrocode}
more text in part four
%    \end{macrocode}

%\iffalse
%</samplepart4>
%\fi
%
% %%%%%%%%%%%%%%%%%%%%%%%%%%%%%%%%%%%%%%
% \paragraph{Forwarding for a Complete Draft.}
%
% The following forwarding file |cdocsdrf.tex|
% compiles the main document in draft mode:
%\iffalse
%<*sampledraft>
%\fi
%    \begin{macrocode}
\def\version{draft}
% \iffalse
%
% childdoc.dtx Copyright (C) 2017-2018 Niklas Beisert
%
% This work may be distributed and/or modified under the
% conditions of the LaTeX Project Public License, either version 1.3
% of this license or (at your option) any later version.
% The latest version of this license is in
%   http://www.latex-project.org/lppl.txt
% and version 1.3 or later is part of all distributions of LaTeX
% version 2005/12/01 or later.
%
% This work has the LPPL maintenance status `maintained'.
%
% The Current Maintainer of this work is Niklas Beisert.
%
% This work consists of the files childdoc.dtx and childdoc.ins
% and the derived files childdoc.def and cdocsamp.tex with
% cdocsch1.tex, cdocsch2.tex, cdocsdrf.tex, cdocsfn1.tex, cdocsfn2.tex.
%
%<package>\ifdefined\childdocmain\endinput\fi
%<package>\ProvidesFile{childdoc.def}[2018/12/30 v2.0 child document driver]
%<samplemain>\ProvidesFile{cdocsamp.tex}[2018/12/30 v2.0 sample for childdoc]
%<*driver>
%\ProvidesFile{childdoc.drv}[2018/12/30 v2.0 childdoc reference manual file]
\PassOptionsToClass{10pt,a4paper}{article}
\documentclass{ltxdoc}

\usepackage[margin=35mm]{geometry}
\usepackage{hyperref}
\usepackage{hyperxmp}
\usepackage[usenames]{color}

\hypersetup{colorlinks=true}
\hypersetup{pdfstartview=FitH}
\hypersetup{pdfpagemode=UseNone}
\hypersetup{pdfsource={}}
\hypersetup{pdflang={en-UK}}
\hypersetup{pdfcopyright={Copyright 2017-2018 Niklas Beisert.
  This work may be distributed and/or modified under the
  conditions of the LaTeX Project Public License, either version 1.3
  of this license or (at your option) any later version.}}
\hypersetup{pdflicenseurl={http://www.latex-project.org/lppl.txt}}
\hypersetup{pdfcontactaddress={ETH Zurich, ITP, HIT K,
  Wolfgang-Pauli-Strasse 27}}
\hypersetup{pdfcontactpostcode={8093}}
\hypersetup{pdfcontactcity={Zurich}}
\hypersetup{pdfcontactcountry={Switzerland}}
\hypersetup{pdfcontactemail={nbeisert@itp.phys.ethz.ch}}
\hypersetup{pdfcontacturl={http://people.phys.ethz.ch/\xmptilde nbeisert/}}

\newcommand{\secref}[1]{\hyperref[#1]{section \ref*{#1}}}

\parskip1ex
\parindent0pt
\let\olditemize\itemize
\def\itemize{\olditemize\parskip0pt}

\begin{document}

\title{The \textsf{childdoc} Package}
\hypersetup{pdftitle={The childdoc Package}}
\author{Niklas Beisert\\[2ex]
  Institut f\"ur Theoretische Physik\\
  Eidgen\"ossische Technische Hochschule Z\"urich\\
  Wolfgang-Pauli-Strasse 27, 8093 Z\"urich, Switzerland\\[1ex]
  \href{mailto:nbeisert@itp.phys.ethz.ch}
  {\texttt{nbeisert@itp.phys.ethz.ch}}}
\hypersetup{pdfauthor={Niklas Beisert}}
\hypersetup{pdfsubject={Manual for the LaTeX2e Package childdoc}}
\date{30 December 2018, \textsf{v2.0}}
\maketitle

\begin{abstract}\noindent
\textsf{childdoc} is a \LaTeXe{} package
that enables the direct compilation
of document sections included by |\include|
to individual files.
\end{abstract}

\begingroup
\parskip0ex
\tableofcontents
\endgroup

%%%%%%%%%%%%%%%%%%%%%%%%%%%%%%%%%%%%%%%%%%%%%%%%%%%%%%%%%%%%%%%%%%%%%%%%%%%%%%%%
%%%%%%%%%%%%%%%%%%%%%%%%%%%%%%%%%%%%%%%%%%%%%%%%%%%%%%%%%%%%%%%%%%%%%%%%%%%%%%%%
\section{Introduction}

\LaTeX{} provides a mechanism to structure a large document (such as a book)
into a main file and several child files (containing the chapters)
using the |\include| command.
This mechanism is beneficial for documents
which span hundreds of pages in order to
make the source file(s) more manageable.
Moreover, compilation can be restricted to
selected child files by means of the |\includeonly| command.
The latter feature can be used to reduce the compilation time while editing
(this was significantly more useful in the earlier days of \LaTeX{})
or to generate a smaller document which is easier to navigate.
Another application of |\includeonly| is to generate
documents consisting of selected parts of the complete document.

However, there are a few drawbacks of the plain |\include| mechanism:
\begin{itemize}
\item
The child files cannot be compiled on their own,
they can only be compiled via the main file.
A naive editing environment
(such as a text editor with an option
to have the current file processed by \LaTeX)
may require one to switch to the main file before compiling;
attempting to compile the child file produces errors.
\item
The main file must be modified (each time)
to adjust the |\includeonly| command
to the present needs. This easily leaves the main file in a messy state.
\item
The generated document will always carry the filename
of the main document. This is inconvenient if
several child files are to be compiled and
to be kept for distribution.
\end{itemize}

The present package provides a simple interface
to make child files individually compilable by \LaTeX{}.
Compiling a child file then has the same effect as compiling
the main file with an |\includeonly| command
to select the appropriate child.
Moreover the generated document will carry the name of the child
rather than the main file.
This resolves all three above issues.

This feature is meant to make the editing of books,
thesis documents and lecture notes somewhat more convenient.
However, the package can also be used efficiently for
composing a series of documents (such as exercise sheets)
which are typically distributed individually.
It then assists the author in generating the individual documents
(potentially in different versions)
as well as a document containing the collected series.
Another application is in developing style files
or other kinds of included material
where compilation of the style file could redirect
to a sample or test file.

%%%%%%%%%%%%%%%%%%%%%%%%%%%%%%%%%%%%%%%%%%%%%%%%%%%%%%%%%%%%%%%%%%%%%%%%%%%%%%%%
%%%%%%%%%%%%%%%%%%%%%%%%%%%%%%%%%%%%%%%%%%%%%%%%%%%%%%%%%%%%%%%%%%%%%%%%%%%%%%%%
\section{Usage}

First of all, the package \textsf{childdoc} is \emph{not} a standard
\LaTeXe{} |.sty| style file! Therefore it needs to be invoked in
a non-standard way.

%%%%%%%%%%%%%%%%%%%%%%%%%%%%%%%%%%%%%%%%%%%%%%%%%%%%%%%%%%%%%%%%%%%%%%%%%%%%%%%%
\subsection{Included Files}
\label{sec:include}

%%%%%%%%%%%%%%%%%%%%%%%%%%%%%%%%%%%%%%%%
\DescribeMacro{\childdocmain}
To use the package, add the commands
\begin{center}
\begin{tabular}{l}
|\input{childdoc.def}|\\
|\childdocmain{}|\\
\end{tabular}
\end{center}
at the very top of the main \LaTeX{} file,
in particular \emph{before} the |\documentclass| statement!
The argument of |\childdocmain| should be left empty
(but it must be present).

%%%%%%%%%%%%%%%%%%%%%%%%%%%%%%%%%%%%%%%%
\DescribeMacro{\childdocof}
Furthermore, add the commands
\begin{center}
\begin{tabular}{l}
|\input{childdoc.def}|\\
|\childdocof{|\textit{main}|}|\\
\end{tabular}
\end{center}
at the top of every child file \textit{child}
which is included by |\include{|\textit{child}|}|
from within the main file
(or at least for those files to be compiled individually).
The argument \textit{main} must be the filename of the main file.

There are a couple of
considerations in setting up the main and child documents:

%%%%%%%%%%%%%%%%%%%%%%%%%%%%%%%%%%%%%%%%
\paragraph{Restrictions.}

Please note the following restrictions:
\begin{itemize}
\item
|\childdocmain| must be called with one argument \textit{main}
to ensure compatibility with earlier version of the package.
It must either be empty (|\childdocmain{}|)
or precisely match the filename of the main file in which it is specified.
See \secref{sec:detection} for further information.
\item
The filename \textit{main} must be specified without the |.tex| extension.
\item
The filename \textit{main} is case sensitive
(even in case-insensitive file systems)
due to internal string comparison.
\item
The argument \textit{main} should be fully expanded, it cannot be a macro.
\item
Subdirectories and special characters should be avoided in filenames.
\item
The command |\childdocmain{|\textit{main}|}| must be followed by a whitespace.
It should not be followed immediately by another command
or by a comment mark `|%|'.
This is because the \TeX{} parser reads the token immediately following
the argument of |\childdocmain| and puts it
at the beginning of every child section;
however, a white\-space is ignored.
\end{itemize}

%%%%%%%%%%%%%%%%%%%%%%%%%%%%%%%%%%%%%%%%
\paragraph{Content of Main File.}

It is advisable to place all content in the child files included by |\include|.
Any output contained in the main file will appear in all child documents
unless suppressed manually;
it cannot be suppressed automatically by the |\includeonly| directive
and thus should normally be avoided.
A method to include some content in the main file
by means of conditional processing is described in \secref{sec:conditional}.

%%%%%%%%%%%%%%%%%%%%%%%%%%%%%%%%%%%%%%%%
\paragraph{Page Numbering.}

When only a part of the document is compiled,
the appropriate numbering of pages
(as well as other status parameters)
is determined from the |.aux| files.
The latter contain information from previous passes.
However this information needs to propagate through
all intermediate child documents.
Therefore the page numbering in child documents may well
be inconsistent until the complete document is compiled at least once.

A useful (if unconventional) way to always ensure a consistent
page numbering is to restart the numbering in each child document
and denote the pages by `\textit{child}|.|\textit{page}'
where \textit{child} represents the chapter/section number of the child file.
This can be achieved by the command
|\numberwithin{page}{|\textit{child}|}|
of the \textsf{amsmath} package
where \textit{child} can be |chapter| or |section|
depending on the chosen structuring.
Alternatively, one can modify the macro |\thepage| appropriately
and reset the counter |page| at the start of each child file.

%%%%%%%%%%%%%%%%%%%%%%%%%%%%%%%%%%%%%%%%%%%%%%%%%%%%%%%%%%%%%%%%%%%%%%%%%%%%%%%%
\subsection{Conditional Processing}
\label{sec:conditional}

The package provides a mechanism to compile different versions
of a document. To customise the versions further some conditional processing
can come in handy to distinguish which version is being compiled.
The package provides two macros to describe the compilation context:

%%%%%%%%%%%%%%%%%%%%%%%%%%%%%%%%%%%%%%%%
\DescribeMacro{\ifchilddoc}
The conditional |\ifchilddoc| distinguishes between the compilation of
child documents and the main document:
%
\begin{center}
|\ifchilddoc |\textit{child-code}| |[|\||else |\textit{main-code}]| \||fi|
\end{center}

%%%%%%%%%%%%%%%%%%%%%%%%%%%%%%%%%%%%%%%%
\DescribeMacro{\childdocname}
\DescribeMacro{\childdocjob}
The macro |\childdocname| contains the filename (without extension)
of the main or child file being processed.
Note that |\childdocjob| will always contain the name of the main file.

%%%%%%%%%%%%%%%%%%%%%%%%%%%%%%%%%%%%%%%%
\paragraph{Title Page.}

Conditional processing can be used to include a title or banner page
in the main document when proper precautions are taken.
Importantly, the code in the main file should ensure that the page counter
(as well as other status parameters which are stored in the |.aux| files)
takes the same value after the conditional processing.
Otherwise the page numbers may take divergent values
depending on which part is compiled.

For example, a title page could be declared by:
%
\begin{center}
\begin{tabular}{l}
|\ifchilddoc\||else|\\
|\addtocounter{page}{-1}|\\
\textit{code for title page}\\
|\newpage|\\
|\||fi|
\end{tabular}
\end{center}
%
A banner page for the child documents can be generated by:
%
\begin{center}
\begin{tabular}{l}
|\ifchilddoc|\\
|\addtocounter{page}{-1}|\\
\textit{code for banner page}\\
|\newpage|\\
|\||fi|
\end{tabular}
\end{center}
%
Here one could write a message such as:
\begin{center}
|This is the part \childdocname{} of \childdocjob{}.|
\end{center}

%%%%%%%%%%%%%%%%%%%%%%%%%%%%%%%%%%%%%%%%%%%%%%%%%%%%%%%%%%%%%%%%%%%%%%%%%%%%%%%%
\subsection{Flags}
\label{sec:flags}

The package makes it easy to generate different versions
of the main or child documents.
To this end compilation flags can be defined
and assigned different default values.
They will be particularly useful in conjunction
with the forwarding mechanism described in \secref{sec:forward}.

For example, it may be useful to have a flag |\version|
which can be set to |draft| or |final|.
The document source will contain some conditional code
depending on the value of |\version|.
Suppose further, the flag should default to |final| for the main file
and to |draft| for child files
which is a natural assignment for editing the document.
This is achieved by placing the following code
in the preamble of the main document
(below the |\childdocmain| directive):
%
\begin{center}
\begin{tabular}{l}
|\ifchilddoc|\\
|\providecommand{\version}{draft}|\\
|\||else|\\
|\providecommand{\version}{final}|\\
|\||fi|
\end{tabular}
\end{center}
%
The definition by |\providecommand| makes sure
that previous definitions are not overwritten.
Further statements |\providecommand{\version}{...}|
can thus be added before the above code to override it.

For the main file, one might add a line
(between |\childdocmain| and the above block)
%
\begin{center}
|%\ifchilddoc\||else\providecommand{\version}{draft}\||fi|
\end{center}
%
which can be uncommented to produce a draft version.
Likewise one can add a line to the very top of a child file
(above the |\childdocof{|\textit{main}|}| directive)
%
\begin{center}
|%\providecommand{\version}{final}|
\end{center}
%
which can be uncommented to produce the final version of this child document.

%%%%%%%%%%%%%%%%%%%%%%%%%%%%%%%%%%%%%%%%%%%%%%%%%%%%%%%%%%%%%%%%%%%%%%%%%%%%%%%%
\subsection{Forwarding}
\label{sec:forward}

Different versions of the main or child documents
using compilation flags as described in \secref{sec:flags}
can be (permanently) stored in different files
for convenient compilation, viewing and distribution.
To this end, the package defines a command
to pass on compilation to a different file:

%%%%%%%%%%%%%%%%%%%%%%%%%%%%%%%%%%%%%%%%
\DescribeMacro{\childdocforward}
The command |\childdocforward| redirects processing to
another source file:
%
\begin{center}
\begin{tabular}{l}
|\input{childdoc.def}|\\
|\childdocforward[|\textit{main}|]{|\textit{dest}|}|\\
\end{tabular}
\end{center}
%
The argument \textit{dest} is the destination file
(without extension).
It should be the main file or one of the child files.
Note that further \textsf{childdoc} directives
such as |\childdocof| and |\childdocforward|
in the indicated file will be processed in this form.
The optional argument \textit{main}
passes on directly to the main file \textit{main}
while pretending to compile the child \textit{dest}.
This form behaves as if \textit{dest}
issues |\childdocof{|\textit{main}|}| right away,
and no further \textsf{childdoc} directives will be processed.

%%%%%%%%%%%%%%%%%%%%%%%%%%%%%%%%%%%%%%%%
\DescribeMacro{\...prefix}
In the alternative form |\childdocforwardprefix|,
%
\begin{center}
\begin{tabular}{l}
|\input{childdoc.def}|\\
|\childdocforwardprefix[|\textit{main}|]{|\textit{prefix}|}{|\textit{dest}|}|
\end{tabular}
\end{center}
%
the destination file is determined by a pattern
depending on the current file:
To make this work, the current file must be called
`{\textit{prefix}\hspace{0.2em}\textit{suffix}}'
with \textit{prefix} matching precisely the argument.
Processing is then passed on to the file
`{\textit{dest}\hspace{0.2em}\textit{suffix}}'.
Surely, the same effect is achieved by
directly specifying the
argument `{\textit{dest}\hspace{0.2em}\textit{suffix}}'
in the first form.
However, that requires to set up a different file
for each child. With the alternative form of the command
all these files can have exactly the same content
which simplifies setting them up and maintaining them.

For example, the following file |draft.tex|
with a compilation flag |\version| as described in \secref{sec:flags}
compiles the main document as a draft:
%
\begin{center}
\begin{tabular}{l}
|\def\version{draft}|\\
|\input{childdoc.def}|\\
|\childdocforward{|\textit{main}|}|
\end{tabular}
\end{center}
%
Likewise, the following files |final|\textit{nn}|.tex|
compile the final version of the child document
|child|\textit{nn}|.tex|:
%
\begin{center}
\begin{tabular}{l}
|\def\version{final}|\\
|\input{childdoc.def}|\\
|\childdocforwardprefix{final}{child}|
\end{tabular}
\end{center}
%

Note that when several versions of a main file and/or of each child file
are to be generated, it may be convenient to set up a |Makefile| or
shell script to automatise the process.

%%%%%%%%%%%%%%%%%%%%%%%%%%%%%%%%%%%%%%%%%%%%%%%%%%%%%%%%%%%%%%%%%%%%%%%%%%%%%%%%
\subsection{Command Line Processing}
\label{sec:commandline}

The effect of redirection files can also be achieved by invoking
the \LaTeX{} compiler with a more elaborate command line.
Most conveniently this should be done as part
of a shell script or a |Makefile|.

When using \textsf{childdoc} in the main file, the following
command lines effectively perform a redirection
(note that depending on the shell being used,
backslashes may have to be doubled: `|\|' $\to$ `|\\|'):
%
\begin{center}
|... -jobname "|\textit{target}|" |\\|"|[\textit{flags}]%
|\input{childdoc.def}\childdocforward[|\textit{main}|]{|\textit{dest}|}"|
\end{center}
%
Here \textit{target} is the name of the output file,
\textit{main} is the name of the main file
and \textit{dest} is the name of the main or child file to be processed
(all filenames without extensions).
The optional argument \textit{main} can be omitted
if \textit{main} matches \textit{dest}.
Optionally, compilation \textit{flags} can be defined via |\def| commands.
This command line makes the \TeX{} engine believe
it is compiling the file \textit{target}
whose content is specified as the latter parameter.
The provided code then forwards the processing to
\textit{main} or \textit{dest} as described in \secref{sec:forward}.

%%%%%%%%%%%%%%%%%%%%%%%%%%%%%%%%%%%%%%%%%%%%%%%%%%%%%%%%%%%%%%%%%%%%%%%%%%%%%%%%
\subsection{Include by Input}
\label{sec:input}

Including child documents by |\include| has some restrictions by design.
Most notably, the content of a child document always occupies
its own set of pages; pages cannot be shared between child documents.
Usually, this behaviour makes perfect sense
because each child document contain an essential part of the document.
However, in some situations it may be desirable to compose
a document from a collection of parts
without having mandatory page breaks between then.
For this case, the package
provides a mechanism to include parts
by |\input| which can also be processed individually.
However, by construction this mechanism
requires manual handling of the content to be output.

%%%%%%%%%%%%%%%%%%%%%%%%%%%%%%%%%%%%%%%%
\DescribeMacro{\ifchilddocmanual}
The main file should be prepared as usual, see \secref{sec:include}.
However, the document body must make a distinction
between processing of an individual part and of the main document, e.g.:
%
\begin{center}
\begin{tabular}{l}
|\ifchilddocmanual|\\
|\input{\childdocname}|\\
|\||else|\\
\textit{document body with }|\input{|\textit{part}|}|\\
|\||fi|
\end{tabular}
\end{center}
%
The conditional |\ifchilddocmanual| is true whenever
a part to be included by |\input| is being compiled,
and the name of the part is stored in |\childdocname|.

%%%%%%%%%%%%%%%%%%%%%%%%%%%%%%%%%%%%%%%%
\DescribeMacro{\childdocby}
Each part to be included by |\input| should start with:
%
\begin{center}
\begin{tabular}{l}
|\input{childdoc.def}|\\
|\childdocby{|\textit{main}|}|\\
\end{tabular}
\end{center}
%
The directive |\childdocby| is similar to |\childdocof|
described in \secref{sec:include},
but the subsequent selection of content must be done manually.
To that end, both |\ifchilddoc| and |\ifchilddocmanual|
will be true upon processing of a part,
and the name of the part is stored in |\childdocname|.
Note that |\jobname| will be set to the filename of the current part
so that each part receives an individual |.aux| file
that does not interfere with the |.aux| file(s) of the main document.
This behaviour can be altered by the alternative form
|\childdocby[*]{|\textit{main}|}| (with a non-empty optional argument)
which uses the |.aux| file of the main document
by setting |\jobname| to \textit{main}.

%%%%%%%%%%%%%%%%%%%%%%%%%%%%%%%%%%%%%%%%%%%%%%%%%%%%%%%%%%%%%%%%%%%%%%%%%%%%%%%%
\subsection{Driver Development}
\label{sec:driver}

The \textsf{childdoc} mechanism can also be use for the development
of definition files such as \LaTeX{} styles or classes.
This case differs from the above setup with multiple parts
included by |\include| in that no |\includeonly| should be invoked.
This can be achieved by starting the include file
(before |\ProvidesPackage|) with:
%
\begin{center}
\begin{tabular}{l}
|\input{childdoc.def}|\\
|\childdocforward{|\textit{main}|}|\\
\end{tabular}
\end{center}
%
or alternatively with:
%
\begin{center}
\begin{tabular}{l}
|\input{childdoc.def}|\\
|\childdocby{|\textit{main}|}|\\
\end{tabular}
\end{center}
%
Both forms have slightly different effects as described above.
The main file is prepared as usual, see \secref{sec:include}.

%%%%%%%%%%%%%%%%%%%%%%%%%%%%%%%%%%%%%%%%%%%%%%%%%%%%%%%%%%%%%%%%%%%%%%%%%%%%%%%%
\subsection{Legacy Detection}
\label{sec:detection}

The directive |\childdocmain| in the main file can detect
whether the complete document or merely a child is to be compiled
even without using the directive |\childdocof|.
This method is deprecated because it is less robust
and there is no compelling reason to use it;
it is merely provided for backward compatibility
and it may be removed in future versions.

If the detection mechanism is to be used,
it is mandatory to correctly specify
the filename of the main file as the argument of |\childdocmain|:
%
\begin{center}
\begin{tabular}{l}
|\input{childdoc.def}|\\
|\childdocmain{|\textit{main}|}|\\
\end{tabular}
\end{center}
%
If |\jobname| does not match the argument \textit{main} of |\childdocmain|,
it is assumed that |\jobname| points to the child file to be compiled.
When using |\childdocmain| with the main file specified as argument,
it suffices to start a child file
with just |\input{|\textit{main}|}|
without loading of the package and using |\childdocof|.
If instead all processing is done
with the appropriate \textsf{childdoc} directives,
the argument of \textit{main} of |\childdocmain| can be empty.

An alternative version of the command line processing described
in \secref{sec:commandline} using the detection mechanism reads:
%
\begin{center}
|... -jobname "|\textit{target}|" "|[\textit{flags}]%
[|\def\jobname{|\textit{dest}|}|]|\input{|\textit{main}|}"|
\end{center}

%%%%%%%%%%%%%%%%%%%%%%%%%%%%%%%%%%%%%%%%%%%%%%%%%%%%%%%%%%%%%%%%%%%%%%%%%%%%%%%%
\subsection{Manual Code}
\label{sec:manual}

In case one cannot be certain whether the definitions file |childdoc.def|
is installed on the target \TeX{} distribution
and one prefers not to ship it,
it is conceivable to paste a few relevant commands into the sources.

To that end, drop all statements |\input{childdoc.def}|
and perform the replacements as outlined below.
Instead of |\childdocmain{|\textit{main}|}| add the following code
to the top of the main file:
%
\begin{center}
\begin{tabular}{l}
|\||ifdefined\childdocname\endinput\||fi\newif\ifchilddoc|\\
|\edef\childdocname{\scantokens\expandafter{\jobname\noexpand}}|\\
|\def\childdocmain{|\textit{main}|}\||ifx\childdocmain\childdocname\||else|\\
|\childdoctrue\includeonly{\childdocname}\let\jobname\childdocmain\||fi|\\
\end{tabular}
\end{center}
%
Instead of |\childdocof{|\textit{main}|}| just include the main file
at the top of each child file:
%
\begin{center}
|\input{|\textit{main}|}|
\end{center}
%
A simple redirection |\childdocforward{|\textit{dest}|}| is achieved by:
%
\begin{center}
|\def\jobname{|\textit{dest}|}\input{\jobname}|
\end{center}
%
The redirection with prefix
|\childdocforwardprefix[|\textit{prefix}|]{|\textit{dest}|}|
is accomplished by:
%
\begin{center}
\begin{tabular}{l}
|{\edef\jobname{\scantokens\expandafter{\jobname\noexpand}}|\\
|\def\redirectjob |\textit{prefix}|#1~~~{\gdef\jobname{|\textit{dest}|#1}}|\\
|\expandafter\redirectjob\jobname~~~}\input{\jobname}|
\end{tabular}
\end{center}

In an alternative approach,
child documents can be compiled by a specific command line
without additional code or specific definitions:
%
\begin{center}
|... -jobname "|\textit{target}|" "|[\textit{flags}]%
|\includeonly{|\textit{dest}|}\input{|\textit{main}|}"|
\end{center}
%

%%%%%%%%%%%%%%%%%%%%%%%%%%%%%%%%%%%%%%%%%%%%%%%%%%%%%%%%%%%%%%%%%%%%%%%%%%%%%%%%
%%%%%%%%%%%%%%%%%%%%%%%%%%%%%%%%%%%%%%%%%%%%%%%%%%%%%%%%%%%%%%%%%%%%%%%%%%%%%%%%
\section{Information}

%%%%%%%%%%%%%%%%%%%%%%%%%%%%%%%%%%%%%%%%%%%%%%%%%%%%%%%%%%%%%%%%%%%%%%%%%%%%%%%%
\subsection{Copyright}

Copyright \copyright{} 2017--2018 Niklas Beisert

This work may be distributed and/or modified under the
conditions of the \LaTeX{} Project Public License, either version 1.3
of this license or (at your option) any later version.
The latest version of this license is in
  \url{http://www.latex-project.org/lppl.txt}
and version 1.3 or later is part of all distributions of \LaTeX{}
version 2005/12/01 or later.

This work has the LPPL maintenance status `maintained'.

The Current Maintainer of this work is Niklas Beisert.

This work consists of the files |README.txt|, |childdoc.ins| and |childdoc.dtx|
as well as the derived files |childdoc.def|, |cdocsamp.tex|
with |cdocsch1.tex|, |cdocsch2.tex|, |cdocspt3.tex|, |cdocspt4.tex|,
|cdocsdrf.tex|, |cdocsfn1.tex|, |cdocsfn2.tex|
as well as |childdoc.pdf|.

%%%%%%%%%%%%%%%%%%%%%%%%%%%%%%%%%%%%%%%%%%%%%%%%%%%%%%%%%%%%%%%%%%%%%%%%%%%%%%%%
\subsection{Files and Installation}

The package consists of the files:
%
\begin{center}
\begin{tabular}{ll}
    |README.txt|   & readme file \\
    |childdoc.ins| & installation file \\
    |childdoc.dtx| & source file \\
    |childdoc.def| & definition file \\
    |cdocsamp.tex| & sample main file \\
    |cdocsch1.tex| & sample include file \\
    |cdocsch2.tex| & sample include file \\
    |cdocspt3.tex| & sample part file \\
    |cdocspt4.tex| & sample part file \\
    |cdocsdrf.tex| & sample redirection file \\
    |cdocsfn1.tex| & sample redirection file \\
    |cdocsfn2.tex| & sample redirection file \\
    |childdoc.pdf| & manual
\end{tabular}
\end{center}
%
The distribution consists of the files
|README.txt|, |childdoc.ins| and |childdoc.dtx|.
%
\begin{itemize}
\item
Run (pdf)\LaTeX{} on |childdoc.dtx|
to compile the manual |childdoc.pdf| (this file).
\item
Run \LaTeX{} on |childdoc.ins| to create the definitions file |childdoc.def|
and the sample |cdocsamp.tex| with include files
|cdocsch1.tex|, |cdocsch2.tex|, |cdocspt3.tex|, |cdocspt4.tex|,
|cdocsdrf.tex|, |cdocsfn1.tex|, |cdocsfn2.tex|.
Then copy the file |childdoc.def| to an appropriate directory of your \LaTeX{}
distribution, e.g.\ \textit{texmf-root}|/tex/latex/childdoc|.
\end{itemize}

%%%%%%%%%%%%%%%%%%%%%%%%%%%%%%%%%%%%%%%%%%%%%%%%%%%%%%%%%%%%%%%%%%%%%%%%%%%%%%%%
\subsection{Related CTAN Packages}

There are several other packages which offer a similar functionality:
%
\begin{itemize}
\item
The packages
\href{http://ctan.org/pkg/docmute}{\textsf{docmute}},
\href{http://ctan.org/pkg/includex}{\textsf{includex}} and
\href{http://ctan.org/pkg/standalone}{\textsf{standalone}}
provide commands to include only the document body of
a child file thus allowing both files to be compiled individually.
\item
The packages \href{http://ctan.org/pkg/subdocs}{\textsf{subdocs}}
and \href{http://ctan.org/pkg/subfiles}{\textsf{subfiles}}
provide structures in which the main and child documents can be
encapsulated and allowing them to be compiled individually.
The inclusion mechanism is different from the conventional |\include|.
\item
The package \href{http://ctan.org/pkg/combine}{\textsf{combine}}
is an elaborate solution to combine several documents into one.
\end{itemize}
%
See also the CTAN topic \href{http://ctan.org/topic/subdocs}{\textsf{subdocs}}
for further related packages.
The present package differs from the above solutions in that
a document structure constructed with the conventional |\include| mechanism
just needs two extra commands at the top of every file
such that all constituent files can be compiled individually.

%%%%%%%%%%%%%%%%%%%%%%%%%%%%%%%%%%%%%%%%%%%%%%%%%%%%%%%%%%%%%%%%%%%%%%%%%%%%%%%%
%\subsection{Feature Suggestions}
%
%The following is a list of features which may be useful for future
%versions of this package:
%%
%\begin{itemize}
%\item
%\ldots
%\end{itemize}

%%%%%%%%%%%%%%%%%%%%%%%%%%%%%%%%%%%%%%%%%%%%%%%%%%%%%%%%%%%%%%%%%%%%%%%%%%%%%%%%
\subsection{Revision History}

%%%%%%%%%%%%%%%%%%%%%%%%%%%%%%%%%%%%%%%%
\paragraph{v2.0:} 2018/12/30

\begin{itemize}
\item
immediate forward processing
\item
added |\childdocby| mechanism
\item
manual restructured
\end{itemize}

%%%%%%%%%%%%%%%%%%%%%%%%%%%%%%%%%%%%%%%%
\paragraph{v1.6:} 2018/01/17

\begin{itemize}
\item
application for development of include files
\item
corrections to manual
\end{itemize}

%%%%%%%%%%%%%%%%%%%%%%%%%%%%%%%%%%%%%%%%
\paragraph{v1.5:} 2017/05/21

\begin{itemize}
\item
more complete structuring introduced
\item
|\childdocof| introduced
\item
|\childdoc| renamed to |\childdocmain|
\item
|\childredirect| renamed to |\childdocforward| and |\childdocforwardprefix|
and functionality expanded
\end{itemize}

%%%%%%%%%%%%%%%%%%%%%%%%%%%%%%%%%%%%%%%%
\paragraph{v1.0:} 2017/04/27

\begin{itemize}
\item
manual and install package
\item
first version published on CTAN
\end{itemize}

%%%%%%%%%%%%%%%%%%%%%%%%%%%%%%%%%%%%%%%%
\paragraph{v0.6:} 2017/04/26

\begin{itemize}
\item
redirection mechanism added
\end{itemize}

%%%%%%%%%%%%%%%%%%%%%%%%%%%%%%%%%%%%%%%%
\paragraph{v0.5:} 2017/04/26

\begin{itemize}
\item
functionality in definition file
\end{itemize}


%%%%%%%%%%%%%%%%%%%%%%%%%%%%%%%%%%%%%%%%%%%%%%%%%%%%%%%%%%%%%%%%%%%%%%%%%%%%%%%%
%%%%%%%%%%%%%%%%%%%%%%%%%%%%%%%%%%%%%%%%%%%%%%%%%%%%%%%%%%%%%%%%%%%%%%%%%%%%%%%%
%%%%%%%%%%%%%%%%%%%%%%%%%%%%%%%%%%%%%%%%%%%%%%%%%%%%%%%%%%%%%%%%%%%%%%%%%%%%%%%%
\appendix

\settowidth\MacroIndent{\rmfamily\scriptsize 000\ }

 \DocInput{childdoc.dtx}

\end{document}
%</driver>
% \fi
%
% %%%%%%%%%%%%%%%%%%%%%%%%%%%%%%%%%%%%%%%%%%%%%%%%%%%%%%%%%%%%%%%%%%%%%%%%%%%%%%
% %%%%%%%%%%%%%%%%%%%%%%%%%%%%%%%%%%%%%%%%%%%%%%%%%%%%%%%%%%%%%%%%%%%%%%%%%%%%%%
% \section{Sample}
%\iffalse
%<*samplemain>
%\fi
%
% The following presents a sample document
% with two chapters, two parts, a title page,
% a compile flag as well as three forwarding files to set the flag.
% It consists of eight |.tex| files:
% \begin{center}
% \begin{tabular}{ll}
% |cdocsamp.tex|&main file\\
% |cdocsch1.tex|&include file for chapter 1\\
% |cdocsch2.tex|&include file for chapter 2\\
% |cdocspt3.tex|&include file for part 3\\
% |cdocspt4.tex|&include file for part 4\\
% |cdocsdrf.tex|&forwarding file for main file in draft mode\\
% |cdocsfi1.tex|&forwarding file for final version of chapter 1\\
% |cdocsfi2.tex|&forwarding file for final version of chapter 2\\
% \end{tabular}
% \end{center}
% Each of the eight files can be compiled directly by the \LaTeX{} compiler.
%
% %%%%%%%%%%%%%%%%%%%%%%%%%%%%%%%%%%%%%%
% \paragraph{Main File.}
%
% The main file is called |cdocsamp.tex|.
%
% Load the \textsf{childdoc} definitions and
% declare the filename for the main document:
%    \begin{macrocode}
\input{childdoc.def}
\childdocmain{}
%    \end{macrocode}

% Optional override for |\version| flag:
%    \begin{macrocode}
%%\ifchilddoc\else\providecommand{\version}{draft}\fi
%    \end{macrocode}

% Define the default values for the |\version| flag
% (|final| for the main file and |draft| for childs):
%    \begin{macrocode}
\ifchilddoc
\providecommand{\version}{draft}
\else
\providecommand{\version}{final}
\fi
%    \end{macrocode}

% Load the standard document class:
%    \begin{macrocode}
\documentclass[12pt]{article}
%    \end{macrocode}

% Start the document body:
%    \begin{macrocode}
\begin{document}
%    \end{macrocode}

% Declare a title page.
% Print title, part of document being processed and version flag:
%    \begin{macrocode}
\addtocounter{page}{-1}
\begin{center}
{\LARGE\bfseries{}childdoc example\par}
\vspace{1cm}
\ifchilddoc
\ifchilddocmanual part\else chapter\fi:
`\childdocname' of `\childdocjob'\par
\else
main document: `\childdocjob'\par
\fi
version: \version\par
\end{center}
\newpage
%    \end{macrocode}

% Manually include selected file,
% otherwise process as usual:
%    \begin{macrocode}
\ifchilddocmanual
\section*{part `\childdocname'}
\input{\childdocname}
\else
%    \end{macrocode}

% Include the two chapters:
%    \begin{macrocode}
\include{cdocsch1}
\include{cdocsch2}
%    \end{macrocode}

% Include the two parts unless only chapters should be displayed:
%    \begin{macrocode}
\ifchilddoc\else
\section{part three}
\input{cdocspt3}
\section{part four}
\input{cdocspt4}
\fi
%    \end{macrocode}

% Process as usual until here:
%    \begin{macrocode}
\fi
%    \end{macrocode}

% End of document body:
%    \begin{macrocode}
\end{document}
%    \end{macrocode}
%\iffalse
%</samplemain>
%\fi
%
% %%%%%%%%%%%%%%%%%%%%%%%%%%%%%%%%%%%%%%
% \paragraph{Chapter Include Files.}
%
% The include files are called |cdocsch1.tex| and |cdocsch2.tex|.
%
%\iffalse
%<*samplechap1|samplechap2>
%\fi

% Optional override for |\version| flag:
%    \begin{macrocode}
%%\providecommand{\version}{final}
%    \end{macrocode}

% Include the main document:
%    \begin{macrocode}
\input{childdoc.def}
\childdocof{cdocsamp}
%    \end{macrocode}

%\iffalse
%</samplechap1|samplechap2>
%\fi
%
%\iffalse
%<*samplechap1>
%\fi
% Some text for chapter 1:
%    \begin{macrocode}
\section{one}
some text in chapter one
%    \end{macrocode}

%\iffalse
%</samplechap1>
%\fi
% Some text for chapter 2:
%\iffalse
%<*samplechap2>
%\fi
%    \begin{macrocode}
\section{two}
more text in chapter two
%    \end{macrocode}

%\iffalse
%</samplechap2>
%\fi
%
% %%%%%%%%%%%%%%%%%%%%%%%%%%%%%%%%%%%%%%
% \paragraph{Part Include Files.}
%
% The include files are called |cdocspt3.tex| and |cdocspt4.tex|.
%
%\iffalse
%<*samplepart3|samplepart4>
%\fi

% Optional override for |\version| flag:
%    \begin{macrocode}
%%\providecommand{\version}{final}
%    \end{macrocode}

% Include the main document:
%    \begin{macrocode}
\input{childdoc.def}
\childdocby{cdocsamp}
%    \end{macrocode}

%\iffalse
%</samplepart3|samplepart4>
%\fi
%
%\iffalse
%<*samplepart3>
%\fi
% Some text for part 3:
%    \begin{macrocode}
some text in part three
%    \end{macrocode}

%\iffalse
%</samplepart3>
%\fi
% Some text for part 4:
%\iffalse
%<*samplepart4>
%\fi
%    \begin{macrocode}
more text in part four
%    \end{macrocode}

%\iffalse
%</samplepart4>
%\fi
%
% %%%%%%%%%%%%%%%%%%%%%%%%%%%%%%%%%%%%%%
% \paragraph{Forwarding for a Complete Draft.}
%
% The following forwarding file |cdocsdrf.tex|
% compiles the main document in draft mode:
%\iffalse
%<*sampledraft>
%\fi
%    \begin{macrocode}
\def\version{draft}
\input{childdoc.def}
\childdocforward{cdocsamp}
%    \end{macrocode}

%\iffalse
%</sampledraft>
%\fi
%
% %%%%%%%%%%%%%%%%%%%%%%%%%%%%%%%%%%%%%%
% \paragraph{Forwarding for Final Version of the Chapters.}
%
% The following forwarding files |cdocsfn1.tex| and |cdocsfn2.tex|
% (with identical content)
% compile the final versions of the child documents
% |cdocsch1.tex| and |cdocsch2.tex|, respectively:
%\iffalse
%<*samplefinal>
%\fi
%    \begin{macrocode}
\def\version{final}
\input{childdoc.def}
\childdocforwardprefix[cdocsamp]{cdocsfn}{cdocsch}
%    \end{macrocode}

%\iffalse
%</samplefinal>
%\fi
%
% %%%%%%%%%%%%%%%%%%%%%%%%%%%%%%%%%%%%%%
% \paragraph{Command Line Processing.}
%
% The following three command lines generate the output files
% |cdocscld|, |cdocscl1| and |cdocscl2|
% which should be identical to
% |cdocsdrf|, |cdocsch1| and |cdocsfn2|, respectively:
% \begin{center}
% \begin{tabular}{l}
% |latex -jobname cdocscld \|\\
% |  "\def\version{draft}\input{childdoc.def}\childdocforward{cdocsamp}"|\\
% |latex -jobname cdocscl1 \|\\
% |  "\input{childdoc.def}\childdocforward[cdocsamp]{cdocsch1}"|\\
% |latex -jobname cdocscl2 \|\\
% |  "\def\version{final}\input{childdoc.def}\childdocforward{cdocsch2}"|
% \end{tabular}
% \end{center}
% Note that the trailing backslash on each first line
% merely continues the input to the second line
% (for convenient cut ant paste).
% Furthermore, the command |latex| can be replaced by any
% of its alternative versions such as |pdflatex|.
%
% %%%%%%%%%%%%%%%%%%%%%%%%%%%%%%%%%%%%%%%%%%%%%%%%%%%%%%%%%%%%%%%%%%%%%%%%%%%%%%
% %%%%%%%%%%%%%%%%%%%%%%%%%%%%%%%%%%%%%%%%%%%%%%%%%%%%%%%%%%%%%%%%%%%%%%%%%%%%%%
% \section{Implementation}
%\iffalse
%<*package>
%\fi
%
% This section describes the definitions file |childdoc.def|.

% The definitions cannot be loaded using |\usepackage| or |\RequirePackage|
% which has a mechanism to prevent loading a style file more than once.
% When loading the definitions by means of |\input|
% multiple instances have to be prevented manually:
%\iffalse
%This code needs to be before the `\ProvidesFile' directive
%which is defined at the beginning of this file.
%Therefore it is also placed there and commented out here.
%</package>
%<*discard>
%\fi
%    \begin{macrocode}
\ifdefined\childdocmain\endinput\fi
%    \end{macrocode}
%\iffalse
%</discard>
%<*package>
%\fi
%
% \macro{\ifchilddoc}
% \macro{\ifchilddocmanual}
% The conditional |\ifchilddoc| tells whether a
% child (true) or main (false) document is being compiled.
% The conditional |\ifchilddocmanual| tells whether
% the |\includeonly| mechanism is used (false) or
% the selection of child files must be performed manually (true).
% The definitions initialise to false:
%    \begin{macrocode}
\newif\ifchilddoc
\newif\ifchilddocmanual
%    \end{macrocode}

% \macro{\childdocname}
% \macro{\childdocjob}
% The macro |\childdocname| stores the name of the main document
% to be compiled. The macro |\childdocjob| stores the name of
% the document on which the \LaTeX{} compiler was originally invoked.
% The content of |\jobname| cannot be compared
% to filenames specified in the source due to different catcodes.
% The following code rescans |\jobname|, stores the result
% in |\childdocname| and saves a copy in |\childdocjob|:
%    \begin{macrocode}
\edef\childdocname{\scantokens\expandafter{\jobname\noexpand}}
\let\childdocjob\childdocname
%    \end{macrocode}

% \macro{\childdocdisable}
% The macro |\childdocdisable| prevents the main file
% from being processed more than once.
% At this stage, the main document command |\childdocmain|
% is assumed to be called once again where it should do nothing.
% Any subsequent call to it should prevent
% a secondary processing of the main document
% It overwrites the forwarding commands
% |\childdocof| and |\childdocforward|
% with empty macros to prevent further inclusions of the main document:
%    \begin{macrocode}
\newcommand{\childdocdisable}
{
  \renewcommand{\childdocmain}[1]{\renewcommand{\childdocmain}[1]{\endinput}}
  \renewcommand{\childdocof}[1]{}
  \renewcommand{\childdocby}[2][]{}
  \renewcommand{\childdocforward}[2][]{}
  \renewcommand{\childdocdisable}{}
}
%    \end{macrocode}

% \macro{\childdocmain}
% The macro |\childdocmain| is to be called at the top of the main file
% with nothing or the main filename (without extension) as argument.
% First, it breaks loops.
% If the argument is not empty and does not match |\childdocname|
% (which is set by the first inclusion of |childdoc.def|),
% |\ifchilddoc| is set to true, |\includeonly| is applied to the child file
% and |\jobname| is set to the main file
% (for proper handling of |.aux| files):
%    \begin{macrocode}
\newcommand{\childdocmain}[1]
{
  \childdocdisable\childdocmain{}
  \if?#1?\else
    \begingroup
      \def\childdoctmp{#1}
      \ifx\childdoctmp\childdocname
        \def\childdoctmp{}
      \else
        \def\childdoctmp
        {
          \childdoctrue
          \includeonly{\childdocname}
          \def\childdocjob{#1}
          \def\jobname{#1}
        }
      \fi
      \expandafter
    \endgroup
    \childdoctmp
  \fi
}
%    \end{macrocode}

% \macro{\childdocof}
% The command |\childdocof| redirects
% compilation to the main file |#1|.
%    \begin{macrocode}
\newcommand{\childdocof}[1]
{
  \childdocdisable
  \childdoctrue
  \includeonly{\childdocname}
  \def\jobname{#1}
  \def\childdocjob{#1}
  \input{#1}
}
%    \end{macrocode}

% \macro{\childdocby}
% The command |\childdocby| ....
%    \begin{macrocode}
\newcommand{\childdocby}[2][]
{
  \childdocdisable
  \childdoctrue
  \childdocmanualtrue
  \if?#1?\else
    \def\jobname{#2}
  \fi
  \def\childdocjob{#2}
  \input{#2}
  \endinput
}
%    \end{macrocode}

% \macro{\childdocforward}
% The command |\childdocforward| redirects
% compilation to the main file or
% (if the optional argument is given) a child file.
% Parameters are set as if the main file
% or a child file starting with |\childdocof| was compiled.
% Then compilation is handed over to the main file:
%    \begin{macrocode}
\newcommand{\childdocforward}[2][]
{
  \begingroup
    \if?#1?
      \def\childdoctmp
      {
        \def\childdocname{#2}
        \def\childdocjob{#2}
        \def\jobname{#2}
        \input{#2}
        \endinput
      }
    \else
      \def\childdoctmp
      {
        \childdocdisable
        \def\childdocname{#2}
        \childdoctrue
        \includeonly{#2}
        \def\childdocjob{#1}
        \def\jobname{#1}
        \input{#1}
        \endinput
      }
    \fi
    \expandafter
  \endgroup
  \childdoctmp
}
%    \end{macrocode}

% \macro{\childdocforwardprefix}
% The command |\childdocforwardprefix| redirects
% compilation to the main or a child file by means of a pattern.
% The prefix |#1| in the current filename is replaced by |#2|
% and the suffix of the current filename is kept
% (it is assumed that the filename does not contain the substring `|~~~|'
% which is used as a delimiter).
% Compilation is handed over to the new file by |\childdocforward|:
%    \begin{macrocode}
\newcommand{\childdocforwardprefix}[3][]
{
  \begingroup
    \def\childdocextract #2##1~~~{\def\childdoctmp{\childdocforward[#1]{#3##1}}}
    \expandafter\childdocextract\childdocname~~~
    \expandafter
  \endgroup
  \childdoctmp
}
%    \end{macrocode}

% \macro{\childdoc}
% The deprecated macro |\childdoc| is a legacy version of |\childdocmain|:
%    \begin{macrocode}
\newcommand{\childdoc}{\childdocmain}
%    \end{macrocode}

% \macro{\childdocredirect}
% The deprecated macro |\childdocredirect| is a legacy version
% of |\childdocforward| and |\childdocforwardprefix|:
%    \begin{macrocode}
\newcommand{\childdocredirect}[2][]
{
  \begingroup
    \if?#1?
      \def\childdoctmp{\childdocforward{#2}}
    \else
      \def\childdoctmp{\childdocforwardprefix{#1}{#2}}
    \fi
    \expandafter
  \endgroup
  \childdoctmp
}
%    \end{macrocode}

%\iffalse
%</package>
%\fi
%
\endinput

\childdocforward{cdocsamp}
%    \end{macrocode}

%\iffalse
%</sampledraft>
%\fi
%
% %%%%%%%%%%%%%%%%%%%%%%%%%%%%%%%%%%%%%%
% \paragraph{Forwarding for Final Version of the Chapters.}
%
% The following forwarding files |cdocsfn1.tex| and |cdocsfn2.tex|
% (with identical content)
% compile the final versions of the child documents
% |cdocsch1.tex| and |cdocsch2.tex|, respectively:
%\iffalse
%<*samplefinal>
%\fi
%    \begin{macrocode}
\def\version{final}
% \iffalse
%
% childdoc.dtx Copyright (C) 2017-2018 Niklas Beisert
%
% This work may be distributed and/or modified under the
% conditions of the LaTeX Project Public License, either version 1.3
% of this license or (at your option) any later version.
% The latest version of this license is in
%   http://www.latex-project.org/lppl.txt
% and version 1.3 or later is part of all distributions of LaTeX
% version 2005/12/01 or later.
%
% This work has the LPPL maintenance status `maintained'.
%
% The Current Maintainer of this work is Niklas Beisert.
%
% This work consists of the files childdoc.dtx and childdoc.ins
% and the derived files childdoc.def and cdocsamp.tex with
% cdocsch1.tex, cdocsch2.tex, cdocsdrf.tex, cdocsfn1.tex, cdocsfn2.tex.
%
%<package>\ifdefined\childdocmain\endinput\fi
%<package>\ProvidesFile{childdoc.def}[2018/12/30 v2.0 child document driver]
%<samplemain>\ProvidesFile{cdocsamp.tex}[2018/12/30 v2.0 sample for childdoc]
%<*driver>
%\ProvidesFile{childdoc.drv}[2018/12/30 v2.0 childdoc reference manual file]
\PassOptionsToClass{10pt,a4paper}{article}
\documentclass{ltxdoc}

\usepackage[margin=35mm]{geometry}
\usepackage{hyperref}
\usepackage{hyperxmp}
\usepackage[usenames]{color}

\hypersetup{colorlinks=true}
\hypersetup{pdfstartview=FitH}
\hypersetup{pdfpagemode=UseNone}
\hypersetup{pdfsource={}}
\hypersetup{pdflang={en-UK}}
\hypersetup{pdfcopyright={Copyright 2017-2018 Niklas Beisert.
  This work may be distributed and/or modified under the
  conditions of the LaTeX Project Public License, either version 1.3
  of this license or (at your option) any later version.}}
\hypersetup{pdflicenseurl={http://www.latex-project.org/lppl.txt}}
\hypersetup{pdfcontactaddress={ETH Zurich, ITP, HIT K,
  Wolfgang-Pauli-Strasse 27}}
\hypersetup{pdfcontactpostcode={8093}}
\hypersetup{pdfcontactcity={Zurich}}
\hypersetup{pdfcontactcountry={Switzerland}}
\hypersetup{pdfcontactemail={nbeisert@itp.phys.ethz.ch}}
\hypersetup{pdfcontacturl={http://people.phys.ethz.ch/\xmptilde nbeisert/}}

\newcommand{\secref}[1]{\hyperref[#1]{section \ref*{#1}}}

\parskip1ex
\parindent0pt
\let\olditemize\itemize
\def\itemize{\olditemize\parskip0pt}

\begin{document}

\title{The \textsf{childdoc} Package}
\hypersetup{pdftitle={The childdoc Package}}
\author{Niklas Beisert\\[2ex]
  Institut f\"ur Theoretische Physik\\
  Eidgen\"ossische Technische Hochschule Z\"urich\\
  Wolfgang-Pauli-Strasse 27, 8093 Z\"urich, Switzerland\\[1ex]
  \href{mailto:nbeisert@itp.phys.ethz.ch}
  {\texttt{nbeisert@itp.phys.ethz.ch}}}
\hypersetup{pdfauthor={Niklas Beisert}}
\hypersetup{pdfsubject={Manual for the LaTeX2e Package childdoc}}
\date{30 December 2018, \textsf{v2.0}}
\maketitle

\begin{abstract}\noindent
\textsf{childdoc} is a \LaTeXe{} package
that enables the direct compilation
of document sections included by |\include|
to individual files.
\end{abstract}

\begingroup
\parskip0ex
\tableofcontents
\endgroup

%%%%%%%%%%%%%%%%%%%%%%%%%%%%%%%%%%%%%%%%%%%%%%%%%%%%%%%%%%%%%%%%%%%%%%%%%%%%%%%%
%%%%%%%%%%%%%%%%%%%%%%%%%%%%%%%%%%%%%%%%%%%%%%%%%%%%%%%%%%%%%%%%%%%%%%%%%%%%%%%%
\section{Introduction}

\LaTeX{} provides a mechanism to structure a large document (such as a book)
into a main file and several child files (containing the chapters)
using the |\include| command.
This mechanism is beneficial for documents
which span hundreds of pages in order to
make the source file(s) more manageable.
Moreover, compilation can be restricted to
selected child files by means of the |\includeonly| command.
The latter feature can be used to reduce the compilation time while editing
(this was significantly more useful in the earlier days of \LaTeX{})
or to generate a smaller document which is easier to navigate.
Another application of |\includeonly| is to generate
documents consisting of selected parts of the complete document.

However, there are a few drawbacks of the plain |\include| mechanism:
\begin{itemize}
\item
The child files cannot be compiled on their own,
they can only be compiled via the main file.
A naive editing environment
(such as a text editor with an option
to have the current file processed by \LaTeX)
may require one to switch to the main file before compiling;
attempting to compile the child file produces errors.
\item
The main file must be modified (each time)
to adjust the |\includeonly| command
to the present needs. This easily leaves the main file in a messy state.
\item
The generated document will always carry the filename
of the main document. This is inconvenient if
several child files are to be compiled and
to be kept for distribution.
\end{itemize}

The present package provides a simple interface
to make child files individually compilable by \LaTeX{}.
Compiling a child file then has the same effect as compiling
the main file with an |\includeonly| command
to select the appropriate child.
Moreover the generated document will carry the name of the child
rather than the main file.
This resolves all three above issues.

This feature is meant to make the editing of books,
thesis documents and lecture notes somewhat more convenient.
However, the package can also be used efficiently for
composing a series of documents (such as exercise sheets)
which are typically distributed individually.
It then assists the author in generating the individual documents
(potentially in different versions)
as well as a document containing the collected series.
Another application is in developing style files
or other kinds of included material
where compilation of the style file could redirect
to a sample or test file.

%%%%%%%%%%%%%%%%%%%%%%%%%%%%%%%%%%%%%%%%%%%%%%%%%%%%%%%%%%%%%%%%%%%%%%%%%%%%%%%%
%%%%%%%%%%%%%%%%%%%%%%%%%%%%%%%%%%%%%%%%%%%%%%%%%%%%%%%%%%%%%%%%%%%%%%%%%%%%%%%%
\section{Usage}

First of all, the package \textsf{childdoc} is \emph{not} a standard
\LaTeXe{} |.sty| style file! Therefore it needs to be invoked in
a non-standard way.

%%%%%%%%%%%%%%%%%%%%%%%%%%%%%%%%%%%%%%%%%%%%%%%%%%%%%%%%%%%%%%%%%%%%%%%%%%%%%%%%
\subsection{Included Files}
\label{sec:include}

%%%%%%%%%%%%%%%%%%%%%%%%%%%%%%%%%%%%%%%%
\DescribeMacro{\childdocmain}
To use the package, add the commands
\begin{center}
\begin{tabular}{l}
|\input{childdoc.def}|\\
|\childdocmain{}|\\
\end{tabular}
\end{center}
at the very top of the main \LaTeX{} file,
in particular \emph{before} the |\documentclass| statement!
The argument of |\childdocmain| should be left empty
(but it must be present).

%%%%%%%%%%%%%%%%%%%%%%%%%%%%%%%%%%%%%%%%
\DescribeMacro{\childdocof}
Furthermore, add the commands
\begin{center}
\begin{tabular}{l}
|\input{childdoc.def}|\\
|\childdocof{|\textit{main}|}|\\
\end{tabular}
\end{center}
at the top of every child file \textit{child}
which is included by |\include{|\textit{child}|}|
from within the main file
(or at least for those files to be compiled individually).
The argument \textit{main} must be the filename of the main file.

There are a couple of
considerations in setting up the main and child documents:

%%%%%%%%%%%%%%%%%%%%%%%%%%%%%%%%%%%%%%%%
\paragraph{Restrictions.}

Please note the following restrictions:
\begin{itemize}
\item
|\childdocmain| must be called with one argument \textit{main}
to ensure compatibility with earlier version of the package.
It must either be empty (|\childdocmain{}|)
or precisely match the filename of the main file in which it is specified.
See \secref{sec:detection} for further information.
\item
The filename \textit{main} must be specified without the |.tex| extension.
\item
The filename \textit{main} is case sensitive
(even in case-insensitive file systems)
due to internal string comparison.
\item
The argument \textit{main} should be fully expanded, it cannot be a macro.
\item
Subdirectories and special characters should be avoided in filenames.
\item
The command |\childdocmain{|\textit{main}|}| must be followed by a whitespace.
It should not be followed immediately by another command
or by a comment mark `|%|'.
This is because the \TeX{} parser reads the token immediately following
the argument of |\childdocmain| and puts it
at the beginning of every child section;
however, a white\-space is ignored.
\end{itemize}

%%%%%%%%%%%%%%%%%%%%%%%%%%%%%%%%%%%%%%%%
\paragraph{Content of Main File.}

It is advisable to place all content in the child files included by |\include|.
Any output contained in the main file will appear in all child documents
unless suppressed manually;
it cannot be suppressed automatically by the |\includeonly| directive
and thus should normally be avoided.
A method to include some content in the main file
by means of conditional processing is described in \secref{sec:conditional}.

%%%%%%%%%%%%%%%%%%%%%%%%%%%%%%%%%%%%%%%%
\paragraph{Page Numbering.}

When only a part of the document is compiled,
the appropriate numbering of pages
(as well as other status parameters)
is determined from the |.aux| files.
The latter contain information from previous passes.
However this information needs to propagate through
all intermediate child documents.
Therefore the page numbering in child documents may well
be inconsistent until the complete document is compiled at least once.

A useful (if unconventional) way to always ensure a consistent
page numbering is to restart the numbering in each child document
and denote the pages by `\textit{child}|.|\textit{page}'
where \textit{child} represents the chapter/section number of the child file.
This can be achieved by the command
|\numberwithin{page}{|\textit{child}|}|
of the \textsf{amsmath} package
where \textit{child} can be |chapter| or |section|
depending on the chosen structuring.
Alternatively, one can modify the macro |\thepage| appropriately
and reset the counter |page| at the start of each child file.

%%%%%%%%%%%%%%%%%%%%%%%%%%%%%%%%%%%%%%%%%%%%%%%%%%%%%%%%%%%%%%%%%%%%%%%%%%%%%%%%
\subsection{Conditional Processing}
\label{sec:conditional}

The package provides a mechanism to compile different versions
of a document. To customise the versions further some conditional processing
can come in handy to distinguish which version is being compiled.
The package provides two macros to describe the compilation context:

%%%%%%%%%%%%%%%%%%%%%%%%%%%%%%%%%%%%%%%%
\DescribeMacro{\ifchilddoc}
The conditional |\ifchilddoc| distinguishes between the compilation of
child documents and the main document:
%
\begin{center}
|\ifchilddoc |\textit{child-code}| |[|\||else |\textit{main-code}]| \||fi|
\end{center}

%%%%%%%%%%%%%%%%%%%%%%%%%%%%%%%%%%%%%%%%
\DescribeMacro{\childdocname}
\DescribeMacro{\childdocjob}
The macro |\childdocname| contains the filename (without extension)
of the main or child file being processed.
Note that |\childdocjob| will always contain the name of the main file.

%%%%%%%%%%%%%%%%%%%%%%%%%%%%%%%%%%%%%%%%
\paragraph{Title Page.}

Conditional processing can be used to include a title or banner page
in the main document when proper precautions are taken.
Importantly, the code in the main file should ensure that the page counter
(as well as other status parameters which are stored in the |.aux| files)
takes the same value after the conditional processing.
Otherwise the page numbers may take divergent values
depending on which part is compiled.

For example, a title page could be declared by:
%
\begin{center}
\begin{tabular}{l}
|\ifchilddoc\||else|\\
|\addtocounter{page}{-1}|\\
\textit{code for title page}\\
|\newpage|\\
|\||fi|
\end{tabular}
\end{center}
%
A banner page for the child documents can be generated by:
%
\begin{center}
\begin{tabular}{l}
|\ifchilddoc|\\
|\addtocounter{page}{-1}|\\
\textit{code for banner page}\\
|\newpage|\\
|\||fi|
\end{tabular}
\end{center}
%
Here one could write a message such as:
\begin{center}
|This is the part \childdocname{} of \childdocjob{}.|
\end{center}

%%%%%%%%%%%%%%%%%%%%%%%%%%%%%%%%%%%%%%%%%%%%%%%%%%%%%%%%%%%%%%%%%%%%%%%%%%%%%%%%
\subsection{Flags}
\label{sec:flags}

The package makes it easy to generate different versions
of the main or child documents.
To this end compilation flags can be defined
and assigned different default values.
They will be particularly useful in conjunction
with the forwarding mechanism described in \secref{sec:forward}.

For example, it may be useful to have a flag |\version|
which can be set to |draft| or |final|.
The document source will contain some conditional code
depending on the value of |\version|.
Suppose further, the flag should default to |final| for the main file
and to |draft| for child files
which is a natural assignment for editing the document.
This is achieved by placing the following code
in the preamble of the main document
(below the |\childdocmain| directive):
%
\begin{center}
\begin{tabular}{l}
|\ifchilddoc|\\
|\providecommand{\version}{draft}|\\
|\||else|\\
|\providecommand{\version}{final}|\\
|\||fi|
\end{tabular}
\end{center}
%
The definition by |\providecommand| makes sure
that previous definitions are not overwritten.
Further statements |\providecommand{\version}{...}|
can thus be added before the above code to override it.

For the main file, one might add a line
(between |\childdocmain| and the above block)
%
\begin{center}
|%\ifchilddoc\||else\providecommand{\version}{draft}\||fi|
\end{center}
%
which can be uncommented to produce a draft version.
Likewise one can add a line to the very top of a child file
(above the |\childdocof{|\textit{main}|}| directive)
%
\begin{center}
|%\providecommand{\version}{final}|
\end{center}
%
which can be uncommented to produce the final version of this child document.

%%%%%%%%%%%%%%%%%%%%%%%%%%%%%%%%%%%%%%%%%%%%%%%%%%%%%%%%%%%%%%%%%%%%%%%%%%%%%%%%
\subsection{Forwarding}
\label{sec:forward}

Different versions of the main or child documents
using compilation flags as described in \secref{sec:flags}
can be (permanently) stored in different files
for convenient compilation, viewing and distribution.
To this end, the package defines a command
to pass on compilation to a different file:

%%%%%%%%%%%%%%%%%%%%%%%%%%%%%%%%%%%%%%%%
\DescribeMacro{\childdocforward}
The command |\childdocforward| redirects processing to
another source file:
%
\begin{center}
\begin{tabular}{l}
|\input{childdoc.def}|\\
|\childdocforward[|\textit{main}|]{|\textit{dest}|}|\\
\end{tabular}
\end{center}
%
The argument \textit{dest} is the destination file
(without extension).
It should be the main file or one of the child files.
Note that further \textsf{childdoc} directives
such as |\childdocof| and |\childdocforward|
in the indicated file will be processed in this form.
The optional argument \textit{main}
passes on directly to the main file \textit{main}
while pretending to compile the child \textit{dest}.
This form behaves as if \textit{dest}
issues |\childdocof{|\textit{main}|}| right away,
and no further \textsf{childdoc} directives will be processed.

%%%%%%%%%%%%%%%%%%%%%%%%%%%%%%%%%%%%%%%%
\DescribeMacro{\...prefix}
In the alternative form |\childdocforwardprefix|,
%
\begin{center}
\begin{tabular}{l}
|\input{childdoc.def}|\\
|\childdocforwardprefix[|\textit{main}|]{|\textit{prefix}|}{|\textit{dest}|}|
\end{tabular}
\end{center}
%
the destination file is determined by a pattern
depending on the current file:
To make this work, the current file must be called
`{\textit{prefix}\hspace{0.2em}\textit{suffix}}'
with \textit{prefix} matching precisely the argument.
Processing is then passed on to the file
`{\textit{dest}\hspace{0.2em}\textit{suffix}}'.
Surely, the same effect is achieved by
directly specifying the
argument `{\textit{dest}\hspace{0.2em}\textit{suffix}}'
in the first form.
However, that requires to set up a different file
for each child. With the alternative form of the command
all these files can have exactly the same content
which simplifies setting them up and maintaining them.

For example, the following file |draft.tex|
with a compilation flag |\version| as described in \secref{sec:flags}
compiles the main document as a draft:
%
\begin{center}
\begin{tabular}{l}
|\def\version{draft}|\\
|\input{childdoc.def}|\\
|\childdocforward{|\textit{main}|}|
\end{tabular}
\end{center}
%
Likewise, the following files |final|\textit{nn}|.tex|
compile the final version of the child document
|child|\textit{nn}|.tex|:
%
\begin{center}
\begin{tabular}{l}
|\def\version{final}|\\
|\input{childdoc.def}|\\
|\childdocforwardprefix{final}{child}|
\end{tabular}
\end{center}
%

Note that when several versions of a main file and/or of each child file
are to be generated, it may be convenient to set up a |Makefile| or
shell script to automatise the process.

%%%%%%%%%%%%%%%%%%%%%%%%%%%%%%%%%%%%%%%%%%%%%%%%%%%%%%%%%%%%%%%%%%%%%%%%%%%%%%%%
\subsection{Command Line Processing}
\label{sec:commandline}

The effect of redirection files can also be achieved by invoking
the \LaTeX{} compiler with a more elaborate command line.
Most conveniently this should be done as part
of a shell script or a |Makefile|.

When using \textsf{childdoc} in the main file, the following
command lines effectively perform a redirection
(note that depending on the shell being used,
backslashes may have to be doubled: `|\|' $\to$ `|\\|'):
%
\begin{center}
|... -jobname "|\textit{target}|" |\\|"|[\textit{flags}]%
|\input{childdoc.def}\childdocforward[|\textit{main}|]{|\textit{dest}|}"|
\end{center}
%
Here \textit{target} is the name of the output file,
\textit{main} is the name of the main file
and \textit{dest} is the name of the main or child file to be processed
(all filenames without extensions).
The optional argument \textit{main} can be omitted
if \textit{main} matches \textit{dest}.
Optionally, compilation \textit{flags} can be defined via |\def| commands.
This command line makes the \TeX{} engine believe
it is compiling the file \textit{target}
whose content is specified as the latter parameter.
The provided code then forwards the processing to
\textit{main} or \textit{dest} as described in \secref{sec:forward}.

%%%%%%%%%%%%%%%%%%%%%%%%%%%%%%%%%%%%%%%%%%%%%%%%%%%%%%%%%%%%%%%%%%%%%%%%%%%%%%%%
\subsection{Include by Input}
\label{sec:input}

Including child documents by |\include| has some restrictions by design.
Most notably, the content of a child document always occupies
its own set of pages; pages cannot be shared between child documents.
Usually, this behaviour makes perfect sense
because each child document contain an essential part of the document.
However, in some situations it may be desirable to compose
a document from a collection of parts
without having mandatory page breaks between then.
For this case, the package
provides a mechanism to include parts
by |\input| which can also be processed individually.
However, by construction this mechanism
requires manual handling of the content to be output.

%%%%%%%%%%%%%%%%%%%%%%%%%%%%%%%%%%%%%%%%
\DescribeMacro{\ifchilddocmanual}
The main file should be prepared as usual, see \secref{sec:include}.
However, the document body must make a distinction
between processing of an individual part and of the main document, e.g.:
%
\begin{center}
\begin{tabular}{l}
|\ifchilddocmanual|\\
|\input{\childdocname}|\\
|\||else|\\
\textit{document body with }|\input{|\textit{part}|}|\\
|\||fi|
\end{tabular}
\end{center}
%
The conditional |\ifchilddocmanual| is true whenever
a part to be included by |\input| is being compiled,
and the name of the part is stored in |\childdocname|.

%%%%%%%%%%%%%%%%%%%%%%%%%%%%%%%%%%%%%%%%
\DescribeMacro{\childdocby}
Each part to be included by |\input| should start with:
%
\begin{center}
\begin{tabular}{l}
|\input{childdoc.def}|\\
|\childdocby{|\textit{main}|}|\\
\end{tabular}
\end{center}
%
The directive |\childdocby| is similar to |\childdocof|
described in \secref{sec:include},
but the subsequent selection of content must be done manually.
To that end, both |\ifchilddoc| and |\ifchilddocmanual|
will be true upon processing of a part,
and the name of the part is stored in |\childdocname|.
Note that |\jobname| will be set to the filename of the current part
so that each part receives an individual |.aux| file
that does not interfere with the |.aux| file(s) of the main document.
This behaviour can be altered by the alternative form
|\childdocby[*]{|\textit{main}|}| (with a non-empty optional argument)
which uses the |.aux| file of the main document
by setting |\jobname| to \textit{main}.

%%%%%%%%%%%%%%%%%%%%%%%%%%%%%%%%%%%%%%%%%%%%%%%%%%%%%%%%%%%%%%%%%%%%%%%%%%%%%%%%
\subsection{Driver Development}
\label{sec:driver}

The \textsf{childdoc} mechanism can also be use for the development
of definition files such as \LaTeX{} styles or classes.
This case differs from the above setup with multiple parts
included by |\include| in that no |\includeonly| should be invoked.
This can be achieved by starting the include file
(before |\ProvidesPackage|) with:
%
\begin{center}
\begin{tabular}{l}
|\input{childdoc.def}|\\
|\childdocforward{|\textit{main}|}|\\
\end{tabular}
\end{center}
%
or alternatively with:
%
\begin{center}
\begin{tabular}{l}
|\input{childdoc.def}|\\
|\childdocby{|\textit{main}|}|\\
\end{tabular}
\end{center}
%
Both forms have slightly different effects as described above.
The main file is prepared as usual, see \secref{sec:include}.

%%%%%%%%%%%%%%%%%%%%%%%%%%%%%%%%%%%%%%%%%%%%%%%%%%%%%%%%%%%%%%%%%%%%%%%%%%%%%%%%
\subsection{Legacy Detection}
\label{sec:detection}

The directive |\childdocmain| in the main file can detect
whether the complete document or merely a child is to be compiled
even without using the directive |\childdocof|.
This method is deprecated because it is less robust
and there is no compelling reason to use it;
it is merely provided for backward compatibility
and it may be removed in future versions.

If the detection mechanism is to be used,
it is mandatory to correctly specify
the filename of the main file as the argument of |\childdocmain|:
%
\begin{center}
\begin{tabular}{l}
|\input{childdoc.def}|\\
|\childdocmain{|\textit{main}|}|\\
\end{tabular}
\end{center}
%
If |\jobname| does not match the argument \textit{main} of |\childdocmain|,
it is assumed that |\jobname| points to the child file to be compiled.
When using |\childdocmain| with the main file specified as argument,
it suffices to start a child file
with just |\input{|\textit{main}|}|
without loading of the package and using |\childdocof|.
If instead all processing is done
with the appropriate \textsf{childdoc} directives,
the argument of \textit{main} of |\childdocmain| can be empty.

An alternative version of the command line processing described
in \secref{sec:commandline} using the detection mechanism reads:
%
\begin{center}
|... -jobname "|\textit{target}|" "|[\textit{flags}]%
[|\def\jobname{|\textit{dest}|}|]|\input{|\textit{main}|}"|
\end{center}

%%%%%%%%%%%%%%%%%%%%%%%%%%%%%%%%%%%%%%%%%%%%%%%%%%%%%%%%%%%%%%%%%%%%%%%%%%%%%%%%
\subsection{Manual Code}
\label{sec:manual}

In case one cannot be certain whether the definitions file |childdoc.def|
is installed on the target \TeX{} distribution
and one prefers not to ship it,
it is conceivable to paste a few relevant commands into the sources.

To that end, drop all statements |\input{childdoc.def}|
and perform the replacements as outlined below.
Instead of |\childdocmain{|\textit{main}|}| add the following code
to the top of the main file:
%
\begin{center}
\begin{tabular}{l}
|\||ifdefined\childdocname\endinput\||fi\newif\ifchilddoc|\\
|\edef\childdocname{\scantokens\expandafter{\jobname\noexpand}}|\\
|\def\childdocmain{|\textit{main}|}\||ifx\childdocmain\childdocname\||else|\\
|\childdoctrue\includeonly{\childdocname}\let\jobname\childdocmain\||fi|\\
\end{tabular}
\end{center}
%
Instead of |\childdocof{|\textit{main}|}| just include the main file
at the top of each child file:
%
\begin{center}
|\input{|\textit{main}|}|
\end{center}
%
A simple redirection |\childdocforward{|\textit{dest}|}| is achieved by:
%
\begin{center}
|\def\jobname{|\textit{dest}|}\input{\jobname}|
\end{center}
%
The redirection with prefix
|\childdocforwardprefix[|\textit{prefix}|]{|\textit{dest}|}|
is accomplished by:
%
\begin{center}
\begin{tabular}{l}
|{\edef\jobname{\scantokens\expandafter{\jobname\noexpand}}|\\
|\def\redirectjob |\textit{prefix}|#1~~~{\gdef\jobname{|\textit{dest}|#1}}|\\
|\expandafter\redirectjob\jobname~~~}\input{\jobname}|
\end{tabular}
\end{center}

In an alternative approach,
child documents can be compiled by a specific command line
without additional code or specific definitions:
%
\begin{center}
|... -jobname "|\textit{target}|" "|[\textit{flags}]%
|\includeonly{|\textit{dest}|}\input{|\textit{main}|}"|
\end{center}
%

%%%%%%%%%%%%%%%%%%%%%%%%%%%%%%%%%%%%%%%%%%%%%%%%%%%%%%%%%%%%%%%%%%%%%%%%%%%%%%%%
%%%%%%%%%%%%%%%%%%%%%%%%%%%%%%%%%%%%%%%%%%%%%%%%%%%%%%%%%%%%%%%%%%%%%%%%%%%%%%%%
\section{Information}

%%%%%%%%%%%%%%%%%%%%%%%%%%%%%%%%%%%%%%%%%%%%%%%%%%%%%%%%%%%%%%%%%%%%%%%%%%%%%%%%
\subsection{Copyright}

Copyright \copyright{} 2017--2018 Niklas Beisert

This work may be distributed and/or modified under the
conditions of the \LaTeX{} Project Public License, either version 1.3
of this license or (at your option) any later version.
The latest version of this license is in
  \url{http://www.latex-project.org/lppl.txt}
and version 1.3 or later is part of all distributions of \LaTeX{}
version 2005/12/01 or later.

This work has the LPPL maintenance status `maintained'.

The Current Maintainer of this work is Niklas Beisert.

This work consists of the files |README.txt|, |childdoc.ins| and |childdoc.dtx|
as well as the derived files |childdoc.def|, |cdocsamp.tex|
with |cdocsch1.tex|, |cdocsch2.tex|, |cdocspt3.tex|, |cdocspt4.tex|,
|cdocsdrf.tex|, |cdocsfn1.tex|, |cdocsfn2.tex|
as well as |childdoc.pdf|.

%%%%%%%%%%%%%%%%%%%%%%%%%%%%%%%%%%%%%%%%%%%%%%%%%%%%%%%%%%%%%%%%%%%%%%%%%%%%%%%%
\subsection{Files and Installation}

The package consists of the files:
%
\begin{center}
\begin{tabular}{ll}
    |README.txt|   & readme file \\
    |childdoc.ins| & installation file \\
    |childdoc.dtx| & source file \\
    |childdoc.def| & definition file \\
    |cdocsamp.tex| & sample main file \\
    |cdocsch1.tex| & sample include file \\
    |cdocsch2.tex| & sample include file \\
    |cdocspt3.tex| & sample part file \\
    |cdocspt4.tex| & sample part file \\
    |cdocsdrf.tex| & sample redirection file \\
    |cdocsfn1.tex| & sample redirection file \\
    |cdocsfn2.tex| & sample redirection file \\
    |childdoc.pdf| & manual
\end{tabular}
\end{center}
%
The distribution consists of the files
|README.txt|, |childdoc.ins| and |childdoc.dtx|.
%
\begin{itemize}
\item
Run (pdf)\LaTeX{} on |childdoc.dtx|
to compile the manual |childdoc.pdf| (this file).
\item
Run \LaTeX{} on |childdoc.ins| to create the definitions file |childdoc.def|
and the sample |cdocsamp.tex| with include files
|cdocsch1.tex|, |cdocsch2.tex|, |cdocspt3.tex|, |cdocspt4.tex|,
|cdocsdrf.tex|, |cdocsfn1.tex|, |cdocsfn2.tex|.
Then copy the file |childdoc.def| to an appropriate directory of your \LaTeX{}
distribution, e.g.\ \textit{texmf-root}|/tex/latex/childdoc|.
\end{itemize}

%%%%%%%%%%%%%%%%%%%%%%%%%%%%%%%%%%%%%%%%%%%%%%%%%%%%%%%%%%%%%%%%%%%%%%%%%%%%%%%%
\subsection{Related CTAN Packages}

There are several other packages which offer a similar functionality:
%
\begin{itemize}
\item
The packages
\href{http://ctan.org/pkg/docmute}{\textsf{docmute}},
\href{http://ctan.org/pkg/includex}{\textsf{includex}} and
\href{http://ctan.org/pkg/standalone}{\textsf{standalone}}
provide commands to include only the document body of
a child file thus allowing both files to be compiled individually.
\item
The packages \href{http://ctan.org/pkg/subdocs}{\textsf{subdocs}}
and \href{http://ctan.org/pkg/subfiles}{\textsf{subfiles}}
provide structures in which the main and child documents can be
encapsulated and allowing them to be compiled individually.
The inclusion mechanism is different from the conventional |\include|.
\item
The package \href{http://ctan.org/pkg/combine}{\textsf{combine}}
is an elaborate solution to combine several documents into one.
\end{itemize}
%
See also the CTAN topic \href{http://ctan.org/topic/subdocs}{\textsf{subdocs}}
for further related packages.
The present package differs from the above solutions in that
a document structure constructed with the conventional |\include| mechanism
just needs two extra commands at the top of every file
such that all constituent files can be compiled individually.

%%%%%%%%%%%%%%%%%%%%%%%%%%%%%%%%%%%%%%%%%%%%%%%%%%%%%%%%%%%%%%%%%%%%%%%%%%%%%%%%
%\subsection{Feature Suggestions}
%
%The following is a list of features which may be useful for future
%versions of this package:
%%
%\begin{itemize}
%\item
%\ldots
%\end{itemize}

%%%%%%%%%%%%%%%%%%%%%%%%%%%%%%%%%%%%%%%%%%%%%%%%%%%%%%%%%%%%%%%%%%%%%%%%%%%%%%%%
\subsection{Revision History}

%%%%%%%%%%%%%%%%%%%%%%%%%%%%%%%%%%%%%%%%
\paragraph{v2.0:} 2018/12/30

\begin{itemize}
\item
immediate forward processing
\item
added |\childdocby| mechanism
\item
manual restructured
\end{itemize}

%%%%%%%%%%%%%%%%%%%%%%%%%%%%%%%%%%%%%%%%
\paragraph{v1.6:} 2018/01/17

\begin{itemize}
\item
application for development of include files
\item
corrections to manual
\end{itemize}

%%%%%%%%%%%%%%%%%%%%%%%%%%%%%%%%%%%%%%%%
\paragraph{v1.5:} 2017/05/21

\begin{itemize}
\item
more complete structuring introduced
\item
|\childdocof| introduced
\item
|\childdoc| renamed to |\childdocmain|
\item
|\childredirect| renamed to |\childdocforward| and |\childdocforwardprefix|
and functionality expanded
\end{itemize}

%%%%%%%%%%%%%%%%%%%%%%%%%%%%%%%%%%%%%%%%
\paragraph{v1.0:} 2017/04/27

\begin{itemize}
\item
manual and install package
\item
first version published on CTAN
\end{itemize}

%%%%%%%%%%%%%%%%%%%%%%%%%%%%%%%%%%%%%%%%
\paragraph{v0.6:} 2017/04/26

\begin{itemize}
\item
redirection mechanism added
\end{itemize}

%%%%%%%%%%%%%%%%%%%%%%%%%%%%%%%%%%%%%%%%
\paragraph{v0.5:} 2017/04/26

\begin{itemize}
\item
functionality in definition file
\end{itemize}


%%%%%%%%%%%%%%%%%%%%%%%%%%%%%%%%%%%%%%%%%%%%%%%%%%%%%%%%%%%%%%%%%%%%%%%%%%%%%%%%
%%%%%%%%%%%%%%%%%%%%%%%%%%%%%%%%%%%%%%%%%%%%%%%%%%%%%%%%%%%%%%%%%%%%%%%%%%%%%%%%
%%%%%%%%%%%%%%%%%%%%%%%%%%%%%%%%%%%%%%%%%%%%%%%%%%%%%%%%%%%%%%%%%%%%%%%%%%%%%%%%
\appendix

\settowidth\MacroIndent{\rmfamily\scriptsize 000\ }

 \DocInput{childdoc.dtx}

\end{document}
%</driver>
% \fi
%
% %%%%%%%%%%%%%%%%%%%%%%%%%%%%%%%%%%%%%%%%%%%%%%%%%%%%%%%%%%%%%%%%%%%%%%%%%%%%%%
% %%%%%%%%%%%%%%%%%%%%%%%%%%%%%%%%%%%%%%%%%%%%%%%%%%%%%%%%%%%%%%%%%%%%%%%%%%%%%%
% \section{Sample}
%\iffalse
%<*samplemain>
%\fi
%
% The following presents a sample document
% with two chapters, two parts, a title page,
% a compile flag as well as three forwarding files to set the flag.
% It consists of eight |.tex| files:
% \begin{center}
% \begin{tabular}{ll}
% |cdocsamp.tex|&main file\\
% |cdocsch1.tex|&include file for chapter 1\\
% |cdocsch2.tex|&include file for chapter 2\\
% |cdocspt3.tex|&include file for part 3\\
% |cdocspt4.tex|&include file for part 4\\
% |cdocsdrf.tex|&forwarding file for main file in draft mode\\
% |cdocsfi1.tex|&forwarding file for final version of chapter 1\\
% |cdocsfi2.tex|&forwarding file for final version of chapter 2\\
% \end{tabular}
% \end{center}
% Each of the eight files can be compiled directly by the \LaTeX{} compiler.
%
% %%%%%%%%%%%%%%%%%%%%%%%%%%%%%%%%%%%%%%
% \paragraph{Main File.}
%
% The main file is called |cdocsamp.tex|.
%
% Load the \textsf{childdoc} definitions and
% declare the filename for the main document:
%    \begin{macrocode}
\input{childdoc.def}
\childdocmain{}
%    \end{macrocode}

% Optional override for |\version| flag:
%    \begin{macrocode}
%%\ifchilddoc\else\providecommand{\version}{draft}\fi
%    \end{macrocode}

% Define the default values for the |\version| flag
% (|final| for the main file and |draft| for childs):
%    \begin{macrocode}
\ifchilddoc
\providecommand{\version}{draft}
\else
\providecommand{\version}{final}
\fi
%    \end{macrocode}

% Load the standard document class:
%    \begin{macrocode}
\documentclass[12pt]{article}
%    \end{macrocode}

% Start the document body:
%    \begin{macrocode}
\begin{document}
%    \end{macrocode}

% Declare a title page.
% Print title, part of document being processed and version flag:
%    \begin{macrocode}
\addtocounter{page}{-1}
\begin{center}
{\LARGE\bfseries{}childdoc example\par}
\vspace{1cm}
\ifchilddoc
\ifchilddocmanual part\else chapter\fi:
`\childdocname' of `\childdocjob'\par
\else
main document: `\childdocjob'\par
\fi
version: \version\par
\end{center}
\newpage
%    \end{macrocode}

% Manually include selected file,
% otherwise process as usual:
%    \begin{macrocode}
\ifchilddocmanual
\section*{part `\childdocname'}
\input{\childdocname}
\else
%    \end{macrocode}

% Include the two chapters:
%    \begin{macrocode}
\include{cdocsch1}
\include{cdocsch2}
%    \end{macrocode}

% Include the two parts unless only chapters should be displayed:
%    \begin{macrocode}
\ifchilddoc\else
\section{part three}
\input{cdocspt3}
\section{part four}
\input{cdocspt4}
\fi
%    \end{macrocode}

% Process as usual until here:
%    \begin{macrocode}
\fi
%    \end{macrocode}

% End of document body:
%    \begin{macrocode}
\end{document}
%    \end{macrocode}
%\iffalse
%</samplemain>
%\fi
%
% %%%%%%%%%%%%%%%%%%%%%%%%%%%%%%%%%%%%%%
% \paragraph{Chapter Include Files.}
%
% The include files are called |cdocsch1.tex| and |cdocsch2.tex|.
%
%\iffalse
%<*samplechap1|samplechap2>
%\fi

% Optional override for |\version| flag:
%    \begin{macrocode}
%%\providecommand{\version}{final}
%    \end{macrocode}

% Include the main document:
%    \begin{macrocode}
\input{childdoc.def}
\childdocof{cdocsamp}
%    \end{macrocode}

%\iffalse
%</samplechap1|samplechap2>
%\fi
%
%\iffalse
%<*samplechap1>
%\fi
% Some text for chapter 1:
%    \begin{macrocode}
\section{one}
some text in chapter one
%    \end{macrocode}

%\iffalse
%</samplechap1>
%\fi
% Some text for chapter 2:
%\iffalse
%<*samplechap2>
%\fi
%    \begin{macrocode}
\section{two}
more text in chapter two
%    \end{macrocode}

%\iffalse
%</samplechap2>
%\fi
%
% %%%%%%%%%%%%%%%%%%%%%%%%%%%%%%%%%%%%%%
% \paragraph{Part Include Files.}
%
% The include files are called |cdocspt3.tex| and |cdocspt4.tex|.
%
%\iffalse
%<*samplepart3|samplepart4>
%\fi

% Optional override for |\version| flag:
%    \begin{macrocode}
%%\providecommand{\version}{final}
%    \end{macrocode}

% Include the main document:
%    \begin{macrocode}
\input{childdoc.def}
\childdocby{cdocsamp}
%    \end{macrocode}

%\iffalse
%</samplepart3|samplepart4>
%\fi
%
%\iffalse
%<*samplepart3>
%\fi
% Some text for part 3:
%    \begin{macrocode}
some text in part three
%    \end{macrocode}

%\iffalse
%</samplepart3>
%\fi
% Some text for part 4:
%\iffalse
%<*samplepart4>
%\fi
%    \begin{macrocode}
more text in part four
%    \end{macrocode}

%\iffalse
%</samplepart4>
%\fi
%
% %%%%%%%%%%%%%%%%%%%%%%%%%%%%%%%%%%%%%%
% \paragraph{Forwarding for a Complete Draft.}
%
% The following forwarding file |cdocsdrf.tex|
% compiles the main document in draft mode:
%\iffalse
%<*sampledraft>
%\fi
%    \begin{macrocode}
\def\version{draft}
\input{childdoc.def}
\childdocforward{cdocsamp}
%    \end{macrocode}

%\iffalse
%</sampledraft>
%\fi
%
% %%%%%%%%%%%%%%%%%%%%%%%%%%%%%%%%%%%%%%
% \paragraph{Forwarding for Final Version of the Chapters.}
%
% The following forwarding files |cdocsfn1.tex| and |cdocsfn2.tex|
% (with identical content)
% compile the final versions of the child documents
% |cdocsch1.tex| and |cdocsch2.tex|, respectively:
%\iffalse
%<*samplefinal>
%\fi
%    \begin{macrocode}
\def\version{final}
\input{childdoc.def}
\childdocforwardprefix[cdocsamp]{cdocsfn}{cdocsch}
%    \end{macrocode}

%\iffalse
%</samplefinal>
%\fi
%
% %%%%%%%%%%%%%%%%%%%%%%%%%%%%%%%%%%%%%%
% \paragraph{Command Line Processing.}
%
% The following three command lines generate the output files
% |cdocscld|, |cdocscl1| and |cdocscl2|
% which should be identical to
% |cdocsdrf|, |cdocsch1| and |cdocsfn2|, respectively:
% \begin{center}
% \begin{tabular}{l}
% |latex -jobname cdocscld \|\\
% |  "\def\version{draft}\input{childdoc.def}\childdocforward{cdocsamp}"|\\
% |latex -jobname cdocscl1 \|\\
% |  "\input{childdoc.def}\childdocforward[cdocsamp]{cdocsch1}"|\\
% |latex -jobname cdocscl2 \|\\
% |  "\def\version{final}\input{childdoc.def}\childdocforward{cdocsch2}"|
% \end{tabular}
% \end{center}
% Note that the trailing backslash on each first line
% merely continues the input to the second line
% (for convenient cut ant paste).
% Furthermore, the command |latex| can be replaced by any
% of its alternative versions such as |pdflatex|.
%
% %%%%%%%%%%%%%%%%%%%%%%%%%%%%%%%%%%%%%%%%%%%%%%%%%%%%%%%%%%%%%%%%%%%%%%%%%%%%%%
% %%%%%%%%%%%%%%%%%%%%%%%%%%%%%%%%%%%%%%%%%%%%%%%%%%%%%%%%%%%%%%%%%%%%%%%%%%%%%%
% \section{Implementation}
%\iffalse
%<*package>
%\fi
%
% This section describes the definitions file |childdoc.def|.

% The definitions cannot be loaded using |\usepackage| or |\RequirePackage|
% which has a mechanism to prevent loading a style file more than once.
% When loading the definitions by means of |\input|
% multiple instances have to be prevented manually:
%\iffalse
%This code needs to be before the `\ProvidesFile' directive
%which is defined at the beginning of this file.
%Therefore it is also placed there and commented out here.
%</package>
%<*discard>
%\fi
%    \begin{macrocode}
\ifdefined\childdocmain\endinput\fi
%    \end{macrocode}
%\iffalse
%</discard>
%<*package>
%\fi
%
% \macro{\ifchilddoc}
% \macro{\ifchilddocmanual}
% The conditional |\ifchilddoc| tells whether a
% child (true) or main (false) document is being compiled.
% The conditional |\ifchilddocmanual| tells whether
% the |\includeonly| mechanism is used (false) or
% the selection of child files must be performed manually (true).
% The definitions initialise to false:
%    \begin{macrocode}
\newif\ifchilddoc
\newif\ifchilddocmanual
%    \end{macrocode}

% \macro{\childdocname}
% \macro{\childdocjob}
% The macro |\childdocname| stores the name of the main document
% to be compiled. The macro |\childdocjob| stores the name of
% the document on which the \LaTeX{} compiler was originally invoked.
% The content of |\jobname| cannot be compared
% to filenames specified in the source due to different catcodes.
% The following code rescans |\jobname|, stores the result
% in |\childdocname| and saves a copy in |\childdocjob|:
%    \begin{macrocode}
\edef\childdocname{\scantokens\expandafter{\jobname\noexpand}}
\let\childdocjob\childdocname
%    \end{macrocode}

% \macro{\childdocdisable}
% The macro |\childdocdisable| prevents the main file
% from being processed more than once.
% At this stage, the main document command |\childdocmain|
% is assumed to be called once again where it should do nothing.
% Any subsequent call to it should prevent
% a secondary processing of the main document
% It overwrites the forwarding commands
% |\childdocof| and |\childdocforward|
% with empty macros to prevent further inclusions of the main document:
%    \begin{macrocode}
\newcommand{\childdocdisable}
{
  \renewcommand{\childdocmain}[1]{\renewcommand{\childdocmain}[1]{\endinput}}
  \renewcommand{\childdocof}[1]{}
  \renewcommand{\childdocby}[2][]{}
  \renewcommand{\childdocforward}[2][]{}
  \renewcommand{\childdocdisable}{}
}
%    \end{macrocode}

% \macro{\childdocmain}
% The macro |\childdocmain| is to be called at the top of the main file
% with nothing or the main filename (without extension) as argument.
% First, it breaks loops.
% If the argument is not empty and does not match |\childdocname|
% (which is set by the first inclusion of |childdoc.def|),
% |\ifchilddoc| is set to true, |\includeonly| is applied to the child file
% and |\jobname| is set to the main file
% (for proper handling of |.aux| files):
%    \begin{macrocode}
\newcommand{\childdocmain}[1]
{
  \childdocdisable\childdocmain{}
  \if?#1?\else
    \begingroup
      \def\childdoctmp{#1}
      \ifx\childdoctmp\childdocname
        \def\childdoctmp{}
      \else
        \def\childdoctmp
        {
          \childdoctrue
          \includeonly{\childdocname}
          \def\childdocjob{#1}
          \def\jobname{#1}
        }
      \fi
      \expandafter
    \endgroup
    \childdoctmp
  \fi
}
%    \end{macrocode}

% \macro{\childdocof}
% The command |\childdocof| redirects
% compilation to the main file |#1|.
%    \begin{macrocode}
\newcommand{\childdocof}[1]
{
  \childdocdisable
  \childdoctrue
  \includeonly{\childdocname}
  \def\jobname{#1}
  \def\childdocjob{#1}
  \input{#1}
}
%    \end{macrocode}

% \macro{\childdocby}
% The command |\childdocby| ....
%    \begin{macrocode}
\newcommand{\childdocby}[2][]
{
  \childdocdisable
  \childdoctrue
  \childdocmanualtrue
  \if?#1?\else
    \def\jobname{#2}
  \fi
  \def\childdocjob{#2}
  \input{#2}
  \endinput
}
%    \end{macrocode}

% \macro{\childdocforward}
% The command |\childdocforward| redirects
% compilation to the main file or
% (if the optional argument is given) a child file.
% Parameters are set as if the main file
% or a child file starting with |\childdocof| was compiled.
% Then compilation is handed over to the main file:
%    \begin{macrocode}
\newcommand{\childdocforward}[2][]
{
  \begingroup
    \if?#1?
      \def\childdoctmp
      {
        \def\childdocname{#2}
        \def\childdocjob{#2}
        \def\jobname{#2}
        \input{#2}
        \endinput
      }
    \else
      \def\childdoctmp
      {
        \childdocdisable
        \def\childdocname{#2}
        \childdoctrue
        \includeonly{#2}
        \def\childdocjob{#1}
        \def\jobname{#1}
        \input{#1}
        \endinput
      }
    \fi
    \expandafter
  \endgroup
  \childdoctmp
}
%    \end{macrocode}

% \macro{\childdocforwardprefix}
% The command |\childdocforwardprefix| redirects
% compilation to the main or a child file by means of a pattern.
% The prefix |#1| in the current filename is replaced by |#2|
% and the suffix of the current filename is kept
% (it is assumed that the filename does not contain the substring `|~~~|'
% which is used as a delimiter).
% Compilation is handed over to the new file by |\childdocforward|:
%    \begin{macrocode}
\newcommand{\childdocforwardprefix}[3][]
{
  \begingroup
    \def\childdocextract #2##1~~~{\def\childdoctmp{\childdocforward[#1]{#3##1}}}
    \expandafter\childdocextract\childdocname~~~
    \expandafter
  \endgroup
  \childdoctmp
}
%    \end{macrocode}

% \macro{\childdoc}
% The deprecated macro |\childdoc| is a legacy version of |\childdocmain|:
%    \begin{macrocode}
\newcommand{\childdoc}{\childdocmain}
%    \end{macrocode}

% \macro{\childdocredirect}
% The deprecated macro |\childdocredirect| is a legacy version
% of |\childdocforward| and |\childdocforwardprefix|:
%    \begin{macrocode}
\newcommand{\childdocredirect}[2][]
{
  \begingroup
    \if?#1?
      \def\childdoctmp{\childdocforward{#2}}
    \else
      \def\childdoctmp{\childdocforwardprefix{#1}{#2}}
    \fi
    \expandafter
  \endgroup
  \childdoctmp
}
%    \end{macrocode}

%\iffalse
%</package>
%\fi
%
\endinput

\childdocforwardprefix[cdocsamp]{cdocsfn}{cdocsch}
%    \end{macrocode}

%\iffalse
%</samplefinal>
%\fi
%
% %%%%%%%%%%%%%%%%%%%%%%%%%%%%%%%%%%%%%%
% \paragraph{Command Line Processing.}
%
% The following three command lines generate the output files
% |cdocscld|, |cdocscl1| and |cdocscl2|
% which should be identical to
% |cdocsdrf|, |cdocsch1| and |cdocsfn2|, respectively:
% \begin{center}
% \begin{tabular}{l}
% |latex -jobname cdocscld \|\\
% |  "\def\version{draft}% \iffalse
%
% childdoc.dtx Copyright (C) 2017-2018 Niklas Beisert
%
% This work may be distributed and/or modified under the
% conditions of the LaTeX Project Public License, either version 1.3
% of this license or (at your option) any later version.
% The latest version of this license is in
%   http://www.latex-project.org/lppl.txt
% and version 1.3 or later is part of all distributions of LaTeX
% version 2005/12/01 or later.
%
% This work has the LPPL maintenance status `maintained'.
%
% The Current Maintainer of this work is Niklas Beisert.
%
% This work consists of the files childdoc.dtx and childdoc.ins
% and the derived files childdoc.def and cdocsamp.tex with
% cdocsch1.tex, cdocsch2.tex, cdocsdrf.tex, cdocsfn1.tex, cdocsfn2.tex.
%
%<package>\ifdefined\childdocmain\endinput\fi
%<package>\ProvidesFile{childdoc.def}[2018/12/30 v2.0 child document driver]
%<samplemain>\ProvidesFile{cdocsamp.tex}[2018/12/30 v2.0 sample for childdoc]
%<*driver>
%\ProvidesFile{childdoc.drv}[2018/12/30 v2.0 childdoc reference manual file]
\PassOptionsToClass{10pt,a4paper}{article}
\documentclass{ltxdoc}

\usepackage[margin=35mm]{geometry}
\usepackage{hyperref}
\usepackage{hyperxmp}
\usepackage[usenames]{color}

\hypersetup{colorlinks=true}
\hypersetup{pdfstartview=FitH}
\hypersetup{pdfpagemode=UseNone}
\hypersetup{pdfsource={}}
\hypersetup{pdflang={en-UK}}
\hypersetup{pdfcopyright={Copyright 2017-2018 Niklas Beisert.
  This work may be distributed and/or modified under the
  conditions of the LaTeX Project Public License, either version 1.3
  of this license or (at your option) any later version.}}
\hypersetup{pdflicenseurl={http://www.latex-project.org/lppl.txt}}
\hypersetup{pdfcontactaddress={ETH Zurich, ITP, HIT K,
  Wolfgang-Pauli-Strasse 27}}
\hypersetup{pdfcontactpostcode={8093}}
\hypersetup{pdfcontactcity={Zurich}}
\hypersetup{pdfcontactcountry={Switzerland}}
\hypersetup{pdfcontactemail={nbeisert@itp.phys.ethz.ch}}
\hypersetup{pdfcontacturl={http://people.phys.ethz.ch/\xmptilde nbeisert/}}

\newcommand{\secref}[1]{\hyperref[#1]{section \ref*{#1}}}

\parskip1ex
\parindent0pt
\let\olditemize\itemize
\def\itemize{\olditemize\parskip0pt}

\begin{document}

\title{The \textsf{childdoc} Package}
\hypersetup{pdftitle={The childdoc Package}}
\author{Niklas Beisert\\[2ex]
  Institut f\"ur Theoretische Physik\\
  Eidgen\"ossische Technische Hochschule Z\"urich\\
  Wolfgang-Pauli-Strasse 27, 8093 Z\"urich, Switzerland\\[1ex]
  \href{mailto:nbeisert@itp.phys.ethz.ch}
  {\texttt{nbeisert@itp.phys.ethz.ch}}}
\hypersetup{pdfauthor={Niklas Beisert}}
\hypersetup{pdfsubject={Manual for the LaTeX2e Package childdoc}}
\date{30 December 2018, \textsf{v2.0}}
\maketitle

\begin{abstract}\noindent
\textsf{childdoc} is a \LaTeXe{} package
that enables the direct compilation
of document sections included by |\include|
to individual files.
\end{abstract}

\begingroup
\parskip0ex
\tableofcontents
\endgroup

%%%%%%%%%%%%%%%%%%%%%%%%%%%%%%%%%%%%%%%%%%%%%%%%%%%%%%%%%%%%%%%%%%%%%%%%%%%%%%%%
%%%%%%%%%%%%%%%%%%%%%%%%%%%%%%%%%%%%%%%%%%%%%%%%%%%%%%%%%%%%%%%%%%%%%%%%%%%%%%%%
\section{Introduction}

\LaTeX{} provides a mechanism to structure a large document (such as a book)
into a main file and several child files (containing the chapters)
using the |\include| command.
This mechanism is beneficial for documents
which span hundreds of pages in order to
make the source file(s) more manageable.
Moreover, compilation can be restricted to
selected child files by means of the |\includeonly| command.
The latter feature can be used to reduce the compilation time while editing
(this was significantly more useful in the earlier days of \LaTeX{})
or to generate a smaller document which is easier to navigate.
Another application of |\includeonly| is to generate
documents consisting of selected parts of the complete document.

However, there are a few drawbacks of the plain |\include| mechanism:
\begin{itemize}
\item
The child files cannot be compiled on their own,
they can only be compiled via the main file.
A naive editing environment
(such as a text editor with an option
to have the current file processed by \LaTeX)
may require one to switch to the main file before compiling;
attempting to compile the child file produces errors.
\item
The main file must be modified (each time)
to adjust the |\includeonly| command
to the present needs. This easily leaves the main file in a messy state.
\item
The generated document will always carry the filename
of the main document. This is inconvenient if
several child files are to be compiled and
to be kept for distribution.
\end{itemize}

The present package provides a simple interface
to make child files individually compilable by \LaTeX{}.
Compiling a child file then has the same effect as compiling
the main file with an |\includeonly| command
to select the appropriate child.
Moreover the generated document will carry the name of the child
rather than the main file.
This resolves all three above issues.

This feature is meant to make the editing of books,
thesis documents and lecture notes somewhat more convenient.
However, the package can also be used efficiently for
composing a series of documents (such as exercise sheets)
which are typically distributed individually.
It then assists the author in generating the individual documents
(potentially in different versions)
as well as a document containing the collected series.
Another application is in developing style files
or other kinds of included material
where compilation of the style file could redirect
to a sample or test file.

%%%%%%%%%%%%%%%%%%%%%%%%%%%%%%%%%%%%%%%%%%%%%%%%%%%%%%%%%%%%%%%%%%%%%%%%%%%%%%%%
%%%%%%%%%%%%%%%%%%%%%%%%%%%%%%%%%%%%%%%%%%%%%%%%%%%%%%%%%%%%%%%%%%%%%%%%%%%%%%%%
\section{Usage}

First of all, the package \textsf{childdoc} is \emph{not} a standard
\LaTeXe{} |.sty| style file! Therefore it needs to be invoked in
a non-standard way.

%%%%%%%%%%%%%%%%%%%%%%%%%%%%%%%%%%%%%%%%%%%%%%%%%%%%%%%%%%%%%%%%%%%%%%%%%%%%%%%%
\subsection{Included Files}
\label{sec:include}

%%%%%%%%%%%%%%%%%%%%%%%%%%%%%%%%%%%%%%%%
\DescribeMacro{\childdocmain}
To use the package, add the commands
\begin{center}
\begin{tabular}{l}
|\input{childdoc.def}|\\
|\childdocmain{}|\\
\end{tabular}
\end{center}
at the very top of the main \LaTeX{} file,
in particular \emph{before} the |\documentclass| statement!
The argument of |\childdocmain| should be left empty
(but it must be present).

%%%%%%%%%%%%%%%%%%%%%%%%%%%%%%%%%%%%%%%%
\DescribeMacro{\childdocof}
Furthermore, add the commands
\begin{center}
\begin{tabular}{l}
|\input{childdoc.def}|\\
|\childdocof{|\textit{main}|}|\\
\end{tabular}
\end{center}
at the top of every child file \textit{child}
which is included by |\include{|\textit{child}|}|
from within the main file
(or at least for those files to be compiled individually).
The argument \textit{main} must be the filename of the main file.

There are a couple of
considerations in setting up the main and child documents:

%%%%%%%%%%%%%%%%%%%%%%%%%%%%%%%%%%%%%%%%
\paragraph{Restrictions.}

Please note the following restrictions:
\begin{itemize}
\item
|\childdocmain| must be called with one argument \textit{main}
to ensure compatibility with earlier version of the package.
It must either be empty (|\childdocmain{}|)
or precisely match the filename of the main file in which it is specified.
See \secref{sec:detection} for further information.
\item
The filename \textit{main} must be specified without the |.tex| extension.
\item
The filename \textit{main} is case sensitive
(even in case-insensitive file systems)
due to internal string comparison.
\item
The argument \textit{main} should be fully expanded, it cannot be a macro.
\item
Subdirectories and special characters should be avoided in filenames.
\item
The command |\childdocmain{|\textit{main}|}| must be followed by a whitespace.
It should not be followed immediately by another command
or by a comment mark `|%|'.
This is because the \TeX{} parser reads the token immediately following
the argument of |\childdocmain| and puts it
at the beginning of every child section;
however, a white\-space is ignored.
\end{itemize}

%%%%%%%%%%%%%%%%%%%%%%%%%%%%%%%%%%%%%%%%
\paragraph{Content of Main File.}

It is advisable to place all content in the child files included by |\include|.
Any output contained in the main file will appear in all child documents
unless suppressed manually;
it cannot be suppressed automatically by the |\includeonly| directive
and thus should normally be avoided.
A method to include some content in the main file
by means of conditional processing is described in \secref{sec:conditional}.

%%%%%%%%%%%%%%%%%%%%%%%%%%%%%%%%%%%%%%%%
\paragraph{Page Numbering.}

When only a part of the document is compiled,
the appropriate numbering of pages
(as well as other status parameters)
is determined from the |.aux| files.
The latter contain information from previous passes.
However this information needs to propagate through
all intermediate child documents.
Therefore the page numbering in child documents may well
be inconsistent until the complete document is compiled at least once.

A useful (if unconventional) way to always ensure a consistent
page numbering is to restart the numbering in each child document
and denote the pages by `\textit{child}|.|\textit{page}'
where \textit{child} represents the chapter/section number of the child file.
This can be achieved by the command
|\numberwithin{page}{|\textit{child}|}|
of the \textsf{amsmath} package
where \textit{child} can be |chapter| or |section|
depending on the chosen structuring.
Alternatively, one can modify the macro |\thepage| appropriately
and reset the counter |page| at the start of each child file.

%%%%%%%%%%%%%%%%%%%%%%%%%%%%%%%%%%%%%%%%%%%%%%%%%%%%%%%%%%%%%%%%%%%%%%%%%%%%%%%%
\subsection{Conditional Processing}
\label{sec:conditional}

The package provides a mechanism to compile different versions
of a document. To customise the versions further some conditional processing
can come in handy to distinguish which version is being compiled.
The package provides two macros to describe the compilation context:

%%%%%%%%%%%%%%%%%%%%%%%%%%%%%%%%%%%%%%%%
\DescribeMacro{\ifchilddoc}
The conditional |\ifchilddoc| distinguishes between the compilation of
child documents and the main document:
%
\begin{center}
|\ifchilddoc |\textit{child-code}| |[|\||else |\textit{main-code}]| \||fi|
\end{center}

%%%%%%%%%%%%%%%%%%%%%%%%%%%%%%%%%%%%%%%%
\DescribeMacro{\childdocname}
\DescribeMacro{\childdocjob}
The macro |\childdocname| contains the filename (without extension)
of the main or child file being processed.
Note that |\childdocjob| will always contain the name of the main file.

%%%%%%%%%%%%%%%%%%%%%%%%%%%%%%%%%%%%%%%%
\paragraph{Title Page.}

Conditional processing can be used to include a title or banner page
in the main document when proper precautions are taken.
Importantly, the code in the main file should ensure that the page counter
(as well as other status parameters which are stored in the |.aux| files)
takes the same value after the conditional processing.
Otherwise the page numbers may take divergent values
depending on which part is compiled.

For example, a title page could be declared by:
%
\begin{center}
\begin{tabular}{l}
|\ifchilddoc\||else|\\
|\addtocounter{page}{-1}|\\
\textit{code for title page}\\
|\newpage|\\
|\||fi|
\end{tabular}
\end{center}
%
A banner page for the child documents can be generated by:
%
\begin{center}
\begin{tabular}{l}
|\ifchilddoc|\\
|\addtocounter{page}{-1}|\\
\textit{code for banner page}\\
|\newpage|\\
|\||fi|
\end{tabular}
\end{center}
%
Here one could write a message such as:
\begin{center}
|This is the part \childdocname{} of \childdocjob{}.|
\end{center}

%%%%%%%%%%%%%%%%%%%%%%%%%%%%%%%%%%%%%%%%%%%%%%%%%%%%%%%%%%%%%%%%%%%%%%%%%%%%%%%%
\subsection{Flags}
\label{sec:flags}

The package makes it easy to generate different versions
of the main or child documents.
To this end compilation flags can be defined
and assigned different default values.
They will be particularly useful in conjunction
with the forwarding mechanism described in \secref{sec:forward}.

For example, it may be useful to have a flag |\version|
which can be set to |draft| or |final|.
The document source will contain some conditional code
depending on the value of |\version|.
Suppose further, the flag should default to |final| for the main file
and to |draft| for child files
which is a natural assignment for editing the document.
This is achieved by placing the following code
in the preamble of the main document
(below the |\childdocmain| directive):
%
\begin{center}
\begin{tabular}{l}
|\ifchilddoc|\\
|\providecommand{\version}{draft}|\\
|\||else|\\
|\providecommand{\version}{final}|\\
|\||fi|
\end{tabular}
\end{center}
%
The definition by |\providecommand| makes sure
that previous definitions are not overwritten.
Further statements |\providecommand{\version}{...}|
can thus be added before the above code to override it.

For the main file, one might add a line
(between |\childdocmain| and the above block)
%
\begin{center}
|%\ifchilddoc\||else\providecommand{\version}{draft}\||fi|
\end{center}
%
which can be uncommented to produce a draft version.
Likewise one can add a line to the very top of a child file
(above the |\childdocof{|\textit{main}|}| directive)
%
\begin{center}
|%\providecommand{\version}{final}|
\end{center}
%
which can be uncommented to produce the final version of this child document.

%%%%%%%%%%%%%%%%%%%%%%%%%%%%%%%%%%%%%%%%%%%%%%%%%%%%%%%%%%%%%%%%%%%%%%%%%%%%%%%%
\subsection{Forwarding}
\label{sec:forward}

Different versions of the main or child documents
using compilation flags as described in \secref{sec:flags}
can be (permanently) stored in different files
for convenient compilation, viewing and distribution.
To this end, the package defines a command
to pass on compilation to a different file:

%%%%%%%%%%%%%%%%%%%%%%%%%%%%%%%%%%%%%%%%
\DescribeMacro{\childdocforward}
The command |\childdocforward| redirects processing to
another source file:
%
\begin{center}
\begin{tabular}{l}
|\input{childdoc.def}|\\
|\childdocforward[|\textit{main}|]{|\textit{dest}|}|\\
\end{tabular}
\end{center}
%
The argument \textit{dest} is the destination file
(without extension).
It should be the main file or one of the child files.
Note that further \textsf{childdoc} directives
such as |\childdocof| and |\childdocforward|
in the indicated file will be processed in this form.
The optional argument \textit{main}
passes on directly to the main file \textit{main}
while pretending to compile the child \textit{dest}.
This form behaves as if \textit{dest}
issues |\childdocof{|\textit{main}|}| right away,
and no further \textsf{childdoc} directives will be processed.

%%%%%%%%%%%%%%%%%%%%%%%%%%%%%%%%%%%%%%%%
\DescribeMacro{\...prefix}
In the alternative form |\childdocforwardprefix|,
%
\begin{center}
\begin{tabular}{l}
|\input{childdoc.def}|\\
|\childdocforwardprefix[|\textit{main}|]{|\textit{prefix}|}{|\textit{dest}|}|
\end{tabular}
\end{center}
%
the destination file is determined by a pattern
depending on the current file:
To make this work, the current file must be called
`{\textit{prefix}\hspace{0.2em}\textit{suffix}}'
with \textit{prefix} matching precisely the argument.
Processing is then passed on to the file
`{\textit{dest}\hspace{0.2em}\textit{suffix}}'.
Surely, the same effect is achieved by
directly specifying the
argument `{\textit{dest}\hspace{0.2em}\textit{suffix}}'
in the first form.
However, that requires to set up a different file
for each child. With the alternative form of the command
all these files can have exactly the same content
which simplifies setting them up and maintaining them.

For example, the following file |draft.tex|
with a compilation flag |\version| as described in \secref{sec:flags}
compiles the main document as a draft:
%
\begin{center}
\begin{tabular}{l}
|\def\version{draft}|\\
|\input{childdoc.def}|\\
|\childdocforward{|\textit{main}|}|
\end{tabular}
\end{center}
%
Likewise, the following files |final|\textit{nn}|.tex|
compile the final version of the child document
|child|\textit{nn}|.tex|:
%
\begin{center}
\begin{tabular}{l}
|\def\version{final}|\\
|\input{childdoc.def}|\\
|\childdocforwardprefix{final}{child}|
\end{tabular}
\end{center}
%

Note that when several versions of a main file and/or of each child file
are to be generated, it may be convenient to set up a |Makefile| or
shell script to automatise the process.

%%%%%%%%%%%%%%%%%%%%%%%%%%%%%%%%%%%%%%%%%%%%%%%%%%%%%%%%%%%%%%%%%%%%%%%%%%%%%%%%
\subsection{Command Line Processing}
\label{sec:commandline}

The effect of redirection files can also be achieved by invoking
the \LaTeX{} compiler with a more elaborate command line.
Most conveniently this should be done as part
of a shell script or a |Makefile|.

When using \textsf{childdoc} in the main file, the following
command lines effectively perform a redirection
(note that depending on the shell being used,
backslashes may have to be doubled: `|\|' $\to$ `|\\|'):
%
\begin{center}
|... -jobname "|\textit{target}|" |\\|"|[\textit{flags}]%
|\input{childdoc.def}\childdocforward[|\textit{main}|]{|\textit{dest}|}"|
\end{center}
%
Here \textit{target} is the name of the output file,
\textit{main} is the name of the main file
and \textit{dest} is the name of the main or child file to be processed
(all filenames without extensions).
The optional argument \textit{main} can be omitted
if \textit{main} matches \textit{dest}.
Optionally, compilation \textit{flags} can be defined via |\def| commands.
This command line makes the \TeX{} engine believe
it is compiling the file \textit{target}
whose content is specified as the latter parameter.
The provided code then forwards the processing to
\textit{main} or \textit{dest} as described in \secref{sec:forward}.

%%%%%%%%%%%%%%%%%%%%%%%%%%%%%%%%%%%%%%%%%%%%%%%%%%%%%%%%%%%%%%%%%%%%%%%%%%%%%%%%
\subsection{Include by Input}
\label{sec:input}

Including child documents by |\include| has some restrictions by design.
Most notably, the content of a child document always occupies
its own set of pages; pages cannot be shared between child documents.
Usually, this behaviour makes perfect sense
because each child document contain an essential part of the document.
However, in some situations it may be desirable to compose
a document from a collection of parts
without having mandatory page breaks between then.
For this case, the package
provides a mechanism to include parts
by |\input| which can also be processed individually.
However, by construction this mechanism
requires manual handling of the content to be output.

%%%%%%%%%%%%%%%%%%%%%%%%%%%%%%%%%%%%%%%%
\DescribeMacro{\ifchilddocmanual}
The main file should be prepared as usual, see \secref{sec:include}.
However, the document body must make a distinction
between processing of an individual part and of the main document, e.g.:
%
\begin{center}
\begin{tabular}{l}
|\ifchilddocmanual|\\
|\input{\childdocname}|\\
|\||else|\\
\textit{document body with }|\input{|\textit{part}|}|\\
|\||fi|
\end{tabular}
\end{center}
%
The conditional |\ifchilddocmanual| is true whenever
a part to be included by |\input| is being compiled,
and the name of the part is stored in |\childdocname|.

%%%%%%%%%%%%%%%%%%%%%%%%%%%%%%%%%%%%%%%%
\DescribeMacro{\childdocby}
Each part to be included by |\input| should start with:
%
\begin{center}
\begin{tabular}{l}
|\input{childdoc.def}|\\
|\childdocby{|\textit{main}|}|\\
\end{tabular}
\end{center}
%
The directive |\childdocby| is similar to |\childdocof|
described in \secref{sec:include},
but the subsequent selection of content must be done manually.
To that end, both |\ifchilddoc| and |\ifchilddocmanual|
will be true upon processing of a part,
and the name of the part is stored in |\childdocname|.
Note that |\jobname| will be set to the filename of the current part
so that each part receives an individual |.aux| file
that does not interfere with the |.aux| file(s) of the main document.
This behaviour can be altered by the alternative form
|\childdocby[*]{|\textit{main}|}| (with a non-empty optional argument)
which uses the |.aux| file of the main document
by setting |\jobname| to \textit{main}.

%%%%%%%%%%%%%%%%%%%%%%%%%%%%%%%%%%%%%%%%%%%%%%%%%%%%%%%%%%%%%%%%%%%%%%%%%%%%%%%%
\subsection{Driver Development}
\label{sec:driver}

The \textsf{childdoc} mechanism can also be use for the development
of definition files such as \LaTeX{} styles or classes.
This case differs from the above setup with multiple parts
included by |\include| in that no |\includeonly| should be invoked.
This can be achieved by starting the include file
(before |\ProvidesPackage|) with:
%
\begin{center}
\begin{tabular}{l}
|\input{childdoc.def}|\\
|\childdocforward{|\textit{main}|}|\\
\end{tabular}
\end{center}
%
or alternatively with:
%
\begin{center}
\begin{tabular}{l}
|\input{childdoc.def}|\\
|\childdocby{|\textit{main}|}|\\
\end{tabular}
\end{center}
%
Both forms have slightly different effects as described above.
The main file is prepared as usual, see \secref{sec:include}.

%%%%%%%%%%%%%%%%%%%%%%%%%%%%%%%%%%%%%%%%%%%%%%%%%%%%%%%%%%%%%%%%%%%%%%%%%%%%%%%%
\subsection{Legacy Detection}
\label{sec:detection}

The directive |\childdocmain| in the main file can detect
whether the complete document or merely a child is to be compiled
even without using the directive |\childdocof|.
This method is deprecated because it is less robust
and there is no compelling reason to use it;
it is merely provided for backward compatibility
and it may be removed in future versions.

If the detection mechanism is to be used,
it is mandatory to correctly specify
the filename of the main file as the argument of |\childdocmain|:
%
\begin{center}
\begin{tabular}{l}
|\input{childdoc.def}|\\
|\childdocmain{|\textit{main}|}|\\
\end{tabular}
\end{center}
%
If |\jobname| does not match the argument \textit{main} of |\childdocmain|,
it is assumed that |\jobname| points to the child file to be compiled.
When using |\childdocmain| with the main file specified as argument,
it suffices to start a child file
with just |\input{|\textit{main}|}|
without loading of the package and using |\childdocof|.
If instead all processing is done
with the appropriate \textsf{childdoc} directives,
the argument of \textit{main} of |\childdocmain| can be empty.

An alternative version of the command line processing described
in \secref{sec:commandline} using the detection mechanism reads:
%
\begin{center}
|... -jobname "|\textit{target}|" "|[\textit{flags}]%
[|\def\jobname{|\textit{dest}|}|]|\input{|\textit{main}|}"|
\end{center}

%%%%%%%%%%%%%%%%%%%%%%%%%%%%%%%%%%%%%%%%%%%%%%%%%%%%%%%%%%%%%%%%%%%%%%%%%%%%%%%%
\subsection{Manual Code}
\label{sec:manual}

In case one cannot be certain whether the definitions file |childdoc.def|
is installed on the target \TeX{} distribution
and one prefers not to ship it,
it is conceivable to paste a few relevant commands into the sources.

To that end, drop all statements |\input{childdoc.def}|
and perform the replacements as outlined below.
Instead of |\childdocmain{|\textit{main}|}| add the following code
to the top of the main file:
%
\begin{center}
\begin{tabular}{l}
|\||ifdefined\childdocname\endinput\||fi\newif\ifchilddoc|\\
|\edef\childdocname{\scantokens\expandafter{\jobname\noexpand}}|\\
|\def\childdocmain{|\textit{main}|}\||ifx\childdocmain\childdocname\||else|\\
|\childdoctrue\includeonly{\childdocname}\let\jobname\childdocmain\||fi|\\
\end{tabular}
\end{center}
%
Instead of |\childdocof{|\textit{main}|}| just include the main file
at the top of each child file:
%
\begin{center}
|\input{|\textit{main}|}|
\end{center}
%
A simple redirection |\childdocforward{|\textit{dest}|}| is achieved by:
%
\begin{center}
|\def\jobname{|\textit{dest}|}\input{\jobname}|
\end{center}
%
The redirection with prefix
|\childdocforwardprefix[|\textit{prefix}|]{|\textit{dest}|}|
is accomplished by:
%
\begin{center}
\begin{tabular}{l}
|{\edef\jobname{\scantokens\expandafter{\jobname\noexpand}}|\\
|\def\redirectjob |\textit{prefix}|#1~~~{\gdef\jobname{|\textit{dest}|#1}}|\\
|\expandafter\redirectjob\jobname~~~}\input{\jobname}|
\end{tabular}
\end{center}

In an alternative approach,
child documents can be compiled by a specific command line
without additional code or specific definitions:
%
\begin{center}
|... -jobname "|\textit{target}|" "|[\textit{flags}]%
|\includeonly{|\textit{dest}|}\input{|\textit{main}|}"|
\end{center}
%

%%%%%%%%%%%%%%%%%%%%%%%%%%%%%%%%%%%%%%%%%%%%%%%%%%%%%%%%%%%%%%%%%%%%%%%%%%%%%%%%
%%%%%%%%%%%%%%%%%%%%%%%%%%%%%%%%%%%%%%%%%%%%%%%%%%%%%%%%%%%%%%%%%%%%%%%%%%%%%%%%
\section{Information}

%%%%%%%%%%%%%%%%%%%%%%%%%%%%%%%%%%%%%%%%%%%%%%%%%%%%%%%%%%%%%%%%%%%%%%%%%%%%%%%%
\subsection{Copyright}

Copyright \copyright{} 2017--2018 Niklas Beisert

This work may be distributed and/or modified under the
conditions of the \LaTeX{} Project Public License, either version 1.3
of this license or (at your option) any later version.
The latest version of this license is in
  \url{http://www.latex-project.org/lppl.txt}
and version 1.3 or later is part of all distributions of \LaTeX{}
version 2005/12/01 or later.

This work has the LPPL maintenance status `maintained'.

The Current Maintainer of this work is Niklas Beisert.

This work consists of the files |README.txt|, |childdoc.ins| and |childdoc.dtx|
as well as the derived files |childdoc.def|, |cdocsamp.tex|
with |cdocsch1.tex|, |cdocsch2.tex|, |cdocspt3.tex|, |cdocspt4.tex|,
|cdocsdrf.tex|, |cdocsfn1.tex|, |cdocsfn2.tex|
as well as |childdoc.pdf|.

%%%%%%%%%%%%%%%%%%%%%%%%%%%%%%%%%%%%%%%%%%%%%%%%%%%%%%%%%%%%%%%%%%%%%%%%%%%%%%%%
\subsection{Files and Installation}

The package consists of the files:
%
\begin{center}
\begin{tabular}{ll}
    |README.txt|   & readme file \\
    |childdoc.ins| & installation file \\
    |childdoc.dtx| & source file \\
    |childdoc.def| & definition file \\
    |cdocsamp.tex| & sample main file \\
    |cdocsch1.tex| & sample include file \\
    |cdocsch2.tex| & sample include file \\
    |cdocspt3.tex| & sample part file \\
    |cdocspt4.tex| & sample part file \\
    |cdocsdrf.tex| & sample redirection file \\
    |cdocsfn1.tex| & sample redirection file \\
    |cdocsfn2.tex| & sample redirection file \\
    |childdoc.pdf| & manual
\end{tabular}
\end{center}
%
The distribution consists of the files
|README.txt|, |childdoc.ins| and |childdoc.dtx|.
%
\begin{itemize}
\item
Run (pdf)\LaTeX{} on |childdoc.dtx|
to compile the manual |childdoc.pdf| (this file).
\item
Run \LaTeX{} on |childdoc.ins| to create the definitions file |childdoc.def|
and the sample |cdocsamp.tex| with include files
|cdocsch1.tex|, |cdocsch2.tex|, |cdocspt3.tex|, |cdocspt4.tex|,
|cdocsdrf.tex|, |cdocsfn1.tex|, |cdocsfn2.tex|.
Then copy the file |childdoc.def| to an appropriate directory of your \LaTeX{}
distribution, e.g.\ \textit{texmf-root}|/tex/latex/childdoc|.
\end{itemize}

%%%%%%%%%%%%%%%%%%%%%%%%%%%%%%%%%%%%%%%%%%%%%%%%%%%%%%%%%%%%%%%%%%%%%%%%%%%%%%%%
\subsection{Related CTAN Packages}

There are several other packages which offer a similar functionality:
%
\begin{itemize}
\item
The packages
\href{http://ctan.org/pkg/docmute}{\textsf{docmute}},
\href{http://ctan.org/pkg/includex}{\textsf{includex}} and
\href{http://ctan.org/pkg/standalone}{\textsf{standalone}}
provide commands to include only the document body of
a child file thus allowing both files to be compiled individually.
\item
The packages \href{http://ctan.org/pkg/subdocs}{\textsf{subdocs}}
and \href{http://ctan.org/pkg/subfiles}{\textsf{subfiles}}
provide structures in which the main and child documents can be
encapsulated and allowing them to be compiled individually.
The inclusion mechanism is different from the conventional |\include|.
\item
The package \href{http://ctan.org/pkg/combine}{\textsf{combine}}
is an elaborate solution to combine several documents into one.
\end{itemize}
%
See also the CTAN topic \href{http://ctan.org/topic/subdocs}{\textsf{subdocs}}
for further related packages.
The present package differs from the above solutions in that
a document structure constructed with the conventional |\include| mechanism
just needs two extra commands at the top of every file
such that all constituent files can be compiled individually.

%%%%%%%%%%%%%%%%%%%%%%%%%%%%%%%%%%%%%%%%%%%%%%%%%%%%%%%%%%%%%%%%%%%%%%%%%%%%%%%%
%\subsection{Feature Suggestions}
%
%The following is a list of features which may be useful for future
%versions of this package:
%%
%\begin{itemize}
%\item
%\ldots
%\end{itemize}

%%%%%%%%%%%%%%%%%%%%%%%%%%%%%%%%%%%%%%%%%%%%%%%%%%%%%%%%%%%%%%%%%%%%%%%%%%%%%%%%
\subsection{Revision History}

%%%%%%%%%%%%%%%%%%%%%%%%%%%%%%%%%%%%%%%%
\paragraph{v2.0:} 2018/12/30

\begin{itemize}
\item
immediate forward processing
\item
added |\childdocby| mechanism
\item
manual restructured
\end{itemize}

%%%%%%%%%%%%%%%%%%%%%%%%%%%%%%%%%%%%%%%%
\paragraph{v1.6:} 2018/01/17

\begin{itemize}
\item
application for development of include files
\item
corrections to manual
\end{itemize}

%%%%%%%%%%%%%%%%%%%%%%%%%%%%%%%%%%%%%%%%
\paragraph{v1.5:} 2017/05/21

\begin{itemize}
\item
more complete structuring introduced
\item
|\childdocof| introduced
\item
|\childdoc| renamed to |\childdocmain|
\item
|\childredirect| renamed to |\childdocforward| and |\childdocforwardprefix|
and functionality expanded
\end{itemize}

%%%%%%%%%%%%%%%%%%%%%%%%%%%%%%%%%%%%%%%%
\paragraph{v1.0:} 2017/04/27

\begin{itemize}
\item
manual and install package
\item
first version published on CTAN
\end{itemize}

%%%%%%%%%%%%%%%%%%%%%%%%%%%%%%%%%%%%%%%%
\paragraph{v0.6:} 2017/04/26

\begin{itemize}
\item
redirection mechanism added
\end{itemize}

%%%%%%%%%%%%%%%%%%%%%%%%%%%%%%%%%%%%%%%%
\paragraph{v0.5:} 2017/04/26

\begin{itemize}
\item
functionality in definition file
\end{itemize}


%%%%%%%%%%%%%%%%%%%%%%%%%%%%%%%%%%%%%%%%%%%%%%%%%%%%%%%%%%%%%%%%%%%%%%%%%%%%%%%%
%%%%%%%%%%%%%%%%%%%%%%%%%%%%%%%%%%%%%%%%%%%%%%%%%%%%%%%%%%%%%%%%%%%%%%%%%%%%%%%%
%%%%%%%%%%%%%%%%%%%%%%%%%%%%%%%%%%%%%%%%%%%%%%%%%%%%%%%%%%%%%%%%%%%%%%%%%%%%%%%%
\appendix

\settowidth\MacroIndent{\rmfamily\scriptsize 000\ }

 \DocInput{childdoc.dtx}

\end{document}
%</driver>
% \fi
%
% %%%%%%%%%%%%%%%%%%%%%%%%%%%%%%%%%%%%%%%%%%%%%%%%%%%%%%%%%%%%%%%%%%%%%%%%%%%%%%
% %%%%%%%%%%%%%%%%%%%%%%%%%%%%%%%%%%%%%%%%%%%%%%%%%%%%%%%%%%%%%%%%%%%%%%%%%%%%%%
% \section{Sample}
%\iffalse
%<*samplemain>
%\fi
%
% The following presents a sample document
% with two chapters, two parts, a title page,
% a compile flag as well as three forwarding files to set the flag.
% It consists of eight |.tex| files:
% \begin{center}
% \begin{tabular}{ll}
% |cdocsamp.tex|&main file\\
% |cdocsch1.tex|&include file for chapter 1\\
% |cdocsch2.tex|&include file for chapter 2\\
% |cdocspt3.tex|&include file for part 3\\
% |cdocspt4.tex|&include file for part 4\\
% |cdocsdrf.tex|&forwarding file for main file in draft mode\\
% |cdocsfi1.tex|&forwarding file for final version of chapter 1\\
% |cdocsfi2.tex|&forwarding file for final version of chapter 2\\
% \end{tabular}
% \end{center}
% Each of the eight files can be compiled directly by the \LaTeX{} compiler.
%
% %%%%%%%%%%%%%%%%%%%%%%%%%%%%%%%%%%%%%%
% \paragraph{Main File.}
%
% The main file is called |cdocsamp.tex|.
%
% Load the \textsf{childdoc} definitions and
% declare the filename for the main document:
%    \begin{macrocode}
\input{childdoc.def}
\childdocmain{}
%    \end{macrocode}

% Optional override for |\version| flag:
%    \begin{macrocode}
%%\ifchilddoc\else\providecommand{\version}{draft}\fi
%    \end{macrocode}

% Define the default values for the |\version| flag
% (|final| for the main file and |draft| for childs):
%    \begin{macrocode}
\ifchilddoc
\providecommand{\version}{draft}
\else
\providecommand{\version}{final}
\fi
%    \end{macrocode}

% Load the standard document class:
%    \begin{macrocode}
\documentclass[12pt]{article}
%    \end{macrocode}

% Start the document body:
%    \begin{macrocode}
\begin{document}
%    \end{macrocode}

% Declare a title page.
% Print title, part of document being processed and version flag:
%    \begin{macrocode}
\addtocounter{page}{-1}
\begin{center}
{\LARGE\bfseries{}childdoc example\par}
\vspace{1cm}
\ifchilddoc
\ifchilddocmanual part\else chapter\fi:
`\childdocname' of `\childdocjob'\par
\else
main document: `\childdocjob'\par
\fi
version: \version\par
\end{center}
\newpage
%    \end{macrocode}

% Manually include selected file,
% otherwise process as usual:
%    \begin{macrocode}
\ifchilddocmanual
\section*{part `\childdocname'}
\input{\childdocname}
\else
%    \end{macrocode}

% Include the two chapters:
%    \begin{macrocode}
\include{cdocsch1}
\include{cdocsch2}
%    \end{macrocode}

% Include the two parts unless only chapters should be displayed:
%    \begin{macrocode}
\ifchilddoc\else
\section{part three}
\input{cdocspt3}
\section{part four}
\input{cdocspt4}
\fi
%    \end{macrocode}

% Process as usual until here:
%    \begin{macrocode}
\fi
%    \end{macrocode}

% End of document body:
%    \begin{macrocode}
\end{document}
%    \end{macrocode}
%\iffalse
%</samplemain>
%\fi
%
% %%%%%%%%%%%%%%%%%%%%%%%%%%%%%%%%%%%%%%
% \paragraph{Chapter Include Files.}
%
% The include files are called |cdocsch1.tex| and |cdocsch2.tex|.
%
%\iffalse
%<*samplechap1|samplechap2>
%\fi

% Optional override for |\version| flag:
%    \begin{macrocode}
%%\providecommand{\version}{final}
%    \end{macrocode}

% Include the main document:
%    \begin{macrocode}
\input{childdoc.def}
\childdocof{cdocsamp}
%    \end{macrocode}

%\iffalse
%</samplechap1|samplechap2>
%\fi
%
%\iffalse
%<*samplechap1>
%\fi
% Some text for chapter 1:
%    \begin{macrocode}
\section{one}
some text in chapter one
%    \end{macrocode}

%\iffalse
%</samplechap1>
%\fi
% Some text for chapter 2:
%\iffalse
%<*samplechap2>
%\fi
%    \begin{macrocode}
\section{two}
more text in chapter two
%    \end{macrocode}

%\iffalse
%</samplechap2>
%\fi
%
% %%%%%%%%%%%%%%%%%%%%%%%%%%%%%%%%%%%%%%
% \paragraph{Part Include Files.}
%
% The include files are called |cdocspt3.tex| and |cdocspt4.tex|.
%
%\iffalse
%<*samplepart3|samplepart4>
%\fi

% Optional override for |\version| flag:
%    \begin{macrocode}
%%\providecommand{\version}{final}
%    \end{macrocode}

% Include the main document:
%    \begin{macrocode}
\input{childdoc.def}
\childdocby{cdocsamp}
%    \end{macrocode}

%\iffalse
%</samplepart3|samplepart4>
%\fi
%
%\iffalse
%<*samplepart3>
%\fi
% Some text for part 3:
%    \begin{macrocode}
some text in part three
%    \end{macrocode}

%\iffalse
%</samplepart3>
%\fi
% Some text for part 4:
%\iffalse
%<*samplepart4>
%\fi
%    \begin{macrocode}
more text in part four
%    \end{macrocode}

%\iffalse
%</samplepart4>
%\fi
%
% %%%%%%%%%%%%%%%%%%%%%%%%%%%%%%%%%%%%%%
% \paragraph{Forwarding for a Complete Draft.}
%
% The following forwarding file |cdocsdrf.tex|
% compiles the main document in draft mode:
%\iffalse
%<*sampledraft>
%\fi
%    \begin{macrocode}
\def\version{draft}
\input{childdoc.def}
\childdocforward{cdocsamp}
%    \end{macrocode}

%\iffalse
%</sampledraft>
%\fi
%
% %%%%%%%%%%%%%%%%%%%%%%%%%%%%%%%%%%%%%%
% \paragraph{Forwarding for Final Version of the Chapters.}
%
% The following forwarding files |cdocsfn1.tex| and |cdocsfn2.tex|
% (with identical content)
% compile the final versions of the child documents
% |cdocsch1.tex| and |cdocsch2.tex|, respectively:
%\iffalse
%<*samplefinal>
%\fi
%    \begin{macrocode}
\def\version{final}
\input{childdoc.def}
\childdocforwardprefix[cdocsamp]{cdocsfn}{cdocsch}
%    \end{macrocode}

%\iffalse
%</samplefinal>
%\fi
%
% %%%%%%%%%%%%%%%%%%%%%%%%%%%%%%%%%%%%%%
% \paragraph{Command Line Processing.}
%
% The following three command lines generate the output files
% |cdocscld|, |cdocscl1| and |cdocscl2|
% which should be identical to
% |cdocsdrf|, |cdocsch1| and |cdocsfn2|, respectively:
% \begin{center}
% \begin{tabular}{l}
% |latex -jobname cdocscld \|\\
% |  "\def\version{draft}\input{childdoc.def}\childdocforward{cdocsamp}"|\\
% |latex -jobname cdocscl1 \|\\
% |  "\input{childdoc.def}\childdocforward[cdocsamp]{cdocsch1}"|\\
% |latex -jobname cdocscl2 \|\\
% |  "\def\version{final}\input{childdoc.def}\childdocforward{cdocsch2}"|
% \end{tabular}
% \end{center}
% Note that the trailing backslash on each first line
% merely continues the input to the second line
% (for convenient cut ant paste).
% Furthermore, the command |latex| can be replaced by any
% of its alternative versions such as |pdflatex|.
%
% %%%%%%%%%%%%%%%%%%%%%%%%%%%%%%%%%%%%%%%%%%%%%%%%%%%%%%%%%%%%%%%%%%%%%%%%%%%%%%
% %%%%%%%%%%%%%%%%%%%%%%%%%%%%%%%%%%%%%%%%%%%%%%%%%%%%%%%%%%%%%%%%%%%%%%%%%%%%%%
% \section{Implementation}
%\iffalse
%<*package>
%\fi
%
% This section describes the definitions file |childdoc.def|.

% The definitions cannot be loaded using |\usepackage| or |\RequirePackage|
% which has a mechanism to prevent loading a style file more than once.
% When loading the definitions by means of |\input|
% multiple instances have to be prevented manually:
%\iffalse
%This code needs to be before the `\ProvidesFile' directive
%which is defined at the beginning of this file.
%Therefore it is also placed there and commented out here.
%</package>
%<*discard>
%\fi
%    \begin{macrocode}
\ifdefined\childdocmain\endinput\fi
%    \end{macrocode}
%\iffalse
%</discard>
%<*package>
%\fi
%
% \macro{\ifchilddoc}
% \macro{\ifchilddocmanual}
% The conditional |\ifchilddoc| tells whether a
% child (true) or main (false) document is being compiled.
% The conditional |\ifchilddocmanual| tells whether
% the |\includeonly| mechanism is used (false) or
% the selection of child files must be performed manually (true).
% The definitions initialise to false:
%    \begin{macrocode}
\newif\ifchilddoc
\newif\ifchilddocmanual
%    \end{macrocode}

% \macro{\childdocname}
% \macro{\childdocjob}
% The macro |\childdocname| stores the name of the main document
% to be compiled. The macro |\childdocjob| stores the name of
% the document on which the \LaTeX{} compiler was originally invoked.
% The content of |\jobname| cannot be compared
% to filenames specified in the source due to different catcodes.
% The following code rescans |\jobname|, stores the result
% in |\childdocname| and saves a copy in |\childdocjob|:
%    \begin{macrocode}
\edef\childdocname{\scantokens\expandafter{\jobname\noexpand}}
\let\childdocjob\childdocname
%    \end{macrocode}

% \macro{\childdocdisable}
% The macro |\childdocdisable| prevents the main file
% from being processed more than once.
% At this stage, the main document command |\childdocmain|
% is assumed to be called once again where it should do nothing.
% Any subsequent call to it should prevent
% a secondary processing of the main document
% It overwrites the forwarding commands
% |\childdocof| and |\childdocforward|
% with empty macros to prevent further inclusions of the main document:
%    \begin{macrocode}
\newcommand{\childdocdisable}
{
  \renewcommand{\childdocmain}[1]{\renewcommand{\childdocmain}[1]{\endinput}}
  \renewcommand{\childdocof}[1]{}
  \renewcommand{\childdocby}[2][]{}
  \renewcommand{\childdocforward}[2][]{}
  \renewcommand{\childdocdisable}{}
}
%    \end{macrocode}

% \macro{\childdocmain}
% The macro |\childdocmain| is to be called at the top of the main file
% with nothing or the main filename (without extension) as argument.
% First, it breaks loops.
% If the argument is not empty and does not match |\childdocname|
% (which is set by the first inclusion of |childdoc.def|),
% |\ifchilddoc| is set to true, |\includeonly| is applied to the child file
% and |\jobname| is set to the main file
% (for proper handling of |.aux| files):
%    \begin{macrocode}
\newcommand{\childdocmain}[1]
{
  \childdocdisable\childdocmain{}
  \if?#1?\else
    \begingroup
      \def\childdoctmp{#1}
      \ifx\childdoctmp\childdocname
        \def\childdoctmp{}
      \else
        \def\childdoctmp
        {
          \childdoctrue
          \includeonly{\childdocname}
          \def\childdocjob{#1}
          \def\jobname{#1}
        }
      \fi
      \expandafter
    \endgroup
    \childdoctmp
  \fi
}
%    \end{macrocode}

% \macro{\childdocof}
% The command |\childdocof| redirects
% compilation to the main file |#1|.
%    \begin{macrocode}
\newcommand{\childdocof}[1]
{
  \childdocdisable
  \childdoctrue
  \includeonly{\childdocname}
  \def\jobname{#1}
  \def\childdocjob{#1}
  \input{#1}
}
%    \end{macrocode}

% \macro{\childdocby}
% The command |\childdocby| ....
%    \begin{macrocode}
\newcommand{\childdocby}[2][]
{
  \childdocdisable
  \childdoctrue
  \childdocmanualtrue
  \if?#1?\else
    \def\jobname{#2}
  \fi
  \def\childdocjob{#2}
  \input{#2}
  \endinput
}
%    \end{macrocode}

% \macro{\childdocforward}
% The command |\childdocforward| redirects
% compilation to the main file or
% (if the optional argument is given) a child file.
% Parameters are set as if the main file
% or a child file starting with |\childdocof| was compiled.
% Then compilation is handed over to the main file:
%    \begin{macrocode}
\newcommand{\childdocforward}[2][]
{
  \begingroup
    \if?#1?
      \def\childdoctmp
      {
        \def\childdocname{#2}
        \def\childdocjob{#2}
        \def\jobname{#2}
        \input{#2}
        \endinput
      }
    \else
      \def\childdoctmp
      {
        \childdocdisable
        \def\childdocname{#2}
        \childdoctrue
        \includeonly{#2}
        \def\childdocjob{#1}
        \def\jobname{#1}
        \input{#1}
        \endinput
      }
    \fi
    \expandafter
  \endgroup
  \childdoctmp
}
%    \end{macrocode}

% \macro{\childdocforwardprefix}
% The command |\childdocforwardprefix| redirects
% compilation to the main or a child file by means of a pattern.
% The prefix |#1| in the current filename is replaced by |#2|
% and the suffix of the current filename is kept
% (it is assumed that the filename does not contain the substring `|~~~|'
% which is used as a delimiter).
% Compilation is handed over to the new file by |\childdocforward|:
%    \begin{macrocode}
\newcommand{\childdocforwardprefix}[3][]
{
  \begingroup
    \def\childdocextract #2##1~~~{\def\childdoctmp{\childdocforward[#1]{#3##1}}}
    \expandafter\childdocextract\childdocname~~~
    \expandafter
  \endgroup
  \childdoctmp
}
%    \end{macrocode}

% \macro{\childdoc}
% The deprecated macro |\childdoc| is a legacy version of |\childdocmain|:
%    \begin{macrocode}
\newcommand{\childdoc}{\childdocmain}
%    \end{macrocode}

% \macro{\childdocredirect}
% The deprecated macro |\childdocredirect| is a legacy version
% of |\childdocforward| and |\childdocforwardprefix|:
%    \begin{macrocode}
\newcommand{\childdocredirect}[2][]
{
  \begingroup
    \if?#1?
      \def\childdoctmp{\childdocforward{#2}}
    \else
      \def\childdoctmp{\childdocforwardprefix{#1}{#2}}
    \fi
    \expandafter
  \endgroup
  \childdoctmp
}
%    \end{macrocode}

%\iffalse
%</package>
%\fi
%
\endinput
\childdocforward{cdocsamp}"|\\
% |latex -jobname cdocscl1 \|\\
% |  "% \iffalse
%
% childdoc.dtx Copyright (C) 2017-2018 Niklas Beisert
%
% This work may be distributed and/or modified under the
% conditions of the LaTeX Project Public License, either version 1.3
% of this license or (at your option) any later version.
% The latest version of this license is in
%   http://www.latex-project.org/lppl.txt
% and version 1.3 or later is part of all distributions of LaTeX
% version 2005/12/01 or later.
%
% This work has the LPPL maintenance status `maintained'.
%
% The Current Maintainer of this work is Niklas Beisert.
%
% This work consists of the files childdoc.dtx and childdoc.ins
% and the derived files childdoc.def and cdocsamp.tex with
% cdocsch1.tex, cdocsch2.tex, cdocsdrf.tex, cdocsfn1.tex, cdocsfn2.tex.
%
%<package>\ifdefined\childdocmain\endinput\fi
%<package>\ProvidesFile{childdoc.def}[2018/12/30 v2.0 child document driver]
%<samplemain>\ProvidesFile{cdocsamp.tex}[2018/12/30 v2.0 sample for childdoc]
%<*driver>
%\ProvidesFile{childdoc.drv}[2018/12/30 v2.0 childdoc reference manual file]
\PassOptionsToClass{10pt,a4paper}{article}
\documentclass{ltxdoc}

\usepackage[margin=35mm]{geometry}
\usepackage{hyperref}
\usepackage{hyperxmp}
\usepackage[usenames]{color}

\hypersetup{colorlinks=true}
\hypersetup{pdfstartview=FitH}
\hypersetup{pdfpagemode=UseNone}
\hypersetup{pdfsource={}}
\hypersetup{pdflang={en-UK}}
\hypersetup{pdfcopyright={Copyright 2017-2018 Niklas Beisert.
  This work may be distributed and/or modified under the
  conditions of the LaTeX Project Public License, either version 1.3
  of this license or (at your option) any later version.}}
\hypersetup{pdflicenseurl={http://www.latex-project.org/lppl.txt}}
\hypersetup{pdfcontactaddress={ETH Zurich, ITP, HIT K,
  Wolfgang-Pauli-Strasse 27}}
\hypersetup{pdfcontactpostcode={8093}}
\hypersetup{pdfcontactcity={Zurich}}
\hypersetup{pdfcontactcountry={Switzerland}}
\hypersetup{pdfcontactemail={nbeisert@itp.phys.ethz.ch}}
\hypersetup{pdfcontacturl={http://people.phys.ethz.ch/\xmptilde nbeisert/}}

\newcommand{\secref}[1]{\hyperref[#1]{section \ref*{#1}}}

\parskip1ex
\parindent0pt
\let\olditemize\itemize
\def\itemize{\olditemize\parskip0pt}

\begin{document}

\title{The \textsf{childdoc} Package}
\hypersetup{pdftitle={The childdoc Package}}
\author{Niklas Beisert\\[2ex]
  Institut f\"ur Theoretische Physik\\
  Eidgen\"ossische Technische Hochschule Z\"urich\\
  Wolfgang-Pauli-Strasse 27, 8093 Z\"urich, Switzerland\\[1ex]
  \href{mailto:nbeisert@itp.phys.ethz.ch}
  {\texttt{nbeisert@itp.phys.ethz.ch}}}
\hypersetup{pdfauthor={Niklas Beisert}}
\hypersetup{pdfsubject={Manual for the LaTeX2e Package childdoc}}
\date{30 December 2018, \textsf{v2.0}}
\maketitle

\begin{abstract}\noindent
\textsf{childdoc} is a \LaTeXe{} package
that enables the direct compilation
of document sections included by |\include|
to individual files.
\end{abstract}

\begingroup
\parskip0ex
\tableofcontents
\endgroup

%%%%%%%%%%%%%%%%%%%%%%%%%%%%%%%%%%%%%%%%%%%%%%%%%%%%%%%%%%%%%%%%%%%%%%%%%%%%%%%%
%%%%%%%%%%%%%%%%%%%%%%%%%%%%%%%%%%%%%%%%%%%%%%%%%%%%%%%%%%%%%%%%%%%%%%%%%%%%%%%%
\section{Introduction}

\LaTeX{} provides a mechanism to structure a large document (such as a book)
into a main file and several child files (containing the chapters)
using the |\include| command.
This mechanism is beneficial for documents
which span hundreds of pages in order to
make the source file(s) more manageable.
Moreover, compilation can be restricted to
selected child files by means of the |\includeonly| command.
The latter feature can be used to reduce the compilation time while editing
(this was significantly more useful in the earlier days of \LaTeX{})
or to generate a smaller document which is easier to navigate.
Another application of |\includeonly| is to generate
documents consisting of selected parts of the complete document.

However, there are a few drawbacks of the plain |\include| mechanism:
\begin{itemize}
\item
The child files cannot be compiled on their own,
they can only be compiled via the main file.
A naive editing environment
(such as a text editor with an option
to have the current file processed by \LaTeX)
may require one to switch to the main file before compiling;
attempting to compile the child file produces errors.
\item
The main file must be modified (each time)
to adjust the |\includeonly| command
to the present needs. This easily leaves the main file in a messy state.
\item
The generated document will always carry the filename
of the main document. This is inconvenient if
several child files are to be compiled and
to be kept for distribution.
\end{itemize}

The present package provides a simple interface
to make child files individually compilable by \LaTeX{}.
Compiling a child file then has the same effect as compiling
the main file with an |\includeonly| command
to select the appropriate child.
Moreover the generated document will carry the name of the child
rather than the main file.
This resolves all three above issues.

This feature is meant to make the editing of books,
thesis documents and lecture notes somewhat more convenient.
However, the package can also be used efficiently for
composing a series of documents (such as exercise sheets)
which are typically distributed individually.
It then assists the author in generating the individual documents
(potentially in different versions)
as well as a document containing the collected series.
Another application is in developing style files
or other kinds of included material
where compilation of the style file could redirect
to a sample or test file.

%%%%%%%%%%%%%%%%%%%%%%%%%%%%%%%%%%%%%%%%%%%%%%%%%%%%%%%%%%%%%%%%%%%%%%%%%%%%%%%%
%%%%%%%%%%%%%%%%%%%%%%%%%%%%%%%%%%%%%%%%%%%%%%%%%%%%%%%%%%%%%%%%%%%%%%%%%%%%%%%%
\section{Usage}

First of all, the package \textsf{childdoc} is \emph{not} a standard
\LaTeXe{} |.sty| style file! Therefore it needs to be invoked in
a non-standard way.

%%%%%%%%%%%%%%%%%%%%%%%%%%%%%%%%%%%%%%%%%%%%%%%%%%%%%%%%%%%%%%%%%%%%%%%%%%%%%%%%
\subsection{Included Files}
\label{sec:include}

%%%%%%%%%%%%%%%%%%%%%%%%%%%%%%%%%%%%%%%%
\DescribeMacro{\childdocmain}
To use the package, add the commands
\begin{center}
\begin{tabular}{l}
|\input{childdoc.def}|\\
|\childdocmain{}|\\
\end{tabular}
\end{center}
at the very top of the main \LaTeX{} file,
in particular \emph{before} the |\documentclass| statement!
The argument of |\childdocmain| should be left empty
(but it must be present).

%%%%%%%%%%%%%%%%%%%%%%%%%%%%%%%%%%%%%%%%
\DescribeMacro{\childdocof}
Furthermore, add the commands
\begin{center}
\begin{tabular}{l}
|\input{childdoc.def}|\\
|\childdocof{|\textit{main}|}|\\
\end{tabular}
\end{center}
at the top of every child file \textit{child}
which is included by |\include{|\textit{child}|}|
from within the main file
(or at least for those files to be compiled individually).
The argument \textit{main} must be the filename of the main file.

There are a couple of
considerations in setting up the main and child documents:

%%%%%%%%%%%%%%%%%%%%%%%%%%%%%%%%%%%%%%%%
\paragraph{Restrictions.}

Please note the following restrictions:
\begin{itemize}
\item
|\childdocmain| must be called with one argument \textit{main}
to ensure compatibility with earlier version of the package.
It must either be empty (|\childdocmain{}|)
or precisely match the filename of the main file in which it is specified.
See \secref{sec:detection} for further information.
\item
The filename \textit{main} must be specified without the |.tex| extension.
\item
The filename \textit{main} is case sensitive
(even in case-insensitive file systems)
due to internal string comparison.
\item
The argument \textit{main} should be fully expanded, it cannot be a macro.
\item
Subdirectories and special characters should be avoided in filenames.
\item
The command |\childdocmain{|\textit{main}|}| must be followed by a whitespace.
It should not be followed immediately by another command
or by a comment mark `|%|'.
This is because the \TeX{} parser reads the token immediately following
the argument of |\childdocmain| and puts it
at the beginning of every child section;
however, a white\-space is ignored.
\end{itemize}

%%%%%%%%%%%%%%%%%%%%%%%%%%%%%%%%%%%%%%%%
\paragraph{Content of Main File.}

It is advisable to place all content in the child files included by |\include|.
Any output contained in the main file will appear in all child documents
unless suppressed manually;
it cannot be suppressed automatically by the |\includeonly| directive
and thus should normally be avoided.
A method to include some content in the main file
by means of conditional processing is described in \secref{sec:conditional}.

%%%%%%%%%%%%%%%%%%%%%%%%%%%%%%%%%%%%%%%%
\paragraph{Page Numbering.}

When only a part of the document is compiled,
the appropriate numbering of pages
(as well as other status parameters)
is determined from the |.aux| files.
The latter contain information from previous passes.
However this information needs to propagate through
all intermediate child documents.
Therefore the page numbering in child documents may well
be inconsistent until the complete document is compiled at least once.

A useful (if unconventional) way to always ensure a consistent
page numbering is to restart the numbering in each child document
and denote the pages by `\textit{child}|.|\textit{page}'
where \textit{child} represents the chapter/section number of the child file.
This can be achieved by the command
|\numberwithin{page}{|\textit{child}|}|
of the \textsf{amsmath} package
where \textit{child} can be |chapter| or |section|
depending on the chosen structuring.
Alternatively, one can modify the macro |\thepage| appropriately
and reset the counter |page| at the start of each child file.

%%%%%%%%%%%%%%%%%%%%%%%%%%%%%%%%%%%%%%%%%%%%%%%%%%%%%%%%%%%%%%%%%%%%%%%%%%%%%%%%
\subsection{Conditional Processing}
\label{sec:conditional}

The package provides a mechanism to compile different versions
of a document. To customise the versions further some conditional processing
can come in handy to distinguish which version is being compiled.
The package provides two macros to describe the compilation context:

%%%%%%%%%%%%%%%%%%%%%%%%%%%%%%%%%%%%%%%%
\DescribeMacro{\ifchilddoc}
The conditional |\ifchilddoc| distinguishes between the compilation of
child documents and the main document:
%
\begin{center}
|\ifchilddoc |\textit{child-code}| |[|\||else |\textit{main-code}]| \||fi|
\end{center}

%%%%%%%%%%%%%%%%%%%%%%%%%%%%%%%%%%%%%%%%
\DescribeMacro{\childdocname}
\DescribeMacro{\childdocjob}
The macro |\childdocname| contains the filename (without extension)
of the main or child file being processed.
Note that |\childdocjob| will always contain the name of the main file.

%%%%%%%%%%%%%%%%%%%%%%%%%%%%%%%%%%%%%%%%
\paragraph{Title Page.}

Conditional processing can be used to include a title or banner page
in the main document when proper precautions are taken.
Importantly, the code in the main file should ensure that the page counter
(as well as other status parameters which are stored in the |.aux| files)
takes the same value after the conditional processing.
Otherwise the page numbers may take divergent values
depending on which part is compiled.

For example, a title page could be declared by:
%
\begin{center}
\begin{tabular}{l}
|\ifchilddoc\||else|\\
|\addtocounter{page}{-1}|\\
\textit{code for title page}\\
|\newpage|\\
|\||fi|
\end{tabular}
\end{center}
%
A banner page for the child documents can be generated by:
%
\begin{center}
\begin{tabular}{l}
|\ifchilddoc|\\
|\addtocounter{page}{-1}|\\
\textit{code for banner page}\\
|\newpage|\\
|\||fi|
\end{tabular}
\end{center}
%
Here one could write a message such as:
\begin{center}
|This is the part \childdocname{} of \childdocjob{}.|
\end{center}

%%%%%%%%%%%%%%%%%%%%%%%%%%%%%%%%%%%%%%%%%%%%%%%%%%%%%%%%%%%%%%%%%%%%%%%%%%%%%%%%
\subsection{Flags}
\label{sec:flags}

The package makes it easy to generate different versions
of the main or child documents.
To this end compilation flags can be defined
and assigned different default values.
They will be particularly useful in conjunction
with the forwarding mechanism described in \secref{sec:forward}.

For example, it may be useful to have a flag |\version|
which can be set to |draft| or |final|.
The document source will contain some conditional code
depending on the value of |\version|.
Suppose further, the flag should default to |final| for the main file
and to |draft| for child files
which is a natural assignment for editing the document.
This is achieved by placing the following code
in the preamble of the main document
(below the |\childdocmain| directive):
%
\begin{center}
\begin{tabular}{l}
|\ifchilddoc|\\
|\providecommand{\version}{draft}|\\
|\||else|\\
|\providecommand{\version}{final}|\\
|\||fi|
\end{tabular}
\end{center}
%
The definition by |\providecommand| makes sure
that previous definitions are not overwritten.
Further statements |\providecommand{\version}{...}|
can thus be added before the above code to override it.

For the main file, one might add a line
(between |\childdocmain| and the above block)
%
\begin{center}
|%\ifchilddoc\||else\providecommand{\version}{draft}\||fi|
\end{center}
%
which can be uncommented to produce a draft version.
Likewise one can add a line to the very top of a child file
(above the |\childdocof{|\textit{main}|}| directive)
%
\begin{center}
|%\providecommand{\version}{final}|
\end{center}
%
which can be uncommented to produce the final version of this child document.

%%%%%%%%%%%%%%%%%%%%%%%%%%%%%%%%%%%%%%%%%%%%%%%%%%%%%%%%%%%%%%%%%%%%%%%%%%%%%%%%
\subsection{Forwarding}
\label{sec:forward}

Different versions of the main or child documents
using compilation flags as described in \secref{sec:flags}
can be (permanently) stored in different files
for convenient compilation, viewing and distribution.
To this end, the package defines a command
to pass on compilation to a different file:

%%%%%%%%%%%%%%%%%%%%%%%%%%%%%%%%%%%%%%%%
\DescribeMacro{\childdocforward}
The command |\childdocforward| redirects processing to
another source file:
%
\begin{center}
\begin{tabular}{l}
|\input{childdoc.def}|\\
|\childdocforward[|\textit{main}|]{|\textit{dest}|}|\\
\end{tabular}
\end{center}
%
The argument \textit{dest} is the destination file
(without extension).
It should be the main file or one of the child files.
Note that further \textsf{childdoc} directives
such as |\childdocof| and |\childdocforward|
in the indicated file will be processed in this form.
The optional argument \textit{main}
passes on directly to the main file \textit{main}
while pretending to compile the child \textit{dest}.
This form behaves as if \textit{dest}
issues |\childdocof{|\textit{main}|}| right away,
and no further \textsf{childdoc} directives will be processed.

%%%%%%%%%%%%%%%%%%%%%%%%%%%%%%%%%%%%%%%%
\DescribeMacro{\...prefix}
In the alternative form |\childdocforwardprefix|,
%
\begin{center}
\begin{tabular}{l}
|\input{childdoc.def}|\\
|\childdocforwardprefix[|\textit{main}|]{|\textit{prefix}|}{|\textit{dest}|}|
\end{tabular}
\end{center}
%
the destination file is determined by a pattern
depending on the current file:
To make this work, the current file must be called
`{\textit{prefix}\hspace{0.2em}\textit{suffix}}'
with \textit{prefix} matching precisely the argument.
Processing is then passed on to the file
`{\textit{dest}\hspace{0.2em}\textit{suffix}}'.
Surely, the same effect is achieved by
directly specifying the
argument `{\textit{dest}\hspace{0.2em}\textit{suffix}}'
in the first form.
However, that requires to set up a different file
for each child. With the alternative form of the command
all these files can have exactly the same content
which simplifies setting them up and maintaining them.

For example, the following file |draft.tex|
with a compilation flag |\version| as described in \secref{sec:flags}
compiles the main document as a draft:
%
\begin{center}
\begin{tabular}{l}
|\def\version{draft}|\\
|\input{childdoc.def}|\\
|\childdocforward{|\textit{main}|}|
\end{tabular}
\end{center}
%
Likewise, the following files |final|\textit{nn}|.tex|
compile the final version of the child document
|child|\textit{nn}|.tex|:
%
\begin{center}
\begin{tabular}{l}
|\def\version{final}|\\
|\input{childdoc.def}|\\
|\childdocforwardprefix{final}{child}|
\end{tabular}
\end{center}
%

Note that when several versions of a main file and/or of each child file
are to be generated, it may be convenient to set up a |Makefile| or
shell script to automatise the process.

%%%%%%%%%%%%%%%%%%%%%%%%%%%%%%%%%%%%%%%%%%%%%%%%%%%%%%%%%%%%%%%%%%%%%%%%%%%%%%%%
\subsection{Command Line Processing}
\label{sec:commandline}

The effect of redirection files can also be achieved by invoking
the \LaTeX{} compiler with a more elaborate command line.
Most conveniently this should be done as part
of a shell script or a |Makefile|.

When using \textsf{childdoc} in the main file, the following
command lines effectively perform a redirection
(note that depending on the shell being used,
backslashes may have to be doubled: `|\|' $\to$ `|\\|'):
%
\begin{center}
|... -jobname "|\textit{target}|" |\\|"|[\textit{flags}]%
|\input{childdoc.def}\childdocforward[|\textit{main}|]{|\textit{dest}|}"|
\end{center}
%
Here \textit{target} is the name of the output file,
\textit{main} is the name of the main file
and \textit{dest} is the name of the main or child file to be processed
(all filenames without extensions).
The optional argument \textit{main} can be omitted
if \textit{main} matches \textit{dest}.
Optionally, compilation \textit{flags} can be defined via |\def| commands.
This command line makes the \TeX{} engine believe
it is compiling the file \textit{target}
whose content is specified as the latter parameter.
The provided code then forwards the processing to
\textit{main} or \textit{dest} as described in \secref{sec:forward}.

%%%%%%%%%%%%%%%%%%%%%%%%%%%%%%%%%%%%%%%%%%%%%%%%%%%%%%%%%%%%%%%%%%%%%%%%%%%%%%%%
\subsection{Include by Input}
\label{sec:input}

Including child documents by |\include| has some restrictions by design.
Most notably, the content of a child document always occupies
its own set of pages; pages cannot be shared between child documents.
Usually, this behaviour makes perfect sense
because each child document contain an essential part of the document.
However, in some situations it may be desirable to compose
a document from a collection of parts
without having mandatory page breaks between then.
For this case, the package
provides a mechanism to include parts
by |\input| which can also be processed individually.
However, by construction this mechanism
requires manual handling of the content to be output.

%%%%%%%%%%%%%%%%%%%%%%%%%%%%%%%%%%%%%%%%
\DescribeMacro{\ifchilddocmanual}
The main file should be prepared as usual, see \secref{sec:include}.
However, the document body must make a distinction
between processing of an individual part and of the main document, e.g.:
%
\begin{center}
\begin{tabular}{l}
|\ifchilddocmanual|\\
|\input{\childdocname}|\\
|\||else|\\
\textit{document body with }|\input{|\textit{part}|}|\\
|\||fi|
\end{tabular}
\end{center}
%
The conditional |\ifchilddocmanual| is true whenever
a part to be included by |\input| is being compiled,
and the name of the part is stored in |\childdocname|.

%%%%%%%%%%%%%%%%%%%%%%%%%%%%%%%%%%%%%%%%
\DescribeMacro{\childdocby}
Each part to be included by |\input| should start with:
%
\begin{center}
\begin{tabular}{l}
|\input{childdoc.def}|\\
|\childdocby{|\textit{main}|}|\\
\end{tabular}
\end{center}
%
The directive |\childdocby| is similar to |\childdocof|
described in \secref{sec:include},
but the subsequent selection of content must be done manually.
To that end, both |\ifchilddoc| and |\ifchilddocmanual|
will be true upon processing of a part,
and the name of the part is stored in |\childdocname|.
Note that |\jobname| will be set to the filename of the current part
so that each part receives an individual |.aux| file
that does not interfere with the |.aux| file(s) of the main document.
This behaviour can be altered by the alternative form
|\childdocby[*]{|\textit{main}|}| (with a non-empty optional argument)
which uses the |.aux| file of the main document
by setting |\jobname| to \textit{main}.

%%%%%%%%%%%%%%%%%%%%%%%%%%%%%%%%%%%%%%%%%%%%%%%%%%%%%%%%%%%%%%%%%%%%%%%%%%%%%%%%
\subsection{Driver Development}
\label{sec:driver}

The \textsf{childdoc} mechanism can also be use for the development
of definition files such as \LaTeX{} styles or classes.
This case differs from the above setup with multiple parts
included by |\include| in that no |\includeonly| should be invoked.
This can be achieved by starting the include file
(before |\ProvidesPackage|) with:
%
\begin{center}
\begin{tabular}{l}
|\input{childdoc.def}|\\
|\childdocforward{|\textit{main}|}|\\
\end{tabular}
\end{center}
%
or alternatively with:
%
\begin{center}
\begin{tabular}{l}
|\input{childdoc.def}|\\
|\childdocby{|\textit{main}|}|\\
\end{tabular}
\end{center}
%
Both forms have slightly different effects as described above.
The main file is prepared as usual, see \secref{sec:include}.

%%%%%%%%%%%%%%%%%%%%%%%%%%%%%%%%%%%%%%%%%%%%%%%%%%%%%%%%%%%%%%%%%%%%%%%%%%%%%%%%
\subsection{Legacy Detection}
\label{sec:detection}

The directive |\childdocmain| in the main file can detect
whether the complete document or merely a child is to be compiled
even without using the directive |\childdocof|.
This method is deprecated because it is less robust
and there is no compelling reason to use it;
it is merely provided for backward compatibility
and it may be removed in future versions.

If the detection mechanism is to be used,
it is mandatory to correctly specify
the filename of the main file as the argument of |\childdocmain|:
%
\begin{center}
\begin{tabular}{l}
|\input{childdoc.def}|\\
|\childdocmain{|\textit{main}|}|\\
\end{tabular}
\end{center}
%
If |\jobname| does not match the argument \textit{main} of |\childdocmain|,
it is assumed that |\jobname| points to the child file to be compiled.
When using |\childdocmain| with the main file specified as argument,
it suffices to start a child file
with just |\input{|\textit{main}|}|
without loading of the package and using |\childdocof|.
If instead all processing is done
with the appropriate \textsf{childdoc} directives,
the argument of \textit{main} of |\childdocmain| can be empty.

An alternative version of the command line processing described
in \secref{sec:commandline} using the detection mechanism reads:
%
\begin{center}
|... -jobname "|\textit{target}|" "|[\textit{flags}]%
[|\def\jobname{|\textit{dest}|}|]|\input{|\textit{main}|}"|
\end{center}

%%%%%%%%%%%%%%%%%%%%%%%%%%%%%%%%%%%%%%%%%%%%%%%%%%%%%%%%%%%%%%%%%%%%%%%%%%%%%%%%
\subsection{Manual Code}
\label{sec:manual}

In case one cannot be certain whether the definitions file |childdoc.def|
is installed on the target \TeX{} distribution
and one prefers not to ship it,
it is conceivable to paste a few relevant commands into the sources.

To that end, drop all statements |\input{childdoc.def}|
and perform the replacements as outlined below.
Instead of |\childdocmain{|\textit{main}|}| add the following code
to the top of the main file:
%
\begin{center}
\begin{tabular}{l}
|\||ifdefined\childdocname\endinput\||fi\newif\ifchilddoc|\\
|\edef\childdocname{\scantokens\expandafter{\jobname\noexpand}}|\\
|\def\childdocmain{|\textit{main}|}\||ifx\childdocmain\childdocname\||else|\\
|\childdoctrue\includeonly{\childdocname}\let\jobname\childdocmain\||fi|\\
\end{tabular}
\end{center}
%
Instead of |\childdocof{|\textit{main}|}| just include the main file
at the top of each child file:
%
\begin{center}
|\input{|\textit{main}|}|
\end{center}
%
A simple redirection |\childdocforward{|\textit{dest}|}| is achieved by:
%
\begin{center}
|\def\jobname{|\textit{dest}|}\input{\jobname}|
\end{center}
%
The redirection with prefix
|\childdocforwardprefix[|\textit{prefix}|]{|\textit{dest}|}|
is accomplished by:
%
\begin{center}
\begin{tabular}{l}
|{\edef\jobname{\scantokens\expandafter{\jobname\noexpand}}|\\
|\def\redirectjob |\textit{prefix}|#1~~~{\gdef\jobname{|\textit{dest}|#1}}|\\
|\expandafter\redirectjob\jobname~~~}\input{\jobname}|
\end{tabular}
\end{center}

In an alternative approach,
child documents can be compiled by a specific command line
without additional code or specific definitions:
%
\begin{center}
|... -jobname "|\textit{target}|" "|[\textit{flags}]%
|\includeonly{|\textit{dest}|}\input{|\textit{main}|}"|
\end{center}
%

%%%%%%%%%%%%%%%%%%%%%%%%%%%%%%%%%%%%%%%%%%%%%%%%%%%%%%%%%%%%%%%%%%%%%%%%%%%%%%%%
%%%%%%%%%%%%%%%%%%%%%%%%%%%%%%%%%%%%%%%%%%%%%%%%%%%%%%%%%%%%%%%%%%%%%%%%%%%%%%%%
\section{Information}

%%%%%%%%%%%%%%%%%%%%%%%%%%%%%%%%%%%%%%%%%%%%%%%%%%%%%%%%%%%%%%%%%%%%%%%%%%%%%%%%
\subsection{Copyright}

Copyright \copyright{} 2017--2018 Niklas Beisert

This work may be distributed and/or modified under the
conditions of the \LaTeX{} Project Public License, either version 1.3
of this license or (at your option) any later version.
The latest version of this license is in
  \url{http://www.latex-project.org/lppl.txt}
and version 1.3 or later is part of all distributions of \LaTeX{}
version 2005/12/01 or later.

This work has the LPPL maintenance status `maintained'.

The Current Maintainer of this work is Niklas Beisert.

This work consists of the files |README.txt|, |childdoc.ins| and |childdoc.dtx|
as well as the derived files |childdoc.def|, |cdocsamp.tex|
with |cdocsch1.tex|, |cdocsch2.tex|, |cdocspt3.tex|, |cdocspt4.tex|,
|cdocsdrf.tex|, |cdocsfn1.tex|, |cdocsfn2.tex|
as well as |childdoc.pdf|.

%%%%%%%%%%%%%%%%%%%%%%%%%%%%%%%%%%%%%%%%%%%%%%%%%%%%%%%%%%%%%%%%%%%%%%%%%%%%%%%%
\subsection{Files and Installation}

The package consists of the files:
%
\begin{center}
\begin{tabular}{ll}
    |README.txt|   & readme file \\
    |childdoc.ins| & installation file \\
    |childdoc.dtx| & source file \\
    |childdoc.def| & definition file \\
    |cdocsamp.tex| & sample main file \\
    |cdocsch1.tex| & sample include file \\
    |cdocsch2.tex| & sample include file \\
    |cdocspt3.tex| & sample part file \\
    |cdocspt4.tex| & sample part file \\
    |cdocsdrf.tex| & sample redirection file \\
    |cdocsfn1.tex| & sample redirection file \\
    |cdocsfn2.tex| & sample redirection file \\
    |childdoc.pdf| & manual
\end{tabular}
\end{center}
%
The distribution consists of the files
|README.txt|, |childdoc.ins| and |childdoc.dtx|.
%
\begin{itemize}
\item
Run (pdf)\LaTeX{} on |childdoc.dtx|
to compile the manual |childdoc.pdf| (this file).
\item
Run \LaTeX{} on |childdoc.ins| to create the definitions file |childdoc.def|
and the sample |cdocsamp.tex| with include files
|cdocsch1.tex|, |cdocsch2.tex|, |cdocspt3.tex|, |cdocspt4.tex|,
|cdocsdrf.tex|, |cdocsfn1.tex|, |cdocsfn2.tex|.
Then copy the file |childdoc.def| to an appropriate directory of your \LaTeX{}
distribution, e.g.\ \textit{texmf-root}|/tex/latex/childdoc|.
\end{itemize}

%%%%%%%%%%%%%%%%%%%%%%%%%%%%%%%%%%%%%%%%%%%%%%%%%%%%%%%%%%%%%%%%%%%%%%%%%%%%%%%%
\subsection{Related CTAN Packages}

There are several other packages which offer a similar functionality:
%
\begin{itemize}
\item
The packages
\href{http://ctan.org/pkg/docmute}{\textsf{docmute}},
\href{http://ctan.org/pkg/includex}{\textsf{includex}} and
\href{http://ctan.org/pkg/standalone}{\textsf{standalone}}
provide commands to include only the document body of
a child file thus allowing both files to be compiled individually.
\item
The packages \href{http://ctan.org/pkg/subdocs}{\textsf{subdocs}}
and \href{http://ctan.org/pkg/subfiles}{\textsf{subfiles}}
provide structures in which the main and child documents can be
encapsulated and allowing them to be compiled individually.
The inclusion mechanism is different from the conventional |\include|.
\item
The package \href{http://ctan.org/pkg/combine}{\textsf{combine}}
is an elaborate solution to combine several documents into one.
\end{itemize}
%
See also the CTAN topic \href{http://ctan.org/topic/subdocs}{\textsf{subdocs}}
for further related packages.
The present package differs from the above solutions in that
a document structure constructed with the conventional |\include| mechanism
just needs two extra commands at the top of every file
such that all constituent files can be compiled individually.

%%%%%%%%%%%%%%%%%%%%%%%%%%%%%%%%%%%%%%%%%%%%%%%%%%%%%%%%%%%%%%%%%%%%%%%%%%%%%%%%
%\subsection{Feature Suggestions}
%
%The following is a list of features which may be useful for future
%versions of this package:
%%
%\begin{itemize}
%\item
%\ldots
%\end{itemize}

%%%%%%%%%%%%%%%%%%%%%%%%%%%%%%%%%%%%%%%%%%%%%%%%%%%%%%%%%%%%%%%%%%%%%%%%%%%%%%%%
\subsection{Revision History}

%%%%%%%%%%%%%%%%%%%%%%%%%%%%%%%%%%%%%%%%
\paragraph{v2.0:} 2018/12/30

\begin{itemize}
\item
immediate forward processing
\item
added |\childdocby| mechanism
\item
manual restructured
\end{itemize}

%%%%%%%%%%%%%%%%%%%%%%%%%%%%%%%%%%%%%%%%
\paragraph{v1.6:} 2018/01/17

\begin{itemize}
\item
application for development of include files
\item
corrections to manual
\end{itemize}

%%%%%%%%%%%%%%%%%%%%%%%%%%%%%%%%%%%%%%%%
\paragraph{v1.5:} 2017/05/21

\begin{itemize}
\item
more complete structuring introduced
\item
|\childdocof| introduced
\item
|\childdoc| renamed to |\childdocmain|
\item
|\childredirect| renamed to |\childdocforward| and |\childdocforwardprefix|
and functionality expanded
\end{itemize}

%%%%%%%%%%%%%%%%%%%%%%%%%%%%%%%%%%%%%%%%
\paragraph{v1.0:} 2017/04/27

\begin{itemize}
\item
manual and install package
\item
first version published on CTAN
\end{itemize}

%%%%%%%%%%%%%%%%%%%%%%%%%%%%%%%%%%%%%%%%
\paragraph{v0.6:} 2017/04/26

\begin{itemize}
\item
redirection mechanism added
\end{itemize}

%%%%%%%%%%%%%%%%%%%%%%%%%%%%%%%%%%%%%%%%
\paragraph{v0.5:} 2017/04/26

\begin{itemize}
\item
functionality in definition file
\end{itemize}


%%%%%%%%%%%%%%%%%%%%%%%%%%%%%%%%%%%%%%%%%%%%%%%%%%%%%%%%%%%%%%%%%%%%%%%%%%%%%%%%
%%%%%%%%%%%%%%%%%%%%%%%%%%%%%%%%%%%%%%%%%%%%%%%%%%%%%%%%%%%%%%%%%%%%%%%%%%%%%%%%
%%%%%%%%%%%%%%%%%%%%%%%%%%%%%%%%%%%%%%%%%%%%%%%%%%%%%%%%%%%%%%%%%%%%%%%%%%%%%%%%
\appendix

\settowidth\MacroIndent{\rmfamily\scriptsize 000\ }

 \DocInput{childdoc.dtx}

\end{document}
%</driver>
% \fi
%
% %%%%%%%%%%%%%%%%%%%%%%%%%%%%%%%%%%%%%%%%%%%%%%%%%%%%%%%%%%%%%%%%%%%%%%%%%%%%%%
% %%%%%%%%%%%%%%%%%%%%%%%%%%%%%%%%%%%%%%%%%%%%%%%%%%%%%%%%%%%%%%%%%%%%%%%%%%%%%%
% \section{Sample}
%\iffalse
%<*samplemain>
%\fi
%
% The following presents a sample document
% with two chapters, two parts, a title page,
% a compile flag as well as three forwarding files to set the flag.
% It consists of eight |.tex| files:
% \begin{center}
% \begin{tabular}{ll}
% |cdocsamp.tex|&main file\\
% |cdocsch1.tex|&include file for chapter 1\\
% |cdocsch2.tex|&include file for chapter 2\\
% |cdocspt3.tex|&include file for part 3\\
% |cdocspt4.tex|&include file for part 4\\
% |cdocsdrf.tex|&forwarding file for main file in draft mode\\
% |cdocsfi1.tex|&forwarding file for final version of chapter 1\\
% |cdocsfi2.tex|&forwarding file for final version of chapter 2\\
% \end{tabular}
% \end{center}
% Each of the eight files can be compiled directly by the \LaTeX{} compiler.
%
% %%%%%%%%%%%%%%%%%%%%%%%%%%%%%%%%%%%%%%
% \paragraph{Main File.}
%
% The main file is called |cdocsamp.tex|.
%
% Load the \textsf{childdoc} definitions and
% declare the filename for the main document:
%    \begin{macrocode}
\input{childdoc.def}
\childdocmain{}
%    \end{macrocode}

% Optional override for |\version| flag:
%    \begin{macrocode}
%%\ifchilddoc\else\providecommand{\version}{draft}\fi
%    \end{macrocode}

% Define the default values for the |\version| flag
% (|final| for the main file and |draft| for childs):
%    \begin{macrocode}
\ifchilddoc
\providecommand{\version}{draft}
\else
\providecommand{\version}{final}
\fi
%    \end{macrocode}

% Load the standard document class:
%    \begin{macrocode}
\documentclass[12pt]{article}
%    \end{macrocode}

% Start the document body:
%    \begin{macrocode}
\begin{document}
%    \end{macrocode}

% Declare a title page.
% Print title, part of document being processed and version flag:
%    \begin{macrocode}
\addtocounter{page}{-1}
\begin{center}
{\LARGE\bfseries{}childdoc example\par}
\vspace{1cm}
\ifchilddoc
\ifchilddocmanual part\else chapter\fi:
`\childdocname' of `\childdocjob'\par
\else
main document: `\childdocjob'\par
\fi
version: \version\par
\end{center}
\newpage
%    \end{macrocode}

% Manually include selected file,
% otherwise process as usual:
%    \begin{macrocode}
\ifchilddocmanual
\section*{part `\childdocname'}
\input{\childdocname}
\else
%    \end{macrocode}

% Include the two chapters:
%    \begin{macrocode}
\include{cdocsch1}
\include{cdocsch2}
%    \end{macrocode}

% Include the two parts unless only chapters should be displayed:
%    \begin{macrocode}
\ifchilddoc\else
\section{part three}
\input{cdocspt3}
\section{part four}
\input{cdocspt4}
\fi
%    \end{macrocode}

% Process as usual until here:
%    \begin{macrocode}
\fi
%    \end{macrocode}

% End of document body:
%    \begin{macrocode}
\end{document}
%    \end{macrocode}
%\iffalse
%</samplemain>
%\fi
%
% %%%%%%%%%%%%%%%%%%%%%%%%%%%%%%%%%%%%%%
% \paragraph{Chapter Include Files.}
%
% The include files are called |cdocsch1.tex| and |cdocsch2.tex|.
%
%\iffalse
%<*samplechap1|samplechap2>
%\fi

% Optional override for |\version| flag:
%    \begin{macrocode}
%%\providecommand{\version}{final}
%    \end{macrocode}

% Include the main document:
%    \begin{macrocode}
\input{childdoc.def}
\childdocof{cdocsamp}
%    \end{macrocode}

%\iffalse
%</samplechap1|samplechap2>
%\fi
%
%\iffalse
%<*samplechap1>
%\fi
% Some text for chapter 1:
%    \begin{macrocode}
\section{one}
some text in chapter one
%    \end{macrocode}

%\iffalse
%</samplechap1>
%\fi
% Some text for chapter 2:
%\iffalse
%<*samplechap2>
%\fi
%    \begin{macrocode}
\section{two}
more text in chapter two
%    \end{macrocode}

%\iffalse
%</samplechap2>
%\fi
%
% %%%%%%%%%%%%%%%%%%%%%%%%%%%%%%%%%%%%%%
% \paragraph{Part Include Files.}
%
% The include files are called |cdocspt3.tex| and |cdocspt4.tex|.
%
%\iffalse
%<*samplepart3|samplepart4>
%\fi

% Optional override for |\version| flag:
%    \begin{macrocode}
%%\providecommand{\version}{final}
%    \end{macrocode}

% Include the main document:
%    \begin{macrocode}
\input{childdoc.def}
\childdocby{cdocsamp}
%    \end{macrocode}

%\iffalse
%</samplepart3|samplepart4>
%\fi
%
%\iffalse
%<*samplepart3>
%\fi
% Some text for part 3:
%    \begin{macrocode}
some text in part three
%    \end{macrocode}

%\iffalse
%</samplepart3>
%\fi
% Some text for part 4:
%\iffalse
%<*samplepart4>
%\fi
%    \begin{macrocode}
more text in part four
%    \end{macrocode}

%\iffalse
%</samplepart4>
%\fi
%
% %%%%%%%%%%%%%%%%%%%%%%%%%%%%%%%%%%%%%%
% \paragraph{Forwarding for a Complete Draft.}
%
% The following forwarding file |cdocsdrf.tex|
% compiles the main document in draft mode:
%\iffalse
%<*sampledraft>
%\fi
%    \begin{macrocode}
\def\version{draft}
\input{childdoc.def}
\childdocforward{cdocsamp}
%    \end{macrocode}

%\iffalse
%</sampledraft>
%\fi
%
% %%%%%%%%%%%%%%%%%%%%%%%%%%%%%%%%%%%%%%
% \paragraph{Forwarding for Final Version of the Chapters.}
%
% The following forwarding files |cdocsfn1.tex| and |cdocsfn2.tex|
% (with identical content)
% compile the final versions of the child documents
% |cdocsch1.tex| and |cdocsch2.tex|, respectively:
%\iffalse
%<*samplefinal>
%\fi
%    \begin{macrocode}
\def\version{final}
\input{childdoc.def}
\childdocforwardprefix[cdocsamp]{cdocsfn}{cdocsch}
%    \end{macrocode}

%\iffalse
%</samplefinal>
%\fi
%
% %%%%%%%%%%%%%%%%%%%%%%%%%%%%%%%%%%%%%%
% \paragraph{Command Line Processing.}
%
% The following three command lines generate the output files
% |cdocscld|, |cdocscl1| and |cdocscl2|
% which should be identical to
% |cdocsdrf|, |cdocsch1| and |cdocsfn2|, respectively:
% \begin{center}
% \begin{tabular}{l}
% |latex -jobname cdocscld \|\\
% |  "\def\version{draft}\input{childdoc.def}\childdocforward{cdocsamp}"|\\
% |latex -jobname cdocscl1 \|\\
% |  "\input{childdoc.def}\childdocforward[cdocsamp]{cdocsch1}"|\\
% |latex -jobname cdocscl2 \|\\
% |  "\def\version{final}\input{childdoc.def}\childdocforward{cdocsch2}"|
% \end{tabular}
% \end{center}
% Note that the trailing backslash on each first line
% merely continues the input to the second line
% (for convenient cut ant paste).
% Furthermore, the command |latex| can be replaced by any
% of its alternative versions such as |pdflatex|.
%
% %%%%%%%%%%%%%%%%%%%%%%%%%%%%%%%%%%%%%%%%%%%%%%%%%%%%%%%%%%%%%%%%%%%%%%%%%%%%%%
% %%%%%%%%%%%%%%%%%%%%%%%%%%%%%%%%%%%%%%%%%%%%%%%%%%%%%%%%%%%%%%%%%%%%%%%%%%%%%%
% \section{Implementation}
%\iffalse
%<*package>
%\fi
%
% This section describes the definitions file |childdoc.def|.

% The definitions cannot be loaded using |\usepackage| or |\RequirePackage|
% which has a mechanism to prevent loading a style file more than once.
% When loading the definitions by means of |\input|
% multiple instances have to be prevented manually:
%\iffalse
%This code needs to be before the `\ProvidesFile' directive
%which is defined at the beginning of this file.
%Therefore it is also placed there and commented out here.
%</package>
%<*discard>
%\fi
%    \begin{macrocode}
\ifdefined\childdocmain\endinput\fi
%    \end{macrocode}
%\iffalse
%</discard>
%<*package>
%\fi
%
% \macro{\ifchilddoc}
% \macro{\ifchilddocmanual}
% The conditional |\ifchilddoc| tells whether a
% child (true) or main (false) document is being compiled.
% The conditional |\ifchilddocmanual| tells whether
% the |\includeonly| mechanism is used (false) or
% the selection of child files must be performed manually (true).
% The definitions initialise to false:
%    \begin{macrocode}
\newif\ifchilddoc
\newif\ifchilddocmanual
%    \end{macrocode}

% \macro{\childdocname}
% \macro{\childdocjob}
% The macro |\childdocname| stores the name of the main document
% to be compiled. The macro |\childdocjob| stores the name of
% the document on which the \LaTeX{} compiler was originally invoked.
% The content of |\jobname| cannot be compared
% to filenames specified in the source due to different catcodes.
% The following code rescans |\jobname|, stores the result
% in |\childdocname| and saves a copy in |\childdocjob|:
%    \begin{macrocode}
\edef\childdocname{\scantokens\expandafter{\jobname\noexpand}}
\let\childdocjob\childdocname
%    \end{macrocode}

% \macro{\childdocdisable}
% The macro |\childdocdisable| prevents the main file
% from being processed more than once.
% At this stage, the main document command |\childdocmain|
% is assumed to be called once again where it should do nothing.
% Any subsequent call to it should prevent
% a secondary processing of the main document
% It overwrites the forwarding commands
% |\childdocof| and |\childdocforward|
% with empty macros to prevent further inclusions of the main document:
%    \begin{macrocode}
\newcommand{\childdocdisable}
{
  \renewcommand{\childdocmain}[1]{\renewcommand{\childdocmain}[1]{\endinput}}
  \renewcommand{\childdocof}[1]{}
  \renewcommand{\childdocby}[2][]{}
  \renewcommand{\childdocforward}[2][]{}
  \renewcommand{\childdocdisable}{}
}
%    \end{macrocode}

% \macro{\childdocmain}
% The macro |\childdocmain| is to be called at the top of the main file
% with nothing or the main filename (without extension) as argument.
% First, it breaks loops.
% If the argument is not empty and does not match |\childdocname|
% (which is set by the first inclusion of |childdoc.def|),
% |\ifchilddoc| is set to true, |\includeonly| is applied to the child file
% and |\jobname| is set to the main file
% (for proper handling of |.aux| files):
%    \begin{macrocode}
\newcommand{\childdocmain}[1]
{
  \childdocdisable\childdocmain{}
  \if?#1?\else
    \begingroup
      \def\childdoctmp{#1}
      \ifx\childdoctmp\childdocname
        \def\childdoctmp{}
      \else
        \def\childdoctmp
        {
          \childdoctrue
          \includeonly{\childdocname}
          \def\childdocjob{#1}
          \def\jobname{#1}
        }
      \fi
      \expandafter
    \endgroup
    \childdoctmp
  \fi
}
%    \end{macrocode}

% \macro{\childdocof}
% The command |\childdocof| redirects
% compilation to the main file |#1|.
%    \begin{macrocode}
\newcommand{\childdocof}[1]
{
  \childdocdisable
  \childdoctrue
  \includeonly{\childdocname}
  \def\jobname{#1}
  \def\childdocjob{#1}
  \input{#1}
}
%    \end{macrocode}

% \macro{\childdocby}
% The command |\childdocby| ....
%    \begin{macrocode}
\newcommand{\childdocby}[2][]
{
  \childdocdisable
  \childdoctrue
  \childdocmanualtrue
  \if?#1?\else
    \def\jobname{#2}
  \fi
  \def\childdocjob{#2}
  \input{#2}
  \endinput
}
%    \end{macrocode}

% \macro{\childdocforward}
% The command |\childdocforward| redirects
% compilation to the main file or
% (if the optional argument is given) a child file.
% Parameters are set as if the main file
% or a child file starting with |\childdocof| was compiled.
% Then compilation is handed over to the main file:
%    \begin{macrocode}
\newcommand{\childdocforward}[2][]
{
  \begingroup
    \if?#1?
      \def\childdoctmp
      {
        \def\childdocname{#2}
        \def\childdocjob{#2}
        \def\jobname{#2}
        \input{#2}
        \endinput
      }
    \else
      \def\childdoctmp
      {
        \childdocdisable
        \def\childdocname{#2}
        \childdoctrue
        \includeonly{#2}
        \def\childdocjob{#1}
        \def\jobname{#1}
        \input{#1}
        \endinput
      }
    \fi
    \expandafter
  \endgroup
  \childdoctmp
}
%    \end{macrocode}

% \macro{\childdocforwardprefix}
% The command |\childdocforwardprefix| redirects
% compilation to the main or a child file by means of a pattern.
% The prefix |#1| in the current filename is replaced by |#2|
% and the suffix of the current filename is kept
% (it is assumed that the filename does not contain the substring `|~~~|'
% which is used as a delimiter).
% Compilation is handed over to the new file by |\childdocforward|:
%    \begin{macrocode}
\newcommand{\childdocforwardprefix}[3][]
{
  \begingroup
    \def\childdocextract #2##1~~~{\def\childdoctmp{\childdocforward[#1]{#3##1}}}
    \expandafter\childdocextract\childdocname~~~
    \expandafter
  \endgroup
  \childdoctmp
}
%    \end{macrocode}

% \macro{\childdoc}
% The deprecated macro |\childdoc| is a legacy version of |\childdocmain|:
%    \begin{macrocode}
\newcommand{\childdoc}{\childdocmain}
%    \end{macrocode}

% \macro{\childdocredirect}
% The deprecated macro |\childdocredirect| is a legacy version
% of |\childdocforward| and |\childdocforwardprefix|:
%    \begin{macrocode}
\newcommand{\childdocredirect}[2][]
{
  \begingroup
    \if?#1?
      \def\childdoctmp{\childdocforward{#2}}
    \else
      \def\childdoctmp{\childdocforwardprefix{#1}{#2}}
    \fi
    \expandafter
  \endgroup
  \childdoctmp
}
%    \end{macrocode}

%\iffalse
%</package>
%\fi
%
\endinput
\childdocforward[cdocsamp]{cdocsch1}"|\\
% |latex -jobname cdocscl2 \|\\
% |  "\def\version{final}% \iffalse
%
% childdoc.dtx Copyright (C) 2017-2018 Niklas Beisert
%
% This work may be distributed and/or modified under the
% conditions of the LaTeX Project Public License, either version 1.3
% of this license or (at your option) any later version.
% The latest version of this license is in
%   http://www.latex-project.org/lppl.txt
% and version 1.3 or later is part of all distributions of LaTeX
% version 2005/12/01 or later.
%
% This work has the LPPL maintenance status `maintained'.
%
% The Current Maintainer of this work is Niklas Beisert.
%
% This work consists of the files childdoc.dtx and childdoc.ins
% and the derived files childdoc.def and cdocsamp.tex with
% cdocsch1.tex, cdocsch2.tex, cdocsdrf.tex, cdocsfn1.tex, cdocsfn2.tex.
%
%<package>\ifdefined\childdocmain\endinput\fi
%<package>\ProvidesFile{childdoc.def}[2018/12/30 v2.0 child document driver]
%<samplemain>\ProvidesFile{cdocsamp.tex}[2018/12/30 v2.0 sample for childdoc]
%<*driver>
%\ProvidesFile{childdoc.drv}[2018/12/30 v2.0 childdoc reference manual file]
\PassOptionsToClass{10pt,a4paper}{article}
\documentclass{ltxdoc}

\usepackage[margin=35mm]{geometry}
\usepackage{hyperref}
\usepackage{hyperxmp}
\usepackage[usenames]{color}

\hypersetup{colorlinks=true}
\hypersetup{pdfstartview=FitH}
\hypersetup{pdfpagemode=UseNone}
\hypersetup{pdfsource={}}
\hypersetup{pdflang={en-UK}}
\hypersetup{pdfcopyright={Copyright 2017-2018 Niklas Beisert.
  This work may be distributed and/or modified under the
  conditions of the LaTeX Project Public License, either version 1.3
  of this license or (at your option) any later version.}}
\hypersetup{pdflicenseurl={http://www.latex-project.org/lppl.txt}}
\hypersetup{pdfcontactaddress={ETH Zurich, ITP, HIT K,
  Wolfgang-Pauli-Strasse 27}}
\hypersetup{pdfcontactpostcode={8093}}
\hypersetup{pdfcontactcity={Zurich}}
\hypersetup{pdfcontactcountry={Switzerland}}
\hypersetup{pdfcontactemail={nbeisert@itp.phys.ethz.ch}}
\hypersetup{pdfcontacturl={http://people.phys.ethz.ch/\xmptilde nbeisert/}}

\newcommand{\secref}[1]{\hyperref[#1]{section \ref*{#1}}}

\parskip1ex
\parindent0pt
\let\olditemize\itemize
\def\itemize{\olditemize\parskip0pt}

\begin{document}

\title{The \textsf{childdoc} Package}
\hypersetup{pdftitle={The childdoc Package}}
\author{Niklas Beisert\\[2ex]
  Institut f\"ur Theoretische Physik\\
  Eidgen\"ossische Technische Hochschule Z\"urich\\
  Wolfgang-Pauli-Strasse 27, 8093 Z\"urich, Switzerland\\[1ex]
  \href{mailto:nbeisert@itp.phys.ethz.ch}
  {\texttt{nbeisert@itp.phys.ethz.ch}}}
\hypersetup{pdfauthor={Niklas Beisert}}
\hypersetup{pdfsubject={Manual for the LaTeX2e Package childdoc}}
\date{30 December 2018, \textsf{v2.0}}
\maketitle

\begin{abstract}\noindent
\textsf{childdoc} is a \LaTeXe{} package
that enables the direct compilation
of document sections included by |\include|
to individual files.
\end{abstract}

\begingroup
\parskip0ex
\tableofcontents
\endgroup

%%%%%%%%%%%%%%%%%%%%%%%%%%%%%%%%%%%%%%%%%%%%%%%%%%%%%%%%%%%%%%%%%%%%%%%%%%%%%%%%
%%%%%%%%%%%%%%%%%%%%%%%%%%%%%%%%%%%%%%%%%%%%%%%%%%%%%%%%%%%%%%%%%%%%%%%%%%%%%%%%
\section{Introduction}

\LaTeX{} provides a mechanism to structure a large document (such as a book)
into a main file and several child files (containing the chapters)
using the |\include| command.
This mechanism is beneficial for documents
which span hundreds of pages in order to
make the source file(s) more manageable.
Moreover, compilation can be restricted to
selected child files by means of the |\includeonly| command.
The latter feature can be used to reduce the compilation time while editing
(this was significantly more useful in the earlier days of \LaTeX{})
or to generate a smaller document which is easier to navigate.
Another application of |\includeonly| is to generate
documents consisting of selected parts of the complete document.

However, there are a few drawbacks of the plain |\include| mechanism:
\begin{itemize}
\item
The child files cannot be compiled on their own,
they can only be compiled via the main file.
A naive editing environment
(such as a text editor with an option
to have the current file processed by \LaTeX)
may require one to switch to the main file before compiling;
attempting to compile the child file produces errors.
\item
The main file must be modified (each time)
to adjust the |\includeonly| command
to the present needs. This easily leaves the main file in a messy state.
\item
The generated document will always carry the filename
of the main document. This is inconvenient if
several child files are to be compiled and
to be kept for distribution.
\end{itemize}

The present package provides a simple interface
to make child files individually compilable by \LaTeX{}.
Compiling a child file then has the same effect as compiling
the main file with an |\includeonly| command
to select the appropriate child.
Moreover the generated document will carry the name of the child
rather than the main file.
This resolves all three above issues.

This feature is meant to make the editing of books,
thesis documents and lecture notes somewhat more convenient.
However, the package can also be used efficiently for
composing a series of documents (such as exercise sheets)
which are typically distributed individually.
It then assists the author in generating the individual documents
(potentially in different versions)
as well as a document containing the collected series.
Another application is in developing style files
or other kinds of included material
where compilation of the style file could redirect
to a sample or test file.

%%%%%%%%%%%%%%%%%%%%%%%%%%%%%%%%%%%%%%%%%%%%%%%%%%%%%%%%%%%%%%%%%%%%%%%%%%%%%%%%
%%%%%%%%%%%%%%%%%%%%%%%%%%%%%%%%%%%%%%%%%%%%%%%%%%%%%%%%%%%%%%%%%%%%%%%%%%%%%%%%
\section{Usage}

First of all, the package \textsf{childdoc} is \emph{not} a standard
\LaTeXe{} |.sty| style file! Therefore it needs to be invoked in
a non-standard way.

%%%%%%%%%%%%%%%%%%%%%%%%%%%%%%%%%%%%%%%%%%%%%%%%%%%%%%%%%%%%%%%%%%%%%%%%%%%%%%%%
\subsection{Included Files}
\label{sec:include}

%%%%%%%%%%%%%%%%%%%%%%%%%%%%%%%%%%%%%%%%
\DescribeMacro{\childdocmain}
To use the package, add the commands
\begin{center}
\begin{tabular}{l}
|\input{childdoc.def}|\\
|\childdocmain{}|\\
\end{tabular}
\end{center}
at the very top of the main \LaTeX{} file,
in particular \emph{before} the |\documentclass| statement!
The argument of |\childdocmain| should be left empty
(but it must be present).

%%%%%%%%%%%%%%%%%%%%%%%%%%%%%%%%%%%%%%%%
\DescribeMacro{\childdocof}
Furthermore, add the commands
\begin{center}
\begin{tabular}{l}
|\input{childdoc.def}|\\
|\childdocof{|\textit{main}|}|\\
\end{tabular}
\end{center}
at the top of every child file \textit{child}
which is included by |\include{|\textit{child}|}|
from within the main file
(or at least for those files to be compiled individually).
The argument \textit{main} must be the filename of the main file.

There are a couple of
considerations in setting up the main and child documents:

%%%%%%%%%%%%%%%%%%%%%%%%%%%%%%%%%%%%%%%%
\paragraph{Restrictions.}

Please note the following restrictions:
\begin{itemize}
\item
|\childdocmain| must be called with one argument \textit{main}
to ensure compatibility with earlier version of the package.
It must either be empty (|\childdocmain{}|)
or precisely match the filename of the main file in which it is specified.
See \secref{sec:detection} for further information.
\item
The filename \textit{main} must be specified without the |.tex| extension.
\item
The filename \textit{main} is case sensitive
(even in case-insensitive file systems)
due to internal string comparison.
\item
The argument \textit{main} should be fully expanded, it cannot be a macro.
\item
Subdirectories and special characters should be avoided in filenames.
\item
The command |\childdocmain{|\textit{main}|}| must be followed by a whitespace.
It should not be followed immediately by another command
or by a comment mark `|%|'.
This is because the \TeX{} parser reads the token immediately following
the argument of |\childdocmain| and puts it
at the beginning of every child section;
however, a white\-space is ignored.
\end{itemize}

%%%%%%%%%%%%%%%%%%%%%%%%%%%%%%%%%%%%%%%%
\paragraph{Content of Main File.}

It is advisable to place all content in the child files included by |\include|.
Any output contained in the main file will appear in all child documents
unless suppressed manually;
it cannot be suppressed automatically by the |\includeonly| directive
and thus should normally be avoided.
A method to include some content in the main file
by means of conditional processing is described in \secref{sec:conditional}.

%%%%%%%%%%%%%%%%%%%%%%%%%%%%%%%%%%%%%%%%
\paragraph{Page Numbering.}

When only a part of the document is compiled,
the appropriate numbering of pages
(as well as other status parameters)
is determined from the |.aux| files.
The latter contain information from previous passes.
However this information needs to propagate through
all intermediate child documents.
Therefore the page numbering in child documents may well
be inconsistent until the complete document is compiled at least once.

A useful (if unconventional) way to always ensure a consistent
page numbering is to restart the numbering in each child document
and denote the pages by `\textit{child}|.|\textit{page}'
where \textit{child} represents the chapter/section number of the child file.
This can be achieved by the command
|\numberwithin{page}{|\textit{child}|}|
of the \textsf{amsmath} package
where \textit{child} can be |chapter| or |section|
depending on the chosen structuring.
Alternatively, one can modify the macro |\thepage| appropriately
and reset the counter |page| at the start of each child file.

%%%%%%%%%%%%%%%%%%%%%%%%%%%%%%%%%%%%%%%%%%%%%%%%%%%%%%%%%%%%%%%%%%%%%%%%%%%%%%%%
\subsection{Conditional Processing}
\label{sec:conditional}

The package provides a mechanism to compile different versions
of a document. To customise the versions further some conditional processing
can come in handy to distinguish which version is being compiled.
The package provides two macros to describe the compilation context:

%%%%%%%%%%%%%%%%%%%%%%%%%%%%%%%%%%%%%%%%
\DescribeMacro{\ifchilddoc}
The conditional |\ifchilddoc| distinguishes between the compilation of
child documents and the main document:
%
\begin{center}
|\ifchilddoc |\textit{child-code}| |[|\||else |\textit{main-code}]| \||fi|
\end{center}

%%%%%%%%%%%%%%%%%%%%%%%%%%%%%%%%%%%%%%%%
\DescribeMacro{\childdocname}
\DescribeMacro{\childdocjob}
The macro |\childdocname| contains the filename (without extension)
of the main or child file being processed.
Note that |\childdocjob| will always contain the name of the main file.

%%%%%%%%%%%%%%%%%%%%%%%%%%%%%%%%%%%%%%%%
\paragraph{Title Page.}

Conditional processing can be used to include a title or banner page
in the main document when proper precautions are taken.
Importantly, the code in the main file should ensure that the page counter
(as well as other status parameters which are stored in the |.aux| files)
takes the same value after the conditional processing.
Otherwise the page numbers may take divergent values
depending on which part is compiled.

For example, a title page could be declared by:
%
\begin{center}
\begin{tabular}{l}
|\ifchilddoc\||else|\\
|\addtocounter{page}{-1}|\\
\textit{code for title page}\\
|\newpage|\\
|\||fi|
\end{tabular}
\end{center}
%
A banner page for the child documents can be generated by:
%
\begin{center}
\begin{tabular}{l}
|\ifchilddoc|\\
|\addtocounter{page}{-1}|\\
\textit{code for banner page}\\
|\newpage|\\
|\||fi|
\end{tabular}
\end{center}
%
Here one could write a message such as:
\begin{center}
|This is the part \childdocname{} of \childdocjob{}.|
\end{center}

%%%%%%%%%%%%%%%%%%%%%%%%%%%%%%%%%%%%%%%%%%%%%%%%%%%%%%%%%%%%%%%%%%%%%%%%%%%%%%%%
\subsection{Flags}
\label{sec:flags}

The package makes it easy to generate different versions
of the main or child documents.
To this end compilation flags can be defined
and assigned different default values.
They will be particularly useful in conjunction
with the forwarding mechanism described in \secref{sec:forward}.

For example, it may be useful to have a flag |\version|
which can be set to |draft| or |final|.
The document source will contain some conditional code
depending on the value of |\version|.
Suppose further, the flag should default to |final| for the main file
and to |draft| for child files
which is a natural assignment for editing the document.
This is achieved by placing the following code
in the preamble of the main document
(below the |\childdocmain| directive):
%
\begin{center}
\begin{tabular}{l}
|\ifchilddoc|\\
|\providecommand{\version}{draft}|\\
|\||else|\\
|\providecommand{\version}{final}|\\
|\||fi|
\end{tabular}
\end{center}
%
The definition by |\providecommand| makes sure
that previous definitions are not overwritten.
Further statements |\providecommand{\version}{...}|
can thus be added before the above code to override it.

For the main file, one might add a line
(between |\childdocmain| and the above block)
%
\begin{center}
|%\ifchilddoc\||else\providecommand{\version}{draft}\||fi|
\end{center}
%
which can be uncommented to produce a draft version.
Likewise one can add a line to the very top of a child file
(above the |\childdocof{|\textit{main}|}| directive)
%
\begin{center}
|%\providecommand{\version}{final}|
\end{center}
%
which can be uncommented to produce the final version of this child document.

%%%%%%%%%%%%%%%%%%%%%%%%%%%%%%%%%%%%%%%%%%%%%%%%%%%%%%%%%%%%%%%%%%%%%%%%%%%%%%%%
\subsection{Forwarding}
\label{sec:forward}

Different versions of the main or child documents
using compilation flags as described in \secref{sec:flags}
can be (permanently) stored in different files
for convenient compilation, viewing and distribution.
To this end, the package defines a command
to pass on compilation to a different file:

%%%%%%%%%%%%%%%%%%%%%%%%%%%%%%%%%%%%%%%%
\DescribeMacro{\childdocforward}
The command |\childdocforward| redirects processing to
another source file:
%
\begin{center}
\begin{tabular}{l}
|\input{childdoc.def}|\\
|\childdocforward[|\textit{main}|]{|\textit{dest}|}|\\
\end{tabular}
\end{center}
%
The argument \textit{dest} is the destination file
(without extension).
It should be the main file or one of the child files.
Note that further \textsf{childdoc} directives
such as |\childdocof| and |\childdocforward|
in the indicated file will be processed in this form.
The optional argument \textit{main}
passes on directly to the main file \textit{main}
while pretending to compile the child \textit{dest}.
This form behaves as if \textit{dest}
issues |\childdocof{|\textit{main}|}| right away,
and no further \textsf{childdoc} directives will be processed.

%%%%%%%%%%%%%%%%%%%%%%%%%%%%%%%%%%%%%%%%
\DescribeMacro{\...prefix}
In the alternative form |\childdocforwardprefix|,
%
\begin{center}
\begin{tabular}{l}
|\input{childdoc.def}|\\
|\childdocforwardprefix[|\textit{main}|]{|\textit{prefix}|}{|\textit{dest}|}|
\end{tabular}
\end{center}
%
the destination file is determined by a pattern
depending on the current file:
To make this work, the current file must be called
`{\textit{prefix}\hspace{0.2em}\textit{suffix}}'
with \textit{prefix} matching precisely the argument.
Processing is then passed on to the file
`{\textit{dest}\hspace{0.2em}\textit{suffix}}'.
Surely, the same effect is achieved by
directly specifying the
argument `{\textit{dest}\hspace{0.2em}\textit{suffix}}'
in the first form.
However, that requires to set up a different file
for each child. With the alternative form of the command
all these files can have exactly the same content
which simplifies setting them up and maintaining them.

For example, the following file |draft.tex|
with a compilation flag |\version| as described in \secref{sec:flags}
compiles the main document as a draft:
%
\begin{center}
\begin{tabular}{l}
|\def\version{draft}|\\
|\input{childdoc.def}|\\
|\childdocforward{|\textit{main}|}|
\end{tabular}
\end{center}
%
Likewise, the following files |final|\textit{nn}|.tex|
compile the final version of the child document
|child|\textit{nn}|.tex|:
%
\begin{center}
\begin{tabular}{l}
|\def\version{final}|\\
|\input{childdoc.def}|\\
|\childdocforwardprefix{final}{child}|
\end{tabular}
\end{center}
%

Note that when several versions of a main file and/or of each child file
are to be generated, it may be convenient to set up a |Makefile| or
shell script to automatise the process.

%%%%%%%%%%%%%%%%%%%%%%%%%%%%%%%%%%%%%%%%%%%%%%%%%%%%%%%%%%%%%%%%%%%%%%%%%%%%%%%%
\subsection{Command Line Processing}
\label{sec:commandline}

The effect of redirection files can also be achieved by invoking
the \LaTeX{} compiler with a more elaborate command line.
Most conveniently this should be done as part
of a shell script or a |Makefile|.

When using \textsf{childdoc} in the main file, the following
command lines effectively perform a redirection
(note that depending on the shell being used,
backslashes may have to be doubled: `|\|' $\to$ `|\\|'):
%
\begin{center}
|... -jobname "|\textit{target}|" |\\|"|[\textit{flags}]%
|\input{childdoc.def}\childdocforward[|\textit{main}|]{|\textit{dest}|}"|
\end{center}
%
Here \textit{target} is the name of the output file,
\textit{main} is the name of the main file
and \textit{dest} is the name of the main or child file to be processed
(all filenames without extensions).
The optional argument \textit{main} can be omitted
if \textit{main} matches \textit{dest}.
Optionally, compilation \textit{flags} can be defined via |\def| commands.
This command line makes the \TeX{} engine believe
it is compiling the file \textit{target}
whose content is specified as the latter parameter.
The provided code then forwards the processing to
\textit{main} or \textit{dest} as described in \secref{sec:forward}.

%%%%%%%%%%%%%%%%%%%%%%%%%%%%%%%%%%%%%%%%%%%%%%%%%%%%%%%%%%%%%%%%%%%%%%%%%%%%%%%%
\subsection{Include by Input}
\label{sec:input}

Including child documents by |\include| has some restrictions by design.
Most notably, the content of a child document always occupies
its own set of pages; pages cannot be shared between child documents.
Usually, this behaviour makes perfect sense
because each child document contain an essential part of the document.
However, in some situations it may be desirable to compose
a document from a collection of parts
without having mandatory page breaks between then.
For this case, the package
provides a mechanism to include parts
by |\input| which can also be processed individually.
However, by construction this mechanism
requires manual handling of the content to be output.

%%%%%%%%%%%%%%%%%%%%%%%%%%%%%%%%%%%%%%%%
\DescribeMacro{\ifchilddocmanual}
The main file should be prepared as usual, see \secref{sec:include}.
However, the document body must make a distinction
between processing of an individual part and of the main document, e.g.:
%
\begin{center}
\begin{tabular}{l}
|\ifchilddocmanual|\\
|\input{\childdocname}|\\
|\||else|\\
\textit{document body with }|\input{|\textit{part}|}|\\
|\||fi|
\end{tabular}
\end{center}
%
The conditional |\ifchilddocmanual| is true whenever
a part to be included by |\input| is being compiled,
and the name of the part is stored in |\childdocname|.

%%%%%%%%%%%%%%%%%%%%%%%%%%%%%%%%%%%%%%%%
\DescribeMacro{\childdocby}
Each part to be included by |\input| should start with:
%
\begin{center}
\begin{tabular}{l}
|\input{childdoc.def}|\\
|\childdocby{|\textit{main}|}|\\
\end{tabular}
\end{center}
%
The directive |\childdocby| is similar to |\childdocof|
described in \secref{sec:include},
but the subsequent selection of content must be done manually.
To that end, both |\ifchilddoc| and |\ifchilddocmanual|
will be true upon processing of a part,
and the name of the part is stored in |\childdocname|.
Note that |\jobname| will be set to the filename of the current part
so that each part receives an individual |.aux| file
that does not interfere with the |.aux| file(s) of the main document.
This behaviour can be altered by the alternative form
|\childdocby[*]{|\textit{main}|}| (with a non-empty optional argument)
which uses the |.aux| file of the main document
by setting |\jobname| to \textit{main}.

%%%%%%%%%%%%%%%%%%%%%%%%%%%%%%%%%%%%%%%%%%%%%%%%%%%%%%%%%%%%%%%%%%%%%%%%%%%%%%%%
\subsection{Driver Development}
\label{sec:driver}

The \textsf{childdoc} mechanism can also be use for the development
of definition files such as \LaTeX{} styles or classes.
This case differs from the above setup with multiple parts
included by |\include| in that no |\includeonly| should be invoked.
This can be achieved by starting the include file
(before |\ProvidesPackage|) with:
%
\begin{center}
\begin{tabular}{l}
|\input{childdoc.def}|\\
|\childdocforward{|\textit{main}|}|\\
\end{tabular}
\end{center}
%
or alternatively with:
%
\begin{center}
\begin{tabular}{l}
|\input{childdoc.def}|\\
|\childdocby{|\textit{main}|}|\\
\end{tabular}
\end{center}
%
Both forms have slightly different effects as described above.
The main file is prepared as usual, see \secref{sec:include}.

%%%%%%%%%%%%%%%%%%%%%%%%%%%%%%%%%%%%%%%%%%%%%%%%%%%%%%%%%%%%%%%%%%%%%%%%%%%%%%%%
\subsection{Legacy Detection}
\label{sec:detection}

The directive |\childdocmain| in the main file can detect
whether the complete document or merely a child is to be compiled
even without using the directive |\childdocof|.
This method is deprecated because it is less robust
and there is no compelling reason to use it;
it is merely provided for backward compatibility
and it may be removed in future versions.

If the detection mechanism is to be used,
it is mandatory to correctly specify
the filename of the main file as the argument of |\childdocmain|:
%
\begin{center}
\begin{tabular}{l}
|\input{childdoc.def}|\\
|\childdocmain{|\textit{main}|}|\\
\end{tabular}
\end{center}
%
If |\jobname| does not match the argument \textit{main} of |\childdocmain|,
it is assumed that |\jobname| points to the child file to be compiled.
When using |\childdocmain| with the main file specified as argument,
it suffices to start a child file
with just |\input{|\textit{main}|}|
without loading of the package and using |\childdocof|.
If instead all processing is done
with the appropriate \textsf{childdoc} directives,
the argument of \textit{main} of |\childdocmain| can be empty.

An alternative version of the command line processing described
in \secref{sec:commandline} using the detection mechanism reads:
%
\begin{center}
|... -jobname "|\textit{target}|" "|[\textit{flags}]%
[|\def\jobname{|\textit{dest}|}|]|\input{|\textit{main}|}"|
\end{center}

%%%%%%%%%%%%%%%%%%%%%%%%%%%%%%%%%%%%%%%%%%%%%%%%%%%%%%%%%%%%%%%%%%%%%%%%%%%%%%%%
\subsection{Manual Code}
\label{sec:manual}

In case one cannot be certain whether the definitions file |childdoc.def|
is installed on the target \TeX{} distribution
and one prefers not to ship it,
it is conceivable to paste a few relevant commands into the sources.

To that end, drop all statements |\input{childdoc.def}|
and perform the replacements as outlined below.
Instead of |\childdocmain{|\textit{main}|}| add the following code
to the top of the main file:
%
\begin{center}
\begin{tabular}{l}
|\||ifdefined\childdocname\endinput\||fi\newif\ifchilddoc|\\
|\edef\childdocname{\scantokens\expandafter{\jobname\noexpand}}|\\
|\def\childdocmain{|\textit{main}|}\||ifx\childdocmain\childdocname\||else|\\
|\childdoctrue\includeonly{\childdocname}\let\jobname\childdocmain\||fi|\\
\end{tabular}
\end{center}
%
Instead of |\childdocof{|\textit{main}|}| just include the main file
at the top of each child file:
%
\begin{center}
|\input{|\textit{main}|}|
\end{center}
%
A simple redirection |\childdocforward{|\textit{dest}|}| is achieved by:
%
\begin{center}
|\def\jobname{|\textit{dest}|}\input{\jobname}|
\end{center}
%
The redirection with prefix
|\childdocforwardprefix[|\textit{prefix}|]{|\textit{dest}|}|
is accomplished by:
%
\begin{center}
\begin{tabular}{l}
|{\edef\jobname{\scantokens\expandafter{\jobname\noexpand}}|\\
|\def\redirectjob |\textit{prefix}|#1~~~{\gdef\jobname{|\textit{dest}|#1}}|\\
|\expandafter\redirectjob\jobname~~~}\input{\jobname}|
\end{tabular}
\end{center}

In an alternative approach,
child documents can be compiled by a specific command line
without additional code or specific definitions:
%
\begin{center}
|... -jobname "|\textit{target}|" "|[\textit{flags}]%
|\includeonly{|\textit{dest}|}\input{|\textit{main}|}"|
\end{center}
%

%%%%%%%%%%%%%%%%%%%%%%%%%%%%%%%%%%%%%%%%%%%%%%%%%%%%%%%%%%%%%%%%%%%%%%%%%%%%%%%%
%%%%%%%%%%%%%%%%%%%%%%%%%%%%%%%%%%%%%%%%%%%%%%%%%%%%%%%%%%%%%%%%%%%%%%%%%%%%%%%%
\section{Information}

%%%%%%%%%%%%%%%%%%%%%%%%%%%%%%%%%%%%%%%%%%%%%%%%%%%%%%%%%%%%%%%%%%%%%%%%%%%%%%%%
\subsection{Copyright}

Copyright \copyright{} 2017--2018 Niklas Beisert

This work may be distributed and/or modified under the
conditions of the \LaTeX{} Project Public License, either version 1.3
of this license or (at your option) any later version.
The latest version of this license is in
  \url{http://www.latex-project.org/lppl.txt}
and version 1.3 or later is part of all distributions of \LaTeX{}
version 2005/12/01 or later.

This work has the LPPL maintenance status `maintained'.

The Current Maintainer of this work is Niklas Beisert.

This work consists of the files |README.txt|, |childdoc.ins| and |childdoc.dtx|
as well as the derived files |childdoc.def|, |cdocsamp.tex|
with |cdocsch1.tex|, |cdocsch2.tex|, |cdocspt3.tex|, |cdocspt4.tex|,
|cdocsdrf.tex|, |cdocsfn1.tex|, |cdocsfn2.tex|
as well as |childdoc.pdf|.

%%%%%%%%%%%%%%%%%%%%%%%%%%%%%%%%%%%%%%%%%%%%%%%%%%%%%%%%%%%%%%%%%%%%%%%%%%%%%%%%
\subsection{Files and Installation}

The package consists of the files:
%
\begin{center}
\begin{tabular}{ll}
    |README.txt|   & readme file \\
    |childdoc.ins| & installation file \\
    |childdoc.dtx| & source file \\
    |childdoc.def| & definition file \\
    |cdocsamp.tex| & sample main file \\
    |cdocsch1.tex| & sample include file \\
    |cdocsch2.tex| & sample include file \\
    |cdocspt3.tex| & sample part file \\
    |cdocspt4.tex| & sample part file \\
    |cdocsdrf.tex| & sample redirection file \\
    |cdocsfn1.tex| & sample redirection file \\
    |cdocsfn2.tex| & sample redirection file \\
    |childdoc.pdf| & manual
\end{tabular}
\end{center}
%
The distribution consists of the files
|README.txt|, |childdoc.ins| and |childdoc.dtx|.
%
\begin{itemize}
\item
Run (pdf)\LaTeX{} on |childdoc.dtx|
to compile the manual |childdoc.pdf| (this file).
\item
Run \LaTeX{} on |childdoc.ins| to create the definitions file |childdoc.def|
and the sample |cdocsamp.tex| with include files
|cdocsch1.tex|, |cdocsch2.tex|, |cdocspt3.tex|, |cdocspt4.tex|,
|cdocsdrf.tex|, |cdocsfn1.tex|, |cdocsfn2.tex|.
Then copy the file |childdoc.def| to an appropriate directory of your \LaTeX{}
distribution, e.g.\ \textit{texmf-root}|/tex/latex/childdoc|.
\end{itemize}

%%%%%%%%%%%%%%%%%%%%%%%%%%%%%%%%%%%%%%%%%%%%%%%%%%%%%%%%%%%%%%%%%%%%%%%%%%%%%%%%
\subsection{Related CTAN Packages}

There are several other packages which offer a similar functionality:
%
\begin{itemize}
\item
The packages
\href{http://ctan.org/pkg/docmute}{\textsf{docmute}},
\href{http://ctan.org/pkg/includex}{\textsf{includex}} and
\href{http://ctan.org/pkg/standalone}{\textsf{standalone}}
provide commands to include only the document body of
a child file thus allowing both files to be compiled individually.
\item
The packages \href{http://ctan.org/pkg/subdocs}{\textsf{subdocs}}
and \href{http://ctan.org/pkg/subfiles}{\textsf{subfiles}}
provide structures in which the main and child documents can be
encapsulated and allowing them to be compiled individually.
The inclusion mechanism is different from the conventional |\include|.
\item
The package \href{http://ctan.org/pkg/combine}{\textsf{combine}}
is an elaborate solution to combine several documents into one.
\end{itemize}
%
See also the CTAN topic \href{http://ctan.org/topic/subdocs}{\textsf{subdocs}}
for further related packages.
The present package differs from the above solutions in that
a document structure constructed with the conventional |\include| mechanism
just needs two extra commands at the top of every file
such that all constituent files can be compiled individually.

%%%%%%%%%%%%%%%%%%%%%%%%%%%%%%%%%%%%%%%%%%%%%%%%%%%%%%%%%%%%%%%%%%%%%%%%%%%%%%%%
%\subsection{Feature Suggestions}
%
%The following is a list of features which may be useful for future
%versions of this package:
%%
%\begin{itemize}
%\item
%\ldots
%\end{itemize}

%%%%%%%%%%%%%%%%%%%%%%%%%%%%%%%%%%%%%%%%%%%%%%%%%%%%%%%%%%%%%%%%%%%%%%%%%%%%%%%%
\subsection{Revision History}

%%%%%%%%%%%%%%%%%%%%%%%%%%%%%%%%%%%%%%%%
\paragraph{v2.0:} 2018/12/30

\begin{itemize}
\item
immediate forward processing
\item
added |\childdocby| mechanism
\item
manual restructured
\end{itemize}

%%%%%%%%%%%%%%%%%%%%%%%%%%%%%%%%%%%%%%%%
\paragraph{v1.6:} 2018/01/17

\begin{itemize}
\item
application for development of include files
\item
corrections to manual
\end{itemize}

%%%%%%%%%%%%%%%%%%%%%%%%%%%%%%%%%%%%%%%%
\paragraph{v1.5:} 2017/05/21

\begin{itemize}
\item
more complete structuring introduced
\item
|\childdocof| introduced
\item
|\childdoc| renamed to |\childdocmain|
\item
|\childredirect| renamed to |\childdocforward| and |\childdocforwardprefix|
and functionality expanded
\end{itemize}

%%%%%%%%%%%%%%%%%%%%%%%%%%%%%%%%%%%%%%%%
\paragraph{v1.0:} 2017/04/27

\begin{itemize}
\item
manual and install package
\item
first version published on CTAN
\end{itemize}

%%%%%%%%%%%%%%%%%%%%%%%%%%%%%%%%%%%%%%%%
\paragraph{v0.6:} 2017/04/26

\begin{itemize}
\item
redirection mechanism added
\end{itemize}

%%%%%%%%%%%%%%%%%%%%%%%%%%%%%%%%%%%%%%%%
\paragraph{v0.5:} 2017/04/26

\begin{itemize}
\item
functionality in definition file
\end{itemize}


%%%%%%%%%%%%%%%%%%%%%%%%%%%%%%%%%%%%%%%%%%%%%%%%%%%%%%%%%%%%%%%%%%%%%%%%%%%%%%%%
%%%%%%%%%%%%%%%%%%%%%%%%%%%%%%%%%%%%%%%%%%%%%%%%%%%%%%%%%%%%%%%%%%%%%%%%%%%%%%%%
%%%%%%%%%%%%%%%%%%%%%%%%%%%%%%%%%%%%%%%%%%%%%%%%%%%%%%%%%%%%%%%%%%%%%%%%%%%%%%%%
\appendix

\settowidth\MacroIndent{\rmfamily\scriptsize 000\ }

 \DocInput{childdoc.dtx}

\end{document}
%</driver>
% \fi
%
% %%%%%%%%%%%%%%%%%%%%%%%%%%%%%%%%%%%%%%%%%%%%%%%%%%%%%%%%%%%%%%%%%%%%%%%%%%%%%%
% %%%%%%%%%%%%%%%%%%%%%%%%%%%%%%%%%%%%%%%%%%%%%%%%%%%%%%%%%%%%%%%%%%%%%%%%%%%%%%
% \section{Sample}
%\iffalse
%<*samplemain>
%\fi
%
% The following presents a sample document
% with two chapters, two parts, a title page,
% a compile flag as well as three forwarding files to set the flag.
% It consists of eight |.tex| files:
% \begin{center}
% \begin{tabular}{ll}
% |cdocsamp.tex|&main file\\
% |cdocsch1.tex|&include file for chapter 1\\
% |cdocsch2.tex|&include file for chapter 2\\
% |cdocspt3.tex|&include file for part 3\\
% |cdocspt4.tex|&include file for part 4\\
% |cdocsdrf.tex|&forwarding file for main file in draft mode\\
% |cdocsfi1.tex|&forwarding file for final version of chapter 1\\
% |cdocsfi2.tex|&forwarding file for final version of chapter 2\\
% \end{tabular}
% \end{center}
% Each of the eight files can be compiled directly by the \LaTeX{} compiler.
%
% %%%%%%%%%%%%%%%%%%%%%%%%%%%%%%%%%%%%%%
% \paragraph{Main File.}
%
% The main file is called |cdocsamp.tex|.
%
% Load the \textsf{childdoc} definitions and
% declare the filename for the main document:
%    \begin{macrocode}
\input{childdoc.def}
\childdocmain{}
%    \end{macrocode}

% Optional override for |\version| flag:
%    \begin{macrocode}
%%\ifchilddoc\else\providecommand{\version}{draft}\fi
%    \end{macrocode}

% Define the default values for the |\version| flag
% (|final| for the main file and |draft| for childs):
%    \begin{macrocode}
\ifchilddoc
\providecommand{\version}{draft}
\else
\providecommand{\version}{final}
\fi
%    \end{macrocode}

% Load the standard document class:
%    \begin{macrocode}
\documentclass[12pt]{article}
%    \end{macrocode}

% Start the document body:
%    \begin{macrocode}
\begin{document}
%    \end{macrocode}

% Declare a title page.
% Print title, part of document being processed and version flag:
%    \begin{macrocode}
\addtocounter{page}{-1}
\begin{center}
{\LARGE\bfseries{}childdoc example\par}
\vspace{1cm}
\ifchilddoc
\ifchilddocmanual part\else chapter\fi:
`\childdocname' of `\childdocjob'\par
\else
main document: `\childdocjob'\par
\fi
version: \version\par
\end{center}
\newpage
%    \end{macrocode}

% Manually include selected file,
% otherwise process as usual:
%    \begin{macrocode}
\ifchilddocmanual
\section*{part `\childdocname'}
\input{\childdocname}
\else
%    \end{macrocode}

% Include the two chapters:
%    \begin{macrocode}
\include{cdocsch1}
\include{cdocsch2}
%    \end{macrocode}

% Include the two parts unless only chapters should be displayed:
%    \begin{macrocode}
\ifchilddoc\else
\section{part three}
\input{cdocspt3}
\section{part four}
\input{cdocspt4}
\fi
%    \end{macrocode}

% Process as usual until here:
%    \begin{macrocode}
\fi
%    \end{macrocode}

% End of document body:
%    \begin{macrocode}
\end{document}
%    \end{macrocode}
%\iffalse
%</samplemain>
%\fi
%
% %%%%%%%%%%%%%%%%%%%%%%%%%%%%%%%%%%%%%%
% \paragraph{Chapter Include Files.}
%
% The include files are called |cdocsch1.tex| and |cdocsch2.tex|.
%
%\iffalse
%<*samplechap1|samplechap2>
%\fi

% Optional override for |\version| flag:
%    \begin{macrocode}
%%\providecommand{\version}{final}
%    \end{macrocode}

% Include the main document:
%    \begin{macrocode}
\input{childdoc.def}
\childdocof{cdocsamp}
%    \end{macrocode}

%\iffalse
%</samplechap1|samplechap2>
%\fi
%
%\iffalse
%<*samplechap1>
%\fi
% Some text for chapter 1:
%    \begin{macrocode}
\section{one}
some text in chapter one
%    \end{macrocode}

%\iffalse
%</samplechap1>
%\fi
% Some text for chapter 2:
%\iffalse
%<*samplechap2>
%\fi
%    \begin{macrocode}
\section{two}
more text in chapter two
%    \end{macrocode}

%\iffalse
%</samplechap2>
%\fi
%
% %%%%%%%%%%%%%%%%%%%%%%%%%%%%%%%%%%%%%%
% \paragraph{Part Include Files.}
%
% The include files are called |cdocspt3.tex| and |cdocspt4.tex|.
%
%\iffalse
%<*samplepart3|samplepart4>
%\fi

% Optional override for |\version| flag:
%    \begin{macrocode}
%%\providecommand{\version}{final}
%    \end{macrocode}

% Include the main document:
%    \begin{macrocode}
\input{childdoc.def}
\childdocby{cdocsamp}
%    \end{macrocode}

%\iffalse
%</samplepart3|samplepart4>
%\fi
%
%\iffalse
%<*samplepart3>
%\fi
% Some text for part 3:
%    \begin{macrocode}
some text in part three
%    \end{macrocode}

%\iffalse
%</samplepart3>
%\fi
% Some text for part 4:
%\iffalse
%<*samplepart4>
%\fi
%    \begin{macrocode}
more text in part four
%    \end{macrocode}

%\iffalse
%</samplepart4>
%\fi
%
% %%%%%%%%%%%%%%%%%%%%%%%%%%%%%%%%%%%%%%
% \paragraph{Forwarding for a Complete Draft.}
%
% The following forwarding file |cdocsdrf.tex|
% compiles the main document in draft mode:
%\iffalse
%<*sampledraft>
%\fi
%    \begin{macrocode}
\def\version{draft}
\input{childdoc.def}
\childdocforward{cdocsamp}
%    \end{macrocode}

%\iffalse
%</sampledraft>
%\fi
%
% %%%%%%%%%%%%%%%%%%%%%%%%%%%%%%%%%%%%%%
% \paragraph{Forwarding for Final Version of the Chapters.}
%
% The following forwarding files |cdocsfn1.tex| and |cdocsfn2.tex|
% (with identical content)
% compile the final versions of the child documents
% |cdocsch1.tex| and |cdocsch2.tex|, respectively:
%\iffalse
%<*samplefinal>
%\fi
%    \begin{macrocode}
\def\version{final}
\input{childdoc.def}
\childdocforwardprefix[cdocsamp]{cdocsfn}{cdocsch}
%    \end{macrocode}

%\iffalse
%</samplefinal>
%\fi
%
% %%%%%%%%%%%%%%%%%%%%%%%%%%%%%%%%%%%%%%
% \paragraph{Command Line Processing.}
%
% The following three command lines generate the output files
% |cdocscld|, |cdocscl1| and |cdocscl2|
% which should be identical to
% |cdocsdrf|, |cdocsch1| and |cdocsfn2|, respectively:
% \begin{center}
% \begin{tabular}{l}
% |latex -jobname cdocscld \|\\
% |  "\def\version{draft}\input{childdoc.def}\childdocforward{cdocsamp}"|\\
% |latex -jobname cdocscl1 \|\\
% |  "\input{childdoc.def}\childdocforward[cdocsamp]{cdocsch1}"|\\
% |latex -jobname cdocscl2 \|\\
% |  "\def\version{final}\input{childdoc.def}\childdocforward{cdocsch2}"|
% \end{tabular}
% \end{center}
% Note that the trailing backslash on each first line
% merely continues the input to the second line
% (for convenient cut ant paste).
% Furthermore, the command |latex| can be replaced by any
% of its alternative versions such as |pdflatex|.
%
% %%%%%%%%%%%%%%%%%%%%%%%%%%%%%%%%%%%%%%%%%%%%%%%%%%%%%%%%%%%%%%%%%%%%%%%%%%%%%%
% %%%%%%%%%%%%%%%%%%%%%%%%%%%%%%%%%%%%%%%%%%%%%%%%%%%%%%%%%%%%%%%%%%%%%%%%%%%%%%
% \section{Implementation}
%\iffalse
%<*package>
%\fi
%
% This section describes the definitions file |childdoc.def|.

% The definitions cannot be loaded using |\usepackage| or |\RequirePackage|
% which has a mechanism to prevent loading a style file more than once.
% When loading the definitions by means of |\input|
% multiple instances have to be prevented manually:
%\iffalse
%This code needs to be before the `\ProvidesFile' directive
%which is defined at the beginning of this file.
%Therefore it is also placed there and commented out here.
%</package>
%<*discard>
%\fi
%    \begin{macrocode}
\ifdefined\childdocmain\endinput\fi
%    \end{macrocode}
%\iffalse
%</discard>
%<*package>
%\fi
%
% \macro{\ifchilddoc}
% \macro{\ifchilddocmanual}
% The conditional |\ifchilddoc| tells whether a
% child (true) or main (false) document is being compiled.
% The conditional |\ifchilddocmanual| tells whether
% the |\includeonly| mechanism is used (false) or
% the selection of child files must be performed manually (true).
% The definitions initialise to false:
%    \begin{macrocode}
\newif\ifchilddoc
\newif\ifchilddocmanual
%    \end{macrocode}

% \macro{\childdocname}
% \macro{\childdocjob}
% The macro |\childdocname| stores the name of the main document
% to be compiled. The macro |\childdocjob| stores the name of
% the document on which the \LaTeX{} compiler was originally invoked.
% The content of |\jobname| cannot be compared
% to filenames specified in the source due to different catcodes.
% The following code rescans |\jobname|, stores the result
% in |\childdocname| and saves a copy in |\childdocjob|:
%    \begin{macrocode}
\edef\childdocname{\scantokens\expandafter{\jobname\noexpand}}
\let\childdocjob\childdocname
%    \end{macrocode}

% \macro{\childdocdisable}
% The macro |\childdocdisable| prevents the main file
% from being processed more than once.
% At this stage, the main document command |\childdocmain|
% is assumed to be called once again where it should do nothing.
% Any subsequent call to it should prevent
% a secondary processing of the main document
% It overwrites the forwarding commands
% |\childdocof| and |\childdocforward|
% with empty macros to prevent further inclusions of the main document:
%    \begin{macrocode}
\newcommand{\childdocdisable}
{
  \renewcommand{\childdocmain}[1]{\renewcommand{\childdocmain}[1]{\endinput}}
  \renewcommand{\childdocof}[1]{}
  \renewcommand{\childdocby}[2][]{}
  \renewcommand{\childdocforward}[2][]{}
  \renewcommand{\childdocdisable}{}
}
%    \end{macrocode}

% \macro{\childdocmain}
% The macro |\childdocmain| is to be called at the top of the main file
% with nothing or the main filename (without extension) as argument.
% First, it breaks loops.
% If the argument is not empty and does not match |\childdocname|
% (which is set by the first inclusion of |childdoc.def|),
% |\ifchilddoc| is set to true, |\includeonly| is applied to the child file
% and |\jobname| is set to the main file
% (for proper handling of |.aux| files):
%    \begin{macrocode}
\newcommand{\childdocmain}[1]
{
  \childdocdisable\childdocmain{}
  \if?#1?\else
    \begingroup
      \def\childdoctmp{#1}
      \ifx\childdoctmp\childdocname
        \def\childdoctmp{}
      \else
        \def\childdoctmp
        {
          \childdoctrue
          \includeonly{\childdocname}
          \def\childdocjob{#1}
          \def\jobname{#1}
        }
      \fi
      \expandafter
    \endgroup
    \childdoctmp
  \fi
}
%    \end{macrocode}

% \macro{\childdocof}
% The command |\childdocof| redirects
% compilation to the main file |#1|.
%    \begin{macrocode}
\newcommand{\childdocof}[1]
{
  \childdocdisable
  \childdoctrue
  \includeonly{\childdocname}
  \def\jobname{#1}
  \def\childdocjob{#1}
  \input{#1}
}
%    \end{macrocode}

% \macro{\childdocby}
% The command |\childdocby| ....
%    \begin{macrocode}
\newcommand{\childdocby}[2][]
{
  \childdocdisable
  \childdoctrue
  \childdocmanualtrue
  \if?#1?\else
    \def\jobname{#2}
  \fi
  \def\childdocjob{#2}
  \input{#2}
  \endinput
}
%    \end{macrocode}

% \macro{\childdocforward}
% The command |\childdocforward| redirects
% compilation to the main file or
% (if the optional argument is given) a child file.
% Parameters are set as if the main file
% or a child file starting with |\childdocof| was compiled.
% Then compilation is handed over to the main file:
%    \begin{macrocode}
\newcommand{\childdocforward}[2][]
{
  \begingroup
    \if?#1?
      \def\childdoctmp
      {
        \def\childdocname{#2}
        \def\childdocjob{#2}
        \def\jobname{#2}
        \input{#2}
        \endinput
      }
    \else
      \def\childdoctmp
      {
        \childdocdisable
        \def\childdocname{#2}
        \childdoctrue
        \includeonly{#2}
        \def\childdocjob{#1}
        \def\jobname{#1}
        \input{#1}
        \endinput
      }
    \fi
    \expandafter
  \endgroup
  \childdoctmp
}
%    \end{macrocode}

% \macro{\childdocforwardprefix}
% The command |\childdocforwardprefix| redirects
% compilation to the main or a child file by means of a pattern.
% The prefix |#1| in the current filename is replaced by |#2|
% and the suffix of the current filename is kept
% (it is assumed that the filename does not contain the substring `|~~~|'
% which is used as a delimiter).
% Compilation is handed over to the new file by |\childdocforward|:
%    \begin{macrocode}
\newcommand{\childdocforwardprefix}[3][]
{
  \begingroup
    \def\childdocextract #2##1~~~{\def\childdoctmp{\childdocforward[#1]{#3##1}}}
    \expandafter\childdocextract\childdocname~~~
    \expandafter
  \endgroup
  \childdoctmp
}
%    \end{macrocode}

% \macro{\childdoc}
% The deprecated macro |\childdoc| is a legacy version of |\childdocmain|:
%    \begin{macrocode}
\newcommand{\childdoc}{\childdocmain}
%    \end{macrocode}

% \macro{\childdocredirect}
% The deprecated macro |\childdocredirect| is a legacy version
% of |\childdocforward| and |\childdocforwardprefix|:
%    \begin{macrocode}
\newcommand{\childdocredirect}[2][]
{
  \begingroup
    \if?#1?
      \def\childdoctmp{\childdocforward{#2}}
    \else
      \def\childdoctmp{\childdocforwardprefix{#1}{#2}}
    \fi
    \expandafter
  \endgroup
  \childdoctmp
}
%    \end{macrocode}

%\iffalse
%</package>
%\fi
%
\endinput
\childdocforward{cdocsch2}"|
% \end{tabular}
% \end{center}
% Note that the trailing backslash on each first line
% merely continues the input to the second line
% (for convenient cut ant paste).
% Furthermore, the command |latex| can be replaced by any
% of its alternative versions such as |pdflatex|.
%
% %%%%%%%%%%%%%%%%%%%%%%%%%%%%%%%%%%%%%%%%%%%%%%%%%%%%%%%%%%%%%%%%%%%%%%%%%%%%%%
% %%%%%%%%%%%%%%%%%%%%%%%%%%%%%%%%%%%%%%%%%%%%%%%%%%%%%%%%%%%%%%%%%%%%%%%%%%%%%%
% \section{Implementation}
%\iffalse
%<*package>
%\fi
%
% This section describes the definitions file |childdoc.def|.

% The definitions cannot be loaded using |\usepackage| or |\RequirePackage|
% which has a mechanism to prevent loading a style file more than once.
% When loading the definitions by means of |\input|
% multiple instances have to be prevented manually:
%\iffalse
%This code needs to be before the `\ProvidesFile' directive
%which is defined at the beginning of this file.
%Therefore it is also placed there and commented out here.
%</package>
%<*discard>
%\fi
%    \begin{macrocode}
\ifdefined\childdocmain\endinput\fi
%    \end{macrocode}
%\iffalse
%</discard>
%<*package>
%\fi
%
% \macro{\ifchilddoc}
% \macro{\ifchilddocmanual}
% The conditional |\ifchilddoc| tells whether a
% child (true) or main (false) document is being compiled.
% The conditional |\ifchilddocmanual| tells whether
% the |\includeonly| mechanism is used (false) or
% the selection of child files must be performed manually (true).
% The definitions initialise to false:
%    \begin{macrocode}
\newif\ifchilddoc
\newif\ifchilddocmanual
%    \end{macrocode}

% \macro{\childdocname}
% \macro{\childdocjob}
% The macro |\childdocname| stores the name of the main document
% to be compiled. The macro |\childdocjob| stores the name of
% the document on which the \LaTeX{} compiler was originally invoked.
% The content of |\jobname| cannot be compared
% to filenames specified in the source due to different catcodes.
% The following code rescans |\jobname|, stores the result
% in |\childdocname| and saves a copy in |\childdocjob|:
%    \begin{macrocode}
\edef\childdocname{\scantokens\expandafter{\jobname\noexpand}}
\let\childdocjob\childdocname
%    \end{macrocode}

% \macro{\childdocdisable}
% The macro |\childdocdisable| prevents the main file
% from being processed more than once.
% At this stage, the main document command |\childdocmain|
% is assumed to be called once again where it should do nothing.
% Any subsequent call to it should prevent
% a secondary processing of the main document
% It overwrites the forwarding commands
% |\childdocof| and |\childdocforward|
% with empty macros to prevent further inclusions of the main document:
%    \begin{macrocode}
\newcommand{\childdocdisable}
{
  \renewcommand{\childdocmain}[1]{\renewcommand{\childdocmain}[1]{\endinput}}
  \renewcommand{\childdocof}[1]{}
  \renewcommand{\childdocby}[2][]{}
  \renewcommand{\childdocforward}[2][]{}
  \renewcommand{\childdocdisable}{}
}
%    \end{macrocode}

% \macro{\childdocmain}
% The macro |\childdocmain| is to be called at the top of the main file
% with nothing or the main filename (without extension) as argument.
% First, it breaks loops.
% If the argument is not empty and does not match |\childdocname|
% (which is set by the first inclusion of |childdoc.def|),
% |\ifchilddoc| is set to true, |\includeonly| is applied to the child file
% and |\jobname| is set to the main file
% (for proper handling of |.aux| files):
%    \begin{macrocode}
\newcommand{\childdocmain}[1]
{
  \childdocdisable\childdocmain{}
  \if?#1?\else
    \begingroup
      \def\childdoctmp{#1}
      \ifx\childdoctmp\childdocname
        \def\childdoctmp{}
      \else
        \def\childdoctmp
        {
          \childdoctrue
          \includeonly{\childdocname}
          \def\childdocjob{#1}
          \def\jobname{#1}
        }
      \fi
      \expandafter
    \endgroup
    \childdoctmp
  \fi
}
%    \end{macrocode}

% \macro{\childdocof}
% The command |\childdocof| redirects
% compilation to the main file |#1|.
%    \begin{macrocode}
\newcommand{\childdocof}[1]
{
  \childdocdisable
  \childdoctrue
  \includeonly{\childdocname}
  \def\jobname{#1}
  \def\childdocjob{#1}
  \input{#1}
}
%    \end{macrocode}

% \macro{\childdocby}
% The command |\childdocby| ....
%    \begin{macrocode}
\newcommand{\childdocby}[2][]
{
  \childdocdisable
  \childdoctrue
  \childdocmanualtrue
  \if?#1?\else
    \def\jobname{#2}
  \fi
  \def\childdocjob{#2}
  \input{#2}
  \endinput
}
%    \end{macrocode}

% \macro{\childdocforward}
% The command |\childdocforward| redirects
% compilation to the main file or
% (if the optional argument is given) a child file.
% Parameters are set as if the main file
% or a child file starting with |\childdocof| was compiled.
% Then compilation is handed over to the main file:
%    \begin{macrocode}
\newcommand{\childdocforward}[2][]
{
  \begingroup
    \if?#1?
      \def\childdoctmp
      {
        \def\childdocname{#2}
        \def\childdocjob{#2}
        \def\jobname{#2}
        \input{#2}
        \endinput
      }
    \else
      \def\childdoctmp
      {
        \childdocdisable
        \def\childdocname{#2}
        \childdoctrue
        \includeonly{#2}
        \def\childdocjob{#1}
        \def\jobname{#1}
        \input{#1}
        \endinput
      }
    \fi
    \expandafter
  \endgroup
  \childdoctmp
}
%    \end{macrocode}

% \macro{\childdocforwardprefix}
% The command |\childdocforwardprefix| redirects
% compilation to the main or a child file by means of a pattern.
% The prefix |#1| in the current filename is replaced by |#2|
% and the suffix of the current filename is kept
% (it is assumed that the filename does not contain the substring `|~~~|'
% which is used as a delimiter).
% Compilation is handed over to the new file by |\childdocforward|:
%    \begin{macrocode}
\newcommand{\childdocforwardprefix}[3][]
{
  \begingroup
    \def\childdocextract #2##1~~~{\def\childdoctmp{\childdocforward[#1]{#3##1}}}
    \expandafter\childdocextract\childdocname~~~
    \expandafter
  \endgroup
  \childdoctmp
}
%    \end{macrocode}

% \macro{\childdoc}
% The deprecated macro |\childdoc| is a legacy version of |\childdocmain|:
%    \begin{macrocode}
\newcommand{\childdoc}{\childdocmain}
%    \end{macrocode}

% \macro{\childdocredirect}
% The deprecated macro |\childdocredirect| is a legacy version
% of |\childdocforward| and |\childdocforwardprefix|:
%    \begin{macrocode}
\newcommand{\childdocredirect}[2][]
{
  \begingroup
    \if?#1?
      \def\childdoctmp{\childdocforward{#2}}
    \else
      \def\childdoctmp{\childdocforwardprefix{#1}{#2}}
    \fi
    \expandafter
  \endgroup
  \childdoctmp
}
%    \end{macrocode}

%\iffalse
%</package>
%\fi
%
\endinput
|\\
|\childdocmain{|\textit{main}|}|\\
\end{tabular}
\end{center}
%
If |\jobname| does not match the argument \textit{main} of |\childdocmain|,
it is assumed that |\jobname| points to the child file to be compiled.
When using |\childdocmain| with the main file specified as argument,
it suffices to start a child file
with just |\input{|\textit{main}|}|
without loading of the package and using |\childdocof|.
If instead all processing is done
with the appropriate \textsf{childdoc} directives,
the argument of \textit{main} of |\childdocmain| can be empty.

An alternative version of the command line processing described
in \secref{sec:commandline} using the detection mechanism reads:
%
\begin{center}
|... -jobname "|\textit{target}|" "|[\textit{flags}]%
[|\def\jobname{|\textit{dest}|}|]|\input{|\textit{main}|}"|
\end{center}

%%%%%%%%%%%%%%%%%%%%%%%%%%%%%%%%%%%%%%%%%%%%%%%%%%%%%%%%%%%%%%%%%%%%%%%%%%%%%%%%
\subsection{Manual Code}
\label{sec:manual}

In case one cannot be certain whether the definitions file |childdoc.def|
is installed on the target \TeX{} distribution
and one prefers not to ship it,
it is conceivable to paste a few relevant commands into the sources.

To that end, drop all statements |% \iffalse
%
% childdoc.dtx Copyright (C) 2017-2018 Niklas Beisert
%
% This work may be distributed and/or modified under the
% conditions of the LaTeX Project Public License, either version 1.3
% of this license or (at your option) any later version.
% The latest version of this license is in
%   http://www.latex-project.org/lppl.txt
% and version 1.3 or later is part of all distributions of LaTeX
% version 2005/12/01 or later.
%
% This work has the LPPL maintenance status `maintained'.
%
% The Current Maintainer of this work is Niklas Beisert.
%
% This work consists of the files childdoc.dtx and childdoc.ins
% and the derived files childdoc.def and cdocsamp.tex with
% cdocsch1.tex, cdocsch2.tex, cdocsdrf.tex, cdocsfn1.tex, cdocsfn2.tex.
%
%<package>\ifdefined\childdocmain\endinput\fi
%<package>\ProvidesFile{childdoc.def}[2018/12/30 v2.0 child document driver]
%<samplemain>\ProvidesFile{cdocsamp.tex}[2018/12/30 v2.0 sample for childdoc]
%<*driver>
%\ProvidesFile{childdoc.drv}[2018/12/30 v2.0 childdoc reference manual file]
\PassOptionsToClass{10pt,a4paper}{article}
\documentclass{ltxdoc}

\usepackage[margin=35mm]{geometry}
\usepackage{hyperref}
\usepackage{hyperxmp}
\usepackage[usenames]{color}

\hypersetup{colorlinks=true}
\hypersetup{pdfstartview=FitH}
\hypersetup{pdfpagemode=UseNone}
\hypersetup{pdfsource={}}
\hypersetup{pdflang={en-UK}}
\hypersetup{pdfcopyright={Copyright 2017-2018 Niklas Beisert.
  This work may be distributed and/or modified under the
  conditions of the LaTeX Project Public License, either version 1.3
  of this license or (at your option) any later version.}}
\hypersetup{pdflicenseurl={http://www.latex-project.org/lppl.txt}}
\hypersetup{pdfcontactaddress={ETH Zurich, ITP, HIT K,
  Wolfgang-Pauli-Strasse 27}}
\hypersetup{pdfcontactpostcode={8093}}
\hypersetup{pdfcontactcity={Zurich}}
\hypersetup{pdfcontactcountry={Switzerland}}
\hypersetup{pdfcontactemail={nbeisert@itp.phys.ethz.ch}}
\hypersetup{pdfcontacturl={http://people.phys.ethz.ch/\xmptilde nbeisert/}}

\newcommand{\secref}[1]{\hyperref[#1]{section \ref*{#1}}}

\parskip1ex
\parindent0pt
\let\olditemize\itemize
\def\itemize{\olditemize\parskip0pt}

\begin{document}

\title{The \textsf{childdoc} Package}
\hypersetup{pdftitle={The childdoc Package}}
\author{Niklas Beisert\\[2ex]
  Institut f\"ur Theoretische Physik\\
  Eidgen\"ossische Technische Hochschule Z\"urich\\
  Wolfgang-Pauli-Strasse 27, 8093 Z\"urich, Switzerland\\[1ex]
  \href{mailto:nbeisert@itp.phys.ethz.ch}
  {\texttt{nbeisert@itp.phys.ethz.ch}}}
\hypersetup{pdfauthor={Niklas Beisert}}
\hypersetup{pdfsubject={Manual for the LaTeX2e Package childdoc}}
\date{30 December 2018, \textsf{v2.0}}
\maketitle

\begin{abstract}\noindent
\textsf{childdoc} is a \LaTeXe{} package
that enables the direct compilation
of document sections included by |\include|
to individual files.
\end{abstract}

\begingroup
\parskip0ex
\tableofcontents
\endgroup

%%%%%%%%%%%%%%%%%%%%%%%%%%%%%%%%%%%%%%%%%%%%%%%%%%%%%%%%%%%%%%%%%%%%%%%%%%%%%%%%
%%%%%%%%%%%%%%%%%%%%%%%%%%%%%%%%%%%%%%%%%%%%%%%%%%%%%%%%%%%%%%%%%%%%%%%%%%%%%%%%
\section{Introduction}

\LaTeX{} provides a mechanism to structure a large document (such as a book)
into a main file and several child files (containing the chapters)
using the |\include| command.
This mechanism is beneficial for documents
which span hundreds of pages in order to
make the source file(s) more manageable.
Moreover, compilation can be restricted to
selected child files by means of the |\includeonly| command.
The latter feature can be used to reduce the compilation time while editing
(this was significantly more useful in the earlier days of \LaTeX{})
or to generate a smaller document which is easier to navigate.
Another application of |\includeonly| is to generate
documents consisting of selected parts of the complete document.

However, there are a few drawbacks of the plain |\include| mechanism:
\begin{itemize}
\item
The child files cannot be compiled on their own,
they can only be compiled via the main file.
A naive editing environment
(such as a text editor with an option
to have the current file processed by \LaTeX)
may require one to switch to the main file before compiling;
attempting to compile the child file produces errors.
\item
The main file must be modified (each time)
to adjust the |\includeonly| command
to the present needs. This easily leaves the main file in a messy state.
\item
The generated document will always carry the filename
of the main document. This is inconvenient if
several child files are to be compiled and
to be kept for distribution.
\end{itemize}

The present package provides a simple interface
to make child files individually compilable by \LaTeX{}.
Compiling a child file then has the same effect as compiling
the main file with an |\includeonly| command
to select the appropriate child.
Moreover the generated document will carry the name of the child
rather than the main file.
This resolves all three above issues.

This feature is meant to make the editing of books,
thesis documents and lecture notes somewhat more convenient.
However, the package can also be used efficiently for
composing a series of documents (such as exercise sheets)
which are typically distributed individually.
It then assists the author in generating the individual documents
(potentially in different versions)
as well as a document containing the collected series.
Another application is in developing style files
or other kinds of included material
where compilation of the style file could redirect
to a sample or test file.

%%%%%%%%%%%%%%%%%%%%%%%%%%%%%%%%%%%%%%%%%%%%%%%%%%%%%%%%%%%%%%%%%%%%%%%%%%%%%%%%
%%%%%%%%%%%%%%%%%%%%%%%%%%%%%%%%%%%%%%%%%%%%%%%%%%%%%%%%%%%%%%%%%%%%%%%%%%%%%%%%
\section{Usage}

First of all, the package \textsf{childdoc} is \emph{not} a standard
\LaTeXe{} |.sty| style file! Therefore it needs to be invoked in
a non-standard way.

%%%%%%%%%%%%%%%%%%%%%%%%%%%%%%%%%%%%%%%%%%%%%%%%%%%%%%%%%%%%%%%%%%%%%%%%%%%%%%%%
\subsection{Included Files}
\label{sec:include}

%%%%%%%%%%%%%%%%%%%%%%%%%%%%%%%%%%%%%%%%
\DescribeMacro{\childdocmain}
To use the package, add the commands
\begin{center}
\begin{tabular}{l}
|% \iffalse
%
% childdoc.dtx Copyright (C) 2017-2018 Niklas Beisert
%
% This work may be distributed and/or modified under the
% conditions of the LaTeX Project Public License, either version 1.3
% of this license or (at your option) any later version.
% The latest version of this license is in
%   http://www.latex-project.org/lppl.txt
% and version 1.3 or later is part of all distributions of LaTeX
% version 2005/12/01 or later.
%
% This work has the LPPL maintenance status `maintained'.
%
% The Current Maintainer of this work is Niklas Beisert.
%
% This work consists of the files childdoc.dtx and childdoc.ins
% and the derived files childdoc.def and cdocsamp.tex with
% cdocsch1.tex, cdocsch2.tex, cdocsdrf.tex, cdocsfn1.tex, cdocsfn2.tex.
%
%<package>\ifdefined\childdocmain\endinput\fi
%<package>\ProvidesFile{childdoc.def}[2018/12/30 v2.0 child document driver]
%<samplemain>\ProvidesFile{cdocsamp.tex}[2018/12/30 v2.0 sample for childdoc]
%<*driver>
%\ProvidesFile{childdoc.drv}[2018/12/30 v2.0 childdoc reference manual file]
\PassOptionsToClass{10pt,a4paper}{article}
\documentclass{ltxdoc}

\usepackage[margin=35mm]{geometry}
\usepackage{hyperref}
\usepackage{hyperxmp}
\usepackage[usenames]{color}

\hypersetup{colorlinks=true}
\hypersetup{pdfstartview=FitH}
\hypersetup{pdfpagemode=UseNone}
\hypersetup{pdfsource={}}
\hypersetup{pdflang={en-UK}}
\hypersetup{pdfcopyright={Copyright 2017-2018 Niklas Beisert.
  This work may be distributed and/or modified under the
  conditions of the LaTeX Project Public License, either version 1.3
  of this license or (at your option) any later version.}}
\hypersetup{pdflicenseurl={http://www.latex-project.org/lppl.txt}}
\hypersetup{pdfcontactaddress={ETH Zurich, ITP, HIT K,
  Wolfgang-Pauli-Strasse 27}}
\hypersetup{pdfcontactpostcode={8093}}
\hypersetup{pdfcontactcity={Zurich}}
\hypersetup{pdfcontactcountry={Switzerland}}
\hypersetup{pdfcontactemail={nbeisert@itp.phys.ethz.ch}}
\hypersetup{pdfcontacturl={http://people.phys.ethz.ch/\xmptilde nbeisert/}}

\newcommand{\secref}[1]{\hyperref[#1]{section \ref*{#1}}}

\parskip1ex
\parindent0pt
\let\olditemize\itemize
\def\itemize{\olditemize\parskip0pt}

\begin{document}

\title{The \textsf{childdoc} Package}
\hypersetup{pdftitle={The childdoc Package}}
\author{Niklas Beisert\\[2ex]
  Institut f\"ur Theoretische Physik\\
  Eidgen\"ossische Technische Hochschule Z\"urich\\
  Wolfgang-Pauli-Strasse 27, 8093 Z\"urich, Switzerland\\[1ex]
  \href{mailto:nbeisert@itp.phys.ethz.ch}
  {\texttt{nbeisert@itp.phys.ethz.ch}}}
\hypersetup{pdfauthor={Niklas Beisert}}
\hypersetup{pdfsubject={Manual for the LaTeX2e Package childdoc}}
\date{30 December 2018, \textsf{v2.0}}
\maketitle

\begin{abstract}\noindent
\textsf{childdoc} is a \LaTeXe{} package
that enables the direct compilation
of document sections included by |\include|
to individual files.
\end{abstract}

\begingroup
\parskip0ex
\tableofcontents
\endgroup

%%%%%%%%%%%%%%%%%%%%%%%%%%%%%%%%%%%%%%%%%%%%%%%%%%%%%%%%%%%%%%%%%%%%%%%%%%%%%%%%
%%%%%%%%%%%%%%%%%%%%%%%%%%%%%%%%%%%%%%%%%%%%%%%%%%%%%%%%%%%%%%%%%%%%%%%%%%%%%%%%
\section{Introduction}

\LaTeX{} provides a mechanism to structure a large document (such as a book)
into a main file and several child files (containing the chapters)
using the |\include| command.
This mechanism is beneficial for documents
which span hundreds of pages in order to
make the source file(s) more manageable.
Moreover, compilation can be restricted to
selected child files by means of the |\includeonly| command.
The latter feature can be used to reduce the compilation time while editing
(this was significantly more useful in the earlier days of \LaTeX{})
or to generate a smaller document which is easier to navigate.
Another application of |\includeonly| is to generate
documents consisting of selected parts of the complete document.

However, there are a few drawbacks of the plain |\include| mechanism:
\begin{itemize}
\item
The child files cannot be compiled on their own,
they can only be compiled via the main file.
A naive editing environment
(such as a text editor with an option
to have the current file processed by \LaTeX)
may require one to switch to the main file before compiling;
attempting to compile the child file produces errors.
\item
The main file must be modified (each time)
to adjust the |\includeonly| command
to the present needs. This easily leaves the main file in a messy state.
\item
The generated document will always carry the filename
of the main document. This is inconvenient if
several child files are to be compiled and
to be kept for distribution.
\end{itemize}

The present package provides a simple interface
to make child files individually compilable by \LaTeX{}.
Compiling a child file then has the same effect as compiling
the main file with an |\includeonly| command
to select the appropriate child.
Moreover the generated document will carry the name of the child
rather than the main file.
This resolves all three above issues.

This feature is meant to make the editing of books,
thesis documents and lecture notes somewhat more convenient.
However, the package can also be used efficiently for
composing a series of documents (such as exercise sheets)
which are typically distributed individually.
It then assists the author in generating the individual documents
(potentially in different versions)
as well as a document containing the collected series.
Another application is in developing style files
or other kinds of included material
where compilation of the style file could redirect
to a sample or test file.

%%%%%%%%%%%%%%%%%%%%%%%%%%%%%%%%%%%%%%%%%%%%%%%%%%%%%%%%%%%%%%%%%%%%%%%%%%%%%%%%
%%%%%%%%%%%%%%%%%%%%%%%%%%%%%%%%%%%%%%%%%%%%%%%%%%%%%%%%%%%%%%%%%%%%%%%%%%%%%%%%
\section{Usage}

First of all, the package \textsf{childdoc} is \emph{not} a standard
\LaTeXe{} |.sty| style file! Therefore it needs to be invoked in
a non-standard way.

%%%%%%%%%%%%%%%%%%%%%%%%%%%%%%%%%%%%%%%%%%%%%%%%%%%%%%%%%%%%%%%%%%%%%%%%%%%%%%%%
\subsection{Included Files}
\label{sec:include}

%%%%%%%%%%%%%%%%%%%%%%%%%%%%%%%%%%%%%%%%
\DescribeMacro{\childdocmain}
To use the package, add the commands
\begin{center}
\begin{tabular}{l}
|\input{childdoc.def}|\\
|\childdocmain{}|\\
\end{tabular}
\end{center}
at the very top of the main \LaTeX{} file,
in particular \emph{before} the |\documentclass| statement!
The argument of |\childdocmain| should be left empty
(but it must be present).

%%%%%%%%%%%%%%%%%%%%%%%%%%%%%%%%%%%%%%%%
\DescribeMacro{\childdocof}
Furthermore, add the commands
\begin{center}
\begin{tabular}{l}
|\input{childdoc.def}|\\
|\childdocof{|\textit{main}|}|\\
\end{tabular}
\end{center}
at the top of every child file \textit{child}
which is included by |\include{|\textit{child}|}|
from within the main file
(or at least for those files to be compiled individually).
The argument \textit{main} must be the filename of the main file.

There are a couple of
considerations in setting up the main and child documents:

%%%%%%%%%%%%%%%%%%%%%%%%%%%%%%%%%%%%%%%%
\paragraph{Restrictions.}

Please note the following restrictions:
\begin{itemize}
\item
|\childdocmain| must be called with one argument \textit{main}
to ensure compatibility with earlier version of the package.
It must either be empty (|\childdocmain{}|)
or precisely match the filename of the main file in which it is specified.
See \secref{sec:detection} for further information.
\item
The filename \textit{main} must be specified without the |.tex| extension.
\item
The filename \textit{main} is case sensitive
(even in case-insensitive file systems)
due to internal string comparison.
\item
The argument \textit{main} should be fully expanded, it cannot be a macro.
\item
Subdirectories and special characters should be avoided in filenames.
\item
The command |\childdocmain{|\textit{main}|}| must be followed by a whitespace.
It should not be followed immediately by another command
or by a comment mark `|%|'.
This is because the \TeX{} parser reads the token immediately following
the argument of |\childdocmain| and puts it
at the beginning of every child section;
however, a white\-space is ignored.
\end{itemize}

%%%%%%%%%%%%%%%%%%%%%%%%%%%%%%%%%%%%%%%%
\paragraph{Content of Main File.}

It is advisable to place all content in the child files included by |\include|.
Any output contained in the main file will appear in all child documents
unless suppressed manually;
it cannot be suppressed automatically by the |\includeonly| directive
and thus should normally be avoided.
A method to include some content in the main file
by means of conditional processing is described in \secref{sec:conditional}.

%%%%%%%%%%%%%%%%%%%%%%%%%%%%%%%%%%%%%%%%
\paragraph{Page Numbering.}

When only a part of the document is compiled,
the appropriate numbering of pages
(as well as other status parameters)
is determined from the |.aux| files.
The latter contain information from previous passes.
However this information needs to propagate through
all intermediate child documents.
Therefore the page numbering in child documents may well
be inconsistent until the complete document is compiled at least once.

A useful (if unconventional) way to always ensure a consistent
page numbering is to restart the numbering in each child document
and denote the pages by `\textit{child}|.|\textit{page}'
where \textit{child} represents the chapter/section number of the child file.
This can be achieved by the command
|\numberwithin{page}{|\textit{child}|}|
of the \textsf{amsmath} package
where \textit{child} can be |chapter| or |section|
depending on the chosen structuring.
Alternatively, one can modify the macro |\thepage| appropriately
and reset the counter |page| at the start of each child file.

%%%%%%%%%%%%%%%%%%%%%%%%%%%%%%%%%%%%%%%%%%%%%%%%%%%%%%%%%%%%%%%%%%%%%%%%%%%%%%%%
\subsection{Conditional Processing}
\label{sec:conditional}

The package provides a mechanism to compile different versions
of a document. To customise the versions further some conditional processing
can come in handy to distinguish which version is being compiled.
The package provides two macros to describe the compilation context:

%%%%%%%%%%%%%%%%%%%%%%%%%%%%%%%%%%%%%%%%
\DescribeMacro{\ifchilddoc}
The conditional |\ifchilddoc| distinguishes between the compilation of
child documents and the main document:
%
\begin{center}
|\ifchilddoc |\textit{child-code}| |[|\||else |\textit{main-code}]| \||fi|
\end{center}

%%%%%%%%%%%%%%%%%%%%%%%%%%%%%%%%%%%%%%%%
\DescribeMacro{\childdocname}
\DescribeMacro{\childdocjob}
The macro |\childdocname| contains the filename (without extension)
of the main or child file being processed.
Note that |\childdocjob| will always contain the name of the main file.

%%%%%%%%%%%%%%%%%%%%%%%%%%%%%%%%%%%%%%%%
\paragraph{Title Page.}

Conditional processing can be used to include a title or banner page
in the main document when proper precautions are taken.
Importantly, the code in the main file should ensure that the page counter
(as well as other status parameters which are stored in the |.aux| files)
takes the same value after the conditional processing.
Otherwise the page numbers may take divergent values
depending on which part is compiled.

For example, a title page could be declared by:
%
\begin{center}
\begin{tabular}{l}
|\ifchilddoc\||else|\\
|\addtocounter{page}{-1}|\\
\textit{code for title page}\\
|\newpage|\\
|\||fi|
\end{tabular}
\end{center}
%
A banner page for the child documents can be generated by:
%
\begin{center}
\begin{tabular}{l}
|\ifchilddoc|\\
|\addtocounter{page}{-1}|\\
\textit{code for banner page}\\
|\newpage|\\
|\||fi|
\end{tabular}
\end{center}
%
Here one could write a message such as:
\begin{center}
|This is the part \childdocname{} of \childdocjob{}.|
\end{center}

%%%%%%%%%%%%%%%%%%%%%%%%%%%%%%%%%%%%%%%%%%%%%%%%%%%%%%%%%%%%%%%%%%%%%%%%%%%%%%%%
\subsection{Flags}
\label{sec:flags}

The package makes it easy to generate different versions
of the main or child documents.
To this end compilation flags can be defined
and assigned different default values.
They will be particularly useful in conjunction
with the forwarding mechanism described in \secref{sec:forward}.

For example, it may be useful to have a flag |\version|
which can be set to |draft| or |final|.
The document source will contain some conditional code
depending on the value of |\version|.
Suppose further, the flag should default to |final| for the main file
and to |draft| for child files
which is a natural assignment for editing the document.
This is achieved by placing the following code
in the preamble of the main document
(below the |\childdocmain| directive):
%
\begin{center}
\begin{tabular}{l}
|\ifchilddoc|\\
|\providecommand{\version}{draft}|\\
|\||else|\\
|\providecommand{\version}{final}|\\
|\||fi|
\end{tabular}
\end{center}
%
The definition by |\providecommand| makes sure
that previous definitions are not overwritten.
Further statements |\providecommand{\version}{...}|
can thus be added before the above code to override it.

For the main file, one might add a line
(between |\childdocmain| and the above block)
%
\begin{center}
|%\ifchilddoc\||else\providecommand{\version}{draft}\||fi|
\end{center}
%
which can be uncommented to produce a draft version.
Likewise one can add a line to the very top of a child file
(above the |\childdocof{|\textit{main}|}| directive)
%
\begin{center}
|%\providecommand{\version}{final}|
\end{center}
%
which can be uncommented to produce the final version of this child document.

%%%%%%%%%%%%%%%%%%%%%%%%%%%%%%%%%%%%%%%%%%%%%%%%%%%%%%%%%%%%%%%%%%%%%%%%%%%%%%%%
\subsection{Forwarding}
\label{sec:forward}

Different versions of the main or child documents
using compilation flags as described in \secref{sec:flags}
can be (permanently) stored in different files
for convenient compilation, viewing and distribution.
To this end, the package defines a command
to pass on compilation to a different file:

%%%%%%%%%%%%%%%%%%%%%%%%%%%%%%%%%%%%%%%%
\DescribeMacro{\childdocforward}
The command |\childdocforward| redirects processing to
another source file:
%
\begin{center}
\begin{tabular}{l}
|\input{childdoc.def}|\\
|\childdocforward[|\textit{main}|]{|\textit{dest}|}|\\
\end{tabular}
\end{center}
%
The argument \textit{dest} is the destination file
(without extension).
It should be the main file or one of the child files.
Note that further \textsf{childdoc} directives
such as |\childdocof| and |\childdocforward|
in the indicated file will be processed in this form.
The optional argument \textit{main}
passes on directly to the main file \textit{main}
while pretending to compile the child \textit{dest}.
This form behaves as if \textit{dest}
issues |\childdocof{|\textit{main}|}| right away,
and no further \textsf{childdoc} directives will be processed.

%%%%%%%%%%%%%%%%%%%%%%%%%%%%%%%%%%%%%%%%
\DescribeMacro{\...prefix}
In the alternative form |\childdocforwardprefix|,
%
\begin{center}
\begin{tabular}{l}
|\input{childdoc.def}|\\
|\childdocforwardprefix[|\textit{main}|]{|\textit{prefix}|}{|\textit{dest}|}|
\end{tabular}
\end{center}
%
the destination file is determined by a pattern
depending on the current file:
To make this work, the current file must be called
`{\textit{prefix}\hspace{0.2em}\textit{suffix}}'
with \textit{prefix} matching precisely the argument.
Processing is then passed on to the file
`{\textit{dest}\hspace{0.2em}\textit{suffix}}'.
Surely, the same effect is achieved by
directly specifying the
argument `{\textit{dest}\hspace{0.2em}\textit{suffix}}'
in the first form.
However, that requires to set up a different file
for each child. With the alternative form of the command
all these files can have exactly the same content
which simplifies setting them up and maintaining them.

For example, the following file |draft.tex|
with a compilation flag |\version| as described in \secref{sec:flags}
compiles the main document as a draft:
%
\begin{center}
\begin{tabular}{l}
|\def\version{draft}|\\
|\input{childdoc.def}|\\
|\childdocforward{|\textit{main}|}|
\end{tabular}
\end{center}
%
Likewise, the following files |final|\textit{nn}|.tex|
compile the final version of the child document
|child|\textit{nn}|.tex|:
%
\begin{center}
\begin{tabular}{l}
|\def\version{final}|\\
|\input{childdoc.def}|\\
|\childdocforwardprefix{final}{child}|
\end{tabular}
\end{center}
%

Note that when several versions of a main file and/or of each child file
are to be generated, it may be convenient to set up a |Makefile| or
shell script to automatise the process.

%%%%%%%%%%%%%%%%%%%%%%%%%%%%%%%%%%%%%%%%%%%%%%%%%%%%%%%%%%%%%%%%%%%%%%%%%%%%%%%%
\subsection{Command Line Processing}
\label{sec:commandline}

The effect of redirection files can also be achieved by invoking
the \LaTeX{} compiler with a more elaborate command line.
Most conveniently this should be done as part
of a shell script or a |Makefile|.

When using \textsf{childdoc} in the main file, the following
command lines effectively perform a redirection
(note that depending on the shell being used,
backslashes may have to be doubled: `|\|' $\to$ `|\\|'):
%
\begin{center}
|... -jobname "|\textit{target}|" |\\|"|[\textit{flags}]%
|\input{childdoc.def}\childdocforward[|\textit{main}|]{|\textit{dest}|}"|
\end{center}
%
Here \textit{target} is the name of the output file,
\textit{main} is the name of the main file
and \textit{dest} is the name of the main or child file to be processed
(all filenames without extensions).
The optional argument \textit{main} can be omitted
if \textit{main} matches \textit{dest}.
Optionally, compilation \textit{flags} can be defined via |\def| commands.
This command line makes the \TeX{} engine believe
it is compiling the file \textit{target}
whose content is specified as the latter parameter.
The provided code then forwards the processing to
\textit{main} or \textit{dest} as described in \secref{sec:forward}.

%%%%%%%%%%%%%%%%%%%%%%%%%%%%%%%%%%%%%%%%%%%%%%%%%%%%%%%%%%%%%%%%%%%%%%%%%%%%%%%%
\subsection{Include by Input}
\label{sec:input}

Including child documents by |\include| has some restrictions by design.
Most notably, the content of a child document always occupies
its own set of pages; pages cannot be shared between child documents.
Usually, this behaviour makes perfect sense
because each child document contain an essential part of the document.
However, in some situations it may be desirable to compose
a document from a collection of parts
without having mandatory page breaks between then.
For this case, the package
provides a mechanism to include parts
by |\input| which can also be processed individually.
However, by construction this mechanism
requires manual handling of the content to be output.

%%%%%%%%%%%%%%%%%%%%%%%%%%%%%%%%%%%%%%%%
\DescribeMacro{\ifchilddocmanual}
The main file should be prepared as usual, see \secref{sec:include}.
However, the document body must make a distinction
between processing of an individual part and of the main document, e.g.:
%
\begin{center}
\begin{tabular}{l}
|\ifchilddocmanual|\\
|\input{\childdocname}|\\
|\||else|\\
\textit{document body with }|\input{|\textit{part}|}|\\
|\||fi|
\end{tabular}
\end{center}
%
The conditional |\ifchilddocmanual| is true whenever
a part to be included by |\input| is being compiled,
and the name of the part is stored in |\childdocname|.

%%%%%%%%%%%%%%%%%%%%%%%%%%%%%%%%%%%%%%%%
\DescribeMacro{\childdocby}
Each part to be included by |\input| should start with:
%
\begin{center}
\begin{tabular}{l}
|\input{childdoc.def}|\\
|\childdocby{|\textit{main}|}|\\
\end{tabular}
\end{center}
%
The directive |\childdocby| is similar to |\childdocof|
described in \secref{sec:include},
but the subsequent selection of content must be done manually.
To that end, both |\ifchilddoc| and |\ifchilddocmanual|
will be true upon processing of a part,
and the name of the part is stored in |\childdocname|.
Note that |\jobname| will be set to the filename of the current part
so that each part receives an individual |.aux| file
that does not interfere with the |.aux| file(s) of the main document.
This behaviour can be altered by the alternative form
|\childdocby[*]{|\textit{main}|}| (with a non-empty optional argument)
which uses the |.aux| file of the main document
by setting |\jobname| to \textit{main}.

%%%%%%%%%%%%%%%%%%%%%%%%%%%%%%%%%%%%%%%%%%%%%%%%%%%%%%%%%%%%%%%%%%%%%%%%%%%%%%%%
\subsection{Driver Development}
\label{sec:driver}

The \textsf{childdoc} mechanism can also be use for the development
of definition files such as \LaTeX{} styles or classes.
This case differs from the above setup with multiple parts
included by |\include| in that no |\includeonly| should be invoked.
This can be achieved by starting the include file
(before |\ProvidesPackage|) with:
%
\begin{center}
\begin{tabular}{l}
|\input{childdoc.def}|\\
|\childdocforward{|\textit{main}|}|\\
\end{tabular}
\end{center}
%
or alternatively with:
%
\begin{center}
\begin{tabular}{l}
|\input{childdoc.def}|\\
|\childdocby{|\textit{main}|}|\\
\end{tabular}
\end{center}
%
Both forms have slightly different effects as described above.
The main file is prepared as usual, see \secref{sec:include}.

%%%%%%%%%%%%%%%%%%%%%%%%%%%%%%%%%%%%%%%%%%%%%%%%%%%%%%%%%%%%%%%%%%%%%%%%%%%%%%%%
\subsection{Legacy Detection}
\label{sec:detection}

The directive |\childdocmain| in the main file can detect
whether the complete document or merely a child is to be compiled
even without using the directive |\childdocof|.
This method is deprecated because it is less robust
and there is no compelling reason to use it;
it is merely provided for backward compatibility
and it may be removed in future versions.

If the detection mechanism is to be used,
it is mandatory to correctly specify
the filename of the main file as the argument of |\childdocmain|:
%
\begin{center}
\begin{tabular}{l}
|\input{childdoc.def}|\\
|\childdocmain{|\textit{main}|}|\\
\end{tabular}
\end{center}
%
If |\jobname| does not match the argument \textit{main} of |\childdocmain|,
it is assumed that |\jobname| points to the child file to be compiled.
When using |\childdocmain| with the main file specified as argument,
it suffices to start a child file
with just |\input{|\textit{main}|}|
without loading of the package and using |\childdocof|.
If instead all processing is done
with the appropriate \textsf{childdoc} directives,
the argument of \textit{main} of |\childdocmain| can be empty.

An alternative version of the command line processing described
in \secref{sec:commandline} using the detection mechanism reads:
%
\begin{center}
|... -jobname "|\textit{target}|" "|[\textit{flags}]%
[|\def\jobname{|\textit{dest}|}|]|\input{|\textit{main}|}"|
\end{center}

%%%%%%%%%%%%%%%%%%%%%%%%%%%%%%%%%%%%%%%%%%%%%%%%%%%%%%%%%%%%%%%%%%%%%%%%%%%%%%%%
\subsection{Manual Code}
\label{sec:manual}

In case one cannot be certain whether the definitions file |childdoc.def|
is installed on the target \TeX{} distribution
and one prefers not to ship it,
it is conceivable to paste a few relevant commands into the sources.

To that end, drop all statements |\input{childdoc.def}|
and perform the replacements as outlined below.
Instead of |\childdocmain{|\textit{main}|}| add the following code
to the top of the main file:
%
\begin{center}
\begin{tabular}{l}
|\||ifdefined\childdocname\endinput\||fi\newif\ifchilddoc|\\
|\edef\childdocname{\scantokens\expandafter{\jobname\noexpand}}|\\
|\def\childdocmain{|\textit{main}|}\||ifx\childdocmain\childdocname\||else|\\
|\childdoctrue\includeonly{\childdocname}\let\jobname\childdocmain\||fi|\\
\end{tabular}
\end{center}
%
Instead of |\childdocof{|\textit{main}|}| just include the main file
at the top of each child file:
%
\begin{center}
|\input{|\textit{main}|}|
\end{center}
%
A simple redirection |\childdocforward{|\textit{dest}|}| is achieved by:
%
\begin{center}
|\def\jobname{|\textit{dest}|}\input{\jobname}|
\end{center}
%
The redirection with prefix
|\childdocforwardprefix[|\textit{prefix}|]{|\textit{dest}|}|
is accomplished by:
%
\begin{center}
\begin{tabular}{l}
|{\edef\jobname{\scantokens\expandafter{\jobname\noexpand}}|\\
|\def\redirectjob |\textit{prefix}|#1~~~{\gdef\jobname{|\textit{dest}|#1}}|\\
|\expandafter\redirectjob\jobname~~~}\input{\jobname}|
\end{tabular}
\end{center}

In an alternative approach,
child documents can be compiled by a specific command line
without additional code or specific definitions:
%
\begin{center}
|... -jobname "|\textit{target}|" "|[\textit{flags}]%
|\includeonly{|\textit{dest}|}\input{|\textit{main}|}"|
\end{center}
%

%%%%%%%%%%%%%%%%%%%%%%%%%%%%%%%%%%%%%%%%%%%%%%%%%%%%%%%%%%%%%%%%%%%%%%%%%%%%%%%%
%%%%%%%%%%%%%%%%%%%%%%%%%%%%%%%%%%%%%%%%%%%%%%%%%%%%%%%%%%%%%%%%%%%%%%%%%%%%%%%%
\section{Information}

%%%%%%%%%%%%%%%%%%%%%%%%%%%%%%%%%%%%%%%%%%%%%%%%%%%%%%%%%%%%%%%%%%%%%%%%%%%%%%%%
\subsection{Copyright}

Copyright \copyright{} 2017--2018 Niklas Beisert

This work may be distributed and/or modified under the
conditions of the \LaTeX{} Project Public License, either version 1.3
of this license or (at your option) any later version.
The latest version of this license is in
  \url{http://www.latex-project.org/lppl.txt}
and version 1.3 or later is part of all distributions of \LaTeX{}
version 2005/12/01 or later.

This work has the LPPL maintenance status `maintained'.

The Current Maintainer of this work is Niklas Beisert.

This work consists of the files |README.txt|, |childdoc.ins| and |childdoc.dtx|
as well as the derived files |childdoc.def|, |cdocsamp.tex|
with |cdocsch1.tex|, |cdocsch2.tex|, |cdocspt3.tex|, |cdocspt4.tex|,
|cdocsdrf.tex|, |cdocsfn1.tex|, |cdocsfn2.tex|
as well as |childdoc.pdf|.

%%%%%%%%%%%%%%%%%%%%%%%%%%%%%%%%%%%%%%%%%%%%%%%%%%%%%%%%%%%%%%%%%%%%%%%%%%%%%%%%
\subsection{Files and Installation}

The package consists of the files:
%
\begin{center}
\begin{tabular}{ll}
    |README.txt|   & readme file \\
    |childdoc.ins| & installation file \\
    |childdoc.dtx| & source file \\
    |childdoc.def| & definition file \\
    |cdocsamp.tex| & sample main file \\
    |cdocsch1.tex| & sample include file \\
    |cdocsch2.tex| & sample include file \\
    |cdocspt3.tex| & sample part file \\
    |cdocspt4.tex| & sample part file \\
    |cdocsdrf.tex| & sample redirection file \\
    |cdocsfn1.tex| & sample redirection file \\
    |cdocsfn2.tex| & sample redirection file \\
    |childdoc.pdf| & manual
\end{tabular}
\end{center}
%
The distribution consists of the files
|README.txt|, |childdoc.ins| and |childdoc.dtx|.
%
\begin{itemize}
\item
Run (pdf)\LaTeX{} on |childdoc.dtx|
to compile the manual |childdoc.pdf| (this file).
\item
Run \LaTeX{} on |childdoc.ins| to create the definitions file |childdoc.def|
and the sample |cdocsamp.tex| with include files
|cdocsch1.tex|, |cdocsch2.tex|, |cdocspt3.tex|, |cdocspt4.tex|,
|cdocsdrf.tex|, |cdocsfn1.tex|, |cdocsfn2.tex|.
Then copy the file |childdoc.def| to an appropriate directory of your \LaTeX{}
distribution, e.g.\ \textit{texmf-root}|/tex/latex/childdoc|.
\end{itemize}

%%%%%%%%%%%%%%%%%%%%%%%%%%%%%%%%%%%%%%%%%%%%%%%%%%%%%%%%%%%%%%%%%%%%%%%%%%%%%%%%
\subsection{Related CTAN Packages}

There are several other packages which offer a similar functionality:
%
\begin{itemize}
\item
The packages
\href{http://ctan.org/pkg/docmute}{\textsf{docmute}},
\href{http://ctan.org/pkg/includex}{\textsf{includex}} and
\href{http://ctan.org/pkg/standalone}{\textsf{standalone}}
provide commands to include only the document body of
a child file thus allowing both files to be compiled individually.
\item
The packages \href{http://ctan.org/pkg/subdocs}{\textsf{subdocs}}
and \href{http://ctan.org/pkg/subfiles}{\textsf{subfiles}}
provide structures in which the main and child documents can be
encapsulated and allowing them to be compiled individually.
The inclusion mechanism is different from the conventional |\include|.
\item
The package \href{http://ctan.org/pkg/combine}{\textsf{combine}}
is an elaborate solution to combine several documents into one.
\end{itemize}
%
See also the CTAN topic \href{http://ctan.org/topic/subdocs}{\textsf{subdocs}}
for further related packages.
The present package differs from the above solutions in that
a document structure constructed with the conventional |\include| mechanism
just needs two extra commands at the top of every file
such that all constituent files can be compiled individually.

%%%%%%%%%%%%%%%%%%%%%%%%%%%%%%%%%%%%%%%%%%%%%%%%%%%%%%%%%%%%%%%%%%%%%%%%%%%%%%%%
%\subsection{Feature Suggestions}
%
%The following is a list of features which may be useful for future
%versions of this package:
%%
%\begin{itemize}
%\item
%\ldots
%\end{itemize}

%%%%%%%%%%%%%%%%%%%%%%%%%%%%%%%%%%%%%%%%%%%%%%%%%%%%%%%%%%%%%%%%%%%%%%%%%%%%%%%%
\subsection{Revision History}

%%%%%%%%%%%%%%%%%%%%%%%%%%%%%%%%%%%%%%%%
\paragraph{v2.0:} 2018/12/30

\begin{itemize}
\item
immediate forward processing
\item
added |\childdocby| mechanism
\item
manual restructured
\end{itemize}

%%%%%%%%%%%%%%%%%%%%%%%%%%%%%%%%%%%%%%%%
\paragraph{v1.6:} 2018/01/17

\begin{itemize}
\item
application for development of include files
\item
corrections to manual
\end{itemize}

%%%%%%%%%%%%%%%%%%%%%%%%%%%%%%%%%%%%%%%%
\paragraph{v1.5:} 2017/05/21

\begin{itemize}
\item
more complete structuring introduced
\item
|\childdocof| introduced
\item
|\childdoc| renamed to |\childdocmain|
\item
|\childredirect| renamed to |\childdocforward| and |\childdocforwardprefix|
and functionality expanded
\end{itemize}

%%%%%%%%%%%%%%%%%%%%%%%%%%%%%%%%%%%%%%%%
\paragraph{v1.0:} 2017/04/27

\begin{itemize}
\item
manual and install package
\item
first version published on CTAN
\end{itemize}

%%%%%%%%%%%%%%%%%%%%%%%%%%%%%%%%%%%%%%%%
\paragraph{v0.6:} 2017/04/26

\begin{itemize}
\item
redirection mechanism added
\end{itemize}

%%%%%%%%%%%%%%%%%%%%%%%%%%%%%%%%%%%%%%%%
\paragraph{v0.5:} 2017/04/26

\begin{itemize}
\item
functionality in definition file
\end{itemize}


%%%%%%%%%%%%%%%%%%%%%%%%%%%%%%%%%%%%%%%%%%%%%%%%%%%%%%%%%%%%%%%%%%%%%%%%%%%%%%%%
%%%%%%%%%%%%%%%%%%%%%%%%%%%%%%%%%%%%%%%%%%%%%%%%%%%%%%%%%%%%%%%%%%%%%%%%%%%%%%%%
%%%%%%%%%%%%%%%%%%%%%%%%%%%%%%%%%%%%%%%%%%%%%%%%%%%%%%%%%%%%%%%%%%%%%%%%%%%%%%%%
\appendix

\settowidth\MacroIndent{\rmfamily\scriptsize 000\ }

 \DocInput{childdoc.dtx}

\end{document}
%</driver>
% \fi
%
% %%%%%%%%%%%%%%%%%%%%%%%%%%%%%%%%%%%%%%%%%%%%%%%%%%%%%%%%%%%%%%%%%%%%%%%%%%%%%%
% %%%%%%%%%%%%%%%%%%%%%%%%%%%%%%%%%%%%%%%%%%%%%%%%%%%%%%%%%%%%%%%%%%%%%%%%%%%%%%
% \section{Sample}
%\iffalse
%<*samplemain>
%\fi
%
% The following presents a sample document
% with two chapters, two parts, a title page,
% a compile flag as well as three forwarding files to set the flag.
% It consists of eight |.tex| files:
% \begin{center}
% \begin{tabular}{ll}
% |cdocsamp.tex|&main file\\
% |cdocsch1.tex|&include file for chapter 1\\
% |cdocsch2.tex|&include file for chapter 2\\
% |cdocspt3.tex|&include file for part 3\\
% |cdocspt4.tex|&include file for part 4\\
% |cdocsdrf.tex|&forwarding file for main file in draft mode\\
% |cdocsfi1.tex|&forwarding file for final version of chapter 1\\
% |cdocsfi2.tex|&forwarding file for final version of chapter 2\\
% \end{tabular}
% \end{center}
% Each of the eight files can be compiled directly by the \LaTeX{} compiler.
%
% %%%%%%%%%%%%%%%%%%%%%%%%%%%%%%%%%%%%%%
% \paragraph{Main File.}
%
% The main file is called |cdocsamp.tex|.
%
% Load the \textsf{childdoc} definitions and
% declare the filename for the main document:
%    \begin{macrocode}
\input{childdoc.def}
\childdocmain{}
%    \end{macrocode}

% Optional override for |\version| flag:
%    \begin{macrocode}
%%\ifchilddoc\else\providecommand{\version}{draft}\fi
%    \end{macrocode}

% Define the default values for the |\version| flag
% (|final| for the main file and |draft| for childs):
%    \begin{macrocode}
\ifchilddoc
\providecommand{\version}{draft}
\else
\providecommand{\version}{final}
\fi
%    \end{macrocode}

% Load the standard document class:
%    \begin{macrocode}
\documentclass[12pt]{article}
%    \end{macrocode}

% Start the document body:
%    \begin{macrocode}
\begin{document}
%    \end{macrocode}

% Declare a title page.
% Print title, part of document being processed and version flag:
%    \begin{macrocode}
\addtocounter{page}{-1}
\begin{center}
{\LARGE\bfseries{}childdoc example\par}
\vspace{1cm}
\ifchilddoc
\ifchilddocmanual part\else chapter\fi:
`\childdocname' of `\childdocjob'\par
\else
main document: `\childdocjob'\par
\fi
version: \version\par
\end{center}
\newpage
%    \end{macrocode}

% Manually include selected file,
% otherwise process as usual:
%    \begin{macrocode}
\ifchilddocmanual
\section*{part `\childdocname'}
\input{\childdocname}
\else
%    \end{macrocode}

% Include the two chapters:
%    \begin{macrocode}
\include{cdocsch1}
\include{cdocsch2}
%    \end{macrocode}

% Include the two parts unless only chapters should be displayed:
%    \begin{macrocode}
\ifchilddoc\else
\section{part three}
\input{cdocspt3}
\section{part four}
\input{cdocspt4}
\fi
%    \end{macrocode}

% Process as usual until here:
%    \begin{macrocode}
\fi
%    \end{macrocode}

% End of document body:
%    \begin{macrocode}
\end{document}
%    \end{macrocode}
%\iffalse
%</samplemain>
%\fi
%
% %%%%%%%%%%%%%%%%%%%%%%%%%%%%%%%%%%%%%%
% \paragraph{Chapter Include Files.}
%
% The include files are called |cdocsch1.tex| and |cdocsch2.tex|.
%
%\iffalse
%<*samplechap1|samplechap2>
%\fi

% Optional override for |\version| flag:
%    \begin{macrocode}
%%\providecommand{\version}{final}
%    \end{macrocode}

% Include the main document:
%    \begin{macrocode}
\input{childdoc.def}
\childdocof{cdocsamp}
%    \end{macrocode}

%\iffalse
%</samplechap1|samplechap2>
%\fi
%
%\iffalse
%<*samplechap1>
%\fi
% Some text for chapter 1:
%    \begin{macrocode}
\section{one}
some text in chapter one
%    \end{macrocode}

%\iffalse
%</samplechap1>
%\fi
% Some text for chapter 2:
%\iffalse
%<*samplechap2>
%\fi
%    \begin{macrocode}
\section{two}
more text in chapter two
%    \end{macrocode}

%\iffalse
%</samplechap2>
%\fi
%
% %%%%%%%%%%%%%%%%%%%%%%%%%%%%%%%%%%%%%%
% \paragraph{Part Include Files.}
%
% The include files are called |cdocspt3.tex| and |cdocspt4.tex|.
%
%\iffalse
%<*samplepart3|samplepart4>
%\fi

% Optional override for |\version| flag:
%    \begin{macrocode}
%%\providecommand{\version}{final}
%    \end{macrocode}

% Include the main document:
%    \begin{macrocode}
\input{childdoc.def}
\childdocby{cdocsamp}
%    \end{macrocode}

%\iffalse
%</samplepart3|samplepart4>
%\fi
%
%\iffalse
%<*samplepart3>
%\fi
% Some text for part 3:
%    \begin{macrocode}
some text in part three
%    \end{macrocode}

%\iffalse
%</samplepart3>
%\fi
% Some text for part 4:
%\iffalse
%<*samplepart4>
%\fi
%    \begin{macrocode}
more text in part four
%    \end{macrocode}

%\iffalse
%</samplepart4>
%\fi
%
% %%%%%%%%%%%%%%%%%%%%%%%%%%%%%%%%%%%%%%
% \paragraph{Forwarding for a Complete Draft.}
%
% The following forwarding file |cdocsdrf.tex|
% compiles the main document in draft mode:
%\iffalse
%<*sampledraft>
%\fi
%    \begin{macrocode}
\def\version{draft}
\input{childdoc.def}
\childdocforward{cdocsamp}
%    \end{macrocode}

%\iffalse
%</sampledraft>
%\fi
%
% %%%%%%%%%%%%%%%%%%%%%%%%%%%%%%%%%%%%%%
% \paragraph{Forwarding for Final Version of the Chapters.}
%
% The following forwarding files |cdocsfn1.tex| and |cdocsfn2.tex|
% (with identical content)
% compile the final versions of the child documents
% |cdocsch1.tex| and |cdocsch2.tex|, respectively:
%\iffalse
%<*samplefinal>
%\fi
%    \begin{macrocode}
\def\version{final}
\input{childdoc.def}
\childdocforwardprefix[cdocsamp]{cdocsfn}{cdocsch}
%    \end{macrocode}

%\iffalse
%</samplefinal>
%\fi
%
% %%%%%%%%%%%%%%%%%%%%%%%%%%%%%%%%%%%%%%
% \paragraph{Command Line Processing.}
%
% The following three command lines generate the output files
% |cdocscld|, |cdocscl1| and |cdocscl2|
% which should be identical to
% |cdocsdrf|, |cdocsch1| and |cdocsfn2|, respectively:
% \begin{center}
% \begin{tabular}{l}
% |latex -jobname cdocscld \|\\
% |  "\def\version{draft}\input{childdoc.def}\childdocforward{cdocsamp}"|\\
% |latex -jobname cdocscl1 \|\\
% |  "\input{childdoc.def}\childdocforward[cdocsamp]{cdocsch1}"|\\
% |latex -jobname cdocscl2 \|\\
% |  "\def\version{final}\input{childdoc.def}\childdocforward{cdocsch2}"|
% \end{tabular}
% \end{center}
% Note that the trailing backslash on each first line
% merely continues the input to the second line
% (for convenient cut ant paste).
% Furthermore, the command |latex| can be replaced by any
% of its alternative versions such as |pdflatex|.
%
% %%%%%%%%%%%%%%%%%%%%%%%%%%%%%%%%%%%%%%%%%%%%%%%%%%%%%%%%%%%%%%%%%%%%%%%%%%%%%%
% %%%%%%%%%%%%%%%%%%%%%%%%%%%%%%%%%%%%%%%%%%%%%%%%%%%%%%%%%%%%%%%%%%%%%%%%%%%%%%
% \section{Implementation}
%\iffalse
%<*package>
%\fi
%
% This section describes the definitions file |childdoc.def|.

% The definitions cannot be loaded using |\usepackage| or |\RequirePackage|
% which has a mechanism to prevent loading a style file more than once.
% When loading the definitions by means of |\input|
% multiple instances have to be prevented manually:
%\iffalse
%This code needs to be before the `\ProvidesFile' directive
%which is defined at the beginning of this file.
%Therefore it is also placed there and commented out here.
%</package>
%<*discard>
%\fi
%    \begin{macrocode}
\ifdefined\childdocmain\endinput\fi
%    \end{macrocode}
%\iffalse
%</discard>
%<*package>
%\fi
%
% \macro{\ifchilddoc}
% \macro{\ifchilddocmanual}
% The conditional |\ifchilddoc| tells whether a
% child (true) or main (false) document is being compiled.
% The conditional |\ifchilddocmanual| tells whether
% the |\includeonly| mechanism is used (false) or
% the selection of child files must be performed manually (true).
% The definitions initialise to false:
%    \begin{macrocode}
\newif\ifchilddoc
\newif\ifchilddocmanual
%    \end{macrocode}

% \macro{\childdocname}
% \macro{\childdocjob}
% The macro |\childdocname| stores the name of the main document
% to be compiled. The macro |\childdocjob| stores the name of
% the document on which the \LaTeX{} compiler was originally invoked.
% The content of |\jobname| cannot be compared
% to filenames specified in the source due to different catcodes.
% The following code rescans |\jobname|, stores the result
% in |\childdocname| and saves a copy in |\childdocjob|:
%    \begin{macrocode}
\edef\childdocname{\scantokens\expandafter{\jobname\noexpand}}
\let\childdocjob\childdocname
%    \end{macrocode}

% \macro{\childdocdisable}
% The macro |\childdocdisable| prevents the main file
% from being processed more than once.
% At this stage, the main document command |\childdocmain|
% is assumed to be called once again where it should do nothing.
% Any subsequent call to it should prevent
% a secondary processing of the main document
% It overwrites the forwarding commands
% |\childdocof| and |\childdocforward|
% with empty macros to prevent further inclusions of the main document:
%    \begin{macrocode}
\newcommand{\childdocdisable}
{
  \renewcommand{\childdocmain}[1]{\renewcommand{\childdocmain}[1]{\endinput}}
  \renewcommand{\childdocof}[1]{}
  \renewcommand{\childdocby}[2][]{}
  \renewcommand{\childdocforward}[2][]{}
  \renewcommand{\childdocdisable}{}
}
%    \end{macrocode}

% \macro{\childdocmain}
% The macro |\childdocmain| is to be called at the top of the main file
% with nothing or the main filename (without extension) as argument.
% First, it breaks loops.
% If the argument is not empty and does not match |\childdocname|
% (which is set by the first inclusion of |childdoc.def|),
% |\ifchilddoc| is set to true, |\includeonly| is applied to the child file
% and |\jobname| is set to the main file
% (for proper handling of |.aux| files):
%    \begin{macrocode}
\newcommand{\childdocmain}[1]
{
  \childdocdisable\childdocmain{}
  \if?#1?\else
    \begingroup
      \def\childdoctmp{#1}
      \ifx\childdoctmp\childdocname
        \def\childdoctmp{}
      \else
        \def\childdoctmp
        {
          \childdoctrue
          \includeonly{\childdocname}
          \def\childdocjob{#1}
          \def\jobname{#1}
        }
      \fi
      \expandafter
    \endgroup
    \childdoctmp
  \fi
}
%    \end{macrocode}

% \macro{\childdocof}
% The command |\childdocof| redirects
% compilation to the main file |#1|.
%    \begin{macrocode}
\newcommand{\childdocof}[1]
{
  \childdocdisable
  \childdoctrue
  \includeonly{\childdocname}
  \def\jobname{#1}
  \def\childdocjob{#1}
  \input{#1}
}
%    \end{macrocode}

% \macro{\childdocby}
% The command |\childdocby| ....
%    \begin{macrocode}
\newcommand{\childdocby}[2][]
{
  \childdocdisable
  \childdoctrue
  \childdocmanualtrue
  \if?#1?\else
    \def\jobname{#2}
  \fi
  \def\childdocjob{#2}
  \input{#2}
  \endinput
}
%    \end{macrocode}

% \macro{\childdocforward}
% The command |\childdocforward| redirects
% compilation to the main file or
% (if the optional argument is given) a child file.
% Parameters are set as if the main file
% or a child file starting with |\childdocof| was compiled.
% Then compilation is handed over to the main file:
%    \begin{macrocode}
\newcommand{\childdocforward}[2][]
{
  \begingroup
    \if?#1?
      \def\childdoctmp
      {
        \def\childdocname{#2}
        \def\childdocjob{#2}
        \def\jobname{#2}
        \input{#2}
        \endinput
      }
    \else
      \def\childdoctmp
      {
        \childdocdisable
        \def\childdocname{#2}
        \childdoctrue
        \includeonly{#2}
        \def\childdocjob{#1}
        \def\jobname{#1}
        \input{#1}
        \endinput
      }
    \fi
    \expandafter
  \endgroup
  \childdoctmp
}
%    \end{macrocode}

% \macro{\childdocforwardprefix}
% The command |\childdocforwardprefix| redirects
% compilation to the main or a child file by means of a pattern.
% The prefix |#1| in the current filename is replaced by |#2|
% and the suffix of the current filename is kept
% (it is assumed that the filename does not contain the substring `|~~~|'
% which is used as a delimiter).
% Compilation is handed over to the new file by |\childdocforward|:
%    \begin{macrocode}
\newcommand{\childdocforwardprefix}[3][]
{
  \begingroup
    \def\childdocextract #2##1~~~{\def\childdoctmp{\childdocforward[#1]{#3##1}}}
    \expandafter\childdocextract\childdocname~~~
    \expandafter
  \endgroup
  \childdoctmp
}
%    \end{macrocode}

% \macro{\childdoc}
% The deprecated macro |\childdoc| is a legacy version of |\childdocmain|:
%    \begin{macrocode}
\newcommand{\childdoc}{\childdocmain}
%    \end{macrocode}

% \macro{\childdocredirect}
% The deprecated macro |\childdocredirect| is a legacy version
% of |\childdocforward| and |\childdocforwardprefix|:
%    \begin{macrocode}
\newcommand{\childdocredirect}[2][]
{
  \begingroup
    \if?#1?
      \def\childdoctmp{\childdocforward{#2}}
    \else
      \def\childdoctmp{\childdocforwardprefix{#1}{#2}}
    \fi
    \expandafter
  \endgroup
  \childdoctmp
}
%    \end{macrocode}

%\iffalse
%</package>
%\fi
%
\endinput
|\\
|\childdocmain{}|\\
\end{tabular}
\end{center}
at the very top of the main \LaTeX{} file,
in particular \emph{before} the |\documentclass| statement!
The argument of |\childdocmain| should be left empty
(but it must be present).

%%%%%%%%%%%%%%%%%%%%%%%%%%%%%%%%%%%%%%%%
\DescribeMacro{\childdocof}
Furthermore, add the commands
\begin{center}
\begin{tabular}{l}
|% \iffalse
%
% childdoc.dtx Copyright (C) 2017-2018 Niklas Beisert
%
% This work may be distributed and/or modified under the
% conditions of the LaTeX Project Public License, either version 1.3
% of this license or (at your option) any later version.
% The latest version of this license is in
%   http://www.latex-project.org/lppl.txt
% and version 1.3 or later is part of all distributions of LaTeX
% version 2005/12/01 or later.
%
% This work has the LPPL maintenance status `maintained'.
%
% The Current Maintainer of this work is Niklas Beisert.
%
% This work consists of the files childdoc.dtx and childdoc.ins
% and the derived files childdoc.def and cdocsamp.tex with
% cdocsch1.tex, cdocsch2.tex, cdocsdrf.tex, cdocsfn1.tex, cdocsfn2.tex.
%
%<package>\ifdefined\childdocmain\endinput\fi
%<package>\ProvidesFile{childdoc.def}[2018/12/30 v2.0 child document driver]
%<samplemain>\ProvidesFile{cdocsamp.tex}[2018/12/30 v2.0 sample for childdoc]
%<*driver>
%\ProvidesFile{childdoc.drv}[2018/12/30 v2.0 childdoc reference manual file]
\PassOptionsToClass{10pt,a4paper}{article}
\documentclass{ltxdoc}

\usepackage[margin=35mm]{geometry}
\usepackage{hyperref}
\usepackage{hyperxmp}
\usepackage[usenames]{color}

\hypersetup{colorlinks=true}
\hypersetup{pdfstartview=FitH}
\hypersetup{pdfpagemode=UseNone}
\hypersetup{pdfsource={}}
\hypersetup{pdflang={en-UK}}
\hypersetup{pdfcopyright={Copyright 2017-2018 Niklas Beisert.
  This work may be distributed and/or modified under the
  conditions of the LaTeX Project Public License, either version 1.3
  of this license or (at your option) any later version.}}
\hypersetup{pdflicenseurl={http://www.latex-project.org/lppl.txt}}
\hypersetup{pdfcontactaddress={ETH Zurich, ITP, HIT K,
  Wolfgang-Pauli-Strasse 27}}
\hypersetup{pdfcontactpostcode={8093}}
\hypersetup{pdfcontactcity={Zurich}}
\hypersetup{pdfcontactcountry={Switzerland}}
\hypersetup{pdfcontactemail={nbeisert@itp.phys.ethz.ch}}
\hypersetup{pdfcontacturl={http://people.phys.ethz.ch/\xmptilde nbeisert/}}

\newcommand{\secref}[1]{\hyperref[#1]{section \ref*{#1}}}

\parskip1ex
\parindent0pt
\let\olditemize\itemize
\def\itemize{\olditemize\parskip0pt}

\begin{document}

\title{The \textsf{childdoc} Package}
\hypersetup{pdftitle={The childdoc Package}}
\author{Niklas Beisert\\[2ex]
  Institut f\"ur Theoretische Physik\\
  Eidgen\"ossische Technische Hochschule Z\"urich\\
  Wolfgang-Pauli-Strasse 27, 8093 Z\"urich, Switzerland\\[1ex]
  \href{mailto:nbeisert@itp.phys.ethz.ch}
  {\texttt{nbeisert@itp.phys.ethz.ch}}}
\hypersetup{pdfauthor={Niklas Beisert}}
\hypersetup{pdfsubject={Manual for the LaTeX2e Package childdoc}}
\date{30 December 2018, \textsf{v2.0}}
\maketitle

\begin{abstract}\noindent
\textsf{childdoc} is a \LaTeXe{} package
that enables the direct compilation
of document sections included by |\include|
to individual files.
\end{abstract}

\begingroup
\parskip0ex
\tableofcontents
\endgroup

%%%%%%%%%%%%%%%%%%%%%%%%%%%%%%%%%%%%%%%%%%%%%%%%%%%%%%%%%%%%%%%%%%%%%%%%%%%%%%%%
%%%%%%%%%%%%%%%%%%%%%%%%%%%%%%%%%%%%%%%%%%%%%%%%%%%%%%%%%%%%%%%%%%%%%%%%%%%%%%%%
\section{Introduction}

\LaTeX{} provides a mechanism to structure a large document (such as a book)
into a main file and several child files (containing the chapters)
using the |\include| command.
This mechanism is beneficial for documents
which span hundreds of pages in order to
make the source file(s) more manageable.
Moreover, compilation can be restricted to
selected child files by means of the |\includeonly| command.
The latter feature can be used to reduce the compilation time while editing
(this was significantly more useful in the earlier days of \LaTeX{})
or to generate a smaller document which is easier to navigate.
Another application of |\includeonly| is to generate
documents consisting of selected parts of the complete document.

However, there are a few drawbacks of the plain |\include| mechanism:
\begin{itemize}
\item
The child files cannot be compiled on their own,
they can only be compiled via the main file.
A naive editing environment
(such as a text editor with an option
to have the current file processed by \LaTeX)
may require one to switch to the main file before compiling;
attempting to compile the child file produces errors.
\item
The main file must be modified (each time)
to adjust the |\includeonly| command
to the present needs. This easily leaves the main file in a messy state.
\item
The generated document will always carry the filename
of the main document. This is inconvenient if
several child files are to be compiled and
to be kept for distribution.
\end{itemize}

The present package provides a simple interface
to make child files individually compilable by \LaTeX{}.
Compiling a child file then has the same effect as compiling
the main file with an |\includeonly| command
to select the appropriate child.
Moreover the generated document will carry the name of the child
rather than the main file.
This resolves all three above issues.

This feature is meant to make the editing of books,
thesis documents and lecture notes somewhat more convenient.
However, the package can also be used efficiently for
composing a series of documents (such as exercise sheets)
which are typically distributed individually.
It then assists the author in generating the individual documents
(potentially in different versions)
as well as a document containing the collected series.
Another application is in developing style files
or other kinds of included material
where compilation of the style file could redirect
to a sample or test file.

%%%%%%%%%%%%%%%%%%%%%%%%%%%%%%%%%%%%%%%%%%%%%%%%%%%%%%%%%%%%%%%%%%%%%%%%%%%%%%%%
%%%%%%%%%%%%%%%%%%%%%%%%%%%%%%%%%%%%%%%%%%%%%%%%%%%%%%%%%%%%%%%%%%%%%%%%%%%%%%%%
\section{Usage}

First of all, the package \textsf{childdoc} is \emph{not} a standard
\LaTeXe{} |.sty| style file! Therefore it needs to be invoked in
a non-standard way.

%%%%%%%%%%%%%%%%%%%%%%%%%%%%%%%%%%%%%%%%%%%%%%%%%%%%%%%%%%%%%%%%%%%%%%%%%%%%%%%%
\subsection{Included Files}
\label{sec:include}

%%%%%%%%%%%%%%%%%%%%%%%%%%%%%%%%%%%%%%%%
\DescribeMacro{\childdocmain}
To use the package, add the commands
\begin{center}
\begin{tabular}{l}
|\input{childdoc.def}|\\
|\childdocmain{}|\\
\end{tabular}
\end{center}
at the very top of the main \LaTeX{} file,
in particular \emph{before} the |\documentclass| statement!
The argument of |\childdocmain| should be left empty
(but it must be present).

%%%%%%%%%%%%%%%%%%%%%%%%%%%%%%%%%%%%%%%%
\DescribeMacro{\childdocof}
Furthermore, add the commands
\begin{center}
\begin{tabular}{l}
|\input{childdoc.def}|\\
|\childdocof{|\textit{main}|}|\\
\end{tabular}
\end{center}
at the top of every child file \textit{child}
which is included by |\include{|\textit{child}|}|
from within the main file
(or at least for those files to be compiled individually).
The argument \textit{main} must be the filename of the main file.

There are a couple of
considerations in setting up the main and child documents:

%%%%%%%%%%%%%%%%%%%%%%%%%%%%%%%%%%%%%%%%
\paragraph{Restrictions.}

Please note the following restrictions:
\begin{itemize}
\item
|\childdocmain| must be called with one argument \textit{main}
to ensure compatibility with earlier version of the package.
It must either be empty (|\childdocmain{}|)
or precisely match the filename of the main file in which it is specified.
See \secref{sec:detection} for further information.
\item
The filename \textit{main} must be specified without the |.tex| extension.
\item
The filename \textit{main} is case sensitive
(even in case-insensitive file systems)
due to internal string comparison.
\item
The argument \textit{main} should be fully expanded, it cannot be a macro.
\item
Subdirectories and special characters should be avoided in filenames.
\item
The command |\childdocmain{|\textit{main}|}| must be followed by a whitespace.
It should not be followed immediately by another command
or by a comment mark `|%|'.
This is because the \TeX{} parser reads the token immediately following
the argument of |\childdocmain| and puts it
at the beginning of every child section;
however, a white\-space is ignored.
\end{itemize}

%%%%%%%%%%%%%%%%%%%%%%%%%%%%%%%%%%%%%%%%
\paragraph{Content of Main File.}

It is advisable to place all content in the child files included by |\include|.
Any output contained in the main file will appear in all child documents
unless suppressed manually;
it cannot be suppressed automatically by the |\includeonly| directive
and thus should normally be avoided.
A method to include some content in the main file
by means of conditional processing is described in \secref{sec:conditional}.

%%%%%%%%%%%%%%%%%%%%%%%%%%%%%%%%%%%%%%%%
\paragraph{Page Numbering.}

When only a part of the document is compiled,
the appropriate numbering of pages
(as well as other status parameters)
is determined from the |.aux| files.
The latter contain information from previous passes.
However this information needs to propagate through
all intermediate child documents.
Therefore the page numbering in child documents may well
be inconsistent until the complete document is compiled at least once.

A useful (if unconventional) way to always ensure a consistent
page numbering is to restart the numbering in each child document
and denote the pages by `\textit{child}|.|\textit{page}'
where \textit{child} represents the chapter/section number of the child file.
This can be achieved by the command
|\numberwithin{page}{|\textit{child}|}|
of the \textsf{amsmath} package
where \textit{child} can be |chapter| or |section|
depending on the chosen structuring.
Alternatively, one can modify the macro |\thepage| appropriately
and reset the counter |page| at the start of each child file.

%%%%%%%%%%%%%%%%%%%%%%%%%%%%%%%%%%%%%%%%%%%%%%%%%%%%%%%%%%%%%%%%%%%%%%%%%%%%%%%%
\subsection{Conditional Processing}
\label{sec:conditional}

The package provides a mechanism to compile different versions
of a document. To customise the versions further some conditional processing
can come in handy to distinguish which version is being compiled.
The package provides two macros to describe the compilation context:

%%%%%%%%%%%%%%%%%%%%%%%%%%%%%%%%%%%%%%%%
\DescribeMacro{\ifchilddoc}
The conditional |\ifchilddoc| distinguishes between the compilation of
child documents and the main document:
%
\begin{center}
|\ifchilddoc |\textit{child-code}| |[|\||else |\textit{main-code}]| \||fi|
\end{center}

%%%%%%%%%%%%%%%%%%%%%%%%%%%%%%%%%%%%%%%%
\DescribeMacro{\childdocname}
\DescribeMacro{\childdocjob}
The macro |\childdocname| contains the filename (without extension)
of the main or child file being processed.
Note that |\childdocjob| will always contain the name of the main file.

%%%%%%%%%%%%%%%%%%%%%%%%%%%%%%%%%%%%%%%%
\paragraph{Title Page.}

Conditional processing can be used to include a title or banner page
in the main document when proper precautions are taken.
Importantly, the code in the main file should ensure that the page counter
(as well as other status parameters which are stored in the |.aux| files)
takes the same value after the conditional processing.
Otherwise the page numbers may take divergent values
depending on which part is compiled.

For example, a title page could be declared by:
%
\begin{center}
\begin{tabular}{l}
|\ifchilddoc\||else|\\
|\addtocounter{page}{-1}|\\
\textit{code for title page}\\
|\newpage|\\
|\||fi|
\end{tabular}
\end{center}
%
A banner page for the child documents can be generated by:
%
\begin{center}
\begin{tabular}{l}
|\ifchilddoc|\\
|\addtocounter{page}{-1}|\\
\textit{code for banner page}\\
|\newpage|\\
|\||fi|
\end{tabular}
\end{center}
%
Here one could write a message such as:
\begin{center}
|This is the part \childdocname{} of \childdocjob{}.|
\end{center}

%%%%%%%%%%%%%%%%%%%%%%%%%%%%%%%%%%%%%%%%%%%%%%%%%%%%%%%%%%%%%%%%%%%%%%%%%%%%%%%%
\subsection{Flags}
\label{sec:flags}

The package makes it easy to generate different versions
of the main or child documents.
To this end compilation flags can be defined
and assigned different default values.
They will be particularly useful in conjunction
with the forwarding mechanism described in \secref{sec:forward}.

For example, it may be useful to have a flag |\version|
which can be set to |draft| or |final|.
The document source will contain some conditional code
depending on the value of |\version|.
Suppose further, the flag should default to |final| for the main file
and to |draft| for child files
which is a natural assignment for editing the document.
This is achieved by placing the following code
in the preamble of the main document
(below the |\childdocmain| directive):
%
\begin{center}
\begin{tabular}{l}
|\ifchilddoc|\\
|\providecommand{\version}{draft}|\\
|\||else|\\
|\providecommand{\version}{final}|\\
|\||fi|
\end{tabular}
\end{center}
%
The definition by |\providecommand| makes sure
that previous definitions are not overwritten.
Further statements |\providecommand{\version}{...}|
can thus be added before the above code to override it.

For the main file, one might add a line
(between |\childdocmain| and the above block)
%
\begin{center}
|%\ifchilddoc\||else\providecommand{\version}{draft}\||fi|
\end{center}
%
which can be uncommented to produce a draft version.
Likewise one can add a line to the very top of a child file
(above the |\childdocof{|\textit{main}|}| directive)
%
\begin{center}
|%\providecommand{\version}{final}|
\end{center}
%
which can be uncommented to produce the final version of this child document.

%%%%%%%%%%%%%%%%%%%%%%%%%%%%%%%%%%%%%%%%%%%%%%%%%%%%%%%%%%%%%%%%%%%%%%%%%%%%%%%%
\subsection{Forwarding}
\label{sec:forward}

Different versions of the main or child documents
using compilation flags as described in \secref{sec:flags}
can be (permanently) stored in different files
for convenient compilation, viewing and distribution.
To this end, the package defines a command
to pass on compilation to a different file:

%%%%%%%%%%%%%%%%%%%%%%%%%%%%%%%%%%%%%%%%
\DescribeMacro{\childdocforward}
The command |\childdocforward| redirects processing to
another source file:
%
\begin{center}
\begin{tabular}{l}
|\input{childdoc.def}|\\
|\childdocforward[|\textit{main}|]{|\textit{dest}|}|\\
\end{tabular}
\end{center}
%
The argument \textit{dest} is the destination file
(without extension).
It should be the main file or one of the child files.
Note that further \textsf{childdoc} directives
such as |\childdocof| and |\childdocforward|
in the indicated file will be processed in this form.
The optional argument \textit{main}
passes on directly to the main file \textit{main}
while pretending to compile the child \textit{dest}.
This form behaves as if \textit{dest}
issues |\childdocof{|\textit{main}|}| right away,
and no further \textsf{childdoc} directives will be processed.

%%%%%%%%%%%%%%%%%%%%%%%%%%%%%%%%%%%%%%%%
\DescribeMacro{\...prefix}
In the alternative form |\childdocforwardprefix|,
%
\begin{center}
\begin{tabular}{l}
|\input{childdoc.def}|\\
|\childdocforwardprefix[|\textit{main}|]{|\textit{prefix}|}{|\textit{dest}|}|
\end{tabular}
\end{center}
%
the destination file is determined by a pattern
depending on the current file:
To make this work, the current file must be called
`{\textit{prefix}\hspace{0.2em}\textit{suffix}}'
with \textit{prefix} matching precisely the argument.
Processing is then passed on to the file
`{\textit{dest}\hspace{0.2em}\textit{suffix}}'.
Surely, the same effect is achieved by
directly specifying the
argument `{\textit{dest}\hspace{0.2em}\textit{suffix}}'
in the first form.
However, that requires to set up a different file
for each child. With the alternative form of the command
all these files can have exactly the same content
which simplifies setting them up and maintaining them.

For example, the following file |draft.tex|
with a compilation flag |\version| as described in \secref{sec:flags}
compiles the main document as a draft:
%
\begin{center}
\begin{tabular}{l}
|\def\version{draft}|\\
|\input{childdoc.def}|\\
|\childdocforward{|\textit{main}|}|
\end{tabular}
\end{center}
%
Likewise, the following files |final|\textit{nn}|.tex|
compile the final version of the child document
|child|\textit{nn}|.tex|:
%
\begin{center}
\begin{tabular}{l}
|\def\version{final}|\\
|\input{childdoc.def}|\\
|\childdocforwardprefix{final}{child}|
\end{tabular}
\end{center}
%

Note that when several versions of a main file and/or of each child file
are to be generated, it may be convenient to set up a |Makefile| or
shell script to automatise the process.

%%%%%%%%%%%%%%%%%%%%%%%%%%%%%%%%%%%%%%%%%%%%%%%%%%%%%%%%%%%%%%%%%%%%%%%%%%%%%%%%
\subsection{Command Line Processing}
\label{sec:commandline}

The effect of redirection files can also be achieved by invoking
the \LaTeX{} compiler with a more elaborate command line.
Most conveniently this should be done as part
of a shell script or a |Makefile|.

When using \textsf{childdoc} in the main file, the following
command lines effectively perform a redirection
(note that depending on the shell being used,
backslashes may have to be doubled: `|\|' $\to$ `|\\|'):
%
\begin{center}
|... -jobname "|\textit{target}|" |\\|"|[\textit{flags}]%
|\input{childdoc.def}\childdocforward[|\textit{main}|]{|\textit{dest}|}"|
\end{center}
%
Here \textit{target} is the name of the output file,
\textit{main} is the name of the main file
and \textit{dest} is the name of the main or child file to be processed
(all filenames without extensions).
The optional argument \textit{main} can be omitted
if \textit{main} matches \textit{dest}.
Optionally, compilation \textit{flags} can be defined via |\def| commands.
This command line makes the \TeX{} engine believe
it is compiling the file \textit{target}
whose content is specified as the latter parameter.
The provided code then forwards the processing to
\textit{main} or \textit{dest} as described in \secref{sec:forward}.

%%%%%%%%%%%%%%%%%%%%%%%%%%%%%%%%%%%%%%%%%%%%%%%%%%%%%%%%%%%%%%%%%%%%%%%%%%%%%%%%
\subsection{Include by Input}
\label{sec:input}

Including child documents by |\include| has some restrictions by design.
Most notably, the content of a child document always occupies
its own set of pages; pages cannot be shared between child documents.
Usually, this behaviour makes perfect sense
because each child document contain an essential part of the document.
However, in some situations it may be desirable to compose
a document from a collection of parts
without having mandatory page breaks between then.
For this case, the package
provides a mechanism to include parts
by |\input| which can also be processed individually.
However, by construction this mechanism
requires manual handling of the content to be output.

%%%%%%%%%%%%%%%%%%%%%%%%%%%%%%%%%%%%%%%%
\DescribeMacro{\ifchilddocmanual}
The main file should be prepared as usual, see \secref{sec:include}.
However, the document body must make a distinction
between processing of an individual part and of the main document, e.g.:
%
\begin{center}
\begin{tabular}{l}
|\ifchilddocmanual|\\
|\input{\childdocname}|\\
|\||else|\\
\textit{document body with }|\input{|\textit{part}|}|\\
|\||fi|
\end{tabular}
\end{center}
%
The conditional |\ifchilddocmanual| is true whenever
a part to be included by |\input| is being compiled,
and the name of the part is stored in |\childdocname|.

%%%%%%%%%%%%%%%%%%%%%%%%%%%%%%%%%%%%%%%%
\DescribeMacro{\childdocby}
Each part to be included by |\input| should start with:
%
\begin{center}
\begin{tabular}{l}
|\input{childdoc.def}|\\
|\childdocby{|\textit{main}|}|\\
\end{tabular}
\end{center}
%
The directive |\childdocby| is similar to |\childdocof|
described in \secref{sec:include},
but the subsequent selection of content must be done manually.
To that end, both |\ifchilddoc| and |\ifchilddocmanual|
will be true upon processing of a part,
and the name of the part is stored in |\childdocname|.
Note that |\jobname| will be set to the filename of the current part
so that each part receives an individual |.aux| file
that does not interfere with the |.aux| file(s) of the main document.
This behaviour can be altered by the alternative form
|\childdocby[*]{|\textit{main}|}| (with a non-empty optional argument)
which uses the |.aux| file of the main document
by setting |\jobname| to \textit{main}.

%%%%%%%%%%%%%%%%%%%%%%%%%%%%%%%%%%%%%%%%%%%%%%%%%%%%%%%%%%%%%%%%%%%%%%%%%%%%%%%%
\subsection{Driver Development}
\label{sec:driver}

The \textsf{childdoc} mechanism can also be use for the development
of definition files such as \LaTeX{} styles or classes.
This case differs from the above setup with multiple parts
included by |\include| in that no |\includeonly| should be invoked.
This can be achieved by starting the include file
(before |\ProvidesPackage|) with:
%
\begin{center}
\begin{tabular}{l}
|\input{childdoc.def}|\\
|\childdocforward{|\textit{main}|}|\\
\end{tabular}
\end{center}
%
or alternatively with:
%
\begin{center}
\begin{tabular}{l}
|\input{childdoc.def}|\\
|\childdocby{|\textit{main}|}|\\
\end{tabular}
\end{center}
%
Both forms have slightly different effects as described above.
The main file is prepared as usual, see \secref{sec:include}.

%%%%%%%%%%%%%%%%%%%%%%%%%%%%%%%%%%%%%%%%%%%%%%%%%%%%%%%%%%%%%%%%%%%%%%%%%%%%%%%%
\subsection{Legacy Detection}
\label{sec:detection}

The directive |\childdocmain| in the main file can detect
whether the complete document or merely a child is to be compiled
even without using the directive |\childdocof|.
This method is deprecated because it is less robust
and there is no compelling reason to use it;
it is merely provided for backward compatibility
and it may be removed in future versions.

If the detection mechanism is to be used,
it is mandatory to correctly specify
the filename of the main file as the argument of |\childdocmain|:
%
\begin{center}
\begin{tabular}{l}
|\input{childdoc.def}|\\
|\childdocmain{|\textit{main}|}|\\
\end{tabular}
\end{center}
%
If |\jobname| does not match the argument \textit{main} of |\childdocmain|,
it is assumed that |\jobname| points to the child file to be compiled.
When using |\childdocmain| with the main file specified as argument,
it suffices to start a child file
with just |\input{|\textit{main}|}|
without loading of the package and using |\childdocof|.
If instead all processing is done
with the appropriate \textsf{childdoc} directives,
the argument of \textit{main} of |\childdocmain| can be empty.

An alternative version of the command line processing described
in \secref{sec:commandline} using the detection mechanism reads:
%
\begin{center}
|... -jobname "|\textit{target}|" "|[\textit{flags}]%
[|\def\jobname{|\textit{dest}|}|]|\input{|\textit{main}|}"|
\end{center}

%%%%%%%%%%%%%%%%%%%%%%%%%%%%%%%%%%%%%%%%%%%%%%%%%%%%%%%%%%%%%%%%%%%%%%%%%%%%%%%%
\subsection{Manual Code}
\label{sec:manual}

In case one cannot be certain whether the definitions file |childdoc.def|
is installed on the target \TeX{} distribution
and one prefers not to ship it,
it is conceivable to paste a few relevant commands into the sources.

To that end, drop all statements |\input{childdoc.def}|
and perform the replacements as outlined below.
Instead of |\childdocmain{|\textit{main}|}| add the following code
to the top of the main file:
%
\begin{center}
\begin{tabular}{l}
|\||ifdefined\childdocname\endinput\||fi\newif\ifchilddoc|\\
|\edef\childdocname{\scantokens\expandafter{\jobname\noexpand}}|\\
|\def\childdocmain{|\textit{main}|}\||ifx\childdocmain\childdocname\||else|\\
|\childdoctrue\includeonly{\childdocname}\let\jobname\childdocmain\||fi|\\
\end{tabular}
\end{center}
%
Instead of |\childdocof{|\textit{main}|}| just include the main file
at the top of each child file:
%
\begin{center}
|\input{|\textit{main}|}|
\end{center}
%
A simple redirection |\childdocforward{|\textit{dest}|}| is achieved by:
%
\begin{center}
|\def\jobname{|\textit{dest}|}\input{\jobname}|
\end{center}
%
The redirection with prefix
|\childdocforwardprefix[|\textit{prefix}|]{|\textit{dest}|}|
is accomplished by:
%
\begin{center}
\begin{tabular}{l}
|{\edef\jobname{\scantokens\expandafter{\jobname\noexpand}}|\\
|\def\redirectjob |\textit{prefix}|#1~~~{\gdef\jobname{|\textit{dest}|#1}}|\\
|\expandafter\redirectjob\jobname~~~}\input{\jobname}|
\end{tabular}
\end{center}

In an alternative approach,
child documents can be compiled by a specific command line
without additional code or specific definitions:
%
\begin{center}
|... -jobname "|\textit{target}|" "|[\textit{flags}]%
|\includeonly{|\textit{dest}|}\input{|\textit{main}|}"|
\end{center}
%

%%%%%%%%%%%%%%%%%%%%%%%%%%%%%%%%%%%%%%%%%%%%%%%%%%%%%%%%%%%%%%%%%%%%%%%%%%%%%%%%
%%%%%%%%%%%%%%%%%%%%%%%%%%%%%%%%%%%%%%%%%%%%%%%%%%%%%%%%%%%%%%%%%%%%%%%%%%%%%%%%
\section{Information}

%%%%%%%%%%%%%%%%%%%%%%%%%%%%%%%%%%%%%%%%%%%%%%%%%%%%%%%%%%%%%%%%%%%%%%%%%%%%%%%%
\subsection{Copyright}

Copyright \copyright{} 2017--2018 Niklas Beisert

This work may be distributed and/or modified under the
conditions of the \LaTeX{} Project Public License, either version 1.3
of this license or (at your option) any later version.
The latest version of this license is in
  \url{http://www.latex-project.org/lppl.txt}
and version 1.3 or later is part of all distributions of \LaTeX{}
version 2005/12/01 or later.

This work has the LPPL maintenance status `maintained'.

The Current Maintainer of this work is Niklas Beisert.

This work consists of the files |README.txt|, |childdoc.ins| and |childdoc.dtx|
as well as the derived files |childdoc.def|, |cdocsamp.tex|
with |cdocsch1.tex|, |cdocsch2.tex|, |cdocspt3.tex|, |cdocspt4.tex|,
|cdocsdrf.tex|, |cdocsfn1.tex|, |cdocsfn2.tex|
as well as |childdoc.pdf|.

%%%%%%%%%%%%%%%%%%%%%%%%%%%%%%%%%%%%%%%%%%%%%%%%%%%%%%%%%%%%%%%%%%%%%%%%%%%%%%%%
\subsection{Files and Installation}

The package consists of the files:
%
\begin{center}
\begin{tabular}{ll}
    |README.txt|   & readme file \\
    |childdoc.ins| & installation file \\
    |childdoc.dtx| & source file \\
    |childdoc.def| & definition file \\
    |cdocsamp.tex| & sample main file \\
    |cdocsch1.tex| & sample include file \\
    |cdocsch2.tex| & sample include file \\
    |cdocspt3.tex| & sample part file \\
    |cdocspt4.tex| & sample part file \\
    |cdocsdrf.tex| & sample redirection file \\
    |cdocsfn1.tex| & sample redirection file \\
    |cdocsfn2.tex| & sample redirection file \\
    |childdoc.pdf| & manual
\end{tabular}
\end{center}
%
The distribution consists of the files
|README.txt|, |childdoc.ins| and |childdoc.dtx|.
%
\begin{itemize}
\item
Run (pdf)\LaTeX{} on |childdoc.dtx|
to compile the manual |childdoc.pdf| (this file).
\item
Run \LaTeX{} on |childdoc.ins| to create the definitions file |childdoc.def|
and the sample |cdocsamp.tex| with include files
|cdocsch1.tex|, |cdocsch2.tex|, |cdocspt3.tex|, |cdocspt4.tex|,
|cdocsdrf.tex|, |cdocsfn1.tex|, |cdocsfn2.tex|.
Then copy the file |childdoc.def| to an appropriate directory of your \LaTeX{}
distribution, e.g.\ \textit{texmf-root}|/tex/latex/childdoc|.
\end{itemize}

%%%%%%%%%%%%%%%%%%%%%%%%%%%%%%%%%%%%%%%%%%%%%%%%%%%%%%%%%%%%%%%%%%%%%%%%%%%%%%%%
\subsection{Related CTAN Packages}

There are several other packages which offer a similar functionality:
%
\begin{itemize}
\item
The packages
\href{http://ctan.org/pkg/docmute}{\textsf{docmute}},
\href{http://ctan.org/pkg/includex}{\textsf{includex}} and
\href{http://ctan.org/pkg/standalone}{\textsf{standalone}}
provide commands to include only the document body of
a child file thus allowing both files to be compiled individually.
\item
The packages \href{http://ctan.org/pkg/subdocs}{\textsf{subdocs}}
and \href{http://ctan.org/pkg/subfiles}{\textsf{subfiles}}
provide structures in which the main and child documents can be
encapsulated and allowing them to be compiled individually.
The inclusion mechanism is different from the conventional |\include|.
\item
The package \href{http://ctan.org/pkg/combine}{\textsf{combine}}
is an elaborate solution to combine several documents into one.
\end{itemize}
%
See also the CTAN topic \href{http://ctan.org/topic/subdocs}{\textsf{subdocs}}
for further related packages.
The present package differs from the above solutions in that
a document structure constructed with the conventional |\include| mechanism
just needs two extra commands at the top of every file
such that all constituent files can be compiled individually.

%%%%%%%%%%%%%%%%%%%%%%%%%%%%%%%%%%%%%%%%%%%%%%%%%%%%%%%%%%%%%%%%%%%%%%%%%%%%%%%%
%\subsection{Feature Suggestions}
%
%The following is a list of features which may be useful for future
%versions of this package:
%%
%\begin{itemize}
%\item
%\ldots
%\end{itemize}

%%%%%%%%%%%%%%%%%%%%%%%%%%%%%%%%%%%%%%%%%%%%%%%%%%%%%%%%%%%%%%%%%%%%%%%%%%%%%%%%
\subsection{Revision History}

%%%%%%%%%%%%%%%%%%%%%%%%%%%%%%%%%%%%%%%%
\paragraph{v2.0:} 2018/12/30

\begin{itemize}
\item
immediate forward processing
\item
added |\childdocby| mechanism
\item
manual restructured
\end{itemize}

%%%%%%%%%%%%%%%%%%%%%%%%%%%%%%%%%%%%%%%%
\paragraph{v1.6:} 2018/01/17

\begin{itemize}
\item
application for development of include files
\item
corrections to manual
\end{itemize}

%%%%%%%%%%%%%%%%%%%%%%%%%%%%%%%%%%%%%%%%
\paragraph{v1.5:} 2017/05/21

\begin{itemize}
\item
more complete structuring introduced
\item
|\childdocof| introduced
\item
|\childdoc| renamed to |\childdocmain|
\item
|\childredirect| renamed to |\childdocforward| and |\childdocforwardprefix|
and functionality expanded
\end{itemize}

%%%%%%%%%%%%%%%%%%%%%%%%%%%%%%%%%%%%%%%%
\paragraph{v1.0:} 2017/04/27

\begin{itemize}
\item
manual and install package
\item
first version published on CTAN
\end{itemize}

%%%%%%%%%%%%%%%%%%%%%%%%%%%%%%%%%%%%%%%%
\paragraph{v0.6:} 2017/04/26

\begin{itemize}
\item
redirection mechanism added
\end{itemize}

%%%%%%%%%%%%%%%%%%%%%%%%%%%%%%%%%%%%%%%%
\paragraph{v0.5:} 2017/04/26

\begin{itemize}
\item
functionality in definition file
\end{itemize}


%%%%%%%%%%%%%%%%%%%%%%%%%%%%%%%%%%%%%%%%%%%%%%%%%%%%%%%%%%%%%%%%%%%%%%%%%%%%%%%%
%%%%%%%%%%%%%%%%%%%%%%%%%%%%%%%%%%%%%%%%%%%%%%%%%%%%%%%%%%%%%%%%%%%%%%%%%%%%%%%%
%%%%%%%%%%%%%%%%%%%%%%%%%%%%%%%%%%%%%%%%%%%%%%%%%%%%%%%%%%%%%%%%%%%%%%%%%%%%%%%%
\appendix

\settowidth\MacroIndent{\rmfamily\scriptsize 000\ }

 \DocInput{childdoc.dtx}

\end{document}
%</driver>
% \fi
%
% %%%%%%%%%%%%%%%%%%%%%%%%%%%%%%%%%%%%%%%%%%%%%%%%%%%%%%%%%%%%%%%%%%%%%%%%%%%%%%
% %%%%%%%%%%%%%%%%%%%%%%%%%%%%%%%%%%%%%%%%%%%%%%%%%%%%%%%%%%%%%%%%%%%%%%%%%%%%%%
% \section{Sample}
%\iffalse
%<*samplemain>
%\fi
%
% The following presents a sample document
% with two chapters, two parts, a title page,
% a compile flag as well as three forwarding files to set the flag.
% It consists of eight |.tex| files:
% \begin{center}
% \begin{tabular}{ll}
% |cdocsamp.tex|&main file\\
% |cdocsch1.tex|&include file for chapter 1\\
% |cdocsch2.tex|&include file for chapter 2\\
% |cdocspt3.tex|&include file for part 3\\
% |cdocspt4.tex|&include file for part 4\\
% |cdocsdrf.tex|&forwarding file for main file in draft mode\\
% |cdocsfi1.tex|&forwarding file for final version of chapter 1\\
% |cdocsfi2.tex|&forwarding file for final version of chapter 2\\
% \end{tabular}
% \end{center}
% Each of the eight files can be compiled directly by the \LaTeX{} compiler.
%
% %%%%%%%%%%%%%%%%%%%%%%%%%%%%%%%%%%%%%%
% \paragraph{Main File.}
%
% The main file is called |cdocsamp.tex|.
%
% Load the \textsf{childdoc} definitions and
% declare the filename for the main document:
%    \begin{macrocode}
\input{childdoc.def}
\childdocmain{}
%    \end{macrocode}

% Optional override for |\version| flag:
%    \begin{macrocode}
%%\ifchilddoc\else\providecommand{\version}{draft}\fi
%    \end{macrocode}

% Define the default values for the |\version| flag
% (|final| for the main file and |draft| for childs):
%    \begin{macrocode}
\ifchilddoc
\providecommand{\version}{draft}
\else
\providecommand{\version}{final}
\fi
%    \end{macrocode}

% Load the standard document class:
%    \begin{macrocode}
\documentclass[12pt]{article}
%    \end{macrocode}

% Start the document body:
%    \begin{macrocode}
\begin{document}
%    \end{macrocode}

% Declare a title page.
% Print title, part of document being processed and version flag:
%    \begin{macrocode}
\addtocounter{page}{-1}
\begin{center}
{\LARGE\bfseries{}childdoc example\par}
\vspace{1cm}
\ifchilddoc
\ifchilddocmanual part\else chapter\fi:
`\childdocname' of `\childdocjob'\par
\else
main document: `\childdocjob'\par
\fi
version: \version\par
\end{center}
\newpage
%    \end{macrocode}

% Manually include selected file,
% otherwise process as usual:
%    \begin{macrocode}
\ifchilddocmanual
\section*{part `\childdocname'}
\input{\childdocname}
\else
%    \end{macrocode}

% Include the two chapters:
%    \begin{macrocode}
\include{cdocsch1}
\include{cdocsch2}
%    \end{macrocode}

% Include the two parts unless only chapters should be displayed:
%    \begin{macrocode}
\ifchilddoc\else
\section{part three}
\input{cdocspt3}
\section{part four}
\input{cdocspt4}
\fi
%    \end{macrocode}

% Process as usual until here:
%    \begin{macrocode}
\fi
%    \end{macrocode}

% End of document body:
%    \begin{macrocode}
\end{document}
%    \end{macrocode}
%\iffalse
%</samplemain>
%\fi
%
% %%%%%%%%%%%%%%%%%%%%%%%%%%%%%%%%%%%%%%
% \paragraph{Chapter Include Files.}
%
% The include files are called |cdocsch1.tex| and |cdocsch2.tex|.
%
%\iffalse
%<*samplechap1|samplechap2>
%\fi

% Optional override for |\version| flag:
%    \begin{macrocode}
%%\providecommand{\version}{final}
%    \end{macrocode}

% Include the main document:
%    \begin{macrocode}
\input{childdoc.def}
\childdocof{cdocsamp}
%    \end{macrocode}

%\iffalse
%</samplechap1|samplechap2>
%\fi
%
%\iffalse
%<*samplechap1>
%\fi
% Some text for chapter 1:
%    \begin{macrocode}
\section{one}
some text in chapter one
%    \end{macrocode}

%\iffalse
%</samplechap1>
%\fi
% Some text for chapter 2:
%\iffalse
%<*samplechap2>
%\fi
%    \begin{macrocode}
\section{two}
more text in chapter two
%    \end{macrocode}

%\iffalse
%</samplechap2>
%\fi
%
% %%%%%%%%%%%%%%%%%%%%%%%%%%%%%%%%%%%%%%
% \paragraph{Part Include Files.}
%
% The include files are called |cdocspt3.tex| and |cdocspt4.tex|.
%
%\iffalse
%<*samplepart3|samplepart4>
%\fi

% Optional override for |\version| flag:
%    \begin{macrocode}
%%\providecommand{\version}{final}
%    \end{macrocode}

% Include the main document:
%    \begin{macrocode}
\input{childdoc.def}
\childdocby{cdocsamp}
%    \end{macrocode}

%\iffalse
%</samplepart3|samplepart4>
%\fi
%
%\iffalse
%<*samplepart3>
%\fi
% Some text for part 3:
%    \begin{macrocode}
some text in part three
%    \end{macrocode}

%\iffalse
%</samplepart3>
%\fi
% Some text for part 4:
%\iffalse
%<*samplepart4>
%\fi
%    \begin{macrocode}
more text in part four
%    \end{macrocode}

%\iffalse
%</samplepart4>
%\fi
%
% %%%%%%%%%%%%%%%%%%%%%%%%%%%%%%%%%%%%%%
% \paragraph{Forwarding for a Complete Draft.}
%
% The following forwarding file |cdocsdrf.tex|
% compiles the main document in draft mode:
%\iffalse
%<*sampledraft>
%\fi
%    \begin{macrocode}
\def\version{draft}
\input{childdoc.def}
\childdocforward{cdocsamp}
%    \end{macrocode}

%\iffalse
%</sampledraft>
%\fi
%
% %%%%%%%%%%%%%%%%%%%%%%%%%%%%%%%%%%%%%%
% \paragraph{Forwarding for Final Version of the Chapters.}
%
% The following forwarding files |cdocsfn1.tex| and |cdocsfn2.tex|
% (with identical content)
% compile the final versions of the child documents
% |cdocsch1.tex| and |cdocsch2.tex|, respectively:
%\iffalse
%<*samplefinal>
%\fi
%    \begin{macrocode}
\def\version{final}
\input{childdoc.def}
\childdocforwardprefix[cdocsamp]{cdocsfn}{cdocsch}
%    \end{macrocode}

%\iffalse
%</samplefinal>
%\fi
%
% %%%%%%%%%%%%%%%%%%%%%%%%%%%%%%%%%%%%%%
% \paragraph{Command Line Processing.}
%
% The following three command lines generate the output files
% |cdocscld|, |cdocscl1| and |cdocscl2|
% which should be identical to
% |cdocsdrf|, |cdocsch1| and |cdocsfn2|, respectively:
% \begin{center}
% \begin{tabular}{l}
% |latex -jobname cdocscld \|\\
% |  "\def\version{draft}\input{childdoc.def}\childdocforward{cdocsamp}"|\\
% |latex -jobname cdocscl1 \|\\
% |  "\input{childdoc.def}\childdocforward[cdocsamp]{cdocsch1}"|\\
% |latex -jobname cdocscl2 \|\\
% |  "\def\version{final}\input{childdoc.def}\childdocforward{cdocsch2}"|
% \end{tabular}
% \end{center}
% Note that the trailing backslash on each first line
% merely continues the input to the second line
% (for convenient cut ant paste).
% Furthermore, the command |latex| can be replaced by any
% of its alternative versions such as |pdflatex|.
%
% %%%%%%%%%%%%%%%%%%%%%%%%%%%%%%%%%%%%%%%%%%%%%%%%%%%%%%%%%%%%%%%%%%%%%%%%%%%%%%
% %%%%%%%%%%%%%%%%%%%%%%%%%%%%%%%%%%%%%%%%%%%%%%%%%%%%%%%%%%%%%%%%%%%%%%%%%%%%%%
% \section{Implementation}
%\iffalse
%<*package>
%\fi
%
% This section describes the definitions file |childdoc.def|.

% The definitions cannot be loaded using |\usepackage| or |\RequirePackage|
% which has a mechanism to prevent loading a style file more than once.
% When loading the definitions by means of |\input|
% multiple instances have to be prevented manually:
%\iffalse
%This code needs to be before the `\ProvidesFile' directive
%which is defined at the beginning of this file.
%Therefore it is also placed there and commented out here.
%</package>
%<*discard>
%\fi
%    \begin{macrocode}
\ifdefined\childdocmain\endinput\fi
%    \end{macrocode}
%\iffalse
%</discard>
%<*package>
%\fi
%
% \macro{\ifchilddoc}
% \macro{\ifchilddocmanual}
% The conditional |\ifchilddoc| tells whether a
% child (true) or main (false) document is being compiled.
% The conditional |\ifchilddocmanual| tells whether
% the |\includeonly| mechanism is used (false) or
% the selection of child files must be performed manually (true).
% The definitions initialise to false:
%    \begin{macrocode}
\newif\ifchilddoc
\newif\ifchilddocmanual
%    \end{macrocode}

% \macro{\childdocname}
% \macro{\childdocjob}
% The macro |\childdocname| stores the name of the main document
% to be compiled. The macro |\childdocjob| stores the name of
% the document on which the \LaTeX{} compiler was originally invoked.
% The content of |\jobname| cannot be compared
% to filenames specified in the source due to different catcodes.
% The following code rescans |\jobname|, stores the result
% in |\childdocname| and saves a copy in |\childdocjob|:
%    \begin{macrocode}
\edef\childdocname{\scantokens\expandafter{\jobname\noexpand}}
\let\childdocjob\childdocname
%    \end{macrocode}

% \macro{\childdocdisable}
% The macro |\childdocdisable| prevents the main file
% from being processed more than once.
% At this stage, the main document command |\childdocmain|
% is assumed to be called once again where it should do nothing.
% Any subsequent call to it should prevent
% a secondary processing of the main document
% It overwrites the forwarding commands
% |\childdocof| and |\childdocforward|
% with empty macros to prevent further inclusions of the main document:
%    \begin{macrocode}
\newcommand{\childdocdisable}
{
  \renewcommand{\childdocmain}[1]{\renewcommand{\childdocmain}[1]{\endinput}}
  \renewcommand{\childdocof}[1]{}
  \renewcommand{\childdocby}[2][]{}
  \renewcommand{\childdocforward}[2][]{}
  \renewcommand{\childdocdisable}{}
}
%    \end{macrocode}

% \macro{\childdocmain}
% The macro |\childdocmain| is to be called at the top of the main file
% with nothing or the main filename (without extension) as argument.
% First, it breaks loops.
% If the argument is not empty and does not match |\childdocname|
% (which is set by the first inclusion of |childdoc.def|),
% |\ifchilddoc| is set to true, |\includeonly| is applied to the child file
% and |\jobname| is set to the main file
% (for proper handling of |.aux| files):
%    \begin{macrocode}
\newcommand{\childdocmain}[1]
{
  \childdocdisable\childdocmain{}
  \if?#1?\else
    \begingroup
      \def\childdoctmp{#1}
      \ifx\childdoctmp\childdocname
        \def\childdoctmp{}
      \else
        \def\childdoctmp
        {
          \childdoctrue
          \includeonly{\childdocname}
          \def\childdocjob{#1}
          \def\jobname{#1}
        }
      \fi
      \expandafter
    \endgroup
    \childdoctmp
  \fi
}
%    \end{macrocode}

% \macro{\childdocof}
% The command |\childdocof| redirects
% compilation to the main file |#1|.
%    \begin{macrocode}
\newcommand{\childdocof}[1]
{
  \childdocdisable
  \childdoctrue
  \includeonly{\childdocname}
  \def\jobname{#1}
  \def\childdocjob{#1}
  \input{#1}
}
%    \end{macrocode}

% \macro{\childdocby}
% The command |\childdocby| ....
%    \begin{macrocode}
\newcommand{\childdocby}[2][]
{
  \childdocdisable
  \childdoctrue
  \childdocmanualtrue
  \if?#1?\else
    \def\jobname{#2}
  \fi
  \def\childdocjob{#2}
  \input{#2}
  \endinput
}
%    \end{macrocode}

% \macro{\childdocforward}
% The command |\childdocforward| redirects
% compilation to the main file or
% (if the optional argument is given) a child file.
% Parameters are set as if the main file
% or a child file starting with |\childdocof| was compiled.
% Then compilation is handed over to the main file:
%    \begin{macrocode}
\newcommand{\childdocforward}[2][]
{
  \begingroup
    \if?#1?
      \def\childdoctmp
      {
        \def\childdocname{#2}
        \def\childdocjob{#2}
        \def\jobname{#2}
        \input{#2}
        \endinput
      }
    \else
      \def\childdoctmp
      {
        \childdocdisable
        \def\childdocname{#2}
        \childdoctrue
        \includeonly{#2}
        \def\childdocjob{#1}
        \def\jobname{#1}
        \input{#1}
        \endinput
      }
    \fi
    \expandafter
  \endgroup
  \childdoctmp
}
%    \end{macrocode}

% \macro{\childdocforwardprefix}
% The command |\childdocforwardprefix| redirects
% compilation to the main or a child file by means of a pattern.
% The prefix |#1| in the current filename is replaced by |#2|
% and the suffix of the current filename is kept
% (it is assumed that the filename does not contain the substring `|~~~|'
% which is used as a delimiter).
% Compilation is handed over to the new file by |\childdocforward|:
%    \begin{macrocode}
\newcommand{\childdocforwardprefix}[3][]
{
  \begingroup
    \def\childdocextract #2##1~~~{\def\childdoctmp{\childdocforward[#1]{#3##1}}}
    \expandafter\childdocextract\childdocname~~~
    \expandafter
  \endgroup
  \childdoctmp
}
%    \end{macrocode}

% \macro{\childdoc}
% The deprecated macro |\childdoc| is a legacy version of |\childdocmain|:
%    \begin{macrocode}
\newcommand{\childdoc}{\childdocmain}
%    \end{macrocode}

% \macro{\childdocredirect}
% The deprecated macro |\childdocredirect| is a legacy version
% of |\childdocforward| and |\childdocforwardprefix|:
%    \begin{macrocode}
\newcommand{\childdocredirect}[2][]
{
  \begingroup
    \if?#1?
      \def\childdoctmp{\childdocforward{#2}}
    \else
      \def\childdoctmp{\childdocforwardprefix{#1}{#2}}
    \fi
    \expandafter
  \endgroup
  \childdoctmp
}
%    \end{macrocode}

%\iffalse
%</package>
%\fi
%
\endinput
|\\
|\childdocof{|\textit{main}|}|\\
\end{tabular}
\end{center}
at the top of every child file \textit{child}
which is included by |\include{|\textit{child}|}|
from within the main file
(or at least for those files to be compiled individually).
The argument \textit{main} must be the filename of the main file.

There are a couple of
considerations in setting up the main and child documents:

%%%%%%%%%%%%%%%%%%%%%%%%%%%%%%%%%%%%%%%%
\paragraph{Restrictions.}

Please note the following restrictions:
\begin{itemize}
\item
|\childdocmain| must be called with one argument \textit{main}
to ensure compatibility with earlier version of the package.
It must either be empty (|\childdocmain{}|)
or precisely match the filename of the main file in which it is specified.
See \secref{sec:detection} for further information.
\item
The filename \textit{main} must be specified without the |.tex| extension.
\item
The filename \textit{main} is case sensitive
(even in case-insensitive file systems)
due to internal string comparison.
\item
The argument \textit{main} should be fully expanded, it cannot be a macro.
\item
Subdirectories and special characters should be avoided in filenames.
\item
The command |\childdocmain{|\textit{main}|}| must be followed by a whitespace.
It should not be followed immediately by another command
or by a comment mark `|%|'.
This is because the \TeX{} parser reads the token immediately following
the argument of |\childdocmain| and puts it
at the beginning of every child section;
however, a white\-space is ignored.
\end{itemize}

%%%%%%%%%%%%%%%%%%%%%%%%%%%%%%%%%%%%%%%%
\paragraph{Content of Main File.}

It is advisable to place all content in the child files included by |\include|.
Any output contained in the main file will appear in all child documents
unless suppressed manually;
it cannot be suppressed automatically by the |\includeonly| directive
and thus should normally be avoided.
A method to include some content in the main file
by means of conditional processing is described in \secref{sec:conditional}.

%%%%%%%%%%%%%%%%%%%%%%%%%%%%%%%%%%%%%%%%
\paragraph{Page Numbering.}

When only a part of the document is compiled,
the appropriate numbering of pages
(as well as other status parameters)
is determined from the |.aux| files.
The latter contain information from previous passes.
However this information needs to propagate through
all intermediate child documents.
Therefore the page numbering in child documents may well
be inconsistent until the complete document is compiled at least once.

A useful (if unconventional) way to always ensure a consistent
page numbering is to restart the numbering in each child document
and denote the pages by `\textit{child}|.|\textit{page}'
where \textit{child} represents the chapter/section number of the child file.
This can be achieved by the command
|\numberwithin{page}{|\textit{child}|}|
of the \textsf{amsmath} package
where \textit{child} can be |chapter| or |section|
depending on the chosen structuring.
Alternatively, one can modify the macro |\thepage| appropriately
and reset the counter |page| at the start of each child file.

%%%%%%%%%%%%%%%%%%%%%%%%%%%%%%%%%%%%%%%%%%%%%%%%%%%%%%%%%%%%%%%%%%%%%%%%%%%%%%%%
\subsection{Conditional Processing}
\label{sec:conditional}

The package provides a mechanism to compile different versions
of a document. To customise the versions further some conditional processing
can come in handy to distinguish which version is being compiled.
The package provides two macros to describe the compilation context:

%%%%%%%%%%%%%%%%%%%%%%%%%%%%%%%%%%%%%%%%
\DescribeMacro{\ifchilddoc}
The conditional |\ifchilddoc| distinguishes between the compilation of
child documents and the main document:
%
\begin{center}
|\ifchilddoc |\textit{child-code}| |[|\||else |\textit{main-code}]| \||fi|
\end{center}

%%%%%%%%%%%%%%%%%%%%%%%%%%%%%%%%%%%%%%%%
\DescribeMacro{\childdocname}
\DescribeMacro{\childdocjob}
The macro |\childdocname| contains the filename (without extension)
of the main or child file being processed.
Note that |\childdocjob| will always contain the name of the main file.

%%%%%%%%%%%%%%%%%%%%%%%%%%%%%%%%%%%%%%%%
\paragraph{Title Page.}

Conditional processing can be used to include a title or banner page
in the main document when proper precautions are taken.
Importantly, the code in the main file should ensure that the page counter
(as well as other status parameters which are stored in the |.aux| files)
takes the same value after the conditional processing.
Otherwise the page numbers may take divergent values
depending on which part is compiled.

For example, a title page could be declared by:
%
\begin{center}
\begin{tabular}{l}
|\ifchilddoc\||else|\\
|\addtocounter{page}{-1}|\\
\textit{code for title page}\\
|\newpage|\\
|\||fi|
\end{tabular}
\end{center}
%
A banner page for the child documents can be generated by:
%
\begin{center}
\begin{tabular}{l}
|\ifchilddoc|\\
|\addtocounter{page}{-1}|\\
\textit{code for banner page}\\
|\newpage|\\
|\||fi|
\end{tabular}
\end{center}
%
Here one could write a message such as:
\begin{center}
|This is the part \childdocname{} of \childdocjob{}.|
\end{center}

%%%%%%%%%%%%%%%%%%%%%%%%%%%%%%%%%%%%%%%%%%%%%%%%%%%%%%%%%%%%%%%%%%%%%%%%%%%%%%%%
\subsection{Flags}
\label{sec:flags}

The package makes it easy to generate different versions
of the main or child documents.
To this end compilation flags can be defined
and assigned different default values.
They will be particularly useful in conjunction
with the forwarding mechanism described in \secref{sec:forward}.

For example, it may be useful to have a flag |\version|
which can be set to |draft| or |final|.
The document source will contain some conditional code
depending on the value of |\version|.
Suppose further, the flag should default to |final| for the main file
and to |draft| for child files
which is a natural assignment for editing the document.
This is achieved by placing the following code
in the preamble of the main document
(below the |\childdocmain| directive):
%
\begin{center}
\begin{tabular}{l}
|\ifchilddoc|\\
|\providecommand{\version}{draft}|\\
|\||else|\\
|\providecommand{\version}{final}|\\
|\||fi|
\end{tabular}
\end{center}
%
The definition by |\providecommand| makes sure
that previous definitions are not overwritten.
Further statements |\providecommand{\version}{...}|
can thus be added before the above code to override it.

For the main file, one might add a line
(between |\childdocmain| and the above block)
%
\begin{center}
|%\ifchilddoc\||else\providecommand{\version}{draft}\||fi|
\end{center}
%
which can be uncommented to produce a draft version.
Likewise one can add a line to the very top of a child file
(above the |\childdocof{|\textit{main}|}| directive)
%
\begin{center}
|%\providecommand{\version}{final}|
\end{center}
%
which can be uncommented to produce the final version of this child document.

%%%%%%%%%%%%%%%%%%%%%%%%%%%%%%%%%%%%%%%%%%%%%%%%%%%%%%%%%%%%%%%%%%%%%%%%%%%%%%%%
\subsection{Forwarding}
\label{sec:forward}

Different versions of the main or child documents
using compilation flags as described in \secref{sec:flags}
can be (permanently) stored in different files
for convenient compilation, viewing and distribution.
To this end, the package defines a command
to pass on compilation to a different file:

%%%%%%%%%%%%%%%%%%%%%%%%%%%%%%%%%%%%%%%%
\DescribeMacro{\childdocforward}
The command |\childdocforward| redirects processing to
another source file:
%
\begin{center}
\begin{tabular}{l}
|% \iffalse
%
% childdoc.dtx Copyright (C) 2017-2018 Niklas Beisert
%
% This work may be distributed and/or modified under the
% conditions of the LaTeX Project Public License, either version 1.3
% of this license or (at your option) any later version.
% The latest version of this license is in
%   http://www.latex-project.org/lppl.txt
% and version 1.3 or later is part of all distributions of LaTeX
% version 2005/12/01 or later.
%
% This work has the LPPL maintenance status `maintained'.
%
% The Current Maintainer of this work is Niklas Beisert.
%
% This work consists of the files childdoc.dtx and childdoc.ins
% and the derived files childdoc.def and cdocsamp.tex with
% cdocsch1.tex, cdocsch2.tex, cdocsdrf.tex, cdocsfn1.tex, cdocsfn2.tex.
%
%<package>\ifdefined\childdocmain\endinput\fi
%<package>\ProvidesFile{childdoc.def}[2018/12/30 v2.0 child document driver]
%<samplemain>\ProvidesFile{cdocsamp.tex}[2018/12/30 v2.0 sample for childdoc]
%<*driver>
%\ProvidesFile{childdoc.drv}[2018/12/30 v2.0 childdoc reference manual file]
\PassOptionsToClass{10pt,a4paper}{article}
\documentclass{ltxdoc}

\usepackage[margin=35mm]{geometry}
\usepackage{hyperref}
\usepackage{hyperxmp}
\usepackage[usenames]{color}

\hypersetup{colorlinks=true}
\hypersetup{pdfstartview=FitH}
\hypersetup{pdfpagemode=UseNone}
\hypersetup{pdfsource={}}
\hypersetup{pdflang={en-UK}}
\hypersetup{pdfcopyright={Copyright 2017-2018 Niklas Beisert.
  This work may be distributed and/or modified under the
  conditions of the LaTeX Project Public License, either version 1.3
  of this license or (at your option) any later version.}}
\hypersetup{pdflicenseurl={http://www.latex-project.org/lppl.txt}}
\hypersetup{pdfcontactaddress={ETH Zurich, ITP, HIT K,
  Wolfgang-Pauli-Strasse 27}}
\hypersetup{pdfcontactpostcode={8093}}
\hypersetup{pdfcontactcity={Zurich}}
\hypersetup{pdfcontactcountry={Switzerland}}
\hypersetup{pdfcontactemail={nbeisert@itp.phys.ethz.ch}}
\hypersetup{pdfcontacturl={http://people.phys.ethz.ch/\xmptilde nbeisert/}}

\newcommand{\secref}[1]{\hyperref[#1]{section \ref*{#1}}}

\parskip1ex
\parindent0pt
\let\olditemize\itemize
\def\itemize{\olditemize\parskip0pt}

\begin{document}

\title{The \textsf{childdoc} Package}
\hypersetup{pdftitle={The childdoc Package}}
\author{Niklas Beisert\\[2ex]
  Institut f\"ur Theoretische Physik\\
  Eidgen\"ossische Technische Hochschule Z\"urich\\
  Wolfgang-Pauli-Strasse 27, 8093 Z\"urich, Switzerland\\[1ex]
  \href{mailto:nbeisert@itp.phys.ethz.ch}
  {\texttt{nbeisert@itp.phys.ethz.ch}}}
\hypersetup{pdfauthor={Niklas Beisert}}
\hypersetup{pdfsubject={Manual for the LaTeX2e Package childdoc}}
\date{30 December 2018, \textsf{v2.0}}
\maketitle

\begin{abstract}\noindent
\textsf{childdoc} is a \LaTeXe{} package
that enables the direct compilation
of document sections included by |\include|
to individual files.
\end{abstract}

\begingroup
\parskip0ex
\tableofcontents
\endgroup

%%%%%%%%%%%%%%%%%%%%%%%%%%%%%%%%%%%%%%%%%%%%%%%%%%%%%%%%%%%%%%%%%%%%%%%%%%%%%%%%
%%%%%%%%%%%%%%%%%%%%%%%%%%%%%%%%%%%%%%%%%%%%%%%%%%%%%%%%%%%%%%%%%%%%%%%%%%%%%%%%
\section{Introduction}

\LaTeX{} provides a mechanism to structure a large document (such as a book)
into a main file and several child files (containing the chapters)
using the |\include| command.
This mechanism is beneficial for documents
which span hundreds of pages in order to
make the source file(s) more manageable.
Moreover, compilation can be restricted to
selected child files by means of the |\includeonly| command.
The latter feature can be used to reduce the compilation time while editing
(this was significantly more useful in the earlier days of \LaTeX{})
or to generate a smaller document which is easier to navigate.
Another application of |\includeonly| is to generate
documents consisting of selected parts of the complete document.

However, there are a few drawbacks of the plain |\include| mechanism:
\begin{itemize}
\item
The child files cannot be compiled on their own,
they can only be compiled via the main file.
A naive editing environment
(such as a text editor with an option
to have the current file processed by \LaTeX)
may require one to switch to the main file before compiling;
attempting to compile the child file produces errors.
\item
The main file must be modified (each time)
to adjust the |\includeonly| command
to the present needs. This easily leaves the main file in a messy state.
\item
The generated document will always carry the filename
of the main document. This is inconvenient if
several child files are to be compiled and
to be kept for distribution.
\end{itemize}

The present package provides a simple interface
to make child files individually compilable by \LaTeX{}.
Compiling a child file then has the same effect as compiling
the main file with an |\includeonly| command
to select the appropriate child.
Moreover the generated document will carry the name of the child
rather than the main file.
This resolves all three above issues.

This feature is meant to make the editing of books,
thesis documents and lecture notes somewhat more convenient.
However, the package can also be used efficiently for
composing a series of documents (such as exercise sheets)
which are typically distributed individually.
It then assists the author in generating the individual documents
(potentially in different versions)
as well as a document containing the collected series.
Another application is in developing style files
or other kinds of included material
where compilation of the style file could redirect
to a sample or test file.

%%%%%%%%%%%%%%%%%%%%%%%%%%%%%%%%%%%%%%%%%%%%%%%%%%%%%%%%%%%%%%%%%%%%%%%%%%%%%%%%
%%%%%%%%%%%%%%%%%%%%%%%%%%%%%%%%%%%%%%%%%%%%%%%%%%%%%%%%%%%%%%%%%%%%%%%%%%%%%%%%
\section{Usage}

First of all, the package \textsf{childdoc} is \emph{not} a standard
\LaTeXe{} |.sty| style file! Therefore it needs to be invoked in
a non-standard way.

%%%%%%%%%%%%%%%%%%%%%%%%%%%%%%%%%%%%%%%%%%%%%%%%%%%%%%%%%%%%%%%%%%%%%%%%%%%%%%%%
\subsection{Included Files}
\label{sec:include}

%%%%%%%%%%%%%%%%%%%%%%%%%%%%%%%%%%%%%%%%
\DescribeMacro{\childdocmain}
To use the package, add the commands
\begin{center}
\begin{tabular}{l}
|\input{childdoc.def}|\\
|\childdocmain{}|\\
\end{tabular}
\end{center}
at the very top of the main \LaTeX{} file,
in particular \emph{before} the |\documentclass| statement!
The argument of |\childdocmain| should be left empty
(but it must be present).

%%%%%%%%%%%%%%%%%%%%%%%%%%%%%%%%%%%%%%%%
\DescribeMacro{\childdocof}
Furthermore, add the commands
\begin{center}
\begin{tabular}{l}
|\input{childdoc.def}|\\
|\childdocof{|\textit{main}|}|\\
\end{tabular}
\end{center}
at the top of every child file \textit{child}
which is included by |\include{|\textit{child}|}|
from within the main file
(or at least for those files to be compiled individually).
The argument \textit{main} must be the filename of the main file.

There are a couple of
considerations in setting up the main and child documents:

%%%%%%%%%%%%%%%%%%%%%%%%%%%%%%%%%%%%%%%%
\paragraph{Restrictions.}

Please note the following restrictions:
\begin{itemize}
\item
|\childdocmain| must be called with one argument \textit{main}
to ensure compatibility with earlier version of the package.
It must either be empty (|\childdocmain{}|)
or precisely match the filename of the main file in which it is specified.
See \secref{sec:detection} for further information.
\item
The filename \textit{main} must be specified without the |.tex| extension.
\item
The filename \textit{main} is case sensitive
(even in case-insensitive file systems)
due to internal string comparison.
\item
The argument \textit{main} should be fully expanded, it cannot be a macro.
\item
Subdirectories and special characters should be avoided in filenames.
\item
The command |\childdocmain{|\textit{main}|}| must be followed by a whitespace.
It should not be followed immediately by another command
or by a comment mark `|%|'.
This is because the \TeX{} parser reads the token immediately following
the argument of |\childdocmain| and puts it
at the beginning of every child section;
however, a white\-space is ignored.
\end{itemize}

%%%%%%%%%%%%%%%%%%%%%%%%%%%%%%%%%%%%%%%%
\paragraph{Content of Main File.}

It is advisable to place all content in the child files included by |\include|.
Any output contained in the main file will appear in all child documents
unless suppressed manually;
it cannot be suppressed automatically by the |\includeonly| directive
and thus should normally be avoided.
A method to include some content in the main file
by means of conditional processing is described in \secref{sec:conditional}.

%%%%%%%%%%%%%%%%%%%%%%%%%%%%%%%%%%%%%%%%
\paragraph{Page Numbering.}

When only a part of the document is compiled,
the appropriate numbering of pages
(as well as other status parameters)
is determined from the |.aux| files.
The latter contain information from previous passes.
However this information needs to propagate through
all intermediate child documents.
Therefore the page numbering in child documents may well
be inconsistent until the complete document is compiled at least once.

A useful (if unconventional) way to always ensure a consistent
page numbering is to restart the numbering in each child document
and denote the pages by `\textit{child}|.|\textit{page}'
where \textit{child} represents the chapter/section number of the child file.
This can be achieved by the command
|\numberwithin{page}{|\textit{child}|}|
of the \textsf{amsmath} package
where \textit{child} can be |chapter| or |section|
depending on the chosen structuring.
Alternatively, one can modify the macro |\thepage| appropriately
and reset the counter |page| at the start of each child file.

%%%%%%%%%%%%%%%%%%%%%%%%%%%%%%%%%%%%%%%%%%%%%%%%%%%%%%%%%%%%%%%%%%%%%%%%%%%%%%%%
\subsection{Conditional Processing}
\label{sec:conditional}

The package provides a mechanism to compile different versions
of a document. To customise the versions further some conditional processing
can come in handy to distinguish which version is being compiled.
The package provides two macros to describe the compilation context:

%%%%%%%%%%%%%%%%%%%%%%%%%%%%%%%%%%%%%%%%
\DescribeMacro{\ifchilddoc}
The conditional |\ifchilddoc| distinguishes between the compilation of
child documents and the main document:
%
\begin{center}
|\ifchilddoc |\textit{child-code}| |[|\||else |\textit{main-code}]| \||fi|
\end{center}

%%%%%%%%%%%%%%%%%%%%%%%%%%%%%%%%%%%%%%%%
\DescribeMacro{\childdocname}
\DescribeMacro{\childdocjob}
The macro |\childdocname| contains the filename (without extension)
of the main or child file being processed.
Note that |\childdocjob| will always contain the name of the main file.

%%%%%%%%%%%%%%%%%%%%%%%%%%%%%%%%%%%%%%%%
\paragraph{Title Page.}

Conditional processing can be used to include a title or banner page
in the main document when proper precautions are taken.
Importantly, the code in the main file should ensure that the page counter
(as well as other status parameters which are stored in the |.aux| files)
takes the same value after the conditional processing.
Otherwise the page numbers may take divergent values
depending on which part is compiled.

For example, a title page could be declared by:
%
\begin{center}
\begin{tabular}{l}
|\ifchilddoc\||else|\\
|\addtocounter{page}{-1}|\\
\textit{code for title page}\\
|\newpage|\\
|\||fi|
\end{tabular}
\end{center}
%
A banner page for the child documents can be generated by:
%
\begin{center}
\begin{tabular}{l}
|\ifchilddoc|\\
|\addtocounter{page}{-1}|\\
\textit{code for banner page}\\
|\newpage|\\
|\||fi|
\end{tabular}
\end{center}
%
Here one could write a message such as:
\begin{center}
|This is the part \childdocname{} of \childdocjob{}.|
\end{center}

%%%%%%%%%%%%%%%%%%%%%%%%%%%%%%%%%%%%%%%%%%%%%%%%%%%%%%%%%%%%%%%%%%%%%%%%%%%%%%%%
\subsection{Flags}
\label{sec:flags}

The package makes it easy to generate different versions
of the main or child documents.
To this end compilation flags can be defined
and assigned different default values.
They will be particularly useful in conjunction
with the forwarding mechanism described in \secref{sec:forward}.

For example, it may be useful to have a flag |\version|
which can be set to |draft| or |final|.
The document source will contain some conditional code
depending on the value of |\version|.
Suppose further, the flag should default to |final| for the main file
and to |draft| for child files
which is a natural assignment for editing the document.
This is achieved by placing the following code
in the preamble of the main document
(below the |\childdocmain| directive):
%
\begin{center}
\begin{tabular}{l}
|\ifchilddoc|\\
|\providecommand{\version}{draft}|\\
|\||else|\\
|\providecommand{\version}{final}|\\
|\||fi|
\end{tabular}
\end{center}
%
The definition by |\providecommand| makes sure
that previous definitions are not overwritten.
Further statements |\providecommand{\version}{...}|
can thus be added before the above code to override it.

For the main file, one might add a line
(between |\childdocmain| and the above block)
%
\begin{center}
|%\ifchilddoc\||else\providecommand{\version}{draft}\||fi|
\end{center}
%
which can be uncommented to produce a draft version.
Likewise one can add a line to the very top of a child file
(above the |\childdocof{|\textit{main}|}| directive)
%
\begin{center}
|%\providecommand{\version}{final}|
\end{center}
%
which can be uncommented to produce the final version of this child document.

%%%%%%%%%%%%%%%%%%%%%%%%%%%%%%%%%%%%%%%%%%%%%%%%%%%%%%%%%%%%%%%%%%%%%%%%%%%%%%%%
\subsection{Forwarding}
\label{sec:forward}

Different versions of the main or child documents
using compilation flags as described in \secref{sec:flags}
can be (permanently) stored in different files
for convenient compilation, viewing and distribution.
To this end, the package defines a command
to pass on compilation to a different file:

%%%%%%%%%%%%%%%%%%%%%%%%%%%%%%%%%%%%%%%%
\DescribeMacro{\childdocforward}
The command |\childdocforward| redirects processing to
another source file:
%
\begin{center}
\begin{tabular}{l}
|\input{childdoc.def}|\\
|\childdocforward[|\textit{main}|]{|\textit{dest}|}|\\
\end{tabular}
\end{center}
%
The argument \textit{dest} is the destination file
(without extension).
It should be the main file or one of the child files.
Note that further \textsf{childdoc} directives
such as |\childdocof| and |\childdocforward|
in the indicated file will be processed in this form.
The optional argument \textit{main}
passes on directly to the main file \textit{main}
while pretending to compile the child \textit{dest}.
This form behaves as if \textit{dest}
issues |\childdocof{|\textit{main}|}| right away,
and no further \textsf{childdoc} directives will be processed.

%%%%%%%%%%%%%%%%%%%%%%%%%%%%%%%%%%%%%%%%
\DescribeMacro{\...prefix}
In the alternative form |\childdocforwardprefix|,
%
\begin{center}
\begin{tabular}{l}
|\input{childdoc.def}|\\
|\childdocforwardprefix[|\textit{main}|]{|\textit{prefix}|}{|\textit{dest}|}|
\end{tabular}
\end{center}
%
the destination file is determined by a pattern
depending on the current file:
To make this work, the current file must be called
`{\textit{prefix}\hspace{0.2em}\textit{suffix}}'
with \textit{prefix} matching precisely the argument.
Processing is then passed on to the file
`{\textit{dest}\hspace{0.2em}\textit{suffix}}'.
Surely, the same effect is achieved by
directly specifying the
argument `{\textit{dest}\hspace{0.2em}\textit{suffix}}'
in the first form.
However, that requires to set up a different file
for each child. With the alternative form of the command
all these files can have exactly the same content
which simplifies setting them up and maintaining them.

For example, the following file |draft.tex|
with a compilation flag |\version| as described in \secref{sec:flags}
compiles the main document as a draft:
%
\begin{center}
\begin{tabular}{l}
|\def\version{draft}|\\
|\input{childdoc.def}|\\
|\childdocforward{|\textit{main}|}|
\end{tabular}
\end{center}
%
Likewise, the following files |final|\textit{nn}|.tex|
compile the final version of the child document
|child|\textit{nn}|.tex|:
%
\begin{center}
\begin{tabular}{l}
|\def\version{final}|\\
|\input{childdoc.def}|\\
|\childdocforwardprefix{final}{child}|
\end{tabular}
\end{center}
%

Note that when several versions of a main file and/or of each child file
are to be generated, it may be convenient to set up a |Makefile| or
shell script to automatise the process.

%%%%%%%%%%%%%%%%%%%%%%%%%%%%%%%%%%%%%%%%%%%%%%%%%%%%%%%%%%%%%%%%%%%%%%%%%%%%%%%%
\subsection{Command Line Processing}
\label{sec:commandline}

The effect of redirection files can also be achieved by invoking
the \LaTeX{} compiler with a more elaborate command line.
Most conveniently this should be done as part
of a shell script or a |Makefile|.

When using \textsf{childdoc} in the main file, the following
command lines effectively perform a redirection
(note that depending on the shell being used,
backslashes may have to be doubled: `|\|' $\to$ `|\\|'):
%
\begin{center}
|... -jobname "|\textit{target}|" |\\|"|[\textit{flags}]%
|\input{childdoc.def}\childdocforward[|\textit{main}|]{|\textit{dest}|}"|
\end{center}
%
Here \textit{target} is the name of the output file,
\textit{main} is the name of the main file
and \textit{dest} is the name of the main or child file to be processed
(all filenames without extensions).
The optional argument \textit{main} can be omitted
if \textit{main} matches \textit{dest}.
Optionally, compilation \textit{flags} can be defined via |\def| commands.
This command line makes the \TeX{} engine believe
it is compiling the file \textit{target}
whose content is specified as the latter parameter.
The provided code then forwards the processing to
\textit{main} or \textit{dest} as described in \secref{sec:forward}.

%%%%%%%%%%%%%%%%%%%%%%%%%%%%%%%%%%%%%%%%%%%%%%%%%%%%%%%%%%%%%%%%%%%%%%%%%%%%%%%%
\subsection{Include by Input}
\label{sec:input}

Including child documents by |\include| has some restrictions by design.
Most notably, the content of a child document always occupies
its own set of pages; pages cannot be shared between child documents.
Usually, this behaviour makes perfect sense
because each child document contain an essential part of the document.
However, in some situations it may be desirable to compose
a document from a collection of parts
without having mandatory page breaks between then.
For this case, the package
provides a mechanism to include parts
by |\input| which can also be processed individually.
However, by construction this mechanism
requires manual handling of the content to be output.

%%%%%%%%%%%%%%%%%%%%%%%%%%%%%%%%%%%%%%%%
\DescribeMacro{\ifchilddocmanual}
The main file should be prepared as usual, see \secref{sec:include}.
However, the document body must make a distinction
between processing of an individual part and of the main document, e.g.:
%
\begin{center}
\begin{tabular}{l}
|\ifchilddocmanual|\\
|\input{\childdocname}|\\
|\||else|\\
\textit{document body with }|\input{|\textit{part}|}|\\
|\||fi|
\end{tabular}
\end{center}
%
The conditional |\ifchilddocmanual| is true whenever
a part to be included by |\input| is being compiled,
and the name of the part is stored in |\childdocname|.

%%%%%%%%%%%%%%%%%%%%%%%%%%%%%%%%%%%%%%%%
\DescribeMacro{\childdocby}
Each part to be included by |\input| should start with:
%
\begin{center}
\begin{tabular}{l}
|\input{childdoc.def}|\\
|\childdocby{|\textit{main}|}|\\
\end{tabular}
\end{center}
%
The directive |\childdocby| is similar to |\childdocof|
described in \secref{sec:include},
but the subsequent selection of content must be done manually.
To that end, both |\ifchilddoc| and |\ifchilddocmanual|
will be true upon processing of a part,
and the name of the part is stored in |\childdocname|.
Note that |\jobname| will be set to the filename of the current part
so that each part receives an individual |.aux| file
that does not interfere with the |.aux| file(s) of the main document.
This behaviour can be altered by the alternative form
|\childdocby[*]{|\textit{main}|}| (with a non-empty optional argument)
which uses the |.aux| file of the main document
by setting |\jobname| to \textit{main}.

%%%%%%%%%%%%%%%%%%%%%%%%%%%%%%%%%%%%%%%%%%%%%%%%%%%%%%%%%%%%%%%%%%%%%%%%%%%%%%%%
\subsection{Driver Development}
\label{sec:driver}

The \textsf{childdoc} mechanism can also be use for the development
of definition files such as \LaTeX{} styles or classes.
This case differs from the above setup with multiple parts
included by |\include| in that no |\includeonly| should be invoked.
This can be achieved by starting the include file
(before |\ProvidesPackage|) with:
%
\begin{center}
\begin{tabular}{l}
|\input{childdoc.def}|\\
|\childdocforward{|\textit{main}|}|\\
\end{tabular}
\end{center}
%
or alternatively with:
%
\begin{center}
\begin{tabular}{l}
|\input{childdoc.def}|\\
|\childdocby{|\textit{main}|}|\\
\end{tabular}
\end{center}
%
Both forms have slightly different effects as described above.
The main file is prepared as usual, see \secref{sec:include}.

%%%%%%%%%%%%%%%%%%%%%%%%%%%%%%%%%%%%%%%%%%%%%%%%%%%%%%%%%%%%%%%%%%%%%%%%%%%%%%%%
\subsection{Legacy Detection}
\label{sec:detection}

The directive |\childdocmain| in the main file can detect
whether the complete document or merely a child is to be compiled
even without using the directive |\childdocof|.
This method is deprecated because it is less robust
and there is no compelling reason to use it;
it is merely provided for backward compatibility
and it may be removed in future versions.

If the detection mechanism is to be used,
it is mandatory to correctly specify
the filename of the main file as the argument of |\childdocmain|:
%
\begin{center}
\begin{tabular}{l}
|\input{childdoc.def}|\\
|\childdocmain{|\textit{main}|}|\\
\end{tabular}
\end{center}
%
If |\jobname| does not match the argument \textit{main} of |\childdocmain|,
it is assumed that |\jobname| points to the child file to be compiled.
When using |\childdocmain| with the main file specified as argument,
it suffices to start a child file
with just |\input{|\textit{main}|}|
without loading of the package and using |\childdocof|.
If instead all processing is done
with the appropriate \textsf{childdoc} directives,
the argument of \textit{main} of |\childdocmain| can be empty.

An alternative version of the command line processing described
in \secref{sec:commandline} using the detection mechanism reads:
%
\begin{center}
|... -jobname "|\textit{target}|" "|[\textit{flags}]%
[|\def\jobname{|\textit{dest}|}|]|\input{|\textit{main}|}"|
\end{center}

%%%%%%%%%%%%%%%%%%%%%%%%%%%%%%%%%%%%%%%%%%%%%%%%%%%%%%%%%%%%%%%%%%%%%%%%%%%%%%%%
\subsection{Manual Code}
\label{sec:manual}

In case one cannot be certain whether the definitions file |childdoc.def|
is installed on the target \TeX{} distribution
and one prefers not to ship it,
it is conceivable to paste a few relevant commands into the sources.

To that end, drop all statements |\input{childdoc.def}|
and perform the replacements as outlined below.
Instead of |\childdocmain{|\textit{main}|}| add the following code
to the top of the main file:
%
\begin{center}
\begin{tabular}{l}
|\||ifdefined\childdocname\endinput\||fi\newif\ifchilddoc|\\
|\edef\childdocname{\scantokens\expandafter{\jobname\noexpand}}|\\
|\def\childdocmain{|\textit{main}|}\||ifx\childdocmain\childdocname\||else|\\
|\childdoctrue\includeonly{\childdocname}\let\jobname\childdocmain\||fi|\\
\end{tabular}
\end{center}
%
Instead of |\childdocof{|\textit{main}|}| just include the main file
at the top of each child file:
%
\begin{center}
|\input{|\textit{main}|}|
\end{center}
%
A simple redirection |\childdocforward{|\textit{dest}|}| is achieved by:
%
\begin{center}
|\def\jobname{|\textit{dest}|}\input{\jobname}|
\end{center}
%
The redirection with prefix
|\childdocforwardprefix[|\textit{prefix}|]{|\textit{dest}|}|
is accomplished by:
%
\begin{center}
\begin{tabular}{l}
|{\edef\jobname{\scantokens\expandafter{\jobname\noexpand}}|\\
|\def\redirectjob |\textit{prefix}|#1~~~{\gdef\jobname{|\textit{dest}|#1}}|\\
|\expandafter\redirectjob\jobname~~~}\input{\jobname}|
\end{tabular}
\end{center}

In an alternative approach,
child documents can be compiled by a specific command line
without additional code or specific definitions:
%
\begin{center}
|... -jobname "|\textit{target}|" "|[\textit{flags}]%
|\includeonly{|\textit{dest}|}\input{|\textit{main}|}"|
\end{center}
%

%%%%%%%%%%%%%%%%%%%%%%%%%%%%%%%%%%%%%%%%%%%%%%%%%%%%%%%%%%%%%%%%%%%%%%%%%%%%%%%%
%%%%%%%%%%%%%%%%%%%%%%%%%%%%%%%%%%%%%%%%%%%%%%%%%%%%%%%%%%%%%%%%%%%%%%%%%%%%%%%%
\section{Information}

%%%%%%%%%%%%%%%%%%%%%%%%%%%%%%%%%%%%%%%%%%%%%%%%%%%%%%%%%%%%%%%%%%%%%%%%%%%%%%%%
\subsection{Copyright}

Copyright \copyright{} 2017--2018 Niklas Beisert

This work may be distributed and/or modified under the
conditions of the \LaTeX{} Project Public License, either version 1.3
of this license or (at your option) any later version.
The latest version of this license is in
  \url{http://www.latex-project.org/lppl.txt}
and version 1.3 or later is part of all distributions of \LaTeX{}
version 2005/12/01 or later.

This work has the LPPL maintenance status `maintained'.

The Current Maintainer of this work is Niklas Beisert.

This work consists of the files |README.txt|, |childdoc.ins| and |childdoc.dtx|
as well as the derived files |childdoc.def|, |cdocsamp.tex|
with |cdocsch1.tex|, |cdocsch2.tex|, |cdocspt3.tex|, |cdocspt4.tex|,
|cdocsdrf.tex|, |cdocsfn1.tex|, |cdocsfn2.tex|
as well as |childdoc.pdf|.

%%%%%%%%%%%%%%%%%%%%%%%%%%%%%%%%%%%%%%%%%%%%%%%%%%%%%%%%%%%%%%%%%%%%%%%%%%%%%%%%
\subsection{Files and Installation}

The package consists of the files:
%
\begin{center}
\begin{tabular}{ll}
    |README.txt|   & readme file \\
    |childdoc.ins| & installation file \\
    |childdoc.dtx| & source file \\
    |childdoc.def| & definition file \\
    |cdocsamp.tex| & sample main file \\
    |cdocsch1.tex| & sample include file \\
    |cdocsch2.tex| & sample include file \\
    |cdocspt3.tex| & sample part file \\
    |cdocspt4.tex| & sample part file \\
    |cdocsdrf.tex| & sample redirection file \\
    |cdocsfn1.tex| & sample redirection file \\
    |cdocsfn2.tex| & sample redirection file \\
    |childdoc.pdf| & manual
\end{tabular}
\end{center}
%
The distribution consists of the files
|README.txt|, |childdoc.ins| and |childdoc.dtx|.
%
\begin{itemize}
\item
Run (pdf)\LaTeX{} on |childdoc.dtx|
to compile the manual |childdoc.pdf| (this file).
\item
Run \LaTeX{} on |childdoc.ins| to create the definitions file |childdoc.def|
and the sample |cdocsamp.tex| with include files
|cdocsch1.tex|, |cdocsch2.tex|, |cdocspt3.tex|, |cdocspt4.tex|,
|cdocsdrf.tex|, |cdocsfn1.tex|, |cdocsfn2.tex|.
Then copy the file |childdoc.def| to an appropriate directory of your \LaTeX{}
distribution, e.g.\ \textit{texmf-root}|/tex/latex/childdoc|.
\end{itemize}

%%%%%%%%%%%%%%%%%%%%%%%%%%%%%%%%%%%%%%%%%%%%%%%%%%%%%%%%%%%%%%%%%%%%%%%%%%%%%%%%
\subsection{Related CTAN Packages}

There are several other packages which offer a similar functionality:
%
\begin{itemize}
\item
The packages
\href{http://ctan.org/pkg/docmute}{\textsf{docmute}},
\href{http://ctan.org/pkg/includex}{\textsf{includex}} and
\href{http://ctan.org/pkg/standalone}{\textsf{standalone}}
provide commands to include only the document body of
a child file thus allowing both files to be compiled individually.
\item
The packages \href{http://ctan.org/pkg/subdocs}{\textsf{subdocs}}
and \href{http://ctan.org/pkg/subfiles}{\textsf{subfiles}}
provide structures in which the main and child documents can be
encapsulated and allowing them to be compiled individually.
The inclusion mechanism is different from the conventional |\include|.
\item
The package \href{http://ctan.org/pkg/combine}{\textsf{combine}}
is an elaborate solution to combine several documents into one.
\end{itemize}
%
See also the CTAN topic \href{http://ctan.org/topic/subdocs}{\textsf{subdocs}}
for further related packages.
The present package differs from the above solutions in that
a document structure constructed with the conventional |\include| mechanism
just needs two extra commands at the top of every file
such that all constituent files can be compiled individually.

%%%%%%%%%%%%%%%%%%%%%%%%%%%%%%%%%%%%%%%%%%%%%%%%%%%%%%%%%%%%%%%%%%%%%%%%%%%%%%%%
%\subsection{Feature Suggestions}
%
%The following is a list of features which may be useful for future
%versions of this package:
%%
%\begin{itemize}
%\item
%\ldots
%\end{itemize}

%%%%%%%%%%%%%%%%%%%%%%%%%%%%%%%%%%%%%%%%%%%%%%%%%%%%%%%%%%%%%%%%%%%%%%%%%%%%%%%%
\subsection{Revision History}

%%%%%%%%%%%%%%%%%%%%%%%%%%%%%%%%%%%%%%%%
\paragraph{v2.0:} 2018/12/30

\begin{itemize}
\item
immediate forward processing
\item
added |\childdocby| mechanism
\item
manual restructured
\end{itemize}

%%%%%%%%%%%%%%%%%%%%%%%%%%%%%%%%%%%%%%%%
\paragraph{v1.6:} 2018/01/17

\begin{itemize}
\item
application for development of include files
\item
corrections to manual
\end{itemize}

%%%%%%%%%%%%%%%%%%%%%%%%%%%%%%%%%%%%%%%%
\paragraph{v1.5:} 2017/05/21

\begin{itemize}
\item
more complete structuring introduced
\item
|\childdocof| introduced
\item
|\childdoc| renamed to |\childdocmain|
\item
|\childredirect| renamed to |\childdocforward| and |\childdocforwardprefix|
and functionality expanded
\end{itemize}

%%%%%%%%%%%%%%%%%%%%%%%%%%%%%%%%%%%%%%%%
\paragraph{v1.0:} 2017/04/27

\begin{itemize}
\item
manual and install package
\item
first version published on CTAN
\end{itemize}

%%%%%%%%%%%%%%%%%%%%%%%%%%%%%%%%%%%%%%%%
\paragraph{v0.6:} 2017/04/26

\begin{itemize}
\item
redirection mechanism added
\end{itemize}

%%%%%%%%%%%%%%%%%%%%%%%%%%%%%%%%%%%%%%%%
\paragraph{v0.5:} 2017/04/26

\begin{itemize}
\item
functionality in definition file
\end{itemize}


%%%%%%%%%%%%%%%%%%%%%%%%%%%%%%%%%%%%%%%%%%%%%%%%%%%%%%%%%%%%%%%%%%%%%%%%%%%%%%%%
%%%%%%%%%%%%%%%%%%%%%%%%%%%%%%%%%%%%%%%%%%%%%%%%%%%%%%%%%%%%%%%%%%%%%%%%%%%%%%%%
%%%%%%%%%%%%%%%%%%%%%%%%%%%%%%%%%%%%%%%%%%%%%%%%%%%%%%%%%%%%%%%%%%%%%%%%%%%%%%%%
\appendix

\settowidth\MacroIndent{\rmfamily\scriptsize 000\ }

 \DocInput{childdoc.dtx}

\end{document}
%</driver>
% \fi
%
% %%%%%%%%%%%%%%%%%%%%%%%%%%%%%%%%%%%%%%%%%%%%%%%%%%%%%%%%%%%%%%%%%%%%%%%%%%%%%%
% %%%%%%%%%%%%%%%%%%%%%%%%%%%%%%%%%%%%%%%%%%%%%%%%%%%%%%%%%%%%%%%%%%%%%%%%%%%%%%
% \section{Sample}
%\iffalse
%<*samplemain>
%\fi
%
% The following presents a sample document
% with two chapters, two parts, a title page,
% a compile flag as well as three forwarding files to set the flag.
% It consists of eight |.tex| files:
% \begin{center}
% \begin{tabular}{ll}
% |cdocsamp.tex|&main file\\
% |cdocsch1.tex|&include file for chapter 1\\
% |cdocsch2.tex|&include file for chapter 2\\
% |cdocspt3.tex|&include file for part 3\\
% |cdocspt4.tex|&include file for part 4\\
% |cdocsdrf.tex|&forwarding file for main file in draft mode\\
% |cdocsfi1.tex|&forwarding file for final version of chapter 1\\
% |cdocsfi2.tex|&forwarding file for final version of chapter 2\\
% \end{tabular}
% \end{center}
% Each of the eight files can be compiled directly by the \LaTeX{} compiler.
%
% %%%%%%%%%%%%%%%%%%%%%%%%%%%%%%%%%%%%%%
% \paragraph{Main File.}
%
% The main file is called |cdocsamp.tex|.
%
% Load the \textsf{childdoc} definitions and
% declare the filename for the main document:
%    \begin{macrocode}
\input{childdoc.def}
\childdocmain{}
%    \end{macrocode}

% Optional override for |\version| flag:
%    \begin{macrocode}
%%\ifchilddoc\else\providecommand{\version}{draft}\fi
%    \end{macrocode}

% Define the default values for the |\version| flag
% (|final| for the main file and |draft| for childs):
%    \begin{macrocode}
\ifchilddoc
\providecommand{\version}{draft}
\else
\providecommand{\version}{final}
\fi
%    \end{macrocode}

% Load the standard document class:
%    \begin{macrocode}
\documentclass[12pt]{article}
%    \end{macrocode}

% Start the document body:
%    \begin{macrocode}
\begin{document}
%    \end{macrocode}

% Declare a title page.
% Print title, part of document being processed and version flag:
%    \begin{macrocode}
\addtocounter{page}{-1}
\begin{center}
{\LARGE\bfseries{}childdoc example\par}
\vspace{1cm}
\ifchilddoc
\ifchilddocmanual part\else chapter\fi:
`\childdocname' of `\childdocjob'\par
\else
main document: `\childdocjob'\par
\fi
version: \version\par
\end{center}
\newpage
%    \end{macrocode}

% Manually include selected file,
% otherwise process as usual:
%    \begin{macrocode}
\ifchilddocmanual
\section*{part `\childdocname'}
\input{\childdocname}
\else
%    \end{macrocode}

% Include the two chapters:
%    \begin{macrocode}
\include{cdocsch1}
\include{cdocsch2}
%    \end{macrocode}

% Include the two parts unless only chapters should be displayed:
%    \begin{macrocode}
\ifchilddoc\else
\section{part three}
\input{cdocspt3}
\section{part four}
\input{cdocspt4}
\fi
%    \end{macrocode}

% Process as usual until here:
%    \begin{macrocode}
\fi
%    \end{macrocode}

% End of document body:
%    \begin{macrocode}
\end{document}
%    \end{macrocode}
%\iffalse
%</samplemain>
%\fi
%
% %%%%%%%%%%%%%%%%%%%%%%%%%%%%%%%%%%%%%%
% \paragraph{Chapter Include Files.}
%
% The include files are called |cdocsch1.tex| and |cdocsch2.tex|.
%
%\iffalse
%<*samplechap1|samplechap2>
%\fi

% Optional override for |\version| flag:
%    \begin{macrocode}
%%\providecommand{\version}{final}
%    \end{macrocode}

% Include the main document:
%    \begin{macrocode}
\input{childdoc.def}
\childdocof{cdocsamp}
%    \end{macrocode}

%\iffalse
%</samplechap1|samplechap2>
%\fi
%
%\iffalse
%<*samplechap1>
%\fi
% Some text for chapter 1:
%    \begin{macrocode}
\section{one}
some text in chapter one
%    \end{macrocode}

%\iffalse
%</samplechap1>
%\fi
% Some text for chapter 2:
%\iffalse
%<*samplechap2>
%\fi
%    \begin{macrocode}
\section{two}
more text in chapter two
%    \end{macrocode}

%\iffalse
%</samplechap2>
%\fi
%
% %%%%%%%%%%%%%%%%%%%%%%%%%%%%%%%%%%%%%%
% \paragraph{Part Include Files.}
%
% The include files are called |cdocspt3.tex| and |cdocspt4.tex|.
%
%\iffalse
%<*samplepart3|samplepart4>
%\fi

% Optional override for |\version| flag:
%    \begin{macrocode}
%%\providecommand{\version}{final}
%    \end{macrocode}

% Include the main document:
%    \begin{macrocode}
\input{childdoc.def}
\childdocby{cdocsamp}
%    \end{macrocode}

%\iffalse
%</samplepart3|samplepart4>
%\fi
%
%\iffalse
%<*samplepart3>
%\fi
% Some text for part 3:
%    \begin{macrocode}
some text in part three
%    \end{macrocode}

%\iffalse
%</samplepart3>
%\fi
% Some text for part 4:
%\iffalse
%<*samplepart4>
%\fi
%    \begin{macrocode}
more text in part four
%    \end{macrocode}

%\iffalse
%</samplepart4>
%\fi
%
% %%%%%%%%%%%%%%%%%%%%%%%%%%%%%%%%%%%%%%
% \paragraph{Forwarding for a Complete Draft.}
%
% The following forwarding file |cdocsdrf.tex|
% compiles the main document in draft mode:
%\iffalse
%<*sampledraft>
%\fi
%    \begin{macrocode}
\def\version{draft}
\input{childdoc.def}
\childdocforward{cdocsamp}
%    \end{macrocode}

%\iffalse
%</sampledraft>
%\fi
%
% %%%%%%%%%%%%%%%%%%%%%%%%%%%%%%%%%%%%%%
% \paragraph{Forwarding for Final Version of the Chapters.}
%
% The following forwarding files |cdocsfn1.tex| and |cdocsfn2.tex|
% (with identical content)
% compile the final versions of the child documents
% |cdocsch1.tex| and |cdocsch2.tex|, respectively:
%\iffalse
%<*samplefinal>
%\fi
%    \begin{macrocode}
\def\version{final}
\input{childdoc.def}
\childdocforwardprefix[cdocsamp]{cdocsfn}{cdocsch}
%    \end{macrocode}

%\iffalse
%</samplefinal>
%\fi
%
% %%%%%%%%%%%%%%%%%%%%%%%%%%%%%%%%%%%%%%
% \paragraph{Command Line Processing.}
%
% The following three command lines generate the output files
% |cdocscld|, |cdocscl1| and |cdocscl2|
% which should be identical to
% |cdocsdrf|, |cdocsch1| and |cdocsfn2|, respectively:
% \begin{center}
% \begin{tabular}{l}
% |latex -jobname cdocscld \|\\
% |  "\def\version{draft}\input{childdoc.def}\childdocforward{cdocsamp}"|\\
% |latex -jobname cdocscl1 \|\\
% |  "\input{childdoc.def}\childdocforward[cdocsamp]{cdocsch1}"|\\
% |latex -jobname cdocscl2 \|\\
% |  "\def\version{final}\input{childdoc.def}\childdocforward{cdocsch2}"|
% \end{tabular}
% \end{center}
% Note that the trailing backslash on each first line
% merely continues the input to the second line
% (for convenient cut ant paste).
% Furthermore, the command |latex| can be replaced by any
% of its alternative versions such as |pdflatex|.
%
% %%%%%%%%%%%%%%%%%%%%%%%%%%%%%%%%%%%%%%%%%%%%%%%%%%%%%%%%%%%%%%%%%%%%%%%%%%%%%%
% %%%%%%%%%%%%%%%%%%%%%%%%%%%%%%%%%%%%%%%%%%%%%%%%%%%%%%%%%%%%%%%%%%%%%%%%%%%%%%
% \section{Implementation}
%\iffalse
%<*package>
%\fi
%
% This section describes the definitions file |childdoc.def|.

% The definitions cannot be loaded using |\usepackage| or |\RequirePackage|
% which has a mechanism to prevent loading a style file more than once.
% When loading the definitions by means of |\input|
% multiple instances have to be prevented manually:
%\iffalse
%This code needs to be before the `\ProvidesFile' directive
%which is defined at the beginning of this file.
%Therefore it is also placed there and commented out here.
%</package>
%<*discard>
%\fi
%    \begin{macrocode}
\ifdefined\childdocmain\endinput\fi
%    \end{macrocode}
%\iffalse
%</discard>
%<*package>
%\fi
%
% \macro{\ifchilddoc}
% \macro{\ifchilddocmanual}
% The conditional |\ifchilddoc| tells whether a
% child (true) or main (false) document is being compiled.
% The conditional |\ifchilddocmanual| tells whether
% the |\includeonly| mechanism is used (false) or
% the selection of child files must be performed manually (true).
% The definitions initialise to false:
%    \begin{macrocode}
\newif\ifchilddoc
\newif\ifchilddocmanual
%    \end{macrocode}

% \macro{\childdocname}
% \macro{\childdocjob}
% The macro |\childdocname| stores the name of the main document
% to be compiled. The macro |\childdocjob| stores the name of
% the document on which the \LaTeX{} compiler was originally invoked.
% The content of |\jobname| cannot be compared
% to filenames specified in the source due to different catcodes.
% The following code rescans |\jobname|, stores the result
% in |\childdocname| and saves a copy in |\childdocjob|:
%    \begin{macrocode}
\edef\childdocname{\scantokens\expandafter{\jobname\noexpand}}
\let\childdocjob\childdocname
%    \end{macrocode}

% \macro{\childdocdisable}
% The macro |\childdocdisable| prevents the main file
% from being processed more than once.
% At this stage, the main document command |\childdocmain|
% is assumed to be called once again where it should do nothing.
% Any subsequent call to it should prevent
% a secondary processing of the main document
% It overwrites the forwarding commands
% |\childdocof| and |\childdocforward|
% with empty macros to prevent further inclusions of the main document:
%    \begin{macrocode}
\newcommand{\childdocdisable}
{
  \renewcommand{\childdocmain}[1]{\renewcommand{\childdocmain}[1]{\endinput}}
  \renewcommand{\childdocof}[1]{}
  \renewcommand{\childdocby}[2][]{}
  \renewcommand{\childdocforward}[2][]{}
  \renewcommand{\childdocdisable}{}
}
%    \end{macrocode}

% \macro{\childdocmain}
% The macro |\childdocmain| is to be called at the top of the main file
% with nothing or the main filename (without extension) as argument.
% First, it breaks loops.
% If the argument is not empty and does not match |\childdocname|
% (which is set by the first inclusion of |childdoc.def|),
% |\ifchilddoc| is set to true, |\includeonly| is applied to the child file
% and |\jobname| is set to the main file
% (for proper handling of |.aux| files):
%    \begin{macrocode}
\newcommand{\childdocmain}[1]
{
  \childdocdisable\childdocmain{}
  \if?#1?\else
    \begingroup
      \def\childdoctmp{#1}
      \ifx\childdoctmp\childdocname
        \def\childdoctmp{}
      \else
        \def\childdoctmp
        {
          \childdoctrue
          \includeonly{\childdocname}
          \def\childdocjob{#1}
          \def\jobname{#1}
        }
      \fi
      \expandafter
    \endgroup
    \childdoctmp
  \fi
}
%    \end{macrocode}

% \macro{\childdocof}
% The command |\childdocof| redirects
% compilation to the main file |#1|.
%    \begin{macrocode}
\newcommand{\childdocof}[1]
{
  \childdocdisable
  \childdoctrue
  \includeonly{\childdocname}
  \def\jobname{#1}
  \def\childdocjob{#1}
  \input{#1}
}
%    \end{macrocode}

% \macro{\childdocby}
% The command |\childdocby| ....
%    \begin{macrocode}
\newcommand{\childdocby}[2][]
{
  \childdocdisable
  \childdoctrue
  \childdocmanualtrue
  \if?#1?\else
    \def\jobname{#2}
  \fi
  \def\childdocjob{#2}
  \input{#2}
  \endinput
}
%    \end{macrocode}

% \macro{\childdocforward}
% The command |\childdocforward| redirects
% compilation to the main file or
% (if the optional argument is given) a child file.
% Parameters are set as if the main file
% or a child file starting with |\childdocof| was compiled.
% Then compilation is handed over to the main file:
%    \begin{macrocode}
\newcommand{\childdocforward}[2][]
{
  \begingroup
    \if?#1?
      \def\childdoctmp
      {
        \def\childdocname{#2}
        \def\childdocjob{#2}
        \def\jobname{#2}
        \input{#2}
        \endinput
      }
    \else
      \def\childdoctmp
      {
        \childdocdisable
        \def\childdocname{#2}
        \childdoctrue
        \includeonly{#2}
        \def\childdocjob{#1}
        \def\jobname{#1}
        \input{#1}
        \endinput
      }
    \fi
    \expandafter
  \endgroup
  \childdoctmp
}
%    \end{macrocode}

% \macro{\childdocforwardprefix}
% The command |\childdocforwardprefix| redirects
% compilation to the main or a child file by means of a pattern.
% The prefix |#1| in the current filename is replaced by |#2|
% and the suffix of the current filename is kept
% (it is assumed that the filename does not contain the substring `|~~~|'
% which is used as a delimiter).
% Compilation is handed over to the new file by |\childdocforward|:
%    \begin{macrocode}
\newcommand{\childdocforwardprefix}[3][]
{
  \begingroup
    \def\childdocextract #2##1~~~{\def\childdoctmp{\childdocforward[#1]{#3##1}}}
    \expandafter\childdocextract\childdocname~~~
    \expandafter
  \endgroup
  \childdoctmp
}
%    \end{macrocode}

% \macro{\childdoc}
% The deprecated macro |\childdoc| is a legacy version of |\childdocmain|:
%    \begin{macrocode}
\newcommand{\childdoc}{\childdocmain}
%    \end{macrocode}

% \macro{\childdocredirect}
% The deprecated macro |\childdocredirect| is a legacy version
% of |\childdocforward| and |\childdocforwardprefix|:
%    \begin{macrocode}
\newcommand{\childdocredirect}[2][]
{
  \begingroup
    \if?#1?
      \def\childdoctmp{\childdocforward{#2}}
    \else
      \def\childdoctmp{\childdocforwardprefix{#1}{#2}}
    \fi
    \expandafter
  \endgroup
  \childdoctmp
}
%    \end{macrocode}

%\iffalse
%</package>
%\fi
%
\endinput
|\\
|\childdocforward[|\textit{main}|]{|\textit{dest}|}|\\
\end{tabular}
\end{center}
%
The argument \textit{dest} is the destination file
(without extension).
It should be the main file or one of the child files.
Note that further \textsf{childdoc} directives
such as |\childdocof| and |\childdocforward|
in the indicated file will be processed in this form.
The optional argument \textit{main}
passes on directly to the main file \textit{main}
while pretending to compile the child \textit{dest}.
This form behaves as if \textit{dest}
issues |\childdocof{|\textit{main}|}| right away,
and no further \textsf{childdoc} directives will be processed.

%%%%%%%%%%%%%%%%%%%%%%%%%%%%%%%%%%%%%%%%
\DescribeMacro{\...prefix}
In the alternative form |\childdocforwardprefix|,
%
\begin{center}
\begin{tabular}{l}
|% \iffalse
%
% childdoc.dtx Copyright (C) 2017-2018 Niklas Beisert
%
% This work may be distributed and/or modified under the
% conditions of the LaTeX Project Public License, either version 1.3
% of this license or (at your option) any later version.
% The latest version of this license is in
%   http://www.latex-project.org/lppl.txt
% and version 1.3 or later is part of all distributions of LaTeX
% version 2005/12/01 or later.
%
% This work has the LPPL maintenance status `maintained'.
%
% The Current Maintainer of this work is Niklas Beisert.
%
% This work consists of the files childdoc.dtx and childdoc.ins
% and the derived files childdoc.def and cdocsamp.tex with
% cdocsch1.tex, cdocsch2.tex, cdocsdrf.tex, cdocsfn1.tex, cdocsfn2.tex.
%
%<package>\ifdefined\childdocmain\endinput\fi
%<package>\ProvidesFile{childdoc.def}[2018/12/30 v2.0 child document driver]
%<samplemain>\ProvidesFile{cdocsamp.tex}[2018/12/30 v2.0 sample for childdoc]
%<*driver>
%\ProvidesFile{childdoc.drv}[2018/12/30 v2.0 childdoc reference manual file]
\PassOptionsToClass{10pt,a4paper}{article}
\documentclass{ltxdoc}

\usepackage[margin=35mm]{geometry}
\usepackage{hyperref}
\usepackage{hyperxmp}
\usepackage[usenames]{color}

\hypersetup{colorlinks=true}
\hypersetup{pdfstartview=FitH}
\hypersetup{pdfpagemode=UseNone}
\hypersetup{pdfsource={}}
\hypersetup{pdflang={en-UK}}
\hypersetup{pdfcopyright={Copyright 2017-2018 Niklas Beisert.
  This work may be distributed and/or modified under the
  conditions of the LaTeX Project Public License, either version 1.3
  of this license or (at your option) any later version.}}
\hypersetup{pdflicenseurl={http://www.latex-project.org/lppl.txt}}
\hypersetup{pdfcontactaddress={ETH Zurich, ITP, HIT K,
  Wolfgang-Pauli-Strasse 27}}
\hypersetup{pdfcontactpostcode={8093}}
\hypersetup{pdfcontactcity={Zurich}}
\hypersetup{pdfcontactcountry={Switzerland}}
\hypersetup{pdfcontactemail={nbeisert@itp.phys.ethz.ch}}
\hypersetup{pdfcontacturl={http://people.phys.ethz.ch/\xmptilde nbeisert/}}

\newcommand{\secref}[1]{\hyperref[#1]{section \ref*{#1}}}

\parskip1ex
\parindent0pt
\let\olditemize\itemize
\def\itemize{\olditemize\parskip0pt}

\begin{document}

\title{The \textsf{childdoc} Package}
\hypersetup{pdftitle={The childdoc Package}}
\author{Niklas Beisert\\[2ex]
  Institut f\"ur Theoretische Physik\\
  Eidgen\"ossische Technische Hochschule Z\"urich\\
  Wolfgang-Pauli-Strasse 27, 8093 Z\"urich, Switzerland\\[1ex]
  \href{mailto:nbeisert@itp.phys.ethz.ch}
  {\texttt{nbeisert@itp.phys.ethz.ch}}}
\hypersetup{pdfauthor={Niklas Beisert}}
\hypersetup{pdfsubject={Manual for the LaTeX2e Package childdoc}}
\date{30 December 2018, \textsf{v2.0}}
\maketitle

\begin{abstract}\noindent
\textsf{childdoc} is a \LaTeXe{} package
that enables the direct compilation
of document sections included by |\include|
to individual files.
\end{abstract}

\begingroup
\parskip0ex
\tableofcontents
\endgroup

%%%%%%%%%%%%%%%%%%%%%%%%%%%%%%%%%%%%%%%%%%%%%%%%%%%%%%%%%%%%%%%%%%%%%%%%%%%%%%%%
%%%%%%%%%%%%%%%%%%%%%%%%%%%%%%%%%%%%%%%%%%%%%%%%%%%%%%%%%%%%%%%%%%%%%%%%%%%%%%%%
\section{Introduction}

\LaTeX{} provides a mechanism to structure a large document (such as a book)
into a main file and several child files (containing the chapters)
using the |\include| command.
This mechanism is beneficial for documents
which span hundreds of pages in order to
make the source file(s) more manageable.
Moreover, compilation can be restricted to
selected child files by means of the |\includeonly| command.
The latter feature can be used to reduce the compilation time while editing
(this was significantly more useful in the earlier days of \LaTeX{})
or to generate a smaller document which is easier to navigate.
Another application of |\includeonly| is to generate
documents consisting of selected parts of the complete document.

However, there are a few drawbacks of the plain |\include| mechanism:
\begin{itemize}
\item
The child files cannot be compiled on their own,
they can only be compiled via the main file.
A naive editing environment
(such as a text editor with an option
to have the current file processed by \LaTeX)
may require one to switch to the main file before compiling;
attempting to compile the child file produces errors.
\item
The main file must be modified (each time)
to adjust the |\includeonly| command
to the present needs. This easily leaves the main file in a messy state.
\item
The generated document will always carry the filename
of the main document. This is inconvenient if
several child files are to be compiled and
to be kept for distribution.
\end{itemize}

The present package provides a simple interface
to make child files individually compilable by \LaTeX{}.
Compiling a child file then has the same effect as compiling
the main file with an |\includeonly| command
to select the appropriate child.
Moreover the generated document will carry the name of the child
rather than the main file.
This resolves all three above issues.

This feature is meant to make the editing of books,
thesis documents and lecture notes somewhat more convenient.
However, the package can also be used efficiently for
composing a series of documents (such as exercise sheets)
which are typically distributed individually.
It then assists the author in generating the individual documents
(potentially in different versions)
as well as a document containing the collected series.
Another application is in developing style files
or other kinds of included material
where compilation of the style file could redirect
to a sample or test file.

%%%%%%%%%%%%%%%%%%%%%%%%%%%%%%%%%%%%%%%%%%%%%%%%%%%%%%%%%%%%%%%%%%%%%%%%%%%%%%%%
%%%%%%%%%%%%%%%%%%%%%%%%%%%%%%%%%%%%%%%%%%%%%%%%%%%%%%%%%%%%%%%%%%%%%%%%%%%%%%%%
\section{Usage}

First of all, the package \textsf{childdoc} is \emph{not} a standard
\LaTeXe{} |.sty| style file! Therefore it needs to be invoked in
a non-standard way.

%%%%%%%%%%%%%%%%%%%%%%%%%%%%%%%%%%%%%%%%%%%%%%%%%%%%%%%%%%%%%%%%%%%%%%%%%%%%%%%%
\subsection{Included Files}
\label{sec:include}

%%%%%%%%%%%%%%%%%%%%%%%%%%%%%%%%%%%%%%%%
\DescribeMacro{\childdocmain}
To use the package, add the commands
\begin{center}
\begin{tabular}{l}
|\input{childdoc.def}|\\
|\childdocmain{}|\\
\end{tabular}
\end{center}
at the very top of the main \LaTeX{} file,
in particular \emph{before} the |\documentclass| statement!
The argument of |\childdocmain| should be left empty
(but it must be present).

%%%%%%%%%%%%%%%%%%%%%%%%%%%%%%%%%%%%%%%%
\DescribeMacro{\childdocof}
Furthermore, add the commands
\begin{center}
\begin{tabular}{l}
|\input{childdoc.def}|\\
|\childdocof{|\textit{main}|}|\\
\end{tabular}
\end{center}
at the top of every child file \textit{child}
which is included by |\include{|\textit{child}|}|
from within the main file
(or at least for those files to be compiled individually).
The argument \textit{main} must be the filename of the main file.

There are a couple of
considerations in setting up the main and child documents:

%%%%%%%%%%%%%%%%%%%%%%%%%%%%%%%%%%%%%%%%
\paragraph{Restrictions.}

Please note the following restrictions:
\begin{itemize}
\item
|\childdocmain| must be called with one argument \textit{main}
to ensure compatibility with earlier version of the package.
It must either be empty (|\childdocmain{}|)
or precisely match the filename of the main file in which it is specified.
See \secref{sec:detection} for further information.
\item
The filename \textit{main} must be specified without the |.tex| extension.
\item
The filename \textit{main} is case sensitive
(even in case-insensitive file systems)
due to internal string comparison.
\item
The argument \textit{main} should be fully expanded, it cannot be a macro.
\item
Subdirectories and special characters should be avoided in filenames.
\item
The command |\childdocmain{|\textit{main}|}| must be followed by a whitespace.
It should not be followed immediately by another command
or by a comment mark `|%|'.
This is because the \TeX{} parser reads the token immediately following
the argument of |\childdocmain| and puts it
at the beginning of every child section;
however, a white\-space is ignored.
\end{itemize}

%%%%%%%%%%%%%%%%%%%%%%%%%%%%%%%%%%%%%%%%
\paragraph{Content of Main File.}

It is advisable to place all content in the child files included by |\include|.
Any output contained in the main file will appear in all child documents
unless suppressed manually;
it cannot be suppressed automatically by the |\includeonly| directive
and thus should normally be avoided.
A method to include some content in the main file
by means of conditional processing is described in \secref{sec:conditional}.

%%%%%%%%%%%%%%%%%%%%%%%%%%%%%%%%%%%%%%%%
\paragraph{Page Numbering.}

When only a part of the document is compiled,
the appropriate numbering of pages
(as well as other status parameters)
is determined from the |.aux| files.
The latter contain information from previous passes.
However this information needs to propagate through
all intermediate child documents.
Therefore the page numbering in child documents may well
be inconsistent until the complete document is compiled at least once.

A useful (if unconventional) way to always ensure a consistent
page numbering is to restart the numbering in each child document
and denote the pages by `\textit{child}|.|\textit{page}'
where \textit{child} represents the chapter/section number of the child file.
This can be achieved by the command
|\numberwithin{page}{|\textit{child}|}|
of the \textsf{amsmath} package
where \textit{child} can be |chapter| or |section|
depending on the chosen structuring.
Alternatively, one can modify the macro |\thepage| appropriately
and reset the counter |page| at the start of each child file.

%%%%%%%%%%%%%%%%%%%%%%%%%%%%%%%%%%%%%%%%%%%%%%%%%%%%%%%%%%%%%%%%%%%%%%%%%%%%%%%%
\subsection{Conditional Processing}
\label{sec:conditional}

The package provides a mechanism to compile different versions
of a document. To customise the versions further some conditional processing
can come in handy to distinguish which version is being compiled.
The package provides two macros to describe the compilation context:

%%%%%%%%%%%%%%%%%%%%%%%%%%%%%%%%%%%%%%%%
\DescribeMacro{\ifchilddoc}
The conditional |\ifchilddoc| distinguishes between the compilation of
child documents and the main document:
%
\begin{center}
|\ifchilddoc |\textit{child-code}| |[|\||else |\textit{main-code}]| \||fi|
\end{center}

%%%%%%%%%%%%%%%%%%%%%%%%%%%%%%%%%%%%%%%%
\DescribeMacro{\childdocname}
\DescribeMacro{\childdocjob}
The macro |\childdocname| contains the filename (without extension)
of the main or child file being processed.
Note that |\childdocjob| will always contain the name of the main file.

%%%%%%%%%%%%%%%%%%%%%%%%%%%%%%%%%%%%%%%%
\paragraph{Title Page.}

Conditional processing can be used to include a title or banner page
in the main document when proper precautions are taken.
Importantly, the code in the main file should ensure that the page counter
(as well as other status parameters which are stored in the |.aux| files)
takes the same value after the conditional processing.
Otherwise the page numbers may take divergent values
depending on which part is compiled.

For example, a title page could be declared by:
%
\begin{center}
\begin{tabular}{l}
|\ifchilddoc\||else|\\
|\addtocounter{page}{-1}|\\
\textit{code for title page}\\
|\newpage|\\
|\||fi|
\end{tabular}
\end{center}
%
A banner page for the child documents can be generated by:
%
\begin{center}
\begin{tabular}{l}
|\ifchilddoc|\\
|\addtocounter{page}{-1}|\\
\textit{code for banner page}\\
|\newpage|\\
|\||fi|
\end{tabular}
\end{center}
%
Here one could write a message such as:
\begin{center}
|This is the part \childdocname{} of \childdocjob{}.|
\end{center}

%%%%%%%%%%%%%%%%%%%%%%%%%%%%%%%%%%%%%%%%%%%%%%%%%%%%%%%%%%%%%%%%%%%%%%%%%%%%%%%%
\subsection{Flags}
\label{sec:flags}

The package makes it easy to generate different versions
of the main or child documents.
To this end compilation flags can be defined
and assigned different default values.
They will be particularly useful in conjunction
with the forwarding mechanism described in \secref{sec:forward}.

For example, it may be useful to have a flag |\version|
which can be set to |draft| or |final|.
The document source will contain some conditional code
depending on the value of |\version|.
Suppose further, the flag should default to |final| for the main file
and to |draft| for child files
which is a natural assignment for editing the document.
This is achieved by placing the following code
in the preamble of the main document
(below the |\childdocmain| directive):
%
\begin{center}
\begin{tabular}{l}
|\ifchilddoc|\\
|\providecommand{\version}{draft}|\\
|\||else|\\
|\providecommand{\version}{final}|\\
|\||fi|
\end{tabular}
\end{center}
%
The definition by |\providecommand| makes sure
that previous definitions are not overwritten.
Further statements |\providecommand{\version}{...}|
can thus be added before the above code to override it.

For the main file, one might add a line
(between |\childdocmain| and the above block)
%
\begin{center}
|%\ifchilddoc\||else\providecommand{\version}{draft}\||fi|
\end{center}
%
which can be uncommented to produce a draft version.
Likewise one can add a line to the very top of a child file
(above the |\childdocof{|\textit{main}|}| directive)
%
\begin{center}
|%\providecommand{\version}{final}|
\end{center}
%
which can be uncommented to produce the final version of this child document.

%%%%%%%%%%%%%%%%%%%%%%%%%%%%%%%%%%%%%%%%%%%%%%%%%%%%%%%%%%%%%%%%%%%%%%%%%%%%%%%%
\subsection{Forwarding}
\label{sec:forward}

Different versions of the main or child documents
using compilation flags as described in \secref{sec:flags}
can be (permanently) stored in different files
for convenient compilation, viewing and distribution.
To this end, the package defines a command
to pass on compilation to a different file:

%%%%%%%%%%%%%%%%%%%%%%%%%%%%%%%%%%%%%%%%
\DescribeMacro{\childdocforward}
The command |\childdocforward| redirects processing to
another source file:
%
\begin{center}
\begin{tabular}{l}
|\input{childdoc.def}|\\
|\childdocforward[|\textit{main}|]{|\textit{dest}|}|\\
\end{tabular}
\end{center}
%
The argument \textit{dest} is the destination file
(without extension).
It should be the main file or one of the child files.
Note that further \textsf{childdoc} directives
such as |\childdocof| and |\childdocforward|
in the indicated file will be processed in this form.
The optional argument \textit{main}
passes on directly to the main file \textit{main}
while pretending to compile the child \textit{dest}.
This form behaves as if \textit{dest}
issues |\childdocof{|\textit{main}|}| right away,
and no further \textsf{childdoc} directives will be processed.

%%%%%%%%%%%%%%%%%%%%%%%%%%%%%%%%%%%%%%%%
\DescribeMacro{\...prefix}
In the alternative form |\childdocforwardprefix|,
%
\begin{center}
\begin{tabular}{l}
|\input{childdoc.def}|\\
|\childdocforwardprefix[|\textit{main}|]{|\textit{prefix}|}{|\textit{dest}|}|
\end{tabular}
\end{center}
%
the destination file is determined by a pattern
depending on the current file:
To make this work, the current file must be called
`{\textit{prefix}\hspace{0.2em}\textit{suffix}}'
with \textit{prefix} matching precisely the argument.
Processing is then passed on to the file
`{\textit{dest}\hspace{0.2em}\textit{suffix}}'.
Surely, the same effect is achieved by
directly specifying the
argument `{\textit{dest}\hspace{0.2em}\textit{suffix}}'
in the first form.
However, that requires to set up a different file
for each child. With the alternative form of the command
all these files can have exactly the same content
which simplifies setting them up and maintaining them.

For example, the following file |draft.tex|
with a compilation flag |\version| as described in \secref{sec:flags}
compiles the main document as a draft:
%
\begin{center}
\begin{tabular}{l}
|\def\version{draft}|\\
|\input{childdoc.def}|\\
|\childdocforward{|\textit{main}|}|
\end{tabular}
\end{center}
%
Likewise, the following files |final|\textit{nn}|.tex|
compile the final version of the child document
|child|\textit{nn}|.tex|:
%
\begin{center}
\begin{tabular}{l}
|\def\version{final}|\\
|\input{childdoc.def}|\\
|\childdocforwardprefix{final}{child}|
\end{tabular}
\end{center}
%

Note that when several versions of a main file and/or of each child file
are to be generated, it may be convenient to set up a |Makefile| or
shell script to automatise the process.

%%%%%%%%%%%%%%%%%%%%%%%%%%%%%%%%%%%%%%%%%%%%%%%%%%%%%%%%%%%%%%%%%%%%%%%%%%%%%%%%
\subsection{Command Line Processing}
\label{sec:commandline}

The effect of redirection files can also be achieved by invoking
the \LaTeX{} compiler with a more elaborate command line.
Most conveniently this should be done as part
of a shell script or a |Makefile|.

When using \textsf{childdoc} in the main file, the following
command lines effectively perform a redirection
(note that depending on the shell being used,
backslashes may have to be doubled: `|\|' $\to$ `|\\|'):
%
\begin{center}
|... -jobname "|\textit{target}|" |\\|"|[\textit{flags}]%
|\input{childdoc.def}\childdocforward[|\textit{main}|]{|\textit{dest}|}"|
\end{center}
%
Here \textit{target} is the name of the output file,
\textit{main} is the name of the main file
and \textit{dest} is the name of the main or child file to be processed
(all filenames without extensions).
The optional argument \textit{main} can be omitted
if \textit{main} matches \textit{dest}.
Optionally, compilation \textit{flags} can be defined via |\def| commands.
This command line makes the \TeX{} engine believe
it is compiling the file \textit{target}
whose content is specified as the latter parameter.
The provided code then forwards the processing to
\textit{main} or \textit{dest} as described in \secref{sec:forward}.

%%%%%%%%%%%%%%%%%%%%%%%%%%%%%%%%%%%%%%%%%%%%%%%%%%%%%%%%%%%%%%%%%%%%%%%%%%%%%%%%
\subsection{Include by Input}
\label{sec:input}

Including child documents by |\include| has some restrictions by design.
Most notably, the content of a child document always occupies
its own set of pages; pages cannot be shared between child documents.
Usually, this behaviour makes perfect sense
because each child document contain an essential part of the document.
However, in some situations it may be desirable to compose
a document from a collection of parts
without having mandatory page breaks between then.
For this case, the package
provides a mechanism to include parts
by |\input| which can also be processed individually.
However, by construction this mechanism
requires manual handling of the content to be output.

%%%%%%%%%%%%%%%%%%%%%%%%%%%%%%%%%%%%%%%%
\DescribeMacro{\ifchilddocmanual}
The main file should be prepared as usual, see \secref{sec:include}.
However, the document body must make a distinction
between processing of an individual part and of the main document, e.g.:
%
\begin{center}
\begin{tabular}{l}
|\ifchilddocmanual|\\
|\input{\childdocname}|\\
|\||else|\\
\textit{document body with }|\input{|\textit{part}|}|\\
|\||fi|
\end{tabular}
\end{center}
%
The conditional |\ifchilddocmanual| is true whenever
a part to be included by |\input| is being compiled,
and the name of the part is stored in |\childdocname|.

%%%%%%%%%%%%%%%%%%%%%%%%%%%%%%%%%%%%%%%%
\DescribeMacro{\childdocby}
Each part to be included by |\input| should start with:
%
\begin{center}
\begin{tabular}{l}
|\input{childdoc.def}|\\
|\childdocby{|\textit{main}|}|\\
\end{tabular}
\end{center}
%
The directive |\childdocby| is similar to |\childdocof|
described in \secref{sec:include},
but the subsequent selection of content must be done manually.
To that end, both |\ifchilddoc| and |\ifchilddocmanual|
will be true upon processing of a part,
and the name of the part is stored in |\childdocname|.
Note that |\jobname| will be set to the filename of the current part
so that each part receives an individual |.aux| file
that does not interfere with the |.aux| file(s) of the main document.
This behaviour can be altered by the alternative form
|\childdocby[*]{|\textit{main}|}| (with a non-empty optional argument)
which uses the |.aux| file of the main document
by setting |\jobname| to \textit{main}.

%%%%%%%%%%%%%%%%%%%%%%%%%%%%%%%%%%%%%%%%%%%%%%%%%%%%%%%%%%%%%%%%%%%%%%%%%%%%%%%%
\subsection{Driver Development}
\label{sec:driver}

The \textsf{childdoc} mechanism can also be use for the development
of definition files such as \LaTeX{} styles or classes.
This case differs from the above setup with multiple parts
included by |\include| in that no |\includeonly| should be invoked.
This can be achieved by starting the include file
(before |\ProvidesPackage|) with:
%
\begin{center}
\begin{tabular}{l}
|\input{childdoc.def}|\\
|\childdocforward{|\textit{main}|}|\\
\end{tabular}
\end{center}
%
or alternatively with:
%
\begin{center}
\begin{tabular}{l}
|\input{childdoc.def}|\\
|\childdocby{|\textit{main}|}|\\
\end{tabular}
\end{center}
%
Both forms have slightly different effects as described above.
The main file is prepared as usual, see \secref{sec:include}.

%%%%%%%%%%%%%%%%%%%%%%%%%%%%%%%%%%%%%%%%%%%%%%%%%%%%%%%%%%%%%%%%%%%%%%%%%%%%%%%%
\subsection{Legacy Detection}
\label{sec:detection}

The directive |\childdocmain| in the main file can detect
whether the complete document or merely a child is to be compiled
even without using the directive |\childdocof|.
This method is deprecated because it is less robust
and there is no compelling reason to use it;
it is merely provided for backward compatibility
and it may be removed in future versions.

If the detection mechanism is to be used,
it is mandatory to correctly specify
the filename of the main file as the argument of |\childdocmain|:
%
\begin{center}
\begin{tabular}{l}
|\input{childdoc.def}|\\
|\childdocmain{|\textit{main}|}|\\
\end{tabular}
\end{center}
%
If |\jobname| does not match the argument \textit{main} of |\childdocmain|,
it is assumed that |\jobname| points to the child file to be compiled.
When using |\childdocmain| with the main file specified as argument,
it suffices to start a child file
with just |\input{|\textit{main}|}|
without loading of the package and using |\childdocof|.
If instead all processing is done
with the appropriate \textsf{childdoc} directives,
the argument of \textit{main} of |\childdocmain| can be empty.

An alternative version of the command line processing described
in \secref{sec:commandline} using the detection mechanism reads:
%
\begin{center}
|... -jobname "|\textit{target}|" "|[\textit{flags}]%
[|\def\jobname{|\textit{dest}|}|]|\input{|\textit{main}|}"|
\end{center}

%%%%%%%%%%%%%%%%%%%%%%%%%%%%%%%%%%%%%%%%%%%%%%%%%%%%%%%%%%%%%%%%%%%%%%%%%%%%%%%%
\subsection{Manual Code}
\label{sec:manual}

In case one cannot be certain whether the definitions file |childdoc.def|
is installed on the target \TeX{} distribution
and one prefers not to ship it,
it is conceivable to paste a few relevant commands into the sources.

To that end, drop all statements |\input{childdoc.def}|
and perform the replacements as outlined below.
Instead of |\childdocmain{|\textit{main}|}| add the following code
to the top of the main file:
%
\begin{center}
\begin{tabular}{l}
|\||ifdefined\childdocname\endinput\||fi\newif\ifchilddoc|\\
|\edef\childdocname{\scantokens\expandafter{\jobname\noexpand}}|\\
|\def\childdocmain{|\textit{main}|}\||ifx\childdocmain\childdocname\||else|\\
|\childdoctrue\includeonly{\childdocname}\let\jobname\childdocmain\||fi|\\
\end{tabular}
\end{center}
%
Instead of |\childdocof{|\textit{main}|}| just include the main file
at the top of each child file:
%
\begin{center}
|\input{|\textit{main}|}|
\end{center}
%
A simple redirection |\childdocforward{|\textit{dest}|}| is achieved by:
%
\begin{center}
|\def\jobname{|\textit{dest}|}\input{\jobname}|
\end{center}
%
The redirection with prefix
|\childdocforwardprefix[|\textit{prefix}|]{|\textit{dest}|}|
is accomplished by:
%
\begin{center}
\begin{tabular}{l}
|{\edef\jobname{\scantokens\expandafter{\jobname\noexpand}}|\\
|\def\redirectjob |\textit{prefix}|#1~~~{\gdef\jobname{|\textit{dest}|#1}}|\\
|\expandafter\redirectjob\jobname~~~}\input{\jobname}|
\end{tabular}
\end{center}

In an alternative approach,
child documents can be compiled by a specific command line
without additional code or specific definitions:
%
\begin{center}
|... -jobname "|\textit{target}|" "|[\textit{flags}]%
|\includeonly{|\textit{dest}|}\input{|\textit{main}|}"|
\end{center}
%

%%%%%%%%%%%%%%%%%%%%%%%%%%%%%%%%%%%%%%%%%%%%%%%%%%%%%%%%%%%%%%%%%%%%%%%%%%%%%%%%
%%%%%%%%%%%%%%%%%%%%%%%%%%%%%%%%%%%%%%%%%%%%%%%%%%%%%%%%%%%%%%%%%%%%%%%%%%%%%%%%
\section{Information}

%%%%%%%%%%%%%%%%%%%%%%%%%%%%%%%%%%%%%%%%%%%%%%%%%%%%%%%%%%%%%%%%%%%%%%%%%%%%%%%%
\subsection{Copyright}

Copyright \copyright{} 2017--2018 Niklas Beisert

This work may be distributed and/or modified under the
conditions of the \LaTeX{} Project Public License, either version 1.3
of this license or (at your option) any later version.
The latest version of this license is in
  \url{http://www.latex-project.org/lppl.txt}
and version 1.3 or later is part of all distributions of \LaTeX{}
version 2005/12/01 or later.

This work has the LPPL maintenance status `maintained'.

The Current Maintainer of this work is Niklas Beisert.

This work consists of the files |README.txt|, |childdoc.ins| and |childdoc.dtx|
as well as the derived files |childdoc.def|, |cdocsamp.tex|
with |cdocsch1.tex|, |cdocsch2.tex|, |cdocspt3.tex|, |cdocspt4.tex|,
|cdocsdrf.tex|, |cdocsfn1.tex|, |cdocsfn2.tex|
as well as |childdoc.pdf|.

%%%%%%%%%%%%%%%%%%%%%%%%%%%%%%%%%%%%%%%%%%%%%%%%%%%%%%%%%%%%%%%%%%%%%%%%%%%%%%%%
\subsection{Files and Installation}

The package consists of the files:
%
\begin{center}
\begin{tabular}{ll}
    |README.txt|   & readme file \\
    |childdoc.ins| & installation file \\
    |childdoc.dtx| & source file \\
    |childdoc.def| & definition file \\
    |cdocsamp.tex| & sample main file \\
    |cdocsch1.tex| & sample include file \\
    |cdocsch2.tex| & sample include file \\
    |cdocspt3.tex| & sample part file \\
    |cdocspt4.tex| & sample part file \\
    |cdocsdrf.tex| & sample redirection file \\
    |cdocsfn1.tex| & sample redirection file \\
    |cdocsfn2.tex| & sample redirection file \\
    |childdoc.pdf| & manual
\end{tabular}
\end{center}
%
The distribution consists of the files
|README.txt|, |childdoc.ins| and |childdoc.dtx|.
%
\begin{itemize}
\item
Run (pdf)\LaTeX{} on |childdoc.dtx|
to compile the manual |childdoc.pdf| (this file).
\item
Run \LaTeX{} on |childdoc.ins| to create the definitions file |childdoc.def|
and the sample |cdocsamp.tex| with include files
|cdocsch1.tex|, |cdocsch2.tex|, |cdocspt3.tex|, |cdocspt4.tex|,
|cdocsdrf.tex|, |cdocsfn1.tex|, |cdocsfn2.tex|.
Then copy the file |childdoc.def| to an appropriate directory of your \LaTeX{}
distribution, e.g.\ \textit{texmf-root}|/tex/latex/childdoc|.
\end{itemize}

%%%%%%%%%%%%%%%%%%%%%%%%%%%%%%%%%%%%%%%%%%%%%%%%%%%%%%%%%%%%%%%%%%%%%%%%%%%%%%%%
\subsection{Related CTAN Packages}

There are several other packages which offer a similar functionality:
%
\begin{itemize}
\item
The packages
\href{http://ctan.org/pkg/docmute}{\textsf{docmute}},
\href{http://ctan.org/pkg/includex}{\textsf{includex}} and
\href{http://ctan.org/pkg/standalone}{\textsf{standalone}}
provide commands to include only the document body of
a child file thus allowing both files to be compiled individually.
\item
The packages \href{http://ctan.org/pkg/subdocs}{\textsf{subdocs}}
and \href{http://ctan.org/pkg/subfiles}{\textsf{subfiles}}
provide structures in which the main and child documents can be
encapsulated and allowing them to be compiled individually.
The inclusion mechanism is different from the conventional |\include|.
\item
The package \href{http://ctan.org/pkg/combine}{\textsf{combine}}
is an elaborate solution to combine several documents into one.
\end{itemize}
%
See also the CTAN topic \href{http://ctan.org/topic/subdocs}{\textsf{subdocs}}
for further related packages.
The present package differs from the above solutions in that
a document structure constructed with the conventional |\include| mechanism
just needs two extra commands at the top of every file
such that all constituent files can be compiled individually.

%%%%%%%%%%%%%%%%%%%%%%%%%%%%%%%%%%%%%%%%%%%%%%%%%%%%%%%%%%%%%%%%%%%%%%%%%%%%%%%%
%\subsection{Feature Suggestions}
%
%The following is a list of features which may be useful for future
%versions of this package:
%%
%\begin{itemize}
%\item
%\ldots
%\end{itemize}

%%%%%%%%%%%%%%%%%%%%%%%%%%%%%%%%%%%%%%%%%%%%%%%%%%%%%%%%%%%%%%%%%%%%%%%%%%%%%%%%
\subsection{Revision History}

%%%%%%%%%%%%%%%%%%%%%%%%%%%%%%%%%%%%%%%%
\paragraph{v2.0:} 2018/12/30

\begin{itemize}
\item
immediate forward processing
\item
added |\childdocby| mechanism
\item
manual restructured
\end{itemize}

%%%%%%%%%%%%%%%%%%%%%%%%%%%%%%%%%%%%%%%%
\paragraph{v1.6:} 2018/01/17

\begin{itemize}
\item
application for development of include files
\item
corrections to manual
\end{itemize}

%%%%%%%%%%%%%%%%%%%%%%%%%%%%%%%%%%%%%%%%
\paragraph{v1.5:} 2017/05/21

\begin{itemize}
\item
more complete structuring introduced
\item
|\childdocof| introduced
\item
|\childdoc| renamed to |\childdocmain|
\item
|\childredirect| renamed to |\childdocforward| and |\childdocforwardprefix|
and functionality expanded
\end{itemize}

%%%%%%%%%%%%%%%%%%%%%%%%%%%%%%%%%%%%%%%%
\paragraph{v1.0:} 2017/04/27

\begin{itemize}
\item
manual and install package
\item
first version published on CTAN
\end{itemize}

%%%%%%%%%%%%%%%%%%%%%%%%%%%%%%%%%%%%%%%%
\paragraph{v0.6:} 2017/04/26

\begin{itemize}
\item
redirection mechanism added
\end{itemize}

%%%%%%%%%%%%%%%%%%%%%%%%%%%%%%%%%%%%%%%%
\paragraph{v0.5:} 2017/04/26

\begin{itemize}
\item
functionality in definition file
\end{itemize}


%%%%%%%%%%%%%%%%%%%%%%%%%%%%%%%%%%%%%%%%%%%%%%%%%%%%%%%%%%%%%%%%%%%%%%%%%%%%%%%%
%%%%%%%%%%%%%%%%%%%%%%%%%%%%%%%%%%%%%%%%%%%%%%%%%%%%%%%%%%%%%%%%%%%%%%%%%%%%%%%%
%%%%%%%%%%%%%%%%%%%%%%%%%%%%%%%%%%%%%%%%%%%%%%%%%%%%%%%%%%%%%%%%%%%%%%%%%%%%%%%%
\appendix

\settowidth\MacroIndent{\rmfamily\scriptsize 000\ }

 \DocInput{childdoc.dtx}

\end{document}
%</driver>
% \fi
%
% %%%%%%%%%%%%%%%%%%%%%%%%%%%%%%%%%%%%%%%%%%%%%%%%%%%%%%%%%%%%%%%%%%%%%%%%%%%%%%
% %%%%%%%%%%%%%%%%%%%%%%%%%%%%%%%%%%%%%%%%%%%%%%%%%%%%%%%%%%%%%%%%%%%%%%%%%%%%%%
% \section{Sample}
%\iffalse
%<*samplemain>
%\fi
%
% The following presents a sample document
% with two chapters, two parts, a title page,
% a compile flag as well as three forwarding files to set the flag.
% It consists of eight |.tex| files:
% \begin{center}
% \begin{tabular}{ll}
% |cdocsamp.tex|&main file\\
% |cdocsch1.tex|&include file for chapter 1\\
% |cdocsch2.tex|&include file for chapter 2\\
% |cdocspt3.tex|&include file for part 3\\
% |cdocspt4.tex|&include file for part 4\\
% |cdocsdrf.tex|&forwarding file for main file in draft mode\\
% |cdocsfi1.tex|&forwarding file for final version of chapter 1\\
% |cdocsfi2.tex|&forwarding file for final version of chapter 2\\
% \end{tabular}
% \end{center}
% Each of the eight files can be compiled directly by the \LaTeX{} compiler.
%
% %%%%%%%%%%%%%%%%%%%%%%%%%%%%%%%%%%%%%%
% \paragraph{Main File.}
%
% The main file is called |cdocsamp.tex|.
%
% Load the \textsf{childdoc} definitions and
% declare the filename for the main document:
%    \begin{macrocode}
\input{childdoc.def}
\childdocmain{}
%    \end{macrocode}

% Optional override for |\version| flag:
%    \begin{macrocode}
%%\ifchilddoc\else\providecommand{\version}{draft}\fi
%    \end{macrocode}

% Define the default values for the |\version| flag
% (|final| for the main file and |draft| for childs):
%    \begin{macrocode}
\ifchilddoc
\providecommand{\version}{draft}
\else
\providecommand{\version}{final}
\fi
%    \end{macrocode}

% Load the standard document class:
%    \begin{macrocode}
\documentclass[12pt]{article}
%    \end{macrocode}

% Start the document body:
%    \begin{macrocode}
\begin{document}
%    \end{macrocode}

% Declare a title page.
% Print title, part of document being processed and version flag:
%    \begin{macrocode}
\addtocounter{page}{-1}
\begin{center}
{\LARGE\bfseries{}childdoc example\par}
\vspace{1cm}
\ifchilddoc
\ifchilddocmanual part\else chapter\fi:
`\childdocname' of `\childdocjob'\par
\else
main document: `\childdocjob'\par
\fi
version: \version\par
\end{center}
\newpage
%    \end{macrocode}

% Manually include selected file,
% otherwise process as usual:
%    \begin{macrocode}
\ifchilddocmanual
\section*{part `\childdocname'}
\input{\childdocname}
\else
%    \end{macrocode}

% Include the two chapters:
%    \begin{macrocode}
\include{cdocsch1}
\include{cdocsch2}
%    \end{macrocode}

% Include the two parts unless only chapters should be displayed:
%    \begin{macrocode}
\ifchilddoc\else
\section{part three}
\input{cdocspt3}
\section{part four}
\input{cdocspt4}
\fi
%    \end{macrocode}

% Process as usual until here:
%    \begin{macrocode}
\fi
%    \end{macrocode}

% End of document body:
%    \begin{macrocode}
\end{document}
%    \end{macrocode}
%\iffalse
%</samplemain>
%\fi
%
% %%%%%%%%%%%%%%%%%%%%%%%%%%%%%%%%%%%%%%
% \paragraph{Chapter Include Files.}
%
% The include files are called |cdocsch1.tex| and |cdocsch2.tex|.
%
%\iffalse
%<*samplechap1|samplechap2>
%\fi

% Optional override for |\version| flag:
%    \begin{macrocode}
%%\providecommand{\version}{final}
%    \end{macrocode}

% Include the main document:
%    \begin{macrocode}
\input{childdoc.def}
\childdocof{cdocsamp}
%    \end{macrocode}

%\iffalse
%</samplechap1|samplechap2>
%\fi
%
%\iffalse
%<*samplechap1>
%\fi
% Some text for chapter 1:
%    \begin{macrocode}
\section{one}
some text in chapter one
%    \end{macrocode}

%\iffalse
%</samplechap1>
%\fi
% Some text for chapter 2:
%\iffalse
%<*samplechap2>
%\fi
%    \begin{macrocode}
\section{two}
more text in chapter two
%    \end{macrocode}

%\iffalse
%</samplechap2>
%\fi
%
% %%%%%%%%%%%%%%%%%%%%%%%%%%%%%%%%%%%%%%
% \paragraph{Part Include Files.}
%
% The include files are called |cdocspt3.tex| and |cdocspt4.tex|.
%
%\iffalse
%<*samplepart3|samplepart4>
%\fi

% Optional override for |\version| flag:
%    \begin{macrocode}
%%\providecommand{\version}{final}
%    \end{macrocode}

% Include the main document:
%    \begin{macrocode}
\input{childdoc.def}
\childdocby{cdocsamp}
%    \end{macrocode}

%\iffalse
%</samplepart3|samplepart4>
%\fi
%
%\iffalse
%<*samplepart3>
%\fi
% Some text for part 3:
%    \begin{macrocode}
some text in part three
%    \end{macrocode}

%\iffalse
%</samplepart3>
%\fi
% Some text for part 4:
%\iffalse
%<*samplepart4>
%\fi
%    \begin{macrocode}
more text in part four
%    \end{macrocode}

%\iffalse
%</samplepart4>
%\fi
%
% %%%%%%%%%%%%%%%%%%%%%%%%%%%%%%%%%%%%%%
% \paragraph{Forwarding for a Complete Draft.}
%
% The following forwarding file |cdocsdrf.tex|
% compiles the main document in draft mode:
%\iffalse
%<*sampledraft>
%\fi
%    \begin{macrocode}
\def\version{draft}
\input{childdoc.def}
\childdocforward{cdocsamp}
%    \end{macrocode}

%\iffalse
%</sampledraft>
%\fi
%
% %%%%%%%%%%%%%%%%%%%%%%%%%%%%%%%%%%%%%%
% \paragraph{Forwarding for Final Version of the Chapters.}
%
% The following forwarding files |cdocsfn1.tex| and |cdocsfn2.tex|
% (with identical content)
% compile the final versions of the child documents
% |cdocsch1.tex| and |cdocsch2.tex|, respectively:
%\iffalse
%<*samplefinal>
%\fi
%    \begin{macrocode}
\def\version{final}
\input{childdoc.def}
\childdocforwardprefix[cdocsamp]{cdocsfn}{cdocsch}
%    \end{macrocode}

%\iffalse
%</samplefinal>
%\fi
%
% %%%%%%%%%%%%%%%%%%%%%%%%%%%%%%%%%%%%%%
% \paragraph{Command Line Processing.}
%
% The following three command lines generate the output files
% |cdocscld|, |cdocscl1| and |cdocscl2|
% which should be identical to
% |cdocsdrf|, |cdocsch1| and |cdocsfn2|, respectively:
% \begin{center}
% \begin{tabular}{l}
% |latex -jobname cdocscld \|\\
% |  "\def\version{draft}\input{childdoc.def}\childdocforward{cdocsamp}"|\\
% |latex -jobname cdocscl1 \|\\
% |  "\input{childdoc.def}\childdocforward[cdocsamp]{cdocsch1}"|\\
% |latex -jobname cdocscl2 \|\\
% |  "\def\version{final}\input{childdoc.def}\childdocforward{cdocsch2}"|
% \end{tabular}
% \end{center}
% Note that the trailing backslash on each first line
% merely continues the input to the second line
% (for convenient cut ant paste).
% Furthermore, the command |latex| can be replaced by any
% of its alternative versions such as |pdflatex|.
%
% %%%%%%%%%%%%%%%%%%%%%%%%%%%%%%%%%%%%%%%%%%%%%%%%%%%%%%%%%%%%%%%%%%%%%%%%%%%%%%
% %%%%%%%%%%%%%%%%%%%%%%%%%%%%%%%%%%%%%%%%%%%%%%%%%%%%%%%%%%%%%%%%%%%%%%%%%%%%%%
% \section{Implementation}
%\iffalse
%<*package>
%\fi
%
% This section describes the definitions file |childdoc.def|.

% The definitions cannot be loaded using |\usepackage| or |\RequirePackage|
% which has a mechanism to prevent loading a style file more than once.
% When loading the definitions by means of |\input|
% multiple instances have to be prevented manually:
%\iffalse
%This code needs to be before the `\ProvidesFile' directive
%which is defined at the beginning of this file.
%Therefore it is also placed there and commented out here.
%</package>
%<*discard>
%\fi
%    \begin{macrocode}
\ifdefined\childdocmain\endinput\fi
%    \end{macrocode}
%\iffalse
%</discard>
%<*package>
%\fi
%
% \macro{\ifchilddoc}
% \macro{\ifchilddocmanual}
% The conditional |\ifchilddoc| tells whether a
% child (true) or main (false) document is being compiled.
% The conditional |\ifchilddocmanual| tells whether
% the |\includeonly| mechanism is used (false) or
% the selection of child files must be performed manually (true).
% The definitions initialise to false:
%    \begin{macrocode}
\newif\ifchilddoc
\newif\ifchilddocmanual
%    \end{macrocode}

% \macro{\childdocname}
% \macro{\childdocjob}
% The macro |\childdocname| stores the name of the main document
% to be compiled. The macro |\childdocjob| stores the name of
% the document on which the \LaTeX{} compiler was originally invoked.
% The content of |\jobname| cannot be compared
% to filenames specified in the source due to different catcodes.
% The following code rescans |\jobname|, stores the result
% in |\childdocname| and saves a copy in |\childdocjob|:
%    \begin{macrocode}
\edef\childdocname{\scantokens\expandafter{\jobname\noexpand}}
\let\childdocjob\childdocname
%    \end{macrocode}

% \macro{\childdocdisable}
% The macro |\childdocdisable| prevents the main file
% from being processed more than once.
% At this stage, the main document command |\childdocmain|
% is assumed to be called once again where it should do nothing.
% Any subsequent call to it should prevent
% a secondary processing of the main document
% It overwrites the forwarding commands
% |\childdocof| and |\childdocforward|
% with empty macros to prevent further inclusions of the main document:
%    \begin{macrocode}
\newcommand{\childdocdisable}
{
  \renewcommand{\childdocmain}[1]{\renewcommand{\childdocmain}[1]{\endinput}}
  \renewcommand{\childdocof}[1]{}
  \renewcommand{\childdocby}[2][]{}
  \renewcommand{\childdocforward}[2][]{}
  \renewcommand{\childdocdisable}{}
}
%    \end{macrocode}

% \macro{\childdocmain}
% The macro |\childdocmain| is to be called at the top of the main file
% with nothing or the main filename (without extension) as argument.
% First, it breaks loops.
% If the argument is not empty and does not match |\childdocname|
% (which is set by the first inclusion of |childdoc.def|),
% |\ifchilddoc| is set to true, |\includeonly| is applied to the child file
% and |\jobname| is set to the main file
% (for proper handling of |.aux| files):
%    \begin{macrocode}
\newcommand{\childdocmain}[1]
{
  \childdocdisable\childdocmain{}
  \if?#1?\else
    \begingroup
      \def\childdoctmp{#1}
      \ifx\childdoctmp\childdocname
        \def\childdoctmp{}
      \else
        \def\childdoctmp
        {
          \childdoctrue
          \includeonly{\childdocname}
          \def\childdocjob{#1}
          \def\jobname{#1}
        }
      \fi
      \expandafter
    \endgroup
    \childdoctmp
  \fi
}
%    \end{macrocode}

% \macro{\childdocof}
% The command |\childdocof| redirects
% compilation to the main file |#1|.
%    \begin{macrocode}
\newcommand{\childdocof}[1]
{
  \childdocdisable
  \childdoctrue
  \includeonly{\childdocname}
  \def\jobname{#1}
  \def\childdocjob{#1}
  \input{#1}
}
%    \end{macrocode}

% \macro{\childdocby}
% The command |\childdocby| ....
%    \begin{macrocode}
\newcommand{\childdocby}[2][]
{
  \childdocdisable
  \childdoctrue
  \childdocmanualtrue
  \if?#1?\else
    \def\jobname{#2}
  \fi
  \def\childdocjob{#2}
  \input{#2}
  \endinput
}
%    \end{macrocode}

% \macro{\childdocforward}
% The command |\childdocforward| redirects
% compilation to the main file or
% (if the optional argument is given) a child file.
% Parameters are set as if the main file
% or a child file starting with |\childdocof| was compiled.
% Then compilation is handed over to the main file:
%    \begin{macrocode}
\newcommand{\childdocforward}[2][]
{
  \begingroup
    \if?#1?
      \def\childdoctmp
      {
        \def\childdocname{#2}
        \def\childdocjob{#2}
        \def\jobname{#2}
        \input{#2}
        \endinput
      }
    \else
      \def\childdoctmp
      {
        \childdocdisable
        \def\childdocname{#2}
        \childdoctrue
        \includeonly{#2}
        \def\childdocjob{#1}
        \def\jobname{#1}
        \input{#1}
        \endinput
      }
    \fi
    \expandafter
  \endgroup
  \childdoctmp
}
%    \end{macrocode}

% \macro{\childdocforwardprefix}
% The command |\childdocforwardprefix| redirects
% compilation to the main or a child file by means of a pattern.
% The prefix |#1| in the current filename is replaced by |#2|
% and the suffix of the current filename is kept
% (it is assumed that the filename does not contain the substring `|~~~|'
% which is used as a delimiter).
% Compilation is handed over to the new file by |\childdocforward|:
%    \begin{macrocode}
\newcommand{\childdocforwardprefix}[3][]
{
  \begingroup
    \def\childdocextract #2##1~~~{\def\childdoctmp{\childdocforward[#1]{#3##1}}}
    \expandafter\childdocextract\childdocname~~~
    \expandafter
  \endgroup
  \childdoctmp
}
%    \end{macrocode}

% \macro{\childdoc}
% The deprecated macro |\childdoc| is a legacy version of |\childdocmain|:
%    \begin{macrocode}
\newcommand{\childdoc}{\childdocmain}
%    \end{macrocode}

% \macro{\childdocredirect}
% The deprecated macro |\childdocredirect| is a legacy version
% of |\childdocforward| and |\childdocforwardprefix|:
%    \begin{macrocode}
\newcommand{\childdocredirect}[2][]
{
  \begingroup
    \if?#1?
      \def\childdoctmp{\childdocforward{#2}}
    \else
      \def\childdoctmp{\childdocforwardprefix{#1}{#2}}
    \fi
    \expandafter
  \endgroup
  \childdoctmp
}
%    \end{macrocode}

%\iffalse
%</package>
%\fi
%
\endinput
|\\
|\childdocforwardprefix[|\textit{main}|]{|\textit{prefix}|}{|\textit{dest}|}|
\end{tabular}
\end{center}
%
the destination file is determined by a pattern
depending on the current file:
To make this work, the current file must be called
`{\textit{prefix}\hspace{0.2em}\textit{suffix}}'
with \textit{prefix} matching precisely the argument.
Processing is then passed on to the file
`{\textit{dest}\hspace{0.2em}\textit{suffix}}'.
Surely, the same effect is achieved by
directly specifying the
argument `{\textit{dest}\hspace{0.2em}\textit{suffix}}'
in the first form.
However, that requires to set up a different file
for each child. With the alternative form of the command
all these files can have exactly the same content
which simplifies setting them up and maintaining them.

For example, the following file |draft.tex|
with a compilation flag |\version| as described in \secref{sec:flags}
compiles the main document as a draft:
%
\begin{center}
\begin{tabular}{l}
|\def\version{draft}|\\
|% \iffalse
%
% childdoc.dtx Copyright (C) 2017-2018 Niklas Beisert
%
% This work may be distributed and/or modified under the
% conditions of the LaTeX Project Public License, either version 1.3
% of this license or (at your option) any later version.
% The latest version of this license is in
%   http://www.latex-project.org/lppl.txt
% and version 1.3 or later is part of all distributions of LaTeX
% version 2005/12/01 or later.
%
% This work has the LPPL maintenance status `maintained'.
%
% The Current Maintainer of this work is Niklas Beisert.
%
% This work consists of the files childdoc.dtx and childdoc.ins
% and the derived files childdoc.def and cdocsamp.tex with
% cdocsch1.tex, cdocsch2.tex, cdocsdrf.tex, cdocsfn1.tex, cdocsfn2.tex.
%
%<package>\ifdefined\childdocmain\endinput\fi
%<package>\ProvidesFile{childdoc.def}[2018/12/30 v2.0 child document driver]
%<samplemain>\ProvidesFile{cdocsamp.tex}[2018/12/30 v2.0 sample for childdoc]
%<*driver>
%\ProvidesFile{childdoc.drv}[2018/12/30 v2.0 childdoc reference manual file]
\PassOptionsToClass{10pt,a4paper}{article}
\documentclass{ltxdoc}

\usepackage[margin=35mm]{geometry}
\usepackage{hyperref}
\usepackage{hyperxmp}
\usepackage[usenames]{color}

\hypersetup{colorlinks=true}
\hypersetup{pdfstartview=FitH}
\hypersetup{pdfpagemode=UseNone}
\hypersetup{pdfsource={}}
\hypersetup{pdflang={en-UK}}
\hypersetup{pdfcopyright={Copyright 2017-2018 Niklas Beisert.
  This work may be distributed and/or modified under the
  conditions of the LaTeX Project Public License, either version 1.3
  of this license or (at your option) any later version.}}
\hypersetup{pdflicenseurl={http://www.latex-project.org/lppl.txt}}
\hypersetup{pdfcontactaddress={ETH Zurich, ITP, HIT K,
  Wolfgang-Pauli-Strasse 27}}
\hypersetup{pdfcontactpostcode={8093}}
\hypersetup{pdfcontactcity={Zurich}}
\hypersetup{pdfcontactcountry={Switzerland}}
\hypersetup{pdfcontactemail={nbeisert@itp.phys.ethz.ch}}
\hypersetup{pdfcontacturl={http://people.phys.ethz.ch/\xmptilde nbeisert/}}

\newcommand{\secref}[1]{\hyperref[#1]{section \ref*{#1}}}

\parskip1ex
\parindent0pt
\let\olditemize\itemize
\def\itemize{\olditemize\parskip0pt}

\begin{document}

\title{The \textsf{childdoc} Package}
\hypersetup{pdftitle={The childdoc Package}}
\author{Niklas Beisert\\[2ex]
  Institut f\"ur Theoretische Physik\\
  Eidgen\"ossische Technische Hochschule Z\"urich\\
  Wolfgang-Pauli-Strasse 27, 8093 Z\"urich, Switzerland\\[1ex]
  \href{mailto:nbeisert@itp.phys.ethz.ch}
  {\texttt{nbeisert@itp.phys.ethz.ch}}}
\hypersetup{pdfauthor={Niklas Beisert}}
\hypersetup{pdfsubject={Manual for the LaTeX2e Package childdoc}}
\date{30 December 2018, \textsf{v2.0}}
\maketitle

\begin{abstract}\noindent
\textsf{childdoc} is a \LaTeXe{} package
that enables the direct compilation
of document sections included by |\include|
to individual files.
\end{abstract}

\begingroup
\parskip0ex
\tableofcontents
\endgroup

%%%%%%%%%%%%%%%%%%%%%%%%%%%%%%%%%%%%%%%%%%%%%%%%%%%%%%%%%%%%%%%%%%%%%%%%%%%%%%%%
%%%%%%%%%%%%%%%%%%%%%%%%%%%%%%%%%%%%%%%%%%%%%%%%%%%%%%%%%%%%%%%%%%%%%%%%%%%%%%%%
\section{Introduction}

\LaTeX{} provides a mechanism to structure a large document (such as a book)
into a main file and several child files (containing the chapters)
using the |\include| command.
This mechanism is beneficial for documents
which span hundreds of pages in order to
make the source file(s) more manageable.
Moreover, compilation can be restricted to
selected child files by means of the |\includeonly| command.
The latter feature can be used to reduce the compilation time while editing
(this was significantly more useful in the earlier days of \LaTeX{})
or to generate a smaller document which is easier to navigate.
Another application of |\includeonly| is to generate
documents consisting of selected parts of the complete document.

However, there are a few drawbacks of the plain |\include| mechanism:
\begin{itemize}
\item
The child files cannot be compiled on their own,
they can only be compiled via the main file.
A naive editing environment
(such as a text editor with an option
to have the current file processed by \LaTeX)
may require one to switch to the main file before compiling;
attempting to compile the child file produces errors.
\item
The main file must be modified (each time)
to adjust the |\includeonly| command
to the present needs. This easily leaves the main file in a messy state.
\item
The generated document will always carry the filename
of the main document. This is inconvenient if
several child files are to be compiled and
to be kept for distribution.
\end{itemize}

The present package provides a simple interface
to make child files individually compilable by \LaTeX{}.
Compiling a child file then has the same effect as compiling
the main file with an |\includeonly| command
to select the appropriate child.
Moreover the generated document will carry the name of the child
rather than the main file.
This resolves all three above issues.

This feature is meant to make the editing of books,
thesis documents and lecture notes somewhat more convenient.
However, the package can also be used efficiently for
composing a series of documents (such as exercise sheets)
which are typically distributed individually.
It then assists the author in generating the individual documents
(potentially in different versions)
as well as a document containing the collected series.
Another application is in developing style files
or other kinds of included material
where compilation of the style file could redirect
to a sample or test file.

%%%%%%%%%%%%%%%%%%%%%%%%%%%%%%%%%%%%%%%%%%%%%%%%%%%%%%%%%%%%%%%%%%%%%%%%%%%%%%%%
%%%%%%%%%%%%%%%%%%%%%%%%%%%%%%%%%%%%%%%%%%%%%%%%%%%%%%%%%%%%%%%%%%%%%%%%%%%%%%%%
\section{Usage}

First of all, the package \textsf{childdoc} is \emph{not} a standard
\LaTeXe{} |.sty| style file! Therefore it needs to be invoked in
a non-standard way.

%%%%%%%%%%%%%%%%%%%%%%%%%%%%%%%%%%%%%%%%%%%%%%%%%%%%%%%%%%%%%%%%%%%%%%%%%%%%%%%%
\subsection{Included Files}
\label{sec:include}

%%%%%%%%%%%%%%%%%%%%%%%%%%%%%%%%%%%%%%%%
\DescribeMacro{\childdocmain}
To use the package, add the commands
\begin{center}
\begin{tabular}{l}
|\input{childdoc.def}|\\
|\childdocmain{}|\\
\end{tabular}
\end{center}
at the very top of the main \LaTeX{} file,
in particular \emph{before} the |\documentclass| statement!
The argument of |\childdocmain| should be left empty
(but it must be present).

%%%%%%%%%%%%%%%%%%%%%%%%%%%%%%%%%%%%%%%%
\DescribeMacro{\childdocof}
Furthermore, add the commands
\begin{center}
\begin{tabular}{l}
|\input{childdoc.def}|\\
|\childdocof{|\textit{main}|}|\\
\end{tabular}
\end{center}
at the top of every child file \textit{child}
which is included by |\include{|\textit{child}|}|
from within the main file
(or at least for those files to be compiled individually).
The argument \textit{main} must be the filename of the main file.

There are a couple of
considerations in setting up the main and child documents:

%%%%%%%%%%%%%%%%%%%%%%%%%%%%%%%%%%%%%%%%
\paragraph{Restrictions.}

Please note the following restrictions:
\begin{itemize}
\item
|\childdocmain| must be called with one argument \textit{main}
to ensure compatibility with earlier version of the package.
It must either be empty (|\childdocmain{}|)
or precisely match the filename of the main file in which it is specified.
See \secref{sec:detection} for further information.
\item
The filename \textit{main} must be specified without the |.tex| extension.
\item
The filename \textit{main} is case sensitive
(even in case-insensitive file systems)
due to internal string comparison.
\item
The argument \textit{main} should be fully expanded, it cannot be a macro.
\item
Subdirectories and special characters should be avoided in filenames.
\item
The command |\childdocmain{|\textit{main}|}| must be followed by a whitespace.
It should not be followed immediately by another command
or by a comment mark `|%|'.
This is because the \TeX{} parser reads the token immediately following
the argument of |\childdocmain| and puts it
at the beginning of every child section;
however, a white\-space is ignored.
\end{itemize}

%%%%%%%%%%%%%%%%%%%%%%%%%%%%%%%%%%%%%%%%
\paragraph{Content of Main File.}

It is advisable to place all content in the child files included by |\include|.
Any output contained in the main file will appear in all child documents
unless suppressed manually;
it cannot be suppressed automatically by the |\includeonly| directive
and thus should normally be avoided.
A method to include some content in the main file
by means of conditional processing is described in \secref{sec:conditional}.

%%%%%%%%%%%%%%%%%%%%%%%%%%%%%%%%%%%%%%%%
\paragraph{Page Numbering.}

When only a part of the document is compiled,
the appropriate numbering of pages
(as well as other status parameters)
is determined from the |.aux| files.
The latter contain information from previous passes.
However this information needs to propagate through
all intermediate child documents.
Therefore the page numbering in child documents may well
be inconsistent until the complete document is compiled at least once.

A useful (if unconventional) way to always ensure a consistent
page numbering is to restart the numbering in each child document
and denote the pages by `\textit{child}|.|\textit{page}'
where \textit{child} represents the chapter/section number of the child file.
This can be achieved by the command
|\numberwithin{page}{|\textit{child}|}|
of the \textsf{amsmath} package
where \textit{child} can be |chapter| or |section|
depending on the chosen structuring.
Alternatively, one can modify the macro |\thepage| appropriately
and reset the counter |page| at the start of each child file.

%%%%%%%%%%%%%%%%%%%%%%%%%%%%%%%%%%%%%%%%%%%%%%%%%%%%%%%%%%%%%%%%%%%%%%%%%%%%%%%%
\subsection{Conditional Processing}
\label{sec:conditional}

The package provides a mechanism to compile different versions
of a document. To customise the versions further some conditional processing
can come in handy to distinguish which version is being compiled.
The package provides two macros to describe the compilation context:

%%%%%%%%%%%%%%%%%%%%%%%%%%%%%%%%%%%%%%%%
\DescribeMacro{\ifchilddoc}
The conditional |\ifchilddoc| distinguishes between the compilation of
child documents and the main document:
%
\begin{center}
|\ifchilddoc |\textit{child-code}| |[|\||else |\textit{main-code}]| \||fi|
\end{center}

%%%%%%%%%%%%%%%%%%%%%%%%%%%%%%%%%%%%%%%%
\DescribeMacro{\childdocname}
\DescribeMacro{\childdocjob}
The macro |\childdocname| contains the filename (without extension)
of the main or child file being processed.
Note that |\childdocjob| will always contain the name of the main file.

%%%%%%%%%%%%%%%%%%%%%%%%%%%%%%%%%%%%%%%%
\paragraph{Title Page.}

Conditional processing can be used to include a title or banner page
in the main document when proper precautions are taken.
Importantly, the code in the main file should ensure that the page counter
(as well as other status parameters which are stored in the |.aux| files)
takes the same value after the conditional processing.
Otherwise the page numbers may take divergent values
depending on which part is compiled.

For example, a title page could be declared by:
%
\begin{center}
\begin{tabular}{l}
|\ifchilddoc\||else|\\
|\addtocounter{page}{-1}|\\
\textit{code for title page}\\
|\newpage|\\
|\||fi|
\end{tabular}
\end{center}
%
A banner page for the child documents can be generated by:
%
\begin{center}
\begin{tabular}{l}
|\ifchilddoc|\\
|\addtocounter{page}{-1}|\\
\textit{code for banner page}\\
|\newpage|\\
|\||fi|
\end{tabular}
\end{center}
%
Here one could write a message such as:
\begin{center}
|This is the part \childdocname{} of \childdocjob{}.|
\end{center}

%%%%%%%%%%%%%%%%%%%%%%%%%%%%%%%%%%%%%%%%%%%%%%%%%%%%%%%%%%%%%%%%%%%%%%%%%%%%%%%%
\subsection{Flags}
\label{sec:flags}

The package makes it easy to generate different versions
of the main or child documents.
To this end compilation flags can be defined
and assigned different default values.
They will be particularly useful in conjunction
with the forwarding mechanism described in \secref{sec:forward}.

For example, it may be useful to have a flag |\version|
which can be set to |draft| or |final|.
The document source will contain some conditional code
depending on the value of |\version|.
Suppose further, the flag should default to |final| for the main file
and to |draft| for child files
which is a natural assignment for editing the document.
This is achieved by placing the following code
in the preamble of the main document
(below the |\childdocmain| directive):
%
\begin{center}
\begin{tabular}{l}
|\ifchilddoc|\\
|\providecommand{\version}{draft}|\\
|\||else|\\
|\providecommand{\version}{final}|\\
|\||fi|
\end{tabular}
\end{center}
%
The definition by |\providecommand| makes sure
that previous definitions are not overwritten.
Further statements |\providecommand{\version}{...}|
can thus be added before the above code to override it.

For the main file, one might add a line
(between |\childdocmain| and the above block)
%
\begin{center}
|%\ifchilddoc\||else\providecommand{\version}{draft}\||fi|
\end{center}
%
which can be uncommented to produce a draft version.
Likewise one can add a line to the very top of a child file
(above the |\childdocof{|\textit{main}|}| directive)
%
\begin{center}
|%\providecommand{\version}{final}|
\end{center}
%
which can be uncommented to produce the final version of this child document.

%%%%%%%%%%%%%%%%%%%%%%%%%%%%%%%%%%%%%%%%%%%%%%%%%%%%%%%%%%%%%%%%%%%%%%%%%%%%%%%%
\subsection{Forwarding}
\label{sec:forward}

Different versions of the main or child documents
using compilation flags as described in \secref{sec:flags}
can be (permanently) stored in different files
for convenient compilation, viewing and distribution.
To this end, the package defines a command
to pass on compilation to a different file:

%%%%%%%%%%%%%%%%%%%%%%%%%%%%%%%%%%%%%%%%
\DescribeMacro{\childdocforward}
The command |\childdocforward| redirects processing to
another source file:
%
\begin{center}
\begin{tabular}{l}
|\input{childdoc.def}|\\
|\childdocforward[|\textit{main}|]{|\textit{dest}|}|\\
\end{tabular}
\end{center}
%
The argument \textit{dest} is the destination file
(without extension).
It should be the main file or one of the child files.
Note that further \textsf{childdoc} directives
such as |\childdocof| and |\childdocforward|
in the indicated file will be processed in this form.
The optional argument \textit{main}
passes on directly to the main file \textit{main}
while pretending to compile the child \textit{dest}.
This form behaves as if \textit{dest}
issues |\childdocof{|\textit{main}|}| right away,
and no further \textsf{childdoc} directives will be processed.

%%%%%%%%%%%%%%%%%%%%%%%%%%%%%%%%%%%%%%%%
\DescribeMacro{\...prefix}
In the alternative form |\childdocforwardprefix|,
%
\begin{center}
\begin{tabular}{l}
|\input{childdoc.def}|\\
|\childdocforwardprefix[|\textit{main}|]{|\textit{prefix}|}{|\textit{dest}|}|
\end{tabular}
\end{center}
%
the destination file is determined by a pattern
depending on the current file:
To make this work, the current file must be called
`{\textit{prefix}\hspace{0.2em}\textit{suffix}}'
with \textit{prefix} matching precisely the argument.
Processing is then passed on to the file
`{\textit{dest}\hspace{0.2em}\textit{suffix}}'.
Surely, the same effect is achieved by
directly specifying the
argument `{\textit{dest}\hspace{0.2em}\textit{suffix}}'
in the first form.
However, that requires to set up a different file
for each child. With the alternative form of the command
all these files can have exactly the same content
which simplifies setting them up and maintaining them.

For example, the following file |draft.tex|
with a compilation flag |\version| as described in \secref{sec:flags}
compiles the main document as a draft:
%
\begin{center}
\begin{tabular}{l}
|\def\version{draft}|\\
|\input{childdoc.def}|\\
|\childdocforward{|\textit{main}|}|
\end{tabular}
\end{center}
%
Likewise, the following files |final|\textit{nn}|.tex|
compile the final version of the child document
|child|\textit{nn}|.tex|:
%
\begin{center}
\begin{tabular}{l}
|\def\version{final}|\\
|\input{childdoc.def}|\\
|\childdocforwardprefix{final}{child}|
\end{tabular}
\end{center}
%

Note that when several versions of a main file and/or of each child file
are to be generated, it may be convenient to set up a |Makefile| or
shell script to automatise the process.

%%%%%%%%%%%%%%%%%%%%%%%%%%%%%%%%%%%%%%%%%%%%%%%%%%%%%%%%%%%%%%%%%%%%%%%%%%%%%%%%
\subsection{Command Line Processing}
\label{sec:commandline}

The effect of redirection files can also be achieved by invoking
the \LaTeX{} compiler with a more elaborate command line.
Most conveniently this should be done as part
of a shell script or a |Makefile|.

When using \textsf{childdoc} in the main file, the following
command lines effectively perform a redirection
(note that depending on the shell being used,
backslashes may have to be doubled: `|\|' $\to$ `|\\|'):
%
\begin{center}
|... -jobname "|\textit{target}|" |\\|"|[\textit{flags}]%
|\input{childdoc.def}\childdocforward[|\textit{main}|]{|\textit{dest}|}"|
\end{center}
%
Here \textit{target} is the name of the output file,
\textit{main} is the name of the main file
and \textit{dest} is the name of the main or child file to be processed
(all filenames without extensions).
The optional argument \textit{main} can be omitted
if \textit{main} matches \textit{dest}.
Optionally, compilation \textit{flags} can be defined via |\def| commands.
This command line makes the \TeX{} engine believe
it is compiling the file \textit{target}
whose content is specified as the latter parameter.
The provided code then forwards the processing to
\textit{main} or \textit{dest} as described in \secref{sec:forward}.

%%%%%%%%%%%%%%%%%%%%%%%%%%%%%%%%%%%%%%%%%%%%%%%%%%%%%%%%%%%%%%%%%%%%%%%%%%%%%%%%
\subsection{Include by Input}
\label{sec:input}

Including child documents by |\include| has some restrictions by design.
Most notably, the content of a child document always occupies
its own set of pages; pages cannot be shared between child documents.
Usually, this behaviour makes perfect sense
because each child document contain an essential part of the document.
However, in some situations it may be desirable to compose
a document from a collection of parts
without having mandatory page breaks between then.
For this case, the package
provides a mechanism to include parts
by |\input| which can also be processed individually.
However, by construction this mechanism
requires manual handling of the content to be output.

%%%%%%%%%%%%%%%%%%%%%%%%%%%%%%%%%%%%%%%%
\DescribeMacro{\ifchilddocmanual}
The main file should be prepared as usual, see \secref{sec:include}.
However, the document body must make a distinction
between processing of an individual part and of the main document, e.g.:
%
\begin{center}
\begin{tabular}{l}
|\ifchilddocmanual|\\
|\input{\childdocname}|\\
|\||else|\\
\textit{document body with }|\input{|\textit{part}|}|\\
|\||fi|
\end{tabular}
\end{center}
%
The conditional |\ifchilddocmanual| is true whenever
a part to be included by |\input| is being compiled,
and the name of the part is stored in |\childdocname|.

%%%%%%%%%%%%%%%%%%%%%%%%%%%%%%%%%%%%%%%%
\DescribeMacro{\childdocby}
Each part to be included by |\input| should start with:
%
\begin{center}
\begin{tabular}{l}
|\input{childdoc.def}|\\
|\childdocby{|\textit{main}|}|\\
\end{tabular}
\end{center}
%
The directive |\childdocby| is similar to |\childdocof|
described in \secref{sec:include},
but the subsequent selection of content must be done manually.
To that end, both |\ifchilddoc| and |\ifchilddocmanual|
will be true upon processing of a part,
and the name of the part is stored in |\childdocname|.
Note that |\jobname| will be set to the filename of the current part
so that each part receives an individual |.aux| file
that does not interfere with the |.aux| file(s) of the main document.
This behaviour can be altered by the alternative form
|\childdocby[*]{|\textit{main}|}| (with a non-empty optional argument)
which uses the |.aux| file of the main document
by setting |\jobname| to \textit{main}.

%%%%%%%%%%%%%%%%%%%%%%%%%%%%%%%%%%%%%%%%%%%%%%%%%%%%%%%%%%%%%%%%%%%%%%%%%%%%%%%%
\subsection{Driver Development}
\label{sec:driver}

The \textsf{childdoc} mechanism can also be use for the development
of definition files such as \LaTeX{} styles or classes.
This case differs from the above setup with multiple parts
included by |\include| in that no |\includeonly| should be invoked.
This can be achieved by starting the include file
(before |\ProvidesPackage|) with:
%
\begin{center}
\begin{tabular}{l}
|\input{childdoc.def}|\\
|\childdocforward{|\textit{main}|}|\\
\end{tabular}
\end{center}
%
or alternatively with:
%
\begin{center}
\begin{tabular}{l}
|\input{childdoc.def}|\\
|\childdocby{|\textit{main}|}|\\
\end{tabular}
\end{center}
%
Both forms have slightly different effects as described above.
The main file is prepared as usual, see \secref{sec:include}.

%%%%%%%%%%%%%%%%%%%%%%%%%%%%%%%%%%%%%%%%%%%%%%%%%%%%%%%%%%%%%%%%%%%%%%%%%%%%%%%%
\subsection{Legacy Detection}
\label{sec:detection}

The directive |\childdocmain| in the main file can detect
whether the complete document or merely a child is to be compiled
even without using the directive |\childdocof|.
This method is deprecated because it is less robust
and there is no compelling reason to use it;
it is merely provided for backward compatibility
and it may be removed in future versions.

If the detection mechanism is to be used,
it is mandatory to correctly specify
the filename of the main file as the argument of |\childdocmain|:
%
\begin{center}
\begin{tabular}{l}
|\input{childdoc.def}|\\
|\childdocmain{|\textit{main}|}|\\
\end{tabular}
\end{center}
%
If |\jobname| does not match the argument \textit{main} of |\childdocmain|,
it is assumed that |\jobname| points to the child file to be compiled.
When using |\childdocmain| with the main file specified as argument,
it suffices to start a child file
with just |\input{|\textit{main}|}|
without loading of the package and using |\childdocof|.
If instead all processing is done
with the appropriate \textsf{childdoc} directives,
the argument of \textit{main} of |\childdocmain| can be empty.

An alternative version of the command line processing described
in \secref{sec:commandline} using the detection mechanism reads:
%
\begin{center}
|... -jobname "|\textit{target}|" "|[\textit{flags}]%
[|\def\jobname{|\textit{dest}|}|]|\input{|\textit{main}|}"|
\end{center}

%%%%%%%%%%%%%%%%%%%%%%%%%%%%%%%%%%%%%%%%%%%%%%%%%%%%%%%%%%%%%%%%%%%%%%%%%%%%%%%%
\subsection{Manual Code}
\label{sec:manual}

In case one cannot be certain whether the definitions file |childdoc.def|
is installed on the target \TeX{} distribution
and one prefers not to ship it,
it is conceivable to paste a few relevant commands into the sources.

To that end, drop all statements |\input{childdoc.def}|
and perform the replacements as outlined below.
Instead of |\childdocmain{|\textit{main}|}| add the following code
to the top of the main file:
%
\begin{center}
\begin{tabular}{l}
|\||ifdefined\childdocname\endinput\||fi\newif\ifchilddoc|\\
|\edef\childdocname{\scantokens\expandafter{\jobname\noexpand}}|\\
|\def\childdocmain{|\textit{main}|}\||ifx\childdocmain\childdocname\||else|\\
|\childdoctrue\includeonly{\childdocname}\let\jobname\childdocmain\||fi|\\
\end{tabular}
\end{center}
%
Instead of |\childdocof{|\textit{main}|}| just include the main file
at the top of each child file:
%
\begin{center}
|\input{|\textit{main}|}|
\end{center}
%
A simple redirection |\childdocforward{|\textit{dest}|}| is achieved by:
%
\begin{center}
|\def\jobname{|\textit{dest}|}\input{\jobname}|
\end{center}
%
The redirection with prefix
|\childdocforwardprefix[|\textit{prefix}|]{|\textit{dest}|}|
is accomplished by:
%
\begin{center}
\begin{tabular}{l}
|{\edef\jobname{\scantokens\expandafter{\jobname\noexpand}}|\\
|\def\redirectjob |\textit{prefix}|#1~~~{\gdef\jobname{|\textit{dest}|#1}}|\\
|\expandafter\redirectjob\jobname~~~}\input{\jobname}|
\end{tabular}
\end{center}

In an alternative approach,
child documents can be compiled by a specific command line
without additional code or specific definitions:
%
\begin{center}
|... -jobname "|\textit{target}|" "|[\textit{flags}]%
|\includeonly{|\textit{dest}|}\input{|\textit{main}|}"|
\end{center}
%

%%%%%%%%%%%%%%%%%%%%%%%%%%%%%%%%%%%%%%%%%%%%%%%%%%%%%%%%%%%%%%%%%%%%%%%%%%%%%%%%
%%%%%%%%%%%%%%%%%%%%%%%%%%%%%%%%%%%%%%%%%%%%%%%%%%%%%%%%%%%%%%%%%%%%%%%%%%%%%%%%
\section{Information}

%%%%%%%%%%%%%%%%%%%%%%%%%%%%%%%%%%%%%%%%%%%%%%%%%%%%%%%%%%%%%%%%%%%%%%%%%%%%%%%%
\subsection{Copyright}

Copyright \copyright{} 2017--2018 Niklas Beisert

This work may be distributed and/or modified under the
conditions of the \LaTeX{} Project Public License, either version 1.3
of this license or (at your option) any later version.
The latest version of this license is in
  \url{http://www.latex-project.org/lppl.txt}
and version 1.3 or later is part of all distributions of \LaTeX{}
version 2005/12/01 or later.

This work has the LPPL maintenance status `maintained'.

The Current Maintainer of this work is Niklas Beisert.

This work consists of the files |README.txt|, |childdoc.ins| and |childdoc.dtx|
as well as the derived files |childdoc.def|, |cdocsamp.tex|
with |cdocsch1.tex|, |cdocsch2.tex|, |cdocspt3.tex|, |cdocspt4.tex|,
|cdocsdrf.tex|, |cdocsfn1.tex|, |cdocsfn2.tex|
as well as |childdoc.pdf|.

%%%%%%%%%%%%%%%%%%%%%%%%%%%%%%%%%%%%%%%%%%%%%%%%%%%%%%%%%%%%%%%%%%%%%%%%%%%%%%%%
\subsection{Files and Installation}

The package consists of the files:
%
\begin{center}
\begin{tabular}{ll}
    |README.txt|   & readme file \\
    |childdoc.ins| & installation file \\
    |childdoc.dtx| & source file \\
    |childdoc.def| & definition file \\
    |cdocsamp.tex| & sample main file \\
    |cdocsch1.tex| & sample include file \\
    |cdocsch2.tex| & sample include file \\
    |cdocspt3.tex| & sample part file \\
    |cdocspt4.tex| & sample part file \\
    |cdocsdrf.tex| & sample redirection file \\
    |cdocsfn1.tex| & sample redirection file \\
    |cdocsfn2.tex| & sample redirection file \\
    |childdoc.pdf| & manual
\end{tabular}
\end{center}
%
The distribution consists of the files
|README.txt|, |childdoc.ins| and |childdoc.dtx|.
%
\begin{itemize}
\item
Run (pdf)\LaTeX{} on |childdoc.dtx|
to compile the manual |childdoc.pdf| (this file).
\item
Run \LaTeX{} on |childdoc.ins| to create the definitions file |childdoc.def|
and the sample |cdocsamp.tex| with include files
|cdocsch1.tex|, |cdocsch2.tex|, |cdocspt3.tex|, |cdocspt4.tex|,
|cdocsdrf.tex|, |cdocsfn1.tex|, |cdocsfn2.tex|.
Then copy the file |childdoc.def| to an appropriate directory of your \LaTeX{}
distribution, e.g.\ \textit{texmf-root}|/tex/latex/childdoc|.
\end{itemize}

%%%%%%%%%%%%%%%%%%%%%%%%%%%%%%%%%%%%%%%%%%%%%%%%%%%%%%%%%%%%%%%%%%%%%%%%%%%%%%%%
\subsection{Related CTAN Packages}

There are several other packages which offer a similar functionality:
%
\begin{itemize}
\item
The packages
\href{http://ctan.org/pkg/docmute}{\textsf{docmute}},
\href{http://ctan.org/pkg/includex}{\textsf{includex}} and
\href{http://ctan.org/pkg/standalone}{\textsf{standalone}}
provide commands to include only the document body of
a child file thus allowing both files to be compiled individually.
\item
The packages \href{http://ctan.org/pkg/subdocs}{\textsf{subdocs}}
and \href{http://ctan.org/pkg/subfiles}{\textsf{subfiles}}
provide structures in which the main and child documents can be
encapsulated and allowing them to be compiled individually.
The inclusion mechanism is different from the conventional |\include|.
\item
The package \href{http://ctan.org/pkg/combine}{\textsf{combine}}
is an elaborate solution to combine several documents into one.
\end{itemize}
%
See also the CTAN topic \href{http://ctan.org/topic/subdocs}{\textsf{subdocs}}
for further related packages.
The present package differs from the above solutions in that
a document structure constructed with the conventional |\include| mechanism
just needs two extra commands at the top of every file
such that all constituent files can be compiled individually.

%%%%%%%%%%%%%%%%%%%%%%%%%%%%%%%%%%%%%%%%%%%%%%%%%%%%%%%%%%%%%%%%%%%%%%%%%%%%%%%%
%\subsection{Feature Suggestions}
%
%The following is a list of features which may be useful for future
%versions of this package:
%%
%\begin{itemize}
%\item
%\ldots
%\end{itemize}

%%%%%%%%%%%%%%%%%%%%%%%%%%%%%%%%%%%%%%%%%%%%%%%%%%%%%%%%%%%%%%%%%%%%%%%%%%%%%%%%
\subsection{Revision History}

%%%%%%%%%%%%%%%%%%%%%%%%%%%%%%%%%%%%%%%%
\paragraph{v2.0:} 2018/12/30

\begin{itemize}
\item
immediate forward processing
\item
added |\childdocby| mechanism
\item
manual restructured
\end{itemize}

%%%%%%%%%%%%%%%%%%%%%%%%%%%%%%%%%%%%%%%%
\paragraph{v1.6:} 2018/01/17

\begin{itemize}
\item
application for development of include files
\item
corrections to manual
\end{itemize}

%%%%%%%%%%%%%%%%%%%%%%%%%%%%%%%%%%%%%%%%
\paragraph{v1.5:} 2017/05/21

\begin{itemize}
\item
more complete structuring introduced
\item
|\childdocof| introduced
\item
|\childdoc| renamed to |\childdocmain|
\item
|\childredirect| renamed to |\childdocforward| and |\childdocforwardprefix|
and functionality expanded
\end{itemize}

%%%%%%%%%%%%%%%%%%%%%%%%%%%%%%%%%%%%%%%%
\paragraph{v1.0:} 2017/04/27

\begin{itemize}
\item
manual and install package
\item
first version published on CTAN
\end{itemize}

%%%%%%%%%%%%%%%%%%%%%%%%%%%%%%%%%%%%%%%%
\paragraph{v0.6:} 2017/04/26

\begin{itemize}
\item
redirection mechanism added
\end{itemize}

%%%%%%%%%%%%%%%%%%%%%%%%%%%%%%%%%%%%%%%%
\paragraph{v0.5:} 2017/04/26

\begin{itemize}
\item
functionality in definition file
\end{itemize}


%%%%%%%%%%%%%%%%%%%%%%%%%%%%%%%%%%%%%%%%%%%%%%%%%%%%%%%%%%%%%%%%%%%%%%%%%%%%%%%%
%%%%%%%%%%%%%%%%%%%%%%%%%%%%%%%%%%%%%%%%%%%%%%%%%%%%%%%%%%%%%%%%%%%%%%%%%%%%%%%%
%%%%%%%%%%%%%%%%%%%%%%%%%%%%%%%%%%%%%%%%%%%%%%%%%%%%%%%%%%%%%%%%%%%%%%%%%%%%%%%%
\appendix

\settowidth\MacroIndent{\rmfamily\scriptsize 000\ }

 \DocInput{childdoc.dtx}

\end{document}
%</driver>
% \fi
%
% %%%%%%%%%%%%%%%%%%%%%%%%%%%%%%%%%%%%%%%%%%%%%%%%%%%%%%%%%%%%%%%%%%%%%%%%%%%%%%
% %%%%%%%%%%%%%%%%%%%%%%%%%%%%%%%%%%%%%%%%%%%%%%%%%%%%%%%%%%%%%%%%%%%%%%%%%%%%%%
% \section{Sample}
%\iffalse
%<*samplemain>
%\fi
%
% The following presents a sample document
% with two chapters, two parts, a title page,
% a compile flag as well as three forwarding files to set the flag.
% It consists of eight |.tex| files:
% \begin{center}
% \begin{tabular}{ll}
% |cdocsamp.tex|&main file\\
% |cdocsch1.tex|&include file for chapter 1\\
% |cdocsch2.tex|&include file for chapter 2\\
% |cdocspt3.tex|&include file for part 3\\
% |cdocspt4.tex|&include file for part 4\\
% |cdocsdrf.tex|&forwarding file for main file in draft mode\\
% |cdocsfi1.tex|&forwarding file for final version of chapter 1\\
% |cdocsfi2.tex|&forwarding file for final version of chapter 2\\
% \end{tabular}
% \end{center}
% Each of the eight files can be compiled directly by the \LaTeX{} compiler.
%
% %%%%%%%%%%%%%%%%%%%%%%%%%%%%%%%%%%%%%%
% \paragraph{Main File.}
%
% The main file is called |cdocsamp.tex|.
%
% Load the \textsf{childdoc} definitions and
% declare the filename for the main document:
%    \begin{macrocode}
\input{childdoc.def}
\childdocmain{}
%    \end{macrocode}

% Optional override for |\version| flag:
%    \begin{macrocode}
%%\ifchilddoc\else\providecommand{\version}{draft}\fi
%    \end{macrocode}

% Define the default values for the |\version| flag
% (|final| for the main file and |draft| for childs):
%    \begin{macrocode}
\ifchilddoc
\providecommand{\version}{draft}
\else
\providecommand{\version}{final}
\fi
%    \end{macrocode}

% Load the standard document class:
%    \begin{macrocode}
\documentclass[12pt]{article}
%    \end{macrocode}

% Start the document body:
%    \begin{macrocode}
\begin{document}
%    \end{macrocode}

% Declare a title page.
% Print title, part of document being processed and version flag:
%    \begin{macrocode}
\addtocounter{page}{-1}
\begin{center}
{\LARGE\bfseries{}childdoc example\par}
\vspace{1cm}
\ifchilddoc
\ifchilddocmanual part\else chapter\fi:
`\childdocname' of `\childdocjob'\par
\else
main document: `\childdocjob'\par
\fi
version: \version\par
\end{center}
\newpage
%    \end{macrocode}

% Manually include selected file,
% otherwise process as usual:
%    \begin{macrocode}
\ifchilddocmanual
\section*{part `\childdocname'}
\input{\childdocname}
\else
%    \end{macrocode}

% Include the two chapters:
%    \begin{macrocode}
\include{cdocsch1}
\include{cdocsch2}
%    \end{macrocode}

% Include the two parts unless only chapters should be displayed:
%    \begin{macrocode}
\ifchilddoc\else
\section{part three}
\input{cdocspt3}
\section{part four}
\input{cdocspt4}
\fi
%    \end{macrocode}

% Process as usual until here:
%    \begin{macrocode}
\fi
%    \end{macrocode}

% End of document body:
%    \begin{macrocode}
\end{document}
%    \end{macrocode}
%\iffalse
%</samplemain>
%\fi
%
% %%%%%%%%%%%%%%%%%%%%%%%%%%%%%%%%%%%%%%
% \paragraph{Chapter Include Files.}
%
% The include files are called |cdocsch1.tex| and |cdocsch2.tex|.
%
%\iffalse
%<*samplechap1|samplechap2>
%\fi

% Optional override for |\version| flag:
%    \begin{macrocode}
%%\providecommand{\version}{final}
%    \end{macrocode}

% Include the main document:
%    \begin{macrocode}
\input{childdoc.def}
\childdocof{cdocsamp}
%    \end{macrocode}

%\iffalse
%</samplechap1|samplechap2>
%\fi
%
%\iffalse
%<*samplechap1>
%\fi
% Some text for chapter 1:
%    \begin{macrocode}
\section{one}
some text in chapter one
%    \end{macrocode}

%\iffalse
%</samplechap1>
%\fi
% Some text for chapter 2:
%\iffalse
%<*samplechap2>
%\fi
%    \begin{macrocode}
\section{two}
more text in chapter two
%    \end{macrocode}

%\iffalse
%</samplechap2>
%\fi
%
% %%%%%%%%%%%%%%%%%%%%%%%%%%%%%%%%%%%%%%
% \paragraph{Part Include Files.}
%
% The include files are called |cdocspt3.tex| and |cdocspt4.tex|.
%
%\iffalse
%<*samplepart3|samplepart4>
%\fi

% Optional override for |\version| flag:
%    \begin{macrocode}
%%\providecommand{\version}{final}
%    \end{macrocode}

% Include the main document:
%    \begin{macrocode}
\input{childdoc.def}
\childdocby{cdocsamp}
%    \end{macrocode}

%\iffalse
%</samplepart3|samplepart4>
%\fi
%
%\iffalse
%<*samplepart3>
%\fi
% Some text for part 3:
%    \begin{macrocode}
some text in part three
%    \end{macrocode}

%\iffalse
%</samplepart3>
%\fi
% Some text for part 4:
%\iffalse
%<*samplepart4>
%\fi
%    \begin{macrocode}
more text in part four
%    \end{macrocode}

%\iffalse
%</samplepart4>
%\fi
%
% %%%%%%%%%%%%%%%%%%%%%%%%%%%%%%%%%%%%%%
% \paragraph{Forwarding for a Complete Draft.}
%
% The following forwarding file |cdocsdrf.tex|
% compiles the main document in draft mode:
%\iffalse
%<*sampledraft>
%\fi
%    \begin{macrocode}
\def\version{draft}
\input{childdoc.def}
\childdocforward{cdocsamp}
%    \end{macrocode}

%\iffalse
%</sampledraft>
%\fi
%
% %%%%%%%%%%%%%%%%%%%%%%%%%%%%%%%%%%%%%%
% \paragraph{Forwarding for Final Version of the Chapters.}
%
% The following forwarding files |cdocsfn1.tex| and |cdocsfn2.tex|
% (with identical content)
% compile the final versions of the child documents
% |cdocsch1.tex| and |cdocsch2.tex|, respectively:
%\iffalse
%<*samplefinal>
%\fi
%    \begin{macrocode}
\def\version{final}
\input{childdoc.def}
\childdocforwardprefix[cdocsamp]{cdocsfn}{cdocsch}
%    \end{macrocode}

%\iffalse
%</samplefinal>
%\fi
%
% %%%%%%%%%%%%%%%%%%%%%%%%%%%%%%%%%%%%%%
% \paragraph{Command Line Processing.}
%
% The following three command lines generate the output files
% |cdocscld|, |cdocscl1| and |cdocscl2|
% which should be identical to
% |cdocsdrf|, |cdocsch1| and |cdocsfn2|, respectively:
% \begin{center}
% \begin{tabular}{l}
% |latex -jobname cdocscld \|\\
% |  "\def\version{draft}\input{childdoc.def}\childdocforward{cdocsamp}"|\\
% |latex -jobname cdocscl1 \|\\
% |  "\input{childdoc.def}\childdocforward[cdocsamp]{cdocsch1}"|\\
% |latex -jobname cdocscl2 \|\\
% |  "\def\version{final}\input{childdoc.def}\childdocforward{cdocsch2}"|
% \end{tabular}
% \end{center}
% Note that the trailing backslash on each first line
% merely continues the input to the second line
% (for convenient cut ant paste).
% Furthermore, the command |latex| can be replaced by any
% of its alternative versions such as |pdflatex|.
%
% %%%%%%%%%%%%%%%%%%%%%%%%%%%%%%%%%%%%%%%%%%%%%%%%%%%%%%%%%%%%%%%%%%%%%%%%%%%%%%
% %%%%%%%%%%%%%%%%%%%%%%%%%%%%%%%%%%%%%%%%%%%%%%%%%%%%%%%%%%%%%%%%%%%%%%%%%%%%%%
% \section{Implementation}
%\iffalse
%<*package>
%\fi
%
% This section describes the definitions file |childdoc.def|.

% The definitions cannot be loaded using |\usepackage| or |\RequirePackage|
% which has a mechanism to prevent loading a style file more than once.
% When loading the definitions by means of |\input|
% multiple instances have to be prevented manually:
%\iffalse
%This code needs to be before the `\ProvidesFile' directive
%which is defined at the beginning of this file.
%Therefore it is also placed there and commented out here.
%</package>
%<*discard>
%\fi
%    \begin{macrocode}
\ifdefined\childdocmain\endinput\fi
%    \end{macrocode}
%\iffalse
%</discard>
%<*package>
%\fi
%
% \macro{\ifchilddoc}
% \macro{\ifchilddocmanual}
% The conditional |\ifchilddoc| tells whether a
% child (true) or main (false) document is being compiled.
% The conditional |\ifchilddocmanual| tells whether
% the |\includeonly| mechanism is used (false) or
% the selection of child files must be performed manually (true).
% The definitions initialise to false:
%    \begin{macrocode}
\newif\ifchilddoc
\newif\ifchilddocmanual
%    \end{macrocode}

% \macro{\childdocname}
% \macro{\childdocjob}
% The macro |\childdocname| stores the name of the main document
% to be compiled. The macro |\childdocjob| stores the name of
% the document on which the \LaTeX{} compiler was originally invoked.
% The content of |\jobname| cannot be compared
% to filenames specified in the source due to different catcodes.
% The following code rescans |\jobname|, stores the result
% in |\childdocname| and saves a copy in |\childdocjob|:
%    \begin{macrocode}
\edef\childdocname{\scantokens\expandafter{\jobname\noexpand}}
\let\childdocjob\childdocname
%    \end{macrocode}

% \macro{\childdocdisable}
% The macro |\childdocdisable| prevents the main file
% from being processed more than once.
% At this stage, the main document command |\childdocmain|
% is assumed to be called once again where it should do nothing.
% Any subsequent call to it should prevent
% a secondary processing of the main document
% It overwrites the forwarding commands
% |\childdocof| and |\childdocforward|
% with empty macros to prevent further inclusions of the main document:
%    \begin{macrocode}
\newcommand{\childdocdisable}
{
  \renewcommand{\childdocmain}[1]{\renewcommand{\childdocmain}[1]{\endinput}}
  \renewcommand{\childdocof}[1]{}
  \renewcommand{\childdocby}[2][]{}
  \renewcommand{\childdocforward}[2][]{}
  \renewcommand{\childdocdisable}{}
}
%    \end{macrocode}

% \macro{\childdocmain}
% The macro |\childdocmain| is to be called at the top of the main file
% with nothing or the main filename (without extension) as argument.
% First, it breaks loops.
% If the argument is not empty and does not match |\childdocname|
% (which is set by the first inclusion of |childdoc.def|),
% |\ifchilddoc| is set to true, |\includeonly| is applied to the child file
% and |\jobname| is set to the main file
% (for proper handling of |.aux| files):
%    \begin{macrocode}
\newcommand{\childdocmain}[1]
{
  \childdocdisable\childdocmain{}
  \if?#1?\else
    \begingroup
      \def\childdoctmp{#1}
      \ifx\childdoctmp\childdocname
        \def\childdoctmp{}
      \else
        \def\childdoctmp
        {
          \childdoctrue
          \includeonly{\childdocname}
          \def\childdocjob{#1}
          \def\jobname{#1}
        }
      \fi
      \expandafter
    \endgroup
    \childdoctmp
  \fi
}
%    \end{macrocode}

% \macro{\childdocof}
% The command |\childdocof| redirects
% compilation to the main file |#1|.
%    \begin{macrocode}
\newcommand{\childdocof}[1]
{
  \childdocdisable
  \childdoctrue
  \includeonly{\childdocname}
  \def\jobname{#1}
  \def\childdocjob{#1}
  \input{#1}
}
%    \end{macrocode}

% \macro{\childdocby}
% The command |\childdocby| ....
%    \begin{macrocode}
\newcommand{\childdocby}[2][]
{
  \childdocdisable
  \childdoctrue
  \childdocmanualtrue
  \if?#1?\else
    \def\jobname{#2}
  \fi
  \def\childdocjob{#2}
  \input{#2}
  \endinput
}
%    \end{macrocode}

% \macro{\childdocforward}
% The command |\childdocforward| redirects
% compilation to the main file or
% (if the optional argument is given) a child file.
% Parameters are set as if the main file
% or a child file starting with |\childdocof| was compiled.
% Then compilation is handed over to the main file:
%    \begin{macrocode}
\newcommand{\childdocforward}[2][]
{
  \begingroup
    \if?#1?
      \def\childdoctmp
      {
        \def\childdocname{#2}
        \def\childdocjob{#2}
        \def\jobname{#2}
        \input{#2}
        \endinput
      }
    \else
      \def\childdoctmp
      {
        \childdocdisable
        \def\childdocname{#2}
        \childdoctrue
        \includeonly{#2}
        \def\childdocjob{#1}
        \def\jobname{#1}
        \input{#1}
        \endinput
      }
    \fi
    \expandafter
  \endgroup
  \childdoctmp
}
%    \end{macrocode}

% \macro{\childdocforwardprefix}
% The command |\childdocforwardprefix| redirects
% compilation to the main or a child file by means of a pattern.
% The prefix |#1| in the current filename is replaced by |#2|
% and the suffix of the current filename is kept
% (it is assumed that the filename does not contain the substring `|~~~|'
% which is used as a delimiter).
% Compilation is handed over to the new file by |\childdocforward|:
%    \begin{macrocode}
\newcommand{\childdocforwardprefix}[3][]
{
  \begingroup
    \def\childdocextract #2##1~~~{\def\childdoctmp{\childdocforward[#1]{#3##1}}}
    \expandafter\childdocextract\childdocname~~~
    \expandafter
  \endgroup
  \childdoctmp
}
%    \end{macrocode}

% \macro{\childdoc}
% The deprecated macro |\childdoc| is a legacy version of |\childdocmain|:
%    \begin{macrocode}
\newcommand{\childdoc}{\childdocmain}
%    \end{macrocode}

% \macro{\childdocredirect}
% The deprecated macro |\childdocredirect| is a legacy version
% of |\childdocforward| and |\childdocforwardprefix|:
%    \begin{macrocode}
\newcommand{\childdocredirect}[2][]
{
  \begingroup
    \if?#1?
      \def\childdoctmp{\childdocforward{#2}}
    \else
      \def\childdoctmp{\childdocforwardprefix{#1}{#2}}
    \fi
    \expandafter
  \endgroup
  \childdoctmp
}
%    \end{macrocode}

%\iffalse
%</package>
%\fi
%
\endinput
|\\
|\childdocforward{|\textit{main}|}|
\end{tabular}
\end{center}
%
Likewise, the following files |final|\textit{nn}|.tex|
compile the final version of the child document
|child|\textit{nn}|.tex|:
%
\begin{center}
\begin{tabular}{l}
|\def\version{final}|\\
|% \iffalse
%
% childdoc.dtx Copyright (C) 2017-2018 Niklas Beisert
%
% This work may be distributed and/or modified under the
% conditions of the LaTeX Project Public License, either version 1.3
% of this license or (at your option) any later version.
% The latest version of this license is in
%   http://www.latex-project.org/lppl.txt
% and version 1.3 or later is part of all distributions of LaTeX
% version 2005/12/01 or later.
%
% This work has the LPPL maintenance status `maintained'.
%
% The Current Maintainer of this work is Niklas Beisert.
%
% This work consists of the files childdoc.dtx and childdoc.ins
% and the derived files childdoc.def and cdocsamp.tex with
% cdocsch1.tex, cdocsch2.tex, cdocsdrf.tex, cdocsfn1.tex, cdocsfn2.tex.
%
%<package>\ifdefined\childdocmain\endinput\fi
%<package>\ProvidesFile{childdoc.def}[2018/12/30 v2.0 child document driver]
%<samplemain>\ProvidesFile{cdocsamp.tex}[2018/12/30 v2.0 sample for childdoc]
%<*driver>
%\ProvidesFile{childdoc.drv}[2018/12/30 v2.0 childdoc reference manual file]
\PassOptionsToClass{10pt,a4paper}{article}
\documentclass{ltxdoc}

\usepackage[margin=35mm]{geometry}
\usepackage{hyperref}
\usepackage{hyperxmp}
\usepackage[usenames]{color}

\hypersetup{colorlinks=true}
\hypersetup{pdfstartview=FitH}
\hypersetup{pdfpagemode=UseNone}
\hypersetup{pdfsource={}}
\hypersetup{pdflang={en-UK}}
\hypersetup{pdfcopyright={Copyright 2017-2018 Niklas Beisert.
  This work may be distributed and/or modified under the
  conditions of the LaTeX Project Public License, either version 1.3
  of this license or (at your option) any later version.}}
\hypersetup{pdflicenseurl={http://www.latex-project.org/lppl.txt}}
\hypersetup{pdfcontactaddress={ETH Zurich, ITP, HIT K,
  Wolfgang-Pauli-Strasse 27}}
\hypersetup{pdfcontactpostcode={8093}}
\hypersetup{pdfcontactcity={Zurich}}
\hypersetup{pdfcontactcountry={Switzerland}}
\hypersetup{pdfcontactemail={nbeisert@itp.phys.ethz.ch}}
\hypersetup{pdfcontacturl={http://people.phys.ethz.ch/\xmptilde nbeisert/}}

\newcommand{\secref}[1]{\hyperref[#1]{section \ref*{#1}}}

\parskip1ex
\parindent0pt
\let\olditemize\itemize
\def\itemize{\olditemize\parskip0pt}

\begin{document}

\title{The \textsf{childdoc} Package}
\hypersetup{pdftitle={The childdoc Package}}
\author{Niklas Beisert\\[2ex]
  Institut f\"ur Theoretische Physik\\
  Eidgen\"ossische Technische Hochschule Z\"urich\\
  Wolfgang-Pauli-Strasse 27, 8093 Z\"urich, Switzerland\\[1ex]
  \href{mailto:nbeisert@itp.phys.ethz.ch}
  {\texttt{nbeisert@itp.phys.ethz.ch}}}
\hypersetup{pdfauthor={Niklas Beisert}}
\hypersetup{pdfsubject={Manual for the LaTeX2e Package childdoc}}
\date{30 December 2018, \textsf{v2.0}}
\maketitle

\begin{abstract}\noindent
\textsf{childdoc} is a \LaTeXe{} package
that enables the direct compilation
of document sections included by |\include|
to individual files.
\end{abstract}

\begingroup
\parskip0ex
\tableofcontents
\endgroup

%%%%%%%%%%%%%%%%%%%%%%%%%%%%%%%%%%%%%%%%%%%%%%%%%%%%%%%%%%%%%%%%%%%%%%%%%%%%%%%%
%%%%%%%%%%%%%%%%%%%%%%%%%%%%%%%%%%%%%%%%%%%%%%%%%%%%%%%%%%%%%%%%%%%%%%%%%%%%%%%%
\section{Introduction}

\LaTeX{} provides a mechanism to structure a large document (such as a book)
into a main file and several child files (containing the chapters)
using the |\include| command.
This mechanism is beneficial for documents
which span hundreds of pages in order to
make the source file(s) more manageable.
Moreover, compilation can be restricted to
selected child files by means of the |\includeonly| command.
The latter feature can be used to reduce the compilation time while editing
(this was significantly more useful in the earlier days of \LaTeX{})
or to generate a smaller document which is easier to navigate.
Another application of |\includeonly| is to generate
documents consisting of selected parts of the complete document.

However, there are a few drawbacks of the plain |\include| mechanism:
\begin{itemize}
\item
The child files cannot be compiled on their own,
they can only be compiled via the main file.
A naive editing environment
(such as a text editor with an option
to have the current file processed by \LaTeX)
may require one to switch to the main file before compiling;
attempting to compile the child file produces errors.
\item
The main file must be modified (each time)
to adjust the |\includeonly| command
to the present needs. This easily leaves the main file in a messy state.
\item
The generated document will always carry the filename
of the main document. This is inconvenient if
several child files are to be compiled and
to be kept for distribution.
\end{itemize}

The present package provides a simple interface
to make child files individually compilable by \LaTeX{}.
Compiling a child file then has the same effect as compiling
the main file with an |\includeonly| command
to select the appropriate child.
Moreover the generated document will carry the name of the child
rather than the main file.
This resolves all three above issues.

This feature is meant to make the editing of books,
thesis documents and lecture notes somewhat more convenient.
However, the package can also be used efficiently for
composing a series of documents (such as exercise sheets)
which are typically distributed individually.
It then assists the author in generating the individual documents
(potentially in different versions)
as well as a document containing the collected series.
Another application is in developing style files
or other kinds of included material
where compilation of the style file could redirect
to a sample or test file.

%%%%%%%%%%%%%%%%%%%%%%%%%%%%%%%%%%%%%%%%%%%%%%%%%%%%%%%%%%%%%%%%%%%%%%%%%%%%%%%%
%%%%%%%%%%%%%%%%%%%%%%%%%%%%%%%%%%%%%%%%%%%%%%%%%%%%%%%%%%%%%%%%%%%%%%%%%%%%%%%%
\section{Usage}

First of all, the package \textsf{childdoc} is \emph{not} a standard
\LaTeXe{} |.sty| style file! Therefore it needs to be invoked in
a non-standard way.

%%%%%%%%%%%%%%%%%%%%%%%%%%%%%%%%%%%%%%%%%%%%%%%%%%%%%%%%%%%%%%%%%%%%%%%%%%%%%%%%
\subsection{Included Files}
\label{sec:include}

%%%%%%%%%%%%%%%%%%%%%%%%%%%%%%%%%%%%%%%%
\DescribeMacro{\childdocmain}
To use the package, add the commands
\begin{center}
\begin{tabular}{l}
|\input{childdoc.def}|\\
|\childdocmain{}|\\
\end{tabular}
\end{center}
at the very top of the main \LaTeX{} file,
in particular \emph{before} the |\documentclass| statement!
The argument of |\childdocmain| should be left empty
(but it must be present).

%%%%%%%%%%%%%%%%%%%%%%%%%%%%%%%%%%%%%%%%
\DescribeMacro{\childdocof}
Furthermore, add the commands
\begin{center}
\begin{tabular}{l}
|\input{childdoc.def}|\\
|\childdocof{|\textit{main}|}|\\
\end{tabular}
\end{center}
at the top of every child file \textit{child}
which is included by |\include{|\textit{child}|}|
from within the main file
(or at least for those files to be compiled individually).
The argument \textit{main} must be the filename of the main file.

There are a couple of
considerations in setting up the main and child documents:

%%%%%%%%%%%%%%%%%%%%%%%%%%%%%%%%%%%%%%%%
\paragraph{Restrictions.}

Please note the following restrictions:
\begin{itemize}
\item
|\childdocmain| must be called with one argument \textit{main}
to ensure compatibility with earlier version of the package.
It must either be empty (|\childdocmain{}|)
or precisely match the filename of the main file in which it is specified.
See \secref{sec:detection} for further information.
\item
The filename \textit{main} must be specified without the |.tex| extension.
\item
The filename \textit{main} is case sensitive
(even in case-insensitive file systems)
due to internal string comparison.
\item
The argument \textit{main} should be fully expanded, it cannot be a macro.
\item
Subdirectories and special characters should be avoided in filenames.
\item
The command |\childdocmain{|\textit{main}|}| must be followed by a whitespace.
It should not be followed immediately by another command
or by a comment mark `|%|'.
This is because the \TeX{} parser reads the token immediately following
the argument of |\childdocmain| and puts it
at the beginning of every child section;
however, a white\-space is ignored.
\end{itemize}

%%%%%%%%%%%%%%%%%%%%%%%%%%%%%%%%%%%%%%%%
\paragraph{Content of Main File.}

It is advisable to place all content in the child files included by |\include|.
Any output contained in the main file will appear in all child documents
unless suppressed manually;
it cannot be suppressed automatically by the |\includeonly| directive
and thus should normally be avoided.
A method to include some content in the main file
by means of conditional processing is described in \secref{sec:conditional}.

%%%%%%%%%%%%%%%%%%%%%%%%%%%%%%%%%%%%%%%%
\paragraph{Page Numbering.}

When only a part of the document is compiled,
the appropriate numbering of pages
(as well as other status parameters)
is determined from the |.aux| files.
The latter contain information from previous passes.
However this information needs to propagate through
all intermediate child documents.
Therefore the page numbering in child documents may well
be inconsistent until the complete document is compiled at least once.

A useful (if unconventional) way to always ensure a consistent
page numbering is to restart the numbering in each child document
and denote the pages by `\textit{child}|.|\textit{page}'
where \textit{child} represents the chapter/section number of the child file.
This can be achieved by the command
|\numberwithin{page}{|\textit{child}|}|
of the \textsf{amsmath} package
where \textit{child} can be |chapter| or |section|
depending on the chosen structuring.
Alternatively, one can modify the macro |\thepage| appropriately
and reset the counter |page| at the start of each child file.

%%%%%%%%%%%%%%%%%%%%%%%%%%%%%%%%%%%%%%%%%%%%%%%%%%%%%%%%%%%%%%%%%%%%%%%%%%%%%%%%
\subsection{Conditional Processing}
\label{sec:conditional}

The package provides a mechanism to compile different versions
of a document. To customise the versions further some conditional processing
can come in handy to distinguish which version is being compiled.
The package provides two macros to describe the compilation context:

%%%%%%%%%%%%%%%%%%%%%%%%%%%%%%%%%%%%%%%%
\DescribeMacro{\ifchilddoc}
The conditional |\ifchilddoc| distinguishes between the compilation of
child documents and the main document:
%
\begin{center}
|\ifchilddoc |\textit{child-code}| |[|\||else |\textit{main-code}]| \||fi|
\end{center}

%%%%%%%%%%%%%%%%%%%%%%%%%%%%%%%%%%%%%%%%
\DescribeMacro{\childdocname}
\DescribeMacro{\childdocjob}
The macro |\childdocname| contains the filename (without extension)
of the main or child file being processed.
Note that |\childdocjob| will always contain the name of the main file.

%%%%%%%%%%%%%%%%%%%%%%%%%%%%%%%%%%%%%%%%
\paragraph{Title Page.}

Conditional processing can be used to include a title or banner page
in the main document when proper precautions are taken.
Importantly, the code in the main file should ensure that the page counter
(as well as other status parameters which are stored in the |.aux| files)
takes the same value after the conditional processing.
Otherwise the page numbers may take divergent values
depending on which part is compiled.

For example, a title page could be declared by:
%
\begin{center}
\begin{tabular}{l}
|\ifchilddoc\||else|\\
|\addtocounter{page}{-1}|\\
\textit{code for title page}\\
|\newpage|\\
|\||fi|
\end{tabular}
\end{center}
%
A banner page for the child documents can be generated by:
%
\begin{center}
\begin{tabular}{l}
|\ifchilddoc|\\
|\addtocounter{page}{-1}|\\
\textit{code for banner page}\\
|\newpage|\\
|\||fi|
\end{tabular}
\end{center}
%
Here one could write a message such as:
\begin{center}
|This is the part \childdocname{} of \childdocjob{}.|
\end{center}

%%%%%%%%%%%%%%%%%%%%%%%%%%%%%%%%%%%%%%%%%%%%%%%%%%%%%%%%%%%%%%%%%%%%%%%%%%%%%%%%
\subsection{Flags}
\label{sec:flags}

The package makes it easy to generate different versions
of the main or child documents.
To this end compilation flags can be defined
and assigned different default values.
They will be particularly useful in conjunction
with the forwarding mechanism described in \secref{sec:forward}.

For example, it may be useful to have a flag |\version|
which can be set to |draft| or |final|.
The document source will contain some conditional code
depending on the value of |\version|.
Suppose further, the flag should default to |final| for the main file
and to |draft| for child files
which is a natural assignment for editing the document.
This is achieved by placing the following code
in the preamble of the main document
(below the |\childdocmain| directive):
%
\begin{center}
\begin{tabular}{l}
|\ifchilddoc|\\
|\providecommand{\version}{draft}|\\
|\||else|\\
|\providecommand{\version}{final}|\\
|\||fi|
\end{tabular}
\end{center}
%
The definition by |\providecommand| makes sure
that previous definitions are not overwritten.
Further statements |\providecommand{\version}{...}|
can thus be added before the above code to override it.

For the main file, one might add a line
(between |\childdocmain| and the above block)
%
\begin{center}
|%\ifchilddoc\||else\providecommand{\version}{draft}\||fi|
\end{center}
%
which can be uncommented to produce a draft version.
Likewise one can add a line to the very top of a child file
(above the |\childdocof{|\textit{main}|}| directive)
%
\begin{center}
|%\providecommand{\version}{final}|
\end{center}
%
which can be uncommented to produce the final version of this child document.

%%%%%%%%%%%%%%%%%%%%%%%%%%%%%%%%%%%%%%%%%%%%%%%%%%%%%%%%%%%%%%%%%%%%%%%%%%%%%%%%
\subsection{Forwarding}
\label{sec:forward}

Different versions of the main or child documents
using compilation flags as described in \secref{sec:flags}
can be (permanently) stored in different files
for convenient compilation, viewing and distribution.
To this end, the package defines a command
to pass on compilation to a different file:

%%%%%%%%%%%%%%%%%%%%%%%%%%%%%%%%%%%%%%%%
\DescribeMacro{\childdocforward}
The command |\childdocforward| redirects processing to
another source file:
%
\begin{center}
\begin{tabular}{l}
|\input{childdoc.def}|\\
|\childdocforward[|\textit{main}|]{|\textit{dest}|}|\\
\end{tabular}
\end{center}
%
The argument \textit{dest} is the destination file
(without extension).
It should be the main file or one of the child files.
Note that further \textsf{childdoc} directives
such as |\childdocof| and |\childdocforward|
in the indicated file will be processed in this form.
The optional argument \textit{main}
passes on directly to the main file \textit{main}
while pretending to compile the child \textit{dest}.
This form behaves as if \textit{dest}
issues |\childdocof{|\textit{main}|}| right away,
and no further \textsf{childdoc} directives will be processed.

%%%%%%%%%%%%%%%%%%%%%%%%%%%%%%%%%%%%%%%%
\DescribeMacro{\...prefix}
In the alternative form |\childdocforwardprefix|,
%
\begin{center}
\begin{tabular}{l}
|\input{childdoc.def}|\\
|\childdocforwardprefix[|\textit{main}|]{|\textit{prefix}|}{|\textit{dest}|}|
\end{tabular}
\end{center}
%
the destination file is determined by a pattern
depending on the current file:
To make this work, the current file must be called
`{\textit{prefix}\hspace{0.2em}\textit{suffix}}'
with \textit{prefix} matching precisely the argument.
Processing is then passed on to the file
`{\textit{dest}\hspace{0.2em}\textit{suffix}}'.
Surely, the same effect is achieved by
directly specifying the
argument `{\textit{dest}\hspace{0.2em}\textit{suffix}}'
in the first form.
However, that requires to set up a different file
for each child. With the alternative form of the command
all these files can have exactly the same content
which simplifies setting them up and maintaining them.

For example, the following file |draft.tex|
with a compilation flag |\version| as described in \secref{sec:flags}
compiles the main document as a draft:
%
\begin{center}
\begin{tabular}{l}
|\def\version{draft}|\\
|\input{childdoc.def}|\\
|\childdocforward{|\textit{main}|}|
\end{tabular}
\end{center}
%
Likewise, the following files |final|\textit{nn}|.tex|
compile the final version of the child document
|child|\textit{nn}|.tex|:
%
\begin{center}
\begin{tabular}{l}
|\def\version{final}|\\
|\input{childdoc.def}|\\
|\childdocforwardprefix{final}{child}|
\end{tabular}
\end{center}
%

Note that when several versions of a main file and/or of each child file
are to be generated, it may be convenient to set up a |Makefile| or
shell script to automatise the process.

%%%%%%%%%%%%%%%%%%%%%%%%%%%%%%%%%%%%%%%%%%%%%%%%%%%%%%%%%%%%%%%%%%%%%%%%%%%%%%%%
\subsection{Command Line Processing}
\label{sec:commandline}

The effect of redirection files can also be achieved by invoking
the \LaTeX{} compiler with a more elaborate command line.
Most conveniently this should be done as part
of a shell script or a |Makefile|.

When using \textsf{childdoc} in the main file, the following
command lines effectively perform a redirection
(note that depending on the shell being used,
backslashes may have to be doubled: `|\|' $\to$ `|\\|'):
%
\begin{center}
|... -jobname "|\textit{target}|" |\\|"|[\textit{flags}]%
|\input{childdoc.def}\childdocforward[|\textit{main}|]{|\textit{dest}|}"|
\end{center}
%
Here \textit{target} is the name of the output file,
\textit{main} is the name of the main file
and \textit{dest} is the name of the main or child file to be processed
(all filenames without extensions).
The optional argument \textit{main} can be omitted
if \textit{main} matches \textit{dest}.
Optionally, compilation \textit{flags} can be defined via |\def| commands.
This command line makes the \TeX{} engine believe
it is compiling the file \textit{target}
whose content is specified as the latter parameter.
The provided code then forwards the processing to
\textit{main} or \textit{dest} as described in \secref{sec:forward}.

%%%%%%%%%%%%%%%%%%%%%%%%%%%%%%%%%%%%%%%%%%%%%%%%%%%%%%%%%%%%%%%%%%%%%%%%%%%%%%%%
\subsection{Include by Input}
\label{sec:input}

Including child documents by |\include| has some restrictions by design.
Most notably, the content of a child document always occupies
its own set of pages; pages cannot be shared between child documents.
Usually, this behaviour makes perfect sense
because each child document contain an essential part of the document.
However, in some situations it may be desirable to compose
a document from a collection of parts
without having mandatory page breaks between then.
For this case, the package
provides a mechanism to include parts
by |\input| which can also be processed individually.
However, by construction this mechanism
requires manual handling of the content to be output.

%%%%%%%%%%%%%%%%%%%%%%%%%%%%%%%%%%%%%%%%
\DescribeMacro{\ifchilddocmanual}
The main file should be prepared as usual, see \secref{sec:include}.
However, the document body must make a distinction
between processing of an individual part and of the main document, e.g.:
%
\begin{center}
\begin{tabular}{l}
|\ifchilddocmanual|\\
|\input{\childdocname}|\\
|\||else|\\
\textit{document body with }|\input{|\textit{part}|}|\\
|\||fi|
\end{tabular}
\end{center}
%
The conditional |\ifchilddocmanual| is true whenever
a part to be included by |\input| is being compiled,
and the name of the part is stored in |\childdocname|.

%%%%%%%%%%%%%%%%%%%%%%%%%%%%%%%%%%%%%%%%
\DescribeMacro{\childdocby}
Each part to be included by |\input| should start with:
%
\begin{center}
\begin{tabular}{l}
|\input{childdoc.def}|\\
|\childdocby{|\textit{main}|}|\\
\end{tabular}
\end{center}
%
The directive |\childdocby| is similar to |\childdocof|
described in \secref{sec:include},
but the subsequent selection of content must be done manually.
To that end, both |\ifchilddoc| and |\ifchilddocmanual|
will be true upon processing of a part,
and the name of the part is stored in |\childdocname|.
Note that |\jobname| will be set to the filename of the current part
so that each part receives an individual |.aux| file
that does not interfere with the |.aux| file(s) of the main document.
This behaviour can be altered by the alternative form
|\childdocby[*]{|\textit{main}|}| (with a non-empty optional argument)
which uses the |.aux| file of the main document
by setting |\jobname| to \textit{main}.

%%%%%%%%%%%%%%%%%%%%%%%%%%%%%%%%%%%%%%%%%%%%%%%%%%%%%%%%%%%%%%%%%%%%%%%%%%%%%%%%
\subsection{Driver Development}
\label{sec:driver}

The \textsf{childdoc} mechanism can also be use for the development
of definition files such as \LaTeX{} styles or classes.
This case differs from the above setup with multiple parts
included by |\include| in that no |\includeonly| should be invoked.
This can be achieved by starting the include file
(before |\ProvidesPackage|) with:
%
\begin{center}
\begin{tabular}{l}
|\input{childdoc.def}|\\
|\childdocforward{|\textit{main}|}|\\
\end{tabular}
\end{center}
%
or alternatively with:
%
\begin{center}
\begin{tabular}{l}
|\input{childdoc.def}|\\
|\childdocby{|\textit{main}|}|\\
\end{tabular}
\end{center}
%
Both forms have slightly different effects as described above.
The main file is prepared as usual, see \secref{sec:include}.

%%%%%%%%%%%%%%%%%%%%%%%%%%%%%%%%%%%%%%%%%%%%%%%%%%%%%%%%%%%%%%%%%%%%%%%%%%%%%%%%
\subsection{Legacy Detection}
\label{sec:detection}

The directive |\childdocmain| in the main file can detect
whether the complete document or merely a child is to be compiled
even without using the directive |\childdocof|.
This method is deprecated because it is less robust
and there is no compelling reason to use it;
it is merely provided for backward compatibility
and it may be removed in future versions.

If the detection mechanism is to be used,
it is mandatory to correctly specify
the filename of the main file as the argument of |\childdocmain|:
%
\begin{center}
\begin{tabular}{l}
|\input{childdoc.def}|\\
|\childdocmain{|\textit{main}|}|\\
\end{tabular}
\end{center}
%
If |\jobname| does not match the argument \textit{main} of |\childdocmain|,
it is assumed that |\jobname| points to the child file to be compiled.
When using |\childdocmain| with the main file specified as argument,
it suffices to start a child file
with just |\input{|\textit{main}|}|
without loading of the package and using |\childdocof|.
If instead all processing is done
with the appropriate \textsf{childdoc} directives,
the argument of \textit{main} of |\childdocmain| can be empty.

An alternative version of the command line processing described
in \secref{sec:commandline} using the detection mechanism reads:
%
\begin{center}
|... -jobname "|\textit{target}|" "|[\textit{flags}]%
[|\def\jobname{|\textit{dest}|}|]|\input{|\textit{main}|}"|
\end{center}

%%%%%%%%%%%%%%%%%%%%%%%%%%%%%%%%%%%%%%%%%%%%%%%%%%%%%%%%%%%%%%%%%%%%%%%%%%%%%%%%
\subsection{Manual Code}
\label{sec:manual}

In case one cannot be certain whether the definitions file |childdoc.def|
is installed on the target \TeX{} distribution
and one prefers not to ship it,
it is conceivable to paste a few relevant commands into the sources.

To that end, drop all statements |\input{childdoc.def}|
and perform the replacements as outlined below.
Instead of |\childdocmain{|\textit{main}|}| add the following code
to the top of the main file:
%
\begin{center}
\begin{tabular}{l}
|\||ifdefined\childdocname\endinput\||fi\newif\ifchilddoc|\\
|\edef\childdocname{\scantokens\expandafter{\jobname\noexpand}}|\\
|\def\childdocmain{|\textit{main}|}\||ifx\childdocmain\childdocname\||else|\\
|\childdoctrue\includeonly{\childdocname}\let\jobname\childdocmain\||fi|\\
\end{tabular}
\end{center}
%
Instead of |\childdocof{|\textit{main}|}| just include the main file
at the top of each child file:
%
\begin{center}
|\input{|\textit{main}|}|
\end{center}
%
A simple redirection |\childdocforward{|\textit{dest}|}| is achieved by:
%
\begin{center}
|\def\jobname{|\textit{dest}|}\input{\jobname}|
\end{center}
%
The redirection with prefix
|\childdocforwardprefix[|\textit{prefix}|]{|\textit{dest}|}|
is accomplished by:
%
\begin{center}
\begin{tabular}{l}
|{\edef\jobname{\scantokens\expandafter{\jobname\noexpand}}|\\
|\def\redirectjob |\textit{prefix}|#1~~~{\gdef\jobname{|\textit{dest}|#1}}|\\
|\expandafter\redirectjob\jobname~~~}\input{\jobname}|
\end{tabular}
\end{center}

In an alternative approach,
child documents can be compiled by a specific command line
without additional code or specific definitions:
%
\begin{center}
|... -jobname "|\textit{target}|" "|[\textit{flags}]%
|\includeonly{|\textit{dest}|}\input{|\textit{main}|}"|
\end{center}
%

%%%%%%%%%%%%%%%%%%%%%%%%%%%%%%%%%%%%%%%%%%%%%%%%%%%%%%%%%%%%%%%%%%%%%%%%%%%%%%%%
%%%%%%%%%%%%%%%%%%%%%%%%%%%%%%%%%%%%%%%%%%%%%%%%%%%%%%%%%%%%%%%%%%%%%%%%%%%%%%%%
\section{Information}

%%%%%%%%%%%%%%%%%%%%%%%%%%%%%%%%%%%%%%%%%%%%%%%%%%%%%%%%%%%%%%%%%%%%%%%%%%%%%%%%
\subsection{Copyright}

Copyright \copyright{} 2017--2018 Niklas Beisert

This work may be distributed and/or modified under the
conditions of the \LaTeX{} Project Public License, either version 1.3
of this license or (at your option) any later version.
The latest version of this license is in
  \url{http://www.latex-project.org/lppl.txt}
and version 1.3 or later is part of all distributions of \LaTeX{}
version 2005/12/01 or later.

This work has the LPPL maintenance status `maintained'.

The Current Maintainer of this work is Niklas Beisert.

This work consists of the files |README.txt|, |childdoc.ins| and |childdoc.dtx|
as well as the derived files |childdoc.def|, |cdocsamp.tex|
with |cdocsch1.tex|, |cdocsch2.tex|, |cdocspt3.tex|, |cdocspt4.tex|,
|cdocsdrf.tex|, |cdocsfn1.tex|, |cdocsfn2.tex|
as well as |childdoc.pdf|.

%%%%%%%%%%%%%%%%%%%%%%%%%%%%%%%%%%%%%%%%%%%%%%%%%%%%%%%%%%%%%%%%%%%%%%%%%%%%%%%%
\subsection{Files and Installation}

The package consists of the files:
%
\begin{center}
\begin{tabular}{ll}
    |README.txt|   & readme file \\
    |childdoc.ins| & installation file \\
    |childdoc.dtx| & source file \\
    |childdoc.def| & definition file \\
    |cdocsamp.tex| & sample main file \\
    |cdocsch1.tex| & sample include file \\
    |cdocsch2.tex| & sample include file \\
    |cdocspt3.tex| & sample part file \\
    |cdocspt4.tex| & sample part file \\
    |cdocsdrf.tex| & sample redirection file \\
    |cdocsfn1.tex| & sample redirection file \\
    |cdocsfn2.tex| & sample redirection file \\
    |childdoc.pdf| & manual
\end{tabular}
\end{center}
%
The distribution consists of the files
|README.txt|, |childdoc.ins| and |childdoc.dtx|.
%
\begin{itemize}
\item
Run (pdf)\LaTeX{} on |childdoc.dtx|
to compile the manual |childdoc.pdf| (this file).
\item
Run \LaTeX{} on |childdoc.ins| to create the definitions file |childdoc.def|
and the sample |cdocsamp.tex| with include files
|cdocsch1.tex|, |cdocsch2.tex|, |cdocspt3.tex|, |cdocspt4.tex|,
|cdocsdrf.tex|, |cdocsfn1.tex|, |cdocsfn2.tex|.
Then copy the file |childdoc.def| to an appropriate directory of your \LaTeX{}
distribution, e.g.\ \textit{texmf-root}|/tex/latex/childdoc|.
\end{itemize}

%%%%%%%%%%%%%%%%%%%%%%%%%%%%%%%%%%%%%%%%%%%%%%%%%%%%%%%%%%%%%%%%%%%%%%%%%%%%%%%%
\subsection{Related CTAN Packages}

There are several other packages which offer a similar functionality:
%
\begin{itemize}
\item
The packages
\href{http://ctan.org/pkg/docmute}{\textsf{docmute}},
\href{http://ctan.org/pkg/includex}{\textsf{includex}} and
\href{http://ctan.org/pkg/standalone}{\textsf{standalone}}
provide commands to include only the document body of
a child file thus allowing both files to be compiled individually.
\item
The packages \href{http://ctan.org/pkg/subdocs}{\textsf{subdocs}}
and \href{http://ctan.org/pkg/subfiles}{\textsf{subfiles}}
provide structures in which the main and child documents can be
encapsulated and allowing them to be compiled individually.
The inclusion mechanism is different from the conventional |\include|.
\item
The package \href{http://ctan.org/pkg/combine}{\textsf{combine}}
is an elaborate solution to combine several documents into one.
\end{itemize}
%
See also the CTAN topic \href{http://ctan.org/topic/subdocs}{\textsf{subdocs}}
for further related packages.
The present package differs from the above solutions in that
a document structure constructed with the conventional |\include| mechanism
just needs two extra commands at the top of every file
such that all constituent files can be compiled individually.

%%%%%%%%%%%%%%%%%%%%%%%%%%%%%%%%%%%%%%%%%%%%%%%%%%%%%%%%%%%%%%%%%%%%%%%%%%%%%%%%
%\subsection{Feature Suggestions}
%
%The following is a list of features which may be useful for future
%versions of this package:
%%
%\begin{itemize}
%\item
%\ldots
%\end{itemize}

%%%%%%%%%%%%%%%%%%%%%%%%%%%%%%%%%%%%%%%%%%%%%%%%%%%%%%%%%%%%%%%%%%%%%%%%%%%%%%%%
\subsection{Revision History}

%%%%%%%%%%%%%%%%%%%%%%%%%%%%%%%%%%%%%%%%
\paragraph{v2.0:} 2018/12/30

\begin{itemize}
\item
immediate forward processing
\item
added |\childdocby| mechanism
\item
manual restructured
\end{itemize}

%%%%%%%%%%%%%%%%%%%%%%%%%%%%%%%%%%%%%%%%
\paragraph{v1.6:} 2018/01/17

\begin{itemize}
\item
application for development of include files
\item
corrections to manual
\end{itemize}

%%%%%%%%%%%%%%%%%%%%%%%%%%%%%%%%%%%%%%%%
\paragraph{v1.5:} 2017/05/21

\begin{itemize}
\item
more complete structuring introduced
\item
|\childdocof| introduced
\item
|\childdoc| renamed to |\childdocmain|
\item
|\childredirect| renamed to |\childdocforward| and |\childdocforwardprefix|
and functionality expanded
\end{itemize}

%%%%%%%%%%%%%%%%%%%%%%%%%%%%%%%%%%%%%%%%
\paragraph{v1.0:} 2017/04/27

\begin{itemize}
\item
manual and install package
\item
first version published on CTAN
\end{itemize}

%%%%%%%%%%%%%%%%%%%%%%%%%%%%%%%%%%%%%%%%
\paragraph{v0.6:} 2017/04/26

\begin{itemize}
\item
redirection mechanism added
\end{itemize}

%%%%%%%%%%%%%%%%%%%%%%%%%%%%%%%%%%%%%%%%
\paragraph{v0.5:} 2017/04/26

\begin{itemize}
\item
functionality in definition file
\end{itemize}


%%%%%%%%%%%%%%%%%%%%%%%%%%%%%%%%%%%%%%%%%%%%%%%%%%%%%%%%%%%%%%%%%%%%%%%%%%%%%%%%
%%%%%%%%%%%%%%%%%%%%%%%%%%%%%%%%%%%%%%%%%%%%%%%%%%%%%%%%%%%%%%%%%%%%%%%%%%%%%%%%
%%%%%%%%%%%%%%%%%%%%%%%%%%%%%%%%%%%%%%%%%%%%%%%%%%%%%%%%%%%%%%%%%%%%%%%%%%%%%%%%
\appendix

\settowidth\MacroIndent{\rmfamily\scriptsize 000\ }

 \DocInput{childdoc.dtx}

\end{document}
%</driver>
% \fi
%
% %%%%%%%%%%%%%%%%%%%%%%%%%%%%%%%%%%%%%%%%%%%%%%%%%%%%%%%%%%%%%%%%%%%%%%%%%%%%%%
% %%%%%%%%%%%%%%%%%%%%%%%%%%%%%%%%%%%%%%%%%%%%%%%%%%%%%%%%%%%%%%%%%%%%%%%%%%%%%%
% \section{Sample}
%\iffalse
%<*samplemain>
%\fi
%
% The following presents a sample document
% with two chapters, two parts, a title page,
% a compile flag as well as three forwarding files to set the flag.
% It consists of eight |.tex| files:
% \begin{center}
% \begin{tabular}{ll}
% |cdocsamp.tex|&main file\\
% |cdocsch1.tex|&include file for chapter 1\\
% |cdocsch2.tex|&include file for chapter 2\\
% |cdocspt3.tex|&include file for part 3\\
% |cdocspt4.tex|&include file for part 4\\
% |cdocsdrf.tex|&forwarding file for main file in draft mode\\
% |cdocsfi1.tex|&forwarding file for final version of chapter 1\\
% |cdocsfi2.tex|&forwarding file for final version of chapter 2\\
% \end{tabular}
% \end{center}
% Each of the eight files can be compiled directly by the \LaTeX{} compiler.
%
% %%%%%%%%%%%%%%%%%%%%%%%%%%%%%%%%%%%%%%
% \paragraph{Main File.}
%
% The main file is called |cdocsamp.tex|.
%
% Load the \textsf{childdoc} definitions and
% declare the filename for the main document:
%    \begin{macrocode}
\input{childdoc.def}
\childdocmain{}
%    \end{macrocode}

% Optional override for |\version| flag:
%    \begin{macrocode}
%%\ifchilddoc\else\providecommand{\version}{draft}\fi
%    \end{macrocode}

% Define the default values for the |\version| flag
% (|final| for the main file and |draft| for childs):
%    \begin{macrocode}
\ifchilddoc
\providecommand{\version}{draft}
\else
\providecommand{\version}{final}
\fi
%    \end{macrocode}

% Load the standard document class:
%    \begin{macrocode}
\documentclass[12pt]{article}
%    \end{macrocode}

% Start the document body:
%    \begin{macrocode}
\begin{document}
%    \end{macrocode}

% Declare a title page.
% Print title, part of document being processed and version flag:
%    \begin{macrocode}
\addtocounter{page}{-1}
\begin{center}
{\LARGE\bfseries{}childdoc example\par}
\vspace{1cm}
\ifchilddoc
\ifchilddocmanual part\else chapter\fi:
`\childdocname' of `\childdocjob'\par
\else
main document: `\childdocjob'\par
\fi
version: \version\par
\end{center}
\newpage
%    \end{macrocode}

% Manually include selected file,
% otherwise process as usual:
%    \begin{macrocode}
\ifchilddocmanual
\section*{part `\childdocname'}
\input{\childdocname}
\else
%    \end{macrocode}

% Include the two chapters:
%    \begin{macrocode}
\include{cdocsch1}
\include{cdocsch2}
%    \end{macrocode}

% Include the two parts unless only chapters should be displayed:
%    \begin{macrocode}
\ifchilddoc\else
\section{part three}
\input{cdocspt3}
\section{part four}
\input{cdocspt4}
\fi
%    \end{macrocode}

% Process as usual until here:
%    \begin{macrocode}
\fi
%    \end{macrocode}

% End of document body:
%    \begin{macrocode}
\end{document}
%    \end{macrocode}
%\iffalse
%</samplemain>
%\fi
%
% %%%%%%%%%%%%%%%%%%%%%%%%%%%%%%%%%%%%%%
% \paragraph{Chapter Include Files.}
%
% The include files are called |cdocsch1.tex| and |cdocsch2.tex|.
%
%\iffalse
%<*samplechap1|samplechap2>
%\fi

% Optional override for |\version| flag:
%    \begin{macrocode}
%%\providecommand{\version}{final}
%    \end{macrocode}

% Include the main document:
%    \begin{macrocode}
\input{childdoc.def}
\childdocof{cdocsamp}
%    \end{macrocode}

%\iffalse
%</samplechap1|samplechap2>
%\fi
%
%\iffalse
%<*samplechap1>
%\fi
% Some text for chapter 1:
%    \begin{macrocode}
\section{one}
some text in chapter one
%    \end{macrocode}

%\iffalse
%</samplechap1>
%\fi
% Some text for chapter 2:
%\iffalse
%<*samplechap2>
%\fi
%    \begin{macrocode}
\section{two}
more text in chapter two
%    \end{macrocode}

%\iffalse
%</samplechap2>
%\fi
%
% %%%%%%%%%%%%%%%%%%%%%%%%%%%%%%%%%%%%%%
% \paragraph{Part Include Files.}
%
% The include files are called |cdocspt3.tex| and |cdocspt4.tex|.
%
%\iffalse
%<*samplepart3|samplepart4>
%\fi

% Optional override for |\version| flag:
%    \begin{macrocode}
%%\providecommand{\version}{final}
%    \end{macrocode}

% Include the main document:
%    \begin{macrocode}
\input{childdoc.def}
\childdocby{cdocsamp}
%    \end{macrocode}

%\iffalse
%</samplepart3|samplepart4>
%\fi
%
%\iffalse
%<*samplepart3>
%\fi
% Some text for part 3:
%    \begin{macrocode}
some text in part three
%    \end{macrocode}

%\iffalse
%</samplepart3>
%\fi
% Some text for part 4:
%\iffalse
%<*samplepart4>
%\fi
%    \begin{macrocode}
more text in part four
%    \end{macrocode}

%\iffalse
%</samplepart4>
%\fi
%
% %%%%%%%%%%%%%%%%%%%%%%%%%%%%%%%%%%%%%%
% \paragraph{Forwarding for a Complete Draft.}
%
% The following forwarding file |cdocsdrf.tex|
% compiles the main document in draft mode:
%\iffalse
%<*sampledraft>
%\fi
%    \begin{macrocode}
\def\version{draft}
\input{childdoc.def}
\childdocforward{cdocsamp}
%    \end{macrocode}

%\iffalse
%</sampledraft>
%\fi
%
% %%%%%%%%%%%%%%%%%%%%%%%%%%%%%%%%%%%%%%
% \paragraph{Forwarding for Final Version of the Chapters.}
%
% The following forwarding files |cdocsfn1.tex| and |cdocsfn2.tex|
% (with identical content)
% compile the final versions of the child documents
% |cdocsch1.tex| and |cdocsch2.tex|, respectively:
%\iffalse
%<*samplefinal>
%\fi
%    \begin{macrocode}
\def\version{final}
\input{childdoc.def}
\childdocforwardprefix[cdocsamp]{cdocsfn}{cdocsch}
%    \end{macrocode}

%\iffalse
%</samplefinal>
%\fi
%
% %%%%%%%%%%%%%%%%%%%%%%%%%%%%%%%%%%%%%%
% \paragraph{Command Line Processing.}
%
% The following three command lines generate the output files
% |cdocscld|, |cdocscl1| and |cdocscl2|
% which should be identical to
% |cdocsdrf|, |cdocsch1| and |cdocsfn2|, respectively:
% \begin{center}
% \begin{tabular}{l}
% |latex -jobname cdocscld \|\\
% |  "\def\version{draft}\input{childdoc.def}\childdocforward{cdocsamp}"|\\
% |latex -jobname cdocscl1 \|\\
% |  "\input{childdoc.def}\childdocforward[cdocsamp]{cdocsch1}"|\\
% |latex -jobname cdocscl2 \|\\
% |  "\def\version{final}\input{childdoc.def}\childdocforward{cdocsch2}"|
% \end{tabular}
% \end{center}
% Note that the trailing backslash on each first line
% merely continues the input to the second line
% (for convenient cut ant paste).
% Furthermore, the command |latex| can be replaced by any
% of its alternative versions such as |pdflatex|.
%
% %%%%%%%%%%%%%%%%%%%%%%%%%%%%%%%%%%%%%%%%%%%%%%%%%%%%%%%%%%%%%%%%%%%%%%%%%%%%%%
% %%%%%%%%%%%%%%%%%%%%%%%%%%%%%%%%%%%%%%%%%%%%%%%%%%%%%%%%%%%%%%%%%%%%%%%%%%%%%%
% \section{Implementation}
%\iffalse
%<*package>
%\fi
%
% This section describes the definitions file |childdoc.def|.

% The definitions cannot be loaded using |\usepackage| or |\RequirePackage|
% which has a mechanism to prevent loading a style file more than once.
% When loading the definitions by means of |\input|
% multiple instances have to be prevented manually:
%\iffalse
%This code needs to be before the `\ProvidesFile' directive
%which is defined at the beginning of this file.
%Therefore it is also placed there and commented out here.
%</package>
%<*discard>
%\fi
%    \begin{macrocode}
\ifdefined\childdocmain\endinput\fi
%    \end{macrocode}
%\iffalse
%</discard>
%<*package>
%\fi
%
% \macro{\ifchilddoc}
% \macro{\ifchilddocmanual}
% The conditional |\ifchilddoc| tells whether a
% child (true) or main (false) document is being compiled.
% The conditional |\ifchilddocmanual| tells whether
% the |\includeonly| mechanism is used (false) or
% the selection of child files must be performed manually (true).
% The definitions initialise to false:
%    \begin{macrocode}
\newif\ifchilddoc
\newif\ifchilddocmanual
%    \end{macrocode}

% \macro{\childdocname}
% \macro{\childdocjob}
% The macro |\childdocname| stores the name of the main document
% to be compiled. The macro |\childdocjob| stores the name of
% the document on which the \LaTeX{} compiler was originally invoked.
% The content of |\jobname| cannot be compared
% to filenames specified in the source due to different catcodes.
% The following code rescans |\jobname|, stores the result
% in |\childdocname| and saves a copy in |\childdocjob|:
%    \begin{macrocode}
\edef\childdocname{\scantokens\expandafter{\jobname\noexpand}}
\let\childdocjob\childdocname
%    \end{macrocode}

% \macro{\childdocdisable}
% The macro |\childdocdisable| prevents the main file
% from being processed more than once.
% At this stage, the main document command |\childdocmain|
% is assumed to be called once again where it should do nothing.
% Any subsequent call to it should prevent
% a secondary processing of the main document
% It overwrites the forwarding commands
% |\childdocof| and |\childdocforward|
% with empty macros to prevent further inclusions of the main document:
%    \begin{macrocode}
\newcommand{\childdocdisable}
{
  \renewcommand{\childdocmain}[1]{\renewcommand{\childdocmain}[1]{\endinput}}
  \renewcommand{\childdocof}[1]{}
  \renewcommand{\childdocby}[2][]{}
  \renewcommand{\childdocforward}[2][]{}
  \renewcommand{\childdocdisable}{}
}
%    \end{macrocode}

% \macro{\childdocmain}
% The macro |\childdocmain| is to be called at the top of the main file
% with nothing or the main filename (without extension) as argument.
% First, it breaks loops.
% If the argument is not empty and does not match |\childdocname|
% (which is set by the first inclusion of |childdoc.def|),
% |\ifchilddoc| is set to true, |\includeonly| is applied to the child file
% and |\jobname| is set to the main file
% (for proper handling of |.aux| files):
%    \begin{macrocode}
\newcommand{\childdocmain}[1]
{
  \childdocdisable\childdocmain{}
  \if?#1?\else
    \begingroup
      \def\childdoctmp{#1}
      \ifx\childdoctmp\childdocname
        \def\childdoctmp{}
      \else
        \def\childdoctmp
        {
          \childdoctrue
          \includeonly{\childdocname}
          \def\childdocjob{#1}
          \def\jobname{#1}
        }
      \fi
      \expandafter
    \endgroup
    \childdoctmp
  \fi
}
%    \end{macrocode}

% \macro{\childdocof}
% The command |\childdocof| redirects
% compilation to the main file |#1|.
%    \begin{macrocode}
\newcommand{\childdocof}[1]
{
  \childdocdisable
  \childdoctrue
  \includeonly{\childdocname}
  \def\jobname{#1}
  \def\childdocjob{#1}
  \input{#1}
}
%    \end{macrocode}

% \macro{\childdocby}
% The command |\childdocby| ....
%    \begin{macrocode}
\newcommand{\childdocby}[2][]
{
  \childdocdisable
  \childdoctrue
  \childdocmanualtrue
  \if?#1?\else
    \def\jobname{#2}
  \fi
  \def\childdocjob{#2}
  \input{#2}
  \endinput
}
%    \end{macrocode}

% \macro{\childdocforward}
% The command |\childdocforward| redirects
% compilation to the main file or
% (if the optional argument is given) a child file.
% Parameters are set as if the main file
% or a child file starting with |\childdocof| was compiled.
% Then compilation is handed over to the main file:
%    \begin{macrocode}
\newcommand{\childdocforward}[2][]
{
  \begingroup
    \if?#1?
      \def\childdoctmp
      {
        \def\childdocname{#2}
        \def\childdocjob{#2}
        \def\jobname{#2}
        \input{#2}
        \endinput
      }
    \else
      \def\childdoctmp
      {
        \childdocdisable
        \def\childdocname{#2}
        \childdoctrue
        \includeonly{#2}
        \def\childdocjob{#1}
        \def\jobname{#1}
        \input{#1}
        \endinput
      }
    \fi
    \expandafter
  \endgroup
  \childdoctmp
}
%    \end{macrocode}

% \macro{\childdocforwardprefix}
% The command |\childdocforwardprefix| redirects
% compilation to the main or a child file by means of a pattern.
% The prefix |#1| in the current filename is replaced by |#2|
% and the suffix of the current filename is kept
% (it is assumed that the filename does not contain the substring `|~~~|'
% which is used as a delimiter).
% Compilation is handed over to the new file by |\childdocforward|:
%    \begin{macrocode}
\newcommand{\childdocforwardprefix}[3][]
{
  \begingroup
    \def\childdocextract #2##1~~~{\def\childdoctmp{\childdocforward[#1]{#3##1}}}
    \expandafter\childdocextract\childdocname~~~
    \expandafter
  \endgroup
  \childdoctmp
}
%    \end{macrocode}

% \macro{\childdoc}
% The deprecated macro |\childdoc| is a legacy version of |\childdocmain|:
%    \begin{macrocode}
\newcommand{\childdoc}{\childdocmain}
%    \end{macrocode}

% \macro{\childdocredirect}
% The deprecated macro |\childdocredirect| is a legacy version
% of |\childdocforward| and |\childdocforwardprefix|:
%    \begin{macrocode}
\newcommand{\childdocredirect}[2][]
{
  \begingroup
    \if?#1?
      \def\childdoctmp{\childdocforward{#2}}
    \else
      \def\childdoctmp{\childdocforwardprefix{#1}{#2}}
    \fi
    \expandafter
  \endgroup
  \childdoctmp
}
%    \end{macrocode}

%\iffalse
%</package>
%\fi
%
\endinput
|\\
|\childdocforwardprefix{final}{child}|
\end{tabular}
\end{center}
%

Note that when several versions of a main file and/or of each child file
are to be generated, it may be convenient to set up a |Makefile| or
shell script to automatise the process.

%%%%%%%%%%%%%%%%%%%%%%%%%%%%%%%%%%%%%%%%%%%%%%%%%%%%%%%%%%%%%%%%%%%%%%%%%%%%%%%%
\subsection{Command Line Processing}
\label{sec:commandline}

The effect of redirection files can also be achieved by invoking
the \LaTeX{} compiler with a more elaborate command line.
Most conveniently this should be done as part
of a shell script or a |Makefile|.

When using \textsf{childdoc} in the main file, the following
command lines effectively perform a redirection
(note that depending on the shell being used,
backslashes may have to be doubled: `|\|' $\to$ `|\\|'):
%
\begin{center}
|... -jobname "|\textit{target}|" |\\|"|[\textit{flags}]%
|% \iffalse
%
% childdoc.dtx Copyright (C) 2017-2018 Niklas Beisert
%
% This work may be distributed and/or modified under the
% conditions of the LaTeX Project Public License, either version 1.3
% of this license or (at your option) any later version.
% The latest version of this license is in
%   http://www.latex-project.org/lppl.txt
% and version 1.3 or later is part of all distributions of LaTeX
% version 2005/12/01 or later.
%
% This work has the LPPL maintenance status `maintained'.
%
% The Current Maintainer of this work is Niklas Beisert.
%
% This work consists of the files childdoc.dtx and childdoc.ins
% and the derived files childdoc.def and cdocsamp.tex with
% cdocsch1.tex, cdocsch2.tex, cdocsdrf.tex, cdocsfn1.tex, cdocsfn2.tex.
%
%<package>\ifdefined\childdocmain\endinput\fi
%<package>\ProvidesFile{childdoc.def}[2018/12/30 v2.0 child document driver]
%<samplemain>\ProvidesFile{cdocsamp.tex}[2018/12/30 v2.0 sample for childdoc]
%<*driver>
%\ProvidesFile{childdoc.drv}[2018/12/30 v2.0 childdoc reference manual file]
\PassOptionsToClass{10pt,a4paper}{article}
\documentclass{ltxdoc}

\usepackage[margin=35mm]{geometry}
\usepackage{hyperref}
\usepackage{hyperxmp}
\usepackage[usenames]{color}

\hypersetup{colorlinks=true}
\hypersetup{pdfstartview=FitH}
\hypersetup{pdfpagemode=UseNone}
\hypersetup{pdfsource={}}
\hypersetup{pdflang={en-UK}}
\hypersetup{pdfcopyright={Copyright 2017-2018 Niklas Beisert.
  This work may be distributed and/or modified under the
  conditions of the LaTeX Project Public License, either version 1.3
  of this license or (at your option) any later version.}}
\hypersetup{pdflicenseurl={http://www.latex-project.org/lppl.txt}}
\hypersetup{pdfcontactaddress={ETH Zurich, ITP, HIT K,
  Wolfgang-Pauli-Strasse 27}}
\hypersetup{pdfcontactpostcode={8093}}
\hypersetup{pdfcontactcity={Zurich}}
\hypersetup{pdfcontactcountry={Switzerland}}
\hypersetup{pdfcontactemail={nbeisert@itp.phys.ethz.ch}}
\hypersetup{pdfcontacturl={http://people.phys.ethz.ch/\xmptilde nbeisert/}}

\newcommand{\secref}[1]{\hyperref[#1]{section \ref*{#1}}}

\parskip1ex
\parindent0pt
\let\olditemize\itemize
\def\itemize{\olditemize\parskip0pt}

\begin{document}

\title{The \textsf{childdoc} Package}
\hypersetup{pdftitle={The childdoc Package}}
\author{Niklas Beisert\\[2ex]
  Institut f\"ur Theoretische Physik\\
  Eidgen\"ossische Technische Hochschule Z\"urich\\
  Wolfgang-Pauli-Strasse 27, 8093 Z\"urich, Switzerland\\[1ex]
  \href{mailto:nbeisert@itp.phys.ethz.ch}
  {\texttt{nbeisert@itp.phys.ethz.ch}}}
\hypersetup{pdfauthor={Niklas Beisert}}
\hypersetup{pdfsubject={Manual for the LaTeX2e Package childdoc}}
\date{30 December 2018, \textsf{v2.0}}
\maketitle

\begin{abstract}\noindent
\textsf{childdoc} is a \LaTeXe{} package
that enables the direct compilation
of document sections included by |\include|
to individual files.
\end{abstract}

\begingroup
\parskip0ex
\tableofcontents
\endgroup

%%%%%%%%%%%%%%%%%%%%%%%%%%%%%%%%%%%%%%%%%%%%%%%%%%%%%%%%%%%%%%%%%%%%%%%%%%%%%%%%
%%%%%%%%%%%%%%%%%%%%%%%%%%%%%%%%%%%%%%%%%%%%%%%%%%%%%%%%%%%%%%%%%%%%%%%%%%%%%%%%
\section{Introduction}

\LaTeX{} provides a mechanism to structure a large document (such as a book)
into a main file and several child files (containing the chapters)
using the |\include| command.
This mechanism is beneficial for documents
which span hundreds of pages in order to
make the source file(s) more manageable.
Moreover, compilation can be restricted to
selected child files by means of the |\includeonly| command.
The latter feature can be used to reduce the compilation time while editing
(this was significantly more useful in the earlier days of \LaTeX{})
or to generate a smaller document which is easier to navigate.
Another application of |\includeonly| is to generate
documents consisting of selected parts of the complete document.

However, there are a few drawbacks of the plain |\include| mechanism:
\begin{itemize}
\item
The child files cannot be compiled on their own,
they can only be compiled via the main file.
A naive editing environment
(such as a text editor with an option
to have the current file processed by \LaTeX)
may require one to switch to the main file before compiling;
attempting to compile the child file produces errors.
\item
The main file must be modified (each time)
to adjust the |\includeonly| command
to the present needs. This easily leaves the main file in a messy state.
\item
The generated document will always carry the filename
of the main document. This is inconvenient if
several child files are to be compiled and
to be kept for distribution.
\end{itemize}

The present package provides a simple interface
to make child files individually compilable by \LaTeX{}.
Compiling a child file then has the same effect as compiling
the main file with an |\includeonly| command
to select the appropriate child.
Moreover the generated document will carry the name of the child
rather than the main file.
This resolves all three above issues.

This feature is meant to make the editing of books,
thesis documents and lecture notes somewhat more convenient.
However, the package can also be used efficiently for
composing a series of documents (such as exercise sheets)
which are typically distributed individually.
It then assists the author in generating the individual documents
(potentially in different versions)
as well as a document containing the collected series.
Another application is in developing style files
or other kinds of included material
where compilation of the style file could redirect
to a sample or test file.

%%%%%%%%%%%%%%%%%%%%%%%%%%%%%%%%%%%%%%%%%%%%%%%%%%%%%%%%%%%%%%%%%%%%%%%%%%%%%%%%
%%%%%%%%%%%%%%%%%%%%%%%%%%%%%%%%%%%%%%%%%%%%%%%%%%%%%%%%%%%%%%%%%%%%%%%%%%%%%%%%
\section{Usage}

First of all, the package \textsf{childdoc} is \emph{not} a standard
\LaTeXe{} |.sty| style file! Therefore it needs to be invoked in
a non-standard way.

%%%%%%%%%%%%%%%%%%%%%%%%%%%%%%%%%%%%%%%%%%%%%%%%%%%%%%%%%%%%%%%%%%%%%%%%%%%%%%%%
\subsection{Included Files}
\label{sec:include}

%%%%%%%%%%%%%%%%%%%%%%%%%%%%%%%%%%%%%%%%
\DescribeMacro{\childdocmain}
To use the package, add the commands
\begin{center}
\begin{tabular}{l}
|\input{childdoc.def}|\\
|\childdocmain{}|\\
\end{tabular}
\end{center}
at the very top of the main \LaTeX{} file,
in particular \emph{before} the |\documentclass| statement!
The argument of |\childdocmain| should be left empty
(but it must be present).

%%%%%%%%%%%%%%%%%%%%%%%%%%%%%%%%%%%%%%%%
\DescribeMacro{\childdocof}
Furthermore, add the commands
\begin{center}
\begin{tabular}{l}
|\input{childdoc.def}|\\
|\childdocof{|\textit{main}|}|\\
\end{tabular}
\end{center}
at the top of every child file \textit{child}
which is included by |\include{|\textit{child}|}|
from within the main file
(or at least for those files to be compiled individually).
The argument \textit{main} must be the filename of the main file.

There are a couple of
considerations in setting up the main and child documents:

%%%%%%%%%%%%%%%%%%%%%%%%%%%%%%%%%%%%%%%%
\paragraph{Restrictions.}

Please note the following restrictions:
\begin{itemize}
\item
|\childdocmain| must be called with one argument \textit{main}
to ensure compatibility with earlier version of the package.
It must either be empty (|\childdocmain{}|)
or precisely match the filename of the main file in which it is specified.
See \secref{sec:detection} for further information.
\item
The filename \textit{main} must be specified without the |.tex| extension.
\item
The filename \textit{main} is case sensitive
(even in case-insensitive file systems)
due to internal string comparison.
\item
The argument \textit{main} should be fully expanded, it cannot be a macro.
\item
Subdirectories and special characters should be avoided in filenames.
\item
The command |\childdocmain{|\textit{main}|}| must be followed by a whitespace.
It should not be followed immediately by another command
or by a comment mark `|%|'.
This is because the \TeX{} parser reads the token immediately following
the argument of |\childdocmain| and puts it
at the beginning of every child section;
however, a white\-space is ignored.
\end{itemize}

%%%%%%%%%%%%%%%%%%%%%%%%%%%%%%%%%%%%%%%%
\paragraph{Content of Main File.}

It is advisable to place all content in the child files included by |\include|.
Any output contained in the main file will appear in all child documents
unless suppressed manually;
it cannot be suppressed automatically by the |\includeonly| directive
and thus should normally be avoided.
A method to include some content in the main file
by means of conditional processing is described in \secref{sec:conditional}.

%%%%%%%%%%%%%%%%%%%%%%%%%%%%%%%%%%%%%%%%
\paragraph{Page Numbering.}

When only a part of the document is compiled,
the appropriate numbering of pages
(as well as other status parameters)
is determined from the |.aux| files.
The latter contain information from previous passes.
However this information needs to propagate through
all intermediate child documents.
Therefore the page numbering in child documents may well
be inconsistent until the complete document is compiled at least once.

A useful (if unconventional) way to always ensure a consistent
page numbering is to restart the numbering in each child document
and denote the pages by `\textit{child}|.|\textit{page}'
where \textit{child} represents the chapter/section number of the child file.
This can be achieved by the command
|\numberwithin{page}{|\textit{child}|}|
of the \textsf{amsmath} package
where \textit{child} can be |chapter| or |section|
depending on the chosen structuring.
Alternatively, one can modify the macro |\thepage| appropriately
and reset the counter |page| at the start of each child file.

%%%%%%%%%%%%%%%%%%%%%%%%%%%%%%%%%%%%%%%%%%%%%%%%%%%%%%%%%%%%%%%%%%%%%%%%%%%%%%%%
\subsection{Conditional Processing}
\label{sec:conditional}

The package provides a mechanism to compile different versions
of a document. To customise the versions further some conditional processing
can come in handy to distinguish which version is being compiled.
The package provides two macros to describe the compilation context:

%%%%%%%%%%%%%%%%%%%%%%%%%%%%%%%%%%%%%%%%
\DescribeMacro{\ifchilddoc}
The conditional |\ifchilddoc| distinguishes between the compilation of
child documents and the main document:
%
\begin{center}
|\ifchilddoc |\textit{child-code}| |[|\||else |\textit{main-code}]| \||fi|
\end{center}

%%%%%%%%%%%%%%%%%%%%%%%%%%%%%%%%%%%%%%%%
\DescribeMacro{\childdocname}
\DescribeMacro{\childdocjob}
The macro |\childdocname| contains the filename (without extension)
of the main or child file being processed.
Note that |\childdocjob| will always contain the name of the main file.

%%%%%%%%%%%%%%%%%%%%%%%%%%%%%%%%%%%%%%%%
\paragraph{Title Page.}

Conditional processing can be used to include a title or banner page
in the main document when proper precautions are taken.
Importantly, the code in the main file should ensure that the page counter
(as well as other status parameters which are stored in the |.aux| files)
takes the same value after the conditional processing.
Otherwise the page numbers may take divergent values
depending on which part is compiled.

For example, a title page could be declared by:
%
\begin{center}
\begin{tabular}{l}
|\ifchilddoc\||else|\\
|\addtocounter{page}{-1}|\\
\textit{code for title page}\\
|\newpage|\\
|\||fi|
\end{tabular}
\end{center}
%
A banner page for the child documents can be generated by:
%
\begin{center}
\begin{tabular}{l}
|\ifchilddoc|\\
|\addtocounter{page}{-1}|\\
\textit{code for banner page}\\
|\newpage|\\
|\||fi|
\end{tabular}
\end{center}
%
Here one could write a message such as:
\begin{center}
|This is the part \childdocname{} of \childdocjob{}.|
\end{center}

%%%%%%%%%%%%%%%%%%%%%%%%%%%%%%%%%%%%%%%%%%%%%%%%%%%%%%%%%%%%%%%%%%%%%%%%%%%%%%%%
\subsection{Flags}
\label{sec:flags}

The package makes it easy to generate different versions
of the main or child documents.
To this end compilation flags can be defined
and assigned different default values.
They will be particularly useful in conjunction
with the forwarding mechanism described in \secref{sec:forward}.

For example, it may be useful to have a flag |\version|
which can be set to |draft| or |final|.
The document source will contain some conditional code
depending on the value of |\version|.
Suppose further, the flag should default to |final| for the main file
and to |draft| for child files
which is a natural assignment for editing the document.
This is achieved by placing the following code
in the preamble of the main document
(below the |\childdocmain| directive):
%
\begin{center}
\begin{tabular}{l}
|\ifchilddoc|\\
|\providecommand{\version}{draft}|\\
|\||else|\\
|\providecommand{\version}{final}|\\
|\||fi|
\end{tabular}
\end{center}
%
The definition by |\providecommand| makes sure
that previous definitions are not overwritten.
Further statements |\providecommand{\version}{...}|
can thus be added before the above code to override it.

For the main file, one might add a line
(between |\childdocmain| and the above block)
%
\begin{center}
|%\ifchilddoc\||else\providecommand{\version}{draft}\||fi|
\end{center}
%
which can be uncommented to produce a draft version.
Likewise one can add a line to the very top of a child file
(above the |\childdocof{|\textit{main}|}| directive)
%
\begin{center}
|%\providecommand{\version}{final}|
\end{center}
%
which can be uncommented to produce the final version of this child document.

%%%%%%%%%%%%%%%%%%%%%%%%%%%%%%%%%%%%%%%%%%%%%%%%%%%%%%%%%%%%%%%%%%%%%%%%%%%%%%%%
\subsection{Forwarding}
\label{sec:forward}

Different versions of the main or child documents
using compilation flags as described in \secref{sec:flags}
can be (permanently) stored in different files
for convenient compilation, viewing and distribution.
To this end, the package defines a command
to pass on compilation to a different file:

%%%%%%%%%%%%%%%%%%%%%%%%%%%%%%%%%%%%%%%%
\DescribeMacro{\childdocforward}
The command |\childdocforward| redirects processing to
another source file:
%
\begin{center}
\begin{tabular}{l}
|\input{childdoc.def}|\\
|\childdocforward[|\textit{main}|]{|\textit{dest}|}|\\
\end{tabular}
\end{center}
%
The argument \textit{dest} is the destination file
(without extension).
It should be the main file or one of the child files.
Note that further \textsf{childdoc} directives
such as |\childdocof| and |\childdocforward|
in the indicated file will be processed in this form.
The optional argument \textit{main}
passes on directly to the main file \textit{main}
while pretending to compile the child \textit{dest}.
This form behaves as if \textit{dest}
issues |\childdocof{|\textit{main}|}| right away,
and no further \textsf{childdoc} directives will be processed.

%%%%%%%%%%%%%%%%%%%%%%%%%%%%%%%%%%%%%%%%
\DescribeMacro{\...prefix}
In the alternative form |\childdocforwardprefix|,
%
\begin{center}
\begin{tabular}{l}
|\input{childdoc.def}|\\
|\childdocforwardprefix[|\textit{main}|]{|\textit{prefix}|}{|\textit{dest}|}|
\end{tabular}
\end{center}
%
the destination file is determined by a pattern
depending on the current file:
To make this work, the current file must be called
`{\textit{prefix}\hspace{0.2em}\textit{suffix}}'
with \textit{prefix} matching precisely the argument.
Processing is then passed on to the file
`{\textit{dest}\hspace{0.2em}\textit{suffix}}'.
Surely, the same effect is achieved by
directly specifying the
argument `{\textit{dest}\hspace{0.2em}\textit{suffix}}'
in the first form.
However, that requires to set up a different file
for each child. With the alternative form of the command
all these files can have exactly the same content
which simplifies setting them up and maintaining them.

For example, the following file |draft.tex|
with a compilation flag |\version| as described in \secref{sec:flags}
compiles the main document as a draft:
%
\begin{center}
\begin{tabular}{l}
|\def\version{draft}|\\
|\input{childdoc.def}|\\
|\childdocforward{|\textit{main}|}|
\end{tabular}
\end{center}
%
Likewise, the following files |final|\textit{nn}|.tex|
compile the final version of the child document
|child|\textit{nn}|.tex|:
%
\begin{center}
\begin{tabular}{l}
|\def\version{final}|\\
|\input{childdoc.def}|\\
|\childdocforwardprefix{final}{child}|
\end{tabular}
\end{center}
%

Note that when several versions of a main file and/or of each child file
are to be generated, it may be convenient to set up a |Makefile| or
shell script to automatise the process.

%%%%%%%%%%%%%%%%%%%%%%%%%%%%%%%%%%%%%%%%%%%%%%%%%%%%%%%%%%%%%%%%%%%%%%%%%%%%%%%%
\subsection{Command Line Processing}
\label{sec:commandline}

The effect of redirection files can also be achieved by invoking
the \LaTeX{} compiler with a more elaborate command line.
Most conveniently this should be done as part
of a shell script or a |Makefile|.

When using \textsf{childdoc} in the main file, the following
command lines effectively perform a redirection
(note that depending on the shell being used,
backslashes may have to be doubled: `|\|' $\to$ `|\\|'):
%
\begin{center}
|... -jobname "|\textit{target}|" |\\|"|[\textit{flags}]%
|\input{childdoc.def}\childdocforward[|\textit{main}|]{|\textit{dest}|}"|
\end{center}
%
Here \textit{target} is the name of the output file,
\textit{main} is the name of the main file
and \textit{dest} is the name of the main or child file to be processed
(all filenames without extensions).
The optional argument \textit{main} can be omitted
if \textit{main} matches \textit{dest}.
Optionally, compilation \textit{flags} can be defined via |\def| commands.
This command line makes the \TeX{} engine believe
it is compiling the file \textit{target}
whose content is specified as the latter parameter.
The provided code then forwards the processing to
\textit{main} or \textit{dest} as described in \secref{sec:forward}.

%%%%%%%%%%%%%%%%%%%%%%%%%%%%%%%%%%%%%%%%%%%%%%%%%%%%%%%%%%%%%%%%%%%%%%%%%%%%%%%%
\subsection{Include by Input}
\label{sec:input}

Including child documents by |\include| has some restrictions by design.
Most notably, the content of a child document always occupies
its own set of pages; pages cannot be shared between child documents.
Usually, this behaviour makes perfect sense
because each child document contain an essential part of the document.
However, in some situations it may be desirable to compose
a document from a collection of parts
without having mandatory page breaks between then.
For this case, the package
provides a mechanism to include parts
by |\input| which can also be processed individually.
However, by construction this mechanism
requires manual handling of the content to be output.

%%%%%%%%%%%%%%%%%%%%%%%%%%%%%%%%%%%%%%%%
\DescribeMacro{\ifchilddocmanual}
The main file should be prepared as usual, see \secref{sec:include}.
However, the document body must make a distinction
between processing of an individual part and of the main document, e.g.:
%
\begin{center}
\begin{tabular}{l}
|\ifchilddocmanual|\\
|\input{\childdocname}|\\
|\||else|\\
\textit{document body with }|\input{|\textit{part}|}|\\
|\||fi|
\end{tabular}
\end{center}
%
The conditional |\ifchilddocmanual| is true whenever
a part to be included by |\input| is being compiled,
and the name of the part is stored in |\childdocname|.

%%%%%%%%%%%%%%%%%%%%%%%%%%%%%%%%%%%%%%%%
\DescribeMacro{\childdocby}
Each part to be included by |\input| should start with:
%
\begin{center}
\begin{tabular}{l}
|\input{childdoc.def}|\\
|\childdocby{|\textit{main}|}|\\
\end{tabular}
\end{center}
%
The directive |\childdocby| is similar to |\childdocof|
described in \secref{sec:include},
but the subsequent selection of content must be done manually.
To that end, both |\ifchilddoc| and |\ifchilddocmanual|
will be true upon processing of a part,
and the name of the part is stored in |\childdocname|.
Note that |\jobname| will be set to the filename of the current part
so that each part receives an individual |.aux| file
that does not interfere with the |.aux| file(s) of the main document.
This behaviour can be altered by the alternative form
|\childdocby[*]{|\textit{main}|}| (with a non-empty optional argument)
which uses the |.aux| file of the main document
by setting |\jobname| to \textit{main}.

%%%%%%%%%%%%%%%%%%%%%%%%%%%%%%%%%%%%%%%%%%%%%%%%%%%%%%%%%%%%%%%%%%%%%%%%%%%%%%%%
\subsection{Driver Development}
\label{sec:driver}

The \textsf{childdoc} mechanism can also be use for the development
of definition files such as \LaTeX{} styles or classes.
This case differs from the above setup with multiple parts
included by |\include| in that no |\includeonly| should be invoked.
This can be achieved by starting the include file
(before |\ProvidesPackage|) with:
%
\begin{center}
\begin{tabular}{l}
|\input{childdoc.def}|\\
|\childdocforward{|\textit{main}|}|\\
\end{tabular}
\end{center}
%
or alternatively with:
%
\begin{center}
\begin{tabular}{l}
|\input{childdoc.def}|\\
|\childdocby{|\textit{main}|}|\\
\end{tabular}
\end{center}
%
Both forms have slightly different effects as described above.
The main file is prepared as usual, see \secref{sec:include}.

%%%%%%%%%%%%%%%%%%%%%%%%%%%%%%%%%%%%%%%%%%%%%%%%%%%%%%%%%%%%%%%%%%%%%%%%%%%%%%%%
\subsection{Legacy Detection}
\label{sec:detection}

The directive |\childdocmain| in the main file can detect
whether the complete document or merely a child is to be compiled
even without using the directive |\childdocof|.
This method is deprecated because it is less robust
and there is no compelling reason to use it;
it is merely provided for backward compatibility
and it may be removed in future versions.

If the detection mechanism is to be used,
it is mandatory to correctly specify
the filename of the main file as the argument of |\childdocmain|:
%
\begin{center}
\begin{tabular}{l}
|\input{childdoc.def}|\\
|\childdocmain{|\textit{main}|}|\\
\end{tabular}
\end{center}
%
If |\jobname| does not match the argument \textit{main} of |\childdocmain|,
it is assumed that |\jobname| points to the child file to be compiled.
When using |\childdocmain| with the main file specified as argument,
it suffices to start a child file
with just |\input{|\textit{main}|}|
without loading of the package and using |\childdocof|.
If instead all processing is done
with the appropriate \textsf{childdoc} directives,
the argument of \textit{main} of |\childdocmain| can be empty.

An alternative version of the command line processing described
in \secref{sec:commandline} using the detection mechanism reads:
%
\begin{center}
|... -jobname "|\textit{target}|" "|[\textit{flags}]%
[|\def\jobname{|\textit{dest}|}|]|\input{|\textit{main}|}"|
\end{center}

%%%%%%%%%%%%%%%%%%%%%%%%%%%%%%%%%%%%%%%%%%%%%%%%%%%%%%%%%%%%%%%%%%%%%%%%%%%%%%%%
\subsection{Manual Code}
\label{sec:manual}

In case one cannot be certain whether the definitions file |childdoc.def|
is installed on the target \TeX{} distribution
and one prefers not to ship it,
it is conceivable to paste a few relevant commands into the sources.

To that end, drop all statements |\input{childdoc.def}|
and perform the replacements as outlined below.
Instead of |\childdocmain{|\textit{main}|}| add the following code
to the top of the main file:
%
\begin{center}
\begin{tabular}{l}
|\||ifdefined\childdocname\endinput\||fi\newif\ifchilddoc|\\
|\edef\childdocname{\scantokens\expandafter{\jobname\noexpand}}|\\
|\def\childdocmain{|\textit{main}|}\||ifx\childdocmain\childdocname\||else|\\
|\childdoctrue\includeonly{\childdocname}\let\jobname\childdocmain\||fi|\\
\end{tabular}
\end{center}
%
Instead of |\childdocof{|\textit{main}|}| just include the main file
at the top of each child file:
%
\begin{center}
|\input{|\textit{main}|}|
\end{center}
%
A simple redirection |\childdocforward{|\textit{dest}|}| is achieved by:
%
\begin{center}
|\def\jobname{|\textit{dest}|}\input{\jobname}|
\end{center}
%
The redirection with prefix
|\childdocforwardprefix[|\textit{prefix}|]{|\textit{dest}|}|
is accomplished by:
%
\begin{center}
\begin{tabular}{l}
|{\edef\jobname{\scantokens\expandafter{\jobname\noexpand}}|\\
|\def\redirectjob |\textit{prefix}|#1~~~{\gdef\jobname{|\textit{dest}|#1}}|\\
|\expandafter\redirectjob\jobname~~~}\input{\jobname}|
\end{tabular}
\end{center}

In an alternative approach,
child documents can be compiled by a specific command line
without additional code or specific definitions:
%
\begin{center}
|... -jobname "|\textit{target}|" "|[\textit{flags}]%
|\includeonly{|\textit{dest}|}\input{|\textit{main}|}"|
\end{center}
%

%%%%%%%%%%%%%%%%%%%%%%%%%%%%%%%%%%%%%%%%%%%%%%%%%%%%%%%%%%%%%%%%%%%%%%%%%%%%%%%%
%%%%%%%%%%%%%%%%%%%%%%%%%%%%%%%%%%%%%%%%%%%%%%%%%%%%%%%%%%%%%%%%%%%%%%%%%%%%%%%%
\section{Information}

%%%%%%%%%%%%%%%%%%%%%%%%%%%%%%%%%%%%%%%%%%%%%%%%%%%%%%%%%%%%%%%%%%%%%%%%%%%%%%%%
\subsection{Copyright}

Copyright \copyright{} 2017--2018 Niklas Beisert

This work may be distributed and/or modified under the
conditions of the \LaTeX{} Project Public License, either version 1.3
of this license or (at your option) any later version.
The latest version of this license is in
  \url{http://www.latex-project.org/lppl.txt}
and version 1.3 or later is part of all distributions of \LaTeX{}
version 2005/12/01 or later.

This work has the LPPL maintenance status `maintained'.

The Current Maintainer of this work is Niklas Beisert.

This work consists of the files |README.txt|, |childdoc.ins| and |childdoc.dtx|
as well as the derived files |childdoc.def|, |cdocsamp.tex|
with |cdocsch1.tex|, |cdocsch2.tex|, |cdocspt3.tex|, |cdocspt4.tex|,
|cdocsdrf.tex|, |cdocsfn1.tex|, |cdocsfn2.tex|
as well as |childdoc.pdf|.

%%%%%%%%%%%%%%%%%%%%%%%%%%%%%%%%%%%%%%%%%%%%%%%%%%%%%%%%%%%%%%%%%%%%%%%%%%%%%%%%
\subsection{Files and Installation}

The package consists of the files:
%
\begin{center}
\begin{tabular}{ll}
    |README.txt|   & readme file \\
    |childdoc.ins| & installation file \\
    |childdoc.dtx| & source file \\
    |childdoc.def| & definition file \\
    |cdocsamp.tex| & sample main file \\
    |cdocsch1.tex| & sample include file \\
    |cdocsch2.tex| & sample include file \\
    |cdocspt3.tex| & sample part file \\
    |cdocspt4.tex| & sample part file \\
    |cdocsdrf.tex| & sample redirection file \\
    |cdocsfn1.tex| & sample redirection file \\
    |cdocsfn2.tex| & sample redirection file \\
    |childdoc.pdf| & manual
\end{tabular}
\end{center}
%
The distribution consists of the files
|README.txt|, |childdoc.ins| and |childdoc.dtx|.
%
\begin{itemize}
\item
Run (pdf)\LaTeX{} on |childdoc.dtx|
to compile the manual |childdoc.pdf| (this file).
\item
Run \LaTeX{} on |childdoc.ins| to create the definitions file |childdoc.def|
and the sample |cdocsamp.tex| with include files
|cdocsch1.tex|, |cdocsch2.tex|, |cdocspt3.tex|, |cdocspt4.tex|,
|cdocsdrf.tex|, |cdocsfn1.tex|, |cdocsfn2.tex|.
Then copy the file |childdoc.def| to an appropriate directory of your \LaTeX{}
distribution, e.g.\ \textit{texmf-root}|/tex/latex/childdoc|.
\end{itemize}

%%%%%%%%%%%%%%%%%%%%%%%%%%%%%%%%%%%%%%%%%%%%%%%%%%%%%%%%%%%%%%%%%%%%%%%%%%%%%%%%
\subsection{Related CTAN Packages}

There are several other packages which offer a similar functionality:
%
\begin{itemize}
\item
The packages
\href{http://ctan.org/pkg/docmute}{\textsf{docmute}},
\href{http://ctan.org/pkg/includex}{\textsf{includex}} and
\href{http://ctan.org/pkg/standalone}{\textsf{standalone}}
provide commands to include only the document body of
a child file thus allowing both files to be compiled individually.
\item
The packages \href{http://ctan.org/pkg/subdocs}{\textsf{subdocs}}
and \href{http://ctan.org/pkg/subfiles}{\textsf{subfiles}}
provide structures in which the main and child documents can be
encapsulated and allowing them to be compiled individually.
The inclusion mechanism is different from the conventional |\include|.
\item
The package \href{http://ctan.org/pkg/combine}{\textsf{combine}}
is an elaborate solution to combine several documents into one.
\end{itemize}
%
See also the CTAN topic \href{http://ctan.org/topic/subdocs}{\textsf{subdocs}}
for further related packages.
The present package differs from the above solutions in that
a document structure constructed with the conventional |\include| mechanism
just needs two extra commands at the top of every file
such that all constituent files can be compiled individually.

%%%%%%%%%%%%%%%%%%%%%%%%%%%%%%%%%%%%%%%%%%%%%%%%%%%%%%%%%%%%%%%%%%%%%%%%%%%%%%%%
%\subsection{Feature Suggestions}
%
%The following is a list of features which may be useful for future
%versions of this package:
%%
%\begin{itemize}
%\item
%\ldots
%\end{itemize}

%%%%%%%%%%%%%%%%%%%%%%%%%%%%%%%%%%%%%%%%%%%%%%%%%%%%%%%%%%%%%%%%%%%%%%%%%%%%%%%%
\subsection{Revision History}

%%%%%%%%%%%%%%%%%%%%%%%%%%%%%%%%%%%%%%%%
\paragraph{v2.0:} 2018/12/30

\begin{itemize}
\item
immediate forward processing
\item
added |\childdocby| mechanism
\item
manual restructured
\end{itemize}

%%%%%%%%%%%%%%%%%%%%%%%%%%%%%%%%%%%%%%%%
\paragraph{v1.6:} 2018/01/17

\begin{itemize}
\item
application for development of include files
\item
corrections to manual
\end{itemize}

%%%%%%%%%%%%%%%%%%%%%%%%%%%%%%%%%%%%%%%%
\paragraph{v1.5:} 2017/05/21

\begin{itemize}
\item
more complete structuring introduced
\item
|\childdocof| introduced
\item
|\childdoc| renamed to |\childdocmain|
\item
|\childredirect| renamed to |\childdocforward| and |\childdocforwardprefix|
and functionality expanded
\end{itemize}

%%%%%%%%%%%%%%%%%%%%%%%%%%%%%%%%%%%%%%%%
\paragraph{v1.0:} 2017/04/27

\begin{itemize}
\item
manual and install package
\item
first version published on CTAN
\end{itemize}

%%%%%%%%%%%%%%%%%%%%%%%%%%%%%%%%%%%%%%%%
\paragraph{v0.6:} 2017/04/26

\begin{itemize}
\item
redirection mechanism added
\end{itemize}

%%%%%%%%%%%%%%%%%%%%%%%%%%%%%%%%%%%%%%%%
\paragraph{v0.5:} 2017/04/26

\begin{itemize}
\item
functionality in definition file
\end{itemize}


%%%%%%%%%%%%%%%%%%%%%%%%%%%%%%%%%%%%%%%%%%%%%%%%%%%%%%%%%%%%%%%%%%%%%%%%%%%%%%%%
%%%%%%%%%%%%%%%%%%%%%%%%%%%%%%%%%%%%%%%%%%%%%%%%%%%%%%%%%%%%%%%%%%%%%%%%%%%%%%%%
%%%%%%%%%%%%%%%%%%%%%%%%%%%%%%%%%%%%%%%%%%%%%%%%%%%%%%%%%%%%%%%%%%%%%%%%%%%%%%%%
\appendix

\settowidth\MacroIndent{\rmfamily\scriptsize 000\ }

 \DocInput{childdoc.dtx}

\end{document}
%</driver>
% \fi
%
% %%%%%%%%%%%%%%%%%%%%%%%%%%%%%%%%%%%%%%%%%%%%%%%%%%%%%%%%%%%%%%%%%%%%%%%%%%%%%%
% %%%%%%%%%%%%%%%%%%%%%%%%%%%%%%%%%%%%%%%%%%%%%%%%%%%%%%%%%%%%%%%%%%%%%%%%%%%%%%
% \section{Sample}
%\iffalse
%<*samplemain>
%\fi
%
% The following presents a sample document
% with two chapters, two parts, a title page,
% a compile flag as well as three forwarding files to set the flag.
% It consists of eight |.tex| files:
% \begin{center}
% \begin{tabular}{ll}
% |cdocsamp.tex|&main file\\
% |cdocsch1.tex|&include file for chapter 1\\
% |cdocsch2.tex|&include file for chapter 2\\
% |cdocspt3.tex|&include file for part 3\\
% |cdocspt4.tex|&include file for part 4\\
% |cdocsdrf.tex|&forwarding file for main file in draft mode\\
% |cdocsfi1.tex|&forwarding file for final version of chapter 1\\
% |cdocsfi2.tex|&forwarding file for final version of chapter 2\\
% \end{tabular}
% \end{center}
% Each of the eight files can be compiled directly by the \LaTeX{} compiler.
%
% %%%%%%%%%%%%%%%%%%%%%%%%%%%%%%%%%%%%%%
% \paragraph{Main File.}
%
% The main file is called |cdocsamp.tex|.
%
% Load the \textsf{childdoc} definitions and
% declare the filename for the main document:
%    \begin{macrocode}
\input{childdoc.def}
\childdocmain{}
%    \end{macrocode}

% Optional override for |\version| flag:
%    \begin{macrocode}
%%\ifchilddoc\else\providecommand{\version}{draft}\fi
%    \end{macrocode}

% Define the default values for the |\version| flag
% (|final| for the main file and |draft| for childs):
%    \begin{macrocode}
\ifchilddoc
\providecommand{\version}{draft}
\else
\providecommand{\version}{final}
\fi
%    \end{macrocode}

% Load the standard document class:
%    \begin{macrocode}
\documentclass[12pt]{article}
%    \end{macrocode}

% Start the document body:
%    \begin{macrocode}
\begin{document}
%    \end{macrocode}

% Declare a title page.
% Print title, part of document being processed and version flag:
%    \begin{macrocode}
\addtocounter{page}{-1}
\begin{center}
{\LARGE\bfseries{}childdoc example\par}
\vspace{1cm}
\ifchilddoc
\ifchilddocmanual part\else chapter\fi:
`\childdocname' of `\childdocjob'\par
\else
main document: `\childdocjob'\par
\fi
version: \version\par
\end{center}
\newpage
%    \end{macrocode}

% Manually include selected file,
% otherwise process as usual:
%    \begin{macrocode}
\ifchilddocmanual
\section*{part `\childdocname'}
\input{\childdocname}
\else
%    \end{macrocode}

% Include the two chapters:
%    \begin{macrocode}
\include{cdocsch1}
\include{cdocsch2}
%    \end{macrocode}

% Include the two parts unless only chapters should be displayed:
%    \begin{macrocode}
\ifchilddoc\else
\section{part three}
\input{cdocspt3}
\section{part four}
\input{cdocspt4}
\fi
%    \end{macrocode}

% Process as usual until here:
%    \begin{macrocode}
\fi
%    \end{macrocode}

% End of document body:
%    \begin{macrocode}
\end{document}
%    \end{macrocode}
%\iffalse
%</samplemain>
%\fi
%
% %%%%%%%%%%%%%%%%%%%%%%%%%%%%%%%%%%%%%%
% \paragraph{Chapter Include Files.}
%
% The include files are called |cdocsch1.tex| and |cdocsch2.tex|.
%
%\iffalse
%<*samplechap1|samplechap2>
%\fi

% Optional override for |\version| flag:
%    \begin{macrocode}
%%\providecommand{\version}{final}
%    \end{macrocode}

% Include the main document:
%    \begin{macrocode}
\input{childdoc.def}
\childdocof{cdocsamp}
%    \end{macrocode}

%\iffalse
%</samplechap1|samplechap2>
%\fi
%
%\iffalse
%<*samplechap1>
%\fi
% Some text for chapter 1:
%    \begin{macrocode}
\section{one}
some text in chapter one
%    \end{macrocode}

%\iffalse
%</samplechap1>
%\fi
% Some text for chapter 2:
%\iffalse
%<*samplechap2>
%\fi
%    \begin{macrocode}
\section{two}
more text in chapter two
%    \end{macrocode}

%\iffalse
%</samplechap2>
%\fi
%
% %%%%%%%%%%%%%%%%%%%%%%%%%%%%%%%%%%%%%%
% \paragraph{Part Include Files.}
%
% The include files are called |cdocspt3.tex| and |cdocspt4.tex|.
%
%\iffalse
%<*samplepart3|samplepart4>
%\fi

% Optional override for |\version| flag:
%    \begin{macrocode}
%%\providecommand{\version}{final}
%    \end{macrocode}

% Include the main document:
%    \begin{macrocode}
\input{childdoc.def}
\childdocby{cdocsamp}
%    \end{macrocode}

%\iffalse
%</samplepart3|samplepart4>
%\fi
%
%\iffalse
%<*samplepart3>
%\fi
% Some text for part 3:
%    \begin{macrocode}
some text in part three
%    \end{macrocode}

%\iffalse
%</samplepart3>
%\fi
% Some text for part 4:
%\iffalse
%<*samplepart4>
%\fi
%    \begin{macrocode}
more text in part four
%    \end{macrocode}

%\iffalse
%</samplepart4>
%\fi
%
% %%%%%%%%%%%%%%%%%%%%%%%%%%%%%%%%%%%%%%
% \paragraph{Forwarding for a Complete Draft.}
%
% The following forwarding file |cdocsdrf.tex|
% compiles the main document in draft mode:
%\iffalse
%<*sampledraft>
%\fi
%    \begin{macrocode}
\def\version{draft}
\input{childdoc.def}
\childdocforward{cdocsamp}
%    \end{macrocode}

%\iffalse
%</sampledraft>
%\fi
%
% %%%%%%%%%%%%%%%%%%%%%%%%%%%%%%%%%%%%%%
% \paragraph{Forwarding for Final Version of the Chapters.}
%
% The following forwarding files |cdocsfn1.tex| and |cdocsfn2.tex|
% (with identical content)
% compile the final versions of the child documents
% |cdocsch1.tex| and |cdocsch2.tex|, respectively:
%\iffalse
%<*samplefinal>
%\fi
%    \begin{macrocode}
\def\version{final}
\input{childdoc.def}
\childdocforwardprefix[cdocsamp]{cdocsfn}{cdocsch}
%    \end{macrocode}

%\iffalse
%</samplefinal>
%\fi
%
% %%%%%%%%%%%%%%%%%%%%%%%%%%%%%%%%%%%%%%
% \paragraph{Command Line Processing.}
%
% The following three command lines generate the output files
% |cdocscld|, |cdocscl1| and |cdocscl2|
% which should be identical to
% |cdocsdrf|, |cdocsch1| and |cdocsfn2|, respectively:
% \begin{center}
% \begin{tabular}{l}
% |latex -jobname cdocscld \|\\
% |  "\def\version{draft}\input{childdoc.def}\childdocforward{cdocsamp}"|\\
% |latex -jobname cdocscl1 \|\\
% |  "\input{childdoc.def}\childdocforward[cdocsamp]{cdocsch1}"|\\
% |latex -jobname cdocscl2 \|\\
% |  "\def\version{final}\input{childdoc.def}\childdocforward{cdocsch2}"|
% \end{tabular}
% \end{center}
% Note that the trailing backslash on each first line
% merely continues the input to the second line
% (for convenient cut ant paste).
% Furthermore, the command |latex| can be replaced by any
% of its alternative versions such as |pdflatex|.
%
% %%%%%%%%%%%%%%%%%%%%%%%%%%%%%%%%%%%%%%%%%%%%%%%%%%%%%%%%%%%%%%%%%%%%%%%%%%%%%%
% %%%%%%%%%%%%%%%%%%%%%%%%%%%%%%%%%%%%%%%%%%%%%%%%%%%%%%%%%%%%%%%%%%%%%%%%%%%%%%
% \section{Implementation}
%\iffalse
%<*package>
%\fi
%
% This section describes the definitions file |childdoc.def|.

% The definitions cannot be loaded using |\usepackage| or |\RequirePackage|
% which has a mechanism to prevent loading a style file more than once.
% When loading the definitions by means of |\input|
% multiple instances have to be prevented manually:
%\iffalse
%This code needs to be before the `\ProvidesFile' directive
%which is defined at the beginning of this file.
%Therefore it is also placed there and commented out here.
%</package>
%<*discard>
%\fi
%    \begin{macrocode}
\ifdefined\childdocmain\endinput\fi
%    \end{macrocode}
%\iffalse
%</discard>
%<*package>
%\fi
%
% \macro{\ifchilddoc}
% \macro{\ifchilddocmanual}
% The conditional |\ifchilddoc| tells whether a
% child (true) or main (false) document is being compiled.
% The conditional |\ifchilddocmanual| tells whether
% the |\includeonly| mechanism is used (false) or
% the selection of child files must be performed manually (true).
% The definitions initialise to false:
%    \begin{macrocode}
\newif\ifchilddoc
\newif\ifchilddocmanual
%    \end{macrocode}

% \macro{\childdocname}
% \macro{\childdocjob}
% The macro |\childdocname| stores the name of the main document
% to be compiled. The macro |\childdocjob| stores the name of
% the document on which the \LaTeX{} compiler was originally invoked.
% The content of |\jobname| cannot be compared
% to filenames specified in the source due to different catcodes.
% The following code rescans |\jobname|, stores the result
% in |\childdocname| and saves a copy in |\childdocjob|:
%    \begin{macrocode}
\edef\childdocname{\scantokens\expandafter{\jobname\noexpand}}
\let\childdocjob\childdocname
%    \end{macrocode}

% \macro{\childdocdisable}
% The macro |\childdocdisable| prevents the main file
% from being processed more than once.
% At this stage, the main document command |\childdocmain|
% is assumed to be called once again where it should do nothing.
% Any subsequent call to it should prevent
% a secondary processing of the main document
% It overwrites the forwarding commands
% |\childdocof| and |\childdocforward|
% with empty macros to prevent further inclusions of the main document:
%    \begin{macrocode}
\newcommand{\childdocdisable}
{
  \renewcommand{\childdocmain}[1]{\renewcommand{\childdocmain}[1]{\endinput}}
  \renewcommand{\childdocof}[1]{}
  \renewcommand{\childdocby}[2][]{}
  \renewcommand{\childdocforward}[2][]{}
  \renewcommand{\childdocdisable}{}
}
%    \end{macrocode}

% \macro{\childdocmain}
% The macro |\childdocmain| is to be called at the top of the main file
% with nothing or the main filename (without extension) as argument.
% First, it breaks loops.
% If the argument is not empty and does not match |\childdocname|
% (which is set by the first inclusion of |childdoc.def|),
% |\ifchilddoc| is set to true, |\includeonly| is applied to the child file
% and |\jobname| is set to the main file
% (for proper handling of |.aux| files):
%    \begin{macrocode}
\newcommand{\childdocmain}[1]
{
  \childdocdisable\childdocmain{}
  \if?#1?\else
    \begingroup
      \def\childdoctmp{#1}
      \ifx\childdoctmp\childdocname
        \def\childdoctmp{}
      \else
        \def\childdoctmp
        {
          \childdoctrue
          \includeonly{\childdocname}
          \def\childdocjob{#1}
          \def\jobname{#1}
        }
      \fi
      \expandafter
    \endgroup
    \childdoctmp
  \fi
}
%    \end{macrocode}

% \macro{\childdocof}
% The command |\childdocof| redirects
% compilation to the main file |#1|.
%    \begin{macrocode}
\newcommand{\childdocof}[1]
{
  \childdocdisable
  \childdoctrue
  \includeonly{\childdocname}
  \def\jobname{#1}
  \def\childdocjob{#1}
  \input{#1}
}
%    \end{macrocode}

% \macro{\childdocby}
% The command |\childdocby| ....
%    \begin{macrocode}
\newcommand{\childdocby}[2][]
{
  \childdocdisable
  \childdoctrue
  \childdocmanualtrue
  \if?#1?\else
    \def\jobname{#2}
  \fi
  \def\childdocjob{#2}
  \input{#2}
  \endinput
}
%    \end{macrocode}

% \macro{\childdocforward}
% The command |\childdocforward| redirects
% compilation to the main file or
% (if the optional argument is given) a child file.
% Parameters are set as if the main file
% or a child file starting with |\childdocof| was compiled.
% Then compilation is handed over to the main file:
%    \begin{macrocode}
\newcommand{\childdocforward}[2][]
{
  \begingroup
    \if?#1?
      \def\childdoctmp
      {
        \def\childdocname{#2}
        \def\childdocjob{#2}
        \def\jobname{#2}
        \input{#2}
        \endinput
      }
    \else
      \def\childdoctmp
      {
        \childdocdisable
        \def\childdocname{#2}
        \childdoctrue
        \includeonly{#2}
        \def\childdocjob{#1}
        \def\jobname{#1}
        \input{#1}
        \endinput
      }
    \fi
    \expandafter
  \endgroup
  \childdoctmp
}
%    \end{macrocode}

% \macro{\childdocforwardprefix}
% The command |\childdocforwardprefix| redirects
% compilation to the main or a child file by means of a pattern.
% The prefix |#1| in the current filename is replaced by |#2|
% and the suffix of the current filename is kept
% (it is assumed that the filename does not contain the substring `|~~~|'
% which is used as a delimiter).
% Compilation is handed over to the new file by |\childdocforward|:
%    \begin{macrocode}
\newcommand{\childdocforwardprefix}[3][]
{
  \begingroup
    \def\childdocextract #2##1~~~{\def\childdoctmp{\childdocforward[#1]{#3##1}}}
    \expandafter\childdocextract\childdocname~~~
    \expandafter
  \endgroup
  \childdoctmp
}
%    \end{macrocode}

% \macro{\childdoc}
% The deprecated macro |\childdoc| is a legacy version of |\childdocmain|:
%    \begin{macrocode}
\newcommand{\childdoc}{\childdocmain}
%    \end{macrocode}

% \macro{\childdocredirect}
% The deprecated macro |\childdocredirect| is a legacy version
% of |\childdocforward| and |\childdocforwardprefix|:
%    \begin{macrocode}
\newcommand{\childdocredirect}[2][]
{
  \begingroup
    \if?#1?
      \def\childdoctmp{\childdocforward{#2}}
    \else
      \def\childdoctmp{\childdocforwardprefix{#1}{#2}}
    \fi
    \expandafter
  \endgroup
  \childdoctmp
}
%    \end{macrocode}

%\iffalse
%</package>
%\fi
%
\endinput
\childdocforward[|\textit{main}|]{|\textit{dest}|}"|
\end{center}
%
Here \textit{target} is the name of the output file,
\textit{main} is the name of the main file
and \textit{dest} is the name of the main or child file to be processed
(all filenames without extensions).
The optional argument \textit{main} can be omitted
if \textit{main} matches \textit{dest}.
Optionally, compilation \textit{flags} can be defined via |\def| commands.
This command line makes the \TeX{} engine believe
it is compiling the file \textit{target}
whose content is specified as the latter parameter.
The provided code then forwards the processing to
\textit{main} or \textit{dest} as described in \secref{sec:forward}.

%%%%%%%%%%%%%%%%%%%%%%%%%%%%%%%%%%%%%%%%%%%%%%%%%%%%%%%%%%%%%%%%%%%%%%%%%%%%%%%%
\subsection{Include by Input}
\label{sec:input}

Including child documents by |\include| has some restrictions by design.
Most notably, the content of a child document always occupies
its own set of pages; pages cannot be shared between child documents.
Usually, this behaviour makes perfect sense
because each child document contain an essential part of the document.
However, in some situations it may be desirable to compose
a document from a collection of parts
without having mandatory page breaks between then.
For this case, the package
provides a mechanism to include parts
by |\input| which can also be processed individually.
However, by construction this mechanism
requires manual handling of the content to be output.

%%%%%%%%%%%%%%%%%%%%%%%%%%%%%%%%%%%%%%%%
\DescribeMacro{\ifchilddocmanual}
The main file should be prepared as usual, see \secref{sec:include}.
However, the document body must make a distinction
between processing of an individual part and of the main document, e.g.:
%
\begin{center}
\begin{tabular}{l}
|\ifchilddocmanual|\\
|\input{\childdocname}|\\
|\||else|\\
\textit{document body with }|\input{|\textit{part}|}|\\
|\||fi|
\end{tabular}
\end{center}
%
The conditional |\ifchilddocmanual| is true whenever
a part to be included by |\input| is being compiled,
and the name of the part is stored in |\childdocname|.

%%%%%%%%%%%%%%%%%%%%%%%%%%%%%%%%%%%%%%%%
\DescribeMacro{\childdocby}
Each part to be included by |\input| should start with:
%
\begin{center}
\begin{tabular}{l}
|% \iffalse
%
% childdoc.dtx Copyright (C) 2017-2018 Niklas Beisert
%
% This work may be distributed and/or modified under the
% conditions of the LaTeX Project Public License, either version 1.3
% of this license or (at your option) any later version.
% The latest version of this license is in
%   http://www.latex-project.org/lppl.txt
% and version 1.3 or later is part of all distributions of LaTeX
% version 2005/12/01 or later.
%
% This work has the LPPL maintenance status `maintained'.
%
% The Current Maintainer of this work is Niklas Beisert.
%
% This work consists of the files childdoc.dtx and childdoc.ins
% and the derived files childdoc.def and cdocsamp.tex with
% cdocsch1.tex, cdocsch2.tex, cdocsdrf.tex, cdocsfn1.tex, cdocsfn2.tex.
%
%<package>\ifdefined\childdocmain\endinput\fi
%<package>\ProvidesFile{childdoc.def}[2018/12/30 v2.0 child document driver]
%<samplemain>\ProvidesFile{cdocsamp.tex}[2018/12/30 v2.0 sample for childdoc]
%<*driver>
%\ProvidesFile{childdoc.drv}[2018/12/30 v2.0 childdoc reference manual file]
\PassOptionsToClass{10pt,a4paper}{article}
\documentclass{ltxdoc}

\usepackage[margin=35mm]{geometry}
\usepackage{hyperref}
\usepackage{hyperxmp}
\usepackage[usenames]{color}

\hypersetup{colorlinks=true}
\hypersetup{pdfstartview=FitH}
\hypersetup{pdfpagemode=UseNone}
\hypersetup{pdfsource={}}
\hypersetup{pdflang={en-UK}}
\hypersetup{pdfcopyright={Copyright 2017-2018 Niklas Beisert.
  This work may be distributed and/or modified under the
  conditions of the LaTeX Project Public License, either version 1.3
  of this license or (at your option) any later version.}}
\hypersetup{pdflicenseurl={http://www.latex-project.org/lppl.txt}}
\hypersetup{pdfcontactaddress={ETH Zurich, ITP, HIT K,
  Wolfgang-Pauli-Strasse 27}}
\hypersetup{pdfcontactpostcode={8093}}
\hypersetup{pdfcontactcity={Zurich}}
\hypersetup{pdfcontactcountry={Switzerland}}
\hypersetup{pdfcontactemail={nbeisert@itp.phys.ethz.ch}}
\hypersetup{pdfcontacturl={http://people.phys.ethz.ch/\xmptilde nbeisert/}}

\newcommand{\secref}[1]{\hyperref[#1]{section \ref*{#1}}}

\parskip1ex
\parindent0pt
\let\olditemize\itemize
\def\itemize{\olditemize\parskip0pt}

\begin{document}

\title{The \textsf{childdoc} Package}
\hypersetup{pdftitle={The childdoc Package}}
\author{Niklas Beisert\\[2ex]
  Institut f\"ur Theoretische Physik\\
  Eidgen\"ossische Technische Hochschule Z\"urich\\
  Wolfgang-Pauli-Strasse 27, 8093 Z\"urich, Switzerland\\[1ex]
  \href{mailto:nbeisert@itp.phys.ethz.ch}
  {\texttt{nbeisert@itp.phys.ethz.ch}}}
\hypersetup{pdfauthor={Niklas Beisert}}
\hypersetup{pdfsubject={Manual for the LaTeX2e Package childdoc}}
\date{30 December 2018, \textsf{v2.0}}
\maketitle

\begin{abstract}\noindent
\textsf{childdoc} is a \LaTeXe{} package
that enables the direct compilation
of document sections included by |\include|
to individual files.
\end{abstract}

\begingroup
\parskip0ex
\tableofcontents
\endgroup

%%%%%%%%%%%%%%%%%%%%%%%%%%%%%%%%%%%%%%%%%%%%%%%%%%%%%%%%%%%%%%%%%%%%%%%%%%%%%%%%
%%%%%%%%%%%%%%%%%%%%%%%%%%%%%%%%%%%%%%%%%%%%%%%%%%%%%%%%%%%%%%%%%%%%%%%%%%%%%%%%
\section{Introduction}

\LaTeX{} provides a mechanism to structure a large document (such as a book)
into a main file and several child files (containing the chapters)
using the |\include| command.
This mechanism is beneficial for documents
which span hundreds of pages in order to
make the source file(s) more manageable.
Moreover, compilation can be restricted to
selected child files by means of the |\includeonly| command.
The latter feature can be used to reduce the compilation time while editing
(this was significantly more useful in the earlier days of \LaTeX{})
or to generate a smaller document which is easier to navigate.
Another application of |\includeonly| is to generate
documents consisting of selected parts of the complete document.

However, there are a few drawbacks of the plain |\include| mechanism:
\begin{itemize}
\item
The child files cannot be compiled on their own,
they can only be compiled via the main file.
A naive editing environment
(such as a text editor with an option
to have the current file processed by \LaTeX)
may require one to switch to the main file before compiling;
attempting to compile the child file produces errors.
\item
The main file must be modified (each time)
to adjust the |\includeonly| command
to the present needs. This easily leaves the main file in a messy state.
\item
The generated document will always carry the filename
of the main document. This is inconvenient if
several child files are to be compiled and
to be kept for distribution.
\end{itemize}

The present package provides a simple interface
to make child files individually compilable by \LaTeX{}.
Compiling a child file then has the same effect as compiling
the main file with an |\includeonly| command
to select the appropriate child.
Moreover the generated document will carry the name of the child
rather than the main file.
This resolves all three above issues.

This feature is meant to make the editing of books,
thesis documents and lecture notes somewhat more convenient.
However, the package can also be used efficiently for
composing a series of documents (such as exercise sheets)
which are typically distributed individually.
It then assists the author in generating the individual documents
(potentially in different versions)
as well as a document containing the collected series.
Another application is in developing style files
or other kinds of included material
where compilation of the style file could redirect
to a sample or test file.

%%%%%%%%%%%%%%%%%%%%%%%%%%%%%%%%%%%%%%%%%%%%%%%%%%%%%%%%%%%%%%%%%%%%%%%%%%%%%%%%
%%%%%%%%%%%%%%%%%%%%%%%%%%%%%%%%%%%%%%%%%%%%%%%%%%%%%%%%%%%%%%%%%%%%%%%%%%%%%%%%
\section{Usage}

First of all, the package \textsf{childdoc} is \emph{not} a standard
\LaTeXe{} |.sty| style file! Therefore it needs to be invoked in
a non-standard way.

%%%%%%%%%%%%%%%%%%%%%%%%%%%%%%%%%%%%%%%%%%%%%%%%%%%%%%%%%%%%%%%%%%%%%%%%%%%%%%%%
\subsection{Included Files}
\label{sec:include}

%%%%%%%%%%%%%%%%%%%%%%%%%%%%%%%%%%%%%%%%
\DescribeMacro{\childdocmain}
To use the package, add the commands
\begin{center}
\begin{tabular}{l}
|\input{childdoc.def}|\\
|\childdocmain{}|\\
\end{tabular}
\end{center}
at the very top of the main \LaTeX{} file,
in particular \emph{before} the |\documentclass| statement!
The argument of |\childdocmain| should be left empty
(but it must be present).

%%%%%%%%%%%%%%%%%%%%%%%%%%%%%%%%%%%%%%%%
\DescribeMacro{\childdocof}
Furthermore, add the commands
\begin{center}
\begin{tabular}{l}
|\input{childdoc.def}|\\
|\childdocof{|\textit{main}|}|\\
\end{tabular}
\end{center}
at the top of every child file \textit{child}
which is included by |\include{|\textit{child}|}|
from within the main file
(or at least for those files to be compiled individually).
The argument \textit{main} must be the filename of the main file.

There are a couple of
considerations in setting up the main and child documents:

%%%%%%%%%%%%%%%%%%%%%%%%%%%%%%%%%%%%%%%%
\paragraph{Restrictions.}

Please note the following restrictions:
\begin{itemize}
\item
|\childdocmain| must be called with one argument \textit{main}
to ensure compatibility with earlier version of the package.
It must either be empty (|\childdocmain{}|)
or precisely match the filename of the main file in which it is specified.
See \secref{sec:detection} for further information.
\item
The filename \textit{main} must be specified without the |.tex| extension.
\item
The filename \textit{main} is case sensitive
(even in case-insensitive file systems)
due to internal string comparison.
\item
The argument \textit{main} should be fully expanded, it cannot be a macro.
\item
Subdirectories and special characters should be avoided in filenames.
\item
The command |\childdocmain{|\textit{main}|}| must be followed by a whitespace.
It should not be followed immediately by another command
or by a comment mark `|%|'.
This is because the \TeX{} parser reads the token immediately following
the argument of |\childdocmain| and puts it
at the beginning of every child section;
however, a white\-space is ignored.
\end{itemize}

%%%%%%%%%%%%%%%%%%%%%%%%%%%%%%%%%%%%%%%%
\paragraph{Content of Main File.}

It is advisable to place all content in the child files included by |\include|.
Any output contained in the main file will appear in all child documents
unless suppressed manually;
it cannot be suppressed automatically by the |\includeonly| directive
and thus should normally be avoided.
A method to include some content in the main file
by means of conditional processing is described in \secref{sec:conditional}.

%%%%%%%%%%%%%%%%%%%%%%%%%%%%%%%%%%%%%%%%
\paragraph{Page Numbering.}

When only a part of the document is compiled,
the appropriate numbering of pages
(as well as other status parameters)
is determined from the |.aux| files.
The latter contain information from previous passes.
However this information needs to propagate through
all intermediate child documents.
Therefore the page numbering in child documents may well
be inconsistent until the complete document is compiled at least once.

A useful (if unconventional) way to always ensure a consistent
page numbering is to restart the numbering in each child document
and denote the pages by `\textit{child}|.|\textit{page}'
where \textit{child} represents the chapter/section number of the child file.
This can be achieved by the command
|\numberwithin{page}{|\textit{child}|}|
of the \textsf{amsmath} package
where \textit{child} can be |chapter| or |section|
depending on the chosen structuring.
Alternatively, one can modify the macro |\thepage| appropriately
and reset the counter |page| at the start of each child file.

%%%%%%%%%%%%%%%%%%%%%%%%%%%%%%%%%%%%%%%%%%%%%%%%%%%%%%%%%%%%%%%%%%%%%%%%%%%%%%%%
\subsection{Conditional Processing}
\label{sec:conditional}

The package provides a mechanism to compile different versions
of a document. To customise the versions further some conditional processing
can come in handy to distinguish which version is being compiled.
The package provides two macros to describe the compilation context:

%%%%%%%%%%%%%%%%%%%%%%%%%%%%%%%%%%%%%%%%
\DescribeMacro{\ifchilddoc}
The conditional |\ifchilddoc| distinguishes between the compilation of
child documents and the main document:
%
\begin{center}
|\ifchilddoc |\textit{child-code}| |[|\||else |\textit{main-code}]| \||fi|
\end{center}

%%%%%%%%%%%%%%%%%%%%%%%%%%%%%%%%%%%%%%%%
\DescribeMacro{\childdocname}
\DescribeMacro{\childdocjob}
The macro |\childdocname| contains the filename (without extension)
of the main or child file being processed.
Note that |\childdocjob| will always contain the name of the main file.

%%%%%%%%%%%%%%%%%%%%%%%%%%%%%%%%%%%%%%%%
\paragraph{Title Page.}

Conditional processing can be used to include a title or banner page
in the main document when proper precautions are taken.
Importantly, the code in the main file should ensure that the page counter
(as well as other status parameters which are stored in the |.aux| files)
takes the same value after the conditional processing.
Otherwise the page numbers may take divergent values
depending on which part is compiled.

For example, a title page could be declared by:
%
\begin{center}
\begin{tabular}{l}
|\ifchilddoc\||else|\\
|\addtocounter{page}{-1}|\\
\textit{code for title page}\\
|\newpage|\\
|\||fi|
\end{tabular}
\end{center}
%
A banner page for the child documents can be generated by:
%
\begin{center}
\begin{tabular}{l}
|\ifchilddoc|\\
|\addtocounter{page}{-1}|\\
\textit{code for banner page}\\
|\newpage|\\
|\||fi|
\end{tabular}
\end{center}
%
Here one could write a message such as:
\begin{center}
|This is the part \childdocname{} of \childdocjob{}.|
\end{center}

%%%%%%%%%%%%%%%%%%%%%%%%%%%%%%%%%%%%%%%%%%%%%%%%%%%%%%%%%%%%%%%%%%%%%%%%%%%%%%%%
\subsection{Flags}
\label{sec:flags}

The package makes it easy to generate different versions
of the main or child documents.
To this end compilation flags can be defined
and assigned different default values.
They will be particularly useful in conjunction
with the forwarding mechanism described in \secref{sec:forward}.

For example, it may be useful to have a flag |\version|
which can be set to |draft| or |final|.
The document source will contain some conditional code
depending on the value of |\version|.
Suppose further, the flag should default to |final| for the main file
and to |draft| for child files
which is a natural assignment for editing the document.
This is achieved by placing the following code
in the preamble of the main document
(below the |\childdocmain| directive):
%
\begin{center}
\begin{tabular}{l}
|\ifchilddoc|\\
|\providecommand{\version}{draft}|\\
|\||else|\\
|\providecommand{\version}{final}|\\
|\||fi|
\end{tabular}
\end{center}
%
The definition by |\providecommand| makes sure
that previous definitions are not overwritten.
Further statements |\providecommand{\version}{...}|
can thus be added before the above code to override it.

For the main file, one might add a line
(between |\childdocmain| and the above block)
%
\begin{center}
|%\ifchilddoc\||else\providecommand{\version}{draft}\||fi|
\end{center}
%
which can be uncommented to produce a draft version.
Likewise one can add a line to the very top of a child file
(above the |\childdocof{|\textit{main}|}| directive)
%
\begin{center}
|%\providecommand{\version}{final}|
\end{center}
%
which can be uncommented to produce the final version of this child document.

%%%%%%%%%%%%%%%%%%%%%%%%%%%%%%%%%%%%%%%%%%%%%%%%%%%%%%%%%%%%%%%%%%%%%%%%%%%%%%%%
\subsection{Forwarding}
\label{sec:forward}

Different versions of the main or child documents
using compilation flags as described in \secref{sec:flags}
can be (permanently) stored in different files
for convenient compilation, viewing and distribution.
To this end, the package defines a command
to pass on compilation to a different file:

%%%%%%%%%%%%%%%%%%%%%%%%%%%%%%%%%%%%%%%%
\DescribeMacro{\childdocforward}
The command |\childdocforward| redirects processing to
another source file:
%
\begin{center}
\begin{tabular}{l}
|\input{childdoc.def}|\\
|\childdocforward[|\textit{main}|]{|\textit{dest}|}|\\
\end{tabular}
\end{center}
%
The argument \textit{dest} is the destination file
(without extension).
It should be the main file or one of the child files.
Note that further \textsf{childdoc} directives
such as |\childdocof| and |\childdocforward|
in the indicated file will be processed in this form.
The optional argument \textit{main}
passes on directly to the main file \textit{main}
while pretending to compile the child \textit{dest}.
This form behaves as if \textit{dest}
issues |\childdocof{|\textit{main}|}| right away,
and no further \textsf{childdoc} directives will be processed.

%%%%%%%%%%%%%%%%%%%%%%%%%%%%%%%%%%%%%%%%
\DescribeMacro{\...prefix}
In the alternative form |\childdocforwardprefix|,
%
\begin{center}
\begin{tabular}{l}
|\input{childdoc.def}|\\
|\childdocforwardprefix[|\textit{main}|]{|\textit{prefix}|}{|\textit{dest}|}|
\end{tabular}
\end{center}
%
the destination file is determined by a pattern
depending on the current file:
To make this work, the current file must be called
`{\textit{prefix}\hspace{0.2em}\textit{suffix}}'
with \textit{prefix} matching precisely the argument.
Processing is then passed on to the file
`{\textit{dest}\hspace{0.2em}\textit{suffix}}'.
Surely, the same effect is achieved by
directly specifying the
argument `{\textit{dest}\hspace{0.2em}\textit{suffix}}'
in the first form.
However, that requires to set up a different file
for each child. With the alternative form of the command
all these files can have exactly the same content
which simplifies setting them up and maintaining them.

For example, the following file |draft.tex|
with a compilation flag |\version| as described in \secref{sec:flags}
compiles the main document as a draft:
%
\begin{center}
\begin{tabular}{l}
|\def\version{draft}|\\
|\input{childdoc.def}|\\
|\childdocforward{|\textit{main}|}|
\end{tabular}
\end{center}
%
Likewise, the following files |final|\textit{nn}|.tex|
compile the final version of the child document
|child|\textit{nn}|.tex|:
%
\begin{center}
\begin{tabular}{l}
|\def\version{final}|\\
|\input{childdoc.def}|\\
|\childdocforwardprefix{final}{child}|
\end{tabular}
\end{center}
%

Note that when several versions of a main file and/or of each child file
are to be generated, it may be convenient to set up a |Makefile| or
shell script to automatise the process.

%%%%%%%%%%%%%%%%%%%%%%%%%%%%%%%%%%%%%%%%%%%%%%%%%%%%%%%%%%%%%%%%%%%%%%%%%%%%%%%%
\subsection{Command Line Processing}
\label{sec:commandline}

The effect of redirection files can also be achieved by invoking
the \LaTeX{} compiler with a more elaborate command line.
Most conveniently this should be done as part
of a shell script or a |Makefile|.

When using \textsf{childdoc} in the main file, the following
command lines effectively perform a redirection
(note that depending on the shell being used,
backslashes may have to be doubled: `|\|' $\to$ `|\\|'):
%
\begin{center}
|... -jobname "|\textit{target}|" |\\|"|[\textit{flags}]%
|\input{childdoc.def}\childdocforward[|\textit{main}|]{|\textit{dest}|}"|
\end{center}
%
Here \textit{target} is the name of the output file,
\textit{main} is the name of the main file
and \textit{dest} is the name of the main or child file to be processed
(all filenames without extensions).
The optional argument \textit{main} can be omitted
if \textit{main} matches \textit{dest}.
Optionally, compilation \textit{flags} can be defined via |\def| commands.
This command line makes the \TeX{} engine believe
it is compiling the file \textit{target}
whose content is specified as the latter parameter.
The provided code then forwards the processing to
\textit{main} or \textit{dest} as described in \secref{sec:forward}.

%%%%%%%%%%%%%%%%%%%%%%%%%%%%%%%%%%%%%%%%%%%%%%%%%%%%%%%%%%%%%%%%%%%%%%%%%%%%%%%%
\subsection{Include by Input}
\label{sec:input}

Including child documents by |\include| has some restrictions by design.
Most notably, the content of a child document always occupies
its own set of pages; pages cannot be shared between child documents.
Usually, this behaviour makes perfect sense
because each child document contain an essential part of the document.
However, in some situations it may be desirable to compose
a document from a collection of parts
without having mandatory page breaks between then.
For this case, the package
provides a mechanism to include parts
by |\input| which can also be processed individually.
However, by construction this mechanism
requires manual handling of the content to be output.

%%%%%%%%%%%%%%%%%%%%%%%%%%%%%%%%%%%%%%%%
\DescribeMacro{\ifchilddocmanual}
The main file should be prepared as usual, see \secref{sec:include}.
However, the document body must make a distinction
between processing of an individual part and of the main document, e.g.:
%
\begin{center}
\begin{tabular}{l}
|\ifchilddocmanual|\\
|\input{\childdocname}|\\
|\||else|\\
\textit{document body with }|\input{|\textit{part}|}|\\
|\||fi|
\end{tabular}
\end{center}
%
The conditional |\ifchilddocmanual| is true whenever
a part to be included by |\input| is being compiled,
and the name of the part is stored in |\childdocname|.

%%%%%%%%%%%%%%%%%%%%%%%%%%%%%%%%%%%%%%%%
\DescribeMacro{\childdocby}
Each part to be included by |\input| should start with:
%
\begin{center}
\begin{tabular}{l}
|\input{childdoc.def}|\\
|\childdocby{|\textit{main}|}|\\
\end{tabular}
\end{center}
%
The directive |\childdocby| is similar to |\childdocof|
described in \secref{sec:include},
but the subsequent selection of content must be done manually.
To that end, both |\ifchilddoc| and |\ifchilddocmanual|
will be true upon processing of a part,
and the name of the part is stored in |\childdocname|.
Note that |\jobname| will be set to the filename of the current part
so that each part receives an individual |.aux| file
that does not interfere with the |.aux| file(s) of the main document.
This behaviour can be altered by the alternative form
|\childdocby[*]{|\textit{main}|}| (with a non-empty optional argument)
which uses the |.aux| file of the main document
by setting |\jobname| to \textit{main}.

%%%%%%%%%%%%%%%%%%%%%%%%%%%%%%%%%%%%%%%%%%%%%%%%%%%%%%%%%%%%%%%%%%%%%%%%%%%%%%%%
\subsection{Driver Development}
\label{sec:driver}

The \textsf{childdoc} mechanism can also be use for the development
of definition files such as \LaTeX{} styles or classes.
This case differs from the above setup with multiple parts
included by |\include| in that no |\includeonly| should be invoked.
This can be achieved by starting the include file
(before |\ProvidesPackage|) with:
%
\begin{center}
\begin{tabular}{l}
|\input{childdoc.def}|\\
|\childdocforward{|\textit{main}|}|\\
\end{tabular}
\end{center}
%
or alternatively with:
%
\begin{center}
\begin{tabular}{l}
|\input{childdoc.def}|\\
|\childdocby{|\textit{main}|}|\\
\end{tabular}
\end{center}
%
Both forms have slightly different effects as described above.
The main file is prepared as usual, see \secref{sec:include}.

%%%%%%%%%%%%%%%%%%%%%%%%%%%%%%%%%%%%%%%%%%%%%%%%%%%%%%%%%%%%%%%%%%%%%%%%%%%%%%%%
\subsection{Legacy Detection}
\label{sec:detection}

The directive |\childdocmain| in the main file can detect
whether the complete document or merely a child is to be compiled
even without using the directive |\childdocof|.
This method is deprecated because it is less robust
and there is no compelling reason to use it;
it is merely provided for backward compatibility
and it may be removed in future versions.

If the detection mechanism is to be used,
it is mandatory to correctly specify
the filename of the main file as the argument of |\childdocmain|:
%
\begin{center}
\begin{tabular}{l}
|\input{childdoc.def}|\\
|\childdocmain{|\textit{main}|}|\\
\end{tabular}
\end{center}
%
If |\jobname| does not match the argument \textit{main} of |\childdocmain|,
it is assumed that |\jobname| points to the child file to be compiled.
When using |\childdocmain| with the main file specified as argument,
it suffices to start a child file
with just |\input{|\textit{main}|}|
without loading of the package and using |\childdocof|.
If instead all processing is done
with the appropriate \textsf{childdoc} directives,
the argument of \textit{main} of |\childdocmain| can be empty.

An alternative version of the command line processing described
in \secref{sec:commandline} using the detection mechanism reads:
%
\begin{center}
|... -jobname "|\textit{target}|" "|[\textit{flags}]%
[|\def\jobname{|\textit{dest}|}|]|\input{|\textit{main}|}"|
\end{center}

%%%%%%%%%%%%%%%%%%%%%%%%%%%%%%%%%%%%%%%%%%%%%%%%%%%%%%%%%%%%%%%%%%%%%%%%%%%%%%%%
\subsection{Manual Code}
\label{sec:manual}

In case one cannot be certain whether the definitions file |childdoc.def|
is installed on the target \TeX{} distribution
and one prefers not to ship it,
it is conceivable to paste a few relevant commands into the sources.

To that end, drop all statements |\input{childdoc.def}|
and perform the replacements as outlined below.
Instead of |\childdocmain{|\textit{main}|}| add the following code
to the top of the main file:
%
\begin{center}
\begin{tabular}{l}
|\||ifdefined\childdocname\endinput\||fi\newif\ifchilddoc|\\
|\edef\childdocname{\scantokens\expandafter{\jobname\noexpand}}|\\
|\def\childdocmain{|\textit{main}|}\||ifx\childdocmain\childdocname\||else|\\
|\childdoctrue\includeonly{\childdocname}\let\jobname\childdocmain\||fi|\\
\end{tabular}
\end{center}
%
Instead of |\childdocof{|\textit{main}|}| just include the main file
at the top of each child file:
%
\begin{center}
|\input{|\textit{main}|}|
\end{center}
%
A simple redirection |\childdocforward{|\textit{dest}|}| is achieved by:
%
\begin{center}
|\def\jobname{|\textit{dest}|}\input{\jobname}|
\end{center}
%
The redirection with prefix
|\childdocforwardprefix[|\textit{prefix}|]{|\textit{dest}|}|
is accomplished by:
%
\begin{center}
\begin{tabular}{l}
|{\edef\jobname{\scantokens\expandafter{\jobname\noexpand}}|\\
|\def\redirectjob |\textit{prefix}|#1~~~{\gdef\jobname{|\textit{dest}|#1}}|\\
|\expandafter\redirectjob\jobname~~~}\input{\jobname}|
\end{tabular}
\end{center}

In an alternative approach,
child documents can be compiled by a specific command line
without additional code or specific definitions:
%
\begin{center}
|... -jobname "|\textit{target}|" "|[\textit{flags}]%
|\includeonly{|\textit{dest}|}\input{|\textit{main}|}"|
\end{center}
%

%%%%%%%%%%%%%%%%%%%%%%%%%%%%%%%%%%%%%%%%%%%%%%%%%%%%%%%%%%%%%%%%%%%%%%%%%%%%%%%%
%%%%%%%%%%%%%%%%%%%%%%%%%%%%%%%%%%%%%%%%%%%%%%%%%%%%%%%%%%%%%%%%%%%%%%%%%%%%%%%%
\section{Information}

%%%%%%%%%%%%%%%%%%%%%%%%%%%%%%%%%%%%%%%%%%%%%%%%%%%%%%%%%%%%%%%%%%%%%%%%%%%%%%%%
\subsection{Copyright}

Copyright \copyright{} 2017--2018 Niklas Beisert

This work may be distributed and/or modified under the
conditions of the \LaTeX{} Project Public License, either version 1.3
of this license or (at your option) any later version.
The latest version of this license is in
  \url{http://www.latex-project.org/lppl.txt}
and version 1.3 or later is part of all distributions of \LaTeX{}
version 2005/12/01 or later.

This work has the LPPL maintenance status `maintained'.

The Current Maintainer of this work is Niklas Beisert.

This work consists of the files |README.txt|, |childdoc.ins| and |childdoc.dtx|
as well as the derived files |childdoc.def|, |cdocsamp.tex|
with |cdocsch1.tex|, |cdocsch2.tex|, |cdocspt3.tex|, |cdocspt4.tex|,
|cdocsdrf.tex|, |cdocsfn1.tex|, |cdocsfn2.tex|
as well as |childdoc.pdf|.

%%%%%%%%%%%%%%%%%%%%%%%%%%%%%%%%%%%%%%%%%%%%%%%%%%%%%%%%%%%%%%%%%%%%%%%%%%%%%%%%
\subsection{Files and Installation}

The package consists of the files:
%
\begin{center}
\begin{tabular}{ll}
    |README.txt|   & readme file \\
    |childdoc.ins| & installation file \\
    |childdoc.dtx| & source file \\
    |childdoc.def| & definition file \\
    |cdocsamp.tex| & sample main file \\
    |cdocsch1.tex| & sample include file \\
    |cdocsch2.tex| & sample include file \\
    |cdocspt3.tex| & sample part file \\
    |cdocspt4.tex| & sample part file \\
    |cdocsdrf.tex| & sample redirection file \\
    |cdocsfn1.tex| & sample redirection file \\
    |cdocsfn2.tex| & sample redirection file \\
    |childdoc.pdf| & manual
\end{tabular}
\end{center}
%
The distribution consists of the files
|README.txt|, |childdoc.ins| and |childdoc.dtx|.
%
\begin{itemize}
\item
Run (pdf)\LaTeX{} on |childdoc.dtx|
to compile the manual |childdoc.pdf| (this file).
\item
Run \LaTeX{} on |childdoc.ins| to create the definitions file |childdoc.def|
and the sample |cdocsamp.tex| with include files
|cdocsch1.tex|, |cdocsch2.tex|, |cdocspt3.tex|, |cdocspt4.tex|,
|cdocsdrf.tex|, |cdocsfn1.tex|, |cdocsfn2.tex|.
Then copy the file |childdoc.def| to an appropriate directory of your \LaTeX{}
distribution, e.g.\ \textit{texmf-root}|/tex/latex/childdoc|.
\end{itemize}

%%%%%%%%%%%%%%%%%%%%%%%%%%%%%%%%%%%%%%%%%%%%%%%%%%%%%%%%%%%%%%%%%%%%%%%%%%%%%%%%
\subsection{Related CTAN Packages}

There are several other packages which offer a similar functionality:
%
\begin{itemize}
\item
The packages
\href{http://ctan.org/pkg/docmute}{\textsf{docmute}},
\href{http://ctan.org/pkg/includex}{\textsf{includex}} and
\href{http://ctan.org/pkg/standalone}{\textsf{standalone}}
provide commands to include only the document body of
a child file thus allowing both files to be compiled individually.
\item
The packages \href{http://ctan.org/pkg/subdocs}{\textsf{subdocs}}
and \href{http://ctan.org/pkg/subfiles}{\textsf{subfiles}}
provide structures in which the main and child documents can be
encapsulated and allowing them to be compiled individually.
The inclusion mechanism is different from the conventional |\include|.
\item
The package \href{http://ctan.org/pkg/combine}{\textsf{combine}}
is an elaborate solution to combine several documents into one.
\end{itemize}
%
See also the CTAN topic \href{http://ctan.org/topic/subdocs}{\textsf{subdocs}}
for further related packages.
The present package differs from the above solutions in that
a document structure constructed with the conventional |\include| mechanism
just needs two extra commands at the top of every file
such that all constituent files can be compiled individually.

%%%%%%%%%%%%%%%%%%%%%%%%%%%%%%%%%%%%%%%%%%%%%%%%%%%%%%%%%%%%%%%%%%%%%%%%%%%%%%%%
%\subsection{Feature Suggestions}
%
%The following is a list of features which may be useful for future
%versions of this package:
%%
%\begin{itemize}
%\item
%\ldots
%\end{itemize}

%%%%%%%%%%%%%%%%%%%%%%%%%%%%%%%%%%%%%%%%%%%%%%%%%%%%%%%%%%%%%%%%%%%%%%%%%%%%%%%%
\subsection{Revision History}

%%%%%%%%%%%%%%%%%%%%%%%%%%%%%%%%%%%%%%%%
\paragraph{v2.0:} 2018/12/30

\begin{itemize}
\item
immediate forward processing
\item
added |\childdocby| mechanism
\item
manual restructured
\end{itemize}

%%%%%%%%%%%%%%%%%%%%%%%%%%%%%%%%%%%%%%%%
\paragraph{v1.6:} 2018/01/17

\begin{itemize}
\item
application for development of include files
\item
corrections to manual
\end{itemize}

%%%%%%%%%%%%%%%%%%%%%%%%%%%%%%%%%%%%%%%%
\paragraph{v1.5:} 2017/05/21

\begin{itemize}
\item
more complete structuring introduced
\item
|\childdocof| introduced
\item
|\childdoc| renamed to |\childdocmain|
\item
|\childredirect| renamed to |\childdocforward| and |\childdocforwardprefix|
and functionality expanded
\end{itemize}

%%%%%%%%%%%%%%%%%%%%%%%%%%%%%%%%%%%%%%%%
\paragraph{v1.0:} 2017/04/27

\begin{itemize}
\item
manual and install package
\item
first version published on CTAN
\end{itemize}

%%%%%%%%%%%%%%%%%%%%%%%%%%%%%%%%%%%%%%%%
\paragraph{v0.6:} 2017/04/26

\begin{itemize}
\item
redirection mechanism added
\end{itemize}

%%%%%%%%%%%%%%%%%%%%%%%%%%%%%%%%%%%%%%%%
\paragraph{v0.5:} 2017/04/26

\begin{itemize}
\item
functionality in definition file
\end{itemize}


%%%%%%%%%%%%%%%%%%%%%%%%%%%%%%%%%%%%%%%%%%%%%%%%%%%%%%%%%%%%%%%%%%%%%%%%%%%%%%%%
%%%%%%%%%%%%%%%%%%%%%%%%%%%%%%%%%%%%%%%%%%%%%%%%%%%%%%%%%%%%%%%%%%%%%%%%%%%%%%%%
%%%%%%%%%%%%%%%%%%%%%%%%%%%%%%%%%%%%%%%%%%%%%%%%%%%%%%%%%%%%%%%%%%%%%%%%%%%%%%%%
\appendix

\settowidth\MacroIndent{\rmfamily\scriptsize 000\ }

 \DocInput{childdoc.dtx}

\end{document}
%</driver>
% \fi
%
% %%%%%%%%%%%%%%%%%%%%%%%%%%%%%%%%%%%%%%%%%%%%%%%%%%%%%%%%%%%%%%%%%%%%%%%%%%%%%%
% %%%%%%%%%%%%%%%%%%%%%%%%%%%%%%%%%%%%%%%%%%%%%%%%%%%%%%%%%%%%%%%%%%%%%%%%%%%%%%
% \section{Sample}
%\iffalse
%<*samplemain>
%\fi
%
% The following presents a sample document
% with two chapters, two parts, a title page,
% a compile flag as well as three forwarding files to set the flag.
% It consists of eight |.tex| files:
% \begin{center}
% \begin{tabular}{ll}
% |cdocsamp.tex|&main file\\
% |cdocsch1.tex|&include file for chapter 1\\
% |cdocsch2.tex|&include file for chapter 2\\
% |cdocspt3.tex|&include file for part 3\\
% |cdocspt4.tex|&include file for part 4\\
% |cdocsdrf.tex|&forwarding file for main file in draft mode\\
% |cdocsfi1.tex|&forwarding file for final version of chapter 1\\
% |cdocsfi2.tex|&forwarding file for final version of chapter 2\\
% \end{tabular}
% \end{center}
% Each of the eight files can be compiled directly by the \LaTeX{} compiler.
%
% %%%%%%%%%%%%%%%%%%%%%%%%%%%%%%%%%%%%%%
% \paragraph{Main File.}
%
% The main file is called |cdocsamp.tex|.
%
% Load the \textsf{childdoc} definitions and
% declare the filename for the main document:
%    \begin{macrocode}
\input{childdoc.def}
\childdocmain{}
%    \end{macrocode}

% Optional override for |\version| flag:
%    \begin{macrocode}
%%\ifchilddoc\else\providecommand{\version}{draft}\fi
%    \end{macrocode}

% Define the default values for the |\version| flag
% (|final| for the main file and |draft| for childs):
%    \begin{macrocode}
\ifchilddoc
\providecommand{\version}{draft}
\else
\providecommand{\version}{final}
\fi
%    \end{macrocode}

% Load the standard document class:
%    \begin{macrocode}
\documentclass[12pt]{article}
%    \end{macrocode}

% Start the document body:
%    \begin{macrocode}
\begin{document}
%    \end{macrocode}

% Declare a title page.
% Print title, part of document being processed and version flag:
%    \begin{macrocode}
\addtocounter{page}{-1}
\begin{center}
{\LARGE\bfseries{}childdoc example\par}
\vspace{1cm}
\ifchilddoc
\ifchilddocmanual part\else chapter\fi:
`\childdocname' of `\childdocjob'\par
\else
main document: `\childdocjob'\par
\fi
version: \version\par
\end{center}
\newpage
%    \end{macrocode}

% Manually include selected file,
% otherwise process as usual:
%    \begin{macrocode}
\ifchilddocmanual
\section*{part `\childdocname'}
\input{\childdocname}
\else
%    \end{macrocode}

% Include the two chapters:
%    \begin{macrocode}
\include{cdocsch1}
\include{cdocsch2}
%    \end{macrocode}

% Include the two parts unless only chapters should be displayed:
%    \begin{macrocode}
\ifchilddoc\else
\section{part three}
\input{cdocspt3}
\section{part four}
\input{cdocspt4}
\fi
%    \end{macrocode}

% Process as usual until here:
%    \begin{macrocode}
\fi
%    \end{macrocode}

% End of document body:
%    \begin{macrocode}
\end{document}
%    \end{macrocode}
%\iffalse
%</samplemain>
%\fi
%
% %%%%%%%%%%%%%%%%%%%%%%%%%%%%%%%%%%%%%%
% \paragraph{Chapter Include Files.}
%
% The include files are called |cdocsch1.tex| and |cdocsch2.tex|.
%
%\iffalse
%<*samplechap1|samplechap2>
%\fi

% Optional override for |\version| flag:
%    \begin{macrocode}
%%\providecommand{\version}{final}
%    \end{macrocode}

% Include the main document:
%    \begin{macrocode}
\input{childdoc.def}
\childdocof{cdocsamp}
%    \end{macrocode}

%\iffalse
%</samplechap1|samplechap2>
%\fi
%
%\iffalse
%<*samplechap1>
%\fi
% Some text for chapter 1:
%    \begin{macrocode}
\section{one}
some text in chapter one
%    \end{macrocode}

%\iffalse
%</samplechap1>
%\fi
% Some text for chapter 2:
%\iffalse
%<*samplechap2>
%\fi
%    \begin{macrocode}
\section{two}
more text in chapter two
%    \end{macrocode}

%\iffalse
%</samplechap2>
%\fi
%
% %%%%%%%%%%%%%%%%%%%%%%%%%%%%%%%%%%%%%%
% \paragraph{Part Include Files.}
%
% The include files are called |cdocspt3.tex| and |cdocspt4.tex|.
%
%\iffalse
%<*samplepart3|samplepart4>
%\fi

% Optional override for |\version| flag:
%    \begin{macrocode}
%%\providecommand{\version}{final}
%    \end{macrocode}

% Include the main document:
%    \begin{macrocode}
\input{childdoc.def}
\childdocby{cdocsamp}
%    \end{macrocode}

%\iffalse
%</samplepart3|samplepart4>
%\fi
%
%\iffalse
%<*samplepart3>
%\fi
% Some text for part 3:
%    \begin{macrocode}
some text in part three
%    \end{macrocode}

%\iffalse
%</samplepart3>
%\fi
% Some text for part 4:
%\iffalse
%<*samplepart4>
%\fi
%    \begin{macrocode}
more text in part four
%    \end{macrocode}

%\iffalse
%</samplepart4>
%\fi
%
% %%%%%%%%%%%%%%%%%%%%%%%%%%%%%%%%%%%%%%
% \paragraph{Forwarding for a Complete Draft.}
%
% The following forwarding file |cdocsdrf.tex|
% compiles the main document in draft mode:
%\iffalse
%<*sampledraft>
%\fi
%    \begin{macrocode}
\def\version{draft}
\input{childdoc.def}
\childdocforward{cdocsamp}
%    \end{macrocode}

%\iffalse
%</sampledraft>
%\fi
%
% %%%%%%%%%%%%%%%%%%%%%%%%%%%%%%%%%%%%%%
% \paragraph{Forwarding for Final Version of the Chapters.}
%
% The following forwarding files |cdocsfn1.tex| and |cdocsfn2.tex|
% (with identical content)
% compile the final versions of the child documents
% |cdocsch1.tex| and |cdocsch2.tex|, respectively:
%\iffalse
%<*samplefinal>
%\fi
%    \begin{macrocode}
\def\version{final}
\input{childdoc.def}
\childdocforwardprefix[cdocsamp]{cdocsfn}{cdocsch}
%    \end{macrocode}

%\iffalse
%</samplefinal>
%\fi
%
% %%%%%%%%%%%%%%%%%%%%%%%%%%%%%%%%%%%%%%
% \paragraph{Command Line Processing.}
%
% The following three command lines generate the output files
% |cdocscld|, |cdocscl1| and |cdocscl2|
% which should be identical to
% |cdocsdrf|, |cdocsch1| and |cdocsfn2|, respectively:
% \begin{center}
% \begin{tabular}{l}
% |latex -jobname cdocscld \|\\
% |  "\def\version{draft}\input{childdoc.def}\childdocforward{cdocsamp}"|\\
% |latex -jobname cdocscl1 \|\\
% |  "\input{childdoc.def}\childdocforward[cdocsamp]{cdocsch1}"|\\
% |latex -jobname cdocscl2 \|\\
% |  "\def\version{final}\input{childdoc.def}\childdocforward{cdocsch2}"|
% \end{tabular}
% \end{center}
% Note that the trailing backslash on each first line
% merely continues the input to the second line
% (for convenient cut ant paste).
% Furthermore, the command |latex| can be replaced by any
% of its alternative versions such as |pdflatex|.
%
% %%%%%%%%%%%%%%%%%%%%%%%%%%%%%%%%%%%%%%%%%%%%%%%%%%%%%%%%%%%%%%%%%%%%%%%%%%%%%%
% %%%%%%%%%%%%%%%%%%%%%%%%%%%%%%%%%%%%%%%%%%%%%%%%%%%%%%%%%%%%%%%%%%%%%%%%%%%%%%
% \section{Implementation}
%\iffalse
%<*package>
%\fi
%
% This section describes the definitions file |childdoc.def|.

% The definitions cannot be loaded using |\usepackage| or |\RequirePackage|
% which has a mechanism to prevent loading a style file more than once.
% When loading the definitions by means of |\input|
% multiple instances have to be prevented manually:
%\iffalse
%This code needs to be before the `\ProvidesFile' directive
%which is defined at the beginning of this file.
%Therefore it is also placed there and commented out here.
%</package>
%<*discard>
%\fi
%    \begin{macrocode}
\ifdefined\childdocmain\endinput\fi
%    \end{macrocode}
%\iffalse
%</discard>
%<*package>
%\fi
%
% \macro{\ifchilddoc}
% \macro{\ifchilddocmanual}
% The conditional |\ifchilddoc| tells whether a
% child (true) or main (false) document is being compiled.
% The conditional |\ifchilddocmanual| tells whether
% the |\includeonly| mechanism is used (false) or
% the selection of child files must be performed manually (true).
% The definitions initialise to false:
%    \begin{macrocode}
\newif\ifchilddoc
\newif\ifchilddocmanual
%    \end{macrocode}

% \macro{\childdocname}
% \macro{\childdocjob}
% The macro |\childdocname| stores the name of the main document
% to be compiled. The macro |\childdocjob| stores the name of
% the document on which the \LaTeX{} compiler was originally invoked.
% The content of |\jobname| cannot be compared
% to filenames specified in the source due to different catcodes.
% The following code rescans |\jobname|, stores the result
% in |\childdocname| and saves a copy in |\childdocjob|:
%    \begin{macrocode}
\edef\childdocname{\scantokens\expandafter{\jobname\noexpand}}
\let\childdocjob\childdocname
%    \end{macrocode}

% \macro{\childdocdisable}
% The macro |\childdocdisable| prevents the main file
% from being processed more than once.
% At this stage, the main document command |\childdocmain|
% is assumed to be called once again where it should do nothing.
% Any subsequent call to it should prevent
% a secondary processing of the main document
% It overwrites the forwarding commands
% |\childdocof| and |\childdocforward|
% with empty macros to prevent further inclusions of the main document:
%    \begin{macrocode}
\newcommand{\childdocdisable}
{
  \renewcommand{\childdocmain}[1]{\renewcommand{\childdocmain}[1]{\endinput}}
  \renewcommand{\childdocof}[1]{}
  \renewcommand{\childdocby}[2][]{}
  \renewcommand{\childdocforward}[2][]{}
  \renewcommand{\childdocdisable}{}
}
%    \end{macrocode}

% \macro{\childdocmain}
% The macro |\childdocmain| is to be called at the top of the main file
% with nothing or the main filename (without extension) as argument.
% First, it breaks loops.
% If the argument is not empty and does not match |\childdocname|
% (which is set by the first inclusion of |childdoc.def|),
% |\ifchilddoc| is set to true, |\includeonly| is applied to the child file
% and |\jobname| is set to the main file
% (for proper handling of |.aux| files):
%    \begin{macrocode}
\newcommand{\childdocmain}[1]
{
  \childdocdisable\childdocmain{}
  \if?#1?\else
    \begingroup
      \def\childdoctmp{#1}
      \ifx\childdoctmp\childdocname
        \def\childdoctmp{}
      \else
        \def\childdoctmp
        {
          \childdoctrue
          \includeonly{\childdocname}
          \def\childdocjob{#1}
          \def\jobname{#1}
        }
      \fi
      \expandafter
    \endgroup
    \childdoctmp
  \fi
}
%    \end{macrocode}

% \macro{\childdocof}
% The command |\childdocof| redirects
% compilation to the main file |#1|.
%    \begin{macrocode}
\newcommand{\childdocof}[1]
{
  \childdocdisable
  \childdoctrue
  \includeonly{\childdocname}
  \def\jobname{#1}
  \def\childdocjob{#1}
  \input{#1}
}
%    \end{macrocode}

% \macro{\childdocby}
% The command |\childdocby| ....
%    \begin{macrocode}
\newcommand{\childdocby}[2][]
{
  \childdocdisable
  \childdoctrue
  \childdocmanualtrue
  \if?#1?\else
    \def\jobname{#2}
  \fi
  \def\childdocjob{#2}
  \input{#2}
  \endinput
}
%    \end{macrocode}

% \macro{\childdocforward}
% The command |\childdocforward| redirects
% compilation to the main file or
% (if the optional argument is given) a child file.
% Parameters are set as if the main file
% or a child file starting with |\childdocof| was compiled.
% Then compilation is handed over to the main file:
%    \begin{macrocode}
\newcommand{\childdocforward}[2][]
{
  \begingroup
    \if?#1?
      \def\childdoctmp
      {
        \def\childdocname{#2}
        \def\childdocjob{#2}
        \def\jobname{#2}
        \input{#2}
        \endinput
      }
    \else
      \def\childdoctmp
      {
        \childdocdisable
        \def\childdocname{#2}
        \childdoctrue
        \includeonly{#2}
        \def\childdocjob{#1}
        \def\jobname{#1}
        \input{#1}
        \endinput
      }
    \fi
    \expandafter
  \endgroup
  \childdoctmp
}
%    \end{macrocode}

% \macro{\childdocforwardprefix}
% The command |\childdocforwardprefix| redirects
% compilation to the main or a child file by means of a pattern.
% The prefix |#1| in the current filename is replaced by |#2|
% and the suffix of the current filename is kept
% (it is assumed that the filename does not contain the substring `|~~~|'
% which is used as a delimiter).
% Compilation is handed over to the new file by |\childdocforward|:
%    \begin{macrocode}
\newcommand{\childdocforwardprefix}[3][]
{
  \begingroup
    \def\childdocextract #2##1~~~{\def\childdoctmp{\childdocforward[#1]{#3##1}}}
    \expandafter\childdocextract\childdocname~~~
    \expandafter
  \endgroup
  \childdoctmp
}
%    \end{macrocode}

% \macro{\childdoc}
% The deprecated macro |\childdoc| is a legacy version of |\childdocmain|:
%    \begin{macrocode}
\newcommand{\childdoc}{\childdocmain}
%    \end{macrocode}

% \macro{\childdocredirect}
% The deprecated macro |\childdocredirect| is a legacy version
% of |\childdocforward| and |\childdocforwardprefix|:
%    \begin{macrocode}
\newcommand{\childdocredirect}[2][]
{
  \begingroup
    \if?#1?
      \def\childdoctmp{\childdocforward{#2}}
    \else
      \def\childdoctmp{\childdocforwardprefix{#1}{#2}}
    \fi
    \expandafter
  \endgroup
  \childdoctmp
}
%    \end{macrocode}

%\iffalse
%</package>
%\fi
%
\endinput
|\\
|\childdocby{|\textit{main}|}|\\
\end{tabular}
\end{center}
%
The directive |\childdocby| is similar to |\childdocof|
described in \secref{sec:include},
but the subsequent selection of content must be done manually.
To that end, both |\ifchilddoc| and |\ifchilddocmanual|
will be true upon processing of a part,
and the name of the part is stored in |\childdocname|.
Note that |\jobname| will be set to the filename of the current part
so that each part receives an individual |.aux| file
that does not interfere with the |.aux| file(s) of the main document.
This behaviour can be altered by the alternative form
|\childdocby[*]{|\textit{main}|}| (with a non-empty optional argument)
which uses the |.aux| file of the main document
by setting |\jobname| to \textit{main}.

%%%%%%%%%%%%%%%%%%%%%%%%%%%%%%%%%%%%%%%%%%%%%%%%%%%%%%%%%%%%%%%%%%%%%%%%%%%%%%%%
\subsection{Driver Development}
\label{sec:driver}

The \textsf{childdoc} mechanism can also be use for the development
of definition files such as \LaTeX{} styles or classes.
This case differs from the above setup with multiple parts
included by |\include| in that no |\includeonly| should be invoked.
This can be achieved by starting the include file
(before |\ProvidesPackage|) with:
%
\begin{center}
\begin{tabular}{l}
|% \iffalse
%
% childdoc.dtx Copyright (C) 2017-2018 Niklas Beisert
%
% This work may be distributed and/or modified under the
% conditions of the LaTeX Project Public License, either version 1.3
% of this license or (at your option) any later version.
% The latest version of this license is in
%   http://www.latex-project.org/lppl.txt
% and version 1.3 or later is part of all distributions of LaTeX
% version 2005/12/01 or later.
%
% This work has the LPPL maintenance status `maintained'.
%
% The Current Maintainer of this work is Niklas Beisert.
%
% This work consists of the files childdoc.dtx and childdoc.ins
% and the derived files childdoc.def and cdocsamp.tex with
% cdocsch1.tex, cdocsch2.tex, cdocsdrf.tex, cdocsfn1.tex, cdocsfn2.tex.
%
%<package>\ifdefined\childdocmain\endinput\fi
%<package>\ProvidesFile{childdoc.def}[2018/12/30 v2.0 child document driver]
%<samplemain>\ProvidesFile{cdocsamp.tex}[2018/12/30 v2.0 sample for childdoc]
%<*driver>
%\ProvidesFile{childdoc.drv}[2018/12/30 v2.0 childdoc reference manual file]
\PassOptionsToClass{10pt,a4paper}{article}
\documentclass{ltxdoc}

\usepackage[margin=35mm]{geometry}
\usepackage{hyperref}
\usepackage{hyperxmp}
\usepackage[usenames]{color}

\hypersetup{colorlinks=true}
\hypersetup{pdfstartview=FitH}
\hypersetup{pdfpagemode=UseNone}
\hypersetup{pdfsource={}}
\hypersetup{pdflang={en-UK}}
\hypersetup{pdfcopyright={Copyright 2017-2018 Niklas Beisert.
  This work may be distributed and/or modified under the
  conditions of the LaTeX Project Public License, either version 1.3
  of this license or (at your option) any later version.}}
\hypersetup{pdflicenseurl={http://www.latex-project.org/lppl.txt}}
\hypersetup{pdfcontactaddress={ETH Zurich, ITP, HIT K,
  Wolfgang-Pauli-Strasse 27}}
\hypersetup{pdfcontactpostcode={8093}}
\hypersetup{pdfcontactcity={Zurich}}
\hypersetup{pdfcontactcountry={Switzerland}}
\hypersetup{pdfcontactemail={nbeisert@itp.phys.ethz.ch}}
\hypersetup{pdfcontacturl={http://people.phys.ethz.ch/\xmptilde nbeisert/}}

\newcommand{\secref}[1]{\hyperref[#1]{section \ref*{#1}}}

\parskip1ex
\parindent0pt
\let\olditemize\itemize
\def\itemize{\olditemize\parskip0pt}

\begin{document}

\title{The \textsf{childdoc} Package}
\hypersetup{pdftitle={The childdoc Package}}
\author{Niklas Beisert\\[2ex]
  Institut f\"ur Theoretische Physik\\
  Eidgen\"ossische Technische Hochschule Z\"urich\\
  Wolfgang-Pauli-Strasse 27, 8093 Z\"urich, Switzerland\\[1ex]
  \href{mailto:nbeisert@itp.phys.ethz.ch}
  {\texttt{nbeisert@itp.phys.ethz.ch}}}
\hypersetup{pdfauthor={Niklas Beisert}}
\hypersetup{pdfsubject={Manual for the LaTeX2e Package childdoc}}
\date{30 December 2018, \textsf{v2.0}}
\maketitle

\begin{abstract}\noindent
\textsf{childdoc} is a \LaTeXe{} package
that enables the direct compilation
of document sections included by |\include|
to individual files.
\end{abstract}

\begingroup
\parskip0ex
\tableofcontents
\endgroup

%%%%%%%%%%%%%%%%%%%%%%%%%%%%%%%%%%%%%%%%%%%%%%%%%%%%%%%%%%%%%%%%%%%%%%%%%%%%%%%%
%%%%%%%%%%%%%%%%%%%%%%%%%%%%%%%%%%%%%%%%%%%%%%%%%%%%%%%%%%%%%%%%%%%%%%%%%%%%%%%%
\section{Introduction}

\LaTeX{} provides a mechanism to structure a large document (such as a book)
into a main file and several child files (containing the chapters)
using the |\include| command.
This mechanism is beneficial for documents
which span hundreds of pages in order to
make the source file(s) more manageable.
Moreover, compilation can be restricted to
selected child files by means of the |\includeonly| command.
The latter feature can be used to reduce the compilation time while editing
(this was significantly more useful in the earlier days of \LaTeX{})
or to generate a smaller document which is easier to navigate.
Another application of |\includeonly| is to generate
documents consisting of selected parts of the complete document.

However, there are a few drawbacks of the plain |\include| mechanism:
\begin{itemize}
\item
The child files cannot be compiled on their own,
they can only be compiled via the main file.
A naive editing environment
(such as a text editor with an option
to have the current file processed by \LaTeX)
may require one to switch to the main file before compiling;
attempting to compile the child file produces errors.
\item
The main file must be modified (each time)
to adjust the |\includeonly| command
to the present needs. This easily leaves the main file in a messy state.
\item
The generated document will always carry the filename
of the main document. This is inconvenient if
several child files are to be compiled and
to be kept for distribution.
\end{itemize}

The present package provides a simple interface
to make child files individually compilable by \LaTeX{}.
Compiling a child file then has the same effect as compiling
the main file with an |\includeonly| command
to select the appropriate child.
Moreover the generated document will carry the name of the child
rather than the main file.
This resolves all three above issues.

This feature is meant to make the editing of books,
thesis documents and lecture notes somewhat more convenient.
However, the package can also be used efficiently for
composing a series of documents (such as exercise sheets)
which are typically distributed individually.
It then assists the author in generating the individual documents
(potentially in different versions)
as well as a document containing the collected series.
Another application is in developing style files
or other kinds of included material
where compilation of the style file could redirect
to a sample or test file.

%%%%%%%%%%%%%%%%%%%%%%%%%%%%%%%%%%%%%%%%%%%%%%%%%%%%%%%%%%%%%%%%%%%%%%%%%%%%%%%%
%%%%%%%%%%%%%%%%%%%%%%%%%%%%%%%%%%%%%%%%%%%%%%%%%%%%%%%%%%%%%%%%%%%%%%%%%%%%%%%%
\section{Usage}

First of all, the package \textsf{childdoc} is \emph{not} a standard
\LaTeXe{} |.sty| style file! Therefore it needs to be invoked in
a non-standard way.

%%%%%%%%%%%%%%%%%%%%%%%%%%%%%%%%%%%%%%%%%%%%%%%%%%%%%%%%%%%%%%%%%%%%%%%%%%%%%%%%
\subsection{Included Files}
\label{sec:include}

%%%%%%%%%%%%%%%%%%%%%%%%%%%%%%%%%%%%%%%%
\DescribeMacro{\childdocmain}
To use the package, add the commands
\begin{center}
\begin{tabular}{l}
|\input{childdoc.def}|\\
|\childdocmain{}|\\
\end{tabular}
\end{center}
at the very top of the main \LaTeX{} file,
in particular \emph{before} the |\documentclass| statement!
The argument of |\childdocmain| should be left empty
(but it must be present).

%%%%%%%%%%%%%%%%%%%%%%%%%%%%%%%%%%%%%%%%
\DescribeMacro{\childdocof}
Furthermore, add the commands
\begin{center}
\begin{tabular}{l}
|\input{childdoc.def}|\\
|\childdocof{|\textit{main}|}|\\
\end{tabular}
\end{center}
at the top of every child file \textit{child}
which is included by |\include{|\textit{child}|}|
from within the main file
(or at least for those files to be compiled individually).
The argument \textit{main} must be the filename of the main file.

There are a couple of
considerations in setting up the main and child documents:

%%%%%%%%%%%%%%%%%%%%%%%%%%%%%%%%%%%%%%%%
\paragraph{Restrictions.}

Please note the following restrictions:
\begin{itemize}
\item
|\childdocmain| must be called with one argument \textit{main}
to ensure compatibility with earlier version of the package.
It must either be empty (|\childdocmain{}|)
or precisely match the filename of the main file in which it is specified.
See \secref{sec:detection} for further information.
\item
The filename \textit{main} must be specified without the |.tex| extension.
\item
The filename \textit{main} is case sensitive
(even in case-insensitive file systems)
due to internal string comparison.
\item
The argument \textit{main} should be fully expanded, it cannot be a macro.
\item
Subdirectories and special characters should be avoided in filenames.
\item
The command |\childdocmain{|\textit{main}|}| must be followed by a whitespace.
It should not be followed immediately by another command
or by a comment mark `|%|'.
This is because the \TeX{} parser reads the token immediately following
the argument of |\childdocmain| and puts it
at the beginning of every child section;
however, a white\-space is ignored.
\end{itemize}

%%%%%%%%%%%%%%%%%%%%%%%%%%%%%%%%%%%%%%%%
\paragraph{Content of Main File.}

It is advisable to place all content in the child files included by |\include|.
Any output contained in the main file will appear in all child documents
unless suppressed manually;
it cannot be suppressed automatically by the |\includeonly| directive
and thus should normally be avoided.
A method to include some content in the main file
by means of conditional processing is described in \secref{sec:conditional}.

%%%%%%%%%%%%%%%%%%%%%%%%%%%%%%%%%%%%%%%%
\paragraph{Page Numbering.}

When only a part of the document is compiled,
the appropriate numbering of pages
(as well as other status parameters)
is determined from the |.aux| files.
The latter contain information from previous passes.
However this information needs to propagate through
all intermediate child documents.
Therefore the page numbering in child documents may well
be inconsistent until the complete document is compiled at least once.

A useful (if unconventional) way to always ensure a consistent
page numbering is to restart the numbering in each child document
and denote the pages by `\textit{child}|.|\textit{page}'
where \textit{child} represents the chapter/section number of the child file.
This can be achieved by the command
|\numberwithin{page}{|\textit{child}|}|
of the \textsf{amsmath} package
where \textit{child} can be |chapter| or |section|
depending on the chosen structuring.
Alternatively, one can modify the macro |\thepage| appropriately
and reset the counter |page| at the start of each child file.

%%%%%%%%%%%%%%%%%%%%%%%%%%%%%%%%%%%%%%%%%%%%%%%%%%%%%%%%%%%%%%%%%%%%%%%%%%%%%%%%
\subsection{Conditional Processing}
\label{sec:conditional}

The package provides a mechanism to compile different versions
of a document. To customise the versions further some conditional processing
can come in handy to distinguish which version is being compiled.
The package provides two macros to describe the compilation context:

%%%%%%%%%%%%%%%%%%%%%%%%%%%%%%%%%%%%%%%%
\DescribeMacro{\ifchilddoc}
The conditional |\ifchilddoc| distinguishes between the compilation of
child documents and the main document:
%
\begin{center}
|\ifchilddoc |\textit{child-code}| |[|\||else |\textit{main-code}]| \||fi|
\end{center}

%%%%%%%%%%%%%%%%%%%%%%%%%%%%%%%%%%%%%%%%
\DescribeMacro{\childdocname}
\DescribeMacro{\childdocjob}
The macro |\childdocname| contains the filename (without extension)
of the main or child file being processed.
Note that |\childdocjob| will always contain the name of the main file.

%%%%%%%%%%%%%%%%%%%%%%%%%%%%%%%%%%%%%%%%
\paragraph{Title Page.}

Conditional processing can be used to include a title or banner page
in the main document when proper precautions are taken.
Importantly, the code in the main file should ensure that the page counter
(as well as other status parameters which are stored in the |.aux| files)
takes the same value after the conditional processing.
Otherwise the page numbers may take divergent values
depending on which part is compiled.

For example, a title page could be declared by:
%
\begin{center}
\begin{tabular}{l}
|\ifchilddoc\||else|\\
|\addtocounter{page}{-1}|\\
\textit{code for title page}\\
|\newpage|\\
|\||fi|
\end{tabular}
\end{center}
%
A banner page for the child documents can be generated by:
%
\begin{center}
\begin{tabular}{l}
|\ifchilddoc|\\
|\addtocounter{page}{-1}|\\
\textit{code for banner page}\\
|\newpage|\\
|\||fi|
\end{tabular}
\end{center}
%
Here one could write a message such as:
\begin{center}
|This is the part \childdocname{} of \childdocjob{}.|
\end{center}

%%%%%%%%%%%%%%%%%%%%%%%%%%%%%%%%%%%%%%%%%%%%%%%%%%%%%%%%%%%%%%%%%%%%%%%%%%%%%%%%
\subsection{Flags}
\label{sec:flags}

The package makes it easy to generate different versions
of the main or child documents.
To this end compilation flags can be defined
and assigned different default values.
They will be particularly useful in conjunction
with the forwarding mechanism described in \secref{sec:forward}.

For example, it may be useful to have a flag |\version|
which can be set to |draft| or |final|.
The document source will contain some conditional code
depending on the value of |\version|.
Suppose further, the flag should default to |final| for the main file
and to |draft| for child files
which is a natural assignment for editing the document.
This is achieved by placing the following code
in the preamble of the main document
(below the |\childdocmain| directive):
%
\begin{center}
\begin{tabular}{l}
|\ifchilddoc|\\
|\providecommand{\version}{draft}|\\
|\||else|\\
|\providecommand{\version}{final}|\\
|\||fi|
\end{tabular}
\end{center}
%
The definition by |\providecommand| makes sure
that previous definitions are not overwritten.
Further statements |\providecommand{\version}{...}|
can thus be added before the above code to override it.

For the main file, one might add a line
(between |\childdocmain| and the above block)
%
\begin{center}
|%\ifchilddoc\||else\providecommand{\version}{draft}\||fi|
\end{center}
%
which can be uncommented to produce a draft version.
Likewise one can add a line to the very top of a child file
(above the |\childdocof{|\textit{main}|}| directive)
%
\begin{center}
|%\providecommand{\version}{final}|
\end{center}
%
which can be uncommented to produce the final version of this child document.

%%%%%%%%%%%%%%%%%%%%%%%%%%%%%%%%%%%%%%%%%%%%%%%%%%%%%%%%%%%%%%%%%%%%%%%%%%%%%%%%
\subsection{Forwarding}
\label{sec:forward}

Different versions of the main or child documents
using compilation flags as described in \secref{sec:flags}
can be (permanently) stored in different files
for convenient compilation, viewing and distribution.
To this end, the package defines a command
to pass on compilation to a different file:

%%%%%%%%%%%%%%%%%%%%%%%%%%%%%%%%%%%%%%%%
\DescribeMacro{\childdocforward}
The command |\childdocforward| redirects processing to
another source file:
%
\begin{center}
\begin{tabular}{l}
|\input{childdoc.def}|\\
|\childdocforward[|\textit{main}|]{|\textit{dest}|}|\\
\end{tabular}
\end{center}
%
The argument \textit{dest} is the destination file
(without extension).
It should be the main file or one of the child files.
Note that further \textsf{childdoc} directives
such as |\childdocof| and |\childdocforward|
in the indicated file will be processed in this form.
The optional argument \textit{main}
passes on directly to the main file \textit{main}
while pretending to compile the child \textit{dest}.
This form behaves as if \textit{dest}
issues |\childdocof{|\textit{main}|}| right away,
and no further \textsf{childdoc} directives will be processed.

%%%%%%%%%%%%%%%%%%%%%%%%%%%%%%%%%%%%%%%%
\DescribeMacro{\...prefix}
In the alternative form |\childdocforwardprefix|,
%
\begin{center}
\begin{tabular}{l}
|\input{childdoc.def}|\\
|\childdocforwardprefix[|\textit{main}|]{|\textit{prefix}|}{|\textit{dest}|}|
\end{tabular}
\end{center}
%
the destination file is determined by a pattern
depending on the current file:
To make this work, the current file must be called
`{\textit{prefix}\hspace{0.2em}\textit{suffix}}'
with \textit{prefix} matching precisely the argument.
Processing is then passed on to the file
`{\textit{dest}\hspace{0.2em}\textit{suffix}}'.
Surely, the same effect is achieved by
directly specifying the
argument `{\textit{dest}\hspace{0.2em}\textit{suffix}}'
in the first form.
However, that requires to set up a different file
for each child. With the alternative form of the command
all these files can have exactly the same content
which simplifies setting them up and maintaining them.

For example, the following file |draft.tex|
with a compilation flag |\version| as described in \secref{sec:flags}
compiles the main document as a draft:
%
\begin{center}
\begin{tabular}{l}
|\def\version{draft}|\\
|\input{childdoc.def}|\\
|\childdocforward{|\textit{main}|}|
\end{tabular}
\end{center}
%
Likewise, the following files |final|\textit{nn}|.tex|
compile the final version of the child document
|child|\textit{nn}|.tex|:
%
\begin{center}
\begin{tabular}{l}
|\def\version{final}|\\
|\input{childdoc.def}|\\
|\childdocforwardprefix{final}{child}|
\end{tabular}
\end{center}
%

Note that when several versions of a main file and/or of each child file
are to be generated, it may be convenient to set up a |Makefile| or
shell script to automatise the process.

%%%%%%%%%%%%%%%%%%%%%%%%%%%%%%%%%%%%%%%%%%%%%%%%%%%%%%%%%%%%%%%%%%%%%%%%%%%%%%%%
\subsection{Command Line Processing}
\label{sec:commandline}

The effect of redirection files can also be achieved by invoking
the \LaTeX{} compiler with a more elaborate command line.
Most conveniently this should be done as part
of a shell script or a |Makefile|.

When using \textsf{childdoc} in the main file, the following
command lines effectively perform a redirection
(note that depending on the shell being used,
backslashes may have to be doubled: `|\|' $\to$ `|\\|'):
%
\begin{center}
|... -jobname "|\textit{target}|" |\\|"|[\textit{flags}]%
|\input{childdoc.def}\childdocforward[|\textit{main}|]{|\textit{dest}|}"|
\end{center}
%
Here \textit{target} is the name of the output file,
\textit{main} is the name of the main file
and \textit{dest} is the name of the main or child file to be processed
(all filenames without extensions).
The optional argument \textit{main} can be omitted
if \textit{main} matches \textit{dest}.
Optionally, compilation \textit{flags} can be defined via |\def| commands.
This command line makes the \TeX{} engine believe
it is compiling the file \textit{target}
whose content is specified as the latter parameter.
The provided code then forwards the processing to
\textit{main} or \textit{dest} as described in \secref{sec:forward}.

%%%%%%%%%%%%%%%%%%%%%%%%%%%%%%%%%%%%%%%%%%%%%%%%%%%%%%%%%%%%%%%%%%%%%%%%%%%%%%%%
\subsection{Include by Input}
\label{sec:input}

Including child documents by |\include| has some restrictions by design.
Most notably, the content of a child document always occupies
its own set of pages; pages cannot be shared between child documents.
Usually, this behaviour makes perfect sense
because each child document contain an essential part of the document.
However, in some situations it may be desirable to compose
a document from a collection of parts
without having mandatory page breaks between then.
For this case, the package
provides a mechanism to include parts
by |\input| which can also be processed individually.
However, by construction this mechanism
requires manual handling of the content to be output.

%%%%%%%%%%%%%%%%%%%%%%%%%%%%%%%%%%%%%%%%
\DescribeMacro{\ifchilddocmanual}
The main file should be prepared as usual, see \secref{sec:include}.
However, the document body must make a distinction
between processing of an individual part and of the main document, e.g.:
%
\begin{center}
\begin{tabular}{l}
|\ifchilddocmanual|\\
|\input{\childdocname}|\\
|\||else|\\
\textit{document body with }|\input{|\textit{part}|}|\\
|\||fi|
\end{tabular}
\end{center}
%
The conditional |\ifchilddocmanual| is true whenever
a part to be included by |\input| is being compiled,
and the name of the part is stored in |\childdocname|.

%%%%%%%%%%%%%%%%%%%%%%%%%%%%%%%%%%%%%%%%
\DescribeMacro{\childdocby}
Each part to be included by |\input| should start with:
%
\begin{center}
\begin{tabular}{l}
|\input{childdoc.def}|\\
|\childdocby{|\textit{main}|}|\\
\end{tabular}
\end{center}
%
The directive |\childdocby| is similar to |\childdocof|
described in \secref{sec:include},
but the subsequent selection of content must be done manually.
To that end, both |\ifchilddoc| and |\ifchilddocmanual|
will be true upon processing of a part,
and the name of the part is stored in |\childdocname|.
Note that |\jobname| will be set to the filename of the current part
so that each part receives an individual |.aux| file
that does not interfere with the |.aux| file(s) of the main document.
This behaviour can be altered by the alternative form
|\childdocby[*]{|\textit{main}|}| (with a non-empty optional argument)
which uses the |.aux| file of the main document
by setting |\jobname| to \textit{main}.

%%%%%%%%%%%%%%%%%%%%%%%%%%%%%%%%%%%%%%%%%%%%%%%%%%%%%%%%%%%%%%%%%%%%%%%%%%%%%%%%
\subsection{Driver Development}
\label{sec:driver}

The \textsf{childdoc} mechanism can also be use for the development
of definition files such as \LaTeX{} styles or classes.
This case differs from the above setup with multiple parts
included by |\include| in that no |\includeonly| should be invoked.
This can be achieved by starting the include file
(before |\ProvidesPackage|) with:
%
\begin{center}
\begin{tabular}{l}
|\input{childdoc.def}|\\
|\childdocforward{|\textit{main}|}|\\
\end{tabular}
\end{center}
%
or alternatively with:
%
\begin{center}
\begin{tabular}{l}
|\input{childdoc.def}|\\
|\childdocby{|\textit{main}|}|\\
\end{tabular}
\end{center}
%
Both forms have slightly different effects as described above.
The main file is prepared as usual, see \secref{sec:include}.

%%%%%%%%%%%%%%%%%%%%%%%%%%%%%%%%%%%%%%%%%%%%%%%%%%%%%%%%%%%%%%%%%%%%%%%%%%%%%%%%
\subsection{Legacy Detection}
\label{sec:detection}

The directive |\childdocmain| in the main file can detect
whether the complete document or merely a child is to be compiled
even without using the directive |\childdocof|.
This method is deprecated because it is less robust
and there is no compelling reason to use it;
it is merely provided for backward compatibility
and it may be removed in future versions.

If the detection mechanism is to be used,
it is mandatory to correctly specify
the filename of the main file as the argument of |\childdocmain|:
%
\begin{center}
\begin{tabular}{l}
|\input{childdoc.def}|\\
|\childdocmain{|\textit{main}|}|\\
\end{tabular}
\end{center}
%
If |\jobname| does not match the argument \textit{main} of |\childdocmain|,
it is assumed that |\jobname| points to the child file to be compiled.
When using |\childdocmain| with the main file specified as argument,
it suffices to start a child file
with just |\input{|\textit{main}|}|
without loading of the package and using |\childdocof|.
If instead all processing is done
with the appropriate \textsf{childdoc} directives,
the argument of \textit{main} of |\childdocmain| can be empty.

An alternative version of the command line processing described
in \secref{sec:commandline} using the detection mechanism reads:
%
\begin{center}
|... -jobname "|\textit{target}|" "|[\textit{flags}]%
[|\def\jobname{|\textit{dest}|}|]|\input{|\textit{main}|}"|
\end{center}

%%%%%%%%%%%%%%%%%%%%%%%%%%%%%%%%%%%%%%%%%%%%%%%%%%%%%%%%%%%%%%%%%%%%%%%%%%%%%%%%
\subsection{Manual Code}
\label{sec:manual}

In case one cannot be certain whether the definitions file |childdoc.def|
is installed on the target \TeX{} distribution
and one prefers not to ship it,
it is conceivable to paste a few relevant commands into the sources.

To that end, drop all statements |\input{childdoc.def}|
and perform the replacements as outlined below.
Instead of |\childdocmain{|\textit{main}|}| add the following code
to the top of the main file:
%
\begin{center}
\begin{tabular}{l}
|\||ifdefined\childdocname\endinput\||fi\newif\ifchilddoc|\\
|\edef\childdocname{\scantokens\expandafter{\jobname\noexpand}}|\\
|\def\childdocmain{|\textit{main}|}\||ifx\childdocmain\childdocname\||else|\\
|\childdoctrue\includeonly{\childdocname}\let\jobname\childdocmain\||fi|\\
\end{tabular}
\end{center}
%
Instead of |\childdocof{|\textit{main}|}| just include the main file
at the top of each child file:
%
\begin{center}
|\input{|\textit{main}|}|
\end{center}
%
A simple redirection |\childdocforward{|\textit{dest}|}| is achieved by:
%
\begin{center}
|\def\jobname{|\textit{dest}|}\input{\jobname}|
\end{center}
%
The redirection with prefix
|\childdocforwardprefix[|\textit{prefix}|]{|\textit{dest}|}|
is accomplished by:
%
\begin{center}
\begin{tabular}{l}
|{\edef\jobname{\scantokens\expandafter{\jobname\noexpand}}|\\
|\def\redirectjob |\textit{prefix}|#1~~~{\gdef\jobname{|\textit{dest}|#1}}|\\
|\expandafter\redirectjob\jobname~~~}\input{\jobname}|
\end{tabular}
\end{center}

In an alternative approach,
child documents can be compiled by a specific command line
without additional code or specific definitions:
%
\begin{center}
|... -jobname "|\textit{target}|" "|[\textit{flags}]%
|\includeonly{|\textit{dest}|}\input{|\textit{main}|}"|
\end{center}
%

%%%%%%%%%%%%%%%%%%%%%%%%%%%%%%%%%%%%%%%%%%%%%%%%%%%%%%%%%%%%%%%%%%%%%%%%%%%%%%%%
%%%%%%%%%%%%%%%%%%%%%%%%%%%%%%%%%%%%%%%%%%%%%%%%%%%%%%%%%%%%%%%%%%%%%%%%%%%%%%%%
\section{Information}

%%%%%%%%%%%%%%%%%%%%%%%%%%%%%%%%%%%%%%%%%%%%%%%%%%%%%%%%%%%%%%%%%%%%%%%%%%%%%%%%
\subsection{Copyright}

Copyright \copyright{} 2017--2018 Niklas Beisert

This work may be distributed and/or modified under the
conditions of the \LaTeX{} Project Public License, either version 1.3
of this license or (at your option) any later version.
The latest version of this license is in
  \url{http://www.latex-project.org/lppl.txt}
and version 1.3 or later is part of all distributions of \LaTeX{}
version 2005/12/01 or later.

This work has the LPPL maintenance status `maintained'.

The Current Maintainer of this work is Niklas Beisert.

This work consists of the files |README.txt|, |childdoc.ins| and |childdoc.dtx|
as well as the derived files |childdoc.def|, |cdocsamp.tex|
with |cdocsch1.tex|, |cdocsch2.tex|, |cdocspt3.tex|, |cdocspt4.tex|,
|cdocsdrf.tex|, |cdocsfn1.tex|, |cdocsfn2.tex|
as well as |childdoc.pdf|.

%%%%%%%%%%%%%%%%%%%%%%%%%%%%%%%%%%%%%%%%%%%%%%%%%%%%%%%%%%%%%%%%%%%%%%%%%%%%%%%%
\subsection{Files and Installation}

The package consists of the files:
%
\begin{center}
\begin{tabular}{ll}
    |README.txt|   & readme file \\
    |childdoc.ins| & installation file \\
    |childdoc.dtx| & source file \\
    |childdoc.def| & definition file \\
    |cdocsamp.tex| & sample main file \\
    |cdocsch1.tex| & sample include file \\
    |cdocsch2.tex| & sample include file \\
    |cdocspt3.tex| & sample part file \\
    |cdocspt4.tex| & sample part file \\
    |cdocsdrf.tex| & sample redirection file \\
    |cdocsfn1.tex| & sample redirection file \\
    |cdocsfn2.tex| & sample redirection file \\
    |childdoc.pdf| & manual
\end{tabular}
\end{center}
%
The distribution consists of the files
|README.txt|, |childdoc.ins| and |childdoc.dtx|.
%
\begin{itemize}
\item
Run (pdf)\LaTeX{} on |childdoc.dtx|
to compile the manual |childdoc.pdf| (this file).
\item
Run \LaTeX{} on |childdoc.ins| to create the definitions file |childdoc.def|
and the sample |cdocsamp.tex| with include files
|cdocsch1.tex|, |cdocsch2.tex|, |cdocspt3.tex|, |cdocspt4.tex|,
|cdocsdrf.tex|, |cdocsfn1.tex|, |cdocsfn2.tex|.
Then copy the file |childdoc.def| to an appropriate directory of your \LaTeX{}
distribution, e.g.\ \textit{texmf-root}|/tex/latex/childdoc|.
\end{itemize}

%%%%%%%%%%%%%%%%%%%%%%%%%%%%%%%%%%%%%%%%%%%%%%%%%%%%%%%%%%%%%%%%%%%%%%%%%%%%%%%%
\subsection{Related CTAN Packages}

There are several other packages which offer a similar functionality:
%
\begin{itemize}
\item
The packages
\href{http://ctan.org/pkg/docmute}{\textsf{docmute}},
\href{http://ctan.org/pkg/includex}{\textsf{includex}} and
\href{http://ctan.org/pkg/standalone}{\textsf{standalone}}
provide commands to include only the document body of
a child file thus allowing both files to be compiled individually.
\item
The packages \href{http://ctan.org/pkg/subdocs}{\textsf{subdocs}}
and \href{http://ctan.org/pkg/subfiles}{\textsf{subfiles}}
provide structures in which the main and child documents can be
encapsulated and allowing them to be compiled individually.
The inclusion mechanism is different from the conventional |\include|.
\item
The package \href{http://ctan.org/pkg/combine}{\textsf{combine}}
is an elaborate solution to combine several documents into one.
\end{itemize}
%
See also the CTAN topic \href{http://ctan.org/topic/subdocs}{\textsf{subdocs}}
for further related packages.
The present package differs from the above solutions in that
a document structure constructed with the conventional |\include| mechanism
just needs two extra commands at the top of every file
such that all constituent files can be compiled individually.

%%%%%%%%%%%%%%%%%%%%%%%%%%%%%%%%%%%%%%%%%%%%%%%%%%%%%%%%%%%%%%%%%%%%%%%%%%%%%%%%
%\subsection{Feature Suggestions}
%
%The following is a list of features which may be useful for future
%versions of this package:
%%
%\begin{itemize}
%\item
%\ldots
%\end{itemize}

%%%%%%%%%%%%%%%%%%%%%%%%%%%%%%%%%%%%%%%%%%%%%%%%%%%%%%%%%%%%%%%%%%%%%%%%%%%%%%%%
\subsection{Revision History}

%%%%%%%%%%%%%%%%%%%%%%%%%%%%%%%%%%%%%%%%
\paragraph{v2.0:} 2018/12/30

\begin{itemize}
\item
immediate forward processing
\item
added |\childdocby| mechanism
\item
manual restructured
\end{itemize}

%%%%%%%%%%%%%%%%%%%%%%%%%%%%%%%%%%%%%%%%
\paragraph{v1.6:} 2018/01/17

\begin{itemize}
\item
application for development of include files
\item
corrections to manual
\end{itemize}

%%%%%%%%%%%%%%%%%%%%%%%%%%%%%%%%%%%%%%%%
\paragraph{v1.5:} 2017/05/21

\begin{itemize}
\item
more complete structuring introduced
\item
|\childdocof| introduced
\item
|\childdoc| renamed to |\childdocmain|
\item
|\childredirect| renamed to |\childdocforward| and |\childdocforwardprefix|
and functionality expanded
\end{itemize}

%%%%%%%%%%%%%%%%%%%%%%%%%%%%%%%%%%%%%%%%
\paragraph{v1.0:} 2017/04/27

\begin{itemize}
\item
manual and install package
\item
first version published on CTAN
\end{itemize}

%%%%%%%%%%%%%%%%%%%%%%%%%%%%%%%%%%%%%%%%
\paragraph{v0.6:} 2017/04/26

\begin{itemize}
\item
redirection mechanism added
\end{itemize}

%%%%%%%%%%%%%%%%%%%%%%%%%%%%%%%%%%%%%%%%
\paragraph{v0.5:} 2017/04/26

\begin{itemize}
\item
functionality in definition file
\end{itemize}


%%%%%%%%%%%%%%%%%%%%%%%%%%%%%%%%%%%%%%%%%%%%%%%%%%%%%%%%%%%%%%%%%%%%%%%%%%%%%%%%
%%%%%%%%%%%%%%%%%%%%%%%%%%%%%%%%%%%%%%%%%%%%%%%%%%%%%%%%%%%%%%%%%%%%%%%%%%%%%%%%
%%%%%%%%%%%%%%%%%%%%%%%%%%%%%%%%%%%%%%%%%%%%%%%%%%%%%%%%%%%%%%%%%%%%%%%%%%%%%%%%
\appendix

\settowidth\MacroIndent{\rmfamily\scriptsize 000\ }

 \DocInput{childdoc.dtx}

\end{document}
%</driver>
% \fi
%
% %%%%%%%%%%%%%%%%%%%%%%%%%%%%%%%%%%%%%%%%%%%%%%%%%%%%%%%%%%%%%%%%%%%%%%%%%%%%%%
% %%%%%%%%%%%%%%%%%%%%%%%%%%%%%%%%%%%%%%%%%%%%%%%%%%%%%%%%%%%%%%%%%%%%%%%%%%%%%%
% \section{Sample}
%\iffalse
%<*samplemain>
%\fi
%
% The following presents a sample document
% with two chapters, two parts, a title page,
% a compile flag as well as three forwarding files to set the flag.
% It consists of eight |.tex| files:
% \begin{center}
% \begin{tabular}{ll}
% |cdocsamp.tex|&main file\\
% |cdocsch1.tex|&include file for chapter 1\\
% |cdocsch2.tex|&include file for chapter 2\\
% |cdocspt3.tex|&include file for part 3\\
% |cdocspt4.tex|&include file for part 4\\
% |cdocsdrf.tex|&forwarding file for main file in draft mode\\
% |cdocsfi1.tex|&forwarding file for final version of chapter 1\\
% |cdocsfi2.tex|&forwarding file for final version of chapter 2\\
% \end{tabular}
% \end{center}
% Each of the eight files can be compiled directly by the \LaTeX{} compiler.
%
% %%%%%%%%%%%%%%%%%%%%%%%%%%%%%%%%%%%%%%
% \paragraph{Main File.}
%
% The main file is called |cdocsamp.tex|.
%
% Load the \textsf{childdoc} definitions and
% declare the filename for the main document:
%    \begin{macrocode}
\input{childdoc.def}
\childdocmain{}
%    \end{macrocode}

% Optional override for |\version| flag:
%    \begin{macrocode}
%%\ifchilddoc\else\providecommand{\version}{draft}\fi
%    \end{macrocode}

% Define the default values for the |\version| flag
% (|final| for the main file and |draft| for childs):
%    \begin{macrocode}
\ifchilddoc
\providecommand{\version}{draft}
\else
\providecommand{\version}{final}
\fi
%    \end{macrocode}

% Load the standard document class:
%    \begin{macrocode}
\documentclass[12pt]{article}
%    \end{macrocode}

% Start the document body:
%    \begin{macrocode}
\begin{document}
%    \end{macrocode}

% Declare a title page.
% Print title, part of document being processed and version flag:
%    \begin{macrocode}
\addtocounter{page}{-1}
\begin{center}
{\LARGE\bfseries{}childdoc example\par}
\vspace{1cm}
\ifchilddoc
\ifchilddocmanual part\else chapter\fi:
`\childdocname' of `\childdocjob'\par
\else
main document: `\childdocjob'\par
\fi
version: \version\par
\end{center}
\newpage
%    \end{macrocode}

% Manually include selected file,
% otherwise process as usual:
%    \begin{macrocode}
\ifchilddocmanual
\section*{part `\childdocname'}
\input{\childdocname}
\else
%    \end{macrocode}

% Include the two chapters:
%    \begin{macrocode}
\include{cdocsch1}
\include{cdocsch2}
%    \end{macrocode}

% Include the two parts unless only chapters should be displayed:
%    \begin{macrocode}
\ifchilddoc\else
\section{part three}
\input{cdocspt3}
\section{part four}
\input{cdocspt4}
\fi
%    \end{macrocode}

% Process as usual until here:
%    \begin{macrocode}
\fi
%    \end{macrocode}

% End of document body:
%    \begin{macrocode}
\end{document}
%    \end{macrocode}
%\iffalse
%</samplemain>
%\fi
%
% %%%%%%%%%%%%%%%%%%%%%%%%%%%%%%%%%%%%%%
% \paragraph{Chapter Include Files.}
%
% The include files are called |cdocsch1.tex| and |cdocsch2.tex|.
%
%\iffalse
%<*samplechap1|samplechap2>
%\fi

% Optional override for |\version| flag:
%    \begin{macrocode}
%%\providecommand{\version}{final}
%    \end{macrocode}

% Include the main document:
%    \begin{macrocode}
\input{childdoc.def}
\childdocof{cdocsamp}
%    \end{macrocode}

%\iffalse
%</samplechap1|samplechap2>
%\fi
%
%\iffalse
%<*samplechap1>
%\fi
% Some text for chapter 1:
%    \begin{macrocode}
\section{one}
some text in chapter one
%    \end{macrocode}

%\iffalse
%</samplechap1>
%\fi
% Some text for chapter 2:
%\iffalse
%<*samplechap2>
%\fi
%    \begin{macrocode}
\section{two}
more text in chapter two
%    \end{macrocode}

%\iffalse
%</samplechap2>
%\fi
%
% %%%%%%%%%%%%%%%%%%%%%%%%%%%%%%%%%%%%%%
% \paragraph{Part Include Files.}
%
% The include files are called |cdocspt3.tex| and |cdocspt4.tex|.
%
%\iffalse
%<*samplepart3|samplepart4>
%\fi

% Optional override for |\version| flag:
%    \begin{macrocode}
%%\providecommand{\version}{final}
%    \end{macrocode}

% Include the main document:
%    \begin{macrocode}
\input{childdoc.def}
\childdocby{cdocsamp}
%    \end{macrocode}

%\iffalse
%</samplepart3|samplepart4>
%\fi
%
%\iffalse
%<*samplepart3>
%\fi
% Some text for part 3:
%    \begin{macrocode}
some text in part three
%    \end{macrocode}

%\iffalse
%</samplepart3>
%\fi
% Some text for part 4:
%\iffalse
%<*samplepart4>
%\fi
%    \begin{macrocode}
more text in part four
%    \end{macrocode}

%\iffalse
%</samplepart4>
%\fi
%
% %%%%%%%%%%%%%%%%%%%%%%%%%%%%%%%%%%%%%%
% \paragraph{Forwarding for a Complete Draft.}
%
% The following forwarding file |cdocsdrf.tex|
% compiles the main document in draft mode:
%\iffalse
%<*sampledraft>
%\fi
%    \begin{macrocode}
\def\version{draft}
\input{childdoc.def}
\childdocforward{cdocsamp}
%    \end{macrocode}

%\iffalse
%</sampledraft>
%\fi
%
% %%%%%%%%%%%%%%%%%%%%%%%%%%%%%%%%%%%%%%
% \paragraph{Forwarding for Final Version of the Chapters.}
%
% The following forwarding files |cdocsfn1.tex| and |cdocsfn2.tex|
% (with identical content)
% compile the final versions of the child documents
% |cdocsch1.tex| and |cdocsch2.tex|, respectively:
%\iffalse
%<*samplefinal>
%\fi
%    \begin{macrocode}
\def\version{final}
\input{childdoc.def}
\childdocforwardprefix[cdocsamp]{cdocsfn}{cdocsch}
%    \end{macrocode}

%\iffalse
%</samplefinal>
%\fi
%
% %%%%%%%%%%%%%%%%%%%%%%%%%%%%%%%%%%%%%%
% \paragraph{Command Line Processing.}
%
% The following three command lines generate the output files
% |cdocscld|, |cdocscl1| and |cdocscl2|
% which should be identical to
% |cdocsdrf|, |cdocsch1| and |cdocsfn2|, respectively:
% \begin{center}
% \begin{tabular}{l}
% |latex -jobname cdocscld \|\\
% |  "\def\version{draft}\input{childdoc.def}\childdocforward{cdocsamp}"|\\
% |latex -jobname cdocscl1 \|\\
% |  "\input{childdoc.def}\childdocforward[cdocsamp]{cdocsch1}"|\\
% |latex -jobname cdocscl2 \|\\
% |  "\def\version{final}\input{childdoc.def}\childdocforward{cdocsch2}"|
% \end{tabular}
% \end{center}
% Note that the trailing backslash on each first line
% merely continues the input to the second line
% (for convenient cut ant paste).
% Furthermore, the command |latex| can be replaced by any
% of its alternative versions such as |pdflatex|.
%
% %%%%%%%%%%%%%%%%%%%%%%%%%%%%%%%%%%%%%%%%%%%%%%%%%%%%%%%%%%%%%%%%%%%%%%%%%%%%%%
% %%%%%%%%%%%%%%%%%%%%%%%%%%%%%%%%%%%%%%%%%%%%%%%%%%%%%%%%%%%%%%%%%%%%%%%%%%%%%%
% \section{Implementation}
%\iffalse
%<*package>
%\fi
%
% This section describes the definitions file |childdoc.def|.

% The definitions cannot be loaded using |\usepackage| or |\RequirePackage|
% which has a mechanism to prevent loading a style file more than once.
% When loading the definitions by means of |\input|
% multiple instances have to be prevented manually:
%\iffalse
%This code needs to be before the `\ProvidesFile' directive
%which is defined at the beginning of this file.
%Therefore it is also placed there and commented out here.
%</package>
%<*discard>
%\fi
%    \begin{macrocode}
\ifdefined\childdocmain\endinput\fi
%    \end{macrocode}
%\iffalse
%</discard>
%<*package>
%\fi
%
% \macro{\ifchilddoc}
% \macro{\ifchilddocmanual}
% The conditional |\ifchilddoc| tells whether a
% child (true) or main (false) document is being compiled.
% The conditional |\ifchilddocmanual| tells whether
% the |\includeonly| mechanism is used (false) or
% the selection of child files must be performed manually (true).
% The definitions initialise to false:
%    \begin{macrocode}
\newif\ifchilddoc
\newif\ifchilddocmanual
%    \end{macrocode}

% \macro{\childdocname}
% \macro{\childdocjob}
% The macro |\childdocname| stores the name of the main document
% to be compiled. The macro |\childdocjob| stores the name of
% the document on which the \LaTeX{} compiler was originally invoked.
% The content of |\jobname| cannot be compared
% to filenames specified in the source due to different catcodes.
% The following code rescans |\jobname|, stores the result
% in |\childdocname| and saves a copy in |\childdocjob|:
%    \begin{macrocode}
\edef\childdocname{\scantokens\expandafter{\jobname\noexpand}}
\let\childdocjob\childdocname
%    \end{macrocode}

% \macro{\childdocdisable}
% The macro |\childdocdisable| prevents the main file
% from being processed more than once.
% At this stage, the main document command |\childdocmain|
% is assumed to be called once again where it should do nothing.
% Any subsequent call to it should prevent
% a secondary processing of the main document
% It overwrites the forwarding commands
% |\childdocof| and |\childdocforward|
% with empty macros to prevent further inclusions of the main document:
%    \begin{macrocode}
\newcommand{\childdocdisable}
{
  \renewcommand{\childdocmain}[1]{\renewcommand{\childdocmain}[1]{\endinput}}
  \renewcommand{\childdocof}[1]{}
  \renewcommand{\childdocby}[2][]{}
  \renewcommand{\childdocforward}[2][]{}
  \renewcommand{\childdocdisable}{}
}
%    \end{macrocode}

% \macro{\childdocmain}
% The macro |\childdocmain| is to be called at the top of the main file
% with nothing or the main filename (without extension) as argument.
% First, it breaks loops.
% If the argument is not empty and does not match |\childdocname|
% (which is set by the first inclusion of |childdoc.def|),
% |\ifchilddoc| is set to true, |\includeonly| is applied to the child file
% and |\jobname| is set to the main file
% (for proper handling of |.aux| files):
%    \begin{macrocode}
\newcommand{\childdocmain}[1]
{
  \childdocdisable\childdocmain{}
  \if?#1?\else
    \begingroup
      \def\childdoctmp{#1}
      \ifx\childdoctmp\childdocname
        \def\childdoctmp{}
      \else
        \def\childdoctmp
        {
          \childdoctrue
          \includeonly{\childdocname}
          \def\childdocjob{#1}
          \def\jobname{#1}
        }
      \fi
      \expandafter
    \endgroup
    \childdoctmp
  \fi
}
%    \end{macrocode}

% \macro{\childdocof}
% The command |\childdocof| redirects
% compilation to the main file |#1|.
%    \begin{macrocode}
\newcommand{\childdocof}[1]
{
  \childdocdisable
  \childdoctrue
  \includeonly{\childdocname}
  \def\jobname{#1}
  \def\childdocjob{#1}
  \input{#1}
}
%    \end{macrocode}

% \macro{\childdocby}
% The command |\childdocby| ....
%    \begin{macrocode}
\newcommand{\childdocby}[2][]
{
  \childdocdisable
  \childdoctrue
  \childdocmanualtrue
  \if?#1?\else
    \def\jobname{#2}
  \fi
  \def\childdocjob{#2}
  \input{#2}
  \endinput
}
%    \end{macrocode}

% \macro{\childdocforward}
% The command |\childdocforward| redirects
% compilation to the main file or
% (if the optional argument is given) a child file.
% Parameters are set as if the main file
% or a child file starting with |\childdocof| was compiled.
% Then compilation is handed over to the main file:
%    \begin{macrocode}
\newcommand{\childdocforward}[2][]
{
  \begingroup
    \if?#1?
      \def\childdoctmp
      {
        \def\childdocname{#2}
        \def\childdocjob{#2}
        \def\jobname{#2}
        \input{#2}
        \endinput
      }
    \else
      \def\childdoctmp
      {
        \childdocdisable
        \def\childdocname{#2}
        \childdoctrue
        \includeonly{#2}
        \def\childdocjob{#1}
        \def\jobname{#1}
        \input{#1}
        \endinput
      }
    \fi
    \expandafter
  \endgroup
  \childdoctmp
}
%    \end{macrocode}

% \macro{\childdocforwardprefix}
% The command |\childdocforwardprefix| redirects
% compilation to the main or a child file by means of a pattern.
% The prefix |#1| in the current filename is replaced by |#2|
% and the suffix of the current filename is kept
% (it is assumed that the filename does not contain the substring `|~~~|'
% which is used as a delimiter).
% Compilation is handed over to the new file by |\childdocforward|:
%    \begin{macrocode}
\newcommand{\childdocforwardprefix}[3][]
{
  \begingroup
    \def\childdocextract #2##1~~~{\def\childdoctmp{\childdocforward[#1]{#3##1}}}
    \expandafter\childdocextract\childdocname~~~
    \expandafter
  \endgroup
  \childdoctmp
}
%    \end{macrocode}

% \macro{\childdoc}
% The deprecated macro |\childdoc| is a legacy version of |\childdocmain|:
%    \begin{macrocode}
\newcommand{\childdoc}{\childdocmain}
%    \end{macrocode}

% \macro{\childdocredirect}
% The deprecated macro |\childdocredirect| is a legacy version
% of |\childdocforward| and |\childdocforwardprefix|:
%    \begin{macrocode}
\newcommand{\childdocredirect}[2][]
{
  \begingroup
    \if?#1?
      \def\childdoctmp{\childdocforward{#2}}
    \else
      \def\childdoctmp{\childdocforwardprefix{#1}{#2}}
    \fi
    \expandafter
  \endgroup
  \childdoctmp
}
%    \end{macrocode}

%\iffalse
%</package>
%\fi
%
\endinput
|\\
|\childdocforward{|\textit{main}|}|\\
\end{tabular}
\end{center}
%
or alternatively with:
%
\begin{center}
\begin{tabular}{l}
|% \iffalse
%
% childdoc.dtx Copyright (C) 2017-2018 Niklas Beisert
%
% This work may be distributed and/or modified under the
% conditions of the LaTeX Project Public License, either version 1.3
% of this license or (at your option) any later version.
% The latest version of this license is in
%   http://www.latex-project.org/lppl.txt
% and version 1.3 or later is part of all distributions of LaTeX
% version 2005/12/01 or later.
%
% This work has the LPPL maintenance status `maintained'.
%
% The Current Maintainer of this work is Niklas Beisert.
%
% This work consists of the files childdoc.dtx and childdoc.ins
% and the derived files childdoc.def and cdocsamp.tex with
% cdocsch1.tex, cdocsch2.tex, cdocsdrf.tex, cdocsfn1.tex, cdocsfn2.tex.
%
%<package>\ifdefined\childdocmain\endinput\fi
%<package>\ProvidesFile{childdoc.def}[2018/12/30 v2.0 child document driver]
%<samplemain>\ProvidesFile{cdocsamp.tex}[2018/12/30 v2.0 sample for childdoc]
%<*driver>
%\ProvidesFile{childdoc.drv}[2018/12/30 v2.0 childdoc reference manual file]
\PassOptionsToClass{10pt,a4paper}{article}
\documentclass{ltxdoc}

\usepackage[margin=35mm]{geometry}
\usepackage{hyperref}
\usepackage{hyperxmp}
\usepackage[usenames]{color}

\hypersetup{colorlinks=true}
\hypersetup{pdfstartview=FitH}
\hypersetup{pdfpagemode=UseNone}
\hypersetup{pdfsource={}}
\hypersetup{pdflang={en-UK}}
\hypersetup{pdfcopyright={Copyright 2017-2018 Niklas Beisert.
  This work may be distributed and/or modified under the
  conditions of the LaTeX Project Public License, either version 1.3
  of this license or (at your option) any later version.}}
\hypersetup{pdflicenseurl={http://www.latex-project.org/lppl.txt}}
\hypersetup{pdfcontactaddress={ETH Zurich, ITP, HIT K,
  Wolfgang-Pauli-Strasse 27}}
\hypersetup{pdfcontactpostcode={8093}}
\hypersetup{pdfcontactcity={Zurich}}
\hypersetup{pdfcontactcountry={Switzerland}}
\hypersetup{pdfcontactemail={nbeisert@itp.phys.ethz.ch}}
\hypersetup{pdfcontacturl={http://people.phys.ethz.ch/\xmptilde nbeisert/}}

\newcommand{\secref}[1]{\hyperref[#1]{section \ref*{#1}}}

\parskip1ex
\parindent0pt
\let\olditemize\itemize
\def\itemize{\olditemize\parskip0pt}

\begin{document}

\title{The \textsf{childdoc} Package}
\hypersetup{pdftitle={The childdoc Package}}
\author{Niklas Beisert\\[2ex]
  Institut f\"ur Theoretische Physik\\
  Eidgen\"ossische Technische Hochschule Z\"urich\\
  Wolfgang-Pauli-Strasse 27, 8093 Z\"urich, Switzerland\\[1ex]
  \href{mailto:nbeisert@itp.phys.ethz.ch}
  {\texttt{nbeisert@itp.phys.ethz.ch}}}
\hypersetup{pdfauthor={Niklas Beisert}}
\hypersetup{pdfsubject={Manual for the LaTeX2e Package childdoc}}
\date{30 December 2018, \textsf{v2.0}}
\maketitle

\begin{abstract}\noindent
\textsf{childdoc} is a \LaTeXe{} package
that enables the direct compilation
of document sections included by |\include|
to individual files.
\end{abstract}

\begingroup
\parskip0ex
\tableofcontents
\endgroup

%%%%%%%%%%%%%%%%%%%%%%%%%%%%%%%%%%%%%%%%%%%%%%%%%%%%%%%%%%%%%%%%%%%%%%%%%%%%%%%%
%%%%%%%%%%%%%%%%%%%%%%%%%%%%%%%%%%%%%%%%%%%%%%%%%%%%%%%%%%%%%%%%%%%%%%%%%%%%%%%%
\section{Introduction}

\LaTeX{} provides a mechanism to structure a large document (such as a book)
into a main file and several child files (containing the chapters)
using the |\include| command.
This mechanism is beneficial for documents
which span hundreds of pages in order to
make the source file(s) more manageable.
Moreover, compilation can be restricted to
selected child files by means of the |\includeonly| command.
The latter feature can be used to reduce the compilation time while editing
(this was significantly more useful in the earlier days of \LaTeX{})
or to generate a smaller document which is easier to navigate.
Another application of |\includeonly| is to generate
documents consisting of selected parts of the complete document.

However, there are a few drawbacks of the plain |\include| mechanism:
\begin{itemize}
\item
The child files cannot be compiled on their own,
they can only be compiled via the main file.
A naive editing environment
(such as a text editor with an option
to have the current file processed by \LaTeX)
may require one to switch to the main file before compiling;
attempting to compile the child file produces errors.
\item
The main file must be modified (each time)
to adjust the |\includeonly| command
to the present needs. This easily leaves the main file in a messy state.
\item
The generated document will always carry the filename
of the main document. This is inconvenient if
several child files are to be compiled and
to be kept for distribution.
\end{itemize}

The present package provides a simple interface
to make child files individually compilable by \LaTeX{}.
Compiling a child file then has the same effect as compiling
the main file with an |\includeonly| command
to select the appropriate child.
Moreover the generated document will carry the name of the child
rather than the main file.
This resolves all three above issues.

This feature is meant to make the editing of books,
thesis documents and lecture notes somewhat more convenient.
However, the package can also be used efficiently for
composing a series of documents (such as exercise sheets)
which are typically distributed individually.
It then assists the author in generating the individual documents
(potentially in different versions)
as well as a document containing the collected series.
Another application is in developing style files
or other kinds of included material
where compilation of the style file could redirect
to a sample or test file.

%%%%%%%%%%%%%%%%%%%%%%%%%%%%%%%%%%%%%%%%%%%%%%%%%%%%%%%%%%%%%%%%%%%%%%%%%%%%%%%%
%%%%%%%%%%%%%%%%%%%%%%%%%%%%%%%%%%%%%%%%%%%%%%%%%%%%%%%%%%%%%%%%%%%%%%%%%%%%%%%%
\section{Usage}

First of all, the package \textsf{childdoc} is \emph{not} a standard
\LaTeXe{} |.sty| style file! Therefore it needs to be invoked in
a non-standard way.

%%%%%%%%%%%%%%%%%%%%%%%%%%%%%%%%%%%%%%%%%%%%%%%%%%%%%%%%%%%%%%%%%%%%%%%%%%%%%%%%
\subsection{Included Files}
\label{sec:include}

%%%%%%%%%%%%%%%%%%%%%%%%%%%%%%%%%%%%%%%%
\DescribeMacro{\childdocmain}
To use the package, add the commands
\begin{center}
\begin{tabular}{l}
|\input{childdoc.def}|\\
|\childdocmain{}|\\
\end{tabular}
\end{center}
at the very top of the main \LaTeX{} file,
in particular \emph{before} the |\documentclass| statement!
The argument of |\childdocmain| should be left empty
(but it must be present).

%%%%%%%%%%%%%%%%%%%%%%%%%%%%%%%%%%%%%%%%
\DescribeMacro{\childdocof}
Furthermore, add the commands
\begin{center}
\begin{tabular}{l}
|\input{childdoc.def}|\\
|\childdocof{|\textit{main}|}|\\
\end{tabular}
\end{center}
at the top of every child file \textit{child}
which is included by |\include{|\textit{child}|}|
from within the main file
(or at least for those files to be compiled individually).
The argument \textit{main} must be the filename of the main file.

There are a couple of
considerations in setting up the main and child documents:

%%%%%%%%%%%%%%%%%%%%%%%%%%%%%%%%%%%%%%%%
\paragraph{Restrictions.}

Please note the following restrictions:
\begin{itemize}
\item
|\childdocmain| must be called with one argument \textit{main}
to ensure compatibility with earlier version of the package.
It must either be empty (|\childdocmain{}|)
or precisely match the filename of the main file in which it is specified.
See \secref{sec:detection} for further information.
\item
The filename \textit{main} must be specified without the |.tex| extension.
\item
The filename \textit{main} is case sensitive
(even in case-insensitive file systems)
due to internal string comparison.
\item
The argument \textit{main} should be fully expanded, it cannot be a macro.
\item
Subdirectories and special characters should be avoided in filenames.
\item
The command |\childdocmain{|\textit{main}|}| must be followed by a whitespace.
It should not be followed immediately by another command
or by a comment mark `|%|'.
This is because the \TeX{} parser reads the token immediately following
the argument of |\childdocmain| and puts it
at the beginning of every child section;
however, a white\-space is ignored.
\end{itemize}

%%%%%%%%%%%%%%%%%%%%%%%%%%%%%%%%%%%%%%%%
\paragraph{Content of Main File.}

It is advisable to place all content in the child files included by |\include|.
Any output contained in the main file will appear in all child documents
unless suppressed manually;
it cannot be suppressed automatically by the |\includeonly| directive
and thus should normally be avoided.
A method to include some content in the main file
by means of conditional processing is described in \secref{sec:conditional}.

%%%%%%%%%%%%%%%%%%%%%%%%%%%%%%%%%%%%%%%%
\paragraph{Page Numbering.}

When only a part of the document is compiled,
the appropriate numbering of pages
(as well as other status parameters)
is determined from the |.aux| files.
The latter contain information from previous passes.
However this information needs to propagate through
all intermediate child documents.
Therefore the page numbering in child documents may well
be inconsistent until the complete document is compiled at least once.

A useful (if unconventional) way to always ensure a consistent
page numbering is to restart the numbering in each child document
and denote the pages by `\textit{child}|.|\textit{page}'
where \textit{child} represents the chapter/section number of the child file.
This can be achieved by the command
|\numberwithin{page}{|\textit{child}|}|
of the \textsf{amsmath} package
where \textit{child} can be |chapter| or |section|
depending on the chosen structuring.
Alternatively, one can modify the macro |\thepage| appropriately
and reset the counter |page| at the start of each child file.

%%%%%%%%%%%%%%%%%%%%%%%%%%%%%%%%%%%%%%%%%%%%%%%%%%%%%%%%%%%%%%%%%%%%%%%%%%%%%%%%
\subsection{Conditional Processing}
\label{sec:conditional}

The package provides a mechanism to compile different versions
of a document. To customise the versions further some conditional processing
can come in handy to distinguish which version is being compiled.
The package provides two macros to describe the compilation context:

%%%%%%%%%%%%%%%%%%%%%%%%%%%%%%%%%%%%%%%%
\DescribeMacro{\ifchilddoc}
The conditional |\ifchilddoc| distinguishes between the compilation of
child documents and the main document:
%
\begin{center}
|\ifchilddoc |\textit{child-code}| |[|\||else |\textit{main-code}]| \||fi|
\end{center}

%%%%%%%%%%%%%%%%%%%%%%%%%%%%%%%%%%%%%%%%
\DescribeMacro{\childdocname}
\DescribeMacro{\childdocjob}
The macro |\childdocname| contains the filename (without extension)
of the main or child file being processed.
Note that |\childdocjob| will always contain the name of the main file.

%%%%%%%%%%%%%%%%%%%%%%%%%%%%%%%%%%%%%%%%
\paragraph{Title Page.}

Conditional processing can be used to include a title or banner page
in the main document when proper precautions are taken.
Importantly, the code in the main file should ensure that the page counter
(as well as other status parameters which are stored in the |.aux| files)
takes the same value after the conditional processing.
Otherwise the page numbers may take divergent values
depending on which part is compiled.

For example, a title page could be declared by:
%
\begin{center}
\begin{tabular}{l}
|\ifchilddoc\||else|\\
|\addtocounter{page}{-1}|\\
\textit{code for title page}\\
|\newpage|\\
|\||fi|
\end{tabular}
\end{center}
%
A banner page for the child documents can be generated by:
%
\begin{center}
\begin{tabular}{l}
|\ifchilddoc|\\
|\addtocounter{page}{-1}|\\
\textit{code for banner page}\\
|\newpage|\\
|\||fi|
\end{tabular}
\end{center}
%
Here one could write a message such as:
\begin{center}
|This is the part \childdocname{} of \childdocjob{}.|
\end{center}

%%%%%%%%%%%%%%%%%%%%%%%%%%%%%%%%%%%%%%%%%%%%%%%%%%%%%%%%%%%%%%%%%%%%%%%%%%%%%%%%
\subsection{Flags}
\label{sec:flags}

The package makes it easy to generate different versions
of the main or child documents.
To this end compilation flags can be defined
and assigned different default values.
They will be particularly useful in conjunction
with the forwarding mechanism described in \secref{sec:forward}.

For example, it may be useful to have a flag |\version|
which can be set to |draft| or |final|.
The document source will contain some conditional code
depending on the value of |\version|.
Suppose further, the flag should default to |final| for the main file
and to |draft| for child files
which is a natural assignment for editing the document.
This is achieved by placing the following code
in the preamble of the main document
(below the |\childdocmain| directive):
%
\begin{center}
\begin{tabular}{l}
|\ifchilddoc|\\
|\providecommand{\version}{draft}|\\
|\||else|\\
|\providecommand{\version}{final}|\\
|\||fi|
\end{tabular}
\end{center}
%
The definition by |\providecommand| makes sure
that previous definitions are not overwritten.
Further statements |\providecommand{\version}{...}|
can thus be added before the above code to override it.

For the main file, one might add a line
(between |\childdocmain| and the above block)
%
\begin{center}
|%\ifchilddoc\||else\providecommand{\version}{draft}\||fi|
\end{center}
%
which can be uncommented to produce a draft version.
Likewise one can add a line to the very top of a child file
(above the |\childdocof{|\textit{main}|}| directive)
%
\begin{center}
|%\providecommand{\version}{final}|
\end{center}
%
which can be uncommented to produce the final version of this child document.

%%%%%%%%%%%%%%%%%%%%%%%%%%%%%%%%%%%%%%%%%%%%%%%%%%%%%%%%%%%%%%%%%%%%%%%%%%%%%%%%
\subsection{Forwarding}
\label{sec:forward}

Different versions of the main or child documents
using compilation flags as described in \secref{sec:flags}
can be (permanently) stored in different files
for convenient compilation, viewing and distribution.
To this end, the package defines a command
to pass on compilation to a different file:

%%%%%%%%%%%%%%%%%%%%%%%%%%%%%%%%%%%%%%%%
\DescribeMacro{\childdocforward}
The command |\childdocforward| redirects processing to
another source file:
%
\begin{center}
\begin{tabular}{l}
|\input{childdoc.def}|\\
|\childdocforward[|\textit{main}|]{|\textit{dest}|}|\\
\end{tabular}
\end{center}
%
The argument \textit{dest} is the destination file
(without extension).
It should be the main file or one of the child files.
Note that further \textsf{childdoc} directives
such as |\childdocof| and |\childdocforward|
in the indicated file will be processed in this form.
The optional argument \textit{main}
passes on directly to the main file \textit{main}
while pretending to compile the child \textit{dest}.
This form behaves as if \textit{dest}
issues |\childdocof{|\textit{main}|}| right away,
and no further \textsf{childdoc} directives will be processed.

%%%%%%%%%%%%%%%%%%%%%%%%%%%%%%%%%%%%%%%%
\DescribeMacro{\...prefix}
In the alternative form |\childdocforwardprefix|,
%
\begin{center}
\begin{tabular}{l}
|\input{childdoc.def}|\\
|\childdocforwardprefix[|\textit{main}|]{|\textit{prefix}|}{|\textit{dest}|}|
\end{tabular}
\end{center}
%
the destination file is determined by a pattern
depending on the current file:
To make this work, the current file must be called
`{\textit{prefix}\hspace{0.2em}\textit{suffix}}'
with \textit{prefix} matching precisely the argument.
Processing is then passed on to the file
`{\textit{dest}\hspace{0.2em}\textit{suffix}}'.
Surely, the same effect is achieved by
directly specifying the
argument `{\textit{dest}\hspace{0.2em}\textit{suffix}}'
in the first form.
However, that requires to set up a different file
for each child. With the alternative form of the command
all these files can have exactly the same content
which simplifies setting them up and maintaining them.

For example, the following file |draft.tex|
with a compilation flag |\version| as described in \secref{sec:flags}
compiles the main document as a draft:
%
\begin{center}
\begin{tabular}{l}
|\def\version{draft}|\\
|\input{childdoc.def}|\\
|\childdocforward{|\textit{main}|}|
\end{tabular}
\end{center}
%
Likewise, the following files |final|\textit{nn}|.tex|
compile the final version of the child document
|child|\textit{nn}|.tex|:
%
\begin{center}
\begin{tabular}{l}
|\def\version{final}|\\
|\input{childdoc.def}|\\
|\childdocforwardprefix{final}{child}|
\end{tabular}
\end{center}
%

Note that when several versions of a main file and/or of each child file
are to be generated, it may be convenient to set up a |Makefile| or
shell script to automatise the process.

%%%%%%%%%%%%%%%%%%%%%%%%%%%%%%%%%%%%%%%%%%%%%%%%%%%%%%%%%%%%%%%%%%%%%%%%%%%%%%%%
\subsection{Command Line Processing}
\label{sec:commandline}

The effect of redirection files can also be achieved by invoking
the \LaTeX{} compiler with a more elaborate command line.
Most conveniently this should be done as part
of a shell script or a |Makefile|.

When using \textsf{childdoc} in the main file, the following
command lines effectively perform a redirection
(note that depending on the shell being used,
backslashes may have to be doubled: `|\|' $\to$ `|\\|'):
%
\begin{center}
|... -jobname "|\textit{target}|" |\\|"|[\textit{flags}]%
|\input{childdoc.def}\childdocforward[|\textit{main}|]{|\textit{dest}|}"|
\end{center}
%
Here \textit{target} is the name of the output file,
\textit{main} is the name of the main file
and \textit{dest} is the name of the main or child file to be processed
(all filenames without extensions).
The optional argument \textit{main} can be omitted
if \textit{main} matches \textit{dest}.
Optionally, compilation \textit{flags} can be defined via |\def| commands.
This command line makes the \TeX{} engine believe
it is compiling the file \textit{target}
whose content is specified as the latter parameter.
The provided code then forwards the processing to
\textit{main} or \textit{dest} as described in \secref{sec:forward}.

%%%%%%%%%%%%%%%%%%%%%%%%%%%%%%%%%%%%%%%%%%%%%%%%%%%%%%%%%%%%%%%%%%%%%%%%%%%%%%%%
\subsection{Include by Input}
\label{sec:input}

Including child documents by |\include| has some restrictions by design.
Most notably, the content of a child document always occupies
its own set of pages; pages cannot be shared between child documents.
Usually, this behaviour makes perfect sense
because each child document contain an essential part of the document.
However, in some situations it may be desirable to compose
a document from a collection of parts
without having mandatory page breaks between then.
For this case, the package
provides a mechanism to include parts
by |\input| which can also be processed individually.
However, by construction this mechanism
requires manual handling of the content to be output.

%%%%%%%%%%%%%%%%%%%%%%%%%%%%%%%%%%%%%%%%
\DescribeMacro{\ifchilddocmanual}
The main file should be prepared as usual, see \secref{sec:include}.
However, the document body must make a distinction
between processing of an individual part and of the main document, e.g.:
%
\begin{center}
\begin{tabular}{l}
|\ifchilddocmanual|\\
|\input{\childdocname}|\\
|\||else|\\
\textit{document body with }|\input{|\textit{part}|}|\\
|\||fi|
\end{tabular}
\end{center}
%
The conditional |\ifchilddocmanual| is true whenever
a part to be included by |\input| is being compiled,
and the name of the part is stored in |\childdocname|.

%%%%%%%%%%%%%%%%%%%%%%%%%%%%%%%%%%%%%%%%
\DescribeMacro{\childdocby}
Each part to be included by |\input| should start with:
%
\begin{center}
\begin{tabular}{l}
|\input{childdoc.def}|\\
|\childdocby{|\textit{main}|}|\\
\end{tabular}
\end{center}
%
The directive |\childdocby| is similar to |\childdocof|
described in \secref{sec:include},
but the subsequent selection of content must be done manually.
To that end, both |\ifchilddoc| and |\ifchilddocmanual|
will be true upon processing of a part,
and the name of the part is stored in |\childdocname|.
Note that |\jobname| will be set to the filename of the current part
so that each part receives an individual |.aux| file
that does not interfere with the |.aux| file(s) of the main document.
This behaviour can be altered by the alternative form
|\childdocby[*]{|\textit{main}|}| (with a non-empty optional argument)
which uses the |.aux| file of the main document
by setting |\jobname| to \textit{main}.

%%%%%%%%%%%%%%%%%%%%%%%%%%%%%%%%%%%%%%%%%%%%%%%%%%%%%%%%%%%%%%%%%%%%%%%%%%%%%%%%
\subsection{Driver Development}
\label{sec:driver}

The \textsf{childdoc} mechanism can also be use for the development
of definition files such as \LaTeX{} styles or classes.
This case differs from the above setup with multiple parts
included by |\include| in that no |\includeonly| should be invoked.
This can be achieved by starting the include file
(before |\ProvidesPackage|) with:
%
\begin{center}
\begin{tabular}{l}
|\input{childdoc.def}|\\
|\childdocforward{|\textit{main}|}|\\
\end{tabular}
\end{center}
%
or alternatively with:
%
\begin{center}
\begin{tabular}{l}
|\input{childdoc.def}|\\
|\childdocby{|\textit{main}|}|\\
\end{tabular}
\end{center}
%
Both forms have slightly different effects as described above.
The main file is prepared as usual, see \secref{sec:include}.

%%%%%%%%%%%%%%%%%%%%%%%%%%%%%%%%%%%%%%%%%%%%%%%%%%%%%%%%%%%%%%%%%%%%%%%%%%%%%%%%
\subsection{Legacy Detection}
\label{sec:detection}

The directive |\childdocmain| in the main file can detect
whether the complete document or merely a child is to be compiled
even without using the directive |\childdocof|.
This method is deprecated because it is less robust
and there is no compelling reason to use it;
it is merely provided for backward compatibility
and it may be removed in future versions.

If the detection mechanism is to be used,
it is mandatory to correctly specify
the filename of the main file as the argument of |\childdocmain|:
%
\begin{center}
\begin{tabular}{l}
|\input{childdoc.def}|\\
|\childdocmain{|\textit{main}|}|\\
\end{tabular}
\end{center}
%
If |\jobname| does not match the argument \textit{main} of |\childdocmain|,
it is assumed that |\jobname| points to the child file to be compiled.
When using |\childdocmain| with the main file specified as argument,
it suffices to start a child file
with just |\input{|\textit{main}|}|
without loading of the package and using |\childdocof|.
If instead all processing is done
with the appropriate \textsf{childdoc} directives,
the argument of \textit{main} of |\childdocmain| can be empty.

An alternative version of the command line processing described
in \secref{sec:commandline} using the detection mechanism reads:
%
\begin{center}
|... -jobname "|\textit{target}|" "|[\textit{flags}]%
[|\def\jobname{|\textit{dest}|}|]|\input{|\textit{main}|}"|
\end{center}

%%%%%%%%%%%%%%%%%%%%%%%%%%%%%%%%%%%%%%%%%%%%%%%%%%%%%%%%%%%%%%%%%%%%%%%%%%%%%%%%
\subsection{Manual Code}
\label{sec:manual}

In case one cannot be certain whether the definitions file |childdoc.def|
is installed on the target \TeX{} distribution
and one prefers not to ship it,
it is conceivable to paste a few relevant commands into the sources.

To that end, drop all statements |\input{childdoc.def}|
and perform the replacements as outlined below.
Instead of |\childdocmain{|\textit{main}|}| add the following code
to the top of the main file:
%
\begin{center}
\begin{tabular}{l}
|\||ifdefined\childdocname\endinput\||fi\newif\ifchilddoc|\\
|\edef\childdocname{\scantokens\expandafter{\jobname\noexpand}}|\\
|\def\childdocmain{|\textit{main}|}\||ifx\childdocmain\childdocname\||else|\\
|\childdoctrue\includeonly{\childdocname}\let\jobname\childdocmain\||fi|\\
\end{tabular}
\end{center}
%
Instead of |\childdocof{|\textit{main}|}| just include the main file
at the top of each child file:
%
\begin{center}
|\input{|\textit{main}|}|
\end{center}
%
A simple redirection |\childdocforward{|\textit{dest}|}| is achieved by:
%
\begin{center}
|\def\jobname{|\textit{dest}|}\input{\jobname}|
\end{center}
%
The redirection with prefix
|\childdocforwardprefix[|\textit{prefix}|]{|\textit{dest}|}|
is accomplished by:
%
\begin{center}
\begin{tabular}{l}
|{\edef\jobname{\scantokens\expandafter{\jobname\noexpand}}|\\
|\def\redirectjob |\textit{prefix}|#1~~~{\gdef\jobname{|\textit{dest}|#1}}|\\
|\expandafter\redirectjob\jobname~~~}\input{\jobname}|
\end{tabular}
\end{center}

In an alternative approach,
child documents can be compiled by a specific command line
without additional code or specific definitions:
%
\begin{center}
|... -jobname "|\textit{target}|" "|[\textit{flags}]%
|\includeonly{|\textit{dest}|}\input{|\textit{main}|}"|
\end{center}
%

%%%%%%%%%%%%%%%%%%%%%%%%%%%%%%%%%%%%%%%%%%%%%%%%%%%%%%%%%%%%%%%%%%%%%%%%%%%%%%%%
%%%%%%%%%%%%%%%%%%%%%%%%%%%%%%%%%%%%%%%%%%%%%%%%%%%%%%%%%%%%%%%%%%%%%%%%%%%%%%%%
\section{Information}

%%%%%%%%%%%%%%%%%%%%%%%%%%%%%%%%%%%%%%%%%%%%%%%%%%%%%%%%%%%%%%%%%%%%%%%%%%%%%%%%
\subsection{Copyright}

Copyright \copyright{} 2017--2018 Niklas Beisert

This work may be distributed and/or modified under the
conditions of the \LaTeX{} Project Public License, either version 1.3
of this license or (at your option) any later version.
The latest version of this license is in
  \url{http://www.latex-project.org/lppl.txt}
and version 1.3 or later is part of all distributions of \LaTeX{}
version 2005/12/01 or later.

This work has the LPPL maintenance status `maintained'.

The Current Maintainer of this work is Niklas Beisert.

This work consists of the files |README.txt|, |childdoc.ins| and |childdoc.dtx|
as well as the derived files |childdoc.def|, |cdocsamp.tex|
with |cdocsch1.tex|, |cdocsch2.tex|, |cdocspt3.tex|, |cdocspt4.tex|,
|cdocsdrf.tex|, |cdocsfn1.tex|, |cdocsfn2.tex|
as well as |childdoc.pdf|.

%%%%%%%%%%%%%%%%%%%%%%%%%%%%%%%%%%%%%%%%%%%%%%%%%%%%%%%%%%%%%%%%%%%%%%%%%%%%%%%%
\subsection{Files and Installation}

The package consists of the files:
%
\begin{center}
\begin{tabular}{ll}
    |README.txt|   & readme file \\
    |childdoc.ins| & installation file \\
    |childdoc.dtx| & source file \\
    |childdoc.def| & definition file \\
    |cdocsamp.tex| & sample main file \\
    |cdocsch1.tex| & sample include file \\
    |cdocsch2.tex| & sample include file \\
    |cdocspt3.tex| & sample part file \\
    |cdocspt4.tex| & sample part file \\
    |cdocsdrf.tex| & sample redirection file \\
    |cdocsfn1.tex| & sample redirection file \\
    |cdocsfn2.tex| & sample redirection file \\
    |childdoc.pdf| & manual
\end{tabular}
\end{center}
%
The distribution consists of the files
|README.txt|, |childdoc.ins| and |childdoc.dtx|.
%
\begin{itemize}
\item
Run (pdf)\LaTeX{} on |childdoc.dtx|
to compile the manual |childdoc.pdf| (this file).
\item
Run \LaTeX{} on |childdoc.ins| to create the definitions file |childdoc.def|
and the sample |cdocsamp.tex| with include files
|cdocsch1.tex|, |cdocsch2.tex|, |cdocspt3.tex|, |cdocspt4.tex|,
|cdocsdrf.tex|, |cdocsfn1.tex|, |cdocsfn2.tex|.
Then copy the file |childdoc.def| to an appropriate directory of your \LaTeX{}
distribution, e.g.\ \textit{texmf-root}|/tex/latex/childdoc|.
\end{itemize}

%%%%%%%%%%%%%%%%%%%%%%%%%%%%%%%%%%%%%%%%%%%%%%%%%%%%%%%%%%%%%%%%%%%%%%%%%%%%%%%%
\subsection{Related CTAN Packages}

There are several other packages which offer a similar functionality:
%
\begin{itemize}
\item
The packages
\href{http://ctan.org/pkg/docmute}{\textsf{docmute}},
\href{http://ctan.org/pkg/includex}{\textsf{includex}} and
\href{http://ctan.org/pkg/standalone}{\textsf{standalone}}
provide commands to include only the document body of
a child file thus allowing both files to be compiled individually.
\item
The packages \href{http://ctan.org/pkg/subdocs}{\textsf{subdocs}}
and \href{http://ctan.org/pkg/subfiles}{\textsf{subfiles}}
provide structures in which the main and child documents can be
encapsulated and allowing them to be compiled individually.
The inclusion mechanism is different from the conventional |\include|.
\item
The package \href{http://ctan.org/pkg/combine}{\textsf{combine}}
is an elaborate solution to combine several documents into one.
\end{itemize}
%
See also the CTAN topic \href{http://ctan.org/topic/subdocs}{\textsf{subdocs}}
for further related packages.
The present package differs from the above solutions in that
a document structure constructed with the conventional |\include| mechanism
just needs two extra commands at the top of every file
such that all constituent files can be compiled individually.

%%%%%%%%%%%%%%%%%%%%%%%%%%%%%%%%%%%%%%%%%%%%%%%%%%%%%%%%%%%%%%%%%%%%%%%%%%%%%%%%
%\subsection{Feature Suggestions}
%
%The following is a list of features which may be useful for future
%versions of this package:
%%
%\begin{itemize}
%\item
%\ldots
%\end{itemize}

%%%%%%%%%%%%%%%%%%%%%%%%%%%%%%%%%%%%%%%%%%%%%%%%%%%%%%%%%%%%%%%%%%%%%%%%%%%%%%%%
\subsection{Revision History}

%%%%%%%%%%%%%%%%%%%%%%%%%%%%%%%%%%%%%%%%
\paragraph{v2.0:} 2018/12/30

\begin{itemize}
\item
immediate forward processing
\item
added |\childdocby| mechanism
\item
manual restructured
\end{itemize}

%%%%%%%%%%%%%%%%%%%%%%%%%%%%%%%%%%%%%%%%
\paragraph{v1.6:} 2018/01/17

\begin{itemize}
\item
application for development of include files
\item
corrections to manual
\end{itemize}

%%%%%%%%%%%%%%%%%%%%%%%%%%%%%%%%%%%%%%%%
\paragraph{v1.5:} 2017/05/21

\begin{itemize}
\item
more complete structuring introduced
\item
|\childdocof| introduced
\item
|\childdoc| renamed to |\childdocmain|
\item
|\childredirect| renamed to |\childdocforward| and |\childdocforwardprefix|
and functionality expanded
\end{itemize}

%%%%%%%%%%%%%%%%%%%%%%%%%%%%%%%%%%%%%%%%
\paragraph{v1.0:} 2017/04/27

\begin{itemize}
\item
manual and install package
\item
first version published on CTAN
\end{itemize}

%%%%%%%%%%%%%%%%%%%%%%%%%%%%%%%%%%%%%%%%
\paragraph{v0.6:} 2017/04/26

\begin{itemize}
\item
redirection mechanism added
\end{itemize}

%%%%%%%%%%%%%%%%%%%%%%%%%%%%%%%%%%%%%%%%
\paragraph{v0.5:} 2017/04/26

\begin{itemize}
\item
functionality in definition file
\end{itemize}


%%%%%%%%%%%%%%%%%%%%%%%%%%%%%%%%%%%%%%%%%%%%%%%%%%%%%%%%%%%%%%%%%%%%%%%%%%%%%%%%
%%%%%%%%%%%%%%%%%%%%%%%%%%%%%%%%%%%%%%%%%%%%%%%%%%%%%%%%%%%%%%%%%%%%%%%%%%%%%%%%
%%%%%%%%%%%%%%%%%%%%%%%%%%%%%%%%%%%%%%%%%%%%%%%%%%%%%%%%%%%%%%%%%%%%%%%%%%%%%%%%
\appendix

\settowidth\MacroIndent{\rmfamily\scriptsize 000\ }

 \DocInput{childdoc.dtx}

\end{document}
%</driver>
% \fi
%
% %%%%%%%%%%%%%%%%%%%%%%%%%%%%%%%%%%%%%%%%%%%%%%%%%%%%%%%%%%%%%%%%%%%%%%%%%%%%%%
% %%%%%%%%%%%%%%%%%%%%%%%%%%%%%%%%%%%%%%%%%%%%%%%%%%%%%%%%%%%%%%%%%%%%%%%%%%%%%%
% \section{Sample}
%\iffalse
%<*samplemain>
%\fi
%
% The following presents a sample document
% with two chapters, two parts, a title page,
% a compile flag as well as three forwarding files to set the flag.
% It consists of eight |.tex| files:
% \begin{center}
% \begin{tabular}{ll}
% |cdocsamp.tex|&main file\\
% |cdocsch1.tex|&include file for chapter 1\\
% |cdocsch2.tex|&include file for chapter 2\\
% |cdocspt3.tex|&include file for part 3\\
% |cdocspt4.tex|&include file for part 4\\
% |cdocsdrf.tex|&forwarding file for main file in draft mode\\
% |cdocsfi1.tex|&forwarding file for final version of chapter 1\\
% |cdocsfi2.tex|&forwarding file for final version of chapter 2\\
% \end{tabular}
% \end{center}
% Each of the eight files can be compiled directly by the \LaTeX{} compiler.
%
% %%%%%%%%%%%%%%%%%%%%%%%%%%%%%%%%%%%%%%
% \paragraph{Main File.}
%
% The main file is called |cdocsamp.tex|.
%
% Load the \textsf{childdoc} definitions and
% declare the filename for the main document:
%    \begin{macrocode}
\input{childdoc.def}
\childdocmain{}
%    \end{macrocode}

% Optional override for |\version| flag:
%    \begin{macrocode}
%%\ifchilddoc\else\providecommand{\version}{draft}\fi
%    \end{macrocode}

% Define the default values for the |\version| flag
% (|final| for the main file and |draft| for childs):
%    \begin{macrocode}
\ifchilddoc
\providecommand{\version}{draft}
\else
\providecommand{\version}{final}
\fi
%    \end{macrocode}

% Load the standard document class:
%    \begin{macrocode}
\documentclass[12pt]{article}
%    \end{macrocode}

% Start the document body:
%    \begin{macrocode}
\begin{document}
%    \end{macrocode}

% Declare a title page.
% Print title, part of document being processed and version flag:
%    \begin{macrocode}
\addtocounter{page}{-1}
\begin{center}
{\LARGE\bfseries{}childdoc example\par}
\vspace{1cm}
\ifchilddoc
\ifchilddocmanual part\else chapter\fi:
`\childdocname' of `\childdocjob'\par
\else
main document: `\childdocjob'\par
\fi
version: \version\par
\end{center}
\newpage
%    \end{macrocode}

% Manually include selected file,
% otherwise process as usual:
%    \begin{macrocode}
\ifchilddocmanual
\section*{part `\childdocname'}
\input{\childdocname}
\else
%    \end{macrocode}

% Include the two chapters:
%    \begin{macrocode}
\include{cdocsch1}
\include{cdocsch2}
%    \end{macrocode}

% Include the two parts unless only chapters should be displayed:
%    \begin{macrocode}
\ifchilddoc\else
\section{part three}
\input{cdocspt3}
\section{part four}
\input{cdocspt4}
\fi
%    \end{macrocode}

% Process as usual until here:
%    \begin{macrocode}
\fi
%    \end{macrocode}

% End of document body:
%    \begin{macrocode}
\end{document}
%    \end{macrocode}
%\iffalse
%</samplemain>
%\fi
%
% %%%%%%%%%%%%%%%%%%%%%%%%%%%%%%%%%%%%%%
% \paragraph{Chapter Include Files.}
%
% The include files are called |cdocsch1.tex| and |cdocsch2.tex|.
%
%\iffalse
%<*samplechap1|samplechap2>
%\fi

% Optional override for |\version| flag:
%    \begin{macrocode}
%%\providecommand{\version}{final}
%    \end{macrocode}

% Include the main document:
%    \begin{macrocode}
\input{childdoc.def}
\childdocof{cdocsamp}
%    \end{macrocode}

%\iffalse
%</samplechap1|samplechap2>
%\fi
%
%\iffalse
%<*samplechap1>
%\fi
% Some text for chapter 1:
%    \begin{macrocode}
\section{one}
some text in chapter one
%    \end{macrocode}

%\iffalse
%</samplechap1>
%\fi
% Some text for chapter 2:
%\iffalse
%<*samplechap2>
%\fi
%    \begin{macrocode}
\section{two}
more text in chapter two
%    \end{macrocode}

%\iffalse
%</samplechap2>
%\fi
%
% %%%%%%%%%%%%%%%%%%%%%%%%%%%%%%%%%%%%%%
% \paragraph{Part Include Files.}
%
% The include files are called |cdocspt3.tex| and |cdocspt4.tex|.
%
%\iffalse
%<*samplepart3|samplepart4>
%\fi

% Optional override for |\version| flag:
%    \begin{macrocode}
%%\providecommand{\version}{final}
%    \end{macrocode}

% Include the main document:
%    \begin{macrocode}
\input{childdoc.def}
\childdocby{cdocsamp}
%    \end{macrocode}

%\iffalse
%</samplepart3|samplepart4>
%\fi
%
%\iffalse
%<*samplepart3>
%\fi
% Some text for part 3:
%    \begin{macrocode}
some text in part three
%    \end{macrocode}

%\iffalse
%</samplepart3>
%\fi
% Some text for part 4:
%\iffalse
%<*samplepart4>
%\fi
%    \begin{macrocode}
more text in part four
%    \end{macrocode}

%\iffalse
%</samplepart4>
%\fi
%
% %%%%%%%%%%%%%%%%%%%%%%%%%%%%%%%%%%%%%%
% \paragraph{Forwarding for a Complete Draft.}
%
% The following forwarding file |cdocsdrf.tex|
% compiles the main document in draft mode:
%\iffalse
%<*sampledraft>
%\fi
%    \begin{macrocode}
\def\version{draft}
\input{childdoc.def}
\childdocforward{cdocsamp}
%    \end{macrocode}

%\iffalse
%</sampledraft>
%\fi
%
% %%%%%%%%%%%%%%%%%%%%%%%%%%%%%%%%%%%%%%
% \paragraph{Forwarding for Final Version of the Chapters.}
%
% The following forwarding files |cdocsfn1.tex| and |cdocsfn2.tex|
% (with identical content)
% compile the final versions of the child documents
% |cdocsch1.tex| and |cdocsch2.tex|, respectively:
%\iffalse
%<*samplefinal>
%\fi
%    \begin{macrocode}
\def\version{final}
\input{childdoc.def}
\childdocforwardprefix[cdocsamp]{cdocsfn}{cdocsch}
%    \end{macrocode}

%\iffalse
%</samplefinal>
%\fi
%
% %%%%%%%%%%%%%%%%%%%%%%%%%%%%%%%%%%%%%%
% \paragraph{Command Line Processing.}
%
% The following three command lines generate the output files
% |cdocscld|, |cdocscl1| and |cdocscl2|
% which should be identical to
% |cdocsdrf|, |cdocsch1| and |cdocsfn2|, respectively:
% \begin{center}
% \begin{tabular}{l}
% |latex -jobname cdocscld \|\\
% |  "\def\version{draft}\input{childdoc.def}\childdocforward{cdocsamp}"|\\
% |latex -jobname cdocscl1 \|\\
% |  "\input{childdoc.def}\childdocforward[cdocsamp]{cdocsch1}"|\\
% |latex -jobname cdocscl2 \|\\
% |  "\def\version{final}\input{childdoc.def}\childdocforward{cdocsch2}"|
% \end{tabular}
% \end{center}
% Note that the trailing backslash on each first line
% merely continues the input to the second line
% (for convenient cut ant paste).
% Furthermore, the command |latex| can be replaced by any
% of its alternative versions such as |pdflatex|.
%
% %%%%%%%%%%%%%%%%%%%%%%%%%%%%%%%%%%%%%%%%%%%%%%%%%%%%%%%%%%%%%%%%%%%%%%%%%%%%%%
% %%%%%%%%%%%%%%%%%%%%%%%%%%%%%%%%%%%%%%%%%%%%%%%%%%%%%%%%%%%%%%%%%%%%%%%%%%%%%%
% \section{Implementation}
%\iffalse
%<*package>
%\fi
%
% This section describes the definitions file |childdoc.def|.

% The definitions cannot be loaded using |\usepackage| or |\RequirePackage|
% which has a mechanism to prevent loading a style file more than once.
% When loading the definitions by means of |\input|
% multiple instances have to be prevented manually:
%\iffalse
%This code needs to be before the `\ProvidesFile' directive
%which is defined at the beginning of this file.
%Therefore it is also placed there and commented out here.
%</package>
%<*discard>
%\fi
%    \begin{macrocode}
\ifdefined\childdocmain\endinput\fi
%    \end{macrocode}
%\iffalse
%</discard>
%<*package>
%\fi
%
% \macro{\ifchilddoc}
% \macro{\ifchilddocmanual}
% The conditional |\ifchilddoc| tells whether a
% child (true) or main (false) document is being compiled.
% The conditional |\ifchilddocmanual| tells whether
% the |\includeonly| mechanism is used (false) or
% the selection of child files must be performed manually (true).
% The definitions initialise to false:
%    \begin{macrocode}
\newif\ifchilddoc
\newif\ifchilddocmanual
%    \end{macrocode}

% \macro{\childdocname}
% \macro{\childdocjob}
% The macro |\childdocname| stores the name of the main document
% to be compiled. The macro |\childdocjob| stores the name of
% the document on which the \LaTeX{} compiler was originally invoked.
% The content of |\jobname| cannot be compared
% to filenames specified in the source due to different catcodes.
% The following code rescans |\jobname|, stores the result
% in |\childdocname| and saves a copy in |\childdocjob|:
%    \begin{macrocode}
\edef\childdocname{\scantokens\expandafter{\jobname\noexpand}}
\let\childdocjob\childdocname
%    \end{macrocode}

% \macro{\childdocdisable}
% The macro |\childdocdisable| prevents the main file
% from being processed more than once.
% At this stage, the main document command |\childdocmain|
% is assumed to be called once again where it should do nothing.
% Any subsequent call to it should prevent
% a secondary processing of the main document
% It overwrites the forwarding commands
% |\childdocof| and |\childdocforward|
% with empty macros to prevent further inclusions of the main document:
%    \begin{macrocode}
\newcommand{\childdocdisable}
{
  \renewcommand{\childdocmain}[1]{\renewcommand{\childdocmain}[1]{\endinput}}
  \renewcommand{\childdocof}[1]{}
  \renewcommand{\childdocby}[2][]{}
  \renewcommand{\childdocforward}[2][]{}
  \renewcommand{\childdocdisable}{}
}
%    \end{macrocode}

% \macro{\childdocmain}
% The macro |\childdocmain| is to be called at the top of the main file
% with nothing or the main filename (without extension) as argument.
% First, it breaks loops.
% If the argument is not empty and does not match |\childdocname|
% (which is set by the first inclusion of |childdoc.def|),
% |\ifchilddoc| is set to true, |\includeonly| is applied to the child file
% and |\jobname| is set to the main file
% (for proper handling of |.aux| files):
%    \begin{macrocode}
\newcommand{\childdocmain}[1]
{
  \childdocdisable\childdocmain{}
  \if?#1?\else
    \begingroup
      \def\childdoctmp{#1}
      \ifx\childdoctmp\childdocname
        \def\childdoctmp{}
      \else
        \def\childdoctmp
        {
          \childdoctrue
          \includeonly{\childdocname}
          \def\childdocjob{#1}
          \def\jobname{#1}
        }
      \fi
      \expandafter
    \endgroup
    \childdoctmp
  \fi
}
%    \end{macrocode}

% \macro{\childdocof}
% The command |\childdocof| redirects
% compilation to the main file |#1|.
%    \begin{macrocode}
\newcommand{\childdocof}[1]
{
  \childdocdisable
  \childdoctrue
  \includeonly{\childdocname}
  \def\jobname{#1}
  \def\childdocjob{#1}
  \input{#1}
}
%    \end{macrocode}

% \macro{\childdocby}
% The command |\childdocby| ....
%    \begin{macrocode}
\newcommand{\childdocby}[2][]
{
  \childdocdisable
  \childdoctrue
  \childdocmanualtrue
  \if?#1?\else
    \def\jobname{#2}
  \fi
  \def\childdocjob{#2}
  \input{#2}
  \endinput
}
%    \end{macrocode}

% \macro{\childdocforward}
% The command |\childdocforward| redirects
% compilation to the main file or
% (if the optional argument is given) a child file.
% Parameters are set as if the main file
% or a child file starting with |\childdocof| was compiled.
% Then compilation is handed over to the main file:
%    \begin{macrocode}
\newcommand{\childdocforward}[2][]
{
  \begingroup
    \if?#1?
      \def\childdoctmp
      {
        \def\childdocname{#2}
        \def\childdocjob{#2}
        \def\jobname{#2}
        \input{#2}
        \endinput
      }
    \else
      \def\childdoctmp
      {
        \childdocdisable
        \def\childdocname{#2}
        \childdoctrue
        \includeonly{#2}
        \def\childdocjob{#1}
        \def\jobname{#1}
        \input{#1}
        \endinput
      }
    \fi
    \expandafter
  \endgroup
  \childdoctmp
}
%    \end{macrocode}

% \macro{\childdocforwardprefix}
% The command |\childdocforwardprefix| redirects
% compilation to the main or a child file by means of a pattern.
% The prefix |#1| in the current filename is replaced by |#2|
% and the suffix of the current filename is kept
% (it is assumed that the filename does not contain the substring `|~~~|'
% which is used as a delimiter).
% Compilation is handed over to the new file by |\childdocforward|:
%    \begin{macrocode}
\newcommand{\childdocforwardprefix}[3][]
{
  \begingroup
    \def\childdocextract #2##1~~~{\def\childdoctmp{\childdocforward[#1]{#3##1}}}
    \expandafter\childdocextract\childdocname~~~
    \expandafter
  \endgroup
  \childdoctmp
}
%    \end{macrocode}

% \macro{\childdoc}
% The deprecated macro |\childdoc| is a legacy version of |\childdocmain|:
%    \begin{macrocode}
\newcommand{\childdoc}{\childdocmain}
%    \end{macrocode}

% \macro{\childdocredirect}
% The deprecated macro |\childdocredirect| is a legacy version
% of |\childdocforward| and |\childdocforwardprefix|:
%    \begin{macrocode}
\newcommand{\childdocredirect}[2][]
{
  \begingroup
    \if?#1?
      \def\childdoctmp{\childdocforward{#2}}
    \else
      \def\childdoctmp{\childdocforwardprefix{#1}{#2}}
    \fi
    \expandafter
  \endgroup
  \childdoctmp
}
%    \end{macrocode}

%\iffalse
%</package>
%\fi
%
\endinput
|\\
|\childdocby{|\textit{main}|}|\\
\end{tabular}
\end{center}
%
Both forms have slightly different effects as described above.
The main file is prepared as usual, see \secref{sec:include}.

%%%%%%%%%%%%%%%%%%%%%%%%%%%%%%%%%%%%%%%%%%%%%%%%%%%%%%%%%%%%%%%%%%%%%%%%%%%%%%%%
\subsection{Legacy Detection}
\label{sec:detection}

The directive |\childdocmain| in the main file can detect
whether the complete document or merely a child is to be compiled
even without using the directive |\childdocof|.
This method is deprecated because it is less robust
and there is no compelling reason to use it;
it is merely provided for backward compatibility
and it may be removed in future versions.

If the detection mechanism is to be used,
it is mandatory to correctly specify
the filename of the main file as the argument of |\childdocmain|:
%
\begin{center}
\begin{tabular}{l}
|% \iffalse
%
% childdoc.dtx Copyright (C) 2017-2018 Niklas Beisert
%
% This work may be distributed and/or modified under the
% conditions of the LaTeX Project Public License, either version 1.3
% of this license or (at your option) any later version.
% The latest version of this license is in
%   http://www.latex-project.org/lppl.txt
% and version 1.3 or later is part of all distributions of LaTeX
% version 2005/12/01 or later.
%
% This work has the LPPL maintenance status `maintained'.
%
% The Current Maintainer of this work is Niklas Beisert.
%
% This work consists of the files childdoc.dtx and childdoc.ins
% and the derived files childdoc.def and cdocsamp.tex with
% cdocsch1.tex, cdocsch2.tex, cdocsdrf.tex, cdocsfn1.tex, cdocsfn2.tex.
%
%<package>\ifdefined\childdocmain\endinput\fi
%<package>\ProvidesFile{childdoc.def}[2018/12/30 v2.0 child document driver]
%<samplemain>\ProvidesFile{cdocsamp.tex}[2018/12/30 v2.0 sample for childdoc]
%<*driver>
%\ProvidesFile{childdoc.drv}[2018/12/30 v2.0 childdoc reference manual file]
\PassOptionsToClass{10pt,a4paper}{article}
\documentclass{ltxdoc}

\usepackage[margin=35mm]{geometry}
\usepackage{hyperref}
\usepackage{hyperxmp}
\usepackage[usenames]{color}

\hypersetup{colorlinks=true}
\hypersetup{pdfstartview=FitH}
\hypersetup{pdfpagemode=UseNone}
\hypersetup{pdfsource={}}
\hypersetup{pdflang={en-UK}}
\hypersetup{pdfcopyright={Copyright 2017-2018 Niklas Beisert.
  This work may be distributed and/or modified under the
  conditions of the LaTeX Project Public License, either version 1.3
  of this license or (at your option) any later version.}}
\hypersetup{pdflicenseurl={http://www.latex-project.org/lppl.txt}}
\hypersetup{pdfcontactaddress={ETH Zurich, ITP, HIT K,
  Wolfgang-Pauli-Strasse 27}}
\hypersetup{pdfcontactpostcode={8093}}
\hypersetup{pdfcontactcity={Zurich}}
\hypersetup{pdfcontactcountry={Switzerland}}
\hypersetup{pdfcontactemail={nbeisert@itp.phys.ethz.ch}}
\hypersetup{pdfcontacturl={http://people.phys.ethz.ch/\xmptilde nbeisert/}}

\newcommand{\secref}[1]{\hyperref[#1]{section \ref*{#1}}}

\parskip1ex
\parindent0pt
\let\olditemize\itemize
\def\itemize{\olditemize\parskip0pt}

\begin{document}

\title{The \textsf{childdoc} Package}
\hypersetup{pdftitle={The childdoc Package}}
\author{Niklas Beisert\\[2ex]
  Institut f\"ur Theoretische Physik\\
  Eidgen\"ossische Technische Hochschule Z\"urich\\
  Wolfgang-Pauli-Strasse 27, 8093 Z\"urich, Switzerland\\[1ex]
  \href{mailto:nbeisert@itp.phys.ethz.ch}
  {\texttt{nbeisert@itp.phys.ethz.ch}}}
\hypersetup{pdfauthor={Niklas Beisert}}
\hypersetup{pdfsubject={Manual for the LaTeX2e Package childdoc}}
\date{30 December 2018, \textsf{v2.0}}
\maketitle

\begin{abstract}\noindent
\textsf{childdoc} is a \LaTeXe{} package
that enables the direct compilation
of document sections included by |\include|
to individual files.
\end{abstract}

\begingroup
\parskip0ex
\tableofcontents
\endgroup

%%%%%%%%%%%%%%%%%%%%%%%%%%%%%%%%%%%%%%%%%%%%%%%%%%%%%%%%%%%%%%%%%%%%%%%%%%%%%%%%
%%%%%%%%%%%%%%%%%%%%%%%%%%%%%%%%%%%%%%%%%%%%%%%%%%%%%%%%%%%%%%%%%%%%%%%%%%%%%%%%
\section{Introduction}

\LaTeX{} provides a mechanism to structure a large document (such as a book)
into a main file and several child files (containing the chapters)
using the |\include| command.
This mechanism is beneficial for documents
which span hundreds of pages in order to
make the source file(s) more manageable.
Moreover, compilation can be restricted to
selected child files by means of the |\includeonly| command.
The latter feature can be used to reduce the compilation time while editing
(this was significantly more useful in the earlier days of \LaTeX{})
or to generate a smaller document which is easier to navigate.
Another application of |\includeonly| is to generate
documents consisting of selected parts of the complete document.

However, there are a few drawbacks of the plain |\include| mechanism:
\begin{itemize}
\item
The child files cannot be compiled on their own,
they can only be compiled via the main file.
A naive editing environment
(such as a text editor with an option
to have the current file processed by \LaTeX)
may require one to switch to the main file before compiling;
attempting to compile the child file produces errors.
\item
The main file must be modified (each time)
to adjust the |\includeonly| command
to the present needs. This easily leaves the main file in a messy state.
\item
The generated document will always carry the filename
of the main document. This is inconvenient if
several child files are to be compiled and
to be kept for distribution.
\end{itemize}

The present package provides a simple interface
to make child files individually compilable by \LaTeX{}.
Compiling a child file then has the same effect as compiling
the main file with an |\includeonly| command
to select the appropriate child.
Moreover the generated document will carry the name of the child
rather than the main file.
This resolves all three above issues.

This feature is meant to make the editing of books,
thesis documents and lecture notes somewhat more convenient.
However, the package can also be used efficiently for
composing a series of documents (such as exercise sheets)
which are typically distributed individually.
It then assists the author in generating the individual documents
(potentially in different versions)
as well as a document containing the collected series.
Another application is in developing style files
or other kinds of included material
where compilation of the style file could redirect
to a sample or test file.

%%%%%%%%%%%%%%%%%%%%%%%%%%%%%%%%%%%%%%%%%%%%%%%%%%%%%%%%%%%%%%%%%%%%%%%%%%%%%%%%
%%%%%%%%%%%%%%%%%%%%%%%%%%%%%%%%%%%%%%%%%%%%%%%%%%%%%%%%%%%%%%%%%%%%%%%%%%%%%%%%
\section{Usage}

First of all, the package \textsf{childdoc} is \emph{not} a standard
\LaTeXe{} |.sty| style file! Therefore it needs to be invoked in
a non-standard way.

%%%%%%%%%%%%%%%%%%%%%%%%%%%%%%%%%%%%%%%%%%%%%%%%%%%%%%%%%%%%%%%%%%%%%%%%%%%%%%%%
\subsection{Included Files}
\label{sec:include}

%%%%%%%%%%%%%%%%%%%%%%%%%%%%%%%%%%%%%%%%
\DescribeMacro{\childdocmain}
To use the package, add the commands
\begin{center}
\begin{tabular}{l}
|\input{childdoc.def}|\\
|\childdocmain{}|\\
\end{tabular}
\end{center}
at the very top of the main \LaTeX{} file,
in particular \emph{before} the |\documentclass| statement!
The argument of |\childdocmain| should be left empty
(but it must be present).

%%%%%%%%%%%%%%%%%%%%%%%%%%%%%%%%%%%%%%%%
\DescribeMacro{\childdocof}
Furthermore, add the commands
\begin{center}
\begin{tabular}{l}
|\input{childdoc.def}|\\
|\childdocof{|\textit{main}|}|\\
\end{tabular}
\end{center}
at the top of every child file \textit{child}
which is included by |\include{|\textit{child}|}|
from within the main file
(or at least for those files to be compiled individually).
The argument \textit{main} must be the filename of the main file.

There are a couple of
considerations in setting up the main and child documents:

%%%%%%%%%%%%%%%%%%%%%%%%%%%%%%%%%%%%%%%%
\paragraph{Restrictions.}

Please note the following restrictions:
\begin{itemize}
\item
|\childdocmain| must be called with one argument \textit{main}
to ensure compatibility with earlier version of the package.
It must either be empty (|\childdocmain{}|)
or precisely match the filename of the main file in which it is specified.
See \secref{sec:detection} for further information.
\item
The filename \textit{main} must be specified without the |.tex| extension.
\item
The filename \textit{main} is case sensitive
(even in case-insensitive file systems)
due to internal string comparison.
\item
The argument \textit{main} should be fully expanded, it cannot be a macro.
\item
Subdirectories and special characters should be avoided in filenames.
\item
The command |\childdocmain{|\textit{main}|}| must be followed by a whitespace.
It should not be followed immediately by another command
or by a comment mark `|%|'.
This is because the \TeX{} parser reads the token immediately following
the argument of |\childdocmain| and puts it
at the beginning of every child section;
however, a white\-space is ignored.
\end{itemize}

%%%%%%%%%%%%%%%%%%%%%%%%%%%%%%%%%%%%%%%%
\paragraph{Content of Main File.}

It is advisable to place all content in the child files included by |\include|.
Any output contained in the main file will appear in all child documents
unless suppressed manually;
it cannot be suppressed automatically by the |\includeonly| directive
and thus should normally be avoided.
A method to include some content in the main file
by means of conditional processing is described in \secref{sec:conditional}.

%%%%%%%%%%%%%%%%%%%%%%%%%%%%%%%%%%%%%%%%
\paragraph{Page Numbering.}

When only a part of the document is compiled,
the appropriate numbering of pages
(as well as other status parameters)
is determined from the |.aux| files.
The latter contain information from previous passes.
However this information needs to propagate through
all intermediate child documents.
Therefore the page numbering in child documents may well
be inconsistent until the complete document is compiled at least once.

A useful (if unconventional) way to always ensure a consistent
page numbering is to restart the numbering in each child document
and denote the pages by `\textit{child}|.|\textit{page}'
where \textit{child} represents the chapter/section number of the child file.
This can be achieved by the command
|\numberwithin{page}{|\textit{child}|}|
of the \textsf{amsmath} package
where \textit{child} can be |chapter| or |section|
depending on the chosen structuring.
Alternatively, one can modify the macro |\thepage| appropriately
and reset the counter |page| at the start of each child file.

%%%%%%%%%%%%%%%%%%%%%%%%%%%%%%%%%%%%%%%%%%%%%%%%%%%%%%%%%%%%%%%%%%%%%%%%%%%%%%%%
\subsection{Conditional Processing}
\label{sec:conditional}

The package provides a mechanism to compile different versions
of a document. To customise the versions further some conditional processing
can come in handy to distinguish which version is being compiled.
The package provides two macros to describe the compilation context:

%%%%%%%%%%%%%%%%%%%%%%%%%%%%%%%%%%%%%%%%
\DescribeMacro{\ifchilddoc}
The conditional |\ifchilddoc| distinguishes between the compilation of
child documents and the main document:
%
\begin{center}
|\ifchilddoc |\textit{child-code}| |[|\||else |\textit{main-code}]| \||fi|
\end{center}

%%%%%%%%%%%%%%%%%%%%%%%%%%%%%%%%%%%%%%%%
\DescribeMacro{\childdocname}
\DescribeMacro{\childdocjob}
The macro |\childdocname| contains the filename (without extension)
of the main or child file being processed.
Note that |\childdocjob| will always contain the name of the main file.

%%%%%%%%%%%%%%%%%%%%%%%%%%%%%%%%%%%%%%%%
\paragraph{Title Page.}

Conditional processing can be used to include a title or banner page
in the main document when proper precautions are taken.
Importantly, the code in the main file should ensure that the page counter
(as well as other status parameters which are stored in the |.aux| files)
takes the same value after the conditional processing.
Otherwise the page numbers may take divergent values
depending on which part is compiled.

For example, a title page could be declared by:
%
\begin{center}
\begin{tabular}{l}
|\ifchilddoc\||else|\\
|\addtocounter{page}{-1}|\\
\textit{code for title page}\\
|\newpage|\\
|\||fi|
\end{tabular}
\end{center}
%
A banner page for the child documents can be generated by:
%
\begin{center}
\begin{tabular}{l}
|\ifchilddoc|\\
|\addtocounter{page}{-1}|\\
\textit{code for banner page}\\
|\newpage|\\
|\||fi|
\end{tabular}
\end{center}
%
Here one could write a message such as:
\begin{center}
|This is the part \childdocname{} of \childdocjob{}.|
\end{center}

%%%%%%%%%%%%%%%%%%%%%%%%%%%%%%%%%%%%%%%%%%%%%%%%%%%%%%%%%%%%%%%%%%%%%%%%%%%%%%%%
\subsection{Flags}
\label{sec:flags}

The package makes it easy to generate different versions
of the main or child documents.
To this end compilation flags can be defined
and assigned different default values.
They will be particularly useful in conjunction
with the forwarding mechanism described in \secref{sec:forward}.

For example, it may be useful to have a flag |\version|
which can be set to |draft| or |final|.
The document source will contain some conditional code
depending on the value of |\version|.
Suppose further, the flag should default to |final| for the main file
and to |draft| for child files
which is a natural assignment for editing the document.
This is achieved by placing the following code
in the preamble of the main document
(below the |\childdocmain| directive):
%
\begin{center}
\begin{tabular}{l}
|\ifchilddoc|\\
|\providecommand{\version}{draft}|\\
|\||else|\\
|\providecommand{\version}{final}|\\
|\||fi|
\end{tabular}
\end{center}
%
The definition by |\providecommand| makes sure
that previous definitions are not overwritten.
Further statements |\providecommand{\version}{...}|
can thus be added before the above code to override it.

For the main file, one might add a line
(between |\childdocmain| and the above block)
%
\begin{center}
|%\ifchilddoc\||else\providecommand{\version}{draft}\||fi|
\end{center}
%
which can be uncommented to produce a draft version.
Likewise one can add a line to the very top of a child file
(above the |\childdocof{|\textit{main}|}| directive)
%
\begin{center}
|%\providecommand{\version}{final}|
\end{center}
%
which can be uncommented to produce the final version of this child document.

%%%%%%%%%%%%%%%%%%%%%%%%%%%%%%%%%%%%%%%%%%%%%%%%%%%%%%%%%%%%%%%%%%%%%%%%%%%%%%%%
\subsection{Forwarding}
\label{sec:forward}

Different versions of the main or child documents
using compilation flags as described in \secref{sec:flags}
can be (permanently) stored in different files
for convenient compilation, viewing and distribution.
To this end, the package defines a command
to pass on compilation to a different file:

%%%%%%%%%%%%%%%%%%%%%%%%%%%%%%%%%%%%%%%%
\DescribeMacro{\childdocforward}
The command |\childdocforward| redirects processing to
another source file:
%
\begin{center}
\begin{tabular}{l}
|\input{childdoc.def}|\\
|\childdocforward[|\textit{main}|]{|\textit{dest}|}|\\
\end{tabular}
\end{center}
%
The argument \textit{dest} is the destination file
(without extension).
It should be the main file or one of the child files.
Note that further \textsf{childdoc} directives
such as |\childdocof| and |\childdocforward|
in the indicated file will be processed in this form.
The optional argument \textit{main}
passes on directly to the main file \textit{main}
while pretending to compile the child \textit{dest}.
This form behaves as if \textit{dest}
issues |\childdocof{|\textit{main}|}| right away,
and no further \textsf{childdoc} directives will be processed.

%%%%%%%%%%%%%%%%%%%%%%%%%%%%%%%%%%%%%%%%
\DescribeMacro{\...prefix}
In the alternative form |\childdocforwardprefix|,
%
\begin{center}
\begin{tabular}{l}
|\input{childdoc.def}|\\
|\childdocforwardprefix[|\textit{main}|]{|\textit{prefix}|}{|\textit{dest}|}|
\end{tabular}
\end{center}
%
the destination file is determined by a pattern
depending on the current file:
To make this work, the current file must be called
`{\textit{prefix}\hspace{0.2em}\textit{suffix}}'
with \textit{prefix} matching precisely the argument.
Processing is then passed on to the file
`{\textit{dest}\hspace{0.2em}\textit{suffix}}'.
Surely, the same effect is achieved by
directly specifying the
argument `{\textit{dest}\hspace{0.2em}\textit{suffix}}'
in the first form.
However, that requires to set up a different file
for each child. With the alternative form of the command
all these files can have exactly the same content
which simplifies setting them up and maintaining them.

For example, the following file |draft.tex|
with a compilation flag |\version| as described in \secref{sec:flags}
compiles the main document as a draft:
%
\begin{center}
\begin{tabular}{l}
|\def\version{draft}|\\
|\input{childdoc.def}|\\
|\childdocforward{|\textit{main}|}|
\end{tabular}
\end{center}
%
Likewise, the following files |final|\textit{nn}|.tex|
compile the final version of the child document
|child|\textit{nn}|.tex|:
%
\begin{center}
\begin{tabular}{l}
|\def\version{final}|\\
|\input{childdoc.def}|\\
|\childdocforwardprefix{final}{child}|
\end{tabular}
\end{center}
%

Note that when several versions of a main file and/or of each child file
are to be generated, it may be convenient to set up a |Makefile| or
shell script to automatise the process.

%%%%%%%%%%%%%%%%%%%%%%%%%%%%%%%%%%%%%%%%%%%%%%%%%%%%%%%%%%%%%%%%%%%%%%%%%%%%%%%%
\subsection{Command Line Processing}
\label{sec:commandline}

The effect of redirection files can also be achieved by invoking
the \LaTeX{} compiler with a more elaborate command line.
Most conveniently this should be done as part
of a shell script or a |Makefile|.

When using \textsf{childdoc} in the main file, the following
command lines effectively perform a redirection
(note that depending on the shell being used,
backslashes may have to be doubled: `|\|' $\to$ `|\\|'):
%
\begin{center}
|... -jobname "|\textit{target}|" |\\|"|[\textit{flags}]%
|\input{childdoc.def}\childdocforward[|\textit{main}|]{|\textit{dest}|}"|
\end{center}
%
Here \textit{target} is the name of the output file,
\textit{main} is the name of the main file
and \textit{dest} is the name of the main or child file to be processed
(all filenames without extensions).
The optional argument \textit{main} can be omitted
if \textit{main} matches \textit{dest}.
Optionally, compilation \textit{flags} can be defined via |\def| commands.
This command line makes the \TeX{} engine believe
it is compiling the file \textit{target}
whose content is specified as the latter parameter.
The provided code then forwards the processing to
\textit{main} or \textit{dest} as described in \secref{sec:forward}.

%%%%%%%%%%%%%%%%%%%%%%%%%%%%%%%%%%%%%%%%%%%%%%%%%%%%%%%%%%%%%%%%%%%%%%%%%%%%%%%%
\subsection{Include by Input}
\label{sec:input}

Including child documents by |\include| has some restrictions by design.
Most notably, the content of a child document always occupies
its own set of pages; pages cannot be shared between child documents.
Usually, this behaviour makes perfect sense
because each child document contain an essential part of the document.
However, in some situations it may be desirable to compose
a document from a collection of parts
without having mandatory page breaks between then.
For this case, the package
provides a mechanism to include parts
by |\input| which can also be processed individually.
However, by construction this mechanism
requires manual handling of the content to be output.

%%%%%%%%%%%%%%%%%%%%%%%%%%%%%%%%%%%%%%%%
\DescribeMacro{\ifchilddocmanual}
The main file should be prepared as usual, see \secref{sec:include}.
However, the document body must make a distinction
between processing of an individual part and of the main document, e.g.:
%
\begin{center}
\begin{tabular}{l}
|\ifchilddocmanual|\\
|\input{\childdocname}|\\
|\||else|\\
\textit{document body with }|\input{|\textit{part}|}|\\
|\||fi|
\end{tabular}
\end{center}
%
The conditional |\ifchilddocmanual| is true whenever
a part to be included by |\input| is being compiled,
and the name of the part is stored in |\childdocname|.

%%%%%%%%%%%%%%%%%%%%%%%%%%%%%%%%%%%%%%%%
\DescribeMacro{\childdocby}
Each part to be included by |\input| should start with:
%
\begin{center}
\begin{tabular}{l}
|\input{childdoc.def}|\\
|\childdocby{|\textit{main}|}|\\
\end{tabular}
\end{center}
%
The directive |\childdocby| is similar to |\childdocof|
described in \secref{sec:include},
but the subsequent selection of content must be done manually.
To that end, both |\ifchilddoc| and |\ifchilddocmanual|
will be true upon processing of a part,
and the name of the part is stored in |\childdocname|.
Note that |\jobname| will be set to the filename of the current part
so that each part receives an individual |.aux| file
that does not interfere with the |.aux| file(s) of the main document.
This behaviour can be altered by the alternative form
|\childdocby[*]{|\textit{main}|}| (with a non-empty optional argument)
which uses the |.aux| file of the main document
by setting |\jobname| to \textit{main}.

%%%%%%%%%%%%%%%%%%%%%%%%%%%%%%%%%%%%%%%%%%%%%%%%%%%%%%%%%%%%%%%%%%%%%%%%%%%%%%%%
\subsection{Driver Development}
\label{sec:driver}

The \textsf{childdoc} mechanism can also be use for the development
of definition files such as \LaTeX{} styles or classes.
This case differs from the above setup with multiple parts
included by |\include| in that no |\includeonly| should be invoked.
This can be achieved by starting the include file
(before |\ProvidesPackage|) with:
%
\begin{center}
\begin{tabular}{l}
|\input{childdoc.def}|\\
|\childdocforward{|\textit{main}|}|\\
\end{tabular}
\end{center}
%
or alternatively with:
%
\begin{center}
\begin{tabular}{l}
|\input{childdoc.def}|\\
|\childdocby{|\textit{main}|}|\\
\end{tabular}
\end{center}
%
Both forms have slightly different effects as described above.
The main file is prepared as usual, see \secref{sec:include}.

%%%%%%%%%%%%%%%%%%%%%%%%%%%%%%%%%%%%%%%%%%%%%%%%%%%%%%%%%%%%%%%%%%%%%%%%%%%%%%%%
\subsection{Legacy Detection}
\label{sec:detection}

The directive |\childdocmain| in the main file can detect
whether the complete document or merely a child is to be compiled
even without using the directive |\childdocof|.
This method is deprecated because it is less robust
and there is no compelling reason to use it;
it is merely provided for backward compatibility
and it may be removed in future versions.

If the detection mechanism is to be used,
it is mandatory to correctly specify
the filename of the main file as the argument of |\childdocmain|:
%
\begin{center}
\begin{tabular}{l}
|\input{childdoc.def}|\\
|\childdocmain{|\textit{main}|}|\\
\end{tabular}
\end{center}
%
If |\jobname| does not match the argument \textit{main} of |\childdocmain|,
it is assumed that |\jobname| points to the child file to be compiled.
When using |\childdocmain| with the main file specified as argument,
it suffices to start a child file
with just |\input{|\textit{main}|}|
without loading of the package and using |\childdocof|.
If instead all processing is done
with the appropriate \textsf{childdoc} directives,
the argument of \textit{main} of |\childdocmain| can be empty.

An alternative version of the command line processing described
in \secref{sec:commandline} using the detection mechanism reads:
%
\begin{center}
|... -jobname "|\textit{target}|" "|[\textit{flags}]%
[|\def\jobname{|\textit{dest}|}|]|\input{|\textit{main}|}"|
\end{center}

%%%%%%%%%%%%%%%%%%%%%%%%%%%%%%%%%%%%%%%%%%%%%%%%%%%%%%%%%%%%%%%%%%%%%%%%%%%%%%%%
\subsection{Manual Code}
\label{sec:manual}

In case one cannot be certain whether the definitions file |childdoc.def|
is installed on the target \TeX{} distribution
and one prefers not to ship it,
it is conceivable to paste a few relevant commands into the sources.

To that end, drop all statements |\input{childdoc.def}|
and perform the replacements as outlined below.
Instead of |\childdocmain{|\textit{main}|}| add the following code
to the top of the main file:
%
\begin{center}
\begin{tabular}{l}
|\||ifdefined\childdocname\endinput\||fi\newif\ifchilddoc|\\
|\edef\childdocname{\scantokens\expandafter{\jobname\noexpand}}|\\
|\def\childdocmain{|\textit{main}|}\||ifx\childdocmain\childdocname\||else|\\
|\childdoctrue\includeonly{\childdocname}\let\jobname\childdocmain\||fi|\\
\end{tabular}
\end{center}
%
Instead of |\childdocof{|\textit{main}|}| just include the main file
at the top of each child file:
%
\begin{center}
|\input{|\textit{main}|}|
\end{center}
%
A simple redirection |\childdocforward{|\textit{dest}|}| is achieved by:
%
\begin{center}
|\def\jobname{|\textit{dest}|}\input{\jobname}|
\end{center}
%
The redirection with prefix
|\childdocforwardprefix[|\textit{prefix}|]{|\textit{dest}|}|
is accomplished by:
%
\begin{center}
\begin{tabular}{l}
|{\edef\jobname{\scantokens\expandafter{\jobname\noexpand}}|\\
|\def\redirectjob |\textit{prefix}|#1~~~{\gdef\jobname{|\textit{dest}|#1}}|\\
|\expandafter\redirectjob\jobname~~~}\input{\jobname}|
\end{tabular}
\end{center}

In an alternative approach,
child documents can be compiled by a specific command line
without additional code or specific definitions:
%
\begin{center}
|... -jobname "|\textit{target}|" "|[\textit{flags}]%
|\includeonly{|\textit{dest}|}\input{|\textit{main}|}"|
\end{center}
%

%%%%%%%%%%%%%%%%%%%%%%%%%%%%%%%%%%%%%%%%%%%%%%%%%%%%%%%%%%%%%%%%%%%%%%%%%%%%%%%%
%%%%%%%%%%%%%%%%%%%%%%%%%%%%%%%%%%%%%%%%%%%%%%%%%%%%%%%%%%%%%%%%%%%%%%%%%%%%%%%%
\section{Information}

%%%%%%%%%%%%%%%%%%%%%%%%%%%%%%%%%%%%%%%%%%%%%%%%%%%%%%%%%%%%%%%%%%%%%%%%%%%%%%%%
\subsection{Copyright}

Copyright \copyright{} 2017--2018 Niklas Beisert

This work may be distributed and/or modified under the
conditions of the \LaTeX{} Project Public License, either version 1.3
of this license or (at your option) any later version.
The latest version of this license is in
  \url{http://www.latex-project.org/lppl.txt}
and version 1.3 or later is part of all distributions of \LaTeX{}
version 2005/12/01 or later.

This work has the LPPL maintenance status `maintained'.

The Current Maintainer of this work is Niklas Beisert.

This work consists of the files |README.txt|, |childdoc.ins| and |childdoc.dtx|
as well as the derived files |childdoc.def|, |cdocsamp.tex|
with |cdocsch1.tex|, |cdocsch2.tex|, |cdocspt3.tex|, |cdocspt4.tex|,
|cdocsdrf.tex|, |cdocsfn1.tex|, |cdocsfn2.tex|
as well as |childdoc.pdf|.

%%%%%%%%%%%%%%%%%%%%%%%%%%%%%%%%%%%%%%%%%%%%%%%%%%%%%%%%%%%%%%%%%%%%%%%%%%%%%%%%
\subsection{Files and Installation}

The package consists of the files:
%
\begin{center}
\begin{tabular}{ll}
    |README.txt|   & readme file \\
    |childdoc.ins| & installation file \\
    |childdoc.dtx| & source file \\
    |childdoc.def| & definition file \\
    |cdocsamp.tex| & sample main file \\
    |cdocsch1.tex| & sample include file \\
    |cdocsch2.tex| & sample include file \\
    |cdocspt3.tex| & sample part file \\
    |cdocspt4.tex| & sample part file \\
    |cdocsdrf.tex| & sample redirection file \\
    |cdocsfn1.tex| & sample redirection file \\
    |cdocsfn2.tex| & sample redirection file \\
    |childdoc.pdf| & manual
\end{tabular}
\end{center}
%
The distribution consists of the files
|README.txt|, |childdoc.ins| and |childdoc.dtx|.
%
\begin{itemize}
\item
Run (pdf)\LaTeX{} on |childdoc.dtx|
to compile the manual |childdoc.pdf| (this file).
\item
Run \LaTeX{} on |childdoc.ins| to create the definitions file |childdoc.def|
and the sample |cdocsamp.tex| with include files
|cdocsch1.tex|, |cdocsch2.tex|, |cdocspt3.tex|, |cdocspt4.tex|,
|cdocsdrf.tex|, |cdocsfn1.tex|, |cdocsfn2.tex|.
Then copy the file |childdoc.def| to an appropriate directory of your \LaTeX{}
distribution, e.g.\ \textit{texmf-root}|/tex/latex/childdoc|.
\end{itemize}

%%%%%%%%%%%%%%%%%%%%%%%%%%%%%%%%%%%%%%%%%%%%%%%%%%%%%%%%%%%%%%%%%%%%%%%%%%%%%%%%
\subsection{Related CTAN Packages}

There are several other packages which offer a similar functionality:
%
\begin{itemize}
\item
The packages
\href{http://ctan.org/pkg/docmute}{\textsf{docmute}},
\href{http://ctan.org/pkg/includex}{\textsf{includex}} and
\href{http://ctan.org/pkg/standalone}{\textsf{standalone}}
provide commands to include only the document body of
a child file thus allowing both files to be compiled individually.
\item
The packages \href{http://ctan.org/pkg/subdocs}{\textsf{subdocs}}
and \href{http://ctan.org/pkg/subfiles}{\textsf{subfiles}}
provide structures in which the main and child documents can be
encapsulated and allowing them to be compiled individually.
The inclusion mechanism is different from the conventional |\include|.
\item
The package \href{http://ctan.org/pkg/combine}{\textsf{combine}}
is an elaborate solution to combine several documents into one.
\end{itemize}
%
See also the CTAN topic \href{http://ctan.org/topic/subdocs}{\textsf{subdocs}}
for further related packages.
The present package differs from the above solutions in that
a document structure constructed with the conventional |\include| mechanism
just needs two extra commands at the top of every file
such that all constituent files can be compiled individually.

%%%%%%%%%%%%%%%%%%%%%%%%%%%%%%%%%%%%%%%%%%%%%%%%%%%%%%%%%%%%%%%%%%%%%%%%%%%%%%%%
%\subsection{Feature Suggestions}
%
%The following is a list of features which may be useful for future
%versions of this package:
%%
%\begin{itemize}
%\item
%\ldots
%\end{itemize}

%%%%%%%%%%%%%%%%%%%%%%%%%%%%%%%%%%%%%%%%%%%%%%%%%%%%%%%%%%%%%%%%%%%%%%%%%%%%%%%%
\subsection{Revision History}

%%%%%%%%%%%%%%%%%%%%%%%%%%%%%%%%%%%%%%%%
\paragraph{v2.0:} 2018/12/30

\begin{itemize}
\item
immediate forward processing
\item
added |\childdocby| mechanism
\item
manual restructured
\end{itemize}

%%%%%%%%%%%%%%%%%%%%%%%%%%%%%%%%%%%%%%%%
\paragraph{v1.6:} 2018/01/17

\begin{itemize}
\item
application for development of include files
\item
corrections to manual
\end{itemize}

%%%%%%%%%%%%%%%%%%%%%%%%%%%%%%%%%%%%%%%%
\paragraph{v1.5:} 2017/05/21

\begin{itemize}
\item
more complete structuring introduced
\item
|\childdocof| introduced
\item
|\childdoc| renamed to |\childdocmain|
\item
|\childredirect| renamed to |\childdocforward| and |\childdocforwardprefix|
and functionality expanded
\end{itemize}

%%%%%%%%%%%%%%%%%%%%%%%%%%%%%%%%%%%%%%%%
\paragraph{v1.0:} 2017/04/27

\begin{itemize}
\item
manual and install package
\item
first version published on CTAN
\end{itemize}

%%%%%%%%%%%%%%%%%%%%%%%%%%%%%%%%%%%%%%%%
\paragraph{v0.6:} 2017/04/26

\begin{itemize}
\item
redirection mechanism added
\end{itemize}

%%%%%%%%%%%%%%%%%%%%%%%%%%%%%%%%%%%%%%%%
\paragraph{v0.5:} 2017/04/26

\begin{itemize}
\item
functionality in definition file
\end{itemize}


%%%%%%%%%%%%%%%%%%%%%%%%%%%%%%%%%%%%%%%%%%%%%%%%%%%%%%%%%%%%%%%%%%%%%%%%%%%%%%%%
%%%%%%%%%%%%%%%%%%%%%%%%%%%%%%%%%%%%%%%%%%%%%%%%%%%%%%%%%%%%%%%%%%%%%%%%%%%%%%%%
%%%%%%%%%%%%%%%%%%%%%%%%%%%%%%%%%%%%%%%%%%%%%%%%%%%%%%%%%%%%%%%%%%%%%%%%%%%%%%%%
\appendix

\settowidth\MacroIndent{\rmfamily\scriptsize 000\ }

 \DocInput{childdoc.dtx}

\end{document}
%</driver>
% \fi
%
% %%%%%%%%%%%%%%%%%%%%%%%%%%%%%%%%%%%%%%%%%%%%%%%%%%%%%%%%%%%%%%%%%%%%%%%%%%%%%%
% %%%%%%%%%%%%%%%%%%%%%%%%%%%%%%%%%%%%%%%%%%%%%%%%%%%%%%%%%%%%%%%%%%%%%%%%%%%%%%
% \section{Sample}
%\iffalse
%<*samplemain>
%\fi
%
% The following presents a sample document
% with two chapters, two parts, a title page,
% a compile flag as well as three forwarding files to set the flag.
% It consists of eight |.tex| files:
% \begin{center}
% \begin{tabular}{ll}
% |cdocsamp.tex|&main file\\
% |cdocsch1.tex|&include file for chapter 1\\
% |cdocsch2.tex|&include file for chapter 2\\
% |cdocspt3.tex|&include file for part 3\\
% |cdocspt4.tex|&include file for part 4\\
% |cdocsdrf.tex|&forwarding file for main file in draft mode\\
% |cdocsfi1.tex|&forwarding file for final version of chapter 1\\
% |cdocsfi2.tex|&forwarding file for final version of chapter 2\\
% \end{tabular}
% \end{center}
% Each of the eight files can be compiled directly by the \LaTeX{} compiler.
%
% %%%%%%%%%%%%%%%%%%%%%%%%%%%%%%%%%%%%%%
% \paragraph{Main File.}
%
% The main file is called |cdocsamp.tex|.
%
% Load the \textsf{childdoc} definitions and
% declare the filename for the main document:
%    \begin{macrocode}
\input{childdoc.def}
\childdocmain{}
%    \end{macrocode}

% Optional override for |\version| flag:
%    \begin{macrocode}
%%\ifchilddoc\else\providecommand{\version}{draft}\fi
%    \end{macrocode}

% Define the default values for the |\version| flag
% (|final| for the main file and |draft| for childs):
%    \begin{macrocode}
\ifchilddoc
\providecommand{\version}{draft}
\else
\providecommand{\version}{final}
\fi
%    \end{macrocode}

% Load the standard document class:
%    \begin{macrocode}
\documentclass[12pt]{article}
%    \end{macrocode}

% Start the document body:
%    \begin{macrocode}
\begin{document}
%    \end{macrocode}

% Declare a title page.
% Print title, part of document being processed and version flag:
%    \begin{macrocode}
\addtocounter{page}{-1}
\begin{center}
{\LARGE\bfseries{}childdoc example\par}
\vspace{1cm}
\ifchilddoc
\ifchilddocmanual part\else chapter\fi:
`\childdocname' of `\childdocjob'\par
\else
main document: `\childdocjob'\par
\fi
version: \version\par
\end{center}
\newpage
%    \end{macrocode}

% Manually include selected file,
% otherwise process as usual:
%    \begin{macrocode}
\ifchilddocmanual
\section*{part `\childdocname'}
\input{\childdocname}
\else
%    \end{macrocode}

% Include the two chapters:
%    \begin{macrocode}
\include{cdocsch1}
\include{cdocsch2}
%    \end{macrocode}

% Include the two parts unless only chapters should be displayed:
%    \begin{macrocode}
\ifchilddoc\else
\section{part three}
\input{cdocspt3}
\section{part four}
\input{cdocspt4}
\fi
%    \end{macrocode}

% Process as usual until here:
%    \begin{macrocode}
\fi
%    \end{macrocode}

% End of document body:
%    \begin{macrocode}
\end{document}
%    \end{macrocode}
%\iffalse
%</samplemain>
%\fi
%
% %%%%%%%%%%%%%%%%%%%%%%%%%%%%%%%%%%%%%%
% \paragraph{Chapter Include Files.}
%
% The include files are called |cdocsch1.tex| and |cdocsch2.tex|.
%
%\iffalse
%<*samplechap1|samplechap2>
%\fi

% Optional override for |\version| flag:
%    \begin{macrocode}
%%\providecommand{\version}{final}
%    \end{macrocode}

% Include the main document:
%    \begin{macrocode}
\input{childdoc.def}
\childdocof{cdocsamp}
%    \end{macrocode}

%\iffalse
%</samplechap1|samplechap2>
%\fi
%
%\iffalse
%<*samplechap1>
%\fi
% Some text for chapter 1:
%    \begin{macrocode}
\section{one}
some text in chapter one
%    \end{macrocode}

%\iffalse
%</samplechap1>
%\fi
% Some text for chapter 2:
%\iffalse
%<*samplechap2>
%\fi
%    \begin{macrocode}
\section{two}
more text in chapter two
%    \end{macrocode}

%\iffalse
%</samplechap2>
%\fi
%
% %%%%%%%%%%%%%%%%%%%%%%%%%%%%%%%%%%%%%%
% \paragraph{Part Include Files.}
%
% The include files are called |cdocspt3.tex| and |cdocspt4.tex|.
%
%\iffalse
%<*samplepart3|samplepart4>
%\fi

% Optional override for |\version| flag:
%    \begin{macrocode}
%%\providecommand{\version}{final}
%    \end{macrocode}

% Include the main document:
%    \begin{macrocode}
\input{childdoc.def}
\childdocby{cdocsamp}
%    \end{macrocode}

%\iffalse
%</samplepart3|samplepart4>
%\fi
%
%\iffalse
%<*samplepart3>
%\fi
% Some text for part 3:
%    \begin{macrocode}
some text in part three
%    \end{macrocode}

%\iffalse
%</samplepart3>
%\fi
% Some text for part 4:
%\iffalse
%<*samplepart4>
%\fi
%    \begin{macrocode}
more text in part four
%    \end{macrocode}

%\iffalse
%</samplepart4>
%\fi
%
% %%%%%%%%%%%%%%%%%%%%%%%%%%%%%%%%%%%%%%
% \paragraph{Forwarding for a Complete Draft.}
%
% The following forwarding file |cdocsdrf.tex|
% compiles the main document in draft mode:
%\iffalse
%<*sampledraft>
%\fi
%    \begin{macrocode}
\def\version{draft}
\input{childdoc.def}
\childdocforward{cdocsamp}
%    \end{macrocode}

%\iffalse
%</sampledraft>
%\fi
%
% %%%%%%%%%%%%%%%%%%%%%%%%%%%%%%%%%%%%%%
% \paragraph{Forwarding for Final Version of the Chapters.}
%
% The following forwarding files |cdocsfn1.tex| and |cdocsfn2.tex|
% (with identical content)
% compile the final versions of the child documents
% |cdocsch1.tex| and |cdocsch2.tex|, respectively:
%\iffalse
%<*samplefinal>
%\fi
%    \begin{macrocode}
\def\version{final}
\input{childdoc.def}
\childdocforwardprefix[cdocsamp]{cdocsfn}{cdocsch}
%    \end{macrocode}

%\iffalse
%</samplefinal>
%\fi
%
% %%%%%%%%%%%%%%%%%%%%%%%%%%%%%%%%%%%%%%
% \paragraph{Command Line Processing.}
%
% The following three command lines generate the output files
% |cdocscld|, |cdocscl1| and |cdocscl2|
% which should be identical to
% |cdocsdrf|, |cdocsch1| and |cdocsfn2|, respectively:
% \begin{center}
% \begin{tabular}{l}
% |latex -jobname cdocscld \|\\
% |  "\def\version{draft}\input{childdoc.def}\childdocforward{cdocsamp}"|\\
% |latex -jobname cdocscl1 \|\\
% |  "\input{childdoc.def}\childdocforward[cdocsamp]{cdocsch1}"|\\
% |latex -jobname cdocscl2 \|\\
% |  "\def\version{final}\input{childdoc.def}\childdocforward{cdocsch2}"|
% \end{tabular}
% \end{center}
% Note that the trailing backslash on each first line
% merely continues the input to the second line
% (for convenient cut ant paste).
% Furthermore, the command |latex| can be replaced by any
% of its alternative versions such as |pdflatex|.
%
% %%%%%%%%%%%%%%%%%%%%%%%%%%%%%%%%%%%%%%%%%%%%%%%%%%%%%%%%%%%%%%%%%%%%%%%%%%%%%%
% %%%%%%%%%%%%%%%%%%%%%%%%%%%%%%%%%%%%%%%%%%%%%%%%%%%%%%%%%%%%%%%%%%%%%%%%%%%%%%
% \section{Implementation}
%\iffalse
%<*package>
%\fi
%
% This section describes the definitions file |childdoc.def|.

% The definitions cannot be loaded using |\usepackage| or |\RequirePackage|
% which has a mechanism to prevent loading a style file more than once.
% When loading the definitions by means of |\input|
% multiple instances have to be prevented manually:
%\iffalse
%This code needs to be before the `\ProvidesFile' directive
%which is defined at the beginning of this file.
%Therefore it is also placed there and commented out here.
%</package>
%<*discard>
%\fi
%    \begin{macrocode}
\ifdefined\childdocmain\endinput\fi
%    \end{macrocode}
%\iffalse
%</discard>
%<*package>
%\fi
%
% \macro{\ifchilddoc}
% \macro{\ifchilddocmanual}
% The conditional |\ifchilddoc| tells whether a
% child (true) or main (false) document is being compiled.
% The conditional |\ifchilddocmanual| tells whether
% the |\includeonly| mechanism is used (false) or
% the selection of child files must be performed manually (true).
% The definitions initialise to false:
%    \begin{macrocode}
\newif\ifchilddoc
\newif\ifchilddocmanual
%    \end{macrocode}

% \macro{\childdocname}
% \macro{\childdocjob}
% The macro |\childdocname| stores the name of the main document
% to be compiled. The macro |\childdocjob| stores the name of
% the document on which the \LaTeX{} compiler was originally invoked.
% The content of |\jobname| cannot be compared
% to filenames specified in the source due to different catcodes.
% The following code rescans |\jobname|, stores the result
% in |\childdocname| and saves a copy in |\childdocjob|:
%    \begin{macrocode}
\edef\childdocname{\scantokens\expandafter{\jobname\noexpand}}
\let\childdocjob\childdocname
%    \end{macrocode}

% \macro{\childdocdisable}
% The macro |\childdocdisable| prevents the main file
% from being processed more than once.
% At this stage, the main document command |\childdocmain|
% is assumed to be called once again where it should do nothing.
% Any subsequent call to it should prevent
% a secondary processing of the main document
% It overwrites the forwarding commands
% |\childdocof| and |\childdocforward|
% with empty macros to prevent further inclusions of the main document:
%    \begin{macrocode}
\newcommand{\childdocdisable}
{
  \renewcommand{\childdocmain}[1]{\renewcommand{\childdocmain}[1]{\endinput}}
  \renewcommand{\childdocof}[1]{}
  \renewcommand{\childdocby}[2][]{}
  \renewcommand{\childdocforward}[2][]{}
  \renewcommand{\childdocdisable}{}
}
%    \end{macrocode}

% \macro{\childdocmain}
% The macro |\childdocmain| is to be called at the top of the main file
% with nothing or the main filename (without extension) as argument.
% First, it breaks loops.
% If the argument is not empty and does not match |\childdocname|
% (which is set by the first inclusion of |childdoc.def|),
% |\ifchilddoc| is set to true, |\includeonly| is applied to the child file
% and |\jobname| is set to the main file
% (for proper handling of |.aux| files):
%    \begin{macrocode}
\newcommand{\childdocmain}[1]
{
  \childdocdisable\childdocmain{}
  \if?#1?\else
    \begingroup
      \def\childdoctmp{#1}
      \ifx\childdoctmp\childdocname
        \def\childdoctmp{}
      \else
        \def\childdoctmp
        {
          \childdoctrue
          \includeonly{\childdocname}
          \def\childdocjob{#1}
          \def\jobname{#1}
        }
      \fi
      \expandafter
    \endgroup
    \childdoctmp
  \fi
}
%    \end{macrocode}

% \macro{\childdocof}
% The command |\childdocof| redirects
% compilation to the main file |#1|.
%    \begin{macrocode}
\newcommand{\childdocof}[1]
{
  \childdocdisable
  \childdoctrue
  \includeonly{\childdocname}
  \def\jobname{#1}
  \def\childdocjob{#1}
  \input{#1}
}
%    \end{macrocode}

% \macro{\childdocby}
% The command |\childdocby| ....
%    \begin{macrocode}
\newcommand{\childdocby}[2][]
{
  \childdocdisable
  \childdoctrue
  \childdocmanualtrue
  \if?#1?\else
    \def\jobname{#2}
  \fi
  \def\childdocjob{#2}
  \input{#2}
  \endinput
}
%    \end{macrocode}

% \macro{\childdocforward}
% The command |\childdocforward| redirects
% compilation to the main file or
% (if the optional argument is given) a child file.
% Parameters are set as if the main file
% or a child file starting with |\childdocof| was compiled.
% Then compilation is handed over to the main file:
%    \begin{macrocode}
\newcommand{\childdocforward}[2][]
{
  \begingroup
    \if?#1?
      \def\childdoctmp
      {
        \def\childdocname{#2}
        \def\childdocjob{#2}
        \def\jobname{#2}
        \input{#2}
        \endinput
      }
    \else
      \def\childdoctmp
      {
        \childdocdisable
        \def\childdocname{#2}
        \childdoctrue
        \includeonly{#2}
        \def\childdocjob{#1}
        \def\jobname{#1}
        \input{#1}
        \endinput
      }
    \fi
    \expandafter
  \endgroup
  \childdoctmp
}
%    \end{macrocode}

% \macro{\childdocforwardprefix}
% The command |\childdocforwardprefix| redirects
% compilation to the main or a child file by means of a pattern.
% The prefix |#1| in the current filename is replaced by |#2|
% and the suffix of the current filename is kept
% (it is assumed that the filename does not contain the substring `|~~~|'
% which is used as a delimiter).
% Compilation is handed over to the new file by |\childdocforward|:
%    \begin{macrocode}
\newcommand{\childdocforwardprefix}[3][]
{
  \begingroup
    \def\childdocextract #2##1~~~{\def\childdoctmp{\childdocforward[#1]{#3##1}}}
    \expandafter\childdocextract\childdocname~~~
    \expandafter
  \endgroup
  \childdoctmp
}
%    \end{macrocode}

% \macro{\childdoc}
% The deprecated macro |\childdoc| is a legacy version of |\childdocmain|:
%    \begin{macrocode}
\newcommand{\childdoc}{\childdocmain}
%    \end{macrocode}

% \macro{\childdocredirect}
% The deprecated macro |\childdocredirect| is a legacy version
% of |\childdocforward| and |\childdocforwardprefix|:
%    \begin{macrocode}
\newcommand{\childdocredirect}[2][]
{
  \begingroup
    \if?#1?
      \def\childdoctmp{\childdocforward{#2}}
    \else
      \def\childdoctmp{\childdocforwardprefix{#1}{#2}}
    \fi
    \expandafter
  \endgroup
  \childdoctmp
}
%    \end{macrocode}

%\iffalse
%</package>
%\fi
%
\endinput
|\\
|\childdocmain{|\textit{main}|}|\\
\end{tabular}
\end{center}
%
If |\jobname| does not match the argument \textit{main} of |\childdocmain|,
it is assumed that |\jobname| points to the child file to be compiled.
When using |\childdocmain| with the main file specified as argument,
it suffices to start a child file
with just |\input{|\textit{main}|}|
without loading of the package and using |\childdocof|.
If instead all processing is done
with the appropriate \textsf{childdoc} directives,
the argument of \textit{main} of |\childdocmain| can be empty.

An alternative version of the command line processing described
in \secref{sec:commandline} using the detection mechanism reads:
%
\begin{center}
|... -jobname "|\textit{target}|" "|[\textit{flags}]%
[|\def\jobname{|\textit{dest}|}|]|\input{|\textit{main}|}"|
\end{center}

%%%%%%%%%%%%%%%%%%%%%%%%%%%%%%%%%%%%%%%%%%%%%%%%%%%%%%%%%%%%%%%%%%%%%%%%%%%%%%%%
\subsection{Manual Code}
\label{sec:manual}

In case one cannot be certain whether the definitions file |childdoc.def|
is installed on the target \TeX{} distribution
and one prefers not to ship it,
it is conceivable to paste a few relevant commands into the sources.

To that end, drop all statements |% \iffalse
%
% childdoc.dtx Copyright (C) 2017-2018 Niklas Beisert
%
% This work may be distributed and/or modified under the
% conditions of the LaTeX Project Public License, either version 1.3
% of this license or (at your option) any later version.
% The latest version of this license is in
%   http://www.latex-project.org/lppl.txt
% and version 1.3 or later is part of all distributions of LaTeX
% version 2005/12/01 or later.
%
% This work has the LPPL maintenance status `maintained'.
%
% The Current Maintainer of this work is Niklas Beisert.
%
% This work consists of the files childdoc.dtx and childdoc.ins
% and the derived files childdoc.def and cdocsamp.tex with
% cdocsch1.tex, cdocsch2.tex, cdocsdrf.tex, cdocsfn1.tex, cdocsfn2.tex.
%
%<package>\ifdefined\childdocmain\endinput\fi
%<package>\ProvidesFile{childdoc.def}[2018/12/30 v2.0 child document driver]
%<samplemain>\ProvidesFile{cdocsamp.tex}[2018/12/30 v2.0 sample for childdoc]
%<*driver>
%\ProvidesFile{childdoc.drv}[2018/12/30 v2.0 childdoc reference manual file]
\PassOptionsToClass{10pt,a4paper}{article}
\documentclass{ltxdoc}

\usepackage[margin=35mm]{geometry}
\usepackage{hyperref}
\usepackage{hyperxmp}
\usepackage[usenames]{color}

\hypersetup{colorlinks=true}
\hypersetup{pdfstartview=FitH}
\hypersetup{pdfpagemode=UseNone}
\hypersetup{pdfsource={}}
\hypersetup{pdflang={en-UK}}
\hypersetup{pdfcopyright={Copyright 2017-2018 Niklas Beisert.
  This work may be distributed and/or modified under the
  conditions of the LaTeX Project Public License, either version 1.3
  of this license or (at your option) any later version.}}
\hypersetup{pdflicenseurl={http://www.latex-project.org/lppl.txt}}
\hypersetup{pdfcontactaddress={ETH Zurich, ITP, HIT K,
  Wolfgang-Pauli-Strasse 27}}
\hypersetup{pdfcontactpostcode={8093}}
\hypersetup{pdfcontactcity={Zurich}}
\hypersetup{pdfcontactcountry={Switzerland}}
\hypersetup{pdfcontactemail={nbeisert@itp.phys.ethz.ch}}
\hypersetup{pdfcontacturl={http://people.phys.ethz.ch/\xmptilde nbeisert/}}

\newcommand{\secref}[1]{\hyperref[#1]{section \ref*{#1}}}

\parskip1ex
\parindent0pt
\let\olditemize\itemize
\def\itemize{\olditemize\parskip0pt}

\begin{document}

\title{The \textsf{childdoc} Package}
\hypersetup{pdftitle={The childdoc Package}}
\author{Niklas Beisert\\[2ex]
  Institut f\"ur Theoretische Physik\\
  Eidgen\"ossische Technische Hochschule Z\"urich\\
  Wolfgang-Pauli-Strasse 27, 8093 Z\"urich, Switzerland\\[1ex]
  \href{mailto:nbeisert@itp.phys.ethz.ch}
  {\texttt{nbeisert@itp.phys.ethz.ch}}}
\hypersetup{pdfauthor={Niklas Beisert}}
\hypersetup{pdfsubject={Manual for the LaTeX2e Package childdoc}}
\date{30 December 2018, \textsf{v2.0}}
\maketitle

\begin{abstract}\noindent
\textsf{childdoc} is a \LaTeXe{} package
that enables the direct compilation
of document sections included by |\include|
to individual files.
\end{abstract}

\begingroup
\parskip0ex
\tableofcontents
\endgroup

%%%%%%%%%%%%%%%%%%%%%%%%%%%%%%%%%%%%%%%%%%%%%%%%%%%%%%%%%%%%%%%%%%%%%%%%%%%%%%%%
%%%%%%%%%%%%%%%%%%%%%%%%%%%%%%%%%%%%%%%%%%%%%%%%%%%%%%%%%%%%%%%%%%%%%%%%%%%%%%%%
\section{Introduction}

\LaTeX{} provides a mechanism to structure a large document (such as a book)
into a main file and several child files (containing the chapters)
using the |\include| command.
This mechanism is beneficial for documents
which span hundreds of pages in order to
make the source file(s) more manageable.
Moreover, compilation can be restricted to
selected child files by means of the |\includeonly| command.
The latter feature can be used to reduce the compilation time while editing
(this was significantly more useful in the earlier days of \LaTeX{})
or to generate a smaller document which is easier to navigate.
Another application of |\includeonly| is to generate
documents consisting of selected parts of the complete document.

However, there are a few drawbacks of the plain |\include| mechanism:
\begin{itemize}
\item
The child files cannot be compiled on their own,
they can only be compiled via the main file.
A naive editing environment
(such as a text editor with an option
to have the current file processed by \LaTeX)
may require one to switch to the main file before compiling;
attempting to compile the child file produces errors.
\item
The main file must be modified (each time)
to adjust the |\includeonly| command
to the present needs. This easily leaves the main file in a messy state.
\item
The generated document will always carry the filename
of the main document. This is inconvenient if
several child files are to be compiled and
to be kept for distribution.
\end{itemize}

The present package provides a simple interface
to make child files individually compilable by \LaTeX{}.
Compiling a child file then has the same effect as compiling
the main file with an |\includeonly| command
to select the appropriate child.
Moreover the generated document will carry the name of the child
rather than the main file.
This resolves all three above issues.

This feature is meant to make the editing of books,
thesis documents and lecture notes somewhat more convenient.
However, the package can also be used efficiently for
composing a series of documents (such as exercise sheets)
which are typically distributed individually.
It then assists the author in generating the individual documents
(potentially in different versions)
as well as a document containing the collected series.
Another application is in developing style files
or other kinds of included material
where compilation of the style file could redirect
to a sample or test file.

%%%%%%%%%%%%%%%%%%%%%%%%%%%%%%%%%%%%%%%%%%%%%%%%%%%%%%%%%%%%%%%%%%%%%%%%%%%%%%%%
%%%%%%%%%%%%%%%%%%%%%%%%%%%%%%%%%%%%%%%%%%%%%%%%%%%%%%%%%%%%%%%%%%%%%%%%%%%%%%%%
\section{Usage}

First of all, the package \textsf{childdoc} is \emph{not} a standard
\LaTeXe{} |.sty| style file! Therefore it needs to be invoked in
a non-standard way.

%%%%%%%%%%%%%%%%%%%%%%%%%%%%%%%%%%%%%%%%%%%%%%%%%%%%%%%%%%%%%%%%%%%%%%%%%%%%%%%%
\subsection{Included Files}
\label{sec:include}

%%%%%%%%%%%%%%%%%%%%%%%%%%%%%%%%%%%%%%%%
\DescribeMacro{\childdocmain}
To use the package, add the commands
\begin{center}
\begin{tabular}{l}
|\input{childdoc.def}|\\
|\childdocmain{}|\\
\end{tabular}
\end{center}
at the very top of the main \LaTeX{} file,
in particular \emph{before} the |\documentclass| statement!
The argument of |\childdocmain| should be left empty
(but it must be present).

%%%%%%%%%%%%%%%%%%%%%%%%%%%%%%%%%%%%%%%%
\DescribeMacro{\childdocof}
Furthermore, add the commands
\begin{center}
\begin{tabular}{l}
|\input{childdoc.def}|\\
|\childdocof{|\textit{main}|}|\\
\end{tabular}
\end{center}
at the top of every child file \textit{child}
which is included by |\include{|\textit{child}|}|
from within the main file
(or at least for those files to be compiled individually).
The argument \textit{main} must be the filename of the main file.

There are a couple of
considerations in setting up the main and child documents:

%%%%%%%%%%%%%%%%%%%%%%%%%%%%%%%%%%%%%%%%
\paragraph{Restrictions.}

Please note the following restrictions:
\begin{itemize}
\item
|\childdocmain| must be called with one argument \textit{main}
to ensure compatibility with earlier version of the package.
It must either be empty (|\childdocmain{}|)
or precisely match the filename of the main file in which it is specified.
See \secref{sec:detection} for further information.
\item
The filename \textit{main} must be specified without the |.tex| extension.
\item
The filename \textit{main} is case sensitive
(even in case-insensitive file systems)
due to internal string comparison.
\item
The argument \textit{main} should be fully expanded, it cannot be a macro.
\item
Subdirectories and special characters should be avoided in filenames.
\item
The command |\childdocmain{|\textit{main}|}| must be followed by a whitespace.
It should not be followed immediately by another command
or by a comment mark `|%|'.
This is because the \TeX{} parser reads the token immediately following
the argument of |\childdocmain| and puts it
at the beginning of every child section;
however, a white\-space is ignored.
\end{itemize}

%%%%%%%%%%%%%%%%%%%%%%%%%%%%%%%%%%%%%%%%
\paragraph{Content of Main File.}

It is advisable to place all content in the child files included by |\include|.
Any output contained in the main file will appear in all child documents
unless suppressed manually;
it cannot be suppressed automatically by the |\includeonly| directive
and thus should normally be avoided.
A method to include some content in the main file
by means of conditional processing is described in \secref{sec:conditional}.

%%%%%%%%%%%%%%%%%%%%%%%%%%%%%%%%%%%%%%%%
\paragraph{Page Numbering.}

When only a part of the document is compiled,
the appropriate numbering of pages
(as well as other status parameters)
is determined from the |.aux| files.
The latter contain information from previous passes.
However this information needs to propagate through
all intermediate child documents.
Therefore the page numbering in child documents may well
be inconsistent until the complete document is compiled at least once.

A useful (if unconventional) way to always ensure a consistent
page numbering is to restart the numbering in each child document
and denote the pages by `\textit{child}|.|\textit{page}'
where \textit{child} represents the chapter/section number of the child file.
This can be achieved by the command
|\numberwithin{page}{|\textit{child}|}|
of the \textsf{amsmath} package
where \textit{child} can be |chapter| or |section|
depending on the chosen structuring.
Alternatively, one can modify the macro |\thepage| appropriately
and reset the counter |page| at the start of each child file.

%%%%%%%%%%%%%%%%%%%%%%%%%%%%%%%%%%%%%%%%%%%%%%%%%%%%%%%%%%%%%%%%%%%%%%%%%%%%%%%%
\subsection{Conditional Processing}
\label{sec:conditional}

The package provides a mechanism to compile different versions
of a document. To customise the versions further some conditional processing
can come in handy to distinguish which version is being compiled.
The package provides two macros to describe the compilation context:

%%%%%%%%%%%%%%%%%%%%%%%%%%%%%%%%%%%%%%%%
\DescribeMacro{\ifchilddoc}
The conditional |\ifchilddoc| distinguishes between the compilation of
child documents and the main document:
%
\begin{center}
|\ifchilddoc |\textit{child-code}| |[|\||else |\textit{main-code}]| \||fi|
\end{center}

%%%%%%%%%%%%%%%%%%%%%%%%%%%%%%%%%%%%%%%%
\DescribeMacro{\childdocname}
\DescribeMacro{\childdocjob}
The macro |\childdocname| contains the filename (without extension)
of the main or child file being processed.
Note that |\childdocjob| will always contain the name of the main file.

%%%%%%%%%%%%%%%%%%%%%%%%%%%%%%%%%%%%%%%%
\paragraph{Title Page.}

Conditional processing can be used to include a title or banner page
in the main document when proper precautions are taken.
Importantly, the code in the main file should ensure that the page counter
(as well as other status parameters which are stored in the |.aux| files)
takes the same value after the conditional processing.
Otherwise the page numbers may take divergent values
depending on which part is compiled.

For example, a title page could be declared by:
%
\begin{center}
\begin{tabular}{l}
|\ifchilddoc\||else|\\
|\addtocounter{page}{-1}|\\
\textit{code for title page}\\
|\newpage|\\
|\||fi|
\end{tabular}
\end{center}
%
A banner page for the child documents can be generated by:
%
\begin{center}
\begin{tabular}{l}
|\ifchilddoc|\\
|\addtocounter{page}{-1}|\\
\textit{code for banner page}\\
|\newpage|\\
|\||fi|
\end{tabular}
\end{center}
%
Here one could write a message such as:
\begin{center}
|This is the part \childdocname{} of \childdocjob{}.|
\end{center}

%%%%%%%%%%%%%%%%%%%%%%%%%%%%%%%%%%%%%%%%%%%%%%%%%%%%%%%%%%%%%%%%%%%%%%%%%%%%%%%%
\subsection{Flags}
\label{sec:flags}

The package makes it easy to generate different versions
of the main or child documents.
To this end compilation flags can be defined
and assigned different default values.
They will be particularly useful in conjunction
with the forwarding mechanism described in \secref{sec:forward}.

For example, it may be useful to have a flag |\version|
which can be set to |draft| or |final|.
The document source will contain some conditional code
depending on the value of |\version|.
Suppose further, the flag should default to |final| for the main file
and to |draft| for child files
which is a natural assignment for editing the document.
This is achieved by placing the following code
in the preamble of the main document
(below the |\childdocmain| directive):
%
\begin{center}
\begin{tabular}{l}
|\ifchilddoc|\\
|\providecommand{\version}{draft}|\\
|\||else|\\
|\providecommand{\version}{final}|\\
|\||fi|
\end{tabular}
\end{center}
%
The definition by |\providecommand| makes sure
that previous definitions are not overwritten.
Further statements |\providecommand{\version}{...}|
can thus be added before the above code to override it.

For the main file, one might add a line
(between |\childdocmain| and the above block)
%
\begin{center}
|%\ifchilddoc\||else\providecommand{\version}{draft}\||fi|
\end{center}
%
which can be uncommented to produce a draft version.
Likewise one can add a line to the very top of a child file
(above the |\childdocof{|\textit{main}|}| directive)
%
\begin{center}
|%\providecommand{\version}{final}|
\end{center}
%
which can be uncommented to produce the final version of this child document.

%%%%%%%%%%%%%%%%%%%%%%%%%%%%%%%%%%%%%%%%%%%%%%%%%%%%%%%%%%%%%%%%%%%%%%%%%%%%%%%%
\subsection{Forwarding}
\label{sec:forward}

Different versions of the main or child documents
using compilation flags as described in \secref{sec:flags}
can be (permanently) stored in different files
for convenient compilation, viewing and distribution.
To this end, the package defines a command
to pass on compilation to a different file:

%%%%%%%%%%%%%%%%%%%%%%%%%%%%%%%%%%%%%%%%
\DescribeMacro{\childdocforward}
The command |\childdocforward| redirects processing to
another source file:
%
\begin{center}
\begin{tabular}{l}
|\input{childdoc.def}|\\
|\childdocforward[|\textit{main}|]{|\textit{dest}|}|\\
\end{tabular}
\end{center}
%
The argument \textit{dest} is the destination file
(without extension).
It should be the main file or one of the child files.
Note that further \textsf{childdoc} directives
such as |\childdocof| and |\childdocforward|
in the indicated file will be processed in this form.
The optional argument \textit{main}
passes on directly to the main file \textit{main}
while pretending to compile the child \textit{dest}.
This form behaves as if \textit{dest}
issues |\childdocof{|\textit{main}|}| right away,
and no further \textsf{childdoc} directives will be processed.

%%%%%%%%%%%%%%%%%%%%%%%%%%%%%%%%%%%%%%%%
\DescribeMacro{\...prefix}
In the alternative form |\childdocforwardprefix|,
%
\begin{center}
\begin{tabular}{l}
|\input{childdoc.def}|\\
|\childdocforwardprefix[|\textit{main}|]{|\textit{prefix}|}{|\textit{dest}|}|
\end{tabular}
\end{center}
%
the destination file is determined by a pattern
depending on the current file:
To make this work, the current file must be called
`{\textit{prefix}\hspace{0.2em}\textit{suffix}}'
with \textit{prefix} matching precisely the argument.
Processing is then passed on to the file
`{\textit{dest}\hspace{0.2em}\textit{suffix}}'.
Surely, the same effect is achieved by
directly specifying the
argument `{\textit{dest}\hspace{0.2em}\textit{suffix}}'
in the first form.
However, that requires to set up a different file
for each child. With the alternative form of the command
all these files can have exactly the same content
which simplifies setting them up and maintaining them.

For example, the following file |draft.tex|
with a compilation flag |\version| as described in \secref{sec:flags}
compiles the main document as a draft:
%
\begin{center}
\begin{tabular}{l}
|\def\version{draft}|\\
|\input{childdoc.def}|\\
|\childdocforward{|\textit{main}|}|
\end{tabular}
\end{center}
%
Likewise, the following files |final|\textit{nn}|.tex|
compile the final version of the child document
|child|\textit{nn}|.tex|:
%
\begin{center}
\begin{tabular}{l}
|\def\version{final}|\\
|\input{childdoc.def}|\\
|\childdocforwardprefix{final}{child}|
\end{tabular}
\end{center}
%

Note that when several versions of a main file and/or of each child file
are to be generated, it may be convenient to set up a |Makefile| or
shell script to automatise the process.

%%%%%%%%%%%%%%%%%%%%%%%%%%%%%%%%%%%%%%%%%%%%%%%%%%%%%%%%%%%%%%%%%%%%%%%%%%%%%%%%
\subsection{Command Line Processing}
\label{sec:commandline}

The effect of redirection files can also be achieved by invoking
the \LaTeX{} compiler with a more elaborate command line.
Most conveniently this should be done as part
of a shell script or a |Makefile|.

When using \textsf{childdoc} in the main file, the following
command lines effectively perform a redirection
(note that depending on the shell being used,
backslashes may have to be doubled: `|\|' $\to$ `|\\|'):
%
\begin{center}
|... -jobname "|\textit{target}|" |\\|"|[\textit{flags}]%
|\input{childdoc.def}\childdocforward[|\textit{main}|]{|\textit{dest}|}"|
\end{center}
%
Here \textit{target} is the name of the output file,
\textit{main} is the name of the main file
and \textit{dest} is the name of the main or child file to be processed
(all filenames without extensions).
The optional argument \textit{main} can be omitted
if \textit{main} matches \textit{dest}.
Optionally, compilation \textit{flags} can be defined via |\def| commands.
This command line makes the \TeX{} engine believe
it is compiling the file \textit{target}
whose content is specified as the latter parameter.
The provided code then forwards the processing to
\textit{main} or \textit{dest} as described in \secref{sec:forward}.

%%%%%%%%%%%%%%%%%%%%%%%%%%%%%%%%%%%%%%%%%%%%%%%%%%%%%%%%%%%%%%%%%%%%%%%%%%%%%%%%
\subsection{Include by Input}
\label{sec:input}

Including child documents by |\include| has some restrictions by design.
Most notably, the content of a child document always occupies
its own set of pages; pages cannot be shared between child documents.
Usually, this behaviour makes perfect sense
because each child document contain an essential part of the document.
However, in some situations it may be desirable to compose
a document from a collection of parts
without having mandatory page breaks between then.
For this case, the package
provides a mechanism to include parts
by |\input| which can also be processed individually.
However, by construction this mechanism
requires manual handling of the content to be output.

%%%%%%%%%%%%%%%%%%%%%%%%%%%%%%%%%%%%%%%%
\DescribeMacro{\ifchilddocmanual}
The main file should be prepared as usual, see \secref{sec:include}.
However, the document body must make a distinction
between processing of an individual part and of the main document, e.g.:
%
\begin{center}
\begin{tabular}{l}
|\ifchilddocmanual|\\
|\input{\childdocname}|\\
|\||else|\\
\textit{document body with }|\input{|\textit{part}|}|\\
|\||fi|
\end{tabular}
\end{center}
%
The conditional |\ifchilddocmanual| is true whenever
a part to be included by |\input| is being compiled,
and the name of the part is stored in |\childdocname|.

%%%%%%%%%%%%%%%%%%%%%%%%%%%%%%%%%%%%%%%%
\DescribeMacro{\childdocby}
Each part to be included by |\input| should start with:
%
\begin{center}
\begin{tabular}{l}
|\input{childdoc.def}|\\
|\childdocby{|\textit{main}|}|\\
\end{tabular}
\end{center}
%
The directive |\childdocby| is similar to |\childdocof|
described in \secref{sec:include},
but the subsequent selection of content must be done manually.
To that end, both |\ifchilddoc| and |\ifchilddocmanual|
will be true upon processing of a part,
and the name of the part is stored in |\childdocname|.
Note that |\jobname| will be set to the filename of the current part
so that each part receives an individual |.aux| file
that does not interfere with the |.aux| file(s) of the main document.
This behaviour can be altered by the alternative form
|\childdocby[*]{|\textit{main}|}| (with a non-empty optional argument)
which uses the |.aux| file of the main document
by setting |\jobname| to \textit{main}.

%%%%%%%%%%%%%%%%%%%%%%%%%%%%%%%%%%%%%%%%%%%%%%%%%%%%%%%%%%%%%%%%%%%%%%%%%%%%%%%%
\subsection{Driver Development}
\label{sec:driver}

The \textsf{childdoc} mechanism can also be use for the development
of definition files such as \LaTeX{} styles or classes.
This case differs from the above setup with multiple parts
included by |\include| in that no |\includeonly| should be invoked.
This can be achieved by starting the include file
(before |\ProvidesPackage|) with:
%
\begin{center}
\begin{tabular}{l}
|\input{childdoc.def}|\\
|\childdocforward{|\textit{main}|}|\\
\end{tabular}
\end{center}
%
or alternatively with:
%
\begin{center}
\begin{tabular}{l}
|\input{childdoc.def}|\\
|\childdocby{|\textit{main}|}|\\
\end{tabular}
\end{center}
%
Both forms have slightly different effects as described above.
The main file is prepared as usual, see \secref{sec:include}.

%%%%%%%%%%%%%%%%%%%%%%%%%%%%%%%%%%%%%%%%%%%%%%%%%%%%%%%%%%%%%%%%%%%%%%%%%%%%%%%%
\subsection{Legacy Detection}
\label{sec:detection}

The directive |\childdocmain| in the main file can detect
whether the complete document or merely a child is to be compiled
even without using the directive |\childdocof|.
This method is deprecated because it is less robust
and there is no compelling reason to use it;
it is merely provided for backward compatibility
and it may be removed in future versions.

If the detection mechanism is to be used,
it is mandatory to correctly specify
the filename of the main file as the argument of |\childdocmain|:
%
\begin{center}
\begin{tabular}{l}
|\input{childdoc.def}|\\
|\childdocmain{|\textit{main}|}|\\
\end{tabular}
\end{center}
%
If |\jobname| does not match the argument \textit{main} of |\childdocmain|,
it is assumed that |\jobname| points to the child file to be compiled.
When using |\childdocmain| with the main file specified as argument,
it suffices to start a child file
with just |\input{|\textit{main}|}|
without loading of the package and using |\childdocof|.
If instead all processing is done
with the appropriate \textsf{childdoc} directives,
the argument of \textit{main} of |\childdocmain| can be empty.

An alternative version of the command line processing described
in \secref{sec:commandline} using the detection mechanism reads:
%
\begin{center}
|... -jobname "|\textit{target}|" "|[\textit{flags}]%
[|\def\jobname{|\textit{dest}|}|]|\input{|\textit{main}|}"|
\end{center}

%%%%%%%%%%%%%%%%%%%%%%%%%%%%%%%%%%%%%%%%%%%%%%%%%%%%%%%%%%%%%%%%%%%%%%%%%%%%%%%%
\subsection{Manual Code}
\label{sec:manual}

In case one cannot be certain whether the definitions file |childdoc.def|
is installed on the target \TeX{} distribution
and one prefers not to ship it,
it is conceivable to paste a few relevant commands into the sources.

To that end, drop all statements |\input{childdoc.def}|
and perform the replacements as outlined below.
Instead of |\childdocmain{|\textit{main}|}| add the following code
to the top of the main file:
%
\begin{center}
\begin{tabular}{l}
|\||ifdefined\childdocname\endinput\||fi\newif\ifchilddoc|\\
|\edef\childdocname{\scantokens\expandafter{\jobname\noexpand}}|\\
|\def\childdocmain{|\textit{main}|}\||ifx\childdocmain\childdocname\||else|\\
|\childdoctrue\includeonly{\childdocname}\let\jobname\childdocmain\||fi|\\
\end{tabular}
\end{center}
%
Instead of |\childdocof{|\textit{main}|}| just include the main file
at the top of each child file:
%
\begin{center}
|\input{|\textit{main}|}|
\end{center}
%
A simple redirection |\childdocforward{|\textit{dest}|}| is achieved by:
%
\begin{center}
|\def\jobname{|\textit{dest}|}\input{\jobname}|
\end{center}
%
The redirection with prefix
|\childdocforwardprefix[|\textit{prefix}|]{|\textit{dest}|}|
is accomplished by:
%
\begin{center}
\begin{tabular}{l}
|{\edef\jobname{\scantokens\expandafter{\jobname\noexpand}}|\\
|\def\redirectjob |\textit{prefix}|#1~~~{\gdef\jobname{|\textit{dest}|#1}}|\\
|\expandafter\redirectjob\jobname~~~}\input{\jobname}|
\end{tabular}
\end{center}

In an alternative approach,
child documents can be compiled by a specific command line
without additional code or specific definitions:
%
\begin{center}
|... -jobname "|\textit{target}|" "|[\textit{flags}]%
|\includeonly{|\textit{dest}|}\input{|\textit{main}|}"|
\end{center}
%

%%%%%%%%%%%%%%%%%%%%%%%%%%%%%%%%%%%%%%%%%%%%%%%%%%%%%%%%%%%%%%%%%%%%%%%%%%%%%%%%
%%%%%%%%%%%%%%%%%%%%%%%%%%%%%%%%%%%%%%%%%%%%%%%%%%%%%%%%%%%%%%%%%%%%%%%%%%%%%%%%
\section{Information}

%%%%%%%%%%%%%%%%%%%%%%%%%%%%%%%%%%%%%%%%%%%%%%%%%%%%%%%%%%%%%%%%%%%%%%%%%%%%%%%%
\subsection{Copyright}

Copyright \copyright{} 2017--2018 Niklas Beisert

This work may be distributed and/or modified under the
conditions of the \LaTeX{} Project Public License, either version 1.3
of this license or (at your option) any later version.
The latest version of this license is in
  \url{http://www.latex-project.org/lppl.txt}
and version 1.3 or later is part of all distributions of \LaTeX{}
version 2005/12/01 or later.

This work has the LPPL maintenance status `maintained'.

The Current Maintainer of this work is Niklas Beisert.

This work consists of the files |README.txt|, |childdoc.ins| and |childdoc.dtx|
as well as the derived files |childdoc.def|, |cdocsamp.tex|
with |cdocsch1.tex|, |cdocsch2.tex|, |cdocspt3.tex|, |cdocspt4.tex|,
|cdocsdrf.tex|, |cdocsfn1.tex|, |cdocsfn2.tex|
as well as |childdoc.pdf|.

%%%%%%%%%%%%%%%%%%%%%%%%%%%%%%%%%%%%%%%%%%%%%%%%%%%%%%%%%%%%%%%%%%%%%%%%%%%%%%%%
\subsection{Files and Installation}

The package consists of the files:
%
\begin{center}
\begin{tabular}{ll}
    |README.txt|   & readme file \\
    |childdoc.ins| & installation file \\
    |childdoc.dtx| & source file \\
    |childdoc.def| & definition file \\
    |cdocsamp.tex| & sample main file \\
    |cdocsch1.tex| & sample include file \\
    |cdocsch2.tex| & sample include file \\
    |cdocspt3.tex| & sample part file \\
    |cdocspt4.tex| & sample part file \\
    |cdocsdrf.tex| & sample redirection file \\
    |cdocsfn1.tex| & sample redirection file \\
    |cdocsfn2.tex| & sample redirection file \\
    |childdoc.pdf| & manual
\end{tabular}
\end{center}
%
The distribution consists of the files
|README.txt|, |childdoc.ins| and |childdoc.dtx|.
%
\begin{itemize}
\item
Run (pdf)\LaTeX{} on |childdoc.dtx|
to compile the manual |childdoc.pdf| (this file).
\item
Run \LaTeX{} on |childdoc.ins| to create the definitions file |childdoc.def|
and the sample |cdocsamp.tex| with include files
|cdocsch1.tex|, |cdocsch2.tex|, |cdocspt3.tex|, |cdocspt4.tex|,
|cdocsdrf.tex|, |cdocsfn1.tex|, |cdocsfn2.tex|.
Then copy the file |childdoc.def| to an appropriate directory of your \LaTeX{}
distribution, e.g.\ \textit{texmf-root}|/tex/latex/childdoc|.
\end{itemize}

%%%%%%%%%%%%%%%%%%%%%%%%%%%%%%%%%%%%%%%%%%%%%%%%%%%%%%%%%%%%%%%%%%%%%%%%%%%%%%%%
\subsection{Related CTAN Packages}

There are several other packages which offer a similar functionality:
%
\begin{itemize}
\item
The packages
\href{http://ctan.org/pkg/docmute}{\textsf{docmute}},
\href{http://ctan.org/pkg/includex}{\textsf{includex}} and
\href{http://ctan.org/pkg/standalone}{\textsf{standalone}}
provide commands to include only the document body of
a child file thus allowing both files to be compiled individually.
\item
The packages \href{http://ctan.org/pkg/subdocs}{\textsf{subdocs}}
and \href{http://ctan.org/pkg/subfiles}{\textsf{subfiles}}
provide structures in which the main and child documents can be
encapsulated and allowing them to be compiled individually.
The inclusion mechanism is different from the conventional |\include|.
\item
The package \href{http://ctan.org/pkg/combine}{\textsf{combine}}
is an elaborate solution to combine several documents into one.
\end{itemize}
%
See also the CTAN topic \href{http://ctan.org/topic/subdocs}{\textsf{subdocs}}
for further related packages.
The present package differs from the above solutions in that
a document structure constructed with the conventional |\include| mechanism
just needs two extra commands at the top of every file
such that all constituent files can be compiled individually.

%%%%%%%%%%%%%%%%%%%%%%%%%%%%%%%%%%%%%%%%%%%%%%%%%%%%%%%%%%%%%%%%%%%%%%%%%%%%%%%%
%\subsection{Feature Suggestions}
%
%The following is a list of features which may be useful for future
%versions of this package:
%%
%\begin{itemize}
%\item
%\ldots
%\end{itemize}

%%%%%%%%%%%%%%%%%%%%%%%%%%%%%%%%%%%%%%%%%%%%%%%%%%%%%%%%%%%%%%%%%%%%%%%%%%%%%%%%
\subsection{Revision History}

%%%%%%%%%%%%%%%%%%%%%%%%%%%%%%%%%%%%%%%%
\paragraph{v2.0:} 2018/12/30

\begin{itemize}
\item
immediate forward processing
\item
added |\childdocby| mechanism
\item
manual restructured
\end{itemize}

%%%%%%%%%%%%%%%%%%%%%%%%%%%%%%%%%%%%%%%%
\paragraph{v1.6:} 2018/01/17

\begin{itemize}
\item
application for development of include files
\item
corrections to manual
\end{itemize}

%%%%%%%%%%%%%%%%%%%%%%%%%%%%%%%%%%%%%%%%
\paragraph{v1.5:} 2017/05/21

\begin{itemize}
\item
more complete structuring introduced
\item
|\childdocof| introduced
\item
|\childdoc| renamed to |\childdocmain|
\item
|\childredirect| renamed to |\childdocforward| and |\childdocforwardprefix|
and functionality expanded
\end{itemize}

%%%%%%%%%%%%%%%%%%%%%%%%%%%%%%%%%%%%%%%%
\paragraph{v1.0:} 2017/04/27

\begin{itemize}
\item
manual and install package
\item
first version published on CTAN
\end{itemize}

%%%%%%%%%%%%%%%%%%%%%%%%%%%%%%%%%%%%%%%%
\paragraph{v0.6:} 2017/04/26

\begin{itemize}
\item
redirection mechanism added
\end{itemize}

%%%%%%%%%%%%%%%%%%%%%%%%%%%%%%%%%%%%%%%%
\paragraph{v0.5:} 2017/04/26

\begin{itemize}
\item
functionality in definition file
\end{itemize}


%%%%%%%%%%%%%%%%%%%%%%%%%%%%%%%%%%%%%%%%%%%%%%%%%%%%%%%%%%%%%%%%%%%%%%%%%%%%%%%%
%%%%%%%%%%%%%%%%%%%%%%%%%%%%%%%%%%%%%%%%%%%%%%%%%%%%%%%%%%%%%%%%%%%%%%%%%%%%%%%%
%%%%%%%%%%%%%%%%%%%%%%%%%%%%%%%%%%%%%%%%%%%%%%%%%%%%%%%%%%%%%%%%%%%%%%%%%%%%%%%%
\appendix

\settowidth\MacroIndent{\rmfamily\scriptsize 000\ }

 \DocInput{childdoc.dtx}

\end{document}
%</driver>
% \fi
%
% %%%%%%%%%%%%%%%%%%%%%%%%%%%%%%%%%%%%%%%%%%%%%%%%%%%%%%%%%%%%%%%%%%%%%%%%%%%%%%
% %%%%%%%%%%%%%%%%%%%%%%%%%%%%%%%%%%%%%%%%%%%%%%%%%%%%%%%%%%%%%%%%%%%%%%%%%%%%%%
% \section{Sample}
%\iffalse
%<*samplemain>
%\fi
%
% The following presents a sample document
% with two chapters, two parts, a title page,
% a compile flag as well as three forwarding files to set the flag.
% It consists of eight |.tex| files:
% \begin{center}
% \begin{tabular}{ll}
% |cdocsamp.tex|&main file\\
% |cdocsch1.tex|&include file for chapter 1\\
% |cdocsch2.tex|&include file for chapter 2\\
% |cdocspt3.tex|&include file for part 3\\
% |cdocspt4.tex|&include file for part 4\\
% |cdocsdrf.tex|&forwarding file for main file in draft mode\\
% |cdocsfi1.tex|&forwarding file for final version of chapter 1\\
% |cdocsfi2.tex|&forwarding file for final version of chapter 2\\
% \end{tabular}
% \end{center}
% Each of the eight files can be compiled directly by the \LaTeX{} compiler.
%
% %%%%%%%%%%%%%%%%%%%%%%%%%%%%%%%%%%%%%%
% \paragraph{Main File.}
%
% The main file is called |cdocsamp.tex|.
%
% Load the \textsf{childdoc} definitions and
% declare the filename for the main document:
%    \begin{macrocode}
\input{childdoc.def}
\childdocmain{}
%    \end{macrocode}

% Optional override for |\version| flag:
%    \begin{macrocode}
%%\ifchilddoc\else\providecommand{\version}{draft}\fi
%    \end{macrocode}

% Define the default values for the |\version| flag
% (|final| for the main file and |draft| for childs):
%    \begin{macrocode}
\ifchilddoc
\providecommand{\version}{draft}
\else
\providecommand{\version}{final}
\fi
%    \end{macrocode}

% Load the standard document class:
%    \begin{macrocode}
\documentclass[12pt]{article}
%    \end{macrocode}

% Start the document body:
%    \begin{macrocode}
\begin{document}
%    \end{macrocode}

% Declare a title page.
% Print title, part of document being processed and version flag:
%    \begin{macrocode}
\addtocounter{page}{-1}
\begin{center}
{\LARGE\bfseries{}childdoc example\par}
\vspace{1cm}
\ifchilddoc
\ifchilddocmanual part\else chapter\fi:
`\childdocname' of `\childdocjob'\par
\else
main document: `\childdocjob'\par
\fi
version: \version\par
\end{center}
\newpage
%    \end{macrocode}

% Manually include selected file,
% otherwise process as usual:
%    \begin{macrocode}
\ifchilddocmanual
\section*{part `\childdocname'}
\input{\childdocname}
\else
%    \end{macrocode}

% Include the two chapters:
%    \begin{macrocode}
\include{cdocsch1}
\include{cdocsch2}
%    \end{macrocode}

% Include the two parts unless only chapters should be displayed:
%    \begin{macrocode}
\ifchilddoc\else
\section{part three}
\input{cdocspt3}
\section{part four}
\input{cdocspt4}
\fi
%    \end{macrocode}

% Process as usual until here:
%    \begin{macrocode}
\fi
%    \end{macrocode}

% End of document body:
%    \begin{macrocode}
\end{document}
%    \end{macrocode}
%\iffalse
%</samplemain>
%\fi
%
% %%%%%%%%%%%%%%%%%%%%%%%%%%%%%%%%%%%%%%
% \paragraph{Chapter Include Files.}
%
% The include files are called |cdocsch1.tex| and |cdocsch2.tex|.
%
%\iffalse
%<*samplechap1|samplechap2>
%\fi

% Optional override for |\version| flag:
%    \begin{macrocode}
%%\providecommand{\version}{final}
%    \end{macrocode}

% Include the main document:
%    \begin{macrocode}
\input{childdoc.def}
\childdocof{cdocsamp}
%    \end{macrocode}

%\iffalse
%</samplechap1|samplechap2>
%\fi
%
%\iffalse
%<*samplechap1>
%\fi
% Some text for chapter 1:
%    \begin{macrocode}
\section{one}
some text in chapter one
%    \end{macrocode}

%\iffalse
%</samplechap1>
%\fi
% Some text for chapter 2:
%\iffalse
%<*samplechap2>
%\fi
%    \begin{macrocode}
\section{two}
more text in chapter two
%    \end{macrocode}

%\iffalse
%</samplechap2>
%\fi
%
% %%%%%%%%%%%%%%%%%%%%%%%%%%%%%%%%%%%%%%
% \paragraph{Part Include Files.}
%
% The include files are called |cdocspt3.tex| and |cdocspt4.tex|.
%
%\iffalse
%<*samplepart3|samplepart4>
%\fi

% Optional override for |\version| flag:
%    \begin{macrocode}
%%\providecommand{\version}{final}
%    \end{macrocode}

% Include the main document:
%    \begin{macrocode}
\input{childdoc.def}
\childdocby{cdocsamp}
%    \end{macrocode}

%\iffalse
%</samplepart3|samplepart4>
%\fi
%
%\iffalse
%<*samplepart3>
%\fi
% Some text for part 3:
%    \begin{macrocode}
some text in part three
%    \end{macrocode}

%\iffalse
%</samplepart3>
%\fi
% Some text for part 4:
%\iffalse
%<*samplepart4>
%\fi
%    \begin{macrocode}
more text in part four
%    \end{macrocode}

%\iffalse
%</samplepart4>
%\fi
%
% %%%%%%%%%%%%%%%%%%%%%%%%%%%%%%%%%%%%%%
% \paragraph{Forwarding for a Complete Draft.}
%
% The following forwarding file |cdocsdrf.tex|
% compiles the main document in draft mode:
%\iffalse
%<*sampledraft>
%\fi
%    \begin{macrocode}
\def\version{draft}
\input{childdoc.def}
\childdocforward{cdocsamp}
%    \end{macrocode}

%\iffalse
%</sampledraft>
%\fi
%
% %%%%%%%%%%%%%%%%%%%%%%%%%%%%%%%%%%%%%%
% \paragraph{Forwarding for Final Version of the Chapters.}
%
% The following forwarding files |cdocsfn1.tex| and |cdocsfn2.tex|
% (with identical content)
% compile the final versions of the child documents
% |cdocsch1.tex| and |cdocsch2.tex|, respectively:
%\iffalse
%<*samplefinal>
%\fi
%    \begin{macrocode}
\def\version{final}
\input{childdoc.def}
\childdocforwardprefix[cdocsamp]{cdocsfn}{cdocsch}
%    \end{macrocode}

%\iffalse
%</samplefinal>
%\fi
%
% %%%%%%%%%%%%%%%%%%%%%%%%%%%%%%%%%%%%%%
% \paragraph{Command Line Processing.}
%
% The following three command lines generate the output files
% |cdocscld|, |cdocscl1| and |cdocscl2|
% which should be identical to
% |cdocsdrf|, |cdocsch1| and |cdocsfn2|, respectively:
% \begin{center}
% \begin{tabular}{l}
% |latex -jobname cdocscld \|\\
% |  "\def\version{draft}\input{childdoc.def}\childdocforward{cdocsamp}"|\\
% |latex -jobname cdocscl1 \|\\
% |  "\input{childdoc.def}\childdocforward[cdocsamp]{cdocsch1}"|\\
% |latex -jobname cdocscl2 \|\\
% |  "\def\version{final}\input{childdoc.def}\childdocforward{cdocsch2}"|
% \end{tabular}
% \end{center}
% Note that the trailing backslash on each first line
% merely continues the input to the second line
% (for convenient cut ant paste).
% Furthermore, the command |latex| can be replaced by any
% of its alternative versions such as |pdflatex|.
%
% %%%%%%%%%%%%%%%%%%%%%%%%%%%%%%%%%%%%%%%%%%%%%%%%%%%%%%%%%%%%%%%%%%%%%%%%%%%%%%
% %%%%%%%%%%%%%%%%%%%%%%%%%%%%%%%%%%%%%%%%%%%%%%%%%%%%%%%%%%%%%%%%%%%%%%%%%%%%%%
% \section{Implementation}
%\iffalse
%<*package>
%\fi
%
% This section describes the definitions file |childdoc.def|.

% The definitions cannot be loaded using |\usepackage| or |\RequirePackage|
% which has a mechanism to prevent loading a style file more than once.
% When loading the definitions by means of |\input|
% multiple instances have to be prevented manually:
%\iffalse
%This code needs to be before the `\ProvidesFile' directive
%which is defined at the beginning of this file.
%Therefore it is also placed there and commented out here.
%</package>
%<*discard>
%\fi
%    \begin{macrocode}
\ifdefined\childdocmain\endinput\fi
%    \end{macrocode}
%\iffalse
%</discard>
%<*package>
%\fi
%
% \macro{\ifchilddoc}
% \macro{\ifchilddocmanual}
% The conditional |\ifchilddoc| tells whether a
% child (true) or main (false) document is being compiled.
% The conditional |\ifchilddocmanual| tells whether
% the |\includeonly| mechanism is used (false) or
% the selection of child files must be performed manually (true).
% The definitions initialise to false:
%    \begin{macrocode}
\newif\ifchilddoc
\newif\ifchilddocmanual
%    \end{macrocode}

% \macro{\childdocname}
% \macro{\childdocjob}
% The macro |\childdocname| stores the name of the main document
% to be compiled. The macro |\childdocjob| stores the name of
% the document on which the \LaTeX{} compiler was originally invoked.
% The content of |\jobname| cannot be compared
% to filenames specified in the source due to different catcodes.
% The following code rescans |\jobname|, stores the result
% in |\childdocname| and saves a copy in |\childdocjob|:
%    \begin{macrocode}
\edef\childdocname{\scantokens\expandafter{\jobname\noexpand}}
\let\childdocjob\childdocname
%    \end{macrocode}

% \macro{\childdocdisable}
% The macro |\childdocdisable| prevents the main file
% from being processed more than once.
% At this stage, the main document command |\childdocmain|
% is assumed to be called once again where it should do nothing.
% Any subsequent call to it should prevent
% a secondary processing of the main document
% It overwrites the forwarding commands
% |\childdocof| and |\childdocforward|
% with empty macros to prevent further inclusions of the main document:
%    \begin{macrocode}
\newcommand{\childdocdisable}
{
  \renewcommand{\childdocmain}[1]{\renewcommand{\childdocmain}[1]{\endinput}}
  \renewcommand{\childdocof}[1]{}
  \renewcommand{\childdocby}[2][]{}
  \renewcommand{\childdocforward}[2][]{}
  \renewcommand{\childdocdisable}{}
}
%    \end{macrocode}

% \macro{\childdocmain}
% The macro |\childdocmain| is to be called at the top of the main file
% with nothing or the main filename (without extension) as argument.
% First, it breaks loops.
% If the argument is not empty and does not match |\childdocname|
% (which is set by the first inclusion of |childdoc.def|),
% |\ifchilddoc| is set to true, |\includeonly| is applied to the child file
% and |\jobname| is set to the main file
% (for proper handling of |.aux| files):
%    \begin{macrocode}
\newcommand{\childdocmain}[1]
{
  \childdocdisable\childdocmain{}
  \if?#1?\else
    \begingroup
      \def\childdoctmp{#1}
      \ifx\childdoctmp\childdocname
        \def\childdoctmp{}
      \else
        \def\childdoctmp
        {
          \childdoctrue
          \includeonly{\childdocname}
          \def\childdocjob{#1}
          \def\jobname{#1}
        }
      \fi
      \expandafter
    \endgroup
    \childdoctmp
  \fi
}
%    \end{macrocode}

% \macro{\childdocof}
% The command |\childdocof| redirects
% compilation to the main file |#1|.
%    \begin{macrocode}
\newcommand{\childdocof}[1]
{
  \childdocdisable
  \childdoctrue
  \includeonly{\childdocname}
  \def\jobname{#1}
  \def\childdocjob{#1}
  \input{#1}
}
%    \end{macrocode}

% \macro{\childdocby}
% The command |\childdocby| ....
%    \begin{macrocode}
\newcommand{\childdocby}[2][]
{
  \childdocdisable
  \childdoctrue
  \childdocmanualtrue
  \if?#1?\else
    \def\jobname{#2}
  \fi
  \def\childdocjob{#2}
  \input{#2}
  \endinput
}
%    \end{macrocode}

% \macro{\childdocforward}
% The command |\childdocforward| redirects
% compilation to the main file or
% (if the optional argument is given) a child file.
% Parameters are set as if the main file
% or a child file starting with |\childdocof| was compiled.
% Then compilation is handed over to the main file:
%    \begin{macrocode}
\newcommand{\childdocforward}[2][]
{
  \begingroup
    \if?#1?
      \def\childdoctmp
      {
        \def\childdocname{#2}
        \def\childdocjob{#2}
        \def\jobname{#2}
        \input{#2}
        \endinput
      }
    \else
      \def\childdoctmp
      {
        \childdocdisable
        \def\childdocname{#2}
        \childdoctrue
        \includeonly{#2}
        \def\childdocjob{#1}
        \def\jobname{#1}
        \input{#1}
        \endinput
      }
    \fi
    \expandafter
  \endgroup
  \childdoctmp
}
%    \end{macrocode}

% \macro{\childdocforwardprefix}
% The command |\childdocforwardprefix| redirects
% compilation to the main or a child file by means of a pattern.
% The prefix |#1| in the current filename is replaced by |#2|
% and the suffix of the current filename is kept
% (it is assumed that the filename does not contain the substring `|~~~|'
% which is used as a delimiter).
% Compilation is handed over to the new file by |\childdocforward|:
%    \begin{macrocode}
\newcommand{\childdocforwardprefix}[3][]
{
  \begingroup
    \def\childdocextract #2##1~~~{\def\childdoctmp{\childdocforward[#1]{#3##1}}}
    \expandafter\childdocextract\childdocname~~~
    \expandafter
  \endgroup
  \childdoctmp
}
%    \end{macrocode}

% \macro{\childdoc}
% The deprecated macro |\childdoc| is a legacy version of |\childdocmain|:
%    \begin{macrocode}
\newcommand{\childdoc}{\childdocmain}
%    \end{macrocode}

% \macro{\childdocredirect}
% The deprecated macro |\childdocredirect| is a legacy version
% of |\childdocforward| and |\childdocforwardprefix|:
%    \begin{macrocode}
\newcommand{\childdocredirect}[2][]
{
  \begingroup
    \if?#1?
      \def\childdoctmp{\childdocforward{#2}}
    \else
      \def\childdoctmp{\childdocforwardprefix{#1}{#2}}
    \fi
    \expandafter
  \endgroup
  \childdoctmp
}
%    \end{macrocode}

%\iffalse
%</package>
%\fi
%
\endinput
|
and perform the replacements as outlined below.
Instead of |\childdocmain{|\textit{main}|}| add the following code
to the top of the main file:
%
\begin{center}
\begin{tabular}{l}
|\||ifdefined\childdocname\endinput\||fi\newif\ifchilddoc|\\
|\edef\childdocname{\scantokens\expandafter{\jobname\noexpand}}|\\
|\def\childdocmain{|\textit{main}|}\||ifx\childdocmain\childdocname\||else|\\
|\childdoctrue\includeonly{\childdocname}\let\jobname\childdocmain\||fi|\\
\end{tabular}
\end{center}
%
Instead of |\childdocof{|\textit{main}|}| just include the main file
at the top of each child file:
%
\begin{center}
|\input{|\textit{main}|}|
\end{center}
%
A simple redirection |\childdocforward{|\textit{dest}|}| is achieved by:
%
\begin{center}
|\def\jobname{|\textit{dest}|}\input{\jobname}|
\end{center}
%
The redirection with prefix
|\childdocforwardprefix[|\textit{prefix}|]{|\textit{dest}|}|
is accomplished by:
%
\begin{center}
\begin{tabular}{l}
|{\edef\jobname{\scantokens\expandafter{\jobname\noexpand}}|\\
|\def\redirectjob |\textit{prefix}|#1~~~{\gdef\jobname{|\textit{dest}|#1}}|\\
|\expandafter\redirectjob\jobname~~~}\input{\jobname}|
\end{tabular}
\end{center}

In an alternative approach,
child documents can be compiled by a specific command line
without additional code or specific definitions:
%
\begin{center}
|... -jobname "|\textit{target}|" "|[\textit{flags}]%
|\includeonly{|\textit{dest}|}\input{|\textit{main}|}"|
\end{center}
%

%%%%%%%%%%%%%%%%%%%%%%%%%%%%%%%%%%%%%%%%%%%%%%%%%%%%%%%%%%%%%%%%%%%%%%%%%%%%%%%%
%%%%%%%%%%%%%%%%%%%%%%%%%%%%%%%%%%%%%%%%%%%%%%%%%%%%%%%%%%%%%%%%%%%%%%%%%%%%%%%%
\section{Information}

%%%%%%%%%%%%%%%%%%%%%%%%%%%%%%%%%%%%%%%%%%%%%%%%%%%%%%%%%%%%%%%%%%%%%%%%%%%%%%%%
\subsection{Copyright}

Copyright \copyright{} 2017--2018 Niklas Beisert

This work may be distributed and/or modified under the
conditions of the \LaTeX{} Project Public License, either version 1.3
of this license or (at your option) any later version.
The latest version of this license is in
  \url{http://www.latex-project.org/lppl.txt}
and version 1.3 or later is part of all distributions of \LaTeX{}
version 2005/12/01 or later.

This work has the LPPL maintenance status `maintained'.

The Current Maintainer of this work is Niklas Beisert.

This work consists of the files |README.txt|, |childdoc.ins| and |childdoc.dtx|
as well as the derived files |childdoc.def|, |cdocsamp.tex|
with |cdocsch1.tex|, |cdocsch2.tex|, |cdocspt3.tex|, |cdocspt4.tex|,
|cdocsdrf.tex|, |cdocsfn1.tex|, |cdocsfn2.tex|
as well as |childdoc.pdf|.

%%%%%%%%%%%%%%%%%%%%%%%%%%%%%%%%%%%%%%%%%%%%%%%%%%%%%%%%%%%%%%%%%%%%%%%%%%%%%%%%
\subsection{Files and Installation}

The package consists of the files:
%
\begin{center}
\begin{tabular}{ll}
    |README.txt|   & readme file \\
    |childdoc.ins| & installation file \\
    |childdoc.dtx| & source file \\
    |childdoc.def| & definition file \\
    |cdocsamp.tex| & sample main file \\
    |cdocsch1.tex| & sample include file \\
    |cdocsch2.tex| & sample include file \\
    |cdocspt3.tex| & sample part file \\
    |cdocspt4.tex| & sample part file \\
    |cdocsdrf.tex| & sample redirection file \\
    |cdocsfn1.tex| & sample redirection file \\
    |cdocsfn2.tex| & sample redirection file \\
    |childdoc.pdf| & manual
\end{tabular}
\end{center}
%
The distribution consists of the files
|README.txt|, |childdoc.ins| and |childdoc.dtx|.
%
\begin{itemize}
\item
Run (pdf)\LaTeX{} on |childdoc.dtx|
to compile the manual |childdoc.pdf| (this file).
\item
Run \LaTeX{} on |childdoc.ins| to create the definitions file |childdoc.def|
and the sample |cdocsamp.tex| with include files
|cdocsch1.tex|, |cdocsch2.tex|, |cdocspt3.tex|, |cdocspt4.tex|,
|cdocsdrf.tex|, |cdocsfn1.tex|, |cdocsfn2.tex|.
Then copy the file |childdoc.def| to an appropriate directory of your \LaTeX{}
distribution, e.g.\ \textit{texmf-root}|/tex/latex/childdoc|.
\end{itemize}

%%%%%%%%%%%%%%%%%%%%%%%%%%%%%%%%%%%%%%%%%%%%%%%%%%%%%%%%%%%%%%%%%%%%%%%%%%%%%%%%
\subsection{Related CTAN Packages}

There are several other packages which offer a similar functionality:
%
\begin{itemize}
\item
The packages
\href{http://ctan.org/pkg/docmute}{\textsf{docmute}},
\href{http://ctan.org/pkg/includex}{\textsf{includex}} and
\href{http://ctan.org/pkg/standalone}{\textsf{standalone}}
provide commands to include only the document body of
a child file thus allowing both files to be compiled individually.
\item
The packages \href{http://ctan.org/pkg/subdocs}{\textsf{subdocs}}
and \href{http://ctan.org/pkg/subfiles}{\textsf{subfiles}}
provide structures in which the main and child documents can be
encapsulated and allowing them to be compiled individually.
The inclusion mechanism is different from the conventional |\include|.
\item
The package \href{http://ctan.org/pkg/combine}{\textsf{combine}}
is an elaborate solution to combine several documents into one.
\end{itemize}
%
See also the CTAN topic \href{http://ctan.org/topic/subdocs}{\textsf{subdocs}}
for further related packages.
The present package differs from the above solutions in that
a document structure constructed with the conventional |\include| mechanism
just needs two extra commands at the top of every file
such that all constituent files can be compiled individually.

%%%%%%%%%%%%%%%%%%%%%%%%%%%%%%%%%%%%%%%%%%%%%%%%%%%%%%%%%%%%%%%%%%%%%%%%%%%%%%%%
%\subsection{Feature Suggestions}
%
%The following is a list of features which may be useful for future
%versions of this package:
%%
%\begin{itemize}
%\item
%\ldots
%\end{itemize}

%%%%%%%%%%%%%%%%%%%%%%%%%%%%%%%%%%%%%%%%%%%%%%%%%%%%%%%%%%%%%%%%%%%%%%%%%%%%%%%%
\subsection{Revision History}

%%%%%%%%%%%%%%%%%%%%%%%%%%%%%%%%%%%%%%%%
\paragraph{v2.0:} 2018/12/30

\begin{itemize}
\item
immediate forward processing
\item
added |\childdocby| mechanism
\item
manual restructured
\end{itemize}

%%%%%%%%%%%%%%%%%%%%%%%%%%%%%%%%%%%%%%%%
\paragraph{v1.6:} 2018/01/17

\begin{itemize}
\item
application for development of include files
\item
corrections to manual
\end{itemize}

%%%%%%%%%%%%%%%%%%%%%%%%%%%%%%%%%%%%%%%%
\paragraph{v1.5:} 2017/05/21

\begin{itemize}
\item
more complete structuring introduced
\item
|\childdocof| introduced
\item
|\childdoc| renamed to |\childdocmain|
\item
|\childredirect| renamed to |\childdocforward| and |\childdocforwardprefix|
and functionality expanded
\end{itemize}

%%%%%%%%%%%%%%%%%%%%%%%%%%%%%%%%%%%%%%%%
\paragraph{v1.0:} 2017/04/27

\begin{itemize}
\item
manual and install package
\item
first version published on CTAN
\end{itemize}

%%%%%%%%%%%%%%%%%%%%%%%%%%%%%%%%%%%%%%%%
\paragraph{v0.6:} 2017/04/26

\begin{itemize}
\item
redirection mechanism added
\end{itemize}

%%%%%%%%%%%%%%%%%%%%%%%%%%%%%%%%%%%%%%%%
\paragraph{v0.5:} 2017/04/26

\begin{itemize}
\item
functionality in definition file
\end{itemize}


%%%%%%%%%%%%%%%%%%%%%%%%%%%%%%%%%%%%%%%%%%%%%%%%%%%%%%%%%%%%%%%%%%%%%%%%%%%%%%%%
%%%%%%%%%%%%%%%%%%%%%%%%%%%%%%%%%%%%%%%%%%%%%%%%%%%%%%%%%%%%%%%%%%%%%%%%%%%%%%%%
%%%%%%%%%%%%%%%%%%%%%%%%%%%%%%%%%%%%%%%%%%%%%%%%%%%%%%%%%%%%%%%%%%%%%%%%%%%%%%%%
\appendix

\settowidth\MacroIndent{\rmfamily\scriptsize 000\ }

 \DocInput{childdoc.dtx}

\end{document}
%</driver>
% \fi
%
% %%%%%%%%%%%%%%%%%%%%%%%%%%%%%%%%%%%%%%%%%%%%%%%%%%%%%%%%%%%%%%%%%%%%%%%%%%%%%%
% %%%%%%%%%%%%%%%%%%%%%%%%%%%%%%%%%%%%%%%%%%%%%%%%%%%%%%%%%%%%%%%%%%%%%%%%%%%%%%
% \section{Sample}
%\iffalse
%<*samplemain>
%\fi
%
% The following presents a sample document
% with two chapters, two parts, a title page,
% a compile flag as well as three forwarding files to set the flag.
% It consists of eight |.tex| files:
% \begin{center}
% \begin{tabular}{ll}
% |cdocsamp.tex|&main file\\
% |cdocsch1.tex|&include file for chapter 1\\
% |cdocsch2.tex|&include file for chapter 2\\
% |cdocspt3.tex|&include file for part 3\\
% |cdocspt4.tex|&include file for part 4\\
% |cdocsdrf.tex|&forwarding file for main file in draft mode\\
% |cdocsfi1.tex|&forwarding file for final version of chapter 1\\
% |cdocsfi2.tex|&forwarding file for final version of chapter 2\\
% \end{tabular}
% \end{center}
% Each of the eight files can be compiled directly by the \LaTeX{} compiler.
%
% %%%%%%%%%%%%%%%%%%%%%%%%%%%%%%%%%%%%%%
% \paragraph{Main File.}
%
% The main file is called |cdocsamp.tex|.
%
% Load the \textsf{childdoc} definitions and
% declare the filename for the main document:
%    \begin{macrocode}
% \iffalse
%
% childdoc.dtx Copyright (C) 2017-2018 Niklas Beisert
%
% This work may be distributed and/or modified under the
% conditions of the LaTeX Project Public License, either version 1.3
% of this license or (at your option) any later version.
% The latest version of this license is in
%   http://www.latex-project.org/lppl.txt
% and version 1.3 or later is part of all distributions of LaTeX
% version 2005/12/01 or later.
%
% This work has the LPPL maintenance status `maintained'.
%
% The Current Maintainer of this work is Niklas Beisert.
%
% This work consists of the files childdoc.dtx and childdoc.ins
% and the derived files childdoc.def and cdocsamp.tex with
% cdocsch1.tex, cdocsch2.tex, cdocsdrf.tex, cdocsfn1.tex, cdocsfn2.tex.
%
%<package>\ifdefined\childdocmain\endinput\fi
%<package>\ProvidesFile{childdoc.def}[2018/12/30 v2.0 child document driver]
%<samplemain>\ProvidesFile{cdocsamp.tex}[2018/12/30 v2.0 sample for childdoc]
%<*driver>
%\ProvidesFile{childdoc.drv}[2018/12/30 v2.0 childdoc reference manual file]
\PassOptionsToClass{10pt,a4paper}{article}
\documentclass{ltxdoc}

\usepackage[margin=35mm]{geometry}
\usepackage{hyperref}
\usepackage{hyperxmp}
\usepackage[usenames]{color}

\hypersetup{colorlinks=true}
\hypersetup{pdfstartview=FitH}
\hypersetup{pdfpagemode=UseNone}
\hypersetup{pdfsource={}}
\hypersetup{pdflang={en-UK}}
\hypersetup{pdfcopyright={Copyright 2017-2018 Niklas Beisert.
  This work may be distributed and/or modified under the
  conditions of the LaTeX Project Public License, either version 1.3
  of this license or (at your option) any later version.}}
\hypersetup{pdflicenseurl={http://www.latex-project.org/lppl.txt}}
\hypersetup{pdfcontactaddress={ETH Zurich, ITP, HIT K,
  Wolfgang-Pauli-Strasse 27}}
\hypersetup{pdfcontactpostcode={8093}}
\hypersetup{pdfcontactcity={Zurich}}
\hypersetup{pdfcontactcountry={Switzerland}}
\hypersetup{pdfcontactemail={nbeisert@itp.phys.ethz.ch}}
\hypersetup{pdfcontacturl={http://people.phys.ethz.ch/\xmptilde nbeisert/}}

\newcommand{\secref}[1]{\hyperref[#1]{section \ref*{#1}}}

\parskip1ex
\parindent0pt
\let\olditemize\itemize
\def\itemize{\olditemize\parskip0pt}

\begin{document}

\title{The \textsf{childdoc} Package}
\hypersetup{pdftitle={The childdoc Package}}
\author{Niklas Beisert\\[2ex]
  Institut f\"ur Theoretische Physik\\
  Eidgen\"ossische Technische Hochschule Z\"urich\\
  Wolfgang-Pauli-Strasse 27, 8093 Z\"urich, Switzerland\\[1ex]
  \href{mailto:nbeisert@itp.phys.ethz.ch}
  {\texttt{nbeisert@itp.phys.ethz.ch}}}
\hypersetup{pdfauthor={Niklas Beisert}}
\hypersetup{pdfsubject={Manual for the LaTeX2e Package childdoc}}
\date{30 December 2018, \textsf{v2.0}}
\maketitle

\begin{abstract}\noindent
\textsf{childdoc} is a \LaTeXe{} package
that enables the direct compilation
of document sections included by |\include|
to individual files.
\end{abstract}

\begingroup
\parskip0ex
\tableofcontents
\endgroup

%%%%%%%%%%%%%%%%%%%%%%%%%%%%%%%%%%%%%%%%%%%%%%%%%%%%%%%%%%%%%%%%%%%%%%%%%%%%%%%%
%%%%%%%%%%%%%%%%%%%%%%%%%%%%%%%%%%%%%%%%%%%%%%%%%%%%%%%%%%%%%%%%%%%%%%%%%%%%%%%%
\section{Introduction}

\LaTeX{} provides a mechanism to structure a large document (such as a book)
into a main file and several child files (containing the chapters)
using the |\include| command.
This mechanism is beneficial for documents
which span hundreds of pages in order to
make the source file(s) more manageable.
Moreover, compilation can be restricted to
selected child files by means of the |\includeonly| command.
The latter feature can be used to reduce the compilation time while editing
(this was significantly more useful in the earlier days of \LaTeX{})
or to generate a smaller document which is easier to navigate.
Another application of |\includeonly| is to generate
documents consisting of selected parts of the complete document.

However, there are a few drawbacks of the plain |\include| mechanism:
\begin{itemize}
\item
The child files cannot be compiled on their own,
they can only be compiled via the main file.
A naive editing environment
(such as a text editor with an option
to have the current file processed by \LaTeX)
may require one to switch to the main file before compiling;
attempting to compile the child file produces errors.
\item
The main file must be modified (each time)
to adjust the |\includeonly| command
to the present needs. This easily leaves the main file in a messy state.
\item
The generated document will always carry the filename
of the main document. This is inconvenient if
several child files are to be compiled and
to be kept for distribution.
\end{itemize}

The present package provides a simple interface
to make child files individually compilable by \LaTeX{}.
Compiling a child file then has the same effect as compiling
the main file with an |\includeonly| command
to select the appropriate child.
Moreover the generated document will carry the name of the child
rather than the main file.
This resolves all three above issues.

This feature is meant to make the editing of books,
thesis documents and lecture notes somewhat more convenient.
However, the package can also be used efficiently for
composing a series of documents (such as exercise sheets)
which are typically distributed individually.
It then assists the author in generating the individual documents
(potentially in different versions)
as well as a document containing the collected series.
Another application is in developing style files
or other kinds of included material
where compilation of the style file could redirect
to a sample or test file.

%%%%%%%%%%%%%%%%%%%%%%%%%%%%%%%%%%%%%%%%%%%%%%%%%%%%%%%%%%%%%%%%%%%%%%%%%%%%%%%%
%%%%%%%%%%%%%%%%%%%%%%%%%%%%%%%%%%%%%%%%%%%%%%%%%%%%%%%%%%%%%%%%%%%%%%%%%%%%%%%%
\section{Usage}

First of all, the package \textsf{childdoc} is \emph{not} a standard
\LaTeXe{} |.sty| style file! Therefore it needs to be invoked in
a non-standard way.

%%%%%%%%%%%%%%%%%%%%%%%%%%%%%%%%%%%%%%%%%%%%%%%%%%%%%%%%%%%%%%%%%%%%%%%%%%%%%%%%
\subsection{Included Files}
\label{sec:include}

%%%%%%%%%%%%%%%%%%%%%%%%%%%%%%%%%%%%%%%%
\DescribeMacro{\childdocmain}
To use the package, add the commands
\begin{center}
\begin{tabular}{l}
|\input{childdoc.def}|\\
|\childdocmain{}|\\
\end{tabular}
\end{center}
at the very top of the main \LaTeX{} file,
in particular \emph{before} the |\documentclass| statement!
The argument of |\childdocmain| should be left empty
(but it must be present).

%%%%%%%%%%%%%%%%%%%%%%%%%%%%%%%%%%%%%%%%
\DescribeMacro{\childdocof}
Furthermore, add the commands
\begin{center}
\begin{tabular}{l}
|\input{childdoc.def}|\\
|\childdocof{|\textit{main}|}|\\
\end{tabular}
\end{center}
at the top of every child file \textit{child}
which is included by |\include{|\textit{child}|}|
from within the main file
(or at least for those files to be compiled individually).
The argument \textit{main} must be the filename of the main file.

There are a couple of
considerations in setting up the main and child documents:

%%%%%%%%%%%%%%%%%%%%%%%%%%%%%%%%%%%%%%%%
\paragraph{Restrictions.}

Please note the following restrictions:
\begin{itemize}
\item
|\childdocmain| must be called with one argument \textit{main}
to ensure compatibility with earlier version of the package.
It must either be empty (|\childdocmain{}|)
or precisely match the filename of the main file in which it is specified.
See \secref{sec:detection} for further information.
\item
The filename \textit{main} must be specified without the |.tex| extension.
\item
The filename \textit{main} is case sensitive
(even in case-insensitive file systems)
due to internal string comparison.
\item
The argument \textit{main} should be fully expanded, it cannot be a macro.
\item
Subdirectories and special characters should be avoided in filenames.
\item
The command |\childdocmain{|\textit{main}|}| must be followed by a whitespace.
It should not be followed immediately by another command
or by a comment mark `|%|'.
This is because the \TeX{} parser reads the token immediately following
the argument of |\childdocmain| and puts it
at the beginning of every child section;
however, a white\-space is ignored.
\end{itemize}

%%%%%%%%%%%%%%%%%%%%%%%%%%%%%%%%%%%%%%%%
\paragraph{Content of Main File.}

It is advisable to place all content in the child files included by |\include|.
Any output contained in the main file will appear in all child documents
unless suppressed manually;
it cannot be suppressed automatically by the |\includeonly| directive
and thus should normally be avoided.
A method to include some content in the main file
by means of conditional processing is described in \secref{sec:conditional}.

%%%%%%%%%%%%%%%%%%%%%%%%%%%%%%%%%%%%%%%%
\paragraph{Page Numbering.}

When only a part of the document is compiled,
the appropriate numbering of pages
(as well as other status parameters)
is determined from the |.aux| files.
The latter contain information from previous passes.
However this information needs to propagate through
all intermediate child documents.
Therefore the page numbering in child documents may well
be inconsistent until the complete document is compiled at least once.

A useful (if unconventional) way to always ensure a consistent
page numbering is to restart the numbering in each child document
and denote the pages by `\textit{child}|.|\textit{page}'
where \textit{child} represents the chapter/section number of the child file.
This can be achieved by the command
|\numberwithin{page}{|\textit{child}|}|
of the \textsf{amsmath} package
where \textit{child} can be |chapter| or |section|
depending on the chosen structuring.
Alternatively, one can modify the macro |\thepage| appropriately
and reset the counter |page| at the start of each child file.

%%%%%%%%%%%%%%%%%%%%%%%%%%%%%%%%%%%%%%%%%%%%%%%%%%%%%%%%%%%%%%%%%%%%%%%%%%%%%%%%
\subsection{Conditional Processing}
\label{sec:conditional}

The package provides a mechanism to compile different versions
of a document. To customise the versions further some conditional processing
can come in handy to distinguish which version is being compiled.
The package provides two macros to describe the compilation context:

%%%%%%%%%%%%%%%%%%%%%%%%%%%%%%%%%%%%%%%%
\DescribeMacro{\ifchilddoc}
The conditional |\ifchilddoc| distinguishes between the compilation of
child documents and the main document:
%
\begin{center}
|\ifchilddoc |\textit{child-code}| |[|\||else |\textit{main-code}]| \||fi|
\end{center}

%%%%%%%%%%%%%%%%%%%%%%%%%%%%%%%%%%%%%%%%
\DescribeMacro{\childdocname}
\DescribeMacro{\childdocjob}
The macro |\childdocname| contains the filename (without extension)
of the main or child file being processed.
Note that |\childdocjob| will always contain the name of the main file.

%%%%%%%%%%%%%%%%%%%%%%%%%%%%%%%%%%%%%%%%
\paragraph{Title Page.}

Conditional processing can be used to include a title or banner page
in the main document when proper precautions are taken.
Importantly, the code in the main file should ensure that the page counter
(as well as other status parameters which are stored in the |.aux| files)
takes the same value after the conditional processing.
Otherwise the page numbers may take divergent values
depending on which part is compiled.

For example, a title page could be declared by:
%
\begin{center}
\begin{tabular}{l}
|\ifchilddoc\||else|\\
|\addtocounter{page}{-1}|\\
\textit{code for title page}\\
|\newpage|\\
|\||fi|
\end{tabular}
\end{center}
%
A banner page for the child documents can be generated by:
%
\begin{center}
\begin{tabular}{l}
|\ifchilddoc|\\
|\addtocounter{page}{-1}|\\
\textit{code for banner page}\\
|\newpage|\\
|\||fi|
\end{tabular}
\end{center}
%
Here one could write a message such as:
\begin{center}
|This is the part \childdocname{} of \childdocjob{}.|
\end{center}

%%%%%%%%%%%%%%%%%%%%%%%%%%%%%%%%%%%%%%%%%%%%%%%%%%%%%%%%%%%%%%%%%%%%%%%%%%%%%%%%
\subsection{Flags}
\label{sec:flags}

The package makes it easy to generate different versions
of the main or child documents.
To this end compilation flags can be defined
and assigned different default values.
They will be particularly useful in conjunction
with the forwarding mechanism described in \secref{sec:forward}.

For example, it may be useful to have a flag |\version|
which can be set to |draft| or |final|.
The document source will contain some conditional code
depending on the value of |\version|.
Suppose further, the flag should default to |final| for the main file
and to |draft| for child files
which is a natural assignment for editing the document.
This is achieved by placing the following code
in the preamble of the main document
(below the |\childdocmain| directive):
%
\begin{center}
\begin{tabular}{l}
|\ifchilddoc|\\
|\providecommand{\version}{draft}|\\
|\||else|\\
|\providecommand{\version}{final}|\\
|\||fi|
\end{tabular}
\end{center}
%
The definition by |\providecommand| makes sure
that previous definitions are not overwritten.
Further statements |\providecommand{\version}{...}|
can thus be added before the above code to override it.

For the main file, one might add a line
(between |\childdocmain| and the above block)
%
\begin{center}
|%\ifchilddoc\||else\providecommand{\version}{draft}\||fi|
\end{center}
%
which can be uncommented to produce a draft version.
Likewise one can add a line to the very top of a child file
(above the |\childdocof{|\textit{main}|}| directive)
%
\begin{center}
|%\providecommand{\version}{final}|
\end{center}
%
which can be uncommented to produce the final version of this child document.

%%%%%%%%%%%%%%%%%%%%%%%%%%%%%%%%%%%%%%%%%%%%%%%%%%%%%%%%%%%%%%%%%%%%%%%%%%%%%%%%
\subsection{Forwarding}
\label{sec:forward}

Different versions of the main or child documents
using compilation flags as described in \secref{sec:flags}
can be (permanently) stored in different files
for convenient compilation, viewing and distribution.
To this end, the package defines a command
to pass on compilation to a different file:

%%%%%%%%%%%%%%%%%%%%%%%%%%%%%%%%%%%%%%%%
\DescribeMacro{\childdocforward}
The command |\childdocforward| redirects processing to
another source file:
%
\begin{center}
\begin{tabular}{l}
|\input{childdoc.def}|\\
|\childdocforward[|\textit{main}|]{|\textit{dest}|}|\\
\end{tabular}
\end{center}
%
The argument \textit{dest} is the destination file
(without extension).
It should be the main file or one of the child files.
Note that further \textsf{childdoc} directives
such as |\childdocof| and |\childdocforward|
in the indicated file will be processed in this form.
The optional argument \textit{main}
passes on directly to the main file \textit{main}
while pretending to compile the child \textit{dest}.
This form behaves as if \textit{dest}
issues |\childdocof{|\textit{main}|}| right away,
and no further \textsf{childdoc} directives will be processed.

%%%%%%%%%%%%%%%%%%%%%%%%%%%%%%%%%%%%%%%%
\DescribeMacro{\...prefix}
In the alternative form |\childdocforwardprefix|,
%
\begin{center}
\begin{tabular}{l}
|\input{childdoc.def}|\\
|\childdocforwardprefix[|\textit{main}|]{|\textit{prefix}|}{|\textit{dest}|}|
\end{tabular}
\end{center}
%
the destination file is determined by a pattern
depending on the current file:
To make this work, the current file must be called
`{\textit{prefix}\hspace{0.2em}\textit{suffix}}'
with \textit{prefix} matching precisely the argument.
Processing is then passed on to the file
`{\textit{dest}\hspace{0.2em}\textit{suffix}}'.
Surely, the same effect is achieved by
directly specifying the
argument `{\textit{dest}\hspace{0.2em}\textit{suffix}}'
in the first form.
However, that requires to set up a different file
for each child. With the alternative form of the command
all these files can have exactly the same content
which simplifies setting them up and maintaining them.

For example, the following file |draft.tex|
with a compilation flag |\version| as described in \secref{sec:flags}
compiles the main document as a draft:
%
\begin{center}
\begin{tabular}{l}
|\def\version{draft}|\\
|\input{childdoc.def}|\\
|\childdocforward{|\textit{main}|}|
\end{tabular}
\end{center}
%
Likewise, the following files |final|\textit{nn}|.tex|
compile the final version of the child document
|child|\textit{nn}|.tex|:
%
\begin{center}
\begin{tabular}{l}
|\def\version{final}|\\
|\input{childdoc.def}|\\
|\childdocforwardprefix{final}{child}|
\end{tabular}
\end{center}
%

Note that when several versions of a main file and/or of each child file
are to be generated, it may be convenient to set up a |Makefile| or
shell script to automatise the process.

%%%%%%%%%%%%%%%%%%%%%%%%%%%%%%%%%%%%%%%%%%%%%%%%%%%%%%%%%%%%%%%%%%%%%%%%%%%%%%%%
\subsection{Command Line Processing}
\label{sec:commandline}

The effect of redirection files can also be achieved by invoking
the \LaTeX{} compiler with a more elaborate command line.
Most conveniently this should be done as part
of a shell script or a |Makefile|.

When using \textsf{childdoc} in the main file, the following
command lines effectively perform a redirection
(note that depending on the shell being used,
backslashes may have to be doubled: `|\|' $\to$ `|\\|'):
%
\begin{center}
|... -jobname "|\textit{target}|" |\\|"|[\textit{flags}]%
|\input{childdoc.def}\childdocforward[|\textit{main}|]{|\textit{dest}|}"|
\end{center}
%
Here \textit{target} is the name of the output file,
\textit{main} is the name of the main file
and \textit{dest} is the name of the main or child file to be processed
(all filenames without extensions).
The optional argument \textit{main} can be omitted
if \textit{main} matches \textit{dest}.
Optionally, compilation \textit{flags} can be defined via |\def| commands.
This command line makes the \TeX{} engine believe
it is compiling the file \textit{target}
whose content is specified as the latter parameter.
The provided code then forwards the processing to
\textit{main} or \textit{dest} as described in \secref{sec:forward}.

%%%%%%%%%%%%%%%%%%%%%%%%%%%%%%%%%%%%%%%%%%%%%%%%%%%%%%%%%%%%%%%%%%%%%%%%%%%%%%%%
\subsection{Include by Input}
\label{sec:input}

Including child documents by |\include| has some restrictions by design.
Most notably, the content of a child document always occupies
its own set of pages; pages cannot be shared between child documents.
Usually, this behaviour makes perfect sense
because each child document contain an essential part of the document.
However, in some situations it may be desirable to compose
a document from a collection of parts
without having mandatory page breaks between then.
For this case, the package
provides a mechanism to include parts
by |\input| which can also be processed individually.
However, by construction this mechanism
requires manual handling of the content to be output.

%%%%%%%%%%%%%%%%%%%%%%%%%%%%%%%%%%%%%%%%
\DescribeMacro{\ifchilddocmanual}
The main file should be prepared as usual, see \secref{sec:include}.
However, the document body must make a distinction
between processing of an individual part and of the main document, e.g.:
%
\begin{center}
\begin{tabular}{l}
|\ifchilddocmanual|\\
|\input{\childdocname}|\\
|\||else|\\
\textit{document body with }|\input{|\textit{part}|}|\\
|\||fi|
\end{tabular}
\end{center}
%
The conditional |\ifchilddocmanual| is true whenever
a part to be included by |\input| is being compiled,
and the name of the part is stored in |\childdocname|.

%%%%%%%%%%%%%%%%%%%%%%%%%%%%%%%%%%%%%%%%
\DescribeMacro{\childdocby}
Each part to be included by |\input| should start with:
%
\begin{center}
\begin{tabular}{l}
|\input{childdoc.def}|\\
|\childdocby{|\textit{main}|}|\\
\end{tabular}
\end{center}
%
The directive |\childdocby| is similar to |\childdocof|
described in \secref{sec:include},
but the subsequent selection of content must be done manually.
To that end, both |\ifchilddoc| and |\ifchilddocmanual|
will be true upon processing of a part,
and the name of the part is stored in |\childdocname|.
Note that |\jobname| will be set to the filename of the current part
so that each part receives an individual |.aux| file
that does not interfere with the |.aux| file(s) of the main document.
This behaviour can be altered by the alternative form
|\childdocby[*]{|\textit{main}|}| (with a non-empty optional argument)
which uses the |.aux| file of the main document
by setting |\jobname| to \textit{main}.

%%%%%%%%%%%%%%%%%%%%%%%%%%%%%%%%%%%%%%%%%%%%%%%%%%%%%%%%%%%%%%%%%%%%%%%%%%%%%%%%
\subsection{Driver Development}
\label{sec:driver}

The \textsf{childdoc} mechanism can also be use for the development
of definition files such as \LaTeX{} styles or classes.
This case differs from the above setup with multiple parts
included by |\include| in that no |\includeonly| should be invoked.
This can be achieved by starting the include file
(before |\ProvidesPackage|) with:
%
\begin{center}
\begin{tabular}{l}
|\input{childdoc.def}|\\
|\childdocforward{|\textit{main}|}|\\
\end{tabular}
\end{center}
%
or alternatively with:
%
\begin{center}
\begin{tabular}{l}
|\input{childdoc.def}|\\
|\childdocby{|\textit{main}|}|\\
\end{tabular}
\end{center}
%
Both forms have slightly different effects as described above.
The main file is prepared as usual, see \secref{sec:include}.

%%%%%%%%%%%%%%%%%%%%%%%%%%%%%%%%%%%%%%%%%%%%%%%%%%%%%%%%%%%%%%%%%%%%%%%%%%%%%%%%
\subsection{Legacy Detection}
\label{sec:detection}

The directive |\childdocmain| in the main file can detect
whether the complete document or merely a child is to be compiled
even without using the directive |\childdocof|.
This method is deprecated because it is less robust
and there is no compelling reason to use it;
it is merely provided for backward compatibility
and it may be removed in future versions.

If the detection mechanism is to be used,
it is mandatory to correctly specify
the filename of the main file as the argument of |\childdocmain|:
%
\begin{center}
\begin{tabular}{l}
|\input{childdoc.def}|\\
|\childdocmain{|\textit{main}|}|\\
\end{tabular}
\end{center}
%
If |\jobname| does not match the argument \textit{main} of |\childdocmain|,
it is assumed that |\jobname| points to the child file to be compiled.
When using |\childdocmain| with the main file specified as argument,
it suffices to start a child file
with just |\input{|\textit{main}|}|
without loading of the package and using |\childdocof|.
If instead all processing is done
with the appropriate \textsf{childdoc} directives,
the argument of \textit{main} of |\childdocmain| can be empty.

An alternative version of the command line processing described
in \secref{sec:commandline} using the detection mechanism reads:
%
\begin{center}
|... -jobname "|\textit{target}|" "|[\textit{flags}]%
[|\def\jobname{|\textit{dest}|}|]|\input{|\textit{main}|}"|
\end{center}

%%%%%%%%%%%%%%%%%%%%%%%%%%%%%%%%%%%%%%%%%%%%%%%%%%%%%%%%%%%%%%%%%%%%%%%%%%%%%%%%
\subsection{Manual Code}
\label{sec:manual}

In case one cannot be certain whether the definitions file |childdoc.def|
is installed on the target \TeX{} distribution
and one prefers not to ship it,
it is conceivable to paste a few relevant commands into the sources.

To that end, drop all statements |\input{childdoc.def}|
and perform the replacements as outlined below.
Instead of |\childdocmain{|\textit{main}|}| add the following code
to the top of the main file:
%
\begin{center}
\begin{tabular}{l}
|\||ifdefined\childdocname\endinput\||fi\newif\ifchilddoc|\\
|\edef\childdocname{\scantokens\expandafter{\jobname\noexpand}}|\\
|\def\childdocmain{|\textit{main}|}\||ifx\childdocmain\childdocname\||else|\\
|\childdoctrue\includeonly{\childdocname}\let\jobname\childdocmain\||fi|\\
\end{tabular}
\end{center}
%
Instead of |\childdocof{|\textit{main}|}| just include the main file
at the top of each child file:
%
\begin{center}
|\input{|\textit{main}|}|
\end{center}
%
A simple redirection |\childdocforward{|\textit{dest}|}| is achieved by:
%
\begin{center}
|\def\jobname{|\textit{dest}|}\input{\jobname}|
\end{center}
%
The redirection with prefix
|\childdocforwardprefix[|\textit{prefix}|]{|\textit{dest}|}|
is accomplished by:
%
\begin{center}
\begin{tabular}{l}
|{\edef\jobname{\scantokens\expandafter{\jobname\noexpand}}|\\
|\def\redirectjob |\textit{prefix}|#1~~~{\gdef\jobname{|\textit{dest}|#1}}|\\
|\expandafter\redirectjob\jobname~~~}\input{\jobname}|
\end{tabular}
\end{center}

In an alternative approach,
child documents can be compiled by a specific command line
without additional code or specific definitions:
%
\begin{center}
|... -jobname "|\textit{target}|" "|[\textit{flags}]%
|\includeonly{|\textit{dest}|}\input{|\textit{main}|}"|
\end{center}
%

%%%%%%%%%%%%%%%%%%%%%%%%%%%%%%%%%%%%%%%%%%%%%%%%%%%%%%%%%%%%%%%%%%%%%%%%%%%%%%%%
%%%%%%%%%%%%%%%%%%%%%%%%%%%%%%%%%%%%%%%%%%%%%%%%%%%%%%%%%%%%%%%%%%%%%%%%%%%%%%%%
\section{Information}

%%%%%%%%%%%%%%%%%%%%%%%%%%%%%%%%%%%%%%%%%%%%%%%%%%%%%%%%%%%%%%%%%%%%%%%%%%%%%%%%
\subsection{Copyright}

Copyright \copyright{} 2017--2018 Niklas Beisert

This work may be distributed and/or modified under the
conditions of the \LaTeX{} Project Public License, either version 1.3
of this license or (at your option) any later version.
The latest version of this license is in
  \url{http://www.latex-project.org/lppl.txt}
and version 1.3 or later is part of all distributions of \LaTeX{}
version 2005/12/01 or later.

This work has the LPPL maintenance status `maintained'.

The Current Maintainer of this work is Niklas Beisert.

This work consists of the files |README.txt|, |childdoc.ins| and |childdoc.dtx|
as well as the derived files |childdoc.def|, |cdocsamp.tex|
with |cdocsch1.tex|, |cdocsch2.tex|, |cdocspt3.tex|, |cdocspt4.tex|,
|cdocsdrf.tex|, |cdocsfn1.tex|, |cdocsfn2.tex|
as well as |childdoc.pdf|.

%%%%%%%%%%%%%%%%%%%%%%%%%%%%%%%%%%%%%%%%%%%%%%%%%%%%%%%%%%%%%%%%%%%%%%%%%%%%%%%%
\subsection{Files and Installation}

The package consists of the files:
%
\begin{center}
\begin{tabular}{ll}
    |README.txt|   & readme file \\
    |childdoc.ins| & installation file \\
    |childdoc.dtx| & source file \\
    |childdoc.def| & definition file \\
    |cdocsamp.tex| & sample main file \\
    |cdocsch1.tex| & sample include file \\
    |cdocsch2.tex| & sample include file \\
    |cdocspt3.tex| & sample part file \\
    |cdocspt4.tex| & sample part file \\
    |cdocsdrf.tex| & sample redirection file \\
    |cdocsfn1.tex| & sample redirection file \\
    |cdocsfn2.tex| & sample redirection file \\
    |childdoc.pdf| & manual
\end{tabular}
\end{center}
%
The distribution consists of the files
|README.txt|, |childdoc.ins| and |childdoc.dtx|.
%
\begin{itemize}
\item
Run (pdf)\LaTeX{} on |childdoc.dtx|
to compile the manual |childdoc.pdf| (this file).
\item
Run \LaTeX{} on |childdoc.ins| to create the definitions file |childdoc.def|
and the sample |cdocsamp.tex| with include files
|cdocsch1.tex|, |cdocsch2.tex|, |cdocspt3.tex|, |cdocspt4.tex|,
|cdocsdrf.tex|, |cdocsfn1.tex|, |cdocsfn2.tex|.
Then copy the file |childdoc.def| to an appropriate directory of your \LaTeX{}
distribution, e.g.\ \textit{texmf-root}|/tex/latex/childdoc|.
\end{itemize}

%%%%%%%%%%%%%%%%%%%%%%%%%%%%%%%%%%%%%%%%%%%%%%%%%%%%%%%%%%%%%%%%%%%%%%%%%%%%%%%%
\subsection{Related CTAN Packages}

There are several other packages which offer a similar functionality:
%
\begin{itemize}
\item
The packages
\href{http://ctan.org/pkg/docmute}{\textsf{docmute}},
\href{http://ctan.org/pkg/includex}{\textsf{includex}} and
\href{http://ctan.org/pkg/standalone}{\textsf{standalone}}
provide commands to include only the document body of
a child file thus allowing both files to be compiled individually.
\item
The packages \href{http://ctan.org/pkg/subdocs}{\textsf{subdocs}}
and \href{http://ctan.org/pkg/subfiles}{\textsf{subfiles}}
provide structures in which the main and child documents can be
encapsulated and allowing them to be compiled individually.
The inclusion mechanism is different from the conventional |\include|.
\item
The package \href{http://ctan.org/pkg/combine}{\textsf{combine}}
is an elaborate solution to combine several documents into one.
\end{itemize}
%
See also the CTAN topic \href{http://ctan.org/topic/subdocs}{\textsf{subdocs}}
for further related packages.
The present package differs from the above solutions in that
a document structure constructed with the conventional |\include| mechanism
just needs two extra commands at the top of every file
such that all constituent files can be compiled individually.

%%%%%%%%%%%%%%%%%%%%%%%%%%%%%%%%%%%%%%%%%%%%%%%%%%%%%%%%%%%%%%%%%%%%%%%%%%%%%%%%
%\subsection{Feature Suggestions}
%
%The following is a list of features which may be useful for future
%versions of this package:
%%
%\begin{itemize}
%\item
%\ldots
%\end{itemize}

%%%%%%%%%%%%%%%%%%%%%%%%%%%%%%%%%%%%%%%%%%%%%%%%%%%%%%%%%%%%%%%%%%%%%%%%%%%%%%%%
\subsection{Revision History}

%%%%%%%%%%%%%%%%%%%%%%%%%%%%%%%%%%%%%%%%
\paragraph{v2.0:} 2018/12/30

\begin{itemize}
\item
immediate forward processing
\item
added |\childdocby| mechanism
\item
manual restructured
\end{itemize}

%%%%%%%%%%%%%%%%%%%%%%%%%%%%%%%%%%%%%%%%
\paragraph{v1.6:} 2018/01/17

\begin{itemize}
\item
application for development of include files
\item
corrections to manual
\end{itemize}

%%%%%%%%%%%%%%%%%%%%%%%%%%%%%%%%%%%%%%%%
\paragraph{v1.5:} 2017/05/21

\begin{itemize}
\item
more complete structuring introduced
\item
|\childdocof| introduced
\item
|\childdoc| renamed to |\childdocmain|
\item
|\childredirect| renamed to |\childdocforward| and |\childdocforwardprefix|
and functionality expanded
\end{itemize}

%%%%%%%%%%%%%%%%%%%%%%%%%%%%%%%%%%%%%%%%
\paragraph{v1.0:} 2017/04/27

\begin{itemize}
\item
manual and install package
\item
first version published on CTAN
\end{itemize}

%%%%%%%%%%%%%%%%%%%%%%%%%%%%%%%%%%%%%%%%
\paragraph{v0.6:} 2017/04/26

\begin{itemize}
\item
redirection mechanism added
\end{itemize}

%%%%%%%%%%%%%%%%%%%%%%%%%%%%%%%%%%%%%%%%
\paragraph{v0.5:} 2017/04/26

\begin{itemize}
\item
functionality in definition file
\end{itemize}


%%%%%%%%%%%%%%%%%%%%%%%%%%%%%%%%%%%%%%%%%%%%%%%%%%%%%%%%%%%%%%%%%%%%%%%%%%%%%%%%
%%%%%%%%%%%%%%%%%%%%%%%%%%%%%%%%%%%%%%%%%%%%%%%%%%%%%%%%%%%%%%%%%%%%%%%%%%%%%%%%
%%%%%%%%%%%%%%%%%%%%%%%%%%%%%%%%%%%%%%%%%%%%%%%%%%%%%%%%%%%%%%%%%%%%%%%%%%%%%%%%
\appendix

\settowidth\MacroIndent{\rmfamily\scriptsize 000\ }

 \DocInput{childdoc.dtx}

\end{document}
%</driver>
% \fi
%
% %%%%%%%%%%%%%%%%%%%%%%%%%%%%%%%%%%%%%%%%%%%%%%%%%%%%%%%%%%%%%%%%%%%%%%%%%%%%%%
% %%%%%%%%%%%%%%%%%%%%%%%%%%%%%%%%%%%%%%%%%%%%%%%%%%%%%%%%%%%%%%%%%%%%%%%%%%%%%%
% \section{Sample}
%\iffalse
%<*samplemain>
%\fi
%
% The following presents a sample document
% with two chapters, two parts, a title page,
% a compile flag as well as three forwarding files to set the flag.
% It consists of eight |.tex| files:
% \begin{center}
% \begin{tabular}{ll}
% |cdocsamp.tex|&main file\\
% |cdocsch1.tex|&include file for chapter 1\\
% |cdocsch2.tex|&include file for chapter 2\\
% |cdocspt3.tex|&include file for part 3\\
% |cdocspt4.tex|&include file for part 4\\
% |cdocsdrf.tex|&forwarding file for main file in draft mode\\
% |cdocsfi1.tex|&forwarding file for final version of chapter 1\\
% |cdocsfi2.tex|&forwarding file for final version of chapter 2\\
% \end{tabular}
% \end{center}
% Each of the eight files can be compiled directly by the \LaTeX{} compiler.
%
% %%%%%%%%%%%%%%%%%%%%%%%%%%%%%%%%%%%%%%
% \paragraph{Main File.}
%
% The main file is called |cdocsamp.tex|.
%
% Load the \textsf{childdoc} definitions and
% declare the filename for the main document:
%    \begin{macrocode}
\input{childdoc.def}
\childdocmain{}
%    \end{macrocode}

% Optional override for |\version| flag:
%    \begin{macrocode}
%%\ifchilddoc\else\providecommand{\version}{draft}\fi
%    \end{macrocode}

% Define the default values for the |\version| flag
% (|final| for the main file and |draft| for childs):
%    \begin{macrocode}
\ifchilddoc
\providecommand{\version}{draft}
\else
\providecommand{\version}{final}
\fi
%    \end{macrocode}

% Load the standard document class:
%    \begin{macrocode}
\documentclass[12pt]{article}
%    \end{macrocode}

% Start the document body:
%    \begin{macrocode}
\begin{document}
%    \end{macrocode}

% Declare a title page.
% Print title, part of document being processed and version flag:
%    \begin{macrocode}
\addtocounter{page}{-1}
\begin{center}
{\LARGE\bfseries{}childdoc example\par}
\vspace{1cm}
\ifchilddoc
\ifchilddocmanual part\else chapter\fi:
`\childdocname' of `\childdocjob'\par
\else
main document: `\childdocjob'\par
\fi
version: \version\par
\end{center}
\newpage
%    \end{macrocode}

% Manually include selected file,
% otherwise process as usual:
%    \begin{macrocode}
\ifchilddocmanual
\section*{part `\childdocname'}
\input{\childdocname}
\else
%    \end{macrocode}

% Include the two chapters:
%    \begin{macrocode}
\include{cdocsch1}
\include{cdocsch2}
%    \end{macrocode}

% Include the two parts unless only chapters should be displayed:
%    \begin{macrocode}
\ifchilddoc\else
\section{part three}
\input{cdocspt3}
\section{part four}
\input{cdocspt4}
\fi
%    \end{macrocode}

% Process as usual until here:
%    \begin{macrocode}
\fi
%    \end{macrocode}

% End of document body:
%    \begin{macrocode}
\end{document}
%    \end{macrocode}
%\iffalse
%</samplemain>
%\fi
%
% %%%%%%%%%%%%%%%%%%%%%%%%%%%%%%%%%%%%%%
% \paragraph{Chapter Include Files.}
%
% The include files are called |cdocsch1.tex| and |cdocsch2.tex|.
%
%\iffalse
%<*samplechap1|samplechap2>
%\fi

% Optional override for |\version| flag:
%    \begin{macrocode}
%%\providecommand{\version}{final}
%    \end{macrocode}

% Include the main document:
%    \begin{macrocode}
\input{childdoc.def}
\childdocof{cdocsamp}
%    \end{macrocode}

%\iffalse
%</samplechap1|samplechap2>
%\fi
%
%\iffalse
%<*samplechap1>
%\fi
% Some text for chapter 1:
%    \begin{macrocode}
\section{one}
some text in chapter one
%    \end{macrocode}

%\iffalse
%</samplechap1>
%\fi
% Some text for chapter 2:
%\iffalse
%<*samplechap2>
%\fi
%    \begin{macrocode}
\section{two}
more text in chapter two
%    \end{macrocode}

%\iffalse
%</samplechap2>
%\fi
%
% %%%%%%%%%%%%%%%%%%%%%%%%%%%%%%%%%%%%%%
% \paragraph{Part Include Files.}
%
% The include files are called |cdocspt3.tex| and |cdocspt4.tex|.
%
%\iffalse
%<*samplepart3|samplepart4>
%\fi

% Optional override for |\version| flag:
%    \begin{macrocode}
%%\providecommand{\version}{final}
%    \end{macrocode}

% Include the main document:
%    \begin{macrocode}
\input{childdoc.def}
\childdocby{cdocsamp}
%    \end{macrocode}

%\iffalse
%</samplepart3|samplepart4>
%\fi
%
%\iffalse
%<*samplepart3>
%\fi
% Some text for part 3:
%    \begin{macrocode}
some text in part three
%    \end{macrocode}

%\iffalse
%</samplepart3>
%\fi
% Some text for part 4:
%\iffalse
%<*samplepart4>
%\fi
%    \begin{macrocode}
more text in part four
%    \end{macrocode}

%\iffalse
%</samplepart4>
%\fi
%
% %%%%%%%%%%%%%%%%%%%%%%%%%%%%%%%%%%%%%%
% \paragraph{Forwarding for a Complete Draft.}
%
% The following forwarding file |cdocsdrf.tex|
% compiles the main document in draft mode:
%\iffalse
%<*sampledraft>
%\fi
%    \begin{macrocode}
\def\version{draft}
\input{childdoc.def}
\childdocforward{cdocsamp}
%    \end{macrocode}

%\iffalse
%</sampledraft>
%\fi
%
% %%%%%%%%%%%%%%%%%%%%%%%%%%%%%%%%%%%%%%
% \paragraph{Forwarding for Final Version of the Chapters.}
%
% The following forwarding files |cdocsfn1.tex| and |cdocsfn2.tex|
% (with identical content)
% compile the final versions of the child documents
% |cdocsch1.tex| and |cdocsch2.tex|, respectively:
%\iffalse
%<*samplefinal>
%\fi
%    \begin{macrocode}
\def\version{final}
\input{childdoc.def}
\childdocforwardprefix[cdocsamp]{cdocsfn}{cdocsch}
%    \end{macrocode}

%\iffalse
%</samplefinal>
%\fi
%
% %%%%%%%%%%%%%%%%%%%%%%%%%%%%%%%%%%%%%%
% \paragraph{Command Line Processing.}
%
% The following three command lines generate the output files
% |cdocscld|, |cdocscl1| and |cdocscl2|
% which should be identical to
% |cdocsdrf|, |cdocsch1| and |cdocsfn2|, respectively:
% \begin{center}
% \begin{tabular}{l}
% |latex -jobname cdocscld \|\\
% |  "\def\version{draft}\input{childdoc.def}\childdocforward{cdocsamp}"|\\
% |latex -jobname cdocscl1 \|\\
% |  "\input{childdoc.def}\childdocforward[cdocsamp]{cdocsch1}"|\\
% |latex -jobname cdocscl2 \|\\
% |  "\def\version{final}\input{childdoc.def}\childdocforward{cdocsch2}"|
% \end{tabular}
% \end{center}
% Note that the trailing backslash on each first line
% merely continues the input to the second line
% (for convenient cut ant paste).
% Furthermore, the command |latex| can be replaced by any
% of its alternative versions such as |pdflatex|.
%
% %%%%%%%%%%%%%%%%%%%%%%%%%%%%%%%%%%%%%%%%%%%%%%%%%%%%%%%%%%%%%%%%%%%%%%%%%%%%%%
% %%%%%%%%%%%%%%%%%%%%%%%%%%%%%%%%%%%%%%%%%%%%%%%%%%%%%%%%%%%%%%%%%%%%%%%%%%%%%%
% \section{Implementation}
%\iffalse
%<*package>
%\fi
%
% This section describes the definitions file |childdoc.def|.

% The definitions cannot be loaded using |\usepackage| or |\RequirePackage|
% which has a mechanism to prevent loading a style file more than once.
% When loading the definitions by means of |\input|
% multiple instances have to be prevented manually:
%\iffalse
%This code needs to be before the `\ProvidesFile' directive
%which is defined at the beginning of this file.
%Therefore it is also placed there and commented out here.
%</package>
%<*discard>
%\fi
%    \begin{macrocode}
\ifdefined\childdocmain\endinput\fi
%    \end{macrocode}
%\iffalse
%</discard>
%<*package>
%\fi
%
% \macro{\ifchilddoc}
% \macro{\ifchilddocmanual}
% The conditional |\ifchilddoc| tells whether a
% child (true) or main (false) document is being compiled.
% The conditional |\ifchilddocmanual| tells whether
% the |\includeonly| mechanism is used (false) or
% the selection of child files must be performed manually (true).
% The definitions initialise to false:
%    \begin{macrocode}
\newif\ifchilddoc
\newif\ifchilddocmanual
%    \end{macrocode}

% \macro{\childdocname}
% \macro{\childdocjob}
% The macro |\childdocname| stores the name of the main document
% to be compiled. The macro |\childdocjob| stores the name of
% the document on which the \LaTeX{} compiler was originally invoked.
% The content of |\jobname| cannot be compared
% to filenames specified in the source due to different catcodes.
% The following code rescans |\jobname|, stores the result
% in |\childdocname| and saves a copy in |\childdocjob|:
%    \begin{macrocode}
\edef\childdocname{\scantokens\expandafter{\jobname\noexpand}}
\let\childdocjob\childdocname
%    \end{macrocode}

% \macro{\childdocdisable}
% The macro |\childdocdisable| prevents the main file
% from being processed more than once.
% At this stage, the main document command |\childdocmain|
% is assumed to be called once again where it should do nothing.
% Any subsequent call to it should prevent
% a secondary processing of the main document
% It overwrites the forwarding commands
% |\childdocof| and |\childdocforward|
% with empty macros to prevent further inclusions of the main document:
%    \begin{macrocode}
\newcommand{\childdocdisable}
{
  \renewcommand{\childdocmain}[1]{\renewcommand{\childdocmain}[1]{\endinput}}
  \renewcommand{\childdocof}[1]{}
  \renewcommand{\childdocby}[2][]{}
  \renewcommand{\childdocforward}[2][]{}
  \renewcommand{\childdocdisable}{}
}
%    \end{macrocode}

% \macro{\childdocmain}
% The macro |\childdocmain| is to be called at the top of the main file
% with nothing or the main filename (without extension) as argument.
% First, it breaks loops.
% If the argument is not empty and does not match |\childdocname|
% (which is set by the first inclusion of |childdoc.def|),
% |\ifchilddoc| is set to true, |\includeonly| is applied to the child file
% and |\jobname| is set to the main file
% (for proper handling of |.aux| files):
%    \begin{macrocode}
\newcommand{\childdocmain}[1]
{
  \childdocdisable\childdocmain{}
  \if?#1?\else
    \begingroup
      \def\childdoctmp{#1}
      \ifx\childdoctmp\childdocname
        \def\childdoctmp{}
      \else
        \def\childdoctmp
        {
          \childdoctrue
          \includeonly{\childdocname}
          \def\childdocjob{#1}
          \def\jobname{#1}
        }
      \fi
      \expandafter
    \endgroup
    \childdoctmp
  \fi
}
%    \end{macrocode}

% \macro{\childdocof}
% The command |\childdocof| redirects
% compilation to the main file |#1|.
%    \begin{macrocode}
\newcommand{\childdocof}[1]
{
  \childdocdisable
  \childdoctrue
  \includeonly{\childdocname}
  \def\jobname{#1}
  \def\childdocjob{#1}
  \input{#1}
}
%    \end{macrocode}

% \macro{\childdocby}
% The command |\childdocby| ....
%    \begin{macrocode}
\newcommand{\childdocby}[2][]
{
  \childdocdisable
  \childdoctrue
  \childdocmanualtrue
  \if?#1?\else
    \def\jobname{#2}
  \fi
  \def\childdocjob{#2}
  \input{#2}
  \endinput
}
%    \end{macrocode}

% \macro{\childdocforward}
% The command |\childdocforward| redirects
% compilation to the main file or
% (if the optional argument is given) a child file.
% Parameters are set as if the main file
% or a child file starting with |\childdocof| was compiled.
% Then compilation is handed over to the main file:
%    \begin{macrocode}
\newcommand{\childdocforward}[2][]
{
  \begingroup
    \if?#1?
      \def\childdoctmp
      {
        \def\childdocname{#2}
        \def\childdocjob{#2}
        \def\jobname{#2}
        \input{#2}
        \endinput
      }
    \else
      \def\childdoctmp
      {
        \childdocdisable
        \def\childdocname{#2}
        \childdoctrue
        \includeonly{#2}
        \def\childdocjob{#1}
        \def\jobname{#1}
        \input{#1}
        \endinput
      }
    \fi
    \expandafter
  \endgroup
  \childdoctmp
}
%    \end{macrocode}

% \macro{\childdocforwardprefix}
% The command |\childdocforwardprefix| redirects
% compilation to the main or a child file by means of a pattern.
% The prefix |#1| in the current filename is replaced by |#2|
% and the suffix of the current filename is kept
% (it is assumed that the filename does not contain the substring `|~~~|'
% which is used as a delimiter).
% Compilation is handed over to the new file by |\childdocforward|:
%    \begin{macrocode}
\newcommand{\childdocforwardprefix}[3][]
{
  \begingroup
    \def\childdocextract #2##1~~~{\def\childdoctmp{\childdocforward[#1]{#3##1}}}
    \expandafter\childdocextract\childdocname~~~
    \expandafter
  \endgroup
  \childdoctmp
}
%    \end{macrocode}

% \macro{\childdoc}
% The deprecated macro |\childdoc| is a legacy version of |\childdocmain|:
%    \begin{macrocode}
\newcommand{\childdoc}{\childdocmain}
%    \end{macrocode}

% \macro{\childdocredirect}
% The deprecated macro |\childdocredirect| is a legacy version
% of |\childdocforward| and |\childdocforwardprefix|:
%    \begin{macrocode}
\newcommand{\childdocredirect}[2][]
{
  \begingroup
    \if?#1?
      \def\childdoctmp{\childdocforward{#2}}
    \else
      \def\childdoctmp{\childdocforwardprefix{#1}{#2}}
    \fi
    \expandafter
  \endgroup
  \childdoctmp
}
%    \end{macrocode}

%\iffalse
%</package>
%\fi
%
\endinput

\childdocmain{}
%    \end{macrocode}

% Optional override for |\version| flag:
%    \begin{macrocode}
%%\ifchilddoc\else\providecommand{\version}{draft}\fi
%    \end{macrocode}

% Define the default values for the |\version| flag
% (|final| for the main file and |draft| for childs):
%    \begin{macrocode}
\ifchilddoc
\providecommand{\version}{draft}
\else
\providecommand{\version}{final}
\fi
%    \end{macrocode}

% Load the standard document class:
%    \begin{macrocode}
\documentclass[12pt]{article}
%    \end{macrocode}

% Start the document body:
%    \begin{macrocode}
\begin{document}
%    \end{macrocode}

% Declare a title page.
% Print title, part of document being processed and version flag:
%    \begin{macrocode}
\addtocounter{page}{-1}
\begin{center}
{\LARGE\bfseries{}childdoc example\par}
\vspace{1cm}
\ifchilddoc
\ifchilddocmanual part\else chapter\fi:
`\childdocname' of `\childdocjob'\par
\else
main document: `\childdocjob'\par
\fi
version: \version\par
\end{center}
\newpage
%    \end{macrocode}

% Manually include selected file,
% otherwise process as usual:
%    \begin{macrocode}
\ifchilddocmanual
\section*{part `\childdocname'}
\input{\childdocname}
\else
%    \end{macrocode}

% Include the two chapters:
%    \begin{macrocode}
\include{cdocsch1}
\include{cdocsch2}
%    \end{macrocode}

% Include the two parts unless only chapters should be displayed:
%    \begin{macrocode}
\ifchilddoc\else
\section{part three}
\input{cdocspt3}
\section{part four}
\input{cdocspt4}
\fi
%    \end{macrocode}

% Process as usual until here:
%    \begin{macrocode}
\fi
%    \end{macrocode}

% End of document body:
%    \begin{macrocode}
\end{document}
%    \end{macrocode}
%\iffalse
%</samplemain>
%\fi
%
% %%%%%%%%%%%%%%%%%%%%%%%%%%%%%%%%%%%%%%
% \paragraph{Chapter Include Files.}
%
% The include files are called |cdocsch1.tex| and |cdocsch2.tex|.
%
%\iffalse
%<*samplechap1|samplechap2>
%\fi

% Optional override for |\version| flag:
%    \begin{macrocode}
%%\providecommand{\version}{final}
%    \end{macrocode}

% Include the main document:
%    \begin{macrocode}
% \iffalse
%
% childdoc.dtx Copyright (C) 2017-2018 Niklas Beisert
%
% This work may be distributed and/or modified under the
% conditions of the LaTeX Project Public License, either version 1.3
% of this license or (at your option) any later version.
% The latest version of this license is in
%   http://www.latex-project.org/lppl.txt
% and version 1.3 or later is part of all distributions of LaTeX
% version 2005/12/01 or later.
%
% This work has the LPPL maintenance status `maintained'.
%
% The Current Maintainer of this work is Niklas Beisert.
%
% This work consists of the files childdoc.dtx and childdoc.ins
% and the derived files childdoc.def and cdocsamp.tex with
% cdocsch1.tex, cdocsch2.tex, cdocsdrf.tex, cdocsfn1.tex, cdocsfn2.tex.
%
%<package>\ifdefined\childdocmain\endinput\fi
%<package>\ProvidesFile{childdoc.def}[2018/12/30 v2.0 child document driver]
%<samplemain>\ProvidesFile{cdocsamp.tex}[2018/12/30 v2.0 sample for childdoc]
%<*driver>
%\ProvidesFile{childdoc.drv}[2018/12/30 v2.0 childdoc reference manual file]
\PassOptionsToClass{10pt,a4paper}{article}
\documentclass{ltxdoc}

\usepackage[margin=35mm]{geometry}
\usepackage{hyperref}
\usepackage{hyperxmp}
\usepackage[usenames]{color}

\hypersetup{colorlinks=true}
\hypersetup{pdfstartview=FitH}
\hypersetup{pdfpagemode=UseNone}
\hypersetup{pdfsource={}}
\hypersetup{pdflang={en-UK}}
\hypersetup{pdfcopyright={Copyright 2017-2018 Niklas Beisert.
  This work may be distributed and/or modified under the
  conditions of the LaTeX Project Public License, either version 1.3
  of this license or (at your option) any later version.}}
\hypersetup{pdflicenseurl={http://www.latex-project.org/lppl.txt}}
\hypersetup{pdfcontactaddress={ETH Zurich, ITP, HIT K,
  Wolfgang-Pauli-Strasse 27}}
\hypersetup{pdfcontactpostcode={8093}}
\hypersetup{pdfcontactcity={Zurich}}
\hypersetup{pdfcontactcountry={Switzerland}}
\hypersetup{pdfcontactemail={nbeisert@itp.phys.ethz.ch}}
\hypersetup{pdfcontacturl={http://people.phys.ethz.ch/\xmptilde nbeisert/}}

\newcommand{\secref}[1]{\hyperref[#1]{section \ref*{#1}}}

\parskip1ex
\parindent0pt
\let\olditemize\itemize
\def\itemize{\olditemize\parskip0pt}

\begin{document}

\title{The \textsf{childdoc} Package}
\hypersetup{pdftitle={The childdoc Package}}
\author{Niklas Beisert\\[2ex]
  Institut f\"ur Theoretische Physik\\
  Eidgen\"ossische Technische Hochschule Z\"urich\\
  Wolfgang-Pauli-Strasse 27, 8093 Z\"urich, Switzerland\\[1ex]
  \href{mailto:nbeisert@itp.phys.ethz.ch}
  {\texttt{nbeisert@itp.phys.ethz.ch}}}
\hypersetup{pdfauthor={Niklas Beisert}}
\hypersetup{pdfsubject={Manual for the LaTeX2e Package childdoc}}
\date{30 December 2018, \textsf{v2.0}}
\maketitle

\begin{abstract}\noindent
\textsf{childdoc} is a \LaTeXe{} package
that enables the direct compilation
of document sections included by |\include|
to individual files.
\end{abstract}

\begingroup
\parskip0ex
\tableofcontents
\endgroup

%%%%%%%%%%%%%%%%%%%%%%%%%%%%%%%%%%%%%%%%%%%%%%%%%%%%%%%%%%%%%%%%%%%%%%%%%%%%%%%%
%%%%%%%%%%%%%%%%%%%%%%%%%%%%%%%%%%%%%%%%%%%%%%%%%%%%%%%%%%%%%%%%%%%%%%%%%%%%%%%%
\section{Introduction}

\LaTeX{} provides a mechanism to structure a large document (such as a book)
into a main file and several child files (containing the chapters)
using the |\include| command.
This mechanism is beneficial for documents
which span hundreds of pages in order to
make the source file(s) more manageable.
Moreover, compilation can be restricted to
selected child files by means of the |\includeonly| command.
The latter feature can be used to reduce the compilation time while editing
(this was significantly more useful in the earlier days of \LaTeX{})
or to generate a smaller document which is easier to navigate.
Another application of |\includeonly| is to generate
documents consisting of selected parts of the complete document.

However, there are a few drawbacks of the plain |\include| mechanism:
\begin{itemize}
\item
The child files cannot be compiled on their own,
they can only be compiled via the main file.
A naive editing environment
(such as a text editor with an option
to have the current file processed by \LaTeX)
may require one to switch to the main file before compiling;
attempting to compile the child file produces errors.
\item
The main file must be modified (each time)
to adjust the |\includeonly| command
to the present needs. This easily leaves the main file in a messy state.
\item
The generated document will always carry the filename
of the main document. This is inconvenient if
several child files are to be compiled and
to be kept for distribution.
\end{itemize}

The present package provides a simple interface
to make child files individually compilable by \LaTeX{}.
Compiling a child file then has the same effect as compiling
the main file with an |\includeonly| command
to select the appropriate child.
Moreover the generated document will carry the name of the child
rather than the main file.
This resolves all three above issues.

This feature is meant to make the editing of books,
thesis documents and lecture notes somewhat more convenient.
However, the package can also be used efficiently for
composing a series of documents (such as exercise sheets)
which are typically distributed individually.
It then assists the author in generating the individual documents
(potentially in different versions)
as well as a document containing the collected series.
Another application is in developing style files
or other kinds of included material
where compilation of the style file could redirect
to a sample or test file.

%%%%%%%%%%%%%%%%%%%%%%%%%%%%%%%%%%%%%%%%%%%%%%%%%%%%%%%%%%%%%%%%%%%%%%%%%%%%%%%%
%%%%%%%%%%%%%%%%%%%%%%%%%%%%%%%%%%%%%%%%%%%%%%%%%%%%%%%%%%%%%%%%%%%%%%%%%%%%%%%%
\section{Usage}

First of all, the package \textsf{childdoc} is \emph{not} a standard
\LaTeXe{} |.sty| style file! Therefore it needs to be invoked in
a non-standard way.

%%%%%%%%%%%%%%%%%%%%%%%%%%%%%%%%%%%%%%%%%%%%%%%%%%%%%%%%%%%%%%%%%%%%%%%%%%%%%%%%
\subsection{Included Files}
\label{sec:include}

%%%%%%%%%%%%%%%%%%%%%%%%%%%%%%%%%%%%%%%%
\DescribeMacro{\childdocmain}
To use the package, add the commands
\begin{center}
\begin{tabular}{l}
|\input{childdoc.def}|\\
|\childdocmain{}|\\
\end{tabular}
\end{center}
at the very top of the main \LaTeX{} file,
in particular \emph{before} the |\documentclass| statement!
The argument of |\childdocmain| should be left empty
(but it must be present).

%%%%%%%%%%%%%%%%%%%%%%%%%%%%%%%%%%%%%%%%
\DescribeMacro{\childdocof}
Furthermore, add the commands
\begin{center}
\begin{tabular}{l}
|\input{childdoc.def}|\\
|\childdocof{|\textit{main}|}|\\
\end{tabular}
\end{center}
at the top of every child file \textit{child}
which is included by |\include{|\textit{child}|}|
from within the main file
(or at least for those files to be compiled individually).
The argument \textit{main} must be the filename of the main file.

There are a couple of
considerations in setting up the main and child documents:

%%%%%%%%%%%%%%%%%%%%%%%%%%%%%%%%%%%%%%%%
\paragraph{Restrictions.}

Please note the following restrictions:
\begin{itemize}
\item
|\childdocmain| must be called with one argument \textit{main}
to ensure compatibility with earlier version of the package.
It must either be empty (|\childdocmain{}|)
or precisely match the filename of the main file in which it is specified.
See \secref{sec:detection} for further information.
\item
The filename \textit{main} must be specified without the |.tex| extension.
\item
The filename \textit{main} is case sensitive
(even in case-insensitive file systems)
due to internal string comparison.
\item
The argument \textit{main} should be fully expanded, it cannot be a macro.
\item
Subdirectories and special characters should be avoided in filenames.
\item
The command |\childdocmain{|\textit{main}|}| must be followed by a whitespace.
It should not be followed immediately by another command
or by a comment mark `|%|'.
This is because the \TeX{} parser reads the token immediately following
the argument of |\childdocmain| and puts it
at the beginning of every child section;
however, a white\-space is ignored.
\end{itemize}

%%%%%%%%%%%%%%%%%%%%%%%%%%%%%%%%%%%%%%%%
\paragraph{Content of Main File.}

It is advisable to place all content in the child files included by |\include|.
Any output contained in the main file will appear in all child documents
unless suppressed manually;
it cannot be suppressed automatically by the |\includeonly| directive
and thus should normally be avoided.
A method to include some content in the main file
by means of conditional processing is described in \secref{sec:conditional}.

%%%%%%%%%%%%%%%%%%%%%%%%%%%%%%%%%%%%%%%%
\paragraph{Page Numbering.}

When only a part of the document is compiled,
the appropriate numbering of pages
(as well as other status parameters)
is determined from the |.aux| files.
The latter contain information from previous passes.
However this information needs to propagate through
all intermediate child documents.
Therefore the page numbering in child documents may well
be inconsistent until the complete document is compiled at least once.

A useful (if unconventional) way to always ensure a consistent
page numbering is to restart the numbering in each child document
and denote the pages by `\textit{child}|.|\textit{page}'
where \textit{child} represents the chapter/section number of the child file.
This can be achieved by the command
|\numberwithin{page}{|\textit{child}|}|
of the \textsf{amsmath} package
where \textit{child} can be |chapter| or |section|
depending on the chosen structuring.
Alternatively, one can modify the macro |\thepage| appropriately
and reset the counter |page| at the start of each child file.

%%%%%%%%%%%%%%%%%%%%%%%%%%%%%%%%%%%%%%%%%%%%%%%%%%%%%%%%%%%%%%%%%%%%%%%%%%%%%%%%
\subsection{Conditional Processing}
\label{sec:conditional}

The package provides a mechanism to compile different versions
of a document. To customise the versions further some conditional processing
can come in handy to distinguish which version is being compiled.
The package provides two macros to describe the compilation context:

%%%%%%%%%%%%%%%%%%%%%%%%%%%%%%%%%%%%%%%%
\DescribeMacro{\ifchilddoc}
The conditional |\ifchilddoc| distinguishes between the compilation of
child documents and the main document:
%
\begin{center}
|\ifchilddoc |\textit{child-code}| |[|\||else |\textit{main-code}]| \||fi|
\end{center}

%%%%%%%%%%%%%%%%%%%%%%%%%%%%%%%%%%%%%%%%
\DescribeMacro{\childdocname}
\DescribeMacro{\childdocjob}
The macro |\childdocname| contains the filename (without extension)
of the main or child file being processed.
Note that |\childdocjob| will always contain the name of the main file.

%%%%%%%%%%%%%%%%%%%%%%%%%%%%%%%%%%%%%%%%
\paragraph{Title Page.}

Conditional processing can be used to include a title or banner page
in the main document when proper precautions are taken.
Importantly, the code in the main file should ensure that the page counter
(as well as other status parameters which are stored in the |.aux| files)
takes the same value after the conditional processing.
Otherwise the page numbers may take divergent values
depending on which part is compiled.

For example, a title page could be declared by:
%
\begin{center}
\begin{tabular}{l}
|\ifchilddoc\||else|\\
|\addtocounter{page}{-1}|\\
\textit{code for title page}\\
|\newpage|\\
|\||fi|
\end{tabular}
\end{center}
%
A banner page for the child documents can be generated by:
%
\begin{center}
\begin{tabular}{l}
|\ifchilddoc|\\
|\addtocounter{page}{-1}|\\
\textit{code for banner page}\\
|\newpage|\\
|\||fi|
\end{tabular}
\end{center}
%
Here one could write a message such as:
\begin{center}
|This is the part \childdocname{} of \childdocjob{}.|
\end{center}

%%%%%%%%%%%%%%%%%%%%%%%%%%%%%%%%%%%%%%%%%%%%%%%%%%%%%%%%%%%%%%%%%%%%%%%%%%%%%%%%
\subsection{Flags}
\label{sec:flags}

The package makes it easy to generate different versions
of the main or child documents.
To this end compilation flags can be defined
and assigned different default values.
They will be particularly useful in conjunction
with the forwarding mechanism described in \secref{sec:forward}.

For example, it may be useful to have a flag |\version|
which can be set to |draft| or |final|.
The document source will contain some conditional code
depending on the value of |\version|.
Suppose further, the flag should default to |final| for the main file
and to |draft| for child files
which is a natural assignment for editing the document.
This is achieved by placing the following code
in the preamble of the main document
(below the |\childdocmain| directive):
%
\begin{center}
\begin{tabular}{l}
|\ifchilddoc|\\
|\providecommand{\version}{draft}|\\
|\||else|\\
|\providecommand{\version}{final}|\\
|\||fi|
\end{tabular}
\end{center}
%
The definition by |\providecommand| makes sure
that previous definitions are not overwritten.
Further statements |\providecommand{\version}{...}|
can thus be added before the above code to override it.

For the main file, one might add a line
(between |\childdocmain| and the above block)
%
\begin{center}
|%\ifchilddoc\||else\providecommand{\version}{draft}\||fi|
\end{center}
%
which can be uncommented to produce a draft version.
Likewise one can add a line to the very top of a child file
(above the |\childdocof{|\textit{main}|}| directive)
%
\begin{center}
|%\providecommand{\version}{final}|
\end{center}
%
which can be uncommented to produce the final version of this child document.

%%%%%%%%%%%%%%%%%%%%%%%%%%%%%%%%%%%%%%%%%%%%%%%%%%%%%%%%%%%%%%%%%%%%%%%%%%%%%%%%
\subsection{Forwarding}
\label{sec:forward}

Different versions of the main or child documents
using compilation flags as described in \secref{sec:flags}
can be (permanently) stored in different files
for convenient compilation, viewing and distribution.
To this end, the package defines a command
to pass on compilation to a different file:

%%%%%%%%%%%%%%%%%%%%%%%%%%%%%%%%%%%%%%%%
\DescribeMacro{\childdocforward}
The command |\childdocforward| redirects processing to
another source file:
%
\begin{center}
\begin{tabular}{l}
|\input{childdoc.def}|\\
|\childdocforward[|\textit{main}|]{|\textit{dest}|}|\\
\end{tabular}
\end{center}
%
The argument \textit{dest} is the destination file
(without extension).
It should be the main file or one of the child files.
Note that further \textsf{childdoc} directives
such as |\childdocof| and |\childdocforward|
in the indicated file will be processed in this form.
The optional argument \textit{main}
passes on directly to the main file \textit{main}
while pretending to compile the child \textit{dest}.
This form behaves as if \textit{dest}
issues |\childdocof{|\textit{main}|}| right away,
and no further \textsf{childdoc} directives will be processed.

%%%%%%%%%%%%%%%%%%%%%%%%%%%%%%%%%%%%%%%%
\DescribeMacro{\...prefix}
In the alternative form |\childdocforwardprefix|,
%
\begin{center}
\begin{tabular}{l}
|\input{childdoc.def}|\\
|\childdocforwardprefix[|\textit{main}|]{|\textit{prefix}|}{|\textit{dest}|}|
\end{tabular}
\end{center}
%
the destination file is determined by a pattern
depending on the current file:
To make this work, the current file must be called
`{\textit{prefix}\hspace{0.2em}\textit{suffix}}'
with \textit{prefix} matching precisely the argument.
Processing is then passed on to the file
`{\textit{dest}\hspace{0.2em}\textit{suffix}}'.
Surely, the same effect is achieved by
directly specifying the
argument `{\textit{dest}\hspace{0.2em}\textit{suffix}}'
in the first form.
However, that requires to set up a different file
for each child. With the alternative form of the command
all these files can have exactly the same content
which simplifies setting them up and maintaining them.

For example, the following file |draft.tex|
with a compilation flag |\version| as described in \secref{sec:flags}
compiles the main document as a draft:
%
\begin{center}
\begin{tabular}{l}
|\def\version{draft}|\\
|\input{childdoc.def}|\\
|\childdocforward{|\textit{main}|}|
\end{tabular}
\end{center}
%
Likewise, the following files |final|\textit{nn}|.tex|
compile the final version of the child document
|child|\textit{nn}|.tex|:
%
\begin{center}
\begin{tabular}{l}
|\def\version{final}|\\
|\input{childdoc.def}|\\
|\childdocforwardprefix{final}{child}|
\end{tabular}
\end{center}
%

Note that when several versions of a main file and/or of each child file
are to be generated, it may be convenient to set up a |Makefile| or
shell script to automatise the process.

%%%%%%%%%%%%%%%%%%%%%%%%%%%%%%%%%%%%%%%%%%%%%%%%%%%%%%%%%%%%%%%%%%%%%%%%%%%%%%%%
\subsection{Command Line Processing}
\label{sec:commandline}

The effect of redirection files can also be achieved by invoking
the \LaTeX{} compiler with a more elaborate command line.
Most conveniently this should be done as part
of a shell script or a |Makefile|.

When using \textsf{childdoc} in the main file, the following
command lines effectively perform a redirection
(note that depending on the shell being used,
backslashes may have to be doubled: `|\|' $\to$ `|\\|'):
%
\begin{center}
|... -jobname "|\textit{target}|" |\\|"|[\textit{flags}]%
|\input{childdoc.def}\childdocforward[|\textit{main}|]{|\textit{dest}|}"|
\end{center}
%
Here \textit{target} is the name of the output file,
\textit{main} is the name of the main file
and \textit{dest} is the name of the main or child file to be processed
(all filenames without extensions).
The optional argument \textit{main} can be omitted
if \textit{main} matches \textit{dest}.
Optionally, compilation \textit{flags} can be defined via |\def| commands.
This command line makes the \TeX{} engine believe
it is compiling the file \textit{target}
whose content is specified as the latter parameter.
The provided code then forwards the processing to
\textit{main} or \textit{dest} as described in \secref{sec:forward}.

%%%%%%%%%%%%%%%%%%%%%%%%%%%%%%%%%%%%%%%%%%%%%%%%%%%%%%%%%%%%%%%%%%%%%%%%%%%%%%%%
\subsection{Include by Input}
\label{sec:input}

Including child documents by |\include| has some restrictions by design.
Most notably, the content of a child document always occupies
its own set of pages; pages cannot be shared between child documents.
Usually, this behaviour makes perfect sense
because each child document contain an essential part of the document.
However, in some situations it may be desirable to compose
a document from a collection of parts
without having mandatory page breaks between then.
For this case, the package
provides a mechanism to include parts
by |\input| which can also be processed individually.
However, by construction this mechanism
requires manual handling of the content to be output.

%%%%%%%%%%%%%%%%%%%%%%%%%%%%%%%%%%%%%%%%
\DescribeMacro{\ifchilddocmanual}
The main file should be prepared as usual, see \secref{sec:include}.
However, the document body must make a distinction
between processing of an individual part and of the main document, e.g.:
%
\begin{center}
\begin{tabular}{l}
|\ifchilddocmanual|\\
|\input{\childdocname}|\\
|\||else|\\
\textit{document body with }|\input{|\textit{part}|}|\\
|\||fi|
\end{tabular}
\end{center}
%
The conditional |\ifchilddocmanual| is true whenever
a part to be included by |\input| is being compiled,
and the name of the part is stored in |\childdocname|.

%%%%%%%%%%%%%%%%%%%%%%%%%%%%%%%%%%%%%%%%
\DescribeMacro{\childdocby}
Each part to be included by |\input| should start with:
%
\begin{center}
\begin{tabular}{l}
|\input{childdoc.def}|\\
|\childdocby{|\textit{main}|}|\\
\end{tabular}
\end{center}
%
The directive |\childdocby| is similar to |\childdocof|
described in \secref{sec:include},
but the subsequent selection of content must be done manually.
To that end, both |\ifchilddoc| and |\ifchilddocmanual|
will be true upon processing of a part,
and the name of the part is stored in |\childdocname|.
Note that |\jobname| will be set to the filename of the current part
so that each part receives an individual |.aux| file
that does not interfere with the |.aux| file(s) of the main document.
This behaviour can be altered by the alternative form
|\childdocby[*]{|\textit{main}|}| (with a non-empty optional argument)
which uses the |.aux| file of the main document
by setting |\jobname| to \textit{main}.

%%%%%%%%%%%%%%%%%%%%%%%%%%%%%%%%%%%%%%%%%%%%%%%%%%%%%%%%%%%%%%%%%%%%%%%%%%%%%%%%
\subsection{Driver Development}
\label{sec:driver}

The \textsf{childdoc} mechanism can also be use for the development
of definition files such as \LaTeX{} styles or classes.
This case differs from the above setup with multiple parts
included by |\include| in that no |\includeonly| should be invoked.
This can be achieved by starting the include file
(before |\ProvidesPackage|) with:
%
\begin{center}
\begin{tabular}{l}
|\input{childdoc.def}|\\
|\childdocforward{|\textit{main}|}|\\
\end{tabular}
\end{center}
%
or alternatively with:
%
\begin{center}
\begin{tabular}{l}
|\input{childdoc.def}|\\
|\childdocby{|\textit{main}|}|\\
\end{tabular}
\end{center}
%
Both forms have slightly different effects as described above.
The main file is prepared as usual, see \secref{sec:include}.

%%%%%%%%%%%%%%%%%%%%%%%%%%%%%%%%%%%%%%%%%%%%%%%%%%%%%%%%%%%%%%%%%%%%%%%%%%%%%%%%
\subsection{Legacy Detection}
\label{sec:detection}

The directive |\childdocmain| in the main file can detect
whether the complete document or merely a child is to be compiled
even without using the directive |\childdocof|.
This method is deprecated because it is less robust
and there is no compelling reason to use it;
it is merely provided for backward compatibility
and it may be removed in future versions.

If the detection mechanism is to be used,
it is mandatory to correctly specify
the filename of the main file as the argument of |\childdocmain|:
%
\begin{center}
\begin{tabular}{l}
|\input{childdoc.def}|\\
|\childdocmain{|\textit{main}|}|\\
\end{tabular}
\end{center}
%
If |\jobname| does not match the argument \textit{main} of |\childdocmain|,
it is assumed that |\jobname| points to the child file to be compiled.
When using |\childdocmain| with the main file specified as argument,
it suffices to start a child file
with just |\input{|\textit{main}|}|
without loading of the package and using |\childdocof|.
If instead all processing is done
with the appropriate \textsf{childdoc} directives,
the argument of \textit{main} of |\childdocmain| can be empty.

An alternative version of the command line processing described
in \secref{sec:commandline} using the detection mechanism reads:
%
\begin{center}
|... -jobname "|\textit{target}|" "|[\textit{flags}]%
[|\def\jobname{|\textit{dest}|}|]|\input{|\textit{main}|}"|
\end{center}

%%%%%%%%%%%%%%%%%%%%%%%%%%%%%%%%%%%%%%%%%%%%%%%%%%%%%%%%%%%%%%%%%%%%%%%%%%%%%%%%
\subsection{Manual Code}
\label{sec:manual}

In case one cannot be certain whether the definitions file |childdoc.def|
is installed on the target \TeX{} distribution
and one prefers not to ship it,
it is conceivable to paste a few relevant commands into the sources.

To that end, drop all statements |\input{childdoc.def}|
and perform the replacements as outlined below.
Instead of |\childdocmain{|\textit{main}|}| add the following code
to the top of the main file:
%
\begin{center}
\begin{tabular}{l}
|\||ifdefined\childdocname\endinput\||fi\newif\ifchilddoc|\\
|\edef\childdocname{\scantokens\expandafter{\jobname\noexpand}}|\\
|\def\childdocmain{|\textit{main}|}\||ifx\childdocmain\childdocname\||else|\\
|\childdoctrue\includeonly{\childdocname}\let\jobname\childdocmain\||fi|\\
\end{tabular}
\end{center}
%
Instead of |\childdocof{|\textit{main}|}| just include the main file
at the top of each child file:
%
\begin{center}
|\input{|\textit{main}|}|
\end{center}
%
A simple redirection |\childdocforward{|\textit{dest}|}| is achieved by:
%
\begin{center}
|\def\jobname{|\textit{dest}|}\input{\jobname}|
\end{center}
%
The redirection with prefix
|\childdocforwardprefix[|\textit{prefix}|]{|\textit{dest}|}|
is accomplished by:
%
\begin{center}
\begin{tabular}{l}
|{\edef\jobname{\scantokens\expandafter{\jobname\noexpand}}|\\
|\def\redirectjob |\textit{prefix}|#1~~~{\gdef\jobname{|\textit{dest}|#1}}|\\
|\expandafter\redirectjob\jobname~~~}\input{\jobname}|
\end{tabular}
\end{center}

In an alternative approach,
child documents can be compiled by a specific command line
without additional code or specific definitions:
%
\begin{center}
|... -jobname "|\textit{target}|" "|[\textit{flags}]%
|\includeonly{|\textit{dest}|}\input{|\textit{main}|}"|
\end{center}
%

%%%%%%%%%%%%%%%%%%%%%%%%%%%%%%%%%%%%%%%%%%%%%%%%%%%%%%%%%%%%%%%%%%%%%%%%%%%%%%%%
%%%%%%%%%%%%%%%%%%%%%%%%%%%%%%%%%%%%%%%%%%%%%%%%%%%%%%%%%%%%%%%%%%%%%%%%%%%%%%%%
\section{Information}

%%%%%%%%%%%%%%%%%%%%%%%%%%%%%%%%%%%%%%%%%%%%%%%%%%%%%%%%%%%%%%%%%%%%%%%%%%%%%%%%
\subsection{Copyright}

Copyright \copyright{} 2017--2018 Niklas Beisert

This work may be distributed and/or modified under the
conditions of the \LaTeX{} Project Public License, either version 1.3
of this license or (at your option) any later version.
The latest version of this license is in
  \url{http://www.latex-project.org/lppl.txt}
and version 1.3 or later is part of all distributions of \LaTeX{}
version 2005/12/01 or later.

This work has the LPPL maintenance status `maintained'.

The Current Maintainer of this work is Niklas Beisert.

This work consists of the files |README.txt|, |childdoc.ins| and |childdoc.dtx|
as well as the derived files |childdoc.def|, |cdocsamp.tex|
with |cdocsch1.tex|, |cdocsch2.tex|, |cdocspt3.tex|, |cdocspt4.tex|,
|cdocsdrf.tex|, |cdocsfn1.tex|, |cdocsfn2.tex|
as well as |childdoc.pdf|.

%%%%%%%%%%%%%%%%%%%%%%%%%%%%%%%%%%%%%%%%%%%%%%%%%%%%%%%%%%%%%%%%%%%%%%%%%%%%%%%%
\subsection{Files and Installation}

The package consists of the files:
%
\begin{center}
\begin{tabular}{ll}
    |README.txt|   & readme file \\
    |childdoc.ins| & installation file \\
    |childdoc.dtx| & source file \\
    |childdoc.def| & definition file \\
    |cdocsamp.tex| & sample main file \\
    |cdocsch1.tex| & sample include file \\
    |cdocsch2.tex| & sample include file \\
    |cdocspt3.tex| & sample part file \\
    |cdocspt4.tex| & sample part file \\
    |cdocsdrf.tex| & sample redirection file \\
    |cdocsfn1.tex| & sample redirection file \\
    |cdocsfn2.tex| & sample redirection file \\
    |childdoc.pdf| & manual
\end{tabular}
\end{center}
%
The distribution consists of the files
|README.txt|, |childdoc.ins| and |childdoc.dtx|.
%
\begin{itemize}
\item
Run (pdf)\LaTeX{} on |childdoc.dtx|
to compile the manual |childdoc.pdf| (this file).
\item
Run \LaTeX{} on |childdoc.ins| to create the definitions file |childdoc.def|
and the sample |cdocsamp.tex| with include files
|cdocsch1.tex|, |cdocsch2.tex|, |cdocspt3.tex|, |cdocspt4.tex|,
|cdocsdrf.tex|, |cdocsfn1.tex|, |cdocsfn2.tex|.
Then copy the file |childdoc.def| to an appropriate directory of your \LaTeX{}
distribution, e.g.\ \textit{texmf-root}|/tex/latex/childdoc|.
\end{itemize}

%%%%%%%%%%%%%%%%%%%%%%%%%%%%%%%%%%%%%%%%%%%%%%%%%%%%%%%%%%%%%%%%%%%%%%%%%%%%%%%%
\subsection{Related CTAN Packages}

There are several other packages which offer a similar functionality:
%
\begin{itemize}
\item
The packages
\href{http://ctan.org/pkg/docmute}{\textsf{docmute}},
\href{http://ctan.org/pkg/includex}{\textsf{includex}} and
\href{http://ctan.org/pkg/standalone}{\textsf{standalone}}
provide commands to include only the document body of
a child file thus allowing both files to be compiled individually.
\item
The packages \href{http://ctan.org/pkg/subdocs}{\textsf{subdocs}}
and \href{http://ctan.org/pkg/subfiles}{\textsf{subfiles}}
provide structures in which the main and child documents can be
encapsulated and allowing them to be compiled individually.
The inclusion mechanism is different from the conventional |\include|.
\item
The package \href{http://ctan.org/pkg/combine}{\textsf{combine}}
is an elaborate solution to combine several documents into one.
\end{itemize}
%
See also the CTAN topic \href{http://ctan.org/topic/subdocs}{\textsf{subdocs}}
for further related packages.
The present package differs from the above solutions in that
a document structure constructed with the conventional |\include| mechanism
just needs two extra commands at the top of every file
such that all constituent files can be compiled individually.

%%%%%%%%%%%%%%%%%%%%%%%%%%%%%%%%%%%%%%%%%%%%%%%%%%%%%%%%%%%%%%%%%%%%%%%%%%%%%%%%
%\subsection{Feature Suggestions}
%
%The following is a list of features which may be useful for future
%versions of this package:
%%
%\begin{itemize}
%\item
%\ldots
%\end{itemize}

%%%%%%%%%%%%%%%%%%%%%%%%%%%%%%%%%%%%%%%%%%%%%%%%%%%%%%%%%%%%%%%%%%%%%%%%%%%%%%%%
\subsection{Revision History}

%%%%%%%%%%%%%%%%%%%%%%%%%%%%%%%%%%%%%%%%
\paragraph{v2.0:} 2018/12/30

\begin{itemize}
\item
immediate forward processing
\item
added |\childdocby| mechanism
\item
manual restructured
\end{itemize}

%%%%%%%%%%%%%%%%%%%%%%%%%%%%%%%%%%%%%%%%
\paragraph{v1.6:} 2018/01/17

\begin{itemize}
\item
application for development of include files
\item
corrections to manual
\end{itemize}

%%%%%%%%%%%%%%%%%%%%%%%%%%%%%%%%%%%%%%%%
\paragraph{v1.5:} 2017/05/21

\begin{itemize}
\item
more complete structuring introduced
\item
|\childdocof| introduced
\item
|\childdoc| renamed to |\childdocmain|
\item
|\childredirect| renamed to |\childdocforward| and |\childdocforwardprefix|
and functionality expanded
\end{itemize}

%%%%%%%%%%%%%%%%%%%%%%%%%%%%%%%%%%%%%%%%
\paragraph{v1.0:} 2017/04/27

\begin{itemize}
\item
manual and install package
\item
first version published on CTAN
\end{itemize}

%%%%%%%%%%%%%%%%%%%%%%%%%%%%%%%%%%%%%%%%
\paragraph{v0.6:} 2017/04/26

\begin{itemize}
\item
redirection mechanism added
\end{itemize}

%%%%%%%%%%%%%%%%%%%%%%%%%%%%%%%%%%%%%%%%
\paragraph{v0.5:} 2017/04/26

\begin{itemize}
\item
functionality in definition file
\end{itemize}


%%%%%%%%%%%%%%%%%%%%%%%%%%%%%%%%%%%%%%%%%%%%%%%%%%%%%%%%%%%%%%%%%%%%%%%%%%%%%%%%
%%%%%%%%%%%%%%%%%%%%%%%%%%%%%%%%%%%%%%%%%%%%%%%%%%%%%%%%%%%%%%%%%%%%%%%%%%%%%%%%
%%%%%%%%%%%%%%%%%%%%%%%%%%%%%%%%%%%%%%%%%%%%%%%%%%%%%%%%%%%%%%%%%%%%%%%%%%%%%%%%
\appendix

\settowidth\MacroIndent{\rmfamily\scriptsize 000\ }

 \DocInput{childdoc.dtx}

\end{document}
%</driver>
% \fi
%
% %%%%%%%%%%%%%%%%%%%%%%%%%%%%%%%%%%%%%%%%%%%%%%%%%%%%%%%%%%%%%%%%%%%%%%%%%%%%%%
% %%%%%%%%%%%%%%%%%%%%%%%%%%%%%%%%%%%%%%%%%%%%%%%%%%%%%%%%%%%%%%%%%%%%%%%%%%%%%%
% \section{Sample}
%\iffalse
%<*samplemain>
%\fi
%
% The following presents a sample document
% with two chapters, two parts, a title page,
% a compile flag as well as three forwarding files to set the flag.
% It consists of eight |.tex| files:
% \begin{center}
% \begin{tabular}{ll}
% |cdocsamp.tex|&main file\\
% |cdocsch1.tex|&include file for chapter 1\\
% |cdocsch2.tex|&include file for chapter 2\\
% |cdocspt3.tex|&include file for part 3\\
% |cdocspt4.tex|&include file for part 4\\
% |cdocsdrf.tex|&forwarding file for main file in draft mode\\
% |cdocsfi1.tex|&forwarding file for final version of chapter 1\\
% |cdocsfi2.tex|&forwarding file for final version of chapter 2\\
% \end{tabular}
% \end{center}
% Each of the eight files can be compiled directly by the \LaTeX{} compiler.
%
% %%%%%%%%%%%%%%%%%%%%%%%%%%%%%%%%%%%%%%
% \paragraph{Main File.}
%
% The main file is called |cdocsamp.tex|.
%
% Load the \textsf{childdoc} definitions and
% declare the filename for the main document:
%    \begin{macrocode}
\input{childdoc.def}
\childdocmain{}
%    \end{macrocode}

% Optional override for |\version| flag:
%    \begin{macrocode}
%%\ifchilddoc\else\providecommand{\version}{draft}\fi
%    \end{macrocode}

% Define the default values for the |\version| flag
% (|final| for the main file and |draft| for childs):
%    \begin{macrocode}
\ifchilddoc
\providecommand{\version}{draft}
\else
\providecommand{\version}{final}
\fi
%    \end{macrocode}

% Load the standard document class:
%    \begin{macrocode}
\documentclass[12pt]{article}
%    \end{macrocode}

% Start the document body:
%    \begin{macrocode}
\begin{document}
%    \end{macrocode}

% Declare a title page.
% Print title, part of document being processed and version flag:
%    \begin{macrocode}
\addtocounter{page}{-1}
\begin{center}
{\LARGE\bfseries{}childdoc example\par}
\vspace{1cm}
\ifchilddoc
\ifchilddocmanual part\else chapter\fi:
`\childdocname' of `\childdocjob'\par
\else
main document: `\childdocjob'\par
\fi
version: \version\par
\end{center}
\newpage
%    \end{macrocode}

% Manually include selected file,
% otherwise process as usual:
%    \begin{macrocode}
\ifchilddocmanual
\section*{part `\childdocname'}
\input{\childdocname}
\else
%    \end{macrocode}

% Include the two chapters:
%    \begin{macrocode}
\include{cdocsch1}
\include{cdocsch2}
%    \end{macrocode}

% Include the two parts unless only chapters should be displayed:
%    \begin{macrocode}
\ifchilddoc\else
\section{part three}
\input{cdocspt3}
\section{part four}
\input{cdocspt4}
\fi
%    \end{macrocode}

% Process as usual until here:
%    \begin{macrocode}
\fi
%    \end{macrocode}

% End of document body:
%    \begin{macrocode}
\end{document}
%    \end{macrocode}
%\iffalse
%</samplemain>
%\fi
%
% %%%%%%%%%%%%%%%%%%%%%%%%%%%%%%%%%%%%%%
% \paragraph{Chapter Include Files.}
%
% The include files are called |cdocsch1.tex| and |cdocsch2.tex|.
%
%\iffalse
%<*samplechap1|samplechap2>
%\fi

% Optional override for |\version| flag:
%    \begin{macrocode}
%%\providecommand{\version}{final}
%    \end{macrocode}

% Include the main document:
%    \begin{macrocode}
\input{childdoc.def}
\childdocof{cdocsamp}
%    \end{macrocode}

%\iffalse
%</samplechap1|samplechap2>
%\fi
%
%\iffalse
%<*samplechap1>
%\fi
% Some text for chapter 1:
%    \begin{macrocode}
\section{one}
some text in chapter one
%    \end{macrocode}

%\iffalse
%</samplechap1>
%\fi
% Some text for chapter 2:
%\iffalse
%<*samplechap2>
%\fi
%    \begin{macrocode}
\section{two}
more text in chapter two
%    \end{macrocode}

%\iffalse
%</samplechap2>
%\fi
%
% %%%%%%%%%%%%%%%%%%%%%%%%%%%%%%%%%%%%%%
% \paragraph{Part Include Files.}
%
% The include files are called |cdocspt3.tex| and |cdocspt4.tex|.
%
%\iffalse
%<*samplepart3|samplepart4>
%\fi

% Optional override for |\version| flag:
%    \begin{macrocode}
%%\providecommand{\version}{final}
%    \end{macrocode}

% Include the main document:
%    \begin{macrocode}
\input{childdoc.def}
\childdocby{cdocsamp}
%    \end{macrocode}

%\iffalse
%</samplepart3|samplepart4>
%\fi
%
%\iffalse
%<*samplepart3>
%\fi
% Some text for part 3:
%    \begin{macrocode}
some text in part three
%    \end{macrocode}

%\iffalse
%</samplepart3>
%\fi
% Some text for part 4:
%\iffalse
%<*samplepart4>
%\fi
%    \begin{macrocode}
more text in part four
%    \end{macrocode}

%\iffalse
%</samplepart4>
%\fi
%
% %%%%%%%%%%%%%%%%%%%%%%%%%%%%%%%%%%%%%%
% \paragraph{Forwarding for a Complete Draft.}
%
% The following forwarding file |cdocsdrf.tex|
% compiles the main document in draft mode:
%\iffalse
%<*sampledraft>
%\fi
%    \begin{macrocode}
\def\version{draft}
\input{childdoc.def}
\childdocforward{cdocsamp}
%    \end{macrocode}

%\iffalse
%</sampledraft>
%\fi
%
% %%%%%%%%%%%%%%%%%%%%%%%%%%%%%%%%%%%%%%
% \paragraph{Forwarding for Final Version of the Chapters.}
%
% The following forwarding files |cdocsfn1.tex| and |cdocsfn2.tex|
% (with identical content)
% compile the final versions of the child documents
% |cdocsch1.tex| and |cdocsch2.tex|, respectively:
%\iffalse
%<*samplefinal>
%\fi
%    \begin{macrocode}
\def\version{final}
\input{childdoc.def}
\childdocforwardprefix[cdocsamp]{cdocsfn}{cdocsch}
%    \end{macrocode}

%\iffalse
%</samplefinal>
%\fi
%
% %%%%%%%%%%%%%%%%%%%%%%%%%%%%%%%%%%%%%%
% \paragraph{Command Line Processing.}
%
% The following three command lines generate the output files
% |cdocscld|, |cdocscl1| and |cdocscl2|
% which should be identical to
% |cdocsdrf|, |cdocsch1| and |cdocsfn2|, respectively:
% \begin{center}
% \begin{tabular}{l}
% |latex -jobname cdocscld \|\\
% |  "\def\version{draft}\input{childdoc.def}\childdocforward{cdocsamp}"|\\
% |latex -jobname cdocscl1 \|\\
% |  "\input{childdoc.def}\childdocforward[cdocsamp]{cdocsch1}"|\\
% |latex -jobname cdocscl2 \|\\
% |  "\def\version{final}\input{childdoc.def}\childdocforward{cdocsch2}"|
% \end{tabular}
% \end{center}
% Note that the trailing backslash on each first line
% merely continues the input to the second line
% (for convenient cut ant paste).
% Furthermore, the command |latex| can be replaced by any
% of its alternative versions such as |pdflatex|.
%
% %%%%%%%%%%%%%%%%%%%%%%%%%%%%%%%%%%%%%%%%%%%%%%%%%%%%%%%%%%%%%%%%%%%%%%%%%%%%%%
% %%%%%%%%%%%%%%%%%%%%%%%%%%%%%%%%%%%%%%%%%%%%%%%%%%%%%%%%%%%%%%%%%%%%%%%%%%%%%%
% \section{Implementation}
%\iffalse
%<*package>
%\fi
%
% This section describes the definitions file |childdoc.def|.

% The definitions cannot be loaded using |\usepackage| or |\RequirePackage|
% which has a mechanism to prevent loading a style file more than once.
% When loading the definitions by means of |\input|
% multiple instances have to be prevented manually:
%\iffalse
%This code needs to be before the `\ProvidesFile' directive
%which is defined at the beginning of this file.
%Therefore it is also placed there and commented out here.
%</package>
%<*discard>
%\fi
%    \begin{macrocode}
\ifdefined\childdocmain\endinput\fi
%    \end{macrocode}
%\iffalse
%</discard>
%<*package>
%\fi
%
% \macro{\ifchilddoc}
% \macro{\ifchilddocmanual}
% The conditional |\ifchilddoc| tells whether a
% child (true) or main (false) document is being compiled.
% The conditional |\ifchilddocmanual| tells whether
% the |\includeonly| mechanism is used (false) or
% the selection of child files must be performed manually (true).
% The definitions initialise to false:
%    \begin{macrocode}
\newif\ifchilddoc
\newif\ifchilddocmanual
%    \end{macrocode}

% \macro{\childdocname}
% \macro{\childdocjob}
% The macro |\childdocname| stores the name of the main document
% to be compiled. The macro |\childdocjob| stores the name of
% the document on which the \LaTeX{} compiler was originally invoked.
% The content of |\jobname| cannot be compared
% to filenames specified in the source due to different catcodes.
% The following code rescans |\jobname|, stores the result
% in |\childdocname| and saves a copy in |\childdocjob|:
%    \begin{macrocode}
\edef\childdocname{\scantokens\expandafter{\jobname\noexpand}}
\let\childdocjob\childdocname
%    \end{macrocode}

% \macro{\childdocdisable}
% The macro |\childdocdisable| prevents the main file
% from being processed more than once.
% At this stage, the main document command |\childdocmain|
% is assumed to be called once again where it should do nothing.
% Any subsequent call to it should prevent
% a secondary processing of the main document
% It overwrites the forwarding commands
% |\childdocof| and |\childdocforward|
% with empty macros to prevent further inclusions of the main document:
%    \begin{macrocode}
\newcommand{\childdocdisable}
{
  \renewcommand{\childdocmain}[1]{\renewcommand{\childdocmain}[1]{\endinput}}
  \renewcommand{\childdocof}[1]{}
  \renewcommand{\childdocby}[2][]{}
  \renewcommand{\childdocforward}[2][]{}
  \renewcommand{\childdocdisable}{}
}
%    \end{macrocode}

% \macro{\childdocmain}
% The macro |\childdocmain| is to be called at the top of the main file
% with nothing or the main filename (without extension) as argument.
% First, it breaks loops.
% If the argument is not empty and does not match |\childdocname|
% (which is set by the first inclusion of |childdoc.def|),
% |\ifchilddoc| is set to true, |\includeonly| is applied to the child file
% and |\jobname| is set to the main file
% (for proper handling of |.aux| files):
%    \begin{macrocode}
\newcommand{\childdocmain}[1]
{
  \childdocdisable\childdocmain{}
  \if?#1?\else
    \begingroup
      \def\childdoctmp{#1}
      \ifx\childdoctmp\childdocname
        \def\childdoctmp{}
      \else
        \def\childdoctmp
        {
          \childdoctrue
          \includeonly{\childdocname}
          \def\childdocjob{#1}
          \def\jobname{#1}
        }
      \fi
      \expandafter
    \endgroup
    \childdoctmp
  \fi
}
%    \end{macrocode}

% \macro{\childdocof}
% The command |\childdocof| redirects
% compilation to the main file |#1|.
%    \begin{macrocode}
\newcommand{\childdocof}[1]
{
  \childdocdisable
  \childdoctrue
  \includeonly{\childdocname}
  \def\jobname{#1}
  \def\childdocjob{#1}
  \input{#1}
}
%    \end{macrocode}

% \macro{\childdocby}
% The command |\childdocby| ....
%    \begin{macrocode}
\newcommand{\childdocby}[2][]
{
  \childdocdisable
  \childdoctrue
  \childdocmanualtrue
  \if?#1?\else
    \def\jobname{#2}
  \fi
  \def\childdocjob{#2}
  \input{#2}
  \endinput
}
%    \end{macrocode}

% \macro{\childdocforward}
% The command |\childdocforward| redirects
% compilation to the main file or
% (if the optional argument is given) a child file.
% Parameters are set as if the main file
% or a child file starting with |\childdocof| was compiled.
% Then compilation is handed over to the main file:
%    \begin{macrocode}
\newcommand{\childdocforward}[2][]
{
  \begingroup
    \if?#1?
      \def\childdoctmp
      {
        \def\childdocname{#2}
        \def\childdocjob{#2}
        \def\jobname{#2}
        \input{#2}
        \endinput
      }
    \else
      \def\childdoctmp
      {
        \childdocdisable
        \def\childdocname{#2}
        \childdoctrue
        \includeonly{#2}
        \def\childdocjob{#1}
        \def\jobname{#1}
        \input{#1}
        \endinput
      }
    \fi
    \expandafter
  \endgroup
  \childdoctmp
}
%    \end{macrocode}

% \macro{\childdocforwardprefix}
% The command |\childdocforwardprefix| redirects
% compilation to the main or a child file by means of a pattern.
% The prefix |#1| in the current filename is replaced by |#2|
% and the suffix of the current filename is kept
% (it is assumed that the filename does not contain the substring `|~~~|'
% which is used as a delimiter).
% Compilation is handed over to the new file by |\childdocforward|:
%    \begin{macrocode}
\newcommand{\childdocforwardprefix}[3][]
{
  \begingroup
    \def\childdocextract #2##1~~~{\def\childdoctmp{\childdocforward[#1]{#3##1}}}
    \expandafter\childdocextract\childdocname~~~
    \expandafter
  \endgroup
  \childdoctmp
}
%    \end{macrocode}

% \macro{\childdoc}
% The deprecated macro |\childdoc| is a legacy version of |\childdocmain|:
%    \begin{macrocode}
\newcommand{\childdoc}{\childdocmain}
%    \end{macrocode}

% \macro{\childdocredirect}
% The deprecated macro |\childdocredirect| is a legacy version
% of |\childdocforward| and |\childdocforwardprefix|:
%    \begin{macrocode}
\newcommand{\childdocredirect}[2][]
{
  \begingroup
    \if?#1?
      \def\childdoctmp{\childdocforward{#2}}
    \else
      \def\childdoctmp{\childdocforwardprefix{#1}{#2}}
    \fi
    \expandafter
  \endgroup
  \childdoctmp
}
%    \end{macrocode}

%\iffalse
%</package>
%\fi
%
\endinput

\childdocof{cdocsamp}
%    \end{macrocode}

%\iffalse
%</samplechap1|samplechap2>
%\fi
%
%\iffalse
%<*samplechap1>
%\fi
% Some text for chapter 1:
%    \begin{macrocode}
\section{one}
some text in chapter one
%    \end{macrocode}

%\iffalse
%</samplechap1>
%\fi
% Some text for chapter 2:
%\iffalse
%<*samplechap2>
%\fi
%    \begin{macrocode}
\section{two}
more text in chapter two
%    \end{macrocode}

%\iffalse
%</samplechap2>
%\fi
%
% %%%%%%%%%%%%%%%%%%%%%%%%%%%%%%%%%%%%%%
% \paragraph{Part Include Files.}
%
% The include files are called |cdocspt3.tex| and |cdocspt4.tex|.
%
%\iffalse
%<*samplepart3|samplepart4>
%\fi

% Optional override for |\version| flag:
%    \begin{macrocode}
%%\providecommand{\version}{final}
%    \end{macrocode}

% Include the main document:
%    \begin{macrocode}
% \iffalse
%
% childdoc.dtx Copyright (C) 2017-2018 Niklas Beisert
%
% This work may be distributed and/or modified under the
% conditions of the LaTeX Project Public License, either version 1.3
% of this license or (at your option) any later version.
% The latest version of this license is in
%   http://www.latex-project.org/lppl.txt
% and version 1.3 or later is part of all distributions of LaTeX
% version 2005/12/01 or later.
%
% This work has the LPPL maintenance status `maintained'.
%
% The Current Maintainer of this work is Niklas Beisert.
%
% This work consists of the files childdoc.dtx and childdoc.ins
% and the derived files childdoc.def and cdocsamp.tex with
% cdocsch1.tex, cdocsch2.tex, cdocsdrf.tex, cdocsfn1.tex, cdocsfn2.tex.
%
%<package>\ifdefined\childdocmain\endinput\fi
%<package>\ProvidesFile{childdoc.def}[2018/12/30 v2.0 child document driver]
%<samplemain>\ProvidesFile{cdocsamp.tex}[2018/12/30 v2.0 sample for childdoc]
%<*driver>
%\ProvidesFile{childdoc.drv}[2018/12/30 v2.0 childdoc reference manual file]
\PassOptionsToClass{10pt,a4paper}{article}
\documentclass{ltxdoc}

\usepackage[margin=35mm]{geometry}
\usepackage{hyperref}
\usepackage{hyperxmp}
\usepackage[usenames]{color}

\hypersetup{colorlinks=true}
\hypersetup{pdfstartview=FitH}
\hypersetup{pdfpagemode=UseNone}
\hypersetup{pdfsource={}}
\hypersetup{pdflang={en-UK}}
\hypersetup{pdfcopyright={Copyright 2017-2018 Niklas Beisert.
  This work may be distributed and/or modified under the
  conditions of the LaTeX Project Public License, either version 1.3
  of this license or (at your option) any later version.}}
\hypersetup{pdflicenseurl={http://www.latex-project.org/lppl.txt}}
\hypersetup{pdfcontactaddress={ETH Zurich, ITP, HIT K,
  Wolfgang-Pauli-Strasse 27}}
\hypersetup{pdfcontactpostcode={8093}}
\hypersetup{pdfcontactcity={Zurich}}
\hypersetup{pdfcontactcountry={Switzerland}}
\hypersetup{pdfcontactemail={nbeisert@itp.phys.ethz.ch}}
\hypersetup{pdfcontacturl={http://people.phys.ethz.ch/\xmptilde nbeisert/}}

\newcommand{\secref}[1]{\hyperref[#1]{section \ref*{#1}}}

\parskip1ex
\parindent0pt
\let\olditemize\itemize
\def\itemize{\olditemize\parskip0pt}

\begin{document}

\title{The \textsf{childdoc} Package}
\hypersetup{pdftitle={The childdoc Package}}
\author{Niklas Beisert\\[2ex]
  Institut f\"ur Theoretische Physik\\
  Eidgen\"ossische Technische Hochschule Z\"urich\\
  Wolfgang-Pauli-Strasse 27, 8093 Z\"urich, Switzerland\\[1ex]
  \href{mailto:nbeisert@itp.phys.ethz.ch}
  {\texttt{nbeisert@itp.phys.ethz.ch}}}
\hypersetup{pdfauthor={Niklas Beisert}}
\hypersetup{pdfsubject={Manual for the LaTeX2e Package childdoc}}
\date{30 December 2018, \textsf{v2.0}}
\maketitle

\begin{abstract}\noindent
\textsf{childdoc} is a \LaTeXe{} package
that enables the direct compilation
of document sections included by |\include|
to individual files.
\end{abstract}

\begingroup
\parskip0ex
\tableofcontents
\endgroup

%%%%%%%%%%%%%%%%%%%%%%%%%%%%%%%%%%%%%%%%%%%%%%%%%%%%%%%%%%%%%%%%%%%%%%%%%%%%%%%%
%%%%%%%%%%%%%%%%%%%%%%%%%%%%%%%%%%%%%%%%%%%%%%%%%%%%%%%%%%%%%%%%%%%%%%%%%%%%%%%%
\section{Introduction}

\LaTeX{} provides a mechanism to structure a large document (such as a book)
into a main file and several child files (containing the chapters)
using the |\include| command.
This mechanism is beneficial for documents
which span hundreds of pages in order to
make the source file(s) more manageable.
Moreover, compilation can be restricted to
selected child files by means of the |\includeonly| command.
The latter feature can be used to reduce the compilation time while editing
(this was significantly more useful in the earlier days of \LaTeX{})
or to generate a smaller document which is easier to navigate.
Another application of |\includeonly| is to generate
documents consisting of selected parts of the complete document.

However, there are a few drawbacks of the plain |\include| mechanism:
\begin{itemize}
\item
The child files cannot be compiled on their own,
they can only be compiled via the main file.
A naive editing environment
(such as a text editor with an option
to have the current file processed by \LaTeX)
may require one to switch to the main file before compiling;
attempting to compile the child file produces errors.
\item
The main file must be modified (each time)
to adjust the |\includeonly| command
to the present needs. This easily leaves the main file in a messy state.
\item
The generated document will always carry the filename
of the main document. This is inconvenient if
several child files are to be compiled and
to be kept for distribution.
\end{itemize}

The present package provides a simple interface
to make child files individually compilable by \LaTeX{}.
Compiling a child file then has the same effect as compiling
the main file with an |\includeonly| command
to select the appropriate child.
Moreover the generated document will carry the name of the child
rather than the main file.
This resolves all three above issues.

This feature is meant to make the editing of books,
thesis documents and lecture notes somewhat more convenient.
However, the package can also be used efficiently for
composing a series of documents (such as exercise sheets)
which are typically distributed individually.
It then assists the author in generating the individual documents
(potentially in different versions)
as well as a document containing the collected series.
Another application is in developing style files
or other kinds of included material
where compilation of the style file could redirect
to a sample or test file.

%%%%%%%%%%%%%%%%%%%%%%%%%%%%%%%%%%%%%%%%%%%%%%%%%%%%%%%%%%%%%%%%%%%%%%%%%%%%%%%%
%%%%%%%%%%%%%%%%%%%%%%%%%%%%%%%%%%%%%%%%%%%%%%%%%%%%%%%%%%%%%%%%%%%%%%%%%%%%%%%%
\section{Usage}

First of all, the package \textsf{childdoc} is \emph{not} a standard
\LaTeXe{} |.sty| style file! Therefore it needs to be invoked in
a non-standard way.

%%%%%%%%%%%%%%%%%%%%%%%%%%%%%%%%%%%%%%%%%%%%%%%%%%%%%%%%%%%%%%%%%%%%%%%%%%%%%%%%
\subsection{Included Files}
\label{sec:include}

%%%%%%%%%%%%%%%%%%%%%%%%%%%%%%%%%%%%%%%%
\DescribeMacro{\childdocmain}
To use the package, add the commands
\begin{center}
\begin{tabular}{l}
|\input{childdoc.def}|\\
|\childdocmain{}|\\
\end{tabular}
\end{center}
at the very top of the main \LaTeX{} file,
in particular \emph{before} the |\documentclass| statement!
The argument of |\childdocmain| should be left empty
(but it must be present).

%%%%%%%%%%%%%%%%%%%%%%%%%%%%%%%%%%%%%%%%
\DescribeMacro{\childdocof}
Furthermore, add the commands
\begin{center}
\begin{tabular}{l}
|\input{childdoc.def}|\\
|\childdocof{|\textit{main}|}|\\
\end{tabular}
\end{center}
at the top of every child file \textit{child}
which is included by |\include{|\textit{child}|}|
from within the main file
(or at least for those files to be compiled individually).
The argument \textit{main} must be the filename of the main file.

There are a couple of
considerations in setting up the main and child documents:

%%%%%%%%%%%%%%%%%%%%%%%%%%%%%%%%%%%%%%%%
\paragraph{Restrictions.}

Please note the following restrictions:
\begin{itemize}
\item
|\childdocmain| must be called with one argument \textit{main}
to ensure compatibility with earlier version of the package.
It must either be empty (|\childdocmain{}|)
or precisely match the filename of the main file in which it is specified.
See \secref{sec:detection} for further information.
\item
The filename \textit{main} must be specified without the |.tex| extension.
\item
The filename \textit{main} is case sensitive
(even in case-insensitive file systems)
due to internal string comparison.
\item
The argument \textit{main} should be fully expanded, it cannot be a macro.
\item
Subdirectories and special characters should be avoided in filenames.
\item
The command |\childdocmain{|\textit{main}|}| must be followed by a whitespace.
It should not be followed immediately by another command
or by a comment mark `|%|'.
This is because the \TeX{} parser reads the token immediately following
the argument of |\childdocmain| and puts it
at the beginning of every child section;
however, a white\-space is ignored.
\end{itemize}

%%%%%%%%%%%%%%%%%%%%%%%%%%%%%%%%%%%%%%%%
\paragraph{Content of Main File.}

It is advisable to place all content in the child files included by |\include|.
Any output contained in the main file will appear in all child documents
unless suppressed manually;
it cannot be suppressed automatically by the |\includeonly| directive
and thus should normally be avoided.
A method to include some content in the main file
by means of conditional processing is described in \secref{sec:conditional}.

%%%%%%%%%%%%%%%%%%%%%%%%%%%%%%%%%%%%%%%%
\paragraph{Page Numbering.}

When only a part of the document is compiled,
the appropriate numbering of pages
(as well as other status parameters)
is determined from the |.aux| files.
The latter contain information from previous passes.
However this information needs to propagate through
all intermediate child documents.
Therefore the page numbering in child documents may well
be inconsistent until the complete document is compiled at least once.

A useful (if unconventional) way to always ensure a consistent
page numbering is to restart the numbering in each child document
and denote the pages by `\textit{child}|.|\textit{page}'
where \textit{child} represents the chapter/section number of the child file.
This can be achieved by the command
|\numberwithin{page}{|\textit{child}|}|
of the \textsf{amsmath} package
where \textit{child} can be |chapter| or |section|
depending on the chosen structuring.
Alternatively, one can modify the macro |\thepage| appropriately
and reset the counter |page| at the start of each child file.

%%%%%%%%%%%%%%%%%%%%%%%%%%%%%%%%%%%%%%%%%%%%%%%%%%%%%%%%%%%%%%%%%%%%%%%%%%%%%%%%
\subsection{Conditional Processing}
\label{sec:conditional}

The package provides a mechanism to compile different versions
of a document. To customise the versions further some conditional processing
can come in handy to distinguish which version is being compiled.
The package provides two macros to describe the compilation context:

%%%%%%%%%%%%%%%%%%%%%%%%%%%%%%%%%%%%%%%%
\DescribeMacro{\ifchilddoc}
The conditional |\ifchilddoc| distinguishes between the compilation of
child documents and the main document:
%
\begin{center}
|\ifchilddoc |\textit{child-code}| |[|\||else |\textit{main-code}]| \||fi|
\end{center}

%%%%%%%%%%%%%%%%%%%%%%%%%%%%%%%%%%%%%%%%
\DescribeMacro{\childdocname}
\DescribeMacro{\childdocjob}
The macro |\childdocname| contains the filename (without extension)
of the main or child file being processed.
Note that |\childdocjob| will always contain the name of the main file.

%%%%%%%%%%%%%%%%%%%%%%%%%%%%%%%%%%%%%%%%
\paragraph{Title Page.}

Conditional processing can be used to include a title or banner page
in the main document when proper precautions are taken.
Importantly, the code in the main file should ensure that the page counter
(as well as other status parameters which are stored in the |.aux| files)
takes the same value after the conditional processing.
Otherwise the page numbers may take divergent values
depending on which part is compiled.

For example, a title page could be declared by:
%
\begin{center}
\begin{tabular}{l}
|\ifchilddoc\||else|\\
|\addtocounter{page}{-1}|\\
\textit{code for title page}\\
|\newpage|\\
|\||fi|
\end{tabular}
\end{center}
%
A banner page for the child documents can be generated by:
%
\begin{center}
\begin{tabular}{l}
|\ifchilddoc|\\
|\addtocounter{page}{-1}|\\
\textit{code for banner page}\\
|\newpage|\\
|\||fi|
\end{tabular}
\end{center}
%
Here one could write a message such as:
\begin{center}
|This is the part \childdocname{} of \childdocjob{}.|
\end{center}

%%%%%%%%%%%%%%%%%%%%%%%%%%%%%%%%%%%%%%%%%%%%%%%%%%%%%%%%%%%%%%%%%%%%%%%%%%%%%%%%
\subsection{Flags}
\label{sec:flags}

The package makes it easy to generate different versions
of the main or child documents.
To this end compilation flags can be defined
and assigned different default values.
They will be particularly useful in conjunction
with the forwarding mechanism described in \secref{sec:forward}.

For example, it may be useful to have a flag |\version|
which can be set to |draft| or |final|.
The document source will contain some conditional code
depending on the value of |\version|.
Suppose further, the flag should default to |final| for the main file
and to |draft| for child files
which is a natural assignment for editing the document.
This is achieved by placing the following code
in the preamble of the main document
(below the |\childdocmain| directive):
%
\begin{center}
\begin{tabular}{l}
|\ifchilddoc|\\
|\providecommand{\version}{draft}|\\
|\||else|\\
|\providecommand{\version}{final}|\\
|\||fi|
\end{tabular}
\end{center}
%
The definition by |\providecommand| makes sure
that previous definitions are not overwritten.
Further statements |\providecommand{\version}{...}|
can thus be added before the above code to override it.

For the main file, one might add a line
(between |\childdocmain| and the above block)
%
\begin{center}
|%\ifchilddoc\||else\providecommand{\version}{draft}\||fi|
\end{center}
%
which can be uncommented to produce a draft version.
Likewise one can add a line to the very top of a child file
(above the |\childdocof{|\textit{main}|}| directive)
%
\begin{center}
|%\providecommand{\version}{final}|
\end{center}
%
which can be uncommented to produce the final version of this child document.

%%%%%%%%%%%%%%%%%%%%%%%%%%%%%%%%%%%%%%%%%%%%%%%%%%%%%%%%%%%%%%%%%%%%%%%%%%%%%%%%
\subsection{Forwarding}
\label{sec:forward}

Different versions of the main or child documents
using compilation flags as described in \secref{sec:flags}
can be (permanently) stored in different files
for convenient compilation, viewing and distribution.
To this end, the package defines a command
to pass on compilation to a different file:

%%%%%%%%%%%%%%%%%%%%%%%%%%%%%%%%%%%%%%%%
\DescribeMacro{\childdocforward}
The command |\childdocforward| redirects processing to
another source file:
%
\begin{center}
\begin{tabular}{l}
|\input{childdoc.def}|\\
|\childdocforward[|\textit{main}|]{|\textit{dest}|}|\\
\end{tabular}
\end{center}
%
The argument \textit{dest} is the destination file
(without extension).
It should be the main file or one of the child files.
Note that further \textsf{childdoc} directives
such as |\childdocof| and |\childdocforward|
in the indicated file will be processed in this form.
The optional argument \textit{main}
passes on directly to the main file \textit{main}
while pretending to compile the child \textit{dest}.
This form behaves as if \textit{dest}
issues |\childdocof{|\textit{main}|}| right away,
and no further \textsf{childdoc} directives will be processed.

%%%%%%%%%%%%%%%%%%%%%%%%%%%%%%%%%%%%%%%%
\DescribeMacro{\...prefix}
In the alternative form |\childdocforwardprefix|,
%
\begin{center}
\begin{tabular}{l}
|\input{childdoc.def}|\\
|\childdocforwardprefix[|\textit{main}|]{|\textit{prefix}|}{|\textit{dest}|}|
\end{tabular}
\end{center}
%
the destination file is determined by a pattern
depending on the current file:
To make this work, the current file must be called
`{\textit{prefix}\hspace{0.2em}\textit{suffix}}'
with \textit{prefix} matching precisely the argument.
Processing is then passed on to the file
`{\textit{dest}\hspace{0.2em}\textit{suffix}}'.
Surely, the same effect is achieved by
directly specifying the
argument `{\textit{dest}\hspace{0.2em}\textit{suffix}}'
in the first form.
However, that requires to set up a different file
for each child. With the alternative form of the command
all these files can have exactly the same content
which simplifies setting them up and maintaining them.

For example, the following file |draft.tex|
with a compilation flag |\version| as described in \secref{sec:flags}
compiles the main document as a draft:
%
\begin{center}
\begin{tabular}{l}
|\def\version{draft}|\\
|\input{childdoc.def}|\\
|\childdocforward{|\textit{main}|}|
\end{tabular}
\end{center}
%
Likewise, the following files |final|\textit{nn}|.tex|
compile the final version of the child document
|child|\textit{nn}|.tex|:
%
\begin{center}
\begin{tabular}{l}
|\def\version{final}|\\
|\input{childdoc.def}|\\
|\childdocforwardprefix{final}{child}|
\end{tabular}
\end{center}
%

Note that when several versions of a main file and/or of each child file
are to be generated, it may be convenient to set up a |Makefile| or
shell script to automatise the process.

%%%%%%%%%%%%%%%%%%%%%%%%%%%%%%%%%%%%%%%%%%%%%%%%%%%%%%%%%%%%%%%%%%%%%%%%%%%%%%%%
\subsection{Command Line Processing}
\label{sec:commandline}

The effect of redirection files can also be achieved by invoking
the \LaTeX{} compiler with a more elaborate command line.
Most conveniently this should be done as part
of a shell script or a |Makefile|.

When using \textsf{childdoc} in the main file, the following
command lines effectively perform a redirection
(note that depending on the shell being used,
backslashes may have to be doubled: `|\|' $\to$ `|\\|'):
%
\begin{center}
|... -jobname "|\textit{target}|" |\\|"|[\textit{flags}]%
|\input{childdoc.def}\childdocforward[|\textit{main}|]{|\textit{dest}|}"|
\end{center}
%
Here \textit{target} is the name of the output file,
\textit{main} is the name of the main file
and \textit{dest} is the name of the main or child file to be processed
(all filenames without extensions).
The optional argument \textit{main} can be omitted
if \textit{main} matches \textit{dest}.
Optionally, compilation \textit{flags} can be defined via |\def| commands.
This command line makes the \TeX{} engine believe
it is compiling the file \textit{target}
whose content is specified as the latter parameter.
The provided code then forwards the processing to
\textit{main} or \textit{dest} as described in \secref{sec:forward}.

%%%%%%%%%%%%%%%%%%%%%%%%%%%%%%%%%%%%%%%%%%%%%%%%%%%%%%%%%%%%%%%%%%%%%%%%%%%%%%%%
\subsection{Include by Input}
\label{sec:input}

Including child documents by |\include| has some restrictions by design.
Most notably, the content of a child document always occupies
its own set of pages; pages cannot be shared between child documents.
Usually, this behaviour makes perfect sense
because each child document contain an essential part of the document.
However, in some situations it may be desirable to compose
a document from a collection of parts
without having mandatory page breaks between then.
For this case, the package
provides a mechanism to include parts
by |\input| which can also be processed individually.
However, by construction this mechanism
requires manual handling of the content to be output.

%%%%%%%%%%%%%%%%%%%%%%%%%%%%%%%%%%%%%%%%
\DescribeMacro{\ifchilddocmanual}
The main file should be prepared as usual, see \secref{sec:include}.
However, the document body must make a distinction
between processing of an individual part and of the main document, e.g.:
%
\begin{center}
\begin{tabular}{l}
|\ifchilddocmanual|\\
|\input{\childdocname}|\\
|\||else|\\
\textit{document body with }|\input{|\textit{part}|}|\\
|\||fi|
\end{tabular}
\end{center}
%
The conditional |\ifchilddocmanual| is true whenever
a part to be included by |\input| is being compiled,
and the name of the part is stored in |\childdocname|.

%%%%%%%%%%%%%%%%%%%%%%%%%%%%%%%%%%%%%%%%
\DescribeMacro{\childdocby}
Each part to be included by |\input| should start with:
%
\begin{center}
\begin{tabular}{l}
|\input{childdoc.def}|\\
|\childdocby{|\textit{main}|}|\\
\end{tabular}
\end{center}
%
The directive |\childdocby| is similar to |\childdocof|
described in \secref{sec:include},
but the subsequent selection of content must be done manually.
To that end, both |\ifchilddoc| and |\ifchilddocmanual|
will be true upon processing of a part,
and the name of the part is stored in |\childdocname|.
Note that |\jobname| will be set to the filename of the current part
so that each part receives an individual |.aux| file
that does not interfere with the |.aux| file(s) of the main document.
This behaviour can be altered by the alternative form
|\childdocby[*]{|\textit{main}|}| (with a non-empty optional argument)
which uses the |.aux| file of the main document
by setting |\jobname| to \textit{main}.

%%%%%%%%%%%%%%%%%%%%%%%%%%%%%%%%%%%%%%%%%%%%%%%%%%%%%%%%%%%%%%%%%%%%%%%%%%%%%%%%
\subsection{Driver Development}
\label{sec:driver}

The \textsf{childdoc} mechanism can also be use for the development
of definition files such as \LaTeX{} styles or classes.
This case differs from the above setup with multiple parts
included by |\include| in that no |\includeonly| should be invoked.
This can be achieved by starting the include file
(before |\ProvidesPackage|) with:
%
\begin{center}
\begin{tabular}{l}
|\input{childdoc.def}|\\
|\childdocforward{|\textit{main}|}|\\
\end{tabular}
\end{center}
%
or alternatively with:
%
\begin{center}
\begin{tabular}{l}
|\input{childdoc.def}|\\
|\childdocby{|\textit{main}|}|\\
\end{tabular}
\end{center}
%
Both forms have slightly different effects as described above.
The main file is prepared as usual, see \secref{sec:include}.

%%%%%%%%%%%%%%%%%%%%%%%%%%%%%%%%%%%%%%%%%%%%%%%%%%%%%%%%%%%%%%%%%%%%%%%%%%%%%%%%
\subsection{Legacy Detection}
\label{sec:detection}

The directive |\childdocmain| in the main file can detect
whether the complete document or merely a child is to be compiled
even without using the directive |\childdocof|.
This method is deprecated because it is less robust
and there is no compelling reason to use it;
it is merely provided for backward compatibility
and it may be removed in future versions.

If the detection mechanism is to be used,
it is mandatory to correctly specify
the filename of the main file as the argument of |\childdocmain|:
%
\begin{center}
\begin{tabular}{l}
|\input{childdoc.def}|\\
|\childdocmain{|\textit{main}|}|\\
\end{tabular}
\end{center}
%
If |\jobname| does not match the argument \textit{main} of |\childdocmain|,
it is assumed that |\jobname| points to the child file to be compiled.
When using |\childdocmain| with the main file specified as argument,
it suffices to start a child file
with just |\input{|\textit{main}|}|
without loading of the package and using |\childdocof|.
If instead all processing is done
with the appropriate \textsf{childdoc} directives,
the argument of \textit{main} of |\childdocmain| can be empty.

An alternative version of the command line processing described
in \secref{sec:commandline} using the detection mechanism reads:
%
\begin{center}
|... -jobname "|\textit{target}|" "|[\textit{flags}]%
[|\def\jobname{|\textit{dest}|}|]|\input{|\textit{main}|}"|
\end{center}

%%%%%%%%%%%%%%%%%%%%%%%%%%%%%%%%%%%%%%%%%%%%%%%%%%%%%%%%%%%%%%%%%%%%%%%%%%%%%%%%
\subsection{Manual Code}
\label{sec:manual}

In case one cannot be certain whether the definitions file |childdoc.def|
is installed on the target \TeX{} distribution
and one prefers not to ship it,
it is conceivable to paste a few relevant commands into the sources.

To that end, drop all statements |\input{childdoc.def}|
and perform the replacements as outlined below.
Instead of |\childdocmain{|\textit{main}|}| add the following code
to the top of the main file:
%
\begin{center}
\begin{tabular}{l}
|\||ifdefined\childdocname\endinput\||fi\newif\ifchilddoc|\\
|\edef\childdocname{\scantokens\expandafter{\jobname\noexpand}}|\\
|\def\childdocmain{|\textit{main}|}\||ifx\childdocmain\childdocname\||else|\\
|\childdoctrue\includeonly{\childdocname}\let\jobname\childdocmain\||fi|\\
\end{tabular}
\end{center}
%
Instead of |\childdocof{|\textit{main}|}| just include the main file
at the top of each child file:
%
\begin{center}
|\input{|\textit{main}|}|
\end{center}
%
A simple redirection |\childdocforward{|\textit{dest}|}| is achieved by:
%
\begin{center}
|\def\jobname{|\textit{dest}|}\input{\jobname}|
\end{center}
%
The redirection with prefix
|\childdocforwardprefix[|\textit{prefix}|]{|\textit{dest}|}|
is accomplished by:
%
\begin{center}
\begin{tabular}{l}
|{\edef\jobname{\scantokens\expandafter{\jobname\noexpand}}|\\
|\def\redirectjob |\textit{prefix}|#1~~~{\gdef\jobname{|\textit{dest}|#1}}|\\
|\expandafter\redirectjob\jobname~~~}\input{\jobname}|
\end{tabular}
\end{center}

In an alternative approach,
child documents can be compiled by a specific command line
without additional code or specific definitions:
%
\begin{center}
|... -jobname "|\textit{target}|" "|[\textit{flags}]%
|\includeonly{|\textit{dest}|}\input{|\textit{main}|}"|
\end{center}
%

%%%%%%%%%%%%%%%%%%%%%%%%%%%%%%%%%%%%%%%%%%%%%%%%%%%%%%%%%%%%%%%%%%%%%%%%%%%%%%%%
%%%%%%%%%%%%%%%%%%%%%%%%%%%%%%%%%%%%%%%%%%%%%%%%%%%%%%%%%%%%%%%%%%%%%%%%%%%%%%%%
\section{Information}

%%%%%%%%%%%%%%%%%%%%%%%%%%%%%%%%%%%%%%%%%%%%%%%%%%%%%%%%%%%%%%%%%%%%%%%%%%%%%%%%
\subsection{Copyright}

Copyright \copyright{} 2017--2018 Niklas Beisert

This work may be distributed and/or modified under the
conditions of the \LaTeX{} Project Public License, either version 1.3
of this license or (at your option) any later version.
The latest version of this license is in
  \url{http://www.latex-project.org/lppl.txt}
and version 1.3 or later is part of all distributions of \LaTeX{}
version 2005/12/01 or later.

This work has the LPPL maintenance status `maintained'.

The Current Maintainer of this work is Niklas Beisert.

This work consists of the files |README.txt|, |childdoc.ins| and |childdoc.dtx|
as well as the derived files |childdoc.def|, |cdocsamp.tex|
with |cdocsch1.tex|, |cdocsch2.tex|, |cdocspt3.tex|, |cdocspt4.tex|,
|cdocsdrf.tex|, |cdocsfn1.tex|, |cdocsfn2.tex|
as well as |childdoc.pdf|.

%%%%%%%%%%%%%%%%%%%%%%%%%%%%%%%%%%%%%%%%%%%%%%%%%%%%%%%%%%%%%%%%%%%%%%%%%%%%%%%%
\subsection{Files and Installation}

The package consists of the files:
%
\begin{center}
\begin{tabular}{ll}
    |README.txt|   & readme file \\
    |childdoc.ins| & installation file \\
    |childdoc.dtx| & source file \\
    |childdoc.def| & definition file \\
    |cdocsamp.tex| & sample main file \\
    |cdocsch1.tex| & sample include file \\
    |cdocsch2.tex| & sample include file \\
    |cdocspt3.tex| & sample part file \\
    |cdocspt4.tex| & sample part file \\
    |cdocsdrf.tex| & sample redirection file \\
    |cdocsfn1.tex| & sample redirection file \\
    |cdocsfn2.tex| & sample redirection file \\
    |childdoc.pdf| & manual
\end{tabular}
\end{center}
%
The distribution consists of the files
|README.txt|, |childdoc.ins| and |childdoc.dtx|.
%
\begin{itemize}
\item
Run (pdf)\LaTeX{} on |childdoc.dtx|
to compile the manual |childdoc.pdf| (this file).
\item
Run \LaTeX{} on |childdoc.ins| to create the definitions file |childdoc.def|
and the sample |cdocsamp.tex| with include files
|cdocsch1.tex|, |cdocsch2.tex|, |cdocspt3.tex|, |cdocspt4.tex|,
|cdocsdrf.tex|, |cdocsfn1.tex|, |cdocsfn2.tex|.
Then copy the file |childdoc.def| to an appropriate directory of your \LaTeX{}
distribution, e.g.\ \textit{texmf-root}|/tex/latex/childdoc|.
\end{itemize}

%%%%%%%%%%%%%%%%%%%%%%%%%%%%%%%%%%%%%%%%%%%%%%%%%%%%%%%%%%%%%%%%%%%%%%%%%%%%%%%%
\subsection{Related CTAN Packages}

There are several other packages which offer a similar functionality:
%
\begin{itemize}
\item
The packages
\href{http://ctan.org/pkg/docmute}{\textsf{docmute}},
\href{http://ctan.org/pkg/includex}{\textsf{includex}} and
\href{http://ctan.org/pkg/standalone}{\textsf{standalone}}
provide commands to include only the document body of
a child file thus allowing both files to be compiled individually.
\item
The packages \href{http://ctan.org/pkg/subdocs}{\textsf{subdocs}}
and \href{http://ctan.org/pkg/subfiles}{\textsf{subfiles}}
provide structures in which the main and child documents can be
encapsulated and allowing them to be compiled individually.
The inclusion mechanism is different from the conventional |\include|.
\item
The package \href{http://ctan.org/pkg/combine}{\textsf{combine}}
is an elaborate solution to combine several documents into one.
\end{itemize}
%
See also the CTAN topic \href{http://ctan.org/topic/subdocs}{\textsf{subdocs}}
for further related packages.
The present package differs from the above solutions in that
a document structure constructed with the conventional |\include| mechanism
just needs two extra commands at the top of every file
such that all constituent files can be compiled individually.

%%%%%%%%%%%%%%%%%%%%%%%%%%%%%%%%%%%%%%%%%%%%%%%%%%%%%%%%%%%%%%%%%%%%%%%%%%%%%%%%
%\subsection{Feature Suggestions}
%
%The following is a list of features which may be useful for future
%versions of this package:
%%
%\begin{itemize}
%\item
%\ldots
%\end{itemize}

%%%%%%%%%%%%%%%%%%%%%%%%%%%%%%%%%%%%%%%%%%%%%%%%%%%%%%%%%%%%%%%%%%%%%%%%%%%%%%%%
\subsection{Revision History}

%%%%%%%%%%%%%%%%%%%%%%%%%%%%%%%%%%%%%%%%
\paragraph{v2.0:} 2018/12/30

\begin{itemize}
\item
immediate forward processing
\item
added |\childdocby| mechanism
\item
manual restructured
\end{itemize}

%%%%%%%%%%%%%%%%%%%%%%%%%%%%%%%%%%%%%%%%
\paragraph{v1.6:} 2018/01/17

\begin{itemize}
\item
application for development of include files
\item
corrections to manual
\end{itemize}

%%%%%%%%%%%%%%%%%%%%%%%%%%%%%%%%%%%%%%%%
\paragraph{v1.5:} 2017/05/21

\begin{itemize}
\item
more complete structuring introduced
\item
|\childdocof| introduced
\item
|\childdoc| renamed to |\childdocmain|
\item
|\childredirect| renamed to |\childdocforward| and |\childdocforwardprefix|
and functionality expanded
\end{itemize}

%%%%%%%%%%%%%%%%%%%%%%%%%%%%%%%%%%%%%%%%
\paragraph{v1.0:} 2017/04/27

\begin{itemize}
\item
manual and install package
\item
first version published on CTAN
\end{itemize}

%%%%%%%%%%%%%%%%%%%%%%%%%%%%%%%%%%%%%%%%
\paragraph{v0.6:} 2017/04/26

\begin{itemize}
\item
redirection mechanism added
\end{itemize}

%%%%%%%%%%%%%%%%%%%%%%%%%%%%%%%%%%%%%%%%
\paragraph{v0.5:} 2017/04/26

\begin{itemize}
\item
functionality in definition file
\end{itemize}


%%%%%%%%%%%%%%%%%%%%%%%%%%%%%%%%%%%%%%%%%%%%%%%%%%%%%%%%%%%%%%%%%%%%%%%%%%%%%%%%
%%%%%%%%%%%%%%%%%%%%%%%%%%%%%%%%%%%%%%%%%%%%%%%%%%%%%%%%%%%%%%%%%%%%%%%%%%%%%%%%
%%%%%%%%%%%%%%%%%%%%%%%%%%%%%%%%%%%%%%%%%%%%%%%%%%%%%%%%%%%%%%%%%%%%%%%%%%%%%%%%
\appendix

\settowidth\MacroIndent{\rmfamily\scriptsize 000\ }

 \DocInput{childdoc.dtx}

\end{document}
%</driver>
% \fi
%
% %%%%%%%%%%%%%%%%%%%%%%%%%%%%%%%%%%%%%%%%%%%%%%%%%%%%%%%%%%%%%%%%%%%%%%%%%%%%%%
% %%%%%%%%%%%%%%%%%%%%%%%%%%%%%%%%%%%%%%%%%%%%%%%%%%%%%%%%%%%%%%%%%%%%%%%%%%%%%%
% \section{Sample}
%\iffalse
%<*samplemain>
%\fi
%
% The following presents a sample document
% with two chapters, two parts, a title page,
% a compile flag as well as three forwarding files to set the flag.
% It consists of eight |.tex| files:
% \begin{center}
% \begin{tabular}{ll}
% |cdocsamp.tex|&main file\\
% |cdocsch1.tex|&include file for chapter 1\\
% |cdocsch2.tex|&include file for chapter 2\\
% |cdocspt3.tex|&include file for part 3\\
% |cdocspt4.tex|&include file for part 4\\
% |cdocsdrf.tex|&forwarding file for main file in draft mode\\
% |cdocsfi1.tex|&forwarding file for final version of chapter 1\\
% |cdocsfi2.tex|&forwarding file for final version of chapter 2\\
% \end{tabular}
% \end{center}
% Each of the eight files can be compiled directly by the \LaTeX{} compiler.
%
% %%%%%%%%%%%%%%%%%%%%%%%%%%%%%%%%%%%%%%
% \paragraph{Main File.}
%
% The main file is called |cdocsamp.tex|.
%
% Load the \textsf{childdoc} definitions and
% declare the filename for the main document:
%    \begin{macrocode}
\input{childdoc.def}
\childdocmain{}
%    \end{macrocode}

% Optional override for |\version| flag:
%    \begin{macrocode}
%%\ifchilddoc\else\providecommand{\version}{draft}\fi
%    \end{macrocode}

% Define the default values for the |\version| flag
% (|final| for the main file and |draft| for childs):
%    \begin{macrocode}
\ifchilddoc
\providecommand{\version}{draft}
\else
\providecommand{\version}{final}
\fi
%    \end{macrocode}

% Load the standard document class:
%    \begin{macrocode}
\documentclass[12pt]{article}
%    \end{macrocode}

% Start the document body:
%    \begin{macrocode}
\begin{document}
%    \end{macrocode}

% Declare a title page.
% Print title, part of document being processed and version flag:
%    \begin{macrocode}
\addtocounter{page}{-1}
\begin{center}
{\LARGE\bfseries{}childdoc example\par}
\vspace{1cm}
\ifchilddoc
\ifchilddocmanual part\else chapter\fi:
`\childdocname' of `\childdocjob'\par
\else
main document: `\childdocjob'\par
\fi
version: \version\par
\end{center}
\newpage
%    \end{macrocode}

% Manually include selected file,
% otherwise process as usual:
%    \begin{macrocode}
\ifchilddocmanual
\section*{part `\childdocname'}
\input{\childdocname}
\else
%    \end{macrocode}

% Include the two chapters:
%    \begin{macrocode}
\include{cdocsch1}
\include{cdocsch2}
%    \end{macrocode}

% Include the two parts unless only chapters should be displayed:
%    \begin{macrocode}
\ifchilddoc\else
\section{part three}
\input{cdocspt3}
\section{part four}
\input{cdocspt4}
\fi
%    \end{macrocode}

% Process as usual until here:
%    \begin{macrocode}
\fi
%    \end{macrocode}

% End of document body:
%    \begin{macrocode}
\end{document}
%    \end{macrocode}
%\iffalse
%</samplemain>
%\fi
%
% %%%%%%%%%%%%%%%%%%%%%%%%%%%%%%%%%%%%%%
% \paragraph{Chapter Include Files.}
%
% The include files are called |cdocsch1.tex| and |cdocsch2.tex|.
%
%\iffalse
%<*samplechap1|samplechap2>
%\fi

% Optional override for |\version| flag:
%    \begin{macrocode}
%%\providecommand{\version}{final}
%    \end{macrocode}

% Include the main document:
%    \begin{macrocode}
\input{childdoc.def}
\childdocof{cdocsamp}
%    \end{macrocode}

%\iffalse
%</samplechap1|samplechap2>
%\fi
%
%\iffalse
%<*samplechap1>
%\fi
% Some text for chapter 1:
%    \begin{macrocode}
\section{one}
some text in chapter one
%    \end{macrocode}

%\iffalse
%</samplechap1>
%\fi
% Some text for chapter 2:
%\iffalse
%<*samplechap2>
%\fi
%    \begin{macrocode}
\section{two}
more text in chapter two
%    \end{macrocode}

%\iffalse
%</samplechap2>
%\fi
%
% %%%%%%%%%%%%%%%%%%%%%%%%%%%%%%%%%%%%%%
% \paragraph{Part Include Files.}
%
% The include files are called |cdocspt3.tex| and |cdocspt4.tex|.
%
%\iffalse
%<*samplepart3|samplepart4>
%\fi

% Optional override for |\version| flag:
%    \begin{macrocode}
%%\providecommand{\version}{final}
%    \end{macrocode}

% Include the main document:
%    \begin{macrocode}
\input{childdoc.def}
\childdocby{cdocsamp}
%    \end{macrocode}

%\iffalse
%</samplepart3|samplepart4>
%\fi
%
%\iffalse
%<*samplepart3>
%\fi
% Some text for part 3:
%    \begin{macrocode}
some text in part three
%    \end{macrocode}

%\iffalse
%</samplepart3>
%\fi
% Some text for part 4:
%\iffalse
%<*samplepart4>
%\fi
%    \begin{macrocode}
more text in part four
%    \end{macrocode}

%\iffalse
%</samplepart4>
%\fi
%
% %%%%%%%%%%%%%%%%%%%%%%%%%%%%%%%%%%%%%%
% \paragraph{Forwarding for a Complete Draft.}
%
% The following forwarding file |cdocsdrf.tex|
% compiles the main document in draft mode:
%\iffalse
%<*sampledraft>
%\fi
%    \begin{macrocode}
\def\version{draft}
\input{childdoc.def}
\childdocforward{cdocsamp}
%    \end{macrocode}

%\iffalse
%</sampledraft>
%\fi
%
% %%%%%%%%%%%%%%%%%%%%%%%%%%%%%%%%%%%%%%
% \paragraph{Forwarding for Final Version of the Chapters.}
%
% The following forwarding files |cdocsfn1.tex| and |cdocsfn2.tex|
% (with identical content)
% compile the final versions of the child documents
% |cdocsch1.tex| and |cdocsch2.tex|, respectively:
%\iffalse
%<*samplefinal>
%\fi
%    \begin{macrocode}
\def\version{final}
\input{childdoc.def}
\childdocforwardprefix[cdocsamp]{cdocsfn}{cdocsch}
%    \end{macrocode}

%\iffalse
%</samplefinal>
%\fi
%
% %%%%%%%%%%%%%%%%%%%%%%%%%%%%%%%%%%%%%%
% \paragraph{Command Line Processing.}
%
% The following three command lines generate the output files
% |cdocscld|, |cdocscl1| and |cdocscl2|
% which should be identical to
% |cdocsdrf|, |cdocsch1| and |cdocsfn2|, respectively:
% \begin{center}
% \begin{tabular}{l}
% |latex -jobname cdocscld \|\\
% |  "\def\version{draft}\input{childdoc.def}\childdocforward{cdocsamp}"|\\
% |latex -jobname cdocscl1 \|\\
% |  "\input{childdoc.def}\childdocforward[cdocsamp]{cdocsch1}"|\\
% |latex -jobname cdocscl2 \|\\
% |  "\def\version{final}\input{childdoc.def}\childdocforward{cdocsch2}"|
% \end{tabular}
% \end{center}
% Note that the trailing backslash on each first line
% merely continues the input to the second line
% (for convenient cut ant paste).
% Furthermore, the command |latex| can be replaced by any
% of its alternative versions such as |pdflatex|.
%
% %%%%%%%%%%%%%%%%%%%%%%%%%%%%%%%%%%%%%%%%%%%%%%%%%%%%%%%%%%%%%%%%%%%%%%%%%%%%%%
% %%%%%%%%%%%%%%%%%%%%%%%%%%%%%%%%%%%%%%%%%%%%%%%%%%%%%%%%%%%%%%%%%%%%%%%%%%%%%%
% \section{Implementation}
%\iffalse
%<*package>
%\fi
%
% This section describes the definitions file |childdoc.def|.

% The definitions cannot be loaded using |\usepackage| or |\RequirePackage|
% which has a mechanism to prevent loading a style file more than once.
% When loading the definitions by means of |\input|
% multiple instances have to be prevented manually:
%\iffalse
%This code needs to be before the `\ProvidesFile' directive
%which is defined at the beginning of this file.
%Therefore it is also placed there and commented out here.
%</package>
%<*discard>
%\fi
%    \begin{macrocode}
\ifdefined\childdocmain\endinput\fi
%    \end{macrocode}
%\iffalse
%</discard>
%<*package>
%\fi
%
% \macro{\ifchilddoc}
% \macro{\ifchilddocmanual}
% The conditional |\ifchilddoc| tells whether a
% child (true) or main (false) document is being compiled.
% The conditional |\ifchilddocmanual| tells whether
% the |\includeonly| mechanism is used (false) or
% the selection of child files must be performed manually (true).
% The definitions initialise to false:
%    \begin{macrocode}
\newif\ifchilddoc
\newif\ifchilddocmanual
%    \end{macrocode}

% \macro{\childdocname}
% \macro{\childdocjob}
% The macro |\childdocname| stores the name of the main document
% to be compiled. The macro |\childdocjob| stores the name of
% the document on which the \LaTeX{} compiler was originally invoked.
% The content of |\jobname| cannot be compared
% to filenames specified in the source due to different catcodes.
% The following code rescans |\jobname|, stores the result
% in |\childdocname| and saves a copy in |\childdocjob|:
%    \begin{macrocode}
\edef\childdocname{\scantokens\expandafter{\jobname\noexpand}}
\let\childdocjob\childdocname
%    \end{macrocode}

% \macro{\childdocdisable}
% The macro |\childdocdisable| prevents the main file
% from being processed more than once.
% At this stage, the main document command |\childdocmain|
% is assumed to be called once again where it should do nothing.
% Any subsequent call to it should prevent
% a secondary processing of the main document
% It overwrites the forwarding commands
% |\childdocof| and |\childdocforward|
% with empty macros to prevent further inclusions of the main document:
%    \begin{macrocode}
\newcommand{\childdocdisable}
{
  \renewcommand{\childdocmain}[1]{\renewcommand{\childdocmain}[1]{\endinput}}
  \renewcommand{\childdocof}[1]{}
  \renewcommand{\childdocby}[2][]{}
  \renewcommand{\childdocforward}[2][]{}
  \renewcommand{\childdocdisable}{}
}
%    \end{macrocode}

% \macro{\childdocmain}
% The macro |\childdocmain| is to be called at the top of the main file
% with nothing or the main filename (without extension) as argument.
% First, it breaks loops.
% If the argument is not empty and does not match |\childdocname|
% (which is set by the first inclusion of |childdoc.def|),
% |\ifchilddoc| is set to true, |\includeonly| is applied to the child file
% and |\jobname| is set to the main file
% (for proper handling of |.aux| files):
%    \begin{macrocode}
\newcommand{\childdocmain}[1]
{
  \childdocdisable\childdocmain{}
  \if?#1?\else
    \begingroup
      \def\childdoctmp{#1}
      \ifx\childdoctmp\childdocname
        \def\childdoctmp{}
      \else
        \def\childdoctmp
        {
          \childdoctrue
          \includeonly{\childdocname}
          \def\childdocjob{#1}
          \def\jobname{#1}
        }
      \fi
      \expandafter
    \endgroup
    \childdoctmp
  \fi
}
%    \end{macrocode}

% \macro{\childdocof}
% The command |\childdocof| redirects
% compilation to the main file |#1|.
%    \begin{macrocode}
\newcommand{\childdocof}[1]
{
  \childdocdisable
  \childdoctrue
  \includeonly{\childdocname}
  \def\jobname{#1}
  \def\childdocjob{#1}
  \input{#1}
}
%    \end{macrocode}

% \macro{\childdocby}
% The command |\childdocby| ....
%    \begin{macrocode}
\newcommand{\childdocby}[2][]
{
  \childdocdisable
  \childdoctrue
  \childdocmanualtrue
  \if?#1?\else
    \def\jobname{#2}
  \fi
  \def\childdocjob{#2}
  \input{#2}
  \endinput
}
%    \end{macrocode}

% \macro{\childdocforward}
% The command |\childdocforward| redirects
% compilation to the main file or
% (if the optional argument is given) a child file.
% Parameters are set as if the main file
% or a child file starting with |\childdocof| was compiled.
% Then compilation is handed over to the main file:
%    \begin{macrocode}
\newcommand{\childdocforward}[2][]
{
  \begingroup
    \if?#1?
      \def\childdoctmp
      {
        \def\childdocname{#2}
        \def\childdocjob{#2}
        \def\jobname{#2}
        \input{#2}
        \endinput
      }
    \else
      \def\childdoctmp
      {
        \childdocdisable
        \def\childdocname{#2}
        \childdoctrue
        \includeonly{#2}
        \def\childdocjob{#1}
        \def\jobname{#1}
        \input{#1}
        \endinput
      }
    \fi
    \expandafter
  \endgroup
  \childdoctmp
}
%    \end{macrocode}

% \macro{\childdocforwardprefix}
% The command |\childdocforwardprefix| redirects
% compilation to the main or a child file by means of a pattern.
% The prefix |#1| in the current filename is replaced by |#2|
% and the suffix of the current filename is kept
% (it is assumed that the filename does not contain the substring `|~~~|'
% which is used as a delimiter).
% Compilation is handed over to the new file by |\childdocforward|:
%    \begin{macrocode}
\newcommand{\childdocforwardprefix}[3][]
{
  \begingroup
    \def\childdocextract #2##1~~~{\def\childdoctmp{\childdocforward[#1]{#3##1}}}
    \expandafter\childdocextract\childdocname~~~
    \expandafter
  \endgroup
  \childdoctmp
}
%    \end{macrocode}

% \macro{\childdoc}
% The deprecated macro |\childdoc| is a legacy version of |\childdocmain|:
%    \begin{macrocode}
\newcommand{\childdoc}{\childdocmain}
%    \end{macrocode}

% \macro{\childdocredirect}
% The deprecated macro |\childdocredirect| is a legacy version
% of |\childdocforward| and |\childdocforwardprefix|:
%    \begin{macrocode}
\newcommand{\childdocredirect}[2][]
{
  \begingroup
    \if?#1?
      \def\childdoctmp{\childdocforward{#2}}
    \else
      \def\childdoctmp{\childdocforwardprefix{#1}{#2}}
    \fi
    \expandafter
  \endgroup
  \childdoctmp
}
%    \end{macrocode}

%\iffalse
%</package>
%\fi
%
\endinput

\childdocby{cdocsamp}
%    \end{macrocode}

%\iffalse
%</samplepart3|samplepart4>
%\fi
%
%\iffalse
%<*samplepart3>
%\fi
% Some text for part 3:
%    \begin{macrocode}
some text in part three
%    \end{macrocode}

%\iffalse
%</samplepart3>
%\fi
% Some text for part 4:
%\iffalse
%<*samplepart4>
%\fi
%    \begin{macrocode}
more text in part four
%    \end{macrocode}

%\iffalse
%</samplepart4>
%\fi
%
% %%%%%%%%%%%%%%%%%%%%%%%%%%%%%%%%%%%%%%
% \paragraph{Forwarding for a Complete Draft.}
%
% The following forwarding file |cdocsdrf.tex|
% compiles the main document in draft mode:
%\iffalse
%<*sampledraft>
%\fi
%    \begin{macrocode}
\def\version{draft}
% \iffalse
%
% childdoc.dtx Copyright (C) 2017-2018 Niklas Beisert
%
% This work may be distributed and/or modified under the
% conditions of the LaTeX Project Public License, either version 1.3
% of this license or (at your option) any later version.
% The latest version of this license is in
%   http://www.latex-project.org/lppl.txt
% and version 1.3 or later is part of all distributions of LaTeX
% version 2005/12/01 or later.
%
% This work has the LPPL maintenance status `maintained'.
%
% The Current Maintainer of this work is Niklas Beisert.
%
% This work consists of the files childdoc.dtx and childdoc.ins
% and the derived files childdoc.def and cdocsamp.tex with
% cdocsch1.tex, cdocsch2.tex, cdocsdrf.tex, cdocsfn1.tex, cdocsfn2.tex.
%
%<package>\ifdefined\childdocmain\endinput\fi
%<package>\ProvidesFile{childdoc.def}[2018/12/30 v2.0 child document driver]
%<samplemain>\ProvidesFile{cdocsamp.tex}[2018/12/30 v2.0 sample for childdoc]
%<*driver>
%\ProvidesFile{childdoc.drv}[2018/12/30 v2.0 childdoc reference manual file]
\PassOptionsToClass{10pt,a4paper}{article}
\documentclass{ltxdoc}

\usepackage[margin=35mm]{geometry}
\usepackage{hyperref}
\usepackage{hyperxmp}
\usepackage[usenames]{color}

\hypersetup{colorlinks=true}
\hypersetup{pdfstartview=FitH}
\hypersetup{pdfpagemode=UseNone}
\hypersetup{pdfsource={}}
\hypersetup{pdflang={en-UK}}
\hypersetup{pdfcopyright={Copyright 2017-2018 Niklas Beisert.
  This work may be distributed and/or modified under the
  conditions of the LaTeX Project Public License, either version 1.3
  of this license or (at your option) any later version.}}
\hypersetup{pdflicenseurl={http://www.latex-project.org/lppl.txt}}
\hypersetup{pdfcontactaddress={ETH Zurich, ITP, HIT K,
  Wolfgang-Pauli-Strasse 27}}
\hypersetup{pdfcontactpostcode={8093}}
\hypersetup{pdfcontactcity={Zurich}}
\hypersetup{pdfcontactcountry={Switzerland}}
\hypersetup{pdfcontactemail={nbeisert@itp.phys.ethz.ch}}
\hypersetup{pdfcontacturl={http://people.phys.ethz.ch/\xmptilde nbeisert/}}

\newcommand{\secref}[1]{\hyperref[#1]{section \ref*{#1}}}

\parskip1ex
\parindent0pt
\let\olditemize\itemize
\def\itemize{\olditemize\parskip0pt}

\begin{document}

\title{The \textsf{childdoc} Package}
\hypersetup{pdftitle={The childdoc Package}}
\author{Niklas Beisert\\[2ex]
  Institut f\"ur Theoretische Physik\\
  Eidgen\"ossische Technische Hochschule Z\"urich\\
  Wolfgang-Pauli-Strasse 27, 8093 Z\"urich, Switzerland\\[1ex]
  \href{mailto:nbeisert@itp.phys.ethz.ch}
  {\texttt{nbeisert@itp.phys.ethz.ch}}}
\hypersetup{pdfauthor={Niklas Beisert}}
\hypersetup{pdfsubject={Manual for the LaTeX2e Package childdoc}}
\date{30 December 2018, \textsf{v2.0}}
\maketitle

\begin{abstract}\noindent
\textsf{childdoc} is a \LaTeXe{} package
that enables the direct compilation
of document sections included by |\include|
to individual files.
\end{abstract}

\begingroup
\parskip0ex
\tableofcontents
\endgroup

%%%%%%%%%%%%%%%%%%%%%%%%%%%%%%%%%%%%%%%%%%%%%%%%%%%%%%%%%%%%%%%%%%%%%%%%%%%%%%%%
%%%%%%%%%%%%%%%%%%%%%%%%%%%%%%%%%%%%%%%%%%%%%%%%%%%%%%%%%%%%%%%%%%%%%%%%%%%%%%%%
\section{Introduction}

\LaTeX{} provides a mechanism to structure a large document (such as a book)
into a main file and several child files (containing the chapters)
using the |\include| command.
This mechanism is beneficial for documents
which span hundreds of pages in order to
make the source file(s) more manageable.
Moreover, compilation can be restricted to
selected child files by means of the |\includeonly| command.
The latter feature can be used to reduce the compilation time while editing
(this was significantly more useful in the earlier days of \LaTeX{})
or to generate a smaller document which is easier to navigate.
Another application of |\includeonly| is to generate
documents consisting of selected parts of the complete document.

However, there are a few drawbacks of the plain |\include| mechanism:
\begin{itemize}
\item
The child files cannot be compiled on their own,
they can only be compiled via the main file.
A naive editing environment
(such as a text editor with an option
to have the current file processed by \LaTeX)
may require one to switch to the main file before compiling;
attempting to compile the child file produces errors.
\item
The main file must be modified (each time)
to adjust the |\includeonly| command
to the present needs. This easily leaves the main file in a messy state.
\item
The generated document will always carry the filename
of the main document. This is inconvenient if
several child files are to be compiled and
to be kept for distribution.
\end{itemize}

The present package provides a simple interface
to make child files individually compilable by \LaTeX{}.
Compiling a child file then has the same effect as compiling
the main file with an |\includeonly| command
to select the appropriate child.
Moreover the generated document will carry the name of the child
rather than the main file.
This resolves all three above issues.

This feature is meant to make the editing of books,
thesis documents and lecture notes somewhat more convenient.
However, the package can also be used efficiently for
composing a series of documents (such as exercise sheets)
which are typically distributed individually.
It then assists the author in generating the individual documents
(potentially in different versions)
as well as a document containing the collected series.
Another application is in developing style files
or other kinds of included material
where compilation of the style file could redirect
to a sample or test file.

%%%%%%%%%%%%%%%%%%%%%%%%%%%%%%%%%%%%%%%%%%%%%%%%%%%%%%%%%%%%%%%%%%%%%%%%%%%%%%%%
%%%%%%%%%%%%%%%%%%%%%%%%%%%%%%%%%%%%%%%%%%%%%%%%%%%%%%%%%%%%%%%%%%%%%%%%%%%%%%%%
\section{Usage}

First of all, the package \textsf{childdoc} is \emph{not} a standard
\LaTeXe{} |.sty| style file! Therefore it needs to be invoked in
a non-standard way.

%%%%%%%%%%%%%%%%%%%%%%%%%%%%%%%%%%%%%%%%%%%%%%%%%%%%%%%%%%%%%%%%%%%%%%%%%%%%%%%%
\subsection{Included Files}
\label{sec:include}

%%%%%%%%%%%%%%%%%%%%%%%%%%%%%%%%%%%%%%%%
\DescribeMacro{\childdocmain}
To use the package, add the commands
\begin{center}
\begin{tabular}{l}
|\input{childdoc.def}|\\
|\childdocmain{}|\\
\end{tabular}
\end{center}
at the very top of the main \LaTeX{} file,
in particular \emph{before} the |\documentclass| statement!
The argument of |\childdocmain| should be left empty
(but it must be present).

%%%%%%%%%%%%%%%%%%%%%%%%%%%%%%%%%%%%%%%%
\DescribeMacro{\childdocof}
Furthermore, add the commands
\begin{center}
\begin{tabular}{l}
|\input{childdoc.def}|\\
|\childdocof{|\textit{main}|}|\\
\end{tabular}
\end{center}
at the top of every child file \textit{child}
which is included by |\include{|\textit{child}|}|
from within the main file
(or at least for those files to be compiled individually).
The argument \textit{main} must be the filename of the main file.

There are a couple of
considerations in setting up the main and child documents:

%%%%%%%%%%%%%%%%%%%%%%%%%%%%%%%%%%%%%%%%
\paragraph{Restrictions.}

Please note the following restrictions:
\begin{itemize}
\item
|\childdocmain| must be called with one argument \textit{main}
to ensure compatibility with earlier version of the package.
It must either be empty (|\childdocmain{}|)
or precisely match the filename of the main file in which it is specified.
See \secref{sec:detection} for further information.
\item
The filename \textit{main} must be specified without the |.tex| extension.
\item
The filename \textit{main} is case sensitive
(even in case-insensitive file systems)
due to internal string comparison.
\item
The argument \textit{main} should be fully expanded, it cannot be a macro.
\item
Subdirectories and special characters should be avoided in filenames.
\item
The command |\childdocmain{|\textit{main}|}| must be followed by a whitespace.
It should not be followed immediately by another command
or by a comment mark `|%|'.
This is because the \TeX{} parser reads the token immediately following
the argument of |\childdocmain| and puts it
at the beginning of every child section;
however, a white\-space is ignored.
\end{itemize}

%%%%%%%%%%%%%%%%%%%%%%%%%%%%%%%%%%%%%%%%
\paragraph{Content of Main File.}

It is advisable to place all content in the child files included by |\include|.
Any output contained in the main file will appear in all child documents
unless suppressed manually;
it cannot be suppressed automatically by the |\includeonly| directive
and thus should normally be avoided.
A method to include some content in the main file
by means of conditional processing is described in \secref{sec:conditional}.

%%%%%%%%%%%%%%%%%%%%%%%%%%%%%%%%%%%%%%%%
\paragraph{Page Numbering.}

When only a part of the document is compiled,
the appropriate numbering of pages
(as well as other status parameters)
is determined from the |.aux| files.
The latter contain information from previous passes.
However this information needs to propagate through
all intermediate child documents.
Therefore the page numbering in child documents may well
be inconsistent until the complete document is compiled at least once.

A useful (if unconventional) way to always ensure a consistent
page numbering is to restart the numbering in each child document
and denote the pages by `\textit{child}|.|\textit{page}'
where \textit{child} represents the chapter/section number of the child file.
This can be achieved by the command
|\numberwithin{page}{|\textit{child}|}|
of the \textsf{amsmath} package
where \textit{child} can be |chapter| or |section|
depending on the chosen structuring.
Alternatively, one can modify the macro |\thepage| appropriately
and reset the counter |page| at the start of each child file.

%%%%%%%%%%%%%%%%%%%%%%%%%%%%%%%%%%%%%%%%%%%%%%%%%%%%%%%%%%%%%%%%%%%%%%%%%%%%%%%%
\subsection{Conditional Processing}
\label{sec:conditional}

The package provides a mechanism to compile different versions
of a document. To customise the versions further some conditional processing
can come in handy to distinguish which version is being compiled.
The package provides two macros to describe the compilation context:

%%%%%%%%%%%%%%%%%%%%%%%%%%%%%%%%%%%%%%%%
\DescribeMacro{\ifchilddoc}
The conditional |\ifchilddoc| distinguishes between the compilation of
child documents and the main document:
%
\begin{center}
|\ifchilddoc |\textit{child-code}| |[|\||else |\textit{main-code}]| \||fi|
\end{center}

%%%%%%%%%%%%%%%%%%%%%%%%%%%%%%%%%%%%%%%%
\DescribeMacro{\childdocname}
\DescribeMacro{\childdocjob}
The macro |\childdocname| contains the filename (without extension)
of the main or child file being processed.
Note that |\childdocjob| will always contain the name of the main file.

%%%%%%%%%%%%%%%%%%%%%%%%%%%%%%%%%%%%%%%%
\paragraph{Title Page.}

Conditional processing can be used to include a title or banner page
in the main document when proper precautions are taken.
Importantly, the code in the main file should ensure that the page counter
(as well as other status parameters which are stored in the |.aux| files)
takes the same value after the conditional processing.
Otherwise the page numbers may take divergent values
depending on which part is compiled.

For example, a title page could be declared by:
%
\begin{center}
\begin{tabular}{l}
|\ifchilddoc\||else|\\
|\addtocounter{page}{-1}|\\
\textit{code for title page}\\
|\newpage|\\
|\||fi|
\end{tabular}
\end{center}
%
A banner page for the child documents can be generated by:
%
\begin{center}
\begin{tabular}{l}
|\ifchilddoc|\\
|\addtocounter{page}{-1}|\\
\textit{code for banner page}\\
|\newpage|\\
|\||fi|
\end{tabular}
\end{center}
%
Here one could write a message such as:
\begin{center}
|This is the part \childdocname{} of \childdocjob{}.|
\end{center}

%%%%%%%%%%%%%%%%%%%%%%%%%%%%%%%%%%%%%%%%%%%%%%%%%%%%%%%%%%%%%%%%%%%%%%%%%%%%%%%%
\subsection{Flags}
\label{sec:flags}

The package makes it easy to generate different versions
of the main or child documents.
To this end compilation flags can be defined
and assigned different default values.
They will be particularly useful in conjunction
with the forwarding mechanism described in \secref{sec:forward}.

For example, it may be useful to have a flag |\version|
which can be set to |draft| or |final|.
The document source will contain some conditional code
depending on the value of |\version|.
Suppose further, the flag should default to |final| for the main file
and to |draft| for child files
which is a natural assignment for editing the document.
This is achieved by placing the following code
in the preamble of the main document
(below the |\childdocmain| directive):
%
\begin{center}
\begin{tabular}{l}
|\ifchilddoc|\\
|\providecommand{\version}{draft}|\\
|\||else|\\
|\providecommand{\version}{final}|\\
|\||fi|
\end{tabular}
\end{center}
%
The definition by |\providecommand| makes sure
that previous definitions are not overwritten.
Further statements |\providecommand{\version}{...}|
can thus be added before the above code to override it.

For the main file, one might add a line
(between |\childdocmain| and the above block)
%
\begin{center}
|%\ifchilddoc\||else\providecommand{\version}{draft}\||fi|
\end{center}
%
which can be uncommented to produce a draft version.
Likewise one can add a line to the very top of a child file
(above the |\childdocof{|\textit{main}|}| directive)
%
\begin{center}
|%\providecommand{\version}{final}|
\end{center}
%
which can be uncommented to produce the final version of this child document.

%%%%%%%%%%%%%%%%%%%%%%%%%%%%%%%%%%%%%%%%%%%%%%%%%%%%%%%%%%%%%%%%%%%%%%%%%%%%%%%%
\subsection{Forwarding}
\label{sec:forward}

Different versions of the main or child documents
using compilation flags as described in \secref{sec:flags}
can be (permanently) stored in different files
for convenient compilation, viewing and distribution.
To this end, the package defines a command
to pass on compilation to a different file:

%%%%%%%%%%%%%%%%%%%%%%%%%%%%%%%%%%%%%%%%
\DescribeMacro{\childdocforward}
The command |\childdocforward| redirects processing to
another source file:
%
\begin{center}
\begin{tabular}{l}
|\input{childdoc.def}|\\
|\childdocforward[|\textit{main}|]{|\textit{dest}|}|\\
\end{tabular}
\end{center}
%
The argument \textit{dest} is the destination file
(without extension).
It should be the main file or one of the child files.
Note that further \textsf{childdoc} directives
such as |\childdocof| and |\childdocforward|
in the indicated file will be processed in this form.
The optional argument \textit{main}
passes on directly to the main file \textit{main}
while pretending to compile the child \textit{dest}.
This form behaves as if \textit{dest}
issues |\childdocof{|\textit{main}|}| right away,
and no further \textsf{childdoc} directives will be processed.

%%%%%%%%%%%%%%%%%%%%%%%%%%%%%%%%%%%%%%%%
\DescribeMacro{\...prefix}
In the alternative form |\childdocforwardprefix|,
%
\begin{center}
\begin{tabular}{l}
|\input{childdoc.def}|\\
|\childdocforwardprefix[|\textit{main}|]{|\textit{prefix}|}{|\textit{dest}|}|
\end{tabular}
\end{center}
%
the destination file is determined by a pattern
depending on the current file:
To make this work, the current file must be called
`{\textit{prefix}\hspace{0.2em}\textit{suffix}}'
with \textit{prefix} matching precisely the argument.
Processing is then passed on to the file
`{\textit{dest}\hspace{0.2em}\textit{suffix}}'.
Surely, the same effect is achieved by
directly specifying the
argument `{\textit{dest}\hspace{0.2em}\textit{suffix}}'
in the first form.
However, that requires to set up a different file
for each child. With the alternative form of the command
all these files can have exactly the same content
which simplifies setting them up and maintaining them.

For example, the following file |draft.tex|
with a compilation flag |\version| as described in \secref{sec:flags}
compiles the main document as a draft:
%
\begin{center}
\begin{tabular}{l}
|\def\version{draft}|\\
|\input{childdoc.def}|\\
|\childdocforward{|\textit{main}|}|
\end{tabular}
\end{center}
%
Likewise, the following files |final|\textit{nn}|.tex|
compile the final version of the child document
|child|\textit{nn}|.tex|:
%
\begin{center}
\begin{tabular}{l}
|\def\version{final}|\\
|\input{childdoc.def}|\\
|\childdocforwardprefix{final}{child}|
\end{tabular}
\end{center}
%

Note that when several versions of a main file and/or of each child file
are to be generated, it may be convenient to set up a |Makefile| or
shell script to automatise the process.

%%%%%%%%%%%%%%%%%%%%%%%%%%%%%%%%%%%%%%%%%%%%%%%%%%%%%%%%%%%%%%%%%%%%%%%%%%%%%%%%
\subsection{Command Line Processing}
\label{sec:commandline}

The effect of redirection files can also be achieved by invoking
the \LaTeX{} compiler with a more elaborate command line.
Most conveniently this should be done as part
of a shell script or a |Makefile|.

When using \textsf{childdoc} in the main file, the following
command lines effectively perform a redirection
(note that depending on the shell being used,
backslashes may have to be doubled: `|\|' $\to$ `|\\|'):
%
\begin{center}
|... -jobname "|\textit{target}|" |\\|"|[\textit{flags}]%
|\input{childdoc.def}\childdocforward[|\textit{main}|]{|\textit{dest}|}"|
\end{center}
%
Here \textit{target} is the name of the output file,
\textit{main} is the name of the main file
and \textit{dest} is the name of the main or child file to be processed
(all filenames without extensions).
The optional argument \textit{main} can be omitted
if \textit{main} matches \textit{dest}.
Optionally, compilation \textit{flags} can be defined via |\def| commands.
This command line makes the \TeX{} engine believe
it is compiling the file \textit{target}
whose content is specified as the latter parameter.
The provided code then forwards the processing to
\textit{main} or \textit{dest} as described in \secref{sec:forward}.

%%%%%%%%%%%%%%%%%%%%%%%%%%%%%%%%%%%%%%%%%%%%%%%%%%%%%%%%%%%%%%%%%%%%%%%%%%%%%%%%
\subsection{Include by Input}
\label{sec:input}

Including child documents by |\include| has some restrictions by design.
Most notably, the content of a child document always occupies
its own set of pages; pages cannot be shared between child documents.
Usually, this behaviour makes perfect sense
because each child document contain an essential part of the document.
However, in some situations it may be desirable to compose
a document from a collection of parts
without having mandatory page breaks between then.
For this case, the package
provides a mechanism to include parts
by |\input| which can also be processed individually.
However, by construction this mechanism
requires manual handling of the content to be output.

%%%%%%%%%%%%%%%%%%%%%%%%%%%%%%%%%%%%%%%%
\DescribeMacro{\ifchilddocmanual}
The main file should be prepared as usual, see \secref{sec:include}.
However, the document body must make a distinction
between processing of an individual part and of the main document, e.g.:
%
\begin{center}
\begin{tabular}{l}
|\ifchilddocmanual|\\
|\input{\childdocname}|\\
|\||else|\\
\textit{document body with }|\input{|\textit{part}|}|\\
|\||fi|
\end{tabular}
\end{center}
%
The conditional |\ifchilddocmanual| is true whenever
a part to be included by |\input| is being compiled,
and the name of the part is stored in |\childdocname|.

%%%%%%%%%%%%%%%%%%%%%%%%%%%%%%%%%%%%%%%%
\DescribeMacro{\childdocby}
Each part to be included by |\input| should start with:
%
\begin{center}
\begin{tabular}{l}
|\input{childdoc.def}|\\
|\childdocby{|\textit{main}|}|\\
\end{tabular}
\end{center}
%
The directive |\childdocby| is similar to |\childdocof|
described in \secref{sec:include},
but the subsequent selection of content must be done manually.
To that end, both |\ifchilddoc| and |\ifchilddocmanual|
will be true upon processing of a part,
and the name of the part is stored in |\childdocname|.
Note that |\jobname| will be set to the filename of the current part
so that each part receives an individual |.aux| file
that does not interfere with the |.aux| file(s) of the main document.
This behaviour can be altered by the alternative form
|\childdocby[*]{|\textit{main}|}| (with a non-empty optional argument)
which uses the |.aux| file of the main document
by setting |\jobname| to \textit{main}.

%%%%%%%%%%%%%%%%%%%%%%%%%%%%%%%%%%%%%%%%%%%%%%%%%%%%%%%%%%%%%%%%%%%%%%%%%%%%%%%%
\subsection{Driver Development}
\label{sec:driver}

The \textsf{childdoc} mechanism can also be use for the development
of definition files such as \LaTeX{} styles or classes.
This case differs from the above setup with multiple parts
included by |\include| in that no |\includeonly| should be invoked.
This can be achieved by starting the include file
(before |\ProvidesPackage|) with:
%
\begin{center}
\begin{tabular}{l}
|\input{childdoc.def}|\\
|\childdocforward{|\textit{main}|}|\\
\end{tabular}
\end{center}
%
or alternatively with:
%
\begin{center}
\begin{tabular}{l}
|\input{childdoc.def}|\\
|\childdocby{|\textit{main}|}|\\
\end{tabular}
\end{center}
%
Both forms have slightly different effects as described above.
The main file is prepared as usual, see \secref{sec:include}.

%%%%%%%%%%%%%%%%%%%%%%%%%%%%%%%%%%%%%%%%%%%%%%%%%%%%%%%%%%%%%%%%%%%%%%%%%%%%%%%%
\subsection{Legacy Detection}
\label{sec:detection}

The directive |\childdocmain| in the main file can detect
whether the complete document or merely a child is to be compiled
even without using the directive |\childdocof|.
This method is deprecated because it is less robust
and there is no compelling reason to use it;
it is merely provided for backward compatibility
and it may be removed in future versions.

If the detection mechanism is to be used,
it is mandatory to correctly specify
the filename of the main file as the argument of |\childdocmain|:
%
\begin{center}
\begin{tabular}{l}
|\input{childdoc.def}|\\
|\childdocmain{|\textit{main}|}|\\
\end{tabular}
\end{center}
%
If |\jobname| does not match the argument \textit{main} of |\childdocmain|,
it is assumed that |\jobname| points to the child file to be compiled.
When using |\childdocmain| with the main file specified as argument,
it suffices to start a child file
with just |\input{|\textit{main}|}|
without loading of the package and using |\childdocof|.
If instead all processing is done
with the appropriate \textsf{childdoc} directives,
the argument of \textit{main} of |\childdocmain| can be empty.

An alternative version of the command line processing described
in \secref{sec:commandline} using the detection mechanism reads:
%
\begin{center}
|... -jobname "|\textit{target}|" "|[\textit{flags}]%
[|\def\jobname{|\textit{dest}|}|]|\input{|\textit{main}|}"|
\end{center}

%%%%%%%%%%%%%%%%%%%%%%%%%%%%%%%%%%%%%%%%%%%%%%%%%%%%%%%%%%%%%%%%%%%%%%%%%%%%%%%%
\subsection{Manual Code}
\label{sec:manual}

In case one cannot be certain whether the definitions file |childdoc.def|
is installed on the target \TeX{} distribution
and one prefers not to ship it,
it is conceivable to paste a few relevant commands into the sources.

To that end, drop all statements |\input{childdoc.def}|
and perform the replacements as outlined below.
Instead of |\childdocmain{|\textit{main}|}| add the following code
to the top of the main file:
%
\begin{center}
\begin{tabular}{l}
|\||ifdefined\childdocname\endinput\||fi\newif\ifchilddoc|\\
|\edef\childdocname{\scantokens\expandafter{\jobname\noexpand}}|\\
|\def\childdocmain{|\textit{main}|}\||ifx\childdocmain\childdocname\||else|\\
|\childdoctrue\includeonly{\childdocname}\let\jobname\childdocmain\||fi|\\
\end{tabular}
\end{center}
%
Instead of |\childdocof{|\textit{main}|}| just include the main file
at the top of each child file:
%
\begin{center}
|\input{|\textit{main}|}|
\end{center}
%
A simple redirection |\childdocforward{|\textit{dest}|}| is achieved by:
%
\begin{center}
|\def\jobname{|\textit{dest}|}\input{\jobname}|
\end{center}
%
The redirection with prefix
|\childdocforwardprefix[|\textit{prefix}|]{|\textit{dest}|}|
is accomplished by:
%
\begin{center}
\begin{tabular}{l}
|{\edef\jobname{\scantokens\expandafter{\jobname\noexpand}}|\\
|\def\redirectjob |\textit{prefix}|#1~~~{\gdef\jobname{|\textit{dest}|#1}}|\\
|\expandafter\redirectjob\jobname~~~}\input{\jobname}|
\end{tabular}
\end{center}

In an alternative approach,
child documents can be compiled by a specific command line
without additional code or specific definitions:
%
\begin{center}
|... -jobname "|\textit{target}|" "|[\textit{flags}]%
|\includeonly{|\textit{dest}|}\input{|\textit{main}|}"|
\end{center}
%

%%%%%%%%%%%%%%%%%%%%%%%%%%%%%%%%%%%%%%%%%%%%%%%%%%%%%%%%%%%%%%%%%%%%%%%%%%%%%%%%
%%%%%%%%%%%%%%%%%%%%%%%%%%%%%%%%%%%%%%%%%%%%%%%%%%%%%%%%%%%%%%%%%%%%%%%%%%%%%%%%
\section{Information}

%%%%%%%%%%%%%%%%%%%%%%%%%%%%%%%%%%%%%%%%%%%%%%%%%%%%%%%%%%%%%%%%%%%%%%%%%%%%%%%%
\subsection{Copyright}

Copyright \copyright{} 2017--2018 Niklas Beisert

This work may be distributed and/or modified under the
conditions of the \LaTeX{} Project Public License, either version 1.3
of this license or (at your option) any later version.
The latest version of this license is in
  \url{http://www.latex-project.org/lppl.txt}
and version 1.3 or later is part of all distributions of \LaTeX{}
version 2005/12/01 or later.

This work has the LPPL maintenance status `maintained'.

The Current Maintainer of this work is Niklas Beisert.

This work consists of the files |README.txt|, |childdoc.ins| and |childdoc.dtx|
as well as the derived files |childdoc.def|, |cdocsamp.tex|
with |cdocsch1.tex|, |cdocsch2.tex|, |cdocspt3.tex|, |cdocspt4.tex|,
|cdocsdrf.tex|, |cdocsfn1.tex|, |cdocsfn2.tex|
as well as |childdoc.pdf|.

%%%%%%%%%%%%%%%%%%%%%%%%%%%%%%%%%%%%%%%%%%%%%%%%%%%%%%%%%%%%%%%%%%%%%%%%%%%%%%%%
\subsection{Files and Installation}

The package consists of the files:
%
\begin{center}
\begin{tabular}{ll}
    |README.txt|   & readme file \\
    |childdoc.ins| & installation file \\
    |childdoc.dtx| & source file \\
    |childdoc.def| & definition file \\
    |cdocsamp.tex| & sample main file \\
    |cdocsch1.tex| & sample include file \\
    |cdocsch2.tex| & sample include file \\
    |cdocspt3.tex| & sample part file \\
    |cdocspt4.tex| & sample part file \\
    |cdocsdrf.tex| & sample redirection file \\
    |cdocsfn1.tex| & sample redirection file \\
    |cdocsfn2.tex| & sample redirection file \\
    |childdoc.pdf| & manual
\end{tabular}
\end{center}
%
The distribution consists of the files
|README.txt|, |childdoc.ins| and |childdoc.dtx|.
%
\begin{itemize}
\item
Run (pdf)\LaTeX{} on |childdoc.dtx|
to compile the manual |childdoc.pdf| (this file).
\item
Run \LaTeX{} on |childdoc.ins| to create the definitions file |childdoc.def|
and the sample |cdocsamp.tex| with include files
|cdocsch1.tex|, |cdocsch2.tex|, |cdocspt3.tex|, |cdocspt4.tex|,
|cdocsdrf.tex|, |cdocsfn1.tex|, |cdocsfn2.tex|.
Then copy the file |childdoc.def| to an appropriate directory of your \LaTeX{}
distribution, e.g.\ \textit{texmf-root}|/tex/latex/childdoc|.
\end{itemize}

%%%%%%%%%%%%%%%%%%%%%%%%%%%%%%%%%%%%%%%%%%%%%%%%%%%%%%%%%%%%%%%%%%%%%%%%%%%%%%%%
\subsection{Related CTAN Packages}

There are several other packages which offer a similar functionality:
%
\begin{itemize}
\item
The packages
\href{http://ctan.org/pkg/docmute}{\textsf{docmute}},
\href{http://ctan.org/pkg/includex}{\textsf{includex}} and
\href{http://ctan.org/pkg/standalone}{\textsf{standalone}}
provide commands to include only the document body of
a child file thus allowing both files to be compiled individually.
\item
The packages \href{http://ctan.org/pkg/subdocs}{\textsf{subdocs}}
and \href{http://ctan.org/pkg/subfiles}{\textsf{subfiles}}
provide structures in which the main and child documents can be
encapsulated and allowing them to be compiled individually.
The inclusion mechanism is different from the conventional |\include|.
\item
The package \href{http://ctan.org/pkg/combine}{\textsf{combine}}
is an elaborate solution to combine several documents into one.
\end{itemize}
%
See also the CTAN topic \href{http://ctan.org/topic/subdocs}{\textsf{subdocs}}
for further related packages.
The present package differs from the above solutions in that
a document structure constructed with the conventional |\include| mechanism
just needs two extra commands at the top of every file
such that all constituent files can be compiled individually.

%%%%%%%%%%%%%%%%%%%%%%%%%%%%%%%%%%%%%%%%%%%%%%%%%%%%%%%%%%%%%%%%%%%%%%%%%%%%%%%%
%\subsection{Feature Suggestions}
%
%The following is a list of features which may be useful for future
%versions of this package:
%%
%\begin{itemize}
%\item
%\ldots
%\end{itemize}

%%%%%%%%%%%%%%%%%%%%%%%%%%%%%%%%%%%%%%%%%%%%%%%%%%%%%%%%%%%%%%%%%%%%%%%%%%%%%%%%
\subsection{Revision History}

%%%%%%%%%%%%%%%%%%%%%%%%%%%%%%%%%%%%%%%%
\paragraph{v2.0:} 2018/12/30

\begin{itemize}
\item
immediate forward processing
\item
added |\childdocby| mechanism
\item
manual restructured
\end{itemize}

%%%%%%%%%%%%%%%%%%%%%%%%%%%%%%%%%%%%%%%%
\paragraph{v1.6:} 2018/01/17

\begin{itemize}
\item
application for development of include files
\item
corrections to manual
\end{itemize}

%%%%%%%%%%%%%%%%%%%%%%%%%%%%%%%%%%%%%%%%
\paragraph{v1.5:} 2017/05/21

\begin{itemize}
\item
more complete structuring introduced
\item
|\childdocof| introduced
\item
|\childdoc| renamed to |\childdocmain|
\item
|\childredirect| renamed to |\childdocforward| and |\childdocforwardprefix|
and functionality expanded
\end{itemize}

%%%%%%%%%%%%%%%%%%%%%%%%%%%%%%%%%%%%%%%%
\paragraph{v1.0:} 2017/04/27

\begin{itemize}
\item
manual and install package
\item
first version published on CTAN
\end{itemize}

%%%%%%%%%%%%%%%%%%%%%%%%%%%%%%%%%%%%%%%%
\paragraph{v0.6:} 2017/04/26

\begin{itemize}
\item
redirection mechanism added
\end{itemize}

%%%%%%%%%%%%%%%%%%%%%%%%%%%%%%%%%%%%%%%%
\paragraph{v0.5:} 2017/04/26

\begin{itemize}
\item
functionality in definition file
\end{itemize}


%%%%%%%%%%%%%%%%%%%%%%%%%%%%%%%%%%%%%%%%%%%%%%%%%%%%%%%%%%%%%%%%%%%%%%%%%%%%%%%%
%%%%%%%%%%%%%%%%%%%%%%%%%%%%%%%%%%%%%%%%%%%%%%%%%%%%%%%%%%%%%%%%%%%%%%%%%%%%%%%%
%%%%%%%%%%%%%%%%%%%%%%%%%%%%%%%%%%%%%%%%%%%%%%%%%%%%%%%%%%%%%%%%%%%%%%%%%%%%%%%%
\appendix

\settowidth\MacroIndent{\rmfamily\scriptsize 000\ }

 \DocInput{childdoc.dtx}

\end{document}
%</driver>
% \fi
%
% %%%%%%%%%%%%%%%%%%%%%%%%%%%%%%%%%%%%%%%%%%%%%%%%%%%%%%%%%%%%%%%%%%%%%%%%%%%%%%
% %%%%%%%%%%%%%%%%%%%%%%%%%%%%%%%%%%%%%%%%%%%%%%%%%%%%%%%%%%%%%%%%%%%%%%%%%%%%%%
% \section{Sample}
%\iffalse
%<*samplemain>
%\fi
%
% The following presents a sample document
% with two chapters, two parts, a title page,
% a compile flag as well as three forwarding files to set the flag.
% It consists of eight |.tex| files:
% \begin{center}
% \begin{tabular}{ll}
% |cdocsamp.tex|&main file\\
% |cdocsch1.tex|&include file for chapter 1\\
% |cdocsch2.tex|&include file for chapter 2\\
% |cdocspt3.tex|&include file for part 3\\
% |cdocspt4.tex|&include file for part 4\\
% |cdocsdrf.tex|&forwarding file for main file in draft mode\\
% |cdocsfi1.tex|&forwarding file for final version of chapter 1\\
% |cdocsfi2.tex|&forwarding file for final version of chapter 2\\
% \end{tabular}
% \end{center}
% Each of the eight files can be compiled directly by the \LaTeX{} compiler.
%
% %%%%%%%%%%%%%%%%%%%%%%%%%%%%%%%%%%%%%%
% \paragraph{Main File.}
%
% The main file is called |cdocsamp.tex|.
%
% Load the \textsf{childdoc} definitions and
% declare the filename for the main document:
%    \begin{macrocode}
\input{childdoc.def}
\childdocmain{}
%    \end{macrocode}

% Optional override for |\version| flag:
%    \begin{macrocode}
%%\ifchilddoc\else\providecommand{\version}{draft}\fi
%    \end{macrocode}

% Define the default values for the |\version| flag
% (|final| for the main file and |draft| for childs):
%    \begin{macrocode}
\ifchilddoc
\providecommand{\version}{draft}
\else
\providecommand{\version}{final}
\fi
%    \end{macrocode}

% Load the standard document class:
%    \begin{macrocode}
\documentclass[12pt]{article}
%    \end{macrocode}

% Start the document body:
%    \begin{macrocode}
\begin{document}
%    \end{macrocode}

% Declare a title page.
% Print title, part of document being processed and version flag:
%    \begin{macrocode}
\addtocounter{page}{-1}
\begin{center}
{\LARGE\bfseries{}childdoc example\par}
\vspace{1cm}
\ifchilddoc
\ifchilddocmanual part\else chapter\fi:
`\childdocname' of `\childdocjob'\par
\else
main document: `\childdocjob'\par
\fi
version: \version\par
\end{center}
\newpage
%    \end{macrocode}

% Manually include selected file,
% otherwise process as usual:
%    \begin{macrocode}
\ifchilddocmanual
\section*{part `\childdocname'}
\input{\childdocname}
\else
%    \end{macrocode}

% Include the two chapters:
%    \begin{macrocode}
\include{cdocsch1}
\include{cdocsch2}
%    \end{macrocode}

% Include the two parts unless only chapters should be displayed:
%    \begin{macrocode}
\ifchilddoc\else
\section{part three}
\input{cdocspt3}
\section{part four}
\input{cdocspt4}
\fi
%    \end{macrocode}

% Process as usual until here:
%    \begin{macrocode}
\fi
%    \end{macrocode}

% End of document body:
%    \begin{macrocode}
\end{document}
%    \end{macrocode}
%\iffalse
%</samplemain>
%\fi
%
% %%%%%%%%%%%%%%%%%%%%%%%%%%%%%%%%%%%%%%
% \paragraph{Chapter Include Files.}
%
% The include files are called |cdocsch1.tex| and |cdocsch2.tex|.
%
%\iffalse
%<*samplechap1|samplechap2>
%\fi

% Optional override for |\version| flag:
%    \begin{macrocode}
%%\providecommand{\version}{final}
%    \end{macrocode}

% Include the main document:
%    \begin{macrocode}
\input{childdoc.def}
\childdocof{cdocsamp}
%    \end{macrocode}

%\iffalse
%</samplechap1|samplechap2>
%\fi
%
%\iffalse
%<*samplechap1>
%\fi
% Some text for chapter 1:
%    \begin{macrocode}
\section{one}
some text in chapter one
%    \end{macrocode}

%\iffalse
%</samplechap1>
%\fi
% Some text for chapter 2:
%\iffalse
%<*samplechap2>
%\fi
%    \begin{macrocode}
\section{two}
more text in chapter two
%    \end{macrocode}

%\iffalse
%</samplechap2>
%\fi
%
% %%%%%%%%%%%%%%%%%%%%%%%%%%%%%%%%%%%%%%
% \paragraph{Part Include Files.}
%
% The include files are called |cdocspt3.tex| and |cdocspt4.tex|.
%
%\iffalse
%<*samplepart3|samplepart4>
%\fi

% Optional override for |\version| flag:
%    \begin{macrocode}
%%\providecommand{\version}{final}
%    \end{macrocode}

% Include the main document:
%    \begin{macrocode}
\input{childdoc.def}
\childdocby{cdocsamp}
%    \end{macrocode}

%\iffalse
%</samplepart3|samplepart4>
%\fi
%
%\iffalse
%<*samplepart3>
%\fi
% Some text for part 3:
%    \begin{macrocode}
some text in part three
%    \end{macrocode}

%\iffalse
%</samplepart3>
%\fi
% Some text for part 4:
%\iffalse
%<*samplepart4>
%\fi
%    \begin{macrocode}
more text in part four
%    \end{macrocode}

%\iffalse
%</samplepart4>
%\fi
%
% %%%%%%%%%%%%%%%%%%%%%%%%%%%%%%%%%%%%%%
% \paragraph{Forwarding for a Complete Draft.}
%
% The following forwarding file |cdocsdrf.tex|
% compiles the main document in draft mode:
%\iffalse
%<*sampledraft>
%\fi
%    \begin{macrocode}
\def\version{draft}
\input{childdoc.def}
\childdocforward{cdocsamp}
%    \end{macrocode}

%\iffalse
%</sampledraft>
%\fi
%
% %%%%%%%%%%%%%%%%%%%%%%%%%%%%%%%%%%%%%%
% \paragraph{Forwarding for Final Version of the Chapters.}
%
% The following forwarding files |cdocsfn1.tex| and |cdocsfn2.tex|
% (with identical content)
% compile the final versions of the child documents
% |cdocsch1.tex| and |cdocsch2.tex|, respectively:
%\iffalse
%<*samplefinal>
%\fi
%    \begin{macrocode}
\def\version{final}
\input{childdoc.def}
\childdocforwardprefix[cdocsamp]{cdocsfn}{cdocsch}
%    \end{macrocode}

%\iffalse
%</samplefinal>
%\fi
%
% %%%%%%%%%%%%%%%%%%%%%%%%%%%%%%%%%%%%%%
% \paragraph{Command Line Processing.}
%
% The following three command lines generate the output files
% |cdocscld|, |cdocscl1| and |cdocscl2|
% which should be identical to
% |cdocsdrf|, |cdocsch1| and |cdocsfn2|, respectively:
% \begin{center}
% \begin{tabular}{l}
% |latex -jobname cdocscld \|\\
% |  "\def\version{draft}\input{childdoc.def}\childdocforward{cdocsamp}"|\\
% |latex -jobname cdocscl1 \|\\
% |  "\input{childdoc.def}\childdocforward[cdocsamp]{cdocsch1}"|\\
% |latex -jobname cdocscl2 \|\\
% |  "\def\version{final}\input{childdoc.def}\childdocforward{cdocsch2}"|
% \end{tabular}
% \end{center}
% Note that the trailing backslash on each first line
% merely continues the input to the second line
% (for convenient cut ant paste).
% Furthermore, the command |latex| can be replaced by any
% of its alternative versions such as |pdflatex|.
%
% %%%%%%%%%%%%%%%%%%%%%%%%%%%%%%%%%%%%%%%%%%%%%%%%%%%%%%%%%%%%%%%%%%%%%%%%%%%%%%
% %%%%%%%%%%%%%%%%%%%%%%%%%%%%%%%%%%%%%%%%%%%%%%%%%%%%%%%%%%%%%%%%%%%%%%%%%%%%%%
% \section{Implementation}
%\iffalse
%<*package>
%\fi
%
% This section describes the definitions file |childdoc.def|.

% The definitions cannot be loaded using |\usepackage| or |\RequirePackage|
% which has a mechanism to prevent loading a style file more than once.
% When loading the definitions by means of |\input|
% multiple instances have to be prevented manually:
%\iffalse
%This code needs to be before the `\ProvidesFile' directive
%which is defined at the beginning of this file.
%Therefore it is also placed there and commented out here.
%</package>
%<*discard>
%\fi
%    \begin{macrocode}
\ifdefined\childdocmain\endinput\fi
%    \end{macrocode}
%\iffalse
%</discard>
%<*package>
%\fi
%
% \macro{\ifchilddoc}
% \macro{\ifchilddocmanual}
% The conditional |\ifchilddoc| tells whether a
% child (true) or main (false) document is being compiled.
% The conditional |\ifchilddocmanual| tells whether
% the |\includeonly| mechanism is used (false) or
% the selection of child files must be performed manually (true).
% The definitions initialise to false:
%    \begin{macrocode}
\newif\ifchilddoc
\newif\ifchilddocmanual
%    \end{macrocode}

% \macro{\childdocname}
% \macro{\childdocjob}
% The macro |\childdocname| stores the name of the main document
% to be compiled. The macro |\childdocjob| stores the name of
% the document on which the \LaTeX{} compiler was originally invoked.
% The content of |\jobname| cannot be compared
% to filenames specified in the source due to different catcodes.
% The following code rescans |\jobname|, stores the result
% in |\childdocname| and saves a copy in |\childdocjob|:
%    \begin{macrocode}
\edef\childdocname{\scantokens\expandafter{\jobname\noexpand}}
\let\childdocjob\childdocname
%    \end{macrocode}

% \macro{\childdocdisable}
% The macro |\childdocdisable| prevents the main file
% from being processed more than once.
% At this stage, the main document command |\childdocmain|
% is assumed to be called once again where it should do nothing.
% Any subsequent call to it should prevent
% a secondary processing of the main document
% It overwrites the forwarding commands
% |\childdocof| and |\childdocforward|
% with empty macros to prevent further inclusions of the main document:
%    \begin{macrocode}
\newcommand{\childdocdisable}
{
  \renewcommand{\childdocmain}[1]{\renewcommand{\childdocmain}[1]{\endinput}}
  \renewcommand{\childdocof}[1]{}
  \renewcommand{\childdocby}[2][]{}
  \renewcommand{\childdocforward}[2][]{}
  \renewcommand{\childdocdisable}{}
}
%    \end{macrocode}

% \macro{\childdocmain}
% The macro |\childdocmain| is to be called at the top of the main file
% with nothing or the main filename (without extension) as argument.
% First, it breaks loops.
% If the argument is not empty and does not match |\childdocname|
% (which is set by the first inclusion of |childdoc.def|),
% |\ifchilddoc| is set to true, |\includeonly| is applied to the child file
% and |\jobname| is set to the main file
% (for proper handling of |.aux| files):
%    \begin{macrocode}
\newcommand{\childdocmain}[1]
{
  \childdocdisable\childdocmain{}
  \if?#1?\else
    \begingroup
      \def\childdoctmp{#1}
      \ifx\childdoctmp\childdocname
        \def\childdoctmp{}
      \else
        \def\childdoctmp
        {
          \childdoctrue
          \includeonly{\childdocname}
          \def\childdocjob{#1}
          \def\jobname{#1}
        }
      \fi
      \expandafter
    \endgroup
    \childdoctmp
  \fi
}
%    \end{macrocode}

% \macro{\childdocof}
% The command |\childdocof| redirects
% compilation to the main file |#1|.
%    \begin{macrocode}
\newcommand{\childdocof}[1]
{
  \childdocdisable
  \childdoctrue
  \includeonly{\childdocname}
  \def\jobname{#1}
  \def\childdocjob{#1}
  \input{#1}
}
%    \end{macrocode}

% \macro{\childdocby}
% The command |\childdocby| ....
%    \begin{macrocode}
\newcommand{\childdocby}[2][]
{
  \childdocdisable
  \childdoctrue
  \childdocmanualtrue
  \if?#1?\else
    \def\jobname{#2}
  \fi
  \def\childdocjob{#2}
  \input{#2}
  \endinput
}
%    \end{macrocode}

% \macro{\childdocforward}
% The command |\childdocforward| redirects
% compilation to the main file or
% (if the optional argument is given) a child file.
% Parameters are set as if the main file
% or a child file starting with |\childdocof| was compiled.
% Then compilation is handed over to the main file:
%    \begin{macrocode}
\newcommand{\childdocforward}[2][]
{
  \begingroup
    \if?#1?
      \def\childdoctmp
      {
        \def\childdocname{#2}
        \def\childdocjob{#2}
        \def\jobname{#2}
        \input{#2}
        \endinput
      }
    \else
      \def\childdoctmp
      {
        \childdocdisable
        \def\childdocname{#2}
        \childdoctrue
        \includeonly{#2}
        \def\childdocjob{#1}
        \def\jobname{#1}
        \input{#1}
        \endinput
      }
    \fi
    \expandafter
  \endgroup
  \childdoctmp
}
%    \end{macrocode}

% \macro{\childdocforwardprefix}
% The command |\childdocforwardprefix| redirects
% compilation to the main or a child file by means of a pattern.
% The prefix |#1| in the current filename is replaced by |#2|
% and the suffix of the current filename is kept
% (it is assumed that the filename does not contain the substring `|~~~|'
% which is used as a delimiter).
% Compilation is handed over to the new file by |\childdocforward|:
%    \begin{macrocode}
\newcommand{\childdocforwardprefix}[3][]
{
  \begingroup
    \def\childdocextract #2##1~~~{\def\childdoctmp{\childdocforward[#1]{#3##1}}}
    \expandafter\childdocextract\childdocname~~~
    \expandafter
  \endgroup
  \childdoctmp
}
%    \end{macrocode}

% \macro{\childdoc}
% The deprecated macro |\childdoc| is a legacy version of |\childdocmain|:
%    \begin{macrocode}
\newcommand{\childdoc}{\childdocmain}
%    \end{macrocode}

% \macro{\childdocredirect}
% The deprecated macro |\childdocredirect| is a legacy version
% of |\childdocforward| and |\childdocforwardprefix|:
%    \begin{macrocode}
\newcommand{\childdocredirect}[2][]
{
  \begingroup
    \if?#1?
      \def\childdoctmp{\childdocforward{#2}}
    \else
      \def\childdoctmp{\childdocforwardprefix{#1}{#2}}
    \fi
    \expandafter
  \endgroup
  \childdoctmp
}
%    \end{macrocode}

%\iffalse
%</package>
%\fi
%
\endinput

\childdocforward{cdocsamp}
%    \end{macrocode}

%\iffalse
%</sampledraft>
%\fi
%
% %%%%%%%%%%%%%%%%%%%%%%%%%%%%%%%%%%%%%%
% \paragraph{Forwarding for Final Version of the Chapters.}
%
% The following forwarding files |cdocsfn1.tex| and |cdocsfn2.tex|
% (with identical content)
% compile the final versions of the child documents
% |cdocsch1.tex| and |cdocsch2.tex|, respectively:
%\iffalse
%<*samplefinal>
%\fi
%    \begin{macrocode}
\def\version{final}
% \iffalse
%
% childdoc.dtx Copyright (C) 2017-2018 Niklas Beisert
%
% This work may be distributed and/or modified under the
% conditions of the LaTeX Project Public License, either version 1.3
% of this license or (at your option) any later version.
% The latest version of this license is in
%   http://www.latex-project.org/lppl.txt
% and version 1.3 or later is part of all distributions of LaTeX
% version 2005/12/01 or later.
%
% This work has the LPPL maintenance status `maintained'.
%
% The Current Maintainer of this work is Niklas Beisert.
%
% This work consists of the files childdoc.dtx and childdoc.ins
% and the derived files childdoc.def and cdocsamp.tex with
% cdocsch1.tex, cdocsch2.tex, cdocsdrf.tex, cdocsfn1.tex, cdocsfn2.tex.
%
%<package>\ifdefined\childdocmain\endinput\fi
%<package>\ProvidesFile{childdoc.def}[2018/12/30 v2.0 child document driver]
%<samplemain>\ProvidesFile{cdocsamp.tex}[2018/12/30 v2.0 sample for childdoc]
%<*driver>
%\ProvidesFile{childdoc.drv}[2018/12/30 v2.0 childdoc reference manual file]
\PassOptionsToClass{10pt,a4paper}{article}
\documentclass{ltxdoc}

\usepackage[margin=35mm]{geometry}
\usepackage{hyperref}
\usepackage{hyperxmp}
\usepackage[usenames]{color}

\hypersetup{colorlinks=true}
\hypersetup{pdfstartview=FitH}
\hypersetup{pdfpagemode=UseNone}
\hypersetup{pdfsource={}}
\hypersetup{pdflang={en-UK}}
\hypersetup{pdfcopyright={Copyright 2017-2018 Niklas Beisert.
  This work may be distributed and/or modified under the
  conditions of the LaTeX Project Public License, either version 1.3
  of this license or (at your option) any later version.}}
\hypersetup{pdflicenseurl={http://www.latex-project.org/lppl.txt}}
\hypersetup{pdfcontactaddress={ETH Zurich, ITP, HIT K,
  Wolfgang-Pauli-Strasse 27}}
\hypersetup{pdfcontactpostcode={8093}}
\hypersetup{pdfcontactcity={Zurich}}
\hypersetup{pdfcontactcountry={Switzerland}}
\hypersetup{pdfcontactemail={nbeisert@itp.phys.ethz.ch}}
\hypersetup{pdfcontacturl={http://people.phys.ethz.ch/\xmptilde nbeisert/}}

\newcommand{\secref}[1]{\hyperref[#1]{section \ref*{#1}}}

\parskip1ex
\parindent0pt
\let\olditemize\itemize
\def\itemize{\olditemize\parskip0pt}

\begin{document}

\title{The \textsf{childdoc} Package}
\hypersetup{pdftitle={The childdoc Package}}
\author{Niklas Beisert\\[2ex]
  Institut f\"ur Theoretische Physik\\
  Eidgen\"ossische Technische Hochschule Z\"urich\\
  Wolfgang-Pauli-Strasse 27, 8093 Z\"urich, Switzerland\\[1ex]
  \href{mailto:nbeisert@itp.phys.ethz.ch}
  {\texttt{nbeisert@itp.phys.ethz.ch}}}
\hypersetup{pdfauthor={Niklas Beisert}}
\hypersetup{pdfsubject={Manual for the LaTeX2e Package childdoc}}
\date{30 December 2018, \textsf{v2.0}}
\maketitle

\begin{abstract}\noindent
\textsf{childdoc} is a \LaTeXe{} package
that enables the direct compilation
of document sections included by |\include|
to individual files.
\end{abstract}

\begingroup
\parskip0ex
\tableofcontents
\endgroup

%%%%%%%%%%%%%%%%%%%%%%%%%%%%%%%%%%%%%%%%%%%%%%%%%%%%%%%%%%%%%%%%%%%%%%%%%%%%%%%%
%%%%%%%%%%%%%%%%%%%%%%%%%%%%%%%%%%%%%%%%%%%%%%%%%%%%%%%%%%%%%%%%%%%%%%%%%%%%%%%%
\section{Introduction}

\LaTeX{} provides a mechanism to structure a large document (such as a book)
into a main file and several child files (containing the chapters)
using the |\include| command.
This mechanism is beneficial for documents
which span hundreds of pages in order to
make the source file(s) more manageable.
Moreover, compilation can be restricted to
selected child files by means of the |\includeonly| command.
The latter feature can be used to reduce the compilation time while editing
(this was significantly more useful in the earlier days of \LaTeX{})
or to generate a smaller document which is easier to navigate.
Another application of |\includeonly| is to generate
documents consisting of selected parts of the complete document.

However, there are a few drawbacks of the plain |\include| mechanism:
\begin{itemize}
\item
The child files cannot be compiled on their own,
they can only be compiled via the main file.
A naive editing environment
(such as a text editor with an option
to have the current file processed by \LaTeX)
may require one to switch to the main file before compiling;
attempting to compile the child file produces errors.
\item
The main file must be modified (each time)
to adjust the |\includeonly| command
to the present needs. This easily leaves the main file in a messy state.
\item
The generated document will always carry the filename
of the main document. This is inconvenient if
several child files are to be compiled and
to be kept for distribution.
\end{itemize}

The present package provides a simple interface
to make child files individually compilable by \LaTeX{}.
Compiling a child file then has the same effect as compiling
the main file with an |\includeonly| command
to select the appropriate child.
Moreover the generated document will carry the name of the child
rather than the main file.
This resolves all three above issues.

This feature is meant to make the editing of books,
thesis documents and lecture notes somewhat more convenient.
However, the package can also be used efficiently for
composing a series of documents (such as exercise sheets)
which are typically distributed individually.
It then assists the author in generating the individual documents
(potentially in different versions)
as well as a document containing the collected series.
Another application is in developing style files
or other kinds of included material
where compilation of the style file could redirect
to a sample or test file.

%%%%%%%%%%%%%%%%%%%%%%%%%%%%%%%%%%%%%%%%%%%%%%%%%%%%%%%%%%%%%%%%%%%%%%%%%%%%%%%%
%%%%%%%%%%%%%%%%%%%%%%%%%%%%%%%%%%%%%%%%%%%%%%%%%%%%%%%%%%%%%%%%%%%%%%%%%%%%%%%%
\section{Usage}

First of all, the package \textsf{childdoc} is \emph{not} a standard
\LaTeXe{} |.sty| style file! Therefore it needs to be invoked in
a non-standard way.

%%%%%%%%%%%%%%%%%%%%%%%%%%%%%%%%%%%%%%%%%%%%%%%%%%%%%%%%%%%%%%%%%%%%%%%%%%%%%%%%
\subsection{Included Files}
\label{sec:include}

%%%%%%%%%%%%%%%%%%%%%%%%%%%%%%%%%%%%%%%%
\DescribeMacro{\childdocmain}
To use the package, add the commands
\begin{center}
\begin{tabular}{l}
|\input{childdoc.def}|\\
|\childdocmain{}|\\
\end{tabular}
\end{center}
at the very top of the main \LaTeX{} file,
in particular \emph{before} the |\documentclass| statement!
The argument of |\childdocmain| should be left empty
(but it must be present).

%%%%%%%%%%%%%%%%%%%%%%%%%%%%%%%%%%%%%%%%
\DescribeMacro{\childdocof}
Furthermore, add the commands
\begin{center}
\begin{tabular}{l}
|\input{childdoc.def}|\\
|\childdocof{|\textit{main}|}|\\
\end{tabular}
\end{center}
at the top of every child file \textit{child}
which is included by |\include{|\textit{child}|}|
from within the main file
(or at least for those files to be compiled individually).
The argument \textit{main} must be the filename of the main file.

There are a couple of
considerations in setting up the main and child documents:

%%%%%%%%%%%%%%%%%%%%%%%%%%%%%%%%%%%%%%%%
\paragraph{Restrictions.}

Please note the following restrictions:
\begin{itemize}
\item
|\childdocmain| must be called with one argument \textit{main}
to ensure compatibility with earlier version of the package.
It must either be empty (|\childdocmain{}|)
or precisely match the filename of the main file in which it is specified.
See \secref{sec:detection} for further information.
\item
The filename \textit{main} must be specified without the |.tex| extension.
\item
The filename \textit{main} is case sensitive
(even in case-insensitive file systems)
due to internal string comparison.
\item
The argument \textit{main} should be fully expanded, it cannot be a macro.
\item
Subdirectories and special characters should be avoided in filenames.
\item
The command |\childdocmain{|\textit{main}|}| must be followed by a whitespace.
It should not be followed immediately by another command
or by a comment mark `|%|'.
This is because the \TeX{} parser reads the token immediately following
the argument of |\childdocmain| and puts it
at the beginning of every child section;
however, a white\-space is ignored.
\end{itemize}

%%%%%%%%%%%%%%%%%%%%%%%%%%%%%%%%%%%%%%%%
\paragraph{Content of Main File.}

It is advisable to place all content in the child files included by |\include|.
Any output contained in the main file will appear in all child documents
unless suppressed manually;
it cannot be suppressed automatically by the |\includeonly| directive
and thus should normally be avoided.
A method to include some content in the main file
by means of conditional processing is described in \secref{sec:conditional}.

%%%%%%%%%%%%%%%%%%%%%%%%%%%%%%%%%%%%%%%%
\paragraph{Page Numbering.}

When only a part of the document is compiled,
the appropriate numbering of pages
(as well as other status parameters)
is determined from the |.aux| files.
The latter contain information from previous passes.
However this information needs to propagate through
all intermediate child documents.
Therefore the page numbering in child documents may well
be inconsistent until the complete document is compiled at least once.

A useful (if unconventional) way to always ensure a consistent
page numbering is to restart the numbering in each child document
and denote the pages by `\textit{child}|.|\textit{page}'
where \textit{child} represents the chapter/section number of the child file.
This can be achieved by the command
|\numberwithin{page}{|\textit{child}|}|
of the \textsf{amsmath} package
where \textit{child} can be |chapter| or |section|
depending on the chosen structuring.
Alternatively, one can modify the macro |\thepage| appropriately
and reset the counter |page| at the start of each child file.

%%%%%%%%%%%%%%%%%%%%%%%%%%%%%%%%%%%%%%%%%%%%%%%%%%%%%%%%%%%%%%%%%%%%%%%%%%%%%%%%
\subsection{Conditional Processing}
\label{sec:conditional}

The package provides a mechanism to compile different versions
of a document. To customise the versions further some conditional processing
can come in handy to distinguish which version is being compiled.
The package provides two macros to describe the compilation context:

%%%%%%%%%%%%%%%%%%%%%%%%%%%%%%%%%%%%%%%%
\DescribeMacro{\ifchilddoc}
The conditional |\ifchilddoc| distinguishes between the compilation of
child documents and the main document:
%
\begin{center}
|\ifchilddoc |\textit{child-code}| |[|\||else |\textit{main-code}]| \||fi|
\end{center}

%%%%%%%%%%%%%%%%%%%%%%%%%%%%%%%%%%%%%%%%
\DescribeMacro{\childdocname}
\DescribeMacro{\childdocjob}
The macro |\childdocname| contains the filename (without extension)
of the main or child file being processed.
Note that |\childdocjob| will always contain the name of the main file.

%%%%%%%%%%%%%%%%%%%%%%%%%%%%%%%%%%%%%%%%
\paragraph{Title Page.}

Conditional processing can be used to include a title or banner page
in the main document when proper precautions are taken.
Importantly, the code in the main file should ensure that the page counter
(as well as other status parameters which are stored in the |.aux| files)
takes the same value after the conditional processing.
Otherwise the page numbers may take divergent values
depending on which part is compiled.

For example, a title page could be declared by:
%
\begin{center}
\begin{tabular}{l}
|\ifchilddoc\||else|\\
|\addtocounter{page}{-1}|\\
\textit{code for title page}\\
|\newpage|\\
|\||fi|
\end{tabular}
\end{center}
%
A banner page for the child documents can be generated by:
%
\begin{center}
\begin{tabular}{l}
|\ifchilddoc|\\
|\addtocounter{page}{-1}|\\
\textit{code for banner page}\\
|\newpage|\\
|\||fi|
\end{tabular}
\end{center}
%
Here one could write a message such as:
\begin{center}
|This is the part \childdocname{} of \childdocjob{}.|
\end{center}

%%%%%%%%%%%%%%%%%%%%%%%%%%%%%%%%%%%%%%%%%%%%%%%%%%%%%%%%%%%%%%%%%%%%%%%%%%%%%%%%
\subsection{Flags}
\label{sec:flags}

The package makes it easy to generate different versions
of the main or child documents.
To this end compilation flags can be defined
and assigned different default values.
They will be particularly useful in conjunction
with the forwarding mechanism described in \secref{sec:forward}.

For example, it may be useful to have a flag |\version|
which can be set to |draft| or |final|.
The document source will contain some conditional code
depending on the value of |\version|.
Suppose further, the flag should default to |final| for the main file
and to |draft| for child files
which is a natural assignment for editing the document.
This is achieved by placing the following code
in the preamble of the main document
(below the |\childdocmain| directive):
%
\begin{center}
\begin{tabular}{l}
|\ifchilddoc|\\
|\providecommand{\version}{draft}|\\
|\||else|\\
|\providecommand{\version}{final}|\\
|\||fi|
\end{tabular}
\end{center}
%
The definition by |\providecommand| makes sure
that previous definitions are not overwritten.
Further statements |\providecommand{\version}{...}|
can thus be added before the above code to override it.

For the main file, one might add a line
(between |\childdocmain| and the above block)
%
\begin{center}
|%\ifchilddoc\||else\providecommand{\version}{draft}\||fi|
\end{center}
%
which can be uncommented to produce a draft version.
Likewise one can add a line to the very top of a child file
(above the |\childdocof{|\textit{main}|}| directive)
%
\begin{center}
|%\providecommand{\version}{final}|
\end{center}
%
which can be uncommented to produce the final version of this child document.

%%%%%%%%%%%%%%%%%%%%%%%%%%%%%%%%%%%%%%%%%%%%%%%%%%%%%%%%%%%%%%%%%%%%%%%%%%%%%%%%
\subsection{Forwarding}
\label{sec:forward}

Different versions of the main or child documents
using compilation flags as described in \secref{sec:flags}
can be (permanently) stored in different files
for convenient compilation, viewing and distribution.
To this end, the package defines a command
to pass on compilation to a different file:

%%%%%%%%%%%%%%%%%%%%%%%%%%%%%%%%%%%%%%%%
\DescribeMacro{\childdocforward}
The command |\childdocforward| redirects processing to
another source file:
%
\begin{center}
\begin{tabular}{l}
|\input{childdoc.def}|\\
|\childdocforward[|\textit{main}|]{|\textit{dest}|}|\\
\end{tabular}
\end{center}
%
The argument \textit{dest} is the destination file
(without extension).
It should be the main file or one of the child files.
Note that further \textsf{childdoc} directives
such as |\childdocof| and |\childdocforward|
in the indicated file will be processed in this form.
The optional argument \textit{main}
passes on directly to the main file \textit{main}
while pretending to compile the child \textit{dest}.
This form behaves as if \textit{dest}
issues |\childdocof{|\textit{main}|}| right away,
and no further \textsf{childdoc} directives will be processed.

%%%%%%%%%%%%%%%%%%%%%%%%%%%%%%%%%%%%%%%%
\DescribeMacro{\...prefix}
In the alternative form |\childdocforwardprefix|,
%
\begin{center}
\begin{tabular}{l}
|\input{childdoc.def}|\\
|\childdocforwardprefix[|\textit{main}|]{|\textit{prefix}|}{|\textit{dest}|}|
\end{tabular}
\end{center}
%
the destination file is determined by a pattern
depending on the current file:
To make this work, the current file must be called
`{\textit{prefix}\hspace{0.2em}\textit{suffix}}'
with \textit{prefix} matching precisely the argument.
Processing is then passed on to the file
`{\textit{dest}\hspace{0.2em}\textit{suffix}}'.
Surely, the same effect is achieved by
directly specifying the
argument `{\textit{dest}\hspace{0.2em}\textit{suffix}}'
in the first form.
However, that requires to set up a different file
for each child. With the alternative form of the command
all these files can have exactly the same content
which simplifies setting them up and maintaining them.

For example, the following file |draft.tex|
with a compilation flag |\version| as described in \secref{sec:flags}
compiles the main document as a draft:
%
\begin{center}
\begin{tabular}{l}
|\def\version{draft}|\\
|\input{childdoc.def}|\\
|\childdocforward{|\textit{main}|}|
\end{tabular}
\end{center}
%
Likewise, the following files |final|\textit{nn}|.tex|
compile the final version of the child document
|child|\textit{nn}|.tex|:
%
\begin{center}
\begin{tabular}{l}
|\def\version{final}|\\
|\input{childdoc.def}|\\
|\childdocforwardprefix{final}{child}|
\end{tabular}
\end{center}
%

Note that when several versions of a main file and/or of each child file
are to be generated, it may be convenient to set up a |Makefile| or
shell script to automatise the process.

%%%%%%%%%%%%%%%%%%%%%%%%%%%%%%%%%%%%%%%%%%%%%%%%%%%%%%%%%%%%%%%%%%%%%%%%%%%%%%%%
\subsection{Command Line Processing}
\label{sec:commandline}

The effect of redirection files can also be achieved by invoking
the \LaTeX{} compiler with a more elaborate command line.
Most conveniently this should be done as part
of a shell script or a |Makefile|.

When using \textsf{childdoc} in the main file, the following
command lines effectively perform a redirection
(note that depending on the shell being used,
backslashes may have to be doubled: `|\|' $\to$ `|\\|'):
%
\begin{center}
|... -jobname "|\textit{target}|" |\\|"|[\textit{flags}]%
|\input{childdoc.def}\childdocforward[|\textit{main}|]{|\textit{dest}|}"|
\end{center}
%
Here \textit{target} is the name of the output file,
\textit{main} is the name of the main file
and \textit{dest} is the name of the main or child file to be processed
(all filenames without extensions).
The optional argument \textit{main} can be omitted
if \textit{main} matches \textit{dest}.
Optionally, compilation \textit{flags} can be defined via |\def| commands.
This command line makes the \TeX{} engine believe
it is compiling the file \textit{target}
whose content is specified as the latter parameter.
The provided code then forwards the processing to
\textit{main} or \textit{dest} as described in \secref{sec:forward}.

%%%%%%%%%%%%%%%%%%%%%%%%%%%%%%%%%%%%%%%%%%%%%%%%%%%%%%%%%%%%%%%%%%%%%%%%%%%%%%%%
\subsection{Include by Input}
\label{sec:input}

Including child documents by |\include| has some restrictions by design.
Most notably, the content of a child document always occupies
its own set of pages; pages cannot be shared between child documents.
Usually, this behaviour makes perfect sense
because each child document contain an essential part of the document.
However, in some situations it may be desirable to compose
a document from a collection of parts
without having mandatory page breaks between then.
For this case, the package
provides a mechanism to include parts
by |\input| which can also be processed individually.
However, by construction this mechanism
requires manual handling of the content to be output.

%%%%%%%%%%%%%%%%%%%%%%%%%%%%%%%%%%%%%%%%
\DescribeMacro{\ifchilddocmanual}
The main file should be prepared as usual, see \secref{sec:include}.
However, the document body must make a distinction
between processing of an individual part and of the main document, e.g.:
%
\begin{center}
\begin{tabular}{l}
|\ifchilddocmanual|\\
|\input{\childdocname}|\\
|\||else|\\
\textit{document body with }|\input{|\textit{part}|}|\\
|\||fi|
\end{tabular}
\end{center}
%
The conditional |\ifchilddocmanual| is true whenever
a part to be included by |\input| is being compiled,
and the name of the part is stored in |\childdocname|.

%%%%%%%%%%%%%%%%%%%%%%%%%%%%%%%%%%%%%%%%
\DescribeMacro{\childdocby}
Each part to be included by |\input| should start with:
%
\begin{center}
\begin{tabular}{l}
|\input{childdoc.def}|\\
|\childdocby{|\textit{main}|}|\\
\end{tabular}
\end{center}
%
The directive |\childdocby| is similar to |\childdocof|
described in \secref{sec:include},
but the subsequent selection of content must be done manually.
To that end, both |\ifchilddoc| and |\ifchilddocmanual|
will be true upon processing of a part,
and the name of the part is stored in |\childdocname|.
Note that |\jobname| will be set to the filename of the current part
so that each part receives an individual |.aux| file
that does not interfere with the |.aux| file(s) of the main document.
This behaviour can be altered by the alternative form
|\childdocby[*]{|\textit{main}|}| (with a non-empty optional argument)
which uses the |.aux| file of the main document
by setting |\jobname| to \textit{main}.

%%%%%%%%%%%%%%%%%%%%%%%%%%%%%%%%%%%%%%%%%%%%%%%%%%%%%%%%%%%%%%%%%%%%%%%%%%%%%%%%
\subsection{Driver Development}
\label{sec:driver}

The \textsf{childdoc} mechanism can also be use for the development
of definition files such as \LaTeX{} styles or classes.
This case differs from the above setup with multiple parts
included by |\include| in that no |\includeonly| should be invoked.
This can be achieved by starting the include file
(before |\ProvidesPackage|) with:
%
\begin{center}
\begin{tabular}{l}
|\input{childdoc.def}|\\
|\childdocforward{|\textit{main}|}|\\
\end{tabular}
\end{center}
%
or alternatively with:
%
\begin{center}
\begin{tabular}{l}
|\input{childdoc.def}|\\
|\childdocby{|\textit{main}|}|\\
\end{tabular}
\end{center}
%
Both forms have slightly different effects as described above.
The main file is prepared as usual, see \secref{sec:include}.

%%%%%%%%%%%%%%%%%%%%%%%%%%%%%%%%%%%%%%%%%%%%%%%%%%%%%%%%%%%%%%%%%%%%%%%%%%%%%%%%
\subsection{Legacy Detection}
\label{sec:detection}

The directive |\childdocmain| in the main file can detect
whether the complete document or merely a child is to be compiled
even without using the directive |\childdocof|.
This method is deprecated because it is less robust
and there is no compelling reason to use it;
it is merely provided for backward compatibility
and it may be removed in future versions.

If the detection mechanism is to be used,
it is mandatory to correctly specify
the filename of the main file as the argument of |\childdocmain|:
%
\begin{center}
\begin{tabular}{l}
|\input{childdoc.def}|\\
|\childdocmain{|\textit{main}|}|\\
\end{tabular}
\end{center}
%
If |\jobname| does not match the argument \textit{main} of |\childdocmain|,
it is assumed that |\jobname| points to the child file to be compiled.
When using |\childdocmain| with the main file specified as argument,
it suffices to start a child file
with just |\input{|\textit{main}|}|
without loading of the package and using |\childdocof|.
If instead all processing is done
with the appropriate \textsf{childdoc} directives,
the argument of \textit{main} of |\childdocmain| can be empty.

An alternative version of the command line processing described
in \secref{sec:commandline} using the detection mechanism reads:
%
\begin{center}
|... -jobname "|\textit{target}|" "|[\textit{flags}]%
[|\def\jobname{|\textit{dest}|}|]|\input{|\textit{main}|}"|
\end{center}

%%%%%%%%%%%%%%%%%%%%%%%%%%%%%%%%%%%%%%%%%%%%%%%%%%%%%%%%%%%%%%%%%%%%%%%%%%%%%%%%
\subsection{Manual Code}
\label{sec:manual}

In case one cannot be certain whether the definitions file |childdoc.def|
is installed on the target \TeX{} distribution
and one prefers not to ship it,
it is conceivable to paste a few relevant commands into the sources.

To that end, drop all statements |\input{childdoc.def}|
and perform the replacements as outlined below.
Instead of |\childdocmain{|\textit{main}|}| add the following code
to the top of the main file:
%
\begin{center}
\begin{tabular}{l}
|\||ifdefined\childdocname\endinput\||fi\newif\ifchilddoc|\\
|\edef\childdocname{\scantokens\expandafter{\jobname\noexpand}}|\\
|\def\childdocmain{|\textit{main}|}\||ifx\childdocmain\childdocname\||else|\\
|\childdoctrue\includeonly{\childdocname}\let\jobname\childdocmain\||fi|\\
\end{tabular}
\end{center}
%
Instead of |\childdocof{|\textit{main}|}| just include the main file
at the top of each child file:
%
\begin{center}
|\input{|\textit{main}|}|
\end{center}
%
A simple redirection |\childdocforward{|\textit{dest}|}| is achieved by:
%
\begin{center}
|\def\jobname{|\textit{dest}|}\input{\jobname}|
\end{center}
%
The redirection with prefix
|\childdocforwardprefix[|\textit{prefix}|]{|\textit{dest}|}|
is accomplished by:
%
\begin{center}
\begin{tabular}{l}
|{\edef\jobname{\scantokens\expandafter{\jobname\noexpand}}|\\
|\def\redirectjob |\textit{prefix}|#1~~~{\gdef\jobname{|\textit{dest}|#1}}|\\
|\expandafter\redirectjob\jobname~~~}\input{\jobname}|
\end{tabular}
\end{center}

In an alternative approach,
child documents can be compiled by a specific command line
without additional code or specific definitions:
%
\begin{center}
|... -jobname "|\textit{target}|" "|[\textit{flags}]%
|\includeonly{|\textit{dest}|}\input{|\textit{main}|}"|
\end{center}
%

%%%%%%%%%%%%%%%%%%%%%%%%%%%%%%%%%%%%%%%%%%%%%%%%%%%%%%%%%%%%%%%%%%%%%%%%%%%%%%%%
%%%%%%%%%%%%%%%%%%%%%%%%%%%%%%%%%%%%%%%%%%%%%%%%%%%%%%%%%%%%%%%%%%%%%%%%%%%%%%%%
\section{Information}

%%%%%%%%%%%%%%%%%%%%%%%%%%%%%%%%%%%%%%%%%%%%%%%%%%%%%%%%%%%%%%%%%%%%%%%%%%%%%%%%
\subsection{Copyright}

Copyright \copyright{} 2017--2018 Niklas Beisert

This work may be distributed and/or modified under the
conditions of the \LaTeX{} Project Public License, either version 1.3
of this license or (at your option) any later version.
The latest version of this license is in
  \url{http://www.latex-project.org/lppl.txt}
and version 1.3 or later is part of all distributions of \LaTeX{}
version 2005/12/01 or later.

This work has the LPPL maintenance status `maintained'.

The Current Maintainer of this work is Niklas Beisert.

This work consists of the files |README.txt|, |childdoc.ins| and |childdoc.dtx|
as well as the derived files |childdoc.def|, |cdocsamp.tex|
with |cdocsch1.tex|, |cdocsch2.tex|, |cdocspt3.tex|, |cdocspt4.tex|,
|cdocsdrf.tex|, |cdocsfn1.tex|, |cdocsfn2.tex|
as well as |childdoc.pdf|.

%%%%%%%%%%%%%%%%%%%%%%%%%%%%%%%%%%%%%%%%%%%%%%%%%%%%%%%%%%%%%%%%%%%%%%%%%%%%%%%%
\subsection{Files and Installation}

The package consists of the files:
%
\begin{center}
\begin{tabular}{ll}
    |README.txt|   & readme file \\
    |childdoc.ins| & installation file \\
    |childdoc.dtx| & source file \\
    |childdoc.def| & definition file \\
    |cdocsamp.tex| & sample main file \\
    |cdocsch1.tex| & sample include file \\
    |cdocsch2.tex| & sample include file \\
    |cdocspt3.tex| & sample part file \\
    |cdocspt4.tex| & sample part file \\
    |cdocsdrf.tex| & sample redirection file \\
    |cdocsfn1.tex| & sample redirection file \\
    |cdocsfn2.tex| & sample redirection file \\
    |childdoc.pdf| & manual
\end{tabular}
\end{center}
%
The distribution consists of the files
|README.txt|, |childdoc.ins| and |childdoc.dtx|.
%
\begin{itemize}
\item
Run (pdf)\LaTeX{} on |childdoc.dtx|
to compile the manual |childdoc.pdf| (this file).
\item
Run \LaTeX{} on |childdoc.ins| to create the definitions file |childdoc.def|
and the sample |cdocsamp.tex| with include files
|cdocsch1.tex|, |cdocsch2.tex|, |cdocspt3.tex|, |cdocspt4.tex|,
|cdocsdrf.tex|, |cdocsfn1.tex|, |cdocsfn2.tex|.
Then copy the file |childdoc.def| to an appropriate directory of your \LaTeX{}
distribution, e.g.\ \textit{texmf-root}|/tex/latex/childdoc|.
\end{itemize}

%%%%%%%%%%%%%%%%%%%%%%%%%%%%%%%%%%%%%%%%%%%%%%%%%%%%%%%%%%%%%%%%%%%%%%%%%%%%%%%%
\subsection{Related CTAN Packages}

There are several other packages which offer a similar functionality:
%
\begin{itemize}
\item
The packages
\href{http://ctan.org/pkg/docmute}{\textsf{docmute}},
\href{http://ctan.org/pkg/includex}{\textsf{includex}} and
\href{http://ctan.org/pkg/standalone}{\textsf{standalone}}
provide commands to include only the document body of
a child file thus allowing both files to be compiled individually.
\item
The packages \href{http://ctan.org/pkg/subdocs}{\textsf{subdocs}}
and \href{http://ctan.org/pkg/subfiles}{\textsf{subfiles}}
provide structures in which the main and child documents can be
encapsulated and allowing them to be compiled individually.
The inclusion mechanism is different from the conventional |\include|.
\item
The package \href{http://ctan.org/pkg/combine}{\textsf{combine}}
is an elaborate solution to combine several documents into one.
\end{itemize}
%
See also the CTAN topic \href{http://ctan.org/topic/subdocs}{\textsf{subdocs}}
for further related packages.
The present package differs from the above solutions in that
a document structure constructed with the conventional |\include| mechanism
just needs two extra commands at the top of every file
such that all constituent files can be compiled individually.

%%%%%%%%%%%%%%%%%%%%%%%%%%%%%%%%%%%%%%%%%%%%%%%%%%%%%%%%%%%%%%%%%%%%%%%%%%%%%%%%
%\subsection{Feature Suggestions}
%
%The following is a list of features which may be useful for future
%versions of this package:
%%
%\begin{itemize}
%\item
%\ldots
%\end{itemize}

%%%%%%%%%%%%%%%%%%%%%%%%%%%%%%%%%%%%%%%%%%%%%%%%%%%%%%%%%%%%%%%%%%%%%%%%%%%%%%%%
\subsection{Revision History}

%%%%%%%%%%%%%%%%%%%%%%%%%%%%%%%%%%%%%%%%
\paragraph{v2.0:} 2018/12/30

\begin{itemize}
\item
immediate forward processing
\item
added |\childdocby| mechanism
\item
manual restructured
\end{itemize}

%%%%%%%%%%%%%%%%%%%%%%%%%%%%%%%%%%%%%%%%
\paragraph{v1.6:} 2018/01/17

\begin{itemize}
\item
application for development of include files
\item
corrections to manual
\end{itemize}

%%%%%%%%%%%%%%%%%%%%%%%%%%%%%%%%%%%%%%%%
\paragraph{v1.5:} 2017/05/21

\begin{itemize}
\item
more complete structuring introduced
\item
|\childdocof| introduced
\item
|\childdoc| renamed to |\childdocmain|
\item
|\childredirect| renamed to |\childdocforward| and |\childdocforwardprefix|
and functionality expanded
\end{itemize}

%%%%%%%%%%%%%%%%%%%%%%%%%%%%%%%%%%%%%%%%
\paragraph{v1.0:} 2017/04/27

\begin{itemize}
\item
manual and install package
\item
first version published on CTAN
\end{itemize}

%%%%%%%%%%%%%%%%%%%%%%%%%%%%%%%%%%%%%%%%
\paragraph{v0.6:} 2017/04/26

\begin{itemize}
\item
redirection mechanism added
\end{itemize}

%%%%%%%%%%%%%%%%%%%%%%%%%%%%%%%%%%%%%%%%
\paragraph{v0.5:} 2017/04/26

\begin{itemize}
\item
functionality in definition file
\end{itemize}


%%%%%%%%%%%%%%%%%%%%%%%%%%%%%%%%%%%%%%%%%%%%%%%%%%%%%%%%%%%%%%%%%%%%%%%%%%%%%%%%
%%%%%%%%%%%%%%%%%%%%%%%%%%%%%%%%%%%%%%%%%%%%%%%%%%%%%%%%%%%%%%%%%%%%%%%%%%%%%%%%
%%%%%%%%%%%%%%%%%%%%%%%%%%%%%%%%%%%%%%%%%%%%%%%%%%%%%%%%%%%%%%%%%%%%%%%%%%%%%%%%
\appendix

\settowidth\MacroIndent{\rmfamily\scriptsize 000\ }

 \DocInput{childdoc.dtx}

\end{document}
%</driver>
% \fi
%
% %%%%%%%%%%%%%%%%%%%%%%%%%%%%%%%%%%%%%%%%%%%%%%%%%%%%%%%%%%%%%%%%%%%%%%%%%%%%%%
% %%%%%%%%%%%%%%%%%%%%%%%%%%%%%%%%%%%%%%%%%%%%%%%%%%%%%%%%%%%%%%%%%%%%%%%%%%%%%%
% \section{Sample}
%\iffalse
%<*samplemain>
%\fi
%
% The following presents a sample document
% with two chapters, two parts, a title page,
% a compile flag as well as three forwarding files to set the flag.
% It consists of eight |.tex| files:
% \begin{center}
% \begin{tabular}{ll}
% |cdocsamp.tex|&main file\\
% |cdocsch1.tex|&include file for chapter 1\\
% |cdocsch2.tex|&include file for chapter 2\\
% |cdocspt3.tex|&include file for part 3\\
% |cdocspt4.tex|&include file for part 4\\
% |cdocsdrf.tex|&forwarding file for main file in draft mode\\
% |cdocsfi1.tex|&forwarding file for final version of chapter 1\\
% |cdocsfi2.tex|&forwarding file for final version of chapter 2\\
% \end{tabular}
% \end{center}
% Each of the eight files can be compiled directly by the \LaTeX{} compiler.
%
% %%%%%%%%%%%%%%%%%%%%%%%%%%%%%%%%%%%%%%
% \paragraph{Main File.}
%
% The main file is called |cdocsamp.tex|.
%
% Load the \textsf{childdoc} definitions and
% declare the filename for the main document:
%    \begin{macrocode}
\input{childdoc.def}
\childdocmain{}
%    \end{macrocode}

% Optional override for |\version| flag:
%    \begin{macrocode}
%%\ifchilddoc\else\providecommand{\version}{draft}\fi
%    \end{macrocode}

% Define the default values for the |\version| flag
% (|final| for the main file and |draft| for childs):
%    \begin{macrocode}
\ifchilddoc
\providecommand{\version}{draft}
\else
\providecommand{\version}{final}
\fi
%    \end{macrocode}

% Load the standard document class:
%    \begin{macrocode}
\documentclass[12pt]{article}
%    \end{macrocode}

% Start the document body:
%    \begin{macrocode}
\begin{document}
%    \end{macrocode}

% Declare a title page.
% Print title, part of document being processed and version flag:
%    \begin{macrocode}
\addtocounter{page}{-1}
\begin{center}
{\LARGE\bfseries{}childdoc example\par}
\vspace{1cm}
\ifchilddoc
\ifchilddocmanual part\else chapter\fi:
`\childdocname' of `\childdocjob'\par
\else
main document: `\childdocjob'\par
\fi
version: \version\par
\end{center}
\newpage
%    \end{macrocode}

% Manually include selected file,
% otherwise process as usual:
%    \begin{macrocode}
\ifchilddocmanual
\section*{part `\childdocname'}
\input{\childdocname}
\else
%    \end{macrocode}

% Include the two chapters:
%    \begin{macrocode}
\include{cdocsch1}
\include{cdocsch2}
%    \end{macrocode}

% Include the two parts unless only chapters should be displayed:
%    \begin{macrocode}
\ifchilddoc\else
\section{part three}
\input{cdocspt3}
\section{part four}
\input{cdocspt4}
\fi
%    \end{macrocode}

% Process as usual until here:
%    \begin{macrocode}
\fi
%    \end{macrocode}

% End of document body:
%    \begin{macrocode}
\end{document}
%    \end{macrocode}
%\iffalse
%</samplemain>
%\fi
%
% %%%%%%%%%%%%%%%%%%%%%%%%%%%%%%%%%%%%%%
% \paragraph{Chapter Include Files.}
%
% The include files are called |cdocsch1.tex| and |cdocsch2.tex|.
%
%\iffalse
%<*samplechap1|samplechap2>
%\fi

% Optional override for |\version| flag:
%    \begin{macrocode}
%%\providecommand{\version}{final}
%    \end{macrocode}

% Include the main document:
%    \begin{macrocode}
\input{childdoc.def}
\childdocof{cdocsamp}
%    \end{macrocode}

%\iffalse
%</samplechap1|samplechap2>
%\fi
%
%\iffalse
%<*samplechap1>
%\fi
% Some text for chapter 1:
%    \begin{macrocode}
\section{one}
some text in chapter one
%    \end{macrocode}

%\iffalse
%</samplechap1>
%\fi
% Some text for chapter 2:
%\iffalse
%<*samplechap2>
%\fi
%    \begin{macrocode}
\section{two}
more text in chapter two
%    \end{macrocode}

%\iffalse
%</samplechap2>
%\fi
%
% %%%%%%%%%%%%%%%%%%%%%%%%%%%%%%%%%%%%%%
% \paragraph{Part Include Files.}
%
% The include files are called |cdocspt3.tex| and |cdocspt4.tex|.
%
%\iffalse
%<*samplepart3|samplepart4>
%\fi

% Optional override for |\version| flag:
%    \begin{macrocode}
%%\providecommand{\version}{final}
%    \end{macrocode}

% Include the main document:
%    \begin{macrocode}
\input{childdoc.def}
\childdocby{cdocsamp}
%    \end{macrocode}

%\iffalse
%</samplepart3|samplepart4>
%\fi
%
%\iffalse
%<*samplepart3>
%\fi
% Some text for part 3:
%    \begin{macrocode}
some text in part three
%    \end{macrocode}

%\iffalse
%</samplepart3>
%\fi
% Some text for part 4:
%\iffalse
%<*samplepart4>
%\fi
%    \begin{macrocode}
more text in part four
%    \end{macrocode}

%\iffalse
%</samplepart4>
%\fi
%
% %%%%%%%%%%%%%%%%%%%%%%%%%%%%%%%%%%%%%%
% \paragraph{Forwarding for a Complete Draft.}
%
% The following forwarding file |cdocsdrf.tex|
% compiles the main document in draft mode:
%\iffalse
%<*sampledraft>
%\fi
%    \begin{macrocode}
\def\version{draft}
\input{childdoc.def}
\childdocforward{cdocsamp}
%    \end{macrocode}

%\iffalse
%</sampledraft>
%\fi
%
% %%%%%%%%%%%%%%%%%%%%%%%%%%%%%%%%%%%%%%
% \paragraph{Forwarding for Final Version of the Chapters.}
%
% The following forwarding files |cdocsfn1.tex| and |cdocsfn2.tex|
% (with identical content)
% compile the final versions of the child documents
% |cdocsch1.tex| and |cdocsch2.tex|, respectively:
%\iffalse
%<*samplefinal>
%\fi
%    \begin{macrocode}
\def\version{final}
\input{childdoc.def}
\childdocforwardprefix[cdocsamp]{cdocsfn}{cdocsch}
%    \end{macrocode}

%\iffalse
%</samplefinal>
%\fi
%
% %%%%%%%%%%%%%%%%%%%%%%%%%%%%%%%%%%%%%%
% \paragraph{Command Line Processing.}
%
% The following three command lines generate the output files
% |cdocscld|, |cdocscl1| and |cdocscl2|
% which should be identical to
% |cdocsdrf|, |cdocsch1| and |cdocsfn2|, respectively:
% \begin{center}
% \begin{tabular}{l}
% |latex -jobname cdocscld \|\\
% |  "\def\version{draft}\input{childdoc.def}\childdocforward{cdocsamp}"|\\
% |latex -jobname cdocscl1 \|\\
% |  "\input{childdoc.def}\childdocforward[cdocsamp]{cdocsch1}"|\\
% |latex -jobname cdocscl2 \|\\
% |  "\def\version{final}\input{childdoc.def}\childdocforward{cdocsch2}"|
% \end{tabular}
% \end{center}
% Note that the trailing backslash on each first line
% merely continues the input to the second line
% (for convenient cut ant paste).
% Furthermore, the command |latex| can be replaced by any
% of its alternative versions such as |pdflatex|.
%
% %%%%%%%%%%%%%%%%%%%%%%%%%%%%%%%%%%%%%%%%%%%%%%%%%%%%%%%%%%%%%%%%%%%%%%%%%%%%%%
% %%%%%%%%%%%%%%%%%%%%%%%%%%%%%%%%%%%%%%%%%%%%%%%%%%%%%%%%%%%%%%%%%%%%%%%%%%%%%%
% \section{Implementation}
%\iffalse
%<*package>
%\fi
%
% This section describes the definitions file |childdoc.def|.

% The definitions cannot be loaded using |\usepackage| or |\RequirePackage|
% which has a mechanism to prevent loading a style file more than once.
% When loading the definitions by means of |\input|
% multiple instances have to be prevented manually:
%\iffalse
%This code needs to be before the `\ProvidesFile' directive
%which is defined at the beginning of this file.
%Therefore it is also placed there and commented out here.
%</package>
%<*discard>
%\fi
%    \begin{macrocode}
\ifdefined\childdocmain\endinput\fi
%    \end{macrocode}
%\iffalse
%</discard>
%<*package>
%\fi
%
% \macro{\ifchilddoc}
% \macro{\ifchilddocmanual}
% The conditional |\ifchilddoc| tells whether a
% child (true) or main (false) document is being compiled.
% The conditional |\ifchilddocmanual| tells whether
% the |\includeonly| mechanism is used (false) or
% the selection of child files must be performed manually (true).
% The definitions initialise to false:
%    \begin{macrocode}
\newif\ifchilddoc
\newif\ifchilddocmanual
%    \end{macrocode}

% \macro{\childdocname}
% \macro{\childdocjob}
% The macro |\childdocname| stores the name of the main document
% to be compiled. The macro |\childdocjob| stores the name of
% the document on which the \LaTeX{} compiler was originally invoked.
% The content of |\jobname| cannot be compared
% to filenames specified in the source due to different catcodes.
% The following code rescans |\jobname|, stores the result
% in |\childdocname| and saves a copy in |\childdocjob|:
%    \begin{macrocode}
\edef\childdocname{\scantokens\expandafter{\jobname\noexpand}}
\let\childdocjob\childdocname
%    \end{macrocode}

% \macro{\childdocdisable}
% The macro |\childdocdisable| prevents the main file
% from being processed more than once.
% At this stage, the main document command |\childdocmain|
% is assumed to be called once again where it should do nothing.
% Any subsequent call to it should prevent
% a secondary processing of the main document
% It overwrites the forwarding commands
% |\childdocof| and |\childdocforward|
% with empty macros to prevent further inclusions of the main document:
%    \begin{macrocode}
\newcommand{\childdocdisable}
{
  \renewcommand{\childdocmain}[1]{\renewcommand{\childdocmain}[1]{\endinput}}
  \renewcommand{\childdocof}[1]{}
  \renewcommand{\childdocby}[2][]{}
  \renewcommand{\childdocforward}[2][]{}
  \renewcommand{\childdocdisable}{}
}
%    \end{macrocode}

% \macro{\childdocmain}
% The macro |\childdocmain| is to be called at the top of the main file
% with nothing or the main filename (without extension) as argument.
% First, it breaks loops.
% If the argument is not empty and does not match |\childdocname|
% (which is set by the first inclusion of |childdoc.def|),
% |\ifchilddoc| is set to true, |\includeonly| is applied to the child file
% and |\jobname| is set to the main file
% (for proper handling of |.aux| files):
%    \begin{macrocode}
\newcommand{\childdocmain}[1]
{
  \childdocdisable\childdocmain{}
  \if?#1?\else
    \begingroup
      \def\childdoctmp{#1}
      \ifx\childdoctmp\childdocname
        \def\childdoctmp{}
      \else
        \def\childdoctmp
        {
          \childdoctrue
          \includeonly{\childdocname}
          \def\childdocjob{#1}
          \def\jobname{#1}
        }
      \fi
      \expandafter
    \endgroup
    \childdoctmp
  \fi
}
%    \end{macrocode}

% \macro{\childdocof}
% The command |\childdocof| redirects
% compilation to the main file |#1|.
%    \begin{macrocode}
\newcommand{\childdocof}[1]
{
  \childdocdisable
  \childdoctrue
  \includeonly{\childdocname}
  \def\jobname{#1}
  \def\childdocjob{#1}
  \input{#1}
}
%    \end{macrocode}

% \macro{\childdocby}
% The command |\childdocby| ....
%    \begin{macrocode}
\newcommand{\childdocby}[2][]
{
  \childdocdisable
  \childdoctrue
  \childdocmanualtrue
  \if?#1?\else
    \def\jobname{#2}
  \fi
  \def\childdocjob{#2}
  \input{#2}
  \endinput
}
%    \end{macrocode}

% \macro{\childdocforward}
% The command |\childdocforward| redirects
% compilation to the main file or
% (if the optional argument is given) a child file.
% Parameters are set as if the main file
% or a child file starting with |\childdocof| was compiled.
% Then compilation is handed over to the main file:
%    \begin{macrocode}
\newcommand{\childdocforward}[2][]
{
  \begingroup
    \if?#1?
      \def\childdoctmp
      {
        \def\childdocname{#2}
        \def\childdocjob{#2}
        \def\jobname{#2}
        \input{#2}
        \endinput
      }
    \else
      \def\childdoctmp
      {
        \childdocdisable
        \def\childdocname{#2}
        \childdoctrue
        \includeonly{#2}
        \def\childdocjob{#1}
        \def\jobname{#1}
        \input{#1}
        \endinput
      }
    \fi
    \expandafter
  \endgroup
  \childdoctmp
}
%    \end{macrocode}

% \macro{\childdocforwardprefix}
% The command |\childdocforwardprefix| redirects
% compilation to the main or a child file by means of a pattern.
% The prefix |#1| in the current filename is replaced by |#2|
% and the suffix of the current filename is kept
% (it is assumed that the filename does not contain the substring `|~~~|'
% which is used as a delimiter).
% Compilation is handed over to the new file by |\childdocforward|:
%    \begin{macrocode}
\newcommand{\childdocforwardprefix}[3][]
{
  \begingroup
    \def\childdocextract #2##1~~~{\def\childdoctmp{\childdocforward[#1]{#3##1}}}
    \expandafter\childdocextract\childdocname~~~
    \expandafter
  \endgroup
  \childdoctmp
}
%    \end{macrocode}

% \macro{\childdoc}
% The deprecated macro |\childdoc| is a legacy version of |\childdocmain|:
%    \begin{macrocode}
\newcommand{\childdoc}{\childdocmain}
%    \end{macrocode}

% \macro{\childdocredirect}
% The deprecated macro |\childdocredirect| is a legacy version
% of |\childdocforward| and |\childdocforwardprefix|:
%    \begin{macrocode}
\newcommand{\childdocredirect}[2][]
{
  \begingroup
    \if?#1?
      \def\childdoctmp{\childdocforward{#2}}
    \else
      \def\childdoctmp{\childdocforwardprefix{#1}{#2}}
    \fi
    \expandafter
  \endgroup
  \childdoctmp
}
%    \end{macrocode}

%\iffalse
%</package>
%\fi
%
\endinput

\childdocforwardprefix[cdocsamp]{cdocsfn}{cdocsch}
%    \end{macrocode}

%\iffalse
%</samplefinal>
%\fi
%
% %%%%%%%%%%%%%%%%%%%%%%%%%%%%%%%%%%%%%%
% \paragraph{Command Line Processing.}
%
% The following three command lines generate the output files
% |cdocscld|, |cdocscl1| and |cdocscl2|
% which should be identical to
% |cdocsdrf|, |cdocsch1| and |cdocsfn2|, respectively:
% \begin{center}
% \begin{tabular}{l}
% |latex -jobname cdocscld \|\\
% |  "\def\version{draft}% \iffalse
%
% childdoc.dtx Copyright (C) 2017-2018 Niklas Beisert
%
% This work may be distributed and/or modified under the
% conditions of the LaTeX Project Public License, either version 1.3
% of this license or (at your option) any later version.
% The latest version of this license is in
%   http://www.latex-project.org/lppl.txt
% and version 1.3 or later is part of all distributions of LaTeX
% version 2005/12/01 or later.
%
% This work has the LPPL maintenance status `maintained'.
%
% The Current Maintainer of this work is Niklas Beisert.
%
% This work consists of the files childdoc.dtx and childdoc.ins
% and the derived files childdoc.def and cdocsamp.tex with
% cdocsch1.tex, cdocsch2.tex, cdocsdrf.tex, cdocsfn1.tex, cdocsfn2.tex.
%
%<package>\ifdefined\childdocmain\endinput\fi
%<package>\ProvidesFile{childdoc.def}[2018/12/30 v2.0 child document driver]
%<samplemain>\ProvidesFile{cdocsamp.tex}[2018/12/30 v2.0 sample for childdoc]
%<*driver>
%\ProvidesFile{childdoc.drv}[2018/12/30 v2.0 childdoc reference manual file]
\PassOptionsToClass{10pt,a4paper}{article}
\documentclass{ltxdoc}

\usepackage[margin=35mm]{geometry}
\usepackage{hyperref}
\usepackage{hyperxmp}
\usepackage[usenames]{color}

\hypersetup{colorlinks=true}
\hypersetup{pdfstartview=FitH}
\hypersetup{pdfpagemode=UseNone}
\hypersetup{pdfsource={}}
\hypersetup{pdflang={en-UK}}
\hypersetup{pdfcopyright={Copyright 2017-2018 Niklas Beisert.
  This work may be distributed and/or modified under the
  conditions of the LaTeX Project Public License, either version 1.3
  of this license or (at your option) any later version.}}
\hypersetup{pdflicenseurl={http://www.latex-project.org/lppl.txt}}
\hypersetup{pdfcontactaddress={ETH Zurich, ITP, HIT K,
  Wolfgang-Pauli-Strasse 27}}
\hypersetup{pdfcontactpostcode={8093}}
\hypersetup{pdfcontactcity={Zurich}}
\hypersetup{pdfcontactcountry={Switzerland}}
\hypersetup{pdfcontactemail={nbeisert@itp.phys.ethz.ch}}
\hypersetup{pdfcontacturl={http://people.phys.ethz.ch/\xmptilde nbeisert/}}

\newcommand{\secref}[1]{\hyperref[#1]{section \ref*{#1}}}

\parskip1ex
\parindent0pt
\let\olditemize\itemize
\def\itemize{\olditemize\parskip0pt}

\begin{document}

\title{The \textsf{childdoc} Package}
\hypersetup{pdftitle={The childdoc Package}}
\author{Niklas Beisert\\[2ex]
  Institut f\"ur Theoretische Physik\\
  Eidgen\"ossische Technische Hochschule Z\"urich\\
  Wolfgang-Pauli-Strasse 27, 8093 Z\"urich, Switzerland\\[1ex]
  \href{mailto:nbeisert@itp.phys.ethz.ch}
  {\texttt{nbeisert@itp.phys.ethz.ch}}}
\hypersetup{pdfauthor={Niklas Beisert}}
\hypersetup{pdfsubject={Manual for the LaTeX2e Package childdoc}}
\date{30 December 2018, \textsf{v2.0}}
\maketitle

\begin{abstract}\noindent
\textsf{childdoc} is a \LaTeXe{} package
that enables the direct compilation
of document sections included by |\include|
to individual files.
\end{abstract}

\begingroup
\parskip0ex
\tableofcontents
\endgroup

%%%%%%%%%%%%%%%%%%%%%%%%%%%%%%%%%%%%%%%%%%%%%%%%%%%%%%%%%%%%%%%%%%%%%%%%%%%%%%%%
%%%%%%%%%%%%%%%%%%%%%%%%%%%%%%%%%%%%%%%%%%%%%%%%%%%%%%%%%%%%%%%%%%%%%%%%%%%%%%%%
\section{Introduction}

\LaTeX{} provides a mechanism to structure a large document (such as a book)
into a main file and several child files (containing the chapters)
using the |\include| command.
This mechanism is beneficial for documents
which span hundreds of pages in order to
make the source file(s) more manageable.
Moreover, compilation can be restricted to
selected child files by means of the |\includeonly| command.
The latter feature can be used to reduce the compilation time while editing
(this was significantly more useful in the earlier days of \LaTeX{})
or to generate a smaller document which is easier to navigate.
Another application of |\includeonly| is to generate
documents consisting of selected parts of the complete document.

However, there are a few drawbacks of the plain |\include| mechanism:
\begin{itemize}
\item
The child files cannot be compiled on their own,
they can only be compiled via the main file.
A naive editing environment
(such as a text editor with an option
to have the current file processed by \LaTeX)
may require one to switch to the main file before compiling;
attempting to compile the child file produces errors.
\item
The main file must be modified (each time)
to adjust the |\includeonly| command
to the present needs. This easily leaves the main file in a messy state.
\item
The generated document will always carry the filename
of the main document. This is inconvenient if
several child files are to be compiled and
to be kept for distribution.
\end{itemize}

The present package provides a simple interface
to make child files individually compilable by \LaTeX{}.
Compiling a child file then has the same effect as compiling
the main file with an |\includeonly| command
to select the appropriate child.
Moreover the generated document will carry the name of the child
rather than the main file.
This resolves all three above issues.

This feature is meant to make the editing of books,
thesis documents and lecture notes somewhat more convenient.
However, the package can also be used efficiently for
composing a series of documents (such as exercise sheets)
which are typically distributed individually.
It then assists the author in generating the individual documents
(potentially in different versions)
as well as a document containing the collected series.
Another application is in developing style files
or other kinds of included material
where compilation of the style file could redirect
to a sample or test file.

%%%%%%%%%%%%%%%%%%%%%%%%%%%%%%%%%%%%%%%%%%%%%%%%%%%%%%%%%%%%%%%%%%%%%%%%%%%%%%%%
%%%%%%%%%%%%%%%%%%%%%%%%%%%%%%%%%%%%%%%%%%%%%%%%%%%%%%%%%%%%%%%%%%%%%%%%%%%%%%%%
\section{Usage}

First of all, the package \textsf{childdoc} is \emph{not} a standard
\LaTeXe{} |.sty| style file! Therefore it needs to be invoked in
a non-standard way.

%%%%%%%%%%%%%%%%%%%%%%%%%%%%%%%%%%%%%%%%%%%%%%%%%%%%%%%%%%%%%%%%%%%%%%%%%%%%%%%%
\subsection{Included Files}
\label{sec:include}

%%%%%%%%%%%%%%%%%%%%%%%%%%%%%%%%%%%%%%%%
\DescribeMacro{\childdocmain}
To use the package, add the commands
\begin{center}
\begin{tabular}{l}
|\input{childdoc.def}|\\
|\childdocmain{}|\\
\end{tabular}
\end{center}
at the very top of the main \LaTeX{} file,
in particular \emph{before} the |\documentclass| statement!
The argument of |\childdocmain| should be left empty
(but it must be present).

%%%%%%%%%%%%%%%%%%%%%%%%%%%%%%%%%%%%%%%%
\DescribeMacro{\childdocof}
Furthermore, add the commands
\begin{center}
\begin{tabular}{l}
|\input{childdoc.def}|\\
|\childdocof{|\textit{main}|}|\\
\end{tabular}
\end{center}
at the top of every child file \textit{child}
which is included by |\include{|\textit{child}|}|
from within the main file
(or at least for those files to be compiled individually).
The argument \textit{main} must be the filename of the main file.

There are a couple of
considerations in setting up the main and child documents:

%%%%%%%%%%%%%%%%%%%%%%%%%%%%%%%%%%%%%%%%
\paragraph{Restrictions.}

Please note the following restrictions:
\begin{itemize}
\item
|\childdocmain| must be called with one argument \textit{main}
to ensure compatibility with earlier version of the package.
It must either be empty (|\childdocmain{}|)
or precisely match the filename of the main file in which it is specified.
See \secref{sec:detection} for further information.
\item
The filename \textit{main} must be specified without the |.tex| extension.
\item
The filename \textit{main} is case sensitive
(even in case-insensitive file systems)
due to internal string comparison.
\item
The argument \textit{main} should be fully expanded, it cannot be a macro.
\item
Subdirectories and special characters should be avoided in filenames.
\item
The command |\childdocmain{|\textit{main}|}| must be followed by a whitespace.
It should not be followed immediately by another command
or by a comment mark `|%|'.
This is because the \TeX{} parser reads the token immediately following
the argument of |\childdocmain| and puts it
at the beginning of every child section;
however, a white\-space is ignored.
\end{itemize}

%%%%%%%%%%%%%%%%%%%%%%%%%%%%%%%%%%%%%%%%
\paragraph{Content of Main File.}

It is advisable to place all content in the child files included by |\include|.
Any output contained in the main file will appear in all child documents
unless suppressed manually;
it cannot be suppressed automatically by the |\includeonly| directive
and thus should normally be avoided.
A method to include some content in the main file
by means of conditional processing is described in \secref{sec:conditional}.

%%%%%%%%%%%%%%%%%%%%%%%%%%%%%%%%%%%%%%%%
\paragraph{Page Numbering.}

When only a part of the document is compiled,
the appropriate numbering of pages
(as well as other status parameters)
is determined from the |.aux| files.
The latter contain information from previous passes.
However this information needs to propagate through
all intermediate child documents.
Therefore the page numbering in child documents may well
be inconsistent until the complete document is compiled at least once.

A useful (if unconventional) way to always ensure a consistent
page numbering is to restart the numbering in each child document
and denote the pages by `\textit{child}|.|\textit{page}'
where \textit{child} represents the chapter/section number of the child file.
This can be achieved by the command
|\numberwithin{page}{|\textit{child}|}|
of the \textsf{amsmath} package
where \textit{child} can be |chapter| or |section|
depending on the chosen structuring.
Alternatively, one can modify the macro |\thepage| appropriately
and reset the counter |page| at the start of each child file.

%%%%%%%%%%%%%%%%%%%%%%%%%%%%%%%%%%%%%%%%%%%%%%%%%%%%%%%%%%%%%%%%%%%%%%%%%%%%%%%%
\subsection{Conditional Processing}
\label{sec:conditional}

The package provides a mechanism to compile different versions
of a document. To customise the versions further some conditional processing
can come in handy to distinguish which version is being compiled.
The package provides two macros to describe the compilation context:

%%%%%%%%%%%%%%%%%%%%%%%%%%%%%%%%%%%%%%%%
\DescribeMacro{\ifchilddoc}
The conditional |\ifchilddoc| distinguishes between the compilation of
child documents and the main document:
%
\begin{center}
|\ifchilddoc |\textit{child-code}| |[|\||else |\textit{main-code}]| \||fi|
\end{center}

%%%%%%%%%%%%%%%%%%%%%%%%%%%%%%%%%%%%%%%%
\DescribeMacro{\childdocname}
\DescribeMacro{\childdocjob}
The macro |\childdocname| contains the filename (without extension)
of the main or child file being processed.
Note that |\childdocjob| will always contain the name of the main file.

%%%%%%%%%%%%%%%%%%%%%%%%%%%%%%%%%%%%%%%%
\paragraph{Title Page.}

Conditional processing can be used to include a title or banner page
in the main document when proper precautions are taken.
Importantly, the code in the main file should ensure that the page counter
(as well as other status parameters which are stored in the |.aux| files)
takes the same value after the conditional processing.
Otherwise the page numbers may take divergent values
depending on which part is compiled.

For example, a title page could be declared by:
%
\begin{center}
\begin{tabular}{l}
|\ifchilddoc\||else|\\
|\addtocounter{page}{-1}|\\
\textit{code for title page}\\
|\newpage|\\
|\||fi|
\end{tabular}
\end{center}
%
A banner page for the child documents can be generated by:
%
\begin{center}
\begin{tabular}{l}
|\ifchilddoc|\\
|\addtocounter{page}{-1}|\\
\textit{code for banner page}\\
|\newpage|\\
|\||fi|
\end{tabular}
\end{center}
%
Here one could write a message such as:
\begin{center}
|This is the part \childdocname{} of \childdocjob{}.|
\end{center}

%%%%%%%%%%%%%%%%%%%%%%%%%%%%%%%%%%%%%%%%%%%%%%%%%%%%%%%%%%%%%%%%%%%%%%%%%%%%%%%%
\subsection{Flags}
\label{sec:flags}

The package makes it easy to generate different versions
of the main or child documents.
To this end compilation flags can be defined
and assigned different default values.
They will be particularly useful in conjunction
with the forwarding mechanism described in \secref{sec:forward}.

For example, it may be useful to have a flag |\version|
which can be set to |draft| or |final|.
The document source will contain some conditional code
depending on the value of |\version|.
Suppose further, the flag should default to |final| for the main file
and to |draft| for child files
which is a natural assignment for editing the document.
This is achieved by placing the following code
in the preamble of the main document
(below the |\childdocmain| directive):
%
\begin{center}
\begin{tabular}{l}
|\ifchilddoc|\\
|\providecommand{\version}{draft}|\\
|\||else|\\
|\providecommand{\version}{final}|\\
|\||fi|
\end{tabular}
\end{center}
%
The definition by |\providecommand| makes sure
that previous definitions are not overwritten.
Further statements |\providecommand{\version}{...}|
can thus be added before the above code to override it.

For the main file, one might add a line
(between |\childdocmain| and the above block)
%
\begin{center}
|%\ifchilddoc\||else\providecommand{\version}{draft}\||fi|
\end{center}
%
which can be uncommented to produce a draft version.
Likewise one can add a line to the very top of a child file
(above the |\childdocof{|\textit{main}|}| directive)
%
\begin{center}
|%\providecommand{\version}{final}|
\end{center}
%
which can be uncommented to produce the final version of this child document.

%%%%%%%%%%%%%%%%%%%%%%%%%%%%%%%%%%%%%%%%%%%%%%%%%%%%%%%%%%%%%%%%%%%%%%%%%%%%%%%%
\subsection{Forwarding}
\label{sec:forward}

Different versions of the main or child documents
using compilation flags as described in \secref{sec:flags}
can be (permanently) stored in different files
for convenient compilation, viewing and distribution.
To this end, the package defines a command
to pass on compilation to a different file:

%%%%%%%%%%%%%%%%%%%%%%%%%%%%%%%%%%%%%%%%
\DescribeMacro{\childdocforward}
The command |\childdocforward| redirects processing to
another source file:
%
\begin{center}
\begin{tabular}{l}
|\input{childdoc.def}|\\
|\childdocforward[|\textit{main}|]{|\textit{dest}|}|\\
\end{tabular}
\end{center}
%
The argument \textit{dest} is the destination file
(without extension).
It should be the main file or one of the child files.
Note that further \textsf{childdoc} directives
such as |\childdocof| and |\childdocforward|
in the indicated file will be processed in this form.
The optional argument \textit{main}
passes on directly to the main file \textit{main}
while pretending to compile the child \textit{dest}.
This form behaves as if \textit{dest}
issues |\childdocof{|\textit{main}|}| right away,
and no further \textsf{childdoc} directives will be processed.

%%%%%%%%%%%%%%%%%%%%%%%%%%%%%%%%%%%%%%%%
\DescribeMacro{\...prefix}
In the alternative form |\childdocforwardprefix|,
%
\begin{center}
\begin{tabular}{l}
|\input{childdoc.def}|\\
|\childdocforwardprefix[|\textit{main}|]{|\textit{prefix}|}{|\textit{dest}|}|
\end{tabular}
\end{center}
%
the destination file is determined by a pattern
depending on the current file:
To make this work, the current file must be called
`{\textit{prefix}\hspace{0.2em}\textit{suffix}}'
with \textit{prefix} matching precisely the argument.
Processing is then passed on to the file
`{\textit{dest}\hspace{0.2em}\textit{suffix}}'.
Surely, the same effect is achieved by
directly specifying the
argument `{\textit{dest}\hspace{0.2em}\textit{suffix}}'
in the first form.
However, that requires to set up a different file
for each child. With the alternative form of the command
all these files can have exactly the same content
which simplifies setting them up and maintaining them.

For example, the following file |draft.tex|
with a compilation flag |\version| as described in \secref{sec:flags}
compiles the main document as a draft:
%
\begin{center}
\begin{tabular}{l}
|\def\version{draft}|\\
|\input{childdoc.def}|\\
|\childdocforward{|\textit{main}|}|
\end{tabular}
\end{center}
%
Likewise, the following files |final|\textit{nn}|.tex|
compile the final version of the child document
|child|\textit{nn}|.tex|:
%
\begin{center}
\begin{tabular}{l}
|\def\version{final}|\\
|\input{childdoc.def}|\\
|\childdocforwardprefix{final}{child}|
\end{tabular}
\end{center}
%

Note that when several versions of a main file and/or of each child file
are to be generated, it may be convenient to set up a |Makefile| or
shell script to automatise the process.

%%%%%%%%%%%%%%%%%%%%%%%%%%%%%%%%%%%%%%%%%%%%%%%%%%%%%%%%%%%%%%%%%%%%%%%%%%%%%%%%
\subsection{Command Line Processing}
\label{sec:commandline}

The effect of redirection files can also be achieved by invoking
the \LaTeX{} compiler with a more elaborate command line.
Most conveniently this should be done as part
of a shell script or a |Makefile|.

When using \textsf{childdoc} in the main file, the following
command lines effectively perform a redirection
(note that depending on the shell being used,
backslashes may have to be doubled: `|\|' $\to$ `|\\|'):
%
\begin{center}
|... -jobname "|\textit{target}|" |\\|"|[\textit{flags}]%
|\input{childdoc.def}\childdocforward[|\textit{main}|]{|\textit{dest}|}"|
\end{center}
%
Here \textit{target} is the name of the output file,
\textit{main} is the name of the main file
and \textit{dest} is the name of the main or child file to be processed
(all filenames without extensions).
The optional argument \textit{main} can be omitted
if \textit{main} matches \textit{dest}.
Optionally, compilation \textit{flags} can be defined via |\def| commands.
This command line makes the \TeX{} engine believe
it is compiling the file \textit{target}
whose content is specified as the latter parameter.
The provided code then forwards the processing to
\textit{main} or \textit{dest} as described in \secref{sec:forward}.

%%%%%%%%%%%%%%%%%%%%%%%%%%%%%%%%%%%%%%%%%%%%%%%%%%%%%%%%%%%%%%%%%%%%%%%%%%%%%%%%
\subsection{Include by Input}
\label{sec:input}

Including child documents by |\include| has some restrictions by design.
Most notably, the content of a child document always occupies
its own set of pages; pages cannot be shared between child documents.
Usually, this behaviour makes perfect sense
because each child document contain an essential part of the document.
However, in some situations it may be desirable to compose
a document from a collection of parts
without having mandatory page breaks between then.
For this case, the package
provides a mechanism to include parts
by |\input| which can also be processed individually.
However, by construction this mechanism
requires manual handling of the content to be output.

%%%%%%%%%%%%%%%%%%%%%%%%%%%%%%%%%%%%%%%%
\DescribeMacro{\ifchilddocmanual}
The main file should be prepared as usual, see \secref{sec:include}.
However, the document body must make a distinction
between processing of an individual part and of the main document, e.g.:
%
\begin{center}
\begin{tabular}{l}
|\ifchilddocmanual|\\
|\input{\childdocname}|\\
|\||else|\\
\textit{document body with }|\input{|\textit{part}|}|\\
|\||fi|
\end{tabular}
\end{center}
%
The conditional |\ifchilddocmanual| is true whenever
a part to be included by |\input| is being compiled,
and the name of the part is stored in |\childdocname|.

%%%%%%%%%%%%%%%%%%%%%%%%%%%%%%%%%%%%%%%%
\DescribeMacro{\childdocby}
Each part to be included by |\input| should start with:
%
\begin{center}
\begin{tabular}{l}
|\input{childdoc.def}|\\
|\childdocby{|\textit{main}|}|\\
\end{tabular}
\end{center}
%
The directive |\childdocby| is similar to |\childdocof|
described in \secref{sec:include},
but the subsequent selection of content must be done manually.
To that end, both |\ifchilddoc| and |\ifchilddocmanual|
will be true upon processing of a part,
and the name of the part is stored in |\childdocname|.
Note that |\jobname| will be set to the filename of the current part
so that each part receives an individual |.aux| file
that does not interfere with the |.aux| file(s) of the main document.
This behaviour can be altered by the alternative form
|\childdocby[*]{|\textit{main}|}| (with a non-empty optional argument)
which uses the |.aux| file of the main document
by setting |\jobname| to \textit{main}.

%%%%%%%%%%%%%%%%%%%%%%%%%%%%%%%%%%%%%%%%%%%%%%%%%%%%%%%%%%%%%%%%%%%%%%%%%%%%%%%%
\subsection{Driver Development}
\label{sec:driver}

The \textsf{childdoc} mechanism can also be use for the development
of definition files such as \LaTeX{} styles or classes.
This case differs from the above setup with multiple parts
included by |\include| in that no |\includeonly| should be invoked.
This can be achieved by starting the include file
(before |\ProvidesPackage|) with:
%
\begin{center}
\begin{tabular}{l}
|\input{childdoc.def}|\\
|\childdocforward{|\textit{main}|}|\\
\end{tabular}
\end{center}
%
or alternatively with:
%
\begin{center}
\begin{tabular}{l}
|\input{childdoc.def}|\\
|\childdocby{|\textit{main}|}|\\
\end{tabular}
\end{center}
%
Both forms have slightly different effects as described above.
The main file is prepared as usual, see \secref{sec:include}.

%%%%%%%%%%%%%%%%%%%%%%%%%%%%%%%%%%%%%%%%%%%%%%%%%%%%%%%%%%%%%%%%%%%%%%%%%%%%%%%%
\subsection{Legacy Detection}
\label{sec:detection}

The directive |\childdocmain| in the main file can detect
whether the complete document or merely a child is to be compiled
even without using the directive |\childdocof|.
This method is deprecated because it is less robust
and there is no compelling reason to use it;
it is merely provided for backward compatibility
and it may be removed in future versions.

If the detection mechanism is to be used,
it is mandatory to correctly specify
the filename of the main file as the argument of |\childdocmain|:
%
\begin{center}
\begin{tabular}{l}
|\input{childdoc.def}|\\
|\childdocmain{|\textit{main}|}|\\
\end{tabular}
\end{center}
%
If |\jobname| does not match the argument \textit{main} of |\childdocmain|,
it is assumed that |\jobname| points to the child file to be compiled.
When using |\childdocmain| with the main file specified as argument,
it suffices to start a child file
with just |\input{|\textit{main}|}|
without loading of the package and using |\childdocof|.
If instead all processing is done
with the appropriate \textsf{childdoc} directives,
the argument of \textit{main} of |\childdocmain| can be empty.

An alternative version of the command line processing described
in \secref{sec:commandline} using the detection mechanism reads:
%
\begin{center}
|... -jobname "|\textit{target}|" "|[\textit{flags}]%
[|\def\jobname{|\textit{dest}|}|]|\input{|\textit{main}|}"|
\end{center}

%%%%%%%%%%%%%%%%%%%%%%%%%%%%%%%%%%%%%%%%%%%%%%%%%%%%%%%%%%%%%%%%%%%%%%%%%%%%%%%%
\subsection{Manual Code}
\label{sec:manual}

In case one cannot be certain whether the definitions file |childdoc.def|
is installed on the target \TeX{} distribution
and one prefers not to ship it,
it is conceivable to paste a few relevant commands into the sources.

To that end, drop all statements |\input{childdoc.def}|
and perform the replacements as outlined below.
Instead of |\childdocmain{|\textit{main}|}| add the following code
to the top of the main file:
%
\begin{center}
\begin{tabular}{l}
|\||ifdefined\childdocname\endinput\||fi\newif\ifchilddoc|\\
|\edef\childdocname{\scantokens\expandafter{\jobname\noexpand}}|\\
|\def\childdocmain{|\textit{main}|}\||ifx\childdocmain\childdocname\||else|\\
|\childdoctrue\includeonly{\childdocname}\let\jobname\childdocmain\||fi|\\
\end{tabular}
\end{center}
%
Instead of |\childdocof{|\textit{main}|}| just include the main file
at the top of each child file:
%
\begin{center}
|\input{|\textit{main}|}|
\end{center}
%
A simple redirection |\childdocforward{|\textit{dest}|}| is achieved by:
%
\begin{center}
|\def\jobname{|\textit{dest}|}\input{\jobname}|
\end{center}
%
The redirection with prefix
|\childdocforwardprefix[|\textit{prefix}|]{|\textit{dest}|}|
is accomplished by:
%
\begin{center}
\begin{tabular}{l}
|{\edef\jobname{\scantokens\expandafter{\jobname\noexpand}}|\\
|\def\redirectjob |\textit{prefix}|#1~~~{\gdef\jobname{|\textit{dest}|#1}}|\\
|\expandafter\redirectjob\jobname~~~}\input{\jobname}|
\end{tabular}
\end{center}

In an alternative approach,
child documents can be compiled by a specific command line
without additional code or specific definitions:
%
\begin{center}
|... -jobname "|\textit{target}|" "|[\textit{flags}]%
|\includeonly{|\textit{dest}|}\input{|\textit{main}|}"|
\end{center}
%

%%%%%%%%%%%%%%%%%%%%%%%%%%%%%%%%%%%%%%%%%%%%%%%%%%%%%%%%%%%%%%%%%%%%%%%%%%%%%%%%
%%%%%%%%%%%%%%%%%%%%%%%%%%%%%%%%%%%%%%%%%%%%%%%%%%%%%%%%%%%%%%%%%%%%%%%%%%%%%%%%
\section{Information}

%%%%%%%%%%%%%%%%%%%%%%%%%%%%%%%%%%%%%%%%%%%%%%%%%%%%%%%%%%%%%%%%%%%%%%%%%%%%%%%%
\subsection{Copyright}

Copyright \copyright{} 2017--2018 Niklas Beisert

This work may be distributed and/or modified under the
conditions of the \LaTeX{} Project Public License, either version 1.3
of this license or (at your option) any later version.
The latest version of this license is in
  \url{http://www.latex-project.org/lppl.txt}
and version 1.3 or later is part of all distributions of \LaTeX{}
version 2005/12/01 or later.

This work has the LPPL maintenance status `maintained'.

The Current Maintainer of this work is Niklas Beisert.

This work consists of the files |README.txt|, |childdoc.ins| and |childdoc.dtx|
as well as the derived files |childdoc.def|, |cdocsamp.tex|
with |cdocsch1.tex|, |cdocsch2.tex|, |cdocspt3.tex|, |cdocspt4.tex|,
|cdocsdrf.tex|, |cdocsfn1.tex|, |cdocsfn2.tex|
as well as |childdoc.pdf|.

%%%%%%%%%%%%%%%%%%%%%%%%%%%%%%%%%%%%%%%%%%%%%%%%%%%%%%%%%%%%%%%%%%%%%%%%%%%%%%%%
\subsection{Files and Installation}

The package consists of the files:
%
\begin{center}
\begin{tabular}{ll}
    |README.txt|   & readme file \\
    |childdoc.ins| & installation file \\
    |childdoc.dtx| & source file \\
    |childdoc.def| & definition file \\
    |cdocsamp.tex| & sample main file \\
    |cdocsch1.tex| & sample include file \\
    |cdocsch2.tex| & sample include file \\
    |cdocspt3.tex| & sample part file \\
    |cdocspt4.tex| & sample part file \\
    |cdocsdrf.tex| & sample redirection file \\
    |cdocsfn1.tex| & sample redirection file \\
    |cdocsfn2.tex| & sample redirection file \\
    |childdoc.pdf| & manual
\end{tabular}
\end{center}
%
The distribution consists of the files
|README.txt|, |childdoc.ins| and |childdoc.dtx|.
%
\begin{itemize}
\item
Run (pdf)\LaTeX{} on |childdoc.dtx|
to compile the manual |childdoc.pdf| (this file).
\item
Run \LaTeX{} on |childdoc.ins| to create the definitions file |childdoc.def|
and the sample |cdocsamp.tex| with include files
|cdocsch1.tex|, |cdocsch2.tex|, |cdocspt3.tex|, |cdocspt4.tex|,
|cdocsdrf.tex|, |cdocsfn1.tex|, |cdocsfn2.tex|.
Then copy the file |childdoc.def| to an appropriate directory of your \LaTeX{}
distribution, e.g.\ \textit{texmf-root}|/tex/latex/childdoc|.
\end{itemize}

%%%%%%%%%%%%%%%%%%%%%%%%%%%%%%%%%%%%%%%%%%%%%%%%%%%%%%%%%%%%%%%%%%%%%%%%%%%%%%%%
\subsection{Related CTAN Packages}

There are several other packages which offer a similar functionality:
%
\begin{itemize}
\item
The packages
\href{http://ctan.org/pkg/docmute}{\textsf{docmute}},
\href{http://ctan.org/pkg/includex}{\textsf{includex}} and
\href{http://ctan.org/pkg/standalone}{\textsf{standalone}}
provide commands to include only the document body of
a child file thus allowing both files to be compiled individually.
\item
The packages \href{http://ctan.org/pkg/subdocs}{\textsf{subdocs}}
and \href{http://ctan.org/pkg/subfiles}{\textsf{subfiles}}
provide structures in which the main and child documents can be
encapsulated and allowing them to be compiled individually.
The inclusion mechanism is different from the conventional |\include|.
\item
The package \href{http://ctan.org/pkg/combine}{\textsf{combine}}
is an elaborate solution to combine several documents into one.
\end{itemize}
%
See also the CTAN topic \href{http://ctan.org/topic/subdocs}{\textsf{subdocs}}
for further related packages.
The present package differs from the above solutions in that
a document structure constructed with the conventional |\include| mechanism
just needs two extra commands at the top of every file
such that all constituent files can be compiled individually.

%%%%%%%%%%%%%%%%%%%%%%%%%%%%%%%%%%%%%%%%%%%%%%%%%%%%%%%%%%%%%%%%%%%%%%%%%%%%%%%%
%\subsection{Feature Suggestions}
%
%The following is a list of features which may be useful for future
%versions of this package:
%%
%\begin{itemize}
%\item
%\ldots
%\end{itemize}

%%%%%%%%%%%%%%%%%%%%%%%%%%%%%%%%%%%%%%%%%%%%%%%%%%%%%%%%%%%%%%%%%%%%%%%%%%%%%%%%
\subsection{Revision History}

%%%%%%%%%%%%%%%%%%%%%%%%%%%%%%%%%%%%%%%%
\paragraph{v2.0:} 2018/12/30

\begin{itemize}
\item
immediate forward processing
\item
added |\childdocby| mechanism
\item
manual restructured
\end{itemize}

%%%%%%%%%%%%%%%%%%%%%%%%%%%%%%%%%%%%%%%%
\paragraph{v1.6:} 2018/01/17

\begin{itemize}
\item
application for development of include files
\item
corrections to manual
\end{itemize}

%%%%%%%%%%%%%%%%%%%%%%%%%%%%%%%%%%%%%%%%
\paragraph{v1.5:} 2017/05/21

\begin{itemize}
\item
more complete structuring introduced
\item
|\childdocof| introduced
\item
|\childdoc| renamed to |\childdocmain|
\item
|\childredirect| renamed to |\childdocforward| and |\childdocforwardprefix|
and functionality expanded
\end{itemize}

%%%%%%%%%%%%%%%%%%%%%%%%%%%%%%%%%%%%%%%%
\paragraph{v1.0:} 2017/04/27

\begin{itemize}
\item
manual and install package
\item
first version published on CTAN
\end{itemize}

%%%%%%%%%%%%%%%%%%%%%%%%%%%%%%%%%%%%%%%%
\paragraph{v0.6:} 2017/04/26

\begin{itemize}
\item
redirection mechanism added
\end{itemize}

%%%%%%%%%%%%%%%%%%%%%%%%%%%%%%%%%%%%%%%%
\paragraph{v0.5:} 2017/04/26

\begin{itemize}
\item
functionality in definition file
\end{itemize}


%%%%%%%%%%%%%%%%%%%%%%%%%%%%%%%%%%%%%%%%%%%%%%%%%%%%%%%%%%%%%%%%%%%%%%%%%%%%%%%%
%%%%%%%%%%%%%%%%%%%%%%%%%%%%%%%%%%%%%%%%%%%%%%%%%%%%%%%%%%%%%%%%%%%%%%%%%%%%%%%%
%%%%%%%%%%%%%%%%%%%%%%%%%%%%%%%%%%%%%%%%%%%%%%%%%%%%%%%%%%%%%%%%%%%%%%%%%%%%%%%%
\appendix

\settowidth\MacroIndent{\rmfamily\scriptsize 000\ }

 \DocInput{childdoc.dtx}

\end{document}
%</driver>
% \fi
%
% %%%%%%%%%%%%%%%%%%%%%%%%%%%%%%%%%%%%%%%%%%%%%%%%%%%%%%%%%%%%%%%%%%%%%%%%%%%%%%
% %%%%%%%%%%%%%%%%%%%%%%%%%%%%%%%%%%%%%%%%%%%%%%%%%%%%%%%%%%%%%%%%%%%%%%%%%%%%%%
% \section{Sample}
%\iffalse
%<*samplemain>
%\fi
%
% The following presents a sample document
% with two chapters, two parts, a title page,
% a compile flag as well as three forwarding files to set the flag.
% It consists of eight |.tex| files:
% \begin{center}
% \begin{tabular}{ll}
% |cdocsamp.tex|&main file\\
% |cdocsch1.tex|&include file for chapter 1\\
% |cdocsch2.tex|&include file for chapter 2\\
% |cdocspt3.tex|&include file for part 3\\
% |cdocspt4.tex|&include file for part 4\\
% |cdocsdrf.tex|&forwarding file for main file in draft mode\\
% |cdocsfi1.tex|&forwarding file for final version of chapter 1\\
% |cdocsfi2.tex|&forwarding file for final version of chapter 2\\
% \end{tabular}
% \end{center}
% Each of the eight files can be compiled directly by the \LaTeX{} compiler.
%
% %%%%%%%%%%%%%%%%%%%%%%%%%%%%%%%%%%%%%%
% \paragraph{Main File.}
%
% The main file is called |cdocsamp.tex|.
%
% Load the \textsf{childdoc} definitions and
% declare the filename for the main document:
%    \begin{macrocode}
\input{childdoc.def}
\childdocmain{}
%    \end{macrocode}

% Optional override for |\version| flag:
%    \begin{macrocode}
%%\ifchilddoc\else\providecommand{\version}{draft}\fi
%    \end{macrocode}

% Define the default values for the |\version| flag
% (|final| for the main file and |draft| for childs):
%    \begin{macrocode}
\ifchilddoc
\providecommand{\version}{draft}
\else
\providecommand{\version}{final}
\fi
%    \end{macrocode}

% Load the standard document class:
%    \begin{macrocode}
\documentclass[12pt]{article}
%    \end{macrocode}

% Start the document body:
%    \begin{macrocode}
\begin{document}
%    \end{macrocode}

% Declare a title page.
% Print title, part of document being processed and version flag:
%    \begin{macrocode}
\addtocounter{page}{-1}
\begin{center}
{\LARGE\bfseries{}childdoc example\par}
\vspace{1cm}
\ifchilddoc
\ifchilddocmanual part\else chapter\fi:
`\childdocname' of `\childdocjob'\par
\else
main document: `\childdocjob'\par
\fi
version: \version\par
\end{center}
\newpage
%    \end{macrocode}

% Manually include selected file,
% otherwise process as usual:
%    \begin{macrocode}
\ifchilddocmanual
\section*{part `\childdocname'}
\input{\childdocname}
\else
%    \end{macrocode}

% Include the two chapters:
%    \begin{macrocode}
\include{cdocsch1}
\include{cdocsch2}
%    \end{macrocode}

% Include the two parts unless only chapters should be displayed:
%    \begin{macrocode}
\ifchilddoc\else
\section{part three}
\input{cdocspt3}
\section{part four}
\input{cdocspt4}
\fi
%    \end{macrocode}

% Process as usual until here:
%    \begin{macrocode}
\fi
%    \end{macrocode}

% End of document body:
%    \begin{macrocode}
\end{document}
%    \end{macrocode}
%\iffalse
%</samplemain>
%\fi
%
% %%%%%%%%%%%%%%%%%%%%%%%%%%%%%%%%%%%%%%
% \paragraph{Chapter Include Files.}
%
% The include files are called |cdocsch1.tex| and |cdocsch2.tex|.
%
%\iffalse
%<*samplechap1|samplechap2>
%\fi

% Optional override for |\version| flag:
%    \begin{macrocode}
%%\providecommand{\version}{final}
%    \end{macrocode}

% Include the main document:
%    \begin{macrocode}
\input{childdoc.def}
\childdocof{cdocsamp}
%    \end{macrocode}

%\iffalse
%</samplechap1|samplechap2>
%\fi
%
%\iffalse
%<*samplechap1>
%\fi
% Some text for chapter 1:
%    \begin{macrocode}
\section{one}
some text in chapter one
%    \end{macrocode}

%\iffalse
%</samplechap1>
%\fi
% Some text for chapter 2:
%\iffalse
%<*samplechap2>
%\fi
%    \begin{macrocode}
\section{two}
more text in chapter two
%    \end{macrocode}

%\iffalse
%</samplechap2>
%\fi
%
% %%%%%%%%%%%%%%%%%%%%%%%%%%%%%%%%%%%%%%
% \paragraph{Part Include Files.}
%
% The include files are called |cdocspt3.tex| and |cdocspt4.tex|.
%
%\iffalse
%<*samplepart3|samplepart4>
%\fi

% Optional override for |\version| flag:
%    \begin{macrocode}
%%\providecommand{\version}{final}
%    \end{macrocode}

% Include the main document:
%    \begin{macrocode}
\input{childdoc.def}
\childdocby{cdocsamp}
%    \end{macrocode}

%\iffalse
%</samplepart3|samplepart4>
%\fi
%
%\iffalse
%<*samplepart3>
%\fi
% Some text for part 3:
%    \begin{macrocode}
some text in part three
%    \end{macrocode}

%\iffalse
%</samplepart3>
%\fi
% Some text for part 4:
%\iffalse
%<*samplepart4>
%\fi
%    \begin{macrocode}
more text in part four
%    \end{macrocode}

%\iffalse
%</samplepart4>
%\fi
%
% %%%%%%%%%%%%%%%%%%%%%%%%%%%%%%%%%%%%%%
% \paragraph{Forwarding for a Complete Draft.}
%
% The following forwarding file |cdocsdrf.tex|
% compiles the main document in draft mode:
%\iffalse
%<*sampledraft>
%\fi
%    \begin{macrocode}
\def\version{draft}
\input{childdoc.def}
\childdocforward{cdocsamp}
%    \end{macrocode}

%\iffalse
%</sampledraft>
%\fi
%
% %%%%%%%%%%%%%%%%%%%%%%%%%%%%%%%%%%%%%%
% \paragraph{Forwarding for Final Version of the Chapters.}
%
% The following forwarding files |cdocsfn1.tex| and |cdocsfn2.tex|
% (with identical content)
% compile the final versions of the child documents
% |cdocsch1.tex| and |cdocsch2.tex|, respectively:
%\iffalse
%<*samplefinal>
%\fi
%    \begin{macrocode}
\def\version{final}
\input{childdoc.def}
\childdocforwardprefix[cdocsamp]{cdocsfn}{cdocsch}
%    \end{macrocode}

%\iffalse
%</samplefinal>
%\fi
%
% %%%%%%%%%%%%%%%%%%%%%%%%%%%%%%%%%%%%%%
% \paragraph{Command Line Processing.}
%
% The following three command lines generate the output files
% |cdocscld|, |cdocscl1| and |cdocscl2|
% which should be identical to
% |cdocsdrf|, |cdocsch1| and |cdocsfn2|, respectively:
% \begin{center}
% \begin{tabular}{l}
% |latex -jobname cdocscld \|\\
% |  "\def\version{draft}\input{childdoc.def}\childdocforward{cdocsamp}"|\\
% |latex -jobname cdocscl1 \|\\
% |  "\input{childdoc.def}\childdocforward[cdocsamp]{cdocsch1}"|\\
% |latex -jobname cdocscl2 \|\\
% |  "\def\version{final}\input{childdoc.def}\childdocforward{cdocsch2}"|
% \end{tabular}
% \end{center}
% Note that the trailing backslash on each first line
% merely continues the input to the second line
% (for convenient cut ant paste).
% Furthermore, the command |latex| can be replaced by any
% of its alternative versions such as |pdflatex|.
%
% %%%%%%%%%%%%%%%%%%%%%%%%%%%%%%%%%%%%%%%%%%%%%%%%%%%%%%%%%%%%%%%%%%%%%%%%%%%%%%
% %%%%%%%%%%%%%%%%%%%%%%%%%%%%%%%%%%%%%%%%%%%%%%%%%%%%%%%%%%%%%%%%%%%%%%%%%%%%%%
% \section{Implementation}
%\iffalse
%<*package>
%\fi
%
% This section describes the definitions file |childdoc.def|.

% The definitions cannot be loaded using |\usepackage| or |\RequirePackage|
% which has a mechanism to prevent loading a style file more than once.
% When loading the definitions by means of |\input|
% multiple instances have to be prevented manually:
%\iffalse
%This code needs to be before the `\ProvidesFile' directive
%which is defined at the beginning of this file.
%Therefore it is also placed there and commented out here.
%</package>
%<*discard>
%\fi
%    \begin{macrocode}
\ifdefined\childdocmain\endinput\fi
%    \end{macrocode}
%\iffalse
%</discard>
%<*package>
%\fi
%
% \macro{\ifchilddoc}
% \macro{\ifchilddocmanual}
% The conditional |\ifchilddoc| tells whether a
% child (true) or main (false) document is being compiled.
% The conditional |\ifchilddocmanual| tells whether
% the |\includeonly| mechanism is used (false) or
% the selection of child files must be performed manually (true).
% The definitions initialise to false:
%    \begin{macrocode}
\newif\ifchilddoc
\newif\ifchilddocmanual
%    \end{macrocode}

% \macro{\childdocname}
% \macro{\childdocjob}
% The macro |\childdocname| stores the name of the main document
% to be compiled. The macro |\childdocjob| stores the name of
% the document on which the \LaTeX{} compiler was originally invoked.
% The content of |\jobname| cannot be compared
% to filenames specified in the source due to different catcodes.
% The following code rescans |\jobname|, stores the result
% in |\childdocname| and saves a copy in |\childdocjob|:
%    \begin{macrocode}
\edef\childdocname{\scantokens\expandafter{\jobname\noexpand}}
\let\childdocjob\childdocname
%    \end{macrocode}

% \macro{\childdocdisable}
% The macro |\childdocdisable| prevents the main file
% from being processed more than once.
% At this stage, the main document command |\childdocmain|
% is assumed to be called once again where it should do nothing.
% Any subsequent call to it should prevent
% a secondary processing of the main document
% It overwrites the forwarding commands
% |\childdocof| and |\childdocforward|
% with empty macros to prevent further inclusions of the main document:
%    \begin{macrocode}
\newcommand{\childdocdisable}
{
  \renewcommand{\childdocmain}[1]{\renewcommand{\childdocmain}[1]{\endinput}}
  \renewcommand{\childdocof}[1]{}
  \renewcommand{\childdocby}[2][]{}
  \renewcommand{\childdocforward}[2][]{}
  \renewcommand{\childdocdisable}{}
}
%    \end{macrocode}

% \macro{\childdocmain}
% The macro |\childdocmain| is to be called at the top of the main file
% with nothing or the main filename (without extension) as argument.
% First, it breaks loops.
% If the argument is not empty and does not match |\childdocname|
% (which is set by the first inclusion of |childdoc.def|),
% |\ifchilddoc| is set to true, |\includeonly| is applied to the child file
% and |\jobname| is set to the main file
% (for proper handling of |.aux| files):
%    \begin{macrocode}
\newcommand{\childdocmain}[1]
{
  \childdocdisable\childdocmain{}
  \if?#1?\else
    \begingroup
      \def\childdoctmp{#1}
      \ifx\childdoctmp\childdocname
        \def\childdoctmp{}
      \else
        \def\childdoctmp
        {
          \childdoctrue
          \includeonly{\childdocname}
          \def\childdocjob{#1}
          \def\jobname{#1}
        }
      \fi
      \expandafter
    \endgroup
    \childdoctmp
  \fi
}
%    \end{macrocode}

% \macro{\childdocof}
% The command |\childdocof| redirects
% compilation to the main file |#1|.
%    \begin{macrocode}
\newcommand{\childdocof}[1]
{
  \childdocdisable
  \childdoctrue
  \includeonly{\childdocname}
  \def\jobname{#1}
  \def\childdocjob{#1}
  \input{#1}
}
%    \end{macrocode}

% \macro{\childdocby}
% The command |\childdocby| ....
%    \begin{macrocode}
\newcommand{\childdocby}[2][]
{
  \childdocdisable
  \childdoctrue
  \childdocmanualtrue
  \if?#1?\else
    \def\jobname{#2}
  \fi
  \def\childdocjob{#2}
  \input{#2}
  \endinput
}
%    \end{macrocode}

% \macro{\childdocforward}
% The command |\childdocforward| redirects
% compilation to the main file or
% (if the optional argument is given) a child file.
% Parameters are set as if the main file
% or a child file starting with |\childdocof| was compiled.
% Then compilation is handed over to the main file:
%    \begin{macrocode}
\newcommand{\childdocforward}[2][]
{
  \begingroup
    \if?#1?
      \def\childdoctmp
      {
        \def\childdocname{#2}
        \def\childdocjob{#2}
        \def\jobname{#2}
        \input{#2}
        \endinput
      }
    \else
      \def\childdoctmp
      {
        \childdocdisable
        \def\childdocname{#2}
        \childdoctrue
        \includeonly{#2}
        \def\childdocjob{#1}
        \def\jobname{#1}
        \input{#1}
        \endinput
      }
    \fi
    \expandafter
  \endgroup
  \childdoctmp
}
%    \end{macrocode}

% \macro{\childdocforwardprefix}
% The command |\childdocforwardprefix| redirects
% compilation to the main or a child file by means of a pattern.
% The prefix |#1| in the current filename is replaced by |#2|
% and the suffix of the current filename is kept
% (it is assumed that the filename does not contain the substring `|~~~|'
% which is used as a delimiter).
% Compilation is handed over to the new file by |\childdocforward|:
%    \begin{macrocode}
\newcommand{\childdocforwardprefix}[3][]
{
  \begingroup
    \def\childdocextract #2##1~~~{\def\childdoctmp{\childdocforward[#1]{#3##1}}}
    \expandafter\childdocextract\childdocname~~~
    \expandafter
  \endgroup
  \childdoctmp
}
%    \end{macrocode}

% \macro{\childdoc}
% The deprecated macro |\childdoc| is a legacy version of |\childdocmain|:
%    \begin{macrocode}
\newcommand{\childdoc}{\childdocmain}
%    \end{macrocode}

% \macro{\childdocredirect}
% The deprecated macro |\childdocredirect| is a legacy version
% of |\childdocforward| and |\childdocforwardprefix|:
%    \begin{macrocode}
\newcommand{\childdocredirect}[2][]
{
  \begingroup
    \if?#1?
      \def\childdoctmp{\childdocforward{#2}}
    \else
      \def\childdoctmp{\childdocforwardprefix{#1}{#2}}
    \fi
    \expandafter
  \endgroup
  \childdoctmp
}
%    \end{macrocode}

%\iffalse
%</package>
%\fi
%
\endinput
\childdocforward{cdocsamp}"|\\
% |latex -jobname cdocscl1 \|\\
% |  "% \iffalse
%
% childdoc.dtx Copyright (C) 2017-2018 Niklas Beisert
%
% This work may be distributed and/or modified under the
% conditions of the LaTeX Project Public License, either version 1.3
% of this license or (at your option) any later version.
% The latest version of this license is in
%   http://www.latex-project.org/lppl.txt
% and version 1.3 or later is part of all distributions of LaTeX
% version 2005/12/01 or later.
%
% This work has the LPPL maintenance status `maintained'.
%
% The Current Maintainer of this work is Niklas Beisert.
%
% This work consists of the files childdoc.dtx and childdoc.ins
% and the derived files childdoc.def and cdocsamp.tex with
% cdocsch1.tex, cdocsch2.tex, cdocsdrf.tex, cdocsfn1.tex, cdocsfn2.tex.
%
%<package>\ifdefined\childdocmain\endinput\fi
%<package>\ProvidesFile{childdoc.def}[2018/12/30 v2.0 child document driver]
%<samplemain>\ProvidesFile{cdocsamp.tex}[2018/12/30 v2.0 sample for childdoc]
%<*driver>
%\ProvidesFile{childdoc.drv}[2018/12/30 v2.0 childdoc reference manual file]
\PassOptionsToClass{10pt,a4paper}{article}
\documentclass{ltxdoc}

\usepackage[margin=35mm]{geometry}
\usepackage{hyperref}
\usepackage{hyperxmp}
\usepackage[usenames]{color}

\hypersetup{colorlinks=true}
\hypersetup{pdfstartview=FitH}
\hypersetup{pdfpagemode=UseNone}
\hypersetup{pdfsource={}}
\hypersetup{pdflang={en-UK}}
\hypersetup{pdfcopyright={Copyright 2017-2018 Niklas Beisert.
  This work may be distributed and/or modified under the
  conditions of the LaTeX Project Public License, either version 1.3
  of this license or (at your option) any later version.}}
\hypersetup{pdflicenseurl={http://www.latex-project.org/lppl.txt}}
\hypersetup{pdfcontactaddress={ETH Zurich, ITP, HIT K,
  Wolfgang-Pauli-Strasse 27}}
\hypersetup{pdfcontactpostcode={8093}}
\hypersetup{pdfcontactcity={Zurich}}
\hypersetup{pdfcontactcountry={Switzerland}}
\hypersetup{pdfcontactemail={nbeisert@itp.phys.ethz.ch}}
\hypersetup{pdfcontacturl={http://people.phys.ethz.ch/\xmptilde nbeisert/}}

\newcommand{\secref}[1]{\hyperref[#1]{section \ref*{#1}}}

\parskip1ex
\parindent0pt
\let\olditemize\itemize
\def\itemize{\olditemize\parskip0pt}

\begin{document}

\title{The \textsf{childdoc} Package}
\hypersetup{pdftitle={The childdoc Package}}
\author{Niklas Beisert\\[2ex]
  Institut f\"ur Theoretische Physik\\
  Eidgen\"ossische Technische Hochschule Z\"urich\\
  Wolfgang-Pauli-Strasse 27, 8093 Z\"urich, Switzerland\\[1ex]
  \href{mailto:nbeisert@itp.phys.ethz.ch}
  {\texttt{nbeisert@itp.phys.ethz.ch}}}
\hypersetup{pdfauthor={Niklas Beisert}}
\hypersetup{pdfsubject={Manual for the LaTeX2e Package childdoc}}
\date{30 December 2018, \textsf{v2.0}}
\maketitle

\begin{abstract}\noindent
\textsf{childdoc} is a \LaTeXe{} package
that enables the direct compilation
of document sections included by |\include|
to individual files.
\end{abstract}

\begingroup
\parskip0ex
\tableofcontents
\endgroup

%%%%%%%%%%%%%%%%%%%%%%%%%%%%%%%%%%%%%%%%%%%%%%%%%%%%%%%%%%%%%%%%%%%%%%%%%%%%%%%%
%%%%%%%%%%%%%%%%%%%%%%%%%%%%%%%%%%%%%%%%%%%%%%%%%%%%%%%%%%%%%%%%%%%%%%%%%%%%%%%%
\section{Introduction}

\LaTeX{} provides a mechanism to structure a large document (such as a book)
into a main file and several child files (containing the chapters)
using the |\include| command.
This mechanism is beneficial for documents
which span hundreds of pages in order to
make the source file(s) more manageable.
Moreover, compilation can be restricted to
selected child files by means of the |\includeonly| command.
The latter feature can be used to reduce the compilation time while editing
(this was significantly more useful in the earlier days of \LaTeX{})
or to generate a smaller document which is easier to navigate.
Another application of |\includeonly| is to generate
documents consisting of selected parts of the complete document.

However, there are a few drawbacks of the plain |\include| mechanism:
\begin{itemize}
\item
The child files cannot be compiled on their own,
they can only be compiled via the main file.
A naive editing environment
(such as a text editor with an option
to have the current file processed by \LaTeX)
may require one to switch to the main file before compiling;
attempting to compile the child file produces errors.
\item
The main file must be modified (each time)
to adjust the |\includeonly| command
to the present needs. This easily leaves the main file in a messy state.
\item
The generated document will always carry the filename
of the main document. This is inconvenient if
several child files are to be compiled and
to be kept for distribution.
\end{itemize}

The present package provides a simple interface
to make child files individually compilable by \LaTeX{}.
Compiling a child file then has the same effect as compiling
the main file with an |\includeonly| command
to select the appropriate child.
Moreover the generated document will carry the name of the child
rather than the main file.
This resolves all three above issues.

This feature is meant to make the editing of books,
thesis documents and lecture notes somewhat more convenient.
However, the package can also be used efficiently for
composing a series of documents (such as exercise sheets)
which are typically distributed individually.
It then assists the author in generating the individual documents
(potentially in different versions)
as well as a document containing the collected series.
Another application is in developing style files
or other kinds of included material
where compilation of the style file could redirect
to a sample or test file.

%%%%%%%%%%%%%%%%%%%%%%%%%%%%%%%%%%%%%%%%%%%%%%%%%%%%%%%%%%%%%%%%%%%%%%%%%%%%%%%%
%%%%%%%%%%%%%%%%%%%%%%%%%%%%%%%%%%%%%%%%%%%%%%%%%%%%%%%%%%%%%%%%%%%%%%%%%%%%%%%%
\section{Usage}

First of all, the package \textsf{childdoc} is \emph{not} a standard
\LaTeXe{} |.sty| style file! Therefore it needs to be invoked in
a non-standard way.

%%%%%%%%%%%%%%%%%%%%%%%%%%%%%%%%%%%%%%%%%%%%%%%%%%%%%%%%%%%%%%%%%%%%%%%%%%%%%%%%
\subsection{Included Files}
\label{sec:include}

%%%%%%%%%%%%%%%%%%%%%%%%%%%%%%%%%%%%%%%%
\DescribeMacro{\childdocmain}
To use the package, add the commands
\begin{center}
\begin{tabular}{l}
|\input{childdoc.def}|\\
|\childdocmain{}|\\
\end{tabular}
\end{center}
at the very top of the main \LaTeX{} file,
in particular \emph{before} the |\documentclass| statement!
The argument of |\childdocmain| should be left empty
(but it must be present).

%%%%%%%%%%%%%%%%%%%%%%%%%%%%%%%%%%%%%%%%
\DescribeMacro{\childdocof}
Furthermore, add the commands
\begin{center}
\begin{tabular}{l}
|\input{childdoc.def}|\\
|\childdocof{|\textit{main}|}|\\
\end{tabular}
\end{center}
at the top of every child file \textit{child}
which is included by |\include{|\textit{child}|}|
from within the main file
(or at least for those files to be compiled individually).
The argument \textit{main} must be the filename of the main file.

There are a couple of
considerations in setting up the main and child documents:

%%%%%%%%%%%%%%%%%%%%%%%%%%%%%%%%%%%%%%%%
\paragraph{Restrictions.}

Please note the following restrictions:
\begin{itemize}
\item
|\childdocmain| must be called with one argument \textit{main}
to ensure compatibility with earlier version of the package.
It must either be empty (|\childdocmain{}|)
or precisely match the filename of the main file in which it is specified.
See \secref{sec:detection} for further information.
\item
The filename \textit{main} must be specified without the |.tex| extension.
\item
The filename \textit{main} is case sensitive
(even in case-insensitive file systems)
due to internal string comparison.
\item
The argument \textit{main} should be fully expanded, it cannot be a macro.
\item
Subdirectories and special characters should be avoided in filenames.
\item
The command |\childdocmain{|\textit{main}|}| must be followed by a whitespace.
It should not be followed immediately by another command
or by a comment mark `|%|'.
This is because the \TeX{} parser reads the token immediately following
the argument of |\childdocmain| and puts it
at the beginning of every child section;
however, a white\-space is ignored.
\end{itemize}

%%%%%%%%%%%%%%%%%%%%%%%%%%%%%%%%%%%%%%%%
\paragraph{Content of Main File.}

It is advisable to place all content in the child files included by |\include|.
Any output contained in the main file will appear in all child documents
unless suppressed manually;
it cannot be suppressed automatically by the |\includeonly| directive
and thus should normally be avoided.
A method to include some content in the main file
by means of conditional processing is described in \secref{sec:conditional}.

%%%%%%%%%%%%%%%%%%%%%%%%%%%%%%%%%%%%%%%%
\paragraph{Page Numbering.}

When only a part of the document is compiled,
the appropriate numbering of pages
(as well as other status parameters)
is determined from the |.aux| files.
The latter contain information from previous passes.
However this information needs to propagate through
all intermediate child documents.
Therefore the page numbering in child documents may well
be inconsistent until the complete document is compiled at least once.

A useful (if unconventional) way to always ensure a consistent
page numbering is to restart the numbering in each child document
and denote the pages by `\textit{child}|.|\textit{page}'
where \textit{child} represents the chapter/section number of the child file.
This can be achieved by the command
|\numberwithin{page}{|\textit{child}|}|
of the \textsf{amsmath} package
where \textit{child} can be |chapter| or |section|
depending on the chosen structuring.
Alternatively, one can modify the macro |\thepage| appropriately
and reset the counter |page| at the start of each child file.

%%%%%%%%%%%%%%%%%%%%%%%%%%%%%%%%%%%%%%%%%%%%%%%%%%%%%%%%%%%%%%%%%%%%%%%%%%%%%%%%
\subsection{Conditional Processing}
\label{sec:conditional}

The package provides a mechanism to compile different versions
of a document. To customise the versions further some conditional processing
can come in handy to distinguish which version is being compiled.
The package provides two macros to describe the compilation context:

%%%%%%%%%%%%%%%%%%%%%%%%%%%%%%%%%%%%%%%%
\DescribeMacro{\ifchilddoc}
The conditional |\ifchilddoc| distinguishes between the compilation of
child documents and the main document:
%
\begin{center}
|\ifchilddoc |\textit{child-code}| |[|\||else |\textit{main-code}]| \||fi|
\end{center}

%%%%%%%%%%%%%%%%%%%%%%%%%%%%%%%%%%%%%%%%
\DescribeMacro{\childdocname}
\DescribeMacro{\childdocjob}
The macro |\childdocname| contains the filename (without extension)
of the main or child file being processed.
Note that |\childdocjob| will always contain the name of the main file.

%%%%%%%%%%%%%%%%%%%%%%%%%%%%%%%%%%%%%%%%
\paragraph{Title Page.}

Conditional processing can be used to include a title or banner page
in the main document when proper precautions are taken.
Importantly, the code in the main file should ensure that the page counter
(as well as other status parameters which are stored in the |.aux| files)
takes the same value after the conditional processing.
Otherwise the page numbers may take divergent values
depending on which part is compiled.

For example, a title page could be declared by:
%
\begin{center}
\begin{tabular}{l}
|\ifchilddoc\||else|\\
|\addtocounter{page}{-1}|\\
\textit{code for title page}\\
|\newpage|\\
|\||fi|
\end{tabular}
\end{center}
%
A banner page for the child documents can be generated by:
%
\begin{center}
\begin{tabular}{l}
|\ifchilddoc|\\
|\addtocounter{page}{-1}|\\
\textit{code for banner page}\\
|\newpage|\\
|\||fi|
\end{tabular}
\end{center}
%
Here one could write a message such as:
\begin{center}
|This is the part \childdocname{} of \childdocjob{}.|
\end{center}

%%%%%%%%%%%%%%%%%%%%%%%%%%%%%%%%%%%%%%%%%%%%%%%%%%%%%%%%%%%%%%%%%%%%%%%%%%%%%%%%
\subsection{Flags}
\label{sec:flags}

The package makes it easy to generate different versions
of the main or child documents.
To this end compilation flags can be defined
and assigned different default values.
They will be particularly useful in conjunction
with the forwarding mechanism described in \secref{sec:forward}.

For example, it may be useful to have a flag |\version|
which can be set to |draft| or |final|.
The document source will contain some conditional code
depending on the value of |\version|.
Suppose further, the flag should default to |final| for the main file
and to |draft| for child files
which is a natural assignment for editing the document.
This is achieved by placing the following code
in the preamble of the main document
(below the |\childdocmain| directive):
%
\begin{center}
\begin{tabular}{l}
|\ifchilddoc|\\
|\providecommand{\version}{draft}|\\
|\||else|\\
|\providecommand{\version}{final}|\\
|\||fi|
\end{tabular}
\end{center}
%
The definition by |\providecommand| makes sure
that previous definitions are not overwritten.
Further statements |\providecommand{\version}{...}|
can thus be added before the above code to override it.

For the main file, one might add a line
(between |\childdocmain| and the above block)
%
\begin{center}
|%\ifchilddoc\||else\providecommand{\version}{draft}\||fi|
\end{center}
%
which can be uncommented to produce a draft version.
Likewise one can add a line to the very top of a child file
(above the |\childdocof{|\textit{main}|}| directive)
%
\begin{center}
|%\providecommand{\version}{final}|
\end{center}
%
which can be uncommented to produce the final version of this child document.

%%%%%%%%%%%%%%%%%%%%%%%%%%%%%%%%%%%%%%%%%%%%%%%%%%%%%%%%%%%%%%%%%%%%%%%%%%%%%%%%
\subsection{Forwarding}
\label{sec:forward}

Different versions of the main or child documents
using compilation flags as described in \secref{sec:flags}
can be (permanently) stored in different files
for convenient compilation, viewing and distribution.
To this end, the package defines a command
to pass on compilation to a different file:

%%%%%%%%%%%%%%%%%%%%%%%%%%%%%%%%%%%%%%%%
\DescribeMacro{\childdocforward}
The command |\childdocforward| redirects processing to
another source file:
%
\begin{center}
\begin{tabular}{l}
|\input{childdoc.def}|\\
|\childdocforward[|\textit{main}|]{|\textit{dest}|}|\\
\end{tabular}
\end{center}
%
The argument \textit{dest} is the destination file
(without extension).
It should be the main file or one of the child files.
Note that further \textsf{childdoc} directives
such as |\childdocof| and |\childdocforward|
in the indicated file will be processed in this form.
The optional argument \textit{main}
passes on directly to the main file \textit{main}
while pretending to compile the child \textit{dest}.
This form behaves as if \textit{dest}
issues |\childdocof{|\textit{main}|}| right away,
and no further \textsf{childdoc} directives will be processed.

%%%%%%%%%%%%%%%%%%%%%%%%%%%%%%%%%%%%%%%%
\DescribeMacro{\...prefix}
In the alternative form |\childdocforwardprefix|,
%
\begin{center}
\begin{tabular}{l}
|\input{childdoc.def}|\\
|\childdocforwardprefix[|\textit{main}|]{|\textit{prefix}|}{|\textit{dest}|}|
\end{tabular}
\end{center}
%
the destination file is determined by a pattern
depending on the current file:
To make this work, the current file must be called
`{\textit{prefix}\hspace{0.2em}\textit{suffix}}'
with \textit{prefix} matching precisely the argument.
Processing is then passed on to the file
`{\textit{dest}\hspace{0.2em}\textit{suffix}}'.
Surely, the same effect is achieved by
directly specifying the
argument `{\textit{dest}\hspace{0.2em}\textit{suffix}}'
in the first form.
However, that requires to set up a different file
for each child. With the alternative form of the command
all these files can have exactly the same content
which simplifies setting them up and maintaining them.

For example, the following file |draft.tex|
with a compilation flag |\version| as described in \secref{sec:flags}
compiles the main document as a draft:
%
\begin{center}
\begin{tabular}{l}
|\def\version{draft}|\\
|\input{childdoc.def}|\\
|\childdocforward{|\textit{main}|}|
\end{tabular}
\end{center}
%
Likewise, the following files |final|\textit{nn}|.tex|
compile the final version of the child document
|child|\textit{nn}|.tex|:
%
\begin{center}
\begin{tabular}{l}
|\def\version{final}|\\
|\input{childdoc.def}|\\
|\childdocforwardprefix{final}{child}|
\end{tabular}
\end{center}
%

Note that when several versions of a main file and/or of each child file
are to be generated, it may be convenient to set up a |Makefile| or
shell script to automatise the process.

%%%%%%%%%%%%%%%%%%%%%%%%%%%%%%%%%%%%%%%%%%%%%%%%%%%%%%%%%%%%%%%%%%%%%%%%%%%%%%%%
\subsection{Command Line Processing}
\label{sec:commandline}

The effect of redirection files can also be achieved by invoking
the \LaTeX{} compiler with a more elaborate command line.
Most conveniently this should be done as part
of a shell script or a |Makefile|.

When using \textsf{childdoc} in the main file, the following
command lines effectively perform a redirection
(note that depending on the shell being used,
backslashes may have to be doubled: `|\|' $\to$ `|\\|'):
%
\begin{center}
|... -jobname "|\textit{target}|" |\\|"|[\textit{flags}]%
|\input{childdoc.def}\childdocforward[|\textit{main}|]{|\textit{dest}|}"|
\end{center}
%
Here \textit{target} is the name of the output file,
\textit{main} is the name of the main file
and \textit{dest} is the name of the main or child file to be processed
(all filenames without extensions).
The optional argument \textit{main} can be omitted
if \textit{main} matches \textit{dest}.
Optionally, compilation \textit{flags} can be defined via |\def| commands.
This command line makes the \TeX{} engine believe
it is compiling the file \textit{target}
whose content is specified as the latter parameter.
The provided code then forwards the processing to
\textit{main} or \textit{dest} as described in \secref{sec:forward}.

%%%%%%%%%%%%%%%%%%%%%%%%%%%%%%%%%%%%%%%%%%%%%%%%%%%%%%%%%%%%%%%%%%%%%%%%%%%%%%%%
\subsection{Include by Input}
\label{sec:input}

Including child documents by |\include| has some restrictions by design.
Most notably, the content of a child document always occupies
its own set of pages; pages cannot be shared between child documents.
Usually, this behaviour makes perfect sense
because each child document contain an essential part of the document.
However, in some situations it may be desirable to compose
a document from a collection of parts
without having mandatory page breaks between then.
For this case, the package
provides a mechanism to include parts
by |\input| which can also be processed individually.
However, by construction this mechanism
requires manual handling of the content to be output.

%%%%%%%%%%%%%%%%%%%%%%%%%%%%%%%%%%%%%%%%
\DescribeMacro{\ifchilddocmanual}
The main file should be prepared as usual, see \secref{sec:include}.
However, the document body must make a distinction
between processing of an individual part and of the main document, e.g.:
%
\begin{center}
\begin{tabular}{l}
|\ifchilddocmanual|\\
|\input{\childdocname}|\\
|\||else|\\
\textit{document body with }|\input{|\textit{part}|}|\\
|\||fi|
\end{tabular}
\end{center}
%
The conditional |\ifchilddocmanual| is true whenever
a part to be included by |\input| is being compiled,
and the name of the part is stored in |\childdocname|.

%%%%%%%%%%%%%%%%%%%%%%%%%%%%%%%%%%%%%%%%
\DescribeMacro{\childdocby}
Each part to be included by |\input| should start with:
%
\begin{center}
\begin{tabular}{l}
|\input{childdoc.def}|\\
|\childdocby{|\textit{main}|}|\\
\end{tabular}
\end{center}
%
The directive |\childdocby| is similar to |\childdocof|
described in \secref{sec:include},
but the subsequent selection of content must be done manually.
To that end, both |\ifchilddoc| and |\ifchilddocmanual|
will be true upon processing of a part,
and the name of the part is stored in |\childdocname|.
Note that |\jobname| will be set to the filename of the current part
so that each part receives an individual |.aux| file
that does not interfere with the |.aux| file(s) of the main document.
This behaviour can be altered by the alternative form
|\childdocby[*]{|\textit{main}|}| (with a non-empty optional argument)
which uses the |.aux| file of the main document
by setting |\jobname| to \textit{main}.

%%%%%%%%%%%%%%%%%%%%%%%%%%%%%%%%%%%%%%%%%%%%%%%%%%%%%%%%%%%%%%%%%%%%%%%%%%%%%%%%
\subsection{Driver Development}
\label{sec:driver}

The \textsf{childdoc} mechanism can also be use for the development
of definition files such as \LaTeX{} styles or classes.
This case differs from the above setup with multiple parts
included by |\include| in that no |\includeonly| should be invoked.
This can be achieved by starting the include file
(before |\ProvidesPackage|) with:
%
\begin{center}
\begin{tabular}{l}
|\input{childdoc.def}|\\
|\childdocforward{|\textit{main}|}|\\
\end{tabular}
\end{center}
%
or alternatively with:
%
\begin{center}
\begin{tabular}{l}
|\input{childdoc.def}|\\
|\childdocby{|\textit{main}|}|\\
\end{tabular}
\end{center}
%
Both forms have slightly different effects as described above.
The main file is prepared as usual, see \secref{sec:include}.

%%%%%%%%%%%%%%%%%%%%%%%%%%%%%%%%%%%%%%%%%%%%%%%%%%%%%%%%%%%%%%%%%%%%%%%%%%%%%%%%
\subsection{Legacy Detection}
\label{sec:detection}

The directive |\childdocmain| in the main file can detect
whether the complete document or merely a child is to be compiled
even without using the directive |\childdocof|.
This method is deprecated because it is less robust
and there is no compelling reason to use it;
it is merely provided for backward compatibility
and it may be removed in future versions.

If the detection mechanism is to be used,
it is mandatory to correctly specify
the filename of the main file as the argument of |\childdocmain|:
%
\begin{center}
\begin{tabular}{l}
|\input{childdoc.def}|\\
|\childdocmain{|\textit{main}|}|\\
\end{tabular}
\end{center}
%
If |\jobname| does not match the argument \textit{main} of |\childdocmain|,
it is assumed that |\jobname| points to the child file to be compiled.
When using |\childdocmain| with the main file specified as argument,
it suffices to start a child file
with just |\input{|\textit{main}|}|
without loading of the package and using |\childdocof|.
If instead all processing is done
with the appropriate \textsf{childdoc} directives,
the argument of \textit{main} of |\childdocmain| can be empty.

An alternative version of the command line processing described
in \secref{sec:commandline} using the detection mechanism reads:
%
\begin{center}
|... -jobname "|\textit{target}|" "|[\textit{flags}]%
[|\def\jobname{|\textit{dest}|}|]|\input{|\textit{main}|}"|
\end{center}

%%%%%%%%%%%%%%%%%%%%%%%%%%%%%%%%%%%%%%%%%%%%%%%%%%%%%%%%%%%%%%%%%%%%%%%%%%%%%%%%
\subsection{Manual Code}
\label{sec:manual}

In case one cannot be certain whether the definitions file |childdoc.def|
is installed on the target \TeX{} distribution
and one prefers not to ship it,
it is conceivable to paste a few relevant commands into the sources.

To that end, drop all statements |\input{childdoc.def}|
and perform the replacements as outlined below.
Instead of |\childdocmain{|\textit{main}|}| add the following code
to the top of the main file:
%
\begin{center}
\begin{tabular}{l}
|\||ifdefined\childdocname\endinput\||fi\newif\ifchilddoc|\\
|\edef\childdocname{\scantokens\expandafter{\jobname\noexpand}}|\\
|\def\childdocmain{|\textit{main}|}\||ifx\childdocmain\childdocname\||else|\\
|\childdoctrue\includeonly{\childdocname}\let\jobname\childdocmain\||fi|\\
\end{tabular}
\end{center}
%
Instead of |\childdocof{|\textit{main}|}| just include the main file
at the top of each child file:
%
\begin{center}
|\input{|\textit{main}|}|
\end{center}
%
A simple redirection |\childdocforward{|\textit{dest}|}| is achieved by:
%
\begin{center}
|\def\jobname{|\textit{dest}|}\input{\jobname}|
\end{center}
%
The redirection with prefix
|\childdocforwardprefix[|\textit{prefix}|]{|\textit{dest}|}|
is accomplished by:
%
\begin{center}
\begin{tabular}{l}
|{\edef\jobname{\scantokens\expandafter{\jobname\noexpand}}|\\
|\def\redirectjob |\textit{prefix}|#1~~~{\gdef\jobname{|\textit{dest}|#1}}|\\
|\expandafter\redirectjob\jobname~~~}\input{\jobname}|
\end{tabular}
\end{center}

In an alternative approach,
child documents can be compiled by a specific command line
without additional code or specific definitions:
%
\begin{center}
|... -jobname "|\textit{target}|" "|[\textit{flags}]%
|\includeonly{|\textit{dest}|}\input{|\textit{main}|}"|
\end{center}
%

%%%%%%%%%%%%%%%%%%%%%%%%%%%%%%%%%%%%%%%%%%%%%%%%%%%%%%%%%%%%%%%%%%%%%%%%%%%%%%%%
%%%%%%%%%%%%%%%%%%%%%%%%%%%%%%%%%%%%%%%%%%%%%%%%%%%%%%%%%%%%%%%%%%%%%%%%%%%%%%%%
\section{Information}

%%%%%%%%%%%%%%%%%%%%%%%%%%%%%%%%%%%%%%%%%%%%%%%%%%%%%%%%%%%%%%%%%%%%%%%%%%%%%%%%
\subsection{Copyright}

Copyright \copyright{} 2017--2018 Niklas Beisert

This work may be distributed and/or modified under the
conditions of the \LaTeX{} Project Public License, either version 1.3
of this license or (at your option) any later version.
The latest version of this license is in
  \url{http://www.latex-project.org/lppl.txt}
and version 1.3 or later is part of all distributions of \LaTeX{}
version 2005/12/01 or later.

This work has the LPPL maintenance status `maintained'.

The Current Maintainer of this work is Niklas Beisert.

This work consists of the files |README.txt|, |childdoc.ins| and |childdoc.dtx|
as well as the derived files |childdoc.def|, |cdocsamp.tex|
with |cdocsch1.tex|, |cdocsch2.tex|, |cdocspt3.tex|, |cdocspt4.tex|,
|cdocsdrf.tex|, |cdocsfn1.tex|, |cdocsfn2.tex|
as well as |childdoc.pdf|.

%%%%%%%%%%%%%%%%%%%%%%%%%%%%%%%%%%%%%%%%%%%%%%%%%%%%%%%%%%%%%%%%%%%%%%%%%%%%%%%%
\subsection{Files and Installation}

The package consists of the files:
%
\begin{center}
\begin{tabular}{ll}
    |README.txt|   & readme file \\
    |childdoc.ins| & installation file \\
    |childdoc.dtx| & source file \\
    |childdoc.def| & definition file \\
    |cdocsamp.tex| & sample main file \\
    |cdocsch1.tex| & sample include file \\
    |cdocsch2.tex| & sample include file \\
    |cdocspt3.tex| & sample part file \\
    |cdocspt4.tex| & sample part file \\
    |cdocsdrf.tex| & sample redirection file \\
    |cdocsfn1.tex| & sample redirection file \\
    |cdocsfn2.tex| & sample redirection file \\
    |childdoc.pdf| & manual
\end{tabular}
\end{center}
%
The distribution consists of the files
|README.txt|, |childdoc.ins| and |childdoc.dtx|.
%
\begin{itemize}
\item
Run (pdf)\LaTeX{} on |childdoc.dtx|
to compile the manual |childdoc.pdf| (this file).
\item
Run \LaTeX{} on |childdoc.ins| to create the definitions file |childdoc.def|
and the sample |cdocsamp.tex| with include files
|cdocsch1.tex|, |cdocsch2.tex|, |cdocspt3.tex|, |cdocspt4.tex|,
|cdocsdrf.tex|, |cdocsfn1.tex|, |cdocsfn2.tex|.
Then copy the file |childdoc.def| to an appropriate directory of your \LaTeX{}
distribution, e.g.\ \textit{texmf-root}|/tex/latex/childdoc|.
\end{itemize}

%%%%%%%%%%%%%%%%%%%%%%%%%%%%%%%%%%%%%%%%%%%%%%%%%%%%%%%%%%%%%%%%%%%%%%%%%%%%%%%%
\subsection{Related CTAN Packages}

There are several other packages which offer a similar functionality:
%
\begin{itemize}
\item
The packages
\href{http://ctan.org/pkg/docmute}{\textsf{docmute}},
\href{http://ctan.org/pkg/includex}{\textsf{includex}} and
\href{http://ctan.org/pkg/standalone}{\textsf{standalone}}
provide commands to include only the document body of
a child file thus allowing both files to be compiled individually.
\item
The packages \href{http://ctan.org/pkg/subdocs}{\textsf{subdocs}}
and \href{http://ctan.org/pkg/subfiles}{\textsf{subfiles}}
provide structures in which the main and child documents can be
encapsulated and allowing them to be compiled individually.
The inclusion mechanism is different from the conventional |\include|.
\item
The package \href{http://ctan.org/pkg/combine}{\textsf{combine}}
is an elaborate solution to combine several documents into one.
\end{itemize}
%
See also the CTAN topic \href{http://ctan.org/topic/subdocs}{\textsf{subdocs}}
for further related packages.
The present package differs from the above solutions in that
a document structure constructed with the conventional |\include| mechanism
just needs two extra commands at the top of every file
such that all constituent files can be compiled individually.

%%%%%%%%%%%%%%%%%%%%%%%%%%%%%%%%%%%%%%%%%%%%%%%%%%%%%%%%%%%%%%%%%%%%%%%%%%%%%%%%
%\subsection{Feature Suggestions}
%
%The following is a list of features which may be useful for future
%versions of this package:
%%
%\begin{itemize}
%\item
%\ldots
%\end{itemize}

%%%%%%%%%%%%%%%%%%%%%%%%%%%%%%%%%%%%%%%%%%%%%%%%%%%%%%%%%%%%%%%%%%%%%%%%%%%%%%%%
\subsection{Revision History}

%%%%%%%%%%%%%%%%%%%%%%%%%%%%%%%%%%%%%%%%
\paragraph{v2.0:} 2018/12/30

\begin{itemize}
\item
immediate forward processing
\item
added |\childdocby| mechanism
\item
manual restructured
\end{itemize}

%%%%%%%%%%%%%%%%%%%%%%%%%%%%%%%%%%%%%%%%
\paragraph{v1.6:} 2018/01/17

\begin{itemize}
\item
application for development of include files
\item
corrections to manual
\end{itemize}

%%%%%%%%%%%%%%%%%%%%%%%%%%%%%%%%%%%%%%%%
\paragraph{v1.5:} 2017/05/21

\begin{itemize}
\item
more complete structuring introduced
\item
|\childdocof| introduced
\item
|\childdoc| renamed to |\childdocmain|
\item
|\childredirect| renamed to |\childdocforward| and |\childdocforwardprefix|
and functionality expanded
\end{itemize}

%%%%%%%%%%%%%%%%%%%%%%%%%%%%%%%%%%%%%%%%
\paragraph{v1.0:} 2017/04/27

\begin{itemize}
\item
manual and install package
\item
first version published on CTAN
\end{itemize}

%%%%%%%%%%%%%%%%%%%%%%%%%%%%%%%%%%%%%%%%
\paragraph{v0.6:} 2017/04/26

\begin{itemize}
\item
redirection mechanism added
\end{itemize}

%%%%%%%%%%%%%%%%%%%%%%%%%%%%%%%%%%%%%%%%
\paragraph{v0.5:} 2017/04/26

\begin{itemize}
\item
functionality in definition file
\end{itemize}


%%%%%%%%%%%%%%%%%%%%%%%%%%%%%%%%%%%%%%%%%%%%%%%%%%%%%%%%%%%%%%%%%%%%%%%%%%%%%%%%
%%%%%%%%%%%%%%%%%%%%%%%%%%%%%%%%%%%%%%%%%%%%%%%%%%%%%%%%%%%%%%%%%%%%%%%%%%%%%%%%
%%%%%%%%%%%%%%%%%%%%%%%%%%%%%%%%%%%%%%%%%%%%%%%%%%%%%%%%%%%%%%%%%%%%%%%%%%%%%%%%
\appendix

\settowidth\MacroIndent{\rmfamily\scriptsize 000\ }

 \DocInput{childdoc.dtx}

\end{document}
%</driver>
% \fi
%
% %%%%%%%%%%%%%%%%%%%%%%%%%%%%%%%%%%%%%%%%%%%%%%%%%%%%%%%%%%%%%%%%%%%%%%%%%%%%%%
% %%%%%%%%%%%%%%%%%%%%%%%%%%%%%%%%%%%%%%%%%%%%%%%%%%%%%%%%%%%%%%%%%%%%%%%%%%%%%%
% \section{Sample}
%\iffalse
%<*samplemain>
%\fi
%
% The following presents a sample document
% with two chapters, two parts, a title page,
% a compile flag as well as three forwarding files to set the flag.
% It consists of eight |.tex| files:
% \begin{center}
% \begin{tabular}{ll}
% |cdocsamp.tex|&main file\\
% |cdocsch1.tex|&include file for chapter 1\\
% |cdocsch2.tex|&include file for chapter 2\\
% |cdocspt3.tex|&include file for part 3\\
% |cdocspt4.tex|&include file for part 4\\
% |cdocsdrf.tex|&forwarding file for main file in draft mode\\
% |cdocsfi1.tex|&forwarding file for final version of chapter 1\\
% |cdocsfi2.tex|&forwarding file for final version of chapter 2\\
% \end{tabular}
% \end{center}
% Each of the eight files can be compiled directly by the \LaTeX{} compiler.
%
% %%%%%%%%%%%%%%%%%%%%%%%%%%%%%%%%%%%%%%
% \paragraph{Main File.}
%
% The main file is called |cdocsamp.tex|.
%
% Load the \textsf{childdoc} definitions and
% declare the filename for the main document:
%    \begin{macrocode}
\input{childdoc.def}
\childdocmain{}
%    \end{macrocode}

% Optional override for |\version| flag:
%    \begin{macrocode}
%%\ifchilddoc\else\providecommand{\version}{draft}\fi
%    \end{macrocode}

% Define the default values for the |\version| flag
% (|final| for the main file and |draft| for childs):
%    \begin{macrocode}
\ifchilddoc
\providecommand{\version}{draft}
\else
\providecommand{\version}{final}
\fi
%    \end{macrocode}

% Load the standard document class:
%    \begin{macrocode}
\documentclass[12pt]{article}
%    \end{macrocode}

% Start the document body:
%    \begin{macrocode}
\begin{document}
%    \end{macrocode}

% Declare a title page.
% Print title, part of document being processed and version flag:
%    \begin{macrocode}
\addtocounter{page}{-1}
\begin{center}
{\LARGE\bfseries{}childdoc example\par}
\vspace{1cm}
\ifchilddoc
\ifchilddocmanual part\else chapter\fi:
`\childdocname' of `\childdocjob'\par
\else
main document: `\childdocjob'\par
\fi
version: \version\par
\end{center}
\newpage
%    \end{macrocode}

% Manually include selected file,
% otherwise process as usual:
%    \begin{macrocode}
\ifchilddocmanual
\section*{part `\childdocname'}
\input{\childdocname}
\else
%    \end{macrocode}

% Include the two chapters:
%    \begin{macrocode}
\include{cdocsch1}
\include{cdocsch2}
%    \end{macrocode}

% Include the two parts unless only chapters should be displayed:
%    \begin{macrocode}
\ifchilddoc\else
\section{part three}
\input{cdocspt3}
\section{part four}
\input{cdocspt4}
\fi
%    \end{macrocode}

% Process as usual until here:
%    \begin{macrocode}
\fi
%    \end{macrocode}

% End of document body:
%    \begin{macrocode}
\end{document}
%    \end{macrocode}
%\iffalse
%</samplemain>
%\fi
%
% %%%%%%%%%%%%%%%%%%%%%%%%%%%%%%%%%%%%%%
% \paragraph{Chapter Include Files.}
%
% The include files are called |cdocsch1.tex| and |cdocsch2.tex|.
%
%\iffalse
%<*samplechap1|samplechap2>
%\fi

% Optional override for |\version| flag:
%    \begin{macrocode}
%%\providecommand{\version}{final}
%    \end{macrocode}

% Include the main document:
%    \begin{macrocode}
\input{childdoc.def}
\childdocof{cdocsamp}
%    \end{macrocode}

%\iffalse
%</samplechap1|samplechap2>
%\fi
%
%\iffalse
%<*samplechap1>
%\fi
% Some text for chapter 1:
%    \begin{macrocode}
\section{one}
some text in chapter one
%    \end{macrocode}

%\iffalse
%</samplechap1>
%\fi
% Some text for chapter 2:
%\iffalse
%<*samplechap2>
%\fi
%    \begin{macrocode}
\section{two}
more text in chapter two
%    \end{macrocode}

%\iffalse
%</samplechap2>
%\fi
%
% %%%%%%%%%%%%%%%%%%%%%%%%%%%%%%%%%%%%%%
% \paragraph{Part Include Files.}
%
% The include files are called |cdocspt3.tex| and |cdocspt4.tex|.
%
%\iffalse
%<*samplepart3|samplepart4>
%\fi

% Optional override for |\version| flag:
%    \begin{macrocode}
%%\providecommand{\version}{final}
%    \end{macrocode}

% Include the main document:
%    \begin{macrocode}
\input{childdoc.def}
\childdocby{cdocsamp}
%    \end{macrocode}

%\iffalse
%</samplepart3|samplepart4>
%\fi
%
%\iffalse
%<*samplepart3>
%\fi
% Some text for part 3:
%    \begin{macrocode}
some text in part three
%    \end{macrocode}

%\iffalse
%</samplepart3>
%\fi
% Some text for part 4:
%\iffalse
%<*samplepart4>
%\fi
%    \begin{macrocode}
more text in part four
%    \end{macrocode}

%\iffalse
%</samplepart4>
%\fi
%
% %%%%%%%%%%%%%%%%%%%%%%%%%%%%%%%%%%%%%%
% \paragraph{Forwarding for a Complete Draft.}
%
% The following forwarding file |cdocsdrf.tex|
% compiles the main document in draft mode:
%\iffalse
%<*sampledraft>
%\fi
%    \begin{macrocode}
\def\version{draft}
\input{childdoc.def}
\childdocforward{cdocsamp}
%    \end{macrocode}

%\iffalse
%</sampledraft>
%\fi
%
% %%%%%%%%%%%%%%%%%%%%%%%%%%%%%%%%%%%%%%
% \paragraph{Forwarding for Final Version of the Chapters.}
%
% The following forwarding files |cdocsfn1.tex| and |cdocsfn2.tex|
% (with identical content)
% compile the final versions of the child documents
% |cdocsch1.tex| and |cdocsch2.tex|, respectively:
%\iffalse
%<*samplefinal>
%\fi
%    \begin{macrocode}
\def\version{final}
\input{childdoc.def}
\childdocforwardprefix[cdocsamp]{cdocsfn}{cdocsch}
%    \end{macrocode}

%\iffalse
%</samplefinal>
%\fi
%
% %%%%%%%%%%%%%%%%%%%%%%%%%%%%%%%%%%%%%%
% \paragraph{Command Line Processing.}
%
% The following three command lines generate the output files
% |cdocscld|, |cdocscl1| and |cdocscl2|
% which should be identical to
% |cdocsdrf|, |cdocsch1| and |cdocsfn2|, respectively:
% \begin{center}
% \begin{tabular}{l}
% |latex -jobname cdocscld \|\\
% |  "\def\version{draft}\input{childdoc.def}\childdocforward{cdocsamp}"|\\
% |latex -jobname cdocscl1 \|\\
% |  "\input{childdoc.def}\childdocforward[cdocsamp]{cdocsch1}"|\\
% |latex -jobname cdocscl2 \|\\
% |  "\def\version{final}\input{childdoc.def}\childdocforward{cdocsch2}"|
% \end{tabular}
% \end{center}
% Note that the trailing backslash on each first line
% merely continues the input to the second line
% (for convenient cut ant paste).
% Furthermore, the command |latex| can be replaced by any
% of its alternative versions such as |pdflatex|.
%
% %%%%%%%%%%%%%%%%%%%%%%%%%%%%%%%%%%%%%%%%%%%%%%%%%%%%%%%%%%%%%%%%%%%%%%%%%%%%%%
% %%%%%%%%%%%%%%%%%%%%%%%%%%%%%%%%%%%%%%%%%%%%%%%%%%%%%%%%%%%%%%%%%%%%%%%%%%%%%%
% \section{Implementation}
%\iffalse
%<*package>
%\fi
%
% This section describes the definitions file |childdoc.def|.

% The definitions cannot be loaded using |\usepackage| or |\RequirePackage|
% which has a mechanism to prevent loading a style file more than once.
% When loading the definitions by means of |\input|
% multiple instances have to be prevented manually:
%\iffalse
%This code needs to be before the `\ProvidesFile' directive
%which is defined at the beginning of this file.
%Therefore it is also placed there and commented out here.
%</package>
%<*discard>
%\fi
%    \begin{macrocode}
\ifdefined\childdocmain\endinput\fi
%    \end{macrocode}
%\iffalse
%</discard>
%<*package>
%\fi
%
% \macro{\ifchilddoc}
% \macro{\ifchilddocmanual}
% The conditional |\ifchilddoc| tells whether a
% child (true) or main (false) document is being compiled.
% The conditional |\ifchilddocmanual| tells whether
% the |\includeonly| mechanism is used (false) or
% the selection of child files must be performed manually (true).
% The definitions initialise to false:
%    \begin{macrocode}
\newif\ifchilddoc
\newif\ifchilddocmanual
%    \end{macrocode}

% \macro{\childdocname}
% \macro{\childdocjob}
% The macro |\childdocname| stores the name of the main document
% to be compiled. The macro |\childdocjob| stores the name of
% the document on which the \LaTeX{} compiler was originally invoked.
% The content of |\jobname| cannot be compared
% to filenames specified in the source due to different catcodes.
% The following code rescans |\jobname|, stores the result
% in |\childdocname| and saves a copy in |\childdocjob|:
%    \begin{macrocode}
\edef\childdocname{\scantokens\expandafter{\jobname\noexpand}}
\let\childdocjob\childdocname
%    \end{macrocode}

% \macro{\childdocdisable}
% The macro |\childdocdisable| prevents the main file
% from being processed more than once.
% At this stage, the main document command |\childdocmain|
% is assumed to be called once again where it should do nothing.
% Any subsequent call to it should prevent
% a secondary processing of the main document
% It overwrites the forwarding commands
% |\childdocof| and |\childdocforward|
% with empty macros to prevent further inclusions of the main document:
%    \begin{macrocode}
\newcommand{\childdocdisable}
{
  \renewcommand{\childdocmain}[1]{\renewcommand{\childdocmain}[1]{\endinput}}
  \renewcommand{\childdocof}[1]{}
  \renewcommand{\childdocby}[2][]{}
  \renewcommand{\childdocforward}[2][]{}
  \renewcommand{\childdocdisable}{}
}
%    \end{macrocode}

% \macro{\childdocmain}
% The macro |\childdocmain| is to be called at the top of the main file
% with nothing or the main filename (without extension) as argument.
% First, it breaks loops.
% If the argument is not empty and does not match |\childdocname|
% (which is set by the first inclusion of |childdoc.def|),
% |\ifchilddoc| is set to true, |\includeonly| is applied to the child file
% and |\jobname| is set to the main file
% (for proper handling of |.aux| files):
%    \begin{macrocode}
\newcommand{\childdocmain}[1]
{
  \childdocdisable\childdocmain{}
  \if?#1?\else
    \begingroup
      \def\childdoctmp{#1}
      \ifx\childdoctmp\childdocname
        \def\childdoctmp{}
      \else
        \def\childdoctmp
        {
          \childdoctrue
          \includeonly{\childdocname}
          \def\childdocjob{#1}
          \def\jobname{#1}
        }
      \fi
      \expandafter
    \endgroup
    \childdoctmp
  \fi
}
%    \end{macrocode}

% \macro{\childdocof}
% The command |\childdocof| redirects
% compilation to the main file |#1|.
%    \begin{macrocode}
\newcommand{\childdocof}[1]
{
  \childdocdisable
  \childdoctrue
  \includeonly{\childdocname}
  \def\jobname{#1}
  \def\childdocjob{#1}
  \input{#1}
}
%    \end{macrocode}

% \macro{\childdocby}
% The command |\childdocby| ....
%    \begin{macrocode}
\newcommand{\childdocby}[2][]
{
  \childdocdisable
  \childdoctrue
  \childdocmanualtrue
  \if?#1?\else
    \def\jobname{#2}
  \fi
  \def\childdocjob{#2}
  \input{#2}
  \endinput
}
%    \end{macrocode}

% \macro{\childdocforward}
% The command |\childdocforward| redirects
% compilation to the main file or
% (if the optional argument is given) a child file.
% Parameters are set as if the main file
% or a child file starting with |\childdocof| was compiled.
% Then compilation is handed over to the main file:
%    \begin{macrocode}
\newcommand{\childdocforward}[2][]
{
  \begingroup
    \if?#1?
      \def\childdoctmp
      {
        \def\childdocname{#2}
        \def\childdocjob{#2}
        \def\jobname{#2}
        \input{#2}
        \endinput
      }
    \else
      \def\childdoctmp
      {
        \childdocdisable
        \def\childdocname{#2}
        \childdoctrue
        \includeonly{#2}
        \def\childdocjob{#1}
        \def\jobname{#1}
        \input{#1}
        \endinput
      }
    \fi
    \expandafter
  \endgroup
  \childdoctmp
}
%    \end{macrocode}

% \macro{\childdocforwardprefix}
% The command |\childdocforwardprefix| redirects
% compilation to the main or a child file by means of a pattern.
% The prefix |#1| in the current filename is replaced by |#2|
% and the suffix of the current filename is kept
% (it is assumed that the filename does not contain the substring `|~~~|'
% which is used as a delimiter).
% Compilation is handed over to the new file by |\childdocforward|:
%    \begin{macrocode}
\newcommand{\childdocforwardprefix}[3][]
{
  \begingroup
    \def\childdocextract #2##1~~~{\def\childdoctmp{\childdocforward[#1]{#3##1}}}
    \expandafter\childdocextract\childdocname~~~
    \expandafter
  \endgroup
  \childdoctmp
}
%    \end{macrocode}

% \macro{\childdoc}
% The deprecated macro |\childdoc| is a legacy version of |\childdocmain|:
%    \begin{macrocode}
\newcommand{\childdoc}{\childdocmain}
%    \end{macrocode}

% \macro{\childdocredirect}
% The deprecated macro |\childdocredirect| is a legacy version
% of |\childdocforward| and |\childdocforwardprefix|:
%    \begin{macrocode}
\newcommand{\childdocredirect}[2][]
{
  \begingroup
    \if?#1?
      \def\childdoctmp{\childdocforward{#2}}
    \else
      \def\childdoctmp{\childdocforwardprefix{#1}{#2}}
    \fi
    \expandafter
  \endgroup
  \childdoctmp
}
%    \end{macrocode}

%\iffalse
%</package>
%\fi
%
\endinput
\childdocforward[cdocsamp]{cdocsch1}"|\\
% |latex -jobname cdocscl2 \|\\
% |  "\def\version{final}% \iffalse
%
% childdoc.dtx Copyright (C) 2017-2018 Niklas Beisert
%
% This work may be distributed and/or modified under the
% conditions of the LaTeX Project Public License, either version 1.3
% of this license or (at your option) any later version.
% The latest version of this license is in
%   http://www.latex-project.org/lppl.txt
% and version 1.3 or later is part of all distributions of LaTeX
% version 2005/12/01 or later.
%
% This work has the LPPL maintenance status `maintained'.
%
% The Current Maintainer of this work is Niklas Beisert.
%
% This work consists of the files childdoc.dtx and childdoc.ins
% and the derived files childdoc.def and cdocsamp.tex with
% cdocsch1.tex, cdocsch2.tex, cdocsdrf.tex, cdocsfn1.tex, cdocsfn2.tex.
%
%<package>\ifdefined\childdocmain\endinput\fi
%<package>\ProvidesFile{childdoc.def}[2018/12/30 v2.0 child document driver]
%<samplemain>\ProvidesFile{cdocsamp.tex}[2018/12/30 v2.0 sample for childdoc]
%<*driver>
%\ProvidesFile{childdoc.drv}[2018/12/30 v2.0 childdoc reference manual file]
\PassOptionsToClass{10pt,a4paper}{article}
\documentclass{ltxdoc}

\usepackage[margin=35mm]{geometry}
\usepackage{hyperref}
\usepackage{hyperxmp}
\usepackage[usenames]{color}

\hypersetup{colorlinks=true}
\hypersetup{pdfstartview=FitH}
\hypersetup{pdfpagemode=UseNone}
\hypersetup{pdfsource={}}
\hypersetup{pdflang={en-UK}}
\hypersetup{pdfcopyright={Copyright 2017-2018 Niklas Beisert.
  This work may be distributed and/or modified under the
  conditions of the LaTeX Project Public License, either version 1.3
  of this license or (at your option) any later version.}}
\hypersetup{pdflicenseurl={http://www.latex-project.org/lppl.txt}}
\hypersetup{pdfcontactaddress={ETH Zurich, ITP, HIT K,
  Wolfgang-Pauli-Strasse 27}}
\hypersetup{pdfcontactpostcode={8093}}
\hypersetup{pdfcontactcity={Zurich}}
\hypersetup{pdfcontactcountry={Switzerland}}
\hypersetup{pdfcontactemail={nbeisert@itp.phys.ethz.ch}}
\hypersetup{pdfcontacturl={http://people.phys.ethz.ch/\xmptilde nbeisert/}}

\newcommand{\secref}[1]{\hyperref[#1]{section \ref*{#1}}}

\parskip1ex
\parindent0pt
\let\olditemize\itemize
\def\itemize{\olditemize\parskip0pt}

\begin{document}

\title{The \textsf{childdoc} Package}
\hypersetup{pdftitle={The childdoc Package}}
\author{Niklas Beisert\\[2ex]
  Institut f\"ur Theoretische Physik\\
  Eidgen\"ossische Technische Hochschule Z\"urich\\
  Wolfgang-Pauli-Strasse 27, 8093 Z\"urich, Switzerland\\[1ex]
  \href{mailto:nbeisert@itp.phys.ethz.ch}
  {\texttt{nbeisert@itp.phys.ethz.ch}}}
\hypersetup{pdfauthor={Niklas Beisert}}
\hypersetup{pdfsubject={Manual for the LaTeX2e Package childdoc}}
\date{30 December 2018, \textsf{v2.0}}
\maketitle

\begin{abstract}\noindent
\textsf{childdoc} is a \LaTeXe{} package
that enables the direct compilation
of document sections included by |\include|
to individual files.
\end{abstract}

\begingroup
\parskip0ex
\tableofcontents
\endgroup

%%%%%%%%%%%%%%%%%%%%%%%%%%%%%%%%%%%%%%%%%%%%%%%%%%%%%%%%%%%%%%%%%%%%%%%%%%%%%%%%
%%%%%%%%%%%%%%%%%%%%%%%%%%%%%%%%%%%%%%%%%%%%%%%%%%%%%%%%%%%%%%%%%%%%%%%%%%%%%%%%
\section{Introduction}

\LaTeX{} provides a mechanism to structure a large document (such as a book)
into a main file and several child files (containing the chapters)
using the |\include| command.
This mechanism is beneficial for documents
which span hundreds of pages in order to
make the source file(s) more manageable.
Moreover, compilation can be restricted to
selected child files by means of the |\includeonly| command.
The latter feature can be used to reduce the compilation time while editing
(this was significantly more useful in the earlier days of \LaTeX{})
or to generate a smaller document which is easier to navigate.
Another application of |\includeonly| is to generate
documents consisting of selected parts of the complete document.

However, there are a few drawbacks of the plain |\include| mechanism:
\begin{itemize}
\item
The child files cannot be compiled on their own,
they can only be compiled via the main file.
A naive editing environment
(such as a text editor with an option
to have the current file processed by \LaTeX)
may require one to switch to the main file before compiling;
attempting to compile the child file produces errors.
\item
The main file must be modified (each time)
to adjust the |\includeonly| command
to the present needs. This easily leaves the main file in a messy state.
\item
The generated document will always carry the filename
of the main document. This is inconvenient if
several child files are to be compiled and
to be kept for distribution.
\end{itemize}

The present package provides a simple interface
to make child files individually compilable by \LaTeX{}.
Compiling a child file then has the same effect as compiling
the main file with an |\includeonly| command
to select the appropriate child.
Moreover the generated document will carry the name of the child
rather than the main file.
This resolves all three above issues.

This feature is meant to make the editing of books,
thesis documents and lecture notes somewhat more convenient.
However, the package can also be used efficiently for
composing a series of documents (such as exercise sheets)
which are typically distributed individually.
It then assists the author in generating the individual documents
(potentially in different versions)
as well as a document containing the collected series.
Another application is in developing style files
or other kinds of included material
where compilation of the style file could redirect
to a sample or test file.

%%%%%%%%%%%%%%%%%%%%%%%%%%%%%%%%%%%%%%%%%%%%%%%%%%%%%%%%%%%%%%%%%%%%%%%%%%%%%%%%
%%%%%%%%%%%%%%%%%%%%%%%%%%%%%%%%%%%%%%%%%%%%%%%%%%%%%%%%%%%%%%%%%%%%%%%%%%%%%%%%
\section{Usage}

First of all, the package \textsf{childdoc} is \emph{not} a standard
\LaTeXe{} |.sty| style file! Therefore it needs to be invoked in
a non-standard way.

%%%%%%%%%%%%%%%%%%%%%%%%%%%%%%%%%%%%%%%%%%%%%%%%%%%%%%%%%%%%%%%%%%%%%%%%%%%%%%%%
\subsection{Included Files}
\label{sec:include}

%%%%%%%%%%%%%%%%%%%%%%%%%%%%%%%%%%%%%%%%
\DescribeMacro{\childdocmain}
To use the package, add the commands
\begin{center}
\begin{tabular}{l}
|\input{childdoc.def}|\\
|\childdocmain{}|\\
\end{tabular}
\end{center}
at the very top of the main \LaTeX{} file,
in particular \emph{before} the |\documentclass| statement!
The argument of |\childdocmain| should be left empty
(but it must be present).

%%%%%%%%%%%%%%%%%%%%%%%%%%%%%%%%%%%%%%%%
\DescribeMacro{\childdocof}
Furthermore, add the commands
\begin{center}
\begin{tabular}{l}
|\input{childdoc.def}|\\
|\childdocof{|\textit{main}|}|\\
\end{tabular}
\end{center}
at the top of every child file \textit{child}
which is included by |\include{|\textit{child}|}|
from within the main file
(or at least for those files to be compiled individually).
The argument \textit{main} must be the filename of the main file.

There are a couple of
considerations in setting up the main and child documents:

%%%%%%%%%%%%%%%%%%%%%%%%%%%%%%%%%%%%%%%%
\paragraph{Restrictions.}

Please note the following restrictions:
\begin{itemize}
\item
|\childdocmain| must be called with one argument \textit{main}
to ensure compatibility with earlier version of the package.
It must either be empty (|\childdocmain{}|)
or precisely match the filename of the main file in which it is specified.
See \secref{sec:detection} for further information.
\item
The filename \textit{main} must be specified without the |.tex| extension.
\item
The filename \textit{main} is case sensitive
(even in case-insensitive file systems)
due to internal string comparison.
\item
The argument \textit{main} should be fully expanded, it cannot be a macro.
\item
Subdirectories and special characters should be avoided in filenames.
\item
The command |\childdocmain{|\textit{main}|}| must be followed by a whitespace.
It should not be followed immediately by another command
or by a comment mark `|%|'.
This is because the \TeX{} parser reads the token immediately following
the argument of |\childdocmain| and puts it
at the beginning of every child section;
however, a white\-space is ignored.
\end{itemize}

%%%%%%%%%%%%%%%%%%%%%%%%%%%%%%%%%%%%%%%%
\paragraph{Content of Main File.}

It is advisable to place all content in the child files included by |\include|.
Any output contained in the main file will appear in all child documents
unless suppressed manually;
it cannot be suppressed automatically by the |\includeonly| directive
and thus should normally be avoided.
A method to include some content in the main file
by means of conditional processing is described in \secref{sec:conditional}.

%%%%%%%%%%%%%%%%%%%%%%%%%%%%%%%%%%%%%%%%
\paragraph{Page Numbering.}

When only a part of the document is compiled,
the appropriate numbering of pages
(as well as other status parameters)
is determined from the |.aux| files.
The latter contain information from previous passes.
However this information needs to propagate through
all intermediate child documents.
Therefore the page numbering in child documents may well
be inconsistent until the complete document is compiled at least once.

A useful (if unconventional) way to always ensure a consistent
page numbering is to restart the numbering in each child document
and denote the pages by `\textit{child}|.|\textit{page}'
where \textit{child} represents the chapter/section number of the child file.
This can be achieved by the command
|\numberwithin{page}{|\textit{child}|}|
of the \textsf{amsmath} package
where \textit{child} can be |chapter| or |section|
depending on the chosen structuring.
Alternatively, one can modify the macro |\thepage| appropriately
and reset the counter |page| at the start of each child file.

%%%%%%%%%%%%%%%%%%%%%%%%%%%%%%%%%%%%%%%%%%%%%%%%%%%%%%%%%%%%%%%%%%%%%%%%%%%%%%%%
\subsection{Conditional Processing}
\label{sec:conditional}

The package provides a mechanism to compile different versions
of a document. To customise the versions further some conditional processing
can come in handy to distinguish which version is being compiled.
The package provides two macros to describe the compilation context:

%%%%%%%%%%%%%%%%%%%%%%%%%%%%%%%%%%%%%%%%
\DescribeMacro{\ifchilddoc}
The conditional |\ifchilddoc| distinguishes between the compilation of
child documents and the main document:
%
\begin{center}
|\ifchilddoc |\textit{child-code}| |[|\||else |\textit{main-code}]| \||fi|
\end{center}

%%%%%%%%%%%%%%%%%%%%%%%%%%%%%%%%%%%%%%%%
\DescribeMacro{\childdocname}
\DescribeMacro{\childdocjob}
The macro |\childdocname| contains the filename (without extension)
of the main or child file being processed.
Note that |\childdocjob| will always contain the name of the main file.

%%%%%%%%%%%%%%%%%%%%%%%%%%%%%%%%%%%%%%%%
\paragraph{Title Page.}

Conditional processing can be used to include a title or banner page
in the main document when proper precautions are taken.
Importantly, the code in the main file should ensure that the page counter
(as well as other status parameters which are stored in the |.aux| files)
takes the same value after the conditional processing.
Otherwise the page numbers may take divergent values
depending on which part is compiled.

For example, a title page could be declared by:
%
\begin{center}
\begin{tabular}{l}
|\ifchilddoc\||else|\\
|\addtocounter{page}{-1}|\\
\textit{code for title page}\\
|\newpage|\\
|\||fi|
\end{tabular}
\end{center}
%
A banner page for the child documents can be generated by:
%
\begin{center}
\begin{tabular}{l}
|\ifchilddoc|\\
|\addtocounter{page}{-1}|\\
\textit{code for banner page}\\
|\newpage|\\
|\||fi|
\end{tabular}
\end{center}
%
Here one could write a message such as:
\begin{center}
|This is the part \childdocname{} of \childdocjob{}.|
\end{center}

%%%%%%%%%%%%%%%%%%%%%%%%%%%%%%%%%%%%%%%%%%%%%%%%%%%%%%%%%%%%%%%%%%%%%%%%%%%%%%%%
\subsection{Flags}
\label{sec:flags}

The package makes it easy to generate different versions
of the main or child documents.
To this end compilation flags can be defined
and assigned different default values.
They will be particularly useful in conjunction
with the forwarding mechanism described in \secref{sec:forward}.

For example, it may be useful to have a flag |\version|
which can be set to |draft| or |final|.
The document source will contain some conditional code
depending on the value of |\version|.
Suppose further, the flag should default to |final| for the main file
and to |draft| for child files
which is a natural assignment for editing the document.
This is achieved by placing the following code
in the preamble of the main document
(below the |\childdocmain| directive):
%
\begin{center}
\begin{tabular}{l}
|\ifchilddoc|\\
|\providecommand{\version}{draft}|\\
|\||else|\\
|\providecommand{\version}{final}|\\
|\||fi|
\end{tabular}
\end{center}
%
The definition by |\providecommand| makes sure
that previous definitions are not overwritten.
Further statements |\providecommand{\version}{...}|
can thus be added before the above code to override it.

For the main file, one might add a line
(between |\childdocmain| and the above block)
%
\begin{center}
|%\ifchilddoc\||else\providecommand{\version}{draft}\||fi|
\end{center}
%
which can be uncommented to produce a draft version.
Likewise one can add a line to the very top of a child file
(above the |\childdocof{|\textit{main}|}| directive)
%
\begin{center}
|%\providecommand{\version}{final}|
\end{center}
%
which can be uncommented to produce the final version of this child document.

%%%%%%%%%%%%%%%%%%%%%%%%%%%%%%%%%%%%%%%%%%%%%%%%%%%%%%%%%%%%%%%%%%%%%%%%%%%%%%%%
\subsection{Forwarding}
\label{sec:forward}

Different versions of the main or child documents
using compilation flags as described in \secref{sec:flags}
can be (permanently) stored in different files
for convenient compilation, viewing and distribution.
To this end, the package defines a command
to pass on compilation to a different file:

%%%%%%%%%%%%%%%%%%%%%%%%%%%%%%%%%%%%%%%%
\DescribeMacro{\childdocforward}
The command |\childdocforward| redirects processing to
another source file:
%
\begin{center}
\begin{tabular}{l}
|\input{childdoc.def}|\\
|\childdocforward[|\textit{main}|]{|\textit{dest}|}|\\
\end{tabular}
\end{center}
%
The argument \textit{dest} is the destination file
(without extension).
It should be the main file or one of the child files.
Note that further \textsf{childdoc} directives
such as |\childdocof| and |\childdocforward|
in the indicated file will be processed in this form.
The optional argument \textit{main}
passes on directly to the main file \textit{main}
while pretending to compile the child \textit{dest}.
This form behaves as if \textit{dest}
issues |\childdocof{|\textit{main}|}| right away,
and no further \textsf{childdoc} directives will be processed.

%%%%%%%%%%%%%%%%%%%%%%%%%%%%%%%%%%%%%%%%
\DescribeMacro{\...prefix}
In the alternative form |\childdocforwardprefix|,
%
\begin{center}
\begin{tabular}{l}
|\input{childdoc.def}|\\
|\childdocforwardprefix[|\textit{main}|]{|\textit{prefix}|}{|\textit{dest}|}|
\end{tabular}
\end{center}
%
the destination file is determined by a pattern
depending on the current file:
To make this work, the current file must be called
`{\textit{prefix}\hspace{0.2em}\textit{suffix}}'
with \textit{prefix} matching precisely the argument.
Processing is then passed on to the file
`{\textit{dest}\hspace{0.2em}\textit{suffix}}'.
Surely, the same effect is achieved by
directly specifying the
argument `{\textit{dest}\hspace{0.2em}\textit{suffix}}'
in the first form.
However, that requires to set up a different file
for each child. With the alternative form of the command
all these files can have exactly the same content
which simplifies setting them up and maintaining them.

For example, the following file |draft.tex|
with a compilation flag |\version| as described in \secref{sec:flags}
compiles the main document as a draft:
%
\begin{center}
\begin{tabular}{l}
|\def\version{draft}|\\
|\input{childdoc.def}|\\
|\childdocforward{|\textit{main}|}|
\end{tabular}
\end{center}
%
Likewise, the following files |final|\textit{nn}|.tex|
compile the final version of the child document
|child|\textit{nn}|.tex|:
%
\begin{center}
\begin{tabular}{l}
|\def\version{final}|\\
|\input{childdoc.def}|\\
|\childdocforwardprefix{final}{child}|
\end{tabular}
\end{center}
%

Note that when several versions of a main file and/or of each child file
are to be generated, it may be convenient to set up a |Makefile| or
shell script to automatise the process.

%%%%%%%%%%%%%%%%%%%%%%%%%%%%%%%%%%%%%%%%%%%%%%%%%%%%%%%%%%%%%%%%%%%%%%%%%%%%%%%%
\subsection{Command Line Processing}
\label{sec:commandline}

The effect of redirection files can also be achieved by invoking
the \LaTeX{} compiler with a more elaborate command line.
Most conveniently this should be done as part
of a shell script or a |Makefile|.

When using \textsf{childdoc} in the main file, the following
command lines effectively perform a redirection
(note that depending on the shell being used,
backslashes may have to be doubled: `|\|' $\to$ `|\\|'):
%
\begin{center}
|... -jobname "|\textit{target}|" |\\|"|[\textit{flags}]%
|\input{childdoc.def}\childdocforward[|\textit{main}|]{|\textit{dest}|}"|
\end{center}
%
Here \textit{target} is the name of the output file,
\textit{main} is the name of the main file
and \textit{dest} is the name of the main or child file to be processed
(all filenames without extensions).
The optional argument \textit{main} can be omitted
if \textit{main} matches \textit{dest}.
Optionally, compilation \textit{flags} can be defined via |\def| commands.
This command line makes the \TeX{} engine believe
it is compiling the file \textit{target}
whose content is specified as the latter parameter.
The provided code then forwards the processing to
\textit{main} or \textit{dest} as described in \secref{sec:forward}.

%%%%%%%%%%%%%%%%%%%%%%%%%%%%%%%%%%%%%%%%%%%%%%%%%%%%%%%%%%%%%%%%%%%%%%%%%%%%%%%%
\subsection{Include by Input}
\label{sec:input}

Including child documents by |\include| has some restrictions by design.
Most notably, the content of a child document always occupies
its own set of pages; pages cannot be shared between child documents.
Usually, this behaviour makes perfect sense
because each child document contain an essential part of the document.
However, in some situations it may be desirable to compose
a document from a collection of parts
without having mandatory page breaks between then.
For this case, the package
provides a mechanism to include parts
by |\input| which can also be processed individually.
However, by construction this mechanism
requires manual handling of the content to be output.

%%%%%%%%%%%%%%%%%%%%%%%%%%%%%%%%%%%%%%%%
\DescribeMacro{\ifchilddocmanual}
The main file should be prepared as usual, see \secref{sec:include}.
However, the document body must make a distinction
between processing of an individual part and of the main document, e.g.:
%
\begin{center}
\begin{tabular}{l}
|\ifchilddocmanual|\\
|\input{\childdocname}|\\
|\||else|\\
\textit{document body with }|\input{|\textit{part}|}|\\
|\||fi|
\end{tabular}
\end{center}
%
The conditional |\ifchilddocmanual| is true whenever
a part to be included by |\input| is being compiled,
and the name of the part is stored in |\childdocname|.

%%%%%%%%%%%%%%%%%%%%%%%%%%%%%%%%%%%%%%%%
\DescribeMacro{\childdocby}
Each part to be included by |\input| should start with:
%
\begin{center}
\begin{tabular}{l}
|\input{childdoc.def}|\\
|\childdocby{|\textit{main}|}|\\
\end{tabular}
\end{center}
%
The directive |\childdocby| is similar to |\childdocof|
described in \secref{sec:include},
but the subsequent selection of content must be done manually.
To that end, both |\ifchilddoc| and |\ifchilddocmanual|
will be true upon processing of a part,
and the name of the part is stored in |\childdocname|.
Note that |\jobname| will be set to the filename of the current part
so that each part receives an individual |.aux| file
that does not interfere with the |.aux| file(s) of the main document.
This behaviour can be altered by the alternative form
|\childdocby[*]{|\textit{main}|}| (with a non-empty optional argument)
which uses the |.aux| file of the main document
by setting |\jobname| to \textit{main}.

%%%%%%%%%%%%%%%%%%%%%%%%%%%%%%%%%%%%%%%%%%%%%%%%%%%%%%%%%%%%%%%%%%%%%%%%%%%%%%%%
\subsection{Driver Development}
\label{sec:driver}

The \textsf{childdoc} mechanism can also be use for the development
of definition files such as \LaTeX{} styles or classes.
This case differs from the above setup with multiple parts
included by |\include| in that no |\includeonly| should be invoked.
This can be achieved by starting the include file
(before |\ProvidesPackage|) with:
%
\begin{center}
\begin{tabular}{l}
|\input{childdoc.def}|\\
|\childdocforward{|\textit{main}|}|\\
\end{tabular}
\end{center}
%
or alternatively with:
%
\begin{center}
\begin{tabular}{l}
|\input{childdoc.def}|\\
|\childdocby{|\textit{main}|}|\\
\end{tabular}
\end{center}
%
Both forms have slightly different effects as described above.
The main file is prepared as usual, see \secref{sec:include}.

%%%%%%%%%%%%%%%%%%%%%%%%%%%%%%%%%%%%%%%%%%%%%%%%%%%%%%%%%%%%%%%%%%%%%%%%%%%%%%%%
\subsection{Legacy Detection}
\label{sec:detection}

The directive |\childdocmain| in the main file can detect
whether the complete document or merely a child is to be compiled
even without using the directive |\childdocof|.
This method is deprecated because it is less robust
and there is no compelling reason to use it;
it is merely provided for backward compatibility
and it may be removed in future versions.

If the detection mechanism is to be used,
it is mandatory to correctly specify
the filename of the main file as the argument of |\childdocmain|:
%
\begin{center}
\begin{tabular}{l}
|\input{childdoc.def}|\\
|\childdocmain{|\textit{main}|}|\\
\end{tabular}
\end{center}
%
If |\jobname| does not match the argument \textit{main} of |\childdocmain|,
it is assumed that |\jobname| points to the child file to be compiled.
When using |\childdocmain| with the main file specified as argument,
it suffices to start a child file
with just |\input{|\textit{main}|}|
without loading of the package and using |\childdocof|.
If instead all processing is done
with the appropriate \textsf{childdoc} directives,
the argument of \textit{main} of |\childdocmain| can be empty.

An alternative version of the command line processing described
in \secref{sec:commandline} using the detection mechanism reads:
%
\begin{center}
|... -jobname "|\textit{target}|" "|[\textit{flags}]%
[|\def\jobname{|\textit{dest}|}|]|\input{|\textit{main}|}"|
\end{center}

%%%%%%%%%%%%%%%%%%%%%%%%%%%%%%%%%%%%%%%%%%%%%%%%%%%%%%%%%%%%%%%%%%%%%%%%%%%%%%%%
\subsection{Manual Code}
\label{sec:manual}

In case one cannot be certain whether the definitions file |childdoc.def|
is installed on the target \TeX{} distribution
and one prefers not to ship it,
it is conceivable to paste a few relevant commands into the sources.

To that end, drop all statements |\input{childdoc.def}|
and perform the replacements as outlined below.
Instead of |\childdocmain{|\textit{main}|}| add the following code
to the top of the main file:
%
\begin{center}
\begin{tabular}{l}
|\||ifdefined\childdocname\endinput\||fi\newif\ifchilddoc|\\
|\edef\childdocname{\scantokens\expandafter{\jobname\noexpand}}|\\
|\def\childdocmain{|\textit{main}|}\||ifx\childdocmain\childdocname\||else|\\
|\childdoctrue\includeonly{\childdocname}\let\jobname\childdocmain\||fi|\\
\end{tabular}
\end{center}
%
Instead of |\childdocof{|\textit{main}|}| just include the main file
at the top of each child file:
%
\begin{center}
|\input{|\textit{main}|}|
\end{center}
%
A simple redirection |\childdocforward{|\textit{dest}|}| is achieved by:
%
\begin{center}
|\def\jobname{|\textit{dest}|}\input{\jobname}|
\end{center}
%
The redirection with prefix
|\childdocforwardprefix[|\textit{prefix}|]{|\textit{dest}|}|
is accomplished by:
%
\begin{center}
\begin{tabular}{l}
|{\edef\jobname{\scantokens\expandafter{\jobname\noexpand}}|\\
|\def\redirectjob |\textit{prefix}|#1~~~{\gdef\jobname{|\textit{dest}|#1}}|\\
|\expandafter\redirectjob\jobname~~~}\input{\jobname}|
\end{tabular}
\end{center}

In an alternative approach,
child documents can be compiled by a specific command line
without additional code or specific definitions:
%
\begin{center}
|... -jobname "|\textit{target}|" "|[\textit{flags}]%
|\includeonly{|\textit{dest}|}\input{|\textit{main}|}"|
\end{center}
%

%%%%%%%%%%%%%%%%%%%%%%%%%%%%%%%%%%%%%%%%%%%%%%%%%%%%%%%%%%%%%%%%%%%%%%%%%%%%%%%%
%%%%%%%%%%%%%%%%%%%%%%%%%%%%%%%%%%%%%%%%%%%%%%%%%%%%%%%%%%%%%%%%%%%%%%%%%%%%%%%%
\section{Information}

%%%%%%%%%%%%%%%%%%%%%%%%%%%%%%%%%%%%%%%%%%%%%%%%%%%%%%%%%%%%%%%%%%%%%%%%%%%%%%%%
\subsection{Copyright}

Copyright \copyright{} 2017--2018 Niklas Beisert

This work may be distributed and/or modified under the
conditions of the \LaTeX{} Project Public License, either version 1.3
of this license or (at your option) any later version.
The latest version of this license is in
  \url{http://www.latex-project.org/lppl.txt}
and version 1.3 or later is part of all distributions of \LaTeX{}
version 2005/12/01 or later.

This work has the LPPL maintenance status `maintained'.

The Current Maintainer of this work is Niklas Beisert.

This work consists of the files |README.txt|, |childdoc.ins| and |childdoc.dtx|
as well as the derived files |childdoc.def|, |cdocsamp.tex|
with |cdocsch1.tex|, |cdocsch2.tex|, |cdocspt3.tex|, |cdocspt4.tex|,
|cdocsdrf.tex|, |cdocsfn1.tex|, |cdocsfn2.tex|
as well as |childdoc.pdf|.

%%%%%%%%%%%%%%%%%%%%%%%%%%%%%%%%%%%%%%%%%%%%%%%%%%%%%%%%%%%%%%%%%%%%%%%%%%%%%%%%
\subsection{Files and Installation}

The package consists of the files:
%
\begin{center}
\begin{tabular}{ll}
    |README.txt|   & readme file \\
    |childdoc.ins| & installation file \\
    |childdoc.dtx| & source file \\
    |childdoc.def| & definition file \\
    |cdocsamp.tex| & sample main file \\
    |cdocsch1.tex| & sample include file \\
    |cdocsch2.tex| & sample include file \\
    |cdocspt3.tex| & sample part file \\
    |cdocspt4.tex| & sample part file \\
    |cdocsdrf.tex| & sample redirection file \\
    |cdocsfn1.tex| & sample redirection file \\
    |cdocsfn2.tex| & sample redirection file \\
    |childdoc.pdf| & manual
\end{tabular}
\end{center}
%
The distribution consists of the files
|README.txt|, |childdoc.ins| and |childdoc.dtx|.
%
\begin{itemize}
\item
Run (pdf)\LaTeX{} on |childdoc.dtx|
to compile the manual |childdoc.pdf| (this file).
\item
Run \LaTeX{} on |childdoc.ins| to create the definitions file |childdoc.def|
and the sample |cdocsamp.tex| with include files
|cdocsch1.tex|, |cdocsch2.tex|, |cdocspt3.tex|, |cdocspt4.tex|,
|cdocsdrf.tex|, |cdocsfn1.tex|, |cdocsfn2.tex|.
Then copy the file |childdoc.def| to an appropriate directory of your \LaTeX{}
distribution, e.g.\ \textit{texmf-root}|/tex/latex/childdoc|.
\end{itemize}

%%%%%%%%%%%%%%%%%%%%%%%%%%%%%%%%%%%%%%%%%%%%%%%%%%%%%%%%%%%%%%%%%%%%%%%%%%%%%%%%
\subsection{Related CTAN Packages}

There are several other packages which offer a similar functionality:
%
\begin{itemize}
\item
The packages
\href{http://ctan.org/pkg/docmute}{\textsf{docmute}},
\href{http://ctan.org/pkg/includex}{\textsf{includex}} and
\href{http://ctan.org/pkg/standalone}{\textsf{standalone}}
provide commands to include only the document body of
a child file thus allowing both files to be compiled individually.
\item
The packages \href{http://ctan.org/pkg/subdocs}{\textsf{subdocs}}
and \href{http://ctan.org/pkg/subfiles}{\textsf{subfiles}}
provide structures in which the main and child documents can be
encapsulated and allowing them to be compiled individually.
The inclusion mechanism is different from the conventional |\include|.
\item
The package \href{http://ctan.org/pkg/combine}{\textsf{combine}}
is an elaborate solution to combine several documents into one.
\end{itemize}
%
See also the CTAN topic \href{http://ctan.org/topic/subdocs}{\textsf{subdocs}}
for further related packages.
The present package differs from the above solutions in that
a document structure constructed with the conventional |\include| mechanism
just needs two extra commands at the top of every file
such that all constituent files can be compiled individually.

%%%%%%%%%%%%%%%%%%%%%%%%%%%%%%%%%%%%%%%%%%%%%%%%%%%%%%%%%%%%%%%%%%%%%%%%%%%%%%%%
%\subsection{Feature Suggestions}
%
%The following is a list of features which may be useful for future
%versions of this package:
%%
%\begin{itemize}
%\item
%\ldots
%\end{itemize}

%%%%%%%%%%%%%%%%%%%%%%%%%%%%%%%%%%%%%%%%%%%%%%%%%%%%%%%%%%%%%%%%%%%%%%%%%%%%%%%%
\subsection{Revision History}

%%%%%%%%%%%%%%%%%%%%%%%%%%%%%%%%%%%%%%%%
\paragraph{v2.0:} 2018/12/30

\begin{itemize}
\item
immediate forward processing
\item
added |\childdocby| mechanism
\item
manual restructured
\end{itemize}

%%%%%%%%%%%%%%%%%%%%%%%%%%%%%%%%%%%%%%%%
\paragraph{v1.6:} 2018/01/17

\begin{itemize}
\item
application for development of include files
\item
corrections to manual
\end{itemize}

%%%%%%%%%%%%%%%%%%%%%%%%%%%%%%%%%%%%%%%%
\paragraph{v1.5:} 2017/05/21

\begin{itemize}
\item
more complete structuring introduced
\item
|\childdocof| introduced
\item
|\childdoc| renamed to |\childdocmain|
\item
|\childredirect| renamed to |\childdocforward| and |\childdocforwardprefix|
and functionality expanded
\end{itemize}

%%%%%%%%%%%%%%%%%%%%%%%%%%%%%%%%%%%%%%%%
\paragraph{v1.0:} 2017/04/27

\begin{itemize}
\item
manual and install package
\item
first version published on CTAN
\end{itemize}

%%%%%%%%%%%%%%%%%%%%%%%%%%%%%%%%%%%%%%%%
\paragraph{v0.6:} 2017/04/26

\begin{itemize}
\item
redirection mechanism added
\end{itemize}

%%%%%%%%%%%%%%%%%%%%%%%%%%%%%%%%%%%%%%%%
\paragraph{v0.5:} 2017/04/26

\begin{itemize}
\item
functionality in definition file
\end{itemize}


%%%%%%%%%%%%%%%%%%%%%%%%%%%%%%%%%%%%%%%%%%%%%%%%%%%%%%%%%%%%%%%%%%%%%%%%%%%%%%%%
%%%%%%%%%%%%%%%%%%%%%%%%%%%%%%%%%%%%%%%%%%%%%%%%%%%%%%%%%%%%%%%%%%%%%%%%%%%%%%%%
%%%%%%%%%%%%%%%%%%%%%%%%%%%%%%%%%%%%%%%%%%%%%%%%%%%%%%%%%%%%%%%%%%%%%%%%%%%%%%%%
\appendix

\settowidth\MacroIndent{\rmfamily\scriptsize 000\ }

 \DocInput{childdoc.dtx}

\end{document}
%</driver>
% \fi
%
% %%%%%%%%%%%%%%%%%%%%%%%%%%%%%%%%%%%%%%%%%%%%%%%%%%%%%%%%%%%%%%%%%%%%%%%%%%%%%%
% %%%%%%%%%%%%%%%%%%%%%%%%%%%%%%%%%%%%%%%%%%%%%%%%%%%%%%%%%%%%%%%%%%%%%%%%%%%%%%
% \section{Sample}
%\iffalse
%<*samplemain>
%\fi
%
% The following presents a sample document
% with two chapters, two parts, a title page,
% a compile flag as well as three forwarding files to set the flag.
% It consists of eight |.tex| files:
% \begin{center}
% \begin{tabular}{ll}
% |cdocsamp.tex|&main file\\
% |cdocsch1.tex|&include file for chapter 1\\
% |cdocsch2.tex|&include file for chapter 2\\
% |cdocspt3.tex|&include file for part 3\\
% |cdocspt4.tex|&include file for part 4\\
% |cdocsdrf.tex|&forwarding file for main file in draft mode\\
% |cdocsfi1.tex|&forwarding file for final version of chapter 1\\
% |cdocsfi2.tex|&forwarding file for final version of chapter 2\\
% \end{tabular}
% \end{center}
% Each of the eight files can be compiled directly by the \LaTeX{} compiler.
%
% %%%%%%%%%%%%%%%%%%%%%%%%%%%%%%%%%%%%%%
% \paragraph{Main File.}
%
% The main file is called |cdocsamp.tex|.
%
% Load the \textsf{childdoc} definitions and
% declare the filename for the main document:
%    \begin{macrocode}
\input{childdoc.def}
\childdocmain{}
%    \end{macrocode}

% Optional override for |\version| flag:
%    \begin{macrocode}
%%\ifchilddoc\else\providecommand{\version}{draft}\fi
%    \end{macrocode}

% Define the default values for the |\version| flag
% (|final| for the main file and |draft| for childs):
%    \begin{macrocode}
\ifchilddoc
\providecommand{\version}{draft}
\else
\providecommand{\version}{final}
\fi
%    \end{macrocode}

% Load the standard document class:
%    \begin{macrocode}
\documentclass[12pt]{article}
%    \end{macrocode}

% Start the document body:
%    \begin{macrocode}
\begin{document}
%    \end{macrocode}

% Declare a title page.
% Print title, part of document being processed and version flag:
%    \begin{macrocode}
\addtocounter{page}{-1}
\begin{center}
{\LARGE\bfseries{}childdoc example\par}
\vspace{1cm}
\ifchilddoc
\ifchilddocmanual part\else chapter\fi:
`\childdocname' of `\childdocjob'\par
\else
main document: `\childdocjob'\par
\fi
version: \version\par
\end{center}
\newpage
%    \end{macrocode}

% Manually include selected file,
% otherwise process as usual:
%    \begin{macrocode}
\ifchilddocmanual
\section*{part `\childdocname'}
\input{\childdocname}
\else
%    \end{macrocode}

% Include the two chapters:
%    \begin{macrocode}
\include{cdocsch1}
\include{cdocsch2}
%    \end{macrocode}

% Include the two parts unless only chapters should be displayed:
%    \begin{macrocode}
\ifchilddoc\else
\section{part three}
\input{cdocspt3}
\section{part four}
\input{cdocspt4}
\fi
%    \end{macrocode}

% Process as usual until here:
%    \begin{macrocode}
\fi
%    \end{macrocode}

% End of document body:
%    \begin{macrocode}
\end{document}
%    \end{macrocode}
%\iffalse
%</samplemain>
%\fi
%
% %%%%%%%%%%%%%%%%%%%%%%%%%%%%%%%%%%%%%%
% \paragraph{Chapter Include Files.}
%
% The include files are called |cdocsch1.tex| and |cdocsch2.tex|.
%
%\iffalse
%<*samplechap1|samplechap2>
%\fi

% Optional override for |\version| flag:
%    \begin{macrocode}
%%\providecommand{\version}{final}
%    \end{macrocode}

% Include the main document:
%    \begin{macrocode}
\input{childdoc.def}
\childdocof{cdocsamp}
%    \end{macrocode}

%\iffalse
%</samplechap1|samplechap2>
%\fi
%
%\iffalse
%<*samplechap1>
%\fi
% Some text for chapter 1:
%    \begin{macrocode}
\section{one}
some text in chapter one
%    \end{macrocode}

%\iffalse
%</samplechap1>
%\fi
% Some text for chapter 2:
%\iffalse
%<*samplechap2>
%\fi
%    \begin{macrocode}
\section{two}
more text in chapter two
%    \end{macrocode}

%\iffalse
%</samplechap2>
%\fi
%
% %%%%%%%%%%%%%%%%%%%%%%%%%%%%%%%%%%%%%%
% \paragraph{Part Include Files.}
%
% The include files are called |cdocspt3.tex| and |cdocspt4.tex|.
%
%\iffalse
%<*samplepart3|samplepart4>
%\fi

% Optional override for |\version| flag:
%    \begin{macrocode}
%%\providecommand{\version}{final}
%    \end{macrocode}

% Include the main document:
%    \begin{macrocode}
\input{childdoc.def}
\childdocby{cdocsamp}
%    \end{macrocode}

%\iffalse
%</samplepart3|samplepart4>
%\fi
%
%\iffalse
%<*samplepart3>
%\fi
% Some text for part 3:
%    \begin{macrocode}
some text in part three
%    \end{macrocode}

%\iffalse
%</samplepart3>
%\fi
% Some text for part 4:
%\iffalse
%<*samplepart4>
%\fi
%    \begin{macrocode}
more text in part four
%    \end{macrocode}

%\iffalse
%</samplepart4>
%\fi
%
% %%%%%%%%%%%%%%%%%%%%%%%%%%%%%%%%%%%%%%
% \paragraph{Forwarding for a Complete Draft.}
%
% The following forwarding file |cdocsdrf.tex|
% compiles the main document in draft mode:
%\iffalse
%<*sampledraft>
%\fi
%    \begin{macrocode}
\def\version{draft}
\input{childdoc.def}
\childdocforward{cdocsamp}
%    \end{macrocode}

%\iffalse
%</sampledraft>
%\fi
%
% %%%%%%%%%%%%%%%%%%%%%%%%%%%%%%%%%%%%%%
% \paragraph{Forwarding for Final Version of the Chapters.}
%
% The following forwarding files |cdocsfn1.tex| and |cdocsfn2.tex|
% (with identical content)
% compile the final versions of the child documents
% |cdocsch1.tex| and |cdocsch2.tex|, respectively:
%\iffalse
%<*samplefinal>
%\fi
%    \begin{macrocode}
\def\version{final}
\input{childdoc.def}
\childdocforwardprefix[cdocsamp]{cdocsfn}{cdocsch}
%    \end{macrocode}

%\iffalse
%</samplefinal>
%\fi
%
% %%%%%%%%%%%%%%%%%%%%%%%%%%%%%%%%%%%%%%
% \paragraph{Command Line Processing.}
%
% The following three command lines generate the output files
% |cdocscld|, |cdocscl1| and |cdocscl2|
% which should be identical to
% |cdocsdrf|, |cdocsch1| and |cdocsfn2|, respectively:
% \begin{center}
% \begin{tabular}{l}
% |latex -jobname cdocscld \|\\
% |  "\def\version{draft}\input{childdoc.def}\childdocforward{cdocsamp}"|\\
% |latex -jobname cdocscl1 \|\\
% |  "\input{childdoc.def}\childdocforward[cdocsamp]{cdocsch1}"|\\
% |latex -jobname cdocscl2 \|\\
% |  "\def\version{final}\input{childdoc.def}\childdocforward{cdocsch2}"|
% \end{tabular}
% \end{center}
% Note that the trailing backslash on each first line
% merely continues the input to the second line
% (for convenient cut ant paste).
% Furthermore, the command |latex| can be replaced by any
% of its alternative versions such as |pdflatex|.
%
% %%%%%%%%%%%%%%%%%%%%%%%%%%%%%%%%%%%%%%%%%%%%%%%%%%%%%%%%%%%%%%%%%%%%%%%%%%%%%%
% %%%%%%%%%%%%%%%%%%%%%%%%%%%%%%%%%%%%%%%%%%%%%%%%%%%%%%%%%%%%%%%%%%%%%%%%%%%%%%
% \section{Implementation}
%\iffalse
%<*package>
%\fi
%
% This section describes the definitions file |childdoc.def|.

% The definitions cannot be loaded using |\usepackage| or |\RequirePackage|
% which has a mechanism to prevent loading a style file more than once.
% When loading the definitions by means of |\input|
% multiple instances have to be prevented manually:
%\iffalse
%This code needs to be before the `\ProvidesFile' directive
%which is defined at the beginning of this file.
%Therefore it is also placed there and commented out here.
%</package>
%<*discard>
%\fi
%    \begin{macrocode}
\ifdefined\childdocmain\endinput\fi
%    \end{macrocode}
%\iffalse
%</discard>
%<*package>
%\fi
%
% \macro{\ifchilddoc}
% \macro{\ifchilddocmanual}
% The conditional |\ifchilddoc| tells whether a
% child (true) or main (false) document is being compiled.
% The conditional |\ifchilddocmanual| tells whether
% the |\includeonly| mechanism is used (false) or
% the selection of child files must be performed manually (true).
% The definitions initialise to false:
%    \begin{macrocode}
\newif\ifchilddoc
\newif\ifchilddocmanual
%    \end{macrocode}

% \macro{\childdocname}
% \macro{\childdocjob}
% The macro |\childdocname| stores the name of the main document
% to be compiled. The macro |\childdocjob| stores the name of
% the document on which the \LaTeX{} compiler was originally invoked.
% The content of |\jobname| cannot be compared
% to filenames specified in the source due to different catcodes.
% The following code rescans |\jobname|, stores the result
% in |\childdocname| and saves a copy in |\childdocjob|:
%    \begin{macrocode}
\edef\childdocname{\scantokens\expandafter{\jobname\noexpand}}
\let\childdocjob\childdocname
%    \end{macrocode}

% \macro{\childdocdisable}
% The macro |\childdocdisable| prevents the main file
% from being processed more than once.
% At this stage, the main document command |\childdocmain|
% is assumed to be called once again where it should do nothing.
% Any subsequent call to it should prevent
% a secondary processing of the main document
% It overwrites the forwarding commands
% |\childdocof| and |\childdocforward|
% with empty macros to prevent further inclusions of the main document:
%    \begin{macrocode}
\newcommand{\childdocdisable}
{
  \renewcommand{\childdocmain}[1]{\renewcommand{\childdocmain}[1]{\endinput}}
  \renewcommand{\childdocof}[1]{}
  \renewcommand{\childdocby}[2][]{}
  \renewcommand{\childdocforward}[2][]{}
  \renewcommand{\childdocdisable}{}
}
%    \end{macrocode}

% \macro{\childdocmain}
% The macro |\childdocmain| is to be called at the top of the main file
% with nothing or the main filename (without extension) as argument.
% First, it breaks loops.
% If the argument is not empty and does not match |\childdocname|
% (which is set by the first inclusion of |childdoc.def|),
% |\ifchilddoc| is set to true, |\includeonly| is applied to the child file
% and |\jobname| is set to the main file
% (for proper handling of |.aux| files):
%    \begin{macrocode}
\newcommand{\childdocmain}[1]
{
  \childdocdisable\childdocmain{}
  \if?#1?\else
    \begingroup
      \def\childdoctmp{#1}
      \ifx\childdoctmp\childdocname
        \def\childdoctmp{}
      \else
        \def\childdoctmp
        {
          \childdoctrue
          \includeonly{\childdocname}
          \def\childdocjob{#1}
          \def\jobname{#1}
        }
      \fi
      \expandafter
    \endgroup
    \childdoctmp
  \fi
}
%    \end{macrocode}

% \macro{\childdocof}
% The command |\childdocof| redirects
% compilation to the main file |#1|.
%    \begin{macrocode}
\newcommand{\childdocof}[1]
{
  \childdocdisable
  \childdoctrue
  \includeonly{\childdocname}
  \def\jobname{#1}
  \def\childdocjob{#1}
  \input{#1}
}
%    \end{macrocode}

% \macro{\childdocby}
% The command |\childdocby| ....
%    \begin{macrocode}
\newcommand{\childdocby}[2][]
{
  \childdocdisable
  \childdoctrue
  \childdocmanualtrue
  \if?#1?\else
    \def\jobname{#2}
  \fi
  \def\childdocjob{#2}
  \input{#2}
  \endinput
}
%    \end{macrocode}

% \macro{\childdocforward}
% The command |\childdocforward| redirects
% compilation to the main file or
% (if the optional argument is given) a child file.
% Parameters are set as if the main file
% or a child file starting with |\childdocof| was compiled.
% Then compilation is handed over to the main file:
%    \begin{macrocode}
\newcommand{\childdocforward}[2][]
{
  \begingroup
    \if?#1?
      \def\childdoctmp
      {
        \def\childdocname{#2}
        \def\childdocjob{#2}
        \def\jobname{#2}
        \input{#2}
        \endinput
      }
    \else
      \def\childdoctmp
      {
        \childdocdisable
        \def\childdocname{#2}
        \childdoctrue
        \includeonly{#2}
        \def\childdocjob{#1}
        \def\jobname{#1}
        \input{#1}
        \endinput
      }
    \fi
    \expandafter
  \endgroup
  \childdoctmp
}
%    \end{macrocode}

% \macro{\childdocforwardprefix}
% The command |\childdocforwardprefix| redirects
% compilation to the main or a child file by means of a pattern.
% The prefix |#1| in the current filename is replaced by |#2|
% and the suffix of the current filename is kept
% (it is assumed that the filename does not contain the substring `|~~~|'
% which is used as a delimiter).
% Compilation is handed over to the new file by |\childdocforward|:
%    \begin{macrocode}
\newcommand{\childdocforwardprefix}[3][]
{
  \begingroup
    \def\childdocextract #2##1~~~{\def\childdoctmp{\childdocforward[#1]{#3##1}}}
    \expandafter\childdocextract\childdocname~~~
    \expandafter
  \endgroup
  \childdoctmp
}
%    \end{macrocode}

% \macro{\childdoc}
% The deprecated macro |\childdoc| is a legacy version of |\childdocmain|:
%    \begin{macrocode}
\newcommand{\childdoc}{\childdocmain}
%    \end{macrocode}

% \macro{\childdocredirect}
% The deprecated macro |\childdocredirect| is a legacy version
% of |\childdocforward| and |\childdocforwardprefix|:
%    \begin{macrocode}
\newcommand{\childdocredirect}[2][]
{
  \begingroup
    \if?#1?
      \def\childdoctmp{\childdocforward{#2}}
    \else
      \def\childdoctmp{\childdocforwardprefix{#1}{#2}}
    \fi
    \expandafter
  \endgroup
  \childdoctmp
}
%    \end{macrocode}

%\iffalse
%</package>
%\fi
%
\endinput
\childdocforward{cdocsch2}"|
% \end{tabular}
% \end{center}
% Note that the trailing backslash on each first line
% merely continues the input to the second line
% (for convenient cut ant paste).
% Furthermore, the command |latex| can be replaced by any
% of its alternative versions such as |pdflatex|.
%
% %%%%%%%%%%%%%%%%%%%%%%%%%%%%%%%%%%%%%%%%%%%%%%%%%%%%%%%%%%%%%%%%%%%%%%%%%%%%%%
% %%%%%%%%%%%%%%%%%%%%%%%%%%%%%%%%%%%%%%%%%%%%%%%%%%%%%%%%%%%%%%%%%%%%%%%%%%%%%%
% \section{Implementation}
%\iffalse
%<*package>
%\fi
%
% This section describes the definitions file |childdoc.def|.

% The definitions cannot be loaded using |\usepackage| or |\RequirePackage|
% which has a mechanism to prevent loading a style file more than once.
% When loading the definitions by means of |\input|
% multiple instances have to be prevented manually:
%\iffalse
%This code needs to be before the `\ProvidesFile' directive
%which is defined at the beginning of this file.
%Therefore it is also placed there and commented out here.
%</package>
%<*discard>
%\fi
%    \begin{macrocode}
\ifdefined\childdocmain\endinput\fi
%    \end{macrocode}
%\iffalse
%</discard>
%<*package>
%\fi
%
% \macro{\ifchilddoc}
% \macro{\ifchilddocmanual}
% The conditional |\ifchilddoc| tells whether a
% child (true) or main (false) document is being compiled.
% The conditional |\ifchilddocmanual| tells whether
% the |\includeonly| mechanism is used (false) or
% the selection of child files must be performed manually (true).
% The definitions initialise to false:
%    \begin{macrocode}
\newif\ifchilddoc
\newif\ifchilddocmanual
%    \end{macrocode}

% \macro{\childdocname}
% \macro{\childdocjob}
% The macro |\childdocname| stores the name of the main document
% to be compiled. The macro |\childdocjob| stores the name of
% the document on which the \LaTeX{} compiler was originally invoked.
% The content of |\jobname| cannot be compared
% to filenames specified in the source due to different catcodes.
% The following code rescans |\jobname|, stores the result
% in |\childdocname| and saves a copy in |\childdocjob|:
%    \begin{macrocode}
\edef\childdocname{\scantokens\expandafter{\jobname\noexpand}}
\let\childdocjob\childdocname
%    \end{macrocode}

% \macro{\childdocdisable}
% The macro |\childdocdisable| prevents the main file
% from being processed more than once.
% At this stage, the main document command |\childdocmain|
% is assumed to be called once again where it should do nothing.
% Any subsequent call to it should prevent
% a secondary processing of the main document
% It overwrites the forwarding commands
% |\childdocof| and |\childdocforward|
% with empty macros to prevent further inclusions of the main document:
%    \begin{macrocode}
\newcommand{\childdocdisable}
{
  \renewcommand{\childdocmain}[1]{\renewcommand{\childdocmain}[1]{\endinput}}
  \renewcommand{\childdocof}[1]{}
  \renewcommand{\childdocby}[2][]{}
  \renewcommand{\childdocforward}[2][]{}
  \renewcommand{\childdocdisable}{}
}
%    \end{macrocode}

% \macro{\childdocmain}
% The macro |\childdocmain| is to be called at the top of the main file
% with nothing or the main filename (without extension) as argument.
% First, it breaks loops.
% If the argument is not empty and does not match |\childdocname|
% (which is set by the first inclusion of |childdoc.def|),
% |\ifchilddoc| is set to true, |\includeonly| is applied to the child file
% and |\jobname| is set to the main file
% (for proper handling of |.aux| files):
%    \begin{macrocode}
\newcommand{\childdocmain}[1]
{
  \childdocdisable\childdocmain{}
  \if?#1?\else
    \begingroup
      \def\childdoctmp{#1}
      \ifx\childdoctmp\childdocname
        \def\childdoctmp{}
      \else
        \def\childdoctmp
        {
          \childdoctrue
          \includeonly{\childdocname}
          \def\childdocjob{#1}
          \def\jobname{#1}
        }
      \fi
      \expandafter
    \endgroup
    \childdoctmp
  \fi
}
%    \end{macrocode}

% \macro{\childdocof}
% The command |\childdocof| redirects
% compilation to the main file |#1|.
%    \begin{macrocode}
\newcommand{\childdocof}[1]
{
  \childdocdisable
  \childdoctrue
  \includeonly{\childdocname}
  \def\jobname{#1}
  \def\childdocjob{#1}
  \input{#1}
}
%    \end{macrocode}

% \macro{\childdocby}
% The command |\childdocby| ....
%    \begin{macrocode}
\newcommand{\childdocby}[2][]
{
  \childdocdisable
  \childdoctrue
  \childdocmanualtrue
  \if?#1?\else
    \def\jobname{#2}
  \fi
  \def\childdocjob{#2}
  \input{#2}
  \endinput
}
%    \end{macrocode}

% \macro{\childdocforward}
% The command |\childdocforward| redirects
% compilation to the main file or
% (if the optional argument is given) a child file.
% Parameters are set as if the main file
% or a child file starting with |\childdocof| was compiled.
% Then compilation is handed over to the main file:
%    \begin{macrocode}
\newcommand{\childdocforward}[2][]
{
  \begingroup
    \if?#1?
      \def\childdoctmp
      {
        \def\childdocname{#2}
        \def\childdocjob{#2}
        \def\jobname{#2}
        \input{#2}
        \endinput
      }
    \else
      \def\childdoctmp
      {
        \childdocdisable
        \def\childdocname{#2}
        \childdoctrue
        \includeonly{#2}
        \def\childdocjob{#1}
        \def\jobname{#1}
        \input{#1}
        \endinput
      }
    \fi
    \expandafter
  \endgroup
  \childdoctmp
}
%    \end{macrocode}

% \macro{\childdocforwardprefix}
% The command |\childdocforwardprefix| redirects
% compilation to the main or a child file by means of a pattern.
% The prefix |#1| in the current filename is replaced by |#2|
% and the suffix of the current filename is kept
% (it is assumed that the filename does not contain the substring `|~~~|'
% which is used as a delimiter).
% Compilation is handed over to the new file by |\childdocforward|:
%    \begin{macrocode}
\newcommand{\childdocforwardprefix}[3][]
{
  \begingroup
    \def\childdocextract #2##1~~~{\def\childdoctmp{\childdocforward[#1]{#3##1}}}
    \expandafter\childdocextract\childdocname~~~
    \expandafter
  \endgroup
  \childdoctmp
}
%    \end{macrocode}

% \macro{\childdoc}
% The deprecated macro |\childdoc| is a legacy version of |\childdocmain|:
%    \begin{macrocode}
\newcommand{\childdoc}{\childdocmain}
%    \end{macrocode}

% \macro{\childdocredirect}
% The deprecated macro |\childdocredirect| is a legacy version
% of |\childdocforward| and |\childdocforwardprefix|:
%    \begin{macrocode}
\newcommand{\childdocredirect}[2][]
{
  \begingroup
    \if?#1?
      \def\childdoctmp{\childdocforward{#2}}
    \else
      \def\childdoctmp{\childdocforwardprefix{#1}{#2}}
    \fi
    \expandafter
  \endgroup
  \childdoctmp
}
%    \end{macrocode}

%\iffalse
%</package>
%\fi
%
\endinput
|
and perform the replacements as outlined below.
Instead of |\childdocmain{|\textit{main}|}| add the following code
to the top of the main file:
%
\begin{center}
\begin{tabular}{l}
|\||ifdefined\childdocname\endinput\||fi\newif\ifchilddoc|\\
|\edef\childdocname{\scantokens\expandafter{\jobname\noexpand}}|\\
|\def\childdocmain{|\textit{main}|}\||ifx\childdocmain\childdocname\||else|\\
|\childdoctrue\includeonly{\childdocname}\let\jobname\childdocmain\||fi|\\
\end{tabular}
\end{center}
%
Instead of |\childdocof{|\textit{main}|}| just include the main file
at the top of each child file:
%
\begin{center}
|\input{|\textit{main}|}|
\end{center}
%
A simple redirection |\childdocforward{|\textit{dest}|}| is achieved by:
%
\begin{center}
|\def\jobname{|\textit{dest}|}\input{\jobname}|
\end{center}
%
The redirection with prefix
|\childdocforwardprefix[|\textit{prefix}|]{|\textit{dest}|}|
is accomplished by:
%
\begin{center}
\begin{tabular}{l}
|{\edef\jobname{\scantokens\expandafter{\jobname\noexpand}}|\\
|\def\redirectjob |\textit{prefix}|#1~~~{\gdef\jobname{|\textit{dest}|#1}}|\\
|\expandafter\redirectjob\jobname~~~}\input{\jobname}|
\end{tabular}
\end{center}

In an alternative approach,
child documents can be compiled by a specific command line
without additional code or specific definitions:
%
\begin{center}
|... -jobname "|\textit{target}|" "|[\textit{flags}]%
|\includeonly{|\textit{dest}|}\input{|\textit{main}|}"|
\end{center}
%

%%%%%%%%%%%%%%%%%%%%%%%%%%%%%%%%%%%%%%%%%%%%%%%%%%%%%%%%%%%%%%%%%%%%%%%%%%%%%%%%
%%%%%%%%%%%%%%%%%%%%%%%%%%%%%%%%%%%%%%%%%%%%%%%%%%%%%%%%%%%%%%%%%%%%%%%%%%%%%%%%
\section{Information}

%%%%%%%%%%%%%%%%%%%%%%%%%%%%%%%%%%%%%%%%%%%%%%%%%%%%%%%%%%%%%%%%%%%%%%%%%%%%%%%%
\subsection{Copyright}

Copyright \copyright{} 2017--2018 Niklas Beisert

This work may be distributed and/or modified under the
conditions of the \LaTeX{} Project Public License, either version 1.3
of this license or (at your option) any later version.
The latest version of this license is in
  \url{http://www.latex-project.org/lppl.txt}
and version 1.3 or later is part of all distributions of \LaTeX{}
version 2005/12/01 or later.

This work has the LPPL maintenance status `maintained'.

The Current Maintainer of this work is Niklas Beisert.

This work consists of the files |README.txt|, |childdoc.ins| and |childdoc.dtx|
as well as the derived files |childdoc.def|, |cdocsamp.tex|
with |cdocsch1.tex|, |cdocsch2.tex|, |cdocspt3.tex|, |cdocspt4.tex|,
|cdocsdrf.tex|, |cdocsfn1.tex|, |cdocsfn2.tex|
as well as |childdoc.pdf|.

%%%%%%%%%%%%%%%%%%%%%%%%%%%%%%%%%%%%%%%%%%%%%%%%%%%%%%%%%%%%%%%%%%%%%%%%%%%%%%%%
\subsection{Files and Installation}

The package consists of the files:
%
\begin{center}
\begin{tabular}{ll}
    |README.txt|   & readme file \\
    |childdoc.ins| & installation file \\
    |childdoc.dtx| & source file \\
    |childdoc.def| & definition file \\
    |cdocsamp.tex| & sample main file \\
    |cdocsch1.tex| & sample include file \\
    |cdocsch2.tex| & sample include file \\
    |cdocspt3.tex| & sample part file \\
    |cdocspt4.tex| & sample part file \\
    |cdocsdrf.tex| & sample redirection file \\
    |cdocsfn1.tex| & sample redirection file \\
    |cdocsfn2.tex| & sample redirection file \\
    |childdoc.pdf| & manual
\end{tabular}
\end{center}
%
The distribution consists of the files
|README.txt|, |childdoc.ins| and |childdoc.dtx|.
%
\begin{itemize}
\item
Run (pdf)\LaTeX{} on |childdoc.dtx|
to compile the manual |childdoc.pdf| (this file).
\item
Run \LaTeX{} on |childdoc.ins| to create the definitions file |childdoc.def|
and the sample |cdocsamp.tex| with include files
|cdocsch1.tex|, |cdocsch2.tex|, |cdocspt3.tex|, |cdocspt4.tex|,
|cdocsdrf.tex|, |cdocsfn1.tex|, |cdocsfn2.tex|.
Then copy the file |childdoc.def| to an appropriate directory of your \LaTeX{}
distribution, e.g.\ \textit{texmf-root}|/tex/latex/childdoc|.
\end{itemize}

%%%%%%%%%%%%%%%%%%%%%%%%%%%%%%%%%%%%%%%%%%%%%%%%%%%%%%%%%%%%%%%%%%%%%%%%%%%%%%%%
\subsection{Related CTAN Packages}

There are several other packages which offer a similar functionality:
%
\begin{itemize}
\item
The packages
\href{http://ctan.org/pkg/docmute}{\textsf{docmute}},
\href{http://ctan.org/pkg/includex}{\textsf{includex}} and
\href{http://ctan.org/pkg/standalone}{\textsf{standalone}}
provide commands to include only the document body of
a child file thus allowing both files to be compiled individually.
\item
The packages \href{http://ctan.org/pkg/subdocs}{\textsf{subdocs}}
and \href{http://ctan.org/pkg/subfiles}{\textsf{subfiles}}
provide structures in which the main and child documents can be
encapsulated and allowing them to be compiled individually.
The inclusion mechanism is different from the conventional |\include|.
\item
The package \href{http://ctan.org/pkg/combine}{\textsf{combine}}
is an elaborate solution to combine several documents into one.
\end{itemize}
%
See also the CTAN topic \href{http://ctan.org/topic/subdocs}{\textsf{subdocs}}
for further related packages.
The present package differs from the above solutions in that
a document structure constructed with the conventional |\include| mechanism
just needs two extra commands at the top of every file
such that all constituent files can be compiled individually.

%%%%%%%%%%%%%%%%%%%%%%%%%%%%%%%%%%%%%%%%%%%%%%%%%%%%%%%%%%%%%%%%%%%%%%%%%%%%%%%%
%\subsection{Feature Suggestions}
%
%The following is a list of features which may be useful for future
%versions of this package:
%%
%\begin{itemize}
%\item
%\ldots
%\end{itemize}

%%%%%%%%%%%%%%%%%%%%%%%%%%%%%%%%%%%%%%%%%%%%%%%%%%%%%%%%%%%%%%%%%%%%%%%%%%%%%%%%
\subsection{Revision History}

%%%%%%%%%%%%%%%%%%%%%%%%%%%%%%%%%%%%%%%%
\paragraph{v2.0:} 2018/12/30

\begin{itemize}
\item
immediate forward processing
\item
added |\childdocby| mechanism
\item
manual restructured
\end{itemize}

%%%%%%%%%%%%%%%%%%%%%%%%%%%%%%%%%%%%%%%%
\paragraph{v1.6:} 2018/01/17

\begin{itemize}
\item
application for development of include files
\item
corrections to manual
\end{itemize}

%%%%%%%%%%%%%%%%%%%%%%%%%%%%%%%%%%%%%%%%
\paragraph{v1.5:} 2017/05/21

\begin{itemize}
\item
more complete structuring introduced
\item
|\childdocof| introduced
\item
|\childdoc| renamed to |\childdocmain|
\item
|\childredirect| renamed to |\childdocforward| and |\childdocforwardprefix|
and functionality expanded
\end{itemize}

%%%%%%%%%%%%%%%%%%%%%%%%%%%%%%%%%%%%%%%%
\paragraph{v1.0:} 2017/04/27

\begin{itemize}
\item
manual and install package
\item
first version published on CTAN
\end{itemize}

%%%%%%%%%%%%%%%%%%%%%%%%%%%%%%%%%%%%%%%%
\paragraph{v0.6:} 2017/04/26

\begin{itemize}
\item
redirection mechanism added
\end{itemize}

%%%%%%%%%%%%%%%%%%%%%%%%%%%%%%%%%%%%%%%%
\paragraph{v0.5:} 2017/04/26

\begin{itemize}
\item
functionality in definition file
\end{itemize}


%%%%%%%%%%%%%%%%%%%%%%%%%%%%%%%%%%%%%%%%%%%%%%%%%%%%%%%%%%%%%%%%%%%%%%%%%%%%%%%%
%%%%%%%%%%%%%%%%%%%%%%%%%%%%%%%%%%%%%%%%%%%%%%%%%%%%%%%%%%%%%%%%%%%%%%%%%%%%%%%%
%%%%%%%%%%%%%%%%%%%%%%%%%%%%%%%%%%%%%%%%%%%%%%%%%%%%%%%%%%%%%%%%%%%%%%%%%%%%%%%%
\appendix

\settowidth\MacroIndent{\rmfamily\scriptsize 000\ }

 \DocInput{childdoc.dtx}

\end{document}
%</driver>
% \fi
%
% %%%%%%%%%%%%%%%%%%%%%%%%%%%%%%%%%%%%%%%%%%%%%%%%%%%%%%%%%%%%%%%%%%%%%%%%%%%%%%
% %%%%%%%%%%%%%%%%%%%%%%%%%%%%%%%%%%%%%%%%%%%%%%%%%%%%%%%%%%%%%%%%%%%%%%%%%%%%%%
% \section{Sample}
%\iffalse
%<*samplemain>
%\fi
%
% The following presents a sample document
% with two chapters, two parts, a title page,
% a compile flag as well as three forwarding files to set the flag.
% It consists of eight |.tex| files:
% \begin{center}
% \begin{tabular}{ll}
% |cdocsamp.tex|&main file\\
% |cdocsch1.tex|&include file for chapter 1\\
% |cdocsch2.tex|&include file for chapter 2\\
% |cdocspt3.tex|&include file for part 3\\
% |cdocspt4.tex|&include file for part 4\\
% |cdocsdrf.tex|&forwarding file for main file in draft mode\\
% |cdocsfi1.tex|&forwarding file for final version of chapter 1\\
% |cdocsfi2.tex|&forwarding file for final version of chapter 2\\
% \end{tabular}
% \end{center}
% Each of the eight files can be compiled directly by the \LaTeX{} compiler.
%
% %%%%%%%%%%%%%%%%%%%%%%%%%%%%%%%%%%%%%%
% \paragraph{Main File.}
%
% The main file is called |cdocsamp.tex|.
%
% Load the \textsf{childdoc} definitions and
% declare the filename for the main document:
%    \begin{macrocode}
% \iffalse
%
% childdoc.dtx Copyright (C) 2017-2018 Niklas Beisert
%
% This work may be distributed and/or modified under the
% conditions of the LaTeX Project Public License, either version 1.3
% of this license or (at your option) any later version.
% The latest version of this license is in
%   http://www.latex-project.org/lppl.txt
% and version 1.3 or later is part of all distributions of LaTeX
% version 2005/12/01 or later.
%
% This work has the LPPL maintenance status `maintained'.
%
% The Current Maintainer of this work is Niklas Beisert.
%
% This work consists of the files childdoc.dtx and childdoc.ins
% and the derived files childdoc.def and cdocsamp.tex with
% cdocsch1.tex, cdocsch2.tex, cdocsdrf.tex, cdocsfn1.tex, cdocsfn2.tex.
%
%<package>\ifdefined\childdocmain\endinput\fi
%<package>\ProvidesFile{childdoc.def}[2018/12/30 v2.0 child document driver]
%<samplemain>\ProvidesFile{cdocsamp.tex}[2018/12/30 v2.0 sample for childdoc]
%<*driver>
%\ProvidesFile{childdoc.drv}[2018/12/30 v2.0 childdoc reference manual file]
\PassOptionsToClass{10pt,a4paper}{article}
\documentclass{ltxdoc}

\usepackage[margin=35mm]{geometry}
\usepackage{hyperref}
\usepackage{hyperxmp}
\usepackage[usenames]{color}

\hypersetup{colorlinks=true}
\hypersetup{pdfstartview=FitH}
\hypersetup{pdfpagemode=UseNone}
\hypersetup{pdfsource={}}
\hypersetup{pdflang={en-UK}}
\hypersetup{pdfcopyright={Copyright 2017-2018 Niklas Beisert.
  This work may be distributed and/or modified under the
  conditions of the LaTeX Project Public License, either version 1.3
  of this license or (at your option) any later version.}}
\hypersetup{pdflicenseurl={http://www.latex-project.org/lppl.txt}}
\hypersetup{pdfcontactaddress={ETH Zurich, ITP, HIT K,
  Wolfgang-Pauli-Strasse 27}}
\hypersetup{pdfcontactpostcode={8093}}
\hypersetup{pdfcontactcity={Zurich}}
\hypersetup{pdfcontactcountry={Switzerland}}
\hypersetup{pdfcontactemail={nbeisert@itp.phys.ethz.ch}}
\hypersetup{pdfcontacturl={http://people.phys.ethz.ch/\xmptilde nbeisert/}}

\newcommand{\secref}[1]{\hyperref[#1]{section \ref*{#1}}}

\parskip1ex
\parindent0pt
\let\olditemize\itemize
\def\itemize{\olditemize\parskip0pt}

\begin{document}

\title{The \textsf{childdoc} Package}
\hypersetup{pdftitle={The childdoc Package}}
\author{Niklas Beisert\\[2ex]
  Institut f\"ur Theoretische Physik\\
  Eidgen\"ossische Technische Hochschule Z\"urich\\
  Wolfgang-Pauli-Strasse 27, 8093 Z\"urich, Switzerland\\[1ex]
  \href{mailto:nbeisert@itp.phys.ethz.ch}
  {\texttt{nbeisert@itp.phys.ethz.ch}}}
\hypersetup{pdfauthor={Niklas Beisert}}
\hypersetup{pdfsubject={Manual for the LaTeX2e Package childdoc}}
\date{30 December 2018, \textsf{v2.0}}
\maketitle

\begin{abstract}\noindent
\textsf{childdoc} is a \LaTeXe{} package
that enables the direct compilation
of document sections included by |\include|
to individual files.
\end{abstract}

\begingroup
\parskip0ex
\tableofcontents
\endgroup

%%%%%%%%%%%%%%%%%%%%%%%%%%%%%%%%%%%%%%%%%%%%%%%%%%%%%%%%%%%%%%%%%%%%%%%%%%%%%%%%
%%%%%%%%%%%%%%%%%%%%%%%%%%%%%%%%%%%%%%%%%%%%%%%%%%%%%%%%%%%%%%%%%%%%%%%%%%%%%%%%
\section{Introduction}

\LaTeX{} provides a mechanism to structure a large document (such as a book)
into a main file and several child files (containing the chapters)
using the |\include| command.
This mechanism is beneficial for documents
which span hundreds of pages in order to
make the source file(s) more manageable.
Moreover, compilation can be restricted to
selected child files by means of the |\includeonly| command.
The latter feature can be used to reduce the compilation time while editing
(this was significantly more useful in the earlier days of \LaTeX{})
or to generate a smaller document which is easier to navigate.
Another application of |\includeonly| is to generate
documents consisting of selected parts of the complete document.

However, there are a few drawbacks of the plain |\include| mechanism:
\begin{itemize}
\item
The child files cannot be compiled on their own,
they can only be compiled via the main file.
A naive editing environment
(such as a text editor with an option
to have the current file processed by \LaTeX)
may require one to switch to the main file before compiling;
attempting to compile the child file produces errors.
\item
The main file must be modified (each time)
to adjust the |\includeonly| command
to the present needs. This easily leaves the main file in a messy state.
\item
The generated document will always carry the filename
of the main document. This is inconvenient if
several child files are to be compiled and
to be kept for distribution.
\end{itemize}

The present package provides a simple interface
to make child files individually compilable by \LaTeX{}.
Compiling a child file then has the same effect as compiling
the main file with an |\includeonly| command
to select the appropriate child.
Moreover the generated document will carry the name of the child
rather than the main file.
This resolves all three above issues.

This feature is meant to make the editing of books,
thesis documents and lecture notes somewhat more convenient.
However, the package can also be used efficiently for
composing a series of documents (such as exercise sheets)
which are typically distributed individually.
It then assists the author in generating the individual documents
(potentially in different versions)
as well as a document containing the collected series.
Another application is in developing style files
or other kinds of included material
where compilation of the style file could redirect
to a sample or test file.

%%%%%%%%%%%%%%%%%%%%%%%%%%%%%%%%%%%%%%%%%%%%%%%%%%%%%%%%%%%%%%%%%%%%%%%%%%%%%%%%
%%%%%%%%%%%%%%%%%%%%%%%%%%%%%%%%%%%%%%%%%%%%%%%%%%%%%%%%%%%%%%%%%%%%%%%%%%%%%%%%
\section{Usage}

First of all, the package \textsf{childdoc} is \emph{not} a standard
\LaTeXe{} |.sty| style file! Therefore it needs to be invoked in
a non-standard way.

%%%%%%%%%%%%%%%%%%%%%%%%%%%%%%%%%%%%%%%%%%%%%%%%%%%%%%%%%%%%%%%%%%%%%%%%%%%%%%%%
\subsection{Included Files}
\label{sec:include}

%%%%%%%%%%%%%%%%%%%%%%%%%%%%%%%%%%%%%%%%
\DescribeMacro{\childdocmain}
To use the package, add the commands
\begin{center}
\begin{tabular}{l}
|% \iffalse
%
% childdoc.dtx Copyright (C) 2017-2018 Niklas Beisert
%
% This work may be distributed and/or modified under the
% conditions of the LaTeX Project Public License, either version 1.3
% of this license or (at your option) any later version.
% The latest version of this license is in
%   http://www.latex-project.org/lppl.txt
% and version 1.3 or later is part of all distributions of LaTeX
% version 2005/12/01 or later.
%
% This work has the LPPL maintenance status `maintained'.
%
% The Current Maintainer of this work is Niklas Beisert.
%
% This work consists of the files childdoc.dtx and childdoc.ins
% and the derived files childdoc.def and cdocsamp.tex with
% cdocsch1.tex, cdocsch2.tex, cdocsdrf.tex, cdocsfn1.tex, cdocsfn2.tex.
%
%<package>\ifdefined\childdocmain\endinput\fi
%<package>\ProvidesFile{childdoc.def}[2018/12/30 v2.0 child document driver]
%<samplemain>\ProvidesFile{cdocsamp.tex}[2018/12/30 v2.0 sample for childdoc]
%<*driver>
%\ProvidesFile{childdoc.drv}[2018/12/30 v2.0 childdoc reference manual file]
\PassOptionsToClass{10pt,a4paper}{article}
\documentclass{ltxdoc}

\usepackage[margin=35mm]{geometry}
\usepackage{hyperref}
\usepackage{hyperxmp}
\usepackage[usenames]{color}

\hypersetup{colorlinks=true}
\hypersetup{pdfstartview=FitH}
\hypersetup{pdfpagemode=UseNone}
\hypersetup{pdfsource={}}
\hypersetup{pdflang={en-UK}}
\hypersetup{pdfcopyright={Copyright 2017-2018 Niklas Beisert.
  This work may be distributed and/or modified under the
  conditions of the LaTeX Project Public License, either version 1.3
  of this license or (at your option) any later version.}}
\hypersetup{pdflicenseurl={http://www.latex-project.org/lppl.txt}}
\hypersetup{pdfcontactaddress={ETH Zurich, ITP, HIT K,
  Wolfgang-Pauli-Strasse 27}}
\hypersetup{pdfcontactpostcode={8093}}
\hypersetup{pdfcontactcity={Zurich}}
\hypersetup{pdfcontactcountry={Switzerland}}
\hypersetup{pdfcontactemail={nbeisert@itp.phys.ethz.ch}}
\hypersetup{pdfcontacturl={http://people.phys.ethz.ch/\xmptilde nbeisert/}}

\newcommand{\secref}[1]{\hyperref[#1]{section \ref*{#1}}}

\parskip1ex
\parindent0pt
\let\olditemize\itemize
\def\itemize{\olditemize\parskip0pt}

\begin{document}

\title{The \textsf{childdoc} Package}
\hypersetup{pdftitle={The childdoc Package}}
\author{Niklas Beisert\\[2ex]
  Institut f\"ur Theoretische Physik\\
  Eidgen\"ossische Technische Hochschule Z\"urich\\
  Wolfgang-Pauli-Strasse 27, 8093 Z\"urich, Switzerland\\[1ex]
  \href{mailto:nbeisert@itp.phys.ethz.ch}
  {\texttt{nbeisert@itp.phys.ethz.ch}}}
\hypersetup{pdfauthor={Niklas Beisert}}
\hypersetup{pdfsubject={Manual for the LaTeX2e Package childdoc}}
\date{30 December 2018, \textsf{v2.0}}
\maketitle

\begin{abstract}\noindent
\textsf{childdoc} is a \LaTeXe{} package
that enables the direct compilation
of document sections included by |\include|
to individual files.
\end{abstract}

\begingroup
\parskip0ex
\tableofcontents
\endgroup

%%%%%%%%%%%%%%%%%%%%%%%%%%%%%%%%%%%%%%%%%%%%%%%%%%%%%%%%%%%%%%%%%%%%%%%%%%%%%%%%
%%%%%%%%%%%%%%%%%%%%%%%%%%%%%%%%%%%%%%%%%%%%%%%%%%%%%%%%%%%%%%%%%%%%%%%%%%%%%%%%
\section{Introduction}

\LaTeX{} provides a mechanism to structure a large document (such as a book)
into a main file and several child files (containing the chapters)
using the |\include| command.
This mechanism is beneficial for documents
which span hundreds of pages in order to
make the source file(s) more manageable.
Moreover, compilation can be restricted to
selected child files by means of the |\includeonly| command.
The latter feature can be used to reduce the compilation time while editing
(this was significantly more useful in the earlier days of \LaTeX{})
or to generate a smaller document which is easier to navigate.
Another application of |\includeonly| is to generate
documents consisting of selected parts of the complete document.

However, there are a few drawbacks of the plain |\include| mechanism:
\begin{itemize}
\item
The child files cannot be compiled on their own,
they can only be compiled via the main file.
A naive editing environment
(such as a text editor with an option
to have the current file processed by \LaTeX)
may require one to switch to the main file before compiling;
attempting to compile the child file produces errors.
\item
The main file must be modified (each time)
to adjust the |\includeonly| command
to the present needs. This easily leaves the main file in a messy state.
\item
The generated document will always carry the filename
of the main document. This is inconvenient if
several child files are to be compiled and
to be kept for distribution.
\end{itemize}

The present package provides a simple interface
to make child files individually compilable by \LaTeX{}.
Compiling a child file then has the same effect as compiling
the main file with an |\includeonly| command
to select the appropriate child.
Moreover the generated document will carry the name of the child
rather than the main file.
This resolves all three above issues.

This feature is meant to make the editing of books,
thesis documents and lecture notes somewhat more convenient.
However, the package can also be used efficiently for
composing a series of documents (such as exercise sheets)
which are typically distributed individually.
It then assists the author in generating the individual documents
(potentially in different versions)
as well as a document containing the collected series.
Another application is in developing style files
or other kinds of included material
where compilation of the style file could redirect
to a sample or test file.

%%%%%%%%%%%%%%%%%%%%%%%%%%%%%%%%%%%%%%%%%%%%%%%%%%%%%%%%%%%%%%%%%%%%%%%%%%%%%%%%
%%%%%%%%%%%%%%%%%%%%%%%%%%%%%%%%%%%%%%%%%%%%%%%%%%%%%%%%%%%%%%%%%%%%%%%%%%%%%%%%
\section{Usage}

First of all, the package \textsf{childdoc} is \emph{not} a standard
\LaTeXe{} |.sty| style file! Therefore it needs to be invoked in
a non-standard way.

%%%%%%%%%%%%%%%%%%%%%%%%%%%%%%%%%%%%%%%%%%%%%%%%%%%%%%%%%%%%%%%%%%%%%%%%%%%%%%%%
\subsection{Included Files}
\label{sec:include}

%%%%%%%%%%%%%%%%%%%%%%%%%%%%%%%%%%%%%%%%
\DescribeMacro{\childdocmain}
To use the package, add the commands
\begin{center}
\begin{tabular}{l}
|\input{childdoc.def}|\\
|\childdocmain{}|\\
\end{tabular}
\end{center}
at the very top of the main \LaTeX{} file,
in particular \emph{before} the |\documentclass| statement!
The argument of |\childdocmain| should be left empty
(but it must be present).

%%%%%%%%%%%%%%%%%%%%%%%%%%%%%%%%%%%%%%%%
\DescribeMacro{\childdocof}
Furthermore, add the commands
\begin{center}
\begin{tabular}{l}
|\input{childdoc.def}|\\
|\childdocof{|\textit{main}|}|\\
\end{tabular}
\end{center}
at the top of every child file \textit{child}
which is included by |\include{|\textit{child}|}|
from within the main file
(or at least for those files to be compiled individually).
The argument \textit{main} must be the filename of the main file.

There are a couple of
considerations in setting up the main and child documents:

%%%%%%%%%%%%%%%%%%%%%%%%%%%%%%%%%%%%%%%%
\paragraph{Restrictions.}

Please note the following restrictions:
\begin{itemize}
\item
|\childdocmain| must be called with one argument \textit{main}
to ensure compatibility with earlier version of the package.
It must either be empty (|\childdocmain{}|)
or precisely match the filename of the main file in which it is specified.
See \secref{sec:detection} for further information.
\item
The filename \textit{main} must be specified without the |.tex| extension.
\item
The filename \textit{main} is case sensitive
(even in case-insensitive file systems)
due to internal string comparison.
\item
The argument \textit{main} should be fully expanded, it cannot be a macro.
\item
Subdirectories and special characters should be avoided in filenames.
\item
The command |\childdocmain{|\textit{main}|}| must be followed by a whitespace.
It should not be followed immediately by another command
or by a comment mark `|%|'.
This is because the \TeX{} parser reads the token immediately following
the argument of |\childdocmain| and puts it
at the beginning of every child section;
however, a white\-space is ignored.
\end{itemize}

%%%%%%%%%%%%%%%%%%%%%%%%%%%%%%%%%%%%%%%%
\paragraph{Content of Main File.}

It is advisable to place all content in the child files included by |\include|.
Any output contained in the main file will appear in all child documents
unless suppressed manually;
it cannot be suppressed automatically by the |\includeonly| directive
and thus should normally be avoided.
A method to include some content in the main file
by means of conditional processing is described in \secref{sec:conditional}.

%%%%%%%%%%%%%%%%%%%%%%%%%%%%%%%%%%%%%%%%
\paragraph{Page Numbering.}

When only a part of the document is compiled,
the appropriate numbering of pages
(as well as other status parameters)
is determined from the |.aux| files.
The latter contain information from previous passes.
However this information needs to propagate through
all intermediate child documents.
Therefore the page numbering in child documents may well
be inconsistent until the complete document is compiled at least once.

A useful (if unconventional) way to always ensure a consistent
page numbering is to restart the numbering in each child document
and denote the pages by `\textit{child}|.|\textit{page}'
where \textit{child} represents the chapter/section number of the child file.
This can be achieved by the command
|\numberwithin{page}{|\textit{child}|}|
of the \textsf{amsmath} package
where \textit{child} can be |chapter| or |section|
depending on the chosen structuring.
Alternatively, one can modify the macro |\thepage| appropriately
and reset the counter |page| at the start of each child file.

%%%%%%%%%%%%%%%%%%%%%%%%%%%%%%%%%%%%%%%%%%%%%%%%%%%%%%%%%%%%%%%%%%%%%%%%%%%%%%%%
\subsection{Conditional Processing}
\label{sec:conditional}

The package provides a mechanism to compile different versions
of a document. To customise the versions further some conditional processing
can come in handy to distinguish which version is being compiled.
The package provides two macros to describe the compilation context:

%%%%%%%%%%%%%%%%%%%%%%%%%%%%%%%%%%%%%%%%
\DescribeMacro{\ifchilddoc}
The conditional |\ifchilddoc| distinguishes between the compilation of
child documents and the main document:
%
\begin{center}
|\ifchilddoc |\textit{child-code}| |[|\||else |\textit{main-code}]| \||fi|
\end{center}

%%%%%%%%%%%%%%%%%%%%%%%%%%%%%%%%%%%%%%%%
\DescribeMacro{\childdocname}
\DescribeMacro{\childdocjob}
The macro |\childdocname| contains the filename (without extension)
of the main or child file being processed.
Note that |\childdocjob| will always contain the name of the main file.

%%%%%%%%%%%%%%%%%%%%%%%%%%%%%%%%%%%%%%%%
\paragraph{Title Page.}

Conditional processing can be used to include a title or banner page
in the main document when proper precautions are taken.
Importantly, the code in the main file should ensure that the page counter
(as well as other status parameters which are stored in the |.aux| files)
takes the same value after the conditional processing.
Otherwise the page numbers may take divergent values
depending on which part is compiled.

For example, a title page could be declared by:
%
\begin{center}
\begin{tabular}{l}
|\ifchilddoc\||else|\\
|\addtocounter{page}{-1}|\\
\textit{code for title page}\\
|\newpage|\\
|\||fi|
\end{tabular}
\end{center}
%
A banner page for the child documents can be generated by:
%
\begin{center}
\begin{tabular}{l}
|\ifchilddoc|\\
|\addtocounter{page}{-1}|\\
\textit{code for banner page}\\
|\newpage|\\
|\||fi|
\end{tabular}
\end{center}
%
Here one could write a message such as:
\begin{center}
|This is the part \childdocname{} of \childdocjob{}.|
\end{center}

%%%%%%%%%%%%%%%%%%%%%%%%%%%%%%%%%%%%%%%%%%%%%%%%%%%%%%%%%%%%%%%%%%%%%%%%%%%%%%%%
\subsection{Flags}
\label{sec:flags}

The package makes it easy to generate different versions
of the main or child documents.
To this end compilation flags can be defined
and assigned different default values.
They will be particularly useful in conjunction
with the forwarding mechanism described in \secref{sec:forward}.

For example, it may be useful to have a flag |\version|
which can be set to |draft| or |final|.
The document source will contain some conditional code
depending on the value of |\version|.
Suppose further, the flag should default to |final| for the main file
and to |draft| for child files
which is a natural assignment for editing the document.
This is achieved by placing the following code
in the preamble of the main document
(below the |\childdocmain| directive):
%
\begin{center}
\begin{tabular}{l}
|\ifchilddoc|\\
|\providecommand{\version}{draft}|\\
|\||else|\\
|\providecommand{\version}{final}|\\
|\||fi|
\end{tabular}
\end{center}
%
The definition by |\providecommand| makes sure
that previous definitions are not overwritten.
Further statements |\providecommand{\version}{...}|
can thus be added before the above code to override it.

For the main file, one might add a line
(between |\childdocmain| and the above block)
%
\begin{center}
|%\ifchilddoc\||else\providecommand{\version}{draft}\||fi|
\end{center}
%
which can be uncommented to produce a draft version.
Likewise one can add a line to the very top of a child file
(above the |\childdocof{|\textit{main}|}| directive)
%
\begin{center}
|%\providecommand{\version}{final}|
\end{center}
%
which can be uncommented to produce the final version of this child document.

%%%%%%%%%%%%%%%%%%%%%%%%%%%%%%%%%%%%%%%%%%%%%%%%%%%%%%%%%%%%%%%%%%%%%%%%%%%%%%%%
\subsection{Forwarding}
\label{sec:forward}

Different versions of the main or child documents
using compilation flags as described in \secref{sec:flags}
can be (permanently) stored in different files
for convenient compilation, viewing and distribution.
To this end, the package defines a command
to pass on compilation to a different file:

%%%%%%%%%%%%%%%%%%%%%%%%%%%%%%%%%%%%%%%%
\DescribeMacro{\childdocforward}
The command |\childdocforward| redirects processing to
another source file:
%
\begin{center}
\begin{tabular}{l}
|\input{childdoc.def}|\\
|\childdocforward[|\textit{main}|]{|\textit{dest}|}|\\
\end{tabular}
\end{center}
%
The argument \textit{dest} is the destination file
(without extension).
It should be the main file or one of the child files.
Note that further \textsf{childdoc} directives
such as |\childdocof| and |\childdocforward|
in the indicated file will be processed in this form.
The optional argument \textit{main}
passes on directly to the main file \textit{main}
while pretending to compile the child \textit{dest}.
This form behaves as if \textit{dest}
issues |\childdocof{|\textit{main}|}| right away,
and no further \textsf{childdoc} directives will be processed.

%%%%%%%%%%%%%%%%%%%%%%%%%%%%%%%%%%%%%%%%
\DescribeMacro{\...prefix}
In the alternative form |\childdocforwardprefix|,
%
\begin{center}
\begin{tabular}{l}
|\input{childdoc.def}|\\
|\childdocforwardprefix[|\textit{main}|]{|\textit{prefix}|}{|\textit{dest}|}|
\end{tabular}
\end{center}
%
the destination file is determined by a pattern
depending on the current file:
To make this work, the current file must be called
`{\textit{prefix}\hspace{0.2em}\textit{suffix}}'
with \textit{prefix} matching precisely the argument.
Processing is then passed on to the file
`{\textit{dest}\hspace{0.2em}\textit{suffix}}'.
Surely, the same effect is achieved by
directly specifying the
argument `{\textit{dest}\hspace{0.2em}\textit{suffix}}'
in the first form.
However, that requires to set up a different file
for each child. With the alternative form of the command
all these files can have exactly the same content
which simplifies setting them up and maintaining them.

For example, the following file |draft.tex|
with a compilation flag |\version| as described in \secref{sec:flags}
compiles the main document as a draft:
%
\begin{center}
\begin{tabular}{l}
|\def\version{draft}|\\
|\input{childdoc.def}|\\
|\childdocforward{|\textit{main}|}|
\end{tabular}
\end{center}
%
Likewise, the following files |final|\textit{nn}|.tex|
compile the final version of the child document
|child|\textit{nn}|.tex|:
%
\begin{center}
\begin{tabular}{l}
|\def\version{final}|\\
|\input{childdoc.def}|\\
|\childdocforwardprefix{final}{child}|
\end{tabular}
\end{center}
%

Note that when several versions of a main file and/or of each child file
are to be generated, it may be convenient to set up a |Makefile| or
shell script to automatise the process.

%%%%%%%%%%%%%%%%%%%%%%%%%%%%%%%%%%%%%%%%%%%%%%%%%%%%%%%%%%%%%%%%%%%%%%%%%%%%%%%%
\subsection{Command Line Processing}
\label{sec:commandline}

The effect of redirection files can also be achieved by invoking
the \LaTeX{} compiler with a more elaborate command line.
Most conveniently this should be done as part
of a shell script or a |Makefile|.

When using \textsf{childdoc} in the main file, the following
command lines effectively perform a redirection
(note that depending on the shell being used,
backslashes may have to be doubled: `|\|' $\to$ `|\\|'):
%
\begin{center}
|... -jobname "|\textit{target}|" |\\|"|[\textit{flags}]%
|\input{childdoc.def}\childdocforward[|\textit{main}|]{|\textit{dest}|}"|
\end{center}
%
Here \textit{target} is the name of the output file,
\textit{main} is the name of the main file
and \textit{dest} is the name of the main or child file to be processed
(all filenames without extensions).
The optional argument \textit{main} can be omitted
if \textit{main} matches \textit{dest}.
Optionally, compilation \textit{flags} can be defined via |\def| commands.
This command line makes the \TeX{} engine believe
it is compiling the file \textit{target}
whose content is specified as the latter parameter.
The provided code then forwards the processing to
\textit{main} or \textit{dest} as described in \secref{sec:forward}.

%%%%%%%%%%%%%%%%%%%%%%%%%%%%%%%%%%%%%%%%%%%%%%%%%%%%%%%%%%%%%%%%%%%%%%%%%%%%%%%%
\subsection{Include by Input}
\label{sec:input}

Including child documents by |\include| has some restrictions by design.
Most notably, the content of a child document always occupies
its own set of pages; pages cannot be shared between child documents.
Usually, this behaviour makes perfect sense
because each child document contain an essential part of the document.
However, in some situations it may be desirable to compose
a document from a collection of parts
without having mandatory page breaks between then.
For this case, the package
provides a mechanism to include parts
by |\input| which can also be processed individually.
However, by construction this mechanism
requires manual handling of the content to be output.

%%%%%%%%%%%%%%%%%%%%%%%%%%%%%%%%%%%%%%%%
\DescribeMacro{\ifchilddocmanual}
The main file should be prepared as usual, see \secref{sec:include}.
However, the document body must make a distinction
between processing of an individual part and of the main document, e.g.:
%
\begin{center}
\begin{tabular}{l}
|\ifchilddocmanual|\\
|\input{\childdocname}|\\
|\||else|\\
\textit{document body with }|\input{|\textit{part}|}|\\
|\||fi|
\end{tabular}
\end{center}
%
The conditional |\ifchilddocmanual| is true whenever
a part to be included by |\input| is being compiled,
and the name of the part is stored in |\childdocname|.

%%%%%%%%%%%%%%%%%%%%%%%%%%%%%%%%%%%%%%%%
\DescribeMacro{\childdocby}
Each part to be included by |\input| should start with:
%
\begin{center}
\begin{tabular}{l}
|\input{childdoc.def}|\\
|\childdocby{|\textit{main}|}|\\
\end{tabular}
\end{center}
%
The directive |\childdocby| is similar to |\childdocof|
described in \secref{sec:include},
but the subsequent selection of content must be done manually.
To that end, both |\ifchilddoc| and |\ifchilddocmanual|
will be true upon processing of a part,
and the name of the part is stored in |\childdocname|.
Note that |\jobname| will be set to the filename of the current part
so that each part receives an individual |.aux| file
that does not interfere with the |.aux| file(s) of the main document.
This behaviour can be altered by the alternative form
|\childdocby[*]{|\textit{main}|}| (with a non-empty optional argument)
which uses the |.aux| file of the main document
by setting |\jobname| to \textit{main}.

%%%%%%%%%%%%%%%%%%%%%%%%%%%%%%%%%%%%%%%%%%%%%%%%%%%%%%%%%%%%%%%%%%%%%%%%%%%%%%%%
\subsection{Driver Development}
\label{sec:driver}

The \textsf{childdoc} mechanism can also be use for the development
of definition files such as \LaTeX{} styles or classes.
This case differs from the above setup with multiple parts
included by |\include| in that no |\includeonly| should be invoked.
This can be achieved by starting the include file
(before |\ProvidesPackage|) with:
%
\begin{center}
\begin{tabular}{l}
|\input{childdoc.def}|\\
|\childdocforward{|\textit{main}|}|\\
\end{tabular}
\end{center}
%
or alternatively with:
%
\begin{center}
\begin{tabular}{l}
|\input{childdoc.def}|\\
|\childdocby{|\textit{main}|}|\\
\end{tabular}
\end{center}
%
Both forms have slightly different effects as described above.
The main file is prepared as usual, see \secref{sec:include}.

%%%%%%%%%%%%%%%%%%%%%%%%%%%%%%%%%%%%%%%%%%%%%%%%%%%%%%%%%%%%%%%%%%%%%%%%%%%%%%%%
\subsection{Legacy Detection}
\label{sec:detection}

The directive |\childdocmain| in the main file can detect
whether the complete document or merely a child is to be compiled
even without using the directive |\childdocof|.
This method is deprecated because it is less robust
and there is no compelling reason to use it;
it is merely provided for backward compatibility
and it may be removed in future versions.

If the detection mechanism is to be used,
it is mandatory to correctly specify
the filename of the main file as the argument of |\childdocmain|:
%
\begin{center}
\begin{tabular}{l}
|\input{childdoc.def}|\\
|\childdocmain{|\textit{main}|}|\\
\end{tabular}
\end{center}
%
If |\jobname| does not match the argument \textit{main} of |\childdocmain|,
it is assumed that |\jobname| points to the child file to be compiled.
When using |\childdocmain| with the main file specified as argument,
it suffices to start a child file
with just |\input{|\textit{main}|}|
without loading of the package and using |\childdocof|.
If instead all processing is done
with the appropriate \textsf{childdoc} directives,
the argument of \textit{main} of |\childdocmain| can be empty.

An alternative version of the command line processing described
in \secref{sec:commandline} using the detection mechanism reads:
%
\begin{center}
|... -jobname "|\textit{target}|" "|[\textit{flags}]%
[|\def\jobname{|\textit{dest}|}|]|\input{|\textit{main}|}"|
\end{center}

%%%%%%%%%%%%%%%%%%%%%%%%%%%%%%%%%%%%%%%%%%%%%%%%%%%%%%%%%%%%%%%%%%%%%%%%%%%%%%%%
\subsection{Manual Code}
\label{sec:manual}

In case one cannot be certain whether the definitions file |childdoc.def|
is installed on the target \TeX{} distribution
and one prefers not to ship it,
it is conceivable to paste a few relevant commands into the sources.

To that end, drop all statements |\input{childdoc.def}|
and perform the replacements as outlined below.
Instead of |\childdocmain{|\textit{main}|}| add the following code
to the top of the main file:
%
\begin{center}
\begin{tabular}{l}
|\||ifdefined\childdocname\endinput\||fi\newif\ifchilddoc|\\
|\edef\childdocname{\scantokens\expandafter{\jobname\noexpand}}|\\
|\def\childdocmain{|\textit{main}|}\||ifx\childdocmain\childdocname\||else|\\
|\childdoctrue\includeonly{\childdocname}\let\jobname\childdocmain\||fi|\\
\end{tabular}
\end{center}
%
Instead of |\childdocof{|\textit{main}|}| just include the main file
at the top of each child file:
%
\begin{center}
|\input{|\textit{main}|}|
\end{center}
%
A simple redirection |\childdocforward{|\textit{dest}|}| is achieved by:
%
\begin{center}
|\def\jobname{|\textit{dest}|}\input{\jobname}|
\end{center}
%
The redirection with prefix
|\childdocforwardprefix[|\textit{prefix}|]{|\textit{dest}|}|
is accomplished by:
%
\begin{center}
\begin{tabular}{l}
|{\edef\jobname{\scantokens\expandafter{\jobname\noexpand}}|\\
|\def\redirectjob |\textit{prefix}|#1~~~{\gdef\jobname{|\textit{dest}|#1}}|\\
|\expandafter\redirectjob\jobname~~~}\input{\jobname}|
\end{tabular}
\end{center}

In an alternative approach,
child documents can be compiled by a specific command line
without additional code or specific definitions:
%
\begin{center}
|... -jobname "|\textit{target}|" "|[\textit{flags}]%
|\includeonly{|\textit{dest}|}\input{|\textit{main}|}"|
\end{center}
%

%%%%%%%%%%%%%%%%%%%%%%%%%%%%%%%%%%%%%%%%%%%%%%%%%%%%%%%%%%%%%%%%%%%%%%%%%%%%%%%%
%%%%%%%%%%%%%%%%%%%%%%%%%%%%%%%%%%%%%%%%%%%%%%%%%%%%%%%%%%%%%%%%%%%%%%%%%%%%%%%%
\section{Information}

%%%%%%%%%%%%%%%%%%%%%%%%%%%%%%%%%%%%%%%%%%%%%%%%%%%%%%%%%%%%%%%%%%%%%%%%%%%%%%%%
\subsection{Copyright}

Copyright \copyright{} 2017--2018 Niklas Beisert

This work may be distributed and/or modified under the
conditions of the \LaTeX{} Project Public License, either version 1.3
of this license or (at your option) any later version.
The latest version of this license is in
  \url{http://www.latex-project.org/lppl.txt}
and version 1.3 or later is part of all distributions of \LaTeX{}
version 2005/12/01 or later.

This work has the LPPL maintenance status `maintained'.

The Current Maintainer of this work is Niklas Beisert.

This work consists of the files |README.txt|, |childdoc.ins| and |childdoc.dtx|
as well as the derived files |childdoc.def|, |cdocsamp.tex|
with |cdocsch1.tex|, |cdocsch2.tex|, |cdocspt3.tex|, |cdocspt4.tex|,
|cdocsdrf.tex|, |cdocsfn1.tex|, |cdocsfn2.tex|
as well as |childdoc.pdf|.

%%%%%%%%%%%%%%%%%%%%%%%%%%%%%%%%%%%%%%%%%%%%%%%%%%%%%%%%%%%%%%%%%%%%%%%%%%%%%%%%
\subsection{Files and Installation}

The package consists of the files:
%
\begin{center}
\begin{tabular}{ll}
    |README.txt|   & readme file \\
    |childdoc.ins| & installation file \\
    |childdoc.dtx| & source file \\
    |childdoc.def| & definition file \\
    |cdocsamp.tex| & sample main file \\
    |cdocsch1.tex| & sample include file \\
    |cdocsch2.tex| & sample include file \\
    |cdocspt3.tex| & sample part file \\
    |cdocspt4.tex| & sample part file \\
    |cdocsdrf.tex| & sample redirection file \\
    |cdocsfn1.tex| & sample redirection file \\
    |cdocsfn2.tex| & sample redirection file \\
    |childdoc.pdf| & manual
\end{tabular}
\end{center}
%
The distribution consists of the files
|README.txt|, |childdoc.ins| and |childdoc.dtx|.
%
\begin{itemize}
\item
Run (pdf)\LaTeX{} on |childdoc.dtx|
to compile the manual |childdoc.pdf| (this file).
\item
Run \LaTeX{} on |childdoc.ins| to create the definitions file |childdoc.def|
and the sample |cdocsamp.tex| with include files
|cdocsch1.tex|, |cdocsch2.tex|, |cdocspt3.tex|, |cdocspt4.tex|,
|cdocsdrf.tex|, |cdocsfn1.tex|, |cdocsfn2.tex|.
Then copy the file |childdoc.def| to an appropriate directory of your \LaTeX{}
distribution, e.g.\ \textit{texmf-root}|/tex/latex/childdoc|.
\end{itemize}

%%%%%%%%%%%%%%%%%%%%%%%%%%%%%%%%%%%%%%%%%%%%%%%%%%%%%%%%%%%%%%%%%%%%%%%%%%%%%%%%
\subsection{Related CTAN Packages}

There are several other packages which offer a similar functionality:
%
\begin{itemize}
\item
The packages
\href{http://ctan.org/pkg/docmute}{\textsf{docmute}},
\href{http://ctan.org/pkg/includex}{\textsf{includex}} and
\href{http://ctan.org/pkg/standalone}{\textsf{standalone}}
provide commands to include only the document body of
a child file thus allowing both files to be compiled individually.
\item
The packages \href{http://ctan.org/pkg/subdocs}{\textsf{subdocs}}
and \href{http://ctan.org/pkg/subfiles}{\textsf{subfiles}}
provide structures in which the main and child documents can be
encapsulated and allowing them to be compiled individually.
The inclusion mechanism is different from the conventional |\include|.
\item
The package \href{http://ctan.org/pkg/combine}{\textsf{combine}}
is an elaborate solution to combine several documents into one.
\end{itemize}
%
See also the CTAN topic \href{http://ctan.org/topic/subdocs}{\textsf{subdocs}}
for further related packages.
The present package differs from the above solutions in that
a document structure constructed with the conventional |\include| mechanism
just needs two extra commands at the top of every file
such that all constituent files can be compiled individually.

%%%%%%%%%%%%%%%%%%%%%%%%%%%%%%%%%%%%%%%%%%%%%%%%%%%%%%%%%%%%%%%%%%%%%%%%%%%%%%%%
%\subsection{Feature Suggestions}
%
%The following is a list of features which may be useful for future
%versions of this package:
%%
%\begin{itemize}
%\item
%\ldots
%\end{itemize}

%%%%%%%%%%%%%%%%%%%%%%%%%%%%%%%%%%%%%%%%%%%%%%%%%%%%%%%%%%%%%%%%%%%%%%%%%%%%%%%%
\subsection{Revision History}

%%%%%%%%%%%%%%%%%%%%%%%%%%%%%%%%%%%%%%%%
\paragraph{v2.0:} 2018/12/30

\begin{itemize}
\item
immediate forward processing
\item
added |\childdocby| mechanism
\item
manual restructured
\end{itemize}

%%%%%%%%%%%%%%%%%%%%%%%%%%%%%%%%%%%%%%%%
\paragraph{v1.6:} 2018/01/17

\begin{itemize}
\item
application for development of include files
\item
corrections to manual
\end{itemize}

%%%%%%%%%%%%%%%%%%%%%%%%%%%%%%%%%%%%%%%%
\paragraph{v1.5:} 2017/05/21

\begin{itemize}
\item
more complete structuring introduced
\item
|\childdocof| introduced
\item
|\childdoc| renamed to |\childdocmain|
\item
|\childredirect| renamed to |\childdocforward| and |\childdocforwardprefix|
and functionality expanded
\end{itemize}

%%%%%%%%%%%%%%%%%%%%%%%%%%%%%%%%%%%%%%%%
\paragraph{v1.0:} 2017/04/27

\begin{itemize}
\item
manual and install package
\item
first version published on CTAN
\end{itemize}

%%%%%%%%%%%%%%%%%%%%%%%%%%%%%%%%%%%%%%%%
\paragraph{v0.6:} 2017/04/26

\begin{itemize}
\item
redirection mechanism added
\end{itemize}

%%%%%%%%%%%%%%%%%%%%%%%%%%%%%%%%%%%%%%%%
\paragraph{v0.5:} 2017/04/26

\begin{itemize}
\item
functionality in definition file
\end{itemize}


%%%%%%%%%%%%%%%%%%%%%%%%%%%%%%%%%%%%%%%%%%%%%%%%%%%%%%%%%%%%%%%%%%%%%%%%%%%%%%%%
%%%%%%%%%%%%%%%%%%%%%%%%%%%%%%%%%%%%%%%%%%%%%%%%%%%%%%%%%%%%%%%%%%%%%%%%%%%%%%%%
%%%%%%%%%%%%%%%%%%%%%%%%%%%%%%%%%%%%%%%%%%%%%%%%%%%%%%%%%%%%%%%%%%%%%%%%%%%%%%%%
\appendix

\settowidth\MacroIndent{\rmfamily\scriptsize 000\ }

 \DocInput{childdoc.dtx}

\end{document}
%</driver>
% \fi
%
% %%%%%%%%%%%%%%%%%%%%%%%%%%%%%%%%%%%%%%%%%%%%%%%%%%%%%%%%%%%%%%%%%%%%%%%%%%%%%%
% %%%%%%%%%%%%%%%%%%%%%%%%%%%%%%%%%%%%%%%%%%%%%%%%%%%%%%%%%%%%%%%%%%%%%%%%%%%%%%
% \section{Sample}
%\iffalse
%<*samplemain>
%\fi
%
% The following presents a sample document
% with two chapters, two parts, a title page,
% a compile flag as well as three forwarding files to set the flag.
% It consists of eight |.tex| files:
% \begin{center}
% \begin{tabular}{ll}
% |cdocsamp.tex|&main file\\
% |cdocsch1.tex|&include file for chapter 1\\
% |cdocsch2.tex|&include file for chapter 2\\
% |cdocspt3.tex|&include file for part 3\\
% |cdocspt4.tex|&include file for part 4\\
% |cdocsdrf.tex|&forwarding file for main file in draft mode\\
% |cdocsfi1.tex|&forwarding file for final version of chapter 1\\
% |cdocsfi2.tex|&forwarding file for final version of chapter 2\\
% \end{tabular}
% \end{center}
% Each of the eight files can be compiled directly by the \LaTeX{} compiler.
%
% %%%%%%%%%%%%%%%%%%%%%%%%%%%%%%%%%%%%%%
% \paragraph{Main File.}
%
% The main file is called |cdocsamp.tex|.
%
% Load the \textsf{childdoc} definitions and
% declare the filename for the main document:
%    \begin{macrocode}
\input{childdoc.def}
\childdocmain{}
%    \end{macrocode}

% Optional override for |\version| flag:
%    \begin{macrocode}
%%\ifchilddoc\else\providecommand{\version}{draft}\fi
%    \end{macrocode}

% Define the default values for the |\version| flag
% (|final| for the main file and |draft| for childs):
%    \begin{macrocode}
\ifchilddoc
\providecommand{\version}{draft}
\else
\providecommand{\version}{final}
\fi
%    \end{macrocode}

% Load the standard document class:
%    \begin{macrocode}
\documentclass[12pt]{article}
%    \end{macrocode}

% Start the document body:
%    \begin{macrocode}
\begin{document}
%    \end{macrocode}

% Declare a title page.
% Print title, part of document being processed and version flag:
%    \begin{macrocode}
\addtocounter{page}{-1}
\begin{center}
{\LARGE\bfseries{}childdoc example\par}
\vspace{1cm}
\ifchilddoc
\ifchilddocmanual part\else chapter\fi:
`\childdocname' of `\childdocjob'\par
\else
main document: `\childdocjob'\par
\fi
version: \version\par
\end{center}
\newpage
%    \end{macrocode}

% Manually include selected file,
% otherwise process as usual:
%    \begin{macrocode}
\ifchilddocmanual
\section*{part `\childdocname'}
\input{\childdocname}
\else
%    \end{macrocode}

% Include the two chapters:
%    \begin{macrocode}
\include{cdocsch1}
\include{cdocsch2}
%    \end{macrocode}

% Include the two parts unless only chapters should be displayed:
%    \begin{macrocode}
\ifchilddoc\else
\section{part three}
\input{cdocspt3}
\section{part four}
\input{cdocspt4}
\fi
%    \end{macrocode}

% Process as usual until here:
%    \begin{macrocode}
\fi
%    \end{macrocode}

% End of document body:
%    \begin{macrocode}
\end{document}
%    \end{macrocode}
%\iffalse
%</samplemain>
%\fi
%
% %%%%%%%%%%%%%%%%%%%%%%%%%%%%%%%%%%%%%%
% \paragraph{Chapter Include Files.}
%
% The include files are called |cdocsch1.tex| and |cdocsch2.tex|.
%
%\iffalse
%<*samplechap1|samplechap2>
%\fi

% Optional override for |\version| flag:
%    \begin{macrocode}
%%\providecommand{\version}{final}
%    \end{macrocode}

% Include the main document:
%    \begin{macrocode}
\input{childdoc.def}
\childdocof{cdocsamp}
%    \end{macrocode}

%\iffalse
%</samplechap1|samplechap2>
%\fi
%
%\iffalse
%<*samplechap1>
%\fi
% Some text for chapter 1:
%    \begin{macrocode}
\section{one}
some text in chapter one
%    \end{macrocode}

%\iffalse
%</samplechap1>
%\fi
% Some text for chapter 2:
%\iffalse
%<*samplechap2>
%\fi
%    \begin{macrocode}
\section{two}
more text in chapter two
%    \end{macrocode}

%\iffalse
%</samplechap2>
%\fi
%
% %%%%%%%%%%%%%%%%%%%%%%%%%%%%%%%%%%%%%%
% \paragraph{Part Include Files.}
%
% The include files are called |cdocspt3.tex| and |cdocspt4.tex|.
%
%\iffalse
%<*samplepart3|samplepart4>
%\fi

% Optional override for |\version| flag:
%    \begin{macrocode}
%%\providecommand{\version}{final}
%    \end{macrocode}

% Include the main document:
%    \begin{macrocode}
\input{childdoc.def}
\childdocby{cdocsamp}
%    \end{macrocode}

%\iffalse
%</samplepart3|samplepart4>
%\fi
%
%\iffalse
%<*samplepart3>
%\fi
% Some text for part 3:
%    \begin{macrocode}
some text in part three
%    \end{macrocode}

%\iffalse
%</samplepart3>
%\fi
% Some text for part 4:
%\iffalse
%<*samplepart4>
%\fi
%    \begin{macrocode}
more text in part four
%    \end{macrocode}

%\iffalse
%</samplepart4>
%\fi
%
% %%%%%%%%%%%%%%%%%%%%%%%%%%%%%%%%%%%%%%
% \paragraph{Forwarding for a Complete Draft.}
%
% The following forwarding file |cdocsdrf.tex|
% compiles the main document in draft mode:
%\iffalse
%<*sampledraft>
%\fi
%    \begin{macrocode}
\def\version{draft}
\input{childdoc.def}
\childdocforward{cdocsamp}
%    \end{macrocode}

%\iffalse
%</sampledraft>
%\fi
%
% %%%%%%%%%%%%%%%%%%%%%%%%%%%%%%%%%%%%%%
% \paragraph{Forwarding for Final Version of the Chapters.}
%
% The following forwarding files |cdocsfn1.tex| and |cdocsfn2.tex|
% (with identical content)
% compile the final versions of the child documents
% |cdocsch1.tex| and |cdocsch2.tex|, respectively:
%\iffalse
%<*samplefinal>
%\fi
%    \begin{macrocode}
\def\version{final}
\input{childdoc.def}
\childdocforwardprefix[cdocsamp]{cdocsfn}{cdocsch}
%    \end{macrocode}

%\iffalse
%</samplefinal>
%\fi
%
% %%%%%%%%%%%%%%%%%%%%%%%%%%%%%%%%%%%%%%
% \paragraph{Command Line Processing.}
%
% The following three command lines generate the output files
% |cdocscld|, |cdocscl1| and |cdocscl2|
% which should be identical to
% |cdocsdrf|, |cdocsch1| and |cdocsfn2|, respectively:
% \begin{center}
% \begin{tabular}{l}
% |latex -jobname cdocscld \|\\
% |  "\def\version{draft}\input{childdoc.def}\childdocforward{cdocsamp}"|\\
% |latex -jobname cdocscl1 \|\\
% |  "\input{childdoc.def}\childdocforward[cdocsamp]{cdocsch1}"|\\
% |latex -jobname cdocscl2 \|\\
% |  "\def\version{final}\input{childdoc.def}\childdocforward{cdocsch2}"|
% \end{tabular}
% \end{center}
% Note that the trailing backslash on each first line
% merely continues the input to the second line
% (for convenient cut ant paste).
% Furthermore, the command |latex| can be replaced by any
% of its alternative versions such as |pdflatex|.
%
% %%%%%%%%%%%%%%%%%%%%%%%%%%%%%%%%%%%%%%%%%%%%%%%%%%%%%%%%%%%%%%%%%%%%%%%%%%%%%%
% %%%%%%%%%%%%%%%%%%%%%%%%%%%%%%%%%%%%%%%%%%%%%%%%%%%%%%%%%%%%%%%%%%%%%%%%%%%%%%
% \section{Implementation}
%\iffalse
%<*package>
%\fi
%
% This section describes the definitions file |childdoc.def|.

% The definitions cannot be loaded using |\usepackage| or |\RequirePackage|
% which has a mechanism to prevent loading a style file more than once.
% When loading the definitions by means of |\input|
% multiple instances have to be prevented manually:
%\iffalse
%This code needs to be before the `\ProvidesFile' directive
%which is defined at the beginning of this file.
%Therefore it is also placed there and commented out here.
%</package>
%<*discard>
%\fi
%    \begin{macrocode}
\ifdefined\childdocmain\endinput\fi
%    \end{macrocode}
%\iffalse
%</discard>
%<*package>
%\fi
%
% \macro{\ifchilddoc}
% \macro{\ifchilddocmanual}
% The conditional |\ifchilddoc| tells whether a
% child (true) or main (false) document is being compiled.
% The conditional |\ifchilddocmanual| tells whether
% the |\includeonly| mechanism is used (false) or
% the selection of child files must be performed manually (true).
% The definitions initialise to false:
%    \begin{macrocode}
\newif\ifchilddoc
\newif\ifchilddocmanual
%    \end{macrocode}

% \macro{\childdocname}
% \macro{\childdocjob}
% The macro |\childdocname| stores the name of the main document
% to be compiled. The macro |\childdocjob| stores the name of
% the document on which the \LaTeX{} compiler was originally invoked.
% The content of |\jobname| cannot be compared
% to filenames specified in the source due to different catcodes.
% The following code rescans |\jobname|, stores the result
% in |\childdocname| and saves a copy in |\childdocjob|:
%    \begin{macrocode}
\edef\childdocname{\scantokens\expandafter{\jobname\noexpand}}
\let\childdocjob\childdocname
%    \end{macrocode}

% \macro{\childdocdisable}
% The macro |\childdocdisable| prevents the main file
% from being processed more than once.
% At this stage, the main document command |\childdocmain|
% is assumed to be called once again where it should do nothing.
% Any subsequent call to it should prevent
% a secondary processing of the main document
% It overwrites the forwarding commands
% |\childdocof| and |\childdocforward|
% with empty macros to prevent further inclusions of the main document:
%    \begin{macrocode}
\newcommand{\childdocdisable}
{
  \renewcommand{\childdocmain}[1]{\renewcommand{\childdocmain}[1]{\endinput}}
  \renewcommand{\childdocof}[1]{}
  \renewcommand{\childdocby}[2][]{}
  \renewcommand{\childdocforward}[2][]{}
  \renewcommand{\childdocdisable}{}
}
%    \end{macrocode}

% \macro{\childdocmain}
% The macro |\childdocmain| is to be called at the top of the main file
% with nothing or the main filename (without extension) as argument.
% First, it breaks loops.
% If the argument is not empty and does not match |\childdocname|
% (which is set by the first inclusion of |childdoc.def|),
% |\ifchilddoc| is set to true, |\includeonly| is applied to the child file
% and |\jobname| is set to the main file
% (for proper handling of |.aux| files):
%    \begin{macrocode}
\newcommand{\childdocmain}[1]
{
  \childdocdisable\childdocmain{}
  \if?#1?\else
    \begingroup
      \def\childdoctmp{#1}
      \ifx\childdoctmp\childdocname
        \def\childdoctmp{}
      \else
        \def\childdoctmp
        {
          \childdoctrue
          \includeonly{\childdocname}
          \def\childdocjob{#1}
          \def\jobname{#1}
        }
      \fi
      \expandafter
    \endgroup
    \childdoctmp
  \fi
}
%    \end{macrocode}

% \macro{\childdocof}
% The command |\childdocof| redirects
% compilation to the main file |#1|.
%    \begin{macrocode}
\newcommand{\childdocof}[1]
{
  \childdocdisable
  \childdoctrue
  \includeonly{\childdocname}
  \def\jobname{#1}
  \def\childdocjob{#1}
  \input{#1}
}
%    \end{macrocode}

% \macro{\childdocby}
% The command |\childdocby| ....
%    \begin{macrocode}
\newcommand{\childdocby}[2][]
{
  \childdocdisable
  \childdoctrue
  \childdocmanualtrue
  \if?#1?\else
    \def\jobname{#2}
  \fi
  \def\childdocjob{#2}
  \input{#2}
  \endinput
}
%    \end{macrocode}

% \macro{\childdocforward}
% The command |\childdocforward| redirects
% compilation to the main file or
% (if the optional argument is given) a child file.
% Parameters are set as if the main file
% or a child file starting with |\childdocof| was compiled.
% Then compilation is handed over to the main file:
%    \begin{macrocode}
\newcommand{\childdocforward}[2][]
{
  \begingroup
    \if?#1?
      \def\childdoctmp
      {
        \def\childdocname{#2}
        \def\childdocjob{#2}
        \def\jobname{#2}
        \input{#2}
        \endinput
      }
    \else
      \def\childdoctmp
      {
        \childdocdisable
        \def\childdocname{#2}
        \childdoctrue
        \includeonly{#2}
        \def\childdocjob{#1}
        \def\jobname{#1}
        \input{#1}
        \endinput
      }
    \fi
    \expandafter
  \endgroup
  \childdoctmp
}
%    \end{macrocode}

% \macro{\childdocforwardprefix}
% The command |\childdocforwardprefix| redirects
% compilation to the main or a child file by means of a pattern.
% The prefix |#1| in the current filename is replaced by |#2|
% and the suffix of the current filename is kept
% (it is assumed that the filename does not contain the substring `|~~~|'
% which is used as a delimiter).
% Compilation is handed over to the new file by |\childdocforward|:
%    \begin{macrocode}
\newcommand{\childdocforwardprefix}[3][]
{
  \begingroup
    \def\childdocextract #2##1~~~{\def\childdoctmp{\childdocforward[#1]{#3##1}}}
    \expandafter\childdocextract\childdocname~~~
    \expandafter
  \endgroup
  \childdoctmp
}
%    \end{macrocode}

% \macro{\childdoc}
% The deprecated macro |\childdoc| is a legacy version of |\childdocmain|:
%    \begin{macrocode}
\newcommand{\childdoc}{\childdocmain}
%    \end{macrocode}

% \macro{\childdocredirect}
% The deprecated macro |\childdocredirect| is a legacy version
% of |\childdocforward| and |\childdocforwardprefix|:
%    \begin{macrocode}
\newcommand{\childdocredirect}[2][]
{
  \begingroup
    \if?#1?
      \def\childdoctmp{\childdocforward{#2}}
    \else
      \def\childdoctmp{\childdocforwardprefix{#1}{#2}}
    \fi
    \expandafter
  \endgroup
  \childdoctmp
}
%    \end{macrocode}

%\iffalse
%</package>
%\fi
%
\endinput
|\\
|\childdocmain{}|\\
\end{tabular}
\end{center}
at the very top of the main \LaTeX{} file,
in particular \emph{before} the |\documentclass| statement!
The argument of |\childdocmain| should be left empty
(but it must be present).

%%%%%%%%%%%%%%%%%%%%%%%%%%%%%%%%%%%%%%%%
\DescribeMacro{\childdocof}
Furthermore, add the commands
\begin{center}
\begin{tabular}{l}
|% \iffalse
%
% childdoc.dtx Copyright (C) 2017-2018 Niklas Beisert
%
% This work may be distributed and/or modified under the
% conditions of the LaTeX Project Public License, either version 1.3
% of this license or (at your option) any later version.
% The latest version of this license is in
%   http://www.latex-project.org/lppl.txt
% and version 1.3 or later is part of all distributions of LaTeX
% version 2005/12/01 or later.
%
% This work has the LPPL maintenance status `maintained'.
%
% The Current Maintainer of this work is Niklas Beisert.
%
% This work consists of the files childdoc.dtx and childdoc.ins
% and the derived files childdoc.def and cdocsamp.tex with
% cdocsch1.tex, cdocsch2.tex, cdocsdrf.tex, cdocsfn1.tex, cdocsfn2.tex.
%
%<package>\ifdefined\childdocmain\endinput\fi
%<package>\ProvidesFile{childdoc.def}[2018/12/30 v2.0 child document driver]
%<samplemain>\ProvidesFile{cdocsamp.tex}[2018/12/30 v2.0 sample for childdoc]
%<*driver>
%\ProvidesFile{childdoc.drv}[2018/12/30 v2.0 childdoc reference manual file]
\PassOptionsToClass{10pt,a4paper}{article}
\documentclass{ltxdoc}

\usepackage[margin=35mm]{geometry}
\usepackage{hyperref}
\usepackage{hyperxmp}
\usepackage[usenames]{color}

\hypersetup{colorlinks=true}
\hypersetup{pdfstartview=FitH}
\hypersetup{pdfpagemode=UseNone}
\hypersetup{pdfsource={}}
\hypersetup{pdflang={en-UK}}
\hypersetup{pdfcopyright={Copyright 2017-2018 Niklas Beisert.
  This work may be distributed and/or modified under the
  conditions of the LaTeX Project Public License, either version 1.3
  of this license or (at your option) any later version.}}
\hypersetup{pdflicenseurl={http://www.latex-project.org/lppl.txt}}
\hypersetup{pdfcontactaddress={ETH Zurich, ITP, HIT K,
  Wolfgang-Pauli-Strasse 27}}
\hypersetup{pdfcontactpostcode={8093}}
\hypersetup{pdfcontactcity={Zurich}}
\hypersetup{pdfcontactcountry={Switzerland}}
\hypersetup{pdfcontactemail={nbeisert@itp.phys.ethz.ch}}
\hypersetup{pdfcontacturl={http://people.phys.ethz.ch/\xmptilde nbeisert/}}

\newcommand{\secref}[1]{\hyperref[#1]{section \ref*{#1}}}

\parskip1ex
\parindent0pt
\let\olditemize\itemize
\def\itemize{\olditemize\parskip0pt}

\begin{document}

\title{The \textsf{childdoc} Package}
\hypersetup{pdftitle={The childdoc Package}}
\author{Niklas Beisert\\[2ex]
  Institut f\"ur Theoretische Physik\\
  Eidgen\"ossische Technische Hochschule Z\"urich\\
  Wolfgang-Pauli-Strasse 27, 8093 Z\"urich, Switzerland\\[1ex]
  \href{mailto:nbeisert@itp.phys.ethz.ch}
  {\texttt{nbeisert@itp.phys.ethz.ch}}}
\hypersetup{pdfauthor={Niklas Beisert}}
\hypersetup{pdfsubject={Manual for the LaTeX2e Package childdoc}}
\date{30 December 2018, \textsf{v2.0}}
\maketitle

\begin{abstract}\noindent
\textsf{childdoc} is a \LaTeXe{} package
that enables the direct compilation
of document sections included by |\include|
to individual files.
\end{abstract}

\begingroup
\parskip0ex
\tableofcontents
\endgroup

%%%%%%%%%%%%%%%%%%%%%%%%%%%%%%%%%%%%%%%%%%%%%%%%%%%%%%%%%%%%%%%%%%%%%%%%%%%%%%%%
%%%%%%%%%%%%%%%%%%%%%%%%%%%%%%%%%%%%%%%%%%%%%%%%%%%%%%%%%%%%%%%%%%%%%%%%%%%%%%%%
\section{Introduction}

\LaTeX{} provides a mechanism to structure a large document (such as a book)
into a main file and several child files (containing the chapters)
using the |\include| command.
This mechanism is beneficial for documents
which span hundreds of pages in order to
make the source file(s) more manageable.
Moreover, compilation can be restricted to
selected child files by means of the |\includeonly| command.
The latter feature can be used to reduce the compilation time while editing
(this was significantly more useful in the earlier days of \LaTeX{})
or to generate a smaller document which is easier to navigate.
Another application of |\includeonly| is to generate
documents consisting of selected parts of the complete document.

However, there are a few drawbacks of the plain |\include| mechanism:
\begin{itemize}
\item
The child files cannot be compiled on their own,
they can only be compiled via the main file.
A naive editing environment
(such as a text editor with an option
to have the current file processed by \LaTeX)
may require one to switch to the main file before compiling;
attempting to compile the child file produces errors.
\item
The main file must be modified (each time)
to adjust the |\includeonly| command
to the present needs. This easily leaves the main file in a messy state.
\item
The generated document will always carry the filename
of the main document. This is inconvenient if
several child files are to be compiled and
to be kept for distribution.
\end{itemize}

The present package provides a simple interface
to make child files individually compilable by \LaTeX{}.
Compiling a child file then has the same effect as compiling
the main file with an |\includeonly| command
to select the appropriate child.
Moreover the generated document will carry the name of the child
rather than the main file.
This resolves all three above issues.

This feature is meant to make the editing of books,
thesis documents and lecture notes somewhat more convenient.
However, the package can also be used efficiently for
composing a series of documents (such as exercise sheets)
which are typically distributed individually.
It then assists the author in generating the individual documents
(potentially in different versions)
as well as a document containing the collected series.
Another application is in developing style files
or other kinds of included material
where compilation of the style file could redirect
to a sample or test file.

%%%%%%%%%%%%%%%%%%%%%%%%%%%%%%%%%%%%%%%%%%%%%%%%%%%%%%%%%%%%%%%%%%%%%%%%%%%%%%%%
%%%%%%%%%%%%%%%%%%%%%%%%%%%%%%%%%%%%%%%%%%%%%%%%%%%%%%%%%%%%%%%%%%%%%%%%%%%%%%%%
\section{Usage}

First of all, the package \textsf{childdoc} is \emph{not} a standard
\LaTeXe{} |.sty| style file! Therefore it needs to be invoked in
a non-standard way.

%%%%%%%%%%%%%%%%%%%%%%%%%%%%%%%%%%%%%%%%%%%%%%%%%%%%%%%%%%%%%%%%%%%%%%%%%%%%%%%%
\subsection{Included Files}
\label{sec:include}

%%%%%%%%%%%%%%%%%%%%%%%%%%%%%%%%%%%%%%%%
\DescribeMacro{\childdocmain}
To use the package, add the commands
\begin{center}
\begin{tabular}{l}
|\input{childdoc.def}|\\
|\childdocmain{}|\\
\end{tabular}
\end{center}
at the very top of the main \LaTeX{} file,
in particular \emph{before} the |\documentclass| statement!
The argument of |\childdocmain| should be left empty
(but it must be present).

%%%%%%%%%%%%%%%%%%%%%%%%%%%%%%%%%%%%%%%%
\DescribeMacro{\childdocof}
Furthermore, add the commands
\begin{center}
\begin{tabular}{l}
|\input{childdoc.def}|\\
|\childdocof{|\textit{main}|}|\\
\end{tabular}
\end{center}
at the top of every child file \textit{child}
which is included by |\include{|\textit{child}|}|
from within the main file
(or at least for those files to be compiled individually).
The argument \textit{main} must be the filename of the main file.

There are a couple of
considerations in setting up the main and child documents:

%%%%%%%%%%%%%%%%%%%%%%%%%%%%%%%%%%%%%%%%
\paragraph{Restrictions.}

Please note the following restrictions:
\begin{itemize}
\item
|\childdocmain| must be called with one argument \textit{main}
to ensure compatibility with earlier version of the package.
It must either be empty (|\childdocmain{}|)
or precisely match the filename of the main file in which it is specified.
See \secref{sec:detection} for further information.
\item
The filename \textit{main} must be specified without the |.tex| extension.
\item
The filename \textit{main} is case sensitive
(even in case-insensitive file systems)
due to internal string comparison.
\item
The argument \textit{main} should be fully expanded, it cannot be a macro.
\item
Subdirectories and special characters should be avoided in filenames.
\item
The command |\childdocmain{|\textit{main}|}| must be followed by a whitespace.
It should not be followed immediately by another command
or by a comment mark `|%|'.
This is because the \TeX{} parser reads the token immediately following
the argument of |\childdocmain| and puts it
at the beginning of every child section;
however, a white\-space is ignored.
\end{itemize}

%%%%%%%%%%%%%%%%%%%%%%%%%%%%%%%%%%%%%%%%
\paragraph{Content of Main File.}

It is advisable to place all content in the child files included by |\include|.
Any output contained in the main file will appear in all child documents
unless suppressed manually;
it cannot be suppressed automatically by the |\includeonly| directive
and thus should normally be avoided.
A method to include some content in the main file
by means of conditional processing is described in \secref{sec:conditional}.

%%%%%%%%%%%%%%%%%%%%%%%%%%%%%%%%%%%%%%%%
\paragraph{Page Numbering.}

When only a part of the document is compiled,
the appropriate numbering of pages
(as well as other status parameters)
is determined from the |.aux| files.
The latter contain information from previous passes.
However this information needs to propagate through
all intermediate child documents.
Therefore the page numbering in child documents may well
be inconsistent until the complete document is compiled at least once.

A useful (if unconventional) way to always ensure a consistent
page numbering is to restart the numbering in each child document
and denote the pages by `\textit{child}|.|\textit{page}'
where \textit{child} represents the chapter/section number of the child file.
This can be achieved by the command
|\numberwithin{page}{|\textit{child}|}|
of the \textsf{amsmath} package
where \textit{child} can be |chapter| or |section|
depending on the chosen structuring.
Alternatively, one can modify the macro |\thepage| appropriately
and reset the counter |page| at the start of each child file.

%%%%%%%%%%%%%%%%%%%%%%%%%%%%%%%%%%%%%%%%%%%%%%%%%%%%%%%%%%%%%%%%%%%%%%%%%%%%%%%%
\subsection{Conditional Processing}
\label{sec:conditional}

The package provides a mechanism to compile different versions
of a document. To customise the versions further some conditional processing
can come in handy to distinguish which version is being compiled.
The package provides two macros to describe the compilation context:

%%%%%%%%%%%%%%%%%%%%%%%%%%%%%%%%%%%%%%%%
\DescribeMacro{\ifchilddoc}
The conditional |\ifchilddoc| distinguishes between the compilation of
child documents and the main document:
%
\begin{center}
|\ifchilddoc |\textit{child-code}| |[|\||else |\textit{main-code}]| \||fi|
\end{center}

%%%%%%%%%%%%%%%%%%%%%%%%%%%%%%%%%%%%%%%%
\DescribeMacro{\childdocname}
\DescribeMacro{\childdocjob}
The macro |\childdocname| contains the filename (without extension)
of the main or child file being processed.
Note that |\childdocjob| will always contain the name of the main file.

%%%%%%%%%%%%%%%%%%%%%%%%%%%%%%%%%%%%%%%%
\paragraph{Title Page.}

Conditional processing can be used to include a title or banner page
in the main document when proper precautions are taken.
Importantly, the code in the main file should ensure that the page counter
(as well as other status parameters which are stored in the |.aux| files)
takes the same value after the conditional processing.
Otherwise the page numbers may take divergent values
depending on which part is compiled.

For example, a title page could be declared by:
%
\begin{center}
\begin{tabular}{l}
|\ifchilddoc\||else|\\
|\addtocounter{page}{-1}|\\
\textit{code for title page}\\
|\newpage|\\
|\||fi|
\end{tabular}
\end{center}
%
A banner page for the child documents can be generated by:
%
\begin{center}
\begin{tabular}{l}
|\ifchilddoc|\\
|\addtocounter{page}{-1}|\\
\textit{code for banner page}\\
|\newpage|\\
|\||fi|
\end{tabular}
\end{center}
%
Here one could write a message such as:
\begin{center}
|This is the part \childdocname{} of \childdocjob{}.|
\end{center}

%%%%%%%%%%%%%%%%%%%%%%%%%%%%%%%%%%%%%%%%%%%%%%%%%%%%%%%%%%%%%%%%%%%%%%%%%%%%%%%%
\subsection{Flags}
\label{sec:flags}

The package makes it easy to generate different versions
of the main or child documents.
To this end compilation flags can be defined
and assigned different default values.
They will be particularly useful in conjunction
with the forwarding mechanism described in \secref{sec:forward}.

For example, it may be useful to have a flag |\version|
which can be set to |draft| or |final|.
The document source will contain some conditional code
depending on the value of |\version|.
Suppose further, the flag should default to |final| for the main file
and to |draft| for child files
which is a natural assignment for editing the document.
This is achieved by placing the following code
in the preamble of the main document
(below the |\childdocmain| directive):
%
\begin{center}
\begin{tabular}{l}
|\ifchilddoc|\\
|\providecommand{\version}{draft}|\\
|\||else|\\
|\providecommand{\version}{final}|\\
|\||fi|
\end{tabular}
\end{center}
%
The definition by |\providecommand| makes sure
that previous definitions are not overwritten.
Further statements |\providecommand{\version}{...}|
can thus be added before the above code to override it.

For the main file, one might add a line
(between |\childdocmain| and the above block)
%
\begin{center}
|%\ifchilddoc\||else\providecommand{\version}{draft}\||fi|
\end{center}
%
which can be uncommented to produce a draft version.
Likewise one can add a line to the very top of a child file
(above the |\childdocof{|\textit{main}|}| directive)
%
\begin{center}
|%\providecommand{\version}{final}|
\end{center}
%
which can be uncommented to produce the final version of this child document.

%%%%%%%%%%%%%%%%%%%%%%%%%%%%%%%%%%%%%%%%%%%%%%%%%%%%%%%%%%%%%%%%%%%%%%%%%%%%%%%%
\subsection{Forwarding}
\label{sec:forward}

Different versions of the main or child documents
using compilation flags as described in \secref{sec:flags}
can be (permanently) stored in different files
for convenient compilation, viewing and distribution.
To this end, the package defines a command
to pass on compilation to a different file:

%%%%%%%%%%%%%%%%%%%%%%%%%%%%%%%%%%%%%%%%
\DescribeMacro{\childdocforward}
The command |\childdocforward| redirects processing to
another source file:
%
\begin{center}
\begin{tabular}{l}
|\input{childdoc.def}|\\
|\childdocforward[|\textit{main}|]{|\textit{dest}|}|\\
\end{tabular}
\end{center}
%
The argument \textit{dest} is the destination file
(without extension).
It should be the main file or one of the child files.
Note that further \textsf{childdoc} directives
such as |\childdocof| and |\childdocforward|
in the indicated file will be processed in this form.
The optional argument \textit{main}
passes on directly to the main file \textit{main}
while pretending to compile the child \textit{dest}.
This form behaves as if \textit{dest}
issues |\childdocof{|\textit{main}|}| right away,
and no further \textsf{childdoc} directives will be processed.

%%%%%%%%%%%%%%%%%%%%%%%%%%%%%%%%%%%%%%%%
\DescribeMacro{\...prefix}
In the alternative form |\childdocforwardprefix|,
%
\begin{center}
\begin{tabular}{l}
|\input{childdoc.def}|\\
|\childdocforwardprefix[|\textit{main}|]{|\textit{prefix}|}{|\textit{dest}|}|
\end{tabular}
\end{center}
%
the destination file is determined by a pattern
depending on the current file:
To make this work, the current file must be called
`{\textit{prefix}\hspace{0.2em}\textit{suffix}}'
with \textit{prefix} matching precisely the argument.
Processing is then passed on to the file
`{\textit{dest}\hspace{0.2em}\textit{suffix}}'.
Surely, the same effect is achieved by
directly specifying the
argument `{\textit{dest}\hspace{0.2em}\textit{suffix}}'
in the first form.
However, that requires to set up a different file
for each child. With the alternative form of the command
all these files can have exactly the same content
which simplifies setting them up and maintaining them.

For example, the following file |draft.tex|
with a compilation flag |\version| as described in \secref{sec:flags}
compiles the main document as a draft:
%
\begin{center}
\begin{tabular}{l}
|\def\version{draft}|\\
|\input{childdoc.def}|\\
|\childdocforward{|\textit{main}|}|
\end{tabular}
\end{center}
%
Likewise, the following files |final|\textit{nn}|.tex|
compile the final version of the child document
|child|\textit{nn}|.tex|:
%
\begin{center}
\begin{tabular}{l}
|\def\version{final}|\\
|\input{childdoc.def}|\\
|\childdocforwardprefix{final}{child}|
\end{tabular}
\end{center}
%

Note that when several versions of a main file and/or of each child file
are to be generated, it may be convenient to set up a |Makefile| or
shell script to automatise the process.

%%%%%%%%%%%%%%%%%%%%%%%%%%%%%%%%%%%%%%%%%%%%%%%%%%%%%%%%%%%%%%%%%%%%%%%%%%%%%%%%
\subsection{Command Line Processing}
\label{sec:commandline}

The effect of redirection files can also be achieved by invoking
the \LaTeX{} compiler with a more elaborate command line.
Most conveniently this should be done as part
of a shell script or a |Makefile|.

When using \textsf{childdoc} in the main file, the following
command lines effectively perform a redirection
(note that depending on the shell being used,
backslashes may have to be doubled: `|\|' $\to$ `|\\|'):
%
\begin{center}
|... -jobname "|\textit{target}|" |\\|"|[\textit{flags}]%
|\input{childdoc.def}\childdocforward[|\textit{main}|]{|\textit{dest}|}"|
\end{center}
%
Here \textit{target} is the name of the output file,
\textit{main} is the name of the main file
and \textit{dest} is the name of the main or child file to be processed
(all filenames without extensions).
The optional argument \textit{main} can be omitted
if \textit{main} matches \textit{dest}.
Optionally, compilation \textit{flags} can be defined via |\def| commands.
This command line makes the \TeX{} engine believe
it is compiling the file \textit{target}
whose content is specified as the latter parameter.
The provided code then forwards the processing to
\textit{main} or \textit{dest} as described in \secref{sec:forward}.

%%%%%%%%%%%%%%%%%%%%%%%%%%%%%%%%%%%%%%%%%%%%%%%%%%%%%%%%%%%%%%%%%%%%%%%%%%%%%%%%
\subsection{Include by Input}
\label{sec:input}

Including child documents by |\include| has some restrictions by design.
Most notably, the content of a child document always occupies
its own set of pages; pages cannot be shared between child documents.
Usually, this behaviour makes perfect sense
because each child document contain an essential part of the document.
However, in some situations it may be desirable to compose
a document from a collection of parts
without having mandatory page breaks between then.
For this case, the package
provides a mechanism to include parts
by |\input| which can also be processed individually.
However, by construction this mechanism
requires manual handling of the content to be output.

%%%%%%%%%%%%%%%%%%%%%%%%%%%%%%%%%%%%%%%%
\DescribeMacro{\ifchilddocmanual}
The main file should be prepared as usual, see \secref{sec:include}.
However, the document body must make a distinction
between processing of an individual part and of the main document, e.g.:
%
\begin{center}
\begin{tabular}{l}
|\ifchilddocmanual|\\
|\input{\childdocname}|\\
|\||else|\\
\textit{document body with }|\input{|\textit{part}|}|\\
|\||fi|
\end{tabular}
\end{center}
%
The conditional |\ifchilddocmanual| is true whenever
a part to be included by |\input| is being compiled,
and the name of the part is stored in |\childdocname|.

%%%%%%%%%%%%%%%%%%%%%%%%%%%%%%%%%%%%%%%%
\DescribeMacro{\childdocby}
Each part to be included by |\input| should start with:
%
\begin{center}
\begin{tabular}{l}
|\input{childdoc.def}|\\
|\childdocby{|\textit{main}|}|\\
\end{tabular}
\end{center}
%
The directive |\childdocby| is similar to |\childdocof|
described in \secref{sec:include},
but the subsequent selection of content must be done manually.
To that end, both |\ifchilddoc| and |\ifchilddocmanual|
will be true upon processing of a part,
and the name of the part is stored in |\childdocname|.
Note that |\jobname| will be set to the filename of the current part
so that each part receives an individual |.aux| file
that does not interfere with the |.aux| file(s) of the main document.
This behaviour can be altered by the alternative form
|\childdocby[*]{|\textit{main}|}| (with a non-empty optional argument)
which uses the |.aux| file of the main document
by setting |\jobname| to \textit{main}.

%%%%%%%%%%%%%%%%%%%%%%%%%%%%%%%%%%%%%%%%%%%%%%%%%%%%%%%%%%%%%%%%%%%%%%%%%%%%%%%%
\subsection{Driver Development}
\label{sec:driver}

The \textsf{childdoc} mechanism can also be use for the development
of definition files such as \LaTeX{} styles or classes.
This case differs from the above setup with multiple parts
included by |\include| in that no |\includeonly| should be invoked.
This can be achieved by starting the include file
(before |\ProvidesPackage|) with:
%
\begin{center}
\begin{tabular}{l}
|\input{childdoc.def}|\\
|\childdocforward{|\textit{main}|}|\\
\end{tabular}
\end{center}
%
or alternatively with:
%
\begin{center}
\begin{tabular}{l}
|\input{childdoc.def}|\\
|\childdocby{|\textit{main}|}|\\
\end{tabular}
\end{center}
%
Both forms have slightly different effects as described above.
The main file is prepared as usual, see \secref{sec:include}.

%%%%%%%%%%%%%%%%%%%%%%%%%%%%%%%%%%%%%%%%%%%%%%%%%%%%%%%%%%%%%%%%%%%%%%%%%%%%%%%%
\subsection{Legacy Detection}
\label{sec:detection}

The directive |\childdocmain| in the main file can detect
whether the complete document or merely a child is to be compiled
even without using the directive |\childdocof|.
This method is deprecated because it is less robust
and there is no compelling reason to use it;
it is merely provided for backward compatibility
and it may be removed in future versions.

If the detection mechanism is to be used,
it is mandatory to correctly specify
the filename of the main file as the argument of |\childdocmain|:
%
\begin{center}
\begin{tabular}{l}
|\input{childdoc.def}|\\
|\childdocmain{|\textit{main}|}|\\
\end{tabular}
\end{center}
%
If |\jobname| does not match the argument \textit{main} of |\childdocmain|,
it is assumed that |\jobname| points to the child file to be compiled.
When using |\childdocmain| with the main file specified as argument,
it suffices to start a child file
with just |\input{|\textit{main}|}|
without loading of the package and using |\childdocof|.
If instead all processing is done
with the appropriate \textsf{childdoc} directives,
the argument of \textit{main} of |\childdocmain| can be empty.

An alternative version of the command line processing described
in \secref{sec:commandline} using the detection mechanism reads:
%
\begin{center}
|... -jobname "|\textit{target}|" "|[\textit{flags}]%
[|\def\jobname{|\textit{dest}|}|]|\input{|\textit{main}|}"|
\end{center}

%%%%%%%%%%%%%%%%%%%%%%%%%%%%%%%%%%%%%%%%%%%%%%%%%%%%%%%%%%%%%%%%%%%%%%%%%%%%%%%%
\subsection{Manual Code}
\label{sec:manual}

In case one cannot be certain whether the definitions file |childdoc.def|
is installed on the target \TeX{} distribution
and one prefers not to ship it,
it is conceivable to paste a few relevant commands into the sources.

To that end, drop all statements |\input{childdoc.def}|
and perform the replacements as outlined below.
Instead of |\childdocmain{|\textit{main}|}| add the following code
to the top of the main file:
%
\begin{center}
\begin{tabular}{l}
|\||ifdefined\childdocname\endinput\||fi\newif\ifchilddoc|\\
|\edef\childdocname{\scantokens\expandafter{\jobname\noexpand}}|\\
|\def\childdocmain{|\textit{main}|}\||ifx\childdocmain\childdocname\||else|\\
|\childdoctrue\includeonly{\childdocname}\let\jobname\childdocmain\||fi|\\
\end{tabular}
\end{center}
%
Instead of |\childdocof{|\textit{main}|}| just include the main file
at the top of each child file:
%
\begin{center}
|\input{|\textit{main}|}|
\end{center}
%
A simple redirection |\childdocforward{|\textit{dest}|}| is achieved by:
%
\begin{center}
|\def\jobname{|\textit{dest}|}\input{\jobname}|
\end{center}
%
The redirection with prefix
|\childdocforwardprefix[|\textit{prefix}|]{|\textit{dest}|}|
is accomplished by:
%
\begin{center}
\begin{tabular}{l}
|{\edef\jobname{\scantokens\expandafter{\jobname\noexpand}}|\\
|\def\redirectjob |\textit{prefix}|#1~~~{\gdef\jobname{|\textit{dest}|#1}}|\\
|\expandafter\redirectjob\jobname~~~}\input{\jobname}|
\end{tabular}
\end{center}

In an alternative approach,
child documents can be compiled by a specific command line
without additional code or specific definitions:
%
\begin{center}
|... -jobname "|\textit{target}|" "|[\textit{flags}]%
|\includeonly{|\textit{dest}|}\input{|\textit{main}|}"|
\end{center}
%

%%%%%%%%%%%%%%%%%%%%%%%%%%%%%%%%%%%%%%%%%%%%%%%%%%%%%%%%%%%%%%%%%%%%%%%%%%%%%%%%
%%%%%%%%%%%%%%%%%%%%%%%%%%%%%%%%%%%%%%%%%%%%%%%%%%%%%%%%%%%%%%%%%%%%%%%%%%%%%%%%
\section{Information}

%%%%%%%%%%%%%%%%%%%%%%%%%%%%%%%%%%%%%%%%%%%%%%%%%%%%%%%%%%%%%%%%%%%%%%%%%%%%%%%%
\subsection{Copyright}

Copyright \copyright{} 2017--2018 Niklas Beisert

This work may be distributed and/or modified under the
conditions of the \LaTeX{} Project Public License, either version 1.3
of this license or (at your option) any later version.
The latest version of this license is in
  \url{http://www.latex-project.org/lppl.txt}
and version 1.3 or later is part of all distributions of \LaTeX{}
version 2005/12/01 or later.

This work has the LPPL maintenance status `maintained'.

The Current Maintainer of this work is Niklas Beisert.

This work consists of the files |README.txt|, |childdoc.ins| and |childdoc.dtx|
as well as the derived files |childdoc.def|, |cdocsamp.tex|
with |cdocsch1.tex|, |cdocsch2.tex|, |cdocspt3.tex|, |cdocspt4.tex|,
|cdocsdrf.tex|, |cdocsfn1.tex|, |cdocsfn2.tex|
as well as |childdoc.pdf|.

%%%%%%%%%%%%%%%%%%%%%%%%%%%%%%%%%%%%%%%%%%%%%%%%%%%%%%%%%%%%%%%%%%%%%%%%%%%%%%%%
\subsection{Files and Installation}

The package consists of the files:
%
\begin{center}
\begin{tabular}{ll}
    |README.txt|   & readme file \\
    |childdoc.ins| & installation file \\
    |childdoc.dtx| & source file \\
    |childdoc.def| & definition file \\
    |cdocsamp.tex| & sample main file \\
    |cdocsch1.tex| & sample include file \\
    |cdocsch2.tex| & sample include file \\
    |cdocspt3.tex| & sample part file \\
    |cdocspt4.tex| & sample part file \\
    |cdocsdrf.tex| & sample redirection file \\
    |cdocsfn1.tex| & sample redirection file \\
    |cdocsfn2.tex| & sample redirection file \\
    |childdoc.pdf| & manual
\end{tabular}
\end{center}
%
The distribution consists of the files
|README.txt|, |childdoc.ins| and |childdoc.dtx|.
%
\begin{itemize}
\item
Run (pdf)\LaTeX{} on |childdoc.dtx|
to compile the manual |childdoc.pdf| (this file).
\item
Run \LaTeX{} on |childdoc.ins| to create the definitions file |childdoc.def|
and the sample |cdocsamp.tex| with include files
|cdocsch1.tex|, |cdocsch2.tex|, |cdocspt3.tex|, |cdocspt4.tex|,
|cdocsdrf.tex|, |cdocsfn1.tex|, |cdocsfn2.tex|.
Then copy the file |childdoc.def| to an appropriate directory of your \LaTeX{}
distribution, e.g.\ \textit{texmf-root}|/tex/latex/childdoc|.
\end{itemize}

%%%%%%%%%%%%%%%%%%%%%%%%%%%%%%%%%%%%%%%%%%%%%%%%%%%%%%%%%%%%%%%%%%%%%%%%%%%%%%%%
\subsection{Related CTAN Packages}

There are several other packages which offer a similar functionality:
%
\begin{itemize}
\item
The packages
\href{http://ctan.org/pkg/docmute}{\textsf{docmute}},
\href{http://ctan.org/pkg/includex}{\textsf{includex}} and
\href{http://ctan.org/pkg/standalone}{\textsf{standalone}}
provide commands to include only the document body of
a child file thus allowing both files to be compiled individually.
\item
The packages \href{http://ctan.org/pkg/subdocs}{\textsf{subdocs}}
and \href{http://ctan.org/pkg/subfiles}{\textsf{subfiles}}
provide structures in which the main and child documents can be
encapsulated and allowing them to be compiled individually.
The inclusion mechanism is different from the conventional |\include|.
\item
The package \href{http://ctan.org/pkg/combine}{\textsf{combine}}
is an elaborate solution to combine several documents into one.
\end{itemize}
%
See also the CTAN topic \href{http://ctan.org/topic/subdocs}{\textsf{subdocs}}
for further related packages.
The present package differs from the above solutions in that
a document structure constructed with the conventional |\include| mechanism
just needs two extra commands at the top of every file
such that all constituent files can be compiled individually.

%%%%%%%%%%%%%%%%%%%%%%%%%%%%%%%%%%%%%%%%%%%%%%%%%%%%%%%%%%%%%%%%%%%%%%%%%%%%%%%%
%\subsection{Feature Suggestions}
%
%The following is a list of features which may be useful for future
%versions of this package:
%%
%\begin{itemize}
%\item
%\ldots
%\end{itemize}

%%%%%%%%%%%%%%%%%%%%%%%%%%%%%%%%%%%%%%%%%%%%%%%%%%%%%%%%%%%%%%%%%%%%%%%%%%%%%%%%
\subsection{Revision History}

%%%%%%%%%%%%%%%%%%%%%%%%%%%%%%%%%%%%%%%%
\paragraph{v2.0:} 2018/12/30

\begin{itemize}
\item
immediate forward processing
\item
added |\childdocby| mechanism
\item
manual restructured
\end{itemize}

%%%%%%%%%%%%%%%%%%%%%%%%%%%%%%%%%%%%%%%%
\paragraph{v1.6:} 2018/01/17

\begin{itemize}
\item
application for development of include files
\item
corrections to manual
\end{itemize}

%%%%%%%%%%%%%%%%%%%%%%%%%%%%%%%%%%%%%%%%
\paragraph{v1.5:} 2017/05/21

\begin{itemize}
\item
more complete structuring introduced
\item
|\childdocof| introduced
\item
|\childdoc| renamed to |\childdocmain|
\item
|\childredirect| renamed to |\childdocforward| and |\childdocforwardprefix|
and functionality expanded
\end{itemize}

%%%%%%%%%%%%%%%%%%%%%%%%%%%%%%%%%%%%%%%%
\paragraph{v1.0:} 2017/04/27

\begin{itemize}
\item
manual and install package
\item
first version published on CTAN
\end{itemize}

%%%%%%%%%%%%%%%%%%%%%%%%%%%%%%%%%%%%%%%%
\paragraph{v0.6:} 2017/04/26

\begin{itemize}
\item
redirection mechanism added
\end{itemize}

%%%%%%%%%%%%%%%%%%%%%%%%%%%%%%%%%%%%%%%%
\paragraph{v0.5:} 2017/04/26

\begin{itemize}
\item
functionality in definition file
\end{itemize}


%%%%%%%%%%%%%%%%%%%%%%%%%%%%%%%%%%%%%%%%%%%%%%%%%%%%%%%%%%%%%%%%%%%%%%%%%%%%%%%%
%%%%%%%%%%%%%%%%%%%%%%%%%%%%%%%%%%%%%%%%%%%%%%%%%%%%%%%%%%%%%%%%%%%%%%%%%%%%%%%%
%%%%%%%%%%%%%%%%%%%%%%%%%%%%%%%%%%%%%%%%%%%%%%%%%%%%%%%%%%%%%%%%%%%%%%%%%%%%%%%%
\appendix

\settowidth\MacroIndent{\rmfamily\scriptsize 000\ }

 \DocInput{childdoc.dtx}

\end{document}
%</driver>
% \fi
%
% %%%%%%%%%%%%%%%%%%%%%%%%%%%%%%%%%%%%%%%%%%%%%%%%%%%%%%%%%%%%%%%%%%%%%%%%%%%%%%
% %%%%%%%%%%%%%%%%%%%%%%%%%%%%%%%%%%%%%%%%%%%%%%%%%%%%%%%%%%%%%%%%%%%%%%%%%%%%%%
% \section{Sample}
%\iffalse
%<*samplemain>
%\fi
%
% The following presents a sample document
% with two chapters, two parts, a title page,
% a compile flag as well as three forwarding files to set the flag.
% It consists of eight |.tex| files:
% \begin{center}
% \begin{tabular}{ll}
% |cdocsamp.tex|&main file\\
% |cdocsch1.tex|&include file for chapter 1\\
% |cdocsch2.tex|&include file for chapter 2\\
% |cdocspt3.tex|&include file for part 3\\
% |cdocspt4.tex|&include file for part 4\\
% |cdocsdrf.tex|&forwarding file for main file in draft mode\\
% |cdocsfi1.tex|&forwarding file for final version of chapter 1\\
% |cdocsfi2.tex|&forwarding file for final version of chapter 2\\
% \end{tabular}
% \end{center}
% Each of the eight files can be compiled directly by the \LaTeX{} compiler.
%
% %%%%%%%%%%%%%%%%%%%%%%%%%%%%%%%%%%%%%%
% \paragraph{Main File.}
%
% The main file is called |cdocsamp.tex|.
%
% Load the \textsf{childdoc} definitions and
% declare the filename for the main document:
%    \begin{macrocode}
\input{childdoc.def}
\childdocmain{}
%    \end{macrocode}

% Optional override for |\version| flag:
%    \begin{macrocode}
%%\ifchilddoc\else\providecommand{\version}{draft}\fi
%    \end{macrocode}

% Define the default values for the |\version| flag
% (|final| for the main file and |draft| for childs):
%    \begin{macrocode}
\ifchilddoc
\providecommand{\version}{draft}
\else
\providecommand{\version}{final}
\fi
%    \end{macrocode}

% Load the standard document class:
%    \begin{macrocode}
\documentclass[12pt]{article}
%    \end{macrocode}

% Start the document body:
%    \begin{macrocode}
\begin{document}
%    \end{macrocode}

% Declare a title page.
% Print title, part of document being processed and version flag:
%    \begin{macrocode}
\addtocounter{page}{-1}
\begin{center}
{\LARGE\bfseries{}childdoc example\par}
\vspace{1cm}
\ifchilddoc
\ifchilddocmanual part\else chapter\fi:
`\childdocname' of `\childdocjob'\par
\else
main document: `\childdocjob'\par
\fi
version: \version\par
\end{center}
\newpage
%    \end{macrocode}

% Manually include selected file,
% otherwise process as usual:
%    \begin{macrocode}
\ifchilddocmanual
\section*{part `\childdocname'}
\input{\childdocname}
\else
%    \end{macrocode}

% Include the two chapters:
%    \begin{macrocode}
\include{cdocsch1}
\include{cdocsch2}
%    \end{macrocode}

% Include the two parts unless only chapters should be displayed:
%    \begin{macrocode}
\ifchilddoc\else
\section{part three}
\input{cdocspt3}
\section{part four}
\input{cdocspt4}
\fi
%    \end{macrocode}

% Process as usual until here:
%    \begin{macrocode}
\fi
%    \end{macrocode}

% End of document body:
%    \begin{macrocode}
\end{document}
%    \end{macrocode}
%\iffalse
%</samplemain>
%\fi
%
% %%%%%%%%%%%%%%%%%%%%%%%%%%%%%%%%%%%%%%
% \paragraph{Chapter Include Files.}
%
% The include files are called |cdocsch1.tex| and |cdocsch2.tex|.
%
%\iffalse
%<*samplechap1|samplechap2>
%\fi

% Optional override for |\version| flag:
%    \begin{macrocode}
%%\providecommand{\version}{final}
%    \end{macrocode}

% Include the main document:
%    \begin{macrocode}
\input{childdoc.def}
\childdocof{cdocsamp}
%    \end{macrocode}

%\iffalse
%</samplechap1|samplechap2>
%\fi
%
%\iffalse
%<*samplechap1>
%\fi
% Some text for chapter 1:
%    \begin{macrocode}
\section{one}
some text in chapter one
%    \end{macrocode}

%\iffalse
%</samplechap1>
%\fi
% Some text for chapter 2:
%\iffalse
%<*samplechap2>
%\fi
%    \begin{macrocode}
\section{two}
more text in chapter two
%    \end{macrocode}

%\iffalse
%</samplechap2>
%\fi
%
% %%%%%%%%%%%%%%%%%%%%%%%%%%%%%%%%%%%%%%
% \paragraph{Part Include Files.}
%
% The include files are called |cdocspt3.tex| and |cdocspt4.tex|.
%
%\iffalse
%<*samplepart3|samplepart4>
%\fi

% Optional override for |\version| flag:
%    \begin{macrocode}
%%\providecommand{\version}{final}
%    \end{macrocode}

% Include the main document:
%    \begin{macrocode}
\input{childdoc.def}
\childdocby{cdocsamp}
%    \end{macrocode}

%\iffalse
%</samplepart3|samplepart4>
%\fi
%
%\iffalse
%<*samplepart3>
%\fi
% Some text for part 3:
%    \begin{macrocode}
some text in part three
%    \end{macrocode}

%\iffalse
%</samplepart3>
%\fi
% Some text for part 4:
%\iffalse
%<*samplepart4>
%\fi
%    \begin{macrocode}
more text in part four
%    \end{macrocode}

%\iffalse
%</samplepart4>
%\fi
%
% %%%%%%%%%%%%%%%%%%%%%%%%%%%%%%%%%%%%%%
% \paragraph{Forwarding for a Complete Draft.}
%
% The following forwarding file |cdocsdrf.tex|
% compiles the main document in draft mode:
%\iffalse
%<*sampledraft>
%\fi
%    \begin{macrocode}
\def\version{draft}
\input{childdoc.def}
\childdocforward{cdocsamp}
%    \end{macrocode}

%\iffalse
%</sampledraft>
%\fi
%
% %%%%%%%%%%%%%%%%%%%%%%%%%%%%%%%%%%%%%%
% \paragraph{Forwarding for Final Version of the Chapters.}
%
% The following forwarding files |cdocsfn1.tex| and |cdocsfn2.tex|
% (with identical content)
% compile the final versions of the child documents
% |cdocsch1.tex| and |cdocsch2.tex|, respectively:
%\iffalse
%<*samplefinal>
%\fi
%    \begin{macrocode}
\def\version{final}
\input{childdoc.def}
\childdocforwardprefix[cdocsamp]{cdocsfn}{cdocsch}
%    \end{macrocode}

%\iffalse
%</samplefinal>
%\fi
%
% %%%%%%%%%%%%%%%%%%%%%%%%%%%%%%%%%%%%%%
% \paragraph{Command Line Processing.}
%
% The following three command lines generate the output files
% |cdocscld|, |cdocscl1| and |cdocscl2|
% which should be identical to
% |cdocsdrf|, |cdocsch1| and |cdocsfn2|, respectively:
% \begin{center}
% \begin{tabular}{l}
% |latex -jobname cdocscld \|\\
% |  "\def\version{draft}\input{childdoc.def}\childdocforward{cdocsamp}"|\\
% |latex -jobname cdocscl1 \|\\
% |  "\input{childdoc.def}\childdocforward[cdocsamp]{cdocsch1}"|\\
% |latex -jobname cdocscl2 \|\\
% |  "\def\version{final}\input{childdoc.def}\childdocforward{cdocsch2}"|
% \end{tabular}
% \end{center}
% Note that the trailing backslash on each first line
% merely continues the input to the second line
% (for convenient cut ant paste).
% Furthermore, the command |latex| can be replaced by any
% of its alternative versions such as |pdflatex|.
%
% %%%%%%%%%%%%%%%%%%%%%%%%%%%%%%%%%%%%%%%%%%%%%%%%%%%%%%%%%%%%%%%%%%%%%%%%%%%%%%
% %%%%%%%%%%%%%%%%%%%%%%%%%%%%%%%%%%%%%%%%%%%%%%%%%%%%%%%%%%%%%%%%%%%%%%%%%%%%%%
% \section{Implementation}
%\iffalse
%<*package>
%\fi
%
% This section describes the definitions file |childdoc.def|.

% The definitions cannot be loaded using |\usepackage| or |\RequirePackage|
% which has a mechanism to prevent loading a style file more than once.
% When loading the definitions by means of |\input|
% multiple instances have to be prevented manually:
%\iffalse
%This code needs to be before the `\ProvidesFile' directive
%which is defined at the beginning of this file.
%Therefore it is also placed there and commented out here.
%</package>
%<*discard>
%\fi
%    \begin{macrocode}
\ifdefined\childdocmain\endinput\fi
%    \end{macrocode}
%\iffalse
%</discard>
%<*package>
%\fi
%
% \macro{\ifchilddoc}
% \macro{\ifchilddocmanual}
% The conditional |\ifchilddoc| tells whether a
% child (true) or main (false) document is being compiled.
% The conditional |\ifchilddocmanual| tells whether
% the |\includeonly| mechanism is used (false) or
% the selection of child files must be performed manually (true).
% The definitions initialise to false:
%    \begin{macrocode}
\newif\ifchilddoc
\newif\ifchilddocmanual
%    \end{macrocode}

% \macro{\childdocname}
% \macro{\childdocjob}
% The macro |\childdocname| stores the name of the main document
% to be compiled. The macro |\childdocjob| stores the name of
% the document on which the \LaTeX{} compiler was originally invoked.
% The content of |\jobname| cannot be compared
% to filenames specified in the source due to different catcodes.
% The following code rescans |\jobname|, stores the result
% in |\childdocname| and saves a copy in |\childdocjob|:
%    \begin{macrocode}
\edef\childdocname{\scantokens\expandafter{\jobname\noexpand}}
\let\childdocjob\childdocname
%    \end{macrocode}

% \macro{\childdocdisable}
% The macro |\childdocdisable| prevents the main file
% from being processed more than once.
% At this stage, the main document command |\childdocmain|
% is assumed to be called once again where it should do nothing.
% Any subsequent call to it should prevent
% a secondary processing of the main document
% It overwrites the forwarding commands
% |\childdocof| and |\childdocforward|
% with empty macros to prevent further inclusions of the main document:
%    \begin{macrocode}
\newcommand{\childdocdisable}
{
  \renewcommand{\childdocmain}[1]{\renewcommand{\childdocmain}[1]{\endinput}}
  \renewcommand{\childdocof}[1]{}
  \renewcommand{\childdocby}[2][]{}
  \renewcommand{\childdocforward}[2][]{}
  \renewcommand{\childdocdisable}{}
}
%    \end{macrocode}

% \macro{\childdocmain}
% The macro |\childdocmain| is to be called at the top of the main file
% with nothing or the main filename (without extension) as argument.
% First, it breaks loops.
% If the argument is not empty and does not match |\childdocname|
% (which is set by the first inclusion of |childdoc.def|),
% |\ifchilddoc| is set to true, |\includeonly| is applied to the child file
% and |\jobname| is set to the main file
% (for proper handling of |.aux| files):
%    \begin{macrocode}
\newcommand{\childdocmain}[1]
{
  \childdocdisable\childdocmain{}
  \if?#1?\else
    \begingroup
      \def\childdoctmp{#1}
      \ifx\childdoctmp\childdocname
        \def\childdoctmp{}
      \else
        \def\childdoctmp
        {
          \childdoctrue
          \includeonly{\childdocname}
          \def\childdocjob{#1}
          \def\jobname{#1}
        }
      \fi
      \expandafter
    \endgroup
    \childdoctmp
  \fi
}
%    \end{macrocode}

% \macro{\childdocof}
% The command |\childdocof| redirects
% compilation to the main file |#1|.
%    \begin{macrocode}
\newcommand{\childdocof}[1]
{
  \childdocdisable
  \childdoctrue
  \includeonly{\childdocname}
  \def\jobname{#1}
  \def\childdocjob{#1}
  \input{#1}
}
%    \end{macrocode}

% \macro{\childdocby}
% The command |\childdocby| ....
%    \begin{macrocode}
\newcommand{\childdocby}[2][]
{
  \childdocdisable
  \childdoctrue
  \childdocmanualtrue
  \if?#1?\else
    \def\jobname{#2}
  \fi
  \def\childdocjob{#2}
  \input{#2}
  \endinput
}
%    \end{macrocode}

% \macro{\childdocforward}
% The command |\childdocforward| redirects
% compilation to the main file or
% (if the optional argument is given) a child file.
% Parameters are set as if the main file
% or a child file starting with |\childdocof| was compiled.
% Then compilation is handed over to the main file:
%    \begin{macrocode}
\newcommand{\childdocforward}[2][]
{
  \begingroup
    \if?#1?
      \def\childdoctmp
      {
        \def\childdocname{#2}
        \def\childdocjob{#2}
        \def\jobname{#2}
        \input{#2}
        \endinput
      }
    \else
      \def\childdoctmp
      {
        \childdocdisable
        \def\childdocname{#2}
        \childdoctrue
        \includeonly{#2}
        \def\childdocjob{#1}
        \def\jobname{#1}
        \input{#1}
        \endinput
      }
    \fi
    \expandafter
  \endgroup
  \childdoctmp
}
%    \end{macrocode}

% \macro{\childdocforwardprefix}
% The command |\childdocforwardprefix| redirects
% compilation to the main or a child file by means of a pattern.
% The prefix |#1| in the current filename is replaced by |#2|
% and the suffix of the current filename is kept
% (it is assumed that the filename does not contain the substring `|~~~|'
% which is used as a delimiter).
% Compilation is handed over to the new file by |\childdocforward|:
%    \begin{macrocode}
\newcommand{\childdocforwardprefix}[3][]
{
  \begingroup
    \def\childdocextract #2##1~~~{\def\childdoctmp{\childdocforward[#1]{#3##1}}}
    \expandafter\childdocextract\childdocname~~~
    \expandafter
  \endgroup
  \childdoctmp
}
%    \end{macrocode}

% \macro{\childdoc}
% The deprecated macro |\childdoc| is a legacy version of |\childdocmain|:
%    \begin{macrocode}
\newcommand{\childdoc}{\childdocmain}
%    \end{macrocode}

% \macro{\childdocredirect}
% The deprecated macro |\childdocredirect| is a legacy version
% of |\childdocforward| and |\childdocforwardprefix|:
%    \begin{macrocode}
\newcommand{\childdocredirect}[2][]
{
  \begingroup
    \if?#1?
      \def\childdoctmp{\childdocforward{#2}}
    \else
      \def\childdoctmp{\childdocforwardprefix{#1}{#2}}
    \fi
    \expandafter
  \endgroup
  \childdoctmp
}
%    \end{macrocode}

%\iffalse
%</package>
%\fi
%
\endinput
|\\
|\childdocof{|\textit{main}|}|\\
\end{tabular}
\end{center}
at the top of every child file \textit{child}
which is included by |\include{|\textit{child}|}|
from within the main file
(or at least for those files to be compiled individually).
The argument \textit{main} must be the filename of the main file.

There are a couple of
considerations in setting up the main and child documents:

%%%%%%%%%%%%%%%%%%%%%%%%%%%%%%%%%%%%%%%%
\paragraph{Restrictions.}

Please note the following restrictions:
\begin{itemize}
\item
|\childdocmain| must be called with one argument \textit{main}
to ensure compatibility with earlier version of the package.
It must either be empty (|\childdocmain{}|)
or precisely match the filename of the main file in which it is specified.
See \secref{sec:detection} for further information.
\item
The filename \textit{main} must be specified without the |.tex| extension.
\item
The filename \textit{main} is case sensitive
(even in case-insensitive file systems)
due to internal string comparison.
\item
The argument \textit{main} should be fully expanded, it cannot be a macro.
\item
Subdirectories and special characters should be avoided in filenames.
\item
The command |\childdocmain{|\textit{main}|}| must be followed by a whitespace.
It should not be followed immediately by another command
or by a comment mark `|%|'.
This is because the \TeX{} parser reads the token immediately following
the argument of |\childdocmain| and puts it
at the beginning of every child section;
however, a white\-space is ignored.
\end{itemize}

%%%%%%%%%%%%%%%%%%%%%%%%%%%%%%%%%%%%%%%%
\paragraph{Content of Main File.}

It is advisable to place all content in the child files included by |\include|.
Any output contained in the main file will appear in all child documents
unless suppressed manually;
it cannot be suppressed automatically by the |\includeonly| directive
and thus should normally be avoided.
A method to include some content in the main file
by means of conditional processing is described in \secref{sec:conditional}.

%%%%%%%%%%%%%%%%%%%%%%%%%%%%%%%%%%%%%%%%
\paragraph{Page Numbering.}

When only a part of the document is compiled,
the appropriate numbering of pages
(as well as other status parameters)
is determined from the |.aux| files.
The latter contain information from previous passes.
However this information needs to propagate through
all intermediate child documents.
Therefore the page numbering in child documents may well
be inconsistent until the complete document is compiled at least once.

A useful (if unconventional) way to always ensure a consistent
page numbering is to restart the numbering in each child document
and denote the pages by `\textit{child}|.|\textit{page}'
where \textit{child} represents the chapter/section number of the child file.
This can be achieved by the command
|\numberwithin{page}{|\textit{child}|}|
of the \textsf{amsmath} package
where \textit{child} can be |chapter| or |section|
depending on the chosen structuring.
Alternatively, one can modify the macro |\thepage| appropriately
and reset the counter |page| at the start of each child file.

%%%%%%%%%%%%%%%%%%%%%%%%%%%%%%%%%%%%%%%%%%%%%%%%%%%%%%%%%%%%%%%%%%%%%%%%%%%%%%%%
\subsection{Conditional Processing}
\label{sec:conditional}

The package provides a mechanism to compile different versions
of a document. To customise the versions further some conditional processing
can come in handy to distinguish which version is being compiled.
The package provides two macros to describe the compilation context:

%%%%%%%%%%%%%%%%%%%%%%%%%%%%%%%%%%%%%%%%
\DescribeMacro{\ifchilddoc}
The conditional |\ifchilddoc| distinguishes between the compilation of
child documents and the main document:
%
\begin{center}
|\ifchilddoc |\textit{child-code}| |[|\||else |\textit{main-code}]| \||fi|
\end{center}

%%%%%%%%%%%%%%%%%%%%%%%%%%%%%%%%%%%%%%%%
\DescribeMacro{\childdocname}
\DescribeMacro{\childdocjob}
The macro |\childdocname| contains the filename (without extension)
of the main or child file being processed.
Note that |\childdocjob| will always contain the name of the main file.

%%%%%%%%%%%%%%%%%%%%%%%%%%%%%%%%%%%%%%%%
\paragraph{Title Page.}

Conditional processing can be used to include a title or banner page
in the main document when proper precautions are taken.
Importantly, the code in the main file should ensure that the page counter
(as well as other status parameters which are stored in the |.aux| files)
takes the same value after the conditional processing.
Otherwise the page numbers may take divergent values
depending on which part is compiled.

For example, a title page could be declared by:
%
\begin{center}
\begin{tabular}{l}
|\ifchilddoc\||else|\\
|\addtocounter{page}{-1}|\\
\textit{code for title page}\\
|\newpage|\\
|\||fi|
\end{tabular}
\end{center}
%
A banner page for the child documents can be generated by:
%
\begin{center}
\begin{tabular}{l}
|\ifchilddoc|\\
|\addtocounter{page}{-1}|\\
\textit{code for banner page}\\
|\newpage|\\
|\||fi|
\end{tabular}
\end{center}
%
Here one could write a message such as:
\begin{center}
|This is the part \childdocname{} of \childdocjob{}.|
\end{center}

%%%%%%%%%%%%%%%%%%%%%%%%%%%%%%%%%%%%%%%%%%%%%%%%%%%%%%%%%%%%%%%%%%%%%%%%%%%%%%%%
\subsection{Flags}
\label{sec:flags}

The package makes it easy to generate different versions
of the main or child documents.
To this end compilation flags can be defined
and assigned different default values.
They will be particularly useful in conjunction
with the forwarding mechanism described in \secref{sec:forward}.

For example, it may be useful to have a flag |\version|
which can be set to |draft| or |final|.
The document source will contain some conditional code
depending on the value of |\version|.
Suppose further, the flag should default to |final| for the main file
and to |draft| for child files
which is a natural assignment for editing the document.
This is achieved by placing the following code
in the preamble of the main document
(below the |\childdocmain| directive):
%
\begin{center}
\begin{tabular}{l}
|\ifchilddoc|\\
|\providecommand{\version}{draft}|\\
|\||else|\\
|\providecommand{\version}{final}|\\
|\||fi|
\end{tabular}
\end{center}
%
The definition by |\providecommand| makes sure
that previous definitions are not overwritten.
Further statements |\providecommand{\version}{...}|
can thus be added before the above code to override it.

For the main file, one might add a line
(between |\childdocmain| and the above block)
%
\begin{center}
|%\ifchilddoc\||else\providecommand{\version}{draft}\||fi|
\end{center}
%
which can be uncommented to produce a draft version.
Likewise one can add a line to the very top of a child file
(above the |\childdocof{|\textit{main}|}| directive)
%
\begin{center}
|%\providecommand{\version}{final}|
\end{center}
%
which can be uncommented to produce the final version of this child document.

%%%%%%%%%%%%%%%%%%%%%%%%%%%%%%%%%%%%%%%%%%%%%%%%%%%%%%%%%%%%%%%%%%%%%%%%%%%%%%%%
\subsection{Forwarding}
\label{sec:forward}

Different versions of the main or child documents
using compilation flags as described in \secref{sec:flags}
can be (permanently) stored in different files
for convenient compilation, viewing and distribution.
To this end, the package defines a command
to pass on compilation to a different file:

%%%%%%%%%%%%%%%%%%%%%%%%%%%%%%%%%%%%%%%%
\DescribeMacro{\childdocforward}
The command |\childdocforward| redirects processing to
another source file:
%
\begin{center}
\begin{tabular}{l}
|% \iffalse
%
% childdoc.dtx Copyright (C) 2017-2018 Niklas Beisert
%
% This work may be distributed and/or modified under the
% conditions of the LaTeX Project Public License, either version 1.3
% of this license or (at your option) any later version.
% The latest version of this license is in
%   http://www.latex-project.org/lppl.txt
% and version 1.3 or later is part of all distributions of LaTeX
% version 2005/12/01 or later.
%
% This work has the LPPL maintenance status `maintained'.
%
% The Current Maintainer of this work is Niklas Beisert.
%
% This work consists of the files childdoc.dtx and childdoc.ins
% and the derived files childdoc.def and cdocsamp.tex with
% cdocsch1.tex, cdocsch2.tex, cdocsdrf.tex, cdocsfn1.tex, cdocsfn2.tex.
%
%<package>\ifdefined\childdocmain\endinput\fi
%<package>\ProvidesFile{childdoc.def}[2018/12/30 v2.0 child document driver]
%<samplemain>\ProvidesFile{cdocsamp.tex}[2018/12/30 v2.0 sample for childdoc]
%<*driver>
%\ProvidesFile{childdoc.drv}[2018/12/30 v2.0 childdoc reference manual file]
\PassOptionsToClass{10pt,a4paper}{article}
\documentclass{ltxdoc}

\usepackage[margin=35mm]{geometry}
\usepackage{hyperref}
\usepackage{hyperxmp}
\usepackage[usenames]{color}

\hypersetup{colorlinks=true}
\hypersetup{pdfstartview=FitH}
\hypersetup{pdfpagemode=UseNone}
\hypersetup{pdfsource={}}
\hypersetup{pdflang={en-UK}}
\hypersetup{pdfcopyright={Copyright 2017-2018 Niklas Beisert.
  This work may be distributed and/or modified under the
  conditions of the LaTeX Project Public License, either version 1.3
  of this license or (at your option) any later version.}}
\hypersetup{pdflicenseurl={http://www.latex-project.org/lppl.txt}}
\hypersetup{pdfcontactaddress={ETH Zurich, ITP, HIT K,
  Wolfgang-Pauli-Strasse 27}}
\hypersetup{pdfcontactpostcode={8093}}
\hypersetup{pdfcontactcity={Zurich}}
\hypersetup{pdfcontactcountry={Switzerland}}
\hypersetup{pdfcontactemail={nbeisert@itp.phys.ethz.ch}}
\hypersetup{pdfcontacturl={http://people.phys.ethz.ch/\xmptilde nbeisert/}}

\newcommand{\secref}[1]{\hyperref[#1]{section \ref*{#1}}}

\parskip1ex
\parindent0pt
\let\olditemize\itemize
\def\itemize{\olditemize\parskip0pt}

\begin{document}

\title{The \textsf{childdoc} Package}
\hypersetup{pdftitle={The childdoc Package}}
\author{Niklas Beisert\\[2ex]
  Institut f\"ur Theoretische Physik\\
  Eidgen\"ossische Technische Hochschule Z\"urich\\
  Wolfgang-Pauli-Strasse 27, 8093 Z\"urich, Switzerland\\[1ex]
  \href{mailto:nbeisert@itp.phys.ethz.ch}
  {\texttt{nbeisert@itp.phys.ethz.ch}}}
\hypersetup{pdfauthor={Niklas Beisert}}
\hypersetup{pdfsubject={Manual for the LaTeX2e Package childdoc}}
\date{30 December 2018, \textsf{v2.0}}
\maketitle

\begin{abstract}\noindent
\textsf{childdoc} is a \LaTeXe{} package
that enables the direct compilation
of document sections included by |\include|
to individual files.
\end{abstract}

\begingroup
\parskip0ex
\tableofcontents
\endgroup

%%%%%%%%%%%%%%%%%%%%%%%%%%%%%%%%%%%%%%%%%%%%%%%%%%%%%%%%%%%%%%%%%%%%%%%%%%%%%%%%
%%%%%%%%%%%%%%%%%%%%%%%%%%%%%%%%%%%%%%%%%%%%%%%%%%%%%%%%%%%%%%%%%%%%%%%%%%%%%%%%
\section{Introduction}

\LaTeX{} provides a mechanism to structure a large document (such as a book)
into a main file and several child files (containing the chapters)
using the |\include| command.
This mechanism is beneficial for documents
which span hundreds of pages in order to
make the source file(s) more manageable.
Moreover, compilation can be restricted to
selected child files by means of the |\includeonly| command.
The latter feature can be used to reduce the compilation time while editing
(this was significantly more useful in the earlier days of \LaTeX{})
or to generate a smaller document which is easier to navigate.
Another application of |\includeonly| is to generate
documents consisting of selected parts of the complete document.

However, there are a few drawbacks of the plain |\include| mechanism:
\begin{itemize}
\item
The child files cannot be compiled on their own,
they can only be compiled via the main file.
A naive editing environment
(such as a text editor with an option
to have the current file processed by \LaTeX)
may require one to switch to the main file before compiling;
attempting to compile the child file produces errors.
\item
The main file must be modified (each time)
to adjust the |\includeonly| command
to the present needs. This easily leaves the main file in a messy state.
\item
The generated document will always carry the filename
of the main document. This is inconvenient if
several child files are to be compiled and
to be kept for distribution.
\end{itemize}

The present package provides a simple interface
to make child files individually compilable by \LaTeX{}.
Compiling a child file then has the same effect as compiling
the main file with an |\includeonly| command
to select the appropriate child.
Moreover the generated document will carry the name of the child
rather than the main file.
This resolves all three above issues.

This feature is meant to make the editing of books,
thesis documents and lecture notes somewhat more convenient.
However, the package can also be used efficiently for
composing a series of documents (such as exercise sheets)
which are typically distributed individually.
It then assists the author in generating the individual documents
(potentially in different versions)
as well as a document containing the collected series.
Another application is in developing style files
or other kinds of included material
where compilation of the style file could redirect
to a sample or test file.

%%%%%%%%%%%%%%%%%%%%%%%%%%%%%%%%%%%%%%%%%%%%%%%%%%%%%%%%%%%%%%%%%%%%%%%%%%%%%%%%
%%%%%%%%%%%%%%%%%%%%%%%%%%%%%%%%%%%%%%%%%%%%%%%%%%%%%%%%%%%%%%%%%%%%%%%%%%%%%%%%
\section{Usage}

First of all, the package \textsf{childdoc} is \emph{not} a standard
\LaTeXe{} |.sty| style file! Therefore it needs to be invoked in
a non-standard way.

%%%%%%%%%%%%%%%%%%%%%%%%%%%%%%%%%%%%%%%%%%%%%%%%%%%%%%%%%%%%%%%%%%%%%%%%%%%%%%%%
\subsection{Included Files}
\label{sec:include}

%%%%%%%%%%%%%%%%%%%%%%%%%%%%%%%%%%%%%%%%
\DescribeMacro{\childdocmain}
To use the package, add the commands
\begin{center}
\begin{tabular}{l}
|\input{childdoc.def}|\\
|\childdocmain{}|\\
\end{tabular}
\end{center}
at the very top of the main \LaTeX{} file,
in particular \emph{before} the |\documentclass| statement!
The argument of |\childdocmain| should be left empty
(but it must be present).

%%%%%%%%%%%%%%%%%%%%%%%%%%%%%%%%%%%%%%%%
\DescribeMacro{\childdocof}
Furthermore, add the commands
\begin{center}
\begin{tabular}{l}
|\input{childdoc.def}|\\
|\childdocof{|\textit{main}|}|\\
\end{tabular}
\end{center}
at the top of every child file \textit{child}
which is included by |\include{|\textit{child}|}|
from within the main file
(or at least for those files to be compiled individually).
The argument \textit{main} must be the filename of the main file.

There are a couple of
considerations in setting up the main and child documents:

%%%%%%%%%%%%%%%%%%%%%%%%%%%%%%%%%%%%%%%%
\paragraph{Restrictions.}

Please note the following restrictions:
\begin{itemize}
\item
|\childdocmain| must be called with one argument \textit{main}
to ensure compatibility with earlier version of the package.
It must either be empty (|\childdocmain{}|)
or precisely match the filename of the main file in which it is specified.
See \secref{sec:detection} for further information.
\item
The filename \textit{main} must be specified without the |.tex| extension.
\item
The filename \textit{main} is case sensitive
(even in case-insensitive file systems)
due to internal string comparison.
\item
The argument \textit{main} should be fully expanded, it cannot be a macro.
\item
Subdirectories and special characters should be avoided in filenames.
\item
The command |\childdocmain{|\textit{main}|}| must be followed by a whitespace.
It should not be followed immediately by another command
or by a comment mark `|%|'.
This is because the \TeX{} parser reads the token immediately following
the argument of |\childdocmain| and puts it
at the beginning of every child section;
however, a white\-space is ignored.
\end{itemize}

%%%%%%%%%%%%%%%%%%%%%%%%%%%%%%%%%%%%%%%%
\paragraph{Content of Main File.}

It is advisable to place all content in the child files included by |\include|.
Any output contained in the main file will appear in all child documents
unless suppressed manually;
it cannot be suppressed automatically by the |\includeonly| directive
and thus should normally be avoided.
A method to include some content in the main file
by means of conditional processing is described in \secref{sec:conditional}.

%%%%%%%%%%%%%%%%%%%%%%%%%%%%%%%%%%%%%%%%
\paragraph{Page Numbering.}

When only a part of the document is compiled,
the appropriate numbering of pages
(as well as other status parameters)
is determined from the |.aux| files.
The latter contain information from previous passes.
However this information needs to propagate through
all intermediate child documents.
Therefore the page numbering in child documents may well
be inconsistent until the complete document is compiled at least once.

A useful (if unconventional) way to always ensure a consistent
page numbering is to restart the numbering in each child document
and denote the pages by `\textit{child}|.|\textit{page}'
where \textit{child} represents the chapter/section number of the child file.
This can be achieved by the command
|\numberwithin{page}{|\textit{child}|}|
of the \textsf{amsmath} package
where \textit{child} can be |chapter| or |section|
depending on the chosen structuring.
Alternatively, one can modify the macro |\thepage| appropriately
and reset the counter |page| at the start of each child file.

%%%%%%%%%%%%%%%%%%%%%%%%%%%%%%%%%%%%%%%%%%%%%%%%%%%%%%%%%%%%%%%%%%%%%%%%%%%%%%%%
\subsection{Conditional Processing}
\label{sec:conditional}

The package provides a mechanism to compile different versions
of a document. To customise the versions further some conditional processing
can come in handy to distinguish which version is being compiled.
The package provides two macros to describe the compilation context:

%%%%%%%%%%%%%%%%%%%%%%%%%%%%%%%%%%%%%%%%
\DescribeMacro{\ifchilddoc}
The conditional |\ifchilddoc| distinguishes between the compilation of
child documents and the main document:
%
\begin{center}
|\ifchilddoc |\textit{child-code}| |[|\||else |\textit{main-code}]| \||fi|
\end{center}

%%%%%%%%%%%%%%%%%%%%%%%%%%%%%%%%%%%%%%%%
\DescribeMacro{\childdocname}
\DescribeMacro{\childdocjob}
The macro |\childdocname| contains the filename (without extension)
of the main or child file being processed.
Note that |\childdocjob| will always contain the name of the main file.

%%%%%%%%%%%%%%%%%%%%%%%%%%%%%%%%%%%%%%%%
\paragraph{Title Page.}

Conditional processing can be used to include a title or banner page
in the main document when proper precautions are taken.
Importantly, the code in the main file should ensure that the page counter
(as well as other status parameters which are stored in the |.aux| files)
takes the same value after the conditional processing.
Otherwise the page numbers may take divergent values
depending on which part is compiled.

For example, a title page could be declared by:
%
\begin{center}
\begin{tabular}{l}
|\ifchilddoc\||else|\\
|\addtocounter{page}{-1}|\\
\textit{code for title page}\\
|\newpage|\\
|\||fi|
\end{tabular}
\end{center}
%
A banner page for the child documents can be generated by:
%
\begin{center}
\begin{tabular}{l}
|\ifchilddoc|\\
|\addtocounter{page}{-1}|\\
\textit{code for banner page}\\
|\newpage|\\
|\||fi|
\end{tabular}
\end{center}
%
Here one could write a message such as:
\begin{center}
|This is the part \childdocname{} of \childdocjob{}.|
\end{center}

%%%%%%%%%%%%%%%%%%%%%%%%%%%%%%%%%%%%%%%%%%%%%%%%%%%%%%%%%%%%%%%%%%%%%%%%%%%%%%%%
\subsection{Flags}
\label{sec:flags}

The package makes it easy to generate different versions
of the main or child documents.
To this end compilation flags can be defined
and assigned different default values.
They will be particularly useful in conjunction
with the forwarding mechanism described in \secref{sec:forward}.

For example, it may be useful to have a flag |\version|
which can be set to |draft| or |final|.
The document source will contain some conditional code
depending on the value of |\version|.
Suppose further, the flag should default to |final| for the main file
and to |draft| for child files
which is a natural assignment for editing the document.
This is achieved by placing the following code
in the preamble of the main document
(below the |\childdocmain| directive):
%
\begin{center}
\begin{tabular}{l}
|\ifchilddoc|\\
|\providecommand{\version}{draft}|\\
|\||else|\\
|\providecommand{\version}{final}|\\
|\||fi|
\end{tabular}
\end{center}
%
The definition by |\providecommand| makes sure
that previous definitions are not overwritten.
Further statements |\providecommand{\version}{...}|
can thus be added before the above code to override it.

For the main file, one might add a line
(between |\childdocmain| and the above block)
%
\begin{center}
|%\ifchilddoc\||else\providecommand{\version}{draft}\||fi|
\end{center}
%
which can be uncommented to produce a draft version.
Likewise one can add a line to the very top of a child file
(above the |\childdocof{|\textit{main}|}| directive)
%
\begin{center}
|%\providecommand{\version}{final}|
\end{center}
%
which can be uncommented to produce the final version of this child document.

%%%%%%%%%%%%%%%%%%%%%%%%%%%%%%%%%%%%%%%%%%%%%%%%%%%%%%%%%%%%%%%%%%%%%%%%%%%%%%%%
\subsection{Forwarding}
\label{sec:forward}

Different versions of the main or child documents
using compilation flags as described in \secref{sec:flags}
can be (permanently) stored in different files
for convenient compilation, viewing and distribution.
To this end, the package defines a command
to pass on compilation to a different file:

%%%%%%%%%%%%%%%%%%%%%%%%%%%%%%%%%%%%%%%%
\DescribeMacro{\childdocforward}
The command |\childdocforward| redirects processing to
another source file:
%
\begin{center}
\begin{tabular}{l}
|\input{childdoc.def}|\\
|\childdocforward[|\textit{main}|]{|\textit{dest}|}|\\
\end{tabular}
\end{center}
%
The argument \textit{dest} is the destination file
(without extension).
It should be the main file or one of the child files.
Note that further \textsf{childdoc} directives
such as |\childdocof| and |\childdocforward|
in the indicated file will be processed in this form.
The optional argument \textit{main}
passes on directly to the main file \textit{main}
while pretending to compile the child \textit{dest}.
This form behaves as if \textit{dest}
issues |\childdocof{|\textit{main}|}| right away,
and no further \textsf{childdoc} directives will be processed.

%%%%%%%%%%%%%%%%%%%%%%%%%%%%%%%%%%%%%%%%
\DescribeMacro{\...prefix}
In the alternative form |\childdocforwardprefix|,
%
\begin{center}
\begin{tabular}{l}
|\input{childdoc.def}|\\
|\childdocforwardprefix[|\textit{main}|]{|\textit{prefix}|}{|\textit{dest}|}|
\end{tabular}
\end{center}
%
the destination file is determined by a pattern
depending on the current file:
To make this work, the current file must be called
`{\textit{prefix}\hspace{0.2em}\textit{suffix}}'
with \textit{prefix} matching precisely the argument.
Processing is then passed on to the file
`{\textit{dest}\hspace{0.2em}\textit{suffix}}'.
Surely, the same effect is achieved by
directly specifying the
argument `{\textit{dest}\hspace{0.2em}\textit{suffix}}'
in the first form.
However, that requires to set up a different file
for each child. With the alternative form of the command
all these files can have exactly the same content
which simplifies setting them up and maintaining them.

For example, the following file |draft.tex|
with a compilation flag |\version| as described in \secref{sec:flags}
compiles the main document as a draft:
%
\begin{center}
\begin{tabular}{l}
|\def\version{draft}|\\
|\input{childdoc.def}|\\
|\childdocforward{|\textit{main}|}|
\end{tabular}
\end{center}
%
Likewise, the following files |final|\textit{nn}|.tex|
compile the final version of the child document
|child|\textit{nn}|.tex|:
%
\begin{center}
\begin{tabular}{l}
|\def\version{final}|\\
|\input{childdoc.def}|\\
|\childdocforwardprefix{final}{child}|
\end{tabular}
\end{center}
%

Note that when several versions of a main file and/or of each child file
are to be generated, it may be convenient to set up a |Makefile| or
shell script to automatise the process.

%%%%%%%%%%%%%%%%%%%%%%%%%%%%%%%%%%%%%%%%%%%%%%%%%%%%%%%%%%%%%%%%%%%%%%%%%%%%%%%%
\subsection{Command Line Processing}
\label{sec:commandline}

The effect of redirection files can also be achieved by invoking
the \LaTeX{} compiler with a more elaborate command line.
Most conveniently this should be done as part
of a shell script or a |Makefile|.

When using \textsf{childdoc} in the main file, the following
command lines effectively perform a redirection
(note that depending on the shell being used,
backslashes may have to be doubled: `|\|' $\to$ `|\\|'):
%
\begin{center}
|... -jobname "|\textit{target}|" |\\|"|[\textit{flags}]%
|\input{childdoc.def}\childdocforward[|\textit{main}|]{|\textit{dest}|}"|
\end{center}
%
Here \textit{target} is the name of the output file,
\textit{main} is the name of the main file
and \textit{dest} is the name of the main or child file to be processed
(all filenames without extensions).
The optional argument \textit{main} can be omitted
if \textit{main} matches \textit{dest}.
Optionally, compilation \textit{flags} can be defined via |\def| commands.
This command line makes the \TeX{} engine believe
it is compiling the file \textit{target}
whose content is specified as the latter parameter.
The provided code then forwards the processing to
\textit{main} or \textit{dest} as described in \secref{sec:forward}.

%%%%%%%%%%%%%%%%%%%%%%%%%%%%%%%%%%%%%%%%%%%%%%%%%%%%%%%%%%%%%%%%%%%%%%%%%%%%%%%%
\subsection{Include by Input}
\label{sec:input}

Including child documents by |\include| has some restrictions by design.
Most notably, the content of a child document always occupies
its own set of pages; pages cannot be shared between child documents.
Usually, this behaviour makes perfect sense
because each child document contain an essential part of the document.
However, in some situations it may be desirable to compose
a document from a collection of parts
without having mandatory page breaks between then.
For this case, the package
provides a mechanism to include parts
by |\input| which can also be processed individually.
However, by construction this mechanism
requires manual handling of the content to be output.

%%%%%%%%%%%%%%%%%%%%%%%%%%%%%%%%%%%%%%%%
\DescribeMacro{\ifchilddocmanual}
The main file should be prepared as usual, see \secref{sec:include}.
However, the document body must make a distinction
between processing of an individual part and of the main document, e.g.:
%
\begin{center}
\begin{tabular}{l}
|\ifchilddocmanual|\\
|\input{\childdocname}|\\
|\||else|\\
\textit{document body with }|\input{|\textit{part}|}|\\
|\||fi|
\end{tabular}
\end{center}
%
The conditional |\ifchilddocmanual| is true whenever
a part to be included by |\input| is being compiled,
and the name of the part is stored in |\childdocname|.

%%%%%%%%%%%%%%%%%%%%%%%%%%%%%%%%%%%%%%%%
\DescribeMacro{\childdocby}
Each part to be included by |\input| should start with:
%
\begin{center}
\begin{tabular}{l}
|\input{childdoc.def}|\\
|\childdocby{|\textit{main}|}|\\
\end{tabular}
\end{center}
%
The directive |\childdocby| is similar to |\childdocof|
described in \secref{sec:include},
but the subsequent selection of content must be done manually.
To that end, both |\ifchilddoc| and |\ifchilddocmanual|
will be true upon processing of a part,
and the name of the part is stored in |\childdocname|.
Note that |\jobname| will be set to the filename of the current part
so that each part receives an individual |.aux| file
that does not interfere with the |.aux| file(s) of the main document.
This behaviour can be altered by the alternative form
|\childdocby[*]{|\textit{main}|}| (with a non-empty optional argument)
which uses the |.aux| file of the main document
by setting |\jobname| to \textit{main}.

%%%%%%%%%%%%%%%%%%%%%%%%%%%%%%%%%%%%%%%%%%%%%%%%%%%%%%%%%%%%%%%%%%%%%%%%%%%%%%%%
\subsection{Driver Development}
\label{sec:driver}

The \textsf{childdoc} mechanism can also be use for the development
of definition files such as \LaTeX{} styles or classes.
This case differs from the above setup with multiple parts
included by |\include| in that no |\includeonly| should be invoked.
This can be achieved by starting the include file
(before |\ProvidesPackage|) with:
%
\begin{center}
\begin{tabular}{l}
|\input{childdoc.def}|\\
|\childdocforward{|\textit{main}|}|\\
\end{tabular}
\end{center}
%
or alternatively with:
%
\begin{center}
\begin{tabular}{l}
|\input{childdoc.def}|\\
|\childdocby{|\textit{main}|}|\\
\end{tabular}
\end{center}
%
Both forms have slightly different effects as described above.
The main file is prepared as usual, see \secref{sec:include}.

%%%%%%%%%%%%%%%%%%%%%%%%%%%%%%%%%%%%%%%%%%%%%%%%%%%%%%%%%%%%%%%%%%%%%%%%%%%%%%%%
\subsection{Legacy Detection}
\label{sec:detection}

The directive |\childdocmain| in the main file can detect
whether the complete document or merely a child is to be compiled
even without using the directive |\childdocof|.
This method is deprecated because it is less robust
and there is no compelling reason to use it;
it is merely provided for backward compatibility
and it may be removed in future versions.

If the detection mechanism is to be used,
it is mandatory to correctly specify
the filename of the main file as the argument of |\childdocmain|:
%
\begin{center}
\begin{tabular}{l}
|\input{childdoc.def}|\\
|\childdocmain{|\textit{main}|}|\\
\end{tabular}
\end{center}
%
If |\jobname| does not match the argument \textit{main} of |\childdocmain|,
it is assumed that |\jobname| points to the child file to be compiled.
When using |\childdocmain| with the main file specified as argument,
it suffices to start a child file
with just |\input{|\textit{main}|}|
without loading of the package and using |\childdocof|.
If instead all processing is done
with the appropriate \textsf{childdoc} directives,
the argument of \textit{main} of |\childdocmain| can be empty.

An alternative version of the command line processing described
in \secref{sec:commandline} using the detection mechanism reads:
%
\begin{center}
|... -jobname "|\textit{target}|" "|[\textit{flags}]%
[|\def\jobname{|\textit{dest}|}|]|\input{|\textit{main}|}"|
\end{center}

%%%%%%%%%%%%%%%%%%%%%%%%%%%%%%%%%%%%%%%%%%%%%%%%%%%%%%%%%%%%%%%%%%%%%%%%%%%%%%%%
\subsection{Manual Code}
\label{sec:manual}

In case one cannot be certain whether the definitions file |childdoc.def|
is installed on the target \TeX{} distribution
and one prefers not to ship it,
it is conceivable to paste a few relevant commands into the sources.

To that end, drop all statements |\input{childdoc.def}|
and perform the replacements as outlined below.
Instead of |\childdocmain{|\textit{main}|}| add the following code
to the top of the main file:
%
\begin{center}
\begin{tabular}{l}
|\||ifdefined\childdocname\endinput\||fi\newif\ifchilddoc|\\
|\edef\childdocname{\scantokens\expandafter{\jobname\noexpand}}|\\
|\def\childdocmain{|\textit{main}|}\||ifx\childdocmain\childdocname\||else|\\
|\childdoctrue\includeonly{\childdocname}\let\jobname\childdocmain\||fi|\\
\end{tabular}
\end{center}
%
Instead of |\childdocof{|\textit{main}|}| just include the main file
at the top of each child file:
%
\begin{center}
|\input{|\textit{main}|}|
\end{center}
%
A simple redirection |\childdocforward{|\textit{dest}|}| is achieved by:
%
\begin{center}
|\def\jobname{|\textit{dest}|}\input{\jobname}|
\end{center}
%
The redirection with prefix
|\childdocforwardprefix[|\textit{prefix}|]{|\textit{dest}|}|
is accomplished by:
%
\begin{center}
\begin{tabular}{l}
|{\edef\jobname{\scantokens\expandafter{\jobname\noexpand}}|\\
|\def\redirectjob |\textit{prefix}|#1~~~{\gdef\jobname{|\textit{dest}|#1}}|\\
|\expandafter\redirectjob\jobname~~~}\input{\jobname}|
\end{tabular}
\end{center}

In an alternative approach,
child documents can be compiled by a specific command line
without additional code or specific definitions:
%
\begin{center}
|... -jobname "|\textit{target}|" "|[\textit{flags}]%
|\includeonly{|\textit{dest}|}\input{|\textit{main}|}"|
\end{center}
%

%%%%%%%%%%%%%%%%%%%%%%%%%%%%%%%%%%%%%%%%%%%%%%%%%%%%%%%%%%%%%%%%%%%%%%%%%%%%%%%%
%%%%%%%%%%%%%%%%%%%%%%%%%%%%%%%%%%%%%%%%%%%%%%%%%%%%%%%%%%%%%%%%%%%%%%%%%%%%%%%%
\section{Information}

%%%%%%%%%%%%%%%%%%%%%%%%%%%%%%%%%%%%%%%%%%%%%%%%%%%%%%%%%%%%%%%%%%%%%%%%%%%%%%%%
\subsection{Copyright}

Copyright \copyright{} 2017--2018 Niklas Beisert

This work may be distributed and/or modified under the
conditions of the \LaTeX{} Project Public License, either version 1.3
of this license or (at your option) any later version.
The latest version of this license is in
  \url{http://www.latex-project.org/lppl.txt}
and version 1.3 or later is part of all distributions of \LaTeX{}
version 2005/12/01 or later.

This work has the LPPL maintenance status `maintained'.

The Current Maintainer of this work is Niklas Beisert.

This work consists of the files |README.txt|, |childdoc.ins| and |childdoc.dtx|
as well as the derived files |childdoc.def|, |cdocsamp.tex|
with |cdocsch1.tex|, |cdocsch2.tex|, |cdocspt3.tex|, |cdocspt4.tex|,
|cdocsdrf.tex|, |cdocsfn1.tex|, |cdocsfn2.tex|
as well as |childdoc.pdf|.

%%%%%%%%%%%%%%%%%%%%%%%%%%%%%%%%%%%%%%%%%%%%%%%%%%%%%%%%%%%%%%%%%%%%%%%%%%%%%%%%
\subsection{Files and Installation}

The package consists of the files:
%
\begin{center}
\begin{tabular}{ll}
    |README.txt|   & readme file \\
    |childdoc.ins| & installation file \\
    |childdoc.dtx| & source file \\
    |childdoc.def| & definition file \\
    |cdocsamp.tex| & sample main file \\
    |cdocsch1.tex| & sample include file \\
    |cdocsch2.tex| & sample include file \\
    |cdocspt3.tex| & sample part file \\
    |cdocspt4.tex| & sample part file \\
    |cdocsdrf.tex| & sample redirection file \\
    |cdocsfn1.tex| & sample redirection file \\
    |cdocsfn2.tex| & sample redirection file \\
    |childdoc.pdf| & manual
\end{tabular}
\end{center}
%
The distribution consists of the files
|README.txt|, |childdoc.ins| and |childdoc.dtx|.
%
\begin{itemize}
\item
Run (pdf)\LaTeX{} on |childdoc.dtx|
to compile the manual |childdoc.pdf| (this file).
\item
Run \LaTeX{} on |childdoc.ins| to create the definitions file |childdoc.def|
and the sample |cdocsamp.tex| with include files
|cdocsch1.tex|, |cdocsch2.tex|, |cdocspt3.tex|, |cdocspt4.tex|,
|cdocsdrf.tex|, |cdocsfn1.tex|, |cdocsfn2.tex|.
Then copy the file |childdoc.def| to an appropriate directory of your \LaTeX{}
distribution, e.g.\ \textit{texmf-root}|/tex/latex/childdoc|.
\end{itemize}

%%%%%%%%%%%%%%%%%%%%%%%%%%%%%%%%%%%%%%%%%%%%%%%%%%%%%%%%%%%%%%%%%%%%%%%%%%%%%%%%
\subsection{Related CTAN Packages}

There are several other packages which offer a similar functionality:
%
\begin{itemize}
\item
The packages
\href{http://ctan.org/pkg/docmute}{\textsf{docmute}},
\href{http://ctan.org/pkg/includex}{\textsf{includex}} and
\href{http://ctan.org/pkg/standalone}{\textsf{standalone}}
provide commands to include only the document body of
a child file thus allowing both files to be compiled individually.
\item
The packages \href{http://ctan.org/pkg/subdocs}{\textsf{subdocs}}
and \href{http://ctan.org/pkg/subfiles}{\textsf{subfiles}}
provide structures in which the main and child documents can be
encapsulated and allowing them to be compiled individually.
The inclusion mechanism is different from the conventional |\include|.
\item
The package \href{http://ctan.org/pkg/combine}{\textsf{combine}}
is an elaborate solution to combine several documents into one.
\end{itemize}
%
See also the CTAN topic \href{http://ctan.org/topic/subdocs}{\textsf{subdocs}}
for further related packages.
The present package differs from the above solutions in that
a document structure constructed with the conventional |\include| mechanism
just needs two extra commands at the top of every file
such that all constituent files can be compiled individually.

%%%%%%%%%%%%%%%%%%%%%%%%%%%%%%%%%%%%%%%%%%%%%%%%%%%%%%%%%%%%%%%%%%%%%%%%%%%%%%%%
%\subsection{Feature Suggestions}
%
%The following is a list of features which may be useful for future
%versions of this package:
%%
%\begin{itemize}
%\item
%\ldots
%\end{itemize}

%%%%%%%%%%%%%%%%%%%%%%%%%%%%%%%%%%%%%%%%%%%%%%%%%%%%%%%%%%%%%%%%%%%%%%%%%%%%%%%%
\subsection{Revision History}

%%%%%%%%%%%%%%%%%%%%%%%%%%%%%%%%%%%%%%%%
\paragraph{v2.0:} 2018/12/30

\begin{itemize}
\item
immediate forward processing
\item
added |\childdocby| mechanism
\item
manual restructured
\end{itemize}

%%%%%%%%%%%%%%%%%%%%%%%%%%%%%%%%%%%%%%%%
\paragraph{v1.6:} 2018/01/17

\begin{itemize}
\item
application for development of include files
\item
corrections to manual
\end{itemize}

%%%%%%%%%%%%%%%%%%%%%%%%%%%%%%%%%%%%%%%%
\paragraph{v1.5:} 2017/05/21

\begin{itemize}
\item
more complete structuring introduced
\item
|\childdocof| introduced
\item
|\childdoc| renamed to |\childdocmain|
\item
|\childredirect| renamed to |\childdocforward| and |\childdocforwardprefix|
and functionality expanded
\end{itemize}

%%%%%%%%%%%%%%%%%%%%%%%%%%%%%%%%%%%%%%%%
\paragraph{v1.0:} 2017/04/27

\begin{itemize}
\item
manual and install package
\item
first version published on CTAN
\end{itemize}

%%%%%%%%%%%%%%%%%%%%%%%%%%%%%%%%%%%%%%%%
\paragraph{v0.6:} 2017/04/26

\begin{itemize}
\item
redirection mechanism added
\end{itemize}

%%%%%%%%%%%%%%%%%%%%%%%%%%%%%%%%%%%%%%%%
\paragraph{v0.5:} 2017/04/26

\begin{itemize}
\item
functionality in definition file
\end{itemize}


%%%%%%%%%%%%%%%%%%%%%%%%%%%%%%%%%%%%%%%%%%%%%%%%%%%%%%%%%%%%%%%%%%%%%%%%%%%%%%%%
%%%%%%%%%%%%%%%%%%%%%%%%%%%%%%%%%%%%%%%%%%%%%%%%%%%%%%%%%%%%%%%%%%%%%%%%%%%%%%%%
%%%%%%%%%%%%%%%%%%%%%%%%%%%%%%%%%%%%%%%%%%%%%%%%%%%%%%%%%%%%%%%%%%%%%%%%%%%%%%%%
\appendix

\settowidth\MacroIndent{\rmfamily\scriptsize 000\ }

 \DocInput{childdoc.dtx}

\end{document}
%</driver>
% \fi
%
% %%%%%%%%%%%%%%%%%%%%%%%%%%%%%%%%%%%%%%%%%%%%%%%%%%%%%%%%%%%%%%%%%%%%%%%%%%%%%%
% %%%%%%%%%%%%%%%%%%%%%%%%%%%%%%%%%%%%%%%%%%%%%%%%%%%%%%%%%%%%%%%%%%%%%%%%%%%%%%
% \section{Sample}
%\iffalse
%<*samplemain>
%\fi
%
% The following presents a sample document
% with two chapters, two parts, a title page,
% a compile flag as well as three forwarding files to set the flag.
% It consists of eight |.tex| files:
% \begin{center}
% \begin{tabular}{ll}
% |cdocsamp.tex|&main file\\
% |cdocsch1.tex|&include file for chapter 1\\
% |cdocsch2.tex|&include file for chapter 2\\
% |cdocspt3.tex|&include file for part 3\\
% |cdocspt4.tex|&include file for part 4\\
% |cdocsdrf.tex|&forwarding file for main file in draft mode\\
% |cdocsfi1.tex|&forwarding file for final version of chapter 1\\
% |cdocsfi2.tex|&forwarding file for final version of chapter 2\\
% \end{tabular}
% \end{center}
% Each of the eight files can be compiled directly by the \LaTeX{} compiler.
%
% %%%%%%%%%%%%%%%%%%%%%%%%%%%%%%%%%%%%%%
% \paragraph{Main File.}
%
% The main file is called |cdocsamp.tex|.
%
% Load the \textsf{childdoc} definitions and
% declare the filename for the main document:
%    \begin{macrocode}
\input{childdoc.def}
\childdocmain{}
%    \end{macrocode}

% Optional override for |\version| flag:
%    \begin{macrocode}
%%\ifchilddoc\else\providecommand{\version}{draft}\fi
%    \end{macrocode}

% Define the default values for the |\version| flag
% (|final| for the main file and |draft| for childs):
%    \begin{macrocode}
\ifchilddoc
\providecommand{\version}{draft}
\else
\providecommand{\version}{final}
\fi
%    \end{macrocode}

% Load the standard document class:
%    \begin{macrocode}
\documentclass[12pt]{article}
%    \end{macrocode}

% Start the document body:
%    \begin{macrocode}
\begin{document}
%    \end{macrocode}

% Declare a title page.
% Print title, part of document being processed and version flag:
%    \begin{macrocode}
\addtocounter{page}{-1}
\begin{center}
{\LARGE\bfseries{}childdoc example\par}
\vspace{1cm}
\ifchilddoc
\ifchilddocmanual part\else chapter\fi:
`\childdocname' of `\childdocjob'\par
\else
main document: `\childdocjob'\par
\fi
version: \version\par
\end{center}
\newpage
%    \end{macrocode}

% Manually include selected file,
% otherwise process as usual:
%    \begin{macrocode}
\ifchilddocmanual
\section*{part `\childdocname'}
\input{\childdocname}
\else
%    \end{macrocode}

% Include the two chapters:
%    \begin{macrocode}
\include{cdocsch1}
\include{cdocsch2}
%    \end{macrocode}

% Include the two parts unless only chapters should be displayed:
%    \begin{macrocode}
\ifchilddoc\else
\section{part three}
\input{cdocspt3}
\section{part four}
\input{cdocspt4}
\fi
%    \end{macrocode}

% Process as usual until here:
%    \begin{macrocode}
\fi
%    \end{macrocode}

% End of document body:
%    \begin{macrocode}
\end{document}
%    \end{macrocode}
%\iffalse
%</samplemain>
%\fi
%
% %%%%%%%%%%%%%%%%%%%%%%%%%%%%%%%%%%%%%%
% \paragraph{Chapter Include Files.}
%
% The include files are called |cdocsch1.tex| and |cdocsch2.tex|.
%
%\iffalse
%<*samplechap1|samplechap2>
%\fi

% Optional override for |\version| flag:
%    \begin{macrocode}
%%\providecommand{\version}{final}
%    \end{macrocode}

% Include the main document:
%    \begin{macrocode}
\input{childdoc.def}
\childdocof{cdocsamp}
%    \end{macrocode}

%\iffalse
%</samplechap1|samplechap2>
%\fi
%
%\iffalse
%<*samplechap1>
%\fi
% Some text for chapter 1:
%    \begin{macrocode}
\section{one}
some text in chapter one
%    \end{macrocode}

%\iffalse
%</samplechap1>
%\fi
% Some text for chapter 2:
%\iffalse
%<*samplechap2>
%\fi
%    \begin{macrocode}
\section{two}
more text in chapter two
%    \end{macrocode}

%\iffalse
%</samplechap2>
%\fi
%
% %%%%%%%%%%%%%%%%%%%%%%%%%%%%%%%%%%%%%%
% \paragraph{Part Include Files.}
%
% The include files are called |cdocspt3.tex| and |cdocspt4.tex|.
%
%\iffalse
%<*samplepart3|samplepart4>
%\fi

% Optional override for |\version| flag:
%    \begin{macrocode}
%%\providecommand{\version}{final}
%    \end{macrocode}

% Include the main document:
%    \begin{macrocode}
\input{childdoc.def}
\childdocby{cdocsamp}
%    \end{macrocode}

%\iffalse
%</samplepart3|samplepart4>
%\fi
%
%\iffalse
%<*samplepart3>
%\fi
% Some text for part 3:
%    \begin{macrocode}
some text in part three
%    \end{macrocode}

%\iffalse
%</samplepart3>
%\fi
% Some text for part 4:
%\iffalse
%<*samplepart4>
%\fi
%    \begin{macrocode}
more text in part four
%    \end{macrocode}

%\iffalse
%</samplepart4>
%\fi
%
% %%%%%%%%%%%%%%%%%%%%%%%%%%%%%%%%%%%%%%
% \paragraph{Forwarding for a Complete Draft.}
%
% The following forwarding file |cdocsdrf.tex|
% compiles the main document in draft mode:
%\iffalse
%<*sampledraft>
%\fi
%    \begin{macrocode}
\def\version{draft}
\input{childdoc.def}
\childdocforward{cdocsamp}
%    \end{macrocode}

%\iffalse
%</sampledraft>
%\fi
%
% %%%%%%%%%%%%%%%%%%%%%%%%%%%%%%%%%%%%%%
% \paragraph{Forwarding for Final Version of the Chapters.}
%
% The following forwarding files |cdocsfn1.tex| and |cdocsfn2.tex|
% (with identical content)
% compile the final versions of the child documents
% |cdocsch1.tex| and |cdocsch2.tex|, respectively:
%\iffalse
%<*samplefinal>
%\fi
%    \begin{macrocode}
\def\version{final}
\input{childdoc.def}
\childdocforwardprefix[cdocsamp]{cdocsfn}{cdocsch}
%    \end{macrocode}

%\iffalse
%</samplefinal>
%\fi
%
% %%%%%%%%%%%%%%%%%%%%%%%%%%%%%%%%%%%%%%
% \paragraph{Command Line Processing.}
%
% The following three command lines generate the output files
% |cdocscld|, |cdocscl1| and |cdocscl2|
% which should be identical to
% |cdocsdrf|, |cdocsch1| and |cdocsfn2|, respectively:
% \begin{center}
% \begin{tabular}{l}
% |latex -jobname cdocscld \|\\
% |  "\def\version{draft}\input{childdoc.def}\childdocforward{cdocsamp}"|\\
% |latex -jobname cdocscl1 \|\\
% |  "\input{childdoc.def}\childdocforward[cdocsamp]{cdocsch1}"|\\
% |latex -jobname cdocscl2 \|\\
% |  "\def\version{final}\input{childdoc.def}\childdocforward{cdocsch2}"|
% \end{tabular}
% \end{center}
% Note that the trailing backslash on each first line
% merely continues the input to the second line
% (for convenient cut ant paste).
% Furthermore, the command |latex| can be replaced by any
% of its alternative versions such as |pdflatex|.
%
% %%%%%%%%%%%%%%%%%%%%%%%%%%%%%%%%%%%%%%%%%%%%%%%%%%%%%%%%%%%%%%%%%%%%%%%%%%%%%%
% %%%%%%%%%%%%%%%%%%%%%%%%%%%%%%%%%%%%%%%%%%%%%%%%%%%%%%%%%%%%%%%%%%%%%%%%%%%%%%
% \section{Implementation}
%\iffalse
%<*package>
%\fi
%
% This section describes the definitions file |childdoc.def|.

% The definitions cannot be loaded using |\usepackage| or |\RequirePackage|
% which has a mechanism to prevent loading a style file more than once.
% When loading the definitions by means of |\input|
% multiple instances have to be prevented manually:
%\iffalse
%This code needs to be before the `\ProvidesFile' directive
%which is defined at the beginning of this file.
%Therefore it is also placed there and commented out here.
%</package>
%<*discard>
%\fi
%    \begin{macrocode}
\ifdefined\childdocmain\endinput\fi
%    \end{macrocode}
%\iffalse
%</discard>
%<*package>
%\fi
%
% \macro{\ifchilddoc}
% \macro{\ifchilddocmanual}
% The conditional |\ifchilddoc| tells whether a
% child (true) or main (false) document is being compiled.
% The conditional |\ifchilddocmanual| tells whether
% the |\includeonly| mechanism is used (false) or
% the selection of child files must be performed manually (true).
% The definitions initialise to false:
%    \begin{macrocode}
\newif\ifchilddoc
\newif\ifchilddocmanual
%    \end{macrocode}

% \macro{\childdocname}
% \macro{\childdocjob}
% The macro |\childdocname| stores the name of the main document
% to be compiled. The macro |\childdocjob| stores the name of
% the document on which the \LaTeX{} compiler was originally invoked.
% The content of |\jobname| cannot be compared
% to filenames specified in the source due to different catcodes.
% The following code rescans |\jobname|, stores the result
% in |\childdocname| and saves a copy in |\childdocjob|:
%    \begin{macrocode}
\edef\childdocname{\scantokens\expandafter{\jobname\noexpand}}
\let\childdocjob\childdocname
%    \end{macrocode}

% \macro{\childdocdisable}
% The macro |\childdocdisable| prevents the main file
% from being processed more than once.
% At this stage, the main document command |\childdocmain|
% is assumed to be called once again where it should do nothing.
% Any subsequent call to it should prevent
% a secondary processing of the main document
% It overwrites the forwarding commands
% |\childdocof| and |\childdocforward|
% with empty macros to prevent further inclusions of the main document:
%    \begin{macrocode}
\newcommand{\childdocdisable}
{
  \renewcommand{\childdocmain}[1]{\renewcommand{\childdocmain}[1]{\endinput}}
  \renewcommand{\childdocof}[1]{}
  \renewcommand{\childdocby}[2][]{}
  \renewcommand{\childdocforward}[2][]{}
  \renewcommand{\childdocdisable}{}
}
%    \end{macrocode}

% \macro{\childdocmain}
% The macro |\childdocmain| is to be called at the top of the main file
% with nothing or the main filename (without extension) as argument.
% First, it breaks loops.
% If the argument is not empty and does not match |\childdocname|
% (which is set by the first inclusion of |childdoc.def|),
% |\ifchilddoc| is set to true, |\includeonly| is applied to the child file
% and |\jobname| is set to the main file
% (for proper handling of |.aux| files):
%    \begin{macrocode}
\newcommand{\childdocmain}[1]
{
  \childdocdisable\childdocmain{}
  \if?#1?\else
    \begingroup
      \def\childdoctmp{#1}
      \ifx\childdoctmp\childdocname
        \def\childdoctmp{}
      \else
        \def\childdoctmp
        {
          \childdoctrue
          \includeonly{\childdocname}
          \def\childdocjob{#1}
          \def\jobname{#1}
        }
      \fi
      \expandafter
    \endgroup
    \childdoctmp
  \fi
}
%    \end{macrocode}

% \macro{\childdocof}
% The command |\childdocof| redirects
% compilation to the main file |#1|.
%    \begin{macrocode}
\newcommand{\childdocof}[1]
{
  \childdocdisable
  \childdoctrue
  \includeonly{\childdocname}
  \def\jobname{#1}
  \def\childdocjob{#1}
  \input{#1}
}
%    \end{macrocode}

% \macro{\childdocby}
% The command |\childdocby| ....
%    \begin{macrocode}
\newcommand{\childdocby}[2][]
{
  \childdocdisable
  \childdoctrue
  \childdocmanualtrue
  \if?#1?\else
    \def\jobname{#2}
  \fi
  \def\childdocjob{#2}
  \input{#2}
  \endinput
}
%    \end{macrocode}

% \macro{\childdocforward}
% The command |\childdocforward| redirects
% compilation to the main file or
% (if the optional argument is given) a child file.
% Parameters are set as if the main file
% or a child file starting with |\childdocof| was compiled.
% Then compilation is handed over to the main file:
%    \begin{macrocode}
\newcommand{\childdocforward}[2][]
{
  \begingroup
    \if?#1?
      \def\childdoctmp
      {
        \def\childdocname{#2}
        \def\childdocjob{#2}
        \def\jobname{#2}
        \input{#2}
        \endinput
      }
    \else
      \def\childdoctmp
      {
        \childdocdisable
        \def\childdocname{#2}
        \childdoctrue
        \includeonly{#2}
        \def\childdocjob{#1}
        \def\jobname{#1}
        \input{#1}
        \endinput
      }
    \fi
    \expandafter
  \endgroup
  \childdoctmp
}
%    \end{macrocode}

% \macro{\childdocforwardprefix}
% The command |\childdocforwardprefix| redirects
% compilation to the main or a child file by means of a pattern.
% The prefix |#1| in the current filename is replaced by |#2|
% and the suffix of the current filename is kept
% (it is assumed that the filename does not contain the substring `|~~~|'
% which is used as a delimiter).
% Compilation is handed over to the new file by |\childdocforward|:
%    \begin{macrocode}
\newcommand{\childdocforwardprefix}[3][]
{
  \begingroup
    \def\childdocextract #2##1~~~{\def\childdoctmp{\childdocforward[#1]{#3##1}}}
    \expandafter\childdocextract\childdocname~~~
    \expandafter
  \endgroup
  \childdoctmp
}
%    \end{macrocode}

% \macro{\childdoc}
% The deprecated macro |\childdoc| is a legacy version of |\childdocmain|:
%    \begin{macrocode}
\newcommand{\childdoc}{\childdocmain}
%    \end{macrocode}

% \macro{\childdocredirect}
% The deprecated macro |\childdocredirect| is a legacy version
% of |\childdocforward| and |\childdocforwardprefix|:
%    \begin{macrocode}
\newcommand{\childdocredirect}[2][]
{
  \begingroup
    \if?#1?
      \def\childdoctmp{\childdocforward{#2}}
    \else
      \def\childdoctmp{\childdocforwardprefix{#1}{#2}}
    \fi
    \expandafter
  \endgroup
  \childdoctmp
}
%    \end{macrocode}

%\iffalse
%</package>
%\fi
%
\endinput
|\\
|\childdocforward[|\textit{main}|]{|\textit{dest}|}|\\
\end{tabular}
\end{center}
%
The argument \textit{dest} is the destination file
(without extension).
It should be the main file or one of the child files.
Note that further \textsf{childdoc} directives
such as |\childdocof| and |\childdocforward|
in the indicated file will be processed in this form.
The optional argument \textit{main}
passes on directly to the main file \textit{main}
while pretending to compile the child \textit{dest}.
This form behaves as if \textit{dest}
issues |\childdocof{|\textit{main}|}| right away,
and no further \textsf{childdoc} directives will be processed.

%%%%%%%%%%%%%%%%%%%%%%%%%%%%%%%%%%%%%%%%
\DescribeMacro{\...prefix}
In the alternative form |\childdocforwardprefix|,
%
\begin{center}
\begin{tabular}{l}
|% \iffalse
%
% childdoc.dtx Copyright (C) 2017-2018 Niklas Beisert
%
% This work may be distributed and/or modified under the
% conditions of the LaTeX Project Public License, either version 1.3
% of this license or (at your option) any later version.
% The latest version of this license is in
%   http://www.latex-project.org/lppl.txt
% and version 1.3 or later is part of all distributions of LaTeX
% version 2005/12/01 or later.
%
% This work has the LPPL maintenance status `maintained'.
%
% The Current Maintainer of this work is Niklas Beisert.
%
% This work consists of the files childdoc.dtx and childdoc.ins
% and the derived files childdoc.def and cdocsamp.tex with
% cdocsch1.tex, cdocsch2.tex, cdocsdrf.tex, cdocsfn1.tex, cdocsfn2.tex.
%
%<package>\ifdefined\childdocmain\endinput\fi
%<package>\ProvidesFile{childdoc.def}[2018/12/30 v2.0 child document driver]
%<samplemain>\ProvidesFile{cdocsamp.tex}[2018/12/30 v2.0 sample for childdoc]
%<*driver>
%\ProvidesFile{childdoc.drv}[2018/12/30 v2.0 childdoc reference manual file]
\PassOptionsToClass{10pt,a4paper}{article}
\documentclass{ltxdoc}

\usepackage[margin=35mm]{geometry}
\usepackage{hyperref}
\usepackage{hyperxmp}
\usepackage[usenames]{color}

\hypersetup{colorlinks=true}
\hypersetup{pdfstartview=FitH}
\hypersetup{pdfpagemode=UseNone}
\hypersetup{pdfsource={}}
\hypersetup{pdflang={en-UK}}
\hypersetup{pdfcopyright={Copyright 2017-2018 Niklas Beisert.
  This work may be distributed and/or modified under the
  conditions of the LaTeX Project Public License, either version 1.3
  of this license or (at your option) any later version.}}
\hypersetup{pdflicenseurl={http://www.latex-project.org/lppl.txt}}
\hypersetup{pdfcontactaddress={ETH Zurich, ITP, HIT K,
  Wolfgang-Pauli-Strasse 27}}
\hypersetup{pdfcontactpostcode={8093}}
\hypersetup{pdfcontactcity={Zurich}}
\hypersetup{pdfcontactcountry={Switzerland}}
\hypersetup{pdfcontactemail={nbeisert@itp.phys.ethz.ch}}
\hypersetup{pdfcontacturl={http://people.phys.ethz.ch/\xmptilde nbeisert/}}

\newcommand{\secref}[1]{\hyperref[#1]{section \ref*{#1}}}

\parskip1ex
\parindent0pt
\let\olditemize\itemize
\def\itemize{\olditemize\parskip0pt}

\begin{document}

\title{The \textsf{childdoc} Package}
\hypersetup{pdftitle={The childdoc Package}}
\author{Niklas Beisert\\[2ex]
  Institut f\"ur Theoretische Physik\\
  Eidgen\"ossische Technische Hochschule Z\"urich\\
  Wolfgang-Pauli-Strasse 27, 8093 Z\"urich, Switzerland\\[1ex]
  \href{mailto:nbeisert@itp.phys.ethz.ch}
  {\texttt{nbeisert@itp.phys.ethz.ch}}}
\hypersetup{pdfauthor={Niklas Beisert}}
\hypersetup{pdfsubject={Manual for the LaTeX2e Package childdoc}}
\date{30 December 2018, \textsf{v2.0}}
\maketitle

\begin{abstract}\noindent
\textsf{childdoc} is a \LaTeXe{} package
that enables the direct compilation
of document sections included by |\include|
to individual files.
\end{abstract}

\begingroup
\parskip0ex
\tableofcontents
\endgroup

%%%%%%%%%%%%%%%%%%%%%%%%%%%%%%%%%%%%%%%%%%%%%%%%%%%%%%%%%%%%%%%%%%%%%%%%%%%%%%%%
%%%%%%%%%%%%%%%%%%%%%%%%%%%%%%%%%%%%%%%%%%%%%%%%%%%%%%%%%%%%%%%%%%%%%%%%%%%%%%%%
\section{Introduction}

\LaTeX{} provides a mechanism to structure a large document (such as a book)
into a main file and several child files (containing the chapters)
using the |\include| command.
This mechanism is beneficial for documents
which span hundreds of pages in order to
make the source file(s) more manageable.
Moreover, compilation can be restricted to
selected child files by means of the |\includeonly| command.
The latter feature can be used to reduce the compilation time while editing
(this was significantly more useful in the earlier days of \LaTeX{})
or to generate a smaller document which is easier to navigate.
Another application of |\includeonly| is to generate
documents consisting of selected parts of the complete document.

However, there are a few drawbacks of the plain |\include| mechanism:
\begin{itemize}
\item
The child files cannot be compiled on their own,
they can only be compiled via the main file.
A naive editing environment
(such as a text editor with an option
to have the current file processed by \LaTeX)
may require one to switch to the main file before compiling;
attempting to compile the child file produces errors.
\item
The main file must be modified (each time)
to adjust the |\includeonly| command
to the present needs. This easily leaves the main file in a messy state.
\item
The generated document will always carry the filename
of the main document. This is inconvenient if
several child files are to be compiled and
to be kept for distribution.
\end{itemize}

The present package provides a simple interface
to make child files individually compilable by \LaTeX{}.
Compiling a child file then has the same effect as compiling
the main file with an |\includeonly| command
to select the appropriate child.
Moreover the generated document will carry the name of the child
rather than the main file.
This resolves all three above issues.

This feature is meant to make the editing of books,
thesis documents and lecture notes somewhat more convenient.
However, the package can also be used efficiently for
composing a series of documents (such as exercise sheets)
which are typically distributed individually.
It then assists the author in generating the individual documents
(potentially in different versions)
as well as a document containing the collected series.
Another application is in developing style files
or other kinds of included material
where compilation of the style file could redirect
to a sample or test file.

%%%%%%%%%%%%%%%%%%%%%%%%%%%%%%%%%%%%%%%%%%%%%%%%%%%%%%%%%%%%%%%%%%%%%%%%%%%%%%%%
%%%%%%%%%%%%%%%%%%%%%%%%%%%%%%%%%%%%%%%%%%%%%%%%%%%%%%%%%%%%%%%%%%%%%%%%%%%%%%%%
\section{Usage}

First of all, the package \textsf{childdoc} is \emph{not} a standard
\LaTeXe{} |.sty| style file! Therefore it needs to be invoked in
a non-standard way.

%%%%%%%%%%%%%%%%%%%%%%%%%%%%%%%%%%%%%%%%%%%%%%%%%%%%%%%%%%%%%%%%%%%%%%%%%%%%%%%%
\subsection{Included Files}
\label{sec:include}

%%%%%%%%%%%%%%%%%%%%%%%%%%%%%%%%%%%%%%%%
\DescribeMacro{\childdocmain}
To use the package, add the commands
\begin{center}
\begin{tabular}{l}
|\input{childdoc.def}|\\
|\childdocmain{}|\\
\end{tabular}
\end{center}
at the very top of the main \LaTeX{} file,
in particular \emph{before} the |\documentclass| statement!
The argument of |\childdocmain| should be left empty
(but it must be present).

%%%%%%%%%%%%%%%%%%%%%%%%%%%%%%%%%%%%%%%%
\DescribeMacro{\childdocof}
Furthermore, add the commands
\begin{center}
\begin{tabular}{l}
|\input{childdoc.def}|\\
|\childdocof{|\textit{main}|}|\\
\end{tabular}
\end{center}
at the top of every child file \textit{child}
which is included by |\include{|\textit{child}|}|
from within the main file
(or at least for those files to be compiled individually).
The argument \textit{main} must be the filename of the main file.

There are a couple of
considerations in setting up the main and child documents:

%%%%%%%%%%%%%%%%%%%%%%%%%%%%%%%%%%%%%%%%
\paragraph{Restrictions.}

Please note the following restrictions:
\begin{itemize}
\item
|\childdocmain| must be called with one argument \textit{main}
to ensure compatibility with earlier version of the package.
It must either be empty (|\childdocmain{}|)
or precisely match the filename of the main file in which it is specified.
See \secref{sec:detection} for further information.
\item
The filename \textit{main} must be specified without the |.tex| extension.
\item
The filename \textit{main} is case sensitive
(even in case-insensitive file systems)
due to internal string comparison.
\item
The argument \textit{main} should be fully expanded, it cannot be a macro.
\item
Subdirectories and special characters should be avoided in filenames.
\item
The command |\childdocmain{|\textit{main}|}| must be followed by a whitespace.
It should not be followed immediately by another command
or by a comment mark `|%|'.
This is because the \TeX{} parser reads the token immediately following
the argument of |\childdocmain| and puts it
at the beginning of every child section;
however, a white\-space is ignored.
\end{itemize}

%%%%%%%%%%%%%%%%%%%%%%%%%%%%%%%%%%%%%%%%
\paragraph{Content of Main File.}

It is advisable to place all content in the child files included by |\include|.
Any output contained in the main file will appear in all child documents
unless suppressed manually;
it cannot be suppressed automatically by the |\includeonly| directive
and thus should normally be avoided.
A method to include some content in the main file
by means of conditional processing is described in \secref{sec:conditional}.

%%%%%%%%%%%%%%%%%%%%%%%%%%%%%%%%%%%%%%%%
\paragraph{Page Numbering.}

When only a part of the document is compiled,
the appropriate numbering of pages
(as well as other status parameters)
is determined from the |.aux| files.
The latter contain information from previous passes.
However this information needs to propagate through
all intermediate child documents.
Therefore the page numbering in child documents may well
be inconsistent until the complete document is compiled at least once.

A useful (if unconventional) way to always ensure a consistent
page numbering is to restart the numbering in each child document
and denote the pages by `\textit{child}|.|\textit{page}'
where \textit{child} represents the chapter/section number of the child file.
This can be achieved by the command
|\numberwithin{page}{|\textit{child}|}|
of the \textsf{amsmath} package
where \textit{child} can be |chapter| or |section|
depending on the chosen structuring.
Alternatively, one can modify the macro |\thepage| appropriately
and reset the counter |page| at the start of each child file.

%%%%%%%%%%%%%%%%%%%%%%%%%%%%%%%%%%%%%%%%%%%%%%%%%%%%%%%%%%%%%%%%%%%%%%%%%%%%%%%%
\subsection{Conditional Processing}
\label{sec:conditional}

The package provides a mechanism to compile different versions
of a document. To customise the versions further some conditional processing
can come in handy to distinguish which version is being compiled.
The package provides two macros to describe the compilation context:

%%%%%%%%%%%%%%%%%%%%%%%%%%%%%%%%%%%%%%%%
\DescribeMacro{\ifchilddoc}
The conditional |\ifchilddoc| distinguishes between the compilation of
child documents and the main document:
%
\begin{center}
|\ifchilddoc |\textit{child-code}| |[|\||else |\textit{main-code}]| \||fi|
\end{center}

%%%%%%%%%%%%%%%%%%%%%%%%%%%%%%%%%%%%%%%%
\DescribeMacro{\childdocname}
\DescribeMacro{\childdocjob}
The macro |\childdocname| contains the filename (without extension)
of the main or child file being processed.
Note that |\childdocjob| will always contain the name of the main file.

%%%%%%%%%%%%%%%%%%%%%%%%%%%%%%%%%%%%%%%%
\paragraph{Title Page.}

Conditional processing can be used to include a title or banner page
in the main document when proper precautions are taken.
Importantly, the code in the main file should ensure that the page counter
(as well as other status parameters which are stored in the |.aux| files)
takes the same value after the conditional processing.
Otherwise the page numbers may take divergent values
depending on which part is compiled.

For example, a title page could be declared by:
%
\begin{center}
\begin{tabular}{l}
|\ifchilddoc\||else|\\
|\addtocounter{page}{-1}|\\
\textit{code for title page}\\
|\newpage|\\
|\||fi|
\end{tabular}
\end{center}
%
A banner page for the child documents can be generated by:
%
\begin{center}
\begin{tabular}{l}
|\ifchilddoc|\\
|\addtocounter{page}{-1}|\\
\textit{code for banner page}\\
|\newpage|\\
|\||fi|
\end{tabular}
\end{center}
%
Here one could write a message such as:
\begin{center}
|This is the part \childdocname{} of \childdocjob{}.|
\end{center}

%%%%%%%%%%%%%%%%%%%%%%%%%%%%%%%%%%%%%%%%%%%%%%%%%%%%%%%%%%%%%%%%%%%%%%%%%%%%%%%%
\subsection{Flags}
\label{sec:flags}

The package makes it easy to generate different versions
of the main or child documents.
To this end compilation flags can be defined
and assigned different default values.
They will be particularly useful in conjunction
with the forwarding mechanism described in \secref{sec:forward}.

For example, it may be useful to have a flag |\version|
which can be set to |draft| or |final|.
The document source will contain some conditional code
depending on the value of |\version|.
Suppose further, the flag should default to |final| for the main file
and to |draft| for child files
which is a natural assignment for editing the document.
This is achieved by placing the following code
in the preamble of the main document
(below the |\childdocmain| directive):
%
\begin{center}
\begin{tabular}{l}
|\ifchilddoc|\\
|\providecommand{\version}{draft}|\\
|\||else|\\
|\providecommand{\version}{final}|\\
|\||fi|
\end{tabular}
\end{center}
%
The definition by |\providecommand| makes sure
that previous definitions are not overwritten.
Further statements |\providecommand{\version}{...}|
can thus be added before the above code to override it.

For the main file, one might add a line
(between |\childdocmain| and the above block)
%
\begin{center}
|%\ifchilddoc\||else\providecommand{\version}{draft}\||fi|
\end{center}
%
which can be uncommented to produce a draft version.
Likewise one can add a line to the very top of a child file
(above the |\childdocof{|\textit{main}|}| directive)
%
\begin{center}
|%\providecommand{\version}{final}|
\end{center}
%
which can be uncommented to produce the final version of this child document.

%%%%%%%%%%%%%%%%%%%%%%%%%%%%%%%%%%%%%%%%%%%%%%%%%%%%%%%%%%%%%%%%%%%%%%%%%%%%%%%%
\subsection{Forwarding}
\label{sec:forward}

Different versions of the main or child documents
using compilation flags as described in \secref{sec:flags}
can be (permanently) stored in different files
for convenient compilation, viewing and distribution.
To this end, the package defines a command
to pass on compilation to a different file:

%%%%%%%%%%%%%%%%%%%%%%%%%%%%%%%%%%%%%%%%
\DescribeMacro{\childdocforward}
The command |\childdocforward| redirects processing to
another source file:
%
\begin{center}
\begin{tabular}{l}
|\input{childdoc.def}|\\
|\childdocforward[|\textit{main}|]{|\textit{dest}|}|\\
\end{tabular}
\end{center}
%
The argument \textit{dest} is the destination file
(without extension).
It should be the main file or one of the child files.
Note that further \textsf{childdoc} directives
such as |\childdocof| and |\childdocforward|
in the indicated file will be processed in this form.
The optional argument \textit{main}
passes on directly to the main file \textit{main}
while pretending to compile the child \textit{dest}.
This form behaves as if \textit{dest}
issues |\childdocof{|\textit{main}|}| right away,
and no further \textsf{childdoc} directives will be processed.

%%%%%%%%%%%%%%%%%%%%%%%%%%%%%%%%%%%%%%%%
\DescribeMacro{\...prefix}
In the alternative form |\childdocforwardprefix|,
%
\begin{center}
\begin{tabular}{l}
|\input{childdoc.def}|\\
|\childdocforwardprefix[|\textit{main}|]{|\textit{prefix}|}{|\textit{dest}|}|
\end{tabular}
\end{center}
%
the destination file is determined by a pattern
depending on the current file:
To make this work, the current file must be called
`{\textit{prefix}\hspace{0.2em}\textit{suffix}}'
with \textit{prefix} matching precisely the argument.
Processing is then passed on to the file
`{\textit{dest}\hspace{0.2em}\textit{suffix}}'.
Surely, the same effect is achieved by
directly specifying the
argument `{\textit{dest}\hspace{0.2em}\textit{suffix}}'
in the first form.
However, that requires to set up a different file
for each child. With the alternative form of the command
all these files can have exactly the same content
which simplifies setting them up and maintaining them.

For example, the following file |draft.tex|
with a compilation flag |\version| as described in \secref{sec:flags}
compiles the main document as a draft:
%
\begin{center}
\begin{tabular}{l}
|\def\version{draft}|\\
|\input{childdoc.def}|\\
|\childdocforward{|\textit{main}|}|
\end{tabular}
\end{center}
%
Likewise, the following files |final|\textit{nn}|.tex|
compile the final version of the child document
|child|\textit{nn}|.tex|:
%
\begin{center}
\begin{tabular}{l}
|\def\version{final}|\\
|\input{childdoc.def}|\\
|\childdocforwardprefix{final}{child}|
\end{tabular}
\end{center}
%

Note that when several versions of a main file and/or of each child file
are to be generated, it may be convenient to set up a |Makefile| or
shell script to automatise the process.

%%%%%%%%%%%%%%%%%%%%%%%%%%%%%%%%%%%%%%%%%%%%%%%%%%%%%%%%%%%%%%%%%%%%%%%%%%%%%%%%
\subsection{Command Line Processing}
\label{sec:commandline}

The effect of redirection files can also be achieved by invoking
the \LaTeX{} compiler with a more elaborate command line.
Most conveniently this should be done as part
of a shell script or a |Makefile|.

When using \textsf{childdoc} in the main file, the following
command lines effectively perform a redirection
(note that depending on the shell being used,
backslashes may have to be doubled: `|\|' $\to$ `|\\|'):
%
\begin{center}
|... -jobname "|\textit{target}|" |\\|"|[\textit{flags}]%
|\input{childdoc.def}\childdocforward[|\textit{main}|]{|\textit{dest}|}"|
\end{center}
%
Here \textit{target} is the name of the output file,
\textit{main} is the name of the main file
and \textit{dest} is the name of the main or child file to be processed
(all filenames without extensions).
The optional argument \textit{main} can be omitted
if \textit{main} matches \textit{dest}.
Optionally, compilation \textit{flags} can be defined via |\def| commands.
This command line makes the \TeX{} engine believe
it is compiling the file \textit{target}
whose content is specified as the latter parameter.
The provided code then forwards the processing to
\textit{main} or \textit{dest} as described in \secref{sec:forward}.

%%%%%%%%%%%%%%%%%%%%%%%%%%%%%%%%%%%%%%%%%%%%%%%%%%%%%%%%%%%%%%%%%%%%%%%%%%%%%%%%
\subsection{Include by Input}
\label{sec:input}

Including child documents by |\include| has some restrictions by design.
Most notably, the content of a child document always occupies
its own set of pages; pages cannot be shared between child documents.
Usually, this behaviour makes perfect sense
because each child document contain an essential part of the document.
However, in some situations it may be desirable to compose
a document from a collection of parts
without having mandatory page breaks between then.
For this case, the package
provides a mechanism to include parts
by |\input| which can also be processed individually.
However, by construction this mechanism
requires manual handling of the content to be output.

%%%%%%%%%%%%%%%%%%%%%%%%%%%%%%%%%%%%%%%%
\DescribeMacro{\ifchilddocmanual}
The main file should be prepared as usual, see \secref{sec:include}.
However, the document body must make a distinction
between processing of an individual part and of the main document, e.g.:
%
\begin{center}
\begin{tabular}{l}
|\ifchilddocmanual|\\
|\input{\childdocname}|\\
|\||else|\\
\textit{document body with }|\input{|\textit{part}|}|\\
|\||fi|
\end{tabular}
\end{center}
%
The conditional |\ifchilddocmanual| is true whenever
a part to be included by |\input| is being compiled,
and the name of the part is stored in |\childdocname|.

%%%%%%%%%%%%%%%%%%%%%%%%%%%%%%%%%%%%%%%%
\DescribeMacro{\childdocby}
Each part to be included by |\input| should start with:
%
\begin{center}
\begin{tabular}{l}
|\input{childdoc.def}|\\
|\childdocby{|\textit{main}|}|\\
\end{tabular}
\end{center}
%
The directive |\childdocby| is similar to |\childdocof|
described in \secref{sec:include},
but the subsequent selection of content must be done manually.
To that end, both |\ifchilddoc| and |\ifchilddocmanual|
will be true upon processing of a part,
and the name of the part is stored in |\childdocname|.
Note that |\jobname| will be set to the filename of the current part
so that each part receives an individual |.aux| file
that does not interfere with the |.aux| file(s) of the main document.
This behaviour can be altered by the alternative form
|\childdocby[*]{|\textit{main}|}| (with a non-empty optional argument)
which uses the |.aux| file of the main document
by setting |\jobname| to \textit{main}.

%%%%%%%%%%%%%%%%%%%%%%%%%%%%%%%%%%%%%%%%%%%%%%%%%%%%%%%%%%%%%%%%%%%%%%%%%%%%%%%%
\subsection{Driver Development}
\label{sec:driver}

The \textsf{childdoc} mechanism can also be use for the development
of definition files such as \LaTeX{} styles or classes.
This case differs from the above setup with multiple parts
included by |\include| in that no |\includeonly| should be invoked.
This can be achieved by starting the include file
(before |\ProvidesPackage|) with:
%
\begin{center}
\begin{tabular}{l}
|\input{childdoc.def}|\\
|\childdocforward{|\textit{main}|}|\\
\end{tabular}
\end{center}
%
or alternatively with:
%
\begin{center}
\begin{tabular}{l}
|\input{childdoc.def}|\\
|\childdocby{|\textit{main}|}|\\
\end{tabular}
\end{center}
%
Both forms have slightly different effects as described above.
The main file is prepared as usual, see \secref{sec:include}.

%%%%%%%%%%%%%%%%%%%%%%%%%%%%%%%%%%%%%%%%%%%%%%%%%%%%%%%%%%%%%%%%%%%%%%%%%%%%%%%%
\subsection{Legacy Detection}
\label{sec:detection}

The directive |\childdocmain| in the main file can detect
whether the complete document or merely a child is to be compiled
even without using the directive |\childdocof|.
This method is deprecated because it is less robust
and there is no compelling reason to use it;
it is merely provided for backward compatibility
and it may be removed in future versions.

If the detection mechanism is to be used,
it is mandatory to correctly specify
the filename of the main file as the argument of |\childdocmain|:
%
\begin{center}
\begin{tabular}{l}
|\input{childdoc.def}|\\
|\childdocmain{|\textit{main}|}|\\
\end{tabular}
\end{center}
%
If |\jobname| does not match the argument \textit{main} of |\childdocmain|,
it is assumed that |\jobname| points to the child file to be compiled.
When using |\childdocmain| with the main file specified as argument,
it suffices to start a child file
with just |\input{|\textit{main}|}|
without loading of the package and using |\childdocof|.
If instead all processing is done
with the appropriate \textsf{childdoc} directives,
the argument of \textit{main} of |\childdocmain| can be empty.

An alternative version of the command line processing described
in \secref{sec:commandline} using the detection mechanism reads:
%
\begin{center}
|... -jobname "|\textit{target}|" "|[\textit{flags}]%
[|\def\jobname{|\textit{dest}|}|]|\input{|\textit{main}|}"|
\end{center}

%%%%%%%%%%%%%%%%%%%%%%%%%%%%%%%%%%%%%%%%%%%%%%%%%%%%%%%%%%%%%%%%%%%%%%%%%%%%%%%%
\subsection{Manual Code}
\label{sec:manual}

In case one cannot be certain whether the definitions file |childdoc.def|
is installed on the target \TeX{} distribution
and one prefers not to ship it,
it is conceivable to paste a few relevant commands into the sources.

To that end, drop all statements |\input{childdoc.def}|
and perform the replacements as outlined below.
Instead of |\childdocmain{|\textit{main}|}| add the following code
to the top of the main file:
%
\begin{center}
\begin{tabular}{l}
|\||ifdefined\childdocname\endinput\||fi\newif\ifchilddoc|\\
|\edef\childdocname{\scantokens\expandafter{\jobname\noexpand}}|\\
|\def\childdocmain{|\textit{main}|}\||ifx\childdocmain\childdocname\||else|\\
|\childdoctrue\includeonly{\childdocname}\let\jobname\childdocmain\||fi|\\
\end{tabular}
\end{center}
%
Instead of |\childdocof{|\textit{main}|}| just include the main file
at the top of each child file:
%
\begin{center}
|\input{|\textit{main}|}|
\end{center}
%
A simple redirection |\childdocforward{|\textit{dest}|}| is achieved by:
%
\begin{center}
|\def\jobname{|\textit{dest}|}\input{\jobname}|
\end{center}
%
The redirection with prefix
|\childdocforwardprefix[|\textit{prefix}|]{|\textit{dest}|}|
is accomplished by:
%
\begin{center}
\begin{tabular}{l}
|{\edef\jobname{\scantokens\expandafter{\jobname\noexpand}}|\\
|\def\redirectjob |\textit{prefix}|#1~~~{\gdef\jobname{|\textit{dest}|#1}}|\\
|\expandafter\redirectjob\jobname~~~}\input{\jobname}|
\end{tabular}
\end{center}

In an alternative approach,
child documents can be compiled by a specific command line
without additional code or specific definitions:
%
\begin{center}
|... -jobname "|\textit{target}|" "|[\textit{flags}]%
|\includeonly{|\textit{dest}|}\input{|\textit{main}|}"|
\end{center}
%

%%%%%%%%%%%%%%%%%%%%%%%%%%%%%%%%%%%%%%%%%%%%%%%%%%%%%%%%%%%%%%%%%%%%%%%%%%%%%%%%
%%%%%%%%%%%%%%%%%%%%%%%%%%%%%%%%%%%%%%%%%%%%%%%%%%%%%%%%%%%%%%%%%%%%%%%%%%%%%%%%
\section{Information}

%%%%%%%%%%%%%%%%%%%%%%%%%%%%%%%%%%%%%%%%%%%%%%%%%%%%%%%%%%%%%%%%%%%%%%%%%%%%%%%%
\subsection{Copyright}

Copyright \copyright{} 2017--2018 Niklas Beisert

This work may be distributed and/or modified under the
conditions of the \LaTeX{} Project Public License, either version 1.3
of this license or (at your option) any later version.
The latest version of this license is in
  \url{http://www.latex-project.org/lppl.txt}
and version 1.3 or later is part of all distributions of \LaTeX{}
version 2005/12/01 or later.

This work has the LPPL maintenance status `maintained'.

The Current Maintainer of this work is Niklas Beisert.

This work consists of the files |README.txt|, |childdoc.ins| and |childdoc.dtx|
as well as the derived files |childdoc.def|, |cdocsamp.tex|
with |cdocsch1.tex|, |cdocsch2.tex|, |cdocspt3.tex|, |cdocspt4.tex|,
|cdocsdrf.tex|, |cdocsfn1.tex|, |cdocsfn2.tex|
as well as |childdoc.pdf|.

%%%%%%%%%%%%%%%%%%%%%%%%%%%%%%%%%%%%%%%%%%%%%%%%%%%%%%%%%%%%%%%%%%%%%%%%%%%%%%%%
\subsection{Files and Installation}

The package consists of the files:
%
\begin{center}
\begin{tabular}{ll}
    |README.txt|   & readme file \\
    |childdoc.ins| & installation file \\
    |childdoc.dtx| & source file \\
    |childdoc.def| & definition file \\
    |cdocsamp.tex| & sample main file \\
    |cdocsch1.tex| & sample include file \\
    |cdocsch2.tex| & sample include file \\
    |cdocspt3.tex| & sample part file \\
    |cdocspt4.tex| & sample part file \\
    |cdocsdrf.tex| & sample redirection file \\
    |cdocsfn1.tex| & sample redirection file \\
    |cdocsfn2.tex| & sample redirection file \\
    |childdoc.pdf| & manual
\end{tabular}
\end{center}
%
The distribution consists of the files
|README.txt|, |childdoc.ins| and |childdoc.dtx|.
%
\begin{itemize}
\item
Run (pdf)\LaTeX{} on |childdoc.dtx|
to compile the manual |childdoc.pdf| (this file).
\item
Run \LaTeX{} on |childdoc.ins| to create the definitions file |childdoc.def|
and the sample |cdocsamp.tex| with include files
|cdocsch1.tex|, |cdocsch2.tex|, |cdocspt3.tex|, |cdocspt4.tex|,
|cdocsdrf.tex|, |cdocsfn1.tex|, |cdocsfn2.tex|.
Then copy the file |childdoc.def| to an appropriate directory of your \LaTeX{}
distribution, e.g.\ \textit{texmf-root}|/tex/latex/childdoc|.
\end{itemize}

%%%%%%%%%%%%%%%%%%%%%%%%%%%%%%%%%%%%%%%%%%%%%%%%%%%%%%%%%%%%%%%%%%%%%%%%%%%%%%%%
\subsection{Related CTAN Packages}

There are several other packages which offer a similar functionality:
%
\begin{itemize}
\item
The packages
\href{http://ctan.org/pkg/docmute}{\textsf{docmute}},
\href{http://ctan.org/pkg/includex}{\textsf{includex}} and
\href{http://ctan.org/pkg/standalone}{\textsf{standalone}}
provide commands to include only the document body of
a child file thus allowing both files to be compiled individually.
\item
The packages \href{http://ctan.org/pkg/subdocs}{\textsf{subdocs}}
and \href{http://ctan.org/pkg/subfiles}{\textsf{subfiles}}
provide structures in which the main and child documents can be
encapsulated and allowing them to be compiled individually.
The inclusion mechanism is different from the conventional |\include|.
\item
The package \href{http://ctan.org/pkg/combine}{\textsf{combine}}
is an elaborate solution to combine several documents into one.
\end{itemize}
%
See also the CTAN topic \href{http://ctan.org/topic/subdocs}{\textsf{subdocs}}
for further related packages.
The present package differs from the above solutions in that
a document structure constructed with the conventional |\include| mechanism
just needs two extra commands at the top of every file
such that all constituent files can be compiled individually.

%%%%%%%%%%%%%%%%%%%%%%%%%%%%%%%%%%%%%%%%%%%%%%%%%%%%%%%%%%%%%%%%%%%%%%%%%%%%%%%%
%\subsection{Feature Suggestions}
%
%The following is a list of features which may be useful for future
%versions of this package:
%%
%\begin{itemize}
%\item
%\ldots
%\end{itemize}

%%%%%%%%%%%%%%%%%%%%%%%%%%%%%%%%%%%%%%%%%%%%%%%%%%%%%%%%%%%%%%%%%%%%%%%%%%%%%%%%
\subsection{Revision History}

%%%%%%%%%%%%%%%%%%%%%%%%%%%%%%%%%%%%%%%%
\paragraph{v2.0:} 2018/12/30

\begin{itemize}
\item
immediate forward processing
\item
added |\childdocby| mechanism
\item
manual restructured
\end{itemize}

%%%%%%%%%%%%%%%%%%%%%%%%%%%%%%%%%%%%%%%%
\paragraph{v1.6:} 2018/01/17

\begin{itemize}
\item
application for development of include files
\item
corrections to manual
\end{itemize}

%%%%%%%%%%%%%%%%%%%%%%%%%%%%%%%%%%%%%%%%
\paragraph{v1.5:} 2017/05/21

\begin{itemize}
\item
more complete structuring introduced
\item
|\childdocof| introduced
\item
|\childdoc| renamed to |\childdocmain|
\item
|\childredirect| renamed to |\childdocforward| and |\childdocforwardprefix|
and functionality expanded
\end{itemize}

%%%%%%%%%%%%%%%%%%%%%%%%%%%%%%%%%%%%%%%%
\paragraph{v1.0:} 2017/04/27

\begin{itemize}
\item
manual and install package
\item
first version published on CTAN
\end{itemize}

%%%%%%%%%%%%%%%%%%%%%%%%%%%%%%%%%%%%%%%%
\paragraph{v0.6:} 2017/04/26

\begin{itemize}
\item
redirection mechanism added
\end{itemize}

%%%%%%%%%%%%%%%%%%%%%%%%%%%%%%%%%%%%%%%%
\paragraph{v0.5:} 2017/04/26

\begin{itemize}
\item
functionality in definition file
\end{itemize}


%%%%%%%%%%%%%%%%%%%%%%%%%%%%%%%%%%%%%%%%%%%%%%%%%%%%%%%%%%%%%%%%%%%%%%%%%%%%%%%%
%%%%%%%%%%%%%%%%%%%%%%%%%%%%%%%%%%%%%%%%%%%%%%%%%%%%%%%%%%%%%%%%%%%%%%%%%%%%%%%%
%%%%%%%%%%%%%%%%%%%%%%%%%%%%%%%%%%%%%%%%%%%%%%%%%%%%%%%%%%%%%%%%%%%%%%%%%%%%%%%%
\appendix

\settowidth\MacroIndent{\rmfamily\scriptsize 000\ }

 \DocInput{childdoc.dtx}

\end{document}
%</driver>
% \fi
%
% %%%%%%%%%%%%%%%%%%%%%%%%%%%%%%%%%%%%%%%%%%%%%%%%%%%%%%%%%%%%%%%%%%%%%%%%%%%%%%
% %%%%%%%%%%%%%%%%%%%%%%%%%%%%%%%%%%%%%%%%%%%%%%%%%%%%%%%%%%%%%%%%%%%%%%%%%%%%%%
% \section{Sample}
%\iffalse
%<*samplemain>
%\fi
%
% The following presents a sample document
% with two chapters, two parts, a title page,
% a compile flag as well as three forwarding files to set the flag.
% It consists of eight |.tex| files:
% \begin{center}
% \begin{tabular}{ll}
% |cdocsamp.tex|&main file\\
% |cdocsch1.tex|&include file for chapter 1\\
% |cdocsch2.tex|&include file for chapter 2\\
% |cdocspt3.tex|&include file for part 3\\
% |cdocspt4.tex|&include file for part 4\\
% |cdocsdrf.tex|&forwarding file for main file in draft mode\\
% |cdocsfi1.tex|&forwarding file for final version of chapter 1\\
% |cdocsfi2.tex|&forwarding file for final version of chapter 2\\
% \end{tabular}
% \end{center}
% Each of the eight files can be compiled directly by the \LaTeX{} compiler.
%
% %%%%%%%%%%%%%%%%%%%%%%%%%%%%%%%%%%%%%%
% \paragraph{Main File.}
%
% The main file is called |cdocsamp.tex|.
%
% Load the \textsf{childdoc} definitions and
% declare the filename for the main document:
%    \begin{macrocode}
\input{childdoc.def}
\childdocmain{}
%    \end{macrocode}

% Optional override for |\version| flag:
%    \begin{macrocode}
%%\ifchilddoc\else\providecommand{\version}{draft}\fi
%    \end{macrocode}

% Define the default values for the |\version| flag
% (|final| for the main file and |draft| for childs):
%    \begin{macrocode}
\ifchilddoc
\providecommand{\version}{draft}
\else
\providecommand{\version}{final}
\fi
%    \end{macrocode}

% Load the standard document class:
%    \begin{macrocode}
\documentclass[12pt]{article}
%    \end{macrocode}

% Start the document body:
%    \begin{macrocode}
\begin{document}
%    \end{macrocode}

% Declare a title page.
% Print title, part of document being processed and version flag:
%    \begin{macrocode}
\addtocounter{page}{-1}
\begin{center}
{\LARGE\bfseries{}childdoc example\par}
\vspace{1cm}
\ifchilddoc
\ifchilddocmanual part\else chapter\fi:
`\childdocname' of `\childdocjob'\par
\else
main document: `\childdocjob'\par
\fi
version: \version\par
\end{center}
\newpage
%    \end{macrocode}

% Manually include selected file,
% otherwise process as usual:
%    \begin{macrocode}
\ifchilddocmanual
\section*{part `\childdocname'}
\input{\childdocname}
\else
%    \end{macrocode}

% Include the two chapters:
%    \begin{macrocode}
\include{cdocsch1}
\include{cdocsch2}
%    \end{macrocode}

% Include the two parts unless only chapters should be displayed:
%    \begin{macrocode}
\ifchilddoc\else
\section{part three}
\input{cdocspt3}
\section{part four}
\input{cdocspt4}
\fi
%    \end{macrocode}

% Process as usual until here:
%    \begin{macrocode}
\fi
%    \end{macrocode}

% End of document body:
%    \begin{macrocode}
\end{document}
%    \end{macrocode}
%\iffalse
%</samplemain>
%\fi
%
% %%%%%%%%%%%%%%%%%%%%%%%%%%%%%%%%%%%%%%
% \paragraph{Chapter Include Files.}
%
% The include files are called |cdocsch1.tex| and |cdocsch2.tex|.
%
%\iffalse
%<*samplechap1|samplechap2>
%\fi

% Optional override for |\version| flag:
%    \begin{macrocode}
%%\providecommand{\version}{final}
%    \end{macrocode}

% Include the main document:
%    \begin{macrocode}
\input{childdoc.def}
\childdocof{cdocsamp}
%    \end{macrocode}

%\iffalse
%</samplechap1|samplechap2>
%\fi
%
%\iffalse
%<*samplechap1>
%\fi
% Some text for chapter 1:
%    \begin{macrocode}
\section{one}
some text in chapter one
%    \end{macrocode}

%\iffalse
%</samplechap1>
%\fi
% Some text for chapter 2:
%\iffalse
%<*samplechap2>
%\fi
%    \begin{macrocode}
\section{two}
more text in chapter two
%    \end{macrocode}

%\iffalse
%</samplechap2>
%\fi
%
% %%%%%%%%%%%%%%%%%%%%%%%%%%%%%%%%%%%%%%
% \paragraph{Part Include Files.}
%
% The include files are called |cdocspt3.tex| and |cdocspt4.tex|.
%
%\iffalse
%<*samplepart3|samplepart4>
%\fi

% Optional override for |\version| flag:
%    \begin{macrocode}
%%\providecommand{\version}{final}
%    \end{macrocode}

% Include the main document:
%    \begin{macrocode}
\input{childdoc.def}
\childdocby{cdocsamp}
%    \end{macrocode}

%\iffalse
%</samplepart3|samplepart4>
%\fi
%
%\iffalse
%<*samplepart3>
%\fi
% Some text for part 3:
%    \begin{macrocode}
some text in part three
%    \end{macrocode}

%\iffalse
%</samplepart3>
%\fi
% Some text for part 4:
%\iffalse
%<*samplepart4>
%\fi
%    \begin{macrocode}
more text in part four
%    \end{macrocode}

%\iffalse
%</samplepart4>
%\fi
%
% %%%%%%%%%%%%%%%%%%%%%%%%%%%%%%%%%%%%%%
% \paragraph{Forwarding for a Complete Draft.}
%
% The following forwarding file |cdocsdrf.tex|
% compiles the main document in draft mode:
%\iffalse
%<*sampledraft>
%\fi
%    \begin{macrocode}
\def\version{draft}
\input{childdoc.def}
\childdocforward{cdocsamp}
%    \end{macrocode}

%\iffalse
%</sampledraft>
%\fi
%
% %%%%%%%%%%%%%%%%%%%%%%%%%%%%%%%%%%%%%%
% \paragraph{Forwarding for Final Version of the Chapters.}
%
% The following forwarding files |cdocsfn1.tex| and |cdocsfn2.tex|
% (with identical content)
% compile the final versions of the child documents
% |cdocsch1.tex| and |cdocsch2.tex|, respectively:
%\iffalse
%<*samplefinal>
%\fi
%    \begin{macrocode}
\def\version{final}
\input{childdoc.def}
\childdocforwardprefix[cdocsamp]{cdocsfn}{cdocsch}
%    \end{macrocode}

%\iffalse
%</samplefinal>
%\fi
%
% %%%%%%%%%%%%%%%%%%%%%%%%%%%%%%%%%%%%%%
% \paragraph{Command Line Processing.}
%
% The following three command lines generate the output files
% |cdocscld|, |cdocscl1| and |cdocscl2|
% which should be identical to
% |cdocsdrf|, |cdocsch1| and |cdocsfn2|, respectively:
% \begin{center}
% \begin{tabular}{l}
% |latex -jobname cdocscld \|\\
% |  "\def\version{draft}\input{childdoc.def}\childdocforward{cdocsamp}"|\\
% |latex -jobname cdocscl1 \|\\
% |  "\input{childdoc.def}\childdocforward[cdocsamp]{cdocsch1}"|\\
% |latex -jobname cdocscl2 \|\\
% |  "\def\version{final}\input{childdoc.def}\childdocforward{cdocsch2}"|
% \end{tabular}
% \end{center}
% Note that the trailing backslash on each first line
% merely continues the input to the second line
% (for convenient cut ant paste).
% Furthermore, the command |latex| can be replaced by any
% of its alternative versions such as |pdflatex|.
%
% %%%%%%%%%%%%%%%%%%%%%%%%%%%%%%%%%%%%%%%%%%%%%%%%%%%%%%%%%%%%%%%%%%%%%%%%%%%%%%
% %%%%%%%%%%%%%%%%%%%%%%%%%%%%%%%%%%%%%%%%%%%%%%%%%%%%%%%%%%%%%%%%%%%%%%%%%%%%%%
% \section{Implementation}
%\iffalse
%<*package>
%\fi
%
% This section describes the definitions file |childdoc.def|.

% The definitions cannot be loaded using |\usepackage| or |\RequirePackage|
% which has a mechanism to prevent loading a style file more than once.
% When loading the definitions by means of |\input|
% multiple instances have to be prevented manually:
%\iffalse
%This code needs to be before the `\ProvidesFile' directive
%which is defined at the beginning of this file.
%Therefore it is also placed there and commented out here.
%</package>
%<*discard>
%\fi
%    \begin{macrocode}
\ifdefined\childdocmain\endinput\fi
%    \end{macrocode}
%\iffalse
%</discard>
%<*package>
%\fi
%
% \macro{\ifchilddoc}
% \macro{\ifchilddocmanual}
% The conditional |\ifchilddoc| tells whether a
% child (true) or main (false) document is being compiled.
% The conditional |\ifchilddocmanual| tells whether
% the |\includeonly| mechanism is used (false) or
% the selection of child files must be performed manually (true).
% The definitions initialise to false:
%    \begin{macrocode}
\newif\ifchilddoc
\newif\ifchilddocmanual
%    \end{macrocode}

% \macro{\childdocname}
% \macro{\childdocjob}
% The macro |\childdocname| stores the name of the main document
% to be compiled. The macro |\childdocjob| stores the name of
% the document on which the \LaTeX{} compiler was originally invoked.
% The content of |\jobname| cannot be compared
% to filenames specified in the source due to different catcodes.
% The following code rescans |\jobname|, stores the result
% in |\childdocname| and saves a copy in |\childdocjob|:
%    \begin{macrocode}
\edef\childdocname{\scantokens\expandafter{\jobname\noexpand}}
\let\childdocjob\childdocname
%    \end{macrocode}

% \macro{\childdocdisable}
% The macro |\childdocdisable| prevents the main file
% from being processed more than once.
% At this stage, the main document command |\childdocmain|
% is assumed to be called once again where it should do nothing.
% Any subsequent call to it should prevent
% a secondary processing of the main document
% It overwrites the forwarding commands
% |\childdocof| and |\childdocforward|
% with empty macros to prevent further inclusions of the main document:
%    \begin{macrocode}
\newcommand{\childdocdisable}
{
  \renewcommand{\childdocmain}[1]{\renewcommand{\childdocmain}[1]{\endinput}}
  \renewcommand{\childdocof}[1]{}
  \renewcommand{\childdocby}[2][]{}
  \renewcommand{\childdocforward}[2][]{}
  \renewcommand{\childdocdisable}{}
}
%    \end{macrocode}

% \macro{\childdocmain}
% The macro |\childdocmain| is to be called at the top of the main file
% with nothing or the main filename (without extension) as argument.
% First, it breaks loops.
% If the argument is not empty and does not match |\childdocname|
% (which is set by the first inclusion of |childdoc.def|),
% |\ifchilddoc| is set to true, |\includeonly| is applied to the child file
% and |\jobname| is set to the main file
% (for proper handling of |.aux| files):
%    \begin{macrocode}
\newcommand{\childdocmain}[1]
{
  \childdocdisable\childdocmain{}
  \if?#1?\else
    \begingroup
      \def\childdoctmp{#1}
      \ifx\childdoctmp\childdocname
        \def\childdoctmp{}
      \else
        \def\childdoctmp
        {
          \childdoctrue
          \includeonly{\childdocname}
          \def\childdocjob{#1}
          \def\jobname{#1}
        }
      \fi
      \expandafter
    \endgroup
    \childdoctmp
  \fi
}
%    \end{macrocode}

% \macro{\childdocof}
% The command |\childdocof| redirects
% compilation to the main file |#1|.
%    \begin{macrocode}
\newcommand{\childdocof}[1]
{
  \childdocdisable
  \childdoctrue
  \includeonly{\childdocname}
  \def\jobname{#1}
  \def\childdocjob{#1}
  \input{#1}
}
%    \end{macrocode}

% \macro{\childdocby}
% The command |\childdocby| ....
%    \begin{macrocode}
\newcommand{\childdocby}[2][]
{
  \childdocdisable
  \childdoctrue
  \childdocmanualtrue
  \if?#1?\else
    \def\jobname{#2}
  \fi
  \def\childdocjob{#2}
  \input{#2}
  \endinput
}
%    \end{macrocode}

% \macro{\childdocforward}
% The command |\childdocforward| redirects
% compilation to the main file or
% (if the optional argument is given) a child file.
% Parameters are set as if the main file
% or a child file starting with |\childdocof| was compiled.
% Then compilation is handed over to the main file:
%    \begin{macrocode}
\newcommand{\childdocforward}[2][]
{
  \begingroup
    \if?#1?
      \def\childdoctmp
      {
        \def\childdocname{#2}
        \def\childdocjob{#2}
        \def\jobname{#2}
        \input{#2}
        \endinput
      }
    \else
      \def\childdoctmp
      {
        \childdocdisable
        \def\childdocname{#2}
        \childdoctrue
        \includeonly{#2}
        \def\childdocjob{#1}
        \def\jobname{#1}
        \input{#1}
        \endinput
      }
    \fi
    \expandafter
  \endgroup
  \childdoctmp
}
%    \end{macrocode}

% \macro{\childdocforwardprefix}
% The command |\childdocforwardprefix| redirects
% compilation to the main or a child file by means of a pattern.
% The prefix |#1| in the current filename is replaced by |#2|
% and the suffix of the current filename is kept
% (it is assumed that the filename does not contain the substring `|~~~|'
% which is used as a delimiter).
% Compilation is handed over to the new file by |\childdocforward|:
%    \begin{macrocode}
\newcommand{\childdocforwardprefix}[3][]
{
  \begingroup
    \def\childdocextract #2##1~~~{\def\childdoctmp{\childdocforward[#1]{#3##1}}}
    \expandafter\childdocextract\childdocname~~~
    \expandafter
  \endgroup
  \childdoctmp
}
%    \end{macrocode}

% \macro{\childdoc}
% The deprecated macro |\childdoc| is a legacy version of |\childdocmain|:
%    \begin{macrocode}
\newcommand{\childdoc}{\childdocmain}
%    \end{macrocode}

% \macro{\childdocredirect}
% The deprecated macro |\childdocredirect| is a legacy version
% of |\childdocforward| and |\childdocforwardprefix|:
%    \begin{macrocode}
\newcommand{\childdocredirect}[2][]
{
  \begingroup
    \if?#1?
      \def\childdoctmp{\childdocforward{#2}}
    \else
      \def\childdoctmp{\childdocforwardprefix{#1}{#2}}
    \fi
    \expandafter
  \endgroup
  \childdoctmp
}
%    \end{macrocode}

%\iffalse
%</package>
%\fi
%
\endinput
|\\
|\childdocforwardprefix[|\textit{main}|]{|\textit{prefix}|}{|\textit{dest}|}|
\end{tabular}
\end{center}
%
the destination file is determined by a pattern
depending on the current file:
To make this work, the current file must be called
`{\textit{prefix}\hspace{0.2em}\textit{suffix}}'
with \textit{prefix} matching precisely the argument.
Processing is then passed on to the file
`{\textit{dest}\hspace{0.2em}\textit{suffix}}'.
Surely, the same effect is achieved by
directly specifying the
argument `{\textit{dest}\hspace{0.2em}\textit{suffix}}'
in the first form.
However, that requires to set up a different file
for each child. With the alternative form of the command
all these files can have exactly the same content
which simplifies setting them up and maintaining them.

For example, the following file |draft.tex|
with a compilation flag |\version| as described in \secref{sec:flags}
compiles the main document as a draft:
%
\begin{center}
\begin{tabular}{l}
|\def\version{draft}|\\
|% \iffalse
%
% childdoc.dtx Copyright (C) 2017-2018 Niklas Beisert
%
% This work may be distributed and/or modified under the
% conditions of the LaTeX Project Public License, either version 1.3
% of this license or (at your option) any later version.
% The latest version of this license is in
%   http://www.latex-project.org/lppl.txt
% and version 1.3 or later is part of all distributions of LaTeX
% version 2005/12/01 or later.
%
% This work has the LPPL maintenance status `maintained'.
%
% The Current Maintainer of this work is Niklas Beisert.
%
% This work consists of the files childdoc.dtx and childdoc.ins
% and the derived files childdoc.def and cdocsamp.tex with
% cdocsch1.tex, cdocsch2.tex, cdocsdrf.tex, cdocsfn1.tex, cdocsfn2.tex.
%
%<package>\ifdefined\childdocmain\endinput\fi
%<package>\ProvidesFile{childdoc.def}[2018/12/30 v2.0 child document driver]
%<samplemain>\ProvidesFile{cdocsamp.tex}[2018/12/30 v2.0 sample for childdoc]
%<*driver>
%\ProvidesFile{childdoc.drv}[2018/12/30 v2.0 childdoc reference manual file]
\PassOptionsToClass{10pt,a4paper}{article}
\documentclass{ltxdoc}

\usepackage[margin=35mm]{geometry}
\usepackage{hyperref}
\usepackage{hyperxmp}
\usepackage[usenames]{color}

\hypersetup{colorlinks=true}
\hypersetup{pdfstartview=FitH}
\hypersetup{pdfpagemode=UseNone}
\hypersetup{pdfsource={}}
\hypersetup{pdflang={en-UK}}
\hypersetup{pdfcopyright={Copyright 2017-2018 Niklas Beisert.
  This work may be distributed and/or modified under the
  conditions of the LaTeX Project Public License, either version 1.3
  of this license or (at your option) any later version.}}
\hypersetup{pdflicenseurl={http://www.latex-project.org/lppl.txt}}
\hypersetup{pdfcontactaddress={ETH Zurich, ITP, HIT K,
  Wolfgang-Pauli-Strasse 27}}
\hypersetup{pdfcontactpostcode={8093}}
\hypersetup{pdfcontactcity={Zurich}}
\hypersetup{pdfcontactcountry={Switzerland}}
\hypersetup{pdfcontactemail={nbeisert@itp.phys.ethz.ch}}
\hypersetup{pdfcontacturl={http://people.phys.ethz.ch/\xmptilde nbeisert/}}

\newcommand{\secref}[1]{\hyperref[#1]{section \ref*{#1}}}

\parskip1ex
\parindent0pt
\let\olditemize\itemize
\def\itemize{\olditemize\parskip0pt}

\begin{document}

\title{The \textsf{childdoc} Package}
\hypersetup{pdftitle={The childdoc Package}}
\author{Niklas Beisert\\[2ex]
  Institut f\"ur Theoretische Physik\\
  Eidgen\"ossische Technische Hochschule Z\"urich\\
  Wolfgang-Pauli-Strasse 27, 8093 Z\"urich, Switzerland\\[1ex]
  \href{mailto:nbeisert@itp.phys.ethz.ch}
  {\texttt{nbeisert@itp.phys.ethz.ch}}}
\hypersetup{pdfauthor={Niklas Beisert}}
\hypersetup{pdfsubject={Manual for the LaTeX2e Package childdoc}}
\date{30 December 2018, \textsf{v2.0}}
\maketitle

\begin{abstract}\noindent
\textsf{childdoc} is a \LaTeXe{} package
that enables the direct compilation
of document sections included by |\include|
to individual files.
\end{abstract}

\begingroup
\parskip0ex
\tableofcontents
\endgroup

%%%%%%%%%%%%%%%%%%%%%%%%%%%%%%%%%%%%%%%%%%%%%%%%%%%%%%%%%%%%%%%%%%%%%%%%%%%%%%%%
%%%%%%%%%%%%%%%%%%%%%%%%%%%%%%%%%%%%%%%%%%%%%%%%%%%%%%%%%%%%%%%%%%%%%%%%%%%%%%%%
\section{Introduction}

\LaTeX{} provides a mechanism to structure a large document (such as a book)
into a main file and several child files (containing the chapters)
using the |\include| command.
This mechanism is beneficial for documents
which span hundreds of pages in order to
make the source file(s) more manageable.
Moreover, compilation can be restricted to
selected child files by means of the |\includeonly| command.
The latter feature can be used to reduce the compilation time while editing
(this was significantly more useful in the earlier days of \LaTeX{})
or to generate a smaller document which is easier to navigate.
Another application of |\includeonly| is to generate
documents consisting of selected parts of the complete document.

However, there are a few drawbacks of the plain |\include| mechanism:
\begin{itemize}
\item
The child files cannot be compiled on their own,
they can only be compiled via the main file.
A naive editing environment
(such as a text editor with an option
to have the current file processed by \LaTeX)
may require one to switch to the main file before compiling;
attempting to compile the child file produces errors.
\item
The main file must be modified (each time)
to adjust the |\includeonly| command
to the present needs. This easily leaves the main file in a messy state.
\item
The generated document will always carry the filename
of the main document. This is inconvenient if
several child files are to be compiled and
to be kept for distribution.
\end{itemize}

The present package provides a simple interface
to make child files individually compilable by \LaTeX{}.
Compiling a child file then has the same effect as compiling
the main file with an |\includeonly| command
to select the appropriate child.
Moreover the generated document will carry the name of the child
rather than the main file.
This resolves all three above issues.

This feature is meant to make the editing of books,
thesis documents and lecture notes somewhat more convenient.
However, the package can also be used efficiently for
composing a series of documents (such as exercise sheets)
which are typically distributed individually.
It then assists the author in generating the individual documents
(potentially in different versions)
as well as a document containing the collected series.
Another application is in developing style files
or other kinds of included material
where compilation of the style file could redirect
to a sample or test file.

%%%%%%%%%%%%%%%%%%%%%%%%%%%%%%%%%%%%%%%%%%%%%%%%%%%%%%%%%%%%%%%%%%%%%%%%%%%%%%%%
%%%%%%%%%%%%%%%%%%%%%%%%%%%%%%%%%%%%%%%%%%%%%%%%%%%%%%%%%%%%%%%%%%%%%%%%%%%%%%%%
\section{Usage}

First of all, the package \textsf{childdoc} is \emph{not} a standard
\LaTeXe{} |.sty| style file! Therefore it needs to be invoked in
a non-standard way.

%%%%%%%%%%%%%%%%%%%%%%%%%%%%%%%%%%%%%%%%%%%%%%%%%%%%%%%%%%%%%%%%%%%%%%%%%%%%%%%%
\subsection{Included Files}
\label{sec:include}

%%%%%%%%%%%%%%%%%%%%%%%%%%%%%%%%%%%%%%%%
\DescribeMacro{\childdocmain}
To use the package, add the commands
\begin{center}
\begin{tabular}{l}
|\input{childdoc.def}|\\
|\childdocmain{}|\\
\end{tabular}
\end{center}
at the very top of the main \LaTeX{} file,
in particular \emph{before} the |\documentclass| statement!
The argument of |\childdocmain| should be left empty
(but it must be present).

%%%%%%%%%%%%%%%%%%%%%%%%%%%%%%%%%%%%%%%%
\DescribeMacro{\childdocof}
Furthermore, add the commands
\begin{center}
\begin{tabular}{l}
|\input{childdoc.def}|\\
|\childdocof{|\textit{main}|}|\\
\end{tabular}
\end{center}
at the top of every child file \textit{child}
which is included by |\include{|\textit{child}|}|
from within the main file
(or at least for those files to be compiled individually).
The argument \textit{main} must be the filename of the main file.

There are a couple of
considerations in setting up the main and child documents:

%%%%%%%%%%%%%%%%%%%%%%%%%%%%%%%%%%%%%%%%
\paragraph{Restrictions.}

Please note the following restrictions:
\begin{itemize}
\item
|\childdocmain| must be called with one argument \textit{main}
to ensure compatibility with earlier version of the package.
It must either be empty (|\childdocmain{}|)
or precisely match the filename of the main file in which it is specified.
See \secref{sec:detection} for further information.
\item
The filename \textit{main} must be specified without the |.tex| extension.
\item
The filename \textit{main} is case sensitive
(even in case-insensitive file systems)
due to internal string comparison.
\item
The argument \textit{main} should be fully expanded, it cannot be a macro.
\item
Subdirectories and special characters should be avoided in filenames.
\item
The command |\childdocmain{|\textit{main}|}| must be followed by a whitespace.
It should not be followed immediately by another command
or by a comment mark `|%|'.
This is because the \TeX{} parser reads the token immediately following
the argument of |\childdocmain| and puts it
at the beginning of every child section;
however, a white\-space is ignored.
\end{itemize}

%%%%%%%%%%%%%%%%%%%%%%%%%%%%%%%%%%%%%%%%
\paragraph{Content of Main File.}

It is advisable to place all content in the child files included by |\include|.
Any output contained in the main file will appear in all child documents
unless suppressed manually;
it cannot be suppressed automatically by the |\includeonly| directive
and thus should normally be avoided.
A method to include some content in the main file
by means of conditional processing is described in \secref{sec:conditional}.

%%%%%%%%%%%%%%%%%%%%%%%%%%%%%%%%%%%%%%%%
\paragraph{Page Numbering.}

When only a part of the document is compiled,
the appropriate numbering of pages
(as well as other status parameters)
is determined from the |.aux| files.
The latter contain information from previous passes.
However this information needs to propagate through
all intermediate child documents.
Therefore the page numbering in child documents may well
be inconsistent until the complete document is compiled at least once.

A useful (if unconventional) way to always ensure a consistent
page numbering is to restart the numbering in each child document
and denote the pages by `\textit{child}|.|\textit{page}'
where \textit{child} represents the chapter/section number of the child file.
This can be achieved by the command
|\numberwithin{page}{|\textit{child}|}|
of the \textsf{amsmath} package
where \textit{child} can be |chapter| or |section|
depending on the chosen structuring.
Alternatively, one can modify the macro |\thepage| appropriately
and reset the counter |page| at the start of each child file.

%%%%%%%%%%%%%%%%%%%%%%%%%%%%%%%%%%%%%%%%%%%%%%%%%%%%%%%%%%%%%%%%%%%%%%%%%%%%%%%%
\subsection{Conditional Processing}
\label{sec:conditional}

The package provides a mechanism to compile different versions
of a document. To customise the versions further some conditional processing
can come in handy to distinguish which version is being compiled.
The package provides two macros to describe the compilation context:

%%%%%%%%%%%%%%%%%%%%%%%%%%%%%%%%%%%%%%%%
\DescribeMacro{\ifchilddoc}
The conditional |\ifchilddoc| distinguishes between the compilation of
child documents and the main document:
%
\begin{center}
|\ifchilddoc |\textit{child-code}| |[|\||else |\textit{main-code}]| \||fi|
\end{center}

%%%%%%%%%%%%%%%%%%%%%%%%%%%%%%%%%%%%%%%%
\DescribeMacro{\childdocname}
\DescribeMacro{\childdocjob}
The macro |\childdocname| contains the filename (without extension)
of the main or child file being processed.
Note that |\childdocjob| will always contain the name of the main file.

%%%%%%%%%%%%%%%%%%%%%%%%%%%%%%%%%%%%%%%%
\paragraph{Title Page.}

Conditional processing can be used to include a title or banner page
in the main document when proper precautions are taken.
Importantly, the code in the main file should ensure that the page counter
(as well as other status parameters which are stored in the |.aux| files)
takes the same value after the conditional processing.
Otherwise the page numbers may take divergent values
depending on which part is compiled.

For example, a title page could be declared by:
%
\begin{center}
\begin{tabular}{l}
|\ifchilddoc\||else|\\
|\addtocounter{page}{-1}|\\
\textit{code for title page}\\
|\newpage|\\
|\||fi|
\end{tabular}
\end{center}
%
A banner page for the child documents can be generated by:
%
\begin{center}
\begin{tabular}{l}
|\ifchilddoc|\\
|\addtocounter{page}{-1}|\\
\textit{code for banner page}\\
|\newpage|\\
|\||fi|
\end{tabular}
\end{center}
%
Here one could write a message such as:
\begin{center}
|This is the part \childdocname{} of \childdocjob{}.|
\end{center}

%%%%%%%%%%%%%%%%%%%%%%%%%%%%%%%%%%%%%%%%%%%%%%%%%%%%%%%%%%%%%%%%%%%%%%%%%%%%%%%%
\subsection{Flags}
\label{sec:flags}

The package makes it easy to generate different versions
of the main or child documents.
To this end compilation flags can be defined
and assigned different default values.
They will be particularly useful in conjunction
with the forwarding mechanism described in \secref{sec:forward}.

For example, it may be useful to have a flag |\version|
which can be set to |draft| or |final|.
The document source will contain some conditional code
depending on the value of |\version|.
Suppose further, the flag should default to |final| for the main file
and to |draft| for child files
which is a natural assignment for editing the document.
This is achieved by placing the following code
in the preamble of the main document
(below the |\childdocmain| directive):
%
\begin{center}
\begin{tabular}{l}
|\ifchilddoc|\\
|\providecommand{\version}{draft}|\\
|\||else|\\
|\providecommand{\version}{final}|\\
|\||fi|
\end{tabular}
\end{center}
%
The definition by |\providecommand| makes sure
that previous definitions are not overwritten.
Further statements |\providecommand{\version}{...}|
can thus be added before the above code to override it.

For the main file, one might add a line
(between |\childdocmain| and the above block)
%
\begin{center}
|%\ifchilddoc\||else\providecommand{\version}{draft}\||fi|
\end{center}
%
which can be uncommented to produce a draft version.
Likewise one can add a line to the very top of a child file
(above the |\childdocof{|\textit{main}|}| directive)
%
\begin{center}
|%\providecommand{\version}{final}|
\end{center}
%
which can be uncommented to produce the final version of this child document.

%%%%%%%%%%%%%%%%%%%%%%%%%%%%%%%%%%%%%%%%%%%%%%%%%%%%%%%%%%%%%%%%%%%%%%%%%%%%%%%%
\subsection{Forwarding}
\label{sec:forward}

Different versions of the main or child documents
using compilation flags as described in \secref{sec:flags}
can be (permanently) stored in different files
for convenient compilation, viewing and distribution.
To this end, the package defines a command
to pass on compilation to a different file:

%%%%%%%%%%%%%%%%%%%%%%%%%%%%%%%%%%%%%%%%
\DescribeMacro{\childdocforward}
The command |\childdocforward| redirects processing to
another source file:
%
\begin{center}
\begin{tabular}{l}
|\input{childdoc.def}|\\
|\childdocforward[|\textit{main}|]{|\textit{dest}|}|\\
\end{tabular}
\end{center}
%
The argument \textit{dest} is the destination file
(without extension).
It should be the main file or one of the child files.
Note that further \textsf{childdoc} directives
such as |\childdocof| and |\childdocforward|
in the indicated file will be processed in this form.
The optional argument \textit{main}
passes on directly to the main file \textit{main}
while pretending to compile the child \textit{dest}.
This form behaves as if \textit{dest}
issues |\childdocof{|\textit{main}|}| right away,
and no further \textsf{childdoc} directives will be processed.

%%%%%%%%%%%%%%%%%%%%%%%%%%%%%%%%%%%%%%%%
\DescribeMacro{\...prefix}
In the alternative form |\childdocforwardprefix|,
%
\begin{center}
\begin{tabular}{l}
|\input{childdoc.def}|\\
|\childdocforwardprefix[|\textit{main}|]{|\textit{prefix}|}{|\textit{dest}|}|
\end{tabular}
\end{center}
%
the destination file is determined by a pattern
depending on the current file:
To make this work, the current file must be called
`{\textit{prefix}\hspace{0.2em}\textit{suffix}}'
with \textit{prefix} matching precisely the argument.
Processing is then passed on to the file
`{\textit{dest}\hspace{0.2em}\textit{suffix}}'.
Surely, the same effect is achieved by
directly specifying the
argument `{\textit{dest}\hspace{0.2em}\textit{suffix}}'
in the first form.
However, that requires to set up a different file
for each child. With the alternative form of the command
all these files can have exactly the same content
which simplifies setting them up and maintaining them.

For example, the following file |draft.tex|
with a compilation flag |\version| as described in \secref{sec:flags}
compiles the main document as a draft:
%
\begin{center}
\begin{tabular}{l}
|\def\version{draft}|\\
|\input{childdoc.def}|\\
|\childdocforward{|\textit{main}|}|
\end{tabular}
\end{center}
%
Likewise, the following files |final|\textit{nn}|.tex|
compile the final version of the child document
|child|\textit{nn}|.tex|:
%
\begin{center}
\begin{tabular}{l}
|\def\version{final}|\\
|\input{childdoc.def}|\\
|\childdocforwardprefix{final}{child}|
\end{tabular}
\end{center}
%

Note that when several versions of a main file and/or of each child file
are to be generated, it may be convenient to set up a |Makefile| or
shell script to automatise the process.

%%%%%%%%%%%%%%%%%%%%%%%%%%%%%%%%%%%%%%%%%%%%%%%%%%%%%%%%%%%%%%%%%%%%%%%%%%%%%%%%
\subsection{Command Line Processing}
\label{sec:commandline}

The effect of redirection files can also be achieved by invoking
the \LaTeX{} compiler with a more elaborate command line.
Most conveniently this should be done as part
of a shell script or a |Makefile|.

When using \textsf{childdoc} in the main file, the following
command lines effectively perform a redirection
(note that depending on the shell being used,
backslashes may have to be doubled: `|\|' $\to$ `|\\|'):
%
\begin{center}
|... -jobname "|\textit{target}|" |\\|"|[\textit{flags}]%
|\input{childdoc.def}\childdocforward[|\textit{main}|]{|\textit{dest}|}"|
\end{center}
%
Here \textit{target} is the name of the output file,
\textit{main} is the name of the main file
and \textit{dest} is the name of the main or child file to be processed
(all filenames without extensions).
The optional argument \textit{main} can be omitted
if \textit{main} matches \textit{dest}.
Optionally, compilation \textit{flags} can be defined via |\def| commands.
This command line makes the \TeX{} engine believe
it is compiling the file \textit{target}
whose content is specified as the latter parameter.
The provided code then forwards the processing to
\textit{main} or \textit{dest} as described in \secref{sec:forward}.

%%%%%%%%%%%%%%%%%%%%%%%%%%%%%%%%%%%%%%%%%%%%%%%%%%%%%%%%%%%%%%%%%%%%%%%%%%%%%%%%
\subsection{Include by Input}
\label{sec:input}

Including child documents by |\include| has some restrictions by design.
Most notably, the content of a child document always occupies
its own set of pages; pages cannot be shared between child documents.
Usually, this behaviour makes perfect sense
because each child document contain an essential part of the document.
However, in some situations it may be desirable to compose
a document from a collection of parts
without having mandatory page breaks between then.
For this case, the package
provides a mechanism to include parts
by |\input| which can also be processed individually.
However, by construction this mechanism
requires manual handling of the content to be output.

%%%%%%%%%%%%%%%%%%%%%%%%%%%%%%%%%%%%%%%%
\DescribeMacro{\ifchilddocmanual}
The main file should be prepared as usual, see \secref{sec:include}.
However, the document body must make a distinction
between processing of an individual part and of the main document, e.g.:
%
\begin{center}
\begin{tabular}{l}
|\ifchilddocmanual|\\
|\input{\childdocname}|\\
|\||else|\\
\textit{document body with }|\input{|\textit{part}|}|\\
|\||fi|
\end{tabular}
\end{center}
%
The conditional |\ifchilddocmanual| is true whenever
a part to be included by |\input| is being compiled,
and the name of the part is stored in |\childdocname|.

%%%%%%%%%%%%%%%%%%%%%%%%%%%%%%%%%%%%%%%%
\DescribeMacro{\childdocby}
Each part to be included by |\input| should start with:
%
\begin{center}
\begin{tabular}{l}
|\input{childdoc.def}|\\
|\childdocby{|\textit{main}|}|\\
\end{tabular}
\end{center}
%
The directive |\childdocby| is similar to |\childdocof|
described in \secref{sec:include},
but the subsequent selection of content must be done manually.
To that end, both |\ifchilddoc| and |\ifchilddocmanual|
will be true upon processing of a part,
and the name of the part is stored in |\childdocname|.
Note that |\jobname| will be set to the filename of the current part
so that each part receives an individual |.aux| file
that does not interfere with the |.aux| file(s) of the main document.
This behaviour can be altered by the alternative form
|\childdocby[*]{|\textit{main}|}| (with a non-empty optional argument)
which uses the |.aux| file of the main document
by setting |\jobname| to \textit{main}.

%%%%%%%%%%%%%%%%%%%%%%%%%%%%%%%%%%%%%%%%%%%%%%%%%%%%%%%%%%%%%%%%%%%%%%%%%%%%%%%%
\subsection{Driver Development}
\label{sec:driver}

The \textsf{childdoc} mechanism can also be use for the development
of definition files such as \LaTeX{} styles or classes.
This case differs from the above setup with multiple parts
included by |\include| in that no |\includeonly| should be invoked.
This can be achieved by starting the include file
(before |\ProvidesPackage|) with:
%
\begin{center}
\begin{tabular}{l}
|\input{childdoc.def}|\\
|\childdocforward{|\textit{main}|}|\\
\end{tabular}
\end{center}
%
or alternatively with:
%
\begin{center}
\begin{tabular}{l}
|\input{childdoc.def}|\\
|\childdocby{|\textit{main}|}|\\
\end{tabular}
\end{center}
%
Both forms have slightly different effects as described above.
The main file is prepared as usual, see \secref{sec:include}.

%%%%%%%%%%%%%%%%%%%%%%%%%%%%%%%%%%%%%%%%%%%%%%%%%%%%%%%%%%%%%%%%%%%%%%%%%%%%%%%%
\subsection{Legacy Detection}
\label{sec:detection}

The directive |\childdocmain| in the main file can detect
whether the complete document or merely a child is to be compiled
even without using the directive |\childdocof|.
This method is deprecated because it is less robust
and there is no compelling reason to use it;
it is merely provided for backward compatibility
and it may be removed in future versions.

If the detection mechanism is to be used,
it is mandatory to correctly specify
the filename of the main file as the argument of |\childdocmain|:
%
\begin{center}
\begin{tabular}{l}
|\input{childdoc.def}|\\
|\childdocmain{|\textit{main}|}|\\
\end{tabular}
\end{center}
%
If |\jobname| does not match the argument \textit{main} of |\childdocmain|,
it is assumed that |\jobname| points to the child file to be compiled.
When using |\childdocmain| with the main file specified as argument,
it suffices to start a child file
with just |\input{|\textit{main}|}|
without loading of the package and using |\childdocof|.
If instead all processing is done
with the appropriate \textsf{childdoc} directives,
the argument of \textit{main} of |\childdocmain| can be empty.

An alternative version of the command line processing described
in \secref{sec:commandline} using the detection mechanism reads:
%
\begin{center}
|... -jobname "|\textit{target}|" "|[\textit{flags}]%
[|\def\jobname{|\textit{dest}|}|]|\input{|\textit{main}|}"|
\end{center}

%%%%%%%%%%%%%%%%%%%%%%%%%%%%%%%%%%%%%%%%%%%%%%%%%%%%%%%%%%%%%%%%%%%%%%%%%%%%%%%%
\subsection{Manual Code}
\label{sec:manual}

In case one cannot be certain whether the definitions file |childdoc.def|
is installed on the target \TeX{} distribution
and one prefers not to ship it,
it is conceivable to paste a few relevant commands into the sources.

To that end, drop all statements |\input{childdoc.def}|
and perform the replacements as outlined below.
Instead of |\childdocmain{|\textit{main}|}| add the following code
to the top of the main file:
%
\begin{center}
\begin{tabular}{l}
|\||ifdefined\childdocname\endinput\||fi\newif\ifchilddoc|\\
|\edef\childdocname{\scantokens\expandafter{\jobname\noexpand}}|\\
|\def\childdocmain{|\textit{main}|}\||ifx\childdocmain\childdocname\||else|\\
|\childdoctrue\includeonly{\childdocname}\let\jobname\childdocmain\||fi|\\
\end{tabular}
\end{center}
%
Instead of |\childdocof{|\textit{main}|}| just include the main file
at the top of each child file:
%
\begin{center}
|\input{|\textit{main}|}|
\end{center}
%
A simple redirection |\childdocforward{|\textit{dest}|}| is achieved by:
%
\begin{center}
|\def\jobname{|\textit{dest}|}\input{\jobname}|
\end{center}
%
The redirection with prefix
|\childdocforwardprefix[|\textit{prefix}|]{|\textit{dest}|}|
is accomplished by:
%
\begin{center}
\begin{tabular}{l}
|{\edef\jobname{\scantokens\expandafter{\jobname\noexpand}}|\\
|\def\redirectjob |\textit{prefix}|#1~~~{\gdef\jobname{|\textit{dest}|#1}}|\\
|\expandafter\redirectjob\jobname~~~}\input{\jobname}|
\end{tabular}
\end{center}

In an alternative approach,
child documents can be compiled by a specific command line
without additional code or specific definitions:
%
\begin{center}
|... -jobname "|\textit{target}|" "|[\textit{flags}]%
|\includeonly{|\textit{dest}|}\input{|\textit{main}|}"|
\end{center}
%

%%%%%%%%%%%%%%%%%%%%%%%%%%%%%%%%%%%%%%%%%%%%%%%%%%%%%%%%%%%%%%%%%%%%%%%%%%%%%%%%
%%%%%%%%%%%%%%%%%%%%%%%%%%%%%%%%%%%%%%%%%%%%%%%%%%%%%%%%%%%%%%%%%%%%%%%%%%%%%%%%
\section{Information}

%%%%%%%%%%%%%%%%%%%%%%%%%%%%%%%%%%%%%%%%%%%%%%%%%%%%%%%%%%%%%%%%%%%%%%%%%%%%%%%%
\subsection{Copyright}

Copyright \copyright{} 2017--2018 Niklas Beisert

This work may be distributed and/or modified under the
conditions of the \LaTeX{} Project Public License, either version 1.3
of this license or (at your option) any later version.
The latest version of this license is in
  \url{http://www.latex-project.org/lppl.txt}
and version 1.3 or later is part of all distributions of \LaTeX{}
version 2005/12/01 or later.

This work has the LPPL maintenance status `maintained'.

The Current Maintainer of this work is Niklas Beisert.

This work consists of the files |README.txt|, |childdoc.ins| and |childdoc.dtx|
as well as the derived files |childdoc.def|, |cdocsamp.tex|
with |cdocsch1.tex|, |cdocsch2.tex|, |cdocspt3.tex|, |cdocspt4.tex|,
|cdocsdrf.tex|, |cdocsfn1.tex|, |cdocsfn2.tex|
as well as |childdoc.pdf|.

%%%%%%%%%%%%%%%%%%%%%%%%%%%%%%%%%%%%%%%%%%%%%%%%%%%%%%%%%%%%%%%%%%%%%%%%%%%%%%%%
\subsection{Files and Installation}

The package consists of the files:
%
\begin{center}
\begin{tabular}{ll}
    |README.txt|   & readme file \\
    |childdoc.ins| & installation file \\
    |childdoc.dtx| & source file \\
    |childdoc.def| & definition file \\
    |cdocsamp.tex| & sample main file \\
    |cdocsch1.tex| & sample include file \\
    |cdocsch2.tex| & sample include file \\
    |cdocspt3.tex| & sample part file \\
    |cdocspt4.tex| & sample part file \\
    |cdocsdrf.tex| & sample redirection file \\
    |cdocsfn1.tex| & sample redirection file \\
    |cdocsfn2.tex| & sample redirection file \\
    |childdoc.pdf| & manual
\end{tabular}
\end{center}
%
The distribution consists of the files
|README.txt|, |childdoc.ins| and |childdoc.dtx|.
%
\begin{itemize}
\item
Run (pdf)\LaTeX{} on |childdoc.dtx|
to compile the manual |childdoc.pdf| (this file).
\item
Run \LaTeX{} on |childdoc.ins| to create the definitions file |childdoc.def|
and the sample |cdocsamp.tex| with include files
|cdocsch1.tex|, |cdocsch2.tex|, |cdocspt3.tex|, |cdocspt4.tex|,
|cdocsdrf.tex|, |cdocsfn1.tex|, |cdocsfn2.tex|.
Then copy the file |childdoc.def| to an appropriate directory of your \LaTeX{}
distribution, e.g.\ \textit{texmf-root}|/tex/latex/childdoc|.
\end{itemize}

%%%%%%%%%%%%%%%%%%%%%%%%%%%%%%%%%%%%%%%%%%%%%%%%%%%%%%%%%%%%%%%%%%%%%%%%%%%%%%%%
\subsection{Related CTAN Packages}

There are several other packages which offer a similar functionality:
%
\begin{itemize}
\item
The packages
\href{http://ctan.org/pkg/docmute}{\textsf{docmute}},
\href{http://ctan.org/pkg/includex}{\textsf{includex}} and
\href{http://ctan.org/pkg/standalone}{\textsf{standalone}}
provide commands to include only the document body of
a child file thus allowing both files to be compiled individually.
\item
The packages \href{http://ctan.org/pkg/subdocs}{\textsf{subdocs}}
and \href{http://ctan.org/pkg/subfiles}{\textsf{subfiles}}
provide structures in which the main and child documents can be
encapsulated and allowing them to be compiled individually.
The inclusion mechanism is different from the conventional |\include|.
\item
The package \href{http://ctan.org/pkg/combine}{\textsf{combine}}
is an elaborate solution to combine several documents into one.
\end{itemize}
%
See also the CTAN topic \href{http://ctan.org/topic/subdocs}{\textsf{subdocs}}
for further related packages.
The present package differs from the above solutions in that
a document structure constructed with the conventional |\include| mechanism
just needs two extra commands at the top of every file
such that all constituent files can be compiled individually.

%%%%%%%%%%%%%%%%%%%%%%%%%%%%%%%%%%%%%%%%%%%%%%%%%%%%%%%%%%%%%%%%%%%%%%%%%%%%%%%%
%\subsection{Feature Suggestions}
%
%The following is a list of features which may be useful for future
%versions of this package:
%%
%\begin{itemize}
%\item
%\ldots
%\end{itemize}

%%%%%%%%%%%%%%%%%%%%%%%%%%%%%%%%%%%%%%%%%%%%%%%%%%%%%%%%%%%%%%%%%%%%%%%%%%%%%%%%
\subsection{Revision History}

%%%%%%%%%%%%%%%%%%%%%%%%%%%%%%%%%%%%%%%%
\paragraph{v2.0:} 2018/12/30

\begin{itemize}
\item
immediate forward processing
\item
added |\childdocby| mechanism
\item
manual restructured
\end{itemize}

%%%%%%%%%%%%%%%%%%%%%%%%%%%%%%%%%%%%%%%%
\paragraph{v1.6:} 2018/01/17

\begin{itemize}
\item
application for development of include files
\item
corrections to manual
\end{itemize}

%%%%%%%%%%%%%%%%%%%%%%%%%%%%%%%%%%%%%%%%
\paragraph{v1.5:} 2017/05/21

\begin{itemize}
\item
more complete structuring introduced
\item
|\childdocof| introduced
\item
|\childdoc| renamed to |\childdocmain|
\item
|\childredirect| renamed to |\childdocforward| and |\childdocforwardprefix|
and functionality expanded
\end{itemize}

%%%%%%%%%%%%%%%%%%%%%%%%%%%%%%%%%%%%%%%%
\paragraph{v1.0:} 2017/04/27

\begin{itemize}
\item
manual and install package
\item
first version published on CTAN
\end{itemize}

%%%%%%%%%%%%%%%%%%%%%%%%%%%%%%%%%%%%%%%%
\paragraph{v0.6:} 2017/04/26

\begin{itemize}
\item
redirection mechanism added
\end{itemize}

%%%%%%%%%%%%%%%%%%%%%%%%%%%%%%%%%%%%%%%%
\paragraph{v0.5:} 2017/04/26

\begin{itemize}
\item
functionality in definition file
\end{itemize}


%%%%%%%%%%%%%%%%%%%%%%%%%%%%%%%%%%%%%%%%%%%%%%%%%%%%%%%%%%%%%%%%%%%%%%%%%%%%%%%%
%%%%%%%%%%%%%%%%%%%%%%%%%%%%%%%%%%%%%%%%%%%%%%%%%%%%%%%%%%%%%%%%%%%%%%%%%%%%%%%%
%%%%%%%%%%%%%%%%%%%%%%%%%%%%%%%%%%%%%%%%%%%%%%%%%%%%%%%%%%%%%%%%%%%%%%%%%%%%%%%%
\appendix

\settowidth\MacroIndent{\rmfamily\scriptsize 000\ }

 \DocInput{childdoc.dtx}

\end{document}
%</driver>
% \fi
%
% %%%%%%%%%%%%%%%%%%%%%%%%%%%%%%%%%%%%%%%%%%%%%%%%%%%%%%%%%%%%%%%%%%%%%%%%%%%%%%
% %%%%%%%%%%%%%%%%%%%%%%%%%%%%%%%%%%%%%%%%%%%%%%%%%%%%%%%%%%%%%%%%%%%%%%%%%%%%%%
% \section{Sample}
%\iffalse
%<*samplemain>
%\fi
%
% The following presents a sample document
% with two chapters, two parts, a title page,
% a compile flag as well as three forwarding files to set the flag.
% It consists of eight |.tex| files:
% \begin{center}
% \begin{tabular}{ll}
% |cdocsamp.tex|&main file\\
% |cdocsch1.tex|&include file for chapter 1\\
% |cdocsch2.tex|&include file for chapter 2\\
% |cdocspt3.tex|&include file for part 3\\
% |cdocspt4.tex|&include file for part 4\\
% |cdocsdrf.tex|&forwarding file for main file in draft mode\\
% |cdocsfi1.tex|&forwarding file for final version of chapter 1\\
% |cdocsfi2.tex|&forwarding file for final version of chapter 2\\
% \end{tabular}
% \end{center}
% Each of the eight files can be compiled directly by the \LaTeX{} compiler.
%
% %%%%%%%%%%%%%%%%%%%%%%%%%%%%%%%%%%%%%%
% \paragraph{Main File.}
%
% The main file is called |cdocsamp.tex|.
%
% Load the \textsf{childdoc} definitions and
% declare the filename for the main document:
%    \begin{macrocode}
\input{childdoc.def}
\childdocmain{}
%    \end{macrocode}

% Optional override for |\version| flag:
%    \begin{macrocode}
%%\ifchilddoc\else\providecommand{\version}{draft}\fi
%    \end{macrocode}

% Define the default values for the |\version| flag
% (|final| for the main file and |draft| for childs):
%    \begin{macrocode}
\ifchilddoc
\providecommand{\version}{draft}
\else
\providecommand{\version}{final}
\fi
%    \end{macrocode}

% Load the standard document class:
%    \begin{macrocode}
\documentclass[12pt]{article}
%    \end{macrocode}

% Start the document body:
%    \begin{macrocode}
\begin{document}
%    \end{macrocode}

% Declare a title page.
% Print title, part of document being processed and version flag:
%    \begin{macrocode}
\addtocounter{page}{-1}
\begin{center}
{\LARGE\bfseries{}childdoc example\par}
\vspace{1cm}
\ifchilddoc
\ifchilddocmanual part\else chapter\fi:
`\childdocname' of `\childdocjob'\par
\else
main document: `\childdocjob'\par
\fi
version: \version\par
\end{center}
\newpage
%    \end{macrocode}

% Manually include selected file,
% otherwise process as usual:
%    \begin{macrocode}
\ifchilddocmanual
\section*{part `\childdocname'}
\input{\childdocname}
\else
%    \end{macrocode}

% Include the two chapters:
%    \begin{macrocode}
\include{cdocsch1}
\include{cdocsch2}
%    \end{macrocode}

% Include the two parts unless only chapters should be displayed:
%    \begin{macrocode}
\ifchilddoc\else
\section{part three}
\input{cdocspt3}
\section{part four}
\input{cdocspt4}
\fi
%    \end{macrocode}

% Process as usual until here:
%    \begin{macrocode}
\fi
%    \end{macrocode}

% End of document body:
%    \begin{macrocode}
\end{document}
%    \end{macrocode}
%\iffalse
%</samplemain>
%\fi
%
% %%%%%%%%%%%%%%%%%%%%%%%%%%%%%%%%%%%%%%
% \paragraph{Chapter Include Files.}
%
% The include files are called |cdocsch1.tex| and |cdocsch2.tex|.
%
%\iffalse
%<*samplechap1|samplechap2>
%\fi

% Optional override for |\version| flag:
%    \begin{macrocode}
%%\providecommand{\version}{final}
%    \end{macrocode}

% Include the main document:
%    \begin{macrocode}
\input{childdoc.def}
\childdocof{cdocsamp}
%    \end{macrocode}

%\iffalse
%</samplechap1|samplechap2>
%\fi
%
%\iffalse
%<*samplechap1>
%\fi
% Some text for chapter 1:
%    \begin{macrocode}
\section{one}
some text in chapter one
%    \end{macrocode}

%\iffalse
%</samplechap1>
%\fi
% Some text for chapter 2:
%\iffalse
%<*samplechap2>
%\fi
%    \begin{macrocode}
\section{two}
more text in chapter two
%    \end{macrocode}

%\iffalse
%</samplechap2>
%\fi
%
% %%%%%%%%%%%%%%%%%%%%%%%%%%%%%%%%%%%%%%
% \paragraph{Part Include Files.}
%
% The include files are called |cdocspt3.tex| and |cdocspt4.tex|.
%
%\iffalse
%<*samplepart3|samplepart4>
%\fi

% Optional override for |\version| flag:
%    \begin{macrocode}
%%\providecommand{\version}{final}
%    \end{macrocode}

% Include the main document:
%    \begin{macrocode}
\input{childdoc.def}
\childdocby{cdocsamp}
%    \end{macrocode}

%\iffalse
%</samplepart3|samplepart4>
%\fi
%
%\iffalse
%<*samplepart3>
%\fi
% Some text for part 3:
%    \begin{macrocode}
some text in part three
%    \end{macrocode}

%\iffalse
%</samplepart3>
%\fi
% Some text for part 4:
%\iffalse
%<*samplepart4>
%\fi
%    \begin{macrocode}
more text in part four
%    \end{macrocode}

%\iffalse
%</samplepart4>
%\fi
%
% %%%%%%%%%%%%%%%%%%%%%%%%%%%%%%%%%%%%%%
% \paragraph{Forwarding for a Complete Draft.}
%
% The following forwarding file |cdocsdrf.tex|
% compiles the main document in draft mode:
%\iffalse
%<*sampledraft>
%\fi
%    \begin{macrocode}
\def\version{draft}
\input{childdoc.def}
\childdocforward{cdocsamp}
%    \end{macrocode}

%\iffalse
%</sampledraft>
%\fi
%
% %%%%%%%%%%%%%%%%%%%%%%%%%%%%%%%%%%%%%%
% \paragraph{Forwarding for Final Version of the Chapters.}
%
% The following forwarding files |cdocsfn1.tex| and |cdocsfn2.tex|
% (with identical content)
% compile the final versions of the child documents
% |cdocsch1.tex| and |cdocsch2.tex|, respectively:
%\iffalse
%<*samplefinal>
%\fi
%    \begin{macrocode}
\def\version{final}
\input{childdoc.def}
\childdocforwardprefix[cdocsamp]{cdocsfn}{cdocsch}
%    \end{macrocode}

%\iffalse
%</samplefinal>
%\fi
%
% %%%%%%%%%%%%%%%%%%%%%%%%%%%%%%%%%%%%%%
% \paragraph{Command Line Processing.}
%
% The following three command lines generate the output files
% |cdocscld|, |cdocscl1| and |cdocscl2|
% which should be identical to
% |cdocsdrf|, |cdocsch1| and |cdocsfn2|, respectively:
% \begin{center}
% \begin{tabular}{l}
% |latex -jobname cdocscld \|\\
% |  "\def\version{draft}\input{childdoc.def}\childdocforward{cdocsamp}"|\\
% |latex -jobname cdocscl1 \|\\
% |  "\input{childdoc.def}\childdocforward[cdocsamp]{cdocsch1}"|\\
% |latex -jobname cdocscl2 \|\\
% |  "\def\version{final}\input{childdoc.def}\childdocforward{cdocsch2}"|
% \end{tabular}
% \end{center}
% Note that the trailing backslash on each first line
% merely continues the input to the second line
% (for convenient cut ant paste).
% Furthermore, the command |latex| can be replaced by any
% of its alternative versions such as |pdflatex|.
%
% %%%%%%%%%%%%%%%%%%%%%%%%%%%%%%%%%%%%%%%%%%%%%%%%%%%%%%%%%%%%%%%%%%%%%%%%%%%%%%
% %%%%%%%%%%%%%%%%%%%%%%%%%%%%%%%%%%%%%%%%%%%%%%%%%%%%%%%%%%%%%%%%%%%%%%%%%%%%%%
% \section{Implementation}
%\iffalse
%<*package>
%\fi
%
% This section describes the definitions file |childdoc.def|.

% The definitions cannot be loaded using |\usepackage| or |\RequirePackage|
% which has a mechanism to prevent loading a style file more than once.
% When loading the definitions by means of |\input|
% multiple instances have to be prevented manually:
%\iffalse
%This code needs to be before the `\ProvidesFile' directive
%which is defined at the beginning of this file.
%Therefore it is also placed there and commented out here.
%</package>
%<*discard>
%\fi
%    \begin{macrocode}
\ifdefined\childdocmain\endinput\fi
%    \end{macrocode}
%\iffalse
%</discard>
%<*package>
%\fi
%
% \macro{\ifchilddoc}
% \macro{\ifchilddocmanual}
% The conditional |\ifchilddoc| tells whether a
% child (true) or main (false) document is being compiled.
% The conditional |\ifchilddocmanual| tells whether
% the |\includeonly| mechanism is used (false) or
% the selection of child files must be performed manually (true).
% The definitions initialise to false:
%    \begin{macrocode}
\newif\ifchilddoc
\newif\ifchilddocmanual
%    \end{macrocode}

% \macro{\childdocname}
% \macro{\childdocjob}
% The macro |\childdocname| stores the name of the main document
% to be compiled. The macro |\childdocjob| stores the name of
% the document on which the \LaTeX{} compiler was originally invoked.
% The content of |\jobname| cannot be compared
% to filenames specified in the source due to different catcodes.
% The following code rescans |\jobname|, stores the result
% in |\childdocname| and saves a copy in |\childdocjob|:
%    \begin{macrocode}
\edef\childdocname{\scantokens\expandafter{\jobname\noexpand}}
\let\childdocjob\childdocname
%    \end{macrocode}

% \macro{\childdocdisable}
% The macro |\childdocdisable| prevents the main file
% from being processed more than once.
% At this stage, the main document command |\childdocmain|
% is assumed to be called once again where it should do nothing.
% Any subsequent call to it should prevent
% a secondary processing of the main document
% It overwrites the forwarding commands
% |\childdocof| and |\childdocforward|
% with empty macros to prevent further inclusions of the main document:
%    \begin{macrocode}
\newcommand{\childdocdisable}
{
  \renewcommand{\childdocmain}[1]{\renewcommand{\childdocmain}[1]{\endinput}}
  \renewcommand{\childdocof}[1]{}
  \renewcommand{\childdocby}[2][]{}
  \renewcommand{\childdocforward}[2][]{}
  \renewcommand{\childdocdisable}{}
}
%    \end{macrocode}

% \macro{\childdocmain}
% The macro |\childdocmain| is to be called at the top of the main file
% with nothing or the main filename (without extension) as argument.
% First, it breaks loops.
% If the argument is not empty and does not match |\childdocname|
% (which is set by the first inclusion of |childdoc.def|),
% |\ifchilddoc| is set to true, |\includeonly| is applied to the child file
% and |\jobname| is set to the main file
% (for proper handling of |.aux| files):
%    \begin{macrocode}
\newcommand{\childdocmain}[1]
{
  \childdocdisable\childdocmain{}
  \if?#1?\else
    \begingroup
      \def\childdoctmp{#1}
      \ifx\childdoctmp\childdocname
        \def\childdoctmp{}
      \else
        \def\childdoctmp
        {
          \childdoctrue
          \includeonly{\childdocname}
          \def\childdocjob{#1}
          \def\jobname{#1}
        }
      \fi
      \expandafter
    \endgroup
    \childdoctmp
  \fi
}
%    \end{macrocode}

% \macro{\childdocof}
% The command |\childdocof| redirects
% compilation to the main file |#1|.
%    \begin{macrocode}
\newcommand{\childdocof}[1]
{
  \childdocdisable
  \childdoctrue
  \includeonly{\childdocname}
  \def\jobname{#1}
  \def\childdocjob{#1}
  \input{#1}
}
%    \end{macrocode}

% \macro{\childdocby}
% The command |\childdocby| ....
%    \begin{macrocode}
\newcommand{\childdocby}[2][]
{
  \childdocdisable
  \childdoctrue
  \childdocmanualtrue
  \if?#1?\else
    \def\jobname{#2}
  \fi
  \def\childdocjob{#2}
  \input{#2}
  \endinput
}
%    \end{macrocode}

% \macro{\childdocforward}
% The command |\childdocforward| redirects
% compilation to the main file or
% (if the optional argument is given) a child file.
% Parameters are set as if the main file
% or a child file starting with |\childdocof| was compiled.
% Then compilation is handed over to the main file:
%    \begin{macrocode}
\newcommand{\childdocforward}[2][]
{
  \begingroup
    \if?#1?
      \def\childdoctmp
      {
        \def\childdocname{#2}
        \def\childdocjob{#2}
        \def\jobname{#2}
        \input{#2}
        \endinput
      }
    \else
      \def\childdoctmp
      {
        \childdocdisable
        \def\childdocname{#2}
        \childdoctrue
        \includeonly{#2}
        \def\childdocjob{#1}
        \def\jobname{#1}
        \input{#1}
        \endinput
      }
    \fi
    \expandafter
  \endgroup
  \childdoctmp
}
%    \end{macrocode}

% \macro{\childdocforwardprefix}
% The command |\childdocforwardprefix| redirects
% compilation to the main or a child file by means of a pattern.
% The prefix |#1| in the current filename is replaced by |#2|
% and the suffix of the current filename is kept
% (it is assumed that the filename does not contain the substring `|~~~|'
% which is used as a delimiter).
% Compilation is handed over to the new file by |\childdocforward|:
%    \begin{macrocode}
\newcommand{\childdocforwardprefix}[3][]
{
  \begingroup
    \def\childdocextract #2##1~~~{\def\childdoctmp{\childdocforward[#1]{#3##1}}}
    \expandafter\childdocextract\childdocname~~~
    \expandafter
  \endgroup
  \childdoctmp
}
%    \end{macrocode}

% \macro{\childdoc}
% The deprecated macro |\childdoc| is a legacy version of |\childdocmain|:
%    \begin{macrocode}
\newcommand{\childdoc}{\childdocmain}
%    \end{macrocode}

% \macro{\childdocredirect}
% The deprecated macro |\childdocredirect| is a legacy version
% of |\childdocforward| and |\childdocforwardprefix|:
%    \begin{macrocode}
\newcommand{\childdocredirect}[2][]
{
  \begingroup
    \if?#1?
      \def\childdoctmp{\childdocforward{#2}}
    \else
      \def\childdoctmp{\childdocforwardprefix{#1}{#2}}
    \fi
    \expandafter
  \endgroup
  \childdoctmp
}
%    \end{macrocode}

%\iffalse
%</package>
%\fi
%
\endinput
|\\
|\childdocforward{|\textit{main}|}|
\end{tabular}
\end{center}
%
Likewise, the following files |final|\textit{nn}|.tex|
compile the final version of the child document
|child|\textit{nn}|.tex|:
%
\begin{center}
\begin{tabular}{l}
|\def\version{final}|\\
|% \iffalse
%
% childdoc.dtx Copyright (C) 2017-2018 Niklas Beisert
%
% This work may be distributed and/or modified under the
% conditions of the LaTeX Project Public License, either version 1.3
% of this license or (at your option) any later version.
% The latest version of this license is in
%   http://www.latex-project.org/lppl.txt
% and version 1.3 or later is part of all distributions of LaTeX
% version 2005/12/01 or later.
%
% This work has the LPPL maintenance status `maintained'.
%
% The Current Maintainer of this work is Niklas Beisert.
%
% This work consists of the files childdoc.dtx and childdoc.ins
% and the derived files childdoc.def and cdocsamp.tex with
% cdocsch1.tex, cdocsch2.tex, cdocsdrf.tex, cdocsfn1.tex, cdocsfn2.tex.
%
%<package>\ifdefined\childdocmain\endinput\fi
%<package>\ProvidesFile{childdoc.def}[2018/12/30 v2.0 child document driver]
%<samplemain>\ProvidesFile{cdocsamp.tex}[2018/12/30 v2.0 sample for childdoc]
%<*driver>
%\ProvidesFile{childdoc.drv}[2018/12/30 v2.0 childdoc reference manual file]
\PassOptionsToClass{10pt,a4paper}{article}
\documentclass{ltxdoc}

\usepackage[margin=35mm]{geometry}
\usepackage{hyperref}
\usepackage{hyperxmp}
\usepackage[usenames]{color}

\hypersetup{colorlinks=true}
\hypersetup{pdfstartview=FitH}
\hypersetup{pdfpagemode=UseNone}
\hypersetup{pdfsource={}}
\hypersetup{pdflang={en-UK}}
\hypersetup{pdfcopyright={Copyright 2017-2018 Niklas Beisert.
  This work may be distributed and/or modified under the
  conditions of the LaTeX Project Public License, either version 1.3
  of this license or (at your option) any later version.}}
\hypersetup{pdflicenseurl={http://www.latex-project.org/lppl.txt}}
\hypersetup{pdfcontactaddress={ETH Zurich, ITP, HIT K,
  Wolfgang-Pauli-Strasse 27}}
\hypersetup{pdfcontactpostcode={8093}}
\hypersetup{pdfcontactcity={Zurich}}
\hypersetup{pdfcontactcountry={Switzerland}}
\hypersetup{pdfcontactemail={nbeisert@itp.phys.ethz.ch}}
\hypersetup{pdfcontacturl={http://people.phys.ethz.ch/\xmptilde nbeisert/}}

\newcommand{\secref}[1]{\hyperref[#1]{section \ref*{#1}}}

\parskip1ex
\parindent0pt
\let\olditemize\itemize
\def\itemize{\olditemize\parskip0pt}

\begin{document}

\title{The \textsf{childdoc} Package}
\hypersetup{pdftitle={The childdoc Package}}
\author{Niklas Beisert\\[2ex]
  Institut f\"ur Theoretische Physik\\
  Eidgen\"ossische Technische Hochschule Z\"urich\\
  Wolfgang-Pauli-Strasse 27, 8093 Z\"urich, Switzerland\\[1ex]
  \href{mailto:nbeisert@itp.phys.ethz.ch}
  {\texttt{nbeisert@itp.phys.ethz.ch}}}
\hypersetup{pdfauthor={Niklas Beisert}}
\hypersetup{pdfsubject={Manual for the LaTeX2e Package childdoc}}
\date{30 December 2018, \textsf{v2.0}}
\maketitle

\begin{abstract}\noindent
\textsf{childdoc} is a \LaTeXe{} package
that enables the direct compilation
of document sections included by |\include|
to individual files.
\end{abstract}

\begingroup
\parskip0ex
\tableofcontents
\endgroup

%%%%%%%%%%%%%%%%%%%%%%%%%%%%%%%%%%%%%%%%%%%%%%%%%%%%%%%%%%%%%%%%%%%%%%%%%%%%%%%%
%%%%%%%%%%%%%%%%%%%%%%%%%%%%%%%%%%%%%%%%%%%%%%%%%%%%%%%%%%%%%%%%%%%%%%%%%%%%%%%%
\section{Introduction}

\LaTeX{} provides a mechanism to structure a large document (such as a book)
into a main file and several child files (containing the chapters)
using the |\include| command.
This mechanism is beneficial for documents
which span hundreds of pages in order to
make the source file(s) more manageable.
Moreover, compilation can be restricted to
selected child files by means of the |\includeonly| command.
The latter feature can be used to reduce the compilation time while editing
(this was significantly more useful in the earlier days of \LaTeX{})
or to generate a smaller document which is easier to navigate.
Another application of |\includeonly| is to generate
documents consisting of selected parts of the complete document.

However, there are a few drawbacks of the plain |\include| mechanism:
\begin{itemize}
\item
The child files cannot be compiled on their own,
they can only be compiled via the main file.
A naive editing environment
(such as a text editor with an option
to have the current file processed by \LaTeX)
may require one to switch to the main file before compiling;
attempting to compile the child file produces errors.
\item
The main file must be modified (each time)
to adjust the |\includeonly| command
to the present needs. This easily leaves the main file in a messy state.
\item
The generated document will always carry the filename
of the main document. This is inconvenient if
several child files are to be compiled and
to be kept for distribution.
\end{itemize}

The present package provides a simple interface
to make child files individually compilable by \LaTeX{}.
Compiling a child file then has the same effect as compiling
the main file with an |\includeonly| command
to select the appropriate child.
Moreover the generated document will carry the name of the child
rather than the main file.
This resolves all three above issues.

This feature is meant to make the editing of books,
thesis documents and lecture notes somewhat more convenient.
However, the package can also be used efficiently for
composing a series of documents (such as exercise sheets)
which are typically distributed individually.
It then assists the author in generating the individual documents
(potentially in different versions)
as well as a document containing the collected series.
Another application is in developing style files
or other kinds of included material
where compilation of the style file could redirect
to a sample or test file.

%%%%%%%%%%%%%%%%%%%%%%%%%%%%%%%%%%%%%%%%%%%%%%%%%%%%%%%%%%%%%%%%%%%%%%%%%%%%%%%%
%%%%%%%%%%%%%%%%%%%%%%%%%%%%%%%%%%%%%%%%%%%%%%%%%%%%%%%%%%%%%%%%%%%%%%%%%%%%%%%%
\section{Usage}

First of all, the package \textsf{childdoc} is \emph{not} a standard
\LaTeXe{} |.sty| style file! Therefore it needs to be invoked in
a non-standard way.

%%%%%%%%%%%%%%%%%%%%%%%%%%%%%%%%%%%%%%%%%%%%%%%%%%%%%%%%%%%%%%%%%%%%%%%%%%%%%%%%
\subsection{Included Files}
\label{sec:include}

%%%%%%%%%%%%%%%%%%%%%%%%%%%%%%%%%%%%%%%%
\DescribeMacro{\childdocmain}
To use the package, add the commands
\begin{center}
\begin{tabular}{l}
|\input{childdoc.def}|\\
|\childdocmain{}|\\
\end{tabular}
\end{center}
at the very top of the main \LaTeX{} file,
in particular \emph{before} the |\documentclass| statement!
The argument of |\childdocmain| should be left empty
(but it must be present).

%%%%%%%%%%%%%%%%%%%%%%%%%%%%%%%%%%%%%%%%
\DescribeMacro{\childdocof}
Furthermore, add the commands
\begin{center}
\begin{tabular}{l}
|\input{childdoc.def}|\\
|\childdocof{|\textit{main}|}|\\
\end{tabular}
\end{center}
at the top of every child file \textit{child}
which is included by |\include{|\textit{child}|}|
from within the main file
(or at least for those files to be compiled individually).
The argument \textit{main} must be the filename of the main file.

There are a couple of
considerations in setting up the main and child documents:

%%%%%%%%%%%%%%%%%%%%%%%%%%%%%%%%%%%%%%%%
\paragraph{Restrictions.}

Please note the following restrictions:
\begin{itemize}
\item
|\childdocmain| must be called with one argument \textit{main}
to ensure compatibility with earlier version of the package.
It must either be empty (|\childdocmain{}|)
or precisely match the filename of the main file in which it is specified.
See \secref{sec:detection} for further information.
\item
The filename \textit{main} must be specified without the |.tex| extension.
\item
The filename \textit{main} is case sensitive
(even in case-insensitive file systems)
due to internal string comparison.
\item
The argument \textit{main} should be fully expanded, it cannot be a macro.
\item
Subdirectories and special characters should be avoided in filenames.
\item
The command |\childdocmain{|\textit{main}|}| must be followed by a whitespace.
It should not be followed immediately by another command
or by a comment mark `|%|'.
This is because the \TeX{} parser reads the token immediately following
the argument of |\childdocmain| and puts it
at the beginning of every child section;
however, a white\-space is ignored.
\end{itemize}

%%%%%%%%%%%%%%%%%%%%%%%%%%%%%%%%%%%%%%%%
\paragraph{Content of Main File.}

It is advisable to place all content in the child files included by |\include|.
Any output contained in the main file will appear in all child documents
unless suppressed manually;
it cannot be suppressed automatically by the |\includeonly| directive
and thus should normally be avoided.
A method to include some content in the main file
by means of conditional processing is described in \secref{sec:conditional}.

%%%%%%%%%%%%%%%%%%%%%%%%%%%%%%%%%%%%%%%%
\paragraph{Page Numbering.}

When only a part of the document is compiled,
the appropriate numbering of pages
(as well as other status parameters)
is determined from the |.aux| files.
The latter contain information from previous passes.
However this information needs to propagate through
all intermediate child documents.
Therefore the page numbering in child documents may well
be inconsistent until the complete document is compiled at least once.

A useful (if unconventional) way to always ensure a consistent
page numbering is to restart the numbering in each child document
and denote the pages by `\textit{child}|.|\textit{page}'
where \textit{child} represents the chapter/section number of the child file.
This can be achieved by the command
|\numberwithin{page}{|\textit{child}|}|
of the \textsf{amsmath} package
where \textit{child} can be |chapter| or |section|
depending on the chosen structuring.
Alternatively, one can modify the macro |\thepage| appropriately
and reset the counter |page| at the start of each child file.

%%%%%%%%%%%%%%%%%%%%%%%%%%%%%%%%%%%%%%%%%%%%%%%%%%%%%%%%%%%%%%%%%%%%%%%%%%%%%%%%
\subsection{Conditional Processing}
\label{sec:conditional}

The package provides a mechanism to compile different versions
of a document. To customise the versions further some conditional processing
can come in handy to distinguish which version is being compiled.
The package provides two macros to describe the compilation context:

%%%%%%%%%%%%%%%%%%%%%%%%%%%%%%%%%%%%%%%%
\DescribeMacro{\ifchilddoc}
The conditional |\ifchilddoc| distinguishes between the compilation of
child documents and the main document:
%
\begin{center}
|\ifchilddoc |\textit{child-code}| |[|\||else |\textit{main-code}]| \||fi|
\end{center}

%%%%%%%%%%%%%%%%%%%%%%%%%%%%%%%%%%%%%%%%
\DescribeMacro{\childdocname}
\DescribeMacro{\childdocjob}
The macro |\childdocname| contains the filename (without extension)
of the main or child file being processed.
Note that |\childdocjob| will always contain the name of the main file.

%%%%%%%%%%%%%%%%%%%%%%%%%%%%%%%%%%%%%%%%
\paragraph{Title Page.}

Conditional processing can be used to include a title or banner page
in the main document when proper precautions are taken.
Importantly, the code in the main file should ensure that the page counter
(as well as other status parameters which are stored in the |.aux| files)
takes the same value after the conditional processing.
Otherwise the page numbers may take divergent values
depending on which part is compiled.

For example, a title page could be declared by:
%
\begin{center}
\begin{tabular}{l}
|\ifchilddoc\||else|\\
|\addtocounter{page}{-1}|\\
\textit{code for title page}\\
|\newpage|\\
|\||fi|
\end{tabular}
\end{center}
%
A banner page for the child documents can be generated by:
%
\begin{center}
\begin{tabular}{l}
|\ifchilddoc|\\
|\addtocounter{page}{-1}|\\
\textit{code for banner page}\\
|\newpage|\\
|\||fi|
\end{tabular}
\end{center}
%
Here one could write a message such as:
\begin{center}
|This is the part \childdocname{} of \childdocjob{}.|
\end{center}

%%%%%%%%%%%%%%%%%%%%%%%%%%%%%%%%%%%%%%%%%%%%%%%%%%%%%%%%%%%%%%%%%%%%%%%%%%%%%%%%
\subsection{Flags}
\label{sec:flags}

The package makes it easy to generate different versions
of the main or child documents.
To this end compilation flags can be defined
and assigned different default values.
They will be particularly useful in conjunction
with the forwarding mechanism described in \secref{sec:forward}.

For example, it may be useful to have a flag |\version|
which can be set to |draft| or |final|.
The document source will contain some conditional code
depending on the value of |\version|.
Suppose further, the flag should default to |final| for the main file
and to |draft| for child files
which is a natural assignment for editing the document.
This is achieved by placing the following code
in the preamble of the main document
(below the |\childdocmain| directive):
%
\begin{center}
\begin{tabular}{l}
|\ifchilddoc|\\
|\providecommand{\version}{draft}|\\
|\||else|\\
|\providecommand{\version}{final}|\\
|\||fi|
\end{tabular}
\end{center}
%
The definition by |\providecommand| makes sure
that previous definitions are not overwritten.
Further statements |\providecommand{\version}{...}|
can thus be added before the above code to override it.

For the main file, one might add a line
(between |\childdocmain| and the above block)
%
\begin{center}
|%\ifchilddoc\||else\providecommand{\version}{draft}\||fi|
\end{center}
%
which can be uncommented to produce a draft version.
Likewise one can add a line to the very top of a child file
(above the |\childdocof{|\textit{main}|}| directive)
%
\begin{center}
|%\providecommand{\version}{final}|
\end{center}
%
which can be uncommented to produce the final version of this child document.

%%%%%%%%%%%%%%%%%%%%%%%%%%%%%%%%%%%%%%%%%%%%%%%%%%%%%%%%%%%%%%%%%%%%%%%%%%%%%%%%
\subsection{Forwarding}
\label{sec:forward}

Different versions of the main or child documents
using compilation flags as described in \secref{sec:flags}
can be (permanently) stored in different files
for convenient compilation, viewing and distribution.
To this end, the package defines a command
to pass on compilation to a different file:

%%%%%%%%%%%%%%%%%%%%%%%%%%%%%%%%%%%%%%%%
\DescribeMacro{\childdocforward}
The command |\childdocforward| redirects processing to
another source file:
%
\begin{center}
\begin{tabular}{l}
|\input{childdoc.def}|\\
|\childdocforward[|\textit{main}|]{|\textit{dest}|}|\\
\end{tabular}
\end{center}
%
The argument \textit{dest} is the destination file
(without extension).
It should be the main file or one of the child files.
Note that further \textsf{childdoc} directives
such as |\childdocof| and |\childdocforward|
in the indicated file will be processed in this form.
The optional argument \textit{main}
passes on directly to the main file \textit{main}
while pretending to compile the child \textit{dest}.
This form behaves as if \textit{dest}
issues |\childdocof{|\textit{main}|}| right away,
and no further \textsf{childdoc} directives will be processed.

%%%%%%%%%%%%%%%%%%%%%%%%%%%%%%%%%%%%%%%%
\DescribeMacro{\...prefix}
In the alternative form |\childdocforwardprefix|,
%
\begin{center}
\begin{tabular}{l}
|\input{childdoc.def}|\\
|\childdocforwardprefix[|\textit{main}|]{|\textit{prefix}|}{|\textit{dest}|}|
\end{tabular}
\end{center}
%
the destination file is determined by a pattern
depending on the current file:
To make this work, the current file must be called
`{\textit{prefix}\hspace{0.2em}\textit{suffix}}'
with \textit{prefix} matching precisely the argument.
Processing is then passed on to the file
`{\textit{dest}\hspace{0.2em}\textit{suffix}}'.
Surely, the same effect is achieved by
directly specifying the
argument `{\textit{dest}\hspace{0.2em}\textit{suffix}}'
in the first form.
However, that requires to set up a different file
for each child. With the alternative form of the command
all these files can have exactly the same content
which simplifies setting them up and maintaining them.

For example, the following file |draft.tex|
with a compilation flag |\version| as described in \secref{sec:flags}
compiles the main document as a draft:
%
\begin{center}
\begin{tabular}{l}
|\def\version{draft}|\\
|\input{childdoc.def}|\\
|\childdocforward{|\textit{main}|}|
\end{tabular}
\end{center}
%
Likewise, the following files |final|\textit{nn}|.tex|
compile the final version of the child document
|child|\textit{nn}|.tex|:
%
\begin{center}
\begin{tabular}{l}
|\def\version{final}|\\
|\input{childdoc.def}|\\
|\childdocforwardprefix{final}{child}|
\end{tabular}
\end{center}
%

Note that when several versions of a main file and/or of each child file
are to be generated, it may be convenient to set up a |Makefile| or
shell script to automatise the process.

%%%%%%%%%%%%%%%%%%%%%%%%%%%%%%%%%%%%%%%%%%%%%%%%%%%%%%%%%%%%%%%%%%%%%%%%%%%%%%%%
\subsection{Command Line Processing}
\label{sec:commandline}

The effect of redirection files can also be achieved by invoking
the \LaTeX{} compiler with a more elaborate command line.
Most conveniently this should be done as part
of a shell script or a |Makefile|.

When using \textsf{childdoc} in the main file, the following
command lines effectively perform a redirection
(note that depending on the shell being used,
backslashes may have to be doubled: `|\|' $\to$ `|\\|'):
%
\begin{center}
|... -jobname "|\textit{target}|" |\\|"|[\textit{flags}]%
|\input{childdoc.def}\childdocforward[|\textit{main}|]{|\textit{dest}|}"|
\end{center}
%
Here \textit{target} is the name of the output file,
\textit{main} is the name of the main file
and \textit{dest} is the name of the main or child file to be processed
(all filenames without extensions).
The optional argument \textit{main} can be omitted
if \textit{main} matches \textit{dest}.
Optionally, compilation \textit{flags} can be defined via |\def| commands.
This command line makes the \TeX{} engine believe
it is compiling the file \textit{target}
whose content is specified as the latter parameter.
The provided code then forwards the processing to
\textit{main} or \textit{dest} as described in \secref{sec:forward}.

%%%%%%%%%%%%%%%%%%%%%%%%%%%%%%%%%%%%%%%%%%%%%%%%%%%%%%%%%%%%%%%%%%%%%%%%%%%%%%%%
\subsection{Include by Input}
\label{sec:input}

Including child documents by |\include| has some restrictions by design.
Most notably, the content of a child document always occupies
its own set of pages; pages cannot be shared between child documents.
Usually, this behaviour makes perfect sense
because each child document contain an essential part of the document.
However, in some situations it may be desirable to compose
a document from a collection of parts
without having mandatory page breaks between then.
For this case, the package
provides a mechanism to include parts
by |\input| which can also be processed individually.
However, by construction this mechanism
requires manual handling of the content to be output.

%%%%%%%%%%%%%%%%%%%%%%%%%%%%%%%%%%%%%%%%
\DescribeMacro{\ifchilddocmanual}
The main file should be prepared as usual, see \secref{sec:include}.
However, the document body must make a distinction
between processing of an individual part and of the main document, e.g.:
%
\begin{center}
\begin{tabular}{l}
|\ifchilddocmanual|\\
|\input{\childdocname}|\\
|\||else|\\
\textit{document body with }|\input{|\textit{part}|}|\\
|\||fi|
\end{tabular}
\end{center}
%
The conditional |\ifchilddocmanual| is true whenever
a part to be included by |\input| is being compiled,
and the name of the part is stored in |\childdocname|.

%%%%%%%%%%%%%%%%%%%%%%%%%%%%%%%%%%%%%%%%
\DescribeMacro{\childdocby}
Each part to be included by |\input| should start with:
%
\begin{center}
\begin{tabular}{l}
|\input{childdoc.def}|\\
|\childdocby{|\textit{main}|}|\\
\end{tabular}
\end{center}
%
The directive |\childdocby| is similar to |\childdocof|
described in \secref{sec:include},
but the subsequent selection of content must be done manually.
To that end, both |\ifchilddoc| and |\ifchilddocmanual|
will be true upon processing of a part,
and the name of the part is stored in |\childdocname|.
Note that |\jobname| will be set to the filename of the current part
so that each part receives an individual |.aux| file
that does not interfere with the |.aux| file(s) of the main document.
This behaviour can be altered by the alternative form
|\childdocby[*]{|\textit{main}|}| (with a non-empty optional argument)
which uses the |.aux| file of the main document
by setting |\jobname| to \textit{main}.

%%%%%%%%%%%%%%%%%%%%%%%%%%%%%%%%%%%%%%%%%%%%%%%%%%%%%%%%%%%%%%%%%%%%%%%%%%%%%%%%
\subsection{Driver Development}
\label{sec:driver}

The \textsf{childdoc} mechanism can also be use for the development
of definition files such as \LaTeX{} styles or classes.
This case differs from the above setup with multiple parts
included by |\include| in that no |\includeonly| should be invoked.
This can be achieved by starting the include file
(before |\ProvidesPackage|) with:
%
\begin{center}
\begin{tabular}{l}
|\input{childdoc.def}|\\
|\childdocforward{|\textit{main}|}|\\
\end{tabular}
\end{center}
%
or alternatively with:
%
\begin{center}
\begin{tabular}{l}
|\input{childdoc.def}|\\
|\childdocby{|\textit{main}|}|\\
\end{tabular}
\end{center}
%
Both forms have slightly different effects as described above.
The main file is prepared as usual, see \secref{sec:include}.

%%%%%%%%%%%%%%%%%%%%%%%%%%%%%%%%%%%%%%%%%%%%%%%%%%%%%%%%%%%%%%%%%%%%%%%%%%%%%%%%
\subsection{Legacy Detection}
\label{sec:detection}

The directive |\childdocmain| in the main file can detect
whether the complete document or merely a child is to be compiled
even without using the directive |\childdocof|.
This method is deprecated because it is less robust
and there is no compelling reason to use it;
it is merely provided for backward compatibility
and it may be removed in future versions.

If the detection mechanism is to be used,
it is mandatory to correctly specify
the filename of the main file as the argument of |\childdocmain|:
%
\begin{center}
\begin{tabular}{l}
|\input{childdoc.def}|\\
|\childdocmain{|\textit{main}|}|\\
\end{tabular}
\end{center}
%
If |\jobname| does not match the argument \textit{main} of |\childdocmain|,
it is assumed that |\jobname| points to the child file to be compiled.
When using |\childdocmain| with the main file specified as argument,
it suffices to start a child file
with just |\input{|\textit{main}|}|
without loading of the package and using |\childdocof|.
If instead all processing is done
with the appropriate \textsf{childdoc} directives,
the argument of \textit{main} of |\childdocmain| can be empty.

An alternative version of the command line processing described
in \secref{sec:commandline} using the detection mechanism reads:
%
\begin{center}
|... -jobname "|\textit{target}|" "|[\textit{flags}]%
[|\def\jobname{|\textit{dest}|}|]|\input{|\textit{main}|}"|
\end{center}

%%%%%%%%%%%%%%%%%%%%%%%%%%%%%%%%%%%%%%%%%%%%%%%%%%%%%%%%%%%%%%%%%%%%%%%%%%%%%%%%
\subsection{Manual Code}
\label{sec:manual}

In case one cannot be certain whether the definitions file |childdoc.def|
is installed on the target \TeX{} distribution
and one prefers not to ship it,
it is conceivable to paste a few relevant commands into the sources.

To that end, drop all statements |\input{childdoc.def}|
and perform the replacements as outlined below.
Instead of |\childdocmain{|\textit{main}|}| add the following code
to the top of the main file:
%
\begin{center}
\begin{tabular}{l}
|\||ifdefined\childdocname\endinput\||fi\newif\ifchilddoc|\\
|\edef\childdocname{\scantokens\expandafter{\jobname\noexpand}}|\\
|\def\childdocmain{|\textit{main}|}\||ifx\childdocmain\childdocname\||else|\\
|\childdoctrue\includeonly{\childdocname}\let\jobname\childdocmain\||fi|\\
\end{tabular}
\end{center}
%
Instead of |\childdocof{|\textit{main}|}| just include the main file
at the top of each child file:
%
\begin{center}
|\input{|\textit{main}|}|
\end{center}
%
A simple redirection |\childdocforward{|\textit{dest}|}| is achieved by:
%
\begin{center}
|\def\jobname{|\textit{dest}|}\input{\jobname}|
\end{center}
%
The redirection with prefix
|\childdocforwardprefix[|\textit{prefix}|]{|\textit{dest}|}|
is accomplished by:
%
\begin{center}
\begin{tabular}{l}
|{\edef\jobname{\scantokens\expandafter{\jobname\noexpand}}|\\
|\def\redirectjob |\textit{prefix}|#1~~~{\gdef\jobname{|\textit{dest}|#1}}|\\
|\expandafter\redirectjob\jobname~~~}\input{\jobname}|
\end{tabular}
\end{center}

In an alternative approach,
child documents can be compiled by a specific command line
without additional code or specific definitions:
%
\begin{center}
|... -jobname "|\textit{target}|" "|[\textit{flags}]%
|\includeonly{|\textit{dest}|}\input{|\textit{main}|}"|
\end{center}
%

%%%%%%%%%%%%%%%%%%%%%%%%%%%%%%%%%%%%%%%%%%%%%%%%%%%%%%%%%%%%%%%%%%%%%%%%%%%%%%%%
%%%%%%%%%%%%%%%%%%%%%%%%%%%%%%%%%%%%%%%%%%%%%%%%%%%%%%%%%%%%%%%%%%%%%%%%%%%%%%%%
\section{Information}

%%%%%%%%%%%%%%%%%%%%%%%%%%%%%%%%%%%%%%%%%%%%%%%%%%%%%%%%%%%%%%%%%%%%%%%%%%%%%%%%
\subsection{Copyright}

Copyright \copyright{} 2017--2018 Niklas Beisert

This work may be distributed and/or modified under the
conditions of the \LaTeX{} Project Public License, either version 1.3
of this license or (at your option) any later version.
The latest version of this license is in
  \url{http://www.latex-project.org/lppl.txt}
and version 1.3 or later is part of all distributions of \LaTeX{}
version 2005/12/01 or later.

This work has the LPPL maintenance status `maintained'.

The Current Maintainer of this work is Niklas Beisert.

This work consists of the files |README.txt|, |childdoc.ins| and |childdoc.dtx|
as well as the derived files |childdoc.def|, |cdocsamp.tex|
with |cdocsch1.tex|, |cdocsch2.tex|, |cdocspt3.tex|, |cdocspt4.tex|,
|cdocsdrf.tex|, |cdocsfn1.tex|, |cdocsfn2.tex|
as well as |childdoc.pdf|.

%%%%%%%%%%%%%%%%%%%%%%%%%%%%%%%%%%%%%%%%%%%%%%%%%%%%%%%%%%%%%%%%%%%%%%%%%%%%%%%%
\subsection{Files and Installation}

The package consists of the files:
%
\begin{center}
\begin{tabular}{ll}
    |README.txt|   & readme file \\
    |childdoc.ins| & installation file \\
    |childdoc.dtx| & source file \\
    |childdoc.def| & definition file \\
    |cdocsamp.tex| & sample main file \\
    |cdocsch1.tex| & sample include file \\
    |cdocsch2.tex| & sample include file \\
    |cdocspt3.tex| & sample part file \\
    |cdocspt4.tex| & sample part file \\
    |cdocsdrf.tex| & sample redirection file \\
    |cdocsfn1.tex| & sample redirection file \\
    |cdocsfn2.tex| & sample redirection file \\
    |childdoc.pdf| & manual
\end{tabular}
\end{center}
%
The distribution consists of the files
|README.txt|, |childdoc.ins| and |childdoc.dtx|.
%
\begin{itemize}
\item
Run (pdf)\LaTeX{} on |childdoc.dtx|
to compile the manual |childdoc.pdf| (this file).
\item
Run \LaTeX{} on |childdoc.ins| to create the definitions file |childdoc.def|
and the sample |cdocsamp.tex| with include files
|cdocsch1.tex|, |cdocsch2.tex|, |cdocspt3.tex|, |cdocspt4.tex|,
|cdocsdrf.tex|, |cdocsfn1.tex|, |cdocsfn2.tex|.
Then copy the file |childdoc.def| to an appropriate directory of your \LaTeX{}
distribution, e.g.\ \textit{texmf-root}|/tex/latex/childdoc|.
\end{itemize}

%%%%%%%%%%%%%%%%%%%%%%%%%%%%%%%%%%%%%%%%%%%%%%%%%%%%%%%%%%%%%%%%%%%%%%%%%%%%%%%%
\subsection{Related CTAN Packages}

There are several other packages which offer a similar functionality:
%
\begin{itemize}
\item
The packages
\href{http://ctan.org/pkg/docmute}{\textsf{docmute}},
\href{http://ctan.org/pkg/includex}{\textsf{includex}} and
\href{http://ctan.org/pkg/standalone}{\textsf{standalone}}
provide commands to include only the document body of
a child file thus allowing both files to be compiled individually.
\item
The packages \href{http://ctan.org/pkg/subdocs}{\textsf{subdocs}}
and \href{http://ctan.org/pkg/subfiles}{\textsf{subfiles}}
provide structures in which the main and child documents can be
encapsulated and allowing them to be compiled individually.
The inclusion mechanism is different from the conventional |\include|.
\item
The package \href{http://ctan.org/pkg/combine}{\textsf{combine}}
is an elaborate solution to combine several documents into one.
\end{itemize}
%
See also the CTAN topic \href{http://ctan.org/topic/subdocs}{\textsf{subdocs}}
for further related packages.
The present package differs from the above solutions in that
a document structure constructed with the conventional |\include| mechanism
just needs two extra commands at the top of every file
such that all constituent files can be compiled individually.

%%%%%%%%%%%%%%%%%%%%%%%%%%%%%%%%%%%%%%%%%%%%%%%%%%%%%%%%%%%%%%%%%%%%%%%%%%%%%%%%
%\subsection{Feature Suggestions}
%
%The following is a list of features which may be useful for future
%versions of this package:
%%
%\begin{itemize}
%\item
%\ldots
%\end{itemize}

%%%%%%%%%%%%%%%%%%%%%%%%%%%%%%%%%%%%%%%%%%%%%%%%%%%%%%%%%%%%%%%%%%%%%%%%%%%%%%%%
\subsection{Revision History}

%%%%%%%%%%%%%%%%%%%%%%%%%%%%%%%%%%%%%%%%
\paragraph{v2.0:} 2018/12/30

\begin{itemize}
\item
immediate forward processing
\item
added |\childdocby| mechanism
\item
manual restructured
\end{itemize}

%%%%%%%%%%%%%%%%%%%%%%%%%%%%%%%%%%%%%%%%
\paragraph{v1.6:} 2018/01/17

\begin{itemize}
\item
application for development of include files
\item
corrections to manual
\end{itemize}

%%%%%%%%%%%%%%%%%%%%%%%%%%%%%%%%%%%%%%%%
\paragraph{v1.5:} 2017/05/21

\begin{itemize}
\item
more complete structuring introduced
\item
|\childdocof| introduced
\item
|\childdoc| renamed to |\childdocmain|
\item
|\childredirect| renamed to |\childdocforward| and |\childdocforwardprefix|
and functionality expanded
\end{itemize}

%%%%%%%%%%%%%%%%%%%%%%%%%%%%%%%%%%%%%%%%
\paragraph{v1.0:} 2017/04/27

\begin{itemize}
\item
manual and install package
\item
first version published on CTAN
\end{itemize}

%%%%%%%%%%%%%%%%%%%%%%%%%%%%%%%%%%%%%%%%
\paragraph{v0.6:} 2017/04/26

\begin{itemize}
\item
redirection mechanism added
\end{itemize}

%%%%%%%%%%%%%%%%%%%%%%%%%%%%%%%%%%%%%%%%
\paragraph{v0.5:} 2017/04/26

\begin{itemize}
\item
functionality in definition file
\end{itemize}


%%%%%%%%%%%%%%%%%%%%%%%%%%%%%%%%%%%%%%%%%%%%%%%%%%%%%%%%%%%%%%%%%%%%%%%%%%%%%%%%
%%%%%%%%%%%%%%%%%%%%%%%%%%%%%%%%%%%%%%%%%%%%%%%%%%%%%%%%%%%%%%%%%%%%%%%%%%%%%%%%
%%%%%%%%%%%%%%%%%%%%%%%%%%%%%%%%%%%%%%%%%%%%%%%%%%%%%%%%%%%%%%%%%%%%%%%%%%%%%%%%
\appendix

\settowidth\MacroIndent{\rmfamily\scriptsize 000\ }

 \DocInput{childdoc.dtx}

\end{document}
%</driver>
% \fi
%
% %%%%%%%%%%%%%%%%%%%%%%%%%%%%%%%%%%%%%%%%%%%%%%%%%%%%%%%%%%%%%%%%%%%%%%%%%%%%%%
% %%%%%%%%%%%%%%%%%%%%%%%%%%%%%%%%%%%%%%%%%%%%%%%%%%%%%%%%%%%%%%%%%%%%%%%%%%%%%%
% \section{Sample}
%\iffalse
%<*samplemain>
%\fi
%
% The following presents a sample document
% with two chapters, two parts, a title page,
% a compile flag as well as three forwarding files to set the flag.
% It consists of eight |.tex| files:
% \begin{center}
% \begin{tabular}{ll}
% |cdocsamp.tex|&main file\\
% |cdocsch1.tex|&include file for chapter 1\\
% |cdocsch2.tex|&include file for chapter 2\\
% |cdocspt3.tex|&include file for part 3\\
% |cdocspt4.tex|&include file for part 4\\
% |cdocsdrf.tex|&forwarding file for main file in draft mode\\
% |cdocsfi1.tex|&forwarding file for final version of chapter 1\\
% |cdocsfi2.tex|&forwarding file for final version of chapter 2\\
% \end{tabular}
% \end{center}
% Each of the eight files can be compiled directly by the \LaTeX{} compiler.
%
% %%%%%%%%%%%%%%%%%%%%%%%%%%%%%%%%%%%%%%
% \paragraph{Main File.}
%
% The main file is called |cdocsamp.tex|.
%
% Load the \textsf{childdoc} definitions and
% declare the filename for the main document:
%    \begin{macrocode}
\input{childdoc.def}
\childdocmain{}
%    \end{macrocode}

% Optional override for |\version| flag:
%    \begin{macrocode}
%%\ifchilddoc\else\providecommand{\version}{draft}\fi
%    \end{macrocode}

% Define the default values for the |\version| flag
% (|final| for the main file and |draft| for childs):
%    \begin{macrocode}
\ifchilddoc
\providecommand{\version}{draft}
\else
\providecommand{\version}{final}
\fi
%    \end{macrocode}

% Load the standard document class:
%    \begin{macrocode}
\documentclass[12pt]{article}
%    \end{macrocode}

% Start the document body:
%    \begin{macrocode}
\begin{document}
%    \end{macrocode}

% Declare a title page.
% Print title, part of document being processed and version flag:
%    \begin{macrocode}
\addtocounter{page}{-1}
\begin{center}
{\LARGE\bfseries{}childdoc example\par}
\vspace{1cm}
\ifchilddoc
\ifchilddocmanual part\else chapter\fi:
`\childdocname' of `\childdocjob'\par
\else
main document: `\childdocjob'\par
\fi
version: \version\par
\end{center}
\newpage
%    \end{macrocode}

% Manually include selected file,
% otherwise process as usual:
%    \begin{macrocode}
\ifchilddocmanual
\section*{part `\childdocname'}
\input{\childdocname}
\else
%    \end{macrocode}

% Include the two chapters:
%    \begin{macrocode}
\include{cdocsch1}
\include{cdocsch2}
%    \end{macrocode}

% Include the two parts unless only chapters should be displayed:
%    \begin{macrocode}
\ifchilddoc\else
\section{part three}
\input{cdocspt3}
\section{part four}
\input{cdocspt4}
\fi
%    \end{macrocode}

% Process as usual until here:
%    \begin{macrocode}
\fi
%    \end{macrocode}

% End of document body:
%    \begin{macrocode}
\end{document}
%    \end{macrocode}
%\iffalse
%</samplemain>
%\fi
%
% %%%%%%%%%%%%%%%%%%%%%%%%%%%%%%%%%%%%%%
% \paragraph{Chapter Include Files.}
%
% The include files are called |cdocsch1.tex| and |cdocsch2.tex|.
%
%\iffalse
%<*samplechap1|samplechap2>
%\fi

% Optional override for |\version| flag:
%    \begin{macrocode}
%%\providecommand{\version}{final}
%    \end{macrocode}

% Include the main document:
%    \begin{macrocode}
\input{childdoc.def}
\childdocof{cdocsamp}
%    \end{macrocode}

%\iffalse
%</samplechap1|samplechap2>
%\fi
%
%\iffalse
%<*samplechap1>
%\fi
% Some text for chapter 1:
%    \begin{macrocode}
\section{one}
some text in chapter one
%    \end{macrocode}

%\iffalse
%</samplechap1>
%\fi
% Some text for chapter 2:
%\iffalse
%<*samplechap2>
%\fi
%    \begin{macrocode}
\section{two}
more text in chapter two
%    \end{macrocode}

%\iffalse
%</samplechap2>
%\fi
%
% %%%%%%%%%%%%%%%%%%%%%%%%%%%%%%%%%%%%%%
% \paragraph{Part Include Files.}
%
% The include files are called |cdocspt3.tex| and |cdocspt4.tex|.
%
%\iffalse
%<*samplepart3|samplepart4>
%\fi

% Optional override for |\version| flag:
%    \begin{macrocode}
%%\providecommand{\version}{final}
%    \end{macrocode}

% Include the main document:
%    \begin{macrocode}
\input{childdoc.def}
\childdocby{cdocsamp}
%    \end{macrocode}

%\iffalse
%</samplepart3|samplepart4>
%\fi
%
%\iffalse
%<*samplepart3>
%\fi
% Some text for part 3:
%    \begin{macrocode}
some text in part three
%    \end{macrocode}

%\iffalse
%</samplepart3>
%\fi
% Some text for part 4:
%\iffalse
%<*samplepart4>
%\fi
%    \begin{macrocode}
more text in part four
%    \end{macrocode}

%\iffalse
%</samplepart4>
%\fi
%
% %%%%%%%%%%%%%%%%%%%%%%%%%%%%%%%%%%%%%%
% \paragraph{Forwarding for a Complete Draft.}
%
% The following forwarding file |cdocsdrf.tex|
% compiles the main document in draft mode:
%\iffalse
%<*sampledraft>
%\fi
%    \begin{macrocode}
\def\version{draft}
\input{childdoc.def}
\childdocforward{cdocsamp}
%    \end{macrocode}

%\iffalse
%</sampledraft>
%\fi
%
% %%%%%%%%%%%%%%%%%%%%%%%%%%%%%%%%%%%%%%
% \paragraph{Forwarding for Final Version of the Chapters.}
%
% The following forwarding files |cdocsfn1.tex| and |cdocsfn2.tex|
% (with identical content)
% compile the final versions of the child documents
% |cdocsch1.tex| and |cdocsch2.tex|, respectively:
%\iffalse
%<*samplefinal>
%\fi
%    \begin{macrocode}
\def\version{final}
\input{childdoc.def}
\childdocforwardprefix[cdocsamp]{cdocsfn}{cdocsch}
%    \end{macrocode}

%\iffalse
%</samplefinal>
%\fi
%
% %%%%%%%%%%%%%%%%%%%%%%%%%%%%%%%%%%%%%%
% \paragraph{Command Line Processing.}
%
% The following three command lines generate the output files
% |cdocscld|, |cdocscl1| and |cdocscl2|
% which should be identical to
% |cdocsdrf|, |cdocsch1| and |cdocsfn2|, respectively:
% \begin{center}
% \begin{tabular}{l}
% |latex -jobname cdocscld \|\\
% |  "\def\version{draft}\input{childdoc.def}\childdocforward{cdocsamp}"|\\
% |latex -jobname cdocscl1 \|\\
% |  "\input{childdoc.def}\childdocforward[cdocsamp]{cdocsch1}"|\\
% |latex -jobname cdocscl2 \|\\
% |  "\def\version{final}\input{childdoc.def}\childdocforward{cdocsch2}"|
% \end{tabular}
% \end{center}
% Note that the trailing backslash on each first line
% merely continues the input to the second line
% (for convenient cut ant paste).
% Furthermore, the command |latex| can be replaced by any
% of its alternative versions such as |pdflatex|.
%
% %%%%%%%%%%%%%%%%%%%%%%%%%%%%%%%%%%%%%%%%%%%%%%%%%%%%%%%%%%%%%%%%%%%%%%%%%%%%%%
% %%%%%%%%%%%%%%%%%%%%%%%%%%%%%%%%%%%%%%%%%%%%%%%%%%%%%%%%%%%%%%%%%%%%%%%%%%%%%%
% \section{Implementation}
%\iffalse
%<*package>
%\fi
%
% This section describes the definitions file |childdoc.def|.

% The definitions cannot be loaded using |\usepackage| or |\RequirePackage|
% which has a mechanism to prevent loading a style file more than once.
% When loading the definitions by means of |\input|
% multiple instances have to be prevented manually:
%\iffalse
%This code needs to be before the `\ProvidesFile' directive
%which is defined at the beginning of this file.
%Therefore it is also placed there and commented out here.
%</package>
%<*discard>
%\fi
%    \begin{macrocode}
\ifdefined\childdocmain\endinput\fi
%    \end{macrocode}
%\iffalse
%</discard>
%<*package>
%\fi
%
% \macro{\ifchilddoc}
% \macro{\ifchilddocmanual}
% The conditional |\ifchilddoc| tells whether a
% child (true) or main (false) document is being compiled.
% The conditional |\ifchilddocmanual| tells whether
% the |\includeonly| mechanism is used (false) or
% the selection of child files must be performed manually (true).
% The definitions initialise to false:
%    \begin{macrocode}
\newif\ifchilddoc
\newif\ifchilddocmanual
%    \end{macrocode}

% \macro{\childdocname}
% \macro{\childdocjob}
% The macro |\childdocname| stores the name of the main document
% to be compiled. The macro |\childdocjob| stores the name of
% the document on which the \LaTeX{} compiler was originally invoked.
% The content of |\jobname| cannot be compared
% to filenames specified in the source due to different catcodes.
% The following code rescans |\jobname|, stores the result
% in |\childdocname| and saves a copy in |\childdocjob|:
%    \begin{macrocode}
\edef\childdocname{\scantokens\expandafter{\jobname\noexpand}}
\let\childdocjob\childdocname
%    \end{macrocode}

% \macro{\childdocdisable}
% The macro |\childdocdisable| prevents the main file
% from being processed more than once.
% At this stage, the main document command |\childdocmain|
% is assumed to be called once again where it should do nothing.
% Any subsequent call to it should prevent
% a secondary processing of the main document
% It overwrites the forwarding commands
% |\childdocof| and |\childdocforward|
% with empty macros to prevent further inclusions of the main document:
%    \begin{macrocode}
\newcommand{\childdocdisable}
{
  \renewcommand{\childdocmain}[1]{\renewcommand{\childdocmain}[1]{\endinput}}
  \renewcommand{\childdocof}[1]{}
  \renewcommand{\childdocby}[2][]{}
  \renewcommand{\childdocforward}[2][]{}
  \renewcommand{\childdocdisable}{}
}
%    \end{macrocode}

% \macro{\childdocmain}
% The macro |\childdocmain| is to be called at the top of the main file
% with nothing or the main filename (without extension) as argument.
% First, it breaks loops.
% If the argument is not empty and does not match |\childdocname|
% (which is set by the first inclusion of |childdoc.def|),
% |\ifchilddoc| is set to true, |\includeonly| is applied to the child file
% and |\jobname| is set to the main file
% (for proper handling of |.aux| files):
%    \begin{macrocode}
\newcommand{\childdocmain}[1]
{
  \childdocdisable\childdocmain{}
  \if?#1?\else
    \begingroup
      \def\childdoctmp{#1}
      \ifx\childdoctmp\childdocname
        \def\childdoctmp{}
      \else
        \def\childdoctmp
        {
          \childdoctrue
          \includeonly{\childdocname}
          \def\childdocjob{#1}
          \def\jobname{#1}
        }
      \fi
      \expandafter
    \endgroup
    \childdoctmp
  \fi
}
%    \end{macrocode}

% \macro{\childdocof}
% The command |\childdocof| redirects
% compilation to the main file |#1|.
%    \begin{macrocode}
\newcommand{\childdocof}[1]
{
  \childdocdisable
  \childdoctrue
  \includeonly{\childdocname}
  \def\jobname{#1}
  \def\childdocjob{#1}
  \input{#1}
}
%    \end{macrocode}

% \macro{\childdocby}
% The command |\childdocby| ....
%    \begin{macrocode}
\newcommand{\childdocby}[2][]
{
  \childdocdisable
  \childdoctrue
  \childdocmanualtrue
  \if?#1?\else
    \def\jobname{#2}
  \fi
  \def\childdocjob{#2}
  \input{#2}
  \endinput
}
%    \end{macrocode}

% \macro{\childdocforward}
% The command |\childdocforward| redirects
% compilation to the main file or
% (if the optional argument is given) a child file.
% Parameters are set as if the main file
% or a child file starting with |\childdocof| was compiled.
% Then compilation is handed over to the main file:
%    \begin{macrocode}
\newcommand{\childdocforward}[2][]
{
  \begingroup
    \if?#1?
      \def\childdoctmp
      {
        \def\childdocname{#2}
        \def\childdocjob{#2}
        \def\jobname{#2}
        \input{#2}
        \endinput
      }
    \else
      \def\childdoctmp
      {
        \childdocdisable
        \def\childdocname{#2}
        \childdoctrue
        \includeonly{#2}
        \def\childdocjob{#1}
        \def\jobname{#1}
        \input{#1}
        \endinput
      }
    \fi
    \expandafter
  \endgroup
  \childdoctmp
}
%    \end{macrocode}

% \macro{\childdocforwardprefix}
% The command |\childdocforwardprefix| redirects
% compilation to the main or a child file by means of a pattern.
% The prefix |#1| in the current filename is replaced by |#2|
% and the suffix of the current filename is kept
% (it is assumed that the filename does not contain the substring `|~~~|'
% which is used as a delimiter).
% Compilation is handed over to the new file by |\childdocforward|:
%    \begin{macrocode}
\newcommand{\childdocforwardprefix}[3][]
{
  \begingroup
    \def\childdocextract #2##1~~~{\def\childdoctmp{\childdocforward[#1]{#3##1}}}
    \expandafter\childdocextract\childdocname~~~
    \expandafter
  \endgroup
  \childdoctmp
}
%    \end{macrocode}

% \macro{\childdoc}
% The deprecated macro |\childdoc| is a legacy version of |\childdocmain|:
%    \begin{macrocode}
\newcommand{\childdoc}{\childdocmain}
%    \end{macrocode}

% \macro{\childdocredirect}
% The deprecated macro |\childdocredirect| is a legacy version
% of |\childdocforward| and |\childdocforwardprefix|:
%    \begin{macrocode}
\newcommand{\childdocredirect}[2][]
{
  \begingroup
    \if?#1?
      \def\childdoctmp{\childdocforward{#2}}
    \else
      \def\childdoctmp{\childdocforwardprefix{#1}{#2}}
    \fi
    \expandafter
  \endgroup
  \childdoctmp
}
%    \end{macrocode}

%\iffalse
%</package>
%\fi
%
\endinput
|\\
|\childdocforwardprefix{final}{child}|
\end{tabular}
\end{center}
%

Note that when several versions of a main file and/or of each child file
are to be generated, it may be convenient to set up a |Makefile| or
shell script to automatise the process.

%%%%%%%%%%%%%%%%%%%%%%%%%%%%%%%%%%%%%%%%%%%%%%%%%%%%%%%%%%%%%%%%%%%%%%%%%%%%%%%%
\subsection{Command Line Processing}
\label{sec:commandline}

The effect of redirection files can also be achieved by invoking
the \LaTeX{} compiler with a more elaborate command line.
Most conveniently this should be done as part
of a shell script or a |Makefile|.

When using \textsf{childdoc} in the main file, the following
command lines effectively perform a redirection
(note that depending on the shell being used,
backslashes may have to be doubled: `|\|' $\to$ `|\\|'):
%
\begin{center}
|... -jobname "|\textit{target}|" |\\|"|[\textit{flags}]%
|% \iffalse
%
% childdoc.dtx Copyright (C) 2017-2018 Niklas Beisert
%
% This work may be distributed and/or modified under the
% conditions of the LaTeX Project Public License, either version 1.3
% of this license or (at your option) any later version.
% The latest version of this license is in
%   http://www.latex-project.org/lppl.txt
% and version 1.3 or later is part of all distributions of LaTeX
% version 2005/12/01 or later.
%
% This work has the LPPL maintenance status `maintained'.
%
% The Current Maintainer of this work is Niklas Beisert.
%
% This work consists of the files childdoc.dtx and childdoc.ins
% and the derived files childdoc.def and cdocsamp.tex with
% cdocsch1.tex, cdocsch2.tex, cdocsdrf.tex, cdocsfn1.tex, cdocsfn2.tex.
%
%<package>\ifdefined\childdocmain\endinput\fi
%<package>\ProvidesFile{childdoc.def}[2018/12/30 v2.0 child document driver]
%<samplemain>\ProvidesFile{cdocsamp.tex}[2018/12/30 v2.0 sample for childdoc]
%<*driver>
%\ProvidesFile{childdoc.drv}[2018/12/30 v2.0 childdoc reference manual file]
\PassOptionsToClass{10pt,a4paper}{article}
\documentclass{ltxdoc}

\usepackage[margin=35mm]{geometry}
\usepackage{hyperref}
\usepackage{hyperxmp}
\usepackage[usenames]{color}

\hypersetup{colorlinks=true}
\hypersetup{pdfstartview=FitH}
\hypersetup{pdfpagemode=UseNone}
\hypersetup{pdfsource={}}
\hypersetup{pdflang={en-UK}}
\hypersetup{pdfcopyright={Copyright 2017-2018 Niklas Beisert.
  This work may be distributed and/or modified under the
  conditions of the LaTeX Project Public License, either version 1.3
  of this license or (at your option) any later version.}}
\hypersetup{pdflicenseurl={http://www.latex-project.org/lppl.txt}}
\hypersetup{pdfcontactaddress={ETH Zurich, ITP, HIT K,
  Wolfgang-Pauli-Strasse 27}}
\hypersetup{pdfcontactpostcode={8093}}
\hypersetup{pdfcontactcity={Zurich}}
\hypersetup{pdfcontactcountry={Switzerland}}
\hypersetup{pdfcontactemail={nbeisert@itp.phys.ethz.ch}}
\hypersetup{pdfcontacturl={http://people.phys.ethz.ch/\xmptilde nbeisert/}}

\newcommand{\secref}[1]{\hyperref[#1]{section \ref*{#1}}}

\parskip1ex
\parindent0pt
\let\olditemize\itemize
\def\itemize{\olditemize\parskip0pt}

\begin{document}

\title{The \textsf{childdoc} Package}
\hypersetup{pdftitle={The childdoc Package}}
\author{Niklas Beisert\\[2ex]
  Institut f\"ur Theoretische Physik\\
  Eidgen\"ossische Technische Hochschule Z\"urich\\
  Wolfgang-Pauli-Strasse 27, 8093 Z\"urich, Switzerland\\[1ex]
  \href{mailto:nbeisert@itp.phys.ethz.ch}
  {\texttt{nbeisert@itp.phys.ethz.ch}}}
\hypersetup{pdfauthor={Niklas Beisert}}
\hypersetup{pdfsubject={Manual for the LaTeX2e Package childdoc}}
\date{30 December 2018, \textsf{v2.0}}
\maketitle

\begin{abstract}\noindent
\textsf{childdoc} is a \LaTeXe{} package
that enables the direct compilation
of document sections included by |\include|
to individual files.
\end{abstract}

\begingroup
\parskip0ex
\tableofcontents
\endgroup

%%%%%%%%%%%%%%%%%%%%%%%%%%%%%%%%%%%%%%%%%%%%%%%%%%%%%%%%%%%%%%%%%%%%%%%%%%%%%%%%
%%%%%%%%%%%%%%%%%%%%%%%%%%%%%%%%%%%%%%%%%%%%%%%%%%%%%%%%%%%%%%%%%%%%%%%%%%%%%%%%
\section{Introduction}

\LaTeX{} provides a mechanism to structure a large document (such as a book)
into a main file and several child files (containing the chapters)
using the |\include| command.
This mechanism is beneficial for documents
which span hundreds of pages in order to
make the source file(s) more manageable.
Moreover, compilation can be restricted to
selected child files by means of the |\includeonly| command.
The latter feature can be used to reduce the compilation time while editing
(this was significantly more useful in the earlier days of \LaTeX{})
or to generate a smaller document which is easier to navigate.
Another application of |\includeonly| is to generate
documents consisting of selected parts of the complete document.

However, there are a few drawbacks of the plain |\include| mechanism:
\begin{itemize}
\item
The child files cannot be compiled on their own,
they can only be compiled via the main file.
A naive editing environment
(such as a text editor with an option
to have the current file processed by \LaTeX)
may require one to switch to the main file before compiling;
attempting to compile the child file produces errors.
\item
The main file must be modified (each time)
to adjust the |\includeonly| command
to the present needs. This easily leaves the main file in a messy state.
\item
The generated document will always carry the filename
of the main document. This is inconvenient if
several child files are to be compiled and
to be kept for distribution.
\end{itemize}

The present package provides a simple interface
to make child files individually compilable by \LaTeX{}.
Compiling a child file then has the same effect as compiling
the main file with an |\includeonly| command
to select the appropriate child.
Moreover the generated document will carry the name of the child
rather than the main file.
This resolves all three above issues.

This feature is meant to make the editing of books,
thesis documents and lecture notes somewhat more convenient.
However, the package can also be used efficiently for
composing a series of documents (such as exercise sheets)
which are typically distributed individually.
It then assists the author in generating the individual documents
(potentially in different versions)
as well as a document containing the collected series.
Another application is in developing style files
or other kinds of included material
where compilation of the style file could redirect
to a sample or test file.

%%%%%%%%%%%%%%%%%%%%%%%%%%%%%%%%%%%%%%%%%%%%%%%%%%%%%%%%%%%%%%%%%%%%%%%%%%%%%%%%
%%%%%%%%%%%%%%%%%%%%%%%%%%%%%%%%%%%%%%%%%%%%%%%%%%%%%%%%%%%%%%%%%%%%%%%%%%%%%%%%
\section{Usage}

First of all, the package \textsf{childdoc} is \emph{not} a standard
\LaTeXe{} |.sty| style file! Therefore it needs to be invoked in
a non-standard way.

%%%%%%%%%%%%%%%%%%%%%%%%%%%%%%%%%%%%%%%%%%%%%%%%%%%%%%%%%%%%%%%%%%%%%%%%%%%%%%%%
\subsection{Included Files}
\label{sec:include}

%%%%%%%%%%%%%%%%%%%%%%%%%%%%%%%%%%%%%%%%
\DescribeMacro{\childdocmain}
To use the package, add the commands
\begin{center}
\begin{tabular}{l}
|\input{childdoc.def}|\\
|\childdocmain{}|\\
\end{tabular}
\end{center}
at the very top of the main \LaTeX{} file,
in particular \emph{before} the |\documentclass| statement!
The argument of |\childdocmain| should be left empty
(but it must be present).

%%%%%%%%%%%%%%%%%%%%%%%%%%%%%%%%%%%%%%%%
\DescribeMacro{\childdocof}
Furthermore, add the commands
\begin{center}
\begin{tabular}{l}
|\input{childdoc.def}|\\
|\childdocof{|\textit{main}|}|\\
\end{tabular}
\end{center}
at the top of every child file \textit{child}
which is included by |\include{|\textit{child}|}|
from within the main file
(or at least for those files to be compiled individually).
The argument \textit{main} must be the filename of the main file.

There are a couple of
considerations in setting up the main and child documents:

%%%%%%%%%%%%%%%%%%%%%%%%%%%%%%%%%%%%%%%%
\paragraph{Restrictions.}

Please note the following restrictions:
\begin{itemize}
\item
|\childdocmain| must be called with one argument \textit{main}
to ensure compatibility with earlier version of the package.
It must either be empty (|\childdocmain{}|)
or precisely match the filename of the main file in which it is specified.
See \secref{sec:detection} for further information.
\item
The filename \textit{main} must be specified without the |.tex| extension.
\item
The filename \textit{main} is case sensitive
(even in case-insensitive file systems)
due to internal string comparison.
\item
The argument \textit{main} should be fully expanded, it cannot be a macro.
\item
Subdirectories and special characters should be avoided in filenames.
\item
The command |\childdocmain{|\textit{main}|}| must be followed by a whitespace.
It should not be followed immediately by another command
or by a comment mark `|%|'.
This is because the \TeX{} parser reads the token immediately following
the argument of |\childdocmain| and puts it
at the beginning of every child section;
however, a white\-space is ignored.
\end{itemize}

%%%%%%%%%%%%%%%%%%%%%%%%%%%%%%%%%%%%%%%%
\paragraph{Content of Main File.}

It is advisable to place all content in the child files included by |\include|.
Any output contained in the main file will appear in all child documents
unless suppressed manually;
it cannot be suppressed automatically by the |\includeonly| directive
and thus should normally be avoided.
A method to include some content in the main file
by means of conditional processing is described in \secref{sec:conditional}.

%%%%%%%%%%%%%%%%%%%%%%%%%%%%%%%%%%%%%%%%
\paragraph{Page Numbering.}

When only a part of the document is compiled,
the appropriate numbering of pages
(as well as other status parameters)
is determined from the |.aux| files.
The latter contain information from previous passes.
However this information needs to propagate through
all intermediate child documents.
Therefore the page numbering in child documents may well
be inconsistent until the complete document is compiled at least once.

A useful (if unconventional) way to always ensure a consistent
page numbering is to restart the numbering in each child document
and denote the pages by `\textit{child}|.|\textit{page}'
where \textit{child} represents the chapter/section number of the child file.
This can be achieved by the command
|\numberwithin{page}{|\textit{child}|}|
of the \textsf{amsmath} package
where \textit{child} can be |chapter| or |section|
depending on the chosen structuring.
Alternatively, one can modify the macro |\thepage| appropriately
and reset the counter |page| at the start of each child file.

%%%%%%%%%%%%%%%%%%%%%%%%%%%%%%%%%%%%%%%%%%%%%%%%%%%%%%%%%%%%%%%%%%%%%%%%%%%%%%%%
\subsection{Conditional Processing}
\label{sec:conditional}

The package provides a mechanism to compile different versions
of a document. To customise the versions further some conditional processing
can come in handy to distinguish which version is being compiled.
The package provides two macros to describe the compilation context:

%%%%%%%%%%%%%%%%%%%%%%%%%%%%%%%%%%%%%%%%
\DescribeMacro{\ifchilddoc}
The conditional |\ifchilddoc| distinguishes between the compilation of
child documents and the main document:
%
\begin{center}
|\ifchilddoc |\textit{child-code}| |[|\||else |\textit{main-code}]| \||fi|
\end{center}

%%%%%%%%%%%%%%%%%%%%%%%%%%%%%%%%%%%%%%%%
\DescribeMacro{\childdocname}
\DescribeMacro{\childdocjob}
The macro |\childdocname| contains the filename (without extension)
of the main or child file being processed.
Note that |\childdocjob| will always contain the name of the main file.

%%%%%%%%%%%%%%%%%%%%%%%%%%%%%%%%%%%%%%%%
\paragraph{Title Page.}

Conditional processing can be used to include a title or banner page
in the main document when proper precautions are taken.
Importantly, the code in the main file should ensure that the page counter
(as well as other status parameters which are stored in the |.aux| files)
takes the same value after the conditional processing.
Otherwise the page numbers may take divergent values
depending on which part is compiled.

For example, a title page could be declared by:
%
\begin{center}
\begin{tabular}{l}
|\ifchilddoc\||else|\\
|\addtocounter{page}{-1}|\\
\textit{code for title page}\\
|\newpage|\\
|\||fi|
\end{tabular}
\end{center}
%
A banner page for the child documents can be generated by:
%
\begin{center}
\begin{tabular}{l}
|\ifchilddoc|\\
|\addtocounter{page}{-1}|\\
\textit{code for banner page}\\
|\newpage|\\
|\||fi|
\end{tabular}
\end{center}
%
Here one could write a message such as:
\begin{center}
|This is the part \childdocname{} of \childdocjob{}.|
\end{center}

%%%%%%%%%%%%%%%%%%%%%%%%%%%%%%%%%%%%%%%%%%%%%%%%%%%%%%%%%%%%%%%%%%%%%%%%%%%%%%%%
\subsection{Flags}
\label{sec:flags}

The package makes it easy to generate different versions
of the main or child documents.
To this end compilation flags can be defined
and assigned different default values.
They will be particularly useful in conjunction
with the forwarding mechanism described in \secref{sec:forward}.

For example, it may be useful to have a flag |\version|
which can be set to |draft| or |final|.
The document source will contain some conditional code
depending on the value of |\version|.
Suppose further, the flag should default to |final| for the main file
and to |draft| for child files
which is a natural assignment for editing the document.
This is achieved by placing the following code
in the preamble of the main document
(below the |\childdocmain| directive):
%
\begin{center}
\begin{tabular}{l}
|\ifchilddoc|\\
|\providecommand{\version}{draft}|\\
|\||else|\\
|\providecommand{\version}{final}|\\
|\||fi|
\end{tabular}
\end{center}
%
The definition by |\providecommand| makes sure
that previous definitions are not overwritten.
Further statements |\providecommand{\version}{...}|
can thus be added before the above code to override it.

For the main file, one might add a line
(between |\childdocmain| and the above block)
%
\begin{center}
|%\ifchilddoc\||else\providecommand{\version}{draft}\||fi|
\end{center}
%
which can be uncommented to produce a draft version.
Likewise one can add a line to the very top of a child file
(above the |\childdocof{|\textit{main}|}| directive)
%
\begin{center}
|%\providecommand{\version}{final}|
\end{center}
%
which can be uncommented to produce the final version of this child document.

%%%%%%%%%%%%%%%%%%%%%%%%%%%%%%%%%%%%%%%%%%%%%%%%%%%%%%%%%%%%%%%%%%%%%%%%%%%%%%%%
\subsection{Forwarding}
\label{sec:forward}

Different versions of the main or child documents
using compilation flags as described in \secref{sec:flags}
can be (permanently) stored in different files
for convenient compilation, viewing and distribution.
To this end, the package defines a command
to pass on compilation to a different file:

%%%%%%%%%%%%%%%%%%%%%%%%%%%%%%%%%%%%%%%%
\DescribeMacro{\childdocforward}
The command |\childdocforward| redirects processing to
another source file:
%
\begin{center}
\begin{tabular}{l}
|\input{childdoc.def}|\\
|\childdocforward[|\textit{main}|]{|\textit{dest}|}|\\
\end{tabular}
\end{center}
%
The argument \textit{dest} is the destination file
(without extension).
It should be the main file or one of the child files.
Note that further \textsf{childdoc} directives
such as |\childdocof| and |\childdocforward|
in the indicated file will be processed in this form.
The optional argument \textit{main}
passes on directly to the main file \textit{main}
while pretending to compile the child \textit{dest}.
This form behaves as if \textit{dest}
issues |\childdocof{|\textit{main}|}| right away,
and no further \textsf{childdoc} directives will be processed.

%%%%%%%%%%%%%%%%%%%%%%%%%%%%%%%%%%%%%%%%
\DescribeMacro{\...prefix}
In the alternative form |\childdocforwardprefix|,
%
\begin{center}
\begin{tabular}{l}
|\input{childdoc.def}|\\
|\childdocforwardprefix[|\textit{main}|]{|\textit{prefix}|}{|\textit{dest}|}|
\end{tabular}
\end{center}
%
the destination file is determined by a pattern
depending on the current file:
To make this work, the current file must be called
`{\textit{prefix}\hspace{0.2em}\textit{suffix}}'
with \textit{prefix} matching precisely the argument.
Processing is then passed on to the file
`{\textit{dest}\hspace{0.2em}\textit{suffix}}'.
Surely, the same effect is achieved by
directly specifying the
argument `{\textit{dest}\hspace{0.2em}\textit{suffix}}'
in the first form.
However, that requires to set up a different file
for each child. With the alternative form of the command
all these files can have exactly the same content
which simplifies setting them up and maintaining them.

For example, the following file |draft.tex|
with a compilation flag |\version| as described in \secref{sec:flags}
compiles the main document as a draft:
%
\begin{center}
\begin{tabular}{l}
|\def\version{draft}|\\
|\input{childdoc.def}|\\
|\childdocforward{|\textit{main}|}|
\end{tabular}
\end{center}
%
Likewise, the following files |final|\textit{nn}|.tex|
compile the final version of the child document
|child|\textit{nn}|.tex|:
%
\begin{center}
\begin{tabular}{l}
|\def\version{final}|\\
|\input{childdoc.def}|\\
|\childdocforwardprefix{final}{child}|
\end{tabular}
\end{center}
%

Note that when several versions of a main file and/or of each child file
are to be generated, it may be convenient to set up a |Makefile| or
shell script to automatise the process.

%%%%%%%%%%%%%%%%%%%%%%%%%%%%%%%%%%%%%%%%%%%%%%%%%%%%%%%%%%%%%%%%%%%%%%%%%%%%%%%%
\subsection{Command Line Processing}
\label{sec:commandline}

The effect of redirection files can also be achieved by invoking
the \LaTeX{} compiler with a more elaborate command line.
Most conveniently this should be done as part
of a shell script or a |Makefile|.

When using \textsf{childdoc} in the main file, the following
command lines effectively perform a redirection
(note that depending on the shell being used,
backslashes may have to be doubled: `|\|' $\to$ `|\\|'):
%
\begin{center}
|... -jobname "|\textit{target}|" |\\|"|[\textit{flags}]%
|\input{childdoc.def}\childdocforward[|\textit{main}|]{|\textit{dest}|}"|
\end{center}
%
Here \textit{target} is the name of the output file,
\textit{main} is the name of the main file
and \textit{dest} is the name of the main or child file to be processed
(all filenames without extensions).
The optional argument \textit{main} can be omitted
if \textit{main} matches \textit{dest}.
Optionally, compilation \textit{flags} can be defined via |\def| commands.
This command line makes the \TeX{} engine believe
it is compiling the file \textit{target}
whose content is specified as the latter parameter.
The provided code then forwards the processing to
\textit{main} or \textit{dest} as described in \secref{sec:forward}.

%%%%%%%%%%%%%%%%%%%%%%%%%%%%%%%%%%%%%%%%%%%%%%%%%%%%%%%%%%%%%%%%%%%%%%%%%%%%%%%%
\subsection{Include by Input}
\label{sec:input}

Including child documents by |\include| has some restrictions by design.
Most notably, the content of a child document always occupies
its own set of pages; pages cannot be shared between child documents.
Usually, this behaviour makes perfect sense
because each child document contain an essential part of the document.
However, in some situations it may be desirable to compose
a document from a collection of parts
without having mandatory page breaks between then.
For this case, the package
provides a mechanism to include parts
by |\input| which can also be processed individually.
However, by construction this mechanism
requires manual handling of the content to be output.

%%%%%%%%%%%%%%%%%%%%%%%%%%%%%%%%%%%%%%%%
\DescribeMacro{\ifchilddocmanual}
The main file should be prepared as usual, see \secref{sec:include}.
However, the document body must make a distinction
between processing of an individual part and of the main document, e.g.:
%
\begin{center}
\begin{tabular}{l}
|\ifchilddocmanual|\\
|\input{\childdocname}|\\
|\||else|\\
\textit{document body with }|\input{|\textit{part}|}|\\
|\||fi|
\end{tabular}
\end{center}
%
The conditional |\ifchilddocmanual| is true whenever
a part to be included by |\input| is being compiled,
and the name of the part is stored in |\childdocname|.

%%%%%%%%%%%%%%%%%%%%%%%%%%%%%%%%%%%%%%%%
\DescribeMacro{\childdocby}
Each part to be included by |\input| should start with:
%
\begin{center}
\begin{tabular}{l}
|\input{childdoc.def}|\\
|\childdocby{|\textit{main}|}|\\
\end{tabular}
\end{center}
%
The directive |\childdocby| is similar to |\childdocof|
described in \secref{sec:include},
but the subsequent selection of content must be done manually.
To that end, both |\ifchilddoc| and |\ifchilddocmanual|
will be true upon processing of a part,
and the name of the part is stored in |\childdocname|.
Note that |\jobname| will be set to the filename of the current part
so that each part receives an individual |.aux| file
that does not interfere with the |.aux| file(s) of the main document.
This behaviour can be altered by the alternative form
|\childdocby[*]{|\textit{main}|}| (with a non-empty optional argument)
which uses the |.aux| file of the main document
by setting |\jobname| to \textit{main}.

%%%%%%%%%%%%%%%%%%%%%%%%%%%%%%%%%%%%%%%%%%%%%%%%%%%%%%%%%%%%%%%%%%%%%%%%%%%%%%%%
\subsection{Driver Development}
\label{sec:driver}

The \textsf{childdoc} mechanism can also be use for the development
of definition files such as \LaTeX{} styles or classes.
This case differs from the above setup with multiple parts
included by |\include| in that no |\includeonly| should be invoked.
This can be achieved by starting the include file
(before |\ProvidesPackage|) with:
%
\begin{center}
\begin{tabular}{l}
|\input{childdoc.def}|\\
|\childdocforward{|\textit{main}|}|\\
\end{tabular}
\end{center}
%
or alternatively with:
%
\begin{center}
\begin{tabular}{l}
|\input{childdoc.def}|\\
|\childdocby{|\textit{main}|}|\\
\end{tabular}
\end{center}
%
Both forms have slightly different effects as described above.
The main file is prepared as usual, see \secref{sec:include}.

%%%%%%%%%%%%%%%%%%%%%%%%%%%%%%%%%%%%%%%%%%%%%%%%%%%%%%%%%%%%%%%%%%%%%%%%%%%%%%%%
\subsection{Legacy Detection}
\label{sec:detection}

The directive |\childdocmain| in the main file can detect
whether the complete document or merely a child is to be compiled
even without using the directive |\childdocof|.
This method is deprecated because it is less robust
and there is no compelling reason to use it;
it is merely provided for backward compatibility
and it may be removed in future versions.

If the detection mechanism is to be used,
it is mandatory to correctly specify
the filename of the main file as the argument of |\childdocmain|:
%
\begin{center}
\begin{tabular}{l}
|\input{childdoc.def}|\\
|\childdocmain{|\textit{main}|}|\\
\end{tabular}
\end{center}
%
If |\jobname| does not match the argument \textit{main} of |\childdocmain|,
it is assumed that |\jobname| points to the child file to be compiled.
When using |\childdocmain| with the main file specified as argument,
it suffices to start a child file
with just |\input{|\textit{main}|}|
without loading of the package and using |\childdocof|.
If instead all processing is done
with the appropriate \textsf{childdoc} directives,
the argument of \textit{main} of |\childdocmain| can be empty.

An alternative version of the command line processing described
in \secref{sec:commandline} using the detection mechanism reads:
%
\begin{center}
|... -jobname "|\textit{target}|" "|[\textit{flags}]%
[|\def\jobname{|\textit{dest}|}|]|\input{|\textit{main}|}"|
\end{center}

%%%%%%%%%%%%%%%%%%%%%%%%%%%%%%%%%%%%%%%%%%%%%%%%%%%%%%%%%%%%%%%%%%%%%%%%%%%%%%%%
\subsection{Manual Code}
\label{sec:manual}

In case one cannot be certain whether the definitions file |childdoc.def|
is installed on the target \TeX{} distribution
and one prefers not to ship it,
it is conceivable to paste a few relevant commands into the sources.

To that end, drop all statements |\input{childdoc.def}|
and perform the replacements as outlined below.
Instead of |\childdocmain{|\textit{main}|}| add the following code
to the top of the main file:
%
\begin{center}
\begin{tabular}{l}
|\||ifdefined\childdocname\endinput\||fi\newif\ifchilddoc|\\
|\edef\childdocname{\scantokens\expandafter{\jobname\noexpand}}|\\
|\def\childdocmain{|\textit{main}|}\||ifx\childdocmain\childdocname\||else|\\
|\childdoctrue\includeonly{\childdocname}\let\jobname\childdocmain\||fi|\\
\end{tabular}
\end{center}
%
Instead of |\childdocof{|\textit{main}|}| just include the main file
at the top of each child file:
%
\begin{center}
|\input{|\textit{main}|}|
\end{center}
%
A simple redirection |\childdocforward{|\textit{dest}|}| is achieved by:
%
\begin{center}
|\def\jobname{|\textit{dest}|}\input{\jobname}|
\end{center}
%
The redirection with prefix
|\childdocforwardprefix[|\textit{prefix}|]{|\textit{dest}|}|
is accomplished by:
%
\begin{center}
\begin{tabular}{l}
|{\edef\jobname{\scantokens\expandafter{\jobname\noexpand}}|\\
|\def\redirectjob |\textit{prefix}|#1~~~{\gdef\jobname{|\textit{dest}|#1}}|\\
|\expandafter\redirectjob\jobname~~~}\input{\jobname}|
\end{tabular}
\end{center}

In an alternative approach,
child documents can be compiled by a specific command line
without additional code or specific definitions:
%
\begin{center}
|... -jobname "|\textit{target}|" "|[\textit{flags}]%
|\includeonly{|\textit{dest}|}\input{|\textit{main}|}"|
\end{center}
%

%%%%%%%%%%%%%%%%%%%%%%%%%%%%%%%%%%%%%%%%%%%%%%%%%%%%%%%%%%%%%%%%%%%%%%%%%%%%%%%%
%%%%%%%%%%%%%%%%%%%%%%%%%%%%%%%%%%%%%%%%%%%%%%%%%%%%%%%%%%%%%%%%%%%%%%%%%%%%%%%%
\section{Information}

%%%%%%%%%%%%%%%%%%%%%%%%%%%%%%%%%%%%%%%%%%%%%%%%%%%%%%%%%%%%%%%%%%%%%%%%%%%%%%%%
\subsection{Copyright}

Copyright \copyright{} 2017--2018 Niklas Beisert

This work may be distributed and/or modified under the
conditions of the \LaTeX{} Project Public License, either version 1.3
of this license or (at your option) any later version.
The latest version of this license is in
  \url{http://www.latex-project.org/lppl.txt}
and version 1.3 or later is part of all distributions of \LaTeX{}
version 2005/12/01 or later.

This work has the LPPL maintenance status `maintained'.

The Current Maintainer of this work is Niklas Beisert.

This work consists of the files |README.txt|, |childdoc.ins| and |childdoc.dtx|
as well as the derived files |childdoc.def|, |cdocsamp.tex|
with |cdocsch1.tex|, |cdocsch2.tex|, |cdocspt3.tex|, |cdocspt4.tex|,
|cdocsdrf.tex|, |cdocsfn1.tex|, |cdocsfn2.tex|
as well as |childdoc.pdf|.

%%%%%%%%%%%%%%%%%%%%%%%%%%%%%%%%%%%%%%%%%%%%%%%%%%%%%%%%%%%%%%%%%%%%%%%%%%%%%%%%
\subsection{Files and Installation}

The package consists of the files:
%
\begin{center}
\begin{tabular}{ll}
    |README.txt|   & readme file \\
    |childdoc.ins| & installation file \\
    |childdoc.dtx| & source file \\
    |childdoc.def| & definition file \\
    |cdocsamp.tex| & sample main file \\
    |cdocsch1.tex| & sample include file \\
    |cdocsch2.tex| & sample include file \\
    |cdocspt3.tex| & sample part file \\
    |cdocspt4.tex| & sample part file \\
    |cdocsdrf.tex| & sample redirection file \\
    |cdocsfn1.tex| & sample redirection file \\
    |cdocsfn2.tex| & sample redirection file \\
    |childdoc.pdf| & manual
\end{tabular}
\end{center}
%
The distribution consists of the files
|README.txt|, |childdoc.ins| and |childdoc.dtx|.
%
\begin{itemize}
\item
Run (pdf)\LaTeX{} on |childdoc.dtx|
to compile the manual |childdoc.pdf| (this file).
\item
Run \LaTeX{} on |childdoc.ins| to create the definitions file |childdoc.def|
and the sample |cdocsamp.tex| with include files
|cdocsch1.tex|, |cdocsch2.tex|, |cdocspt3.tex|, |cdocspt4.tex|,
|cdocsdrf.tex|, |cdocsfn1.tex|, |cdocsfn2.tex|.
Then copy the file |childdoc.def| to an appropriate directory of your \LaTeX{}
distribution, e.g.\ \textit{texmf-root}|/tex/latex/childdoc|.
\end{itemize}

%%%%%%%%%%%%%%%%%%%%%%%%%%%%%%%%%%%%%%%%%%%%%%%%%%%%%%%%%%%%%%%%%%%%%%%%%%%%%%%%
\subsection{Related CTAN Packages}

There are several other packages which offer a similar functionality:
%
\begin{itemize}
\item
The packages
\href{http://ctan.org/pkg/docmute}{\textsf{docmute}},
\href{http://ctan.org/pkg/includex}{\textsf{includex}} and
\href{http://ctan.org/pkg/standalone}{\textsf{standalone}}
provide commands to include only the document body of
a child file thus allowing both files to be compiled individually.
\item
The packages \href{http://ctan.org/pkg/subdocs}{\textsf{subdocs}}
and \href{http://ctan.org/pkg/subfiles}{\textsf{subfiles}}
provide structures in which the main and child documents can be
encapsulated and allowing them to be compiled individually.
The inclusion mechanism is different from the conventional |\include|.
\item
The package \href{http://ctan.org/pkg/combine}{\textsf{combine}}
is an elaborate solution to combine several documents into one.
\end{itemize}
%
See also the CTAN topic \href{http://ctan.org/topic/subdocs}{\textsf{subdocs}}
for further related packages.
The present package differs from the above solutions in that
a document structure constructed with the conventional |\include| mechanism
just needs two extra commands at the top of every file
such that all constituent files can be compiled individually.

%%%%%%%%%%%%%%%%%%%%%%%%%%%%%%%%%%%%%%%%%%%%%%%%%%%%%%%%%%%%%%%%%%%%%%%%%%%%%%%%
%\subsection{Feature Suggestions}
%
%The following is a list of features which may be useful for future
%versions of this package:
%%
%\begin{itemize}
%\item
%\ldots
%\end{itemize}

%%%%%%%%%%%%%%%%%%%%%%%%%%%%%%%%%%%%%%%%%%%%%%%%%%%%%%%%%%%%%%%%%%%%%%%%%%%%%%%%
\subsection{Revision History}

%%%%%%%%%%%%%%%%%%%%%%%%%%%%%%%%%%%%%%%%
\paragraph{v2.0:} 2018/12/30

\begin{itemize}
\item
immediate forward processing
\item
added |\childdocby| mechanism
\item
manual restructured
\end{itemize}

%%%%%%%%%%%%%%%%%%%%%%%%%%%%%%%%%%%%%%%%
\paragraph{v1.6:} 2018/01/17

\begin{itemize}
\item
application for development of include files
\item
corrections to manual
\end{itemize}

%%%%%%%%%%%%%%%%%%%%%%%%%%%%%%%%%%%%%%%%
\paragraph{v1.5:} 2017/05/21

\begin{itemize}
\item
more complete structuring introduced
\item
|\childdocof| introduced
\item
|\childdoc| renamed to |\childdocmain|
\item
|\childredirect| renamed to |\childdocforward| and |\childdocforwardprefix|
and functionality expanded
\end{itemize}

%%%%%%%%%%%%%%%%%%%%%%%%%%%%%%%%%%%%%%%%
\paragraph{v1.0:} 2017/04/27

\begin{itemize}
\item
manual and install package
\item
first version published on CTAN
\end{itemize}

%%%%%%%%%%%%%%%%%%%%%%%%%%%%%%%%%%%%%%%%
\paragraph{v0.6:} 2017/04/26

\begin{itemize}
\item
redirection mechanism added
\end{itemize}

%%%%%%%%%%%%%%%%%%%%%%%%%%%%%%%%%%%%%%%%
\paragraph{v0.5:} 2017/04/26

\begin{itemize}
\item
functionality in definition file
\end{itemize}


%%%%%%%%%%%%%%%%%%%%%%%%%%%%%%%%%%%%%%%%%%%%%%%%%%%%%%%%%%%%%%%%%%%%%%%%%%%%%%%%
%%%%%%%%%%%%%%%%%%%%%%%%%%%%%%%%%%%%%%%%%%%%%%%%%%%%%%%%%%%%%%%%%%%%%%%%%%%%%%%%
%%%%%%%%%%%%%%%%%%%%%%%%%%%%%%%%%%%%%%%%%%%%%%%%%%%%%%%%%%%%%%%%%%%%%%%%%%%%%%%%
\appendix

\settowidth\MacroIndent{\rmfamily\scriptsize 000\ }

 \DocInput{childdoc.dtx}

\end{document}
%</driver>
% \fi
%
% %%%%%%%%%%%%%%%%%%%%%%%%%%%%%%%%%%%%%%%%%%%%%%%%%%%%%%%%%%%%%%%%%%%%%%%%%%%%%%
% %%%%%%%%%%%%%%%%%%%%%%%%%%%%%%%%%%%%%%%%%%%%%%%%%%%%%%%%%%%%%%%%%%%%%%%%%%%%%%
% \section{Sample}
%\iffalse
%<*samplemain>
%\fi
%
% The following presents a sample document
% with two chapters, two parts, a title page,
% a compile flag as well as three forwarding files to set the flag.
% It consists of eight |.tex| files:
% \begin{center}
% \begin{tabular}{ll}
% |cdocsamp.tex|&main file\\
% |cdocsch1.tex|&include file for chapter 1\\
% |cdocsch2.tex|&include file for chapter 2\\
% |cdocspt3.tex|&include file for part 3\\
% |cdocspt4.tex|&include file for part 4\\
% |cdocsdrf.tex|&forwarding file for main file in draft mode\\
% |cdocsfi1.tex|&forwarding file for final version of chapter 1\\
% |cdocsfi2.tex|&forwarding file for final version of chapter 2\\
% \end{tabular}
% \end{center}
% Each of the eight files can be compiled directly by the \LaTeX{} compiler.
%
% %%%%%%%%%%%%%%%%%%%%%%%%%%%%%%%%%%%%%%
% \paragraph{Main File.}
%
% The main file is called |cdocsamp.tex|.
%
% Load the \textsf{childdoc} definitions and
% declare the filename for the main document:
%    \begin{macrocode}
\input{childdoc.def}
\childdocmain{}
%    \end{macrocode}

% Optional override for |\version| flag:
%    \begin{macrocode}
%%\ifchilddoc\else\providecommand{\version}{draft}\fi
%    \end{macrocode}

% Define the default values for the |\version| flag
% (|final| for the main file and |draft| for childs):
%    \begin{macrocode}
\ifchilddoc
\providecommand{\version}{draft}
\else
\providecommand{\version}{final}
\fi
%    \end{macrocode}

% Load the standard document class:
%    \begin{macrocode}
\documentclass[12pt]{article}
%    \end{macrocode}

% Start the document body:
%    \begin{macrocode}
\begin{document}
%    \end{macrocode}

% Declare a title page.
% Print title, part of document being processed and version flag:
%    \begin{macrocode}
\addtocounter{page}{-1}
\begin{center}
{\LARGE\bfseries{}childdoc example\par}
\vspace{1cm}
\ifchilddoc
\ifchilddocmanual part\else chapter\fi:
`\childdocname' of `\childdocjob'\par
\else
main document: `\childdocjob'\par
\fi
version: \version\par
\end{center}
\newpage
%    \end{macrocode}

% Manually include selected file,
% otherwise process as usual:
%    \begin{macrocode}
\ifchilddocmanual
\section*{part `\childdocname'}
\input{\childdocname}
\else
%    \end{macrocode}

% Include the two chapters:
%    \begin{macrocode}
\include{cdocsch1}
\include{cdocsch2}
%    \end{macrocode}

% Include the two parts unless only chapters should be displayed:
%    \begin{macrocode}
\ifchilddoc\else
\section{part three}
\input{cdocspt3}
\section{part four}
\input{cdocspt4}
\fi
%    \end{macrocode}

% Process as usual until here:
%    \begin{macrocode}
\fi
%    \end{macrocode}

% End of document body:
%    \begin{macrocode}
\end{document}
%    \end{macrocode}
%\iffalse
%</samplemain>
%\fi
%
% %%%%%%%%%%%%%%%%%%%%%%%%%%%%%%%%%%%%%%
% \paragraph{Chapter Include Files.}
%
% The include files are called |cdocsch1.tex| and |cdocsch2.tex|.
%
%\iffalse
%<*samplechap1|samplechap2>
%\fi

% Optional override for |\version| flag:
%    \begin{macrocode}
%%\providecommand{\version}{final}
%    \end{macrocode}

% Include the main document:
%    \begin{macrocode}
\input{childdoc.def}
\childdocof{cdocsamp}
%    \end{macrocode}

%\iffalse
%</samplechap1|samplechap2>
%\fi
%
%\iffalse
%<*samplechap1>
%\fi
% Some text for chapter 1:
%    \begin{macrocode}
\section{one}
some text in chapter one
%    \end{macrocode}

%\iffalse
%</samplechap1>
%\fi
% Some text for chapter 2:
%\iffalse
%<*samplechap2>
%\fi
%    \begin{macrocode}
\section{two}
more text in chapter two
%    \end{macrocode}

%\iffalse
%</samplechap2>
%\fi
%
% %%%%%%%%%%%%%%%%%%%%%%%%%%%%%%%%%%%%%%
% \paragraph{Part Include Files.}
%
% The include files are called |cdocspt3.tex| and |cdocspt4.tex|.
%
%\iffalse
%<*samplepart3|samplepart4>
%\fi

% Optional override for |\version| flag:
%    \begin{macrocode}
%%\providecommand{\version}{final}
%    \end{macrocode}

% Include the main document:
%    \begin{macrocode}
\input{childdoc.def}
\childdocby{cdocsamp}
%    \end{macrocode}

%\iffalse
%</samplepart3|samplepart4>
%\fi
%
%\iffalse
%<*samplepart3>
%\fi
% Some text for part 3:
%    \begin{macrocode}
some text in part three
%    \end{macrocode}

%\iffalse
%</samplepart3>
%\fi
% Some text for part 4:
%\iffalse
%<*samplepart4>
%\fi
%    \begin{macrocode}
more text in part four
%    \end{macrocode}

%\iffalse
%</samplepart4>
%\fi
%
% %%%%%%%%%%%%%%%%%%%%%%%%%%%%%%%%%%%%%%
% \paragraph{Forwarding for a Complete Draft.}
%
% The following forwarding file |cdocsdrf.tex|
% compiles the main document in draft mode:
%\iffalse
%<*sampledraft>
%\fi
%    \begin{macrocode}
\def\version{draft}
\input{childdoc.def}
\childdocforward{cdocsamp}
%    \end{macrocode}

%\iffalse
%</sampledraft>
%\fi
%
% %%%%%%%%%%%%%%%%%%%%%%%%%%%%%%%%%%%%%%
% \paragraph{Forwarding for Final Version of the Chapters.}
%
% The following forwarding files |cdocsfn1.tex| and |cdocsfn2.tex|
% (with identical content)
% compile the final versions of the child documents
% |cdocsch1.tex| and |cdocsch2.tex|, respectively:
%\iffalse
%<*samplefinal>
%\fi
%    \begin{macrocode}
\def\version{final}
\input{childdoc.def}
\childdocforwardprefix[cdocsamp]{cdocsfn}{cdocsch}
%    \end{macrocode}

%\iffalse
%</samplefinal>
%\fi
%
% %%%%%%%%%%%%%%%%%%%%%%%%%%%%%%%%%%%%%%
% \paragraph{Command Line Processing.}
%
% The following three command lines generate the output files
% |cdocscld|, |cdocscl1| and |cdocscl2|
% which should be identical to
% |cdocsdrf|, |cdocsch1| and |cdocsfn2|, respectively:
% \begin{center}
% \begin{tabular}{l}
% |latex -jobname cdocscld \|\\
% |  "\def\version{draft}\input{childdoc.def}\childdocforward{cdocsamp}"|\\
% |latex -jobname cdocscl1 \|\\
% |  "\input{childdoc.def}\childdocforward[cdocsamp]{cdocsch1}"|\\
% |latex -jobname cdocscl2 \|\\
% |  "\def\version{final}\input{childdoc.def}\childdocforward{cdocsch2}"|
% \end{tabular}
% \end{center}
% Note that the trailing backslash on each first line
% merely continues the input to the second line
% (for convenient cut ant paste).
% Furthermore, the command |latex| can be replaced by any
% of its alternative versions such as |pdflatex|.
%
% %%%%%%%%%%%%%%%%%%%%%%%%%%%%%%%%%%%%%%%%%%%%%%%%%%%%%%%%%%%%%%%%%%%%%%%%%%%%%%
% %%%%%%%%%%%%%%%%%%%%%%%%%%%%%%%%%%%%%%%%%%%%%%%%%%%%%%%%%%%%%%%%%%%%%%%%%%%%%%
% \section{Implementation}
%\iffalse
%<*package>
%\fi
%
% This section describes the definitions file |childdoc.def|.

% The definitions cannot be loaded using |\usepackage| or |\RequirePackage|
% which has a mechanism to prevent loading a style file more than once.
% When loading the definitions by means of |\input|
% multiple instances have to be prevented manually:
%\iffalse
%This code needs to be before the `\ProvidesFile' directive
%which is defined at the beginning of this file.
%Therefore it is also placed there and commented out here.
%</package>
%<*discard>
%\fi
%    \begin{macrocode}
\ifdefined\childdocmain\endinput\fi
%    \end{macrocode}
%\iffalse
%</discard>
%<*package>
%\fi
%
% \macro{\ifchilddoc}
% \macro{\ifchilddocmanual}
% The conditional |\ifchilddoc| tells whether a
% child (true) or main (false) document is being compiled.
% The conditional |\ifchilddocmanual| tells whether
% the |\includeonly| mechanism is used (false) or
% the selection of child files must be performed manually (true).
% The definitions initialise to false:
%    \begin{macrocode}
\newif\ifchilddoc
\newif\ifchilddocmanual
%    \end{macrocode}

% \macro{\childdocname}
% \macro{\childdocjob}
% The macro |\childdocname| stores the name of the main document
% to be compiled. The macro |\childdocjob| stores the name of
% the document on which the \LaTeX{} compiler was originally invoked.
% The content of |\jobname| cannot be compared
% to filenames specified in the source due to different catcodes.
% The following code rescans |\jobname|, stores the result
% in |\childdocname| and saves a copy in |\childdocjob|:
%    \begin{macrocode}
\edef\childdocname{\scantokens\expandafter{\jobname\noexpand}}
\let\childdocjob\childdocname
%    \end{macrocode}

% \macro{\childdocdisable}
% The macro |\childdocdisable| prevents the main file
% from being processed more than once.
% At this stage, the main document command |\childdocmain|
% is assumed to be called once again where it should do nothing.
% Any subsequent call to it should prevent
% a secondary processing of the main document
% It overwrites the forwarding commands
% |\childdocof| and |\childdocforward|
% with empty macros to prevent further inclusions of the main document:
%    \begin{macrocode}
\newcommand{\childdocdisable}
{
  \renewcommand{\childdocmain}[1]{\renewcommand{\childdocmain}[1]{\endinput}}
  \renewcommand{\childdocof}[1]{}
  \renewcommand{\childdocby}[2][]{}
  \renewcommand{\childdocforward}[2][]{}
  \renewcommand{\childdocdisable}{}
}
%    \end{macrocode}

% \macro{\childdocmain}
% The macro |\childdocmain| is to be called at the top of the main file
% with nothing or the main filename (without extension) as argument.
% First, it breaks loops.
% If the argument is not empty and does not match |\childdocname|
% (which is set by the first inclusion of |childdoc.def|),
% |\ifchilddoc| is set to true, |\includeonly| is applied to the child file
% and |\jobname| is set to the main file
% (for proper handling of |.aux| files):
%    \begin{macrocode}
\newcommand{\childdocmain}[1]
{
  \childdocdisable\childdocmain{}
  \if?#1?\else
    \begingroup
      \def\childdoctmp{#1}
      \ifx\childdoctmp\childdocname
        \def\childdoctmp{}
      \else
        \def\childdoctmp
        {
          \childdoctrue
          \includeonly{\childdocname}
          \def\childdocjob{#1}
          \def\jobname{#1}
        }
      \fi
      \expandafter
    \endgroup
    \childdoctmp
  \fi
}
%    \end{macrocode}

% \macro{\childdocof}
% The command |\childdocof| redirects
% compilation to the main file |#1|.
%    \begin{macrocode}
\newcommand{\childdocof}[1]
{
  \childdocdisable
  \childdoctrue
  \includeonly{\childdocname}
  \def\jobname{#1}
  \def\childdocjob{#1}
  \input{#1}
}
%    \end{macrocode}

% \macro{\childdocby}
% The command |\childdocby| ....
%    \begin{macrocode}
\newcommand{\childdocby}[2][]
{
  \childdocdisable
  \childdoctrue
  \childdocmanualtrue
  \if?#1?\else
    \def\jobname{#2}
  \fi
  \def\childdocjob{#2}
  \input{#2}
  \endinput
}
%    \end{macrocode}

% \macro{\childdocforward}
% The command |\childdocforward| redirects
% compilation to the main file or
% (if the optional argument is given) a child file.
% Parameters are set as if the main file
% or a child file starting with |\childdocof| was compiled.
% Then compilation is handed over to the main file:
%    \begin{macrocode}
\newcommand{\childdocforward}[2][]
{
  \begingroup
    \if?#1?
      \def\childdoctmp
      {
        \def\childdocname{#2}
        \def\childdocjob{#2}
        \def\jobname{#2}
        \input{#2}
        \endinput
      }
    \else
      \def\childdoctmp
      {
        \childdocdisable
        \def\childdocname{#2}
        \childdoctrue
        \includeonly{#2}
        \def\childdocjob{#1}
        \def\jobname{#1}
        \input{#1}
        \endinput
      }
    \fi
    \expandafter
  \endgroup
  \childdoctmp
}
%    \end{macrocode}

% \macro{\childdocforwardprefix}
% The command |\childdocforwardprefix| redirects
% compilation to the main or a child file by means of a pattern.
% The prefix |#1| in the current filename is replaced by |#2|
% and the suffix of the current filename is kept
% (it is assumed that the filename does not contain the substring `|~~~|'
% which is used as a delimiter).
% Compilation is handed over to the new file by |\childdocforward|:
%    \begin{macrocode}
\newcommand{\childdocforwardprefix}[3][]
{
  \begingroup
    \def\childdocextract #2##1~~~{\def\childdoctmp{\childdocforward[#1]{#3##1}}}
    \expandafter\childdocextract\childdocname~~~
    \expandafter
  \endgroup
  \childdoctmp
}
%    \end{macrocode}

% \macro{\childdoc}
% The deprecated macro |\childdoc| is a legacy version of |\childdocmain|:
%    \begin{macrocode}
\newcommand{\childdoc}{\childdocmain}
%    \end{macrocode}

% \macro{\childdocredirect}
% The deprecated macro |\childdocredirect| is a legacy version
% of |\childdocforward| and |\childdocforwardprefix|:
%    \begin{macrocode}
\newcommand{\childdocredirect}[2][]
{
  \begingroup
    \if?#1?
      \def\childdoctmp{\childdocforward{#2}}
    \else
      \def\childdoctmp{\childdocforwardprefix{#1}{#2}}
    \fi
    \expandafter
  \endgroup
  \childdoctmp
}
%    \end{macrocode}

%\iffalse
%</package>
%\fi
%
\endinput
\childdocforward[|\textit{main}|]{|\textit{dest}|}"|
\end{center}
%
Here \textit{target} is the name of the output file,
\textit{main} is the name of the main file
and \textit{dest} is the name of the main or child file to be processed
(all filenames without extensions).
The optional argument \textit{main} can be omitted
if \textit{main} matches \textit{dest}.
Optionally, compilation \textit{flags} can be defined via |\def| commands.
This command line makes the \TeX{} engine believe
it is compiling the file \textit{target}
whose content is specified as the latter parameter.
The provided code then forwards the processing to
\textit{main} or \textit{dest} as described in \secref{sec:forward}.

%%%%%%%%%%%%%%%%%%%%%%%%%%%%%%%%%%%%%%%%%%%%%%%%%%%%%%%%%%%%%%%%%%%%%%%%%%%%%%%%
\subsection{Include by Input}
\label{sec:input}

Including child documents by |\include| has some restrictions by design.
Most notably, the content of a child document always occupies
its own set of pages; pages cannot be shared between child documents.
Usually, this behaviour makes perfect sense
because each child document contain an essential part of the document.
However, in some situations it may be desirable to compose
a document from a collection of parts
without having mandatory page breaks between then.
For this case, the package
provides a mechanism to include parts
by |\input| which can also be processed individually.
However, by construction this mechanism
requires manual handling of the content to be output.

%%%%%%%%%%%%%%%%%%%%%%%%%%%%%%%%%%%%%%%%
\DescribeMacro{\ifchilddocmanual}
The main file should be prepared as usual, see \secref{sec:include}.
However, the document body must make a distinction
between processing of an individual part and of the main document, e.g.:
%
\begin{center}
\begin{tabular}{l}
|\ifchilddocmanual|\\
|\input{\childdocname}|\\
|\||else|\\
\textit{document body with }|\input{|\textit{part}|}|\\
|\||fi|
\end{tabular}
\end{center}
%
The conditional |\ifchilddocmanual| is true whenever
a part to be included by |\input| is being compiled,
and the name of the part is stored in |\childdocname|.

%%%%%%%%%%%%%%%%%%%%%%%%%%%%%%%%%%%%%%%%
\DescribeMacro{\childdocby}
Each part to be included by |\input| should start with:
%
\begin{center}
\begin{tabular}{l}
|% \iffalse
%
% childdoc.dtx Copyright (C) 2017-2018 Niklas Beisert
%
% This work may be distributed and/or modified under the
% conditions of the LaTeX Project Public License, either version 1.3
% of this license or (at your option) any later version.
% The latest version of this license is in
%   http://www.latex-project.org/lppl.txt
% and version 1.3 or later is part of all distributions of LaTeX
% version 2005/12/01 or later.
%
% This work has the LPPL maintenance status `maintained'.
%
% The Current Maintainer of this work is Niklas Beisert.
%
% This work consists of the files childdoc.dtx and childdoc.ins
% and the derived files childdoc.def and cdocsamp.tex with
% cdocsch1.tex, cdocsch2.tex, cdocsdrf.tex, cdocsfn1.tex, cdocsfn2.tex.
%
%<package>\ifdefined\childdocmain\endinput\fi
%<package>\ProvidesFile{childdoc.def}[2018/12/30 v2.0 child document driver]
%<samplemain>\ProvidesFile{cdocsamp.tex}[2018/12/30 v2.0 sample for childdoc]
%<*driver>
%\ProvidesFile{childdoc.drv}[2018/12/30 v2.0 childdoc reference manual file]
\PassOptionsToClass{10pt,a4paper}{article}
\documentclass{ltxdoc}

\usepackage[margin=35mm]{geometry}
\usepackage{hyperref}
\usepackage{hyperxmp}
\usepackage[usenames]{color}

\hypersetup{colorlinks=true}
\hypersetup{pdfstartview=FitH}
\hypersetup{pdfpagemode=UseNone}
\hypersetup{pdfsource={}}
\hypersetup{pdflang={en-UK}}
\hypersetup{pdfcopyright={Copyright 2017-2018 Niklas Beisert.
  This work may be distributed and/or modified under the
  conditions of the LaTeX Project Public License, either version 1.3
  of this license or (at your option) any later version.}}
\hypersetup{pdflicenseurl={http://www.latex-project.org/lppl.txt}}
\hypersetup{pdfcontactaddress={ETH Zurich, ITP, HIT K,
  Wolfgang-Pauli-Strasse 27}}
\hypersetup{pdfcontactpostcode={8093}}
\hypersetup{pdfcontactcity={Zurich}}
\hypersetup{pdfcontactcountry={Switzerland}}
\hypersetup{pdfcontactemail={nbeisert@itp.phys.ethz.ch}}
\hypersetup{pdfcontacturl={http://people.phys.ethz.ch/\xmptilde nbeisert/}}

\newcommand{\secref}[1]{\hyperref[#1]{section \ref*{#1}}}

\parskip1ex
\parindent0pt
\let\olditemize\itemize
\def\itemize{\olditemize\parskip0pt}

\begin{document}

\title{The \textsf{childdoc} Package}
\hypersetup{pdftitle={The childdoc Package}}
\author{Niklas Beisert\\[2ex]
  Institut f\"ur Theoretische Physik\\
  Eidgen\"ossische Technische Hochschule Z\"urich\\
  Wolfgang-Pauli-Strasse 27, 8093 Z\"urich, Switzerland\\[1ex]
  \href{mailto:nbeisert@itp.phys.ethz.ch}
  {\texttt{nbeisert@itp.phys.ethz.ch}}}
\hypersetup{pdfauthor={Niklas Beisert}}
\hypersetup{pdfsubject={Manual for the LaTeX2e Package childdoc}}
\date{30 December 2018, \textsf{v2.0}}
\maketitle

\begin{abstract}\noindent
\textsf{childdoc} is a \LaTeXe{} package
that enables the direct compilation
of document sections included by |\include|
to individual files.
\end{abstract}

\begingroup
\parskip0ex
\tableofcontents
\endgroup

%%%%%%%%%%%%%%%%%%%%%%%%%%%%%%%%%%%%%%%%%%%%%%%%%%%%%%%%%%%%%%%%%%%%%%%%%%%%%%%%
%%%%%%%%%%%%%%%%%%%%%%%%%%%%%%%%%%%%%%%%%%%%%%%%%%%%%%%%%%%%%%%%%%%%%%%%%%%%%%%%
\section{Introduction}

\LaTeX{} provides a mechanism to structure a large document (such as a book)
into a main file and several child files (containing the chapters)
using the |\include| command.
This mechanism is beneficial for documents
which span hundreds of pages in order to
make the source file(s) more manageable.
Moreover, compilation can be restricted to
selected child files by means of the |\includeonly| command.
The latter feature can be used to reduce the compilation time while editing
(this was significantly more useful in the earlier days of \LaTeX{})
or to generate a smaller document which is easier to navigate.
Another application of |\includeonly| is to generate
documents consisting of selected parts of the complete document.

However, there are a few drawbacks of the plain |\include| mechanism:
\begin{itemize}
\item
The child files cannot be compiled on their own,
they can only be compiled via the main file.
A naive editing environment
(such as a text editor with an option
to have the current file processed by \LaTeX)
may require one to switch to the main file before compiling;
attempting to compile the child file produces errors.
\item
The main file must be modified (each time)
to adjust the |\includeonly| command
to the present needs. This easily leaves the main file in a messy state.
\item
The generated document will always carry the filename
of the main document. This is inconvenient if
several child files are to be compiled and
to be kept for distribution.
\end{itemize}

The present package provides a simple interface
to make child files individually compilable by \LaTeX{}.
Compiling a child file then has the same effect as compiling
the main file with an |\includeonly| command
to select the appropriate child.
Moreover the generated document will carry the name of the child
rather than the main file.
This resolves all three above issues.

This feature is meant to make the editing of books,
thesis documents and lecture notes somewhat more convenient.
However, the package can also be used efficiently for
composing a series of documents (such as exercise sheets)
which are typically distributed individually.
It then assists the author in generating the individual documents
(potentially in different versions)
as well as a document containing the collected series.
Another application is in developing style files
or other kinds of included material
where compilation of the style file could redirect
to a sample or test file.

%%%%%%%%%%%%%%%%%%%%%%%%%%%%%%%%%%%%%%%%%%%%%%%%%%%%%%%%%%%%%%%%%%%%%%%%%%%%%%%%
%%%%%%%%%%%%%%%%%%%%%%%%%%%%%%%%%%%%%%%%%%%%%%%%%%%%%%%%%%%%%%%%%%%%%%%%%%%%%%%%
\section{Usage}

First of all, the package \textsf{childdoc} is \emph{not} a standard
\LaTeXe{} |.sty| style file! Therefore it needs to be invoked in
a non-standard way.

%%%%%%%%%%%%%%%%%%%%%%%%%%%%%%%%%%%%%%%%%%%%%%%%%%%%%%%%%%%%%%%%%%%%%%%%%%%%%%%%
\subsection{Included Files}
\label{sec:include}

%%%%%%%%%%%%%%%%%%%%%%%%%%%%%%%%%%%%%%%%
\DescribeMacro{\childdocmain}
To use the package, add the commands
\begin{center}
\begin{tabular}{l}
|\input{childdoc.def}|\\
|\childdocmain{}|\\
\end{tabular}
\end{center}
at the very top of the main \LaTeX{} file,
in particular \emph{before} the |\documentclass| statement!
The argument of |\childdocmain| should be left empty
(but it must be present).

%%%%%%%%%%%%%%%%%%%%%%%%%%%%%%%%%%%%%%%%
\DescribeMacro{\childdocof}
Furthermore, add the commands
\begin{center}
\begin{tabular}{l}
|\input{childdoc.def}|\\
|\childdocof{|\textit{main}|}|\\
\end{tabular}
\end{center}
at the top of every child file \textit{child}
which is included by |\include{|\textit{child}|}|
from within the main file
(or at least for those files to be compiled individually).
The argument \textit{main} must be the filename of the main file.

There are a couple of
considerations in setting up the main and child documents:

%%%%%%%%%%%%%%%%%%%%%%%%%%%%%%%%%%%%%%%%
\paragraph{Restrictions.}

Please note the following restrictions:
\begin{itemize}
\item
|\childdocmain| must be called with one argument \textit{main}
to ensure compatibility with earlier version of the package.
It must either be empty (|\childdocmain{}|)
or precisely match the filename of the main file in which it is specified.
See \secref{sec:detection} for further information.
\item
The filename \textit{main} must be specified without the |.tex| extension.
\item
The filename \textit{main} is case sensitive
(even in case-insensitive file systems)
due to internal string comparison.
\item
The argument \textit{main} should be fully expanded, it cannot be a macro.
\item
Subdirectories and special characters should be avoided in filenames.
\item
The command |\childdocmain{|\textit{main}|}| must be followed by a whitespace.
It should not be followed immediately by another command
or by a comment mark `|%|'.
This is because the \TeX{} parser reads the token immediately following
the argument of |\childdocmain| and puts it
at the beginning of every child section;
however, a white\-space is ignored.
\end{itemize}

%%%%%%%%%%%%%%%%%%%%%%%%%%%%%%%%%%%%%%%%
\paragraph{Content of Main File.}

It is advisable to place all content in the child files included by |\include|.
Any output contained in the main file will appear in all child documents
unless suppressed manually;
it cannot be suppressed automatically by the |\includeonly| directive
and thus should normally be avoided.
A method to include some content in the main file
by means of conditional processing is described in \secref{sec:conditional}.

%%%%%%%%%%%%%%%%%%%%%%%%%%%%%%%%%%%%%%%%
\paragraph{Page Numbering.}

When only a part of the document is compiled,
the appropriate numbering of pages
(as well as other status parameters)
is determined from the |.aux| files.
The latter contain information from previous passes.
However this information needs to propagate through
all intermediate child documents.
Therefore the page numbering in child documents may well
be inconsistent until the complete document is compiled at least once.

A useful (if unconventional) way to always ensure a consistent
page numbering is to restart the numbering in each child document
and denote the pages by `\textit{child}|.|\textit{page}'
where \textit{child} represents the chapter/section number of the child file.
This can be achieved by the command
|\numberwithin{page}{|\textit{child}|}|
of the \textsf{amsmath} package
where \textit{child} can be |chapter| or |section|
depending on the chosen structuring.
Alternatively, one can modify the macro |\thepage| appropriately
and reset the counter |page| at the start of each child file.

%%%%%%%%%%%%%%%%%%%%%%%%%%%%%%%%%%%%%%%%%%%%%%%%%%%%%%%%%%%%%%%%%%%%%%%%%%%%%%%%
\subsection{Conditional Processing}
\label{sec:conditional}

The package provides a mechanism to compile different versions
of a document. To customise the versions further some conditional processing
can come in handy to distinguish which version is being compiled.
The package provides two macros to describe the compilation context:

%%%%%%%%%%%%%%%%%%%%%%%%%%%%%%%%%%%%%%%%
\DescribeMacro{\ifchilddoc}
The conditional |\ifchilddoc| distinguishes between the compilation of
child documents and the main document:
%
\begin{center}
|\ifchilddoc |\textit{child-code}| |[|\||else |\textit{main-code}]| \||fi|
\end{center}

%%%%%%%%%%%%%%%%%%%%%%%%%%%%%%%%%%%%%%%%
\DescribeMacro{\childdocname}
\DescribeMacro{\childdocjob}
The macro |\childdocname| contains the filename (without extension)
of the main or child file being processed.
Note that |\childdocjob| will always contain the name of the main file.

%%%%%%%%%%%%%%%%%%%%%%%%%%%%%%%%%%%%%%%%
\paragraph{Title Page.}

Conditional processing can be used to include a title or banner page
in the main document when proper precautions are taken.
Importantly, the code in the main file should ensure that the page counter
(as well as other status parameters which are stored in the |.aux| files)
takes the same value after the conditional processing.
Otherwise the page numbers may take divergent values
depending on which part is compiled.

For example, a title page could be declared by:
%
\begin{center}
\begin{tabular}{l}
|\ifchilddoc\||else|\\
|\addtocounter{page}{-1}|\\
\textit{code for title page}\\
|\newpage|\\
|\||fi|
\end{tabular}
\end{center}
%
A banner page for the child documents can be generated by:
%
\begin{center}
\begin{tabular}{l}
|\ifchilddoc|\\
|\addtocounter{page}{-1}|\\
\textit{code for banner page}\\
|\newpage|\\
|\||fi|
\end{tabular}
\end{center}
%
Here one could write a message such as:
\begin{center}
|This is the part \childdocname{} of \childdocjob{}.|
\end{center}

%%%%%%%%%%%%%%%%%%%%%%%%%%%%%%%%%%%%%%%%%%%%%%%%%%%%%%%%%%%%%%%%%%%%%%%%%%%%%%%%
\subsection{Flags}
\label{sec:flags}

The package makes it easy to generate different versions
of the main or child documents.
To this end compilation flags can be defined
and assigned different default values.
They will be particularly useful in conjunction
with the forwarding mechanism described in \secref{sec:forward}.

For example, it may be useful to have a flag |\version|
which can be set to |draft| or |final|.
The document source will contain some conditional code
depending on the value of |\version|.
Suppose further, the flag should default to |final| for the main file
and to |draft| for child files
which is a natural assignment for editing the document.
This is achieved by placing the following code
in the preamble of the main document
(below the |\childdocmain| directive):
%
\begin{center}
\begin{tabular}{l}
|\ifchilddoc|\\
|\providecommand{\version}{draft}|\\
|\||else|\\
|\providecommand{\version}{final}|\\
|\||fi|
\end{tabular}
\end{center}
%
The definition by |\providecommand| makes sure
that previous definitions are not overwritten.
Further statements |\providecommand{\version}{...}|
can thus be added before the above code to override it.

For the main file, one might add a line
(between |\childdocmain| and the above block)
%
\begin{center}
|%\ifchilddoc\||else\providecommand{\version}{draft}\||fi|
\end{center}
%
which can be uncommented to produce a draft version.
Likewise one can add a line to the very top of a child file
(above the |\childdocof{|\textit{main}|}| directive)
%
\begin{center}
|%\providecommand{\version}{final}|
\end{center}
%
which can be uncommented to produce the final version of this child document.

%%%%%%%%%%%%%%%%%%%%%%%%%%%%%%%%%%%%%%%%%%%%%%%%%%%%%%%%%%%%%%%%%%%%%%%%%%%%%%%%
\subsection{Forwarding}
\label{sec:forward}

Different versions of the main or child documents
using compilation flags as described in \secref{sec:flags}
can be (permanently) stored in different files
for convenient compilation, viewing and distribution.
To this end, the package defines a command
to pass on compilation to a different file:

%%%%%%%%%%%%%%%%%%%%%%%%%%%%%%%%%%%%%%%%
\DescribeMacro{\childdocforward}
The command |\childdocforward| redirects processing to
another source file:
%
\begin{center}
\begin{tabular}{l}
|\input{childdoc.def}|\\
|\childdocforward[|\textit{main}|]{|\textit{dest}|}|\\
\end{tabular}
\end{center}
%
The argument \textit{dest} is the destination file
(without extension).
It should be the main file or one of the child files.
Note that further \textsf{childdoc} directives
such as |\childdocof| and |\childdocforward|
in the indicated file will be processed in this form.
The optional argument \textit{main}
passes on directly to the main file \textit{main}
while pretending to compile the child \textit{dest}.
This form behaves as if \textit{dest}
issues |\childdocof{|\textit{main}|}| right away,
and no further \textsf{childdoc} directives will be processed.

%%%%%%%%%%%%%%%%%%%%%%%%%%%%%%%%%%%%%%%%
\DescribeMacro{\...prefix}
In the alternative form |\childdocforwardprefix|,
%
\begin{center}
\begin{tabular}{l}
|\input{childdoc.def}|\\
|\childdocforwardprefix[|\textit{main}|]{|\textit{prefix}|}{|\textit{dest}|}|
\end{tabular}
\end{center}
%
the destination file is determined by a pattern
depending on the current file:
To make this work, the current file must be called
`{\textit{prefix}\hspace{0.2em}\textit{suffix}}'
with \textit{prefix} matching precisely the argument.
Processing is then passed on to the file
`{\textit{dest}\hspace{0.2em}\textit{suffix}}'.
Surely, the same effect is achieved by
directly specifying the
argument `{\textit{dest}\hspace{0.2em}\textit{suffix}}'
in the first form.
However, that requires to set up a different file
for each child. With the alternative form of the command
all these files can have exactly the same content
which simplifies setting them up and maintaining them.

For example, the following file |draft.tex|
with a compilation flag |\version| as described in \secref{sec:flags}
compiles the main document as a draft:
%
\begin{center}
\begin{tabular}{l}
|\def\version{draft}|\\
|\input{childdoc.def}|\\
|\childdocforward{|\textit{main}|}|
\end{tabular}
\end{center}
%
Likewise, the following files |final|\textit{nn}|.tex|
compile the final version of the child document
|child|\textit{nn}|.tex|:
%
\begin{center}
\begin{tabular}{l}
|\def\version{final}|\\
|\input{childdoc.def}|\\
|\childdocforwardprefix{final}{child}|
\end{tabular}
\end{center}
%

Note that when several versions of a main file and/or of each child file
are to be generated, it may be convenient to set up a |Makefile| or
shell script to automatise the process.

%%%%%%%%%%%%%%%%%%%%%%%%%%%%%%%%%%%%%%%%%%%%%%%%%%%%%%%%%%%%%%%%%%%%%%%%%%%%%%%%
\subsection{Command Line Processing}
\label{sec:commandline}

The effect of redirection files can also be achieved by invoking
the \LaTeX{} compiler with a more elaborate command line.
Most conveniently this should be done as part
of a shell script or a |Makefile|.

When using \textsf{childdoc} in the main file, the following
command lines effectively perform a redirection
(note that depending on the shell being used,
backslashes may have to be doubled: `|\|' $\to$ `|\\|'):
%
\begin{center}
|... -jobname "|\textit{target}|" |\\|"|[\textit{flags}]%
|\input{childdoc.def}\childdocforward[|\textit{main}|]{|\textit{dest}|}"|
\end{center}
%
Here \textit{target} is the name of the output file,
\textit{main} is the name of the main file
and \textit{dest} is the name of the main or child file to be processed
(all filenames without extensions).
The optional argument \textit{main} can be omitted
if \textit{main} matches \textit{dest}.
Optionally, compilation \textit{flags} can be defined via |\def| commands.
This command line makes the \TeX{} engine believe
it is compiling the file \textit{target}
whose content is specified as the latter parameter.
The provided code then forwards the processing to
\textit{main} or \textit{dest} as described in \secref{sec:forward}.

%%%%%%%%%%%%%%%%%%%%%%%%%%%%%%%%%%%%%%%%%%%%%%%%%%%%%%%%%%%%%%%%%%%%%%%%%%%%%%%%
\subsection{Include by Input}
\label{sec:input}

Including child documents by |\include| has some restrictions by design.
Most notably, the content of a child document always occupies
its own set of pages; pages cannot be shared between child documents.
Usually, this behaviour makes perfect sense
because each child document contain an essential part of the document.
However, in some situations it may be desirable to compose
a document from a collection of parts
without having mandatory page breaks between then.
For this case, the package
provides a mechanism to include parts
by |\input| which can also be processed individually.
However, by construction this mechanism
requires manual handling of the content to be output.

%%%%%%%%%%%%%%%%%%%%%%%%%%%%%%%%%%%%%%%%
\DescribeMacro{\ifchilddocmanual}
The main file should be prepared as usual, see \secref{sec:include}.
However, the document body must make a distinction
between processing of an individual part and of the main document, e.g.:
%
\begin{center}
\begin{tabular}{l}
|\ifchilddocmanual|\\
|\input{\childdocname}|\\
|\||else|\\
\textit{document body with }|\input{|\textit{part}|}|\\
|\||fi|
\end{tabular}
\end{center}
%
The conditional |\ifchilddocmanual| is true whenever
a part to be included by |\input| is being compiled,
and the name of the part is stored in |\childdocname|.

%%%%%%%%%%%%%%%%%%%%%%%%%%%%%%%%%%%%%%%%
\DescribeMacro{\childdocby}
Each part to be included by |\input| should start with:
%
\begin{center}
\begin{tabular}{l}
|\input{childdoc.def}|\\
|\childdocby{|\textit{main}|}|\\
\end{tabular}
\end{center}
%
The directive |\childdocby| is similar to |\childdocof|
described in \secref{sec:include},
but the subsequent selection of content must be done manually.
To that end, both |\ifchilddoc| and |\ifchilddocmanual|
will be true upon processing of a part,
and the name of the part is stored in |\childdocname|.
Note that |\jobname| will be set to the filename of the current part
so that each part receives an individual |.aux| file
that does not interfere with the |.aux| file(s) of the main document.
This behaviour can be altered by the alternative form
|\childdocby[*]{|\textit{main}|}| (with a non-empty optional argument)
which uses the |.aux| file of the main document
by setting |\jobname| to \textit{main}.

%%%%%%%%%%%%%%%%%%%%%%%%%%%%%%%%%%%%%%%%%%%%%%%%%%%%%%%%%%%%%%%%%%%%%%%%%%%%%%%%
\subsection{Driver Development}
\label{sec:driver}

The \textsf{childdoc} mechanism can also be use for the development
of definition files such as \LaTeX{} styles or classes.
This case differs from the above setup with multiple parts
included by |\include| in that no |\includeonly| should be invoked.
This can be achieved by starting the include file
(before |\ProvidesPackage|) with:
%
\begin{center}
\begin{tabular}{l}
|\input{childdoc.def}|\\
|\childdocforward{|\textit{main}|}|\\
\end{tabular}
\end{center}
%
or alternatively with:
%
\begin{center}
\begin{tabular}{l}
|\input{childdoc.def}|\\
|\childdocby{|\textit{main}|}|\\
\end{tabular}
\end{center}
%
Both forms have slightly different effects as described above.
The main file is prepared as usual, see \secref{sec:include}.

%%%%%%%%%%%%%%%%%%%%%%%%%%%%%%%%%%%%%%%%%%%%%%%%%%%%%%%%%%%%%%%%%%%%%%%%%%%%%%%%
\subsection{Legacy Detection}
\label{sec:detection}

The directive |\childdocmain| in the main file can detect
whether the complete document or merely a child is to be compiled
even without using the directive |\childdocof|.
This method is deprecated because it is less robust
and there is no compelling reason to use it;
it is merely provided for backward compatibility
and it may be removed in future versions.

If the detection mechanism is to be used,
it is mandatory to correctly specify
the filename of the main file as the argument of |\childdocmain|:
%
\begin{center}
\begin{tabular}{l}
|\input{childdoc.def}|\\
|\childdocmain{|\textit{main}|}|\\
\end{tabular}
\end{center}
%
If |\jobname| does not match the argument \textit{main} of |\childdocmain|,
it is assumed that |\jobname| points to the child file to be compiled.
When using |\childdocmain| with the main file specified as argument,
it suffices to start a child file
with just |\input{|\textit{main}|}|
without loading of the package and using |\childdocof|.
If instead all processing is done
with the appropriate \textsf{childdoc} directives,
the argument of \textit{main} of |\childdocmain| can be empty.

An alternative version of the command line processing described
in \secref{sec:commandline} using the detection mechanism reads:
%
\begin{center}
|... -jobname "|\textit{target}|" "|[\textit{flags}]%
[|\def\jobname{|\textit{dest}|}|]|\input{|\textit{main}|}"|
\end{center}

%%%%%%%%%%%%%%%%%%%%%%%%%%%%%%%%%%%%%%%%%%%%%%%%%%%%%%%%%%%%%%%%%%%%%%%%%%%%%%%%
\subsection{Manual Code}
\label{sec:manual}

In case one cannot be certain whether the definitions file |childdoc.def|
is installed on the target \TeX{} distribution
and one prefers not to ship it,
it is conceivable to paste a few relevant commands into the sources.

To that end, drop all statements |\input{childdoc.def}|
and perform the replacements as outlined below.
Instead of |\childdocmain{|\textit{main}|}| add the following code
to the top of the main file:
%
\begin{center}
\begin{tabular}{l}
|\||ifdefined\childdocname\endinput\||fi\newif\ifchilddoc|\\
|\edef\childdocname{\scantokens\expandafter{\jobname\noexpand}}|\\
|\def\childdocmain{|\textit{main}|}\||ifx\childdocmain\childdocname\||else|\\
|\childdoctrue\includeonly{\childdocname}\let\jobname\childdocmain\||fi|\\
\end{tabular}
\end{center}
%
Instead of |\childdocof{|\textit{main}|}| just include the main file
at the top of each child file:
%
\begin{center}
|\input{|\textit{main}|}|
\end{center}
%
A simple redirection |\childdocforward{|\textit{dest}|}| is achieved by:
%
\begin{center}
|\def\jobname{|\textit{dest}|}\input{\jobname}|
\end{center}
%
The redirection with prefix
|\childdocforwardprefix[|\textit{prefix}|]{|\textit{dest}|}|
is accomplished by:
%
\begin{center}
\begin{tabular}{l}
|{\edef\jobname{\scantokens\expandafter{\jobname\noexpand}}|\\
|\def\redirectjob |\textit{prefix}|#1~~~{\gdef\jobname{|\textit{dest}|#1}}|\\
|\expandafter\redirectjob\jobname~~~}\input{\jobname}|
\end{tabular}
\end{center}

In an alternative approach,
child documents can be compiled by a specific command line
without additional code or specific definitions:
%
\begin{center}
|... -jobname "|\textit{target}|" "|[\textit{flags}]%
|\includeonly{|\textit{dest}|}\input{|\textit{main}|}"|
\end{center}
%

%%%%%%%%%%%%%%%%%%%%%%%%%%%%%%%%%%%%%%%%%%%%%%%%%%%%%%%%%%%%%%%%%%%%%%%%%%%%%%%%
%%%%%%%%%%%%%%%%%%%%%%%%%%%%%%%%%%%%%%%%%%%%%%%%%%%%%%%%%%%%%%%%%%%%%%%%%%%%%%%%
\section{Information}

%%%%%%%%%%%%%%%%%%%%%%%%%%%%%%%%%%%%%%%%%%%%%%%%%%%%%%%%%%%%%%%%%%%%%%%%%%%%%%%%
\subsection{Copyright}

Copyright \copyright{} 2017--2018 Niklas Beisert

This work may be distributed and/or modified under the
conditions of the \LaTeX{} Project Public License, either version 1.3
of this license or (at your option) any later version.
The latest version of this license is in
  \url{http://www.latex-project.org/lppl.txt}
and version 1.3 or later is part of all distributions of \LaTeX{}
version 2005/12/01 or later.

This work has the LPPL maintenance status `maintained'.

The Current Maintainer of this work is Niklas Beisert.

This work consists of the files |README.txt|, |childdoc.ins| and |childdoc.dtx|
as well as the derived files |childdoc.def|, |cdocsamp.tex|
with |cdocsch1.tex|, |cdocsch2.tex|, |cdocspt3.tex|, |cdocspt4.tex|,
|cdocsdrf.tex|, |cdocsfn1.tex|, |cdocsfn2.tex|
as well as |childdoc.pdf|.

%%%%%%%%%%%%%%%%%%%%%%%%%%%%%%%%%%%%%%%%%%%%%%%%%%%%%%%%%%%%%%%%%%%%%%%%%%%%%%%%
\subsection{Files and Installation}

The package consists of the files:
%
\begin{center}
\begin{tabular}{ll}
    |README.txt|   & readme file \\
    |childdoc.ins| & installation file \\
    |childdoc.dtx| & source file \\
    |childdoc.def| & definition file \\
    |cdocsamp.tex| & sample main file \\
    |cdocsch1.tex| & sample include file \\
    |cdocsch2.tex| & sample include file \\
    |cdocspt3.tex| & sample part file \\
    |cdocspt4.tex| & sample part file \\
    |cdocsdrf.tex| & sample redirection file \\
    |cdocsfn1.tex| & sample redirection file \\
    |cdocsfn2.tex| & sample redirection file \\
    |childdoc.pdf| & manual
\end{tabular}
\end{center}
%
The distribution consists of the files
|README.txt|, |childdoc.ins| and |childdoc.dtx|.
%
\begin{itemize}
\item
Run (pdf)\LaTeX{} on |childdoc.dtx|
to compile the manual |childdoc.pdf| (this file).
\item
Run \LaTeX{} on |childdoc.ins| to create the definitions file |childdoc.def|
and the sample |cdocsamp.tex| with include files
|cdocsch1.tex|, |cdocsch2.tex|, |cdocspt3.tex|, |cdocspt4.tex|,
|cdocsdrf.tex|, |cdocsfn1.tex|, |cdocsfn2.tex|.
Then copy the file |childdoc.def| to an appropriate directory of your \LaTeX{}
distribution, e.g.\ \textit{texmf-root}|/tex/latex/childdoc|.
\end{itemize}

%%%%%%%%%%%%%%%%%%%%%%%%%%%%%%%%%%%%%%%%%%%%%%%%%%%%%%%%%%%%%%%%%%%%%%%%%%%%%%%%
\subsection{Related CTAN Packages}

There are several other packages which offer a similar functionality:
%
\begin{itemize}
\item
The packages
\href{http://ctan.org/pkg/docmute}{\textsf{docmute}},
\href{http://ctan.org/pkg/includex}{\textsf{includex}} and
\href{http://ctan.org/pkg/standalone}{\textsf{standalone}}
provide commands to include only the document body of
a child file thus allowing both files to be compiled individually.
\item
The packages \href{http://ctan.org/pkg/subdocs}{\textsf{subdocs}}
and \href{http://ctan.org/pkg/subfiles}{\textsf{subfiles}}
provide structures in which the main and child documents can be
encapsulated and allowing them to be compiled individually.
The inclusion mechanism is different from the conventional |\include|.
\item
The package \href{http://ctan.org/pkg/combine}{\textsf{combine}}
is an elaborate solution to combine several documents into one.
\end{itemize}
%
See also the CTAN topic \href{http://ctan.org/topic/subdocs}{\textsf{subdocs}}
for further related packages.
The present package differs from the above solutions in that
a document structure constructed with the conventional |\include| mechanism
just needs two extra commands at the top of every file
such that all constituent files can be compiled individually.

%%%%%%%%%%%%%%%%%%%%%%%%%%%%%%%%%%%%%%%%%%%%%%%%%%%%%%%%%%%%%%%%%%%%%%%%%%%%%%%%
%\subsection{Feature Suggestions}
%
%The following is a list of features which may be useful for future
%versions of this package:
%%
%\begin{itemize}
%\item
%\ldots
%\end{itemize}

%%%%%%%%%%%%%%%%%%%%%%%%%%%%%%%%%%%%%%%%%%%%%%%%%%%%%%%%%%%%%%%%%%%%%%%%%%%%%%%%
\subsection{Revision History}

%%%%%%%%%%%%%%%%%%%%%%%%%%%%%%%%%%%%%%%%
\paragraph{v2.0:} 2018/12/30

\begin{itemize}
\item
immediate forward processing
\item
added |\childdocby| mechanism
\item
manual restructured
\end{itemize}

%%%%%%%%%%%%%%%%%%%%%%%%%%%%%%%%%%%%%%%%
\paragraph{v1.6:} 2018/01/17

\begin{itemize}
\item
application for development of include files
\item
corrections to manual
\end{itemize}

%%%%%%%%%%%%%%%%%%%%%%%%%%%%%%%%%%%%%%%%
\paragraph{v1.5:} 2017/05/21

\begin{itemize}
\item
more complete structuring introduced
\item
|\childdocof| introduced
\item
|\childdoc| renamed to |\childdocmain|
\item
|\childredirect| renamed to |\childdocforward| and |\childdocforwardprefix|
and functionality expanded
\end{itemize}

%%%%%%%%%%%%%%%%%%%%%%%%%%%%%%%%%%%%%%%%
\paragraph{v1.0:} 2017/04/27

\begin{itemize}
\item
manual and install package
\item
first version published on CTAN
\end{itemize}

%%%%%%%%%%%%%%%%%%%%%%%%%%%%%%%%%%%%%%%%
\paragraph{v0.6:} 2017/04/26

\begin{itemize}
\item
redirection mechanism added
\end{itemize}

%%%%%%%%%%%%%%%%%%%%%%%%%%%%%%%%%%%%%%%%
\paragraph{v0.5:} 2017/04/26

\begin{itemize}
\item
functionality in definition file
\end{itemize}


%%%%%%%%%%%%%%%%%%%%%%%%%%%%%%%%%%%%%%%%%%%%%%%%%%%%%%%%%%%%%%%%%%%%%%%%%%%%%%%%
%%%%%%%%%%%%%%%%%%%%%%%%%%%%%%%%%%%%%%%%%%%%%%%%%%%%%%%%%%%%%%%%%%%%%%%%%%%%%%%%
%%%%%%%%%%%%%%%%%%%%%%%%%%%%%%%%%%%%%%%%%%%%%%%%%%%%%%%%%%%%%%%%%%%%%%%%%%%%%%%%
\appendix

\settowidth\MacroIndent{\rmfamily\scriptsize 000\ }

 \DocInput{childdoc.dtx}

\end{document}
%</driver>
% \fi
%
% %%%%%%%%%%%%%%%%%%%%%%%%%%%%%%%%%%%%%%%%%%%%%%%%%%%%%%%%%%%%%%%%%%%%%%%%%%%%%%
% %%%%%%%%%%%%%%%%%%%%%%%%%%%%%%%%%%%%%%%%%%%%%%%%%%%%%%%%%%%%%%%%%%%%%%%%%%%%%%
% \section{Sample}
%\iffalse
%<*samplemain>
%\fi
%
% The following presents a sample document
% with two chapters, two parts, a title page,
% a compile flag as well as three forwarding files to set the flag.
% It consists of eight |.tex| files:
% \begin{center}
% \begin{tabular}{ll}
% |cdocsamp.tex|&main file\\
% |cdocsch1.tex|&include file for chapter 1\\
% |cdocsch2.tex|&include file for chapter 2\\
% |cdocspt3.tex|&include file for part 3\\
% |cdocspt4.tex|&include file for part 4\\
% |cdocsdrf.tex|&forwarding file for main file in draft mode\\
% |cdocsfi1.tex|&forwarding file for final version of chapter 1\\
% |cdocsfi2.tex|&forwarding file for final version of chapter 2\\
% \end{tabular}
% \end{center}
% Each of the eight files can be compiled directly by the \LaTeX{} compiler.
%
% %%%%%%%%%%%%%%%%%%%%%%%%%%%%%%%%%%%%%%
% \paragraph{Main File.}
%
% The main file is called |cdocsamp.tex|.
%
% Load the \textsf{childdoc} definitions and
% declare the filename for the main document:
%    \begin{macrocode}
\input{childdoc.def}
\childdocmain{}
%    \end{macrocode}

% Optional override for |\version| flag:
%    \begin{macrocode}
%%\ifchilddoc\else\providecommand{\version}{draft}\fi
%    \end{macrocode}

% Define the default values for the |\version| flag
% (|final| for the main file and |draft| for childs):
%    \begin{macrocode}
\ifchilddoc
\providecommand{\version}{draft}
\else
\providecommand{\version}{final}
\fi
%    \end{macrocode}

% Load the standard document class:
%    \begin{macrocode}
\documentclass[12pt]{article}
%    \end{macrocode}

% Start the document body:
%    \begin{macrocode}
\begin{document}
%    \end{macrocode}

% Declare a title page.
% Print title, part of document being processed and version flag:
%    \begin{macrocode}
\addtocounter{page}{-1}
\begin{center}
{\LARGE\bfseries{}childdoc example\par}
\vspace{1cm}
\ifchilddoc
\ifchilddocmanual part\else chapter\fi:
`\childdocname' of `\childdocjob'\par
\else
main document: `\childdocjob'\par
\fi
version: \version\par
\end{center}
\newpage
%    \end{macrocode}

% Manually include selected file,
% otherwise process as usual:
%    \begin{macrocode}
\ifchilddocmanual
\section*{part `\childdocname'}
\input{\childdocname}
\else
%    \end{macrocode}

% Include the two chapters:
%    \begin{macrocode}
\include{cdocsch1}
\include{cdocsch2}
%    \end{macrocode}

% Include the two parts unless only chapters should be displayed:
%    \begin{macrocode}
\ifchilddoc\else
\section{part three}
\input{cdocspt3}
\section{part four}
\input{cdocspt4}
\fi
%    \end{macrocode}

% Process as usual until here:
%    \begin{macrocode}
\fi
%    \end{macrocode}

% End of document body:
%    \begin{macrocode}
\end{document}
%    \end{macrocode}
%\iffalse
%</samplemain>
%\fi
%
% %%%%%%%%%%%%%%%%%%%%%%%%%%%%%%%%%%%%%%
% \paragraph{Chapter Include Files.}
%
% The include files are called |cdocsch1.tex| and |cdocsch2.tex|.
%
%\iffalse
%<*samplechap1|samplechap2>
%\fi

% Optional override for |\version| flag:
%    \begin{macrocode}
%%\providecommand{\version}{final}
%    \end{macrocode}

% Include the main document:
%    \begin{macrocode}
\input{childdoc.def}
\childdocof{cdocsamp}
%    \end{macrocode}

%\iffalse
%</samplechap1|samplechap2>
%\fi
%
%\iffalse
%<*samplechap1>
%\fi
% Some text for chapter 1:
%    \begin{macrocode}
\section{one}
some text in chapter one
%    \end{macrocode}

%\iffalse
%</samplechap1>
%\fi
% Some text for chapter 2:
%\iffalse
%<*samplechap2>
%\fi
%    \begin{macrocode}
\section{two}
more text in chapter two
%    \end{macrocode}

%\iffalse
%</samplechap2>
%\fi
%
% %%%%%%%%%%%%%%%%%%%%%%%%%%%%%%%%%%%%%%
% \paragraph{Part Include Files.}
%
% The include files are called |cdocspt3.tex| and |cdocspt4.tex|.
%
%\iffalse
%<*samplepart3|samplepart4>
%\fi

% Optional override for |\version| flag:
%    \begin{macrocode}
%%\providecommand{\version}{final}
%    \end{macrocode}

% Include the main document:
%    \begin{macrocode}
\input{childdoc.def}
\childdocby{cdocsamp}
%    \end{macrocode}

%\iffalse
%</samplepart3|samplepart4>
%\fi
%
%\iffalse
%<*samplepart3>
%\fi
% Some text for part 3:
%    \begin{macrocode}
some text in part three
%    \end{macrocode}

%\iffalse
%</samplepart3>
%\fi
% Some text for part 4:
%\iffalse
%<*samplepart4>
%\fi
%    \begin{macrocode}
more text in part four
%    \end{macrocode}

%\iffalse
%</samplepart4>
%\fi
%
% %%%%%%%%%%%%%%%%%%%%%%%%%%%%%%%%%%%%%%
% \paragraph{Forwarding for a Complete Draft.}
%
% The following forwarding file |cdocsdrf.tex|
% compiles the main document in draft mode:
%\iffalse
%<*sampledraft>
%\fi
%    \begin{macrocode}
\def\version{draft}
\input{childdoc.def}
\childdocforward{cdocsamp}
%    \end{macrocode}

%\iffalse
%</sampledraft>
%\fi
%
% %%%%%%%%%%%%%%%%%%%%%%%%%%%%%%%%%%%%%%
% \paragraph{Forwarding for Final Version of the Chapters.}
%
% The following forwarding files |cdocsfn1.tex| and |cdocsfn2.tex|
% (with identical content)
% compile the final versions of the child documents
% |cdocsch1.tex| and |cdocsch2.tex|, respectively:
%\iffalse
%<*samplefinal>
%\fi
%    \begin{macrocode}
\def\version{final}
\input{childdoc.def}
\childdocforwardprefix[cdocsamp]{cdocsfn}{cdocsch}
%    \end{macrocode}

%\iffalse
%</samplefinal>
%\fi
%
% %%%%%%%%%%%%%%%%%%%%%%%%%%%%%%%%%%%%%%
% \paragraph{Command Line Processing.}
%
% The following three command lines generate the output files
% |cdocscld|, |cdocscl1| and |cdocscl2|
% which should be identical to
% |cdocsdrf|, |cdocsch1| and |cdocsfn2|, respectively:
% \begin{center}
% \begin{tabular}{l}
% |latex -jobname cdocscld \|\\
% |  "\def\version{draft}\input{childdoc.def}\childdocforward{cdocsamp}"|\\
% |latex -jobname cdocscl1 \|\\
% |  "\input{childdoc.def}\childdocforward[cdocsamp]{cdocsch1}"|\\
% |latex -jobname cdocscl2 \|\\
% |  "\def\version{final}\input{childdoc.def}\childdocforward{cdocsch2}"|
% \end{tabular}
% \end{center}
% Note that the trailing backslash on each first line
% merely continues the input to the second line
% (for convenient cut ant paste).
% Furthermore, the command |latex| can be replaced by any
% of its alternative versions such as |pdflatex|.
%
% %%%%%%%%%%%%%%%%%%%%%%%%%%%%%%%%%%%%%%%%%%%%%%%%%%%%%%%%%%%%%%%%%%%%%%%%%%%%%%
% %%%%%%%%%%%%%%%%%%%%%%%%%%%%%%%%%%%%%%%%%%%%%%%%%%%%%%%%%%%%%%%%%%%%%%%%%%%%%%
% \section{Implementation}
%\iffalse
%<*package>
%\fi
%
% This section describes the definitions file |childdoc.def|.

% The definitions cannot be loaded using |\usepackage| or |\RequirePackage|
% which has a mechanism to prevent loading a style file more than once.
% When loading the definitions by means of |\input|
% multiple instances have to be prevented manually:
%\iffalse
%This code needs to be before the `\ProvidesFile' directive
%which is defined at the beginning of this file.
%Therefore it is also placed there and commented out here.
%</package>
%<*discard>
%\fi
%    \begin{macrocode}
\ifdefined\childdocmain\endinput\fi
%    \end{macrocode}
%\iffalse
%</discard>
%<*package>
%\fi
%
% \macro{\ifchilddoc}
% \macro{\ifchilddocmanual}
% The conditional |\ifchilddoc| tells whether a
% child (true) or main (false) document is being compiled.
% The conditional |\ifchilddocmanual| tells whether
% the |\includeonly| mechanism is used (false) or
% the selection of child files must be performed manually (true).
% The definitions initialise to false:
%    \begin{macrocode}
\newif\ifchilddoc
\newif\ifchilddocmanual
%    \end{macrocode}

% \macro{\childdocname}
% \macro{\childdocjob}
% The macro |\childdocname| stores the name of the main document
% to be compiled. The macro |\childdocjob| stores the name of
% the document on which the \LaTeX{} compiler was originally invoked.
% The content of |\jobname| cannot be compared
% to filenames specified in the source due to different catcodes.
% The following code rescans |\jobname|, stores the result
% in |\childdocname| and saves a copy in |\childdocjob|:
%    \begin{macrocode}
\edef\childdocname{\scantokens\expandafter{\jobname\noexpand}}
\let\childdocjob\childdocname
%    \end{macrocode}

% \macro{\childdocdisable}
% The macro |\childdocdisable| prevents the main file
% from being processed more than once.
% At this stage, the main document command |\childdocmain|
% is assumed to be called once again where it should do nothing.
% Any subsequent call to it should prevent
% a secondary processing of the main document
% It overwrites the forwarding commands
% |\childdocof| and |\childdocforward|
% with empty macros to prevent further inclusions of the main document:
%    \begin{macrocode}
\newcommand{\childdocdisable}
{
  \renewcommand{\childdocmain}[1]{\renewcommand{\childdocmain}[1]{\endinput}}
  \renewcommand{\childdocof}[1]{}
  \renewcommand{\childdocby}[2][]{}
  \renewcommand{\childdocforward}[2][]{}
  \renewcommand{\childdocdisable}{}
}
%    \end{macrocode}

% \macro{\childdocmain}
% The macro |\childdocmain| is to be called at the top of the main file
% with nothing or the main filename (without extension) as argument.
% First, it breaks loops.
% If the argument is not empty and does not match |\childdocname|
% (which is set by the first inclusion of |childdoc.def|),
% |\ifchilddoc| is set to true, |\includeonly| is applied to the child file
% and |\jobname| is set to the main file
% (for proper handling of |.aux| files):
%    \begin{macrocode}
\newcommand{\childdocmain}[1]
{
  \childdocdisable\childdocmain{}
  \if?#1?\else
    \begingroup
      \def\childdoctmp{#1}
      \ifx\childdoctmp\childdocname
        \def\childdoctmp{}
      \else
        \def\childdoctmp
        {
          \childdoctrue
          \includeonly{\childdocname}
          \def\childdocjob{#1}
          \def\jobname{#1}
        }
      \fi
      \expandafter
    \endgroup
    \childdoctmp
  \fi
}
%    \end{macrocode}

% \macro{\childdocof}
% The command |\childdocof| redirects
% compilation to the main file |#1|.
%    \begin{macrocode}
\newcommand{\childdocof}[1]
{
  \childdocdisable
  \childdoctrue
  \includeonly{\childdocname}
  \def\jobname{#1}
  \def\childdocjob{#1}
  \input{#1}
}
%    \end{macrocode}

% \macro{\childdocby}
% The command |\childdocby| ....
%    \begin{macrocode}
\newcommand{\childdocby}[2][]
{
  \childdocdisable
  \childdoctrue
  \childdocmanualtrue
  \if?#1?\else
    \def\jobname{#2}
  \fi
  \def\childdocjob{#2}
  \input{#2}
  \endinput
}
%    \end{macrocode}

% \macro{\childdocforward}
% The command |\childdocforward| redirects
% compilation to the main file or
% (if the optional argument is given) a child file.
% Parameters are set as if the main file
% or a child file starting with |\childdocof| was compiled.
% Then compilation is handed over to the main file:
%    \begin{macrocode}
\newcommand{\childdocforward}[2][]
{
  \begingroup
    \if?#1?
      \def\childdoctmp
      {
        \def\childdocname{#2}
        \def\childdocjob{#2}
        \def\jobname{#2}
        \input{#2}
        \endinput
      }
    \else
      \def\childdoctmp
      {
        \childdocdisable
        \def\childdocname{#2}
        \childdoctrue
        \includeonly{#2}
        \def\childdocjob{#1}
        \def\jobname{#1}
        \input{#1}
        \endinput
      }
    \fi
    \expandafter
  \endgroup
  \childdoctmp
}
%    \end{macrocode}

% \macro{\childdocforwardprefix}
% The command |\childdocforwardprefix| redirects
% compilation to the main or a child file by means of a pattern.
% The prefix |#1| in the current filename is replaced by |#2|
% and the suffix of the current filename is kept
% (it is assumed that the filename does not contain the substring `|~~~|'
% which is used as a delimiter).
% Compilation is handed over to the new file by |\childdocforward|:
%    \begin{macrocode}
\newcommand{\childdocforwardprefix}[3][]
{
  \begingroup
    \def\childdocextract #2##1~~~{\def\childdoctmp{\childdocforward[#1]{#3##1}}}
    \expandafter\childdocextract\childdocname~~~
    \expandafter
  \endgroup
  \childdoctmp
}
%    \end{macrocode}

% \macro{\childdoc}
% The deprecated macro |\childdoc| is a legacy version of |\childdocmain|:
%    \begin{macrocode}
\newcommand{\childdoc}{\childdocmain}
%    \end{macrocode}

% \macro{\childdocredirect}
% The deprecated macro |\childdocredirect| is a legacy version
% of |\childdocforward| and |\childdocforwardprefix|:
%    \begin{macrocode}
\newcommand{\childdocredirect}[2][]
{
  \begingroup
    \if?#1?
      \def\childdoctmp{\childdocforward{#2}}
    \else
      \def\childdoctmp{\childdocforwardprefix{#1}{#2}}
    \fi
    \expandafter
  \endgroup
  \childdoctmp
}
%    \end{macrocode}

%\iffalse
%</package>
%\fi
%
\endinput
|\\
|\childdocby{|\textit{main}|}|\\
\end{tabular}
\end{center}
%
The directive |\childdocby| is similar to |\childdocof|
described in \secref{sec:include},
but the subsequent selection of content must be done manually.
To that end, both |\ifchilddoc| and |\ifchilddocmanual|
will be true upon processing of a part,
and the name of the part is stored in |\childdocname|.
Note that |\jobname| will be set to the filename of the current part
so that each part receives an individual |.aux| file
that does not interfere with the |.aux| file(s) of the main document.
This behaviour can be altered by the alternative form
|\childdocby[*]{|\textit{main}|}| (with a non-empty optional argument)
which uses the |.aux| file of the main document
by setting |\jobname| to \textit{main}.

%%%%%%%%%%%%%%%%%%%%%%%%%%%%%%%%%%%%%%%%%%%%%%%%%%%%%%%%%%%%%%%%%%%%%%%%%%%%%%%%
\subsection{Driver Development}
\label{sec:driver}

The \textsf{childdoc} mechanism can also be use for the development
of definition files such as \LaTeX{} styles or classes.
This case differs from the above setup with multiple parts
included by |\include| in that no |\includeonly| should be invoked.
This can be achieved by starting the include file
(before |\ProvidesPackage|) with:
%
\begin{center}
\begin{tabular}{l}
|% \iffalse
%
% childdoc.dtx Copyright (C) 2017-2018 Niklas Beisert
%
% This work may be distributed and/or modified under the
% conditions of the LaTeX Project Public License, either version 1.3
% of this license or (at your option) any later version.
% The latest version of this license is in
%   http://www.latex-project.org/lppl.txt
% and version 1.3 or later is part of all distributions of LaTeX
% version 2005/12/01 or later.
%
% This work has the LPPL maintenance status `maintained'.
%
% The Current Maintainer of this work is Niklas Beisert.
%
% This work consists of the files childdoc.dtx and childdoc.ins
% and the derived files childdoc.def and cdocsamp.tex with
% cdocsch1.tex, cdocsch2.tex, cdocsdrf.tex, cdocsfn1.tex, cdocsfn2.tex.
%
%<package>\ifdefined\childdocmain\endinput\fi
%<package>\ProvidesFile{childdoc.def}[2018/12/30 v2.0 child document driver]
%<samplemain>\ProvidesFile{cdocsamp.tex}[2018/12/30 v2.0 sample for childdoc]
%<*driver>
%\ProvidesFile{childdoc.drv}[2018/12/30 v2.0 childdoc reference manual file]
\PassOptionsToClass{10pt,a4paper}{article}
\documentclass{ltxdoc}

\usepackage[margin=35mm]{geometry}
\usepackage{hyperref}
\usepackage{hyperxmp}
\usepackage[usenames]{color}

\hypersetup{colorlinks=true}
\hypersetup{pdfstartview=FitH}
\hypersetup{pdfpagemode=UseNone}
\hypersetup{pdfsource={}}
\hypersetup{pdflang={en-UK}}
\hypersetup{pdfcopyright={Copyright 2017-2018 Niklas Beisert.
  This work may be distributed and/or modified under the
  conditions of the LaTeX Project Public License, either version 1.3
  of this license or (at your option) any later version.}}
\hypersetup{pdflicenseurl={http://www.latex-project.org/lppl.txt}}
\hypersetup{pdfcontactaddress={ETH Zurich, ITP, HIT K,
  Wolfgang-Pauli-Strasse 27}}
\hypersetup{pdfcontactpostcode={8093}}
\hypersetup{pdfcontactcity={Zurich}}
\hypersetup{pdfcontactcountry={Switzerland}}
\hypersetup{pdfcontactemail={nbeisert@itp.phys.ethz.ch}}
\hypersetup{pdfcontacturl={http://people.phys.ethz.ch/\xmptilde nbeisert/}}

\newcommand{\secref}[1]{\hyperref[#1]{section \ref*{#1}}}

\parskip1ex
\parindent0pt
\let\olditemize\itemize
\def\itemize{\olditemize\parskip0pt}

\begin{document}

\title{The \textsf{childdoc} Package}
\hypersetup{pdftitle={The childdoc Package}}
\author{Niklas Beisert\\[2ex]
  Institut f\"ur Theoretische Physik\\
  Eidgen\"ossische Technische Hochschule Z\"urich\\
  Wolfgang-Pauli-Strasse 27, 8093 Z\"urich, Switzerland\\[1ex]
  \href{mailto:nbeisert@itp.phys.ethz.ch}
  {\texttt{nbeisert@itp.phys.ethz.ch}}}
\hypersetup{pdfauthor={Niklas Beisert}}
\hypersetup{pdfsubject={Manual for the LaTeX2e Package childdoc}}
\date{30 December 2018, \textsf{v2.0}}
\maketitle

\begin{abstract}\noindent
\textsf{childdoc} is a \LaTeXe{} package
that enables the direct compilation
of document sections included by |\include|
to individual files.
\end{abstract}

\begingroup
\parskip0ex
\tableofcontents
\endgroup

%%%%%%%%%%%%%%%%%%%%%%%%%%%%%%%%%%%%%%%%%%%%%%%%%%%%%%%%%%%%%%%%%%%%%%%%%%%%%%%%
%%%%%%%%%%%%%%%%%%%%%%%%%%%%%%%%%%%%%%%%%%%%%%%%%%%%%%%%%%%%%%%%%%%%%%%%%%%%%%%%
\section{Introduction}

\LaTeX{} provides a mechanism to structure a large document (such as a book)
into a main file and several child files (containing the chapters)
using the |\include| command.
This mechanism is beneficial for documents
which span hundreds of pages in order to
make the source file(s) more manageable.
Moreover, compilation can be restricted to
selected child files by means of the |\includeonly| command.
The latter feature can be used to reduce the compilation time while editing
(this was significantly more useful in the earlier days of \LaTeX{})
or to generate a smaller document which is easier to navigate.
Another application of |\includeonly| is to generate
documents consisting of selected parts of the complete document.

However, there are a few drawbacks of the plain |\include| mechanism:
\begin{itemize}
\item
The child files cannot be compiled on their own,
they can only be compiled via the main file.
A naive editing environment
(such as a text editor with an option
to have the current file processed by \LaTeX)
may require one to switch to the main file before compiling;
attempting to compile the child file produces errors.
\item
The main file must be modified (each time)
to adjust the |\includeonly| command
to the present needs. This easily leaves the main file in a messy state.
\item
The generated document will always carry the filename
of the main document. This is inconvenient if
several child files are to be compiled and
to be kept for distribution.
\end{itemize}

The present package provides a simple interface
to make child files individually compilable by \LaTeX{}.
Compiling a child file then has the same effect as compiling
the main file with an |\includeonly| command
to select the appropriate child.
Moreover the generated document will carry the name of the child
rather than the main file.
This resolves all three above issues.

This feature is meant to make the editing of books,
thesis documents and lecture notes somewhat more convenient.
However, the package can also be used efficiently for
composing a series of documents (such as exercise sheets)
which are typically distributed individually.
It then assists the author in generating the individual documents
(potentially in different versions)
as well as a document containing the collected series.
Another application is in developing style files
or other kinds of included material
where compilation of the style file could redirect
to a sample or test file.

%%%%%%%%%%%%%%%%%%%%%%%%%%%%%%%%%%%%%%%%%%%%%%%%%%%%%%%%%%%%%%%%%%%%%%%%%%%%%%%%
%%%%%%%%%%%%%%%%%%%%%%%%%%%%%%%%%%%%%%%%%%%%%%%%%%%%%%%%%%%%%%%%%%%%%%%%%%%%%%%%
\section{Usage}

First of all, the package \textsf{childdoc} is \emph{not} a standard
\LaTeXe{} |.sty| style file! Therefore it needs to be invoked in
a non-standard way.

%%%%%%%%%%%%%%%%%%%%%%%%%%%%%%%%%%%%%%%%%%%%%%%%%%%%%%%%%%%%%%%%%%%%%%%%%%%%%%%%
\subsection{Included Files}
\label{sec:include}

%%%%%%%%%%%%%%%%%%%%%%%%%%%%%%%%%%%%%%%%
\DescribeMacro{\childdocmain}
To use the package, add the commands
\begin{center}
\begin{tabular}{l}
|\input{childdoc.def}|\\
|\childdocmain{}|\\
\end{tabular}
\end{center}
at the very top of the main \LaTeX{} file,
in particular \emph{before} the |\documentclass| statement!
The argument of |\childdocmain| should be left empty
(but it must be present).

%%%%%%%%%%%%%%%%%%%%%%%%%%%%%%%%%%%%%%%%
\DescribeMacro{\childdocof}
Furthermore, add the commands
\begin{center}
\begin{tabular}{l}
|\input{childdoc.def}|\\
|\childdocof{|\textit{main}|}|\\
\end{tabular}
\end{center}
at the top of every child file \textit{child}
which is included by |\include{|\textit{child}|}|
from within the main file
(or at least for those files to be compiled individually).
The argument \textit{main} must be the filename of the main file.

There are a couple of
considerations in setting up the main and child documents:

%%%%%%%%%%%%%%%%%%%%%%%%%%%%%%%%%%%%%%%%
\paragraph{Restrictions.}

Please note the following restrictions:
\begin{itemize}
\item
|\childdocmain| must be called with one argument \textit{main}
to ensure compatibility with earlier version of the package.
It must either be empty (|\childdocmain{}|)
or precisely match the filename of the main file in which it is specified.
See \secref{sec:detection} for further information.
\item
The filename \textit{main} must be specified without the |.tex| extension.
\item
The filename \textit{main} is case sensitive
(even in case-insensitive file systems)
due to internal string comparison.
\item
The argument \textit{main} should be fully expanded, it cannot be a macro.
\item
Subdirectories and special characters should be avoided in filenames.
\item
The command |\childdocmain{|\textit{main}|}| must be followed by a whitespace.
It should not be followed immediately by another command
or by a comment mark `|%|'.
This is because the \TeX{} parser reads the token immediately following
the argument of |\childdocmain| and puts it
at the beginning of every child section;
however, a white\-space is ignored.
\end{itemize}

%%%%%%%%%%%%%%%%%%%%%%%%%%%%%%%%%%%%%%%%
\paragraph{Content of Main File.}

It is advisable to place all content in the child files included by |\include|.
Any output contained in the main file will appear in all child documents
unless suppressed manually;
it cannot be suppressed automatically by the |\includeonly| directive
and thus should normally be avoided.
A method to include some content in the main file
by means of conditional processing is described in \secref{sec:conditional}.

%%%%%%%%%%%%%%%%%%%%%%%%%%%%%%%%%%%%%%%%
\paragraph{Page Numbering.}

When only a part of the document is compiled,
the appropriate numbering of pages
(as well as other status parameters)
is determined from the |.aux| files.
The latter contain information from previous passes.
However this information needs to propagate through
all intermediate child documents.
Therefore the page numbering in child documents may well
be inconsistent until the complete document is compiled at least once.

A useful (if unconventional) way to always ensure a consistent
page numbering is to restart the numbering in each child document
and denote the pages by `\textit{child}|.|\textit{page}'
where \textit{child} represents the chapter/section number of the child file.
This can be achieved by the command
|\numberwithin{page}{|\textit{child}|}|
of the \textsf{amsmath} package
where \textit{child} can be |chapter| or |section|
depending on the chosen structuring.
Alternatively, one can modify the macro |\thepage| appropriately
and reset the counter |page| at the start of each child file.

%%%%%%%%%%%%%%%%%%%%%%%%%%%%%%%%%%%%%%%%%%%%%%%%%%%%%%%%%%%%%%%%%%%%%%%%%%%%%%%%
\subsection{Conditional Processing}
\label{sec:conditional}

The package provides a mechanism to compile different versions
of a document. To customise the versions further some conditional processing
can come in handy to distinguish which version is being compiled.
The package provides two macros to describe the compilation context:

%%%%%%%%%%%%%%%%%%%%%%%%%%%%%%%%%%%%%%%%
\DescribeMacro{\ifchilddoc}
The conditional |\ifchilddoc| distinguishes between the compilation of
child documents and the main document:
%
\begin{center}
|\ifchilddoc |\textit{child-code}| |[|\||else |\textit{main-code}]| \||fi|
\end{center}

%%%%%%%%%%%%%%%%%%%%%%%%%%%%%%%%%%%%%%%%
\DescribeMacro{\childdocname}
\DescribeMacro{\childdocjob}
The macro |\childdocname| contains the filename (without extension)
of the main or child file being processed.
Note that |\childdocjob| will always contain the name of the main file.

%%%%%%%%%%%%%%%%%%%%%%%%%%%%%%%%%%%%%%%%
\paragraph{Title Page.}

Conditional processing can be used to include a title or banner page
in the main document when proper precautions are taken.
Importantly, the code in the main file should ensure that the page counter
(as well as other status parameters which are stored in the |.aux| files)
takes the same value after the conditional processing.
Otherwise the page numbers may take divergent values
depending on which part is compiled.

For example, a title page could be declared by:
%
\begin{center}
\begin{tabular}{l}
|\ifchilddoc\||else|\\
|\addtocounter{page}{-1}|\\
\textit{code for title page}\\
|\newpage|\\
|\||fi|
\end{tabular}
\end{center}
%
A banner page for the child documents can be generated by:
%
\begin{center}
\begin{tabular}{l}
|\ifchilddoc|\\
|\addtocounter{page}{-1}|\\
\textit{code for banner page}\\
|\newpage|\\
|\||fi|
\end{tabular}
\end{center}
%
Here one could write a message such as:
\begin{center}
|This is the part \childdocname{} of \childdocjob{}.|
\end{center}

%%%%%%%%%%%%%%%%%%%%%%%%%%%%%%%%%%%%%%%%%%%%%%%%%%%%%%%%%%%%%%%%%%%%%%%%%%%%%%%%
\subsection{Flags}
\label{sec:flags}

The package makes it easy to generate different versions
of the main or child documents.
To this end compilation flags can be defined
and assigned different default values.
They will be particularly useful in conjunction
with the forwarding mechanism described in \secref{sec:forward}.

For example, it may be useful to have a flag |\version|
which can be set to |draft| or |final|.
The document source will contain some conditional code
depending on the value of |\version|.
Suppose further, the flag should default to |final| for the main file
and to |draft| for child files
which is a natural assignment for editing the document.
This is achieved by placing the following code
in the preamble of the main document
(below the |\childdocmain| directive):
%
\begin{center}
\begin{tabular}{l}
|\ifchilddoc|\\
|\providecommand{\version}{draft}|\\
|\||else|\\
|\providecommand{\version}{final}|\\
|\||fi|
\end{tabular}
\end{center}
%
The definition by |\providecommand| makes sure
that previous definitions are not overwritten.
Further statements |\providecommand{\version}{...}|
can thus be added before the above code to override it.

For the main file, one might add a line
(between |\childdocmain| and the above block)
%
\begin{center}
|%\ifchilddoc\||else\providecommand{\version}{draft}\||fi|
\end{center}
%
which can be uncommented to produce a draft version.
Likewise one can add a line to the very top of a child file
(above the |\childdocof{|\textit{main}|}| directive)
%
\begin{center}
|%\providecommand{\version}{final}|
\end{center}
%
which can be uncommented to produce the final version of this child document.

%%%%%%%%%%%%%%%%%%%%%%%%%%%%%%%%%%%%%%%%%%%%%%%%%%%%%%%%%%%%%%%%%%%%%%%%%%%%%%%%
\subsection{Forwarding}
\label{sec:forward}

Different versions of the main or child documents
using compilation flags as described in \secref{sec:flags}
can be (permanently) stored in different files
for convenient compilation, viewing and distribution.
To this end, the package defines a command
to pass on compilation to a different file:

%%%%%%%%%%%%%%%%%%%%%%%%%%%%%%%%%%%%%%%%
\DescribeMacro{\childdocforward}
The command |\childdocforward| redirects processing to
another source file:
%
\begin{center}
\begin{tabular}{l}
|\input{childdoc.def}|\\
|\childdocforward[|\textit{main}|]{|\textit{dest}|}|\\
\end{tabular}
\end{center}
%
The argument \textit{dest} is the destination file
(without extension).
It should be the main file or one of the child files.
Note that further \textsf{childdoc} directives
such as |\childdocof| and |\childdocforward|
in the indicated file will be processed in this form.
The optional argument \textit{main}
passes on directly to the main file \textit{main}
while pretending to compile the child \textit{dest}.
This form behaves as if \textit{dest}
issues |\childdocof{|\textit{main}|}| right away,
and no further \textsf{childdoc} directives will be processed.

%%%%%%%%%%%%%%%%%%%%%%%%%%%%%%%%%%%%%%%%
\DescribeMacro{\...prefix}
In the alternative form |\childdocforwardprefix|,
%
\begin{center}
\begin{tabular}{l}
|\input{childdoc.def}|\\
|\childdocforwardprefix[|\textit{main}|]{|\textit{prefix}|}{|\textit{dest}|}|
\end{tabular}
\end{center}
%
the destination file is determined by a pattern
depending on the current file:
To make this work, the current file must be called
`{\textit{prefix}\hspace{0.2em}\textit{suffix}}'
with \textit{prefix} matching precisely the argument.
Processing is then passed on to the file
`{\textit{dest}\hspace{0.2em}\textit{suffix}}'.
Surely, the same effect is achieved by
directly specifying the
argument `{\textit{dest}\hspace{0.2em}\textit{suffix}}'
in the first form.
However, that requires to set up a different file
for each child. With the alternative form of the command
all these files can have exactly the same content
which simplifies setting them up and maintaining them.

For example, the following file |draft.tex|
with a compilation flag |\version| as described in \secref{sec:flags}
compiles the main document as a draft:
%
\begin{center}
\begin{tabular}{l}
|\def\version{draft}|\\
|\input{childdoc.def}|\\
|\childdocforward{|\textit{main}|}|
\end{tabular}
\end{center}
%
Likewise, the following files |final|\textit{nn}|.tex|
compile the final version of the child document
|child|\textit{nn}|.tex|:
%
\begin{center}
\begin{tabular}{l}
|\def\version{final}|\\
|\input{childdoc.def}|\\
|\childdocforwardprefix{final}{child}|
\end{tabular}
\end{center}
%

Note that when several versions of a main file and/or of each child file
are to be generated, it may be convenient to set up a |Makefile| or
shell script to automatise the process.

%%%%%%%%%%%%%%%%%%%%%%%%%%%%%%%%%%%%%%%%%%%%%%%%%%%%%%%%%%%%%%%%%%%%%%%%%%%%%%%%
\subsection{Command Line Processing}
\label{sec:commandline}

The effect of redirection files can also be achieved by invoking
the \LaTeX{} compiler with a more elaborate command line.
Most conveniently this should be done as part
of a shell script or a |Makefile|.

When using \textsf{childdoc} in the main file, the following
command lines effectively perform a redirection
(note that depending on the shell being used,
backslashes may have to be doubled: `|\|' $\to$ `|\\|'):
%
\begin{center}
|... -jobname "|\textit{target}|" |\\|"|[\textit{flags}]%
|\input{childdoc.def}\childdocforward[|\textit{main}|]{|\textit{dest}|}"|
\end{center}
%
Here \textit{target} is the name of the output file,
\textit{main} is the name of the main file
and \textit{dest} is the name of the main or child file to be processed
(all filenames without extensions).
The optional argument \textit{main} can be omitted
if \textit{main} matches \textit{dest}.
Optionally, compilation \textit{flags} can be defined via |\def| commands.
This command line makes the \TeX{} engine believe
it is compiling the file \textit{target}
whose content is specified as the latter parameter.
The provided code then forwards the processing to
\textit{main} or \textit{dest} as described in \secref{sec:forward}.

%%%%%%%%%%%%%%%%%%%%%%%%%%%%%%%%%%%%%%%%%%%%%%%%%%%%%%%%%%%%%%%%%%%%%%%%%%%%%%%%
\subsection{Include by Input}
\label{sec:input}

Including child documents by |\include| has some restrictions by design.
Most notably, the content of a child document always occupies
its own set of pages; pages cannot be shared between child documents.
Usually, this behaviour makes perfect sense
because each child document contain an essential part of the document.
However, in some situations it may be desirable to compose
a document from a collection of parts
without having mandatory page breaks between then.
For this case, the package
provides a mechanism to include parts
by |\input| which can also be processed individually.
However, by construction this mechanism
requires manual handling of the content to be output.

%%%%%%%%%%%%%%%%%%%%%%%%%%%%%%%%%%%%%%%%
\DescribeMacro{\ifchilddocmanual}
The main file should be prepared as usual, see \secref{sec:include}.
However, the document body must make a distinction
between processing of an individual part and of the main document, e.g.:
%
\begin{center}
\begin{tabular}{l}
|\ifchilddocmanual|\\
|\input{\childdocname}|\\
|\||else|\\
\textit{document body with }|\input{|\textit{part}|}|\\
|\||fi|
\end{tabular}
\end{center}
%
The conditional |\ifchilddocmanual| is true whenever
a part to be included by |\input| is being compiled,
and the name of the part is stored in |\childdocname|.

%%%%%%%%%%%%%%%%%%%%%%%%%%%%%%%%%%%%%%%%
\DescribeMacro{\childdocby}
Each part to be included by |\input| should start with:
%
\begin{center}
\begin{tabular}{l}
|\input{childdoc.def}|\\
|\childdocby{|\textit{main}|}|\\
\end{tabular}
\end{center}
%
The directive |\childdocby| is similar to |\childdocof|
described in \secref{sec:include},
but the subsequent selection of content must be done manually.
To that end, both |\ifchilddoc| and |\ifchilddocmanual|
will be true upon processing of a part,
and the name of the part is stored in |\childdocname|.
Note that |\jobname| will be set to the filename of the current part
so that each part receives an individual |.aux| file
that does not interfere with the |.aux| file(s) of the main document.
This behaviour can be altered by the alternative form
|\childdocby[*]{|\textit{main}|}| (with a non-empty optional argument)
which uses the |.aux| file of the main document
by setting |\jobname| to \textit{main}.

%%%%%%%%%%%%%%%%%%%%%%%%%%%%%%%%%%%%%%%%%%%%%%%%%%%%%%%%%%%%%%%%%%%%%%%%%%%%%%%%
\subsection{Driver Development}
\label{sec:driver}

The \textsf{childdoc} mechanism can also be use for the development
of definition files such as \LaTeX{} styles or classes.
This case differs from the above setup with multiple parts
included by |\include| in that no |\includeonly| should be invoked.
This can be achieved by starting the include file
(before |\ProvidesPackage|) with:
%
\begin{center}
\begin{tabular}{l}
|\input{childdoc.def}|\\
|\childdocforward{|\textit{main}|}|\\
\end{tabular}
\end{center}
%
or alternatively with:
%
\begin{center}
\begin{tabular}{l}
|\input{childdoc.def}|\\
|\childdocby{|\textit{main}|}|\\
\end{tabular}
\end{center}
%
Both forms have slightly different effects as described above.
The main file is prepared as usual, see \secref{sec:include}.

%%%%%%%%%%%%%%%%%%%%%%%%%%%%%%%%%%%%%%%%%%%%%%%%%%%%%%%%%%%%%%%%%%%%%%%%%%%%%%%%
\subsection{Legacy Detection}
\label{sec:detection}

The directive |\childdocmain| in the main file can detect
whether the complete document or merely a child is to be compiled
even without using the directive |\childdocof|.
This method is deprecated because it is less robust
and there is no compelling reason to use it;
it is merely provided for backward compatibility
and it may be removed in future versions.

If the detection mechanism is to be used,
it is mandatory to correctly specify
the filename of the main file as the argument of |\childdocmain|:
%
\begin{center}
\begin{tabular}{l}
|\input{childdoc.def}|\\
|\childdocmain{|\textit{main}|}|\\
\end{tabular}
\end{center}
%
If |\jobname| does not match the argument \textit{main} of |\childdocmain|,
it is assumed that |\jobname| points to the child file to be compiled.
When using |\childdocmain| with the main file specified as argument,
it suffices to start a child file
with just |\input{|\textit{main}|}|
without loading of the package and using |\childdocof|.
If instead all processing is done
with the appropriate \textsf{childdoc} directives,
the argument of \textit{main} of |\childdocmain| can be empty.

An alternative version of the command line processing described
in \secref{sec:commandline} using the detection mechanism reads:
%
\begin{center}
|... -jobname "|\textit{target}|" "|[\textit{flags}]%
[|\def\jobname{|\textit{dest}|}|]|\input{|\textit{main}|}"|
\end{center}

%%%%%%%%%%%%%%%%%%%%%%%%%%%%%%%%%%%%%%%%%%%%%%%%%%%%%%%%%%%%%%%%%%%%%%%%%%%%%%%%
\subsection{Manual Code}
\label{sec:manual}

In case one cannot be certain whether the definitions file |childdoc.def|
is installed on the target \TeX{} distribution
and one prefers not to ship it,
it is conceivable to paste a few relevant commands into the sources.

To that end, drop all statements |\input{childdoc.def}|
and perform the replacements as outlined below.
Instead of |\childdocmain{|\textit{main}|}| add the following code
to the top of the main file:
%
\begin{center}
\begin{tabular}{l}
|\||ifdefined\childdocname\endinput\||fi\newif\ifchilddoc|\\
|\edef\childdocname{\scantokens\expandafter{\jobname\noexpand}}|\\
|\def\childdocmain{|\textit{main}|}\||ifx\childdocmain\childdocname\||else|\\
|\childdoctrue\includeonly{\childdocname}\let\jobname\childdocmain\||fi|\\
\end{tabular}
\end{center}
%
Instead of |\childdocof{|\textit{main}|}| just include the main file
at the top of each child file:
%
\begin{center}
|\input{|\textit{main}|}|
\end{center}
%
A simple redirection |\childdocforward{|\textit{dest}|}| is achieved by:
%
\begin{center}
|\def\jobname{|\textit{dest}|}\input{\jobname}|
\end{center}
%
The redirection with prefix
|\childdocforwardprefix[|\textit{prefix}|]{|\textit{dest}|}|
is accomplished by:
%
\begin{center}
\begin{tabular}{l}
|{\edef\jobname{\scantokens\expandafter{\jobname\noexpand}}|\\
|\def\redirectjob |\textit{prefix}|#1~~~{\gdef\jobname{|\textit{dest}|#1}}|\\
|\expandafter\redirectjob\jobname~~~}\input{\jobname}|
\end{tabular}
\end{center}

In an alternative approach,
child documents can be compiled by a specific command line
without additional code or specific definitions:
%
\begin{center}
|... -jobname "|\textit{target}|" "|[\textit{flags}]%
|\includeonly{|\textit{dest}|}\input{|\textit{main}|}"|
\end{center}
%

%%%%%%%%%%%%%%%%%%%%%%%%%%%%%%%%%%%%%%%%%%%%%%%%%%%%%%%%%%%%%%%%%%%%%%%%%%%%%%%%
%%%%%%%%%%%%%%%%%%%%%%%%%%%%%%%%%%%%%%%%%%%%%%%%%%%%%%%%%%%%%%%%%%%%%%%%%%%%%%%%
\section{Information}

%%%%%%%%%%%%%%%%%%%%%%%%%%%%%%%%%%%%%%%%%%%%%%%%%%%%%%%%%%%%%%%%%%%%%%%%%%%%%%%%
\subsection{Copyright}

Copyright \copyright{} 2017--2018 Niklas Beisert

This work may be distributed and/or modified under the
conditions of the \LaTeX{} Project Public License, either version 1.3
of this license or (at your option) any later version.
The latest version of this license is in
  \url{http://www.latex-project.org/lppl.txt}
and version 1.3 or later is part of all distributions of \LaTeX{}
version 2005/12/01 or later.

This work has the LPPL maintenance status `maintained'.

The Current Maintainer of this work is Niklas Beisert.

This work consists of the files |README.txt|, |childdoc.ins| and |childdoc.dtx|
as well as the derived files |childdoc.def|, |cdocsamp.tex|
with |cdocsch1.tex|, |cdocsch2.tex|, |cdocspt3.tex|, |cdocspt4.tex|,
|cdocsdrf.tex|, |cdocsfn1.tex|, |cdocsfn2.tex|
as well as |childdoc.pdf|.

%%%%%%%%%%%%%%%%%%%%%%%%%%%%%%%%%%%%%%%%%%%%%%%%%%%%%%%%%%%%%%%%%%%%%%%%%%%%%%%%
\subsection{Files and Installation}

The package consists of the files:
%
\begin{center}
\begin{tabular}{ll}
    |README.txt|   & readme file \\
    |childdoc.ins| & installation file \\
    |childdoc.dtx| & source file \\
    |childdoc.def| & definition file \\
    |cdocsamp.tex| & sample main file \\
    |cdocsch1.tex| & sample include file \\
    |cdocsch2.tex| & sample include file \\
    |cdocspt3.tex| & sample part file \\
    |cdocspt4.tex| & sample part file \\
    |cdocsdrf.tex| & sample redirection file \\
    |cdocsfn1.tex| & sample redirection file \\
    |cdocsfn2.tex| & sample redirection file \\
    |childdoc.pdf| & manual
\end{tabular}
\end{center}
%
The distribution consists of the files
|README.txt|, |childdoc.ins| and |childdoc.dtx|.
%
\begin{itemize}
\item
Run (pdf)\LaTeX{} on |childdoc.dtx|
to compile the manual |childdoc.pdf| (this file).
\item
Run \LaTeX{} on |childdoc.ins| to create the definitions file |childdoc.def|
and the sample |cdocsamp.tex| with include files
|cdocsch1.tex|, |cdocsch2.tex|, |cdocspt3.tex|, |cdocspt4.tex|,
|cdocsdrf.tex|, |cdocsfn1.tex|, |cdocsfn2.tex|.
Then copy the file |childdoc.def| to an appropriate directory of your \LaTeX{}
distribution, e.g.\ \textit{texmf-root}|/tex/latex/childdoc|.
\end{itemize}

%%%%%%%%%%%%%%%%%%%%%%%%%%%%%%%%%%%%%%%%%%%%%%%%%%%%%%%%%%%%%%%%%%%%%%%%%%%%%%%%
\subsection{Related CTAN Packages}

There are several other packages which offer a similar functionality:
%
\begin{itemize}
\item
The packages
\href{http://ctan.org/pkg/docmute}{\textsf{docmute}},
\href{http://ctan.org/pkg/includex}{\textsf{includex}} and
\href{http://ctan.org/pkg/standalone}{\textsf{standalone}}
provide commands to include only the document body of
a child file thus allowing both files to be compiled individually.
\item
The packages \href{http://ctan.org/pkg/subdocs}{\textsf{subdocs}}
and \href{http://ctan.org/pkg/subfiles}{\textsf{subfiles}}
provide structures in which the main and child documents can be
encapsulated and allowing them to be compiled individually.
The inclusion mechanism is different from the conventional |\include|.
\item
The package \href{http://ctan.org/pkg/combine}{\textsf{combine}}
is an elaborate solution to combine several documents into one.
\end{itemize}
%
See also the CTAN topic \href{http://ctan.org/topic/subdocs}{\textsf{subdocs}}
for further related packages.
The present package differs from the above solutions in that
a document structure constructed with the conventional |\include| mechanism
just needs two extra commands at the top of every file
such that all constituent files can be compiled individually.

%%%%%%%%%%%%%%%%%%%%%%%%%%%%%%%%%%%%%%%%%%%%%%%%%%%%%%%%%%%%%%%%%%%%%%%%%%%%%%%%
%\subsection{Feature Suggestions}
%
%The following is a list of features which may be useful for future
%versions of this package:
%%
%\begin{itemize}
%\item
%\ldots
%\end{itemize}

%%%%%%%%%%%%%%%%%%%%%%%%%%%%%%%%%%%%%%%%%%%%%%%%%%%%%%%%%%%%%%%%%%%%%%%%%%%%%%%%
\subsection{Revision History}

%%%%%%%%%%%%%%%%%%%%%%%%%%%%%%%%%%%%%%%%
\paragraph{v2.0:} 2018/12/30

\begin{itemize}
\item
immediate forward processing
\item
added |\childdocby| mechanism
\item
manual restructured
\end{itemize}

%%%%%%%%%%%%%%%%%%%%%%%%%%%%%%%%%%%%%%%%
\paragraph{v1.6:} 2018/01/17

\begin{itemize}
\item
application for development of include files
\item
corrections to manual
\end{itemize}

%%%%%%%%%%%%%%%%%%%%%%%%%%%%%%%%%%%%%%%%
\paragraph{v1.5:} 2017/05/21

\begin{itemize}
\item
more complete structuring introduced
\item
|\childdocof| introduced
\item
|\childdoc| renamed to |\childdocmain|
\item
|\childredirect| renamed to |\childdocforward| and |\childdocforwardprefix|
and functionality expanded
\end{itemize}

%%%%%%%%%%%%%%%%%%%%%%%%%%%%%%%%%%%%%%%%
\paragraph{v1.0:} 2017/04/27

\begin{itemize}
\item
manual and install package
\item
first version published on CTAN
\end{itemize}

%%%%%%%%%%%%%%%%%%%%%%%%%%%%%%%%%%%%%%%%
\paragraph{v0.6:} 2017/04/26

\begin{itemize}
\item
redirection mechanism added
\end{itemize}

%%%%%%%%%%%%%%%%%%%%%%%%%%%%%%%%%%%%%%%%
\paragraph{v0.5:} 2017/04/26

\begin{itemize}
\item
functionality in definition file
\end{itemize}


%%%%%%%%%%%%%%%%%%%%%%%%%%%%%%%%%%%%%%%%%%%%%%%%%%%%%%%%%%%%%%%%%%%%%%%%%%%%%%%%
%%%%%%%%%%%%%%%%%%%%%%%%%%%%%%%%%%%%%%%%%%%%%%%%%%%%%%%%%%%%%%%%%%%%%%%%%%%%%%%%
%%%%%%%%%%%%%%%%%%%%%%%%%%%%%%%%%%%%%%%%%%%%%%%%%%%%%%%%%%%%%%%%%%%%%%%%%%%%%%%%
\appendix

\settowidth\MacroIndent{\rmfamily\scriptsize 000\ }

 \DocInput{childdoc.dtx}

\end{document}
%</driver>
% \fi
%
% %%%%%%%%%%%%%%%%%%%%%%%%%%%%%%%%%%%%%%%%%%%%%%%%%%%%%%%%%%%%%%%%%%%%%%%%%%%%%%
% %%%%%%%%%%%%%%%%%%%%%%%%%%%%%%%%%%%%%%%%%%%%%%%%%%%%%%%%%%%%%%%%%%%%%%%%%%%%%%
% \section{Sample}
%\iffalse
%<*samplemain>
%\fi
%
% The following presents a sample document
% with two chapters, two parts, a title page,
% a compile flag as well as three forwarding files to set the flag.
% It consists of eight |.tex| files:
% \begin{center}
% \begin{tabular}{ll}
% |cdocsamp.tex|&main file\\
% |cdocsch1.tex|&include file for chapter 1\\
% |cdocsch2.tex|&include file for chapter 2\\
% |cdocspt3.tex|&include file for part 3\\
% |cdocspt4.tex|&include file for part 4\\
% |cdocsdrf.tex|&forwarding file for main file in draft mode\\
% |cdocsfi1.tex|&forwarding file for final version of chapter 1\\
% |cdocsfi2.tex|&forwarding file for final version of chapter 2\\
% \end{tabular}
% \end{center}
% Each of the eight files can be compiled directly by the \LaTeX{} compiler.
%
% %%%%%%%%%%%%%%%%%%%%%%%%%%%%%%%%%%%%%%
% \paragraph{Main File.}
%
% The main file is called |cdocsamp.tex|.
%
% Load the \textsf{childdoc} definitions and
% declare the filename for the main document:
%    \begin{macrocode}
\input{childdoc.def}
\childdocmain{}
%    \end{macrocode}

% Optional override for |\version| flag:
%    \begin{macrocode}
%%\ifchilddoc\else\providecommand{\version}{draft}\fi
%    \end{macrocode}

% Define the default values for the |\version| flag
% (|final| for the main file and |draft| for childs):
%    \begin{macrocode}
\ifchilddoc
\providecommand{\version}{draft}
\else
\providecommand{\version}{final}
\fi
%    \end{macrocode}

% Load the standard document class:
%    \begin{macrocode}
\documentclass[12pt]{article}
%    \end{macrocode}

% Start the document body:
%    \begin{macrocode}
\begin{document}
%    \end{macrocode}

% Declare a title page.
% Print title, part of document being processed and version flag:
%    \begin{macrocode}
\addtocounter{page}{-1}
\begin{center}
{\LARGE\bfseries{}childdoc example\par}
\vspace{1cm}
\ifchilddoc
\ifchilddocmanual part\else chapter\fi:
`\childdocname' of `\childdocjob'\par
\else
main document: `\childdocjob'\par
\fi
version: \version\par
\end{center}
\newpage
%    \end{macrocode}

% Manually include selected file,
% otherwise process as usual:
%    \begin{macrocode}
\ifchilddocmanual
\section*{part `\childdocname'}
\input{\childdocname}
\else
%    \end{macrocode}

% Include the two chapters:
%    \begin{macrocode}
\include{cdocsch1}
\include{cdocsch2}
%    \end{macrocode}

% Include the two parts unless only chapters should be displayed:
%    \begin{macrocode}
\ifchilddoc\else
\section{part three}
\input{cdocspt3}
\section{part four}
\input{cdocspt4}
\fi
%    \end{macrocode}

% Process as usual until here:
%    \begin{macrocode}
\fi
%    \end{macrocode}

% End of document body:
%    \begin{macrocode}
\end{document}
%    \end{macrocode}
%\iffalse
%</samplemain>
%\fi
%
% %%%%%%%%%%%%%%%%%%%%%%%%%%%%%%%%%%%%%%
% \paragraph{Chapter Include Files.}
%
% The include files are called |cdocsch1.tex| and |cdocsch2.tex|.
%
%\iffalse
%<*samplechap1|samplechap2>
%\fi

% Optional override for |\version| flag:
%    \begin{macrocode}
%%\providecommand{\version}{final}
%    \end{macrocode}

% Include the main document:
%    \begin{macrocode}
\input{childdoc.def}
\childdocof{cdocsamp}
%    \end{macrocode}

%\iffalse
%</samplechap1|samplechap2>
%\fi
%
%\iffalse
%<*samplechap1>
%\fi
% Some text for chapter 1:
%    \begin{macrocode}
\section{one}
some text in chapter one
%    \end{macrocode}

%\iffalse
%</samplechap1>
%\fi
% Some text for chapter 2:
%\iffalse
%<*samplechap2>
%\fi
%    \begin{macrocode}
\section{two}
more text in chapter two
%    \end{macrocode}

%\iffalse
%</samplechap2>
%\fi
%
% %%%%%%%%%%%%%%%%%%%%%%%%%%%%%%%%%%%%%%
% \paragraph{Part Include Files.}
%
% The include files are called |cdocspt3.tex| and |cdocspt4.tex|.
%
%\iffalse
%<*samplepart3|samplepart4>
%\fi

% Optional override for |\version| flag:
%    \begin{macrocode}
%%\providecommand{\version}{final}
%    \end{macrocode}

% Include the main document:
%    \begin{macrocode}
\input{childdoc.def}
\childdocby{cdocsamp}
%    \end{macrocode}

%\iffalse
%</samplepart3|samplepart4>
%\fi
%
%\iffalse
%<*samplepart3>
%\fi
% Some text for part 3:
%    \begin{macrocode}
some text in part three
%    \end{macrocode}

%\iffalse
%</samplepart3>
%\fi
% Some text for part 4:
%\iffalse
%<*samplepart4>
%\fi
%    \begin{macrocode}
more text in part four
%    \end{macrocode}

%\iffalse
%</samplepart4>
%\fi
%
% %%%%%%%%%%%%%%%%%%%%%%%%%%%%%%%%%%%%%%
% \paragraph{Forwarding for a Complete Draft.}
%
% The following forwarding file |cdocsdrf.tex|
% compiles the main document in draft mode:
%\iffalse
%<*sampledraft>
%\fi
%    \begin{macrocode}
\def\version{draft}
\input{childdoc.def}
\childdocforward{cdocsamp}
%    \end{macrocode}

%\iffalse
%</sampledraft>
%\fi
%
% %%%%%%%%%%%%%%%%%%%%%%%%%%%%%%%%%%%%%%
% \paragraph{Forwarding for Final Version of the Chapters.}
%
% The following forwarding files |cdocsfn1.tex| and |cdocsfn2.tex|
% (with identical content)
% compile the final versions of the child documents
% |cdocsch1.tex| and |cdocsch2.tex|, respectively:
%\iffalse
%<*samplefinal>
%\fi
%    \begin{macrocode}
\def\version{final}
\input{childdoc.def}
\childdocforwardprefix[cdocsamp]{cdocsfn}{cdocsch}
%    \end{macrocode}

%\iffalse
%</samplefinal>
%\fi
%
% %%%%%%%%%%%%%%%%%%%%%%%%%%%%%%%%%%%%%%
% \paragraph{Command Line Processing.}
%
% The following three command lines generate the output files
% |cdocscld|, |cdocscl1| and |cdocscl2|
% which should be identical to
% |cdocsdrf|, |cdocsch1| and |cdocsfn2|, respectively:
% \begin{center}
% \begin{tabular}{l}
% |latex -jobname cdocscld \|\\
% |  "\def\version{draft}\input{childdoc.def}\childdocforward{cdocsamp}"|\\
% |latex -jobname cdocscl1 \|\\
% |  "\input{childdoc.def}\childdocforward[cdocsamp]{cdocsch1}"|\\
% |latex -jobname cdocscl2 \|\\
% |  "\def\version{final}\input{childdoc.def}\childdocforward{cdocsch2}"|
% \end{tabular}
% \end{center}
% Note that the trailing backslash on each first line
% merely continues the input to the second line
% (for convenient cut ant paste).
% Furthermore, the command |latex| can be replaced by any
% of its alternative versions such as |pdflatex|.
%
% %%%%%%%%%%%%%%%%%%%%%%%%%%%%%%%%%%%%%%%%%%%%%%%%%%%%%%%%%%%%%%%%%%%%%%%%%%%%%%
% %%%%%%%%%%%%%%%%%%%%%%%%%%%%%%%%%%%%%%%%%%%%%%%%%%%%%%%%%%%%%%%%%%%%%%%%%%%%%%
% \section{Implementation}
%\iffalse
%<*package>
%\fi
%
% This section describes the definitions file |childdoc.def|.

% The definitions cannot be loaded using |\usepackage| or |\RequirePackage|
% which has a mechanism to prevent loading a style file more than once.
% When loading the definitions by means of |\input|
% multiple instances have to be prevented manually:
%\iffalse
%This code needs to be before the `\ProvidesFile' directive
%which is defined at the beginning of this file.
%Therefore it is also placed there and commented out here.
%</package>
%<*discard>
%\fi
%    \begin{macrocode}
\ifdefined\childdocmain\endinput\fi
%    \end{macrocode}
%\iffalse
%</discard>
%<*package>
%\fi
%
% \macro{\ifchilddoc}
% \macro{\ifchilddocmanual}
% The conditional |\ifchilddoc| tells whether a
% child (true) or main (false) document is being compiled.
% The conditional |\ifchilddocmanual| tells whether
% the |\includeonly| mechanism is used (false) or
% the selection of child files must be performed manually (true).
% The definitions initialise to false:
%    \begin{macrocode}
\newif\ifchilddoc
\newif\ifchilddocmanual
%    \end{macrocode}

% \macro{\childdocname}
% \macro{\childdocjob}
% The macro |\childdocname| stores the name of the main document
% to be compiled. The macro |\childdocjob| stores the name of
% the document on which the \LaTeX{} compiler was originally invoked.
% The content of |\jobname| cannot be compared
% to filenames specified in the source due to different catcodes.
% The following code rescans |\jobname|, stores the result
% in |\childdocname| and saves a copy in |\childdocjob|:
%    \begin{macrocode}
\edef\childdocname{\scantokens\expandafter{\jobname\noexpand}}
\let\childdocjob\childdocname
%    \end{macrocode}

% \macro{\childdocdisable}
% The macro |\childdocdisable| prevents the main file
% from being processed more than once.
% At this stage, the main document command |\childdocmain|
% is assumed to be called once again where it should do nothing.
% Any subsequent call to it should prevent
% a secondary processing of the main document
% It overwrites the forwarding commands
% |\childdocof| and |\childdocforward|
% with empty macros to prevent further inclusions of the main document:
%    \begin{macrocode}
\newcommand{\childdocdisable}
{
  \renewcommand{\childdocmain}[1]{\renewcommand{\childdocmain}[1]{\endinput}}
  \renewcommand{\childdocof}[1]{}
  \renewcommand{\childdocby}[2][]{}
  \renewcommand{\childdocforward}[2][]{}
  \renewcommand{\childdocdisable}{}
}
%    \end{macrocode}

% \macro{\childdocmain}
% The macro |\childdocmain| is to be called at the top of the main file
% with nothing or the main filename (without extension) as argument.
% First, it breaks loops.
% If the argument is not empty and does not match |\childdocname|
% (which is set by the first inclusion of |childdoc.def|),
% |\ifchilddoc| is set to true, |\includeonly| is applied to the child file
% and |\jobname| is set to the main file
% (for proper handling of |.aux| files):
%    \begin{macrocode}
\newcommand{\childdocmain}[1]
{
  \childdocdisable\childdocmain{}
  \if?#1?\else
    \begingroup
      \def\childdoctmp{#1}
      \ifx\childdoctmp\childdocname
        \def\childdoctmp{}
      \else
        \def\childdoctmp
        {
          \childdoctrue
          \includeonly{\childdocname}
          \def\childdocjob{#1}
          \def\jobname{#1}
        }
      \fi
      \expandafter
    \endgroup
    \childdoctmp
  \fi
}
%    \end{macrocode}

% \macro{\childdocof}
% The command |\childdocof| redirects
% compilation to the main file |#1|.
%    \begin{macrocode}
\newcommand{\childdocof}[1]
{
  \childdocdisable
  \childdoctrue
  \includeonly{\childdocname}
  \def\jobname{#1}
  \def\childdocjob{#1}
  \input{#1}
}
%    \end{macrocode}

% \macro{\childdocby}
% The command |\childdocby| ....
%    \begin{macrocode}
\newcommand{\childdocby}[2][]
{
  \childdocdisable
  \childdoctrue
  \childdocmanualtrue
  \if?#1?\else
    \def\jobname{#2}
  \fi
  \def\childdocjob{#2}
  \input{#2}
  \endinput
}
%    \end{macrocode}

% \macro{\childdocforward}
% The command |\childdocforward| redirects
% compilation to the main file or
% (if the optional argument is given) a child file.
% Parameters are set as if the main file
% or a child file starting with |\childdocof| was compiled.
% Then compilation is handed over to the main file:
%    \begin{macrocode}
\newcommand{\childdocforward}[2][]
{
  \begingroup
    \if?#1?
      \def\childdoctmp
      {
        \def\childdocname{#2}
        \def\childdocjob{#2}
        \def\jobname{#2}
        \input{#2}
        \endinput
      }
    \else
      \def\childdoctmp
      {
        \childdocdisable
        \def\childdocname{#2}
        \childdoctrue
        \includeonly{#2}
        \def\childdocjob{#1}
        \def\jobname{#1}
        \input{#1}
        \endinput
      }
    \fi
    \expandafter
  \endgroup
  \childdoctmp
}
%    \end{macrocode}

% \macro{\childdocforwardprefix}
% The command |\childdocforwardprefix| redirects
% compilation to the main or a child file by means of a pattern.
% The prefix |#1| in the current filename is replaced by |#2|
% and the suffix of the current filename is kept
% (it is assumed that the filename does not contain the substring `|~~~|'
% which is used as a delimiter).
% Compilation is handed over to the new file by |\childdocforward|:
%    \begin{macrocode}
\newcommand{\childdocforwardprefix}[3][]
{
  \begingroup
    \def\childdocextract #2##1~~~{\def\childdoctmp{\childdocforward[#1]{#3##1}}}
    \expandafter\childdocextract\childdocname~~~
    \expandafter
  \endgroup
  \childdoctmp
}
%    \end{macrocode}

% \macro{\childdoc}
% The deprecated macro |\childdoc| is a legacy version of |\childdocmain|:
%    \begin{macrocode}
\newcommand{\childdoc}{\childdocmain}
%    \end{macrocode}

% \macro{\childdocredirect}
% The deprecated macro |\childdocredirect| is a legacy version
% of |\childdocforward| and |\childdocforwardprefix|:
%    \begin{macrocode}
\newcommand{\childdocredirect}[2][]
{
  \begingroup
    \if?#1?
      \def\childdoctmp{\childdocforward{#2}}
    \else
      \def\childdoctmp{\childdocforwardprefix{#1}{#2}}
    \fi
    \expandafter
  \endgroup
  \childdoctmp
}
%    \end{macrocode}

%\iffalse
%</package>
%\fi
%
\endinput
|\\
|\childdocforward{|\textit{main}|}|\\
\end{tabular}
\end{center}
%
or alternatively with:
%
\begin{center}
\begin{tabular}{l}
|% \iffalse
%
% childdoc.dtx Copyright (C) 2017-2018 Niklas Beisert
%
% This work may be distributed and/or modified under the
% conditions of the LaTeX Project Public License, either version 1.3
% of this license or (at your option) any later version.
% The latest version of this license is in
%   http://www.latex-project.org/lppl.txt
% and version 1.3 or later is part of all distributions of LaTeX
% version 2005/12/01 or later.
%
% This work has the LPPL maintenance status `maintained'.
%
% The Current Maintainer of this work is Niklas Beisert.
%
% This work consists of the files childdoc.dtx and childdoc.ins
% and the derived files childdoc.def and cdocsamp.tex with
% cdocsch1.tex, cdocsch2.tex, cdocsdrf.tex, cdocsfn1.tex, cdocsfn2.tex.
%
%<package>\ifdefined\childdocmain\endinput\fi
%<package>\ProvidesFile{childdoc.def}[2018/12/30 v2.0 child document driver]
%<samplemain>\ProvidesFile{cdocsamp.tex}[2018/12/30 v2.0 sample for childdoc]
%<*driver>
%\ProvidesFile{childdoc.drv}[2018/12/30 v2.0 childdoc reference manual file]
\PassOptionsToClass{10pt,a4paper}{article}
\documentclass{ltxdoc}

\usepackage[margin=35mm]{geometry}
\usepackage{hyperref}
\usepackage{hyperxmp}
\usepackage[usenames]{color}

\hypersetup{colorlinks=true}
\hypersetup{pdfstartview=FitH}
\hypersetup{pdfpagemode=UseNone}
\hypersetup{pdfsource={}}
\hypersetup{pdflang={en-UK}}
\hypersetup{pdfcopyright={Copyright 2017-2018 Niklas Beisert.
  This work may be distributed and/or modified under the
  conditions of the LaTeX Project Public License, either version 1.3
  of this license or (at your option) any later version.}}
\hypersetup{pdflicenseurl={http://www.latex-project.org/lppl.txt}}
\hypersetup{pdfcontactaddress={ETH Zurich, ITP, HIT K,
  Wolfgang-Pauli-Strasse 27}}
\hypersetup{pdfcontactpostcode={8093}}
\hypersetup{pdfcontactcity={Zurich}}
\hypersetup{pdfcontactcountry={Switzerland}}
\hypersetup{pdfcontactemail={nbeisert@itp.phys.ethz.ch}}
\hypersetup{pdfcontacturl={http://people.phys.ethz.ch/\xmptilde nbeisert/}}

\newcommand{\secref}[1]{\hyperref[#1]{section \ref*{#1}}}

\parskip1ex
\parindent0pt
\let\olditemize\itemize
\def\itemize{\olditemize\parskip0pt}

\begin{document}

\title{The \textsf{childdoc} Package}
\hypersetup{pdftitle={The childdoc Package}}
\author{Niklas Beisert\\[2ex]
  Institut f\"ur Theoretische Physik\\
  Eidgen\"ossische Technische Hochschule Z\"urich\\
  Wolfgang-Pauli-Strasse 27, 8093 Z\"urich, Switzerland\\[1ex]
  \href{mailto:nbeisert@itp.phys.ethz.ch}
  {\texttt{nbeisert@itp.phys.ethz.ch}}}
\hypersetup{pdfauthor={Niklas Beisert}}
\hypersetup{pdfsubject={Manual for the LaTeX2e Package childdoc}}
\date{30 December 2018, \textsf{v2.0}}
\maketitle

\begin{abstract}\noindent
\textsf{childdoc} is a \LaTeXe{} package
that enables the direct compilation
of document sections included by |\include|
to individual files.
\end{abstract}

\begingroup
\parskip0ex
\tableofcontents
\endgroup

%%%%%%%%%%%%%%%%%%%%%%%%%%%%%%%%%%%%%%%%%%%%%%%%%%%%%%%%%%%%%%%%%%%%%%%%%%%%%%%%
%%%%%%%%%%%%%%%%%%%%%%%%%%%%%%%%%%%%%%%%%%%%%%%%%%%%%%%%%%%%%%%%%%%%%%%%%%%%%%%%
\section{Introduction}

\LaTeX{} provides a mechanism to structure a large document (such as a book)
into a main file and several child files (containing the chapters)
using the |\include| command.
This mechanism is beneficial for documents
which span hundreds of pages in order to
make the source file(s) more manageable.
Moreover, compilation can be restricted to
selected child files by means of the |\includeonly| command.
The latter feature can be used to reduce the compilation time while editing
(this was significantly more useful in the earlier days of \LaTeX{})
or to generate a smaller document which is easier to navigate.
Another application of |\includeonly| is to generate
documents consisting of selected parts of the complete document.

However, there are a few drawbacks of the plain |\include| mechanism:
\begin{itemize}
\item
The child files cannot be compiled on their own,
they can only be compiled via the main file.
A naive editing environment
(such as a text editor with an option
to have the current file processed by \LaTeX)
may require one to switch to the main file before compiling;
attempting to compile the child file produces errors.
\item
The main file must be modified (each time)
to adjust the |\includeonly| command
to the present needs. This easily leaves the main file in a messy state.
\item
The generated document will always carry the filename
of the main document. This is inconvenient if
several child files are to be compiled and
to be kept for distribution.
\end{itemize}

The present package provides a simple interface
to make child files individually compilable by \LaTeX{}.
Compiling a child file then has the same effect as compiling
the main file with an |\includeonly| command
to select the appropriate child.
Moreover the generated document will carry the name of the child
rather than the main file.
This resolves all three above issues.

This feature is meant to make the editing of books,
thesis documents and lecture notes somewhat more convenient.
However, the package can also be used efficiently for
composing a series of documents (such as exercise sheets)
which are typically distributed individually.
It then assists the author in generating the individual documents
(potentially in different versions)
as well as a document containing the collected series.
Another application is in developing style files
or other kinds of included material
where compilation of the style file could redirect
to a sample or test file.

%%%%%%%%%%%%%%%%%%%%%%%%%%%%%%%%%%%%%%%%%%%%%%%%%%%%%%%%%%%%%%%%%%%%%%%%%%%%%%%%
%%%%%%%%%%%%%%%%%%%%%%%%%%%%%%%%%%%%%%%%%%%%%%%%%%%%%%%%%%%%%%%%%%%%%%%%%%%%%%%%
\section{Usage}

First of all, the package \textsf{childdoc} is \emph{not} a standard
\LaTeXe{} |.sty| style file! Therefore it needs to be invoked in
a non-standard way.

%%%%%%%%%%%%%%%%%%%%%%%%%%%%%%%%%%%%%%%%%%%%%%%%%%%%%%%%%%%%%%%%%%%%%%%%%%%%%%%%
\subsection{Included Files}
\label{sec:include}

%%%%%%%%%%%%%%%%%%%%%%%%%%%%%%%%%%%%%%%%
\DescribeMacro{\childdocmain}
To use the package, add the commands
\begin{center}
\begin{tabular}{l}
|\input{childdoc.def}|\\
|\childdocmain{}|\\
\end{tabular}
\end{center}
at the very top of the main \LaTeX{} file,
in particular \emph{before} the |\documentclass| statement!
The argument of |\childdocmain| should be left empty
(but it must be present).

%%%%%%%%%%%%%%%%%%%%%%%%%%%%%%%%%%%%%%%%
\DescribeMacro{\childdocof}
Furthermore, add the commands
\begin{center}
\begin{tabular}{l}
|\input{childdoc.def}|\\
|\childdocof{|\textit{main}|}|\\
\end{tabular}
\end{center}
at the top of every child file \textit{child}
which is included by |\include{|\textit{child}|}|
from within the main file
(or at least for those files to be compiled individually).
The argument \textit{main} must be the filename of the main file.

There are a couple of
considerations in setting up the main and child documents:

%%%%%%%%%%%%%%%%%%%%%%%%%%%%%%%%%%%%%%%%
\paragraph{Restrictions.}

Please note the following restrictions:
\begin{itemize}
\item
|\childdocmain| must be called with one argument \textit{main}
to ensure compatibility with earlier version of the package.
It must either be empty (|\childdocmain{}|)
or precisely match the filename of the main file in which it is specified.
See \secref{sec:detection} for further information.
\item
The filename \textit{main} must be specified without the |.tex| extension.
\item
The filename \textit{main} is case sensitive
(even in case-insensitive file systems)
due to internal string comparison.
\item
The argument \textit{main} should be fully expanded, it cannot be a macro.
\item
Subdirectories and special characters should be avoided in filenames.
\item
The command |\childdocmain{|\textit{main}|}| must be followed by a whitespace.
It should not be followed immediately by another command
or by a comment mark `|%|'.
This is because the \TeX{} parser reads the token immediately following
the argument of |\childdocmain| and puts it
at the beginning of every child section;
however, a white\-space is ignored.
\end{itemize}

%%%%%%%%%%%%%%%%%%%%%%%%%%%%%%%%%%%%%%%%
\paragraph{Content of Main File.}

It is advisable to place all content in the child files included by |\include|.
Any output contained in the main file will appear in all child documents
unless suppressed manually;
it cannot be suppressed automatically by the |\includeonly| directive
and thus should normally be avoided.
A method to include some content in the main file
by means of conditional processing is described in \secref{sec:conditional}.

%%%%%%%%%%%%%%%%%%%%%%%%%%%%%%%%%%%%%%%%
\paragraph{Page Numbering.}

When only a part of the document is compiled,
the appropriate numbering of pages
(as well as other status parameters)
is determined from the |.aux| files.
The latter contain information from previous passes.
However this information needs to propagate through
all intermediate child documents.
Therefore the page numbering in child documents may well
be inconsistent until the complete document is compiled at least once.

A useful (if unconventional) way to always ensure a consistent
page numbering is to restart the numbering in each child document
and denote the pages by `\textit{child}|.|\textit{page}'
where \textit{child} represents the chapter/section number of the child file.
This can be achieved by the command
|\numberwithin{page}{|\textit{child}|}|
of the \textsf{amsmath} package
where \textit{child} can be |chapter| or |section|
depending on the chosen structuring.
Alternatively, one can modify the macro |\thepage| appropriately
and reset the counter |page| at the start of each child file.

%%%%%%%%%%%%%%%%%%%%%%%%%%%%%%%%%%%%%%%%%%%%%%%%%%%%%%%%%%%%%%%%%%%%%%%%%%%%%%%%
\subsection{Conditional Processing}
\label{sec:conditional}

The package provides a mechanism to compile different versions
of a document. To customise the versions further some conditional processing
can come in handy to distinguish which version is being compiled.
The package provides two macros to describe the compilation context:

%%%%%%%%%%%%%%%%%%%%%%%%%%%%%%%%%%%%%%%%
\DescribeMacro{\ifchilddoc}
The conditional |\ifchilddoc| distinguishes between the compilation of
child documents and the main document:
%
\begin{center}
|\ifchilddoc |\textit{child-code}| |[|\||else |\textit{main-code}]| \||fi|
\end{center}

%%%%%%%%%%%%%%%%%%%%%%%%%%%%%%%%%%%%%%%%
\DescribeMacro{\childdocname}
\DescribeMacro{\childdocjob}
The macro |\childdocname| contains the filename (without extension)
of the main or child file being processed.
Note that |\childdocjob| will always contain the name of the main file.

%%%%%%%%%%%%%%%%%%%%%%%%%%%%%%%%%%%%%%%%
\paragraph{Title Page.}

Conditional processing can be used to include a title or banner page
in the main document when proper precautions are taken.
Importantly, the code in the main file should ensure that the page counter
(as well as other status parameters which are stored in the |.aux| files)
takes the same value after the conditional processing.
Otherwise the page numbers may take divergent values
depending on which part is compiled.

For example, a title page could be declared by:
%
\begin{center}
\begin{tabular}{l}
|\ifchilddoc\||else|\\
|\addtocounter{page}{-1}|\\
\textit{code for title page}\\
|\newpage|\\
|\||fi|
\end{tabular}
\end{center}
%
A banner page for the child documents can be generated by:
%
\begin{center}
\begin{tabular}{l}
|\ifchilddoc|\\
|\addtocounter{page}{-1}|\\
\textit{code for banner page}\\
|\newpage|\\
|\||fi|
\end{tabular}
\end{center}
%
Here one could write a message such as:
\begin{center}
|This is the part \childdocname{} of \childdocjob{}.|
\end{center}

%%%%%%%%%%%%%%%%%%%%%%%%%%%%%%%%%%%%%%%%%%%%%%%%%%%%%%%%%%%%%%%%%%%%%%%%%%%%%%%%
\subsection{Flags}
\label{sec:flags}

The package makes it easy to generate different versions
of the main or child documents.
To this end compilation flags can be defined
and assigned different default values.
They will be particularly useful in conjunction
with the forwarding mechanism described in \secref{sec:forward}.

For example, it may be useful to have a flag |\version|
which can be set to |draft| or |final|.
The document source will contain some conditional code
depending on the value of |\version|.
Suppose further, the flag should default to |final| for the main file
and to |draft| for child files
which is a natural assignment for editing the document.
This is achieved by placing the following code
in the preamble of the main document
(below the |\childdocmain| directive):
%
\begin{center}
\begin{tabular}{l}
|\ifchilddoc|\\
|\providecommand{\version}{draft}|\\
|\||else|\\
|\providecommand{\version}{final}|\\
|\||fi|
\end{tabular}
\end{center}
%
The definition by |\providecommand| makes sure
that previous definitions are not overwritten.
Further statements |\providecommand{\version}{...}|
can thus be added before the above code to override it.

For the main file, one might add a line
(between |\childdocmain| and the above block)
%
\begin{center}
|%\ifchilddoc\||else\providecommand{\version}{draft}\||fi|
\end{center}
%
which can be uncommented to produce a draft version.
Likewise one can add a line to the very top of a child file
(above the |\childdocof{|\textit{main}|}| directive)
%
\begin{center}
|%\providecommand{\version}{final}|
\end{center}
%
which can be uncommented to produce the final version of this child document.

%%%%%%%%%%%%%%%%%%%%%%%%%%%%%%%%%%%%%%%%%%%%%%%%%%%%%%%%%%%%%%%%%%%%%%%%%%%%%%%%
\subsection{Forwarding}
\label{sec:forward}

Different versions of the main or child documents
using compilation flags as described in \secref{sec:flags}
can be (permanently) stored in different files
for convenient compilation, viewing and distribution.
To this end, the package defines a command
to pass on compilation to a different file:

%%%%%%%%%%%%%%%%%%%%%%%%%%%%%%%%%%%%%%%%
\DescribeMacro{\childdocforward}
The command |\childdocforward| redirects processing to
another source file:
%
\begin{center}
\begin{tabular}{l}
|\input{childdoc.def}|\\
|\childdocforward[|\textit{main}|]{|\textit{dest}|}|\\
\end{tabular}
\end{center}
%
The argument \textit{dest} is the destination file
(without extension).
It should be the main file or one of the child files.
Note that further \textsf{childdoc} directives
such as |\childdocof| and |\childdocforward|
in the indicated file will be processed in this form.
The optional argument \textit{main}
passes on directly to the main file \textit{main}
while pretending to compile the child \textit{dest}.
This form behaves as if \textit{dest}
issues |\childdocof{|\textit{main}|}| right away,
and no further \textsf{childdoc} directives will be processed.

%%%%%%%%%%%%%%%%%%%%%%%%%%%%%%%%%%%%%%%%
\DescribeMacro{\...prefix}
In the alternative form |\childdocforwardprefix|,
%
\begin{center}
\begin{tabular}{l}
|\input{childdoc.def}|\\
|\childdocforwardprefix[|\textit{main}|]{|\textit{prefix}|}{|\textit{dest}|}|
\end{tabular}
\end{center}
%
the destination file is determined by a pattern
depending on the current file:
To make this work, the current file must be called
`{\textit{prefix}\hspace{0.2em}\textit{suffix}}'
with \textit{prefix} matching precisely the argument.
Processing is then passed on to the file
`{\textit{dest}\hspace{0.2em}\textit{suffix}}'.
Surely, the same effect is achieved by
directly specifying the
argument `{\textit{dest}\hspace{0.2em}\textit{suffix}}'
in the first form.
However, that requires to set up a different file
for each child. With the alternative form of the command
all these files can have exactly the same content
which simplifies setting them up and maintaining them.

For example, the following file |draft.tex|
with a compilation flag |\version| as described in \secref{sec:flags}
compiles the main document as a draft:
%
\begin{center}
\begin{tabular}{l}
|\def\version{draft}|\\
|\input{childdoc.def}|\\
|\childdocforward{|\textit{main}|}|
\end{tabular}
\end{center}
%
Likewise, the following files |final|\textit{nn}|.tex|
compile the final version of the child document
|child|\textit{nn}|.tex|:
%
\begin{center}
\begin{tabular}{l}
|\def\version{final}|\\
|\input{childdoc.def}|\\
|\childdocforwardprefix{final}{child}|
\end{tabular}
\end{center}
%

Note that when several versions of a main file and/or of each child file
are to be generated, it may be convenient to set up a |Makefile| or
shell script to automatise the process.

%%%%%%%%%%%%%%%%%%%%%%%%%%%%%%%%%%%%%%%%%%%%%%%%%%%%%%%%%%%%%%%%%%%%%%%%%%%%%%%%
\subsection{Command Line Processing}
\label{sec:commandline}

The effect of redirection files can also be achieved by invoking
the \LaTeX{} compiler with a more elaborate command line.
Most conveniently this should be done as part
of a shell script or a |Makefile|.

When using \textsf{childdoc} in the main file, the following
command lines effectively perform a redirection
(note that depending on the shell being used,
backslashes may have to be doubled: `|\|' $\to$ `|\\|'):
%
\begin{center}
|... -jobname "|\textit{target}|" |\\|"|[\textit{flags}]%
|\input{childdoc.def}\childdocforward[|\textit{main}|]{|\textit{dest}|}"|
\end{center}
%
Here \textit{target} is the name of the output file,
\textit{main} is the name of the main file
and \textit{dest} is the name of the main or child file to be processed
(all filenames without extensions).
The optional argument \textit{main} can be omitted
if \textit{main} matches \textit{dest}.
Optionally, compilation \textit{flags} can be defined via |\def| commands.
This command line makes the \TeX{} engine believe
it is compiling the file \textit{target}
whose content is specified as the latter parameter.
The provided code then forwards the processing to
\textit{main} or \textit{dest} as described in \secref{sec:forward}.

%%%%%%%%%%%%%%%%%%%%%%%%%%%%%%%%%%%%%%%%%%%%%%%%%%%%%%%%%%%%%%%%%%%%%%%%%%%%%%%%
\subsection{Include by Input}
\label{sec:input}

Including child documents by |\include| has some restrictions by design.
Most notably, the content of a child document always occupies
its own set of pages; pages cannot be shared between child documents.
Usually, this behaviour makes perfect sense
because each child document contain an essential part of the document.
However, in some situations it may be desirable to compose
a document from a collection of parts
without having mandatory page breaks between then.
For this case, the package
provides a mechanism to include parts
by |\input| which can also be processed individually.
However, by construction this mechanism
requires manual handling of the content to be output.

%%%%%%%%%%%%%%%%%%%%%%%%%%%%%%%%%%%%%%%%
\DescribeMacro{\ifchilddocmanual}
The main file should be prepared as usual, see \secref{sec:include}.
However, the document body must make a distinction
between processing of an individual part and of the main document, e.g.:
%
\begin{center}
\begin{tabular}{l}
|\ifchilddocmanual|\\
|\input{\childdocname}|\\
|\||else|\\
\textit{document body with }|\input{|\textit{part}|}|\\
|\||fi|
\end{tabular}
\end{center}
%
The conditional |\ifchilddocmanual| is true whenever
a part to be included by |\input| is being compiled,
and the name of the part is stored in |\childdocname|.

%%%%%%%%%%%%%%%%%%%%%%%%%%%%%%%%%%%%%%%%
\DescribeMacro{\childdocby}
Each part to be included by |\input| should start with:
%
\begin{center}
\begin{tabular}{l}
|\input{childdoc.def}|\\
|\childdocby{|\textit{main}|}|\\
\end{tabular}
\end{center}
%
The directive |\childdocby| is similar to |\childdocof|
described in \secref{sec:include},
but the subsequent selection of content must be done manually.
To that end, both |\ifchilddoc| and |\ifchilddocmanual|
will be true upon processing of a part,
and the name of the part is stored in |\childdocname|.
Note that |\jobname| will be set to the filename of the current part
so that each part receives an individual |.aux| file
that does not interfere with the |.aux| file(s) of the main document.
This behaviour can be altered by the alternative form
|\childdocby[*]{|\textit{main}|}| (with a non-empty optional argument)
which uses the |.aux| file of the main document
by setting |\jobname| to \textit{main}.

%%%%%%%%%%%%%%%%%%%%%%%%%%%%%%%%%%%%%%%%%%%%%%%%%%%%%%%%%%%%%%%%%%%%%%%%%%%%%%%%
\subsection{Driver Development}
\label{sec:driver}

The \textsf{childdoc} mechanism can also be use for the development
of definition files such as \LaTeX{} styles or classes.
This case differs from the above setup with multiple parts
included by |\include| in that no |\includeonly| should be invoked.
This can be achieved by starting the include file
(before |\ProvidesPackage|) with:
%
\begin{center}
\begin{tabular}{l}
|\input{childdoc.def}|\\
|\childdocforward{|\textit{main}|}|\\
\end{tabular}
\end{center}
%
or alternatively with:
%
\begin{center}
\begin{tabular}{l}
|\input{childdoc.def}|\\
|\childdocby{|\textit{main}|}|\\
\end{tabular}
\end{center}
%
Both forms have slightly different effects as described above.
The main file is prepared as usual, see \secref{sec:include}.

%%%%%%%%%%%%%%%%%%%%%%%%%%%%%%%%%%%%%%%%%%%%%%%%%%%%%%%%%%%%%%%%%%%%%%%%%%%%%%%%
\subsection{Legacy Detection}
\label{sec:detection}

The directive |\childdocmain| in the main file can detect
whether the complete document or merely a child is to be compiled
even without using the directive |\childdocof|.
This method is deprecated because it is less robust
and there is no compelling reason to use it;
it is merely provided for backward compatibility
and it may be removed in future versions.

If the detection mechanism is to be used,
it is mandatory to correctly specify
the filename of the main file as the argument of |\childdocmain|:
%
\begin{center}
\begin{tabular}{l}
|\input{childdoc.def}|\\
|\childdocmain{|\textit{main}|}|\\
\end{tabular}
\end{center}
%
If |\jobname| does not match the argument \textit{main} of |\childdocmain|,
it is assumed that |\jobname| points to the child file to be compiled.
When using |\childdocmain| with the main file specified as argument,
it suffices to start a child file
with just |\input{|\textit{main}|}|
without loading of the package and using |\childdocof|.
If instead all processing is done
with the appropriate \textsf{childdoc} directives,
the argument of \textit{main} of |\childdocmain| can be empty.

An alternative version of the command line processing described
in \secref{sec:commandline} using the detection mechanism reads:
%
\begin{center}
|... -jobname "|\textit{target}|" "|[\textit{flags}]%
[|\def\jobname{|\textit{dest}|}|]|\input{|\textit{main}|}"|
\end{center}

%%%%%%%%%%%%%%%%%%%%%%%%%%%%%%%%%%%%%%%%%%%%%%%%%%%%%%%%%%%%%%%%%%%%%%%%%%%%%%%%
\subsection{Manual Code}
\label{sec:manual}

In case one cannot be certain whether the definitions file |childdoc.def|
is installed on the target \TeX{} distribution
and one prefers not to ship it,
it is conceivable to paste a few relevant commands into the sources.

To that end, drop all statements |\input{childdoc.def}|
and perform the replacements as outlined below.
Instead of |\childdocmain{|\textit{main}|}| add the following code
to the top of the main file:
%
\begin{center}
\begin{tabular}{l}
|\||ifdefined\childdocname\endinput\||fi\newif\ifchilddoc|\\
|\edef\childdocname{\scantokens\expandafter{\jobname\noexpand}}|\\
|\def\childdocmain{|\textit{main}|}\||ifx\childdocmain\childdocname\||else|\\
|\childdoctrue\includeonly{\childdocname}\let\jobname\childdocmain\||fi|\\
\end{tabular}
\end{center}
%
Instead of |\childdocof{|\textit{main}|}| just include the main file
at the top of each child file:
%
\begin{center}
|\input{|\textit{main}|}|
\end{center}
%
A simple redirection |\childdocforward{|\textit{dest}|}| is achieved by:
%
\begin{center}
|\def\jobname{|\textit{dest}|}\input{\jobname}|
\end{center}
%
The redirection with prefix
|\childdocforwardprefix[|\textit{prefix}|]{|\textit{dest}|}|
is accomplished by:
%
\begin{center}
\begin{tabular}{l}
|{\edef\jobname{\scantokens\expandafter{\jobname\noexpand}}|\\
|\def\redirectjob |\textit{prefix}|#1~~~{\gdef\jobname{|\textit{dest}|#1}}|\\
|\expandafter\redirectjob\jobname~~~}\input{\jobname}|
\end{tabular}
\end{center}

In an alternative approach,
child documents can be compiled by a specific command line
without additional code or specific definitions:
%
\begin{center}
|... -jobname "|\textit{target}|" "|[\textit{flags}]%
|\includeonly{|\textit{dest}|}\input{|\textit{main}|}"|
\end{center}
%

%%%%%%%%%%%%%%%%%%%%%%%%%%%%%%%%%%%%%%%%%%%%%%%%%%%%%%%%%%%%%%%%%%%%%%%%%%%%%%%%
%%%%%%%%%%%%%%%%%%%%%%%%%%%%%%%%%%%%%%%%%%%%%%%%%%%%%%%%%%%%%%%%%%%%%%%%%%%%%%%%
\section{Information}

%%%%%%%%%%%%%%%%%%%%%%%%%%%%%%%%%%%%%%%%%%%%%%%%%%%%%%%%%%%%%%%%%%%%%%%%%%%%%%%%
\subsection{Copyright}

Copyright \copyright{} 2017--2018 Niklas Beisert

This work may be distributed and/or modified under the
conditions of the \LaTeX{} Project Public License, either version 1.3
of this license or (at your option) any later version.
The latest version of this license is in
  \url{http://www.latex-project.org/lppl.txt}
and version 1.3 or later is part of all distributions of \LaTeX{}
version 2005/12/01 or later.

This work has the LPPL maintenance status `maintained'.

The Current Maintainer of this work is Niklas Beisert.

This work consists of the files |README.txt|, |childdoc.ins| and |childdoc.dtx|
as well as the derived files |childdoc.def|, |cdocsamp.tex|
with |cdocsch1.tex|, |cdocsch2.tex|, |cdocspt3.tex|, |cdocspt4.tex|,
|cdocsdrf.tex|, |cdocsfn1.tex|, |cdocsfn2.tex|
as well as |childdoc.pdf|.

%%%%%%%%%%%%%%%%%%%%%%%%%%%%%%%%%%%%%%%%%%%%%%%%%%%%%%%%%%%%%%%%%%%%%%%%%%%%%%%%
\subsection{Files and Installation}

The package consists of the files:
%
\begin{center}
\begin{tabular}{ll}
    |README.txt|   & readme file \\
    |childdoc.ins| & installation file \\
    |childdoc.dtx| & source file \\
    |childdoc.def| & definition file \\
    |cdocsamp.tex| & sample main file \\
    |cdocsch1.tex| & sample include file \\
    |cdocsch2.tex| & sample include file \\
    |cdocspt3.tex| & sample part file \\
    |cdocspt4.tex| & sample part file \\
    |cdocsdrf.tex| & sample redirection file \\
    |cdocsfn1.tex| & sample redirection file \\
    |cdocsfn2.tex| & sample redirection file \\
    |childdoc.pdf| & manual
\end{tabular}
\end{center}
%
The distribution consists of the files
|README.txt|, |childdoc.ins| and |childdoc.dtx|.
%
\begin{itemize}
\item
Run (pdf)\LaTeX{} on |childdoc.dtx|
to compile the manual |childdoc.pdf| (this file).
\item
Run \LaTeX{} on |childdoc.ins| to create the definitions file |childdoc.def|
and the sample |cdocsamp.tex| with include files
|cdocsch1.tex|, |cdocsch2.tex|, |cdocspt3.tex|, |cdocspt4.tex|,
|cdocsdrf.tex|, |cdocsfn1.tex|, |cdocsfn2.tex|.
Then copy the file |childdoc.def| to an appropriate directory of your \LaTeX{}
distribution, e.g.\ \textit{texmf-root}|/tex/latex/childdoc|.
\end{itemize}

%%%%%%%%%%%%%%%%%%%%%%%%%%%%%%%%%%%%%%%%%%%%%%%%%%%%%%%%%%%%%%%%%%%%%%%%%%%%%%%%
\subsection{Related CTAN Packages}

There are several other packages which offer a similar functionality:
%
\begin{itemize}
\item
The packages
\href{http://ctan.org/pkg/docmute}{\textsf{docmute}},
\href{http://ctan.org/pkg/includex}{\textsf{includex}} and
\href{http://ctan.org/pkg/standalone}{\textsf{standalone}}
provide commands to include only the document body of
a child file thus allowing both files to be compiled individually.
\item
The packages \href{http://ctan.org/pkg/subdocs}{\textsf{subdocs}}
and \href{http://ctan.org/pkg/subfiles}{\textsf{subfiles}}
provide structures in which the main and child documents can be
encapsulated and allowing them to be compiled individually.
The inclusion mechanism is different from the conventional |\include|.
\item
The package \href{http://ctan.org/pkg/combine}{\textsf{combine}}
is an elaborate solution to combine several documents into one.
\end{itemize}
%
See also the CTAN topic \href{http://ctan.org/topic/subdocs}{\textsf{subdocs}}
for further related packages.
The present package differs from the above solutions in that
a document structure constructed with the conventional |\include| mechanism
just needs two extra commands at the top of every file
such that all constituent files can be compiled individually.

%%%%%%%%%%%%%%%%%%%%%%%%%%%%%%%%%%%%%%%%%%%%%%%%%%%%%%%%%%%%%%%%%%%%%%%%%%%%%%%%
%\subsection{Feature Suggestions}
%
%The following is a list of features which may be useful for future
%versions of this package:
%%
%\begin{itemize}
%\item
%\ldots
%\end{itemize}

%%%%%%%%%%%%%%%%%%%%%%%%%%%%%%%%%%%%%%%%%%%%%%%%%%%%%%%%%%%%%%%%%%%%%%%%%%%%%%%%
\subsection{Revision History}

%%%%%%%%%%%%%%%%%%%%%%%%%%%%%%%%%%%%%%%%
\paragraph{v2.0:} 2018/12/30

\begin{itemize}
\item
immediate forward processing
\item
added |\childdocby| mechanism
\item
manual restructured
\end{itemize}

%%%%%%%%%%%%%%%%%%%%%%%%%%%%%%%%%%%%%%%%
\paragraph{v1.6:} 2018/01/17

\begin{itemize}
\item
application for development of include files
\item
corrections to manual
\end{itemize}

%%%%%%%%%%%%%%%%%%%%%%%%%%%%%%%%%%%%%%%%
\paragraph{v1.5:} 2017/05/21

\begin{itemize}
\item
more complete structuring introduced
\item
|\childdocof| introduced
\item
|\childdoc| renamed to |\childdocmain|
\item
|\childredirect| renamed to |\childdocforward| and |\childdocforwardprefix|
and functionality expanded
\end{itemize}

%%%%%%%%%%%%%%%%%%%%%%%%%%%%%%%%%%%%%%%%
\paragraph{v1.0:} 2017/04/27

\begin{itemize}
\item
manual and install package
\item
first version published on CTAN
\end{itemize}

%%%%%%%%%%%%%%%%%%%%%%%%%%%%%%%%%%%%%%%%
\paragraph{v0.6:} 2017/04/26

\begin{itemize}
\item
redirection mechanism added
\end{itemize}

%%%%%%%%%%%%%%%%%%%%%%%%%%%%%%%%%%%%%%%%
\paragraph{v0.5:} 2017/04/26

\begin{itemize}
\item
functionality in definition file
\end{itemize}


%%%%%%%%%%%%%%%%%%%%%%%%%%%%%%%%%%%%%%%%%%%%%%%%%%%%%%%%%%%%%%%%%%%%%%%%%%%%%%%%
%%%%%%%%%%%%%%%%%%%%%%%%%%%%%%%%%%%%%%%%%%%%%%%%%%%%%%%%%%%%%%%%%%%%%%%%%%%%%%%%
%%%%%%%%%%%%%%%%%%%%%%%%%%%%%%%%%%%%%%%%%%%%%%%%%%%%%%%%%%%%%%%%%%%%%%%%%%%%%%%%
\appendix

\settowidth\MacroIndent{\rmfamily\scriptsize 000\ }

 \DocInput{childdoc.dtx}

\end{document}
%</driver>
% \fi
%
% %%%%%%%%%%%%%%%%%%%%%%%%%%%%%%%%%%%%%%%%%%%%%%%%%%%%%%%%%%%%%%%%%%%%%%%%%%%%%%
% %%%%%%%%%%%%%%%%%%%%%%%%%%%%%%%%%%%%%%%%%%%%%%%%%%%%%%%%%%%%%%%%%%%%%%%%%%%%%%
% \section{Sample}
%\iffalse
%<*samplemain>
%\fi
%
% The following presents a sample document
% with two chapters, two parts, a title page,
% a compile flag as well as three forwarding files to set the flag.
% It consists of eight |.tex| files:
% \begin{center}
% \begin{tabular}{ll}
% |cdocsamp.tex|&main file\\
% |cdocsch1.tex|&include file for chapter 1\\
% |cdocsch2.tex|&include file for chapter 2\\
% |cdocspt3.tex|&include file for part 3\\
% |cdocspt4.tex|&include file for part 4\\
% |cdocsdrf.tex|&forwarding file for main file in draft mode\\
% |cdocsfi1.tex|&forwarding file for final version of chapter 1\\
% |cdocsfi2.tex|&forwarding file for final version of chapter 2\\
% \end{tabular}
% \end{center}
% Each of the eight files can be compiled directly by the \LaTeX{} compiler.
%
% %%%%%%%%%%%%%%%%%%%%%%%%%%%%%%%%%%%%%%
% \paragraph{Main File.}
%
% The main file is called |cdocsamp.tex|.
%
% Load the \textsf{childdoc} definitions and
% declare the filename for the main document:
%    \begin{macrocode}
\input{childdoc.def}
\childdocmain{}
%    \end{macrocode}

% Optional override for |\version| flag:
%    \begin{macrocode}
%%\ifchilddoc\else\providecommand{\version}{draft}\fi
%    \end{macrocode}

% Define the default values for the |\version| flag
% (|final| for the main file and |draft| for childs):
%    \begin{macrocode}
\ifchilddoc
\providecommand{\version}{draft}
\else
\providecommand{\version}{final}
\fi
%    \end{macrocode}

% Load the standard document class:
%    \begin{macrocode}
\documentclass[12pt]{article}
%    \end{macrocode}

% Start the document body:
%    \begin{macrocode}
\begin{document}
%    \end{macrocode}

% Declare a title page.
% Print title, part of document being processed and version flag:
%    \begin{macrocode}
\addtocounter{page}{-1}
\begin{center}
{\LARGE\bfseries{}childdoc example\par}
\vspace{1cm}
\ifchilddoc
\ifchilddocmanual part\else chapter\fi:
`\childdocname' of `\childdocjob'\par
\else
main document: `\childdocjob'\par
\fi
version: \version\par
\end{center}
\newpage
%    \end{macrocode}

% Manually include selected file,
% otherwise process as usual:
%    \begin{macrocode}
\ifchilddocmanual
\section*{part `\childdocname'}
\input{\childdocname}
\else
%    \end{macrocode}

% Include the two chapters:
%    \begin{macrocode}
\include{cdocsch1}
\include{cdocsch2}
%    \end{macrocode}

% Include the two parts unless only chapters should be displayed:
%    \begin{macrocode}
\ifchilddoc\else
\section{part three}
\input{cdocspt3}
\section{part four}
\input{cdocspt4}
\fi
%    \end{macrocode}

% Process as usual until here:
%    \begin{macrocode}
\fi
%    \end{macrocode}

% End of document body:
%    \begin{macrocode}
\end{document}
%    \end{macrocode}
%\iffalse
%</samplemain>
%\fi
%
% %%%%%%%%%%%%%%%%%%%%%%%%%%%%%%%%%%%%%%
% \paragraph{Chapter Include Files.}
%
% The include files are called |cdocsch1.tex| and |cdocsch2.tex|.
%
%\iffalse
%<*samplechap1|samplechap2>
%\fi

% Optional override for |\version| flag:
%    \begin{macrocode}
%%\providecommand{\version}{final}
%    \end{macrocode}

% Include the main document:
%    \begin{macrocode}
\input{childdoc.def}
\childdocof{cdocsamp}
%    \end{macrocode}

%\iffalse
%</samplechap1|samplechap2>
%\fi
%
%\iffalse
%<*samplechap1>
%\fi
% Some text for chapter 1:
%    \begin{macrocode}
\section{one}
some text in chapter one
%    \end{macrocode}

%\iffalse
%</samplechap1>
%\fi
% Some text for chapter 2:
%\iffalse
%<*samplechap2>
%\fi
%    \begin{macrocode}
\section{two}
more text in chapter two
%    \end{macrocode}

%\iffalse
%</samplechap2>
%\fi
%
% %%%%%%%%%%%%%%%%%%%%%%%%%%%%%%%%%%%%%%
% \paragraph{Part Include Files.}
%
% The include files are called |cdocspt3.tex| and |cdocspt4.tex|.
%
%\iffalse
%<*samplepart3|samplepart4>
%\fi

% Optional override for |\version| flag:
%    \begin{macrocode}
%%\providecommand{\version}{final}
%    \end{macrocode}

% Include the main document:
%    \begin{macrocode}
\input{childdoc.def}
\childdocby{cdocsamp}
%    \end{macrocode}

%\iffalse
%</samplepart3|samplepart4>
%\fi
%
%\iffalse
%<*samplepart3>
%\fi
% Some text for part 3:
%    \begin{macrocode}
some text in part three
%    \end{macrocode}

%\iffalse
%</samplepart3>
%\fi
% Some text for part 4:
%\iffalse
%<*samplepart4>
%\fi
%    \begin{macrocode}
more text in part four
%    \end{macrocode}

%\iffalse
%</samplepart4>
%\fi
%
% %%%%%%%%%%%%%%%%%%%%%%%%%%%%%%%%%%%%%%
% \paragraph{Forwarding for a Complete Draft.}
%
% The following forwarding file |cdocsdrf.tex|
% compiles the main document in draft mode:
%\iffalse
%<*sampledraft>
%\fi
%    \begin{macrocode}
\def\version{draft}
\input{childdoc.def}
\childdocforward{cdocsamp}
%    \end{macrocode}

%\iffalse
%</sampledraft>
%\fi
%
% %%%%%%%%%%%%%%%%%%%%%%%%%%%%%%%%%%%%%%
% \paragraph{Forwarding for Final Version of the Chapters.}
%
% The following forwarding files |cdocsfn1.tex| and |cdocsfn2.tex|
% (with identical content)
% compile the final versions of the child documents
% |cdocsch1.tex| and |cdocsch2.tex|, respectively:
%\iffalse
%<*samplefinal>
%\fi
%    \begin{macrocode}
\def\version{final}
\input{childdoc.def}
\childdocforwardprefix[cdocsamp]{cdocsfn}{cdocsch}
%    \end{macrocode}

%\iffalse
%</samplefinal>
%\fi
%
% %%%%%%%%%%%%%%%%%%%%%%%%%%%%%%%%%%%%%%
% \paragraph{Command Line Processing.}
%
% The following three command lines generate the output files
% |cdocscld|, |cdocscl1| and |cdocscl2|
% which should be identical to
% |cdocsdrf|, |cdocsch1| and |cdocsfn2|, respectively:
% \begin{center}
% \begin{tabular}{l}
% |latex -jobname cdocscld \|\\
% |  "\def\version{draft}\input{childdoc.def}\childdocforward{cdocsamp}"|\\
% |latex -jobname cdocscl1 \|\\
% |  "\input{childdoc.def}\childdocforward[cdocsamp]{cdocsch1}"|\\
% |latex -jobname cdocscl2 \|\\
% |  "\def\version{final}\input{childdoc.def}\childdocforward{cdocsch2}"|
% \end{tabular}
% \end{center}
% Note that the trailing backslash on each first line
% merely continues the input to the second line
% (for convenient cut ant paste).
% Furthermore, the command |latex| can be replaced by any
% of its alternative versions such as |pdflatex|.
%
% %%%%%%%%%%%%%%%%%%%%%%%%%%%%%%%%%%%%%%%%%%%%%%%%%%%%%%%%%%%%%%%%%%%%%%%%%%%%%%
% %%%%%%%%%%%%%%%%%%%%%%%%%%%%%%%%%%%%%%%%%%%%%%%%%%%%%%%%%%%%%%%%%%%%%%%%%%%%%%
% \section{Implementation}
%\iffalse
%<*package>
%\fi
%
% This section describes the definitions file |childdoc.def|.

% The definitions cannot be loaded using |\usepackage| or |\RequirePackage|
% which has a mechanism to prevent loading a style file more than once.
% When loading the definitions by means of |\input|
% multiple instances have to be prevented manually:
%\iffalse
%This code needs to be before the `\ProvidesFile' directive
%which is defined at the beginning of this file.
%Therefore it is also placed there and commented out here.
%</package>
%<*discard>
%\fi
%    \begin{macrocode}
\ifdefined\childdocmain\endinput\fi
%    \end{macrocode}
%\iffalse
%</discard>
%<*package>
%\fi
%
% \macro{\ifchilddoc}
% \macro{\ifchilddocmanual}
% The conditional |\ifchilddoc| tells whether a
% child (true) or main (false) document is being compiled.
% The conditional |\ifchilddocmanual| tells whether
% the |\includeonly| mechanism is used (false) or
% the selection of child files must be performed manually (true).
% The definitions initialise to false:
%    \begin{macrocode}
\newif\ifchilddoc
\newif\ifchilddocmanual
%    \end{macrocode}

% \macro{\childdocname}
% \macro{\childdocjob}
% The macro |\childdocname| stores the name of the main document
% to be compiled. The macro |\childdocjob| stores the name of
% the document on which the \LaTeX{} compiler was originally invoked.
% The content of |\jobname| cannot be compared
% to filenames specified in the source due to different catcodes.
% The following code rescans |\jobname|, stores the result
% in |\childdocname| and saves a copy in |\childdocjob|:
%    \begin{macrocode}
\edef\childdocname{\scantokens\expandafter{\jobname\noexpand}}
\let\childdocjob\childdocname
%    \end{macrocode}

% \macro{\childdocdisable}
% The macro |\childdocdisable| prevents the main file
% from being processed more than once.
% At this stage, the main document command |\childdocmain|
% is assumed to be called once again where it should do nothing.
% Any subsequent call to it should prevent
% a secondary processing of the main document
% It overwrites the forwarding commands
% |\childdocof| and |\childdocforward|
% with empty macros to prevent further inclusions of the main document:
%    \begin{macrocode}
\newcommand{\childdocdisable}
{
  \renewcommand{\childdocmain}[1]{\renewcommand{\childdocmain}[1]{\endinput}}
  \renewcommand{\childdocof}[1]{}
  \renewcommand{\childdocby}[2][]{}
  \renewcommand{\childdocforward}[2][]{}
  \renewcommand{\childdocdisable}{}
}
%    \end{macrocode}

% \macro{\childdocmain}
% The macro |\childdocmain| is to be called at the top of the main file
% with nothing or the main filename (without extension) as argument.
% First, it breaks loops.
% If the argument is not empty and does not match |\childdocname|
% (which is set by the first inclusion of |childdoc.def|),
% |\ifchilddoc| is set to true, |\includeonly| is applied to the child file
% and |\jobname| is set to the main file
% (for proper handling of |.aux| files):
%    \begin{macrocode}
\newcommand{\childdocmain}[1]
{
  \childdocdisable\childdocmain{}
  \if?#1?\else
    \begingroup
      \def\childdoctmp{#1}
      \ifx\childdoctmp\childdocname
        \def\childdoctmp{}
      \else
        \def\childdoctmp
        {
          \childdoctrue
          \includeonly{\childdocname}
          \def\childdocjob{#1}
          \def\jobname{#1}
        }
      \fi
      \expandafter
    \endgroup
    \childdoctmp
  \fi
}
%    \end{macrocode}

% \macro{\childdocof}
% The command |\childdocof| redirects
% compilation to the main file |#1|.
%    \begin{macrocode}
\newcommand{\childdocof}[1]
{
  \childdocdisable
  \childdoctrue
  \includeonly{\childdocname}
  \def\jobname{#1}
  \def\childdocjob{#1}
  \input{#1}
}
%    \end{macrocode}

% \macro{\childdocby}
% The command |\childdocby| ....
%    \begin{macrocode}
\newcommand{\childdocby}[2][]
{
  \childdocdisable
  \childdoctrue
  \childdocmanualtrue
  \if?#1?\else
    \def\jobname{#2}
  \fi
  \def\childdocjob{#2}
  \input{#2}
  \endinput
}
%    \end{macrocode}

% \macro{\childdocforward}
% The command |\childdocforward| redirects
% compilation to the main file or
% (if the optional argument is given) a child file.
% Parameters are set as if the main file
% or a child file starting with |\childdocof| was compiled.
% Then compilation is handed over to the main file:
%    \begin{macrocode}
\newcommand{\childdocforward}[2][]
{
  \begingroup
    \if?#1?
      \def\childdoctmp
      {
        \def\childdocname{#2}
        \def\childdocjob{#2}
        \def\jobname{#2}
        \input{#2}
        \endinput
      }
    \else
      \def\childdoctmp
      {
        \childdocdisable
        \def\childdocname{#2}
        \childdoctrue
        \includeonly{#2}
        \def\childdocjob{#1}
        \def\jobname{#1}
        \input{#1}
        \endinput
      }
    \fi
    \expandafter
  \endgroup
  \childdoctmp
}
%    \end{macrocode}

% \macro{\childdocforwardprefix}
% The command |\childdocforwardprefix| redirects
% compilation to the main or a child file by means of a pattern.
% The prefix |#1| in the current filename is replaced by |#2|
% and the suffix of the current filename is kept
% (it is assumed that the filename does not contain the substring `|~~~|'
% which is used as a delimiter).
% Compilation is handed over to the new file by |\childdocforward|:
%    \begin{macrocode}
\newcommand{\childdocforwardprefix}[3][]
{
  \begingroup
    \def\childdocextract #2##1~~~{\def\childdoctmp{\childdocforward[#1]{#3##1}}}
    \expandafter\childdocextract\childdocname~~~
    \expandafter
  \endgroup
  \childdoctmp
}
%    \end{macrocode}

% \macro{\childdoc}
% The deprecated macro |\childdoc| is a legacy version of |\childdocmain|:
%    \begin{macrocode}
\newcommand{\childdoc}{\childdocmain}
%    \end{macrocode}

% \macro{\childdocredirect}
% The deprecated macro |\childdocredirect| is a legacy version
% of |\childdocforward| and |\childdocforwardprefix|:
%    \begin{macrocode}
\newcommand{\childdocredirect}[2][]
{
  \begingroup
    \if?#1?
      \def\childdoctmp{\childdocforward{#2}}
    \else
      \def\childdoctmp{\childdocforwardprefix{#1}{#2}}
    \fi
    \expandafter
  \endgroup
  \childdoctmp
}
%    \end{macrocode}

%\iffalse
%</package>
%\fi
%
\endinput
|\\
|\childdocby{|\textit{main}|}|\\
\end{tabular}
\end{center}
%
Both forms have slightly different effects as described above.
The main file is prepared as usual, see \secref{sec:include}.

%%%%%%%%%%%%%%%%%%%%%%%%%%%%%%%%%%%%%%%%%%%%%%%%%%%%%%%%%%%%%%%%%%%%%%%%%%%%%%%%
\subsection{Legacy Detection}
\label{sec:detection}

The directive |\childdocmain| in the main file can detect
whether the complete document or merely a child is to be compiled
even without using the directive |\childdocof|.
This method is deprecated because it is less robust
and there is no compelling reason to use it;
it is merely provided for backward compatibility
and it may be removed in future versions.

If the detection mechanism is to be used,
it is mandatory to correctly specify
the filename of the main file as the argument of |\childdocmain|:
%
\begin{center}
\begin{tabular}{l}
|% \iffalse
%
% childdoc.dtx Copyright (C) 2017-2018 Niklas Beisert
%
% This work may be distributed and/or modified under the
% conditions of the LaTeX Project Public License, either version 1.3
% of this license or (at your option) any later version.
% The latest version of this license is in
%   http://www.latex-project.org/lppl.txt
% and version 1.3 or later is part of all distributions of LaTeX
% version 2005/12/01 or later.
%
% This work has the LPPL maintenance status `maintained'.
%
% The Current Maintainer of this work is Niklas Beisert.
%
% This work consists of the files childdoc.dtx and childdoc.ins
% and the derived files childdoc.def and cdocsamp.tex with
% cdocsch1.tex, cdocsch2.tex, cdocsdrf.tex, cdocsfn1.tex, cdocsfn2.tex.
%
%<package>\ifdefined\childdocmain\endinput\fi
%<package>\ProvidesFile{childdoc.def}[2018/12/30 v2.0 child document driver]
%<samplemain>\ProvidesFile{cdocsamp.tex}[2018/12/30 v2.0 sample for childdoc]
%<*driver>
%\ProvidesFile{childdoc.drv}[2018/12/30 v2.0 childdoc reference manual file]
\PassOptionsToClass{10pt,a4paper}{article}
\documentclass{ltxdoc}

\usepackage[margin=35mm]{geometry}
\usepackage{hyperref}
\usepackage{hyperxmp}
\usepackage[usenames]{color}

\hypersetup{colorlinks=true}
\hypersetup{pdfstartview=FitH}
\hypersetup{pdfpagemode=UseNone}
\hypersetup{pdfsource={}}
\hypersetup{pdflang={en-UK}}
\hypersetup{pdfcopyright={Copyright 2017-2018 Niklas Beisert.
  This work may be distributed and/or modified under the
  conditions of the LaTeX Project Public License, either version 1.3
  of this license or (at your option) any later version.}}
\hypersetup{pdflicenseurl={http://www.latex-project.org/lppl.txt}}
\hypersetup{pdfcontactaddress={ETH Zurich, ITP, HIT K,
  Wolfgang-Pauli-Strasse 27}}
\hypersetup{pdfcontactpostcode={8093}}
\hypersetup{pdfcontactcity={Zurich}}
\hypersetup{pdfcontactcountry={Switzerland}}
\hypersetup{pdfcontactemail={nbeisert@itp.phys.ethz.ch}}
\hypersetup{pdfcontacturl={http://people.phys.ethz.ch/\xmptilde nbeisert/}}

\newcommand{\secref}[1]{\hyperref[#1]{section \ref*{#1}}}

\parskip1ex
\parindent0pt
\let\olditemize\itemize
\def\itemize{\olditemize\parskip0pt}

\begin{document}

\title{The \textsf{childdoc} Package}
\hypersetup{pdftitle={The childdoc Package}}
\author{Niklas Beisert\\[2ex]
  Institut f\"ur Theoretische Physik\\
  Eidgen\"ossische Technische Hochschule Z\"urich\\
  Wolfgang-Pauli-Strasse 27, 8093 Z\"urich, Switzerland\\[1ex]
  \href{mailto:nbeisert@itp.phys.ethz.ch}
  {\texttt{nbeisert@itp.phys.ethz.ch}}}
\hypersetup{pdfauthor={Niklas Beisert}}
\hypersetup{pdfsubject={Manual for the LaTeX2e Package childdoc}}
\date{30 December 2018, \textsf{v2.0}}
\maketitle

\begin{abstract}\noindent
\textsf{childdoc} is a \LaTeXe{} package
that enables the direct compilation
of document sections included by |\include|
to individual files.
\end{abstract}

\begingroup
\parskip0ex
\tableofcontents
\endgroup

%%%%%%%%%%%%%%%%%%%%%%%%%%%%%%%%%%%%%%%%%%%%%%%%%%%%%%%%%%%%%%%%%%%%%%%%%%%%%%%%
%%%%%%%%%%%%%%%%%%%%%%%%%%%%%%%%%%%%%%%%%%%%%%%%%%%%%%%%%%%%%%%%%%%%%%%%%%%%%%%%
\section{Introduction}

\LaTeX{} provides a mechanism to structure a large document (such as a book)
into a main file and several child files (containing the chapters)
using the |\include| command.
This mechanism is beneficial for documents
which span hundreds of pages in order to
make the source file(s) more manageable.
Moreover, compilation can be restricted to
selected child files by means of the |\includeonly| command.
The latter feature can be used to reduce the compilation time while editing
(this was significantly more useful in the earlier days of \LaTeX{})
or to generate a smaller document which is easier to navigate.
Another application of |\includeonly| is to generate
documents consisting of selected parts of the complete document.

However, there are a few drawbacks of the plain |\include| mechanism:
\begin{itemize}
\item
The child files cannot be compiled on their own,
they can only be compiled via the main file.
A naive editing environment
(such as a text editor with an option
to have the current file processed by \LaTeX)
may require one to switch to the main file before compiling;
attempting to compile the child file produces errors.
\item
The main file must be modified (each time)
to adjust the |\includeonly| command
to the present needs. This easily leaves the main file in a messy state.
\item
The generated document will always carry the filename
of the main document. This is inconvenient if
several child files are to be compiled and
to be kept for distribution.
\end{itemize}

The present package provides a simple interface
to make child files individually compilable by \LaTeX{}.
Compiling a child file then has the same effect as compiling
the main file with an |\includeonly| command
to select the appropriate child.
Moreover the generated document will carry the name of the child
rather than the main file.
This resolves all three above issues.

This feature is meant to make the editing of books,
thesis documents and lecture notes somewhat more convenient.
However, the package can also be used efficiently for
composing a series of documents (such as exercise sheets)
which are typically distributed individually.
It then assists the author in generating the individual documents
(potentially in different versions)
as well as a document containing the collected series.
Another application is in developing style files
or other kinds of included material
where compilation of the style file could redirect
to a sample or test file.

%%%%%%%%%%%%%%%%%%%%%%%%%%%%%%%%%%%%%%%%%%%%%%%%%%%%%%%%%%%%%%%%%%%%%%%%%%%%%%%%
%%%%%%%%%%%%%%%%%%%%%%%%%%%%%%%%%%%%%%%%%%%%%%%%%%%%%%%%%%%%%%%%%%%%%%%%%%%%%%%%
\section{Usage}

First of all, the package \textsf{childdoc} is \emph{not} a standard
\LaTeXe{} |.sty| style file! Therefore it needs to be invoked in
a non-standard way.

%%%%%%%%%%%%%%%%%%%%%%%%%%%%%%%%%%%%%%%%%%%%%%%%%%%%%%%%%%%%%%%%%%%%%%%%%%%%%%%%
\subsection{Included Files}
\label{sec:include}

%%%%%%%%%%%%%%%%%%%%%%%%%%%%%%%%%%%%%%%%
\DescribeMacro{\childdocmain}
To use the package, add the commands
\begin{center}
\begin{tabular}{l}
|\input{childdoc.def}|\\
|\childdocmain{}|\\
\end{tabular}
\end{center}
at the very top of the main \LaTeX{} file,
in particular \emph{before} the |\documentclass| statement!
The argument of |\childdocmain| should be left empty
(but it must be present).

%%%%%%%%%%%%%%%%%%%%%%%%%%%%%%%%%%%%%%%%
\DescribeMacro{\childdocof}
Furthermore, add the commands
\begin{center}
\begin{tabular}{l}
|\input{childdoc.def}|\\
|\childdocof{|\textit{main}|}|\\
\end{tabular}
\end{center}
at the top of every child file \textit{child}
which is included by |\include{|\textit{child}|}|
from within the main file
(or at least for those files to be compiled individually).
The argument \textit{main} must be the filename of the main file.

There are a couple of
considerations in setting up the main and child documents:

%%%%%%%%%%%%%%%%%%%%%%%%%%%%%%%%%%%%%%%%
\paragraph{Restrictions.}

Please note the following restrictions:
\begin{itemize}
\item
|\childdocmain| must be called with one argument \textit{main}
to ensure compatibility with earlier version of the package.
It must either be empty (|\childdocmain{}|)
or precisely match the filename of the main file in which it is specified.
See \secref{sec:detection} for further information.
\item
The filename \textit{main} must be specified without the |.tex| extension.
\item
The filename \textit{main} is case sensitive
(even in case-insensitive file systems)
due to internal string comparison.
\item
The argument \textit{main} should be fully expanded, it cannot be a macro.
\item
Subdirectories and special characters should be avoided in filenames.
\item
The command |\childdocmain{|\textit{main}|}| must be followed by a whitespace.
It should not be followed immediately by another command
or by a comment mark `|%|'.
This is because the \TeX{} parser reads the token immediately following
the argument of |\childdocmain| and puts it
at the beginning of every child section;
however, a white\-space is ignored.
\end{itemize}

%%%%%%%%%%%%%%%%%%%%%%%%%%%%%%%%%%%%%%%%
\paragraph{Content of Main File.}

It is advisable to place all content in the child files included by |\include|.
Any output contained in the main file will appear in all child documents
unless suppressed manually;
it cannot be suppressed automatically by the |\includeonly| directive
and thus should normally be avoided.
A method to include some content in the main file
by means of conditional processing is described in \secref{sec:conditional}.

%%%%%%%%%%%%%%%%%%%%%%%%%%%%%%%%%%%%%%%%
\paragraph{Page Numbering.}

When only a part of the document is compiled,
the appropriate numbering of pages
(as well as other status parameters)
is determined from the |.aux| files.
The latter contain information from previous passes.
However this information needs to propagate through
all intermediate child documents.
Therefore the page numbering in child documents may well
be inconsistent until the complete document is compiled at least once.

A useful (if unconventional) way to always ensure a consistent
page numbering is to restart the numbering in each child document
and denote the pages by `\textit{child}|.|\textit{page}'
where \textit{child} represents the chapter/section number of the child file.
This can be achieved by the command
|\numberwithin{page}{|\textit{child}|}|
of the \textsf{amsmath} package
where \textit{child} can be |chapter| or |section|
depending on the chosen structuring.
Alternatively, one can modify the macro |\thepage| appropriately
and reset the counter |page| at the start of each child file.

%%%%%%%%%%%%%%%%%%%%%%%%%%%%%%%%%%%%%%%%%%%%%%%%%%%%%%%%%%%%%%%%%%%%%%%%%%%%%%%%
\subsection{Conditional Processing}
\label{sec:conditional}

The package provides a mechanism to compile different versions
of a document. To customise the versions further some conditional processing
can come in handy to distinguish which version is being compiled.
The package provides two macros to describe the compilation context:

%%%%%%%%%%%%%%%%%%%%%%%%%%%%%%%%%%%%%%%%
\DescribeMacro{\ifchilddoc}
The conditional |\ifchilddoc| distinguishes between the compilation of
child documents and the main document:
%
\begin{center}
|\ifchilddoc |\textit{child-code}| |[|\||else |\textit{main-code}]| \||fi|
\end{center}

%%%%%%%%%%%%%%%%%%%%%%%%%%%%%%%%%%%%%%%%
\DescribeMacro{\childdocname}
\DescribeMacro{\childdocjob}
The macro |\childdocname| contains the filename (without extension)
of the main or child file being processed.
Note that |\childdocjob| will always contain the name of the main file.

%%%%%%%%%%%%%%%%%%%%%%%%%%%%%%%%%%%%%%%%
\paragraph{Title Page.}

Conditional processing can be used to include a title or banner page
in the main document when proper precautions are taken.
Importantly, the code in the main file should ensure that the page counter
(as well as other status parameters which are stored in the |.aux| files)
takes the same value after the conditional processing.
Otherwise the page numbers may take divergent values
depending on which part is compiled.

For example, a title page could be declared by:
%
\begin{center}
\begin{tabular}{l}
|\ifchilddoc\||else|\\
|\addtocounter{page}{-1}|\\
\textit{code for title page}\\
|\newpage|\\
|\||fi|
\end{tabular}
\end{center}
%
A banner page for the child documents can be generated by:
%
\begin{center}
\begin{tabular}{l}
|\ifchilddoc|\\
|\addtocounter{page}{-1}|\\
\textit{code for banner page}\\
|\newpage|\\
|\||fi|
\end{tabular}
\end{center}
%
Here one could write a message such as:
\begin{center}
|This is the part \childdocname{} of \childdocjob{}.|
\end{center}

%%%%%%%%%%%%%%%%%%%%%%%%%%%%%%%%%%%%%%%%%%%%%%%%%%%%%%%%%%%%%%%%%%%%%%%%%%%%%%%%
\subsection{Flags}
\label{sec:flags}

The package makes it easy to generate different versions
of the main or child documents.
To this end compilation flags can be defined
and assigned different default values.
They will be particularly useful in conjunction
with the forwarding mechanism described in \secref{sec:forward}.

For example, it may be useful to have a flag |\version|
which can be set to |draft| or |final|.
The document source will contain some conditional code
depending on the value of |\version|.
Suppose further, the flag should default to |final| for the main file
and to |draft| for child files
which is a natural assignment for editing the document.
This is achieved by placing the following code
in the preamble of the main document
(below the |\childdocmain| directive):
%
\begin{center}
\begin{tabular}{l}
|\ifchilddoc|\\
|\providecommand{\version}{draft}|\\
|\||else|\\
|\providecommand{\version}{final}|\\
|\||fi|
\end{tabular}
\end{center}
%
The definition by |\providecommand| makes sure
that previous definitions are not overwritten.
Further statements |\providecommand{\version}{...}|
can thus be added before the above code to override it.

For the main file, one might add a line
(between |\childdocmain| and the above block)
%
\begin{center}
|%\ifchilddoc\||else\providecommand{\version}{draft}\||fi|
\end{center}
%
which can be uncommented to produce a draft version.
Likewise one can add a line to the very top of a child file
(above the |\childdocof{|\textit{main}|}| directive)
%
\begin{center}
|%\providecommand{\version}{final}|
\end{center}
%
which can be uncommented to produce the final version of this child document.

%%%%%%%%%%%%%%%%%%%%%%%%%%%%%%%%%%%%%%%%%%%%%%%%%%%%%%%%%%%%%%%%%%%%%%%%%%%%%%%%
\subsection{Forwarding}
\label{sec:forward}

Different versions of the main or child documents
using compilation flags as described in \secref{sec:flags}
can be (permanently) stored in different files
for convenient compilation, viewing and distribution.
To this end, the package defines a command
to pass on compilation to a different file:

%%%%%%%%%%%%%%%%%%%%%%%%%%%%%%%%%%%%%%%%
\DescribeMacro{\childdocforward}
The command |\childdocforward| redirects processing to
another source file:
%
\begin{center}
\begin{tabular}{l}
|\input{childdoc.def}|\\
|\childdocforward[|\textit{main}|]{|\textit{dest}|}|\\
\end{tabular}
\end{center}
%
The argument \textit{dest} is the destination file
(without extension).
It should be the main file or one of the child files.
Note that further \textsf{childdoc} directives
such as |\childdocof| and |\childdocforward|
in the indicated file will be processed in this form.
The optional argument \textit{main}
passes on directly to the main file \textit{main}
while pretending to compile the child \textit{dest}.
This form behaves as if \textit{dest}
issues |\childdocof{|\textit{main}|}| right away,
and no further \textsf{childdoc} directives will be processed.

%%%%%%%%%%%%%%%%%%%%%%%%%%%%%%%%%%%%%%%%
\DescribeMacro{\...prefix}
In the alternative form |\childdocforwardprefix|,
%
\begin{center}
\begin{tabular}{l}
|\input{childdoc.def}|\\
|\childdocforwardprefix[|\textit{main}|]{|\textit{prefix}|}{|\textit{dest}|}|
\end{tabular}
\end{center}
%
the destination file is determined by a pattern
depending on the current file:
To make this work, the current file must be called
`{\textit{prefix}\hspace{0.2em}\textit{suffix}}'
with \textit{prefix} matching precisely the argument.
Processing is then passed on to the file
`{\textit{dest}\hspace{0.2em}\textit{suffix}}'.
Surely, the same effect is achieved by
directly specifying the
argument `{\textit{dest}\hspace{0.2em}\textit{suffix}}'
in the first form.
However, that requires to set up a different file
for each child. With the alternative form of the command
all these files can have exactly the same content
which simplifies setting them up and maintaining them.

For example, the following file |draft.tex|
with a compilation flag |\version| as described in \secref{sec:flags}
compiles the main document as a draft:
%
\begin{center}
\begin{tabular}{l}
|\def\version{draft}|\\
|\input{childdoc.def}|\\
|\childdocforward{|\textit{main}|}|
\end{tabular}
\end{center}
%
Likewise, the following files |final|\textit{nn}|.tex|
compile the final version of the child document
|child|\textit{nn}|.tex|:
%
\begin{center}
\begin{tabular}{l}
|\def\version{final}|\\
|\input{childdoc.def}|\\
|\childdocforwardprefix{final}{child}|
\end{tabular}
\end{center}
%

Note that when several versions of a main file and/or of each child file
are to be generated, it may be convenient to set up a |Makefile| or
shell script to automatise the process.

%%%%%%%%%%%%%%%%%%%%%%%%%%%%%%%%%%%%%%%%%%%%%%%%%%%%%%%%%%%%%%%%%%%%%%%%%%%%%%%%
\subsection{Command Line Processing}
\label{sec:commandline}

The effect of redirection files can also be achieved by invoking
the \LaTeX{} compiler with a more elaborate command line.
Most conveniently this should be done as part
of a shell script or a |Makefile|.

When using \textsf{childdoc} in the main file, the following
command lines effectively perform a redirection
(note that depending on the shell being used,
backslashes may have to be doubled: `|\|' $\to$ `|\\|'):
%
\begin{center}
|... -jobname "|\textit{target}|" |\\|"|[\textit{flags}]%
|\input{childdoc.def}\childdocforward[|\textit{main}|]{|\textit{dest}|}"|
\end{center}
%
Here \textit{target} is the name of the output file,
\textit{main} is the name of the main file
and \textit{dest} is the name of the main or child file to be processed
(all filenames without extensions).
The optional argument \textit{main} can be omitted
if \textit{main} matches \textit{dest}.
Optionally, compilation \textit{flags} can be defined via |\def| commands.
This command line makes the \TeX{} engine believe
it is compiling the file \textit{target}
whose content is specified as the latter parameter.
The provided code then forwards the processing to
\textit{main} or \textit{dest} as described in \secref{sec:forward}.

%%%%%%%%%%%%%%%%%%%%%%%%%%%%%%%%%%%%%%%%%%%%%%%%%%%%%%%%%%%%%%%%%%%%%%%%%%%%%%%%
\subsection{Include by Input}
\label{sec:input}

Including child documents by |\include| has some restrictions by design.
Most notably, the content of a child document always occupies
its own set of pages; pages cannot be shared between child documents.
Usually, this behaviour makes perfect sense
because each child document contain an essential part of the document.
However, in some situations it may be desirable to compose
a document from a collection of parts
without having mandatory page breaks between then.
For this case, the package
provides a mechanism to include parts
by |\input| which can also be processed individually.
However, by construction this mechanism
requires manual handling of the content to be output.

%%%%%%%%%%%%%%%%%%%%%%%%%%%%%%%%%%%%%%%%
\DescribeMacro{\ifchilddocmanual}
The main file should be prepared as usual, see \secref{sec:include}.
However, the document body must make a distinction
between processing of an individual part and of the main document, e.g.:
%
\begin{center}
\begin{tabular}{l}
|\ifchilddocmanual|\\
|\input{\childdocname}|\\
|\||else|\\
\textit{document body with }|\input{|\textit{part}|}|\\
|\||fi|
\end{tabular}
\end{center}
%
The conditional |\ifchilddocmanual| is true whenever
a part to be included by |\input| is being compiled,
and the name of the part is stored in |\childdocname|.

%%%%%%%%%%%%%%%%%%%%%%%%%%%%%%%%%%%%%%%%
\DescribeMacro{\childdocby}
Each part to be included by |\input| should start with:
%
\begin{center}
\begin{tabular}{l}
|\input{childdoc.def}|\\
|\childdocby{|\textit{main}|}|\\
\end{tabular}
\end{center}
%
The directive |\childdocby| is similar to |\childdocof|
described in \secref{sec:include},
but the subsequent selection of content must be done manually.
To that end, both |\ifchilddoc| and |\ifchilddocmanual|
will be true upon processing of a part,
and the name of the part is stored in |\childdocname|.
Note that |\jobname| will be set to the filename of the current part
so that each part receives an individual |.aux| file
that does not interfere with the |.aux| file(s) of the main document.
This behaviour can be altered by the alternative form
|\childdocby[*]{|\textit{main}|}| (with a non-empty optional argument)
which uses the |.aux| file of the main document
by setting |\jobname| to \textit{main}.

%%%%%%%%%%%%%%%%%%%%%%%%%%%%%%%%%%%%%%%%%%%%%%%%%%%%%%%%%%%%%%%%%%%%%%%%%%%%%%%%
\subsection{Driver Development}
\label{sec:driver}

The \textsf{childdoc} mechanism can also be use for the development
of definition files such as \LaTeX{} styles or classes.
This case differs from the above setup with multiple parts
included by |\include| in that no |\includeonly| should be invoked.
This can be achieved by starting the include file
(before |\ProvidesPackage|) with:
%
\begin{center}
\begin{tabular}{l}
|\input{childdoc.def}|\\
|\childdocforward{|\textit{main}|}|\\
\end{tabular}
\end{center}
%
or alternatively with:
%
\begin{center}
\begin{tabular}{l}
|\input{childdoc.def}|\\
|\childdocby{|\textit{main}|}|\\
\end{tabular}
\end{center}
%
Both forms have slightly different effects as described above.
The main file is prepared as usual, see \secref{sec:include}.

%%%%%%%%%%%%%%%%%%%%%%%%%%%%%%%%%%%%%%%%%%%%%%%%%%%%%%%%%%%%%%%%%%%%%%%%%%%%%%%%
\subsection{Legacy Detection}
\label{sec:detection}

The directive |\childdocmain| in the main file can detect
whether the complete document or merely a child is to be compiled
even without using the directive |\childdocof|.
This method is deprecated because it is less robust
and there is no compelling reason to use it;
it is merely provided for backward compatibility
and it may be removed in future versions.

If the detection mechanism is to be used,
it is mandatory to correctly specify
the filename of the main file as the argument of |\childdocmain|:
%
\begin{center}
\begin{tabular}{l}
|\input{childdoc.def}|\\
|\childdocmain{|\textit{main}|}|\\
\end{tabular}
\end{center}
%
If |\jobname| does not match the argument \textit{main} of |\childdocmain|,
it is assumed that |\jobname| points to the child file to be compiled.
When using |\childdocmain| with the main file specified as argument,
it suffices to start a child file
with just |\input{|\textit{main}|}|
without loading of the package and using |\childdocof|.
If instead all processing is done
with the appropriate \textsf{childdoc} directives,
the argument of \textit{main} of |\childdocmain| can be empty.

An alternative version of the command line processing described
in \secref{sec:commandline} using the detection mechanism reads:
%
\begin{center}
|... -jobname "|\textit{target}|" "|[\textit{flags}]%
[|\def\jobname{|\textit{dest}|}|]|\input{|\textit{main}|}"|
\end{center}

%%%%%%%%%%%%%%%%%%%%%%%%%%%%%%%%%%%%%%%%%%%%%%%%%%%%%%%%%%%%%%%%%%%%%%%%%%%%%%%%
\subsection{Manual Code}
\label{sec:manual}

In case one cannot be certain whether the definitions file |childdoc.def|
is installed on the target \TeX{} distribution
and one prefers not to ship it,
it is conceivable to paste a few relevant commands into the sources.

To that end, drop all statements |\input{childdoc.def}|
and perform the replacements as outlined below.
Instead of |\childdocmain{|\textit{main}|}| add the following code
to the top of the main file:
%
\begin{center}
\begin{tabular}{l}
|\||ifdefined\childdocname\endinput\||fi\newif\ifchilddoc|\\
|\edef\childdocname{\scantokens\expandafter{\jobname\noexpand}}|\\
|\def\childdocmain{|\textit{main}|}\||ifx\childdocmain\childdocname\||else|\\
|\childdoctrue\includeonly{\childdocname}\let\jobname\childdocmain\||fi|\\
\end{tabular}
\end{center}
%
Instead of |\childdocof{|\textit{main}|}| just include the main file
at the top of each child file:
%
\begin{center}
|\input{|\textit{main}|}|
\end{center}
%
A simple redirection |\childdocforward{|\textit{dest}|}| is achieved by:
%
\begin{center}
|\def\jobname{|\textit{dest}|}\input{\jobname}|
\end{center}
%
The redirection with prefix
|\childdocforwardprefix[|\textit{prefix}|]{|\textit{dest}|}|
is accomplished by:
%
\begin{center}
\begin{tabular}{l}
|{\edef\jobname{\scantokens\expandafter{\jobname\noexpand}}|\\
|\def\redirectjob |\textit{prefix}|#1~~~{\gdef\jobname{|\textit{dest}|#1}}|\\
|\expandafter\redirectjob\jobname~~~}\input{\jobname}|
\end{tabular}
\end{center}

In an alternative approach,
child documents can be compiled by a specific command line
without additional code or specific definitions:
%
\begin{center}
|... -jobname "|\textit{target}|" "|[\textit{flags}]%
|\includeonly{|\textit{dest}|}\input{|\textit{main}|}"|
\end{center}
%

%%%%%%%%%%%%%%%%%%%%%%%%%%%%%%%%%%%%%%%%%%%%%%%%%%%%%%%%%%%%%%%%%%%%%%%%%%%%%%%%
%%%%%%%%%%%%%%%%%%%%%%%%%%%%%%%%%%%%%%%%%%%%%%%%%%%%%%%%%%%%%%%%%%%%%%%%%%%%%%%%
\section{Information}

%%%%%%%%%%%%%%%%%%%%%%%%%%%%%%%%%%%%%%%%%%%%%%%%%%%%%%%%%%%%%%%%%%%%%%%%%%%%%%%%
\subsection{Copyright}

Copyright \copyright{} 2017--2018 Niklas Beisert

This work may be distributed and/or modified under the
conditions of the \LaTeX{} Project Public License, either version 1.3
of this license or (at your option) any later version.
The latest version of this license is in
  \url{http://www.latex-project.org/lppl.txt}
and version 1.3 or later is part of all distributions of \LaTeX{}
version 2005/12/01 or later.

This work has the LPPL maintenance status `maintained'.

The Current Maintainer of this work is Niklas Beisert.

This work consists of the files |README.txt|, |childdoc.ins| and |childdoc.dtx|
as well as the derived files |childdoc.def|, |cdocsamp.tex|
with |cdocsch1.tex|, |cdocsch2.tex|, |cdocspt3.tex|, |cdocspt4.tex|,
|cdocsdrf.tex|, |cdocsfn1.tex|, |cdocsfn2.tex|
as well as |childdoc.pdf|.

%%%%%%%%%%%%%%%%%%%%%%%%%%%%%%%%%%%%%%%%%%%%%%%%%%%%%%%%%%%%%%%%%%%%%%%%%%%%%%%%
\subsection{Files and Installation}

The package consists of the files:
%
\begin{center}
\begin{tabular}{ll}
    |README.txt|   & readme file \\
    |childdoc.ins| & installation file \\
    |childdoc.dtx| & source file \\
    |childdoc.def| & definition file \\
    |cdocsamp.tex| & sample main file \\
    |cdocsch1.tex| & sample include file \\
    |cdocsch2.tex| & sample include file \\
    |cdocspt3.tex| & sample part file \\
    |cdocspt4.tex| & sample part file \\
    |cdocsdrf.tex| & sample redirection file \\
    |cdocsfn1.tex| & sample redirection file \\
    |cdocsfn2.tex| & sample redirection file \\
    |childdoc.pdf| & manual
\end{tabular}
\end{center}
%
The distribution consists of the files
|README.txt|, |childdoc.ins| and |childdoc.dtx|.
%
\begin{itemize}
\item
Run (pdf)\LaTeX{} on |childdoc.dtx|
to compile the manual |childdoc.pdf| (this file).
\item
Run \LaTeX{} on |childdoc.ins| to create the definitions file |childdoc.def|
and the sample |cdocsamp.tex| with include files
|cdocsch1.tex|, |cdocsch2.tex|, |cdocspt3.tex|, |cdocspt4.tex|,
|cdocsdrf.tex|, |cdocsfn1.tex|, |cdocsfn2.tex|.
Then copy the file |childdoc.def| to an appropriate directory of your \LaTeX{}
distribution, e.g.\ \textit{texmf-root}|/tex/latex/childdoc|.
\end{itemize}

%%%%%%%%%%%%%%%%%%%%%%%%%%%%%%%%%%%%%%%%%%%%%%%%%%%%%%%%%%%%%%%%%%%%%%%%%%%%%%%%
\subsection{Related CTAN Packages}

There are several other packages which offer a similar functionality:
%
\begin{itemize}
\item
The packages
\href{http://ctan.org/pkg/docmute}{\textsf{docmute}},
\href{http://ctan.org/pkg/includex}{\textsf{includex}} and
\href{http://ctan.org/pkg/standalone}{\textsf{standalone}}
provide commands to include only the document body of
a child file thus allowing both files to be compiled individually.
\item
The packages \href{http://ctan.org/pkg/subdocs}{\textsf{subdocs}}
and \href{http://ctan.org/pkg/subfiles}{\textsf{subfiles}}
provide structures in which the main and child documents can be
encapsulated and allowing them to be compiled individually.
The inclusion mechanism is different from the conventional |\include|.
\item
The package \href{http://ctan.org/pkg/combine}{\textsf{combine}}
is an elaborate solution to combine several documents into one.
\end{itemize}
%
See also the CTAN topic \href{http://ctan.org/topic/subdocs}{\textsf{subdocs}}
for further related packages.
The present package differs from the above solutions in that
a document structure constructed with the conventional |\include| mechanism
just needs two extra commands at the top of every file
such that all constituent files can be compiled individually.

%%%%%%%%%%%%%%%%%%%%%%%%%%%%%%%%%%%%%%%%%%%%%%%%%%%%%%%%%%%%%%%%%%%%%%%%%%%%%%%%
%\subsection{Feature Suggestions}
%
%The following is a list of features which may be useful for future
%versions of this package:
%%
%\begin{itemize}
%\item
%\ldots
%\end{itemize}

%%%%%%%%%%%%%%%%%%%%%%%%%%%%%%%%%%%%%%%%%%%%%%%%%%%%%%%%%%%%%%%%%%%%%%%%%%%%%%%%
\subsection{Revision History}

%%%%%%%%%%%%%%%%%%%%%%%%%%%%%%%%%%%%%%%%
\paragraph{v2.0:} 2018/12/30

\begin{itemize}
\item
immediate forward processing
\item
added |\childdocby| mechanism
\item
manual restructured
\end{itemize}

%%%%%%%%%%%%%%%%%%%%%%%%%%%%%%%%%%%%%%%%
\paragraph{v1.6:} 2018/01/17

\begin{itemize}
\item
application for development of include files
\item
corrections to manual
\end{itemize}

%%%%%%%%%%%%%%%%%%%%%%%%%%%%%%%%%%%%%%%%
\paragraph{v1.5:} 2017/05/21

\begin{itemize}
\item
more complete structuring introduced
\item
|\childdocof| introduced
\item
|\childdoc| renamed to |\childdocmain|
\item
|\childredirect| renamed to |\childdocforward| and |\childdocforwardprefix|
and functionality expanded
\end{itemize}

%%%%%%%%%%%%%%%%%%%%%%%%%%%%%%%%%%%%%%%%
\paragraph{v1.0:} 2017/04/27

\begin{itemize}
\item
manual and install package
\item
first version published on CTAN
\end{itemize}

%%%%%%%%%%%%%%%%%%%%%%%%%%%%%%%%%%%%%%%%
\paragraph{v0.6:} 2017/04/26

\begin{itemize}
\item
redirection mechanism added
\end{itemize}

%%%%%%%%%%%%%%%%%%%%%%%%%%%%%%%%%%%%%%%%
\paragraph{v0.5:} 2017/04/26

\begin{itemize}
\item
functionality in definition file
\end{itemize}


%%%%%%%%%%%%%%%%%%%%%%%%%%%%%%%%%%%%%%%%%%%%%%%%%%%%%%%%%%%%%%%%%%%%%%%%%%%%%%%%
%%%%%%%%%%%%%%%%%%%%%%%%%%%%%%%%%%%%%%%%%%%%%%%%%%%%%%%%%%%%%%%%%%%%%%%%%%%%%%%%
%%%%%%%%%%%%%%%%%%%%%%%%%%%%%%%%%%%%%%%%%%%%%%%%%%%%%%%%%%%%%%%%%%%%%%%%%%%%%%%%
\appendix

\settowidth\MacroIndent{\rmfamily\scriptsize 000\ }

 \DocInput{childdoc.dtx}

\end{document}
%</driver>
% \fi
%
% %%%%%%%%%%%%%%%%%%%%%%%%%%%%%%%%%%%%%%%%%%%%%%%%%%%%%%%%%%%%%%%%%%%%%%%%%%%%%%
% %%%%%%%%%%%%%%%%%%%%%%%%%%%%%%%%%%%%%%%%%%%%%%%%%%%%%%%%%%%%%%%%%%%%%%%%%%%%%%
% \section{Sample}
%\iffalse
%<*samplemain>
%\fi
%
% The following presents a sample document
% with two chapters, two parts, a title page,
% a compile flag as well as three forwarding files to set the flag.
% It consists of eight |.tex| files:
% \begin{center}
% \begin{tabular}{ll}
% |cdocsamp.tex|&main file\\
% |cdocsch1.tex|&include file for chapter 1\\
% |cdocsch2.tex|&include file for chapter 2\\
% |cdocspt3.tex|&include file for part 3\\
% |cdocspt4.tex|&include file for part 4\\
% |cdocsdrf.tex|&forwarding file for main file in draft mode\\
% |cdocsfi1.tex|&forwarding file for final version of chapter 1\\
% |cdocsfi2.tex|&forwarding file for final version of chapter 2\\
% \end{tabular}
% \end{center}
% Each of the eight files can be compiled directly by the \LaTeX{} compiler.
%
% %%%%%%%%%%%%%%%%%%%%%%%%%%%%%%%%%%%%%%
% \paragraph{Main File.}
%
% The main file is called |cdocsamp.tex|.
%
% Load the \textsf{childdoc} definitions and
% declare the filename for the main document:
%    \begin{macrocode}
\input{childdoc.def}
\childdocmain{}
%    \end{macrocode}

% Optional override for |\version| flag:
%    \begin{macrocode}
%%\ifchilddoc\else\providecommand{\version}{draft}\fi
%    \end{macrocode}

% Define the default values for the |\version| flag
% (|final| for the main file and |draft| for childs):
%    \begin{macrocode}
\ifchilddoc
\providecommand{\version}{draft}
\else
\providecommand{\version}{final}
\fi
%    \end{macrocode}

% Load the standard document class:
%    \begin{macrocode}
\documentclass[12pt]{article}
%    \end{macrocode}

% Start the document body:
%    \begin{macrocode}
\begin{document}
%    \end{macrocode}

% Declare a title page.
% Print title, part of document being processed and version flag:
%    \begin{macrocode}
\addtocounter{page}{-1}
\begin{center}
{\LARGE\bfseries{}childdoc example\par}
\vspace{1cm}
\ifchilddoc
\ifchilddocmanual part\else chapter\fi:
`\childdocname' of `\childdocjob'\par
\else
main document: `\childdocjob'\par
\fi
version: \version\par
\end{center}
\newpage
%    \end{macrocode}

% Manually include selected file,
% otherwise process as usual:
%    \begin{macrocode}
\ifchilddocmanual
\section*{part `\childdocname'}
\input{\childdocname}
\else
%    \end{macrocode}

% Include the two chapters:
%    \begin{macrocode}
\include{cdocsch1}
\include{cdocsch2}
%    \end{macrocode}

% Include the two parts unless only chapters should be displayed:
%    \begin{macrocode}
\ifchilddoc\else
\section{part three}
\input{cdocspt3}
\section{part four}
\input{cdocspt4}
\fi
%    \end{macrocode}

% Process as usual until here:
%    \begin{macrocode}
\fi
%    \end{macrocode}

% End of document body:
%    \begin{macrocode}
\end{document}
%    \end{macrocode}
%\iffalse
%</samplemain>
%\fi
%
% %%%%%%%%%%%%%%%%%%%%%%%%%%%%%%%%%%%%%%
% \paragraph{Chapter Include Files.}
%
% The include files are called |cdocsch1.tex| and |cdocsch2.tex|.
%
%\iffalse
%<*samplechap1|samplechap2>
%\fi

% Optional override for |\version| flag:
%    \begin{macrocode}
%%\providecommand{\version}{final}
%    \end{macrocode}

% Include the main document:
%    \begin{macrocode}
\input{childdoc.def}
\childdocof{cdocsamp}
%    \end{macrocode}

%\iffalse
%</samplechap1|samplechap2>
%\fi
%
%\iffalse
%<*samplechap1>
%\fi
% Some text for chapter 1:
%    \begin{macrocode}
\section{one}
some text in chapter one
%    \end{macrocode}

%\iffalse
%</samplechap1>
%\fi
% Some text for chapter 2:
%\iffalse
%<*samplechap2>
%\fi
%    \begin{macrocode}
\section{two}
more text in chapter two
%    \end{macrocode}

%\iffalse
%</samplechap2>
%\fi
%
% %%%%%%%%%%%%%%%%%%%%%%%%%%%%%%%%%%%%%%
% \paragraph{Part Include Files.}
%
% The include files are called |cdocspt3.tex| and |cdocspt4.tex|.
%
%\iffalse
%<*samplepart3|samplepart4>
%\fi

% Optional override for |\version| flag:
%    \begin{macrocode}
%%\providecommand{\version}{final}
%    \end{macrocode}

% Include the main document:
%    \begin{macrocode}
\input{childdoc.def}
\childdocby{cdocsamp}
%    \end{macrocode}

%\iffalse
%</samplepart3|samplepart4>
%\fi
%
%\iffalse
%<*samplepart3>
%\fi
% Some text for part 3:
%    \begin{macrocode}
some text in part three
%    \end{macrocode}

%\iffalse
%</samplepart3>
%\fi
% Some text for part 4:
%\iffalse
%<*samplepart4>
%\fi
%    \begin{macrocode}
more text in part four
%    \end{macrocode}

%\iffalse
%</samplepart4>
%\fi
%
% %%%%%%%%%%%%%%%%%%%%%%%%%%%%%%%%%%%%%%
% \paragraph{Forwarding for a Complete Draft.}
%
% The following forwarding file |cdocsdrf.tex|
% compiles the main document in draft mode:
%\iffalse
%<*sampledraft>
%\fi
%    \begin{macrocode}
\def\version{draft}
\input{childdoc.def}
\childdocforward{cdocsamp}
%    \end{macrocode}

%\iffalse
%</sampledraft>
%\fi
%
% %%%%%%%%%%%%%%%%%%%%%%%%%%%%%%%%%%%%%%
% \paragraph{Forwarding for Final Version of the Chapters.}
%
% The following forwarding files |cdocsfn1.tex| and |cdocsfn2.tex|
% (with identical content)
% compile the final versions of the child documents
% |cdocsch1.tex| and |cdocsch2.tex|, respectively:
%\iffalse
%<*samplefinal>
%\fi
%    \begin{macrocode}
\def\version{final}
\input{childdoc.def}
\childdocforwardprefix[cdocsamp]{cdocsfn}{cdocsch}
%    \end{macrocode}

%\iffalse
%</samplefinal>
%\fi
%
% %%%%%%%%%%%%%%%%%%%%%%%%%%%%%%%%%%%%%%
% \paragraph{Command Line Processing.}
%
% The following three command lines generate the output files
% |cdocscld|, |cdocscl1| and |cdocscl2|
% which should be identical to
% |cdocsdrf|, |cdocsch1| and |cdocsfn2|, respectively:
% \begin{center}
% \begin{tabular}{l}
% |latex -jobname cdocscld \|\\
% |  "\def\version{draft}\input{childdoc.def}\childdocforward{cdocsamp}"|\\
% |latex -jobname cdocscl1 \|\\
% |  "\input{childdoc.def}\childdocforward[cdocsamp]{cdocsch1}"|\\
% |latex -jobname cdocscl2 \|\\
% |  "\def\version{final}\input{childdoc.def}\childdocforward{cdocsch2}"|
% \end{tabular}
% \end{center}
% Note that the trailing backslash on each first line
% merely continues the input to the second line
% (for convenient cut ant paste).
% Furthermore, the command |latex| can be replaced by any
% of its alternative versions such as |pdflatex|.
%
% %%%%%%%%%%%%%%%%%%%%%%%%%%%%%%%%%%%%%%%%%%%%%%%%%%%%%%%%%%%%%%%%%%%%%%%%%%%%%%
% %%%%%%%%%%%%%%%%%%%%%%%%%%%%%%%%%%%%%%%%%%%%%%%%%%%%%%%%%%%%%%%%%%%%%%%%%%%%%%
% \section{Implementation}
%\iffalse
%<*package>
%\fi
%
% This section describes the definitions file |childdoc.def|.

% The definitions cannot be loaded using |\usepackage| or |\RequirePackage|
% which has a mechanism to prevent loading a style file more than once.
% When loading the definitions by means of |\input|
% multiple instances have to be prevented manually:
%\iffalse
%This code needs to be before the `\ProvidesFile' directive
%which is defined at the beginning of this file.
%Therefore it is also placed there and commented out here.
%</package>
%<*discard>
%\fi
%    \begin{macrocode}
\ifdefined\childdocmain\endinput\fi
%    \end{macrocode}
%\iffalse
%</discard>
%<*package>
%\fi
%
% \macro{\ifchilddoc}
% \macro{\ifchilddocmanual}
% The conditional |\ifchilddoc| tells whether a
% child (true) or main (false) document is being compiled.
% The conditional |\ifchilddocmanual| tells whether
% the |\includeonly| mechanism is used (false) or
% the selection of child files must be performed manually (true).
% The definitions initialise to false:
%    \begin{macrocode}
\newif\ifchilddoc
\newif\ifchilddocmanual
%    \end{macrocode}

% \macro{\childdocname}
% \macro{\childdocjob}
% The macro |\childdocname| stores the name of the main document
% to be compiled. The macro |\childdocjob| stores the name of
% the document on which the \LaTeX{} compiler was originally invoked.
% The content of |\jobname| cannot be compared
% to filenames specified in the source due to different catcodes.
% The following code rescans |\jobname|, stores the result
% in |\childdocname| and saves a copy in |\childdocjob|:
%    \begin{macrocode}
\edef\childdocname{\scantokens\expandafter{\jobname\noexpand}}
\let\childdocjob\childdocname
%    \end{macrocode}

% \macro{\childdocdisable}
% The macro |\childdocdisable| prevents the main file
% from being processed more than once.
% At this stage, the main document command |\childdocmain|
% is assumed to be called once again where it should do nothing.
% Any subsequent call to it should prevent
% a secondary processing of the main document
% It overwrites the forwarding commands
% |\childdocof| and |\childdocforward|
% with empty macros to prevent further inclusions of the main document:
%    \begin{macrocode}
\newcommand{\childdocdisable}
{
  \renewcommand{\childdocmain}[1]{\renewcommand{\childdocmain}[1]{\endinput}}
  \renewcommand{\childdocof}[1]{}
  \renewcommand{\childdocby}[2][]{}
  \renewcommand{\childdocforward}[2][]{}
  \renewcommand{\childdocdisable}{}
}
%    \end{macrocode}

% \macro{\childdocmain}
% The macro |\childdocmain| is to be called at the top of the main file
% with nothing or the main filename (without extension) as argument.
% First, it breaks loops.
% If the argument is not empty and does not match |\childdocname|
% (which is set by the first inclusion of |childdoc.def|),
% |\ifchilddoc| is set to true, |\includeonly| is applied to the child file
% and |\jobname| is set to the main file
% (for proper handling of |.aux| files):
%    \begin{macrocode}
\newcommand{\childdocmain}[1]
{
  \childdocdisable\childdocmain{}
  \if?#1?\else
    \begingroup
      \def\childdoctmp{#1}
      \ifx\childdoctmp\childdocname
        \def\childdoctmp{}
      \else
        \def\childdoctmp
        {
          \childdoctrue
          \includeonly{\childdocname}
          \def\childdocjob{#1}
          \def\jobname{#1}
        }
      \fi
      \expandafter
    \endgroup
    \childdoctmp
  \fi
}
%    \end{macrocode}

% \macro{\childdocof}
% The command |\childdocof| redirects
% compilation to the main file |#1|.
%    \begin{macrocode}
\newcommand{\childdocof}[1]
{
  \childdocdisable
  \childdoctrue
  \includeonly{\childdocname}
  \def\jobname{#1}
  \def\childdocjob{#1}
  \input{#1}
}
%    \end{macrocode}

% \macro{\childdocby}
% The command |\childdocby| ....
%    \begin{macrocode}
\newcommand{\childdocby}[2][]
{
  \childdocdisable
  \childdoctrue
  \childdocmanualtrue
  \if?#1?\else
    \def\jobname{#2}
  \fi
  \def\childdocjob{#2}
  \input{#2}
  \endinput
}
%    \end{macrocode}

% \macro{\childdocforward}
% The command |\childdocforward| redirects
% compilation to the main file or
% (if the optional argument is given) a child file.
% Parameters are set as if the main file
% or a child file starting with |\childdocof| was compiled.
% Then compilation is handed over to the main file:
%    \begin{macrocode}
\newcommand{\childdocforward}[2][]
{
  \begingroup
    \if?#1?
      \def\childdoctmp
      {
        \def\childdocname{#2}
        \def\childdocjob{#2}
        \def\jobname{#2}
        \input{#2}
        \endinput
      }
    \else
      \def\childdoctmp
      {
        \childdocdisable
        \def\childdocname{#2}
        \childdoctrue
        \includeonly{#2}
        \def\childdocjob{#1}
        \def\jobname{#1}
        \input{#1}
        \endinput
      }
    \fi
    \expandafter
  \endgroup
  \childdoctmp
}
%    \end{macrocode}

% \macro{\childdocforwardprefix}
% The command |\childdocforwardprefix| redirects
% compilation to the main or a child file by means of a pattern.
% The prefix |#1| in the current filename is replaced by |#2|
% and the suffix of the current filename is kept
% (it is assumed that the filename does not contain the substring `|~~~|'
% which is used as a delimiter).
% Compilation is handed over to the new file by |\childdocforward|:
%    \begin{macrocode}
\newcommand{\childdocforwardprefix}[3][]
{
  \begingroup
    \def\childdocextract #2##1~~~{\def\childdoctmp{\childdocforward[#1]{#3##1}}}
    \expandafter\childdocextract\childdocname~~~
    \expandafter
  \endgroup
  \childdoctmp
}
%    \end{macrocode}

% \macro{\childdoc}
% The deprecated macro |\childdoc| is a legacy version of |\childdocmain|:
%    \begin{macrocode}
\newcommand{\childdoc}{\childdocmain}
%    \end{macrocode}

% \macro{\childdocredirect}
% The deprecated macro |\childdocredirect| is a legacy version
% of |\childdocforward| and |\childdocforwardprefix|:
%    \begin{macrocode}
\newcommand{\childdocredirect}[2][]
{
  \begingroup
    \if?#1?
      \def\childdoctmp{\childdocforward{#2}}
    \else
      \def\childdoctmp{\childdocforwardprefix{#1}{#2}}
    \fi
    \expandafter
  \endgroup
  \childdoctmp
}
%    \end{macrocode}

%\iffalse
%</package>
%\fi
%
\endinput
|\\
|\childdocmain{|\textit{main}|}|\\
\end{tabular}
\end{center}
%
If |\jobname| does not match the argument \textit{main} of |\childdocmain|,
it is assumed that |\jobname| points to the child file to be compiled.
When using |\childdocmain| with the main file specified as argument,
it suffices to start a child file
with just |\input{|\textit{main}|}|
without loading of the package and using |\childdocof|.
If instead all processing is done
with the appropriate \textsf{childdoc} directives,
the argument of \textit{main} of |\childdocmain| can be empty.

An alternative version of the command line processing described
in \secref{sec:commandline} using the detection mechanism reads:
%
\begin{center}
|... -jobname "|\textit{target}|" "|[\textit{flags}]%
[|\def\jobname{|\textit{dest}|}|]|\input{|\textit{main}|}"|
\end{center}

%%%%%%%%%%%%%%%%%%%%%%%%%%%%%%%%%%%%%%%%%%%%%%%%%%%%%%%%%%%%%%%%%%%%%%%%%%%%%%%%
\subsection{Manual Code}
\label{sec:manual}

In case one cannot be certain whether the definitions file |childdoc.def|
is installed on the target \TeX{} distribution
and one prefers not to ship it,
it is conceivable to paste a few relevant commands into the sources.

To that end, drop all statements |% \iffalse
%
% childdoc.dtx Copyright (C) 2017-2018 Niklas Beisert
%
% This work may be distributed and/or modified under the
% conditions of the LaTeX Project Public License, either version 1.3
% of this license or (at your option) any later version.
% The latest version of this license is in
%   http://www.latex-project.org/lppl.txt
% and version 1.3 or later is part of all distributions of LaTeX
% version 2005/12/01 or later.
%
% This work has the LPPL maintenance status `maintained'.
%
% The Current Maintainer of this work is Niklas Beisert.
%
% This work consists of the files childdoc.dtx and childdoc.ins
% and the derived files childdoc.def and cdocsamp.tex with
% cdocsch1.tex, cdocsch2.tex, cdocsdrf.tex, cdocsfn1.tex, cdocsfn2.tex.
%
%<package>\ifdefined\childdocmain\endinput\fi
%<package>\ProvidesFile{childdoc.def}[2018/12/30 v2.0 child document driver]
%<samplemain>\ProvidesFile{cdocsamp.tex}[2018/12/30 v2.0 sample for childdoc]
%<*driver>
%\ProvidesFile{childdoc.drv}[2018/12/30 v2.0 childdoc reference manual file]
\PassOptionsToClass{10pt,a4paper}{article}
\documentclass{ltxdoc}

\usepackage[margin=35mm]{geometry}
\usepackage{hyperref}
\usepackage{hyperxmp}
\usepackage[usenames]{color}

\hypersetup{colorlinks=true}
\hypersetup{pdfstartview=FitH}
\hypersetup{pdfpagemode=UseNone}
\hypersetup{pdfsource={}}
\hypersetup{pdflang={en-UK}}
\hypersetup{pdfcopyright={Copyright 2017-2018 Niklas Beisert.
  This work may be distributed and/or modified under the
  conditions of the LaTeX Project Public License, either version 1.3
  of this license or (at your option) any later version.}}
\hypersetup{pdflicenseurl={http://www.latex-project.org/lppl.txt}}
\hypersetup{pdfcontactaddress={ETH Zurich, ITP, HIT K,
  Wolfgang-Pauli-Strasse 27}}
\hypersetup{pdfcontactpostcode={8093}}
\hypersetup{pdfcontactcity={Zurich}}
\hypersetup{pdfcontactcountry={Switzerland}}
\hypersetup{pdfcontactemail={nbeisert@itp.phys.ethz.ch}}
\hypersetup{pdfcontacturl={http://people.phys.ethz.ch/\xmptilde nbeisert/}}

\newcommand{\secref}[1]{\hyperref[#1]{section \ref*{#1}}}

\parskip1ex
\parindent0pt
\let\olditemize\itemize
\def\itemize{\olditemize\parskip0pt}

\begin{document}

\title{The \textsf{childdoc} Package}
\hypersetup{pdftitle={The childdoc Package}}
\author{Niklas Beisert\\[2ex]
  Institut f\"ur Theoretische Physik\\
  Eidgen\"ossische Technische Hochschule Z\"urich\\
  Wolfgang-Pauli-Strasse 27, 8093 Z\"urich, Switzerland\\[1ex]
  \href{mailto:nbeisert@itp.phys.ethz.ch}
  {\texttt{nbeisert@itp.phys.ethz.ch}}}
\hypersetup{pdfauthor={Niklas Beisert}}
\hypersetup{pdfsubject={Manual for the LaTeX2e Package childdoc}}
\date{30 December 2018, \textsf{v2.0}}
\maketitle

\begin{abstract}\noindent
\textsf{childdoc} is a \LaTeXe{} package
that enables the direct compilation
of document sections included by |\include|
to individual files.
\end{abstract}

\begingroup
\parskip0ex
\tableofcontents
\endgroup

%%%%%%%%%%%%%%%%%%%%%%%%%%%%%%%%%%%%%%%%%%%%%%%%%%%%%%%%%%%%%%%%%%%%%%%%%%%%%%%%
%%%%%%%%%%%%%%%%%%%%%%%%%%%%%%%%%%%%%%%%%%%%%%%%%%%%%%%%%%%%%%%%%%%%%%%%%%%%%%%%
\section{Introduction}

\LaTeX{} provides a mechanism to structure a large document (such as a book)
into a main file and several child files (containing the chapters)
using the |\include| command.
This mechanism is beneficial for documents
which span hundreds of pages in order to
make the source file(s) more manageable.
Moreover, compilation can be restricted to
selected child files by means of the |\includeonly| command.
The latter feature can be used to reduce the compilation time while editing
(this was significantly more useful in the earlier days of \LaTeX{})
or to generate a smaller document which is easier to navigate.
Another application of |\includeonly| is to generate
documents consisting of selected parts of the complete document.

However, there are a few drawbacks of the plain |\include| mechanism:
\begin{itemize}
\item
The child files cannot be compiled on their own,
they can only be compiled via the main file.
A naive editing environment
(such as a text editor with an option
to have the current file processed by \LaTeX)
may require one to switch to the main file before compiling;
attempting to compile the child file produces errors.
\item
The main file must be modified (each time)
to adjust the |\includeonly| command
to the present needs. This easily leaves the main file in a messy state.
\item
The generated document will always carry the filename
of the main document. This is inconvenient if
several child files are to be compiled and
to be kept for distribution.
\end{itemize}

The present package provides a simple interface
to make child files individually compilable by \LaTeX{}.
Compiling a child file then has the same effect as compiling
the main file with an |\includeonly| command
to select the appropriate child.
Moreover the generated document will carry the name of the child
rather than the main file.
This resolves all three above issues.

This feature is meant to make the editing of books,
thesis documents and lecture notes somewhat more convenient.
However, the package can also be used efficiently for
composing a series of documents (such as exercise sheets)
which are typically distributed individually.
It then assists the author in generating the individual documents
(potentially in different versions)
as well as a document containing the collected series.
Another application is in developing style files
or other kinds of included material
where compilation of the style file could redirect
to a sample or test file.

%%%%%%%%%%%%%%%%%%%%%%%%%%%%%%%%%%%%%%%%%%%%%%%%%%%%%%%%%%%%%%%%%%%%%%%%%%%%%%%%
%%%%%%%%%%%%%%%%%%%%%%%%%%%%%%%%%%%%%%%%%%%%%%%%%%%%%%%%%%%%%%%%%%%%%%%%%%%%%%%%
\section{Usage}

First of all, the package \textsf{childdoc} is \emph{not} a standard
\LaTeXe{} |.sty| style file! Therefore it needs to be invoked in
a non-standard way.

%%%%%%%%%%%%%%%%%%%%%%%%%%%%%%%%%%%%%%%%%%%%%%%%%%%%%%%%%%%%%%%%%%%%%%%%%%%%%%%%
\subsection{Included Files}
\label{sec:include}

%%%%%%%%%%%%%%%%%%%%%%%%%%%%%%%%%%%%%%%%
\DescribeMacro{\childdocmain}
To use the package, add the commands
\begin{center}
\begin{tabular}{l}
|\input{childdoc.def}|\\
|\childdocmain{}|\\
\end{tabular}
\end{center}
at the very top of the main \LaTeX{} file,
in particular \emph{before} the |\documentclass| statement!
The argument of |\childdocmain| should be left empty
(but it must be present).

%%%%%%%%%%%%%%%%%%%%%%%%%%%%%%%%%%%%%%%%
\DescribeMacro{\childdocof}
Furthermore, add the commands
\begin{center}
\begin{tabular}{l}
|\input{childdoc.def}|\\
|\childdocof{|\textit{main}|}|\\
\end{tabular}
\end{center}
at the top of every child file \textit{child}
which is included by |\include{|\textit{child}|}|
from within the main file
(or at least for those files to be compiled individually).
The argument \textit{main} must be the filename of the main file.

There are a couple of
considerations in setting up the main and child documents:

%%%%%%%%%%%%%%%%%%%%%%%%%%%%%%%%%%%%%%%%
\paragraph{Restrictions.}

Please note the following restrictions:
\begin{itemize}
\item
|\childdocmain| must be called with one argument \textit{main}
to ensure compatibility with earlier version of the package.
It must either be empty (|\childdocmain{}|)
or precisely match the filename of the main file in which it is specified.
See \secref{sec:detection} for further information.
\item
The filename \textit{main} must be specified without the |.tex| extension.
\item
The filename \textit{main} is case sensitive
(even in case-insensitive file systems)
due to internal string comparison.
\item
The argument \textit{main} should be fully expanded, it cannot be a macro.
\item
Subdirectories and special characters should be avoided in filenames.
\item
The command |\childdocmain{|\textit{main}|}| must be followed by a whitespace.
It should not be followed immediately by another command
or by a comment mark `|%|'.
This is because the \TeX{} parser reads the token immediately following
the argument of |\childdocmain| and puts it
at the beginning of every child section;
however, a white\-space is ignored.
\end{itemize}

%%%%%%%%%%%%%%%%%%%%%%%%%%%%%%%%%%%%%%%%
\paragraph{Content of Main File.}

It is advisable to place all content in the child files included by |\include|.
Any output contained in the main file will appear in all child documents
unless suppressed manually;
it cannot be suppressed automatically by the |\includeonly| directive
and thus should normally be avoided.
A method to include some content in the main file
by means of conditional processing is described in \secref{sec:conditional}.

%%%%%%%%%%%%%%%%%%%%%%%%%%%%%%%%%%%%%%%%
\paragraph{Page Numbering.}

When only a part of the document is compiled,
the appropriate numbering of pages
(as well as other status parameters)
is determined from the |.aux| files.
The latter contain information from previous passes.
However this information needs to propagate through
all intermediate child documents.
Therefore the page numbering in child documents may well
be inconsistent until the complete document is compiled at least once.

A useful (if unconventional) way to always ensure a consistent
page numbering is to restart the numbering in each child document
and denote the pages by `\textit{child}|.|\textit{page}'
where \textit{child} represents the chapter/section number of the child file.
This can be achieved by the command
|\numberwithin{page}{|\textit{child}|}|
of the \textsf{amsmath} package
where \textit{child} can be |chapter| or |section|
depending on the chosen structuring.
Alternatively, one can modify the macro |\thepage| appropriately
and reset the counter |page| at the start of each child file.

%%%%%%%%%%%%%%%%%%%%%%%%%%%%%%%%%%%%%%%%%%%%%%%%%%%%%%%%%%%%%%%%%%%%%%%%%%%%%%%%
\subsection{Conditional Processing}
\label{sec:conditional}

The package provides a mechanism to compile different versions
of a document. To customise the versions further some conditional processing
can come in handy to distinguish which version is being compiled.
The package provides two macros to describe the compilation context:

%%%%%%%%%%%%%%%%%%%%%%%%%%%%%%%%%%%%%%%%
\DescribeMacro{\ifchilddoc}
The conditional |\ifchilddoc| distinguishes between the compilation of
child documents and the main document:
%
\begin{center}
|\ifchilddoc |\textit{child-code}| |[|\||else |\textit{main-code}]| \||fi|
\end{center}

%%%%%%%%%%%%%%%%%%%%%%%%%%%%%%%%%%%%%%%%
\DescribeMacro{\childdocname}
\DescribeMacro{\childdocjob}
The macro |\childdocname| contains the filename (without extension)
of the main or child file being processed.
Note that |\childdocjob| will always contain the name of the main file.

%%%%%%%%%%%%%%%%%%%%%%%%%%%%%%%%%%%%%%%%
\paragraph{Title Page.}

Conditional processing can be used to include a title or banner page
in the main document when proper precautions are taken.
Importantly, the code in the main file should ensure that the page counter
(as well as other status parameters which are stored in the |.aux| files)
takes the same value after the conditional processing.
Otherwise the page numbers may take divergent values
depending on which part is compiled.

For example, a title page could be declared by:
%
\begin{center}
\begin{tabular}{l}
|\ifchilddoc\||else|\\
|\addtocounter{page}{-1}|\\
\textit{code for title page}\\
|\newpage|\\
|\||fi|
\end{tabular}
\end{center}
%
A banner page for the child documents can be generated by:
%
\begin{center}
\begin{tabular}{l}
|\ifchilddoc|\\
|\addtocounter{page}{-1}|\\
\textit{code for banner page}\\
|\newpage|\\
|\||fi|
\end{tabular}
\end{center}
%
Here one could write a message such as:
\begin{center}
|This is the part \childdocname{} of \childdocjob{}.|
\end{center}

%%%%%%%%%%%%%%%%%%%%%%%%%%%%%%%%%%%%%%%%%%%%%%%%%%%%%%%%%%%%%%%%%%%%%%%%%%%%%%%%
\subsection{Flags}
\label{sec:flags}

The package makes it easy to generate different versions
of the main or child documents.
To this end compilation flags can be defined
and assigned different default values.
They will be particularly useful in conjunction
with the forwarding mechanism described in \secref{sec:forward}.

For example, it may be useful to have a flag |\version|
which can be set to |draft| or |final|.
The document source will contain some conditional code
depending on the value of |\version|.
Suppose further, the flag should default to |final| for the main file
and to |draft| for child files
which is a natural assignment for editing the document.
This is achieved by placing the following code
in the preamble of the main document
(below the |\childdocmain| directive):
%
\begin{center}
\begin{tabular}{l}
|\ifchilddoc|\\
|\providecommand{\version}{draft}|\\
|\||else|\\
|\providecommand{\version}{final}|\\
|\||fi|
\end{tabular}
\end{center}
%
The definition by |\providecommand| makes sure
that previous definitions are not overwritten.
Further statements |\providecommand{\version}{...}|
can thus be added before the above code to override it.

For the main file, one might add a line
(between |\childdocmain| and the above block)
%
\begin{center}
|%\ifchilddoc\||else\providecommand{\version}{draft}\||fi|
\end{center}
%
which can be uncommented to produce a draft version.
Likewise one can add a line to the very top of a child file
(above the |\childdocof{|\textit{main}|}| directive)
%
\begin{center}
|%\providecommand{\version}{final}|
\end{center}
%
which can be uncommented to produce the final version of this child document.

%%%%%%%%%%%%%%%%%%%%%%%%%%%%%%%%%%%%%%%%%%%%%%%%%%%%%%%%%%%%%%%%%%%%%%%%%%%%%%%%
\subsection{Forwarding}
\label{sec:forward}

Different versions of the main or child documents
using compilation flags as described in \secref{sec:flags}
can be (permanently) stored in different files
for convenient compilation, viewing and distribution.
To this end, the package defines a command
to pass on compilation to a different file:

%%%%%%%%%%%%%%%%%%%%%%%%%%%%%%%%%%%%%%%%
\DescribeMacro{\childdocforward}
The command |\childdocforward| redirects processing to
another source file:
%
\begin{center}
\begin{tabular}{l}
|\input{childdoc.def}|\\
|\childdocforward[|\textit{main}|]{|\textit{dest}|}|\\
\end{tabular}
\end{center}
%
The argument \textit{dest} is the destination file
(without extension).
It should be the main file or one of the child files.
Note that further \textsf{childdoc} directives
such as |\childdocof| and |\childdocforward|
in the indicated file will be processed in this form.
The optional argument \textit{main}
passes on directly to the main file \textit{main}
while pretending to compile the child \textit{dest}.
This form behaves as if \textit{dest}
issues |\childdocof{|\textit{main}|}| right away,
and no further \textsf{childdoc} directives will be processed.

%%%%%%%%%%%%%%%%%%%%%%%%%%%%%%%%%%%%%%%%
\DescribeMacro{\...prefix}
In the alternative form |\childdocforwardprefix|,
%
\begin{center}
\begin{tabular}{l}
|\input{childdoc.def}|\\
|\childdocforwardprefix[|\textit{main}|]{|\textit{prefix}|}{|\textit{dest}|}|
\end{tabular}
\end{center}
%
the destination file is determined by a pattern
depending on the current file:
To make this work, the current file must be called
`{\textit{prefix}\hspace{0.2em}\textit{suffix}}'
with \textit{prefix} matching precisely the argument.
Processing is then passed on to the file
`{\textit{dest}\hspace{0.2em}\textit{suffix}}'.
Surely, the same effect is achieved by
directly specifying the
argument `{\textit{dest}\hspace{0.2em}\textit{suffix}}'
in the first form.
However, that requires to set up a different file
for each child. With the alternative form of the command
all these files can have exactly the same content
which simplifies setting them up and maintaining them.

For example, the following file |draft.tex|
with a compilation flag |\version| as described in \secref{sec:flags}
compiles the main document as a draft:
%
\begin{center}
\begin{tabular}{l}
|\def\version{draft}|\\
|\input{childdoc.def}|\\
|\childdocforward{|\textit{main}|}|
\end{tabular}
\end{center}
%
Likewise, the following files |final|\textit{nn}|.tex|
compile the final version of the child document
|child|\textit{nn}|.tex|:
%
\begin{center}
\begin{tabular}{l}
|\def\version{final}|\\
|\input{childdoc.def}|\\
|\childdocforwardprefix{final}{child}|
\end{tabular}
\end{center}
%

Note that when several versions of a main file and/or of each child file
are to be generated, it may be convenient to set up a |Makefile| or
shell script to automatise the process.

%%%%%%%%%%%%%%%%%%%%%%%%%%%%%%%%%%%%%%%%%%%%%%%%%%%%%%%%%%%%%%%%%%%%%%%%%%%%%%%%
\subsection{Command Line Processing}
\label{sec:commandline}

The effect of redirection files can also be achieved by invoking
the \LaTeX{} compiler with a more elaborate command line.
Most conveniently this should be done as part
of a shell script or a |Makefile|.

When using \textsf{childdoc} in the main file, the following
command lines effectively perform a redirection
(note that depending on the shell being used,
backslashes may have to be doubled: `|\|' $\to$ `|\\|'):
%
\begin{center}
|... -jobname "|\textit{target}|" |\\|"|[\textit{flags}]%
|\input{childdoc.def}\childdocforward[|\textit{main}|]{|\textit{dest}|}"|
\end{center}
%
Here \textit{target} is the name of the output file,
\textit{main} is the name of the main file
and \textit{dest} is the name of the main or child file to be processed
(all filenames without extensions).
The optional argument \textit{main} can be omitted
if \textit{main} matches \textit{dest}.
Optionally, compilation \textit{flags} can be defined via |\def| commands.
This command line makes the \TeX{} engine believe
it is compiling the file \textit{target}
whose content is specified as the latter parameter.
The provided code then forwards the processing to
\textit{main} or \textit{dest} as described in \secref{sec:forward}.

%%%%%%%%%%%%%%%%%%%%%%%%%%%%%%%%%%%%%%%%%%%%%%%%%%%%%%%%%%%%%%%%%%%%%%%%%%%%%%%%
\subsection{Include by Input}
\label{sec:input}

Including child documents by |\include| has some restrictions by design.
Most notably, the content of a child document always occupies
its own set of pages; pages cannot be shared between child documents.
Usually, this behaviour makes perfect sense
because each child document contain an essential part of the document.
However, in some situations it may be desirable to compose
a document from a collection of parts
without having mandatory page breaks between then.
For this case, the package
provides a mechanism to include parts
by |\input| which can also be processed individually.
However, by construction this mechanism
requires manual handling of the content to be output.

%%%%%%%%%%%%%%%%%%%%%%%%%%%%%%%%%%%%%%%%
\DescribeMacro{\ifchilddocmanual}
The main file should be prepared as usual, see \secref{sec:include}.
However, the document body must make a distinction
between processing of an individual part and of the main document, e.g.:
%
\begin{center}
\begin{tabular}{l}
|\ifchilddocmanual|\\
|\input{\childdocname}|\\
|\||else|\\
\textit{document body with }|\input{|\textit{part}|}|\\
|\||fi|
\end{tabular}
\end{center}
%
The conditional |\ifchilddocmanual| is true whenever
a part to be included by |\input| is being compiled,
and the name of the part is stored in |\childdocname|.

%%%%%%%%%%%%%%%%%%%%%%%%%%%%%%%%%%%%%%%%
\DescribeMacro{\childdocby}
Each part to be included by |\input| should start with:
%
\begin{center}
\begin{tabular}{l}
|\input{childdoc.def}|\\
|\childdocby{|\textit{main}|}|\\
\end{tabular}
\end{center}
%
The directive |\childdocby| is similar to |\childdocof|
described in \secref{sec:include},
but the subsequent selection of content must be done manually.
To that end, both |\ifchilddoc| and |\ifchilddocmanual|
will be true upon processing of a part,
and the name of the part is stored in |\childdocname|.
Note that |\jobname| will be set to the filename of the current part
so that each part receives an individual |.aux| file
that does not interfere with the |.aux| file(s) of the main document.
This behaviour can be altered by the alternative form
|\childdocby[*]{|\textit{main}|}| (with a non-empty optional argument)
which uses the |.aux| file of the main document
by setting |\jobname| to \textit{main}.

%%%%%%%%%%%%%%%%%%%%%%%%%%%%%%%%%%%%%%%%%%%%%%%%%%%%%%%%%%%%%%%%%%%%%%%%%%%%%%%%
\subsection{Driver Development}
\label{sec:driver}

The \textsf{childdoc} mechanism can also be use for the development
of definition files such as \LaTeX{} styles or classes.
This case differs from the above setup with multiple parts
included by |\include| in that no |\includeonly| should be invoked.
This can be achieved by starting the include file
(before |\ProvidesPackage|) with:
%
\begin{center}
\begin{tabular}{l}
|\input{childdoc.def}|\\
|\childdocforward{|\textit{main}|}|\\
\end{tabular}
\end{center}
%
or alternatively with:
%
\begin{center}
\begin{tabular}{l}
|\input{childdoc.def}|\\
|\childdocby{|\textit{main}|}|\\
\end{tabular}
\end{center}
%
Both forms have slightly different effects as described above.
The main file is prepared as usual, see \secref{sec:include}.

%%%%%%%%%%%%%%%%%%%%%%%%%%%%%%%%%%%%%%%%%%%%%%%%%%%%%%%%%%%%%%%%%%%%%%%%%%%%%%%%
\subsection{Legacy Detection}
\label{sec:detection}

The directive |\childdocmain| in the main file can detect
whether the complete document or merely a child is to be compiled
even without using the directive |\childdocof|.
This method is deprecated because it is less robust
and there is no compelling reason to use it;
it is merely provided for backward compatibility
and it may be removed in future versions.

If the detection mechanism is to be used,
it is mandatory to correctly specify
the filename of the main file as the argument of |\childdocmain|:
%
\begin{center}
\begin{tabular}{l}
|\input{childdoc.def}|\\
|\childdocmain{|\textit{main}|}|\\
\end{tabular}
\end{center}
%
If |\jobname| does not match the argument \textit{main} of |\childdocmain|,
it is assumed that |\jobname| points to the child file to be compiled.
When using |\childdocmain| with the main file specified as argument,
it suffices to start a child file
with just |\input{|\textit{main}|}|
without loading of the package and using |\childdocof|.
If instead all processing is done
with the appropriate \textsf{childdoc} directives,
the argument of \textit{main} of |\childdocmain| can be empty.

An alternative version of the command line processing described
in \secref{sec:commandline} using the detection mechanism reads:
%
\begin{center}
|... -jobname "|\textit{target}|" "|[\textit{flags}]%
[|\def\jobname{|\textit{dest}|}|]|\input{|\textit{main}|}"|
\end{center}

%%%%%%%%%%%%%%%%%%%%%%%%%%%%%%%%%%%%%%%%%%%%%%%%%%%%%%%%%%%%%%%%%%%%%%%%%%%%%%%%
\subsection{Manual Code}
\label{sec:manual}

In case one cannot be certain whether the definitions file |childdoc.def|
is installed on the target \TeX{} distribution
and one prefers not to ship it,
it is conceivable to paste a few relevant commands into the sources.

To that end, drop all statements |\input{childdoc.def}|
and perform the replacements as outlined below.
Instead of |\childdocmain{|\textit{main}|}| add the following code
to the top of the main file:
%
\begin{center}
\begin{tabular}{l}
|\||ifdefined\childdocname\endinput\||fi\newif\ifchilddoc|\\
|\edef\childdocname{\scantokens\expandafter{\jobname\noexpand}}|\\
|\def\childdocmain{|\textit{main}|}\||ifx\childdocmain\childdocname\||else|\\
|\childdoctrue\includeonly{\childdocname}\let\jobname\childdocmain\||fi|\\
\end{tabular}
\end{center}
%
Instead of |\childdocof{|\textit{main}|}| just include the main file
at the top of each child file:
%
\begin{center}
|\input{|\textit{main}|}|
\end{center}
%
A simple redirection |\childdocforward{|\textit{dest}|}| is achieved by:
%
\begin{center}
|\def\jobname{|\textit{dest}|}\input{\jobname}|
\end{center}
%
The redirection with prefix
|\childdocforwardprefix[|\textit{prefix}|]{|\textit{dest}|}|
is accomplished by:
%
\begin{center}
\begin{tabular}{l}
|{\edef\jobname{\scantokens\expandafter{\jobname\noexpand}}|\\
|\def\redirectjob |\textit{prefix}|#1~~~{\gdef\jobname{|\textit{dest}|#1}}|\\
|\expandafter\redirectjob\jobname~~~}\input{\jobname}|
\end{tabular}
\end{center}

In an alternative approach,
child documents can be compiled by a specific command line
without additional code or specific definitions:
%
\begin{center}
|... -jobname "|\textit{target}|" "|[\textit{flags}]%
|\includeonly{|\textit{dest}|}\input{|\textit{main}|}"|
\end{center}
%

%%%%%%%%%%%%%%%%%%%%%%%%%%%%%%%%%%%%%%%%%%%%%%%%%%%%%%%%%%%%%%%%%%%%%%%%%%%%%%%%
%%%%%%%%%%%%%%%%%%%%%%%%%%%%%%%%%%%%%%%%%%%%%%%%%%%%%%%%%%%%%%%%%%%%%%%%%%%%%%%%
\section{Information}

%%%%%%%%%%%%%%%%%%%%%%%%%%%%%%%%%%%%%%%%%%%%%%%%%%%%%%%%%%%%%%%%%%%%%%%%%%%%%%%%
\subsection{Copyright}

Copyright \copyright{} 2017--2018 Niklas Beisert

This work may be distributed and/or modified under the
conditions of the \LaTeX{} Project Public License, either version 1.3
of this license or (at your option) any later version.
The latest version of this license is in
  \url{http://www.latex-project.org/lppl.txt}
and version 1.3 or later is part of all distributions of \LaTeX{}
version 2005/12/01 or later.

This work has the LPPL maintenance status `maintained'.

The Current Maintainer of this work is Niklas Beisert.

This work consists of the files |README.txt|, |childdoc.ins| and |childdoc.dtx|
as well as the derived files |childdoc.def|, |cdocsamp.tex|
with |cdocsch1.tex|, |cdocsch2.tex|, |cdocspt3.tex|, |cdocspt4.tex|,
|cdocsdrf.tex|, |cdocsfn1.tex|, |cdocsfn2.tex|
as well as |childdoc.pdf|.

%%%%%%%%%%%%%%%%%%%%%%%%%%%%%%%%%%%%%%%%%%%%%%%%%%%%%%%%%%%%%%%%%%%%%%%%%%%%%%%%
\subsection{Files and Installation}

The package consists of the files:
%
\begin{center}
\begin{tabular}{ll}
    |README.txt|   & readme file \\
    |childdoc.ins| & installation file \\
    |childdoc.dtx| & source file \\
    |childdoc.def| & definition file \\
    |cdocsamp.tex| & sample main file \\
    |cdocsch1.tex| & sample include file \\
    |cdocsch2.tex| & sample include file \\
    |cdocspt3.tex| & sample part file \\
    |cdocspt4.tex| & sample part file \\
    |cdocsdrf.tex| & sample redirection file \\
    |cdocsfn1.tex| & sample redirection file \\
    |cdocsfn2.tex| & sample redirection file \\
    |childdoc.pdf| & manual
\end{tabular}
\end{center}
%
The distribution consists of the files
|README.txt|, |childdoc.ins| and |childdoc.dtx|.
%
\begin{itemize}
\item
Run (pdf)\LaTeX{} on |childdoc.dtx|
to compile the manual |childdoc.pdf| (this file).
\item
Run \LaTeX{} on |childdoc.ins| to create the definitions file |childdoc.def|
and the sample |cdocsamp.tex| with include files
|cdocsch1.tex|, |cdocsch2.tex|, |cdocspt3.tex|, |cdocspt4.tex|,
|cdocsdrf.tex|, |cdocsfn1.tex|, |cdocsfn2.tex|.
Then copy the file |childdoc.def| to an appropriate directory of your \LaTeX{}
distribution, e.g.\ \textit{texmf-root}|/tex/latex/childdoc|.
\end{itemize}

%%%%%%%%%%%%%%%%%%%%%%%%%%%%%%%%%%%%%%%%%%%%%%%%%%%%%%%%%%%%%%%%%%%%%%%%%%%%%%%%
\subsection{Related CTAN Packages}

There are several other packages which offer a similar functionality:
%
\begin{itemize}
\item
The packages
\href{http://ctan.org/pkg/docmute}{\textsf{docmute}},
\href{http://ctan.org/pkg/includex}{\textsf{includex}} and
\href{http://ctan.org/pkg/standalone}{\textsf{standalone}}
provide commands to include only the document body of
a child file thus allowing both files to be compiled individually.
\item
The packages \href{http://ctan.org/pkg/subdocs}{\textsf{subdocs}}
and \href{http://ctan.org/pkg/subfiles}{\textsf{subfiles}}
provide structures in which the main and child documents can be
encapsulated and allowing them to be compiled individually.
The inclusion mechanism is different from the conventional |\include|.
\item
The package \href{http://ctan.org/pkg/combine}{\textsf{combine}}
is an elaborate solution to combine several documents into one.
\end{itemize}
%
See also the CTAN topic \href{http://ctan.org/topic/subdocs}{\textsf{subdocs}}
for further related packages.
The present package differs from the above solutions in that
a document structure constructed with the conventional |\include| mechanism
just needs two extra commands at the top of every file
such that all constituent files can be compiled individually.

%%%%%%%%%%%%%%%%%%%%%%%%%%%%%%%%%%%%%%%%%%%%%%%%%%%%%%%%%%%%%%%%%%%%%%%%%%%%%%%%
%\subsection{Feature Suggestions}
%
%The following is a list of features which may be useful for future
%versions of this package:
%%
%\begin{itemize}
%\item
%\ldots
%\end{itemize}

%%%%%%%%%%%%%%%%%%%%%%%%%%%%%%%%%%%%%%%%%%%%%%%%%%%%%%%%%%%%%%%%%%%%%%%%%%%%%%%%
\subsection{Revision History}

%%%%%%%%%%%%%%%%%%%%%%%%%%%%%%%%%%%%%%%%
\paragraph{v2.0:} 2018/12/30

\begin{itemize}
\item
immediate forward processing
\item
added |\childdocby| mechanism
\item
manual restructured
\end{itemize}

%%%%%%%%%%%%%%%%%%%%%%%%%%%%%%%%%%%%%%%%
\paragraph{v1.6:} 2018/01/17

\begin{itemize}
\item
application for development of include files
\item
corrections to manual
\end{itemize}

%%%%%%%%%%%%%%%%%%%%%%%%%%%%%%%%%%%%%%%%
\paragraph{v1.5:} 2017/05/21

\begin{itemize}
\item
more complete structuring introduced
\item
|\childdocof| introduced
\item
|\childdoc| renamed to |\childdocmain|
\item
|\childredirect| renamed to |\childdocforward| and |\childdocforwardprefix|
and functionality expanded
\end{itemize}

%%%%%%%%%%%%%%%%%%%%%%%%%%%%%%%%%%%%%%%%
\paragraph{v1.0:} 2017/04/27

\begin{itemize}
\item
manual and install package
\item
first version published on CTAN
\end{itemize}

%%%%%%%%%%%%%%%%%%%%%%%%%%%%%%%%%%%%%%%%
\paragraph{v0.6:} 2017/04/26

\begin{itemize}
\item
redirection mechanism added
\end{itemize}

%%%%%%%%%%%%%%%%%%%%%%%%%%%%%%%%%%%%%%%%
\paragraph{v0.5:} 2017/04/26

\begin{itemize}
\item
functionality in definition file
\end{itemize}


%%%%%%%%%%%%%%%%%%%%%%%%%%%%%%%%%%%%%%%%%%%%%%%%%%%%%%%%%%%%%%%%%%%%%%%%%%%%%%%%
%%%%%%%%%%%%%%%%%%%%%%%%%%%%%%%%%%%%%%%%%%%%%%%%%%%%%%%%%%%%%%%%%%%%%%%%%%%%%%%%
%%%%%%%%%%%%%%%%%%%%%%%%%%%%%%%%%%%%%%%%%%%%%%%%%%%%%%%%%%%%%%%%%%%%%%%%%%%%%%%%
\appendix

\settowidth\MacroIndent{\rmfamily\scriptsize 000\ }

 \DocInput{childdoc.dtx}

\end{document}
%</driver>
% \fi
%
% %%%%%%%%%%%%%%%%%%%%%%%%%%%%%%%%%%%%%%%%%%%%%%%%%%%%%%%%%%%%%%%%%%%%%%%%%%%%%%
% %%%%%%%%%%%%%%%%%%%%%%%%%%%%%%%%%%%%%%%%%%%%%%%%%%%%%%%%%%%%%%%%%%%%%%%%%%%%%%
% \section{Sample}
%\iffalse
%<*samplemain>
%\fi
%
% The following presents a sample document
% with two chapters, two parts, a title page,
% a compile flag as well as three forwarding files to set the flag.
% It consists of eight |.tex| files:
% \begin{center}
% \begin{tabular}{ll}
% |cdocsamp.tex|&main file\\
% |cdocsch1.tex|&include file for chapter 1\\
% |cdocsch2.tex|&include file for chapter 2\\
% |cdocspt3.tex|&include file for part 3\\
% |cdocspt4.tex|&include file for part 4\\
% |cdocsdrf.tex|&forwarding file for main file in draft mode\\
% |cdocsfi1.tex|&forwarding file for final version of chapter 1\\
% |cdocsfi2.tex|&forwarding file for final version of chapter 2\\
% \end{tabular}
% \end{center}
% Each of the eight files can be compiled directly by the \LaTeX{} compiler.
%
% %%%%%%%%%%%%%%%%%%%%%%%%%%%%%%%%%%%%%%
% \paragraph{Main File.}
%
% The main file is called |cdocsamp.tex|.
%
% Load the \textsf{childdoc} definitions and
% declare the filename for the main document:
%    \begin{macrocode}
\input{childdoc.def}
\childdocmain{}
%    \end{macrocode}

% Optional override for |\version| flag:
%    \begin{macrocode}
%%\ifchilddoc\else\providecommand{\version}{draft}\fi
%    \end{macrocode}

% Define the default values for the |\version| flag
% (|final| for the main file and |draft| for childs):
%    \begin{macrocode}
\ifchilddoc
\providecommand{\version}{draft}
\else
\providecommand{\version}{final}
\fi
%    \end{macrocode}

% Load the standard document class:
%    \begin{macrocode}
\documentclass[12pt]{article}
%    \end{macrocode}

% Start the document body:
%    \begin{macrocode}
\begin{document}
%    \end{macrocode}

% Declare a title page.
% Print title, part of document being processed and version flag:
%    \begin{macrocode}
\addtocounter{page}{-1}
\begin{center}
{\LARGE\bfseries{}childdoc example\par}
\vspace{1cm}
\ifchilddoc
\ifchilddocmanual part\else chapter\fi:
`\childdocname' of `\childdocjob'\par
\else
main document: `\childdocjob'\par
\fi
version: \version\par
\end{center}
\newpage
%    \end{macrocode}

% Manually include selected file,
% otherwise process as usual:
%    \begin{macrocode}
\ifchilddocmanual
\section*{part `\childdocname'}
\input{\childdocname}
\else
%    \end{macrocode}

% Include the two chapters:
%    \begin{macrocode}
\include{cdocsch1}
\include{cdocsch2}
%    \end{macrocode}

% Include the two parts unless only chapters should be displayed:
%    \begin{macrocode}
\ifchilddoc\else
\section{part three}
\input{cdocspt3}
\section{part four}
\input{cdocspt4}
\fi
%    \end{macrocode}

% Process as usual until here:
%    \begin{macrocode}
\fi
%    \end{macrocode}

% End of document body:
%    \begin{macrocode}
\end{document}
%    \end{macrocode}
%\iffalse
%</samplemain>
%\fi
%
% %%%%%%%%%%%%%%%%%%%%%%%%%%%%%%%%%%%%%%
% \paragraph{Chapter Include Files.}
%
% The include files are called |cdocsch1.tex| and |cdocsch2.tex|.
%
%\iffalse
%<*samplechap1|samplechap2>
%\fi

% Optional override for |\version| flag:
%    \begin{macrocode}
%%\providecommand{\version}{final}
%    \end{macrocode}

% Include the main document:
%    \begin{macrocode}
\input{childdoc.def}
\childdocof{cdocsamp}
%    \end{macrocode}

%\iffalse
%</samplechap1|samplechap2>
%\fi
%
%\iffalse
%<*samplechap1>
%\fi
% Some text for chapter 1:
%    \begin{macrocode}
\section{one}
some text in chapter one
%    \end{macrocode}

%\iffalse
%</samplechap1>
%\fi
% Some text for chapter 2:
%\iffalse
%<*samplechap2>
%\fi
%    \begin{macrocode}
\section{two}
more text in chapter two
%    \end{macrocode}

%\iffalse
%</samplechap2>
%\fi
%
% %%%%%%%%%%%%%%%%%%%%%%%%%%%%%%%%%%%%%%
% \paragraph{Part Include Files.}
%
% The include files are called |cdocspt3.tex| and |cdocspt4.tex|.
%
%\iffalse
%<*samplepart3|samplepart4>
%\fi

% Optional override for |\version| flag:
%    \begin{macrocode}
%%\providecommand{\version}{final}
%    \end{macrocode}

% Include the main document:
%    \begin{macrocode}
\input{childdoc.def}
\childdocby{cdocsamp}
%    \end{macrocode}

%\iffalse
%</samplepart3|samplepart4>
%\fi
%
%\iffalse
%<*samplepart3>
%\fi
% Some text for part 3:
%    \begin{macrocode}
some text in part three
%    \end{macrocode}

%\iffalse
%</samplepart3>
%\fi
% Some text for part 4:
%\iffalse
%<*samplepart4>
%\fi
%    \begin{macrocode}
more text in part four
%    \end{macrocode}

%\iffalse
%</samplepart4>
%\fi
%
% %%%%%%%%%%%%%%%%%%%%%%%%%%%%%%%%%%%%%%
% \paragraph{Forwarding for a Complete Draft.}
%
% The following forwarding file |cdocsdrf.tex|
% compiles the main document in draft mode:
%\iffalse
%<*sampledraft>
%\fi
%    \begin{macrocode}
\def\version{draft}
\input{childdoc.def}
\childdocforward{cdocsamp}
%    \end{macrocode}

%\iffalse
%</sampledraft>
%\fi
%
% %%%%%%%%%%%%%%%%%%%%%%%%%%%%%%%%%%%%%%
% \paragraph{Forwarding for Final Version of the Chapters.}
%
% The following forwarding files |cdocsfn1.tex| and |cdocsfn2.tex|
% (with identical content)
% compile the final versions of the child documents
% |cdocsch1.tex| and |cdocsch2.tex|, respectively:
%\iffalse
%<*samplefinal>
%\fi
%    \begin{macrocode}
\def\version{final}
\input{childdoc.def}
\childdocforwardprefix[cdocsamp]{cdocsfn}{cdocsch}
%    \end{macrocode}

%\iffalse
%</samplefinal>
%\fi
%
% %%%%%%%%%%%%%%%%%%%%%%%%%%%%%%%%%%%%%%
% \paragraph{Command Line Processing.}
%
% The following three command lines generate the output files
% |cdocscld|, |cdocscl1| and |cdocscl2|
% which should be identical to
% |cdocsdrf|, |cdocsch1| and |cdocsfn2|, respectively:
% \begin{center}
% \begin{tabular}{l}
% |latex -jobname cdocscld \|\\
% |  "\def\version{draft}\input{childdoc.def}\childdocforward{cdocsamp}"|\\
% |latex -jobname cdocscl1 \|\\
% |  "\input{childdoc.def}\childdocforward[cdocsamp]{cdocsch1}"|\\
% |latex -jobname cdocscl2 \|\\
% |  "\def\version{final}\input{childdoc.def}\childdocforward{cdocsch2}"|
% \end{tabular}
% \end{center}
% Note that the trailing backslash on each first line
% merely continues the input to the second line
% (for convenient cut ant paste).
% Furthermore, the command |latex| can be replaced by any
% of its alternative versions such as |pdflatex|.
%
% %%%%%%%%%%%%%%%%%%%%%%%%%%%%%%%%%%%%%%%%%%%%%%%%%%%%%%%%%%%%%%%%%%%%%%%%%%%%%%
% %%%%%%%%%%%%%%%%%%%%%%%%%%%%%%%%%%%%%%%%%%%%%%%%%%%%%%%%%%%%%%%%%%%%%%%%%%%%%%
% \section{Implementation}
%\iffalse
%<*package>
%\fi
%
% This section describes the definitions file |childdoc.def|.

% The definitions cannot be loaded using |\usepackage| or |\RequirePackage|
% which has a mechanism to prevent loading a style file more than once.
% When loading the definitions by means of |\input|
% multiple instances have to be prevented manually:
%\iffalse
%This code needs to be before the `\ProvidesFile' directive
%which is defined at the beginning of this file.
%Therefore it is also placed there and commented out here.
%</package>
%<*discard>
%\fi
%    \begin{macrocode}
\ifdefined\childdocmain\endinput\fi
%    \end{macrocode}
%\iffalse
%</discard>
%<*package>
%\fi
%
% \macro{\ifchilddoc}
% \macro{\ifchilddocmanual}
% The conditional |\ifchilddoc| tells whether a
% child (true) or main (false) document is being compiled.
% The conditional |\ifchilddocmanual| tells whether
% the |\includeonly| mechanism is used (false) or
% the selection of child files must be performed manually (true).
% The definitions initialise to false:
%    \begin{macrocode}
\newif\ifchilddoc
\newif\ifchilddocmanual
%    \end{macrocode}

% \macro{\childdocname}
% \macro{\childdocjob}
% The macro |\childdocname| stores the name of the main document
% to be compiled. The macro |\childdocjob| stores the name of
% the document on which the \LaTeX{} compiler was originally invoked.
% The content of |\jobname| cannot be compared
% to filenames specified in the source due to different catcodes.
% The following code rescans |\jobname|, stores the result
% in |\childdocname| and saves a copy in |\childdocjob|:
%    \begin{macrocode}
\edef\childdocname{\scantokens\expandafter{\jobname\noexpand}}
\let\childdocjob\childdocname
%    \end{macrocode}

% \macro{\childdocdisable}
% The macro |\childdocdisable| prevents the main file
% from being processed more than once.
% At this stage, the main document command |\childdocmain|
% is assumed to be called once again where it should do nothing.
% Any subsequent call to it should prevent
% a secondary processing of the main document
% It overwrites the forwarding commands
% |\childdocof| and |\childdocforward|
% with empty macros to prevent further inclusions of the main document:
%    \begin{macrocode}
\newcommand{\childdocdisable}
{
  \renewcommand{\childdocmain}[1]{\renewcommand{\childdocmain}[1]{\endinput}}
  \renewcommand{\childdocof}[1]{}
  \renewcommand{\childdocby}[2][]{}
  \renewcommand{\childdocforward}[2][]{}
  \renewcommand{\childdocdisable}{}
}
%    \end{macrocode}

% \macro{\childdocmain}
% The macro |\childdocmain| is to be called at the top of the main file
% with nothing or the main filename (without extension) as argument.
% First, it breaks loops.
% If the argument is not empty and does not match |\childdocname|
% (which is set by the first inclusion of |childdoc.def|),
% |\ifchilddoc| is set to true, |\includeonly| is applied to the child file
% and |\jobname| is set to the main file
% (for proper handling of |.aux| files):
%    \begin{macrocode}
\newcommand{\childdocmain}[1]
{
  \childdocdisable\childdocmain{}
  \if?#1?\else
    \begingroup
      \def\childdoctmp{#1}
      \ifx\childdoctmp\childdocname
        \def\childdoctmp{}
      \else
        \def\childdoctmp
        {
          \childdoctrue
          \includeonly{\childdocname}
          \def\childdocjob{#1}
          \def\jobname{#1}
        }
      \fi
      \expandafter
    \endgroup
    \childdoctmp
  \fi
}
%    \end{macrocode}

% \macro{\childdocof}
% The command |\childdocof| redirects
% compilation to the main file |#1|.
%    \begin{macrocode}
\newcommand{\childdocof}[1]
{
  \childdocdisable
  \childdoctrue
  \includeonly{\childdocname}
  \def\jobname{#1}
  \def\childdocjob{#1}
  \input{#1}
}
%    \end{macrocode}

% \macro{\childdocby}
% The command |\childdocby| ....
%    \begin{macrocode}
\newcommand{\childdocby}[2][]
{
  \childdocdisable
  \childdoctrue
  \childdocmanualtrue
  \if?#1?\else
    \def\jobname{#2}
  \fi
  \def\childdocjob{#2}
  \input{#2}
  \endinput
}
%    \end{macrocode}

% \macro{\childdocforward}
% The command |\childdocforward| redirects
% compilation to the main file or
% (if the optional argument is given) a child file.
% Parameters are set as if the main file
% or a child file starting with |\childdocof| was compiled.
% Then compilation is handed over to the main file:
%    \begin{macrocode}
\newcommand{\childdocforward}[2][]
{
  \begingroup
    \if?#1?
      \def\childdoctmp
      {
        \def\childdocname{#2}
        \def\childdocjob{#2}
        \def\jobname{#2}
        \input{#2}
        \endinput
      }
    \else
      \def\childdoctmp
      {
        \childdocdisable
        \def\childdocname{#2}
        \childdoctrue
        \includeonly{#2}
        \def\childdocjob{#1}
        \def\jobname{#1}
        \input{#1}
        \endinput
      }
    \fi
    \expandafter
  \endgroup
  \childdoctmp
}
%    \end{macrocode}

% \macro{\childdocforwardprefix}
% The command |\childdocforwardprefix| redirects
% compilation to the main or a child file by means of a pattern.
% The prefix |#1| in the current filename is replaced by |#2|
% and the suffix of the current filename is kept
% (it is assumed that the filename does not contain the substring `|~~~|'
% which is used as a delimiter).
% Compilation is handed over to the new file by |\childdocforward|:
%    \begin{macrocode}
\newcommand{\childdocforwardprefix}[3][]
{
  \begingroup
    \def\childdocextract #2##1~~~{\def\childdoctmp{\childdocforward[#1]{#3##1}}}
    \expandafter\childdocextract\childdocname~~~
    \expandafter
  \endgroup
  \childdoctmp
}
%    \end{macrocode}

% \macro{\childdoc}
% The deprecated macro |\childdoc| is a legacy version of |\childdocmain|:
%    \begin{macrocode}
\newcommand{\childdoc}{\childdocmain}
%    \end{macrocode}

% \macro{\childdocredirect}
% The deprecated macro |\childdocredirect| is a legacy version
% of |\childdocforward| and |\childdocforwardprefix|:
%    \begin{macrocode}
\newcommand{\childdocredirect}[2][]
{
  \begingroup
    \if?#1?
      \def\childdoctmp{\childdocforward{#2}}
    \else
      \def\childdoctmp{\childdocforwardprefix{#1}{#2}}
    \fi
    \expandafter
  \endgroup
  \childdoctmp
}
%    \end{macrocode}

%\iffalse
%</package>
%\fi
%
\endinput
|
and perform the replacements as outlined below.
Instead of |\childdocmain{|\textit{main}|}| add the following code
to the top of the main file:
%
\begin{center}
\begin{tabular}{l}
|\||ifdefined\childdocname\endinput\||fi\newif\ifchilddoc|\\
|\edef\childdocname{\scantokens\expandafter{\jobname\noexpand}}|\\
|\def\childdocmain{|\textit{main}|}\||ifx\childdocmain\childdocname\||else|\\
|\childdoctrue\includeonly{\childdocname}\let\jobname\childdocmain\||fi|\\
\end{tabular}
\end{center}
%
Instead of |\childdocof{|\textit{main}|}| just include the main file
at the top of each child file:
%
\begin{center}
|\input{|\textit{main}|}|
\end{center}
%
A simple redirection |\childdocforward{|\textit{dest}|}| is achieved by:
%
\begin{center}
|\def\jobname{|\textit{dest}|}\input{\jobname}|
\end{center}
%
The redirection with prefix
|\childdocforwardprefix[|\textit{prefix}|]{|\textit{dest}|}|
is accomplished by:
%
\begin{center}
\begin{tabular}{l}
|{\edef\jobname{\scantokens\expandafter{\jobname\noexpand}}|\\
|\def\redirectjob |\textit{prefix}|#1~~~{\gdef\jobname{|\textit{dest}|#1}}|\\
|\expandafter\redirectjob\jobname~~~}\input{\jobname}|
\end{tabular}
\end{center}

In an alternative approach,
child documents can be compiled by a specific command line
without additional code or specific definitions:
%
\begin{center}
|... -jobname "|\textit{target}|" "|[\textit{flags}]%
|\includeonly{|\textit{dest}|}\input{|\textit{main}|}"|
\end{center}
%

%%%%%%%%%%%%%%%%%%%%%%%%%%%%%%%%%%%%%%%%%%%%%%%%%%%%%%%%%%%%%%%%%%%%%%%%%%%%%%%%
%%%%%%%%%%%%%%%%%%%%%%%%%%%%%%%%%%%%%%%%%%%%%%%%%%%%%%%%%%%%%%%%%%%%%%%%%%%%%%%%
\section{Information}

%%%%%%%%%%%%%%%%%%%%%%%%%%%%%%%%%%%%%%%%%%%%%%%%%%%%%%%%%%%%%%%%%%%%%%%%%%%%%%%%
\subsection{Copyright}

Copyright \copyright{} 2017--2018 Niklas Beisert

This work may be distributed and/or modified under the
conditions of the \LaTeX{} Project Public License, either version 1.3
of this license or (at your option) any later version.
The latest version of this license is in
  \url{http://www.latex-project.org/lppl.txt}
and version 1.3 or later is part of all distributions of \LaTeX{}
version 2005/12/01 or later.

This work has the LPPL maintenance status `maintained'.

The Current Maintainer of this work is Niklas Beisert.

This work consists of the files |README.txt|, |childdoc.ins| and |childdoc.dtx|
as well as the derived files |childdoc.def|, |cdocsamp.tex|
with |cdocsch1.tex|, |cdocsch2.tex|, |cdocspt3.tex|, |cdocspt4.tex|,
|cdocsdrf.tex|, |cdocsfn1.tex|, |cdocsfn2.tex|
as well as |childdoc.pdf|.

%%%%%%%%%%%%%%%%%%%%%%%%%%%%%%%%%%%%%%%%%%%%%%%%%%%%%%%%%%%%%%%%%%%%%%%%%%%%%%%%
\subsection{Files and Installation}

The package consists of the files:
%
\begin{center}
\begin{tabular}{ll}
    |README.txt|   & readme file \\
    |childdoc.ins| & installation file \\
    |childdoc.dtx| & source file \\
    |childdoc.def| & definition file \\
    |cdocsamp.tex| & sample main file \\
    |cdocsch1.tex| & sample include file \\
    |cdocsch2.tex| & sample include file \\
    |cdocspt3.tex| & sample part file \\
    |cdocspt4.tex| & sample part file \\
    |cdocsdrf.tex| & sample redirection file \\
    |cdocsfn1.tex| & sample redirection file \\
    |cdocsfn2.tex| & sample redirection file \\
    |childdoc.pdf| & manual
\end{tabular}
\end{center}
%
The distribution consists of the files
|README.txt|, |childdoc.ins| and |childdoc.dtx|.
%
\begin{itemize}
\item
Run (pdf)\LaTeX{} on |childdoc.dtx|
to compile the manual |childdoc.pdf| (this file).
\item
Run \LaTeX{} on |childdoc.ins| to create the definitions file |childdoc.def|
and the sample |cdocsamp.tex| with include files
|cdocsch1.tex|, |cdocsch2.tex|, |cdocspt3.tex|, |cdocspt4.tex|,
|cdocsdrf.tex|, |cdocsfn1.tex|, |cdocsfn2.tex|.
Then copy the file |childdoc.def| to an appropriate directory of your \LaTeX{}
distribution, e.g.\ \textit{texmf-root}|/tex/latex/childdoc|.
\end{itemize}

%%%%%%%%%%%%%%%%%%%%%%%%%%%%%%%%%%%%%%%%%%%%%%%%%%%%%%%%%%%%%%%%%%%%%%%%%%%%%%%%
\subsection{Related CTAN Packages}

There are several other packages which offer a similar functionality:
%
\begin{itemize}
\item
The packages
\href{http://ctan.org/pkg/docmute}{\textsf{docmute}},
\href{http://ctan.org/pkg/includex}{\textsf{includex}} and
\href{http://ctan.org/pkg/standalone}{\textsf{standalone}}
provide commands to include only the document body of
a child file thus allowing both files to be compiled individually.
\item
The packages \href{http://ctan.org/pkg/subdocs}{\textsf{subdocs}}
and \href{http://ctan.org/pkg/subfiles}{\textsf{subfiles}}
provide structures in which the main and child documents can be
encapsulated and allowing them to be compiled individually.
The inclusion mechanism is different from the conventional |\include|.
\item
The package \href{http://ctan.org/pkg/combine}{\textsf{combine}}
is an elaborate solution to combine several documents into one.
\end{itemize}
%
See also the CTAN topic \href{http://ctan.org/topic/subdocs}{\textsf{subdocs}}
for further related packages.
The present package differs from the above solutions in that
a document structure constructed with the conventional |\include| mechanism
just needs two extra commands at the top of every file
such that all constituent files can be compiled individually.

%%%%%%%%%%%%%%%%%%%%%%%%%%%%%%%%%%%%%%%%%%%%%%%%%%%%%%%%%%%%%%%%%%%%%%%%%%%%%%%%
%\subsection{Feature Suggestions}
%
%The following is a list of features which may be useful for future
%versions of this package:
%%
%\begin{itemize}
%\item
%\ldots
%\end{itemize}

%%%%%%%%%%%%%%%%%%%%%%%%%%%%%%%%%%%%%%%%%%%%%%%%%%%%%%%%%%%%%%%%%%%%%%%%%%%%%%%%
\subsection{Revision History}

%%%%%%%%%%%%%%%%%%%%%%%%%%%%%%%%%%%%%%%%
\paragraph{v2.0:} 2018/12/30

\begin{itemize}
\item
immediate forward processing
\item
added |\childdocby| mechanism
\item
manual restructured
\end{itemize}

%%%%%%%%%%%%%%%%%%%%%%%%%%%%%%%%%%%%%%%%
\paragraph{v1.6:} 2018/01/17

\begin{itemize}
\item
application for development of include files
\item
corrections to manual
\end{itemize}

%%%%%%%%%%%%%%%%%%%%%%%%%%%%%%%%%%%%%%%%
\paragraph{v1.5:} 2017/05/21

\begin{itemize}
\item
more complete structuring introduced
\item
|\childdocof| introduced
\item
|\childdoc| renamed to |\childdocmain|
\item
|\childredirect| renamed to |\childdocforward| and |\childdocforwardprefix|
and functionality expanded
\end{itemize}

%%%%%%%%%%%%%%%%%%%%%%%%%%%%%%%%%%%%%%%%
\paragraph{v1.0:} 2017/04/27

\begin{itemize}
\item
manual and install package
\item
first version published on CTAN
\end{itemize}

%%%%%%%%%%%%%%%%%%%%%%%%%%%%%%%%%%%%%%%%
\paragraph{v0.6:} 2017/04/26

\begin{itemize}
\item
redirection mechanism added
\end{itemize}

%%%%%%%%%%%%%%%%%%%%%%%%%%%%%%%%%%%%%%%%
\paragraph{v0.5:} 2017/04/26

\begin{itemize}
\item
functionality in definition file
\end{itemize}


%%%%%%%%%%%%%%%%%%%%%%%%%%%%%%%%%%%%%%%%%%%%%%%%%%%%%%%%%%%%%%%%%%%%%%%%%%%%%%%%
%%%%%%%%%%%%%%%%%%%%%%%%%%%%%%%%%%%%%%%%%%%%%%%%%%%%%%%%%%%%%%%%%%%%%%%%%%%%%%%%
%%%%%%%%%%%%%%%%%%%%%%%%%%%%%%%%%%%%%%%%%%%%%%%%%%%%%%%%%%%%%%%%%%%%%%%%%%%%%%%%
\appendix

\settowidth\MacroIndent{\rmfamily\scriptsize 000\ }

 \DocInput{childdoc.dtx}

\end{document}
%</driver>
% \fi
%
% %%%%%%%%%%%%%%%%%%%%%%%%%%%%%%%%%%%%%%%%%%%%%%%%%%%%%%%%%%%%%%%%%%%%%%%%%%%%%%
% %%%%%%%%%%%%%%%%%%%%%%%%%%%%%%%%%%%%%%%%%%%%%%%%%%%%%%%%%%%%%%%%%%%%%%%%%%%%%%
% \section{Sample}
%\iffalse
%<*samplemain>
%\fi
%
% The following presents a sample document
% with two chapters, two parts, a title page,
% a compile flag as well as three forwarding files to set the flag.
% It consists of eight |.tex| files:
% \begin{center}
% \begin{tabular}{ll}
% |cdocsamp.tex|&main file\\
% |cdocsch1.tex|&include file for chapter 1\\
% |cdocsch2.tex|&include file for chapter 2\\
% |cdocspt3.tex|&include file for part 3\\
% |cdocspt4.tex|&include file for part 4\\
% |cdocsdrf.tex|&forwarding file for main file in draft mode\\
% |cdocsfi1.tex|&forwarding file for final version of chapter 1\\
% |cdocsfi2.tex|&forwarding file for final version of chapter 2\\
% \end{tabular}
% \end{center}
% Each of the eight files can be compiled directly by the \LaTeX{} compiler.
%
% %%%%%%%%%%%%%%%%%%%%%%%%%%%%%%%%%%%%%%
% \paragraph{Main File.}
%
% The main file is called |cdocsamp.tex|.
%
% Load the \textsf{childdoc} definitions and
% declare the filename for the main document:
%    \begin{macrocode}
% \iffalse
%
% childdoc.dtx Copyright (C) 2017-2018 Niklas Beisert
%
% This work may be distributed and/or modified under the
% conditions of the LaTeX Project Public License, either version 1.3
% of this license or (at your option) any later version.
% The latest version of this license is in
%   http://www.latex-project.org/lppl.txt
% and version 1.3 or later is part of all distributions of LaTeX
% version 2005/12/01 or later.
%
% This work has the LPPL maintenance status `maintained'.
%
% The Current Maintainer of this work is Niklas Beisert.
%
% This work consists of the files childdoc.dtx and childdoc.ins
% and the derived files childdoc.def and cdocsamp.tex with
% cdocsch1.tex, cdocsch2.tex, cdocsdrf.tex, cdocsfn1.tex, cdocsfn2.tex.
%
%<package>\ifdefined\childdocmain\endinput\fi
%<package>\ProvidesFile{childdoc.def}[2018/12/30 v2.0 child document driver]
%<samplemain>\ProvidesFile{cdocsamp.tex}[2018/12/30 v2.0 sample for childdoc]
%<*driver>
%\ProvidesFile{childdoc.drv}[2018/12/30 v2.0 childdoc reference manual file]
\PassOptionsToClass{10pt,a4paper}{article}
\documentclass{ltxdoc}

\usepackage[margin=35mm]{geometry}
\usepackage{hyperref}
\usepackage{hyperxmp}
\usepackage[usenames]{color}

\hypersetup{colorlinks=true}
\hypersetup{pdfstartview=FitH}
\hypersetup{pdfpagemode=UseNone}
\hypersetup{pdfsource={}}
\hypersetup{pdflang={en-UK}}
\hypersetup{pdfcopyright={Copyright 2017-2018 Niklas Beisert.
  This work may be distributed and/or modified under the
  conditions of the LaTeX Project Public License, either version 1.3
  of this license or (at your option) any later version.}}
\hypersetup{pdflicenseurl={http://www.latex-project.org/lppl.txt}}
\hypersetup{pdfcontactaddress={ETH Zurich, ITP, HIT K,
  Wolfgang-Pauli-Strasse 27}}
\hypersetup{pdfcontactpostcode={8093}}
\hypersetup{pdfcontactcity={Zurich}}
\hypersetup{pdfcontactcountry={Switzerland}}
\hypersetup{pdfcontactemail={nbeisert@itp.phys.ethz.ch}}
\hypersetup{pdfcontacturl={http://people.phys.ethz.ch/\xmptilde nbeisert/}}

\newcommand{\secref}[1]{\hyperref[#1]{section \ref*{#1}}}

\parskip1ex
\parindent0pt
\let\olditemize\itemize
\def\itemize{\olditemize\parskip0pt}

\begin{document}

\title{The \textsf{childdoc} Package}
\hypersetup{pdftitle={The childdoc Package}}
\author{Niklas Beisert\\[2ex]
  Institut f\"ur Theoretische Physik\\
  Eidgen\"ossische Technische Hochschule Z\"urich\\
  Wolfgang-Pauli-Strasse 27, 8093 Z\"urich, Switzerland\\[1ex]
  \href{mailto:nbeisert@itp.phys.ethz.ch}
  {\texttt{nbeisert@itp.phys.ethz.ch}}}
\hypersetup{pdfauthor={Niklas Beisert}}
\hypersetup{pdfsubject={Manual for the LaTeX2e Package childdoc}}
\date{30 December 2018, \textsf{v2.0}}
\maketitle

\begin{abstract}\noindent
\textsf{childdoc} is a \LaTeXe{} package
that enables the direct compilation
of document sections included by |\include|
to individual files.
\end{abstract}

\begingroup
\parskip0ex
\tableofcontents
\endgroup

%%%%%%%%%%%%%%%%%%%%%%%%%%%%%%%%%%%%%%%%%%%%%%%%%%%%%%%%%%%%%%%%%%%%%%%%%%%%%%%%
%%%%%%%%%%%%%%%%%%%%%%%%%%%%%%%%%%%%%%%%%%%%%%%%%%%%%%%%%%%%%%%%%%%%%%%%%%%%%%%%
\section{Introduction}

\LaTeX{} provides a mechanism to structure a large document (such as a book)
into a main file and several child files (containing the chapters)
using the |\include| command.
This mechanism is beneficial for documents
which span hundreds of pages in order to
make the source file(s) more manageable.
Moreover, compilation can be restricted to
selected child files by means of the |\includeonly| command.
The latter feature can be used to reduce the compilation time while editing
(this was significantly more useful in the earlier days of \LaTeX{})
or to generate a smaller document which is easier to navigate.
Another application of |\includeonly| is to generate
documents consisting of selected parts of the complete document.

However, there are a few drawbacks of the plain |\include| mechanism:
\begin{itemize}
\item
The child files cannot be compiled on their own,
they can only be compiled via the main file.
A naive editing environment
(such as a text editor with an option
to have the current file processed by \LaTeX)
may require one to switch to the main file before compiling;
attempting to compile the child file produces errors.
\item
The main file must be modified (each time)
to adjust the |\includeonly| command
to the present needs. This easily leaves the main file in a messy state.
\item
The generated document will always carry the filename
of the main document. This is inconvenient if
several child files are to be compiled and
to be kept for distribution.
\end{itemize}

The present package provides a simple interface
to make child files individually compilable by \LaTeX{}.
Compiling a child file then has the same effect as compiling
the main file with an |\includeonly| command
to select the appropriate child.
Moreover the generated document will carry the name of the child
rather than the main file.
This resolves all three above issues.

This feature is meant to make the editing of books,
thesis documents and lecture notes somewhat more convenient.
However, the package can also be used efficiently for
composing a series of documents (such as exercise sheets)
which are typically distributed individually.
It then assists the author in generating the individual documents
(potentially in different versions)
as well as a document containing the collected series.
Another application is in developing style files
or other kinds of included material
where compilation of the style file could redirect
to a sample or test file.

%%%%%%%%%%%%%%%%%%%%%%%%%%%%%%%%%%%%%%%%%%%%%%%%%%%%%%%%%%%%%%%%%%%%%%%%%%%%%%%%
%%%%%%%%%%%%%%%%%%%%%%%%%%%%%%%%%%%%%%%%%%%%%%%%%%%%%%%%%%%%%%%%%%%%%%%%%%%%%%%%
\section{Usage}

First of all, the package \textsf{childdoc} is \emph{not} a standard
\LaTeXe{} |.sty| style file! Therefore it needs to be invoked in
a non-standard way.

%%%%%%%%%%%%%%%%%%%%%%%%%%%%%%%%%%%%%%%%%%%%%%%%%%%%%%%%%%%%%%%%%%%%%%%%%%%%%%%%
\subsection{Included Files}
\label{sec:include}

%%%%%%%%%%%%%%%%%%%%%%%%%%%%%%%%%%%%%%%%
\DescribeMacro{\childdocmain}
To use the package, add the commands
\begin{center}
\begin{tabular}{l}
|\input{childdoc.def}|\\
|\childdocmain{}|\\
\end{tabular}
\end{center}
at the very top of the main \LaTeX{} file,
in particular \emph{before} the |\documentclass| statement!
The argument of |\childdocmain| should be left empty
(but it must be present).

%%%%%%%%%%%%%%%%%%%%%%%%%%%%%%%%%%%%%%%%
\DescribeMacro{\childdocof}
Furthermore, add the commands
\begin{center}
\begin{tabular}{l}
|\input{childdoc.def}|\\
|\childdocof{|\textit{main}|}|\\
\end{tabular}
\end{center}
at the top of every child file \textit{child}
which is included by |\include{|\textit{child}|}|
from within the main file
(or at least for those files to be compiled individually).
The argument \textit{main} must be the filename of the main file.

There are a couple of
considerations in setting up the main and child documents:

%%%%%%%%%%%%%%%%%%%%%%%%%%%%%%%%%%%%%%%%
\paragraph{Restrictions.}

Please note the following restrictions:
\begin{itemize}
\item
|\childdocmain| must be called with one argument \textit{main}
to ensure compatibility with earlier version of the package.
It must either be empty (|\childdocmain{}|)
or precisely match the filename of the main file in which it is specified.
See \secref{sec:detection} for further information.
\item
The filename \textit{main} must be specified without the |.tex| extension.
\item
The filename \textit{main} is case sensitive
(even in case-insensitive file systems)
due to internal string comparison.
\item
The argument \textit{main} should be fully expanded, it cannot be a macro.
\item
Subdirectories and special characters should be avoided in filenames.
\item
The command |\childdocmain{|\textit{main}|}| must be followed by a whitespace.
It should not be followed immediately by another command
or by a comment mark `|%|'.
This is because the \TeX{} parser reads the token immediately following
the argument of |\childdocmain| and puts it
at the beginning of every child section;
however, a white\-space is ignored.
\end{itemize}

%%%%%%%%%%%%%%%%%%%%%%%%%%%%%%%%%%%%%%%%
\paragraph{Content of Main File.}

It is advisable to place all content in the child files included by |\include|.
Any output contained in the main file will appear in all child documents
unless suppressed manually;
it cannot be suppressed automatically by the |\includeonly| directive
and thus should normally be avoided.
A method to include some content in the main file
by means of conditional processing is described in \secref{sec:conditional}.

%%%%%%%%%%%%%%%%%%%%%%%%%%%%%%%%%%%%%%%%
\paragraph{Page Numbering.}

When only a part of the document is compiled,
the appropriate numbering of pages
(as well as other status parameters)
is determined from the |.aux| files.
The latter contain information from previous passes.
However this information needs to propagate through
all intermediate child documents.
Therefore the page numbering in child documents may well
be inconsistent until the complete document is compiled at least once.

A useful (if unconventional) way to always ensure a consistent
page numbering is to restart the numbering in each child document
and denote the pages by `\textit{child}|.|\textit{page}'
where \textit{child} represents the chapter/section number of the child file.
This can be achieved by the command
|\numberwithin{page}{|\textit{child}|}|
of the \textsf{amsmath} package
where \textit{child} can be |chapter| or |section|
depending on the chosen structuring.
Alternatively, one can modify the macro |\thepage| appropriately
and reset the counter |page| at the start of each child file.

%%%%%%%%%%%%%%%%%%%%%%%%%%%%%%%%%%%%%%%%%%%%%%%%%%%%%%%%%%%%%%%%%%%%%%%%%%%%%%%%
\subsection{Conditional Processing}
\label{sec:conditional}

The package provides a mechanism to compile different versions
of a document. To customise the versions further some conditional processing
can come in handy to distinguish which version is being compiled.
The package provides two macros to describe the compilation context:

%%%%%%%%%%%%%%%%%%%%%%%%%%%%%%%%%%%%%%%%
\DescribeMacro{\ifchilddoc}
The conditional |\ifchilddoc| distinguishes between the compilation of
child documents and the main document:
%
\begin{center}
|\ifchilddoc |\textit{child-code}| |[|\||else |\textit{main-code}]| \||fi|
\end{center}

%%%%%%%%%%%%%%%%%%%%%%%%%%%%%%%%%%%%%%%%
\DescribeMacro{\childdocname}
\DescribeMacro{\childdocjob}
The macro |\childdocname| contains the filename (without extension)
of the main or child file being processed.
Note that |\childdocjob| will always contain the name of the main file.

%%%%%%%%%%%%%%%%%%%%%%%%%%%%%%%%%%%%%%%%
\paragraph{Title Page.}

Conditional processing can be used to include a title or banner page
in the main document when proper precautions are taken.
Importantly, the code in the main file should ensure that the page counter
(as well as other status parameters which are stored in the |.aux| files)
takes the same value after the conditional processing.
Otherwise the page numbers may take divergent values
depending on which part is compiled.

For example, a title page could be declared by:
%
\begin{center}
\begin{tabular}{l}
|\ifchilddoc\||else|\\
|\addtocounter{page}{-1}|\\
\textit{code for title page}\\
|\newpage|\\
|\||fi|
\end{tabular}
\end{center}
%
A banner page for the child documents can be generated by:
%
\begin{center}
\begin{tabular}{l}
|\ifchilddoc|\\
|\addtocounter{page}{-1}|\\
\textit{code for banner page}\\
|\newpage|\\
|\||fi|
\end{tabular}
\end{center}
%
Here one could write a message such as:
\begin{center}
|This is the part \childdocname{} of \childdocjob{}.|
\end{center}

%%%%%%%%%%%%%%%%%%%%%%%%%%%%%%%%%%%%%%%%%%%%%%%%%%%%%%%%%%%%%%%%%%%%%%%%%%%%%%%%
\subsection{Flags}
\label{sec:flags}

The package makes it easy to generate different versions
of the main or child documents.
To this end compilation flags can be defined
and assigned different default values.
They will be particularly useful in conjunction
with the forwarding mechanism described in \secref{sec:forward}.

For example, it may be useful to have a flag |\version|
which can be set to |draft| or |final|.
The document source will contain some conditional code
depending on the value of |\version|.
Suppose further, the flag should default to |final| for the main file
and to |draft| for child files
which is a natural assignment for editing the document.
This is achieved by placing the following code
in the preamble of the main document
(below the |\childdocmain| directive):
%
\begin{center}
\begin{tabular}{l}
|\ifchilddoc|\\
|\providecommand{\version}{draft}|\\
|\||else|\\
|\providecommand{\version}{final}|\\
|\||fi|
\end{tabular}
\end{center}
%
The definition by |\providecommand| makes sure
that previous definitions are not overwritten.
Further statements |\providecommand{\version}{...}|
can thus be added before the above code to override it.

For the main file, one might add a line
(between |\childdocmain| and the above block)
%
\begin{center}
|%\ifchilddoc\||else\providecommand{\version}{draft}\||fi|
\end{center}
%
which can be uncommented to produce a draft version.
Likewise one can add a line to the very top of a child file
(above the |\childdocof{|\textit{main}|}| directive)
%
\begin{center}
|%\providecommand{\version}{final}|
\end{center}
%
which can be uncommented to produce the final version of this child document.

%%%%%%%%%%%%%%%%%%%%%%%%%%%%%%%%%%%%%%%%%%%%%%%%%%%%%%%%%%%%%%%%%%%%%%%%%%%%%%%%
\subsection{Forwarding}
\label{sec:forward}

Different versions of the main or child documents
using compilation flags as described in \secref{sec:flags}
can be (permanently) stored in different files
for convenient compilation, viewing and distribution.
To this end, the package defines a command
to pass on compilation to a different file:

%%%%%%%%%%%%%%%%%%%%%%%%%%%%%%%%%%%%%%%%
\DescribeMacro{\childdocforward}
The command |\childdocforward| redirects processing to
another source file:
%
\begin{center}
\begin{tabular}{l}
|\input{childdoc.def}|\\
|\childdocforward[|\textit{main}|]{|\textit{dest}|}|\\
\end{tabular}
\end{center}
%
The argument \textit{dest} is the destination file
(without extension).
It should be the main file or one of the child files.
Note that further \textsf{childdoc} directives
such as |\childdocof| and |\childdocforward|
in the indicated file will be processed in this form.
The optional argument \textit{main}
passes on directly to the main file \textit{main}
while pretending to compile the child \textit{dest}.
This form behaves as if \textit{dest}
issues |\childdocof{|\textit{main}|}| right away,
and no further \textsf{childdoc} directives will be processed.

%%%%%%%%%%%%%%%%%%%%%%%%%%%%%%%%%%%%%%%%
\DescribeMacro{\...prefix}
In the alternative form |\childdocforwardprefix|,
%
\begin{center}
\begin{tabular}{l}
|\input{childdoc.def}|\\
|\childdocforwardprefix[|\textit{main}|]{|\textit{prefix}|}{|\textit{dest}|}|
\end{tabular}
\end{center}
%
the destination file is determined by a pattern
depending on the current file:
To make this work, the current file must be called
`{\textit{prefix}\hspace{0.2em}\textit{suffix}}'
with \textit{prefix} matching precisely the argument.
Processing is then passed on to the file
`{\textit{dest}\hspace{0.2em}\textit{suffix}}'.
Surely, the same effect is achieved by
directly specifying the
argument `{\textit{dest}\hspace{0.2em}\textit{suffix}}'
in the first form.
However, that requires to set up a different file
for each child. With the alternative form of the command
all these files can have exactly the same content
which simplifies setting them up and maintaining them.

For example, the following file |draft.tex|
with a compilation flag |\version| as described in \secref{sec:flags}
compiles the main document as a draft:
%
\begin{center}
\begin{tabular}{l}
|\def\version{draft}|\\
|\input{childdoc.def}|\\
|\childdocforward{|\textit{main}|}|
\end{tabular}
\end{center}
%
Likewise, the following files |final|\textit{nn}|.tex|
compile the final version of the child document
|child|\textit{nn}|.tex|:
%
\begin{center}
\begin{tabular}{l}
|\def\version{final}|\\
|\input{childdoc.def}|\\
|\childdocforwardprefix{final}{child}|
\end{tabular}
\end{center}
%

Note that when several versions of a main file and/or of each child file
are to be generated, it may be convenient to set up a |Makefile| or
shell script to automatise the process.

%%%%%%%%%%%%%%%%%%%%%%%%%%%%%%%%%%%%%%%%%%%%%%%%%%%%%%%%%%%%%%%%%%%%%%%%%%%%%%%%
\subsection{Command Line Processing}
\label{sec:commandline}

The effect of redirection files can also be achieved by invoking
the \LaTeX{} compiler with a more elaborate command line.
Most conveniently this should be done as part
of a shell script or a |Makefile|.

When using \textsf{childdoc} in the main file, the following
command lines effectively perform a redirection
(note that depending on the shell being used,
backslashes may have to be doubled: `|\|' $\to$ `|\\|'):
%
\begin{center}
|... -jobname "|\textit{target}|" |\\|"|[\textit{flags}]%
|\input{childdoc.def}\childdocforward[|\textit{main}|]{|\textit{dest}|}"|
\end{center}
%
Here \textit{target} is the name of the output file,
\textit{main} is the name of the main file
and \textit{dest} is the name of the main or child file to be processed
(all filenames without extensions).
The optional argument \textit{main} can be omitted
if \textit{main} matches \textit{dest}.
Optionally, compilation \textit{flags} can be defined via |\def| commands.
This command line makes the \TeX{} engine believe
it is compiling the file \textit{target}
whose content is specified as the latter parameter.
The provided code then forwards the processing to
\textit{main} or \textit{dest} as described in \secref{sec:forward}.

%%%%%%%%%%%%%%%%%%%%%%%%%%%%%%%%%%%%%%%%%%%%%%%%%%%%%%%%%%%%%%%%%%%%%%%%%%%%%%%%
\subsection{Include by Input}
\label{sec:input}

Including child documents by |\include| has some restrictions by design.
Most notably, the content of a child document always occupies
its own set of pages; pages cannot be shared between child documents.
Usually, this behaviour makes perfect sense
because each child document contain an essential part of the document.
However, in some situations it may be desirable to compose
a document from a collection of parts
without having mandatory page breaks between then.
For this case, the package
provides a mechanism to include parts
by |\input| which can also be processed individually.
However, by construction this mechanism
requires manual handling of the content to be output.

%%%%%%%%%%%%%%%%%%%%%%%%%%%%%%%%%%%%%%%%
\DescribeMacro{\ifchilddocmanual}
The main file should be prepared as usual, see \secref{sec:include}.
However, the document body must make a distinction
between processing of an individual part and of the main document, e.g.:
%
\begin{center}
\begin{tabular}{l}
|\ifchilddocmanual|\\
|\input{\childdocname}|\\
|\||else|\\
\textit{document body with }|\input{|\textit{part}|}|\\
|\||fi|
\end{tabular}
\end{center}
%
The conditional |\ifchilddocmanual| is true whenever
a part to be included by |\input| is being compiled,
and the name of the part is stored in |\childdocname|.

%%%%%%%%%%%%%%%%%%%%%%%%%%%%%%%%%%%%%%%%
\DescribeMacro{\childdocby}
Each part to be included by |\input| should start with:
%
\begin{center}
\begin{tabular}{l}
|\input{childdoc.def}|\\
|\childdocby{|\textit{main}|}|\\
\end{tabular}
\end{center}
%
The directive |\childdocby| is similar to |\childdocof|
described in \secref{sec:include},
but the subsequent selection of content must be done manually.
To that end, both |\ifchilddoc| and |\ifchilddocmanual|
will be true upon processing of a part,
and the name of the part is stored in |\childdocname|.
Note that |\jobname| will be set to the filename of the current part
so that each part receives an individual |.aux| file
that does not interfere with the |.aux| file(s) of the main document.
This behaviour can be altered by the alternative form
|\childdocby[*]{|\textit{main}|}| (with a non-empty optional argument)
which uses the |.aux| file of the main document
by setting |\jobname| to \textit{main}.

%%%%%%%%%%%%%%%%%%%%%%%%%%%%%%%%%%%%%%%%%%%%%%%%%%%%%%%%%%%%%%%%%%%%%%%%%%%%%%%%
\subsection{Driver Development}
\label{sec:driver}

The \textsf{childdoc} mechanism can also be use for the development
of definition files such as \LaTeX{} styles or classes.
This case differs from the above setup with multiple parts
included by |\include| in that no |\includeonly| should be invoked.
This can be achieved by starting the include file
(before |\ProvidesPackage|) with:
%
\begin{center}
\begin{tabular}{l}
|\input{childdoc.def}|\\
|\childdocforward{|\textit{main}|}|\\
\end{tabular}
\end{center}
%
or alternatively with:
%
\begin{center}
\begin{tabular}{l}
|\input{childdoc.def}|\\
|\childdocby{|\textit{main}|}|\\
\end{tabular}
\end{center}
%
Both forms have slightly different effects as described above.
The main file is prepared as usual, see \secref{sec:include}.

%%%%%%%%%%%%%%%%%%%%%%%%%%%%%%%%%%%%%%%%%%%%%%%%%%%%%%%%%%%%%%%%%%%%%%%%%%%%%%%%
\subsection{Legacy Detection}
\label{sec:detection}

The directive |\childdocmain| in the main file can detect
whether the complete document or merely a child is to be compiled
even without using the directive |\childdocof|.
This method is deprecated because it is less robust
and there is no compelling reason to use it;
it is merely provided for backward compatibility
and it may be removed in future versions.

If the detection mechanism is to be used,
it is mandatory to correctly specify
the filename of the main file as the argument of |\childdocmain|:
%
\begin{center}
\begin{tabular}{l}
|\input{childdoc.def}|\\
|\childdocmain{|\textit{main}|}|\\
\end{tabular}
\end{center}
%
If |\jobname| does not match the argument \textit{main} of |\childdocmain|,
it is assumed that |\jobname| points to the child file to be compiled.
When using |\childdocmain| with the main file specified as argument,
it suffices to start a child file
with just |\input{|\textit{main}|}|
without loading of the package and using |\childdocof|.
If instead all processing is done
with the appropriate \textsf{childdoc} directives,
the argument of \textit{main} of |\childdocmain| can be empty.

An alternative version of the command line processing described
in \secref{sec:commandline} using the detection mechanism reads:
%
\begin{center}
|... -jobname "|\textit{target}|" "|[\textit{flags}]%
[|\def\jobname{|\textit{dest}|}|]|\input{|\textit{main}|}"|
\end{center}

%%%%%%%%%%%%%%%%%%%%%%%%%%%%%%%%%%%%%%%%%%%%%%%%%%%%%%%%%%%%%%%%%%%%%%%%%%%%%%%%
\subsection{Manual Code}
\label{sec:manual}

In case one cannot be certain whether the definitions file |childdoc.def|
is installed on the target \TeX{} distribution
and one prefers not to ship it,
it is conceivable to paste a few relevant commands into the sources.

To that end, drop all statements |\input{childdoc.def}|
and perform the replacements as outlined below.
Instead of |\childdocmain{|\textit{main}|}| add the following code
to the top of the main file:
%
\begin{center}
\begin{tabular}{l}
|\||ifdefined\childdocname\endinput\||fi\newif\ifchilddoc|\\
|\edef\childdocname{\scantokens\expandafter{\jobname\noexpand}}|\\
|\def\childdocmain{|\textit{main}|}\||ifx\childdocmain\childdocname\||else|\\
|\childdoctrue\includeonly{\childdocname}\let\jobname\childdocmain\||fi|\\
\end{tabular}
\end{center}
%
Instead of |\childdocof{|\textit{main}|}| just include the main file
at the top of each child file:
%
\begin{center}
|\input{|\textit{main}|}|
\end{center}
%
A simple redirection |\childdocforward{|\textit{dest}|}| is achieved by:
%
\begin{center}
|\def\jobname{|\textit{dest}|}\input{\jobname}|
\end{center}
%
The redirection with prefix
|\childdocforwardprefix[|\textit{prefix}|]{|\textit{dest}|}|
is accomplished by:
%
\begin{center}
\begin{tabular}{l}
|{\edef\jobname{\scantokens\expandafter{\jobname\noexpand}}|\\
|\def\redirectjob |\textit{prefix}|#1~~~{\gdef\jobname{|\textit{dest}|#1}}|\\
|\expandafter\redirectjob\jobname~~~}\input{\jobname}|
\end{tabular}
\end{center}

In an alternative approach,
child documents can be compiled by a specific command line
without additional code or specific definitions:
%
\begin{center}
|... -jobname "|\textit{target}|" "|[\textit{flags}]%
|\includeonly{|\textit{dest}|}\input{|\textit{main}|}"|
\end{center}
%

%%%%%%%%%%%%%%%%%%%%%%%%%%%%%%%%%%%%%%%%%%%%%%%%%%%%%%%%%%%%%%%%%%%%%%%%%%%%%%%%
%%%%%%%%%%%%%%%%%%%%%%%%%%%%%%%%%%%%%%%%%%%%%%%%%%%%%%%%%%%%%%%%%%%%%%%%%%%%%%%%
\section{Information}

%%%%%%%%%%%%%%%%%%%%%%%%%%%%%%%%%%%%%%%%%%%%%%%%%%%%%%%%%%%%%%%%%%%%%%%%%%%%%%%%
\subsection{Copyright}

Copyright \copyright{} 2017--2018 Niklas Beisert

This work may be distributed and/or modified under the
conditions of the \LaTeX{} Project Public License, either version 1.3
of this license or (at your option) any later version.
The latest version of this license is in
  \url{http://www.latex-project.org/lppl.txt}
and version 1.3 or later is part of all distributions of \LaTeX{}
version 2005/12/01 or later.

This work has the LPPL maintenance status `maintained'.

The Current Maintainer of this work is Niklas Beisert.

This work consists of the files |README.txt|, |childdoc.ins| and |childdoc.dtx|
as well as the derived files |childdoc.def|, |cdocsamp.tex|
with |cdocsch1.tex|, |cdocsch2.tex|, |cdocspt3.tex|, |cdocspt4.tex|,
|cdocsdrf.tex|, |cdocsfn1.tex|, |cdocsfn2.tex|
as well as |childdoc.pdf|.

%%%%%%%%%%%%%%%%%%%%%%%%%%%%%%%%%%%%%%%%%%%%%%%%%%%%%%%%%%%%%%%%%%%%%%%%%%%%%%%%
\subsection{Files and Installation}

The package consists of the files:
%
\begin{center}
\begin{tabular}{ll}
    |README.txt|   & readme file \\
    |childdoc.ins| & installation file \\
    |childdoc.dtx| & source file \\
    |childdoc.def| & definition file \\
    |cdocsamp.tex| & sample main file \\
    |cdocsch1.tex| & sample include file \\
    |cdocsch2.tex| & sample include file \\
    |cdocspt3.tex| & sample part file \\
    |cdocspt4.tex| & sample part file \\
    |cdocsdrf.tex| & sample redirection file \\
    |cdocsfn1.tex| & sample redirection file \\
    |cdocsfn2.tex| & sample redirection file \\
    |childdoc.pdf| & manual
\end{tabular}
\end{center}
%
The distribution consists of the files
|README.txt|, |childdoc.ins| and |childdoc.dtx|.
%
\begin{itemize}
\item
Run (pdf)\LaTeX{} on |childdoc.dtx|
to compile the manual |childdoc.pdf| (this file).
\item
Run \LaTeX{} on |childdoc.ins| to create the definitions file |childdoc.def|
and the sample |cdocsamp.tex| with include files
|cdocsch1.tex|, |cdocsch2.tex|, |cdocspt3.tex|, |cdocspt4.tex|,
|cdocsdrf.tex|, |cdocsfn1.tex|, |cdocsfn2.tex|.
Then copy the file |childdoc.def| to an appropriate directory of your \LaTeX{}
distribution, e.g.\ \textit{texmf-root}|/tex/latex/childdoc|.
\end{itemize}

%%%%%%%%%%%%%%%%%%%%%%%%%%%%%%%%%%%%%%%%%%%%%%%%%%%%%%%%%%%%%%%%%%%%%%%%%%%%%%%%
\subsection{Related CTAN Packages}

There are several other packages which offer a similar functionality:
%
\begin{itemize}
\item
The packages
\href{http://ctan.org/pkg/docmute}{\textsf{docmute}},
\href{http://ctan.org/pkg/includex}{\textsf{includex}} and
\href{http://ctan.org/pkg/standalone}{\textsf{standalone}}
provide commands to include only the document body of
a child file thus allowing both files to be compiled individually.
\item
The packages \href{http://ctan.org/pkg/subdocs}{\textsf{subdocs}}
and \href{http://ctan.org/pkg/subfiles}{\textsf{subfiles}}
provide structures in which the main and child documents can be
encapsulated and allowing them to be compiled individually.
The inclusion mechanism is different from the conventional |\include|.
\item
The package \href{http://ctan.org/pkg/combine}{\textsf{combine}}
is an elaborate solution to combine several documents into one.
\end{itemize}
%
See also the CTAN topic \href{http://ctan.org/topic/subdocs}{\textsf{subdocs}}
for further related packages.
The present package differs from the above solutions in that
a document structure constructed with the conventional |\include| mechanism
just needs two extra commands at the top of every file
such that all constituent files can be compiled individually.

%%%%%%%%%%%%%%%%%%%%%%%%%%%%%%%%%%%%%%%%%%%%%%%%%%%%%%%%%%%%%%%%%%%%%%%%%%%%%%%%
%\subsection{Feature Suggestions}
%
%The following is a list of features which may be useful for future
%versions of this package:
%%
%\begin{itemize}
%\item
%\ldots
%\end{itemize}

%%%%%%%%%%%%%%%%%%%%%%%%%%%%%%%%%%%%%%%%%%%%%%%%%%%%%%%%%%%%%%%%%%%%%%%%%%%%%%%%
\subsection{Revision History}

%%%%%%%%%%%%%%%%%%%%%%%%%%%%%%%%%%%%%%%%
\paragraph{v2.0:} 2018/12/30

\begin{itemize}
\item
immediate forward processing
\item
added |\childdocby| mechanism
\item
manual restructured
\end{itemize}

%%%%%%%%%%%%%%%%%%%%%%%%%%%%%%%%%%%%%%%%
\paragraph{v1.6:} 2018/01/17

\begin{itemize}
\item
application for development of include files
\item
corrections to manual
\end{itemize}

%%%%%%%%%%%%%%%%%%%%%%%%%%%%%%%%%%%%%%%%
\paragraph{v1.5:} 2017/05/21

\begin{itemize}
\item
more complete structuring introduced
\item
|\childdocof| introduced
\item
|\childdoc| renamed to |\childdocmain|
\item
|\childredirect| renamed to |\childdocforward| and |\childdocforwardprefix|
and functionality expanded
\end{itemize}

%%%%%%%%%%%%%%%%%%%%%%%%%%%%%%%%%%%%%%%%
\paragraph{v1.0:} 2017/04/27

\begin{itemize}
\item
manual and install package
\item
first version published on CTAN
\end{itemize}

%%%%%%%%%%%%%%%%%%%%%%%%%%%%%%%%%%%%%%%%
\paragraph{v0.6:} 2017/04/26

\begin{itemize}
\item
redirection mechanism added
\end{itemize}

%%%%%%%%%%%%%%%%%%%%%%%%%%%%%%%%%%%%%%%%
\paragraph{v0.5:} 2017/04/26

\begin{itemize}
\item
functionality in definition file
\end{itemize}


%%%%%%%%%%%%%%%%%%%%%%%%%%%%%%%%%%%%%%%%%%%%%%%%%%%%%%%%%%%%%%%%%%%%%%%%%%%%%%%%
%%%%%%%%%%%%%%%%%%%%%%%%%%%%%%%%%%%%%%%%%%%%%%%%%%%%%%%%%%%%%%%%%%%%%%%%%%%%%%%%
%%%%%%%%%%%%%%%%%%%%%%%%%%%%%%%%%%%%%%%%%%%%%%%%%%%%%%%%%%%%%%%%%%%%%%%%%%%%%%%%
\appendix

\settowidth\MacroIndent{\rmfamily\scriptsize 000\ }

 \DocInput{childdoc.dtx}

\end{document}
%</driver>
% \fi
%
% %%%%%%%%%%%%%%%%%%%%%%%%%%%%%%%%%%%%%%%%%%%%%%%%%%%%%%%%%%%%%%%%%%%%%%%%%%%%%%
% %%%%%%%%%%%%%%%%%%%%%%%%%%%%%%%%%%%%%%%%%%%%%%%%%%%%%%%%%%%%%%%%%%%%%%%%%%%%%%
% \section{Sample}
%\iffalse
%<*samplemain>
%\fi
%
% The following presents a sample document
% with two chapters, two parts, a title page,
% a compile flag as well as three forwarding files to set the flag.
% It consists of eight |.tex| files:
% \begin{center}
% \begin{tabular}{ll}
% |cdocsamp.tex|&main file\\
% |cdocsch1.tex|&include file for chapter 1\\
% |cdocsch2.tex|&include file for chapter 2\\
% |cdocspt3.tex|&include file for part 3\\
% |cdocspt4.tex|&include file for part 4\\
% |cdocsdrf.tex|&forwarding file for main file in draft mode\\
% |cdocsfi1.tex|&forwarding file for final version of chapter 1\\
% |cdocsfi2.tex|&forwarding file for final version of chapter 2\\
% \end{tabular}
% \end{center}
% Each of the eight files can be compiled directly by the \LaTeX{} compiler.
%
% %%%%%%%%%%%%%%%%%%%%%%%%%%%%%%%%%%%%%%
% \paragraph{Main File.}
%
% The main file is called |cdocsamp.tex|.
%
% Load the \textsf{childdoc} definitions and
% declare the filename for the main document:
%    \begin{macrocode}
\input{childdoc.def}
\childdocmain{}
%    \end{macrocode}

% Optional override for |\version| flag:
%    \begin{macrocode}
%%\ifchilddoc\else\providecommand{\version}{draft}\fi
%    \end{macrocode}

% Define the default values for the |\version| flag
% (|final| for the main file and |draft| for childs):
%    \begin{macrocode}
\ifchilddoc
\providecommand{\version}{draft}
\else
\providecommand{\version}{final}
\fi
%    \end{macrocode}

% Load the standard document class:
%    \begin{macrocode}
\documentclass[12pt]{article}
%    \end{macrocode}

% Start the document body:
%    \begin{macrocode}
\begin{document}
%    \end{macrocode}

% Declare a title page.
% Print title, part of document being processed and version flag:
%    \begin{macrocode}
\addtocounter{page}{-1}
\begin{center}
{\LARGE\bfseries{}childdoc example\par}
\vspace{1cm}
\ifchilddoc
\ifchilddocmanual part\else chapter\fi:
`\childdocname' of `\childdocjob'\par
\else
main document: `\childdocjob'\par
\fi
version: \version\par
\end{center}
\newpage
%    \end{macrocode}

% Manually include selected file,
% otherwise process as usual:
%    \begin{macrocode}
\ifchilddocmanual
\section*{part `\childdocname'}
\input{\childdocname}
\else
%    \end{macrocode}

% Include the two chapters:
%    \begin{macrocode}
\include{cdocsch1}
\include{cdocsch2}
%    \end{macrocode}

% Include the two parts unless only chapters should be displayed:
%    \begin{macrocode}
\ifchilddoc\else
\section{part three}
\input{cdocspt3}
\section{part four}
\input{cdocspt4}
\fi
%    \end{macrocode}

% Process as usual until here:
%    \begin{macrocode}
\fi
%    \end{macrocode}

% End of document body:
%    \begin{macrocode}
\end{document}
%    \end{macrocode}
%\iffalse
%</samplemain>
%\fi
%
% %%%%%%%%%%%%%%%%%%%%%%%%%%%%%%%%%%%%%%
% \paragraph{Chapter Include Files.}
%
% The include files are called |cdocsch1.tex| and |cdocsch2.tex|.
%
%\iffalse
%<*samplechap1|samplechap2>
%\fi

% Optional override for |\version| flag:
%    \begin{macrocode}
%%\providecommand{\version}{final}
%    \end{macrocode}

% Include the main document:
%    \begin{macrocode}
\input{childdoc.def}
\childdocof{cdocsamp}
%    \end{macrocode}

%\iffalse
%</samplechap1|samplechap2>
%\fi
%
%\iffalse
%<*samplechap1>
%\fi
% Some text for chapter 1:
%    \begin{macrocode}
\section{one}
some text in chapter one
%    \end{macrocode}

%\iffalse
%</samplechap1>
%\fi
% Some text for chapter 2:
%\iffalse
%<*samplechap2>
%\fi
%    \begin{macrocode}
\section{two}
more text in chapter two
%    \end{macrocode}

%\iffalse
%</samplechap2>
%\fi
%
% %%%%%%%%%%%%%%%%%%%%%%%%%%%%%%%%%%%%%%
% \paragraph{Part Include Files.}
%
% The include files are called |cdocspt3.tex| and |cdocspt4.tex|.
%
%\iffalse
%<*samplepart3|samplepart4>
%\fi

% Optional override for |\version| flag:
%    \begin{macrocode}
%%\providecommand{\version}{final}
%    \end{macrocode}

% Include the main document:
%    \begin{macrocode}
\input{childdoc.def}
\childdocby{cdocsamp}
%    \end{macrocode}

%\iffalse
%</samplepart3|samplepart4>
%\fi
%
%\iffalse
%<*samplepart3>
%\fi
% Some text for part 3:
%    \begin{macrocode}
some text in part three
%    \end{macrocode}

%\iffalse
%</samplepart3>
%\fi
% Some text for part 4:
%\iffalse
%<*samplepart4>
%\fi
%    \begin{macrocode}
more text in part four
%    \end{macrocode}

%\iffalse
%</samplepart4>
%\fi
%
% %%%%%%%%%%%%%%%%%%%%%%%%%%%%%%%%%%%%%%
% \paragraph{Forwarding for a Complete Draft.}
%
% The following forwarding file |cdocsdrf.tex|
% compiles the main document in draft mode:
%\iffalse
%<*sampledraft>
%\fi
%    \begin{macrocode}
\def\version{draft}
\input{childdoc.def}
\childdocforward{cdocsamp}
%    \end{macrocode}

%\iffalse
%</sampledraft>
%\fi
%
% %%%%%%%%%%%%%%%%%%%%%%%%%%%%%%%%%%%%%%
% \paragraph{Forwarding for Final Version of the Chapters.}
%
% The following forwarding files |cdocsfn1.tex| and |cdocsfn2.tex|
% (with identical content)
% compile the final versions of the child documents
% |cdocsch1.tex| and |cdocsch2.tex|, respectively:
%\iffalse
%<*samplefinal>
%\fi
%    \begin{macrocode}
\def\version{final}
\input{childdoc.def}
\childdocforwardprefix[cdocsamp]{cdocsfn}{cdocsch}
%    \end{macrocode}

%\iffalse
%</samplefinal>
%\fi
%
% %%%%%%%%%%%%%%%%%%%%%%%%%%%%%%%%%%%%%%
% \paragraph{Command Line Processing.}
%
% The following three command lines generate the output files
% |cdocscld|, |cdocscl1| and |cdocscl2|
% which should be identical to
% |cdocsdrf|, |cdocsch1| and |cdocsfn2|, respectively:
% \begin{center}
% \begin{tabular}{l}
% |latex -jobname cdocscld \|\\
% |  "\def\version{draft}\input{childdoc.def}\childdocforward{cdocsamp}"|\\
% |latex -jobname cdocscl1 \|\\
% |  "\input{childdoc.def}\childdocforward[cdocsamp]{cdocsch1}"|\\
% |latex -jobname cdocscl2 \|\\
% |  "\def\version{final}\input{childdoc.def}\childdocforward{cdocsch2}"|
% \end{tabular}
% \end{center}
% Note that the trailing backslash on each first line
% merely continues the input to the second line
% (for convenient cut ant paste).
% Furthermore, the command |latex| can be replaced by any
% of its alternative versions such as |pdflatex|.
%
% %%%%%%%%%%%%%%%%%%%%%%%%%%%%%%%%%%%%%%%%%%%%%%%%%%%%%%%%%%%%%%%%%%%%%%%%%%%%%%
% %%%%%%%%%%%%%%%%%%%%%%%%%%%%%%%%%%%%%%%%%%%%%%%%%%%%%%%%%%%%%%%%%%%%%%%%%%%%%%
% \section{Implementation}
%\iffalse
%<*package>
%\fi
%
% This section describes the definitions file |childdoc.def|.

% The definitions cannot be loaded using |\usepackage| or |\RequirePackage|
% which has a mechanism to prevent loading a style file more than once.
% When loading the definitions by means of |\input|
% multiple instances have to be prevented manually:
%\iffalse
%This code needs to be before the `\ProvidesFile' directive
%which is defined at the beginning of this file.
%Therefore it is also placed there and commented out here.
%</package>
%<*discard>
%\fi
%    \begin{macrocode}
\ifdefined\childdocmain\endinput\fi
%    \end{macrocode}
%\iffalse
%</discard>
%<*package>
%\fi
%
% \macro{\ifchilddoc}
% \macro{\ifchilddocmanual}
% The conditional |\ifchilddoc| tells whether a
% child (true) or main (false) document is being compiled.
% The conditional |\ifchilddocmanual| tells whether
% the |\includeonly| mechanism is used (false) or
% the selection of child files must be performed manually (true).
% The definitions initialise to false:
%    \begin{macrocode}
\newif\ifchilddoc
\newif\ifchilddocmanual
%    \end{macrocode}

% \macro{\childdocname}
% \macro{\childdocjob}
% The macro |\childdocname| stores the name of the main document
% to be compiled. The macro |\childdocjob| stores the name of
% the document on which the \LaTeX{} compiler was originally invoked.
% The content of |\jobname| cannot be compared
% to filenames specified in the source due to different catcodes.
% The following code rescans |\jobname|, stores the result
% in |\childdocname| and saves a copy in |\childdocjob|:
%    \begin{macrocode}
\edef\childdocname{\scantokens\expandafter{\jobname\noexpand}}
\let\childdocjob\childdocname
%    \end{macrocode}

% \macro{\childdocdisable}
% The macro |\childdocdisable| prevents the main file
% from being processed more than once.
% At this stage, the main document command |\childdocmain|
% is assumed to be called once again where it should do nothing.
% Any subsequent call to it should prevent
% a secondary processing of the main document
% It overwrites the forwarding commands
% |\childdocof| and |\childdocforward|
% with empty macros to prevent further inclusions of the main document:
%    \begin{macrocode}
\newcommand{\childdocdisable}
{
  \renewcommand{\childdocmain}[1]{\renewcommand{\childdocmain}[1]{\endinput}}
  \renewcommand{\childdocof}[1]{}
  \renewcommand{\childdocby}[2][]{}
  \renewcommand{\childdocforward}[2][]{}
  \renewcommand{\childdocdisable}{}
}
%    \end{macrocode}

% \macro{\childdocmain}
% The macro |\childdocmain| is to be called at the top of the main file
% with nothing or the main filename (without extension) as argument.
% First, it breaks loops.
% If the argument is not empty and does not match |\childdocname|
% (which is set by the first inclusion of |childdoc.def|),
% |\ifchilddoc| is set to true, |\includeonly| is applied to the child file
% and |\jobname| is set to the main file
% (for proper handling of |.aux| files):
%    \begin{macrocode}
\newcommand{\childdocmain}[1]
{
  \childdocdisable\childdocmain{}
  \if?#1?\else
    \begingroup
      \def\childdoctmp{#1}
      \ifx\childdoctmp\childdocname
        \def\childdoctmp{}
      \else
        \def\childdoctmp
        {
          \childdoctrue
          \includeonly{\childdocname}
          \def\childdocjob{#1}
          \def\jobname{#1}
        }
      \fi
      \expandafter
    \endgroup
    \childdoctmp
  \fi
}
%    \end{macrocode}

% \macro{\childdocof}
% The command |\childdocof| redirects
% compilation to the main file |#1|.
%    \begin{macrocode}
\newcommand{\childdocof}[1]
{
  \childdocdisable
  \childdoctrue
  \includeonly{\childdocname}
  \def\jobname{#1}
  \def\childdocjob{#1}
  \input{#1}
}
%    \end{macrocode}

% \macro{\childdocby}
% The command |\childdocby| ....
%    \begin{macrocode}
\newcommand{\childdocby}[2][]
{
  \childdocdisable
  \childdoctrue
  \childdocmanualtrue
  \if?#1?\else
    \def\jobname{#2}
  \fi
  \def\childdocjob{#2}
  \input{#2}
  \endinput
}
%    \end{macrocode}

% \macro{\childdocforward}
% The command |\childdocforward| redirects
% compilation to the main file or
% (if the optional argument is given) a child file.
% Parameters are set as if the main file
% or a child file starting with |\childdocof| was compiled.
% Then compilation is handed over to the main file:
%    \begin{macrocode}
\newcommand{\childdocforward}[2][]
{
  \begingroup
    \if?#1?
      \def\childdoctmp
      {
        \def\childdocname{#2}
        \def\childdocjob{#2}
        \def\jobname{#2}
        \input{#2}
        \endinput
      }
    \else
      \def\childdoctmp
      {
        \childdocdisable
        \def\childdocname{#2}
        \childdoctrue
        \includeonly{#2}
        \def\childdocjob{#1}
        \def\jobname{#1}
        \input{#1}
        \endinput
      }
    \fi
    \expandafter
  \endgroup
  \childdoctmp
}
%    \end{macrocode}

% \macro{\childdocforwardprefix}
% The command |\childdocforwardprefix| redirects
% compilation to the main or a child file by means of a pattern.
% The prefix |#1| in the current filename is replaced by |#2|
% and the suffix of the current filename is kept
% (it is assumed that the filename does not contain the substring `|~~~|'
% which is used as a delimiter).
% Compilation is handed over to the new file by |\childdocforward|:
%    \begin{macrocode}
\newcommand{\childdocforwardprefix}[3][]
{
  \begingroup
    \def\childdocextract #2##1~~~{\def\childdoctmp{\childdocforward[#1]{#3##1}}}
    \expandafter\childdocextract\childdocname~~~
    \expandafter
  \endgroup
  \childdoctmp
}
%    \end{macrocode}

% \macro{\childdoc}
% The deprecated macro |\childdoc| is a legacy version of |\childdocmain|:
%    \begin{macrocode}
\newcommand{\childdoc}{\childdocmain}
%    \end{macrocode}

% \macro{\childdocredirect}
% The deprecated macro |\childdocredirect| is a legacy version
% of |\childdocforward| and |\childdocforwardprefix|:
%    \begin{macrocode}
\newcommand{\childdocredirect}[2][]
{
  \begingroup
    \if?#1?
      \def\childdoctmp{\childdocforward{#2}}
    \else
      \def\childdoctmp{\childdocforwardprefix{#1}{#2}}
    \fi
    \expandafter
  \endgroup
  \childdoctmp
}
%    \end{macrocode}

%\iffalse
%</package>
%\fi
%
\endinput

\childdocmain{}
%    \end{macrocode}

% Optional override for |\version| flag:
%    \begin{macrocode}
%%\ifchilddoc\else\providecommand{\version}{draft}\fi
%    \end{macrocode}

% Define the default values for the |\version| flag
% (|final| for the main file and |draft| for childs):
%    \begin{macrocode}
\ifchilddoc
\providecommand{\version}{draft}
\else
\providecommand{\version}{final}
\fi
%    \end{macrocode}

% Load the standard document class:
%    \begin{macrocode}
\documentclass[12pt]{article}
%    \end{macrocode}

% Start the document body:
%    \begin{macrocode}
\begin{document}
%    \end{macrocode}

% Declare a title page.
% Print title, part of document being processed and version flag:
%    \begin{macrocode}
\addtocounter{page}{-1}
\begin{center}
{\LARGE\bfseries{}childdoc example\par}
\vspace{1cm}
\ifchilddoc
\ifchilddocmanual part\else chapter\fi:
`\childdocname' of `\childdocjob'\par
\else
main document: `\childdocjob'\par
\fi
version: \version\par
\end{center}
\newpage
%    \end{macrocode}

% Manually include selected file,
% otherwise process as usual:
%    \begin{macrocode}
\ifchilddocmanual
\section*{part `\childdocname'}
\input{\childdocname}
\else
%    \end{macrocode}

% Include the two chapters:
%    \begin{macrocode}
\include{cdocsch1}
\include{cdocsch2}
%    \end{macrocode}

% Include the two parts unless only chapters should be displayed:
%    \begin{macrocode}
\ifchilddoc\else
\section{part three}
\input{cdocspt3}
\section{part four}
\input{cdocspt4}
\fi
%    \end{macrocode}

% Process as usual until here:
%    \begin{macrocode}
\fi
%    \end{macrocode}

% End of document body:
%    \begin{macrocode}
\end{document}
%    \end{macrocode}
%\iffalse
%</samplemain>
%\fi
%
% %%%%%%%%%%%%%%%%%%%%%%%%%%%%%%%%%%%%%%
% \paragraph{Chapter Include Files.}
%
% The include files are called |cdocsch1.tex| and |cdocsch2.tex|.
%
%\iffalse
%<*samplechap1|samplechap2>
%\fi

% Optional override for |\version| flag:
%    \begin{macrocode}
%%\providecommand{\version}{final}
%    \end{macrocode}

% Include the main document:
%    \begin{macrocode}
% \iffalse
%
% childdoc.dtx Copyright (C) 2017-2018 Niklas Beisert
%
% This work may be distributed and/or modified under the
% conditions of the LaTeX Project Public License, either version 1.3
% of this license or (at your option) any later version.
% The latest version of this license is in
%   http://www.latex-project.org/lppl.txt
% and version 1.3 or later is part of all distributions of LaTeX
% version 2005/12/01 or later.
%
% This work has the LPPL maintenance status `maintained'.
%
% The Current Maintainer of this work is Niklas Beisert.
%
% This work consists of the files childdoc.dtx and childdoc.ins
% and the derived files childdoc.def and cdocsamp.tex with
% cdocsch1.tex, cdocsch2.tex, cdocsdrf.tex, cdocsfn1.tex, cdocsfn2.tex.
%
%<package>\ifdefined\childdocmain\endinput\fi
%<package>\ProvidesFile{childdoc.def}[2018/12/30 v2.0 child document driver]
%<samplemain>\ProvidesFile{cdocsamp.tex}[2018/12/30 v2.0 sample for childdoc]
%<*driver>
%\ProvidesFile{childdoc.drv}[2018/12/30 v2.0 childdoc reference manual file]
\PassOptionsToClass{10pt,a4paper}{article}
\documentclass{ltxdoc}

\usepackage[margin=35mm]{geometry}
\usepackage{hyperref}
\usepackage{hyperxmp}
\usepackage[usenames]{color}

\hypersetup{colorlinks=true}
\hypersetup{pdfstartview=FitH}
\hypersetup{pdfpagemode=UseNone}
\hypersetup{pdfsource={}}
\hypersetup{pdflang={en-UK}}
\hypersetup{pdfcopyright={Copyright 2017-2018 Niklas Beisert.
  This work may be distributed and/or modified under the
  conditions of the LaTeX Project Public License, either version 1.3
  of this license or (at your option) any later version.}}
\hypersetup{pdflicenseurl={http://www.latex-project.org/lppl.txt}}
\hypersetup{pdfcontactaddress={ETH Zurich, ITP, HIT K,
  Wolfgang-Pauli-Strasse 27}}
\hypersetup{pdfcontactpostcode={8093}}
\hypersetup{pdfcontactcity={Zurich}}
\hypersetup{pdfcontactcountry={Switzerland}}
\hypersetup{pdfcontactemail={nbeisert@itp.phys.ethz.ch}}
\hypersetup{pdfcontacturl={http://people.phys.ethz.ch/\xmptilde nbeisert/}}

\newcommand{\secref}[1]{\hyperref[#1]{section \ref*{#1}}}

\parskip1ex
\parindent0pt
\let\olditemize\itemize
\def\itemize{\olditemize\parskip0pt}

\begin{document}

\title{The \textsf{childdoc} Package}
\hypersetup{pdftitle={The childdoc Package}}
\author{Niklas Beisert\\[2ex]
  Institut f\"ur Theoretische Physik\\
  Eidgen\"ossische Technische Hochschule Z\"urich\\
  Wolfgang-Pauli-Strasse 27, 8093 Z\"urich, Switzerland\\[1ex]
  \href{mailto:nbeisert@itp.phys.ethz.ch}
  {\texttt{nbeisert@itp.phys.ethz.ch}}}
\hypersetup{pdfauthor={Niklas Beisert}}
\hypersetup{pdfsubject={Manual for the LaTeX2e Package childdoc}}
\date{30 December 2018, \textsf{v2.0}}
\maketitle

\begin{abstract}\noindent
\textsf{childdoc} is a \LaTeXe{} package
that enables the direct compilation
of document sections included by |\include|
to individual files.
\end{abstract}

\begingroup
\parskip0ex
\tableofcontents
\endgroup

%%%%%%%%%%%%%%%%%%%%%%%%%%%%%%%%%%%%%%%%%%%%%%%%%%%%%%%%%%%%%%%%%%%%%%%%%%%%%%%%
%%%%%%%%%%%%%%%%%%%%%%%%%%%%%%%%%%%%%%%%%%%%%%%%%%%%%%%%%%%%%%%%%%%%%%%%%%%%%%%%
\section{Introduction}

\LaTeX{} provides a mechanism to structure a large document (such as a book)
into a main file and several child files (containing the chapters)
using the |\include| command.
This mechanism is beneficial for documents
which span hundreds of pages in order to
make the source file(s) more manageable.
Moreover, compilation can be restricted to
selected child files by means of the |\includeonly| command.
The latter feature can be used to reduce the compilation time while editing
(this was significantly more useful in the earlier days of \LaTeX{})
or to generate a smaller document which is easier to navigate.
Another application of |\includeonly| is to generate
documents consisting of selected parts of the complete document.

However, there are a few drawbacks of the plain |\include| mechanism:
\begin{itemize}
\item
The child files cannot be compiled on their own,
they can only be compiled via the main file.
A naive editing environment
(such as a text editor with an option
to have the current file processed by \LaTeX)
may require one to switch to the main file before compiling;
attempting to compile the child file produces errors.
\item
The main file must be modified (each time)
to adjust the |\includeonly| command
to the present needs. This easily leaves the main file in a messy state.
\item
The generated document will always carry the filename
of the main document. This is inconvenient if
several child files are to be compiled and
to be kept for distribution.
\end{itemize}

The present package provides a simple interface
to make child files individually compilable by \LaTeX{}.
Compiling a child file then has the same effect as compiling
the main file with an |\includeonly| command
to select the appropriate child.
Moreover the generated document will carry the name of the child
rather than the main file.
This resolves all three above issues.

This feature is meant to make the editing of books,
thesis documents and lecture notes somewhat more convenient.
However, the package can also be used efficiently for
composing a series of documents (such as exercise sheets)
which are typically distributed individually.
It then assists the author in generating the individual documents
(potentially in different versions)
as well as a document containing the collected series.
Another application is in developing style files
or other kinds of included material
where compilation of the style file could redirect
to a sample or test file.

%%%%%%%%%%%%%%%%%%%%%%%%%%%%%%%%%%%%%%%%%%%%%%%%%%%%%%%%%%%%%%%%%%%%%%%%%%%%%%%%
%%%%%%%%%%%%%%%%%%%%%%%%%%%%%%%%%%%%%%%%%%%%%%%%%%%%%%%%%%%%%%%%%%%%%%%%%%%%%%%%
\section{Usage}

First of all, the package \textsf{childdoc} is \emph{not} a standard
\LaTeXe{} |.sty| style file! Therefore it needs to be invoked in
a non-standard way.

%%%%%%%%%%%%%%%%%%%%%%%%%%%%%%%%%%%%%%%%%%%%%%%%%%%%%%%%%%%%%%%%%%%%%%%%%%%%%%%%
\subsection{Included Files}
\label{sec:include}

%%%%%%%%%%%%%%%%%%%%%%%%%%%%%%%%%%%%%%%%
\DescribeMacro{\childdocmain}
To use the package, add the commands
\begin{center}
\begin{tabular}{l}
|\input{childdoc.def}|\\
|\childdocmain{}|\\
\end{tabular}
\end{center}
at the very top of the main \LaTeX{} file,
in particular \emph{before} the |\documentclass| statement!
The argument of |\childdocmain| should be left empty
(but it must be present).

%%%%%%%%%%%%%%%%%%%%%%%%%%%%%%%%%%%%%%%%
\DescribeMacro{\childdocof}
Furthermore, add the commands
\begin{center}
\begin{tabular}{l}
|\input{childdoc.def}|\\
|\childdocof{|\textit{main}|}|\\
\end{tabular}
\end{center}
at the top of every child file \textit{child}
which is included by |\include{|\textit{child}|}|
from within the main file
(or at least for those files to be compiled individually).
The argument \textit{main} must be the filename of the main file.

There are a couple of
considerations in setting up the main and child documents:

%%%%%%%%%%%%%%%%%%%%%%%%%%%%%%%%%%%%%%%%
\paragraph{Restrictions.}

Please note the following restrictions:
\begin{itemize}
\item
|\childdocmain| must be called with one argument \textit{main}
to ensure compatibility with earlier version of the package.
It must either be empty (|\childdocmain{}|)
or precisely match the filename of the main file in which it is specified.
See \secref{sec:detection} for further information.
\item
The filename \textit{main} must be specified without the |.tex| extension.
\item
The filename \textit{main} is case sensitive
(even in case-insensitive file systems)
due to internal string comparison.
\item
The argument \textit{main} should be fully expanded, it cannot be a macro.
\item
Subdirectories and special characters should be avoided in filenames.
\item
The command |\childdocmain{|\textit{main}|}| must be followed by a whitespace.
It should not be followed immediately by another command
or by a comment mark `|%|'.
This is because the \TeX{} parser reads the token immediately following
the argument of |\childdocmain| and puts it
at the beginning of every child section;
however, a white\-space is ignored.
\end{itemize}

%%%%%%%%%%%%%%%%%%%%%%%%%%%%%%%%%%%%%%%%
\paragraph{Content of Main File.}

It is advisable to place all content in the child files included by |\include|.
Any output contained in the main file will appear in all child documents
unless suppressed manually;
it cannot be suppressed automatically by the |\includeonly| directive
and thus should normally be avoided.
A method to include some content in the main file
by means of conditional processing is described in \secref{sec:conditional}.

%%%%%%%%%%%%%%%%%%%%%%%%%%%%%%%%%%%%%%%%
\paragraph{Page Numbering.}

When only a part of the document is compiled,
the appropriate numbering of pages
(as well as other status parameters)
is determined from the |.aux| files.
The latter contain information from previous passes.
However this information needs to propagate through
all intermediate child documents.
Therefore the page numbering in child documents may well
be inconsistent until the complete document is compiled at least once.

A useful (if unconventional) way to always ensure a consistent
page numbering is to restart the numbering in each child document
and denote the pages by `\textit{child}|.|\textit{page}'
where \textit{child} represents the chapter/section number of the child file.
This can be achieved by the command
|\numberwithin{page}{|\textit{child}|}|
of the \textsf{amsmath} package
where \textit{child} can be |chapter| or |section|
depending on the chosen structuring.
Alternatively, one can modify the macro |\thepage| appropriately
and reset the counter |page| at the start of each child file.

%%%%%%%%%%%%%%%%%%%%%%%%%%%%%%%%%%%%%%%%%%%%%%%%%%%%%%%%%%%%%%%%%%%%%%%%%%%%%%%%
\subsection{Conditional Processing}
\label{sec:conditional}

The package provides a mechanism to compile different versions
of a document. To customise the versions further some conditional processing
can come in handy to distinguish which version is being compiled.
The package provides two macros to describe the compilation context:

%%%%%%%%%%%%%%%%%%%%%%%%%%%%%%%%%%%%%%%%
\DescribeMacro{\ifchilddoc}
The conditional |\ifchilddoc| distinguishes between the compilation of
child documents and the main document:
%
\begin{center}
|\ifchilddoc |\textit{child-code}| |[|\||else |\textit{main-code}]| \||fi|
\end{center}

%%%%%%%%%%%%%%%%%%%%%%%%%%%%%%%%%%%%%%%%
\DescribeMacro{\childdocname}
\DescribeMacro{\childdocjob}
The macro |\childdocname| contains the filename (without extension)
of the main or child file being processed.
Note that |\childdocjob| will always contain the name of the main file.

%%%%%%%%%%%%%%%%%%%%%%%%%%%%%%%%%%%%%%%%
\paragraph{Title Page.}

Conditional processing can be used to include a title or banner page
in the main document when proper precautions are taken.
Importantly, the code in the main file should ensure that the page counter
(as well as other status parameters which are stored in the |.aux| files)
takes the same value after the conditional processing.
Otherwise the page numbers may take divergent values
depending on which part is compiled.

For example, a title page could be declared by:
%
\begin{center}
\begin{tabular}{l}
|\ifchilddoc\||else|\\
|\addtocounter{page}{-1}|\\
\textit{code for title page}\\
|\newpage|\\
|\||fi|
\end{tabular}
\end{center}
%
A banner page for the child documents can be generated by:
%
\begin{center}
\begin{tabular}{l}
|\ifchilddoc|\\
|\addtocounter{page}{-1}|\\
\textit{code for banner page}\\
|\newpage|\\
|\||fi|
\end{tabular}
\end{center}
%
Here one could write a message such as:
\begin{center}
|This is the part \childdocname{} of \childdocjob{}.|
\end{center}

%%%%%%%%%%%%%%%%%%%%%%%%%%%%%%%%%%%%%%%%%%%%%%%%%%%%%%%%%%%%%%%%%%%%%%%%%%%%%%%%
\subsection{Flags}
\label{sec:flags}

The package makes it easy to generate different versions
of the main or child documents.
To this end compilation flags can be defined
and assigned different default values.
They will be particularly useful in conjunction
with the forwarding mechanism described in \secref{sec:forward}.

For example, it may be useful to have a flag |\version|
which can be set to |draft| or |final|.
The document source will contain some conditional code
depending on the value of |\version|.
Suppose further, the flag should default to |final| for the main file
and to |draft| for child files
which is a natural assignment for editing the document.
This is achieved by placing the following code
in the preamble of the main document
(below the |\childdocmain| directive):
%
\begin{center}
\begin{tabular}{l}
|\ifchilddoc|\\
|\providecommand{\version}{draft}|\\
|\||else|\\
|\providecommand{\version}{final}|\\
|\||fi|
\end{tabular}
\end{center}
%
The definition by |\providecommand| makes sure
that previous definitions are not overwritten.
Further statements |\providecommand{\version}{...}|
can thus be added before the above code to override it.

For the main file, one might add a line
(between |\childdocmain| and the above block)
%
\begin{center}
|%\ifchilddoc\||else\providecommand{\version}{draft}\||fi|
\end{center}
%
which can be uncommented to produce a draft version.
Likewise one can add a line to the very top of a child file
(above the |\childdocof{|\textit{main}|}| directive)
%
\begin{center}
|%\providecommand{\version}{final}|
\end{center}
%
which can be uncommented to produce the final version of this child document.

%%%%%%%%%%%%%%%%%%%%%%%%%%%%%%%%%%%%%%%%%%%%%%%%%%%%%%%%%%%%%%%%%%%%%%%%%%%%%%%%
\subsection{Forwarding}
\label{sec:forward}

Different versions of the main or child documents
using compilation flags as described in \secref{sec:flags}
can be (permanently) stored in different files
for convenient compilation, viewing and distribution.
To this end, the package defines a command
to pass on compilation to a different file:

%%%%%%%%%%%%%%%%%%%%%%%%%%%%%%%%%%%%%%%%
\DescribeMacro{\childdocforward}
The command |\childdocforward| redirects processing to
another source file:
%
\begin{center}
\begin{tabular}{l}
|\input{childdoc.def}|\\
|\childdocforward[|\textit{main}|]{|\textit{dest}|}|\\
\end{tabular}
\end{center}
%
The argument \textit{dest} is the destination file
(without extension).
It should be the main file or one of the child files.
Note that further \textsf{childdoc} directives
such as |\childdocof| and |\childdocforward|
in the indicated file will be processed in this form.
The optional argument \textit{main}
passes on directly to the main file \textit{main}
while pretending to compile the child \textit{dest}.
This form behaves as if \textit{dest}
issues |\childdocof{|\textit{main}|}| right away,
and no further \textsf{childdoc} directives will be processed.

%%%%%%%%%%%%%%%%%%%%%%%%%%%%%%%%%%%%%%%%
\DescribeMacro{\...prefix}
In the alternative form |\childdocforwardprefix|,
%
\begin{center}
\begin{tabular}{l}
|\input{childdoc.def}|\\
|\childdocforwardprefix[|\textit{main}|]{|\textit{prefix}|}{|\textit{dest}|}|
\end{tabular}
\end{center}
%
the destination file is determined by a pattern
depending on the current file:
To make this work, the current file must be called
`{\textit{prefix}\hspace{0.2em}\textit{suffix}}'
with \textit{prefix} matching precisely the argument.
Processing is then passed on to the file
`{\textit{dest}\hspace{0.2em}\textit{suffix}}'.
Surely, the same effect is achieved by
directly specifying the
argument `{\textit{dest}\hspace{0.2em}\textit{suffix}}'
in the first form.
However, that requires to set up a different file
for each child. With the alternative form of the command
all these files can have exactly the same content
which simplifies setting them up and maintaining them.

For example, the following file |draft.tex|
with a compilation flag |\version| as described in \secref{sec:flags}
compiles the main document as a draft:
%
\begin{center}
\begin{tabular}{l}
|\def\version{draft}|\\
|\input{childdoc.def}|\\
|\childdocforward{|\textit{main}|}|
\end{tabular}
\end{center}
%
Likewise, the following files |final|\textit{nn}|.tex|
compile the final version of the child document
|child|\textit{nn}|.tex|:
%
\begin{center}
\begin{tabular}{l}
|\def\version{final}|\\
|\input{childdoc.def}|\\
|\childdocforwardprefix{final}{child}|
\end{tabular}
\end{center}
%

Note that when several versions of a main file and/or of each child file
are to be generated, it may be convenient to set up a |Makefile| or
shell script to automatise the process.

%%%%%%%%%%%%%%%%%%%%%%%%%%%%%%%%%%%%%%%%%%%%%%%%%%%%%%%%%%%%%%%%%%%%%%%%%%%%%%%%
\subsection{Command Line Processing}
\label{sec:commandline}

The effect of redirection files can also be achieved by invoking
the \LaTeX{} compiler with a more elaborate command line.
Most conveniently this should be done as part
of a shell script or a |Makefile|.

When using \textsf{childdoc} in the main file, the following
command lines effectively perform a redirection
(note that depending on the shell being used,
backslashes may have to be doubled: `|\|' $\to$ `|\\|'):
%
\begin{center}
|... -jobname "|\textit{target}|" |\\|"|[\textit{flags}]%
|\input{childdoc.def}\childdocforward[|\textit{main}|]{|\textit{dest}|}"|
\end{center}
%
Here \textit{target} is the name of the output file,
\textit{main} is the name of the main file
and \textit{dest} is the name of the main or child file to be processed
(all filenames without extensions).
The optional argument \textit{main} can be omitted
if \textit{main} matches \textit{dest}.
Optionally, compilation \textit{flags} can be defined via |\def| commands.
This command line makes the \TeX{} engine believe
it is compiling the file \textit{target}
whose content is specified as the latter parameter.
The provided code then forwards the processing to
\textit{main} or \textit{dest} as described in \secref{sec:forward}.

%%%%%%%%%%%%%%%%%%%%%%%%%%%%%%%%%%%%%%%%%%%%%%%%%%%%%%%%%%%%%%%%%%%%%%%%%%%%%%%%
\subsection{Include by Input}
\label{sec:input}

Including child documents by |\include| has some restrictions by design.
Most notably, the content of a child document always occupies
its own set of pages; pages cannot be shared between child documents.
Usually, this behaviour makes perfect sense
because each child document contain an essential part of the document.
However, in some situations it may be desirable to compose
a document from a collection of parts
without having mandatory page breaks between then.
For this case, the package
provides a mechanism to include parts
by |\input| which can also be processed individually.
However, by construction this mechanism
requires manual handling of the content to be output.

%%%%%%%%%%%%%%%%%%%%%%%%%%%%%%%%%%%%%%%%
\DescribeMacro{\ifchilddocmanual}
The main file should be prepared as usual, see \secref{sec:include}.
However, the document body must make a distinction
between processing of an individual part and of the main document, e.g.:
%
\begin{center}
\begin{tabular}{l}
|\ifchilddocmanual|\\
|\input{\childdocname}|\\
|\||else|\\
\textit{document body with }|\input{|\textit{part}|}|\\
|\||fi|
\end{tabular}
\end{center}
%
The conditional |\ifchilddocmanual| is true whenever
a part to be included by |\input| is being compiled,
and the name of the part is stored in |\childdocname|.

%%%%%%%%%%%%%%%%%%%%%%%%%%%%%%%%%%%%%%%%
\DescribeMacro{\childdocby}
Each part to be included by |\input| should start with:
%
\begin{center}
\begin{tabular}{l}
|\input{childdoc.def}|\\
|\childdocby{|\textit{main}|}|\\
\end{tabular}
\end{center}
%
The directive |\childdocby| is similar to |\childdocof|
described in \secref{sec:include},
but the subsequent selection of content must be done manually.
To that end, both |\ifchilddoc| and |\ifchilddocmanual|
will be true upon processing of a part,
and the name of the part is stored in |\childdocname|.
Note that |\jobname| will be set to the filename of the current part
so that each part receives an individual |.aux| file
that does not interfere with the |.aux| file(s) of the main document.
This behaviour can be altered by the alternative form
|\childdocby[*]{|\textit{main}|}| (with a non-empty optional argument)
which uses the |.aux| file of the main document
by setting |\jobname| to \textit{main}.

%%%%%%%%%%%%%%%%%%%%%%%%%%%%%%%%%%%%%%%%%%%%%%%%%%%%%%%%%%%%%%%%%%%%%%%%%%%%%%%%
\subsection{Driver Development}
\label{sec:driver}

The \textsf{childdoc} mechanism can also be use for the development
of definition files such as \LaTeX{} styles or classes.
This case differs from the above setup with multiple parts
included by |\include| in that no |\includeonly| should be invoked.
This can be achieved by starting the include file
(before |\ProvidesPackage|) with:
%
\begin{center}
\begin{tabular}{l}
|\input{childdoc.def}|\\
|\childdocforward{|\textit{main}|}|\\
\end{tabular}
\end{center}
%
or alternatively with:
%
\begin{center}
\begin{tabular}{l}
|\input{childdoc.def}|\\
|\childdocby{|\textit{main}|}|\\
\end{tabular}
\end{center}
%
Both forms have slightly different effects as described above.
The main file is prepared as usual, see \secref{sec:include}.

%%%%%%%%%%%%%%%%%%%%%%%%%%%%%%%%%%%%%%%%%%%%%%%%%%%%%%%%%%%%%%%%%%%%%%%%%%%%%%%%
\subsection{Legacy Detection}
\label{sec:detection}

The directive |\childdocmain| in the main file can detect
whether the complete document or merely a child is to be compiled
even without using the directive |\childdocof|.
This method is deprecated because it is less robust
and there is no compelling reason to use it;
it is merely provided for backward compatibility
and it may be removed in future versions.

If the detection mechanism is to be used,
it is mandatory to correctly specify
the filename of the main file as the argument of |\childdocmain|:
%
\begin{center}
\begin{tabular}{l}
|\input{childdoc.def}|\\
|\childdocmain{|\textit{main}|}|\\
\end{tabular}
\end{center}
%
If |\jobname| does not match the argument \textit{main} of |\childdocmain|,
it is assumed that |\jobname| points to the child file to be compiled.
When using |\childdocmain| with the main file specified as argument,
it suffices to start a child file
with just |\input{|\textit{main}|}|
without loading of the package and using |\childdocof|.
If instead all processing is done
with the appropriate \textsf{childdoc} directives,
the argument of \textit{main} of |\childdocmain| can be empty.

An alternative version of the command line processing described
in \secref{sec:commandline} using the detection mechanism reads:
%
\begin{center}
|... -jobname "|\textit{target}|" "|[\textit{flags}]%
[|\def\jobname{|\textit{dest}|}|]|\input{|\textit{main}|}"|
\end{center}

%%%%%%%%%%%%%%%%%%%%%%%%%%%%%%%%%%%%%%%%%%%%%%%%%%%%%%%%%%%%%%%%%%%%%%%%%%%%%%%%
\subsection{Manual Code}
\label{sec:manual}

In case one cannot be certain whether the definitions file |childdoc.def|
is installed on the target \TeX{} distribution
and one prefers not to ship it,
it is conceivable to paste a few relevant commands into the sources.

To that end, drop all statements |\input{childdoc.def}|
and perform the replacements as outlined below.
Instead of |\childdocmain{|\textit{main}|}| add the following code
to the top of the main file:
%
\begin{center}
\begin{tabular}{l}
|\||ifdefined\childdocname\endinput\||fi\newif\ifchilddoc|\\
|\edef\childdocname{\scantokens\expandafter{\jobname\noexpand}}|\\
|\def\childdocmain{|\textit{main}|}\||ifx\childdocmain\childdocname\||else|\\
|\childdoctrue\includeonly{\childdocname}\let\jobname\childdocmain\||fi|\\
\end{tabular}
\end{center}
%
Instead of |\childdocof{|\textit{main}|}| just include the main file
at the top of each child file:
%
\begin{center}
|\input{|\textit{main}|}|
\end{center}
%
A simple redirection |\childdocforward{|\textit{dest}|}| is achieved by:
%
\begin{center}
|\def\jobname{|\textit{dest}|}\input{\jobname}|
\end{center}
%
The redirection with prefix
|\childdocforwardprefix[|\textit{prefix}|]{|\textit{dest}|}|
is accomplished by:
%
\begin{center}
\begin{tabular}{l}
|{\edef\jobname{\scantokens\expandafter{\jobname\noexpand}}|\\
|\def\redirectjob |\textit{prefix}|#1~~~{\gdef\jobname{|\textit{dest}|#1}}|\\
|\expandafter\redirectjob\jobname~~~}\input{\jobname}|
\end{tabular}
\end{center}

In an alternative approach,
child documents can be compiled by a specific command line
without additional code or specific definitions:
%
\begin{center}
|... -jobname "|\textit{target}|" "|[\textit{flags}]%
|\includeonly{|\textit{dest}|}\input{|\textit{main}|}"|
\end{center}
%

%%%%%%%%%%%%%%%%%%%%%%%%%%%%%%%%%%%%%%%%%%%%%%%%%%%%%%%%%%%%%%%%%%%%%%%%%%%%%%%%
%%%%%%%%%%%%%%%%%%%%%%%%%%%%%%%%%%%%%%%%%%%%%%%%%%%%%%%%%%%%%%%%%%%%%%%%%%%%%%%%
\section{Information}

%%%%%%%%%%%%%%%%%%%%%%%%%%%%%%%%%%%%%%%%%%%%%%%%%%%%%%%%%%%%%%%%%%%%%%%%%%%%%%%%
\subsection{Copyright}

Copyright \copyright{} 2017--2018 Niklas Beisert

This work may be distributed and/or modified under the
conditions of the \LaTeX{} Project Public License, either version 1.3
of this license or (at your option) any later version.
The latest version of this license is in
  \url{http://www.latex-project.org/lppl.txt}
and version 1.3 or later is part of all distributions of \LaTeX{}
version 2005/12/01 or later.

This work has the LPPL maintenance status `maintained'.

The Current Maintainer of this work is Niklas Beisert.

This work consists of the files |README.txt|, |childdoc.ins| and |childdoc.dtx|
as well as the derived files |childdoc.def|, |cdocsamp.tex|
with |cdocsch1.tex|, |cdocsch2.tex|, |cdocspt3.tex|, |cdocspt4.tex|,
|cdocsdrf.tex|, |cdocsfn1.tex|, |cdocsfn2.tex|
as well as |childdoc.pdf|.

%%%%%%%%%%%%%%%%%%%%%%%%%%%%%%%%%%%%%%%%%%%%%%%%%%%%%%%%%%%%%%%%%%%%%%%%%%%%%%%%
\subsection{Files and Installation}

The package consists of the files:
%
\begin{center}
\begin{tabular}{ll}
    |README.txt|   & readme file \\
    |childdoc.ins| & installation file \\
    |childdoc.dtx| & source file \\
    |childdoc.def| & definition file \\
    |cdocsamp.tex| & sample main file \\
    |cdocsch1.tex| & sample include file \\
    |cdocsch2.tex| & sample include file \\
    |cdocspt3.tex| & sample part file \\
    |cdocspt4.tex| & sample part file \\
    |cdocsdrf.tex| & sample redirection file \\
    |cdocsfn1.tex| & sample redirection file \\
    |cdocsfn2.tex| & sample redirection file \\
    |childdoc.pdf| & manual
\end{tabular}
\end{center}
%
The distribution consists of the files
|README.txt|, |childdoc.ins| and |childdoc.dtx|.
%
\begin{itemize}
\item
Run (pdf)\LaTeX{} on |childdoc.dtx|
to compile the manual |childdoc.pdf| (this file).
\item
Run \LaTeX{} on |childdoc.ins| to create the definitions file |childdoc.def|
and the sample |cdocsamp.tex| with include files
|cdocsch1.tex|, |cdocsch2.tex|, |cdocspt3.tex|, |cdocspt4.tex|,
|cdocsdrf.tex|, |cdocsfn1.tex|, |cdocsfn2.tex|.
Then copy the file |childdoc.def| to an appropriate directory of your \LaTeX{}
distribution, e.g.\ \textit{texmf-root}|/tex/latex/childdoc|.
\end{itemize}

%%%%%%%%%%%%%%%%%%%%%%%%%%%%%%%%%%%%%%%%%%%%%%%%%%%%%%%%%%%%%%%%%%%%%%%%%%%%%%%%
\subsection{Related CTAN Packages}

There are several other packages which offer a similar functionality:
%
\begin{itemize}
\item
The packages
\href{http://ctan.org/pkg/docmute}{\textsf{docmute}},
\href{http://ctan.org/pkg/includex}{\textsf{includex}} and
\href{http://ctan.org/pkg/standalone}{\textsf{standalone}}
provide commands to include only the document body of
a child file thus allowing both files to be compiled individually.
\item
The packages \href{http://ctan.org/pkg/subdocs}{\textsf{subdocs}}
and \href{http://ctan.org/pkg/subfiles}{\textsf{subfiles}}
provide structures in which the main and child documents can be
encapsulated and allowing them to be compiled individually.
The inclusion mechanism is different from the conventional |\include|.
\item
The package \href{http://ctan.org/pkg/combine}{\textsf{combine}}
is an elaborate solution to combine several documents into one.
\end{itemize}
%
See also the CTAN topic \href{http://ctan.org/topic/subdocs}{\textsf{subdocs}}
for further related packages.
The present package differs from the above solutions in that
a document structure constructed with the conventional |\include| mechanism
just needs two extra commands at the top of every file
such that all constituent files can be compiled individually.

%%%%%%%%%%%%%%%%%%%%%%%%%%%%%%%%%%%%%%%%%%%%%%%%%%%%%%%%%%%%%%%%%%%%%%%%%%%%%%%%
%\subsection{Feature Suggestions}
%
%The following is a list of features which may be useful for future
%versions of this package:
%%
%\begin{itemize}
%\item
%\ldots
%\end{itemize}

%%%%%%%%%%%%%%%%%%%%%%%%%%%%%%%%%%%%%%%%%%%%%%%%%%%%%%%%%%%%%%%%%%%%%%%%%%%%%%%%
\subsection{Revision History}

%%%%%%%%%%%%%%%%%%%%%%%%%%%%%%%%%%%%%%%%
\paragraph{v2.0:} 2018/12/30

\begin{itemize}
\item
immediate forward processing
\item
added |\childdocby| mechanism
\item
manual restructured
\end{itemize}

%%%%%%%%%%%%%%%%%%%%%%%%%%%%%%%%%%%%%%%%
\paragraph{v1.6:} 2018/01/17

\begin{itemize}
\item
application for development of include files
\item
corrections to manual
\end{itemize}

%%%%%%%%%%%%%%%%%%%%%%%%%%%%%%%%%%%%%%%%
\paragraph{v1.5:} 2017/05/21

\begin{itemize}
\item
more complete structuring introduced
\item
|\childdocof| introduced
\item
|\childdoc| renamed to |\childdocmain|
\item
|\childredirect| renamed to |\childdocforward| and |\childdocforwardprefix|
and functionality expanded
\end{itemize}

%%%%%%%%%%%%%%%%%%%%%%%%%%%%%%%%%%%%%%%%
\paragraph{v1.0:} 2017/04/27

\begin{itemize}
\item
manual and install package
\item
first version published on CTAN
\end{itemize}

%%%%%%%%%%%%%%%%%%%%%%%%%%%%%%%%%%%%%%%%
\paragraph{v0.6:} 2017/04/26

\begin{itemize}
\item
redirection mechanism added
\end{itemize}

%%%%%%%%%%%%%%%%%%%%%%%%%%%%%%%%%%%%%%%%
\paragraph{v0.5:} 2017/04/26

\begin{itemize}
\item
functionality in definition file
\end{itemize}


%%%%%%%%%%%%%%%%%%%%%%%%%%%%%%%%%%%%%%%%%%%%%%%%%%%%%%%%%%%%%%%%%%%%%%%%%%%%%%%%
%%%%%%%%%%%%%%%%%%%%%%%%%%%%%%%%%%%%%%%%%%%%%%%%%%%%%%%%%%%%%%%%%%%%%%%%%%%%%%%%
%%%%%%%%%%%%%%%%%%%%%%%%%%%%%%%%%%%%%%%%%%%%%%%%%%%%%%%%%%%%%%%%%%%%%%%%%%%%%%%%
\appendix

\settowidth\MacroIndent{\rmfamily\scriptsize 000\ }

 \DocInput{childdoc.dtx}

\end{document}
%</driver>
% \fi
%
% %%%%%%%%%%%%%%%%%%%%%%%%%%%%%%%%%%%%%%%%%%%%%%%%%%%%%%%%%%%%%%%%%%%%%%%%%%%%%%
% %%%%%%%%%%%%%%%%%%%%%%%%%%%%%%%%%%%%%%%%%%%%%%%%%%%%%%%%%%%%%%%%%%%%%%%%%%%%%%
% \section{Sample}
%\iffalse
%<*samplemain>
%\fi
%
% The following presents a sample document
% with two chapters, two parts, a title page,
% a compile flag as well as three forwarding files to set the flag.
% It consists of eight |.tex| files:
% \begin{center}
% \begin{tabular}{ll}
% |cdocsamp.tex|&main file\\
% |cdocsch1.tex|&include file for chapter 1\\
% |cdocsch2.tex|&include file for chapter 2\\
% |cdocspt3.tex|&include file for part 3\\
% |cdocspt4.tex|&include file for part 4\\
% |cdocsdrf.tex|&forwarding file for main file in draft mode\\
% |cdocsfi1.tex|&forwarding file for final version of chapter 1\\
% |cdocsfi2.tex|&forwarding file for final version of chapter 2\\
% \end{tabular}
% \end{center}
% Each of the eight files can be compiled directly by the \LaTeX{} compiler.
%
% %%%%%%%%%%%%%%%%%%%%%%%%%%%%%%%%%%%%%%
% \paragraph{Main File.}
%
% The main file is called |cdocsamp.tex|.
%
% Load the \textsf{childdoc} definitions and
% declare the filename for the main document:
%    \begin{macrocode}
\input{childdoc.def}
\childdocmain{}
%    \end{macrocode}

% Optional override for |\version| flag:
%    \begin{macrocode}
%%\ifchilddoc\else\providecommand{\version}{draft}\fi
%    \end{macrocode}

% Define the default values for the |\version| flag
% (|final| for the main file and |draft| for childs):
%    \begin{macrocode}
\ifchilddoc
\providecommand{\version}{draft}
\else
\providecommand{\version}{final}
\fi
%    \end{macrocode}

% Load the standard document class:
%    \begin{macrocode}
\documentclass[12pt]{article}
%    \end{macrocode}

% Start the document body:
%    \begin{macrocode}
\begin{document}
%    \end{macrocode}

% Declare a title page.
% Print title, part of document being processed and version flag:
%    \begin{macrocode}
\addtocounter{page}{-1}
\begin{center}
{\LARGE\bfseries{}childdoc example\par}
\vspace{1cm}
\ifchilddoc
\ifchilddocmanual part\else chapter\fi:
`\childdocname' of `\childdocjob'\par
\else
main document: `\childdocjob'\par
\fi
version: \version\par
\end{center}
\newpage
%    \end{macrocode}

% Manually include selected file,
% otherwise process as usual:
%    \begin{macrocode}
\ifchilddocmanual
\section*{part `\childdocname'}
\input{\childdocname}
\else
%    \end{macrocode}

% Include the two chapters:
%    \begin{macrocode}
\include{cdocsch1}
\include{cdocsch2}
%    \end{macrocode}

% Include the two parts unless only chapters should be displayed:
%    \begin{macrocode}
\ifchilddoc\else
\section{part three}
\input{cdocspt3}
\section{part four}
\input{cdocspt4}
\fi
%    \end{macrocode}

% Process as usual until here:
%    \begin{macrocode}
\fi
%    \end{macrocode}

% End of document body:
%    \begin{macrocode}
\end{document}
%    \end{macrocode}
%\iffalse
%</samplemain>
%\fi
%
% %%%%%%%%%%%%%%%%%%%%%%%%%%%%%%%%%%%%%%
% \paragraph{Chapter Include Files.}
%
% The include files are called |cdocsch1.tex| and |cdocsch2.tex|.
%
%\iffalse
%<*samplechap1|samplechap2>
%\fi

% Optional override for |\version| flag:
%    \begin{macrocode}
%%\providecommand{\version}{final}
%    \end{macrocode}

% Include the main document:
%    \begin{macrocode}
\input{childdoc.def}
\childdocof{cdocsamp}
%    \end{macrocode}

%\iffalse
%</samplechap1|samplechap2>
%\fi
%
%\iffalse
%<*samplechap1>
%\fi
% Some text for chapter 1:
%    \begin{macrocode}
\section{one}
some text in chapter one
%    \end{macrocode}

%\iffalse
%</samplechap1>
%\fi
% Some text for chapter 2:
%\iffalse
%<*samplechap2>
%\fi
%    \begin{macrocode}
\section{two}
more text in chapter two
%    \end{macrocode}

%\iffalse
%</samplechap2>
%\fi
%
% %%%%%%%%%%%%%%%%%%%%%%%%%%%%%%%%%%%%%%
% \paragraph{Part Include Files.}
%
% The include files are called |cdocspt3.tex| and |cdocspt4.tex|.
%
%\iffalse
%<*samplepart3|samplepart4>
%\fi

% Optional override for |\version| flag:
%    \begin{macrocode}
%%\providecommand{\version}{final}
%    \end{macrocode}

% Include the main document:
%    \begin{macrocode}
\input{childdoc.def}
\childdocby{cdocsamp}
%    \end{macrocode}

%\iffalse
%</samplepart3|samplepart4>
%\fi
%
%\iffalse
%<*samplepart3>
%\fi
% Some text for part 3:
%    \begin{macrocode}
some text in part three
%    \end{macrocode}

%\iffalse
%</samplepart3>
%\fi
% Some text for part 4:
%\iffalse
%<*samplepart4>
%\fi
%    \begin{macrocode}
more text in part four
%    \end{macrocode}

%\iffalse
%</samplepart4>
%\fi
%
% %%%%%%%%%%%%%%%%%%%%%%%%%%%%%%%%%%%%%%
% \paragraph{Forwarding for a Complete Draft.}
%
% The following forwarding file |cdocsdrf.tex|
% compiles the main document in draft mode:
%\iffalse
%<*sampledraft>
%\fi
%    \begin{macrocode}
\def\version{draft}
\input{childdoc.def}
\childdocforward{cdocsamp}
%    \end{macrocode}

%\iffalse
%</sampledraft>
%\fi
%
% %%%%%%%%%%%%%%%%%%%%%%%%%%%%%%%%%%%%%%
% \paragraph{Forwarding for Final Version of the Chapters.}
%
% The following forwarding files |cdocsfn1.tex| and |cdocsfn2.tex|
% (with identical content)
% compile the final versions of the child documents
% |cdocsch1.tex| and |cdocsch2.tex|, respectively:
%\iffalse
%<*samplefinal>
%\fi
%    \begin{macrocode}
\def\version{final}
\input{childdoc.def}
\childdocforwardprefix[cdocsamp]{cdocsfn}{cdocsch}
%    \end{macrocode}

%\iffalse
%</samplefinal>
%\fi
%
% %%%%%%%%%%%%%%%%%%%%%%%%%%%%%%%%%%%%%%
% \paragraph{Command Line Processing.}
%
% The following three command lines generate the output files
% |cdocscld|, |cdocscl1| and |cdocscl2|
% which should be identical to
% |cdocsdrf|, |cdocsch1| and |cdocsfn2|, respectively:
% \begin{center}
% \begin{tabular}{l}
% |latex -jobname cdocscld \|\\
% |  "\def\version{draft}\input{childdoc.def}\childdocforward{cdocsamp}"|\\
% |latex -jobname cdocscl1 \|\\
% |  "\input{childdoc.def}\childdocforward[cdocsamp]{cdocsch1}"|\\
% |latex -jobname cdocscl2 \|\\
% |  "\def\version{final}\input{childdoc.def}\childdocforward{cdocsch2}"|
% \end{tabular}
% \end{center}
% Note that the trailing backslash on each first line
% merely continues the input to the second line
% (for convenient cut ant paste).
% Furthermore, the command |latex| can be replaced by any
% of its alternative versions such as |pdflatex|.
%
% %%%%%%%%%%%%%%%%%%%%%%%%%%%%%%%%%%%%%%%%%%%%%%%%%%%%%%%%%%%%%%%%%%%%%%%%%%%%%%
% %%%%%%%%%%%%%%%%%%%%%%%%%%%%%%%%%%%%%%%%%%%%%%%%%%%%%%%%%%%%%%%%%%%%%%%%%%%%%%
% \section{Implementation}
%\iffalse
%<*package>
%\fi
%
% This section describes the definitions file |childdoc.def|.

% The definitions cannot be loaded using |\usepackage| or |\RequirePackage|
% which has a mechanism to prevent loading a style file more than once.
% When loading the definitions by means of |\input|
% multiple instances have to be prevented manually:
%\iffalse
%This code needs to be before the `\ProvidesFile' directive
%which is defined at the beginning of this file.
%Therefore it is also placed there and commented out here.
%</package>
%<*discard>
%\fi
%    \begin{macrocode}
\ifdefined\childdocmain\endinput\fi
%    \end{macrocode}
%\iffalse
%</discard>
%<*package>
%\fi
%
% \macro{\ifchilddoc}
% \macro{\ifchilddocmanual}
% The conditional |\ifchilddoc| tells whether a
% child (true) or main (false) document is being compiled.
% The conditional |\ifchilddocmanual| tells whether
% the |\includeonly| mechanism is used (false) or
% the selection of child files must be performed manually (true).
% The definitions initialise to false:
%    \begin{macrocode}
\newif\ifchilddoc
\newif\ifchilddocmanual
%    \end{macrocode}

% \macro{\childdocname}
% \macro{\childdocjob}
% The macro |\childdocname| stores the name of the main document
% to be compiled. The macro |\childdocjob| stores the name of
% the document on which the \LaTeX{} compiler was originally invoked.
% The content of |\jobname| cannot be compared
% to filenames specified in the source due to different catcodes.
% The following code rescans |\jobname|, stores the result
% in |\childdocname| and saves a copy in |\childdocjob|:
%    \begin{macrocode}
\edef\childdocname{\scantokens\expandafter{\jobname\noexpand}}
\let\childdocjob\childdocname
%    \end{macrocode}

% \macro{\childdocdisable}
% The macro |\childdocdisable| prevents the main file
% from being processed more than once.
% At this stage, the main document command |\childdocmain|
% is assumed to be called once again where it should do nothing.
% Any subsequent call to it should prevent
% a secondary processing of the main document
% It overwrites the forwarding commands
% |\childdocof| and |\childdocforward|
% with empty macros to prevent further inclusions of the main document:
%    \begin{macrocode}
\newcommand{\childdocdisable}
{
  \renewcommand{\childdocmain}[1]{\renewcommand{\childdocmain}[1]{\endinput}}
  \renewcommand{\childdocof}[1]{}
  \renewcommand{\childdocby}[2][]{}
  \renewcommand{\childdocforward}[2][]{}
  \renewcommand{\childdocdisable}{}
}
%    \end{macrocode}

% \macro{\childdocmain}
% The macro |\childdocmain| is to be called at the top of the main file
% with nothing or the main filename (without extension) as argument.
% First, it breaks loops.
% If the argument is not empty and does not match |\childdocname|
% (which is set by the first inclusion of |childdoc.def|),
% |\ifchilddoc| is set to true, |\includeonly| is applied to the child file
% and |\jobname| is set to the main file
% (for proper handling of |.aux| files):
%    \begin{macrocode}
\newcommand{\childdocmain}[1]
{
  \childdocdisable\childdocmain{}
  \if?#1?\else
    \begingroup
      \def\childdoctmp{#1}
      \ifx\childdoctmp\childdocname
        \def\childdoctmp{}
      \else
        \def\childdoctmp
        {
          \childdoctrue
          \includeonly{\childdocname}
          \def\childdocjob{#1}
          \def\jobname{#1}
        }
      \fi
      \expandafter
    \endgroup
    \childdoctmp
  \fi
}
%    \end{macrocode}

% \macro{\childdocof}
% The command |\childdocof| redirects
% compilation to the main file |#1|.
%    \begin{macrocode}
\newcommand{\childdocof}[1]
{
  \childdocdisable
  \childdoctrue
  \includeonly{\childdocname}
  \def\jobname{#1}
  \def\childdocjob{#1}
  \input{#1}
}
%    \end{macrocode}

% \macro{\childdocby}
% The command |\childdocby| ....
%    \begin{macrocode}
\newcommand{\childdocby}[2][]
{
  \childdocdisable
  \childdoctrue
  \childdocmanualtrue
  \if?#1?\else
    \def\jobname{#2}
  \fi
  \def\childdocjob{#2}
  \input{#2}
  \endinput
}
%    \end{macrocode}

% \macro{\childdocforward}
% The command |\childdocforward| redirects
% compilation to the main file or
% (if the optional argument is given) a child file.
% Parameters are set as if the main file
% or a child file starting with |\childdocof| was compiled.
% Then compilation is handed over to the main file:
%    \begin{macrocode}
\newcommand{\childdocforward}[2][]
{
  \begingroup
    \if?#1?
      \def\childdoctmp
      {
        \def\childdocname{#2}
        \def\childdocjob{#2}
        \def\jobname{#2}
        \input{#2}
        \endinput
      }
    \else
      \def\childdoctmp
      {
        \childdocdisable
        \def\childdocname{#2}
        \childdoctrue
        \includeonly{#2}
        \def\childdocjob{#1}
        \def\jobname{#1}
        \input{#1}
        \endinput
      }
    \fi
    \expandafter
  \endgroup
  \childdoctmp
}
%    \end{macrocode}

% \macro{\childdocforwardprefix}
% The command |\childdocforwardprefix| redirects
% compilation to the main or a child file by means of a pattern.
% The prefix |#1| in the current filename is replaced by |#2|
% and the suffix of the current filename is kept
% (it is assumed that the filename does not contain the substring `|~~~|'
% which is used as a delimiter).
% Compilation is handed over to the new file by |\childdocforward|:
%    \begin{macrocode}
\newcommand{\childdocforwardprefix}[3][]
{
  \begingroup
    \def\childdocextract #2##1~~~{\def\childdoctmp{\childdocforward[#1]{#3##1}}}
    \expandafter\childdocextract\childdocname~~~
    \expandafter
  \endgroup
  \childdoctmp
}
%    \end{macrocode}

% \macro{\childdoc}
% The deprecated macro |\childdoc| is a legacy version of |\childdocmain|:
%    \begin{macrocode}
\newcommand{\childdoc}{\childdocmain}
%    \end{macrocode}

% \macro{\childdocredirect}
% The deprecated macro |\childdocredirect| is a legacy version
% of |\childdocforward| and |\childdocforwardprefix|:
%    \begin{macrocode}
\newcommand{\childdocredirect}[2][]
{
  \begingroup
    \if?#1?
      \def\childdoctmp{\childdocforward{#2}}
    \else
      \def\childdoctmp{\childdocforwardprefix{#1}{#2}}
    \fi
    \expandafter
  \endgroup
  \childdoctmp
}
%    \end{macrocode}

%\iffalse
%</package>
%\fi
%
\endinput

\childdocof{cdocsamp}
%    \end{macrocode}

%\iffalse
%</samplechap1|samplechap2>
%\fi
%
%\iffalse
%<*samplechap1>
%\fi
% Some text for chapter 1:
%    \begin{macrocode}
\section{one}
some text in chapter one
%    \end{macrocode}

%\iffalse
%</samplechap1>
%\fi
% Some text for chapter 2:
%\iffalse
%<*samplechap2>
%\fi
%    \begin{macrocode}
\section{two}
more text in chapter two
%    \end{macrocode}

%\iffalse
%</samplechap2>
%\fi
%
% %%%%%%%%%%%%%%%%%%%%%%%%%%%%%%%%%%%%%%
% \paragraph{Part Include Files.}
%
% The include files are called |cdocspt3.tex| and |cdocspt4.tex|.
%
%\iffalse
%<*samplepart3|samplepart4>
%\fi

% Optional override for |\version| flag:
%    \begin{macrocode}
%%\providecommand{\version}{final}
%    \end{macrocode}

% Include the main document:
%    \begin{macrocode}
% \iffalse
%
% childdoc.dtx Copyright (C) 2017-2018 Niklas Beisert
%
% This work may be distributed and/or modified under the
% conditions of the LaTeX Project Public License, either version 1.3
% of this license or (at your option) any later version.
% The latest version of this license is in
%   http://www.latex-project.org/lppl.txt
% and version 1.3 or later is part of all distributions of LaTeX
% version 2005/12/01 or later.
%
% This work has the LPPL maintenance status `maintained'.
%
% The Current Maintainer of this work is Niklas Beisert.
%
% This work consists of the files childdoc.dtx and childdoc.ins
% and the derived files childdoc.def and cdocsamp.tex with
% cdocsch1.tex, cdocsch2.tex, cdocsdrf.tex, cdocsfn1.tex, cdocsfn2.tex.
%
%<package>\ifdefined\childdocmain\endinput\fi
%<package>\ProvidesFile{childdoc.def}[2018/12/30 v2.0 child document driver]
%<samplemain>\ProvidesFile{cdocsamp.tex}[2018/12/30 v2.0 sample for childdoc]
%<*driver>
%\ProvidesFile{childdoc.drv}[2018/12/30 v2.0 childdoc reference manual file]
\PassOptionsToClass{10pt,a4paper}{article}
\documentclass{ltxdoc}

\usepackage[margin=35mm]{geometry}
\usepackage{hyperref}
\usepackage{hyperxmp}
\usepackage[usenames]{color}

\hypersetup{colorlinks=true}
\hypersetup{pdfstartview=FitH}
\hypersetup{pdfpagemode=UseNone}
\hypersetup{pdfsource={}}
\hypersetup{pdflang={en-UK}}
\hypersetup{pdfcopyright={Copyright 2017-2018 Niklas Beisert.
  This work may be distributed and/or modified under the
  conditions of the LaTeX Project Public License, either version 1.3
  of this license or (at your option) any later version.}}
\hypersetup{pdflicenseurl={http://www.latex-project.org/lppl.txt}}
\hypersetup{pdfcontactaddress={ETH Zurich, ITP, HIT K,
  Wolfgang-Pauli-Strasse 27}}
\hypersetup{pdfcontactpostcode={8093}}
\hypersetup{pdfcontactcity={Zurich}}
\hypersetup{pdfcontactcountry={Switzerland}}
\hypersetup{pdfcontactemail={nbeisert@itp.phys.ethz.ch}}
\hypersetup{pdfcontacturl={http://people.phys.ethz.ch/\xmptilde nbeisert/}}

\newcommand{\secref}[1]{\hyperref[#1]{section \ref*{#1}}}

\parskip1ex
\parindent0pt
\let\olditemize\itemize
\def\itemize{\olditemize\parskip0pt}

\begin{document}

\title{The \textsf{childdoc} Package}
\hypersetup{pdftitle={The childdoc Package}}
\author{Niklas Beisert\\[2ex]
  Institut f\"ur Theoretische Physik\\
  Eidgen\"ossische Technische Hochschule Z\"urich\\
  Wolfgang-Pauli-Strasse 27, 8093 Z\"urich, Switzerland\\[1ex]
  \href{mailto:nbeisert@itp.phys.ethz.ch}
  {\texttt{nbeisert@itp.phys.ethz.ch}}}
\hypersetup{pdfauthor={Niklas Beisert}}
\hypersetup{pdfsubject={Manual for the LaTeX2e Package childdoc}}
\date{30 December 2018, \textsf{v2.0}}
\maketitle

\begin{abstract}\noindent
\textsf{childdoc} is a \LaTeXe{} package
that enables the direct compilation
of document sections included by |\include|
to individual files.
\end{abstract}

\begingroup
\parskip0ex
\tableofcontents
\endgroup

%%%%%%%%%%%%%%%%%%%%%%%%%%%%%%%%%%%%%%%%%%%%%%%%%%%%%%%%%%%%%%%%%%%%%%%%%%%%%%%%
%%%%%%%%%%%%%%%%%%%%%%%%%%%%%%%%%%%%%%%%%%%%%%%%%%%%%%%%%%%%%%%%%%%%%%%%%%%%%%%%
\section{Introduction}

\LaTeX{} provides a mechanism to structure a large document (such as a book)
into a main file and several child files (containing the chapters)
using the |\include| command.
This mechanism is beneficial for documents
which span hundreds of pages in order to
make the source file(s) more manageable.
Moreover, compilation can be restricted to
selected child files by means of the |\includeonly| command.
The latter feature can be used to reduce the compilation time while editing
(this was significantly more useful in the earlier days of \LaTeX{})
or to generate a smaller document which is easier to navigate.
Another application of |\includeonly| is to generate
documents consisting of selected parts of the complete document.

However, there are a few drawbacks of the plain |\include| mechanism:
\begin{itemize}
\item
The child files cannot be compiled on their own,
they can only be compiled via the main file.
A naive editing environment
(such as a text editor with an option
to have the current file processed by \LaTeX)
may require one to switch to the main file before compiling;
attempting to compile the child file produces errors.
\item
The main file must be modified (each time)
to adjust the |\includeonly| command
to the present needs. This easily leaves the main file in a messy state.
\item
The generated document will always carry the filename
of the main document. This is inconvenient if
several child files are to be compiled and
to be kept for distribution.
\end{itemize}

The present package provides a simple interface
to make child files individually compilable by \LaTeX{}.
Compiling a child file then has the same effect as compiling
the main file with an |\includeonly| command
to select the appropriate child.
Moreover the generated document will carry the name of the child
rather than the main file.
This resolves all three above issues.

This feature is meant to make the editing of books,
thesis documents and lecture notes somewhat more convenient.
However, the package can also be used efficiently for
composing a series of documents (such as exercise sheets)
which are typically distributed individually.
It then assists the author in generating the individual documents
(potentially in different versions)
as well as a document containing the collected series.
Another application is in developing style files
or other kinds of included material
where compilation of the style file could redirect
to a sample or test file.

%%%%%%%%%%%%%%%%%%%%%%%%%%%%%%%%%%%%%%%%%%%%%%%%%%%%%%%%%%%%%%%%%%%%%%%%%%%%%%%%
%%%%%%%%%%%%%%%%%%%%%%%%%%%%%%%%%%%%%%%%%%%%%%%%%%%%%%%%%%%%%%%%%%%%%%%%%%%%%%%%
\section{Usage}

First of all, the package \textsf{childdoc} is \emph{not} a standard
\LaTeXe{} |.sty| style file! Therefore it needs to be invoked in
a non-standard way.

%%%%%%%%%%%%%%%%%%%%%%%%%%%%%%%%%%%%%%%%%%%%%%%%%%%%%%%%%%%%%%%%%%%%%%%%%%%%%%%%
\subsection{Included Files}
\label{sec:include}

%%%%%%%%%%%%%%%%%%%%%%%%%%%%%%%%%%%%%%%%
\DescribeMacro{\childdocmain}
To use the package, add the commands
\begin{center}
\begin{tabular}{l}
|\input{childdoc.def}|\\
|\childdocmain{}|\\
\end{tabular}
\end{center}
at the very top of the main \LaTeX{} file,
in particular \emph{before} the |\documentclass| statement!
The argument of |\childdocmain| should be left empty
(but it must be present).

%%%%%%%%%%%%%%%%%%%%%%%%%%%%%%%%%%%%%%%%
\DescribeMacro{\childdocof}
Furthermore, add the commands
\begin{center}
\begin{tabular}{l}
|\input{childdoc.def}|\\
|\childdocof{|\textit{main}|}|\\
\end{tabular}
\end{center}
at the top of every child file \textit{child}
which is included by |\include{|\textit{child}|}|
from within the main file
(or at least for those files to be compiled individually).
The argument \textit{main} must be the filename of the main file.

There are a couple of
considerations in setting up the main and child documents:

%%%%%%%%%%%%%%%%%%%%%%%%%%%%%%%%%%%%%%%%
\paragraph{Restrictions.}

Please note the following restrictions:
\begin{itemize}
\item
|\childdocmain| must be called with one argument \textit{main}
to ensure compatibility with earlier version of the package.
It must either be empty (|\childdocmain{}|)
or precisely match the filename of the main file in which it is specified.
See \secref{sec:detection} for further information.
\item
The filename \textit{main} must be specified without the |.tex| extension.
\item
The filename \textit{main} is case sensitive
(even in case-insensitive file systems)
due to internal string comparison.
\item
The argument \textit{main} should be fully expanded, it cannot be a macro.
\item
Subdirectories and special characters should be avoided in filenames.
\item
The command |\childdocmain{|\textit{main}|}| must be followed by a whitespace.
It should not be followed immediately by another command
or by a comment mark `|%|'.
This is because the \TeX{} parser reads the token immediately following
the argument of |\childdocmain| and puts it
at the beginning of every child section;
however, a white\-space is ignored.
\end{itemize}

%%%%%%%%%%%%%%%%%%%%%%%%%%%%%%%%%%%%%%%%
\paragraph{Content of Main File.}

It is advisable to place all content in the child files included by |\include|.
Any output contained in the main file will appear in all child documents
unless suppressed manually;
it cannot be suppressed automatically by the |\includeonly| directive
and thus should normally be avoided.
A method to include some content in the main file
by means of conditional processing is described in \secref{sec:conditional}.

%%%%%%%%%%%%%%%%%%%%%%%%%%%%%%%%%%%%%%%%
\paragraph{Page Numbering.}

When only a part of the document is compiled,
the appropriate numbering of pages
(as well as other status parameters)
is determined from the |.aux| files.
The latter contain information from previous passes.
However this information needs to propagate through
all intermediate child documents.
Therefore the page numbering in child documents may well
be inconsistent until the complete document is compiled at least once.

A useful (if unconventional) way to always ensure a consistent
page numbering is to restart the numbering in each child document
and denote the pages by `\textit{child}|.|\textit{page}'
where \textit{child} represents the chapter/section number of the child file.
This can be achieved by the command
|\numberwithin{page}{|\textit{child}|}|
of the \textsf{amsmath} package
where \textit{child} can be |chapter| or |section|
depending on the chosen structuring.
Alternatively, one can modify the macro |\thepage| appropriately
and reset the counter |page| at the start of each child file.

%%%%%%%%%%%%%%%%%%%%%%%%%%%%%%%%%%%%%%%%%%%%%%%%%%%%%%%%%%%%%%%%%%%%%%%%%%%%%%%%
\subsection{Conditional Processing}
\label{sec:conditional}

The package provides a mechanism to compile different versions
of a document. To customise the versions further some conditional processing
can come in handy to distinguish which version is being compiled.
The package provides two macros to describe the compilation context:

%%%%%%%%%%%%%%%%%%%%%%%%%%%%%%%%%%%%%%%%
\DescribeMacro{\ifchilddoc}
The conditional |\ifchilddoc| distinguishes between the compilation of
child documents and the main document:
%
\begin{center}
|\ifchilddoc |\textit{child-code}| |[|\||else |\textit{main-code}]| \||fi|
\end{center}

%%%%%%%%%%%%%%%%%%%%%%%%%%%%%%%%%%%%%%%%
\DescribeMacro{\childdocname}
\DescribeMacro{\childdocjob}
The macro |\childdocname| contains the filename (without extension)
of the main or child file being processed.
Note that |\childdocjob| will always contain the name of the main file.

%%%%%%%%%%%%%%%%%%%%%%%%%%%%%%%%%%%%%%%%
\paragraph{Title Page.}

Conditional processing can be used to include a title or banner page
in the main document when proper precautions are taken.
Importantly, the code in the main file should ensure that the page counter
(as well as other status parameters which are stored in the |.aux| files)
takes the same value after the conditional processing.
Otherwise the page numbers may take divergent values
depending on which part is compiled.

For example, a title page could be declared by:
%
\begin{center}
\begin{tabular}{l}
|\ifchilddoc\||else|\\
|\addtocounter{page}{-1}|\\
\textit{code for title page}\\
|\newpage|\\
|\||fi|
\end{tabular}
\end{center}
%
A banner page for the child documents can be generated by:
%
\begin{center}
\begin{tabular}{l}
|\ifchilddoc|\\
|\addtocounter{page}{-1}|\\
\textit{code for banner page}\\
|\newpage|\\
|\||fi|
\end{tabular}
\end{center}
%
Here one could write a message such as:
\begin{center}
|This is the part \childdocname{} of \childdocjob{}.|
\end{center}

%%%%%%%%%%%%%%%%%%%%%%%%%%%%%%%%%%%%%%%%%%%%%%%%%%%%%%%%%%%%%%%%%%%%%%%%%%%%%%%%
\subsection{Flags}
\label{sec:flags}

The package makes it easy to generate different versions
of the main or child documents.
To this end compilation flags can be defined
and assigned different default values.
They will be particularly useful in conjunction
with the forwarding mechanism described in \secref{sec:forward}.

For example, it may be useful to have a flag |\version|
which can be set to |draft| or |final|.
The document source will contain some conditional code
depending on the value of |\version|.
Suppose further, the flag should default to |final| for the main file
and to |draft| for child files
which is a natural assignment for editing the document.
This is achieved by placing the following code
in the preamble of the main document
(below the |\childdocmain| directive):
%
\begin{center}
\begin{tabular}{l}
|\ifchilddoc|\\
|\providecommand{\version}{draft}|\\
|\||else|\\
|\providecommand{\version}{final}|\\
|\||fi|
\end{tabular}
\end{center}
%
The definition by |\providecommand| makes sure
that previous definitions are not overwritten.
Further statements |\providecommand{\version}{...}|
can thus be added before the above code to override it.

For the main file, one might add a line
(between |\childdocmain| and the above block)
%
\begin{center}
|%\ifchilddoc\||else\providecommand{\version}{draft}\||fi|
\end{center}
%
which can be uncommented to produce a draft version.
Likewise one can add a line to the very top of a child file
(above the |\childdocof{|\textit{main}|}| directive)
%
\begin{center}
|%\providecommand{\version}{final}|
\end{center}
%
which can be uncommented to produce the final version of this child document.

%%%%%%%%%%%%%%%%%%%%%%%%%%%%%%%%%%%%%%%%%%%%%%%%%%%%%%%%%%%%%%%%%%%%%%%%%%%%%%%%
\subsection{Forwarding}
\label{sec:forward}

Different versions of the main or child documents
using compilation flags as described in \secref{sec:flags}
can be (permanently) stored in different files
for convenient compilation, viewing and distribution.
To this end, the package defines a command
to pass on compilation to a different file:

%%%%%%%%%%%%%%%%%%%%%%%%%%%%%%%%%%%%%%%%
\DescribeMacro{\childdocforward}
The command |\childdocforward| redirects processing to
another source file:
%
\begin{center}
\begin{tabular}{l}
|\input{childdoc.def}|\\
|\childdocforward[|\textit{main}|]{|\textit{dest}|}|\\
\end{tabular}
\end{center}
%
The argument \textit{dest} is the destination file
(without extension).
It should be the main file or one of the child files.
Note that further \textsf{childdoc} directives
such as |\childdocof| and |\childdocforward|
in the indicated file will be processed in this form.
The optional argument \textit{main}
passes on directly to the main file \textit{main}
while pretending to compile the child \textit{dest}.
This form behaves as if \textit{dest}
issues |\childdocof{|\textit{main}|}| right away,
and no further \textsf{childdoc} directives will be processed.

%%%%%%%%%%%%%%%%%%%%%%%%%%%%%%%%%%%%%%%%
\DescribeMacro{\...prefix}
In the alternative form |\childdocforwardprefix|,
%
\begin{center}
\begin{tabular}{l}
|\input{childdoc.def}|\\
|\childdocforwardprefix[|\textit{main}|]{|\textit{prefix}|}{|\textit{dest}|}|
\end{tabular}
\end{center}
%
the destination file is determined by a pattern
depending on the current file:
To make this work, the current file must be called
`{\textit{prefix}\hspace{0.2em}\textit{suffix}}'
with \textit{prefix} matching precisely the argument.
Processing is then passed on to the file
`{\textit{dest}\hspace{0.2em}\textit{suffix}}'.
Surely, the same effect is achieved by
directly specifying the
argument `{\textit{dest}\hspace{0.2em}\textit{suffix}}'
in the first form.
However, that requires to set up a different file
for each child. With the alternative form of the command
all these files can have exactly the same content
which simplifies setting them up and maintaining them.

For example, the following file |draft.tex|
with a compilation flag |\version| as described in \secref{sec:flags}
compiles the main document as a draft:
%
\begin{center}
\begin{tabular}{l}
|\def\version{draft}|\\
|\input{childdoc.def}|\\
|\childdocforward{|\textit{main}|}|
\end{tabular}
\end{center}
%
Likewise, the following files |final|\textit{nn}|.tex|
compile the final version of the child document
|child|\textit{nn}|.tex|:
%
\begin{center}
\begin{tabular}{l}
|\def\version{final}|\\
|\input{childdoc.def}|\\
|\childdocforwardprefix{final}{child}|
\end{tabular}
\end{center}
%

Note that when several versions of a main file and/or of each child file
are to be generated, it may be convenient to set up a |Makefile| or
shell script to automatise the process.

%%%%%%%%%%%%%%%%%%%%%%%%%%%%%%%%%%%%%%%%%%%%%%%%%%%%%%%%%%%%%%%%%%%%%%%%%%%%%%%%
\subsection{Command Line Processing}
\label{sec:commandline}

The effect of redirection files can also be achieved by invoking
the \LaTeX{} compiler with a more elaborate command line.
Most conveniently this should be done as part
of a shell script or a |Makefile|.

When using \textsf{childdoc} in the main file, the following
command lines effectively perform a redirection
(note that depending on the shell being used,
backslashes may have to be doubled: `|\|' $\to$ `|\\|'):
%
\begin{center}
|... -jobname "|\textit{target}|" |\\|"|[\textit{flags}]%
|\input{childdoc.def}\childdocforward[|\textit{main}|]{|\textit{dest}|}"|
\end{center}
%
Here \textit{target} is the name of the output file,
\textit{main} is the name of the main file
and \textit{dest} is the name of the main or child file to be processed
(all filenames without extensions).
The optional argument \textit{main} can be omitted
if \textit{main} matches \textit{dest}.
Optionally, compilation \textit{flags} can be defined via |\def| commands.
This command line makes the \TeX{} engine believe
it is compiling the file \textit{target}
whose content is specified as the latter parameter.
The provided code then forwards the processing to
\textit{main} or \textit{dest} as described in \secref{sec:forward}.

%%%%%%%%%%%%%%%%%%%%%%%%%%%%%%%%%%%%%%%%%%%%%%%%%%%%%%%%%%%%%%%%%%%%%%%%%%%%%%%%
\subsection{Include by Input}
\label{sec:input}

Including child documents by |\include| has some restrictions by design.
Most notably, the content of a child document always occupies
its own set of pages; pages cannot be shared between child documents.
Usually, this behaviour makes perfect sense
because each child document contain an essential part of the document.
However, in some situations it may be desirable to compose
a document from a collection of parts
without having mandatory page breaks between then.
For this case, the package
provides a mechanism to include parts
by |\input| which can also be processed individually.
However, by construction this mechanism
requires manual handling of the content to be output.

%%%%%%%%%%%%%%%%%%%%%%%%%%%%%%%%%%%%%%%%
\DescribeMacro{\ifchilddocmanual}
The main file should be prepared as usual, see \secref{sec:include}.
However, the document body must make a distinction
between processing of an individual part and of the main document, e.g.:
%
\begin{center}
\begin{tabular}{l}
|\ifchilddocmanual|\\
|\input{\childdocname}|\\
|\||else|\\
\textit{document body with }|\input{|\textit{part}|}|\\
|\||fi|
\end{tabular}
\end{center}
%
The conditional |\ifchilddocmanual| is true whenever
a part to be included by |\input| is being compiled,
and the name of the part is stored in |\childdocname|.

%%%%%%%%%%%%%%%%%%%%%%%%%%%%%%%%%%%%%%%%
\DescribeMacro{\childdocby}
Each part to be included by |\input| should start with:
%
\begin{center}
\begin{tabular}{l}
|\input{childdoc.def}|\\
|\childdocby{|\textit{main}|}|\\
\end{tabular}
\end{center}
%
The directive |\childdocby| is similar to |\childdocof|
described in \secref{sec:include},
but the subsequent selection of content must be done manually.
To that end, both |\ifchilddoc| and |\ifchilddocmanual|
will be true upon processing of a part,
and the name of the part is stored in |\childdocname|.
Note that |\jobname| will be set to the filename of the current part
so that each part receives an individual |.aux| file
that does not interfere with the |.aux| file(s) of the main document.
This behaviour can be altered by the alternative form
|\childdocby[*]{|\textit{main}|}| (with a non-empty optional argument)
which uses the |.aux| file of the main document
by setting |\jobname| to \textit{main}.

%%%%%%%%%%%%%%%%%%%%%%%%%%%%%%%%%%%%%%%%%%%%%%%%%%%%%%%%%%%%%%%%%%%%%%%%%%%%%%%%
\subsection{Driver Development}
\label{sec:driver}

The \textsf{childdoc} mechanism can also be use for the development
of definition files such as \LaTeX{} styles or classes.
This case differs from the above setup with multiple parts
included by |\include| in that no |\includeonly| should be invoked.
This can be achieved by starting the include file
(before |\ProvidesPackage|) with:
%
\begin{center}
\begin{tabular}{l}
|\input{childdoc.def}|\\
|\childdocforward{|\textit{main}|}|\\
\end{tabular}
\end{center}
%
or alternatively with:
%
\begin{center}
\begin{tabular}{l}
|\input{childdoc.def}|\\
|\childdocby{|\textit{main}|}|\\
\end{tabular}
\end{center}
%
Both forms have slightly different effects as described above.
The main file is prepared as usual, see \secref{sec:include}.

%%%%%%%%%%%%%%%%%%%%%%%%%%%%%%%%%%%%%%%%%%%%%%%%%%%%%%%%%%%%%%%%%%%%%%%%%%%%%%%%
\subsection{Legacy Detection}
\label{sec:detection}

The directive |\childdocmain| in the main file can detect
whether the complete document or merely a child is to be compiled
even without using the directive |\childdocof|.
This method is deprecated because it is less robust
and there is no compelling reason to use it;
it is merely provided for backward compatibility
and it may be removed in future versions.

If the detection mechanism is to be used,
it is mandatory to correctly specify
the filename of the main file as the argument of |\childdocmain|:
%
\begin{center}
\begin{tabular}{l}
|\input{childdoc.def}|\\
|\childdocmain{|\textit{main}|}|\\
\end{tabular}
\end{center}
%
If |\jobname| does not match the argument \textit{main} of |\childdocmain|,
it is assumed that |\jobname| points to the child file to be compiled.
When using |\childdocmain| with the main file specified as argument,
it suffices to start a child file
with just |\input{|\textit{main}|}|
without loading of the package and using |\childdocof|.
If instead all processing is done
with the appropriate \textsf{childdoc} directives,
the argument of \textit{main} of |\childdocmain| can be empty.

An alternative version of the command line processing described
in \secref{sec:commandline} using the detection mechanism reads:
%
\begin{center}
|... -jobname "|\textit{target}|" "|[\textit{flags}]%
[|\def\jobname{|\textit{dest}|}|]|\input{|\textit{main}|}"|
\end{center}

%%%%%%%%%%%%%%%%%%%%%%%%%%%%%%%%%%%%%%%%%%%%%%%%%%%%%%%%%%%%%%%%%%%%%%%%%%%%%%%%
\subsection{Manual Code}
\label{sec:manual}

In case one cannot be certain whether the definitions file |childdoc.def|
is installed on the target \TeX{} distribution
and one prefers not to ship it,
it is conceivable to paste a few relevant commands into the sources.

To that end, drop all statements |\input{childdoc.def}|
and perform the replacements as outlined below.
Instead of |\childdocmain{|\textit{main}|}| add the following code
to the top of the main file:
%
\begin{center}
\begin{tabular}{l}
|\||ifdefined\childdocname\endinput\||fi\newif\ifchilddoc|\\
|\edef\childdocname{\scantokens\expandafter{\jobname\noexpand}}|\\
|\def\childdocmain{|\textit{main}|}\||ifx\childdocmain\childdocname\||else|\\
|\childdoctrue\includeonly{\childdocname}\let\jobname\childdocmain\||fi|\\
\end{tabular}
\end{center}
%
Instead of |\childdocof{|\textit{main}|}| just include the main file
at the top of each child file:
%
\begin{center}
|\input{|\textit{main}|}|
\end{center}
%
A simple redirection |\childdocforward{|\textit{dest}|}| is achieved by:
%
\begin{center}
|\def\jobname{|\textit{dest}|}\input{\jobname}|
\end{center}
%
The redirection with prefix
|\childdocforwardprefix[|\textit{prefix}|]{|\textit{dest}|}|
is accomplished by:
%
\begin{center}
\begin{tabular}{l}
|{\edef\jobname{\scantokens\expandafter{\jobname\noexpand}}|\\
|\def\redirectjob |\textit{prefix}|#1~~~{\gdef\jobname{|\textit{dest}|#1}}|\\
|\expandafter\redirectjob\jobname~~~}\input{\jobname}|
\end{tabular}
\end{center}

In an alternative approach,
child documents can be compiled by a specific command line
without additional code or specific definitions:
%
\begin{center}
|... -jobname "|\textit{target}|" "|[\textit{flags}]%
|\includeonly{|\textit{dest}|}\input{|\textit{main}|}"|
\end{center}
%

%%%%%%%%%%%%%%%%%%%%%%%%%%%%%%%%%%%%%%%%%%%%%%%%%%%%%%%%%%%%%%%%%%%%%%%%%%%%%%%%
%%%%%%%%%%%%%%%%%%%%%%%%%%%%%%%%%%%%%%%%%%%%%%%%%%%%%%%%%%%%%%%%%%%%%%%%%%%%%%%%
\section{Information}

%%%%%%%%%%%%%%%%%%%%%%%%%%%%%%%%%%%%%%%%%%%%%%%%%%%%%%%%%%%%%%%%%%%%%%%%%%%%%%%%
\subsection{Copyright}

Copyright \copyright{} 2017--2018 Niklas Beisert

This work may be distributed and/or modified under the
conditions of the \LaTeX{} Project Public License, either version 1.3
of this license or (at your option) any later version.
The latest version of this license is in
  \url{http://www.latex-project.org/lppl.txt}
and version 1.3 or later is part of all distributions of \LaTeX{}
version 2005/12/01 or later.

This work has the LPPL maintenance status `maintained'.

The Current Maintainer of this work is Niklas Beisert.

This work consists of the files |README.txt|, |childdoc.ins| and |childdoc.dtx|
as well as the derived files |childdoc.def|, |cdocsamp.tex|
with |cdocsch1.tex|, |cdocsch2.tex|, |cdocspt3.tex|, |cdocspt4.tex|,
|cdocsdrf.tex|, |cdocsfn1.tex|, |cdocsfn2.tex|
as well as |childdoc.pdf|.

%%%%%%%%%%%%%%%%%%%%%%%%%%%%%%%%%%%%%%%%%%%%%%%%%%%%%%%%%%%%%%%%%%%%%%%%%%%%%%%%
\subsection{Files and Installation}

The package consists of the files:
%
\begin{center}
\begin{tabular}{ll}
    |README.txt|   & readme file \\
    |childdoc.ins| & installation file \\
    |childdoc.dtx| & source file \\
    |childdoc.def| & definition file \\
    |cdocsamp.tex| & sample main file \\
    |cdocsch1.tex| & sample include file \\
    |cdocsch2.tex| & sample include file \\
    |cdocspt3.tex| & sample part file \\
    |cdocspt4.tex| & sample part file \\
    |cdocsdrf.tex| & sample redirection file \\
    |cdocsfn1.tex| & sample redirection file \\
    |cdocsfn2.tex| & sample redirection file \\
    |childdoc.pdf| & manual
\end{tabular}
\end{center}
%
The distribution consists of the files
|README.txt|, |childdoc.ins| and |childdoc.dtx|.
%
\begin{itemize}
\item
Run (pdf)\LaTeX{} on |childdoc.dtx|
to compile the manual |childdoc.pdf| (this file).
\item
Run \LaTeX{} on |childdoc.ins| to create the definitions file |childdoc.def|
and the sample |cdocsamp.tex| with include files
|cdocsch1.tex|, |cdocsch2.tex|, |cdocspt3.tex|, |cdocspt4.tex|,
|cdocsdrf.tex|, |cdocsfn1.tex|, |cdocsfn2.tex|.
Then copy the file |childdoc.def| to an appropriate directory of your \LaTeX{}
distribution, e.g.\ \textit{texmf-root}|/tex/latex/childdoc|.
\end{itemize}

%%%%%%%%%%%%%%%%%%%%%%%%%%%%%%%%%%%%%%%%%%%%%%%%%%%%%%%%%%%%%%%%%%%%%%%%%%%%%%%%
\subsection{Related CTAN Packages}

There are several other packages which offer a similar functionality:
%
\begin{itemize}
\item
The packages
\href{http://ctan.org/pkg/docmute}{\textsf{docmute}},
\href{http://ctan.org/pkg/includex}{\textsf{includex}} and
\href{http://ctan.org/pkg/standalone}{\textsf{standalone}}
provide commands to include only the document body of
a child file thus allowing both files to be compiled individually.
\item
The packages \href{http://ctan.org/pkg/subdocs}{\textsf{subdocs}}
and \href{http://ctan.org/pkg/subfiles}{\textsf{subfiles}}
provide structures in which the main and child documents can be
encapsulated and allowing them to be compiled individually.
The inclusion mechanism is different from the conventional |\include|.
\item
The package \href{http://ctan.org/pkg/combine}{\textsf{combine}}
is an elaborate solution to combine several documents into one.
\end{itemize}
%
See also the CTAN topic \href{http://ctan.org/topic/subdocs}{\textsf{subdocs}}
for further related packages.
The present package differs from the above solutions in that
a document structure constructed with the conventional |\include| mechanism
just needs two extra commands at the top of every file
such that all constituent files can be compiled individually.

%%%%%%%%%%%%%%%%%%%%%%%%%%%%%%%%%%%%%%%%%%%%%%%%%%%%%%%%%%%%%%%%%%%%%%%%%%%%%%%%
%\subsection{Feature Suggestions}
%
%The following is a list of features which may be useful for future
%versions of this package:
%%
%\begin{itemize}
%\item
%\ldots
%\end{itemize}

%%%%%%%%%%%%%%%%%%%%%%%%%%%%%%%%%%%%%%%%%%%%%%%%%%%%%%%%%%%%%%%%%%%%%%%%%%%%%%%%
\subsection{Revision History}

%%%%%%%%%%%%%%%%%%%%%%%%%%%%%%%%%%%%%%%%
\paragraph{v2.0:} 2018/12/30

\begin{itemize}
\item
immediate forward processing
\item
added |\childdocby| mechanism
\item
manual restructured
\end{itemize}

%%%%%%%%%%%%%%%%%%%%%%%%%%%%%%%%%%%%%%%%
\paragraph{v1.6:} 2018/01/17

\begin{itemize}
\item
application for development of include files
\item
corrections to manual
\end{itemize}

%%%%%%%%%%%%%%%%%%%%%%%%%%%%%%%%%%%%%%%%
\paragraph{v1.5:} 2017/05/21

\begin{itemize}
\item
more complete structuring introduced
\item
|\childdocof| introduced
\item
|\childdoc| renamed to |\childdocmain|
\item
|\childredirect| renamed to |\childdocforward| and |\childdocforwardprefix|
and functionality expanded
\end{itemize}

%%%%%%%%%%%%%%%%%%%%%%%%%%%%%%%%%%%%%%%%
\paragraph{v1.0:} 2017/04/27

\begin{itemize}
\item
manual and install package
\item
first version published on CTAN
\end{itemize}

%%%%%%%%%%%%%%%%%%%%%%%%%%%%%%%%%%%%%%%%
\paragraph{v0.6:} 2017/04/26

\begin{itemize}
\item
redirection mechanism added
\end{itemize}

%%%%%%%%%%%%%%%%%%%%%%%%%%%%%%%%%%%%%%%%
\paragraph{v0.5:} 2017/04/26

\begin{itemize}
\item
functionality in definition file
\end{itemize}


%%%%%%%%%%%%%%%%%%%%%%%%%%%%%%%%%%%%%%%%%%%%%%%%%%%%%%%%%%%%%%%%%%%%%%%%%%%%%%%%
%%%%%%%%%%%%%%%%%%%%%%%%%%%%%%%%%%%%%%%%%%%%%%%%%%%%%%%%%%%%%%%%%%%%%%%%%%%%%%%%
%%%%%%%%%%%%%%%%%%%%%%%%%%%%%%%%%%%%%%%%%%%%%%%%%%%%%%%%%%%%%%%%%%%%%%%%%%%%%%%%
\appendix

\settowidth\MacroIndent{\rmfamily\scriptsize 000\ }

 \DocInput{childdoc.dtx}

\end{document}
%</driver>
% \fi
%
% %%%%%%%%%%%%%%%%%%%%%%%%%%%%%%%%%%%%%%%%%%%%%%%%%%%%%%%%%%%%%%%%%%%%%%%%%%%%%%
% %%%%%%%%%%%%%%%%%%%%%%%%%%%%%%%%%%%%%%%%%%%%%%%%%%%%%%%%%%%%%%%%%%%%%%%%%%%%%%
% \section{Sample}
%\iffalse
%<*samplemain>
%\fi
%
% The following presents a sample document
% with two chapters, two parts, a title page,
% a compile flag as well as three forwarding files to set the flag.
% It consists of eight |.tex| files:
% \begin{center}
% \begin{tabular}{ll}
% |cdocsamp.tex|&main file\\
% |cdocsch1.tex|&include file for chapter 1\\
% |cdocsch2.tex|&include file for chapter 2\\
% |cdocspt3.tex|&include file for part 3\\
% |cdocspt4.tex|&include file for part 4\\
% |cdocsdrf.tex|&forwarding file for main file in draft mode\\
% |cdocsfi1.tex|&forwarding file for final version of chapter 1\\
% |cdocsfi2.tex|&forwarding file for final version of chapter 2\\
% \end{tabular}
% \end{center}
% Each of the eight files can be compiled directly by the \LaTeX{} compiler.
%
% %%%%%%%%%%%%%%%%%%%%%%%%%%%%%%%%%%%%%%
% \paragraph{Main File.}
%
% The main file is called |cdocsamp.tex|.
%
% Load the \textsf{childdoc} definitions and
% declare the filename for the main document:
%    \begin{macrocode}
\input{childdoc.def}
\childdocmain{}
%    \end{macrocode}

% Optional override for |\version| flag:
%    \begin{macrocode}
%%\ifchilddoc\else\providecommand{\version}{draft}\fi
%    \end{macrocode}

% Define the default values for the |\version| flag
% (|final| for the main file and |draft| for childs):
%    \begin{macrocode}
\ifchilddoc
\providecommand{\version}{draft}
\else
\providecommand{\version}{final}
\fi
%    \end{macrocode}

% Load the standard document class:
%    \begin{macrocode}
\documentclass[12pt]{article}
%    \end{macrocode}

% Start the document body:
%    \begin{macrocode}
\begin{document}
%    \end{macrocode}

% Declare a title page.
% Print title, part of document being processed and version flag:
%    \begin{macrocode}
\addtocounter{page}{-1}
\begin{center}
{\LARGE\bfseries{}childdoc example\par}
\vspace{1cm}
\ifchilddoc
\ifchilddocmanual part\else chapter\fi:
`\childdocname' of `\childdocjob'\par
\else
main document: `\childdocjob'\par
\fi
version: \version\par
\end{center}
\newpage
%    \end{macrocode}

% Manually include selected file,
% otherwise process as usual:
%    \begin{macrocode}
\ifchilddocmanual
\section*{part `\childdocname'}
\input{\childdocname}
\else
%    \end{macrocode}

% Include the two chapters:
%    \begin{macrocode}
\include{cdocsch1}
\include{cdocsch2}
%    \end{macrocode}

% Include the two parts unless only chapters should be displayed:
%    \begin{macrocode}
\ifchilddoc\else
\section{part three}
\input{cdocspt3}
\section{part four}
\input{cdocspt4}
\fi
%    \end{macrocode}

% Process as usual until here:
%    \begin{macrocode}
\fi
%    \end{macrocode}

% End of document body:
%    \begin{macrocode}
\end{document}
%    \end{macrocode}
%\iffalse
%</samplemain>
%\fi
%
% %%%%%%%%%%%%%%%%%%%%%%%%%%%%%%%%%%%%%%
% \paragraph{Chapter Include Files.}
%
% The include files are called |cdocsch1.tex| and |cdocsch2.tex|.
%
%\iffalse
%<*samplechap1|samplechap2>
%\fi

% Optional override for |\version| flag:
%    \begin{macrocode}
%%\providecommand{\version}{final}
%    \end{macrocode}

% Include the main document:
%    \begin{macrocode}
\input{childdoc.def}
\childdocof{cdocsamp}
%    \end{macrocode}

%\iffalse
%</samplechap1|samplechap2>
%\fi
%
%\iffalse
%<*samplechap1>
%\fi
% Some text for chapter 1:
%    \begin{macrocode}
\section{one}
some text in chapter one
%    \end{macrocode}

%\iffalse
%</samplechap1>
%\fi
% Some text for chapter 2:
%\iffalse
%<*samplechap2>
%\fi
%    \begin{macrocode}
\section{two}
more text in chapter two
%    \end{macrocode}

%\iffalse
%</samplechap2>
%\fi
%
% %%%%%%%%%%%%%%%%%%%%%%%%%%%%%%%%%%%%%%
% \paragraph{Part Include Files.}
%
% The include files are called |cdocspt3.tex| and |cdocspt4.tex|.
%
%\iffalse
%<*samplepart3|samplepart4>
%\fi

% Optional override for |\version| flag:
%    \begin{macrocode}
%%\providecommand{\version}{final}
%    \end{macrocode}

% Include the main document:
%    \begin{macrocode}
\input{childdoc.def}
\childdocby{cdocsamp}
%    \end{macrocode}

%\iffalse
%</samplepart3|samplepart4>
%\fi
%
%\iffalse
%<*samplepart3>
%\fi
% Some text for part 3:
%    \begin{macrocode}
some text in part three
%    \end{macrocode}

%\iffalse
%</samplepart3>
%\fi
% Some text for part 4:
%\iffalse
%<*samplepart4>
%\fi
%    \begin{macrocode}
more text in part four
%    \end{macrocode}

%\iffalse
%</samplepart4>
%\fi
%
% %%%%%%%%%%%%%%%%%%%%%%%%%%%%%%%%%%%%%%
% \paragraph{Forwarding for a Complete Draft.}
%
% The following forwarding file |cdocsdrf.tex|
% compiles the main document in draft mode:
%\iffalse
%<*sampledraft>
%\fi
%    \begin{macrocode}
\def\version{draft}
\input{childdoc.def}
\childdocforward{cdocsamp}
%    \end{macrocode}

%\iffalse
%</sampledraft>
%\fi
%
% %%%%%%%%%%%%%%%%%%%%%%%%%%%%%%%%%%%%%%
% \paragraph{Forwarding for Final Version of the Chapters.}
%
% The following forwarding files |cdocsfn1.tex| and |cdocsfn2.tex|
% (with identical content)
% compile the final versions of the child documents
% |cdocsch1.tex| and |cdocsch2.tex|, respectively:
%\iffalse
%<*samplefinal>
%\fi
%    \begin{macrocode}
\def\version{final}
\input{childdoc.def}
\childdocforwardprefix[cdocsamp]{cdocsfn}{cdocsch}
%    \end{macrocode}

%\iffalse
%</samplefinal>
%\fi
%
% %%%%%%%%%%%%%%%%%%%%%%%%%%%%%%%%%%%%%%
% \paragraph{Command Line Processing.}
%
% The following three command lines generate the output files
% |cdocscld|, |cdocscl1| and |cdocscl2|
% which should be identical to
% |cdocsdrf|, |cdocsch1| and |cdocsfn2|, respectively:
% \begin{center}
% \begin{tabular}{l}
% |latex -jobname cdocscld \|\\
% |  "\def\version{draft}\input{childdoc.def}\childdocforward{cdocsamp}"|\\
% |latex -jobname cdocscl1 \|\\
% |  "\input{childdoc.def}\childdocforward[cdocsamp]{cdocsch1}"|\\
% |latex -jobname cdocscl2 \|\\
% |  "\def\version{final}\input{childdoc.def}\childdocforward{cdocsch2}"|
% \end{tabular}
% \end{center}
% Note that the trailing backslash on each first line
% merely continues the input to the second line
% (for convenient cut ant paste).
% Furthermore, the command |latex| can be replaced by any
% of its alternative versions such as |pdflatex|.
%
% %%%%%%%%%%%%%%%%%%%%%%%%%%%%%%%%%%%%%%%%%%%%%%%%%%%%%%%%%%%%%%%%%%%%%%%%%%%%%%
% %%%%%%%%%%%%%%%%%%%%%%%%%%%%%%%%%%%%%%%%%%%%%%%%%%%%%%%%%%%%%%%%%%%%%%%%%%%%%%
% \section{Implementation}
%\iffalse
%<*package>
%\fi
%
% This section describes the definitions file |childdoc.def|.

% The definitions cannot be loaded using |\usepackage| or |\RequirePackage|
% which has a mechanism to prevent loading a style file more than once.
% When loading the definitions by means of |\input|
% multiple instances have to be prevented manually:
%\iffalse
%This code needs to be before the `\ProvidesFile' directive
%which is defined at the beginning of this file.
%Therefore it is also placed there and commented out here.
%</package>
%<*discard>
%\fi
%    \begin{macrocode}
\ifdefined\childdocmain\endinput\fi
%    \end{macrocode}
%\iffalse
%</discard>
%<*package>
%\fi
%
% \macro{\ifchilddoc}
% \macro{\ifchilddocmanual}
% The conditional |\ifchilddoc| tells whether a
% child (true) or main (false) document is being compiled.
% The conditional |\ifchilddocmanual| tells whether
% the |\includeonly| mechanism is used (false) or
% the selection of child files must be performed manually (true).
% The definitions initialise to false:
%    \begin{macrocode}
\newif\ifchilddoc
\newif\ifchilddocmanual
%    \end{macrocode}

% \macro{\childdocname}
% \macro{\childdocjob}
% The macro |\childdocname| stores the name of the main document
% to be compiled. The macro |\childdocjob| stores the name of
% the document on which the \LaTeX{} compiler was originally invoked.
% The content of |\jobname| cannot be compared
% to filenames specified in the source due to different catcodes.
% The following code rescans |\jobname|, stores the result
% in |\childdocname| and saves a copy in |\childdocjob|:
%    \begin{macrocode}
\edef\childdocname{\scantokens\expandafter{\jobname\noexpand}}
\let\childdocjob\childdocname
%    \end{macrocode}

% \macro{\childdocdisable}
% The macro |\childdocdisable| prevents the main file
% from being processed more than once.
% At this stage, the main document command |\childdocmain|
% is assumed to be called once again where it should do nothing.
% Any subsequent call to it should prevent
% a secondary processing of the main document
% It overwrites the forwarding commands
% |\childdocof| and |\childdocforward|
% with empty macros to prevent further inclusions of the main document:
%    \begin{macrocode}
\newcommand{\childdocdisable}
{
  \renewcommand{\childdocmain}[1]{\renewcommand{\childdocmain}[1]{\endinput}}
  \renewcommand{\childdocof}[1]{}
  \renewcommand{\childdocby}[2][]{}
  \renewcommand{\childdocforward}[2][]{}
  \renewcommand{\childdocdisable}{}
}
%    \end{macrocode}

% \macro{\childdocmain}
% The macro |\childdocmain| is to be called at the top of the main file
% with nothing or the main filename (without extension) as argument.
% First, it breaks loops.
% If the argument is not empty and does not match |\childdocname|
% (which is set by the first inclusion of |childdoc.def|),
% |\ifchilddoc| is set to true, |\includeonly| is applied to the child file
% and |\jobname| is set to the main file
% (for proper handling of |.aux| files):
%    \begin{macrocode}
\newcommand{\childdocmain}[1]
{
  \childdocdisable\childdocmain{}
  \if?#1?\else
    \begingroup
      \def\childdoctmp{#1}
      \ifx\childdoctmp\childdocname
        \def\childdoctmp{}
      \else
        \def\childdoctmp
        {
          \childdoctrue
          \includeonly{\childdocname}
          \def\childdocjob{#1}
          \def\jobname{#1}
        }
      \fi
      \expandafter
    \endgroup
    \childdoctmp
  \fi
}
%    \end{macrocode}

% \macro{\childdocof}
% The command |\childdocof| redirects
% compilation to the main file |#1|.
%    \begin{macrocode}
\newcommand{\childdocof}[1]
{
  \childdocdisable
  \childdoctrue
  \includeonly{\childdocname}
  \def\jobname{#1}
  \def\childdocjob{#1}
  \input{#1}
}
%    \end{macrocode}

% \macro{\childdocby}
% The command |\childdocby| ....
%    \begin{macrocode}
\newcommand{\childdocby}[2][]
{
  \childdocdisable
  \childdoctrue
  \childdocmanualtrue
  \if?#1?\else
    \def\jobname{#2}
  \fi
  \def\childdocjob{#2}
  \input{#2}
  \endinput
}
%    \end{macrocode}

% \macro{\childdocforward}
% The command |\childdocforward| redirects
% compilation to the main file or
% (if the optional argument is given) a child file.
% Parameters are set as if the main file
% or a child file starting with |\childdocof| was compiled.
% Then compilation is handed over to the main file:
%    \begin{macrocode}
\newcommand{\childdocforward}[2][]
{
  \begingroup
    \if?#1?
      \def\childdoctmp
      {
        \def\childdocname{#2}
        \def\childdocjob{#2}
        \def\jobname{#2}
        \input{#2}
        \endinput
      }
    \else
      \def\childdoctmp
      {
        \childdocdisable
        \def\childdocname{#2}
        \childdoctrue
        \includeonly{#2}
        \def\childdocjob{#1}
        \def\jobname{#1}
        \input{#1}
        \endinput
      }
    \fi
    \expandafter
  \endgroup
  \childdoctmp
}
%    \end{macrocode}

% \macro{\childdocforwardprefix}
% The command |\childdocforwardprefix| redirects
% compilation to the main or a child file by means of a pattern.
% The prefix |#1| in the current filename is replaced by |#2|
% and the suffix of the current filename is kept
% (it is assumed that the filename does not contain the substring `|~~~|'
% which is used as a delimiter).
% Compilation is handed over to the new file by |\childdocforward|:
%    \begin{macrocode}
\newcommand{\childdocforwardprefix}[3][]
{
  \begingroup
    \def\childdocextract #2##1~~~{\def\childdoctmp{\childdocforward[#1]{#3##1}}}
    \expandafter\childdocextract\childdocname~~~
    \expandafter
  \endgroup
  \childdoctmp
}
%    \end{macrocode}

% \macro{\childdoc}
% The deprecated macro |\childdoc| is a legacy version of |\childdocmain|:
%    \begin{macrocode}
\newcommand{\childdoc}{\childdocmain}
%    \end{macrocode}

% \macro{\childdocredirect}
% The deprecated macro |\childdocredirect| is a legacy version
% of |\childdocforward| and |\childdocforwardprefix|:
%    \begin{macrocode}
\newcommand{\childdocredirect}[2][]
{
  \begingroup
    \if?#1?
      \def\childdoctmp{\childdocforward{#2}}
    \else
      \def\childdoctmp{\childdocforwardprefix{#1}{#2}}
    \fi
    \expandafter
  \endgroup
  \childdoctmp
}
%    \end{macrocode}

%\iffalse
%</package>
%\fi
%
\endinput

\childdocby{cdocsamp}
%    \end{macrocode}

%\iffalse
%</samplepart3|samplepart4>
%\fi
%
%\iffalse
%<*samplepart3>
%\fi
% Some text for part 3:
%    \begin{macrocode}
some text in part three
%    \end{macrocode}

%\iffalse
%</samplepart3>
%\fi
% Some text for part 4:
%\iffalse
%<*samplepart4>
%\fi
%    \begin{macrocode}
more text in part four
%    \end{macrocode}

%\iffalse
%</samplepart4>
%\fi
%
% %%%%%%%%%%%%%%%%%%%%%%%%%%%%%%%%%%%%%%
% \paragraph{Forwarding for a Complete Draft.}
%
% The following forwarding file |cdocsdrf.tex|
% compiles the main document in draft mode:
%\iffalse
%<*sampledraft>
%\fi
%    \begin{macrocode}
\def\version{draft}
% \iffalse
%
% childdoc.dtx Copyright (C) 2017-2018 Niklas Beisert
%
% This work may be distributed and/or modified under the
% conditions of the LaTeX Project Public License, either version 1.3
% of this license or (at your option) any later version.
% The latest version of this license is in
%   http://www.latex-project.org/lppl.txt
% and version 1.3 or later is part of all distributions of LaTeX
% version 2005/12/01 or later.
%
% This work has the LPPL maintenance status `maintained'.
%
% The Current Maintainer of this work is Niklas Beisert.
%
% This work consists of the files childdoc.dtx and childdoc.ins
% and the derived files childdoc.def and cdocsamp.tex with
% cdocsch1.tex, cdocsch2.tex, cdocsdrf.tex, cdocsfn1.tex, cdocsfn2.tex.
%
%<package>\ifdefined\childdocmain\endinput\fi
%<package>\ProvidesFile{childdoc.def}[2018/12/30 v2.0 child document driver]
%<samplemain>\ProvidesFile{cdocsamp.tex}[2018/12/30 v2.0 sample for childdoc]
%<*driver>
%\ProvidesFile{childdoc.drv}[2018/12/30 v2.0 childdoc reference manual file]
\PassOptionsToClass{10pt,a4paper}{article}
\documentclass{ltxdoc}

\usepackage[margin=35mm]{geometry}
\usepackage{hyperref}
\usepackage{hyperxmp}
\usepackage[usenames]{color}

\hypersetup{colorlinks=true}
\hypersetup{pdfstartview=FitH}
\hypersetup{pdfpagemode=UseNone}
\hypersetup{pdfsource={}}
\hypersetup{pdflang={en-UK}}
\hypersetup{pdfcopyright={Copyright 2017-2018 Niklas Beisert.
  This work may be distributed and/or modified under the
  conditions of the LaTeX Project Public License, either version 1.3
  of this license or (at your option) any later version.}}
\hypersetup{pdflicenseurl={http://www.latex-project.org/lppl.txt}}
\hypersetup{pdfcontactaddress={ETH Zurich, ITP, HIT K,
  Wolfgang-Pauli-Strasse 27}}
\hypersetup{pdfcontactpostcode={8093}}
\hypersetup{pdfcontactcity={Zurich}}
\hypersetup{pdfcontactcountry={Switzerland}}
\hypersetup{pdfcontactemail={nbeisert@itp.phys.ethz.ch}}
\hypersetup{pdfcontacturl={http://people.phys.ethz.ch/\xmptilde nbeisert/}}

\newcommand{\secref}[1]{\hyperref[#1]{section \ref*{#1}}}

\parskip1ex
\parindent0pt
\let\olditemize\itemize
\def\itemize{\olditemize\parskip0pt}

\begin{document}

\title{The \textsf{childdoc} Package}
\hypersetup{pdftitle={The childdoc Package}}
\author{Niklas Beisert\\[2ex]
  Institut f\"ur Theoretische Physik\\
  Eidgen\"ossische Technische Hochschule Z\"urich\\
  Wolfgang-Pauli-Strasse 27, 8093 Z\"urich, Switzerland\\[1ex]
  \href{mailto:nbeisert@itp.phys.ethz.ch}
  {\texttt{nbeisert@itp.phys.ethz.ch}}}
\hypersetup{pdfauthor={Niklas Beisert}}
\hypersetup{pdfsubject={Manual for the LaTeX2e Package childdoc}}
\date{30 December 2018, \textsf{v2.0}}
\maketitle

\begin{abstract}\noindent
\textsf{childdoc} is a \LaTeXe{} package
that enables the direct compilation
of document sections included by |\include|
to individual files.
\end{abstract}

\begingroup
\parskip0ex
\tableofcontents
\endgroup

%%%%%%%%%%%%%%%%%%%%%%%%%%%%%%%%%%%%%%%%%%%%%%%%%%%%%%%%%%%%%%%%%%%%%%%%%%%%%%%%
%%%%%%%%%%%%%%%%%%%%%%%%%%%%%%%%%%%%%%%%%%%%%%%%%%%%%%%%%%%%%%%%%%%%%%%%%%%%%%%%
\section{Introduction}

\LaTeX{} provides a mechanism to structure a large document (such as a book)
into a main file and several child files (containing the chapters)
using the |\include| command.
This mechanism is beneficial for documents
which span hundreds of pages in order to
make the source file(s) more manageable.
Moreover, compilation can be restricted to
selected child files by means of the |\includeonly| command.
The latter feature can be used to reduce the compilation time while editing
(this was significantly more useful in the earlier days of \LaTeX{})
or to generate a smaller document which is easier to navigate.
Another application of |\includeonly| is to generate
documents consisting of selected parts of the complete document.

However, there are a few drawbacks of the plain |\include| mechanism:
\begin{itemize}
\item
The child files cannot be compiled on their own,
they can only be compiled via the main file.
A naive editing environment
(such as a text editor with an option
to have the current file processed by \LaTeX)
may require one to switch to the main file before compiling;
attempting to compile the child file produces errors.
\item
The main file must be modified (each time)
to adjust the |\includeonly| command
to the present needs. This easily leaves the main file in a messy state.
\item
The generated document will always carry the filename
of the main document. This is inconvenient if
several child files are to be compiled and
to be kept for distribution.
\end{itemize}

The present package provides a simple interface
to make child files individually compilable by \LaTeX{}.
Compiling a child file then has the same effect as compiling
the main file with an |\includeonly| command
to select the appropriate child.
Moreover the generated document will carry the name of the child
rather than the main file.
This resolves all three above issues.

This feature is meant to make the editing of books,
thesis documents and lecture notes somewhat more convenient.
However, the package can also be used efficiently for
composing a series of documents (such as exercise sheets)
which are typically distributed individually.
It then assists the author in generating the individual documents
(potentially in different versions)
as well as a document containing the collected series.
Another application is in developing style files
or other kinds of included material
where compilation of the style file could redirect
to a sample or test file.

%%%%%%%%%%%%%%%%%%%%%%%%%%%%%%%%%%%%%%%%%%%%%%%%%%%%%%%%%%%%%%%%%%%%%%%%%%%%%%%%
%%%%%%%%%%%%%%%%%%%%%%%%%%%%%%%%%%%%%%%%%%%%%%%%%%%%%%%%%%%%%%%%%%%%%%%%%%%%%%%%
\section{Usage}

First of all, the package \textsf{childdoc} is \emph{not} a standard
\LaTeXe{} |.sty| style file! Therefore it needs to be invoked in
a non-standard way.

%%%%%%%%%%%%%%%%%%%%%%%%%%%%%%%%%%%%%%%%%%%%%%%%%%%%%%%%%%%%%%%%%%%%%%%%%%%%%%%%
\subsection{Included Files}
\label{sec:include}

%%%%%%%%%%%%%%%%%%%%%%%%%%%%%%%%%%%%%%%%
\DescribeMacro{\childdocmain}
To use the package, add the commands
\begin{center}
\begin{tabular}{l}
|\input{childdoc.def}|\\
|\childdocmain{}|\\
\end{tabular}
\end{center}
at the very top of the main \LaTeX{} file,
in particular \emph{before} the |\documentclass| statement!
The argument of |\childdocmain| should be left empty
(but it must be present).

%%%%%%%%%%%%%%%%%%%%%%%%%%%%%%%%%%%%%%%%
\DescribeMacro{\childdocof}
Furthermore, add the commands
\begin{center}
\begin{tabular}{l}
|\input{childdoc.def}|\\
|\childdocof{|\textit{main}|}|\\
\end{tabular}
\end{center}
at the top of every child file \textit{child}
which is included by |\include{|\textit{child}|}|
from within the main file
(or at least for those files to be compiled individually).
The argument \textit{main} must be the filename of the main file.

There are a couple of
considerations in setting up the main and child documents:

%%%%%%%%%%%%%%%%%%%%%%%%%%%%%%%%%%%%%%%%
\paragraph{Restrictions.}

Please note the following restrictions:
\begin{itemize}
\item
|\childdocmain| must be called with one argument \textit{main}
to ensure compatibility with earlier version of the package.
It must either be empty (|\childdocmain{}|)
or precisely match the filename of the main file in which it is specified.
See \secref{sec:detection} for further information.
\item
The filename \textit{main} must be specified without the |.tex| extension.
\item
The filename \textit{main} is case sensitive
(even in case-insensitive file systems)
due to internal string comparison.
\item
The argument \textit{main} should be fully expanded, it cannot be a macro.
\item
Subdirectories and special characters should be avoided in filenames.
\item
The command |\childdocmain{|\textit{main}|}| must be followed by a whitespace.
It should not be followed immediately by another command
or by a comment mark `|%|'.
This is because the \TeX{} parser reads the token immediately following
the argument of |\childdocmain| and puts it
at the beginning of every child section;
however, a white\-space is ignored.
\end{itemize}

%%%%%%%%%%%%%%%%%%%%%%%%%%%%%%%%%%%%%%%%
\paragraph{Content of Main File.}

It is advisable to place all content in the child files included by |\include|.
Any output contained in the main file will appear in all child documents
unless suppressed manually;
it cannot be suppressed automatically by the |\includeonly| directive
and thus should normally be avoided.
A method to include some content in the main file
by means of conditional processing is described in \secref{sec:conditional}.

%%%%%%%%%%%%%%%%%%%%%%%%%%%%%%%%%%%%%%%%
\paragraph{Page Numbering.}

When only a part of the document is compiled,
the appropriate numbering of pages
(as well as other status parameters)
is determined from the |.aux| files.
The latter contain information from previous passes.
However this information needs to propagate through
all intermediate child documents.
Therefore the page numbering in child documents may well
be inconsistent until the complete document is compiled at least once.

A useful (if unconventional) way to always ensure a consistent
page numbering is to restart the numbering in each child document
and denote the pages by `\textit{child}|.|\textit{page}'
where \textit{child} represents the chapter/section number of the child file.
This can be achieved by the command
|\numberwithin{page}{|\textit{child}|}|
of the \textsf{amsmath} package
where \textit{child} can be |chapter| or |section|
depending on the chosen structuring.
Alternatively, one can modify the macro |\thepage| appropriately
and reset the counter |page| at the start of each child file.

%%%%%%%%%%%%%%%%%%%%%%%%%%%%%%%%%%%%%%%%%%%%%%%%%%%%%%%%%%%%%%%%%%%%%%%%%%%%%%%%
\subsection{Conditional Processing}
\label{sec:conditional}

The package provides a mechanism to compile different versions
of a document. To customise the versions further some conditional processing
can come in handy to distinguish which version is being compiled.
The package provides two macros to describe the compilation context:

%%%%%%%%%%%%%%%%%%%%%%%%%%%%%%%%%%%%%%%%
\DescribeMacro{\ifchilddoc}
The conditional |\ifchilddoc| distinguishes between the compilation of
child documents and the main document:
%
\begin{center}
|\ifchilddoc |\textit{child-code}| |[|\||else |\textit{main-code}]| \||fi|
\end{center}

%%%%%%%%%%%%%%%%%%%%%%%%%%%%%%%%%%%%%%%%
\DescribeMacro{\childdocname}
\DescribeMacro{\childdocjob}
The macro |\childdocname| contains the filename (without extension)
of the main or child file being processed.
Note that |\childdocjob| will always contain the name of the main file.

%%%%%%%%%%%%%%%%%%%%%%%%%%%%%%%%%%%%%%%%
\paragraph{Title Page.}

Conditional processing can be used to include a title or banner page
in the main document when proper precautions are taken.
Importantly, the code in the main file should ensure that the page counter
(as well as other status parameters which are stored in the |.aux| files)
takes the same value after the conditional processing.
Otherwise the page numbers may take divergent values
depending on which part is compiled.

For example, a title page could be declared by:
%
\begin{center}
\begin{tabular}{l}
|\ifchilddoc\||else|\\
|\addtocounter{page}{-1}|\\
\textit{code for title page}\\
|\newpage|\\
|\||fi|
\end{tabular}
\end{center}
%
A banner page for the child documents can be generated by:
%
\begin{center}
\begin{tabular}{l}
|\ifchilddoc|\\
|\addtocounter{page}{-1}|\\
\textit{code for banner page}\\
|\newpage|\\
|\||fi|
\end{tabular}
\end{center}
%
Here one could write a message such as:
\begin{center}
|This is the part \childdocname{} of \childdocjob{}.|
\end{center}

%%%%%%%%%%%%%%%%%%%%%%%%%%%%%%%%%%%%%%%%%%%%%%%%%%%%%%%%%%%%%%%%%%%%%%%%%%%%%%%%
\subsection{Flags}
\label{sec:flags}

The package makes it easy to generate different versions
of the main or child documents.
To this end compilation flags can be defined
and assigned different default values.
They will be particularly useful in conjunction
with the forwarding mechanism described in \secref{sec:forward}.

For example, it may be useful to have a flag |\version|
which can be set to |draft| or |final|.
The document source will contain some conditional code
depending on the value of |\version|.
Suppose further, the flag should default to |final| for the main file
and to |draft| for child files
which is a natural assignment for editing the document.
This is achieved by placing the following code
in the preamble of the main document
(below the |\childdocmain| directive):
%
\begin{center}
\begin{tabular}{l}
|\ifchilddoc|\\
|\providecommand{\version}{draft}|\\
|\||else|\\
|\providecommand{\version}{final}|\\
|\||fi|
\end{tabular}
\end{center}
%
The definition by |\providecommand| makes sure
that previous definitions are not overwritten.
Further statements |\providecommand{\version}{...}|
can thus be added before the above code to override it.

For the main file, one might add a line
(between |\childdocmain| and the above block)
%
\begin{center}
|%\ifchilddoc\||else\providecommand{\version}{draft}\||fi|
\end{center}
%
which can be uncommented to produce a draft version.
Likewise one can add a line to the very top of a child file
(above the |\childdocof{|\textit{main}|}| directive)
%
\begin{center}
|%\providecommand{\version}{final}|
\end{center}
%
which can be uncommented to produce the final version of this child document.

%%%%%%%%%%%%%%%%%%%%%%%%%%%%%%%%%%%%%%%%%%%%%%%%%%%%%%%%%%%%%%%%%%%%%%%%%%%%%%%%
\subsection{Forwarding}
\label{sec:forward}

Different versions of the main or child documents
using compilation flags as described in \secref{sec:flags}
can be (permanently) stored in different files
for convenient compilation, viewing and distribution.
To this end, the package defines a command
to pass on compilation to a different file:

%%%%%%%%%%%%%%%%%%%%%%%%%%%%%%%%%%%%%%%%
\DescribeMacro{\childdocforward}
The command |\childdocforward| redirects processing to
another source file:
%
\begin{center}
\begin{tabular}{l}
|\input{childdoc.def}|\\
|\childdocforward[|\textit{main}|]{|\textit{dest}|}|\\
\end{tabular}
\end{center}
%
The argument \textit{dest} is the destination file
(without extension).
It should be the main file or one of the child files.
Note that further \textsf{childdoc} directives
such as |\childdocof| and |\childdocforward|
in the indicated file will be processed in this form.
The optional argument \textit{main}
passes on directly to the main file \textit{main}
while pretending to compile the child \textit{dest}.
This form behaves as if \textit{dest}
issues |\childdocof{|\textit{main}|}| right away,
and no further \textsf{childdoc} directives will be processed.

%%%%%%%%%%%%%%%%%%%%%%%%%%%%%%%%%%%%%%%%
\DescribeMacro{\...prefix}
In the alternative form |\childdocforwardprefix|,
%
\begin{center}
\begin{tabular}{l}
|\input{childdoc.def}|\\
|\childdocforwardprefix[|\textit{main}|]{|\textit{prefix}|}{|\textit{dest}|}|
\end{tabular}
\end{center}
%
the destination file is determined by a pattern
depending on the current file:
To make this work, the current file must be called
`{\textit{prefix}\hspace{0.2em}\textit{suffix}}'
with \textit{prefix} matching precisely the argument.
Processing is then passed on to the file
`{\textit{dest}\hspace{0.2em}\textit{suffix}}'.
Surely, the same effect is achieved by
directly specifying the
argument `{\textit{dest}\hspace{0.2em}\textit{suffix}}'
in the first form.
However, that requires to set up a different file
for each child. With the alternative form of the command
all these files can have exactly the same content
which simplifies setting them up and maintaining them.

For example, the following file |draft.tex|
with a compilation flag |\version| as described in \secref{sec:flags}
compiles the main document as a draft:
%
\begin{center}
\begin{tabular}{l}
|\def\version{draft}|\\
|\input{childdoc.def}|\\
|\childdocforward{|\textit{main}|}|
\end{tabular}
\end{center}
%
Likewise, the following files |final|\textit{nn}|.tex|
compile the final version of the child document
|child|\textit{nn}|.tex|:
%
\begin{center}
\begin{tabular}{l}
|\def\version{final}|\\
|\input{childdoc.def}|\\
|\childdocforwardprefix{final}{child}|
\end{tabular}
\end{center}
%

Note that when several versions of a main file and/or of each child file
are to be generated, it may be convenient to set up a |Makefile| or
shell script to automatise the process.

%%%%%%%%%%%%%%%%%%%%%%%%%%%%%%%%%%%%%%%%%%%%%%%%%%%%%%%%%%%%%%%%%%%%%%%%%%%%%%%%
\subsection{Command Line Processing}
\label{sec:commandline}

The effect of redirection files can also be achieved by invoking
the \LaTeX{} compiler with a more elaborate command line.
Most conveniently this should be done as part
of a shell script or a |Makefile|.

When using \textsf{childdoc} in the main file, the following
command lines effectively perform a redirection
(note that depending on the shell being used,
backslashes may have to be doubled: `|\|' $\to$ `|\\|'):
%
\begin{center}
|... -jobname "|\textit{target}|" |\\|"|[\textit{flags}]%
|\input{childdoc.def}\childdocforward[|\textit{main}|]{|\textit{dest}|}"|
\end{center}
%
Here \textit{target} is the name of the output file,
\textit{main} is the name of the main file
and \textit{dest} is the name of the main or child file to be processed
(all filenames without extensions).
The optional argument \textit{main} can be omitted
if \textit{main} matches \textit{dest}.
Optionally, compilation \textit{flags} can be defined via |\def| commands.
This command line makes the \TeX{} engine believe
it is compiling the file \textit{target}
whose content is specified as the latter parameter.
The provided code then forwards the processing to
\textit{main} or \textit{dest} as described in \secref{sec:forward}.

%%%%%%%%%%%%%%%%%%%%%%%%%%%%%%%%%%%%%%%%%%%%%%%%%%%%%%%%%%%%%%%%%%%%%%%%%%%%%%%%
\subsection{Include by Input}
\label{sec:input}

Including child documents by |\include| has some restrictions by design.
Most notably, the content of a child document always occupies
its own set of pages; pages cannot be shared between child documents.
Usually, this behaviour makes perfect sense
because each child document contain an essential part of the document.
However, in some situations it may be desirable to compose
a document from a collection of parts
without having mandatory page breaks between then.
For this case, the package
provides a mechanism to include parts
by |\input| which can also be processed individually.
However, by construction this mechanism
requires manual handling of the content to be output.

%%%%%%%%%%%%%%%%%%%%%%%%%%%%%%%%%%%%%%%%
\DescribeMacro{\ifchilddocmanual}
The main file should be prepared as usual, see \secref{sec:include}.
However, the document body must make a distinction
between processing of an individual part and of the main document, e.g.:
%
\begin{center}
\begin{tabular}{l}
|\ifchilddocmanual|\\
|\input{\childdocname}|\\
|\||else|\\
\textit{document body with }|\input{|\textit{part}|}|\\
|\||fi|
\end{tabular}
\end{center}
%
The conditional |\ifchilddocmanual| is true whenever
a part to be included by |\input| is being compiled,
and the name of the part is stored in |\childdocname|.

%%%%%%%%%%%%%%%%%%%%%%%%%%%%%%%%%%%%%%%%
\DescribeMacro{\childdocby}
Each part to be included by |\input| should start with:
%
\begin{center}
\begin{tabular}{l}
|\input{childdoc.def}|\\
|\childdocby{|\textit{main}|}|\\
\end{tabular}
\end{center}
%
The directive |\childdocby| is similar to |\childdocof|
described in \secref{sec:include},
but the subsequent selection of content must be done manually.
To that end, both |\ifchilddoc| and |\ifchilddocmanual|
will be true upon processing of a part,
and the name of the part is stored in |\childdocname|.
Note that |\jobname| will be set to the filename of the current part
so that each part receives an individual |.aux| file
that does not interfere with the |.aux| file(s) of the main document.
This behaviour can be altered by the alternative form
|\childdocby[*]{|\textit{main}|}| (with a non-empty optional argument)
which uses the |.aux| file of the main document
by setting |\jobname| to \textit{main}.

%%%%%%%%%%%%%%%%%%%%%%%%%%%%%%%%%%%%%%%%%%%%%%%%%%%%%%%%%%%%%%%%%%%%%%%%%%%%%%%%
\subsection{Driver Development}
\label{sec:driver}

The \textsf{childdoc} mechanism can also be use for the development
of definition files such as \LaTeX{} styles or classes.
This case differs from the above setup with multiple parts
included by |\include| in that no |\includeonly| should be invoked.
This can be achieved by starting the include file
(before |\ProvidesPackage|) with:
%
\begin{center}
\begin{tabular}{l}
|\input{childdoc.def}|\\
|\childdocforward{|\textit{main}|}|\\
\end{tabular}
\end{center}
%
or alternatively with:
%
\begin{center}
\begin{tabular}{l}
|\input{childdoc.def}|\\
|\childdocby{|\textit{main}|}|\\
\end{tabular}
\end{center}
%
Both forms have slightly different effects as described above.
The main file is prepared as usual, see \secref{sec:include}.

%%%%%%%%%%%%%%%%%%%%%%%%%%%%%%%%%%%%%%%%%%%%%%%%%%%%%%%%%%%%%%%%%%%%%%%%%%%%%%%%
\subsection{Legacy Detection}
\label{sec:detection}

The directive |\childdocmain| in the main file can detect
whether the complete document or merely a child is to be compiled
even without using the directive |\childdocof|.
This method is deprecated because it is less robust
and there is no compelling reason to use it;
it is merely provided for backward compatibility
and it may be removed in future versions.

If the detection mechanism is to be used,
it is mandatory to correctly specify
the filename of the main file as the argument of |\childdocmain|:
%
\begin{center}
\begin{tabular}{l}
|\input{childdoc.def}|\\
|\childdocmain{|\textit{main}|}|\\
\end{tabular}
\end{center}
%
If |\jobname| does not match the argument \textit{main} of |\childdocmain|,
it is assumed that |\jobname| points to the child file to be compiled.
When using |\childdocmain| with the main file specified as argument,
it suffices to start a child file
with just |\input{|\textit{main}|}|
without loading of the package and using |\childdocof|.
If instead all processing is done
with the appropriate \textsf{childdoc} directives,
the argument of \textit{main} of |\childdocmain| can be empty.

An alternative version of the command line processing described
in \secref{sec:commandline} using the detection mechanism reads:
%
\begin{center}
|... -jobname "|\textit{target}|" "|[\textit{flags}]%
[|\def\jobname{|\textit{dest}|}|]|\input{|\textit{main}|}"|
\end{center}

%%%%%%%%%%%%%%%%%%%%%%%%%%%%%%%%%%%%%%%%%%%%%%%%%%%%%%%%%%%%%%%%%%%%%%%%%%%%%%%%
\subsection{Manual Code}
\label{sec:manual}

In case one cannot be certain whether the definitions file |childdoc.def|
is installed on the target \TeX{} distribution
and one prefers not to ship it,
it is conceivable to paste a few relevant commands into the sources.

To that end, drop all statements |\input{childdoc.def}|
and perform the replacements as outlined below.
Instead of |\childdocmain{|\textit{main}|}| add the following code
to the top of the main file:
%
\begin{center}
\begin{tabular}{l}
|\||ifdefined\childdocname\endinput\||fi\newif\ifchilddoc|\\
|\edef\childdocname{\scantokens\expandafter{\jobname\noexpand}}|\\
|\def\childdocmain{|\textit{main}|}\||ifx\childdocmain\childdocname\||else|\\
|\childdoctrue\includeonly{\childdocname}\let\jobname\childdocmain\||fi|\\
\end{tabular}
\end{center}
%
Instead of |\childdocof{|\textit{main}|}| just include the main file
at the top of each child file:
%
\begin{center}
|\input{|\textit{main}|}|
\end{center}
%
A simple redirection |\childdocforward{|\textit{dest}|}| is achieved by:
%
\begin{center}
|\def\jobname{|\textit{dest}|}\input{\jobname}|
\end{center}
%
The redirection with prefix
|\childdocforwardprefix[|\textit{prefix}|]{|\textit{dest}|}|
is accomplished by:
%
\begin{center}
\begin{tabular}{l}
|{\edef\jobname{\scantokens\expandafter{\jobname\noexpand}}|\\
|\def\redirectjob |\textit{prefix}|#1~~~{\gdef\jobname{|\textit{dest}|#1}}|\\
|\expandafter\redirectjob\jobname~~~}\input{\jobname}|
\end{tabular}
\end{center}

In an alternative approach,
child documents can be compiled by a specific command line
without additional code or specific definitions:
%
\begin{center}
|... -jobname "|\textit{target}|" "|[\textit{flags}]%
|\includeonly{|\textit{dest}|}\input{|\textit{main}|}"|
\end{center}
%

%%%%%%%%%%%%%%%%%%%%%%%%%%%%%%%%%%%%%%%%%%%%%%%%%%%%%%%%%%%%%%%%%%%%%%%%%%%%%%%%
%%%%%%%%%%%%%%%%%%%%%%%%%%%%%%%%%%%%%%%%%%%%%%%%%%%%%%%%%%%%%%%%%%%%%%%%%%%%%%%%
\section{Information}

%%%%%%%%%%%%%%%%%%%%%%%%%%%%%%%%%%%%%%%%%%%%%%%%%%%%%%%%%%%%%%%%%%%%%%%%%%%%%%%%
\subsection{Copyright}

Copyright \copyright{} 2017--2018 Niklas Beisert

This work may be distributed and/or modified under the
conditions of the \LaTeX{} Project Public License, either version 1.3
of this license or (at your option) any later version.
The latest version of this license is in
  \url{http://www.latex-project.org/lppl.txt}
and version 1.3 or later is part of all distributions of \LaTeX{}
version 2005/12/01 or later.

This work has the LPPL maintenance status `maintained'.

The Current Maintainer of this work is Niklas Beisert.

This work consists of the files |README.txt|, |childdoc.ins| and |childdoc.dtx|
as well as the derived files |childdoc.def|, |cdocsamp.tex|
with |cdocsch1.tex|, |cdocsch2.tex|, |cdocspt3.tex|, |cdocspt4.tex|,
|cdocsdrf.tex|, |cdocsfn1.tex|, |cdocsfn2.tex|
as well as |childdoc.pdf|.

%%%%%%%%%%%%%%%%%%%%%%%%%%%%%%%%%%%%%%%%%%%%%%%%%%%%%%%%%%%%%%%%%%%%%%%%%%%%%%%%
\subsection{Files and Installation}

The package consists of the files:
%
\begin{center}
\begin{tabular}{ll}
    |README.txt|   & readme file \\
    |childdoc.ins| & installation file \\
    |childdoc.dtx| & source file \\
    |childdoc.def| & definition file \\
    |cdocsamp.tex| & sample main file \\
    |cdocsch1.tex| & sample include file \\
    |cdocsch2.tex| & sample include file \\
    |cdocspt3.tex| & sample part file \\
    |cdocspt4.tex| & sample part file \\
    |cdocsdrf.tex| & sample redirection file \\
    |cdocsfn1.tex| & sample redirection file \\
    |cdocsfn2.tex| & sample redirection file \\
    |childdoc.pdf| & manual
\end{tabular}
\end{center}
%
The distribution consists of the files
|README.txt|, |childdoc.ins| and |childdoc.dtx|.
%
\begin{itemize}
\item
Run (pdf)\LaTeX{} on |childdoc.dtx|
to compile the manual |childdoc.pdf| (this file).
\item
Run \LaTeX{} on |childdoc.ins| to create the definitions file |childdoc.def|
and the sample |cdocsamp.tex| with include files
|cdocsch1.tex|, |cdocsch2.tex|, |cdocspt3.tex|, |cdocspt4.tex|,
|cdocsdrf.tex|, |cdocsfn1.tex|, |cdocsfn2.tex|.
Then copy the file |childdoc.def| to an appropriate directory of your \LaTeX{}
distribution, e.g.\ \textit{texmf-root}|/tex/latex/childdoc|.
\end{itemize}

%%%%%%%%%%%%%%%%%%%%%%%%%%%%%%%%%%%%%%%%%%%%%%%%%%%%%%%%%%%%%%%%%%%%%%%%%%%%%%%%
\subsection{Related CTAN Packages}

There are several other packages which offer a similar functionality:
%
\begin{itemize}
\item
The packages
\href{http://ctan.org/pkg/docmute}{\textsf{docmute}},
\href{http://ctan.org/pkg/includex}{\textsf{includex}} and
\href{http://ctan.org/pkg/standalone}{\textsf{standalone}}
provide commands to include only the document body of
a child file thus allowing both files to be compiled individually.
\item
The packages \href{http://ctan.org/pkg/subdocs}{\textsf{subdocs}}
and \href{http://ctan.org/pkg/subfiles}{\textsf{subfiles}}
provide structures in which the main and child documents can be
encapsulated and allowing them to be compiled individually.
The inclusion mechanism is different from the conventional |\include|.
\item
The package \href{http://ctan.org/pkg/combine}{\textsf{combine}}
is an elaborate solution to combine several documents into one.
\end{itemize}
%
See also the CTAN topic \href{http://ctan.org/topic/subdocs}{\textsf{subdocs}}
for further related packages.
The present package differs from the above solutions in that
a document structure constructed with the conventional |\include| mechanism
just needs two extra commands at the top of every file
such that all constituent files can be compiled individually.

%%%%%%%%%%%%%%%%%%%%%%%%%%%%%%%%%%%%%%%%%%%%%%%%%%%%%%%%%%%%%%%%%%%%%%%%%%%%%%%%
%\subsection{Feature Suggestions}
%
%The following is a list of features which may be useful for future
%versions of this package:
%%
%\begin{itemize}
%\item
%\ldots
%\end{itemize}

%%%%%%%%%%%%%%%%%%%%%%%%%%%%%%%%%%%%%%%%%%%%%%%%%%%%%%%%%%%%%%%%%%%%%%%%%%%%%%%%
\subsection{Revision History}

%%%%%%%%%%%%%%%%%%%%%%%%%%%%%%%%%%%%%%%%
\paragraph{v2.0:} 2018/12/30

\begin{itemize}
\item
immediate forward processing
\item
added |\childdocby| mechanism
\item
manual restructured
\end{itemize}

%%%%%%%%%%%%%%%%%%%%%%%%%%%%%%%%%%%%%%%%
\paragraph{v1.6:} 2018/01/17

\begin{itemize}
\item
application for development of include files
\item
corrections to manual
\end{itemize}

%%%%%%%%%%%%%%%%%%%%%%%%%%%%%%%%%%%%%%%%
\paragraph{v1.5:} 2017/05/21

\begin{itemize}
\item
more complete structuring introduced
\item
|\childdocof| introduced
\item
|\childdoc| renamed to |\childdocmain|
\item
|\childredirect| renamed to |\childdocforward| and |\childdocforwardprefix|
and functionality expanded
\end{itemize}

%%%%%%%%%%%%%%%%%%%%%%%%%%%%%%%%%%%%%%%%
\paragraph{v1.0:} 2017/04/27

\begin{itemize}
\item
manual and install package
\item
first version published on CTAN
\end{itemize}

%%%%%%%%%%%%%%%%%%%%%%%%%%%%%%%%%%%%%%%%
\paragraph{v0.6:} 2017/04/26

\begin{itemize}
\item
redirection mechanism added
\end{itemize}

%%%%%%%%%%%%%%%%%%%%%%%%%%%%%%%%%%%%%%%%
\paragraph{v0.5:} 2017/04/26

\begin{itemize}
\item
functionality in definition file
\end{itemize}


%%%%%%%%%%%%%%%%%%%%%%%%%%%%%%%%%%%%%%%%%%%%%%%%%%%%%%%%%%%%%%%%%%%%%%%%%%%%%%%%
%%%%%%%%%%%%%%%%%%%%%%%%%%%%%%%%%%%%%%%%%%%%%%%%%%%%%%%%%%%%%%%%%%%%%%%%%%%%%%%%
%%%%%%%%%%%%%%%%%%%%%%%%%%%%%%%%%%%%%%%%%%%%%%%%%%%%%%%%%%%%%%%%%%%%%%%%%%%%%%%%
\appendix

\settowidth\MacroIndent{\rmfamily\scriptsize 000\ }

 \DocInput{childdoc.dtx}

\end{document}
%</driver>
% \fi
%
% %%%%%%%%%%%%%%%%%%%%%%%%%%%%%%%%%%%%%%%%%%%%%%%%%%%%%%%%%%%%%%%%%%%%%%%%%%%%%%
% %%%%%%%%%%%%%%%%%%%%%%%%%%%%%%%%%%%%%%%%%%%%%%%%%%%%%%%%%%%%%%%%%%%%%%%%%%%%%%
% \section{Sample}
%\iffalse
%<*samplemain>
%\fi
%
% The following presents a sample document
% with two chapters, two parts, a title page,
% a compile flag as well as three forwarding files to set the flag.
% It consists of eight |.tex| files:
% \begin{center}
% \begin{tabular}{ll}
% |cdocsamp.tex|&main file\\
% |cdocsch1.tex|&include file for chapter 1\\
% |cdocsch2.tex|&include file for chapter 2\\
% |cdocspt3.tex|&include file for part 3\\
% |cdocspt4.tex|&include file for part 4\\
% |cdocsdrf.tex|&forwarding file for main file in draft mode\\
% |cdocsfi1.tex|&forwarding file for final version of chapter 1\\
% |cdocsfi2.tex|&forwarding file for final version of chapter 2\\
% \end{tabular}
% \end{center}
% Each of the eight files can be compiled directly by the \LaTeX{} compiler.
%
% %%%%%%%%%%%%%%%%%%%%%%%%%%%%%%%%%%%%%%
% \paragraph{Main File.}
%
% The main file is called |cdocsamp.tex|.
%
% Load the \textsf{childdoc} definitions and
% declare the filename for the main document:
%    \begin{macrocode}
\input{childdoc.def}
\childdocmain{}
%    \end{macrocode}

% Optional override for |\version| flag:
%    \begin{macrocode}
%%\ifchilddoc\else\providecommand{\version}{draft}\fi
%    \end{macrocode}

% Define the default values for the |\version| flag
% (|final| for the main file and |draft| for childs):
%    \begin{macrocode}
\ifchilddoc
\providecommand{\version}{draft}
\else
\providecommand{\version}{final}
\fi
%    \end{macrocode}

% Load the standard document class:
%    \begin{macrocode}
\documentclass[12pt]{article}
%    \end{macrocode}

% Start the document body:
%    \begin{macrocode}
\begin{document}
%    \end{macrocode}

% Declare a title page.
% Print title, part of document being processed and version flag:
%    \begin{macrocode}
\addtocounter{page}{-1}
\begin{center}
{\LARGE\bfseries{}childdoc example\par}
\vspace{1cm}
\ifchilddoc
\ifchilddocmanual part\else chapter\fi:
`\childdocname' of `\childdocjob'\par
\else
main document: `\childdocjob'\par
\fi
version: \version\par
\end{center}
\newpage
%    \end{macrocode}

% Manually include selected file,
% otherwise process as usual:
%    \begin{macrocode}
\ifchilddocmanual
\section*{part `\childdocname'}
\input{\childdocname}
\else
%    \end{macrocode}

% Include the two chapters:
%    \begin{macrocode}
\include{cdocsch1}
\include{cdocsch2}
%    \end{macrocode}

% Include the two parts unless only chapters should be displayed:
%    \begin{macrocode}
\ifchilddoc\else
\section{part three}
\input{cdocspt3}
\section{part four}
\input{cdocspt4}
\fi
%    \end{macrocode}

% Process as usual until here:
%    \begin{macrocode}
\fi
%    \end{macrocode}

% End of document body:
%    \begin{macrocode}
\end{document}
%    \end{macrocode}
%\iffalse
%</samplemain>
%\fi
%
% %%%%%%%%%%%%%%%%%%%%%%%%%%%%%%%%%%%%%%
% \paragraph{Chapter Include Files.}
%
% The include files are called |cdocsch1.tex| and |cdocsch2.tex|.
%
%\iffalse
%<*samplechap1|samplechap2>
%\fi

% Optional override for |\version| flag:
%    \begin{macrocode}
%%\providecommand{\version}{final}
%    \end{macrocode}

% Include the main document:
%    \begin{macrocode}
\input{childdoc.def}
\childdocof{cdocsamp}
%    \end{macrocode}

%\iffalse
%</samplechap1|samplechap2>
%\fi
%
%\iffalse
%<*samplechap1>
%\fi
% Some text for chapter 1:
%    \begin{macrocode}
\section{one}
some text in chapter one
%    \end{macrocode}

%\iffalse
%</samplechap1>
%\fi
% Some text for chapter 2:
%\iffalse
%<*samplechap2>
%\fi
%    \begin{macrocode}
\section{two}
more text in chapter two
%    \end{macrocode}

%\iffalse
%</samplechap2>
%\fi
%
% %%%%%%%%%%%%%%%%%%%%%%%%%%%%%%%%%%%%%%
% \paragraph{Part Include Files.}
%
% The include files are called |cdocspt3.tex| and |cdocspt4.tex|.
%
%\iffalse
%<*samplepart3|samplepart4>
%\fi

% Optional override for |\version| flag:
%    \begin{macrocode}
%%\providecommand{\version}{final}
%    \end{macrocode}

% Include the main document:
%    \begin{macrocode}
\input{childdoc.def}
\childdocby{cdocsamp}
%    \end{macrocode}

%\iffalse
%</samplepart3|samplepart4>
%\fi
%
%\iffalse
%<*samplepart3>
%\fi
% Some text for part 3:
%    \begin{macrocode}
some text in part three
%    \end{macrocode}

%\iffalse
%</samplepart3>
%\fi
% Some text for part 4:
%\iffalse
%<*samplepart4>
%\fi
%    \begin{macrocode}
more text in part four
%    \end{macrocode}

%\iffalse
%</samplepart4>
%\fi
%
% %%%%%%%%%%%%%%%%%%%%%%%%%%%%%%%%%%%%%%
% \paragraph{Forwarding for a Complete Draft.}
%
% The following forwarding file |cdocsdrf.tex|
% compiles the main document in draft mode:
%\iffalse
%<*sampledraft>
%\fi
%    \begin{macrocode}
\def\version{draft}
\input{childdoc.def}
\childdocforward{cdocsamp}
%    \end{macrocode}

%\iffalse
%</sampledraft>
%\fi
%
% %%%%%%%%%%%%%%%%%%%%%%%%%%%%%%%%%%%%%%
% \paragraph{Forwarding for Final Version of the Chapters.}
%
% The following forwarding files |cdocsfn1.tex| and |cdocsfn2.tex|
% (with identical content)
% compile the final versions of the child documents
% |cdocsch1.tex| and |cdocsch2.tex|, respectively:
%\iffalse
%<*samplefinal>
%\fi
%    \begin{macrocode}
\def\version{final}
\input{childdoc.def}
\childdocforwardprefix[cdocsamp]{cdocsfn}{cdocsch}
%    \end{macrocode}

%\iffalse
%</samplefinal>
%\fi
%
% %%%%%%%%%%%%%%%%%%%%%%%%%%%%%%%%%%%%%%
% \paragraph{Command Line Processing.}
%
% The following three command lines generate the output files
% |cdocscld|, |cdocscl1| and |cdocscl2|
% which should be identical to
% |cdocsdrf|, |cdocsch1| and |cdocsfn2|, respectively:
% \begin{center}
% \begin{tabular}{l}
% |latex -jobname cdocscld \|\\
% |  "\def\version{draft}\input{childdoc.def}\childdocforward{cdocsamp}"|\\
% |latex -jobname cdocscl1 \|\\
% |  "\input{childdoc.def}\childdocforward[cdocsamp]{cdocsch1}"|\\
% |latex -jobname cdocscl2 \|\\
% |  "\def\version{final}\input{childdoc.def}\childdocforward{cdocsch2}"|
% \end{tabular}
% \end{center}
% Note that the trailing backslash on each first line
% merely continues the input to the second line
% (for convenient cut ant paste).
% Furthermore, the command |latex| can be replaced by any
% of its alternative versions such as |pdflatex|.
%
% %%%%%%%%%%%%%%%%%%%%%%%%%%%%%%%%%%%%%%%%%%%%%%%%%%%%%%%%%%%%%%%%%%%%%%%%%%%%%%
% %%%%%%%%%%%%%%%%%%%%%%%%%%%%%%%%%%%%%%%%%%%%%%%%%%%%%%%%%%%%%%%%%%%%%%%%%%%%%%
% \section{Implementation}
%\iffalse
%<*package>
%\fi
%
% This section describes the definitions file |childdoc.def|.

% The definitions cannot be loaded using |\usepackage| or |\RequirePackage|
% which has a mechanism to prevent loading a style file more than once.
% When loading the definitions by means of |\input|
% multiple instances have to be prevented manually:
%\iffalse
%This code needs to be before the `\ProvidesFile' directive
%which is defined at the beginning of this file.
%Therefore it is also placed there and commented out here.
%</package>
%<*discard>
%\fi
%    \begin{macrocode}
\ifdefined\childdocmain\endinput\fi
%    \end{macrocode}
%\iffalse
%</discard>
%<*package>
%\fi
%
% \macro{\ifchilddoc}
% \macro{\ifchilddocmanual}
% The conditional |\ifchilddoc| tells whether a
% child (true) or main (false) document is being compiled.
% The conditional |\ifchilddocmanual| tells whether
% the |\includeonly| mechanism is used (false) or
% the selection of child files must be performed manually (true).
% The definitions initialise to false:
%    \begin{macrocode}
\newif\ifchilddoc
\newif\ifchilddocmanual
%    \end{macrocode}

% \macro{\childdocname}
% \macro{\childdocjob}
% The macro |\childdocname| stores the name of the main document
% to be compiled. The macro |\childdocjob| stores the name of
% the document on which the \LaTeX{} compiler was originally invoked.
% The content of |\jobname| cannot be compared
% to filenames specified in the source due to different catcodes.
% The following code rescans |\jobname|, stores the result
% in |\childdocname| and saves a copy in |\childdocjob|:
%    \begin{macrocode}
\edef\childdocname{\scantokens\expandafter{\jobname\noexpand}}
\let\childdocjob\childdocname
%    \end{macrocode}

% \macro{\childdocdisable}
% The macro |\childdocdisable| prevents the main file
% from being processed more than once.
% At this stage, the main document command |\childdocmain|
% is assumed to be called once again where it should do nothing.
% Any subsequent call to it should prevent
% a secondary processing of the main document
% It overwrites the forwarding commands
% |\childdocof| and |\childdocforward|
% with empty macros to prevent further inclusions of the main document:
%    \begin{macrocode}
\newcommand{\childdocdisable}
{
  \renewcommand{\childdocmain}[1]{\renewcommand{\childdocmain}[1]{\endinput}}
  \renewcommand{\childdocof}[1]{}
  \renewcommand{\childdocby}[2][]{}
  \renewcommand{\childdocforward}[2][]{}
  \renewcommand{\childdocdisable}{}
}
%    \end{macrocode}

% \macro{\childdocmain}
% The macro |\childdocmain| is to be called at the top of the main file
% with nothing or the main filename (without extension) as argument.
% First, it breaks loops.
% If the argument is not empty and does not match |\childdocname|
% (which is set by the first inclusion of |childdoc.def|),
% |\ifchilddoc| is set to true, |\includeonly| is applied to the child file
% and |\jobname| is set to the main file
% (for proper handling of |.aux| files):
%    \begin{macrocode}
\newcommand{\childdocmain}[1]
{
  \childdocdisable\childdocmain{}
  \if?#1?\else
    \begingroup
      \def\childdoctmp{#1}
      \ifx\childdoctmp\childdocname
        \def\childdoctmp{}
      \else
        \def\childdoctmp
        {
          \childdoctrue
          \includeonly{\childdocname}
          \def\childdocjob{#1}
          \def\jobname{#1}
        }
      \fi
      \expandafter
    \endgroup
    \childdoctmp
  \fi
}
%    \end{macrocode}

% \macro{\childdocof}
% The command |\childdocof| redirects
% compilation to the main file |#1|.
%    \begin{macrocode}
\newcommand{\childdocof}[1]
{
  \childdocdisable
  \childdoctrue
  \includeonly{\childdocname}
  \def\jobname{#1}
  \def\childdocjob{#1}
  \input{#1}
}
%    \end{macrocode}

% \macro{\childdocby}
% The command |\childdocby| ....
%    \begin{macrocode}
\newcommand{\childdocby}[2][]
{
  \childdocdisable
  \childdoctrue
  \childdocmanualtrue
  \if?#1?\else
    \def\jobname{#2}
  \fi
  \def\childdocjob{#2}
  \input{#2}
  \endinput
}
%    \end{macrocode}

% \macro{\childdocforward}
% The command |\childdocforward| redirects
% compilation to the main file or
% (if the optional argument is given) a child file.
% Parameters are set as if the main file
% or a child file starting with |\childdocof| was compiled.
% Then compilation is handed over to the main file:
%    \begin{macrocode}
\newcommand{\childdocforward}[2][]
{
  \begingroup
    \if?#1?
      \def\childdoctmp
      {
        \def\childdocname{#2}
        \def\childdocjob{#2}
        \def\jobname{#2}
        \input{#2}
        \endinput
      }
    \else
      \def\childdoctmp
      {
        \childdocdisable
        \def\childdocname{#2}
        \childdoctrue
        \includeonly{#2}
        \def\childdocjob{#1}
        \def\jobname{#1}
        \input{#1}
        \endinput
      }
    \fi
    \expandafter
  \endgroup
  \childdoctmp
}
%    \end{macrocode}

% \macro{\childdocforwardprefix}
% The command |\childdocforwardprefix| redirects
% compilation to the main or a child file by means of a pattern.
% The prefix |#1| in the current filename is replaced by |#2|
% and the suffix of the current filename is kept
% (it is assumed that the filename does not contain the substring `|~~~|'
% which is used as a delimiter).
% Compilation is handed over to the new file by |\childdocforward|:
%    \begin{macrocode}
\newcommand{\childdocforwardprefix}[3][]
{
  \begingroup
    \def\childdocextract #2##1~~~{\def\childdoctmp{\childdocforward[#1]{#3##1}}}
    \expandafter\childdocextract\childdocname~~~
    \expandafter
  \endgroup
  \childdoctmp
}
%    \end{macrocode}

% \macro{\childdoc}
% The deprecated macro |\childdoc| is a legacy version of |\childdocmain|:
%    \begin{macrocode}
\newcommand{\childdoc}{\childdocmain}
%    \end{macrocode}

% \macro{\childdocredirect}
% The deprecated macro |\childdocredirect| is a legacy version
% of |\childdocforward| and |\childdocforwardprefix|:
%    \begin{macrocode}
\newcommand{\childdocredirect}[2][]
{
  \begingroup
    \if?#1?
      \def\childdoctmp{\childdocforward{#2}}
    \else
      \def\childdoctmp{\childdocforwardprefix{#1}{#2}}
    \fi
    \expandafter
  \endgroup
  \childdoctmp
}
%    \end{macrocode}

%\iffalse
%</package>
%\fi
%
\endinput

\childdocforward{cdocsamp}
%    \end{macrocode}

%\iffalse
%</sampledraft>
%\fi
%
% %%%%%%%%%%%%%%%%%%%%%%%%%%%%%%%%%%%%%%
% \paragraph{Forwarding for Final Version of the Chapters.}
%
% The following forwarding files |cdocsfn1.tex| and |cdocsfn2.tex|
% (with identical content)
% compile the final versions of the child documents
% |cdocsch1.tex| and |cdocsch2.tex|, respectively:
%\iffalse
%<*samplefinal>
%\fi
%    \begin{macrocode}
\def\version{final}
% \iffalse
%
% childdoc.dtx Copyright (C) 2017-2018 Niklas Beisert
%
% This work may be distributed and/or modified under the
% conditions of the LaTeX Project Public License, either version 1.3
% of this license or (at your option) any later version.
% The latest version of this license is in
%   http://www.latex-project.org/lppl.txt
% and version 1.3 or later is part of all distributions of LaTeX
% version 2005/12/01 or later.
%
% This work has the LPPL maintenance status `maintained'.
%
% The Current Maintainer of this work is Niklas Beisert.
%
% This work consists of the files childdoc.dtx and childdoc.ins
% and the derived files childdoc.def and cdocsamp.tex with
% cdocsch1.tex, cdocsch2.tex, cdocsdrf.tex, cdocsfn1.tex, cdocsfn2.tex.
%
%<package>\ifdefined\childdocmain\endinput\fi
%<package>\ProvidesFile{childdoc.def}[2018/12/30 v2.0 child document driver]
%<samplemain>\ProvidesFile{cdocsamp.tex}[2018/12/30 v2.0 sample for childdoc]
%<*driver>
%\ProvidesFile{childdoc.drv}[2018/12/30 v2.0 childdoc reference manual file]
\PassOptionsToClass{10pt,a4paper}{article}
\documentclass{ltxdoc}

\usepackage[margin=35mm]{geometry}
\usepackage{hyperref}
\usepackage{hyperxmp}
\usepackage[usenames]{color}

\hypersetup{colorlinks=true}
\hypersetup{pdfstartview=FitH}
\hypersetup{pdfpagemode=UseNone}
\hypersetup{pdfsource={}}
\hypersetup{pdflang={en-UK}}
\hypersetup{pdfcopyright={Copyright 2017-2018 Niklas Beisert.
  This work may be distributed and/or modified under the
  conditions of the LaTeX Project Public License, either version 1.3
  of this license or (at your option) any later version.}}
\hypersetup{pdflicenseurl={http://www.latex-project.org/lppl.txt}}
\hypersetup{pdfcontactaddress={ETH Zurich, ITP, HIT K,
  Wolfgang-Pauli-Strasse 27}}
\hypersetup{pdfcontactpostcode={8093}}
\hypersetup{pdfcontactcity={Zurich}}
\hypersetup{pdfcontactcountry={Switzerland}}
\hypersetup{pdfcontactemail={nbeisert@itp.phys.ethz.ch}}
\hypersetup{pdfcontacturl={http://people.phys.ethz.ch/\xmptilde nbeisert/}}

\newcommand{\secref}[1]{\hyperref[#1]{section \ref*{#1}}}

\parskip1ex
\parindent0pt
\let\olditemize\itemize
\def\itemize{\olditemize\parskip0pt}

\begin{document}

\title{The \textsf{childdoc} Package}
\hypersetup{pdftitle={The childdoc Package}}
\author{Niklas Beisert\\[2ex]
  Institut f\"ur Theoretische Physik\\
  Eidgen\"ossische Technische Hochschule Z\"urich\\
  Wolfgang-Pauli-Strasse 27, 8093 Z\"urich, Switzerland\\[1ex]
  \href{mailto:nbeisert@itp.phys.ethz.ch}
  {\texttt{nbeisert@itp.phys.ethz.ch}}}
\hypersetup{pdfauthor={Niklas Beisert}}
\hypersetup{pdfsubject={Manual for the LaTeX2e Package childdoc}}
\date{30 December 2018, \textsf{v2.0}}
\maketitle

\begin{abstract}\noindent
\textsf{childdoc} is a \LaTeXe{} package
that enables the direct compilation
of document sections included by |\include|
to individual files.
\end{abstract}

\begingroup
\parskip0ex
\tableofcontents
\endgroup

%%%%%%%%%%%%%%%%%%%%%%%%%%%%%%%%%%%%%%%%%%%%%%%%%%%%%%%%%%%%%%%%%%%%%%%%%%%%%%%%
%%%%%%%%%%%%%%%%%%%%%%%%%%%%%%%%%%%%%%%%%%%%%%%%%%%%%%%%%%%%%%%%%%%%%%%%%%%%%%%%
\section{Introduction}

\LaTeX{} provides a mechanism to structure a large document (such as a book)
into a main file and several child files (containing the chapters)
using the |\include| command.
This mechanism is beneficial for documents
which span hundreds of pages in order to
make the source file(s) more manageable.
Moreover, compilation can be restricted to
selected child files by means of the |\includeonly| command.
The latter feature can be used to reduce the compilation time while editing
(this was significantly more useful in the earlier days of \LaTeX{})
or to generate a smaller document which is easier to navigate.
Another application of |\includeonly| is to generate
documents consisting of selected parts of the complete document.

However, there are a few drawbacks of the plain |\include| mechanism:
\begin{itemize}
\item
The child files cannot be compiled on their own,
they can only be compiled via the main file.
A naive editing environment
(such as a text editor with an option
to have the current file processed by \LaTeX)
may require one to switch to the main file before compiling;
attempting to compile the child file produces errors.
\item
The main file must be modified (each time)
to adjust the |\includeonly| command
to the present needs. This easily leaves the main file in a messy state.
\item
The generated document will always carry the filename
of the main document. This is inconvenient if
several child files are to be compiled and
to be kept for distribution.
\end{itemize}

The present package provides a simple interface
to make child files individually compilable by \LaTeX{}.
Compiling a child file then has the same effect as compiling
the main file with an |\includeonly| command
to select the appropriate child.
Moreover the generated document will carry the name of the child
rather than the main file.
This resolves all three above issues.

This feature is meant to make the editing of books,
thesis documents and lecture notes somewhat more convenient.
However, the package can also be used efficiently for
composing a series of documents (such as exercise sheets)
which are typically distributed individually.
It then assists the author in generating the individual documents
(potentially in different versions)
as well as a document containing the collected series.
Another application is in developing style files
or other kinds of included material
where compilation of the style file could redirect
to a sample or test file.

%%%%%%%%%%%%%%%%%%%%%%%%%%%%%%%%%%%%%%%%%%%%%%%%%%%%%%%%%%%%%%%%%%%%%%%%%%%%%%%%
%%%%%%%%%%%%%%%%%%%%%%%%%%%%%%%%%%%%%%%%%%%%%%%%%%%%%%%%%%%%%%%%%%%%%%%%%%%%%%%%
\section{Usage}

First of all, the package \textsf{childdoc} is \emph{not} a standard
\LaTeXe{} |.sty| style file! Therefore it needs to be invoked in
a non-standard way.

%%%%%%%%%%%%%%%%%%%%%%%%%%%%%%%%%%%%%%%%%%%%%%%%%%%%%%%%%%%%%%%%%%%%%%%%%%%%%%%%
\subsection{Included Files}
\label{sec:include}

%%%%%%%%%%%%%%%%%%%%%%%%%%%%%%%%%%%%%%%%
\DescribeMacro{\childdocmain}
To use the package, add the commands
\begin{center}
\begin{tabular}{l}
|\input{childdoc.def}|\\
|\childdocmain{}|\\
\end{tabular}
\end{center}
at the very top of the main \LaTeX{} file,
in particular \emph{before} the |\documentclass| statement!
The argument of |\childdocmain| should be left empty
(but it must be present).

%%%%%%%%%%%%%%%%%%%%%%%%%%%%%%%%%%%%%%%%
\DescribeMacro{\childdocof}
Furthermore, add the commands
\begin{center}
\begin{tabular}{l}
|\input{childdoc.def}|\\
|\childdocof{|\textit{main}|}|\\
\end{tabular}
\end{center}
at the top of every child file \textit{child}
which is included by |\include{|\textit{child}|}|
from within the main file
(or at least for those files to be compiled individually).
The argument \textit{main} must be the filename of the main file.

There are a couple of
considerations in setting up the main and child documents:

%%%%%%%%%%%%%%%%%%%%%%%%%%%%%%%%%%%%%%%%
\paragraph{Restrictions.}

Please note the following restrictions:
\begin{itemize}
\item
|\childdocmain| must be called with one argument \textit{main}
to ensure compatibility with earlier version of the package.
It must either be empty (|\childdocmain{}|)
or precisely match the filename of the main file in which it is specified.
See \secref{sec:detection} for further information.
\item
The filename \textit{main} must be specified without the |.tex| extension.
\item
The filename \textit{main} is case sensitive
(even in case-insensitive file systems)
due to internal string comparison.
\item
The argument \textit{main} should be fully expanded, it cannot be a macro.
\item
Subdirectories and special characters should be avoided in filenames.
\item
The command |\childdocmain{|\textit{main}|}| must be followed by a whitespace.
It should not be followed immediately by another command
or by a comment mark `|%|'.
This is because the \TeX{} parser reads the token immediately following
the argument of |\childdocmain| and puts it
at the beginning of every child section;
however, a white\-space is ignored.
\end{itemize}

%%%%%%%%%%%%%%%%%%%%%%%%%%%%%%%%%%%%%%%%
\paragraph{Content of Main File.}

It is advisable to place all content in the child files included by |\include|.
Any output contained in the main file will appear in all child documents
unless suppressed manually;
it cannot be suppressed automatically by the |\includeonly| directive
and thus should normally be avoided.
A method to include some content in the main file
by means of conditional processing is described in \secref{sec:conditional}.

%%%%%%%%%%%%%%%%%%%%%%%%%%%%%%%%%%%%%%%%
\paragraph{Page Numbering.}

When only a part of the document is compiled,
the appropriate numbering of pages
(as well as other status parameters)
is determined from the |.aux| files.
The latter contain information from previous passes.
However this information needs to propagate through
all intermediate child documents.
Therefore the page numbering in child documents may well
be inconsistent until the complete document is compiled at least once.

A useful (if unconventional) way to always ensure a consistent
page numbering is to restart the numbering in each child document
and denote the pages by `\textit{child}|.|\textit{page}'
where \textit{child} represents the chapter/section number of the child file.
This can be achieved by the command
|\numberwithin{page}{|\textit{child}|}|
of the \textsf{amsmath} package
where \textit{child} can be |chapter| or |section|
depending on the chosen structuring.
Alternatively, one can modify the macro |\thepage| appropriately
and reset the counter |page| at the start of each child file.

%%%%%%%%%%%%%%%%%%%%%%%%%%%%%%%%%%%%%%%%%%%%%%%%%%%%%%%%%%%%%%%%%%%%%%%%%%%%%%%%
\subsection{Conditional Processing}
\label{sec:conditional}

The package provides a mechanism to compile different versions
of a document. To customise the versions further some conditional processing
can come in handy to distinguish which version is being compiled.
The package provides two macros to describe the compilation context:

%%%%%%%%%%%%%%%%%%%%%%%%%%%%%%%%%%%%%%%%
\DescribeMacro{\ifchilddoc}
The conditional |\ifchilddoc| distinguishes between the compilation of
child documents and the main document:
%
\begin{center}
|\ifchilddoc |\textit{child-code}| |[|\||else |\textit{main-code}]| \||fi|
\end{center}

%%%%%%%%%%%%%%%%%%%%%%%%%%%%%%%%%%%%%%%%
\DescribeMacro{\childdocname}
\DescribeMacro{\childdocjob}
The macro |\childdocname| contains the filename (without extension)
of the main or child file being processed.
Note that |\childdocjob| will always contain the name of the main file.

%%%%%%%%%%%%%%%%%%%%%%%%%%%%%%%%%%%%%%%%
\paragraph{Title Page.}

Conditional processing can be used to include a title or banner page
in the main document when proper precautions are taken.
Importantly, the code in the main file should ensure that the page counter
(as well as other status parameters which are stored in the |.aux| files)
takes the same value after the conditional processing.
Otherwise the page numbers may take divergent values
depending on which part is compiled.

For example, a title page could be declared by:
%
\begin{center}
\begin{tabular}{l}
|\ifchilddoc\||else|\\
|\addtocounter{page}{-1}|\\
\textit{code for title page}\\
|\newpage|\\
|\||fi|
\end{tabular}
\end{center}
%
A banner page for the child documents can be generated by:
%
\begin{center}
\begin{tabular}{l}
|\ifchilddoc|\\
|\addtocounter{page}{-1}|\\
\textit{code for banner page}\\
|\newpage|\\
|\||fi|
\end{tabular}
\end{center}
%
Here one could write a message such as:
\begin{center}
|This is the part \childdocname{} of \childdocjob{}.|
\end{center}

%%%%%%%%%%%%%%%%%%%%%%%%%%%%%%%%%%%%%%%%%%%%%%%%%%%%%%%%%%%%%%%%%%%%%%%%%%%%%%%%
\subsection{Flags}
\label{sec:flags}

The package makes it easy to generate different versions
of the main or child documents.
To this end compilation flags can be defined
and assigned different default values.
They will be particularly useful in conjunction
with the forwarding mechanism described in \secref{sec:forward}.

For example, it may be useful to have a flag |\version|
which can be set to |draft| or |final|.
The document source will contain some conditional code
depending on the value of |\version|.
Suppose further, the flag should default to |final| for the main file
and to |draft| for child files
which is a natural assignment for editing the document.
This is achieved by placing the following code
in the preamble of the main document
(below the |\childdocmain| directive):
%
\begin{center}
\begin{tabular}{l}
|\ifchilddoc|\\
|\providecommand{\version}{draft}|\\
|\||else|\\
|\providecommand{\version}{final}|\\
|\||fi|
\end{tabular}
\end{center}
%
The definition by |\providecommand| makes sure
that previous definitions are not overwritten.
Further statements |\providecommand{\version}{...}|
can thus be added before the above code to override it.

For the main file, one might add a line
(between |\childdocmain| and the above block)
%
\begin{center}
|%\ifchilddoc\||else\providecommand{\version}{draft}\||fi|
\end{center}
%
which can be uncommented to produce a draft version.
Likewise one can add a line to the very top of a child file
(above the |\childdocof{|\textit{main}|}| directive)
%
\begin{center}
|%\providecommand{\version}{final}|
\end{center}
%
which can be uncommented to produce the final version of this child document.

%%%%%%%%%%%%%%%%%%%%%%%%%%%%%%%%%%%%%%%%%%%%%%%%%%%%%%%%%%%%%%%%%%%%%%%%%%%%%%%%
\subsection{Forwarding}
\label{sec:forward}

Different versions of the main or child documents
using compilation flags as described in \secref{sec:flags}
can be (permanently) stored in different files
for convenient compilation, viewing and distribution.
To this end, the package defines a command
to pass on compilation to a different file:

%%%%%%%%%%%%%%%%%%%%%%%%%%%%%%%%%%%%%%%%
\DescribeMacro{\childdocforward}
The command |\childdocforward| redirects processing to
another source file:
%
\begin{center}
\begin{tabular}{l}
|\input{childdoc.def}|\\
|\childdocforward[|\textit{main}|]{|\textit{dest}|}|\\
\end{tabular}
\end{center}
%
The argument \textit{dest} is the destination file
(without extension).
It should be the main file or one of the child files.
Note that further \textsf{childdoc} directives
such as |\childdocof| and |\childdocforward|
in the indicated file will be processed in this form.
The optional argument \textit{main}
passes on directly to the main file \textit{main}
while pretending to compile the child \textit{dest}.
This form behaves as if \textit{dest}
issues |\childdocof{|\textit{main}|}| right away,
and no further \textsf{childdoc} directives will be processed.

%%%%%%%%%%%%%%%%%%%%%%%%%%%%%%%%%%%%%%%%
\DescribeMacro{\...prefix}
In the alternative form |\childdocforwardprefix|,
%
\begin{center}
\begin{tabular}{l}
|\input{childdoc.def}|\\
|\childdocforwardprefix[|\textit{main}|]{|\textit{prefix}|}{|\textit{dest}|}|
\end{tabular}
\end{center}
%
the destination file is determined by a pattern
depending on the current file:
To make this work, the current file must be called
`{\textit{prefix}\hspace{0.2em}\textit{suffix}}'
with \textit{prefix} matching precisely the argument.
Processing is then passed on to the file
`{\textit{dest}\hspace{0.2em}\textit{suffix}}'.
Surely, the same effect is achieved by
directly specifying the
argument `{\textit{dest}\hspace{0.2em}\textit{suffix}}'
in the first form.
However, that requires to set up a different file
for each child. With the alternative form of the command
all these files can have exactly the same content
which simplifies setting them up and maintaining them.

For example, the following file |draft.tex|
with a compilation flag |\version| as described in \secref{sec:flags}
compiles the main document as a draft:
%
\begin{center}
\begin{tabular}{l}
|\def\version{draft}|\\
|\input{childdoc.def}|\\
|\childdocforward{|\textit{main}|}|
\end{tabular}
\end{center}
%
Likewise, the following files |final|\textit{nn}|.tex|
compile the final version of the child document
|child|\textit{nn}|.tex|:
%
\begin{center}
\begin{tabular}{l}
|\def\version{final}|\\
|\input{childdoc.def}|\\
|\childdocforwardprefix{final}{child}|
\end{tabular}
\end{center}
%

Note that when several versions of a main file and/or of each child file
are to be generated, it may be convenient to set up a |Makefile| or
shell script to automatise the process.

%%%%%%%%%%%%%%%%%%%%%%%%%%%%%%%%%%%%%%%%%%%%%%%%%%%%%%%%%%%%%%%%%%%%%%%%%%%%%%%%
\subsection{Command Line Processing}
\label{sec:commandline}

The effect of redirection files can also be achieved by invoking
the \LaTeX{} compiler with a more elaborate command line.
Most conveniently this should be done as part
of a shell script or a |Makefile|.

When using \textsf{childdoc} in the main file, the following
command lines effectively perform a redirection
(note that depending on the shell being used,
backslashes may have to be doubled: `|\|' $\to$ `|\\|'):
%
\begin{center}
|... -jobname "|\textit{target}|" |\\|"|[\textit{flags}]%
|\input{childdoc.def}\childdocforward[|\textit{main}|]{|\textit{dest}|}"|
\end{center}
%
Here \textit{target} is the name of the output file,
\textit{main} is the name of the main file
and \textit{dest} is the name of the main or child file to be processed
(all filenames without extensions).
The optional argument \textit{main} can be omitted
if \textit{main} matches \textit{dest}.
Optionally, compilation \textit{flags} can be defined via |\def| commands.
This command line makes the \TeX{} engine believe
it is compiling the file \textit{target}
whose content is specified as the latter parameter.
The provided code then forwards the processing to
\textit{main} or \textit{dest} as described in \secref{sec:forward}.

%%%%%%%%%%%%%%%%%%%%%%%%%%%%%%%%%%%%%%%%%%%%%%%%%%%%%%%%%%%%%%%%%%%%%%%%%%%%%%%%
\subsection{Include by Input}
\label{sec:input}

Including child documents by |\include| has some restrictions by design.
Most notably, the content of a child document always occupies
its own set of pages; pages cannot be shared between child documents.
Usually, this behaviour makes perfect sense
because each child document contain an essential part of the document.
However, in some situations it may be desirable to compose
a document from a collection of parts
without having mandatory page breaks between then.
For this case, the package
provides a mechanism to include parts
by |\input| which can also be processed individually.
However, by construction this mechanism
requires manual handling of the content to be output.

%%%%%%%%%%%%%%%%%%%%%%%%%%%%%%%%%%%%%%%%
\DescribeMacro{\ifchilddocmanual}
The main file should be prepared as usual, see \secref{sec:include}.
However, the document body must make a distinction
between processing of an individual part and of the main document, e.g.:
%
\begin{center}
\begin{tabular}{l}
|\ifchilddocmanual|\\
|\input{\childdocname}|\\
|\||else|\\
\textit{document body with }|\input{|\textit{part}|}|\\
|\||fi|
\end{tabular}
\end{center}
%
The conditional |\ifchilddocmanual| is true whenever
a part to be included by |\input| is being compiled,
and the name of the part is stored in |\childdocname|.

%%%%%%%%%%%%%%%%%%%%%%%%%%%%%%%%%%%%%%%%
\DescribeMacro{\childdocby}
Each part to be included by |\input| should start with:
%
\begin{center}
\begin{tabular}{l}
|\input{childdoc.def}|\\
|\childdocby{|\textit{main}|}|\\
\end{tabular}
\end{center}
%
The directive |\childdocby| is similar to |\childdocof|
described in \secref{sec:include},
but the subsequent selection of content must be done manually.
To that end, both |\ifchilddoc| and |\ifchilddocmanual|
will be true upon processing of a part,
and the name of the part is stored in |\childdocname|.
Note that |\jobname| will be set to the filename of the current part
so that each part receives an individual |.aux| file
that does not interfere with the |.aux| file(s) of the main document.
This behaviour can be altered by the alternative form
|\childdocby[*]{|\textit{main}|}| (with a non-empty optional argument)
which uses the |.aux| file of the main document
by setting |\jobname| to \textit{main}.

%%%%%%%%%%%%%%%%%%%%%%%%%%%%%%%%%%%%%%%%%%%%%%%%%%%%%%%%%%%%%%%%%%%%%%%%%%%%%%%%
\subsection{Driver Development}
\label{sec:driver}

The \textsf{childdoc} mechanism can also be use for the development
of definition files such as \LaTeX{} styles or classes.
This case differs from the above setup with multiple parts
included by |\include| in that no |\includeonly| should be invoked.
This can be achieved by starting the include file
(before |\ProvidesPackage|) with:
%
\begin{center}
\begin{tabular}{l}
|\input{childdoc.def}|\\
|\childdocforward{|\textit{main}|}|\\
\end{tabular}
\end{center}
%
or alternatively with:
%
\begin{center}
\begin{tabular}{l}
|\input{childdoc.def}|\\
|\childdocby{|\textit{main}|}|\\
\end{tabular}
\end{center}
%
Both forms have slightly different effects as described above.
The main file is prepared as usual, see \secref{sec:include}.

%%%%%%%%%%%%%%%%%%%%%%%%%%%%%%%%%%%%%%%%%%%%%%%%%%%%%%%%%%%%%%%%%%%%%%%%%%%%%%%%
\subsection{Legacy Detection}
\label{sec:detection}

The directive |\childdocmain| in the main file can detect
whether the complete document or merely a child is to be compiled
even without using the directive |\childdocof|.
This method is deprecated because it is less robust
and there is no compelling reason to use it;
it is merely provided for backward compatibility
and it may be removed in future versions.

If the detection mechanism is to be used,
it is mandatory to correctly specify
the filename of the main file as the argument of |\childdocmain|:
%
\begin{center}
\begin{tabular}{l}
|\input{childdoc.def}|\\
|\childdocmain{|\textit{main}|}|\\
\end{tabular}
\end{center}
%
If |\jobname| does not match the argument \textit{main} of |\childdocmain|,
it is assumed that |\jobname| points to the child file to be compiled.
When using |\childdocmain| with the main file specified as argument,
it suffices to start a child file
with just |\input{|\textit{main}|}|
without loading of the package and using |\childdocof|.
If instead all processing is done
with the appropriate \textsf{childdoc} directives,
the argument of \textit{main} of |\childdocmain| can be empty.

An alternative version of the command line processing described
in \secref{sec:commandline} using the detection mechanism reads:
%
\begin{center}
|... -jobname "|\textit{target}|" "|[\textit{flags}]%
[|\def\jobname{|\textit{dest}|}|]|\input{|\textit{main}|}"|
\end{center}

%%%%%%%%%%%%%%%%%%%%%%%%%%%%%%%%%%%%%%%%%%%%%%%%%%%%%%%%%%%%%%%%%%%%%%%%%%%%%%%%
\subsection{Manual Code}
\label{sec:manual}

In case one cannot be certain whether the definitions file |childdoc.def|
is installed on the target \TeX{} distribution
and one prefers not to ship it,
it is conceivable to paste a few relevant commands into the sources.

To that end, drop all statements |\input{childdoc.def}|
and perform the replacements as outlined below.
Instead of |\childdocmain{|\textit{main}|}| add the following code
to the top of the main file:
%
\begin{center}
\begin{tabular}{l}
|\||ifdefined\childdocname\endinput\||fi\newif\ifchilddoc|\\
|\edef\childdocname{\scantokens\expandafter{\jobname\noexpand}}|\\
|\def\childdocmain{|\textit{main}|}\||ifx\childdocmain\childdocname\||else|\\
|\childdoctrue\includeonly{\childdocname}\let\jobname\childdocmain\||fi|\\
\end{tabular}
\end{center}
%
Instead of |\childdocof{|\textit{main}|}| just include the main file
at the top of each child file:
%
\begin{center}
|\input{|\textit{main}|}|
\end{center}
%
A simple redirection |\childdocforward{|\textit{dest}|}| is achieved by:
%
\begin{center}
|\def\jobname{|\textit{dest}|}\input{\jobname}|
\end{center}
%
The redirection with prefix
|\childdocforwardprefix[|\textit{prefix}|]{|\textit{dest}|}|
is accomplished by:
%
\begin{center}
\begin{tabular}{l}
|{\edef\jobname{\scantokens\expandafter{\jobname\noexpand}}|\\
|\def\redirectjob |\textit{prefix}|#1~~~{\gdef\jobname{|\textit{dest}|#1}}|\\
|\expandafter\redirectjob\jobname~~~}\input{\jobname}|
\end{tabular}
\end{center}

In an alternative approach,
child documents can be compiled by a specific command line
without additional code or specific definitions:
%
\begin{center}
|... -jobname "|\textit{target}|" "|[\textit{flags}]%
|\includeonly{|\textit{dest}|}\input{|\textit{main}|}"|
\end{center}
%

%%%%%%%%%%%%%%%%%%%%%%%%%%%%%%%%%%%%%%%%%%%%%%%%%%%%%%%%%%%%%%%%%%%%%%%%%%%%%%%%
%%%%%%%%%%%%%%%%%%%%%%%%%%%%%%%%%%%%%%%%%%%%%%%%%%%%%%%%%%%%%%%%%%%%%%%%%%%%%%%%
\section{Information}

%%%%%%%%%%%%%%%%%%%%%%%%%%%%%%%%%%%%%%%%%%%%%%%%%%%%%%%%%%%%%%%%%%%%%%%%%%%%%%%%
\subsection{Copyright}

Copyright \copyright{} 2017--2018 Niklas Beisert

This work may be distributed and/or modified under the
conditions of the \LaTeX{} Project Public License, either version 1.3
of this license or (at your option) any later version.
The latest version of this license is in
  \url{http://www.latex-project.org/lppl.txt}
and version 1.3 or later is part of all distributions of \LaTeX{}
version 2005/12/01 or later.

This work has the LPPL maintenance status `maintained'.

The Current Maintainer of this work is Niklas Beisert.

This work consists of the files |README.txt|, |childdoc.ins| and |childdoc.dtx|
as well as the derived files |childdoc.def|, |cdocsamp.tex|
with |cdocsch1.tex|, |cdocsch2.tex|, |cdocspt3.tex|, |cdocspt4.tex|,
|cdocsdrf.tex|, |cdocsfn1.tex|, |cdocsfn2.tex|
as well as |childdoc.pdf|.

%%%%%%%%%%%%%%%%%%%%%%%%%%%%%%%%%%%%%%%%%%%%%%%%%%%%%%%%%%%%%%%%%%%%%%%%%%%%%%%%
\subsection{Files and Installation}

The package consists of the files:
%
\begin{center}
\begin{tabular}{ll}
    |README.txt|   & readme file \\
    |childdoc.ins| & installation file \\
    |childdoc.dtx| & source file \\
    |childdoc.def| & definition file \\
    |cdocsamp.tex| & sample main file \\
    |cdocsch1.tex| & sample include file \\
    |cdocsch2.tex| & sample include file \\
    |cdocspt3.tex| & sample part file \\
    |cdocspt4.tex| & sample part file \\
    |cdocsdrf.tex| & sample redirection file \\
    |cdocsfn1.tex| & sample redirection file \\
    |cdocsfn2.tex| & sample redirection file \\
    |childdoc.pdf| & manual
\end{tabular}
\end{center}
%
The distribution consists of the files
|README.txt|, |childdoc.ins| and |childdoc.dtx|.
%
\begin{itemize}
\item
Run (pdf)\LaTeX{} on |childdoc.dtx|
to compile the manual |childdoc.pdf| (this file).
\item
Run \LaTeX{} on |childdoc.ins| to create the definitions file |childdoc.def|
and the sample |cdocsamp.tex| with include files
|cdocsch1.tex|, |cdocsch2.tex|, |cdocspt3.tex|, |cdocspt4.tex|,
|cdocsdrf.tex|, |cdocsfn1.tex|, |cdocsfn2.tex|.
Then copy the file |childdoc.def| to an appropriate directory of your \LaTeX{}
distribution, e.g.\ \textit{texmf-root}|/tex/latex/childdoc|.
\end{itemize}

%%%%%%%%%%%%%%%%%%%%%%%%%%%%%%%%%%%%%%%%%%%%%%%%%%%%%%%%%%%%%%%%%%%%%%%%%%%%%%%%
\subsection{Related CTAN Packages}

There are several other packages which offer a similar functionality:
%
\begin{itemize}
\item
The packages
\href{http://ctan.org/pkg/docmute}{\textsf{docmute}},
\href{http://ctan.org/pkg/includex}{\textsf{includex}} and
\href{http://ctan.org/pkg/standalone}{\textsf{standalone}}
provide commands to include only the document body of
a child file thus allowing both files to be compiled individually.
\item
The packages \href{http://ctan.org/pkg/subdocs}{\textsf{subdocs}}
and \href{http://ctan.org/pkg/subfiles}{\textsf{subfiles}}
provide structures in which the main and child documents can be
encapsulated and allowing them to be compiled individually.
The inclusion mechanism is different from the conventional |\include|.
\item
The package \href{http://ctan.org/pkg/combine}{\textsf{combine}}
is an elaborate solution to combine several documents into one.
\end{itemize}
%
See also the CTAN topic \href{http://ctan.org/topic/subdocs}{\textsf{subdocs}}
for further related packages.
The present package differs from the above solutions in that
a document structure constructed with the conventional |\include| mechanism
just needs two extra commands at the top of every file
such that all constituent files can be compiled individually.

%%%%%%%%%%%%%%%%%%%%%%%%%%%%%%%%%%%%%%%%%%%%%%%%%%%%%%%%%%%%%%%%%%%%%%%%%%%%%%%%
%\subsection{Feature Suggestions}
%
%The following is a list of features which may be useful for future
%versions of this package:
%%
%\begin{itemize}
%\item
%\ldots
%\end{itemize}

%%%%%%%%%%%%%%%%%%%%%%%%%%%%%%%%%%%%%%%%%%%%%%%%%%%%%%%%%%%%%%%%%%%%%%%%%%%%%%%%
\subsection{Revision History}

%%%%%%%%%%%%%%%%%%%%%%%%%%%%%%%%%%%%%%%%
\paragraph{v2.0:} 2018/12/30

\begin{itemize}
\item
immediate forward processing
\item
added |\childdocby| mechanism
\item
manual restructured
\end{itemize}

%%%%%%%%%%%%%%%%%%%%%%%%%%%%%%%%%%%%%%%%
\paragraph{v1.6:} 2018/01/17

\begin{itemize}
\item
application for development of include files
\item
corrections to manual
\end{itemize}

%%%%%%%%%%%%%%%%%%%%%%%%%%%%%%%%%%%%%%%%
\paragraph{v1.5:} 2017/05/21

\begin{itemize}
\item
more complete structuring introduced
\item
|\childdocof| introduced
\item
|\childdoc| renamed to |\childdocmain|
\item
|\childredirect| renamed to |\childdocforward| and |\childdocforwardprefix|
and functionality expanded
\end{itemize}

%%%%%%%%%%%%%%%%%%%%%%%%%%%%%%%%%%%%%%%%
\paragraph{v1.0:} 2017/04/27

\begin{itemize}
\item
manual and install package
\item
first version published on CTAN
\end{itemize}

%%%%%%%%%%%%%%%%%%%%%%%%%%%%%%%%%%%%%%%%
\paragraph{v0.6:} 2017/04/26

\begin{itemize}
\item
redirection mechanism added
\end{itemize}

%%%%%%%%%%%%%%%%%%%%%%%%%%%%%%%%%%%%%%%%
\paragraph{v0.5:} 2017/04/26

\begin{itemize}
\item
functionality in definition file
\end{itemize}


%%%%%%%%%%%%%%%%%%%%%%%%%%%%%%%%%%%%%%%%%%%%%%%%%%%%%%%%%%%%%%%%%%%%%%%%%%%%%%%%
%%%%%%%%%%%%%%%%%%%%%%%%%%%%%%%%%%%%%%%%%%%%%%%%%%%%%%%%%%%%%%%%%%%%%%%%%%%%%%%%
%%%%%%%%%%%%%%%%%%%%%%%%%%%%%%%%%%%%%%%%%%%%%%%%%%%%%%%%%%%%%%%%%%%%%%%%%%%%%%%%
\appendix

\settowidth\MacroIndent{\rmfamily\scriptsize 000\ }

 \DocInput{childdoc.dtx}

\end{document}
%</driver>
% \fi
%
% %%%%%%%%%%%%%%%%%%%%%%%%%%%%%%%%%%%%%%%%%%%%%%%%%%%%%%%%%%%%%%%%%%%%%%%%%%%%%%
% %%%%%%%%%%%%%%%%%%%%%%%%%%%%%%%%%%%%%%%%%%%%%%%%%%%%%%%%%%%%%%%%%%%%%%%%%%%%%%
% \section{Sample}
%\iffalse
%<*samplemain>
%\fi
%
% The following presents a sample document
% with two chapters, two parts, a title page,
% a compile flag as well as three forwarding files to set the flag.
% It consists of eight |.tex| files:
% \begin{center}
% \begin{tabular}{ll}
% |cdocsamp.tex|&main file\\
% |cdocsch1.tex|&include file for chapter 1\\
% |cdocsch2.tex|&include file for chapter 2\\
% |cdocspt3.tex|&include file for part 3\\
% |cdocspt4.tex|&include file for part 4\\
% |cdocsdrf.tex|&forwarding file for main file in draft mode\\
% |cdocsfi1.tex|&forwarding file for final version of chapter 1\\
% |cdocsfi2.tex|&forwarding file for final version of chapter 2\\
% \end{tabular}
% \end{center}
% Each of the eight files can be compiled directly by the \LaTeX{} compiler.
%
% %%%%%%%%%%%%%%%%%%%%%%%%%%%%%%%%%%%%%%
% \paragraph{Main File.}
%
% The main file is called |cdocsamp.tex|.
%
% Load the \textsf{childdoc} definitions and
% declare the filename for the main document:
%    \begin{macrocode}
\input{childdoc.def}
\childdocmain{}
%    \end{macrocode}

% Optional override for |\version| flag:
%    \begin{macrocode}
%%\ifchilddoc\else\providecommand{\version}{draft}\fi
%    \end{macrocode}

% Define the default values for the |\version| flag
% (|final| for the main file and |draft| for childs):
%    \begin{macrocode}
\ifchilddoc
\providecommand{\version}{draft}
\else
\providecommand{\version}{final}
\fi
%    \end{macrocode}

% Load the standard document class:
%    \begin{macrocode}
\documentclass[12pt]{article}
%    \end{macrocode}

% Start the document body:
%    \begin{macrocode}
\begin{document}
%    \end{macrocode}

% Declare a title page.
% Print title, part of document being processed and version flag:
%    \begin{macrocode}
\addtocounter{page}{-1}
\begin{center}
{\LARGE\bfseries{}childdoc example\par}
\vspace{1cm}
\ifchilddoc
\ifchilddocmanual part\else chapter\fi:
`\childdocname' of `\childdocjob'\par
\else
main document: `\childdocjob'\par
\fi
version: \version\par
\end{center}
\newpage
%    \end{macrocode}

% Manually include selected file,
% otherwise process as usual:
%    \begin{macrocode}
\ifchilddocmanual
\section*{part `\childdocname'}
\input{\childdocname}
\else
%    \end{macrocode}

% Include the two chapters:
%    \begin{macrocode}
\include{cdocsch1}
\include{cdocsch2}
%    \end{macrocode}

% Include the two parts unless only chapters should be displayed:
%    \begin{macrocode}
\ifchilddoc\else
\section{part three}
\input{cdocspt3}
\section{part four}
\input{cdocspt4}
\fi
%    \end{macrocode}

% Process as usual until here:
%    \begin{macrocode}
\fi
%    \end{macrocode}

% End of document body:
%    \begin{macrocode}
\end{document}
%    \end{macrocode}
%\iffalse
%</samplemain>
%\fi
%
% %%%%%%%%%%%%%%%%%%%%%%%%%%%%%%%%%%%%%%
% \paragraph{Chapter Include Files.}
%
% The include files are called |cdocsch1.tex| and |cdocsch2.tex|.
%
%\iffalse
%<*samplechap1|samplechap2>
%\fi

% Optional override for |\version| flag:
%    \begin{macrocode}
%%\providecommand{\version}{final}
%    \end{macrocode}

% Include the main document:
%    \begin{macrocode}
\input{childdoc.def}
\childdocof{cdocsamp}
%    \end{macrocode}

%\iffalse
%</samplechap1|samplechap2>
%\fi
%
%\iffalse
%<*samplechap1>
%\fi
% Some text for chapter 1:
%    \begin{macrocode}
\section{one}
some text in chapter one
%    \end{macrocode}

%\iffalse
%</samplechap1>
%\fi
% Some text for chapter 2:
%\iffalse
%<*samplechap2>
%\fi
%    \begin{macrocode}
\section{two}
more text in chapter two
%    \end{macrocode}

%\iffalse
%</samplechap2>
%\fi
%
% %%%%%%%%%%%%%%%%%%%%%%%%%%%%%%%%%%%%%%
% \paragraph{Part Include Files.}
%
% The include files are called |cdocspt3.tex| and |cdocspt4.tex|.
%
%\iffalse
%<*samplepart3|samplepart4>
%\fi

% Optional override for |\version| flag:
%    \begin{macrocode}
%%\providecommand{\version}{final}
%    \end{macrocode}

% Include the main document:
%    \begin{macrocode}
\input{childdoc.def}
\childdocby{cdocsamp}
%    \end{macrocode}

%\iffalse
%</samplepart3|samplepart4>
%\fi
%
%\iffalse
%<*samplepart3>
%\fi
% Some text for part 3:
%    \begin{macrocode}
some text in part three
%    \end{macrocode}

%\iffalse
%</samplepart3>
%\fi
% Some text for part 4:
%\iffalse
%<*samplepart4>
%\fi
%    \begin{macrocode}
more text in part four
%    \end{macrocode}

%\iffalse
%</samplepart4>
%\fi
%
% %%%%%%%%%%%%%%%%%%%%%%%%%%%%%%%%%%%%%%
% \paragraph{Forwarding for a Complete Draft.}
%
% The following forwarding file |cdocsdrf.tex|
% compiles the main document in draft mode:
%\iffalse
%<*sampledraft>
%\fi
%    \begin{macrocode}
\def\version{draft}
\input{childdoc.def}
\childdocforward{cdocsamp}
%    \end{macrocode}

%\iffalse
%</sampledraft>
%\fi
%
% %%%%%%%%%%%%%%%%%%%%%%%%%%%%%%%%%%%%%%
% \paragraph{Forwarding for Final Version of the Chapters.}
%
% The following forwarding files |cdocsfn1.tex| and |cdocsfn2.tex|
% (with identical content)
% compile the final versions of the child documents
% |cdocsch1.tex| and |cdocsch2.tex|, respectively:
%\iffalse
%<*samplefinal>
%\fi
%    \begin{macrocode}
\def\version{final}
\input{childdoc.def}
\childdocforwardprefix[cdocsamp]{cdocsfn}{cdocsch}
%    \end{macrocode}

%\iffalse
%</samplefinal>
%\fi
%
% %%%%%%%%%%%%%%%%%%%%%%%%%%%%%%%%%%%%%%
% \paragraph{Command Line Processing.}
%
% The following three command lines generate the output files
% |cdocscld|, |cdocscl1| and |cdocscl2|
% which should be identical to
% |cdocsdrf|, |cdocsch1| and |cdocsfn2|, respectively:
% \begin{center}
% \begin{tabular}{l}
% |latex -jobname cdocscld \|\\
% |  "\def\version{draft}\input{childdoc.def}\childdocforward{cdocsamp}"|\\
% |latex -jobname cdocscl1 \|\\
% |  "\input{childdoc.def}\childdocforward[cdocsamp]{cdocsch1}"|\\
% |latex -jobname cdocscl2 \|\\
% |  "\def\version{final}\input{childdoc.def}\childdocforward{cdocsch2}"|
% \end{tabular}
% \end{center}
% Note that the trailing backslash on each first line
% merely continues the input to the second line
% (for convenient cut ant paste).
% Furthermore, the command |latex| can be replaced by any
% of its alternative versions such as |pdflatex|.
%
% %%%%%%%%%%%%%%%%%%%%%%%%%%%%%%%%%%%%%%%%%%%%%%%%%%%%%%%%%%%%%%%%%%%%%%%%%%%%%%
% %%%%%%%%%%%%%%%%%%%%%%%%%%%%%%%%%%%%%%%%%%%%%%%%%%%%%%%%%%%%%%%%%%%%%%%%%%%%%%
% \section{Implementation}
%\iffalse
%<*package>
%\fi
%
% This section describes the definitions file |childdoc.def|.

% The definitions cannot be loaded using |\usepackage| or |\RequirePackage|
% which has a mechanism to prevent loading a style file more than once.
% When loading the definitions by means of |\input|
% multiple instances have to be prevented manually:
%\iffalse
%This code needs to be before the `\ProvidesFile' directive
%which is defined at the beginning of this file.
%Therefore it is also placed there and commented out here.
%</package>
%<*discard>
%\fi
%    \begin{macrocode}
\ifdefined\childdocmain\endinput\fi
%    \end{macrocode}
%\iffalse
%</discard>
%<*package>
%\fi
%
% \macro{\ifchilddoc}
% \macro{\ifchilddocmanual}
% The conditional |\ifchilddoc| tells whether a
% child (true) or main (false) document is being compiled.
% The conditional |\ifchilddocmanual| tells whether
% the |\includeonly| mechanism is used (false) or
% the selection of child files must be performed manually (true).
% The definitions initialise to false:
%    \begin{macrocode}
\newif\ifchilddoc
\newif\ifchilddocmanual
%    \end{macrocode}

% \macro{\childdocname}
% \macro{\childdocjob}
% The macro |\childdocname| stores the name of the main document
% to be compiled. The macro |\childdocjob| stores the name of
% the document on which the \LaTeX{} compiler was originally invoked.
% The content of |\jobname| cannot be compared
% to filenames specified in the source due to different catcodes.
% The following code rescans |\jobname|, stores the result
% in |\childdocname| and saves a copy in |\childdocjob|:
%    \begin{macrocode}
\edef\childdocname{\scantokens\expandafter{\jobname\noexpand}}
\let\childdocjob\childdocname
%    \end{macrocode}

% \macro{\childdocdisable}
% The macro |\childdocdisable| prevents the main file
% from being processed more than once.
% At this stage, the main document command |\childdocmain|
% is assumed to be called once again where it should do nothing.
% Any subsequent call to it should prevent
% a secondary processing of the main document
% It overwrites the forwarding commands
% |\childdocof| and |\childdocforward|
% with empty macros to prevent further inclusions of the main document:
%    \begin{macrocode}
\newcommand{\childdocdisable}
{
  \renewcommand{\childdocmain}[1]{\renewcommand{\childdocmain}[1]{\endinput}}
  \renewcommand{\childdocof}[1]{}
  \renewcommand{\childdocby}[2][]{}
  \renewcommand{\childdocforward}[2][]{}
  \renewcommand{\childdocdisable}{}
}
%    \end{macrocode}

% \macro{\childdocmain}
% The macro |\childdocmain| is to be called at the top of the main file
% with nothing or the main filename (without extension) as argument.
% First, it breaks loops.
% If the argument is not empty and does not match |\childdocname|
% (which is set by the first inclusion of |childdoc.def|),
% |\ifchilddoc| is set to true, |\includeonly| is applied to the child file
% and |\jobname| is set to the main file
% (for proper handling of |.aux| files):
%    \begin{macrocode}
\newcommand{\childdocmain}[1]
{
  \childdocdisable\childdocmain{}
  \if?#1?\else
    \begingroup
      \def\childdoctmp{#1}
      \ifx\childdoctmp\childdocname
        \def\childdoctmp{}
      \else
        \def\childdoctmp
        {
          \childdoctrue
          \includeonly{\childdocname}
          \def\childdocjob{#1}
          \def\jobname{#1}
        }
      \fi
      \expandafter
    \endgroup
    \childdoctmp
  \fi
}
%    \end{macrocode}

% \macro{\childdocof}
% The command |\childdocof| redirects
% compilation to the main file |#1|.
%    \begin{macrocode}
\newcommand{\childdocof}[1]
{
  \childdocdisable
  \childdoctrue
  \includeonly{\childdocname}
  \def\jobname{#1}
  \def\childdocjob{#1}
  \input{#1}
}
%    \end{macrocode}

% \macro{\childdocby}
% The command |\childdocby| ....
%    \begin{macrocode}
\newcommand{\childdocby}[2][]
{
  \childdocdisable
  \childdoctrue
  \childdocmanualtrue
  \if?#1?\else
    \def\jobname{#2}
  \fi
  \def\childdocjob{#2}
  \input{#2}
  \endinput
}
%    \end{macrocode}

% \macro{\childdocforward}
% The command |\childdocforward| redirects
% compilation to the main file or
% (if the optional argument is given) a child file.
% Parameters are set as if the main file
% or a child file starting with |\childdocof| was compiled.
% Then compilation is handed over to the main file:
%    \begin{macrocode}
\newcommand{\childdocforward}[2][]
{
  \begingroup
    \if?#1?
      \def\childdoctmp
      {
        \def\childdocname{#2}
        \def\childdocjob{#2}
        \def\jobname{#2}
        \input{#2}
        \endinput
      }
    \else
      \def\childdoctmp
      {
        \childdocdisable
        \def\childdocname{#2}
        \childdoctrue
        \includeonly{#2}
        \def\childdocjob{#1}
        \def\jobname{#1}
        \input{#1}
        \endinput
      }
    \fi
    \expandafter
  \endgroup
  \childdoctmp
}
%    \end{macrocode}

% \macro{\childdocforwardprefix}
% The command |\childdocforwardprefix| redirects
% compilation to the main or a child file by means of a pattern.
% The prefix |#1| in the current filename is replaced by |#2|
% and the suffix of the current filename is kept
% (it is assumed that the filename does not contain the substring `|~~~|'
% which is used as a delimiter).
% Compilation is handed over to the new file by |\childdocforward|:
%    \begin{macrocode}
\newcommand{\childdocforwardprefix}[3][]
{
  \begingroup
    \def\childdocextract #2##1~~~{\def\childdoctmp{\childdocforward[#1]{#3##1}}}
    \expandafter\childdocextract\childdocname~~~
    \expandafter
  \endgroup
  \childdoctmp
}
%    \end{macrocode}

% \macro{\childdoc}
% The deprecated macro |\childdoc| is a legacy version of |\childdocmain|:
%    \begin{macrocode}
\newcommand{\childdoc}{\childdocmain}
%    \end{macrocode}

% \macro{\childdocredirect}
% The deprecated macro |\childdocredirect| is a legacy version
% of |\childdocforward| and |\childdocforwardprefix|:
%    \begin{macrocode}
\newcommand{\childdocredirect}[2][]
{
  \begingroup
    \if?#1?
      \def\childdoctmp{\childdocforward{#2}}
    \else
      \def\childdoctmp{\childdocforwardprefix{#1}{#2}}
    \fi
    \expandafter
  \endgroup
  \childdoctmp
}
%    \end{macrocode}

%\iffalse
%</package>
%\fi
%
\endinput

\childdocforwardprefix[cdocsamp]{cdocsfn}{cdocsch}
%    \end{macrocode}

%\iffalse
%</samplefinal>
%\fi
%
% %%%%%%%%%%%%%%%%%%%%%%%%%%%%%%%%%%%%%%
% \paragraph{Command Line Processing.}
%
% The following three command lines generate the output files
% |cdocscld|, |cdocscl1| and |cdocscl2|
% which should be identical to
% |cdocsdrf|, |cdocsch1| and |cdocsfn2|, respectively:
% \begin{center}
% \begin{tabular}{l}
% |latex -jobname cdocscld \|\\
% |  "\def\version{draft}% \iffalse
%
% childdoc.dtx Copyright (C) 2017-2018 Niklas Beisert
%
% This work may be distributed and/or modified under the
% conditions of the LaTeX Project Public License, either version 1.3
% of this license or (at your option) any later version.
% The latest version of this license is in
%   http://www.latex-project.org/lppl.txt
% and version 1.3 or later is part of all distributions of LaTeX
% version 2005/12/01 or later.
%
% This work has the LPPL maintenance status `maintained'.
%
% The Current Maintainer of this work is Niklas Beisert.
%
% This work consists of the files childdoc.dtx and childdoc.ins
% and the derived files childdoc.def and cdocsamp.tex with
% cdocsch1.tex, cdocsch2.tex, cdocsdrf.tex, cdocsfn1.tex, cdocsfn2.tex.
%
%<package>\ifdefined\childdocmain\endinput\fi
%<package>\ProvidesFile{childdoc.def}[2018/12/30 v2.0 child document driver]
%<samplemain>\ProvidesFile{cdocsamp.tex}[2018/12/30 v2.0 sample for childdoc]
%<*driver>
%\ProvidesFile{childdoc.drv}[2018/12/30 v2.0 childdoc reference manual file]
\PassOptionsToClass{10pt,a4paper}{article}
\documentclass{ltxdoc}

\usepackage[margin=35mm]{geometry}
\usepackage{hyperref}
\usepackage{hyperxmp}
\usepackage[usenames]{color}

\hypersetup{colorlinks=true}
\hypersetup{pdfstartview=FitH}
\hypersetup{pdfpagemode=UseNone}
\hypersetup{pdfsource={}}
\hypersetup{pdflang={en-UK}}
\hypersetup{pdfcopyright={Copyright 2017-2018 Niklas Beisert.
  This work may be distributed and/or modified under the
  conditions of the LaTeX Project Public License, either version 1.3
  of this license or (at your option) any later version.}}
\hypersetup{pdflicenseurl={http://www.latex-project.org/lppl.txt}}
\hypersetup{pdfcontactaddress={ETH Zurich, ITP, HIT K,
  Wolfgang-Pauli-Strasse 27}}
\hypersetup{pdfcontactpostcode={8093}}
\hypersetup{pdfcontactcity={Zurich}}
\hypersetup{pdfcontactcountry={Switzerland}}
\hypersetup{pdfcontactemail={nbeisert@itp.phys.ethz.ch}}
\hypersetup{pdfcontacturl={http://people.phys.ethz.ch/\xmptilde nbeisert/}}

\newcommand{\secref}[1]{\hyperref[#1]{section \ref*{#1}}}

\parskip1ex
\parindent0pt
\let\olditemize\itemize
\def\itemize{\olditemize\parskip0pt}

\begin{document}

\title{The \textsf{childdoc} Package}
\hypersetup{pdftitle={The childdoc Package}}
\author{Niklas Beisert\\[2ex]
  Institut f\"ur Theoretische Physik\\
  Eidgen\"ossische Technische Hochschule Z\"urich\\
  Wolfgang-Pauli-Strasse 27, 8093 Z\"urich, Switzerland\\[1ex]
  \href{mailto:nbeisert@itp.phys.ethz.ch}
  {\texttt{nbeisert@itp.phys.ethz.ch}}}
\hypersetup{pdfauthor={Niklas Beisert}}
\hypersetup{pdfsubject={Manual for the LaTeX2e Package childdoc}}
\date{30 December 2018, \textsf{v2.0}}
\maketitle

\begin{abstract}\noindent
\textsf{childdoc} is a \LaTeXe{} package
that enables the direct compilation
of document sections included by |\include|
to individual files.
\end{abstract}

\begingroup
\parskip0ex
\tableofcontents
\endgroup

%%%%%%%%%%%%%%%%%%%%%%%%%%%%%%%%%%%%%%%%%%%%%%%%%%%%%%%%%%%%%%%%%%%%%%%%%%%%%%%%
%%%%%%%%%%%%%%%%%%%%%%%%%%%%%%%%%%%%%%%%%%%%%%%%%%%%%%%%%%%%%%%%%%%%%%%%%%%%%%%%
\section{Introduction}

\LaTeX{} provides a mechanism to structure a large document (such as a book)
into a main file and several child files (containing the chapters)
using the |\include| command.
This mechanism is beneficial for documents
which span hundreds of pages in order to
make the source file(s) more manageable.
Moreover, compilation can be restricted to
selected child files by means of the |\includeonly| command.
The latter feature can be used to reduce the compilation time while editing
(this was significantly more useful in the earlier days of \LaTeX{})
or to generate a smaller document which is easier to navigate.
Another application of |\includeonly| is to generate
documents consisting of selected parts of the complete document.

However, there are a few drawbacks of the plain |\include| mechanism:
\begin{itemize}
\item
The child files cannot be compiled on their own,
they can only be compiled via the main file.
A naive editing environment
(such as a text editor with an option
to have the current file processed by \LaTeX)
may require one to switch to the main file before compiling;
attempting to compile the child file produces errors.
\item
The main file must be modified (each time)
to adjust the |\includeonly| command
to the present needs. This easily leaves the main file in a messy state.
\item
The generated document will always carry the filename
of the main document. This is inconvenient if
several child files are to be compiled and
to be kept for distribution.
\end{itemize}

The present package provides a simple interface
to make child files individually compilable by \LaTeX{}.
Compiling a child file then has the same effect as compiling
the main file with an |\includeonly| command
to select the appropriate child.
Moreover the generated document will carry the name of the child
rather than the main file.
This resolves all three above issues.

This feature is meant to make the editing of books,
thesis documents and lecture notes somewhat more convenient.
However, the package can also be used efficiently for
composing a series of documents (such as exercise sheets)
which are typically distributed individually.
It then assists the author in generating the individual documents
(potentially in different versions)
as well as a document containing the collected series.
Another application is in developing style files
or other kinds of included material
where compilation of the style file could redirect
to a sample or test file.

%%%%%%%%%%%%%%%%%%%%%%%%%%%%%%%%%%%%%%%%%%%%%%%%%%%%%%%%%%%%%%%%%%%%%%%%%%%%%%%%
%%%%%%%%%%%%%%%%%%%%%%%%%%%%%%%%%%%%%%%%%%%%%%%%%%%%%%%%%%%%%%%%%%%%%%%%%%%%%%%%
\section{Usage}

First of all, the package \textsf{childdoc} is \emph{not} a standard
\LaTeXe{} |.sty| style file! Therefore it needs to be invoked in
a non-standard way.

%%%%%%%%%%%%%%%%%%%%%%%%%%%%%%%%%%%%%%%%%%%%%%%%%%%%%%%%%%%%%%%%%%%%%%%%%%%%%%%%
\subsection{Included Files}
\label{sec:include}

%%%%%%%%%%%%%%%%%%%%%%%%%%%%%%%%%%%%%%%%
\DescribeMacro{\childdocmain}
To use the package, add the commands
\begin{center}
\begin{tabular}{l}
|\input{childdoc.def}|\\
|\childdocmain{}|\\
\end{tabular}
\end{center}
at the very top of the main \LaTeX{} file,
in particular \emph{before} the |\documentclass| statement!
The argument of |\childdocmain| should be left empty
(but it must be present).

%%%%%%%%%%%%%%%%%%%%%%%%%%%%%%%%%%%%%%%%
\DescribeMacro{\childdocof}
Furthermore, add the commands
\begin{center}
\begin{tabular}{l}
|\input{childdoc.def}|\\
|\childdocof{|\textit{main}|}|\\
\end{tabular}
\end{center}
at the top of every child file \textit{child}
which is included by |\include{|\textit{child}|}|
from within the main file
(or at least for those files to be compiled individually).
The argument \textit{main} must be the filename of the main file.

There are a couple of
considerations in setting up the main and child documents:

%%%%%%%%%%%%%%%%%%%%%%%%%%%%%%%%%%%%%%%%
\paragraph{Restrictions.}

Please note the following restrictions:
\begin{itemize}
\item
|\childdocmain| must be called with one argument \textit{main}
to ensure compatibility with earlier version of the package.
It must either be empty (|\childdocmain{}|)
or precisely match the filename of the main file in which it is specified.
See \secref{sec:detection} for further information.
\item
The filename \textit{main} must be specified without the |.tex| extension.
\item
The filename \textit{main} is case sensitive
(even in case-insensitive file systems)
due to internal string comparison.
\item
The argument \textit{main} should be fully expanded, it cannot be a macro.
\item
Subdirectories and special characters should be avoided in filenames.
\item
The command |\childdocmain{|\textit{main}|}| must be followed by a whitespace.
It should not be followed immediately by another command
or by a comment mark `|%|'.
This is because the \TeX{} parser reads the token immediately following
the argument of |\childdocmain| and puts it
at the beginning of every child section;
however, a white\-space is ignored.
\end{itemize}

%%%%%%%%%%%%%%%%%%%%%%%%%%%%%%%%%%%%%%%%
\paragraph{Content of Main File.}

It is advisable to place all content in the child files included by |\include|.
Any output contained in the main file will appear in all child documents
unless suppressed manually;
it cannot be suppressed automatically by the |\includeonly| directive
and thus should normally be avoided.
A method to include some content in the main file
by means of conditional processing is described in \secref{sec:conditional}.

%%%%%%%%%%%%%%%%%%%%%%%%%%%%%%%%%%%%%%%%
\paragraph{Page Numbering.}

When only a part of the document is compiled,
the appropriate numbering of pages
(as well as other status parameters)
is determined from the |.aux| files.
The latter contain information from previous passes.
However this information needs to propagate through
all intermediate child documents.
Therefore the page numbering in child documents may well
be inconsistent until the complete document is compiled at least once.

A useful (if unconventional) way to always ensure a consistent
page numbering is to restart the numbering in each child document
and denote the pages by `\textit{child}|.|\textit{page}'
where \textit{child} represents the chapter/section number of the child file.
This can be achieved by the command
|\numberwithin{page}{|\textit{child}|}|
of the \textsf{amsmath} package
where \textit{child} can be |chapter| or |section|
depending on the chosen structuring.
Alternatively, one can modify the macro |\thepage| appropriately
and reset the counter |page| at the start of each child file.

%%%%%%%%%%%%%%%%%%%%%%%%%%%%%%%%%%%%%%%%%%%%%%%%%%%%%%%%%%%%%%%%%%%%%%%%%%%%%%%%
\subsection{Conditional Processing}
\label{sec:conditional}

The package provides a mechanism to compile different versions
of a document. To customise the versions further some conditional processing
can come in handy to distinguish which version is being compiled.
The package provides two macros to describe the compilation context:

%%%%%%%%%%%%%%%%%%%%%%%%%%%%%%%%%%%%%%%%
\DescribeMacro{\ifchilddoc}
The conditional |\ifchilddoc| distinguishes between the compilation of
child documents and the main document:
%
\begin{center}
|\ifchilddoc |\textit{child-code}| |[|\||else |\textit{main-code}]| \||fi|
\end{center}

%%%%%%%%%%%%%%%%%%%%%%%%%%%%%%%%%%%%%%%%
\DescribeMacro{\childdocname}
\DescribeMacro{\childdocjob}
The macro |\childdocname| contains the filename (without extension)
of the main or child file being processed.
Note that |\childdocjob| will always contain the name of the main file.

%%%%%%%%%%%%%%%%%%%%%%%%%%%%%%%%%%%%%%%%
\paragraph{Title Page.}

Conditional processing can be used to include a title or banner page
in the main document when proper precautions are taken.
Importantly, the code in the main file should ensure that the page counter
(as well as other status parameters which are stored in the |.aux| files)
takes the same value after the conditional processing.
Otherwise the page numbers may take divergent values
depending on which part is compiled.

For example, a title page could be declared by:
%
\begin{center}
\begin{tabular}{l}
|\ifchilddoc\||else|\\
|\addtocounter{page}{-1}|\\
\textit{code for title page}\\
|\newpage|\\
|\||fi|
\end{tabular}
\end{center}
%
A banner page for the child documents can be generated by:
%
\begin{center}
\begin{tabular}{l}
|\ifchilddoc|\\
|\addtocounter{page}{-1}|\\
\textit{code for banner page}\\
|\newpage|\\
|\||fi|
\end{tabular}
\end{center}
%
Here one could write a message such as:
\begin{center}
|This is the part \childdocname{} of \childdocjob{}.|
\end{center}

%%%%%%%%%%%%%%%%%%%%%%%%%%%%%%%%%%%%%%%%%%%%%%%%%%%%%%%%%%%%%%%%%%%%%%%%%%%%%%%%
\subsection{Flags}
\label{sec:flags}

The package makes it easy to generate different versions
of the main or child documents.
To this end compilation flags can be defined
and assigned different default values.
They will be particularly useful in conjunction
with the forwarding mechanism described in \secref{sec:forward}.

For example, it may be useful to have a flag |\version|
which can be set to |draft| or |final|.
The document source will contain some conditional code
depending on the value of |\version|.
Suppose further, the flag should default to |final| for the main file
and to |draft| for child files
which is a natural assignment for editing the document.
This is achieved by placing the following code
in the preamble of the main document
(below the |\childdocmain| directive):
%
\begin{center}
\begin{tabular}{l}
|\ifchilddoc|\\
|\providecommand{\version}{draft}|\\
|\||else|\\
|\providecommand{\version}{final}|\\
|\||fi|
\end{tabular}
\end{center}
%
The definition by |\providecommand| makes sure
that previous definitions are not overwritten.
Further statements |\providecommand{\version}{...}|
can thus be added before the above code to override it.

For the main file, one might add a line
(between |\childdocmain| and the above block)
%
\begin{center}
|%\ifchilddoc\||else\providecommand{\version}{draft}\||fi|
\end{center}
%
which can be uncommented to produce a draft version.
Likewise one can add a line to the very top of a child file
(above the |\childdocof{|\textit{main}|}| directive)
%
\begin{center}
|%\providecommand{\version}{final}|
\end{center}
%
which can be uncommented to produce the final version of this child document.

%%%%%%%%%%%%%%%%%%%%%%%%%%%%%%%%%%%%%%%%%%%%%%%%%%%%%%%%%%%%%%%%%%%%%%%%%%%%%%%%
\subsection{Forwarding}
\label{sec:forward}

Different versions of the main or child documents
using compilation flags as described in \secref{sec:flags}
can be (permanently) stored in different files
for convenient compilation, viewing and distribution.
To this end, the package defines a command
to pass on compilation to a different file:

%%%%%%%%%%%%%%%%%%%%%%%%%%%%%%%%%%%%%%%%
\DescribeMacro{\childdocforward}
The command |\childdocforward| redirects processing to
another source file:
%
\begin{center}
\begin{tabular}{l}
|\input{childdoc.def}|\\
|\childdocforward[|\textit{main}|]{|\textit{dest}|}|\\
\end{tabular}
\end{center}
%
The argument \textit{dest} is the destination file
(without extension).
It should be the main file or one of the child files.
Note that further \textsf{childdoc} directives
such as |\childdocof| and |\childdocforward|
in the indicated file will be processed in this form.
The optional argument \textit{main}
passes on directly to the main file \textit{main}
while pretending to compile the child \textit{dest}.
This form behaves as if \textit{dest}
issues |\childdocof{|\textit{main}|}| right away,
and no further \textsf{childdoc} directives will be processed.

%%%%%%%%%%%%%%%%%%%%%%%%%%%%%%%%%%%%%%%%
\DescribeMacro{\...prefix}
In the alternative form |\childdocforwardprefix|,
%
\begin{center}
\begin{tabular}{l}
|\input{childdoc.def}|\\
|\childdocforwardprefix[|\textit{main}|]{|\textit{prefix}|}{|\textit{dest}|}|
\end{tabular}
\end{center}
%
the destination file is determined by a pattern
depending on the current file:
To make this work, the current file must be called
`{\textit{prefix}\hspace{0.2em}\textit{suffix}}'
with \textit{prefix} matching precisely the argument.
Processing is then passed on to the file
`{\textit{dest}\hspace{0.2em}\textit{suffix}}'.
Surely, the same effect is achieved by
directly specifying the
argument `{\textit{dest}\hspace{0.2em}\textit{suffix}}'
in the first form.
However, that requires to set up a different file
for each child. With the alternative form of the command
all these files can have exactly the same content
which simplifies setting them up and maintaining them.

For example, the following file |draft.tex|
with a compilation flag |\version| as described in \secref{sec:flags}
compiles the main document as a draft:
%
\begin{center}
\begin{tabular}{l}
|\def\version{draft}|\\
|\input{childdoc.def}|\\
|\childdocforward{|\textit{main}|}|
\end{tabular}
\end{center}
%
Likewise, the following files |final|\textit{nn}|.tex|
compile the final version of the child document
|child|\textit{nn}|.tex|:
%
\begin{center}
\begin{tabular}{l}
|\def\version{final}|\\
|\input{childdoc.def}|\\
|\childdocforwardprefix{final}{child}|
\end{tabular}
\end{center}
%

Note that when several versions of a main file and/or of each child file
are to be generated, it may be convenient to set up a |Makefile| or
shell script to automatise the process.

%%%%%%%%%%%%%%%%%%%%%%%%%%%%%%%%%%%%%%%%%%%%%%%%%%%%%%%%%%%%%%%%%%%%%%%%%%%%%%%%
\subsection{Command Line Processing}
\label{sec:commandline}

The effect of redirection files can also be achieved by invoking
the \LaTeX{} compiler with a more elaborate command line.
Most conveniently this should be done as part
of a shell script or a |Makefile|.

When using \textsf{childdoc} in the main file, the following
command lines effectively perform a redirection
(note that depending on the shell being used,
backslashes may have to be doubled: `|\|' $\to$ `|\\|'):
%
\begin{center}
|... -jobname "|\textit{target}|" |\\|"|[\textit{flags}]%
|\input{childdoc.def}\childdocforward[|\textit{main}|]{|\textit{dest}|}"|
\end{center}
%
Here \textit{target} is the name of the output file,
\textit{main} is the name of the main file
and \textit{dest} is the name of the main or child file to be processed
(all filenames without extensions).
The optional argument \textit{main} can be omitted
if \textit{main} matches \textit{dest}.
Optionally, compilation \textit{flags} can be defined via |\def| commands.
This command line makes the \TeX{} engine believe
it is compiling the file \textit{target}
whose content is specified as the latter parameter.
The provided code then forwards the processing to
\textit{main} or \textit{dest} as described in \secref{sec:forward}.

%%%%%%%%%%%%%%%%%%%%%%%%%%%%%%%%%%%%%%%%%%%%%%%%%%%%%%%%%%%%%%%%%%%%%%%%%%%%%%%%
\subsection{Include by Input}
\label{sec:input}

Including child documents by |\include| has some restrictions by design.
Most notably, the content of a child document always occupies
its own set of pages; pages cannot be shared between child documents.
Usually, this behaviour makes perfect sense
because each child document contain an essential part of the document.
However, in some situations it may be desirable to compose
a document from a collection of parts
without having mandatory page breaks between then.
For this case, the package
provides a mechanism to include parts
by |\input| which can also be processed individually.
However, by construction this mechanism
requires manual handling of the content to be output.

%%%%%%%%%%%%%%%%%%%%%%%%%%%%%%%%%%%%%%%%
\DescribeMacro{\ifchilddocmanual}
The main file should be prepared as usual, see \secref{sec:include}.
However, the document body must make a distinction
between processing of an individual part and of the main document, e.g.:
%
\begin{center}
\begin{tabular}{l}
|\ifchilddocmanual|\\
|\input{\childdocname}|\\
|\||else|\\
\textit{document body with }|\input{|\textit{part}|}|\\
|\||fi|
\end{tabular}
\end{center}
%
The conditional |\ifchilddocmanual| is true whenever
a part to be included by |\input| is being compiled,
and the name of the part is stored in |\childdocname|.

%%%%%%%%%%%%%%%%%%%%%%%%%%%%%%%%%%%%%%%%
\DescribeMacro{\childdocby}
Each part to be included by |\input| should start with:
%
\begin{center}
\begin{tabular}{l}
|\input{childdoc.def}|\\
|\childdocby{|\textit{main}|}|\\
\end{tabular}
\end{center}
%
The directive |\childdocby| is similar to |\childdocof|
described in \secref{sec:include},
but the subsequent selection of content must be done manually.
To that end, both |\ifchilddoc| and |\ifchilddocmanual|
will be true upon processing of a part,
and the name of the part is stored in |\childdocname|.
Note that |\jobname| will be set to the filename of the current part
so that each part receives an individual |.aux| file
that does not interfere with the |.aux| file(s) of the main document.
This behaviour can be altered by the alternative form
|\childdocby[*]{|\textit{main}|}| (with a non-empty optional argument)
which uses the |.aux| file of the main document
by setting |\jobname| to \textit{main}.

%%%%%%%%%%%%%%%%%%%%%%%%%%%%%%%%%%%%%%%%%%%%%%%%%%%%%%%%%%%%%%%%%%%%%%%%%%%%%%%%
\subsection{Driver Development}
\label{sec:driver}

The \textsf{childdoc} mechanism can also be use for the development
of definition files such as \LaTeX{} styles or classes.
This case differs from the above setup with multiple parts
included by |\include| in that no |\includeonly| should be invoked.
This can be achieved by starting the include file
(before |\ProvidesPackage|) with:
%
\begin{center}
\begin{tabular}{l}
|\input{childdoc.def}|\\
|\childdocforward{|\textit{main}|}|\\
\end{tabular}
\end{center}
%
or alternatively with:
%
\begin{center}
\begin{tabular}{l}
|\input{childdoc.def}|\\
|\childdocby{|\textit{main}|}|\\
\end{tabular}
\end{center}
%
Both forms have slightly different effects as described above.
The main file is prepared as usual, see \secref{sec:include}.

%%%%%%%%%%%%%%%%%%%%%%%%%%%%%%%%%%%%%%%%%%%%%%%%%%%%%%%%%%%%%%%%%%%%%%%%%%%%%%%%
\subsection{Legacy Detection}
\label{sec:detection}

The directive |\childdocmain| in the main file can detect
whether the complete document or merely a child is to be compiled
even without using the directive |\childdocof|.
This method is deprecated because it is less robust
and there is no compelling reason to use it;
it is merely provided for backward compatibility
and it may be removed in future versions.

If the detection mechanism is to be used,
it is mandatory to correctly specify
the filename of the main file as the argument of |\childdocmain|:
%
\begin{center}
\begin{tabular}{l}
|\input{childdoc.def}|\\
|\childdocmain{|\textit{main}|}|\\
\end{tabular}
\end{center}
%
If |\jobname| does not match the argument \textit{main} of |\childdocmain|,
it is assumed that |\jobname| points to the child file to be compiled.
When using |\childdocmain| with the main file specified as argument,
it suffices to start a child file
with just |\input{|\textit{main}|}|
without loading of the package and using |\childdocof|.
If instead all processing is done
with the appropriate \textsf{childdoc} directives,
the argument of \textit{main} of |\childdocmain| can be empty.

An alternative version of the command line processing described
in \secref{sec:commandline} using the detection mechanism reads:
%
\begin{center}
|... -jobname "|\textit{target}|" "|[\textit{flags}]%
[|\def\jobname{|\textit{dest}|}|]|\input{|\textit{main}|}"|
\end{center}

%%%%%%%%%%%%%%%%%%%%%%%%%%%%%%%%%%%%%%%%%%%%%%%%%%%%%%%%%%%%%%%%%%%%%%%%%%%%%%%%
\subsection{Manual Code}
\label{sec:manual}

In case one cannot be certain whether the definitions file |childdoc.def|
is installed on the target \TeX{} distribution
and one prefers not to ship it,
it is conceivable to paste a few relevant commands into the sources.

To that end, drop all statements |\input{childdoc.def}|
and perform the replacements as outlined below.
Instead of |\childdocmain{|\textit{main}|}| add the following code
to the top of the main file:
%
\begin{center}
\begin{tabular}{l}
|\||ifdefined\childdocname\endinput\||fi\newif\ifchilddoc|\\
|\edef\childdocname{\scantokens\expandafter{\jobname\noexpand}}|\\
|\def\childdocmain{|\textit{main}|}\||ifx\childdocmain\childdocname\||else|\\
|\childdoctrue\includeonly{\childdocname}\let\jobname\childdocmain\||fi|\\
\end{tabular}
\end{center}
%
Instead of |\childdocof{|\textit{main}|}| just include the main file
at the top of each child file:
%
\begin{center}
|\input{|\textit{main}|}|
\end{center}
%
A simple redirection |\childdocforward{|\textit{dest}|}| is achieved by:
%
\begin{center}
|\def\jobname{|\textit{dest}|}\input{\jobname}|
\end{center}
%
The redirection with prefix
|\childdocforwardprefix[|\textit{prefix}|]{|\textit{dest}|}|
is accomplished by:
%
\begin{center}
\begin{tabular}{l}
|{\edef\jobname{\scantokens\expandafter{\jobname\noexpand}}|\\
|\def\redirectjob |\textit{prefix}|#1~~~{\gdef\jobname{|\textit{dest}|#1}}|\\
|\expandafter\redirectjob\jobname~~~}\input{\jobname}|
\end{tabular}
\end{center}

In an alternative approach,
child documents can be compiled by a specific command line
without additional code or specific definitions:
%
\begin{center}
|... -jobname "|\textit{target}|" "|[\textit{flags}]%
|\includeonly{|\textit{dest}|}\input{|\textit{main}|}"|
\end{center}
%

%%%%%%%%%%%%%%%%%%%%%%%%%%%%%%%%%%%%%%%%%%%%%%%%%%%%%%%%%%%%%%%%%%%%%%%%%%%%%%%%
%%%%%%%%%%%%%%%%%%%%%%%%%%%%%%%%%%%%%%%%%%%%%%%%%%%%%%%%%%%%%%%%%%%%%%%%%%%%%%%%
\section{Information}

%%%%%%%%%%%%%%%%%%%%%%%%%%%%%%%%%%%%%%%%%%%%%%%%%%%%%%%%%%%%%%%%%%%%%%%%%%%%%%%%
\subsection{Copyright}

Copyright \copyright{} 2017--2018 Niklas Beisert

This work may be distributed and/or modified under the
conditions of the \LaTeX{} Project Public License, either version 1.3
of this license or (at your option) any later version.
The latest version of this license is in
  \url{http://www.latex-project.org/lppl.txt}
and version 1.3 or later is part of all distributions of \LaTeX{}
version 2005/12/01 or later.

This work has the LPPL maintenance status `maintained'.

The Current Maintainer of this work is Niklas Beisert.

This work consists of the files |README.txt|, |childdoc.ins| and |childdoc.dtx|
as well as the derived files |childdoc.def|, |cdocsamp.tex|
with |cdocsch1.tex|, |cdocsch2.tex|, |cdocspt3.tex|, |cdocspt4.tex|,
|cdocsdrf.tex|, |cdocsfn1.tex|, |cdocsfn2.tex|
as well as |childdoc.pdf|.

%%%%%%%%%%%%%%%%%%%%%%%%%%%%%%%%%%%%%%%%%%%%%%%%%%%%%%%%%%%%%%%%%%%%%%%%%%%%%%%%
\subsection{Files and Installation}

The package consists of the files:
%
\begin{center}
\begin{tabular}{ll}
    |README.txt|   & readme file \\
    |childdoc.ins| & installation file \\
    |childdoc.dtx| & source file \\
    |childdoc.def| & definition file \\
    |cdocsamp.tex| & sample main file \\
    |cdocsch1.tex| & sample include file \\
    |cdocsch2.tex| & sample include file \\
    |cdocspt3.tex| & sample part file \\
    |cdocspt4.tex| & sample part file \\
    |cdocsdrf.tex| & sample redirection file \\
    |cdocsfn1.tex| & sample redirection file \\
    |cdocsfn2.tex| & sample redirection file \\
    |childdoc.pdf| & manual
\end{tabular}
\end{center}
%
The distribution consists of the files
|README.txt|, |childdoc.ins| and |childdoc.dtx|.
%
\begin{itemize}
\item
Run (pdf)\LaTeX{} on |childdoc.dtx|
to compile the manual |childdoc.pdf| (this file).
\item
Run \LaTeX{} on |childdoc.ins| to create the definitions file |childdoc.def|
and the sample |cdocsamp.tex| with include files
|cdocsch1.tex|, |cdocsch2.tex|, |cdocspt3.tex|, |cdocspt4.tex|,
|cdocsdrf.tex|, |cdocsfn1.tex|, |cdocsfn2.tex|.
Then copy the file |childdoc.def| to an appropriate directory of your \LaTeX{}
distribution, e.g.\ \textit{texmf-root}|/tex/latex/childdoc|.
\end{itemize}

%%%%%%%%%%%%%%%%%%%%%%%%%%%%%%%%%%%%%%%%%%%%%%%%%%%%%%%%%%%%%%%%%%%%%%%%%%%%%%%%
\subsection{Related CTAN Packages}

There are several other packages which offer a similar functionality:
%
\begin{itemize}
\item
The packages
\href{http://ctan.org/pkg/docmute}{\textsf{docmute}},
\href{http://ctan.org/pkg/includex}{\textsf{includex}} and
\href{http://ctan.org/pkg/standalone}{\textsf{standalone}}
provide commands to include only the document body of
a child file thus allowing both files to be compiled individually.
\item
The packages \href{http://ctan.org/pkg/subdocs}{\textsf{subdocs}}
and \href{http://ctan.org/pkg/subfiles}{\textsf{subfiles}}
provide structures in which the main and child documents can be
encapsulated and allowing them to be compiled individually.
The inclusion mechanism is different from the conventional |\include|.
\item
The package \href{http://ctan.org/pkg/combine}{\textsf{combine}}
is an elaborate solution to combine several documents into one.
\end{itemize}
%
See also the CTAN topic \href{http://ctan.org/topic/subdocs}{\textsf{subdocs}}
for further related packages.
The present package differs from the above solutions in that
a document structure constructed with the conventional |\include| mechanism
just needs two extra commands at the top of every file
such that all constituent files can be compiled individually.

%%%%%%%%%%%%%%%%%%%%%%%%%%%%%%%%%%%%%%%%%%%%%%%%%%%%%%%%%%%%%%%%%%%%%%%%%%%%%%%%
%\subsection{Feature Suggestions}
%
%The following is a list of features which may be useful for future
%versions of this package:
%%
%\begin{itemize}
%\item
%\ldots
%\end{itemize}

%%%%%%%%%%%%%%%%%%%%%%%%%%%%%%%%%%%%%%%%%%%%%%%%%%%%%%%%%%%%%%%%%%%%%%%%%%%%%%%%
\subsection{Revision History}

%%%%%%%%%%%%%%%%%%%%%%%%%%%%%%%%%%%%%%%%
\paragraph{v2.0:} 2018/12/30

\begin{itemize}
\item
immediate forward processing
\item
added |\childdocby| mechanism
\item
manual restructured
\end{itemize}

%%%%%%%%%%%%%%%%%%%%%%%%%%%%%%%%%%%%%%%%
\paragraph{v1.6:} 2018/01/17

\begin{itemize}
\item
application for development of include files
\item
corrections to manual
\end{itemize}

%%%%%%%%%%%%%%%%%%%%%%%%%%%%%%%%%%%%%%%%
\paragraph{v1.5:} 2017/05/21

\begin{itemize}
\item
more complete structuring introduced
\item
|\childdocof| introduced
\item
|\childdoc| renamed to |\childdocmain|
\item
|\childredirect| renamed to |\childdocforward| and |\childdocforwardprefix|
and functionality expanded
\end{itemize}

%%%%%%%%%%%%%%%%%%%%%%%%%%%%%%%%%%%%%%%%
\paragraph{v1.0:} 2017/04/27

\begin{itemize}
\item
manual and install package
\item
first version published on CTAN
\end{itemize}

%%%%%%%%%%%%%%%%%%%%%%%%%%%%%%%%%%%%%%%%
\paragraph{v0.6:} 2017/04/26

\begin{itemize}
\item
redirection mechanism added
\end{itemize}

%%%%%%%%%%%%%%%%%%%%%%%%%%%%%%%%%%%%%%%%
\paragraph{v0.5:} 2017/04/26

\begin{itemize}
\item
functionality in definition file
\end{itemize}


%%%%%%%%%%%%%%%%%%%%%%%%%%%%%%%%%%%%%%%%%%%%%%%%%%%%%%%%%%%%%%%%%%%%%%%%%%%%%%%%
%%%%%%%%%%%%%%%%%%%%%%%%%%%%%%%%%%%%%%%%%%%%%%%%%%%%%%%%%%%%%%%%%%%%%%%%%%%%%%%%
%%%%%%%%%%%%%%%%%%%%%%%%%%%%%%%%%%%%%%%%%%%%%%%%%%%%%%%%%%%%%%%%%%%%%%%%%%%%%%%%
\appendix

\settowidth\MacroIndent{\rmfamily\scriptsize 000\ }

 \DocInput{childdoc.dtx}

\end{document}
%</driver>
% \fi
%
% %%%%%%%%%%%%%%%%%%%%%%%%%%%%%%%%%%%%%%%%%%%%%%%%%%%%%%%%%%%%%%%%%%%%%%%%%%%%%%
% %%%%%%%%%%%%%%%%%%%%%%%%%%%%%%%%%%%%%%%%%%%%%%%%%%%%%%%%%%%%%%%%%%%%%%%%%%%%%%
% \section{Sample}
%\iffalse
%<*samplemain>
%\fi
%
% The following presents a sample document
% with two chapters, two parts, a title page,
% a compile flag as well as three forwarding files to set the flag.
% It consists of eight |.tex| files:
% \begin{center}
% \begin{tabular}{ll}
% |cdocsamp.tex|&main file\\
% |cdocsch1.tex|&include file for chapter 1\\
% |cdocsch2.tex|&include file for chapter 2\\
% |cdocspt3.tex|&include file for part 3\\
% |cdocspt4.tex|&include file for part 4\\
% |cdocsdrf.tex|&forwarding file for main file in draft mode\\
% |cdocsfi1.tex|&forwarding file for final version of chapter 1\\
% |cdocsfi2.tex|&forwarding file for final version of chapter 2\\
% \end{tabular}
% \end{center}
% Each of the eight files can be compiled directly by the \LaTeX{} compiler.
%
% %%%%%%%%%%%%%%%%%%%%%%%%%%%%%%%%%%%%%%
% \paragraph{Main File.}
%
% The main file is called |cdocsamp.tex|.
%
% Load the \textsf{childdoc} definitions and
% declare the filename for the main document:
%    \begin{macrocode}
\input{childdoc.def}
\childdocmain{}
%    \end{macrocode}

% Optional override for |\version| flag:
%    \begin{macrocode}
%%\ifchilddoc\else\providecommand{\version}{draft}\fi
%    \end{macrocode}

% Define the default values for the |\version| flag
% (|final| for the main file and |draft| for childs):
%    \begin{macrocode}
\ifchilddoc
\providecommand{\version}{draft}
\else
\providecommand{\version}{final}
\fi
%    \end{macrocode}

% Load the standard document class:
%    \begin{macrocode}
\documentclass[12pt]{article}
%    \end{macrocode}

% Start the document body:
%    \begin{macrocode}
\begin{document}
%    \end{macrocode}

% Declare a title page.
% Print title, part of document being processed and version flag:
%    \begin{macrocode}
\addtocounter{page}{-1}
\begin{center}
{\LARGE\bfseries{}childdoc example\par}
\vspace{1cm}
\ifchilddoc
\ifchilddocmanual part\else chapter\fi:
`\childdocname' of `\childdocjob'\par
\else
main document: `\childdocjob'\par
\fi
version: \version\par
\end{center}
\newpage
%    \end{macrocode}

% Manually include selected file,
% otherwise process as usual:
%    \begin{macrocode}
\ifchilddocmanual
\section*{part `\childdocname'}
\input{\childdocname}
\else
%    \end{macrocode}

% Include the two chapters:
%    \begin{macrocode}
\include{cdocsch1}
\include{cdocsch2}
%    \end{macrocode}

% Include the two parts unless only chapters should be displayed:
%    \begin{macrocode}
\ifchilddoc\else
\section{part three}
\input{cdocspt3}
\section{part four}
\input{cdocspt4}
\fi
%    \end{macrocode}

% Process as usual until here:
%    \begin{macrocode}
\fi
%    \end{macrocode}

% End of document body:
%    \begin{macrocode}
\end{document}
%    \end{macrocode}
%\iffalse
%</samplemain>
%\fi
%
% %%%%%%%%%%%%%%%%%%%%%%%%%%%%%%%%%%%%%%
% \paragraph{Chapter Include Files.}
%
% The include files are called |cdocsch1.tex| and |cdocsch2.tex|.
%
%\iffalse
%<*samplechap1|samplechap2>
%\fi

% Optional override for |\version| flag:
%    \begin{macrocode}
%%\providecommand{\version}{final}
%    \end{macrocode}

% Include the main document:
%    \begin{macrocode}
\input{childdoc.def}
\childdocof{cdocsamp}
%    \end{macrocode}

%\iffalse
%</samplechap1|samplechap2>
%\fi
%
%\iffalse
%<*samplechap1>
%\fi
% Some text for chapter 1:
%    \begin{macrocode}
\section{one}
some text in chapter one
%    \end{macrocode}

%\iffalse
%</samplechap1>
%\fi
% Some text for chapter 2:
%\iffalse
%<*samplechap2>
%\fi
%    \begin{macrocode}
\section{two}
more text in chapter two
%    \end{macrocode}

%\iffalse
%</samplechap2>
%\fi
%
% %%%%%%%%%%%%%%%%%%%%%%%%%%%%%%%%%%%%%%
% \paragraph{Part Include Files.}
%
% The include files are called |cdocspt3.tex| and |cdocspt4.tex|.
%
%\iffalse
%<*samplepart3|samplepart4>
%\fi

% Optional override for |\version| flag:
%    \begin{macrocode}
%%\providecommand{\version}{final}
%    \end{macrocode}

% Include the main document:
%    \begin{macrocode}
\input{childdoc.def}
\childdocby{cdocsamp}
%    \end{macrocode}

%\iffalse
%</samplepart3|samplepart4>
%\fi
%
%\iffalse
%<*samplepart3>
%\fi
% Some text for part 3:
%    \begin{macrocode}
some text in part three
%    \end{macrocode}

%\iffalse
%</samplepart3>
%\fi
% Some text for part 4:
%\iffalse
%<*samplepart4>
%\fi
%    \begin{macrocode}
more text in part four
%    \end{macrocode}

%\iffalse
%</samplepart4>
%\fi
%
% %%%%%%%%%%%%%%%%%%%%%%%%%%%%%%%%%%%%%%
% \paragraph{Forwarding for a Complete Draft.}
%
% The following forwarding file |cdocsdrf.tex|
% compiles the main document in draft mode:
%\iffalse
%<*sampledraft>
%\fi
%    \begin{macrocode}
\def\version{draft}
\input{childdoc.def}
\childdocforward{cdocsamp}
%    \end{macrocode}

%\iffalse
%</sampledraft>
%\fi
%
% %%%%%%%%%%%%%%%%%%%%%%%%%%%%%%%%%%%%%%
% \paragraph{Forwarding for Final Version of the Chapters.}
%
% The following forwarding files |cdocsfn1.tex| and |cdocsfn2.tex|
% (with identical content)
% compile the final versions of the child documents
% |cdocsch1.tex| and |cdocsch2.tex|, respectively:
%\iffalse
%<*samplefinal>
%\fi
%    \begin{macrocode}
\def\version{final}
\input{childdoc.def}
\childdocforwardprefix[cdocsamp]{cdocsfn}{cdocsch}
%    \end{macrocode}

%\iffalse
%</samplefinal>
%\fi
%
% %%%%%%%%%%%%%%%%%%%%%%%%%%%%%%%%%%%%%%
% \paragraph{Command Line Processing.}
%
% The following three command lines generate the output files
% |cdocscld|, |cdocscl1| and |cdocscl2|
% which should be identical to
% |cdocsdrf|, |cdocsch1| and |cdocsfn2|, respectively:
% \begin{center}
% \begin{tabular}{l}
% |latex -jobname cdocscld \|\\
% |  "\def\version{draft}\input{childdoc.def}\childdocforward{cdocsamp}"|\\
% |latex -jobname cdocscl1 \|\\
% |  "\input{childdoc.def}\childdocforward[cdocsamp]{cdocsch1}"|\\
% |latex -jobname cdocscl2 \|\\
% |  "\def\version{final}\input{childdoc.def}\childdocforward{cdocsch2}"|
% \end{tabular}
% \end{center}
% Note that the trailing backslash on each first line
% merely continues the input to the second line
% (for convenient cut ant paste).
% Furthermore, the command |latex| can be replaced by any
% of its alternative versions such as |pdflatex|.
%
% %%%%%%%%%%%%%%%%%%%%%%%%%%%%%%%%%%%%%%%%%%%%%%%%%%%%%%%%%%%%%%%%%%%%%%%%%%%%%%
% %%%%%%%%%%%%%%%%%%%%%%%%%%%%%%%%%%%%%%%%%%%%%%%%%%%%%%%%%%%%%%%%%%%%%%%%%%%%%%
% \section{Implementation}
%\iffalse
%<*package>
%\fi
%
% This section describes the definitions file |childdoc.def|.

% The definitions cannot be loaded using |\usepackage| or |\RequirePackage|
% which has a mechanism to prevent loading a style file more than once.
% When loading the definitions by means of |\input|
% multiple instances have to be prevented manually:
%\iffalse
%This code needs to be before the `\ProvidesFile' directive
%which is defined at the beginning of this file.
%Therefore it is also placed there and commented out here.
%</package>
%<*discard>
%\fi
%    \begin{macrocode}
\ifdefined\childdocmain\endinput\fi
%    \end{macrocode}
%\iffalse
%</discard>
%<*package>
%\fi
%
% \macro{\ifchilddoc}
% \macro{\ifchilddocmanual}
% The conditional |\ifchilddoc| tells whether a
% child (true) or main (false) document is being compiled.
% The conditional |\ifchilddocmanual| tells whether
% the |\includeonly| mechanism is used (false) or
% the selection of child files must be performed manually (true).
% The definitions initialise to false:
%    \begin{macrocode}
\newif\ifchilddoc
\newif\ifchilddocmanual
%    \end{macrocode}

% \macro{\childdocname}
% \macro{\childdocjob}
% The macro |\childdocname| stores the name of the main document
% to be compiled. The macro |\childdocjob| stores the name of
% the document on which the \LaTeX{} compiler was originally invoked.
% The content of |\jobname| cannot be compared
% to filenames specified in the source due to different catcodes.
% The following code rescans |\jobname|, stores the result
% in |\childdocname| and saves a copy in |\childdocjob|:
%    \begin{macrocode}
\edef\childdocname{\scantokens\expandafter{\jobname\noexpand}}
\let\childdocjob\childdocname
%    \end{macrocode}

% \macro{\childdocdisable}
% The macro |\childdocdisable| prevents the main file
% from being processed more than once.
% At this stage, the main document command |\childdocmain|
% is assumed to be called once again where it should do nothing.
% Any subsequent call to it should prevent
% a secondary processing of the main document
% It overwrites the forwarding commands
% |\childdocof| and |\childdocforward|
% with empty macros to prevent further inclusions of the main document:
%    \begin{macrocode}
\newcommand{\childdocdisable}
{
  \renewcommand{\childdocmain}[1]{\renewcommand{\childdocmain}[1]{\endinput}}
  \renewcommand{\childdocof}[1]{}
  \renewcommand{\childdocby}[2][]{}
  \renewcommand{\childdocforward}[2][]{}
  \renewcommand{\childdocdisable}{}
}
%    \end{macrocode}

% \macro{\childdocmain}
% The macro |\childdocmain| is to be called at the top of the main file
% with nothing or the main filename (without extension) as argument.
% First, it breaks loops.
% If the argument is not empty and does not match |\childdocname|
% (which is set by the first inclusion of |childdoc.def|),
% |\ifchilddoc| is set to true, |\includeonly| is applied to the child file
% and |\jobname| is set to the main file
% (for proper handling of |.aux| files):
%    \begin{macrocode}
\newcommand{\childdocmain}[1]
{
  \childdocdisable\childdocmain{}
  \if?#1?\else
    \begingroup
      \def\childdoctmp{#1}
      \ifx\childdoctmp\childdocname
        \def\childdoctmp{}
      \else
        \def\childdoctmp
        {
          \childdoctrue
          \includeonly{\childdocname}
          \def\childdocjob{#1}
          \def\jobname{#1}
        }
      \fi
      \expandafter
    \endgroup
    \childdoctmp
  \fi
}
%    \end{macrocode}

% \macro{\childdocof}
% The command |\childdocof| redirects
% compilation to the main file |#1|.
%    \begin{macrocode}
\newcommand{\childdocof}[1]
{
  \childdocdisable
  \childdoctrue
  \includeonly{\childdocname}
  \def\jobname{#1}
  \def\childdocjob{#1}
  \input{#1}
}
%    \end{macrocode}

% \macro{\childdocby}
% The command |\childdocby| ....
%    \begin{macrocode}
\newcommand{\childdocby}[2][]
{
  \childdocdisable
  \childdoctrue
  \childdocmanualtrue
  \if?#1?\else
    \def\jobname{#2}
  \fi
  \def\childdocjob{#2}
  \input{#2}
  \endinput
}
%    \end{macrocode}

% \macro{\childdocforward}
% The command |\childdocforward| redirects
% compilation to the main file or
% (if the optional argument is given) a child file.
% Parameters are set as if the main file
% or a child file starting with |\childdocof| was compiled.
% Then compilation is handed over to the main file:
%    \begin{macrocode}
\newcommand{\childdocforward}[2][]
{
  \begingroup
    \if?#1?
      \def\childdoctmp
      {
        \def\childdocname{#2}
        \def\childdocjob{#2}
        \def\jobname{#2}
        \input{#2}
        \endinput
      }
    \else
      \def\childdoctmp
      {
        \childdocdisable
        \def\childdocname{#2}
        \childdoctrue
        \includeonly{#2}
        \def\childdocjob{#1}
        \def\jobname{#1}
        \input{#1}
        \endinput
      }
    \fi
    \expandafter
  \endgroup
  \childdoctmp
}
%    \end{macrocode}

% \macro{\childdocforwardprefix}
% The command |\childdocforwardprefix| redirects
% compilation to the main or a child file by means of a pattern.
% The prefix |#1| in the current filename is replaced by |#2|
% and the suffix of the current filename is kept
% (it is assumed that the filename does not contain the substring `|~~~|'
% which is used as a delimiter).
% Compilation is handed over to the new file by |\childdocforward|:
%    \begin{macrocode}
\newcommand{\childdocforwardprefix}[3][]
{
  \begingroup
    \def\childdocextract #2##1~~~{\def\childdoctmp{\childdocforward[#1]{#3##1}}}
    \expandafter\childdocextract\childdocname~~~
    \expandafter
  \endgroup
  \childdoctmp
}
%    \end{macrocode}

% \macro{\childdoc}
% The deprecated macro |\childdoc| is a legacy version of |\childdocmain|:
%    \begin{macrocode}
\newcommand{\childdoc}{\childdocmain}
%    \end{macrocode}

% \macro{\childdocredirect}
% The deprecated macro |\childdocredirect| is a legacy version
% of |\childdocforward| and |\childdocforwardprefix|:
%    \begin{macrocode}
\newcommand{\childdocredirect}[2][]
{
  \begingroup
    \if?#1?
      \def\childdoctmp{\childdocforward{#2}}
    \else
      \def\childdoctmp{\childdocforwardprefix{#1}{#2}}
    \fi
    \expandafter
  \endgroup
  \childdoctmp
}
%    \end{macrocode}

%\iffalse
%</package>
%\fi
%
\endinput
\childdocforward{cdocsamp}"|\\
% |latex -jobname cdocscl1 \|\\
% |  "% \iffalse
%
% childdoc.dtx Copyright (C) 2017-2018 Niklas Beisert
%
% This work may be distributed and/or modified under the
% conditions of the LaTeX Project Public License, either version 1.3
% of this license or (at your option) any later version.
% The latest version of this license is in
%   http://www.latex-project.org/lppl.txt
% and version 1.3 or later is part of all distributions of LaTeX
% version 2005/12/01 or later.
%
% This work has the LPPL maintenance status `maintained'.
%
% The Current Maintainer of this work is Niklas Beisert.
%
% This work consists of the files childdoc.dtx and childdoc.ins
% and the derived files childdoc.def and cdocsamp.tex with
% cdocsch1.tex, cdocsch2.tex, cdocsdrf.tex, cdocsfn1.tex, cdocsfn2.tex.
%
%<package>\ifdefined\childdocmain\endinput\fi
%<package>\ProvidesFile{childdoc.def}[2018/12/30 v2.0 child document driver]
%<samplemain>\ProvidesFile{cdocsamp.tex}[2018/12/30 v2.0 sample for childdoc]
%<*driver>
%\ProvidesFile{childdoc.drv}[2018/12/30 v2.0 childdoc reference manual file]
\PassOptionsToClass{10pt,a4paper}{article}
\documentclass{ltxdoc}

\usepackage[margin=35mm]{geometry}
\usepackage{hyperref}
\usepackage{hyperxmp}
\usepackage[usenames]{color}

\hypersetup{colorlinks=true}
\hypersetup{pdfstartview=FitH}
\hypersetup{pdfpagemode=UseNone}
\hypersetup{pdfsource={}}
\hypersetup{pdflang={en-UK}}
\hypersetup{pdfcopyright={Copyright 2017-2018 Niklas Beisert.
  This work may be distributed and/or modified under the
  conditions of the LaTeX Project Public License, either version 1.3
  of this license or (at your option) any later version.}}
\hypersetup{pdflicenseurl={http://www.latex-project.org/lppl.txt}}
\hypersetup{pdfcontactaddress={ETH Zurich, ITP, HIT K,
  Wolfgang-Pauli-Strasse 27}}
\hypersetup{pdfcontactpostcode={8093}}
\hypersetup{pdfcontactcity={Zurich}}
\hypersetup{pdfcontactcountry={Switzerland}}
\hypersetup{pdfcontactemail={nbeisert@itp.phys.ethz.ch}}
\hypersetup{pdfcontacturl={http://people.phys.ethz.ch/\xmptilde nbeisert/}}

\newcommand{\secref}[1]{\hyperref[#1]{section \ref*{#1}}}

\parskip1ex
\parindent0pt
\let\olditemize\itemize
\def\itemize{\olditemize\parskip0pt}

\begin{document}

\title{The \textsf{childdoc} Package}
\hypersetup{pdftitle={The childdoc Package}}
\author{Niklas Beisert\\[2ex]
  Institut f\"ur Theoretische Physik\\
  Eidgen\"ossische Technische Hochschule Z\"urich\\
  Wolfgang-Pauli-Strasse 27, 8093 Z\"urich, Switzerland\\[1ex]
  \href{mailto:nbeisert@itp.phys.ethz.ch}
  {\texttt{nbeisert@itp.phys.ethz.ch}}}
\hypersetup{pdfauthor={Niklas Beisert}}
\hypersetup{pdfsubject={Manual for the LaTeX2e Package childdoc}}
\date{30 December 2018, \textsf{v2.0}}
\maketitle

\begin{abstract}\noindent
\textsf{childdoc} is a \LaTeXe{} package
that enables the direct compilation
of document sections included by |\include|
to individual files.
\end{abstract}

\begingroup
\parskip0ex
\tableofcontents
\endgroup

%%%%%%%%%%%%%%%%%%%%%%%%%%%%%%%%%%%%%%%%%%%%%%%%%%%%%%%%%%%%%%%%%%%%%%%%%%%%%%%%
%%%%%%%%%%%%%%%%%%%%%%%%%%%%%%%%%%%%%%%%%%%%%%%%%%%%%%%%%%%%%%%%%%%%%%%%%%%%%%%%
\section{Introduction}

\LaTeX{} provides a mechanism to structure a large document (such as a book)
into a main file and several child files (containing the chapters)
using the |\include| command.
This mechanism is beneficial for documents
which span hundreds of pages in order to
make the source file(s) more manageable.
Moreover, compilation can be restricted to
selected child files by means of the |\includeonly| command.
The latter feature can be used to reduce the compilation time while editing
(this was significantly more useful in the earlier days of \LaTeX{})
or to generate a smaller document which is easier to navigate.
Another application of |\includeonly| is to generate
documents consisting of selected parts of the complete document.

However, there are a few drawbacks of the plain |\include| mechanism:
\begin{itemize}
\item
The child files cannot be compiled on their own,
they can only be compiled via the main file.
A naive editing environment
(such as a text editor with an option
to have the current file processed by \LaTeX)
may require one to switch to the main file before compiling;
attempting to compile the child file produces errors.
\item
The main file must be modified (each time)
to adjust the |\includeonly| command
to the present needs. This easily leaves the main file in a messy state.
\item
The generated document will always carry the filename
of the main document. This is inconvenient if
several child files are to be compiled and
to be kept for distribution.
\end{itemize}

The present package provides a simple interface
to make child files individually compilable by \LaTeX{}.
Compiling a child file then has the same effect as compiling
the main file with an |\includeonly| command
to select the appropriate child.
Moreover the generated document will carry the name of the child
rather than the main file.
This resolves all three above issues.

This feature is meant to make the editing of books,
thesis documents and lecture notes somewhat more convenient.
However, the package can also be used efficiently for
composing a series of documents (such as exercise sheets)
which are typically distributed individually.
It then assists the author in generating the individual documents
(potentially in different versions)
as well as a document containing the collected series.
Another application is in developing style files
or other kinds of included material
where compilation of the style file could redirect
to a sample or test file.

%%%%%%%%%%%%%%%%%%%%%%%%%%%%%%%%%%%%%%%%%%%%%%%%%%%%%%%%%%%%%%%%%%%%%%%%%%%%%%%%
%%%%%%%%%%%%%%%%%%%%%%%%%%%%%%%%%%%%%%%%%%%%%%%%%%%%%%%%%%%%%%%%%%%%%%%%%%%%%%%%
\section{Usage}

First of all, the package \textsf{childdoc} is \emph{not} a standard
\LaTeXe{} |.sty| style file! Therefore it needs to be invoked in
a non-standard way.

%%%%%%%%%%%%%%%%%%%%%%%%%%%%%%%%%%%%%%%%%%%%%%%%%%%%%%%%%%%%%%%%%%%%%%%%%%%%%%%%
\subsection{Included Files}
\label{sec:include}

%%%%%%%%%%%%%%%%%%%%%%%%%%%%%%%%%%%%%%%%
\DescribeMacro{\childdocmain}
To use the package, add the commands
\begin{center}
\begin{tabular}{l}
|\input{childdoc.def}|\\
|\childdocmain{}|\\
\end{tabular}
\end{center}
at the very top of the main \LaTeX{} file,
in particular \emph{before} the |\documentclass| statement!
The argument of |\childdocmain| should be left empty
(but it must be present).

%%%%%%%%%%%%%%%%%%%%%%%%%%%%%%%%%%%%%%%%
\DescribeMacro{\childdocof}
Furthermore, add the commands
\begin{center}
\begin{tabular}{l}
|\input{childdoc.def}|\\
|\childdocof{|\textit{main}|}|\\
\end{tabular}
\end{center}
at the top of every child file \textit{child}
which is included by |\include{|\textit{child}|}|
from within the main file
(or at least for those files to be compiled individually).
The argument \textit{main} must be the filename of the main file.

There are a couple of
considerations in setting up the main and child documents:

%%%%%%%%%%%%%%%%%%%%%%%%%%%%%%%%%%%%%%%%
\paragraph{Restrictions.}

Please note the following restrictions:
\begin{itemize}
\item
|\childdocmain| must be called with one argument \textit{main}
to ensure compatibility with earlier version of the package.
It must either be empty (|\childdocmain{}|)
or precisely match the filename of the main file in which it is specified.
See \secref{sec:detection} for further information.
\item
The filename \textit{main} must be specified without the |.tex| extension.
\item
The filename \textit{main} is case sensitive
(even in case-insensitive file systems)
due to internal string comparison.
\item
The argument \textit{main} should be fully expanded, it cannot be a macro.
\item
Subdirectories and special characters should be avoided in filenames.
\item
The command |\childdocmain{|\textit{main}|}| must be followed by a whitespace.
It should not be followed immediately by another command
or by a comment mark `|%|'.
This is because the \TeX{} parser reads the token immediately following
the argument of |\childdocmain| and puts it
at the beginning of every child section;
however, a white\-space is ignored.
\end{itemize}

%%%%%%%%%%%%%%%%%%%%%%%%%%%%%%%%%%%%%%%%
\paragraph{Content of Main File.}

It is advisable to place all content in the child files included by |\include|.
Any output contained in the main file will appear in all child documents
unless suppressed manually;
it cannot be suppressed automatically by the |\includeonly| directive
and thus should normally be avoided.
A method to include some content in the main file
by means of conditional processing is described in \secref{sec:conditional}.

%%%%%%%%%%%%%%%%%%%%%%%%%%%%%%%%%%%%%%%%
\paragraph{Page Numbering.}

When only a part of the document is compiled,
the appropriate numbering of pages
(as well as other status parameters)
is determined from the |.aux| files.
The latter contain information from previous passes.
However this information needs to propagate through
all intermediate child documents.
Therefore the page numbering in child documents may well
be inconsistent until the complete document is compiled at least once.

A useful (if unconventional) way to always ensure a consistent
page numbering is to restart the numbering in each child document
and denote the pages by `\textit{child}|.|\textit{page}'
where \textit{child} represents the chapter/section number of the child file.
This can be achieved by the command
|\numberwithin{page}{|\textit{child}|}|
of the \textsf{amsmath} package
where \textit{child} can be |chapter| or |section|
depending on the chosen structuring.
Alternatively, one can modify the macro |\thepage| appropriately
and reset the counter |page| at the start of each child file.

%%%%%%%%%%%%%%%%%%%%%%%%%%%%%%%%%%%%%%%%%%%%%%%%%%%%%%%%%%%%%%%%%%%%%%%%%%%%%%%%
\subsection{Conditional Processing}
\label{sec:conditional}

The package provides a mechanism to compile different versions
of a document. To customise the versions further some conditional processing
can come in handy to distinguish which version is being compiled.
The package provides two macros to describe the compilation context:

%%%%%%%%%%%%%%%%%%%%%%%%%%%%%%%%%%%%%%%%
\DescribeMacro{\ifchilddoc}
The conditional |\ifchilddoc| distinguishes between the compilation of
child documents and the main document:
%
\begin{center}
|\ifchilddoc |\textit{child-code}| |[|\||else |\textit{main-code}]| \||fi|
\end{center}

%%%%%%%%%%%%%%%%%%%%%%%%%%%%%%%%%%%%%%%%
\DescribeMacro{\childdocname}
\DescribeMacro{\childdocjob}
The macro |\childdocname| contains the filename (without extension)
of the main or child file being processed.
Note that |\childdocjob| will always contain the name of the main file.

%%%%%%%%%%%%%%%%%%%%%%%%%%%%%%%%%%%%%%%%
\paragraph{Title Page.}

Conditional processing can be used to include a title or banner page
in the main document when proper precautions are taken.
Importantly, the code in the main file should ensure that the page counter
(as well as other status parameters which are stored in the |.aux| files)
takes the same value after the conditional processing.
Otherwise the page numbers may take divergent values
depending on which part is compiled.

For example, a title page could be declared by:
%
\begin{center}
\begin{tabular}{l}
|\ifchilddoc\||else|\\
|\addtocounter{page}{-1}|\\
\textit{code for title page}\\
|\newpage|\\
|\||fi|
\end{tabular}
\end{center}
%
A banner page for the child documents can be generated by:
%
\begin{center}
\begin{tabular}{l}
|\ifchilddoc|\\
|\addtocounter{page}{-1}|\\
\textit{code for banner page}\\
|\newpage|\\
|\||fi|
\end{tabular}
\end{center}
%
Here one could write a message such as:
\begin{center}
|This is the part \childdocname{} of \childdocjob{}.|
\end{center}

%%%%%%%%%%%%%%%%%%%%%%%%%%%%%%%%%%%%%%%%%%%%%%%%%%%%%%%%%%%%%%%%%%%%%%%%%%%%%%%%
\subsection{Flags}
\label{sec:flags}

The package makes it easy to generate different versions
of the main or child documents.
To this end compilation flags can be defined
and assigned different default values.
They will be particularly useful in conjunction
with the forwarding mechanism described in \secref{sec:forward}.

For example, it may be useful to have a flag |\version|
which can be set to |draft| or |final|.
The document source will contain some conditional code
depending on the value of |\version|.
Suppose further, the flag should default to |final| for the main file
and to |draft| for child files
which is a natural assignment for editing the document.
This is achieved by placing the following code
in the preamble of the main document
(below the |\childdocmain| directive):
%
\begin{center}
\begin{tabular}{l}
|\ifchilddoc|\\
|\providecommand{\version}{draft}|\\
|\||else|\\
|\providecommand{\version}{final}|\\
|\||fi|
\end{tabular}
\end{center}
%
The definition by |\providecommand| makes sure
that previous definitions are not overwritten.
Further statements |\providecommand{\version}{...}|
can thus be added before the above code to override it.

For the main file, one might add a line
(between |\childdocmain| and the above block)
%
\begin{center}
|%\ifchilddoc\||else\providecommand{\version}{draft}\||fi|
\end{center}
%
which can be uncommented to produce a draft version.
Likewise one can add a line to the very top of a child file
(above the |\childdocof{|\textit{main}|}| directive)
%
\begin{center}
|%\providecommand{\version}{final}|
\end{center}
%
which can be uncommented to produce the final version of this child document.

%%%%%%%%%%%%%%%%%%%%%%%%%%%%%%%%%%%%%%%%%%%%%%%%%%%%%%%%%%%%%%%%%%%%%%%%%%%%%%%%
\subsection{Forwarding}
\label{sec:forward}

Different versions of the main or child documents
using compilation flags as described in \secref{sec:flags}
can be (permanently) stored in different files
for convenient compilation, viewing and distribution.
To this end, the package defines a command
to pass on compilation to a different file:

%%%%%%%%%%%%%%%%%%%%%%%%%%%%%%%%%%%%%%%%
\DescribeMacro{\childdocforward}
The command |\childdocforward| redirects processing to
another source file:
%
\begin{center}
\begin{tabular}{l}
|\input{childdoc.def}|\\
|\childdocforward[|\textit{main}|]{|\textit{dest}|}|\\
\end{tabular}
\end{center}
%
The argument \textit{dest} is the destination file
(without extension).
It should be the main file or one of the child files.
Note that further \textsf{childdoc} directives
such as |\childdocof| and |\childdocforward|
in the indicated file will be processed in this form.
The optional argument \textit{main}
passes on directly to the main file \textit{main}
while pretending to compile the child \textit{dest}.
This form behaves as if \textit{dest}
issues |\childdocof{|\textit{main}|}| right away,
and no further \textsf{childdoc} directives will be processed.

%%%%%%%%%%%%%%%%%%%%%%%%%%%%%%%%%%%%%%%%
\DescribeMacro{\...prefix}
In the alternative form |\childdocforwardprefix|,
%
\begin{center}
\begin{tabular}{l}
|\input{childdoc.def}|\\
|\childdocforwardprefix[|\textit{main}|]{|\textit{prefix}|}{|\textit{dest}|}|
\end{tabular}
\end{center}
%
the destination file is determined by a pattern
depending on the current file:
To make this work, the current file must be called
`{\textit{prefix}\hspace{0.2em}\textit{suffix}}'
with \textit{prefix} matching precisely the argument.
Processing is then passed on to the file
`{\textit{dest}\hspace{0.2em}\textit{suffix}}'.
Surely, the same effect is achieved by
directly specifying the
argument `{\textit{dest}\hspace{0.2em}\textit{suffix}}'
in the first form.
However, that requires to set up a different file
for each child. With the alternative form of the command
all these files can have exactly the same content
which simplifies setting them up and maintaining them.

For example, the following file |draft.tex|
with a compilation flag |\version| as described in \secref{sec:flags}
compiles the main document as a draft:
%
\begin{center}
\begin{tabular}{l}
|\def\version{draft}|\\
|\input{childdoc.def}|\\
|\childdocforward{|\textit{main}|}|
\end{tabular}
\end{center}
%
Likewise, the following files |final|\textit{nn}|.tex|
compile the final version of the child document
|child|\textit{nn}|.tex|:
%
\begin{center}
\begin{tabular}{l}
|\def\version{final}|\\
|\input{childdoc.def}|\\
|\childdocforwardprefix{final}{child}|
\end{tabular}
\end{center}
%

Note that when several versions of a main file and/or of each child file
are to be generated, it may be convenient to set up a |Makefile| or
shell script to automatise the process.

%%%%%%%%%%%%%%%%%%%%%%%%%%%%%%%%%%%%%%%%%%%%%%%%%%%%%%%%%%%%%%%%%%%%%%%%%%%%%%%%
\subsection{Command Line Processing}
\label{sec:commandline}

The effect of redirection files can also be achieved by invoking
the \LaTeX{} compiler with a more elaborate command line.
Most conveniently this should be done as part
of a shell script or a |Makefile|.

When using \textsf{childdoc} in the main file, the following
command lines effectively perform a redirection
(note that depending on the shell being used,
backslashes may have to be doubled: `|\|' $\to$ `|\\|'):
%
\begin{center}
|... -jobname "|\textit{target}|" |\\|"|[\textit{flags}]%
|\input{childdoc.def}\childdocforward[|\textit{main}|]{|\textit{dest}|}"|
\end{center}
%
Here \textit{target} is the name of the output file,
\textit{main} is the name of the main file
and \textit{dest} is the name of the main or child file to be processed
(all filenames without extensions).
The optional argument \textit{main} can be omitted
if \textit{main} matches \textit{dest}.
Optionally, compilation \textit{flags} can be defined via |\def| commands.
This command line makes the \TeX{} engine believe
it is compiling the file \textit{target}
whose content is specified as the latter parameter.
The provided code then forwards the processing to
\textit{main} or \textit{dest} as described in \secref{sec:forward}.

%%%%%%%%%%%%%%%%%%%%%%%%%%%%%%%%%%%%%%%%%%%%%%%%%%%%%%%%%%%%%%%%%%%%%%%%%%%%%%%%
\subsection{Include by Input}
\label{sec:input}

Including child documents by |\include| has some restrictions by design.
Most notably, the content of a child document always occupies
its own set of pages; pages cannot be shared between child documents.
Usually, this behaviour makes perfect sense
because each child document contain an essential part of the document.
However, in some situations it may be desirable to compose
a document from a collection of parts
without having mandatory page breaks between then.
For this case, the package
provides a mechanism to include parts
by |\input| which can also be processed individually.
However, by construction this mechanism
requires manual handling of the content to be output.

%%%%%%%%%%%%%%%%%%%%%%%%%%%%%%%%%%%%%%%%
\DescribeMacro{\ifchilddocmanual}
The main file should be prepared as usual, see \secref{sec:include}.
However, the document body must make a distinction
between processing of an individual part and of the main document, e.g.:
%
\begin{center}
\begin{tabular}{l}
|\ifchilddocmanual|\\
|\input{\childdocname}|\\
|\||else|\\
\textit{document body with }|\input{|\textit{part}|}|\\
|\||fi|
\end{tabular}
\end{center}
%
The conditional |\ifchilddocmanual| is true whenever
a part to be included by |\input| is being compiled,
and the name of the part is stored in |\childdocname|.

%%%%%%%%%%%%%%%%%%%%%%%%%%%%%%%%%%%%%%%%
\DescribeMacro{\childdocby}
Each part to be included by |\input| should start with:
%
\begin{center}
\begin{tabular}{l}
|\input{childdoc.def}|\\
|\childdocby{|\textit{main}|}|\\
\end{tabular}
\end{center}
%
The directive |\childdocby| is similar to |\childdocof|
described in \secref{sec:include},
but the subsequent selection of content must be done manually.
To that end, both |\ifchilddoc| and |\ifchilddocmanual|
will be true upon processing of a part,
and the name of the part is stored in |\childdocname|.
Note that |\jobname| will be set to the filename of the current part
so that each part receives an individual |.aux| file
that does not interfere with the |.aux| file(s) of the main document.
This behaviour can be altered by the alternative form
|\childdocby[*]{|\textit{main}|}| (with a non-empty optional argument)
which uses the |.aux| file of the main document
by setting |\jobname| to \textit{main}.

%%%%%%%%%%%%%%%%%%%%%%%%%%%%%%%%%%%%%%%%%%%%%%%%%%%%%%%%%%%%%%%%%%%%%%%%%%%%%%%%
\subsection{Driver Development}
\label{sec:driver}

The \textsf{childdoc} mechanism can also be use for the development
of definition files such as \LaTeX{} styles or classes.
This case differs from the above setup with multiple parts
included by |\include| in that no |\includeonly| should be invoked.
This can be achieved by starting the include file
(before |\ProvidesPackage|) with:
%
\begin{center}
\begin{tabular}{l}
|\input{childdoc.def}|\\
|\childdocforward{|\textit{main}|}|\\
\end{tabular}
\end{center}
%
or alternatively with:
%
\begin{center}
\begin{tabular}{l}
|\input{childdoc.def}|\\
|\childdocby{|\textit{main}|}|\\
\end{tabular}
\end{center}
%
Both forms have slightly different effects as described above.
The main file is prepared as usual, see \secref{sec:include}.

%%%%%%%%%%%%%%%%%%%%%%%%%%%%%%%%%%%%%%%%%%%%%%%%%%%%%%%%%%%%%%%%%%%%%%%%%%%%%%%%
\subsection{Legacy Detection}
\label{sec:detection}

The directive |\childdocmain| in the main file can detect
whether the complete document or merely a child is to be compiled
even without using the directive |\childdocof|.
This method is deprecated because it is less robust
and there is no compelling reason to use it;
it is merely provided for backward compatibility
and it may be removed in future versions.

If the detection mechanism is to be used,
it is mandatory to correctly specify
the filename of the main file as the argument of |\childdocmain|:
%
\begin{center}
\begin{tabular}{l}
|\input{childdoc.def}|\\
|\childdocmain{|\textit{main}|}|\\
\end{tabular}
\end{center}
%
If |\jobname| does not match the argument \textit{main} of |\childdocmain|,
it is assumed that |\jobname| points to the child file to be compiled.
When using |\childdocmain| with the main file specified as argument,
it suffices to start a child file
with just |\input{|\textit{main}|}|
without loading of the package and using |\childdocof|.
If instead all processing is done
with the appropriate \textsf{childdoc} directives,
the argument of \textit{main} of |\childdocmain| can be empty.

An alternative version of the command line processing described
in \secref{sec:commandline} using the detection mechanism reads:
%
\begin{center}
|... -jobname "|\textit{target}|" "|[\textit{flags}]%
[|\def\jobname{|\textit{dest}|}|]|\input{|\textit{main}|}"|
\end{center}

%%%%%%%%%%%%%%%%%%%%%%%%%%%%%%%%%%%%%%%%%%%%%%%%%%%%%%%%%%%%%%%%%%%%%%%%%%%%%%%%
\subsection{Manual Code}
\label{sec:manual}

In case one cannot be certain whether the definitions file |childdoc.def|
is installed on the target \TeX{} distribution
and one prefers not to ship it,
it is conceivable to paste a few relevant commands into the sources.

To that end, drop all statements |\input{childdoc.def}|
and perform the replacements as outlined below.
Instead of |\childdocmain{|\textit{main}|}| add the following code
to the top of the main file:
%
\begin{center}
\begin{tabular}{l}
|\||ifdefined\childdocname\endinput\||fi\newif\ifchilddoc|\\
|\edef\childdocname{\scantokens\expandafter{\jobname\noexpand}}|\\
|\def\childdocmain{|\textit{main}|}\||ifx\childdocmain\childdocname\||else|\\
|\childdoctrue\includeonly{\childdocname}\let\jobname\childdocmain\||fi|\\
\end{tabular}
\end{center}
%
Instead of |\childdocof{|\textit{main}|}| just include the main file
at the top of each child file:
%
\begin{center}
|\input{|\textit{main}|}|
\end{center}
%
A simple redirection |\childdocforward{|\textit{dest}|}| is achieved by:
%
\begin{center}
|\def\jobname{|\textit{dest}|}\input{\jobname}|
\end{center}
%
The redirection with prefix
|\childdocforwardprefix[|\textit{prefix}|]{|\textit{dest}|}|
is accomplished by:
%
\begin{center}
\begin{tabular}{l}
|{\edef\jobname{\scantokens\expandafter{\jobname\noexpand}}|\\
|\def\redirectjob |\textit{prefix}|#1~~~{\gdef\jobname{|\textit{dest}|#1}}|\\
|\expandafter\redirectjob\jobname~~~}\input{\jobname}|
\end{tabular}
\end{center}

In an alternative approach,
child documents can be compiled by a specific command line
without additional code or specific definitions:
%
\begin{center}
|... -jobname "|\textit{target}|" "|[\textit{flags}]%
|\includeonly{|\textit{dest}|}\input{|\textit{main}|}"|
\end{center}
%

%%%%%%%%%%%%%%%%%%%%%%%%%%%%%%%%%%%%%%%%%%%%%%%%%%%%%%%%%%%%%%%%%%%%%%%%%%%%%%%%
%%%%%%%%%%%%%%%%%%%%%%%%%%%%%%%%%%%%%%%%%%%%%%%%%%%%%%%%%%%%%%%%%%%%%%%%%%%%%%%%
\section{Information}

%%%%%%%%%%%%%%%%%%%%%%%%%%%%%%%%%%%%%%%%%%%%%%%%%%%%%%%%%%%%%%%%%%%%%%%%%%%%%%%%
\subsection{Copyright}

Copyright \copyright{} 2017--2018 Niklas Beisert

This work may be distributed and/or modified under the
conditions of the \LaTeX{} Project Public License, either version 1.3
of this license or (at your option) any later version.
The latest version of this license is in
  \url{http://www.latex-project.org/lppl.txt}
and version 1.3 or later is part of all distributions of \LaTeX{}
version 2005/12/01 or later.

This work has the LPPL maintenance status `maintained'.

The Current Maintainer of this work is Niklas Beisert.

This work consists of the files |README.txt|, |childdoc.ins| and |childdoc.dtx|
as well as the derived files |childdoc.def|, |cdocsamp.tex|
with |cdocsch1.tex|, |cdocsch2.tex|, |cdocspt3.tex|, |cdocspt4.tex|,
|cdocsdrf.tex|, |cdocsfn1.tex|, |cdocsfn2.tex|
as well as |childdoc.pdf|.

%%%%%%%%%%%%%%%%%%%%%%%%%%%%%%%%%%%%%%%%%%%%%%%%%%%%%%%%%%%%%%%%%%%%%%%%%%%%%%%%
\subsection{Files and Installation}

The package consists of the files:
%
\begin{center}
\begin{tabular}{ll}
    |README.txt|   & readme file \\
    |childdoc.ins| & installation file \\
    |childdoc.dtx| & source file \\
    |childdoc.def| & definition file \\
    |cdocsamp.tex| & sample main file \\
    |cdocsch1.tex| & sample include file \\
    |cdocsch2.tex| & sample include file \\
    |cdocspt3.tex| & sample part file \\
    |cdocspt4.tex| & sample part file \\
    |cdocsdrf.tex| & sample redirection file \\
    |cdocsfn1.tex| & sample redirection file \\
    |cdocsfn2.tex| & sample redirection file \\
    |childdoc.pdf| & manual
\end{tabular}
\end{center}
%
The distribution consists of the files
|README.txt|, |childdoc.ins| and |childdoc.dtx|.
%
\begin{itemize}
\item
Run (pdf)\LaTeX{} on |childdoc.dtx|
to compile the manual |childdoc.pdf| (this file).
\item
Run \LaTeX{} on |childdoc.ins| to create the definitions file |childdoc.def|
and the sample |cdocsamp.tex| with include files
|cdocsch1.tex|, |cdocsch2.tex|, |cdocspt3.tex|, |cdocspt4.tex|,
|cdocsdrf.tex|, |cdocsfn1.tex|, |cdocsfn2.tex|.
Then copy the file |childdoc.def| to an appropriate directory of your \LaTeX{}
distribution, e.g.\ \textit{texmf-root}|/tex/latex/childdoc|.
\end{itemize}

%%%%%%%%%%%%%%%%%%%%%%%%%%%%%%%%%%%%%%%%%%%%%%%%%%%%%%%%%%%%%%%%%%%%%%%%%%%%%%%%
\subsection{Related CTAN Packages}

There are several other packages which offer a similar functionality:
%
\begin{itemize}
\item
The packages
\href{http://ctan.org/pkg/docmute}{\textsf{docmute}},
\href{http://ctan.org/pkg/includex}{\textsf{includex}} and
\href{http://ctan.org/pkg/standalone}{\textsf{standalone}}
provide commands to include only the document body of
a child file thus allowing both files to be compiled individually.
\item
The packages \href{http://ctan.org/pkg/subdocs}{\textsf{subdocs}}
and \href{http://ctan.org/pkg/subfiles}{\textsf{subfiles}}
provide structures in which the main and child documents can be
encapsulated and allowing them to be compiled individually.
The inclusion mechanism is different from the conventional |\include|.
\item
The package \href{http://ctan.org/pkg/combine}{\textsf{combine}}
is an elaborate solution to combine several documents into one.
\end{itemize}
%
See also the CTAN topic \href{http://ctan.org/topic/subdocs}{\textsf{subdocs}}
for further related packages.
The present package differs from the above solutions in that
a document structure constructed with the conventional |\include| mechanism
just needs two extra commands at the top of every file
such that all constituent files can be compiled individually.

%%%%%%%%%%%%%%%%%%%%%%%%%%%%%%%%%%%%%%%%%%%%%%%%%%%%%%%%%%%%%%%%%%%%%%%%%%%%%%%%
%\subsection{Feature Suggestions}
%
%The following is a list of features which may be useful for future
%versions of this package:
%%
%\begin{itemize}
%\item
%\ldots
%\end{itemize}

%%%%%%%%%%%%%%%%%%%%%%%%%%%%%%%%%%%%%%%%%%%%%%%%%%%%%%%%%%%%%%%%%%%%%%%%%%%%%%%%
\subsection{Revision History}

%%%%%%%%%%%%%%%%%%%%%%%%%%%%%%%%%%%%%%%%
\paragraph{v2.0:} 2018/12/30

\begin{itemize}
\item
immediate forward processing
\item
added |\childdocby| mechanism
\item
manual restructured
\end{itemize}

%%%%%%%%%%%%%%%%%%%%%%%%%%%%%%%%%%%%%%%%
\paragraph{v1.6:} 2018/01/17

\begin{itemize}
\item
application for development of include files
\item
corrections to manual
\end{itemize}

%%%%%%%%%%%%%%%%%%%%%%%%%%%%%%%%%%%%%%%%
\paragraph{v1.5:} 2017/05/21

\begin{itemize}
\item
more complete structuring introduced
\item
|\childdocof| introduced
\item
|\childdoc| renamed to |\childdocmain|
\item
|\childredirect| renamed to |\childdocforward| and |\childdocforwardprefix|
and functionality expanded
\end{itemize}

%%%%%%%%%%%%%%%%%%%%%%%%%%%%%%%%%%%%%%%%
\paragraph{v1.0:} 2017/04/27

\begin{itemize}
\item
manual and install package
\item
first version published on CTAN
\end{itemize}

%%%%%%%%%%%%%%%%%%%%%%%%%%%%%%%%%%%%%%%%
\paragraph{v0.6:} 2017/04/26

\begin{itemize}
\item
redirection mechanism added
\end{itemize}

%%%%%%%%%%%%%%%%%%%%%%%%%%%%%%%%%%%%%%%%
\paragraph{v0.5:} 2017/04/26

\begin{itemize}
\item
functionality in definition file
\end{itemize}


%%%%%%%%%%%%%%%%%%%%%%%%%%%%%%%%%%%%%%%%%%%%%%%%%%%%%%%%%%%%%%%%%%%%%%%%%%%%%%%%
%%%%%%%%%%%%%%%%%%%%%%%%%%%%%%%%%%%%%%%%%%%%%%%%%%%%%%%%%%%%%%%%%%%%%%%%%%%%%%%%
%%%%%%%%%%%%%%%%%%%%%%%%%%%%%%%%%%%%%%%%%%%%%%%%%%%%%%%%%%%%%%%%%%%%%%%%%%%%%%%%
\appendix

\settowidth\MacroIndent{\rmfamily\scriptsize 000\ }

 \DocInput{childdoc.dtx}

\end{document}
%</driver>
% \fi
%
% %%%%%%%%%%%%%%%%%%%%%%%%%%%%%%%%%%%%%%%%%%%%%%%%%%%%%%%%%%%%%%%%%%%%%%%%%%%%%%
% %%%%%%%%%%%%%%%%%%%%%%%%%%%%%%%%%%%%%%%%%%%%%%%%%%%%%%%%%%%%%%%%%%%%%%%%%%%%%%
% \section{Sample}
%\iffalse
%<*samplemain>
%\fi
%
% The following presents a sample document
% with two chapters, two parts, a title page,
% a compile flag as well as three forwarding files to set the flag.
% It consists of eight |.tex| files:
% \begin{center}
% \begin{tabular}{ll}
% |cdocsamp.tex|&main file\\
% |cdocsch1.tex|&include file for chapter 1\\
% |cdocsch2.tex|&include file for chapter 2\\
% |cdocspt3.tex|&include file for part 3\\
% |cdocspt4.tex|&include file for part 4\\
% |cdocsdrf.tex|&forwarding file for main file in draft mode\\
% |cdocsfi1.tex|&forwarding file for final version of chapter 1\\
% |cdocsfi2.tex|&forwarding file for final version of chapter 2\\
% \end{tabular}
% \end{center}
% Each of the eight files can be compiled directly by the \LaTeX{} compiler.
%
% %%%%%%%%%%%%%%%%%%%%%%%%%%%%%%%%%%%%%%
% \paragraph{Main File.}
%
% The main file is called |cdocsamp.tex|.
%
% Load the \textsf{childdoc} definitions and
% declare the filename for the main document:
%    \begin{macrocode}
\input{childdoc.def}
\childdocmain{}
%    \end{macrocode}

% Optional override for |\version| flag:
%    \begin{macrocode}
%%\ifchilddoc\else\providecommand{\version}{draft}\fi
%    \end{macrocode}

% Define the default values for the |\version| flag
% (|final| for the main file and |draft| for childs):
%    \begin{macrocode}
\ifchilddoc
\providecommand{\version}{draft}
\else
\providecommand{\version}{final}
\fi
%    \end{macrocode}

% Load the standard document class:
%    \begin{macrocode}
\documentclass[12pt]{article}
%    \end{macrocode}

% Start the document body:
%    \begin{macrocode}
\begin{document}
%    \end{macrocode}

% Declare a title page.
% Print title, part of document being processed and version flag:
%    \begin{macrocode}
\addtocounter{page}{-1}
\begin{center}
{\LARGE\bfseries{}childdoc example\par}
\vspace{1cm}
\ifchilddoc
\ifchilddocmanual part\else chapter\fi:
`\childdocname' of `\childdocjob'\par
\else
main document: `\childdocjob'\par
\fi
version: \version\par
\end{center}
\newpage
%    \end{macrocode}

% Manually include selected file,
% otherwise process as usual:
%    \begin{macrocode}
\ifchilddocmanual
\section*{part `\childdocname'}
\input{\childdocname}
\else
%    \end{macrocode}

% Include the two chapters:
%    \begin{macrocode}
\include{cdocsch1}
\include{cdocsch2}
%    \end{macrocode}

% Include the two parts unless only chapters should be displayed:
%    \begin{macrocode}
\ifchilddoc\else
\section{part three}
\input{cdocspt3}
\section{part four}
\input{cdocspt4}
\fi
%    \end{macrocode}

% Process as usual until here:
%    \begin{macrocode}
\fi
%    \end{macrocode}

% End of document body:
%    \begin{macrocode}
\end{document}
%    \end{macrocode}
%\iffalse
%</samplemain>
%\fi
%
% %%%%%%%%%%%%%%%%%%%%%%%%%%%%%%%%%%%%%%
% \paragraph{Chapter Include Files.}
%
% The include files are called |cdocsch1.tex| and |cdocsch2.tex|.
%
%\iffalse
%<*samplechap1|samplechap2>
%\fi

% Optional override for |\version| flag:
%    \begin{macrocode}
%%\providecommand{\version}{final}
%    \end{macrocode}

% Include the main document:
%    \begin{macrocode}
\input{childdoc.def}
\childdocof{cdocsamp}
%    \end{macrocode}

%\iffalse
%</samplechap1|samplechap2>
%\fi
%
%\iffalse
%<*samplechap1>
%\fi
% Some text for chapter 1:
%    \begin{macrocode}
\section{one}
some text in chapter one
%    \end{macrocode}

%\iffalse
%</samplechap1>
%\fi
% Some text for chapter 2:
%\iffalse
%<*samplechap2>
%\fi
%    \begin{macrocode}
\section{two}
more text in chapter two
%    \end{macrocode}

%\iffalse
%</samplechap2>
%\fi
%
% %%%%%%%%%%%%%%%%%%%%%%%%%%%%%%%%%%%%%%
% \paragraph{Part Include Files.}
%
% The include files are called |cdocspt3.tex| and |cdocspt4.tex|.
%
%\iffalse
%<*samplepart3|samplepart4>
%\fi

% Optional override for |\version| flag:
%    \begin{macrocode}
%%\providecommand{\version}{final}
%    \end{macrocode}

% Include the main document:
%    \begin{macrocode}
\input{childdoc.def}
\childdocby{cdocsamp}
%    \end{macrocode}

%\iffalse
%</samplepart3|samplepart4>
%\fi
%
%\iffalse
%<*samplepart3>
%\fi
% Some text for part 3:
%    \begin{macrocode}
some text in part three
%    \end{macrocode}

%\iffalse
%</samplepart3>
%\fi
% Some text for part 4:
%\iffalse
%<*samplepart4>
%\fi
%    \begin{macrocode}
more text in part four
%    \end{macrocode}

%\iffalse
%</samplepart4>
%\fi
%
% %%%%%%%%%%%%%%%%%%%%%%%%%%%%%%%%%%%%%%
% \paragraph{Forwarding for a Complete Draft.}
%
% The following forwarding file |cdocsdrf.tex|
% compiles the main document in draft mode:
%\iffalse
%<*sampledraft>
%\fi
%    \begin{macrocode}
\def\version{draft}
\input{childdoc.def}
\childdocforward{cdocsamp}
%    \end{macrocode}

%\iffalse
%</sampledraft>
%\fi
%
% %%%%%%%%%%%%%%%%%%%%%%%%%%%%%%%%%%%%%%
% \paragraph{Forwarding for Final Version of the Chapters.}
%
% The following forwarding files |cdocsfn1.tex| and |cdocsfn2.tex|
% (with identical content)
% compile the final versions of the child documents
% |cdocsch1.tex| and |cdocsch2.tex|, respectively:
%\iffalse
%<*samplefinal>
%\fi
%    \begin{macrocode}
\def\version{final}
\input{childdoc.def}
\childdocforwardprefix[cdocsamp]{cdocsfn}{cdocsch}
%    \end{macrocode}

%\iffalse
%</samplefinal>
%\fi
%
% %%%%%%%%%%%%%%%%%%%%%%%%%%%%%%%%%%%%%%
% \paragraph{Command Line Processing.}
%
% The following three command lines generate the output files
% |cdocscld|, |cdocscl1| and |cdocscl2|
% which should be identical to
% |cdocsdrf|, |cdocsch1| and |cdocsfn2|, respectively:
% \begin{center}
% \begin{tabular}{l}
% |latex -jobname cdocscld \|\\
% |  "\def\version{draft}\input{childdoc.def}\childdocforward{cdocsamp}"|\\
% |latex -jobname cdocscl1 \|\\
% |  "\input{childdoc.def}\childdocforward[cdocsamp]{cdocsch1}"|\\
% |latex -jobname cdocscl2 \|\\
% |  "\def\version{final}\input{childdoc.def}\childdocforward{cdocsch2}"|
% \end{tabular}
% \end{center}
% Note that the trailing backslash on each first line
% merely continues the input to the second line
% (for convenient cut ant paste).
% Furthermore, the command |latex| can be replaced by any
% of its alternative versions such as |pdflatex|.
%
% %%%%%%%%%%%%%%%%%%%%%%%%%%%%%%%%%%%%%%%%%%%%%%%%%%%%%%%%%%%%%%%%%%%%%%%%%%%%%%
% %%%%%%%%%%%%%%%%%%%%%%%%%%%%%%%%%%%%%%%%%%%%%%%%%%%%%%%%%%%%%%%%%%%%%%%%%%%%%%
% \section{Implementation}
%\iffalse
%<*package>
%\fi
%
% This section describes the definitions file |childdoc.def|.

% The definitions cannot be loaded using |\usepackage| or |\RequirePackage|
% which has a mechanism to prevent loading a style file more than once.
% When loading the definitions by means of |\input|
% multiple instances have to be prevented manually:
%\iffalse
%This code needs to be before the `\ProvidesFile' directive
%which is defined at the beginning of this file.
%Therefore it is also placed there and commented out here.
%</package>
%<*discard>
%\fi
%    \begin{macrocode}
\ifdefined\childdocmain\endinput\fi
%    \end{macrocode}
%\iffalse
%</discard>
%<*package>
%\fi
%
% \macro{\ifchilddoc}
% \macro{\ifchilddocmanual}
% The conditional |\ifchilddoc| tells whether a
% child (true) or main (false) document is being compiled.
% The conditional |\ifchilddocmanual| tells whether
% the |\includeonly| mechanism is used (false) or
% the selection of child files must be performed manually (true).
% The definitions initialise to false:
%    \begin{macrocode}
\newif\ifchilddoc
\newif\ifchilddocmanual
%    \end{macrocode}

% \macro{\childdocname}
% \macro{\childdocjob}
% The macro |\childdocname| stores the name of the main document
% to be compiled. The macro |\childdocjob| stores the name of
% the document on which the \LaTeX{} compiler was originally invoked.
% The content of |\jobname| cannot be compared
% to filenames specified in the source due to different catcodes.
% The following code rescans |\jobname|, stores the result
% in |\childdocname| and saves a copy in |\childdocjob|:
%    \begin{macrocode}
\edef\childdocname{\scantokens\expandafter{\jobname\noexpand}}
\let\childdocjob\childdocname
%    \end{macrocode}

% \macro{\childdocdisable}
% The macro |\childdocdisable| prevents the main file
% from being processed more than once.
% At this stage, the main document command |\childdocmain|
% is assumed to be called once again where it should do nothing.
% Any subsequent call to it should prevent
% a secondary processing of the main document
% It overwrites the forwarding commands
% |\childdocof| and |\childdocforward|
% with empty macros to prevent further inclusions of the main document:
%    \begin{macrocode}
\newcommand{\childdocdisable}
{
  \renewcommand{\childdocmain}[1]{\renewcommand{\childdocmain}[1]{\endinput}}
  \renewcommand{\childdocof}[1]{}
  \renewcommand{\childdocby}[2][]{}
  \renewcommand{\childdocforward}[2][]{}
  \renewcommand{\childdocdisable}{}
}
%    \end{macrocode}

% \macro{\childdocmain}
% The macro |\childdocmain| is to be called at the top of the main file
% with nothing or the main filename (without extension) as argument.
% First, it breaks loops.
% If the argument is not empty and does not match |\childdocname|
% (which is set by the first inclusion of |childdoc.def|),
% |\ifchilddoc| is set to true, |\includeonly| is applied to the child file
% and |\jobname| is set to the main file
% (for proper handling of |.aux| files):
%    \begin{macrocode}
\newcommand{\childdocmain}[1]
{
  \childdocdisable\childdocmain{}
  \if?#1?\else
    \begingroup
      \def\childdoctmp{#1}
      \ifx\childdoctmp\childdocname
        \def\childdoctmp{}
      \else
        \def\childdoctmp
        {
          \childdoctrue
          \includeonly{\childdocname}
          \def\childdocjob{#1}
          \def\jobname{#1}
        }
      \fi
      \expandafter
    \endgroup
    \childdoctmp
  \fi
}
%    \end{macrocode}

% \macro{\childdocof}
% The command |\childdocof| redirects
% compilation to the main file |#1|.
%    \begin{macrocode}
\newcommand{\childdocof}[1]
{
  \childdocdisable
  \childdoctrue
  \includeonly{\childdocname}
  \def\jobname{#1}
  \def\childdocjob{#1}
  \input{#1}
}
%    \end{macrocode}

% \macro{\childdocby}
% The command |\childdocby| ....
%    \begin{macrocode}
\newcommand{\childdocby}[2][]
{
  \childdocdisable
  \childdoctrue
  \childdocmanualtrue
  \if?#1?\else
    \def\jobname{#2}
  \fi
  \def\childdocjob{#2}
  \input{#2}
  \endinput
}
%    \end{macrocode}

% \macro{\childdocforward}
% The command |\childdocforward| redirects
% compilation to the main file or
% (if the optional argument is given) a child file.
% Parameters are set as if the main file
% or a child file starting with |\childdocof| was compiled.
% Then compilation is handed over to the main file:
%    \begin{macrocode}
\newcommand{\childdocforward}[2][]
{
  \begingroup
    \if?#1?
      \def\childdoctmp
      {
        \def\childdocname{#2}
        \def\childdocjob{#2}
        \def\jobname{#2}
        \input{#2}
        \endinput
      }
    \else
      \def\childdoctmp
      {
        \childdocdisable
        \def\childdocname{#2}
        \childdoctrue
        \includeonly{#2}
        \def\childdocjob{#1}
        \def\jobname{#1}
        \input{#1}
        \endinput
      }
    \fi
    \expandafter
  \endgroup
  \childdoctmp
}
%    \end{macrocode}

% \macro{\childdocforwardprefix}
% The command |\childdocforwardprefix| redirects
% compilation to the main or a child file by means of a pattern.
% The prefix |#1| in the current filename is replaced by |#2|
% and the suffix of the current filename is kept
% (it is assumed that the filename does not contain the substring `|~~~|'
% which is used as a delimiter).
% Compilation is handed over to the new file by |\childdocforward|:
%    \begin{macrocode}
\newcommand{\childdocforwardprefix}[3][]
{
  \begingroup
    \def\childdocextract #2##1~~~{\def\childdoctmp{\childdocforward[#1]{#3##1}}}
    \expandafter\childdocextract\childdocname~~~
    \expandafter
  \endgroup
  \childdoctmp
}
%    \end{macrocode}

% \macro{\childdoc}
% The deprecated macro |\childdoc| is a legacy version of |\childdocmain|:
%    \begin{macrocode}
\newcommand{\childdoc}{\childdocmain}
%    \end{macrocode}

% \macro{\childdocredirect}
% The deprecated macro |\childdocredirect| is a legacy version
% of |\childdocforward| and |\childdocforwardprefix|:
%    \begin{macrocode}
\newcommand{\childdocredirect}[2][]
{
  \begingroup
    \if?#1?
      \def\childdoctmp{\childdocforward{#2}}
    \else
      \def\childdoctmp{\childdocforwardprefix{#1}{#2}}
    \fi
    \expandafter
  \endgroup
  \childdoctmp
}
%    \end{macrocode}

%\iffalse
%</package>
%\fi
%
\endinput
\childdocforward[cdocsamp]{cdocsch1}"|\\
% |latex -jobname cdocscl2 \|\\
% |  "\def\version{final}% \iffalse
%
% childdoc.dtx Copyright (C) 2017-2018 Niklas Beisert
%
% This work may be distributed and/or modified under the
% conditions of the LaTeX Project Public License, either version 1.3
% of this license or (at your option) any later version.
% The latest version of this license is in
%   http://www.latex-project.org/lppl.txt
% and version 1.3 or later is part of all distributions of LaTeX
% version 2005/12/01 or later.
%
% This work has the LPPL maintenance status `maintained'.
%
% The Current Maintainer of this work is Niklas Beisert.
%
% This work consists of the files childdoc.dtx and childdoc.ins
% and the derived files childdoc.def and cdocsamp.tex with
% cdocsch1.tex, cdocsch2.tex, cdocsdrf.tex, cdocsfn1.tex, cdocsfn2.tex.
%
%<package>\ifdefined\childdocmain\endinput\fi
%<package>\ProvidesFile{childdoc.def}[2018/12/30 v2.0 child document driver]
%<samplemain>\ProvidesFile{cdocsamp.tex}[2018/12/30 v2.0 sample for childdoc]
%<*driver>
%\ProvidesFile{childdoc.drv}[2018/12/30 v2.0 childdoc reference manual file]
\PassOptionsToClass{10pt,a4paper}{article}
\documentclass{ltxdoc}

\usepackage[margin=35mm]{geometry}
\usepackage{hyperref}
\usepackage{hyperxmp}
\usepackage[usenames]{color}

\hypersetup{colorlinks=true}
\hypersetup{pdfstartview=FitH}
\hypersetup{pdfpagemode=UseNone}
\hypersetup{pdfsource={}}
\hypersetup{pdflang={en-UK}}
\hypersetup{pdfcopyright={Copyright 2017-2018 Niklas Beisert.
  This work may be distributed and/or modified under the
  conditions of the LaTeX Project Public License, either version 1.3
  of this license or (at your option) any later version.}}
\hypersetup{pdflicenseurl={http://www.latex-project.org/lppl.txt}}
\hypersetup{pdfcontactaddress={ETH Zurich, ITP, HIT K,
  Wolfgang-Pauli-Strasse 27}}
\hypersetup{pdfcontactpostcode={8093}}
\hypersetup{pdfcontactcity={Zurich}}
\hypersetup{pdfcontactcountry={Switzerland}}
\hypersetup{pdfcontactemail={nbeisert@itp.phys.ethz.ch}}
\hypersetup{pdfcontacturl={http://people.phys.ethz.ch/\xmptilde nbeisert/}}

\newcommand{\secref}[1]{\hyperref[#1]{section \ref*{#1}}}

\parskip1ex
\parindent0pt
\let\olditemize\itemize
\def\itemize{\olditemize\parskip0pt}

\begin{document}

\title{The \textsf{childdoc} Package}
\hypersetup{pdftitle={The childdoc Package}}
\author{Niklas Beisert\\[2ex]
  Institut f\"ur Theoretische Physik\\
  Eidgen\"ossische Technische Hochschule Z\"urich\\
  Wolfgang-Pauli-Strasse 27, 8093 Z\"urich, Switzerland\\[1ex]
  \href{mailto:nbeisert@itp.phys.ethz.ch}
  {\texttt{nbeisert@itp.phys.ethz.ch}}}
\hypersetup{pdfauthor={Niklas Beisert}}
\hypersetup{pdfsubject={Manual for the LaTeX2e Package childdoc}}
\date{30 December 2018, \textsf{v2.0}}
\maketitle

\begin{abstract}\noindent
\textsf{childdoc} is a \LaTeXe{} package
that enables the direct compilation
of document sections included by |\include|
to individual files.
\end{abstract}

\begingroup
\parskip0ex
\tableofcontents
\endgroup

%%%%%%%%%%%%%%%%%%%%%%%%%%%%%%%%%%%%%%%%%%%%%%%%%%%%%%%%%%%%%%%%%%%%%%%%%%%%%%%%
%%%%%%%%%%%%%%%%%%%%%%%%%%%%%%%%%%%%%%%%%%%%%%%%%%%%%%%%%%%%%%%%%%%%%%%%%%%%%%%%
\section{Introduction}

\LaTeX{} provides a mechanism to structure a large document (such as a book)
into a main file and several child files (containing the chapters)
using the |\include| command.
This mechanism is beneficial for documents
which span hundreds of pages in order to
make the source file(s) more manageable.
Moreover, compilation can be restricted to
selected child files by means of the |\includeonly| command.
The latter feature can be used to reduce the compilation time while editing
(this was significantly more useful in the earlier days of \LaTeX{})
or to generate a smaller document which is easier to navigate.
Another application of |\includeonly| is to generate
documents consisting of selected parts of the complete document.

However, there are a few drawbacks of the plain |\include| mechanism:
\begin{itemize}
\item
The child files cannot be compiled on their own,
they can only be compiled via the main file.
A naive editing environment
(such as a text editor with an option
to have the current file processed by \LaTeX)
may require one to switch to the main file before compiling;
attempting to compile the child file produces errors.
\item
The main file must be modified (each time)
to adjust the |\includeonly| command
to the present needs. This easily leaves the main file in a messy state.
\item
The generated document will always carry the filename
of the main document. This is inconvenient if
several child files are to be compiled and
to be kept for distribution.
\end{itemize}

The present package provides a simple interface
to make child files individually compilable by \LaTeX{}.
Compiling a child file then has the same effect as compiling
the main file with an |\includeonly| command
to select the appropriate child.
Moreover the generated document will carry the name of the child
rather than the main file.
This resolves all three above issues.

This feature is meant to make the editing of books,
thesis documents and lecture notes somewhat more convenient.
However, the package can also be used efficiently for
composing a series of documents (such as exercise sheets)
which are typically distributed individually.
It then assists the author in generating the individual documents
(potentially in different versions)
as well as a document containing the collected series.
Another application is in developing style files
or other kinds of included material
where compilation of the style file could redirect
to a sample or test file.

%%%%%%%%%%%%%%%%%%%%%%%%%%%%%%%%%%%%%%%%%%%%%%%%%%%%%%%%%%%%%%%%%%%%%%%%%%%%%%%%
%%%%%%%%%%%%%%%%%%%%%%%%%%%%%%%%%%%%%%%%%%%%%%%%%%%%%%%%%%%%%%%%%%%%%%%%%%%%%%%%
\section{Usage}

First of all, the package \textsf{childdoc} is \emph{not} a standard
\LaTeXe{} |.sty| style file! Therefore it needs to be invoked in
a non-standard way.

%%%%%%%%%%%%%%%%%%%%%%%%%%%%%%%%%%%%%%%%%%%%%%%%%%%%%%%%%%%%%%%%%%%%%%%%%%%%%%%%
\subsection{Included Files}
\label{sec:include}

%%%%%%%%%%%%%%%%%%%%%%%%%%%%%%%%%%%%%%%%
\DescribeMacro{\childdocmain}
To use the package, add the commands
\begin{center}
\begin{tabular}{l}
|\input{childdoc.def}|\\
|\childdocmain{}|\\
\end{tabular}
\end{center}
at the very top of the main \LaTeX{} file,
in particular \emph{before} the |\documentclass| statement!
The argument of |\childdocmain| should be left empty
(but it must be present).

%%%%%%%%%%%%%%%%%%%%%%%%%%%%%%%%%%%%%%%%
\DescribeMacro{\childdocof}
Furthermore, add the commands
\begin{center}
\begin{tabular}{l}
|\input{childdoc.def}|\\
|\childdocof{|\textit{main}|}|\\
\end{tabular}
\end{center}
at the top of every child file \textit{child}
which is included by |\include{|\textit{child}|}|
from within the main file
(or at least for those files to be compiled individually).
The argument \textit{main} must be the filename of the main file.

There are a couple of
considerations in setting up the main and child documents:

%%%%%%%%%%%%%%%%%%%%%%%%%%%%%%%%%%%%%%%%
\paragraph{Restrictions.}

Please note the following restrictions:
\begin{itemize}
\item
|\childdocmain| must be called with one argument \textit{main}
to ensure compatibility with earlier version of the package.
It must either be empty (|\childdocmain{}|)
or precisely match the filename of the main file in which it is specified.
See \secref{sec:detection} for further information.
\item
The filename \textit{main} must be specified without the |.tex| extension.
\item
The filename \textit{main} is case sensitive
(even in case-insensitive file systems)
due to internal string comparison.
\item
The argument \textit{main} should be fully expanded, it cannot be a macro.
\item
Subdirectories and special characters should be avoided in filenames.
\item
The command |\childdocmain{|\textit{main}|}| must be followed by a whitespace.
It should not be followed immediately by another command
or by a comment mark `|%|'.
This is because the \TeX{} parser reads the token immediately following
the argument of |\childdocmain| and puts it
at the beginning of every child section;
however, a white\-space is ignored.
\end{itemize}

%%%%%%%%%%%%%%%%%%%%%%%%%%%%%%%%%%%%%%%%
\paragraph{Content of Main File.}

It is advisable to place all content in the child files included by |\include|.
Any output contained in the main file will appear in all child documents
unless suppressed manually;
it cannot be suppressed automatically by the |\includeonly| directive
and thus should normally be avoided.
A method to include some content in the main file
by means of conditional processing is described in \secref{sec:conditional}.

%%%%%%%%%%%%%%%%%%%%%%%%%%%%%%%%%%%%%%%%
\paragraph{Page Numbering.}

When only a part of the document is compiled,
the appropriate numbering of pages
(as well as other status parameters)
is determined from the |.aux| files.
The latter contain information from previous passes.
However this information needs to propagate through
all intermediate child documents.
Therefore the page numbering in child documents may well
be inconsistent until the complete document is compiled at least once.

A useful (if unconventional) way to always ensure a consistent
page numbering is to restart the numbering in each child document
and denote the pages by `\textit{child}|.|\textit{page}'
where \textit{child} represents the chapter/section number of the child file.
This can be achieved by the command
|\numberwithin{page}{|\textit{child}|}|
of the \textsf{amsmath} package
where \textit{child} can be |chapter| or |section|
depending on the chosen structuring.
Alternatively, one can modify the macro |\thepage| appropriately
and reset the counter |page| at the start of each child file.

%%%%%%%%%%%%%%%%%%%%%%%%%%%%%%%%%%%%%%%%%%%%%%%%%%%%%%%%%%%%%%%%%%%%%%%%%%%%%%%%
\subsection{Conditional Processing}
\label{sec:conditional}

The package provides a mechanism to compile different versions
of a document. To customise the versions further some conditional processing
can come in handy to distinguish which version is being compiled.
The package provides two macros to describe the compilation context:

%%%%%%%%%%%%%%%%%%%%%%%%%%%%%%%%%%%%%%%%
\DescribeMacro{\ifchilddoc}
The conditional |\ifchilddoc| distinguishes between the compilation of
child documents and the main document:
%
\begin{center}
|\ifchilddoc |\textit{child-code}| |[|\||else |\textit{main-code}]| \||fi|
\end{center}

%%%%%%%%%%%%%%%%%%%%%%%%%%%%%%%%%%%%%%%%
\DescribeMacro{\childdocname}
\DescribeMacro{\childdocjob}
The macro |\childdocname| contains the filename (without extension)
of the main or child file being processed.
Note that |\childdocjob| will always contain the name of the main file.

%%%%%%%%%%%%%%%%%%%%%%%%%%%%%%%%%%%%%%%%
\paragraph{Title Page.}

Conditional processing can be used to include a title or banner page
in the main document when proper precautions are taken.
Importantly, the code in the main file should ensure that the page counter
(as well as other status parameters which are stored in the |.aux| files)
takes the same value after the conditional processing.
Otherwise the page numbers may take divergent values
depending on which part is compiled.

For example, a title page could be declared by:
%
\begin{center}
\begin{tabular}{l}
|\ifchilddoc\||else|\\
|\addtocounter{page}{-1}|\\
\textit{code for title page}\\
|\newpage|\\
|\||fi|
\end{tabular}
\end{center}
%
A banner page for the child documents can be generated by:
%
\begin{center}
\begin{tabular}{l}
|\ifchilddoc|\\
|\addtocounter{page}{-1}|\\
\textit{code for banner page}\\
|\newpage|\\
|\||fi|
\end{tabular}
\end{center}
%
Here one could write a message such as:
\begin{center}
|This is the part \childdocname{} of \childdocjob{}.|
\end{center}

%%%%%%%%%%%%%%%%%%%%%%%%%%%%%%%%%%%%%%%%%%%%%%%%%%%%%%%%%%%%%%%%%%%%%%%%%%%%%%%%
\subsection{Flags}
\label{sec:flags}

The package makes it easy to generate different versions
of the main or child documents.
To this end compilation flags can be defined
and assigned different default values.
They will be particularly useful in conjunction
with the forwarding mechanism described in \secref{sec:forward}.

For example, it may be useful to have a flag |\version|
which can be set to |draft| or |final|.
The document source will contain some conditional code
depending on the value of |\version|.
Suppose further, the flag should default to |final| for the main file
and to |draft| for child files
which is a natural assignment for editing the document.
This is achieved by placing the following code
in the preamble of the main document
(below the |\childdocmain| directive):
%
\begin{center}
\begin{tabular}{l}
|\ifchilddoc|\\
|\providecommand{\version}{draft}|\\
|\||else|\\
|\providecommand{\version}{final}|\\
|\||fi|
\end{tabular}
\end{center}
%
The definition by |\providecommand| makes sure
that previous definitions are not overwritten.
Further statements |\providecommand{\version}{...}|
can thus be added before the above code to override it.

For the main file, one might add a line
(between |\childdocmain| and the above block)
%
\begin{center}
|%\ifchilddoc\||else\providecommand{\version}{draft}\||fi|
\end{center}
%
which can be uncommented to produce a draft version.
Likewise one can add a line to the very top of a child file
(above the |\childdocof{|\textit{main}|}| directive)
%
\begin{center}
|%\providecommand{\version}{final}|
\end{center}
%
which can be uncommented to produce the final version of this child document.

%%%%%%%%%%%%%%%%%%%%%%%%%%%%%%%%%%%%%%%%%%%%%%%%%%%%%%%%%%%%%%%%%%%%%%%%%%%%%%%%
\subsection{Forwarding}
\label{sec:forward}

Different versions of the main or child documents
using compilation flags as described in \secref{sec:flags}
can be (permanently) stored in different files
for convenient compilation, viewing and distribution.
To this end, the package defines a command
to pass on compilation to a different file:

%%%%%%%%%%%%%%%%%%%%%%%%%%%%%%%%%%%%%%%%
\DescribeMacro{\childdocforward}
The command |\childdocforward| redirects processing to
another source file:
%
\begin{center}
\begin{tabular}{l}
|\input{childdoc.def}|\\
|\childdocforward[|\textit{main}|]{|\textit{dest}|}|\\
\end{tabular}
\end{center}
%
The argument \textit{dest} is the destination file
(without extension).
It should be the main file or one of the child files.
Note that further \textsf{childdoc} directives
such as |\childdocof| and |\childdocforward|
in the indicated file will be processed in this form.
The optional argument \textit{main}
passes on directly to the main file \textit{main}
while pretending to compile the child \textit{dest}.
This form behaves as if \textit{dest}
issues |\childdocof{|\textit{main}|}| right away,
and no further \textsf{childdoc} directives will be processed.

%%%%%%%%%%%%%%%%%%%%%%%%%%%%%%%%%%%%%%%%
\DescribeMacro{\...prefix}
In the alternative form |\childdocforwardprefix|,
%
\begin{center}
\begin{tabular}{l}
|\input{childdoc.def}|\\
|\childdocforwardprefix[|\textit{main}|]{|\textit{prefix}|}{|\textit{dest}|}|
\end{tabular}
\end{center}
%
the destination file is determined by a pattern
depending on the current file:
To make this work, the current file must be called
`{\textit{prefix}\hspace{0.2em}\textit{suffix}}'
with \textit{prefix} matching precisely the argument.
Processing is then passed on to the file
`{\textit{dest}\hspace{0.2em}\textit{suffix}}'.
Surely, the same effect is achieved by
directly specifying the
argument `{\textit{dest}\hspace{0.2em}\textit{suffix}}'
in the first form.
However, that requires to set up a different file
for each child. With the alternative form of the command
all these files can have exactly the same content
which simplifies setting them up and maintaining them.

For example, the following file |draft.tex|
with a compilation flag |\version| as described in \secref{sec:flags}
compiles the main document as a draft:
%
\begin{center}
\begin{tabular}{l}
|\def\version{draft}|\\
|\input{childdoc.def}|\\
|\childdocforward{|\textit{main}|}|
\end{tabular}
\end{center}
%
Likewise, the following files |final|\textit{nn}|.tex|
compile the final version of the child document
|child|\textit{nn}|.tex|:
%
\begin{center}
\begin{tabular}{l}
|\def\version{final}|\\
|\input{childdoc.def}|\\
|\childdocforwardprefix{final}{child}|
\end{tabular}
\end{center}
%

Note that when several versions of a main file and/or of each child file
are to be generated, it may be convenient to set up a |Makefile| or
shell script to automatise the process.

%%%%%%%%%%%%%%%%%%%%%%%%%%%%%%%%%%%%%%%%%%%%%%%%%%%%%%%%%%%%%%%%%%%%%%%%%%%%%%%%
\subsection{Command Line Processing}
\label{sec:commandline}

The effect of redirection files can also be achieved by invoking
the \LaTeX{} compiler with a more elaborate command line.
Most conveniently this should be done as part
of a shell script or a |Makefile|.

When using \textsf{childdoc} in the main file, the following
command lines effectively perform a redirection
(note that depending on the shell being used,
backslashes may have to be doubled: `|\|' $\to$ `|\\|'):
%
\begin{center}
|... -jobname "|\textit{target}|" |\\|"|[\textit{flags}]%
|\input{childdoc.def}\childdocforward[|\textit{main}|]{|\textit{dest}|}"|
\end{center}
%
Here \textit{target} is the name of the output file,
\textit{main} is the name of the main file
and \textit{dest} is the name of the main or child file to be processed
(all filenames without extensions).
The optional argument \textit{main} can be omitted
if \textit{main} matches \textit{dest}.
Optionally, compilation \textit{flags} can be defined via |\def| commands.
This command line makes the \TeX{} engine believe
it is compiling the file \textit{target}
whose content is specified as the latter parameter.
The provided code then forwards the processing to
\textit{main} or \textit{dest} as described in \secref{sec:forward}.

%%%%%%%%%%%%%%%%%%%%%%%%%%%%%%%%%%%%%%%%%%%%%%%%%%%%%%%%%%%%%%%%%%%%%%%%%%%%%%%%
\subsection{Include by Input}
\label{sec:input}

Including child documents by |\include| has some restrictions by design.
Most notably, the content of a child document always occupies
its own set of pages; pages cannot be shared between child documents.
Usually, this behaviour makes perfect sense
because each child document contain an essential part of the document.
However, in some situations it may be desirable to compose
a document from a collection of parts
without having mandatory page breaks between then.
For this case, the package
provides a mechanism to include parts
by |\input| which can also be processed individually.
However, by construction this mechanism
requires manual handling of the content to be output.

%%%%%%%%%%%%%%%%%%%%%%%%%%%%%%%%%%%%%%%%
\DescribeMacro{\ifchilddocmanual}
The main file should be prepared as usual, see \secref{sec:include}.
However, the document body must make a distinction
between processing of an individual part and of the main document, e.g.:
%
\begin{center}
\begin{tabular}{l}
|\ifchilddocmanual|\\
|\input{\childdocname}|\\
|\||else|\\
\textit{document body with }|\input{|\textit{part}|}|\\
|\||fi|
\end{tabular}
\end{center}
%
The conditional |\ifchilddocmanual| is true whenever
a part to be included by |\input| is being compiled,
and the name of the part is stored in |\childdocname|.

%%%%%%%%%%%%%%%%%%%%%%%%%%%%%%%%%%%%%%%%
\DescribeMacro{\childdocby}
Each part to be included by |\input| should start with:
%
\begin{center}
\begin{tabular}{l}
|\input{childdoc.def}|\\
|\childdocby{|\textit{main}|}|\\
\end{tabular}
\end{center}
%
The directive |\childdocby| is similar to |\childdocof|
described in \secref{sec:include},
but the subsequent selection of content must be done manually.
To that end, both |\ifchilddoc| and |\ifchilddocmanual|
will be true upon processing of a part,
and the name of the part is stored in |\childdocname|.
Note that |\jobname| will be set to the filename of the current part
so that each part receives an individual |.aux| file
that does not interfere with the |.aux| file(s) of the main document.
This behaviour can be altered by the alternative form
|\childdocby[*]{|\textit{main}|}| (with a non-empty optional argument)
which uses the |.aux| file of the main document
by setting |\jobname| to \textit{main}.

%%%%%%%%%%%%%%%%%%%%%%%%%%%%%%%%%%%%%%%%%%%%%%%%%%%%%%%%%%%%%%%%%%%%%%%%%%%%%%%%
\subsection{Driver Development}
\label{sec:driver}

The \textsf{childdoc} mechanism can also be use for the development
of definition files such as \LaTeX{} styles or classes.
This case differs from the above setup with multiple parts
included by |\include| in that no |\includeonly| should be invoked.
This can be achieved by starting the include file
(before |\ProvidesPackage|) with:
%
\begin{center}
\begin{tabular}{l}
|\input{childdoc.def}|\\
|\childdocforward{|\textit{main}|}|\\
\end{tabular}
\end{center}
%
or alternatively with:
%
\begin{center}
\begin{tabular}{l}
|\input{childdoc.def}|\\
|\childdocby{|\textit{main}|}|\\
\end{tabular}
\end{center}
%
Both forms have slightly different effects as described above.
The main file is prepared as usual, see \secref{sec:include}.

%%%%%%%%%%%%%%%%%%%%%%%%%%%%%%%%%%%%%%%%%%%%%%%%%%%%%%%%%%%%%%%%%%%%%%%%%%%%%%%%
\subsection{Legacy Detection}
\label{sec:detection}

The directive |\childdocmain| in the main file can detect
whether the complete document or merely a child is to be compiled
even without using the directive |\childdocof|.
This method is deprecated because it is less robust
and there is no compelling reason to use it;
it is merely provided for backward compatibility
and it may be removed in future versions.

If the detection mechanism is to be used,
it is mandatory to correctly specify
the filename of the main file as the argument of |\childdocmain|:
%
\begin{center}
\begin{tabular}{l}
|\input{childdoc.def}|\\
|\childdocmain{|\textit{main}|}|\\
\end{tabular}
\end{center}
%
If |\jobname| does not match the argument \textit{main} of |\childdocmain|,
it is assumed that |\jobname| points to the child file to be compiled.
When using |\childdocmain| with the main file specified as argument,
it suffices to start a child file
with just |\input{|\textit{main}|}|
without loading of the package and using |\childdocof|.
If instead all processing is done
with the appropriate \textsf{childdoc} directives,
the argument of \textit{main} of |\childdocmain| can be empty.

An alternative version of the command line processing described
in \secref{sec:commandline} using the detection mechanism reads:
%
\begin{center}
|... -jobname "|\textit{target}|" "|[\textit{flags}]%
[|\def\jobname{|\textit{dest}|}|]|\input{|\textit{main}|}"|
\end{center}

%%%%%%%%%%%%%%%%%%%%%%%%%%%%%%%%%%%%%%%%%%%%%%%%%%%%%%%%%%%%%%%%%%%%%%%%%%%%%%%%
\subsection{Manual Code}
\label{sec:manual}

In case one cannot be certain whether the definitions file |childdoc.def|
is installed on the target \TeX{} distribution
and one prefers not to ship it,
it is conceivable to paste a few relevant commands into the sources.

To that end, drop all statements |\input{childdoc.def}|
and perform the replacements as outlined below.
Instead of |\childdocmain{|\textit{main}|}| add the following code
to the top of the main file:
%
\begin{center}
\begin{tabular}{l}
|\||ifdefined\childdocname\endinput\||fi\newif\ifchilddoc|\\
|\edef\childdocname{\scantokens\expandafter{\jobname\noexpand}}|\\
|\def\childdocmain{|\textit{main}|}\||ifx\childdocmain\childdocname\||else|\\
|\childdoctrue\includeonly{\childdocname}\let\jobname\childdocmain\||fi|\\
\end{tabular}
\end{center}
%
Instead of |\childdocof{|\textit{main}|}| just include the main file
at the top of each child file:
%
\begin{center}
|\input{|\textit{main}|}|
\end{center}
%
A simple redirection |\childdocforward{|\textit{dest}|}| is achieved by:
%
\begin{center}
|\def\jobname{|\textit{dest}|}\input{\jobname}|
\end{center}
%
The redirection with prefix
|\childdocforwardprefix[|\textit{prefix}|]{|\textit{dest}|}|
is accomplished by:
%
\begin{center}
\begin{tabular}{l}
|{\edef\jobname{\scantokens\expandafter{\jobname\noexpand}}|\\
|\def\redirectjob |\textit{prefix}|#1~~~{\gdef\jobname{|\textit{dest}|#1}}|\\
|\expandafter\redirectjob\jobname~~~}\input{\jobname}|
\end{tabular}
\end{center}

In an alternative approach,
child documents can be compiled by a specific command line
without additional code or specific definitions:
%
\begin{center}
|... -jobname "|\textit{target}|" "|[\textit{flags}]%
|\includeonly{|\textit{dest}|}\input{|\textit{main}|}"|
\end{center}
%

%%%%%%%%%%%%%%%%%%%%%%%%%%%%%%%%%%%%%%%%%%%%%%%%%%%%%%%%%%%%%%%%%%%%%%%%%%%%%%%%
%%%%%%%%%%%%%%%%%%%%%%%%%%%%%%%%%%%%%%%%%%%%%%%%%%%%%%%%%%%%%%%%%%%%%%%%%%%%%%%%
\section{Information}

%%%%%%%%%%%%%%%%%%%%%%%%%%%%%%%%%%%%%%%%%%%%%%%%%%%%%%%%%%%%%%%%%%%%%%%%%%%%%%%%
\subsection{Copyright}

Copyright \copyright{} 2017--2018 Niklas Beisert

This work may be distributed and/or modified under the
conditions of the \LaTeX{} Project Public License, either version 1.3
of this license or (at your option) any later version.
The latest version of this license is in
  \url{http://www.latex-project.org/lppl.txt}
and version 1.3 or later is part of all distributions of \LaTeX{}
version 2005/12/01 or later.

This work has the LPPL maintenance status `maintained'.

The Current Maintainer of this work is Niklas Beisert.

This work consists of the files |README.txt|, |childdoc.ins| and |childdoc.dtx|
as well as the derived files |childdoc.def|, |cdocsamp.tex|
with |cdocsch1.tex|, |cdocsch2.tex|, |cdocspt3.tex|, |cdocspt4.tex|,
|cdocsdrf.tex|, |cdocsfn1.tex|, |cdocsfn2.tex|
as well as |childdoc.pdf|.

%%%%%%%%%%%%%%%%%%%%%%%%%%%%%%%%%%%%%%%%%%%%%%%%%%%%%%%%%%%%%%%%%%%%%%%%%%%%%%%%
\subsection{Files and Installation}

The package consists of the files:
%
\begin{center}
\begin{tabular}{ll}
    |README.txt|   & readme file \\
    |childdoc.ins| & installation file \\
    |childdoc.dtx| & source file \\
    |childdoc.def| & definition file \\
    |cdocsamp.tex| & sample main file \\
    |cdocsch1.tex| & sample include file \\
    |cdocsch2.tex| & sample include file \\
    |cdocspt3.tex| & sample part file \\
    |cdocspt4.tex| & sample part file \\
    |cdocsdrf.tex| & sample redirection file \\
    |cdocsfn1.tex| & sample redirection file \\
    |cdocsfn2.tex| & sample redirection file \\
    |childdoc.pdf| & manual
\end{tabular}
\end{center}
%
The distribution consists of the files
|README.txt|, |childdoc.ins| and |childdoc.dtx|.
%
\begin{itemize}
\item
Run (pdf)\LaTeX{} on |childdoc.dtx|
to compile the manual |childdoc.pdf| (this file).
\item
Run \LaTeX{} on |childdoc.ins| to create the definitions file |childdoc.def|
and the sample |cdocsamp.tex| with include files
|cdocsch1.tex|, |cdocsch2.tex|, |cdocspt3.tex|, |cdocspt4.tex|,
|cdocsdrf.tex|, |cdocsfn1.tex|, |cdocsfn2.tex|.
Then copy the file |childdoc.def| to an appropriate directory of your \LaTeX{}
distribution, e.g.\ \textit{texmf-root}|/tex/latex/childdoc|.
\end{itemize}

%%%%%%%%%%%%%%%%%%%%%%%%%%%%%%%%%%%%%%%%%%%%%%%%%%%%%%%%%%%%%%%%%%%%%%%%%%%%%%%%
\subsection{Related CTAN Packages}

There are several other packages which offer a similar functionality:
%
\begin{itemize}
\item
The packages
\href{http://ctan.org/pkg/docmute}{\textsf{docmute}},
\href{http://ctan.org/pkg/includex}{\textsf{includex}} and
\href{http://ctan.org/pkg/standalone}{\textsf{standalone}}
provide commands to include only the document body of
a child file thus allowing both files to be compiled individually.
\item
The packages \href{http://ctan.org/pkg/subdocs}{\textsf{subdocs}}
and \href{http://ctan.org/pkg/subfiles}{\textsf{subfiles}}
provide structures in which the main and child documents can be
encapsulated and allowing them to be compiled individually.
The inclusion mechanism is different from the conventional |\include|.
\item
The package \href{http://ctan.org/pkg/combine}{\textsf{combine}}
is an elaborate solution to combine several documents into one.
\end{itemize}
%
See also the CTAN topic \href{http://ctan.org/topic/subdocs}{\textsf{subdocs}}
for further related packages.
The present package differs from the above solutions in that
a document structure constructed with the conventional |\include| mechanism
just needs two extra commands at the top of every file
such that all constituent files can be compiled individually.

%%%%%%%%%%%%%%%%%%%%%%%%%%%%%%%%%%%%%%%%%%%%%%%%%%%%%%%%%%%%%%%%%%%%%%%%%%%%%%%%
%\subsection{Feature Suggestions}
%
%The following is a list of features which may be useful for future
%versions of this package:
%%
%\begin{itemize}
%\item
%\ldots
%\end{itemize}

%%%%%%%%%%%%%%%%%%%%%%%%%%%%%%%%%%%%%%%%%%%%%%%%%%%%%%%%%%%%%%%%%%%%%%%%%%%%%%%%
\subsection{Revision History}

%%%%%%%%%%%%%%%%%%%%%%%%%%%%%%%%%%%%%%%%
\paragraph{v2.0:} 2018/12/30

\begin{itemize}
\item
immediate forward processing
\item
added |\childdocby| mechanism
\item
manual restructured
\end{itemize}

%%%%%%%%%%%%%%%%%%%%%%%%%%%%%%%%%%%%%%%%
\paragraph{v1.6:} 2018/01/17

\begin{itemize}
\item
application for development of include files
\item
corrections to manual
\end{itemize}

%%%%%%%%%%%%%%%%%%%%%%%%%%%%%%%%%%%%%%%%
\paragraph{v1.5:} 2017/05/21

\begin{itemize}
\item
more complete structuring introduced
\item
|\childdocof| introduced
\item
|\childdoc| renamed to |\childdocmain|
\item
|\childredirect| renamed to |\childdocforward| and |\childdocforwardprefix|
and functionality expanded
\end{itemize}

%%%%%%%%%%%%%%%%%%%%%%%%%%%%%%%%%%%%%%%%
\paragraph{v1.0:} 2017/04/27

\begin{itemize}
\item
manual and install package
\item
first version published on CTAN
\end{itemize}

%%%%%%%%%%%%%%%%%%%%%%%%%%%%%%%%%%%%%%%%
\paragraph{v0.6:} 2017/04/26

\begin{itemize}
\item
redirection mechanism added
\end{itemize}

%%%%%%%%%%%%%%%%%%%%%%%%%%%%%%%%%%%%%%%%
\paragraph{v0.5:} 2017/04/26

\begin{itemize}
\item
functionality in definition file
\end{itemize}


%%%%%%%%%%%%%%%%%%%%%%%%%%%%%%%%%%%%%%%%%%%%%%%%%%%%%%%%%%%%%%%%%%%%%%%%%%%%%%%%
%%%%%%%%%%%%%%%%%%%%%%%%%%%%%%%%%%%%%%%%%%%%%%%%%%%%%%%%%%%%%%%%%%%%%%%%%%%%%%%%
%%%%%%%%%%%%%%%%%%%%%%%%%%%%%%%%%%%%%%%%%%%%%%%%%%%%%%%%%%%%%%%%%%%%%%%%%%%%%%%%
\appendix

\settowidth\MacroIndent{\rmfamily\scriptsize 000\ }

 \DocInput{childdoc.dtx}

\end{document}
%</driver>
% \fi
%
% %%%%%%%%%%%%%%%%%%%%%%%%%%%%%%%%%%%%%%%%%%%%%%%%%%%%%%%%%%%%%%%%%%%%%%%%%%%%%%
% %%%%%%%%%%%%%%%%%%%%%%%%%%%%%%%%%%%%%%%%%%%%%%%%%%%%%%%%%%%%%%%%%%%%%%%%%%%%%%
% \section{Sample}
%\iffalse
%<*samplemain>
%\fi
%
% The following presents a sample document
% with two chapters, two parts, a title page,
% a compile flag as well as three forwarding files to set the flag.
% It consists of eight |.tex| files:
% \begin{center}
% \begin{tabular}{ll}
% |cdocsamp.tex|&main file\\
% |cdocsch1.tex|&include file for chapter 1\\
% |cdocsch2.tex|&include file for chapter 2\\
% |cdocspt3.tex|&include file for part 3\\
% |cdocspt4.tex|&include file for part 4\\
% |cdocsdrf.tex|&forwarding file for main file in draft mode\\
% |cdocsfi1.tex|&forwarding file for final version of chapter 1\\
% |cdocsfi2.tex|&forwarding file for final version of chapter 2\\
% \end{tabular}
% \end{center}
% Each of the eight files can be compiled directly by the \LaTeX{} compiler.
%
% %%%%%%%%%%%%%%%%%%%%%%%%%%%%%%%%%%%%%%
% \paragraph{Main File.}
%
% The main file is called |cdocsamp.tex|.
%
% Load the \textsf{childdoc} definitions and
% declare the filename for the main document:
%    \begin{macrocode}
\input{childdoc.def}
\childdocmain{}
%    \end{macrocode}

% Optional override for |\version| flag:
%    \begin{macrocode}
%%\ifchilddoc\else\providecommand{\version}{draft}\fi
%    \end{macrocode}

% Define the default values for the |\version| flag
% (|final| for the main file and |draft| for childs):
%    \begin{macrocode}
\ifchilddoc
\providecommand{\version}{draft}
\else
\providecommand{\version}{final}
\fi
%    \end{macrocode}

% Load the standard document class:
%    \begin{macrocode}
\documentclass[12pt]{article}
%    \end{macrocode}

% Start the document body:
%    \begin{macrocode}
\begin{document}
%    \end{macrocode}

% Declare a title page.
% Print title, part of document being processed and version flag:
%    \begin{macrocode}
\addtocounter{page}{-1}
\begin{center}
{\LARGE\bfseries{}childdoc example\par}
\vspace{1cm}
\ifchilddoc
\ifchilddocmanual part\else chapter\fi:
`\childdocname' of `\childdocjob'\par
\else
main document: `\childdocjob'\par
\fi
version: \version\par
\end{center}
\newpage
%    \end{macrocode}

% Manually include selected file,
% otherwise process as usual:
%    \begin{macrocode}
\ifchilddocmanual
\section*{part `\childdocname'}
\input{\childdocname}
\else
%    \end{macrocode}

% Include the two chapters:
%    \begin{macrocode}
\include{cdocsch1}
\include{cdocsch2}
%    \end{macrocode}

% Include the two parts unless only chapters should be displayed:
%    \begin{macrocode}
\ifchilddoc\else
\section{part three}
\input{cdocspt3}
\section{part four}
\input{cdocspt4}
\fi
%    \end{macrocode}

% Process as usual until here:
%    \begin{macrocode}
\fi
%    \end{macrocode}

% End of document body:
%    \begin{macrocode}
\end{document}
%    \end{macrocode}
%\iffalse
%</samplemain>
%\fi
%
% %%%%%%%%%%%%%%%%%%%%%%%%%%%%%%%%%%%%%%
% \paragraph{Chapter Include Files.}
%
% The include files are called |cdocsch1.tex| and |cdocsch2.tex|.
%
%\iffalse
%<*samplechap1|samplechap2>
%\fi

% Optional override for |\version| flag:
%    \begin{macrocode}
%%\providecommand{\version}{final}
%    \end{macrocode}

% Include the main document:
%    \begin{macrocode}
\input{childdoc.def}
\childdocof{cdocsamp}
%    \end{macrocode}

%\iffalse
%</samplechap1|samplechap2>
%\fi
%
%\iffalse
%<*samplechap1>
%\fi
% Some text for chapter 1:
%    \begin{macrocode}
\section{one}
some text in chapter one
%    \end{macrocode}

%\iffalse
%</samplechap1>
%\fi
% Some text for chapter 2:
%\iffalse
%<*samplechap2>
%\fi
%    \begin{macrocode}
\section{two}
more text in chapter two
%    \end{macrocode}

%\iffalse
%</samplechap2>
%\fi
%
% %%%%%%%%%%%%%%%%%%%%%%%%%%%%%%%%%%%%%%
% \paragraph{Part Include Files.}
%
% The include files are called |cdocspt3.tex| and |cdocspt4.tex|.
%
%\iffalse
%<*samplepart3|samplepart4>
%\fi

% Optional override for |\version| flag:
%    \begin{macrocode}
%%\providecommand{\version}{final}
%    \end{macrocode}

% Include the main document:
%    \begin{macrocode}
\input{childdoc.def}
\childdocby{cdocsamp}
%    \end{macrocode}

%\iffalse
%</samplepart3|samplepart4>
%\fi
%
%\iffalse
%<*samplepart3>
%\fi
% Some text for part 3:
%    \begin{macrocode}
some text in part three
%    \end{macrocode}

%\iffalse
%</samplepart3>
%\fi
% Some text for part 4:
%\iffalse
%<*samplepart4>
%\fi
%    \begin{macrocode}
more text in part four
%    \end{macrocode}

%\iffalse
%</samplepart4>
%\fi
%
% %%%%%%%%%%%%%%%%%%%%%%%%%%%%%%%%%%%%%%
% \paragraph{Forwarding for a Complete Draft.}
%
% The following forwarding file |cdocsdrf.tex|
% compiles the main document in draft mode:
%\iffalse
%<*sampledraft>
%\fi
%    \begin{macrocode}
\def\version{draft}
\input{childdoc.def}
\childdocforward{cdocsamp}
%    \end{macrocode}

%\iffalse
%</sampledraft>
%\fi
%
% %%%%%%%%%%%%%%%%%%%%%%%%%%%%%%%%%%%%%%
% \paragraph{Forwarding for Final Version of the Chapters.}
%
% The following forwarding files |cdocsfn1.tex| and |cdocsfn2.tex|
% (with identical content)
% compile the final versions of the child documents
% |cdocsch1.tex| and |cdocsch2.tex|, respectively:
%\iffalse
%<*samplefinal>
%\fi
%    \begin{macrocode}
\def\version{final}
\input{childdoc.def}
\childdocforwardprefix[cdocsamp]{cdocsfn}{cdocsch}
%    \end{macrocode}

%\iffalse
%</samplefinal>
%\fi
%
% %%%%%%%%%%%%%%%%%%%%%%%%%%%%%%%%%%%%%%
% \paragraph{Command Line Processing.}
%
% The following three command lines generate the output files
% |cdocscld|, |cdocscl1| and |cdocscl2|
% which should be identical to
% |cdocsdrf|, |cdocsch1| and |cdocsfn2|, respectively:
% \begin{center}
% \begin{tabular}{l}
% |latex -jobname cdocscld \|\\
% |  "\def\version{draft}\input{childdoc.def}\childdocforward{cdocsamp}"|\\
% |latex -jobname cdocscl1 \|\\
% |  "\input{childdoc.def}\childdocforward[cdocsamp]{cdocsch1}"|\\
% |latex -jobname cdocscl2 \|\\
% |  "\def\version{final}\input{childdoc.def}\childdocforward{cdocsch2}"|
% \end{tabular}
% \end{center}
% Note that the trailing backslash on each first line
% merely continues the input to the second line
% (for convenient cut ant paste).
% Furthermore, the command |latex| can be replaced by any
% of its alternative versions such as |pdflatex|.
%
% %%%%%%%%%%%%%%%%%%%%%%%%%%%%%%%%%%%%%%%%%%%%%%%%%%%%%%%%%%%%%%%%%%%%%%%%%%%%%%
% %%%%%%%%%%%%%%%%%%%%%%%%%%%%%%%%%%%%%%%%%%%%%%%%%%%%%%%%%%%%%%%%%%%%%%%%%%%%%%
% \section{Implementation}
%\iffalse
%<*package>
%\fi
%
% This section describes the definitions file |childdoc.def|.

% The definitions cannot be loaded using |\usepackage| or |\RequirePackage|
% which has a mechanism to prevent loading a style file more than once.
% When loading the definitions by means of |\input|
% multiple instances have to be prevented manually:
%\iffalse
%This code needs to be before the `\ProvidesFile' directive
%which is defined at the beginning of this file.
%Therefore it is also placed there and commented out here.
%</package>
%<*discard>
%\fi
%    \begin{macrocode}
\ifdefined\childdocmain\endinput\fi
%    \end{macrocode}
%\iffalse
%</discard>
%<*package>
%\fi
%
% \macro{\ifchilddoc}
% \macro{\ifchilddocmanual}
% The conditional |\ifchilddoc| tells whether a
% child (true) or main (false) document is being compiled.
% The conditional |\ifchilddocmanual| tells whether
% the |\includeonly| mechanism is used (false) or
% the selection of child files must be performed manually (true).
% The definitions initialise to false:
%    \begin{macrocode}
\newif\ifchilddoc
\newif\ifchilddocmanual
%    \end{macrocode}

% \macro{\childdocname}
% \macro{\childdocjob}
% The macro |\childdocname| stores the name of the main document
% to be compiled. The macro |\childdocjob| stores the name of
% the document on which the \LaTeX{} compiler was originally invoked.
% The content of |\jobname| cannot be compared
% to filenames specified in the source due to different catcodes.
% The following code rescans |\jobname|, stores the result
% in |\childdocname| and saves a copy in |\childdocjob|:
%    \begin{macrocode}
\edef\childdocname{\scantokens\expandafter{\jobname\noexpand}}
\let\childdocjob\childdocname
%    \end{macrocode}

% \macro{\childdocdisable}
% The macro |\childdocdisable| prevents the main file
% from being processed more than once.
% At this stage, the main document command |\childdocmain|
% is assumed to be called once again where it should do nothing.
% Any subsequent call to it should prevent
% a secondary processing of the main document
% It overwrites the forwarding commands
% |\childdocof| and |\childdocforward|
% with empty macros to prevent further inclusions of the main document:
%    \begin{macrocode}
\newcommand{\childdocdisable}
{
  \renewcommand{\childdocmain}[1]{\renewcommand{\childdocmain}[1]{\endinput}}
  \renewcommand{\childdocof}[1]{}
  \renewcommand{\childdocby}[2][]{}
  \renewcommand{\childdocforward}[2][]{}
  \renewcommand{\childdocdisable}{}
}
%    \end{macrocode}

% \macro{\childdocmain}
% The macro |\childdocmain| is to be called at the top of the main file
% with nothing or the main filename (without extension) as argument.
% First, it breaks loops.
% If the argument is not empty and does not match |\childdocname|
% (which is set by the first inclusion of |childdoc.def|),
% |\ifchilddoc| is set to true, |\includeonly| is applied to the child file
% and |\jobname| is set to the main file
% (for proper handling of |.aux| files):
%    \begin{macrocode}
\newcommand{\childdocmain}[1]
{
  \childdocdisable\childdocmain{}
  \if?#1?\else
    \begingroup
      \def\childdoctmp{#1}
      \ifx\childdoctmp\childdocname
        \def\childdoctmp{}
      \else
        \def\childdoctmp
        {
          \childdoctrue
          \includeonly{\childdocname}
          \def\childdocjob{#1}
          \def\jobname{#1}
        }
      \fi
      \expandafter
    \endgroup
    \childdoctmp
  \fi
}
%    \end{macrocode}

% \macro{\childdocof}
% The command |\childdocof| redirects
% compilation to the main file |#1|.
%    \begin{macrocode}
\newcommand{\childdocof}[1]
{
  \childdocdisable
  \childdoctrue
  \includeonly{\childdocname}
  \def\jobname{#1}
  \def\childdocjob{#1}
  \input{#1}
}
%    \end{macrocode}

% \macro{\childdocby}
% The command |\childdocby| ....
%    \begin{macrocode}
\newcommand{\childdocby}[2][]
{
  \childdocdisable
  \childdoctrue
  \childdocmanualtrue
  \if?#1?\else
    \def\jobname{#2}
  \fi
  \def\childdocjob{#2}
  \input{#2}
  \endinput
}
%    \end{macrocode}

% \macro{\childdocforward}
% The command |\childdocforward| redirects
% compilation to the main file or
% (if the optional argument is given) a child file.
% Parameters are set as if the main file
% or a child file starting with |\childdocof| was compiled.
% Then compilation is handed over to the main file:
%    \begin{macrocode}
\newcommand{\childdocforward}[2][]
{
  \begingroup
    \if?#1?
      \def\childdoctmp
      {
        \def\childdocname{#2}
        \def\childdocjob{#2}
        \def\jobname{#2}
        \input{#2}
        \endinput
      }
    \else
      \def\childdoctmp
      {
        \childdocdisable
        \def\childdocname{#2}
        \childdoctrue
        \includeonly{#2}
        \def\childdocjob{#1}
        \def\jobname{#1}
        \input{#1}
        \endinput
      }
    \fi
    \expandafter
  \endgroup
  \childdoctmp
}
%    \end{macrocode}

% \macro{\childdocforwardprefix}
% The command |\childdocforwardprefix| redirects
% compilation to the main or a child file by means of a pattern.
% The prefix |#1| in the current filename is replaced by |#2|
% and the suffix of the current filename is kept
% (it is assumed that the filename does not contain the substring `|~~~|'
% which is used as a delimiter).
% Compilation is handed over to the new file by |\childdocforward|:
%    \begin{macrocode}
\newcommand{\childdocforwardprefix}[3][]
{
  \begingroup
    \def\childdocextract #2##1~~~{\def\childdoctmp{\childdocforward[#1]{#3##1}}}
    \expandafter\childdocextract\childdocname~~~
    \expandafter
  \endgroup
  \childdoctmp
}
%    \end{macrocode}

% \macro{\childdoc}
% The deprecated macro |\childdoc| is a legacy version of |\childdocmain|:
%    \begin{macrocode}
\newcommand{\childdoc}{\childdocmain}
%    \end{macrocode}

% \macro{\childdocredirect}
% The deprecated macro |\childdocredirect| is a legacy version
% of |\childdocforward| and |\childdocforwardprefix|:
%    \begin{macrocode}
\newcommand{\childdocredirect}[2][]
{
  \begingroup
    \if?#1?
      \def\childdoctmp{\childdocforward{#2}}
    \else
      \def\childdoctmp{\childdocforwardprefix{#1}{#2}}
    \fi
    \expandafter
  \endgroup
  \childdoctmp
}
%    \end{macrocode}

%\iffalse
%</package>
%\fi
%
\endinput
\childdocforward{cdocsch2}"|
% \end{tabular}
% \end{center}
% Note that the trailing backslash on each first line
% merely continues the input to the second line
% (for convenient cut ant paste).
% Furthermore, the command |latex| can be replaced by any
% of its alternative versions such as |pdflatex|.
%
% %%%%%%%%%%%%%%%%%%%%%%%%%%%%%%%%%%%%%%%%%%%%%%%%%%%%%%%%%%%%%%%%%%%%%%%%%%%%%%
% %%%%%%%%%%%%%%%%%%%%%%%%%%%%%%%%%%%%%%%%%%%%%%%%%%%%%%%%%%%%%%%%%%%%%%%%%%%%%%
% \section{Implementation}
%\iffalse
%<*package>
%\fi
%
% This section describes the definitions file |childdoc.def|.

% The definitions cannot be loaded using |\usepackage| or |\RequirePackage|
% which has a mechanism to prevent loading a style file more than once.
% When loading the definitions by means of |\input|
% multiple instances have to be prevented manually:
%\iffalse
%This code needs to be before the `\ProvidesFile' directive
%which is defined at the beginning of this file.
%Therefore it is also placed there and commented out here.
%</package>
%<*discard>
%\fi
%    \begin{macrocode}
\ifdefined\childdocmain\endinput\fi
%    \end{macrocode}
%\iffalse
%</discard>
%<*package>
%\fi
%
% \macro{\ifchilddoc}
% \macro{\ifchilddocmanual}
% The conditional |\ifchilddoc| tells whether a
% child (true) or main (false) document is being compiled.
% The conditional |\ifchilddocmanual| tells whether
% the |\includeonly| mechanism is used (false) or
% the selection of child files must be performed manually (true).
% The definitions initialise to false:
%    \begin{macrocode}
\newif\ifchilddoc
\newif\ifchilddocmanual
%    \end{macrocode}

% \macro{\childdocname}
% \macro{\childdocjob}
% The macro |\childdocname| stores the name of the main document
% to be compiled. The macro |\childdocjob| stores the name of
% the document on which the \LaTeX{} compiler was originally invoked.
% The content of |\jobname| cannot be compared
% to filenames specified in the source due to different catcodes.
% The following code rescans |\jobname|, stores the result
% in |\childdocname| and saves a copy in |\childdocjob|:
%    \begin{macrocode}
\edef\childdocname{\scantokens\expandafter{\jobname\noexpand}}
\let\childdocjob\childdocname
%    \end{macrocode}

% \macro{\childdocdisable}
% The macro |\childdocdisable| prevents the main file
% from being processed more than once.
% At this stage, the main document command |\childdocmain|
% is assumed to be called once again where it should do nothing.
% Any subsequent call to it should prevent
% a secondary processing of the main document
% It overwrites the forwarding commands
% |\childdocof| and |\childdocforward|
% with empty macros to prevent further inclusions of the main document:
%    \begin{macrocode}
\newcommand{\childdocdisable}
{
  \renewcommand{\childdocmain}[1]{\renewcommand{\childdocmain}[1]{\endinput}}
  \renewcommand{\childdocof}[1]{}
  \renewcommand{\childdocby}[2][]{}
  \renewcommand{\childdocforward}[2][]{}
  \renewcommand{\childdocdisable}{}
}
%    \end{macrocode}

% \macro{\childdocmain}
% The macro |\childdocmain| is to be called at the top of the main file
% with nothing or the main filename (without extension) as argument.
% First, it breaks loops.
% If the argument is not empty and does not match |\childdocname|
% (which is set by the first inclusion of |childdoc.def|),
% |\ifchilddoc| is set to true, |\includeonly| is applied to the child file
% and |\jobname| is set to the main file
% (for proper handling of |.aux| files):
%    \begin{macrocode}
\newcommand{\childdocmain}[1]
{
  \childdocdisable\childdocmain{}
  \if?#1?\else
    \begingroup
      \def\childdoctmp{#1}
      \ifx\childdoctmp\childdocname
        \def\childdoctmp{}
      \else
        \def\childdoctmp
        {
          \childdoctrue
          \includeonly{\childdocname}
          \def\childdocjob{#1}
          \def\jobname{#1}
        }
      \fi
      \expandafter
    \endgroup
    \childdoctmp
  \fi
}
%    \end{macrocode}

% \macro{\childdocof}
% The command |\childdocof| redirects
% compilation to the main file |#1|.
%    \begin{macrocode}
\newcommand{\childdocof}[1]
{
  \childdocdisable
  \childdoctrue
  \includeonly{\childdocname}
  \def\jobname{#1}
  \def\childdocjob{#1}
  \input{#1}
}
%    \end{macrocode}

% \macro{\childdocby}
% The command |\childdocby| ....
%    \begin{macrocode}
\newcommand{\childdocby}[2][]
{
  \childdocdisable
  \childdoctrue
  \childdocmanualtrue
  \if?#1?\else
    \def\jobname{#2}
  \fi
  \def\childdocjob{#2}
  \input{#2}
  \endinput
}
%    \end{macrocode}

% \macro{\childdocforward}
% The command |\childdocforward| redirects
% compilation to the main file or
% (if the optional argument is given) a child file.
% Parameters are set as if the main file
% or a child file starting with |\childdocof| was compiled.
% Then compilation is handed over to the main file:
%    \begin{macrocode}
\newcommand{\childdocforward}[2][]
{
  \begingroup
    \if?#1?
      \def\childdoctmp
      {
        \def\childdocname{#2}
        \def\childdocjob{#2}
        \def\jobname{#2}
        \input{#2}
        \endinput
      }
    \else
      \def\childdoctmp
      {
        \childdocdisable
        \def\childdocname{#2}
        \childdoctrue
        \includeonly{#2}
        \def\childdocjob{#1}
        \def\jobname{#1}
        \input{#1}
        \endinput
      }
    \fi
    \expandafter
  \endgroup
  \childdoctmp
}
%    \end{macrocode}

% \macro{\childdocforwardprefix}
% The command |\childdocforwardprefix| redirects
% compilation to the main or a child file by means of a pattern.
% The prefix |#1| in the current filename is replaced by |#2|
% and the suffix of the current filename is kept
% (it is assumed that the filename does not contain the substring `|~~~|'
% which is used as a delimiter).
% Compilation is handed over to the new file by |\childdocforward|:
%    \begin{macrocode}
\newcommand{\childdocforwardprefix}[3][]
{
  \begingroup
    \def\childdocextract #2##1~~~{\def\childdoctmp{\childdocforward[#1]{#3##1}}}
    \expandafter\childdocextract\childdocname~~~
    \expandafter
  \endgroup
  \childdoctmp
}
%    \end{macrocode}

% \macro{\childdoc}
% The deprecated macro |\childdoc| is a legacy version of |\childdocmain|:
%    \begin{macrocode}
\newcommand{\childdoc}{\childdocmain}
%    \end{macrocode}

% \macro{\childdocredirect}
% The deprecated macro |\childdocredirect| is a legacy version
% of |\childdocforward| and |\childdocforwardprefix|:
%    \begin{macrocode}
\newcommand{\childdocredirect}[2][]
{
  \begingroup
    \if?#1?
      \def\childdoctmp{\childdocforward{#2}}
    \else
      \def\childdoctmp{\childdocforwardprefix{#1}{#2}}
    \fi
    \expandafter
  \endgroup
  \childdoctmp
}
%    \end{macrocode}

%\iffalse
%</package>
%\fi
%
\endinput

\childdocmain{}
%    \end{macrocode}

% Optional override for |\version| flag:
%    \begin{macrocode}
%%\ifchilddoc\else\providecommand{\version}{draft}\fi
%    \end{macrocode}

% Define the default values for the |\version| flag
% (|final| for the main file and |draft| for childs):
%    \begin{macrocode}
\ifchilddoc
\providecommand{\version}{draft}
\else
\providecommand{\version}{final}
\fi
%    \end{macrocode}

% Load the standard document class:
%    \begin{macrocode}
\documentclass[12pt]{article}
%    \end{macrocode}

% Start the document body:
%    \begin{macrocode}
\begin{document}
%    \end{macrocode}

% Declare a title page.
% Print title, part of document being processed and version flag:
%    \begin{macrocode}
\addtocounter{page}{-1}
\begin{center}
{\LARGE\bfseries{}childdoc example\par}
\vspace{1cm}
\ifchilddoc
\ifchilddocmanual part\else chapter\fi:
`\childdocname' of `\childdocjob'\par
\else
main document: `\childdocjob'\par
\fi
version: \version\par
\end{center}
\newpage
%    \end{macrocode}

% Manually include selected file,
% otherwise process as usual:
%    \begin{macrocode}
\ifchilddocmanual
\section*{part `\childdocname'}
\input{\childdocname}
\else
%    \end{macrocode}

% Include the two chapters:
%    \begin{macrocode}
\include{cdocsch1}
\include{cdocsch2}
%    \end{macrocode}

% Include the two parts unless only chapters should be displayed:
%    \begin{macrocode}
\ifchilddoc\else
\section{part three}
\input{cdocspt3}
\section{part four}
\input{cdocspt4}
\fi
%    \end{macrocode}

% Process as usual until here:
%    \begin{macrocode}
\fi
%    \end{macrocode}

% End of document body:
%    \begin{macrocode}
\end{document}
%    \end{macrocode}
%\iffalse
%</samplemain>
%\fi
%
% %%%%%%%%%%%%%%%%%%%%%%%%%%%%%%%%%%%%%%
% \paragraph{Chapter Include Files.}
%
% The include files are called |cdocsch1.tex| and |cdocsch2.tex|.
%
%\iffalse
%<*samplechap1|samplechap2>
%\fi

% Optional override for |\version| flag:
%    \begin{macrocode}
%%\providecommand{\version}{final}
%    \end{macrocode}

% Include the main document:
%    \begin{macrocode}
% \iffalse
%
% childdoc.dtx Copyright (C) 2017-2018 Niklas Beisert
%
% This work may be distributed and/or modified under the
% conditions of the LaTeX Project Public License, either version 1.3
% of this license or (at your option) any later version.
% The latest version of this license is in
%   http://www.latex-project.org/lppl.txt
% and version 1.3 or later is part of all distributions of LaTeX
% version 2005/12/01 or later.
%
% This work has the LPPL maintenance status `maintained'.
%
% The Current Maintainer of this work is Niklas Beisert.
%
% This work consists of the files childdoc.dtx and childdoc.ins
% and the derived files childdoc.def and cdocsamp.tex with
% cdocsch1.tex, cdocsch2.tex, cdocsdrf.tex, cdocsfn1.tex, cdocsfn2.tex.
%
%<package>\ifdefined\childdocmain\endinput\fi
%<package>\ProvidesFile{childdoc.def}[2018/12/30 v2.0 child document driver]
%<samplemain>\ProvidesFile{cdocsamp.tex}[2018/12/30 v2.0 sample for childdoc]
%<*driver>
%\ProvidesFile{childdoc.drv}[2018/12/30 v2.0 childdoc reference manual file]
\PassOptionsToClass{10pt,a4paper}{article}
\documentclass{ltxdoc}

\usepackage[margin=35mm]{geometry}
\usepackage{hyperref}
\usepackage{hyperxmp}
\usepackage[usenames]{color}

\hypersetup{colorlinks=true}
\hypersetup{pdfstartview=FitH}
\hypersetup{pdfpagemode=UseNone}
\hypersetup{pdfsource={}}
\hypersetup{pdflang={en-UK}}
\hypersetup{pdfcopyright={Copyright 2017-2018 Niklas Beisert.
  This work may be distributed and/or modified under the
  conditions of the LaTeX Project Public License, either version 1.3
  of this license or (at your option) any later version.}}
\hypersetup{pdflicenseurl={http://www.latex-project.org/lppl.txt}}
\hypersetup{pdfcontactaddress={ETH Zurich, ITP, HIT K,
  Wolfgang-Pauli-Strasse 27}}
\hypersetup{pdfcontactpostcode={8093}}
\hypersetup{pdfcontactcity={Zurich}}
\hypersetup{pdfcontactcountry={Switzerland}}
\hypersetup{pdfcontactemail={nbeisert@itp.phys.ethz.ch}}
\hypersetup{pdfcontacturl={http://people.phys.ethz.ch/\xmptilde nbeisert/}}

\newcommand{\secref}[1]{\hyperref[#1]{section \ref*{#1}}}

\parskip1ex
\parindent0pt
\let\olditemize\itemize
\def\itemize{\olditemize\parskip0pt}

\begin{document}

\title{The \textsf{childdoc} Package}
\hypersetup{pdftitle={The childdoc Package}}
\author{Niklas Beisert\\[2ex]
  Institut f\"ur Theoretische Physik\\
  Eidgen\"ossische Technische Hochschule Z\"urich\\
  Wolfgang-Pauli-Strasse 27, 8093 Z\"urich, Switzerland\\[1ex]
  \href{mailto:nbeisert@itp.phys.ethz.ch}
  {\texttt{nbeisert@itp.phys.ethz.ch}}}
\hypersetup{pdfauthor={Niklas Beisert}}
\hypersetup{pdfsubject={Manual for the LaTeX2e Package childdoc}}
\date{30 December 2018, \textsf{v2.0}}
\maketitle

\begin{abstract}\noindent
\textsf{childdoc} is a \LaTeXe{} package
that enables the direct compilation
of document sections included by |\include|
to individual files.
\end{abstract}

\begingroup
\parskip0ex
\tableofcontents
\endgroup

%%%%%%%%%%%%%%%%%%%%%%%%%%%%%%%%%%%%%%%%%%%%%%%%%%%%%%%%%%%%%%%%%%%%%%%%%%%%%%%%
%%%%%%%%%%%%%%%%%%%%%%%%%%%%%%%%%%%%%%%%%%%%%%%%%%%%%%%%%%%%%%%%%%%%%%%%%%%%%%%%
\section{Introduction}

\LaTeX{} provides a mechanism to structure a large document (such as a book)
into a main file and several child files (containing the chapters)
using the |\include| command.
This mechanism is beneficial for documents
which span hundreds of pages in order to
make the source file(s) more manageable.
Moreover, compilation can be restricted to
selected child files by means of the |\includeonly| command.
The latter feature can be used to reduce the compilation time while editing
(this was significantly more useful in the earlier days of \LaTeX{})
or to generate a smaller document which is easier to navigate.
Another application of |\includeonly| is to generate
documents consisting of selected parts of the complete document.

However, there are a few drawbacks of the plain |\include| mechanism:
\begin{itemize}
\item
The child files cannot be compiled on their own,
they can only be compiled via the main file.
A naive editing environment
(such as a text editor with an option
to have the current file processed by \LaTeX)
may require one to switch to the main file before compiling;
attempting to compile the child file produces errors.
\item
The main file must be modified (each time)
to adjust the |\includeonly| command
to the present needs. This easily leaves the main file in a messy state.
\item
The generated document will always carry the filename
of the main document. This is inconvenient if
several child files are to be compiled and
to be kept for distribution.
\end{itemize}

The present package provides a simple interface
to make child files individually compilable by \LaTeX{}.
Compiling a child file then has the same effect as compiling
the main file with an |\includeonly| command
to select the appropriate child.
Moreover the generated document will carry the name of the child
rather than the main file.
This resolves all three above issues.

This feature is meant to make the editing of books,
thesis documents and lecture notes somewhat more convenient.
However, the package can also be used efficiently for
composing a series of documents (such as exercise sheets)
which are typically distributed individually.
It then assists the author in generating the individual documents
(potentially in different versions)
as well as a document containing the collected series.
Another application is in developing style files
or other kinds of included material
where compilation of the style file could redirect
to a sample or test file.

%%%%%%%%%%%%%%%%%%%%%%%%%%%%%%%%%%%%%%%%%%%%%%%%%%%%%%%%%%%%%%%%%%%%%%%%%%%%%%%%
%%%%%%%%%%%%%%%%%%%%%%%%%%%%%%%%%%%%%%%%%%%%%%%%%%%%%%%%%%%%%%%%%%%%%%%%%%%%%%%%
\section{Usage}

First of all, the package \textsf{childdoc} is \emph{not} a standard
\LaTeXe{} |.sty| style file! Therefore it needs to be invoked in
a non-standard way.

%%%%%%%%%%%%%%%%%%%%%%%%%%%%%%%%%%%%%%%%%%%%%%%%%%%%%%%%%%%%%%%%%%%%%%%%%%%%%%%%
\subsection{Included Files}
\label{sec:include}

%%%%%%%%%%%%%%%%%%%%%%%%%%%%%%%%%%%%%%%%
\DescribeMacro{\childdocmain}
To use the package, add the commands
\begin{center}
\begin{tabular}{l}
|% \iffalse
%
% childdoc.dtx Copyright (C) 2017-2018 Niklas Beisert
%
% This work may be distributed and/or modified under the
% conditions of the LaTeX Project Public License, either version 1.3
% of this license or (at your option) any later version.
% The latest version of this license is in
%   http://www.latex-project.org/lppl.txt
% and version 1.3 or later is part of all distributions of LaTeX
% version 2005/12/01 or later.
%
% This work has the LPPL maintenance status `maintained'.
%
% The Current Maintainer of this work is Niklas Beisert.
%
% This work consists of the files childdoc.dtx and childdoc.ins
% and the derived files childdoc.def and cdocsamp.tex with
% cdocsch1.tex, cdocsch2.tex, cdocsdrf.tex, cdocsfn1.tex, cdocsfn2.tex.
%
%<package>\ifdefined\childdocmain\endinput\fi
%<package>\ProvidesFile{childdoc.def}[2018/12/30 v2.0 child document driver]
%<samplemain>\ProvidesFile{cdocsamp.tex}[2018/12/30 v2.0 sample for childdoc]
%<*driver>
%\ProvidesFile{childdoc.drv}[2018/12/30 v2.0 childdoc reference manual file]
\PassOptionsToClass{10pt,a4paper}{article}
\documentclass{ltxdoc}

\usepackage[margin=35mm]{geometry}
\usepackage{hyperref}
\usepackage{hyperxmp}
\usepackage[usenames]{color}

\hypersetup{colorlinks=true}
\hypersetup{pdfstartview=FitH}
\hypersetup{pdfpagemode=UseNone}
\hypersetup{pdfsource={}}
\hypersetup{pdflang={en-UK}}
\hypersetup{pdfcopyright={Copyright 2017-2018 Niklas Beisert.
  This work may be distributed and/or modified under the
  conditions of the LaTeX Project Public License, either version 1.3
  of this license or (at your option) any later version.}}
\hypersetup{pdflicenseurl={http://www.latex-project.org/lppl.txt}}
\hypersetup{pdfcontactaddress={ETH Zurich, ITP, HIT K,
  Wolfgang-Pauli-Strasse 27}}
\hypersetup{pdfcontactpostcode={8093}}
\hypersetup{pdfcontactcity={Zurich}}
\hypersetup{pdfcontactcountry={Switzerland}}
\hypersetup{pdfcontactemail={nbeisert@itp.phys.ethz.ch}}
\hypersetup{pdfcontacturl={http://people.phys.ethz.ch/\xmptilde nbeisert/}}

\newcommand{\secref}[1]{\hyperref[#1]{section \ref*{#1}}}

\parskip1ex
\parindent0pt
\let\olditemize\itemize
\def\itemize{\olditemize\parskip0pt}

\begin{document}

\title{The \textsf{childdoc} Package}
\hypersetup{pdftitle={The childdoc Package}}
\author{Niklas Beisert\\[2ex]
  Institut f\"ur Theoretische Physik\\
  Eidgen\"ossische Technische Hochschule Z\"urich\\
  Wolfgang-Pauli-Strasse 27, 8093 Z\"urich, Switzerland\\[1ex]
  \href{mailto:nbeisert@itp.phys.ethz.ch}
  {\texttt{nbeisert@itp.phys.ethz.ch}}}
\hypersetup{pdfauthor={Niklas Beisert}}
\hypersetup{pdfsubject={Manual for the LaTeX2e Package childdoc}}
\date{30 December 2018, \textsf{v2.0}}
\maketitle

\begin{abstract}\noindent
\textsf{childdoc} is a \LaTeXe{} package
that enables the direct compilation
of document sections included by |\include|
to individual files.
\end{abstract}

\begingroup
\parskip0ex
\tableofcontents
\endgroup

%%%%%%%%%%%%%%%%%%%%%%%%%%%%%%%%%%%%%%%%%%%%%%%%%%%%%%%%%%%%%%%%%%%%%%%%%%%%%%%%
%%%%%%%%%%%%%%%%%%%%%%%%%%%%%%%%%%%%%%%%%%%%%%%%%%%%%%%%%%%%%%%%%%%%%%%%%%%%%%%%
\section{Introduction}

\LaTeX{} provides a mechanism to structure a large document (such as a book)
into a main file and several child files (containing the chapters)
using the |\include| command.
This mechanism is beneficial for documents
which span hundreds of pages in order to
make the source file(s) more manageable.
Moreover, compilation can be restricted to
selected child files by means of the |\includeonly| command.
The latter feature can be used to reduce the compilation time while editing
(this was significantly more useful in the earlier days of \LaTeX{})
or to generate a smaller document which is easier to navigate.
Another application of |\includeonly| is to generate
documents consisting of selected parts of the complete document.

However, there are a few drawbacks of the plain |\include| mechanism:
\begin{itemize}
\item
The child files cannot be compiled on their own,
they can only be compiled via the main file.
A naive editing environment
(such as a text editor with an option
to have the current file processed by \LaTeX)
may require one to switch to the main file before compiling;
attempting to compile the child file produces errors.
\item
The main file must be modified (each time)
to adjust the |\includeonly| command
to the present needs. This easily leaves the main file in a messy state.
\item
The generated document will always carry the filename
of the main document. This is inconvenient if
several child files are to be compiled and
to be kept for distribution.
\end{itemize}

The present package provides a simple interface
to make child files individually compilable by \LaTeX{}.
Compiling a child file then has the same effect as compiling
the main file with an |\includeonly| command
to select the appropriate child.
Moreover the generated document will carry the name of the child
rather than the main file.
This resolves all three above issues.

This feature is meant to make the editing of books,
thesis documents and lecture notes somewhat more convenient.
However, the package can also be used efficiently for
composing a series of documents (such as exercise sheets)
which are typically distributed individually.
It then assists the author in generating the individual documents
(potentially in different versions)
as well as a document containing the collected series.
Another application is in developing style files
or other kinds of included material
where compilation of the style file could redirect
to a sample or test file.

%%%%%%%%%%%%%%%%%%%%%%%%%%%%%%%%%%%%%%%%%%%%%%%%%%%%%%%%%%%%%%%%%%%%%%%%%%%%%%%%
%%%%%%%%%%%%%%%%%%%%%%%%%%%%%%%%%%%%%%%%%%%%%%%%%%%%%%%%%%%%%%%%%%%%%%%%%%%%%%%%
\section{Usage}

First of all, the package \textsf{childdoc} is \emph{not} a standard
\LaTeXe{} |.sty| style file! Therefore it needs to be invoked in
a non-standard way.

%%%%%%%%%%%%%%%%%%%%%%%%%%%%%%%%%%%%%%%%%%%%%%%%%%%%%%%%%%%%%%%%%%%%%%%%%%%%%%%%
\subsection{Included Files}
\label{sec:include}

%%%%%%%%%%%%%%%%%%%%%%%%%%%%%%%%%%%%%%%%
\DescribeMacro{\childdocmain}
To use the package, add the commands
\begin{center}
\begin{tabular}{l}
|\input{childdoc.def}|\\
|\childdocmain{}|\\
\end{tabular}
\end{center}
at the very top of the main \LaTeX{} file,
in particular \emph{before} the |\documentclass| statement!
The argument of |\childdocmain| should be left empty
(but it must be present).

%%%%%%%%%%%%%%%%%%%%%%%%%%%%%%%%%%%%%%%%
\DescribeMacro{\childdocof}
Furthermore, add the commands
\begin{center}
\begin{tabular}{l}
|\input{childdoc.def}|\\
|\childdocof{|\textit{main}|}|\\
\end{tabular}
\end{center}
at the top of every child file \textit{child}
which is included by |\include{|\textit{child}|}|
from within the main file
(or at least for those files to be compiled individually).
The argument \textit{main} must be the filename of the main file.

There are a couple of
considerations in setting up the main and child documents:

%%%%%%%%%%%%%%%%%%%%%%%%%%%%%%%%%%%%%%%%
\paragraph{Restrictions.}

Please note the following restrictions:
\begin{itemize}
\item
|\childdocmain| must be called with one argument \textit{main}
to ensure compatibility with earlier version of the package.
It must either be empty (|\childdocmain{}|)
or precisely match the filename of the main file in which it is specified.
See \secref{sec:detection} for further information.
\item
The filename \textit{main} must be specified without the |.tex| extension.
\item
The filename \textit{main} is case sensitive
(even in case-insensitive file systems)
due to internal string comparison.
\item
The argument \textit{main} should be fully expanded, it cannot be a macro.
\item
Subdirectories and special characters should be avoided in filenames.
\item
The command |\childdocmain{|\textit{main}|}| must be followed by a whitespace.
It should not be followed immediately by another command
or by a comment mark `|%|'.
This is because the \TeX{} parser reads the token immediately following
the argument of |\childdocmain| and puts it
at the beginning of every child section;
however, a white\-space is ignored.
\end{itemize}

%%%%%%%%%%%%%%%%%%%%%%%%%%%%%%%%%%%%%%%%
\paragraph{Content of Main File.}

It is advisable to place all content in the child files included by |\include|.
Any output contained in the main file will appear in all child documents
unless suppressed manually;
it cannot be suppressed automatically by the |\includeonly| directive
and thus should normally be avoided.
A method to include some content in the main file
by means of conditional processing is described in \secref{sec:conditional}.

%%%%%%%%%%%%%%%%%%%%%%%%%%%%%%%%%%%%%%%%
\paragraph{Page Numbering.}

When only a part of the document is compiled,
the appropriate numbering of pages
(as well as other status parameters)
is determined from the |.aux| files.
The latter contain information from previous passes.
However this information needs to propagate through
all intermediate child documents.
Therefore the page numbering in child documents may well
be inconsistent until the complete document is compiled at least once.

A useful (if unconventional) way to always ensure a consistent
page numbering is to restart the numbering in each child document
and denote the pages by `\textit{child}|.|\textit{page}'
where \textit{child} represents the chapter/section number of the child file.
This can be achieved by the command
|\numberwithin{page}{|\textit{child}|}|
of the \textsf{amsmath} package
where \textit{child} can be |chapter| or |section|
depending on the chosen structuring.
Alternatively, one can modify the macro |\thepage| appropriately
and reset the counter |page| at the start of each child file.

%%%%%%%%%%%%%%%%%%%%%%%%%%%%%%%%%%%%%%%%%%%%%%%%%%%%%%%%%%%%%%%%%%%%%%%%%%%%%%%%
\subsection{Conditional Processing}
\label{sec:conditional}

The package provides a mechanism to compile different versions
of a document. To customise the versions further some conditional processing
can come in handy to distinguish which version is being compiled.
The package provides two macros to describe the compilation context:

%%%%%%%%%%%%%%%%%%%%%%%%%%%%%%%%%%%%%%%%
\DescribeMacro{\ifchilddoc}
The conditional |\ifchilddoc| distinguishes between the compilation of
child documents and the main document:
%
\begin{center}
|\ifchilddoc |\textit{child-code}| |[|\||else |\textit{main-code}]| \||fi|
\end{center}

%%%%%%%%%%%%%%%%%%%%%%%%%%%%%%%%%%%%%%%%
\DescribeMacro{\childdocname}
\DescribeMacro{\childdocjob}
The macro |\childdocname| contains the filename (without extension)
of the main or child file being processed.
Note that |\childdocjob| will always contain the name of the main file.

%%%%%%%%%%%%%%%%%%%%%%%%%%%%%%%%%%%%%%%%
\paragraph{Title Page.}

Conditional processing can be used to include a title or banner page
in the main document when proper precautions are taken.
Importantly, the code in the main file should ensure that the page counter
(as well as other status parameters which are stored in the |.aux| files)
takes the same value after the conditional processing.
Otherwise the page numbers may take divergent values
depending on which part is compiled.

For example, a title page could be declared by:
%
\begin{center}
\begin{tabular}{l}
|\ifchilddoc\||else|\\
|\addtocounter{page}{-1}|\\
\textit{code for title page}\\
|\newpage|\\
|\||fi|
\end{tabular}
\end{center}
%
A banner page for the child documents can be generated by:
%
\begin{center}
\begin{tabular}{l}
|\ifchilddoc|\\
|\addtocounter{page}{-1}|\\
\textit{code for banner page}\\
|\newpage|\\
|\||fi|
\end{tabular}
\end{center}
%
Here one could write a message such as:
\begin{center}
|This is the part \childdocname{} of \childdocjob{}.|
\end{center}

%%%%%%%%%%%%%%%%%%%%%%%%%%%%%%%%%%%%%%%%%%%%%%%%%%%%%%%%%%%%%%%%%%%%%%%%%%%%%%%%
\subsection{Flags}
\label{sec:flags}

The package makes it easy to generate different versions
of the main or child documents.
To this end compilation flags can be defined
and assigned different default values.
They will be particularly useful in conjunction
with the forwarding mechanism described in \secref{sec:forward}.

For example, it may be useful to have a flag |\version|
which can be set to |draft| or |final|.
The document source will contain some conditional code
depending on the value of |\version|.
Suppose further, the flag should default to |final| for the main file
and to |draft| for child files
which is a natural assignment for editing the document.
This is achieved by placing the following code
in the preamble of the main document
(below the |\childdocmain| directive):
%
\begin{center}
\begin{tabular}{l}
|\ifchilddoc|\\
|\providecommand{\version}{draft}|\\
|\||else|\\
|\providecommand{\version}{final}|\\
|\||fi|
\end{tabular}
\end{center}
%
The definition by |\providecommand| makes sure
that previous definitions are not overwritten.
Further statements |\providecommand{\version}{...}|
can thus be added before the above code to override it.

For the main file, one might add a line
(between |\childdocmain| and the above block)
%
\begin{center}
|%\ifchilddoc\||else\providecommand{\version}{draft}\||fi|
\end{center}
%
which can be uncommented to produce a draft version.
Likewise one can add a line to the very top of a child file
(above the |\childdocof{|\textit{main}|}| directive)
%
\begin{center}
|%\providecommand{\version}{final}|
\end{center}
%
which can be uncommented to produce the final version of this child document.

%%%%%%%%%%%%%%%%%%%%%%%%%%%%%%%%%%%%%%%%%%%%%%%%%%%%%%%%%%%%%%%%%%%%%%%%%%%%%%%%
\subsection{Forwarding}
\label{sec:forward}

Different versions of the main or child documents
using compilation flags as described in \secref{sec:flags}
can be (permanently) stored in different files
for convenient compilation, viewing and distribution.
To this end, the package defines a command
to pass on compilation to a different file:

%%%%%%%%%%%%%%%%%%%%%%%%%%%%%%%%%%%%%%%%
\DescribeMacro{\childdocforward}
The command |\childdocforward| redirects processing to
another source file:
%
\begin{center}
\begin{tabular}{l}
|\input{childdoc.def}|\\
|\childdocforward[|\textit{main}|]{|\textit{dest}|}|\\
\end{tabular}
\end{center}
%
The argument \textit{dest} is the destination file
(without extension).
It should be the main file or one of the child files.
Note that further \textsf{childdoc} directives
such as |\childdocof| and |\childdocforward|
in the indicated file will be processed in this form.
The optional argument \textit{main}
passes on directly to the main file \textit{main}
while pretending to compile the child \textit{dest}.
This form behaves as if \textit{dest}
issues |\childdocof{|\textit{main}|}| right away,
and no further \textsf{childdoc} directives will be processed.

%%%%%%%%%%%%%%%%%%%%%%%%%%%%%%%%%%%%%%%%
\DescribeMacro{\...prefix}
In the alternative form |\childdocforwardprefix|,
%
\begin{center}
\begin{tabular}{l}
|\input{childdoc.def}|\\
|\childdocforwardprefix[|\textit{main}|]{|\textit{prefix}|}{|\textit{dest}|}|
\end{tabular}
\end{center}
%
the destination file is determined by a pattern
depending on the current file:
To make this work, the current file must be called
`{\textit{prefix}\hspace{0.2em}\textit{suffix}}'
with \textit{prefix} matching precisely the argument.
Processing is then passed on to the file
`{\textit{dest}\hspace{0.2em}\textit{suffix}}'.
Surely, the same effect is achieved by
directly specifying the
argument `{\textit{dest}\hspace{0.2em}\textit{suffix}}'
in the first form.
However, that requires to set up a different file
for each child. With the alternative form of the command
all these files can have exactly the same content
which simplifies setting them up and maintaining them.

For example, the following file |draft.tex|
with a compilation flag |\version| as described in \secref{sec:flags}
compiles the main document as a draft:
%
\begin{center}
\begin{tabular}{l}
|\def\version{draft}|\\
|\input{childdoc.def}|\\
|\childdocforward{|\textit{main}|}|
\end{tabular}
\end{center}
%
Likewise, the following files |final|\textit{nn}|.tex|
compile the final version of the child document
|child|\textit{nn}|.tex|:
%
\begin{center}
\begin{tabular}{l}
|\def\version{final}|\\
|\input{childdoc.def}|\\
|\childdocforwardprefix{final}{child}|
\end{tabular}
\end{center}
%

Note that when several versions of a main file and/or of each child file
are to be generated, it may be convenient to set up a |Makefile| or
shell script to automatise the process.

%%%%%%%%%%%%%%%%%%%%%%%%%%%%%%%%%%%%%%%%%%%%%%%%%%%%%%%%%%%%%%%%%%%%%%%%%%%%%%%%
\subsection{Command Line Processing}
\label{sec:commandline}

The effect of redirection files can also be achieved by invoking
the \LaTeX{} compiler with a more elaborate command line.
Most conveniently this should be done as part
of a shell script or a |Makefile|.

When using \textsf{childdoc} in the main file, the following
command lines effectively perform a redirection
(note that depending on the shell being used,
backslashes may have to be doubled: `|\|' $\to$ `|\\|'):
%
\begin{center}
|... -jobname "|\textit{target}|" |\\|"|[\textit{flags}]%
|\input{childdoc.def}\childdocforward[|\textit{main}|]{|\textit{dest}|}"|
\end{center}
%
Here \textit{target} is the name of the output file,
\textit{main} is the name of the main file
and \textit{dest} is the name of the main or child file to be processed
(all filenames without extensions).
The optional argument \textit{main} can be omitted
if \textit{main} matches \textit{dest}.
Optionally, compilation \textit{flags} can be defined via |\def| commands.
This command line makes the \TeX{} engine believe
it is compiling the file \textit{target}
whose content is specified as the latter parameter.
The provided code then forwards the processing to
\textit{main} or \textit{dest} as described in \secref{sec:forward}.

%%%%%%%%%%%%%%%%%%%%%%%%%%%%%%%%%%%%%%%%%%%%%%%%%%%%%%%%%%%%%%%%%%%%%%%%%%%%%%%%
\subsection{Include by Input}
\label{sec:input}

Including child documents by |\include| has some restrictions by design.
Most notably, the content of a child document always occupies
its own set of pages; pages cannot be shared between child documents.
Usually, this behaviour makes perfect sense
because each child document contain an essential part of the document.
However, in some situations it may be desirable to compose
a document from a collection of parts
without having mandatory page breaks between then.
For this case, the package
provides a mechanism to include parts
by |\input| which can also be processed individually.
However, by construction this mechanism
requires manual handling of the content to be output.

%%%%%%%%%%%%%%%%%%%%%%%%%%%%%%%%%%%%%%%%
\DescribeMacro{\ifchilddocmanual}
The main file should be prepared as usual, see \secref{sec:include}.
However, the document body must make a distinction
between processing of an individual part and of the main document, e.g.:
%
\begin{center}
\begin{tabular}{l}
|\ifchilddocmanual|\\
|\input{\childdocname}|\\
|\||else|\\
\textit{document body with }|\input{|\textit{part}|}|\\
|\||fi|
\end{tabular}
\end{center}
%
The conditional |\ifchilddocmanual| is true whenever
a part to be included by |\input| is being compiled,
and the name of the part is stored in |\childdocname|.

%%%%%%%%%%%%%%%%%%%%%%%%%%%%%%%%%%%%%%%%
\DescribeMacro{\childdocby}
Each part to be included by |\input| should start with:
%
\begin{center}
\begin{tabular}{l}
|\input{childdoc.def}|\\
|\childdocby{|\textit{main}|}|\\
\end{tabular}
\end{center}
%
The directive |\childdocby| is similar to |\childdocof|
described in \secref{sec:include},
but the subsequent selection of content must be done manually.
To that end, both |\ifchilddoc| and |\ifchilddocmanual|
will be true upon processing of a part,
and the name of the part is stored in |\childdocname|.
Note that |\jobname| will be set to the filename of the current part
so that each part receives an individual |.aux| file
that does not interfere with the |.aux| file(s) of the main document.
This behaviour can be altered by the alternative form
|\childdocby[*]{|\textit{main}|}| (with a non-empty optional argument)
which uses the |.aux| file of the main document
by setting |\jobname| to \textit{main}.

%%%%%%%%%%%%%%%%%%%%%%%%%%%%%%%%%%%%%%%%%%%%%%%%%%%%%%%%%%%%%%%%%%%%%%%%%%%%%%%%
\subsection{Driver Development}
\label{sec:driver}

The \textsf{childdoc} mechanism can also be use for the development
of definition files such as \LaTeX{} styles or classes.
This case differs from the above setup with multiple parts
included by |\include| in that no |\includeonly| should be invoked.
This can be achieved by starting the include file
(before |\ProvidesPackage|) with:
%
\begin{center}
\begin{tabular}{l}
|\input{childdoc.def}|\\
|\childdocforward{|\textit{main}|}|\\
\end{tabular}
\end{center}
%
or alternatively with:
%
\begin{center}
\begin{tabular}{l}
|\input{childdoc.def}|\\
|\childdocby{|\textit{main}|}|\\
\end{tabular}
\end{center}
%
Both forms have slightly different effects as described above.
The main file is prepared as usual, see \secref{sec:include}.

%%%%%%%%%%%%%%%%%%%%%%%%%%%%%%%%%%%%%%%%%%%%%%%%%%%%%%%%%%%%%%%%%%%%%%%%%%%%%%%%
\subsection{Legacy Detection}
\label{sec:detection}

The directive |\childdocmain| in the main file can detect
whether the complete document or merely a child is to be compiled
even without using the directive |\childdocof|.
This method is deprecated because it is less robust
and there is no compelling reason to use it;
it is merely provided for backward compatibility
and it may be removed in future versions.

If the detection mechanism is to be used,
it is mandatory to correctly specify
the filename of the main file as the argument of |\childdocmain|:
%
\begin{center}
\begin{tabular}{l}
|\input{childdoc.def}|\\
|\childdocmain{|\textit{main}|}|\\
\end{tabular}
\end{center}
%
If |\jobname| does not match the argument \textit{main} of |\childdocmain|,
it is assumed that |\jobname| points to the child file to be compiled.
When using |\childdocmain| with the main file specified as argument,
it suffices to start a child file
with just |\input{|\textit{main}|}|
without loading of the package and using |\childdocof|.
If instead all processing is done
with the appropriate \textsf{childdoc} directives,
the argument of \textit{main} of |\childdocmain| can be empty.

An alternative version of the command line processing described
in \secref{sec:commandline} using the detection mechanism reads:
%
\begin{center}
|... -jobname "|\textit{target}|" "|[\textit{flags}]%
[|\def\jobname{|\textit{dest}|}|]|\input{|\textit{main}|}"|
\end{center}

%%%%%%%%%%%%%%%%%%%%%%%%%%%%%%%%%%%%%%%%%%%%%%%%%%%%%%%%%%%%%%%%%%%%%%%%%%%%%%%%
\subsection{Manual Code}
\label{sec:manual}

In case one cannot be certain whether the definitions file |childdoc.def|
is installed on the target \TeX{} distribution
and one prefers not to ship it,
it is conceivable to paste a few relevant commands into the sources.

To that end, drop all statements |\input{childdoc.def}|
and perform the replacements as outlined below.
Instead of |\childdocmain{|\textit{main}|}| add the following code
to the top of the main file:
%
\begin{center}
\begin{tabular}{l}
|\||ifdefined\childdocname\endinput\||fi\newif\ifchilddoc|\\
|\edef\childdocname{\scantokens\expandafter{\jobname\noexpand}}|\\
|\def\childdocmain{|\textit{main}|}\||ifx\childdocmain\childdocname\||else|\\
|\childdoctrue\includeonly{\childdocname}\let\jobname\childdocmain\||fi|\\
\end{tabular}
\end{center}
%
Instead of |\childdocof{|\textit{main}|}| just include the main file
at the top of each child file:
%
\begin{center}
|\input{|\textit{main}|}|
\end{center}
%
A simple redirection |\childdocforward{|\textit{dest}|}| is achieved by:
%
\begin{center}
|\def\jobname{|\textit{dest}|}\input{\jobname}|
\end{center}
%
The redirection with prefix
|\childdocforwardprefix[|\textit{prefix}|]{|\textit{dest}|}|
is accomplished by:
%
\begin{center}
\begin{tabular}{l}
|{\edef\jobname{\scantokens\expandafter{\jobname\noexpand}}|\\
|\def\redirectjob |\textit{prefix}|#1~~~{\gdef\jobname{|\textit{dest}|#1}}|\\
|\expandafter\redirectjob\jobname~~~}\input{\jobname}|
\end{tabular}
\end{center}

In an alternative approach,
child documents can be compiled by a specific command line
without additional code or specific definitions:
%
\begin{center}
|... -jobname "|\textit{target}|" "|[\textit{flags}]%
|\includeonly{|\textit{dest}|}\input{|\textit{main}|}"|
\end{center}
%

%%%%%%%%%%%%%%%%%%%%%%%%%%%%%%%%%%%%%%%%%%%%%%%%%%%%%%%%%%%%%%%%%%%%%%%%%%%%%%%%
%%%%%%%%%%%%%%%%%%%%%%%%%%%%%%%%%%%%%%%%%%%%%%%%%%%%%%%%%%%%%%%%%%%%%%%%%%%%%%%%
\section{Information}

%%%%%%%%%%%%%%%%%%%%%%%%%%%%%%%%%%%%%%%%%%%%%%%%%%%%%%%%%%%%%%%%%%%%%%%%%%%%%%%%
\subsection{Copyright}

Copyright \copyright{} 2017--2018 Niklas Beisert

This work may be distributed and/or modified under the
conditions of the \LaTeX{} Project Public License, either version 1.3
of this license or (at your option) any later version.
The latest version of this license is in
  \url{http://www.latex-project.org/lppl.txt}
and version 1.3 or later is part of all distributions of \LaTeX{}
version 2005/12/01 or later.

This work has the LPPL maintenance status `maintained'.

The Current Maintainer of this work is Niklas Beisert.

This work consists of the files |README.txt|, |childdoc.ins| and |childdoc.dtx|
as well as the derived files |childdoc.def|, |cdocsamp.tex|
with |cdocsch1.tex|, |cdocsch2.tex|, |cdocspt3.tex|, |cdocspt4.tex|,
|cdocsdrf.tex|, |cdocsfn1.tex|, |cdocsfn2.tex|
as well as |childdoc.pdf|.

%%%%%%%%%%%%%%%%%%%%%%%%%%%%%%%%%%%%%%%%%%%%%%%%%%%%%%%%%%%%%%%%%%%%%%%%%%%%%%%%
\subsection{Files and Installation}

The package consists of the files:
%
\begin{center}
\begin{tabular}{ll}
    |README.txt|   & readme file \\
    |childdoc.ins| & installation file \\
    |childdoc.dtx| & source file \\
    |childdoc.def| & definition file \\
    |cdocsamp.tex| & sample main file \\
    |cdocsch1.tex| & sample include file \\
    |cdocsch2.tex| & sample include file \\
    |cdocspt3.tex| & sample part file \\
    |cdocspt4.tex| & sample part file \\
    |cdocsdrf.tex| & sample redirection file \\
    |cdocsfn1.tex| & sample redirection file \\
    |cdocsfn2.tex| & sample redirection file \\
    |childdoc.pdf| & manual
\end{tabular}
\end{center}
%
The distribution consists of the files
|README.txt|, |childdoc.ins| and |childdoc.dtx|.
%
\begin{itemize}
\item
Run (pdf)\LaTeX{} on |childdoc.dtx|
to compile the manual |childdoc.pdf| (this file).
\item
Run \LaTeX{} on |childdoc.ins| to create the definitions file |childdoc.def|
and the sample |cdocsamp.tex| with include files
|cdocsch1.tex|, |cdocsch2.tex|, |cdocspt3.tex|, |cdocspt4.tex|,
|cdocsdrf.tex|, |cdocsfn1.tex|, |cdocsfn2.tex|.
Then copy the file |childdoc.def| to an appropriate directory of your \LaTeX{}
distribution, e.g.\ \textit{texmf-root}|/tex/latex/childdoc|.
\end{itemize}

%%%%%%%%%%%%%%%%%%%%%%%%%%%%%%%%%%%%%%%%%%%%%%%%%%%%%%%%%%%%%%%%%%%%%%%%%%%%%%%%
\subsection{Related CTAN Packages}

There are several other packages which offer a similar functionality:
%
\begin{itemize}
\item
The packages
\href{http://ctan.org/pkg/docmute}{\textsf{docmute}},
\href{http://ctan.org/pkg/includex}{\textsf{includex}} and
\href{http://ctan.org/pkg/standalone}{\textsf{standalone}}
provide commands to include only the document body of
a child file thus allowing both files to be compiled individually.
\item
The packages \href{http://ctan.org/pkg/subdocs}{\textsf{subdocs}}
and \href{http://ctan.org/pkg/subfiles}{\textsf{subfiles}}
provide structures in which the main and child documents can be
encapsulated and allowing them to be compiled individually.
The inclusion mechanism is different from the conventional |\include|.
\item
The package \href{http://ctan.org/pkg/combine}{\textsf{combine}}
is an elaborate solution to combine several documents into one.
\end{itemize}
%
See also the CTAN topic \href{http://ctan.org/topic/subdocs}{\textsf{subdocs}}
for further related packages.
The present package differs from the above solutions in that
a document structure constructed with the conventional |\include| mechanism
just needs two extra commands at the top of every file
such that all constituent files can be compiled individually.

%%%%%%%%%%%%%%%%%%%%%%%%%%%%%%%%%%%%%%%%%%%%%%%%%%%%%%%%%%%%%%%%%%%%%%%%%%%%%%%%
%\subsection{Feature Suggestions}
%
%The following is a list of features which may be useful for future
%versions of this package:
%%
%\begin{itemize}
%\item
%\ldots
%\end{itemize}

%%%%%%%%%%%%%%%%%%%%%%%%%%%%%%%%%%%%%%%%%%%%%%%%%%%%%%%%%%%%%%%%%%%%%%%%%%%%%%%%
\subsection{Revision History}

%%%%%%%%%%%%%%%%%%%%%%%%%%%%%%%%%%%%%%%%
\paragraph{v2.0:} 2018/12/30

\begin{itemize}
\item
immediate forward processing
\item
added |\childdocby| mechanism
\item
manual restructured
\end{itemize}

%%%%%%%%%%%%%%%%%%%%%%%%%%%%%%%%%%%%%%%%
\paragraph{v1.6:} 2018/01/17

\begin{itemize}
\item
application for development of include files
\item
corrections to manual
\end{itemize}

%%%%%%%%%%%%%%%%%%%%%%%%%%%%%%%%%%%%%%%%
\paragraph{v1.5:} 2017/05/21

\begin{itemize}
\item
more complete structuring introduced
\item
|\childdocof| introduced
\item
|\childdoc| renamed to |\childdocmain|
\item
|\childredirect| renamed to |\childdocforward| and |\childdocforwardprefix|
and functionality expanded
\end{itemize}

%%%%%%%%%%%%%%%%%%%%%%%%%%%%%%%%%%%%%%%%
\paragraph{v1.0:} 2017/04/27

\begin{itemize}
\item
manual and install package
\item
first version published on CTAN
\end{itemize}

%%%%%%%%%%%%%%%%%%%%%%%%%%%%%%%%%%%%%%%%
\paragraph{v0.6:} 2017/04/26

\begin{itemize}
\item
redirection mechanism added
\end{itemize}

%%%%%%%%%%%%%%%%%%%%%%%%%%%%%%%%%%%%%%%%
\paragraph{v0.5:} 2017/04/26

\begin{itemize}
\item
functionality in definition file
\end{itemize}


%%%%%%%%%%%%%%%%%%%%%%%%%%%%%%%%%%%%%%%%%%%%%%%%%%%%%%%%%%%%%%%%%%%%%%%%%%%%%%%%
%%%%%%%%%%%%%%%%%%%%%%%%%%%%%%%%%%%%%%%%%%%%%%%%%%%%%%%%%%%%%%%%%%%%%%%%%%%%%%%%
%%%%%%%%%%%%%%%%%%%%%%%%%%%%%%%%%%%%%%%%%%%%%%%%%%%%%%%%%%%%%%%%%%%%%%%%%%%%%%%%
\appendix

\settowidth\MacroIndent{\rmfamily\scriptsize 000\ }

 \DocInput{childdoc.dtx}

\end{document}
%</driver>
% \fi
%
% %%%%%%%%%%%%%%%%%%%%%%%%%%%%%%%%%%%%%%%%%%%%%%%%%%%%%%%%%%%%%%%%%%%%%%%%%%%%%%
% %%%%%%%%%%%%%%%%%%%%%%%%%%%%%%%%%%%%%%%%%%%%%%%%%%%%%%%%%%%%%%%%%%%%%%%%%%%%%%
% \section{Sample}
%\iffalse
%<*samplemain>
%\fi
%
% The following presents a sample document
% with two chapters, two parts, a title page,
% a compile flag as well as three forwarding files to set the flag.
% It consists of eight |.tex| files:
% \begin{center}
% \begin{tabular}{ll}
% |cdocsamp.tex|&main file\\
% |cdocsch1.tex|&include file for chapter 1\\
% |cdocsch2.tex|&include file for chapter 2\\
% |cdocspt3.tex|&include file for part 3\\
% |cdocspt4.tex|&include file for part 4\\
% |cdocsdrf.tex|&forwarding file for main file in draft mode\\
% |cdocsfi1.tex|&forwarding file for final version of chapter 1\\
% |cdocsfi2.tex|&forwarding file for final version of chapter 2\\
% \end{tabular}
% \end{center}
% Each of the eight files can be compiled directly by the \LaTeX{} compiler.
%
% %%%%%%%%%%%%%%%%%%%%%%%%%%%%%%%%%%%%%%
% \paragraph{Main File.}
%
% The main file is called |cdocsamp.tex|.
%
% Load the \textsf{childdoc} definitions and
% declare the filename for the main document:
%    \begin{macrocode}
\input{childdoc.def}
\childdocmain{}
%    \end{macrocode}

% Optional override for |\version| flag:
%    \begin{macrocode}
%%\ifchilddoc\else\providecommand{\version}{draft}\fi
%    \end{macrocode}

% Define the default values for the |\version| flag
% (|final| for the main file and |draft| for childs):
%    \begin{macrocode}
\ifchilddoc
\providecommand{\version}{draft}
\else
\providecommand{\version}{final}
\fi
%    \end{macrocode}

% Load the standard document class:
%    \begin{macrocode}
\documentclass[12pt]{article}
%    \end{macrocode}

% Start the document body:
%    \begin{macrocode}
\begin{document}
%    \end{macrocode}

% Declare a title page.
% Print title, part of document being processed and version flag:
%    \begin{macrocode}
\addtocounter{page}{-1}
\begin{center}
{\LARGE\bfseries{}childdoc example\par}
\vspace{1cm}
\ifchilddoc
\ifchilddocmanual part\else chapter\fi:
`\childdocname' of `\childdocjob'\par
\else
main document: `\childdocjob'\par
\fi
version: \version\par
\end{center}
\newpage
%    \end{macrocode}

% Manually include selected file,
% otherwise process as usual:
%    \begin{macrocode}
\ifchilddocmanual
\section*{part `\childdocname'}
\input{\childdocname}
\else
%    \end{macrocode}

% Include the two chapters:
%    \begin{macrocode}
\include{cdocsch1}
\include{cdocsch2}
%    \end{macrocode}

% Include the two parts unless only chapters should be displayed:
%    \begin{macrocode}
\ifchilddoc\else
\section{part three}
\input{cdocspt3}
\section{part four}
\input{cdocspt4}
\fi
%    \end{macrocode}

% Process as usual until here:
%    \begin{macrocode}
\fi
%    \end{macrocode}

% End of document body:
%    \begin{macrocode}
\end{document}
%    \end{macrocode}
%\iffalse
%</samplemain>
%\fi
%
% %%%%%%%%%%%%%%%%%%%%%%%%%%%%%%%%%%%%%%
% \paragraph{Chapter Include Files.}
%
% The include files are called |cdocsch1.tex| and |cdocsch2.tex|.
%
%\iffalse
%<*samplechap1|samplechap2>
%\fi

% Optional override for |\version| flag:
%    \begin{macrocode}
%%\providecommand{\version}{final}
%    \end{macrocode}

% Include the main document:
%    \begin{macrocode}
\input{childdoc.def}
\childdocof{cdocsamp}
%    \end{macrocode}

%\iffalse
%</samplechap1|samplechap2>
%\fi
%
%\iffalse
%<*samplechap1>
%\fi
% Some text for chapter 1:
%    \begin{macrocode}
\section{one}
some text in chapter one
%    \end{macrocode}

%\iffalse
%</samplechap1>
%\fi
% Some text for chapter 2:
%\iffalse
%<*samplechap2>
%\fi
%    \begin{macrocode}
\section{two}
more text in chapter two
%    \end{macrocode}

%\iffalse
%</samplechap2>
%\fi
%
% %%%%%%%%%%%%%%%%%%%%%%%%%%%%%%%%%%%%%%
% \paragraph{Part Include Files.}
%
% The include files are called |cdocspt3.tex| and |cdocspt4.tex|.
%
%\iffalse
%<*samplepart3|samplepart4>
%\fi

% Optional override for |\version| flag:
%    \begin{macrocode}
%%\providecommand{\version}{final}
%    \end{macrocode}

% Include the main document:
%    \begin{macrocode}
\input{childdoc.def}
\childdocby{cdocsamp}
%    \end{macrocode}

%\iffalse
%</samplepart3|samplepart4>
%\fi
%
%\iffalse
%<*samplepart3>
%\fi
% Some text for part 3:
%    \begin{macrocode}
some text in part three
%    \end{macrocode}

%\iffalse
%</samplepart3>
%\fi
% Some text for part 4:
%\iffalse
%<*samplepart4>
%\fi
%    \begin{macrocode}
more text in part four
%    \end{macrocode}

%\iffalse
%</samplepart4>
%\fi
%
% %%%%%%%%%%%%%%%%%%%%%%%%%%%%%%%%%%%%%%
% \paragraph{Forwarding for a Complete Draft.}
%
% The following forwarding file |cdocsdrf.tex|
% compiles the main document in draft mode:
%\iffalse
%<*sampledraft>
%\fi
%    \begin{macrocode}
\def\version{draft}
\input{childdoc.def}
\childdocforward{cdocsamp}
%    \end{macrocode}

%\iffalse
%</sampledraft>
%\fi
%
% %%%%%%%%%%%%%%%%%%%%%%%%%%%%%%%%%%%%%%
% \paragraph{Forwarding for Final Version of the Chapters.}
%
% The following forwarding files |cdocsfn1.tex| and |cdocsfn2.tex|
% (with identical content)
% compile the final versions of the child documents
% |cdocsch1.tex| and |cdocsch2.tex|, respectively:
%\iffalse
%<*samplefinal>
%\fi
%    \begin{macrocode}
\def\version{final}
\input{childdoc.def}
\childdocforwardprefix[cdocsamp]{cdocsfn}{cdocsch}
%    \end{macrocode}

%\iffalse
%</samplefinal>
%\fi
%
% %%%%%%%%%%%%%%%%%%%%%%%%%%%%%%%%%%%%%%
% \paragraph{Command Line Processing.}
%
% The following three command lines generate the output files
% |cdocscld|, |cdocscl1| and |cdocscl2|
% which should be identical to
% |cdocsdrf|, |cdocsch1| and |cdocsfn2|, respectively:
% \begin{center}
% \begin{tabular}{l}
% |latex -jobname cdocscld \|\\
% |  "\def\version{draft}\input{childdoc.def}\childdocforward{cdocsamp}"|\\
% |latex -jobname cdocscl1 \|\\
% |  "\input{childdoc.def}\childdocforward[cdocsamp]{cdocsch1}"|\\
% |latex -jobname cdocscl2 \|\\
% |  "\def\version{final}\input{childdoc.def}\childdocforward{cdocsch2}"|
% \end{tabular}
% \end{center}
% Note that the trailing backslash on each first line
% merely continues the input to the second line
% (for convenient cut ant paste).
% Furthermore, the command |latex| can be replaced by any
% of its alternative versions such as |pdflatex|.
%
% %%%%%%%%%%%%%%%%%%%%%%%%%%%%%%%%%%%%%%%%%%%%%%%%%%%%%%%%%%%%%%%%%%%%%%%%%%%%%%
% %%%%%%%%%%%%%%%%%%%%%%%%%%%%%%%%%%%%%%%%%%%%%%%%%%%%%%%%%%%%%%%%%%%%%%%%%%%%%%
% \section{Implementation}
%\iffalse
%<*package>
%\fi
%
% This section describes the definitions file |childdoc.def|.

% The definitions cannot be loaded using |\usepackage| or |\RequirePackage|
% which has a mechanism to prevent loading a style file more than once.
% When loading the definitions by means of |\input|
% multiple instances have to be prevented manually:
%\iffalse
%This code needs to be before the `\ProvidesFile' directive
%which is defined at the beginning of this file.
%Therefore it is also placed there and commented out here.
%</package>
%<*discard>
%\fi
%    \begin{macrocode}
\ifdefined\childdocmain\endinput\fi
%    \end{macrocode}
%\iffalse
%</discard>
%<*package>
%\fi
%
% \macro{\ifchilddoc}
% \macro{\ifchilddocmanual}
% The conditional |\ifchilddoc| tells whether a
% child (true) or main (false) document is being compiled.
% The conditional |\ifchilddocmanual| tells whether
% the |\includeonly| mechanism is used (false) or
% the selection of child files must be performed manually (true).
% The definitions initialise to false:
%    \begin{macrocode}
\newif\ifchilddoc
\newif\ifchilddocmanual
%    \end{macrocode}

% \macro{\childdocname}
% \macro{\childdocjob}
% The macro |\childdocname| stores the name of the main document
% to be compiled. The macro |\childdocjob| stores the name of
% the document on which the \LaTeX{} compiler was originally invoked.
% The content of |\jobname| cannot be compared
% to filenames specified in the source due to different catcodes.
% The following code rescans |\jobname|, stores the result
% in |\childdocname| and saves a copy in |\childdocjob|:
%    \begin{macrocode}
\edef\childdocname{\scantokens\expandafter{\jobname\noexpand}}
\let\childdocjob\childdocname
%    \end{macrocode}

% \macro{\childdocdisable}
% The macro |\childdocdisable| prevents the main file
% from being processed more than once.
% At this stage, the main document command |\childdocmain|
% is assumed to be called once again where it should do nothing.
% Any subsequent call to it should prevent
% a secondary processing of the main document
% It overwrites the forwarding commands
% |\childdocof| and |\childdocforward|
% with empty macros to prevent further inclusions of the main document:
%    \begin{macrocode}
\newcommand{\childdocdisable}
{
  \renewcommand{\childdocmain}[1]{\renewcommand{\childdocmain}[1]{\endinput}}
  \renewcommand{\childdocof}[1]{}
  \renewcommand{\childdocby}[2][]{}
  \renewcommand{\childdocforward}[2][]{}
  \renewcommand{\childdocdisable}{}
}
%    \end{macrocode}

% \macro{\childdocmain}
% The macro |\childdocmain| is to be called at the top of the main file
% with nothing or the main filename (without extension) as argument.
% First, it breaks loops.
% If the argument is not empty and does not match |\childdocname|
% (which is set by the first inclusion of |childdoc.def|),
% |\ifchilddoc| is set to true, |\includeonly| is applied to the child file
% and |\jobname| is set to the main file
% (for proper handling of |.aux| files):
%    \begin{macrocode}
\newcommand{\childdocmain}[1]
{
  \childdocdisable\childdocmain{}
  \if?#1?\else
    \begingroup
      \def\childdoctmp{#1}
      \ifx\childdoctmp\childdocname
        \def\childdoctmp{}
      \else
        \def\childdoctmp
        {
          \childdoctrue
          \includeonly{\childdocname}
          \def\childdocjob{#1}
          \def\jobname{#1}
        }
      \fi
      \expandafter
    \endgroup
    \childdoctmp
  \fi
}
%    \end{macrocode}

% \macro{\childdocof}
% The command |\childdocof| redirects
% compilation to the main file |#1|.
%    \begin{macrocode}
\newcommand{\childdocof}[1]
{
  \childdocdisable
  \childdoctrue
  \includeonly{\childdocname}
  \def\jobname{#1}
  \def\childdocjob{#1}
  \input{#1}
}
%    \end{macrocode}

% \macro{\childdocby}
% The command |\childdocby| ....
%    \begin{macrocode}
\newcommand{\childdocby}[2][]
{
  \childdocdisable
  \childdoctrue
  \childdocmanualtrue
  \if?#1?\else
    \def\jobname{#2}
  \fi
  \def\childdocjob{#2}
  \input{#2}
  \endinput
}
%    \end{macrocode}

% \macro{\childdocforward}
% The command |\childdocforward| redirects
% compilation to the main file or
% (if the optional argument is given) a child file.
% Parameters are set as if the main file
% or a child file starting with |\childdocof| was compiled.
% Then compilation is handed over to the main file:
%    \begin{macrocode}
\newcommand{\childdocforward}[2][]
{
  \begingroup
    \if?#1?
      \def\childdoctmp
      {
        \def\childdocname{#2}
        \def\childdocjob{#2}
        \def\jobname{#2}
        \input{#2}
        \endinput
      }
    \else
      \def\childdoctmp
      {
        \childdocdisable
        \def\childdocname{#2}
        \childdoctrue
        \includeonly{#2}
        \def\childdocjob{#1}
        \def\jobname{#1}
        \input{#1}
        \endinput
      }
    \fi
    \expandafter
  \endgroup
  \childdoctmp
}
%    \end{macrocode}

% \macro{\childdocforwardprefix}
% The command |\childdocforwardprefix| redirects
% compilation to the main or a child file by means of a pattern.
% The prefix |#1| in the current filename is replaced by |#2|
% and the suffix of the current filename is kept
% (it is assumed that the filename does not contain the substring `|~~~|'
% which is used as a delimiter).
% Compilation is handed over to the new file by |\childdocforward|:
%    \begin{macrocode}
\newcommand{\childdocforwardprefix}[3][]
{
  \begingroup
    \def\childdocextract #2##1~~~{\def\childdoctmp{\childdocforward[#1]{#3##1}}}
    \expandafter\childdocextract\childdocname~~~
    \expandafter
  \endgroup
  \childdoctmp
}
%    \end{macrocode}

% \macro{\childdoc}
% The deprecated macro |\childdoc| is a legacy version of |\childdocmain|:
%    \begin{macrocode}
\newcommand{\childdoc}{\childdocmain}
%    \end{macrocode}

% \macro{\childdocredirect}
% The deprecated macro |\childdocredirect| is a legacy version
% of |\childdocforward| and |\childdocforwardprefix|:
%    \begin{macrocode}
\newcommand{\childdocredirect}[2][]
{
  \begingroup
    \if?#1?
      \def\childdoctmp{\childdocforward{#2}}
    \else
      \def\childdoctmp{\childdocforwardprefix{#1}{#2}}
    \fi
    \expandafter
  \endgroup
  \childdoctmp
}
%    \end{macrocode}

%\iffalse
%</package>
%\fi
%
\endinput
|\\
|\childdocmain{}|\\
\end{tabular}
\end{center}
at the very top of the main \LaTeX{} file,
in particular \emph{before} the |\documentclass| statement!
The argument of |\childdocmain| should be left empty
(but it must be present).

%%%%%%%%%%%%%%%%%%%%%%%%%%%%%%%%%%%%%%%%
\DescribeMacro{\childdocof}
Furthermore, add the commands
\begin{center}
\begin{tabular}{l}
|% \iffalse
%
% childdoc.dtx Copyright (C) 2017-2018 Niklas Beisert
%
% This work may be distributed and/or modified under the
% conditions of the LaTeX Project Public License, either version 1.3
% of this license or (at your option) any later version.
% The latest version of this license is in
%   http://www.latex-project.org/lppl.txt
% and version 1.3 or later is part of all distributions of LaTeX
% version 2005/12/01 or later.
%
% This work has the LPPL maintenance status `maintained'.
%
% The Current Maintainer of this work is Niklas Beisert.
%
% This work consists of the files childdoc.dtx and childdoc.ins
% and the derived files childdoc.def and cdocsamp.tex with
% cdocsch1.tex, cdocsch2.tex, cdocsdrf.tex, cdocsfn1.tex, cdocsfn2.tex.
%
%<package>\ifdefined\childdocmain\endinput\fi
%<package>\ProvidesFile{childdoc.def}[2018/12/30 v2.0 child document driver]
%<samplemain>\ProvidesFile{cdocsamp.tex}[2018/12/30 v2.0 sample for childdoc]
%<*driver>
%\ProvidesFile{childdoc.drv}[2018/12/30 v2.0 childdoc reference manual file]
\PassOptionsToClass{10pt,a4paper}{article}
\documentclass{ltxdoc}

\usepackage[margin=35mm]{geometry}
\usepackage{hyperref}
\usepackage{hyperxmp}
\usepackage[usenames]{color}

\hypersetup{colorlinks=true}
\hypersetup{pdfstartview=FitH}
\hypersetup{pdfpagemode=UseNone}
\hypersetup{pdfsource={}}
\hypersetup{pdflang={en-UK}}
\hypersetup{pdfcopyright={Copyright 2017-2018 Niklas Beisert.
  This work may be distributed and/or modified under the
  conditions of the LaTeX Project Public License, either version 1.3
  of this license or (at your option) any later version.}}
\hypersetup{pdflicenseurl={http://www.latex-project.org/lppl.txt}}
\hypersetup{pdfcontactaddress={ETH Zurich, ITP, HIT K,
  Wolfgang-Pauli-Strasse 27}}
\hypersetup{pdfcontactpostcode={8093}}
\hypersetup{pdfcontactcity={Zurich}}
\hypersetup{pdfcontactcountry={Switzerland}}
\hypersetup{pdfcontactemail={nbeisert@itp.phys.ethz.ch}}
\hypersetup{pdfcontacturl={http://people.phys.ethz.ch/\xmptilde nbeisert/}}

\newcommand{\secref}[1]{\hyperref[#1]{section \ref*{#1}}}

\parskip1ex
\parindent0pt
\let\olditemize\itemize
\def\itemize{\olditemize\parskip0pt}

\begin{document}

\title{The \textsf{childdoc} Package}
\hypersetup{pdftitle={The childdoc Package}}
\author{Niklas Beisert\\[2ex]
  Institut f\"ur Theoretische Physik\\
  Eidgen\"ossische Technische Hochschule Z\"urich\\
  Wolfgang-Pauli-Strasse 27, 8093 Z\"urich, Switzerland\\[1ex]
  \href{mailto:nbeisert@itp.phys.ethz.ch}
  {\texttt{nbeisert@itp.phys.ethz.ch}}}
\hypersetup{pdfauthor={Niklas Beisert}}
\hypersetup{pdfsubject={Manual for the LaTeX2e Package childdoc}}
\date{30 December 2018, \textsf{v2.0}}
\maketitle

\begin{abstract}\noindent
\textsf{childdoc} is a \LaTeXe{} package
that enables the direct compilation
of document sections included by |\include|
to individual files.
\end{abstract}

\begingroup
\parskip0ex
\tableofcontents
\endgroup

%%%%%%%%%%%%%%%%%%%%%%%%%%%%%%%%%%%%%%%%%%%%%%%%%%%%%%%%%%%%%%%%%%%%%%%%%%%%%%%%
%%%%%%%%%%%%%%%%%%%%%%%%%%%%%%%%%%%%%%%%%%%%%%%%%%%%%%%%%%%%%%%%%%%%%%%%%%%%%%%%
\section{Introduction}

\LaTeX{} provides a mechanism to structure a large document (such as a book)
into a main file and several child files (containing the chapters)
using the |\include| command.
This mechanism is beneficial for documents
which span hundreds of pages in order to
make the source file(s) more manageable.
Moreover, compilation can be restricted to
selected child files by means of the |\includeonly| command.
The latter feature can be used to reduce the compilation time while editing
(this was significantly more useful in the earlier days of \LaTeX{})
or to generate a smaller document which is easier to navigate.
Another application of |\includeonly| is to generate
documents consisting of selected parts of the complete document.

However, there are a few drawbacks of the plain |\include| mechanism:
\begin{itemize}
\item
The child files cannot be compiled on their own,
they can only be compiled via the main file.
A naive editing environment
(such as a text editor with an option
to have the current file processed by \LaTeX)
may require one to switch to the main file before compiling;
attempting to compile the child file produces errors.
\item
The main file must be modified (each time)
to adjust the |\includeonly| command
to the present needs. This easily leaves the main file in a messy state.
\item
The generated document will always carry the filename
of the main document. This is inconvenient if
several child files are to be compiled and
to be kept for distribution.
\end{itemize}

The present package provides a simple interface
to make child files individually compilable by \LaTeX{}.
Compiling a child file then has the same effect as compiling
the main file with an |\includeonly| command
to select the appropriate child.
Moreover the generated document will carry the name of the child
rather than the main file.
This resolves all three above issues.

This feature is meant to make the editing of books,
thesis documents and lecture notes somewhat more convenient.
However, the package can also be used efficiently for
composing a series of documents (such as exercise sheets)
which are typically distributed individually.
It then assists the author in generating the individual documents
(potentially in different versions)
as well as a document containing the collected series.
Another application is in developing style files
or other kinds of included material
where compilation of the style file could redirect
to a sample or test file.

%%%%%%%%%%%%%%%%%%%%%%%%%%%%%%%%%%%%%%%%%%%%%%%%%%%%%%%%%%%%%%%%%%%%%%%%%%%%%%%%
%%%%%%%%%%%%%%%%%%%%%%%%%%%%%%%%%%%%%%%%%%%%%%%%%%%%%%%%%%%%%%%%%%%%%%%%%%%%%%%%
\section{Usage}

First of all, the package \textsf{childdoc} is \emph{not} a standard
\LaTeXe{} |.sty| style file! Therefore it needs to be invoked in
a non-standard way.

%%%%%%%%%%%%%%%%%%%%%%%%%%%%%%%%%%%%%%%%%%%%%%%%%%%%%%%%%%%%%%%%%%%%%%%%%%%%%%%%
\subsection{Included Files}
\label{sec:include}

%%%%%%%%%%%%%%%%%%%%%%%%%%%%%%%%%%%%%%%%
\DescribeMacro{\childdocmain}
To use the package, add the commands
\begin{center}
\begin{tabular}{l}
|\input{childdoc.def}|\\
|\childdocmain{}|\\
\end{tabular}
\end{center}
at the very top of the main \LaTeX{} file,
in particular \emph{before} the |\documentclass| statement!
The argument of |\childdocmain| should be left empty
(but it must be present).

%%%%%%%%%%%%%%%%%%%%%%%%%%%%%%%%%%%%%%%%
\DescribeMacro{\childdocof}
Furthermore, add the commands
\begin{center}
\begin{tabular}{l}
|\input{childdoc.def}|\\
|\childdocof{|\textit{main}|}|\\
\end{tabular}
\end{center}
at the top of every child file \textit{child}
which is included by |\include{|\textit{child}|}|
from within the main file
(or at least for those files to be compiled individually).
The argument \textit{main} must be the filename of the main file.

There are a couple of
considerations in setting up the main and child documents:

%%%%%%%%%%%%%%%%%%%%%%%%%%%%%%%%%%%%%%%%
\paragraph{Restrictions.}

Please note the following restrictions:
\begin{itemize}
\item
|\childdocmain| must be called with one argument \textit{main}
to ensure compatibility with earlier version of the package.
It must either be empty (|\childdocmain{}|)
or precisely match the filename of the main file in which it is specified.
See \secref{sec:detection} for further information.
\item
The filename \textit{main} must be specified without the |.tex| extension.
\item
The filename \textit{main} is case sensitive
(even in case-insensitive file systems)
due to internal string comparison.
\item
The argument \textit{main} should be fully expanded, it cannot be a macro.
\item
Subdirectories and special characters should be avoided in filenames.
\item
The command |\childdocmain{|\textit{main}|}| must be followed by a whitespace.
It should not be followed immediately by another command
or by a comment mark `|%|'.
This is because the \TeX{} parser reads the token immediately following
the argument of |\childdocmain| and puts it
at the beginning of every child section;
however, a white\-space is ignored.
\end{itemize}

%%%%%%%%%%%%%%%%%%%%%%%%%%%%%%%%%%%%%%%%
\paragraph{Content of Main File.}

It is advisable to place all content in the child files included by |\include|.
Any output contained in the main file will appear in all child documents
unless suppressed manually;
it cannot be suppressed automatically by the |\includeonly| directive
and thus should normally be avoided.
A method to include some content in the main file
by means of conditional processing is described in \secref{sec:conditional}.

%%%%%%%%%%%%%%%%%%%%%%%%%%%%%%%%%%%%%%%%
\paragraph{Page Numbering.}

When only a part of the document is compiled,
the appropriate numbering of pages
(as well as other status parameters)
is determined from the |.aux| files.
The latter contain information from previous passes.
However this information needs to propagate through
all intermediate child documents.
Therefore the page numbering in child documents may well
be inconsistent until the complete document is compiled at least once.

A useful (if unconventional) way to always ensure a consistent
page numbering is to restart the numbering in each child document
and denote the pages by `\textit{child}|.|\textit{page}'
where \textit{child} represents the chapter/section number of the child file.
This can be achieved by the command
|\numberwithin{page}{|\textit{child}|}|
of the \textsf{amsmath} package
where \textit{child} can be |chapter| or |section|
depending on the chosen structuring.
Alternatively, one can modify the macro |\thepage| appropriately
and reset the counter |page| at the start of each child file.

%%%%%%%%%%%%%%%%%%%%%%%%%%%%%%%%%%%%%%%%%%%%%%%%%%%%%%%%%%%%%%%%%%%%%%%%%%%%%%%%
\subsection{Conditional Processing}
\label{sec:conditional}

The package provides a mechanism to compile different versions
of a document. To customise the versions further some conditional processing
can come in handy to distinguish which version is being compiled.
The package provides two macros to describe the compilation context:

%%%%%%%%%%%%%%%%%%%%%%%%%%%%%%%%%%%%%%%%
\DescribeMacro{\ifchilddoc}
The conditional |\ifchilddoc| distinguishes between the compilation of
child documents and the main document:
%
\begin{center}
|\ifchilddoc |\textit{child-code}| |[|\||else |\textit{main-code}]| \||fi|
\end{center}

%%%%%%%%%%%%%%%%%%%%%%%%%%%%%%%%%%%%%%%%
\DescribeMacro{\childdocname}
\DescribeMacro{\childdocjob}
The macro |\childdocname| contains the filename (without extension)
of the main or child file being processed.
Note that |\childdocjob| will always contain the name of the main file.

%%%%%%%%%%%%%%%%%%%%%%%%%%%%%%%%%%%%%%%%
\paragraph{Title Page.}

Conditional processing can be used to include a title or banner page
in the main document when proper precautions are taken.
Importantly, the code in the main file should ensure that the page counter
(as well as other status parameters which are stored in the |.aux| files)
takes the same value after the conditional processing.
Otherwise the page numbers may take divergent values
depending on which part is compiled.

For example, a title page could be declared by:
%
\begin{center}
\begin{tabular}{l}
|\ifchilddoc\||else|\\
|\addtocounter{page}{-1}|\\
\textit{code for title page}\\
|\newpage|\\
|\||fi|
\end{tabular}
\end{center}
%
A banner page for the child documents can be generated by:
%
\begin{center}
\begin{tabular}{l}
|\ifchilddoc|\\
|\addtocounter{page}{-1}|\\
\textit{code for banner page}\\
|\newpage|\\
|\||fi|
\end{tabular}
\end{center}
%
Here one could write a message such as:
\begin{center}
|This is the part \childdocname{} of \childdocjob{}.|
\end{center}

%%%%%%%%%%%%%%%%%%%%%%%%%%%%%%%%%%%%%%%%%%%%%%%%%%%%%%%%%%%%%%%%%%%%%%%%%%%%%%%%
\subsection{Flags}
\label{sec:flags}

The package makes it easy to generate different versions
of the main or child documents.
To this end compilation flags can be defined
and assigned different default values.
They will be particularly useful in conjunction
with the forwarding mechanism described in \secref{sec:forward}.

For example, it may be useful to have a flag |\version|
which can be set to |draft| or |final|.
The document source will contain some conditional code
depending on the value of |\version|.
Suppose further, the flag should default to |final| for the main file
and to |draft| for child files
which is a natural assignment for editing the document.
This is achieved by placing the following code
in the preamble of the main document
(below the |\childdocmain| directive):
%
\begin{center}
\begin{tabular}{l}
|\ifchilddoc|\\
|\providecommand{\version}{draft}|\\
|\||else|\\
|\providecommand{\version}{final}|\\
|\||fi|
\end{tabular}
\end{center}
%
The definition by |\providecommand| makes sure
that previous definitions are not overwritten.
Further statements |\providecommand{\version}{...}|
can thus be added before the above code to override it.

For the main file, one might add a line
(between |\childdocmain| and the above block)
%
\begin{center}
|%\ifchilddoc\||else\providecommand{\version}{draft}\||fi|
\end{center}
%
which can be uncommented to produce a draft version.
Likewise one can add a line to the very top of a child file
(above the |\childdocof{|\textit{main}|}| directive)
%
\begin{center}
|%\providecommand{\version}{final}|
\end{center}
%
which can be uncommented to produce the final version of this child document.

%%%%%%%%%%%%%%%%%%%%%%%%%%%%%%%%%%%%%%%%%%%%%%%%%%%%%%%%%%%%%%%%%%%%%%%%%%%%%%%%
\subsection{Forwarding}
\label{sec:forward}

Different versions of the main or child documents
using compilation flags as described in \secref{sec:flags}
can be (permanently) stored in different files
for convenient compilation, viewing and distribution.
To this end, the package defines a command
to pass on compilation to a different file:

%%%%%%%%%%%%%%%%%%%%%%%%%%%%%%%%%%%%%%%%
\DescribeMacro{\childdocforward}
The command |\childdocforward| redirects processing to
another source file:
%
\begin{center}
\begin{tabular}{l}
|\input{childdoc.def}|\\
|\childdocforward[|\textit{main}|]{|\textit{dest}|}|\\
\end{tabular}
\end{center}
%
The argument \textit{dest} is the destination file
(without extension).
It should be the main file or one of the child files.
Note that further \textsf{childdoc} directives
such as |\childdocof| and |\childdocforward|
in the indicated file will be processed in this form.
The optional argument \textit{main}
passes on directly to the main file \textit{main}
while pretending to compile the child \textit{dest}.
This form behaves as if \textit{dest}
issues |\childdocof{|\textit{main}|}| right away,
and no further \textsf{childdoc} directives will be processed.

%%%%%%%%%%%%%%%%%%%%%%%%%%%%%%%%%%%%%%%%
\DescribeMacro{\...prefix}
In the alternative form |\childdocforwardprefix|,
%
\begin{center}
\begin{tabular}{l}
|\input{childdoc.def}|\\
|\childdocforwardprefix[|\textit{main}|]{|\textit{prefix}|}{|\textit{dest}|}|
\end{tabular}
\end{center}
%
the destination file is determined by a pattern
depending on the current file:
To make this work, the current file must be called
`{\textit{prefix}\hspace{0.2em}\textit{suffix}}'
with \textit{prefix} matching precisely the argument.
Processing is then passed on to the file
`{\textit{dest}\hspace{0.2em}\textit{suffix}}'.
Surely, the same effect is achieved by
directly specifying the
argument `{\textit{dest}\hspace{0.2em}\textit{suffix}}'
in the first form.
However, that requires to set up a different file
for each child. With the alternative form of the command
all these files can have exactly the same content
which simplifies setting them up and maintaining them.

For example, the following file |draft.tex|
with a compilation flag |\version| as described in \secref{sec:flags}
compiles the main document as a draft:
%
\begin{center}
\begin{tabular}{l}
|\def\version{draft}|\\
|\input{childdoc.def}|\\
|\childdocforward{|\textit{main}|}|
\end{tabular}
\end{center}
%
Likewise, the following files |final|\textit{nn}|.tex|
compile the final version of the child document
|child|\textit{nn}|.tex|:
%
\begin{center}
\begin{tabular}{l}
|\def\version{final}|\\
|\input{childdoc.def}|\\
|\childdocforwardprefix{final}{child}|
\end{tabular}
\end{center}
%

Note that when several versions of a main file and/or of each child file
are to be generated, it may be convenient to set up a |Makefile| or
shell script to automatise the process.

%%%%%%%%%%%%%%%%%%%%%%%%%%%%%%%%%%%%%%%%%%%%%%%%%%%%%%%%%%%%%%%%%%%%%%%%%%%%%%%%
\subsection{Command Line Processing}
\label{sec:commandline}

The effect of redirection files can also be achieved by invoking
the \LaTeX{} compiler with a more elaborate command line.
Most conveniently this should be done as part
of a shell script or a |Makefile|.

When using \textsf{childdoc} in the main file, the following
command lines effectively perform a redirection
(note that depending on the shell being used,
backslashes may have to be doubled: `|\|' $\to$ `|\\|'):
%
\begin{center}
|... -jobname "|\textit{target}|" |\\|"|[\textit{flags}]%
|\input{childdoc.def}\childdocforward[|\textit{main}|]{|\textit{dest}|}"|
\end{center}
%
Here \textit{target} is the name of the output file,
\textit{main} is the name of the main file
and \textit{dest} is the name of the main or child file to be processed
(all filenames without extensions).
The optional argument \textit{main} can be omitted
if \textit{main} matches \textit{dest}.
Optionally, compilation \textit{flags} can be defined via |\def| commands.
This command line makes the \TeX{} engine believe
it is compiling the file \textit{target}
whose content is specified as the latter parameter.
The provided code then forwards the processing to
\textit{main} or \textit{dest} as described in \secref{sec:forward}.

%%%%%%%%%%%%%%%%%%%%%%%%%%%%%%%%%%%%%%%%%%%%%%%%%%%%%%%%%%%%%%%%%%%%%%%%%%%%%%%%
\subsection{Include by Input}
\label{sec:input}

Including child documents by |\include| has some restrictions by design.
Most notably, the content of a child document always occupies
its own set of pages; pages cannot be shared between child documents.
Usually, this behaviour makes perfect sense
because each child document contain an essential part of the document.
However, in some situations it may be desirable to compose
a document from a collection of parts
without having mandatory page breaks between then.
For this case, the package
provides a mechanism to include parts
by |\input| which can also be processed individually.
However, by construction this mechanism
requires manual handling of the content to be output.

%%%%%%%%%%%%%%%%%%%%%%%%%%%%%%%%%%%%%%%%
\DescribeMacro{\ifchilddocmanual}
The main file should be prepared as usual, see \secref{sec:include}.
However, the document body must make a distinction
between processing of an individual part and of the main document, e.g.:
%
\begin{center}
\begin{tabular}{l}
|\ifchilddocmanual|\\
|\input{\childdocname}|\\
|\||else|\\
\textit{document body with }|\input{|\textit{part}|}|\\
|\||fi|
\end{tabular}
\end{center}
%
The conditional |\ifchilddocmanual| is true whenever
a part to be included by |\input| is being compiled,
and the name of the part is stored in |\childdocname|.

%%%%%%%%%%%%%%%%%%%%%%%%%%%%%%%%%%%%%%%%
\DescribeMacro{\childdocby}
Each part to be included by |\input| should start with:
%
\begin{center}
\begin{tabular}{l}
|\input{childdoc.def}|\\
|\childdocby{|\textit{main}|}|\\
\end{tabular}
\end{center}
%
The directive |\childdocby| is similar to |\childdocof|
described in \secref{sec:include},
but the subsequent selection of content must be done manually.
To that end, both |\ifchilddoc| and |\ifchilddocmanual|
will be true upon processing of a part,
and the name of the part is stored in |\childdocname|.
Note that |\jobname| will be set to the filename of the current part
so that each part receives an individual |.aux| file
that does not interfere with the |.aux| file(s) of the main document.
This behaviour can be altered by the alternative form
|\childdocby[*]{|\textit{main}|}| (with a non-empty optional argument)
which uses the |.aux| file of the main document
by setting |\jobname| to \textit{main}.

%%%%%%%%%%%%%%%%%%%%%%%%%%%%%%%%%%%%%%%%%%%%%%%%%%%%%%%%%%%%%%%%%%%%%%%%%%%%%%%%
\subsection{Driver Development}
\label{sec:driver}

The \textsf{childdoc} mechanism can also be use for the development
of definition files such as \LaTeX{} styles or classes.
This case differs from the above setup with multiple parts
included by |\include| in that no |\includeonly| should be invoked.
This can be achieved by starting the include file
(before |\ProvidesPackage|) with:
%
\begin{center}
\begin{tabular}{l}
|\input{childdoc.def}|\\
|\childdocforward{|\textit{main}|}|\\
\end{tabular}
\end{center}
%
or alternatively with:
%
\begin{center}
\begin{tabular}{l}
|\input{childdoc.def}|\\
|\childdocby{|\textit{main}|}|\\
\end{tabular}
\end{center}
%
Both forms have slightly different effects as described above.
The main file is prepared as usual, see \secref{sec:include}.

%%%%%%%%%%%%%%%%%%%%%%%%%%%%%%%%%%%%%%%%%%%%%%%%%%%%%%%%%%%%%%%%%%%%%%%%%%%%%%%%
\subsection{Legacy Detection}
\label{sec:detection}

The directive |\childdocmain| in the main file can detect
whether the complete document or merely a child is to be compiled
even without using the directive |\childdocof|.
This method is deprecated because it is less robust
and there is no compelling reason to use it;
it is merely provided for backward compatibility
and it may be removed in future versions.

If the detection mechanism is to be used,
it is mandatory to correctly specify
the filename of the main file as the argument of |\childdocmain|:
%
\begin{center}
\begin{tabular}{l}
|\input{childdoc.def}|\\
|\childdocmain{|\textit{main}|}|\\
\end{tabular}
\end{center}
%
If |\jobname| does not match the argument \textit{main} of |\childdocmain|,
it is assumed that |\jobname| points to the child file to be compiled.
When using |\childdocmain| with the main file specified as argument,
it suffices to start a child file
with just |\input{|\textit{main}|}|
without loading of the package and using |\childdocof|.
If instead all processing is done
with the appropriate \textsf{childdoc} directives,
the argument of \textit{main} of |\childdocmain| can be empty.

An alternative version of the command line processing described
in \secref{sec:commandline} using the detection mechanism reads:
%
\begin{center}
|... -jobname "|\textit{target}|" "|[\textit{flags}]%
[|\def\jobname{|\textit{dest}|}|]|\input{|\textit{main}|}"|
\end{center}

%%%%%%%%%%%%%%%%%%%%%%%%%%%%%%%%%%%%%%%%%%%%%%%%%%%%%%%%%%%%%%%%%%%%%%%%%%%%%%%%
\subsection{Manual Code}
\label{sec:manual}

In case one cannot be certain whether the definitions file |childdoc.def|
is installed on the target \TeX{} distribution
and one prefers not to ship it,
it is conceivable to paste a few relevant commands into the sources.

To that end, drop all statements |\input{childdoc.def}|
and perform the replacements as outlined below.
Instead of |\childdocmain{|\textit{main}|}| add the following code
to the top of the main file:
%
\begin{center}
\begin{tabular}{l}
|\||ifdefined\childdocname\endinput\||fi\newif\ifchilddoc|\\
|\edef\childdocname{\scantokens\expandafter{\jobname\noexpand}}|\\
|\def\childdocmain{|\textit{main}|}\||ifx\childdocmain\childdocname\||else|\\
|\childdoctrue\includeonly{\childdocname}\let\jobname\childdocmain\||fi|\\
\end{tabular}
\end{center}
%
Instead of |\childdocof{|\textit{main}|}| just include the main file
at the top of each child file:
%
\begin{center}
|\input{|\textit{main}|}|
\end{center}
%
A simple redirection |\childdocforward{|\textit{dest}|}| is achieved by:
%
\begin{center}
|\def\jobname{|\textit{dest}|}\input{\jobname}|
\end{center}
%
The redirection with prefix
|\childdocforwardprefix[|\textit{prefix}|]{|\textit{dest}|}|
is accomplished by:
%
\begin{center}
\begin{tabular}{l}
|{\edef\jobname{\scantokens\expandafter{\jobname\noexpand}}|\\
|\def\redirectjob |\textit{prefix}|#1~~~{\gdef\jobname{|\textit{dest}|#1}}|\\
|\expandafter\redirectjob\jobname~~~}\input{\jobname}|
\end{tabular}
\end{center}

In an alternative approach,
child documents can be compiled by a specific command line
without additional code or specific definitions:
%
\begin{center}
|... -jobname "|\textit{target}|" "|[\textit{flags}]%
|\includeonly{|\textit{dest}|}\input{|\textit{main}|}"|
\end{center}
%

%%%%%%%%%%%%%%%%%%%%%%%%%%%%%%%%%%%%%%%%%%%%%%%%%%%%%%%%%%%%%%%%%%%%%%%%%%%%%%%%
%%%%%%%%%%%%%%%%%%%%%%%%%%%%%%%%%%%%%%%%%%%%%%%%%%%%%%%%%%%%%%%%%%%%%%%%%%%%%%%%
\section{Information}

%%%%%%%%%%%%%%%%%%%%%%%%%%%%%%%%%%%%%%%%%%%%%%%%%%%%%%%%%%%%%%%%%%%%%%%%%%%%%%%%
\subsection{Copyright}

Copyright \copyright{} 2017--2018 Niklas Beisert

This work may be distributed and/or modified under the
conditions of the \LaTeX{} Project Public License, either version 1.3
of this license or (at your option) any later version.
The latest version of this license is in
  \url{http://www.latex-project.org/lppl.txt}
and version 1.3 or later is part of all distributions of \LaTeX{}
version 2005/12/01 or later.

This work has the LPPL maintenance status `maintained'.

The Current Maintainer of this work is Niklas Beisert.

This work consists of the files |README.txt|, |childdoc.ins| and |childdoc.dtx|
as well as the derived files |childdoc.def|, |cdocsamp.tex|
with |cdocsch1.tex|, |cdocsch2.tex|, |cdocspt3.tex|, |cdocspt4.tex|,
|cdocsdrf.tex|, |cdocsfn1.tex|, |cdocsfn2.tex|
as well as |childdoc.pdf|.

%%%%%%%%%%%%%%%%%%%%%%%%%%%%%%%%%%%%%%%%%%%%%%%%%%%%%%%%%%%%%%%%%%%%%%%%%%%%%%%%
\subsection{Files and Installation}

The package consists of the files:
%
\begin{center}
\begin{tabular}{ll}
    |README.txt|   & readme file \\
    |childdoc.ins| & installation file \\
    |childdoc.dtx| & source file \\
    |childdoc.def| & definition file \\
    |cdocsamp.tex| & sample main file \\
    |cdocsch1.tex| & sample include file \\
    |cdocsch2.tex| & sample include file \\
    |cdocspt3.tex| & sample part file \\
    |cdocspt4.tex| & sample part file \\
    |cdocsdrf.tex| & sample redirection file \\
    |cdocsfn1.tex| & sample redirection file \\
    |cdocsfn2.tex| & sample redirection file \\
    |childdoc.pdf| & manual
\end{tabular}
\end{center}
%
The distribution consists of the files
|README.txt|, |childdoc.ins| and |childdoc.dtx|.
%
\begin{itemize}
\item
Run (pdf)\LaTeX{} on |childdoc.dtx|
to compile the manual |childdoc.pdf| (this file).
\item
Run \LaTeX{} on |childdoc.ins| to create the definitions file |childdoc.def|
and the sample |cdocsamp.tex| with include files
|cdocsch1.tex|, |cdocsch2.tex|, |cdocspt3.tex|, |cdocspt4.tex|,
|cdocsdrf.tex|, |cdocsfn1.tex|, |cdocsfn2.tex|.
Then copy the file |childdoc.def| to an appropriate directory of your \LaTeX{}
distribution, e.g.\ \textit{texmf-root}|/tex/latex/childdoc|.
\end{itemize}

%%%%%%%%%%%%%%%%%%%%%%%%%%%%%%%%%%%%%%%%%%%%%%%%%%%%%%%%%%%%%%%%%%%%%%%%%%%%%%%%
\subsection{Related CTAN Packages}

There are several other packages which offer a similar functionality:
%
\begin{itemize}
\item
The packages
\href{http://ctan.org/pkg/docmute}{\textsf{docmute}},
\href{http://ctan.org/pkg/includex}{\textsf{includex}} and
\href{http://ctan.org/pkg/standalone}{\textsf{standalone}}
provide commands to include only the document body of
a child file thus allowing both files to be compiled individually.
\item
The packages \href{http://ctan.org/pkg/subdocs}{\textsf{subdocs}}
and \href{http://ctan.org/pkg/subfiles}{\textsf{subfiles}}
provide structures in which the main and child documents can be
encapsulated and allowing them to be compiled individually.
The inclusion mechanism is different from the conventional |\include|.
\item
The package \href{http://ctan.org/pkg/combine}{\textsf{combine}}
is an elaborate solution to combine several documents into one.
\end{itemize}
%
See also the CTAN topic \href{http://ctan.org/topic/subdocs}{\textsf{subdocs}}
for further related packages.
The present package differs from the above solutions in that
a document structure constructed with the conventional |\include| mechanism
just needs two extra commands at the top of every file
such that all constituent files can be compiled individually.

%%%%%%%%%%%%%%%%%%%%%%%%%%%%%%%%%%%%%%%%%%%%%%%%%%%%%%%%%%%%%%%%%%%%%%%%%%%%%%%%
%\subsection{Feature Suggestions}
%
%The following is a list of features which may be useful for future
%versions of this package:
%%
%\begin{itemize}
%\item
%\ldots
%\end{itemize}

%%%%%%%%%%%%%%%%%%%%%%%%%%%%%%%%%%%%%%%%%%%%%%%%%%%%%%%%%%%%%%%%%%%%%%%%%%%%%%%%
\subsection{Revision History}

%%%%%%%%%%%%%%%%%%%%%%%%%%%%%%%%%%%%%%%%
\paragraph{v2.0:} 2018/12/30

\begin{itemize}
\item
immediate forward processing
\item
added |\childdocby| mechanism
\item
manual restructured
\end{itemize}

%%%%%%%%%%%%%%%%%%%%%%%%%%%%%%%%%%%%%%%%
\paragraph{v1.6:} 2018/01/17

\begin{itemize}
\item
application for development of include files
\item
corrections to manual
\end{itemize}

%%%%%%%%%%%%%%%%%%%%%%%%%%%%%%%%%%%%%%%%
\paragraph{v1.5:} 2017/05/21

\begin{itemize}
\item
more complete structuring introduced
\item
|\childdocof| introduced
\item
|\childdoc| renamed to |\childdocmain|
\item
|\childredirect| renamed to |\childdocforward| and |\childdocforwardprefix|
and functionality expanded
\end{itemize}

%%%%%%%%%%%%%%%%%%%%%%%%%%%%%%%%%%%%%%%%
\paragraph{v1.0:} 2017/04/27

\begin{itemize}
\item
manual and install package
\item
first version published on CTAN
\end{itemize}

%%%%%%%%%%%%%%%%%%%%%%%%%%%%%%%%%%%%%%%%
\paragraph{v0.6:} 2017/04/26

\begin{itemize}
\item
redirection mechanism added
\end{itemize}

%%%%%%%%%%%%%%%%%%%%%%%%%%%%%%%%%%%%%%%%
\paragraph{v0.5:} 2017/04/26

\begin{itemize}
\item
functionality in definition file
\end{itemize}


%%%%%%%%%%%%%%%%%%%%%%%%%%%%%%%%%%%%%%%%%%%%%%%%%%%%%%%%%%%%%%%%%%%%%%%%%%%%%%%%
%%%%%%%%%%%%%%%%%%%%%%%%%%%%%%%%%%%%%%%%%%%%%%%%%%%%%%%%%%%%%%%%%%%%%%%%%%%%%%%%
%%%%%%%%%%%%%%%%%%%%%%%%%%%%%%%%%%%%%%%%%%%%%%%%%%%%%%%%%%%%%%%%%%%%%%%%%%%%%%%%
\appendix

\settowidth\MacroIndent{\rmfamily\scriptsize 000\ }

 \DocInput{childdoc.dtx}

\end{document}
%</driver>
% \fi
%
% %%%%%%%%%%%%%%%%%%%%%%%%%%%%%%%%%%%%%%%%%%%%%%%%%%%%%%%%%%%%%%%%%%%%%%%%%%%%%%
% %%%%%%%%%%%%%%%%%%%%%%%%%%%%%%%%%%%%%%%%%%%%%%%%%%%%%%%%%%%%%%%%%%%%%%%%%%%%%%
% \section{Sample}
%\iffalse
%<*samplemain>
%\fi
%
% The following presents a sample document
% with two chapters, two parts, a title page,
% a compile flag as well as three forwarding files to set the flag.
% It consists of eight |.tex| files:
% \begin{center}
% \begin{tabular}{ll}
% |cdocsamp.tex|&main file\\
% |cdocsch1.tex|&include file for chapter 1\\
% |cdocsch2.tex|&include file for chapter 2\\
% |cdocspt3.tex|&include file for part 3\\
% |cdocspt4.tex|&include file for part 4\\
% |cdocsdrf.tex|&forwarding file for main file in draft mode\\
% |cdocsfi1.tex|&forwarding file for final version of chapter 1\\
% |cdocsfi2.tex|&forwarding file for final version of chapter 2\\
% \end{tabular}
% \end{center}
% Each of the eight files can be compiled directly by the \LaTeX{} compiler.
%
% %%%%%%%%%%%%%%%%%%%%%%%%%%%%%%%%%%%%%%
% \paragraph{Main File.}
%
% The main file is called |cdocsamp.tex|.
%
% Load the \textsf{childdoc} definitions and
% declare the filename for the main document:
%    \begin{macrocode}
\input{childdoc.def}
\childdocmain{}
%    \end{macrocode}

% Optional override for |\version| flag:
%    \begin{macrocode}
%%\ifchilddoc\else\providecommand{\version}{draft}\fi
%    \end{macrocode}

% Define the default values for the |\version| flag
% (|final| for the main file and |draft| for childs):
%    \begin{macrocode}
\ifchilddoc
\providecommand{\version}{draft}
\else
\providecommand{\version}{final}
\fi
%    \end{macrocode}

% Load the standard document class:
%    \begin{macrocode}
\documentclass[12pt]{article}
%    \end{macrocode}

% Start the document body:
%    \begin{macrocode}
\begin{document}
%    \end{macrocode}

% Declare a title page.
% Print title, part of document being processed and version flag:
%    \begin{macrocode}
\addtocounter{page}{-1}
\begin{center}
{\LARGE\bfseries{}childdoc example\par}
\vspace{1cm}
\ifchilddoc
\ifchilddocmanual part\else chapter\fi:
`\childdocname' of `\childdocjob'\par
\else
main document: `\childdocjob'\par
\fi
version: \version\par
\end{center}
\newpage
%    \end{macrocode}

% Manually include selected file,
% otherwise process as usual:
%    \begin{macrocode}
\ifchilddocmanual
\section*{part `\childdocname'}
\input{\childdocname}
\else
%    \end{macrocode}

% Include the two chapters:
%    \begin{macrocode}
\include{cdocsch1}
\include{cdocsch2}
%    \end{macrocode}

% Include the two parts unless only chapters should be displayed:
%    \begin{macrocode}
\ifchilddoc\else
\section{part three}
\input{cdocspt3}
\section{part four}
\input{cdocspt4}
\fi
%    \end{macrocode}

% Process as usual until here:
%    \begin{macrocode}
\fi
%    \end{macrocode}

% End of document body:
%    \begin{macrocode}
\end{document}
%    \end{macrocode}
%\iffalse
%</samplemain>
%\fi
%
% %%%%%%%%%%%%%%%%%%%%%%%%%%%%%%%%%%%%%%
% \paragraph{Chapter Include Files.}
%
% The include files are called |cdocsch1.tex| and |cdocsch2.tex|.
%
%\iffalse
%<*samplechap1|samplechap2>
%\fi

% Optional override for |\version| flag:
%    \begin{macrocode}
%%\providecommand{\version}{final}
%    \end{macrocode}

% Include the main document:
%    \begin{macrocode}
\input{childdoc.def}
\childdocof{cdocsamp}
%    \end{macrocode}

%\iffalse
%</samplechap1|samplechap2>
%\fi
%
%\iffalse
%<*samplechap1>
%\fi
% Some text for chapter 1:
%    \begin{macrocode}
\section{one}
some text in chapter one
%    \end{macrocode}

%\iffalse
%</samplechap1>
%\fi
% Some text for chapter 2:
%\iffalse
%<*samplechap2>
%\fi
%    \begin{macrocode}
\section{two}
more text in chapter two
%    \end{macrocode}

%\iffalse
%</samplechap2>
%\fi
%
% %%%%%%%%%%%%%%%%%%%%%%%%%%%%%%%%%%%%%%
% \paragraph{Part Include Files.}
%
% The include files are called |cdocspt3.tex| and |cdocspt4.tex|.
%
%\iffalse
%<*samplepart3|samplepart4>
%\fi

% Optional override for |\version| flag:
%    \begin{macrocode}
%%\providecommand{\version}{final}
%    \end{macrocode}

% Include the main document:
%    \begin{macrocode}
\input{childdoc.def}
\childdocby{cdocsamp}
%    \end{macrocode}

%\iffalse
%</samplepart3|samplepart4>
%\fi
%
%\iffalse
%<*samplepart3>
%\fi
% Some text for part 3:
%    \begin{macrocode}
some text in part three
%    \end{macrocode}

%\iffalse
%</samplepart3>
%\fi
% Some text for part 4:
%\iffalse
%<*samplepart4>
%\fi
%    \begin{macrocode}
more text in part four
%    \end{macrocode}

%\iffalse
%</samplepart4>
%\fi
%
% %%%%%%%%%%%%%%%%%%%%%%%%%%%%%%%%%%%%%%
% \paragraph{Forwarding for a Complete Draft.}
%
% The following forwarding file |cdocsdrf.tex|
% compiles the main document in draft mode:
%\iffalse
%<*sampledraft>
%\fi
%    \begin{macrocode}
\def\version{draft}
\input{childdoc.def}
\childdocforward{cdocsamp}
%    \end{macrocode}

%\iffalse
%</sampledraft>
%\fi
%
% %%%%%%%%%%%%%%%%%%%%%%%%%%%%%%%%%%%%%%
% \paragraph{Forwarding for Final Version of the Chapters.}
%
% The following forwarding files |cdocsfn1.tex| and |cdocsfn2.tex|
% (with identical content)
% compile the final versions of the child documents
% |cdocsch1.tex| and |cdocsch2.tex|, respectively:
%\iffalse
%<*samplefinal>
%\fi
%    \begin{macrocode}
\def\version{final}
\input{childdoc.def}
\childdocforwardprefix[cdocsamp]{cdocsfn}{cdocsch}
%    \end{macrocode}

%\iffalse
%</samplefinal>
%\fi
%
% %%%%%%%%%%%%%%%%%%%%%%%%%%%%%%%%%%%%%%
% \paragraph{Command Line Processing.}
%
% The following three command lines generate the output files
% |cdocscld|, |cdocscl1| and |cdocscl2|
% which should be identical to
% |cdocsdrf|, |cdocsch1| and |cdocsfn2|, respectively:
% \begin{center}
% \begin{tabular}{l}
% |latex -jobname cdocscld \|\\
% |  "\def\version{draft}\input{childdoc.def}\childdocforward{cdocsamp}"|\\
% |latex -jobname cdocscl1 \|\\
% |  "\input{childdoc.def}\childdocforward[cdocsamp]{cdocsch1}"|\\
% |latex -jobname cdocscl2 \|\\
% |  "\def\version{final}\input{childdoc.def}\childdocforward{cdocsch2}"|
% \end{tabular}
% \end{center}
% Note that the trailing backslash on each first line
% merely continues the input to the second line
% (for convenient cut ant paste).
% Furthermore, the command |latex| can be replaced by any
% of its alternative versions such as |pdflatex|.
%
% %%%%%%%%%%%%%%%%%%%%%%%%%%%%%%%%%%%%%%%%%%%%%%%%%%%%%%%%%%%%%%%%%%%%%%%%%%%%%%
% %%%%%%%%%%%%%%%%%%%%%%%%%%%%%%%%%%%%%%%%%%%%%%%%%%%%%%%%%%%%%%%%%%%%%%%%%%%%%%
% \section{Implementation}
%\iffalse
%<*package>
%\fi
%
% This section describes the definitions file |childdoc.def|.

% The definitions cannot be loaded using |\usepackage| or |\RequirePackage|
% which has a mechanism to prevent loading a style file more than once.
% When loading the definitions by means of |\input|
% multiple instances have to be prevented manually:
%\iffalse
%This code needs to be before the `\ProvidesFile' directive
%which is defined at the beginning of this file.
%Therefore it is also placed there and commented out here.
%</package>
%<*discard>
%\fi
%    \begin{macrocode}
\ifdefined\childdocmain\endinput\fi
%    \end{macrocode}
%\iffalse
%</discard>
%<*package>
%\fi
%
% \macro{\ifchilddoc}
% \macro{\ifchilddocmanual}
% The conditional |\ifchilddoc| tells whether a
% child (true) or main (false) document is being compiled.
% The conditional |\ifchilddocmanual| tells whether
% the |\includeonly| mechanism is used (false) or
% the selection of child files must be performed manually (true).
% The definitions initialise to false:
%    \begin{macrocode}
\newif\ifchilddoc
\newif\ifchilddocmanual
%    \end{macrocode}

% \macro{\childdocname}
% \macro{\childdocjob}
% The macro |\childdocname| stores the name of the main document
% to be compiled. The macro |\childdocjob| stores the name of
% the document on which the \LaTeX{} compiler was originally invoked.
% The content of |\jobname| cannot be compared
% to filenames specified in the source due to different catcodes.
% The following code rescans |\jobname|, stores the result
% in |\childdocname| and saves a copy in |\childdocjob|:
%    \begin{macrocode}
\edef\childdocname{\scantokens\expandafter{\jobname\noexpand}}
\let\childdocjob\childdocname
%    \end{macrocode}

% \macro{\childdocdisable}
% The macro |\childdocdisable| prevents the main file
% from being processed more than once.
% At this stage, the main document command |\childdocmain|
% is assumed to be called once again where it should do nothing.
% Any subsequent call to it should prevent
% a secondary processing of the main document
% It overwrites the forwarding commands
% |\childdocof| and |\childdocforward|
% with empty macros to prevent further inclusions of the main document:
%    \begin{macrocode}
\newcommand{\childdocdisable}
{
  \renewcommand{\childdocmain}[1]{\renewcommand{\childdocmain}[1]{\endinput}}
  \renewcommand{\childdocof}[1]{}
  \renewcommand{\childdocby}[2][]{}
  \renewcommand{\childdocforward}[2][]{}
  \renewcommand{\childdocdisable}{}
}
%    \end{macrocode}

% \macro{\childdocmain}
% The macro |\childdocmain| is to be called at the top of the main file
% with nothing or the main filename (without extension) as argument.
% First, it breaks loops.
% If the argument is not empty and does not match |\childdocname|
% (which is set by the first inclusion of |childdoc.def|),
% |\ifchilddoc| is set to true, |\includeonly| is applied to the child file
% and |\jobname| is set to the main file
% (for proper handling of |.aux| files):
%    \begin{macrocode}
\newcommand{\childdocmain}[1]
{
  \childdocdisable\childdocmain{}
  \if?#1?\else
    \begingroup
      \def\childdoctmp{#1}
      \ifx\childdoctmp\childdocname
        \def\childdoctmp{}
      \else
        \def\childdoctmp
        {
          \childdoctrue
          \includeonly{\childdocname}
          \def\childdocjob{#1}
          \def\jobname{#1}
        }
      \fi
      \expandafter
    \endgroup
    \childdoctmp
  \fi
}
%    \end{macrocode}

% \macro{\childdocof}
% The command |\childdocof| redirects
% compilation to the main file |#1|.
%    \begin{macrocode}
\newcommand{\childdocof}[1]
{
  \childdocdisable
  \childdoctrue
  \includeonly{\childdocname}
  \def\jobname{#1}
  \def\childdocjob{#1}
  \input{#1}
}
%    \end{macrocode}

% \macro{\childdocby}
% The command |\childdocby| ....
%    \begin{macrocode}
\newcommand{\childdocby}[2][]
{
  \childdocdisable
  \childdoctrue
  \childdocmanualtrue
  \if?#1?\else
    \def\jobname{#2}
  \fi
  \def\childdocjob{#2}
  \input{#2}
  \endinput
}
%    \end{macrocode}

% \macro{\childdocforward}
% The command |\childdocforward| redirects
% compilation to the main file or
% (if the optional argument is given) a child file.
% Parameters are set as if the main file
% or a child file starting with |\childdocof| was compiled.
% Then compilation is handed over to the main file:
%    \begin{macrocode}
\newcommand{\childdocforward}[2][]
{
  \begingroup
    \if?#1?
      \def\childdoctmp
      {
        \def\childdocname{#2}
        \def\childdocjob{#2}
        \def\jobname{#2}
        \input{#2}
        \endinput
      }
    \else
      \def\childdoctmp
      {
        \childdocdisable
        \def\childdocname{#2}
        \childdoctrue
        \includeonly{#2}
        \def\childdocjob{#1}
        \def\jobname{#1}
        \input{#1}
        \endinput
      }
    \fi
    \expandafter
  \endgroup
  \childdoctmp
}
%    \end{macrocode}

% \macro{\childdocforwardprefix}
% The command |\childdocforwardprefix| redirects
% compilation to the main or a child file by means of a pattern.
% The prefix |#1| in the current filename is replaced by |#2|
% and the suffix of the current filename is kept
% (it is assumed that the filename does not contain the substring `|~~~|'
% which is used as a delimiter).
% Compilation is handed over to the new file by |\childdocforward|:
%    \begin{macrocode}
\newcommand{\childdocforwardprefix}[3][]
{
  \begingroup
    \def\childdocextract #2##1~~~{\def\childdoctmp{\childdocforward[#1]{#3##1}}}
    \expandafter\childdocextract\childdocname~~~
    \expandafter
  \endgroup
  \childdoctmp
}
%    \end{macrocode}

% \macro{\childdoc}
% The deprecated macro |\childdoc| is a legacy version of |\childdocmain|:
%    \begin{macrocode}
\newcommand{\childdoc}{\childdocmain}
%    \end{macrocode}

% \macro{\childdocredirect}
% The deprecated macro |\childdocredirect| is a legacy version
% of |\childdocforward| and |\childdocforwardprefix|:
%    \begin{macrocode}
\newcommand{\childdocredirect}[2][]
{
  \begingroup
    \if?#1?
      \def\childdoctmp{\childdocforward{#2}}
    \else
      \def\childdoctmp{\childdocforwardprefix{#1}{#2}}
    \fi
    \expandafter
  \endgroup
  \childdoctmp
}
%    \end{macrocode}

%\iffalse
%</package>
%\fi
%
\endinput
|\\
|\childdocof{|\textit{main}|}|\\
\end{tabular}
\end{center}
at the top of every child file \textit{child}
which is included by |\include{|\textit{child}|}|
from within the main file
(or at least for those files to be compiled individually).
The argument \textit{main} must be the filename of the main file.

There are a couple of
considerations in setting up the main and child documents:

%%%%%%%%%%%%%%%%%%%%%%%%%%%%%%%%%%%%%%%%
\paragraph{Restrictions.}

Please note the following restrictions:
\begin{itemize}
\item
|\childdocmain| must be called with one argument \textit{main}
to ensure compatibility with earlier version of the package.
It must either be empty (|\childdocmain{}|)
or precisely match the filename of the main file in which it is specified.
See \secref{sec:detection} for further information.
\item
The filename \textit{main} must be specified without the |.tex| extension.
\item
The filename \textit{main} is case sensitive
(even in case-insensitive file systems)
due to internal string comparison.
\item
The argument \textit{main} should be fully expanded, it cannot be a macro.
\item
Subdirectories and special characters should be avoided in filenames.
\item
The command |\childdocmain{|\textit{main}|}| must be followed by a whitespace.
It should not be followed immediately by another command
or by a comment mark `|%|'.
This is because the \TeX{} parser reads the token immediately following
the argument of |\childdocmain| and puts it
at the beginning of every child section;
however, a white\-space is ignored.
\end{itemize}

%%%%%%%%%%%%%%%%%%%%%%%%%%%%%%%%%%%%%%%%
\paragraph{Content of Main File.}

It is advisable to place all content in the child files included by |\include|.
Any output contained in the main file will appear in all child documents
unless suppressed manually;
it cannot be suppressed automatically by the |\includeonly| directive
and thus should normally be avoided.
A method to include some content in the main file
by means of conditional processing is described in \secref{sec:conditional}.

%%%%%%%%%%%%%%%%%%%%%%%%%%%%%%%%%%%%%%%%
\paragraph{Page Numbering.}

When only a part of the document is compiled,
the appropriate numbering of pages
(as well as other status parameters)
is determined from the |.aux| files.
The latter contain information from previous passes.
However this information needs to propagate through
all intermediate child documents.
Therefore the page numbering in child documents may well
be inconsistent until the complete document is compiled at least once.

A useful (if unconventional) way to always ensure a consistent
page numbering is to restart the numbering in each child document
and denote the pages by `\textit{child}|.|\textit{page}'
where \textit{child} represents the chapter/section number of the child file.
This can be achieved by the command
|\numberwithin{page}{|\textit{child}|}|
of the \textsf{amsmath} package
where \textit{child} can be |chapter| or |section|
depending on the chosen structuring.
Alternatively, one can modify the macro |\thepage| appropriately
and reset the counter |page| at the start of each child file.

%%%%%%%%%%%%%%%%%%%%%%%%%%%%%%%%%%%%%%%%%%%%%%%%%%%%%%%%%%%%%%%%%%%%%%%%%%%%%%%%
\subsection{Conditional Processing}
\label{sec:conditional}

The package provides a mechanism to compile different versions
of a document. To customise the versions further some conditional processing
can come in handy to distinguish which version is being compiled.
The package provides two macros to describe the compilation context:

%%%%%%%%%%%%%%%%%%%%%%%%%%%%%%%%%%%%%%%%
\DescribeMacro{\ifchilddoc}
The conditional |\ifchilddoc| distinguishes between the compilation of
child documents and the main document:
%
\begin{center}
|\ifchilddoc |\textit{child-code}| |[|\||else |\textit{main-code}]| \||fi|
\end{center}

%%%%%%%%%%%%%%%%%%%%%%%%%%%%%%%%%%%%%%%%
\DescribeMacro{\childdocname}
\DescribeMacro{\childdocjob}
The macro |\childdocname| contains the filename (without extension)
of the main or child file being processed.
Note that |\childdocjob| will always contain the name of the main file.

%%%%%%%%%%%%%%%%%%%%%%%%%%%%%%%%%%%%%%%%
\paragraph{Title Page.}

Conditional processing can be used to include a title or banner page
in the main document when proper precautions are taken.
Importantly, the code in the main file should ensure that the page counter
(as well as other status parameters which are stored in the |.aux| files)
takes the same value after the conditional processing.
Otherwise the page numbers may take divergent values
depending on which part is compiled.

For example, a title page could be declared by:
%
\begin{center}
\begin{tabular}{l}
|\ifchilddoc\||else|\\
|\addtocounter{page}{-1}|\\
\textit{code for title page}\\
|\newpage|\\
|\||fi|
\end{tabular}
\end{center}
%
A banner page for the child documents can be generated by:
%
\begin{center}
\begin{tabular}{l}
|\ifchilddoc|\\
|\addtocounter{page}{-1}|\\
\textit{code for banner page}\\
|\newpage|\\
|\||fi|
\end{tabular}
\end{center}
%
Here one could write a message such as:
\begin{center}
|This is the part \childdocname{} of \childdocjob{}.|
\end{center}

%%%%%%%%%%%%%%%%%%%%%%%%%%%%%%%%%%%%%%%%%%%%%%%%%%%%%%%%%%%%%%%%%%%%%%%%%%%%%%%%
\subsection{Flags}
\label{sec:flags}

The package makes it easy to generate different versions
of the main or child documents.
To this end compilation flags can be defined
and assigned different default values.
They will be particularly useful in conjunction
with the forwarding mechanism described in \secref{sec:forward}.

For example, it may be useful to have a flag |\version|
which can be set to |draft| or |final|.
The document source will contain some conditional code
depending on the value of |\version|.
Suppose further, the flag should default to |final| for the main file
and to |draft| for child files
which is a natural assignment for editing the document.
This is achieved by placing the following code
in the preamble of the main document
(below the |\childdocmain| directive):
%
\begin{center}
\begin{tabular}{l}
|\ifchilddoc|\\
|\providecommand{\version}{draft}|\\
|\||else|\\
|\providecommand{\version}{final}|\\
|\||fi|
\end{tabular}
\end{center}
%
The definition by |\providecommand| makes sure
that previous definitions are not overwritten.
Further statements |\providecommand{\version}{...}|
can thus be added before the above code to override it.

For the main file, one might add a line
(between |\childdocmain| and the above block)
%
\begin{center}
|%\ifchilddoc\||else\providecommand{\version}{draft}\||fi|
\end{center}
%
which can be uncommented to produce a draft version.
Likewise one can add a line to the very top of a child file
(above the |\childdocof{|\textit{main}|}| directive)
%
\begin{center}
|%\providecommand{\version}{final}|
\end{center}
%
which can be uncommented to produce the final version of this child document.

%%%%%%%%%%%%%%%%%%%%%%%%%%%%%%%%%%%%%%%%%%%%%%%%%%%%%%%%%%%%%%%%%%%%%%%%%%%%%%%%
\subsection{Forwarding}
\label{sec:forward}

Different versions of the main or child documents
using compilation flags as described in \secref{sec:flags}
can be (permanently) stored in different files
for convenient compilation, viewing and distribution.
To this end, the package defines a command
to pass on compilation to a different file:

%%%%%%%%%%%%%%%%%%%%%%%%%%%%%%%%%%%%%%%%
\DescribeMacro{\childdocforward}
The command |\childdocforward| redirects processing to
another source file:
%
\begin{center}
\begin{tabular}{l}
|% \iffalse
%
% childdoc.dtx Copyright (C) 2017-2018 Niklas Beisert
%
% This work may be distributed and/or modified under the
% conditions of the LaTeX Project Public License, either version 1.3
% of this license or (at your option) any later version.
% The latest version of this license is in
%   http://www.latex-project.org/lppl.txt
% and version 1.3 or later is part of all distributions of LaTeX
% version 2005/12/01 or later.
%
% This work has the LPPL maintenance status `maintained'.
%
% The Current Maintainer of this work is Niklas Beisert.
%
% This work consists of the files childdoc.dtx and childdoc.ins
% and the derived files childdoc.def and cdocsamp.tex with
% cdocsch1.tex, cdocsch2.tex, cdocsdrf.tex, cdocsfn1.tex, cdocsfn2.tex.
%
%<package>\ifdefined\childdocmain\endinput\fi
%<package>\ProvidesFile{childdoc.def}[2018/12/30 v2.0 child document driver]
%<samplemain>\ProvidesFile{cdocsamp.tex}[2018/12/30 v2.0 sample for childdoc]
%<*driver>
%\ProvidesFile{childdoc.drv}[2018/12/30 v2.0 childdoc reference manual file]
\PassOptionsToClass{10pt,a4paper}{article}
\documentclass{ltxdoc}

\usepackage[margin=35mm]{geometry}
\usepackage{hyperref}
\usepackage{hyperxmp}
\usepackage[usenames]{color}

\hypersetup{colorlinks=true}
\hypersetup{pdfstartview=FitH}
\hypersetup{pdfpagemode=UseNone}
\hypersetup{pdfsource={}}
\hypersetup{pdflang={en-UK}}
\hypersetup{pdfcopyright={Copyright 2017-2018 Niklas Beisert.
  This work may be distributed and/or modified under the
  conditions of the LaTeX Project Public License, either version 1.3
  of this license or (at your option) any later version.}}
\hypersetup{pdflicenseurl={http://www.latex-project.org/lppl.txt}}
\hypersetup{pdfcontactaddress={ETH Zurich, ITP, HIT K,
  Wolfgang-Pauli-Strasse 27}}
\hypersetup{pdfcontactpostcode={8093}}
\hypersetup{pdfcontactcity={Zurich}}
\hypersetup{pdfcontactcountry={Switzerland}}
\hypersetup{pdfcontactemail={nbeisert@itp.phys.ethz.ch}}
\hypersetup{pdfcontacturl={http://people.phys.ethz.ch/\xmptilde nbeisert/}}

\newcommand{\secref}[1]{\hyperref[#1]{section \ref*{#1}}}

\parskip1ex
\parindent0pt
\let\olditemize\itemize
\def\itemize{\olditemize\parskip0pt}

\begin{document}

\title{The \textsf{childdoc} Package}
\hypersetup{pdftitle={The childdoc Package}}
\author{Niklas Beisert\\[2ex]
  Institut f\"ur Theoretische Physik\\
  Eidgen\"ossische Technische Hochschule Z\"urich\\
  Wolfgang-Pauli-Strasse 27, 8093 Z\"urich, Switzerland\\[1ex]
  \href{mailto:nbeisert@itp.phys.ethz.ch}
  {\texttt{nbeisert@itp.phys.ethz.ch}}}
\hypersetup{pdfauthor={Niklas Beisert}}
\hypersetup{pdfsubject={Manual for the LaTeX2e Package childdoc}}
\date{30 December 2018, \textsf{v2.0}}
\maketitle

\begin{abstract}\noindent
\textsf{childdoc} is a \LaTeXe{} package
that enables the direct compilation
of document sections included by |\include|
to individual files.
\end{abstract}

\begingroup
\parskip0ex
\tableofcontents
\endgroup

%%%%%%%%%%%%%%%%%%%%%%%%%%%%%%%%%%%%%%%%%%%%%%%%%%%%%%%%%%%%%%%%%%%%%%%%%%%%%%%%
%%%%%%%%%%%%%%%%%%%%%%%%%%%%%%%%%%%%%%%%%%%%%%%%%%%%%%%%%%%%%%%%%%%%%%%%%%%%%%%%
\section{Introduction}

\LaTeX{} provides a mechanism to structure a large document (such as a book)
into a main file and several child files (containing the chapters)
using the |\include| command.
This mechanism is beneficial for documents
which span hundreds of pages in order to
make the source file(s) more manageable.
Moreover, compilation can be restricted to
selected child files by means of the |\includeonly| command.
The latter feature can be used to reduce the compilation time while editing
(this was significantly more useful in the earlier days of \LaTeX{})
or to generate a smaller document which is easier to navigate.
Another application of |\includeonly| is to generate
documents consisting of selected parts of the complete document.

However, there are a few drawbacks of the plain |\include| mechanism:
\begin{itemize}
\item
The child files cannot be compiled on their own,
they can only be compiled via the main file.
A naive editing environment
(such as a text editor with an option
to have the current file processed by \LaTeX)
may require one to switch to the main file before compiling;
attempting to compile the child file produces errors.
\item
The main file must be modified (each time)
to adjust the |\includeonly| command
to the present needs. This easily leaves the main file in a messy state.
\item
The generated document will always carry the filename
of the main document. This is inconvenient if
several child files are to be compiled and
to be kept for distribution.
\end{itemize}

The present package provides a simple interface
to make child files individually compilable by \LaTeX{}.
Compiling a child file then has the same effect as compiling
the main file with an |\includeonly| command
to select the appropriate child.
Moreover the generated document will carry the name of the child
rather than the main file.
This resolves all three above issues.

This feature is meant to make the editing of books,
thesis documents and lecture notes somewhat more convenient.
However, the package can also be used efficiently for
composing a series of documents (such as exercise sheets)
which are typically distributed individually.
It then assists the author in generating the individual documents
(potentially in different versions)
as well as a document containing the collected series.
Another application is in developing style files
or other kinds of included material
where compilation of the style file could redirect
to a sample or test file.

%%%%%%%%%%%%%%%%%%%%%%%%%%%%%%%%%%%%%%%%%%%%%%%%%%%%%%%%%%%%%%%%%%%%%%%%%%%%%%%%
%%%%%%%%%%%%%%%%%%%%%%%%%%%%%%%%%%%%%%%%%%%%%%%%%%%%%%%%%%%%%%%%%%%%%%%%%%%%%%%%
\section{Usage}

First of all, the package \textsf{childdoc} is \emph{not} a standard
\LaTeXe{} |.sty| style file! Therefore it needs to be invoked in
a non-standard way.

%%%%%%%%%%%%%%%%%%%%%%%%%%%%%%%%%%%%%%%%%%%%%%%%%%%%%%%%%%%%%%%%%%%%%%%%%%%%%%%%
\subsection{Included Files}
\label{sec:include}

%%%%%%%%%%%%%%%%%%%%%%%%%%%%%%%%%%%%%%%%
\DescribeMacro{\childdocmain}
To use the package, add the commands
\begin{center}
\begin{tabular}{l}
|\input{childdoc.def}|\\
|\childdocmain{}|\\
\end{tabular}
\end{center}
at the very top of the main \LaTeX{} file,
in particular \emph{before} the |\documentclass| statement!
The argument of |\childdocmain| should be left empty
(but it must be present).

%%%%%%%%%%%%%%%%%%%%%%%%%%%%%%%%%%%%%%%%
\DescribeMacro{\childdocof}
Furthermore, add the commands
\begin{center}
\begin{tabular}{l}
|\input{childdoc.def}|\\
|\childdocof{|\textit{main}|}|\\
\end{tabular}
\end{center}
at the top of every child file \textit{child}
which is included by |\include{|\textit{child}|}|
from within the main file
(or at least for those files to be compiled individually).
The argument \textit{main} must be the filename of the main file.

There are a couple of
considerations in setting up the main and child documents:

%%%%%%%%%%%%%%%%%%%%%%%%%%%%%%%%%%%%%%%%
\paragraph{Restrictions.}

Please note the following restrictions:
\begin{itemize}
\item
|\childdocmain| must be called with one argument \textit{main}
to ensure compatibility with earlier version of the package.
It must either be empty (|\childdocmain{}|)
or precisely match the filename of the main file in which it is specified.
See \secref{sec:detection} for further information.
\item
The filename \textit{main} must be specified without the |.tex| extension.
\item
The filename \textit{main} is case sensitive
(even in case-insensitive file systems)
due to internal string comparison.
\item
The argument \textit{main} should be fully expanded, it cannot be a macro.
\item
Subdirectories and special characters should be avoided in filenames.
\item
The command |\childdocmain{|\textit{main}|}| must be followed by a whitespace.
It should not be followed immediately by another command
or by a comment mark `|%|'.
This is because the \TeX{} parser reads the token immediately following
the argument of |\childdocmain| and puts it
at the beginning of every child section;
however, a white\-space is ignored.
\end{itemize}

%%%%%%%%%%%%%%%%%%%%%%%%%%%%%%%%%%%%%%%%
\paragraph{Content of Main File.}

It is advisable to place all content in the child files included by |\include|.
Any output contained in the main file will appear in all child documents
unless suppressed manually;
it cannot be suppressed automatically by the |\includeonly| directive
and thus should normally be avoided.
A method to include some content in the main file
by means of conditional processing is described in \secref{sec:conditional}.

%%%%%%%%%%%%%%%%%%%%%%%%%%%%%%%%%%%%%%%%
\paragraph{Page Numbering.}

When only a part of the document is compiled,
the appropriate numbering of pages
(as well as other status parameters)
is determined from the |.aux| files.
The latter contain information from previous passes.
However this information needs to propagate through
all intermediate child documents.
Therefore the page numbering in child documents may well
be inconsistent until the complete document is compiled at least once.

A useful (if unconventional) way to always ensure a consistent
page numbering is to restart the numbering in each child document
and denote the pages by `\textit{child}|.|\textit{page}'
where \textit{child} represents the chapter/section number of the child file.
This can be achieved by the command
|\numberwithin{page}{|\textit{child}|}|
of the \textsf{amsmath} package
where \textit{child} can be |chapter| or |section|
depending on the chosen structuring.
Alternatively, one can modify the macro |\thepage| appropriately
and reset the counter |page| at the start of each child file.

%%%%%%%%%%%%%%%%%%%%%%%%%%%%%%%%%%%%%%%%%%%%%%%%%%%%%%%%%%%%%%%%%%%%%%%%%%%%%%%%
\subsection{Conditional Processing}
\label{sec:conditional}

The package provides a mechanism to compile different versions
of a document. To customise the versions further some conditional processing
can come in handy to distinguish which version is being compiled.
The package provides two macros to describe the compilation context:

%%%%%%%%%%%%%%%%%%%%%%%%%%%%%%%%%%%%%%%%
\DescribeMacro{\ifchilddoc}
The conditional |\ifchilddoc| distinguishes between the compilation of
child documents and the main document:
%
\begin{center}
|\ifchilddoc |\textit{child-code}| |[|\||else |\textit{main-code}]| \||fi|
\end{center}

%%%%%%%%%%%%%%%%%%%%%%%%%%%%%%%%%%%%%%%%
\DescribeMacro{\childdocname}
\DescribeMacro{\childdocjob}
The macro |\childdocname| contains the filename (without extension)
of the main or child file being processed.
Note that |\childdocjob| will always contain the name of the main file.

%%%%%%%%%%%%%%%%%%%%%%%%%%%%%%%%%%%%%%%%
\paragraph{Title Page.}

Conditional processing can be used to include a title or banner page
in the main document when proper precautions are taken.
Importantly, the code in the main file should ensure that the page counter
(as well as other status parameters which are stored in the |.aux| files)
takes the same value after the conditional processing.
Otherwise the page numbers may take divergent values
depending on which part is compiled.

For example, a title page could be declared by:
%
\begin{center}
\begin{tabular}{l}
|\ifchilddoc\||else|\\
|\addtocounter{page}{-1}|\\
\textit{code for title page}\\
|\newpage|\\
|\||fi|
\end{tabular}
\end{center}
%
A banner page for the child documents can be generated by:
%
\begin{center}
\begin{tabular}{l}
|\ifchilddoc|\\
|\addtocounter{page}{-1}|\\
\textit{code for banner page}\\
|\newpage|\\
|\||fi|
\end{tabular}
\end{center}
%
Here one could write a message such as:
\begin{center}
|This is the part \childdocname{} of \childdocjob{}.|
\end{center}

%%%%%%%%%%%%%%%%%%%%%%%%%%%%%%%%%%%%%%%%%%%%%%%%%%%%%%%%%%%%%%%%%%%%%%%%%%%%%%%%
\subsection{Flags}
\label{sec:flags}

The package makes it easy to generate different versions
of the main or child documents.
To this end compilation flags can be defined
and assigned different default values.
They will be particularly useful in conjunction
with the forwarding mechanism described in \secref{sec:forward}.

For example, it may be useful to have a flag |\version|
which can be set to |draft| or |final|.
The document source will contain some conditional code
depending on the value of |\version|.
Suppose further, the flag should default to |final| for the main file
and to |draft| for child files
which is a natural assignment for editing the document.
This is achieved by placing the following code
in the preamble of the main document
(below the |\childdocmain| directive):
%
\begin{center}
\begin{tabular}{l}
|\ifchilddoc|\\
|\providecommand{\version}{draft}|\\
|\||else|\\
|\providecommand{\version}{final}|\\
|\||fi|
\end{tabular}
\end{center}
%
The definition by |\providecommand| makes sure
that previous definitions are not overwritten.
Further statements |\providecommand{\version}{...}|
can thus be added before the above code to override it.

For the main file, one might add a line
(between |\childdocmain| and the above block)
%
\begin{center}
|%\ifchilddoc\||else\providecommand{\version}{draft}\||fi|
\end{center}
%
which can be uncommented to produce a draft version.
Likewise one can add a line to the very top of a child file
(above the |\childdocof{|\textit{main}|}| directive)
%
\begin{center}
|%\providecommand{\version}{final}|
\end{center}
%
which can be uncommented to produce the final version of this child document.

%%%%%%%%%%%%%%%%%%%%%%%%%%%%%%%%%%%%%%%%%%%%%%%%%%%%%%%%%%%%%%%%%%%%%%%%%%%%%%%%
\subsection{Forwarding}
\label{sec:forward}

Different versions of the main or child documents
using compilation flags as described in \secref{sec:flags}
can be (permanently) stored in different files
for convenient compilation, viewing and distribution.
To this end, the package defines a command
to pass on compilation to a different file:

%%%%%%%%%%%%%%%%%%%%%%%%%%%%%%%%%%%%%%%%
\DescribeMacro{\childdocforward}
The command |\childdocforward| redirects processing to
another source file:
%
\begin{center}
\begin{tabular}{l}
|\input{childdoc.def}|\\
|\childdocforward[|\textit{main}|]{|\textit{dest}|}|\\
\end{tabular}
\end{center}
%
The argument \textit{dest} is the destination file
(without extension).
It should be the main file or one of the child files.
Note that further \textsf{childdoc} directives
such as |\childdocof| and |\childdocforward|
in the indicated file will be processed in this form.
The optional argument \textit{main}
passes on directly to the main file \textit{main}
while pretending to compile the child \textit{dest}.
This form behaves as if \textit{dest}
issues |\childdocof{|\textit{main}|}| right away,
and no further \textsf{childdoc} directives will be processed.

%%%%%%%%%%%%%%%%%%%%%%%%%%%%%%%%%%%%%%%%
\DescribeMacro{\...prefix}
In the alternative form |\childdocforwardprefix|,
%
\begin{center}
\begin{tabular}{l}
|\input{childdoc.def}|\\
|\childdocforwardprefix[|\textit{main}|]{|\textit{prefix}|}{|\textit{dest}|}|
\end{tabular}
\end{center}
%
the destination file is determined by a pattern
depending on the current file:
To make this work, the current file must be called
`{\textit{prefix}\hspace{0.2em}\textit{suffix}}'
with \textit{prefix} matching precisely the argument.
Processing is then passed on to the file
`{\textit{dest}\hspace{0.2em}\textit{suffix}}'.
Surely, the same effect is achieved by
directly specifying the
argument `{\textit{dest}\hspace{0.2em}\textit{suffix}}'
in the first form.
However, that requires to set up a different file
for each child. With the alternative form of the command
all these files can have exactly the same content
which simplifies setting them up and maintaining them.

For example, the following file |draft.tex|
with a compilation flag |\version| as described in \secref{sec:flags}
compiles the main document as a draft:
%
\begin{center}
\begin{tabular}{l}
|\def\version{draft}|\\
|\input{childdoc.def}|\\
|\childdocforward{|\textit{main}|}|
\end{tabular}
\end{center}
%
Likewise, the following files |final|\textit{nn}|.tex|
compile the final version of the child document
|child|\textit{nn}|.tex|:
%
\begin{center}
\begin{tabular}{l}
|\def\version{final}|\\
|\input{childdoc.def}|\\
|\childdocforwardprefix{final}{child}|
\end{tabular}
\end{center}
%

Note that when several versions of a main file and/or of each child file
are to be generated, it may be convenient to set up a |Makefile| or
shell script to automatise the process.

%%%%%%%%%%%%%%%%%%%%%%%%%%%%%%%%%%%%%%%%%%%%%%%%%%%%%%%%%%%%%%%%%%%%%%%%%%%%%%%%
\subsection{Command Line Processing}
\label{sec:commandline}

The effect of redirection files can also be achieved by invoking
the \LaTeX{} compiler with a more elaborate command line.
Most conveniently this should be done as part
of a shell script or a |Makefile|.

When using \textsf{childdoc} in the main file, the following
command lines effectively perform a redirection
(note that depending on the shell being used,
backslashes may have to be doubled: `|\|' $\to$ `|\\|'):
%
\begin{center}
|... -jobname "|\textit{target}|" |\\|"|[\textit{flags}]%
|\input{childdoc.def}\childdocforward[|\textit{main}|]{|\textit{dest}|}"|
\end{center}
%
Here \textit{target} is the name of the output file,
\textit{main} is the name of the main file
and \textit{dest} is the name of the main or child file to be processed
(all filenames without extensions).
The optional argument \textit{main} can be omitted
if \textit{main} matches \textit{dest}.
Optionally, compilation \textit{flags} can be defined via |\def| commands.
This command line makes the \TeX{} engine believe
it is compiling the file \textit{target}
whose content is specified as the latter parameter.
The provided code then forwards the processing to
\textit{main} or \textit{dest} as described in \secref{sec:forward}.

%%%%%%%%%%%%%%%%%%%%%%%%%%%%%%%%%%%%%%%%%%%%%%%%%%%%%%%%%%%%%%%%%%%%%%%%%%%%%%%%
\subsection{Include by Input}
\label{sec:input}

Including child documents by |\include| has some restrictions by design.
Most notably, the content of a child document always occupies
its own set of pages; pages cannot be shared between child documents.
Usually, this behaviour makes perfect sense
because each child document contain an essential part of the document.
However, in some situations it may be desirable to compose
a document from a collection of parts
without having mandatory page breaks between then.
For this case, the package
provides a mechanism to include parts
by |\input| which can also be processed individually.
However, by construction this mechanism
requires manual handling of the content to be output.

%%%%%%%%%%%%%%%%%%%%%%%%%%%%%%%%%%%%%%%%
\DescribeMacro{\ifchilddocmanual}
The main file should be prepared as usual, see \secref{sec:include}.
However, the document body must make a distinction
between processing of an individual part and of the main document, e.g.:
%
\begin{center}
\begin{tabular}{l}
|\ifchilddocmanual|\\
|\input{\childdocname}|\\
|\||else|\\
\textit{document body with }|\input{|\textit{part}|}|\\
|\||fi|
\end{tabular}
\end{center}
%
The conditional |\ifchilddocmanual| is true whenever
a part to be included by |\input| is being compiled,
and the name of the part is stored in |\childdocname|.

%%%%%%%%%%%%%%%%%%%%%%%%%%%%%%%%%%%%%%%%
\DescribeMacro{\childdocby}
Each part to be included by |\input| should start with:
%
\begin{center}
\begin{tabular}{l}
|\input{childdoc.def}|\\
|\childdocby{|\textit{main}|}|\\
\end{tabular}
\end{center}
%
The directive |\childdocby| is similar to |\childdocof|
described in \secref{sec:include},
but the subsequent selection of content must be done manually.
To that end, both |\ifchilddoc| and |\ifchilddocmanual|
will be true upon processing of a part,
and the name of the part is stored in |\childdocname|.
Note that |\jobname| will be set to the filename of the current part
so that each part receives an individual |.aux| file
that does not interfere with the |.aux| file(s) of the main document.
This behaviour can be altered by the alternative form
|\childdocby[*]{|\textit{main}|}| (with a non-empty optional argument)
which uses the |.aux| file of the main document
by setting |\jobname| to \textit{main}.

%%%%%%%%%%%%%%%%%%%%%%%%%%%%%%%%%%%%%%%%%%%%%%%%%%%%%%%%%%%%%%%%%%%%%%%%%%%%%%%%
\subsection{Driver Development}
\label{sec:driver}

The \textsf{childdoc} mechanism can also be use for the development
of definition files such as \LaTeX{} styles or classes.
This case differs from the above setup with multiple parts
included by |\include| in that no |\includeonly| should be invoked.
This can be achieved by starting the include file
(before |\ProvidesPackage|) with:
%
\begin{center}
\begin{tabular}{l}
|\input{childdoc.def}|\\
|\childdocforward{|\textit{main}|}|\\
\end{tabular}
\end{center}
%
or alternatively with:
%
\begin{center}
\begin{tabular}{l}
|\input{childdoc.def}|\\
|\childdocby{|\textit{main}|}|\\
\end{tabular}
\end{center}
%
Both forms have slightly different effects as described above.
The main file is prepared as usual, see \secref{sec:include}.

%%%%%%%%%%%%%%%%%%%%%%%%%%%%%%%%%%%%%%%%%%%%%%%%%%%%%%%%%%%%%%%%%%%%%%%%%%%%%%%%
\subsection{Legacy Detection}
\label{sec:detection}

The directive |\childdocmain| in the main file can detect
whether the complete document or merely a child is to be compiled
even without using the directive |\childdocof|.
This method is deprecated because it is less robust
and there is no compelling reason to use it;
it is merely provided for backward compatibility
and it may be removed in future versions.

If the detection mechanism is to be used,
it is mandatory to correctly specify
the filename of the main file as the argument of |\childdocmain|:
%
\begin{center}
\begin{tabular}{l}
|\input{childdoc.def}|\\
|\childdocmain{|\textit{main}|}|\\
\end{tabular}
\end{center}
%
If |\jobname| does not match the argument \textit{main} of |\childdocmain|,
it is assumed that |\jobname| points to the child file to be compiled.
When using |\childdocmain| with the main file specified as argument,
it suffices to start a child file
with just |\input{|\textit{main}|}|
without loading of the package and using |\childdocof|.
If instead all processing is done
with the appropriate \textsf{childdoc} directives,
the argument of \textit{main} of |\childdocmain| can be empty.

An alternative version of the command line processing described
in \secref{sec:commandline} using the detection mechanism reads:
%
\begin{center}
|... -jobname "|\textit{target}|" "|[\textit{flags}]%
[|\def\jobname{|\textit{dest}|}|]|\input{|\textit{main}|}"|
\end{center}

%%%%%%%%%%%%%%%%%%%%%%%%%%%%%%%%%%%%%%%%%%%%%%%%%%%%%%%%%%%%%%%%%%%%%%%%%%%%%%%%
\subsection{Manual Code}
\label{sec:manual}

In case one cannot be certain whether the definitions file |childdoc.def|
is installed on the target \TeX{} distribution
and one prefers not to ship it,
it is conceivable to paste a few relevant commands into the sources.

To that end, drop all statements |\input{childdoc.def}|
and perform the replacements as outlined below.
Instead of |\childdocmain{|\textit{main}|}| add the following code
to the top of the main file:
%
\begin{center}
\begin{tabular}{l}
|\||ifdefined\childdocname\endinput\||fi\newif\ifchilddoc|\\
|\edef\childdocname{\scantokens\expandafter{\jobname\noexpand}}|\\
|\def\childdocmain{|\textit{main}|}\||ifx\childdocmain\childdocname\||else|\\
|\childdoctrue\includeonly{\childdocname}\let\jobname\childdocmain\||fi|\\
\end{tabular}
\end{center}
%
Instead of |\childdocof{|\textit{main}|}| just include the main file
at the top of each child file:
%
\begin{center}
|\input{|\textit{main}|}|
\end{center}
%
A simple redirection |\childdocforward{|\textit{dest}|}| is achieved by:
%
\begin{center}
|\def\jobname{|\textit{dest}|}\input{\jobname}|
\end{center}
%
The redirection with prefix
|\childdocforwardprefix[|\textit{prefix}|]{|\textit{dest}|}|
is accomplished by:
%
\begin{center}
\begin{tabular}{l}
|{\edef\jobname{\scantokens\expandafter{\jobname\noexpand}}|\\
|\def\redirectjob |\textit{prefix}|#1~~~{\gdef\jobname{|\textit{dest}|#1}}|\\
|\expandafter\redirectjob\jobname~~~}\input{\jobname}|
\end{tabular}
\end{center}

In an alternative approach,
child documents can be compiled by a specific command line
without additional code or specific definitions:
%
\begin{center}
|... -jobname "|\textit{target}|" "|[\textit{flags}]%
|\includeonly{|\textit{dest}|}\input{|\textit{main}|}"|
\end{center}
%

%%%%%%%%%%%%%%%%%%%%%%%%%%%%%%%%%%%%%%%%%%%%%%%%%%%%%%%%%%%%%%%%%%%%%%%%%%%%%%%%
%%%%%%%%%%%%%%%%%%%%%%%%%%%%%%%%%%%%%%%%%%%%%%%%%%%%%%%%%%%%%%%%%%%%%%%%%%%%%%%%
\section{Information}

%%%%%%%%%%%%%%%%%%%%%%%%%%%%%%%%%%%%%%%%%%%%%%%%%%%%%%%%%%%%%%%%%%%%%%%%%%%%%%%%
\subsection{Copyright}

Copyright \copyright{} 2017--2018 Niklas Beisert

This work may be distributed and/or modified under the
conditions of the \LaTeX{} Project Public License, either version 1.3
of this license or (at your option) any later version.
The latest version of this license is in
  \url{http://www.latex-project.org/lppl.txt}
and version 1.3 or later is part of all distributions of \LaTeX{}
version 2005/12/01 or later.

This work has the LPPL maintenance status `maintained'.

The Current Maintainer of this work is Niklas Beisert.

This work consists of the files |README.txt|, |childdoc.ins| and |childdoc.dtx|
as well as the derived files |childdoc.def|, |cdocsamp.tex|
with |cdocsch1.tex|, |cdocsch2.tex|, |cdocspt3.tex|, |cdocspt4.tex|,
|cdocsdrf.tex|, |cdocsfn1.tex|, |cdocsfn2.tex|
as well as |childdoc.pdf|.

%%%%%%%%%%%%%%%%%%%%%%%%%%%%%%%%%%%%%%%%%%%%%%%%%%%%%%%%%%%%%%%%%%%%%%%%%%%%%%%%
\subsection{Files and Installation}

The package consists of the files:
%
\begin{center}
\begin{tabular}{ll}
    |README.txt|   & readme file \\
    |childdoc.ins| & installation file \\
    |childdoc.dtx| & source file \\
    |childdoc.def| & definition file \\
    |cdocsamp.tex| & sample main file \\
    |cdocsch1.tex| & sample include file \\
    |cdocsch2.tex| & sample include file \\
    |cdocspt3.tex| & sample part file \\
    |cdocspt4.tex| & sample part file \\
    |cdocsdrf.tex| & sample redirection file \\
    |cdocsfn1.tex| & sample redirection file \\
    |cdocsfn2.tex| & sample redirection file \\
    |childdoc.pdf| & manual
\end{tabular}
\end{center}
%
The distribution consists of the files
|README.txt|, |childdoc.ins| and |childdoc.dtx|.
%
\begin{itemize}
\item
Run (pdf)\LaTeX{} on |childdoc.dtx|
to compile the manual |childdoc.pdf| (this file).
\item
Run \LaTeX{} on |childdoc.ins| to create the definitions file |childdoc.def|
and the sample |cdocsamp.tex| with include files
|cdocsch1.tex|, |cdocsch2.tex|, |cdocspt3.tex|, |cdocspt4.tex|,
|cdocsdrf.tex|, |cdocsfn1.tex|, |cdocsfn2.tex|.
Then copy the file |childdoc.def| to an appropriate directory of your \LaTeX{}
distribution, e.g.\ \textit{texmf-root}|/tex/latex/childdoc|.
\end{itemize}

%%%%%%%%%%%%%%%%%%%%%%%%%%%%%%%%%%%%%%%%%%%%%%%%%%%%%%%%%%%%%%%%%%%%%%%%%%%%%%%%
\subsection{Related CTAN Packages}

There are several other packages which offer a similar functionality:
%
\begin{itemize}
\item
The packages
\href{http://ctan.org/pkg/docmute}{\textsf{docmute}},
\href{http://ctan.org/pkg/includex}{\textsf{includex}} and
\href{http://ctan.org/pkg/standalone}{\textsf{standalone}}
provide commands to include only the document body of
a child file thus allowing both files to be compiled individually.
\item
The packages \href{http://ctan.org/pkg/subdocs}{\textsf{subdocs}}
and \href{http://ctan.org/pkg/subfiles}{\textsf{subfiles}}
provide structures in which the main and child documents can be
encapsulated and allowing them to be compiled individually.
The inclusion mechanism is different from the conventional |\include|.
\item
The package \href{http://ctan.org/pkg/combine}{\textsf{combine}}
is an elaborate solution to combine several documents into one.
\end{itemize}
%
See also the CTAN topic \href{http://ctan.org/topic/subdocs}{\textsf{subdocs}}
for further related packages.
The present package differs from the above solutions in that
a document structure constructed with the conventional |\include| mechanism
just needs two extra commands at the top of every file
such that all constituent files can be compiled individually.

%%%%%%%%%%%%%%%%%%%%%%%%%%%%%%%%%%%%%%%%%%%%%%%%%%%%%%%%%%%%%%%%%%%%%%%%%%%%%%%%
%\subsection{Feature Suggestions}
%
%The following is a list of features which may be useful for future
%versions of this package:
%%
%\begin{itemize}
%\item
%\ldots
%\end{itemize}

%%%%%%%%%%%%%%%%%%%%%%%%%%%%%%%%%%%%%%%%%%%%%%%%%%%%%%%%%%%%%%%%%%%%%%%%%%%%%%%%
\subsection{Revision History}

%%%%%%%%%%%%%%%%%%%%%%%%%%%%%%%%%%%%%%%%
\paragraph{v2.0:} 2018/12/30

\begin{itemize}
\item
immediate forward processing
\item
added |\childdocby| mechanism
\item
manual restructured
\end{itemize}

%%%%%%%%%%%%%%%%%%%%%%%%%%%%%%%%%%%%%%%%
\paragraph{v1.6:} 2018/01/17

\begin{itemize}
\item
application for development of include files
\item
corrections to manual
\end{itemize}

%%%%%%%%%%%%%%%%%%%%%%%%%%%%%%%%%%%%%%%%
\paragraph{v1.5:} 2017/05/21

\begin{itemize}
\item
more complete structuring introduced
\item
|\childdocof| introduced
\item
|\childdoc| renamed to |\childdocmain|
\item
|\childredirect| renamed to |\childdocforward| and |\childdocforwardprefix|
and functionality expanded
\end{itemize}

%%%%%%%%%%%%%%%%%%%%%%%%%%%%%%%%%%%%%%%%
\paragraph{v1.0:} 2017/04/27

\begin{itemize}
\item
manual and install package
\item
first version published on CTAN
\end{itemize}

%%%%%%%%%%%%%%%%%%%%%%%%%%%%%%%%%%%%%%%%
\paragraph{v0.6:} 2017/04/26

\begin{itemize}
\item
redirection mechanism added
\end{itemize}

%%%%%%%%%%%%%%%%%%%%%%%%%%%%%%%%%%%%%%%%
\paragraph{v0.5:} 2017/04/26

\begin{itemize}
\item
functionality in definition file
\end{itemize}


%%%%%%%%%%%%%%%%%%%%%%%%%%%%%%%%%%%%%%%%%%%%%%%%%%%%%%%%%%%%%%%%%%%%%%%%%%%%%%%%
%%%%%%%%%%%%%%%%%%%%%%%%%%%%%%%%%%%%%%%%%%%%%%%%%%%%%%%%%%%%%%%%%%%%%%%%%%%%%%%%
%%%%%%%%%%%%%%%%%%%%%%%%%%%%%%%%%%%%%%%%%%%%%%%%%%%%%%%%%%%%%%%%%%%%%%%%%%%%%%%%
\appendix

\settowidth\MacroIndent{\rmfamily\scriptsize 000\ }

 \DocInput{childdoc.dtx}

\end{document}
%</driver>
% \fi
%
% %%%%%%%%%%%%%%%%%%%%%%%%%%%%%%%%%%%%%%%%%%%%%%%%%%%%%%%%%%%%%%%%%%%%%%%%%%%%%%
% %%%%%%%%%%%%%%%%%%%%%%%%%%%%%%%%%%%%%%%%%%%%%%%%%%%%%%%%%%%%%%%%%%%%%%%%%%%%%%
% \section{Sample}
%\iffalse
%<*samplemain>
%\fi
%
% The following presents a sample document
% with two chapters, two parts, a title page,
% a compile flag as well as three forwarding files to set the flag.
% It consists of eight |.tex| files:
% \begin{center}
% \begin{tabular}{ll}
% |cdocsamp.tex|&main file\\
% |cdocsch1.tex|&include file for chapter 1\\
% |cdocsch2.tex|&include file for chapter 2\\
% |cdocspt3.tex|&include file for part 3\\
% |cdocspt4.tex|&include file for part 4\\
% |cdocsdrf.tex|&forwarding file for main file in draft mode\\
% |cdocsfi1.tex|&forwarding file for final version of chapter 1\\
% |cdocsfi2.tex|&forwarding file for final version of chapter 2\\
% \end{tabular}
% \end{center}
% Each of the eight files can be compiled directly by the \LaTeX{} compiler.
%
% %%%%%%%%%%%%%%%%%%%%%%%%%%%%%%%%%%%%%%
% \paragraph{Main File.}
%
% The main file is called |cdocsamp.tex|.
%
% Load the \textsf{childdoc} definitions and
% declare the filename for the main document:
%    \begin{macrocode}
\input{childdoc.def}
\childdocmain{}
%    \end{macrocode}

% Optional override for |\version| flag:
%    \begin{macrocode}
%%\ifchilddoc\else\providecommand{\version}{draft}\fi
%    \end{macrocode}

% Define the default values for the |\version| flag
% (|final| for the main file and |draft| for childs):
%    \begin{macrocode}
\ifchilddoc
\providecommand{\version}{draft}
\else
\providecommand{\version}{final}
\fi
%    \end{macrocode}

% Load the standard document class:
%    \begin{macrocode}
\documentclass[12pt]{article}
%    \end{macrocode}

% Start the document body:
%    \begin{macrocode}
\begin{document}
%    \end{macrocode}

% Declare a title page.
% Print title, part of document being processed and version flag:
%    \begin{macrocode}
\addtocounter{page}{-1}
\begin{center}
{\LARGE\bfseries{}childdoc example\par}
\vspace{1cm}
\ifchilddoc
\ifchilddocmanual part\else chapter\fi:
`\childdocname' of `\childdocjob'\par
\else
main document: `\childdocjob'\par
\fi
version: \version\par
\end{center}
\newpage
%    \end{macrocode}

% Manually include selected file,
% otherwise process as usual:
%    \begin{macrocode}
\ifchilddocmanual
\section*{part `\childdocname'}
\input{\childdocname}
\else
%    \end{macrocode}

% Include the two chapters:
%    \begin{macrocode}
\include{cdocsch1}
\include{cdocsch2}
%    \end{macrocode}

% Include the two parts unless only chapters should be displayed:
%    \begin{macrocode}
\ifchilddoc\else
\section{part three}
\input{cdocspt3}
\section{part four}
\input{cdocspt4}
\fi
%    \end{macrocode}

% Process as usual until here:
%    \begin{macrocode}
\fi
%    \end{macrocode}

% End of document body:
%    \begin{macrocode}
\end{document}
%    \end{macrocode}
%\iffalse
%</samplemain>
%\fi
%
% %%%%%%%%%%%%%%%%%%%%%%%%%%%%%%%%%%%%%%
% \paragraph{Chapter Include Files.}
%
% The include files are called |cdocsch1.tex| and |cdocsch2.tex|.
%
%\iffalse
%<*samplechap1|samplechap2>
%\fi

% Optional override for |\version| flag:
%    \begin{macrocode}
%%\providecommand{\version}{final}
%    \end{macrocode}

% Include the main document:
%    \begin{macrocode}
\input{childdoc.def}
\childdocof{cdocsamp}
%    \end{macrocode}

%\iffalse
%</samplechap1|samplechap2>
%\fi
%
%\iffalse
%<*samplechap1>
%\fi
% Some text for chapter 1:
%    \begin{macrocode}
\section{one}
some text in chapter one
%    \end{macrocode}

%\iffalse
%</samplechap1>
%\fi
% Some text for chapter 2:
%\iffalse
%<*samplechap2>
%\fi
%    \begin{macrocode}
\section{two}
more text in chapter two
%    \end{macrocode}

%\iffalse
%</samplechap2>
%\fi
%
% %%%%%%%%%%%%%%%%%%%%%%%%%%%%%%%%%%%%%%
% \paragraph{Part Include Files.}
%
% The include files are called |cdocspt3.tex| and |cdocspt4.tex|.
%
%\iffalse
%<*samplepart3|samplepart4>
%\fi

% Optional override for |\version| flag:
%    \begin{macrocode}
%%\providecommand{\version}{final}
%    \end{macrocode}

% Include the main document:
%    \begin{macrocode}
\input{childdoc.def}
\childdocby{cdocsamp}
%    \end{macrocode}

%\iffalse
%</samplepart3|samplepart4>
%\fi
%
%\iffalse
%<*samplepart3>
%\fi
% Some text for part 3:
%    \begin{macrocode}
some text in part three
%    \end{macrocode}

%\iffalse
%</samplepart3>
%\fi
% Some text for part 4:
%\iffalse
%<*samplepart4>
%\fi
%    \begin{macrocode}
more text in part four
%    \end{macrocode}

%\iffalse
%</samplepart4>
%\fi
%
% %%%%%%%%%%%%%%%%%%%%%%%%%%%%%%%%%%%%%%
% \paragraph{Forwarding for a Complete Draft.}
%
% The following forwarding file |cdocsdrf.tex|
% compiles the main document in draft mode:
%\iffalse
%<*sampledraft>
%\fi
%    \begin{macrocode}
\def\version{draft}
\input{childdoc.def}
\childdocforward{cdocsamp}
%    \end{macrocode}

%\iffalse
%</sampledraft>
%\fi
%
% %%%%%%%%%%%%%%%%%%%%%%%%%%%%%%%%%%%%%%
% \paragraph{Forwarding for Final Version of the Chapters.}
%
% The following forwarding files |cdocsfn1.tex| and |cdocsfn2.tex|
% (with identical content)
% compile the final versions of the child documents
% |cdocsch1.tex| and |cdocsch2.tex|, respectively:
%\iffalse
%<*samplefinal>
%\fi
%    \begin{macrocode}
\def\version{final}
\input{childdoc.def}
\childdocforwardprefix[cdocsamp]{cdocsfn}{cdocsch}
%    \end{macrocode}

%\iffalse
%</samplefinal>
%\fi
%
% %%%%%%%%%%%%%%%%%%%%%%%%%%%%%%%%%%%%%%
% \paragraph{Command Line Processing.}
%
% The following three command lines generate the output files
% |cdocscld|, |cdocscl1| and |cdocscl2|
% which should be identical to
% |cdocsdrf|, |cdocsch1| and |cdocsfn2|, respectively:
% \begin{center}
% \begin{tabular}{l}
% |latex -jobname cdocscld \|\\
% |  "\def\version{draft}\input{childdoc.def}\childdocforward{cdocsamp}"|\\
% |latex -jobname cdocscl1 \|\\
% |  "\input{childdoc.def}\childdocforward[cdocsamp]{cdocsch1}"|\\
% |latex -jobname cdocscl2 \|\\
% |  "\def\version{final}\input{childdoc.def}\childdocforward{cdocsch2}"|
% \end{tabular}
% \end{center}
% Note that the trailing backslash on each first line
% merely continues the input to the second line
% (for convenient cut ant paste).
% Furthermore, the command |latex| can be replaced by any
% of its alternative versions such as |pdflatex|.
%
% %%%%%%%%%%%%%%%%%%%%%%%%%%%%%%%%%%%%%%%%%%%%%%%%%%%%%%%%%%%%%%%%%%%%%%%%%%%%%%
% %%%%%%%%%%%%%%%%%%%%%%%%%%%%%%%%%%%%%%%%%%%%%%%%%%%%%%%%%%%%%%%%%%%%%%%%%%%%%%
% \section{Implementation}
%\iffalse
%<*package>
%\fi
%
% This section describes the definitions file |childdoc.def|.

% The definitions cannot be loaded using |\usepackage| or |\RequirePackage|
% which has a mechanism to prevent loading a style file more than once.
% When loading the definitions by means of |\input|
% multiple instances have to be prevented manually:
%\iffalse
%This code needs to be before the `\ProvidesFile' directive
%which is defined at the beginning of this file.
%Therefore it is also placed there and commented out here.
%</package>
%<*discard>
%\fi
%    \begin{macrocode}
\ifdefined\childdocmain\endinput\fi
%    \end{macrocode}
%\iffalse
%</discard>
%<*package>
%\fi
%
% \macro{\ifchilddoc}
% \macro{\ifchilddocmanual}
% The conditional |\ifchilddoc| tells whether a
% child (true) or main (false) document is being compiled.
% The conditional |\ifchilddocmanual| tells whether
% the |\includeonly| mechanism is used (false) or
% the selection of child files must be performed manually (true).
% The definitions initialise to false:
%    \begin{macrocode}
\newif\ifchilddoc
\newif\ifchilddocmanual
%    \end{macrocode}

% \macro{\childdocname}
% \macro{\childdocjob}
% The macro |\childdocname| stores the name of the main document
% to be compiled. The macro |\childdocjob| stores the name of
% the document on which the \LaTeX{} compiler was originally invoked.
% The content of |\jobname| cannot be compared
% to filenames specified in the source due to different catcodes.
% The following code rescans |\jobname|, stores the result
% in |\childdocname| and saves a copy in |\childdocjob|:
%    \begin{macrocode}
\edef\childdocname{\scantokens\expandafter{\jobname\noexpand}}
\let\childdocjob\childdocname
%    \end{macrocode}

% \macro{\childdocdisable}
% The macro |\childdocdisable| prevents the main file
% from being processed more than once.
% At this stage, the main document command |\childdocmain|
% is assumed to be called once again where it should do nothing.
% Any subsequent call to it should prevent
% a secondary processing of the main document
% It overwrites the forwarding commands
% |\childdocof| and |\childdocforward|
% with empty macros to prevent further inclusions of the main document:
%    \begin{macrocode}
\newcommand{\childdocdisable}
{
  \renewcommand{\childdocmain}[1]{\renewcommand{\childdocmain}[1]{\endinput}}
  \renewcommand{\childdocof}[1]{}
  \renewcommand{\childdocby}[2][]{}
  \renewcommand{\childdocforward}[2][]{}
  \renewcommand{\childdocdisable}{}
}
%    \end{macrocode}

% \macro{\childdocmain}
% The macro |\childdocmain| is to be called at the top of the main file
% with nothing or the main filename (without extension) as argument.
% First, it breaks loops.
% If the argument is not empty and does not match |\childdocname|
% (which is set by the first inclusion of |childdoc.def|),
% |\ifchilddoc| is set to true, |\includeonly| is applied to the child file
% and |\jobname| is set to the main file
% (for proper handling of |.aux| files):
%    \begin{macrocode}
\newcommand{\childdocmain}[1]
{
  \childdocdisable\childdocmain{}
  \if?#1?\else
    \begingroup
      \def\childdoctmp{#1}
      \ifx\childdoctmp\childdocname
        \def\childdoctmp{}
      \else
        \def\childdoctmp
        {
          \childdoctrue
          \includeonly{\childdocname}
          \def\childdocjob{#1}
          \def\jobname{#1}
        }
      \fi
      \expandafter
    \endgroup
    \childdoctmp
  \fi
}
%    \end{macrocode}

% \macro{\childdocof}
% The command |\childdocof| redirects
% compilation to the main file |#1|.
%    \begin{macrocode}
\newcommand{\childdocof}[1]
{
  \childdocdisable
  \childdoctrue
  \includeonly{\childdocname}
  \def\jobname{#1}
  \def\childdocjob{#1}
  \input{#1}
}
%    \end{macrocode}

% \macro{\childdocby}
% The command |\childdocby| ....
%    \begin{macrocode}
\newcommand{\childdocby}[2][]
{
  \childdocdisable
  \childdoctrue
  \childdocmanualtrue
  \if?#1?\else
    \def\jobname{#2}
  \fi
  \def\childdocjob{#2}
  \input{#2}
  \endinput
}
%    \end{macrocode}

% \macro{\childdocforward}
% The command |\childdocforward| redirects
% compilation to the main file or
% (if the optional argument is given) a child file.
% Parameters are set as if the main file
% or a child file starting with |\childdocof| was compiled.
% Then compilation is handed over to the main file:
%    \begin{macrocode}
\newcommand{\childdocforward}[2][]
{
  \begingroup
    \if?#1?
      \def\childdoctmp
      {
        \def\childdocname{#2}
        \def\childdocjob{#2}
        \def\jobname{#2}
        \input{#2}
        \endinput
      }
    \else
      \def\childdoctmp
      {
        \childdocdisable
        \def\childdocname{#2}
        \childdoctrue
        \includeonly{#2}
        \def\childdocjob{#1}
        \def\jobname{#1}
        \input{#1}
        \endinput
      }
    \fi
    \expandafter
  \endgroup
  \childdoctmp
}
%    \end{macrocode}

% \macro{\childdocforwardprefix}
% The command |\childdocforwardprefix| redirects
% compilation to the main or a child file by means of a pattern.
% The prefix |#1| in the current filename is replaced by |#2|
% and the suffix of the current filename is kept
% (it is assumed that the filename does not contain the substring `|~~~|'
% which is used as a delimiter).
% Compilation is handed over to the new file by |\childdocforward|:
%    \begin{macrocode}
\newcommand{\childdocforwardprefix}[3][]
{
  \begingroup
    \def\childdocextract #2##1~~~{\def\childdoctmp{\childdocforward[#1]{#3##1}}}
    \expandafter\childdocextract\childdocname~~~
    \expandafter
  \endgroup
  \childdoctmp
}
%    \end{macrocode}

% \macro{\childdoc}
% The deprecated macro |\childdoc| is a legacy version of |\childdocmain|:
%    \begin{macrocode}
\newcommand{\childdoc}{\childdocmain}
%    \end{macrocode}

% \macro{\childdocredirect}
% The deprecated macro |\childdocredirect| is a legacy version
% of |\childdocforward| and |\childdocforwardprefix|:
%    \begin{macrocode}
\newcommand{\childdocredirect}[2][]
{
  \begingroup
    \if?#1?
      \def\childdoctmp{\childdocforward{#2}}
    \else
      \def\childdoctmp{\childdocforwardprefix{#1}{#2}}
    \fi
    \expandafter
  \endgroup
  \childdoctmp
}
%    \end{macrocode}

%\iffalse
%</package>
%\fi
%
\endinput
|\\
|\childdocforward[|\textit{main}|]{|\textit{dest}|}|\\
\end{tabular}
\end{center}
%
The argument \textit{dest} is the destination file
(without extension).
It should be the main file or one of the child files.
Note that further \textsf{childdoc} directives
such as |\childdocof| and |\childdocforward|
in the indicated file will be processed in this form.
The optional argument \textit{main}
passes on directly to the main file \textit{main}
while pretending to compile the child \textit{dest}.
This form behaves as if \textit{dest}
issues |\childdocof{|\textit{main}|}| right away,
and no further \textsf{childdoc} directives will be processed.

%%%%%%%%%%%%%%%%%%%%%%%%%%%%%%%%%%%%%%%%
\DescribeMacro{\...prefix}
In the alternative form |\childdocforwardprefix|,
%
\begin{center}
\begin{tabular}{l}
|% \iffalse
%
% childdoc.dtx Copyright (C) 2017-2018 Niklas Beisert
%
% This work may be distributed and/or modified under the
% conditions of the LaTeX Project Public License, either version 1.3
% of this license or (at your option) any later version.
% The latest version of this license is in
%   http://www.latex-project.org/lppl.txt
% and version 1.3 or later is part of all distributions of LaTeX
% version 2005/12/01 or later.
%
% This work has the LPPL maintenance status `maintained'.
%
% The Current Maintainer of this work is Niklas Beisert.
%
% This work consists of the files childdoc.dtx and childdoc.ins
% and the derived files childdoc.def and cdocsamp.tex with
% cdocsch1.tex, cdocsch2.tex, cdocsdrf.tex, cdocsfn1.tex, cdocsfn2.tex.
%
%<package>\ifdefined\childdocmain\endinput\fi
%<package>\ProvidesFile{childdoc.def}[2018/12/30 v2.0 child document driver]
%<samplemain>\ProvidesFile{cdocsamp.tex}[2018/12/30 v2.0 sample for childdoc]
%<*driver>
%\ProvidesFile{childdoc.drv}[2018/12/30 v2.0 childdoc reference manual file]
\PassOptionsToClass{10pt,a4paper}{article}
\documentclass{ltxdoc}

\usepackage[margin=35mm]{geometry}
\usepackage{hyperref}
\usepackage{hyperxmp}
\usepackage[usenames]{color}

\hypersetup{colorlinks=true}
\hypersetup{pdfstartview=FitH}
\hypersetup{pdfpagemode=UseNone}
\hypersetup{pdfsource={}}
\hypersetup{pdflang={en-UK}}
\hypersetup{pdfcopyright={Copyright 2017-2018 Niklas Beisert.
  This work may be distributed and/or modified under the
  conditions of the LaTeX Project Public License, either version 1.3
  of this license or (at your option) any later version.}}
\hypersetup{pdflicenseurl={http://www.latex-project.org/lppl.txt}}
\hypersetup{pdfcontactaddress={ETH Zurich, ITP, HIT K,
  Wolfgang-Pauli-Strasse 27}}
\hypersetup{pdfcontactpostcode={8093}}
\hypersetup{pdfcontactcity={Zurich}}
\hypersetup{pdfcontactcountry={Switzerland}}
\hypersetup{pdfcontactemail={nbeisert@itp.phys.ethz.ch}}
\hypersetup{pdfcontacturl={http://people.phys.ethz.ch/\xmptilde nbeisert/}}

\newcommand{\secref}[1]{\hyperref[#1]{section \ref*{#1}}}

\parskip1ex
\parindent0pt
\let\olditemize\itemize
\def\itemize{\olditemize\parskip0pt}

\begin{document}

\title{The \textsf{childdoc} Package}
\hypersetup{pdftitle={The childdoc Package}}
\author{Niklas Beisert\\[2ex]
  Institut f\"ur Theoretische Physik\\
  Eidgen\"ossische Technische Hochschule Z\"urich\\
  Wolfgang-Pauli-Strasse 27, 8093 Z\"urich, Switzerland\\[1ex]
  \href{mailto:nbeisert@itp.phys.ethz.ch}
  {\texttt{nbeisert@itp.phys.ethz.ch}}}
\hypersetup{pdfauthor={Niklas Beisert}}
\hypersetup{pdfsubject={Manual for the LaTeX2e Package childdoc}}
\date{30 December 2018, \textsf{v2.0}}
\maketitle

\begin{abstract}\noindent
\textsf{childdoc} is a \LaTeXe{} package
that enables the direct compilation
of document sections included by |\include|
to individual files.
\end{abstract}

\begingroup
\parskip0ex
\tableofcontents
\endgroup

%%%%%%%%%%%%%%%%%%%%%%%%%%%%%%%%%%%%%%%%%%%%%%%%%%%%%%%%%%%%%%%%%%%%%%%%%%%%%%%%
%%%%%%%%%%%%%%%%%%%%%%%%%%%%%%%%%%%%%%%%%%%%%%%%%%%%%%%%%%%%%%%%%%%%%%%%%%%%%%%%
\section{Introduction}

\LaTeX{} provides a mechanism to structure a large document (such as a book)
into a main file and several child files (containing the chapters)
using the |\include| command.
This mechanism is beneficial for documents
which span hundreds of pages in order to
make the source file(s) more manageable.
Moreover, compilation can be restricted to
selected child files by means of the |\includeonly| command.
The latter feature can be used to reduce the compilation time while editing
(this was significantly more useful in the earlier days of \LaTeX{})
or to generate a smaller document which is easier to navigate.
Another application of |\includeonly| is to generate
documents consisting of selected parts of the complete document.

However, there are a few drawbacks of the plain |\include| mechanism:
\begin{itemize}
\item
The child files cannot be compiled on their own,
they can only be compiled via the main file.
A naive editing environment
(such as a text editor with an option
to have the current file processed by \LaTeX)
may require one to switch to the main file before compiling;
attempting to compile the child file produces errors.
\item
The main file must be modified (each time)
to adjust the |\includeonly| command
to the present needs. This easily leaves the main file in a messy state.
\item
The generated document will always carry the filename
of the main document. This is inconvenient if
several child files are to be compiled and
to be kept for distribution.
\end{itemize}

The present package provides a simple interface
to make child files individually compilable by \LaTeX{}.
Compiling a child file then has the same effect as compiling
the main file with an |\includeonly| command
to select the appropriate child.
Moreover the generated document will carry the name of the child
rather than the main file.
This resolves all three above issues.

This feature is meant to make the editing of books,
thesis documents and lecture notes somewhat more convenient.
However, the package can also be used efficiently for
composing a series of documents (such as exercise sheets)
which are typically distributed individually.
It then assists the author in generating the individual documents
(potentially in different versions)
as well as a document containing the collected series.
Another application is in developing style files
or other kinds of included material
where compilation of the style file could redirect
to a sample or test file.

%%%%%%%%%%%%%%%%%%%%%%%%%%%%%%%%%%%%%%%%%%%%%%%%%%%%%%%%%%%%%%%%%%%%%%%%%%%%%%%%
%%%%%%%%%%%%%%%%%%%%%%%%%%%%%%%%%%%%%%%%%%%%%%%%%%%%%%%%%%%%%%%%%%%%%%%%%%%%%%%%
\section{Usage}

First of all, the package \textsf{childdoc} is \emph{not} a standard
\LaTeXe{} |.sty| style file! Therefore it needs to be invoked in
a non-standard way.

%%%%%%%%%%%%%%%%%%%%%%%%%%%%%%%%%%%%%%%%%%%%%%%%%%%%%%%%%%%%%%%%%%%%%%%%%%%%%%%%
\subsection{Included Files}
\label{sec:include}

%%%%%%%%%%%%%%%%%%%%%%%%%%%%%%%%%%%%%%%%
\DescribeMacro{\childdocmain}
To use the package, add the commands
\begin{center}
\begin{tabular}{l}
|\input{childdoc.def}|\\
|\childdocmain{}|\\
\end{tabular}
\end{center}
at the very top of the main \LaTeX{} file,
in particular \emph{before} the |\documentclass| statement!
The argument of |\childdocmain| should be left empty
(but it must be present).

%%%%%%%%%%%%%%%%%%%%%%%%%%%%%%%%%%%%%%%%
\DescribeMacro{\childdocof}
Furthermore, add the commands
\begin{center}
\begin{tabular}{l}
|\input{childdoc.def}|\\
|\childdocof{|\textit{main}|}|\\
\end{tabular}
\end{center}
at the top of every child file \textit{child}
which is included by |\include{|\textit{child}|}|
from within the main file
(or at least for those files to be compiled individually).
The argument \textit{main} must be the filename of the main file.

There are a couple of
considerations in setting up the main and child documents:

%%%%%%%%%%%%%%%%%%%%%%%%%%%%%%%%%%%%%%%%
\paragraph{Restrictions.}

Please note the following restrictions:
\begin{itemize}
\item
|\childdocmain| must be called with one argument \textit{main}
to ensure compatibility with earlier version of the package.
It must either be empty (|\childdocmain{}|)
or precisely match the filename of the main file in which it is specified.
See \secref{sec:detection} for further information.
\item
The filename \textit{main} must be specified without the |.tex| extension.
\item
The filename \textit{main} is case sensitive
(even in case-insensitive file systems)
due to internal string comparison.
\item
The argument \textit{main} should be fully expanded, it cannot be a macro.
\item
Subdirectories and special characters should be avoided in filenames.
\item
The command |\childdocmain{|\textit{main}|}| must be followed by a whitespace.
It should not be followed immediately by another command
or by a comment mark `|%|'.
This is because the \TeX{} parser reads the token immediately following
the argument of |\childdocmain| and puts it
at the beginning of every child section;
however, a white\-space is ignored.
\end{itemize}

%%%%%%%%%%%%%%%%%%%%%%%%%%%%%%%%%%%%%%%%
\paragraph{Content of Main File.}

It is advisable to place all content in the child files included by |\include|.
Any output contained in the main file will appear in all child documents
unless suppressed manually;
it cannot be suppressed automatically by the |\includeonly| directive
and thus should normally be avoided.
A method to include some content in the main file
by means of conditional processing is described in \secref{sec:conditional}.

%%%%%%%%%%%%%%%%%%%%%%%%%%%%%%%%%%%%%%%%
\paragraph{Page Numbering.}

When only a part of the document is compiled,
the appropriate numbering of pages
(as well as other status parameters)
is determined from the |.aux| files.
The latter contain information from previous passes.
However this information needs to propagate through
all intermediate child documents.
Therefore the page numbering in child documents may well
be inconsistent until the complete document is compiled at least once.

A useful (if unconventional) way to always ensure a consistent
page numbering is to restart the numbering in each child document
and denote the pages by `\textit{child}|.|\textit{page}'
where \textit{child} represents the chapter/section number of the child file.
This can be achieved by the command
|\numberwithin{page}{|\textit{child}|}|
of the \textsf{amsmath} package
where \textit{child} can be |chapter| or |section|
depending on the chosen structuring.
Alternatively, one can modify the macro |\thepage| appropriately
and reset the counter |page| at the start of each child file.

%%%%%%%%%%%%%%%%%%%%%%%%%%%%%%%%%%%%%%%%%%%%%%%%%%%%%%%%%%%%%%%%%%%%%%%%%%%%%%%%
\subsection{Conditional Processing}
\label{sec:conditional}

The package provides a mechanism to compile different versions
of a document. To customise the versions further some conditional processing
can come in handy to distinguish which version is being compiled.
The package provides two macros to describe the compilation context:

%%%%%%%%%%%%%%%%%%%%%%%%%%%%%%%%%%%%%%%%
\DescribeMacro{\ifchilddoc}
The conditional |\ifchilddoc| distinguishes between the compilation of
child documents and the main document:
%
\begin{center}
|\ifchilddoc |\textit{child-code}| |[|\||else |\textit{main-code}]| \||fi|
\end{center}

%%%%%%%%%%%%%%%%%%%%%%%%%%%%%%%%%%%%%%%%
\DescribeMacro{\childdocname}
\DescribeMacro{\childdocjob}
The macro |\childdocname| contains the filename (without extension)
of the main or child file being processed.
Note that |\childdocjob| will always contain the name of the main file.

%%%%%%%%%%%%%%%%%%%%%%%%%%%%%%%%%%%%%%%%
\paragraph{Title Page.}

Conditional processing can be used to include a title or banner page
in the main document when proper precautions are taken.
Importantly, the code in the main file should ensure that the page counter
(as well as other status parameters which are stored in the |.aux| files)
takes the same value after the conditional processing.
Otherwise the page numbers may take divergent values
depending on which part is compiled.

For example, a title page could be declared by:
%
\begin{center}
\begin{tabular}{l}
|\ifchilddoc\||else|\\
|\addtocounter{page}{-1}|\\
\textit{code for title page}\\
|\newpage|\\
|\||fi|
\end{tabular}
\end{center}
%
A banner page for the child documents can be generated by:
%
\begin{center}
\begin{tabular}{l}
|\ifchilddoc|\\
|\addtocounter{page}{-1}|\\
\textit{code for banner page}\\
|\newpage|\\
|\||fi|
\end{tabular}
\end{center}
%
Here one could write a message such as:
\begin{center}
|This is the part \childdocname{} of \childdocjob{}.|
\end{center}

%%%%%%%%%%%%%%%%%%%%%%%%%%%%%%%%%%%%%%%%%%%%%%%%%%%%%%%%%%%%%%%%%%%%%%%%%%%%%%%%
\subsection{Flags}
\label{sec:flags}

The package makes it easy to generate different versions
of the main or child documents.
To this end compilation flags can be defined
and assigned different default values.
They will be particularly useful in conjunction
with the forwarding mechanism described in \secref{sec:forward}.

For example, it may be useful to have a flag |\version|
which can be set to |draft| or |final|.
The document source will contain some conditional code
depending on the value of |\version|.
Suppose further, the flag should default to |final| for the main file
and to |draft| for child files
which is a natural assignment for editing the document.
This is achieved by placing the following code
in the preamble of the main document
(below the |\childdocmain| directive):
%
\begin{center}
\begin{tabular}{l}
|\ifchilddoc|\\
|\providecommand{\version}{draft}|\\
|\||else|\\
|\providecommand{\version}{final}|\\
|\||fi|
\end{tabular}
\end{center}
%
The definition by |\providecommand| makes sure
that previous definitions are not overwritten.
Further statements |\providecommand{\version}{...}|
can thus be added before the above code to override it.

For the main file, one might add a line
(between |\childdocmain| and the above block)
%
\begin{center}
|%\ifchilddoc\||else\providecommand{\version}{draft}\||fi|
\end{center}
%
which can be uncommented to produce a draft version.
Likewise one can add a line to the very top of a child file
(above the |\childdocof{|\textit{main}|}| directive)
%
\begin{center}
|%\providecommand{\version}{final}|
\end{center}
%
which can be uncommented to produce the final version of this child document.

%%%%%%%%%%%%%%%%%%%%%%%%%%%%%%%%%%%%%%%%%%%%%%%%%%%%%%%%%%%%%%%%%%%%%%%%%%%%%%%%
\subsection{Forwarding}
\label{sec:forward}

Different versions of the main or child documents
using compilation flags as described in \secref{sec:flags}
can be (permanently) stored in different files
for convenient compilation, viewing and distribution.
To this end, the package defines a command
to pass on compilation to a different file:

%%%%%%%%%%%%%%%%%%%%%%%%%%%%%%%%%%%%%%%%
\DescribeMacro{\childdocforward}
The command |\childdocforward| redirects processing to
another source file:
%
\begin{center}
\begin{tabular}{l}
|\input{childdoc.def}|\\
|\childdocforward[|\textit{main}|]{|\textit{dest}|}|\\
\end{tabular}
\end{center}
%
The argument \textit{dest} is the destination file
(without extension).
It should be the main file or one of the child files.
Note that further \textsf{childdoc} directives
such as |\childdocof| and |\childdocforward|
in the indicated file will be processed in this form.
The optional argument \textit{main}
passes on directly to the main file \textit{main}
while pretending to compile the child \textit{dest}.
This form behaves as if \textit{dest}
issues |\childdocof{|\textit{main}|}| right away,
and no further \textsf{childdoc} directives will be processed.

%%%%%%%%%%%%%%%%%%%%%%%%%%%%%%%%%%%%%%%%
\DescribeMacro{\...prefix}
In the alternative form |\childdocforwardprefix|,
%
\begin{center}
\begin{tabular}{l}
|\input{childdoc.def}|\\
|\childdocforwardprefix[|\textit{main}|]{|\textit{prefix}|}{|\textit{dest}|}|
\end{tabular}
\end{center}
%
the destination file is determined by a pattern
depending on the current file:
To make this work, the current file must be called
`{\textit{prefix}\hspace{0.2em}\textit{suffix}}'
with \textit{prefix} matching precisely the argument.
Processing is then passed on to the file
`{\textit{dest}\hspace{0.2em}\textit{suffix}}'.
Surely, the same effect is achieved by
directly specifying the
argument `{\textit{dest}\hspace{0.2em}\textit{suffix}}'
in the first form.
However, that requires to set up a different file
for each child. With the alternative form of the command
all these files can have exactly the same content
which simplifies setting them up and maintaining them.

For example, the following file |draft.tex|
with a compilation flag |\version| as described in \secref{sec:flags}
compiles the main document as a draft:
%
\begin{center}
\begin{tabular}{l}
|\def\version{draft}|\\
|\input{childdoc.def}|\\
|\childdocforward{|\textit{main}|}|
\end{tabular}
\end{center}
%
Likewise, the following files |final|\textit{nn}|.tex|
compile the final version of the child document
|child|\textit{nn}|.tex|:
%
\begin{center}
\begin{tabular}{l}
|\def\version{final}|\\
|\input{childdoc.def}|\\
|\childdocforwardprefix{final}{child}|
\end{tabular}
\end{center}
%

Note that when several versions of a main file and/or of each child file
are to be generated, it may be convenient to set up a |Makefile| or
shell script to automatise the process.

%%%%%%%%%%%%%%%%%%%%%%%%%%%%%%%%%%%%%%%%%%%%%%%%%%%%%%%%%%%%%%%%%%%%%%%%%%%%%%%%
\subsection{Command Line Processing}
\label{sec:commandline}

The effect of redirection files can also be achieved by invoking
the \LaTeX{} compiler with a more elaborate command line.
Most conveniently this should be done as part
of a shell script or a |Makefile|.

When using \textsf{childdoc} in the main file, the following
command lines effectively perform a redirection
(note that depending on the shell being used,
backslashes may have to be doubled: `|\|' $\to$ `|\\|'):
%
\begin{center}
|... -jobname "|\textit{target}|" |\\|"|[\textit{flags}]%
|\input{childdoc.def}\childdocforward[|\textit{main}|]{|\textit{dest}|}"|
\end{center}
%
Here \textit{target} is the name of the output file,
\textit{main} is the name of the main file
and \textit{dest} is the name of the main or child file to be processed
(all filenames without extensions).
The optional argument \textit{main} can be omitted
if \textit{main} matches \textit{dest}.
Optionally, compilation \textit{flags} can be defined via |\def| commands.
This command line makes the \TeX{} engine believe
it is compiling the file \textit{target}
whose content is specified as the latter parameter.
The provided code then forwards the processing to
\textit{main} or \textit{dest} as described in \secref{sec:forward}.

%%%%%%%%%%%%%%%%%%%%%%%%%%%%%%%%%%%%%%%%%%%%%%%%%%%%%%%%%%%%%%%%%%%%%%%%%%%%%%%%
\subsection{Include by Input}
\label{sec:input}

Including child documents by |\include| has some restrictions by design.
Most notably, the content of a child document always occupies
its own set of pages; pages cannot be shared between child documents.
Usually, this behaviour makes perfect sense
because each child document contain an essential part of the document.
However, in some situations it may be desirable to compose
a document from a collection of parts
without having mandatory page breaks between then.
For this case, the package
provides a mechanism to include parts
by |\input| which can also be processed individually.
However, by construction this mechanism
requires manual handling of the content to be output.

%%%%%%%%%%%%%%%%%%%%%%%%%%%%%%%%%%%%%%%%
\DescribeMacro{\ifchilddocmanual}
The main file should be prepared as usual, see \secref{sec:include}.
However, the document body must make a distinction
between processing of an individual part and of the main document, e.g.:
%
\begin{center}
\begin{tabular}{l}
|\ifchilddocmanual|\\
|\input{\childdocname}|\\
|\||else|\\
\textit{document body with }|\input{|\textit{part}|}|\\
|\||fi|
\end{tabular}
\end{center}
%
The conditional |\ifchilddocmanual| is true whenever
a part to be included by |\input| is being compiled,
and the name of the part is stored in |\childdocname|.

%%%%%%%%%%%%%%%%%%%%%%%%%%%%%%%%%%%%%%%%
\DescribeMacro{\childdocby}
Each part to be included by |\input| should start with:
%
\begin{center}
\begin{tabular}{l}
|\input{childdoc.def}|\\
|\childdocby{|\textit{main}|}|\\
\end{tabular}
\end{center}
%
The directive |\childdocby| is similar to |\childdocof|
described in \secref{sec:include},
but the subsequent selection of content must be done manually.
To that end, both |\ifchilddoc| and |\ifchilddocmanual|
will be true upon processing of a part,
and the name of the part is stored in |\childdocname|.
Note that |\jobname| will be set to the filename of the current part
so that each part receives an individual |.aux| file
that does not interfere with the |.aux| file(s) of the main document.
This behaviour can be altered by the alternative form
|\childdocby[*]{|\textit{main}|}| (with a non-empty optional argument)
which uses the |.aux| file of the main document
by setting |\jobname| to \textit{main}.

%%%%%%%%%%%%%%%%%%%%%%%%%%%%%%%%%%%%%%%%%%%%%%%%%%%%%%%%%%%%%%%%%%%%%%%%%%%%%%%%
\subsection{Driver Development}
\label{sec:driver}

The \textsf{childdoc} mechanism can also be use for the development
of definition files such as \LaTeX{} styles or classes.
This case differs from the above setup with multiple parts
included by |\include| in that no |\includeonly| should be invoked.
This can be achieved by starting the include file
(before |\ProvidesPackage|) with:
%
\begin{center}
\begin{tabular}{l}
|\input{childdoc.def}|\\
|\childdocforward{|\textit{main}|}|\\
\end{tabular}
\end{center}
%
or alternatively with:
%
\begin{center}
\begin{tabular}{l}
|\input{childdoc.def}|\\
|\childdocby{|\textit{main}|}|\\
\end{tabular}
\end{center}
%
Both forms have slightly different effects as described above.
The main file is prepared as usual, see \secref{sec:include}.

%%%%%%%%%%%%%%%%%%%%%%%%%%%%%%%%%%%%%%%%%%%%%%%%%%%%%%%%%%%%%%%%%%%%%%%%%%%%%%%%
\subsection{Legacy Detection}
\label{sec:detection}

The directive |\childdocmain| in the main file can detect
whether the complete document or merely a child is to be compiled
even without using the directive |\childdocof|.
This method is deprecated because it is less robust
and there is no compelling reason to use it;
it is merely provided for backward compatibility
and it may be removed in future versions.

If the detection mechanism is to be used,
it is mandatory to correctly specify
the filename of the main file as the argument of |\childdocmain|:
%
\begin{center}
\begin{tabular}{l}
|\input{childdoc.def}|\\
|\childdocmain{|\textit{main}|}|\\
\end{tabular}
\end{center}
%
If |\jobname| does not match the argument \textit{main} of |\childdocmain|,
it is assumed that |\jobname| points to the child file to be compiled.
When using |\childdocmain| with the main file specified as argument,
it suffices to start a child file
with just |\input{|\textit{main}|}|
without loading of the package and using |\childdocof|.
If instead all processing is done
with the appropriate \textsf{childdoc} directives,
the argument of \textit{main} of |\childdocmain| can be empty.

An alternative version of the command line processing described
in \secref{sec:commandline} using the detection mechanism reads:
%
\begin{center}
|... -jobname "|\textit{target}|" "|[\textit{flags}]%
[|\def\jobname{|\textit{dest}|}|]|\input{|\textit{main}|}"|
\end{center}

%%%%%%%%%%%%%%%%%%%%%%%%%%%%%%%%%%%%%%%%%%%%%%%%%%%%%%%%%%%%%%%%%%%%%%%%%%%%%%%%
\subsection{Manual Code}
\label{sec:manual}

In case one cannot be certain whether the definitions file |childdoc.def|
is installed on the target \TeX{} distribution
and one prefers not to ship it,
it is conceivable to paste a few relevant commands into the sources.

To that end, drop all statements |\input{childdoc.def}|
and perform the replacements as outlined below.
Instead of |\childdocmain{|\textit{main}|}| add the following code
to the top of the main file:
%
\begin{center}
\begin{tabular}{l}
|\||ifdefined\childdocname\endinput\||fi\newif\ifchilddoc|\\
|\edef\childdocname{\scantokens\expandafter{\jobname\noexpand}}|\\
|\def\childdocmain{|\textit{main}|}\||ifx\childdocmain\childdocname\||else|\\
|\childdoctrue\includeonly{\childdocname}\let\jobname\childdocmain\||fi|\\
\end{tabular}
\end{center}
%
Instead of |\childdocof{|\textit{main}|}| just include the main file
at the top of each child file:
%
\begin{center}
|\input{|\textit{main}|}|
\end{center}
%
A simple redirection |\childdocforward{|\textit{dest}|}| is achieved by:
%
\begin{center}
|\def\jobname{|\textit{dest}|}\input{\jobname}|
\end{center}
%
The redirection with prefix
|\childdocforwardprefix[|\textit{prefix}|]{|\textit{dest}|}|
is accomplished by:
%
\begin{center}
\begin{tabular}{l}
|{\edef\jobname{\scantokens\expandafter{\jobname\noexpand}}|\\
|\def\redirectjob |\textit{prefix}|#1~~~{\gdef\jobname{|\textit{dest}|#1}}|\\
|\expandafter\redirectjob\jobname~~~}\input{\jobname}|
\end{tabular}
\end{center}

In an alternative approach,
child documents can be compiled by a specific command line
without additional code or specific definitions:
%
\begin{center}
|... -jobname "|\textit{target}|" "|[\textit{flags}]%
|\includeonly{|\textit{dest}|}\input{|\textit{main}|}"|
\end{center}
%

%%%%%%%%%%%%%%%%%%%%%%%%%%%%%%%%%%%%%%%%%%%%%%%%%%%%%%%%%%%%%%%%%%%%%%%%%%%%%%%%
%%%%%%%%%%%%%%%%%%%%%%%%%%%%%%%%%%%%%%%%%%%%%%%%%%%%%%%%%%%%%%%%%%%%%%%%%%%%%%%%
\section{Information}

%%%%%%%%%%%%%%%%%%%%%%%%%%%%%%%%%%%%%%%%%%%%%%%%%%%%%%%%%%%%%%%%%%%%%%%%%%%%%%%%
\subsection{Copyright}

Copyright \copyright{} 2017--2018 Niklas Beisert

This work may be distributed and/or modified under the
conditions of the \LaTeX{} Project Public License, either version 1.3
of this license or (at your option) any later version.
The latest version of this license is in
  \url{http://www.latex-project.org/lppl.txt}
and version 1.3 or later is part of all distributions of \LaTeX{}
version 2005/12/01 or later.

This work has the LPPL maintenance status `maintained'.

The Current Maintainer of this work is Niklas Beisert.

This work consists of the files |README.txt|, |childdoc.ins| and |childdoc.dtx|
as well as the derived files |childdoc.def|, |cdocsamp.tex|
with |cdocsch1.tex|, |cdocsch2.tex|, |cdocspt3.tex|, |cdocspt4.tex|,
|cdocsdrf.tex|, |cdocsfn1.tex|, |cdocsfn2.tex|
as well as |childdoc.pdf|.

%%%%%%%%%%%%%%%%%%%%%%%%%%%%%%%%%%%%%%%%%%%%%%%%%%%%%%%%%%%%%%%%%%%%%%%%%%%%%%%%
\subsection{Files and Installation}

The package consists of the files:
%
\begin{center}
\begin{tabular}{ll}
    |README.txt|   & readme file \\
    |childdoc.ins| & installation file \\
    |childdoc.dtx| & source file \\
    |childdoc.def| & definition file \\
    |cdocsamp.tex| & sample main file \\
    |cdocsch1.tex| & sample include file \\
    |cdocsch2.tex| & sample include file \\
    |cdocspt3.tex| & sample part file \\
    |cdocspt4.tex| & sample part file \\
    |cdocsdrf.tex| & sample redirection file \\
    |cdocsfn1.tex| & sample redirection file \\
    |cdocsfn2.tex| & sample redirection file \\
    |childdoc.pdf| & manual
\end{tabular}
\end{center}
%
The distribution consists of the files
|README.txt|, |childdoc.ins| and |childdoc.dtx|.
%
\begin{itemize}
\item
Run (pdf)\LaTeX{} on |childdoc.dtx|
to compile the manual |childdoc.pdf| (this file).
\item
Run \LaTeX{} on |childdoc.ins| to create the definitions file |childdoc.def|
and the sample |cdocsamp.tex| with include files
|cdocsch1.tex|, |cdocsch2.tex|, |cdocspt3.tex|, |cdocspt4.tex|,
|cdocsdrf.tex|, |cdocsfn1.tex|, |cdocsfn2.tex|.
Then copy the file |childdoc.def| to an appropriate directory of your \LaTeX{}
distribution, e.g.\ \textit{texmf-root}|/tex/latex/childdoc|.
\end{itemize}

%%%%%%%%%%%%%%%%%%%%%%%%%%%%%%%%%%%%%%%%%%%%%%%%%%%%%%%%%%%%%%%%%%%%%%%%%%%%%%%%
\subsection{Related CTAN Packages}

There are several other packages which offer a similar functionality:
%
\begin{itemize}
\item
The packages
\href{http://ctan.org/pkg/docmute}{\textsf{docmute}},
\href{http://ctan.org/pkg/includex}{\textsf{includex}} and
\href{http://ctan.org/pkg/standalone}{\textsf{standalone}}
provide commands to include only the document body of
a child file thus allowing both files to be compiled individually.
\item
The packages \href{http://ctan.org/pkg/subdocs}{\textsf{subdocs}}
and \href{http://ctan.org/pkg/subfiles}{\textsf{subfiles}}
provide structures in which the main and child documents can be
encapsulated and allowing them to be compiled individually.
The inclusion mechanism is different from the conventional |\include|.
\item
The package \href{http://ctan.org/pkg/combine}{\textsf{combine}}
is an elaborate solution to combine several documents into one.
\end{itemize}
%
See also the CTAN topic \href{http://ctan.org/topic/subdocs}{\textsf{subdocs}}
for further related packages.
The present package differs from the above solutions in that
a document structure constructed with the conventional |\include| mechanism
just needs two extra commands at the top of every file
such that all constituent files can be compiled individually.

%%%%%%%%%%%%%%%%%%%%%%%%%%%%%%%%%%%%%%%%%%%%%%%%%%%%%%%%%%%%%%%%%%%%%%%%%%%%%%%%
%\subsection{Feature Suggestions}
%
%The following is a list of features which may be useful for future
%versions of this package:
%%
%\begin{itemize}
%\item
%\ldots
%\end{itemize}

%%%%%%%%%%%%%%%%%%%%%%%%%%%%%%%%%%%%%%%%%%%%%%%%%%%%%%%%%%%%%%%%%%%%%%%%%%%%%%%%
\subsection{Revision History}

%%%%%%%%%%%%%%%%%%%%%%%%%%%%%%%%%%%%%%%%
\paragraph{v2.0:} 2018/12/30

\begin{itemize}
\item
immediate forward processing
\item
added |\childdocby| mechanism
\item
manual restructured
\end{itemize}

%%%%%%%%%%%%%%%%%%%%%%%%%%%%%%%%%%%%%%%%
\paragraph{v1.6:} 2018/01/17

\begin{itemize}
\item
application for development of include files
\item
corrections to manual
\end{itemize}

%%%%%%%%%%%%%%%%%%%%%%%%%%%%%%%%%%%%%%%%
\paragraph{v1.5:} 2017/05/21

\begin{itemize}
\item
more complete structuring introduced
\item
|\childdocof| introduced
\item
|\childdoc| renamed to |\childdocmain|
\item
|\childredirect| renamed to |\childdocforward| and |\childdocforwardprefix|
and functionality expanded
\end{itemize}

%%%%%%%%%%%%%%%%%%%%%%%%%%%%%%%%%%%%%%%%
\paragraph{v1.0:} 2017/04/27

\begin{itemize}
\item
manual and install package
\item
first version published on CTAN
\end{itemize}

%%%%%%%%%%%%%%%%%%%%%%%%%%%%%%%%%%%%%%%%
\paragraph{v0.6:} 2017/04/26

\begin{itemize}
\item
redirection mechanism added
\end{itemize}

%%%%%%%%%%%%%%%%%%%%%%%%%%%%%%%%%%%%%%%%
\paragraph{v0.5:} 2017/04/26

\begin{itemize}
\item
functionality in definition file
\end{itemize}


%%%%%%%%%%%%%%%%%%%%%%%%%%%%%%%%%%%%%%%%%%%%%%%%%%%%%%%%%%%%%%%%%%%%%%%%%%%%%%%%
%%%%%%%%%%%%%%%%%%%%%%%%%%%%%%%%%%%%%%%%%%%%%%%%%%%%%%%%%%%%%%%%%%%%%%%%%%%%%%%%
%%%%%%%%%%%%%%%%%%%%%%%%%%%%%%%%%%%%%%%%%%%%%%%%%%%%%%%%%%%%%%%%%%%%%%%%%%%%%%%%
\appendix

\settowidth\MacroIndent{\rmfamily\scriptsize 000\ }

 \DocInput{childdoc.dtx}

\end{document}
%</driver>
% \fi
%
% %%%%%%%%%%%%%%%%%%%%%%%%%%%%%%%%%%%%%%%%%%%%%%%%%%%%%%%%%%%%%%%%%%%%%%%%%%%%%%
% %%%%%%%%%%%%%%%%%%%%%%%%%%%%%%%%%%%%%%%%%%%%%%%%%%%%%%%%%%%%%%%%%%%%%%%%%%%%%%
% \section{Sample}
%\iffalse
%<*samplemain>
%\fi
%
% The following presents a sample document
% with two chapters, two parts, a title page,
% a compile flag as well as three forwarding files to set the flag.
% It consists of eight |.tex| files:
% \begin{center}
% \begin{tabular}{ll}
% |cdocsamp.tex|&main file\\
% |cdocsch1.tex|&include file for chapter 1\\
% |cdocsch2.tex|&include file for chapter 2\\
% |cdocspt3.tex|&include file for part 3\\
% |cdocspt4.tex|&include file for part 4\\
% |cdocsdrf.tex|&forwarding file for main file in draft mode\\
% |cdocsfi1.tex|&forwarding file for final version of chapter 1\\
% |cdocsfi2.tex|&forwarding file for final version of chapter 2\\
% \end{tabular}
% \end{center}
% Each of the eight files can be compiled directly by the \LaTeX{} compiler.
%
% %%%%%%%%%%%%%%%%%%%%%%%%%%%%%%%%%%%%%%
% \paragraph{Main File.}
%
% The main file is called |cdocsamp.tex|.
%
% Load the \textsf{childdoc} definitions and
% declare the filename for the main document:
%    \begin{macrocode}
\input{childdoc.def}
\childdocmain{}
%    \end{macrocode}

% Optional override for |\version| flag:
%    \begin{macrocode}
%%\ifchilddoc\else\providecommand{\version}{draft}\fi
%    \end{macrocode}

% Define the default values for the |\version| flag
% (|final| for the main file and |draft| for childs):
%    \begin{macrocode}
\ifchilddoc
\providecommand{\version}{draft}
\else
\providecommand{\version}{final}
\fi
%    \end{macrocode}

% Load the standard document class:
%    \begin{macrocode}
\documentclass[12pt]{article}
%    \end{macrocode}

% Start the document body:
%    \begin{macrocode}
\begin{document}
%    \end{macrocode}

% Declare a title page.
% Print title, part of document being processed and version flag:
%    \begin{macrocode}
\addtocounter{page}{-1}
\begin{center}
{\LARGE\bfseries{}childdoc example\par}
\vspace{1cm}
\ifchilddoc
\ifchilddocmanual part\else chapter\fi:
`\childdocname' of `\childdocjob'\par
\else
main document: `\childdocjob'\par
\fi
version: \version\par
\end{center}
\newpage
%    \end{macrocode}

% Manually include selected file,
% otherwise process as usual:
%    \begin{macrocode}
\ifchilddocmanual
\section*{part `\childdocname'}
\input{\childdocname}
\else
%    \end{macrocode}

% Include the two chapters:
%    \begin{macrocode}
\include{cdocsch1}
\include{cdocsch2}
%    \end{macrocode}

% Include the two parts unless only chapters should be displayed:
%    \begin{macrocode}
\ifchilddoc\else
\section{part three}
\input{cdocspt3}
\section{part four}
\input{cdocspt4}
\fi
%    \end{macrocode}

% Process as usual until here:
%    \begin{macrocode}
\fi
%    \end{macrocode}

% End of document body:
%    \begin{macrocode}
\end{document}
%    \end{macrocode}
%\iffalse
%</samplemain>
%\fi
%
% %%%%%%%%%%%%%%%%%%%%%%%%%%%%%%%%%%%%%%
% \paragraph{Chapter Include Files.}
%
% The include files are called |cdocsch1.tex| and |cdocsch2.tex|.
%
%\iffalse
%<*samplechap1|samplechap2>
%\fi

% Optional override for |\version| flag:
%    \begin{macrocode}
%%\providecommand{\version}{final}
%    \end{macrocode}

% Include the main document:
%    \begin{macrocode}
\input{childdoc.def}
\childdocof{cdocsamp}
%    \end{macrocode}

%\iffalse
%</samplechap1|samplechap2>
%\fi
%
%\iffalse
%<*samplechap1>
%\fi
% Some text for chapter 1:
%    \begin{macrocode}
\section{one}
some text in chapter one
%    \end{macrocode}

%\iffalse
%</samplechap1>
%\fi
% Some text for chapter 2:
%\iffalse
%<*samplechap2>
%\fi
%    \begin{macrocode}
\section{two}
more text in chapter two
%    \end{macrocode}

%\iffalse
%</samplechap2>
%\fi
%
% %%%%%%%%%%%%%%%%%%%%%%%%%%%%%%%%%%%%%%
% \paragraph{Part Include Files.}
%
% The include files are called |cdocspt3.tex| and |cdocspt4.tex|.
%
%\iffalse
%<*samplepart3|samplepart4>
%\fi

% Optional override for |\version| flag:
%    \begin{macrocode}
%%\providecommand{\version}{final}
%    \end{macrocode}

% Include the main document:
%    \begin{macrocode}
\input{childdoc.def}
\childdocby{cdocsamp}
%    \end{macrocode}

%\iffalse
%</samplepart3|samplepart4>
%\fi
%
%\iffalse
%<*samplepart3>
%\fi
% Some text for part 3:
%    \begin{macrocode}
some text in part three
%    \end{macrocode}

%\iffalse
%</samplepart3>
%\fi
% Some text for part 4:
%\iffalse
%<*samplepart4>
%\fi
%    \begin{macrocode}
more text in part four
%    \end{macrocode}

%\iffalse
%</samplepart4>
%\fi
%
% %%%%%%%%%%%%%%%%%%%%%%%%%%%%%%%%%%%%%%
% \paragraph{Forwarding for a Complete Draft.}
%
% The following forwarding file |cdocsdrf.tex|
% compiles the main document in draft mode:
%\iffalse
%<*sampledraft>
%\fi
%    \begin{macrocode}
\def\version{draft}
\input{childdoc.def}
\childdocforward{cdocsamp}
%    \end{macrocode}

%\iffalse
%</sampledraft>
%\fi
%
% %%%%%%%%%%%%%%%%%%%%%%%%%%%%%%%%%%%%%%
% \paragraph{Forwarding for Final Version of the Chapters.}
%
% The following forwarding files |cdocsfn1.tex| and |cdocsfn2.tex|
% (with identical content)
% compile the final versions of the child documents
% |cdocsch1.tex| and |cdocsch2.tex|, respectively:
%\iffalse
%<*samplefinal>
%\fi
%    \begin{macrocode}
\def\version{final}
\input{childdoc.def}
\childdocforwardprefix[cdocsamp]{cdocsfn}{cdocsch}
%    \end{macrocode}

%\iffalse
%</samplefinal>
%\fi
%
% %%%%%%%%%%%%%%%%%%%%%%%%%%%%%%%%%%%%%%
% \paragraph{Command Line Processing.}
%
% The following three command lines generate the output files
% |cdocscld|, |cdocscl1| and |cdocscl2|
% which should be identical to
% |cdocsdrf|, |cdocsch1| and |cdocsfn2|, respectively:
% \begin{center}
% \begin{tabular}{l}
% |latex -jobname cdocscld \|\\
% |  "\def\version{draft}\input{childdoc.def}\childdocforward{cdocsamp}"|\\
% |latex -jobname cdocscl1 \|\\
% |  "\input{childdoc.def}\childdocforward[cdocsamp]{cdocsch1}"|\\
% |latex -jobname cdocscl2 \|\\
% |  "\def\version{final}\input{childdoc.def}\childdocforward{cdocsch2}"|
% \end{tabular}
% \end{center}
% Note that the trailing backslash on each first line
% merely continues the input to the second line
% (for convenient cut ant paste).
% Furthermore, the command |latex| can be replaced by any
% of its alternative versions such as |pdflatex|.
%
% %%%%%%%%%%%%%%%%%%%%%%%%%%%%%%%%%%%%%%%%%%%%%%%%%%%%%%%%%%%%%%%%%%%%%%%%%%%%%%
% %%%%%%%%%%%%%%%%%%%%%%%%%%%%%%%%%%%%%%%%%%%%%%%%%%%%%%%%%%%%%%%%%%%%%%%%%%%%%%
% \section{Implementation}
%\iffalse
%<*package>
%\fi
%
% This section describes the definitions file |childdoc.def|.

% The definitions cannot be loaded using |\usepackage| or |\RequirePackage|
% which has a mechanism to prevent loading a style file more than once.
% When loading the definitions by means of |\input|
% multiple instances have to be prevented manually:
%\iffalse
%This code needs to be before the `\ProvidesFile' directive
%which is defined at the beginning of this file.
%Therefore it is also placed there and commented out here.
%</package>
%<*discard>
%\fi
%    \begin{macrocode}
\ifdefined\childdocmain\endinput\fi
%    \end{macrocode}
%\iffalse
%</discard>
%<*package>
%\fi
%
% \macro{\ifchilddoc}
% \macro{\ifchilddocmanual}
% The conditional |\ifchilddoc| tells whether a
% child (true) or main (false) document is being compiled.
% The conditional |\ifchilddocmanual| tells whether
% the |\includeonly| mechanism is used (false) or
% the selection of child files must be performed manually (true).
% The definitions initialise to false:
%    \begin{macrocode}
\newif\ifchilddoc
\newif\ifchilddocmanual
%    \end{macrocode}

% \macro{\childdocname}
% \macro{\childdocjob}
% The macro |\childdocname| stores the name of the main document
% to be compiled. The macro |\childdocjob| stores the name of
% the document on which the \LaTeX{} compiler was originally invoked.
% The content of |\jobname| cannot be compared
% to filenames specified in the source due to different catcodes.
% The following code rescans |\jobname|, stores the result
% in |\childdocname| and saves a copy in |\childdocjob|:
%    \begin{macrocode}
\edef\childdocname{\scantokens\expandafter{\jobname\noexpand}}
\let\childdocjob\childdocname
%    \end{macrocode}

% \macro{\childdocdisable}
% The macro |\childdocdisable| prevents the main file
% from being processed more than once.
% At this stage, the main document command |\childdocmain|
% is assumed to be called once again where it should do nothing.
% Any subsequent call to it should prevent
% a secondary processing of the main document
% It overwrites the forwarding commands
% |\childdocof| and |\childdocforward|
% with empty macros to prevent further inclusions of the main document:
%    \begin{macrocode}
\newcommand{\childdocdisable}
{
  \renewcommand{\childdocmain}[1]{\renewcommand{\childdocmain}[1]{\endinput}}
  \renewcommand{\childdocof}[1]{}
  \renewcommand{\childdocby}[2][]{}
  \renewcommand{\childdocforward}[2][]{}
  \renewcommand{\childdocdisable}{}
}
%    \end{macrocode}

% \macro{\childdocmain}
% The macro |\childdocmain| is to be called at the top of the main file
% with nothing or the main filename (without extension) as argument.
% First, it breaks loops.
% If the argument is not empty and does not match |\childdocname|
% (which is set by the first inclusion of |childdoc.def|),
% |\ifchilddoc| is set to true, |\includeonly| is applied to the child file
% and |\jobname| is set to the main file
% (for proper handling of |.aux| files):
%    \begin{macrocode}
\newcommand{\childdocmain}[1]
{
  \childdocdisable\childdocmain{}
  \if?#1?\else
    \begingroup
      \def\childdoctmp{#1}
      \ifx\childdoctmp\childdocname
        \def\childdoctmp{}
      \else
        \def\childdoctmp
        {
          \childdoctrue
          \includeonly{\childdocname}
          \def\childdocjob{#1}
          \def\jobname{#1}
        }
      \fi
      \expandafter
    \endgroup
    \childdoctmp
  \fi
}
%    \end{macrocode}

% \macro{\childdocof}
% The command |\childdocof| redirects
% compilation to the main file |#1|.
%    \begin{macrocode}
\newcommand{\childdocof}[1]
{
  \childdocdisable
  \childdoctrue
  \includeonly{\childdocname}
  \def\jobname{#1}
  \def\childdocjob{#1}
  \input{#1}
}
%    \end{macrocode}

% \macro{\childdocby}
% The command |\childdocby| ....
%    \begin{macrocode}
\newcommand{\childdocby}[2][]
{
  \childdocdisable
  \childdoctrue
  \childdocmanualtrue
  \if?#1?\else
    \def\jobname{#2}
  \fi
  \def\childdocjob{#2}
  \input{#2}
  \endinput
}
%    \end{macrocode}

% \macro{\childdocforward}
% The command |\childdocforward| redirects
% compilation to the main file or
% (if the optional argument is given) a child file.
% Parameters are set as if the main file
% or a child file starting with |\childdocof| was compiled.
% Then compilation is handed over to the main file:
%    \begin{macrocode}
\newcommand{\childdocforward}[2][]
{
  \begingroup
    \if?#1?
      \def\childdoctmp
      {
        \def\childdocname{#2}
        \def\childdocjob{#2}
        \def\jobname{#2}
        \input{#2}
        \endinput
      }
    \else
      \def\childdoctmp
      {
        \childdocdisable
        \def\childdocname{#2}
        \childdoctrue
        \includeonly{#2}
        \def\childdocjob{#1}
        \def\jobname{#1}
        \input{#1}
        \endinput
      }
    \fi
    \expandafter
  \endgroup
  \childdoctmp
}
%    \end{macrocode}

% \macro{\childdocforwardprefix}
% The command |\childdocforwardprefix| redirects
% compilation to the main or a child file by means of a pattern.
% The prefix |#1| in the current filename is replaced by |#2|
% and the suffix of the current filename is kept
% (it is assumed that the filename does not contain the substring `|~~~|'
% which is used as a delimiter).
% Compilation is handed over to the new file by |\childdocforward|:
%    \begin{macrocode}
\newcommand{\childdocforwardprefix}[3][]
{
  \begingroup
    \def\childdocextract #2##1~~~{\def\childdoctmp{\childdocforward[#1]{#3##1}}}
    \expandafter\childdocextract\childdocname~~~
    \expandafter
  \endgroup
  \childdoctmp
}
%    \end{macrocode}

% \macro{\childdoc}
% The deprecated macro |\childdoc| is a legacy version of |\childdocmain|:
%    \begin{macrocode}
\newcommand{\childdoc}{\childdocmain}
%    \end{macrocode}

% \macro{\childdocredirect}
% The deprecated macro |\childdocredirect| is a legacy version
% of |\childdocforward| and |\childdocforwardprefix|:
%    \begin{macrocode}
\newcommand{\childdocredirect}[2][]
{
  \begingroup
    \if?#1?
      \def\childdoctmp{\childdocforward{#2}}
    \else
      \def\childdoctmp{\childdocforwardprefix{#1}{#2}}
    \fi
    \expandafter
  \endgroup
  \childdoctmp
}
%    \end{macrocode}

%\iffalse
%</package>
%\fi
%
\endinput
|\\
|\childdocforwardprefix[|\textit{main}|]{|\textit{prefix}|}{|\textit{dest}|}|
\end{tabular}
\end{center}
%
the destination file is determined by a pattern
depending on the current file:
To make this work, the current file must be called
`{\textit{prefix}\hspace{0.2em}\textit{suffix}}'
with \textit{prefix} matching precisely the argument.
Processing is then passed on to the file
`{\textit{dest}\hspace{0.2em}\textit{suffix}}'.
Surely, the same effect is achieved by
directly specifying the
argument `{\textit{dest}\hspace{0.2em}\textit{suffix}}'
in the first form.
However, that requires to set up a different file
for each child. With the alternative form of the command
all these files can have exactly the same content
which simplifies setting them up and maintaining them.

For example, the following file |draft.tex|
with a compilation flag |\version| as described in \secref{sec:flags}
compiles the main document as a draft:
%
\begin{center}
\begin{tabular}{l}
|\def\version{draft}|\\
|% \iffalse
%
% childdoc.dtx Copyright (C) 2017-2018 Niklas Beisert
%
% This work may be distributed and/or modified under the
% conditions of the LaTeX Project Public License, either version 1.3
% of this license or (at your option) any later version.
% The latest version of this license is in
%   http://www.latex-project.org/lppl.txt
% and version 1.3 or later is part of all distributions of LaTeX
% version 2005/12/01 or later.
%
% This work has the LPPL maintenance status `maintained'.
%
% The Current Maintainer of this work is Niklas Beisert.
%
% This work consists of the files childdoc.dtx and childdoc.ins
% and the derived files childdoc.def and cdocsamp.tex with
% cdocsch1.tex, cdocsch2.tex, cdocsdrf.tex, cdocsfn1.tex, cdocsfn2.tex.
%
%<package>\ifdefined\childdocmain\endinput\fi
%<package>\ProvidesFile{childdoc.def}[2018/12/30 v2.0 child document driver]
%<samplemain>\ProvidesFile{cdocsamp.tex}[2018/12/30 v2.0 sample for childdoc]
%<*driver>
%\ProvidesFile{childdoc.drv}[2018/12/30 v2.0 childdoc reference manual file]
\PassOptionsToClass{10pt,a4paper}{article}
\documentclass{ltxdoc}

\usepackage[margin=35mm]{geometry}
\usepackage{hyperref}
\usepackage{hyperxmp}
\usepackage[usenames]{color}

\hypersetup{colorlinks=true}
\hypersetup{pdfstartview=FitH}
\hypersetup{pdfpagemode=UseNone}
\hypersetup{pdfsource={}}
\hypersetup{pdflang={en-UK}}
\hypersetup{pdfcopyright={Copyright 2017-2018 Niklas Beisert.
  This work may be distributed and/or modified under the
  conditions of the LaTeX Project Public License, either version 1.3
  of this license or (at your option) any later version.}}
\hypersetup{pdflicenseurl={http://www.latex-project.org/lppl.txt}}
\hypersetup{pdfcontactaddress={ETH Zurich, ITP, HIT K,
  Wolfgang-Pauli-Strasse 27}}
\hypersetup{pdfcontactpostcode={8093}}
\hypersetup{pdfcontactcity={Zurich}}
\hypersetup{pdfcontactcountry={Switzerland}}
\hypersetup{pdfcontactemail={nbeisert@itp.phys.ethz.ch}}
\hypersetup{pdfcontacturl={http://people.phys.ethz.ch/\xmptilde nbeisert/}}

\newcommand{\secref}[1]{\hyperref[#1]{section \ref*{#1}}}

\parskip1ex
\parindent0pt
\let\olditemize\itemize
\def\itemize{\olditemize\parskip0pt}

\begin{document}

\title{The \textsf{childdoc} Package}
\hypersetup{pdftitle={The childdoc Package}}
\author{Niklas Beisert\\[2ex]
  Institut f\"ur Theoretische Physik\\
  Eidgen\"ossische Technische Hochschule Z\"urich\\
  Wolfgang-Pauli-Strasse 27, 8093 Z\"urich, Switzerland\\[1ex]
  \href{mailto:nbeisert@itp.phys.ethz.ch}
  {\texttt{nbeisert@itp.phys.ethz.ch}}}
\hypersetup{pdfauthor={Niklas Beisert}}
\hypersetup{pdfsubject={Manual for the LaTeX2e Package childdoc}}
\date{30 December 2018, \textsf{v2.0}}
\maketitle

\begin{abstract}\noindent
\textsf{childdoc} is a \LaTeXe{} package
that enables the direct compilation
of document sections included by |\include|
to individual files.
\end{abstract}

\begingroup
\parskip0ex
\tableofcontents
\endgroup

%%%%%%%%%%%%%%%%%%%%%%%%%%%%%%%%%%%%%%%%%%%%%%%%%%%%%%%%%%%%%%%%%%%%%%%%%%%%%%%%
%%%%%%%%%%%%%%%%%%%%%%%%%%%%%%%%%%%%%%%%%%%%%%%%%%%%%%%%%%%%%%%%%%%%%%%%%%%%%%%%
\section{Introduction}

\LaTeX{} provides a mechanism to structure a large document (such as a book)
into a main file and several child files (containing the chapters)
using the |\include| command.
This mechanism is beneficial for documents
which span hundreds of pages in order to
make the source file(s) more manageable.
Moreover, compilation can be restricted to
selected child files by means of the |\includeonly| command.
The latter feature can be used to reduce the compilation time while editing
(this was significantly more useful in the earlier days of \LaTeX{})
or to generate a smaller document which is easier to navigate.
Another application of |\includeonly| is to generate
documents consisting of selected parts of the complete document.

However, there are a few drawbacks of the plain |\include| mechanism:
\begin{itemize}
\item
The child files cannot be compiled on their own,
they can only be compiled via the main file.
A naive editing environment
(such as a text editor with an option
to have the current file processed by \LaTeX)
may require one to switch to the main file before compiling;
attempting to compile the child file produces errors.
\item
The main file must be modified (each time)
to adjust the |\includeonly| command
to the present needs. This easily leaves the main file in a messy state.
\item
The generated document will always carry the filename
of the main document. This is inconvenient if
several child files are to be compiled and
to be kept for distribution.
\end{itemize}

The present package provides a simple interface
to make child files individually compilable by \LaTeX{}.
Compiling a child file then has the same effect as compiling
the main file with an |\includeonly| command
to select the appropriate child.
Moreover the generated document will carry the name of the child
rather than the main file.
This resolves all three above issues.

This feature is meant to make the editing of books,
thesis documents and lecture notes somewhat more convenient.
However, the package can also be used efficiently for
composing a series of documents (such as exercise sheets)
which are typically distributed individually.
It then assists the author in generating the individual documents
(potentially in different versions)
as well as a document containing the collected series.
Another application is in developing style files
or other kinds of included material
where compilation of the style file could redirect
to a sample or test file.

%%%%%%%%%%%%%%%%%%%%%%%%%%%%%%%%%%%%%%%%%%%%%%%%%%%%%%%%%%%%%%%%%%%%%%%%%%%%%%%%
%%%%%%%%%%%%%%%%%%%%%%%%%%%%%%%%%%%%%%%%%%%%%%%%%%%%%%%%%%%%%%%%%%%%%%%%%%%%%%%%
\section{Usage}

First of all, the package \textsf{childdoc} is \emph{not} a standard
\LaTeXe{} |.sty| style file! Therefore it needs to be invoked in
a non-standard way.

%%%%%%%%%%%%%%%%%%%%%%%%%%%%%%%%%%%%%%%%%%%%%%%%%%%%%%%%%%%%%%%%%%%%%%%%%%%%%%%%
\subsection{Included Files}
\label{sec:include}

%%%%%%%%%%%%%%%%%%%%%%%%%%%%%%%%%%%%%%%%
\DescribeMacro{\childdocmain}
To use the package, add the commands
\begin{center}
\begin{tabular}{l}
|\input{childdoc.def}|\\
|\childdocmain{}|\\
\end{tabular}
\end{center}
at the very top of the main \LaTeX{} file,
in particular \emph{before} the |\documentclass| statement!
The argument of |\childdocmain| should be left empty
(but it must be present).

%%%%%%%%%%%%%%%%%%%%%%%%%%%%%%%%%%%%%%%%
\DescribeMacro{\childdocof}
Furthermore, add the commands
\begin{center}
\begin{tabular}{l}
|\input{childdoc.def}|\\
|\childdocof{|\textit{main}|}|\\
\end{tabular}
\end{center}
at the top of every child file \textit{child}
which is included by |\include{|\textit{child}|}|
from within the main file
(or at least for those files to be compiled individually).
The argument \textit{main} must be the filename of the main file.

There are a couple of
considerations in setting up the main and child documents:

%%%%%%%%%%%%%%%%%%%%%%%%%%%%%%%%%%%%%%%%
\paragraph{Restrictions.}

Please note the following restrictions:
\begin{itemize}
\item
|\childdocmain| must be called with one argument \textit{main}
to ensure compatibility with earlier version of the package.
It must either be empty (|\childdocmain{}|)
or precisely match the filename of the main file in which it is specified.
See \secref{sec:detection} for further information.
\item
The filename \textit{main} must be specified without the |.tex| extension.
\item
The filename \textit{main} is case sensitive
(even in case-insensitive file systems)
due to internal string comparison.
\item
The argument \textit{main} should be fully expanded, it cannot be a macro.
\item
Subdirectories and special characters should be avoided in filenames.
\item
The command |\childdocmain{|\textit{main}|}| must be followed by a whitespace.
It should not be followed immediately by another command
or by a comment mark `|%|'.
This is because the \TeX{} parser reads the token immediately following
the argument of |\childdocmain| and puts it
at the beginning of every child section;
however, a white\-space is ignored.
\end{itemize}

%%%%%%%%%%%%%%%%%%%%%%%%%%%%%%%%%%%%%%%%
\paragraph{Content of Main File.}

It is advisable to place all content in the child files included by |\include|.
Any output contained in the main file will appear in all child documents
unless suppressed manually;
it cannot be suppressed automatically by the |\includeonly| directive
and thus should normally be avoided.
A method to include some content in the main file
by means of conditional processing is described in \secref{sec:conditional}.

%%%%%%%%%%%%%%%%%%%%%%%%%%%%%%%%%%%%%%%%
\paragraph{Page Numbering.}

When only a part of the document is compiled,
the appropriate numbering of pages
(as well as other status parameters)
is determined from the |.aux| files.
The latter contain information from previous passes.
However this information needs to propagate through
all intermediate child documents.
Therefore the page numbering in child documents may well
be inconsistent until the complete document is compiled at least once.

A useful (if unconventional) way to always ensure a consistent
page numbering is to restart the numbering in each child document
and denote the pages by `\textit{child}|.|\textit{page}'
where \textit{child} represents the chapter/section number of the child file.
This can be achieved by the command
|\numberwithin{page}{|\textit{child}|}|
of the \textsf{amsmath} package
where \textit{child} can be |chapter| or |section|
depending on the chosen structuring.
Alternatively, one can modify the macro |\thepage| appropriately
and reset the counter |page| at the start of each child file.

%%%%%%%%%%%%%%%%%%%%%%%%%%%%%%%%%%%%%%%%%%%%%%%%%%%%%%%%%%%%%%%%%%%%%%%%%%%%%%%%
\subsection{Conditional Processing}
\label{sec:conditional}

The package provides a mechanism to compile different versions
of a document. To customise the versions further some conditional processing
can come in handy to distinguish which version is being compiled.
The package provides two macros to describe the compilation context:

%%%%%%%%%%%%%%%%%%%%%%%%%%%%%%%%%%%%%%%%
\DescribeMacro{\ifchilddoc}
The conditional |\ifchilddoc| distinguishes between the compilation of
child documents and the main document:
%
\begin{center}
|\ifchilddoc |\textit{child-code}| |[|\||else |\textit{main-code}]| \||fi|
\end{center}

%%%%%%%%%%%%%%%%%%%%%%%%%%%%%%%%%%%%%%%%
\DescribeMacro{\childdocname}
\DescribeMacro{\childdocjob}
The macro |\childdocname| contains the filename (without extension)
of the main or child file being processed.
Note that |\childdocjob| will always contain the name of the main file.

%%%%%%%%%%%%%%%%%%%%%%%%%%%%%%%%%%%%%%%%
\paragraph{Title Page.}

Conditional processing can be used to include a title or banner page
in the main document when proper precautions are taken.
Importantly, the code in the main file should ensure that the page counter
(as well as other status parameters which are stored in the |.aux| files)
takes the same value after the conditional processing.
Otherwise the page numbers may take divergent values
depending on which part is compiled.

For example, a title page could be declared by:
%
\begin{center}
\begin{tabular}{l}
|\ifchilddoc\||else|\\
|\addtocounter{page}{-1}|\\
\textit{code for title page}\\
|\newpage|\\
|\||fi|
\end{tabular}
\end{center}
%
A banner page for the child documents can be generated by:
%
\begin{center}
\begin{tabular}{l}
|\ifchilddoc|\\
|\addtocounter{page}{-1}|\\
\textit{code for banner page}\\
|\newpage|\\
|\||fi|
\end{tabular}
\end{center}
%
Here one could write a message such as:
\begin{center}
|This is the part \childdocname{} of \childdocjob{}.|
\end{center}

%%%%%%%%%%%%%%%%%%%%%%%%%%%%%%%%%%%%%%%%%%%%%%%%%%%%%%%%%%%%%%%%%%%%%%%%%%%%%%%%
\subsection{Flags}
\label{sec:flags}

The package makes it easy to generate different versions
of the main or child documents.
To this end compilation flags can be defined
and assigned different default values.
They will be particularly useful in conjunction
with the forwarding mechanism described in \secref{sec:forward}.

For example, it may be useful to have a flag |\version|
which can be set to |draft| or |final|.
The document source will contain some conditional code
depending on the value of |\version|.
Suppose further, the flag should default to |final| for the main file
and to |draft| for child files
which is a natural assignment for editing the document.
This is achieved by placing the following code
in the preamble of the main document
(below the |\childdocmain| directive):
%
\begin{center}
\begin{tabular}{l}
|\ifchilddoc|\\
|\providecommand{\version}{draft}|\\
|\||else|\\
|\providecommand{\version}{final}|\\
|\||fi|
\end{tabular}
\end{center}
%
The definition by |\providecommand| makes sure
that previous definitions are not overwritten.
Further statements |\providecommand{\version}{...}|
can thus be added before the above code to override it.

For the main file, one might add a line
(between |\childdocmain| and the above block)
%
\begin{center}
|%\ifchilddoc\||else\providecommand{\version}{draft}\||fi|
\end{center}
%
which can be uncommented to produce a draft version.
Likewise one can add a line to the very top of a child file
(above the |\childdocof{|\textit{main}|}| directive)
%
\begin{center}
|%\providecommand{\version}{final}|
\end{center}
%
which can be uncommented to produce the final version of this child document.

%%%%%%%%%%%%%%%%%%%%%%%%%%%%%%%%%%%%%%%%%%%%%%%%%%%%%%%%%%%%%%%%%%%%%%%%%%%%%%%%
\subsection{Forwarding}
\label{sec:forward}

Different versions of the main or child documents
using compilation flags as described in \secref{sec:flags}
can be (permanently) stored in different files
for convenient compilation, viewing and distribution.
To this end, the package defines a command
to pass on compilation to a different file:

%%%%%%%%%%%%%%%%%%%%%%%%%%%%%%%%%%%%%%%%
\DescribeMacro{\childdocforward}
The command |\childdocforward| redirects processing to
another source file:
%
\begin{center}
\begin{tabular}{l}
|\input{childdoc.def}|\\
|\childdocforward[|\textit{main}|]{|\textit{dest}|}|\\
\end{tabular}
\end{center}
%
The argument \textit{dest} is the destination file
(without extension).
It should be the main file or one of the child files.
Note that further \textsf{childdoc} directives
such as |\childdocof| and |\childdocforward|
in the indicated file will be processed in this form.
The optional argument \textit{main}
passes on directly to the main file \textit{main}
while pretending to compile the child \textit{dest}.
This form behaves as if \textit{dest}
issues |\childdocof{|\textit{main}|}| right away,
and no further \textsf{childdoc} directives will be processed.

%%%%%%%%%%%%%%%%%%%%%%%%%%%%%%%%%%%%%%%%
\DescribeMacro{\...prefix}
In the alternative form |\childdocforwardprefix|,
%
\begin{center}
\begin{tabular}{l}
|\input{childdoc.def}|\\
|\childdocforwardprefix[|\textit{main}|]{|\textit{prefix}|}{|\textit{dest}|}|
\end{tabular}
\end{center}
%
the destination file is determined by a pattern
depending on the current file:
To make this work, the current file must be called
`{\textit{prefix}\hspace{0.2em}\textit{suffix}}'
with \textit{prefix} matching precisely the argument.
Processing is then passed on to the file
`{\textit{dest}\hspace{0.2em}\textit{suffix}}'.
Surely, the same effect is achieved by
directly specifying the
argument `{\textit{dest}\hspace{0.2em}\textit{suffix}}'
in the first form.
However, that requires to set up a different file
for each child. With the alternative form of the command
all these files can have exactly the same content
which simplifies setting them up and maintaining them.

For example, the following file |draft.tex|
with a compilation flag |\version| as described in \secref{sec:flags}
compiles the main document as a draft:
%
\begin{center}
\begin{tabular}{l}
|\def\version{draft}|\\
|\input{childdoc.def}|\\
|\childdocforward{|\textit{main}|}|
\end{tabular}
\end{center}
%
Likewise, the following files |final|\textit{nn}|.tex|
compile the final version of the child document
|child|\textit{nn}|.tex|:
%
\begin{center}
\begin{tabular}{l}
|\def\version{final}|\\
|\input{childdoc.def}|\\
|\childdocforwardprefix{final}{child}|
\end{tabular}
\end{center}
%

Note that when several versions of a main file and/or of each child file
are to be generated, it may be convenient to set up a |Makefile| or
shell script to automatise the process.

%%%%%%%%%%%%%%%%%%%%%%%%%%%%%%%%%%%%%%%%%%%%%%%%%%%%%%%%%%%%%%%%%%%%%%%%%%%%%%%%
\subsection{Command Line Processing}
\label{sec:commandline}

The effect of redirection files can also be achieved by invoking
the \LaTeX{} compiler with a more elaborate command line.
Most conveniently this should be done as part
of a shell script or a |Makefile|.

When using \textsf{childdoc} in the main file, the following
command lines effectively perform a redirection
(note that depending on the shell being used,
backslashes may have to be doubled: `|\|' $\to$ `|\\|'):
%
\begin{center}
|... -jobname "|\textit{target}|" |\\|"|[\textit{flags}]%
|\input{childdoc.def}\childdocforward[|\textit{main}|]{|\textit{dest}|}"|
\end{center}
%
Here \textit{target} is the name of the output file,
\textit{main} is the name of the main file
and \textit{dest} is the name of the main or child file to be processed
(all filenames without extensions).
The optional argument \textit{main} can be omitted
if \textit{main} matches \textit{dest}.
Optionally, compilation \textit{flags} can be defined via |\def| commands.
This command line makes the \TeX{} engine believe
it is compiling the file \textit{target}
whose content is specified as the latter parameter.
The provided code then forwards the processing to
\textit{main} or \textit{dest} as described in \secref{sec:forward}.

%%%%%%%%%%%%%%%%%%%%%%%%%%%%%%%%%%%%%%%%%%%%%%%%%%%%%%%%%%%%%%%%%%%%%%%%%%%%%%%%
\subsection{Include by Input}
\label{sec:input}

Including child documents by |\include| has some restrictions by design.
Most notably, the content of a child document always occupies
its own set of pages; pages cannot be shared between child documents.
Usually, this behaviour makes perfect sense
because each child document contain an essential part of the document.
However, in some situations it may be desirable to compose
a document from a collection of parts
without having mandatory page breaks between then.
For this case, the package
provides a mechanism to include parts
by |\input| which can also be processed individually.
However, by construction this mechanism
requires manual handling of the content to be output.

%%%%%%%%%%%%%%%%%%%%%%%%%%%%%%%%%%%%%%%%
\DescribeMacro{\ifchilddocmanual}
The main file should be prepared as usual, see \secref{sec:include}.
However, the document body must make a distinction
between processing of an individual part and of the main document, e.g.:
%
\begin{center}
\begin{tabular}{l}
|\ifchilddocmanual|\\
|\input{\childdocname}|\\
|\||else|\\
\textit{document body with }|\input{|\textit{part}|}|\\
|\||fi|
\end{tabular}
\end{center}
%
The conditional |\ifchilddocmanual| is true whenever
a part to be included by |\input| is being compiled,
and the name of the part is stored in |\childdocname|.

%%%%%%%%%%%%%%%%%%%%%%%%%%%%%%%%%%%%%%%%
\DescribeMacro{\childdocby}
Each part to be included by |\input| should start with:
%
\begin{center}
\begin{tabular}{l}
|\input{childdoc.def}|\\
|\childdocby{|\textit{main}|}|\\
\end{tabular}
\end{center}
%
The directive |\childdocby| is similar to |\childdocof|
described in \secref{sec:include},
but the subsequent selection of content must be done manually.
To that end, both |\ifchilddoc| and |\ifchilddocmanual|
will be true upon processing of a part,
and the name of the part is stored in |\childdocname|.
Note that |\jobname| will be set to the filename of the current part
so that each part receives an individual |.aux| file
that does not interfere with the |.aux| file(s) of the main document.
This behaviour can be altered by the alternative form
|\childdocby[*]{|\textit{main}|}| (with a non-empty optional argument)
which uses the |.aux| file of the main document
by setting |\jobname| to \textit{main}.

%%%%%%%%%%%%%%%%%%%%%%%%%%%%%%%%%%%%%%%%%%%%%%%%%%%%%%%%%%%%%%%%%%%%%%%%%%%%%%%%
\subsection{Driver Development}
\label{sec:driver}

The \textsf{childdoc} mechanism can also be use for the development
of definition files such as \LaTeX{} styles or classes.
This case differs from the above setup with multiple parts
included by |\include| in that no |\includeonly| should be invoked.
This can be achieved by starting the include file
(before |\ProvidesPackage|) with:
%
\begin{center}
\begin{tabular}{l}
|\input{childdoc.def}|\\
|\childdocforward{|\textit{main}|}|\\
\end{tabular}
\end{center}
%
or alternatively with:
%
\begin{center}
\begin{tabular}{l}
|\input{childdoc.def}|\\
|\childdocby{|\textit{main}|}|\\
\end{tabular}
\end{center}
%
Both forms have slightly different effects as described above.
The main file is prepared as usual, see \secref{sec:include}.

%%%%%%%%%%%%%%%%%%%%%%%%%%%%%%%%%%%%%%%%%%%%%%%%%%%%%%%%%%%%%%%%%%%%%%%%%%%%%%%%
\subsection{Legacy Detection}
\label{sec:detection}

The directive |\childdocmain| in the main file can detect
whether the complete document or merely a child is to be compiled
even without using the directive |\childdocof|.
This method is deprecated because it is less robust
and there is no compelling reason to use it;
it is merely provided for backward compatibility
and it may be removed in future versions.

If the detection mechanism is to be used,
it is mandatory to correctly specify
the filename of the main file as the argument of |\childdocmain|:
%
\begin{center}
\begin{tabular}{l}
|\input{childdoc.def}|\\
|\childdocmain{|\textit{main}|}|\\
\end{tabular}
\end{center}
%
If |\jobname| does not match the argument \textit{main} of |\childdocmain|,
it is assumed that |\jobname| points to the child file to be compiled.
When using |\childdocmain| with the main file specified as argument,
it suffices to start a child file
with just |\input{|\textit{main}|}|
without loading of the package and using |\childdocof|.
If instead all processing is done
with the appropriate \textsf{childdoc} directives,
the argument of \textit{main} of |\childdocmain| can be empty.

An alternative version of the command line processing described
in \secref{sec:commandline} using the detection mechanism reads:
%
\begin{center}
|... -jobname "|\textit{target}|" "|[\textit{flags}]%
[|\def\jobname{|\textit{dest}|}|]|\input{|\textit{main}|}"|
\end{center}

%%%%%%%%%%%%%%%%%%%%%%%%%%%%%%%%%%%%%%%%%%%%%%%%%%%%%%%%%%%%%%%%%%%%%%%%%%%%%%%%
\subsection{Manual Code}
\label{sec:manual}

In case one cannot be certain whether the definitions file |childdoc.def|
is installed on the target \TeX{} distribution
and one prefers not to ship it,
it is conceivable to paste a few relevant commands into the sources.

To that end, drop all statements |\input{childdoc.def}|
and perform the replacements as outlined below.
Instead of |\childdocmain{|\textit{main}|}| add the following code
to the top of the main file:
%
\begin{center}
\begin{tabular}{l}
|\||ifdefined\childdocname\endinput\||fi\newif\ifchilddoc|\\
|\edef\childdocname{\scantokens\expandafter{\jobname\noexpand}}|\\
|\def\childdocmain{|\textit{main}|}\||ifx\childdocmain\childdocname\||else|\\
|\childdoctrue\includeonly{\childdocname}\let\jobname\childdocmain\||fi|\\
\end{tabular}
\end{center}
%
Instead of |\childdocof{|\textit{main}|}| just include the main file
at the top of each child file:
%
\begin{center}
|\input{|\textit{main}|}|
\end{center}
%
A simple redirection |\childdocforward{|\textit{dest}|}| is achieved by:
%
\begin{center}
|\def\jobname{|\textit{dest}|}\input{\jobname}|
\end{center}
%
The redirection with prefix
|\childdocforwardprefix[|\textit{prefix}|]{|\textit{dest}|}|
is accomplished by:
%
\begin{center}
\begin{tabular}{l}
|{\edef\jobname{\scantokens\expandafter{\jobname\noexpand}}|\\
|\def\redirectjob |\textit{prefix}|#1~~~{\gdef\jobname{|\textit{dest}|#1}}|\\
|\expandafter\redirectjob\jobname~~~}\input{\jobname}|
\end{tabular}
\end{center}

In an alternative approach,
child documents can be compiled by a specific command line
without additional code or specific definitions:
%
\begin{center}
|... -jobname "|\textit{target}|" "|[\textit{flags}]%
|\includeonly{|\textit{dest}|}\input{|\textit{main}|}"|
\end{center}
%

%%%%%%%%%%%%%%%%%%%%%%%%%%%%%%%%%%%%%%%%%%%%%%%%%%%%%%%%%%%%%%%%%%%%%%%%%%%%%%%%
%%%%%%%%%%%%%%%%%%%%%%%%%%%%%%%%%%%%%%%%%%%%%%%%%%%%%%%%%%%%%%%%%%%%%%%%%%%%%%%%
\section{Information}

%%%%%%%%%%%%%%%%%%%%%%%%%%%%%%%%%%%%%%%%%%%%%%%%%%%%%%%%%%%%%%%%%%%%%%%%%%%%%%%%
\subsection{Copyright}

Copyright \copyright{} 2017--2018 Niklas Beisert

This work may be distributed and/or modified under the
conditions of the \LaTeX{} Project Public License, either version 1.3
of this license or (at your option) any later version.
The latest version of this license is in
  \url{http://www.latex-project.org/lppl.txt}
and version 1.3 or later is part of all distributions of \LaTeX{}
version 2005/12/01 or later.

This work has the LPPL maintenance status `maintained'.

The Current Maintainer of this work is Niklas Beisert.

This work consists of the files |README.txt|, |childdoc.ins| and |childdoc.dtx|
as well as the derived files |childdoc.def|, |cdocsamp.tex|
with |cdocsch1.tex|, |cdocsch2.tex|, |cdocspt3.tex|, |cdocspt4.tex|,
|cdocsdrf.tex|, |cdocsfn1.tex|, |cdocsfn2.tex|
as well as |childdoc.pdf|.

%%%%%%%%%%%%%%%%%%%%%%%%%%%%%%%%%%%%%%%%%%%%%%%%%%%%%%%%%%%%%%%%%%%%%%%%%%%%%%%%
\subsection{Files and Installation}

The package consists of the files:
%
\begin{center}
\begin{tabular}{ll}
    |README.txt|   & readme file \\
    |childdoc.ins| & installation file \\
    |childdoc.dtx| & source file \\
    |childdoc.def| & definition file \\
    |cdocsamp.tex| & sample main file \\
    |cdocsch1.tex| & sample include file \\
    |cdocsch2.tex| & sample include file \\
    |cdocspt3.tex| & sample part file \\
    |cdocspt4.tex| & sample part file \\
    |cdocsdrf.tex| & sample redirection file \\
    |cdocsfn1.tex| & sample redirection file \\
    |cdocsfn2.tex| & sample redirection file \\
    |childdoc.pdf| & manual
\end{tabular}
\end{center}
%
The distribution consists of the files
|README.txt|, |childdoc.ins| and |childdoc.dtx|.
%
\begin{itemize}
\item
Run (pdf)\LaTeX{} on |childdoc.dtx|
to compile the manual |childdoc.pdf| (this file).
\item
Run \LaTeX{} on |childdoc.ins| to create the definitions file |childdoc.def|
and the sample |cdocsamp.tex| with include files
|cdocsch1.tex|, |cdocsch2.tex|, |cdocspt3.tex|, |cdocspt4.tex|,
|cdocsdrf.tex|, |cdocsfn1.tex|, |cdocsfn2.tex|.
Then copy the file |childdoc.def| to an appropriate directory of your \LaTeX{}
distribution, e.g.\ \textit{texmf-root}|/tex/latex/childdoc|.
\end{itemize}

%%%%%%%%%%%%%%%%%%%%%%%%%%%%%%%%%%%%%%%%%%%%%%%%%%%%%%%%%%%%%%%%%%%%%%%%%%%%%%%%
\subsection{Related CTAN Packages}

There are several other packages which offer a similar functionality:
%
\begin{itemize}
\item
The packages
\href{http://ctan.org/pkg/docmute}{\textsf{docmute}},
\href{http://ctan.org/pkg/includex}{\textsf{includex}} and
\href{http://ctan.org/pkg/standalone}{\textsf{standalone}}
provide commands to include only the document body of
a child file thus allowing both files to be compiled individually.
\item
The packages \href{http://ctan.org/pkg/subdocs}{\textsf{subdocs}}
and \href{http://ctan.org/pkg/subfiles}{\textsf{subfiles}}
provide structures in which the main and child documents can be
encapsulated and allowing them to be compiled individually.
The inclusion mechanism is different from the conventional |\include|.
\item
The package \href{http://ctan.org/pkg/combine}{\textsf{combine}}
is an elaborate solution to combine several documents into one.
\end{itemize}
%
See also the CTAN topic \href{http://ctan.org/topic/subdocs}{\textsf{subdocs}}
for further related packages.
The present package differs from the above solutions in that
a document structure constructed with the conventional |\include| mechanism
just needs two extra commands at the top of every file
such that all constituent files can be compiled individually.

%%%%%%%%%%%%%%%%%%%%%%%%%%%%%%%%%%%%%%%%%%%%%%%%%%%%%%%%%%%%%%%%%%%%%%%%%%%%%%%%
%\subsection{Feature Suggestions}
%
%The following is a list of features which may be useful for future
%versions of this package:
%%
%\begin{itemize}
%\item
%\ldots
%\end{itemize}

%%%%%%%%%%%%%%%%%%%%%%%%%%%%%%%%%%%%%%%%%%%%%%%%%%%%%%%%%%%%%%%%%%%%%%%%%%%%%%%%
\subsection{Revision History}

%%%%%%%%%%%%%%%%%%%%%%%%%%%%%%%%%%%%%%%%
\paragraph{v2.0:} 2018/12/30

\begin{itemize}
\item
immediate forward processing
\item
added |\childdocby| mechanism
\item
manual restructured
\end{itemize}

%%%%%%%%%%%%%%%%%%%%%%%%%%%%%%%%%%%%%%%%
\paragraph{v1.6:} 2018/01/17

\begin{itemize}
\item
application for development of include files
\item
corrections to manual
\end{itemize}

%%%%%%%%%%%%%%%%%%%%%%%%%%%%%%%%%%%%%%%%
\paragraph{v1.5:} 2017/05/21

\begin{itemize}
\item
more complete structuring introduced
\item
|\childdocof| introduced
\item
|\childdoc| renamed to |\childdocmain|
\item
|\childredirect| renamed to |\childdocforward| and |\childdocforwardprefix|
and functionality expanded
\end{itemize}

%%%%%%%%%%%%%%%%%%%%%%%%%%%%%%%%%%%%%%%%
\paragraph{v1.0:} 2017/04/27

\begin{itemize}
\item
manual and install package
\item
first version published on CTAN
\end{itemize}

%%%%%%%%%%%%%%%%%%%%%%%%%%%%%%%%%%%%%%%%
\paragraph{v0.6:} 2017/04/26

\begin{itemize}
\item
redirection mechanism added
\end{itemize}

%%%%%%%%%%%%%%%%%%%%%%%%%%%%%%%%%%%%%%%%
\paragraph{v0.5:} 2017/04/26

\begin{itemize}
\item
functionality in definition file
\end{itemize}


%%%%%%%%%%%%%%%%%%%%%%%%%%%%%%%%%%%%%%%%%%%%%%%%%%%%%%%%%%%%%%%%%%%%%%%%%%%%%%%%
%%%%%%%%%%%%%%%%%%%%%%%%%%%%%%%%%%%%%%%%%%%%%%%%%%%%%%%%%%%%%%%%%%%%%%%%%%%%%%%%
%%%%%%%%%%%%%%%%%%%%%%%%%%%%%%%%%%%%%%%%%%%%%%%%%%%%%%%%%%%%%%%%%%%%%%%%%%%%%%%%
\appendix

\settowidth\MacroIndent{\rmfamily\scriptsize 000\ }

 \DocInput{childdoc.dtx}

\end{document}
%</driver>
% \fi
%
% %%%%%%%%%%%%%%%%%%%%%%%%%%%%%%%%%%%%%%%%%%%%%%%%%%%%%%%%%%%%%%%%%%%%%%%%%%%%%%
% %%%%%%%%%%%%%%%%%%%%%%%%%%%%%%%%%%%%%%%%%%%%%%%%%%%%%%%%%%%%%%%%%%%%%%%%%%%%%%
% \section{Sample}
%\iffalse
%<*samplemain>
%\fi
%
% The following presents a sample document
% with two chapters, two parts, a title page,
% a compile flag as well as three forwarding files to set the flag.
% It consists of eight |.tex| files:
% \begin{center}
% \begin{tabular}{ll}
% |cdocsamp.tex|&main file\\
% |cdocsch1.tex|&include file for chapter 1\\
% |cdocsch2.tex|&include file for chapter 2\\
% |cdocspt3.tex|&include file for part 3\\
% |cdocspt4.tex|&include file for part 4\\
% |cdocsdrf.tex|&forwarding file for main file in draft mode\\
% |cdocsfi1.tex|&forwarding file for final version of chapter 1\\
% |cdocsfi2.tex|&forwarding file for final version of chapter 2\\
% \end{tabular}
% \end{center}
% Each of the eight files can be compiled directly by the \LaTeX{} compiler.
%
% %%%%%%%%%%%%%%%%%%%%%%%%%%%%%%%%%%%%%%
% \paragraph{Main File.}
%
% The main file is called |cdocsamp.tex|.
%
% Load the \textsf{childdoc} definitions and
% declare the filename for the main document:
%    \begin{macrocode}
\input{childdoc.def}
\childdocmain{}
%    \end{macrocode}

% Optional override for |\version| flag:
%    \begin{macrocode}
%%\ifchilddoc\else\providecommand{\version}{draft}\fi
%    \end{macrocode}

% Define the default values for the |\version| flag
% (|final| for the main file and |draft| for childs):
%    \begin{macrocode}
\ifchilddoc
\providecommand{\version}{draft}
\else
\providecommand{\version}{final}
\fi
%    \end{macrocode}

% Load the standard document class:
%    \begin{macrocode}
\documentclass[12pt]{article}
%    \end{macrocode}

% Start the document body:
%    \begin{macrocode}
\begin{document}
%    \end{macrocode}

% Declare a title page.
% Print title, part of document being processed and version flag:
%    \begin{macrocode}
\addtocounter{page}{-1}
\begin{center}
{\LARGE\bfseries{}childdoc example\par}
\vspace{1cm}
\ifchilddoc
\ifchilddocmanual part\else chapter\fi:
`\childdocname' of `\childdocjob'\par
\else
main document: `\childdocjob'\par
\fi
version: \version\par
\end{center}
\newpage
%    \end{macrocode}

% Manually include selected file,
% otherwise process as usual:
%    \begin{macrocode}
\ifchilddocmanual
\section*{part `\childdocname'}
\input{\childdocname}
\else
%    \end{macrocode}

% Include the two chapters:
%    \begin{macrocode}
\include{cdocsch1}
\include{cdocsch2}
%    \end{macrocode}

% Include the two parts unless only chapters should be displayed:
%    \begin{macrocode}
\ifchilddoc\else
\section{part three}
\input{cdocspt3}
\section{part four}
\input{cdocspt4}
\fi
%    \end{macrocode}

% Process as usual until here:
%    \begin{macrocode}
\fi
%    \end{macrocode}

% End of document body:
%    \begin{macrocode}
\end{document}
%    \end{macrocode}
%\iffalse
%</samplemain>
%\fi
%
% %%%%%%%%%%%%%%%%%%%%%%%%%%%%%%%%%%%%%%
% \paragraph{Chapter Include Files.}
%
% The include files are called |cdocsch1.tex| and |cdocsch2.tex|.
%
%\iffalse
%<*samplechap1|samplechap2>
%\fi

% Optional override for |\version| flag:
%    \begin{macrocode}
%%\providecommand{\version}{final}
%    \end{macrocode}

% Include the main document:
%    \begin{macrocode}
\input{childdoc.def}
\childdocof{cdocsamp}
%    \end{macrocode}

%\iffalse
%</samplechap1|samplechap2>
%\fi
%
%\iffalse
%<*samplechap1>
%\fi
% Some text for chapter 1:
%    \begin{macrocode}
\section{one}
some text in chapter one
%    \end{macrocode}

%\iffalse
%</samplechap1>
%\fi
% Some text for chapter 2:
%\iffalse
%<*samplechap2>
%\fi
%    \begin{macrocode}
\section{two}
more text in chapter two
%    \end{macrocode}

%\iffalse
%</samplechap2>
%\fi
%
% %%%%%%%%%%%%%%%%%%%%%%%%%%%%%%%%%%%%%%
% \paragraph{Part Include Files.}
%
% The include files are called |cdocspt3.tex| and |cdocspt4.tex|.
%
%\iffalse
%<*samplepart3|samplepart4>
%\fi

% Optional override for |\version| flag:
%    \begin{macrocode}
%%\providecommand{\version}{final}
%    \end{macrocode}

% Include the main document:
%    \begin{macrocode}
\input{childdoc.def}
\childdocby{cdocsamp}
%    \end{macrocode}

%\iffalse
%</samplepart3|samplepart4>
%\fi
%
%\iffalse
%<*samplepart3>
%\fi
% Some text for part 3:
%    \begin{macrocode}
some text in part three
%    \end{macrocode}

%\iffalse
%</samplepart3>
%\fi
% Some text for part 4:
%\iffalse
%<*samplepart4>
%\fi
%    \begin{macrocode}
more text in part four
%    \end{macrocode}

%\iffalse
%</samplepart4>
%\fi
%
% %%%%%%%%%%%%%%%%%%%%%%%%%%%%%%%%%%%%%%
% \paragraph{Forwarding for a Complete Draft.}
%
% The following forwarding file |cdocsdrf.tex|
% compiles the main document in draft mode:
%\iffalse
%<*sampledraft>
%\fi
%    \begin{macrocode}
\def\version{draft}
\input{childdoc.def}
\childdocforward{cdocsamp}
%    \end{macrocode}

%\iffalse
%</sampledraft>
%\fi
%
% %%%%%%%%%%%%%%%%%%%%%%%%%%%%%%%%%%%%%%
% \paragraph{Forwarding for Final Version of the Chapters.}
%
% The following forwarding files |cdocsfn1.tex| and |cdocsfn2.tex|
% (with identical content)
% compile the final versions of the child documents
% |cdocsch1.tex| and |cdocsch2.tex|, respectively:
%\iffalse
%<*samplefinal>
%\fi
%    \begin{macrocode}
\def\version{final}
\input{childdoc.def}
\childdocforwardprefix[cdocsamp]{cdocsfn}{cdocsch}
%    \end{macrocode}

%\iffalse
%</samplefinal>
%\fi
%
% %%%%%%%%%%%%%%%%%%%%%%%%%%%%%%%%%%%%%%
% \paragraph{Command Line Processing.}
%
% The following three command lines generate the output files
% |cdocscld|, |cdocscl1| and |cdocscl2|
% which should be identical to
% |cdocsdrf|, |cdocsch1| and |cdocsfn2|, respectively:
% \begin{center}
% \begin{tabular}{l}
% |latex -jobname cdocscld \|\\
% |  "\def\version{draft}\input{childdoc.def}\childdocforward{cdocsamp}"|\\
% |latex -jobname cdocscl1 \|\\
% |  "\input{childdoc.def}\childdocforward[cdocsamp]{cdocsch1}"|\\
% |latex -jobname cdocscl2 \|\\
% |  "\def\version{final}\input{childdoc.def}\childdocforward{cdocsch2}"|
% \end{tabular}
% \end{center}
% Note that the trailing backslash on each first line
% merely continues the input to the second line
% (for convenient cut ant paste).
% Furthermore, the command |latex| can be replaced by any
% of its alternative versions such as |pdflatex|.
%
% %%%%%%%%%%%%%%%%%%%%%%%%%%%%%%%%%%%%%%%%%%%%%%%%%%%%%%%%%%%%%%%%%%%%%%%%%%%%%%
% %%%%%%%%%%%%%%%%%%%%%%%%%%%%%%%%%%%%%%%%%%%%%%%%%%%%%%%%%%%%%%%%%%%%%%%%%%%%%%
% \section{Implementation}
%\iffalse
%<*package>
%\fi
%
% This section describes the definitions file |childdoc.def|.

% The definitions cannot be loaded using |\usepackage| or |\RequirePackage|
% which has a mechanism to prevent loading a style file more than once.
% When loading the definitions by means of |\input|
% multiple instances have to be prevented manually:
%\iffalse
%This code needs to be before the `\ProvidesFile' directive
%which is defined at the beginning of this file.
%Therefore it is also placed there and commented out here.
%</package>
%<*discard>
%\fi
%    \begin{macrocode}
\ifdefined\childdocmain\endinput\fi
%    \end{macrocode}
%\iffalse
%</discard>
%<*package>
%\fi
%
% \macro{\ifchilddoc}
% \macro{\ifchilddocmanual}
% The conditional |\ifchilddoc| tells whether a
% child (true) or main (false) document is being compiled.
% The conditional |\ifchilddocmanual| tells whether
% the |\includeonly| mechanism is used (false) or
% the selection of child files must be performed manually (true).
% The definitions initialise to false:
%    \begin{macrocode}
\newif\ifchilddoc
\newif\ifchilddocmanual
%    \end{macrocode}

% \macro{\childdocname}
% \macro{\childdocjob}
% The macro |\childdocname| stores the name of the main document
% to be compiled. The macro |\childdocjob| stores the name of
% the document on which the \LaTeX{} compiler was originally invoked.
% The content of |\jobname| cannot be compared
% to filenames specified in the source due to different catcodes.
% The following code rescans |\jobname|, stores the result
% in |\childdocname| and saves a copy in |\childdocjob|:
%    \begin{macrocode}
\edef\childdocname{\scantokens\expandafter{\jobname\noexpand}}
\let\childdocjob\childdocname
%    \end{macrocode}

% \macro{\childdocdisable}
% The macro |\childdocdisable| prevents the main file
% from being processed more than once.
% At this stage, the main document command |\childdocmain|
% is assumed to be called once again where it should do nothing.
% Any subsequent call to it should prevent
% a secondary processing of the main document
% It overwrites the forwarding commands
% |\childdocof| and |\childdocforward|
% with empty macros to prevent further inclusions of the main document:
%    \begin{macrocode}
\newcommand{\childdocdisable}
{
  \renewcommand{\childdocmain}[1]{\renewcommand{\childdocmain}[1]{\endinput}}
  \renewcommand{\childdocof}[1]{}
  \renewcommand{\childdocby}[2][]{}
  \renewcommand{\childdocforward}[2][]{}
  \renewcommand{\childdocdisable}{}
}
%    \end{macrocode}

% \macro{\childdocmain}
% The macro |\childdocmain| is to be called at the top of the main file
% with nothing or the main filename (without extension) as argument.
% First, it breaks loops.
% If the argument is not empty and does not match |\childdocname|
% (which is set by the first inclusion of |childdoc.def|),
% |\ifchilddoc| is set to true, |\includeonly| is applied to the child file
% and |\jobname| is set to the main file
% (for proper handling of |.aux| files):
%    \begin{macrocode}
\newcommand{\childdocmain}[1]
{
  \childdocdisable\childdocmain{}
  \if?#1?\else
    \begingroup
      \def\childdoctmp{#1}
      \ifx\childdoctmp\childdocname
        \def\childdoctmp{}
      \else
        \def\childdoctmp
        {
          \childdoctrue
          \includeonly{\childdocname}
          \def\childdocjob{#1}
          \def\jobname{#1}
        }
      \fi
      \expandafter
    \endgroup
    \childdoctmp
  \fi
}
%    \end{macrocode}

% \macro{\childdocof}
% The command |\childdocof| redirects
% compilation to the main file |#1|.
%    \begin{macrocode}
\newcommand{\childdocof}[1]
{
  \childdocdisable
  \childdoctrue
  \includeonly{\childdocname}
  \def\jobname{#1}
  \def\childdocjob{#1}
  \input{#1}
}
%    \end{macrocode}

% \macro{\childdocby}
% The command |\childdocby| ....
%    \begin{macrocode}
\newcommand{\childdocby}[2][]
{
  \childdocdisable
  \childdoctrue
  \childdocmanualtrue
  \if?#1?\else
    \def\jobname{#2}
  \fi
  \def\childdocjob{#2}
  \input{#2}
  \endinput
}
%    \end{macrocode}

% \macro{\childdocforward}
% The command |\childdocforward| redirects
% compilation to the main file or
% (if the optional argument is given) a child file.
% Parameters are set as if the main file
% or a child file starting with |\childdocof| was compiled.
% Then compilation is handed over to the main file:
%    \begin{macrocode}
\newcommand{\childdocforward}[2][]
{
  \begingroup
    \if?#1?
      \def\childdoctmp
      {
        \def\childdocname{#2}
        \def\childdocjob{#2}
        \def\jobname{#2}
        \input{#2}
        \endinput
      }
    \else
      \def\childdoctmp
      {
        \childdocdisable
        \def\childdocname{#2}
        \childdoctrue
        \includeonly{#2}
        \def\childdocjob{#1}
        \def\jobname{#1}
        \input{#1}
        \endinput
      }
    \fi
    \expandafter
  \endgroup
  \childdoctmp
}
%    \end{macrocode}

% \macro{\childdocforwardprefix}
% The command |\childdocforwardprefix| redirects
% compilation to the main or a child file by means of a pattern.
% The prefix |#1| in the current filename is replaced by |#2|
% and the suffix of the current filename is kept
% (it is assumed that the filename does not contain the substring `|~~~|'
% which is used as a delimiter).
% Compilation is handed over to the new file by |\childdocforward|:
%    \begin{macrocode}
\newcommand{\childdocforwardprefix}[3][]
{
  \begingroup
    \def\childdocextract #2##1~~~{\def\childdoctmp{\childdocforward[#1]{#3##1}}}
    \expandafter\childdocextract\childdocname~~~
    \expandafter
  \endgroup
  \childdoctmp
}
%    \end{macrocode}

% \macro{\childdoc}
% The deprecated macro |\childdoc| is a legacy version of |\childdocmain|:
%    \begin{macrocode}
\newcommand{\childdoc}{\childdocmain}
%    \end{macrocode}

% \macro{\childdocredirect}
% The deprecated macro |\childdocredirect| is a legacy version
% of |\childdocforward| and |\childdocforwardprefix|:
%    \begin{macrocode}
\newcommand{\childdocredirect}[2][]
{
  \begingroup
    \if?#1?
      \def\childdoctmp{\childdocforward{#2}}
    \else
      \def\childdoctmp{\childdocforwardprefix{#1}{#2}}
    \fi
    \expandafter
  \endgroup
  \childdoctmp
}
%    \end{macrocode}

%\iffalse
%</package>
%\fi
%
\endinput
|\\
|\childdocforward{|\textit{main}|}|
\end{tabular}
\end{center}
%
Likewise, the following files |final|\textit{nn}|.tex|
compile the final version of the child document
|child|\textit{nn}|.tex|:
%
\begin{center}
\begin{tabular}{l}
|\def\version{final}|\\
|% \iffalse
%
% childdoc.dtx Copyright (C) 2017-2018 Niklas Beisert
%
% This work may be distributed and/or modified under the
% conditions of the LaTeX Project Public License, either version 1.3
% of this license or (at your option) any later version.
% The latest version of this license is in
%   http://www.latex-project.org/lppl.txt
% and version 1.3 or later is part of all distributions of LaTeX
% version 2005/12/01 or later.
%
% This work has the LPPL maintenance status `maintained'.
%
% The Current Maintainer of this work is Niklas Beisert.
%
% This work consists of the files childdoc.dtx and childdoc.ins
% and the derived files childdoc.def and cdocsamp.tex with
% cdocsch1.tex, cdocsch2.tex, cdocsdrf.tex, cdocsfn1.tex, cdocsfn2.tex.
%
%<package>\ifdefined\childdocmain\endinput\fi
%<package>\ProvidesFile{childdoc.def}[2018/12/30 v2.0 child document driver]
%<samplemain>\ProvidesFile{cdocsamp.tex}[2018/12/30 v2.0 sample for childdoc]
%<*driver>
%\ProvidesFile{childdoc.drv}[2018/12/30 v2.0 childdoc reference manual file]
\PassOptionsToClass{10pt,a4paper}{article}
\documentclass{ltxdoc}

\usepackage[margin=35mm]{geometry}
\usepackage{hyperref}
\usepackage{hyperxmp}
\usepackage[usenames]{color}

\hypersetup{colorlinks=true}
\hypersetup{pdfstartview=FitH}
\hypersetup{pdfpagemode=UseNone}
\hypersetup{pdfsource={}}
\hypersetup{pdflang={en-UK}}
\hypersetup{pdfcopyright={Copyright 2017-2018 Niklas Beisert.
  This work may be distributed and/or modified under the
  conditions of the LaTeX Project Public License, either version 1.3
  of this license or (at your option) any later version.}}
\hypersetup{pdflicenseurl={http://www.latex-project.org/lppl.txt}}
\hypersetup{pdfcontactaddress={ETH Zurich, ITP, HIT K,
  Wolfgang-Pauli-Strasse 27}}
\hypersetup{pdfcontactpostcode={8093}}
\hypersetup{pdfcontactcity={Zurich}}
\hypersetup{pdfcontactcountry={Switzerland}}
\hypersetup{pdfcontactemail={nbeisert@itp.phys.ethz.ch}}
\hypersetup{pdfcontacturl={http://people.phys.ethz.ch/\xmptilde nbeisert/}}

\newcommand{\secref}[1]{\hyperref[#1]{section \ref*{#1}}}

\parskip1ex
\parindent0pt
\let\olditemize\itemize
\def\itemize{\olditemize\parskip0pt}

\begin{document}

\title{The \textsf{childdoc} Package}
\hypersetup{pdftitle={The childdoc Package}}
\author{Niklas Beisert\\[2ex]
  Institut f\"ur Theoretische Physik\\
  Eidgen\"ossische Technische Hochschule Z\"urich\\
  Wolfgang-Pauli-Strasse 27, 8093 Z\"urich, Switzerland\\[1ex]
  \href{mailto:nbeisert@itp.phys.ethz.ch}
  {\texttt{nbeisert@itp.phys.ethz.ch}}}
\hypersetup{pdfauthor={Niklas Beisert}}
\hypersetup{pdfsubject={Manual for the LaTeX2e Package childdoc}}
\date{30 December 2018, \textsf{v2.0}}
\maketitle

\begin{abstract}\noindent
\textsf{childdoc} is a \LaTeXe{} package
that enables the direct compilation
of document sections included by |\include|
to individual files.
\end{abstract}

\begingroup
\parskip0ex
\tableofcontents
\endgroup

%%%%%%%%%%%%%%%%%%%%%%%%%%%%%%%%%%%%%%%%%%%%%%%%%%%%%%%%%%%%%%%%%%%%%%%%%%%%%%%%
%%%%%%%%%%%%%%%%%%%%%%%%%%%%%%%%%%%%%%%%%%%%%%%%%%%%%%%%%%%%%%%%%%%%%%%%%%%%%%%%
\section{Introduction}

\LaTeX{} provides a mechanism to structure a large document (such as a book)
into a main file and several child files (containing the chapters)
using the |\include| command.
This mechanism is beneficial for documents
which span hundreds of pages in order to
make the source file(s) more manageable.
Moreover, compilation can be restricted to
selected child files by means of the |\includeonly| command.
The latter feature can be used to reduce the compilation time while editing
(this was significantly more useful in the earlier days of \LaTeX{})
or to generate a smaller document which is easier to navigate.
Another application of |\includeonly| is to generate
documents consisting of selected parts of the complete document.

However, there are a few drawbacks of the plain |\include| mechanism:
\begin{itemize}
\item
The child files cannot be compiled on their own,
they can only be compiled via the main file.
A naive editing environment
(such as a text editor with an option
to have the current file processed by \LaTeX)
may require one to switch to the main file before compiling;
attempting to compile the child file produces errors.
\item
The main file must be modified (each time)
to adjust the |\includeonly| command
to the present needs. This easily leaves the main file in a messy state.
\item
The generated document will always carry the filename
of the main document. This is inconvenient if
several child files are to be compiled and
to be kept for distribution.
\end{itemize}

The present package provides a simple interface
to make child files individually compilable by \LaTeX{}.
Compiling a child file then has the same effect as compiling
the main file with an |\includeonly| command
to select the appropriate child.
Moreover the generated document will carry the name of the child
rather than the main file.
This resolves all three above issues.

This feature is meant to make the editing of books,
thesis documents and lecture notes somewhat more convenient.
However, the package can also be used efficiently for
composing a series of documents (such as exercise sheets)
which are typically distributed individually.
It then assists the author in generating the individual documents
(potentially in different versions)
as well as a document containing the collected series.
Another application is in developing style files
or other kinds of included material
where compilation of the style file could redirect
to a sample or test file.

%%%%%%%%%%%%%%%%%%%%%%%%%%%%%%%%%%%%%%%%%%%%%%%%%%%%%%%%%%%%%%%%%%%%%%%%%%%%%%%%
%%%%%%%%%%%%%%%%%%%%%%%%%%%%%%%%%%%%%%%%%%%%%%%%%%%%%%%%%%%%%%%%%%%%%%%%%%%%%%%%
\section{Usage}

First of all, the package \textsf{childdoc} is \emph{not} a standard
\LaTeXe{} |.sty| style file! Therefore it needs to be invoked in
a non-standard way.

%%%%%%%%%%%%%%%%%%%%%%%%%%%%%%%%%%%%%%%%%%%%%%%%%%%%%%%%%%%%%%%%%%%%%%%%%%%%%%%%
\subsection{Included Files}
\label{sec:include}

%%%%%%%%%%%%%%%%%%%%%%%%%%%%%%%%%%%%%%%%
\DescribeMacro{\childdocmain}
To use the package, add the commands
\begin{center}
\begin{tabular}{l}
|\input{childdoc.def}|\\
|\childdocmain{}|\\
\end{tabular}
\end{center}
at the very top of the main \LaTeX{} file,
in particular \emph{before} the |\documentclass| statement!
The argument of |\childdocmain| should be left empty
(but it must be present).

%%%%%%%%%%%%%%%%%%%%%%%%%%%%%%%%%%%%%%%%
\DescribeMacro{\childdocof}
Furthermore, add the commands
\begin{center}
\begin{tabular}{l}
|\input{childdoc.def}|\\
|\childdocof{|\textit{main}|}|\\
\end{tabular}
\end{center}
at the top of every child file \textit{child}
which is included by |\include{|\textit{child}|}|
from within the main file
(or at least for those files to be compiled individually).
The argument \textit{main} must be the filename of the main file.

There are a couple of
considerations in setting up the main and child documents:

%%%%%%%%%%%%%%%%%%%%%%%%%%%%%%%%%%%%%%%%
\paragraph{Restrictions.}

Please note the following restrictions:
\begin{itemize}
\item
|\childdocmain| must be called with one argument \textit{main}
to ensure compatibility with earlier version of the package.
It must either be empty (|\childdocmain{}|)
or precisely match the filename of the main file in which it is specified.
See \secref{sec:detection} for further information.
\item
The filename \textit{main} must be specified without the |.tex| extension.
\item
The filename \textit{main} is case sensitive
(even in case-insensitive file systems)
due to internal string comparison.
\item
The argument \textit{main} should be fully expanded, it cannot be a macro.
\item
Subdirectories and special characters should be avoided in filenames.
\item
The command |\childdocmain{|\textit{main}|}| must be followed by a whitespace.
It should not be followed immediately by another command
or by a comment mark `|%|'.
This is because the \TeX{} parser reads the token immediately following
the argument of |\childdocmain| and puts it
at the beginning of every child section;
however, a white\-space is ignored.
\end{itemize}

%%%%%%%%%%%%%%%%%%%%%%%%%%%%%%%%%%%%%%%%
\paragraph{Content of Main File.}

It is advisable to place all content in the child files included by |\include|.
Any output contained in the main file will appear in all child documents
unless suppressed manually;
it cannot be suppressed automatically by the |\includeonly| directive
and thus should normally be avoided.
A method to include some content in the main file
by means of conditional processing is described in \secref{sec:conditional}.

%%%%%%%%%%%%%%%%%%%%%%%%%%%%%%%%%%%%%%%%
\paragraph{Page Numbering.}

When only a part of the document is compiled,
the appropriate numbering of pages
(as well as other status parameters)
is determined from the |.aux| files.
The latter contain information from previous passes.
However this information needs to propagate through
all intermediate child documents.
Therefore the page numbering in child documents may well
be inconsistent until the complete document is compiled at least once.

A useful (if unconventional) way to always ensure a consistent
page numbering is to restart the numbering in each child document
and denote the pages by `\textit{child}|.|\textit{page}'
where \textit{child} represents the chapter/section number of the child file.
This can be achieved by the command
|\numberwithin{page}{|\textit{child}|}|
of the \textsf{amsmath} package
where \textit{child} can be |chapter| or |section|
depending on the chosen structuring.
Alternatively, one can modify the macro |\thepage| appropriately
and reset the counter |page| at the start of each child file.

%%%%%%%%%%%%%%%%%%%%%%%%%%%%%%%%%%%%%%%%%%%%%%%%%%%%%%%%%%%%%%%%%%%%%%%%%%%%%%%%
\subsection{Conditional Processing}
\label{sec:conditional}

The package provides a mechanism to compile different versions
of a document. To customise the versions further some conditional processing
can come in handy to distinguish which version is being compiled.
The package provides two macros to describe the compilation context:

%%%%%%%%%%%%%%%%%%%%%%%%%%%%%%%%%%%%%%%%
\DescribeMacro{\ifchilddoc}
The conditional |\ifchilddoc| distinguishes between the compilation of
child documents and the main document:
%
\begin{center}
|\ifchilddoc |\textit{child-code}| |[|\||else |\textit{main-code}]| \||fi|
\end{center}

%%%%%%%%%%%%%%%%%%%%%%%%%%%%%%%%%%%%%%%%
\DescribeMacro{\childdocname}
\DescribeMacro{\childdocjob}
The macro |\childdocname| contains the filename (without extension)
of the main or child file being processed.
Note that |\childdocjob| will always contain the name of the main file.

%%%%%%%%%%%%%%%%%%%%%%%%%%%%%%%%%%%%%%%%
\paragraph{Title Page.}

Conditional processing can be used to include a title or banner page
in the main document when proper precautions are taken.
Importantly, the code in the main file should ensure that the page counter
(as well as other status parameters which are stored in the |.aux| files)
takes the same value after the conditional processing.
Otherwise the page numbers may take divergent values
depending on which part is compiled.

For example, a title page could be declared by:
%
\begin{center}
\begin{tabular}{l}
|\ifchilddoc\||else|\\
|\addtocounter{page}{-1}|\\
\textit{code for title page}\\
|\newpage|\\
|\||fi|
\end{tabular}
\end{center}
%
A banner page for the child documents can be generated by:
%
\begin{center}
\begin{tabular}{l}
|\ifchilddoc|\\
|\addtocounter{page}{-1}|\\
\textit{code for banner page}\\
|\newpage|\\
|\||fi|
\end{tabular}
\end{center}
%
Here one could write a message such as:
\begin{center}
|This is the part \childdocname{} of \childdocjob{}.|
\end{center}

%%%%%%%%%%%%%%%%%%%%%%%%%%%%%%%%%%%%%%%%%%%%%%%%%%%%%%%%%%%%%%%%%%%%%%%%%%%%%%%%
\subsection{Flags}
\label{sec:flags}

The package makes it easy to generate different versions
of the main or child documents.
To this end compilation flags can be defined
and assigned different default values.
They will be particularly useful in conjunction
with the forwarding mechanism described in \secref{sec:forward}.

For example, it may be useful to have a flag |\version|
which can be set to |draft| or |final|.
The document source will contain some conditional code
depending on the value of |\version|.
Suppose further, the flag should default to |final| for the main file
and to |draft| for child files
which is a natural assignment for editing the document.
This is achieved by placing the following code
in the preamble of the main document
(below the |\childdocmain| directive):
%
\begin{center}
\begin{tabular}{l}
|\ifchilddoc|\\
|\providecommand{\version}{draft}|\\
|\||else|\\
|\providecommand{\version}{final}|\\
|\||fi|
\end{tabular}
\end{center}
%
The definition by |\providecommand| makes sure
that previous definitions are not overwritten.
Further statements |\providecommand{\version}{...}|
can thus be added before the above code to override it.

For the main file, one might add a line
(between |\childdocmain| and the above block)
%
\begin{center}
|%\ifchilddoc\||else\providecommand{\version}{draft}\||fi|
\end{center}
%
which can be uncommented to produce a draft version.
Likewise one can add a line to the very top of a child file
(above the |\childdocof{|\textit{main}|}| directive)
%
\begin{center}
|%\providecommand{\version}{final}|
\end{center}
%
which can be uncommented to produce the final version of this child document.

%%%%%%%%%%%%%%%%%%%%%%%%%%%%%%%%%%%%%%%%%%%%%%%%%%%%%%%%%%%%%%%%%%%%%%%%%%%%%%%%
\subsection{Forwarding}
\label{sec:forward}

Different versions of the main or child documents
using compilation flags as described in \secref{sec:flags}
can be (permanently) stored in different files
for convenient compilation, viewing and distribution.
To this end, the package defines a command
to pass on compilation to a different file:

%%%%%%%%%%%%%%%%%%%%%%%%%%%%%%%%%%%%%%%%
\DescribeMacro{\childdocforward}
The command |\childdocforward| redirects processing to
another source file:
%
\begin{center}
\begin{tabular}{l}
|\input{childdoc.def}|\\
|\childdocforward[|\textit{main}|]{|\textit{dest}|}|\\
\end{tabular}
\end{center}
%
The argument \textit{dest} is the destination file
(without extension).
It should be the main file or one of the child files.
Note that further \textsf{childdoc} directives
such as |\childdocof| and |\childdocforward|
in the indicated file will be processed in this form.
The optional argument \textit{main}
passes on directly to the main file \textit{main}
while pretending to compile the child \textit{dest}.
This form behaves as if \textit{dest}
issues |\childdocof{|\textit{main}|}| right away,
and no further \textsf{childdoc} directives will be processed.

%%%%%%%%%%%%%%%%%%%%%%%%%%%%%%%%%%%%%%%%
\DescribeMacro{\...prefix}
In the alternative form |\childdocforwardprefix|,
%
\begin{center}
\begin{tabular}{l}
|\input{childdoc.def}|\\
|\childdocforwardprefix[|\textit{main}|]{|\textit{prefix}|}{|\textit{dest}|}|
\end{tabular}
\end{center}
%
the destination file is determined by a pattern
depending on the current file:
To make this work, the current file must be called
`{\textit{prefix}\hspace{0.2em}\textit{suffix}}'
with \textit{prefix} matching precisely the argument.
Processing is then passed on to the file
`{\textit{dest}\hspace{0.2em}\textit{suffix}}'.
Surely, the same effect is achieved by
directly specifying the
argument `{\textit{dest}\hspace{0.2em}\textit{suffix}}'
in the first form.
However, that requires to set up a different file
for each child. With the alternative form of the command
all these files can have exactly the same content
which simplifies setting them up and maintaining them.

For example, the following file |draft.tex|
with a compilation flag |\version| as described in \secref{sec:flags}
compiles the main document as a draft:
%
\begin{center}
\begin{tabular}{l}
|\def\version{draft}|\\
|\input{childdoc.def}|\\
|\childdocforward{|\textit{main}|}|
\end{tabular}
\end{center}
%
Likewise, the following files |final|\textit{nn}|.tex|
compile the final version of the child document
|child|\textit{nn}|.tex|:
%
\begin{center}
\begin{tabular}{l}
|\def\version{final}|\\
|\input{childdoc.def}|\\
|\childdocforwardprefix{final}{child}|
\end{tabular}
\end{center}
%

Note that when several versions of a main file and/or of each child file
are to be generated, it may be convenient to set up a |Makefile| or
shell script to automatise the process.

%%%%%%%%%%%%%%%%%%%%%%%%%%%%%%%%%%%%%%%%%%%%%%%%%%%%%%%%%%%%%%%%%%%%%%%%%%%%%%%%
\subsection{Command Line Processing}
\label{sec:commandline}

The effect of redirection files can also be achieved by invoking
the \LaTeX{} compiler with a more elaborate command line.
Most conveniently this should be done as part
of a shell script or a |Makefile|.

When using \textsf{childdoc} in the main file, the following
command lines effectively perform a redirection
(note that depending on the shell being used,
backslashes may have to be doubled: `|\|' $\to$ `|\\|'):
%
\begin{center}
|... -jobname "|\textit{target}|" |\\|"|[\textit{flags}]%
|\input{childdoc.def}\childdocforward[|\textit{main}|]{|\textit{dest}|}"|
\end{center}
%
Here \textit{target} is the name of the output file,
\textit{main} is the name of the main file
and \textit{dest} is the name of the main or child file to be processed
(all filenames without extensions).
The optional argument \textit{main} can be omitted
if \textit{main} matches \textit{dest}.
Optionally, compilation \textit{flags} can be defined via |\def| commands.
This command line makes the \TeX{} engine believe
it is compiling the file \textit{target}
whose content is specified as the latter parameter.
The provided code then forwards the processing to
\textit{main} or \textit{dest} as described in \secref{sec:forward}.

%%%%%%%%%%%%%%%%%%%%%%%%%%%%%%%%%%%%%%%%%%%%%%%%%%%%%%%%%%%%%%%%%%%%%%%%%%%%%%%%
\subsection{Include by Input}
\label{sec:input}

Including child documents by |\include| has some restrictions by design.
Most notably, the content of a child document always occupies
its own set of pages; pages cannot be shared between child documents.
Usually, this behaviour makes perfect sense
because each child document contain an essential part of the document.
However, in some situations it may be desirable to compose
a document from a collection of parts
without having mandatory page breaks between then.
For this case, the package
provides a mechanism to include parts
by |\input| which can also be processed individually.
However, by construction this mechanism
requires manual handling of the content to be output.

%%%%%%%%%%%%%%%%%%%%%%%%%%%%%%%%%%%%%%%%
\DescribeMacro{\ifchilddocmanual}
The main file should be prepared as usual, see \secref{sec:include}.
However, the document body must make a distinction
between processing of an individual part and of the main document, e.g.:
%
\begin{center}
\begin{tabular}{l}
|\ifchilddocmanual|\\
|\input{\childdocname}|\\
|\||else|\\
\textit{document body with }|\input{|\textit{part}|}|\\
|\||fi|
\end{tabular}
\end{center}
%
The conditional |\ifchilddocmanual| is true whenever
a part to be included by |\input| is being compiled,
and the name of the part is stored in |\childdocname|.

%%%%%%%%%%%%%%%%%%%%%%%%%%%%%%%%%%%%%%%%
\DescribeMacro{\childdocby}
Each part to be included by |\input| should start with:
%
\begin{center}
\begin{tabular}{l}
|\input{childdoc.def}|\\
|\childdocby{|\textit{main}|}|\\
\end{tabular}
\end{center}
%
The directive |\childdocby| is similar to |\childdocof|
described in \secref{sec:include},
but the subsequent selection of content must be done manually.
To that end, both |\ifchilddoc| and |\ifchilddocmanual|
will be true upon processing of a part,
and the name of the part is stored in |\childdocname|.
Note that |\jobname| will be set to the filename of the current part
so that each part receives an individual |.aux| file
that does not interfere with the |.aux| file(s) of the main document.
This behaviour can be altered by the alternative form
|\childdocby[*]{|\textit{main}|}| (with a non-empty optional argument)
which uses the |.aux| file of the main document
by setting |\jobname| to \textit{main}.

%%%%%%%%%%%%%%%%%%%%%%%%%%%%%%%%%%%%%%%%%%%%%%%%%%%%%%%%%%%%%%%%%%%%%%%%%%%%%%%%
\subsection{Driver Development}
\label{sec:driver}

The \textsf{childdoc} mechanism can also be use for the development
of definition files such as \LaTeX{} styles or classes.
This case differs from the above setup with multiple parts
included by |\include| in that no |\includeonly| should be invoked.
This can be achieved by starting the include file
(before |\ProvidesPackage|) with:
%
\begin{center}
\begin{tabular}{l}
|\input{childdoc.def}|\\
|\childdocforward{|\textit{main}|}|\\
\end{tabular}
\end{center}
%
or alternatively with:
%
\begin{center}
\begin{tabular}{l}
|\input{childdoc.def}|\\
|\childdocby{|\textit{main}|}|\\
\end{tabular}
\end{center}
%
Both forms have slightly different effects as described above.
The main file is prepared as usual, see \secref{sec:include}.

%%%%%%%%%%%%%%%%%%%%%%%%%%%%%%%%%%%%%%%%%%%%%%%%%%%%%%%%%%%%%%%%%%%%%%%%%%%%%%%%
\subsection{Legacy Detection}
\label{sec:detection}

The directive |\childdocmain| in the main file can detect
whether the complete document or merely a child is to be compiled
even without using the directive |\childdocof|.
This method is deprecated because it is less robust
and there is no compelling reason to use it;
it is merely provided for backward compatibility
and it may be removed in future versions.

If the detection mechanism is to be used,
it is mandatory to correctly specify
the filename of the main file as the argument of |\childdocmain|:
%
\begin{center}
\begin{tabular}{l}
|\input{childdoc.def}|\\
|\childdocmain{|\textit{main}|}|\\
\end{tabular}
\end{center}
%
If |\jobname| does not match the argument \textit{main} of |\childdocmain|,
it is assumed that |\jobname| points to the child file to be compiled.
When using |\childdocmain| with the main file specified as argument,
it suffices to start a child file
with just |\input{|\textit{main}|}|
without loading of the package and using |\childdocof|.
If instead all processing is done
with the appropriate \textsf{childdoc} directives,
the argument of \textit{main} of |\childdocmain| can be empty.

An alternative version of the command line processing described
in \secref{sec:commandline} using the detection mechanism reads:
%
\begin{center}
|... -jobname "|\textit{target}|" "|[\textit{flags}]%
[|\def\jobname{|\textit{dest}|}|]|\input{|\textit{main}|}"|
\end{center}

%%%%%%%%%%%%%%%%%%%%%%%%%%%%%%%%%%%%%%%%%%%%%%%%%%%%%%%%%%%%%%%%%%%%%%%%%%%%%%%%
\subsection{Manual Code}
\label{sec:manual}

In case one cannot be certain whether the definitions file |childdoc.def|
is installed on the target \TeX{} distribution
and one prefers not to ship it,
it is conceivable to paste a few relevant commands into the sources.

To that end, drop all statements |\input{childdoc.def}|
and perform the replacements as outlined below.
Instead of |\childdocmain{|\textit{main}|}| add the following code
to the top of the main file:
%
\begin{center}
\begin{tabular}{l}
|\||ifdefined\childdocname\endinput\||fi\newif\ifchilddoc|\\
|\edef\childdocname{\scantokens\expandafter{\jobname\noexpand}}|\\
|\def\childdocmain{|\textit{main}|}\||ifx\childdocmain\childdocname\||else|\\
|\childdoctrue\includeonly{\childdocname}\let\jobname\childdocmain\||fi|\\
\end{tabular}
\end{center}
%
Instead of |\childdocof{|\textit{main}|}| just include the main file
at the top of each child file:
%
\begin{center}
|\input{|\textit{main}|}|
\end{center}
%
A simple redirection |\childdocforward{|\textit{dest}|}| is achieved by:
%
\begin{center}
|\def\jobname{|\textit{dest}|}\input{\jobname}|
\end{center}
%
The redirection with prefix
|\childdocforwardprefix[|\textit{prefix}|]{|\textit{dest}|}|
is accomplished by:
%
\begin{center}
\begin{tabular}{l}
|{\edef\jobname{\scantokens\expandafter{\jobname\noexpand}}|\\
|\def\redirectjob |\textit{prefix}|#1~~~{\gdef\jobname{|\textit{dest}|#1}}|\\
|\expandafter\redirectjob\jobname~~~}\input{\jobname}|
\end{tabular}
\end{center}

In an alternative approach,
child documents can be compiled by a specific command line
without additional code or specific definitions:
%
\begin{center}
|... -jobname "|\textit{target}|" "|[\textit{flags}]%
|\includeonly{|\textit{dest}|}\input{|\textit{main}|}"|
\end{center}
%

%%%%%%%%%%%%%%%%%%%%%%%%%%%%%%%%%%%%%%%%%%%%%%%%%%%%%%%%%%%%%%%%%%%%%%%%%%%%%%%%
%%%%%%%%%%%%%%%%%%%%%%%%%%%%%%%%%%%%%%%%%%%%%%%%%%%%%%%%%%%%%%%%%%%%%%%%%%%%%%%%
\section{Information}

%%%%%%%%%%%%%%%%%%%%%%%%%%%%%%%%%%%%%%%%%%%%%%%%%%%%%%%%%%%%%%%%%%%%%%%%%%%%%%%%
\subsection{Copyright}

Copyright \copyright{} 2017--2018 Niklas Beisert

This work may be distributed and/or modified under the
conditions of the \LaTeX{} Project Public License, either version 1.3
of this license or (at your option) any later version.
The latest version of this license is in
  \url{http://www.latex-project.org/lppl.txt}
and version 1.3 or later is part of all distributions of \LaTeX{}
version 2005/12/01 or later.

This work has the LPPL maintenance status `maintained'.

The Current Maintainer of this work is Niklas Beisert.

This work consists of the files |README.txt|, |childdoc.ins| and |childdoc.dtx|
as well as the derived files |childdoc.def|, |cdocsamp.tex|
with |cdocsch1.tex|, |cdocsch2.tex|, |cdocspt3.tex|, |cdocspt4.tex|,
|cdocsdrf.tex|, |cdocsfn1.tex|, |cdocsfn2.tex|
as well as |childdoc.pdf|.

%%%%%%%%%%%%%%%%%%%%%%%%%%%%%%%%%%%%%%%%%%%%%%%%%%%%%%%%%%%%%%%%%%%%%%%%%%%%%%%%
\subsection{Files and Installation}

The package consists of the files:
%
\begin{center}
\begin{tabular}{ll}
    |README.txt|   & readme file \\
    |childdoc.ins| & installation file \\
    |childdoc.dtx| & source file \\
    |childdoc.def| & definition file \\
    |cdocsamp.tex| & sample main file \\
    |cdocsch1.tex| & sample include file \\
    |cdocsch2.tex| & sample include file \\
    |cdocspt3.tex| & sample part file \\
    |cdocspt4.tex| & sample part file \\
    |cdocsdrf.tex| & sample redirection file \\
    |cdocsfn1.tex| & sample redirection file \\
    |cdocsfn2.tex| & sample redirection file \\
    |childdoc.pdf| & manual
\end{tabular}
\end{center}
%
The distribution consists of the files
|README.txt|, |childdoc.ins| and |childdoc.dtx|.
%
\begin{itemize}
\item
Run (pdf)\LaTeX{} on |childdoc.dtx|
to compile the manual |childdoc.pdf| (this file).
\item
Run \LaTeX{} on |childdoc.ins| to create the definitions file |childdoc.def|
and the sample |cdocsamp.tex| with include files
|cdocsch1.tex|, |cdocsch2.tex|, |cdocspt3.tex|, |cdocspt4.tex|,
|cdocsdrf.tex|, |cdocsfn1.tex|, |cdocsfn2.tex|.
Then copy the file |childdoc.def| to an appropriate directory of your \LaTeX{}
distribution, e.g.\ \textit{texmf-root}|/tex/latex/childdoc|.
\end{itemize}

%%%%%%%%%%%%%%%%%%%%%%%%%%%%%%%%%%%%%%%%%%%%%%%%%%%%%%%%%%%%%%%%%%%%%%%%%%%%%%%%
\subsection{Related CTAN Packages}

There are several other packages which offer a similar functionality:
%
\begin{itemize}
\item
The packages
\href{http://ctan.org/pkg/docmute}{\textsf{docmute}},
\href{http://ctan.org/pkg/includex}{\textsf{includex}} and
\href{http://ctan.org/pkg/standalone}{\textsf{standalone}}
provide commands to include only the document body of
a child file thus allowing both files to be compiled individually.
\item
The packages \href{http://ctan.org/pkg/subdocs}{\textsf{subdocs}}
and \href{http://ctan.org/pkg/subfiles}{\textsf{subfiles}}
provide structures in which the main and child documents can be
encapsulated and allowing them to be compiled individually.
The inclusion mechanism is different from the conventional |\include|.
\item
The package \href{http://ctan.org/pkg/combine}{\textsf{combine}}
is an elaborate solution to combine several documents into one.
\end{itemize}
%
See also the CTAN topic \href{http://ctan.org/topic/subdocs}{\textsf{subdocs}}
for further related packages.
The present package differs from the above solutions in that
a document structure constructed with the conventional |\include| mechanism
just needs two extra commands at the top of every file
such that all constituent files can be compiled individually.

%%%%%%%%%%%%%%%%%%%%%%%%%%%%%%%%%%%%%%%%%%%%%%%%%%%%%%%%%%%%%%%%%%%%%%%%%%%%%%%%
%\subsection{Feature Suggestions}
%
%The following is a list of features which may be useful for future
%versions of this package:
%%
%\begin{itemize}
%\item
%\ldots
%\end{itemize}

%%%%%%%%%%%%%%%%%%%%%%%%%%%%%%%%%%%%%%%%%%%%%%%%%%%%%%%%%%%%%%%%%%%%%%%%%%%%%%%%
\subsection{Revision History}

%%%%%%%%%%%%%%%%%%%%%%%%%%%%%%%%%%%%%%%%
\paragraph{v2.0:} 2018/12/30

\begin{itemize}
\item
immediate forward processing
\item
added |\childdocby| mechanism
\item
manual restructured
\end{itemize}

%%%%%%%%%%%%%%%%%%%%%%%%%%%%%%%%%%%%%%%%
\paragraph{v1.6:} 2018/01/17

\begin{itemize}
\item
application for development of include files
\item
corrections to manual
\end{itemize}

%%%%%%%%%%%%%%%%%%%%%%%%%%%%%%%%%%%%%%%%
\paragraph{v1.5:} 2017/05/21

\begin{itemize}
\item
more complete structuring introduced
\item
|\childdocof| introduced
\item
|\childdoc| renamed to |\childdocmain|
\item
|\childredirect| renamed to |\childdocforward| and |\childdocforwardprefix|
and functionality expanded
\end{itemize}

%%%%%%%%%%%%%%%%%%%%%%%%%%%%%%%%%%%%%%%%
\paragraph{v1.0:} 2017/04/27

\begin{itemize}
\item
manual and install package
\item
first version published on CTAN
\end{itemize}

%%%%%%%%%%%%%%%%%%%%%%%%%%%%%%%%%%%%%%%%
\paragraph{v0.6:} 2017/04/26

\begin{itemize}
\item
redirection mechanism added
\end{itemize}

%%%%%%%%%%%%%%%%%%%%%%%%%%%%%%%%%%%%%%%%
\paragraph{v0.5:} 2017/04/26

\begin{itemize}
\item
functionality in definition file
\end{itemize}


%%%%%%%%%%%%%%%%%%%%%%%%%%%%%%%%%%%%%%%%%%%%%%%%%%%%%%%%%%%%%%%%%%%%%%%%%%%%%%%%
%%%%%%%%%%%%%%%%%%%%%%%%%%%%%%%%%%%%%%%%%%%%%%%%%%%%%%%%%%%%%%%%%%%%%%%%%%%%%%%%
%%%%%%%%%%%%%%%%%%%%%%%%%%%%%%%%%%%%%%%%%%%%%%%%%%%%%%%%%%%%%%%%%%%%%%%%%%%%%%%%
\appendix

\settowidth\MacroIndent{\rmfamily\scriptsize 000\ }

 \DocInput{childdoc.dtx}

\end{document}
%</driver>
% \fi
%
% %%%%%%%%%%%%%%%%%%%%%%%%%%%%%%%%%%%%%%%%%%%%%%%%%%%%%%%%%%%%%%%%%%%%%%%%%%%%%%
% %%%%%%%%%%%%%%%%%%%%%%%%%%%%%%%%%%%%%%%%%%%%%%%%%%%%%%%%%%%%%%%%%%%%%%%%%%%%%%
% \section{Sample}
%\iffalse
%<*samplemain>
%\fi
%
% The following presents a sample document
% with two chapters, two parts, a title page,
% a compile flag as well as three forwarding files to set the flag.
% It consists of eight |.tex| files:
% \begin{center}
% \begin{tabular}{ll}
% |cdocsamp.tex|&main file\\
% |cdocsch1.tex|&include file for chapter 1\\
% |cdocsch2.tex|&include file for chapter 2\\
% |cdocspt3.tex|&include file for part 3\\
% |cdocspt4.tex|&include file for part 4\\
% |cdocsdrf.tex|&forwarding file for main file in draft mode\\
% |cdocsfi1.tex|&forwarding file for final version of chapter 1\\
% |cdocsfi2.tex|&forwarding file for final version of chapter 2\\
% \end{tabular}
% \end{center}
% Each of the eight files can be compiled directly by the \LaTeX{} compiler.
%
% %%%%%%%%%%%%%%%%%%%%%%%%%%%%%%%%%%%%%%
% \paragraph{Main File.}
%
% The main file is called |cdocsamp.tex|.
%
% Load the \textsf{childdoc} definitions and
% declare the filename for the main document:
%    \begin{macrocode}
\input{childdoc.def}
\childdocmain{}
%    \end{macrocode}

% Optional override for |\version| flag:
%    \begin{macrocode}
%%\ifchilddoc\else\providecommand{\version}{draft}\fi
%    \end{macrocode}

% Define the default values for the |\version| flag
% (|final| for the main file and |draft| for childs):
%    \begin{macrocode}
\ifchilddoc
\providecommand{\version}{draft}
\else
\providecommand{\version}{final}
\fi
%    \end{macrocode}

% Load the standard document class:
%    \begin{macrocode}
\documentclass[12pt]{article}
%    \end{macrocode}

% Start the document body:
%    \begin{macrocode}
\begin{document}
%    \end{macrocode}

% Declare a title page.
% Print title, part of document being processed and version flag:
%    \begin{macrocode}
\addtocounter{page}{-1}
\begin{center}
{\LARGE\bfseries{}childdoc example\par}
\vspace{1cm}
\ifchilddoc
\ifchilddocmanual part\else chapter\fi:
`\childdocname' of `\childdocjob'\par
\else
main document: `\childdocjob'\par
\fi
version: \version\par
\end{center}
\newpage
%    \end{macrocode}

% Manually include selected file,
% otherwise process as usual:
%    \begin{macrocode}
\ifchilddocmanual
\section*{part `\childdocname'}
\input{\childdocname}
\else
%    \end{macrocode}

% Include the two chapters:
%    \begin{macrocode}
\include{cdocsch1}
\include{cdocsch2}
%    \end{macrocode}

% Include the two parts unless only chapters should be displayed:
%    \begin{macrocode}
\ifchilddoc\else
\section{part three}
\input{cdocspt3}
\section{part four}
\input{cdocspt4}
\fi
%    \end{macrocode}

% Process as usual until here:
%    \begin{macrocode}
\fi
%    \end{macrocode}

% End of document body:
%    \begin{macrocode}
\end{document}
%    \end{macrocode}
%\iffalse
%</samplemain>
%\fi
%
% %%%%%%%%%%%%%%%%%%%%%%%%%%%%%%%%%%%%%%
% \paragraph{Chapter Include Files.}
%
% The include files are called |cdocsch1.tex| and |cdocsch2.tex|.
%
%\iffalse
%<*samplechap1|samplechap2>
%\fi

% Optional override for |\version| flag:
%    \begin{macrocode}
%%\providecommand{\version}{final}
%    \end{macrocode}

% Include the main document:
%    \begin{macrocode}
\input{childdoc.def}
\childdocof{cdocsamp}
%    \end{macrocode}

%\iffalse
%</samplechap1|samplechap2>
%\fi
%
%\iffalse
%<*samplechap1>
%\fi
% Some text for chapter 1:
%    \begin{macrocode}
\section{one}
some text in chapter one
%    \end{macrocode}

%\iffalse
%</samplechap1>
%\fi
% Some text for chapter 2:
%\iffalse
%<*samplechap2>
%\fi
%    \begin{macrocode}
\section{two}
more text in chapter two
%    \end{macrocode}

%\iffalse
%</samplechap2>
%\fi
%
% %%%%%%%%%%%%%%%%%%%%%%%%%%%%%%%%%%%%%%
% \paragraph{Part Include Files.}
%
% The include files are called |cdocspt3.tex| and |cdocspt4.tex|.
%
%\iffalse
%<*samplepart3|samplepart4>
%\fi

% Optional override for |\version| flag:
%    \begin{macrocode}
%%\providecommand{\version}{final}
%    \end{macrocode}

% Include the main document:
%    \begin{macrocode}
\input{childdoc.def}
\childdocby{cdocsamp}
%    \end{macrocode}

%\iffalse
%</samplepart3|samplepart4>
%\fi
%
%\iffalse
%<*samplepart3>
%\fi
% Some text for part 3:
%    \begin{macrocode}
some text in part three
%    \end{macrocode}

%\iffalse
%</samplepart3>
%\fi
% Some text for part 4:
%\iffalse
%<*samplepart4>
%\fi
%    \begin{macrocode}
more text in part four
%    \end{macrocode}

%\iffalse
%</samplepart4>
%\fi
%
% %%%%%%%%%%%%%%%%%%%%%%%%%%%%%%%%%%%%%%
% \paragraph{Forwarding for a Complete Draft.}
%
% The following forwarding file |cdocsdrf.tex|
% compiles the main document in draft mode:
%\iffalse
%<*sampledraft>
%\fi
%    \begin{macrocode}
\def\version{draft}
\input{childdoc.def}
\childdocforward{cdocsamp}
%    \end{macrocode}

%\iffalse
%</sampledraft>
%\fi
%
% %%%%%%%%%%%%%%%%%%%%%%%%%%%%%%%%%%%%%%
% \paragraph{Forwarding for Final Version of the Chapters.}
%
% The following forwarding files |cdocsfn1.tex| and |cdocsfn2.tex|
% (with identical content)
% compile the final versions of the child documents
% |cdocsch1.tex| and |cdocsch2.tex|, respectively:
%\iffalse
%<*samplefinal>
%\fi
%    \begin{macrocode}
\def\version{final}
\input{childdoc.def}
\childdocforwardprefix[cdocsamp]{cdocsfn}{cdocsch}
%    \end{macrocode}

%\iffalse
%</samplefinal>
%\fi
%
% %%%%%%%%%%%%%%%%%%%%%%%%%%%%%%%%%%%%%%
% \paragraph{Command Line Processing.}
%
% The following three command lines generate the output files
% |cdocscld|, |cdocscl1| and |cdocscl2|
% which should be identical to
% |cdocsdrf|, |cdocsch1| and |cdocsfn2|, respectively:
% \begin{center}
% \begin{tabular}{l}
% |latex -jobname cdocscld \|\\
% |  "\def\version{draft}\input{childdoc.def}\childdocforward{cdocsamp}"|\\
% |latex -jobname cdocscl1 \|\\
% |  "\input{childdoc.def}\childdocforward[cdocsamp]{cdocsch1}"|\\
% |latex -jobname cdocscl2 \|\\
% |  "\def\version{final}\input{childdoc.def}\childdocforward{cdocsch2}"|
% \end{tabular}
% \end{center}
% Note that the trailing backslash on each first line
% merely continues the input to the second line
% (for convenient cut ant paste).
% Furthermore, the command |latex| can be replaced by any
% of its alternative versions such as |pdflatex|.
%
% %%%%%%%%%%%%%%%%%%%%%%%%%%%%%%%%%%%%%%%%%%%%%%%%%%%%%%%%%%%%%%%%%%%%%%%%%%%%%%
% %%%%%%%%%%%%%%%%%%%%%%%%%%%%%%%%%%%%%%%%%%%%%%%%%%%%%%%%%%%%%%%%%%%%%%%%%%%%%%
% \section{Implementation}
%\iffalse
%<*package>
%\fi
%
% This section describes the definitions file |childdoc.def|.

% The definitions cannot be loaded using |\usepackage| or |\RequirePackage|
% which has a mechanism to prevent loading a style file more than once.
% When loading the definitions by means of |\input|
% multiple instances have to be prevented manually:
%\iffalse
%This code needs to be before the `\ProvidesFile' directive
%which is defined at the beginning of this file.
%Therefore it is also placed there and commented out here.
%</package>
%<*discard>
%\fi
%    \begin{macrocode}
\ifdefined\childdocmain\endinput\fi
%    \end{macrocode}
%\iffalse
%</discard>
%<*package>
%\fi
%
% \macro{\ifchilddoc}
% \macro{\ifchilddocmanual}
% The conditional |\ifchilddoc| tells whether a
% child (true) or main (false) document is being compiled.
% The conditional |\ifchilddocmanual| tells whether
% the |\includeonly| mechanism is used (false) or
% the selection of child files must be performed manually (true).
% The definitions initialise to false:
%    \begin{macrocode}
\newif\ifchilddoc
\newif\ifchilddocmanual
%    \end{macrocode}

% \macro{\childdocname}
% \macro{\childdocjob}
% The macro |\childdocname| stores the name of the main document
% to be compiled. The macro |\childdocjob| stores the name of
% the document on which the \LaTeX{} compiler was originally invoked.
% The content of |\jobname| cannot be compared
% to filenames specified in the source due to different catcodes.
% The following code rescans |\jobname|, stores the result
% in |\childdocname| and saves a copy in |\childdocjob|:
%    \begin{macrocode}
\edef\childdocname{\scantokens\expandafter{\jobname\noexpand}}
\let\childdocjob\childdocname
%    \end{macrocode}

% \macro{\childdocdisable}
% The macro |\childdocdisable| prevents the main file
% from being processed more than once.
% At this stage, the main document command |\childdocmain|
% is assumed to be called once again where it should do nothing.
% Any subsequent call to it should prevent
% a secondary processing of the main document
% It overwrites the forwarding commands
% |\childdocof| and |\childdocforward|
% with empty macros to prevent further inclusions of the main document:
%    \begin{macrocode}
\newcommand{\childdocdisable}
{
  \renewcommand{\childdocmain}[1]{\renewcommand{\childdocmain}[1]{\endinput}}
  \renewcommand{\childdocof}[1]{}
  \renewcommand{\childdocby}[2][]{}
  \renewcommand{\childdocforward}[2][]{}
  \renewcommand{\childdocdisable}{}
}
%    \end{macrocode}

% \macro{\childdocmain}
% The macro |\childdocmain| is to be called at the top of the main file
% with nothing or the main filename (without extension) as argument.
% First, it breaks loops.
% If the argument is not empty and does not match |\childdocname|
% (which is set by the first inclusion of |childdoc.def|),
% |\ifchilddoc| is set to true, |\includeonly| is applied to the child file
% and |\jobname| is set to the main file
% (for proper handling of |.aux| files):
%    \begin{macrocode}
\newcommand{\childdocmain}[1]
{
  \childdocdisable\childdocmain{}
  \if?#1?\else
    \begingroup
      \def\childdoctmp{#1}
      \ifx\childdoctmp\childdocname
        \def\childdoctmp{}
      \else
        \def\childdoctmp
        {
          \childdoctrue
          \includeonly{\childdocname}
          \def\childdocjob{#1}
          \def\jobname{#1}
        }
      \fi
      \expandafter
    \endgroup
    \childdoctmp
  \fi
}
%    \end{macrocode}

% \macro{\childdocof}
% The command |\childdocof| redirects
% compilation to the main file |#1|.
%    \begin{macrocode}
\newcommand{\childdocof}[1]
{
  \childdocdisable
  \childdoctrue
  \includeonly{\childdocname}
  \def\jobname{#1}
  \def\childdocjob{#1}
  \input{#1}
}
%    \end{macrocode}

% \macro{\childdocby}
% The command |\childdocby| ....
%    \begin{macrocode}
\newcommand{\childdocby}[2][]
{
  \childdocdisable
  \childdoctrue
  \childdocmanualtrue
  \if?#1?\else
    \def\jobname{#2}
  \fi
  \def\childdocjob{#2}
  \input{#2}
  \endinput
}
%    \end{macrocode}

% \macro{\childdocforward}
% The command |\childdocforward| redirects
% compilation to the main file or
% (if the optional argument is given) a child file.
% Parameters are set as if the main file
% or a child file starting with |\childdocof| was compiled.
% Then compilation is handed over to the main file:
%    \begin{macrocode}
\newcommand{\childdocforward}[2][]
{
  \begingroup
    \if?#1?
      \def\childdoctmp
      {
        \def\childdocname{#2}
        \def\childdocjob{#2}
        \def\jobname{#2}
        \input{#2}
        \endinput
      }
    \else
      \def\childdoctmp
      {
        \childdocdisable
        \def\childdocname{#2}
        \childdoctrue
        \includeonly{#2}
        \def\childdocjob{#1}
        \def\jobname{#1}
        \input{#1}
        \endinput
      }
    \fi
    \expandafter
  \endgroup
  \childdoctmp
}
%    \end{macrocode}

% \macro{\childdocforwardprefix}
% The command |\childdocforwardprefix| redirects
% compilation to the main or a child file by means of a pattern.
% The prefix |#1| in the current filename is replaced by |#2|
% and the suffix of the current filename is kept
% (it is assumed that the filename does not contain the substring `|~~~|'
% which is used as a delimiter).
% Compilation is handed over to the new file by |\childdocforward|:
%    \begin{macrocode}
\newcommand{\childdocforwardprefix}[3][]
{
  \begingroup
    \def\childdocextract #2##1~~~{\def\childdoctmp{\childdocforward[#1]{#3##1}}}
    \expandafter\childdocextract\childdocname~~~
    \expandafter
  \endgroup
  \childdoctmp
}
%    \end{macrocode}

% \macro{\childdoc}
% The deprecated macro |\childdoc| is a legacy version of |\childdocmain|:
%    \begin{macrocode}
\newcommand{\childdoc}{\childdocmain}
%    \end{macrocode}

% \macro{\childdocredirect}
% The deprecated macro |\childdocredirect| is a legacy version
% of |\childdocforward| and |\childdocforwardprefix|:
%    \begin{macrocode}
\newcommand{\childdocredirect}[2][]
{
  \begingroup
    \if?#1?
      \def\childdoctmp{\childdocforward{#2}}
    \else
      \def\childdoctmp{\childdocforwardprefix{#1}{#2}}
    \fi
    \expandafter
  \endgroup
  \childdoctmp
}
%    \end{macrocode}

%\iffalse
%</package>
%\fi
%
\endinput
|\\
|\childdocforwardprefix{final}{child}|
\end{tabular}
\end{center}
%

Note that when several versions of a main file and/or of each child file
are to be generated, it may be convenient to set up a |Makefile| or
shell script to automatise the process.

%%%%%%%%%%%%%%%%%%%%%%%%%%%%%%%%%%%%%%%%%%%%%%%%%%%%%%%%%%%%%%%%%%%%%%%%%%%%%%%%
\subsection{Command Line Processing}
\label{sec:commandline}

The effect of redirection files can also be achieved by invoking
the \LaTeX{} compiler with a more elaborate command line.
Most conveniently this should be done as part
of a shell script or a |Makefile|.

When using \textsf{childdoc} in the main file, the following
command lines effectively perform a redirection
(note that depending on the shell being used,
backslashes may have to be doubled: `|\|' $\to$ `|\\|'):
%
\begin{center}
|... -jobname "|\textit{target}|" |\\|"|[\textit{flags}]%
|% \iffalse
%
% childdoc.dtx Copyright (C) 2017-2018 Niklas Beisert
%
% This work may be distributed and/or modified under the
% conditions of the LaTeX Project Public License, either version 1.3
% of this license or (at your option) any later version.
% The latest version of this license is in
%   http://www.latex-project.org/lppl.txt
% and version 1.3 or later is part of all distributions of LaTeX
% version 2005/12/01 or later.
%
% This work has the LPPL maintenance status `maintained'.
%
% The Current Maintainer of this work is Niklas Beisert.
%
% This work consists of the files childdoc.dtx and childdoc.ins
% and the derived files childdoc.def and cdocsamp.tex with
% cdocsch1.tex, cdocsch2.tex, cdocsdrf.tex, cdocsfn1.tex, cdocsfn2.tex.
%
%<package>\ifdefined\childdocmain\endinput\fi
%<package>\ProvidesFile{childdoc.def}[2018/12/30 v2.0 child document driver]
%<samplemain>\ProvidesFile{cdocsamp.tex}[2018/12/30 v2.0 sample for childdoc]
%<*driver>
%\ProvidesFile{childdoc.drv}[2018/12/30 v2.0 childdoc reference manual file]
\PassOptionsToClass{10pt,a4paper}{article}
\documentclass{ltxdoc}

\usepackage[margin=35mm]{geometry}
\usepackage{hyperref}
\usepackage{hyperxmp}
\usepackage[usenames]{color}

\hypersetup{colorlinks=true}
\hypersetup{pdfstartview=FitH}
\hypersetup{pdfpagemode=UseNone}
\hypersetup{pdfsource={}}
\hypersetup{pdflang={en-UK}}
\hypersetup{pdfcopyright={Copyright 2017-2018 Niklas Beisert.
  This work may be distributed and/or modified under the
  conditions of the LaTeX Project Public License, either version 1.3
  of this license or (at your option) any later version.}}
\hypersetup{pdflicenseurl={http://www.latex-project.org/lppl.txt}}
\hypersetup{pdfcontactaddress={ETH Zurich, ITP, HIT K,
  Wolfgang-Pauli-Strasse 27}}
\hypersetup{pdfcontactpostcode={8093}}
\hypersetup{pdfcontactcity={Zurich}}
\hypersetup{pdfcontactcountry={Switzerland}}
\hypersetup{pdfcontactemail={nbeisert@itp.phys.ethz.ch}}
\hypersetup{pdfcontacturl={http://people.phys.ethz.ch/\xmptilde nbeisert/}}

\newcommand{\secref}[1]{\hyperref[#1]{section \ref*{#1}}}

\parskip1ex
\parindent0pt
\let\olditemize\itemize
\def\itemize{\olditemize\parskip0pt}

\begin{document}

\title{The \textsf{childdoc} Package}
\hypersetup{pdftitle={The childdoc Package}}
\author{Niklas Beisert\\[2ex]
  Institut f\"ur Theoretische Physik\\
  Eidgen\"ossische Technische Hochschule Z\"urich\\
  Wolfgang-Pauli-Strasse 27, 8093 Z\"urich, Switzerland\\[1ex]
  \href{mailto:nbeisert@itp.phys.ethz.ch}
  {\texttt{nbeisert@itp.phys.ethz.ch}}}
\hypersetup{pdfauthor={Niklas Beisert}}
\hypersetup{pdfsubject={Manual for the LaTeX2e Package childdoc}}
\date{30 December 2018, \textsf{v2.0}}
\maketitle

\begin{abstract}\noindent
\textsf{childdoc} is a \LaTeXe{} package
that enables the direct compilation
of document sections included by |\include|
to individual files.
\end{abstract}

\begingroup
\parskip0ex
\tableofcontents
\endgroup

%%%%%%%%%%%%%%%%%%%%%%%%%%%%%%%%%%%%%%%%%%%%%%%%%%%%%%%%%%%%%%%%%%%%%%%%%%%%%%%%
%%%%%%%%%%%%%%%%%%%%%%%%%%%%%%%%%%%%%%%%%%%%%%%%%%%%%%%%%%%%%%%%%%%%%%%%%%%%%%%%
\section{Introduction}

\LaTeX{} provides a mechanism to structure a large document (such as a book)
into a main file and several child files (containing the chapters)
using the |\include| command.
This mechanism is beneficial for documents
which span hundreds of pages in order to
make the source file(s) more manageable.
Moreover, compilation can be restricted to
selected child files by means of the |\includeonly| command.
The latter feature can be used to reduce the compilation time while editing
(this was significantly more useful in the earlier days of \LaTeX{})
or to generate a smaller document which is easier to navigate.
Another application of |\includeonly| is to generate
documents consisting of selected parts of the complete document.

However, there are a few drawbacks of the plain |\include| mechanism:
\begin{itemize}
\item
The child files cannot be compiled on their own,
they can only be compiled via the main file.
A naive editing environment
(such as a text editor with an option
to have the current file processed by \LaTeX)
may require one to switch to the main file before compiling;
attempting to compile the child file produces errors.
\item
The main file must be modified (each time)
to adjust the |\includeonly| command
to the present needs. This easily leaves the main file in a messy state.
\item
The generated document will always carry the filename
of the main document. This is inconvenient if
several child files are to be compiled and
to be kept for distribution.
\end{itemize}

The present package provides a simple interface
to make child files individually compilable by \LaTeX{}.
Compiling a child file then has the same effect as compiling
the main file with an |\includeonly| command
to select the appropriate child.
Moreover the generated document will carry the name of the child
rather than the main file.
This resolves all three above issues.

This feature is meant to make the editing of books,
thesis documents and lecture notes somewhat more convenient.
However, the package can also be used efficiently for
composing a series of documents (such as exercise sheets)
which are typically distributed individually.
It then assists the author in generating the individual documents
(potentially in different versions)
as well as a document containing the collected series.
Another application is in developing style files
or other kinds of included material
where compilation of the style file could redirect
to a sample or test file.

%%%%%%%%%%%%%%%%%%%%%%%%%%%%%%%%%%%%%%%%%%%%%%%%%%%%%%%%%%%%%%%%%%%%%%%%%%%%%%%%
%%%%%%%%%%%%%%%%%%%%%%%%%%%%%%%%%%%%%%%%%%%%%%%%%%%%%%%%%%%%%%%%%%%%%%%%%%%%%%%%
\section{Usage}

First of all, the package \textsf{childdoc} is \emph{not} a standard
\LaTeXe{} |.sty| style file! Therefore it needs to be invoked in
a non-standard way.

%%%%%%%%%%%%%%%%%%%%%%%%%%%%%%%%%%%%%%%%%%%%%%%%%%%%%%%%%%%%%%%%%%%%%%%%%%%%%%%%
\subsection{Included Files}
\label{sec:include}

%%%%%%%%%%%%%%%%%%%%%%%%%%%%%%%%%%%%%%%%
\DescribeMacro{\childdocmain}
To use the package, add the commands
\begin{center}
\begin{tabular}{l}
|\input{childdoc.def}|\\
|\childdocmain{}|\\
\end{tabular}
\end{center}
at the very top of the main \LaTeX{} file,
in particular \emph{before} the |\documentclass| statement!
The argument of |\childdocmain| should be left empty
(but it must be present).

%%%%%%%%%%%%%%%%%%%%%%%%%%%%%%%%%%%%%%%%
\DescribeMacro{\childdocof}
Furthermore, add the commands
\begin{center}
\begin{tabular}{l}
|\input{childdoc.def}|\\
|\childdocof{|\textit{main}|}|\\
\end{tabular}
\end{center}
at the top of every child file \textit{child}
which is included by |\include{|\textit{child}|}|
from within the main file
(or at least for those files to be compiled individually).
The argument \textit{main} must be the filename of the main file.

There are a couple of
considerations in setting up the main and child documents:

%%%%%%%%%%%%%%%%%%%%%%%%%%%%%%%%%%%%%%%%
\paragraph{Restrictions.}

Please note the following restrictions:
\begin{itemize}
\item
|\childdocmain| must be called with one argument \textit{main}
to ensure compatibility with earlier version of the package.
It must either be empty (|\childdocmain{}|)
or precisely match the filename of the main file in which it is specified.
See \secref{sec:detection} for further information.
\item
The filename \textit{main} must be specified without the |.tex| extension.
\item
The filename \textit{main} is case sensitive
(even in case-insensitive file systems)
due to internal string comparison.
\item
The argument \textit{main} should be fully expanded, it cannot be a macro.
\item
Subdirectories and special characters should be avoided in filenames.
\item
The command |\childdocmain{|\textit{main}|}| must be followed by a whitespace.
It should not be followed immediately by another command
or by a comment mark `|%|'.
This is because the \TeX{} parser reads the token immediately following
the argument of |\childdocmain| and puts it
at the beginning of every child section;
however, a white\-space is ignored.
\end{itemize}

%%%%%%%%%%%%%%%%%%%%%%%%%%%%%%%%%%%%%%%%
\paragraph{Content of Main File.}

It is advisable to place all content in the child files included by |\include|.
Any output contained in the main file will appear in all child documents
unless suppressed manually;
it cannot be suppressed automatically by the |\includeonly| directive
and thus should normally be avoided.
A method to include some content in the main file
by means of conditional processing is described in \secref{sec:conditional}.

%%%%%%%%%%%%%%%%%%%%%%%%%%%%%%%%%%%%%%%%
\paragraph{Page Numbering.}

When only a part of the document is compiled,
the appropriate numbering of pages
(as well as other status parameters)
is determined from the |.aux| files.
The latter contain information from previous passes.
However this information needs to propagate through
all intermediate child documents.
Therefore the page numbering in child documents may well
be inconsistent until the complete document is compiled at least once.

A useful (if unconventional) way to always ensure a consistent
page numbering is to restart the numbering in each child document
and denote the pages by `\textit{child}|.|\textit{page}'
where \textit{child} represents the chapter/section number of the child file.
This can be achieved by the command
|\numberwithin{page}{|\textit{child}|}|
of the \textsf{amsmath} package
where \textit{child} can be |chapter| or |section|
depending on the chosen structuring.
Alternatively, one can modify the macro |\thepage| appropriately
and reset the counter |page| at the start of each child file.

%%%%%%%%%%%%%%%%%%%%%%%%%%%%%%%%%%%%%%%%%%%%%%%%%%%%%%%%%%%%%%%%%%%%%%%%%%%%%%%%
\subsection{Conditional Processing}
\label{sec:conditional}

The package provides a mechanism to compile different versions
of a document. To customise the versions further some conditional processing
can come in handy to distinguish which version is being compiled.
The package provides two macros to describe the compilation context:

%%%%%%%%%%%%%%%%%%%%%%%%%%%%%%%%%%%%%%%%
\DescribeMacro{\ifchilddoc}
The conditional |\ifchilddoc| distinguishes between the compilation of
child documents and the main document:
%
\begin{center}
|\ifchilddoc |\textit{child-code}| |[|\||else |\textit{main-code}]| \||fi|
\end{center}

%%%%%%%%%%%%%%%%%%%%%%%%%%%%%%%%%%%%%%%%
\DescribeMacro{\childdocname}
\DescribeMacro{\childdocjob}
The macro |\childdocname| contains the filename (without extension)
of the main or child file being processed.
Note that |\childdocjob| will always contain the name of the main file.

%%%%%%%%%%%%%%%%%%%%%%%%%%%%%%%%%%%%%%%%
\paragraph{Title Page.}

Conditional processing can be used to include a title or banner page
in the main document when proper precautions are taken.
Importantly, the code in the main file should ensure that the page counter
(as well as other status parameters which are stored in the |.aux| files)
takes the same value after the conditional processing.
Otherwise the page numbers may take divergent values
depending on which part is compiled.

For example, a title page could be declared by:
%
\begin{center}
\begin{tabular}{l}
|\ifchilddoc\||else|\\
|\addtocounter{page}{-1}|\\
\textit{code for title page}\\
|\newpage|\\
|\||fi|
\end{tabular}
\end{center}
%
A banner page for the child documents can be generated by:
%
\begin{center}
\begin{tabular}{l}
|\ifchilddoc|\\
|\addtocounter{page}{-1}|\\
\textit{code for banner page}\\
|\newpage|\\
|\||fi|
\end{tabular}
\end{center}
%
Here one could write a message such as:
\begin{center}
|This is the part \childdocname{} of \childdocjob{}.|
\end{center}

%%%%%%%%%%%%%%%%%%%%%%%%%%%%%%%%%%%%%%%%%%%%%%%%%%%%%%%%%%%%%%%%%%%%%%%%%%%%%%%%
\subsection{Flags}
\label{sec:flags}

The package makes it easy to generate different versions
of the main or child documents.
To this end compilation flags can be defined
and assigned different default values.
They will be particularly useful in conjunction
with the forwarding mechanism described in \secref{sec:forward}.

For example, it may be useful to have a flag |\version|
which can be set to |draft| or |final|.
The document source will contain some conditional code
depending on the value of |\version|.
Suppose further, the flag should default to |final| for the main file
and to |draft| for child files
which is a natural assignment for editing the document.
This is achieved by placing the following code
in the preamble of the main document
(below the |\childdocmain| directive):
%
\begin{center}
\begin{tabular}{l}
|\ifchilddoc|\\
|\providecommand{\version}{draft}|\\
|\||else|\\
|\providecommand{\version}{final}|\\
|\||fi|
\end{tabular}
\end{center}
%
The definition by |\providecommand| makes sure
that previous definitions are not overwritten.
Further statements |\providecommand{\version}{...}|
can thus be added before the above code to override it.

For the main file, one might add a line
(between |\childdocmain| and the above block)
%
\begin{center}
|%\ifchilddoc\||else\providecommand{\version}{draft}\||fi|
\end{center}
%
which can be uncommented to produce a draft version.
Likewise one can add a line to the very top of a child file
(above the |\childdocof{|\textit{main}|}| directive)
%
\begin{center}
|%\providecommand{\version}{final}|
\end{center}
%
which can be uncommented to produce the final version of this child document.

%%%%%%%%%%%%%%%%%%%%%%%%%%%%%%%%%%%%%%%%%%%%%%%%%%%%%%%%%%%%%%%%%%%%%%%%%%%%%%%%
\subsection{Forwarding}
\label{sec:forward}

Different versions of the main or child documents
using compilation flags as described in \secref{sec:flags}
can be (permanently) stored in different files
for convenient compilation, viewing and distribution.
To this end, the package defines a command
to pass on compilation to a different file:

%%%%%%%%%%%%%%%%%%%%%%%%%%%%%%%%%%%%%%%%
\DescribeMacro{\childdocforward}
The command |\childdocforward| redirects processing to
another source file:
%
\begin{center}
\begin{tabular}{l}
|\input{childdoc.def}|\\
|\childdocforward[|\textit{main}|]{|\textit{dest}|}|\\
\end{tabular}
\end{center}
%
The argument \textit{dest} is the destination file
(without extension).
It should be the main file or one of the child files.
Note that further \textsf{childdoc} directives
such as |\childdocof| and |\childdocforward|
in the indicated file will be processed in this form.
The optional argument \textit{main}
passes on directly to the main file \textit{main}
while pretending to compile the child \textit{dest}.
This form behaves as if \textit{dest}
issues |\childdocof{|\textit{main}|}| right away,
and no further \textsf{childdoc} directives will be processed.

%%%%%%%%%%%%%%%%%%%%%%%%%%%%%%%%%%%%%%%%
\DescribeMacro{\...prefix}
In the alternative form |\childdocforwardprefix|,
%
\begin{center}
\begin{tabular}{l}
|\input{childdoc.def}|\\
|\childdocforwardprefix[|\textit{main}|]{|\textit{prefix}|}{|\textit{dest}|}|
\end{tabular}
\end{center}
%
the destination file is determined by a pattern
depending on the current file:
To make this work, the current file must be called
`{\textit{prefix}\hspace{0.2em}\textit{suffix}}'
with \textit{prefix} matching precisely the argument.
Processing is then passed on to the file
`{\textit{dest}\hspace{0.2em}\textit{suffix}}'.
Surely, the same effect is achieved by
directly specifying the
argument `{\textit{dest}\hspace{0.2em}\textit{suffix}}'
in the first form.
However, that requires to set up a different file
for each child. With the alternative form of the command
all these files can have exactly the same content
which simplifies setting them up and maintaining them.

For example, the following file |draft.tex|
with a compilation flag |\version| as described in \secref{sec:flags}
compiles the main document as a draft:
%
\begin{center}
\begin{tabular}{l}
|\def\version{draft}|\\
|\input{childdoc.def}|\\
|\childdocforward{|\textit{main}|}|
\end{tabular}
\end{center}
%
Likewise, the following files |final|\textit{nn}|.tex|
compile the final version of the child document
|child|\textit{nn}|.tex|:
%
\begin{center}
\begin{tabular}{l}
|\def\version{final}|\\
|\input{childdoc.def}|\\
|\childdocforwardprefix{final}{child}|
\end{tabular}
\end{center}
%

Note that when several versions of a main file and/or of each child file
are to be generated, it may be convenient to set up a |Makefile| or
shell script to automatise the process.

%%%%%%%%%%%%%%%%%%%%%%%%%%%%%%%%%%%%%%%%%%%%%%%%%%%%%%%%%%%%%%%%%%%%%%%%%%%%%%%%
\subsection{Command Line Processing}
\label{sec:commandline}

The effect of redirection files can also be achieved by invoking
the \LaTeX{} compiler with a more elaborate command line.
Most conveniently this should be done as part
of a shell script or a |Makefile|.

When using \textsf{childdoc} in the main file, the following
command lines effectively perform a redirection
(note that depending on the shell being used,
backslashes may have to be doubled: `|\|' $\to$ `|\\|'):
%
\begin{center}
|... -jobname "|\textit{target}|" |\\|"|[\textit{flags}]%
|\input{childdoc.def}\childdocforward[|\textit{main}|]{|\textit{dest}|}"|
\end{center}
%
Here \textit{target} is the name of the output file,
\textit{main} is the name of the main file
and \textit{dest} is the name of the main or child file to be processed
(all filenames without extensions).
The optional argument \textit{main} can be omitted
if \textit{main} matches \textit{dest}.
Optionally, compilation \textit{flags} can be defined via |\def| commands.
This command line makes the \TeX{} engine believe
it is compiling the file \textit{target}
whose content is specified as the latter parameter.
The provided code then forwards the processing to
\textit{main} or \textit{dest} as described in \secref{sec:forward}.

%%%%%%%%%%%%%%%%%%%%%%%%%%%%%%%%%%%%%%%%%%%%%%%%%%%%%%%%%%%%%%%%%%%%%%%%%%%%%%%%
\subsection{Include by Input}
\label{sec:input}

Including child documents by |\include| has some restrictions by design.
Most notably, the content of a child document always occupies
its own set of pages; pages cannot be shared between child documents.
Usually, this behaviour makes perfect sense
because each child document contain an essential part of the document.
However, in some situations it may be desirable to compose
a document from a collection of parts
without having mandatory page breaks between then.
For this case, the package
provides a mechanism to include parts
by |\input| which can also be processed individually.
However, by construction this mechanism
requires manual handling of the content to be output.

%%%%%%%%%%%%%%%%%%%%%%%%%%%%%%%%%%%%%%%%
\DescribeMacro{\ifchilddocmanual}
The main file should be prepared as usual, see \secref{sec:include}.
However, the document body must make a distinction
between processing of an individual part and of the main document, e.g.:
%
\begin{center}
\begin{tabular}{l}
|\ifchilddocmanual|\\
|\input{\childdocname}|\\
|\||else|\\
\textit{document body with }|\input{|\textit{part}|}|\\
|\||fi|
\end{tabular}
\end{center}
%
The conditional |\ifchilddocmanual| is true whenever
a part to be included by |\input| is being compiled,
and the name of the part is stored in |\childdocname|.

%%%%%%%%%%%%%%%%%%%%%%%%%%%%%%%%%%%%%%%%
\DescribeMacro{\childdocby}
Each part to be included by |\input| should start with:
%
\begin{center}
\begin{tabular}{l}
|\input{childdoc.def}|\\
|\childdocby{|\textit{main}|}|\\
\end{tabular}
\end{center}
%
The directive |\childdocby| is similar to |\childdocof|
described in \secref{sec:include},
but the subsequent selection of content must be done manually.
To that end, both |\ifchilddoc| and |\ifchilddocmanual|
will be true upon processing of a part,
and the name of the part is stored in |\childdocname|.
Note that |\jobname| will be set to the filename of the current part
so that each part receives an individual |.aux| file
that does not interfere with the |.aux| file(s) of the main document.
This behaviour can be altered by the alternative form
|\childdocby[*]{|\textit{main}|}| (with a non-empty optional argument)
which uses the |.aux| file of the main document
by setting |\jobname| to \textit{main}.

%%%%%%%%%%%%%%%%%%%%%%%%%%%%%%%%%%%%%%%%%%%%%%%%%%%%%%%%%%%%%%%%%%%%%%%%%%%%%%%%
\subsection{Driver Development}
\label{sec:driver}

The \textsf{childdoc} mechanism can also be use for the development
of definition files such as \LaTeX{} styles or classes.
This case differs from the above setup with multiple parts
included by |\include| in that no |\includeonly| should be invoked.
This can be achieved by starting the include file
(before |\ProvidesPackage|) with:
%
\begin{center}
\begin{tabular}{l}
|\input{childdoc.def}|\\
|\childdocforward{|\textit{main}|}|\\
\end{tabular}
\end{center}
%
or alternatively with:
%
\begin{center}
\begin{tabular}{l}
|\input{childdoc.def}|\\
|\childdocby{|\textit{main}|}|\\
\end{tabular}
\end{center}
%
Both forms have slightly different effects as described above.
The main file is prepared as usual, see \secref{sec:include}.

%%%%%%%%%%%%%%%%%%%%%%%%%%%%%%%%%%%%%%%%%%%%%%%%%%%%%%%%%%%%%%%%%%%%%%%%%%%%%%%%
\subsection{Legacy Detection}
\label{sec:detection}

The directive |\childdocmain| in the main file can detect
whether the complete document or merely a child is to be compiled
even without using the directive |\childdocof|.
This method is deprecated because it is less robust
and there is no compelling reason to use it;
it is merely provided for backward compatibility
and it may be removed in future versions.

If the detection mechanism is to be used,
it is mandatory to correctly specify
the filename of the main file as the argument of |\childdocmain|:
%
\begin{center}
\begin{tabular}{l}
|\input{childdoc.def}|\\
|\childdocmain{|\textit{main}|}|\\
\end{tabular}
\end{center}
%
If |\jobname| does not match the argument \textit{main} of |\childdocmain|,
it is assumed that |\jobname| points to the child file to be compiled.
When using |\childdocmain| with the main file specified as argument,
it suffices to start a child file
with just |\input{|\textit{main}|}|
without loading of the package and using |\childdocof|.
If instead all processing is done
with the appropriate \textsf{childdoc} directives,
the argument of \textit{main} of |\childdocmain| can be empty.

An alternative version of the command line processing described
in \secref{sec:commandline} using the detection mechanism reads:
%
\begin{center}
|... -jobname "|\textit{target}|" "|[\textit{flags}]%
[|\def\jobname{|\textit{dest}|}|]|\input{|\textit{main}|}"|
\end{center}

%%%%%%%%%%%%%%%%%%%%%%%%%%%%%%%%%%%%%%%%%%%%%%%%%%%%%%%%%%%%%%%%%%%%%%%%%%%%%%%%
\subsection{Manual Code}
\label{sec:manual}

In case one cannot be certain whether the definitions file |childdoc.def|
is installed on the target \TeX{} distribution
and one prefers not to ship it,
it is conceivable to paste a few relevant commands into the sources.

To that end, drop all statements |\input{childdoc.def}|
and perform the replacements as outlined below.
Instead of |\childdocmain{|\textit{main}|}| add the following code
to the top of the main file:
%
\begin{center}
\begin{tabular}{l}
|\||ifdefined\childdocname\endinput\||fi\newif\ifchilddoc|\\
|\edef\childdocname{\scantokens\expandafter{\jobname\noexpand}}|\\
|\def\childdocmain{|\textit{main}|}\||ifx\childdocmain\childdocname\||else|\\
|\childdoctrue\includeonly{\childdocname}\let\jobname\childdocmain\||fi|\\
\end{tabular}
\end{center}
%
Instead of |\childdocof{|\textit{main}|}| just include the main file
at the top of each child file:
%
\begin{center}
|\input{|\textit{main}|}|
\end{center}
%
A simple redirection |\childdocforward{|\textit{dest}|}| is achieved by:
%
\begin{center}
|\def\jobname{|\textit{dest}|}\input{\jobname}|
\end{center}
%
The redirection with prefix
|\childdocforwardprefix[|\textit{prefix}|]{|\textit{dest}|}|
is accomplished by:
%
\begin{center}
\begin{tabular}{l}
|{\edef\jobname{\scantokens\expandafter{\jobname\noexpand}}|\\
|\def\redirectjob |\textit{prefix}|#1~~~{\gdef\jobname{|\textit{dest}|#1}}|\\
|\expandafter\redirectjob\jobname~~~}\input{\jobname}|
\end{tabular}
\end{center}

In an alternative approach,
child documents can be compiled by a specific command line
without additional code or specific definitions:
%
\begin{center}
|... -jobname "|\textit{target}|" "|[\textit{flags}]%
|\includeonly{|\textit{dest}|}\input{|\textit{main}|}"|
\end{center}
%

%%%%%%%%%%%%%%%%%%%%%%%%%%%%%%%%%%%%%%%%%%%%%%%%%%%%%%%%%%%%%%%%%%%%%%%%%%%%%%%%
%%%%%%%%%%%%%%%%%%%%%%%%%%%%%%%%%%%%%%%%%%%%%%%%%%%%%%%%%%%%%%%%%%%%%%%%%%%%%%%%
\section{Information}

%%%%%%%%%%%%%%%%%%%%%%%%%%%%%%%%%%%%%%%%%%%%%%%%%%%%%%%%%%%%%%%%%%%%%%%%%%%%%%%%
\subsection{Copyright}

Copyright \copyright{} 2017--2018 Niklas Beisert

This work may be distributed and/or modified under the
conditions of the \LaTeX{} Project Public License, either version 1.3
of this license or (at your option) any later version.
The latest version of this license is in
  \url{http://www.latex-project.org/lppl.txt}
and version 1.3 or later is part of all distributions of \LaTeX{}
version 2005/12/01 or later.

This work has the LPPL maintenance status `maintained'.

The Current Maintainer of this work is Niklas Beisert.

This work consists of the files |README.txt|, |childdoc.ins| and |childdoc.dtx|
as well as the derived files |childdoc.def|, |cdocsamp.tex|
with |cdocsch1.tex|, |cdocsch2.tex|, |cdocspt3.tex|, |cdocspt4.tex|,
|cdocsdrf.tex|, |cdocsfn1.tex|, |cdocsfn2.tex|
as well as |childdoc.pdf|.

%%%%%%%%%%%%%%%%%%%%%%%%%%%%%%%%%%%%%%%%%%%%%%%%%%%%%%%%%%%%%%%%%%%%%%%%%%%%%%%%
\subsection{Files and Installation}

The package consists of the files:
%
\begin{center}
\begin{tabular}{ll}
    |README.txt|   & readme file \\
    |childdoc.ins| & installation file \\
    |childdoc.dtx| & source file \\
    |childdoc.def| & definition file \\
    |cdocsamp.tex| & sample main file \\
    |cdocsch1.tex| & sample include file \\
    |cdocsch2.tex| & sample include file \\
    |cdocspt3.tex| & sample part file \\
    |cdocspt4.tex| & sample part file \\
    |cdocsdrf.tex| & sample redirection file \\
    |cdocsfn1.tex| & sample redirection file \\
    |cdocsfn2.tex| & sample redirection file \\
    |childdoc.pdf| & manual
\end{tabular}
\end{center}
%
The distribution consists of the files
|README.txt|, |childdoc.ins| and |childdoc.dtx|.
%
\begin{itemize}
\item
Run (pdf)\LaTeX{} on |childdoc.dtx|
to compile the manual |childdoc.pdf| (this file).
\item
Run \LaTeX{} on |childdoc.ins| to create the definitions file |childdoc.def|
and the sample |cdocsamp.tex| with include files
|cdocsch1.tex|, |cdocsch2.tex|, |cdocspt3.tex|, |cdocspt4.tex|,
|cdocsdrf.tex|, |cdocsfn1.tex|, |cdocsfn2.tex|.
Then copy the file |childdoc.def| to an appropriate directory of your \LaTeX{}
distribution, e.g.\ \textit{texmf-root}|/tex/latex/childdoc|.
\end{itemize}

%%%%%%%%%%%%%%%%%%%%%%%%%%%%%%%%%%%%%%%%%%%%%%%%%%%%%%%%%%%%%%%%%%%%%%%%%%%%%%%%
\subsection{Related CTAN Packages}

There are several other packages which offer a similar functionality:
%
\begin{itemize}
\item
The packages
\href{http://ctan.org/pkg/docmute}{\textsf{docmute}},
\href{http://ctan.org/pkg/includex}{\textsf{includex}} and
\href{http://ctan.org/pkg/standalone}{\textsf{standalone}}
provide commands to include only the document body of
a child file thus allowing both files to be compiled individually.
\item
The packages \href{http://ctan.org/pkg/subdocs}{\textsf{subdocs}}
and \href{http://ctan.org/pkg/subfiles}{\textsf{subfiles}}
provide structures in which the main and child documents can be
encapsulated and allowing them to be compiled individually.
The inclusion mechanism is different from the conventional |\include|.
\item
The package \href{http://ctan.org/pkg/combine}{\textsf{combine}}
is an elaborate solution to combine several documents into one.
\end{itemize}
%
See also the CTAN topic \href{http://ctan.org/topic/subdocs}{\textsf{subdocs}}
for further related packages.
The present package differs from the above solutions in that
a document structure constructed with the conventional |\include| mechanism
just needs two extra commands at the top of every file
such that all constituent files can be compiled individually.

%%%%%%%%%%%%%%%%%%%%%%%%%%%%%%%%%%%%%%%%%%%%%%%%%%%%%%%%%%%%%%%%%%%%%%%%%%%%%%%%
%\subsection{Feature Suggestions}
%
%The following is a list of features which may be useful for future
%versions of this package:
%%
%\begin{itemize}
%\item
%\ldots
%\end{itemize}

%%%%%%%%%%%%%%%%%%%%%%%%%%%%%%%%%%%%%%%%%%%%%%%%%%%%%%%%%%%%%%%%%%%%%%%%%%%%%%%%
\subsection{Revision History}

%%%%%%%%%%%%%%%%%%%%%%%%%%%%%%%%%%%%%%%%
\paragraph{v2.0:} 2018/12/30

\begin{itemize}
\item
immediate forward processing
\item
added |\childdocby| mechanism
\item
manual restructured
\end{itemize}

%%%%%%%%%%%%%%%%%%%%%%%%%%%%%%%%%%%%%%%%
\paragraph{v1.6:} 2018/01/17

\begin{itemize}
\item
application for development of include files
\item
corrections to manual
\end{itemize}

%%%%%%%%%%%%%%%%%%%%%%%%%%%%%%%%%%%%%%%%
\paragraph{v1.5:} 2017/05/21

\begin{itemize}
\item
more complete structuring introduced
\item
|\childdocof| introduced
\item
|\childdoc| renamed to |\childdocmain|
\item
|\childredirect| renamed to |\childdocforward| and |\childdocforwardprefix|
and functionality expanded
\end{itemize}

%%%%%%%%%%%%%%%%%%%%%%%%%%%%%%%%%%%%%%%%
\paragraph{v1.0:} 2017/04/27

\begin{itemize}
\item
manual and install package
\item
first version published on CTAN
\end{itemize}

%%%%%%%%%%%%%%%%%%%%%%%%%%%%%%%%%%%%%%%%
\paragraph{v0.6:} 2017/04/26

\begin{itemize}
\item
redirection mechanism added
\end{itemize}

%%%%%%%%%%%%%%%%%%%%%%%%%%%%%%%%%%%%%%%%
\paragraph{v0.5:} 2017/04/26

\begin{itemize}
\item
functionality in definition file
\end{itemize}


%%%%%%%%%%%%%%%%%%%%%%%%%%%%%%%%%%%%%%%%%%%%%%%%%%%%%%%%%%%%%%%%%%%%%%%%%%%%%%%%
%%%%%%%%%%%%%%%%%%%%%%%%%%%%%%%%%%%%%%%%%%%%%%%%%%%%%%%%%%%%%%%%%%%%%%%%%%%%%%%%
%%%%%%%%%%%%%%%%%%%%%%%%%%%%%%%%%%%%%%%%%%%%%%%%%%%%%%%%%%%%%%%%%%%%%%%%%%%%%%%%
\appendix

\settowidth\MacroIndent{\rmfamily\scriptsize 000\ }

 \DocInput{childdoc.dtx}

\end{document}
%</driver>
% \fi
%
% %%%%%%%%%%%%%%%%%%%%%%%%%%%%%%%%%%%%%%%%%%%%%%%%%%%%%%%%%%%%%%%%%%%%%%%%%%%%%%
% %%%%%%%%%%%%%%%%%%%%%%%%%%%%%%%%%%%%%%%%%%%%%%%%%%%%%%%%%%%%%%%%%%%%%%%%%%%%%%
% \section{Sample}
%\iffalse
%<*samplemain>
%\fi
%
% The following presents a sample document
% with two chapters, two parts, a title page,
% a compile flag as well as three forwarding files to set the flag.
% It consists of eight |.tex| files:
% \begin{center}
% \begin{tabular}{ll}
% |cdocsamp.tex|&main file\\
% |cdocsch1.tex|&include file for chapter 1\\
% |cdocsch2.tex|&include file for chapter 2\\
% |cdocspt3.tex|&include file for part 3\\
% |cdocspt4.tex|&include file for part 4\\
% |cdocsdrf.tex|&forwarding file for main file in draft mode\\
% |cdocsfi1.tex|&forwarding file for final version of chapter 1\\
% |cdocsfi2.tex|&forwarding file for final version of chapter 2\\
% \end{tabular}
% \end{center}
% Each of the eight files can be compiled directly by the \LaTeX{} compiler.
%
% %%%%%%%%%%%%%%%%%%%%%%%%%%%%%%%%%%%%%%
% \paragraph{Main File.}
%
% The main file is called |cdocsamp.tex|.
%
% Load the \textsf{childdoc} definitions and
% declare the filename for the main document:
%    \begin{macrocode}
\input{childdoc.def}
\childdocmain{}
%    \end{macrocode}

% Optional override for |\version| flag:
%    \begin{macrocode}
%%\ifchilddoc\else\providecommand{\version}{draft}\fi
%    \end{macrocode}

% Define the default values for the |\version| flag
% (|final| for the main file and |draft| for childs):
%    \begin{macrocode}
\ifchilddoc
\providecommand{\version}{draft}
\else
\providecommand{\version}{final}
\fi
%    \end{macrocode}

% Load the standard document class:
%    \begin{macrocode}
\documentclass[12pt]{article}
%    \end{macrocode}

% Start the document body:
%    \begin{macrocode}
\begin{document}
%    \end{macrocode}

% Declare a title page.
% Print title, part of document being processed and version flag:
%    \begin{macrocode}
\addtocounter{page}{-1}
\begin{center}
{\LARGE\bfseries{}childdoc example\par}
\vspace{1cm}
\ifchilddoc
\ifchilddocmanual part\else chapter\fi:
`\childdocname' of `\childdocjob'\par
\else
main document: `\childdocjob'\par
\fi
version: \version\par
\end{center}
\newpage
%    \end{macrocode}

% Manually include selected file,
% otherwise process as usual:
%    \begin{macrocode}
\ifchilddocmanual
\section*{part `\childdocname'}
\input{\childdocname}
\else
%    \end{macrocode}

% Include the two chapters:
%    \begin{macrocode}
\include{cdocsch1}
\include{cdocsch2}
%    \end{macrocode}

% Include the two parts unless only chapters should be displayed:
%    \begin{macrocode}
\ifchilddoc\else
\section{part three}
\input{cdocspt3}
\section{part four}
\input{cdocspt4}
\fi
%    \end{macrocode}

% Process as usual until here:
%    \begin{macrocode}
\fi
%    \end{macrocode}

% End of document body:
%    \begin{macrocode}
\end{document}
%    \end{macrocode}
%\iffalse
%</samplemain>
%\fi
%
% %%%%%%%%%%%%%%%%%%%%%%%%%%%%%%%%%%%%%%
% \paragraph{Chapter Include Files.}
%
% The include files are called |cdocsch1.tex| and |cdocsch2.tex|.
%
%\iffalse
%<*samplechap1|samplechap2>
%\fi

% Optional override for |\version| flag:
%    \begin{macrocode}
%%\providecommand{\version}{final}
%    \end{macrocode}

% Include the main document:
%    \begin{macrocode}
\input{childdoc.def}
\childdocof{cdocsamp}
%    \end{macrocode}

%\iffalse
%</samplechap1|samplechap2>
%\fi
%
%\iffalse
%<*samplechap1>
%\fi
% Some text for chapter 1:
%    \begin{macrocode}
\section{one}
some text in chapter one
%    \end{macrocode}

%\iffalse
%</samplechap1>
%\fi
% Some text for chapter 2:
%\iffalse
%<*samplechap2>
%\fi
%    \begin{macrocode}
\section{two}
more text in chapter two
%    \end{macrocode}

%\iffalse
%</samplechap2>
%\fi
%
% %%%%%%%%%%%%%%%%%%%%%%%%%%%%%%%%%%%%%%
% \paragraph{Part Include Files.}
%
% The include files are called |cdocspt3.tex| and |cdocspt4.tex|.
%
%\iffalse
%<*samplepart3|samplepart4>
%\fi

% Optional override for |\version| flag:
%    \begin{macrocode}
%%\providecommand{\version}{final}
%    \end{macrocode}

% Include the main document:
%    \begin{macrocode}
\input{childdoc.def}
\childdocby{cdocsamp}
%    \end{macrocode}

%\iffalse
%</samplepart3|samplepart4>
%\fi
%
%\iffalse
%<*samplepart3>
%\fi
% Some text for part 3:
%    \begin{macrocode}
some text in part three
%    \end{macrocode}

%\iffalse
%</samplepart3>
%\fi
% Some text for part 4:
%\iffalse
%<*samplepart4>
%\fi
%    \begin{macrocode}
more text in part four
%    \end{macrocode}

%\iffalse
%</samplepart4>
%\fi
%
% %%%%%%%%%%%%%%%%%%%%%%%%%%%%%%%%%%%%%%
% \paragraph{Forwarding for a Complete Draft.}
%
% The following forwarding file |cdocsdrf.tex|
% compiles the main document in draft mode:
%\iffalse
%<*sampledraft>
%\fi
%    \begin{macrocode}
\def\version{draft}
\input{childdoc.def}
\childdocforward{cdocsamp}
%    \end{macrocode}

%\iffalse
%</sampledraft>
%\fi
%
% %%%%%%%%%%%%%%%%%%%%%%%%%%%%%%%%%%%%%%
% \paragraph{Forwarding for Final Version of the Chapters.}
%
% The following forwarding files |cdocsfn1.tex| and |cdocsfn2.tex|
% (with identical content)
% compile the final versions of the child documents
% |cdocsch1.tex| and |cdocsch2.tex|, respectively:
%\iffalse
%<*samplefinal>
%\fi
%    \begin{macrocode}
\def\version{final}
\input{childdoc.def}
\childdocforwardprefix[cdocsamp]{cdocsfn}{cdocsch}
%    \end{macrocode}

%\iffalse
%</samplefinal>
%\fi
%
% %%%%%%%%%%%%%%%%%%%%%%%%%%%%%%%%%%%%%%
% \paragraph{Command Line Processing.}
%
% The following three command lines generate the output files
% |cdocscld|, |cdocscl1| and |cdocscl2|
% which should be identical to
% |cdocsdrf|, |cdocsch1| and |cdocsfn2|, respectively:
% \begin{center}
% \begin{tabular}{l}
% |latex -jobname cdocscld \|\\
% |  "\def\version{draft}\input{childdoc.def}\childdocforward{cdocsamp}"|\\
% |latex -jobname cdocscl1 \|\\
% |  "\input{childdoc.def}\childdocforward[cdocsamp]{cdocsch1}"|\\
% |latex -jobname cdocscl2 \|\\
% |  "\def\version{final}\input{childdoc.def}\childdocforward{cdocsch2}"|
% \end{tabular}
% \end{center}
% Note that the trailing backslash on each first line
% merely continues the input to the second line
% (for convenient cut ant paste).
% Furthermore, the command |latex| can be replaced by any
% of its alternative versions such as |pdflatex|.
%
% %%%%%%%%%%%%%%%%%%%%%%%%%%%%%%%%%%%%%%%%%%%%%%%%%%%%%%%%%%%%%%%%%%%%%%%%%%%%%%
% %%%%%%%%%%%%%%%%%%%%%%%%%%%%%%%%%%%%%%%%%%%%%%%%%%%%%%%%%%%%%%%%%%%%%%%%%%%%%%
% \section{Implementation}
%\iffalse
%<*package>
%\fi
%
% This section describes the definitions file |childdoc.def|.

% The definitions cannot be loaded using |\usepackage| or |\RequirePackage|
% which has a mechanism to prevent loading a style file more than once.
% When loading the definitions by means of |\input|
% multiple instances have to be prevented manually:
%\iffalse
%This code needs to be before the `\ProvidesFile' directive
%which is defined at the beginning of this file.
%Therefore it is also placed there and commented out here.
%</package>
%<*discard>
%\fi
%    \begin{macrocode}
\ifdefined\childdocmain\endinput\fi
%    \end{macrocode}
%\iffalse
%</discard>
%<*package>
%\fi
%
% \macro{\ifchilddoc}
% \macro{\ifchilddocmanual}
% The conditional |\ifchilddoc| tells whether a
% child (true) or main (false) document is being compiled.
% The conditional |\ifchilddocmanual| tells whether
% the |\includeonly| mechanism is used (false) or
% the selection of child files must be performed manually (true).
% The definitions initialise to false:
%    \begin{macrocode}
\newif\ifchilddoc
\newif\ifchilddocmanual
%    \end{macrocode}

% \macro{\childdocname}
% \macro{\childdocjob}
% The macro |\childdocname| stores the name of the main document
% to be compiled. The macro |\childdocjob| stores the name of
% the document on which the \LaTeX{} compiler was originally invoked.
% The content of |\jobname| cannot be compared
% to filenames specified in the source due to different catcodes.
% The following code rescans |\jobname|, stores the result
% in |\childdocname| and saves a copy in |\childdocjob|:
%    \begin{macrocode}
\edef\childdocname{\scantokens\expandafter{\jobname\noexpand}}
\let\childdocjob\childdocname
%    \end{macrocode}

% \macro{\childdocdisable}
% The macro |\childdocdisable| prevents the main file
% from being processed more than once.
% At this stage, the main document command |\childdocmain|
% is assumed to be called once again where it should do nothing.
% Any subsequent call to it should prevent
% a secondary processing of the main document
% It overwrites the forwarding commands
% |\childdocof| and |\childdocforward|
% with empty macros to prevent further inclusions of the main document:
%    \begin{macrocode}
\newcommand{\childdocdisable}
{
  \renewcommand{\childdocmain}[1]{\renewcommand{\childdocmain}[1]{\endinput}}
  \renewcommand{\childdocof}[1]{}
  \renewcommand{\childdocby}[2][]{}
  \renewcommand{\childdocforward}[2][]{}
  \renewcommand{\childdocdisable}{}
}
%    \end{macrocode}

% \macro{\childdocmain}
% The macro |\childdocmain| is to be called at the top of the main file
% with nothing or the main filename (without extension) as argument.
% First, it breaks loops.
% If the argument is not empty and does not match |\childdocname|
% (which is set by the first inclusion of |childdoc.def|),
% |\ifchilddoc| is set to true, |\includeonly| is applied to the child file
% and |\jobname| is set to the main file
% (for proper handling of |.aux| files):
%    \begin{macrocode}
\newcommand{\childdocmain}[1]
{
  \childdocdisable\childdocmain{}
  \if?#1?\else
    \begingroup
      \def\childdoctmp{#1}
      \ifx\childdoctmp\childdocname
        \def\childdoctmp{}
      \else
        \def\childdoctmp
        {
          \childdoctrue
          \includeonly{\childdocname}
          \def\childdocjob{#1}
          \def\jobname{#1}
        }
      \fi
      \expandafter
    \endgroup
    \childdoctmp
  \fi
}
%    \end{macrocode}

% \macro{\childdocof}
% The command |\childdocof| redirects
% compilation to the main file |#1|.
%    \begin{macrocode}
\newcommand{\childdocof}[1]
{
  \childdocdisable
  \childdoctrue
  \includeonly{\childdocname}
  \def\jobname{#1}
  \def\childdocjob{#1}
  \input{#1}
}
%    \end{macrocode}

% \macro{\childdocby}
% The command |\childdocby| ....
%    \begin{macrocode}
\newcommand{\childdocby}[2][]
{
  \childdocdisable
  \childdoctrue
  \childdocmanualtrue
  \if?#1?\else
    \def\jobname{#2}
  \fi
  \def\childdocjob{#2}
  \input{#2}
  \endinput
}
%    \end{macrocode}

% \macro{\childdocforward}
% The command |\childdocforward| redirects
% compilation to the main file or
% (if the optional argument is given) a child file.
% Parameters are set as if the main file
% or a child file starting with |\childdocof| was compiled.
% Then compilation is handed over to the main file:
%    \begin{macrocode}
\newcommand{\childdocforward}[2][]
{
  \begingroup
    \if?#1?
      \def\childdoctmp
      {
        \def\childdocname{#2}
        \def\childdocjob{#2}
        \def\jobname{#2}
        \input{#2}
        \endinput
      }
    \else
      \def\childdoctmp
      {
        \childdocdisable
        \def\childdocname{#2}
        \childdoctrue
        \includeonly{#2}
        \def\childdocjob{#1}
        \def\jobname{#1}
        \input{#1}
        \endinput
      }
    \fi
    \expandafter
  \endgroup
  \childdoctmp
}
%    \end{macrocode}

% \macro{\childdocforwardprefix}
% The command |\childdocforwardprefix| redirects
% compilation to the main or a child file by means of a pattern.
% The prefix |#1| in the current filename is replaced by |#2|
% and the suffix of the current filename is kept
% (it is assumed that the filename does not contain the substring `|~~~|'
% which is used as a delimiter).
% Compilation is handed over to the new file by |\childdocforward|:
%    \begin{macrocode}
\newcommand{\childdocforwardprefix}[3][]
{
  \begingroup
    \def\childdocextract #2##1~~~{\def\childdoctmp{\childdocforward[#1]{#3##1}}}
    \expandafter\childdocextract\childdocname~~~
    \expandafter
  \endgroup
  \childdoctmp
}
%    \end{macrocode}

% \macro{\childdoc}
% The deprecated macro |\childdoc| is a legacy version of |\childdocmain|:
%    \begin{macrocode}
\newcommand{\childdoc}{\childdocmain}
%    \end{macrocode}

% \macro{\childdocredirect}
% The deprecated macro |\childdocredirect| is a legacy version
% of |\childdocforward| and |\childdocforwardprefix|:
%    \begin{macrocode}
\newcommand{\childdocredirect}[2][]
{
  \begingroup
    \if?#1?
      \def\childdoctmp{\childdocforward{#2}}
    \else
      \def\childdoctmp{\childdocforwardprefix{#1}{#2}}
    \fi
    \expandafter
  \endgroup
  \childdoctmp
}
%    \end{macrocode}

%\iffalse
%</package>
%\fi
%
\endinput
\childdocforward[|\textit{main}|]{|\textit{dest}|}"|
\end{center}
%
Here \textit{target} is the name of the output file,
\textit{main} is the name of the main file
and \textit{dest} is the name of the main or child file to be processed
(all filenames without extensions).
The optional argument \textit{main} can be omitted
if \textit{main} matches \textit{dest}.
Optionally, compilation \textit{flags} can be defined via |\def| commands.
This command line makes the \TeX{} engine believe
it is compiling the file \textit{target}
whose content is specified as the latter parameter.
The provided code then forwards the processing to
\textit{main} or \textit{dest} as described in \secref{sec:forward}.

%%%%%%%%%%%%%%%%%%%%%%%%%%%%%%%%%%%%%%%%%%%%%%%%%%%%%%%%%%%%%%%%%%%%%%%%%%%%%%%%
\subsection{Include by Input}
\label{sec:input}

Including child documents by |\include| has some restrictions by design.
Most notably, the content of a child document always occupies
its own set of pages; pages cannot be shared between child documents.
Usually, this behaviour makes perfect sense
because each child document contain an essential part of the document.
However, in some situations it may be desirable to compose
a document from a collection of parts
without having mandatory page breaks between then.
For this case, the package
provides a mechanism to include parts
by |\input| which can also be processed individually.
However, by construction this mechanism
requires manual handling of the content to be output.

%%%%%%%%%%%%%%%%%%%%%%%%%%%%%%%%%%%%%%%%
\DescribeMacro{\ifchilddocmanual}
The main file should be prepared as usual, see \secref{sec:include}.
However, the document body must make a distinction
between processing of an individual part and of the main document, e.g.:
%
\begin{center}
\begin{tabular}{l}
|\ifchilddocmanual|\\
|\input{\childdocname}|\\
|\||else|\\
\textit{document body with }|\input{|\textit{part}|}|\\
|\||fi|
\end{tabular}
\end{center}
%
The conditional |\ifchilddocmanual| is true whenever
a part to be included by |\input| is being compiled,
and the name of the part is stored in |\childdocname|.

%%%%%%%%%%%%%%%%%%%%%%%%%%%%%%%%%%%%%%%%
\DescribeMacro{\childdocby}
Each part to be included by |\input| should start with:
%
\begin{center}
\begin{tabular}{l}
|% \iffalse
%
% childdoc.dtx Copyright (C) 2017-2018 Niklas Beisert
%
% This work may be distributed and/or modified under the
% conditions of the LaTeX Project Public License, either version 1.3
% of this license or (at your option) any later version.
% The latest version of this license is in
%   http://www.latex-project.org/lppl.txt
% and version 1.3 or later is part of all distributions of LaTeX
% version 2005/12/01 or later.
%
% This work has the LPPL maintenance status `maintained'.
%
% The Current Maintainer of this work is Niklas Beisert.
%
% This work consists of the files childdoc.dtx and childdoc.ins
% and the derived files childdoc.def and cdocsamp.tex with
% cdocsch1.tex, cdocsch2.tex, cdocsdrf.tex, cdocsfn1.tex, cdocsfn2.tex.
%
%<package>\ifdefined\childdocmain\endinput\fi
%<package>\ProvidesFile{childdoc.def}[2018/12/30 v2.0 child document driver]
%<samplemain>\ProvidesFile{cdocsamp.tex}[2018/12/30 v2.0 sample for childdoc]
%<*driver>
%\ProvidesFile{childdoc.drv}[2018/12/30 v2.0 childdoc reference manual file]
\PassOptionsToClass{10pt,a4paper}{article}
\documentclass{ltxdoc}

\usepackage[margin=35mm]{geometry}
\usepackage{hyperref}
\usepackage{hyperxmp}
\usepackage[usenames]{color}

\hypersetup{colorlinks=true}
\hypersetup{pdfstartview=FitH}
\hypersetup{pdfpagemode=UseNone}
\hypersetup{pdfsource={}}
\hypersetup{pdflang={en-UK}}
\hypersetup{pdfcopyright={Copyright 2017-2018 Niklas Beisert.
  This work may be distributed and/or modified under the
  conditions of the LaTeX Project Public License, either version 1.3
  of this license or (at your option) any later version.}}
\hypersetup{pdflicenseurl={http://www.latex-project.org/lppl.txt}}
\hypersetup{pdfcontactaddress={ETH Zurich, ITP, HIT K,
  Wolfgang-Pauli-Strasse 27}}
\hypersetup{pdfcontactpostcode={8093}}
\hypersetup{pdfcontactcity={Zurich}}
\hypersetup{pdfcontactcountry={Switzerland}}
\hypersetup{pdfcontactemail={nbeisert@itp.phys.ethz.ch}}
\hypersetup{pdfcontacturl={http://people.phys.ethz.ch/\xmptilde nbeisert/}}

\newcommand{\secref}[1]{\hyperref[#1]{section \ref*{#1}}}

\parskip1ex
\parindent0pt
\let\olditemize\itemize
\def\itemize{\olditemize\parskip0pt}

\begin{document}

\title{The \textsf{childdoc} Package}
\hypersetup{pdftitle={The childdoc Package}}
\author{Niklas Beisert\\[2ex]
  Institut f\"ur Theoretische Physik\\
  Eidgen\"ossische Technische Hochschule Z\"urich\\
  Wolfgang-Pauli-Strasse 27, 8093 Z\"urich, Switzerland\\[1ex]
  \href{mailto:nbeisert@itp.phys.ethz.ch}
  {\texttt{nbeisert@itp.phys.ethz.ch}}}
\hypersetup{pdfauthor={Niklas Beisert}}
\hypersetup{pdfsubject={Manual for the LaTeX2e Package childdoc}}
\date{30 December 2018, \textsf{v2.0}}
\maketitle

\begin{abstract}\noindent
\textsf{childdoc} is a \LaTeXe{} package
that enables the direct compilation
of document sections included by |\include|
to individual files.
\end{abstract}

\begingroup
\parskip0ex
\tableofcontents
\endgroup

%%%%%%%%%%%%%%%%%%%%%%%%%%%%%%%%%%%%%%%%%%%%%%%%%%%%%%%%%%%%%%%%%%%%%%%%%%%%%%%%
%%%%%%%%%%%%%%%%%%%%%%%%%%%%%%%%%%%%%%%%%%%%%%%%%%%%%%%%%%%%%%%%%%%%%%%%%%%%%%%%
\section{Introduction}

\LaTeX{} provides a mechanism to structure a large document (such as a book)
into a main file and several child files (containing the chapters)
using the |\include| command.
This mechanism is beneficial for documents
which span hundreds of pages in order to
make the source file(s) more manageable.
Moreover, compilation can be restricted to
selected child files by means of the |\includeonly| command.
The latter feature can be used to reduce the compilation time while editing
(this was significantly more useful in the earlier days of \LaTeX{})
or to generate a smaller document which is easier to navigate.
Another application of |\includeonly| is to generate
documents consisting of selected parts of the complete document.

However, there are a few drawbacks of the plain |\include| mechanism:
\begin{itemize}
\item
The child files cannot be compiled on their own,
they can only be compiled via the main file.
A naive editing environment
(such as a text editor with an option
to have the current file processed by \LaTeX)
may require one to switch to the main file before compiling;
attempting to compile the child file produces errors.
\item
The main file must be modified (each time)
to adjust the |\includeonly| command
to the present needs. This easily leaves the main file in a messy state.
\item
The generated document will always carry the filename
of the main document. This is inconvenient if
several child files are to be compiled and
to be kept for distribution.
\end{itemize}

The present package provides a simple interface
to make child files individually compilable by \LaTeX{}.
Compiling a child file then has the same effect as compiling
the main file with an |\includeonly| command
to select the appropriate child.
Moreover the generated document will carry the name of the child
rather than the main file.
This resolves all three above issues.

This feature is meant to make the editing of books,
thesis documents and lecture notes somewhat more convenient.
However, the package can also be used efficiently for
composing a series of documents (such as exercise sheets)
which are typically distributed individually.
It then assists the author in generating the individual documents
(potentially in different versions)
as well as a document containing the collected series.
Another application is in developing style files
or other kinds of included material
where compilation of the style file could redirect
to a sample or test file.

%%%%%%%%%%%%%%%%%%%%%%%%%%%%%%%%%%%%%%%%%%%%%%%%%%%%%%%%%%%%%%%%%%%%%%%%%%%%%%%%
%%%%%%%%%%%%%%%%%%%%%%%%%%%%%%%%%%%%%%%%%%%%%%%%%%%%%%%%%%%%%%%%%%%%%%%%%%%%%%%%
\section{Usage}

First of all, the package \textsf{childdoc} is \emph{not} a standard
\LaTeXe{} |.sty| style file! Therefore it needs to be invoked in
a non-standard way.

%%%%%%%%%%%%%%%%%%%%%%%%%%%%%%%%%%%%%%%%%%%%%%%%%%%%%%%%%%%%%%%%%%%%%%%%%%%%%%%%
\subsection{Included Files}
\label{sec:include}

%%%%%%%%%%%%%%%%%%%%%%%%%%%%%%%%%%%%%%%%
\DescribeMacro{\childdocmain}
To use the package, add the commands
\begin{center}
\begin{tabular}{l}
|\input{childdoc.def}|\\
|\childdocmain{}|\\
\end{tabular}
\end{center}
at the very top of the main \LaTeX{} file,
in particular \emph{before} the |\documentclass| statement!
The argument of |\childdocmain| should be left empty
(but it must be present).

%%%%%%%%%%%%%%%%%%%%%%%%%%%%%%%%%%%%%%%%
\DescribeMacro{\childdocof}
Furthermore, add the commands
\begin{center}
\begin{tabular}{l}
|\input{childdoc.def}|\\
|\childdocof{|\textit{main}|}|\\
\end{tabular}
\end{center}
at the top of every child file \textit{child}
which is included by |\include{|\textit{child}|}|
from within the main file
(or at least for those files to be compiled individually).
The argument \textit{main} must be the filename of the main file.

There are a couple of
considerations in setting up the main and child documents:

%%%%%%%%%%%%%%%%%%%%%%%%%%%%%%%%%%%%%%%%
\paragraph{Restrictions.}

Please note the following restrictions:
\begin{itemize}
\item
|\childdocmain| must be called with one argument \textit{main}
to ensure compatibility with earlier version of the package.
It must either be empty (|\childdocmain{}|)
or precisely match the filename of the main file in which it is specified.
See \secref{sec:detection} for further information.
\item
The filename \textit{main} must be specified without the |.tex| extension.
\item
The filename \textit{main} is case sensitive
(even in case-insensitive file systems)
due to internal string comparison.
\item
The argument \textit{main} should be fully expanded, it cannot be a macro.
\item
Subdirectories and special characters should be avoided in filenames.
\item
The command |\childdocmain{|\textit{main}|}| must be followed by a whitespace.
It should not be followed immediately by another command
or by a comment mark `|%|'.
This is because the \TeX{} parser reads the token immediately following
the argument of |\childdocmain| and puts it
at the beginning of every child section;
however, a white\-space is ignored.
\end{itemize}

%%%%%%%%%%%%%%%%%%%%%%%%%%%%%%%%%%%%%%%%
\paragraph{Content of Main File.}

It is advisable to place all content in the child files included by |\include|.
Any output contained in the main file will appear in all child documents
unless suppressed manually;
it cannot be suppressed automatically by the |\includeonly| directive
and thus should normally be avoided.
A method to include some content in the main file
by means of conditional processing is described in \secref{sec:conditional}.

%%%%%%%%%%%%%%%%%%%%%%%%%%%%%%%%%%%%%%%%
\paragraph{Page Numbering.}

When only a part of the document is compiled,
the appropriate numbering of pages
(as well as other status parameters)
is determined from the |.aux| files.
The latter contain information from previous passes.
However this information needs to propagate through
all intermediate child documents.
Therefore the page numbering in child documents may well
be inconsistent until the complete document is compiled at least once.

A useful (if unconventional) way to always ensure a consistent
page numbering is to restart the numbering in each child document
and denote the pages by `\textit{child}|.|\textit{page}'
where \textit{child} represents the chapter/section number of the child file.
This can be achieved by the command
|\numberwithin{page}{|\textit{child}|}|
of the \textsf{amsmath} package
where \textit{child} can be |chapter| or |section|
depending on the chosen structuring.
Alternatively, one can modify the macro |\thepage| appropriately
and reset the counter |page| at the start of each child file.

%%%%%%%%%%%%%%%%%%%%%%%%%%%%%%%%%%%%%%%%%%%%%%%%%%%%%%%%%%%%%%%%%%%%%%%%%%%%%%%%
\subsection{Conditional Processing}
\label{sec:conditional}

The package provides a mechanism to compile different versions
of a document. To customise the versions further some conditional processing
can come in handy to distinguish which version is being compiled.
The package provides two macros to describe the compilation context:

%%%%%%%%%%%%%%%%%%%%%%%%%%%%%%%%%%%%%%%%
\DescribeMacro{\ifchilddoc}
The conditional |\ifchilddoc| distinguishes between the compilation of
child documents and the main document:
%
\begin{center}
|\ifchilddoc |\textit{child-code}| |[|\||else |\textit{main-code}]| \||fi|
\end{center}

%%%%%%%%%%%%%%%%%%%%%%%%%%%%%%%%%%%%%%%%
\DescribeMacro{\childdocname}
\DescribeMacro{\childdocjob}
The macro |\childdocname| contains the filename (without extension)
of the main or child file being processed.
Note that |\childdocjob| will always contain the name of the main file.

%%%%%%%%%%%%%%%%%%%%%%%%%%%%%%%%%%%%%%%%
\paragraph{Title Page.}

Conditional processing can be used to include a title or banner page
in the main document when proper precautions are taken.
Importantly, the code in the main file should ensure that the page counter
(as well as other status parameters which are stored in the |.aux| files)
takes the same value after the conditional processing.
Otherwise the page numbers may take divergent values
depending on which part is compiled.

For example, a title page could be declared by:
%
\begin{center}
\begin{tabular}{l}
|\ifchilddoc\||else|\\
|\addtocounter{page}{-1}|\\
\textit{code for title page}\\
|\newpage|\\
|\||fi|
\end{tabular}
\end{center}
%
A banner page for the child documents can be generated by:
%
\begin{center}
\begin{tabular}{l}
|\ifchilddoc|\\
|\addtocounter{page}{-1}|\\
\textit{code for banner page}\\
|\newpage|\\
|\||fi|
\end{tabular}
\end{center}
%
Here one could write a message such as:
\begin{center}
|This is the part \childdocname{} of \childdocjob{}.|
\end{center}

%%%%%%%%%%%%%%%%%%%%%%%%%%%%%%%%%%%%%%%%%%%%%%%%%%%%%%%%%%%%%%%%%%%%%%%%%%%%%%%%
\subsection{Flags}
\label{sec:flags}

The package makes it easy to generate different versions
of the main or child documents.
To this end compilation flags can be defined
and assigned different default values.
They will be particularly useful in conjunction
with the forwarding mechanism described in \secref{sec:forward}.

For example, it may be useful to have a flag |\version|
which can be set to |draft| or |final|.
The document source will contain some conditional code
depending on the value of |\version|.
Suppose further, the flag should default to |final| for the main file
and to |draft| for child files
which is a natural assignment for editing the document.
This is achieved by placing the following code
in the preamble of the main document
(below the |\childdocmain| directive):
%
\begin{center}
\begin{tabular}{l}
|\ifchilddoc|\\
|\providecommand{\version}{draft}|\\
|\||else|\\
|\providecommand{\version}{final}|\\
|\||fi|
\end{tabular}
\end{center}
%
The definition by |\providecommand| makes sure
that previous definitions are not overwritten.
Further statements |\providecommand{\version}{...}|
can thus be added before the above code to override it.

For the main file, one might add a line
(between |\childdocmain| and the above block)
%
\begin{center}
|%\ifchilddoc\||else\providecommand{\version}{draft}\||fi|
\end{center}
%
which can be uncommented to produce a draft version.
Likewise one can add a line to the very top of a child file
(above the |\childdocof{|\textit{main}|}| directive)
%
\begin{center}
|%\providecommand{\version}{final}|
\end{center}
%
which can be uncommented to produce the final version of this child document.

%%%%%%%%%%%%%%%%%%%%%%%%%%%%%%%%%%%%%%%%%%%%%%%%%%%%%%%%%%%%%%%%%%%%%%%%%%%%%%%%
\subsection{Forwarding}
\label{sec:forward}

Different versions of the main or child documents
using compilation flags as described in \secref{sec:flags}
can be (permanently) stored in different files
for convenient compilation, viewing and distribution.
To this end, the package defines a command
to pass on compilation to a different file:

%%%%%%%%%%%%%%%%%%%%%%%%%%%%%%%%%%%%%%%%
\DescribeMacro{\childdocforward}
The command |\childdocforward| redirects processing to
another source file:
%
\begin{center}
\begin{tabular}{l}
|\input{childdoc.def}|\\
|\childdocforward[|\textit{main}|]{|\textit{dest}|}|\\
\end{tabular}
\end{center}
%
The argument \textit{dest} is the destination file
(without extension).
It should be the main file or one of the child files.
Note that further \textsf{childdoc} directives
such as |\childdocof| and |\childdocforward|
in the indicated file will be processed in this form.
The optional argument \textit{main}
passes on directly to the main file \textit{main}
while pretending to compile the child \textit{dest}.
This form behaves as if \textit{dest}
issues |\childdocof{|\textit{main}|}| right away,
and no further \textsf{childdoc} directives will be processed.

%%%%%%%%%%%%%%%%%%%%%%%%%%%%%%%%%%%%%%%%
\DescribeMacro{\...prefix}
In the alternative form |\childdocforwardprefix|,
%
\begin{center}
\begin{tabular}{l}
|\input{childdoc.def}|\\
|\childdocforwardprefix[|\textit{main}|]{|\textit{prefix}|}{|\textit{dest}|}|
\end{tabular}
\end{center}
%
the destination file is determined by a pattern
depending on the current file:
To make this work, the current file must be called
`{\textit{prefix}\hspace{0.2em}\textit{suffix}}'
with \textit{prefix} matching precisely the argument.
Processing is then passed on to the file
`{\textit{dest}\hspace{0.2em}\textit{suffix}}'.
Surely, the same effect is achieved by
directly specifying the
argument `{\textit{dest}\hspace{0.2em}\textit{suffix}}'
in the first form.
However, that requires to set up a different file
for each child. With the alternative form of the command
all these files can have exactly the same content
which simplifies setting them up and maintaining them.

For example, the following file |draft.tex|
with a compilation flag |\version| as described in \secref{sec:flags}
compiles the main document as a draft:
%
\begin{center}
\begin{tabular}{l}
|\def\version{draft}|\\
|\input{childdoc.def}|\\
|\childdocforward{|\textit{main}|}|
\end{tabular}
\end{center}
%
Likewise, the following files |final|\textit{nn}|.tex|
compile the final version of the child document
|child|\textit{nn}|.tex|:
%
\begin{center}
\begin{tabular}{l}
|\def\version{final}|\\
|\input{childdoc.def}|\\
|\childdocforwardprefix{final}{child}|
\end{tabular}
\end{center}
%

Note that when several versions of a main file and/or of each child file
are to be generated, it may be convenient to set up a |Makefile| or
shell script to automatise the process.

%%%%%%%%%%%%%%%%%%%%%%%%%%%%%%%%%%%%%%%%%%%%%%%%%%%%%%%%%%%%%%%%%%%%%%%%%%%%%%%%
\subsection{Command Line Processing}
\label{sec:commandline}

The effect of redirection files can also be achieved by invoking
the \LaTeX{} compiler with a more elaborate command line.
Most conveniently this should be done as part
of a shell script or a |Makefile|.

When using \textsf{childdoc} in the main file, the following
command lines effectively perform a redirection
(note that depending on the shell being used,
backslashes may have to be doubled: `|\|' $\to$ `|\\|'):
%
\begin{center}
|... -jobname "|\textit{target}|" |\\|"|[\textit{flags}]%
|\input{childdoc.def}\childdocforward[|\textit{main}|]{|\textit{dest}|}"|
\end{center}
%
Here \textit{target} is the name of the output file,
\textit{main} is the name of the main file
and \textit{dest} is the name of the main or child file to be processed
(all filenames without extensions).
The optional argument \textit{main} can be omitted
if \textit{main} matches \textit{dest}.
Optionally, compilation \textit{flags} can be defined via |\def| commands.
This command line makes the \TeX{} engine believe
it is compiling the file \textit{target}
whose content is specified as the latter parameter.
The provided code then forwards the processing to
\textit{main} or \textit{dest} as described in \secref{sec:forward}.

%%%%%%%%%%%%%%%%%%%%%%%%%%%%%%%%%%%%%%%%%%%%%%%%%%%%%%%%%%%%%%%%%%%%%%%%%%%%%%%%
\subsection{Include by Input}
\label{sec:input}

Including child documents by |\include| has some restrictions by design.
Most notably, the content of a child document always occupies
its own set of pages; pages cannot be shared between child documents.
Usually, this behaviour makes perfect sense
because each child document contain an essential part of the document.
However, in some situations it may be desirable to compose
a document from a collection of parts
without having mandatory page breaks between then.
For this case, the package
provides a mechanism to include parts
by |\input| which can also be processed individually.
However, by construction this mechanism
requires manual handling of the content to be output.

%%%%%%%%%%%%%%%%%%%%%%%%%%%%%%%%%%%%%%%%
\DescribeMacro{\ifchilddocmanual}
The main file should be prepared as usual, see \secref{sec:include}.
However, the document body must make a distinction
between processing of an individual part and of the main document, e.g.:
%
\begin{center}
\begin{tabular}{l}
|\ifchilddocmanual|\\
|\input{\childdocname}|\\
|\||else|\\
\textit{document body with }|\input{|\textit{part}|}|\\
|\||fi|
\end{tabular}
\end{center}
%
The conditional |\ifchilddocmanual| is true whenever
a part to be included by |\input| is being compiled,
and the name of the part is stored in |\childdocname|.

%%%%%%%%%%%%%%%%%%%%%%%%%%%%%%%%%%%%%%%%
\DescribeMacro{\childdocby}
Each part to be included by |\input| should start with:
%
\begin{center}
\begin{tabular}{l}
|\input{childdoc.def}|\\
|\childdocby{|\textit{main}|}|\\
\end{tabular}
\end{center}
%
The directive |\childdocby| is similar to |\childdocof|
described in \secref{sec:include},
but the subsequent selection of content must be done manually.
To that end, both |\ifchilddoc| and |\ifchilddocmanual|
will be true upon processing of a part,
and the name of the part is stored in |\childdocname|.
Note that |\jobname| will be set to the filename of the current part
so that each part receives an individual |.aux| file
that does not interfere with the |.aux| file(s) of the main document.
This behaviour can be altered by the alternative form
|\childdocby[*]{|\textit{main}|}| (with a non-empty optional argument)
which uses the |.aux| file of the main document
by setting |\jobname| to \textit{main}.

%%%%%%%%%%%%%%%%%%%%%%%%%%%%%%%%%%%%%%%%%%%%%%%%%%%%%%%%%%%%%%%%%%%%%%%%%%%%%%%%
\subsection{Driver Development}
\label{sec:driver}

The \textsf{childdoc} mechanism can also be use for the development
of definition files such as \LaTeX{} styles or classes.
This case differs from the above setup with multiple parts
included by |\include| in that no |\includeonly| should be invoked.
This can be achieved by starting the include file
(before |\ProvidesPackage|) with:
%
\begin{center}
\begin{tabular}{l}
|\input{childdoc.def}|\\
|\childdocforward{|\textit{main}|}|\\
\end{tabular}
\end{center}
%
or alternatively with:
%
\begin{center}
\begin{tabular}{l}
|\input{childdoc.def}|\\
|\childdocby{|\textit{main}|}|\\
\end{tabular}
\end{center}
%
Both forms have slightly different effects as described above.
The main file is prepared as usual, see \secref{sec:include}.

%%%%%%%%%%%%%%%%%%%%%%%%%%%%%%%%%%%%%%%%%%%%%%%%%%%%%%%%%%%%%%%%%%%%%%%%%%%%%%%%
\subsection{Legacy Detection}
\label{sec:detection}

The directive |\childdocmain| in the main file can detect
whether the complete document or merely a child is to be compiled
even without using the directive |\childdocof|.
This method is deprecated because it is less robust
and there is no compelling reason to use it;
it is merely provided for backward compatibility
and it may be removed in future versions.

If the detection mechanism is to be used,
it is mandatory to correctly specify
the filename of the main file as the argument of |\childdocmain|:
%
\begin{center}
\begin{tabular}{l}
|\input{childdoc.def}|\\
|\childdocmain{|\textit{main}|}|\\
\end{tabular}
\end{center}
%
If |\jobname| does not match the argument \textit{main} of |\childdocmain|,
it is assumed that |\jobname| points to the child file to be compiled.
When using |\childdocmain| with the main file specified as argument,
it suffices to start a child file
with just |\input{|\textit{main}|}|
without loading of the package and using |\childdocof|.
If instead all processing is done
with the appropriate \textsf{childdoc} directives,
the argument of \textit{main} of |\childdocmain| can be empty.

An alternative version of the command line processing described
in \secref{sec:commandline} using the detection mechanism reads:
%
\begin{center}
|... -jobname "|\textit{target}|" "|[\textit{flags}]%
[|\def\jobname{|\textit{dest}|}|]|\input{|\textit{main}|}"|
\end{center}

%%%%%%%%%%%%%%%%%%%%%%%%%%%%%%%%%%%%%%%%%%%%%%%%%%%%%%%%%%%%%%%%%%%%%%%%%%%%%%%%
\subsection{Manual Code}
\label{sec:manual}

In case one cannot be certain whether the definitions file |childdoc.def|
is installed on the target \TeX{} distribution
and one prefers not to ship it,
it is conceivable to paste a few relevant commands into the sources.

To that end, drop all statements |\input{childdoc.def}|
and perform the replacements as outlined below.
Instead of |\childdocmain{|\textit{main}|}| add the following code
to the top of the main file:
%
\begin{center}
\begin{tabular}{l}
|\||ifdefined\childdocname\endinput\||fi\newif\ifchilddoc|\\
|\edef\childdocname{\scantokens\expandafter{\jobname\noexpand}}|\\
|\def\childdocmain{|\textit{main}|}\||ifx\childdocmain\childdocname\||else|\\
|\childdoctrue\includeonly{\childdocname}\let\jobname\childdocmain\||fi|\\
\end{tabular}
\end{center}
%
Instead of |\childdocof{|\textit{main}|}| just include the main file
at the top of each child file:
%
\begin{center}
|\input{|\textit{main}|}|
\end{center}
%
A simple redirection |\childdocforward{|\textit{dest}|}| is achieved by:
%
\begin{center}
|\def\jobname{|\textit{dest}|}\input{\jobname}|
\end{center}
%
The redirection with prefix
|\childdocforwardprefix[|\textit{prefix}|]{|\textit{dest}|}|
is accomplished by:
%
\begin{center}
\begin{tabular}{l}
|{\edef\jobname{\scantokens\expandafter{\jobname\noexpand}}|\\
|\def\redirectjob |\textit{prefix}|#1~~~{\gdef\jobname{|\textit{dest}|#1}}|\\
|\expandafter\redirectjob\jobname~~~}\input{\jobname}|
\end{tabular}
\end{center}

In an alternative approach,
child documents can be compiled by a specific command line
without additional code or specific definitions:
%
\begin{center}
|... -jobname "|\textit{target}|" "|[\textit{flags}]%
|\includeonly{|\textit{dest}|}\input{|\textit{main}|}"|
\end{center}
%

%%%%%%%%%%%%%%%%%%%%%%%%%%%%%%%%%%%%%%%%%%%%%%%%%%%%%%%%%%%%%%%%%%%%%%%%%%%%%%%%
%%%%%%%%%%%%%%%%%%%%%%%%%%%%%%%%%%%%%%%%%%%%%%%%%%%%%%%%%%%%%%%%%%%%%%%%%%%%%%%%
\section{Information}

%%%%%%%%%%%%%%%%%%%%%%%%%%%%%%%%%%%%%%%%%%%%%%%%%%%%%%%%%%%%%%%%%%%%%%%%%%%%%%%%
\subsection{Copyright}

Copyright \copyright{} 2017--2018 Niklas Beisert

This work may be distributed and/or modified under the
conditions of the \LaTeX{} Project Public License, either version 1.3
of this license or (at your option) any later version.
The latest version of this license is in
  \url{http://www.latex-project.org/lppl.txt}
and version 1.3 or later is part of all distributions of \LaTeX{}
version 2005/12/01 or later.

This work has the LPPL maintenance status `maintained'.

The Current Maintainer of this work is Niklas Beisert.

This work consists of the files |README.txt|, |childdoc.ins| and |childdoc.dtx|
as well as the derived files |childdoc.def|, |cdocsamp.tex|
with |cdocsch1.tex|, |cdocsch2.tex|, |cdocspt3.tex|, |cdocspt4.tex|,
|cdocsdrf.tex|, |cdocsfn1.tex|, |cdocsfn2.tex|
as well as |childdoc.pdf|.

%%%%%%%%%%%%%%%%%%%%%%%%%%%%%%%%%%%%%%%%%%%%%%%%%%%%%%%%%%%%%%%%%%%%%%%%%%%%%%%%
\subsection{Files and Installation}

The package consists of the files:
%
\begin{center}
\begin{tabular}{ll}
    |README.txt|   & readme file \\
    |childdoc.ins| & installation file \\
    |childdoc.dtx| & source file \\
    |childdoc.def| & definition file \\
    |cdocsamp.tex| & sample main file \\
    |cdocsch1.tex| & sample include file \\
    |cdocsch2.tex| & sample include file \\
    |cdocspt3.tex| & sample part file \\
    |cdocspt4.tex| & sample part file \\
    |cdocsdrf.tex| & sample redirection file \\
    |cdocsfn1.tex| & sample redirection file \\
    |cdocsfn2.tex| & sample redirection file \\
    |childdoc.pdf| & manual
\end{tabular}
\end{center}
%
The distribution consists of the files
|README.txt|, |childdoc.ins| and |childdoc.dtx|.
%
\begin{itemize}
\item
Run (pdf)\LaTeX{} on |childdoc.dtx|
to compile the manual |childdoc.pdf| (this file).
\item
Run \LaTeX{} on |childdoc.ins| to create the definitions file |childdoc.def|
and the sample |cdocsamp.tex| with include files
|cdocsch1.tex|, |cdocsch2.tex|, |cdocspt3.tex|, |cdocspt4.tex|,
|cdocsdrf.tex|, |cdocsfn1.tex|, |cdocsfn2.tex|.
Then copy the file |childdoc.def| to an appropriate directory of your \LaTeX{}
distribution, e.g.\ \textit{texmf-root}|/tex/latex/childdoc|.
\end{itemize}

%%%%%%%%%%%%%%%%%%%%%%%%%%%%%%%%%%%%%%%%%%%%%%%%%%%%%%%%%%%%%%%%%%%%%%%%%%%%%%%%
\subsection{Related CTAN Packages}

There are several other packages which offer a similar functionality:
%
\begin{itemize}
\item
The packages
\href{http://ctan.org/pkg/docmute}{\textsf{docmute}},
\href{http://ctan.org/pkg/includex}{\textsf{includex}} and
\href{http://ctan.org/pkg/standalone}{\textsf{standalone}}
provide commands to include only the document body of
a child file thus allowing both files to be compiled individually.
\item
The packages \href{http://ctan.org/pkg/subdocs}{\textsf{subdocs}}
and \href{http://ctan.org/pkg/subfiles}{\textsf{subfiles}}
provide structures in which the main and child documents can be
encapsulated and allowing them to be compiled individually.
The inclusion mechanism is different from the conventional |\include|.
\item
The package \href{http://ctan.org/pkg/combine}{\textsf{combine}}
is an elaborate solution to combine several documents into one.
\end{itemize}
%
See also the CTAN topic \href{http://ctan.org/topic/subdocs}{\textsf{subdocs}}
for further related packages.
The present package differs from the above solutions in that
a document structure constructed with the conventional |\include| mechanism
just needs two extra commands at the top of every file
such that all constituent files can be compiled individually.

%%%%%%%%%%%%%%%%%%%%%%%%%%%%%%%%%%%%%%%%%%%%%%%%%%%%%%%%%%%%%%%%%%%%%%%%%%%%%%%%
%\subsection{Feature Suggestions}
%
%The following is a list of features which may be useful for future
%versions of this package:
%%
%\begin{itemize}
%\item
%\ldots
%\end{itemize}

%%%%%%%%%%%%%%%%%%%%%%%%%%%%%%%%%%%%%%%%%%%%%%%%%%%%%%%%%%%%%%%%%%%%%%%%%%%%%%%%
\subsection{Revision History}

%%%%%%%%%%%%%%%%%%%%%%%%%%%%%%%%%%%%%%%%
\paragraph{v2.0:} 2018/12/30

\begin{itemize}
\item
immediate forward processing
\item
added |\childdocby| mechanism
\item
manual restructured
\end{itemize}

%%%%%%%%%%%%%%%%%%%%%%%%%%%%%%%%%%%%%%%%
\paragraph{v1.6:} 2018/01/17

\begin{itemize}
\item
application for development of include files
\item
corrections to manual
\end{itemize}

%%%%%%%%%%%%%%%%%%%%%%%%%%%%%%%%%%%%%%%%
\paragraph{v1.5:} 2017/05/21

\begin{itemize}
\item
more complete structuring introduced
\item
|\childdocof| introduced
\item
|\childdoc| renamed to |\childdocmain|
\item
|\childredirect| renamed to |\childdocforward| and |\childdocforwardprefix|
and functionality expanded
\end{itemize}

%%%%%%%%%%%%%%%%%%%%%%%%%%%%%%%%%%%%%%%%
\paragraph{v1.0:} 2017/04/27

\begin{itemize}
\item
manual and install package
\item
first version published on CTAN
\end{itemize}

%%%%%%%%%%%%%%%%%%%%%%%%%%%%%%%%%%%%%%%%
\paragraph{v0.6:} 2017/04/26

\begin{itemize}
\item
redirection mechanism added
\end{itemize}

%%%%%%%%%%%%%%%%%%%%%%%%%%%%%%%%%%%%%%%%
\paragraph{v0.5:} 2017/04/26

\begin{itemize}
\item
functionality in definition file
\end{itemize}


%%%%%%%%%%%%%%%%%%%%%%%%%%%%%%%%%%%%%%%%%%%%%%%%%%%%%%%%%%%%%%%%%%%%%%%%%%%%%%%%
%%%%%%%%%%%%%%%%%%%%%%%%%%%%%%%%%%%%%%%%%%%%%%%%%%%%%%%%%%%%%%%%%%%%%%%%%%%%%%%%
%%%%%%%%%%%%%%%%%%%%%%%%%%%%%%%%%%%%%%%%%%%%%%%%%%%%%%%%%%%%%%%%%%%%%%%%%%%%%%%%
\appendix

\settowidth\MacroIndent{\rmfamily\scriptsize 000\ }

 \DocInput{childdoc.dtx}

\end{document}
%</driver>
% \fi
%
% %%%%%%%%%%%%%%%%%%%%%%%%%%%%%%%%%%%%%%%%%%%%%%%%%%%%%%%%%%%%%%%%%%%%%%%%%%%%%%
% %%%%%%%%%%%%%%%%%%%%%%%%%%%%%%%%%%%%%%%%%%%%%%%%%%%%%%%%%%%%%%%%%%%%%%%%%%%%%%
% \section{Sample}
%\iffalse
%<*samplemain>
%\fi
%
% The following presents a sample document
% with two chapters, two parts, a title page,
% a compile flag as well as three forwarding files to set the flag.
% It consists of eight |.tex| files:
% \begin{center}
% \begin{tabular}{ll}
% |cdocsamp.tex|&main file\\
% |cdocsch1.tex|&include file for chapter 1\\
% |cdocsch2.tex|&include file for chapter 2\\
% |cdocspt3.tex|&include file for part 3\\
% |cdocspt4.tex|&include file for part 4\\
% |cdocsdrf.tex|&forwarding file for main file in draft mode\\
% |cdocsfi1.tex|&forwarding file for final version of chapter 1\\
% |cdocsfi2.tex|&forwarding file for final version of chapter 2\\
% \end{tabular}
% \end{center}
% Each of the eight files can be compiled directly by the \LaTeX{} compiler.
%
% %%%%%%%%%%%%%%%%%%%%%%%%%%%%%%%%%%%%%%
% \paragraph{Main File.}
%
% The main file is called |cdocsamp.tex|.
%
% Load the \textsf{childdoc} definitions and
% declare the filename for the main document:
%    \begin{macrocode}
\input{childdoc.def}
\childdocmain{}
%    \end{macrocode}

% Optional override for |\version| flag:
%    \begin{macrocode}
%%\ifchilddoc\else\providecommand{\version}{draft}\fi
%    \end{macrocode}

% Define the default values for the |\version| flag
% (|final| for the main file and |draft| for childs):
%    \begin{macrocode}
\ifchilddoc
\providecommand{\version}{draft}
\else
\providecommand{\version}{final}
\fi
%    \end{macrocode}

% Load the standard document class:
%    \begin{macrocode}
\documentclass[12pt]{article}
%    \end{macrocode}

% Start the document body:
%    \begin{macrocode}
\begin{document}
%    \end{macrocode}

% Declare a title page.
% Print title, part of document being processed and version flag:
%    \begin{macrocode}
\addtocounter{page}{-1}
\begin{center}
{\LARGE\bfseries{}childdoc example\par}
\vspace{1cm}
\ifchilddoc
\ifchilddocmanual part\else chapter\fi:
`\childdocname' of `\childdocjob'\par
\else
main document: `\childdocjob'\par
\fi
version: \version\par
\end{center}
\newpage
%    \end{macrocode}

% Manually include selected file,
% otherwise process as usual:
%    \begin{macrocode}
\ifchilddocmanual
\section*{part `\childdocname'}
\input{\childdocname}
\else
%    \end{macrocode}

% Include the two chapters:
%    \begin{macrocode}
\include{cdocsch1}
\include{cdocsch2}
%    \end{macrocode}

% Include the two parts unless only chapters should be displayed:
%    \begin{macrocode}
\ifchilddoc\else
\section{part three}
\input{cdocspt3}
\section{part four}
\input{cdocspt4}
\fi
%    \end{macrocode}

% Process as usual until here:
%    \begin{macrocode}
\fi
%    \end{macrocode}

% End of document body:
%    \begin{macrocode}
\end{document}
%    \end{macrocode}
%\iffalse
%</samplemain>
%\fi
%
% %%%%%%%%%%%%%%%%%%%%%%%%%%%%%%%%%%%%%%
% \paragraph{Chapter Include Files.}
%
% The include files are called |cdocsch1.tex| and |cdocsch2.tex|.
%
%\iffalse
%<*samplechap1|samplechap2>
%\fi

% Optional override for |\version| flag:
%    \begin{macrocode}
%%\providecommand{\version}{final}
%    \end{macrocode}

% Include the main document:
%    \begin{macrocode}
\input{childdoc.def}
\childdocof{cdocsamp}
%    \end{macrocode}

%\iffalse
%</samplechap1|samplechap2>
%\fi
%
%\iffalse
%<*samplechap1>
%\fi
% Some text for chapter 1:
%    \begin{macrocode}
\section{one}
some text in chapter one
%    \end{macrocode}

%\iffalse
%</samplechap1>
%\fi
% Some text for chapter 2:
%\iffalse
%<*samplechap2>
%\fi
%    \begin{macrocode}
\section{two}
more text in chapter two
%    \end{macrocode}

%\iffalse
%</samplechap2>
%\fi
%
% %%%%%%%%%%%%%%%%%%%%%%%%%%%%%%%%%%%%%%
% \paragraph{Part Include Files.}
%
% The include files are called |cdocspt3.tex| and |cdocspt4.tex|.
%
%\iffalse
%<*samplepart3|samplepart4>
%\fi

% Optional override for |\version| flag:
%    \begin{macrocode}
%%\providecommand{\version}{final}
%    \end{macrocode}

% Include the main document:
%    \begin{macrocode}
\input{childdoc.def}
\childdocby{cdocsamp}
%    \end{macrocode}

%\iffalse
%</samplepart3|samplepart4>
%\fi
%
%\iffalse
%<*samplepart3>
%\fi
% Some text for part 3:
%    \begin{macrocode}
some text in part three
%    \end{macrocode}

%\iffalse
%</samplepart3>
%\fi
% Some text for part 4:
%\iffalse
%<*samplepart4>
%\fi
%    \begin{macrocode}
more text in part four
%    \end{macrocode}

%\iffalse
%</samplepart4>
%\fi
%
% %%%%%%%%%%%%%%%%%%%%%%%%%%%%%%%%%%%%%%
% \paragraph{Forwarding for a Complete Draft.}
%
% The following forwarding file |cdocsdrf.tex|
% compiles the main document in draft mode:
%\iffalse
%<*sampledraft>
%\fi
%    \begin{macrocode}
\def\version{draft}
\input{childdoc.def}
\childdocforward{cdocsamp}
%    \end{macrocode}

%\iffalse
%</sampledraft>
%\fi
%
% %%%%%%%%%%%%%%%%%%%%%%%%%%%%%%%%%%%%%%
% \paragraph{Forwarding for Final Version of the Chapters.}
%
% The following forwarding files |cdocsfn1.tex| and |cdocsfn2.tex|
% (with identical content)
% compile the final versions of the child documents
% |cdocsch1.tex| and |cdocsch2.tex|, respectively:
%\iffalse
%<*samplefinal>
%\fi
%    \begin{macrocode}
\def\version{final}
\input{childdoc.def}
\childdocforwardprefix[cdocsamp]{cdocsfn}{cdocsch}
%    \end{macrocode}

%\iffalse
%</samplefinal>
%\fi
%
% %%%%%%%%%%%%%%%%%%%%%%%%%%%%%%%%%%%%%%
% \paragraph{Command Line Processing.}
%
% The following three command lines generate the output files
% |cdocscld|, |cdocscl1| and |cdocscl2|
% which should be identical to
% |cdocsdrf|, |cdocsch1| and |cdocsfn2|, respectively:
% \begin{center}
% \begin{tabular}{l}
% |latex -jobname cdocscld \|\\
% |  "\def\version{draft}\input{childdoc.def}\childdocforward{cdocsamp}"|\\
% |latex -jobname cdocscl1 \|\\
% |  "\input{childdoc.def}\childdocforward[cdocsamp]{cdocsch1}"|\\
% |latex -jobname cdocscl2 \|\\
% |  "\def\version{final}\input{childdoc.def}\childdocforward{cdocsch2}"|
% \end{tabular}
% \end{center}
% Note that the trailing backslash on each first line
% merely continues the input to the second line
% (for convenient cut ant paste).
% Furthermore, the command |latex| can be replaced by any
% of its alternative versions such as |pdflatex|.
%
% %%%%%%%%%%%%%%%%%%%%%%%%%%%%%%%%%%%%%%%%%%%%%%%%%%%%%%%%%%%%%%%%%%%%%%%%%%%%%%
% %%%%%%%%%%%%%%%%%%%%%%%%%%%%%%%%%%%%%%%%%%%%%%%%%%%%%%%%%%%%%%%%%%%%%%%%%%%%%%
% \section{Implementation}
%\iffalse
%<*package>
%\fi
%
% This section describes the definitions file |childdoc.def|.

% The definitions cannot be loaded using |\usepackage| or |\RequirePackage|
% which has a mechanism to prevent loading a style file more than once.
% When loading the definitions by means of |\input|
% multiple instances have to be prevented manually:
%\iffalse
%This code needs to be before the `\ProvidesFile' directive
%which is defined at the beginning of this file.
%Therefore it is also placed there and commented out here.
%</package>
%<*discard>
%\fi
%    \begin{macrocode}
\ifdefined\childdocmain\endinput\fi
%    \end{macrocode}
%\iffalse
%</discard>
%<*package>
%\fi
%
% \macro{\ifchilddoc}
% \macro{\ifchilddocmanual}
% The conditional |\ifchilddoc| tells whether a
% child (true) or main (false) document is being compiled.
% The conditional |\ifchilddocmanual| tells whether
% the |\includeonly| mechanism is used (false) or
% the selection of child files must be performed manually (true).
% The definitions initialise to false:
%    \begin{macrocode}
\newif\ifchilddoc
\newif\ifchilddocmanual
%    \end{macrocode}

% \macro{\childdocname}
% \macro{\childdocjob}
% The macro |\childdocname| stores the name of the main document
% to be compiled. The macro |\childdocjob| stores the name of
% the document on which the \LaTeX{} compiler was originally invoked.
% The content of |\jobname| cannot be compared
% to filenames specified in the source due to different catcodes.
% The following code rescans |\jobname|, stores the result
% in |\childdocname| and saves a copy in |\childdocjob|:
%    \begin{macrocode}
\edef\childdocname{\scantokens\expandafter{\jobname\noexpand}}
\let\childdocjob\childdocname
%    \end{macrocode}

% \macro{\childdocdisable}
% The macro |\childdocdisable| prevents the main file
% from being processed more than once.
% At this stage, the main document command |\childdocmain|
% is assumed to be called once again where it should do nothing.
% Any subsequent call to it should prevent
% a secondary processing of the main document
% It overwrites the forwarding commands
% |\childdocof| and |\childdocforward|
% with empty macros to prevent further inclusions of the main document:
%    \begin{macrocode}
\newcommand{\childdocdisable}
{
  \renewcommand{\childdocmain}[1]{\renewcommand{\childdocmain}[1]{\endinput}}
  \renewcommand{\childdocof}[1]{}
  \renewcommand{\childdocby}[2][]{}
  \renewcommand{\childdocforward}[2][]{}
  \renewcommand{\childdocdisable}{}
}
%    \end{macrocode}

% \macro{\childdocmain}
% The macro |\childdocmain| is to be called at the top of the main file
% with nothing or the main filename (without extension) as argument.
% First, it breaks loops.
% If the argument is not empty and does not match |\childdocname|
% (which is set by the first inclusion of |childdoc.def|),
% |\ifchilddoc| is set to true, |\includeonly| is applied to the child file
% and |\jobname| is set to the main file
% (for proper handling of |.aux| files):
%    \begin{macrocode}
\newcommand{\childdocmain}[1]
{
  \childdocdisable\childdocmain{}
  \if?#1?\else
    \begingroup
      \def\childdoctmp{#1}
      \ifx\childdoctmp\childdocname
        \def\childdoctmp{}
      \else
        \def\childdoctmp
        {
          \childdoctrue
          \includeonly{\childdocname}
          \def\childdocjob{#1}
          \def\jobname{#1}
        }
      \fi
      \expandafter
    \endgroup
    \childdoctmp
  \fi
}
%    \end{macrocode}

% \macro{\childdocof}
% The command |\childdocof| redirects
% compilation to the main file |#1|.
%    \begin{macrocode}
\newcommand{\childdocof}[1]
{
  \childdocdisable
  \childdoctrue
  \includeonly{\childdocname}
  \def\jobname{#1}
  \def\childdocjob{#1}
  \input{#1}
}
%    \end{macrocode}

% \macro{\childdocby}
% The command |\childdocby| ....
%    \begin{macrocode}
\newcommand{\childdocby}[2][]
{
  \childdocdisable
  \childdoctrue
  \childdocmanualtrue
  \if?#1?\else
    \def\jobname{#2}
  \fi
  \def\childdocjob{#2}
  \input{#2}
  \endinput
}
%    \end{macrocode}

% \macro{\childdocforward}
% The command |\childdocforward| redirects
% compilation to the main file or
% (if the optional argument is given) a child file.
% Parameters are set as if the main file
% or a child file starting with |\childdocof| was compiled.
% Then compilation is handed over to the main file:
%    \begin{macrocode}
\newcommand{\childdocforward}[2][]
{
  \begingroup
    \if?#1?
      \def\childdoctmp
      {
        \def\childdocname{#2}
        \def\childdocjob{#2}
        \def\jobname{#2}
        \input{#2}
        \endinput
      }
    \else
      \def\childdoctmp
      {
        \childdocdisable
        \def\childdocname{#2}
        \childdoctrue
        \includeonly{#2}
        \def\childdocjob{#1}
        \def\jobname{#1}
        \input{#1}
        \endinput
      }
    \fi
    \expandafter
  \endgroup
  \childdoctmp
}
%    \end{macrocode}

% \macro{\childdocforwardprefix}
% The command |\childdocforwardprefix| redirects
% compilation to the main or a child file by means of a pattern.
% The prefix |#1| in the current filename is replaced by |#2|
% and the suffix of the current filename is kept
% (it is assumed that the filename does not contain the substring `|~~~|'
% which is used as a delimiter).
% Compilation is handed over to the new file by |\childdocforward|:
%    \begin{macrocode}
\newcommand{\childdocforwardprefix}[3][]
{
  \begingroup
    \def\childdocextract #2##1~~~{\def\childdoctmp{\childdocforward[#1]{#3##1}}}
    \expandafter\childdocextract\childdocname~~~
    \expandafter
  \endgroup
  \childdoctmp
}
%    \end{macrocode}

% \macro{\childdoc}
% The deprecated macro |\childdoc| is a legacy version of |\childdocmain|:
%    \begin{macrocode}
\newcommand{\childdoc}{\childdocmain}
%    \end{macrocode}

% \macro{\childdocredirect}
% The deprecated macro |\childdocredirect| is a legacy version
% of |\childdocforward| and |\childdocforwardprefix|:
%    \begin{macrocode}
\newcommand{\childdocredirect}[2][]
{
  \begingroup
    \if?#1?
      \def\childdoctmp{\childdocforward{#2}}
    \else
      \def\childdoctmp{\childdocforwardprefix{#1}{#2}}
    \fi
    \expandafter
  \endgroup
  \childdoctmp
}
%    \end{macrocode}

%\iffalse
%</package>
%\fi
%
\endinput
|\\
|\childdocby{|\textit{main}|}|\\
\end{tabular}
\end{center}
%
The directive |\childdocby| is similar to |\childdocof|
described in \secref{sec:include},
but the subsequent selection of content must be done manually.
To that end, both |\ifchilddoc| and |\ifchilddocmanual|
will be true upon processing of a part,
and the name of the part is stored in |\childdocname|.
Note that |\jobname| will be set to the filename of the current part
so that each part receives an individual |.aux| file
that does not interfere with the |.aux| file(s) of the main document.
This behaviour can be altered by the alternative form
|\childdocby[*]{|\textit{main}|}| (with a non-empty optional argument)
which uses the |.aux| file of the main document
by setting |\jobname| to \textit{main}.

%%%%%%%%%%%%%%%%%%%%%%%%%%%%%%%%%%%%%%%%%%%%%%%%%%%%%%%%%%%%%%%%%%%%%%%%%%%%%%%%
\subsection{Driver Development}
\label{sec:driver}

The \textsf{childdoc} mechanism can also be use for the development
of definition files such as \LaTeX{} styles or classes.
This case differs from the above setup with multiple parts
included by |\include| in that no |\includeonly| should be invoked.
This can be achieved by starting the include file
(before |\ProvidesPackage|) with:
%
\begin{center}
\begin{tabular}{l}
|% \iffalse
%
% childdoc.dtx Copyright (C) 2017-2018 Niklas Beisert
%
% This work may be distributed and/or modified under the
% conditions of the LaTeX Project Public License, either version 1.3
% of this license or (at your option) any later version.
% The latest version of this license is in
%   http://www.latex-project.org/lppl.txt
% and version 1.3 or later is part of all distributions of LaTeX
% version 2005/12/01 or later.
%
% This work has the LPPL maintenance status `maintained'.
%
% The Current Maintainer of this work is Niklas Beisert.
%
% This work consists of the files childdoc.dtx and childdoc.ins
% and the derived files childdoc.def and cdocsamp.tex with
% cdocsch1.tex, cdocsch2.tex, cdocsdrf.tex, cdocsfn1.tex, cdocsfn2.tex.
%
%<package>\ifdefined\childdocmain\endinput\fi
%<package>\ProvidesFile{childdoc.def}[2018/12/30 v2.0 child document driver]
%<samplemain>\ProvidesFile{cdocsamp.tex}[2018/12/30 v2.0 sample for childdoc]
%<*driver>
%\ProvidesFile{childdoc.drv}[2018/12/30 v2.0 childdoc reference manual file]
\PassOptionsToClass{10pt,a4paper}{article}
\documentclass{ltxdoc}

\usepackage[margin=35mm]{geometry}
\usepackage{hyperref}
\usepackage{hyperxmp}
\usepackage[usenames]{color}

\hypersetup{colorlinks=true}
\hypersetup{pdfstartview=FitH}
\hypersetup{pdfpagemode=UseNone}
\hypersetup{pdfsource={}}
\hypersetup{pdflang={en-UK}}
\hypersetup{pdfcopyright={Copyright 2017-2018 Niklas Beisert.
  This work may be distributed and/or modified under the
  conditions of the LaTeX Project Public License, either version 1.3
  of this license or (at your option) any later version.}}
\hypersetup{pdflicenseurl={http://www.latex-project.org/lppl.txt}}
\hypersetup{pdfcontactaddress={ETH Zurich, ITP, HIT K,
  Wolfgang-Pauli-Strasse 27}}
\hypersetup{pdfcontactpostcode={8093}}
\hypersetup{pdfcontactcity={Zurich}}
\hypersetup{pdfcontactcountry={Switzerland}}
\hypersetup{pdfcontactemail={nbeisert@itp.phys.ethz.ch}}
\hypersetup{pdfcontacturl={http://people.phys.ethz.ch/\xmptilde nbeisert/}}

\newcommand{\secref}[1]{\hyperref[#1]{section \ref*{#1}}}

\parskip1ex
\parindent0pt
\let\olditemize\itemize
\def\itemize{\olditemize\parskip0pt}

\begin{document}

\title{The \textsf{childdoc} Package}
\hypersetup{pdftitle={The childdoc Package}}
\author{Niklas Beisert\\[2ex]
  Institut f\"ur Theoretische Physik\\
  Eidgen\"ossische Technische Hochschule Z\"urich\\
  Wolfgang-Pauli-Strasse 27, 8093 Z\"urich, Switzerland\\[1ex]
  \href{mailto:nbeisert@itp.phys.ethz.ch}
  {\texttt{nbeisert@itp.phys.ethz.ch}}}
\hypersetup{pdfauthor={Niklas Beisert}}
\hypersetup{pdfsubject={Manual for the LaTeX2e Package childdoc}}
\date{30 December 2018, \textsf{v2.0}}
\maketitle

\begin{abstract}\noindent
\textsf{childdoc} is a \LaTeXe{} package
that enables the direct compilation
of document sections included by |\include|
to individual files.
\end{abstract}

\begingroup
\parskip0ex
\tableofcontents
\endgroup

%%%%%%%%%%%%%%%%%%%%%%%%%%%%%%%%%%%%%%%%%%%%%%%%%%%%%%%%%%%%%%%%%%%%%%%%%%%%%%%%
%%%%%%%%%%%%%%%%%%%%%%%%%%%%%%%%%%%%%%%%%%%%%%%%%%%%%%%%%%%%%%%%%%%%%%%%%%%%%%%%
\section{Introduction}

\LaTeX{} provides a mechanism to structure a large document (such as a book)
into a main file and several child files (containing the chapters)
using the |\include| command.
This mechanism is beneficial for documents
which span hundreds of pages in order to
make the source file(s) more manageable.
Moreover, compilation can be restricted to
selected child files by means of the |\includeonly| command.
The latter feature can be used to reduce the compilation time while editing
(this was significantly more useful in the earlier days of \LaTeX{})
or to generate a smaller document which is easier to navigate.
Another application of |\includeonly| is to generate
documents consisting of selected parts of the complete document.

However, there are a few drawbacks of the plain |\include| mechanism:
\begin{itemize}
\item
The child files cannot be compiled on their own,
they can only be compiled via the main file.
A naive editing environment
(such as a text editor with an option
to have the current file processed by \LaTeX)
may require one to switch to the main file before compiling;
attempting to compile the child file produces errors.
\item
The main file must be modified (each time)
to adjust the |\includeonly| command
to the present needs. This easily leaves the main file in a messy state.
\item
The generated document will always carry the filename
of the main document. This is inconvenient if
several child files are to be compiled and
to be kept for distribution.
\end{itemize}

The present package provides a simple interface
to make child files individually compilable by \LaTeX{}.
Compiling a child file then has the same effect as compiling
the main file with an |\includeonly| command
to select the appropriate child.
Moreover the generated document will carry the name of the child
rather than the main file.
This resolves all three above issues.

This feature is meant to make the editing of books,
thesis documents and lecture notes somewhat more convenient.
However, the package can also be used efficiently for
composing a series of documents (such as exercise sheets)
which are typically distributed individually.
It then assists the author in generating the individual documents
(potentially in different versions)
as well as a document containing the collected series.
Another application is in developing style files
or other kinds of included material
where compilation of the style file could redirect
to a sample or test file.

%%%%%%%%%%%%%%%%%%%%%%%%%%%%%%%%%%%%%%%%%%%%%%%%%%%%%%%%%%%%%%%%%%%%%%%%%%%%%%%%
%%%%%%%%%%%%%%%%%%%%%%%%%%%%%%%%%%%%%%%%%%%%%%%%%%%%%%%%%%%%%%%%%%%%%%%%%%%%%%%%
\section{Usage}

First of all, the package \textsf{childdoc} is \emph{not} a standard
\LaTeXe{} |.sty| style file! Therefore it needs to be invoked in
a non-standard way.

%%%%%%%%%%%%%%%%%%%%%%%%%%%%%%%%%%%%%%%%%%%%%%%%%%%%%%%%%%%%%%%%%%%%%%%%%%%%%%%%
\subsection{Included Files}
\label{sec:include}

%%%%%%%%%%%%%%%%%%%%%%%%%%%%%%%%%%%%%%%%
\DescribeMacro{\childdocmain}
To use the package, add the commands
\begin{center}
\begin{tabular}{l}
|\input{childdoc.def}|\\
|\childdocmain{}|\\
\end{tabular}
\end{center}
at the very top of the main \LaTeX{} file,
in particular \emph{before} the |\documentclass| statement!
The argument of |\childdocmain| should be left empty
(but it must be present).

%%%%%%%%%%%%%%%%%%%%%%%%%%%%%%%%%%%%%%%%
\DescribeMacro{\childdocof}
Furthermore, add the commands
\begin{center}
\begin{tabular}{l}
|\input{childdoc.def}|\\
|\childdocof{|\textit{main}|}|\\
\end{tabular}
\end{center}
at the top of every child file \textit{child}
which is included by |\include{|\textit{child}|}|
from within the main file
(or at least for those files to be compiled individually).
The argument \textit{main} must be the filename of the main file.

There are a couple of
considerations in setting up the main and child documents:

%%%%%%%%%%%%%%%%%%%%%%%%%%%%%%%%%%%%%%%%
\paragraph{Restrictions.}

Please note the following restrictions:
\begin{itemize}
\item
|\childdocmain| must be called with one argument \textit{main}
to ensure compatibility with earlier version of the package.
It must either be empty (|\childdocmain{}|)
or precisely match the filename of the main file in which it is specified.
See \secref{sec:detection} for further information.
\item
The filename \textit{main} must be specified without the |.tex| extension.
\item
The filename \textit{main} is case sensitive
(even in case-insensitive file systems)
due to internal string comparison.
\item
The argument \textit{main} should be fully expanded, it cannot be a macro.
\item
Subdirectories and special characters should be avoided in filenames.
\item
The command |\childdocmain{|\textit{main}|}| must be followed by a whitespace.
It should not be followed immediately by another command
or by a comment mark `|%|'.
This is because the \TeX{} parser reads the token immediately following
the argument of |\childdocmain| and puts it
at the beginning of every child section;
however, a white\-space is ignored.
\end{itemize}

%%%%%%%%%%%%%%%%%%%%%%%%%%%%%%%%%%%%%%%%
\paragraph{Content of Main File.}

It is advisable to place all content in the child files included by |\include|.
Any output contained in the main file will appear in all child documents
unless suppressed manually;
it cannot be suppressed automatically by the |\includeonly| directive
and thus should normally be avoided.
A method to include some content in the main file
by means of conditional processing is described in \secref{sec:conditional}.

%%%%%%%%%%%%%%%%%%%%%%%%%%%%%%%%%%%%%%%%
\paragraph{Page Numbering.}

When only a part of the document is compiled,
the appropriate numbering of pages
(as well as other status parameters)
is determined from the |.aux| files.
The latter contain information from previous passes.
However this information needs to propagate through
all intermediate child documents.
Therefore the page numbering in child documents may well
be inconsistent until the complete document is compiled at least once.

A useful (if unconventional) way to always ensure a consistent
page numbering is to restart the numbering in each child document
and denote the pages by `\textit{child}|.|\textit{page}'
where \textit{child} represents the chapter/section number of the child file.
This can be achieved by the command
|\numberwithin{page}{|\textit{child}|}|
of the \textsf{amsmath} package
where \textit{child} can be |chapter| or |section|
depending on the chosen structuring.
Alternatively, one can modify the macro |\thepage| appropriately
and reset the counter |page| at the start of each child file.

%%%%%%%%%%%%%%%%%%%%%%%%%%%%%%%%%%%%%%%%%%%%%%%%%%%%%%%%%%%%%%%%%%%%%%%%%%%%%%%%
\subsection{Conditional Processing}
\label{sec:conditional}

The package provides a mechanism to compile different versions
of a document. To customise the versions further some conditional processing
can come in handy to distinguish which version is being compiled.
The package provides two macros to describe the compilation context:

%%%%%%%%%%%%%%%%%%%%%%%%%%%%%%%%%%%%%%%%
\DescribeMacro{\ifchilddoc}
The conditional |\ifchilddoc| distinguishes between the compilation of
child documents and the main document:
%
\begin{center}
|\ifchilddoc |\textit{child-code}| |[|\||else |\textit{main-code}]| \||fi|
\end{center}

%%%%%%%%%%%%%%%%%%%%%%%%%%%%%%%%%%%%%%%%
\DescribeMacro{\childdocname}
\DescribeMacro{\childdocjob}
The macro |\childdocname| contains the filename (without extension)
of the main or child file being processed.
Note that |\childdocjob| will always contain the name of the main file.

%%%%%%%%%%%%%%%%%%%%%%%%%%%%%%%%%%%%%%%%
\paragraph{Title Page.}

Conditional processing can be used to include a title or banner page
in the main document when proper precautions are taken.
Importantly, the code in the main file should ensure that the page counter
(as well as other status parameters which are stored in the |.aux| files)
takes the same value after the conditional processing.
Otherwise the page numbers may take divergent values
depending on which part is compiled.

For example, a title page could be declared by:
%
\begin{center}
\begin{tabular}{l}
|\ifchilddoc\||else|\\
|\addtocounter{page}{-1}|\\
\textit{code for title page}\\
|\newpage|\\
|\||fi|
\end{tabular}
\end{center}
%
A banner page for the child documents can be generated by:
%
\begin{center}
\begin{tabular}{l}
|\ifchilddoc|\\
|\addtocounter{page}{-1}|\\
\textit{code for banner page}\\
|\newpage|\\
|\||fi|
\end{tabular}
\end{center}
%
Here one could write a message such as:
\begin{center}
|This is the part \childdocname{} of \childdocjob{}.|
\end{center}

%%%%%%%%%%%%%%%%%%%%%%%%%%%%%%%%%%%%%%%%%%%%%%%%%%%%%%%%%%%%%%%%%%%%%%%%%%%%%%%%
\subsection{Flags}
\label{sec:flags}

The package makes it easy to generate different versions
of the main or child documents.
To this end compilation flags can be defined
and assigned different default values.
They will be particularly useful in conjunction
with the forwarding mechanism described in \secref{sec:forward}.

For example, it may be useful to have a flag |\version|
which can be set to |draft| or |final|.
The document source will contain some conditional code
depending on the value of |\version|.
Suppose further, the flag should default to |final| for the main file
and to |draft| for child files
which is a natural assignment for editing the document.
This is achieved by placing the following code
in the preamble of the main document
(below the |\childdocmain| directive):
%
\begin{center}
\begin{tabular}{l}
|\ifchilddoc|\\
|\providecommand{\version}{draft}|\\
|\||else|\\
|\providecommand{\version}{final}|\\
|\||fi|
\end{tabular}
\end{center}
%
The definition by |\providecommand| makes sure
that previous definitions are not overwritten.
Further statements |\providecommand{\version}{...}|
can thus be added before the above code to override it.

For the main file, one might add a line
(between |\childdocmain| and the above block)
%
\begin{center}
|%\ifchilddoc\||else\providecommand{\version}{draft}\||fi|
\end{center}
%
which can be uncommented to produce a draft version.
Likewise one can add a line to the very top of a child file
(above the |\childdocof{|\textit{main}|}| directive)
%
\begin{center}
|%\providecommand{\version}{final}|
\end{center}
%
which can be uncommented to produce the final version of this child document.

%%%%%%%%%%%%%%%%%%%%%%%%%%%%%%%%%%%%%%%%%%%%%%%%%%%%%%%%%%%%%%%%%%%%%%%%%%%%%%%%
\subsection{Forwarding}
\label{sec:forward}

Different versions of the main or child documents
using compilation flags as described in \secref{sec:flags}
can be (permanently) stored in different files
for convenient compilation, viewing and distribution.
To this end, the package defines a command
to pass on compilation to a different file:

%%%%%%%%%%%%%%%%%%%%%%%%%%%%%%%%%%%%%%%%
\DescribeMacro{\childdocforward}
The command |\childdocforward| redirects processing to
another source file:
%
\begin{center}
\begin{tabular}{l}
|\input{childdoc.def}|\\
|\childdocforward[|\textit{main}|]{|\textit{dest}|}|\\
\end{tabular}
\end{center}
%
The argument \textit{dest} is the destination file
(without extension).
It should be the main file or one of the child files.
Note that further \textsf{childdoc} directives
such as |\childdocof| and |\childdocforward|
in the indicated file will be processed in this form.
The optional argument \textit{main}
passes on directly to the main file \textit{main}
while pretending to compile the child \textit{dest}.
This form behaves as if \textit{dest}
issues |\childdocof{|\textit{main}|}| right away,
and no further \textsf{childdoc} directives will be processed.

%%%%%%%%%%%%%%%%%%%%%%%%%%%%%%%%%%%%%%%%
\DescribeMacro{\...prefix}
In the alternative form |\childdocforwardprefix|,
%
\begin{center}
\begin{tabular}{l}
|\input{childdoc.def}|\\
|\childdocforwardprefix[|\textit{main}|]{|\textit{prefix}|}{|\textit{dest}|}|
\end{tabular}
\end{center}
%
the destination file is determined by a pattern
depending on the current file:
To make this work, the current file must be called
`{\textit{prefix}\hspace{0.2em}\textit{suffix}}'
with \textit{prefix} matching precisely the argument.
Processing is then passed on to the file
`{\textit{dest}\hspace{0.2em}\textit{suffix}}'.
Surely, the same effect is achieved by
directly specifying the
argument `{\textit{dest}\hspace{0.2em}\textit{suffix}}'
in the first form.
However, that requires to set up a different file
for each child. With the alternative form of the command
all these files can have exactly the same content
which simplifies setting them up and maintaining them.

For example, the following file |draft.tex|
with a compilation flag |\version| as described in \secref{sec:flags}
compiles the main document as a draft:
%
\begin{center}
\begin{tabular}{l}
|\def\version{draft}|\\
|\input{childdoc.def}|\\
|\childdocforward{|\textit{main}|}|
\end{tabular}
\end{center}
%
Likewise, the following files |final|\textit{nn}|.tex|
compile the final version of the child document
|child|\textit{nn}|.tex|:
%
\begin{center}
\begin{tabular}{l}
|\def\version{final}|\\
|\input{childdoc.def}|\\
|\childdocforwardprefix{final}{child}|
\end{tabular}
\end{center}
%

Note that when several versions of a main file and/or of each child file
are to be generated, it may be convenient to set up a |Makefile| or
shell script to automatise the process.

%%%%%%%%%%%%%%%%%%%%%%%%%%%%%%%%%%%%%%%%%%%%%%%%%%%%%%%%%%%%%%%%%%%%%%%%%%%%%%%%
\subsection{Command Line Processing}
\label{sec:commandline}

The effect of redirection files can also be achieved by invoking
the \LaTeX{} compiler with a more elaborate command line.
Most conveniently this should be done as part
of a shell script or a |Makefile|.

When using \textsf{childdoc} in the main file, the following
command lines effectively perform a redirection
(note that depending on the shell being used,
backslashes may have to be doubled: `|\|' $\to$ `|\\|'):
%
\begin{center}
|... -jobname "|\textit{target}|" |\\|"|[\textit{flags}]%
|\input{childdoc.def}\childdocforward[|\textit{main}|]{|\textit{dest}|}"|
\end{center}
%
Here \textit{target} is the name of the output file,
\textit{main} is the name of the main file
and \textit{dest} is the name of the main or child file to be processed
(all filenames without extensions).
The optional argument \textit{main} can be omitted
if \textit{main} matches \textit{dest}.
Optionally, compilation \textit{flags} can be defined via |\def| commands.
This command line makes the \TeX{} engine believe
it is compiling the file \textit{target}
whose content is specified as the latter parameter.
The provided code then forwards the processing to
\textit{main} or \textit{dest} as described in \secref{sec:forward}.

%%%%%%%%%%%%%%%%%%%%%%%%%%%%%%%%%%%%%%%%%%%%%%%%%%%%%%%%%%%%%%%%%%%%%%%%%%%%%%%%
\subsection{Include by Input}
\label{sec:input}

Including child documents by |\include| has some restrictions by design.
Most notably, the content of a child document always occupies
its own set of pages; pages cannot be shared between child documents.
Usually, this behaviour makes perfect sense
because each child document contain an essential part of the document.
However, in some situations it may be desirable to compose
a document from a collection of parts
without having mandatory page breaks between then.
For this case, the package
provides a mechanism to include parts
by |\input| which can also be processed individually.
However, by construction this mechanism
requires manual handling of the content to be output.

%%%%%%%%%%%%%%%%%%%%%%%%%%%%%%%%%%%%%%%%
\DescribeMacro{\ifchilddocmanual}
The main file should be prepared as usual, see \secref{sec:include}.
However, the document body must make a distinction
between processing of an individual part and of the main document, e.g.:
%
\begin{center}
\begin{tabular}{l}
|\ifchilddocmanual|\\
|\input{\childdocname}|\\
|\||else|\\
\textit{document body with }|\input{|\textit{part}|}|\\
|\||fi|
\end{tabular}
\end{center}
%
The conditional |\ifchilddocmanual| is true whenever
a part to be included by |\input| is being compiled,
and the name of the part is stored in |\childdocname|.

%%%%%%%%%%%%%%%%%%%%%%%%%%%%%%%%%%%%%%%%
\DescribeMacro{\childdocby}
Each part to be included by |\input| should start with:
%
\begin{center}
\begin{tabular}{l}
|\input{childdoc.def}|\\
|\childdocby{|\textit{main}|}|\\
\end{tabular}
\end{center}
%
The directive |\childdocby| is similar to |\childdocof|
described in \secref{sec:include},
but the subsequent selection of content must be done manually.
To that end, both |\ifchilddoc| and |\ifchilddocmanual|
will be true upon processing of a part,
and the name of the part is stored in |\childdocname|.
Note that |\jobname| will be set to the filename of the current part
so that each part receives an individual |.aux| file
that does not interfere with the |.aux| file(s) of the main document.
This behaviour can be altered by the alternative form
|\childdocby[*]{|\textit{main}|}| (with a non-empty optional argument)
which uses the |.aux| file of the main document
by setting |\jobname| to \textit{main}.

%%%%%%%%%%%%%%%%%%%%%%%%%%%%%%%%%%%%%%%%%%%%%%%%%%%%%%%%%%%%%%%%%%%%%%%%%%%%%%%%
\subsection{Driver Development}
\label{sec:driver}

The \textsf{childdoc} mechanism can also be use for the development
of definition files such as \LaTeX{} styles or classes.
This case differs from the above setup with multiple parts
included by |\include| in that no |\includeonly| should be invoked.
This can be achieved by starting the include file
(before |\ProvidesPackage|) with:
%
\begin{center}
\begin{tabular}{l}
|\input{childdoc.def}|\\
|\childdocforward{|\textit{main}|}|\\
\end{tabular}
\end{center}
%
or alternatively with:
%
\begin{center}
\begin{tabular}{l}
|\input{childdoc.def}|\\
|\childdocby{|\textit{main}|}|\\
\end{tabular}
\end{center}
%
Both forms have slightly different effects as described above.
The main file is prepared as usual, see \secref{sec:include}.

%%%%%%%%%%%%%%%%%%%%%%%%%%%%%%%%%%%%%%%%%%%%%%%%%%%%%%%%%%%%%%%%%%%%%%%%%%%%%%%%
\subsection{Legacy Detection}
\label{sec:detection}

The directive |\childdocmain| in the main file can detect
whether the complete document or merely a child is to be compiled
even without using the directive |\childdocof|.
This method is deprecated because it is less robust
and there is no compelling reason to use it;
it is merely provided for backward compatibility
and it may be removed in future versions.

If the detection mechanism is to be used,
it is mandatory to correctly specify
the filename of the main file as the argument of |\childdocmain|:
%
\begin{center}
\begin{tabular}{l}
|\input{childdoc.def}|\\
|\childdocmain{|\textit{main}|}|\\
\end{tabular}
\end{center}
%
If |\jobname| does not match the argument \textit{main} of |\childdocmain|,
it is assumed that |\jobname| points to the child file to be compiled.
When using |\childdocmain| with the main file specified as argument,
it suffices to start a child file
with just |\input{|\textit{main}|}|
without loading of the package and using |\childdocof|.
If instead all processing is done
with the appropriate \textsf{childdoc} directives,
the argument of \textit{main} of |\childdocmain| can be empty.

An alternative version of the command line processing described
in \secref{sec:commandline} using the detection mechanism reads:
%
\begin{center}
|... -jobname "|\textit{target}|" "|[\textit{flags}]%
[|\def\jobname{|\textit{dest}|}|]|\input{|\textit{main}|}"|
\end{center}

%%%%%%%%%%%%%%%%%%%%%%%%%%%%%%%%%%%%%%%%%%%%%%%%%%%%%%%%%%%%%%%%%%%%%%%%%%%%%%%%
\subsection{Manual Code}
\label{sec:manual}

In case one cannot be certain whether the definitions file |childdoc.def|
is installed on the target \TeX{} distribution
and one prefers not to ship it,
it is conceivable to paste a few relevant commands into the sources.

To that end, drop all statements |\input{childdoc.def}|
and perform the replacements as outlined below.
Instead of |\childdocmain{|\textit{main}|}| add the following code
to the top of the main file:
%
\begin{center}
\begin{tabular}{l}
|\||ifdefined\childdocname\endinput\||fi\newif\ifchilddoc|\\
|\edef\childdocname{\scantokens\expandafter{\jobname\noexpand}}|\\
|\def\childdocmain{|\textit{main}|}\||ifx\childdocmain\childdocname\||else|\\
|\childdoctrue\includeonly{\childdocname}\let\jobname\childdocmain\||fi|\\
\end{tabular}
\end{center}
%
Instead of |\childdocof{|\textit{main}|}| just include the main file
at the top of each child file:
%
\begin{center}
|\input{|\textit{main}|}|
\end{center}
%
A simple redirection |\childdocforward{|\textit{dest}|}| is achieved by:
%
\begin{center}
|\def\jobname{|\textit{dest}|}\input{\jobname}|
\end{center}
%
The redirection with prefix
|\childdocforwardprefix[|\textit{prefix}|]{|\textit{dest}|}|
is accomplished by:
%
\begin{center}
\begin{tabular}{l}
|{\edef\jobname{\scantokens\expandafter{\jobname\noexpand}}|\\
|\def\redirectjob |\textit{prefix}|#1~~~{\gdef\jobname{|\textit{dest}|#1}}|\\
|\expandafter\redirectjob\jobname~~~}\input{\jobname}|
\end{tabular}
\end{center}

In an alternative approach,
child documents can be compiled by a specific command line
without additional code or specific definitions:
%
\begin{center}
|... -jobname "|\textit{target}|" "|[\textit{flags}]%
|\includeonly{|\textit{dest}|}\input{|\textit{main}|}"|
\end{center}
%

%%%%%%%%%%%%%%%%%%%%%%%%%%%%%%%%%%%%%%%%%%%%%%%%%%%%%%%%%%%%%%%%%%%%%%%%%%%%%%%%
%%%%%%%%%%%%%%%%%%%%%%%%%%%%%%%%%%%%%%%%%%%%%%%%%%%%%%%%%%%%%%%%%%%%%%%%%%%%%%%%
\section{Information}

%%%%%%%%%%%%%%%%%%%%%%%%%%%%%%%%%%%%%%%%%%%%%%%%%%%%%%%%%%%%%%%%%%%%%%%%%%%%%%%%
\subsection{Copyright}

Copyright \copyright{} 2017--2018 Niklas Beisert

This work may be distributed and/or modified under the
conditions of the \LaTeX{} Project Public License, either version 1.3
of this license or (at your option) any later version.
The latest version of this license is in
  \url{http://www.latex-project.org/lppl.txt}
and version 1.3 or later is part of all distributions of \LaTeX{}
version 2005/12/01 or later.

This work has the LPPL maintenance status `maintained'.

The Current Maintainer of this work is Niklas Beisert.

This work consists of the files |README.txt|, |childdoc.ins| and |childdoc.dtx|
as well as the derived files |childdoc.def|, |cdocsamp.tex|
with |cdocsch1.tex|, |cdocsch2.tex|, |cdocspt3.tex|, |cdocspt4.tex|,
|cdocsdrf.tex|, |cdocsfn1.tex|, |cdocsfn2.tex|
as well as |childdoc.pdf|.

%%%%%%%%%%%%%%%%%%%%%%%%%%%%%%%%%%%%%%%%%%%%%%%%%%%%%%%%%%%%%%%%%%%%%%%%%%%%%%%%
\subsection{Files and Installation}

The package consists of the files:
%
\begin{center}
\begin{tabular}{ll}
    |README.txt|   & readme file \\
    |childdoc.ins| & installation file \\
    |childdoc.dtx| & source file \\
    |childdoc.def| & definition file \\
    |cdocsamp.tex| & sample main file \\
    |cdocsch1.tex| & sample include file \\
    |cdocsch2.tex| & sample include file \\
    |cdocspt3.tex| & sample part file \\
    |cdocspt4.tex| & sample part file \\
    |cdocsdrf.tex| & sample redirection file \\
    |cdocsfn1.tex| & sample redirection file \\
    |cdocsfn2.tex| & sample redirection file \\
    |childdoc.pdf| & manual
\end{tabular}
\end{center}
%
The distribution consists of the files
|README.txt|, |childdoc.ins| and |childdoc.dtx|.
%
\begin{itemize}
\item
Run (pdf)\LaTeX{} on |childdoc.dtx|
to compile the manual |childdoc.pdf| (this file).
\item
Run \LaTeX{} on |childdoc.ins| to create the definitions file |childdoc.def|
and the sample |cdocsamp.tex| with include files
|cdocsch1.tex|, |cdocsch2.tex|, |cdocspt3.tex|, |cdocspt4.tex|,
|cdocsdrf.tex|, |cdocsfn1.tex|, |cdocsfn2.tex|.
Then copy the file |childdoc.def| to an appropriate directory of your \LaTeX{}
distribution, e.g.\ \textit{texmf-root}|/tex/latex/childdoc|.
\end{itemize}

%%%%%%%%%%%%%%%%%%%%%%%%%%%%%%%%%%%%%%%%%%%%%%%%%%%%%%%%%%%%%%%%%%%%%%%%%%%%%%%%
\subsection{Related CTAN Packages}

There are several other packages which offer a similar functionality:
%
\begin{itemize}
\item
The packages
\href{http://ctan.org/pkg/docmute}{\textsf{docmute}},
\href{http://ctan.org/pkg/includex}{\textsf{includex}} and
\href{http://ctan.org/pkg/standalone}{\textsf{standalone}}
provide commands to include only the document body of
a child file thus allowing both files to be compiled individually.
\item
The packages \href{http://ctan.org/pkg/subdocs}{\textsf{subdocs}}
and \href{http://ctan.org/pkg/subfiles}{\textsf{subfiles}}
provide structures in which the main and child documents can be
encapsulated and allowing them to be compiled individually.
The inclusion mechanism is different from the conventional |\include|.
\item
The package \href{http://ctan.org/pkg/combine}{\textsf{combine}}
is an elaborate solution to combine several documents into one.
\end{itemize}
%
See also the CTAN topic \href{http://ctan.org/topic/subdocs}{\textsf{subdocs}}
for further related packages.
The present package differs from the above solutions in that
a document structure constructed with the conventional |\include| mechanism
just needs two extra commands at the top of every file
such that all constituent files can be compiled individually.

%%%%%%%%%%%%%%%%%%%%%%%%%%%%%%%%%%%%%%%%%%%%%%%%%%%%%%%%%%%%%%%%%%%%%%%%%%%%%%%%
%\subsection{Feature Suggestions}
%
%The following is a list of features which may be useful for future
%versions of this package:
%%
%\begin{itemize}
%\item
%\ldots
%\end{itemize}

%%%%%%%%%%%%%%%%%%%%%%%%%%%%%%%%%%%%%%%%%%%%%%%%%%%%%%%%%%%%%%%%%%%%%%%%%%%%%%%%
\subsection{Revision History}

%%%%%%%%%%%%%%%%%%%%%%%%%%%%%%%%%%%%%%%%
\paragraph{v2.0:} 2018/12/30

\begin{itemize}
\item
immediate forward processing
\item
added |\childdocby| mechanism
\item
manual restructured
\end{itemize}

%%%%%%%%%%%%%%%%%%%%%%%%%%%%%%%%%%%%%%%%
\paragraph{v1.6:} 2018/01/17

\begin{itemize}
\item
application for development of include files
\item
corrections to manual
\end{itemize}

%%%%%%%%%%%%%%%%%%%%%%%%%%%%%%%%%%%%%%%%
\paragraph{v1.5:} 2017/05/21

\begin{itemize}
\item
more complete structuring introduced
\item
|\childdocof| introduced
\item
|\childdoc| renamed to |\childdocmain|
\item
|\childredirect| renamed to |\childdocforward| and |\childdocforwardprefix|
and functionality expanded
\end{itemize}

%%%%%%%%%%%%%%%%%%%%%%%%%%%%%%%%%%%%%%%%
\paragraph{v1.0:} 2017/04/27

\begin{itemize}
\item
manual and install package
\item
first version published on CTAN
\end{itemize}

%%%%%%%%%%%%%%%%%%%%%%%%%%%%%%%%%%%%%%%%
\paragraph{v0.6:} 2017/04/26

\begin{itemize}
\item
redirection mechanism added
\end{itemize}

%%%%%%%%%%%%%%%%%%%%%%%%%%%%%%%%%%%%%%%%
\paragraph{v0.5:} 2017/04/26

\begin{itemize}
\item
functionality in definition file
\end{itemize}


%%%%%%%%%%%%%%%%%%%%%%%%%%%%%%%%%%%%%%%%%%%%%%%%%%%%%%%%%%%%%%%%%%%%%%%%%%%%%%%%
%%%%%%%%%%%%%%%%%%%%%%%%%%%%%%%%%%%%%%%%%%%%%%%%%%%%%%%%%%%%%%%%%%%%%%%%%%%%%%%%
%%%%%%%%%%%%%%%%%%%%%%%%%%%%%%%%%%%%%%%%%%%%%%%%%%%%%%%%%%%%%%%%%%%%%%%%%%%%%%%%
\appendix

\settowidth\MacroIndent{\rmfamily\scriptsize 000\ }

 \DocInput{childdoc.dtx}

\end{document}
%</driver>
% \fi
%
% %%%%%%%%%%%%%%%%%%%%%%%%%%%%%%%%%%%%%%%%%%%%%%%%%%%%%%%%%%%%%%%%%%%%%%%%%%%%%%
% %%%%%%%%%%%%%%%%%%%%%%%%%%%%%%%%%%%%%%%%%%%%%%%%%%%%%%%%%%%%%%%%%%%%%%%%%%%%%%
% \section{Sample}
%\iffalse
%<*samplemain>
%\fi
%
% The following presents a sample document
% with two chapters, two parts, a title page,
% a compile flag as well as three forwarding files to set the flag.
% It consists of eight |.tex| files:
% \begin{center}
% \begin{tabular}{ll}
% |cdocsamp.tex|&main file\\
% |cdocsch1.tex|&include file for chapter 1\\
% |cdocsch2.tex|&include file for chapter 2\\
% |cdocspt3.tex|&include file for part 3\\
% |cdocspt4.tex|&include file for part 4\\
% |cdocsdrf.tex|&forwarding file for main file in draft mode\\
% |cdocsfi1.tex|&forwarding file for final version of chapter 1\\
% |cdocsfi2.tex|&forwarding file for final version of chapter 2\\
% \end{tabular}
% \end{center}
% Each of the eight files can be compiled directly by the \LaTeX{} compiler.
%
% %%%%%%%%%%%%%%%%%%%%%%%%%%%%%%%%%%%%%%
% \paragraph{Main File.}
%
% The main file is called |cdocsamp.tex|.
%
% Load the \textsf{childdoc} definitions and
% declare the filename for the main document:
%    \begin{macrocode}
\input{childdoc.def}
\childdocmain{}
%    \end{macrocode}

% Optional override for |\version| flag:
%    \begin{macrocode}
%%\ifchilddoc\else\providecommand{\version}{draft}\fi
%    \end{macrocode}

% Define the default values for the |\version| flag
% (|final| for the main file and |draft| for childs):
%    \begin{macrocode}
\ifchilddoc
\providecommand{\version}{draft}
\else
\providecommand{\version}{final}
\fi
%    \end{macrocode}

% Load the standard document class:
%    \begin{macrocode}
\documentclass[12pt]{article}
%    \end{macrocode}

% Start the document body:
%    \begin{macrocode}
\begin{document}
%    \end{macrocode}

% Declare a title page.
% Print title, part of document being processed and version flag:
%    \begin{macrocode}
\addtocounter{page}{-1}
\begin{center}
{\LARGE\bfseries{}childdoc example\par}
\vspace{1cm}
\ifchilddoc
\ifchilddocmanual part\else chapter\fi:
`\childdocname' of `\childdocjob'\par
\else
main document: `\childdocjob'\par
\fi
version: \version\par
\end{center}
\newpage
%    \end{macrocode}

% Manually include selected file,
% otherwise process as usual:
%    \begin{macrocode}
\ifchilddocmanual
\section*{part `\childdocname'}
\input{\childdocname}
\else
%    \end{macrocode}

% Include the two chapters:
%    \begin{macrocode}
\include{cdocsch1}
\include{cdocsch2}
%    \end{macrocode}

% Include the two parts unless only chapters should be displayed:
%    \begin{macrocode}
\ifchilddoc\else
\section{part three}
\input{cdocspt3}
\section{part four}
\input{cdocspt4}
\fi
%    \end{macrocode}

% Process as usual until here:
%    \begin{macrocode}
\fi
%    \end{macrocode}

% End of document body:
%    \begin{macrocode}
\end{document}
%    \end{macrocode}
%\iffalse
%</samplemain>
%\fi
%
% %%%%%%%%%%%%%%%%%%%%%%%%%%%%%%%%%%%%%%
% \paragraph{Chapter Include Files.}
%
% The include files are called |cdocsch1.tex| and |cdocsch2.tex|.
%
%\iffalse
%<*samplechap1|samplechap2>
%\fi

% Optional override for |\version| flag:
%    \begin{macrocode}
%%\providecommand{\version}{final}
%    \end{macrocode}

% Include the main document:
%    \begin{macrocode}
\input{childdoc.def}
\childdocof{cdocsamp}
%    \end{macrocode}

%\iffalse
%</samplechap1|samplechap2>
%\fi
%
%\iffalse
%<*samplechap1>
%\fi
% Some text for chapter 1:
%    \begin{macrocode}
\section{one}
some text in chapter one
%    \end{macrocode}

%\iffalse
%</samplechap1>
%\fi
% Some text for chapter 2:
%\iffalse
%<*samplechap2>
%\fi
%    \begin{macrocode}
\section{two}
more text in chapter two
%    \end{macrocode}

%\iffalse
%</samplechap2>
%\fi
%
% %%%%%%%%%%%%%%%%%%%%%%%%%%%%%%%%%%%%%%
% \paragraph{Part Include Files.}
%
% The include files are called |cdocspt3.tex| and |cdocspt4.tex|.
%
%\iffalse
%<*samplepart3|samplepart4>
%\fi

% Optional override for |\version| flag:
%    \begin{macrocode}
%%\providecommand{\version}{final}
%    \end{macrocode}

% Include the main document:
%    \begin{macrocode}
\input{childdoc.def}
\childdocby{cdocsamp}
%    \end{macrocode}

%\iffalse
%</samplepart3|samplepart4>
%\fi
%
%\iffalse
%<*samplepart3>
%\fi
% Some text for part 3:
%    \begin{macrocode}
some text in part three
%    \end{macrocode}

%\iffalse
%</samplepart3>
%\fi
% Some text for part 4:
%\iffalse
%<*samplepart4>
%\fi
%    \begin{macrocode}
more text in part four
%    \end{macrocode}

%\iffalse
%</samplepart4>
%\fi
%
% %%%%%%%%%%%%%%%%%%%%%%%%%%%%%%%%%%%%%%
% \paragraph{Forwarding for a Complete Draft.}
%
% The following forwarding file |cdocsdrf.tex|
% compiles the main document in draft mode:
%\iffalse
%<*sampledraft>
%\fi
%    \begin{macrocode}
\def\version{draft}
\input{childdoc.def}
\childdocforward{cdocsamp}
%    \end{macrocode}

%\iffalse
%</sampledraft>
%\fi
%
% %%%%%%%%%%%%%%%%%%%%%%%%%%%%%%%%%%%%%%
% \paragraph{Forwarding for Final Version of the Chapters.}
%
% The following forwarding files |cdocsfn1.tex| and |cdocsfn2.tex|
% (with identical content)
% compile the final versions of the child documents
% |cdocsch1.tex| and |cdocsch2.tex|, respectively:
%\iffalse
%<*samplefinal>
%\fi
%    \begin{macrocode}
\def\version{final}
\input{childdoc.def}
\childdocforwardprefix[cdocsamp]{cdocsfn}{cdocsch}
%    \end{macrocode}

%\iffalse
%</samplefinal>
%\fi
%
% %%%%%%%%%%%%%%%%%%%%%%%%%%%%%%%%%%%%%%
% \paragraph{Command Line Processing.}
%
% The following three command lines generate the output files
% |cdocscld|, |cdocscl1| and |cdocscl2|
% which should be identical to
% |cdocsdrf|, |cdocsch1| and |cdocsfn2|, respectively:
% \begin{center}
% \begin{tabular}{l}
% |latex -jobname cdocscld \|\\
% |  "\def\version{draft}\input{childdoc.def}\childdocforward{cdocsamp}"|\\
% |latex -jobname cdocscl1 \|\\
% |  "\input{childdoc.def}\childdocforward[cdocsamp]{cdocsch1}"|\\
% |latex -jobname cdocscl2 \|\\
% |  "\def\version{final}\input{childdoc.def}\childdocforward{cdocsch2}"|
% \end{tabular}
% \end{center}
% Note that the trailing backslash on each first line
% merely continues the input to the second line
% (for convenient cut ant paste).
% Furthermore, the command |latex| can be replaced by any
% of its alternative versions such as |pdflatex|.
%
% %%%%%%%%%%%%%%%%%%%%%%%%%%%%%%%%%%%%%%%%%%%%%%%%%%%%%%%%%%%%%%%%%%%%%%%%%%%%%%
% %%%%%%%%%%%%%%%%%%%%%%%%%%%%%%%%%%%%%%%%%%%%%%%%%%%%%%%%%%%%%%%%%%%%%%%%%%%%%%
% \section{Implementation}
%\iffalse
%<*package>
%\fi
%
% This section describes the definitions file |childdoc.def|.

% The definitions cannot be loaded using |\usepackage| or |\RequirePackage|
% which has a mechanism to prevent loading a style file more than once.
% When loading the definitions by means of |\input|
% multiple instances have to be prevented manually:
%\iffalse
%This code needs to be before the `\ProvidesFile' directive
%which is defined at the beginning of this file.
%Therefore it is also placed there and commented out here.
%</package>
%<*discard>
%\fi
%    \begin{macrocode}
\ifdefined\childdocmain\endinput\fi
%    \end{macrocode}
%\iffalse
%</discard>
%<*package>
%\fi
%
% \macro{\ifchilddoc}
% \macro{\ifchilddocmanual}
% The conditional |\ifchilddoc| tells whether a
% child (true) or main (false) document is being compiled.
% The conditional |\ifchilddocmanual| tells whether
% the |\includeonly| mechanism is used (false) or
% the selection of child files must be performed manually (true).
% The definitions initialise to false:
%    \begin{macrocode}
\newif\ifchilddoc
\newif\ifchilddocmanual
%    \end{macrocode}

% \macro{\childdocname}
% \macro{\childdocjob}
% The macro |\childdocname| stores the name of the main document
% to be compiled. The macro |\childdocjob| stores the name of
% the document on which the \LaTeX{} compiler was originally invoked.
% The content of |\jobname| cannot be compared
% to filenames specified in the source due to different catcodes.
% The following code rescans |\jobname|, stores the result
% in |\childdocname| and saves a copy in |\childdocjob|:
%    \begin{macrocode}
\edef\childdocname{\scantokens\expandafter{\jobname\noexpand}}
\let\childdocjob\childdocname
%    \end{macrocode}

% \macro{\childdocdisable}
% The macro |\childdocdisable| prevents the main file
% from being processed more than once.
% At this stage, the main document command |\childdocmain|
% is assumed to be called once again where it should do nothing.
% Any subsequent call to it should prevent
% a secondary processing of the main document
% It overwrites the forwarding commands
% |\childdocof| and |\childdocforward|
% with empty macros to prevent further inclusions of the main document:
%    \begin{macrocode}
\newcommand{\childdocdisable}
{
  \renewcommand{\childdocmain}[1]{\renewcommand{\childdocmain}[1]{\endinput}}
  \renewcommand{\childdocof}[1]{}
  \renewcommand{\childdocby}[2][]{}
  \renewcommand{\childdocforward}[2][]{}
  \renewcommand{\childdocdisable}{}
}
%    \end{macrocode}

% \macro{\childdocmain}
% The macro |\childdocmain| is to be called at the top of the main file
% with nothing or the main filename (without extension) as argument.
% First, it breaks loops.
% If the argument is not empty and does not match |\childdocname|
% (which is set by the first inclusion of |childdoc.def|),
% |\ifchilddoc| is set to true, |\includeonly| is applied to the child file
% and |\jobname| is set to the main file
% (for proper handling of |.aux| files):
%    \begin{macrocode}
\newcommand{\childdocmain}[1]
{
  \childdocdisable\childdocmain{}
  \if?#1?\else
    \begingroup
      \def\childdoctmp{#1}
      \ifx\childdoctmp\childdocname
        \def\childdoctmp{}
      \else
        \def\childdoctmp
        {
          \childdoctrue
          \includeonly{\childdocname}
          \def\childdocjob{#1}
          \def\jobname{#1}
        }
      \fi
      \expandafter
    \endgroup
    \childdoctmp
  \fi
}
%    \end{macrocode}

% \macro{\childdocof}
% The command |\childdocof| redirects
% compilation to the main file |#1|.
%    \begin{macrocode}
\newcommand{\childdocof}[1]
{
  \childdocdisable
  \childdoctrue
  \includeonly{\childdocname}
  \def\jobname{#1}
  \def\childdocjob{#1}
  \input{#1}
}
%    \end{macrocode}

% \macro{\childdocby}
% The command |\childdocby| ....
%    \begin{macrocode}
\newcommand{\childdocby}[2][]
{
  \childdocdisable
  \childdoctrue
  \childdocmanualtrue
  \if?#1?\else
    \def\jobname{#2}
  \fi
  \def\childdocjob{#2}
  \input{#2}
  \endinput
}
%    \end{macrocode}

% \macro{\childdocforward}
% The command |\childdocforward| redirects
% compilation to the main file or
% (if the optional argument is given) a child file.
% Parameters are set as if the main file
% or a child file starting with |\childdocof| was compiled.
% Then compilation is handed over to the main file:
%    \begin{macrocode}
\newcommand{\childdocforward}[2][]
{
  \begingroup
    \if?#1?
      \def\childdoctmp
      {
        \def\childdocname{#2}
        \def\childdocjob{#2}
        \def\jobname{#2}
        \input{#2}
        \endinput
      }
    \else
      \def\childdoctmp
      {
        \childdocdisable
        \def\childdocname{#2}
        \childdoctrue
        \includeonly{#2}
        \def\childdocjob{#1}
        \def\jobname{#1}
        \input{#1}
        \endinput
      }
    \fi
    \expandafter
  \endgroup
  \childdoctmp
}
%    \end{macrocode}

% \macro{\childdocforwardprefix}
% The command |\childdocforwardprefix| redirects
% compilation to the main or a child file by means of a pattern.
% The prefix |#1| in the current filename is replaced by |#2|
% and the suffix of the current filename is kept
% (it is assumed that the filename does not contain the substring `|~~~|'
% which is used as a delimiter).
% Compilation is handed over to the new file by |\childdocforward|:
%    \begin{macrocode}
\newcommand{\childdocforwardprefix}[3][]
{
  \begingroup
    \def\childdocextract #2##1~~~{\def\childdoctmp{\childdocforward[#1]{#3##1}}}
    \expandafter\childdocextract\childdocname~~~
    \expandafter
  \endgroup
  \childdoctmp
}
%    \end{macrocode}

% \macro{\childdoc}
% The deprecated macro |\childdoc| is a legacy version of |\childdocmain|:
%    \begin{macrocode}
\newcommand{\childdoc}{\childdocmain}
%    \end{macrocode}

% \macro{\childdocredirect}
% The deprecated macro |\childdocredirect| is a legacy version
% of |\childdocforward| and |\childdocforwardprefix|:
%    \begin{macrocode}
\newcommand{\childdocredirect}[2][]
{
  \begingroup
    \if?#1?
      \def\childdoctmp{\childdocforward{#2}}
    \else
      \def\childdoctmp{\childdocforwardprefix{#1}{#2}}
    \fi
    \expandafter
  \endgroup
  \childdoctmp
}
%    \end{macrocode}

%\iffalse
%</package>
%\fi
%
\endinput
|\\
|\childdocforward{|\textit{main}|}|\\
\end{tabular}
\end{center}
%
or alternatively with:
%
\begin{center}
\begin{tabular}{l}
|% \iffalse
%
% childdoc.dtx Copyright (C) 2017-2018 Niklas Beisert
%
% This work may be distributed and/or modified under the
% conditions of the LaTeX Project Public License, either version 1.3
% of this license or (at your option) any later version.
% The latest version of this license is in
%   http://www.latex-project.org/lppl.txt
% and version 1.3 or later is part of all distributions of LaTeX
% version 2005/12/01 or later.
%
% This work has the LPPL maintenance status `maintained'.
%
% The Current Maintainer of this work is Niklas Beisert.
%
% This work consists of the files childdoc.dtx and childdoc.ins
% and the derived files childdoc.def and cdocsamp.tex with
% cdocsch1.tex, cdocsch2.tex, cdocsdrf.tex, cdocsfn1.tex, cdocsfn2.tex.
%
%<package>\ifdefined\childdocmain\endinput\fi
%<package>\ProvidesFile{childdoc.def}[2018/12/30 v2.0 child document driver]
%<samplemain>\ProvidesFile{cdocsamp.tex}[2018/12/30 v2.0 sample for childdoc]
%<*driver>
%\ProvidesFile{childdoc.drv}[2018/12/30 v2.0 childdoc reference manual file]
\PassOptionsToClass{10pt,a4paper}{article}
\documentclass{ltxdoc}

\usepackage[margin=35mm]{geometry}
\usepackage{hyperref}
\usepackage{hyperxmp}
\usepackage[usenames]{color}

\hypersetup{colorlinks=true}
\hypersetup{pdfstartview=FitH}
\hypersetup{pdfpagemode=UseNone}
\hypersetup{pdfsource={}}
\hypersetup{pdflang={en-UK}}
\hypersetup{pdfcopyright={Copyright 2017-2018 Niklas Beisert.
  This work may be distributed and/or modified under the
  conditions of the LaTeX Project Public License, either version 1.3
  of this license or (at your option) any later version.}}
\hypersetup{pdflicenseurl={http://www.latex-project.org/lppl.txt}}
\hypersetup{pdfcontactaddress={ETH Zurich, ITP, HIT K,
  Wolfgang-Pauli-Strasse 27}}
\hypersetup{pdfcontactpostcode={8093}}
\hypersetup{pdfcontactcity={Zurich}}
\hypersetup{pdfcontactcountry={Switzerland}}
\hypersetup{pdfcontactemail={nbeisert@itp.phys.ethz.ch}}
\hypersetup{pdfcontacturl={http://people.phys.ethz.ch/\xmptilde nbeisert/}}

\newcommand{\secref}[1]{\hyperref[#1]{section \ref*{#1}}}

\parskip1ex
\parindent0pt
\let\olditemize\itemize
\def\itemize{\olditemize\parskip0pt}

\begin{document}

\title{The \textsf{childdoc} Package}
\hypersetup{pdftitle={The childdoc Package}}
\author{Niklas Beisert\\[2ex]
  Institut f\"ur Theoretische Physik\\
  Eidgen\"ossische Technische Hochschule Z\"urich\\
  Wolfgang-Pauli-Strasse 27, 8093 Z\"urich, Switzerland\\[1ex]
  \href{mailto:nbeisert@itp.phys.ethz.ch}
  {\texttt{nbeisert@itp.phys.ethz.ch}}}
\hypersetup{pdfauthor={Niklas Beisert}}
\hypersetup{pdfsubject={Manual for the LaTeX2e Package childdoc}}
\date{30 December 2018, \textsf{v2.0}}
\maketitle

\begin{abstract}\noindent
\textsf{childdoc} is a \LaTeXe{} package
that enables the direct compilation
of document sections included by |\include|
to individual files.
\end{abstract}

\begingroup
\parskip0ex
\tableofcontents
\endgroup

%%%%%%%%%%%%%%%%%%%%%%%%%%%%%%%%%%%%%%%%%%%%%%%%%%%%%%%%%%%%%%%%%%%%%%%%%%%%%%%%
%%%%%%%%%%%%%%%%%%%%%%%%%%%%%%%%%%%%%%%%%%%%%%%%%%%%%%%%%%%%%%%%%%%%%%%%%%%%%%%%
\section{Introduction}

\LaTeX{} provides a mechanism to structure a large document (such as a book)
into a main file and several child files (containing the chapters)
using the |\include| command.
This mechanism is beneficial for documents
which span hundreds of pages in order to
make the source file(s) more manageable.
Moreover, compilation can be restricted to
selected child files by means of the |\includeonly| command.
The latter feature can be used to reduce the compilation time while editing
(this was significantly more useful in the earlier days of \LaTeX{})
or to generate a smaller document which is easier to navigate.
Another application of |\includeonly| is to generate
documents consisting of selected parts of the complete document.

However, there are a few drawbacks of the plain |\include| mechanism:
\begin{itemize}
\item
The child files cannot be compiled on their own,
they can only be compiled via the main file.
A naive editing environment
(such as a text editor with an option
to have the current file processed by \LaTeX)
may require one to switch to the main file before compiling;
attempting to compile the child file produces errors.
\item
The main file must be modified (each time)
to adjust the |\includeonly| command
to the present needs. This easily leaves the main file in a messy state.
\item
The generated document will always carry the filename
of the main document. This is inconvenient if
several child files are to be compiled and
to be kept for distribution.
\end{itemize}

The present package provides a simple interface
to make child files individually compilable by \LaTeX{}.
Compiling a child file then has the same effect as compiling
the main file with an |\includeonly| command
to select the appropriate child.
Moreover the generated document will carry the name of the child
rather than the main file.
This resolves all three above issues.

This feature is meant to make the editing of books,
thesis documents and lecture notes somewhat more convenient.
However, the package can also be used efficiently for
composing a series of documents (such as exercise sheets)
which are typically distributed individually.
It then assists the author in generating the individual documents
(potentially in different versions)
as well as a document containing the collected series.
Another application is in developing style files
or other kinds of included material
where compilation of the style file could redirect
to a sample or test file.

%%%%%%%%%%%%%%%%%%%%%%%%%%%%%%%%%%%%%%%%%%%%%%%%%%%%%%%%%%%%%%%%%%%%%%%%%%%%%%%%
%%%%%%%%%%%%%%%%%%%%%%%%%%%%%%%%%%%%%%%%%%%%%%%%%%%%%%%%%%%%%%%%%%%%%%%%%%%%%%%%
\section{Usage}

First of all, the package \textsf{childdoc} is \emph{not} a standard
\LaTeXe{} |.sty| style file! Therefore it needs to be invoked in
a non-standard way.

%%%%%%%%%%%%%%%%%%%%%%%%%%%%%%%%%%%%%%%%%%%%%%%%%%%%%%%%%%%%%%%%%%%%%%%%%%%%%%%%
\subsection{Included Files}
\label{sec:include}

%%%%%%%%%%%%%%%%%%%%%%%%%%%%%%%%%%%%%%%%
\DescribeMacro{\childdocmain}
To use the package, add the commands
\begin{center}
\begin{tabular}{l}
|\input{childdoc.def}|\\
|\childdocmain{}|\\
\end{tabular}
\end{center}
at the very top of the main \LaTeX{} file,
in particular \emph{before} the |\documentclass| statement!
The argument of |\childdocmain| should be left empty
(but it must be present).

%%%%%%%%%%%%%%%%%%%%%%%%%%%%%%%%%%%%%%%%
\DescribeMacro{\childdocof}
Furthermore, add the commands
\begin{center}
\begin{tabular}{l}
|\input{childdoc.def}|\\
|\childdocof{|\textit{main}|}|\\
\end{tabular}
\end{center}
at the top of every child file \textit{child}
which is included by |\include{|\textit{child}|}|
from within the main file
(or at least for those files to be compiled individually).
The argument \textit{main} must be the filename of the main file.

There are a couple of
considerations in setting up the main and child documents:

%%%%%%%%%%%%%%%%%%%%%%%%%%%%%%%%%%%%%%%%
\paragraph{Restrictions.}

Please note the following restrictions:
\begin{itemize}
\item
|\childdocmain| must be called with one argument \textit{main}
to ensure compatibility with earlier version of the package.
It must either be empty (|\childdocmain{}|)
or precisely match the filename of the main file in which it is specified.
See \secref{sec:detection} for further information.
\item
The filename \textit{main} must be specified without the |.tex| extension.
\item
The filename \textit{main} is case sensitive
(even in case-insensitive file systems)
due to internal string comparison.
\item
The argument \textit{main} should be fully expanded, it cannot be a macro.
\item
Subdirectories and special characters should be avoided in filenames.
\item
The command |\childdocmain{|\textit{main}|}| must be followed by a whitespace.
It should not be followed immediately by another command
or by a comment mark `|%|'.
This is because the \TeX{} parser reads the token immediately following
the argument of |\childdocmain| and puts it
at the beginning of every child section;
however, a white\-space is ignored.
\end{itemize}

%%%%%%%%%%%%%%%%%%%%%%%%%%%%%%%%%%%%%%%%
\paragraph{Content of Main File.}

It is advisable to place all content in the child files included by |\include|.
Any output contained in the main file will appear in all child documents
unless suppressed manually;
it cannot be suppressed automatically by the |\includeonly| directive
and thus should normally be avoided.
A method to include some content in the main file
by means of conditional processing is described in \secref{sec:conditional}.

%%%%%%%%%%%%%%%%%%%%%%%%%%%%%%%%%%%%%%%%
\paragraph{Page Numbering.}

When only a part of the document is compiled,
the appropriate numbering of pages
(as well as other status parameters)
is determined from the |.aux| files.
The latter contain information from previous passes.
However this information needs to propagate through
all intermediate child documents.
Therefore the page numbering in child documents may well
be inconsistent until the complete document is compiled at least once.

A useful (if unconventional) way to always ensure a consistent
page numbering is to restart the numbering in each child document
and denote the pages by `\textit{child}|.|\textit{page}'
where \textit{child} represents the chapter/section number of the child file.
This can be achieved by the command
|\numberwithin{page}{|\textit{child}|}|
of the \textsf{amsmath} package
where \textit{child} can be |chapter| or |section|
depending on the chosen structuring.
Alternatively, one can modify the macro |\thepage| appropriately
and reset the counter |page| at the start of each child file.

%%%%%%%%%%%%%%%%%%%%%%%%%%%%%%%%%%%%%%%%%%%%%%%%%%%%%%%%%%%%%%%%%%%%%%%%%%%%%%%%
\subsection{Conditional Processing}
\label{sec:conditional}

The package provides a mechanism to compile different versions
of a document. To customise the versions further some conditional processing
can come in handy to distinguish which version is being compiled.
The package provides two macros to describe the compilation context:

%%%%%%%%%%%%%%%%%%%%%%%%%%%%%%%%%%%%%%%%
\DescribeMacro{\ifchilddoc}
The conditional |\ifchilddoc| distinguishes between the compilation of
child documents and the main document:
%
\begin{center}
|\ifchilddoc |\textit{child-code}| |[|\||else |\textit{main-code}]| \||fi|
\end{center}

%%%%%%%%%%%%%%%%%%%%%%%%%%%%%%%%%%%%%%%%
\DescribeMacro{\childdocname}
\DescribeMacro{\childdocjob}
The macro |\childdocname| contains the filename (without extension)
of the main or child file being processed.
Note that |\childdocjob| will always contain the name of the main file.

%%%%%%%%%%%%%%%%%%%%%%%%%%%%%%%%%%%%%%%%
\paragraph{Title Page.}

Conditional processing can be used to include a title or banner page
in the main document when proper precautions are taken.
Importantly, the code in the main file should ensure that the page counter
(as well as other status parameters which are stored in the |.aux| files)
takes the same value after the conditional processing.
Otherwise the page numbers may take divergent values
depending on which part is compiled.

For example, a title page could be declared by:
%
\begin{center}
\begin{tabular}{l}
|\ifchilddoc\||else|\\
|\addtocounter{page}{-1}|\\
\textit{code for title page}\\
|\newpage|\\
|\||fi|
\end{tabular}
\end{center}
%
A banner page for the child documents can be generated by:
%
\begin{center}
\begin{tabular}{l}
|\ifchilddoc|\\
|\addtocounter{page}{-1}|\\
\textit{code for banner page}\\
|\newpage|\\
|\||fi|
\end{tabular}
\end{center}
%
Here one could write a message such as:
\begin{center}
|This is the part \childdocname{} of \childdocjob{}.|
\end{center}

%%%%%%%%%%%%%%%%%%%%%%%%%%%%%%%%%%%%%%%%%%%%%%%%%%%%%%%%%%%%%%%%%%%%%%%%%%%%%%%%
\subsection{Flags}
\label{sec:flags}

The package makes it easy to generate different versions
of the main or child documents.
To this end compilation flags can be defined
and assigned different default values.
They will be particularly useful in conjunction
with the forwarding mechanism described in \secref{sec:forward}.

For example, it may be useful to have a flag |\version|
which can be set to |draft| or |final|.
The document source will contain some conditional code
depending on the value of |\version|.
Suppose further, the flag should default to |final| for the main file
and to |draft| for child files
which is a natural assignment for editing the document.
This is achieved by placing the following code
in the preamble of the main document
(below the |\childdocmain| directive):
%
\begin{center}
\begin{tabular}{l}
|\ifchilddoc|\\
|\providecommand{\version}{draft}|\\
|\||else|\\
|\providecommand{\version}{final}|\\
|\||fi|
\end{tabular}
\end{center}
%
The definition by |\providecommand| makes sure
that previous definitions are not overwritten.
Further statements |\providecommand{\version}{...}|
can thus be added before the above code to override it.

For the main file, one might add a line
(between |\childdocmain| and the above block)
%
\begin{center}
|%\ifchilddoc\||else\providecommand{\version}{draft}\||fi|
\end{center}
%
which can be uncommented to produce a draft version.
Likewise one can add a line to the very top of a child file
(above the |\childdocof{|\textit{main}|}| directive)
%
\begin{center}
|%\providecommand{\version}{final}|
\end{center}
%
which can be uncommented to produce the final version of this child document.

%%%%%%%%%%%%%%%%%%%%%%%%%%%%%%%%%%%%%%%%%%%%%%%%%%%%%%%%%%%%%%%%%%%%%%%%%%%%%%%%
\subsection{Forwarding}
\label{sec:forward}

Different versions of the main or child documents
using compilation flags as described in \secref{sec:flags}
can be (permanently) stored in different files
for convenient compilation, viewing and distribution.
To this end, the package defines a command
to pass on compilation to a different file:

%%%%%%%%%%%%%%%%%%%%%%%%%%%%%%%%%%%%%%%%
\DescribeMacro{\childdocforward}
The command |\childdocforward| redirects processing to
another source file:
%
\begin{center}
\begin{tabular}{l}
|\input{childdoc.def}|\\
|\childdocforward[|\textit{main}|]{|\textit{dest}|}|\\
\end{tabular}
\end{center}
%
The argument \textit{dest} is the destination file
(without extension).
It should be the main file or one of the child files.
Note that further \textsf{childdoc} directives
such as |\childdocof| and |\childdocforward|
in the indicated file will be processed in this form.
The optional argument \textit{main}
passes on directly to the main file \textit{main}
while pretending to compile the child \textit{dest}.
This form behaves as if \textit{dest}
issues |\childdocof{|\textit{main}|}| right away,
and no further \textsf{childdoc} directives will be processed.

%%%%%%%%%%%%%%%%%%%%%%%%%%%%%%%%%%%%%%%%
\DescribeMacro{\...prefix}
In the alternative form |\childdocforwardprefix|,
%
\begin{center}
\begin{tabular}{l}
|\input{childdoc.def}|\\
|\childdocforwardprefix[|\textit{main}|]{|\textit{prefix}|}{|\textit{dest}|}|
\end{tabular}
\end{center}
%
the destination file is determined by a pattern
depending on the current file:
To make this work, the current file must be called
`{\textit{prefix}\hspace{0.2em}\textit{suffix}}'
with \textit{prefix} matching precisely the argument.
Processing is then passed on to the file
`{\textit{dest}\hspace{0.2em}\textit{suffix}}'.
Surely, the same effect is achieved by
directly specifying the
argument `{\textit{dest}\hspace{0.2em}\textit{suffix}}'
in the first form.
However, that requires to set up a different file
for each child. With the alternative form of the command
all these files can have exactly the same content
which simplifies setting them up and maintaining them.

For example, the following file |draft.tex|
with a compilation flag |\version| as described in \secref{sec:flags}
compiles the main document as a draft:
%
\begin{center}
\begin{tabular}{l}
|\def\version{draft}|\\
|\input{childdoc.def}|\\
|\childdocforward{|\textit{main}|}|
\end{tabular}
\end{center}
%
Likewise, the following files |final|\textit{nn}|.tex|
compile the final version of the child document
|child|\textit{nn}|.tex|:
%
\begin{center}
\begin{tabular}{l}
|\def\version{final}|\\
|\input{childdoc.def}|\\
|\childdocforwardprefix{final}{child}|
\end{tabular}
\end{center}
%

Note that when several versions of a main file and/or of each child file
are to be generated, it may be convenient to set up a |Makefile| or
shell script to automatise the process.

%%%%%%%%%%%%%%%%%%%%%%%%%%%%%%%%%%%%%%%%%%%%%%%%%%%%%%%%%%%%%%%%%%%%%%%%%%%%%%%%
\subsection{Command Line Processing}
\label{sec:commandline}

The effect of redirection files can also be achieved by invoking
the \LaTeX{} compiler with a more elaborate command line.
Most conveniently this should be done as part
of a shell script or a |Makefile|.

When using \textsf{childdoc} in the main file, the following
command lines effectively perform a redirection
(note that depending on the shell being used,
backslashes may have to be doubled: `|\|' $\to$ `|\\|'):
%
\begin{center}
|... -jobname "|\textit{target}|" |\\|"|[\textit{flags}]%
|\input{childdoc.def}\childdocforward[|\textit{main}|]{|\textit{dest}|}"|
\end{center}
%
Here \textit{target} is the name of the output file,
\textit{main} is the name of the main file
and \textit{dest} is the name of the main or child file to be processed
(all filenames without extensions).
The optional argument \textit{main} can be omitted
if \textit{main} matches \textit{dest}.
Optionally, compilation \textit{flags} can be defined via |\def| commands.
This command line makes the \TeX{} engine believe
it is compiling the file \textit{target}
whose content is specified as the latter parameter.
The provided code then forwards the processing to
\textit{main} or \textit{dest} as described in \secref{sec:forward}.

%%%%%%%%%%%%%%%%%%%%%%%%%%%%%%%%%%%%%%%%%%%%%%%%%%%%%%%%%%%%%%%%%%%%%%%%%%%%%%%%
\subsection{Include by Input}
\label{sec:input}

Including child documents by |\include| has some restrictions by design.
Most notably, the content of a child document always occupies
its own set of pages; pages cannot be shared between child documents.
Usually, this behaviour makes perfect sense
because each child document contain an essential part of the document.
However, in some situations it may be desirable to compose
a document from a collection of parts
without having mandatory page breaks between then.
For this case, the package
provides a mechanism to include parts
by |\input| which can also be processed individually.
However, by construction this mechanism
requires manual handling of the content to be output.

%%%%%%%%%%%%%%%%%%%%%%%%%%%%%%%%%%%%%%%%
\DescribeMacro{\ifchilddocmanual}
The main file should be prepared as usual, see \secref{sec:include}.
However, the document body must make a distinction
between processing of an individual part and of the main document, e.g.:
%
\begin{center}
\begin{tabular}{l}
|\ifchilddocmanual|\\
|\input{\childdocname}|\\
|\||else|\\
\textit{document body with }|\input{|\textit{part}|}|\\
|\||fi|
\end{tabular}
\end{center}
%
The conditional |\ifchilddocmanual| is true whenever
a part to be included by |\input| is being compiled,
and the name of the part is stored in |\childdocname|.

%%%%%%%%%%%%%%%%%%%%%%%%%%%%%%%%%%%%%%%%
\DescribeMacro{\childdocby}
Each part to be included by |\input| should start with:
%
\begin{center}
\begin{tabular}{l}
|\input{childdoc.def}|\\
|\childdocby{|\textit{main}|}|\\
\end{tabular}
\end{center}
%
The directive |\childdocby| is similar to |\childdocof|
described in \secref{sec:include},
but the subsequent selection of content must be done manually.
To that end, both |\ifchilddoc| and |\ifchilddocmanual|
will be true upon processing of a part,
and the name of the part is stored in |\childdocname|.
Note that |\jobname| will be set to the filename of the current part
so that each part receives an individual |.aux| file
that does not interfere with the |.aux| file(s) of the main document.
This behaviour can be altered by the alternative form
|\childdocby[*]{|\textit{main}|}| (with a non-empty optional argument)
which uses the |.aux| file of the main document
by setting |\jobname| to \textit{main}.

%%%%%%%%%%%%%%%%%%%%%%%%%%%%%%%%%%%%%%%%%%%%%%%%%%%%%%%%%%%%%%%%%%%%%%%%%%%%%%%%
\subsection{Driver Development}
\label{sec:driver}

The \textsf{childdoc} mechanism can also be use for the development
of definition files such as \LaTeX{} styles or classes.
This case differs from the above setup with multiple parts
included by |\include| in that no |\includeonly| should be invoked.
This can be achieved by starting the include file
(before |\ProvidesPackage|) with:
%
\begin{center}
\begin{tabular}{l}
|\input{childdoc.def}|\\
|\childdocforward{|\textit{main}|}|\\
\end{tabular}
\end{center}
%
or alternatively with:
%
\begin{center}
\begin{tabular}{l}
|\input{childdoc.def}|\\
|\childdocby{|\textit{main}|}|\\
\end{tabular}
\end{center}
%
Both forms have slightly different effects as described above.
The main file is prepared as usual, see \secref{sec:include}.

%%%%%%%%%%%%%%%%%%%%%%%%%%%%%%%%%%%%%%%%%%%%%%%%%%%%%%%%%%%%%%%%%%%%%%%%%%%%%%%%
\subsection{Legacy Detection}
\label{sec:detection}

The directive |\childdocmain| in the main file can detect
whether the complete document or merely a child is to be compiled
even without using the directive |\childdocof|.
This method is deprecated because it is less robust
and there is no compelling reason to use it;
it is merely provided for backward compatibility
and it may be removed in future versions.

If the detection mechanism is to be used,
it is mandatory to correctly specify
the filename of the main file as the argument of |\childdocmain|:
%
\begin{center}
\begin{tabular}{l}
|\input{childdoc.def}|\\
|\childdocmain{|\textit{main}|}|\\
\end{tabular}
\end{center}
%
If |\jobname| does not match the argument \textit{main} of |\childdocmain|,
it is assumed that |\jobname| points to the child file to be compiled.
When using |\childdocmain| with the main file specified as argument,
it suffices to start a child file
with just |\input{|\textit{main}|}|
without loading of the package and using |\childdocof|.
If instead all processing is done
with the appropriate \textsf{childdoc} directives,
the argument of \textit{main} of |\childdocmain| can be empty.

An alternative version of the command line processing described
in \secref{sec:commandline} using the detection mechanism reads:
%
\begin{center}
|... -jobname "|\textit{target}|" "|[\textit{flags}]%
[|\def\jobname{|\textit{dest}|}|]|\input{|\textit{main}|}"|
\end{center}

%%%%%%%%%%%%%%%%%%%%%%%%%%%%%%%%%%%%%%%%%%%%%%%%%%%%%%%%%%%%%%%%%%%%%%%%%%%%%%%%
\subsection{Manual Code}
\label{sec:manual}

In case one cannot be certain whether the definitions file |childdoc.def|
is installed on the target \TeX{} distribution
and one prefers not to ship it,
it is conceivable to paste a few relevant commands into the sources.

To that end, drop all statements |\input{childdoc.def}|
and perform the replacements as outlined below.
Instead of |\childdocmain{|\textit{main}|}| add the following code
to the top of the main file:
%
\begin{center}
\begin{tabular}{l}
|\||ifdefined\childdocname\endinput\||fi\newif\ifchilddoc|\\
|\edef\childdocname{\scantokens\expandafter{\jobname\noexpand}}|\\
|\def\childdocmain{|\textit{main}|}\||ifx\childdocmain\childdocname\||else|\\
|\childdoctrue\includeonly{\childdocname}\let\jobname\childdocmain\||fi|\\
\end{tabular}
\end{center}
%
Instead of |\childdocof{|\textit{main}|}| just include the main file
at the top of each child file:
%
\begin{center}
|\input{|\textit{main}|}|
\end{center}
%
A simple redirection |\childdocforward{|\textit{dest}|}| is achieved by:
%
\begin{center}
|\def\jobname{|\textit{dest}|}\input{\jobname}|
\end{center}
%
The redirection with prefix
|\childdocforwardprefix[|\textit{prefix}|]{|\textit{dest}|}|
is accomplished by:
%
\begin{center}
\begin{tabular}{l}
|{\edef\jobname{\scantokens\expandafter{\jobname\noexpand}}|\\
|\def\redirectjob |\textit{prefix}|#1~~~{\gdef\jobname{|\textit{dest}|#1}}|\\
|\expandafter\redirectjob\jobname~~~}\input{\jobname}|
\end{tabular}
\end{center}

In an alternative approach,
child documents can be compiled by a specific command line
without additional code or specific definitions:
%
\begin{center}
|... -jobname "|\textit{target}|" "|[\textit{flags}]%
|\includeonly{|\textit{dest}|}\input{|\textit{main}|}"|
\end{center}
%

%%%%%%%%%%%%%%%%%%%%%%%%%%%%%%%%%%%%%%%%%%%%%%%%%%%%%%%%%%%%%%%%%%%%%%%%%%%%%%%%
%%%%%%%%%%%%%%%%%%%%%%%%%%%%%%%%%%%%%%%%%%%%%%%%%%%%%%%%%%%%%%%%%%%%%%%%%%%%%%%%
\section{Information}

%%%%%%%%%%%%%%%%%%%%%%%%%%%%%%%%%%%%%%%%%%%%%%%%%%%%%%%%%%%%%%%%%%%%%%%%%%%%%%%%
\subsection{Copyright}

Copyright \copyright{} 2017--2018 Niklas Beisert

This work may be distributed and/or modified under the
conditions of the \LaTeX{} Project Public License, either version 1.3
of this license or (at your option) any later version.
The latest version of this license is in
  \url{http://www.latex-project.org/lppl.txt}
and version 1.3 or later is part of all distributions of \LaTeX{}
version 2005/12/01 or later.

This work has the LPPL maintenance status `maintained'.

The Current Maintainer of this work is Niklas Beisert.

This work consists of the files |README.txt|, |childdoc.ins| and |childdoc.dtx|
as well as the derived files |childdoc.def|, |cdocsamp.tex|
with |cdocsch1.tex|, |cdocsch2.tex|, |cdocspt3.tex|, |cdocspt4.tex|,
|cdocsdrf.tex|, |cdocsfn1.tex|, |cdocsfn2.tex|
as well as |childdoc.pdf|.

%%%%%%%%%%%%%%%%%%%%%%%%%%%%%%%%%%%%%%%%%%%%%%%%%%%%%%%%%%%%%%%%%%%%%%%%%%%%%%%%
\subsection{Files and Installation}

The package consists of the files:
%
\begin{center}
\begin{tabular}{ll}
    |README.txt|   & readme file \\
    |childdoc.ins| & installation file \\
    |childdoc.dtx| & source file \\
    |childdoc.def| & definition file \\
    |cdocsamp.tex| & sample main file \\
    |cdocsch1.tex| & sample include file \\
    |cdocsch2.tex| & sample include file \\
    |cdocspt3.tex| & sample part file \\
    |cdocspt4.tex| & sample part file \\
    |cdocsdrf.tex| & sample redirection file \\
    |cdocsfn1.tex| & sample redirection file \\
    |cdocsfn2.tex| & sample redirection file \\
    |childdoc.pdf| & manual
\end{tabular}
\end{center}
%
The distribution consists of the files
|README.txt|, |childdoc.ins| and |childdoc.dtx|.
%
\begin{itemize}
\item
Run (pdf)\LaTeX{} on |childdoc.dtx|
to compile the manual |childdoc.pdf| (this file).
\item
Run \LaTeX{} on |childdoc.ins| to create the definitions file |childdoc.def|
and the sample |cdocsamp.tex| with include files
|cdocsch1.tex|, |cdocsch2.tex|, |cdocspt3.tex|, |cdocspt4.tex|,
|cdocsdrf.tex|, |cdocsfn1.tex|, |cdocsfn2.tex|.
Then copy the file |childdoc.def| to an appropriate directory of your \LaTeX{}
distribution, e.g.\ \textit{texmf-root}|/tex/latex/childdoc|.
\end{itemize}

%%%%%%%%%%%%%%%%%%%%%%%%%%%%%%%%%%%%%%%%%%%%%%%%%%%%%%%%%%%%%%%%%%%%%%%%%%%%%%%%
\subsection{Related CTAN Packages}

There are several other packages which offer a similar functionality:
%
\begin{itemize}
\item
The packages
\href{http://ctan.org/pkg/docmute}{\textsf{docmute}},
\href{http://ctan.org/pkg/includex}{\textsf{includex}} and
\href{http://ctan.org/pkg/standalone}{\textsf{standalone}}
provide commands to include only the document body of
a child file thus allowing both files to be compiled individually.
\item
The packages \href{http://ctan.org/pkg/subdocs}{\textsf{subdocs}}
and \href{http://ctan.org/pkg/subfiles}{\textsf{subfiles}}
provide structures in which the main and child documents can be
encapsulated and allowing them to be compiled individually.
The inclusion mechanism is different from the conventional |\include|.
\item
The package \href{http://ctan.org/pkg/combine}{\textsf{combine}}
is an elaborate solution to combine several documents into one.
\end{itemize}
%
See also the CTAN topic \href{http://ctan.org/topic/subdocs}{\textsf{subdocs}}
for further related packages.
The present package differs from the above solutions in that
a document structure constructed with the conventional |\include| mechanism
just needs two extra commands at the top of every file
such that all constituent files can be compiled individually.

%%%%%%%%%%%%%%%%%%%%%%%%%%%%%%%%%%%%%%%%%%%%%%%%%%%%%%%%%%%%%%%%%%%%%%%%%%%%%%%%
%\subsection{Feature Suggestions}
%
%The following is a list of features which may be useful for future
%versions of this package:
%%
%\begin{itemize}
%\item
%\ldots
%\end{itemize}

%%%%%%%%%%%%%%%%%%%%%%%%%%%%%%%%%%%%%%%%%%%%%%%%%%%%%%%%%%%%%%%%%%%%%%%%%%%%%%%%
\subsection{Revision History}

%%%%%%%%%%%%%%%%%%%%%%%%%%%%%%%%%%%%%%%%
\paragraph{v2.0:} 2018/12/30

\begin{itemize}
\item
immediate forward processing
\item
added |\childdocby| mechanism
\item
manual restructured
\end{itemize}

%%%%%%%%%%%%%%%%%%%%%%%%%%%%%%%%%%%%%%%%
\paragraph{v1.6:} 2018/01/17

\begin{itemize}
\item
application for development of include files
\item
corrections to manual
\end{itemize}

%%%%%%%%%%%%%%%%%%%%%%%%%%%%%%%%%%%%%%%%
\paragraph{v1.5:} 2017/05/21

\begin{itemize}
\item
more complete structuring introduced
\item
|\childdocof| introduced
\item
|\childdoc| renamed to |\childdocmain|
\item
|\childredirect| renamed to |\childdocforward| and |\childdocforwardprefix|
and functionality expanded
\end{itemize}

%%%%%%%%%%%%%%%%%%%%%%%%%%%%%%%%%%%%%%%%
\paragraph{v1.0:} 2017/04/27

\begin{itemize}
\item
manual and install package
\item
first version published on CTAN
\end{itemize}

%%%%%%%%%%%%%%%%%%%%%%%%%%%%%%%%%%%%%%%%
\paragraph{v0.6:} 2017/04/26

\begin{itemize}
\item
redirection mechanism added
\end{itemize}

%%%%%%%%%%%%%%%%%%%%%%%%%%%%%%%%%%%%%%%%
\paragraph{v0.5:} 2017/04/26

\begin{itemize}
\item
functionality in definition file
\end{itemize}


%%%%%%%%%%%%%%%%%%%%%%%%%%%%%%%%%%%%%%%%%%%%%%%%%%%%%%%%%%%%%%%%%%%%%%%%%%%%%%%%
%%%%%%%%%%%%%%%%%%%%%%%%%%%%%%%%%%%%%%%%%%%%%%%%%%%%%%%%%%%%%%%%%%%%%%%%%%%%%%%%
%%%%%%%%%%%%%%%%%%%%%%%%%%%%%%%%%%%%%%%%%%%%%%%%%%%%%%%%%%%%%%%%%%%%%%%%%%%%%%%%
\appendix

\settowidth\MacroIndent{\rmfamily\scriptsize 000\ }

 \DocInput{childdoc.dtx}

\end{document}
%</driver>
% \fi
%
% %%%%%%%%%%%%%%%%%%%%%%%%%%%%%%%%%%%%%%%%%%%%%%%%%%%%%%%%%%%%%%%%%%%%%%%%%%%%%%
% %%%%%%%%%%%%%%%%%%%%%%%%%%%%%%%%%%%%%%%%%%%%%%%%%%%%%%%%%%%%%%%%%%%%%%%%%%%%%%
% \section{Sample}
%\iffalse
%<*samplemain>
%\fi
%
% The following presents a sample document
% with two chapters, two parts, a title page,
% a compile flag as well as three forwarding files to set the flag.
% It consists of eight |.tex| files:
% \begin{center}
% \begin{tabular}{ll}
% |cdocsamp.tex|&main file\\
% |cdocsch1.tex|&include file for chapter 1\\
% |cdocsch2.tex|&include file for chapter 2\\
% |cdocspt3.tex|&include file for part 3\\
% |cdocspt4.tex|&include file for part 4\\
% |cdocsdrf.tex|&forwarding file for main file in draft mode\\
% |cdocsfi1.tex|&forwarding file for final version of chapter 1\\
% |cdocsfi2.tex|&forwarding file for final version of chapter 2\\
% \end{tabular}
% \end{center}
% Each of the eight files can be compiled directly by the \LaTeX{} compiler.
%
% %%%%%%%%%%%%%%%%%%%%%%%%%%%%%%%%%%%%%%
% \paragraph{Main File.}
%
% The main file is called |cdocsamp.tex|.
%
% Load the \textsf{childdoc} definitions and
% declare the filename for the main document:
%    \begin{macrocode}
\input{childdoc.def}
\childdocmain{}
%    \end{macrocode}

% Optional override for |\version| flag:
%    \begin{macrocode}
%%\ifchilddoc\else\providecommand{\version}{draft}\fi
%    \end{macrocode}

% Define the default values for the |\version| flag
% (|final| for the main file and |draft| for childs):
%    \begin{macrocode}
\ifchilddoc
\providecommand{\version}{draft}
\else
\providecommand{\version}{final}
\fi
%    \end{macrocode}

% Load the standard document class:
%    \begin{macrocode}
\documentclass[12pt]{article}
%    \end{macrocode}

% Start the document body:
%    \begin{macrocode}
\begin{document}
%    \end{macrocode}

% Declare a title page.
% Print title, part of document being processed and version flag:
%    \begin{macrocode}
\addtocounter{page}{-1}
\begin{center}
{\LARGE\bfseries{}childdoc example\par}
\vspace{1cm}
\ifchilddoc
\ifchilddocmanual part\else chapter\fi:
`\childdocname' of `\childdocjob'\par
\else
main document: `\childdocjob'\par
\fi
version: \version\par
\end{center}
\newpage
%    \end{macrocode}

% Manually include selected file,
% otherwise process as usual:
%    \begin{macrocode}
\ifchilddocmanual
\section*{part `\childdocname'}
\input{\childdocname}
\else
%    \end{macrocode}

% Include the two chapters:
%    \begin{macrocode}
\include{cdocsch1}
\include{cdocsch2}
%    \end{macrocode}

% Include the two parts unless only chapters should be displayed:
%    \begin{macrocode}
\ifchilddoc\else
\section{part three}
\input{cdocspt3}
\section{part four}
\input{cdocspt4}
\fi
%    \end{macrocode}

% Process as usual until here:
%    \begin{macrocode}
\fi
%    \end{macrocode}

% End of document body:
%    \begin{macrocode}
\end{document}
%    \end{macrocode}
%\iffalse
%</samplemain>
%\fi
%
% %%%%%%%%%%%%%%%%%%%%%%%%%%%%%%%%%%%%%%
% \paragraph{Chapter Include Files.}
%
% The include files are called |cdocsch1.tex| and |cdocsch2.tex|.
%
%\iffalse
%<*samplechap1|samplechap2>
%\fi

% Optional override for |\version| flag:
%    \begin{macrocode}
%%\providecommand{\version}{final}
%    \end{macrocode}

% Include the main document:
%    \begin{macrocode}
\input{childdoc.def}
\childdocof{cdocsamp}
%    \end{macrocode}

%\iffalse
%</samplechap1|samplechap2>
%\fi
%
%\iffalse
%<*samplechap1>
%\fi
% Some text for chapter 1:
%    \begin{macrocode}
\section{one}
some text in chapter one
%    \end{macrocode}

%\iffalse
%</samplechap1>
%\fi
% Some text for chapter 2:
%\iffalse
%<*samplechap2>
%\fi
%    \begin{macrocode}
\section{two}
more text in chapter two
%    \end{macrocode}

%\iffalse
%</samplechap2>
%\fi
%
% %%%%%%%%%%%%%%%%%%%%%%%%%%%%%%%%%%%%%%
% \paragraph{Part Include Files.}
%
% The include files are called |cdocspt3.tex| and |cdocspt4.tex|.
%
%\iffalse
%<*samplepart3|samplepart4>
%\fi

% Optional override for |\version| flag:
%    \begin{macrocode}
%%\providecommand{\version}{final}
%    \end{macrocode}

% Include the main document:
%    \begin{macrocode}
\input{childdoc.def}
\childdocby{cdocsamp}
%    \end{macrocode}

%\iffalse
%</samplepart3|samplepart4>
%\fi
%
%\iffalse
%<*samplepart3>
%\fi
% Some text for part 3:
%    \begin{macrocode}
some text in part three
%    \end{macrocode}

%\iffalse
%</samplepart3>
%\fi
% Some text for part 4:
%\iffalse
%<*samplepart4>
%\fi
%    \begin{macrocode}
more text in part four
%    \end{macrocode}

%\iffalse
%</samplepart4>
%\fi
%
% %%%%%%%%%%%%%%%%%%%%%%%%%%%%%%%%%%%%%%
% \paragraph{Forwarding for a Complete Draft.}
%
% The following forwarding file |cdocsdrf.tex|
% compiles the main document in draft mode:
%\iffalse
%<*sampledraft>
%\fi
%    \begin{macrocode}
\def\version{draft}
\input{childdoc.def}
\childdocforward{cdocsamp}
%    \end{macrocode}

%\iffalse
%</sampledraft>
%\fi
%
% %%%%%%%%%%%%%%%%%%%%%%%%%%%%%%%%%%%%%%
% \paragraph{Forwarding for Final Version of the Chapters.}
%
% The following forwarding files |cdocsfn1.tex| and |cdocsfn2.tex|
% (with identical content)
% compile the final versions of the child documents
% |cdocsch1.tex| and |cdocsch2.tex|, respectively:
%\iffalse
%<*samplefinal>
%\fi
%    \begin{macrocode}
\def\version{final}
\input{childdoc.def}
\childdocforwardprefix[cdocsamp]{cdocsfn}{cdocsch}
%    \end{macrocode}

%\iffalse
%</samplefinal>
%\fi
%
% %%%%%%%%%%%%%%%%%%%%%%%%%%%%%%%%%%%%%%
% \paragraph{Command Line Processing.}
%
% The following three command lines generate the output files
% |cdocscld|, |cdocscl1| and |cdocscl2|
% which should be identical to
% |cdocsdrf|, |cdocsch1| and |cdocsfn2|, respectively:
% \begin{center}
% \begin{tabular}{l}
% |latex -jobname cdocscld \|\\
% |  "\def\version{draft}\input{childdoc.def}\childdocforward{cdocsamp}"|\\
% |latex -jobname cdocscl1 \|\\
% |  "\input{childdoc.def}\childdocforward[cdocsamp]{cdocsch1}"|\\
% |latex -jobname cdocscl2 \|\\
% |  "\def\version{final}\input{childdoc.def}\childdocforward{cdocsch2}"|
% \end{tabular}
% \end{center}
% Note that the trailing backslash on each first line
% merely continues the input to the second line
% (for convenient cut ant paste).
% Furthermore, the command |latex| can be replaced by any
% of its alternative versions such as |pdflatex|.
%
% %%%%%%%%%%%%%%%%%%%%%%%%%%%%%%%%%%%%%%%%%%%%%%%%%%%%%%%%%%%%%%%%%%%%%%%%%%%%%%
% %%%%%%%%%%%%%%%%%%%%%%%%%%%%%%%%%%%%%%%%%%%%%%%%%%%%%%%%%%%%%%%%%%%%%%%%%%%%%%
% \section{Implementation}
%\iffalse
%<*package>
%\fi
%
% This section describes the definitions file |childdoc.def|.

% The definitions cannot be loaded using |\usepackage| or |\RequirePackage|
% which has a mechanism to prevent loading a style file more than once.
% When loading the definitions by means of |\input|
% multiple instances have to be prevented manually:
%\iffalse
%This code needs to be before the `\ProvidesFile' directive
%which is defined at the beginning of this file.
%Therefore it is also placed there and commented out here.
%</package>
%<*discard>
%\fi
%    \begin{macrocode}
\ifdefined\childdocmain\endinput\fi
%    \end{macrocode}
%\iffalse
%</discard>
%<*package>
%\fi
%
% \macro{\ifchilddoc}
% \macro{\ifchilddocmanual}
% The conditional |\ifchilddoc| tells whether a
% child (true) or main (false) document is being compiled.
% The conditional |\ifchilddocmanual| tells whether
% the |\includeonly| mechanism is used (false) or
% the selection of child files must be performed manually (true).
% The definitions initialise to false:
%    \begin{macrocode}
\newif\ifchilddoc
\newif\ifchilddocmanual
%    \end{macrocode}

% \macro{\childdocname}
% \macro{\childdocjob}
% The macro |\childdocname| stores the name of the main document
% to be compiled. The macro |\childdocjob| stores the name of
% the document on which the \LaTeX{} compiler was originally invoked.
% The content of |\jobname| cannot be compared
% to filenames specified in the source due to different catcodes.
% The following code rescans |\jobname|, stores the result
% in |\childdocname| and saves a copy in |\childdocjob|:
%    \begin{macrocode}
\edef\childdocname{\scantokens\expandafter{\jobname\noexpand}}
\let\childdocjob\childdocname
%    \end{macrocode}

% \macro{\childdocdisable}
% The macro |\childdocdisable| prevents the main file
% from being processed more than once.
% At this stage, the main document command |\childdocmain|
% is assumed to be called once again where it should do nothing.
% Any subsequent call to it should prevent
% a secondary processing of the main document
% It overwrites the forwarding commands
% |\childdocof| and |\childdocforward|
% with empty macros to prevent further inclusions of the main document:
%    \begin{macrocode}
\newcommand{\childdocdisable}
{
  \renewcommand{\childdocmain}[1]{\renewcommand{\childdocmain}[1]{\endinput}}
  \renewcommand{\childdocof}[1]{}
  \renewcommand{\childdocby}[2][]{}
  \renewcommand{\childdocforward}[2][]{}
  \renewcommand{\childdocdisable}{}
}
%    \end{macrocode}

% \macro{\childdocmain}
% The macro |\childdocmain| is to be called at the top of the main file
% with nothing or the main filename (without extension) as argument.
% First, it breaks loops.
% If the argument is not empty and does not match |\childdocname|
% (which is set by the first inclusion of |childdoc.def|),
% |\ifchilddoc| is set to true, |\includeonly| is applied to the child file
% and |\jobname| is set to the main file
% (for proper handling of |.aux| files):
%    \begin{macrocode}
\newcommand{\childdocmain}[1]
{
  \childdocdisable\childdocmain{}
  \if?#1?\else
    \begingroup
      \def\childdoctmp{#1}
      \ifx\childdoctmp\childdocname
        \def\childdoctmp{}
      \else
        \def\childdoctmp
        {
          \childdoctrue
          \includeonly{\childdocname}
          \def\childdocjob{#1}
          \def\jobname{#1}
        }
      \fi
      \expandafter
    \endgroup
    \childdoctmp
  \fi
}
%    \end{macrocode}

% \macro{\childdocof}
% The command |\childdocof| redirects
% compilation to the main file |#1|.
%    \begin{macrocode}
\newcommand{\childdocof}[1]
{
  \childdocdisable
  \childdoctrue
  \includeonly{\childdocname}
  \def\jobname{#1}
  \def\childdocjob{#1}
  \input{#1}
}
%    \end{macrocode}

% \macro{\childdocby}
% The command |\childdocby| ....
%    \begin{macrocode}
\newcommand{\childdocby}[2][]
{
  \childdocdisable
  \childdoctrue
  \childdocmanualtrue
  \if?#1?\else
    \def\jobname{#2}
  \fi
  \def\childdocjob{#2}
  \input{#2}
  \endinput
}
%    \end{macrocode}

% \macro{\childdocforward}
% The command |\childdocforward| redirects
% compilation to the main file or
% (if the optional argument is given) a child file.
% Parameters are set as if the main file
% or a child file starting with |\childdocof| was compiled.
% Then compilation is handed over to the main file:
%    \begin{macrocode}
\newcommand{\childdocforward}[2][]
{
  \begingroup
    \if?#1?
      \def\childdoctmp
      {
        \def\childdocname{#2}
        \def\childdocjob{#2}
        \def\jobname{#2}
        \input{#2}
        \endinput
      }
    \else
      \def\childdoctmp
      {
        \childdocdisable
        \def\childdocname{#2}
        \childdoctrue
        \includeonly{#2}
        \def\childdocjob{#1}
        \def\jobname{#1}
        \input{#1}
        \endinput
      }
    \fi
    \expandafter
  \endgroup
  \childdoctmp
}
%    \end{macrocode}

% \macro{\childdocforwardprefix}
% The command |\childdocforwardprefix| redirects
% compilation to the main or a child file by means of a pattern.
% The prefix |#1| in the current filename is replaced by |#2|
% and the suffix of the current filename is kept
% (it is assumed that the filename does not contain the substring `|~~~|'
% which is used as a delimiter).
% Compilation is handed over to the new file by |\childdocforward|:
%    \begin{macrocode}
\newcommand{\childdocforwardprefix}[3][]
{
  \begingroup
    \def\childdocextract #2##1~~~{\def\childdoctmp{\childdocforward[#1]{#3##1}}}
    \expandafter\childdocextract\childdocname~~~
    \expandafter
  \endgroup
  \childdoctmp
}
%    \end{macrocode}

% \macro{\childdoc}
% The deprecated macro |\childdoc| is a legacy version of |\childdocmain|:
%    \begin{macrocode}
\newcommand{\childdoc}{\childdocmain}
%    \end{macrocode}

% \macro{\childdocredirect}
% The deprecated macro |\childdocredirect| is a legacy version
% of |\childdocforward| and |\childdocforwardprefix|:
%    \begin{macrocode}
\newcommand{\childdocredirect}[2][]
{
  \begingroup
    \if?#1?
      \def\childdoctmp{\childdocforward{#2}}
    \else
      \def\childdoctmp{\childdocforwardprefix{#1}{#2}}
    \fi
    \expandafter
  \endgroup
  \childdoctmp
}
%    \end{macrocode}

%\iffalse
%</package>
%\fi
%
\endinput
|\\
|\childdocby{|\textit{main}|}|\\
\end{tabular}
\end{center}
%
Both forms have slightly different effects as described above.
The main file is prepared as usual, see \secref{sec:include}.

%%%%%%%%%%%%%%%%%%%%%%%%%%%%%%%%%%%%%%%%%%%%%%%%%%%%%%%%%%%%%%%%%%%%%%%%%%%%%%%%
\subsection{Legacy Detection}
\label{sec:detection}

The directive |\childdocmain| in the main file can detect
whether the complete document or merely a child is to be compiled
even without using the directive |\childdocof|.
This method is deprecated because it is less robust
and there is no compelling reason to use it;
it is merely provided for backward compatibility
and it may be removed in future versions.

If the detection mechanism is to be used,
it is mandatory to correctly specify
the filename of the main file as the argument of |\childdocmain|:
%
\begin{center}
\begin{tabular}{l}
|% \iffalse
%
% childdoc.dtx Copyright (C) 2017-2018 Niklas Beisert
%
% This work may be distributed and/or modified under the
% conditions of the LaTeX Project Public License, either version 1.3
% of this license or (at your option) any later version.
% The latest version of this license is in
%   http://www.latex-project.org/lppl.txt
% and version 1.3 or later is part of all distributions of LaTeX
% version 2005/12/01 or later.
%
% This work has the LPPL maintenance status `maintained'.
%
% The Current Maintainer of this work is Niklas Beisert.
%
% This work consists of the files childdoc.dtx and childdoc.ins
% and the derived files childdoc.def and cdocsamp.tex with
% cdocsch1.tex, cdocsch2.tex, cdocsdrf.tex, cdocsfn1.tex, cdocsfn2.tex.
%
%<package>\ifdefined\childdocmain\endinput\fi
%<package>\ProvidesFile{childdoc.def}[2018/12/30 v2.0 child document driver]
%<samplemain>\ProvidesFile{cdocsamp.tex}[2018/12/30 v2.0 sample for childdoc]
%<*driver>
%\ProvidesFile{childdoc.drv}[2018/12/30 v2.0 childdoc reference manual file]
\PassOptionsToClass{10pt,a4paper}{article}
\documentclass{ltxdoc}

\usepackage[margin=35mm]{geometry}
\usepackage{hyperref}
\usepackage{hyperxmp}
\usepackage[usenames]{color}

\hypersetup{colorlinks=true}
\hypersetup{pdfstartview=FitH}
\hypersetup{pdfpagemode=UseNone}
\hypersetup{pdfsource={}}
\hypersetup{pdflang={en-UK}}
\hypersetup{pdfcopyright={Copyright 2017-2018 Niklas Beisert.
  This work may be distributed and/or modified under the
  conditions of the LaTeX Project Public License, either version 1.3
  of this license or (at your option) any later version.}}
\hypersetup{pdflicenseurl={http://www.latex-project.org/lppl.txt}}
\hypersetup{pdfcontactaddress={ETH Zurich, ITP, HIT K,
  Wolfgang-Pauli-Strasse 27}}
\hypersetup{pdfcontactpostcode={8093}}
\hypersetup{pdfcontactcity={Zurich}}
\hypersetup{pdfcontactcountry={Switzerland}}
\hypersetup{pdfcontactemail={nbeisert@itp.phys.ethz.ch}}
\hypersetup{pdfcontacturl={http://people.phys.ethz.ch/\xmptilde nbeisert/}}

\newcommand{\secref}[1]{\hyperref[#1]{section \ref*{#1}}}

\parskip1ex
\parindent0pt
\let\olditemize\itemize
\def\itemize{\olditemize\parskip0pt}

\begin{document}

\title{The \textsf{childdoc} Package}
\hypersetup{pdftitle={The childdoc Package}}
\author{Niklas Beisert\\[2ex]
  Institut f\"ur Theoretische Physik\\
  Eidgen\"ossische Technische Hochschule Z\"urich\\
  Wolfgang-Pauli-Strasse 27, 8093 Z\"urich, Switzerland\\[1ex]
  \href{mailto:nbeisert@itp.phys.ethz.ch}
  {\texttt{nbeisert@itp.phys.ethz.ch}}}
\hypersetup{pdfauthor={Niklas Beisert}}
\hypersetup{pdfsubject={Manual for the LaTeX2e Package childdoc}}
\date{30 December 2018, \textsf{v2.0}}
\maketitle

\begin{abstract}\noindent
\textsf{childdoc} is a \LaTeXe{} package
that enables the direct compilation
of document sections included by |\include|
to individual files.
\end{abstract}

\begingroup
\parskip0ex
\tableofcontents
\endgroup

%%%%%%%%%%%%%%%%%%%%%%%%%%%%%%%%%%%%%%%%%%%%%%%%%%%%%%%%%%%%%%%%%%%%%%%%%%%%%%%%
%%%%%%%%%%%%%%%%%%%%%%%%%%%%%%%%%%%%%%%%%%%%%%%%%%%%%%%%%%%%%%%%%%%%%%%%%%%%%%%%
\section{Introduction}

\LaTeX{} provides a mechanism to structure a large document (such as a book)
into a main file and several child files (containing the chapters)
using the |\include| command.
This mechanism is beneficial for documents
which span hundreds of pages in order to
make the source file(s) more manageable.
Moreover, compilation can be restricted to
selected child files by means of the |\includeonly| command.
The latter feature can be used to reduce the compilation time while editing
(this was significantly more useful in the earlier days of \LaTeX{})
or to generate a smaller document which is easier to navigate.
Another application of |\includeonly| is to generate
documents consisting of selected parts of the complete document.

However, there are a few drawbacks of the plain |\include| mechanism:
\begin{itemize}
\item
The child files cannot be compiled on their own,
they can only be compiled via the main file.
A naive editing environment
(such as a text editor with an option
to have the current file processed by \LaTeX)
may require one to switch to the main file before compiling;
attempting to compile the child file produces errors.
\item
The main file must be modified (each time)
to adjust the |\includeonly| command
to the present needs. This easily leaves the main file in a messy state.
\item
The generated document will always carry the filename
of the main document. This is inconvenient if
several child files are to be compiled and
to be kept for distribution.
\end{itemize}

The present package provides a simple interface
to make child files individually compilable by \LaTeX{}.
Compiling a child file then has the same effect as compiling
the main file with an |\includeonly| command
to select the appropriate child.
Moreover the generated document will carry the name of the child
rather than the main file.
This resolves all three above issues.

This feature is meant to make the editing of books,
thesis documents and lecture notes somewhat more convenient.
However, the package can also be used efficiently for
composing a series of documents (such as exercise sheets)
which are typically distributed individually.
It then assists the author in generating the individual documents
(potentially in different versions)
as well as a document containing the collected series.
Another application is in developing style files
or other kinds of included material
where compilation of the style file could redirect
to a sample or test file.

%%%%%%%%%%%%%%%%%%%%%%%%%%%%%%%%%%%%%%%%%%%%%%%%%%%%%%%%%%%%%%%%%%%%%%%%%%%%%%%%
%%%%%%%%%%%%%%%%%%%%%%%%%%%%%%%%%%%%%%%%%%%%%%%%%%%%%%%%%%%%%%%%%%%%%%%%%%%%%%%%
\section{Usage}

First of all, the package \textsf{childdoc} is \emph{not} a standard
\LaTeXe{} |.sty| style file! Therefore it needs to be invoked in
a non-standard way.

%%%%%%%%%%%%%%%%%%%%%%%%%%%%%%%%%%%%%%%%%%%%%%%%%%%%%%%%%%%%%%%%%%%%%%%%%%%%%%%%
\subsection{Included Files}
\label{sec:include}

%%%%%%%%%%%%%%%%%%%%%%%%%%%%%%%%%%%%%%%%
\DescribeMacro{\childdocmain}
To use the package, add the commands
\begin{center}
\begin{tabular}{l}
|\input{childdoc.def}|\\
|\childdocmain{}|\\
\end{tabular}
\end{center}
at the very top of the main \LaTeX{} file,
in particular \emph{before} the |\documentclass| statement!
The argument of |\childdocmain| should be left empty
(but it must be present).

%%%%%%%%%%%%%%%%%%%%%%%%%%%%%%%%%%%%%%%%
\DescribeMacro{\childdocof}
Furthermore, add the commands
\begin{center}
\begin{tabular}{l}
|\input{childdoc.def}|\\
|\childdocof{|\textit{main}|}|\\
\end{tabular}
\end{center}
at the top of every child file \textit{child}
which is included by |\include{|\textit{child}|}|
from within the main file
(or at least for those files to be compiled individually).
The argument \textit{main} must be the filename of the main file.

There are a couple of
considerations in setting up the main and child documents:

%%%%%%%%%%%%%%%%%%%%%%%%%%%%%%%%%%%%%%%%
\paragraph{Restrictions.}

Please note the following restrictions:
\begin{itemize}
\item
|\childdocmain| must be called with one argument \textit{main}
to ensure compatibility with earlier version of the package.
It must either be empty (|\childdocmain{}|)
or precisely match the filename of the main file in which it is specified.
See \secref{sec:detection} for further information.
\item
The filename \textit{main} must be specified without the |.tex| extension.
\item
The filename \textit{main} is case sensitive
(even in case-insensitive file systems)
due to internal string comparison.
\item
The argument \textit{main} should be fully expanded, it cannot be a macro.
\item
Subdirectories and special characters should be avoided in filenames.
\item
The command |\childdocmain{|\textit{main}|}| must be followed by a whitespace.
It should not be followed immediately by another command
or by a comment mark `|%|'.
This is because the \TeX{} parser reads the token immediately following
the argument of |\childdocmain| and puts it
at the beginning of every child section;
however, a white\-space is ignored.
\end{itemize}

%%%%%%%%%%%%%%%%%%%%%%%%%%%%%%%%%%%%%%%%
\paragraph{Content of Main File.}

It is advisable to place all content in the child files included by |\include|.
Any output contained in the main file will appear in all child documents
unless suppressed manually;
it cannot be suppressed automatically by the |\includeonly| directive
and thus should normally be avoided.
A method to include some content in the main file
by means of conditional processing is described in \secref{sec:conditional}.

%%%%%%%%%%%%%%%%%%%%%%%%%%%%%%%%%%%%%%%%
\paragraph{Page Numbering.}

When only a part of the document is compiled,
the appropriate numbering of pages
(as well as other status parameters)
is determined from the |.aux| files.
The latter contain information from previous passes.
However this information needs to propagate through
all intermediate child documents.
Therefore the page numbering in child documents may well
be inconsistent until the complete document is compiled at least once.

A useful (if unconventional) way to always ensure a consistent
page numbering is to restart the numbering in each child document
and denote the pages by `\textit{child}|.|\textit{page}'
where \textit{child} represents the chapter/section number of the child file.
This can be achieved by the command
|\numberwithin{page}{|\textit{child}|}|
of the \textsf{amsmath} package
where \textit{child} can be |chapter| or |section|
depending on the chosen structuring.
Alternatively, one can modify the macro |\thepage| appropriately
and reset the counter |page| at the start of each child file.

%%%%%%%%%%%%%%%%%%%%%%%%%%%%%%%%%%%%%%%%%%%%%%%%%%%%%%%%%%%%%%%%%%%%%%%%%%%%%%%%
\subsection{Conditional Processing}
\label{sec:conditional}

The package provides a mechanism to compile different versions
of a document. To customise the versions further some conditional processing
can come in handy to distinguish which version is being compiled.
The package provides two macros to describe the compilation context:

%%%%%%%%%%%%%%%%%%%%%%%%%%%%%%%%%%%%%%%%
\DescribeMacro{\ifchilddoc}
The conditional |\ifchilddoc| distinguishes between the compilation of
child documents and the main document:
%
\begin{center}
|\ifchilddoc |\textit{child-code}| |[|\||else |\textit{main-code}]| \||fi|
\end{center}

%%%%%%%%%%%%%%%%%%%%%%%%%%%%%%%%%%%%%%%%
\DescribeMacro{\childdocname}
\DescribeMacro{\childdocjob}
The macro |\childdocname| contains the filename (without extension)
of the main or child file being processed.
Note that |\childdocjob| will always contain the name of the main file.

%%%%%%%%%%%%%%%%%%%%%%%%%%%%%%%%%%%%%%%%
\paragraph{Title Page.}

Conditional processing can be used to include a title or banner page
in the main document when proper precautions are taken.
Importantly, the code in the main file should ensure that the page counter
(as well as other status parameters which are stored in the |.aux| files)
takes the same value after the conditional processing.
Otherwise the page numbers may take divergent values
depending on which part is compiled.

For example, a title page could be declared by:
%
\begin{center}
\begin{tabular}{l}
|\ifchilddoc\||else|\\
|\addtocounter{page}{-1}|\\
\textit{code for title page}\\
|\newpage|\\
|\||fi|
\end{tabular}
\end{center}
%
A banner page for the child documents can be generated by:
%
\begin{center}
\begin{tabular}{l}
|\ifchilddoc|\\
|\addtocounter{page}{-1}|\\
\textit{code for banner page}\\
|\newpage|\\
|\||fi|
\end{tabular}
\end{center}
%
Here one could write a message such as:
\begin{center}
|This is the part \childdocname{} of \childdocjob{}.|
\end{center}

%%%%%%%%%%%%%%%%%%%%%%%%%%%%%%%%%%%%%%%%%%%%%%%%%%%%%%%%%%%%%%%%%%%%%%%%%%%%%%%%
\subsection{Flags}
\label{sec:flags}

The package makes it easy to generate different versions
of the main or child documents.
To this end compilation flags can be defined
and assigned different default values.
They will be particularly useful in conjunction
with the forwarding mechanism described in \secref{sec:forward}.

For example, it may be useful to have a flag |\version|
which can be set to |draft| or |final|.
The document source will contain some conditional code
depending on the value of |\version|.
Suppose further, the flag should default to |final| for the main file
and to |draft| for child files
which is a natural assignment for editing the document.
This is achieved by placing the following code
in the preamble of the main document
(below the |\childdocmain| directive):
%
\begin{center}
\begin{tabular}{l}
|\ifchilddoc|\\
|\providecommand{\version}{draft}|\\
|\||else|\\
|\providecommand{\version}{final}|\\
|\||fi|
\end{tabular}
\end{center}
%
The definition by |\providecommand| makes sure
that previous definitions are not overwritten.
Further statements |\providecommand{\version}{...}|
can thus be added before the above code to override it.

For the main file, one might add a line
(between |\childdocmain| and the above block)
%
\begin{center}
|%\ifchilddoc\||else\providecommand{\version}{draft}\||fi|
\end{center}
%
which can be uncommented to produce a draft version.
Likewise one can add a line to the very top of a child file
(above the |\childdocof{|\textit{main}|}| directive)
%
\begin{center}
|%\providecommand{\version}{final}|
\end{center}
%
which can be uncommented to produce the final version of this child document.

%%%%%%%%%%%%%%%%%%%%%%%%%%%%%%%%%%%%%%%%%%%%%%%%%%%%%%%%%%%%%%%%%%%%%%%%%%%%%%%%
\subsection{Forwarding}
\label{sec:forward}

Different versions of the main or child documents
using compilation flags as described in \secref{sec:flags}
can be (permanently) stored in different files
for convenient compilation, viewing and distribution.
To this end, the package defines a command
to pass on compilation to a different file:

%%%%%%%%%%%%%%%%%%%%%%%%%%%%%%%%%%%%%%%%
\DescribeMacro{\childdocforward}
The command |\childdocforward| redirects processing to
another source file:
%
\begin{center}
\begin{tabular}{l}
|\input{childdoc.def}|\\
|\childdocforward[|\textit{main}|]{|\textit{dest}|}|\\
\end{tabular}
\end{center}
%
The argument \textit{dest} is the destination file
(without extension).
It should be the main file or one of the child files.
Note that further \textsf{childdoc} directives
such as |\childdocof| and |\childdocforward|
in the indicated file will be processed in this form.
The optional argument \textit{main}
passes on directly to the main file \textit{main}
while pretending to compile the child \textit{dest}.
This form behaves as if \textit{dest}
issues |\childdocof{|\textit{main}|}| right away,
and no further \textsf{childdoc} directives will be processed.

%%%%%%%%%%%%%%%%%%%%%%%%%%%%%%%%%%%%%%%%
\DescribeMacro{\...prefix}
In the alternative form |\childdocforwardprefix|,
%
\begin{center}
\begin{tabular}{l}
|\input{childdoc.def}|\\
|\childdocforwardprefix[|\textit{main}|]{|\textit{prefix}|}{|\textit{dest}|}|
\end{tabular}
\end{center}
%
the destination file is determined by a pattern
depending on the current file:
To make this work, the current file must be called
`{\textit{prefix}\hspace{0.2em}\textit{suffix}}'
with \textit{prefix} matching precisely the argument.
Processing is then passed on to the file
`{\textit{dest}\hspace{0.2em}\textit{suffix}}'.
Surely, the same effect is achieved by
directly specifying the
argument `{\textit{dest}\hspace{0.2em}\textit{suffix}}'
in the first form.
However, that requires to set up a different file
for each child. With the alternative form of the command
all these files can have exactly the same content
which simplifies setting them up and maintaining them.

For example, the following file |draft.tex|
with a compilation flag |\version| as described in \secref{sec:flags}
compiles the main document as a draft:
%
\begin{center}
\begin{tabular}{l}
|\def\version{draft}|\\
|\input{childdoc.def}|\\
|\childdocforward{|\textit{main}|}|
\end{tabular}
\end{center}
%
Likewise, the following files |final|\textit{nn}|.tex|
compile the final version of the child document
|child|\textit{nn}|.tex|:
%
\begin{center}
\begin{tabular}{l}
|\def\version{final}|\\
|\input{childdoc.def}|\\
|\childdocforwardprefix{final}{child}|
\end{tabular}
\end{center}
%

Note that when several versions of a main file and/or of each child file
are to be generated, it may be convenient to set up a |Makefile| or
shell script to automatise the process.

%%%%%%%%%%%%%%%%%%%%%%%%%%%%%%%%%%%%%%%%%%%%%%%%%%%%%%%%%%%%%%%%%%%%%%%%%%%%%%%%
\subsection{Command Line Processing}
\label{sec:commandline}

The effect of redirection files can also be achieved by invoking
the \LaTeX{} compiler with a more elaborate command line.
Most conveniently this should be done as part
of a shell script or a |Makefile|.

When using \textsf{childdoc} in the main file, the following
command lines effectively perform a redirection
(note that depending on the shell being used,
backslashes may have to be doubled: `|\|' $\to$ `|\\|'):
%
\begin{center}
|... -jobname "|\textit{target}|" |\\|"|[\textit{flags}]%
|\input{childdoc.def}\childdocforward[|\textit{main}|]{|\textit{dest}|}"|
\end{center}
%
Here \textit{target} is the name of the output file,
\textit{main} is the name of the main file
and \textit{dest} is the name of the main or child file to be processed
(all filenames without extensions).
The optional argument \textit{main} can be omitted
if \textit{main} matches \textit{dest}.
Optionally, compilation \textit{flags} can be defined via |\def| commands.
This command line makes the \TeX{} engine believe
it is compiling the file \textit{target}
whose content is specified as the latter parameter.
The provided code then forwards the processing to
\textit{main} or \textit{dest} as described in \secref{sec:forward}.

%%%%%%%%%%%%%%%%%%%%%%%%%%%%%%%%%%%%%%%%%%%%%%%%%%%%%%%%%%%%%%%%%%%%%%%%%%%%%%%%
\subsection{Include by Input}
\label{sec:input}

Including child documents by |\include| has some restrictions by design.
Most notably, the content of a child document always occupies
its own set of pages; pages cannot be shared between child documents.
Usually, this behaviour makes perfect sense
because each child document contain an essential part of the document.
However, in some situations it may be desirable to compose
a document from a collection of parts
without having mandatory page breaks between then.
For this case, the package
provides a mechanism to include parts
by |\input| which can also be processed individually.
However, by construction this mechanism
requires manual handling of the content to be output.

%%%%%%%%%%%%%%%%%%%%%%%%%%%%%%%%%%%%%%%%
\DescribeMacro{\ifchilddocmanual}
The main file should be prepared as usual, see \secref{sec:include}.
However, the document body must make a distinction
between processing of an individual part and of the main document, e.g.:
%
\begin{center}
\begin{tabular}{l}
|\ifchilddocmanual|\\
|\input{\childdocname}|\\
|\||else|\\
\textit{document body with }|\input{|\textit{part}|}|\\
|\||fi|
\end{tabular}
\end{center}
%
The conditional |\ifchilddocmanual| is true whenever
a part to be included by |\input| is being compiled,
and the name of the part is stored in |\childdocname|.

%%%%%%%%%%%%%%%%%%%%%%%%%%%%%%%%%%%%%%%%
\DescribeMacro{\childdocby}
Each part to be included by |\input| should start with:
%
\begin{center}
\begin{tabular}{l}
|\input{childdoc.def}|\\
|\childdocby{|\textit{main}|}|\\
\end{tabular}
\end{center}
%
The directive |\childdocby| is similar to |\childdocof|
described in \secref{sec:include},
but the subsequent selection of content must be done manually.
To that end, both |\ifchilddoc| and |\ifchilddocmanual|
will be true upon processing of a part,
and the name of the part is stored in |\childdocname|.
Note that |\jobname| will be set to the filename of the current part
so that each part receives an individual |.aux| file
that does not interfere with the |.aux| file(s) of the main document.
This behaviour can be altered by the alternative form
|\childdocby[*]{|\textit{main}|}| (with a non-empty optional argument)
which uses the |.aux| file of the main document
by setting |\jobname| to \textit{main}.

%%%%%%%%%%%%%%%%%%%%%%%%%%%%%%%%%%%%%%%%%%%%%%%%%%%%%%%%%%%%%%%%%%%%%%%%%%%%%%%%
\subsection{Driver Development}
\label{sec:driver}

The \textsf{childdoc} mechanism can also be use for the development
of definition files such as \LaTeX{} styles or classes.
This case differs from the above setup with multiple parts
included by |\include| in that no |\includeonly| should be invoked.
This can be achieved by starting the include file
(before |\ProvidesPackage|) with:
%
\begin{center}
\begin{tabular}{l}
|\input{childdoc.def}|\\
|\childdocforward{|\textit{main}|}|\\
\end{tabular}
\end{center}
%
or alternatively with:
%
\begin{center}
\begin{tabular}{l}
|\input{childdoc.def}|\\
|\childdocby{|\textit{main}|}|\\
\end{tabular}
\end{center}
%
Both forms have slightly different effects as described above.
The main file is prepared as usual, see \secref{sec:include}.

%%%%%%%%%%%%%%%%%%%%%%%%%%%%%%%%%%%%%%%%%%%%%%%%%%%%%%%%%%%%%%%%%%%%%%%%%%%%%%%%
\subsection{Legacy Detection}
\label{sec:detection}

The directive |\childdocmain| in the main file can detect
whether the complete document or merely a child is to be compiled
even without using the directive |\childdocof|.
This method is deprecated because it is less robust
and there is no compelling reason to use it;
it is merely provided for backward compatibility
and it may be removed in future versions.

If the detection mechanism is to be used,
it is mandatory to correctly specify
the filename of the main file as the argument of |\childdocmain|:
%
\begin{center}
\begin{tabular}{l}
|\input{childdoc.def}|\\
|\childdocmain{|\textit{main}|}|\\
\end{tabular}
\end{center}
%
If |\jobname| does not match the argument \textit{main} of |\childdocmain|,
it is assumed that |\jobname| points to the child file to be compiled.
When using |\childdocmain| with the main file specified as argument,
it suffices to start a child file
with just |\input{|\textit{main}|}|
without loading of the package and using |\childdocof|.
If instead all processing is done
with the appropriate \textsf{childdoc} directives,
the argument of \textit{main} of |\childdocmain| can be empty.

An alternative version of the command line processing described
in \secref{sec:commandline} using the detection mechanism reads:
%
\begin{center}
|... -jobname "|\textit{target}|" "|[\textit{flags}]%
[|\def\jobname{|\textit{dest}|}|]|\input{|\textit{main}|}"|
\end{center}

%%%%%%%%%%%%%%%%%%%%%%%%%%%%%%%%%%%%%%%%%%%%%%%%%%%%%%%%%%%%%%%%%%%%%%%%%%%%%%%%
\subsection{Manual Code}
\label{sec:manual}

In case one cannot be certain whether the definitions file |childdoc.def|
is installed on the target \TeX{} distribution
and one prefers not to ship it,
it is conceivable to paste a few relevant commands into the sources.

To that end, drop all statements |\input{childdoc.def}|
and perform the replacements as outlined below.
Instead of |\childdocmain{|\textit{main}|}| add the following code
to the top of the main file:
%
\begin{center}
\begin{tabular}{l}
|\||ifdefined\childdocname\endinput\||fi\newif\ifchilddoc|\\
|\edef\childdocname{\scantokens\expandafter{\jobname\noexpand}}|\\
|\def\childdocmain{|\textit{main}|}\||ifx\childdocmain\childdocname\||else|\\
|\childdoctrue\includeonly{\childdocname}\let\jobname\childdocmain\||fi|\\
\end{tabular}
\end{center}
%
Instead of |\childdocof{|\textit{main}|}| just include the main file
at the top of each child file:
%
\begin{center}
|\input{|\textit{main}|}|
\end{center}
%
A simple redirection |\childdocforward{|\textit{dest}|}| is achieved by:
%
\begin{center}
|\def\jobname{|\textit{dest}|}\input{\jobname}|
\end{center}
%
The redirection with prefix
|\childdocforwardprefix[|\textit{prefix}|]{|\textit{dest}|}|
is accomplished by:
%
\begin{center}
\begin{tabular}{l}
|{\edef\jobname{\scantokens\expandafter{\jobname\noexpand}}|\\
|\def\redirectjob |\textit{prefix}|#1~~~{\gdef\jobname{|\textit{dest}|#1}}|\\
|\expandafter\redirectjob\jobname~~~}\input{\jobname}|
\end{tabular}
\end{center}

In an alternative approach,
child documents can be compiled by a specific command line
without additional code or specific definitions:
%
\begin{center}
|... -jobname "|\textit{target}|" "|[\textit{flags}]%
|\includeonly{|\textit{dest}|}\input{|\textit{main}|}"|
\end{center}
%

%%%%%%%%%%%%%%%%%%%%%%%%%%%%%%%%%%%%%%%%%%%%%%%%%%%%%%%%%%%%%%%%%%%%%%%%%%%%%%%%
%%%%%%%%%%%%%%%%%%%%%%%%%%%%%%%%%%%%%%%%%%%%%%%%%%%%%%%%%%%%%%%%%%%%%%%%%%%%%%%%
\section{Information}

%%%%%%%%%%%%%%%%%%%%%%%%%%%%%%%%%%%%%%%%%%%%%%%%%%%%%%%%%%%%%%%%%%%%%%%%%%%%%%%%
\subsection{Copyright}

Copyright \copyright{} 2017--2018 Niklas Beisert

This work may be distributed and/or modified under the
conditions of the \LaTeX{} Project Public License, either version 1.3
of this license or (at your option) any later version.
The latest version of this license is in
  \url{http://www.latex-project.org/lppl.txt}
and version 1.3 or later is part of all distributions of \LaTeX{}
version 2005/12/01 or later.

This work has the LPPL maintenance status `maintained'.

The Current Maintainer of this work is Niklas Beisert.

This work consists of the files |README.txt|, |childdoc.ins| and |childdoc.dtx|
as well as the derived files |childdoc.def|, |cdocsamp.tex|
with |cdocsch1.tex|, |cdocsch2.tex|, |cdocspt3.tex|, |cdocspt4.tex|,
|cdocsdrf.tex|, |cdocsfn1.tex|, |cdocsfn2.tex|
as well as |childdoc.pdf|.

%%%%%%%%%%%%%%%%%%%%%%%%%%%%%%%%%%%%%%%%%%%%%%%%%%%%%%%%%%%%%%%%%%%%%%%%%%%%%%%%
\subsection{Files and Installation}

The package consists of the files:
%
\begin{center}
\begin{tabular}{ll}
    |README.txt|   & readme file \\
    |childdoc.ins| & installation file \\
    |childdoc.dtx| & source file \\
    |childdoc.def| & definition file \\
    |cdocsamp.tex| & sample main file \\
    |cdocsch1.tex| & sample include file \\
    |cdocsch2.tex| & sample include file \\
    |cdocspt3.tex| & sample part file \\
    |cdocspt4.tex| & sample part file \\
    |cdocsdrf.tex| & sample redirection file \\
    |cdocsfn1.tex| & sample redirection file \\
    |cdocsfn2.tex| & sample redirection file \\
    |childdoc.pdf| & manual
\end{tabular}
\end{center}
%
The distribution consists of the files
|README.txt|, |childdoc.ins| and |childdoc.dtx|.
%
\begin{itemize}
\item
Run (pdf)\LaTeX{} on |childdoc.dtx|
to compile the manual |childdoc.pdf| (this file).
\item
Run \LaTeX{} on |childdoc.ins| to create the definitions file |childdoc.def|
and the sample |cdocsamp.tex| with include files
|cdocsch1.tex|, |cdocsch2.tex|, |cdocspt3.tex|, |cdocspt4.tex|,
|cdocsdrf.tex|, |cdocsfn1.tex|, |cdocsfn2.tex|.
Then copy the file |childdoc.def| to an appropriate directory of your \LaTeX{}
distribution, e.g.\ \textit{texmf-root}|/tex/latex/childdoc|.
\end{itemize}

%%%%%%%%%%%%%%%%%%%%%%%%%%%%%%%%%%%%%%%%%%%%%%%%%%%%%%%%%%%%%%%%%%%%%%%%%%%%%%%%
\subsection{Related CTAN Packages}

There are several other packages which offer a similar functionality:
%
\begin{itemize}
\item
The packages
\href{http://ctan.org/pkg/docmute}{\textsf{docmute}},
\href{http://ctan.org/pkg/includex}{\textsf{includex}} and
\href{http://ctan.org/pkg/standalone}{\textsf{standalone}}
provide commands to include only the document body of
a child file thus allowing both files to be compiled individually.
\item
The packages \href{http://ctan.org/pkg/subdocs}{\textsf{subdocs}}
and \href{http://ctan.org/pkg/subfiles}{\textsf{subfiles}}
provide structures in which the main and child documents can be
encapsulated and allowing them to be compiled individually.
The inclusion mechanism is different from the conventional |\include|.
\item
The package \href{http://ctan.org/pkg/combine}{\textsf{combine}}
is an elaborate solution to combine several documents into one.
\end{itemize}
%
See also the CTAN topic \href{http://ctan.org/topic/subdocs}{\textsf{subdocs}}
for further related packages.
The present package differs from the above solutions in that
a document structure constructed with the conventional |\include| mechanism
just needs two extra commands at the top of every file
such that all constituent files can be compiled individually.

%%%%%%%%%%%%%%%%%%%%%%%%%%%%%%%%%%%%%%%%%%%%%%%%%%%%%%%%%%%%%%%%%%%%%%%%%%%%%%%%
%\subsection{Feature Suggestions}
%
%The following is a list of features which may be useful for future
%versions of this package:
%%
%\begin{itemize}
%\item
%\ldots
%\end{itemize}

%%%%%%%%%%%%%%%%%%%%%%%%%%%%%%%%%%%%%%%%%%%%%%%%%%%%%%%%%%%%%%%%%%%%%%%%%%%%%%%%
\subsection{Revision History}

%%%%%%%%%%%%%%%%%%%%%%%%%%%%%%%%%%%%%%%%
\paragraph{v2.0:} 2018/12/30

\begin{itemize}
\item
immediate forward processing
\item
added |\childdocby| mechanism
\item
manual restructured
\end{itemize}

%%%%%%%%%%%%%%%%%%%%%%%%%%%%%%%%%%%%%%%%
\paragraph{v1.6:} 2018/01/17

\begin{itemize}
\item
application for development of include files
\item
corrections to manual
\end{itemize}

%%%%%%%%%%%%%%%%%%%%%%%%%%%%%%%%%%%%%%%%
\paragraph{v1.5:} 2017/05/21

\begin{itemize}
\item
more complete structuring introduced
\item
|\childdocof| introduced
\item
|\childdoc| renamed to |\childdocmain|
\item
|\childredirect| renamed to |\childdocforward| and |\childdocforwardprefix|
and functionality expanded
\end{itemize}

%%%%%%%%%%%%%%%%%%%%%%%%%%%%%%%%%%%%%%%%
\paragraph{v1.0:} 2017/04/27

\begin{itemize}
\item
manual and install package
\item
first version published on CTAN
\end{itemize}

%%%%%%%%%%%%%%%%%%%%%%%%%%%%%%%%%%%%%%%%
\paragraph{v0.6:} 2017/04/26

\begin{itemize}
\item
redirection mechanism added
\end{itemize}

%%%%%%%%%%%%%%%%%%%%%%%%%%%%%%%%%%%%%%%%
\paragraph{v0.5:} 2017/04/26

\begin{itemize}
\item
functionality in definition file
\end{itemize}


%%%%%%%%%%%%%%%%%%%%%%%%%%%%%%%%%%%%%%%%%%%%%%%%%%%%%%%%%%%%%%%%%%%%%%%%%%%%%%%%
%%%%%%%%%%%%%%%%%%%%%%%%%%%%%%%%%%%%%%%%%%%%%%%%%%%%%%%%%%%%%%%%%%%%%%%%%%%%%%%%
%%%%%%%%%%%%%%%%%%%%%%%%%%%%%%%%%%%%%%%%%%%%%%%%%%%%%%%%%%%%%%%%%%%%%%%%%%%%%%%%
\appendix

\settowidth\MacroIndent{\rmfamily\scriptsize 000\ }

 \DocInput{childdoc.dtx}

\end{document}
%</driver>
% \fi
%
% %%%%%%%%%%%%%%%%%%%%%%%%%%%%%%%%%%%%%%%%%%%%%%%%%%%%%%%%%%%%%%%%%%%%%%%%%%%%%%
% %%%%%%%%%%%%%%%%%%%%%%%%%%%%%%%%%%%%%%%%%%%%%%%%%%%%%%%%%%%%%%%%%%%%%%%%%%%%%%
% \section{Sample}
%\iffalse
%<*samplemain>
%\fi
%
% The following presents a sample document
% with two chapters, two parts, a title page,
% a compile flag as well as three forwarding files to set the flag.
% It consists of eight |.tex| files:
% \begin{center}
% \begin{tabular}{ll}
% |cdocsamp.tex|&main file\\
% |cdocsch1.tex|&include file for chapter 1\\
% |cdocsch2.tex|&include file for chapter 2\\
% |cdocspt3.tex|&include file for part 3\\
% |cdocspt4.tex|&include file for part 4\\
% |cdocsdrf.tex|&forwarding file for main file in draft mode\\
% |cdocsfi1.tex|&forwarding file for final version of chapter 1\\
% |cdocsfi2.tex|&forwarding file for final version of chapter 2\\
% \end{tabular}
% \end{center}
% Each of the eight files can be compiled directly by the \LaTeX{} compiler.
%
% %%%%%%%%%%%%%%%%%%%%%%%%%%%%%%%%%%%%%%
% \paragraph{Main File.}
%
% The main file is called |cdocsamp.tex|.
%
% Load the \textsf{childdoc} definitions and
% declare the filename for the main document:
%    \begin{macrocode}
\input{childdoc.def}
\childdocmain{}
%    \end{macrocode}

% Optional override for |\version| flag:
%    \begin{macrocode}
%%\ifchilddoc\else\providecommand{\version}{draft}\fi
%    \end{macrocode}

% Define the default values for the |\version| flag
% (|final| for the main file and |draft| for childs):
%    \begin{macrocode}
\ifchilddoc
\providecommand{\version}{draft}
\else
\providecommand{\version}{final}
\fi
%    \end{macrocode}

% Load the standard document class:
%    \begin{macrocode}
\documentclass[12pt]{article}
%    \end{macrocode}

% Start the document body:
%    \begin{macrocode}
\begin{document}
%    \end{macrocode}

% Declare a title page.
% Print title, part of document being processed and version flag:
%    \begin{macrocode}
\addtocounter{page}{-1}
\begin{center}
{\LARGE\bfseries{}childdoc example\par}
\vspace{1cm}
\ifchilddoc
\ifchilddocmanual part\else chapter\fi:
`\childdocname' of `\childdocjob'\par
\else
main document: `\childdocjob'\par
\fi
version: \version\par
\end{center}
\newpage
%    \end{macrocode}

% Manually include selected file,
% otherwise process as usual:
%    \begin{macrocode}
\ifchilddocmanual
\section*{part `\childdocname'}
\input{\childdocname}
\else
%    \end{macrocode}

% Include the two chapters:
%    \begin{macrocode}
\include{cdocsch1}
\include{cdocsch2}
%    \end{macrocode}

% Include the two parts unless only chapters should be displayed:
%    \begin{macrocode}
\ifchilddoc\else
\section{part three}
\input{cdocspt3}
\section{part four}
\input{cdocspt4}
\fi
%    \end{macrocode}

% Process as usual until here:
%    \begin{macrocode}
\fi
%    \end{macrocode}

% End of document body:
%    \begin{macrocode}
\end{document}
%    \end{macrocode}
%\iffalse
%</samplemain>
%\fi
%
% %%%%%%%%%%%%%%%%%%%%%%%%%%%%%%%%%%%%%%
% \paragraph{Chapter Include Files.}
%
% The include files are called |cdocsch1.tex| and |cdocsch2.tex|.
%
%\iffalse
%<*samplechap1|samplechap2>
%\fi

% Optional override for |\version| flag:
%    \begin{macrocode}
%%\providecommand{\version}{final}
%    \end{macrocode}

% Include the main document:
%    \begin{macrocode}
\input{childdoc.def}
\childdocof{cdocsamp}
%    \end{macrocode}

%\iffalse
%</samplechap1|samplechap2>
%\fi
%
%\iffalse
%<*samplechap1>
%\fi
% Some text for chapter 1:
%    \begin{macrocode}
\section{one}
some text in chapter one
%    \end{macrocode}

%\iffalse
%</samplechap1>
%\fi
% Some text for chapter 2:
%\iffalse
%<*samplechap2>
%\fi
%    \begin{macrocode}
\section{two}
more text in chapter two
%    \end{macrocode}

%\iffalse
%</samplechap2>
%\fi
%
% %%%%%%%%%%%%%%%%%%%%%%%%%%%%%%%%%%%%%%
% \paragraph{Part Include Files.}
%
% The include files are called |cdocspt3.tex| and |cdocspt4.tex|.
%
%\iffalse
%<*samplepart3|samplepart4>
%\fi

% Optional override for |\version| flag:
%    \begin{macrocode}
%%\providecommand{\version}{final}
%    \end{macrocode}

% Include the main document:
%    \begin{macrocode}
\input{childdoc.def}
\childdocby{cdocsamp}
%    \end{macrocode}

%\iffalse
%</samplepart3|samplepart4>
%\fi
%
%\iffalse
%<*samplepart3>
%\fi
% Some text for part 3:
%    \begin{macrocode}
some text in part three
%    \end{macrocode}

%\iffalse
%</samplepart3>
%\fi
% Some text for part 4:
%\iffalse
%<*samplepart4>
%\fi
%    \begin{macrocode}
more text in part four
%    \end{macrocode}

%\iffalse
%</samplepart4>
%\fi
%
% %%%%%%%%%%%%%%%%%%%%%%%%%%%%%%%%%%%%%%
% \paragraph{Forwarding for a Complete Draft.}
%
% The following forwarding file |cdocsdrf.tex|
% compiles the main document in draft mode:
%\iffalse
%<*sampledraft>
%\fi
%    \begin{macrocode}
\def\version{draft}
\input{childdoc.def}
\childdocforward{cdocsamp}
%    \end{macrocode}

%\iffalse
%</sampledraft>
%\fi
%
% %%%%%%%%%%%%%%%%%%%%%%%%%%%%%%%%%%%%%%
% \paragraph{Forwarding for Final Version of the Chapters.}
%
% The following forwarding files |cdocsfn1.tex| and |cdocsfn2.tex|
% (with identical content)
% compile the final versions of the child documents
% |cdocsch1.tex| and |cdocsch2.tex|, respectively:
%\iffalse
%<*samplefinal>
%\fi
%    \begin{macrocode}
\def\version{final}
\input{childdoc.def}
\childdocforwardprefix[cdocsamp]{cdocsfn}{cdocsch}
%    \end{macrocode}

%\iffalse
%</samplefinal>
%\fi
%
% %%%%%%%%%%%%%%%%%%%%%%%%%%%%%%%%%%%%%%
% \paragraph{Command Line Processing.}
%
% The following three command lines generate the output files
% |cdocscld|, |cdocscl1| and |cdocscl2|
% which should be identical to
% |cdocsdrf|, |cdocsch1| and |cdocsfn2|, respectively:
% \begin{center}
% \begin{tabular}{l}
% |latex -jobname cdocscld \|\\
% |  "\def\version{draft}\input{childdoc.def}\childdocforward{cdocsamp}"|\\
% |latex -jobname cdocscl1 \|\\
% |  "\input{childdoc.def}\childdocforward[cdocsamp]{cdocsch1}"|\\
% |latex -jobname cdocscl2 \|\\
% |  "\def\version{final}\input{childdoc.def}\childdocforward{cdocsch2}"|
% \end{tabular}
% \end{center}
% Note that the trailing backslash on each first line
% merely continues the input to the second line
% (for convenient cut ant paste).
% Furthermore, the command |latex| can be replaced by any
% of its alternative versions such as |pdflatex|.
%
% %%%%%%%%%%%%%%%%%%%%%%%%%%%%%%%%%%%%%%%%%%%%%%%%%%%%%%%%%%%%%%%%%%%%%%%%%%%%%%
% %%%%%%%%%%%%%%%%%%%%%%%%%%%%%%%%%%%%%%%%%%%%%%%%%%%%%%%%%%%%%%%%%%%%%%%%%%%%%%
% \section{Implementation}
%\iffalse
%<*package>
%\fi
%
% This section describes the definitions file |childdoc.def|.

% The definitions cannot be loaded using |\usepackage| or |\RequirePackage|
% which has a mechanism to prevent loading a style file more than once.
% When loading the definitions by means of |\input|
% multiple instances have to be prevented manually:
%\iffalse
%This code needs to be before the `\ProvidesFile' directive
%which is defined at the beginning of this file.
%Therefore it is also placed there and commented out here.
%</package>
%<*discard>
%\fi
%    \begin{macrocode}
\ifdefined\childdocmain\endinput\fi
%    \end{macrocode}
%\iffalse
%</discard>
%<*package>
%\fi
%
% \macro{\ifchilddoc}
% \macro{\ifchilddocmanual}
% The conditional |\ifchilddoc| tells whether a
% child (true) or main (false) document is being compiled.
% The conditional |\ifchilddocmanual| tells whether
% the |\includeonly| mechanism is used (false) or
% the selection of child files must be performed manually (true).
% The definitions initialise to false:
%    \begin{macrocode}
\newif\ifchilddoc
\newif\ifchilddocmanual
%    \end{macrocode}

% \macro{\childdocname}
% \macro{\childdocjob}
% The macro |\childdocname| stores the name of the main document
% to be compiled. The macro |\childdocjob| stores the name of
% the document on which the \LaTeX{} compiler was originally invoked.
% The content of |\jobname| cannot be compared
% to filenames specified in the source due to different catcodes.
% The following code rescans |\jobname|, stores the result
% in |\childdocname| and saves a copy in |\childdocjob|:
%    \begin{macrocode}
\edef\childdocname{\scantokens\expandafter{\jobname\noexpand}}
\let\childdocjob\childdocname
%    \end{macrocode}

% \macro{\childdocdisable}
% The macro |\childdocdisable| prevents the main file
% from being processed more than once.
% At this stage, the main document command |\childdocmain|
% is assumed to be called once again where it should do nothing.
% Any subsequent call to it should prevent
% a secondary processing of the main document
% It overwrites the forwarding commands
% |\childdocof| and |\childdocforward|
% with empty macros to prevent further inclusions of the main document:
%    \begin{macrocode}
\newcommand{\childdocdisable}
{
  \renewcommand{\childdocmain}[1]{\renewcommand{\childdocmain}[1]{\endinput}}
  \renewcommand{\childdocof}[1]{}
  \renewcommand{\childdocby}[2][]{}
  \renewcommand{\childdocforward}[2][]{}
  \renewcommand{\childdocdisable}{}
}
%    \end{macrocode}

% \macro{\childdocmain}
% The macro |\childdocmain| is to be called at the top of the main file
% with nothing or the main filename (without extension) as argument.
% First, it breaks loops.
% If the argument is not empty and does not match |\childdocname|
% (which is set by the first inclusion of |childdoc.def|),
% |\ifchilddoc| is set to true, |\includeonly| is applied to the child file
% and |\jobname| is set to the main file
% (for proper handling of |.aux| files):
%    \begin{macrocode}
\newcommand{\childdocmain}[1]
{
  \childdocdisable\childdocmain{}
  \if?#1?\else
    \begingroup
      \def\childdoctmp{#1}
      \ifx\childdoctmp\childdocname
        \def\childdoctmp{}
      \else
        \def\childdoctmp
        {
          \childdoctrue
          \includeonly{\childdocname}
          \def\childdocjob{#1}
          \def\jobname{#1}
        }
      \fi
      \expandafter
    \endgroup
    \childdoctmp
  \fi
}
%    \end{macrocode}

% \macro{\childdocof}
% The command |\childdocof| redirects
% compilation to the main file |#1|.
%    \begin{macrocode}
\newcommand{\childdocof}[1]
{
  \childdocdisable
  \childdoctrue
  \includeonly{\childdocname}
  \def\jobname{#1}
  \def\childdocjob{#1}
  \input{#1}
}
%    \end{macrocode}

% \macro{\childdocby}
% The command |\childdocby| ....
%    \begin{macrocode}
\newcommand{\childdocby}[2][]
{
  \childdocdisable
  \childdoctrue
  \childdocmanualtrue
  \if?#1?\else
    \def\jobname{#2}
  \fi
  \def\childdocjob{#2}
  \input{#2}
  \endinput
}
%    \end{macrocode}

% \macro{\childdocforward}
% The command |\childdocforward| redirects
% compilation to the main file or
% (if the optional argument is given) a child file.
% Parameters are set as if the main file
% or a child file starting with |\childdocof| was compiled.
% Then compilation is handed over to the main file:
%    \begin{macrocode}
\newcommand{\childdocforward}[2][]
{
  \begingroup
    \if?#1?
      \def\childdoctmp
      {
        \def\childdocname{#2}
        \def\childdocjob{#2}
        \def\jobname{#2}
        \input{#2}
        \endinput
      }
    \else
      \def\childdoctmp
      {
        \childdocdisable
        \def\childdocname{#2}
        \childdoctrue
        \includeonly{#2}
        \def\childdocjob{#1}
        \def\jobname{#1}
        \input{#1}
        \endinput
      }
    \fi
    \expandafter
  \endgroup
  \childdoctmp
}
%    \end{macrocode}

% \macro{\childdocforwardprefix}
% The command |\childdocforwardprefix| redirects
% compilation to the main or a child file by means of a pattern.
% The prefix |#1| in the current filename is replaced by |#2|
% and the suffix of the current filename is kept
% (it is assumed that the filename does not contain the substring `|~~~|'
% which is used as a delimiter).
% Compilation is handed over to the new file by |\childdocforward|:
%    \begin{macrocode}
\newcommand{\childdocforwardprefix}[3][]
{
  \begingroup
    \def\childdocextract #2##1~~~{\def\childdoctmp{\childdocforward[#1]{#3##1}}}
    \expandafter\childdocextract\childdocname~~~
    \expandafter
  \endgroup
  \childdoctmp
}
%    \end{macrocode}

% \macro{\childdoc}
% The deprecated macro |\childdoc| is a legacy version of |\childdocmain|:
%    \begin{macrocode}
\newcommand{\childdoc}{\childdocmain}
%    \end{macrocode}

% \macro{\childdocredirect}
% The deprecated macro |\childdocredirect| is a legacy version
% of |\childdocforward| and |\childdocforwardprefix|:
%    \begin{macrocode}
\newcommand{\childdocredirect}[2][]
{
  \begingroup
    \if?#1?
      \def\childdoctmp{\childdocforward{#2}}
    \else
      \def\childdoctmp{\childdocforwardprefix{#1}{#2}}
    \fi
    \expandafter
  \endgroup
  \childdoctmp
}
%    \end{macrocode}

%\iffalse
%</package>
%\fi
%
\endinput
|\\
|\childdocmain{|\textit{main}|}|\\
\end{tabular}
\end{center}
%
If |\jobname| does not match the argument \textit{main} of |\childdocmain|,
it is assumed that |\jobname| points to the child file to be compiled.
When using |\childdocmain| with the main file specified as argument,
it suffices to start a child file
with just |\input{|\textit{main}|}|
without loading of the package and using |\childdocof|.
If instead all processing is done
with the appropriate \textsf{childdoc} directives,
the argument of \textit{main} of |\childdocmain| can be empty.

An alternative version of the command line processing described
in \secref{sec:commandline} using the detection mechanism reads:
%
\begin{center}
|... -jobname "|\textit{target}|" "|[\textit{flags}]%
[|\def\jobname{|\textit{dest}|}|]|\input{|\textit{main}|}"|
\end{center}

%%%%%%%%%%%%%%%%%%%%%%%%%%%%%%%%%%%%%%%%%%%%%%%%%%%%%%%%%%%%%%%%%%%%%%%%%%%%%%%%
\subsection{Manual Code}
\label{sec:manual}

In case one cannot be certain whether the definitions file |childdoc.def|
is installed on the target \TeX{} distribution
and one prefers not to ship it,
it is conceivable to paste a few relevant commands into the sources.

To that end, drop all statements |% \iffalse
%
% childdoc.dtx Copyright (C) 2017-2018 Niklas Beisert
%
% This work may be distributed and/or modified under the
% conditions of the LaTeX Project Public License, either version 1.3
% of this license or (at your option) any later version.
% The latest version of this license is in
%   http://www.latex-project.org/lppl.txt
% and version 1.3 or later is part of all distributions of LaTeX
% version 2005/12/01 or later.
%
% This work has the LPPL maintenance status `maintained'.
%
% The Current Maintainer of this work is Niklas Beisert.
%
% This work consists of the files childdoc.dtx and childdoc.ins
% and the derived files childdoc.def and cdocsamp.tex with
% cdocsch1.tex, cdocsch2.tex, cdocsdrf.tex, cdocsfn1.tex, cdocsfn2.tex.
%
%<package>\ifdefined\childdocmain\endinput\fi
%<package>\ProvidesFile{childdoc.def}[2018/12/30 v2.0 child document driver]
%<samplemain>\ProvidesFile{cdocsamp.tex}[2018/12/30 v2.0 sample for childdoc]
%<*driver>
%\ProvidesFile{childdoc.drv}[2018/12/30 v2.0 childdoc reference manual file]
\PassOptionsToClass{10pt,a4paper}{article}
\documentclass{ltxdoc}

\usepackage[margin=35mm]{geometry}
\usepackage{hyperref}
\usepackage{hyperxmp}
\usepackage[usenames]{color}

\hypersetup{colorlinks=true}
\hypersetup{pdfstartview=FitH}
\hypersetup{pdfpagemode=UseNone}
\hypersetup{pdfsource={}}
\hypersetup{pdflang={en-UK}}
\hypersetup{pdfcopyright={Copyright 2017-2018 Niklas Beisert.
  This work may be distributed and/or modified under the
  conditions of the LaTeX Project Public License, either version 1.3
  of this license or (at your option) any later version.}}
\hypersetup{pdflicenseurl={http://www.latex-project.org/lppl.txt}}
\hypersetup{pdfcontactaddress={ETH Zurich, ITP, HIT K,
  Wolfgang-Pauli-Strasse 27}}
\hypersetup{pdfcontactpostcode={8093}}
\hypersetup{pdfcontactcity={Zurich}}
\hypersetup{pdfcontactcountry={Switzerland}}
\hypersetup{pdfcontactemail={nbeisert@itp.phys.ethz.ch}}
\hypersetup{pdfcontacturl={http://people.phys.ethz.ch/\xmptilde nbeisert/}}

\newcommand{\secref}[1]{\hyperref[#1]{section \ref*{#1}}}

\parskip1ex
\parindent0pt
\let\olditemize\itemize
\def\itemize{\olditemize\parskip0pt}

\begin{document}

\title{The \textsf{childdoc} Package}
\hypersetup{pdftitle={The childdoc Package}}
\author{Niklas Beisert\\[2ex]
  Institut f\"ur Theoretische Physik\\
  Eidgen\"ossische Technische Hochschule Z\"urich\\
  Wolfgang-Pauli-Strasse 27, 8093 Z\"urich, Switzerland\\[1ex]
  \href{mailto:nbeisert@itp.phys.ethz.ch}
  {\texttt{nbeisert@itp.phys.ethz.ch}}}
\hypersetup{pdfauthor={Niklas Beisert}}
\hypersetup{pdfsubject={Manual for the LaTeX2e Package childdoc}}
\date{30 December 2018, \textsf{v2.0}}
\maketitle

\begin{abstract}\noindent
\textsf{childdoc} is a \LaTeXe{} package
that enables the direct compilation
of document sections included by |\include|
to individual files.
\end{abstract}

\begingroup
\parskip0ex
\tableofcontents
\endgroup

%%%%%%%%%%%%%%%%%%%%%%%%%%%%%%%%%%%%%%%%%%%%%%%%%%%%%%%%%%%%%%%%%%%%%%%%%%%%%%%%
%%%%%%%%%%%%%%%%%%%%%%%%%%%%%%%%%%%%%%%%%%%%%%%%%%%%%%%%%%%%%%%%%%%%%%%%%%%%%%%%
\section{Introduction}

\LaTeX{} provides a mechanism to structure a large document (such as a book)
into a main file and several child files (containing the chapters)
using the |\include| command.
This mechanism is beneficial for documents
which span hundreds of pages in order to
make the source file(s) more manageable.
Moreover, compilation can be restricted to
selected child files by means of the |\includeonly| command.
The latter feature can be used to reduce the compilation time while editing
(this was significantly more useful in the earlier days of \LaTeX{})
or to generate a smaller document which is easier to navigate.
Another application of |\includeonly| is to generate
documents consisting of selected parts of the complete document.

However, there are a few drawbacks of the plain |\include| mechanism:
\begin{itemize}
\item
The child files cannot be compiled on their own,
they can only be compiled via the main file.
A naive editing environment
(such as a text editor with an option
to have the current file processed by \LaTeX)
may require one to switch to the main file before compiling;
attempting to compile the child file produces errors.
\item
The main file must be modified (each time)
to adjust the |\includeonly| command
to the present needs. This easily leaves the main file in a messy state.
\item
The generated document will always carry the filename
of the main document. This is inconvenient if
several child files are to be compiled and
to be kept for distribution.
\end{itemize}

The present package provides a simple interface
to make child files individually compilable by \LaTeX{}.
Compiling a child file then has the same effect as compiling
the main file with an |\includeonly| command
to select the appropriate child.
Moreover the generated document will carry the name of the child
rather than the main file.
This resolves all three above issues.

This feature is meant to make the editing of books,
thesis documents and lecture notes somewhat more convenient.
However, the package can also be used efficiently for
composing a series of documents (such as exercise sheets)
which are typically distributed individually.
It then assists the author in generating the individual documents
(potentially in different versions)
as well as a document containing the collected series.
Another application is in developing style files
or other kinds of included material
where compilation of the style file could redirect
to a sample or test file.

%%%%%%%%%%%%%%%%%%%%%%%%%%%%%%%%%%%%%%%%%%%%%%%%%%%%%%%%%%%%%%%%%%%%%%%%%%%%%%%%
%%%%%%%%%%%%%%%%%%%%%%%%%%%%%%%%%%%%%%%%%%%%%%%%%%%%%%%%%%%%%%%%%%%%%%%%%%%%%%%%
\section{Usage}

First of all, the package \textsf{childdoc} is \emph{not} a standard
\LaTeXe{} |.sty| style file! Therefore it needs to be invoked in
a non-standard way.

%%%%%%%%%%%%%%%%%%%%%%%%%%%%%%%%%%%%%%%%%%%%%%%%%%%%%%%%%%%%%%%%%%%%%%%%%%%%%%%%
\subsection{Included Files}
\label{sec:include}

%%%%%%%%%%%%%%%%%%%%%%%%%%%%%%%%%%%%%%%%
\DescribeMacro{\childdocmain}
To use the package, add the commands
\begin{center}
\begin{tabular}{l}
|\input{childdoc.def}|\\
|\childdocmain{}|\\
\end{tabular}
\end{center}
at the very top of the main \LaTeX{} file,
in particular \emph{before} the |\documentclass| statement!
The argument of |\childdocmain| should be left empty
(but it must be present).

%%%%%%%%%%%%%%%%%%%%%%%%%%%%%%%%%%%%%%%%
\DescribeMacro{\childdocof}
Furthermore, add the commands
\begin{center}
\begin{tabular}{l}
|\input{childdoc.def}|\\
|\childdocof{|\textit{main}|}|\\
\end{tabular}
\end{center}
at the top of every child file \textit{child}
which is included by |\include{|\textit{child}|}|
from within the main file
(or at least for those files to be compiled individually).
The argument \textit{main} must be the filename of the main file.

There are a couple of
considerations in setting up the main and child documents:

%%%%%%%%%%%%%%%%%%%%%%%%%%%%%%%%%%%%%%%%
\paragraph{Restrictions.}

Please note the following restrictions:
\begin{itemize}
\item
|\childdocmain| must be called with one argument \textit{main}
to ensure compatibility with earlier version of the package.
It must either be empty (|\childdocmain{}|)
or precisely match the filename of the main file in which it is specified.
See \secref{sec:detection} for further information.
\item
The filename \textit{main} must be specified without the |.tex| extension.
\item
The filename \textit{main} is case sensitive
(even in case-insensitive file systems)
due to internal string comparison.
\item
The argument \textit{main} should be fully expanded, it cannot be a macro.
\item
Subdirectories and special characters should be avoided in filenames.
\item
The command |\childdocmain{|\textit{main}|}| must be followed by a whitespace.
It should not be followed immediately by another command
or by a comment mark `|%|'.
This is because the \TeX{} parser reads the token immediately following
the argument of |\childdocmain| and puts it
at the beginning of every child section;
however, a white\-space is ignored.
\end{itemize}

%%%%%%%%%%%%%%%%%%%%%%%%%%%%%%%%%%%%%%%%
\paragraph{Content of Main File.}

It is advisable to place all content in the child files included by |\include|.
Any output contained in the main file will appear in all child documents
unless suppressed manually;
it cannot be suppressed automatically by the |\includeonly| directive
and thus should normally be avoided.
A method to include some content in the main file
by means of conditional processing is described in \secref{sec:conditional}.

%%%%%%%%%%%%%%%%%%%%%%%%%%%%%%%%%%%%%%%%
\paragraph{Page Numbering.}

When only a part of the document is compiled,
the appropriate numbering of pages
(as well as other status parameters)
is determined from the |.aux| files.
The latter contain information from previous passes.
However this information needs to propagate through
all intermediate child documents.
Therefore the page numbering in child documents may well
be inconsistent until the complete document is compiled at least once.

A useful (if unconventional) way to always ensure a consistent
page numbering is to restart the numbering in each child document
and denote the pages by `\textit{child}|.|\textit{page}'
where \textit{child} represents the chapter/section number of the child file.
This can be achieved by the command
|\numberwithin{page}{|\textit{child}|}|
of the \textsf{amsmath} package
where \textit{child} can be |chapter| or |section|
depending on the chosen structuring.
Alternatively, one can modify the macro |\thepage| appropriately
and reset the counter |page| at the start of each child file.

%%%%%%%%%%%%%%%%%%%%%%%%%%%%%%%%%%%%%%%%%%%%%%%%%%%%%%%%%%%%%%%%%%%%%%%%%%%%%%%%
\subsection{Conditional Processing}
\label{sec:conditional}

The package provides a mechanism to compile different versions
of a document. To customise the versions further some conditional processing
can come in handy to distinguish which version is being compiled.
The package provides two macros to describe the compilation context:

%%%%%%%%%%%%%%%%%%%%%%%%%%%%%%%%%%%%%%%%
\DescribeMacro{\ifchilddoc}
The conditional |\ifchilddoc| distinguishes between the compilation of
child documents and the main document:
%
\begin{center}
|\ifchilddoc |\textit{child-code}| |[|\||else |\textit{main-code}]| \||fi|
\end{center}

%%%%%%%%%%%%%%%%%%%%%%%%%%%%%%%%%%%%%%%%
\DescribeMacro{\childdocname}
\DescribeMacro{\childdocjob}
The macro |\childdocname| contains the filename (without extension)
of the main or child file being processed.
Note that |\childdocjob| will always contain the name of the main file.

%%%%%%%%%%%%%%%%%%%%%%%%%%%%%%%%%%%%%%%%
\paragraph{Title Page.}

Conditional processing can be used to include a title or banner page
in the main document when proper precautions are taken.
Importantly, the code in the main file should ensure that the page counter
(as well as other status parameters which are stored in the |.aux| files)
takes the same value after the conditional processing.
Otherwise the page numbers may take divergent values
depending on which part is compiled.

For example, a title page could be declared by:
%
\begin{center}
\begin{tabular}{l}
|\ifchilddoc\||else|\\
|\addtocounter{page}{-1}|\\
\textit{code for title page}\\
|\newpage|\\
|\||fi|
\end{tabular}
\end{center}
%
A banner page for the child documents can be generated by:
%
\begin{center}
\begin{tabular}{l}
|\ifchilddoc|\\
|\addtocounter{page}{-1}|\\
\textit{code for banner page}\\
|\newpage|\\
|\||fi|
\end{tabular}
\end{center}
%
Here one could write a message such as:
\begin{center}
|This is the part \childdocname{} of \childdocjob{}.|
\end{center}

%%%%%%%%%%%%%%%%%%%%%%%%%%%%%%%%%%%%%%%%%%%%%%%%%%%%%%%%%%%%%%%%%%%%%%%%%%%%%%%%
\subsection{Flags}
\label{sec:flags}

The package makes it easy to generate different versions
of the main or child documents.
To this end compilation flags can be defined
and assigned different default values.
They will be particularly useful in conjunction
with the forwarding mechanism described in \secref{sec:forward}.

For example, it may be useful to have a flag |\version|
which can be set to |draft| or |final|.
The document source will contain some conditional code
depending on the value of |\version|.
Suppose further, the flag should default to |final| for the main file
and to |draft| for child files
which is a natural assignment for editing the document.
This is achieved by placing the following code
in the preamble of the main document
(below the |\childdocmain| directive):
%
\begin{center}
\begin{tabular}{l}
|\ifchilddoc|\\
|\providecommand{\version}{draft}|\\
|\||else|\\
|\providecommand{\version}{final}|\\
|\||fi|
\end{tabular}
\end{center}
%
The definition by |\providecommand| makes sure
that previous definitions are not overwritten.
Further statements |\providecommand{\version}{...}|
can thus be added before the above code to override it.

For the main file, one might add a line
(between |\childdocmain| and the above block)
%
\begin{center}
|%\ifchilddoc\||else\providecommand{\version}{draft}\||fi|
\end{center}
%
which can be uncommented to produce a draft version.
Likewise one can add a line to the very top of a child file
(above the |\childdocof{|\textit{main}|}| directive)
%
\begin{center}
|%\providecommand{\version}{final}|
\end{center}
%
which can be uncommented to produce the final version of this child document.

%%%%%%%%%%%%%%%%%%%%%%%%%%%%%%%%%%%%%%%%%%%%%%%%%%%%%%%%%%%%%%%%%%%%%%%%%%%%%%%%
\subsection{Forwarding}
\label{sec:forward}

Different versions of the main or child documents
using compilation flags as described in \secref{sec:flags}
can be (permanently) stored in different files
for convenient compilation, viewing and distribution.
To this end, the package defines a command
to pass on compilation to a different file:

%%%%%%%%%%%%%%%%%%%%%%%%%%%%%%%%%%%%%%%%
\DescribeMacro{\childdocforward}
The command |\childdocforward| redirects processing to
another source file:
%
\begin{center}
\begin{tabular}{l}
|\input{childdoc.def}|\\
|\childdocforward[|\textit{main}|]{|\textit{dest}|}|\\
\end{tabular}
\end{center}
%
The argument \textit{dest} is the destination file
(without extension).
It should be the main file or one of the child files.
Note that further \textsf{childdoc} directives
such as |\childdocof| and |\childdocforward|
in the indicated file will be processed in this form.
The optional argument \textit{main}
passes on directly to the main file \textit{main}
while pretending to compile the child \textit{dest}.
This form behaves as if \textit{dest}
issues |\childdocof{|\textit{main}|}| right away,
and no further \textsf{childdoc} directives will be processed.

%%%%%%%%%%%%%%%%%%%%%%%%%%%%%%%%%%%%%%%%
\DescribeMacro{\...prefix}
In the alternative form |\childdocforwardprefix|,
%
\begin{center}
\begin{tabular}{l}
|\input{childdoc.def}|\\
|\childdocforwardprefix[|\textit{main}|]{|\textit{prefix}|}{|\textit{dest}|}|
\end{tabular}
\end{center}
%
the destination file is determined by a pattern
depending on the current file:
To make this work, the current file must be called
`{\textit{prefix}\hspace{0.2em}\textit{suffix}}'
with \textit{prefix} matching precisely the argument.
Processing is then passed on to the file
`{\textit{dest}\hspace{0.2em}\textit{suffix}}'.
Surely, the same effect is achieved by
directly specifying the
argument `{\textit{dest}\hspace{0.2em}\textit{suffix}}'
in the first form.
However, that requires to set up a different file
for each child. With the alternative form of the command
all these files can have exactly the same content
which simplifies setting them up and maintaining them.

For example, the following file |draft.tex|
with a compilation flag |\version| as described in \secref{sec:flags}
compiles the main document as a draft:
%
\begin{center}
\begin{tabular}{l}
|\def\version{draft}|\\
|\input{childdoc.def}|\\
|\childdocforward{|\textit{main}|}|
\end{tabular}
\end{center}
%
Likewise, the following files |final|\textit{nn}|.tex|
compile the final version of the child document
|child|\textit{nn}|.tex|:
%
\begin{center}
\begin{tabular}{l}
|\def\version{final}|\\
|\input{childdoc.def}|\\
|\childdocforwardprefix{final}{child}|
\end{tabular}
\end{center}
%

Note that when several versions of a main file and/or of each child file
are to be generated, it may be convenient to set up a |Makefile| or
shell script to automatise the process.

%%%%%%%%%%%%%%%%%%%%%%%%%%%%%%%%%%%%%%%%%%%%%%%%%%%%%%%%%%%%%%%%%%%%%%%%%%%%%%%%
\subsection{Command Line Processing}
\label{sec:commandline}

The effect of redirection files can also be achieved by invoking
the \LaTeX{} compiler with a more elaborate command line.
Most conveniently this should be done as part
of a shell script or a |Makefile|.

When using \textsf{childdoc} in the main file, the following
command lines effectively perform a redirection
(note that depending on the shell being used,
backslashes may have to be doubled: `|\|' $\to$ `|\\|'):
%
\begin{center}
|... -jobname "|\textit{target}|" |\\|"|[\textit{flags}]%
|\input{childdoc.def}\childdocforward[|\textit{main}|]{|\textit{dest}|}"|
\end{center}
%
Here \textit{target} is the name of the output file,
\textit{main} is the name of the main file
and \textit{dest} is the name of the main or child file to be processed
(all filenames without extensions).
The optional argument \textit{main} can be omitted
if \textit{main} matches \textit{dest}.
Optionally, compilation \textit{flags} can be defined via |\def| commands.
This command line makes the \TeX{} engine believe
it is compiling the file \textit{target}
whose content is specified as the latter parameter.
The provided code then forwards the processing to
\textit{main} or \textit{dest} as described in \secref{sec:forward}.

%%%%%%%%%%%%%%%%%%%%%%%%%%%%%%%%%%%%%%%%%%%%%%%%%%%%%%%%%%%%%%%%%%%%%%%%%%%%%%%%
\subsection{Include by Input}
\label{sec:input}

Including child documents by |\include| has some restrictions by design.
Most notably, the content of a child document always occupies
its own set of pages; pages cannot be shared between child documents.
Usually, this behaviour makes perfect sense
because each child document contain an essential part of the document.
However, in some situations it may be desirable to compose
a document from a collection of parts
without having mandatory page breaks between then.
For this case, the package
provides a mechanism to include parts
by |\input| which can also be processed individually.
However, by construction this mechanism
requires manual handling of the content to be output.

%%%%%%%%%%%%%%%%%%%%%%%%%%%%%%%%%%%%%%%%
\DescribeMacro{\ifchilddocmanual}
The main file should be prepared as usual, see \secref{sec:include}.
However, the document body must make a distinction
between processing of an individual part and of the main document, e.g.:
%
\begin{center}
\begin{tabular}{l}
|\ifchilddocmanual|\\
|\input{\childdocname}|\\
|\||else|\\
\textit{document body with }|\input{|\textit{part}|}|\\
|\||fi|
\end{tabular}
\end{center}
%
The conditional |\ifchilddocmanual| is true whenever
a part to be included by |\input| is being compiled,
and the name of the part is stored in |\childdocname|.

%%%%%%%%%%%%%%%%%%%%%%%%%%%%%%%%%%%%%%%%
\DescribeMacro{\childdocby}
Each part to be included by |\input| should start with:
%
\begin{center}
\begin{tabular}{l}
|\input{childdoc.def}|\\
|\childdocby{|\textit{main}|}|\\
\end{tabular}
\end{center}
%
The directive |\childdocby| is similar to |\childdocof|
described in \secref{sec:include},
but the subsequent selection of content must be done manually.
To that end, both |\ifchilddoc| and |\ifchilddocmanual|
will be true upon processing of a part,
and the name of the part is stored in |\childdocname|.
Note that |\jobname| will be set to the filename of the current part
so that each part receives an individual |.aux| file
that does not interfere with the |.aux| file(s) of the main document.
This behaviour can be altered by the alternative form
|\childdocby[*]{|\textit{main}|}| (with a non-empty optional argument)
which uses the |.aux| file of the main document
by setting |\jobname| to \textit{main}.

%%%%%%%%%%%%%%%%%%%%%%%%%%%%%%%%%%%%%%%%%%%%%%%%%%%%%%%%%%%%%%%%%%%%%%%%%%%%%%%%
\subsection{Driver Development}
\label{sec:driver}

The \textsf{childdoc} mechanism can also be use for the development
of definition files such as \LaTeX{} styles or classes.
This case differs from the above setup with multiple parts
included by |\include| in that no |\includeonly| should be invoked.
This can be achieved by starting the include file
(before |\ProvidesPackage|) with:
%
\begin{center}
\begin{tabular}{l}
|\input{childdoc.def}|\\
|\childdocforward{|\textit{main}|}|\\
\end{tabular}
\end{center}
%
or alternatively with:
%
\begin{center}
\begin{tabular}{l}
|\input{childdoc.def}|\\
|\childdocby{|\textit{main}|}|\\
\end{tabular}
\end{center}
%
Both forms have slightly different effects as described above.
The main file is prepared as usual, see \secref{sec:include}.

%%%%%%%%%%%%%%%%%%%%%%%%%%%%%%%%%%%%%%%%%%%%%%%%%%%%%%%%%%%%%%%%%%%%%%%%%%%%%%%%
\subsection{Legacy Detection}
\label{sec:detection}

The directive |\childdocmain| in the main file can detect
whether the complete document or merely a child is to be compiled
even without using the directive |\childdocof|.
This method is deprecated because it is less robust
and there is no compelling reason to use it;
it is merely provided for backward compatibility
and it may be removed in future versions.

If the detection mechanism is to be used,
it is mandatory to correctly specify
the filename of the main file as the argument of |\childdocmain|:
%
\begin{center}
\begin{tabular}{l}
|\input{childdoc.def}|\\
|\childdocmain{|\textit{main}|}|\\
\end{tabular}
\end{center}
%
If |\jobname| does not match the argument \textit{main} of |\childdocmain|,
it is assumed that |\jobname| points to the child file to be compiled.
When using |\childdocmain| with the main file specified as argument,
it suffices to start a child file
with just |\input{|\textit{main}|}|
without loading of the package and using |\childdocof|.
If instead all processing is done
with the appropriate \textsf{childdoc} directives,
the argument of \textit{main} of |\childdocmain| can be empty.

An alternative version of the command line processing described
in \secref{sec:commandline} using the detection mechanism reads:
%
\begin{center}
|... -jobname "|\textit{target}|" "|[\textit{flags}]%
[|\def\jobname{|\textit{dest}|}|]|\input{|\textit{main}|}"|
\end{center}

%%%%%%%%%%%%%%%%%%%%%%%%%%%%%%%%%%%%%%%%%%%%%%%%%%%%%%%%%%%%%%%%%%%%%%%%%%%%%%%%
\subsection{Manual Code}
\label{sec:manual}

In case one cannot be certain whether the definitions file |childdoc.def|
is installed on the target \TeX{} distribution
and one prefers not to ship it,
it is conceivable to paste a few relevant commands into the sources.

To that end, drop all statements |\input{childdoc.def}|
and perform the replacements as outlined below.
Instead of |\childdocmain{|\textit{main}|}| add the following code
to the top of the main file:
%
\begin{center}
\begin{tabular}{l}
|\||ifdefined\childdocname\endinput\||fi\newif\ifchilddoc|\\
|\edef\childdocname{\scantokens\expandafter{\jobname\noexpand}}|\\
|\def\childdocmain{|\textit{main}|}\||ifx\childdocmain\childdocname\||else|\\
|\childdoctrue\includeonly{\childdocname}\let\jobname\childdocmain\||fi|\\
\end{tabular}
\end{center}
%
Instead of |\childdocof{|\textit{main}|}| just include the main file
at the top of each child file:
%
\begin{center}
|\input{|\textit{main}|}|
\end{center}
%
A simple redirection |\childdocforward{|\textit{dest}|}| is achieved by:
%
\begin{center}
|\def\jobname{|\textit{dest}|}\input{\jobname}|
\end{center}
%
The redirection with prefix
|\childdocforwardprefix[|\textit{prefix}|]{|\textit{dest}|}|
is accomplished by:
%
\begin{center}
\begin{tabular}{l}
|{\edef\jobname{\scantokens\expandafter{\jobname\noexpand}}|\\
|\def\redirectjob |\textit{prefix}|#1~~~{\gdef\jobname{|\textit{dest}|#1}}|\\
|\expandafter\redirectjob\jobname~~~}\input{\jobname}|
\end{tabular}
\end{center}

In an alternative approach,
child documents can be compiled by a specific command line
without additional code or specific definitions:
%
\begin{center}
|... -jobname "|\textit{target}|" "|[\textit{flags}]%
|\includeonly{|\textit{dest}|}\input{|\textit{main}|}"|
\end{center}
%

%%%%%%%%%%%%%%%%%%%%%%%%%%%%%%%%%%%%%%%%%%%%%%%%%%%%%%%%%%%%%%%%%%%%%%%%%%%%%%%%
%%%%%%%%%%%%%%%%%%%%%%%%%%%%%%%%%%%%%%%%%%%%%%%%%%%%%%%%%%%%%%%%%%%%%%%%%%%%%%%%
\section{Information}

%%%%%%%%%%%%%%%%%%%%%%%%%%%%%%%%%%%%%%%%%%%%%%%%%%%%%%%%%%%%%%%%%%%%%%%%%%%%%%%%
\subsection{Copyright}

Copyright \copyright{} 2017--2018 Niklas Beisert

This work may be distributed and/or modified under the
conditions of the \LaTeX{} Project Public License, either version 1.3
of this license or (at your option) any later version.
The latest version of this license is in
  \url{http://www.latex-project.org/lppl.txt}
and version 1.3 or later is part of all distributions of \LaTeX{}
version 2005/12/01 or later.

This work has the LPPL maintenance status `maintained'.

The Current Maintainer of this work is Niklas Beisert.

This work consists of the files |README.txt|, |childdoc.ins| and |childdoc.dtx|
as well as the derived files |childdoc.def|, |cdocsamp.tex|
with |cdocsch1.tex|, |cdocsch2.tex|, |cdocspt3.tex|, |cdocspt4.tex|,
|cdocsdrf.tex|, |cdocsfn1.tex|, |cdocsfn2.tex|
as well as |childdoc.pdf|.

%%%%%%%%%%%%%%%%%%%%%%%%%%%%%%%%%%%%%%%%%%%%%%%%%%%%%%%%%%%%%%%%%%%%%%%%%%%%%%%%
\subsection{Files and Installation}

The package consists of the files:
%
\begin{center}
\begin{tabular}{ll}
    |README.txt|   & readme file \\
    |childdoc.ins| & installation file \\
    |childdoc.dtx| & source file \\
    |childdoc.def| & definition file \\
    |cdocsamp.tex| & sample main file \\
    |cdocsch1.tex| & sample include file \\
    |cdocsch2.tex| & sample include file \\
    |cdocspt3.tex| & sample part file \\
    |cdocspt4.tex| & sample part file \\
    |cdocsdrf.tex| & sample redirection file \\
    |cdocsfn1.tex| & sample redirection file \\
    |cdocsfn2.tex| & sample redirection file \\
    |childdoc.pdf| & manual
\end{tabular}
\end{center}
%
The distribution consists of the files
|README.txt|, |childdoc.ins| and |childdoc.dtx|.
%
\begin{itemize}
\item
Run (pdf)\LaTeX{} on |childdoc.dtx|
to compile the manual |childdoc.pdf| (this file).
\item
Run \LaTeX{} on |childdoc.ins| to create the definitions file |childdoc.def|
and the sample |cdocsamp.tex| with include files
|cdocsch1.tex|, |cdocsch2.tex|, |cdocspt3.tex|, |cdocspt4.tex|,
|cdocsdrf.tex|, |cdocsfn1.tex|, |cdocsfn2.tex|.
Then copy the file |childdoc.def| to an appropriate directory of your \LaTeX{}
distribution, e.g.\ \textit{texmf-root}|/tex/latex/childdoc|.
\end{itemize}

%%%%%%%%%%%%%%%%%%%%%%%%%%%%%%%%%%%%%%%%%%%%%%%%%%%%%%%%%%%%%%%%%%%%%%%%%%%%%%%%
\subsection{Related CTAN Packages}

There are several other packages which offer a similar functionality:
%
\begin{itemize}
\item
The packages
\href{http://ctan.org/pkg/docmute}{\textsf{docmute}},
\href{http://ctan.org/pkg/includex}{\textsf{includex}} and
\href{http://ctan.org/pkg/standalone}{\textsf{standalone}}
provide commands to include only the document body of
a child file thus allowing both files to be compiled individually.
\item
The packages \href{http://ctan.org/pkg/subdocs}{\textsf{subdocs}}
and \href{http://ctan.org/pkg/subfiles}{\textsf{subfiles}}
provide structures in which the main and child documents can be
encapsulated and allowing them to be compiled individually.
The inclusion mechanism is different from the conventional |\include|.
\item
The package \href{http://ctan.org/pkg/combine}{\textsf{combine}}
is an elaborate solution to combine several documents into one.
\end{itemize}
%
See also the CTAN topic \href{http://ctan.org/topic/subdocs}{\textsf{subdocs}}
for further related packages.
The present package differs from the above solutions in that
a document structure constructed with the conventional |\include| mechanism
just needs two extra commands at the top of every file
such that all constituent files can be compiled individually.

%%%%%%%%%%%%%%%%%%%%%%%%%%%%%%%%%%%%%%%%%%%%%%%%%%%%%%%%%%%%%%%%%%%%%%%%%%%%%%%%
%\subsection{Feature Suggestions}
%
%The following is a list of features which may be useful for future
%versions of this package:
%%
%\begin{itemize}
%\item
%\ldots
%\end{itemize}

%%%%%%%%%%%%%%%%%%%%%%%%%%%%%%%%%%%%%%%%%%%%%%%%%%%%%%%%%%%%%%%%%%%%%%%%%%%%%%%%
\subsection{Revision History}

%%%%%%%%%%%%%%%%%%%%%%%%%%%%%%%%%%%%%%%%
\paragraph{v2.0:} 2018/12/30

\begin{itemize}
\item
immediate forward processing
\item
added |\childdocby| mechanism
\item
manual restructured
\end{itemize}

%%%%%%%%%%%%%%%%%%%%%%%%%%%%%%%%%%%%%%%%
\paragraph{v1.6:} 2018/01/17

\begin{itemize}
\item
application for development of include files
\item
corrections to manual
\end{itemize}

%%%%%%%%%%%%%%%%%%%%%%%%%%%%%%%%%%%%%%%%
\paragraph{v1.5:} 2017/05/21

\begin{itemize}
\item
more complete structuring introduced
\item
|\childdocof| introduced
\item
|\childdoc| renamed to |\childdocmain|
\item
|\childredirect| renamed to |\childdocforward| and |\childdocforwardprefix|
and functionality expanded
\end{itemize}

%%%%%%%%%%%%%%%%%%%%%%%%%%%%%%%%%%%%%%%%
\paragraph{v1.0:} 2017/04/27

\begin{itemize}
\item
manual and install package
\item
first version published on CTAN
\end{itemize}

%%%%%%%%%%%%%%%%%%%%%%%%%%%%%%%%%%%%%%%%
\paragraph{v0.6:} 2017/04/26

\begin{itemize}
\item
redirection mechanism added
\end{itemize}

%%%%%%%%%%%%%%%%%%%%%%%%%%%%%%%%%%%%%%%%
\paragraph{v0.5:} 2017/04/26

\begin{itemize}
\item
functionality in definition file
\end{itemize}


%%%%%%%%%%%%%%%%%%%%%%%%%%%%%%%%%%%%%%%%%%%%%%%%%%%%%%%%%%%%%%%%%%%%%%%%%%%%%%%%
%%%%%%%%%%%%%%%%%%%%%%%%%%%%%%%%%%%%%%%%%%%%%%%%%%%%%%%%%%%%%%%%%%%%%%%%%%%%%%%%
%%%%%%%%%%%%%%%%%%%%%%%%%%%%%%%%%%%%%%%%%%%%%%%%%%%%%%%%%%%%%%%%%%%%%%%%%%%%%%%%
\appendix

\settowidth\MacroIndent{\rmfamily\scriptsize 000\ }

 \DocInput{childdoc.dtx}

\end{document}
%</driver>
% \fi
%
% %%%%%%%%%%%%%%%%%%%%%%%%%%%%%%%%%%%%%%%%%%%%%%%%%%%%%%%%%%%%%%%%%%%%%%%%%%%%%%
% %%%%%%%%%%%%%%%%%%%%%%%%%%%%%%%%%%%%%%%%%%%%%%%%%%%%%%%%%%%%%%%%%%%%%%%%%%%%%%
% \section{Sample}
%\iffalse
%<*samplemain>
%\fi
%
% The following presents a sample document
% with two chapters, two parts, a title page,
% a compile flag as well as three forwarding files to set the flag.
% It consists of eight |.tex| files:
% \begin{center}
% \begin{tabular}{ll}
% |cdocsamp.tex|&main file\\
% |cdocsch1.tex|&include file for chapter 1\\
% |cdocsch2.tex|&include file for chapter 2\\
% |cdocspt3.tex|&include file for part 3\\
% |cdocspt4.tex|&include file for part 4\\
% |cdocsdrf.tex|&forwarding file for main file in draft mode\\
% |cdocsfi1.tex|&forwarding file for final version of chapter 1\\
% |cdocsfi2.tex|&forwarding file for final version of chapter 2\\
% \end{tabular}
% \end{center}
% Each of the eight files can be compiled directly by the \LaTeX{} compiler.
%
% %%%%%%%%%%%%%%%%%%%%%%%%%%%%%%%%%%%%%%
% \paragraph{Main File.}
%
% The main file is called |cdocsamp.tex|.
%
% Load the \textsf{childdoc} definitions and
% declare the filename for the main document:
%    \begin{macrocode}
\input{childdoc.def}
\childdocmain{}
%    \end{macrocode}

% Optional override for |\version| flag:
%    \begin{macrocode}
%%\ifchilddoc\else\providecommand{\version}{draft}\fi
%    \end{macrocode}

% Define the default values for the |\version| flag
% (|final| for the main file and |draft| for childs):
%    \begin{macrocode}
\ifchilddoc
\providecommand{\version}{draft}
\else
\providecommand{\version}{final}
\fi
%    \end{macrocode}

% Load the standard document class:
%    \begin{macrocode}
\documentclass[12pt]{article}
%    \end{macrocode}

% Start the document body:
%    \begin{macrocode}
\begin{document}
%    \end{macrocode}

% Declare a title page.
% Print title, part of document being processed and version flag:
%    \begin{macrocode}
\addtocounter{page}{-1}
\begin{center}
{\LARGE\bfseries{}childdoc example\par}
\vspace{1cm}
\ifchilddoc
\ifchilddocmanual part\else chapter\fi:
`\childdocname' of `\childdocjob'\par
\else
main document: `\childdocjob'\par
\fi
version: \version\par
\end{center}
\newpage
%    \end{macrocode}

% Manually include selected file,
% otherwise process as usual:
%    \begin{macrocode}
\ifchilddocmanual
\section*{part `\childdocname'}
\input{\childdocname}
\else
%    \end{macrocode}

% Include the two chapters:
%    \begin{macrocode}
\include{cdocsch1}
\include{cdocsch2}
%    \end{macrocode}

% Include the two parts unless only chapters should be displayed:
%    \begin{macrocode}
\ifchilddoc\else
\section{part three}
\input{cdocspt3}
\section{part four}
\input{cdocspt4}
\fi
%    \end{macrocode}

% Process as usual until here:
%    \begin{macrocode}
\fi
%    \end{macrocode}

% End of document body:
%    \begin{macrocode}
\end{document}
%    \end{macrocode}
%\iffalse
%</samplemain>
%\fi
%
% %%%%%%%%%%%%%%%%%%%%%%%%%%%%%%%%%%%%%%
% \paragraph{Chapter Include Files.}
%
% The include files are called |cdocsch1.tex| and |cdocsch2.tex|.
%
%\iffalse
%<*samplechap1|samplechap2>
%\fi

% Optional override for |\version| flag:
%    \begin{macrocode}
%%\providecommand{\version}{final}
%    \end{macrocode}

% Include the main document:
%    \begin{macrocode}
\input{childdoc.def}
\childdocof{cdocsamp}
%    \end{macrocode}

%\iffalse
%</samplechap1|samplechap2>
%\fi
%
%\iffalse
%<*samplechap1>
%\fi
% Some text for chapter 1:
%    \begin{macrocode}
\section{one}
some text in chapter one
%    \end{macrocode}

%\iffalse
%</samplechap1>
%\fi
% Some text for chapter 2:
%\iffalse
%<*samplechap2>
%\fi
%    \begin{macrocode}
\section{two}
more text in chapter two
%    \end{macrocode}

%\iffalse
%</samplechap2>
%\fi
%
% %%%%%%%%%%%%%%%%%%%%%%%%%%%%%%%%%%%%%%
% \paragraph{Part Include Files.}
%
% The include files are called |cdocspt3.tex| and |cdocspt4.tex|.
%
%\iffalse
%<*samplepart3|samplepart4>
%\fi

% Optional override for |\version| flag:
%    \begin{macrocode}
%%\providecommand{\version}{final}
%    \end{macrocode}

% Include the main document:
%    \begin{macrocode}
\input{childdoc.def}
\childdocby{cdocsamp}
%    \end{macrocode}

%\iffalse
%</samplepart3|samplepart4>
%\fi
%
%\iffalse
%<*samplepart3>
%\fi
% Some text for part 3:
%    \begin{macrocode}
some text in part three
%    \end{macrocode}

%\iffalse
%</samplepart3>
%\fi
% Some text for part 4:
%\iffalse
%<*samplepart4>
%\fi
%    \begin{macrocode}
more text in part four
%    \end{macrocode}

%\iffalse
%</samplepart4>
%\fi
%
% %%%%%%%%%%%%%%%%%%%%%%%%%%%%%%%%%%%%%%
% \paragraph{Forwarding for a Complete Draft.}
%
% The following forwarding file |cdocsdrf.tex|
% compiles the main document in draft mode:
%\iffalse
%<*sampledraft>
%\fi
%    \begin{macrocode}
\def\version{draft}
\input{childdoc.def}
\childdocforward{cdocsamp}
%    \end{macrocode}

%\iffalse
%</sampledraft>
%\fi
%
% %%%%%%%%%%%%%%%%%%%%%%%%%%%%%%%%%%%%%%
% \paragraph{Forwarding for Final Version of the Chapters.}
%
% The following forwarding files |cdocsfn1.tex| and |cdocsfn2.tex|
% (with identical content)
% compile the final versions of the child documents
% |cdocsch1.tex| and |cdocsch2.tex|, respectively:
%\iffalse
%<*samplefinal>
%\fi
%    \begin{macrocode}
\def\version{final}
\input{childdoc.def}
\childdocforwardprefix[cdocsamp]{cdocsfn}{cdocsch}
%    \end{macrocode}

%\iffalse
%</samplefinal>
%\fi
%
% %%%%%%%%%%%%%%%%%%%%%%%%%%%%%%%%%%%%%%
% \paragraph{Command Line Processing.}
%
% The following three command lines generate the output files
% |cdocscld|, |cdocscl1| and |cdocscl2|
% which should be identical to
% |cdocsdrf|, |cdocsch1| and |cdocsfn2|, respectively:
% \begin{center}
% \begin{tabular}{l}
% |latex -jobname cdocscld \|\\
% |  "\def\version{draft}\input{childdoc.def}\childdocforward{cdocsamp}"|\\
% |latex -jobname cdocscl1 \|\\
% |  "\input{childdoc.def}\childdocforward[cdocsamp]{cdocsch1}"|\\
% |latex -jobname cdocscl2 \|\\
% |  "\def\version{final}\input{childdoc.def}\childdocforward{cdocsch2}"|
% \end{tabular}
% \end{center}
% Note that the trailing backslash on each first line
% merely continues the input to the second line
% (for convenient cut ant paste).
% Furthermore, the command |latex| can be replaced by any
% of its alternative versions such as |pdflatex|.
%
% %%%%%%%%%%%%%%%%%%%%%%%%%%%%%%%%%%%%%%%%%%%%%%%%%%%%%%%%%%%%%%%%%%%%%%%%%%%%%%
% %%%%%%%%%%%%%%%%%%%%%%%%%%%%%%%%%%%%%%%%%%%%%%%%%%%%%%%%%%%%%%%%%%%%%%%%%%%%%%
% \section{Implementation}
%\iffalse
%<*package>
%\fi
%
% This section describes the definitions file |childdoc.def|.

% The definitions cannot be loaded using |\usepackage| or |\RequirePackage|
% which has a mechanism to prevent loading a style file more than once.
% When loading the definitions by means of |\input|
% multiple instances have to be prevented manually:
%\iffalse
%This code needs to be before the `\ProvidesFile' directive
%which is defined at the beginning of this file.
%Therefore it is also placed there and commented out here.
%</package>
%<*discard>
%\fi
%    \begin{macrocode}
\ifdefined\childdocmain\endinput\fi
%    \end{macrocode}
%\iffalse
%</discard>
%<*package>
%\fi
%
% \macro{\ifchilddoc}
% \macro{\ifchilddocmanual}
% The conditional |\ifchilddoc| tells whether a
% child (true) or main (false) document is being compiled.
% The conditional |\ifchilddocmanual| tells whether
% the |\includeonly| mechanism is used (false) or
% the selection of child files must be performed manually (true).
% The definitions initialise to false:
%    \begin{macrocode}
\newif\ifchilddoc
\newif\ifchilddocmanual
%    \end{macrocode}

% \macro{\childdocname}
% \macro{\childdocjob}
% The macro |\childdocname| stores the name of the main document
% to be compiled. The macro |\childdocjob| stores the name of
% the document on which the \LaTeX{} compiler was originally invoked.
% The content of |\jobname| cannot be compared
% to filenames specified in the source due to different catcodes.
% The following code rescans |\jobname|, stores the result
% in |\childdocname| and saves a copy in |\childdocjob|:
%    \begin{macrocode}
\edef\childdocname{\scantokens\expandafter{\jobname\noexpand}}
\let\childdocjob\childdocname
%    \end{macrocode}

% \macro{\childdocdisable}
% The macro |\childdocdisable| prevents the main file
% from being processed more than once.
% At this stage, the main document command |\childdocmain|
% is assumed to be called once again where it should do nothing.
% Any subsequent call to it should prevent
% a secondary processing of the main document
% It overwrites the forwarding commands
% |\childdocof| and |\childdocforward|
% with empty macros to prevent further inclusions of the main document:
%    \begin{macrocode}
\newcommand{\childdocdisable}
{
  \renewcommand{\childdocmain}[1]{\renewcommand{\childdocmain}[1]{\endinput}}
  \renewcommand{\childdocof}[1]{}
  \renewcommand{\childdocby}[2][]{}
  \renewcommand{\childdocforward}[2][]{}
  \renewcommand{\childdocdisable}{}
}
%    \end{macrocode}

% \macro{\childdocmain}
% The macro |\childdocmain| is to be called at the top of the main file
% with nothing or the main filename (without extension) as argument.
% First, it breaks loops.
% If the argument is not empty and does not match |\childdocname|
% (which is set by the first inclusion of |childdoc.def|),
% |\ifchilddoc| is set to true, |\includeonly| is applied to the child file
% and |\jobname| is set to the main file
% (for proper handling of |.aux| files):
%    \begin{macrocode}
\newcommand{\childdocmain}[1]
{
  \childdocdisable\childdocmain{}
  \if?#1?\else
    \begingroup
      \def\childdoctmp{#1}
      \ifx\childdoctmp\childdocname
        \def\childdoctmp{}
      \else
        \def\childdoctmp
        {
          \childdoctrue
          \includeonly{\childdocname}
          \def\childdocjob{#1}
          \def\jobname{#1}
        }
      \fi
      \expandafter
    \endgroup
    \childdoctmp
  \fi
}
%    \end{macrocode}

% \macro{\childdocof}
% The command |\childdocof| redirects
% compilation to the main file |#1|.
%    \begin{macrocode}
\newcommand{\childdocof}[1]
{
  \childdocdisable
  \childdoctrue
  \includeonly{\childdocname}
  \def\jobname{#1}
  \def\childdocjob{#1}
  \input{#1}
}
%    \end{macrocode}

% \macro{\childdocby}
% The command |\childdocby| ....
%    \begin{macrocode}
\newcommand{\childdocby}[2][]
{
  \childdocdisable
  \childdoctrue
  \childdocmanualtrue
  \if?#1?\else
    \def\jobname{#2}
  \fi
  \def\childdocjob{#2}
  \input{#2}
  \endinput
}
%    \end{macrocode}

% \macro{\childdocforward}
% The command |\childdocforward| redirects
% compilation to the main file or
% (if the optional argument is given) a child file.
% Parameters are set as if the main file
% or a child file starting with |\childdocof| was compiled.
% Then compilation is handed over to the main file:
%    \begin{macrocode}
\newcommand{\childdocforward}[2][]
{
  \begingroup
    \if?#1?
      \def\childdoctmp
      {
        \def\childdocname{#2}
        \def\childdocjob{#2}
        \def\jobname{#2}
        \input{#2}
        \endinput
      }
    \else
      \def\childdoctmp
      {
        \childdocdisable
        \def\childdocname{#2}
        \childdoctrue
        \includeonly{#2}
        \def\childdocjob{#1}
        \def\jobname{#1}
        \input{#1}
        \endinput
      }
    \fi
    \expandafter
  \endgroup
  \childdoctmp
}
%    \end{macrocode}

% \macro{\childdocforwardprefix}
% The command |\childdocforwardprefix| redirects
% compilation to the main or a child file by means of a pattern.
% The prefix |#1| in the current filename is replaced by |#2|
% and the suffix of the current filename is kept
% (it is assumed that the filename does not contain the substring `|~~~|'
% which is used as a delimiter).
% Compilation is handed over to the new file by |\childdocforward|:
%    \begin{macrocode}
\newcommand{\childdocforwardprefix}[3][]
{
  \begingroup
    \def\childdocextract #2##1~~~{\def\childdoctmp{\childdocforward[#1]{#3##1}}}
    \expandafter\childdocextract\childdocname~~~
    \expandafter
  \endgroup
  \childdoctmp
}
%    \end{macrocode}

% \macro{\childdoc}
% The deprecated macro |\childdoc| is a legacy version of |\childdocmain|:
%    \begin{macrocode}
\newcommand{\childdoc}{\childdocmain}
%    \end{macrocode}

% \macro{\childdocredirect}
% The deprecated macro |\childdocredirect| is a legacy version
% of |\childdocforward| and |\childdocforwardprefix|:
%    \begin{macrocode}
\newcommand{\childdocredirect}[2][]
{
  \begingroup
    \if?#1?
      \def\childdoctmp{\childdocforward{#2}}
    \else
      \def\childdoctmp{\childdocforwardprefix{#1}{#2}}
    \fi
    \expandafter
  \endgroup
  \childdoctmp
}
%    \end{macrocode}

%\iffalse
%</package>
%\fi
%
\endinput
|
and perform the replacements as outlined below.
Instead of |\childdocmain{|\textit{main}|}| add the following code
to the top of the main file:
%
\begin{center}
\begin{tabular}{l}
|\||ifdefined\childdocname\endinput\||fi\newif\ifchilddoc|\\
|\edef\childdocname{\scantokens\expandafter{\jobname\noexpand}}|\\
|\def\childdocmain{|\textit{main}|}\||ifx\childdocmain\childdocname\||else|\\
|\childdoctrue\includeonly{\childdocname}\let\jobname\childdocmain\||fi|\\
\end{tabular}
\end{center}
%
Instead of |\childdocof{|\textit{main}|}| just include the main file
at the top of each child file:
%
\begin{center}
|\input{|\textit{main}|}|
\end{center}
%
A simple redirection |\childdocforward{|\textit{dest}|}| is achieved by:
%
\begin{center}
|\def\jobname{|\textit{dest}|}\input{\jobname}|
\end{center}
%
The redirection with prefix
|\childdocforwardprefix[|\textit{prefix}|]{|\textit{dest}|}|
is accomplished by:
%
\begin{center}
\begin{tabular}{l}
|{\edef\jobname{\scantokens\expandafter{\jobname\noexpand}}|\\
|\def\redirectjob |\textit{prefix}|#1~~~{\gdef\jobname{|\textit{dest}|#1}}|\\
|\expandafter\redirectjob\jobname~~~}\input{\jobname}|
\end{tabular}
\end{center}

In an alternative approach,
child documents can be compiled by a specific command line
without additional code or specific definitions:
%
\begin{center}
|... -jobname "|\textit{target}|" "|[\textit{flags}]%
|\includeonly{|\textit{dest}|}\input{|\textit{main}|}"|
\end{center}
%

%%%%%%%%%%%%%%%%%%%%%%%%%%%%%%%%%%%%%%%%%%%%%%%%%%%%%%%%%%%%%%%%%%%%%%%%%%%%%%%%
%%%%%%%%%%%%%%%%%%%%%%%%%%%%%%%%%%%%%%%%%%%%%%%%%%%%%%%%%%%%%%%%%%%%%%%%%%%%%%%%
\section{Information}

%%%%%%%%%%%%%%%%%%%%%%%%%%%%%%%%%%%%%%%%%%%%%%%%%%%%%%%%%%%%%%%%%%%%%%%%%%%%%%%%
\subsection{Copyright}

Copyright \copyright{} 2017--2018 Niklas Beisert

This work may be distributed and/or modified under the
conditions of the \LaTeX{} Project Public License, either version 1.3
of this license or (at your option) any later version.
The latest version of this license is in
  \url{http://www.latex-project.org/lppl.txt}
and version 1.3 or later is part of all distributions of \LaTeX{}
version 2005/12/01 or later.

This work has the LPPL maintenance status `maintained'.

The Current Maintainer of this work is Niklas Beisert.

This work consists of the files |README.txt|, |childdoc.ins| and |childdoc.dtx|
as well as the derived files |childdoc.def|, |cdocsamp.tex|
with |cdocsch1.tex|, |cdocsch2.tex|, |cdocspt3.tex|, |cdocspt4.tex|,
|cdocsdrf.tex|, |cdocsfn1.tex|, |cdocsfn2.tex|
as well as |childdoc.pdf|.

%%%%%%%%%%%%%%%%%%%%%%%%%%%%%%%%%%%%%%%%%%%%%%%%%%%%%%%%%%%%%%%%%%%%%%%%%%%%%%%%
\subsection{Files and Installation}

The package consists of the files:
%
\begin{center}
\begin{tabular}{ll}
    |README.txt|   & readme file \\
    |childdoc.ins| & installation file \\
    |childdoc.dtx| & source file \\
    |childdoc.def| & definition file \\
    |cdocsamp.tex| & sample main file \\
    |cdocsch1.tex| & sample include file \\
    |cdocsch2.tex| & sample include file \\
    |cdocspt3.tex| & sample part file \\
    |cdocspt4.tex| & sample part file \\
    |cdocsdrf.tex| & sample redirection file \\
    |cdocsfn1.tex| & sample redirection file \\
    |cdocsfn2.tex| & sample redirection file \\
    |childdoc.pdf| & manual
\end{tabular}
\end{center}
%
The distribution consists of the files
|README.txt|, |childdoc.ins| and |childdoc.dtx|.
%
\begin{itemize}
\item
Run (pdf)\LaTeX{} on |childdoc.dtx|
to compile the manual |childdoc.pdf| (this file).
\item
Run \LaTeX{} on |childdoc.ins| to create the definitions file |childdoc.def|
and the sample |cdocsamp.tex| with include files
|cdocsch1.tex|, |cdocsch2.tex|, |cdocspt3.tex|, |cdocspt4.tex|,
|cdocsdrf.tex|, |cdocsfn1.tex|, |cdocsfn2.tex|.
Then copy the file |childdoc.def| to an appropriate directory of your \LaTeX{}
distribution, e.g.\ \textit{texmf-root}|/tex/latex/childdoc|.
\end{itemize}

%%%%%%%%%%%%%%%%%%%%%%%%%%%%%%%%%%%%%%%%%%%%%%%%%%%%%%%%%%%%%%%%%%%%%%%%%%%%%%%%
\subsection{Related CTAN Packages}

There are several other packages which offer a similar functionality:
%
\begin{itemize}
\item
The packages
\href{http://ctan.org/pkg/docmute}{\textsf{docmute}},
\href{http://ctan.org/pkg/includex}{\textsf{includex}} and
\href{http://ctan.org/pkg/standalone}{\textsf{standalone}}
provide commands to include only the document body of
a child file thus allowing both files to be compiled individually.
\item
The packages \href{http://ctan.org/pkg/subdocs}{\textsf{subdocs}}
and \href{http://ctan.org/pkg/subfiles}{\textsf{subfiles}}
provide structures in which the main and child documents can be
encapsulated and allowing them to be compiled individually.
The inclusion mechanism is different from the conventional |\include|.
\item
The package \href{http://ctan.org/pkg/combine}{\textsf{combine}}
is an elaborate solution to combine several documents into one.
\end{itemize}
%
See also the CTAN topic \href{http://ctan.org/topic/subdocs}{\textsf{subdocs}}
for further related packages.
The present package differs from the above solutions in that
a document structure constructed with the conventional |\include| mechanism
just needs two extra commands at the top of every file
such that all constituent files can be compiled individually.

%%%%%%%%%%%%%%%%%%%%%%%%%%%%%%%%%%%%%%%%%%%%%%%%%%%%%%%%%%%%%%%%%%%%%%%%%%%%%%%%
%\subsection{Feature Suggestions}
%
%The following is a list of features which may be useful for future
%versions of this package:
%%
%\begin{itemize}
%\item
%\ldots
%\end{itemize}

%%%%%%%%%%%%%%%%%%%%%%%%%%%%%%%%%%%%%%%%%%%%%%%%%%%%%%%%%%%%%%%%%%%%%%%%%%%%%%%%
\subsection{Revision History}

%%%%%%%%%%%%%%%%%%%%%%%%%%%%%%%%%%%%%%%%
\paragraph{v2.0:} 2018/12/30

\begin{itemize}
\item
immediate forward processing
\item
added |\childdocby| mechanism
\item
manual restructured
\end{itemize}

%%%%%%%%%%%%%%%%%%%%%%%%%%%%%%%%%%%%%%%%
\paragraph{v1.6:} 2018/01/17

\begin{itemize}
\item
application for development of include files
\item
corrections to manual
\end{itemize}

%%%%%%%%%%%%%%%%%%%%%%%%%%%%%%%%%%%%%%%%
\paragraph{v1.5:} 2017/05/21

\begin{itemize}
\item
more complete structuring introduced
\item
|\childdocof| introduced
\item
|\childdoc| renamed to |\childdocmain|
\item
|\childredirect| renamed to |\childdocforward| and |\childdocforwardprefix|
and functionality expanded
\end{itemize}

%%%%%%%%%%%%%%%%%%%%%%%%%%%%%%%%%%%%%%%%
\paragraph{v1.0:} 2017/04/27

\begin{itemize}
\item
manual and install package
\item
first version published on CTAN
\end{itemize}

%%%%%%%%%%%%%%%%%%%%%%%%%%%%%%%%%%%%%%%%
\paragraph{v0.6:} 2017/04/26

\begin{itemize}
\item
redirection mechanism added
\end{itemize}

%%%%%%%%%%%%%%%%%%%%%%%%%%%%%%%%%%%%%%%%
\paragraph{v0.5:} 2017/04/26

\begin{itemize}
\item
functionality in definition file
\end{itemize}


%%%%%%%%%%%%%%%%%%%%%%%%%%%%%%%%%%%%%%%%%%%%%%%%%%%%%%%%%%%%%%%%%%%%%%%%%%%%%%%%
%%%%%%%%%%%%%%%%%%%%%%%%%%%%%%%%%%%%%%%%%%%%%%%%%%%%%%%%%%%%%%%%%%%%%%%%%%%%%%%%
%%%%%%%%%%%%%%%%%%%%%%%%%%%%%%%%%%%%%%%%%%%%%%%%%%%%%%%%%%%%%%%%%%%%%%%%%%%%%%%%
\appendix

\settowidth\MacroIndent{\rmfamily\scriptsize 000\ }

 \DocInput{childdoc.dtx}

\end{document}
%</driver>
% \fi
%
% %%%%%%%%%%%%%%%%%%%%%%%%%%%%%%%%%%%%%%%%%%%%%%%%%%%%%%%%%%%%%%%%%%%%%%%%%%%%%%
% %%%%%%%%%%%%%%%%%%%%%%%%%%%%%%%%%%%%%%%%%%%%%%%%%%%%%%%%%%%%%%%%%%%%%%%%%%%%%%
% \section{Sample}
%\iffalse
%<*samplemain>
%\fi
%
% The following presents a sample document
% with two chapters, two parts, a title page,
% a compile flag as well as three forwarding files to set the flag.
% It consists of eight |.tex| files:
% \begin{center}
% \begin{tabular}{ll}
% |cdocsamp.tex|&main file\\
% |cdocsch1.tex|&include file for chapter 1\\
% |cdocsch2.tex|&include file for chapter 2\\
% |cdocspt3.tex|&include file for part 3\\
% |cdocspt4.tex|&include file for part 4\\
% |cdocsdrf.tex|&forwarding file for main file in draft mode\\
% |cdocsfi1.tex|&forwarding file for final version of chapter 1\\
% |cdocsfi2.tex|&forwarding file for final version of chapter 2\\
% \end{tabular}
% \end{center}
% Each of the eight files can be compiled directly by the \LaTeX{} compiler.
%
% %%%%%%%%%%%%%%%%%%%%%%%%%%%%%%%%%%%%%%
% \paragraph{Main File.}
%
% The main file is called |cdocsamp.tex|.
%
% Load the \textsf{childdoc} definitions and
% declare the filename for the main document:
%    \begin{macrocode}
% \iffalse
%
% childdoc.dtx Copyright (C) 2017-2018 Niklas Beisert
%
% This work may be distributed and/or modified under the
% conditions of the LaTeX Project Public License, either version 1.3
% of this license or (at your option) any later version.
% The latest version of this license is in
%   http://www.latex-project.org/lppl.txt
% and version 1.3 or later is part of all distributions of LaTeX
% version 2005/12/01 or later.
%
% This work has the LPPL maintenance status `maintained'.
%
% The Current Maintainer of this work is Niklas Beisert.
%
% This work consists of the files childdoc.dtx and childdoc.ins
% and the derived files childdoc.def and cdocsamp.tex with
% cdocsch1.tex, cdocsch2.tex, cdocsdrf.tex, cdocsfn1.tex, cdocsfn2.tex.
%
%<package>\ifdefined\childdocmain\endinput\fi
%<package>\ProvidesFile{childdoc.def}[2018/12/30 v2.0 child document driver]
%<samplemain>\ProvidesFile{cdocsamp.tex}[2018/12/30 v2.0 sample for childdoc]
%<*driver>
%\ProvidesFile{childdoc.drv}[2018/12/30 v2.0 childdoc reference manual file]
\PassOptionsToClass{10pt,a4paper}{article}
\documentclass{ltxdoc}

\usepackage[margin=35mm]{geometry}
\usepackage{hyperref}
\usepackage{hyperxmp}
\usepackage[usenames]{color}

\hypersetup{colorlinks=true}
\hypersetup{pdfstartview=FitH}
\hypersetup{pdfpagemode=UseNone}
\hypersetup{pdfsource={}}
\hypersetup{pdflang={en-UK}}
\hypersetup{pdfcopyright={Copyright 2017-2018 Niklas Beisert.
  This work may be distributed and/or modified under the
  conditions of the LaTeX Project Public License, either version 1.3
  of this license or (at your option) any later version.}}
\hypersetup{pdflicenseurl={http://www.latex-project.org/lppl.txt}}
\hypersetup{pdfcontactaddress={ETH Zurich, ITP, HIT K,
  Wolfgang-Pauli-Strasse 27}}
\hypersetup{pdfcontactpostcode={8093}}
\hypersetup{pdfcontactcity={Zurich}}
\hypersetup{pdfcontactcountry={Switzerland}}
\hypersetup{pdfcontactemail={nbeisert@itp.phys.ethz.ch}}
\hypersetup{pdfcontacturl={http://people.phys.ethz.ch/\xmptilde nbeisert/}}

\newcommand{\secref}[1]{\hyperref[#1]{section \ref*{#1}}}

\parskip1ex
\parindent0pt
\let\olditemize\itemize
\def\itemize{\olditemize\parskip0pt}

\begin{document}

\title{The \textsf{childdoc} Package}
\hypersetup{pdftitle={The childdoc Package}}
\author{Niklas Beisert\\[2ex]
  Institut f\"ur Theoretische Physik\\
  Eidgen\"ossische Technische Hochschule Z\"urich\\
  Wolfgang-Pauli-Strasse 27, 8093 Z\"urich, Switzerland\\[1ex]
  \href{mailto:nbeisert@itp.phys.ethz.ch}
  {\texttt{nbeisert@itp.phys.ethz.ch}}}
\hypersetup{pdfauthor={Niklas Beisert}}
\hypersetup{pdfsubject={Manual for the LaTeX2e Package childdoc}}
\date{30 December 2018, \textsf{v2.0}}
\maketitle

\begin{abstract}\noindent
\textsf{childdoc} is a \LaTeXe{} package
that enables the direct compilation
of document sections included by |\include|
to individual files.
\end{abstract}

\begingroup
\parskip0ex
\tableofcontents
\endgroup

%%%%%%%%%%%%%%%%%%%%%%%%%%%%%%%%%%%%%%%%%%%%%%%%%%%%%%%%%%%%%%%%%%%%%%%%%%%%%%%%
%%%%%%%%%%%%%%%%%%%%%%%%%%%%%%%%%%%%%%%%%%%%%%%%%%%%%%%%%%%%%%%%%%%%%%%%%%%%%%%%
\section{Introduction}

\LaTeX{} provides a mechanism to structure a large document (such as a book)
into a main file and several child files (containing the chapters)
using the |\include| command.
This mechanism is beneficial for documents
which span hundreds of pages in order to
make the source file(s) more manageable.
Moreover, compilation can be restricted to
selected child files by means of the |\includeonly| command.
The latter feature can be used to reduce the compilation time while editing
(this was significantly more useful in the earlier days of \LaTeX{})
or to generate a smaller document which is easier to navigate.
Another application of |\includeonly| is to generate
documents consisting of selected parts of the complete document.

However, there are a few drawbacks of the plain |\include| mechanism:
\begin{itemize}
\item
The child files cannot be compiled on their own,
they can only be compiled via the main file.
A naive editing environment
(such as a text editor with an option
to have the current file processed by \LaTeX)
may require one to switch to the main file before compiling;
attempting to compile the child file produces errors.
\item
The main file must be modified (each time)
to adjust the |\includeonly| command
to the present needs. This easily leaves the main file in a messy state.
\item
The generated document will always carry the filename
of the main document. This is inconvenient if
several child files are to be compiled and
to be kept for distribution.
\end{itemize}

The present package provides a simple interface
to make child files individually compilable by \LaTeX{}.
Compiling a child file then has the same effect as compiling
the main file with an |\includeonly| command
to select the appropriate child.
Moreover the generated document will carry the name of the child
rather than the main file.
This resolves all three above issues.

This feature is meant to make the editing of books,
thesis documents and lecture notes somewhat more convenient.
However, the package can also be used efficiently for
composing a series of documents (such as exercise sheets)
which are typically distributed individually.
It then assists the author in generating the individual documents
(potentially in different versions)
as well as a document containing the collected series.
Another application is in developing style files
or other kinds of included material
where compilation of the style file could redirect
to a sample or test file.

%%%%%%%%%%%%%%%%%%%%%%%%%%%%%%%%%%%%%%%%%%%%%%%%%%%%%%%%%%%%%%%%%%%%%%%%%%%%%%%%
%%%%%%%%%%%%%%%%%%%%%%%%%%%%%%%%%%%%%%%%%%%%%%%%%%%%%%%%%%%%%%%%%%%%%%%%%%%%%%%%
\section{Usage}

First of all, the package \textsf{childdoc} is \emph{not} a standard
\LaTeXe{} |.sty| style file! Therefore it needs to be invoked in
a non-standard way.

%%%%%%%%%%%%%%%%%%%%%%%%%%%%%%%%%%%%%%%%%%%%%%%%%%%%%%%%%%%%%%%%%%%%%%%%%%%%%%%%
\subsection{Included Files}
\label{sec:include}

%%%%%%%%%%%%%%%%%%%%%%%%%%%%%%%%%%%%%%%%
\DescribeMacro{\childdocmain}
To use the package, add the commands
\begin{center}
\begin{tabular}{l}
|\input{childdoc.def}|\\
|\childdocmain{}|\\
\end{tabular}
\end{center}
at the very top of the main \LaTeX{} file,
in particular \emph{before} the |\documentclass| statement!
The argument of |\childdocmain| should be left empty
(but it must be present).

%%%%%%%%%%%%%%%%%%%%%%%%%%%%%%%%%%%%%%%%
\DescribeMacro{\childdocof}
Furthermore, add the commands
\begin{center}
\begin{tabular}{l}
|\input{childdoc.def}|\\
|\childdocof{|\textit{main}|}|\\
\end{tabular}
\end{center}
at the top of every child file \textit{child}
which is included by |\include{|\textit{child}|}|
from within the main file
(or at least for those files to be compiled individually).
The argument \textit{main} must be the filename of the main file.

There are a couple of
considerations in setting up the main and child documents:

%%%%%%%%%%%%%%%%%%%%%%%%%%%%%%%%%%%%%%%%
\paragraph{Restrictions.}

Please note the following restrictions:
\begin{itemize}
\item
|\childdocmain| must be called with one argument \textit{main}
to ensure compatibility with earlier version of the package.
It must either be empty (|\childdocmain{}|)
or precisely match the filename of the main file in which it is specified.
See \secref{sec:detection} for further information.
\item
The filename \textit{main} must be specified without the |.tex| extension.
\item
The filename \textit{main} is case sensitive
(even in case-insensitive file systems)
due to internal string comparison.
\item
The argument \textit{main} should be fully expanded, it cannot be a macro.
\item
Subdirectories and special characters should be avoided in filenames.
\item
The command |\childdocmain{|\textit{main}|}| must be followed by a whitespace.
It should not be followed immediately by another command
or by a comment mark `|%|'.
This is because the \TeX{} parser reads the token immediately following
the argument of |\childdocmain| and puts it
at the beginning of every child section;
however, a white\-space is ignored.
\end{itemize}

%%%%%%%%%%%%%%%%%%%%%%%%%%%%%%%%%%%%%%%%
\paragraph{Content of Main File.}

It is advisable to place all content in the child files included by |\include|.
Any output contained in the main file will appear in all child documents
unless suppressed manually;
it cannot be suppressed automatically by the |\includeonly| directive
and thus should normally be avoided.
A method to include some content in the main file
by means of conditional processing is described in \secref{sec:conditional}.

%%%%%%%%%%%%%%%%%%%%%%%%%%%%%%%%%%%%%%%%
\paragraph{Page Numbering.}

When only a part of the document is compiled,
the appropriate numbering of pages
(as well as other status parameters)
is determined from the |.aux| files.
The latter contain information from previous passes.
However this information needs to propagate through
all intermediate child documents.
Therefore the page numbering in child documents may well
be inconsistent until the complete document is compiled at least once.

A useful (if unconventional) way to always ensure a consistent
page numbering is to restart the numbering in each child document
and denote the pages by `\textit{child}|.|\textit{page}'
where \textit{child} represents the chapter/section number of the child file.
This can be achieved by the command
|\numberwithin{page}{|\textit{child}|}|
of the \textsf{amsmath} package
where \textit{child} can be |chapter| or |section|
depending on the chosen structuring.
Alternatively, one can modify the macro |\thepage| appropriately
and reset the counter |page| at the start of each child file.

%%%%%%%%%%%%%%%%%%%%%%%%%%%%%%%%%%%%%%%%%%%%%%%%%%%%%%%%%%%%%%%%%%%%%%%%%%%%%%%%
\subsection{Conditional Processing}
\label{sec:conditional}

The package provides a mechanism to compile different versions
of a document. To customise the versions further some conditional processing
can come in handy to distinguish which version is being compiled.
The package provides two macros to describe the compilation context:

%%%%%%%%%%%%%%%%%%%%%%%%%%%%%%%%%%%%%%%%
\DescribeMacro{\ifchilddoc}
The conditional |\ifchilddoc| distinguishes between the compilation of
child documents and the main document:
%
\begin{center}
|\ifchilddoc |\textit{child-code}| |[|\||else |\textit{main-code}]| \||fi|
\end{center}

%%%%%%%%%%%%%%%%%%%%%%%%%%%%%%%%%%%%%%%%
\DescribeMacro{\childdocname}
\DescribeMacro{\childdocjob}
The macro |\childdocname| contains the filename (without extension)
of the main or child file being processed.
Note that |\childdocjob| will always contain the name of the main file.

%%%%%%%%%%%%%%%%%%%%%%%%%%%%%%%%%%%%%%%%
\paragraph{Title Page.}

Conditional processing can be used to include a title or banner page
in the main document when proper precautions are taken.
Importantly, the code in the main file should ensure that the page counter
(as well as other status parameters which are stored in the |.aux| files)
takes the same value after the conditional processing.
Otherwise the page numbers may take divergent values
depending on which part is compiled.

For example, a title page could be declared by:
%
\begin{center}
\begin{tabular}{l}
|\ifchilddoc\||else|\\
|\addtocounter{page}{-1}|\\
\textit{code for title page}\\
|\newpage|\\
|\||fi|
\end{tabular}
\end{center}
%
A banner page for the child documents can be generated by:
%
\begin{center}
\begin{tabular}{l}
|\ifchilddoc|\\
|\addtocounter{page}{-1}|\\
\textit{code for banner page}\\
|\newpage|\\
|\||fi|
\end{tabular}
\end{center}
%
Here one could write a message such as:
\begin{center}
|This is the part \childdocname{} of \childdocjob{}.|
\end{center}

%%%%%%%%%%%%%%%%%%%%%%%%%%%%%%%%%%%%%%%%%%%%%%%%%%%%%%%%%%%%%%%%%%%%%%%%%%%%%%%%
\subsection{Flags}
\label{sec:flags}

The package makes it easy to generate different versions
of the main or child documents.
To this end compilation flags can be defined
and assigned different default values.
They will be particularly useful in conjunction
with the forwarding mechanism described in \secref{sec:forward}.

For example, it may be useful to have a flag |\version|
which can be set to |draft| or |final|.
The document source will contain some conditional code
depending on the value of |\version|.
Suppose further, the flag should default to |final| for the main file
and to |draft| for child files
which is a natural assignment for editing the document.
This is achieved by placing the following code
in the preamble of the main document
(below the |\childdocmain| directive):
%
\begin{center}
\begin{tabular}{l}
|\ifchilddoc|\\
|\providecommand{\version}{draft}|\\
|\||else|\\
|\providecommand{\version}{final}|\\
|\||fi|
\end{tabular}
\end{center}
%
The definition by |\providecommand| makes sure
that previous definitions are not overwritten.
Further statements |\providecommand{\version}{...}|
can thus be added before the above code to override it.

For the main file, one might add a line
(between |\childdocmain| and the above block)
%
\begin{center}
|%\ifchilddoc\||else\providecommand{\version}{draft}\||fi|
\end{center}
%
which can be uncommented to produce a draft version.
Likewise one can add a line to the very top of a child file
(above the |\childdocof{|\textit{main}|}| directive)
%
\begin{center}
|%\providecommand{\version}{final}|
\end{center}
%
which can be uncommented to produce the final version of this child document.

%%%%%%%%%%%%%%%%%%%%%%%%%%%%%%%%%%%%%%%%%%%%%%%%%%%%%%%%%%%%%%%%%%%%%%%%%%%%%%%%
\subsection{Forwarding}
\label{sec:forward}

Different versions of the main or child documents
using compilation flags as described in \secref{sec:flags}
can be (permanently) stored in different files
for convenient compilation, viewing and distribution.
To this end, the package defines a command
to pass on compilation to a different file:

%%%%%%%%%%%%%%%%%%%%%%%%%%%%%%%%%%%%%%%%
\DescribeMacro{\childdocforward}
The command |\childdocforward| redirects processing to
another source file:
%
\begin{center}
\begin{tabular}{l}
|\input{childdoc.def}|\\
|\childdocforward[|\textit{main}|]{|\textit{dest}|}|\\
\end{tabular}
\end{center}
%
The argument \textit{dest} is the destination file
(without extension).
It should be the main file or one of the child files.
Note that further \textsf{childdoc} directives
such as |\childdocof| and |\childdocforward|
in the indicated file will be processed in this form.
The optional argument \textit{main}
passes on directly to the main file \textit{main}
while pretending to compile the child \textit{dest}.
This form behaves as if \textit{dest}
issues |\childdocof{|\textit{main}|}| right away,
and no further \textsf{childdoc} directives will be processed.

%%%%%%%%%%%%%%%%%%%%%%%%%%%%%%%%%%%%%%%%
\DescribeMacro{\...prefix}
In the alternative form |\childdocforwardprefix|,
%
\begin{center}
\begin{tabular}{l}
|\input{childdoc.def}|\\
|\childdocforwardprefix[|\textit{main}|]{|\textit{prefix}|}{|\textit{dest}|}|
\end{tabular}
\end{center}
%
the destination file is determined by a pattern
depending on the current file:
To make this work, the current file must be called
`{\textit{prefix}\hspace{0.2em}\textit{suffix}}'
with \textit{prefix} matching precisely the argument.
Processing is then passed on to the file
`{\textit{dest}\hspace{0.2em}\textit{suffix}}'.
Surely, the same effect is achieved by
directly specifying the
argument `{\textit{dest}\hspace{0.2em}\textit{suffix}}'
in the first form.
However, that requires to set up a different file
for each child. With the alternative form of the command
all these files can have exactly the same content
which simplifies setting them up and maintaining them.

For example, the following file |draft.tex|
with a compilation flag |\version| as described in \secref{sec:flags}
compiles the main document as a draft:
%
\begin{center}
\begin{tabular}{l}
|\def\version{draft}|\\
|\input{childdoc.def}|\\
|\childdocforward{|\textit{main}|}|
\end{tabular}
\end{center}
%
Likewise, the following files |final|\textit{nn}|.tex|
compile the final version of the child document
|child|\textit{nn}|.tex|:
%
\begin{center}
\begin{tabular}{l}
|\def\version{final}|\\
|\input{childdoc.def}|\\
|\childdocforwardprefix{final}{child}|
\end{tabular}
\end{center}
%

Note that when several versions of a main file and/or of each child file
are to be generated, it may be convenient to set up a |Makefile| or
shell script to automatise the process.

%%%%%%%%%%%%%%%%%%%%%%%%%%%%%%%%%%%%%%%%%%%%%%%%%%%%%%%%%%%%%%%%%%%%%%%%%%%%%%%%
\subsection{Command Line Processing}
\label{sec:commandline}

The effect of redirection files can also be achieved by invoking
the \LaTeX{} compiler with a more elaborate command line.
Most conveniently this should be done as part
of a shell script or a |Makefile|.

When using \textsf{childdoc} in the main file, the following
command lines effectively perform a redirection
(note that depending on the shell being used,
backslashes may have to be doubled: `|\|' $\to$ `|\\|'):
%
\begin{center}
|... -jobname "|\textit{target}|" |\\|"|[\textit{flags}]%
|\input{childdoc.def}\childdocforward[|\textit{main}|]{|\textit{dest}|}"|
\end{center}
%
Here \textit{target} is the name of the output file,
\textit{main} is the name of the main file
and \textit{dest} is the name of the main or child file to be processed
(all filenames without extensions).
The optional argument \textit{main} can be omitted
if \textit{main} matches \textit{dest}.
Optionally, compilation \textit{flags} can be defined via |\def| commands.
This command line makes the \TeX{} engine believe
it is compiling the file \textit{target}
whose content is specified as the latter parameter.
The provided code then forwards the processing to
\textit{main} or \textit{dest} as described in \secref{sec:forward}.

%%%%%%%%%%%%%%%%%%%%%%%%%%%%%%%%%%%%%%%%%%%%%%%%%%%%%%%%%%%%%%%%%%%%%%%%%%%%%%%%
\subsection{Include by Input}
\label{sec:input}

Including child documents by |\include| has some restrictions by design.
Most notably, the content of a child document always occupies
its own set of pages; pages cannot be shared between child documents.
Usually, this behaviour makes perfect sense
because each child document contain an essential part of the document.
However, in some situations it may be desirable to compose
a document from a collection of parts
without having mandatory page breaks between then.
For this case, the package
provides a mechanism to include parts
by |\input| which can also be processed individually.
However, by construction this mechanism
requires manual handling of the content to be output.

%%%%%%%%%%%%%%%%%%%%%%%%%%%%%%%%%%%%%%%%
\DescribeMacro{\ifchilddocmanual}
The main file should be prepared as usual, see \secref{sec:include}.
However, the document body must make a distinction
between processing of an individual part and of the main document, e.g.:
%
\begin{center}
\begin{tabular}{l}
|\ifchilddocmanual|\\
|\input{\childdocname}|\\
|\||else|\\
\textit{document body with }|\input{|\textit{part}|}|\\
|\||fi|
\end{tabular}
\end{center}
%
The conditional |\ifchilddocmanual| is true whenever
a part to be included by |\input| is being compiled,
and the name of the part is stored in |\childdocname|.

%%%%%%%%%%%%%%%%%%%%%%%%%%%%%%%%%%%%%%%%
\DescribeMacro{\childdocby}
Each part to be included by |\input| should start with:
%
\begin{center}
\begin{tabular}{l}
|\input{childdoc.def}|\\
|\childdocby{|\textit{main}|}|\\
\end{tabular}
\end{center}
%
The directive |\childdocby| is similar to |\childdocof|
described in \secref{sec:include},
but the subsequent selection of content must be done manually.
To that end, both |\ifchilddoc| and |\ifchilddocmanual|
will be true upon processing of a part,
and the name of the part is stored in |\childdocname|.
Note that |\jobname| will be set to the filename of the current part
so that each part receives an individual |.aux| file
that does not interfere with the |.aux| file(s) of the main document.
This behaviour can be altered by the alternative form
|\childdocby[*]{|\textit{main}|}| (with a non-empty optional argument)
which uses the |.aux| file of the main document
by setting |\jobname| to \textit{main}.

%%%%%%%%%%%%%%%%%%%%%%%%%%%%%%%%%%%%%%%%%%%%%%%%%%%%%%%%%%%%%%%%%%%%%%%%%%%%%%%%
\subsection{Driver Development}
\label{sec:driver}

The \textsf{childdoc} mechanism can also be use for the development
of definition files such as \LaTeX{} styles or classes.
This case differs from the above setup with multiple parts
included by |\include| in that no |\includeonly| should be invoked.
This can be achieved by starting the include file
(before |\ProvidesPackage|) with:
%
\begin{center}
\begin{tabular}{l}
|\input{childdoc.def}|\\
|\childdocforward{|\textit{main}|}|\\
\end{tabular}
\end{center}
%
or alternatively with:
%
\begin{center}
\begin{tabular}{l}
|\input{childdoc.def}|\\
|\childdocby{|\textit{main}|}|\\
\end{tabular}
\end{center}
%
Both forms have slightly different effects as described above.
The main file is prepared as usual, see \secref{sec:include}.

%%%%%%%%%%%%%%%%%%%%%%%%%%%%%%%%%%%%%%%%%%%%%%%%%%%%%%%%%%%%%%%%%%%%%%%%%%%%%%%%
\subsection{Legacy Detection}
\label{sec:detection}

The directive |\childdocmain| in the main file can detect
whether the complete document or merely a child is to be compiled
even without using the directive |\childdocof|.
This method is deprecated because it is less robust
and there is no compelling reason to use it;
it is merely provided for backward compatibility
and it may be removed in future versions.

If the detection mechanism is to be used,
it is mandatory to correctly specify
the filename of the main file as the argument of |\childdocmain|:
%
\begin{center}
\begin{tabular}{l}
|\input{childdoc.def}|\\
|\childdocmain{|\textit{main}|}|\\
\end{tabular}
\end{center}
%
If |\jobname| does not match the argument \textit{main} of |\childdocmain|,
it is assumed that |\jobname| points to the child file to be compiled.
When using |\childdocmain| with the main file specified as argument,
it suffices to start a child file
with just |\input{|\textit{main}|}|
without loading of the package and using |\childdocof|.
If instead all processing is done
with the appropriate \textsf{childdoc} directives,
the argument of \textit{main} of |\childdocmain| can be empty.

An alternative version of the command line processing described
in \secref{sec:commandline} using the detection mechanism reads:
%
\begin{center}
|... -jobname "|\textit{target}|" "|[\textit{flags}]%
[|\def\jobname{|\textit{dest}|}|]|\input{|\textit{main}|}"|
\end{center}

%%%%%%%%%%%%%%%%%%%%%%%%%%%%%%%%%%%%%%%%%%%%%%%%%%%%%%%%%%%%%%%%%%%%%%%%%%%%%%%%
\subsection{Manual Code}
\label{sec:manual}

In case one cannot be certain whether the definitions file |childdoc.def|
is installed on the target \TeX{} distribution
and one prefers not to ship it,
it is conceivable to paste a few relevant commands into the sources.

To that end, drop all statements |\input{childdoc.def}|
and perform the replacements as outlined below.
Instead of |\childdocmain{|\textit{main}|}| add the following code
to the top of the main file:
%
\begin{center}
\begin{tabular}{l}
|\||ifdefined\childdocname\endinput\||fi\newif\ifchilddoc|\\
|\edef\childdocname{\scantokens\expandafter{\jobname\noexpand}}|\\
|\def\childdocmain{|\textit{main}|}\||ifx\childdocmain\childdocname\||else|\\
|\childdoctrue\includeonly{\childdocname}\let\jobname\childdocmain\||fi|\\
\end{tabular}
\end{center}
%
Instead of |\childdocof{|\textit{main}|}| just include the main file
at the top of each child file:
%
\begin{center}
|\input{|\textit{main}|}|
\end{center}
%
A simple redirection |\childdocforward{|\textit{dest}|}| is achieved by:
%
\begin{center}
|\def\jobname{|\textit{dest}|}\input{\jobname}|
\end{center}
%
The redirection with prefix
|\childdocforwardprefix[|\textit{prefix}|]{|\textit{dest}|}|
is accomplished by:
%
\begin{center}
\begin{tabular}{l}
|{\edef\jobname{\scantokens\expandafter{\jobname\noexpand}}|\\
|\def\redirectjob |\textit{prefix}|#1~~~{\gdef\jobname{|\textit{dest}|#1}}|\\
|\expandafter\redirectjob\jobname~~~}\input{\jobname}|
\end{tabular}
\end{center}

In an alternative approach,
child documents can be compiled by a specific command line
without additional code or specific definitions:
%
\begin{center}
|... -jobname "|\textit{target}|" "|[\textit{flags}]%
|\includeonly{|\textit{dest}|}\input{|\textit{main}|}"|
\end{center}
%

%%%%%%%%%%%%%%%%%%%%%%%%%%%%%%%%%%%%%%%%%%%%%%%%%%%%%%%%%%%%%%%%%%%%%%%%%%%%%%%%
%%%%%%%%%%%%%%%%%%%%%%%%%%%%%%%%%%%%%%%%%%%%%%%%%%%%%%%%%%%%%%%%%%%%%%%%%%%%%%%%
\section{Information}

%%%%%%%%%%%%%%%%%%%%%%%%%%%%%%%%%%%%%%%%%%%%%%%%%%%%%%%%%%%%%%%%%%%%%%%%%%%%%%%%
\subsection{Copyright}

Copyright \copyright{} 2017--2018 Niklas Beisert

This work may be distributed and/or modified under the
conditions of the \LaTeX{} Project Public License, either version 1.3
of this license or (at your option) any later version.
The latest version of this license is in
  \url{http://www.latex-project.org/lppl.txt}
and version 1.3 or later is part of all distributions of \LaTeX{}
version 2005/12/01 or later.

This work has the LPPL maintenance status `maintained'.

The Current Maintainer of this work is Niklas Beisert.

This work consists of the files |README.txt|, |childdoc.ins| and |childdoc.dtx|
as well as the derived files |childdoc.def|, |cdocsamp.tex|
with |cdocsch1.tex|, |cdocsch2.tex|, |cdocspt3.tex|, |cdocspt4.tex|,
|cdocsdrf.tex|, |cdocsfn1.tex|, |cdocsfn2.tex|
as well as |childdoc.pdf|.

%%%%%%%%%%%%%%%%%%%%%%%%%%%%%%%%%%%%%%%%%%%%%%%%%%%%%%%%%%%%%%%%%%%%%%%%%%%%%%%%
\subsection{Files and Installation}

The package consists of the files:
%
\begin{center}
\begin{tabular}{ll}
    |README.txt|   & readme file \\
    |childdoc.ins| & installation file \\
    |childdoc.dtx| & source file \\
    |childdoc.def| & definition file \\
    |cdocsamp.tex| & sample main file \\
    |cdocsch1.tex| & sample include file \\
    |cdocsch2.tex| & sample include file \\
    |cdocspt3.tex| & sample part file \\
    |cdocspt4.tex| & sample part file \\
    |cdocsdrf.tex| & sample redirection file \\
    |cdocsfn1.tex| & sample redirection file \\
    |cdocsfn2.tex| & sample redirection file \\
    |childdoc.pdf| & manual
\end{tabular}
\end{center}
%
The distribution consists of the files
|README.txt|, |childdoc.ins| and |childdoc.dtx|.
%
\begin{itemize}
\item
Run (pdf)\LaTeX{} on |childdoc.dtx|
to compile the manual |childdoc.pdf| (this file).
\item
Run \LaTeX{} on |childdoc.ins| to create the definitions file |childdoc.def|
and the sample |cdocsamp.tex| with include files
|cdocsch1.tex|, |cdocsch2.tex|, |cdocspt3.tex|, |cdocspt4.tex|,
|cdocsdrf.tex|, |cdocsfn1.tex|, |cdocsfn2.tex|.
Then copy the file |childdoc.def| to an appropriate directory of your \LaTeX{}
distribution, e.g.\ \textit{texmf-root}|/tex/latex/childdoc|.
\end{itemize}

%%%%%%%%%%%%%%%%%%%%%%%%%%%%%%%%%%%%%%%%%%%%%%%%%%%%%%%%%%%%%%%%%%%%%%%%%%%%%%%%
\subsection{Related CTAN Packages}

There are several other packages which offer a similar functionality:
%
\begin{itemize}
\item
The packages
\href{http://ctan.org/pkg/docmute}{\textsf{docmute}},
\href{http://ctan.org/pkg/includex}{\textsf{includex}} and
\href{http://ctan.org/pkg/standalone}{\textsf{standalone}}
provide commands to include only the document body of
a child file thus allowing both files to be compiled individually.
\item
The packages \href{http://ctan.org/pkg/subdocs}{\textsf{subdocs}}
and \href{http://ctan.org/pkg/subfiles}{\textsf{subfiles}}
provide structures in which the main and child documents can be
encapsulated and allowing them to be compiled individually.
The inclusion mechanism is different from the conventional |\include|.
\item
The package \href{http://ctan.org/pkg/combine}{\textsf{combine}}
is an elaborate solution to combine several documents into one.
\end{itemize}
%
See also the CTAN topic \href{http://ctan.org/topic/subdocs}{\textsf{subdocs}}
for further related packages.
The present package differs from the above solutions in that
a document structure constructed with the conventional |\include| mechanism
just needs two extra commands at the top of every file
such that all constituent files can be compiled individually.

%%%%%%%%%%%%%%%%%%%%%%%%%%%%%%%%%%%%%%%%%%%%%%%%%%%%%%%%%%%%%%%%%%%%%%%%%%%%%%%%
%\subsection{Feature Suggestions}
%
%The following is a list of features which may be useful for future
%versions of this package:
%%
%\begin{itemize}
%\item
%\ldots
%\end{itemize}

%%%%%%%%%%%%%%%%%%%%%%%%%%%%%%%%%%%%%%%%%%%%%%%%%%%%%%%%%%%%%%%%%%%%%%%%%%%%%%%%
\subsection{Revision History}

%%%%%%%%%%%%%%%%%%%%%%%%%%%%%%%%%%%%%%%%
\paragraph{v2.0:} 2018/12/30

\begin{itemize}
\item
immediate forward processing
\item
added |\childdocby| mechanism
\item
manual restructured
\end{itemize}

%%%%%%%%%%%%%%%%%%%%%%%%%%%%%%%%%%%%%%%%
\paragraph{v1.6:} 2018/01/17

\begin{itemize}
\item
application for development of include files
\item
corrections to manual
\end{itemize}

%%%%%%%%%%%%%%%%%%%%%%%%%%%%%%%%%%%%%%%%
\paragraph{v1.5:} 2017/05/21

\begin{itemize}
\item
more complete structuring introduced
\item
|\childdocof| introduced
\item
|\childdoc| renamed to |\childdocmain|
\item
|\childredirect| renamed to |\childdocforward| and |\childdocforwardprefix|
and functionality expanded
\end{itemize}

%%%%%%%%%%%%%%%%%%%%%%%%%%%%%%%%%%%%%%%%
\paragraph{v1.0:} 2017/04/27

\begin{itemize}
\item
manual and install package
\item
first version published on CTAN
\end{itemize}

%%%%%%%%%%%%%%%%%%%%%%%%%%%%%%%%%%%%%%%%
\paragraph{v0.6:} 2017/04/26

\begin{itemize}
\item
redirection mechanism added
\end{itemize}

%%%%%%%%%%%%%%%%%%%%%%%%%%%%%%%%%%%%%%%%
\paragraph{v0.5:} 2017/04/26

\begin{itemize}
\item
functionality in definition file
\end{itemize}


%%%%%%%%%%%%%%%%%%%%%%%%%%%%%%%%%%%%%%%%%%%%%%%%%%%%%%%%%%%%%%%%%%%%%%%%%%%%%%%%
%%%%%%%%%%%%%%%%%%%%%%%%%%%%%%%%%%%%%%%%%%%%%%%%%%%%%%%%%%%%%%%%%%%%%%%%%%%%%%%%
%%%%%%%%%%%%%%%%%%%%%%%%%%%%%%%%%%%%%%%%%%%%%%%%%%%%%%%%%%%%%%%%%%%%%%%%%%%%%%%%
\appendix

\settowidth\MacroIndent{\rmfamily\scriptsize 000\ }

 \DocInput{childdoc.dtx}

\end{document}
%</driver>
% \fi
%
% %%%%%%%%%%%%%%%%%%%%%%%%%%%%%%%%%%%%%%%%%%%%%%%%%%%%%%%%%%%%%%%%%%%%%%%%%%%%%%
% %%%%%%%%%%%%%%%%%%%%%%%%%%%%%%%%%%%%%%%%%%%%%%%%%%%%%%%%%%%%%%%%%%%%%%%%%%%%%%
% \section{Sample}
%\iffalse
%<*samplemain>
%\fi
%
% The following presents a sample document
% with two chapters, two parts, a title page,
% a compile flag as well as three forwarding files to set the flag.
% It consists of eight |.tex| files:
% \begin{center}
% \begin{tabular}{ll}
% |cdocsamp.tex|&main file\\
% |cdocsch1.tex|&include file for chapter 1\\
% |cdocsch2.tex|&include file for chapter 2\\
% |cdocspt3.tex|&include file for part 3\\
% |cdocspt4.tex|&include file for part 4\\
% |cdocsdrf.tex|&forwarding file for main file in draft mode\\
% |cdocsfi1.tex|&forwarding file for final version of chapter 1\\
% |cdocsfi2.tex|&forwarding file for final version of chapter 2\\
% \end{tabular}
% \end{center}
% Each of the eight files can be compiled directly by the \LaTeX{} compiler.
%
% %%%%%%%%%%%%%%%%%%%%%%%%%%%%%%%%%%%%%%
% \paragraph{Main File.}
%
% The main file is called |cdocsamp.tex|.
%
% Load the \textsf{childdoc} definitions and
% declare the filename for the main document:
%    \begin{macrocode}
\input{childdoc.def}
\childdocmain{}
%    \end{macrocode}

% Optional override for |\version| flag:
%    \begin{macrocode}
%%\ifchilddoc\else\providecommand{\version}{draft}\fi
%    \end{macrocode}

% Define the default values for the |\version| flag
% (|final| for the main file and |draft| for childs):
%    \begin{macrocode}
\ifchilddoc
\providecommand{\version}{draft}
\else
\providecommand{\version}{final}
\fi
%    \end{macrocode}

% Load the standard document class:
%    \begin{macrocode}
\documentclass[12pt]{article}
%    \end{macrocode}

% Start the document body:
%    \begin{macrocode}
\begin{document}
%    \end{macrocode}

% Declare a title page.
% Print title, part of document being processed and version flag:
%    \begin{macrocode}
\addtocounter{page}{-1}
\begin{center}
{\LARGE\bfseries{}childdoc example\par}
\vspace{1cm}
\ifchilddoc
\ifchilddocmanual part\else chapter\fi:
`\childdocname' of `\childdocjob'\par
\else
main document: `\childdocjob'\par
\fi
version: \version\par
\end{center}
\newpage
%    \end{macrocode}

% Manually include selected file,
% otherwise process as usual:
%    \begin{macrocode}
\ifchilddocmanual
\section*{part `\childdocname'}
\input{\childdocname}
\else
%    \end{macrocode}

% Include the two chapters:
%    \begin{macrocode}
\include{cdocsch1}
\include{cdocsch2}
%    \end{macrocode}

% Include the two parts unless only chapters should be displayed:
%    \begin{macrocode}
\ifchilddoc\else
\section{part three}
\input{cdocspt3}
\section{part four}
\input{cdocspt4}
\fi
%    \end{macrocode}

% Process as usual until here:
%    \begin{macrocode}
\fi
%    \end{macrocode}

% End of document body:
%    \begin{macrocode}
\end{document}
%    \end{macrocode}
%\iffalse
%</samplemain>
%\fi
%
% %%%%%%%%%%%%%%%%%%%%%%%%%%%%%%%%%%%%%%
% \paragraph{Chapter Include Files.}
%
% The include files are called |cdocsch1.tex| and |cdocsch2.tex|.
%
%\iffalse
%<*samplechap1|samplechap2>
%\fi

% Optional override for |\version| flag:
%    \begin{macrocode}
%%\providecommand{\version}{final}
%    \end{macrocode}

% Include the main document:
%    \begin{macrocode}
\input{childdoc.def}
\childdocof{cdocsamp}
%    \end{macrocode}

%\iffalse
%</samplechap1|samplechap2>
%\fi
%
%\iffalse
%<*samplechap1>
%\fi
% Some text for chapter 1:
%    \begin{macrocode}
\section{one}
some text in chapter one
%    \end{macrocode}

%\iffalse
%</samplechap1>
%\fi
% Some text for chapter 2:
%\iffalse
%<*samplechap2>
%\fi
%    \begin{macrocode}
\section{two}
more text in chapter two
%    \end{macrocode}

%\iffalse
%</samplechap2>
%\fi
%
% %%%%%%%%%%%%%%%%%%%%%%%%%%%%%%%%%%%%%%
% \paragraph{Part Include Files.}
%
% The include files are called |cdocspt3.tex| and |cdocspt4.tex|.
%
%\iffalse
%<*samplepart3|samplepart4>
%\fi

% Optional override for |\version| flag:
%    \begin{macrocode}
%%\providecommand{\version}{final}
%    \end{macrocode}

% Include the main document:
%    \begin{macrocode}
\input{childdoc.def}
\childdocby{cdocsamp}
%    \end{macrocode}

%\iffalse
%</samplepart3|samplepart4>
%\fi
%
%\iffalse
%<*samplepart3>
%\fi
% Some text for part 3:
%    \begin{macrocode}
some text in part three
%    \end{macrocode}

%\iffalse
%</samplepart3>
%\fi
% Some text for part 4:
%\iffalse
%<*samplepart4>
%\fi
%    \begin{macrocode}
more text in part four
%    \end{macrocode}

%\iffalse
%</samplepart4>
%\fi
%
% %%%%%%%%%%%%%%%%%%%%%%%%%%%%%%%%%%%%%%
% \paragraph{Forwarding for a Complete Draft.}
%
% The following forwarding file |cdocsdrf.tex|
% compiles the main document in draft mode:
%\iffalse
%<*sampledraft>
%\fi
%    \begin{macrocode}
\def\version{draft}
\input{childdoc.def}
\childdocforward{cdocsamp}
%    \end{macrocode}

%\iffalse
%</sampledraft>
%\fi
%
% %%%%%%%%%%%%%%%%%%%%%%%%%%%%%%%%%%%%%%
% \paragraph{Forwarding for Final Version of the Chapters.}
%
% The following forwarding files |cdocsfn1.tex| and |cdocsfn2.tex|
% (with identical content)
% compile the final versions of the child documents
% |cdocsch1.tex| and |cdocsch2.tex|, respectively:
%\iffalse
%<*samplefinal>
%\fi
%    \begin{macrocode}
\def\version{final}
\input{childdoc.def}
\childdocforwardprefix[cdocsamp]{cdocsfn}{cdocsch}
%    \end{macrocode}

%\iffalse
%</samplefinal>
%\fi
%
% %%%%%%%%%%%%%%%%%%%%%%%%%%%%%%%%%%%%%%
% \paragraph{Command Line Processing.}
%
% The following three command lines generate the output files
% |cdocscld|, |cdocscl1| and |cdocscl2|
% which should be identical to
% |cdocsdrf|, |cdocsch1| and |cdocsfn2|, respectively:
% \begin{center}
% \begin{tabular}{l}
% |latex -jobname cdocscld \|\\
% |  "\def\version{draft}\input{childdoc.def}\childdocforward{cdocsamp}"|\\
% |latex -jobname cdocscl1 \|\\
% |  "\input{childdoc.def}\childdocforward[cdocsamp]{cdocsch1}"|\\
% |latex -jobname cdocscl2 \|\\
% |  "\def\version{final}\input{childdoc.def}\childdocforward{cdocsch2}"|
% \end{tabular}
% \end{center}
% Note that the trailing backslash on each first line
% merely continues the input to the second line
% (for convenient cut ant paste).
% Furthermore, the command |latex| can be replaced by any
% of its alternative versions such as |pdflatex|.
%
% %%%%%%%%%%%%%%%%%%%%%%%%%%%%%%%%%%%%%%%%%%%%%%%%%%%%%%%%%%%%%%%%%%%%%%%%%%%%%%
% %%%%%%%%%%%%%%%%%%%%%%%%%%%%%%%%%%%%%%%%%%%%%%%%%%%%%%%%%%%%%%%%%%%%%%%%%%%%%%
% \section{Implementation}
%\iffalse
%<*package>
%\fi
%
% This section describes the definitions file |childdoc.def|.

% The definitions cannot be loaded using |\usepackage| or |\RequirePackage|
% which has a mechanism to prevent loading a style file more than once.
% When loading the definitions by means of |\input|
% multiple instances have to be prevented manually:
%\iffalse
%This code needs to be before the `\ProvidesFile' directive
%which is defined at the beginning of this file.
%Therefore it is also placed there and commented out here.
%</package>
%<*discard>
%\fi
%    \begin{macrocode}
\ifdefined\childdocmain\endinput\fi
%    \end{macrocode}
%\iffalse
%</discard>
%<*package>
%\fi
%
% \macro{\ifchilddoc}
% \macro{\ifchilddocmanual}
% The conditional |\ifchilddoc| tells whether a
% child (true) or main (false) document is being compiled.
% The conditional |\ifchilddocmanual| tells whether
% the |\includeonly| mechanism is used (false) or
% the selection of child files must be performed manually (true).
% The definitions initialise to false:
%    \begin{macrocode}
\newif\ifchilddoc
\newif\ifchilddocmanual
%    \end{macrocode}

% \macro{\childdocname}
% \macro{\childdocjob}
% The macro |\childdocname| stores the name of the main document
% to be compiled. The macro |\childdocjob| stores the name of
% the document on which the \LaTeX{} compiler was originally invoked.
% The content of |\jobname| cannot be compared
% to filenames specified in the source due to different catcodes.
% The following code rescans |\jobname|, stores the result
% in |\childdocname| and saves a copy in |\childdocjob|:
%    \begin{macrocode}
\edef\childdocname{\scantokens\expandafter{\jobname\noexpand}}
\let\childdocjob\childdocname
%    \end{macrocode}

% \macro{\childdocdisable}
% The macro |\childdocdisable| prevents the main file
% from being processed more than once.
% At this stage, the main document command |\childdocmain|
% is assumed to be called once again where it should do nothing.
% Any subsequent call to it should prevent
% a secondary processing of the main document
% It overwrites the forwarding commands
% |\childdocof| and |\childdocforward|
% with empty macros to prevent further inclusions of the main document:
%    \begin{macrocode}
\newcommand{\childdocdisable}
{
  \renewcommand{\childdocmain}[1]{\renewcommand{\childdocmain}[1]{\endinput}}
  \renewcommand{\childdocof}[1]{}
  \renewcommand{\childdocby}[2][]{}
  \renewcommand{\childdocforward}[2][]{}
  \renewcommand{\childdocdisable}{}
}
%    \end{macrocode}

% \macro{\childdocmain}
% The macro |\childdocmain| is to be called at the top of the main file
% with nothing or the main filename (without extension) as argument.
% First, it breaks loops.
% If the argument is not empty and does not match |\childdocname|
% (which is set by the first inclusion of |childdoc.def|),
% |\ifchilddoc| is set to true, |\includeonly| is applied to the child file
% and |\jobname| is set to the main file
% (for proper handling of |.aux| files):
%    \begin{macrocode}
\newcommand{\childdocmain}[1]
{
  \childdocdisable\childdocmain{}
  \if?#1?\else
    \begingroup
      \def\childdoctmp{#1}
      \ifx\childdoctmp\childdocname
        \def\childdoctmp{}
      \else
        \def\childdoctmp
        {
          \childdoctrue
          \includeonly{\childdocname}
          \def\childdocjob{#1}
          \def\jobname{#1}
        }
      \fi
      \expandafter
    \endgroup
    \childdoctmp
  \fi
}
%    \end{macrocode}

% \macro{\childdocof}
% The command |\childdocof| redirects
% compilation to the main file |#1|.
%    \begin{macrocode}
\newcommand{\childdocof}[1]
{
  \childdocdisable
  \childdoctrue
  \includeonly{\childdocname}
  \def\jobname{#1}
  \def\childdocjob{#1}
  \input{#1}
}
%    \end{macrocode}

% \macro{\childdocby}
% The command |\childdocby| ....
%    \begin{macrocode}
\newcommand{\childdocby}[2][]
{
  \childdocdisable
  \childdoctrue
  \childdocmanualtrue
  \if?#1?\else
    \def\jobname{#2}
  \fi
  \def\childdocjob{#2}
  \input{#2}
  \endinput
}
%    \end{macrocode}

% \macro{\childdocforward}
% The command |\childdocforward| redirects
% compilation to the main file or
% (if the optional argument is given) a child file.
% Parameters are set as if the main file
% or a child file starting with |\childdocof| was compiled.
% Then compilation is handed over to the main file:
%    \begin{macrocode}
\newcommand{\childdocforward}[2][]
{
  \begingroup
    \if?#1?
      \def\childdoctmp
      {
        \def\childdocname{#2}
        \def\childdocjob{#2}
        \def\jobname{#2}
        \input{#2}
        \endinput
      }
    \else
      \def\childdoctmp
      {
        \childdocdisable
        \def\childdocname{#2}
        \childdoctrue
        \includeonly{#2}
        \def\childdocjob{#1}
        \def\jobname{#1}
        \input{#1}
        \endinput
      }
    \fi
    \expandafter
  \endgroup
  \childdoctmp
}
%    \end{macrocode}

% \macro{\childdocforwardprefix}
% The command |\childdocforwardprefix| redirects
% compilation to the main or a child file by means of a pattern.
% The prefix |#1| in the current filename is replaced by |#2|
% and the suffix of the current filename is kept
% (it is assumed that the filename does not contain the substring `|~~~|'
% which is used as a delimiter).
% Compilation is handed over to the new file by |\childdocforward|:
%    \begin{macrocode}
\newcommand{\childdocforwardprefix}[3][]
{
  \begingroup
    \def\childdocextract #2##1~~~{\def\childdoctmp{\childdocforward[#1]{#3##1}}}
    \expandafter\childdocextract\childdocname~~~
    \expandafter
  \endgroup
  \childdoctmp
}
%    \end{macrocode}

% \macro{\childdoc}
% The deprecated macro |\childdoc| is a legacy version of |\childdocmain|:
%    \begin{macrocode}
\newcommand{\childdoc}{\childdocmain}
%    \end{macrocode}

% \macro{\childdocredirect}
% The deprecated macro |\childdocredirect| is a legacy version
% of |\childdocforward| and |\childdocforwardprefix|:
%    \begin{macrocode}
\newcommand{\childdocredirect}[2][]
{
  \begingroup
    \if?#1?
      \def\childdoctmp{\childdocforward{#2}}
    \else
      \def\childdoctmp{\childdocforwardprefix{#1}{#2}}
    \fi
    \expandafter
  \endgroup
  \childdoctmp
}
%    \end{macrocode}

%\iffalse
%</package>
%\fi
%
\endinput

\childdocmain{}
%    \end{macrocode}

% Optional override for |\version| flag:
%    \begin{macrocode}
%%\ifchilddoc\else\providecommand{\version}{draft}\fi
%    \end{macrocode}

% Define the default values for the |\version| flag
% (|final| for the main file and |draft| for childs):
%    \begin{macrocode}
\ifchilddoc
\providecommand{\version}{draft}
\else
\providecommand{\version}{final}
\fi
%    \end{macrocode}

% Load the standard document class:
%    \begin{macrocode}
\documentclass[12pt]{article}
%    \end{macrocode}

% Start the document body:
%    \begin{macrocode}
\begin{document}
%    \end{macrocode}

% Declare a title page.
% Print title, part of document being processed and version flag:
%    \begin{macrocode}
\addtocounter{page}{-1}
\begin{center}
{\LARGE\bfseries{}childdoc example\par}
\vspace{1cm}
\ifchilddoc
\ifchilddocmanual part\else chapter\fi:
`\childdocname' of `\childdocjob'\par
\else
main document: `\childdocjob'\par
\fi
version: \version\par
\end{center}
\newpage
%    \end{macrocode}

% Manually include selected file,
% otherwise process as usual:
%    \begin{macrocode}
\ifchilddocmanual
\section*{part `\childdocname'}
\input{\childdocname}
\else
%    \end{macrocode}

% Include the two chapters:
%    \begin{macrocode}
\include{cdocsch1}
\include{cdocsch2}
%    \end{macrocode}

% Include the two parts unless only chapters should be displayed:
%    \begin{macrocode}
\ifchilddoc\else
\section{part three}
\input{cdocspt3}
\section{part four}
\input{cdocspt4}
\fi
%    \end{macrocode}

% Process as usual until here:
%    \begin{macrocode}
\fi
%    \end{macrocode}

% End of document body:
%    \begin{macrocode}
\end{document}
%    \end{macrocode}
%\iffalse
%</samplemain>
%\fi
%
% %%%%%%%%%%%%%%%%%%%%%%%%%%%%%%%%%%%%%%
% \paragraph{Chapter Include Files.}
%
% The include files are called |cdocsch1.tex| and |cdocsch2.tex|.
%
%\iffalse
%<*samplechap1|samplechap2>
%\fi

% Optional override for |\version| flag:
%    \begin{macrocode}
%%\providecommand{\version}{final}
%    \end{macrocode}

% Include the main document:
%    \begin{macrocode}
% \iffalse
%
% childdoc.dtx Copyright (C) 2017-2018 Niklas Beisert
%
% This work may be distributed and/or modified under the
% conditions of the LaTeX Project Public License, either version 1.3
% of this license or (at your option) any later version.
% The latest version of this license is in
%   http://www.latex-project.org/lppl.txt
% and version 1.3 or later is part of all distributions of LaTeX
% version 2005/12/01 or later.
%
% This work has the LPPL maintenance status `maintained'.
%
% The Current Maintainer of this work is Niklas Beisert.
%
% This work consists of the files childdoc.dtx and childdoc.ins
% and the derived files childdoc.def and cdocsamp.tex with
% cdocsch1.tex, cdocsch2.tex, cdocsdrf.tex, cdocsfn1.tex, cdocsfn2.tex.
%
%<package>\ifdefined\childdocmain\endinput\fi
%<package>\ProvidesFile{childdoc.def}[2018/12/30 v2.0 child document driver]
%<samplemain>\ProvidesFile{cdocsamp.tex}[2018/12/30 v2.0 sample for childdoc]
%<*driver>
%\ProvidesFile{childdoc.drv}[2018/12/30 v2.0 childdoc reference manual file]
\PassOptionsToClass{10pt,a4paper}{article}
\documentclass{ltxdoc}

\usepackage[margin=35mm]{geometry}
\usepackage{hyperref}
\usepackage{hyperxmp}
\usepackage[usenames]{color}

\hypersetup{colorlinks=true}
\hypersetup{pdfstartview=FitH}
\hypersetup{pdfpagemode=UseNone}
\hypersetup{pdfsource={}}
\hypersetup{pdflang={en-UK}}
\hypersetup{pdfcopyright={Copyright 2017-2018 Niklas Beisert.
  This work may be distributed and/or modified under the
  conditions of the LaTeX Project Public License, either version 1.3
  of this license or (at your option) any later version.}}
\hypersetup{pdflicenseurl={http://www.latex-project.org/lppl.txt}}
\hypersetup{pdfcontactaddress={ETH Zurich, ITP, HIT K,
  Wolfgang-Pauli-Strasse 27}}
\hypersetup{pdfcontactpostcode={8093}}
\hypersetup{pdfcontactcity={Zurich}}
\hypersetup{pdfcontactcountry={Switzerland}}
\hypersetup{pdfcontactemail={nbeisert@itp.phys.ethz.ch}}
\hypersetup{pdfcontacturl={http://people.phys.ethz.ch/\xmptilde nbeisert/}}

\newcommand{\secref}[1]{\hyperref[#1]{section \ref*{#1}}}

\parskip1ex
\parindent0pt
\let\olditemize\itemize
\def\itemize{\olditemize\parskip0pt}

\begin{document}

\title{The \textsf{childdoc} Package}
\hypersetup{pdftitle={The childdoc Package}}
\author{Niklas Beisert\\[2ex]
  Institut f\"ur Theoretische Physik\\
  Eidgen\"ossische Technische Hochschule Z\"urich\\
  Wolfgang-Pauli-Strasse 27, 8093 Z\"urich, Switzerland\\[1ex]
  \href{mailto:nbeisert@itp.phys.ethz.ch}
  {\texttt{nbeisert@itp.phys.ethz.ch}}}
\hypersetup{pdfauthor={Niklas Beisert}}
\hypersetup{pdfsubject={Manual for the LaTeX2e Package childdoc}}
\date{30 December 2018, \textsf{v2.0}}
\maketitle

\begin{abstract}\noindent
\textsf{childdoc} is a \LaTeXe{} package
that enables the direct compilation
of document sections included by |\include|
to individual files.
\end{abstract}

\begingroup
\parskip0ex
\tableofcontents
\endgroup

%%%%%%%%%%%%%%%%%%%%%%%%%%%%%%%%%%%%%%%%%%%%%%%%%%%%%%%%%%%%%%%%%%%%%%%%%%%%%%%%
%%%%%%%%%%%%%%%%%%%%%%%%%%%%%%%%%%%%%%%%%%%%%%%%%%%%%%%%%%%%%%%%%%%%%%%%%%%%%%%%
\section{Introduction}

\LaTeX{} provides a mechanism to structure a large document (such as a book)
into a main file and several child files (containing the chapters)
using the |\include| command.
This mechanism is beneficial for documents
which span hundreds of pages in order to
make the source file(s) more manageable.
Moreover, compilation can be restricted to
selected child files by means of the |\includeonly| command.
The latter feature can be used to reduce the compilation time while editing
(this was significantly more useful in the earlier days of \LaTeX{})
or to generate a smaller document which is easier to navigate.
Another application of |\includeonly| is to generate
documents consisting of selected parts of the complete document.

However, there are a few drawbacks of the plain |\include| mechanism:
\begin{itemize}
\item
The child files cannot be compiled on their own,
they can only be compiled via the main file.
A naive editing environment
(such as a text editor with an option
to have the current file processed by \LaTeX)
may require one to switch to the main file before compiling;
attempting to compile the child file produces errors.
\item
The main file must be modified (each time)
to adjust the |\includeonly| command
to the present needs. This easily leaves the main file in a messy state.
\item
The generated document will always carry the filename
of the main document. This is inconvenient if
several child files are to be compiled and
to be kept for distribution.
\end{itemize}

The present package provides a simple interface
to make child files individually compilable by \LaTeX{}.
Compiling a child file then has the same effect as compiling
the main file with an |\includeonly| command
to select the appropriate child.
Moreover the generated document will carry the name of the child
rather than the main file.
This resolves all three above issues.

This feature is meant to make the editing of books,
thesis documents and lecture notes somewhat more convenient.
However, the package can also be used efficiently for
composing a series of documents (such as exercise sheets)
which are typically distributed individually.
It then assists the author in generating the individual documents
(potentially in different versions)
as well as a document containing the collected series.
Another application is in developing style files
or other kinds of included material
where compilation of the style file could redirect
to a sample or test file.

%%%%%%%%%%%%%%%%%%%%%%%%%%%%%%%%%%%%%%%%%%%%%%%%%%%%%%%%%%%%%%%%%%%%%%%%%%%%%%%%
%%%%%%%%%%%%%%%%%%%%%%%%%%%%%%%%%%%%%%%%%%%%%%%%%%%%%%%%%%%%%%%%%%%%%%%%%%%%%%%%
\section{Usage}

First of all, the package \textsf{childdoc} is \emph{not} a standard
\LaTeXe{} |.sty| style file! Therefore it needs to be invoked in
a non-standard way.

%%%%%%%%%%%%%%%%%%%%%%%%%%%%%%%%%%%%%%%%%%%%%%%%%%%%%%%%%%%%%%%%%%%%%%%%%%%%%%%%
\subsection{Included Files}
\label{sec:include}

%%%%%%%%%%%%%%%%%%%%%%%%%%%%%%%%%%%%%%%%
\DescribeMacro{\childdocmain}
To use the package, add the commands
\begin{center}
\begin{tabular}{l}
|\input{childdoc.def}|\\
|\childdocmain{}|\\
\end{tabular}
\end{center}
at the very top of the main \LaTeX{} file,
in particular \emph{before} the |\documentclass| statement!
The argument of |\childdocmain| should be left empty
(but it must be present).

%%%%%%%%%%%%%%%%%%%%%%%%%%%%%%%%%%%%%%%%
\DescribeMacro{\childdocof}
Furthermore, add the commands
\begin{center}
\begin{tabular}{l}
|\input{childdoc.def}|\\
|\childdocof{|\textit{main}|}|\\
\end{tabular}
\end{center}
at the top of every child file \textit{child}
which is included by |\include{|\textit{child}|}|
from within the main file
(or at least for those files to be compiled individually).
The argument \textit{main} must be the filename of the main file.

There are a couple of
considerations in setting up the main and child documents:

%%%%%%%%%%%%%%%%%%%%%%%%%%%%%%%%%%%%%%%%
\paragraph{Restrictions.}

Please note the following restrictions:
\begin{itemize}
\item
|\childdocmain| must be called with one argument \textit{main}
to ensure compatibility with earlier version of the package.
It must either be empty (|\childdocmain{}|)
or precisely match the filename of the main file in which it is specified.
See \secref{sec:detection} for further information.
\item
The filename \textit{main} must be specified without the |.tex| extension.
\item
The filename \textit{main} is case sensitive
(even in case-insensitive file systems)
due to internal string comparison.
\item
The argument \textit{main} should be fully expanded, it cannot be a macro.
\item
Subdirectories and special characters should be avoided in filenames.
\item
The command |\childdocmain{|\textit{main}|}| must be followed by a whitespace.
It should not be followed immediately by another command
or by a comment mark `|%|'.
This is because the \TeX{} parser reads the token immediately following
the argument of |\childdocmain| and puts it
at the beginning of every child section;
however, a white\-space is ignored.
\end{itemize}

%%%%%%%%%%%%%%%%%%%%%%%%%%%%%%%%%%%%%%%%
\paragraph{Content of Main File.}

It is advisable to place all content in the child files included by |\include|.
Any output contained in the main file will appear in all child documents
unless suppressed manually;
it cannot be suppressed automatically by the |\includeonly| directive
and thus should normally be avoided.
A method to include some content in the main file
by means of conditional processing is described in \secref{sec:conditional}.

%%%%%%%%%%%%%%%%%%%%%%%%%%%%%%%%%%%%%%%%
\paragraph{Page Numbering.}

When only a part of the document is compiled,
the appropriate numbering of pages
(as well as other status parameters)
is determined from the |.aux| files.
The latter contain information from previous passes.
However this information needs to propagate through
all intermediate child documents.
Therefore the page numbering in child documents may well
be inconsistent until the complete document is compiled at least once.

A useful (if unconventional) way to always ensure a consistent
page numbering is to restart the numbering in each child document
and denote the pages by `\textit{child}|.|\textit{page}'
where \textit{child} represents the chapter/section number of the child file.
This can be achieved by the command
|\numberwithin{page}{|\textit{child}|}|
of the \textsf{amsmath} package
where \textit{child} can be |chapter| or |section|
depending on the chosen structuring.
Alternatively, one can modify the macro |\thepage| appropriately
and reset the counter |page| at the start of each child file.

%%%%%%%%%%%%%%%%%%%%%%%%%%%%%%%%%%%%%%%%%%%%%%%%%%%%%%%%%%%%%%%%%%%%%%%%%%%%%%%%
\subsection{Conditional Processing}
\label{sec:conditional}

The package provides a mechanism to compile different versions
of a document. To customise the versions further some conditional processing
can come in handy to distinguish which version is being compiled.
The package provides two macros to describe the compilation context:

%%%%%%%%%%%%%%%%%%%%%%%%%%%%%%%%%%%%%%%%
\DescribeMacro{\ifchilddoc}
The conditional |\ifchilddoc| distinguishes between the compilation of
child documents and the main document:
%
\begin{center}
|\ifchilddoc |\textit{child-code}| |[|\||else |\textit{main-code}]| \||fi|
\end{center}

%%%%%%%%%%%%%%%%%%%%%%%%%%%%%%%%%%%%%%%%
\DescribeMacro{\childdocname}
\DescribeMacro{\childdocjob}
The macro |\childdocname| contains the filename (without extension)
of the main or child file being processed.
Note that |\childdocjob| will always contain the name of the main file.

%%%%%%%%%%%%%%%%%%%%%%%%%%%%%%%%%%%%%%%%
\paragraph{Title Page.}

Conditional processing can be used to include a title or banner page
in the main document when proper precautions are taken.
Importantly, the code in the main file should ensure that the page counter
(as well as other status parameters which are stored in the |.aux| files)
takes the same value after the conditional processing.
Otherwise the page numbers may take divergent values
depending on which part is compiled.

For example, a title page could be declared by:
%
\begin{center}
\begin{tabular}{l}
|\ifchilddoc\||else|\\
|\addtocounter{page}{-1}|\\
\textit{code for title page}\\
|\newpage|\\
|\||fi|
\end{tabular}
\end{center}
%
A banner page for the child documents can be generated by:
%
\begin{center}
\begin{tabular}{l}
|\ifchilddoc|\\
|\addtocounter{page}{-1}|\\
\textit{code for banner page}\\
|\newpage|\\
|\||fi|
\end{tabular}
\end{center}
%
Here one could write a message such as:
\begin{center}
|This is the part \childdocname{} of \childdocjob{}.|
\end{center}

%%%%%%%%%%%%%%%%%%%%%%%%%%%%%%%%%%%%%%%%%%%%%%%%%%%%%%%%%%%%%%%%%%%%%%%%%%%%%%%%
\subsection{Flags}
\label{sec:flags}

The package makes it easy to generate different versions
of the main or child documents.
To this end compilation flags can be defined
and assigned different default values.
They will be particularly useful in conjunction
with the forwarding mechanism described in \secref{sec:forward}.

For example, it may be useful to have a flag |\version|
which can be set to |draft| or |final|.
The document source will contain some conditional code
depending on the value of |\version|.
Suppose further, the flag should default to |final| for the main file
and to |draft| for child files
which is a natural assignment for editing the document.
This is achieved by placing the following code
in the preamble of the main document
(below the |\childdocmain| directive):
%
\begin{center}
\begin{tabular}{l}
|\ifchilddoc|\\
|\providecommand{\version}{draft}|\\
|\||else|\\
|\providecommand{\version}{final}|\\
|\||fi|
\end{tabular}
\end{center}
%
The definition by |\providecommand| makes sure
that previous definitions are not overwritten.
Further statements |\providecommand{\version}{...}|
can thus be added before the above code to override it.

For the main file, one might add a line
(between |\childdocmain| and the above block)
%
\begin{center}
|%\ifchilddoc\||else\providecommand{\version}{draft}\||fi|
\end{center}
%
which can be uncommented to produce a draft version.
Likewise one can add a line to the very top of a child file
(above the |\childdocof{|\textit{main}|}| directive)
%
\begin{center}
|%\providecommand{\version}{final}|
\end{center}
%
which can be uncommented to produce the final version of this child document.

%%%%%%%%%%%%%%%%%%%%%%%%%%%%%%%%%%%%%%%%%%%%%%%%%%%%%%%%%%%%%%%%%%%%%%%%%%%%%%%%
\subsection{Forwarding}
\label{sec:forward}

Different versions of the main or child documents
using compilation flags as described in \secref{sec:flags}
can be (permanently) stored in different files
for convenient compilation, viewing and distribution.
To this end, the package defines a command
to pass on compilation to a different file:

%%%%%%%%%%%%%%%%%%%%%%%%%%%%%%%%%%%%%%%%
\DescribeMacro{\childdocforward}
The command |\childdocforward| redirects processing to
another source file:
%
\begin{center}
\begin{tabular}{l}
|\input{childdoc.def}|\\
|\childdocforward[|\textit{main}|]{|\textit{dest}|}|\\
\end{tabular}
\end{center}
%
The argument \textit{dest} is the destination file
(without extension).
It should be the main file or one of the child files.
Note that further \textsf{childdoc} directives
such as |\childdocof| and |\childdocforward|
in the indicated file will be processed in this form.
The optional argument \textit{main}
passes on directly to the main file \textit{main}
while pretending to compile the child \textit{dest}.
This form behaves as if \textit{dest}
issues |\childdocof{|\textit{main}|}| right away,
and no further \textsf{childdoc} directives will be processed.

%%%%%%%%%%%%%%%%%%%%%%%%%%%%%%%%%%%%%%%%
\DescribeMacro{\...prefix}
In the alternative form |\childdocforwardprefix|,
%
\begin{center}
\begin{tabular}{l}
|\input{childdoc.def}|\\
|\childdocforwardprefix[|\textit{main}|]{|\textit{prefix}|}{|\textit{dest}|}|
\end{tabular}
\end{center}
%
the destination file is determined by a pattern
depending on the current file:
To make this work, the current file must be called
`{\textit{prefix}\hspace{0.2em}\textit{suffix}}'
with \textit{prefix} matching precisely the argument.
Processing is then passed on to the file
`{\textit{dest}\hspace{0.2em}\textit{suffix}}'.
Surely, the same effect is achieved by
directly specifying the
argument `{\textit{dest}\hspace{0.2em}\textit{suffix}}'
in the first form.
However, that requires to set up a different file
for each child. With the alternative form of the command
all these files can have exactly the same content
which simplifies setting them up and maintaining them.

For example, the following file |draft.tex|
with a compilation flag |\version| as described in \secref{sec:flags}
compiles the main document as a draft:
%
\begin{center}
\begin{tabular}{l}
|\def\version{draft}|\\
|\input{childdoc.def}|\\
|\childdocforward{|\textit{main}|}|
\end{tabular}
\end{center}
%
Likewise, the following files |final|\textit{nn}|.tex|
compile the final version of the child document
|child|\textit{nn}|.tex|:
%
\begin{center}
\begin{tabular}{l}
|\def\version{final}|\\
|\input{childdoc.def}|\\
|\childdocforwardprefix{final}{child}|
\end{tabular}
\end{center}
%

Note that when several versions of a main file and/or of each child file
are to be generated, it may be convenient to set up a |Makefile| or
shell script to automatise the process.

%%%%%%%%%%%%%%%%%%%%%%%%%%%%%%%%%%%%%%%%%%%%%%%%%%%%%%%%%%%%%%%%%%%%%%%%%%%%%%%%
\subsection{Command Line Processing}
\label{sec:commandline}

The effect of redirection files can also be achieved by invoking
the \LaTeX{} compiler with a more elaborate command line.
Most conveniently this should be done as part
of a shell script or a |Makefile|.

When using \textsf{childdoc} in the main file, the following
command lines effectively perform a redirection
(note that depending on the shell being used,
backslashes may have to be doubled: `|\|' $\to$ `|\\|'):
%
\begin{center}
|... -jobname "|\textit{target}|" |\\|"|[\textit{flags}]%
|\input{childdoc.def}\childdocforward[|\textit{main}|]{|\textit{dest}|}"|
\end{center}
%
Here \textit{target} is the name of the output file,
\textit{main} is the name of the main file
and \textit{dest} is the name of the main or child file to be processed
(all filenames without extensions).
The optional argument \textit{main} can be omitted
if \textit{main} matches \textit{dest}.
Optionally, compilation \textit{flags} can be defined via |\def| commands.
This command line makes the \TeX{} engine believe
it is compiling the file \textit{target}
whose content is specified as the latter parameter.
The provided code then forwards the processing to
\textit{main} or \textit{dest} as described in \secref{sec:forward}.

%%%%%%%%%%%%%%%%%%%%%%%%%%%%%%%%%%%%%%%%%%%%%%%%%%%%%%%%%%%%%%%%%%%%%%%%%%%%%%%%
\subsection{Include by Input}
\label{sec:input}

Including child documents by |\include| has some restrictions by design.
Most notably, the content of a child document always occupies
its own set of pages; pages cannot be shared between child documents.
Usually, this behaviour makes perfect sense
because each child document contain an essential part of the document.
However, in some situations it may be desirable to compose
a document from a collection of parts
without having mandatory page breaks between then.
For this case, the package
provides a mechanism to include parts
by |\input| which can also be processed individually.
However, by construction this mechanism
requires manual handling of the content to be output.

%%%%%%%%%%%%%%%%%%%%%%%%%%%%%%%%%%%%%%%%
\DescribeMacro{\ifchilddocmanual}
The main file should be prepared as usual, see \secref{sec:include}.
However, the document body must make a distinction
between processing of an individual part and of the main document, e.g.:
%
\begin{center}
\begin{tabular}{l}
|\ifchilddocmanual|\\
|\input{\childdocname}|\\
|\||else|\\
\textit{document body with }|\input{|\textit{part}|}|\\
|\||fi|
\end{tabular}
\end{center}
%
The conditional |\ifchilddocmanual| is true whenever
a part to be included by |\input| is being compiled,
and the name of the part is stored in |\childdocname|.

%%%%%%%%%%%%%%%%%%%%%%%%%%%%%%%%%%%%%%%%
\DescribeMacro{\childdocby}
Each part to be included by |\input| should start with:
%
\begin{center}
\begin{tabular}{l}
|\input{childdoc.def}|\\
|\childdocby{|\textit{main}|}|\\
\end{tabular}
\end{center}
%
The directive |\childdocby| is similar to |\childdocof|
described in \secref{sec:include},
but the subsequent selection of content must be done manually.
To that end, both |\ifchilddoc| and |\ifchilddocmanual|
will be true upon processing of a part,
and the name of the part is stored in |\childdocname|.
Note that |\jobname| will be set to the filename of the current part
so that each part receives an individual |.aux| file
that does not interfere with the |.aux| file(s) of the main document.
This behaviour can be altered by the alternative form
|\childdocby[*]{|\textit{main}|}| (with a non-empty optional argument)
which uses the |.aux| file of the main document
by setting |\jobname| to \textit{main}.

%%%%%%%%%%%%%%%%%%%%%%%%%%%%%%%%%%%%%%%%%%%%%%%%%%%%%%%%%%%%%%%%%%%%%%%%%%%%%%%%
\subsection{Driver Development}
\label{sec:driver}

The \textsf{childdoc} mechanism can also be use for the development
of definition files such as \LaTeX{} styles or classes.
This case differs from the above setup with multiple parts
included by |\include| in that no |\includeonly| should be invoked.
This can be achieved by starting the include file
(before |\ProvidesPackage|) with:
%
\begin{center}
\begin{tabular}{l}
|\input{childdoc.def}|\\
|\childdocforward{|\textit{main}|}|\\
\end{tabular}
\end{center}
%
or alternatively with:
%
\begin{center}
\begin{tabular}{l}
|\input{childdoc.def}|\\
|\childdocby{|\textit{main}|}|\\
\end{tabular}
\end{center}
%
Both forms have slightly different effects as described above.
The main file is prepared as usual, see \secref{sec:include}.

%%%%%%%%%%%%%%%%%%%%%%%%%%%%%%%%%%%%%%%%%%%%%%%%%%%%%%%%%%%%%%%%%%%%%%%%%%%%%%%%
\subsection{Legacy Detection}
\label{sec:detection}

The directive |\childdocmain| in the main file can detect
whether the complete document or merely a child is to be compiled
even without using the directive |\childdocof|.
This method is deprecated because it is less robust
and there is no compelling reason to use it;
it is merely provided for backward compatibility
and it may be removed in future versions.

If the detection mechanism is to be used,
it is mandatory to correctly specify
the filename of the main file as the argument of |\childdocmain|:
%
\begin{center}
\begin{tabular}{l}
|\input{childdoc.def}|\\
|\childdocmain{|\textit{main}|}|\\
\end{tabular}
\end{center}
%
If |\jobname| does not match the argument \textit{main} of |\childdocmain|,
it is assumed that |\jobname| points to the child file to be compiled.
When using |\childdocmain| with the main file specified as argument,
it suffices to start a child file
with just |\input{|\textit{main}|}|
without loading of the package and using |\childdocof|.
If instead all processing is done
with the appropriate \textsf{childdoc} directives,
the argument of \textit{main} of |\childdocmain| can be empty.

An alternative version of the command line processing described
in \secref{sec:commandline} using the detection mechanism reads:
%
\begin{center}
|... -jobname "|\textit{target}|" "|[\textit{flags}]%
[|\def\jobname{|\textit{dest}|}|]|\input{|\textit{main}|}"|
\end{center}

%%%%%%%%%%%%%%%%%%%%%%%%%%%%%%%%%%%%%%%%%%%%%%%%%%%%%%%%%%%%%%%%%%%%%%%%%%%%%%%%
\subsection{Manual Code}
\label{sec:manual}

In case one cannot be certain whether the definitions file |childdoc.def|
is installed on the target \TeX{} distribution
and one prefers not to ship it,
it is conceivable to paste a few relevant commands into the sources.

To that end, drop all statements |\input{childdoc.def}|
and perform the replacements as outlined below.
Instead of |\childdocmain{|\textit{main}|}| add the following code
to the top of the main file:
%
\begin{center}
\begin{tabular}{l}
|\||ifdefined\childdocname\endinput\||fi\newif\ifchilddoc|\\
|\edef\childdocname{\scantokens\expandafter{\jobname\noexpand}}|\\
|\def\childdocmain{|\textit{main}|}\||ifx\childdocmain\childdocname\||else|\\
|\childdoctrue\includeonly{\childdocname}\let\jobname\childdocmain\||fi|\\
\end{tabular}
\end{center}
%
Instead of |\childdocof{|\textit{main}|}| just include the main file
at the top of each child file:
%
\begin{center}
|\input{|\textit{main}|}|
\end{center}
%
A simple redirection |\childdocforward{|\textit{dest}|}| is achieved by:
%
\begin{center}
|\def\jobname{|\textit{dest}|}\input{\jobname}|
\end{center}
%
The redirection with prefix
|\childdocforwardprefix[|\textit{prefix}|]{|\textit{dest}|}|
is accomplished by:
%
\begin{center}
\begin{tabular}{l}
|{\edef\jobname{\scantokens\expandafter{\jobname\noexpand}}|\\
|\def\redirectjob |\textit{prefix}|#1~~~{\gdef\jobname{|\textit{dest}|#1}}|\\
|\expandafter\redirectjob\jobname~~~}\input{\jobname}|
\end{tabular}
\end{center}

In an alternative approach,
child documents can be compiled by a specific command line
without additional code or specific definitions:
%
\begin{center}
|... -jobname "|\textit{target}|" "|[\textit{flags}]%
|\includeonly{|\textit{dest}|}\input{|\textit{main}|}"|
\end{center}
%

%%%%%%%%%%%%%%%%%%%%%%%%%%%%%%%%%%%%%%%%%%%%%%%%%%%%%%%%%%%%%%%%%%%%%%%%%%%%%%%%
%%%%%%%%%%%%%%%%%%%%%%%%%%%%%%%%%%%%%%%%%%%%%%%%%%%%%%%%%%%%%%%%%%%%%%%%%%%%%%%%
\section{Information}

%%%%%%%%%%%%%%%%%%%%%%%%%%%%%%%%%%%%%%%%%%%%%%%%%%%%%%%%%%%%%%%%%%%%%%%%%%%%%%%%
\subsection{Copyright}

Copyright \copyright{} 2017--2018 Niklas Beisert

This work may be distributed and/or modified under the
conditions of the \LaTeX{} Project Public License, either version 1.3
of this license or (at your option) any later version.
The latest version of this license is in
  \url{http://www.latex-project.org/lppl.txt}
and version 1.3 or later is part of all distributions of \LaTeX{}
version 2005/12/01 or later.

This work has the LPPL maintenance status `maintained'.

The Current Maintainer of this work is Niklas Beisert.

This work consists of the files |README.txt|, |childdoc.ins| and |childdoc.dtx|
as well as the derived files |childdoc.def|, |cdocsamp.tex|
with |cdocsch1.tex|, |cdocsch2.tex|, |cdocspt3.tex|, |cdocspt4.tex|,
|cdocsdrf.tex|, |cdocsfn1.tex|, |cdocsfn2.tex|
as well as |childdoc.pdf|.

%%%%%%%%%%%%%%%%%%%%%%%%%%%%%%%%%%%%%%%%%%%%%%%%%%%%%%%%%%%%%%%%%%%%%%%%%%%%%%%%
\subsection{Files and Installation}

The package consists of the files:
%
\begin{center}
\begin{tabular}{ll}
    |README.txt|   & readme file \\
    |childdoc.ins| & installation file \\
    |childdoc.dtx| & source file \\
    |childdoc.def| & definition file \\
    |cdocsamp.tex| & sample main file \\
    |cdocsch1.tex| & sample include file \\
    |cdocsch2.tex| & sample include file \\
    |cdocspt3.tex| & sample part file \\
    |cdocspt4.tex| & sample part file \\
    |cdocsdrf.tex| & sample redirection file \\
    |cdocsfn1.tex| & sample redirection file \\
    |cdocsfn2.tex| & sample redirection file \\
    |childdoc.pdf| & manual
\end{tabular}
\end{center}
%
The distribution consists of the files
|README.txt|, |childdoc.ins| and |childdoc.dtx|.
%
\begin{itemize}
\item
Run (pdf)\LaTeX{} on |childdoc.dtx|
to compile the manual |childdoc.pdf| (this file).
\item
Run \LaTeX{} on |childdoc.ins| to create the definitions file |childdoc.def|
and the sample |cdocsamp.tex| with include files
|cdocsch1.tex|, |cdocsch2.tex|, |cdocspt3.tex|, |cdocspt4.tex|,
|cdocsdrf.tex|, |cdocsfn1.tex|, |cdocsfn2.tex|.
Then copy the file |childdoc.def| to an appropriate directory of your \LaTeX{}
distribution, e.g.\ \textit{texmf-root}|/tex/latex/childdoc|.
\end{itemize}

%%%%%%%%%%%%%%%%%%%%%%%%%%%%%%%%%%%%%%%%%%%%%%%%%%%%%%%%%%%%%%%%%%%%%%%%%%%%%%%%
\subsection{Related CTAN Packages}

There are several other packages which offer a similar functionality:
%
\begin{itemize}
\item
The packages
\href{http://ctan.org/pkg/docmute}{\textsf{docmute}},
\href{http://ctan.org/pkg/includex}{\textsf{includex}} and
\href{http://ctan.org/pkg/standalone}{\textsf{standalone}}
provide commands to include only the document body of
a child file thus allowing both files to be compiled individually.
\item
The packages \href{http://ctan.org/pkg/subdocs}{\textsf{subdocs}}
and \href{http://ctan.org/pkg/subfiles}{\textsf{subfiles}}
provide structures in which the main and child documents can be
encapsulated and allowing them to be compiled individually.
The inclusion mechanism is different from the conventional |\include|.
\item
The package \href{http://ctan.org/pkg/combine}{\textsf{combine}}
is an elaborate solution to combine several documents into one.
\end{itemize}
%
See also the CTAN topic \href{http://ctan.org/topic/subdocs}{\textsf{subdocs}}
for further related packages.
The present package differs from the above solutions in that
a document structure constructed with the conventional |\include| mechanism
just needs two extra commands at the top of every file
such that all constituent files can be compiled individually.

%%%%%%%%%%%%%%%%%%%%%%%%%%%%%%%%%%%%%%%%%%%%%%%%%%%%%%%%%%%%%%%%%%%%%%%%%%%%%%%%
%\subsection{Feature Suggestions}
%
%The following is a list of features which may be useful for future
%versions of this package:
%%
%\begin{itemize}
%\item
%\ldots
%\end{itemize}

%%%%%%%%%%%%%%%%%%%%%%%%%%%%%%%%%%%%%%%%%%%%%%%%%%%%%%%%%%%%%%%%%%%%%%%%%%%%%%%%
\subsection{Revision History}

%%%%%%%%%%%%%%%%%%%%%%%%%%%%%%%%%%%%%%%%
\paragraph{v2.0:} 2018/12/30

\begin{itemize}
\item
immediate forward processing
\item
added |\childdocby| mechanism
\item
manual restructured
\end{itemize}

%%%%%%%%%%%%%%%%%%%%%%%%%%%%%%%%%%%%%%%%
\paragraph{v1.6:} 2018/01/17

\begin{itemize}
\item
application for development of include files
\item
corrections to manual
\end{itemize}

%%%%%%%%%%%%%%%%%%%%%%%%%%%%%%%%%%%%%%%%
\paragraph{v1.5:} 2017/05/21

\begin{itemize}
\item
more complete structuring introduced
\item
|\childdocof| introduced
\item
|\childdoc| renamed to |\childdocmain|
\item
|\childredirect| renamed to |\childdocforward| and |\childdocforwardprefix|
and functionality expanded
\end{itemize}

%%%%%%%%%%%%%%%%%%%%%%%%%%%%%%%%%%%%%%%%
\paragraph{v1.0:} 2017/04/27

\begin{itemize}
\item
manual and install package
\item
first version published on CTAN
\end{itemize}

%%%%%%%%%%%%%%%%%%%%%%%%%%%%%%%%%%%%%%%%
\paragraph{v0.6:} 2017/04/26

\begin{itemize}
\item
redirection mechanism added
\end{itemize}

%%%%%%%%%%%%%%%%%%%%%%%%%%%%%%%%%%%%%%%%
\paragraph{v0.5:} 2017/04/26

\begin{itemize}
\item
functionality in definition file
\end{itemize}


%%%%%%%%%%%%%%%%%%%%%%%%%%%%%%%%%%%%%%%%%%%%%%%%%%%%%%%%%%%%%%%%%%%%%%%%%%%%%%%%
%%%%%%%%%%%%%%%%%%%%%%%%%%%%%%%%%%%%%%%%%%%%%%%%%%%%%%%%%%%%%%%%%%%%%%%%%%%%%%%%
%%%%%%%%%%%%%%%%%%%%%%%%%%%%%%%%%%%%%%%%%%%%%%%%%%%%%%%%%%%%%%%%%%%%%%%%%%%%%%%%
\appendix

\settowidth\MacroIndent{\rmfamily\scriptsize 000\ }

 \DocInput{childdoc.dtx}

\end{document}
%</driver>
% \fi
%
% %%%%%%%%%%%%%%%%%%%%%%%%%%%%%%%%%%%%%%%%%%%%%%%%%%%%%%%%%%%%%%%%%%%%%%%%%%%%%%
% %%%%%%%%%%%%%%%%%%%%%%%%%%%%%%%%%%%%%%%%%%%%%%%%%%%%%%%%%%%%%%%%%%%%%%%%%%%%%%
% \section{Sample}
%\iffalse
%<*samplemain>
%\fi
%
% The following presents a sample document
% with two chapters, two parts, a title page,
% a compile flag as well as three forwarding files to set the flag.
% It consists of eight |.tex| files:
% \begin{center}
% \begin{tabular}{ll}
% |cdocsamp.tex|&main file\\
% |cdocsch1.tex|&include file for chapter 1\\
% |cdocsch2.tex|&include file for chapter 2\\
% |cdocspt3.tex|&include file for part 3\\
% |cdocspt4.tex|&include file for part 4\\
% |cdocsdrf.tex|&forwarding file for main file in draft mode\\
% |cdocsfi1.tex|&forwarding file for final version of chapter 1\\
% |cdocsfi2.tex|&forwarding file for final version of chapter 2\\
% \end{tabular}
% \end{center}
% Each of the eight files can be compiled directly by the \LaTeX{} compiler.
%
% %%%%%%%%%%%%%%%%%%%%%%%%%%%%%%%%%%%%%%
% \paragraph{Main File.}
%
% The main file is called |cdocsamp.tex|.
%
% Load the \textsf{childdoc} definitions and
% declare the filename for the main document:
%    \begin{macrocode}
\input{childdoc.def}
\childdocmain{}
%    \end{macrocode}

% Optional override for |\version| flag:
%    \begin{macrocode}
%%\ifchilddoc\else\providecommand{\version}{draft}\fi
%    \end{macrocode}

% Define the default values for the |\version| flag
% (|final| for the main file and |draft| for childs):
%    \begin{macrocode}
\ifchilddoc
\providecommand{\version}{draft}
\else
\providecommand{\version}{final}
\fi
%    \end{macrocode}

% Load the standard document class:
%    \begin{macrocode}
\documentclass[12pt]{article}
%    \end{macrocode}

% Start the document body:
%    \begin{macrocode}
\begin{document}
%    \end{macrocode}

% Declare a title page.
% Print title, part of document being processed and version flag:
%    \begin{macrocode}
\addtocounter{page}{-1}
\begin{center}
{\LARGE\bfseries{}childdoc example\par}
\vspace{1cm}
\ifchilddoc
\ifchilddocmanual part\else chapter\fi:
`\childdocname' of `\childdocjob'\par
\else
main document: `\childdocjob'\par
\fi
version: \version\par
\end{center}
\newpage
%    \end{macrocode}

% Manually include selected file,
% otherwise process as usual:
%    \begin{macrocode}
\ifchilddocmanual
\section*{part `\childdocname'}
\input{\childdocname}
\else
%    \end{macrocode}

% Include the two chapters:
%    \begin{macrocode}
\include{cdocsch1}
\include{cdocsch2}
%    \end{macrocode}

% Include the two parts unless only chapters should be displayed:
%    \begin{macrocode}
\ifchilddoc\else
\section{part three}
\input{cdocspt3}
\section{part four}
\input{cdocspt4}
\fi
%    \end{macrocode}

% Process as usual until here:
%    \begin{macrocode}
\fi
%    \end{macrocode}

% End of document body:
%    \begin{macrocode}
\end{document}
%    \end{macrocode}
%\iffalse
%</samplemain>
%\fi
%
% %%%%%%%%%%%%%%%%%%%%%%%%%%%%%%%%%%%%%%
% \paragraph{Chapter Include Files.}
%
% The include files are called |cdocsch1.tex| and |cdocsch2.tex|.
%
%\iffalse
%<*samplechap1|samplechap2>
%\fi

% Optional override for |\version| flag:
%    \begin{macrocode}
%%\providecommand{\version}{final}
%    \end{macrocode}

% Include the main document:
%    \begin{macrocode}
\input{childdoc.def}
\childdocof{cdocsamp}
%    \end{macrocode}

%\iffalse
%</samplechap1|samplechap2>
%\fi
%
%\iffalse
%<*samplechap1>
%\fi
% Some text for chapter 1:
%    \begin{macrocode}
\section{one}
some text in chapter one
%    \end{macrocode}

%\iffalse
%</samplechap1>
%\fi
% Some text for chapter 2:
%\iffalse
%<*samplechap2>
%\fi
%    \begin{macrocode}
\section{two}
more text in chapter two
%    \end{macrocode}

%\iffalse
%</samplechap2>
%\fi
%
% %%%%%%%%%%%%%%%%%%%%%%%%%%%%%%%%%%%%%%
% \paragraph{Part Include Files.}
%
% The include files are called |cdocspt3.tex| and |cdocspt4.tex|.
%
%\iffalse
%<*samplepart3|samplepart4>
%\fi

% Optional override for |\version| flag:
%    \begin{macrocode}
%%\providecommand{\version}{final}
%    \end{macrocode}

% Include the main document:
%    \begin{macrocode}
\input{childdoc.def}
\childdocby{cdocsamp}
%    \end{macrocode}

%\iffalse
%</samplepart3|samplepart4>
%\fi
%
%\iffalse
%<*samplepart3>
%\fi
% Some text for part 3:
%    \begin{macrocode}
some text in part three
%    \end{macrocode}

%\iffalse
%</samplepart3>
%\fi
% Some text for part 4:
%\iffalse
%<*samplepart4>
%\fi
%    \begin{macrocode}
more text in part four
%    \end{macrocode}

%\iffalse
%</samplepart4>
%\fi
%
% %%%%%%%%%%%%%%%%%%%%%%%%%%%%%%%%%%%%%%
% \paragraph{Forwarding for a Complete Draft.}
%
% The following forwarding file |cdocsdrf.tex|
% compiles the main document in draft mode:
%\iffalse
%<*sampledraft>
%\fi
%    \begin{macrocode}
\def\version{draft}
\input{childdoc.def}
\childdocforward{cdocsamp}
%    \end{macrocode}

%\iffalse
%</sampledraft>
%\fi
%
% %%%%%%%%%%%%%%%%%%%%%%%%%%%%%%%%%%%%%%
% \paragraph{Forwarding for Final Version of the Chapters.}
%
% The following forwarding files |cdocsfn1.tex| and |cdocsfn2.tex|
% (with identical content)
% compile the final versions of the child documents
% |cdocsch1.tex| and |cdocsch2.tex|, respectively:
%\iffalse
%<*samplefinal>
%\fi
%    \begin{macrocode}
\def\version{final}
\input{childdoc.def}
\childdocforwardprefix[cdocsamp]{cdocsfn}{cdocsch}
%    \end{macrocode}

%\iffalse
%</samplefinal>
%\fi
%
% %%%%%%%%%%%%%%%%%%%%%%%%%%%%%%%%%%%%%%
% \paragraph{Command Line Processing.}
%
% The following three command lines generate the output files
% |cdocscld|, |cdocscl1| and |cdocscl2|
% which should be identical to
% |cdocsdrf|, |cdocsch1| and |cdocsfn2|, respectively:
% \begin{center}
% \begin{tabular}{l}
% |latex -jobname cdocscld \|\\
% |  "\def\version{draft}\input{childdoc.def}\childdocforward{cdocsamp}"|\\
% |latex -jobname cdocscl1 \|\\
% |  "\input{childdoc.def}\childdocforward[cdocsamp]{cdocsch1}"|\\
% |latex -jobname cdocscl2 \|\\
% |  "\def\version{final}\input{childdoc.def}\childdocforward{cdocsch2}"|
% \end{tabular}
% \end{center}
% Note that the trailing backslash on each first line
% merely continues the input to the second line
% (for convenient cut ant paste).
% Furthermore, the command |latex| can be replaced by any
% of its alternative versions such as |pdflatex|.
%
% %%%%%%%%%%%%%%%%%%%%%%%%%%%%%%%%%%%%%%%%%%%%%%%%%%%%%%%%%%%%%%%%%%%%%%%%%%%%%%
% %%%%%%%%%%%%%%%%%%%%%%%%%%%%%%%%%%%%%%%%%%%%%%%%%%%%%%%%%%%%%%%%%%%%%%%%%%%%%%
% \section{Implementation}
%\iffalse
%<*package>
%\fi
%
% This section describes the definitions file |childdoc.def|.

% The definitions cannot be loaded using |\usepackage| or |\RequirePackage|
% which has a mechanism to prevent loading a style file more than once.
% When loading the definitions by means of |\input|
% multiple instances have to be prevented manually:
%\iffalse
%This code needs to be before the `\ProvidesFile' directive
%which is defined at the beginning of this file.
%Therefore it is also placed there and commented out here.
%</package>
%<*discard>
%\fi
%    \begin{macrocode}
\ifdefined\childdocmain\endinput\fi
%    \end{macrocode}
%\iffalse
%</discard>
%<*package>
%\fi
%
% \macro{\ifchilddoc}
% \macro{\ifchilddocmanual}
% The conditional |\ifchilddoc| tells whether a
% child (true) or main (false) document is being compiled.
% The conditional |\ifchilddocmanual| tells whether
% the |\includeonly| mechanism is used (false) or
% the selection of child files must be performed manually (true).
% The definitions initialise to false:
%    \begin{macrocode}
\newif\ifchilddoc
\newif\ifchilddocmanual
%    \end{macrocode}

% \macro{\childdocname}
% \macro{\childdocjob}
% The macro |\childdocname| stores the name of the main document
% to be compiled. The macro |\childdocjob| stores the name of
% the document on which the \LaTeX{} compiler was originally invoked.
% The content of |\jobname| cannot be compared
% to filenames specified in the source due to different catcodes.
% The following code rescans |\jobname|, stores the result
% in |\childdocname| and saves a copy in |\childdocjob|:
%    \begin{macrocode}
\edef\childdocname{\scantokens\expandafter{\jobname\noexpand}}
\let\childdocjob\childdocname
%    \end{macrocode}

% \macro{\childdocdisable}
% The macro |\childdocdisable| prevents the main file
% from being processed more than once.
% At this stage, the main document command |\childdocmain|
% is assumed to be called once again where it should do nothing.
% Any subsequent call to it should prevent
% a secondary processing of the main document
% It overwrites the forwarding commands
% |\childdocof| and |\childdocforward|
% with empty macros to prevent further inclusions of the main document:
%    \begin{macrocode}
\newcommand{\childdocdisable}
{
  \renewcommand{\childdocmain}[1]{\renewcommand{\childdocmain}[1]{\endinput}}
  \renewcommand{\childdocof}[1]{}
  \renewcommand{\childdocby}[2][]{}
  \renewcommand{\childdocforward}[2][]{}
  \renewcommand{\childdocdisable}{}
}
%    \end{macrocode}

% \macro{\childdocmain}
% The macro |\childdocmain| is to be called at the top of the main file
% with nothing or the main filename (without extension) as argument.
% First, it breaks loops.
% If the argument is not empty and does not match |\childdocname|
% (which is set by the first inclusion of |childdoc.def|),
% |\ifchilddoc| is set to true, |\includeonly| is applied to the child file
% and |\jobname| is set to the main file
% (for proper handling of |.aux| files):
%    \begin{macrocode}
\newcommand{\childdocmain}[1]
{
  \childdocdisable\childdocmain{}
  \if?#1?\else
    \begingroup
      \def\childdoctmp{#1}
      \ifx\childdoctmp\childdocname
        \def\childdoctmp{}
      \else
        \def\childdoctmp
        {
          \childdoctrue
          \includeonly{\childdocname}
          \def\childdocjob{#1}
          \def\jobname{#1}
        }
      \fi
      \expandafter
    \endgroup
    \childdoctmp
  \fi
}
%    \end{macrocode}

% \macro{\childdocof}
% The command |\childdocof| redirects
% compilation to the main file |#1|.
%    \begin{macrocode}
\newcommand{\childdocof}[1]
{
  \childdocdisable
  \childdoctrue
  \includeonly{\childdocname}
  \def\jobname{#1}
  \def\childdocjob{#1}
  \input{#1}
}
%    \end{macrocode}

% \macro{\childdocby}
% The command |\childdocby| ....
%    \begin{macrocode}
\newcommand{\childdocby}[2][]
{
  \childdocdisable
  \childdoctrue
  \childdocmanualtrue
  \if?#1?\else
    \def\jobname{#2}
  \fi
  \def\childdocjob{#2}
  \input{#2}
  \endinput
}
%    \end{macrocode}

% \macro{\childdocforward}
% The command |\childdocforward| redirects
% compilation to the main file or
% (if the optional argument is given) a child file.
% Parameters are set as if the main file
% or a child file starting with |\childdocof| was compiled.
% Then compilation is handed over to the main file:
%    \begin{macrocode}
\newcommand{\childdocforward}[2][]
{
  \begingroup
    \if?#1?
      \def\childdoctmp
      {
        \def\childdocname{#2}
        \def\childdocjob{#2}
        \def\jobname{#2}
        \input{#2}
        \endinput
      }
    \else
      \def\childdoctmp
      {
        \childdocdisable
        \def\childdocname{#2}
        \childdoctrue
        \includeonly{#2}
        \def\childdocjob{#1}
        \def\jobname{#1}
        \input{#1}
        \endinput
      }
    \fi
    \expandafter
  \endgroup
  \childdoctmp
}
%    \end{macrocode}

% \macro{\childdocforwardprefix}
% The command |\childdocforwardprefix| redirects
% compilation to the main or a child file by means of a pattern.
% The prefix |#1| in the current filename is replaced by |#2|
% and the suffix of the current filename is kept
% (it is assumed that the filename does not contain the substring `|~~~|'
% which is used as a delimiter).
% Compilation is handed over to the new file by |\childdocforward|:
%    \begin{macrocode}
\newcommand{\childdocforwardprefix}[3][]
{
  \begingroup
    \def\childdocextract #2##1~~~{\def\childdoctmp{\childdocforward[#1]{#3##1}}}
    \expandafter\childdocextract\childdocname~~~
    \expandafter
  \endgroup
  \childdoctmp
}
%    \end{macrocode}

% \macro{\childdoc}
% The deprecated macro |\childdoc| is a legacy version of |\childdocmain|:
%    \begin{macrocode}
\newcommand{\childdoc}{\childdocmain}
%    \end{macrocode}

% \macro{\childdocredirect}
% The deprecated macro |\childdocredirect| is a legacy version
% of |\childdocforward| and |\childdocforwardprefix|:
%    \begin{macrocode}
\newcommand{\childdocredirect}[2][]
{
  \begingroup
    \if?#1?
      \def\childdoctmp{\childdocforward{#2}}
    \else
      \def\childdoctmp{\childdocforwardprefix{#1}{#2}}
    \fi
    \expandafter
  \endgroup
  \childdoctmp
}
%    \end{macrocode}

%\iffalse
%</package>
%\fi
%
\endinput

\childdocof{cdocsamp}
%    \end{macrocode}

%\iffalse
%</samplechap1|samplechap2>
%\fi
%
%\iffalse
%<*samplechap1>
%\fi
% Some text for chapter 1:
%    \begin{macrocode}
\section{one}
some text in chapter one
%    \end{macrocode}

%\iffalse
%</samplechap1>
%\fi
% Some text for chapter 2:
%\iffalse
%<*samplechap2>
%\fi
%    \begin{macrocode}
\section{two}
more text in chapter two
%    \end{macrocode}

%\iffalse
%</samplechap2>
%\fi
%
% %%%%%%%%%%%%%%%%%%%%%%%%%%%%%%%%%%%%%%
% \paragraph{Part Include Files.}
%
% The include files are called |cdocspt3.tex| and |cdocspt4.tex|.
%
%\iffalse
%<*samplepart3|samplepart4>
%\fi

% Optional override for |\version| flag:
%    \begin{macrocode}
%%\providecommand{\version}{final}
%    \end{macrocode}

% Include the main document:
%    \begin{macrocode}
% \iffalse
%
% childdoc.dtx Copyright (C) 2017-2018 Niklas Beisert
%
% This work may be distributed and/or modified under the
% conditions of the LaTeX Project Public License, either version 1.3
% of this license or (at your option) any later version.
% The latest version of this license is in
%   http://www.latex-project.org/lppl.txt
% and version 1.3 or later is part of all distributions of LaTeX
% version 2005/12/01 or later.
%
% This work has the LPPL maintenance status `maintained'.
%
% The Current Maintainer of this work is Niklas Beisert.
%
% This work consists of the files childdoc.dtx and childdoc.ins
% and the derived files childdoc.def and cdocsamp.tex with
% cdocsch1.tex, cdocsch2.tex, cdocsdrf.tex, cdocsfn1.tex, cdocsfn2.tex.
%
%<package>\ifdefined\childdocmain\endinput\fi
%<package>\ProvidesFile{childdoc.def}[2018/12/30 v2.0 child document driver]
%<samplemain>\ProvidesFile{cdocsamp.tex}[2018/12/30 v2.0 sample for childdoc]
%<*driver>
%\ProvidesFile{childdoc.drv}[2018/12/30 v2.0 childdoc reference manual file]
\PassOptionsToClass{10pt,a4paper}{article}
\documentclass{ltxdoc}

\usepackage[margin=35mm]{geometry}
\usepackage{hyperref}
\usepackage{hyperxmp}
\usepackage[usenames]{color}

\hypersetup{colorlinks=true}
\hypersetup{pdfstartview=FitH}
\hypersetup{pdfpagemode=UseNone}
\hypersetup{pdfsource={}}
\hypersetup{pdflang={en-UK}}
\hypersetup{pdfcopyright={Copyright 2017-2018 Niklas Beisert.
  This work may be distributed and/or modified under the
  conditions of the LaTeX Project Public License, either version 1.3
  of this license or (at your option) any later version.}}
\hypersetup{pdflicenseurl={http://www.latex-project.org/lppl.txt}}
\hypersetup{pdfcontactaddress={ETH Zurich, ITP, HIT K,
  Wolfgang-Pauli-Strasse 27}}
\hypersetup{pdfcontactpostcode={8093}}
\hypersetup{pdfcontactcity={Zurich}}
\hypersetup{pdfcontactcountry={Switzerland}}
\hypersetup{pdfcontactemail={nbeisert@itp.phys.ethz.ch}}
\hypersetup{pdfcontacturl={http://people.phys.ethz.ch/\xmptilde nbeisert/}}

\newcommand{\secref}[1]{\hyperref[#1]{section \ref*{#1}}}

\parskip1ex
\parindent0pt
\let\olditemize\itemize
\def\itemize{\olditemize\parskip0pt}

\begin{document}

\title{The \textsf{childdoc} Package}
\hypersetup{pdftitle={The childdoc Package}}
\author{Niklas Beisert\\[2ex]
  Institut f\"ur Theoretische Physik\\
  Eidgen\"ossische Technische Hochschule Z\"urich\\
  Wolfgang-Pauli-Strasse 27, 8093 Z\"urich, Switzerland\\[1ex]
  \href{mailto:nbeisert@itp.phys.ethz.ch}
  {\texttt{nbeisert@itp.phys.ethz.ch}}}
\hypersetup{pdfauthor={Niklas Beisert}}
\hypersetup{pdfsubject={Manual for the LaTeX2e Package childdoc}}
\date{30 December 2018, \textsf{v2.0}}
\maketitle

\begin{abstract}\noindent
\textsf{childdoc} is a \LaTeXe{} package
that enables the direct compilation
of document sections included by |\include|
to individual files.
\end{abstract}

\begingroup
\parskip0ex
\tableofcontents
\endgroup

%%%%%%%%%%%%%%%%%%%%%%%%%%%%%%%%%%%%%%%%%%%%%%%%%%%%%%%%%%%%%%%%%%%%%%%%%%%%%%%%
%%%%%%%%%%%%%%%%%%%%%%%%%%%%%%%%%%%%%%%%%%%%%%%%%%%%%%%%%%%%%%%%%%%%%%%%%%%%%%%%
\section{Introduction}

\LaTeX{} provides a mechanism to structure a large document (such as a book)
into a main file and several child files (containing the chapters)
using the |\include| command.
This mechanism is beneficial for documents
which span hundreds of pages in order to
make the source file(s) more manageable.
Moreover, compilation can be restricted to
selected child files by means of the |\includeonly| command.
The latter feature can be used to reduce the compilation time while editing
(this was significantly more useful in the earlier days of \LaTeX{})
or to generate a smaller document which is easier to navigate.
Another application of |\includeonly| is to generate
documents consisting of selected parts of the complete document.

However, there are a few drawbacks of the plain |\include| mechanism:
\begin{itemize}
\item
The child files cannot be compiled on their own,
they can only be compiled via the main file.
A naive editing environment
(such as a text editor with an option
to have the current file processed by \LaTeX)
may require one to switch to the main file before compiling;
attempting to compile the child file produces errors.
\item
The main file must be modified (each time)
to adjust the |\includeonly| command
to the present needs. This easily leaves the main file in a messy state.
\item
The generated document will always carry the filename
of the main document. This is inconvenient if
several child files are to be compiled and
to be kept for distribution.
\end{itemize}

The present package provides a simple interface
to make child files individually compilable by \LaTeX{}.
Compiling a child file then has the same effect as compiling
the main file with an |\includeonly| command
to select the appropriate child.
Moreover the generated document will carry the name of the child
rather than the main file.
This resolves all three above issues.

This feature is meant to make the editing of books,
thesis documents and lecture notes somewhat more convenient.
However, the package can also be used efficiently for
composing a series of documents (such as exercise sheets)
which are typically distributed individually.
It then assists the author in generating the individual documents
(potentially in different versions)
as well as a document containing the collected series.
Another application is in developing style files
or other kinds of included material
where compilation of the style file could redirect
to a sample or test file.

%%%%%%%%%%%%%%%%%%%%%%%%%%%%%%%%%%%%%%%%%%%%%%%%%%%%%%%%%%%%%%%%%%%%%%%%%%%%%%%%
%%%%%%%%%%%%%%%%%%%%%%%%%%%%%%%%%%%%%%%%%%%%%%%%%%%%%%%%%%%%%%%%%%%%%%%%%%%%%%%%
\section{Usage}

First of all, the package \textsf{childdoc} is \emph{not} a standard
\LaTeXe{} |.sty| style file! Therefore it needs to be invoked in
a non-standard way.

%%%%%%%%%%%%%%%%%%%%%%%%%%%%%%%%%%%%%%%%%%%%%%%%%%%%%%%%%%%%%%%%%%%%%%%%%%%%%%%%
\subsection{Included Files}
\label{sec:include}

%%%%%%%%%%%%%%%%%%%%%%%%%%%%%%%%%%%%%%%%
\DescribeMacro{\childdocmain}
To use the package, add the commands
\begin{center}
\begin{tabular}{l}
|\input{childdoc.def}|\\
|\childdocmain{}|\\
\end{tabular}
\end{center}
at the very top of the main \LaTeX{} file,
in particular \emph{before} the |\documentclass| statement!
The argument of |\childdocmain| should be left empty
(but it must be present).

%%%%%%%%%%%%%%%%%%%%%%%%%%%%%%%%%%%%%%%%
\DescribeMacro{\childdocof}
Furthermore, add the commands
\begin{center}
\begin{tabular}{l}
|\input{childdoc.def}|\\
|\childdocof{|\textit{main}|}|\\
\end{tabular}
\end{center}
at the top of every child file \textit{child}
which is included by |\include{|\textit{child}|}|
from within the main file
(or at least for those files to be compiled individually).
The argument \textit{main} must be the filename of the main file.

There are a couple of
considerations in setting up the main and child documents:

%%%%%%%%%%%%%%%%%%%%%%%%%%%%%%%%%%%%%%%%
\paragraph{Restrictions.}

Please note the following restrictions:
\begin{itemize}
\item
|\childdocmain| must be called with one argument \textit{main}
to ensure compatibility with earlier version of the package.
It must either be empty (|\childdocmain{}|)
or precisely match the filename of the main file in which it is specified.
See \secref{sec:detection} for further information.
\item
The filename \textit{main} must be specified without the |.tex| extension.
\item
The filename \textit{main} is case sensitive
(even in case-insensitive file systems)
due to internal string comparison.
\item
The argument \textit{main} should be fully expanded, it cannot be a macro.
\item
Subdirectories and special characters should be avoided in filenames.
\item
The command |\childdocmain{|\textit{main}|}| must be followed by a whitespace.
It should not be followed immediately by another command
or by a comment mark `|%|'.
This is because the \TeX{} parser reads the token immediately following
the argument of |\childdocmain| and puts it
at the beginning of every child section;
however, a white\-space is ignored.
\end{itemize}

%%%%%%%%%%%%%%%%%%%%%%%%%%%%%%%%%%%%%%%%
\paragraph{Content of Main File.}

It is advisable to place all content in the child files included by |\include|.
Any output contained in the main file will appear in all child documents
unless suppressed manually;
it cannot be suppressed automatically by the |\includeonly| directive
and thus should normally be avoided.
A method to include some content in the main file
by means of conditional processing is described in \secref{sec:conditional}.

%%%%%%%%%%%%%%%%%%%%%%%%%%%%%%%%%%%%%%%%
\paragraph{Page Numbering.}

When only a part of the document is compiled,
the appropriate numbering of pages
(as well as other status parameters)
is determined from the |.aux| files.
The latter contain information from previous passes.
However this information needs to propagate through
all intermediate child documents.
Therefore the page numbering in child documents may well
be inconsistent until the complete document is compiled at least once.

A useful (if unconventional) way to always ensure a consistent
page numbering is to restart the numbering in each child document
and denote the pages by `\textit{child}|.|\textit{page}'
where \textit{child} represents the chapter/section number of the child file.
This can be achieved by the command
|\numberwithin{page}{|\textit{child}|}|
of the \textsf{amsmath} package
where \textit{child} can be |chapter| or |section|
depending on the chosen structuring.
Alternatively, one can modify the macro |\thepage| appropriately
and reset the counter |page| at the start of each child file.

%%%%%%%%%%%%%%%%%%%%%%%%%%%%%%%%%%%%%%%%%%%%%%%%%%%%%%%%%%%%%%%%%%%%%%%%%%%%%%%%
\subsection{Conditional Processing}
\label{sec:conditional}

The package provides a mechanism to compile different versions
of a document. To customise the versions further some conditional processing
can come in handy to distinguish which version is being compiled.
The package provides two macros to describe the compilation context:

%%%%%%%%%%%%%%%%%%%%%%%%%%%%%%%%%%%%%%%%
\DescribeMacro{\ifchilddoc}
The conditional |\ifchilddoc| distinguishes between the compilation of
child documents and the main document:
%
\begin{center}
|\ifchilddoc |\textit{child-code}| |[|\||else |\textit{main-code}]| \||fi|
\end{center}

%%%%%%%%%%%%%%%%%%%%%%%%%%%%%%%%%%%%%%%%
\DescribeMacro{\childdocname}
\DescribeMacro{\childdocjob}
The macro |\childdocname| contains the filename (without extension)
of the main or child file being processed.
Note that |\childdocjob| will always contain the name of the main file.

%%%%%%%%%%%%%%%%%%%%%%%%%%%%%%%%%%%%%%%%
\paragraph{Title Page.}

Conditional processing can be used to include a title or banner page
in the main document when proper precautions are taken.
Importantly, the code in the main file should ensure that the page counter
(as well as other status parameters which are stored in the |.aux| files)
takes the same value after the conditional processing.
Otherwise the page numbers may take divergent values
depending on which part is compiled.

For example, a title page could be declared by:
%
\begin{center}
\begin{tabular}{l}
|\ifchilddoc\||else|\\
|\addtocounter{page}{-1}|\\
\textit{code for title page}\\
|\newpage|\\
|\||fi|
\end{tabular}
\end{center}
%
A banner page for the child documents can be generated by:
%
\begin{center}
\begin{tabular}{l}
|\ifchilddoc|\\
|\addtocounter{page}{-1}|\\
\textit{code for banner page}\\
|\newpage|\\
|\||fi|
\end{tabular}
\end{center}
%
Here one could write a message such as:
\begin{center}
|This is the part \childdocname{} of \childdocjob{}.|
\end{center}

%%%%%%%%%%%%%%%%%%%%%%%%%%%%%%%%%%%%%%%%%%%%%%%%%%%%%%%%%%%%%%%%%%%%%%%%%%%%%%%%
\subsection{Flags}
\label{sec:flags}

The package makes it easy to generate different versions
of the main or child documents.
To this end compilation flags can be defined
and assigned different default values.
They will be particularly useful in conjunction
with the forwarding mechanism described in \secref{sec:forward}.

For example, it may be useful to have a flag |\version|
which can be set to |draft| or |final|.
The document source will contain some conditional code
depending on the value of |\version|.
Suppose further, the flag should default to |final| for the main file
and to |draft| for child files
which is a natural assignment for editing the document.
This is achieved by placing the following code
in the preamble of the main document
(below the |\childdocmain| directive):
%
\begin{center}
\begin{tabular}{l}
|\ifchilddoc|\\
|\providecommand{\version}{draft}|\\
|\||else|\\
|\providecommand{\version}{final}|\\
|\||fi|
\end{tabular}
\end{center}
%
The definition by |\providecommand| makes sure
that previous definitions are not overwritten.
Further statements |\providecommand{\version}{...}|
can thus be added before the above code to override it.

For the main file, one might add a line
(between |\childdocmain| and the above block)
%
\begin{center}
|%\ifchilddoc\||else\providecommand{\version}{draft}\||fi|
\end{center}
%
which can be uncommented to produce a draft version.
Likewise one can add a line to the very top of a child file
(above the |\childdocof{|\textit{main}|}| directive)
%
\begin{center}
|%\providecommand{\version}{final}|
\end{center}
%
which can be uncommented to produce the final version of this child document.

%%%%%%%%%%%%%%%%%%%%%%%%%%%%%%%%%%%%%%%%%%%%%%%%%%%%%%%%%%%%%%%%%%%%%%%%%%%%%%%%
\subsection{Forwarding}
\label{sec:forward}

Different versions of the main or child documents
using compilation flags as described in \secref{sec:flags}
can be (permanently) stored in different files
for convenient compilation, viewing and distribution.
To this end, the package defines a command
to pass on compilation to a different file:

%%%%%%%%%%%%%%%%%%%%%%%%%%%%%%%%%%%%%%%%
\DescribeMacro{\childdocforward}
The command |\childdocforward| redirects processing to
another source file:
%
\begin{center}
\begin{tabular}{l}
|\input{childdoc.def}|\\
|\childdocforward[|\textit{main}|]{|\textit{dest}|}|\\
\end{tabular}
\end{center}
%
The argument \textit{dest} is the destination file
(without extension).
It should be the main file or one of the child files.
Note that further \textsf{childdoc} directives
such as |\childdocof| and |\childdocforward|
in the indicated file will be processed in this form.
The optional argument \textit{main}
passes on directly to the main file \textit{main}
while pretending to compile the child \textit{dest}.
This form behaves as if \textit{dest}
issues |\childdocof{|\textit{main}|}| right away,
and no further \textsf{childdoc} directives will be processed.

%%%%%%%%%%%%%%%%%%%%%%%%%%%%%%%%%%%%%%%%
\DescribeMacro{\...prefix}
In the alternative form |\childdocforwardprefix|,
%
\begin{center}
\begin{tabular}{l}
|\input{childdoc.def}|\\
|\childdocforwardprefix[|\textit{main}|]{|\textit{prefix}|}{|\textit{dest}|}|
\end{tabular}
\end{center}
%
the destination file is determined by a pattern
depending on the current file:
To make this work, the current file must be called
`{\textit{prefix}\hspace{0.2em}\textit{suffix}}'
with \textit{prefix} matching precisely the argument.
Processing is then passed on to the file
`{\textit{dest}\hspace{0.2em}\textit{suffix}}'.
Surely, the same effect is achieved by
directly specifying the
argument `{\textit{dest}\hspace{0.2em}\textit{suffix}}'
in the first form.
However, that requires to set up a different file
for each child. With the alternative form of the command
all these files can have exactly the same content
which simplifies setting them up and maintaining them.

For example, the following file |draft.tex|
with a compilation flag |\version| as described in \secref{sec:flags}
compiles the main document as a draft:
%
\begin{center}
\begin{tabular}{l}
|\def\version{draft}|\\
|\input{childdoc.def}|\\
|\childdocforward{|\textit{main}|}|
\end{tabular}
\end{center}
%
Likewise, the following files |final|\textit{nn}|.tex|
compile the final version of the child document
|child|\textit{nn}|.tex|:
%
\begin{center}
\begin{tabular}{l}
|\def\version{final}|\\
|\input{childdoc.def}|\\
|\childdocforwardprefix{final}{child}|
\end{tabular}
\end{center}
%

Note that when several versions of a main file and/or of each child file
are to be generated, it may be convenient to set up a |Makefile| or
shell script to automatise the process.

%%%%%%%%%%%%%%%%%%%%%%%%%%%%%%%%%%%%%%%%%%%%%%%%%%%%%%%%%%%%%%%%%%%%%%%%%%%%%%%%
\subsection{Command Line Processing}
\label{sec:commandline}

The effect of redirection files can also be achieved by invoking
the \LaTeX{} compiler with a more elaborate command line.
Most conveniently this should be done as part
of a shell script or a |Makefile|.

When using \textsf{childdoc} in the main file, the following
command lines effectively perform a redirection
(note that depending on the shell being used,
backslashes may have to be doubled: `|\|' $\to$ `|\\|'):
%
\begin{center}
|... -jobname "|\textit{target}|" |\\|"|[\textit{flags}]%
|\input{childdoc.def}\childdocforward[|\textit{main}|]{|\textit{dest}|}"|
\end{center}
%
Here \textit{target} is the name of the output file,
\textit{main} is the name of the main file
and \textit{dest} is the name of the main or child file to be processed
(all filenames without extensions).
The optional argument \textit{main} can be omitted
if \textit{main} matches \textit{dest}.
Optionally, compilation \textit{flags} can be defined via |\def| commands.
This command line makes the \TeX{} engine believe
it is compiling the file \textit{target}
whose content is specified as the latter parameter.
The provided code then forwards the processing to
\textit{main} or \textit{dest} as described in \secref{sec:forward}.

%%%%%%%%%%%%%%%%%%%%%%%%%%%%%%%%%%%%%%%%%%%%%%%%%%%%%%%%%%%%%%%%%%%%%%%%%%%%%%%%
\subsection{Include by Input}
\label{sec:input}

Including child documents by |\include| has some restrictions by design.
Most notably, the content of a child document always occupies
its own set of pages; pages cannot be shared between child documents.
Usually, this behaviour makes perfect sense
because each child document contain an essential part of the document.
However, in some situations it may be desirable to compose
a document from a collection of parts
without having mandatory page breaks between then.
For this case, the package
provides a mechanism to include parts
by |\input| which can also be processed individually.
However, by construction this mechanism
requires manual handling of the content to be output.

%%%%%%%%%%%%%%%%%%%%%%%%%%%%%%%%%%%%%%%%
\DescribeMacro{\ifchilddocmanual}
The main file should be prepared as usual, see \secref{sec:include}.
However, the document body must make a distinction
between processing of an individual part and of the main document, e.g.:
%
\begin{center}
\begin{tabular}{l}
|\ifchilddocmanual|\\
|\input{\childdocname}|\\
|\||else|\\
\textit{document body with }|\input{|\textit{part}|}|\\
|\||fi|
\end{tabular}
\end{center}
%
The conditional |\ifchilddocmanual| is true whenever
a part to be included by |\input| is being compiled,
and the name of the part is stored in |\childdocname|.

%%%%%%%%%%%%%%%%%%%%%%%%%%%%%%%%%%%%%%%%
\DescribeMacro{\childdocby}
Each part to be included by |\input| should start with:
%
\begin{center}
\begin{tabular}{l}
|\input{childdoc.def}|\\
|\childdocby{|\textit{main}|}|\\
\end{tabular}
\end{center}
%
The directive |\childdocby| is similar to |\childdocof|
described in \secref{sec:include},
but the subsequent selection of content must be done manually.
To that end, both |\ifchilddoc| and |\ifchilddocmanual|
will be true upon processing of a part,
and the name of the part is stored in |\childdocname|.
Note that |\jobname| will be set to the filename of the current part
so that each part receives an individual |.aux| file
that does not interfere with the |.aux| file(s) of the main document.
This behaviour can be altered by the alternative form
|\childdocby[*]{|\textit{main}|}| (with a non-empty optional argument)
which uses the |.aux| file of the main document
by setting |\jobname| to \textit{main}.

%%%%%%%%%%%%%%%%%%%%%%%%%%%%%%%%%%%%%%%%%%%%%%%%%%%%%%%%%%%%%%%%%%%%%%%%%%%%%%%%
\subsection{Driver Development}
\label{sec:driver}

The \textsf{childdoc} mechanism can also be use for the development
of definition files such as \LaTeX{} styles or classes.
This case differs from the above setup with multiple parts
included by |\include| in that no |\includeonly| should be invoked.
This can be achieved by starting the include file
(before |\ProvidesPackage|) with:
%
\begin{center}
\begin{tabular}{l}
|\input{childdoc.def}|\\
|\childdocforward{|\textit{main}|}|\\
\end{tabular}
\end{center}
%
or alternatively with:
%
\begin{center}
\begin{tabular}{l}
|\input{childdoc.def}|\\
|\childdocby{|\textit{main}|}|\\
\end{tabular}
\end{center}
%
Both forms have slightly different effects as described above.
The main file is prepared as usual, see \secref{sec:include}.

%%%%%%%%%%%%%%%%%%%%%%%%%%%%%%%%%%%%%%%%%%%%%%%%%%%%%%%%%%%%%%%%%%%%%%%%%%%%%%%%
\subsection{Legacy Detection}
\label{sec:detection}

The directive |\childdocmain| in the main file can detect
whether the complete document or merely a child is to be compiled
even without using the directive |\childdocof|.
This method is deprecated because it is less robust
and there is no compelling reason to use it;
it is merely provided for backward compatibility
and it may be removed in future versions.

If the detection mechanism is to be used,
it is mandatory to correctly specify
the filename of the main file as the argument of |\childdocmain|:
%
\begin{center}
\begin{tabular}{l}
|\input{childdoc.def}|\\
|\childdocmain{|\textit{main}|}|\\
\end{tabular}
\end{center}
%
If |\jobname| does not match the argument \textit{main} of |\childdocmain|,
it is assumed that |\jobname| points to the child file to be compiled.
When using |\childdocmain| with the main file specified as argument,
it suffices to start a child file
with just |\input{|\textit{main}|}|
without loading of the package and using |\childdocof|.
If instead all processing is done
with the appropriate \textsf{childdoc} directives,
the argument of \textit{main} of |\childdocmain| can be empty.

An alternative version of the command line processing described
in \secref{sec:commandline} using the detection mechanism reads:
%
\begin{center}
|... -jobname "|\textit{target}|" "|[\textit{flags}]%
[|\def\jobname{|\textit{dest}|}|]|\input{|\textit{main}|}"|
\end{center}

%%%%%%%%%%%%%%%%%%%%%%%%%%%%%%%%%%%%%%%%%%%%%%%%%%%%%%%%%%%%%%%%%%%%%%%%%%%%%%%%
\subsection{Manual Code}
\label{sec:manual}

In case one cannot be certain whether the definitions file |childdoc.def|
is installed on the target \TeX{} distribution
and one prefers not to ship it,
it is conceivable to paste a few relevant commands into the sources.

To that end, drop all statements |\input{childdoc.def}|
and perform the replacements as outlined below.
Instead of |\childdocmain{|\textit{main}|}| add the following code
to the top of the main file:
%
\begin{center}
\begin{tabular}{l}
|\||ifdefined\childdocname\endinput\||fi\newif\ifchilddoc|\\
|\edef\childdocname{\scantokens\expandafter{\jobname\noexpand}}|\\
|\def\childdocmain{|\textit{main}|}\||ifx\childdocmain\childdocname\||else|\\
|\childdoctrue\includeonly{\childdocname}\let\jobname\childdocmain\||fi|\\
\end{tabular}
\end{center}
%
Instead of |\childdocof{|\textit{main}|}| just include the main file
at the top of each child file:
%
\begin{center}
|\input{|\textit{main}|}|
\end{center}
%
A simple redirection |\childdocforward{|\textit{dest}|}| is achieved by:
%
\begin{center}
|\def\jobname{|\textit{dest}|}\input{\jobname}|
\end{center}
%
The redirection with prefix
|\childdocforwardprefix[|\textit{prefix}|]{|\textit{dest}|}|
is accomplished by:
%
\begin{center}
\begin{tabular}{l}
|{\edef\jobname{\scantokens\expandafter{\jobname\noexpand}}|\\
|\def\redirectjob |\textit{prefix}|#1~~~{\gdef\jobname{|\textit{dest}|#1}}|\\
|\expandafter\redirectjob\jobname~~~}\input{\jobname}|
\end{tabular}
\end{center}

In an alternative approach,
child documents can be compiled by a specific command line
without additional code or specific definitions:
%
\begin{center}
|... -jobname "|\textit{target}|" "|[\textit{flags}]%
|\includeonly{|\textit{dest}|}\input{|\textit{main}|}"|
\end{center}
%

%%%%%%%%%%%%%%%%%%%%%%%%%%%%%%%%%%%%%%%%%%%%%%%%%%%%%%%%%%%%%%%%%%%%%%%%%%%%%%%%
%%%%%%%%%%%%%%%%%%%%%%%%%%%%%%%%%%%%%%%%%%%%%%%%%%%%%%%%%%%%%%%%%%%%%%%%%%%%%%%%
\section{Information}

%%%%%%%%%%%%%%%%%%%%%%%%%%%%%%%%%%%%%%%%%%%%%%%%%%%%%%%%%%%%%%%%%%%%%%%%%%%%%%%%
\subsection{Copyright}

Copyright \copyright{} 2017--2018 Niklas Beisert

This work may be distributed and/or modified under the
conditions of the \LaTeX{} Project Public License, either version 1.3
of this license or (at your option) any later version.
The latest version of this license is in
  \url{http://www.latex-project.org/lppl.txt}
and version 1.3 or later is part of all distributions of \LaTeX{}
version 2005/12/01 or later.

This work has the LPPL maintenance status `maintained'.

The Current Maintainer of this work is Niklas Beisert.

This work consists of the files |README.txt|, |childdoc.ins| and |childdoc.dtx|
as well as the derived files |childdoc.def|, |cdocsamp.tex|
with |cdocsch1.tex|, |cdocsch2.tex|, |cdocspt3.tex|, |cdocspt4.tex|,
|cdocsdrf.tex|, |cdocsfn1.tex|, |cdocsfn2.tex|
as well as |childdoc.pdf|.

%%%%%%%%%%%%%%%%%%%%%%%%%%%%%%%%%%%%%%%%%%%%%%%%%%%%%%%%%%%%%%%%%%%%%%%%%%%%%%%%
\subsection{Files and Installation}

The package consists of the files:
%
\begin{center}
\begin{tabular}{ll}
    |README.txt|   & readme file \\
    |childdoc.ins| & installation file \\
    |childdoc.dtx| & source file \\
    |childdoc.def| & definition file \\
    |cdocsamp.tex| & sample main file \\
    |cdocsch1.tex| & sample include file \\
    |cdocsch2.tex| & sample include file \\
    |cdocspt3.tex| & sample part file \\
    |cdocspt4.tex| & sample part file \\
    |cdocsdrf.tex| & sample redirection file \\
    |cdocsfn1.tex| & sample redirection file \\
    |cdocsfn2.tex| & sample redirection file \\
    |childdoc.pdf| & manual
\end{tabular}
\end{center}
%
The distribution consists of the files
|README.txt|, |childdoc.ins| and |childdoc.dtx|.
%
\begin{itemize}
\item
Run (pdf)\LaTeX{} on |childdoc.dtx|
to compile the manual |childdoc.pdf| (this file).
\item
Run \LaTeX{} on |childdoc.ins| to create the definitions file |childdoc.def|
and the sample |cdocsamp.tex| with include files
|cdocsch1.tex|, |cdocsch2.tex|, |cdocspt3.tex|, |cdocspt4.tex|,
|cdocsdrf.tex|, |cdocsfn1.tex|, |cdocsfn2.tex|.
Then copy the file |childdoc.def| to an appropriate directory of your \LaTeX{}
distribution, e.g.\ \textit{texmf-root}|/tex/latex/childdoc|.
\end{itemize}

%%%%%%%%%%%%%%%%%%%%%%%%%%%%%%%%%%%%%%%%%%%%%%%%%%%%%%%%%%%%%%%%%%%%%%%%%%%%%%%%
\subsection{Related CTAN Packages}

There are several other packages which offer a similar functionality:
%
\begin{itemize}
\item
The packages
\href{http://ctan.org/pkg/docmute}{\textsf{docmute}},
\href{http://ctan.org/pkg/includex}{\textsf{includex}} and
\href{http://ctan.org/pkg/standalone}{\textsf{standalone}}
provide commands to include only the document body of
a child file thus allowing both files to be compiled individually.
\item
The packages \href{http://ctan.org/pkg/subdocs}{\textsf{subdocs}}
and \href{http://ctan.org/pkg/subfiles}{\textsf{subfiles}}
provide structures in which the main and child documents can be
encapsulated and allowing them to be compiled individually.
The inclusion mechanism is different from the conventional |\include|.
\item
The package \href{http://ctan.org/pkg/combine}{\textsf{combine}}
is an elaborate solution to combine several documents into one.
\end{itemize}
%
See also the CTAN topic \href{http://ctan.org/topic/subdocs}{\textsf{subdocs}}
for further related packages.
The present package differs from the above solutions in that
a document structure constructed with the conventional |\include| mechanism
just needs two extra commands at the top of every file
such that all constituent files can be compiled individually.

%%%%%%%%%%%%%%%%%%%%%%%%%%%%%%%%%%%%%%%%%%%%%%%%%%%%%%%%%%%%%%%%%%%%%%%%%%%%%%%%
%\subsection{Feature Suggestions}
%
%The following is a list of features which may be useful for future
%versions of this package:
%%
%\begin{itemize}
%\item
%\ldots
%\end{itemize}

%%%%%%%%%%%%%%%%%%%%%%%%%%%%%%%%%%%%%%%%%%%%%%%%%%%%%%%%%%%%%%%%%%%%%%%%%%%%%%%%
\subsection{Revision History}

%%%%%%%%%%%%%%%%%%%%%%%%%%%%%%%%%%%%%%%%
\paragraph{v2.0:} 2018/12/30

\begin{itemize}
\item
immediate forward processing
\item
added |\childdocby| mechanism
\item
manual restructured
\end{itemize}

%%%%%%%%%%%%%%%%%%%%%%%%%%%%%%%%%%%%%%%%
\paragraph{v1.6:} 2018/01/17

\begin{itemize}
\item
application for development of include files
\item
corrections to manual
\end{itemize}

%%%%%%%%%%%%%%%%%%%%%%%%%%%%%%%%%%%%%%%%
\paragraph{v1.5:} 2017/05/21

\begin{itemize}
\item
more complete structuring introduced
\item
|\childdocof| introduced
\item
|\childdoc| renamed to |\childdocmain|
\item
|\childredirect| renamed to |\childdocforward| and |\childdocforwardprefix|
and functionality expanded
\end{itemize}

%%%%%%%%%%%%%%%%%%%%%%%%%%%%%%%%%%%%%%%%
\paragraph{v1.0:} 2017/04/27

\begin{itemize}
\item
manual and install package
\item
first version published on CTAN
\end{itemize}

%%%%%%%%%%%%%%%%%%%%%%%%%%%%%%%%%%%%%%%%
\paragraph{v0.6:} 2017/04/26

\begin{itemize}
\item
redirection mechanism added
\end{itemize}

%%%%%%%%%%%%%%%%%%%%%%%%%%%%%%%%%%%%%%%%
\paragraph{v0.5:} 2017/04/26

\begin{itemize}
\item
functionality in definition file
\end{itemize}


%%%%%%%%%%%%%%%%%%%%%%%%%%%%%%%%%%%%%%%%%%%%%%%%%%%%%%%%%%%%%%%%%%%%%%%%%%%%%%%%
%%%%%%%%%%%%%%%%%%%%%%%%%%%%%%%%%%%%%%%%%%%%%%%%%%%%%%%%%%%%%%%%%%%%%%%%%%%%%%%%
%%%%%%%%%%%%%%%%%%%%%%%%%%%%%%%%%%%%%%%%%%%%%%%%%%%%%%%%%%%%%%%%%%%%%%%%%%%%%%%%
\appendix

\settowidth\MacroIndent{\rmfamily\scriptsize 000\ }

 \DocInput{childdoc.dtx}

\end{document}
%</driver>
% \fi
%
% %%%%%%%%%%%%%%%%%%%%%%%%%%%%%%%%%%%%%%%%%%%%%%%%%%%%%%%%%%%%%%%%%%%%%%%%%%%%%%
% %%%%%%%%%%%%%%%%%%%%%%%%%%%%%%%%%%%%%%%%%%%%%%%%%%%%%%%%%%%%%%%%%%%%%%%%%%%%%%
% \section{Sample}
%\iffalse
%<*samplemain>
%\fi
%
% The following presents a sample document
% with two chapters, two parts, a title page,
% a compile flag as well as three forwarding files to set the flag.
% It consists of eight |.tex| files:
% \begin{center}
% \begin{tabular}{ll}
% |cdocsamp.tex|&main file\\
% |cdocsch1.tex|&include file for chapter 1\\
% |cdocsch2.tex|&include file for chapter 2\\
% |cdocspt3.tex|&include file for part 3\\
% |cdocspt4.tex|&include file for part 4\\
% |cdocsdrf.tex|&forwarding file for main file in draft mode\\
% |cdocsfi1.tex|&forwarding file for final version of chapter 1\\
% |cdocsfi2.tex|&forwarding file for final version of chapter 2\\
% \end{tabular}
% \end{center}
% Each of the eight files can be compiled directly by the \LaTeX{} compiler.
%
% %%%%%%%%%%%%%%%%%%%%%%%%%%%%%%%%%%%%%%
% \paragraph{Main File.}
%
% The main file is called |cdocsamp.tex|.
%
% Load the \textsf{childdoc} definitions and
% declare the filename for the main document:
%    \begin{macrocode}
\input{childdoc.def}
\childdocmain{}
%    \end{macrocode}

% Optional override for |\version| flag:
%    \begin{macrocode}
%%\ifchilddoc\else\providecommand{\version}{draft}\fi
%    \end{macrocode}

% Define the default values for the |\version| flag
% (|final| for the main file and |draft| for childs):
%    \begin{macrocode}
\ifchilddoc
\providecommand{\version}{draft}
\else
\providecommand{\version}{final}
\fi
%    \end{macrocode}

% Load the standard document class:
%    \begin{macrocode}
\documentclass[12pt]{article}
%    \end{macrocode}

% Start the document body:
%    \begin{macrocode}
\begin{document}
%    \end{macrocode}

% Declare a title page.
% Print title, part of document being processed and version flag:
%    \begin{macrocode}
\addtocounter{page}{-1}
\begin{center}
{\LARGE\bfseries{}childdoc example\par}
\vspace{1cm}
\ifchilddoc
\ifchilddocmanual part\else chapter\fi:
`\childdocname' of `\childdocjob'\par
\else
main document: `\childdocjob'\par
\fi
version: \version\par
\end{center}
\newpage
%    \end{macrocode}

% Manually include selected file,
% otherwise process as usual:
%    \begin{macrocode}
\ifchilddocmanual
\section*{part `\childdocname'}
\input{\childdocname}
\else
%    \end{macrocode}

% Include the two chapters:
%    \begin{macrocode}
\include{cdocsch1}
\include{cdocsch2}
%    \end{macrocode}

% Include the two parts unless only chapters should be displayed:
%    \begin{macrocode}
\ifchilddoc\else
\section{part three}
\input{cdocspt3}
\section{part four}
\input{cdocspt4}
\fi
%    \end{macrocode}

% Process as usual until here:
%    \begin{macrocode}
\fi
%    \end{macrocode}

% End of document body:
%    \begin{macrocode}
\end{document}
%    \end{macrocode}
%\iffalse
%</samplemain>
%\fi
%
% %%%%%%%%%%%%%%%%%%%%%%%%%%%%%%%%%%%%%%
% \paragraph{Chapter Include Files.}
%
% The include files are called |cdocsch1.tex| and |cdocsch2.tex|.
%
%\iffalse
%<*samplechap1|samplechap2>
%\fi

% Optional override for |\version| flag:
%    \begin{macrocode}
%%\providecommand{\version}{final}
%    \end{macrocode}

% Include the main document:
%    \begin{macrocode}
\input{childdoc.def}
\childdocof{cdocsamp}
%    \end{macrocode}

%\iffalse
%</samplechap1|samplechap2>
%\fi
%
%\iffalse
%<*samplechap1>
%\fi
% Some text for chapter 1:
%    \begin{macrocode}
\section{one}
some text in chapter one
%    \end{macrocode}

%\iffalse
%</samplechap1>
%\fi
% Some text for chapter 2:
%\iffalse
%<*samplechap2>
%\fi
%    \begin{macrocode}
\section{two}
more text in chapter two
%    \end{macrocode}

%\iffalse
%</samplechap2>
%\fi
%
% %%%%%%%%%%%%%%%%%%%%%%%%%%%%%%%%%%%%%%
% \paragraph{Part Include Files.}
%
% The include files are called |cdocspt3.tex| and |cdocspt4.tex|.
%
%\iffalse
%<*samplepart3|samplepart4>
%\fi

% Optional override for |\version| flag:
%    \begin{macrocode}
%%\providecommand{\version}{final}
%    \end{macrocode}

% Include the main document:
%    \begin{macrocode}
\input{childdoc.def}
\childdocby{cdocsamp}
%    \end{macrocode}

%\iffalse
%</samplepart3|samplepart4>
%\fi
%
%\iffalse
%<*samplepart3>
%\fi
% Some text for part 3:
%    \begin{macrocode}
some text in part three
%    \end{macrocode}

%\iffalse
%</samplepart3>
%\fi
% Some text for part 4:
%\iffalse
%<*samplepart4>
%\fi
%    \begin{macrocode}
more text in part four
%    \end{macrocode}

%\iffalse
%</samplepart4>
%\fi
%
% %%%%%%%%%%%%%%%%%%%%%%%%%%%%%%%%%%%%%%
% \paragraph{Forwarding for a Complete Draft.}
%
% The following forwarding file |cdocsdrf.tex|
% compiles the main document in draft mode:
%\iffalse
%<*sampledraft>
%\fi
%    \begin{macrocode}
\def\version{draft}
\input{childdoc.def}
\childdocforward{cdocsamp}
%    \end{macrocode}

%\iffalse
%</sampledraft>
%\fi
%
% %%%%%%%%%%%%%%%%%%%%%%%%%%%%%%%%%%%%%%
% \paragraph{Forwarding for Final Version of the Chapters.}
%
% The following forwarding files |cdocsfn1.tex| and |cdocsfn2.tex|
% (with identical content)
% compile the final versions of the child documents
% |cdocsch1.tex| and |cdocsch2.tex|, respectively:
%\iffalse
%<*samplefinal>
%\fi
%    \begin{macrocode}
\def\version{final}
\input{childdoc.def}
\childdocforwardprefix[cdocsamp]{cdocsfn}{cdocsch}
%    \end{macrocode}

%\iffalse
%</samplefinal>
%\fi
%
% %%%%%%%%%%%%%%%%%%%%%%%%%%%%%%%%%%%%%%
% \paragraph{Command Line Processing.}
%
% The following three command lines generate the output files
% |cdocscld|, |cdocscl1| and |cdocscl2|
% which should be identical to
% |cdocsdrf|, |cdocsch1| and |cdocsfn2|, respectively:
% \begin{center}
% \begin{tabular}{l}
% |latex -jobname cdocscld \|\\
% |  "\def\version{draft}\input{childdoc.def}\childdocforward{cdocsamp}"|\\
% |latex -jobname cdocscl1 \|\\
% |  "\input{childdoc.def}\childdocforward[cdocsamp]{cdocsch1}"|\\
% |latex -jobname cdocscl2 \|\\
% |  "\def\version{final}\input{childdoc.def}\childdocforward{cdocsch2}"|
% \end{tabular}
% \end{center}
% Note that the trailing backslash on each first line
% merely continues the input to the second line
% (for convenient cut ant paste).
% Furthermore, the command |latex| can be replaced by any
% of its alternative versions such as |pdflatex|.
%
% %%%%%%%%%%%%%%%%%%%%%%%%%%%%%%%%%%%%%%%%%%%%%%%%%%%%%%%%%%%%%%%%%%%%%%%%%%%%%%
% %%%%%%%%%%%%%%%%%%%%%%%%%%%%%%%%%%%%%%%%%%%%%%%%%%%%%%%%%%%%%%%%%%%%%%%%%%%%%%
% \section{Implementation}
%\iffalse
%<*package>
%\fi
%
% This section describes the definitions file |childdoc.def|.

% The definitions cannot be loaded using |\usepackage| or |\RequirePackage|
% which has a mechanism to prevent loading a style file more than once.
% When loading the definitions by means of |\input|
% multiple instances have to be prevented manually:
%\iffalse
%This code needs to be before the `\ProvidesFile' directive
%which is defined at the beginning of this file.
%Therefore it is also placed there and commented out here.
%</package>
%<*discard>
%\fi
%    \begin{macrocode}
\ifdefined\childdocmain\endinput\fi
%    \end{macrocode}
%\iffalse
%</discard>
%<*package>
%\fi
%
% \macro{\ifchilddoc}
% \macro{\ifchilddocmanual}
% The conditional |\ifchilddoc| tells whether a
% child (true) or main (false) document is being compiled.
% The conditional |\ifchilddocmanual| tells whether
% the |\includeonly| mechanism is used (false) or
% the selection of child files must be performed manually (true).
% The definitions initialise to false:
%    \begin{macrocode}
\newif\ifchilddoc
\newif\ifchilddocmanual
%    \end{macrocode}

% \macro{\childdocname}
% \macro{\childdocjob}
% The macro |\childdocname| stores the name of the main document
% to be compiled. The macro |\childdocjob| stores the name of
% the document on which the \LaTeX{} compiler was originally invoked.
% The content of |\jobname| cannot be compared
% to filenames specified in the source due to different catcodes.
% The following code rescans |\jobname|, stores the result
% in |\childdocname| and saves a copy in |\childdocjob|:
%    \begin{macrocode}
\edef\childdocname{\scantokens\expandafter{\jobname\noexpand}}
\let\childdocjob\childdocname
%    \end{macrocode}

% \macro{\childdocdisable}
% The macro |\childdocdisable| prevents the main file
% from being processed more than once.
% At this stage, the main document command |\childdocmain|
% is assumed to be called once again where it should do nothing.
% Any subsequent call to it should prevent
% a secondary processing of the main document
% It overwrites the forwarding commands
% |\childdocof| and |\childdocforward|
% with empty macros to prevent further inclusions of the main document:
%    \begin{macrocode}
\newcommand{\childdocdisable}
{
  \renewcommand{\childdocmain}[1]{\renewcommand{\childdocmain}[1]{\endinput}}
  \renewcommand{\childdocof}[1]{}
  \renewcommand{\childdocby}[2][]{}
  \renewcommand{\childdocforward}[2][]{}
  \renewcommand{\childdocdisable}{}
}
%    \end{macrocode}

% \macro{\childdocmain}
% The macro |\childdocmain| is to be called at the top of the main file
% with nothing or the main filename (without extension) as argument.
% First, it breaks loops.
% If the argument is not empty and does not match |\childdocname|
% (which is set by the first inclusion of |childdoc.def|),
% |\ifchilddoc| is set to true, |\includeonly| is applied to the child file
% and |\jobname| is set to the main file
% (for proper handling of |.aux| files):
%    \begin{macrocode}
\newcommand{\childdocmain}[1]
{
  \childdocdisable\childdocmain{}
  \if?#1?\else
    \begingroup
      \def\childdoctmp{#1}
      \ifx\childdoctmp\childdocname
        \def\childdoctmp{}
      \else
        \def\childdoctmp
        {
          \childdoctrue
          \includeonly{\childdocname}
          \def\childdocjob{#1}
          \def\jobname{#1}
        }
      \fi
      \expandafter
    \endgroup
    \childdoctmp
  \fi
}
%    \end{macrocode}

% \macro{\childdocof}
% The command |\childdocof| redirects
% compilation to the main file |#1|.
%    \begin{macrocode}
\newcommand{\childdocof}[1]
{
  \childdocdisable
  \childdoctrue
  \includeonly{\childdocname}
  \def\jobname{#1}
  \def\childdocjob{#1}
  \input{#1}
}
%    \end{macrocode}

% \macro{\childdocby}
% The command |\childdocby| ....
%    \begin{macrocode}
\newcommand{\childdocby}[2][]
{
  \childdocdisable
  \childdoctrue
  \childdocmanualtrue
  \if?#1?\else
    \def\jobname{#2}
  \fi
  \def\childdocjob{#2}
  \input{#2}
  \endinput
}
%    \end{macrocode}

% \macro{\childdocforward}
% The command |\childdocforward| redirects
% compilation to the main file or
% (if the optional argument is given) a child file.
% Parameters are set as if the main file
% or a child file starting with |\childdocof| was compiled.
% Then compilation is handed over to the main file:
%    \begin{macrocode}
\newcommand{\childdocforward}[2][]
{
  \begingroup
    \if?#1?
      \def\childdoctmp
      {
        \def\childdocname{#2}
        \def\childdocjob{#2}
        \def\jobname{#2}
        \input{#2}
        \endinput
      }
    \else
      \def\childdoctmp
      {
        \childdocdisable
        \def\childdocname{#2}
        \childdoctrue
        \includeonly{#2}
        \def\childdocjob{#1}
        \def\jobname{#1}
        \input{#1}
        \endinput
      }
    \fi
    \expandafter
  \endgroup
  \childdoctmp
}
%    \end{macrocode}

% \macro{\childdocforwardprefix}
% The command |\childdocforwardprefix| redirects
% compilation to the main or a child file by means of a pattern.
% The prefix |#1| in the current filename is replaced by |#2|
% and the suffix of the current filename is kept
% (it is assumed that the filename does not contain the substring `|~~~|'
% which is used as a delimiter).
% Compilation is handed over to the new file by |\childdocforward|:
%    \begin{macrocode}
\newcommand{\childdocforwardprefix}[3][]
{
  \begingroup
    \def\childdocextract #2##1~~~{\def\childdoctmp{\childdocforward[#1]{#3##1}}}
    \expandafter\childdocextract\childdocname~~~
    \expandafter
  \endgroup
  \childdoctmp
}
%    \end{macrocode}

% \macro{\childdoc}
% The deprecated macro |\childdoc| is a legacy version of |\childdocmain|:
%    \begin{macrocode}
\newcommand{\childdoc}{\childdocmain}
%    \end{macrocode}

% \macro{\childdocredirect}
% The deprecated macro |\childdocredirect| is a legacy version
% of |\childdocforward| and |\childdocforwardprefix|:
%    \begin{macrocode}
\newcommand{\childdocredirect}[2][]
{
  \begingroup
    \if?#1?
      \def\childdoctmp{\childdocforward{#2}}
    \else
      \def\childdoctmp{\childdocforwardprefix{#1}{#2}}
    \fi
    \expandafter
  \endgroup
  \childdoctmp
}
%    \end{macrocode}

%\iffalse
%</package>
%\fi
%
\endinput

\childdocby{cdocsamp}
%    \end{macrocode}

%\iffalse
%</samplepart3|samplepart4>
%\fi
%
%\iffalse
%<*samplepart3>
%\fi
% Some text for part 3:
%    \begin{macrocode}
some text in part three
%    \end{macrocode}

%\iffalse
%</samplepart3>
%\fi
% Some text for part 4:
%\iffalse
%<*samplepart4>
%\fi
%    \begin{macrocode}
more text in part four
%    \end{macrocode}

%\iffalse
%</samplepart4>
%\fi
%
% %%%%%%%%%%%%%%%%%%%%%%%%%%%%%%%%%%%%%%
% \paragraph{Forwarding for a Complete Draft.}
%
% The following forwarding file |cdocsdrf.tex|
% compiles the main document in draft mode:
%\iffalse
%<*sampledraft>
%\fi
%    \begin{macrocode}
\def\version{draft}
% \iffalse
%
% childdoc.dtx Copyright (C) 2017-2018 Niklas Beisert
%
% This work may be distributed and/or modified under the
% conditions of the LaTeX Project Public License, either version 1.3
% of this license or (at your option) any later version.
% The latest version of this license is in
%   http://www.latex-project.org/lppl.txt
% and version 1.3 or later is part of all distributions of LaTeX
% version 2005/12/01 or later.
%
% This work has the LPPL maintenance status `maintained'.
%
% The Current Maintainer of this work is Niklas Beisert.
%
% This work consists of the files childdoc.dtx and childdoc.ins
% and the derived files childdoc.def and cdocsamp.tex with
% cdocsch1.tex, cdocsch2.tex, cdocsdrf.tex, cdocsfn1.tex, cdocsfn2.tex.
%
%<package>\ifdefined\childdocmain\endinput\fi
%<package>\ProvidesFile{childdoc.def}[2018/12/30 v2.0 child document driver]
%<samplemain>\ProvidesFile{cdocsamp.tex}[2018/12/30 v2.0 sample for childdoc]
%<*driver>
%\ProvidesFile{childdoc.drv}[2018/12/30 v2.0 childdoc reference manual file]
\PassOptionsToClass{10pt,a4paper}{article}
\documentclass{ltxdoc}

\usepackage[margin=35mm]{geometry}
\usepackage{hyperref}
\usepackage{hyperxmp}
\usepackage[usenames]{color}

\hypersetup{colorlinks=true}
\hypersetup{pdfstartview=FitH}
\hypersetup{pdfpagemode=UseNone}
\hypersetup{pdfsource={}}
\hypersetup{pdflang={en-UK}}
\hypersetup{pdfcopyright={Copyright 2017-2018 Niklas Beisert.
  This work may be distributed and/or modified under the
  conditions of the LaTeX Project Public License, either version 1.3
  of this license or (at your option) any later version.}}
\hypersetup{pdflicenseurl={http://www.latex-project.org/lppl.txt}}
\hypersetup{pdfcontactaddress={ETH Zurich, ITP, HIT K,
  Wolfgang-Pauli-Strasse 27}}
\hypersetup{pdfcontactpostcode={8093}}
\hypersetup{pdfcontactcity={Zurich}}
\hypersetup{pdfcontactcountry={Switzerland}}
\hypersetup{pdfcontactemail={nbeisert@itp.phys.ethz.ch}}
\hypersetup{pdfcontacturl={http://people.phys.ethz.ch/\xmptilde nbeisert/}}

\newcommand{\secref}[1]{\hyperref[#1]{section \ref*{#1}}}

\parskip1ex
\parindent0pt
\let\olditemize\itemize
\def\itemize{\olditemize\parskip0pt}

\begin{document}

\title{The \textsf{childdoc} Package}
\hypersetup{pdftitle={The childdoc Package}}
\author{Niklas Beisert\\[2ex]
  Institut f\"ur Theoretische Physik\\
  Eidgen\"ossische Technische Hochschule Z\"urich\\
  Wolfgang-Pauli-Strasse 27, 8093 Z\"urich, Switzerland\\[1ex]
  \href{mailto:nbeisert@itp.phys.ethz.ch}
  {\texttt{nbeisert@itp.phys.ethz.ch}}}
\hypersetup{pdfauthor={Niklas Beisert}}
\hypersetup{pdfsubject={Manual for the LaTeX2e Package childdoc}}
\date{30 December 2018, \textsf{v2.0}}
\maketitle

\begin{abstract}\noindent
\textsf{childdoc} is a \LaTeXe{} package
that enables the direct compilation
of document sections included by |\include|
to individual files.
\end{abstract}

\begingroup
\parskip0ex
\tableofcontents
\endgroup

%%%%%%%%%%%%%%%%%%%%%%%%%%%%%%%%%%%%%%%%%%%%%%%%%%%%%%%%%%%%%%%%%%%%%%%%%%%%%%%%
%%%%%%%%%%%%%%%%%%%%%%%%%%%%%%%%%%%%%%%%%%%%%%%%%%%%%%%%%%%%%%%%%%%%%%%%%%%%%%%%
\section{Introduction}

\LaTeX{} provides a mechanism to structure a large document (such as a book)
into a main file and several child files (containing the chapters)
using the |\include| command.
This mechanism is beneficial for documents
which span hundreds of pages in order to
make the source file(s) more manageable.
Moreover, compilation can be restricted to
selected child files by means of the |\includeonly| command.
The latter feature can be used to reduce the compilation time while editing
(this was significantly more useful in the earlier days of \LaTeX{})
or to generate a smaller document which is easier to navigate.
Another application of |\includeonly| is to generate
documents consisting of selected parts of the complete document.

However, there are a few drawbacks of the plain |\include| mechanism:
\begin{itemize}
\item
The child files cannot be compiled on their own,
they can only be compiled via the main file.
A naive editing environment
(such as a text editor with an option
to have the current file processed by \LaTeX)
may require one to switch to the main file before compiling;
attempting to compile the child file produces errors.
\item
The main file must be modified (each time)
to adjust the |\includeonly| command
to the present needs. This easily leaves the main file in a messy state.
\item
The generated document will always carry the filename
of the main document. This is inconvenient if
several child files are to be compiled and
to be kept for distribution.
\end{itemize}

The present package provides a simple interface
to make child files individually compilable by \LaTeX{}.
Compiling a child file then has the same effect as compiling
the main file with an |\includeonly| command
to select the appropriate child.
Moreover the generated document will carry the name of the child
rather than the main file.
This resolves all three above issues.

This feature is meant to make the editing of books,
thesis documents and lecture notes somewhat more convenient.
However, the package can also be used efficiently for
composing a series of documents (such as exercise sheets)
which are typically distributed individually.
It then assists the author in generating the individual documents
(potentially in different versions)
as well as a document containing the collected series.
Another application is in developing style files
or other kinds of included material
where compilation of the style file could redirect
to a sample or test file.

%%%%%%%%%%%%%%%%%%%%%%%%%%%%%%%%%%%%%%%%%%%%%%%%%%%%%%%%%%%%%%%%%%%%%%%%%%%%%%%%
%%%%%%%%%%%%%%%%%%%%%%%%%%%%%%%%%%%%%%%%%%%%%%%%%%%%%%%%%%%%%%%%%%%%%%%%%%%%%%%%
\section{Usage}

First of all, the package \textsf{childdoc} is \emph{not} a standard
\LaTeXe{} |.sty| style file! Therefore it needs to be invoked in
a non-standard way.

%%%%%%%%%%%%%%%%%%%%%%%%%%%%%%%%%%%%%%%%%%%%%%%%%%%%%%%%%%%%%%%%%%%%%%%%%%%%%%%%
\subsection{Included Files}
\label{sec:include}

%%%%%%%%%%%%%%%%%%%%%%%%%%%%%%%%%%%%%%%%
\DescribeMacro{\childdocmain}
To use the package, add the commands
\begin{center}
\begin{tabular}{l}
|\input{childdoc.def}|\\
|\childdocmain{}|\\
\end{tabular}
\end{center}
at the very top of the main \LaTeX{} file,
in particular \emph{before} the |\documentclass| statement!
The argument of |\childdocmain| should be left empty
(but it must be present).

%%%%%%%%%%%%%%%%%%%%%%%%%%%%%%%%%%%%%%%%
\DescribeMacro{\childdocof}
Furthermore, add the commands
\begin{center}
\begin{tabular}{l}
|\input{childdoc.def}|\\
|\childdocof{|\textit{main}|}|\\
\end{tabular}
\end{center}
at the top of every child file \textit{child}
which is included by |\include{|\textit{child}|}|
from within the main file
(or at least for those files to be compiled individually).
The argument \textit{main} must be the filename of the main file.

There are a couple of
considerations in setting up the main and child documents:

%%%%%%%%%%%%%%%%%%%%%%%%%%%%%%%%%%%%%%%%
\paragraph{Restrictions.}

Please note the following restrictions:
\begin{itemize}
\item
|\childdocmain| must be called with one argument \textit{main}
to ensure compatibility with earlier version of the package.
It must either be empty (|\childdocmain{}|)
or precisely match the filename of the main file in which it is specified.
See \secref{sec:detection} for further information.
\item
The filename \textit{main} must be specified without the |.tex| extension.
\item
The filename \textit{main} is case sensitive
(even in case-insensitive file systems)
due to internal string comparison.
\item
The argument \textit{main} should be fully expanded, it cannot be a macro.
\item
Subdirectories and special characters should be avoided in filenames.
\item
The command |\childdocmain{|\textit{main}|}| must be followed by a whitespace.
It should not be followed immediately by another command
or by a comment mark `|%|'.
This is because the \TeX{} parser reads the token immediately following
the argument of |\childdocmain| and puts it
at the beginning of every child section;
however, a white\-space is ignored.
\end{itemize}

%%%%%%%%%%%%%%%%%%%%%%%%%%%%%%%%%%%%%%%%
\paragraph{Content of Main File.}

It is advisable to place all content in the child files included by |\include|.
Any output contained in the main file will appear in all child documents
unless suppressed manually;
it cannot be suppressed automatically by the |\includeonly| directive
and thus should normally be avoided.
A method to include some content in the main file
by means of conditional processing is described in \secref{sec:conditional}.

%%%%%%%%%%%%%%%%%%%%%%%%%%%%%%%%%%%%%%%%
\paragraph{Page Numbering.}

When only a part of the document is compiled,
the appropriate numbering of pages
(as well as other status parameters)
is determined from the |.aux| files.
The latter contain information from previous passes.
However this information needs to propagate through
all intermediate child documents.
Therefore the page numbering in child documents may well
be inconsistent until the complete document is compiled at least once.

A useful (if unconventional) way to always ensure a consistent
page numbering is to restart the numbering in each child document
and denote the pages by `\textit{child}|.|\textit{page}'
where \textit{child} represents the chapter/section number of the child file.
This can be achieved by the command
|\numberwithin{page}{|\textit{child}|}|
of the \textsf{amsmath} package
where \textit{child} can be |chapter| or |section|
depending on the chosen structuring.
Alternatively, one can modify the macro |\thepage| appropriately
and reset the counter |page| at the start of each child file.

%%%%%%%%%%%%%%%%%%%%%%%%%%%%%%%%%%%%%%%%%%%%%%%%%%%%%%%%%%%%%%%%%%%%%%%%%%%%%%%%
\subsection{Conditional Processing}
\label{sec:conditional}

The package provides a mechanism to compile different versions
of a document. To customise the versions further some conditional processing
can come in handy to distinguish which version is being compiled.
The package provides two macros to describe the compilation context:

%%%%%%%%%%%%%%%%%%%%%%%%%%%%%%%%%%%%%%%%
\DescribeMacro{\ifchilddoc}
The conditional |\ifchilddoc| distinguishes between the compilation of
child documents and the main document:
%
\begin{center}
|\ifchilddoc |\textit{child-code}| |[|\||else |\textit{main-code}]| \||fi|
\end{center}

%%%%%%%%%%%%%%%%%%%%%%%%%%%%%%%%%%%%%%%%
\DescribeMacro{\childdocname}
\DescribeMacro{\childdocjob}
The macro |\childdocname| contains the filename (without extension)
of the main or child file being processed.
Note that |\childdocjob| will always contain the name of the main file.

%%%%%%%%%%%%%%%%%%%%%%%%%%%%%%%%%%%%%%%%
\paragraph{Title Page.}

Conditional processing can be used to include a title or banner page
in the main document when proper precautions are taken.
Importantly, the code in the main file should ensure that the page counter
(as well as other status parameters which are stored in the |.aux| files)
takes the same value after the conditional processing.
Otherwise the page numbers may take divergent values
depending on which part is compiled.

For example, a title page could be declared by:
%
\begin{center}
\begin{tabular}{l}
|\ifchilddoc\||else|\\
|\addtocounter{page}{-1}|\\
\textit{code for title page}\\
|\newpage|\\
|\||fi|
\end{tabular}
\end{center}
%
A banner page for the child documents can be generated by:
%
\begin{center}
\begin{tabular}{l}
|\ifchilddoc|\\
|\addtocounter{page}{-1}|\\
\textit{code for banner page}\\
|\newpage|\\
|\||fi|
\end{tabular}
\end{center}
%
Here one could write a message such as:
\begin{center}
|This is the part \childdocname{} of \childdocjob{}.|
\end{center}

%%%%%%%%%%%%%%%%%%%%%%%%%%%%%%%%%%%%%%%%%%%%%%%%%%%%%%%%%%%%%%%%%%%%%%%%%%%%%%%%
\subsection{Flags}
\label{sec:flags}

The package makes it easy to generate different versions
of the main or child documents.
To this end compilation flags can be defined
and assigned different default values.
They will be particularly useful in conjunction
with the forwarding mechanism described in \secref{sec:forward}.

For example, it may be useful to have a flag |\version|
which can be set to |draft| or |final|.
The document source will contain some conditional code
depending on the value of |\version|.
Suppose further, the flag should default to |final| for the main file
and to |draft| for child files
which is a natural assignment for editing the document.
This is achieved by placing the following code
in the preamble of the main document
(below the |\childdocmain| directive):
%
\begin{center}
\begin{tabular}{l}
|\ifchilddoc|\\
|\providecommand{\version}{draft}|\\
|\||else|\\
|\providecommand{\version}{final}|\\
|\||fi|
\end{tabular}
\end{center}
%
The definition by |\providecommand| makes sure
that previous definitions are not overwritten.
Further statements |\providecommand{\version}{...}|
can thus be added before the above code to override it.

For the main file, one might add a line
(between |\childdocmain| and the above block)
%
\begin{center}
|%\ifchilddoc\||else\providecommand{\version}{draft}\||fi|
\end{center}
%
which can be uncommented to produce a draft version.
Likewise one can add a line to the very top of a child file
(above the |\childdocof{|\textit{main}|}| directive)
%
\begin{center}
|%\providecommand{\version}{final}|
\end{center}
%
which can be uncommented to produce the final version of this child document.

%%%%%%%%%%%%%%%%%%%%%%%%%%%%%%%%%%%%%%%%%%%%%%%%%%%%%%%%%%%%%%%%%%%%%%%%%%%%%%%%
\subsection{Forwarding}
\label{sec:forward}

Different versions of the main or child documents
using compilation flags as described in \secref{sec:flags}
can be (permanently) stored in different files
for convenient compilation, viewing and distribution.
To this end, the package defines a command
to pass on compilation to a different file:

%%%%%%%%%%%%%%%%%%%%%%%%%%%%%%%%%%%%%%%%
\DescribeMacro{\childdocforward}
The command |\childdocforward| redirects processing to
another source file:
%
\begin{center}
\begin{tabular}{l}
|\input{childdoc.def}|\\
|\childdocforward[|\textit{main}|]{|\textit{dest}|}|\\
\end{tabular}
\end{center}
%
The argument \textit{dest} is the destination file
(without extension).
It should be the main file or one of the child files.
Note that further \textsf{childdoc} directives
such as |\childdocof| and |\childdocforward|
in the indicated file will be processed in this form.
The optional argument \textit{main}
passes on directly to the main file \textit{main}
while pretending to compile the child \textit{dest}.
This form behaves as if \textit{dest}
issues |\childdocof{|\textit{main}|}| right away,
and no further \textsf{childdoc} directives will be processed.

%%%%%%%%%%%%%%%%%%%%%%%%%%%%%%%%%%%%%%%%
\DescribeMacro{\...prefix}
In the alternative form |\childdocforwardprefix|,
%
\begin{center}
\begin{tabular}{l}
|\input{childdoc.def}|\\
|\childdocforwardprefix[|\textit{main}|]{|\textit{prefix}|}{|\textit{dest}|}|
\end{tabular}
\end{center}
%
the destination file is determined by a pattern
depending on the current file:
To make this work, the current file must be called
`{\textit{prefix}\hspace{0.2em}\textit{suffix}}'
with \textit{prefix} matching precisely the argument.
Processing is then passed on to the file
`{\textit{dest}\hspace{0.2em}\textit{suffix}}'.
Surely, the same effect is achieved by
directly specifying the
argument `{\textit{dest}\hspace{0.2em}\textit{suffix}}'
in the first form.
However, that requires to set up a different file
for each child. With the alternative form of the command
all these files can have exactly the same content
which simplifies setting them up and maintaining them.

For example, the following file |draft.tex|
with a compilation flag |\version| as described in \secref{sec:flags}
compiles the main document as a draft:
%
\begin{center}
\begin{tabular}{l}
|\def\version{draft}|\\
|\input{childdoc.def}|\\
|\childdocforward{|\textit{main}|}|
\end{tabular}
\end{center}
%
Likewise, the following files |final|\textit{nn}|.tex|
compile the final version of the child document
|child|\textit{nn}|.tex|:
%
\begin{center}
\begin{tabular}{l}
|\def\version{final}|\\
|\input{childdoc.def}|\\
|\childdocforwardprefix{final}{child}|
\end{tabular}
\end{center}
%

Note that when several versions of a main file and/or of each child file
are to be generated, it may be convenient to set up a |Makefile| or
shell script to automatise the process.

%%%%%%%%%%%%%%%%%%%%%%%%%%%%%%%%%%%%%%%%%%%%%%%%%%%%%%%%%%%%%%%%%%%%%%%%%%%%%%%%
\subsection{Command Line Processing}
\label{sec:commandline}

The effect of redirection files can also be achieved by invoking
the \LaTeX{} compiler with a more elaborate command line.
Most conveniently this should be done as part
of a shell script or a |Makefile|.

When using \textsf{childdoc} in the main file, the following
command lines effectively perform a redirection
(note that depending on the shell being used,
backslashes may have to be doubled: `|\|' $\to$ `|\\|'):
%
\begin{center}
|... -jobname "|\textit{target}|" |\\|"|[\textit{flags}]%
|\input{childdoc.def}\childdocforward[|\textit{main}|]{|\textit{dest}|}"|
\end{center}
%
Here \textit{target} is the name of the output file,
\textit{main} is the name of the main file
and \textit{dest} is the name of the main or child file to be processed
(all filenames without extensions).
The optional argument \textit{main} can be omitted
if \textit{main} matches \textit{dest}.
Optionally, compilation \textit{flags} can be defined via |\def| commands.
This command line makes the \TeX{} engine believe
it is compiling the file \textit{target}
whose content is specified as the latter parameter.
The provided code then forwards the processing to
\textit{main} or \textit{dest} as described in \secref{sec:forward}.

%%%%%%%%%%%%%%%%%%%%%%%%%%%%%%%%%%%%%%%%%%%%%%%%%%%%%%%%%%%%%%%%%%%%%%%%%%%%%%%%
\subsection{Include by Input}
\label{sec:input}

Including child documents by |\include| has some restrictions by design.
Most notably, the content of a child document always occupies
its own set of pages; pages cannot be shared between child documents.
Usually, this behaviour makes perfect sense
because each child document contain an essential part of the document.
However, in some situations it may be desirable to compose
a document from a collection of parts
without having mandatory page breaks between then.
For this case, the package
provides a mechanism to include parts
by |\input| which can also be processed individually.
However, by construction this mechanism
requires manual handling of the content to be output.

%%%%%%%%%%%%%%%%%%%%%%%%%%%%%%%%%%%%%%%%
\DescribeMacro{\ifchilddocmanual}
The main file should be prepared as usual, see \secref{sec:include}.
However, the document body must make a distinction
between processing of an individual part and of the main document, e.g.:
%
\begin{center}
\begin{tabular}{l}
|\ifchilddocmanual|\\
|\input{\childdocname}|\\
|\||else|\\
\textit{document body with }|\input{|\textit{part}|}|\\
|\||fi|
\end{tabular}
\end{center}
%
The conditional |\ifchilddocmanual| is true whenever
a part to be included by |\input| is being compiled,
and the name of the part is stored in |\childdocname|.

%%%%%%%%%%%%%%%%%%%%%%%%%%%%%%%%%%%%%%%%
\DescribeMacro{\childdocby}
Each part to be included by |\input| should start with:
%
\begin{center}
\begin{tabular}{l}
|\input{childdoc.def}|\\
|\childdocby{|\textit{main}|}|\\
\end{tabular}
\end{center}
%
The directive |\childdocby| is similar to |\childdocof|
described in \secref{sec:include},
but the subsequent selection of content must be done manually.
To that end, both |\ifchilddoc| and |\ifchilddocmanual|
will be true upon processing of a part,
and the name of the part is stored in |\childdocname|.
Note that |\jobname| will be set to the filename of the current part
so that each part receives an individual |.aux| file
that does not interfere with the |.aux| file(s) of the main document.
This behaviour can be altered by the alternative form
|\childdocby[*]{|\textit{main}|}| (with a non-empty optional argument)
which uses the |.aux| file of the main document
by setting |\jobname| to \textit{main}.

%%%%%%%%%%%%%%%%%%%%%%%%%%%%%%%%%%%%%%%%%%%%%%%%%%%%%%%%%%%%%%%%%%%%%%%%%%%%%%%%
\subsection{Driver Development}
\label{sec:driver}

The \textsf{childdoc} mechanism can also be use for the development
of definition files such as \LaTeX{} styles or classes.
This case differs from the above setup with multiple parts
included by |\include| in that no |\includeonly| should be invoked.
This can be achieved by starting the include file
(before |\ProvidesPackage|) with:
%
\begin{center}
\begin{tabular}{l}
|\input{childdoc.def}|\\
|\childdocforward{|\textit{main}|}|\\
\end{tabular}
\end{center}
%
or alternatively with:
%
\begin{center}
\begin{tabular}{l}
|\input{childdoc.def}|\\
|\childdocby{|\textit{main}|}|\\
\end{tabular}
\end{center}
%
Both forms have slightly different effects as described above.
The main file is prepared as usual, see \secref{sec:include}.

%%%%%%%%%%%%%%%%%%%%%%%%%%%%%%%%%%%%%%%%%%%%%%%%%%%%%%%%%%%%%%%%%%%%%%%%%%%%%%%%
\subsection{Legacy Detection}
\label{sec:detection}

The directive |\childdocmain| in the main file can detect
whether the complete document or merely a child is to be compiled
even without using the directive |\childdocof|.
This method is deprecated because it is less robust
and there is no compelling reason to use it;
it is merely provided for backward compatibility
and it may be removed in future versions.

If the detection mechanism is to be used,
it is mandatory to correctly specify
the filename of the main file as the argument of |\childdocmain|:
%
\begin{center}
\begin{tabular}{l}
|\input{childdoc.def}|\\
|\childdocmain{|\textit{main}|}|\\
\end{tabular}
\end{center}
%
If |\jobname| does not match the argument \textit{main} of |\childdocmain|,
it is assumed that |\jobname| points to the child file to be compiled.
When using |\childdocmain| with the main file specified as argument,
it suffices to start a child file
with just |\input{|\textit{main}|}|
without loading of the package and using |\childdocof|.
If instead all processing is done
with the appropriate \textsf{childdoc} directives,
the argument of \textit{main} of |\childdocmain| can be empty.

An alternative version of the command line processing described
in \secref{sec:commandline} using the detection mechanism reads:
%
\begin{center}
|... -jobname "|\textit{target}|" "|[\textit{flags}]%
[|\def\jobname{|\textit{dest}|}|]|\input{|\textit{main}|}"|
\end{center}

%%%%%%%%%%%%%%%%%%%%%%%%%%%%%%%%%%%%%%%%%%%%%%%%%%%%%%%%%%%%%%%%%%%%%%%%%%%%%%%%
\subsection{Manual Code}
\label{sec:manual}

In case one cannot be certain whether the definitions file |childdoc.def|
is installed on the target \TeX{} distribution
and one prefers not to ship it,
it is conceivable to paste a few relevant commands into the sources.

To that end, drop all statements |\input{childdoc.def}|
and perform the replacements as outlined below.
Instead of |\childdocmain{|\textit{main}|}| add the following code
to the top of the main file:
%
\begin{center}
\begin{tabular}{l}
|\||ifdefined\childdocname\endinput\||fi\newif\ifchilddoc|\\
|\edef\childdocname{\scantokens\expandafter{\jobname\noexpand}}|\\
|\def\childdocmain{|\textit{main}|}\||ifx\childdocmain\childdocname\||else|\\
|\childdoctrue\includeonly{\childdocname}\let\jobname\childdocmain\||fi|\\
\end{tabular}
\end{center}
%
Instead of |\childdocof{|\textit{main}|}| just include the main file
at the top of each child file:
%
\begin{center}
|\input{|\textit{main}|}|
\end{center}
%
A simple redirection |\childdocforward{|\textit{dest}|}| is achieved by:
%
\begin{center}
|\def\jobname{|\textit{dest}|}\input{\jobname}|
\end{center}
%
The redirection with prefix
|\childdocforwardprefix[|\textit{prefix}|]{|\textit{dest}|}|
is accomplished by:
%
\begin{center}
\begin{tabular}{l}
|{\edef\jobname{\scantokens\expandafter{\jobname\noexpand}}|\\
|\def\redirectjob |\textit{prefix}|#1~~~{\gdef\jobname{|\textit{dest}|#1}}|\\
|\expandafter\redirectjob\jobname~~~}\input{\jobname}|
\end{tabular}
\end{center}

In an alternative approach,
child documents can be compiled by a specific command line
without additional code or specific definitions:
%
\begin{center}
|... -jobname "|\textit{target}|" "|[\textit{flags}]%
|\includeonly{|\textit{dest}|}\input{|\textit{main}|}"|
\end{center}
%

%%%%%%%%%%%%%%%%%%%%%%%%%%%%%%%%%%%%%%%%%%%%%%%%%%%%%%%%%%%%%%%%%%%%%%%%%%%%%%%%
%%%%%%%%%%%%%%%%%%%%%%%%%%%%%%%%%%%%%%%%%%%%%%%%%%%%%%%%%%%%%%%%%%%%%%%%%%%%%%%%
\section{Information}

%%%%%%%%%%%%%%%%%%%%%%%%%%%%%%%%%%%%%%%%%%%%%%%%%%%%%%%%%%%%%%%%%%%%%%%%%%%%%%%%
\subsection{Copyright}

Copyright \copyright{} 2017--2018 Niklas Beisert

This work may be distributed and/or modified under the
conditions of the \LaTeX{} Project Public License, either version 1.3
of this license or (at your option) any later version.
The latest version of this license is in
  \url{http://www.latex-project.org/lppl.txt}
and version 1.3 or later is part of all distributions of \LaTeX{}
version 2005/12/01 or later.

This work has the LPPL maintenance status `maintained'.

The Current Maintainer of this work is Niklas Beisert.

This work consists of the files |README.txt|, |childdoc.ins| and |childdoc.dtx|
as well as the derived files |childdoc.def|, |cdocsamp.tex|
with |cdocsch1.tex|, |cdocsch2.tex|, |cdocspt3.tex|, |cdocspt4.tex|,
|cdocsdrf.tex|, |cdocsfn1.tex|, |cdocsfn2.tex|
as well as |childdoc.pdf|.

%%%%%%%%%%%%%%%%%%%%%%%%%%%%%%%%%%%%%%%%%%%%%%%%%%%%%%%%%%%%%%%%%%%%%%%%%%%%%%%%
\subsection{Files and Installation}

The package consists of the files:
%
\begin{center}
\begin{tabular}{ll}
    |README.txt|   & readme file \\
    |childdoc.ins| & installation file \\
    |childdoc.dtx| & source file \\
    |childdoc.def| & definition file \\
    |cdocsamp.tex| & sample main file \\
    |cdocsch1.tex| & sample include file \\
    |cdocsch2.tex| & sample include file \\
    |cdocspt3.tex| & sample part file \\
    |cdocspt4.tex| & sample part file \\
    |cdocsdrf.tex| & sample redirection file \\
    |cdocsfn1.tex| & sample redirection file \\
    |cdocsfn2.tex| & sample redirection file \\
    |childdoc.pdf| & manual
\end{tabular}
\end{center}
%
The distribution consists of the files
|README.txt|, |childdoc.ins| and |childdoc.dtx|.
%
\begin{itemize}
\item
Run (pdf)\LaTeX{} on |childdoc.dtx|
to compile the manual |childdoc.pdf| (this file).
\item
Run \LaTeX{} on |childdoc.ins| to create the definitions file |childdoc.def|
and the sample |cdocsamp.tex| with include files
|cdocsch1.tex|, |cdocsch2.tex|, |cdocspt3.tex|, |cdocspt4.tex|,
|cdocsdrf.tex|, |cdocsfn1.tex|, |cdocsfn2.tex|.
Then copy the file |childdoc.def| to an appropriate directory of your \LaTeX{}
distribution, e.g.\ \textit{texmf-root}|/tex/latex/childdoc|.
\end{itemize}

%%%%%%%%%%%%%%%%%%%%%%%%%%%%%%%%%%%%%%%%%%%%%%%%%%%%%%%%%%%%%%%%%%%%%%%%%%%%%%%%
\subsection{Related CTAN Packages}

There are several other packages which offer a similar functionality:
%
\begin{itemize}
\item
The packages
\href{http://ctan.org/pkg/docmute}{\textsf{docmute}},
\href{http://ctan.org/pkg/includex}{\textsf{includex}} and
\href{http://ctan.org/pkg/standalone}{\textsf{standalone}}
provide commands to include only the document body of
a child file thus allowing both files to be compiled individually.
\item
The packages \href{http://ctan.org/pkg/subdocs}{\textsf{subdocs}}
and \href{http://ctan.org/pkg/subfiles}{\textsf{subfiles}}
provide structures in which the main and child documents can be
encapsulated and allowing them to be compiled individually.
The inclusion mechanism is different from the conventional |\include|.
\item
The package \href{http://ctan.org/pkg/combine}{\textsf{combine}}
is an elaborate solution to combine several documents into one.
\end{itemize}
%
See also the CTAN topic \href{http://ctan.org/topic/subdocs}{\textsf{subdocs}}
for further related packages.
The present package differs from the above solutions in that
a document structure constructed with the conventional |\include| mechanism
just needs two extra commands at the top of every file
such that all constituent files can be compiled individually.

%%%%%%%%%%%%%%%%%%%%%%%%%%%%%%%%%%%%%%%%%%%%%%%%%%%%%%%%%%%%%%%%%%%%%%%%%%%%%%%%
%\subsection{Feature Suggestions}
%
%The following is a list of features which may be useful for future
%versions of this package:
%%
%\begin{itemize}
%\item
%\ldots
%\end{itemize}

%%%%%%%%%%%%%%%%%%%%%%%%%%%%%%%%%%%%%%%%%%%%%%%%%%%%%%%%%%%%%%%%%%%%%%%%%%%%%%%%
\subsection{Revision History}

%%%%%%%%%%%%%%%%%%%%%%%%%%%%%%%%%%%%%%%%
\paragraph{v2.0:} 2018/12/30

\begin{itemize}
\item
immediate forward processing
\item
added |\childdocby| mechanism
\item
manual restructured
\end{itemize}

%%%%%%%%%%%%%%%%%%%%%%%%%%%%%%%%%%%%%%%%
\paragraph{v1.6:} 2018/01/17

\begin{itemize}
\item
application for development of include files
\item
corrections to manual
\end{itemize}

%%%%%%%%%%%%%%%%%%%%%%%%%%%%%%%%%%%%%%%%
\paragraph{v1.5:} 2017/05/21

\begin{itemize}
\item
more complete structuring introduced
\item
|\childdocof| introduced
\item
|\childdoc| renamed to |\childdocmain|
\item
|\childredirect| renamed to |\childdocforward| and |\childdocforwardprefix|
and functionality expanded
\end{itemize}

%%%%%%%%%%%%%%%%%%%%%%%%%%%%%%%%%%%%%%%%
\paragraph{v1.0:} 2017/04/27

\begin{itemize}
\item
manual and install package
\item
first version published on CTAN
\end{itemize}

%%%%%%%%%%%%%%%%%%%%%%%%%%%%%%%%%%%%%%%%
\paragraph{v0.6:} 2017/04/26

\begin{itemize}
\item
redirection mechanism added
\end{itemize}

%%%%%%%%%%%%%%%%%%%%%%%%%%%%%%%%%%%%%%%%
\paragraph{v0.5:} 2017/04/26

\begin{itemize}
\item
functionality in definition file
\end{itemize}


%%%%%%%%%%%%%%%%%%%%%%%%%%%%%%%%%%%%%%%%%%%%%%%%%%%%%%%%%%%%%%%%%%%%%%%%%%%%%%%%
%%%%%%%%%%%%%%%%%%%%%%%%%%%%%%%%%%%%%%%%%%%%%%%%%%%%%%%%%%%%%%%%%%%%%%%%%%%%%%%%
%%%%%%%%%%%%%%%%%%%%%%%%%%%%%%%%%%%%%%%%%%%%%%%%%%%%%%%%%%%%%%%%%%%%%%%%%%%%%%%%
\appendix

\settowidth\MacroIndent{\rmfamily\scriptsize 000\ }

 \DocInput{childdoc.dtx}

\end{document}
%</driver>
% \fi
%
% %%%%%%%%%%%%%%%%%%%%%%%%%%%%%%%%%%%%%%%%%%%%%%%%%%%%%%%%%%%%%%%%%%%%%%%%%%%%%%
% %%%%%%%%%%%%%%%%%%%%%%%%%%%%%%%%%%%%%%%%%%%%%%%%%%%%%%%%%%%%%%%%%%%%%%%%%%%%%%
% \section{Sample}
%\iffalse
%<*samplemain>
%\fi
%
% The following presents a sample document
% with two chapters, two parts, a title page,
% a compile flag as well as three forwarding files to set the flag.
% It consists of eight |.tex| files:
% \begin{center}
% \begin{tabular}{ll}
% |cdocsamp.tex|&main file\\
% |cdocsch1.tex|&include file for chapter 1\\
% |cdocsch2.tex|&include file for chapter 2\\
% |cdocspt3.tex|&include file for part 3\\
% |cdocspt4.tex|&include file for part 4\\
% |cdocsdrf.tex|&forwarding file for main file in draft mode\\
% |cdocsfi1.tex|&forwarding file for final version of chapter 1\\
% |cdocsfi2.tex|&forwarding file for final version of chapter 2\\
% \end{tabular}
% \end{center}
% Each of the eight files can be compiled directly by the \LaTeX{} compiler.
%
% %%%%%%%%%%%%%%%%%%%%%%%%%%%%%%%%%%%%%%
% \paragraph{Main File.}
%
% The main file is called |cdocsamp.tex|.
%
% Load the \textsf{childdoc} definitions and
% declare the filename for the main document:
%    \begin{macrocode}
\input{childdoc.def}
\childdocmain{}
%    \end{macrocode}

% Optional override for |\version| flag:
%    \begin{macrocode}
%%\ifchilddoc\else\providecommand{\version}{draft}\fi
%    \end{macrocode}

% Define the default values for the |\version| flag
% (|final| for the main file and |draft| for childs):
%    \begin{macrocode}
\ifchilddoc
\providecommand{\version}{draft}
\else
\providecommand{\version}{final}
\fi
%    \end{macrocode}

% Load the standard document class:
%    \begin{macrocode}
\documentclass[12pt]{article}
%    \end{macrocode}

% Start the document body:
%    \begin{macrocode}
\begin{document}
%    \end{macrocode}

% Declare a title page.
% Print title, part of document being processed and version flag:
%    \begin{macrocode}
\addtocounter{page}{-1}
\begin{center}
{\LARGE\bfseries{}childdoc example\par}
\vspace{1cm}
\ifchilddoc
\ifchilddocmanual part\else chapter\fi:
`\childdocname' of `\childdocjob'\par
\else
main document: `\childdocjob'\par
\fi
version: \version\par
\end{center}
\newpage
%    \end{macrocode}

% Manually include selected file,
% otherwise process as usual:
%    \begin{macrocode}
\ifchilddocmanual
\section*{part `\childdocname'}
\input{\childdocname}
\else
%    \end{macrocode}

% Include the two chapters:
%    \begin{macrocode}
\include{cdocsch1}
\include{cdocsch2}
%    \end{macrocode}

% Include the two parts unless only chapters should be displayed:
%    \begin{macrocode}
\ifchilddoc\else
\section{part three}
\input{cdocspt3}
\section{part four}
\input{cdocspt4}
\fi
%    \end{macrocode}

% Process as usual until here:
%    \begin{macrocode}
\fi
%    \end{macrocode}

% End of document body:
%    \begin{macrocode}
\end{document}
%    \end{macrocode}
%\iffalse
%</samplemain>
%\fi
%
% %%%%%%%%%%%%%%%%%%%%%%%%%%%%%%%%%%%%%%
% \paragraph{Chapter Include Files.}
%
% The include files are called |cdocsch1.tex| and |cdocsch2.tex|.
%
%\iffalse
%<*samplechap1|samplechap2>
%\fi

% Optional override for |\version| flag:
%    \begin{macrocode}
%%\providecommand{\version}{final}
%    \end{macrocode}

% Include the main document:
%    \begin{macrocode}
\input{childdoc.def}
\childdocof{cdocsamp}
%    \end{macrocode}

%\iffalse
%</samplechap1|samplechap2>
%\fi
%
%\iffalse
%<*samplechap1>
%\fi
% Some text for chapter 1:
%    \begin{macrocode}
\section{one}
some text in chapter one
%    \end{macrocode}

%\iffalse
%</samplechap1>
%\fi
% Some text for chapter 2:
%\iffalse
%<*samplechap2>
%\fi
%    \begin{macrocode}
\section{two}
more text in chapter two
%    \end{macrocode}

%\iffalse
%</samplechap2>
%\fi
%
% %%%%%%%%%%%%%%%%%%%%%%%%%%%%%%%%%%%%%%
% \paragraph{Part Include Files.}
%
% The include files are called |cdocspt3.tex| and |cdocspt4.tex|.
%
%\iffalse
%<*samplepart3|samplepart4>
%\fi

% Optional override for |\version| flag:
%    \begin{macrocode}
%%\providecommand{\version}{final}
%    \end{macrocode}

% Include the main document:
%    \begin{macrocode}
\input{childdoc.def}
\childdocby{cdocsamp}
%    \end{macrocode}

%\iffalse
%</samplepart3|samplepart4>
%\fi
%
%\iffalse
%<*samplepart3>
%\fi
% Some text for part 3:
%    \begin{macrocode}
some text in part three
%    \end{macrocode}

%\iffalse
%</samplepart3>
%\fi
% Some text for part 4:
%\iffalse
%<*samplepart4>
%\fi
%    \begin{macrocode}
more text in part four
%    \end{macrocode}

%\iffalse
%</samplepart4>
%\fi
%
% %%%%%%%%%%%%%%%%%%%%%%%%%%%%%%%%%%%%%%
% \paragraph{Forwarding for a Complete Draft.}
%
% The following forwarding file |cdocsdrf.tex|
% compiles the main document in draft mode:
%\iffalse
%<*sampledraft>
%\fi
%    \begin{macrocode}
\def\version{draft}
\input{childdoc.def}
\childdocforward{cdocsamp}
%    \end{macrocode}

%\iffalse
%</sampledraft>
%\fi
%
% %%%%%%%%%%%%%%%%%%%%%%%%%%%%%%%%%%%%%%
% \paragraph{Forwarding for Final Version of the Chapters.}
%
% The following forwarding files |cdocsfn1.tex| and |cdocsfn2.tex|
% (with identical content)
% compile the final versions of the child documents
% |cdocsch1.tex| and |cdocsch2.tex|, respectively:
%\iffalse
%<*samplefinal>
%\fi
%    \begin{macrocode}
\def\version{final}
\input{childdoc.def}
\childdocforwardprefix[cdocsamp]{cdocsfn}{cdocsch}
%    \end{macrocode}

%\iffalse
%</samplefinal>
%\fi
%
% %%%%%%%%%%%%%%%%%%%%%%%%%%%%%%%%%%%%%%
% \paragraph{Command Line Processing.}
%
% The following three command lines generate the output files
% |cdocscld|, |cdocscl1| and |cdocscl2|
% which should be identical to
% |cdocsdrf|, |cdocsch1| and |cdocsfn2|, respectively:
% \begin{center}
% \begin{tabular}{l}
% |latex -jobname cdocscld \|\\
% |  "\def\version{draft}\input{childdoc.def}\childdocforward{cdocsamp}"|\\
% |latex -jobname cdocscl1 \|\\
% |  "\input{childdoc.def}\childdocforward[cdocsamp]{cdocsch1}"|\\
% |latex -jobname cdocscl2 \|\\
% |  "\def\version{final}\input{childdoc.def}\childdocforward{cdocsch2}"|
% \end{tabular}
% \end{center}
% Note that the trailing backslash on each first line
% merely continues the input to the second line
% (for convenient cut ant paste).
% Furthermore, the command |latex| can be replaced by any
% of its alternative versions such as |pdflatex|.
%
% %%%%%%%%%%%%%%%%%%%%%%%%%%%%%%%%%%%%%%%%%%%%%%%%%%%%%%%%%%%%%%%%%%%%%%%%%%%%%%
% %%%%%%%%%%%%%%%%%%%%%%%%%%%%%%%%%%%%%%%%%%%%%%%%%%%%%%%%%%%%%%%%%%%%%%%%%%%%%%
% \section{Implementation}
%\iffalse
%<*package>
%\fi
%
% This section describes the definitions file |childdoc.def|.

% The definitions cannot be loaded using |\usepackage| or |\RequirePackage|
% which has a mechanism to prevent loading a style file more than once.
% When loading the definitions by means of |\input|
% multiple instances have to be prevented manually:
%\iffalse
%This code needs to be before the `\ProvidesFile' directive
%which is defined at the beginning of this file.
%Therefore it is also placed there and commented out here.
%</package>
%<*discard>
%\fi
%    \begin{macrocode}
\ifdefined\childdocmain\endinput\fi
%    \end{macrocode}
%\iffalse
%</discard>
%<*package>
%\fi
%
% \macro{\ifchilddoc}
% \macro{\ifchilddocmanual}
% The conditional |\ifchilddoc| tells whether a
% child (true) or main (false) document is being compiled.
% The conditional |\ifchilddocmanual| tells whether
% the |\includeonly| mechanism is used (false) or
% the selection of child files must be performed manually (true).
% The definitions initialise to false:
%    \begin{macrocode}
\newif\ifchilddoc
\newif\ifchilddocmanual
%    \end{macrocode}

% \macro{\childdocname}
% \macro{\childdocjob}
% The macro |\childdocname| stores the name of the main document
% to be compiled. The macro |\childdocjob| stores the name of
% the document on which the \LaTeX{} compiler was originally invoked.
% The content of |\jobname| cannot be compared
% to filenames specified in the source due to different catcodes.
% The following code rescans |\jobname|, stores the result
% in |\childdocname| and saves a copy in |\childdocjob|:
%    \begin{macrocode}
\edef\childdocname{\scantokens\expandafter{\jobname\noexpand}}
\let\childdocjob\childdocname
%    \end{macrocode}

% \macro{\childdocdisable}
% The macro |\childdocdisable| prevents the main file
% from being processed more than once.
% At this stage, the main document command |\childdocmain|
% is assumed to be called once again where it should do nothing.
% Any subsequent call to it should prevent
% a secondary processing of the main document
% It overwrites the forwarding commands
% |\childdocof| and |\childdocforward|
% with empty macros to prevent further inclusions of the main document:
%    \begin{macrocode}
\newcommand{\childdocdisable}
{
  \renewcommand{\childdocmain}[1]{\renewcommand{\childdocmain}[1]{\endinput}}
  \renewcommand{\childdocof}[1]{}
  \renewcommand{\childdocby}[2][]{}
  \renewcommand{\childdocforward}[2][]{}
  \renewcommand{\childdocdisable}{}
}
%    \end{macrocode}

% \macro{\childdocmain}
% The macro |\childdocmain| is to be called at the top of the main file
% with nothing or the main filename (without extension) as argument.
% First, it breaks loops.
% If the argument is not empty and does not match |\childdocname|
% (which is set by the first inclusion of |childdoc.def|),
% |\ifchilddoc| is set to true, |\includeonly| is applied to the child file
% and |\jobname| is set to the main file
% (for proper handling of |.aux| files):
%    \begin{macrocode}
\newcommand{\childdocmain}[1]
{
  \childdocdisable\childdocmain{}
  \if?#1?\else
    \begingroup
      \def\childdoctmp{#1}
      \ifx\childdoctmp\childdocname
        \def\childdoctmp{}
      \else
        \def\childdoctmp
        {
          \childdoctrue
          \includeonly{\childdocname}
          \def\childdocjob{#1}
          \def\jobname{#1}
        }
      \fi
      \expandafter
    \endgroup
    \childdoctmp
  \fi
}
%    \end{macrocode}

% \macro{\childdocof}
% The command |\childdocof| redirects
% compilation to the main file |#1|.
%    \begin{macrocode}
\newcommand{\childdocof}[1]
{
  \childdocdisable
  \childdoctrue
  \includeonly{\childdocname}
  \def\jobname{#1}
  \def\childdocjob{#1}
  \input{#1}
}
%    \end{macrocode}

% \macro{\childdocby}
% The command |\childdocby| ....
%    \begin{macrocode}
\newcommand{\childdocby}[2][]
{
  \childdocdisable
  \childdoctrue
  \childdocmanualtrue
  \if?#1?\else
    \def\jobname{#2}
  \fi
  \def\childdocjob{#2}
  \input{#2}
  \endinput
}
%    \end{macrocode}

% \macro{\childdocforward}
% The command |\childdocforward| redirects
% compilation to the main file or
% (if the optional argument is given) a child file.
% Parameters are set as if the main file
% or a child file starting with |\childdocof| was compiled.
% Then compilation is handed over to the main file:
%    \begin{macrocode}
\newcommand{\childdocforward}[2][]
{
  \begingroup
    \if?#1?
      \def\childdoctmp
      {
        \def\childdocname{#2}
        \def\childdocjob{#2}
        \def\jobname{#2}
        \input{#2}
        \endinput
      }
    \else
      \def\childdoctmp
      {
        \childdocdisable
        \def\childdocname{#2}
        \childdoctrue
        \includeonly{#2}
        \def\childdocjob{#1}
        \def\jobname{#1}
        \input{#1}
        \endinput
      }
    \fi
    \expandafter
  \endgroup
  \childdoctmp
}
%    \end{macrocode}

% \macro{\childdocforwardprefix}
% The command |\childdocforwardprefix| redirects
% compilation to the main or a child file by means of a pattern.
% The prefix |#1| in the current filename is replaced by |#2|
% and the suffix of the current filename is kept
% (it is assumed that the filename does not contain the substring `|~~~|'
% which is used as a delimiter).
% Compilation is handed over to the new file by |\childdocforward|:
%    \begin{macrocode}
\newcommand{\childdocforwardprefix}[3][]
{
  \begingroup
    \def\childdocextract #2##1~~~{\def\childdoctmp{\childdocforward[#1]{#3##1}}}
    \expandafter\childdocextract\childdocname~~~
    \expandafter
  \endgroup
  \childdoctmp
}
%    \end{macrocode}

% \macro{\childdoc}
% The deprecated macro |\childdoc| is a legacy version of |\childdocmain|:
%    \begin{macrocode}
\newcommand{\childdoc}{\childdocmain}
%    \end{macrocode}

% \macro{\childdocredirect}
% The deprecated macro |\childdocredirect| is a legacy version
% of |\childdocforward| and |\childdocforwardprefix|:
%    \begin{macrocode}
\newcommand{\childdocredirect}[2][]
{
  \begingroup
    \if?#1?
      \def\childdoctmp{\childdocforward{#2}}
    \else
      \def\childdoctmp{\childdocforwardprefix{#1}{#2}}
    \fi
    \expandafter
  \endgroup
  \childdoctmp
}
%    \end{macrocode}

%\iffalse
%</package>
%\fi
%
\endinput

\childdocforward{cdocsamp}
%    \end{macrocode}

%\iffalse
%</sampledraft>
%\fi
%
% %%%%%%%%%%%%%%%%%%%%%%%%%%%%%%%%%%%%%%
% \paragraph{Forwarding for Final Version of the Chapters.}
%
% The following forwarding files |cdocsfn1.tex| and |cdocsfn2.tex|
% (with identical content)
% compile the final versions of the child documents
% |cdocsch1.tex| and |cdocsch2.tex|, respectively:
%\iffalse
%<*samplefinal>
%\fi
%    \begin{macrocode}
\def\version{final}
% \iffalse
%
% childdoc.dtx Copyright (C) 2017-2018 Niklas Beisert
%
% This work may be distributed and/or modified under the
% conditions of the LaTeX Project Public License, either version 1.3
% of this license or (at your option) any later version.
% The latest version of this license is in
%   http://www.latex-project.org/lppl.txt
% and version 1.3 or later is part of all distributions of LaTeX
% version 2005/12/01 or later.
%
% This work has the LPPL maintenance status `maintained'.
%
% The Current Maintainer of this work is Niklas Beisert.
%
% This work consists of the files childdoc.dtx and childdoc.ins
% and the derived files childdoc.def and cdocsamp.tex with
% cdocsch1.tex, cdocsch2.tex, cdocsdrf.tex, cdocsfn1.tex, cdocsfn2.tex.
%
%<package>\ifdefined\childdocmain\endinput\fi
%<package>\ProvidesFile{childdoc.def}[2018/12/30 v2.0 child document driver]
%<samplemain>\ProvidesFile{cdocsamp.tex}[2018/12/30 v2.0 sample for childdoc]
%<*driver>
%\ProvidesFile{childdoc.drv}[2018/12/30 v2.0 childdoc reference manual file]
\PassOptionsToClass{10pt,a4paper}{article}
\documentclass{ltxdoc}

\usepackage[margin=35mm]{geometry}
\usepackage{hyperref}
\usepackage{hyperxmp}
\usepackage[usenames]{color}

\hypersetup{colorlinks=true}
\hypersetup{pdfstartview=FitH}
\hypersetup{pdfpagemode=UseNone}
\hypersetup{pdfsource={}}
\hypersetup{pdflang={en-UK}}
\hypersetup{pdfcopyright={Copyright 2017-2018 Niklas Beisert.
  This work may be distributed and/or modified under the
  conditions of the LaTeX Project Public License, either version 1.3
  of this license or (at your option) any later version.}}
\hypersetup{pdflicenseurl={http://www.latex-project.org/lppl.txt}}
\hypersetup{pdfcontactaddress={ETH Zurich, ITP, HIT K,
  Wolfgang-Pauli-Strasse 27}}
\hypersetup{pdfcontactpostcode={8093}}
\hypersetup{pdfcontactcity={Zurich}}
\hypersetup{pdfcontactcountry={Switzerland}}
\hypersetup{pdfcontactemail={nbeisert@itp.phys.ethz.ch}}
\hypersetup{pdfcontacturl={http://people.phys.ethz.ch/\xmptilde nbeisert/}}

\newcommand{\secref}[1]{\hyperref[#1]{section \ref*{#1}}}

\parskip1ex
\parindent0pt
\let\olditemize\itemize
\def\itemize{\olditemize\parskip0pt}

\begin{document}

\title{The \textsf{childdoc} Package}
\hypersetup{pdftitle={The childdoc Package}}
\author{Niklas Beisert\\[2ex]
  Institut f\"ur Theoretische Physik\\
  Eidgen\"ossische Technische Hochschule Z\"urich\\
  Wolfgang-Pauli-Strasse 27, 8093 Z\"urich, Switzerland\\[1ex]
  \href{mailto:nbeisert@itp.phys.ethz.ch}
  {\texttt{nbeisert@itp.phys.ethz.ch}}}
\hypersetup{pdfauthor={Niklas Beisert}}
\hypersetup{pdfsubject={Manual for the LaTeX2e Package childdoc}}
\date{30 December 2018, \textsf{v2.0}}
\maketitle

\begin{abstract}\noindent
\textsf{childdoc} is a \LaTeXe{} package
that enables the direct compilation
of document sections included by |\include|
to individual files.
\end{abstract}

\begingroup
\parskip0ex
\tableofcontents
\endgroup

%%%%%%%%%%%%%%%%%%%%%%%%%%%%%%%%%%%%%%%%%%%%%%%%%%%%%%%%%%%%%%%%%%%%%%%%%%%%%%%%
%%%%%%%%%%%%%%%%%%%%%%%%%%%%%%%%%%%%%%%%%%%%%%%%%%%%%%%%%%%%%%%%%%%%%%%%%%%%%%%%
\section{Introduction}

\LaTeX{} provides a mechanism to structure a large document (such as a book)
into a main file and several child files (containing the chapters)
using the |\include| command.
This mechanism is beneficial for documents
which span hundreds of pages in order to
make the source file(s) more manageable.
Moreover, compilation can be restricted to
selected child files by means of the |\includeonly| command.
The latter feature can be used to reduce the compilation time while editing
(this was significantly more useful in the earlier days of \LaTeX{})
or to generate a smaller document which is easier to navigate.
Another application of |\includeonly| is to generate
documents consisting of selected parts of the complete document.

However, there are a few drawbacks of the plain |\include| mechanism:
\begin{itemize}
\item
The child files cannot be compiled on their own,
they can only be compiled via the main file.
A naive editing environment
(such as a text editor with an option
to have the current file processed by \LaTeX)
may require one to switch to the main file before compiling;
attempting to compile the child file produces errors.
\item
The main file must be modified (each time)
to adjust the |\includeonly| command
to the present needs. This easily leaves the main file in a messy state.
\item
The generated document will always carry the filename
of the main document. This is inconvenient if
several child files are to be compiled and
to be kept for distribution.
\end{itemize}

The present package provides a simple interface
to make child files individually compilable by \LaTeX{}.
Compiling a child file then has the same effect as compiling
the main file with an |\includeonly| command
to select the appropriate child.
Moreover the generated document will carry the name of the child
rather than the main file.
This resolves all three above issues.

This feature is meant to make the editing of books,
thesis documents and lecture notes somewhat more convenient.
However, the package can also be used efficiently for
composing a series of documents (such as exercise sheets)
which are typically distributed individually.
It then assists the author in generating the individual documents
(potentially in different versions)
as well as a document containing the collected series.
Another application is in developing style files
or other kinds of included material
where compilation of the style file could redirect
to a sample or test file.

%%%%%%%%%%%%%%%%%%%%%%%%%%%%%%%%%%%%%%%%%%%%%%%%%%%%%%%%%%%%%%%%%%%%%%%%%%%%%%%%
%%%%%%%%%%%%%%%%%%%%%%%%%%%%%%%%%%%%%%%%%%%%%%%%%%%%%%%%%%%%%%%%%%%%%%%%%%%%%%%%
\section{Usage}

First of all, the package \textsf{childdoc} is \emph{not} a standard
\LaTeXe{} |.sty| style file! Therefore it needs to be invoked in
a non-standard way.

%%%%%%%%%%%%%%%%%%%%%%%%%%%%%%%%%%%%%%%%%%%%%%%%%%%%%%%%%%%%%%%%%%%%%%%%%%%%%%%%
\subsection{Included Files}
\label{sec:include}

%%%%%%%%%%%%%%%%%%%%%%%%%%%%%%%%%%%%%%%%
\DescribeMacro{\childdocmain}
To use the package, add the commands
\begin{center}
\begin{tabular}{l}
|\input{childdoc.def}|\\
|\childdocmain{}|\\
\end{tabular}
\end{center}
at the very top of the main \LaTeX{} file,
in particular \emph{before} the |\documentclass| statement!
The argument of |\childdocmain| should be left empty
(but it must be present).

%%%%%%%%%%%%%%%%%%%%%%%%%%%%%%%%%%%%%%%%
\DescribeMacro{\childdocof}
Furthermore, add the commands
\begin{center}
\begin{tabular}{l}
|\input{childdoc.def}|\\
|\childdocof{|\textit{main}|}|\\
\end{tabular}
\end{center}
at the top of every child file \textit{child}
which is included by |\include{|\textit{child}|}|
from within the main file
(or at least for those files to be compiled individually).
The argument \textit{main} must be the filename of the main file.

There are a couple of
considerations in setting up the main and child documents:

%%%%%%%%%%%%%%%%%%%%%%%%%%%%%%%%%%%%%%%%
\paragraph{Restrictions.}

Please note the following restrictions:
\begin{itemize}
\item
|\childdocmain| must be called with one argument \textit{main}
to ensure compatibility with earlier version of the package.
It must either be empty (|\childdocmain{}|)
or precisely match the filename of the main file in which it is specified.
See \secref{sec:detection} for further information.
\item
The filename \textit{main} must be specified without the |.tex| extension.
\item
The filename \textit{main} is case sensitive
(even in case-insensitive file systems)
due to internal string comparison.
\item
The argument \textit{main} should be fully expanded, it cannot be a macro.
\item
Subdirectories and special characters should be avoided in filenames.
\item
The command |\childdocmain{|\textit{main}|}| must be followed by a whitespace.
It should not be followed immediately by another command
or by a comment mark `|%|'.
This is because the \TeX{} parser reads the token immediately following
the argument of |\childdocmain| and puts it
at the beginning of every child section;
however, a white\-space is ignored.
\end{itemize}

%%%%%%%%%%%%%%%%%%%%%%%%%%%%%%%%%%%%%%%%
\paragraph{Content of Main File.}

It is advisable to place all content in the child files included by |\include|.
Any output contained in the main file will appear in all child documents
unless suppressed manually;
it cannot be suppressed automatically by the |\includeonly| directive
and thus should normally be avoided.
A method to include some content in the main file
by means of conditional processing is described in \secref{sec:conditional}.

%%%%%%%%%%%%%%%%%%%%%%%%%%%%%%%%%%%%%%%%
\paragraph{Page Numbering.}

When only a part of the document is compiled,
the appropriate numbering of pages
(as well as other status parameters)
is determined from the |.aux| files.
The latter contain information from previous passes.
However this information needs to propagate through
all intermediate child documents.
Therefore the page numbering in child documents may well
be inconsistent until the complete document is compiled at least once.

A useful (if unconventional) way to always ensure a consistent
page numbering is to restart the numbering in each child document
and denote the pages by `\textit{child}|.|\textit{page}'
where \textit{child} represents the chapter/section number of the child file.
This can be achieved by the command
|\numberwithin{page}{|\textit{child}|}|
of the \textsf{amsmath} package
where \textit{child} can be |chapter| or |section|
depending on the chosen structuring.
Alternatively, one can modify the macro |\thepage| appropriately
and reset the counter |page| at the start of each child file.

%%%%%%%%%%%%%%%%%%%%%%%%%%%%%%%%%%%%%%%%%%%%%%%%%%%%%%%%%%%%%%%%%%%%%%%%%%%%%%%%
\subsection{Conditional Processing}
\label{sec:conditional}

The package provides a mechanism to compile different versions
of a document. To customise the versions further some conditional processing
can come in handy to distinguish which version is being compiled.
The package provides two macros to describe the compilation context:

%%%%%%%%%%%%%%%%%%%%%%%%%%%%%%%%%%%%%%%%
\DescribeMacro{\ifchilddoc}
The conditional |\ifchilddoc| distinguishes between the compilation of
child documents and the main document:
%
\begin{center}
|\ifchilddoc |\textit{child-code}| |[|\||else |\textit{main-code}]| \||fi|
\end{center}

%%%%%%%%%%%%%%%%%%%%%%%%%%%%%%%%%%%%%%%%
\DescribeMacro{\childdocname}
\DescribeMacro{\childdocjob}
The macro |\childdocname| contains the filename (without extension)
of the main or child file being processed.
Note that |\childdocjob| will always contain the name of the main file.

%%%%%%%%%%%%%%%%%%%%%%%%%%%%%%%%%%%%%%%%
\paragraph{Title Page.}

Conditional processing can be used to include a title or banner page
in the main document when proper precautions are taken.
Importantly, the code in the main file should ensure that the page counter
(as well as other status parameters which are stored in the |.aux| files)
takes the same value after the conditional processing.
Otherwise the page numbers may take divergent values
depending on which part is compiled.

For example, a title page could be declared by:
%
\begin{center}
\begin{tabular}{l}
|\ifchilddoc\||else|\\
|\addtocounter{page}{-1}|\\
\textit{code for title page}\\
|\newpage|\\
|\||fi|
\end{tabular}
\end{center}
%
A banner page for the child documents can be generated by:
%
\begin{center}
\begin{tabular}{l}
|\ifchilddoc|\\
|\addtocounter{page}{-1}|\\
\textit{code for banner page}\\
|\newpage|\\
|\||fi|
\end{tabular}
\end{center}
%
Here one could write a message such as:
\begin{center}
|This is the part \childdocname{} of \childdocjob{}.|
\end{center}

%%%%%%%%%%%%%%%%%%%%%%%%%%%%%%%%%%%%%%%%%%%%%%%%%%%%%%%%%%%%%%%%%%%%%%%%%%%%%%%%
\subsection{Flags}
\label{sec:flags}

The package makes it easy to generate different versions
of the main or child documents.
To this end compilation flags can be defined
and assigned different default values.
They will be particularly useful in conjunction
with the forwarding mechanism described in \secref{sec:forward}.

For example, it may be useful to have a flag |\version|
which can be set to |draft| or |final|.
The document source will contain some conditional code
depending on the value of |\version|.
Suppose further, the flag should default to |final| for the main file
and to |draft| for child files
which is a natural assignment for editing the document.
This is achieved by placing the following code
in the preamble of the main document
(below the |\childdocmain| directive):
%
\begin{center}
\begin{tabular}{l}
|\ifchilddoc|\\
|\providecommand{\version}{draft}|\\
|\||else|\\
|\providecommand{\version}{final}|\\
|\||fi|
\end{tabular}
\end{center}
%
The definition by |\providecommand| makes sure
that previous definitions are not overwritten.
Further statements |\providecommand{\version}{...}|
can thus be added before the above code to override it.

For the main file, one might add a line
(between |\childdocmain| and the above block)
%
\begin{center}
|%\ifchilddoc\||else\providecommand{\version}{draft}\||fi|
\end{center}
%
which can be uncommented to produce a draft version.
Likewise one can add a line to the very top of a child file
(above the |\childdocof{|\textit{main}|}| directive)
%
\begin{center}
|%\providecommand{\version}{final}|
\end{center}
%
which can be uncommented to produce the final version of this child document.

%%%%%%%%%%%%%%%%%%%%%%%%%%%%%%%%%%%%%%%%%%%%%%%%%%%%%%%%%%%%%%%%%%%%%%%%%%%%%%%%
\subsection{Forwarding}
\label{sec:forward}

Different versions of the main or child documents
using compilation flags as described in \secref{sec:flags}
can be (permanently) stored in different files
for convenient compilation, viewing and distribution.
To this end, the package defines a command
to pass on compilation to a different file:

%%%%%%%%%%%%%%%%%%%%%%%%%%%%%%%%%%%%%%%%
\DescribeMacro{\childdocforward}
The command |\childdocforward| redirects processing to
another source file:
%
\begin{center}
\begin{tabular}{l}
|\input{childdoc.def}|\\
|\childdocforward[|\textit{main}|]{|\textit{dest}|}|\\
\end{tabular}
\end{center}
%
The argument \textit{dest} is the destination file
(without extension).
It should be the main file or one of the child files.
Note that further \textsf{childdoc} directives
such as |\childdocof| and |\childdocforward|
in the indicated file will be processed in this form.
The optional argument \textit{main}
passes on directly to the main file \textit{main}
while pretending to compile the child \textit{dest}.
This form behaves as if \textit{dest}
issues |\childdocof{|\textit{main}|}| right away,
and no further \textsf{childdoc} directives will be processed.

%%%%%%%%%%%%%%%%%%%%%%%%%%%%%%%%%%%%%%%%
\DescribeMacro{\...prefix}
In the alternative form |\childdocforwardprefix|,
%
\begin{center}
\begin{tabular}{l}
|\input{childdoc.def}|\\
|\childdocforwardprefix[|\textit{main}|]{|\textit{prefix}|}{|\textit{dest}|}|
\end{tabular}
\end{center}
%
the destination file is determined by a pattern
depending on the current file:
To make this work, the current file must be called
`{\textit{prefix}\hspace{0.2em}\textit{suffix}}'
with \textit{prefix} matching precisely the argument.
Processing is then passed on to the file
`{\textit{dest}\hspace{0.2em}\textit{suffix}}'.
Surely, the same effect is achieved by
directly specifying the
argument `{\textit{dest}\hspace{0.2em}\textit{suffix}}'
in the first form.
However, that requires to set up a different file
for each child. With the alternative form of the command
all these files can have exactly the same content
which simplifies setting them up and maintaining them.

For example, the following file |draft.tex|
with a compilation flag |\version| as described in \secref{sec:flags}
compiles the main document as a draft:
%
\begin{center}
\begin{tabular}{l}
|\def\version{draft}|\\
|\input{childdoc.def}|\\
|\childdocforward{|\textit{main}|}|
\end{tabular}
\end{center}
%
Likewise, the following files |final|\textit{nn}|.tex|
compile the final version of the child document
|child|\textit{nn}|.tex|:
%
\begin{center}
\begin{tabular}{l}
|\def\version{final}|\\
|\input{childdoc.def}|\\
|\childdocforwardprefix{final}{child}|
\end{tabular}
\end{center}
%

Note that when several versions of a main file and/or of each child file
are to be generated, it may be convenient to set up a |Makefile| or
shell script to automatise the process.

%%%%%%%%%%%%%%%%%%%%%%%%%%%%%%%%%%%%%%%%%%%%%%%%%%%%%%%%%%%%%%%%%%%%%%%%%%%%%%%%
\subsection{Command Line Processing}
\label{sec:commandline}

The effect of redirection files can also be achieved by invoking
the \LaTeX{} compiler with a more elaborate command line.
Most conveniently this should be done as part
of a shell script or a |Makefile|.

When using \textsf{childdoc} in the main file, the following
command lines effectively perform a redirection
(note that depending on the shell being used,
backslashes may have to be doubled: `|\|' $\to$ `|\\|'):
%
\begin{center}
|... -jobname "|\textit{target}|" |\\|"|[\textit{flags}]%
|\input{childdoc.def}\childdocforward[|\textit{main}|]{|\textit{dest}|}"|
\end{center}
%
Here \textit{target} is the name of the output file,
\textit{main} is the name of the main file
and \textit{dest} is the name of the main or child file to be processed
(all filenames without extensions).
The optional argument \textit{main} can be omitted
if \textit{main} matches \textit{dest}.
Optionally, compilation \textit{flags} can be defined via |\def| commands.
This command line makes the \TeX{} engine believe
it is compiling the file \textit{target}
whose content is specified as the latter parameter.
The provided code then forwards the processing to
\textit{main} or \textit{dest} as described in \secref{sec:forward}.

%%%%%%%%%%%%%%%%%%%%%%%%%%%%%%%%%%%%%%%%%%%%%%%%%%%%%%%%%%%%%%%%%%%%%%%%%%%%%%%%
\subsection{Include by Input}
\label{sec:input}

Including child documents by |\include| has some restrictions by design.
Most notably, the content of a child document always occupies
its own set of pages; pages cannot be shared between child documents.
Usually, this behaviour makes perfect sense
because each child document contain an essential part of the document.
However, in some situations it may be desirable to compose
a document from a collection of parts
without having mandatory page breaks between then.
For this case, the package
provides a mechanism to include parts
by |\input| which can also be processed individually.
However, by construction this mechanism
requires manual handling of the content to be output.

%%%%%%%%%%%%%%%%%%%%%%%%%%%%%%%%%%%%%%%%
\DescribeMacro{\ifchilddocmanual}
The main file should be prepared as usual, see \secref{sec:include}.
However, the document body must make a distinction
between processing of an individual part and of the main document, e.g.:
%
\begin{center}
\begin{tabular}{l}
|\ifchilddocmanual|\\
|\input{\childdocname}|\\
|\||else|\\
\textit{document body with }|\input{|\textit{part}|}|\\
|\||fi|
\end{tabular}
\end{center}
%
The conditional |\ifchilddocmanual| is true whenever
a part to be included by |\input| is being compiled,
and the name of the part is stored in |\childdocname|.

%%%%%%%%%%%%%%%%%%%%%%%%%%%%%%%%%%%%%%%%
\DescribeMacro{\childdocby}
Each part to be included by |\input| should start with:
%
\begin{center}
\begin{tabular}{l}
|\input{childdoc.def}|\\
|\childdocby{|\textit{main}|}|\\
\end{tabular}
\end{center}
%
The directive |\childdocby| is similar to |\childdocof|
described in \secref{sec:include},
but the subsequent selection of content must be done manually.
To that end, both |\ifchilddoc| and |\ifchilddocmanual|
will be true upon processing of a part,
and the name of the part is stored in |\childdocname|.
Note that |\jobname| will be set to the filename of the current part
so that each part receives an individual |.aux| file
that does not interfere with the |.aux| file(s) of the main document.
This behaviour can be altered by the alternative form
|\childdocby[*]{|\textit{main}|}| (with a non-empty optional argument)
which uses the |.aux| file of the main document
by setting |\jobname| to \textit{main}.

%%%%%%%%%%%%%%%%%%%%%%%%%%%%%%%%%%%%%%%%%%%%%%%%%%%%%%%%%%%%%%%%%%%%%%%%%%%%%%%%
\subsection{Driver Development}
\label{sec:driver}

The \textsf{childdoc} mechanism can also be use for the development
of definition files such as \LaTeX{} styles or classes.
This case differs from the above setup with multiple parts
included by |\include| in that no |\includeonly| should be invoked.
This can be achieved by starting the include file
(before |\ProvidesPackage|) with:
%
\begin{center}
\begin{tabular}{l}
|\input{childdoc.def}|\\
|\childdocforward{|\textit{main}|}|\\
\end{tabular}
\end{center}
%
or alternatively with:
%
\begin{center}
\begin{tabular}{l}
|\input{childdoc.def}|\\
|\childdocby{|\textit{main}|}|\\
\end{tabular}
\end{center}
%
Both forms have slightly different effects as described above.
The main file is prepared as usual, see \secref{sec:include}.

%%%%%%%%%%%%%%%%%%%%%%%%%%%%%%%%%%%%%%%%%%%%%%%%%%%%%%%%%%%%%%%%%%%%%%%%%%%%%%%%
\subsection{Legacy Detection}
\label{sec:detection}

The directive |\childdocmain| in the main file can detect
whether the complete document or merely a child is to be compiled
even without using the directive |\childdocof|.
This method is deprecated because it is less robust
and there is no compelling reason to use it;
it is merely provided for backward compatibility
and it may be removed in future versions.

If the detection mechanism is to be used,
it is mandatory to correctly specify
the filename of the main file as the argument of |\childdocmain|:
%
\begin{center}
\begin{tabular}{l}
|\input{childdoc.def}|\\
|\childdocmain{|\textit{main}|}|\\
\end{tabular}
\end{center}
%
If |\jobname| does not match the argument \textit{main} of |\childdocmain|,
it is assumed that |\jobname| points to the child file to be compiled.
When using |\childdocmain| with the main file specified as argument,
it suffices to start a child file
with just |\input{|\textit{main}|}|
without loading of the package and using |\childdocof|.
If instead all processing is done
with the appropriate \textsf{childdoc} directives,
the argument of \textit{main} of |\childdocmain| can be empty.

An alternative version of the command line processing described
in \secref{sec:commandline} using the detection mechanism reads:
%
\begin{center}
|... -jobname "|\textit{target}|" "|[\textit{flags}]%
[|\def\jobname{|\textit{dest}|}|]|\input{|\textit{main}|}"|
\end{center}

%%%%%%%%%%%%%%%%%%%%%%%%%%%%%%%%%%%%%%%%%%%%%%%%%%%%%%%%%%%%%%%%%%%%%%%%%%%%%%%%
\subsection{Manual Code}
\label{sec:manual}

In case one cannot be certain whether the definitions file |childdoc.def|
is installed on the target \TeX{} distribution
and one prefers not to ship it,
it is conceivable to paste a few relevant commands into the sources.

To that end, drop all statements |\input{childdoc.def}|
and perform the replacements as outlined below.
Instead of |\childdocmain{|\textit{main}|}| add the following code
to the top of the main file:
%
\begin{center}
\begin{tabular}{l}
|\||ifdefined\childdocname\endinput\||fi\newif\ifchilddoc|\\
|\edef\childdocname{\scantokens\expandafter{\jobname\noexpand}}|\\
|\def\childdocmain{|\textit{main}|}\||ifx\childdocmain\childdocname\||else|\\
|\childdoctrue\includeonly{\childdocname}\let\jobname\childdocmain\||fi|\\
\end{tabular}
\end{center}
%
Instead of |\childdocof{|\textit{main}|}| just include the main file
at the top of each child file:
%
\begin{center}
|\input{|\textit{main}|}|
\end{center}
%
A simple redirection |\childdocforward{|\textit{dest}|}| is achieved by:
%
\begin{center}
|\def\jobname{|\textit{dest}|}\input{\jobname}|
\end{center}
%
The redirection with prefix
|\childdocforwardprefix[|\textit{prefix}|]{|\textit{dest}|}|
is accomplished by:
%
\begin{center}
\begin{tabular}{l}
|{\edef\jobname{\scantokens\expandafter{\jobname\noexpand}}|\\
|\def\redirectjob |\textit{prefix}|#1~~~{\gdef\jobname{|\textit{dest}|#1}}|\\
|\expandafter\redirectjob\jobname~~~}\input{\jobname}|
\end{tabular}
\end{center}

In an alternative approach,
child documents can be compiled by a specific command line
without additional code or specific definitions:
%
\begin{center}
|... -jobname "|\textit{target}|" "|[\textit{flags}]%
|\includeonly{|\textit{dest}|}\input{|\textit{main}|}"|
\end{center}
%

%%%%%%%%%%%%%%%%%%%%%%%%%%%%%%%%%%%%%%%%%%%%%%%%%%%%%%%%%%%%%%%%%%%%%%%%%%%%%%%%
%%%%%%%%%%%%%%%%%%%%%%%%%%%%%%%%%%%%%%%%%%%%%%%%%%%%%%%%%%%%%%%%%%%%%%%%%%%%%%%%
\section{Information}

%%%%%%%%%%%%%%%%%%%%%%%%%%%%%%%%%%%%%%%%%%%%%%%%%%%%%%%%%%%%%%%%%%%%%%%%%%%%%%%%
\subsection{Copyright}

Copyright \copyright{} 2017--2018 Niklas Beisert

This work may be distributed and/or modified under the
conditions of the \LaTeX{} Project Public License, either version 1.3
of this license or (at your option) any later version.
The latest version of this license is in
  \url{http://www.latex-project.org/lppl.txt}
and version 1.3 or later is part of all distributions of \LaTeX{}
version 2005/12/01 or later.

This work has the LPPL maintenance status `maintained'.

The Current Maintainer of this work is Niklas Beisert.

This work consists of the files |README.txt|, |childdoc.ins| and |childdoc.dtx|
as well as the derived files |childdoc.def|, |cdocsamp.tex|
with |cdocsch1.tex|, |cdocsch2.tex|, |cdocspt3.tex|, |cdocspt4.tex|,
|cdocsdrf.tex|, |cdocsfn1.tex|, |cdocsfn2.tex|
as well as |childdoc.pdf|.

%%%%%%%%%%%%%%%%%%%%%%%%%%%%%%%%%%%%%%%%%%%%%%%%%%%%%%%%%%%%%%%%%%%%%%%%%%%%%%%%
\subsection{Files and Installation}

The package consists of the files:
%
\begin{center}
\begin{tabular}{ll}
    |README.txt|   & readme file \\
    |childdoc.ins| & installation file \\
    |childdoc.dtx| & source file \\
    |childdoc.def| & definition file \\
    |cdocsamp.tex| & sample main file \\
    |cdocsch1.tex| & sample include file \\
    |cdocsch2.tex| & sample include file \\
    |cdocspt3.tex| & sample part file \\
    |cdocspt4.tex| & sample part file \\
    |cdocsdrf.tex| & sample redirection file \\
    |cdocsfn1.tex| & sample redirection file \\
    |cdocsfn2.tex| & sample redirection file \\
    |childdoc.pdf| & manual
\end{tabular}
\end{center}
%
The distribution consists of the files
|README.txt|, |childdoc.ins| and |childdoc.dtx|.
%
\begin{itemize}
\item
Run (pdf)\LaTeX{} on |childdoc.dtx|
to compile the manual |childdoc.pdf| (this file).
\item
Run \LaTeX{} on |childdoc.ins| to create the definitions file |childdoc.def|
and the sample |cdocsamp.tex| with include files
|cdocsch1.tex|, |cdocsch2.tex|, |cdocspt3.tex|, |cdocspt4.tex|,
|cdocsdrf.tex|, |cdocsfn1.tex|, |cdocsfn2.tex|.
Then copy the file |childdoc.def| to an appropriate directory of your \LaTeX{}
distribution, e.g.\ \textit{texmf-root}|/tex/latex/childdoc|.
\end{itemize}

%%%%%%%%%%%%%%%%%%%%%%%%%%%%%%%%%%%%%%%%%%%%%%%%%%%%%%%%%%%%%%%%%%%%%%%%%%%%%%%%
\subsection{Related CTAN Packages}

There are several other packages which offer a similar functionality:
%
\begin{itemize}
\item
The packages
\href{http://ctan.org/pkg/docmute}{\textsf{docmute}},
\href{http://ctan.org/pkg/includex}{\textsf{includex}} and
\href{http://ctan.org/pkg/standalone}{\textsf{standalone}}
provide commands to include only the document body of
a child file thus allowing both files to be compiled individually.
\item
The packages \href{http://ctan.org/pkg/subdocs}{\textsf{subdocs}}
and \href{http://ctan.org/pkg/subfiles}{\textsf{subfiles}}
provide structures in which the main and child documents can be
encapsulated and allowing them to be compiled individually.
The inclusion mechanism is different from the conventional |\include|.
\item
The package \href{http://ctan.org/pkg/combine}{\textsf{combine}}
is an elaborate solution to combine several documents into one.
\end{itemize}
%
See also the CTAN topic \href{http://ctan.org/topic/subdocs}{\textsf{subdocs}}
for further related packages.
The present package differs from the above solutions in that
a document structure constructed with the conventional |\include| mechanism
just needs two extra commands at the top of every file
such that all constituent files can be compiled individually.

%%%%%%%%%%%%%%%%%%%%%%%%%%%%%%%%%%%%%%%%%%%%%%%%%%%%%%%%%%%%%%%%%%%%%%%%%%%%%%%%
%\subsection{Feature Suggestions}
%
%The following is a list of features which may be useful for future
%versions of this package:
%%
%\begin{itemize}
%\item
%\ldots
%\end{itemize}

%%%%%%%%%%%%%%%%%%%%%%%%%%%%%%%%%%%%%%%%%%%%%%%%%%%%%%%%%%%%%%%%%%%%%%%%%%%%%%%%
\subsection{Revision History}

%%%%%%%%%%%%%%%%%%%%%%%%%%%%%%%%%%%%%%%%
\paragraph{v2.0:} 2018/12/30

\begin{itemize}
\item
immediate forward processing
\item
added |\childdocby| mechanism
\item
manual restructured
\end{itemize}

%%%%%%%%%%%%%%%%%%%%%%%%%%%%%%%%%%%%%%%%
\paragraph{v1.6:} 2018/01/17

\begin{itemize}
\item
application for development of include files
\item
corrections to manual
\end{itemize}

%%%%%%%%%%%%%%%%%%%%%%%%%%%%%%%%%%%%%%%%
\paragraph{v1.5:} 2017/05/21

\begin{itemize}
\item
more complete structuring introduced
\item
|\childdocof| introduced
\item
|\childdoc| renamed to |\childdocmain|
\item
|\childredirect| renamed to |\childdocforward| and |\childdocforwardprefix|
and functionality expanded
\end{itemize}

%%%%%%%%%%%%%%%%%%%%%%%%%%%%%%%%%%%%%%%%
\paragraph{v1.0:} 2017/04/27

\begin{itemize}
\item
manual and install package
\item
first version published on CTAN
\end{itemize}

%%%%%%%%%%%%%%%%%%%%%%%%%%%%%%%%%%%%%%%%
\paragraph{v0.6:} 2017/04/26

\begin{itemize}
\item
redirection mechanism added
\end{itemize}

%%%%%%%%%%%%%%%%%%%%%%%%%%%%%%%%%%%%%%%%
\paragraph{v0.5:} 2017/04/26

\begin{itemize}
\item
functionality in definition file
\end{itemize}


%%%%%%%%%%%%%%%%%%%%%%%%%%%%%%%%%%%%%%%%%%%%%%%%%%%%%%%%%%%%%%%%%%%%%%%%%%%%%%%%
%%%%%%%%%%%%%%%%%%%%%%%%%%%%%%%%%%%%%%%%%%%%%%%%%%%%%%%%%%%%%%%%%%%%%%%%%%%%%%%%
%%%%%%%%%%%%%%%%%%%%%%%%%%%%%%%%%%%%%%%%%%%%%%%%%%%%%%%%%%%%%%%%%%%%%%%%%%%%%%%%
\appendix

\settowidth\MacroIndent{\rmfamily\scriptsize 000\ }

 \DocInput{childdoc.dtx}

\end{document}
%</driver>
% \fi
%
% %%%%%%%%%%%%%%%%%%%%%%%%%%%%%%%%%%%%%%%%%%%%%%%%%%%%%%%%%%%%%%%%%%%%%%%%%%%%%%
% %%%%%%%%%%%%%%%%%%%%%%%%%%%%%%%%%%%%%%%%%%%%%%%%%%%%%%%%%%%%%%%%%%%%%%%%%%%%%%
% \section{Sample}
%\iffalse
%<*samplemain>
%\fi
%
% The following presents a sample document
% with two chapters, two parts, a title page,
% a compile flag as well as three forwarding files to set the flag.
% It consists of eight |.tex| files:
% \begin{center}
% \begin{tabular}{ll}
% |cdocsamp.tex|&main file\\
% |cdocsch1.tex|&include file for chapter 1\\
% |cdocsch2.tex|&include file for chapter 2\\
% |cdocspt3.tex|&include file for part 3\\
% |cdocspt4.tex|&include file for part 4\\
% |cdocsdrf.tex|&forwarding file for main file in draft mode\\
% |cdocsfi1.tex|&forwarding file for final version of chapter 1\\
% |cdocsfi2.tex|&forwarding file for final version of chapter 2\\
% \end{tabular}
% \end{center}
% Each of the eight files can be compiled directly by the \LaTeX{} compiler.
%
% %%%%%%%%%%%%%%%%%%%%%%%%%%%%%%%%%%%%%%
% \paragraph{Main File.}
%
% The main file is called |cdocsamp.tex|.
%
% Load the \textsf{childdoc} definitions and
% declare the filename for the main document:
%    \begin{macrocode}
\input{childdoc.def}
\childdocmain{}
%    \end{macrocode}

% Optional override for |\version| flag:
%    \begin{macrocode}
%%\ifchilddoc\else\providecommand{\version}{draft}\fi
%    \end{macrocode}

% Define the default values for the |\version| flag
% (|final| for the main file and |draft| for childs):
%    \begin{macrocode}
\ifchilddoc
\providecommand{\version}{draft}
\else
\providecommand{\version}{final}
\fi
%    \end{macrocode}

% Load the standard document class:
%    \begin{macrocode}
\documentclass[12pt]{article}
%    \end{macrocode}

% Start the document body:
%    \begin{macrocode}
\begin{document}
%    \end{macrocode}

% Declare a title page.
% Print title, part of document being processed and version flag:
%    \begin{macrocode}
\addtocounter{page}{-1}
\begin{center}
{\LARGE\bfseries{}childdoc example\par}
\vspace{1cm}
\ifchilddoc
\ifchilddocmanual part\else chapter\fi:
`\childdocname' of `\childdocjob'\par
\else
main document: `\childdocjob'\par
\fi
version: \version\par
\end{center}
\newpage
%    \end{macrocode}

% Manually include selected file,
% otherwise process as usual:
%    \begin{macrocode}
\ifchilddocmanual
\section*{part `\childdocname'}
\input{\childdocname}
\else
%    \end{macrocode}

% Include the two chapters:
%    \begin{macrocode}
\include{cdocsch1}
\include{cdocsch2}
%    \end{macrocode}

% Include the two parts unless only chapters should be displayed:
%    \begin{macrocode}
\ifchilddoc\else
\section{part three}
\input{cdocspt3}
\section{part four}
\input{cdocspt4}
\fi
%    \end{macrocode}

% Process as usual until here:
%    \begin{macrocode}
\fi
%    \end{macrocode}

% End of document body:
%    \begin{macrocode}
\end{document}
%    \end{macrocode}
%\iffalse
%</samplemain>
%\fi
%
% %%%%%%%%%%%%%%%%%%%%%%%%%%%%%%%%%%%%%%
% \paragraph{Chapter Include Files.}
%
% The include files are called |cdocsch1.tex| and |cdocsch2.tex|.
%
%\iffalse
%<*samplechap1|samplechap2>
%\fi

% Optional override for |\version| flag:
%    \begin{macrocode}
%%\providecommand{\version}{final}
%    \end{macrocode}

% Include the main document:
%    \begin{macrocode}
\input{childdoc.def}
\childdocof{cdocsamp}
%    \end{macrocode}

%\iffalse
%</samplechap1|samplechap2>
%\fi
%
%\iffalse
%<*samplechap1>
%\fi
% Some text for chapter 1:
%    \begin{macrocode}
\section{one}
some text in chapter one
%    \end{macrocode}

%\iffalse
%</samplechap1>
%\fi
% Some text for chapter 2:
%\iffalse
%<*samplechap2>
%\fi
%    \begin{macrocode}
\section{two}
more text in chapter two
%    \end{macrocode}

%\iffalse
%</samplechap2>
%\fi
%
% %%%%%%%%%%%%%%%%%%%%%%%%%%%%%%%%%%%%%%
% \paragraph{Part Include Files.}
%
% The include files are called |cdocspt3.tex| and |cdocspt4.tex|.
%
%\iffalse
%<*samplepart3|samplepart4>
%\fi

% Optional override for |\version| flag:
%    \begin{macrocode}
%%\providecommand{\version}{final}
%    \end{macrocode}

% Include the main document:
%    \begin{macrocode}
\input{childdoc.def}
\childdocby{cdocsamp}
%    \end{macrocode}

%\iffalse
%</samplepart3|samplepart4>
%\fi
%
%\iffalse
%<*samplepart3>
%\fi
% Some text for part 3:
%    \begin{macrocode}
some text in part three
%    \end{macrocode}

%\iffalse
%</samplepart3>
%\fi
% Some text for part 4:
%\iffalse
%<*samplepart4>
%\fi
%    \begin{macrocode}
more text in part four
%    \end{macrocode}

%\iffalse
%</samplepart4>
%\fi
%
% %%%%%%%%%%%%%%%%%%%%%%%%%%%%%%%%%%%%%%
% \paragraph{Forwarding for a Complete Draft.}
%
% The following forwarding file |cdocsdrf.tex|
% compiles the main document in draft mode:
%\iffalse
%<*sampledraft>
%\fi
%    \begin{macrocode}
\def\version{draft}
\input{childdoc.def}
\childdocforward{cdocsamp}
%    \end{macrocode}

%\iffalse
%</sampledraft>
%\fi
%
% %%%%%%%%%%%%%%%%%%%%%%%%%%%%%%%%%%%%%%
% \paragraph{Forwarding for Final Version of the Chapters.}
%
% The following forwarding files |cdocsfn1.tex| and |cdocsfn2.tex|
% (with identical content)
% compile the final versions of the child documents
% |cdocsch1.tex| and |cdocsch2.tex|, respectively:
%\iffalse
%<*samplefinal>
%\fi
%    \begin{macrocode}
\def\version{final}
\input{childdoc.def}
\childdocforwardprefix[cdocsamp]{cdocsfn}{cdocsch}
%    \end{macrocode}

%\iffalse
%</samplefinal>
%\fi
%
% %%%%%%%%%%%%%%%%%%%%%%%%%%%%%%%%%%%%%%
% \paragraph{Command Line Processing.}
%
% The following three command lines generate the output files
% |cdocscld|, |cdocscl1| and |cdocscl2|
% which should be identical to
% |cdocsdrf|, |cdocsch1| and |cdocsfn2|, respectively:
% \begin{center}
% \begin{tabular}{l}
% |latex -jobname cdocscld \|\\
% |  "\def\version{draft}\input{childdoc.def}\childdocforward{cdocsamp}"|\\
% |latex -jobname cdocscl1 \|\\
% |  "\input{childdoc.def}\childdocforward[cdocsamp]{cdocsch1}"|\\
% |latex -jobname cdocscl2 \|\\
% |  "\def\version{final}\input{childdoc.def}\childdocforward{cdocsch2}"|
% \end{tabular}
% \end{center}
% Note that the trailing backslash on each first line
% merely continues the input to the second line
% (for convenient cut ant paste).
% Furthermore, the command |latex| can be replaced by any
% of its alternative versions such as |pdflatex|.
%
% %%%%%%%%%%%%%%%%%%%%%%%%%%%%%%%%%%%%%%%%%%%%%%%%%%%%%%%%%%%%%%%%%%%%%%%%%%%%%%
% %%%%%%%%%%%%%%%%%%%%%%%%%%%%%%%%%%%%%%%%%%%%%%%%%%%%%%%%%%%%%%%%%%%%%%%%%%%%%%
% \section{Implementation}
%\iffalse
%<*package>
%\fi
%
% This section describes the definitions file |childdoc.def|.

% The definitions cannot be loaded using |\usepackage| or |\RequirePackage|
% which has a mechanism to prevent loading a style file more than once.
% When loading the definitions by means of |\input|
% multiple instances have to be prevented manually:
%\iffalse
%This code needs to be before the `\ProvidesFile' directive
%which is defined at the beginning of this file.
%Therefore it is also placed there and commented out here.
%</package>
%<*discard>
%\fi
%    \begin{macrocode}
\ifdefined\childdocmain\endinput\fi
%    \end{macrocode}
%\iffalse
%</discard>
%<*package>
%\fi
%
% \macro{\ifchilddoc}
% \macro{\ifchilddocmanual}
% The conditional |\ifchilddoc| tells whether a
% child (true) or main (false) document is being compiled.
% The conditional |\ifchilddocmanual| tells whether
% the |\includeonly| mechanism is used (false) or
% the selection of child files must be performed manually (true).
% The definitions initialise to false:
%    \begin{macrocode}
\newif\ifchilddoc
\newif\ifchilddocmanual
%    \end{macrocode}

% \macro{\childdocname}
% \macro{\childdocjob}
% The macro |\childdocname| stores the name of the main document
% to be compiled. The macro |\childdocjob| stores the name of
% the document on which the \LaTeX{} compiler was originally invoked.
% The content of |\jobname| cannot be compared
% to filenames specified in the source due to different catcodes.
% The following code rescans |\jobname|, stores the result
% in |\childdocname| and saves a copy in |\childdocjob|:
%    \begin{macrocode}
\edef\childdocname{\scantokens\expandafter{\jobname\noexpand}}
\let\childdocjob\childdocname
%    \end{macrocode}

% \macro{\childdocdisable}
% The macro |\childdocdisable| prevents the main file
% from being processed more than once.
% At this stage, the main document command |\childdocmain|
% is assumed to be called once again where it should do nothing.
% Any subsequent call to it should prevent
% a secondary processing of the main document
% It overwrites the forwarding commands
% |\childdocof| and |\childdocforward|
% with empty macros to prevent further inclusions of the main document:
%    \begin{macrocode}
\newcommand{\childdocdisable}
{
  \renewcommand{\childdocmain}[1]{\renewcommand{\childdocmain}[1]{\endinput}}
  \renewcommand{\childdocof}[1]{}
  \renewcommand{\childdocby}[2][]{}
  \renewcommand{\childdocforward}[2][]{}
  \renewcommand{\childdocdisable}{}
}
%    \end{macrocode}

% \macro{\childdocmain}
% The macro |\childdocmain| is to be called at the top of the main file
% with nothing or the main filename (without extension) as argument.
% First, it breaks loops.
% If the argument is not empty and does not match |\childdocname|
% (which is set by the first inclusion of |childdoc.def|),
% |\ifchilddoc| is set to true, |\includeonly| is applied to the child file
% and |\jobname| is set to the main file
% (for proper handling of |.aux| files):
%    \begin{macrocode}
\newcommand{\childdocmain}[1]
{
  \childdocdisable\childdocmain{}
  \if?#1?\else
    \begingroup
      \def\childdoctmp{#1}
      \ifx\childdoctmp\childdocname
        \def\childdoctmp{}
      \else
        \def\childdoctmp
        {
          \childdoctrue
          \includeonly{\childdocname}
          \def\childdocjob{#1}
          \def\jobname{#1}
        }
      \fi
      \expandafter
    \endgroup
    \childdoctmp
  \fi
}
%    \end{macrocode}

% \macro{\childdocof}
% The command |\childdocof| redirects
% compilation to the main file |#1|.
%    \begin{macrocode}
\newcommand{\childdocof}[1]
{
  \childdocdisable
  \childdoctrue
  \includeonly{\childdocname}
  \def\jobname{#1}
  \def\childdocjob{#1}
  \input{#1}
}
%    \end{macrocode}

% \macro{\childdocby}
% The command |\childdocby| ....
%    \begin{macrocode}
\newcommand{\childdocby}[2][]
{
  \childdocdisable
  \childdoctrue
  \childdocmanualtrue
  \if?#1?\else
    \def\jobname{#2}
  \fi
  \def\childdocjob{#2}
  \input{#2}
  \endinput
}
%    \end{macrocode}

% \macro{\childdocforward}
% The command |\childdocforward| redirects
% compilation to the main file or
% (if the optional argument is given) a child file.
% Parameters are set as if the main file
% or a child file starting with |\childdocof| was compiled.
% Then compilation is handed over to the main file:
%    \begin{macrocode}
\newcommand{\childdocforward}[2][]
{
  \begingroup
    \if?#1?
      \def\childdoctmp
      {
        \def\childdocname{#2}
        \def\childdocjob{#2}
        \def\jobname{#2}
        \input{#2}
        \endinput
      }
    \else
      \def\childdoctmp
      {
        \childdocdisable
        \def\childdocname{#2}
        \childdoctrue
        \includeonly{#2}
        \def\childdocjob{#1}
        \def\jobname{#1}
        \input{#1}
        \endinput
      }
    \fi
    \expandafter
  \endgroup
  \childdoctmp
}
%    \end{macrocode}

% \macro{\childdocforwardprefix}
% The command |\childdocforwardprefix| redirects
% compilation to the main or a child file by means of a pattern.
% The prefix |#1| in the current filename is replaced by |#2|
% and the suffix of the current filename is kept
% (it is assumed that the filename does not contain the substring `|~~~|'
% which is used as a delimiter).
% Compilation is handed over to the new file by |\childdocforward|:
%    \begin{macrocode}
\newcommand{\childdocforwardprefix}[3][]
{
  \begingroup
    \def\childdocextract #2##1~~~{\def\childdoctmp{\childdocforward[#1]{#3##1}}}
    \expandafter\childdocextract\childdocname~~~
    \expandafter
  \endgroup
  \childdoctmp
}
%    \end{macrocode}

% \macro{\childdoc}
% The deprecated macro |\childdoc| is a legacy version of |\childdocmain|:
%    \begin{macrocode}
\newcommand{\childdoc}{\childdocmain}
%    \end{macrocode}

% \macro{\childdocredirect}
% The deprecated macro |\childdocredirect| is a legacy version
% of |\childdocforward| and |\childdocforwardprefix|:
%    \begin{macrocode}
\newcommand{\childdocredirect}[2][]
{
  \begingroup
    \if?#1?
      \def\childdoctmp{\childdocforward{#2}}
    \else
      \def\childdoctmp{\childdocforwardprefix{#1}{#2}}
    \fi
    \expandafter
  \endgroup
  \childdoctmp
}
%    \end{macrocode}

%\iffalse
%</package>
%\fi
%
\endinput

\childdocforwardprefix[cdocsamp]{cdocsfn}{cdocsch}
%    \end{macrocode}

%\iffalse
%</samplefinal>
%\fi
%
% %%%%%%%%%%%%%%%%%%%%%%%%%%%%%%%%%%%%%%
% \paragraph{Command Line Processing.}
%
% The following three command lines generate the output files
% |cdocscld|, |cdocscl1| and |cdocscl2|
% which should be identical to
% |cdocsdrf|, |cdocsch1| and |cdocsfn2|, respectively:
% \begin{center}
% \begin{tabular}{l}
% |latex -jobname cdocscld \|\\
% |  "\def\version{draft}% \iffalse
%
% childdoc.dtx Copyright (C) 2017-2018 Niklas Beisert
%
% This work may be distributed and/or modified under the
% conditions of the LaTeX Project Public License, either version 1.3
% of this license or (at your option) any later version.
% The latest version of this license is in
%   http://www.latex-project.org/lppl.txt
% and version 1.3 or later is part of all distributions of LaTeX
% version 2005/12/01 or later.
%
% This work has the LPPL maintenance status `maintained'.
%
% The Current Maintainer of this work is Niklas Beisert.
%
% This work consists of the files childdoc.dtx and childdoc.ins
% and the derived files childdoc.def and cdocsamp.tex with
% cdocsch1.tex, cdocsch2.tex, cdocsdrf.tex, cdocsfn1.tex, cdocsfn2.tex.
%
%<package>\ifdefined\childdocmain\endinput\fi
%<package>\ProvidesFile{childdoc.def}[2018/12/30 v2.0 child document driver]
%<samplemain>\ProvidesFile{cdocsamp.tex}[2018/12/30 v2.0 sample for childdoc]
%<*driver>
%\ProvidesFile{childdoc.drv}[2018/12/30 v2.0 childdoc reference manual file]
\PassOptionsToClass{10pt,a4paper}{article}
\documentclass{ltxdoc}

\usepackage[margin=35mm]{geometry}
\usepackage{hyperref}
\usepackage{hyperxmp}
\usepackage[usenames]{color}

\hypersetup{colorlinks=true}
\hypersetup{pdfstartview=FitH}
\hypersetup{pdfpagemode=UseNone}
\hypersetup{pdfsource={}}
\hypersetup{pdflang={en-UK}}
\hypersetup{pdfcopyright={Copyright 2017-2018 Niklas Beisert.
  This work may be distributed and/or modified under the
  conditions of the LaTeX Project Public License, either version 1.3
  of this license or (at your option) any later version.}}
\hypersetup{pdflicenseurl={http://www.latex-project.org/lppl.txt}}
\hypersetup{pdfcontactaddress={ETH Zurich, ITP, HIT K,
  Wolfgang-Pauli-Strasse 27}}
\hypersetup{pdfcontactpostcode={8093}}
\hypersetup{pdfcontactcity={Zurich}}
\hypersetup{pdfcontactcountry={Switzerland}}
\hypersetup{pdfcontactemail={nbeisert@itp.phys.ethz.ch}}
\hypersetup{pdfcontacturl={http://people.phys.ethz.ch/\xmptilde nbeisert/}}

\newcommand{\secref}[1]{\hyperref[#1]{section \ref*{#1}}}

\parskip1ex
\parindent0pt
\let\olditemize\itemize
\def\itemize{\olditemize\parskip0pt}

\begin{document}

\title{The \textsf{childdoc} Package}
\hypersetup{pdftitle={The childdoc Package}}
\author{Niklas Beisert\\[2ex]
  Institut f\"ur Theoretische Physik\\
  Eidgen\"ossische Technische Hochschule Z\"urich\\
  Wolfgang-Pauli-Strasse 27, 8093 Z\"urich, Switzerland\\[1ex]
  \href{mailto:nbeisert@itp.phys.ethz.ch}
  {\texttt{nbeisert@itp.phys.ethz.ch}}}
\hypersetup{pdfauthor={Niklas Beisert}}
\hypersetup{pdfsubject={Manual for the LaTeX2e Package childdoc}}
\date{30 December 2018, \textsf{v2.0}}
\maketitle

\begin{abstract}\noindent
\textsf{childdoc} is a \LaTeXe{} package
that enables the direct compilation
of document sections included by |\include|
to individual files.
\end{abstract}

\begingroup
\parskip0ex
\tableofcontents
\endgroup

%%%%%%%%%%%%%%%%%%%%%%%%%%%%%%%%%%%%%%%%%%%%%%%%%%%%%%%%%%%%%%%%%%%%%%%%%%%%%%%%
%%%%%%%%%%%%%%%%%%%%%%%%%%%%%%%%%%%%%%%%%%%%%%%%%%%%%%%%%%%%%%%%%%%%%%%%%%%%%%%%
\section{Introduction}

\LaTeX{} provides a mechanism to structure a large document (such as a book)
into a main file and several child files (containing the chapters)
using the |\include| command.
This mechanism is beneficial for documents
which span hundreds of pages in order to
make the source file(s) more manageable.
Moreover, compilation can be restricted to
selected child files by means of the |\includeonly| command.
The latter feature can be used to reduce the compilation time while editing
(this was significantly more useful in the earlier days of \LaTeX{})
or to generate a smaller document which is easier to navigate.
Another application of |\includeonly| is to generate
documents consisting of selected parts of the complete document.

However, there are a few drawbacks of the plain |\include| mechanism:
\begin{itemize}
\item
The child files cannot be compiled on their own,
they can only be compiled via the main file.
A naive editing environment
(such as a text editor with an option
to have the current file processed by \LaTeX)
may require one to switch to the main file before compiling;
attempting to compile the child file produces errors.
\item
The main file must be modified (each time)
to adjust the |\includeonly| command
to the present needs. This easily leaves the main file in a messy state.
\item
The generated document will always carry the filename
of the main document. This is inconvenient if
several child files are to be compiled and
to be kept for distribution.
\end{itemize}

The present package provides a simple interface
to make child files individually compilable by \LaTeX{}.
Compiling a child file then has the same effect as compiling
the main file with an |\includeonly| command
to select the appropriate child.
Moreover the generated document will carry the name of the child
rather than the main file.
This resolves all three above issues.

This feature is meant to make the editing of books,
thesis documents and lecture notes somewhat more convenient.
However, the package can also be used efficiently for
composing a series of documents (such as exercise sheets)
which are typically distributed individually.
It then assists the author in generating the individual documents
(potentially in different versions)
as well as a document containing the collected series.
Another application is in developing style files
or other kinds of included material
where compilation of the style file could redirect
to a sample or test file.

%%%%%%%%%%%%%%%%%%%%%%%%%%%%%%%%%%%%%%%%%%%%%%%%%%%%%%%%%%%%%%%%%%%%%%%%%%%%%%%%
%%%%%%%%%%%%%%%%%%%%%%%%%%%%%%%%%%%%%%%%%%%%%%%%%%%%%%%%%%%%%%%%%%%%%%%%%%%%%%%%
\section{Usage}

First of all, the package \textsf{childdoc} is \emph{not} a standard
\LaTeXe{} |.sty| style file! Therefore it needs to be invoked in
a non-standard way.

%%%%%%%%%%%%%%%%%%%%%%%%%%%%%%%%%%%%%%%%%%%%%%%%%%%%%%%%%%%%%%%%%%%%%%%%%%%%%%%%
\subsection{Included Files}
\label{sec:include}

%%%%%%%%%%%%%%%%%%%%%%%%%%%%%%%%%%%%%%%%
\DescribeMacro{\childdocmain}
To use the package, add the commands
\begin{center}
\begin{tabular}{l}
|\input{childdoc.def}|\\
|\childdocmain{}|\\
\end{tabular}
\end{center}
at the very top of the main \LaTeX{} file,
in particular \emph{before} the |\documentclass| statement!
The argument of |\childdocmain| should be left empty
(but it must be present).

%%%%%%%%%%%%%%%%%%%%%%%%%%%%%%%%%%%%%%%%
\DescribeMacro{\childdocof}
Furthermore, add the commands
\begin{center}
\begin{tabular}{l}
|\input{childdoc.def}|\\
|\childdocof{|\textit{main}|}|\\
\end{tabular}
\end{center}
at the top of every child file \textit{child}
which is included by |\include{|\textit{child}|}|
from within the main file
(or at least for those files to be compiled individually).
The argument \textit{main} must be the filename of the main file.

There are a couple of
considerations in setting up the main and child documents:

%%%%%%%%%%%%%%%%%%%%%%%%%%%%%%%%%%%%%%%%
\paragraph{Restrictions.}

Please note the following restrictions:
\begin{itemize}
\item
|\childdocmain| must be called with one argument \textit{main}
to ensure compatibility with earlier version of the package.
It must either be empty (|\childdocmain{}|)
or precisely match the filename of the main file in which it is specified.
See \secref{sec:detection} for further information.
\item
The filename \textit{main} must be specified without the |.tex| extension.
\item
The filename \textit{main} is case sensitive
(even in case-insensitive file systems)
due to internal string comparison.
\item
The argument \textit{main} should be fully expanded, it cannot be a macro.
\item
Subdirectories and special characters should be avoided in filenames.
\item
The command |\childdocmain{|\textit{main}|}| must be followed by a whitespace.
It should not be followed immediately by another command
or by a comment mark `|%|'.
This is because the \TeX{} parser reads the token immediately following
the argument of |\childdocmain| and puts it
at the beginning of every child section;
however, a white\-space is ignored.
\end{itemize}

%%%%%%%%%%%%%%%%%%%%%%%%%%%%%%%%%%%%%%%%
\paragraph{Content of Main File.}

It is advisable to place all content in the child files included by |\include|.
Any output contained in the main file will appear in all child documents
unless suppressed manually;
it cannot be suppressed automatically by the |\includeonly| directive
and thus should normally be avoided.
A method to include some content in the main file
by means of conditional processing is described in \secref{sec:conditional}.

%%%%%%%%%%%%%%%%%%%%%%%%%%%%%%%%%%%%%%%%
\paragraph{Page Numbering.}

When only a part of the document is compiled,
the appropriate numbering of pages
(as well as other status parameters)
is determined from the |.aux| files.
The latter contain information from previous passes.
However this information needs to propagate through
all intermediate child documents.
Therefore the page numbering in child documents may well
be inconsistent until the complete document is compiled at least once.

A useful (if unconventional) way to always ensure a consistent
page numbering is to restart the numbering in each child document
and denote the pages by `\textit{child}|.|\textit{page}'
where \textit{child} represents the chapter/section number of the child file.
This can be achieved by the command
|\numberwithin{page}{|\textit{child}|}|
of the \textsf{amsmath} package
where \textit{child} can be |chapter| or |section|
depending on the chosen structuring.
Alternatively, one can modify the macro |\thepage| appropriately
and reset the counter |page| at the start of each child file.

%%%%%%%%%%%%%%%%%%%%%%%%%%%%%%%%%%%%%%%%%%%%%%%%%%%%%%%%%%%%%%%%%%%%%%%%%%%%%%%%
\subsection{Conditional Processing}
\label{sec:conditional}

The package provides a mechanism to compile different versions
of a document. To customise the versions further some conditional processing
can come in handy to distinguish which version is being compiled.
The package provides two macros to describe the compilation context:

%%%%%%%%%%%%%%%%%%%%%%%%%%%%%%%%%%%%%%%%
\DescribeMacro{\ifchilddoc}
The conditional |\ifchilddoc| distinguishes between the compilation of
child documents and the main document:
%
\begin{center}
|\ifchilddoc |\textit{child-code}| |[|\||else |\textit{main-code}]| \||fi|
\end{center}

%%%%%%%%%%%%%%%%%%%%%%%%%%%%%%%%%%%%%%%%
\DescribeMacro{\childdocname}
\DescribeMacro{\childdocjob}
The macro |\childdocname| contains the filename (without extension)
of the main or child file being processed.
Note that |\childdocjob| will always contain the name of the main file.

%%%%%%%%%%%%%%%%%%%%%%%%%%%%%%%%%%%%%%%%
\paragraph{Title Page.}

Conditional processing can be used to include a title or banner page
in the main document when proper precautions are taken.
Importantly, the code in the main file should ensure that the page counter
(as well as other status parameters which are stored in the |.aux| files)
takes the same value after the conditional processing.
Otherwise the page numbers may take divergent values
depending on which part is compiled.

For example, a title page could be declared by:
%
\begin{center}
\begin{tabular}{l}
|\ifchilddoc\||else|\\
|\addtocounter{page}{-1}|\\
\textit{code for title page}\\
|\newpage|\\
|\||fi|
\end{tabular}
\end{center}
%
A banner page for the child documents can be generated by:
%
\begin{center}
\begin{tabular}{l}
|\ifchilddoc|\\
|\addtocounter{page}{-1}|\\
\textit{code for banner page}\\
|\newpage|\\
|\||fi|
\end{tabular}
\end{center}
%
Here one could write a message such as:
\begin{center}
|This is the part \childdocname{} of \childdocjob{}.|
\end{center}

%%%%%%%%%%%%%%%%%%%%%%%%%%%%%%%%%%%%%%%%%%%%%%%%%%%%%%%%%%%%%%%%%%%%%%%%%%%%%%%%
\subsection{Flags}
\label{sec:flags}

The package makes it easy to generate different versions
of the main or child documents.
To this end compilation flags can be defined
and assigned different default values.
They will be particularly useful in conjunction
with the forwarding mechanism described in \secref{sec:forward}.

For example, it may be useful to have a flag |\version|
which can be set to |draft| or |final|.
The document source will contain some conditional code
depending on the value of |\version|.
Suppose further, the flag should default to |final| for the main file
and to |draft| for child files
which is a natural assignment for editing the document.
This is achieved by placing the following code
in the preamble of the main document
(below the |\childdocmain| directive):
%
\begin{center}
\begin{tabular}{l}
|\ifchilddoc|\\
|\providecommand{\version}{draft}|\\
|\||else|\\
|\providecommand{\version}{final}|\\
|\||fi|
\end{tabular}
\end{center}
%
The definition by |\providecommand| makes sure
that previous definitions are not overwritten.
Further statements |\providecommand{\version}{...}|
can thus be added before the above code to override it.

For the main file, one might add a line
(between |\childdocmain| and the above block)
%
\begin{center}
|%\ifchilddoc\||else\providecommand{\version}{draft}\||fi|
\end{center}
%
which can be uncommented to produce a draft version.
Likewise one can add a line to the very top of a child file
(above the |\childdocof{|\textit{main}|}| directive)
%
\begin{center}
|%\providecommand{\version}{final}|
\end{center}
%
which can be uncommented to produce the final version of this child document.

%%%%%%%%%%%%%%%%%%%%%%%%%%%%%%%%%%%%%%%%%%%%%%%%%%%%%%%%%%%%%%%%%%%%%%%%%%%%%%%%
\subsection{Forwarding}
\label{sec:forward}

Different versions of the main or child documents
using compilation flags as described in \secref{sec:flags}
can be (permanently) stored in different files
for convenient compilation, viewing and distribution.
To this end, the package defines a command
to pass on compilation to a different file:

%%%%%%%%%%%%%%%%%%%%%%%%%%%%%%%%%%%%%%%%
\DescribeMacro{\childdocforward}
The command |\childdocforward| redirects processing to
another source file:
%
\begin{center}
\begin{tabular}{l}
|\input{childdoc.def}|\\
|\childdocforward[|\textit{main}|]{|\textit{dest}|}|\\
\end{tabular}
\end{center}
%
The argument \textit{dest} is the destination file
(without extension).
It should be the main file or one of the child files.
Note that further \textsf{childdoc} directives
such as |\childdocof| and |\childdocforward|
in the indicated file will be processed in this form.
The optional argument \textit{main}
passes on directly to the main file \textit{main}
while pretending to compile the child \textit{dest}.
This form behaves as if \textit{dest}
issues |\childdocof{|\textit{main}|}| right away,
and no further \textsf{childdoc} directives will be processed.

%%%%%%%%%%%%%%%%%%%%%%%%%%%%%%%%%%%%%%%%
\DescribeMacro{\...prefix}
In the alternative form |\childdocforwardprefix|,
%
\begin{center}
\begin{tabular}{l}
|\input{childdoc.def}|\\
|\childdocforwardprefix[|\textit{main}|]{|\textit{prefix}|}{|\textit{dest}|}|
\end{tabular}
\end{center}
%
the destination file is determined by a pattern
depending on the current file:
To make this work, the current file must be called
`{\textit{prefix}\hspace{0.2em}\textit{suffix}}'
with \textit{prefix} matching precisely the argument.
Processing is then passed on to the file
`{\textit{dest}\hspace{0.2em}\textit{suffix}}'.
Surely, the same effect is achieved by
directly specifying the
argument `{\textit{dest}\hspace{0.2em}\textit{suffix}}'
in the first form.
However, that requires to set up a different file
for each child. With the alternative form of the command
all these files can have exactly the same content
which simplifies setting them up and maintaining them.

For example, the following file |draft.tex|
with a compilation flag |\version| as described in \secref{sec:flags}
compiles the main document as a draft:
%
\begin{center}
\begin{tabular}{l}
|\def\version{draft}|\\
|\input{childdoc.def}|\\
|\childdocforward{|\textit{main}|}|
\end{tabular}
\end{center}
%
Likewise, the following files |final|\textit{nn}|.tex|
compile the final version of the child document
|child|\textit{nn}|.tex|:
%
\begin{center}
\begin{tabular}{l}
|\def\version{final}|\\
|\input{childdoc.def}|\\
|\childdocforwardprefix{final}{child}|
\end{tabular}
\end{center}
%

Note that when several versions of a main file and/or of each child file
are to be generated, it may be convenient to set up a |Makefile| or
shell script to automatise the process.

%%%%%%%%%%%%%%%%%%%%%%%%%%%%%%%%%%%%%%%%%%%%%%%%%%%%%%%%%%%%%%%%%%%%%%%%%%%%%%%%
\subsection{Command Line Processing}
\label{sec:commandline}

The effect of redirection files can also be achieved by invoking
the \LaTeX{} compiler with a more elaborate command line.
Most conveniently this should be done as part
of a shell script or a |Makefile|.

When using \textsf{childdoc} in the main file, the following
command lines effectively perform a redirection
(note that depending on the shell being used,
backslashes may have to be doubled: `|\|' $\to$ `|\\|'):
%
\begin{center}
|... -jobname "|\textit{target}|" |\\|"|[\textit{flags}]%
|\input{childdoc.def}\childdocforward[|\textit{main}|]{|\textit{dest}|}"|
\end{center}
%
Here \textit{target} is the name of the output file,
\textit{main} is the name of the main file
and \textit{dest} is the name of the main or child file to be processed
(all filenames without extensions).
The optional argument \textit{main} can be omitted
if \textit{main} matches \textit{dest}.
Optionally, compilation \textit{flags} can be defined via |\def| commands.
This command line makes the \TeX{} engine believe
it is compiling the file \textit{target}
whose content is specified as the latter parameter.
The provided code then forwards the processing to
\textit{main} or \textit{dest} as described in \secref{sec:forward}.

%%%%%%%%%%%%%%%%%%%%%%%%%%%%%%%%%%%%%%%%%%%%%%%%%%%%%%%%%%%%%%%%%%%%%%%%%%%%%%%%
\subsection{Include by Input}
\label{sec:input}

Including child documents by |\include| has some restrictions by design.
Most notably, the content of a child document always occupies
its own set of pages; pages cannot be shared between child documents.
Usually, this behaviour makes perfect sense
because each child document contain an essential part of the document.
However, in some situations it may be desirable to compose
a document from a collection of parts
without having mandatory page breaks between then.
For this case, the package
provides a mechanism to include parts
by |\input| which can also be processed individually.
However, by construction this mechanism
requires manual handling of the content to be output.

%%%%%%%%%%%%%%%%%%%%%%%%%%%%%%%%%%%%%%%%
\DescribeMacro{\ifchilddocmanual}
The main file should be prepared as usual, see \secref{sec:include}.
However, the document body must make a distinction
between processing of an individual part and of the main document, e.g.:
%
\begin{center}
\begin{tabular}{l}
|\ifchilddocmanual|\\
|\input{\childdocname}|\\
|\||else|\\
\textit{document body with }|\input{|\textit{part}|}|\\
|\||fi|
\end{tabular}
\end{center}
%
The conditional |\ifchilddocmanual| is true whenever
a part to be included by |\input| is being compiled,
and the name of the part is stored in |\childdocname|.

%%%%%%%%%%%%%%%%%%%%%%%%%%%%%%%%%%%%%%%%
\DescribeMacro{\childdocby}
Each part to be included by |\input| should start with:
%
\begin{center}
\begin{tabular}{l}
|\input{childdoc.def}|\\
|\childdocby{|\textit{main}|}|\\
\end{tabular}
\end{center}
%
The directive |\childdocby| is similar to |\childdocof|
described in \secref{sec:include},
but the subsequent selection of content must be done manually.
To that end, both |\ifchilddoc| and |\ifchilddocmanual|
will be true upon processing of a part,
and the name of the part is stored in |\childdocname|.
Note that |\jobname| will be set to the filename of the current part
so that each part receives an individual |.aux| file
that does not interfere with the |.aux| file(s) of the main document.
This behaviour can be altered by the alternative form
|\childdocby[*]{|\textit{main}|}| (with a non-empty optional argument)
which uses the |.aux| file of the main document
by setting |\jobname| to \textit{main}.

%%%%%%%%%%%%%%%%%%%%%%%%%%%%%%%%%%%%%%%%%%%%%%%%%%%%%%%%%%%%%%%%%%%%%%%%%%%%%%%%
\subsection{Driver Development}
\label{sec:driver}

The \textsf{childdoc} mechanism can also be use for the development
of definition files such as \LaTeX{} styles or classes.
This case differs from the above setup with multiple parts
included by |\include| in that no |\includeonly| should be invoked.
This can be achieved by starting the include file
(before |\ProvidesPackage|) with:
%
\begin{center}
\begin{tabular}{l}
|\input{childdoc.def}|\\
|\childdocforward{|\textit{main}|}|\\
\end{tabular}
\end{center}
%
or alternatively with:
%
\begin{center}
\begin{tabular}{l}
|\input{childdoc.def}|\\
|\childdocby{|\textit{main}|}|\\
\end{tabular}
\end{center}
%
Both forms have slightly different effects as described above.
The main file is prepared as usual, see \secref{sec:include}.

%%%%%%%%%%%%%%%%%%%%%%%%%%%%%%%%%%%%%%%%%%%%%%%%%%%%%%%%%%%%%%%%%%%%%%%%%%%%%%%%
\subsection{Legacy Detection}
\label{sec:detection}

The directive |\childdocmain| in the main file can detect
whether the complete document or merely a child is to be compiled
even without using the directive |\childdocof|.
This method is deprecated because it is less robust
and there is no compelling reason to use it;
it is merely provided for backward compatibility
and it may be removed in future versions.

If the detection mechanism is to be used,
it is mandatory to correctly specify
the filename of the main file as the argument of |\childdocmain|:
%
\begin{center}
\begin{tabular}{l}
|\input{childdoc.def}|\\
|\childdocmain{|\textit{main}|}|\\
\end{tabular}
\end{center}
%
If |\jobname| does not match the argument \textit{main} of |\childdocmain|,
it is assumed that |\jobname| points to the child file to be compiled.
When using |\childdocmain| with the main file specified as argument,
it suffices to start a child file
with just |\input{|\textit{main}|}|
without loading of the package and using |\childdocof|.
If instead all processing is done
with the appropriate \textsf{childdoc} directives,
the argument of \textit{main} of |\childdocmain| can be empty.

An alternative version of the command line processing described
in \secref{sec:commandline} using the detection mechanism reads:
%
\begin{center}
|... -jobname "|\textit{target}|" "|[\textit{flags}]%
[|\def\jobname{|\textit{dest}|}|]|\input{|\textit{main}|}"|
\end{center}

%%%%%%%%%%%%%%%%%%%%%%%%%%%%%%%%%%%%%%%%%%%%%%%%%%%%%%%%%%%%%%%%%%%%%%%%%%%%%%%%
\subsection{Manual Code}
\label{sec:manual}

In case one cannot be certain whether the definitions file |childdoc.def|
is installed on the target \TeX{} distribution
and one prefers not to ship it,
it is conceivable to paste a few relevant commands into the sources.

To that end, drop all statements |\input{childdoc.def}|
and perform the replacements as outlined below.
Instead of |\childdocmain{|\textit{main}|}| add the following code
to the top of the main file:
%
\begin{center}
\begin{tabular}{l}
|\||ifdefined\childdocname\endinput\||fi\newif\ifchilddoc|\\
|\edef\childdocname{\scantokens\expandafter{\jobname\noexpand}}|\\
|\def\childdocmain{|\textit{main}|}\||ifx\childdocmain\childdocname\||else|\\
|\childdoctrue\includeonly{\childdocname}\let\jobname\childdocmain\||fi|\\
\end{tabular}
\end{center}
%
Instead of |\childdocof{|\textit{main}|}| just include the main file
at the top of each child file:
%
\begin{center}
|\input{|\textit{main}|}|
\end{center}
%
A simple redirection |\childdocforward{|\textit{dest}|}| is achieved by:
%
\begin{center}
|\def\jobname{|\textit{dest}|}\input{\jobname}|
\end{center}
%
The redirection with prefix
|\childdocforwardprefix[|\textit{prefix}|]{|\textit{dest}|}|
is accomplished by:
%
\begin{center}
\begin{tabular}{l}
|{\edef\jobname{\scantokens\expandafter{\jobname\noexpand}}|\\
|\def\redirectjob |\textit{prefix}|#1~~~{\gdef\jobname{|\textit{dest}|#1}}|\\
|\expandafter\redirectjob\jobname~~~}\input{\jobname}|
\end{tabular}
\end{center}

In an alternative approach,
child documents can be compiled by a specific command line
without additional code or specific definitions:
%
\begin{center}
|... -jobname "|\textit{target}|" "|[\textit{flags}]%
|\includeonly{|\textit{dest}|}\input{|\textit{main}|}"|
\end{center}
%

%%%%%%%%%%%%%%%%%%%%%%%%%%%%%%%%%%%%%%%%%%%%%%%%%%%%%%%%%%%%%%%%%%%%%%%%%%%%%%%%
%%%%%%%%%%%%%%%%%%%%%%%%%%%%%%%%%%%%%%%%%%%%%%%%%%%%%%%%%%%%%%%%%%%%%%%%%%%%%%%%
\section{Information}

%%%%%%%%%%%%%%%%%%%%%%%%%%%%%%%%%%%%%%%%%%%%%%%%%%%%%%%%%%%%%%%%%%%%%%%%%%%%%%%%
\subsection{Copyright}

Copyright \copyright{} 2017--2018 Niklas Beisert

This work may be distributed and/or modified under the
conditions of the \LaTeX{} Project Public License, either version 1.3
of this license or (at your option) any later version.
The latest version of this license is in
  \url{http://www.latex-project.org/lppl.txt}
and version 1.3 or later is part of all distributions of \LaTeX{}
version 2005/12/01 or later.

This work has the LPPL maintenance status `maintained'.

The Current Maintainer of this work is Niklas Beisert.

This work consists of the files |README.txt|, |childdoc.ins| and |childdoc.dtx|
as well as the derived files |childdoc.def|, |cdocsamp.tex|
with |cdocsch1.tex|, |cdocsch2.tex|, |cdocspt3.tex|, |cdocspt4.tex|,
|cdocsdrf.tex|, |cdocsfn1.tex|, |cdocsfn2.tex|
as well as |childdoc.pdf|.

%%%%%%%%%%%%%%%%%%%%%%%%%%%%%%%%%%%%%%%%%%%%%%%%%%%%%%%%%%%%%%%%%%%%%%%%%%%%%%%%
\subsection{Files and Installation}

The package consists of the files:
%
\begin{center}
\begin{tabular}{ll}
    |README.txt|   & readme file \\
    |childdoc.ins| & installation file \\
    |childdoc.dtx| & source file \\
    |childdoc.def| & definition file \\
    |cdocsamp.tex| & sample main file \\
    |cdocsch1.tex| & sample include file \\
    |cdocsch2.tex| & sample include file \\
    |cdocspt3.tex| & sample part file \\
    |cdocspt4.tex| & sample part file \\
    |cdocsdrf.tex| & sample redirection file \\
    |cdocsfn1.tex| & sample redirection file \\
    |cdocsfn2.tex| & sample redirection file \\
    |childdoc.pdf| & manual
\end{tabular}
\end{center}
%
The distribution consists of the files
|README.txt|, |childdoc.ins| and |childdoc.dtx|.
%
\begin{itemize}
\item
Run (pdf)\LaTeX{} on |childdoc.dtx|
to compile the manual |childdoc.pdf| (this file).
\item
Run \LaTeX{} on |childdoc.ins| to create the definitions file |childdoc.def|
and the sample |cdocsamp.tex| with include files
|cdocsch1.tex|, |cdocsch2.tex|, |cdocspt3.tex|, |cdocspt4.tex|,
|cdocsdrf.tex|, |cdocsfn1.tex|, |cdocsfn2.tex|.
Then copy the file |childdoc.def| to an appropriate directory of your \LaTeX{}
distribution, e.g.\ \textit{texmf-root}|/tex/latex/childdoc|.
\end{itemize}

%%%%%%%%%%%%%%%%%%%%%%%%%%%%%%%%%%%%%%%%%%%%%%%%%%%%%%%%%%%%%%%%%%%%%%%%%%%%%%%%
\subsection{Related CTAN Packages}

There are several other packages which offer a similar functionality:
%
\begin{itemize}
\item
The packages
\href{http://ctan.org/pkg/docmute}{\textsf{docmute}},
\href{http://ctan.org/pkg/includex}{\textsf{includex}} and
\href{http://ctan.org/pkg/standalone}{\textsf{standalone}}
provide commands to include only the document body of
a child file thus allowing both files to be compiled individually.
\item
The packages \href{http://ctan.org/pkg/subdocs}{\textsf{subdocs}}
and \href{http://ctan.org/pkg/subfiles}{\textsf{subfiles}}
provide structures in which the main and child documents can be
encapsulated and allowing them to be compiled individually.
The inclusion mechanism is different from the conventional |\include|.
\item
The package \href{http://ctan.org/pkg/combine}{\textsf{combine}}
is an elaborate solution to combine several documents into one.
\end{itemize}
%
See also the CTAN topic \href{http://ctan.org/topic/subdocs}{\textsf{subdocs}}
for further related packages.
The present package differs from the above solutions in that
a document structure constructed with the conventional |\include| mechanism
just needs two extra commands at the top of every file
such that all constituent files can be compiled individually.

%%%%%%%%%%%%%%%%%%%%%%%%%%%%%%%%%%%%%%%%%%%%%%%%%%%%%%%%%%%%%%%%%%%%%%%%%%%%%%%%
%\subsection{Feature Suggestions}
%
%The following is a list of features which may be useful for future
%versions of this package:
%%
%\begin{itemize}
%\item
%\ldots
%\end{itemize}

%%%%%%%%%%%%%%%%%%%%%%%%%%%%%%%%%%%%%%%%%%%%%%%%%%%%%%%%%%%%%%%%%%%%%%%%%%%%%%%%
\subsection{Revision History}

%%%%%%%%%%%%%%%%%%%%%%%%%%%%%%%%%%%%%%%%
\paragraph{v2.0:} 2018/12/30

\begin{itemize}
\item
immediate forward processing
\item
added |\childdocby| mechanism
\item
manual restructured
\end{itemize}

%%%%%%%%%%%%%%%%%%%%%%%%%%%%%%%%%%%%%%%%
\paragraph{v1.6:} 2018/01/17

\begin{itemize}
\item
application for development of include files
\item
corrections to manual
\end{itemize}

%%%%%%%%%%%%%%%%%%%%%%%%%%%%%%%%%%%%%%%%
\paragraph{v1.5:} 2017/05/21

\begin{itemize}
\item
more complete structuring introduced
\item
|\childdocof| introduced
\item
|\childdoc| renamed to |\childdocmain|
\item
|\childredirect| renamed to |\childdocforward| and |\childdocforwardprefix|
and functionality expanded
\end{itemize}

%%%%%%%%%%%%%%%%%%%%%%%%%%%%%%%%%%%%%%%%
\paragraph{v1.0:} 2017/04/27

\begin{itemize}
\item
manual and install package
\item
first version published on CTAN
\end{itemize}

%%%%%%%%%%%%%%%%%%%%%%%%%%%%%%%%%%%%%%%%
\paragraph{v0.6:} 2017/04/26

\begin{itemize}
\item
redirection mechanism added
\end{itemize}

%%%%%%%%%%%%%%%%%%%%%%%%%%%%%%%%%%%%%%%%
\paragraph{v0.5:} 2017/04/26

\begin{itemize}
\item
functionality in definition file
\end{itemize}


%%%%%%%%%%%%%%%%%%%%%%%%%%%%%%%%%%%%%%%%%%%%%%%%%%%%%%%%%%%%%%%%%%%%%%%%%%%%%%%%
%%%%%%%%%%%%%%%%%%%%%%%%%%%%%%%%%%%%%%%%%%%%%%%%%%%%%%%%%%%%%%%%%%%%%%%%%%%%%%%%
%%%%%%%%%%%%%%%%%%%%%%%%%%%%%%%%%%%%%%%%%%%%%%%%%%%%%%%%%%%%%%%%%%%%%%%%%%%%%%%%
\appendix

\settowidth\MacroIndent{\rmfamily\scriptsize 000\ }

 \DocInput{childdoc.dtx}

\end{document}
%</driver>
% \fi
%
% %%%%%%%%%%%%%%%%%%%%%%%%%%%%%%%%%%%%%%%%%%%%%%%%%%%%%%%%%%%%%%%%%%%%%%%%%%%%%%
% %%%%%%%%%%%%%%%%%%%%%%%%%%%%%%%%%%%%%%%%%%%%%%%%%%%%%%%%%%%%%%%%%%%%%%%%%%%%%%
% \section{Sample}
%\iffalse
%<*samplemain>
%\fi
%
% The following presents a sample document
% with two chapters, two parts, a title page,
% a compile flag as well as three forwarding files to set the flag.
% It consists of eight |.tex| files:
% \begin{center}
% \begin{tabular}{ll}
% |cdocsamp.tex|&main file\\
% |cdocsch1.tex|&include file for chapter 1\\
% |cdocsch2.tex|&include file for chapter 2\\
% |cdocspt3.tex|&include file for part 3\\
% |cdocspt4.tex|&include file for part 4\\
% |cdocsdrf.tex|&forwarding file for main file in draft mode\\
% |cdocsfi1.tex|&forwarding file for final version of chapter 1\\
% |cdocsfi2.tex|&forwarding file for final version of chapter 2\\
% \end{tabular}
% \end{center}
% Each of the eight files can be compiled directly by the \LaTeX{} compiler.
%
% %%%%%%%%%%%%%%%%%%%%%%%%%%%%%%%%%%%%%%
% \paragraph{Main File.}
%
% The main file is called |cdocsamp.tex|.
%
% Load the \textsf{childdoc} definitions and
% declare the filename for the main document:
%    \begin{macrocode}
\input{childdoc.def}
\childdocmain{}
%    \end{macrocode}

% Optional override for |\version| flag:
%    \begin{macrocode}
%%\ifchilddoc\else\providecommand{\version}{draft}\fi
%    \end{macrocode}

% Define the default values for the |\version| flag
% (|final| for the main file and |draft| for childs):
%    \begin{macrocode}
\ifchilddoc
\providecommand{\version}{draft}
\else
\providecommand{\version}{final}
\fi
%    \end{macrocode}

% Load the standard document class:
%    \begin{macrocode}
\documentclass[12pt]{article}
%    \end{macrocode}

% Start the document body:
%    \begin{macrocode}
\begin{document}
%    \end{macrocode}

% Declare a title page.
% Print title, part of document being processed and version flag:
%    \begin{macrocode}
\addtocounter{page}{-1}
\begin{center}
{\LARGE\bfseries{}childdoc example\par}
\vspace{1cm}
\ifchilddoc
\ifchilddocmanual part\else chapter\fi:
`\childdocname' of `\childdocjob'\par
\else
main document: `\childdocjob'\par
\fi
version: \version\par
\end{center}
\newpage
%    \end{macrocode}

% Manually include selected file,
% otherwise process as usual:
%    \begin{macrocode}
\ifchilddocmanual
\section*{part `\childdocname'}
\input{\childdocname}
\else
%    \end{macrocode}

% Include the two chapters:
%    \begin{macrocode}
\include{cdocsch1}
\include{cdocsch2}
%    \end{macrocode}

% Include the two parts unless only chapters should be displayed:
%    \begin{macrocode}
\ifchilddoc\else
\section{part three}
\input{cdocspt3}
\section{part four}
\input{cdocspt4}
\fi
%    \end{macrocode}

% Process as usual until here:
%    \begin{macrocode}
\fi
%    \end{macrocode}

% End of document body:
%    \begin{macrocode}
\end{document}
%    \end{macrocode}
%\iffalse
%</samplemain>
%\fi
%
% %%%%%%%%%%%%%%%%%%%%%%%%%%%%%%%%%%%%%%
% \paragraph{Chapter Include Files.}
%
% The include files are called |cdocsch1.tex| and |cdocsch2.tex|.
%
%\iffalse
%<*samplechap1|samplechap2>
%\fi

% Optional override for |\version| flag:
%    \begin{macrocode}
%%\providecommand{\version}{final}
%    \end{macrocode}

% Include the main document:
%    \begin{macrocode}
\input{childdoc.def}
\childdocof{cdocsamp}
%    \end{macrocode}

%\iffalse
%</samplechap1|samplechap2>
%\fi
%
%\iffalse
%<*samplechap1>
%\fi
% Some text for chapter 1:
%    \begin{macrocode}
\section{one}
some text in chapter one
%    \end{macrocode}

%\iffalse
%</samplechap1>
%\fi
% Some text for chapter 2:
%\iffalse
%<*samplechap2>
%\fi
%    \begin{macrocode}
\section{two}
more text in chapter two
%    \end{macrocode}

%\iffalse
%</samplechap2>
%\fi
%
% %%%%%%%%%%%%%%%%%%%%%%%%%%%%%%%%%%%%%%
% \paragraph{Part Include Files.}
%
% The include files are called |cdocspt3.tex| and |cdocspt4.tex|.
%
%\iffalse
%<*samplepart3|samplepart4>
%\fi

% Optional override for |\version| flag:
%    \begin{macrocode}
%%\providecommand{\version}{final}
%    \end{macrocode}

% Include the main document:
%    \begin{macrocode}
\input{childdoc.def}
\childdocby{cdocsamp}
%    \end{macrocode}

%\iffalse
%</samplepart3|samplepart4>
%\fi
%
%\iffalse
%<*samplepart3>
%\fi
% Some text for part 3:
%    \begin{macrocode}
some text in part three
%    \end{macrocode}

%\iffalse
%</samplepart3>
%\fi
% Some text for part 4:
%\iffalse
%<*samplepart4>
%\fi
%    \begin{macrocode}
more text in part four
%    \end{macrocode}

%\iffalse
%</samplepart4>
%\fi
%
% %%%%%%%%%%%%%%%%%%%%%%%%%%%%%%%%%%%%%%
% \paragraph{Forwarding for a Complete Draft.}
%
% The following forwarding file |cdocsdrf.tex|
% compiles the main document in draft mode:
%\iffalse
%<*sampledraft>
%\fi
%    \begin{macrocode}
\def\version{draft}
\input{childdoc.def}
\childdocforward{cdocsamp}
%    \end{macrocode}

%\iffalse
%</sampledraft>
%\fi
%
% %%%%%%%%%%%%%%%%%%%%%%%%%%%%%%%%%%%%%%
% \paragraph{Forwarding for Final Version of the Chapters.}
%
% The following forwarding files |cdocsfn1.tex| and |cdocsfn2.tex|
% (with identical content)
% compile the final versions of the child documents
% |cdocsch1.tex| and |cdocsch2.tex|, respectively:
%\iffalse
%<*samplefinal>
%\fi
%    \begin{macrocode}
\def\version{final}
\input{childdoc.def}
\childdocforwardprefix[cdocsamp]{cdocsfn}{cdocsch}
%    \end{macrocode}

%\iffalse
%</samplefinal>
%\fi
%
% %%%%%%%%%%%%%%%%%%%%%%%%%%%%%%%%%%%%%%
% \paragraph{Command Line Processing.}
%
% The following three command lines generate the output files
% |cdocscld|, |cdocscl1| and |cdocscl2|
% which should be identical to
% |cdocsdrf|, |cdocsch1| and |cdocsfn2|, respectively:
% \begin{center}
% \begin{tabular}{l}
% |latex -jobname cdocscld \|\\
% |  "\def\version{draft}\input{childdoc.def}\childdocforward{cdocsamp}"|\\
% |latex -jobname cdocscl1 \|\\
% |  "\input{childdoc.def}\childdocforward[cdocsamp]{cdocsch1}"|\\
% |latex -jobname cdocscl2 \|\\
% |  "\def\version{final}\input{childdoc.def}\childdocforward{cdocsch2}"|
% \end{tabular}
% \end{center}
% Note that the trailing backslash on each first line
% merely continues the input to the second line
% (for convenient cut ant paste).
% Furthermore, the command |latex| can be replaced by any
% of its alternative versions such as |pdflatex|.
%
% %%%%%%%%%%%%%%%%%%%%%%%%%%%%%%%%%%%%%%%%%%%%%%%%%%%%%%%%%%%%%%%%%%%%%%%%%%%%%%
% %%%%%%%%%%%%%%%%%%%%%%%%%%%%%%%%%%%%%%%%%%%%%%%%%%%%%%%%%%%%%%%%%%%%%%%%%%%%%%
% \section{Implementation}
%\iffalse
%<*package>
%\fi
%
% This section describes the definitions file |childdoc.def|.

% The definitions cannot be loaded using |\usepackage| or |\RequirePackage|
% which has a mechanism to prevent loading a style file more than once.
% When loading the definitions by means of |\input|
% multiple instances have to be prevented manually:
%\iffalse
%This code needs to be before the `\ProvidesFile' directive
%which is defined at the beginning of this file.
%Therefore it is also placed there and commented out here.
%</package>
%<*discard>
%\fi
%    \begin{macrocode}
\ifdefined\childdocmain\endinput\fi
%    \end{macrocode}
%\iffalse
%</discard>
%<*package>
%\fi
%
% \macro{\ifchilddoc}
% \macro{\ifchilddocmanual}
% The conditional |\ifchilddoc| tells whether a
% child (true) or main (false) document is being compiled.
% The conditional |\ifchilddocmanual| tells whether
% the |\includeonly| mechanism is used (false) or
% the selection of child files must be performed manually (true).
% The definitions initialise to false:
%    \begin{macrocode}
\newif\ifchilddoc
\newif\ifchilddocmanual
%    \end{macrocode}

% \macro{\childdocname}
% \macro{\childdocjob}
% The macro |\childdocname| stores the name of the main document
% to be compiled. The macro |\childdocjob| stores the name of
% the document on which the \LaTeX{} compiler was originally invoked.
% The content of |\jobname| cannot be compared
% to filenames specified in the source due to different catcodes.
% The following code rescans |\jobname|, stores the result
% in |\childdocname| and saves a copy in |\childdocjob|:
%    \begin{macrocode}
\edef\childdocname{\scantokens\expandafter{\jobname\noexpand}}
\let\childdocjob\childdocname
%    \end{macrocode}

% \macro{\childdocdisable}
% The macro |\childdocdisable| prevents the main file
% from being processed more than once.
% At this stage, the main document command |\childdocmain|
% is assumed to be called once again where it should do nothing.
% Any subsequent call to it should prevent
% a secondary processing of the main document
% It overwrites the forwarding commands
% |\childdocof| and |\childdocforward|
% with empty macros to prevent further inclusions of the main document:
%    \begin{macrocode}
\newcommand{\childdocdisable}
{
  \renewcommand{\childdocmain}[1]{\renewcommand{\childdocmain}[1]{\endinput}}
  \renewcommand{\childdocof}[1]{}
  \renewcommand{\childdocby}[2][]{}
  \renewcommand{\childdocforward}[2][]{}
  \renewcommand{\childdocdisable}{}
}
%    \end{macrocode}

% \macro{\childdocmain}
% The macro |\childdocmain| is to be called at the top of the main file
% with nothing or the main filename (without extension) as argument.
% First, it breaks loops.
% If the argument is not empty and does not match |\childdocname|
% (which is set by the first inclusion of |childdoc.def|),
% |\ifchilddoc| is set to true, |\includeonly| is applied to the child file
% and |\jobname| is set to the main file
% (for proper handling of |.aux| files):
%    \begin{macrocode}
\newcommand{\childdocmain}[1]
{
  \childdocdisable\childdocmain{}
  \if?#1?\else
    \begingroup
      \def\childdoctmp{#1}
      \ifx\childdoctmp\childdocname
        \def\childdoctmp{}
      \else
        \def\childdoctmp
        {
          \childdoctrue
          \includeonly{\childdocname}
          \def\childdocjob{#1}
          \def\jobname{#1}
        }
      \fi
      \expandafter
    \endgroup
    \childdoctmp
  \fi
}
%    \end{macrocode}

% \macro{\childdocof}
% The command |\childdocof| redirects
% compilation to the main file |#1|.
%    \begin{macrocode}
\newcommand{\childdocof}[1]
{
  \childdocdisable
  \childdoctrue
  \includeonly{\childdocname}
  \def\jobname{#1}
  \def\childdocjob{#1}
  \input{#1}
}
%    \end{macrocode}

% \macro{\childdocby}
% The command |\childdocby| ....
%    \begin{macrocode}
\newcommand{\childdocby}[2][]
{
  \childdocdisable
  \childdoctrue
  \childdocmanualtrue
  \if?#1?\else
    \def\jobname{#2}
  \fi
  \def\childdocjob{#2}
  \input{#2}
  \endinput
}
%    \end{macrocode}

% \macro{\childdocforward}
% The command |\childdocforward| redirects
% compilation to the main file or
% (if the optional argument is given) a child file.
% Parameters are set as if the main file
% or a child file starting with |\childdocof| was compiled.
% Then compilation is handed over to the main file:
%    \begin{macrocode}
\newcommand{\childdocforward}[2][]
{
  \begingroup
    \if?#1?
      \def\childdoctmp
      {
        \def\childdocname{#2}
        \def\childdocjob{#2}
        \def\jobname{#2}
        \input{#2}
        \endinput
      }
    \else
      \def\childdoctmp
      {
        \childdocdisable
        \def\childdocname{#2}
        \childdoctrue
        \includeonly{#2}
        \def\childdocjob{#1}
        \def\jobname{#1}
        \input{#1}
        \endinput
      }
    \fi
    \expandafter
  \endgroup
  \childdoctmp
}
%    \end{macrocode}

% \macro{\childdocforwardprefix}
% The command |\childdocforwardprefix| redirects
% compilation to the main or a child file by means of a pattern.
% The prefix |#1| in the current filename is replaced by |#2|
% and the suffix of the current filename is kept
% (it is assumed that the filename does not contain the substring `|~~~|'
% which is used as a delimiter).
% Compilation is handed over to the new file by |\childdocforward|:
%    \begin{macrocode}
\newcommand{\childdocforwardprefix}[3][]
{
  \begingroup
    \def\childdocextract #2##1~~~{\def\childdoctmp{\childdocforward[#1]{#3##1}}}
    \expandafter\childdocextract\childdocname~~~
    \expandafter
  \endgroup
  \childdoctmp
}
%    \end{macrocode}

% \macro{\childdoc}
% The deprecated macro |\childdoc| is a legacy version of |\childdocmain|:
%    \begin{macrocode}
\newcommand{\childdoc}{\childdocmain}
%    \end{macrocode}

% \macro{\childdocredirect}
% The deprecated macro |\childdocredirect| is a legacy version
% of |\childdocforward| and |\childdocforwardprefix|:
%    \begin{macrocode}
\newcommand{\childdocredirect}[2][]
{
  \begingroup
    \if?#1?
      \def\childdoctmp{\childdocforward{#2}}
    \else
      \def\childdoctmp{\childdocforwardprefix{#1}{#2}}
    \fi
    \expandafter
  \endgroup
  \childdoctmp
}
%    \end{macrocode}

%\iffalse
%</package>
%\fi
%
\endinput
\childdocforward{cdocsamp}"|\\
% |latex -jobname cdocscl1 \|\\
% |  "% \iffalse
%
% childdoc.dtx Copyright (C) 2017-2018 Niklas Beisert
%
% This work may be distributed and/or modified under the
% conditions of the LaTeX Project Public License, either version 1.3
% of this license or (at your option) any later version.
% The latest version of this license is in
%   http://www.latex-project.org/lppl.txt
% and version 1.3 or later is part of all distributions of LaTeX
% version 2005/12/01 or later.
%
% This work has the LPPL maintenance status `maintained'.
%
% The Current Maintainer of this work is Niklas Beisert.
%
% This work consists of the files childdoc.dtx and childdoc.ins
% and the derived files childdoc.def and cdocsamp.tex with
% cdocsch1.tex, cdocsch2.tex, cdocsdrf.tex, cdocsfn1.tex, cdocsfn2.tex.
%
%<package>\ifdefined\childdocmain\endinput\fi
%<package>\ProvidesFile{childdoc.def}[2018/12/30 v2.0 child document driver]
%<samplemain>\ProvidesFile{cdocsamp.tex}[2018/12/30 v2.0 sample for childdoc]
%<*driver>
%\ProvidesFile{childdoc.drv}[2018/12/30 v2.0 childdoc reference manual file]
\PassOptionsToClass{10pt,a4paper}{article}
\documentclass{ltxdoc}

\usepackage[margin=35mm]{geometry}
\usepackage{hyperref}
\usepackage{hyperxmp}
\usepackage[usenames]{color}

\hypersetup{colorlinks=true}
\hypersetup{pdfstartview=FitH}
\hypersetup{pdfpagemode=UseNone}
\hypersetup{pdfsource={}}
\hypersetup{pdflang={en-UK}}
\hypersetup{pdfcopyright={Copyright 2017-2018 Niklas Beisert.
  This work may be distributed and/or modified under the
  conditions of the LaTeX Project Public License, either version 1.3
  of this license or (at your option) any later version.}}
\hypersetup{pdflicenseurl={http://www.latex-project.org/lppl.txt}}
\hypersetup{pdfcontactaddress={ETH Zurich, ITP, HIT K,
  Wolfgang-Pauli-Strasse 27}}
\hypersetup{pdfcontactpostcode={8093}}
\hypersetup{pdfcontactcity={Zurich}}
\hypersetup{pdfcontactcountry={Switzerland}}
\hypersetup{pdfcontactemail={nbeisert@itp.phys.ethz.ch}}
\hypersetup{pdfcontacturl={http://people.phys.ethz.ch/\xmptilde nbeisert/}}

\newcommand{\secref}[1]{\hyperref[#1]{section \ref*{#1}}}

\parskip1ex
\parindent0pt
\let\olditemize\itemize
\def\itemize{\olditemize\parskip0pt}

\begin{document}

\title{The \textsf{childdoc} Package}
\hypersetup{pdftitle={The childdoc Package}}
\author{Niklas Beisert\\[2ex]
  Institut f\"ur Theoretische Physik\\
  Eidgen\"ossische Technische Hochschule Z\"urich\\
  Wolfgang-Pauli-Strasse 27, 8093 Z\"urich, Switzerland\\[1ex]
  \href{mailto:nbeisert@itp.phys.ethz.ch}
  {\texttt{nbeisert@itp.phys.ethz.ch}}}
\hypersetup{pdfauthor={Niklas Beisert}}
\hypersetup{pdfsubject={Manual for the LaTeX2e Package childdoc}}
\date{30 December 2018, \textsf{v2.0}}
\maketitle

\begin{abstract}\noindent
\textsf{childdoc} is a \LaTeXe{} package
that enables the direct compilation
of document sections included by |\include|
to individual files.
\end{abstract}

\begingroup
\parskip0ex
\tableofcontents
\endgroup

%%%%%%%%%%%%%%%%%%%%%%%%%%%%%%%%%%%%%%%%%%%%%%%%%%%%%%%%%%%%%%%%%%%%%%%%%%%%%%%%
%%%%%%%%%%%%%%%%%%%%%%%%%%%%%%%%%%%%%%%%%%%%%%%%%%%%%%%%%%%%%%%%%%%%%%%%%%%%%%%%
\section{Introduction}

\LaTeX{} provides a mechanism to structure a large document (such as a book)
into a main file and several child files (containing the chapters)
using the |\include| command.
This mechanism is beneficial for documents
which span hundreds of pages in order to
make the source file(s) more manageable.
Moreover, compilation can be restricted to
selected child files by means of the |\includeonly| command.
The latter feature can be used to reduce the compilation time while editing
(this was significantly more useful in the earlier days of \LaTeX{})
or to generate a smaller document which is easier to navigate.
Another application of |\includeonly| is to generate
documents consisting of selected parts of the complete document.

However, there are a few drawbacks of the plain |\include| mechanism:
\begin{itemize}
\item
The child files cannot be compiled on their own,
they can only be compiled via the main file.
A naive editing environment
(such as a text editor with an option
to have the current file processed by \LaTeX)
may require one to switch to the main file before compiling;
attempting to compile the child file produces errors.
\item
The main file must be modified (each time)
to adjust the |\includeonly| command
to the present needs. This easily leaves the main file in a messy state.
\item
The generated document will always carry the filename
of the main document. This is inconvenient if
several child files are to be compiled and
to be kept for distribution.
\end{itemize}

The present package provides a simple interface
to make child files individually compilable by \LaTeX{}.
Compiling a child file then has the same effect as compiling
the main file with an |\includeonly| command
to select the appropriate child.
Moreover the generated document will carry the name of the child
rather than the main file.
This resolves all three above issues.

This feature is meant to make the editing of books,
thesis documents and lecture notes somewhat more convenient.
However, the package can also be used efficiently for
composing a series of documents (such as exercise sheets)
which are typically distributed individually.
It then assists the author in generating the individual documents
(potentially in different versions)
as well as a document containing the collected series.
Another application is in developing style files
or other kinds of included material
where compilation of the style file could redirect
to a sample or test file.

%%%%%%%%%%%%%%%%%%%%%%%%%%%%%%%%%%%%%%%%%%%%%%%%%%%%%%%%%%%%%%%%%%%%%%%%%%%%%%%%
%%%%%%%%%%%%%%%%%%%%%%%%%%%%%%%%%%%%%%%%%%%%%%%%%%%%%%%%%%%%%%%%%%%%%%%%%%%%%%%%
\section{Usage}

First of all, the package \textsf{childdoc} is \emph{not} a standard
\LaTeXe{} |.sty| style file! Therefore it needs to be invoked in
a non-standard way.

%%%%%%%%%%%%%%%%%%%%%%%%%%%%%%%%%%%%%%%%%%%%%%%%%%%%%%%%%%%%%%%%%%%%%%%%%%%%%%%%
\subsection{Included Files}
\label{sec:include}

%%%%%%%%%%%%%%%%%%%%%%%%%%%%%%%%%%%%%%%%
\DescribeMacro{\childdocmain}
To use the package, add the commands
\begin{center}
\begin{tabular}{l}
|\input{childdoc.def}|\\
|\childdocmain{}|\\
\end{tabular}
\end{center}
at the very top of the main \LaTeX{} file,
in particular \emph{before} the |\documentclass| statement!
The argument of |\childdocmain| should be left empty
(but it must be present).

%%%%%%%%%%%%%%%%%%%%%%%%%%%%%%%%%%%%%%%%
\DescribeMacro{\childdocof}
Furthermore, add the commands
\begin{center}
\begin{tabular}{l}
|\input{childdoc.def}|\\
|\childdocof{|\textit{main}|}|\\
\end{tabular}
\end{center}
at the top of every child file \textit{child}
which is included by |\include{|\textit{child}|}|
from within the main file
(or at least for those files to be compiled individually).
The argument \textit{main} must be the filename of the main file.

There are a couple of
considerations in setting up the main and child documents:

%%%%%%%%%%%%%%%%%%%%%%%%%%%%%%%%%%%%%%%%
\paragraph{Restrictions.}

Please note the following restrictions:
\begin{itemize}
\item
|\childdocmain| must be called with one argument \textit{main}
to ensure compatibility with earlier version of the package.
It must either be empty (|\childdocmain{}|)
or precisely match the filename of the main file in which it is specified.
See \secref{sec:detection} for further information.
\item
The filename \textit{main} must be specified without the |.tex| extension.
\item
The filename \textit{main} is case sensitive
(even in case-insensitive file systems)
due to internal string comparison.
\item
The argument \textit{main} should be fully expanded, it cannot be a macro.
\item
Subdirectories and special characters should be avoided in filenames.
\item
The command |\childdocmain{|\textit{main}|}| must be followed by a whitespace.
It should not be followed immediately by another command
or by a comment mark `|%|'.
This is because the \TeX{} parser reads the token immediately following
the argument of |\childdocmain| and puts it
at the beginning of every child section;
however, a white\-space is ignored.
\end{itemize}

%%%%%%%%%%%%%%%%%%%%%%%%%%%%%%%%%%%%%%%%
\paragraph{Content of Main File.}

It is advisable to place all content in the child files included by |\include|.
Any output contained in the main file will appear in all child documents
unless suppressed manually;
it cannot be suppressed automatically by the |\includeonly| directive
and thus should normally be avoided.
A method to include some content in the main file
by means of conditional processing is described in \secref{sec:conditional}.

%%%%%%%%%%%%%%%%%%%%%%%%%%%%%%%%%%%%%%%%
\paragraph{Page Numbering.}

When only a part of the document is compiled,
the appropriate numbering of pages
(as well as other status parameters)
is determined from the |.aux| files.
The latter contain information from previous passes.
However this information needs to propagate through
all intermediate child documents.
Therefore the page numbering in child documents may well
be inconsistent until the complete document is compiled at least once.

A useful (if unconventional) way to always ensure a consistent
page numbering is to restart the numbering in each child document
and denote the pages by `\textit{child}|.|\textit{page}'
where \textit{child} represents the chapter/section number of the child file.
This can be achieved by the command
|\numberwithin{page}{|\textit{child}|}|
of the \textsf{amsmath} package
where \textit{child} can be |chapter| or |section|
depending on the chosen structuring.
Alternatively, one can modify the macro |\thepage| appropriately
and reset the counter |page| at the start of each child file.

%%%%%%%%%%%%%%%%%%%%%%%%%%%%%%%%%%%%%%%%%%%%%%%%%%%%%%%%%%%%%%%%%%%%%%%%%%%%%%%%
\subsection{Conditional Processing}
\label{sec:conditional}

The package provides a mechanism to compile different versions
of a document. To customise the versions further some conditional processing
can come in handy to distinguish which version is being compiled.
The package provides two macros to describe the compilation context:

%%%%%%%%%%%%%%%%%%%%%%%%%%%%%%%%%%%%%%%%
\DescribeMacro{\ifchilddoc}
The conditional |\ifchilddoc| distinguishes between the compilation of
child documents and the main document:
%
\begin{center}
|\ifchilddoc |\textit{child-code}| |[|\||else |\textit{main-code}]| \||fi|
\end{center}

%%%%%%%%%%%%%%%%%%%%%%%%%%%%%%%%%%%%%%%%
\DescribeMacro{\childdocname}
\DescribeMacro{\childdocjob}
The macro |\childdocname| contains the filename (without extension)
of the main or child file being processed.
Note that |\childdocjob| will always contain the name of the main file.

%%%%%%%%%%%%%%%%%%%%%%%%%%%%%%%%%%%%%%%%
\paragraph{Title Page.}

Conditional processing can be used to include a title or banner page
in the main document when proper precautions are taken.
Importantly, the code in the main file should ensure that the page counter
(as well as other status parameters which are stored in the |.aux| files)
takes the same value after the conditional processing.
Otherwise the page numbers may take divergent values
depending on which part is compiled.

For example, a title page could be declared by:
%
\begin{center}
\begin{tabular}{l}
|\ifchilddoc\||else|\\
|\addtocounter{page}{-1}|\\
\textit{code for title page}\\
|\newpage|\\
|\||fi|
\end{tabular}
\end{center}
%
A banner page for the child documents can be generated by:
%
\begin{center}
\begin{tabular}{l}
|\ifchilddoc|\\
|\addtocounter{page}{-1}|\\
\textit{code for banner page}\\
|\newpage|\\
|\||fi|
\end{tabular}
\end{center}
%
Here one could write a message such as:
\begin{center}
|This is the part \childdocname{} of \childdocjob{}.|
\end{center}

%%%%%%%%%%%%%%%%%%%%%%%%%%%%%%%%%%%%%%%%%%%%%%%%%%%%%%%%%%%%%%%%%%%%%%%%%%%%%%%%
\subsection{Flags}
\label{sec:flags}

The package makes it easy to generate different versions
of the main or child documents.
To this end compilation flags can be defined
and assigned different default values.
They will be particularly useful in conjunction
with the forwarding mechanism described in \secref{sec:forward}.

For example, it may be useful to have a flag |\version|
which can be set to |draft| or |final|.
The document source will contain some conditional code
depending on the value of |\version|.
Suppose further, the flag should default to |final| for the main file
and to |draft| for child files
which is a natural assignment for editing the document.
This is achieved by placing the following code
in the preamble of the main document
(below the |\childdocmain| directive):
%
\begin{center}
\begin{tabular}{l}
|\ifchilddoc|\\
|\providecommand{\version}{draft}|\\
|\||else|\\
|\providecommand{\version}{final}|\\
|\||fi|
\end{tabular}
\end{center}
%
The definition by |\providecommand| makes sure
that previous definitions are not overwritten.
Further statements |\providecommand{\version}{...}|
can thus be added before the above code to override it.

For the main file, one might add a line
(between |\childdocmain| and the above block)
%
\begin{center}
|%\ifchilddoc\||else\providecommand{\version}{draft}\||fi|
\end{center}
%
which can be uncommented to produce a draft version.
Likewise one can add a line to the very top of a child file
(above the |\childdocof{|\textit{main}|}| directive)
%
\begin{center}
|%\providecommand{\version}{final}|
\end{center}
%
which can be uncommented to produce the final version of this child document.

%%%%%%%%%%%%%%%%%%%%%%%%%%%%%%%%%%%%%%%%%%%%%%%%%%%%%%%%%%%%%%%%%%%%%%%%%%%%%%%%
\subsection{Forwarding}
\label{sec:forward}

Different versions of the main or child documents
using compilation flags as described in \secref{sec:flags}
can be (permanently) stored in different files
for convenient compilation, viewing and distribution.
To this end, the package defines a command
to pass on compilation to a different file:

%%%%%%%%%%%%%%%%%%%%%%%%%%%%%%%%%%%%%%%%
\DescribeMacro{\childdocforward}
The command |\childdocforward| redirects processing to
another source file:
%
\begin{center}
\begin{tabular}{l}
|\input{childdoc.def}|\\
|\childdocforward[|\textit{main}|]{|\textit{dest}|}|\\
\end{tabular}
\end{center}
%
The argument \textit{dest} is the destination file
(without extension).
It should be the main file or one of the child files.
Note that further \textsf{childdoc} directives
such as |\childdocof| and |\childdocforward|
in the indicated file will be processed in this form.
The optional argument \textit{main}
passes on directly to the main file \textit{main}
while pretending to compile the child \textit{dest}.
This form behaves as if \textit{dest}
issues |\childdocof{|\textit{main}|}| right away,
and no further \textsf{childdoc} directives will be processed.

%%%%%%%%%%%%%%%%%%%%%%%%%%%%%%%%%%%%%%%%
\DescribeMacro{\...prefix}
In the alternative form |\childdocforwardprefix|,
%
\begin{center}
\begin{tabular}{l}
|\input{childdoc.def}|\\
|\childdocforwardprefix[|\textit{main}|]{|\textit{prefix}|}{|\textit{dest}|}|
\end{tabular}
\end{center}
%
the destination file is determined by a pattern
depending on the current file:
To make this work, the current file must be called
`{\textit{prefix}\hspace{0.2em}\textit{suffix}}'
with \textit{prefix} matching precisely the argument.
Processing is then passed on to the file
`{\textit{dest}\hspace{0.2em}\textit{suffix}}'.
Surely, the same effect is achieved by
directly specifying the
argument `{\textit{dest}\hspace{0.2em}\textit{suffix}}'
in the first form.
However, that requires to set up a different file
for each child. With the alternative form of the command
all these files can have exactly the same content
which simplifies setting them up and maintaining them.

For example, the following file |draft.tex|
with a compilation flag |\version| as described in \secref{sec:flags}
compiles the main document as a draft:
%
\begin{center}
\begin{tabular}{l}
|\def\version{draft}|\\
|\input{childdoc.def}|\\
|\childdocforward{|\textit{main}|}|
\end{tabular}
\end{center}
%
Likewise, the following files |final|\textit{nn}|.tex|
compile the final version of the child document
|child|\textit{nn}|.tex|:
%
\begin{center}
\begin{tabular}{l}
|\def\version{final}|\\
|\input{childdoc.def}|\\
|\childdocforwardprefix{final}{child}|
\end{tabular}
\end{center}
%

Note that when several versions of a main file and/or of each child file
are to be generated, it may be convenient to set up a |Makefile| or
shell script to automatise the process.

%%%%%%%%%%%%%%%%%%%%%%%%%%%%%%%%%%%%%%%%%%%%%%%%%%%%%%%%%%%%%%%%%%%%%%%%%%%%%%%%
\subsection{Command Line Processing}
\label{sec:commandline}

The effect of redirection files can also be achieved by invoking
the \LaTeX{} compiler with a more elaborate command line.
Most conveniently this should be done as part
of a shell script or a |Makefile|.

When using \textsf{childdoc} in the main file, the following
command lines effectively perform a redirection
(note that depending on the shell being used,
backslashes may have to be doubled: `|\|' $\to$ `|\\|'):
%
\begin{center}
|... -jobname "|\textit{target}|" |\\|"|[\textit{flags}]%
|\input{childdoc.def}\childdocforward[|\textit{main}|]{|\textit{dest}|}"|
\end{center}
%
Here \textit{target} is the name of the output file,
\textit{main} is the name of the main file
and \textit{dest} is the name of the main or child file to be processed
(all filenames without extensions).
The optional argument \textit{main} can be omitted
if \textit{main} matches \textit{dest}.
Optionally, compilation \textit{flags} can be defined via |\def| commands.
This command line makes the \TeX{} engine believe
it is compiling the file \textit{target}
whose content is specified as the latter parameter.
The provided code then forwards the processing to
\textit{main} or \textit{dest} as described in \secref{sec:forward}.

%%%%%%%%%%%%%%%%%%%%%%%%%%%%%%%%%%%%%%%%%%%%%%%%%%%%%%%%%%%%%%%%%%%%%%%%%%%%%%%%
\subsection{Include by Input}
\label{sec:input}

Including child documents by |\include| has some restrictions by design.
Most notably, the content of a child document always occupies
its own set of pages; pages cannot be shared between child documents.
Usually, this behaviour makes perfect sense
because each child document contain an essential part of the document.
However, in some situations it may be desirable to compose
a document from a collection of parts
without having mandatory page breaks between then.
For this case, the package
provides a mechanism to include parts
by |\input| which can also be processed individually.
However, by construction this mechanism
requires manual handling of the content to be output.

%%%%%%%%%%%%%%%%%%%%%%%%%%%%%%%%%%%%%%%%
\DescribeMacro{\ifchilddocmanual}
The main file should be prepared as usual, see \secref{sec:include}.
However, the document body must make a distinction
between processing of an individual part and of the main document, e.g.:
%
\begin{center}
\begin{tabular}{l}
|\ifchilddocmanual|\\
|\input{\childdocname}|\\
|\||else|\\
\textit{document body with }|\input{|\textit{part}|}|\\
|\||fi|
\end{tabular}
\end{center}
%
The conditional |\ifchilddocmanual| is true whenever
a part to be included by |\input| is being compiled,
and the name of the part is stored in |\childdocname|.

%%%%%%%%%%%%%%%%%%%%%%%%%%%%%%%%%%%%%%%%
\DescribeMacro{\childdocby}
Each part to be included by |\input| should start with:
%
\begin{center}
\begin{tabular}{l}
|\input{childdoc.def}|\\
|\childdocby{|\textit{main}|}|\\
\end{tabular}
\end{center}
%
The directive |\childdocby| is similar to |\childdocof|
described in \secref{sec:include},
but the subsequent selection of content must be done manually.
To that end, both |\ifchilddoc| and |\ifchilddocmanual|
will be true upon processing of a part,
and the name of the part is stored in |\childdocname|.
Note that |\jobname| will be set to the filename of the current part
so that each part receives an individual |.aux| file
that does not interfere with the |.aux| file(s) of the main document.
This behaviour can be altered by the alternative form
|\childdocby[*]{|\textit{main}|}| (with a non-empty optional argument)
which uses the |.aux| file of the main document
by setting |\jobname| to \textit{main}.

%%%%%%%%%%%%%%%%%%%%%%%%%%%%%%%%%%%%%%%%%%%%%%%%%%%%%%%%%%%%%%%%%%%%%%%%%%%%%%%%
\subsection{Driver Development}
\label{sec:driver}

The \textsf{childdoc} mechanism can also be use for the development
of definition files such as \LaTeX{} styles or classes.
This case differs from the above setup with multiple parts
included by |\include| in that no |\includeonly| should be invoked.
This can be achieved by starting the include file
(before |\ProvidesPackage|) with:
%
\begin{center}
\begin{tabular}{l}
|\input{childdoc.def}|\\
|\childdocforward{|\textit{main}|}|\\
\end{tabular}
\end{center}
%
or alternatively with:
%
\begin{center}
\begin{tabular}{l}
|\input{childdoc.def}|\\
|\childdocby{|\textit{main}|}|\\
\end{tabular}
\end{center}
%
Both forms have slightly different effects as described above.
The main file is prepared as usual, see \secref{sec:include}.

%%%%%%%%%%%%%%%%%%%%%%%%%%%%%%%%%%%%%%%%%%%%%%%%%%%%%%%%%%%%%%%%%%%%%%%%%%%%%%%%
\subsection{Legacy Detection}
\label{sec:detection}

The directive |\childdocmain| in the main file can detect
whether the complete document or merely a child is to be compiled
even without using the directive |\childdocof|.
This method is deprecated because it is less robust
and there is no compelling reason to use it;
it is merely provided for backward compatibility
and it may be removed in future versions.

If the detection mechanism is to be used,
it is mandatory to correctly specify
the filename of the main file as the argument of |\childdocmain|:
%
\begin{center}
\begin{tabular}{l}
|\input{childdoc.def}|\\
|\childdocmain{|\textit{main}|}|\\
\end{tabular}
\end{center}
%
If |\jobname| does not match the argument \textit{main} of |\childdocmain|,
it is assumed that |\jobname| points to the child file to be compiled.
When using |\childdocmain| with the main file specified as argument,
it suffices to start a child file
with just |\input{|\textit{main}|}|
without loading of the package and using |\childdocof|.
If instead all processing is done
with the appropriate \textsf{childdoc} directives,
the argument of \textit{main} of |\childdocmain| can be empty.

An alternative version of the command line processing described
in \secref{sec:commandline} using the detection mechanism reads:
%
\begin{center}
|... -jobname "|\textit{target}|" "|[\textit{flags}]%
[|\def\jobname{|\textit{dest}|}|]|\input{|\textit{main}|}"|
\end{center}

%%%%%%%%%%%%%%%%%%%%%%%%%%%%%%%%%%%%%%%%%%%%%%%%%%%%%%%%%%%%%%%%%%%%%%%%%%%%%%%%
\subsection{Manual Code}
\label{sec:manual}

In case one cannot be certain whether the definitions file |childdoc.def|
is installed on the target \TeX{} distribution
and one prefers not to ship it,
it is conceivable to paste a few relevant commands into the sources.

To that end, drop all statements |\input{childdoc.def}|
and perform the replacements as outlined below.
Instead of |\childdocmain{|\textit{main}|}| add the following code
to the top of the main file:
%
\begin{center}
\begin{tabular}{l}
|\||ifdefined\childdocname\endinput\||fi\newif\ifchilddoc|\\
|\edef\childdocname{\scantokens\expandafter{\jobname\noexpand}}|\\
|\def\childdocmain{|\textit{main}|}\||ifx\childdocmain\childdocname\||else|\\
|\childdoctrue\includeonly{\childdocname}\let\jobname\childdocmain\||fi|\\
\end{tabular}
\end{center}
%
Instead of |\childdocof{|\textit{main}|}| just include the main file
at the top of each child file:
%
\begin{center}
|\input{|\textit{main}|}|
\end{center}
%
A simple redirection |\childdocforward{|\textit{dest}|}| is achieved by:
%
\begin{center}
|\def\jobname{|\textit{dest}|}\input{\jobname}|
\end{center}
%
The redirection with prefix
|\childdocforwardprefix[|\textit{prefix}|]{|\textit{dest}|}|
is accomplished by:
%
\begin{center}
\begin{tabular}{l}
|{\edef\jobname{\scantokens\expandafter{\jobname\noexpand}}|\\
|\def\redirectjob |\textit{prefix}|#1~~~{\gdef\jobname{|\textit{dest}|#1}}|\\
|\expandafter\redirectjob\jobname~~~}\input{\jobname}|
\end{tabular}
\end{center}

In an alternative approach,
child documents can be compiled by a specific command line
without additional code or specific definitions:
%
\begin{center}
|... -jobname "|\textit{target}|" "|[\textit{flags}]%
|\includeonly{|\textit{dest}|}\input{|\textit{main}|}"|
\end{center}
%

%%%%%%%%%%%%%%%%%%%%%%%%%%%%%%%%%%%%%%%%%%%%%%%%%%%%%%%%%%%%%%%%%%%%%%%%%%%%%%%%
%%%%%%%%%%%%%%%%%%%%%%%%%%%%%%%%%%%%%%%%%%%%%%%%%%%%%%%%%%%%%%%%%%%%%%%%%%%%%%%%
\section{Information}

%%%%%%%%%%%%%%%%%%%%%%%%%%%%%%%%%%%%%%%%%%%%%%%%%%%%%%%%%%%%%%%%%%%%%%%%%%%%%%%%
\subsection{Copyright}

Copyright \copyright{} 2017--2018 Niklas Beisert

This work may be distributed and/or modified under the
conditions of the \LaTeX{} Project Public License, either version 1.3
of this license or (at your option) any later version.
The latest version of this license is in
  \url{http://www.latex-project.org/lppl.txt}
and version 1.3 or later is part of all distributions of \LaTeX{}
version 2005/12/01 or later.

This work has the LPPL maintenance status `maintained'.

The Current Maintainer of this work is Niklas Beisert.

This work consists of the files |README.txt|, |childdoc.ins| and |childdoc.dtx|
as well as the derived files |childdoc.def|, |cdocsamp.tex|
with |cdocsch1.tex|, |cdocsch2.tex|, |cdocspt3.tex|, |cdocspt4.tex|,
|cdocsdrf.tex|, |cdocsfn1.tex|, |cdocsfn2.tex|
as well as |childdoc.pdf|.

%%%%%%%%%%%%%%%%%%%%%%%%%%%%%%%%%%%%%%%%%%%%%%%%%%%%%%%%%%%%%%%%%%%%%%%%%%%%%%%%
\subsection{Files and Installation}

The package consists of the files:
%
\begin{center}
\begin{tabular}{ll}
    |README.txt|   & readme file \\
    |childdoc.ins| & installation file \\
    |childdoc.dtx| & source file \\
    |childdoc.def| & definition file \\
    |cdocsamp.tex| & sample main file \\
    |cdocsch1.tex| & sample include file \\
    |cdocsch2.tex| & sample include file \\
    |cdocspt3.tex| & sample part file \\
    |cdocspt4.tex| & sample part file \\
    |cdocsdrf.tex| & sample redirection file \\
    |cdocsfn1.tex| & sample redirection file \\
    |cdocsfn2.tex| & sample redirection file \\
    |childdoc.pdf| & manual
\end{tabular}
\end{center}
%
The distribution consists of the files
|README.txt|, |childdoc.ins| and |childdoc.dtx|.
%
\begin{itemize}
\item
Run (pdf)\LaTeX{} on |childdoc.dtx|
to compile the manual |childdoc.pdf| (this file).
\item
Run \LaTeX{} on |childdoc.ins| to create the definitions file |childdoc.def|
and the sample |cdocsamp.tex| with include files
|cdocsch1.tex|, |cdocsch2.tex|, |cdocspt3.tex|, |cdocspt4.tex|,
|cdocsdrf.tex|, |cdocsfn1.tex|, |cdocsfn2.tex|.
Then copy the file |childdoc.def| to an appropriate directory of your \LaTeX{}
distribution, e.g.\ \textit{texmf-root}|/tex/latex/childdoc|.
\end{itemize}

%%%%%%%%%%%%%%%%%%%%%%%%%%%%%%%%%%%%%%%%%%%%%%%%%%%%%%%%%%%%%%%%%%%%%%%%%%%%%%%%
\subsection{Related CTAN Packages}

There are several other packages which offer a similar functionality:
%
\begin{itemize}
\item
The packages
\href{http://ctan.org/pkg/docmute}{\textsf{docmute}},
\href{http://ctan.org/pkg/includex}{\textsf{includex}} and
\href{http://ctan.org/pkg/standalone}{\textsf{standalone}}
provide commands to include only the document body of
a child file thus allowing both files to be compiled individually.
\item
The packages \href{http://ctan.org/pkg/subdocs}{\textsf{subdocs}}
and \href{http://ctan.org/pkg/subfiles}{\textsf{subfiles}}
provide structures in which the main and child documents can be
encapsulated and allowing them to be compiled individually.
The inclusion mechanism is different from the conventional |\include|.
\item
The package \href{http://ctan.org/pkg/combine}{\textsf{combine}}
is an elaborate solution to combine several documents into one.
\end{itemize}
%
See also the CTAN topic \href{http://ctan.org/topic/subdocs}{\textsf{subdocs}}
for further related packages.
The present package differs from the above solutions in that
a document structure constructed with the conventional |\include| mechanism
just needs two extra commands at the top of every file
such that all constituent files can be compiled individually.

%%%%%%%%%%%%%%%%%%%%%%%%%%%%%%%%%%%%%%%%%%%%%%%%%%%%%%%%%%%%%%%%%%%%%%%%%%%%%%%%
%\subsection{Feature Suggestions}
%
%The following is a list of features which may be useful for future
%versions of this package:
%%
%\begin{itemize}
%\item
%\ldots
%\end{itemize}

%%%%%%%%%%%%%%%%%%%%%%%%%%%%%%%%%%%%%%%%%%%%%%%%%%%%%%%%%%%%%%%%%%%%%%%%%%%%%%%%
\subsection{Revision History}

%%%%%%%%%%%%%%%%%%%%%%%%%%%%%%%%%%%%%%%%
\paragraph{v2.0:} 2018/12/30

\begin{itemize}
\item
immediate forward processing
\item
added |\childdocby| mechanism
\item
manual restructured
\end{itemize}

%%%%%%%%%%%%%%%%%%%%%%%%%%%%%%%%%%%%%%%%
\paragraph{v1.6:} 2018/01/17

\begin{itemize}
\item
application for development of include files
\item
corrections to manual
\end{itemize}

%%%%%%%%%%%%%%%%%%%%%%%%%%%%%%%%%%%%%%%%
\paragraph{v1.5:} 2017/05/21

\begin{itemize}
\item
more complete structuring introduced
\item
|\childdocof| introduced
\item
|\childdoc| renamed to |\childdocmain|
\item
|\childredirect| renamed to |\childdocforward| and |\childdocforwardprefix|
and functionality expanded
\end{itemize}

%%%%%%%%%%%%%%%%%%%%%%%%%%%%%%%%%%%%%%%%
\paragraph{v1.0:} 2017/04/27

\begin{itemize}
\item
manual and install package
\item
first version published on CTAN
\end{itemize}

%%%%%%%%%%%%%%%%%%%%%%%%%%%%%%%%%%%%%%%%
\paragraph{v0.6:} 2017/04/26

\begin{itemize}
\item
redirection mechanism added
\end{itemize}

%%%%%%%%%%%%%%%%%%%%%%%%%%%%%%%%%%%%%%%%
\paragraph{v0.5:} 2017/04/26

\begin{itemize}
\item
functionality in definition file
\end{itemize}


%%%%%%%%%%%%%%%%%%%%%%%%%%%%%%%%%%%%%%%%%%%%%%%%%%%%%%%%%%%%%%%%%%%%%%%%%%%%%%%%
%%%%%%%%%%%%%%%%%%%%%%%%%%%%%%%%%%%%%%%%%%%%%%%%%%%%%%%%%%%%%%%%%%%%%%%%%%%%%%%%
%%%%%%%%%%%%%%%%%%%%%%%%%%%%%%%%%%%%%%%%%%%%%%%%%%%%%%%%%%%%%%%%%%%%%%%%%%%%%%%%
\appendix

\settowidth\MacroIndent{\rmfamily\scriptsize 000\ }

 \DocInput{childdoc.dtx}

\end{document}
%</driver>
% \fi
%
% %%%%%%%%%%%%%%%%%%%%%%%%%%%%%%%%%%%%%%%%%%%%%%%%%%%%%%%%%%%%%%%%%%%%%%%%%%%%%%
% %%%%%%%%%%%%%%%%%%%%%%%%%%%%%%%%%%%%%%%%%%%%%%%%%%%%%%%%%%%%%%%%%%%%%%%%%%%%%%
% \section{Sample}
%\iffalse
%<*samplemain>
%\fi
%
% The following presents a sample document
% with two chapters, two parts, a title page,
% a compile flag as well as three forwarding files to set the flag.
% It consists of eight |.tex| files:
% \begin{center}
% \begin{tabular}{ll}
% |cdocsamp.tex|&main file\\
% |cdocsch1.tex|&include file for chapter 1\\
% |cdocsch2.tex|&include file for chapter 2\\
% |cdocspt3.tex|&include file for part 3\\
% |cdocspt4.tex|&include file for part 4\\
% |cdocsdrf.tex|&forwarding file for main file in draft mode\\
% |cdocsfi1.tex|&forwarding file for final version of chapter 1\\
% |cdocsfi2.tex|&forwarding file for final version of chapter 2\\
% \end{tabular}
% \end{center}
% Each of the eight files can be compiled directly by the \LaTeX{} compiler.
%
% %%%%%%%%%%%%%%%%%%%%%%%%%%%%%%%%%%%%%%
% \paragraph{Main File.}
%
% The main file is called |cdocsamp.tex|.
%
% Load the \textsf{childdoc} definitions and
% declare the filename for the main document:
%    \begin{macrocode}
\input{childdoc.def}
\childdocmain{}
%    \end{macrocode}

% Optional override for |\version| flag:
%    \begin{macrocode}
%%\ifchilddoc\else\providecommand{\version}{draft}\fi
%    \end{macrocode}

% Define the default values for the |\version| flag
% (|final| for the main file and |draft| for childs):
%    \begin{macrocode}
\ifchilddoc
\providecommand{\version}{draft}
\else
\providecommand{\version}{final}
\fi
%    \end{macrocode}

% Load the standard document class:
%    \begin{macrocode}
\documentclass[12pt]{article}
%    \end{macrocode}

% Start the document body:
%    \begin{macrocode}
\begin{document}
%    \end{macrocode}

% Declare a title page.
% Print title, part of document being processed and version flag:
%    \begin{macrocode}
\addtocounter{page}{-1}
\begin{center}
{\LARGE\bfseries{}childdoc example\par}
\vspace{1cm}
\ifchilddoc
\ifchilddocmanual part\else chapter\fi:
`\childdocname' of `\childdocjob'\par
\else
main document: `\childdocjob'\par
\fi
version: \version\par
\end{center}
\newpage
%    \end{macrocode}

% Manually include selected file,
% otherwise process as usual:
%    \begin{macrocode}
\ifchilddocmanual
\section*{part `\childdocname'}
\input{\childdocname}
\else
%    \end{macrocode}

% Include the two chapters:
%    \begin{macrocode}
\include{cdocsch1}
\include{cdocsch2}
%    \end{macrocode}

% Include the two parts unless only chapters should be displayed:
%    \begin{macrocode}
\ifchilddoc\else
\section{part three}
\input{cdocspt3}
\section{part four}
\input{cdocspt4}
\fi
%    \end{macrocode}

% Process as usual until here:
%    \begin{macrocode}
\fi
%    \end{macrocode}

% End of document body:
%    \begin{macrocode}
\end{document}
%    \end{macrocode}
%\iffalse
%</samplemain>
%\fi
%
% %%%%%%%%%%%%%%%%%%%%%%%%%%%%%%%%%%%%%%
% \paragraph{Chapter Include Files.}
%
% The include files are called |cdocsch1.tex| and |cdocsch2.tex|.
%
%\iffalse
%<*samplechap1|samplechap2>
%\fi

% Optional override for |\version| flag:
%    \begin{macrocode}
%%\providecommand{\version}{final}
%    \end{macrocode}

% Include the main document:
%    \begin{macrocode}
\input{childdoc.def}
\childdocof{cdocsamp}
%    \end{macrocode}

%\iffalse
%</samplechap1|samplechap2>
%\fi
%
%\iffalse
%<*samplechap1>
%\fi
% Some text for chapter 1:
%    \begin{macrocode}
\section{one}
some text in chapter one
%    \end{macrocode}

%\iffalse
%</samplechap1>
%\fi
% Some text for chapter 2:
%\iffalse
%<*samplechap2>
%\fi
%    \begin{macrocode}
\section{two}
more text in chapter two
%    \end{macrocode}

%\iffalse
%</samplechap2>
%\fi
%
% %%%%%%%%%%%%%%%%%%%%%%%%%%%%%%%%%%%%%%
% \paragraph{Part Include Files.}
%
% The include files are called |cdocspt3.tex| and |cdocspt4.tex|.
%
%\iffalse
%<*samplepart3|samplepart4>
%\fi

% Optional override for |\version| flag:
%    \begin{macrocode}
%%\providecommand{\version}{final}
%    \end{macrocode}

% Include the main document:
%    \begin{macrocode}
\input{childdoc.def}
\childdocby{cdocsamp}
%    \end{macrocode}

%\iffalse
%</samplepart3|samplepart4>
%\fi
%
%\iffalse
%<*samplepart3>
%\fi
% Some text for part 3:
%    \begin{macrocode}
some text in part three
%    \end{macrocode}

%\iffalse
%</samplepart3>
%\fi
% Some text for part 4:
%\iffalse
%<*samplepart4>
%\fi
%    \begin{macrocode}
more text in part four
%    \end{macrocode}

%\iffalse
%</samplepart4>
%\fi
%
% %%%%%%%%%%%%%%%%%%%%%%%%%%%%%%%%%%%%%%
% \paragraph{Forwarding for a Complete Draft.}
%
% The following forwarding file |cdocsdrf.tex|
% compiles the main document in draft mode:
%\iffalse
%<*sampledraft>
%\fi
%    \begin{macrocode}
\def\version{draft}
\input{childdoc.def}
\childdocforward{cdocsamp}
%    \end{macrocode}

%\iffalse
%</sampledraft>
%\fi
%
% %%%%%%%%%%%%%%%%%%%%%%%%%%%%%%%%%%%%%%
% \paragraph{Forwarding for Final Version of the Chapters.}
%
% The following forwarding files |cdocsfn1.tex| and |cdocsfn2.tex|
% (with identical content)
% compile the final versions of the child documents
% |cdocsch1.tex| and |cdocsch2.tex|, respectively:
%\iffalse
%<*samplefinal>
%\fi
%    \begin{macrocode}
\def\version{final}
\input{childdoc.def}
\childdocforwardprefix[cdocsamp]{cdocsfn}{cdocsch}
%    \end{macrocode}

%\iffalse
%</samplefinal>
%\fi
%
% %%%%%%%%%%%%%%%%%%%%%%%%%%%%%%%%%%%%%%
% \paragraph{Command Line Processing.}
%
% The following three command lines generate the output files
% |cdocscld|, |cdocscl1| and |cdocscl2|
% which should be identical to
% |cdocsdrf|, |cdocsch1| and |cdocsfn2|, respectively:
% \begin{center}
% \begin{tabular}{l}
% |latex -jobname cdocscld \|\\
% |  "\def\version{draft}\input{childdoc.def}\childdocforward{cdocsamp}"|\\
% |latex -jobname cdocscl1 \|\\
% |  "\input{childdoc.def}\childdocforward[cdocsamp]{cdocsch1}"|\\
% |latex -jobname cdocscl2 \|\\
% |  "\def\version{final}\input{childdoc.def}\childdocforward{cdocsch2}"|
% \end{tabular}
% \end{center}
% Note that the trailing backslash on each first line
% merely continues the input to the second line
% (for convenient cut ant paste).
% Furthermore, the command |latex| can be replaced by any
% of its alternative versions such as |pdflatex|.
%
% %%%%%%%%%%%%%%%%%%%%%%%%%%%%%%%%%%%%%%%%%%%%%%%%%%%%%%%%%%%%%%%%%%%%%%%%%%%%%%
% %%%%%%%%%%%%%%%%%%%%%%%%%%%%%%%%%%%%%%%%%%%%%%%%%%%%%%%%%%%%%%%%%%%%%%%%%%%%%%
% \section{Implementation}
%\iffalse
%<*package>
%\fi
%
% This section describes the definitions file |childdoc.def|.

% The definitions cannot be loaded using |\usepackage| or |\RequirePackage|
% which has a mechanism to prevent loading a style file more than once.
% When loading the definitions by means of |\input|
% multiple instances have to be prevented manually:
%\iffalse
%This code needs to be before the `\ProvidesFile' directive
%which is defined at the beginning of this file.
%Therefore it is also placed there and commented out here.
%</package>
%<*discard>
%\fi
%    \begin{macrocode}
\ifdefined\childdocmain\endinput\fi
%    \end{macrocode}
%\iffalse
%</discard>
%<*package>
%\fi
%
% \macro{\ifchilddoc}
% \macro{\ifchilddocmanual}
% The conditional |\ifchilddoc| tells whether a
% child (true) or main (false) document is being compiled.
% The conditional |\ifchilddocmanual| tells whether
% the |\includeonly| mechanism is used (false) or
% the selection of child files must be performed manually (true).
% The definitions initialise to false:
%    \begin{macrocode}
\newif\ifchilddoc
\newif\ifchilddocmanual
%    \end{macrocode}

% \macro{\childdocname}
% \macro{\childdocjob}
% The macro |\childdocname| stores the name of the main document
% to be compiled. The macro |\childdocjob| stores the name of
% the document on which the \LaTeX{} compiler was originally invoked.
% The content of |\jobname| cannot be compared
% to filenames specified in the source due to different catcodes.
% The following code rescans |\jobname|, stores the result
% in |\childdocname| and saves a copy in |\childdocjob|:
%    \begin{macrocode}
\edef\childdocname{\scantokens\expandafter{\jobname\noexpand}}
\let\childdocjob\childdocname
%    \end{macrocode}

% \macro{\childdocdisable}
% The macro |\childdocdisable| prevents the main file
% from being processed more than once.
% At this stage, the main document command |\childdocmain|
% is assumed to be called once again where it should do nothing.
% Any subsequent call to it should prevent
% a secondary processing of the main document
% It overwrites the forwarding commands
% |\childdocof| and |\childdocforward|
% with empty macros to prevent further inclusions of the main document:
%    \begin{macrocode}
\newcommand{\childdocdisable}
{
  \renewcommand{\childdocmain}[1]{\renewcommand{\childdocmain}[1]{\endinput}}
  \renewcommand{\childdocof}[1]{}
  \renewcommand{\childdocby}[2][]{}
  \renewcommand{\childdocforward}[2][]{}
  \renewcommand{\childdocdisable}{}
}
%    \end{macrocode}

% \macro{\childdocmain}
% The macro |\childdocmain| is to be called at the top of the main file
% with nothing or the main filename (without extension) as argument.
% First, it breaks loops.
% If the argument is not empty and does not match |\childdocname|
% (which is set by the first inclusion of |childdoc.def|),
% |\ifchilddoc| is set to true, |\includeonly| is applied to the child file
% and |\jobname| is set to the main file
% (for proper handling of |.aux| files):
%    \begin{macrocode}
\newcommand{\childdocmain}[1]
{
  \childdocdisable\childdocmain{}
  \if?#1?\else
    \begingroup
      \def\childdoctmp{#1}
      \ifx\childdoctmp\childdocname
        \def\childdoctmp{}
      \else
        \def\childdoctmp
        {
          \childdoctrue
          \includeonly{\childdocname}
          \def\childdocjob{#1}
          \def\jobname{#1}
        }
      \fi
      \expandafter
    \endgroup
    \childdoctmp
  \fi
}
%    \end{macrocode}

% \macro{\childdocof}
% The command |\childdocof| redirects
% compilation to the main file |#1|.
%    \begin{macrocode}
\newcommand{\childdocof}[1]
{
  \childdocdisable
  \childdoctrue
  \includeonly{\childdocname}
  \def\jobname{#1}
  \def\childdocjob{#1}
  \input{#1}
}
%    \end{macrocode}

% \macro{\childdocby}
% The command |\childdocby| ....
%    \begin{macrocode}
\newcommand{\childdocby}[2][]
{
  \childdocdisable
  \childdoctrue
  \childdocmanualtrue
  \if?#1?\else
    \def\jobname{#2}
  \fi
  \def\childdocjob{#2}
  \input{#2}
  \endinput
}
%    \end{macrocode}

% \macro{\childdocforward}
% The command |\childdocforward| redirects
% compilation to the main file or
% (if the optional argument is given) a child file.
% Parameters are set as if the main file
% or a child file starting with |\childdocof| was compiled.
% Then compilation is handed over to the main file:
%    \begin{macrocode}
\newcommand{\childdocforward}[2][]
{
  \begingroup
    \if?#1?
      \def\childdoctmp
      {
        \def\childdocname{#2}
        \def\childdocjob{#2}
        \def\jobname{#2}
        \input{#2}
        \endinput
      }
    \else
      \def\childdoctmp
      {
        \childdocdisable
        \def\childdocname{#2}
        \childdoctrue
        \includeonly{#2}
        \def\childdocjob{#1}
        \def\jobname{#1}
        \input{#1}
        \endinput
      }
    \fi
    \expandafter
  \endgroup
  \childdoctmp
}
%    \end{macrocode}

% \macro{\childdocforwardprefix}
% The command |\childdocforwardprefix| redirects
% compilation to the main or a child file by means of a pattern.
% The prefix |#1| in the current filename is replaced by |#2|
% and the suffix of the current filename is kept
% (it is assumed that the filename does not contain the substring `|~~~|'
% which is used as a delimiter).
% Compilation is handed over to the new file by |\childdocforward|:
%    \begin{macrocode}
\newcommand{\childdocforwardprefix}[3][]
{
  \begingroup
    \def\childdocextract #2##1~~~{\def\childdoctmp{\childdocforward[#1]{#3##1}}}
    \expandafter\childdocextract\childdocname~~~
    \expandafter
  \endgroup
  \childdoctmp
}
%    \end{macrocode}

% \macro{\childdoc}
% The deprecated macro |\childdoc| is a legacy version of |\childdocmain|:
%    \begin{macrocode}
\newcommand{\childdoc}{\childdocmain}
%    \end{macrocode}

% \macro{\childdocredirect}
% The deprecated macro |\childdocredirect| is a legacy version
% of |\childdocforward| and |\childdocforwardprefix|:
%    \begin{macrocode}
\newcommand{\childdocredirect}[2][]
{
  \begingroup
    \if?#1?
      \def\childdoctmp{\childdocforward{#2}}
    \else
      \def\childdoctmp{\childdocforwardprefix{#1}{#2}}
    \fi
    \expandafter
  \endgroup
  \childdoctmp
}
%    \end{macrocode}

%\iffalse
%</package>
%\fi
%
\endinput
\childdocforward[cdocsamp]{cdocsch1}"|\\
% |latex -jobname cdocscl2 \|\\
% |  "\def\version{final}% \iffalse
%
% childdoc.dtx Copyright (C) 2017-2018 Niklas Beisert
%
% This work may be distributed and/or modified under the
% conditions of the LaTeX Project Public License, either version 1.3
% of this license or (at your option) any later version.
% The latest version of this license is in
%   http://www.latex-project.org/lppl.txt
% and version 1.3 or later is part of all distributions of LaTeX
% version 2005/12/01 or later.
%
% This work has the LPPL maintenance status `maintained'.
%
% The Current Maintainer of this work is Niklas Beisert.
%
% This work consists of the files childdoc.dtx and childdoc.ins
% and the derived files childdoc.def and cdocsamp.tex with
% cdocsch1.tex, cdocsch2.tex, cdocsdrf.tex, cdocsfn1.tex, cdocsfn2.tex.
%
%<package>\ifdefined\childdocmain\endinput\fi
%<package>\ProvidesFile{childdoc.def}[2018/12/30 v2.0 child document driver]
%<samplemain>\ProvidesFile{cdocsamp.tex}[2018/12/30 v2.0 sample for childdoc]
%<*driver>
%\ProvidesFile{childdoc.drv}[2018/12/30 v2.0 childdoc reference manual file]
\PassOptionsToClass{10pt,a4paper}{article}
\documentclass{ltxdoc}

\usepackage[margin=35mm]{geometry}
\usepackage{hyperref}
\usepackage{hyperxmp}
\usepackage[usenames]{color}

\hypersetup{colorlinks=true}
\hypersetup{pdfstartview=FitH}
\hypersetup{pdfpagemode=UseNone}
\hypersetup{pdfsource={}}
\hypersetup{pdflang={en-UK}}
\hypersetup{pdfcopyright={Copyright 2017-2018 Niklas Beisert.
  This work may be distributed and/or modified under the
  conditions of the LaTeX Project Public License, either version 1.3
  of this license or (at your option) any later version.}}
\hypersetup{pdflicenseurl={http://www.latex-project.org/lppl.txt}}
\hypersetup{pdfcontactaddress={ETH Zurich, ITP, HIT K,
  Wolfgang-Pauli-Strasse 27}}
\hypersetup{pdfcontactpostcode={8093}}
\hypersetup{pdfcontactcity={Zurich}}
\hypersetup{pdfcontactcountry={Switzerland}}
\hypersetup{pdfcontactemail={nbeisert@itp.phys.ethz.ch}}
\hypersetup{pdfcontacturl={http://people.phys.ethz.ch/\xmptilde nbeisert/}}

\newcommand{\secref}[1]{\hyperref[#1]{section \ref*{#1}}}

\parskip1ex
\parindent0pt
\let\olditemize\itemize
\def\itemize{\olditemize\parskip0pt}

\begin{document}

\title{The \textsf{childdoc} Package}
\hypersetup{pdftitle={The childdoc Package}}
\author{Niklas Beisert\\[2ex]
  Institut f\"ur Theoretische Physik\\
  Eidgen\"ossische Technische Hochschule Z\"urich\\
  Wolfgang-Pauli-Strasse 27, 8093 Z\"urich, Switzerland\\[1ex]
  \href{mailto:nbeisert@itp.phys.ethz.ch}
  {\texttt{nbeisert@itp.phys.ethz.ch}}}
\hypersetup{pdfauthor={Niklas Beisert}}
\hypersetup{pdfsubject={Manual for the LaTeX2e Package childdoc}}
\date{30 December 2018, \textsf{v2.0}}
\maketitle

\begin{abstract}\noindent
\textsf{childdoc} is a \LaTeXe{} package
that enables the direct compilation
of document sections included by |\include|
to individual files.
\end{abstract}

\begingroup
\parskip0ex
\tableofcontents
\endgroup

%%%%%%%%%%%%%%%%%%%%%%%%%%%%%%%%%%%%%%%%%%%%%%%%%%%%%%%%%%%%%%%%%%%%%%%%%%%%%%%%
%%%%%%%%%%%%%%%%%%%%%%%%%%%%%%%%%%%%%%%%%%%%%%%%%%%%%%%%%%%%%%%%%%%%%%%%%%%%%%%%
\section{Introduction}

\LaTeX{} provides a mechanism to structure a large document (such as a book)
into a main file and several child files (containing the chapters)
using the |\include| command.
This mechanism is beneficial for documents
which span hundreds of pages in order to
make the source file(s) more manageable.
Moreover, compilation can be restricted to
selected child files by means of the |\includeonly| command.
The latter feature can be used to reduce the compilation time while editing
(this was significantly more useful in the earlier days of \LaTeX{})
or to generate a smaller document which is easier to navigate.
Another application of |\includeonly| is to generate
documents consisting of selected parts of the complete document.

However, there are a few drawbacks of the plain |\include| mechanism:
\begin{itemize}
\item
The child files cannot be compiled on their own,
they can only be compiled via the main file.
A naive editing environment
(such as a text editor with an option
to have the current file processed by \LaTeX)
may require one to switch to the main file before compiling;
attempting to compile the child file produces errors.
\item
The main file must be modified (each time)
to adjust the |\includeonly| command
to the present needs. This easily leaves the main file in a messy state.
\item
The generated document will always carry the filename
of the main document. This is inconvenient if
several child files are to be compiled and
to be kept for distribution.
\end{itemize}

The present package provides a simple interface
to make child files individually compilable by \LaTeX{}.
Compiling a child file then has the same effect as compiling
the main file with an |\includeonly| command
to select the appropriate child.
Moreover the generated document will carry the name of the child
rather than the main file.
This resolves all three above issues.

This feature is meant to make the editing of books,
thesis documents and lecture notes somewhat more convenient.
However, the package can also be used efficiently for
composing a series of documents (such as exercise sheets)
which are typically distributed individually.
It then assists the author in generating the individual documents
(potentially in different versions)
as well as a document containing the collected series.
Another application is in developing style files
or other kinds of included material
where compilation of the style file could redirect
to a sample or test file.

%%%%%%%%%%%%%%%%%%%%%%%%%%%%%%%%%%%%%%%%%%%%%%%%%%%%%%%%%%%%%%%%%%%%%%%%%%%%%%%%
%%%%%%%%%%%%%%%%%%%%%%%%%%%%%%%%%%%%%%%%%%%%%%%%%%%%%%%%%%%%%%%%%%%%%%%%%%%%%%%%
\section{Usage}

First of all, the package \textsf{childdoc} is \emph{not} a standard
\LaTeXe{} |.sty| style file! Therefore it needs to be invoked in
a non-standard way.

%%%%%%%%%%%%%%%%%%%%%%%%%%%%%%%%%%%%%%%%%%%%%%%%%%%%%%%%%%%%%%%%%%%%%%%%%%%%%%%%
\subsection{Included Files}
\label{sec:include}

%%%%%%%%%%%%%%%%%%%%%%%%%%%%%%%%%%%%%%%%
\DescribeMacro{\childdocmain}
To use the package, add the commands
\begin{center}
\begin{tabular}{l}
|\input{childdoc.def}|\\
|\childdocmain{}|\\
\end{tabular}
\end{center}
at the very top of the main \LaTeX{} file,
in particular \emph{before} the |\documentclass| statement!
The argument of |\childdocmain| should be left empty
(but it must be present).

%%%%%%%%%%%%%%%%%%%%%%%%%%%%%%%%%%%%%%%%
\DescribeMacro{\childdocof}
Furthermore, add the commands
\begin{center}
\begin{tabular}{l}
|\input{childdoc.def}|\\
|\childdocof{|\textit{main}|}|\\
\end{tabular}
\end{center}
at the top of every child file \textit{child}
which is included by |\include{|\textit{child}|}|
from within the main file
(or at least for those files to be compiled individually).
The argument \textit{main} must be the filename of the main file.

There are a couple of
considerations in setting up the main and child documents:

%%%%%%%%%%%%%%%%%%%%%%%%%%%%%%%%%%%%%%%%
\paragraph{Restrictions.}

Please note the following restrictions:
\begin{itemize}
\item
|\childdocmain| must be called with one argument \textit{main}
to ensure compatibility with earlier version of the package.
It must either be empty (|\childdocmain{}|)
or precisely match the filename of the main file in which it is specified.
See \secref{sec:detection} for further information.
\item
The filename \textit{main} must be specified without the |.tex| extension.
\item
The filename \textit{main} is case sensitive
(even in case-insensitive file systems)
due to internal string comparison.
\item
The argument \textit{main} should be fully expanded, it cannot be a macro.
\item
Subdirectories and special characters should be avoided in filenames.
\item
The command |\childdocmain{|\textit{main}|}| must be followed by a whitespace.
It should not be followed immediately by another command
or by a comment mark `|%|'.
This is because the \TeX{} parser reads the token immediately following
the argument of |\childdocmain| and puts it
at the beginning of every child section;
however, a white\-space is ignored.
\end{itemize}

%%%%%%%%%%%%%%%%%%%%%%%%%%%%%%%%%%%%%%%%
\paragraph{Content of Main File.}

It is advisable to place all content in the child files included by |\include|.
Any output contained in the main file will appear in all child documents
unless suppressed manually;
it cannot be suppressed automatically by the |\includeonly| directive
and thus should normally be avoided.
A method to include some content in the main file
by means of conditional processing is described in \secref{sec:conditional}.

%%%%%%%%%%%%%%%%%%%%%%%%%%%%%%%%%%%%%%%%
\paragraph{Page Numbering.}

When only a part of the document is compiled,
the appropriate numbering of pages
(as well as other status parameters)
is determined from the |.aux| files.
The latter contain information from previous passes.
However this information needs to propagate through
all intermediate child documents.
Therefore the page numbering in child documents may well
be inconsistent until the complete document is compiled at least once.

A useful (if unconventional) way to always ensure a consistent
page numbering is to restart the numbering in each child document
and denote the pages by `\textit{child}|.|\textit{page}'
where \textit{child} represents the chapter/section number of the child file.
This can be achieved by the command
|\numberwithin{page}{|\textit{child}|}|
of the \textsf{amsmath} package
where \textit{child} can be |chapter| or |section|
depending on the chosen structuring.
Alternatively, one can modify the macro |\thepage| appropriately
and reset the counter |page| at the start of each child file.

%%%%%%%%%%%%%%%%%%%%%%%%%%%%%%%%%%%%%%%%%%%%%%%%%%%%%%%%%%%%%%%%%%%%%%%%%%%%%%%%
\subsection{Conditional Processing}
\label{sec:conditional}

The package provides a mechanism to compile different versions
of a document. To customise the versions further some conditional processing
can come in handy to distinguish which version is being compiled.
The package provides two macros to describe the compilation context:

%%%%%%%%%%%%%%%%%%%%%%%%%%%%%%%%%%%%%%%%
\DescribeMacro{\ifchilddoc}
The conditional |\ifchilddoc| distinguishes between the compilation of
child documents and the main document:
%
\begin{center}
|\ifchilddoc |\textit{child-code}| |[|\||else |\textit{main-code}]| \||fi|
\end{center}

%%%%%%%%%%%%%%%%%%%%%%%%%%%%%%%%%%%%%%%%
\DescribeMacro{\childdocname}
\DescribeMacro{\childdocjob}
The macro |\childdocname| contains the filename (without extension)
of the main or child file being processed.
Note that |\childdocjob| will always contain the name of the main file.

%%%%%%%%%%%%%%%%%%%%%%%%%%%%%%%%%%%%%%%%
\paragraph{Title Page.}

Conditional processing can be used to include a title or banner page
in the main document when proper precautions are taken.
Importantly, the code in the main file should ensure that the page counter
(as well as other status parameters which are stored in the |.aux| files)
takes the same value after the conditional processing.
Otherwise the page numbers may take divergent values
depending on which part is compiled.

For example, a title page could be declared by:
%
\begin{center}
\begin{tabular}{l}
|\ifchilddoc\||else|\\
|\addtocounter{page}{-1}|\\
\textit{code for title page}\\
|\newpage|\\
|\||fi|
\end{tabular}
\end{center}
%
A banner page for the child documents can be generated by:
%
\begin{center}
\begin{tabular}{l}
|\ifchilddoc|\\
|\addtocounter{page}{-1}|\\
\textit{code for banner page}\\
|\newpage|\\
|\||fi|
\end{tabular}
\end{center}
%
Here one could write a message such as:
\begin{center}
|This is the part \childdocname{} of \childdocjob{}.|
\end{center}

%%%%%%%%%%%%%%%%%%%%%%%%%%%%%%%%%%%%%%%%%%%%%%%%%%%%%%%%%%%%%%%%%%%%%%%%%%%%%%%%
\subsection{Flags}
\label{sec:flags}

The package makes it easy to generate different versions
of the main or child documents.
To this end compilation flags can be defined
and assigned different default values.
They will be particularly useful in conjunction
with the forwarding mechanism described in \secref{sec:forward}.

For example, it may be useful to have a flag |\version|
which can be set to |draft| or |final|.
The document source will contain some conditional code
depending on the value of |\version|.
Suppose further, the flag should default to |final| for the main file
and to |draft| for child files
which is a natural assignment for editing the document.
This is achieved by placing the following code
in the preamble of the main document
(below the |\childdocmain| directive):
%
\begin{center}
\begin{tabular}{l}
|\ifchilddoc|\\
|\providecommand{\version}{draft}|\\
|\||else|\\
|\providecommand{\version}{final}|\\
|\||fi|
\end{tabular}
\end{center}
%
The definition by |\providecommand| makes sure
that previous definitions are not overwritten.
Further statements |\providecommand{\version}{...}|
can thus be added before the above code to override it.

For the main file, one might add a line
(between |\childdocmain| and the above block)
%
\begin{center}
|%\ifchilddoc\||else\providecommand{\version}{draft}\||fi|
\end{center}
%
which can be uncommented to produce a draft version.
Likewise one can add a line to the very top of a child file
(above the |\childdocof{|\textit{main}|}| directive)
%
\begin{center}
|%\providecommand{\version}{final}|
\end{center}
%
which can be uncommented to produce the final version of this child document.

%%%%%%%%%%%%%%%%%%%%%%%%%%%%%%%%%%%%%%%%%%%%%%%%%%%%%%%%%%%%%%%%%%%%%%%%%%%%%%%%
\subsection{Forwarding}
\label{sec:forward}

Different versions of the main or child documents
using compilation flags as described in \secref{sec:flags}
can be (permanently) stored in different files
for convenient compilation, viewing and distribution.
To this end, the package defines a command
to pass on compilation to a different file:

%%%%%%%%%%%%%%%%%%%%%%%%%%%%%%%%%%%%%%%%
\DescribeMacro{\childdocforward}
The command |\childdocforward| redirects processing to
another source file:
%
\begin{center}
\begin{tabular}{l}
|\input{childdoc.def}|\\
|\childdocforward[|\textit{main}|]{|\textit{dest}|}|\\
\end{tabular}
\end{center}
%
The argument \textit{dest} is the destination file
(without extension).
It should be the main file or one of the child files.
Note that further \textsf{childdoc} directives
such as |\childdocof| and |\childdocforward|
in the indicated file will be processed in this form.
The optional argument \textit{main}
passes on directly to the main file \textit{main}
while pretending to compile the child \textit{dest}.
This form behaves as if \textit{dest}
issues |\childdocof{|\textit{main}|}| right away,
and no further \textsf{childdoc} directives will be processed.

%%%%%%%%%%%%%%%%%%%%%%%%%%%%%%%%%%%%%%%%
\DescribeMacro{\...prefix}
In the alternative form |\childdocforwardprefix|,
%
\begin{center}
\begin{tabular}{l}
|\input{childdoc.def}|\\
|\childdocforwardprefix[|\textit{main}|]{|\textit{prefix}|}{|\textit{dest}|}|
\end{tabular}
\end{center}
%
the destination file is determined by a pattern
depending on the current file:
To make this work, the current file must be called
`{\textit{prefix}\hspace{0.2em}\textit{suffix}}'
with \textit{prefix} matching precisely the argument.
Processing is then passed on to the file
`{\textit{dest}\hspace{0.2em}\textit{suffix}}'.
Surely, the same effect is achieved by
directly specifying the
argument `{\textit{dest}\hspace{0.2em}\textit{suffix}}'
in the first form.
However, that requires to set up a different file
for each child. With the alternative form of the command
all these files can have exactly the same content
which simplifies setting them up and maintaining them.

For example, the following file |draft.tex|
with a compilation flag |\version| as described in \secref{sec:flags}
compiles the main document as a draft:
%
\begin{center}
\begin{tabular}{l}
|\def\version{draft}|\\
|\input{childdoc.def}|\\
|\childdocforward{|\textit{main}|}|
\end{tabular}
\end{center}
%
Likewise, the following files |final|\textit{nn}|.tex|
compile the final version of the child document
|child|\textit{nn}|.tex|:
%
\begin{center}
\begin{tabular}{l}
|\def\version{final}|\\
|\input{childdoc.def}|\\
|\childdocforwardprefix{final}{child}|
\end{tabular}
\end{center}
%

Note that when several versions of a main file and/or of each child file
are to be generated, it may be convenient to set up a |Makefile| or
shell script to automatise the process.

%%%%%%%%%%%%%%%%%%%%%%%%%%%%%%%%%%%%%%%%%%%%%%%%%%%%%%%%%%%%%%%%%%%%%%%%%%%%%%%%
\subsection{Command Line Processing}
\label{sec:commandline}

The effect of redirection files can also be achieved by invoking
the \LaTeX{} compiler with a more elaborate command line.
Most conveniently this should be done as part
of a shell script or a |Makefile|.

When using \textsf{childdoc} in the main file, the following
command lines effectively perform a redirection
(note that depending on the shell being used,
backslashes may have to be doubled: `|\|' $\to$ `|\\|'):
%
\begin{center}
|... -jobname "|\textit{target}|" |\\|"|[\textit{flags}]%
|\input{childdoc.def}\childdocforward[|\textit{main}|]{|\textit{dest}|}"|
\end{center}
%
Here \textit{target} is the name of the output file,
\textit{main} is the name of the main file
and \textit{dest} is the name of the main or child file to be processed
(all filenames without extensions).
The optional argument \textit{main} can be omitted
if \textit{main} matches \textit{dest}.
Optionally, compilation \textit{flags} can be defined via |\def| commands.
This command line makes the \TeX{} engine believe
it is compiling the file \textit{target}
whose content is specified as the latter parameter.
The provided code then forwards the processing to
\textit{main} or \textit{dest} as described in \secref{sec:forward}.

%%%%%%%%%%%%%%%%%%%%%%%%%%%%%%%%%%%%%%%%%%%%%%%%%%%%%%%%%%%%%%%%%%%%%%%%%%%%%%%%
\subsection{Include by Input}
\label{sec:input}

Including child documents by |\include| has some restrictions by design.
Most notably, the content of a child document always occupies
its own set of pages; pages cannot be shared between child documents.
Usually, this behaviour makes perfect sense
because each child document contain an essential part of the document.
However, in some situations it may be desirable to compose
a document from a collection of parts
without having mandatory page breaks between then.
For this case, the package
provides a mechanism to include parts
by |\input| which can also be processed individually.
However, by construction this mechanism
requires manual handling of the content to be output.

%%%%%%%%%%%%%%%%%%%%%%%%%%%%%%%%%%%%%%%%
\DescribeMacro{\ifchilddocmanual}
The main file should be prepared as usual, see \secref{sec:include}.
However, the document body must make a distinction
between processing of an individual part and of the main document, e.g.:
%
\begin{center}
\begin{tabular}{l}
|\ifchilddocmanual|\\
|\input{\childdocname}|\\
|\||else|\\
\textit{document body with }|\input{|\textit{part}|}|\\
|\||fi|
\end{tabular}
\end{center}
%
The conditional |\ifchilddocmanual| is true whenever
a part to be included by |\input| is being compiled,
and the name of the part is stored in |\childdocname|.

%%%%%%%%%%%%%%%%%%%%%%%%%%%%%%%%%%%%%%%%
\DescribeMacro{\childdocby}
Each part to be included by |\input| should start with:
%
\begin{center}
\begin{tabular}{l}
|\input{childdoc.def}|\\
|\childdocby{|\textit{main}|}|\\
\end{tabular}
\end{center}
%
The directive |\childdocby| is similar to |\childdocof|
described in \secref{sec:include},
but the subsequent selection of content must be done manually.
To that end, both |\ifchilddoc| and |\ifchilddocmanual|
will be true upon processing of a part,
and the name of the part is stored in |\childdocname|.
Note that |\jobname| will be set to the filename of the current part
so that each part receives an individual |.aux| file
that does not interfere with the |.aux| file(s) of the main document.
This behaviour can be altered by the alternative form
|\childdocby[*]{|\textit{main}|}| (with a non-empty optional argument)
which uses the |.aux| file of the main document
by setting |\jobname| to \textit{main}.

%%%%%%%%%%%%%%%%%%%%%%%%%%%%%%%%%%%%%%%%%%%%%%%%%%%%%%%%%%%%%%%%%%%%%%%%%%%%%%%%
\subsection{Driver Development}
\label{sec:driver}

The \textsf{childdoc} mechanism can also be use for the development
of definition files such as \LaTeX{} styles or classes.
This case differs from the above setup with multiple parts
included by |\include| in that no |\includeonly| should be invoked.
This can be achieved by starting the include file
(before |\ProvidesPackage|) with:
%
\begin{center}
\begin{tabular}{l}
|\input{childdoc.def}|\\
|\childdocforward{|\textit{main}|}|\\
\end{tabular}
\end{center}
%
or alternatively with:
%
\begin{center}
\begin{tabular}{l}
|\input{childdoc.def}|\\
|\childdocby{|\textit{main}|}|\\
\end{tabular}
\end{center}
%
Both forms have slightly different effects as described above.
The main file is prepared as usual, see \secref{sec:include}.

%%%%%%%%%%%%%%%%%%%%%%%%%%%%%%%%%%%%%%%%%%%%%%%%%%%%%%%%%%%%%%%%%%%%%%%%%%%%%%%%
\subsection{Legacy Detection}
\label{sec:detection}

The directive |\childdocmain| in the main file can detect
whether the complete document or merely a child is to be compiled
even without using the directive |\childdocof|.
This method is deprecated because it is less robust
and there is no compelling reason to use it;
it is merely provided for backward compatibility
and it may be removed in future versions.

If the detection mechanism is to be used,
it is mandatory to correctly specify
the filename of the main file as the argument of |\childdocmain|:
%
\begin{center}
\begin{tabular}{l}
|\input{childdoc.def}|\\
|\childdocmain{|\textit{main}|}|\\
\end{tabular}
\end{center}
%
If |\jobname| does not match the argument \textit{main} of |\childdocmain|,
it is assumed that |\jobname| points to the child file to be compiled.
When using |\childdocmain| with the main file specified as argument,
it suffices to start a child file
with just |\input{|\textit{main}|}|
without loading of the package and using |\childdocof|.
If instead all processing is done
with the appropriate \textsf{childdoc} directives,
the argument of \textit{main} of |\childdocmain| can be empty.

An alternative version of the command line processing described
in \secref{sec:commandline} using the detection mechanism reads:
%
\begin{center}
|... -jobname "|\textit{target}|" "|[\textit{flags}]%
[|\def\jobname{|\textit{dest}|}|]|\input{|\textit{main}|}"|
\end{center}

%%%%%%%%%%%%%%%%%%%%%%%%%%%%%%%%%%%%%%%%%%%%%%%%%%%%%%%%%%%%%%%%%%%%%%%%%%%%%%%%
\subsection{Manual Code}
\label{sec:manual}

In case one cannot be certain whether the definitions file |childdoc.def|
is installed on the target \TeX{} distribution
and one prefers not to ship it,
it is conceivable to paste a few relevant commands into the sources.

To that end, drop all statements |\input{childdoc.def}|
and perform the replacements as outlined below.
Instead of |\childdocmain{|\textit{main}|}| add the following code
to the top of the main file:
%
\begin{center}
\begin{tabular}{l}
|\||ifdefined\childdocname\endinput\||fi\newif\ifchilddoc|\\
|\edef\childdocname{\scantokens\expandafter{\jobname\noexpand}}|\\
|\def\childdocmain{|\textit{main}|}\||ifx\childdocmain\childdocname\||else|\\
|\childdoctrue\includeonly{\childdocname}\let\jobname\childdocmain\||fi|\\
\end{tabular}
\end{center}
%
Instead of |\childdocof{|\textit{main}|}| just include the main file
at the top of each child file:
%
\begin{center}
|\input{|\textit{main}|}|
\end{center}
%
A simple redirection |\childdocforward{|\textit{dest}|}| is achieved by:
%
\begin{center}
|\def\jobname{|\textit{dest}|}\input{\jobname}|
\end{center}
%
The redirection with prefix
|\childdocforwardprefix[|\textit{prefix}|]{|\textit{dest}|}|
is accomplished by:
%
\begin{center}
\begin{tabular}{l}
|{\edef\jobname{\scantokens\expandafter{\jobname\noexpand}}|\\
|\def\redirectjob |\textit{prefix}|#1~~~{\gdef\jobname{|\textit{dest}|#1}}|\\
|\expandafter\redirectjob\jobname~~~}\input{\jobname}|
\end{tabular}
\end{center}

In an alternative approach,
child documents can be compiled by a specific command line
without additional code or specific definitions:
%
\begin{center}
|... -jobname "|\textit{target}|" "|[\textit{flags}]%
|\includeonly{|\textit{dest}|}\input{|\textit{main}|}"|
\end{center}
%

%%%%%%%%%%%%%%%%%%%%%%%%%%%%%%%%%%%%%%%%%%%%%%%%%%%%%%%%%%%%%%%%%%%%%%%%%%%%%%%%
%%%%%%%%%%%%%%%%%%%%%%%%%%%%%%%%%%%%%%%%%%%%%%%%%%%%%%%%%%%%%%%%%%%%%%%%%%%%%%%%
\section{Information}

%%%%%%%%%%%%%%%%%%%%%%%%%%%%%%%%%%%%%%%%%%%%%%%%%%%%%%%%%%%%%%%%%%%%%%%%%%%%%%%%
\subsection{Copyright}

Copyright \copyright{} 2017--2018 Niklas Beisert

This work may be distributed and/or modified under the
conditions of the \LaTeX{} Project Public License, either version 1.3
of this license or (at your option) any later version.
The latest version of this license is in
  \url{http://www.latex-project.org/lppl.txt}
and version 1.3 or later is part of all distributions of \LaTeX{}
version 2005/12/01 or later.

This work has the LPPL maintenance status `maintained'.

The Current Maintainer of this work is Niklas Beisert.

This work consists of the files |README.txt|, |childdoc.ins| and |childdoc.dtx|
as well as the derived files |childdoc.def|, |cdocsamp.tex|
with |cdocsch1.tex|, |cdocsch2.tex|, |cdocspt3.tex|, |cdocspt4.tex|,
|cdocsdrf.tex|, |cdocsfn1.tex|, |cdocsfn2.tex|
as well as |childdoc.pdf|.

%%%%%%%%%%%%%%%%%%%%%%%%%%%%%%%%%%%%%%%%%%%%%%%%%%%%%%%%%%%%%%%%%%%%%%%%%%%%%%%%
\subsection{Files and Installation}

The package consists of the files:
%
\begin{center}
\begin{tabular}{ll}
    |README.txt|   & readme file \\
    |childdoc.ins| & installation file \\
    |childdoc.dtx| & source file \\
    |childdoc.def| & definition file \\
    |cdocsamp.tex| & sample main file \\
    |cdocsch1.tex| & sample include file \\
    |cdocsch2.tex| & sample include file \\
    |cdocspt3.tex| & sample part file \\
    |cdocspt4.tex| & sample part file \\
    |cdocsdrf.tex| & sample redirection file \\
    |cdocsfn1.tex| & sample redirection file \\
    |cdocsfn2.tex| & sample redirection file \\
    |childdoc.pdf| & manual
\end{tabular}
\end{center}
%
The distribution consists of the files
|README.txt|, |childdoc.ins| and |childdoc.dtx|.
%
\begin{itemize}
\item
Run (pdf)\LaTeX{} on |childdoc.dtx|
to compile the manual |childdoc.pdf| (this file).
\item
Run \LaTeX{} on |childdoc.ins| to create the definitions file |childdoc.def|
and the sample |cdocsamp.tex| with include files
|cdocsch1.tex|, |cdocsch2.tex|, |cdocspt3.tex|, |cdocspt4.tex|,
|cdocsdrf.tex|, |cdocsfn1.tex|, |cdocsfn2.tex|.
Then copy the file |childdoc.def| to an appropriate directory of your \LaTeX{}
distribution, e.g.\ \textit{texmf-root}|/tex/latex/childdoc|.
\end{itemize}

%%%%%%%%%%%%%%%%%%%%%%%%%%%%%%%%%%%%%%%%%%%%%%%%%%%%%%%%%%%%%%%%%%%%%%%%%%%%%%%%
\subsection{Related CTAN Packages}

There are several other packages which offer a similar functionality:
%
\begin{itemize}
\item
The packages
\href{http://ctan.org/pkg/docmute}{\textsf{docmute}},
\href{http://ctan.org/pkg/includex}{\textsf{includex}} and
\href{http://ctan.org/pkg/standalone}{\textsf{standalone}}
provide commands to include only the document body of
a child file thus allowing both files to be compiled individually.
\item
The packages \href{http://ctan.org/pkg/subdocs}{\textsf{subdocs}}
and \href{http://ctan.org/pkg/subfiles}{\textsf{subfiles}}
provide structures in which the main and child documents can be
encapsulated and allowing them to be compiled individually.
The inclusion mechanism is different from the conventional |\include|.
\item
The package \href{http://ctan.org/pkg/combine}{\textsf{combine}}
is an elaborate solution to combine several documents into one.
\end{itemize}
%
See also the CTAN topic \href{http://ctan.org/topic/subdocs}{\textsf{subdocs}}
for further related packages.
The present package differs from the above solutions in that
a document structure constructed with the conventional |\include| mechanism
just needs two extra commands at the top of every file
such that all constituent files can be compiled individually.

%%%%%%%%%%%%%%%%%%%%%%%%%%%%%%%%%%%%%%%%%%%%%%%%%%%%%%%%%%%%%%%%%%%%%%%%%%%%%%%%
%\subsection{Feature Suggestions}
%
%The following is a list of features which may be useful for future
%versions of this package:
%%
%\begin{itemize}
%\item
%\ldots
%\end{itemize}

%%%%%%%%%%%%%%%%%%%%%%%%%%%%%%%%%%%%%%%%%%%%%%%%%%%%%%%%%%%%%%%%%%%%%%%%%%%%%%%%
\subsection{Revision History}

%%%%%%%%%%%%%%%%%%%%%%%%%%%%%%%%%%%%%%%%
\paragraph{v2.0:} 2018/12/30

\begin{itemize}
\item
immediate forward processing
\item
added |\childdocby| mechanism
\item
manual restructured
\end{itemize}

%%%%%%%%%%%%%%%%%%%%%%%%%%%%%%%%%%%%%%%%
\paragraph{v1.6:} 2018/01/17

\begin{itemize}
\item
application for development of include files
\item
corrections to manual
\end{itemize}

%%%%%%%%%%%%%%%%%%%%%%%%%%%%%%%%%%%%%%%%
\paragraph{v1.5:} 2017/05/21

\begin{itemize}
\item
more complete structuring introduced
\item
|\childdocof| introduced
\item
|\childdoc| renamed to |\childdocmain|
\item
|\childredirect| renamed to |\childdocforward| and |\childdocforwardprefix|
and functionality expanded
\end{itemize}

%%%%%%%%%%%%%%%%%%%%%%%%%%%%%%%%%%%%%%%%
\paragraph{v1.0:} 2017/04/27

\begin{itemize}
\item
manual and install package
\item
first version published on CTAN
\end{itemize}

%%%%%%%%%%%%%%%%%%%%%%%%%%%%%%%%%%%%%%%%
\paragraph{v0.6:} 2017/04/26

\begin{itemize}
\item
redirection mechanism added
\end{itemize}

%%%%%%%%%%%%%%%%%%%%%%%%%%%%%%%%%%%%%%%%
\paragraph{v0.5:} 2017/04/26

\begin{itemize}
\item
functionality in definition file
\end{itemize}


%%%%%%%%%%%%%%%%%%%%%%%%%%%%%%%%%%%%%%%%%%%%%%%%%%%%%%%%%%%%%%%%%%%%%%%%%%%%%%%%
%%%%%%%%%%%%%%%%%%%%%%%%%%%%%%%%%%%%%%%%%%%%%%%%%%%%%%%%%%%%%%%%%%%%%%%%%%%%%%%%
%%%%%%%%%%%%%%%%%%%%%%%%%%%%%%%%%%%%%%%%%%%%%%%%%%%%%%%%%%%%%%%%%%%%%%%%%%%%%%%%
\appendix

\settowidth\MacroIndent{\rmfamily\scriptsize 000\ }

 \DocInput{childdoc.dtx}

\end{document}
%</driver>
% \fi
%
% %%%%%%%%%%%%%%%%%%%%%%%%%%%%%%%%%%%%%%%%%%%%%%%%%%%%%%%%%%%%%%%%%%%%%%%%%%%%%%
% %%%%%%%%%%%%%%%%%%%%%%%%%%%%%%%%%%%%%%%%%%%%%%%%%%%%%%%%%%%%%%%%%%%%%%%%%%%%%%
% \section{Sample}
%\iffalse
%<*samplemain>
%\fi
%
% The following presents a sample document
% with two chapters, two parts, a title page,
% a compile flag as well as three forwarding files to set the flag.
% It consists of eight |.tex| files:
% \begin{center}
% \begin{tabular}{ll}
% |cdocsamp.tex|&main file\\
% |cdocsch1.tex|&include file for chapter 1\\
% |cdocsch2.tex|&include file for chapter 2\\
% |cdocspt3.tex|&include file for part 3\\
% |cdocspt4.tex|&include file for part 4\\
% |cdocsdrf.tex|&forwarding file for main file in draft mode\\
% |cdocsfi1.tex|&forwarding file for final version of chapter 1\\
% |cdocsfi2.tex|&forwarding file for final version of chapter 2\\
% \end{tabular}
% \end{center}
% Each of the eight files can be compiled directly by the \LaTeX{} compiler.
%
% %%%%%%%%%%%%%%%%%%%%%%%%%%%%%%%%%%%%%%
% \paragraph{Main File.}
%
% The main file is called |cdocsamp.tex|.
%
% Load the \textsf{childdoc} definitions and
% declare the filename for the main document:
%    \begin{macrocode}
\input{childdoc.def}
\childdocmain{}
%    \end{macrocode}

% Optional override for |\version| flag:
%    \begin{macrocode}
%%\ifchilddoc\else\providecommand{\version}{draft}\fi
%    \end{macrocode}

% Define the default values for the |\version| flag
% (|final| for the main file and |draft| for childs):
%    \begin{macrocode}
\ifchilddoc
\providecommand{\version}{draft}
\else
\providecommand{\version}{final}
\fi
%    \end{macrocode}

% Load the standard document class:
%    \begin{macrocode}
\documentclass[12pt]{article}
%    \end{macrocode}

% Start the document body:
%    \begin{macrocode}
\begin{document}
%    \end{macrocode}

% Declare a title page.
% Print title, part of document being processed and version flag:
%    \begin{macrocode}
\addtocounter{page}{-1}
\begin{center}
{\LARGE\bfseries{}childdoc example\par}
\vspace{1cm}
\ifchilddoc
\ifchilddocmanual part\else chapter\fi:
`\childdocname' of `\childdocjob'\par
\else
main document: `\childdocjob'\par
\fi
version: \version\par
\end{center}
\newpage
%    \end{macrocode}

% Manually include selected file,
% otherwise process as usual:
%    \begin{macrocode}
\ifchilddocmanual
\section*{part `\childdocname'}
\input{\childdocname}
\else
%    \end{macrocode}

% Include the two chapters:
%    \begin{macrocode}
\include{cdocsch1}
\include{cdocsch2}
%    \end{macrocode}

% Include the two parts unless only chapters should be displayed:
%    \begin{macrocode}
\ifchilddoc\else
\section{part three}
\input{cdocspt3}
\section{part four}
\input{cdocspt4}
\fi
%    \end{macrocode}

% Process as usual until here:
%    \begin{macrocode}
\fi
%    \end{macrocode}

% End of document body:
%    \begin{macrocode}
\end{document}
%    \end{macrocode}
%\iffalse
%</samplemain>
%\fi
%
% %%%%%%%%%%%%%%%%%%%%%%%%%%%%%%%%%%%%%%
% \paragraph{Chapter Include Files.}
%
% The include files are called |cdocsch1.tex| and |cdocsch2.tex|.
%
%\iffalse
%<*samplechap1|samplechap2>
%\fi

% Optional override for |\version| flag:
%    \begin{macrocode}
%%\providecommand{\version}{final}
%    \end{macrocode}

% Include the main document:
%    \begin{macrocode}
\input{childdoc.def}
\childdocof{cdocsamp}
%    \end{macrocode}

%\iffalse
%</samplechap1|samplechap2>
%\fi
%
%\iffalse
%<*samplechap1>
%\fi
% Some text for chapter 1:
%    \begin{macrocode}
\section{one}
some text in chapter one
%    \end{macrocode}

%\iffalse
%</samplechap1>
%\fi
% Some text for chapter 2:
%\iffalse
%<*samplechap2>
%\fi
%    \begin{macrocode}
\section{two}
more text in chapter two
%    \end{macrocode}

%\iffalse
%</samplechap2>
%\fi
%
% %%%%%%%%%%%%%%%%%%%%%%%%%%%%%%%%%%%%%%
% \paragraph{Part Include Files.}
%
% The include files are called |cdocspt3.tex| and |cdocspt4.tex|.
%
%\iffalse
%<*samplepart3|samplepart4>
%\fi

% Optional override for |\version| flag:
%    \begin{macrocode}
%%\providecommand{\version}{final}
%    \end{macrocode}

% Include the main document:
%    \begin{macrocode}
\input{childdoc.def}
\childdocby{cdocsamp}
%    \end{macrocode}

%\iffalse
%</samplepart3|samplepart4>
%\fi
%
%\iffalse
%<*samplepart3>
%\fi
% Some text for part 3:
%    \begin{macrocode}
some text in part three
%    \end{macrocode}

%\iffalse
%</samplepart3>
%\fi
% Some text for part 4:
%\iffalse
%<*samplepart4>
%\fi
%    \begin{macrocode}
more text in part four
%    \end{macrocode}

%\iffalse
%</samplepart4>
%\fi
%
% %%%%%%%%%%%%%%%%%%%%%%%%%%%%%%%%%%%%%%
% \paragraph{Forwarding for a Complete Draft.}
%
% The following forwarding file |cdocsdrf.tex|
% compiles the main document in draft mode:
%\iffalse
%<*sampledraft>
%\fi
%    \begin{macrocode}
\def\version{draft}
\input{childdoc.def}
\childdocforward{cdocsamp}
%    \end{macrocode}

%\iffalse
%</sampledraft>
%\fi
%
% %%%%%%%%%%%%%%%%%%%%%%%%%%%%%%%%%%%%%%
% \paragraph{Forwarding for Final Version of the Chapters.}
%
% The following forwarding files |cdocsfn1.tex| and |cdocsfn2.tex|
% (with identical content)
% compile the final versions of the child documents
% |cdocsch1.tex| and |cdocsch2.tex|, respectively:
%\iffalse
%<*samplefinal>
%\fi
%    \begin{macrocode}
\def\version{final}
\input{childdoc.def}
\childdocforwardprefix[cdocsamp]{cdocsfn}{cdocsch}
%    \end{macrocode}

%\iffalse
%</samplefinal>
%\fi
%
% %%%%%%%%%%%%%%%%%%%%%%%%%%%%%%%%%%%%%%
% \paragraph{Command Line Processing.}
%
% The following three command lines generate the output files
% |cdocscld|, |cdocscl1| and |cdocscl2|
% which should be identical to
% |cdocsdrf|, |cdocsch1| and |cdocsfn2|, respectively:
% \begin{center}
% \begin{tabular}{l}
% |latex -jobname cdocscld \|\\
% |  "\def\version{draft}\input{childdoc.def}\childdocforward{cdocsamp}"|\\
% |latex -jobname cdocscl1 \|\\
% |  "\input{childdoc.def}\childdocforward[cdocsamp]{cdocsch1}"|\\
% |latex -jobname cdocscl2 \|\\
% |  "\def\version{final}\input{childdoc.def}\childdocforward{cdocsch2}"|
% \end{tabular}
% \end{center}
% Note that the trailing backslash on each first line
% merely continues the input to the second line
% (for convenient cut ant paste).
% Furthermore, the command |latex| can be replaced by any
% of its alternative versions such as |pdflatex|.
%
% %%%%%%%%%%%%%%%%%%%%%%%%%%%%%%%%%%%%%%%%%%%%%%%%%%%%%%%%%%%%%%%%%%%%%%%%%%%%%%
% %%%%%%%%%%%%%%%%%%%%%%%%%%%%%%%%%%%%%%%%%%%%%%%%%%%%%%%%%%%%%%%%%%%%%%%%%%%%%%
% \section{Implementation}
%\iffalse
%<*package>
%\fi
%
% This section describes the definitions file |childdoc.def|.

% The definitions cannot be loaded using |\usepackage| or |\RequirePackage|
% which has a mechanism to prevent loading a style file more than once.
% When loading the definitions by means of |\input|
% multiple instances have to be prevented manually:
%\iffalse
%This code needs to be before the `\ProvidesFile' directive
%which is defined at the beginning of this file.
%Therefore it is also placed there and commented out here.
%</package>
%<*discard>
%\fi
%    \begin{macrocode}
\ifdefined\childdocmain\endinput\fi
%    \end{macrocode}
%\iffalse
%</discard>
%<*package>
%\fi
%
% \macro{\ifchilddoc}
% \macro{\ifchilddocmanual}
% The conditional |\ifchilddoc| tells whether a
% child (true) or main (false) document is being compiled.
% The conditional |\ifchilddocmanual| tells whether
% the |\includeonly| mechanism is used (false) or
% the selection of child files must be performed manually (true).
% The definitions initialise to false:
%    \begin{macrocode}
\newif\ifchilddoc
\newif\ifchilddocmanual
%    \end{macrocode}

% \macro{\childdocname}
% \macro{\childdocjob}
% The macro |\childdocname| stores the name of the main document
% to be compiled. The macro |\childdocjob| stores the name of
% the document on which the \LaTeX{} compiler was originally invoked.
% The content of |\jobname| cannot be compared
% to filenames specified in the source due to different catcodes.
% The following code rescans |\jobname|, stores the result
% in |\childdocname| and saves a copy in |\childdocjob|:
%    \begin{macrocode}
\edef\childdocname{\scantokens\expandafter{\jobname\noexpand}}
\let\childdocjob\childdocname
%    \end{macrocode}

% \macro{\childdocdisable}
% The macro |\childdocdisable| prevents the main file
% from being processed more than once.
% At this stage, the main document command |\childdocmain|
% is assumed to be called once again where it should do nothing.
% Any subsequent call to it should prevent
% a secondary processing of the main document
% It overwrites the forwarding commands
% |\childdocof| and |\childdocforward|
% with empty macros to prevent further inclusions of the main document:
%    \begin{macrocode}
\newcommand{\childdocdisable}
{
  \renewcommand{\childdocmain}[1]{\renewcommand{\childdocmain}[1]{\endinput}}
  \renewcommand{\childdocof}[1]{}
  \renewcommand{\childdocby}[2][]{}
  \renewcommand{\childdocforward}[2][]{}
  \renewcommand{\childdocdisable}{}
}
%    \end{macrocode}

% \macro{\childdocmain}
% The macro |\childdocmain| is to be called at the top of the main file
% with nothing or the main filename (without extension) as argument.
% First, it breaks loops.
% If the argument is not empty and does not match |\childdocname|
% (which is set by the first inclusion of |childdoc.def|),
% |\ifchilddoc| is set to true, |\includeonly| is applied to the child file
% and |\jobname| is set to the main file
% (for proper handling of |.aux| files):
%    \begin{macrocode}
\newcommand{\childdocmain}[1]
{
  \childdocdisable\childdocmain{}
  \if?#1?\else
    \begingroup
      \def\childdoctmp{#1}
      \ifx\childdoctmp\childdocname
        \def\childdoctmp{}
      \else
        \def\childdoctmp
        {
          \childdoctrue
          \includeonly{\childdocname}
          \def\childdocjob{#1}
          \def\jobname{#1}
        }
      \fi
      \expandafter
    \endgroup
    \childdoctmp
  \fi
}
%    \end{macrocode}

% \macro{\childdocof}
% The command |\childdocof| redirects
% compilation to the main file |#1|.
%    \begin{macrocode}
\newcommand{\childdocof}[1]
{
  \childdocdisable
  \childdoctrue
  \includeonly{\childdocname}
  \def\jobname{#1}
  \def\childdocjob{#1}
  \input{#1}
}
%    \end{macrocode}

% \macro{\childdocby}
% The command |\childdocby| ....
%    \begin{macrocode}
\newcommand{\childdocby}[2][]
{
  \childdocdisable
  \childdoctrue
  \childdocmanualtrue
  \if?#1?\else
    \def\jobname{#2}
  \fi
  \def\childdocjob{#2}
  \input{#2}
  \endinput
}
%    \end{macrocode}

% \macro{\childdocforward}
% The command |\childdocforward| redirects
% compilation to the main file or
% (if the optional argument is given) a child file.
% Parameters are set as if the main file
% or a child file starting with |\childdocof| was compiled.
% Then compilation is handed over to the main file:
%    \begin{macrocode}
\newcommand{\childdocforward}[2][]
{
  \begingroup
    \if?#1?
      \def\childdoctmp
      {
        \def\childdocname{#2}
        \def\childdocjob{#2}
        \def\jobname{#2}
        \input{#2}
        \endinput
      }
    \else
      \def\childdoctmp
      {
        \childdocdisable
        \def\childdocname{#2}
        \childdoctrue
        \includeonly{#2}
        \def\childdocjob{#1}
        \def\jobname{#1}
        \input{#1}
        \endinput
      }
    \fi
    \expandafter
  \endgroup
  \childdoctmp
}
%    \end{macrocode}

% \macro{\childdocforwardprefix}
% The command |\childdocforwardprefix| redirects
% compilation to the main or a child file by means of a pattern.
% The prefix |#1| in the current filename is replaced by |#2|
% and the suffix of the current filename is kept
% (it is assumed that the filename does not contain the substring `|~~~|'
% which is used as a delimiter).
% Compilation is handed over to the new file by |\childdocforward|:
%    \begin{macrocode}
\newcommand{\childdocforwardprefix}[3][]
{
  \begingroup
    \def\childdocextract #2##1~~~{\def\childdoctmp{\childdocforward[#1]{#3##1}}}
    \expandafter\childdocextract\childdocname~~~
    \expandafter
  \endgroup
  \childdoctmp
}
%    \end{macrocode}

% \macro{\childdoc}
% The deprecated macro |\childdoc| is a legacy version of |\childdocmain|:
%    \begin{macrocode}
\newcommand{\childdoc}{\childdocmain}
%    \end{macrocode}

% \macro{\childdocredirect}
% The deprecated macro |\childdocredirect| is a legacy version
% of |\childdocforward| and |\childdocforwardprefix|:
%    \begin{macrocode}
\newcommand{\childdocredirect}[2][]
{
  \begingroup
    \if?#1?
      \def\childdoctmp{\childdocforward{#2}}
    \else
      \def\childdoctmp{\childdocforwardprefix{#1}{#2}}
    \fi
    \expandafter
  \endgroup
  \childdoctmp
}
%    \end{macrocode}

%\iffalse
%</package>
%\fi
%
\endinput
\childdocforward{cdocsch2}"|
% \end{tabular}
% \end{center}
% Note that the trailing backslash on each first line
% merely continues the input to the second line
% (for convenient cut ant paste).
% Furthermore, the command |latex| can be replaced by any
% of its alternative versions such as |pdflatex|.
%
% %%%%%%%%%%%%%%%%%%%%%%%%%%%%%%%%%%%%%%%%%%%%%%%%%%%%%%%%%%%%%%%%%%%%%%%%%%%%%%
% %%%%%%%%%%%%%%%%%%%%%%%%%%%%%%%%%%%%%%%%%%%%%%%%%%%%%%%%%%%%%%%%%%%%%%%%%%%%%%
% \section{Implementation}
%\iffalse
%<*package>
%\fi
%
% This section describes the definitions file |childdoc.def|.

% The definitions cannot be loaded using |\usepackage| or |\RequirePackage|
% which has a mechanism to prevent loading a style file more than once.
% When loading the definitions by means of |\input|
% multiple instances have to be prevented manually:
%\iffalse
%This code needs to be before the `\ProvidesFile' directive
%which is defined at the beginning of this file.
%Therefore it is also placed there and commented out here.
%</package>
%<*discard>
%\fi
%    \begin{macrocode}
\ifdefined\childdocmain\endinput\fi
%    \end{macrocode}
%\iffalse
%</discard>
%<*package>
%\fi
%
% \macro{\ifchilddoc}
% \macro{\ifchilddocmanual}
% The conditional |\ifchilddoc| tells whether a
% child (true) or main (false) document is being compiled.
% The conditional |\ifchilddocmanual| tells whether
% the |\includeonly| mechanism is used (false) or
% the selection of child files must be performed manually (true).
% The definitions initialise to false:
%    \begin{macrocode}
\newif\ifchilddoc
\newif\ifchilddocmanual
%    \end{macrocode}

% \macro{\childdocname}
% \macro{\childdocjob}
% The macro |\childdocname| stores the name of the main document
% to be compiled. The macro |\childdocjob| stores the name of
% the document on which the \LaTeX{} compiler was originally invoked.
% The content of |\jobname| cannot be compared
% to filenames specified in the source due to different catcodes.
% The following code rescans |\jobname|, stores the result
% in |\childdocname| and saves a copy in |\childdocjob|:
%    \begin{macrocode}
\edef\childdocname{\scantokens\expandafter{\jobname\noexpand}}
\let\childdocjob\childdocname
%    \end{macrocode}

% \macro{\childdocdisable}
% The macro |\childdocdisable| prevents the main file
% from being processed more than once.
% At this stage, the main document command |\childdocmain|
% is assumed to be called once again where it should do nothing.
% Any subsequent call to it should prevent
% a secondary processing of the main document
% It overwrites the forwarding commands
% |\childdocof| and |\childdocforward|
% with empty macros to prevent further inclusions of the main document:
%    \begin{macrocode}
\newcommand{\childdocdisable}
{
  \renewcommand{\childdocmain}[1]{\renewcommand{\childdocmain}[1]{\endinput}}
  \renewcommand{\childdocof}[1]{}
  \renewcommand{\childdocby}[2][]{}
  \renewcommand{\childdocforward}[2][]{}
  \renewcommand{\childdocdisable}{}
}
%    \end{macrocode}

% \macro{\childdocmain}
% The macro |\childdocmain| is to be called at the top of the main file
% with nothing or the main filename (without extension) as argument.
% First, it breaks loops.
% If the argument is not empty and does not match |\childdocname|
% (which is set by the first inclusion of |childdoc.def|),
% |\ifchilddoc| is set to true, |\includeonly| is applied to the child file
% and |\jobname| is set to the main file
% (for proper handling of |.aux| files):
%    \begin{macrocode}
\newcommand{\childdocmain}[1]
{
  \childdocdisable\childdocmain{}
  \if?#1?\else
    \begingroup
      \def\childdoctmp{#1}
      \ifx\childdoctmp\childdocname
        \def\childdoctmp{}
      \else
        \def\childdoctmp
        {
          \childdoctrue
          \includeonly{\childdocname}
          \def\childdocjob{#1}
          \def\jobname{#1}
        }
      \fi
      \expandafter
    \endgroup
    \childdoctmp
  \fi
}
%    \end{macrocode}

% \macro{\childdocof}
% The command |\childdocof| redirects
% compilation to the main file |#1|.
%    \begin{macrocode}
\newcommand{\childdocof}[1]
{
  \childdocdisable
  \childdoctrue
  \includeonly{\childdocname}
  \def\jobname{#1}
  \def\childdocjob{#1}
  \input{#1}
}
%    \end{macrocode}

% \macro{\childdocby}
% The command |\childdocby| ....
%    \begin{macrocode}
\newcommand{\childdocby}[2][]
{
  \childdocdisable
  \childdoctrue
  \childdocmanualtrue
  \if?#1?\else
    \def\jobname{#2}
  \fi
  \def\childdocjob{#2}
  \input{#2}
  \endinput
}
%    \end{macrocode}

% \macro{\childdocforward}
% The command |\childdocforward| redirects
% compilation to the main file or
% (if the optional argument is given) a child file.
% Parameters are set as if the main file
% or a child file starting with |\childdocof| was compiled.
% Then compilation is handed over to the main file:
%    \begin{macrocode}
\newcommand{\childdocforward}[2][]
{
  \begingroup
    \if?#1?
      \def\childdoctmp
      {
        \def\childdocname{#2}
        \def\childdocjob{#2}
        \def\jobname{#2}
        \input{#2}
        \endinput
      }
    \else
      \def\childdoctmp
      {
        \childdocdisable
        \def\childdocname{#2}
        \childdoctrue
        \includeonly{#2}
        \def\childdocjob{#1}
        \def\jobname{#1}
        \input{#1}
        \endinput
      }
    \fi
    \expandafter
  \endgroup
  \childdoctmp
}
%    \end{macrocode}

% \macro{\childdocforwardprefix}
% The command |\childdocforwardprefix| redirects
% compilation to the main or a child file by means of a pattern.
% The prefix |#1| in the current filename is replaced by |#2|
% and the suffix of the current filename is kept
% (it is assumed that the filename does not contain the substring `|~~~|'
% which is used as a delimiter).
% Compilation is handed over to the new file by |\childdocforward|:
%    \begin{macrocode}
\newcommand{\childdocforwardprefix}[3][]
{
  \begingroup
    \def\childdocextract #2##1~~~{\def\childdoctmp{\childdocforward[#1]{#3##1}}}
    \expandafter\childdocextract\childdocname~~~
    \expandafter
  \endgroup
  \childdoctmp
}
%    \end{macrocode}

% \macro{\childdoc}
% The deprecated macro |\childdoc| is a legacy version of |\childdocmain|:
%    \begin{macrocode}
\newcommand{\childdoc}{\childdocmain}
%    \end{macrocode}

% \macro{\childdocredirect}
% The deprecated macro |\childdocredirect| is a legacy version
% of |\childdocforward| and |\childdocforwardprefix|:
%    \begin{macrocode}
\newcommand{\childdocredirect}[2][]
{
  \begingroup
    \if?#1?
      \def\childdoctmp{\childdocforward{#2}}
    \else
      \def\childdoctmp{\childdocforwardprefix{#1}{#2}}
    \fi
    \expandafter
  \endgroup
  \childdoctmp
}
%    \end{macrocode}

%\iffalse
%</package>
%\fi
%
\endinput

\childdocof{cdocsamp}
%    \end{macrocode}

%\iffalse
%</samplechap1|samplechap2>
%\fi
%
%\iffalse
%<*samplechap1>
%\fi
% Some text for chapter 1:
%    \begin{macrocode}
\section{one}
some text in chapter one
%    \end{macrocode}

%\iffalse
%</samplechap1>
%\fi
% Some text for chapter 2:
%\iffalse
%<*samplechap2>
%\fi
%    \begin{macrocode}
\section{two}
more text in chapter two
%    \end{macrocode}

%\iffalse
%</samplechap2>
%\fi
%
% %%%%%%%%%%%%%%%%%%%%%%%%%%%%%%%%%%%%%%
% \paragraph{Part Include Files.}
%
% The include files are called |cdocspt3.tex| and |cdocspt4.tex|.
%
%\iffalse
%<*samplepart3|samplepart4>
%\fi

% Optional override for |\version| flag:
%    \begin{macrocode}
%%\providecommand{\version}{final}
%    \end{macrocode}

% Include the main document:
%    \begin{macrocode}
% \iffalse
%
% childdoc.dtx Copyright (C) 2017-2018 Niklas Beisert
%
% This work may be distributed and/or modified under the
% conditions of the LaTeX Project Public License, either version 1.3
% of this license or (at your option) any later version.
% The latest version of this license is in
%   http://www.latex-project.org/lppl.txt
% and version 1.3 or later is part of all distributions of LaTeX
% version 2005/12/01 or later.
%
% This work has the LPPL maintenance status `maintained'.
%
% The Current Maintainer of this work is Niklas Beisert.
%
% This work consists of the files childdoc.dtx and childdoc.ins
% and the derived files childdoc.def and cdocsamp.tex with
% cdocsch1.tex, cdocsch2.tex, cdocsdrf.tex, cdocsfn1.tex, cdocsfn2.tex.
%
%<package>\ifdefined\childdocmain\endinput\fi
%<package>\ProvidesFile{childdoc.def}[2018/12/30 v2.0 child document driver]
%<samplemain>\ProvidesFile{cdocsamp.tex}[2018/12/30 v2.0 sample for childdoc]
%<*driver>
%\ProvidesFile{childdoc.drv}[2018/12/30 v2.0 childdoc reference manual file]
\PassOptionsToClass{10pt,a4paper}{article}
\documentclass{ltxdoc}

\usepackage[margin=35mm]{geometry}
\usepackage{hyperref}
\usepackage{hyperxmp}
\usepackage[usenames]{color}

\hypersetup{colorlinks=true}
\hypersetup{pdfstartview=FitH}
\hypersetup{pdfpagemode=UseNone}
\hypersetup{pdfsource={}}
\hypersetup{pdflang={en-UK}}
\hypersetup{pdfcopyright={Copyright 2017-2018 Niklas Beisert.
  This work may be distributed and/or modified under the
  conditions of the LaTeX Project Public License, either version 1.3
  of this license or (at your option) any later version.}}
\hypersetup{pdflicenseurl={http://www.latex-project.org/lppl.txt}}
\hypersetup{pdfcontactaddress={ETH Zurich, ITP, HIT K,
  Wolfgang-Pauli-Strasse 27}}
\hypersetup{pdfcontactpostcode={8093}}
\hypersetup{pdfcontactcity={Zurich}}
\hypersetup{pdfcontactcountry={Switzerland}}
\hypersetup{pdfcontactemail={nbeisert@itp.phys.ethz.ch}}
\hypersetup{pdfcontacturl={http://people.phys.ethz.ch/\xmptilde nbeisert/}}

\newcommand{\secref}[1]{\hyperref[#1]{section \ref*{#1}}}

\parskip1ex
\parindent0pt
\let\olditemize\itemize
\def\itemize{\olditemize\parskip0pt}

\begin{document}

\title{The \textsf{childdoc} Package}
\hypersetup{pdftitle={The childdoc Package}}
\author{Niklas Beisert\\[2ex]
  Institut f\"ur Theoretische Physik\\
  Eidgen\"ossische Technische Hochschule Z\"urich\\
  Wolfgang-Pauli-Strasse 27, 8093 Z\"urich, Switzerland\\[1ex]
  \href{mailto:nbeisert@itp.phys.ethz.ch}
  {\texttt{nbeisert@itp.phys.ethz.ch}}}
\hypersetup{pdfauthor={Niklas Beisert}}
\hypersetup{pdfsubject={Manual for the LaTeX2e Package childdoc}}
\date{30 December 2018, \textsf{v2.0}}
\maketitle

\begin{abstract}\noindent
\textsf{childdoc} is a \LaTeXe{} package
that enables the direct compilation
of document sections included by |\include|
to individual files.
\end{abstract}

\begingroup
\parskip0ex
\tableofcontents
\endgroup

%%%%%%%%%%%%%%%%%%%%%%%%%%%%%%%%%%%%%%%%%%%%%%%%%%%%%%%%%%%%%%%%%%%%%%%%%%%%%%%%
%%%%%%%%%%%%%%%%%%%%%%%%%%%%%%%%%%%%%%%%%%%%%%%%%%%%%%%%%%%%%%%%%%%%%%%%%%%%%%%%
\section{Introduction}

\LaTeX{} provides a mechanism to structure a large document (such as a book)
into a main file and several child files (containing the chapters)
using the |\include| command.
This mechanism is beneficial for documents
which span hundreds of pages in order to
make the source file(s) more manageable.
Moreover, compilation can be restricted to
selected child files by means of the |\includeonly| command.
The latter feature can be used to reduce the compilation time while editing
(this was significantly more useful in the earlier days of \LaTeX{})
or to generate a smaller document which is easier to navigate.
Another application of |\includeonly| is to generate
documents consisting of selected parts of the complete document.

However, there are a few drawbacks of the plain |\include| mechanism:
\begin{itemize}
\item
The child files cannot be compiled on their own,
they can only be compiled via the main file.
A naive editing environment
(such as a text editor with an option
to have the current file processed by \LaTeX)
may require one to switch to the main file before compiling;
attempting to compile the child file produces errors.
\item
The main file must be modified (each time)
to adjust the |\includeonly| command
to the present needs. This easily leaves the main file in a messy state.
\item
The generated document will always carry the filename
of the main document. This is inconvenient if
several child files are to be compiled and
to be kept for distribution.
\end{itemize}

The present package provides a simple interface
to make child files individually compilable by \LaTeX{}.
Compiling a child file then has the same effect as compiling
the main file with an |\includeonly| command
to select the appropriate child.
Moreover the generated document will carry the name of the child
rather than the main file.
This resolves all three above issues.

This feature is meant to make the editing of books,
thesis documents and lecture notes somewhat more convenient.
However, the package can also be used efficiently for
composing a series of documents (such as exercise sheets)
which are typically distributed individually.
It then assists the author in generating the individual documents
(potentially in different versions)
as well as a document containing the collected series.
Another application is in developing style files
or other kinds of included material
where compilation of the style file could redirect
to a sample or test file.

%%%%%%%%%%%%%%%%%%%%%%%%%%%%%%%%%%%%%%%%%%%%%%%%%%%%%%%%%%%%%%%%%%%%%%%%%%%%%%%%
%%%%%%%%%%%%%%%%%%%%%%%%%%%%%%%%%%%%%%%%%%%%%%%%%%%%%%%%%%%%%%%%%%%%%%%%%%%%%%%%
\section{Usage}

First of all, the package \textsf{childdoc} is \emph{not} a standard
\LaTeXe{} |.sty| style file! Therefore it needs to be invoked in
a non-standard way.

%%%%%%%%%%%%%%%%%%%%%%%%%%%%%%%%%%%%%%%%%%%%%%%%%%%%%%%%%%%%%%%%%%%%%%%%%%%%%%%%
\subsection{Included Files}
\label{sec:include}

%%%%%%%%%%%%%%%%%%%%%%%%%%%%%%%%%%%%%%%%
\DescribeMacro{\childdocmain}
To use the package, add the commands
\begin{center}
\begin{tabular}{l}
|% \iffalse
%
% childdoc.dtx Copyright (C) 2017-2018 Niklas Beisert
%
% This work may be distributed and/or modified under the
% conditions of the LaTeX Project Public License, either version 1.3
% of this license or (at your option) any later version.
% The latest version of this license is in
%   http://www.latex-project.org/lppl.txt
% and version 1.3 or later is part of all distributions of LaTeX
% version 2005/12/01 or later.
%
% This work has the LPPL maintenance status `maintained'.
%
% The Current Maintainer of this work is Niklas Beisert.
%
% This work consists of the files childdoc.dtx and childdoc.ins
% and the derived files childdoc.def and cdocsamp.tex with
% cdocsch1.tex, cdocsch2.tex, cdocsdrf.tex, cdocsfn1.tex, cdocsfn2.tex.
%
%<package>\ifdefined\childdocmain\endinput\fi
%<package>\ProvidesFile{childdoc.def}[2018/12/30 v2.0 child document driver]
%<samplemain>\ProvidesFile{cdocsamp.tex}[2018/12/30 v2.0 sample for childdoc]
%<*driver>
%\ProvidesFile{childdoc.drv}[2018/12/30 v2.0 childdoc reference manual file]
\PassOptionsToClass{10pt,a4paper}{article}
\documentclass{ltxdoc}

\usepackage[margin=35mm]{geometry}
\usepackage{hyperref}
\usepackage{hyperxmp}
\usepackage[usenames]{color}

\hypersetup{colorlinks=true}
\hypersetup{pdfstartview=FitH}
\hypersetup{pdfpagemode=UseNone}
\hypersetup{pdfsource={}}
\hypersetup{pdflang={en-UK}}
\hypersetup{pdfcopyright={Copyright 2017-2018 Niklas Beisert.
  This work may be distributed and/or modified under the
  conditions of the LaTeX Project Public License, either version 1.3
  of this license or (at your option) any later version.}}
\hypersetup{pdflicenseurl={http://www.latex-project.org/lppl.txt}}
\hypersetup{pdfcontactaddress={ETH Zurich, ITP, HIT K,
  Wolfgang-Pauli-Strasse 27}}
\hypersetup{pdfcontactpostcode={8093}}
\hypersetup{pdfcontactcity={Zurich}}
\hypersetup{pdfcontactcountry={Switzerland}}
\hypersetup{pdfcontactemail={nbeisert@itp.phys.ethz.ch}}
\hypersetup{pdfcontacturl={http://people.phys.ethz.ch/\xmptilde nbeisert/}}

\newcommand{\secref}[1]{\hyperref[#1]{section \ref*{#1}}}

\parskip1ex
\parindent0pt
\let\olditemize\itemize
\def\itemize{\olditemize\parskip0pt}

\begin{document}

\title{The \textsf{childdoc} Package}
\hypersetup{pdftitle={The childdoc Package}}
\author{Niklas Beisert\\[2ex]
  Institut f\"ur Theoretische Physik\\
  Eidgen\"ossische Technische Hochschule Z\"urich\\
  Wolfgang-Pauli-Strasse 27, 8093 Z\"urich, Switzerland\\[1ex]
  \href{mailto:nbeisert@itp.phys.ethz.ch}
  {\texttt{nbeisert@itp.phys.ethz.ch}}}
\hypersetup{pdfauthor={Niklas Beisert}}
\hypersetup{pdfsubject={Manual for the LaTeX2e Package childdoc}}
\date{30 December 2018, \textsf{v2.0}}
\maketitle

\begin{abstract}\noindent
\textsf{childdoc} is a \LaTeXe{} package
that enables the direct compilation
of document sections included by |\include|
to individual files.
\end{abstract}

\begingroup
\parskip0ex
\tableofcontents
\endgroup

%%%%%%%%%%%%%%%%%%%%%%%%%%%%%%%%%%%%%%%%%%%%%%%%%%%%%%%%%%%%%%%%%%%%%%%%%%%%%%%%
%%%%%%%%%%%%%%%%%%%%%%%%%%%%%%%%%%%%%%%%%%%%%%%%%%%%%%%%%%%%%%%%%%%%%%%%%%%%%%%%
\section{Introduction}

\LaTeX{} provides a mechanism to structure a large document (such as a book)
into a main file and several child files (containing the chapters)
using the |\include| command.
This mechanism is beneficial for documents
which span hundreds of pages in order to
make the source file(s) more manageable.
Moreover, compilation can be restricted to
selected child files by means of the |\includeonly| command.
The latter feature can be used to reduce the compilation time while editing
(this was significantly more useful in the earlier days of \LaTeX{})
or to generate a smaller document which is easier to navigate.
Another application of |\includeonly| is to generate
documents consisting of selected parts of the complete document.

However, there are a few drawbacks of the plain |\include| mechanism:
\begin{itemize}
\item
The child files cannot be compiled on their own,
they can only be compiled via the main file.
A naive editing environment
(such as a text editor with an option
to have the current file processed by \LaTeX)
may require one to switch to the main file before compiling;
attempting to compile the child file produces errors.
\item
The main file must be modified (each time)
to adjust the |\includeonly| command
to the present needs. This easily leaves the main file in a messy state.
\item
The generated document will always carry the filename
of the main document. This is inconvenient if
several child files are to be compiled and
to be kept for distribution.
\end{itemize}

The present package provides a simple interface
to make child files individually compilable by \LaTeX{}.
Compiling a child file then has the same effect as compiling
the main file with an |\includeonly| command
to select the appropriate child.
Moreover the generated document will carry the name of the child
rather than the main file.
This resolves all three above issues.

This feature is meant to make the editing of books,
thesis documents and lecture notes somewhat more convenient.
However, the package can also be used efficiently for
composing a series of documents (such as exercise sheets)
which are typically distributed individually.
It then assists the author in generating the individual documents
(potentially in different versions)
as well as a document containing the collected series.
Another application is in developing style files
or other kinds of included material
where compilation of the style file could redirect
to a sample or test file.

%%%%%%%%%%%%%%%%%%%%%%%%%%%%%%%%%%%%%%%%%%%%%%%%%%%%%%%%%%%%%%%%%%%%%%%%%%%%%%%%
%%%%%%%%%%%%%%%%%%%%%%%%%%%%%%%%%%%%%%%%%%%%%%%%%%%%%%%%%%%%%%%%%%%%%%%%%%%%%%%%
\section{Usage}

First of all, the package \textsf{childdoc} is \emph{not} a standard
\LaTeXe{} |.sty| style file! Therefore it needs to be invoked in
a non-standard way.

%%%%%%%%%%%%%%%%%%%%%%%%%%%%%%%%%%%%%%%%%%%%%%%%%%%%%%%%%%%%%%%%%%%%%%%%%%%%%%%%
\subsection{Included Files}
\label{sec:include}

%%%%%%%%%%%%%%%%%%%%%%%%%%%%%%%%%%%%%%%%
\DescribeMacro{\childdocmain}
To use the package, add the commands
\begin{center}
\begin{tabular}{l}
|\input{childdoc.def}|\\
|\childdocmain{}|\\
\end{tabular}
\end{center}
at the very top of the main \LaTeX{} file,
in particular \emph{before} the |\documentclass| statement!
The argument of |\childdocmain| should be left empty
(but it must be present).

%%%%%%%%%%%%%%%%%%%%%%%%%%%%%%%%%%%%%%%%
\DescribeMacro{\childdocof}
Furthermore, add the commands
\begin{center}
\begin{tabular}{l}
|\input{childdoc.def}|\\
|\childdocof{|\textit{main}|}|\\
\end{tabular}
\end{center}
at the top of every child file \textit{child}
which is included by |\include{|\textit{child}|}|
from within the main file
(or at least for those files to be compiled individually).
The argument \textit{main} must be the filename of the main file.

There are a couple of
considerations in setting up the main and child documents:

%%%%%%%%%%%%%%%%%%%%%%%%%%%%%%%%%%%%%%%%
\paragraph{Restrictions.}

Please note the following restrictions:
\begin{itemize}
\item
|\childdocmain| must be called with one argument \textit{main}
to ensure compatibility with earlier version of the package.
It must either be empty (|\childdocmain{}|)
or precisely match the filename of the main file in which it is specified.
See \secref{sec:detection} for further information.
\item
The filename \textit{main} must be specified without the |.tex| extension.
\item
The filename \textit{main} is case sensitive
(even in case-insensitive file systems)
due to internal string comparison.
\item
The argument \textit{main} should be fully expanded, it cannot be a macro.
\item
Subdirectories and special characters should be avoided in filenames.
\item
The command |\childdocmain{|\textit{main}|}| must be followed by a whitespace.
It should not be followed immediately by another command
or by a comment mark `|%|'.
This is because the \TeX{} parser reads the token immediately following
the argument of |\childdocmain| and puts it
at the beginning of every child section;
however, a white\-space is ignored.
\end{itemize}

%%%%%%%%%%%%%%%%%%%%%%%%%%%%%%%%%%%%%%%%
\paragraph{Content of Main File.}

It is advisable to place all content in the child files included by |\include|.
Any output contained in the main file will appear in all child documents
unless suppressed manually;
it cannot be suppressed automatically by the |\includeonly| directive
and thus should normally be avoided.
A method to include some content in the main file
by means of conditional processing is described in \secref{sec:conditional}.

%%%%%%%%%%%%%%%%%%%%%%%%%%%%%%%%%%%%%%%%
\paragraph{Page Numbering.}

When only a part of the document is compiled,
the appropriate numbering of pages
(as well as other status parameters)
is determined from the |.aux| files.
The latter contain information from previous passes.
However this information needs to propagate through
all intermediate child documents.
Therefore the page numbering in child documents may well
be inconsistent until the complete document is compiled at least once.

A useful (if unconventional) way to always ensure a consistent
page numbering is to restart the numbering in each child document
and denote the pages by `\textit{child}|.|\textit{page}'
where \textit{child} represents the chapter/section number of the child file.
This can be achieved by the command
|\numberwithin{page}{|\textit{child}|}|
of the \textsf{amsmath} package
where \textit{child} can be |chapter| or |section|
depending on the chosen structuring.
Alternatively, one can modify the macro |\thepage| appropriately
and reset the counter |page| at the start of each child file.

%%%%%%%%%%%%%%%%%%%%%%%%%%%%%%%%%%%%%%%%%%%%%%%%%%%%%%%%%%%%%%%%%%%%%%%%%%%%%%%%
\subsection{Conditional Processing}
\label{sec:conditional}

The package provides a mechanism to compile different versions
of a document. To customise the versions further some conditional processing
can come in handy to distinguish which version is being compiled.
The package provides two macros to describe the compilation context:

%%%%%%%%%%%%%%%%%%%%%%%%%%%%%%%%%%%%%%%%
\DescribeMacro{\ifchilddoc}
The conditional |\ifchilddoc| distinguishes between the compilation of
child documents and the main document:
%
\begin{center}
|\ifchilddoc |\textit{child-code}| |[|\||else |\textit{main-code}]| \||fi|
\end{center}

%%%%%%%%%%%%%%%%%%%%%%%%%%%%%%%%%%%%%%%%
\DescribeMacro{\childdocname}
\DescribeMacro{\childdocjob}
The macro |\childdocname| contains the filename (without extension)
of the main or child file being processed.
Note that |\childdocjob| will always contain the name of the main file.

%%%%%%%%%%%%%%%%%%%%%%%%%%%%%%%%%%%%%%%%
\paragraph{Title Page.}

Conditional processing can be used to include a title or banner page
in the main document when proper precautions are taken.
Importantly, the code in the main file should ensure that the page counter
(as well as other status parameters which are stored in the |.aux| files)
takes the same value after the conditional processing.
Otherwise the page numbers may take divergent values
depending on which part is compiled.

For example, a title page could be declared by:
%
\begin{center}
\begin{tabular}{l}
|\ifchilddoc\||else|\\
|\addtocounter{page}{-1}|\\
\textit{code for title page}\\
|\newpage|\\
|\||fi|
\end{tabular}
\end{center}
%
A banner page for the child documents can be generated by:
%
\begin{center}
\begin{tabular}{l}
|\ifchilddoc|\\
|\addtocounter{page}{-1}|\\
\textit{code for banner page}\\
|\newpage|\\
|\||fi|
\end{tabular}
\end{center}
%
Here one could write a message such as:
\begin{center}
|This is the part \childdocname{} of \childdocjob{}.|
\end{center}

%%%%%%%%%%%%%%%%%%%%%%%%%%%%%%%%%%%%%%%%%%%%%%%%%%%%%%%%%%%%%%%%%%%%%%%%%%%%%%%%
\subsection{Flags}
\label{sec:flags}

The package makes it easy to generate different versions
of the main or child documents.
To this end compilation flags can be defined
and assigned different default values.
They will be particularly useful in conjunction
with the forwarding mechanism described in \secref{sec:forward}.

For example, it may be useful to have a flag |\version|
which can be set to |draft| or |final|.
The document source will contain some conditional code
depending on the value of |\version|.
Suppose further, the flag should default to |final| for the main file
and to |draft| for child files
which is a natural assignment for editing the document.
This is achieved by placing the following code
in the preamble of the main document
(below the |\childdocmain| directive):
%
\begin{center}
\begin{tabular}{l}
|\ifchilddoc|\\
|\providecommand{\version}{draft}|\\
|\||else|\\
|\providecommand{\version}{final}|\\
|\||fi|
\end{tabular}
\end{center}
%
The definition by |\providecommand| makes sure
that previous definitions are not overwritten.
Further statements |\providecommand{\version}{...}|
can thus be added before the above code to override it.

For the main file, one might add a line
(between |\childdocmain| and the above block)
%
\begin{center}
|%\ifchilddoc\||else\providecommand{\version}{draft}\||fi|
\end{center}
%
which can be uncommented to produce a draft version.
Likewise one can add a line to the very top of a child file
(above the |\childdocof{|\textit{main}|}| directive)
%
\begin{center}
|%\providecommand{\version}{final}|
\end{center}
%
which can be uncommented to produce the final version of this child document.

%%%%%%%%%%%%%%%%%%%%%%%%%%%%%%%%%%%%%%%%%%%%%%%%%%%%%%%%%%%%%%%%%%%%%%%%%%%%%%%%
\subsection{Forwarding}
\label{sec:forward}

Different versions of the main or child documents
using compilation flags as described in \secref{sec:flags}
can be (permanently) stored in different files
for convenient compilation, viewing and distribution.
To this end, the package defines a command
to pass on compilation to a different file:

%%%%%%%%%%%%%%%%%%%%%%%%%%%%%%%%%%%%%%%%
\DescribeMacro{\childdocforward}
The command |\childdocforward| redirects processing to
another source file:
%
\begin{center}
\begin{tabular}{l}
|\input{childdoc.def}|\\
|\childdocforward[|\textit{main}|]{|\textit{dest}|}|\\
\end{tabular}
\end{center}
%
The argument \textit{dest} is the destination file
(without extension).
It should be the main file or one of the child files.
Note that further \textsf{childdoc} directives
such as |\childdocof| and |\childdocforward|
in the indicated file will be processed in this form.
The optional argument \textit{main}
passes on directly to the main file \textit{main}
while pretending to compile the child \textit{dest}.
This form behaves as if \textit{dest}
issues |\childdocof{|\textit{main}|}| right away,
and no further \textsf{childdoc} directives will be processed.

%%%%%%%%%%%%%%%%%%%%%%%%%%%%%%%%%%%%%%%%
\DescribeMacro{\...prefix}
In the alternative form |\childdocforwardprefix|,
%
\begin{center}
\begin{tabular}{l}
|\input{childdoc.def}|\\
|\childdocforwardprefix[|\textit{main}|]{|\textit{prefix}|}{|\textit{dest}|}|
\end{tabular}
\end{center}
%
the destination file is determined by a pattern
depending on the current file:
To make this work, the current file must be called
`{\textit{prefix}\hspace{0.2em}\textit{suffix}}'
with \textit{prefix} matching precisely the argument.
Processing is then passed on to the file
`{\textit{dest}\hspace{0.2em}\textit{suffix}}'.
Surely, the same effect is achieved by
directly specifying the
argument `{\textit{dest}\hspace{0.2em}\textit{suffix}}'
in the first form.
However, that requires to set up a different file
for each child. With the alternative form of the command
all these files can have exactly the same content
which simplifies setting them up and maintaining them.

For example, the following file |draft.tex|
with a compilation flag |\version| as described in \secref{sec:flags}
compiles the main document as a draft:
%
\begin{center}
\begin{tabular}{l}
|\def\version{draft}|\\
|\input{childdoc.def}|\\
|\childdocforward{|\textit{main}|}|
\end{tabular}
\end{center}
%
Likewise, the following files |final|\textit{nn}|.tex|
compile the final version of the child document
|child|\textit{nn}|.tex|:
%
\begin{center}
\begin{tabular}{l}
|\def\version{final}|\\
|\input{childdoc.def}|\\
|\childdocforwardprefix{final}{child}|
\end{tabular}
\end{center}
%

Note that when several versions of a main file and/or of each child file
are to be generated, it may be convenient to set up a |Makefile| or
shell script to automatise the process.

%%%%%%%%%%%%%%%%%%%%%%%%%%%%%%%%%%%%%%%%%%%%%%%%%%%%%%%%%%%%%%%%%%%%%%%%%%%%%%%%
\subsection{Command Line Processing}
\label{sec:commandline}

The effect of redirection files can also be achieved by invoking
the \LaTeX{} compiler with a more elaborate command line.
Most conveniently this should be done as part
of a shell script or a |Makefile|.

When using \textsf{childdoc} in the main file, the following
command lines effectively perform a redirection
(note that depending on the shell being used,
backslashes may have to be doubled: `|\|' $\to$ `|\\|'):
%
\begin{center}
|... -jobname "|\textit{target}|" |\\|"|[\textit{flags}]%
|\input{childdoc.def}\childdocforward[|\textit{main}|]{|\textit{dest}|}"|
\end{center}
%
Here \textit{target} is the name of the output file,
\textit{main} is the name of the main file
and \textit{dest} is the name of the main or child file to be processed
(all filenames without extensions).
The optional argument \textit{main} can be omitted
if \textit{main} matches \textit{dest}.
Optionally, compilation \textit{flags} can be defined via |\def| commands.
This command line makes the \TeX{} engine believe
it is compiling the file \textit{target}
whose content is specified as the latter parameter.
The provided code then forwards the processing to
\textit{main} or \textit{dest} as described in \secref{sec:forward}.

%%%%%%%%%%%%%%%%%%%%%%%%%%%%%%%%%%%%%%%%%%%%%%%%%%%%%%%%%%%%%%%%%%%%%%%%%%%%%%%%
\subsection{Include by Input}
\label{sec:input}

Including child documents by |\include| has some restrictions by design.
Most notably, the content of a child document always occupies
its own set of pages; pages cannot be shared between child documents.
Usually, this behaviour makes perfect sense
because each child document contain an essential part of the document.
However, in some situations it may be desirable to compose
a document from a collection of parts
without having mandatory page breaks between then.
For this case, the package
provides a mechanism to include parts
by |\input| which can also be processed individually.
However, by construction this mechanism
requires manual handling of the content to be output.

%%%%%%%%%%%%%%%%%%%%%%%%%%%%%%%%%%%%%%%%
\DescribeMacro{\ifchilddocmanual}
The main file should be prepared as usual, see \secref{sec:include}.
However, the document body must make a distinction
between processing of an individual part and of the main document, e.g.:
%
\begin{center}
\begin{tabular}{l}
|\ifchilddocmanual|\\
|\input{\childdocname}|\\
|\||else|\\
\textit{document body with }|\input{|\textit{part}|}|\\
|\||fi|
\end{tabular}
\end{center}
%
The conditional |\ifchilddocmanual| is true whenever
a part to be included by |\input| is being compiled,
and the name of the part is stored in |\childdocname|.

%%%%%%%%%%%%%%%%%%%%%%%%%%%%%%%%%%%%%%%%
\DescribeMacro{\childdocby}
Each part to be included by |\input| should start with:
%
\begin{center}
\begin{tabular}{l}
|\input{childdoc.def}|\\
|\childdocby{|\textit{main}|}|\\
\end{tabular}
\end{center}
%
The directive |\childdocby| is similar to |\childdocof|
described in \secref{sec:include},
but the subsequent selection of content must be done manually.
To that end, both |\ifchilddoc| and |\ifchilddocmanual|
will be true upon processing of a part,
and the name of the part is stored in |\childdocname|.
Note that |\jobname| will be set to the filename of the current part
so that each part receives an individual |.aux| file
that does not interfere with the |.aux| file(s) of the main document.
This behaviour can be altered by the alternative form
|\childdocby[*]{|\textit{main}|}| (with a non-empty optional argument)
which uses the |.aux| file of the main document
by setting |\jobname| to \textit{main}.

%%%%%%%%%%%%%%%%%%%%%%%%%%%%%%%%%%%%%%%%%%%%%%%%%%%%%%%%%%%%%%%%%%%%%%%%%%%%%%%%
\subsection{Driver Development}
\label{sec:driver}

The \textsf{childdoc} mechanism can also be use for the development
of definition files such as \LaTeX{} styles or classes.
This case differs from the above setup with multiple parts
included by |\include| in that no |\includeonly| should be invoked.
This can be achieved by starting the include file
(before |\ProvidesPackage|) with:
%
\begin{center}
\begin{tabular}{l}
|\input{childdoc.def}|\\
|\childdocforward{|\textit{main}|}|\\
\end{tabular}
\end{center}
%
or alternatively with:
%
\begin{center}
\begin{tabular}{l}
|\input{childdoc.def}|\\
|\childdocby{|\textit{main}|}|\\
\end{tabular}
\end{center}
%
Both forms have slightly different effects as described above.
The main file is prepared as usual, see \secref{sec:include}.

%%%%%%%%%%%%%%%%%%%%%%%%%%%%%%%%%%%%%%%%%%%%%%%%%%%%%%%%%%%%%%%%%%%%%%%%%%%%%%%%
\subsection{Legacy Detection}
\label{sec:detection}

The directive |\childdocmain| in the main file can detect
whether the complete document or merely a child is to be compiled
even without using the directive |\childdocof|.
This method is deprecated because it is less robust
and there is no compelling reason to use it;
it is merely provided for backward compatibility
and it may be removed in future versions.

If the detection mechanism is to be used,
it is mandatory to correctly specify
the filename of the main file as the argument of |\childdocmain|:
%
\begin{center}
\begin{tabular}{l}
|\input{childdoc.def}|\\
|\childdocmain{|\textit{main}|}|\\
\end{tabular}
\end{center}
%
If |\jobname| does not match the argument \textit{main} of |\childdocmain|,
it is assumed that |\jobname| points to the child file to be compiled.
When using |\childdocmain| with the main file specified as argument,
it suffices to start a child file
with just |\input{|\textit{main}|}|
without loading of the package and using |\childdocof|.
If instead all processing is done
with the appropriate \textsf{childdoc} directives,
the argument of \textit{main} of |\childdocmain| can be empty.

An alternative version of the command line processing described
in \secref{sec:commandline} using the detection mechanism reads:
%
\begin{center}
|... -jobname "|\textit{target}|" "|[\textit{flags}]%
[|\def\jobname{|\textit{dest}|}|]|\input{|\textit{main}|}"|
\end{center}

%%%%%%%%%%%%%%%%%%%%%%%%%%%%%%%%%%%%%%%%%%%%%%%%%%%%%%%%%%%%%%%%%%%%%%%%%%%%%%%%
\subsection{Manual Code}
\label{sec:manual}

In case one cannot be certain whether the definitions file |childdoc.def|
is installed on the target \TeX{} distribution
and one prefers not to ship it,
it is conceivable to paste a few relevant commands into the sources.

To that end, drop all statements |\input{childdoc.def}|
and perform the replacements as outlined below.
Instead of |\childdocmain{|\textit{main}|}| add the following code
to the top of the main file:
%
\begin{center}
\begin{tabular}{l}
|\||ifdefined\childdocname\endinput\||fi\newif\ifchilddoc|\\
|\edef\childdocname{\scantokens\expandafter{\jobname\noexpand}}|\\
|\def\childdocmain{|\textit{main}|}\||ifx\childdocmain\childdocname\||else|\\
|\childdoctrue\includeonly{\childdocname}\let\jobname\childdocmain\||fi|\\
\end{tabular}
\end{center}
%
Instead of |\childdocof{|\textit{main}|}| just include the main file
at the top of each child file:
%
\begin{center}
|\input{|\textit{main}|}|
\end{center}
%
A simple redirection |\childdocforward{|\textit{dest}|}| is achieved by:
%
\begin{center}
|\def\jobname{|\textit{dest}|}\input{\jobname}|
\end{center}
%
The redirection with prefix
|\childdocforwardprefix[|\textit{prefix}|]{|\textit{dest}|}|
is accomplished by:
%
\begin{center}
\begin{tabular}{l}
|{\edef\jobname{\scantokens\expandafter{\jobname\noexpand}}|\\
|\def\redirectjob |\textit{prefix}|#1~~~{\gdef\jobname{|\textit{dest}|#1}}|\\
|\expandafter\redirectjob\jobname~~~}\input{\jobname}|
\end{tabular}
\end{center}

In an alternative approach,
child documents can be compiled by a specific command line
without additional code or specific definitions:
%
\begin{center}
|... -jobname "|\textit{target}|" "|[\textit{flags}]%
|\includeonly{|\textit{dest}|}\input{|\textit{main}|}"|
\end{center}
%

%%%%%%%%%%%%%%%%%%%%%%%%%%%%%%%%%%%%%%%%%%%%%%%%%%%%%%%%%%%%%%%%%%%%%%%%%%%%%%%%
%%%%%%%%%%%%%%%%%%%%%%%%%%%%%%%%%%%%%%%%%%%%%%%%%%%%%%%%%%%%%%%%%%%%%%%%%%%%%%%%
\section{Information}

%%%%%%%%%%%%%%%%%%%%%%%%%%%%%%%%%%%%%%%%%%%%%%%%%%%%%%%%%%%%%%%%%%%%%%%%%%%%%%%%
\subsection{Copyright}

Copyright \copyright{} 2017--2018 Niklas Beisert

This work may be distributed and/or modified under the
conditions of the \LaTeX{} Project Public License, either version 1.3
of this license or (at your option) any later version.
The latest version of this license is in
  \url{http://www.latex-project.org/lppl.txt}
and version 1.3 or later is part of all distributions of \LaTeX{}
version 2005/12/01 or later.

This work has the LPPL maintenance status `maintained'.

The Current Maintainer of this work is Niklas Beisert.

This work consists of the files |README.txt|, |childdoc.ins| and |childdoc.dtx|
as well as the derived files |childdoc.def|, |cdocsamp.tex|
with |cdocsch1.tex|, |cdocsch2.tex|, |cdocspt3.tex|, |cdocspt4.tex|,
|cdocsdrf.tex|, |cdocsfn1.tex|, |cdocsfn2.tex|
as well as |childdoc.pdf|.

%%%%%%%%%%%%%%%%%%%%%%%%%%%%%%%%%%%%%%%%%%%%%%%%%%%%%%%%%%%%%%%%%%%%%%%%%%%%%%%%
\subsection{Files and Installation}

The package consists of the files:
%
\begin{center}
\begin{tabular}{ll}
    |README.txt|   & readme file \\
    |childdoc.ins| & installation file \\
    |childdoc.dtx| & source file \\
    |childdoc.def| & definition file \\
    |cdocsamp.tex| & sample main file \\
    |cdocsch1.tex| & sample include file \\
    |cdocsch2.tex| & sample include file \\
    |cdocspt3.tex| & sample part file \\
    |cdocspt4.tex| & sample part file \\
    |cdocsdrf.tex| & sample redirection file \\
    |cdocsfn1.tex| & sample redirection file \\
    |cdocsfn2.tex| & sample redirection file \\
    |childdoc.pdf| & manual
\end{tabular}
\end{center}
%
The distribution consists of the files
|README.txt|, |childdoc.ins| and |childdoc.dtx|.
%
\begin{itemize}
\item
Run (pdf)\LaTeX{} on |childdoc.dtx|
to compile the manual |childdoc.pdf| (this file).
\item
Run \LaTeX{} on |childdoc.ins| to create the definitions file |childdoc.def|
and the sample |cdocsamp.tex| with include files
|cdocsch1.tex|, |cdocsch2.tex|, |cdocspt3.tex|, |cdocspt4.tex|,
|cdocsdrf.tex|, |cdocsfn1.tex|, |cdocsfn2.tex|.
Then copy the file |childdoc.def| to an appropriate directory of your \LaTeX{}
distribution, e.g.\ \textit{texmf-root}|/tex/latex/childdoc|.
\end{itemize}

%%%%%%%%%%%%%%%%%%%%%%%%%%%%%%%%%%%%%%%%%%%%%%%%%%%%%%%%%%%%%%%%%%%%%%%%%%%%%%%%
\subsection{Related CTAN Packages}

There are several other packages which offer a similar functionality:
%
\begin{itemize}
\item
The packages
\href{http://ctan.org/pkg/docmute}{\textsf{docmute}},
\href{http://ctan.org/pkg/includex}{\textsf{includex}} and
\href{http://ctan.org/pkg/standalone}{\textsf{standalone}}
provide commands to include only the document body of
a child file thus allowing both files to be compiled individually.
\item
The packages \href{http://ctan.org/pkg/subdocs}{\textsf{subdocs}}
and \href{http://ctan.org/pkg/subfiles}{\textsf{subfiles}}
provide structures in which the main and child documents can be
encapsulated and allowing them to be compiled individually.
The inclusion mechanism is different from the conventional |\include|.
\item
The package \href{http://ctan.org/pkg/combine}{\textsf{combine}}
is an elaborate solution to combine several documents into one.
\end{itemize}
%
See also the CTAN topic \href{http://ctan.org/topic/subdocs}{\textsf{subdocs}}
for further related packages.
The present package differs from the above solutions in that
a document structure constructed with the conventional |\include| mechanism
just needs two extra commands at the top of every file
such that all constituent files can be compiled individually.

%%%%%%%%%%%%%%%%%%%%%%%%%%%%%%%%%%%%%%%%%%%%%%%%%%%%%%%%%%%%%%%%%%%%%%%%%%%%%%%%
%\subsection{Feature Suggestions}
%
%The following is a list of features which may be useful for future
%versions of this package:
%%
%\begin{itemize}
%\item
%\ldots
%\end{itemize}

%%%%%%%%%%%%%%%%%%%%%%%%%%%%%%%%%%%%%%%%%%%%%%%%%%%%%%%%%%%%%%%%%%%%%%%%%%%%%%%%
\subsection{Revision History}

%%%%%%%%%%%%%%%%%%%%%%%%%%%%%%%%%%%%%%%%
\paragraph{v2.0:} 2018/12/30

\begin{itemize}
\item
immediate forward processing
\item
added |\childdocby| mechanism
\item
manual restructured
\end{itemize}

%%%%%%%%%%%%%%%%%%%%%%%%%%%%%%%%%%%%%%%%
\paragraph{v1.6:} 2018/01/17

\begin{itemize}
\item
application for development of include files
\item
corrections to manual
\end{itemize}

%%%%%%%%%%%%%%%%%%%%%%%%%%%%%%%%%%%%%%%%
\paragraph{v1.5:} 2017/05/21

\begin{itemize}
\item
more complete structuring introduced
\item
|\childdocof| introduced
\item
|\childdoc| renamed to |\childdocmain|
\item
|\childredirect| renamed to |\childdocforward| and |\childdocforwardprefix|
and functionality expanded
\end{itemize}

%%%%%%%%%%%%%%%%%%%%%%%%%%%%%%%%%%%%%%%%
\paragraph{v1.0:} 2017/04/27

\begin{itemize}
\item
manual and install package
\item
first version published on CTAN
\end{itemize}

%%%%%%%%%%%%%%%%%%%%%%%%%%%%%%%%%%%%%%%%
\paragraph{v0.6:} 2017/04/26

\begin{itemize}
\item
redirection mechanism added
\end{itemize}

%%%%%%%%%%%%%%%%%%%%%%%%%%%%%%%%%%%%%%%%
\paragraph{v0.5:} 2017/04/26

\begin{itemize}
\item
functionality in definition file
\end{itemize}


%%%%%%%%%%%%%%%%%%%%%%%%%%%%%%%%%%%%%%%%%%%%%%%%%%%%%%%%%%%%%%%%%%%%%%%%%%%%%%%%
%%%%%%%%%%%%%%%%%%%%%%%%%%%%%%%%%%%%%%%%%%%%%%%%%%%%%%%%%%%%%%%%%%%%%%%%%%%%%%%%
%%%%%%%%%%%%%%%%%%%%%%%%%%%%%%%%%%%%%%%%%%%%%%%%%%%%%%%%%%%%%%%%%%%%%%%%%%%%%%%%
\appendix

\settowidth\MacroIndent{\rmfamily\scriptsize 000\ }

 \DocInput{childdoc.dtx}

\end{document}
%</driver>
% \fi
%
% %%%%%%%%%%%%%%%%%%%%%%%%%%%%%%%%%%%%%%%%%%%%%%%%%%%%%%%%%%%%%%%%%%%%%%%%%%%%%%
% %%%%%%%%%%%%%%%%%%%%%%%%%%%%%%%%%%%%%%%%%%%%%%%%%%%%%%%%%%%%%%%%%%%%%%%%%%%%%%
% \section{Sample}
%\iffalse
%<*samplemain>
%\fi
%
% The following presents a sample document
% with two chapters, two parts, a title page,
% a compile flag as well as three forwarding files to set the flag.
% It consists of eight |.tex| files:
% \begin{center}
% \begin{tabular}{ll}
% |cdocsamp.tex|&main file\\
% |cdocsch1.tex|&include file for chapter 1\\
% |cdocsch2.tex|&include file for chapter 2\\
% |cdocspt3.tex|&include file for part 3\\
% |cdocspt4.tex|&include file for part 4\\
% |cdocsdrf.tex|&forwarding file for main file in draft mode\\
% |cdocsfi1.tex|&forwarding file for final version of chapter 1\\
% |cdocsfi2.tex|&forwarding file for final version of chapter 2\\
% \end{tabular}
% \end{center}
% Each of the eight files can be compiled directly by the \LaTeX{} compiler.
%
% %%%%%%%%%%%%%%%%%%%%%%%%%%%%%%%%%%%%%%
% \paragraph{Main File.}
%
% The main file is called |cdocsamp.tex|.
%
% Load the \textsf{childdoc} definitions and
% declare the filename for the main document:
%    \begin{macrocode}
\input{childdoc.def}
\childdocmain{}
%    \end{macrocode}

% Optional override for |\version| flag:
%    \begin{macrocode}
%%\ifchilddoc\else\providecommand{\version}{draft}\fi
%    \end{macrocode}

% Define the default values for the |\version| flag
% (|final| for the main file and |draft| for childs):
%    \begin{macrocode}
\ifchilddoc
\providecommand{\version}{draft}
\else
\providecommand{\version}{final}
\fi
%    \end{macrocode}

% Load the standard document class:
%    \begin{macrocode}
\documentclass[12pt]{article}
%    \end{macrocode}

% Start the document body:
%    \begin{macrocode}
\begin{document}
%    \end{macrocode}

% Declare a title page.
% Print title, part of document being processed and version flag:
%    \begin{macrocode}
\addtocounter{page}{-1}
\begin{center}
{\LARGE\bfseries{}childdoc example\par}
\vspace{1cm}
\ifchilddoc
\ifchilddocmanual part\else chapter\fi:
`\childdocname' of `\childdocjob'\par
\else
main document: `\childdocjob'\par
\fi
version: \version\par
\end{center}
\newpage
%    \end{macrocode}

% Manually include selected file,
% otherwise process as usual:
%    \begin{macrocode}
\ifchilddocmanual
\section*{part `\childdocname'}
\input{\childdocname}
\else
%    \end{macrocode}

% Include the two chapters:
%    \begin{macrocode}
\include{cdocsch1}
\include{cdocsch2}
%    \end{macrocode}

% Include the two parts unless only chapters should be displayed:
%    \begin{macrocode}
\ifchilddoc\else
\section{part three}
\input{cdocspt3}
\section{part four}
\input{cdocspt4}
\fi
%    \end{macrocode}

% Process as usual until here:
%    \begin{macrocode}
\fi
%    \end{macrocode}

% End of document body:
%    \begin{macrocode}
\end{document}
%    \end{macrocode}
%\iffalse
%</samplemain>
%\fi
%
% %%%%%%%%%%%%%%%%%%%%%%%%%%%%%%%%%%%%%%
% \paragraph{Chapter Include Files.}
%
% The include files are called |cdocsch1.tex| and |cdocsch2.tex|.
%
%\iffalse
%<*samplechap1|samplechap2>
%\fi

% Optional override for |\version| flag:
%    \begin{macrocode}
%%\providecommand{\version}{final}
%    \end{macrocode}

% Include the main document:
%    \begin{macrocode}
\input{childdoc.def}
\childdocof{cdocsamp}
%    \end{macrocode}

%\iffalse
%</samplechap1|samplechap2>
%\fi
%
%\iffalse
%<*samplechap1>
%\fi
% Some text for chapter 1:
%    \begin{macrocode}
\section{one}
some text in chapter one
%    \end{macrocode}

%\iffalse
%</samplechap1>
%\fi
% Some text for chapter 2:
%\iffalse
%<*samplechap2>
%\fi
%    \begin{macrocode}
\section{two}
more text in chapter two
%    \end{macrocode}

%\iffalse
%</samplechap2>
%\fi
%
% %%%%%%%%%%%%%%%%%%%%%%%%%%%%%%%%%%%%%%
% \paragraph{Part Include Files.}
%
% The include files are called |cdocspt3.tex| and |cdocspt4.tex|.
%
%\iffalse
%<*samplepart3|samplepart4>
%\fi

% Optional override for |\version| flag:
%    \begin{macrocode}
%%\providecommand{\version}{final}
%    \end{macrocode}

% Include the main document:
%    \begin{macrocode}
\input{childdoc.def}
\childdocby{cdocsamp}
%    \end{macrocode}

%\iffalse
%</samplepart3|samplepart4>
%\fi
%
%\iffalse
%<*samplepart3>
%\fi
% Some text for part 3:
%    \begin{macrocode}
some text in part three
%    \end{macrocode}

%\iffalse
%</samplepart3>
%\fi
% Some text for part 4:
%\iffalse
%<*samplepart4>
%\fi
%    \begin{macrocode}
more text in part four
%    \end{macrocode}

%\iffalse
%</samplepart4>
%\fi
%
% %%%%%%%%%%%%%%%%%%%%%%%%%%%%%%%%%%%%%%
% \paragraph{Forwarding for a Complete Draft.}
%
% The following forwarding file |cdocsdrf.tex|
% compiles the main document in draft mode:
%\iffalse
%<*sampledraft>
%\fi
%    \begin{macrocode}
\def\version{draft}
\input{childdoc.def}
\childdocforward{cdocsamp}
%    \end{macrocode}

%\iffalse
%</sampledraft>
%\fi
%
% %%%%%%%%%%%%%%%%%%%%%%%%%%%%%%%%%%%%%%
% \paragraph{Forwarding for Final Version of the Chapters.}
%
% The following forwarding files |cdocsfn1.tex| and |cdocsfn2.tex|
% (with identical content)
% compile the final versions of the child documents
% |cdocsch1.tex| and |cdocsch2.tex|, respectively:
%\iffalse
%<*samplefinal>
%\fi
%    \begin{macrocode}
\def\version{final}
\input{childdoc.def}
\childdocforwardprefix[cdocsamp]{cdocsfn}{cdocsch}
%    \end{macrocode}

%\iffalse
%</samplefinal>
%\fi
%
% %%%%%%%%%%%%%%%%%%%%%%%%%%%%%%%%%%%%%%
% \paragraph{Command Line Processing.}
%
% The following three command lines generate the output files
% |cdocscld|, |cdocscl1| and |cdocscl2|
% which should be identical to
% |cdocsdrf|, |cdocsch1| and |cdocsfn2|, respectively:
% \begin{center}
% \begin{tabular}{l}
% |latex -jobname cdocscld \|\\
% |  "\def\version{draft}\input{childdoc.def}\childdocforward{cdocsamp}"|\\
% |latex -jobname cdocscl1 \|\\
% |  "\input{childdoc.def}\childdocforward[cdocsamp]{cdocsch1}"|\\
% |latex -jobname cdocscl2 \|\\
% |  "\def\version{final}\input{childdoc.def}\childdocforward{cdocsch2}"|
% \end{tabular}
% \end{center}
% Note that the trailing backslash on each first line
% merely continues the input to the second line
% (for convenient cut ant paste).
% Furthermore, the command |latex| can be replaced by any
% of its alternative versions such as |pdflatex|.
%
% %%%%%%%%%%%%%%%%%%%%%%%%%%%%%%%%%%%%%%%%%%%%%%%%%%%%%%%%%%%%%%%%%%%%%%%%%%%%%%
% %%%%%%%%%%%%%%%%%%%%%%%%%%%%%%%%%%%%%%%%%%%%%%%%%%%%%%%%%%%%%%%%%%%%%%%%%%%%%%
% \section{Implementation}
%\iffalse
%<*package>
%\fi
%
% This section describes the definitions file |childdoc.def|.

% The definitions cannot be loaded using |\usepackage| or |\RequirePackage|
% which has a mechanism to prevent loading a style file more than once.
% When loading the definitions by means of |\input|
% multiple instances have to be prevented manually:
%\iffalse
%This code needs to be before the `\ProvidesFile' directive
%which is defined at the beginning of this file.
%Therefore it is also placed there and commented out here.
%</package>
%<*discard>
%\fi
%    \begin{macrocode}
\ifdefined\childdocmain\endinput\fi
%    \end{macrocode}
%\iffalse
%</discard>
%<*package>
%\fi
%
% \macro{\ifchilddoc}
% \macro{\ifchilddocmanual}
% The conditional |\ifchilddoc| tells whether a
% child (true) or main (false) document is being compiled.
% The conditional |\ifchilddocmanual| tells whether
% the |\includeonly| mechanism is used (false) or
% the selection of child files must be performed manually (true).
% The definitions initialise to false:
%    \begin{macrocode}
\newif\ifchilddoc
\newif\ifchilddocmanual
%    \end{macrocode}

% \macro{\childdocname}
% \macro{\childdocjob}
% The macro |\childdocname| stores the name of the main document
% to be compiled. The macro |\childdocjob| stores the name of
% the document on which the \LaTeX{} compiler was originally invoked.
% The content of |\jobname| cannot be compared
% to filenames specified in the source due to different catcodes.
% The following code rescans |\jobname|, stores the result
% in |\childdocname| and saves a copy in |\childdocjob|:
%    \begin{macrocode}
\edef\childdocname{\scantokens\expandafter{\jobname\noexpand}}
\let\childdocjob\childdocname
%    \end{macrocode}

% \macro{\childdocdisable}
% The macro |\childdocdisable| prevents the main file
% from being processed more than once.
% At this stage, the main document command |\childdocmain|
% is assumed to be called once again where it should do nothing.
% Any subsequent call to it should prevent
% a secondary processing of the main document
% It overwrites the forwarding commands
% |\childdocof| and |\childdocforward|
% with empty macros to prevent further inclusions of the main document:
%    \begin{macrocode}
\newcommand{\childdocdisable}
{
  \renewcommand{\childdocmain}[1]{\renewcommand{\childdocmain}[1]{\endinput}}
  \renewcommand{\childdocof}[1]{}
  \renewcommand{\childdocby}[2][]{}
  \renewcommand{\childdocforward}[2][]{}
  \renewcommand{\childdocdisable}{}
}
%    \end{macrocode}

% \macro{\childdocmain}
% The macro |\childdocmain| is to be called at the top of the main file
% with nothing or the main filename (without extension) as argument.
% First, it breaks loops.
% If the argument is not empty and does not match |\childdocname|
% (which is set by the first inclusion of |childdoc.def|),
% |\ifchilddoc| is set to true, |\includeonly| is applied to the child file
% and |\jobname| is set to the main file
% (for proper handling of |.aux| files):
%    \begin{macrocode}
\newcommand{\childdocmain}[1]
{
  \childdocdisable\childdocmain{}
  \if?#1?\else
    \begingroup
      \def\childdoctmp{#1}
      \ifx\childdoctmp\childdocname
        \def\childdoctmp{}
      \else
        \def\childdoctmp
        {
          \childdoctrue
          \includeonly{\childdocname}
          \def\childdocjob{#1}
          \def\jobname{#1}
        }
      \fi
      \expandafter
    \endgroup
    \childdoctmp
  \fi
}
%    \end{macrocode}

% \macro{\childdocof}
% The command |\childdocof| redirects
% compilation to the main file |#1|.
%    \begin{macrocode}
\newcommand{\childdocof}[1]
{
  \childdocdisable
  \childdoctrue
  \includeonly{\childdocname}
  \def\jobname{#1}
  \def\childdocjob{#1}
  \input{#1}
}
%    \end{macrocode}

% \macro{\childdocby}
% The command |\childdocby| ....
%    \begin{macrocode}
\newcommand{\childdocby}[2][]
{
  \childdocdisable
  \childdoctrue
  \childdocmanualtrue
  \if?#1?\else
    \def\jobname{#2}
  \fi
  \def\childdocjob{#2}
  \input{#2}
  \endinput
}
%    \end{macrocode}

% \macro{\childdocforward}
% The command |\childdocforward| redirects
% compilation to the main file or
% (if the optional argument is given) a child file.
% Parameters are set as if the main file
% or a child file starting with |\childdocof| was compiled.
% Then compilation is handed over to the main file:
%    \begin{macrocode}
\newcommand{\childdocforward}[2][]
{
  \begingroup
    \if?#1?
      \def\childdoctmp
      {
        \def\childdocname{#2}
        \def\childdocjob{#2}
        \def\jobname{#2}
        \input{#2}
        \endinput
      }
    \else
      \def\childdoctmp
      {
        \childdocdisable
        \def\childdocname{#2}
        \childdoctrue
        \includeonly{#2}
        \def\childdocjob{#1}
        \def\jobname{#1}
        \input{#1}
        \endinput
      }
    \fi
    \expandafter
  \endgroup
  \childdoctmp
}
%    \end{macrocode}

% \macro{\childdocforwardprefix}
% The command |\childdocforwardprefix| redirects
% compilation to the main or a child file by means of a pattern.
% The prefix |#1| in the current filename is replaced by |#2|
% and the suffix of the current filename is kept
% (it is assumed that the filename does not contain the substring `|~~~|'
% which is used as a delimiter).
% Compilation is handed over to the new file by |\childdocforward|:
%    \begin{macrocode}
\newcommand{\childdocforwardprefix}[3][]
{
  \begingroup
    \def\childdocextract #2##1~~~{\def\childdoctmp{\childdocforward[#1]{#3##1}}}
    \expandafter\childdocextract\childdocname~~~
    \expandafter
  \endgroup
  \childdoctmp
}
%    \end{macrocode}

% \macro{\childdoc}
% The deprecated macro |\childdoc| is a legacy version of |\childdocmain|:
%    \begin{macrocode}
\newcommand{\childdoc}{\childdocmain}
%    \end{macrocode}

% \macro{\childdocredirect}
% The deprecated macro |\childdocredirect| is a legacy version
% of |\childdocforward| and |\childdocforwardprefix|:
%    \begin{macrocode}
\newcommand{\childdocredirect}[2][]
{
  \begingroup
    \if?#1?
      \def\childdoctmp{\childdocforward{#2}}
    \else
      \def\childdoctmp{\childdocforwardprefix{#1}{#2}}
    \fi
    \expandafter
  \endgroup
  \childdoctmp
}
%    \end{macrocode}

%\iffalse
%</package>
%\fi
%
\endinput
|\\
|\childdocmain{}|\\
\end{tabular}
\end{center}
at the very top of the main \LaTeX{} file,
in particular \emph{before} the |\documentclass| statement!
The argument of |\childdocmain| should be left empty
(but it must be present).

%%%%%%%%%%%%%%%%%%%%%%%%%%%%%%%%%%%%%%%%
\DescribeMacro{\childdocof}
Furthermore, add the commands
\begin{center}
\begin{tabular}{l}
|% \iffalse
%
% childdoc.dtx Copyright (C) 2017-2018 Niklas Beisert
%
% This work may be distributed and/or modified under the
% conditions of the LaTeX Project Public License, either version 1.3
% of this license or (at your option) any later version.
% The latest version of this license is in
%   http://www.latex-project.org/lppl.txt
% and version 1.3 or later is part of all distributions of LaTeX
% version 2005/12/01 or later.
%
% This work has the LPPL maintenance status `maintained'.
%
% The Current Maintainer of this work is Niklas Beisert.
%
% This work consists of the files childdoc.dtx and childdoc.ins
% and the derived files childdoc.def and cdocsamp.tex with
% cdocsch1.tex, cdocsch2.tex, cdocsdrf.tex, cdocsfn1.tex, cdocsfn2.tex.
%
%<package>\ifdefined\childdocmain\endinput\fi
%<package>\ProvidesFile{childdoc.def}[2018/12/30 v2.0 child document driver]
%<samplemain>\ProvidesFile{cdocsamp.tex}[2018/12/30 v2.0 sample for childdoc]
%<*driver>
%\ProvidesFile{childdoc.drv}[2018/12/30 v2.0 childdoc reference manual file]
\PassOptionsToClass{10pt,a4paper}{article}
\documentclass{ltxdoc}

\usepackage[margin=35mm]{geometry}
\usepackage{hyperref}
\usepackage{hyperxmp}
\usepackage[usenames]{color}

\hypersetup{colorlinks=true}
\hypersetup{pdfstartview=FitH}
\hypersetup{pdfpagemode=UseNone}
\hypersetup{pdfsource={}}
\hypersetup{pdflang={en-UK}}
\hypersetup{pdfcopyright={Copyright 2017-2018 Niklas Beisert.
  This work may be distributed and/or modified under the
  conditions of the LaTeX Project Public License, either version 1.3
  of this license or (at your option) any later version.}}
\hypersetup{pdflicenseurl={http://www.latex-project.org/lppl.txt}}
\hypersetup{pdfcontactaddress={ETH Zurich, ITP, HIT K,
  Wolfgang-Pauli-Strasse 27}}
\hypersetup{pdfcontactpostcode={8093}}
\hypersetup{pdfcontactcity={Zurich}}
\hypersetup{pdfcontactcountry={Switzerland}}
\hypersetup{pdfcontactemail={nbeisert@itp.phys.ethz.ch}}
\hypersetup{pdfcontacturl={http://people.phys.ethz.ch/\xmptilde nbeisert/}}

\newcommand{\secref}[1]{\hyperref[#1]{section \ref*{#1}}}

\parskip1ex
\parindent0pt
\let\olditemize\itemize
\def\itemize{\olditemize\parskip0pt}

\begin{document}

\title{The \textsf{childdoc} Package}
\hypersetup{pdftitle={The childdoc Package}}
\author{Niklas Beisert\\[2ex]
  Institut f\"ur Theoretische Physik\\
  Eidgen\"ossische Technische Hochschule Z\"urich\\
  Wolfgang-Pauli-Strasse 27, 8093 Z\"urich, Switzerland\\[1ex]
  \href{mailto:nbeisert@itp.phys.ethz.ch}
  {\texttt{nbeisert@itp.phys.ethz.ch}}}
\hypersetup{pdfauthor={Niklas Beisert}}
\hypersetup{pdfsubject={Manual for the LaTeX2e Package childdoc}}
\date{30 December 2018, \textsf{v2.0}}
\maketitle

\begin{abstract}\noindent
\textsf{childdoc} is a \LaTeXe{} package
that enables the direct compilation
of document sections included by |\include|
to individual files.
\end{abstract}

\begingroup
\parskip0ex
\tableofcontents
\endgroup

%%%%%%%%%%%%%%%%%%%%%%%%%%%%%%%%%%%%%%%%%%%%%%%%%%%%%%%%%%%%%%%%%%%%%%%%%%%%%%%%
%%%%%%%%%%%%%%%%%%%%%%%%%%%%%%%%%%%%%%%%%%%%%%%%%%%%%%%%%%%%%%%%%%%%%%%%%%%%%%%%
\section{Introduction}

\LaTeX{} provides a mechanism to structure a large document (such as a book)
into a main file and several child files (containing the chapters)
using the |\include| command.
This mechanism is beneficial for documents
which span hundreds of pages in order to
make the source file(s) more manageable.
Moreover, compilation can be restricted to
selected child files by means of the |\includeonly| command.
The latter feature can be used to reduce the compilation time while editing
(this was significantly more useful in the earlier days of \LaTeX{})
or to generate a smaller document which is easier to navigate.
Another application of |\includeonly| is to generate
documents consisting of selected parts of the complete document.

However, there are a few drawbacks of the plain |\include| mechanism:
\begin{itemize}
\item
The child files cannot be compiled on their own,
they can only be compiled via the main file.
A naive editing environment
(such as a text editor with an option
to have the current file processed by \LaTeX)
may require one to switch to the main file before compiling;
attempting to compile the child file produces errors.
\item
The main file must be modified (each time)
to adjust the |\includeonly| command
to the present needs. This easily leaves the main file in a messy state.
\item
The generated document will always carry the filename
of the main document. This is inconvenient if
several child files are to be compiled and
to be kept for distribution.
\end{itemize}

The present package provides a simple interface
to make child files individually compilable by \LaTeX{}.
Compiling a child file then has the same effect as compiling
the main file with an |\includeonly| command
to select the appropriate child.
Moreover the generated document will carry the name of the child
rather than the main file.
This resolves all three above issues.

This feature is meant to make the editing of books,
thesis documents and lecture notes somewhat more convenient.
However, the package can also be used efficiently for
composing a series of documents (such as exercise sheets)
which are typically distributed individually.
It then assists the author in generating the individual documents
(potentially in different versions)
as well as a document containing the collected series.
Another application is in developing style files
or other kinds of included material
where compilation of the style file could redirect
to a sample or test file.

%%%%%%%%%%%%%%%%%%%%%%%%%%%%%%%%%%%%%%%%%%%%%%%%%%%%%%%%%%%%%%%%%%%%%%%%%%%%%%%%
%%%%%%%%%%%%%%%%%%%%%%%%%%%%%%%%%%%%%%%%%%%%%%%%%%%%%%%%%%%%%%%%%%%%%%%%%%%%%%%%
\section{Usage}

First of all, the package \textsf{childdoc} is \emph{not} a standard
\LaTeXe{} |.sty| style file! Therefore it needs to be invoked in
a non-standard way.

%%%%%%%%%%%%%%%%%%%%%%%%%%%%%%%%%%%%%%%%%%%%%%%%%%%%%%%%%%%%%%%%%%%%%%%%%%%%%%%%
\subsection{Included Files}
\label{sec:include}

%%%%%%%%%%%%%%%%%%%%%%%%%%%%%%%%%%%%%%%%
\DescribeMacro{\childdocmain}
To use the package, add the commands
\begin{center}
\begin{tabular}{l}
|\input{childdoc.def}|\\
|\childdocmain{}|\\
\end{tabular}
\end{center}
at the very top of the main \LaTeX{} file,
in particular \emph{before} the |\documentclass| statement!
The argument of |\childdocmain| should be left empty
(but it must be present).

%%%%%%%%%%%%%%%%%%%%%%%%%%%%%%%%%%%%%%%%
\DescribeMacro{\childdocof}
Furthermore, add the commands
\begin{center}
\begin{tabular}{l}
|\input{childdoc.def}|\\
|\childdocof{|\textit{main}|}|\\
\end{tabular}
\end{center}
at the top of every child file \textit{child}
which is included by |\include{|\textit{child}|}|
from within the main file
(or at least for those files to be compiled individually).
The argument \textit{main} must be the filename of the main file.

There are a couple of
considerations in setting up the main and child documents:

%%%%%%%%%%%%%%%%%%%%%%%%%%%%%%%%%%%%%%%%
\paragraph{Restrictions.}

Please note the following restrictions:
\begin{itemize}
\item
|\childdocmain| must be called with one argument \textit{main}
to ensure compatibility with earlier version of the package.
It must either be empty (|\childdocmain{}|)
or precisely match the filename of the main file in which it is specified.
See \secref{sec:detection} for further information.
\item
The filename \textit{main} must be specified without the |.tex| extension.
\item
The filename \textit{main} is case sensitive
(even in case-insensitive file systems)
due to internal string comparison.
\item
The argument \textit{main} should be fully expanded, it cannot be a macro.
\item
Subdirectories and special characters should be avoided in filenames.
\item
The command |\childdocmain{|\textit{main}|}| must be followed by a whitespace.
It should not be followed immediately by another command
or by a comment mark `|%|'.
This is because the \TeX{} parser reads the token immediately following
the argument of |\childdocmain| and puts it
at the beginning of every child section;
however, a white\-space is ignored.
\end{itemize}

%%%%%%%%%%%%%%%%%%%%%%%%%%%%%%%%%%%%%%%%
\paragraph{Content of Main File.}

It is advisable to place all content in the child files included by |\include|.
Any output contained in the main file will appear in all child documents
unless suppressed manually;
it cannot be suppressed automatically by the |\includeonly| directive
and thus should normally be avoided.
A method to include some content in the main file
by means of conditional processing is described in \secref{sec:conditional}.

%%%%%%%%%%%%%%%%%%%%%%%%%%%%%%%%%%%%%%%%
\paragraph{Page Numbering.}

When only a part of the document is compiled,
the appropriate numbering of pages
(as well as other status parameters)
is determined from the |.aux| files.
The latter contain information from previous passes.
However this information needs to propagate through
all intermediate child documents.
Therefore the page numbering in child documents may well
be inconsistent until the complete document is compiled at least once.

A useful (if unconventional) way to always ensure a consistent
page numbering is to restart the numbering in each child document
and denote the pages by `\textit{child}|.|\textit{page}'
where \textit{child} represents the chapter/section number of the child file.
This can be achieved by the command
|\numberwithin{page}{|\textit{child}|}|
of the \textsf{amsmath} package
where \textit{child} can be |chapter| or |section|
depending on the chosen structuring.
Alternatively, one can modify the macro |\thepage| appropriately
and reset the counter |page| at the start of each child file.

%%%%%%%%%%%%%%%%%%%%%%%%%%%%%%%%%%%%%%%%%%%%%%%%%%%%%%%%%%%%%%%%%%%%%%%%%%%%%%%%
\subsection{Conditional Processing}
\label{sec:conditional}

The package provides a mechanism to compile different versions
of a document. To customise the versions further some conditional processing
can come in handy to distinguish which version is being compiled.
The package provides two macros to describe the compilation context:

%%%%%%%%%%%%%%%%%%%%%%%%%%%%%%%%%%%%%%%%
\DescribeMacro{\ifchilddoc}
The conditional |\ifchilddoc| distinguishes between the compilation of
child documents and the main document:
%
\begin{center}
|\ifchilddoc |\textit{child-code}| |[|\||else |\textit{main-code}]| \||fi|
\end{center}

%%%%%%%%%%%%%%%%%%%%%%%%%%%%%%%%%%%%%%%%
\DescribeMacro{\childdocname}
\DescribeMacro{\childdocjob}
The macro |\childdocname| contains the filename (without extension)
of the main or child file being processed.
Note that |\childdocjob| will always contain the name of the main file.

%%%%%%%%%%%%%%%%%%%%%%%%%%%%%%%%%%%%%%%%
\paragraph{Title Page.}

Conditional processing can be used to include a title or banner page
in the main document when proper precautions are taken.
Importantly, the code in the main file should ensure that the page counter
(as well as other status parameters which are stored in the |.aux| files)
takes the same value after the conditional processing.
Otherwise the page numbers may take divergent values
depending on which part is compiled.

For example, a title page could be declared by:
%
\begin{center}
\begin{tabular}{l}
|\ifchilddoc\||else|\\
|\addtocounter{page}{-1}|\\
\textit{code for title page}\\
|\newpage|\\
|\||fi|
\end{tabular}
\end{center}
%
A banner page for the child documents can be generated by:
%
\begin{center}
\begin{tabular}{l}
|\ifchilddoc|\\
|\addtocounter{page}{-1}|\\
\textit{code for banner page}\\
|\newpage|\\
|\||fi|
\end{tabular}
\end{center}
%
Here one could write a message such as:
\begin{center}
|This is the part \childdocname{} of \childdocjob{}.|
\end{center}

%%%%%%%%%%%%%%%%%%%%%%%%%%%%%%%%%%%%%%%%%%%%%%%%%%%%%%%%%%%%%%%%%%%%%%%%%%%%%%%%
\subsection{Flags}
\label{sec:flags}

The package makes it easy to generate different versions
of the main or child documents.
To this end compilation flags can be defined
and assigned different default values.
They will be particularly useful in conjunction
with the forwarding mechanism described in \secref{sec:forward}.

For example, it may be useful to have a flag |\version|
which can be set to |draft| or |final|.
The document source will contain some conditional code
depending on the value of |\version|.
Suppose further, the flag should default to |final| for the main file
and to |draft| for child files
which is a natural assignment for editing the document.
This is achieved by placing the following code
in the preamble of the main document
(below the |\childdocmain| directive):
%
\begin{center}
\begin{tabular}{l}
|\ifchilddoc|\\
|\providecommand{\version}{draft}|\\
|\||else|\\
|\providecommand{\version}{final}|\\
|\||fi|
\end{tabular}
\end{center}
%
The definition by |\providecommand| makes sure
that previous definitions are not overwritten.
Further statements |\providecommand{\version}{...}|
can thus be added before the above code to override it.

For the main file, one might add a line
(between |\childdocmain| and the above block)
%
\begin{center}
|%\ifchilddoc\||else\providecommand{\version}{draft}\||fi|
\end{center}
%
which can be uncommented to produce a draft version.
Likewise one can add a line to the very top of a child file
(above the |\childdocof{|\textit{main}|}| directive)
%
\begin{center}
|%\providecommand{\version}{final}|
\end{center}
%
which can be uncommented to produce the final version of this child document.

%%%%%%%%%%%%%%%%%%%%%%%%%%%%%%%%%%%%%%%%%%%%%%%%%%%%%%%%%%%%%%%%%%%%%%%%%%%%%%%%
\subsection{Forwarding}
\label{sec:forward}

Different versions of the main or child documents
using compilation flags as described in \secref{sec:flags}
can be (permanently) stored in different files
for convenient compilation, viewing and distribution.
To this end, the package defines a command
to pass on compilation to a different file:

%%%%%%%%%%%%%%%%%%%%%%%%%%%%%%%%%%%%%%%%
\DescribeMacro{\childdocforward}
The command |\childdocforward| redirects processing to
another source file:
%
\begin{center}
\begin{tabular}{l}
|\input{childdoc.def}|\\
|\childdocforward[|\textit{main}|]{|\textit{dest}|}|\\
\end{tabular}
\end{center}
%
The argument \textit{dest} is the destination file
(without extension).
It should be the main file or one of the child files.
Note that further \textsf{childdoc} directives
such as |\childdocof| and |\childdocforward|
in the indicated file will be processed in this form.
The optional argument \textit{main}
passes on directly to the main file \textit{main}
while pretending to compile the child \textit{dest}.
This form behaves as if \textit{dest}
issues |\childdocof{|\textit{main}|}| right away,
and no further \textsf{childdoc} directives will be processed.

%%%%%%%%%%%%%%%%%%%%%%%%%%%%%%%%%%%%%%%%
\DescribeMacro{\...prefix}
In the alternative form |\childdocforwardprefix|,
%
\begin{center}
\begin{tabular}{l}
|\input{childdoc.def}|\\
|\childdocforwardprefix[|\textit{main}|]{|\textit{prefix}|}{|\textit{dest}|}|
\end{tabular}
\end{center}
%
the destination file is determined by a pattern
depending on the current file:
To make this work, the current file must be called
`{\textit{prefix}\hspace{0.2em}\textit{suffix}}'
with \textit{prefix} matching precisely the argument.
Processing is then passed on to the file
`{\textit{dest}\hspace{0.2em}\textit{suffix}}'.
Surely, the same effect is achieved by
directly specifying the
argument `{\textit{dest}\hspace{0.2em}\textit{suffix}}'
in the first form.
However, that requires to set up a different file
for each child. With the alternative form of the command
all these files can have exactly the same content
which simplifies setting them up and maintaining them.

For example, the following file |draft.tex|
with a compilation flag |\version| as described in \secref{sec:flags}
compiles the main document as a draft:
%
\begin{center}
\begin{tabular}{l}
|\def\version{draft}|\\
|\input{childdoc.def}|\\
|\childdocforward{|\textit{main}|}|
\end{tabular}
\end{center}
%
Likewise, the following files |final|\textit{nn}|.tex|
compile the final version of the child document
|child|\textit{nn}|.tex|:
%
\begin{center}
\begin{tabular}{l}
|\def\version{final}|\\
|\input{childdoc.def}|\\
|\childdocforwardprefix{final}{child}|
\end{tabular}
\end{center}
%

Note that when several versions of a main file and/or of each child file
are to be generated, it may be convenient to set up a |Makefile| or
shell script to automatise the process.

%%%%%%%%%%%%%%%%%%%%%%%%%%%%%%%%%%%%%%%%%%%%%%%%%%%%%%%%%%%%%%%%%%%%%%%%%%%%%%%%
\subsection{Command Line Processing}
\label{sec:commandline}

The effect of redirection files can also be achieved by invoking
the \LaTeX{} compiler with a more elaborate command line.
Most conveniently this should be done as part
of a shell script or a |Makefile|.

When using \textsf{childdoc} in the main file, the following
command lines effectively perform a redirection
(note that depending on the shell being used,
backslashes may have to be doubled: `|\|' $\to$ `|\\|'):
%
\begin{center}
|... -jobname "|\textit{target}|" |\\|"|[\textit{flags}]%
|\input{childdoc.def}\childdocforward[|\textit{main}|]{|\textit{dest}|}"|
\end{center}
%
Here \textit{target} is the name of the output file,
\textit{main} is the name of the main file
and \textit{dest} is the name of the main or child file to be processed
(all filenames without extensions).
The optional argument \textit{main} can be omitted
if \textit{main} matches \textit{dest}.
Optionally, compilation \textit{flags} can be defined via |\def| commands.
This command line makes the \TeX{} engine believe
it is compiling the file \textit{target}
whose content is specified as the latter parameter.
The provided code then forwards the processing to
\textit{main} or \textit{dest} as described in \secref{sec:forward}.

%%%%%%%%%%%%%%%%%%%%%%%%%%%%%%%%%%%%%%%%%%%%%%%%%%%%%%%%%%%%%%%%%%%%%%%%%%%%%%%%
\subsection{Include by Input}
\label{sec:input}

Including child documents by |\include| has some restrictions by design.
Most notably, the content of a child document always occupies
its own set of pages; pages cannot be shared between child documents.
Usually, this behaviour makes perfect sense
because each child document contain an essential part of the document.
However, in some situations it may be desirable to compose
a document from a collection of parts
without having mandatory page breaks between then.
For this case, the package
provides a mechanism to include parts
by |\input| which can also be processed individually.
However, by construction this mechanism
requires manual handling of the content to be output.

%%%%%%%%%%%%%%%%%%%%%%%%%%%%%%%%%%%%%%%%
\DescribeMacro{\ifchilddocmanual}
The main file should be prepared as usual, see \secref{sec:include}.
However, the document body must make a distinction
between processing of an individual part and of the main document, e.g.:
%
\begin{center}
\begin{tabular}{l}
|\ifchilddocmanual|\\
|\input{\childdocname}|\\
|\||else|\\
\textit{document body with }|\input{|\textit{part}|}|\\
|\||fi|
\end{tabular}
\end{center}
%
The conditional |\ifchilddocmanual| is true whenever
a part to be included by |\input| is being compiled,
and the name of the part is stored in |\childdocname|.

%%%%%%%%%%%%%%%%%%%%%%%%%%%%%%%%%%%%%%%%
\DescribeMacro{\childdocby}
Each part to be included by |\input| should start with:
%
\begin{center}
\begin{tabular}{l}
|\input{childdoc.def}|\\
|\childdocby{|\textit{main}|}|\\
\end{tabular}
\end{center}
%
The directive |\childdocby| is similar to |\childdocof|
described in \secref{sec:include},
but the subsequent selection of content must be done manually.
To that end, both |\ifchilddoc| and |\ifchilddocmanual|
will be true upon processing of a part,
and the name of the part is stored in |\childdocname|.
Note that |\jobname| will be set to the filename of the current part
so that each part receives an individual |.aux| file
that does not interfere with the |.aux| file(s) of the main document.
This behaviour can be altered by the alternative form
|\childdocby[*]{|\textit{main}|}| (with a non-empty optional argument)
which uses the |.aux| file of the main document
by setting |\jobname| to \textit{main}.

%%%%%%%%%%%%%%%%%%%%%%%%%%%%%%%%%%%%%%%%%%%%%%%%%%%%%%%%%%%%%%%%%%%%%%%%%%%%%%%%
\subsection{Driver Development}
\label{sec:driver}

The \textsf{childdoc} mechanism can also be use for the development
of definition files such as \LaTeX{} styles or classes.
This case differs from the above setup with multiple parts
included by |\include| in that no |\includeonly| should be invoked.
This can be achieved by starting the include file
(before |\ProvidesPackage|) with:
%
\begin{center}
\begin{tabular}{l}
|\input{childdoc.def}|\\
|\childdocforward{|\textit{main}|}|\\
\end{tabular}
\end{center}
%
or alternatively with:
%
\begin{center}
\begin{tabular}{l}
|\input{childdoc.def}|\\
|\childdocby{|\textit{main}|}|\\
\end{tabular}
\end{center}
%
Both forms have slightly different effects as described above.
The main file is prepared as usual, see \secref{sec:include}.

%%%%%%%%%%%%%%%%%%%%%%%%%%%%%%%%%%%%%%%%%%%%%%%%%%%%%%%%%%%%%%%%%%%%%%%%%%%%%%%%
\subsection{Legacy Detection}
\label{sec:detection}

The directive |\childdocmain| in the main file can detect
whether the complete document or merely a child is to be compiled
even without using the directive |\childdocof|.
This method is deprecated because it is less robust
and there is no compelling reason to use it;
it is merely provided for backward compatibility
and it may be removed in future versions.

If the detection mechanism is to be used,
it is mandatory to correctly specify
the filename of the main file as the argument of |\childdocmain|:
%
\begin{center}
\begin{tabular}{l}
|\input{childdoc.def}|\\
|\childdocmain{|\textit{main}|}|\\
\end{tabular}
\end{center}
%
If |\jobname| does not match the argument \textit{main} of |\childdocmain|,
it is assumed that |\jobname| points to the child file to be compiled.
When using |\childdocmain| with the main file specified as argument,
it suffices to start a child file
with just |\input{|\textit{main}|}|
without loading of the package and using |\childdocof|.
If instead all processing is done
with the appropriate \textsf{childdoc} directives,
the argument of \textit{main} of |\childdocmain| can be empty.

An alternative version of the command line processing described
in \secref{sec:commandline} using the detection mechanism reads:
%
\begin{center}
|... -jobname "|\textit{target}|" "|[\textit{flags}]%
[|\def\jobname{|\textit{dest}|}|]|\input{|\textit{main}|}"|
\end{center}

%%%%%%%%%%%%%%%%%%%%%%%%%%%%%%%%%%%%%%%%%%%%%%%%%%%%%%%%%%%%%%%%%%%%%%%%%%%%%%%%
\subsection{Manual Code}
\label{sec:manual}

In case one cannot be certain whether the definitions file |childdoc.def|
is installed on the target \TeX{} distribution
and one prefers not to ship it,
it is conceivable to paste a few relevant commands into the sources.

To that end, drop all statements |\input{childdoc.def}|
and perform the replacements as outlined below.
Instead of |\childdocmain{|\textit{main}|}| add the following code
to the top of the main file:
%
\begin{center}
\begin{tabular}{l}
|\||ifdefined\childdocname\endinput\||fi\newif\ifchilddoc|\\
|\edef\childdocname{\scantokens\expandafter{\jobname\noexpand}}|\\
|\def\childdocmain{|\textit{main}|}\||ifx\childdocmain\childdocname\||else|\\
|\childdoctrue\includeonly{\childdocname}\let\jobname\childdocmain\||fi|\\
\end{tabular}
\end{center}
%
Instead of |\childdocof{|\textit{main}|}| just include the main file
at the top of each child file:
%
\begin{center}
|\input{|\textit{main}|}|
\end{center}
%
A simple redirection |\childdocforward{|\textit{dest}|}| is achieved by:
%
\begin{center}
|\def\jobname{|\textit{dest}|}\input{\jobname}|
\end{center}
%
The redirection with prefix
|\childdocforwardprefix[|\textit{prefix}|]{|\textit{dest}|}|
is accomplished by:
%
\begin{center}
\begin{tabular}{l}
|{\edef\jobname{\scantokens\expandafter{\jobname\noexpand}}|\\
|\def\redirectjob |\textit{prefix}|#1~~~{\gdef\jobname{|\textit{dest}|#1}}|\\
|\expandafter\redirectjob\jobname~~~}\input{\jobname}|
\end{tabular}
\end{center}

In an alternative approach,
child documents can be compiled by a specific command line
without additional code or specific definitions:
%
\begin{center}
|... -jobname "|\textit{target}|" "|[\textit{flags}]%
|\includeonly{|\textit{dest}|}\input{|\textit{main}|}"|
\end{center}
%

%%%%%%%%%%%%%%%%%%%%%%%%%%%%%%%%%%%%%%%%%%%%%%%%%%%%%%%%%%%%%%%%%%%%%%%%%%%%%%%%
%%%%%%%%%%%%%%%%%%%%%%%%%%%%%%%%%%%%%%%%%%%%%%%%%%%%%%%%%%%%%%%%%%%%%%%%%%%%%%%%
\section{Information}

%%%%%%%%%%%%%%%%%%%%%%%%%%%%%%%%%%%%%%%%%%%%%%%%%%%%%%%%%%%%%%%%%%%%%%%%%%%%%%%%
\subsection{Copyright}

Copyright \copyright{} 2017--2018 Niklas Beisert

This work may be distributed and/or modified under the
conditions of the \LaTeX{} Project Public License, either version 1.3
of this license or (at your option) any later version.
The latest version of this license is in
  \url{http://www.latex-project.org/lppl.txt}
and version 1.3 or later is part of all distributions of \LaTeX{}
version 2005/12/01 or later.

This work has the LPPL maintenance status `maintained'.

The Current Maintainer of this work is Niklas Beisert.

This work consists of the files |README.txt|, |childdoc.ins| and |childdoc.dtx|
as well as the derived files |childdoc.def|, |cdocsamp.tex|
with |cdocsch1.tex|, |cdocsch2.tex|, |cdocspt3.tex|, |cdocspt4.tex|,
|cdocsdrf.tex|, |cdocsfn1.tex|, |cdocsfn2.tex|
as well as |childdoc.pdf|.

%%%%%%%%%%%%%%%%%%%%%%%%%%%%%%%%%%%%%%%%%%%%%%%%%%%%%%%%%%%%%%%%%%%%%%%%%%%%%%%%
\subsection{Files and Installation}

The package consists of the files:
%
\begin{center}
\begin{tabular}{ll}
    |README.txt|   & readme file \\
    |childdoc.ins| & installation file \\
    |childdoc.dtx| & source file \\
    |childdoc.def| & definition file \\
    |cdocsamp.tex| & sample main file \\
    |cdocsch1.tex| & sample include file \\
    |cdocsch2.tex| & sample include file \\
    |cdocspt3.tex| & sample part file \\
    |cdocspt4.tex| & sample part file \\
    |cdocsdrf.tex| & sample redirection file \\
    |cdocsfn1.tex| & sample redirection file \\
    |cdocsfn2.tex| & sample redirection file \\
    |childdoc.pdf| & manual
\end{tabular}
\end{center}
%
The distribution consists of the files
|README.txt|, |childdoc.ins| and |childdoc.dtx|.
%
\begin{itemize}
\item
Run (pdf)\LaTeX{} on |childdoc.dtx|
to compile the manual |childdoc.pdf| (this file).
\item
Run \LaTeX{} on |childdoc.ins| to create the definitions file |childdoc.def|
and the sample |cdocsamp.tex| with include files
|cdocsch1.tex|, |cdocsch2.tex|, |cdocspt3.tex|, |cdocspt4.tex|,
|cdocsdrf.tex|, |cdocsfn1.tex|, |cdocsfn2.tex|.
Then copy the file |childdoc.def| to an appropriate directory of your \LaTeX{}
distribution, e.g.\ \textit{texmf-root}|/tex/latex/childdoc|.
\end{itemize}

%%%%%%%%%%%%%%%%%%%%%%%%%%%%%%%%%%%%%%%%%%%%%%%%%%%%%%%%%%%%%%%%%%%%%%%%%%%%%%%%
\subsection{Related CTAN Packages}

There are several other packages which offer a similar functionality:
%
\begin{itemize}
\item
The packages
\href{http://ctan.org/pkg/docmute}{\textsf{docmute}},
\href{http://ctan.org/pkg/includex}{\textsf{includex}} and
\href{http://ctan.org/pkg/standalone}{\textsf{standalone}}
provide commands to include only the document body of
a child file thus allowing both files to be compiled individually.
\item
The packages \href{http://ctan.org/pkg/subdocs}{\textsf{subdocs}}
and \href{http://ctan.org/pkg/subfiles}{\textsf{subfiles}}
provide structures in which the main and child documents can be
encapsulated and allowing them to be compiled individually.
The inclusion mechanism is different from the conventional |\include|.
\item
The package \href{http://ctan.org/pkg/combine}{\textsf{combine}}
is an elaborate solution to combine several documents into one.
\end{itemize}
%
See also the CTAN topic \href{http://ctan.org/topic/subdocs}{\textsf{subdocs}}
for further related packages.
The present package differs from the above solutions in that
a document structure constructed with the conventional |\include| mechanism
just needs two extra commands at the top of every file
such that all constituent files can be compiled individually.

%%%%%%%%%%%%%%%%%%%%%%%%%%%%%%%%%%%%%%%%%%%%%%%%%%%%%%%%%%%%%%%%%%%%%%%%%%%%%%%%
%\subsection{Feature Suggestions}
%
%The following is a list of features which may be useful for future
%versions of this package:
%%
%\begin{itemize}
%\item
%\ldots
%\end{itemize}

%%%%%%%%%%%%%%%%%%%%%%%%%%%%%%%%%%%%%%%%%%%%%%%%%%%%%%%%%%%%%%%%%%%%%%%%%%%%%%%%
\subsection{Revision History}

%%%%%%%%%%%%%%%%%%%%%%%%%%%%%%%%%%%%%%%%
\paragraph{v2.0:} 2018/12/30

\begin{itemize}
\item
immediate forward processing
\item
added |\childdocby| mechanism
\item
manual restructured
\end{itemize}

%%%%%%%%%%%%%%%%%%%%%%%%%%%%%%%%%%%%%%%%
\paragraph{v1.6:} 2018/01/17

\begin{itemize}
\item
application for development of include files
\item
corrections to manual
\end{itemize}

%%%%%%%%%%%%%%%%%%%%%%%%%%%%%%%%%%%%%%%%
\paragraph{v1.5:} 2017/05/21

\begin{itemize}
\item
more complete structuring introduced
\item
|\childdocof| introduced
\item
|\childdoc| renamed to |\childdocmain|
\item
|\childredirect| renamed to |\childdocforward| and |\childdocforwardprefix|
and functionality expanded
\end{itemize}

%%%%%%%%%%%%%%%%%%%%%%%%%%%%%%%%%%%%%%%%
\paragraph{v1.0:} 2017/04/27

\begin{itemize}
\item
manual and install package
\item
first version published on CTAN
\end{itemize}

%%%%%%%%%%%%%%%%%%%%%%%%%%%%%%%%%%%%%%%%
\paragraph{v0.6:} 2017/04/26

\begin{itemize}
\item
redirection mechanism added
\end{itemize}

%%%%%%%%%%%%%%%%%%%%%%%%%%%%%%%%%%%%%%%%
\paragraph{v0.5:} 2017/04/26

\begin{itemize}
\item
functionality in definition file
\end{itemize}


%%%%%%%%%%%%%%%%%%%%%%%%%%%%%%%%%%%%%%%%%%%%%%%%%%%%%%%%%%%%%%%%%%%%%%%%%%%%%%%%
%%%%%%%%%%%%%%%%%%%%%%%%%%%%%%%%%%%%%%%%%%%%%%%%%%%%%%%%%%%%%%%%%%%%%%%%%%%%%%%%
%%%%%%%%%%%%%%%%%%%%%%%%%%%%%%%%%%%%%%%%%%%%%%%%%%%%%%%%%%%%%%%%%%%%%%%%%%%%%%%%
\appendix

\settowidth\MacroIndent{\rmfamily\scriptsize 000\ }

 \DocInput{childdoc.dtx}

\end{document}
%</driver>
% \fi
%
% %%%%%%%%%%%%%%%%%%%%%%%%%%%%%%%%%%%%%%%%%%%%%%%%%%%%%%%%%%%%%%%%%%%%%%%%%%%%%%
% %%%%%%%%%%%%%%%%%%%%%%%%%%%%%%%%%%%%%%%%%%%%%%%%%%%%%%%%%%%%%%%%%%%%%%%%%%%%%%
% \section{Sample}
%\iffalse
%<*samplemain>
%\fi
%
% The following presents a sample document
% with two chapters, two parts, a title page,
% a compile flag as well as three forwarding files to set the flag.
% It consists of eight |.tex| files:
% \begin{center}
% \begin{tabular}{ll}
% |cdocsamp.tex|&main file\\
% |cdocsch1.tex|&include file for chapter 1\\
% |cdocsch2.tex|&include file for chapter 2\\
% |cdocspt3.tex|&include file for part 3\\
% |cdocspt4.tex|&include file for part 4\\
% |cdocsdrf.tex|&forwarding file for main file in draft mode\\
% |cdocsfi1.tex|&forwarding file for final version of chapter 1\\
% |cdocsfi2.tex|&forwarding file for final version of chapter 2\\
% \end{tabular}
% \end{center}
% Each of the eight files can be compiled directly by the \LaTeX{} compiler.
%
% %%%%%%%%%%%%%%%%%%%%%%%%%%%%%%%%%%%%%%
% \paragraph{Main File.}
%
% The main file is called |cdocsamp.tex|.
%
% Load the \textsf{childdoc} definitions and
% declare the filename for the main document:
%    \begin{macrocode}
\input{childdoc.def}
\childdocmain{}
%    \end{macrocode}

% Optional override for |\version| flag:
%    \begin{macrocode}
%%\ifchilddoc\else\providecommand{\version}{draft}\fi
%    \end{macrocode}

% Define the default values for the |\version| flag
% (|final| for the main file and |draft| for childs):
%    \begin{macrocode}
\ifchilddoc
\providecommand{\version}{draft}
\else
\providecommand{\version}{final}
\fi
%    \end{macrocode}

% Load the standard document class:
%    \begin{macrocode}
\documentclass[12pt]{article}
%    \end{macrocode}

% Start the document body:
%    \begin{macrocode}
\begin{document}
%    \end{macrocode}

% Declare a title page.
% Print title, part of document being processed and version flag:
%    \begin{macrocode}
\addtocounter{page}{-1}
\begin{center}
{\LARGE\bfseries{}childdoc example\par}
\vspace{1cm}
\ifchilddoc
\ifchilddocmanual part\else chapter\fi:
`\childdocname' of `\childdocjob'\par
\else
main document: `\childdocjob'\par
\fi
version: \version\par
\end{center}
\newpage
%    \end{macrocode}

% Manually include selected file,
% otherwise process as usual:
%    \begin{macrocode}
\ifchilddocmanual
\section*{part `\childdocname'}
\input{\childdocname}
\else
%    \end{macrocode}

% Include the two chapters:
%    \begin{macrocode}
\include{cdocsch1}
\include{cdocsch2}
%    \end{macrocode}

% Include the two parts unless only chapters should be displayed:
%    \begin{macrocode}
\ifchilddoc\else
\section{part three}
\input{cdocspt3}
\section{part four}
\input{cdocspt4}
\fi
%    \end{macrocode}

% Process as usual until here:
%    \begin{macrocode}
\fi
%    \end{macrocode}

% End of document body:
%    \begin{macrocode}
\end{document}
%    \end{macrocode}
%\iffalse
%</samplemain>
%\fi
%
% %%%%%%%%%%%%%%%%%%%%%%%%%%%%%%%%%%%%%%
% \paragraph{Chapter Include Files.}
%
% The include files are called |cdocsch1.tex| and |cdocsch2.tex|.
%
%\iffalse
%<*samplechap1|samplechap2>
%\fi

% Optional override for |\version| flag:
%    \begin{macrocode}
%%\providecommand{\version}{final}
%    \end{macrocode}

% Include the main document:
%    \begin{macrocode}
\input{childdoc.def}
\childdocof{cdocsamp}
%    \end{macrocode}

%\iffalse
%</samplechap1|samplechap2>
%\fi
%
%\iffalse
%<*samplechap1>
%\fi
% Some text for chapter 1:
%    \begin{macrocode}
\section{one}
some text in chapter one
%    \end{macrocode}

%\iffalse
%</samplechap1>
%\fi
% Some text for chapter 2:
%\iffalse
%<*samplechap2>
%\fi
%    \begin{macrocode}
\section{two}
more text in chapter two
%    \end{macrocode}

%\iffalse
%</samplechap2>
%\fi
%
% %%%%%%%%%%%%%%%%%%%%%%%%%%%%%%%%%%%%%%
% \paragraph{Part Include Files.}
%
% The include files are called |cdocspt3.tex| and |cdocspt4.tex|.
%
%\iffalse
%<*samplepart3|samplepart4>
%\fi

% Optional override for |\version| flag:
%    \begin{macrocode}
%%\providecommand{\version}{final}
%    \end{macrocode}

% Include the main document:
%    \begin{macrocode}
\input{childdoc.def}
\childdocby{cdocsamp}
%    \end{macrocode}

%\iffalse
%</samplepart3|samplepart4>
%\fi
%
%\iffalse
%<*samplepart3>
%\fi
% Some text for part 3:
%    \begin{macrocode}
some text in part three
%    \end{macrocode}

%\iffalse
%</samplepart3>
%\fi
% Some text for part 4:
%\iffalse
%<*samplepart4>
%\fi
%    \begin{macrocode}
more text in part four
%    \end{macrocode}

%\iffalse
%</samplepart4>
%\fi
%
% %%%%%%%%%%%%%%%%%%%%%%%%%%%%%%%%%%%%%%
% \paragraph{Forwarding for a Complete Draft.}
%
% The following forwarding file |cdocsdrf.tex|
% compiles the main document in draft mode:
%\iffalse
%<*sampledraft>
%\fi
%    \begin{macrocode}
\def\version{draft}
\input{childdoc.def}
\childdocforward{cdocsamp}
%    \end{macrocode}

%\iffalse
%</sampledraft>
%\fi
%
% %%%%%%%%%%%%%%%%%%%%%%%%%%%%%%%%%%%%%%
% \paragraph{Forwarding for Final Version of the Chapters.}
%
% The following forwarding files |cdocsfn1.tex| and |cdocsfn2.tex|
% (with identical content)
% compile the final versions of the child documents
% |cdocsch1.tex| and |cdocsch2.tex|, respectively:
%\iffalse
%<*samplefinal>
%\fi
%    \begin{macrocode}
\def\version{final}
\input{childdoc.def}
\childdocforwardprefix[cdocsamp]{cdocsfn}{cdocsch}
%    \end{macrocode}

%\iffalse
%</samplefinal>
%\fi
%
% %%%%%%%%%%%%%%%%%%%%%%%%%%%%%%%%%%%%%%
% \paragraph{Command Line Processing.}
%
% The following three command lines generate the output files
% |cdocscld|, |cdocscl1| and |cdocscl2|
% which should be identical to
% |cdocsdrf|, |cdocsch1| and |cdocsfn2|, respectively:
% \begin{center}
% \begin{tabular}{l}
% |latex -jobname cdocscld \|\\
% |  "\def\version{draft}\input{childdoc.def}\childdocforward{cdocsamp}"|\\
% |latex -jobname cdocscl1 \|\\
% |  "\input{childdoc.def}\childdocforward[cdocsamp]{cdocsch1}"|\\
% |latex -jobname cdocscl2 \|\\
% |  "\def\version{final}\input{childdoc.def}\childdocforward{cdocsch2}"|
% \end{tabular}
% \end{center}
% Note that the trailing backslash on each first line
% merely continues the input to the second line
% (for convenient cut ant paste).
% Furthermore, the command |latex| can be replaced by any
% of its alternative versions such as |pdflatex|.
%
% %%%%%%%%%%%%%%%%%%%%%%%%%%%%%%%%%%%%%%%%%%%%%%%%%%%%%%%%%%%%%%%%%%%%%%%%%%%%%%
% %%%%%%%%%%%%%%%%%%%%%%%%%%%%%%%%%%%%%%%%%%%%%%%%%%%%%%%%%%%%%%%%%%%%%%%%%%%%%%
% \section{Implementation}
%\iffalse
%<*package>
%\fi
%
% This section describes the definitions file |childdoc.def|.

% The definitions cannot be loaded using |\usepackage| or |\RequirePackage|
% which has a mechanism to prevent loading a style file more than once.
% When loading the definitions by means of |\input|
% multiple instances have to be prevented manually:
%\iffalse
%This code needs to be before the `\ProvidesFile' directive
%which is defined at the beginning of this file.
%Therefore it is also placed there and commented out here.
%</package>
%<*discard>
%\fi
%    \begin{macrocode}
\ifdefined\childdocmain\endinput\fi
%    \end{macrocode}
%\iffalse
%</discard>
%<*package>
%\fi
%
% \macro{\ifchilddoc}
% \macro{\ifchilddocmanual}
% The conditional |\ifchilddoc| tells whether a
% child (true) or main (false) document is being compiled.
% The conditional |\ifchilddocmanual| tells whether
% the |\includeonly| mechanism is used (false) or
% the selection of child files must be performed manually (true).
% The definitions initialise to false:
%    \begin{macrocode}
\newif\ifchilddoc
\newif\ifchilddocmanual
%    \end{macrocode}

% \macro{\childdocname}
% \macro{\childdocjob}
% The macro |\childdocname| stores the name of the main document
% to be compiled. The macro |\childdocjob| stores the name of
% the document on which the \LaTeX{} compiler was originally invoked.
% The content of |\jobname| cannot be compared
% to filenames specified in the source due to different catcodes.
% The following code rescans |\jobname|, stores the result
% in |\childdocname| and saves a copy in |\childdocjob|:
%    \begin{macrocode}
\edef\childdocname{\scantokens\expandafter{\jobname\noexpand}}
\let\childdocjob\childdocname
%    \end{macrocode}

% \macro{\childdocdisable}
% The macro |\childdocdisable| prevents the main file
% from being processed more than once.
% At this stage, the main document command |\childdocmain|
% is assumed to be called once again where it should do nothing.
% Any subsequent call to it should prevent
% a secondary processing of the main document
% It overwrites the forwarding commands
% |\childdocof| and |\childdocforward|
% with empty macros to prevent further inclusions of the main document:
%    \begin{macrocode}
\newcommand{\childdocdisable}
{
  \renewcommand{\childdocmain}[1]{\renewcommand{\childdocmain}[1]{\endinput}}
  \renewcommand{\childdocof}[1]{}
  \renewcommand{\childdocby}[2][]{}
  \renewcommand{\childdocforward}[2][]{}
  \renewcommand{\childdocdisable}{}
}
%    \end{macrocode}

% \macro{\childdocmain}
% The macro |\childdocmain| is to be called at the top of the main file
% with nothing or the main filename (without extension) as argument.
% First, it breaks loops.
% If the argument is not empty and does not match |\childdocname|
% (which is set by the first inclusion of |childdoc.def|),
% |\ifchilddoc| is set to true, |\includeonly| is applied to the child file
% and |\jobname| is set to the main file
% (for proper handling of |.aux| files):
%    \begin{macrocode}
\newcommand{\childdocmain}[1]
{
  \childdocdisable\childdocmain{}
  \if?#1?\else
    \begingroup
      \def\childdoctmp{#1}
      \ifx\childdoctmp\childdocname
        \def\childdoctmp{}
      \else
        \def\childdoctmp
        {
          \childdoctrue
          \includeonly{\childdocname}
          \def\childdocjob{#1}
          \def\jobname{#1}
        }
      \fi
      \expandafter
    \endgroup
    \childdoctmp
  \fi
}
%    \end{macrocode}

% \macro{\childdocof}
% The command |\childdocof| redirects
% compilation to the main file |#1|.
%    \begin{macrocode}
\newcommand{\childdocof}[1]
{
  \childdocdisable
  \childdoctrue
  \includeonly{\childdocname}
  \def\jobname{#1}
  \def\childdocjob{#1}
  \input{#1}
}
%    \end{macrocode}

% \macro{\childdocby}
% The command |\childdocby| ....
%    \begin{macrocode}
\newcommand{\childdocby}[2][]
{
  \childdocdisable
  \childdoctrue
  \childdocmanualtrue
  \if?#1?\else
    \def\jobname{#2}
  \fi
  \def\childdocjob{#2}
  \input{#2}
  \endinput
}
%    \end{macrocode}

% \macro{\childdocforward}
% The command |\childdocforward| redirects
% compilation to the main file or
% (if the optional argument is given) a child file.
% Parameters are set as if the main file
% or a child file starting with |\childdocof| was compiled.
% Then compilation is handed over to the main file:
%    \begin{macrocode}
\newcommand{\childdocforward}[2][]
{
  \begingroup
    \if?#1?
      \def\childdoctmp
      {
        \def\childdocname{#2}
        \def\childdocjob{#2}
        \def\jobname{#2}
        \input{#2}
        \endinput
      }
    \else
      \def\childdoctmp
      {
        \childdocdisable
        \def\childdocname{#2}
        \childdoctrue
        \includeonly{#2}
        \def\childdocjob{#1}
        \def\jobname{#1}
        \input{#1}
        \endinput
      }
    \fi
    \expandafter
  \endgroup
  \childdoctmp
}
%    \end{macrocode}

% \macro{\childdocforwardprefix}
% The command |\childdocforwardprefix| redirects
% compilation to the main or a child file by means of a pattern.
% The prefix |#1| in the current filename is replaced by |#2|
% and the suffix of the current filename is kept
% (it is assumed that the filename does not contain the substring `|~~~|'
% which is used as a delimiter).
% Compilation is handed over to the new file by |\childdocforward|:
%    \begin{macrocode}
\newcommand{\childdocforwardprefix}[3][]
{
  \begingroup
    \def\childdocextract #2##1~~~{\def\childdoctmp{\childdocforward[#1]{#3##1}}}
    \expandafter\childdocextract\childdocname~~~
    \expandafter
  \endgroup
  \childdoctmp
}
%    \end{macrocode}

% \macro{\childdoc}
% The deprecated macro |\childdoc| is a legacy version of |\childdocmain|:
%    \begin{macrocode}
\newcommand{\childdoc}{\childdocmain}
%    \end{macrocode}

% \macro{\childdocredirect}
% The deprecated macro |\childdocredirect| is a legacy version
% of |\childdocforward| and |\childdocforwardprefix|:
%    \begin{macrocode}
\newcommand{\childdocredirect}[2][]
{
  \begingroup
    \if?#1?
      \def\childdoctmp{\childdocforward{#2}}
    \else
      \def\childdoctmp{\childdocforwardprefix{#1}{#2}}
    \fi
    \expandafter
  \endgroup
  \childdoctmp
}
%    \end{macrocode}

%\iffalse
%</package>
%\fi
%
\endinput
|\\
|\childdocof{|\textit{main}|}|\\
\end{tabular}
\end{center}
at the top of every child file \textit{child}
which is included by |\include{|\textit{child}|}|
from within the main file
(or at least for those files to be compiled individually).
The argument \textit{main} must be the filename of the main file.

There are a couple of
considerations in setting up the main and child documents:

%%%%%%%%%%%%%%%%%%%%%%%%%%%%%%%%%%%%%%%%
\paragraph{Restrictions.}

Please note the following restrictions:
\begin{itemize}
\item
|\childdocmain| must be called with one argument \textit{main}
to ensure compatibility with earlier version of the package.
It must either be empty (|\childdocmain{}|)
or precisely match the filename of the main file in which it is specified.
See \secref{sec:detection} for further information.
\item
The filename \textit{main} must be specified without the |.tex| extension.
\item
The filename \textit{main} is case sensitive
(even in case-insensitive file systems)
due to internal string comparison.
\item
The argument \textit{main} should be fully expanded, it cannot be a macro.
\item
Subdirectories and special characters should be avoided in filenames.
\item
The command |\childdocmain{|\textit{main}|}| must be followed by a whitespace.
It should not be followed immediately by another command
or by a comment mark `|%|'.
This is because the \TeX{} parser reads the token immediately following
the argument of |\childdocmain| and puts it
at the beginning of every child section;
however, a white\-space is ignored.
\end{itemize}

%%%%%%%%%%%%%%%%%%%%%%%%%%%%%%%%%%%%%%%%
\paragraph{Content of Main File.}

It is advisable to place all content in the child files included by |\include|.
Any output contained in the main file will appear in all child documents
unless suppressed manually;
it cannot be suppressed automatically by the |\includeonly| directive
and thus should normally be avoided.
A method to include some content in the main file
by means of conditional processing is described in \secref{sec:conditional}.

%%%%%%%%%%%%%%%%%%%%%%%%%%%%%%%%%%%%%%%%
\paragraph{Page Numbering.}

When only a part of the document is compiled,
the appropriate numbering of pages
(as well as other status parameters)
is determined from the |.aux| files.
The latter contain information from previous passes.
However this information needs to propagate through
all intermediate child documents.
Therefore the page numbering in child documents may well
be inconsistent until the complete document is compiled at least once.

A useful (if unconventional) way to always ensure a consistent
page numbering is to restart the numbering in each child document
and denote the pages by `\textit{child}|.|\textit{page}'
where \textit{child} represents the chapter/section number of the child file.
This can be achieved by the command
|\numberwithin{page}{|\textit{child}|}|
of the \textsf{amsmath} package
where \textit{child} can be |chapter| or |section|
depending on the chosen structuring.
Alternatively, one can modify the macro |\thepage| appropriately
and reset the counter |page| at the start of each child file.

%%%%%%%%%%%%%%%%%%%%%%%%%%%%%%%%%%%%%%%%%%%%%%%%%%%%%%%%%%%%%%%%%%%%%%%%%%%%%%%%
\subsection{Conditional Processing}
\label{sec:conditional}

The package provides a mechanism to compile different versions
of a document. To customise the versions further some conditional processing
can come in handy to distinguish which version is being compiled.
The package provides two macros to describe the compilation context:

%%%%%%%%%%%%%%%%%%%%%%%%%%%%%%%%%%%%%%%%
\DescribeMacro{\ifchilddoc}
The conditional |\ifchilddoc| distinguishes between the compilation of
child documents and the main document:
%
\begin{center}
|\ifchilddoc |\textit{child-code}| |[|\||else |\textit{main-code}]| \||fi|
\end{center}

%%%%%%%%%%%%%%%%%%%%%%%%%%%%%%%%%%%%%%%%
\DescribeMacro{\childdocname}
\DescribeMacro{\childdocjob}
The macro |\childdocname| contains the filename (without extension)
of the main or child file being processed.
Note that |\childdocjob| will always contain the name of the main file.

%%%%%%%%%%%%%%%%%%%%%%%%%%%%%%%%%%%%%%%%
\paragraph{Title Page.}

Conditional processing can be used to include a title or banner page
in the main document when proper precautions are taken.
Importantly, the code in the main file should ensure that the page counter
(as well as other status parameters which are stored in the |.aux| files)
takes the same value after the conditional processing.
Otherwise the page numbers may take divergent values
depending on which part is compiled.

For example, a title page could be declared by:
%
\begin{center}
\begin{tabular}{l}
|\ifchilddoc\||else|\\
|\addtocounter{page}{-1}|\\
\textit{code for title page}\\
|\newpage|\\
|\||fi|
\end{tabular}
\end{center}
%
A banner page for the child documents can be generated by:
%
\begin{center}
\begin{tabular}{l}
|\ifchilddoc|\\
|\addtocounter{page}{-1}|\\
\textit{code for banner page}\\
|\newpage|\\
|\||fi|
\end{tabular}
\end{center}
%
Here one could write a message such as:
\begin{center}
|This is the part \childdocname{} of \childdocjob{}.|
\end{center}

%%%%%%%%%%%%%%%%%%%%%%%%%%%%%%%%%%%%%%%%%%%%%%%%%%%%%%%%%%%%%%%%%%%%%%%%%%%%%%%%
\subsection{Flags}
\label{sec:flags}

The package makes it easy to generate different versions
of the main or child documents.
To this end compilation flags can be defined
and assigned different default values.
They will be particularly useful in conjunction
with the forwarding mechanism described in \secref{sec:forward}.

For example, it may be useful to have a flag |\version|
which can be set to |draft| or |final|.
The document source will contain some conditional code
depending on the value of |\version|.
Suppose further, the flag should default to |final| for the main file
and to |draft| for child files
which is a natural assignment for editing the document.
This is achieved by placing the following code
in the preamble of the main document
(below the |\childdocmain| directive):
%
\begin{center}
\begin{tabular}{l}
|\ifchilddoc|\\
|\providecommand{\version}{draft}|\\
|\||else|\\
|\providecommand{\version}{final}|\\
|\||fi|
\end{tabular}
\end{center}
%
The definition by |\providecommand| makes sure
that previous definitions are not overwritten.
Further statements |\providecommand{\version}{...}|
can thus be added before the above code to override it.

For the main file, one might add a line
(between |\childdocmain| and the above block)
%
\begin{center}
|%\ifchilddoc\||else\providecommand{\version}{draft}\||fi|
\end{center}
%
which can be uncommented to produce a draft version.
Likewise one can add a line to the very top of a child file
(above the |\childdocof{|\textit{main}|}| directive)
%
\begin{center}
|%\providecommand{\version}{final}|
\end{center}
%
which can be uncommented to produce the final version of this child document.

%%%%%%%%%%%%%%%%%%%%%%%%%%%%%%%%%%%%%%%%%%%%%%%%%%%%%%%%%%%%%%%%%%%%%%%%%%%%%%%%
\subsection{Forwarding}
\label{sec:forward}

Different versions of the main or child documents
using compilation flags as described in \secref{sec:flags}
can be (permanently) stored in different files
for convenient compilation, viewing and distribution.
To this end, the package defines a command
to pass on compilation to a different file:

%%%%%%%%%%%%%%%%%%%%%%%%%%%%%%%%%%%%%%%%
\DescribeMacro{\childdocforward}
The command |\childdocforward| redirects processing to
another source file:
%
\begin{center}
\begin{tabular}{l}
|% \iffalse
%
% childdoc.dtx Copyright (C) 2017-2018 Niklas Beisert
%
% This work may be distributed and/or modified under the
% conditions of the LaTeX Project Public License, either version 1.3
% of this license or (at your option) any later version.
% The latest version of this license is in
%   http://www.latex-project.org/lppl.txt
% and version 1.3 or later is part of all distributions of LaTeX
% version 2005/12/01 or later.
%
% This work has the LPPL maintenance status `maintained'.
%
% The Current Maintainer of this work is Niklas Beisert.
%
% This work consists of the files childdoc.dtx and childdoc.ins
% and the derived files childdoc.def and cdocsamp.tex with
% cdocsch1.tex, cdocsch2.tex, cdocsdrf.tex, cdocsfn1.tex, cdocsfn2.tex.
%
%<package>\ifdefined\childdocmain\endinput\fi
%<package>\ProvidesFile{childdoc.def}[2018/12/30 v2.0 child document driver]
%<samplemain>\ProvidesFile{cdocsamp.tex}[2018/12/30 v2.0 sample for childdoc]
%<*driver>
%\ProvidesFile{childdoc.drv}[2018/12/30 v2.0 childdoc reference manual file]
\PassOptionsToClass{10pt,a4paper}{article}
\documentclass{ltxdoc}

\usepackage[margin=35mm]{geometry}
\usepackage{hyperref}
\usepackage{hyperxmp}
\usepackage[usenames]{color}

\hypersetup{colorlinks=true}
\hypersetup{pdfstartview=FitH}
\hypersetup{pdfpagemode=UseNone}
\hypersetup{pdfsource={}}
\hypersetup{pdflang={en-UK}}
\hypersetup{pdfcopyright={Copyright 2017-2018 Niklas Beisert.
  This work may be distributed and/or modified under the
  conditions of the LaTeX Project Public License, either version 1.3
  of this license or (at your option) any later version.}}
\hypersetup{pdflicenseurl={http://www.latex-project.org/lppl.txt}}
\hypersetup{pdfcontactaddress={ETH Zurich, ITP, HIT K,
  Wolfgang-Pauli-Strasse 27}}
\hypersetup{pdfcontactpostcode={8093}}
\hypersetup{pdfcontactcity={Zurich}}
\hypersetup{pdfcontactcountry={Switzerland}}
\hypersetup{pdfcontactemail={nbeisert@itp.phys.ethz.ch}}
\hypersetup{pdfcontacturl={http://people.phys.ethz.ch/\xmptilde nbeisert/}}

\newcommand{\secref}[1]{\hyperref[#1]{section \ref*{#1}}}

\parskip1ex
\parindent0pt
\let\olditemize\itemize
\def\itemize{\olditemize\parskip0pt}

\begin{document}

\title{The \textsf{childdoc} Package}
\hypersetup{pdftitle={The childdoc Package}}
\author{Niklas Beisert\\[2ex]
  Institut f\"ur Theoretische Physik\\
  Eidgen\"ossische Technische Hochschule Z\"urich\\
  Wolfgang-Pauli-Strasse 27, 8093 Z\"urich, Switzerland\\[1ex]
  \href{mailto:nbeisert@itp.phys.ethz.ch}
  {\texttt{nbeisert@itp.phys.ethz.ch}}}
\hypersetup{pdfauthor={Niklas Beisert}}
\hypersetup{pdfsubject={Manual for the LaTeX2e Package childdoc}}
\date{30 December 2018, \textsf{v2.0}}
\maketitle

\begin{abstract}\noindent
\textsf{childdoc} is a \LaTeXe{} package
that enables the direct compilation
of document sections included by |\include|
to individual files.
\end{abstract}

\begingroup
\parskip0ex
\tableofcontents
\endgroup

%%%%%%%%%%%%%%%%%%%%%%%%%%%%%%%%%%%%%%%%%%%%%%%%%%%%%%%%%%%%%%%%%%%%%%%%%%%%%%%%
%%%%%%%%%%%%%%%%%%%%%%%%%%%%%%%%%%%%%%%%%%%%%%%%%%%%%%%%%%%%%%%%%%%%%%%%%%%%%%%%
\section{Introduction}

\LaTeX{} provides a mechanism to structure a large document (such as a book)
into a main file and several child files (containing the chapters)
using the |\include| command.
This mechanism is beneficial for documents
which span hundreds of pages in order to
make the source file(s) more manageable.
Moreover, compilation can be restricted to
selected child files by means of the |\includeonly| command.
The latter feature can be used to reduce the compilation time while editing
(this was significantly more useful in the earlier days of \LaTeX{})
or to generate a smaller document which is easier to navigate.
Another application of |\includeonly| is to generate
documents consisting of selected parts of the complete document.

However, there are a few drawbacks of the plain |\include| mechanism:
\begin{itemize}
\item
The child files cannot be compiled on their own,
they can only be compiled via the main file.
A naive editing environment
(such as a text editor with an option
to have the current file processed by \LaTeX)
may require one to switch to the main file before compiling;
attempting to compile the child file produces errors.
\item
The main file must be modified (each time)
to adjust the |\includeonly| command
to the present needs. This easily leaves the main file in a messy state.
\item
The generated document will always carry the filename
of the main document. This is inconvenient if
several child files are to be compiled and
to be kept for distribution.
\end{itemize}

The present package provides a simple interface
to make child files individually compilable by \LaTeX{}.
Compiling a child file then has the same effect as compiling
the main file with an |\includeonly| command
to select the appropriate child.
Moreover the generated document will carry the name of the child
rather than the main file.
This resolves all three above issues.

This feature is meant to make the editing of books,
thesis documents and lecture notes somewhat more convenient.
However, the package can also be used efficiently for
composing a series of documents (such as exercise sheets)
which are typically distributed individually.
It then assists the author in generating the individual documents
(potentially in different versions)
as well as a document containing the collected series.
Another application is in developing style files
or other kinds of included material
where compilation of the style file could redirect
to a sample or test file.

%%%%%%%%%%%%%%%%%%%%%%%%%%%%%%%%%%%%%%%%%%%%%%%%%%%%%%%%%%%%%%%%%%%%%%%%%%%%%%%%
%%%%%%%%%%%%%%%%%%%%%%%%%%%%%%%%%%%%%%%%%%%%%%%%%%%%%%%%%%%%%%%%%%%%%%%%%%%%%%%%
\section{Usage}

First of all, the package \textsf{childdoc} is \emph{not} a standard
\LaTeXe{} |.sty| style file! Therefore it needs to be invoked in
a non-standard way.

%%%%%%%%%%%%%%%%%%%%%%%%%%%%%%%%%%%%%%%%%%%%%%%%%%%%%%%%%%%%%%%%%%%%%%%%%%%%%%%%
\subsection{Included Files}
\label{sec:include}

%%%%%%%%%%%%%%%%%%%%%%%%%%%%%%%%%%%%%%%%
\DescribeMacro{\childdocmain}
To use the package, add the commands
\begin{center}
\begin{tabular}{l}
|\input{childdoc.def}|\\
|\childdocmain{}|\\
\end{tabular}
\end{center}
at the very top of the main \LaTeX{} file,
in particular \emph{before} the |\documentclass| statement!
The argument of |\childdocmain| should be left empty
(but it must be present).

%%%%%%%%%%%%%%%%%%%%%%%%%%%%%%%%%%%%%%%%
\DescribeMacro{\childdocof}
Furthermore, add the commands
\begin{center}
\begin{tabular}{l}
|\input{childdoc.def}|\\
|\childdocof{|\textit{main}|}|\\
\end{tabular}
\end{center}
at the top of every child file \textit{child}
which is included by |\include{|\textit{child}|}|
from within the main file
(or at least for those files to be compiled individually).
The argument \textit{main} must be the filename of the main file.

There are a couple of
considerations in setting up the main and child documents:

%%%%%%%%%%%%%%%%%%%%%%%%%%%%%%%%%%%%%%%%
\paragraph{Restrictions.}

Please note the following restrictions:
\begin{itemize}
\item
|\childdocmain| must be called with one argument \textit{main}
to ensure compatibility with earlier version of the package.
It must either be empty (|\childdocmain{}|)
or precisely match the filename of the main file in which it is specified.
See \secref{sec:detection} for further information.
\item
The filename \textit{main} must be specified without the |.tex| extension.
\item
The filename \textit{main} is case sensitive
(even in case-insensitive file systems)
due to internal string comparison.
\item
The argument \textit{main} should be fully expanded, it cannot be a macro.
\item
Subdirectories and special characters should be avoided in filenames.
\item
The command |\childdocmain{|\textit{main}|}| must be followed by a whitespace.
It should not be followed immediately by another command
or by a comment mark `|%|'.
This is because the \TeX{} parser reads the token immediately following
the argument of |\childdocmain| and puts it
at the beginning of every child section;
however, a white\-space is ignored.
\end{itemize}

%%%%%%%%%%%%%%%%%%%%%%%%%%%%%%%%%%%%%%%%
\paragraph{Content of Main File.}

It is advisable to place all content in the child files included by |\include|.
Any output contained in the main file will appear in all child documents
unless suppressed manually;
it cannot be suppressed automatically by the |\includeonly| directive
and thus should normally be avoided.
A method to include some content in the main file
by means of conditional processing is described in \secref{sec:conditional}.

%%%%%%%%%%%%%%%%%%%%%%%%%%%%%%%%%%%%%%%%
\paragraph{Page Numbering.}

When only a part of the document is compiled,
the appropriate numbering of pages
(as well as other status parameters)
is determined from the |.aux| files.
The latter contain information from previous passes.
However this information needs to propagate through
all intermediate child documents.
Therefore the page numbering in child documents may well
be inconsistent until the complete document is compiled at least once.

A useful (if unconventional) way to always ensure a consistent
page numbering is to restart the numbering in each child document
and denote the pages by `\textit{child}|.|\textit{page}'
where \textit{child} represents the chapter/section number of the child file.
This can be achieved by the command
|\numberwithin{page}{|\textit{child}|}|
of the \textsf{amsmath} package
where \textit{child} can be |chapter| or |section|
depending on the chosen structuring.
Alternatively, one can modify the macro |\thepage| appropriately
and reset the counter |page| at the start of each child file.

%%%%%%%%%%%%%%%%%%%%%%%%%%%%%%%%%%%%%%%%%%%%%%%%%%%%%%%%%%%%%%%%%%%%%%%%%%%%%%%%
\subsection{Conditional Processing}
\label{sec:conditional}

The package provides a mechanism to compile different versions
of a document. To customise the versions further some conditional processing
can come in handy to distinguish which version is being compiled.
The package provides two macros to describe the compilation context:

%%%%%%%%%%%%%%%%%%%%%%%%%%%%%%%%%%%%%%%%
\DescribeMacro{\ifchilddoc}
The conditional |\ifchilddoc| distinguishes between the compilation of
child documents and the main document:
%
\begin{center}
|\ifchilddoc |\textit{child-code}| |[|\||else |\textit{main-code}]| \||fi|
\end{center}

%%%%%%%%%%%%%%%%%%%%%%%%%%%%%%%%%%%%%%%%
\DescribeMacro{\childdocname}
\DescribeMacro{\childdocjob}
The macro |\childdocname| contains the filename (without extension)
of the main or child file being processed.
Note that |\childdocjob| will always contain the name of the main file.

%%%%%%%%%%%%%%%%%%%%%%%%%%%%%%%%%%%%%%%%
\paragraph{Title Page.}

Conditional processing can be used to include a title or banner page
in the main document when proper precautions are taken.
Importantly, the code in the main file should ensure that the page counter
(as well as other status parameters which are stored in the |.aux| files)
takes the same value after the conditional processing.
Otherwise the page numbers may take divergent values
depending on which part is compiled.

For example, a title page could be declared by:
%
\begin{center}
\begin{tabular}{l}
|\ifchilddoc\||else|\\
|\addtocounter{page}{-1}|\\
\textit{code for title page}\\
|\newpage|\\
|\||fi|
\end{tabular}
\end{center}
%
A banner page for the child documents can be generated by:
%
\begin{center}
\begin{tabular}{l}
|\ifchilddoc|\\
|\addtocounter{page}{-1}|\\
\textit{code for banner page}\\
|\newpage|\\
|\||fi|
\end{tabular}
\end{center}
%
Here one could write a message such as:
\begin{center}
|This is the part \childdocname{} of \childdocjob{}.|
\end{center}

%%%%%%%%%%%%%%%%%%%%%%%%%%%%%%%%%%%%%%%%%%%%%%%%%%%%%%%%%%%%%%%%%%%%%%%%%%%%%%%%
\subsection{Flags}
\label{sec:flags}

The package makes it easy to generate different versions
of the main or child documents.
To this end compilation flags can be defined
and assigned different default values.
They will be particularly useful in conjunction
with the forwarding mechanism described in \secref{sec:forward}.

For example, it may be useful to have a flag |\version|
which can be set to |draft| or |final|.
The document source will contain some conditional code
depending on the value of |\version|.
Suppose further, the flag should default to |final| for the main file
and to |draft| for child files
which is a natural assignment for editing the document.
This is achieved by placing the following code
in the preamble of the main document
(below the |\childdocmain| directive):
%
\begin{center}
\begin{tabular}{l}
|\ifchilddoc|\\
|\providecommand{\version}{draft}|\\
|\||else|\\
|\providecommand{\version}{final}|\\
|\||fi|
\end{tabular}
\end{center}
%
The definition by |\providecommand| makes sure
that previous definitions are not overwritten.
Further statements |\providecommand{\version}{...}|
can thus be added before the above code to override it.

For the main file, one might add a line
(between |\childdocmain| and the above block)
%
\begin{center}
|%\ifchilddoc\||else\providecommand{\version}{draft}\||fi|
\end{center}
%
which can be uncommented to produce a draft version.
Likewise one can add a line to the very top of a child file
(above the |\childdocof{|\textit{main}|}| directive)
%
\begin{center}
|%\providecommand{\version}{final}|
\end{center}
%
which can be uncommented to produce the final version of this child document.

%%%%%%%%%%%%%%%%%%%%%%%%%%%%%%%%%%%%%%%%%%%%%%%%%%%%%%%%%%%%%%%%%%%%%%%%%%%%%%%%
\subsection{Forwarding}
\label{sec:forward}

Different versions of the main or child documents
using compilation flags as described in \secref{sec:flags}
can be (permanently) stored in different files
for convenient compilation, viewing and distribution.
To this end, the package defines a command
to pass on compilation to a different file:

%%%%%%%%%%%%%%%%%%%%%%%%%%%%%%%%%%%%%%%%
\DescribeMacro{\childdocforward}
The command |\childdocforward| redirects processing to
another source file:
%
\begin{center}
\begin{tabular}{l}
|\input{childdoc.def}|\\
|\childdocforward[|\textit{main}|]{|\textit{dest}|}|\\
\end{tabular}
\end{center}
%
The argument \textit{dest} is the destination file
(without extension).
It should be the main file or one of the child files.
Note that further \textsf{childdoc} directives
such as |\childdocof| and |\childdocforward|
in the indicated file will be processed in this form.
The optional argument \textit{main}
passes on directly to the main file \textit{main}
while pretending to compile the child \textit{dest}.
This form behaves as if \textit{dest}
issues |\childdocof{|\textit{main}|}| right away,
and no further \textsf{childdoc} directives will be processed.

%%%%%%%%%%%%%%%%%%%%%%%%%%%%%%%%%%%%%%%%
\DescribeMacro{\...prefix}
In the alternative form |\childdocforwardprefix|,
%
\begin{center}
\begin{tabular}{l}
|\input{childdoc.def}|\\
|\childdocforwardprefix[|\textit{main}|]{|\textit{prefix}|}{|\textit{dest}|}|
\end{tabular}
\end{center}
%
the destination file is determined by a pattern
depending on the current file:
To make this work, the current file must be called
`{\textit{prefix}\hspace{0.2em}\textit{suffix}}'
with \textit{prefix} matching precisely the argument.
Processing is then passed on to the file
`{\textit{dest}\hspace{0.2em}\textit{suffix}}'.
Surely, the same effect is achieved by
directly specifying the
argument `{\textit{dest}\hspace{0.2em}\textit{suffix}}'
in the first form.
However, that requires to set up a different file
for each child. With the alternative form of the command
all these files can have exactly the same content
which simplifies setting them up and maintaining them.

For example, the following file |draft.tex|
with a compilation flag |\version| as described in \secref{sec:flags}
compiles the main document as a draft:
%
\begin{center}
\begin{tabular}{l}
|\def\version{draft}|\\
|\input{childdoc.def}|\\
|\childdocforward{|\textit{main}|}|
\end{tabular}
\end{center}
%
Likewise, the following files |final|\textit{nn}|.tex|
compile the final version of the child document
|child|\textit{nn}|.tex|:
%
\begin{center}
\begin{tabular}{l}
|\def\version{final}|\\
|\input{childdoc.def}|\\
|\childdocforwardprefix{final}{child}|
\end{tabular}
\end{center}
%

Note that when several versions of a main file and/or of each child file
are to be generated, it may be convenient to set up a |Makefile| or
shell script to automatise the process.

%%%%%%%%%%%%%%%%%%%%%%%%%%%%%%%%%%%%%%%%%%%%%%%%%%%%%%%%%%%%%%%%%%%%%%%%%%%%%%%%
\subsection{Command Line Processing}
\label{sec:commandline}

The effect of redirection files can also be achieved by invoking
the \LaTeX{} compiler with a more elaborate command line.
Most conveniently this should be done as part
of a shell script or a |Makefile|.

When using \textsf{childdoc} in the main file, the following
command lines effectively perform a redirection
(note that depending on the shell being used,
backslashes may have to be doubled: `|\|' $\to$ `|\\|'):
%
\begin{center}
|... -jobname "|\textit{target}|" |\\|"|[\textit{flags}]%
|\input{childdoc.def}\childdocforward[|\textit{main}|]{|\textit{dest}|}"|
\end{center}
%
Here \textit{target} is the name of the output file,
\textit{main} is the name of the main file
and \textit{dest} is the name of the main or child file to be processed
(all filenames without extensions).
The optional argument \textit{main} can be omitted
if \textit{main} matches \textit{dest}.
Optionally, compilation \textit{flags} can be defined via |\def| commands.
This command line makes the \TeX{} engine believe
it is compiling the file \textit{target}
whose content is specified as the latter parameter.
The provided code then forwards the processing to
\textit{main} or \textit{dest} as described in \secref{sec:forward}.

%%%%%%%%%%%%%%%%%%%%%%%%%%%%%%%%%%%%%%%%%%%%%%%%%%%%%%%%%%%%%%%%%%%%%%%%%%%%%%%%
\subsection{Include by Input}
\label{sec:input}

Including child documents by |\include| has some restrictions by design.
Most notably, the content of a child document always occupies
its own set of pages; pages cannot be shared between child documents.
Usually, this behaviour makes perfect sense
because each child document contain an essential part of the document.
However, in some situations it may be desirable to compose
a document from a collection of parts
without having mandatory page breaks between then.
For this case, the package
provides a mechanism to include parts
by |\input| which can also be processed individually.
However, by construction this mechanism
requires manual handling of the content to be output.

%%%%%%%%%%%%%%%%%%%%%%%%%%%%%%%%%%%%%%%%
\DescribeMacro{\ifchilddocmanual}
The main file should be prepared as usual, see \secref{sec:include}.
However, the document body must make a distinction
between processing of an individual part and of the main document, e.g.:
%
\begin{center}
\begin{tabular}{l}
|\ifchilddocmanual|\\
|\input{\childdocname}|\\
|\||else|\\
\textit{document body with }|\input{|\textit{part}|}|\\
|\||fi|
\end{tabular}
\end{center}
%
The conditional |\ifchilddocmanual| is true whenever
a part to be included by |\input| is being compiled,
and the name of the part is stored in |\childdocname|.

%%%%%%%%%%%%%%%%%%%%%%%%%%%%%%%%%%%%%%%%
\DescribeMacro{\childdocby}
Each part to be included by |\input| should start with:
%
\begin{center}
\begin{tabular}{l}
|\input{childdoc.def}|\\
|\childdocby{|\textit{main}|}|\\
\end{tabular}
\end{center}
%
The directive |\childdocby| is similar to |\childdocof|
described in \secref{sec:include},
but the subsequent selection of content must be done manually.
To that end, both |\ifchilddoc| and |\ifchilddocmanual|
will be true upon processing of a part,
and the name of the part is stored in |\childdocname|.
Note that |\jobname| will be set to the filename of the current part
so that each part receives an individual |.aux| file
that does not interfere with the |.aux| file(s) of the main document.
This behaviour can be altered by the alternative form
|\childdocby[*]{|\textit{main}|}| (with a non-empty optional argument)
which uses the |.aux| file of the main document
by setting |\jobname| to \textit{main}.

%%%%%%%%%%%%%%%%%%%%%%%%%%%%%%%%%%%%%%%%%%%%%%%%%%%%%%%%%%%%%%%%%%%%%%%%%%%%%%%%
\subsection{Driver Development}
\label{sec:driver}

The \textsf{childdoc} mechanism can also be use for the development
of definition files such as \LaTeX{} styles or classes.
This case differs from the above setup with multiple parts
included by |\include| in that no |\includeonly| should be invoked.
This can be achieved by starting the include file
(before |\ProvidesPackage|) with:
%
\begin{center}
\begin{tabular}{l}
|\input{childdoc.def}|\\
|\childdocforward{|\textit{main}|}|\\
\end{tabular}
\end{center}
%
or alternatively with:
%
\begin{center}
\begin{tabular}{l}
|\input{childdoc.def}|\\
|\childdocby{|\textit{main}|}|\\
\end{tabular}
\end{center}
%
Both forms have slightly different effects as described above.
The main file is prepared as usual, see \secref{sec:include}.

%%%%%%%%%%%%%%%%%%%%%%%%%%%%%%%%%%%%%%%%%%%%%%%%%%%%%%%%%%%%%%%%%%%%%%%%%%%%%%%%
\subsection{Legacy Detection}
\label{sec:detection}

The directive |\childdocmain| in the main file can detect
whether the complete document or merely a child is to be compiled
even without using the directive |\childdocof|.
This method is deprecated because it is less robust
and there is no compelling reason to use it;
it is merely provided for backward compatibility
and it may be removed in future versions.

If the detection mechanism is to be used,
it is mandatory to correctly specify
the filename of the main file as the argument of |\childdocmain|:
%
\begin{center}
\begin{tabular}{l}
|\input{childdoc.def}|\\
|\childdocmain{|\textit{main}|}|\\
\end{tabular}
\end{center}
%
If |\jobname| does not match the argument \textit{main} of |\childdocmain|,
it is assumed that |\jobname| points to the child file to be compiled.
When using |\childdocmain| with the main file specified as argument,
it suffices to start a child file
with just |\input{|\textit{main}|}|
without loading of the package and using |\childdocof|.
If instead all processing is done
with the appropriate \textsf{childdoc} directives,
the argument of \textit{main} of |\childdocmain| can be empty.

An alternative version of the command line processing described
in \secref{sec:commandline} using the detection mechanism reads:
%
\begin{center}
|... -jobname "|\textit{target}|" "|[\textit{flags}]%
[|\def\jobname{|\textit{dest}|}|]|\input{|\textit{main}|}"|
\end{center}

%%%%%%%%%%%%%%%%%%%%%%%%%%%%%%%%%%%%%%%%%%%%%%%%%%%%%%%%%%%%%%%%%%%%%%%%%%%%%%%%
\subsection{Manual Code}
\label{sec:manual}

In case one cannot be certain whether the definitions file |childdoc.def|
is installed on the target \TeX{} distribution
and one prefers not to ship it,
it is conceivable to paste a few relevant commands into the sources.

To that end, drop all statements |\input{childdoc.def}|
and perform the replacements as outlined below.
Instead of |\childdocmain{|\textit{main}|}| add the following code
to the top of the main file:
%
\begin{center}
\begin{tabular}{l}
|\||ifdefined\childdocname\endinput\||fi\newif\ifchilddoc|\\
|\edef\childdocname{\scantokens\expandafter{\jobname\noexpand}}|\\
|\def\childdocmain{|\textit{main}|}\||ifx\childdocmain\childdocname\||else|\\
|\childdoctrue\includeonly{\childdocname}\let\jobname\childdocmain\||fi|\\
\end{tabular}
\end{center}
%
Instead of |\childdocof{|\textit{main}|}| just include the main file
at the top of each child file:
%
\begin{center}
|\input{|\textit{main}|}|
\end{center}
%
A simple redirection |\childdocforward{|\textit{dest}|}| is achieved by:
%
\begin{center}
|\def\jobname{|\textit{dest}|}\input{\jobname}|
\end{center}
%
The redirection with prefix
|\childdocforwardprefix[|\textit{prefix}|]{|\textit{dest}|}|
is accomplished by:
%
\begin{center}
\begin{tabular}{l}
|{\edef\jobname{\scantokens\expandafter{\jobname\noexpand}}|\\
|\def\redirectjob |\textit{prefix}|#1~~~{\gdef\jobname{|\textit{dest}|#1}}|\\
|\expandafter\redirectjob\jobname~~~}\input{\jobname}|
\end{tabular}
\end{center}

In an alternative approach,
child documents can be compiled by a specific command line
without additional code or specific definitions:
%
\begin{center}
|... -jobname "|\textit{target}|" "|[\textit{flags}]%
|\includeonly{|\textit{dest}|}\input{|\textit{main}|}"|
\end{center}
%

%%%%%%%%%%%%%%%%%%%%%%%%%%%%%%%%%%%%%%%%%%%%%%%%%%%%%%%%%%%%%%%%%%%%%%%%%%%%%%%%
%%%%%%%%%%%%%%%%%%%%%%%%%%%%%%%%%%%%%%%%%%%%%%%%%%%%%%%%%%%%%%%%%%%%%%%%%%%%%%%%
\section{Information}

%%%%%%%%%%%%%%%%%%%%%%%%%%%%%%%%%%%%%%%%%%%%%%%%%%%%%%%%%%%%%%%%%%%%%%%%%%%%%%%%
\subsection{Copyright}

Copyright \copyright{} 2017--2018 Niklas Beisert

This work may be distributed and/or modified under the
conditions of the \LaTeX{} Project Public License, either version 1.3
of this license or (at your option) any later version.
The latest version of this license is in
  \url{http://www.latex-project.org/lppl.txt}
and version 1.3 or later is part of all distributions of \LaTeX{}
version 2005/12/01 or later.

This work has the LPPL maintenance status `maintained'.

The Current Maintainer of this work is Niklas Beisert.

This work consists of the files |README.txt|, |childdoc.ins| and |childdoc.dtx|
as well as the derived files |childdoc.def|, |cdocsamp.tex|
with |cdocsch1.tex|, |cdocsch2.tex|, |cdocspt3.tex|, |cdocspt4.tex|,
|cdocsdrf.tex|, |cdocsfn1.tex|, |cdocsfn2.tex|
as well as |childdoc.pdf|.

%%%%%%%%%%%%%%%%%%%%%%%%%%%%%%%%%%%%%%%%%%%%%%%%%%%%%%%%%%%%%%%%%%%%%%%%%%%%%%%%
\subsection{Files and Installation}

The package consists of the files:
%
\begin{center}
\begin{tabular}{ll}
    |README.txt|   & readme file \\
    |childdoc.ins| & installation file \\
    |childdoc.dtx| & source file \\
    |childdoc.def| & definition file \\
    |cdocsamp.tex| & sample main file \\
    |cdocsch1.tex| & sample include file \\
    |cdocsch2.tex| & sample include file \\
    |cdocspt3.tex| & sample part file \\
    |cdocspt4.tex| & sample part file \\
    |cdocsdrf.tex| & sample redirection file \\
    |cdocsfn1.tex| & sample redirection file \\
    |cdocsfn2.tex| & sample redirection file \\
    |childdoc.pdf| & manual
\end{tabular}
\end{center}
%
The distribution consists of the files
|README.txt|, |childdoc.ins| and |childdoc.dtx|.
%
\begin{itemize}
\item
Run (pdf)\LaTeX{} on |childdoc.dtx|
to compile the manual |childdoc.pdf| (this file).
\item
Run \LaTeX{} on |childdoc.ins| to create the definitions file |childdoc.def|
and the sample |cdocsamp.tex| with include files
|cdocsch1.tex|, |cdocsch2.tex|, |cdocspt3.tex|, |cdocspt4.tex|,
|cdocsdrf.tex|, |cdocsfn1.tex|, |cdocsfn2.tex|.
Then copy the file |childdoc.def| to an appropriate directory of your \LaTeX{}
distribution, e.g.\ \textit{texmf-root}|/tex/latex/childdoc|.
\end{itemize}

%%%%%%%%%%%%%%%%%%%%%%%%%%%%%%%%%%%%%%%%%%%%%%%%%%%%%%%%%%%%%%%%%%%%%%%%%%%%%%%%
\subsection{Related CTAN Packages}

There are several other packages which offer a similar functionality:
%
\begin{itemize}
\item
The packages
\href{http://ctan.org/pkg/docmute}{\textsf{docmute}},
\href{http://ctan.org/pkg/includex}{\textsf{includex}} and
\href{http://ctan.org/pkg/standalone}{\textsf{standalone}}
provide commands to include only the document body of
a child file thus allowing both files to be compiled individually.
\item
The packages \href{http://ctan.org/pkg/subdocs}{\textsf{subdocs}}
and \href{http://ctan.org/pkg/subfiles}{\textsf{subfiles}}
provide structures in which the main and child documents can be
encapsulated and allowing them to be compiled individually.
The inclusion mechanism is different from the conventional |\include|.
\item
The package \href{http://ctan.org/pkg/combine}{\textsf{combine}}
is an elaborate solution to combine several documents into one.
\end{itemize}
%
See also the CTAN topic \href{http://ctan.org/topic/subdocs}{\textsf{subdocs}}
for further related packages.
The present package differs from the above solutions in that
a document structure constructed with the conventional |\include| mechanism
just needs two extra commands at the top of every file
such that all constituent files can be compiled individually.

%%%%%%%%%%%%%%%%%%%%%%%%%%%%%%%%%%%%%%%%%%%%%%%%%%%%%%%%%%%%%%%%%%%%%%%%%%%%%%%%
%\subsection{Feature Suggestions}
%
%The following is a list of features which may be useful for future
%versions of this package:
%%
%\begin{itemize}
%\item
%\ldots
%\end{itemize}

%%%%%%%%%%%%%%%%%%%%%%%%%%%%%%%%%%%%%%%%%%%%%%%%%%%%%%%%%%%%%%%%%%%%%%%%%%%%%%%%
\subsection{Revision History}

%%%%%%%%%%%%%%%%%%%%%%%%%%%%%%%%%%%%%%%%
\paragraph{v2.0:} 2018/12/30

\begin{itemize}
\item
immediate forward processing
\item
added |\childdocby| mechanism
\item
manual restructured
\end{itemize}

%%%%%%%%%%%%%%%%%%%%%%%%%%%%%%%%%%%%%%%%
\paragraph{v1.6:} 2018/01/17

\begin{itemize}
\item
application for development of include files
\item
corrections to manual
\end{itemize}

%%%%%%%%%%%%%%%%%%%%%%%%%%%%%%%%%%%%%%%%
\paragraph{v1.5:} 2017/05/21

\begin{itemize}
\item
more complete structuring introduced
\item
|\childdocof| introduced
\item
|\childdoc| renamed to |\childdocmain|
\item
|\childredirect| renamed to |\childdocforward| and |\childdocforwardprefix|
and functionality expanded
\end{itemize}

%%%%%%%%%%%%%%%%%%%%%%%%%%%%%%%%%%%%%%%%
\paragraph{v1.0:} 2017/04/27

\begin{itemize}
\item
manual and install package
\item
first version published on CTAN
\end{itemize}

%%%%%%%%%%%%%%%%%%%%%%%%%%%%%%%%%%%%%%%%
\paragraph{v0.6:} 2017/04/26

\begin{itemize}
\item
redirection mechanism added
\end{itemize}

%%%%%%%%%%%%%%%%%%%%%%%%%%%%%%%%%%%%%%%%
\paragraph{v0.5:} 2017/04/26

\begin{itemize}
\item
functionality in definition file
\end{itemize}


%%%%%%%%%%%%%%%%%%%%%%%%%%%%%%%%%%%%%%%%%%%%%%%%%%%%%%%%%%%%%%%%%%%%%%%%%%%%%%%%
%%%%%%%%%%%%%%%%%%%%%%%%%%%%%%%%%%%%%%%%%%%%%%%%%%%%%%%%%%%%%%%%%%%%%%%%%%%%%%%%
%%%%%%%%%%%%%%%%%%%%%%%%%%%%%%%%%%%%%%%%%%%%%%%%%%%%%%%%%%%%%%%%%%%%%%%%%%%%%%%%
\appendix

\settowidth\MacroIndent{\rmfamily\scriptsize 000\ }

 \DocInput{childdoc.dtx}

\end{document}
%</driver>
% \fi
%
% %%%%%%%%%%%%%%%%%%%%%%%%%%%%%%%%%%%%%%%%%%%%%%%%%%%%%%%%%%%%%%%%%%%%%%%%%%%%%%
% %%%%%%%%%%%%%%%%%%%%%%%%%%%%%%%%%%%%%%%%%%%%%%%%%%%%%%%%%%%%%%%%%%%%%%%%%%%%%%
% \section{Sample}
%\iffalse
%<*samplemain>
%\fi
%
% The following presents a sample document
% with two chapters, two parts, a title page,
% a compile flag as well as three forwarding files to set the flag.
% It consists of eight |.tex| files:
% \begin{center}
% \begin{tabular}{ll}
% |cdocsamp.tex|&main file\\
% |cdocsch1.tex|&include file for chapter 1\\
% |cdocsch2.tex|&include file for chapter 2\\
% |cdocspt3.tex|&include file for part 3\\
% |cdocspt4.tex|&include file for part 4\\
% |cdocsdrf.tex|&forwarding file for main file in draft mode\\
% |cdocsfi1.tex|&forwarding file for final version of chapter 1\\
% |cdocsfi2.tex|&forwarding file for final version of chapter 2\\
% \end{tabular}
% \end{center}
% Each of the eight files can be compiled directly by the \LaTeX{} compiler.
%
% %%%%%%%%%%%%%%%%%%%%%%%%%%%%%%%%%%%%%%
% \paragraph{Main File.}
%
% The main file is called |cdocsamp.tex|.
%
% Load the \textsf{childdoc} definitions and
% declare the filename for the main document:
%    \begin{macrocode}
\input{childdoc.def}
\childdocmain{}
%    \end{macrocode}

% Optional override for |\version| flag:
%    \begin{macrocode}
%%\ifchilddoc\else\providecommand{\version}{draft}\fi
%    \end{macrocode}

% Define the default values for the |\version| flag
% (|final| for the main file and |draft| for childs):
%    \begin{macrocode}
\ifchilddoc
\providecommand{\version}{draft}
\else
\providecommand{\version}{final}
\fi
%    \end{macrocode}

% Load the standard document class:
%    \begin{macrocode}
\documentclass[12pt]{article}
%    \end{macrocode}

% Start the document body:
%    \begin{macrocode}
\begin{document}
%    \end{macrocode}

% Declare a title page.
% Print title, part of document being processed and version flag:
%    \begin{macrocode}
\addtocounter{page}{-1}
\begin{center}
{\LARGE\bfseries{}childdoc example\par}
\vspace{1cm}
\ifchilddoc
\ifchilddocmanual part\else chapter\fi:
`\childdocname' of `\childdocjob'\par
\else
main document: `\childdocjob'\par
\fi
version: \version\par
\end{center}
\newpage
%    \end{macrocode}

% Manually include selected file,
% otherwise process as usual:
%    \begin{macrocode}
\ifchilddocmanual
\section*{part `\childdocname'}
\input{\childdocname}
\else
%    \end{macrocode}

% Include the two chapters:
%    \begin{macrocode}
\include{cdocsch1}
\include{cdocsch2}
%    \end{macrocode}

% Include the two parts unless only chapters should be displayed:
%    \begin{macrocode}
\ifchilddoc\else
\section{part three}
\input{cdocspt3}
\section{part four}
\input{cdocspt4}
\fi
%    \end{macrocode}

% Process as usual until here:
%    \begin{macrocode}
\fi
%    \end{macrocode}

% End of document body:
%    \begin{macrocode}
\end{document}
%    \end{macrocode}
%\iffalse
%</samplemain>
%\fi
%
% %%%%%%%%%%%%%%%%%%%%%%%%%%%%%%%%%%%%%%
% \paragraph{Chapter Include Files.}
%
% The include files are called |cdocsch1.tex| and |cdocsch2.tex|.
%
%\iffalse
%<*samplechap1|samplechap2>
%\fi

% Optional override for |\version| flag:
%    \begin{macrocode}
%%\providecommand{\version}{final}
%    \end{macrocode}

% Include the main document:
%    \begin{macrocode}
\input{childdoc.def}
\childdocof{cdocsamp}
%    \end{macrocode}

%\iffalse
%</samplechap1|samplechap2>
%\fi
%
%\iffalse
%<*samplechap1>
%\fi
% Some text for chapter 1:
%    \begin{macrocode}
\section{one}
some text in chapter one
%    \end{macrocode}

%\iffalse
%</samplechap1>
%\fi
% Some text for chapter 2:
%\iffalse
%<*samplechap2>
%\fi
%    \begin{macrocode}
\section{two}
more text in chapter two
%    \end{macrocode}

%\iffalse
%</samplechap2>
%\fi
%
% %%%%%%%%%%%%%%%%%%%%%%%%%%%%%%%%%%%%%%
% \paragraph{Part Include Files.}
%
% The include files are called |cdocspt3.tex| and |cdocspt4.tex|.
%
%\iffalse
%<*samplepart3|samplepart4>
%\fi

% Optional override for |\version| flag:
%    \begin{macrocode}
%%\providecommand{\version}{final}
%    \end{macrocode}

% Include the main document:
%    \begin{macrocode}
\input{childdoc.def}
\childdocby{cdocsamp}
%    \end{macrocode}

%\iffalse
%</samplepart3|samplepart4>
%\fi
%
%\iffalse
%<*samplepart3>
%\fi
% Some text for part 3:
%    \begin{macrocode}
some text in part three
%    \end{macrocode}

%\iffalse
%</samplepart3>
%\fi
% Some text for part 4:
%\iffalse
%<*samplepart4>
%\fi
%    \begin{macrocode}
more text in part four
%    \end{macrocode}

%\iffalse
%</samplepart4>
%\fi
%
% %%%%%%%%%%%%%%%%%%%%%%%%%%%%%%%%%%%%%%
% \paragraph{Forwarding for a Complete Draft.}
%
% The following forwarding file |cdocsdrf.tex|
% compiles the main document in draft mode:
%\iffalse
%<*sampledraft>
%\fi
%    \begin{macrocode}
\def\version{draft}
\input{childdoc.def}
\childdocforward{cdocsamp}
%    \end{macrocode}

%\iffalse
%</sampledraft>
%\fi
%
% %%%%%%%%%%%%%%%%%%%%%%%%%%%%%%%%%%%%%%
% \paragraph{Forwarding for Final Version of the Chapters.}
%
% The following forwarding files |cdocsfn1.tex| and |cdocsfn2.tex|
% (with identical content)
% compile the final versions of the child documents
% |cdocsch1.tex| and |cdocsch2.tex|, respectively:
%\iffalse
%<*samplefinal>
%\fi
%    \begin{macrocode}
\def\version{final}
\input{childdoc.def}
\childdocforwardprefix[cdocsamp]{cdocsfn}{cdocsch}
%    \end{macrocode}

%\iffalse
%</samplefinal>
%\fi
%
% %%%%%%%%%%%%%%%%%%%%%%%%%%%%%%%%%%%%%%
% \paragraph{Command Line Processing.}
%
% The following three command lines generate the output files
% |cdocscld|, |cdocscl1| and |cdocscl2|
% which should be identical to
% |cdocsdrf|, |cdocsch1| and |cdocsfn2|, respectively:
% \begin{center}
% \begin{tabular}{l}
% |latex -jobname cdocscld \|\\
% |  "\def\version{draft}\input{childdoc.def}\childdocforward{cdocsamp}"|\\
% |latex -jobname cdocscl1 \|\\
% |  "\input{childdoc.def}\childdocforward[cdocsamp]{cdocsch1}"|\\
% |latex -jobname cdocscl2 \|\\
% |  "\def\version{final}\input{childdoc.def}\childdocforward{cdocsch2}"|
% \end{tabular}
% \end{center}
% Note that the trailing backslash on each first line
% merely continues the input to the second line
% (for convenient cut ant paste).
% Furthermore, the command |latex| can be replaced by any
% of its alternative versions such as |pdflatex|.
%
% %%%%%%%%%%%%%%%%%%%%%%%%%%%%%%%%%%%%%%%%%%%%%%%%%%%%%%%%%%%%%%%%%%%%%%%%%%%%%%
% %%%%%%%%%%%%%%%%%%%%%%%%%%%%%%%%%%%%%%%%%%%%%%%%%%%%%%%%%%%%%%%%%%%%%%%%%%%%%%
% \section{Implementation}
%\iffalse
%<*package>
%\fi
%
% This section describes the definitions file |childdoc.def|.

% The definitions cannot be loaded using |\usepackage| or |\RequirePackage|
% which has a mechanism to prevent loading a style file more than once.
% When loading the definitions by means of |\input|
% multiple instances have to be prevented manually:
%\iffalse
%This code needs to be before the `\ProvidesFile' directive
%which is defined at the beginning of this file.
%Therefore it is also placed there and commented out here.
%</package>
%<*discard>
%\fi
%    \begin{macrocode}
\ifdefined\childdocmain\endinput\fi
%    \end{macrocode}
%\iffalse
%</discard>
%<*package>
%\fi
%
% \macro{\ifchilddoc}
% \macro{\ifchilddocmanual}
% The conditional |\ifchilddoc| tells whether a
% child (true) or main (false) document is being compiled.
% The conditional |\ifchilddocmanual| tells whether
% the |\includeonly| mechanism is used (false) or
% the selection of child files must be performed manually (true).
% The definitions initialise to false:
%    \begin{macrocode}
\newif\ifchilddoc
\newif\ifchilddocmanual
%    \end{macrocode}

% \macro{\childdocname}
% \macro{\childdocjob}
% The macro |\childdocname| stores the name of the main document
% to be compiled. The macro |\childdocjob| stores the name of
% the document on which the \LaTeX{} compiler was originally invoked.
% The content of |\jobname| cannot be compared
% to filenames specified in the source due to different catcodes.
% The following code rescans |\jobname|, stores the result
% in |\childdocname| and saves a copy in |\childdocjob|:
%    \begin{macrocode}
\edef\childdocname{\scantokens\expandafter{\jobname\noexpand}}
\let\childdocjob\childdocname
%    \end{macrocode}

% \macro{\childdocdisable}
% The macro |\childdocdisable| prevents the main file
% from being processed more than once.
% At this stage, the main document command |\childdocmain|
% is assumed to be called once again where it should do nothing.
% Any subsequent call to it should prevent
% a secondary processing of the main document
% It overwrites the forwarding commands
% |\childdocof| and |\childdocforward|
% with empty macros to prevent further inclusions of the main document:
%    \begin{macrocode}
\newcommand{\childdocdisable}
{
  \renewcommand{\childdocmain}[1]{\renewcommand{\childdocmain}[1]{\endinput}}
  \renewcommand{\childdocof}[1]{}
  \renewcommand{\childdocby}[2][]{}
  \renewcommand{\childdocforward}[2][]{}
  \renewcommand{\childdocdisable}{}
}
%    \end{macrocode}

% \macro{\childdocmain}
% The macro |\childdocmain| is to be called at the top of the main file
% with nothing or the main filename (without extension) as argument.
% First, it breaks loops.
% If the argument is not empty and does not match |\childdocname|
% (which is set by the first inclusion of |childdoc.def|),
% |\ifchilddoc| is set to true, |\includeonly| is applied to the child file
% and |\jobname| is set to the main file
% (for proper handling of |.aux| files):
%    \begin{macrocode}
\newcommand{\childdocmain}[1]
{
  \childdocdisable\childdocmain{}
  \if?#1?\else
    \begingroup
      \def\childdoctmp{#1}
      \ifx\childdoctmp\childdocname
        \def\childdoctmp{}
      \else
        \def\childdoctmp
        {
          \childdoctrue
          \includeonly{\childdocname}
          \def\childdocjob{#1}
          \def\jobname{#1}
        }
      \fi
      \expandafter
    \endgroup
    \childdoctmp
  \fi
}
%    \end{macrocode}

% \macro{\childdocof}
% The command |\childdocof| redirects
% compilation to the main file |#1|.
%    \begin{macrocode}
\newcommand{\childdocof}[1]
{
  \childdocdisable
  \childdoctrue
  \includeonly{\childdocname}
  \def\jobname{#1}
  \def\childdocjob{#1}
  \input{#1}
}
%    \end{macrocode}

% \macro{\childdocby}
% The command |\childdocby| ....
%    \begin{macrocode}
\newcommand{\childdocby}[2][]
{
  \childdocdisable
  \childdoctrue
  \childdocmanualtrue
  \if?#1?\else
    \def\jobname{#2}
  \fi
  \def\childdocjob{#2}
  \input{#2}
  \endinput
}
%    \end{macrocode}

% \macro{\childdocforward}
% The command |\childdocforward| redirects
% compilation to the main file or
% (if the optional argument is given) a child file.
% Parameters are set as if the main file
% or a child file starting with |\childdocof| was compiled.
% Then compilation is handed over to the main file:
%    \begin{macrocode}
\newcommand{\childdocforward}[2][]
{
  \begingroup
    \if?#1?
      \def\childdoctmp
      {
        \def\childdocname{#2}
        \def\childdocjob{#2}
        \def\jobname{#2}
        \input{#2}
        \endinput
      }
    \else
      \def\childdoctmp
      {
        \childdocdisable
        \def\childdocname{#2}
        \childdoctrue
        \includeonly{#2}
        \def\childdocjob{#1}
        \def\jobname{#1}
        \input{#1}
        \endinput
      }
    \fi
    \expandafter
  \endgroup
  \childdoctmp
}
%    \end{macrocode}

% \macro{\childdocforwardprefix}
% The command |\childdocforwardprefix| redirects
% compilation to the main or a child file by means of a pattern.
% The prefix |#1| in the current filename is replaced by |#2|
% and the suffix of the current filename is kept
% (it is assumed that the filename does not contain the substring `|~~~|'
% which is used as a delimiter).
% Compilation is handed over to the new file by |\childdocforward|:
%    \begin{macrocode}
\newcommand{\childdocforwardprefix}[3][]
{
  \begingroup
    \def\childdocextract #2##1~~~{\def\childdoctmp{\childdocforward[#1]{#3##1}}}
    \expandafter\childdocextract\childdocname~~~
    \expandafter
  \endgroup
  \childdoctmp
}
%    \end{macrocode}

% \macro{\childdoc}
% The deprecated macro |\childdoc| is a legacy version of |\childdocmain|:
%    \begin{macrocode}
\newcommand{\childdoc}{\childdocmain}
%    \end{macrocode}

% \macro{\childdocredirect}
% The deprecated macro |\childdocredirect| is a legacy version
% of |\childdocforward| and |\childdocforwardprefix|:
%    \begin{macrocode}
\newcommand{\childdocredirect}[2][]
{
  \begingroup
    \if?#1?
      \def\childdoctmp{\childdocforward{#2}}
    \else
      \def\childdoctmp{\childdocforwardprefix{#1}{#2}}
    \fi
    \expandafter
  \endgroup
  \childdoctmp
}
%    \end{macrocode}

%\iffalse
%</package>
%\fi
%
\endinput
|\\
|\childdocforward[|\textit{main}|]{|\textit{dest}|}|\\
\end{tabular}
\end{center}
%
The argument \textit{dest} is the destination file
(without extension).
It should be the main file or one of the child files.
Note that further \textsf{childdoc} directives
such as |\childdocof| and |\childdocforward|
in the indicated file will be processed in this form.
The optional argument \textit{main}
passes on directly to the main file \textit{main}
while pretending to compile the child \textit{dest}.
This form behaves as if \textit{dest}
issues |\childdocof{|\textit{main}|}| right away,
and no further \textsf{childdoc} directives will be processed.

%%%%%%%%%%%%%%%%%%%%%%%%%%%%%%%%%%%%%%%%
\DescribeMacro{\...prefix}
In the alternative form |\childdocforwardprefix|,
%
\begin{center}
\begin{tabular}{l}
|% \iffalse
%
% childdoc.dtx Copyright (C) 2017-2018 Niklas Beisert
%
% This work may be distributed and/or modified under the
% conditions of the LaTeX Project Public License, either version 1.3
% of this license or (at your option) any later version.
% The latest version of this license is in
%   http://www.latex-project.org/lppl.txt
% and version 1.3 or later is part of all distributions of LaTeX
% version 2005/12/01 or later.
%
% This work has the LPPL maintenance status `maintained'.
%
% The Current Maintainer of this work is Niklas Beisert.
%
% This work consists of the files childdoc.dtx and childdoc.ins
% and the derived files childdoc.def and cdocsamp.tex with
% cdocsch1.tex, cdocsch2.tex, cdocsdrf.tex, cdocsfn1.tex, cdocsfn2.tex.
%
%<package>\ifdefined\childdocmain\endinput\fi
%<package>\ProvidesFile{childdoc.def}[2018/12/30 v2.0 child document driver]
%<samplemain>\ProvidesFile{cdocsamp.tex}[2018/12/30 v2.0 sample for childdoc]
%<*driver>
%\ProvidesFile{childdoc.drv}[2018/12/30 v2.0 childdoc reference manual file]
\PassOptionsToClass{10pt,a4paper}{article}
\documentclass{ltxdoc}

\usepackage[margin=35mm]{geometry}
\usepackage{hyperref}
\usepackage{hyperxmp}
\usepackage[usenames]{color}

\hypersetup{colorlinks=true}
\hypersetup{pdfstartview=FitH}
\hypersetup{pdfpagemode=UseNone}
\hypersetup{pdfsource={}}
\hypersetup{pdflang={en-UK}}
\hypersetup{pdfcopyright={Copyright 2017-2018 Niklas Beisert.
  This work may be distributed and/or modified under the
  conditions of the LaTeX Project Public License, either version 1.3
  of this license or (at your option) any later version.}}
\hypersetup{pdflicenseurl={http://www.latex-project.org/lppl.txt}}
\hypersetup{pdfcontactaddress={ETH Zurich, ITP, HIT K,
  Wolfgang-Pauli-Strasse 27}}
\hypersetup{pdfcontactpostcode={8093}}
\hypersetup{pdfcontactcity={Zurich}}
\hypersetup{pdfcontactcountry={Switzerland}}
\hypersetup{pdfcontactemail={nbeisert@itp.phys.ethz.ch}}
\hypersetup{pdfcontacturl={http://people.phys.ethz.ch/\xmptilde nbeisert/}}

\newcommand{\secref}[1]{\hyperref[#1]{section \ref*{#1}}}

\parskip1ex
\parindent0pt
\let\olditemize\itemize
\def\itemize{\olditemize\parskip0pt}

\begin{document}

\title{The \textsf{childdoc} Package}
\hypersetup{pdftitle={The childdoc Package}}
\author{Niklas Beisert\\[2ex]
  Institut f\"ur Theoretische Physik\\
  Eidgen\"ossische Technische Hochschule Z\"urich\\
  Wolfgang-Pauli-Strasse 27, 8093 Z\"urich, Switzerland\\[1ex]
  \href{mailto:nbeisert@itp.phys.ethz.ch}
  {\texttt{nbeisert@itp.phys.ethz.ch}}}
\hypersetup{pdfauthor={Niklas Beisert}}
\hypersetup{pdfsubject={Manual for the LaTeX2e Package childdoc}}
\date{30 December 2018, \textsf{v2.0}}
\maketitle

\begin{abstract}\noindent
\textsf{childdoc} is a \LaTeXe{} package
that enables the direct compilation
of document sections included by |\include|
to individual files.
\end{abstract}

\begingroup
\parskip0ex
\tableofcontents
\endgroup

%%%%%%%%%%%%%%%%%%%%%%%%%%%%%%%%%%%%%%%%%%%%%%%%%%%%%%%%%%%%%%%%%%%%%%%%%%%%%%%%
%%%%%%%%%%%%%%%%%%%%%%%%%%%%%%%%%%%%%%%%%%%%%%%%%%%%%%%%%%%%%%%%%%%%%%%%%%%%%%%%
\section{Introduction}

\LaTeX{} provides a mechanism to structure a large document (such as a book)
into a main file and several child files (containing the chapters)
using the |\include| command.
This mechanism is beneficial for documents
which span hundreds of pages in order to
make the source file(s) more manageable.
Moreover, compilation can be restricted to
selected child files by means of the |\includeonly| command.
The latter feature can be used to reduce the compilation time while editing
(this was significantly more useful in the earlier days of \LaTeX{})
or to generate a smaller document which is easier to navigate.
Another application of |\includeonly| is to generate
documents consisting of selected parts of the complete document.

However, there are a few drawbacks of the plain |\include| mechanism:
\begin{itemize}
\item
The child files cannot be compiled on their own,
they can only be compiled via the main file.
A naive editing environment
(such as a text editor with an option
to have the current file processed by \LaTeX)
may require one to switch to the main file before compiling;
attempting to compile the child file produces errors.
\item
The main file must be modified (each time)
to adjust the |\includeonly| command
to the present needs. This easily leaves the main file in a messy state.
\item
The generated document will always carry the filename
of the main document. This is inconvenient if
several child files are to be compiled and
to be kept for distribution.
\end{itemize}

The present package provides a simple interface
to make child files individually compilable by \LaTeX{}.
Compiling a child file then has the same effect as compiling
the main file with an |\includeonly| command
to select the appropriate child.
Moreover the generated document will carry the name of the child
rather than the main file.
This resolves all three above issues.

This feature is meant to make the editing of books,
thesis documents and lecture notes somewhat more convenient.
However, the package can also be used efficiently for
composing a series of documents (such as exercise sheets)
which are typically distributed individually.
It then assists the author in generating the individual documents
(potentially in different versions)
as well as a document containing the collected series.
Another application is in developing style files
or other kinds of included material
where compilation of the style file could redirect
to a sample or test file.

%%%%%%%%%%%%%%%%%%%%%%%%%%%%%%%%%%%%%%%%%%%%%%%%%%%%%%%%%%%%%%%%%%%%%%%%%%%%%%%%
%%%%%%%%%%%%%%%%%%%%%%%%%%%%%%%%%%%%%%%%%%%%%%%%%%%%%%%%%%%%%%%%%%%%%%%%%%%%%%%%
\section{Usage}

First of all, the package \textsf{childdoc} is \emph{not} a standard
\LaTeXe{} |.sty| style file! Therefore it needs to be invoked in
a non-standard way.

%%%%%%%%%%%%%%%%%%%%%%%%%%%%%%%%%%%%%%%%%%%%%%%%%%%%%%%%%%%%%%%%%%%%%%%%%%%%%%%%
\subsection{Included Files}
\label{sec:include}

%%%%%%%%%%%%%%%%%%%%%%%%%%%%%%%%%%%%%%%%
\DescribeMacro{\childdocmain}
To use the package, add the commands
\begin{center}
\begin{tabular}{l}
|\input{childdoc.def}|\\
|\childdocmain{}|\\
\end{tabular}
\end{center}
at the very top of the main \LaTeX{} file,
in particular \emph{before} the |\documentclass| statement!
The argument of |\childdocmain| should be left empty
(but it must be present).

%%%%%%%%%%%%%%%%%%%%%%%%%%%%%%%%%%%%%%%%
\DescribeMacro{\childdocof}
Furthermore, add the commands
\begin{center}
\begin{tabular}{l}
|\input{childdoc.def}|\\
|\childdocof{|\textit{main}|}|\\
\end{tabular}
\end{center}
at the top of every child file \textit{child}
which is included by |\include{|\textit{child}|}|
from within the main file
(or at least for those files to be compiled individually).
The argument \textit{main} must be the filename of the main file.

There are a couple of
considerations in setting up the main and child documents:

%%%%%%%%%%%%%%%%%%%%%%%%%%%%%%%%%%%%%%%%
\paragraph{Restrictions.}

Please note the following restrictions:
\begin{itemize}
\item
|\childdocmain| must be called with one argument \textit{main}
to ensure compatibility with earlier version of the package.
It must either be empty (|\childdocmain{}|)
or precisely match the filename of the main file in which it is specified.
See \secref{sec:detection} for further information.
\item
The filename \textit{main} must be specified without the |.tex| extension.
\item
The filename \textit{main} is case sensitive
(even in case-insensitive file systems)
due to internal string comparison.
\item
The argument \textit{main} should be fully expanded, it cannot be a macro.
\item
Subdirectories and special characters should be avoided in filenames.
\item
The command |\childdocmain{|\textit{main}|}| must be followed by a whitespace.
It should not be followed immediately by another command
or by a comment mark `|%|'.
This is because the \TeX{} parser reads the token immediately following
the argument of |\childdocmain| and puts it
at the beginning of every child section;
however, a white\-space is ignored.
\end{itemize}

%%%%%%%%%%%%%%%%%%%%%%%%%%%%%%%%%%%%%%%%
\paragraph{Content of Main File.}

It is advisable to place all content in the child files included by |\include|.
Any output contained in the main file will appear in all child documents
unless suppressed manually;
it cannot be suppressed automatically by the |\includeonly| directive
and thus should normally be avoided.
A method to include some content in the main file
by means of conditional processing is described in \secref{sec:conditional}.

%%%%%%%%%%%%%%%%%%%%%%%%%%%%%%%%%%%%%%%%
\paragraph{Page Numbering.}

When only a part of the document is compiled,
the appropriate numbering of pages
(as well as other status parameters)
is determined from the |.aux| files.
The latter contain information from previous passes.
However this information needs to propagate through
all intermediate child documents.
Therefore the page numbering in child documents may well
be inconsistent until the complete document is compiled at least once.

A useful (if unconventional) way to always ensure a consistent
page numbering is to restart the numbering in each child document
and denote the pages by `\textit{child}|.|\textit{page}'
where \textit{child} represents the chapter/section number of the child file.
This can be achieved by the command
|\numberwithin{page}{|\textit{child}|}|
of the \textsf{amsmath} package
where \textit{child} can be |chapter| or |section|
depending on the chosen structuring.
Alternatively, one can modify the macro |\thepage| appropriately
and reset the counter |page| at the start of each child file.

%%%%%%%%%%%%%%%%%%%%%%%%%%%%%%%%%%%%%%%%%%%%%%%%%%%%%%%%%%%%%%%%%%%%%%%%%%%%%%%%
\subsection{Conditional Processing}
\label{sec:conditional}

The package provides a mechanism to compile different versions
of a document. To customise the versions further some conditional processing
can come in handy to distinguish which version is being compiled.
The package provides two macros to describe the compilation context:

%%%%%%%%%%%%%%%%%%%%%%%%%%%%%%%%%%%%%%%%
\DescribeMacro{\ifchilddoc}
The conditional |\ifchilddoc| distinguishes between the compilation of
child documents and the main document:
%
\begin{center}
|\ifchilddoc |\textit{child-code}| |[|\||else |\textit{main-code}]| \||fi|
\end{center}

%%%%%%%%%%%%%%%%%%%%%%%%%%%%%%%%%%%%%%%%
\DescribeMacro{\childdocname}
\DescribeMacro{\childdocjob}
The macro |\childdocname| contains the filename (without extension)
of the main or child file being processed.
Note that |\childdocjob| will always contain the name of the main file.

%%%%%%%%%%%%%%%%%%%%%%%%%%%%%%%%%%%%%%%%
\paragraph{Title Page.}

Conditional processing can be used to include a title or banner page
in the main document when proper precautions are taken.
Importantly, the code in the main file should ensure that the page counter
(as well as other status parameters which are stored in the |.aux| files)
takes the same value after the conditional processing.
Otherwise the page numbers may take divergent values
depending on which part is compiled.

For example, a title page could be declared by:
%
\begin{center}
\begin{tabular}{l}
|\ifchilddoc\||else|\\
|\addtocounter{page}{-1}|\\
\textit{code for title page}\\
|\newpage|\\
|\||fi|
\end{tabular}
\end{center}
%
A banner page for the child documents can be generated by:
%
\begin{center}
\begin{tabular}{l}
|\ifchilddoc|\\
|\addtocounter{page}{-1}|\\
\textit{code for banner page}\\
|\newpage|\\
|\||fi|
\end{tabular}
\end{center}
%
Here one could write a message such as:
\begin{center}
|This is the part \childdocname{} of \childdocjob{}.|
\end{center}

%%%%%%%%%%%%%%%%%%%%%%%%%%%%%%%%%%%%%%%%%%%%%%%%%%%%%%%%%%%%%%%%%%%%%%%%%%%%%%%%
\subsection{Flags}
\label{sec:flags}

The package makes it easy to generate different versions
of the main or child documents.
To this end compilation flags can be defined
and assigned different default values.
They will be particularly useful in conjunction
with the forwarding mechanism described in \secref{sec:forward}.

For example, it may be useful to have a flag |\version|
which can be set to |draft| or |final|.
The document source will contain some conditional code
depending on the value of |\version|.
Suppose further, the flag should default to |final| for the main file
and to |draft| for child files
which is a natural assignment for editing the document.
This is achieved by placing the following code
in the preamble of the main document
(below the |\childdocmain| directive):
%
\begin{center}
\begin{tabular}{l}
|\ifchilddoc|\\
|\providecommand{\version}{draft}|\\
|\||else|\\
|\providecommand{\version}{final}|\\
|\||fi|
\end{tabular}
\end{center}
%
The definition by |\providecommand| makes sure
that previous definitions are not overwritten.
Further statements |\providecommand{\version}{...}|
can thus be added before the above code to override it.

For the main file, one might add a line
(between |\childdocmain| and the above block)
%
\begin{center}
|%\ifchilddoc\||else\providecommand{\version}{draft}\||fi|
\end{center}
%
which can be uncommented to produce a draft version.
Likewise one can add a line to the very top of a child file
(above the |\childdocof{|\textit{main}|}| directive)
%
\begin{center}
|%\providecommand{\version}{final}|
\end{center}
%
which can be uncommented to produce the final version of this child document.

%%%%%%%%%%%%%%%%%%%%%%%%%%%%%%%%%%%%%%%%%%%%%%%%%%%%%%%%%%%%%%%%%%%%%%%%%%%%%%%%
\subsection{Forwarding}
\label{sec:forward}

Different versions of the main or child documents
using compilation flags as described in \secref{sec:flags}
can be (permanently) stored in different files
for convenient compilation, viewing and distribution.
To this end, the package defines a command
to pass on compilation to a different file:

%%%%%%%%%%%%%%%%%%%%%%%%%%%%%%%%%%%%%%%%
\DescribeMacro{\childdocforward}
The command |\childdocforward| redirects processing to
another source file:
%
\begin{center}
\begin{tabular}{l}
|\input{childdoc.def}|\\
|\childdocforward[|\textit{main}|]{|\textit{dest}|}|\\
\end{tabular}
\end{center}
%
The argument \textit{dest} is the destination file
(without extension).
It should be the main file or one of the child files.
Note that further \textsf{childdoc} directives
such as |\childdocof| and |\childdocforward|
in the indicated file will be processed in this form.
The optional argument \textit{main}
passes on directly to the main file \textit{main}
while pretending to compile the child \textit{dest}.
This form behaves as if \textit{dest}
issues |\childdocof{|\textit{main}|}| right away,
and no further \textsf{childdoc} directives will be processed.

%%%%%%%%%%%%%%%%%%%%%%%%%%%%%%%%%%%%%%%%
\DescribeMacro{\...prefix}
In the alternative form |\childdocforwardprefix|,
%
\begin{center}
\begin{tabular}{l}
|\input{childdoc.def}|\\
|\childdocforwardprefix[|\textit{main}|]{|\textit{prefix}|}{|\textit{dest}|}|
\end{tabular}
\end{center}
%
the destination file is determined by a pattern
depending on the current file:
To make this work, the current file must be called
`{\textit{prefix}\hspace{0.2em}\textit{suffix}}'
with \textit{prefix} matching precisely the argument.
Processing is then passed on to the file
`{\textit{dest}\hspace{0.2em}\textit{suffix}}'.
Surely, the same effect is achieved by
directly specifying the
argument `{\textit{dest}\hspace{0.2em}\textit{suffix}}'
in the first form.
However, that requires to set up a different file
for each child. With the alternative form of the command
all these files can have exactly the same content
which simplifies setting them up and maintaining them.

For example, the following file |draft.tex|
with a compilation flag |\version| as described in \secref{sec:flags}
compiles the main document as a draft:
%
\begin{center}
\begin{tabular}{l}
|\def\version{draft}|\\
|\input{childdoc.def}|\\
|\childdocforward{|\textit{main}|}|
\end{tabular}
\end{center}
%
Likewise, the following files |final|\textit{nn}|.tex|
compile the final version of the child document
|child|\textit{nn}|.tex|:
%
\begin{center}
\begin{tabular}{l}
|\def\version{final}|\\
|\input{childdoc.def}|\\
|\childdocforwardprefix{final}{child}|
\end{tabular}
\end{center}
%

Note that when several versions of a main file and/or of each child file
are to be generated, it may be convenient to set up a |Makefile| or
shell script to automatise the process.

%%%%%%%%%%%%%%%%%%%%%%%%%%%%%%%%%%%%%%%%%%%%%%%%%%%%%%%%%%%%%%%%%%%%%%%%%%%%%%%%
\subsection{Command Line Processing}
\label{sec:commandline}

The effect of redirection files can also be achieved by invoking
the \LaTeX{} compiler with a more elaborate command line.
Most conveniently this should be done as part
of a shell script or a |Makefile|.

When using \textsf{childdoc} in the main file, the following
command lines effectively perform a redirection
(note that depending on the shell being used,
backslashes may have to be doubled: `|\|' $\to$ `|\\|'):
%
\begin{center}
|... -jobname "|\textit{target}|" |\\|"|[\textit{flags}]%
|\input{childdoc.def}\childdocforward[|\textit{main}|]{|\textit{dest}|}"|
\end{center}
%
Here \textit{target} is the name of the output file,
\textit{main} is the name of the main file
and \textit{dest} is the name of the main or child file to be processed
(all filenames without extensions).
The optional argument \textit{main} can be omitted
if \textit{main} matches \textit{dest}.
Optionally, compilation \textit{flags} can be defined via |\def| commands.
This command line makes the \TeX{} engine believe
it is compiling the file \textit{target}
whose content is specified as the latter parameter.
The provided code then forwards the processing to
\textit{main} or \textit{dest} as described in \secref{sec:forward}.

%%%%%%%%%%%%%%%%%%%%%%%%%%%%%%%%%%%%%%%%%%%%%%%%%%%%%%%%%%%%%%%%%%%%%%%%%%%%%%%%
\subsection{Include by Input}
\label{sec:input}

Including child documents by |\include| has some restrictions by design.
Most notably, the content of a child document always occupies
its own set of pages; pages cannot be shared between child documents.
Usually, this behaviour makes perfect sense
because each child document contain an essential part of the document.
However, in some situations it may be desirable to compose
a document from a collection of parts
without having mandatory page breaks between then.
For this case, the package
provides a mechanism to include parts
by |\input| which can also be processed individually.
However, by construction this mechanism
requires manual handling of the content to be output.

%%%%%%%%%%%%%%%%%%%%%%%%%%%%%%%%%%%%%%%%
\DescribeMacro{\ifchilddocmanual}
The main file should be prepared as usual, see \secref{sec:include}.
However, the document body must make a distinction
between processing of an individual part and of the main document, e.g.:
%
\begin{center}
\begin{tabular}{l}
|\ifchilddocmanual|\\
|\input{\childdocname}|\\
|\||else|\\
\textit{document body with }|\input{|\textit{part}|}|\\
|\||fi|
\end{tabular}
\end{center}
%
The conditional |\ifchilddocmanual| is true whenever
a part to be included by |\input| is being compiled,
and the name of the part is stored in |\childdocname|.

%%%%%%%%%%%%%%%%%%%%%%%%%%%%%%%%%%%%%%%%
\DescribeMacro{\childdocby}
Each part to be included by |\input| should start with:
%
\begin{center}
\begin{tabular}{l}
|\input{childdoc.def}|\\
|\childdocby{|\textit{main}|}|\\
\end{tabular}
\end{center}
%
The directive |\childdocby| is similar to |\childdocof|
described in \secref{sec:include},
but the subsequent selection of content must be done manually.
To that end, both |\ifchilddoc| and |\ifchilddocmanual|
will be true upon processing of a part,
and the name of the part is stored in |\childdocname|.
Note that |\jobname| will be set to the filename of the current part
so that each part receives an individual |.aux| file
that does not interfere with the |.aux| file(s) of the main document.
This behaviour can be altered by the alternative form
|\childdocby[*]{|\textit{main}|}| (with a non-empty optional argument)
which uses the |.aux| file of the main document
by setting |\jobname| to \textit{main}.

%%%%%%%%%%%%%%%%%%%%%%%%%%%%%%%%%%%%%%%%%%%%%%%%%%%%%%%%%%%%%%%%%%%%%%%%%%%%%%%%
\subsection{Driver Development}
\label{sec:driver}

The \textsf{childdoc} mechanism can also be use for the development
of definition files such as \LaTeX{} styles or classes.
This case differs from the above setup with multiple parts
included by |\include| in that no |\includeonly| should be invoked.
This can be achieved by starting the include file
(before |\ProvidesPackage|) with:
%
\begin{center}
\begin{tabular}{l}
|\input{childdoc.def}|\\
|\childdocforward{|\textit{main}|}|\\
\end{tabular}
\end{center}
%
or alternatively with:
%
\begin{center}
\begin{tabular}{l}
|\input{childdoc.def}|\\
|\childdocby{|\textit{main}|}|\\
\end{tabular}
\end{center}
%
Both forms have slightly different effects as described above.
The main file is prepared as usual, see \secref{sec:include}.

%%%%%%%%%%%%%%%%%%%%%%%%%%%%%%%%%%%%%%%%%%%%%%%%%%%%%%%%%%%%%%%%%%%%%%%%%%%%%%%%
\subsection{Legacy Detection}
\label{sec:detection}

The directive |\childdocmain| in the main file can detect
whether the complete document or merely a child is to be compiled
even without using the directive |\childdocof|.
This method is deprecated because it is less robust
and there is no compelling reason to use it;
it is merely provided for backward compatibility
and it may be removed in future versions.

If the detection mechanism is to be used,
it is mandatory to correctly specify
the filename of the main file as the argument of |\childdocmain|:
%
\begin{center}
\begin{tabular}{l}
|\input{childdoc.def}|\\
|\childdocmain{|\textit{main}|}|\\
\end{tabular}
\end{center}
%
If |\jobname| does not match the argument \textit{main} of |\childdocmain|,
it is assumed that |\jobname| points to the child file to be compiled.
When using |\childdocmain| with the main file specified as argument,
it suffices to start a child file
with just |\input{|\textit{main}|}|
without loading of the package and using |\childdocof|.
If instead all processing is done
with the appropriate \textsf{childdoc} directives,
the argument of \textit{main} of |\childdocmain| can be empty.

An alternative version of the command line processing described
in \secref{sec:commandline} using the detection mechanism reads:
%
\begin{center}
|... -jobname "|\textit{target}|" "|[\textit{flags}]%
[|\def\jobname{|\textit{dest}|}|]|\input{|\textit{main}|}"|
\end{center}

%%%%%%%%%%%%%%%%%%%%%%%%%%%%%%%%%%%%%%%%%%%%%%%%%%%%%%%%%%%%%%%%%%%%%%%%%%%%%%%%
\subsection{Manual Code}
\label{sec:manual}

In case one cannot be certain whether the definitions file |childdoc.def|
is installed on the target \TeX{} distribution
and one prefers not to ship it,
it is conceivable to paste a few relevant commands into the sources.

To that end, drop all statements |\input{childdoc.def}|
and perform the replacements as outlined below.
Instead of |\childdocmain{|\textit{main}|}| add the following code
to the top of the main file:
%
\begin{center}
\begin{tabular}{l}
|\||ifdefined\childdocname\endinput\||fi\newif\ifchilddoc|\\
|\edef\childdocname{\scantokens\expandafter{\jobname\noexpand}}|\\
|\def\childdocmain{|\textit{main}|}\||ifx\childdocmain\childdocname\||else|\\
|\childdoctrue\includeonly{\childdocname}\let\jobname\childdocmain\||fi|\\
\end{tabular}
\end{center}
%
Instead of |\childdocof{|\textit{main}|}| just include the main file
at the top of each child file:
%
\begin{center}
|\input{|\textit{main}|}|
\end{center}
%
A simple redirection |\childdocforward{|\textit{dest}|}| is achieved by:
%
\begin{center}
|\def\jobname{|\textit{dest}|}\input{\jobname}|
\end{center}
%
The redirection with prefix
|\childdocforwardprefix[|\textit{prefix}|]{|\textit{dest}|}|
is accomplished by:
%
\begin{center}
\begin{tabular}{l}
|{\edef\jobname{\scantokens\expandafter{\jobname\noexpand}}|\\
|\def\redirectjob |\textit{prefix}|#1~~~{\gdef\jobname{|\textit{dest}|#1}}|\\
|\expandafter\redirectjob\jobname~~~}\input{\jobname}|
\end{tabular}
\end{center}

In an alternative approach,
child documents can be compiled by a specific command line
without additional code or specific definitions:
%
\begin{center}
|... -jobname "|\textit{target}|" "|[\textit{flags}]%
|\includeonly{|\textit{dest}|}\input{|\textit{main}|}"|
\end{center}
%

%%%%%%%%%%%%%%%%%%%%%%%%%%%%%%%%%%%%%%%%%%%%%%%%%%%%%%%%%%%%%%%%%%%%%%%%%%%%%%%%
%%%%%%%%%%%%%%%%%%%%%%%%%%%%%%%%%%%%%%%%%%%%%%%%%%%%%%%%%%%%%%%%%%%%%%%%%%%%%%%%
\section{Information}

%%%%%%%%%%%%%%%%%%%%%%%%%%%%%%%%%%%%%%%%%%%%%%%%%%%%%%%%%%%%%%%%%%%%%%%%%%%%%%%%
\subsection{Copyright}

Copyright \copyright{} 2017--2018 Niklas Beisert

This work may be distributed and/or modified under the
conditions of the \LaTeX{} Project Public License, either version 1.3
of this license or (at your option) any later version.
The latest version of this license is in
  \url{http://www.latex-project.org/lppl.txt}
and version 1.3 or later is part of all distributions of \LaTeX{}
version 2005/12/01 or later.

This work has the LPPL maintenance status `maintained'.

The Current Maintainer of this work is Niklas Beisert.

This work consists of the files |README.txt|, |childdoc.ins| and |childdoc.dtx|
as well as the derived files |childdoc.def|, |cdocsamp.tex|
with |cdocsch1.tex|, |cdocsch2.tex|, |cdocspt3.tex|, |cdocspt4.tex|,
|cdocsdrf.tex|, |cdocsfn1.tex|, |cdocsfn2.tex|
as well as |childdoc.pdf|.

%%%%%%%%%%%%%%%%%%%%%%%%%%%%%%%%%%%%%%%%%%%%%%%%%%%%%%%%%%%%%%%%%%%%%%%%%%%%%%%%
\subsection{Files and Installation}

The package consists of the files:
%
\begin{center}
\begin{tabular}{ll}
    |README.txt|   & readme file \\
    |childdoc.ins| & installation file \\
    |childdoc.dtx| & source file \\
    |childdoc.def| & definition file \\
    |cdocsamp.tex| & sample main file \\
    |cdocsch1.tex| & sample include file \\
    |cdocsch2.tex| & sample include file \\
    |cdocspt3.tex| & sample part file \\
    |cdocspt4.tex| & sample part file \\
    |cdocsdrf.tex| & sample redirection file \\
    |cdocsfn1.tex| & sample redirection file \\
    |cdocsfn2.tex| & sample redirection file \\
    |childdoc.pdf| & manual
\end{tabular}
\end{center}
%
The distribution consists of the files
|README.txt|, |childdoc.ins| and |childdoc.dtx|.
%
\begin{itemize}
\item
Run (pdf)\LaTeX{} on |childdoc.dtx|
to compile the manual |childdoc.pdf| (this file).
\item
Run \LaTeX{} on |childdoc.ins| to create the definitions file |childdoc.def|
and the sample |cdocsamp.tex| with include files
|cdocsch1.tex|, |cdocsch2.tex|, |cdocspt3.tex|, |cdocspt4.tex|,
|cdocsdrf.tex|, |cdocsfn1.tex|, |cdocsfn2.tex|.
Then copy the file |childdoc.def| to an appropriate directory of your \LaTeX{}
distribution, e.g.\ \textit{texmf-root}|/tex/latex/childdoc|.
\end{itemize}

%%%%%%%%%%%%%%%%%%%%%%%%%%%%%%%%%%%%%%%%%%%%%%%%%%%%%%%%%%%%%%%%%%%%%%%%%%%%%%%%
\subsection{Related CTAN Packages}

There are several other packages which offer a similar functionality:
%
\begin{itemize}
\item
The packages
\href{http://ctan.org/pkg/docmute}{\textsf{docmute}},
\href{http://ctan.org/pkg/includex}{\textsf{includex}} and
\href{http://ctan.org/pkg/standalone}{\textsf{standalone}}
provide commands to include only the document body of
a child file thus allowing both files to be compiled individually.
\item
The packages \href{http://ctan.org/pkg/subdocs}{\textsf{subdocs}}
and \href{http://ctan.org/pkg/subfiles}{\textsf{subfiles}}
provide structures in which the main and child documents can be
encapsulated and allowing them to be compiled individually.
The inclusion mechanism is different from the conventional |\include|.
\item
The package \href{http://ctan.org/pkg/combine}{\textsf{combine}}
is an elaborate solution to combine several documents into one.
\end{itemize}
%
See also the CTAN topic \href{http://ctan.org/topic/subdocs}{\textsf{subdocs}}
for further related packages.
The present package differs from the above solutions in that
a document structure constructed with the conventional |\include| mechanism
just needs two extra commands at the top of every file
such that all constituent files can be compiled individually.

%%%%%%%%%%%%%%%%%%%%%%%%%%%%%%%%%%%%%%%%%%%%%%%%%%%%%%%%%%%%%%%%%%%%%%%%%%%%%%%%
%\subsection{Feature Suggestions}
%
%The following is a list of features which may be useful for future
%versions of this package:
%%
%\begin{itemize}
%\item
%\ldots
%\end{itemize}

%%%%%%%%%%%%%%%%%%%%%%%%%%%%%%%%%%%%%%%%%%%%%%%%%%%%%%%%%%%%%%%%%%%%%%%%%%%%%%%%
\subsection{Revision History}

%%%%%%%%%%%%%%%%%%%%%%%%%%%%%%%%%%%%%%%%
\paragraph{v2.0:} 2018/12/30

\begin{itemize}
\item
immediate forward processing
\item
added |\childdocby| mechanism
\item
manual restructured
\end{itemize}

%%%%%%%%%%%%%%%%%%%%%%%%%%%%%%%%%%%%%%%%
\paragraph{v1.6:} 2018/01/17

\begin{itemize}
\item
application for development of include files
\item
corrections to manual
\end{itemize}

%%%%%%%%%%%%%%%%%%%%%%%%%%%%%%%%%%%%%%%%
\paragraph{v1.5:} 2017/05/21

\begin{itemize}
\item
more complete structuring introduced
\item
|\childdocof| introduced
\item
|\childdoc| renamed to |\childdocmain|
\item
|\childredirect| renamed to |\childdocforward| and |\childdocforwardprefix|
and functionality expanded
\end{itemize}

%%%%%%%%%%%%%%%%%%%%%%%%%%%%%%%%%%%%%%%%
\paragraph{v1.0:} 2017/04/27

\begin{itemize}
\item
manual and install package
\item
first version published on CTAN
\end{itemize}

%%%%%%%%%%%%%%%%%%%%%%%%%%%%%%%%%%%%%%%%
\paragraph{v0.6:} 2017/04/26

\begin{itemize}
\item
redirection mechanism added
\end{itemize}

%%%%%%%%%%%%%%%%%%%%%%%%%%%%%%%%%%%%%%%%
\paragraph{v0.5:} 2017/04/26

\begin{itemize}
\item
functionality in definition file
\end{itemize}


%%%%%%%%%%%%%%%%%%%%%%%%%%%%%%%%%%%%%%%%%%%%%%%%%%%%%%%%%%%%%%%%%%%%%%%%%%%%%%%%
%%%%%%%%%%%%%%%%%%%%%%%%%%%%%%%%%%%%%%%%%%%%%%%%%%%%%%%%%%%%%%%%%%%%%%%%%%%%%%%%
%%%%%%%%%%%%%%%%%%%%%%%%%%%%%%%%%%%%%%%%%%%%%%%%%%%%%%%%%%%%%%%%%%%%%%%%%%%%%%%%
\appendix

\settowidth\MacroIndent{\rmfamily\scriptsize 000\ }

 \DocInput{childdoc.dtx}

\end{document}
%</driver>
% \fi
%
% %%%%%%%%%%%%%%%%%%%%%%%%%%%%%%%%%%%%%%%%%%%%%%%%%%%%%%%%%%%%%%%%%%%%%%%%%%%%%%
% %%%%%%%%%%%%%%%%%%%%%%%%%%%%%%%%%%%%%%%%%%%%%%%%%%%%%%%%%%%%%%%%%%%%%%%%%%%%%%
% \section{Sample}
%\iffalse
%<*samplemain>
%\fi
%
% The following presents a sample document
% with two chapters, two parts, a title page,
% a compile flag as well as three forwarding files to set the flag.
% It consists of eight |.tex| files:
% \begin{center}
% \begin{tabular}{ll}
% |cdocsamp.tex|&main file\\
% |cdocsch1.tex|&include file for chapter 1\\
% |cdocsch2.tex|&include file for chapter 2\\
% |cdocspt3.tex|&include file for part 3\\
% |cdocspt4.tex|&include file for part 4\\
% |cdocsdrf.tex|&forwarding file for main file in draft mode\\
% |cdocsfi1.tex|&forwarding file for final version of chapter 1\\
% |cdocsfi2.tex|&forwarding file for final version of chapter 2\\
% \end{tabular}
% \end{center}
% Each of the eight files can be compiled directly by the \LaTeX{} compiler.
%
% %%%%%%%%%%%%%%%%%%%%%%%%%%%%%%%%%%%%%%
% \paragraph{Main File.}
%
% The main file is called |cdocsamp.tex|.
%
% Load the \textsf{childdoc} definitions and
% declare the filename for the main document:
%    \begin{macrocode}
\input{childdoc.def}
\childdocmain{}
%    \end{macrocode}

% Optional override for |\version| flag:
%    \begin{macrocode}
%%\ifchilddoc\else\providecommand{\version}{draft}\fi
%    \end{macrocode}

% Define the default values for the |\version| flag
% (|final| for the main file and |draft| for childs):
%    \begin{macrocode}
\ifchilddoc
\providecommand{\version}{draft}
\else
\providecommand{\version}{final}
\fi
%    \end{macrocode}

% Load the standard document class:
%    \begin{macrocode}
\documentclass[12pt]{article}
%    \end{macrocode}

% Start the document body:
%    \begin{macrocode}
\begin{document}
%    \end{macrocode}

% Declare a title page.
% Print title, part of document being processed and version flag:
%    \begin{macrocode}
\addtocounter{page}{-1}
\begin{center}
{\LARGE\bfseries{}childdoc example\par}
\vspace{1cm}
\ifchilddoc
\ifchilddocmanual part\else chapter\fi:
`\childdocname' of `\childdocjob'\par
\else
main document: `\childdocjob'\par
\fi
version: \version\par
\end{center}
\newpage
%    \end{macrocode}

% Manually include selected file,
% otherwise process as usual:
%    \begin{macrocode}
\ifchilddocmanual
\section*{part `\childdocname'}
\input{\childdocname}
\else
%    \end{macrocode}

% Include the two chapters:
%    \begin{macrocode}
\include{cdocsch1}
\include{cdocsch2}
%    \end{macrocode}

% Include the two parts unless only chapters should be displayed:
%    \begin{macrocode}
\ifchilddoc\else
\section{part three}
\input{cdocspt3}
\section{part four}
\input{cdocspt4}
\fi
%    \end{macrocode}

% Process as usual until here:
%    \begin{macrocode}
\fi
%    \end{macrocode}

% End of document body:
%    \begin{macrocode}
\end{document}
%    \end{macrocode}
%\iffalse
%</samplemain>
%\fi
%
% %%%%%%%%%%%%%%%%%%%%%%%%%%%%%%%%%%%%%%
% \paragraph{Chapter Include Files.}
%
% The include files are called |cdocsch1.tex| and |cdocsch2.tex|.
%
%\iffalse
%<*samplechap1|samplechap2>
%\fi

% Optional override for |\version| flag:
%    \begin{macrocode}
%%\providecommand{\version}{final}
%    \end{macrocode}

% Include the main document:
%    \begin{macrocode}
\input{childdoc.def}
\childdocof{cdocsamp}
%    \end{macrocode}

%\iffalse
%</samplechap1|samplechap2>
%\fi
%
%\iffalse
%<*samplechap1>
%\fi
% Some text for chapter 1:
%    \begin{macrocode}
\section{one}
some text in chapter one
%    \end{macrocode}

%\iffalse
%</samplechap1>
%\fi
% Some text for chapter 2:
%\iffalse
%<*samplechap2>
%\fi
%    \begin{macrocode}
\section{two}
more text in chapter two
%    \end{macrocode}

%\iffalse
%</samplechap2>
%\fi
%
% %%%%%%%%%%%%%%%%%%%%%%%%%%%%%%%%%%%%%%
% \paragraph{Part Include Files.}
%
% The include files are called |cdocspt3.tex| and |cdocspt4.tex|.
%
%\iffalse
%<*samplepart3|samplepart4>
%\fi

% Optional override for |\version| flag:
%    \begin{macrocode}
%%\providecommand{\version}{final}
%    \end{macrocode}

% Include the main document:
%    \begin{macrocode}
\input{childdoc.def}
\childdocby{cdocsamp}
%    \end{macrocode}

%\iffalse
%</samplepart3|samplepart4>
%\fi
%
%\iffalse
%<*samplepart3>
%\fi
% Some text for part 3:
%    \begin{macrocode}
some text in part three
%    \end{macrocode}

%\iffalse
%</samplepart3>
%\fi
% Some text for part 4:
%\iffalse
%<*samplepart4>
%\fi
%    \begin{macrocode}
more text in part four
%    \end{macrocode}

%\iffalse
%</samplepart4>
%\fi
%
% %%%%%%%%%%%%%%%%%%%%%%%%%%%%%%%%%%%%%%
% \paragraph{Forwarding for a Complete Draft.}
%
% The following forwarding file |cdocsdrf.tex|
% compiles the main document in draft mode:
%\iffalse
%<*sampledraft>
%\fi
%    \begin{macrocode}
\def\version{draft}
\input{childdoc.def}
\childdocforward{cdocsamp}
%    \end{macrocode}

%\iffalse
%</sampledraft>
%\fi
%
% %%%%%%%%%%%%%%%%%%%%%%%%%%%%%%%%%%%%%%
% \paragraph{Forwarding for Final Version of the Chapters.}
%
% The following forwarding files |cdocsfn1.tex| and |cdocsfn2.tex|
% (with identical content)
% compile the final versions of the child documents
% |cdocsch1.tex| and |cdocsch2.tex|, respectively:
%\iffalse
%<*samplefinal>
%\fi
%    \begin{macrocode}
\def\version{final}
\input{childdoc.def}
\childdocforwardprefix[cdocsamp]{cdocsfn}{cdocsch}
%    \end{macrocode}

%\iffalse
%</samplefinal>
%\fi
%
% %%%%%%%%%%%%%%%%%%%%%%%%%%%%%%%%%%%%%%
% \paragraph{Command Line Processing.}
%
% The following three command lines generate the output files
% |cdocscld|, |cdocscl1| and |cdocscl2|
% which should be identical to
% |cdocsdrf|, |cdocsch1| and |cdocsfn2|, respectively:
% \begin{center}
% \begin{tabular}{l}
% |latex -jobname cdocscld \|\\
% |  "\def\version{draft}\input{childdoc.def}\childdocforward{cdocsamp}"|\\
% |latex -jobname cdocscl1 \|\\
% |  "\input{childdoc.def}\childdocforward[cdocsamp]{cdocsch1}"|\\
% |latex -jobname cdocscl2 \|\\
% |  "\def\version{final}\input{childdoc.def}\childdocforward{cdocsch2}"|
% \end{tabular}
% \end{center}
% Note that the trailing backslash on each first line
% merely continues the input to the second line
% (for convenient cut ant paste).
% Furthermore, the command |latex| can be replaced by any
% of its alternative versions such as |pdflatex|.
%
% %%%%%%%%%%%%%%%%%%%%%%%%%%%%%%%%%%%%%%%%%%%%%%%%%%%%%%%%%%%%%%%%%%%%%%%%%%%%%%
% %%%%%%%%%%%%%%%%%%%%%%%%%%%%%%%%%%%%%%%%%%%%%%%%%%%%%%%%%%%%%%%%%%%%%%%%%%%%%%
% \section{Implementation}
%\iffalse
%<*package>
%\fi
%
% This section describes the definitions file |childdoc.def|.

% The definitions cannot be loaded using |\usepackage| or |\RequirePackage|
% which has a mechanism to prevent loading a style file more than once.
% When loading the definitions by means of |\input|
% multiple instances have to be prevented manually:
%\iffalse
%This code needs to be before the `\ProvidesFile' directive
%which is defined at the beginning of this file.
%Therefore it is also placed there and commented out here.
%</package>
%<*discard>
%\fi
%    \begin{macrocode}
\ifdefined\childdocmain\endinput\fi
%    \end{macrocode}
%\iffalse
%</discard>
%<*package>
%\fi
%
% \macro{\ifchilddoc}
% \macro{\ifchilddocmanual}
% The conditional |\ifchilddoc| tells whether a
% child (true) or main (false) document is being compiled.
% The conditional |\ifchilddocmanual| tells whether
% the |\includeonly| mechanism is used (false) or
% the selection of child files must be performed manually (true).
% The definitions initialise to false:
%    \begin{macrocode}
\newif\ifchilddoc
\newif\ifchilddocmanual
%    \end{macrocode}

% \macro{\childdocname}
% \macro{\childdocjob}
% The macro |\childdocname| stores the name of the main document
% to be compiled. The macro |\childdocjob| stores the name of
% the document on which the \LaTeX{} compiler was originally invoked.
% The content of |\jobname| cannot be compared
% to filenames specified in the source due to different catcodes.
% The following code rescans |\jobname|, stores the result
% in |\childdocname| and saves a copy in |\childdocjob|:
%    \begin{macrocode}
\edef\childdocname{\scantokens\expandafter{\jobname\noexpand}}
\let\childdocjob\childdocname
%    \end{macrocode}

% \macro{\childdocdisable}
% The macro |\childdocdisable| prevents the main file
% from being processed more than once.
% At this stage, the main document command |\childdocmain|
% is assumed to be called once again where it should do nothing.
% Any subsequent call to it should prevent
% a secondary processing of the main document
% It overwrites the forwarding commands
% |\childdocof| and |\childdocforward|
% with empty macros to prevent further inclusions of the main document:
%    \begin{macrocode}
\newcommand{\childdocdisable}
{
  \renewcommand{\childdocmain}[1]{\renewcommand{\childdocmain}[1]{\endinput}}
  \renewcommand{\childdocof}[1]{}
  \renewcommand{\childdocby}[2][]{}
  \renewcommand{\childdocforward}[2][]{}
  \renewcommand{\childdocdisable}{}
}
%    \end{macrocode}

% \macro{\childdocmain}
% The macro |\childdocmain| is to be called at the top of the main file
% with nothing or the main filename (without extension) as argument.
% First, it breaks loops.
% If the argument is not empty and does not match |\childdocname|
% (which is set by the first inclusion of |childdoc.def|),
% |\ifchilddoc| is set to true, |\includeonly| is applied to the child file
% and |\jobname| is set to the main file
% (for proper handling of |.aux| files):
%    \begin{macrocode}
\newcommand{\childdocmain}[1]
{
  \childdocdisable\childdocmain{}
  \if?#1?\else
    \begingroup
      \def\childdoctmp{#1}
      \ifx\childdoctmp\childdocname
        \def\childdoctmp{}
      \else
        \def\childdoctmp
        {
          \childdoctrue
          \includeonly{\childdocname}
          \def\childdocjob{#1}
          \def\jobname{#1}
        }
      \fi
      \expandafter
    \endgroup
    \childdoctmp
  \fi
}
%    \end{macrocode}

% \macro{\childdocof}
% The command |\childdocof| redirects
% compilation to the main file |#1|.
%    \begin{macrocode}
\newcommand{\childdocof}[1]
{
  \childdocdisable
  \childdoctrue
  \includeonly{\childdocname}
  \def\jobname{#1}
  \def\childdocjob{#1}
  \input{#1}
}
%    \end{macrocode}

% \macro{\childdocby}
% The command |\childdocby| ....
%    \begin{macrocode}
\newcommand{\childdocby}[2][]
{
  \childdocdisable
  \childdoctrue
  \childdocmanualtrue
  \if?#1?\else
    \def\jobname{#2}
  \fi
  \def\childdocjob{#2}
  \input{#2}
  \endinput
}
%    \end{macrocode}

% \macro{\childdocforward}
% The command |\childdocforward| redirects
% compilation to the main file or
% (if the optional argument is given) a child file.
% Parameters are set as if the main file
% or a child file starting with |\childdocof| was compiled.
% Then compilation is handed over to the main file:
%    \begin{macrocode}
\newcommand{\childdocforward}[2][]
{
  \begingroup
    \if?#1?
      \def\childdoctmp
      {
        \def\childdocname{#2}
        \def\childdocjob{#2}
        \def\jobname{#2}
        \input{#2}
        \endinput
      }
    \else
      \def\childdoctmp
      {
        \childdocdisable
        \def\childdocname{#2}
        \childdoctrue
        \includeonly{#2}
        \def\childdocjob{#1}
        \def\jobname{#1}
        \input{#1}
        \endinput
      }
    \fi
    \expandafter
  \endgroup
  \childdoctmp
}
%    \end{macrocode}

% \macro{\childdocforwardprefix}
% The command |\childdocforwardprefix| redirects
% compilation to the main or a child file by means of a pattern.
% The prefix |#1| in the current filename is replaced by |#2|
% and the suffix of the current filename is kept
% (it is assumed that the filename does not contain the substring `|~~~|'
% which is used as a delimiter).
% Compilation is handed over to the new file by |\childdocforward|:
%    \begin{macrocode}
\newcommand{\childdocforwardprefix}[3][]
{
  \begingroup
    \def\childdocextract #2##1~~~{\def\childdoctmp{\childdocforward[#1]{#3##1}}}
    \expandafter\childdocextract\childdocname~~~
    \expandafter
  \endgroup
  \childdoctmp
}
%    \end{macrocode}

% \macro{\childdoc}
% The deprecated macro |\childdoc| is a legacy version of |\childdocmain|:
%    \begin{macrocode}
\newcommand{\childdoc}{\childdocmain}
%    \end{macrocode}

% \macro{\childdocredirect}
% The deprecated macro |\childdocredirect| is a legacy version
% of |\childdocforward| and |\childdocforwardprefix|:
%    \begin{macrocode}
\newcommand{\childdocredirect}[2][]
{
  \begingroup
    \if?#1?
      \def\childdoctmp{\childdocforward{#2}}
    \else
      \def\childdoctmp{\childdocforwardprefix{#1}{#2}}
    \fi
    \expandafter
  \endgroup
  \childdoctmp
}
%    \end{macrocode}

%\iffalse
%</package>
%\fi
%
\endinput
|\\
|\childdocforwardprefix[|\textit{main}|]{|\textit{prefix}|}{|\textit{dest}|}|
\end{tabular}
\end{center}
%
the destination file is determined by a pattern
depending on the current file:
To make this work, the current file must be called
`{\textit{prefix}\hspace{0.2em}\textit{suffix}}'
with \textit{prefix} matching precisely the argument.
Processing is then passed on to the file
`{\textit{dest}\hspace{0.2em}\textit{suffix}}'.
Surely, the same effect is achieved by
directly specifying the
argument `{\textit{dest}\hspace{0.2em}\textit{suffix}}'
in the first form.
However, that requires to set up a different file
for each child. With the alternative form of the command
all these files can have exactly the same content
which simplifies setting them up and maintaining them.

For example, the following file |draft.tex|
with a compilation flag |\version| as described in \secref{sec:flags}
compiles the main document as a draft:
%
\begin{center}
\begin{tabular}{l}
|\def\version{draft}|\\
|% \iffalse
%
% childdoc.dtx Copyright (C) 2017-2018 Niklas Beisert
%
% This work may be distributed and/or modified under the
% conditions of the LaTeX Project Public License, either version 1.3
% of this license or (at your option) any later version.
% The latest version of this license is in
%   http://www.latex-project.org/lppl.txt
% and version 1.3 or later is part of all distributions of LaTeX
% version 2005/12/01 or later.
%
% This work has the LPPL maintenance status `maintained'.
%
% The Current Maintainer of this work is Niklas Beisert.
%
% This work consists of the files childdoc.dtx and childdoc.ins
% and the derived files childdoc.def and cdocsamp.tex with
% cdocsch1.tex, cdocsch2.tex, cdocsdrf.tex, cdocsfn1.tex, cdocsfn2.tex.
%
%<package>\ifdefined\childdocmain\endinput\fi
%<package>\ProvidesFile{childdoc.def}[2018/12/30 v2.0 child document driver]
%<samplemain>\ProvidesFile{cdocsamp.tex}[2018/12/30 v2.0 sample for childdoc]
%<*driver>
%\ProvidesFile{childdoc.drv}[2018/12/30 v2.0 childdoc reference manual file]
\PassOptionsToClass{10pt,a4paper}{article}
\documentclass{ltxdoc}

\usepackage[margin=35mm]{geometry}
\usepackage{hyperref}
\usepackage{hyperxmp}
\usepackage[usenames]{color}

\hypersetup{colorlinks=true}
\hypersetup{pdfstartview=FitH}
\hypersetup{pdfpagemode=UseNone}
\hypersetup{pdfsource={}}
\hypersetup{pdflang={en-UK}}
\hypersetup{pdfcopyright={Copyright 2017-2018 Niklas Beisert.
  This work may be distributed and/or modified under the
  conditions of the LaTeX Project Public License, either version 1.3
  of this license or (at your option) any later version.}}
\hypersetup{pdflicenseurl={http://www.latex-project.org/lppl.txt}}
\hypersetup{pdfcontactaddress={ETH Zurich, ITP, HIT K,
  Wolfgang-Pauli-Strasse 27}}
\hypersetup{pdfcontactpostcode={8093}}
\hypersetup{pdfcontactcity={Zurich}}
\hypersetup{pdfcontactcountry={Switzerland}}
\hypersetup{pdfcontactemail={nbeisert@itp.phys.ethz.ch}}
\hypersetup{pdfcontacturl={http://people.phys.ethz.ch/\xmptilde nbeisert/}}

\newcommand{\secref}[1]{\hyperref[#1]{section \ref*{#1}}}

\parskip1ex
\parindent0pt
\let\olditemize\itemize
\def\itemize{\olditemize\parskip0pt}

\begin{document}

\title{The \textsf{childdoc} Package}
\hypersetup{pdftitle={The childdoc Package}}
\author{Niklas Beisert\\[2ex]
  Institut f\"ur Theoretische Physik\\
  Eidgen\"ossische Technische Hochschule Z\"urich\\
  Wolfgang-Pauli-Strasse 27, 8093 Z\"urich, Switzerland\\[1ex]
  \href{mailto:nbeisert@itp.phys.ethz.ch}
  {\texttt{nbeisert@itp.phys.ethz.ch}}}
\hypersetup{pdfauthor={Niklas Beisert}}
\hypersetup{pdfsubject={Manual for the LaTeX2e Package childdoc}}
\date{30 December 2018, \textsf{v2.0}}
\maketitle

\begin{abstract}\noindent
\textsf{childdoc} is a \LaTeXe{} package
that enables the direct compilation
of document sections included by |\include|
to individual files.
\end{abstract}

\begingroup
\parskip0ex
\tableofcontents
\endgroup

%%%%%%%%%%%%%%%%%%%%%%%%%%%%%%%%%%%%%%%%%%%%%%%%%%%%%%%%%%%%%%%%%%%%%%%%%%%%%%%%
%%%%%%%%%%%%%%%%%%%%%%%%%%%%%%%%%%%%%%%%%%%%%%%%%%%%%%%%%%%%%%%%%%%%%%%%%%%%%%%%
\section{Introduction}

\LaTeX{} provides a mechanism to structure a large document (such as a book)
into a main file and several child files (containing the chapters)
using the |\include| command.
This mechanism is beneficial for documents
which span hundreds of pages in order to
make the source file(s) more manageable.
Moreover, compilation can be restricted to
selected child files by means of the |\includeonly| command.
The latter feature can be used to reduce the compilation time while editing
(this was significantly more useful in the earlier days of \LaTeX{})
or to generate a smaller document which is easier to navigate.
Another application of |\includeonly| is to generate
documents consisting of selected parts of the complete document.

However, there are a few drawbacks of the plain |\include| mechanism:
\begin{itemize}
\item
The child files cannot be compiled on their own,
they can only be compiled via the main file.
A naive editing environment
(such as a text editor with an option
to have the current file processed by \LaTeX)
may require one to switch to the main file before compiling;
attempting to compile the child file produces errors.
\item
The main file must be modified (each time)
to adjust the |\includeonly| command
to the present needs. This easily leaves the main file in a messy state.
\item
The generated document will always carry the filename
of the main document. This is inconvenient if
several child files are to be compiled and
to be kept for distribution.
\end{itemize}

The present package provides a simple interface
to make child files individually compilable by \LaTeX{}.
Compiling a child file then has the same effect as compiling
the main file with an |\includeonly| command
to select the appropriate child.
Moreover the generated document will carry the name of the child
rather than the main file.
This resolves all three above issues.

This feature is meant to make the editing of books,
thesis documents and lecture notes somewhat more convenient.
However, the package can also be used efficiently for
composing a series of documents (such as exercise sheets)
which are typically distributed individually.
It then assists the author in generating the individual documents
(potentially in different versions)
as well as a document containing the collected series.
Another application is in developing style files
or other kinds of included material
where compilation of the style file could redirect
to a sample or test file.

%%%%%%%%%%%%%%%%%%%%%%%%%%%%%%%%%%%%%%%%%%%%%%%%%%%%%%%%%%%%%%%%%%%%%%%%%%%%%%%%
%%%%%%%%%%%%%%%%%%%%%%%%%%%%%%%%%%%%%%%%%%%%%%%%%%%%%%%%%%%%%%%%%%%%%%%%%%%%%%%%
\section{Usage}

First of all, the package \textsf{childdoc} is \emph{not} a standard
\LaTeXe{} |.sty| style file! Therefore it needs to be invoked in
a non-standard way.

%%%%%%%%%%%%%%%%%%%%%%%%%%%%%%%%%%%%%%%%%%%%%%%%%%%%%%%%%%%%%%%%%%%%%%%%%%%%%%%%
\subsection{Included Files}
\label{sec:include}

%%%%%%%%%%%%%%%%%%%%%%%%%%%%%%%%%%%%%%%%
\DescribeMacro{\childdocmain}
To use the package, add the commands
\begin{center}
\begin{tabular}{l}
|\input{childdoc.def}|\\
|\childdocmain{}|\\
\end{tabular}
\end{center}
at the very top of the main \LaTeX{} file,
in particular \emph{before} the |\documentclass| statement!
The argument of |\childdocmain| should be left empty
(but it must be present).

%%%%%%%%%%%%%%%%%%%%%%%%%%%%%%%%%%%%%%%%
\DescribeMacro{\childdocof}
Furthermore, add the commands
\begin{center}
\begin{tabular}{l}
|\input{childdoc.def}|\\
|\childdocof{|\textit{main}|}|\\
\end{tabular}
\end{center}
at the top of every child file \textit{child}
which is included by |\include{|\textit{child}|}|
from within the main file
(or at least for those files to be compiled individually).
The argument \textit{main} must be the filename of the main file.

There are a couple of
considerations in setting up the main and child documents:

%%%%%%%%%%%%%%%%%%%%%%%%%%%%%%%%%%%%%%%%
\paragraph{Restrictions.}

Please note the following restrictions:
\begin{itemize}
\item
|\childdocmain| must be called with one argument \textit{main}
to ensure compatibility with earlier version of the package.
It must either be empty (|\childdocmain{}|)
or precisely match the filename of the main file in which it is specified.
See \secref{sec:detection} for further information.
\item
The filename \textit{main} must be specified without the |.tex| extension.
\item
The filename \textit{main} is case sensitive
(even in case-insensitive file systems)
due to internal string comparison.
\item
The argument \textit{main} should be fully expanded, it cannot be a macro.
\item
Subdirectories and special characters should be avoided in filenames.
\item
The command |\childdocmain{|\textit{main}|}| must be followed by a whitespace.
It should not be followed immediately by another command
or by a comment mark `|%|'.
This is because the \TeX{} parser reads the token immediately following
the argument of |\childdocmain| and puts it
at the beginning of every child section;
however, a white\-space is ignored.
\end{itemize}

%%%%%%%%%%%%%%%%%%%%%%%%%%%%%%%%%%%%%%%%
\paragraph{Content of Main File.}

It is advisable to place all content in the child files included by |\include|.
Any output contained in the main file will appear in all child documents
unless suppressed manually;
it cannot be suppressed automatically by the |\includeonly| directive
and thus should normally be avoided.
A method to include some content in the main file
by means of conditional processing is described in \secref{sec:conditional}.

%%%%%%%%%%%%%%%%%%%%%%%%%%%%%%%%%%%%%%%%
\paragraph{Page Numbering.}

When only a part of the document is compiled,
the appropriate numbering of pages
(as well as other status parameters)
is determined from the |.aux| files.
The latter contain information from previous passes.
However this information needs to propagate through
all intermediate child documents.
Therefore the page numbering in child documents may well
be inconsistent until the complete document is compiled at least once.

A useful (if unconventional) way to always ensure a consistent
page numbering is to restart the numbering in each child document
and denote the pages by `\textit{child}|.|\textit{page}'
where \textit{child} represents the chapter/section number of the child file.
This can be achieved by the command
|\numberwithin{page}{|\textit{child}|}|
of the \textsf{amsmath} package
where \textit{child} can be |chapter| or |section|
depending on the chosen structuring.
Alternatively, one can modify the macro |\thepage| appropriately
and reset the counter |page| at the start of each child file.

%%%%%%%%%%%%%%%%%%%%%%%%%%%%%%%%%%%%%%%%%%%%%%%%%%%%%%%%%%%%%%%%%%%%%%%%%%%%%%%%
\subsection{Conditional Processing}
\label{sec:conditional}

The package provides a mechanism to compile different versions
of a document. To customise the versions further some conditional processing
can come in handy to distinguish which version is being compiled.
The package provides two macros to describe the compilation context:

%%%%%%%%%%%%%%%%%%%%%%%%%%%%%%%%%%%%%%%%
\DescribeMacro{\ifchilddoc}
The conditional |\ifchilddoc| distinguishes between the compilation of
child documents and the main document:
%
\begin{center}
|\ifchilddoc |\textit{child-code}| |[|\||else |\textit{main-code}]| \||fi|
\end{center}

%%%%%%%%%%%%%%%%%%%%%%%%%%%%%%%%%%%%%%%%
\DescribeMacro{\childdocname}
\DescribeMacro{\childdocjob}
The macro |\childdocname| contains the filename (without extension)
of the main or child file being processed.
Note that |\childdocjob| will always contain the name of the main file.

%%%%%%%%%%%%%%%%%%%%%%%%%%%%%%%%%%%%%%%%
\paragraph{Title Page.}

Conditional processing can be used to include a title or banner page
in the main document when proper precautions are taken.
Importantly, the code in the main file should ensure that the page counter
(as well as other status parameters which are stored in the |.aux| files)
takes the same value after the conditional processing.
Otherwise the page numbers may take divergent values
depending on which part is compiled.

For example, a title page could be declared by:
%
\begin{center}
\begin{tabular}{l}
|\ifchilddoc\||else|\\
|\addtocounter{page}{-1}|\\
\textit{code for title page}\\
|\newpage|\\
|\||fi|
\end{tabular}
\end{center}
%
A banner page for the child documents can be generated by:
%
\begin{center}
\begin{tabular}{l}
|\ifchilddoc|\\
|\addtocounter{page}{-1}|\\
\textit{code for banner page}\\
|\newpage|\\
|\||fi|
\end{tabular}
\end{center}
%
Here one could write a message such as:
\begin{center}
|This is the part \childdocname{} of \childdocjob{}.|
\end{center}

%%%%%%%%%%%%%%%%%%%%%%%%%%%%%%%%%%%%%%%%%%%%%%%%%%%%%%%%%%%%%%%%%%%%%%%%%%%%%%%%
\subsection{Flags}
\label{sec:flags}

The package makes it easy to generate different versions
of the main or child documents.
To this end compilation flags can be defined
and assigned different default values.
They will be particularly useful in conjunction
with the forwarding mechanism described in \secref{sec:forward}.

For example, it may be useful to have a flag |\version|
which can be set to |draft| or |final|.
The document source will contain some conditional code
depending on the value of |\version|.
Suppose further, the flag should default to |final| for the main file
and to |draft| for child files
which is a natural assignment for editing the document.
This is achieved by placing the following code
in the preamble of the main document
(below the |\childdocmain| directive):
%
\begin{center}
\begin{tabular}{l}
|\ifchilddoc|\\
|\providecommand{\version}{draft}|\\
|\||else|\\
|\providecommand{\version}{final}|\\
|\||fi|
\end{tabular}
\end{center}
%
The definition by |\providecommand| makes sure
that previous definitions are not overwritten.
Further statements |\providecommand{\version}{...}|
can thus be added before the above code to override it.

For the main file, one might add a line
(between |\childdocmain| and the above block)
%
\begin{center}
|%\ifchilddoc\||else\providecommand{\version}{draft}\||fi|
\end{center}
%
which can be uncommented to produce a draft version.
Likewise one can add a line to the very top of a child file
(above the |\childdocof{|\textit{main}|}| directive)
%
\begin{center}
|%\providecommand{\version}{final}|
\end{center}
%
which can be uncommented to produce the final version of this child document.

%%%%%%%%%%%%%%%%%%%%%%%%%%%%%%%%%%%%%%%%%%%%%%%%%%%%%%%%%%%%%%%%%%%%%%%%%%%%%%%%
\subsection{Forwarding}
\label{sec:forward}

Different versions of the main or child documents
using compilation flags as described in \secref{sec:flags}
can be (permanently) stored in different files
for convenient compilation, viewing and distribution.
To this end, the package defines a command
to pass on compilation to a different file:

%%%%%%%%%%%%%%%%%%%%%%%%%%%%%%%%%%%%%%%%
\DescribeMacro{\childdocforward}
The command |\childdocforward| redirects processing to
another source file:
%
\begin{center}
\begin{tabular}{l}
|\input{childdoc.def}|\\
|\childdocforward[|\textit{main}|]{|\textit{dest}|}|\\
\end{tabular}
\end{center}
%
The argument \textit{dest} is the destination file
(without extension).
It should be the main file or one of the child files.
Note that further \textsf{childdoc} directives
such as |\childdocof| and |\childdocforward|
in the indicated file will be processed in this form.
The optional argument \textit{main}
passes on directly to the main file \textit{main}
while pretending to compile the child \textit{dest}.
This form behaves as if \textit{dest}
issues |\childdocof{|\textit{main}|}| right away,
and no further \textsf{childdoc} directives will be processed.

%%%%%%%%%%%%%%%%%%%%%%%%%%%%%%%%%%%%%%%%
\DescribeMacro{\...prefix}
In the alternative form |\childdocforwardprefix|,
%
\begin{center}
\begin{tabular}{l}
|\input{childdoc.def}|\\
|\childdocforwardprefix[|\textit{main}|]{|\textit{prefix}|}{|\textit{dest}|}|
\end{tabular}
\end{center}
%
the destination file is determined by a pattern
depending on the current file:
To make this work, the current file must be called
`{\textit{prefix}\hspace{0.2em}\textit{suffix}}'
with \textit{prefix} matching precisely the argument.
Processing is then passed on to the file
`{\textit{dest}\hspace{0.2em}\textit{suffix}}'.
Surely, the same effect is achieved by
directly specifying the
argument `{\textit{dest}\hspace{0.2em}\textit{suffix}}'
in the first form.
However, that requires to set up a different file
for each child. With the alternative form of the command
all these files can have exactly the same content
which simplifies setting them up and maintaining them.

For example, the following file |draft.tex|
with a compilation flag |\version| as described in \secref{sec:flags}
compiles the main document as a draft:
%
\begin{center}
\begin{tabular}{l}
|\def\version{draft}|\\
|\input{childdoc.def}|\\
|\childdocforward{|\textit{main}|}|
\end{tabular}
\end{center}
%
Likewise, the following files |final|\textit{nn}|.tex|
compile the final version of the child document
|child|\textit{nn}|.tex|:
%
\begin{center}
\begin{tabular}{l}
|\def\version{final}|\\
|\input{childdoc.def}|\\
|\childdocforwardprefix{final}{child}|
\end{tabular}
\end{center}
%

Note that when several versions of a main file and/or of each child file
are to be generated, it may be convenient to set up a |Makefile| or
shell script to automatise the process.

%%%%%%%%%%%%%%%%%%%%%%%%%%%%%%%%%%%%%%%%%%%%%%%%%%%%%%%%%%%%%%%%%%%%%%%%%%%%%%%%
\subsection{Command Line Processing}
\label{sec:commandline}

The effect of redirection files can also be achieved by invoking
the \LaTeX{} compiler with a more elaborate command line.
Most conveniently this should be done as part
of a shell script or a |Makefile|.

When using \textsf{childdoc} in the main file, the following
command lines effectively perform a redirection
(note that depending on the shell being used,
backslashes may have to be doubled: `|\|' $\to$ `|\\|'):
%
\begin{center}
|... -jobname "|\textit{target}|" |\\|"|[\textit{flags}]%
|\input{childdoc.def}\childdocforward[|\textit{main}|]{|\textit{dest}|}"|
\end{center}
%
Here \textit{target} is the name of the output file,
\textit{main} is the name of the main file
and \textit{dest} is the name of the main or child file to be processed
(all filenames without extensions).
The optional argument \textit{main} can be omitted
if \textit{main} matches \textit{dest}.
Optionally, compilation \textit{flags} can be defined via |\def| commands.
This command line makes the \TeX{} engine believe
it is compiling the file \textit{target}
whose content is specified as the latter parameter.
The provided code then forwards the processing to
\textit{main} or \textit{dest} as described in \secref{sec:forward}.

%%%%%%%%%%%%%%%%%%%%%%%%%%%%%%%%%%%%%%%%%%%%%%%%%%%%%%%%%%%%%%%%%%%%%%%%%%%%%%%%
\subsection{Include by Input}
\label{sec:input}

Including child documents by |\include| has some restrictions by design.
Most notably, the content of a child document always occupies
its own set of pages; pages cannot be shared between child documents.
Usually, this behaviour makes perfect sense
because each child document contain an essential part of the document.
However, in some situations it may be desirable to compose
a document from a collection of parts
without having mandatory page breaks between then.
For this case, the package
provides a mechanism to include parts
by |\input| which can also be processed individually.
However, by construction this mechanism
requires manual handling of the content to be output.

%%%%%%%%%%%%%%%%%%%%%%%%%%%%%%%%%%%%%%%%
\DescribeMacro{\ifchilddocmanual}
The main file should be prepared as usual, see \secref{sec:include}.
However, the document body must make a distinction
between processing of an individual part and of the main document, e.g.:
%
\begin{center}
\begin{tabular}{l}
|\ifchilddocmanual|\\
|\input{\childdocname}|\\
|\||else|\\
\textit{document body with }|\input{|\textit{part}|}|\\
|\||fi|
\end{tabular}
\end{center}
%
The conditional |\ifchilddocmanual| is true whenever
a part to be included by |\input| is being compiled,
and the name of the part is stored in |\childdocname|.

%%%%%%%%%%%%%%%%%%%%%%%%%%%%%%%%%%%%%%%%
\DescribeMacro{\childdocby}
Each part to be included by |\input| should start with:
%
\begin{center}
\begin{tabular}{l}
|\input{childdoc.def}|\\
|\childdocby{|\textit{main}|}|\\
\end{tabular}
\end{center}
%
The directive |\childdocby| is similar to |\childdocof|
described in \secref{sec:include},
but the subsequent selection of content must be done manually.
To that end, both |\ifchilddoc| and |\ifchilddocmanual|
will be true upon processing of a part,
and the name of the part is stored in |\childdocname|.
Note that |\jobname| will be set to the filename of the current part
so that each part receives an individual |.aux| file
that does not interfere with the |.aux| file(s) of the main document.
This behaviour can be altered by the alternative form
|\childdocby[*]{|\textit{main}|}| (with a non-empty optional argument)
which uses the |.aux| file of the main document
by setting |\jobname| to \textit{main}.

%%%%%%%%%%%%%%%%%%%%%%%%%%%%%%%%%%%%%%%%%%%%%%%%%%%%%%%%%%%%%%%%%%%%%%%%%%%%%%%%
\subsection{Driver Development}
\label{sec:driver}

The \textsf{childdoc} mechanism can also be use for the development
of definition files such as \LaTeX{} styles or classes.
This case differs from the above setup with multiple parts
included by |\include| in that no |\includeonly| should be invoked.
This can be achieved by starting the include file
(before |\ProvidesPackage|) with:
%
\begin{center}
\begin{tabular}{l}
|\input{childdoc.def}|\\
|\childdocforward{|\textit{main}|}|\\
\end{tabular}
\end{center}
%
or alternatively with:
%
\begin{center}
\begin{tabular}{l}
|\input{childdoc.def}|\\
|\childdocby{|\textit{main}|}|\\
\end{tabular}
\end{center}
%
Both forms have slightly different effects as described above.
The main file is prepared as usual, see \secref{sec:include}.

%%%%%%%%%%%%%%%%%%%%%%%%%%%%%%%%%%%%%%%%%%%%%%%%%%%%%%%%%%%%%%%%%%%%%%%%%%%%%%%%
\subsection{Legacy Detection}
\label{sec:detection}

The directive |\childdocmain| in the main file can detect
whether the complete document or merely a child is to be compiled
even without using the directive |\childdocof|.
This method is deprecated because it is less robust
and there is no compelling reason to use it;
it is merely provided for backward compatibility
and it may be removed in future versions.

If the detection mechanism is to be used,
it is mandatory to correctly specify
the filename of the main file as the argument of |\childdocmain|:
%
\begin{center}
\begin{tabular}{l}
|\input{childdoc.def}|\\
|\childdocmain{|\textit{main}|}|\\
\end{tabular}
\end{center}
%
If |\jobname| does not match the argument \textit{main} of |\childdocmain|,
it is assumed that |\jobname| points to the child file to be compiled.
When using |\childdocmain| with the main file specified as argument,
it suffices to start a child file
with just |\input{|\textit{main}|}|
without loading of the package and using |\childdocof|.
If instead all processing is done
with the appropriate \textsf{childdoc} directives,
the argument of \textit{main} of |\childdocmain| can be empty.

An alternative version of the command line processing described
in \secref{sec:commandline} using the detection mechanism reads:
%
\begin{center}
|... -jobname "|\textit{target}|" "|[\textit{flags}]%
[|\def\jobname{|\textit{dest}|}|]|\input{|\textit{main}|}"|
\end{center}

%%%%%%%%%%%%%%%%%%%%%%%%%%%%%%%%%%%%%%%%%%%%%%%%%%%%%%%%%%%%%%%%%%%%%%%%%%%%%%%%
\subsection{Manual Code}
\label{sec:manual}

In case one cannot be certain whether the definitions file |childdoc.def|
is installed on the target \TeX{} distribution
and one prefers not to ship it,
it is conceivable to paste a few relevant commands into the sources.

To that end, drop all statements |\input{childdoc.def}|
and perform the replacements as outlined below.
Instead of |\childdocmain{|\textit{main}|}| add the following code
to the top of the main file:
%
\begin{center}
\begin{tabular}{l}
|\||ifdefined\childdocname\endinput\||fi\newif\ifchilddoc|\\
|\edef\childdocname{\scantokens\expandafter{\jobname\noexpand}}|\\
|\def\childdocmain{|\textit{main}|}\||ifx\childdocmain\childdocname\||else|\\
|\childdoctrue\includeonly{\childdocname}\let\jobname\childdocmain\||fi|\\
\end{tabular}
\end{center}
%
Instead of |\childdocof{|\textit{main}|}| just include the main file
at the top of each child file:
%
\begin{center}
|\input{|\textit{main}|}|
\end{center}
%
A simple redirection |\childdocforward{|\textit{dest}|}| is achieved by:
%
\begin{center}
|\def\jobname{|\textit{dest}|}\input{\jobname}|
\end{center}
%
The redirection with prefix
|\childdocforwardprefix[|\textit{prefix}|]{|\textit{dest}|}|
is accomplished by:
%
\begin{center}
\begin{tabular}{l}
|{\edef\jobname{\scantokens\expandafter{\jobname\noexpand}}|\\
|\def\redirectjob |\textit{prefix}|#1~~~{\gdef\jobname{|\textit{dest}|#1}}|\\
|\expandafter\redirectjob\jobname~~~}\input{\jobname}|
\end{tabular}
\end{center}

In an alternative approach,
child documents can be compiled by a specific command line
without additional code or specific definitions:
%
\begin{center}
|... -jobname "|\textit{target}|" "|[\textit{flags}]%
|\includeonly{|\textit{dest}|}\input{|\textit{main}|}"|
\end{center}
%

%%%%%%%%%%%%%%%%%%%%%%%%%%%%%%%%%%%%%%%%%%%%%%%%%%%%%%%%%%%%%%%%%%%%%%%%%%%%%%%%
%%%%%%%%%%%%%%%%%%%%%%%%%%%%%%%%%%%%%%%%%%%%%%%%%%%%%%%%%%%%%%%%%%%%%%%%%%%%%%%%
\section{Information}

%%%%%%%%%%%%%%%%%%%%%%%%%%%%%%%%%%%%%%%%%%%%%%%%%%%%%%%%%%%%%%%%%%%%%%%%%%%%%%%%
\subsection{Copyright}

Copyright \copyright{} 2017--2018 Niklas Beisert

This work may be distributed and/or modified under the
conditions of the \LaTeX{} Project Public License, either version 1.3
of this license or (at your option) any later version.
The latest version of this license is in
  \url{http://www.latex-project.org/lppl.txt}
and version 1.3 or later is part of all distributions of \LaTeX{}
version 2005/12/01 or later.

This work has the LPPL maintenance status `maintained'.

The Current Maintainer of this work is Niklas Beisert.

This work consists of the files |README.txt|, |childdoc.ins| and |childdoc.dtx|
as well as the derived files |childdoc.def|, |cdocsamp.tex|
with |cdocsch1.tex|, |cdocsch2.tex|, |cdocspt3.tex|, |cdocspt4.tex|,
|cdocsdrf.tex|, |cdocsfn1.tex|, |cdocsfn2.tex|
as well as |childdoc.pdf|.

%%%%%%%%%%%%%%%%%%%%%%%%%%%%%%%%%%%%%%%%%%%%%%%%%%%%%%%%%%%%%%%%%%%%%%%%%%%%%%%%
\subsection{Files and Installation}

The package consists of the files:
%
\begin{center}
\begin{tabular}{ll}
    |README.txt|   & readme file \\
    |childdoc.ins| & installation file \\
    |childdoc.dtx| & source file \\
    |childdoc.def| & definition file \\
    |cdocsamp.tex| & sample main file \\
    |cdocsch1.tex| & sample include file \\
    |cdocsch2.tex| & sample include file \\
    |cdocspt3.tex| & sample part file \\
    |cdocspt4.tex| & sample part file \\
    |cdocsdrf.tex| & sample redirection file \\
    |cdocsfn1.tex| & sample redirection file \\
    |cdocsfn2.tex| & sample redirection file \\
    |childdoc.pdf| & manual
\end{tabular}
\end{center}
%
The distribution consists of the files
|README.txt|, |childdoc.ins| and |childdoc.dtx|.
%
\begin{itemize}
\item
Run (pdf)\LaTeX{} on |childdoc.dtx|
to compile the manual |childdoc.pdf| (this file).
\item
Run \LaTeX{} on |childdoc.ins| to create the definitions file |childdoc.def|
and the sample |cdocsamp.tex| with include files
|cdocsch1.tex|, |cdocsch2.tex|, |cdocspt3.tex|, |cdocspt4.tex|,
|cdocsdrf.tex|, |cdocsfn1.tex|, |cdocsfn2.tex|.
Then copy the file |childdoc.def| to an appropriate directory of your \LaTeX{}
distribution, e.g.\ \textit{texmf-root}|/tex/latex/childdoc|.
\end{itemize}

%%%%%%%%%%%%%%%%%%%%%%%%%%%%%%%%%%%%%%%%%%%%%%%%%%%%%%%%%%%%%%%%%%%%%%%%%%%%%%%%
\subsection{Related CTAN Packages}

There are several other packages which offer a similar functionality:
%
\begin{itemize}
\item
The packages
\href{http://ctan.org/pkg/docmute}{\textsf{docmute}},
\href{http://ctan.org/pkg/includex}{\textsf{includex}} and
\href{http://ctan.org/pkg/standalone}{\textsf{standalone}}
provide commands to include only the document body of
a child file thus allowing both files to be compiled individually.
\item
The packages \href{http://ctan.org/pkg/subdocs}{\textsf{subdocs}}
and \href{http://ctan.org/pkg/subfiles}{\textsf{subfiles}}
provide structures in which the main and child documents can be
encapsulated and allowing them to be compiled individually.
The inclusion mechanism is different from the conventional |\include|.
\item
The package \href{http://ctan.org/pkg/combine}{\textsf{combine}}
is an elaborate solution to combine several documents into one.
\end{itemize}
%
See also the CTAN topic \href{http://ctan.org/topic/subdocs}{\textsf{subdocs}}
for further related packages.
The present package differs from the above solutions in that
a document structure constructed with the conventional |\include| mechanism
just needs two extra commands at the top of every file
such that all constituent files can be compiled individually.

%%%%%%%%%%%%%%%%%%%%%%%%%%%%%%%%%%%%%%%%%%%%%%%%%%%%%%%%%%%%%%%%%%%%%%%%%%%%%%%%
%\subsection{Feature Suggestions}
%
%The following is a list of features which may be useful for future
%versions of this package:
%%
%\begin{itemize}
%\item
%\ldots
%\end{itemize}

%%%%%%%%%%%%%%%%%%%%%%%%%%%%%%%%%%%%%%%%%%%%%%%%%%%%%%%%%%%%%%%%%%%%%%%%%%%%%%%%
\subsection{Revision History}

%%%%%%%%%%%%%%%%%%%%%%%%%%%%%%%%%%%%%%%%
\paragraph{v2.0:} 2018/12/30

\begin{itemize}
\item
immediate forward processing
\item
added |\childdocby| mechanism
\item
manual restructured
\end{itemize}

%%%%%%%%%%%%%%%%%%%%%%%%%%%%%%%%%%%%%%%%
\paragraph{v1.6:} 2018/01/17

\begin{itemize}
\item
application for development of include files
\item
corrections to manual
\end{itemize}

%%%%%%%%%%%%%%%%%%%%%%%%%%%%%%%%%%%%%%%%
\paragraph{v1.5:} 2017/05/21

\begin{itemize}
\item
more complete structuring introduced
\item
|\childdocof| introduced
\item
|\childdoc| renamed to |\childdocmain|
\item
|\childredirect| renamed to |\childdocforward| and |\childdocforwardprefix|
and functionality expanded
\end{itemize}

%%%%%%%%%%%%%%%%%%%%%%%%%%%%%%%%%%%%%%%%
\paragraph{v1.0:} 2017/04/27

\begin{itemize}
\item
manual and install package
\item
first version published on CTAN
\end{itemize}

%%%%%%%%%%%%%%%%%%%%%%%%%%%%%%%%%%%%%%%%
\paragraph{v0.6:} 2017/04/26

\begin{itemize}
\item
redirection mechanism added
\end{itemize}

%%%%%%%%%%%%%%%%%%%%%%%%%%%%%%%%%%%%%%%%
\paragraph{v0.5:} 2017/04/26

\begin{itemize}
\item
functionality in definition file
\end{itemize}


%%%%%%%%%%%%%%%%%%%%%%%%%%%%%%%%%%%%%%%%%%%%%%%%%%%%%%%%%%%%%%%%%%%%%%%%%%%%%%%%
%%%%%%%%%%%%%%%%%%%%%%%%%%%%%%%%%%%%%%%%%%%%%%%%%%%%%%%%%%%%%%%%%%%%%%%%%%%%%%%%
%%%%%%%%%%%%%%%%%%%%%%%%%%%%%%%%%%%%%%%%%%%%%%%%%%%%%%%%%%%%%%%%%%%%%%%%%%%%%%%%
\appendix

\settowidth\MacroIndent{\rmfamily\scriptsize 000\ }

 \DocInput{childdoc.dtx}

\end{document}
%</driver>
% \fi
%
% %%%%%%%%%%%%%%%%%%%%%%%%%%%%%%%%%%%%%%%%%%%%%%%%%%%%%%%%%%%%%%%%%%%%%%%%%%%%%%
% %%%%%%%%%%%%%%%%%%%%%%%%%%%%%%%%%%%%%%%%%%%%%%%%%%%%%%%%%%%%%%%%%%%%%%%%%%%%%%
% \section{Sample}
%\iffalse
%<*samplemain>
%\fi
%
% The following presents a sample document
% with two chapters, two parts, a title page,
% a compile flag as well as three forwarding files to set the flag.
% It consists of eight |.tex| files:
% \begin{center}
% \begin{tabular}{ll}
% |cdocsamp.tex|&main file\\
% |cdocsch1.tex|&include file for chapter 1\\
% |cdocsch2.tex|&include file for chapter 2\\
% |cdocspt3.tex|&include file for part 3\\
% |cdocspt4.tex|&include file for part 4\\
% |cdocsdrf.tex|&forwarding file for main file in draft mode\\
% |cdocsfi1.tex|&forwarding file for final version of chapter 1\\
% |cdocsfi2.tex|&forwarding file for final version of chapter 2\\
% \end{tabular}
% \end{center}
% Each of the eight files can be compiled directly by the \LaTeX{} compiler.
%
% %%%%%%%%%%%%%%%%%%%%%%%%%%%%%%%%%%%%%%
% \paragraph{Main File.}
%
% The main file is called |cdocsamp.tex|.
%
% Load the \textsf{childdoc} definitions and
% declare the filename for the main document:
%    \begin{macrocode}
\input{childdoc.def}
\childdocmain{}
%    \end{macrocode}

% Optional override for |\version| flag:
%    \begin{macrocode}
%%\ifchilddoc\else\providecommand{\version}{draft}\fi
%    \end{macrocode}

% Define the default values for the |\version| flag
% (|final| for the main file and |draft| for childs):
%    \begin{macrocode}
\ifchilddoc
\providecommand{\version}{draft}
\else
\providecommand{\version}{final}
\fi
%    \end{macrocode}

% Load the standard document class:
%    \begin{macrocode}
\documentclass[12pt]{article}
%    \end{macrocode}

% Start the document body:
%    \begin{macrocode}
\begin{document}
%    \end{macrocode}

% Declare a title page.
% Print title, part of document being processed and version flag:
%    \begin{macrocode}
\addtocounter{page}{-1}
\begin{center}
{\LARGE\bfseries{}childdoc example\par}
\vspace{1cm}
\ifchilddoc
\ifchilddocmanual part\else chapter\fi:
`\childdocname' of `\childdocjob'\par
\else
main document: `\childdocjob'\par
\fi
version: \version\par
\end{center}
\newpage
%    \end{macrocode}

% Manually include selected file,
% otherwise process as usual:
%    \begin{macrocode}
\ifchilddocmanual
\section*{part `\childdocname'}
\input{\childdocname}
\else
%    \end{macrocode}

% Include the two chapters:
%    \begin{macrocode}
\include{cdocsch1}
\include{cdocsch2}
%    \end{macrocode}

% Include the two parts unless only chapters should be displayed:
%    \begin{macrocode}
\ifchilddoc\else
\section{part three}
\input{cdocspt3}
\section{part four}
\input{cdocspt4}
\fi
%    \end{macrocode}

% Process as usual until here:
%    \begin{macrocode}
\fi
%    \end{macrocode}

% End of document body:
%    \begin{macrocode}
\end{document}
%    \end{macrocode}
%\iffalse
%</samplemain>
%\fi
%
% %%%%%%%%%%%%%%%%%%%%%%%%%%%%%%%%%%%%%%
% \paragraph{Chapter Include Files.}
%
% The include files are called |cdocsch1.tex| and |cdocsch2.tex|.
%
%\iffalse
%<*samplechap1|samplechap2>
%\fi

% Optional override for |\version| flag:
%    \begin{macrocode}
%%\providecommand{\version}{final}
%    \end{macrocode}

% Include the main document:
%    \begin{macrocode}
\input{childdoc.def}
\childdocof{cdocsamp}
%    \end{macrocode}

%\iffalse
%</samplechap1|samplechap2>
%\fi
%
%\iffalse
%<*samplechap1>
%\fi
% Some text for chapter 1:
%    \begin{macrocode}
\section{one}
some text in chapter one
%    \end{macrocode}

%\iffalse
%</samplechap1>
%\fi
% Some text for chapter 2:
%\iffalse
%<*samplechap2>
%\fi
%    \begin{macrocode}
\section{two}
more text in chapter two
%    \end{macrocode}

%\iffalse
%</samplechap2>
%\fi
%
% %%%%%%%%%%%%%%%%%%%%%%%%%%%%%%%%%%%%%%
% \paragraph{Part Include Files.}
%
% The include files are called |cdocspt3.tex| and |cdocspt4.tex|.
%
%\iffalse
%<*samplepart3|samplepart4>
%\fi

% Optional override for |\version| flag:
%    \begin{macrocode}
%%\providecommand{\version}{final}
%    \end{macrocode}

% Include the main document:
%    \begin{macrocode}
\input{childdoc.def}
\childdocby{cdocsamp}
%    \end{macrocode}

%\iffalse
%</samplepart3|samplepart4>
%\fi
%
%\iffalse
%<*samplepart3>
%\fi
% Some text for part 3:
%    \begin{macrocode}
some text in part three
%    \end{macrocode}

%\iffalse
%</samplepart3>
%\fi
% Some text for part 4:
%\iffalse
%<*samplepart4>
%\fi
%    \begin{macrocode}
more text in part four
%    \end{macrocode}

%\iffalse
%</samplepart4>
%\fi
%
% %%%%%%%%%%%%%%%%%%%%%%%%%%%%%%%%%%%%%%
% \paragraph{Forwarding for a Complete Draft.}
%
% The following forwarding file |cdocsdrf.tex|
% compiles the main document in draft mode:
%\iffalse
%<*sampledraft>
%\fi
%    \begin{macrocode}
\def\version{draft}
\input{childdoc.def}
\childdocforward{cdocsamp}
%    \end{macrocode}

%\iffalse
%</sampledraft>
%\fi
%
% %%%%%%%%%%%%%%%%%%%%%%%%%%%%%%%%%%%%%%
% \paragraph{Forwarding for Final Version of the Chapters.}
%
% The following forwarding files |cdocsfn1.tex| and |cdocsfn2.tex|
% (with identical content)
% compile the final versions of the child documents
% |cdocsch1.tex| and |cdocsch2.tex|, respectively:
%\iffalse
%<*samplefinal>
%\fi
%    \begin{macrocode}
\def\version{final}
\input{childdoc.def}
\childdocforwardprefix[cdocsamp]{cdocsfn}{cdocsch}
%    \end{macrocode}

%\iffalse
%</samplefinal>
%\fi
%
% %%%%%%%%%%%%%%%%%%%%%%%%%%%%%%%%%%%%%%
% \paragraph{Command Line Processing.}
%
% The following three command lines generate the output files
% |cdocscld|, |cdocscl1| and |cdocscl2|
% which should be identical to
% |cdocsdrf|, |cdocsch1| and |cdocsfn2|, respectively:
% \begin{center}
% \begin{tabular}{l}
% |latex -jobname cdocscld \|\\
% |  "\def\version{draft}\input{childdoc.def}\childdocforward{cdocsamp}"|\\
% |latex -jobname cdocscl1 \|\\
% |  "\input{childdoc.def}\childdocforward[cdocsamp]{cdocsch1}"|\\
% |latex -jobname cdocscl2 \|\\
% |  "\def\version{final}\input{childdoc.def}\childdocforward{cdocsch2}"|
% \end{tabular}
% \end{center}
% Note that the trailing backslash on each first line
% merely continues the input to the second line
% (for convenient cut ant paste).
% Furthermore, the command |latex| can be replaced by any
% of its alternative versions such as |pdflatex|.
%
% %%%%%%%%%%%%%%%%%%%%%%%%%%%%%%%%%%%%%%%%%%%%%%%%%%%%%%%%%%%%%%%%%%%%%%%%%%%%%%
% %%%%%%%%%%%%%%%%%%%%%%%%%%%%%%%%%%%%%%%%%%%%%%%%%%%%%%%%%%%%%%%%%%%%%%%%%%%%%%
% \section{Implementation}
%\iffalse
%<*package>
%\fi
%
% This section describes the definitions file |childdoc.def|.

% The definitions cannot be loaded using |\usepackage| or |\RequirePackage|
% which has a mechanism to prevent loading a style file more than once.
% When loading the definitions by means of |\input|
% multiple instances have to be prevented manually:
%\iffalse
%This code needs to be before the `\ProvidesFile' directive
%which is defined at the beginning of this file.
%Therefore it is also placed there and commented out here.
%</package>
%<*discard>
%\fi
%    \begin{macrocode}
\ifdefined\childdocmain\endinput\fi
%    \end{macrocode}
%\iffalse
%</discard>
%<*package>
%\fi
%
% \macro{\ifchilddoc}
% \macro{\ifchilddocmanual}
% The conditional |\ifchilddoc| tells whether a
% child (true) or main (false) document is being compiled.
% The conditional |\ifchilddocmanual| tells whether
% the |\includeonly| mechanism is used (false) or
% the selection of child files must be performed manually (true).
% The definitions initialise to false:
%    \begin{macrocode}
\newif\ifchilddoc
\newif\ifchilddocmanual
%    \end{macrocode}

% \macro{\childdocname}
% \macro{\childdocjob}
% The macro |\childdocname| stores the name of the main document
% to be compiled. The macro |\childdocjob| stores the name of
% the document on which the \LaTeX{} compiler was originally invoked.
% The content of |\jobname| cannot be compared
% to filenames specified in the source due to different catcodes.
% The following code rescans |\jobname|, stores the result
% in |\childdocname| and saves a copy in |\childdocjob|:
%    \begin{macrocode}
\edef\childdocname{\scantokens\expandafter{\jobname\noexpand}}
\let\childdocjob\childdocname
%    \end{macrocode}

% \macro{\childdocdisable}
% The macro |\childdocdisable| prevents the main file
% from being processed more than once.
% At this stage, the main document command |\childdocmain|
% is assumed to be called once again where it should do nothing.
% Any subsequent call to it should prevent
% a secondary processing of the main document
% It overwrites the forwarding commands
% |\childdocof| and |\childdocforward|
% with empty macros to prevent further inclusions of the main document:
%    \begin{macrocode}
\newcommand{\childdocdisable}
{
  \renewcommand{\childdocmain}[1]{\renewcommand{\childdocmain}[1]{\endinput}}
  \renewcommand{\childdocof}[1]{}
  \renewcommand{\childdocby}[2][]{}
  \renewcommand{\childdocforward}[2][]{}
  \renewcommand{\childdocdisable}{}
}
%    \end{macrocode}

% \macro{\childdocmain}
% The macro |\childdocmain| is to be called at the top of the main file
% with nothing or the main filename (without extension) as argument.
% First, it breaks loops.
% If the argument is not empty and does not match |\childdocname|
% (which is set by the first inclusion of |childdoc.def|),
% |\ifchilddoc| is set to true, |\includeonly| is applied to the child file
% and |\jobname| is set to the main file
% (for proper handling of |.aux| files):
%    \begin{macrocode}
\newcommand{\childdocmain}[1]
{
  \childdocdisable\childdocmain{}
  \if?#1?\else
    \begingroup
      \def\childdoctmp{#1}
      \ifx\childdoctmp\childdocname
        \def\childdoctmp{}
      \else
        \def\childdoctmp
        {
          \childdoctrue
          \includeonly{\childdocname}
          \def\childdocjob{#1}
          \def\jobname{#1}
        }
      \fi
      \expandafter
    \endgroup
    \childdoctmp
  \fi
}
%    \end{macrocode}

% \macro{\childdocof}
% The command |\childdocof| redirects
% compilation to the main file |#1|.
%    \begin{macrocode}
\newcommand{\childdocof}[1]
{
  \childdocdisable
  \childdoctrue
  \includeonly{\childdocname}
  \def\jobname{#1}
  \def\childdocjob{#1}
  \input{#1}
}
%    \end{macrocode}

% \macro{\childdocby}
% The command |\childdocby| ....
%    \begin{macrocode}
\newcommand{\childdocby}[2][]
{
  \childdocdisable
  \childdoctrue
  \childdocmanualtrue
  \if?#1?\else
    \def\jobname{#2}
  \fi
  \def\childdocjob{#2}
  \input{#2}
  \endinput
}
%    \end{macrocode}

% \macro{\childdocforward}
% The command |\childdocforward| redirects
% compilation to the main file or
% (if the optional argument is given) a child file.
% Parameters are set as if the main file
% or a child file starting with |\childdocof| was compiled.
% Then compilation is handed over to the main file:
%    \begin{macrocode}
\newcommand{\childdocforward}[2][]
{
  \begingroup
    \if?#1?
      \def\childdoctmp
      {
        \def\childdocname{#2}
        \def\childdocjob{#2}
        \def\jobname{#2}
        \input{#2}
        \endinput
      }
    \else
      \def\childdoctmp
      {
        \childdocdisable
        \def\childdocname{#2}
        \childdoctrue
        \includeonly{#2}
        \def\childdocjob{#1}
        \def\jobname{#1}
        \input{#1}
        \endinput
      }
    \fi
    \expandafter
  \endgroup
  \childdoctmp
}
%    \end{macrocode}

% \macro{\childdocforwardprefix}
% The command |\childdocforwardprefix| redirects
% compilation to the main or a child file by means of a pattern.
% The prefix |#1| in the current filename is replaced by |#2|
% and the suffix of the current filename is kept
% (it is assumed that the filename does not contain the substring `|~~~|'
% which is used as a delimiter).
% Compilation is handed over to the new file by |\childdocforward|:
%    \begin{macrocode}
\newcommand{\childdocforwardprefix}[3][]
{
  \begingroup
    \def\childdocextract #2##1~~~{\def\childdoctmp{\childdocforward[#1]{#3##1}}}
    \expandafter\childdocextract\childdocname~~~
    \expandafter
  \endgroup
  \childdoctmp
}
%    \end{macrocode}

% \macro{\childdoc}
% The deprecated macro |\childdoc| is a legacy version of |\childdocmain|:
%    \begin{macrocode}
\newcommand{\childdoc}{\childdocmain}
%    \end{macrocode}

% \macro{\childdocredirect}
% The deprecated macro |\childdocredirect| is a legacy version
% of |\childdocforward| and |\childdocforwardprefix|:
%    \begin{macrocode}
\newcommand{\childdocredirect}[2][]
{
  \begingroup
    \if?#1?
      \def\childdoctmp{\childdocforward{#2}}
    \else
      \def\childdoctmp{\childdocforwardprefix{#1}{#2}}
    \fi
    \expandafter
  \endgroup
  \childdoctmp
}
%    \end{macrocode}

%\iffalse
%</package>
%\fi
%
\endinput
|\\
|\childdocforward{|\textit{main}|}|
\end{tabular}
\end{center}
%
Likewise, the following files |final|\textit{nn}|.tex|
compile the final version of the child document
|child|\textit{nn}|.tex|:
%
\begin{center}
\begin{tabular}{l}
|\def\version{final}|\\
|% \iffalse
%
% childdoc.dtx Copyright (C) 2017-2018 Niklas Beisert
%
% This work may be distributed and/or modified under the
% conditions of the LaTeX Project Public License, either version 1.3
% of this license or (at your option) any later version.
% The latest version of this license is in
%   http://www.latex-project.org/lppl.txt
% and version 1.3 or later is part of all distributions of LaTeX
% version 2005/12/01 or later.
%
% This work has the LPPL maintenance status `maintained'.
%
% The Current Maintainer of this work is Niklas Beisert.
%
% This work consists of the files childdoc.dtx and childdoc.ins
% and the derived files childdoc.def and cdocsamp.tex with
% cdocsch1.tex, cdocsch2.tex, cdocsdrf.tex, cdocsfn1.tex, cdocsfn2.tex.
%
%<package>\ifdefined\childdocmain\endinput\fi
%<package>\ProvidesFile{childdoc.def}[2018/12/30 v2.0 child document driver]
%<samplemain>\ProvidesFile{cdocsamp.tex}[2018/12/30 v2.0 sample for childdoc]
%<*driver>
%\ProvidesFile{childdoc.drv}[2018/12/30 v2.0 childdoc reference manual file]
\PassOptionsToClass{10pt,a4paper}{article}
\documentclass{ltxdoc}

\usepackage[margin=35mm]{geometry}
\usepackage{hyperref}
\usepackage{hyperxmp}
\usepackage[usenames]{color}

\hypersetup{colorlinks=true}
\hypersetup{pdfstartview=FitH}
\hypersetup{pdfpagemode=UseNone}
\hypersetup{pdfsource={}}
\hypersetup{pdflang={en-UK}}
\hypersetup{pdfcopyright={Copyright 2017-2018 Niklas Beisert.
  This work may be distributed and/or modified under the
  conditions of the LaTeX Project Public License, either version 1.3
  of this license or (at your option) any later version.}}
\hypersetup{pdflicenseurl={http://www.latex-project.org/lppl.txt}}
\hypersetup{pdfcontactaddress={ETH Zurich, ITP, HIT K,
  Wolfgang-Pauli-Strasse 27}}
\hypersetup{pdfcontactpostcode={8093}}
\hypersetup{pdfcontactcity={Zurich}}
\hypersetup{pdfcontactcountry={Switzerland}}
\hypersetup{pdfcontactemail={nbeisert@itp.phys.ethz.ch}}
\hypersetup{pdfcontacturl={http://people.phys.ethz.ch/\xmptilde nbeisert/}}

\newcommand{\secref}[1]{\hyperref[#1]{section \ref*{#1}}}

\parskip1ex
\parindent0pt
\let\olditemize\itemize
\def\itemize{\olditemize\parskip0pt}

\begin{document}

\title{The \textsf{childdoc} Package}
\hypersetup{pdftitle={The childdoc Package}}
\author{Niklas Beisert\\[2ex]
  Institut f\"ur Theoretische Physik\\
  Eidgen\"ossische Technische Hochschule Z\"urich\\
  Wolfgang-Pauli-Strasse 27, 8093 Z\"urich, Switzerland\\[1ex]
  \href{mailto:nbeisert@itp.phys.ethz.ch}
  {\texttt{nbeisert@itp.phys.ethz.ch}}}
\hypersetup{pdfauthor={Niklas Beisert}}
\hypersetup{pdfsubject={Manual for the LaTeX2e Package childdoc}}
\date{30 December 2018, \textsf{v2.0}}
\maketitle

\begin{abstract}\noindent
\textsf{childdoc} is a \LaTeXe{} package
that enables the direct compilation
of document sections included by |\include|
to individual files.
\end{abstract}

\begingroup
\parskip0ex
\tableofcontents
\endgroup

%%%%%%%%%%%%%%%%%%%%%%%%%%%%%%%%%%%%%%%%%%%%%%%%%%%%%%%%%%%%%%%%%%%%%%%%%%%%%%%%
%%%%%%%%%%%%%%%%%%%%%%%%%%%%%%%%%%%%%%%%%%%%%%%%%%%%%%%%%%%%%%%%%%%%%%%%%%%%%%%%
\section{Introduction}

\LaTeX{} provides a mechanism to structure a large document (such as a book)
into a main file and several child files (containing the chapters)
using the |\include| command.
This mechanism is beneficial for documents
which span hundreds of pages in order to
make the source file(s) more manageable.
Moreover, compilation can be restricted to
selected child files by means of the |\includeonly| command.
The latter feature can be used to reduce the compilation time while editing
(this was significantly more useful in the earlier days of \LaTeX{})
or to generate a smaller document which is easier to navigate.
Another application of |\includeonly| is to generate
documents consisting of selected parts of the complete document.

However, there are a few drawbacks of the plain |\include| mechanism:
\begin{itemize}
\item
The child files cannot be compiled on their own,
they can only be compiled via the main file.
A naive editing environment
(such as a text editor with an option
to have the current file processed by \LaTeX)
may require one to switch to the main file before compiling;
attempting to compile the child file produces errors.
\item
The main file must be modified (each time)
to adjust the |\includeonly| command
to the present needs. This easily leaves the main file in a messy state.
\item
The generated document will always carry the filename
of the main document. This is inconvenient if
several child files are to be compiled and
to be kept for distribution.
\end{itemize}

The present package provides a simple interface
to make child files individually compilable by \LaTeX{}.
Compiling a child file then has the same effect as compiling
the main file with an |\includeonly| command
to select the appropriate child.
Moreover the generated document will carry the name of the child
rather than the main file.
This resolves all three above issues.

This feature is meant to make the editing of books,
thesis documents and lecture notes somewhat more convenient.
However, the package can also be used efficiently for
composing a series of documents (such as exercise sheets)
which are typically distributed individually.
It then assists the author in generating the individual documents
(potentially in different versions)
as well as a document containing the collected series.
Another application is in developing style files
or other kinds of included material
where compilation of the style file could redirect
to a sample or test file.

%%%%%%%%%%%%%%%%%%%%%%%%%%%%%%%%%%%%%%%%%%%%%%%%%%%%%%%%%%%%%%%%%%%%%%%%%%%%%%%%
%%%%%%%%%%%%%%%%%%%%%%%%%%%%%%%%%%%%%%%%%%%%%%%%%%%%%%%%%%%%%%%%%%%%%%%%%%%%%%%%
\section{Usage}

First of all, the package \textsf{childdoc} is \emph{not} a standard
\LaTeXe{} |.sty| style file! Therefore it needs to be invoked in
a non-standard way.

%%%%%%%%%%%%%%%%%%%%%%%%%%%%%%%%%%%%%%%%%%%%%%%%%%%%%%%%%%%%%%%%%%%%%%%%%%%%%%%%
\subsection{Included Files}
\label{sec:include}

%%%%%%%%%%%%%%%%%%%%%%%%%%%%%%%%%%%%%%%%
\DescribeMacro{\childdocmain}
To use the package, add the commands
\begin{center}
\begin{tabular}{l}
|\input{childdoc.def}|\\
|\childdocmain{}|\\
\end{tabular}
\end{center}
at the very top of the main \LaTeX{} file,
in particular \emph{before} the |\documentclass| statement!
The argument of |\childdocmain| should be left empty
(but it must be present).

%%%%%%%%%%%%%%%%%%%%%%%%%%%%%%%%%%%%%%%%
\DescribeMacro{\childdocof}
Furthermore, add the commands
\begin{center}
\begin{tabular}{l}
|\input{childdoc.def}|\\
|\childdocof{|\textit{main}|}|\\
\end{tabular}
\end{center}
at the top of every child file \textit{child}
which is included by |\include{|\textit{child}|}|
from within the main file
(or at least for those files to be compiled individually).
The argument \textit{main} must be the filename of the main file.

There are a couple of
considerations in setting up the main and child documents:

%%%%%%%%%%%%%%%%%%%%%%%%%%%%%%%%%%%%%%%%
\paragraph{Restrictions.}

Please note the following restrictions:
\begin{itemize}
\item
|\childdocmain| must be called with one argument \textit{main}
to ensure compatibility with earlier version of the package.
It must either be empty (|\childdocmain{}|)
or precisely match the filename of the main file in which it is specified.
See \secref{sec:detection} for further information.
\item
The filename \textit{main} must be specified without the |.tex| extension.
\item
The filename \textit{main} is case sensitive
(even in case-insensitive file systems)
due to internal string comparison.
\item
The argument \textit{main} should be fully expanded, it cannot be a macro.
\item
Subdirectories and special characters should be avoided in filenames.
\item
The command |\childdocmain{|\textit{main}|}| must be followed by a whitespace.
It should not be followed immediately by another command
or by a comment mark `|%|'.
This is because the \TeX{} parser reads the token immediately following
the argument of |\childdocmain| and puts it
at the beginning of every child section;
however, a white\-space is ignored.
\end{itemize}

%%%%%%%%%%%%%%%%%%%%%%%%%%%%%%%%%%%%%%%%
\paragraph{Content of Main File.}

It is advisable to place all content in the child files included by |\include|.
Any output contained in the main file will appear in all child documents
unless suppressed manually;
it cannot be suppressed automatically by the |\includeonly| directive
and thus should normally be avoided.
A method to include some content in the main file
by means of conditional processing is described in \secref{sec:conditional}.

%%%%%%%%%%%%%%%%%%%%%%%%%%%%%%%%%%%%%%%%
\paragraph{Page Numbering.}

When only a part of the document is compiled,
the appropriate numbering of pages
(as well as other status parameters)
is determined from the |.aux| files.
The latter contain information from previous passes.
However this information needs to propagate through
all intermediate child documents.
Therefore the page numbering in child documents may well
be inconsistent until the complete document is compiled at least once.

A useful (if unconventional) way to always ensure a consistent
page numbering is to restart the numbering in each child document
and denote the pages by `\textit{child}|.|\textit{page}'
where \textit{child} represents the chapter/section number of the child file.
This can be achieved by the command
|\numberwithin{page}{|\textit{child}|}|
of the \textsf{amsmath} package
where \textit{child} can be |chapter| or |section|
depending on the chosen structuring.
Alternatively, one can modify the macro |\thepage| appropriately
and reset the counter |page| at the start of each child file.

%%%%%%%%%%%%%%%%%%%%%%%%%%%%%%%%%%%%%%%%%%%%%%%%%%%%%%%%%%%%%%%%%%%%%%%%%%%%%%%%
\subsection{Conditional Processing}
\label{sec:conditional}

The package provides a mechanism to compile different versions
of a document. To customise the versions further some conditional processing
can come in handy to distinguish which version is being compiled.
The package provides two macros to describe the compilation context:

%%%%%%%%%%%%%%%%%%%%%%%%%%%%%%%%%%%%%%%%
\DescribeMacro{\ifchilddoc}
The conditional |\ifchilddoc| distinguishes between the compilation of
child documents and the main document:
%
\begin{center}
|\ifchilddoc |\textit{child-code}| |[|\||else |\textit{main-code}]| \||fi|
\end{center}

%%%%%%%%%%%%%%%%%%%%%%%%%%%%%%%%%%%%%%%%
\DescribeMacro{\childdocname}
\DescribeMacro{\childdocjob}
The macro |\childdocname| contains the filename (without extension)
of the main or child file being processed.
Note that |\childdocjob| will always contain the name of the main file.

%%%%%%%%%%%%%%%%%%%%%%%%%%%%%%%%%%%%%%%%
\paragraph{Title Page.}

Conditional processing can be used to include a title or banner page
in the main document when proper precautions are taken.
Importantly, the code in the main file should ensure that the page counter
(as well as other status parameters which are stored in the |.aux| files)
takes the same value after the conditional processing.
Otherwise the page numbers may take divergent values
depending on which part is compiled.

For example, a title page could be declared by:
%
\begin{center}
\begin{tabular}{l}
|\ifchilddoc\||else|\\
|\addtocounter{page}{-1}|\\
\textit{code for title page}\\
|\newpage|\\
|\||fi|
\end{tabular}
\end{center}
%
A banner page for the child documents can be generated by:
%
\begin{center}
\begin{tabular}{l}
|\ifchilddoc|\\
|\addtocounter{page}{-1}|\\
\textit{code for banner page}\\
|\newpage|\\
|\||fi|
\end{tabular}
\end{center}
%
Here one could write a message such as:
\begin{center}
|This is the part \childdocname{} of \childdocjob{}.|
\end{center}

%%%%%%%%%%%%%%%%%%%%%%%%%%%%%%%%%%%%%%%%%%%%%%%%%%%%%%%%%%%%%%%%%%%%%%%%%%%%%%%%
\subsection{Flags}
\label{sec:flags}

The package makes it easy to generate different versions
of the main or child documents.
To this end compilation flags can be defined
and assigned different default values.
They will be particularly useful in conjunction
with the forwarding mechanism described in \secref{sec:forward}.

For example, it may be useful to have a flag |\version|
which can be set to |draft| or |final|.
The document source will contain some conditional code
depending on the value of |\version|.
Suppose further, the flag should default to |final| for the main file
and to |draft| for child files
which is a natural assignment for editing the document.
This is achieved by placing the following code
in the preamble of the main document
(below the |\childdocmain| directive):
%
\begin{center}
\begin{tabular}{l}
|\ifchilddoc|\\
|\providecommand{\version}{draft}|\\
|\||else|\\
|\providecommand{\version}{final}|\\
|\||fi|
\end{tabular}
\end{center}
%
The definition by |\providecommand| makes sure
that previous definitions are not overwritten.
Further statements |\providecommand{\version}{...}|
can thus be added before the above code to override it.

For the main file, one might add a line
(between |\childdocmain| and the above block)
%
\begin{center}
|%\ifchilddoc\||else\providecommand{\version}{draft}\||fi|
\end{center}
%
which can be uncommented to produce a draft version.
Likewise one can add a line to the very top of a child file
(above the |\childdocof{|\textit{main}|}| directive)
%
\begin{center}
|%\providecommand{\version}{final}|
\end{center}
%
which can be uncommented to produce the final version of this child document.

%%%%%%%%%%%%%%%%%%%%%%%%%%%%%%%%%%%%%%%%%%%%%%%%%%%%%%%%%%%%%%%%%%%%%%%%%%%%%%%%
\subsection{Forwarding}
\label{sec:forward}

Different versions of the main or child documents
using compilation flags as described in \secref{sec:flags}
can be (permanently) stored in different files
for convenient compilation, viewing and distribution.
To this end, the package defines a command
to pass on compilation to a different file:

%%%%%%%%%%%%%%%%%%%%%%%%%%%%%%%%%%%%%%%%
\DescribeMacro{\childdocforward}
The command |\childdocforward| redirects processing to
another source file:
%
\begin{center}
\begin{tabular}{l}
|\input{childdoc.def}|\\
|\childdocforward[|\textit{main}|]{|\textit{dest}|}|\\
\end{tabular}
\end{center}
%
The argument \textit{dest} is the destination file
(without extension).
It should be the main file or one of the child files.
Note that further \textsf{childdoc} directives
such as |\childdocof| and |\childdocforward|
in the indicated file will be processed in this form.
The optional argument \textit{main}
passes on directly to the main file \textit{main}
while pretending to compile the child \textit{dest}.
This form behaves as if \textit{dest}
issues |\childdocof{|\textit{main}|}| right away,
and no further \textsf{childdoc} directives will be processed.

%%%%%%%%%%%%%%%%%%%%%%%%%%%%%%%%%%%%%%%%
\DescribeMacro{\...prefix}
In the alternative form |\childdocforwardprefix|,
%
\begin{center}
\begin{tabular}{l}
|\input{childdoc.def}|\\
|\childdocforwardprefix[|\textit{main}|]{|\textit{prefix}|}{|\textit{dest}|}|
\end{tabular}
\end{center}
%
the destination file is determined by a pattern
depending on the current file:
To make this work, the current file must be called
`{\textit{prefix}\hspace{0.2em}\textit{suffix}}'
with \textit{prefix} matching precisely the argument.
Processing is then passed on to the file
`{\textit{dest}\hspace{0.2em}\textit{suffix}}'.
Surely, the same effect is achieved by
directly specifying the
argument `{\textit{dest}\hspace{0.2em}\textit{suffix}}'
in the first form.
However, that requires to set up a different file
for each child. With the alternative form of the command
all these files can have exactly the same content
which simplifies setting them up and maintaining them.

For example, the following file |draft.tex|
with a compilation flag |\version| as described in \secref{sec:flags}
compiles the main document as a draft:
%
\begin{center}
\begin{tabular}{l}
|\def\version{draft}|\\
|\input{childdoc.def}|\\
|\childdocforward{|\textit{main}|}|
\end{tabular}
\end{center}
%
Likewise, the following files |final|\textit{nn}|.tex|
compile the final version of the child document
|child|\textit{nn}|.tex|:
%
\begin{center}
\begin{tabular}{l}
|\def\version{final}|\\
|\input{childdoc.def}|\\
|\childdocforwardprefix{final}{child}|
\end{tabular}
\end{center}
%

Note that when several versions of a main file and/or of each child file
are to be generated, it may be convenient to set up a |Makefile| or
shell script to automatise the process.

%%%%%%%%%%%%%%%%%%%%%%%%%%%%%%%%%%%%%%%%%%%%%%%%%%%%%%%%%%%%%%%%%%%%%%%%%%%%%%%%
\subsection{Command Line Processing}
\label{sec:commandline}

The effect of redirection files can also be achieved by invoking
the \LaTeX{} compiler with a more elaborate command line.
Most conveniently this should be done as part
of a shell script or a |Makefile|.

When using \textsf{childdoc} in the main file, the following
command lines effectively perform a redirection
(note that depending on the shell being used,
backslashes may have to be doubled: `|\|' $\to$ `|\\|'):
%
\begin{center}
|... -jobname "|\textit{target}|" |\\|"|[\textit{flags}]%
|\input{childdoc.def}\childdocforward[|\textit{main}|]{|\textit{dest}|}"|
\end{center}
%
Here \textit{target} is the name of the output file,
\textit{main} is the name of the main file
and \textit{dest} is the name of the main or child file to be processed
(all filenames without extensions).
The optional argument \textit{main} can be omitted
if \textit{main} matches \textit{dest}.
Optionally, compilation \textit{flags} can be defined via |\def| commands.
This command line makes the \TeX{} engine believe
it is compiling the file \textit{target}
whose content is specified as the latter parameter.
The provided code then forwards the processing to
\textit{main} or \textit{dest} as described in \secref{sec:forward}.

%%%%%%%%%%%%%%%%%%%%%%%%%%%%%%%%%%%%%%%%%%%%%%%%%%%%%%%%%%%%%%%%%%%%%%%%%%%%%%%%
\subsection{Include by Input}
\label{sec:input}

Including child documents by |\include| has some restrictions by design.
Most notably, the content of a child document always occupies
its own set of pages; pages cannot be shared between child documents.
Usually, this behaviour makes perfect sense
because each child document contain an essential part of the document.
However, in some situations it may be desirable to compose
a document from a collection of parts
without having mandatory page breaks between then.
For this case, the package
provides a mechanism to include parts
by |\input| which can also be processed individually.
However, by construction this mechanism
requires manual handling of the content to be output.

%%%%%%%%%%%%%%%%%%%%%%%%%%%%%%%%%%%%%%%%
\DescribeMacro{\ifchilddocmanual}
The main file should be prepared as usual, see \secref{sec:include}.
However, the document body must make a distinction
between processing of an individual part and of the main document, e.g.:
%
\begin{center}
\begin{tabular}{l}
|\ifchilddocmanual|\\
|\input{\childdocname}|\\
|\||else|\\
\textit{document body with }|\input{|\textit{part}|}|\\
|\||fi|
\end{tabular}
\end{center}
%
The conditional |\ifchilddocmanual| is true whenever
a part to be included by |\input| is being compiled,
and the name of the part is stored in |\childdocname|.

%%%%%%%%%%%%%%%%%%%%%%%%%%%%%%%%%%%%%%%%
\DescribeMacro{\childdocby}
Each part to be included by |\input| should start with:
%
\begin{center}
\begin{tabular}{l}
|\input{childdoc.def}|\\
|\childdocby{|\textit{main}|}|\\
\end{tabular}
\end{center}
%
The directive |\childdocby| is similar to |\childdocof|
described in \secref{sec:include},
but the subsequent selection of content must be done manually.
To that end, both |\ifchilddoc| and |\ifchilddocmanual|
will be true upon processing of a part,
and the name of the part is stored in |\childdocname|.
Note that |\jobname| will be set to the filename of the current part
so that each part receives an individual |.aux| file
that does not interfere with the |.aux| file(s) of the main document.
This behaviour can be altered by the alternative form
|\childdocby[*]{|\textit{main}|}| (with a non-empty optional argument)
which uses the |.aux| file of the main document
by setting |\jobname| to \textit{main}.

%%%%%%%%%%%%%%%%%%%%%%%%%%%%%%%%%%%%%%%%%%%%%%%%%%%%%%%%%%%%%%%%%%%%%%%%%%%%%%%%
\subsection{Driver Development}
\label{sec:driver}

The \textsf{childdoc} mechanism can also be use for the development
of definition files such as \LaTeX{} styles or classes.
This case differs from the above setup with multiple parts
included by |\include| in that no |\includeonly| should be invoked.
This can be achieved by starting the include file
(before |\ProvidesPackage|) with:
%
\begin{center}
\begin{tabular}{l}
|\input{childdoc.def}|\\
|\childdocforward{|\textit{main}|}|\\
\end{tabular}
\end{center}
%
or alternatively with:
%
\begin{center}
\begin{tabular}{l}
|\input{childdoc.def}|\\
|\childdocby{|\textit{main}|}|\\
\end{tabular}
\end{center}
%
Both forms have slightly different effects as described above.
The main file is prepared as usual, see \secref{sec:include}.

%%%%%%%%%%%%%%%%%%%%%%%%%%%%%%%%%%%%%%%%%%%%%%%%%%%%%%%%%%%%%%%%%%%%%%%%%%%%%%%%
\subsection{Legacy Detection}
\label{sec:detection}

The directive |\childdocmain| in the main file can detect
whether the complete document or merely a child is to be compiled
even without using the directive |\childdocof|.
This method is deprecated because it is less robust
and there is no compelling reason to use it;
it is merely provided for backward compatibility
and it may be removed in future versions.

If the detection mechanism is to be used,
it is mandatory to correctly specify
the filename of the main file as the argument of |\childdocmain|:
%
\begin{center}
\begin{tabular}{l}
|\input{childdoc.def}|\\
|\childdocmain{|\textit{main}|}|\\
\end{tabular}
\end{center}
%
If |\jobname| does not match the argument \textit{main} of |\childdocmain|,
it is assumed that |\jobname| points to the child file to be compiled.
When using |\childdocmain| with the main file specified as argument,
it suffices to start a child file
with just |\input{|\textit{main}|}|
without loading of the package and using |\childdocof|.
If instead all processing is done
with the appropriate \textsf{childdoc} directives,
the argument of \textit{main} of |\childdocmain| can be empty.

An alternative version of the command line processing described
in \secref{sec:commandline} using the detection mechanism reads:
%
\begin{center}
|... -jobname "|\textit{target}|" "|[\textit{flags}]%
[|\def\jobname{|\textit{dest}|}|]|\input{|\textit{main}|}"|
\end{center}

%%%%%%%%%%%%%%%%%%%%%%%%%%%%%%%%%%%%%%%%%%%%%%%%%%%%%%%%%%%%%%%%%%%%%%%%%%%%%%%%
\subsection{Manual Code}
\label{sec:manual}

In case one cannot be certain whether the definitions file |childdoc.def|
is installed on the target \TeX{} distribution
and one prefers not to ship it,
it is conceivable to paste a few relevant commands into the sources.

To that end, drop all statements |\input{childdoc.def}|
and perform the replacements as outlined below.
Instead of |\childdocmain{|\textit{main}|}| add the following code
to the top of the main file:
%
\begin{center}
\begin{tabular}{l}
|\||ifdefined\childdocname\endinput\||fi\newif\ifchilddoc|\\
|\edef\childdocname{\scantokens\expandafter{\jobname\noexpand}}|\\
|\def\childdocmain{|\textit{main}|}\||ifx\childdocmain\childdocname\||else|\\
|\childdoctrue\includeonly{\childdocname}\let\jobname\childdocmain\||fi|\\
\end{tabular}
\end{center}
%
Instead of |\childdocof{|\textit{main}|}| just include the main file
at the top of each child file:
%
\begin{center}
|\input{|\textit{main}|}|
\end{center}
%
A simple redirection |\childdocforward{|\textit{dest}|}| is achieved by:
%
\begin{center}
|\def\jobname{|\textit{dest}|}\input{\jobname}|
\end{center}
%
The redirection with prefix
|\childdocforwardprefix[|\textit{prefix}|]{|\textit{dest}|}|
is accomplished by:
%
\begin{center}
\begin{tabular}{l}
|{\edef\jobname{\scantokens\expandafter{\jobname\noexpand}}|\\
|\def\redirectjob |\textit{prefix}|#1~~~{\gdef\jobname{|\textit{dest}|#1}}|\\
|\expandafter\redirectjob\jobname~~~}\input{\jobname}|
\end{tabular}
\end{center}

In an alternative approach,
child documents can be compiled by a specific command line
without additional code or specific definitions:
%
\begin{center}
|... -jobname "|\textit{target}|" "|[\textit{flags}]%
|\includeonly{|\textit{dest}|}\input{|\textit{main}|}"|
\end{center}
%

%%%%%%%%%%%%%%%%%%%%%%%%%%%%%%%%%%%%%%%%%%%%%%%%%%%%%%%%%%%%%%%%%%%%%%%%%%%%%%%%
%%%%%%%%%%%%%%%%%%%%%%%%%%%%%%%%%%%%%%%%%%%%%%%%%%%%%%%%%%%%%%%%%%%%%%%%%%%%%%%%
\section{Information}

%%%%%%%%%%%%%%%%%%%%%%%%%%%%%%%%%%%%%%%%%%%%%%%%%%%%%%%%%%%%%%%%%%%%%%%%%%%%%%%%
\subsection{Copyright}

Copyright \copyright{} 2017--2018 Niklas Beisert

This work may be distributed and/or modified under the
conditions of the \LaTeX{} Project Public License, either version 1.3
of this license or (at your option) any later version.
The latest version of this license is in
  \url{http://www.latex-project.org/lppl.txt}
and version 1.3 or later is part of all distributions of \LaTeX{}
version 2005/12/01 or later.

This work has the LPPL maintenance status `maintained'.

The Current Maintainer of this work is Niklas Beisert.

This work consists of the files |README.txt|, |childdoc.ins| and |childdoc.dtx|
as well as the derived files |childdoc.def|, |cdocsamp.tex|
with |cdocsch1.tex|, |cdocsch2.tex|, |cdocspt3.tex|, |cdocspt4.tex|,
|cdocsdrf.tex|, |cdocsfn1.tex|, |cdocsfn2.tex|
as well as |childdoc.pdf|.

%%%%%%%%%%%%%%%%%%%%%%%%%%%%%%%%%%%%%%%%%%%%%%%%%%%%%%%%%%%%%%%%%%%%%%%%%%%%%%%%
\subsection{Files and Installation}

The package consists of the files:
%
\begin{center}
\begin{tabular}{ll}
    |README.txt|   & readme file \\
    |childdoc.ins| & installation file \\
    |childdoc.dtx| & source file \\
    |childdoc.def| & definition file \\
    |cdocsamp.tex| & sample main file \\
    |cdocsch1.tex| & sample include file \\
    |cdocsch2.tex| & sample include file \\
    |cdocspt3.tex| & sample part file \\
    |cdocspt4.tex| & sample part file \\
    |cdocsdrf.tex| & sample redirection file \\
    |cdocsfn1.tex| & sample redirection file \\
    |cdocsfn2.tex| & sample redirection file \\
    |childdoc.pdf| & manual
\end{tabular}
\end{center}
%
The distribution consists of the files
|README.txt|, |childdoc.ins| and |childdoc.dtx|.
%
\begin{itemize}
\item
Run (pdf)\LaTeX{} on |childdoc.dtx|
to compile the manual |childdoc.pdf| (this file).
\item
Run \LaTeX{} on |childdoc.ins| to create the definitions file |childdoc.def|
and the sample |cdocsamp.tex| with include files
|cdocsch1.tex|, |cdocsch2.tex|, |cdocspt3.tex|, |cdocspt4.tex|,
|cdocsdrf.tex|, |cdocsfn1.tex|, |cdocsfn2.tex|.
Then copy the file |childdoc.def| to an appropriate directory of your \LaTeX{}
distribution, e.g.\ \textit{texmf-root}|/tex/latex/childdoc|.
\end{itemize}

%%%%%%%%%%%%%%%%%%%%%%%%%%%%%%%%%%%%%%%%%%%%%%%%%%%%%%%%%%%%%%%%%%%%%%%%%%%%%%%%
\subsection{Related CTAN Packages}

There are several other packages which offer a similar functionality:
%
\begin{itemize}
\item
The packages
\href{http://ctan.org/pkg/docmute}{\textsf{docmute}},
\href{http://ctan.org/pkg/includex}{\textsf{includex}} and
\href{http://ctan.org/pkg/standalone}{\textsf{standalone}}
provide commands to include only the document body of
a child file thus allowing both files to be compiled individually.
\item
The packages \href{http://ctan.org/pkg/subdocs}{\textsf{subdocs}}
and \href{http://ctan.org/pkg/subfiles}{\textsf{subfiles}}
provide structures in which the main and child documents can be
encapsulated and allowing them to be compiled individually.
The inclusion mechanism is different from the conventional |\include|.
\item
The package \href{http://ctan.org/pkg/combine}{\textsf{combine}}
is an elaborate solution to combine several documents into one.
\end{itemize}
%
See also the CTAN topic \href{http://ctan.org/topic/subdocs}{\textsf{subdocs}}
for further related packages.
The present package differs from the above solutions in that
a document structure constructed with the conventional |\include| mechanism
just needs two extra commands at the top of every file
such that all constituent files can be compiled individually.

%%%%%%%%%%%%%%%%%%%%%%%%%%%%%%%%%%%%%%%%%%%%%%%%%%%%%%%%%%%%%%%%%%%%%%%%%%%%%%%%
%\subsection{Feature Suggestions}
%
%The following is a list of features which may be useful for future
%versions of this package:
%%
%\begin{itemize}
%\item
%\ldots
%\end{itemize}

%%%%%%%%%%%%%%%%%%%%%%%%%%%%%%%%%%%%%%%%%%%%%%%%%%%%%%%%%%%%%%%%%%%%%%%%%%%%%%%%
\subsection{Revision History}

%%%%%%%%%%%%%%%%%%%%%%%%%%%%%%%%%%%%%%%%
\paragraph{v2.0:} 2018/12/30

\begin{itemize}
\item
immediate forward processing
\item
added |\childdocby| mechanism
\item
manual restructured
\end{itemize}

%%%%%%%%%%%%%%%%%%%%%%%%%%%%%%%%%%%%%%%%
\paragraph{v1.6:} 2018/01/17

\begin{itemize}
\item
application for development of include files
\item
corrections to manual
\end{itemize}

%%%%%%%%%%%%%%%%%%%%%%%%%%%%%%%%%%%%%%%%
\paragraph{v1.5:} 2017/05/21

\begin{itemize}
\item
more complete structuring introduced
\item
|\childdocof| introduced
\item
|\childdoc| renamed to |\childdocmain|
\item
|\childredirect| renamed to |\childdocforward| and |\childdocforwardprefix|
and functionality expanded
\end{itemize}

%%%%%%%%%%%%%%%%%%%%%%%%%%%%%%%%%%%%%%%%
\paragraph{v1.0:} 2017/04/27

\begin{itemize}
\item
manual and install package
\item
first version published on CTAN
\end{itemize}

%%%%%%%%%%%%%%%%%%%%%%%%%%%%%%%%%%%%%%%%
\paragraph{v0.6:} 2017/04/26

\begin{itemize}
\item
redirection mechanism added
\end{itemize}

%%%%%%%%%%%%%%%%%%%%%%%%%%%%%%%%%%%%%%%%
\paragraph{v0.5:} 2017/04/26

\begin{itemize}
\item
functionality in definition file
\end{itemize}


%%%%%%%%%%%%%%%%%%%%%%%%%%%%%%%%%%%%%%%%%%%%%%%%%%%%%%%%%%%%%%%%%%%%%%%%%%%%%%%%
%%%%%%%%%%%%%%%%%%%%%%%%%%%%%%%%%%%%%%%%%%%%%%%%%%%%%%%%%%%%%%%%%%%%%%%%%%%%%%%%
%%%%%%%%%%%%%%%%%%%%%%%%%%%%%%%%%%%%%%%%%%%%%%%%%%%%%%%%%%%%%%%%%%%%%%%%%%%%%%%%
\appendix

\settowidth\MacroIndent{\rmfamily\scriptsize 000\ }

 \DocInput{childdoc.dtx}

\end{document}
%</driver>
% \fi
%
% %%%%%%%%%%%%%%%%%%%%%%%%%%%%%%%%%%%%%%%%%%%%%%%%%%%%%%%%%%%%%%%%%%%%%%%%%%%%%%
% %%%%%%%%%%%%%%%%%%%%%%%%%%%%%%%%%%%%%%%%%%%%%%%%%%%%%%%%%%%%%%%%%%%%%%%%%%%%%%
% \section{Sample}
%\iffalse
%<*samplemain>
%\fi
%
% The following presents a sample document
% with two chapters, two parts, a title page,
% a compile flag as well as three forwarding files to set the flag.
% It consists of eight |.tex| files:
% \begin{center}
% \begin{tabular}{ll}
% |cdocsamp.tex|&main file\\
% |cdocsch1.tex|&include file for chapter 1\\
% |cdocsch2.tex|&include file for chapter 2\\
% |cdocspt3.tex|&include file for part 3\\
% |cdocspt4.tex|&include file for part 4\\
% |cdocsdrf.tex|&forwarding file for main file in draft mode\\
% |cdocsfi1.tex|&forwarding file for final version of chapter 1\\
% |cdocsfi2.tex|&forwarding file for final version of chapter 2\\
% \end{tabular}
% \end{center}
% Each of the eight files can be compiled directly by the \LaTeX{} compiler.
%
% %%%%%%%%%%%%%%%%%%%%%%%%%%%%%%%%%%%%%%
% \paragraph{Main File.}
%
% The main file is called |cdocsamp.tex|.
%
% Load the \textsf{childdoc} definitions and
% declare the filename for the main document:
%    \begin{macrocode}
\input{childdoc.def}
\childdocmain{}
%    \end{macrocode}

% Optional override for |\version| flag:
%    \begin{macrocode}
%%\ifchilddoc\else\providecommand{\version}{draft}\fi
%    \end{macrocode}

% Define the default values for the |\version| flag
% (|final| for the main file and |draft| for childs):
%    \begin{macrocode}
\ifchilddoc
\providecommand{\version}{draft}
\else
\providecommand{\version}{final}
\fi
%    \end{macrocode}

% Load the standard document class:
%    \begin{macrocode}
\documentclass[12pt]{article}
%    \end{macrocode}

% Start the document body:
%    \begin{macrocode}
\begin{document}
%    \end{macrocode}

% Declare a title page.
% Print title, part of document being processed and version flag:
%    \begin{macrocode}
\addtocounter{page}{-1}
\begin{center}
{\LARGE\bfseries{}childdoc example\par}
\vspace{1cm}
\ifchilddoc
\ifchilddocmanual part\else chapter\fi:
`\childdocname' of `\childdocjob'\par
\else
main document: `\childdocjob'\par
\fi
version: \version\par
\end{center}
\newpage
%    \end{macrocode}

% Manually include selected file,
% otherwise process as usual:
%    \begin{macrocode}
\ifchilddocmanual
\section*{part `\childdocname'}
\input{\childdocname}
\else
%    \end{macrocode}

% Include the two chapters:
%    \begin{macrocode}
\include{cdocsch1}
\include{cdocsch2}
%    \end{macrocode}

% Include the two parts unless only chapters should be displayed:
%    \begin{macrocode}
\ifchilddoc\else
\section{part three}
\input{cdocspt3}
\section{part four}
\input{cdocspt4}
\fi
%    \end{macrocode}

% Process as usual until here:
%    \begin{macrocode}
\fi
%    \end{macrocode}

% End of document body:
%    \begin{macrocode}
\end{document}
%    \end{macrocode}
%\iffalse
%</samplemain>
%\fi
%
% %%%%%%%%%%%%%%%%%%%%%%%%%%%%%%%%%%%%%%
% \paragraph{Chapter Include Files.}
%
% The include files are called |cdocsch1.tex| and |cdocsch2.tex|.
%
%\iffalse
%<*samplechap1|samplechap2>
%\fi

% Optional override for |\version| flag:
%    \begin{macrocode}
%%\providecommand{\version}{final}
%    \end{macrocode}

% Include the main document:
%    \begin{macrocode}
\input{childdoc.def}
\childdocof{cdocsamp}
%    \end{macrocode}

%\iffalse
%</samplechap1|samplechap2>
%\fi
%
%\iffalse
%<*samplechap1>
%\fi
% Some text for chapter 1:
%    \begin{macrocode}
\section{one}
some text in chapter one
%    \end{macrocode}

%\iffalse
%</samplechap1>
%\fi
% Some text for chapter 2:
%\iffalse
%<*samplechap2>
%\fi
%    \begin{macrocode}
\section{two}
more text in chapter two
%    \end{macrocode}

%\iffalse
%</samplechap2>
%\fi
%
% %%%%%%%%%%%%%%%%%%%%%%%%%%%%%%%%%%%%%%
% \paragraph{Part Include Files.}
%
% The include files are called |cdocspt3.tex| and |cdocspt4.tex|.
%
%\iffalse
%<*samplepart3|samplepart4>
%\fi

% Optional override for |\version| flag:
%    \begin{macrocode}
%%\providecommand{\version}{final}
%    \end{macrocode}

% Include the main document:
%    \begin{macrocode}
\input{childdoc.def}
\childdocby{cdocsamp}
%    \end{macrocode}

%\iffalse
%</samplepart3|samplepart4>
%\fi
%
%\iffalse
%<*samplepart3>
%\fi
% Some text for part 3:
%    \begin{macrocode}
some text in part three
%    \end{macrocode}

%\iffalse
%</samplepart3>
%\fi
% Some text for part 4:
%\iffalse
%<*samplepart4>
%\fi
%    \begin{macrocode}
more text in part four
%    \end{macrocode}

%\iffalse
%</samplepart4>
%\fi
%
% %%%%%%%%%%%%%%%%%%%%%%%%%%%%%%%%%%%%%%
% \paragraph{Forwarding for a Complete Draft.}
%
% The following forwarding file |cdocsdrf.tex|
% compiles the main document in draft mode:
%\iffalse
%<*sampledraft>
%\fi
%    \begin{macrocode}
\def\version{draft}
\input{childdoc.def}
\childdocforward{cdocsamp}
%    \end{macrocode}

%\iffalse
%</sampledraft>
%\fi
%
% %%%%%%%%%%%%%%%%%%%%%%%%%%%%%%%%%%%%%%
% \paragraph{Forwarding for Final Version of the Chapters.}
%
% The following forwarding files |cdocsfn1.tex| and |cdocsfn2.tex|
% (with identical content)
% compile the final versions of the child documents
% |cdocsch1.tex| and |cdocsch2.tex|, respectively:
%\iffalse
%<*samplefinal>
%\fi
%    \begin{macrocode}
\def\version{final}
\input{childdoc.def}
\childdocforwardprefix[cdocsamp]{cdocsfn}{cdocsch}
%    \end{macrocode}

%\iffalse
%</samplefinal>
%\fi
%
% %%%%%%%%%%%%%%%%%%%%%%%%%%%%%%%%%%%%%%
% \paragraph{Command Line Processing.}
%
% The following three command lines generate the output files
% |cdocscld|, |cdocscl1| and |cdocscl2|
% which should be identical to
% |cdocsdrf|, |cdocsch1| and |cdocsfn2|, respectively:
% \begin{center}
% \begin{tabular}{l}
% |latex -jobname cdocscld \|\\
% |  "\def\version{draft}\input{childdoc.def}\childdocforward{cdocsamp}"|\\
% |latex -jobname cdocscl1 \|\\
% |  "\input{childdoc.def}\childdocforward[cdocsamp]{cdocsch1}"|\\
% |latex -jobname cdocscl2 \|\\
% |  "\def\version{final}\input{childdoc.def}\childdocforward{cdocsch2}"|
% \end{tabular}
% \end{center}
% Note that the trailing backslash on each first line
% merely continues the input to the second line
% (for convenient cut ant paste).
% Furthermore, the command |latex| can be replaced by any
% of its alternative versions such as |pdflatex|.
%
% %%%%%%%%%%%%%%%%%%%%%%%%%%%%%%%%%%%%%%%%%%%%%%%%%%%%%%%%%%%%%%%%%%%%%%%%%%%%%%
% %%%%%%%%%%%%%%%%%%%%%%%%%%%%%%%%%%%%%%%%%%%%%%%%%%%%%%%%%%%%%%%%%%%%%%%%%%%%%%
% \section{Implementation}
%\iffalse
%<*package>
%\fi
%
% This section describes the definitions file |childdoc.def|.

% The definitions cannot be loaded using |\usepackage| or |\RequirePackage|
% which has a mechanism to prevent loading a style file more than once.
% When loading the definitions by means of |\input|
% multiple instances have to be prevented manually:
%\iffalse
%This code needs to be before the `\ProvidesFile' directive
%which is defined at the beginning of this file.
%Therefore it is also placed there and commented out here.
%</package>
%<*discard>
%\fi
%    \begin{macrocode}
\ifdefined\childdocmain\endinput\fi
%    \end{macrocode}
%\iffalse
%</discard>
%<*package>
%\fi
%
% \macro{\ifchilddoc}
% \macro{\ifchilddocmanual}
% The conditional |\ifchilddoc| tells whether a
% child (true) or main (false) document is being compiled.
% The conditional |\ifchilddocmanual| tells whether
% the |\includeonly| mechanism is used (false) or
% the selection of child files must be performed manually (true).
% The definitions initialise to false:
%    \begin{macrocode}
\newif\ifchilddoc
\newif\ifchilddocmanual
%    \end{macrocode}

% \macro{\childdocname}
% \macro{\childdocjob}
% The macro |\childdocname| stores the name of the main document
% to be compiled. The macro |\childdocjob| stores the name of
% the document on which the \LaTeX{} compiler was originally invoked.
% The content of |\jobname| cannot be compared
% to filenames specified in the source due to different catcodes.
% The following code rescans |\jobname|, stores the result
% in |\childdocname| and saves a copy in |\childdocjob|:
%    \begin{macrocode}
\edef\childdocname{\scantokens\expandafter{\jobname\noexpand}}
\let\childdocjob\childdocname
%    \end{macrocode}

% \macro{\childdocdisable}
% The macro |\childdocdisable| prevents the main file
% from being processed more than once.
% At this stage, the main document command |\childdocmain|
% is assumed to be called once again where it should do nothing.
% Any subsequent call to it should prevent
% a secondary processing of the main document
% It overwrites the forwarding commands
% |\childdocof| and |\childdocforward|
% with empty macros to prevent further inclusions of the main document:
%    \begin{macrocode}
\newcommand{\childdocdisable}
{
  \renewcommand{\childdocmain}[1]{\renewcommand{\childdocmain}[1]{\endinput}}
  \renewcommand{\childdocof}[1]{}
  \renewcommand{\childdocby}[2][]{}
  \renewcommand{\childdocforward}[2][]{}
  \renewcommand{\childdocdisable}{}
}
%    \end{macrocode}

% \macro{\childdocmain}
% The macro |\childdocmain| is to be called at the top of the main file
% with nothing or the main filename (without extension) as argument.
% First, it breaks loops.
% If the argument is not empty and does not match |\childdocname|
% (which is set by the first inclusion of |childdoc.def|),
% |\ifchilddoc| is set to true, |\includeonly| is applied to the child file
% and |\jobname| is set to the main file
% (for proper handling of |.aux| files):
%    \begin{macrocode}
\newcommand{\childdocmain}[1]
{
  \childdocdisable\childdocmain{}
  \if?#1?\else
    \begingroup
      \def\childdoctmp{#1}
      \ifx\childdoctmp\childdocname
        \def\childdoctmp{}
      \else
        \def\childdoctmp
        {
          \childdoctrue
          \includeonly{\childdocname}
          \def\childdocjob{#1}
          \def\jobname{#1}
        }
      \fi
      \expandafter
    \endgroup
    \childdoctmp
  \fi
}
%    \end{macrocode}

% \macro{\childdocof}
% The command |\childdocof| redirects
% compilation to the main file |#1|.
%    \begin{macrocode}
\newcommand{\childdocof}[1]
{
  \childdocdisable
  \childdoctrue
  \includeonly{\childdocname}
  \def\jobname{#1}
  \def\childdocjob{#1}
  \input{#1}
}
%    \end{macrocode}

% \macro{\childdocby}
% The command |\childdocby| ....
%    \begin{macrocode}
\newcommand{\childdocby}[2][]
{
  \childdocdisable
  \childdoctrue
  \childdocmanualtrue
  \if?#1?\else
    \def\jobname{#2}
  \fi
  \def\childdocjob{#2}
  \input{#2}
  \endinput
}
%    \end{macrocode}

% \macro{\childdocforward}
% The command |\childdocforward| redirects
% compilation to the main file or
% (if the optional argument is given) a child file.
% Parameters are set as if the main file
% or a child file starting with |\childdocof| was compiled.
% Then compilation is handed over to the main file:
%    \begin{macrocode}
\newcommand{\childdocforward}[2][]
{
  \begingroup
    \if?#1?
      \def\childdoctmp
      {
        \def\childdocname{#2}
        \def\childdocjob{#2}
        \def\jobname{#2}
        \input{#2}
        \endinput
      }
    \else
      \def\childdoctmp
      {
        \childdocdisable
        \def\childdocname{#2}
        \childdoctrue
        \includeonly{#2}
        \def\childdocjob{#1}
        \def\jobname{#1}
        \input{#1}
        \endinput
      }
    \fi
    \expandafter
  \endgroup
  \childdoctmp
}
%    \end{macrocode}

% \macro{\childdocforwardprefix}
% The command |\childdocforwardprefix| redirects
% compilation to the main or a child file by means of a pattern.
% The prefix |#1| in the current filename is replaced by |#2|
% and the suffix of the current filename is kept
% (it is assumed that the filename does not contain the substring `|~~~|'
% which is used as a delimiter).
% Compilation is handed over to the new file by |\childdocforward|:
%    \begin{macrocode}
\newcommand{\childdocforwardprefix}[3][]
{
  \begingroup
    \def\childdocextract #2##1~~~{\def\childdoctmp{\childdocforward[#1]{#3##1}}}
    \expandafter\childdocextract\childdocname~~~
    \expandafter
  \endgroup
  \childdoctmp
}
%    \end{macrocode}

% \macro{\childdoc}
% The deprecated macro |\childdoc| is a legacy version of |\childdocmain|:
%    \begin{macrocode}
\newcommand{\childdoc}{\childdocmain}
%    \end{macrocode}

% \macro{\childdocredirect}
% The deprecated macro |\childdocredirect| is a legacy version
% of |\childdocforward| and |\childdocforwardprefix|:
%    \begin{macrocode}
\newcommand{\childdocredirect}[2][]
{
  \begingroup
    \if?#1?
      \def\childdoctmp{\childdocforward{#2}}
    \else
      \def\childdoctmp{\childdocforwardprefix{#1}{#2}}
    \fi
    \expandafter
  \endgroup
  \childdoctmp
}
%    \end{macrocode}

%\iffalse
%</package>
%\fi
%
\endinput
|\\
|\childdocforwardprefix{final}{child}|
\end{tabular}
\end{center}
%

Note that when several versions of a main file and/or of each child file
are to be generated, it may be convenient to set up a |Makefile| or
shell script to automatise the process.

%%%%%%%%%%%%%%%%%%%%%%%%%%%%%%%%%%%%%%%%%%%%%%%%%%%%%%%%%%%%%%%%%%%%%%%%%%%%%%%%
\subsection{Command Line Processing}
\label{sec:commandline}

The effect of redirection files can also be achieved by invoking
the \LaTeX{} compiler with a more elaborate command line.
Most conveniently this should be done as part
of a shell script or a |Makefile|.

When using \textsf{childdoc} in the main file, the following
command lines effectively perform a redirection
(note that depending on the shell being used,
backslashes may have to be doubled: `|\|' $\to$ `|\\|'):
%
\begin{center}
|... -jobname "|\textit{target}|" |\\|"|[\textit{flags}]%
|% \iffalse
%
% childdoc.dtx Copyright (C) 2017-2018 Niklas Beisert
%
% This work may be distributed and/or modified under the
% conditions of the LaTeX Project Public License, either version 1.3
% of this license or (at your option) any later version.
% The latest version of this license is in
%   http://www.latex-project.org/lppl.txt
% and version 1.3 or later is part of all distributions of LaTeX
% version 2005/12/01 or later.
%
% This work has the LPPL maintenance status `maintained'.
%
% The Current Maintainer of this work is Niklas Beisert.
%
% This work consists of the files childdoc.dtx and childdoc.ins
% and the derived files childdoc.def and cdocsamp.tex with
% cdocsch1.tex, cdocsch2.tex, cdocsdrf.tex, cdocsfn1.tex, cdocsfn2.tex.
%
%<package>\ifdefined\childdocmain\endinput\fi
%<package>\ProvidesFile{childdoc.def}[2018/12/30 v2.0 child document driver]
%<samplemain>\ProvidesFile{cdocsamp.tex}[2018/12/30 v2.0 sample for childdoc]
%<*driver>
%\ProvidesFile{childdoc.drv}[2018/12/30 v2.0 childdoc reference manual file]
\PassOptionsToClass{10pt,a4paper}{article}
\documentclass{ltxdoc}

\usepackage[margin=35mm]{geometry}
\usepackage{hyperref}
\usepackage{hyperxmp}
\usepackage[usenames]{color}

\hypersetup{colorlinks=true}
\hypersetup{pdfstartview=FitH}
\hypersetup{pdfpagemode=UseNone}
\hypersetup{pdfsource={}}
\hypersetup{pdflang={en-UK}}
\hypersetup{pdfcopyright={Copyright 2017-2018 Niklas Beisert.
  This work may be distributed and/or modified under the
  conditions of the LaTeX Project Public License, either version 1.3
  of this license or (at your option) any later version.}}
\hypersetup{pdflicenseurl={http://www.latex-project.org/lppl.txt}}
\hypersetup{pdfcontactaddress={ETH Zurich, ITP, HIT K,
  Wolfgang-Pauli-Strasse 27}}
\hypersetup{pdfcontactpostcode={8093}}
\hypersetup{pdfcontactcity={Zurich}}
\hypersetup{pdfcontactcountry={Switzerland}}
\hypersetup{pdfcontactemail={nbeisert@itp.phys.ethz.ch}}
\hypersetup{pdfcontacturl={http://people.phys.ethz.ch/\xmptilde nbeisert/}}

\newcommand{\secref}[1]{\hyperref[#1]{section \ref*{#1}}}

\parskip1ex
\parindent0pt
\let\olditemize\itemize
\def\itemize{\olditemize\parskip0pt}

\begin{document}

\title{The \textsf{childdoc} Package}
\hypersetup{pdftitle={The childdoc Package}}
\author{Niklas Beisert\\[2ex]
  Institut f\"ur Theoretische Physik\\
  Eidgen\"ossische Technische Hochschule Z\"urich\\
  Wolfgang-Pauli-Strasse 27, 8093 Z\"urich, Switzerland\\[1ex]
  \href{mailto:nbeisert@itp.phys.ethz.ch}
  {\texttt{nbeisert@itp.phys.ethz.ch}}}
\hypersetup{pdfauthor={Niklas Beisert}}
\hypersetup{pdfsubject={Manual for the LaTeX2e Package childdoc}}
\date{30 December 2018, \textsf{v2.0}}
\maketitle

\begin{abstract}\noindent
\textsf{childdoc} is a \LaTeXe{} package
that enables the direct compilation
of document sections included by |\include|
to individual files.
\end{abstract}

\begingroup
\parskip0ex
\tableofcontents
\endgroup

%%%%%%%%%%%%%%%%%%%%%%%%%%%%%%%%%%%%%%%%%%%%%%%%%%%%%%%%%%%%%%%%%%%%%%%%%%%%%%%%
%%%%%%%%%%%%%%%%%%%%%%%%%%%%%%%%%%%%%%%%%%%%%%%%%%%%%%%%%%%%%%%%%%%%%%%%%%%%%%%%
\section{Introduction}

\LaTeX{} provides a mechanism to structure a large document (such as a book)
into a main file and several child files (containing the chapters)
using the |\include| command.
This mechanism is beneficial for documents
which span hundreds of pages in order to
make the source file(s) more manageable.
Moreover, compilation can be restricted to
selected child files by means of the |\includeonly| command.
The latter feature can be used to reduce the compilation time while editing
(this was significantly more useful in the earlier days of \LaTeX{})
or to generate a smaller document which is easier to navigate.
Another application of |\includeonly| is to generate
documents consisting of selected parts of the complete document.

However, there are a few drawbacks of the plain |\include| mechanism:
\begin{itemize}
\item
The child files cannot be compiled on their own,
they can only be compiled via the main file.
A naive editing environment
(such as a text editor with an option
to have the current file processed by \LaTeX)
may require one to switch to the main file before compiling;
attempting to compile the child file produces errors.
\item
The main file must be modified (each time)
to adjust the |\includeonly| command
to the present needs. This easily leaves the main file in a messy state.
\item
The generated document will always carry the filename
of the main document. This is inconvenient if
several child files are to be compiled and
to be kept for distribution.
\end{itemize}

The present package provides a simple interface
to make child files individually compilable by \LaTeX{}.
Compiling a child file then has the same effect as compiling
the main file with an |\includeonly| command
to select the appropriate child.
Moreover the generated document will carry the name of the child
rather than the main file.
This resolves all three above issues.

This feature is meant to make the editing of books,
thesis documents and lecture notes somewhat more convenient.
However, the package can also be used efficiently for
composing a series of documents (such as exercise sheets)
which are typically distributed individually.
It then assists the author in generating the individual documents
(potentially in different versions)
as well as a document containing the collected series.
Another application is in developing style files
or other kinds of included material
where compilation of the style file could redirect
to a sample or test file.

%%%%%%%%%%%%%%%%%%%%%%%%%%%%%%%%%%%%%%%%%%%%%%%%%%%%%%%%%%%%%%%%%%%%%%%%%%%%%%%%
%%%%%%%%%%%%%%%%%%%%%%%%%%%%%%%%%%%%%%%%%%%%%%%%%%%%%%%%%%%%%%%%%%%%%%%%%%%%%%%%
\section{Usage}

First of all, the package \textsf{childdoc} is \emph{not} a standard
\LaTeXe{} |.sty| style file! Therefore it needs to be invoked in
a non-standard way.

%%%%%%%%%%%%%%%%%%%%%%%%%%%%%%%%%%%%%%%%%%%%%%%%%%%%%%%%%%%%%%%%%%%%%%%%%%%%%%%%
\subsection{Included Files}
\label{sec:include}

%%%%%%%%%%%%%%%%%%%%%%%%%%%%%%%%%%%%%%%%
\DescribeMacro{\childdocmain}
To use the package, add the commands
\begin{center}
\begin{tabular}{l}
|\input{childdoc.def}|\\
|\childdocmain{}|\\
\end{tabular}
\end{center}
at the very top of the main \LaTeX{} file,
in particular \emph{before} the |\documentclass| statement!
The argument of |\childdocmain| should be left empty
(but it must be present).

%%%%%%%%%%%%%%%%%%%%%%%%%%%%%%%%%%%%%%%%
\DescribeMacro{\childdocof}
Furthermore, add the commands
\begin{center}
\begin{tabular}{l}
|\input{childdoc.def}|\\
|\childdocof{|\textit{main}|}|\\
\end{tabular}
\end{center}
at the top of every child file \textit{child}
which is included by |\include{|\textit{child}|}|
from within the main file
(or at least for those files to be compiled individually).
The argument \textit{main} must be the filename of the main file.

There are a couple of
considerations in setting up the main and child documents:

%%%%%%%%%%%%%%%%%%%%%%%%%%%%%%%%%%%%%%%%
\paragraph{Restrictions.}

Please note the following restrictions:
\begin{itemize}
\item
|\childdocmain| must be called with one argument \textit{main}
to ensure compatibility with earlier version of the package.
It must either be empty (|\childdocmain{}|)
or precisely match the filename of the main file in which it is specified.
See \secref{sec:detection} for further information.
\item
The filename \textit{main} must be specified without the |.tex| extension.
\item
The filename \textit{main} is case sensitive
(even in case-insensitive file systems)
due to internal string comparison.
\item
The argument \textit{main} should be fully expanded, it cannot be a macro.
\item
Subdirectories and special characters should be avoided in filenames.
\item
The command |\childdocmain{|\textit{main}|}| must be followed by a whitespace.
It should not be followed immediately by another command
or by a comment mark `|%|'.
This is because the \TeX{} parser reads the token immediately following
the argument of |\childdocmain| and puts it
at the beginning of every child section;
however, a white\-space is ignored.
\end{itemize}

%%%%%%%%%%%%%%%%%%%%%%%%%%%%%%%%%%%%%%%%
\paragraph{Content of Main File.}

It is advisable to place all content in the child files included by |\include|.
Any output contained in the main file will appear in all child documents
unless suppressed manually;
it cannot be suppressed automatically by the |\includeonly| directive
and thus should normally be avoided.
A method to include some content in the main file
by means of conditional processing is described in \secref{sec:conditional}.

%%%%%%%%%%%%%%%%%%%%%%%%%%%%%%%%%%%%%%%%
\paragraph{Page Numbering.}

When only a part of the document is compiled,
the appropriate numbering of pages
(as well as other status parameters)
is determined from the |.aux| files.
The latter contain information from previous passes.
However this information needs to propagate through
all intermediate child documents.
Therefore the page numbering in child documents may well
be inconsistent until the complete document is compiled at least once.

A useful (if unconventional) way to always ensure a consistent
page numbering is to restart the numbering in each child document
and denote the pages by `\textit{child}|.|\textit{page}'
where \textit{child} represents the chapter/section number of the child file.
This can be achieved by the command
|\numberwithin{page}{|\textit{child}|}|
of the \textsf{amsmath} package
where \textit{child} can be |chapter| or |section|
depending on the chosen structuring.
Alternatively, one can modify the macro |\thepage| appropriately
and reset the counter |page| at the start of each child file.

%%%%%%%%%%%%%%%%%%%%%%%%%%%%%%%%%%%%%%%%%%%%%%%%%%%%%%%%%%%%%%%%%%%%%%%%%%%%%%%%
\subsection{Conditional Processing}
\label{sec:conditional}

The package provides a mechanism to compile different versions
of a document. To customise the versions further some conditional processing
can come in handy to distinguish which version is being compiled.
The package provides two macros to describe the compilation context:

%%%%%%%%%%%%%%%%%%%%%%%%%%%%%%%%%%%%%%%%
\DescribeMacro{\ifchilddoc}
The conditional |\ifchilddoc| distinguishes between the compilation of
child documents and the main document:
%
\begin{center}
|\ifchilddoc |\textit{child-code}| |[|\||else |\textit{main-code}]| \||fi|
\end{center}

%%%%%%%%%%%%%%%%%%%%%%%%%%%%%%%%%%%%%%%%
\DescribeMacro{\childdocname}
\DescribeMacro{\childdocjob}
The macro |\childdocname| contains the filename (without extension)
of the main or child file being processed.
Note that |\childdocjob| will always contain the name of the main file.

%%%%%%%%%%%%%%%%%%%%%%%%%%%%%%%%%%%%%%%%
\paragraph{Title Page.}

Conditional processing can be used to include a title or banner page
in the main document when proper precautions are taken.
Importantly, the code in the main file should ensure that the page counter
(as well as other status parameters which are stored in the |.aux| files)
takes the same value after the conditional processing.
Otherwise the page numbers may take divergent values
depending on which part is compiled.

For example, a title page could be declared by:
%
\begin{center}
\begin{tabular}{l}
|\ifchilddoc\||else|\\
|\addtocounter{page}{-1}|\\
\textit{code for title page}\\
|\newpage|\\
|\||fi|
\end{tabular}
\end{center}
%
A banner page for the child documents can be generated by:
%
\begin{center}
\begin{tabular}{l}
|\ifchilddoc|\\
|\addtocounter{page}{-1}|\\
\textit{code for banner page}\\
|\newpage|\\
|\||fi|
\end{tabular}
\end{center}
%
Here one could write a message such as:
\begin{center}
|This is the part \childdocname{} of \childdocjob{}.|
\end{center}

%%%%%%%%%%%%%%%%%%%%%%%%%%%%%%%%%%%%%%%%%%%%%%%%%%%%%%%%%%%%%%%%%%%%%%%%%%%%%%%%
\subsection{Flags}
\label{sec:flags}

The package makes it easy to generate different versions
of the main or child documents.
To this end compilation flags can be defined
and assigned different default values.
They will be particularly useful in conjunction
with the forwarding mechanism described in \secref{sec:forward}.

For example, it may be useful to have a flag |\version|
which can be set to |draft| or |final|.
The document source will contain some conditional code
depending on the value of |\version|.
Suppose further, the flag should default to |final| for the main file
and to |draft| for child files
which is a natural assignment for editing the document.
This is achieved by placing the following code
in the preamble of the main document
(below the |\childdocmain| directive):
%
\begin{center}
\begin{tabular}{l}
|\ifchilddoc|\\
|\providecommand{\version}{draft}|\\
|\||else|\\
|\providecommand{\version}{final}|\\
|\||fi|
\end{tabular}
\end{center}
%
The definition by |\providecommand| makes sure
that previous definitions are not overwritten.
Further statements |\providecommand{\version}{...}|
can thus be added before the above code to override it.

For the main file, one might add a line
(between |\childdocmain| and the above block)
%
\begin{center}
|%\ifchilddoc\||else\providecommand{\version}{draft}\||fi|
\end{center}
%
which can be uncommented to produce a draft version.
Likewise one can add a line to the very top of a child file
(above the |\childdocof{|\textit{main}|}| directive)
%
\begin{center}
|%\providecommand{\version}{final}|
\end{center}
%
which can be uncommented to produce the final version of this child document.

%%%%%%%%%%%%%%%%%%%%%%%%%%%%%%%%%%%%%%%%%%%%%%%%%%%%%%%%%%%%%%%%%%%%%%%%%%%%%%%%
\subsection{Forwarding}
\label{sec:forward}

Different versions of the main or child documents
using compilation flags as described in \secref{sec:flags}
can be (permanently) stored in different files
for convenient compilation, viewing and distribution.
To this end, the package defines a command
to pass on compilation to a different file:

%%%%%%%%%%%%%%%%%%%%%%%%%%%%%%%%%%%%%%%%
\DescribeMacro{\childdocforward}
The command |\childdocforward| redirects processing to
another source file:
%
\begin{center}
\begin{tabular}{l}
|\input{childdoc.def}|\\
|\childdocforward[|\textit{main}|]{|\textit{dest}|}|\\
\end{tabular}
\end{center}
%
The argument \textit{dest} is the destination file
(without extension).
It should be the main file or one of the child files.
Note that further \textsf{childdoc} directives
such as |\childdocof| and |\childdocforward|
in the indicated file will be processed in this form.
The optional argument \textit{main}
passes on directly to the main file \textit{main}
while pretending to compile the child \textit{dest}.
This form behaves as if \textit{dest}
issues |\childdocof{|\textit{main}|}| right away,
and no further \textsf{childdoc} directives will be processed.

%%%%%%%%%%%%%%%%%%%%%%%%%%%%%%%%%%%%%%%%
\DescribeMacro{\...prefix}
In the alternative form |\childdocforwardprefix|,
%
\begin{center}
\begin{tabular}{l}
|\input{childdoc.def}|\\
|\childdocforwardprefix[|\textit{main}|]{|\textit{prefix}|}{|\textit{dest}|}|
\end{tabular}
\end{center}
%
the destination file is determined by a pattern
depending on the current file:
To make this work, the current file must be called
`{\textit{prefix}\hspace{0.2em}\textit{suffix}}'
with \textit{prefix} matching precisely the argument.
Processing is then passed on to the file
`{\textit{dest}\hspace{0.2em}\textit{suffix}}'.
Surely, the same effect is achieved by
directly specifying the
argument `{\textit{dest}\hspace{0.2em}\textit{suffix}}'
in the first form.
However, that requires to set up a different file
for each child. With the alternative form of the command
all these files can have exactly the same content
which simplifies setting them up and maintaining them.

For example, the following file |draft.tex|
with a compilation flag |\version| as described in \secref{sec:flags}
compiles the main document as a draft:
%
\begin{center}
\begin{tabular}{l}
|\def\version{draft}|\\
|\input{childdoc.def}|\\
|\childdocforward{|\textit{main}|}|
\end{tabular}
\end{center}
%
Likewise, the following files |final|\textit{nn}|.tex|
compile the final version of the child document
|child|\textit{nn}|.tex|:
%
\begin{center}
\begin{tabular}{l}
|\def\version{final}|\\
|\input{childdoc.def}|\\
|\childdocforwardprefix{final}{child}|
\end{tabular}
\end{center}
%

Note that when several versions of a main file and/or of each child file
are to be generated, it may be convenient to set up a |Makefile| or
shell script to automatise the process.

%%%%%%%%%%%%%%%%%%%%%%%%%%%%%%%%%%%%%%%%%%%%%%%%%%%%%%%%%%%%%%%%%%%%%%%%%%%%%%%%
\subsection{Command Line Processing}
\label{sec:commandline}

The effect of redirection files can also be achieved by invoking
the \LaTeX{} compiler with a more elaborate command line.
Most conveniently this should be done as part
of a shell script or a |Makefile|.

When using \textsf{childdoc} in the main file, the following
command lines effectively perform a redirection
(note that depending on the shell being used,
backslashes may have to be doubled: `|\|' $\to$ `|\\|'):
%
\begin{center}
|... -jobname "|\textit{target}|" |\\|"|[\textit{flags}]%
|\input{childdoc.def}\childdocforward[|\textit{main}|]{|\textit{dest}|}"|
\end{center}
%
Here \textit{target} is the name of the output file,
\textit{main} is the name of the main file
and \textit{dest} is the name of the main or child file to be processed
(all filenames without extensions).
The optional argument \textit{main} can be omitted
if \textit{main} matches \textit{dest}.
Optionally, compilation \textit{flags} can be defined via |\def| commands.
This command line makes the \TeX{} engine believe
it is compiling the file \textit{target}
whose content is specified as the latter parameter.
The provided code then forwards the processing to
\textit{main} or \textit{dest} as described in \secref{sec:forward}.

%%%%%%%%%%%%%%%%%%%%%%%%%%%%%%%%%%%%%%%%%%%%%%%%%%%%%%%%%%%%%%%%%%%%%%%%%%%%%%%%
\subsection{Include by Input}
\label{sec:input}

Including child documents by |\include| has some restrictions by design.
Most notably, the content of a child document always occupies
its own set of pages; pages cannot be shared between child documents.
Usually, this behaviour makes perfect sense
because each child document contain an essential part of the document.
However, in some situations it may be desirable to compose
a document from a collection of parts
without having mandatory page breaks between then.
For this case, the package
provides a mechanism to include parts
by |\input| which can also be processed individually.
However, by construction this mechanism
requires manual handling of the content to be output.

%%%%%%%%%%%%%%%%%%%%%%%%%%%%%%%%%%%%%%%%
\DescribeMacro{\ifchilddocmanual}
The main file should be prepared as usual, see \secref{sec:include}.
However, the document body must make a distinction
between processing of an individual part and of the main document, e.g.:
%
\begin{center}
\begin{tabular}{l}
|\ifchilddocmanual|\\
|\input{\childdocname}|\\
|\||else|\\
\textit{document body with }|\input{|\textit{part}|}|\\
|\||fi|
\end{tabular}
\end{center}
%
The conditional |\ifchilddocmanual| is true whenever
a part to be included by |\input| is being compiled,
and the name of the part is stored in |\childdocname|.

%%%%%%%%%%%%%%%%%%%%%%%%%%%%%%%%%%%%%%%%
\DescribeMacro{\childdocby}
Each part to be included by |\input| should start with:
%
\begin{center}
\begin{tabular}{l}
|\input{childdoc.def}|\\
|\childdocby{|\textit{main}|}|\\
\end{tabular}
\end{center}
%
The directive |\childdocby| is similar to |\childdocof|
described in \secref{sec:include},
but the subsequent selection of content must be done manually.
To that end, both |\ifchilddoc| and |\ifchilddocmanual|
will be true upon processing of a part,
and the name of the part is stored in |\childdocname|.
Note that |\jobname| will be set to the filename of the current part
so that each part receives an individual |.aux| file
that does not interfere with the |.aux| file(s) of the main document.
This behaviour can be altered by the alternative form
|\childdocby[*]{|\textit{main}|}| (with a non-empty optional argument)
which uses the |.aux| file of the main document
by setting |\jobname| to \textit{main}.

%%%%%%%%%%%%%%%%%%%%%%%%%%%%%%%%%%%%%%%%%%%%%%%%%%%%%%%%%%%%%%%%%%%%%%%%%%%%%%%%
\subsection{Driver Development}
\label{sec:driver}

The \textsf{childdoc} mechanism can also be use for the development
of definition files such as \LaTeX{} styles or classes.
This case differs from the above setup with multiple parts
included by |\include| in that no |\includeonly| should be invoked.
This can be achieved by starting the include file
(before |\ProvidesPackage|) with:
%
\begin{center}
\begin{tabular}{l}
|\input{childdoc.def}|\\
|\childdocforward{|\textit{main}|}|\\
\end{tabular}
\end{center}
%
or alternatively with:
%
\begin{center}
\begin{tabular}{l}
|\input{childdoc.def}|\\
|\childdocby{|\textit{main}|}|\\
\end{tabular}
\end{center}
%
Both forms have slightly different effects as described above.
The main file is prepared as usual, see \secref{sec:include}.

%%%%%%%%%%%%%%%%%%%%%%%%%%%%%%%%%%%%%%%%%%%%%%%%%%%%%%%%%%%%%%%%%%%%%%%%%%%%%%%%
\subsection{Legacy Detection}
\label{sec:detection}

The directive |\childdocmain| in the main file can detect
whether the complete document or merely a child is to be compiled
even without using the directive |\childdocof|.
This method is deprecated because it is less robust
and there is no compelling reason to use it;
it is merely provided for backward compatibility
and it may be removed in future versions.

If the detection mechanism is to be used,
it is mandatory to correctly specify
the filename of the main file as the argument of |\childdocmain|:
%
\begin{center}
\begin{tabular}{l}
|\input{childdoc.def}|\\
|\childdocmain{|\textit{main}|}|\\
\end{tabular}
\end{center}
%
If |\jobname| does not match the argument \textit{main} of |\childdocmain|,
it is assumed that |\jobname| points to the child file to be compiled.
When using |\childdocmain| with the main file specified as argument,
it suffices to start a child file
with just |\input{|\textit{main}|}|
without loading of the package and using |\childdocof|.
If instead all processing is done
with the appropriate \textsf{childdoc} directives,
the argument of \textit{main} of |\childdocmain| can be empty.

An alternative version of the command line processing described
in \secref{sec:commandline} using the detection mechanism reads:
%
\begin{center}
|... -jobname "|\textit{target}|" "|[\textit{flags}]%
[|\def\jobname{|\textit{dest}|}|]|\input{|\textit{main}|}"|
\end{center}

%%%%%%%%%%%%%%%%%%%%%%%%%%%%%%%%%%%%%%%%%%%%%%%%%%%%%%%%%%%%%%%%%%%%%%%%%%%%%%%%
\subsection{Manual Code}
\label{sec:manual}

In case one cannot be certain whether the definitions file |childdoc.def|
is installed on the target \TeX{} distribution
and one prefers not to ship it,
it is conceivable to paste a few relevant commands into the sources.

To that end, drop all statements |\input{childdoc.def}|
and perform the replacements as outlined below.
Instead of |\childdocmain{|\textit{main}|}| add the following code
to the top of the main file:
%
\begin{center}
\begin{tabular}{l}
|\||ifdefined\childdocname\endinput\||fi\newif\ifchilddoc|\\
|\edef\childdocname{\scantokens\expandafter{\jobname\noexpand}}|\\
|\def\childdocmain{|\textit{main}|}\||ifx\childdocmain\childdocname\||else|\\
|\childdoctrue\includeonly{\childdocname}\let\jobname\childdocmain\||fi|\\
\end{tabular}
\end{center}
%
Instead of |\childdocof{|\textit{main}|}| just include the main file
at the top of each child file:
%
\begin{center}
|\input{|\textit{main}|}|
\end{center}
%
A simple redirection |\childdocforward{|\textit{dest}|}| is achieved by:
%
\begin{center}
|\def\jobname{|\textit{dest}|}\input{\jobname}|
\end{center}
%
The redirection with prefix
|\childdocforwardprefix[|\textit{prefix}|]{|\textit{dest}|}|
is accomplished by:
%
\begin{center}
\begin{tabular}{l}
|{\edef\jobname{\scantokens\expandafter{\jobname\noexpand}}|\\
|\def\redirectjob |\textit{prefix}|#1~~~{\gdef\jobname{|\textit{dest}|#1}}|\\
|\expandafter\redirectjob\jobname~~~}\input{\jobname}|
\end{tabular}
\end{center}

In an alternative approach,
child documents can be compiled by a specific command line
without additional code or specific definitions:
%
\begin{center}
|... -jobname "|\textit{target}|" "|[\textit{flags}]%
|\includeonly{|\textit{dest}|}\input{|\textit{main}|}"|
\end{center}
%

%%%%%%%%%%%%%%%%%%%%%%%%%%%%%%%%%%%%%%%%%%%%%%%%%%%%%%%%%%%%%%%%%%%%%%%%%%%%%%%%
%%%%%%%%%%%%%%%%%%%%%%%%%%%%%%%%%%%%%%%%%%%%%%%%%%%%%%%%%%%%%%%%%%%%%%%%%%%%%%%%
\section{Information}

%%%%%%%%%%%%%%%%%%%%%%%%%%%%%%%%%%%%%%%%%%%%%%%%%%%%%%%%%%%%%%%%%%%%%%%%%%%%%%%%
\subsection{Copyright}

Copyright \copyright{} 2017--2018 Niklas Beisert

This work may be distributed and/or modified under the
conditions of the \LaTeX{} Project Public License, either version 1.3
of this license or (at your option) any later version.
The latest version of this license is in
  \url{http://www.latex-project.org/lppl.txt}
and version 1.3 or later is part of all distributions of \LaTeX{}
version 2005/12/01 or later.

This work has the LPPL maintenance status `maintained'.

The Current Maintainer of this work is Niklas Beisert.

This work consists of the files |README.txt|, |childdoc.ins| and |childdoc.dtx|
as well as the derived files |childdoc.def|, |cdocsamp.tex|
with |cdocsch1.tex|, |cdocsch2.tex|, |cdocspt3.tex|, |cdocspt4.tex|,
|cdocsdrf.tex|, |cdocsfn1.tex|, |cdocsfn2.tex|
as well as |childdoc.pdf|.

%%%%%%%%%%%%%%%%%%%%%%%%%%%%%%%%%%%%%%%%%%%%%%%%%%%%%%%%%%%%%%%%%%%%%%%%%%%%%%%%
\subsection{Files and Installation}

The package consists of the files:
%
\begin{center}
\begin{tabular}{ll}
    |README.txt|   & readme file \\
    |childdoc.ins| & installation file \\
    |childdoc.dtx| & source file \\
    |childdoc.def| & definition file \\
    |cdocsamp.tex| & sample main file \\
    |cdocsch1.tex| & sample include file \\
    |cdocsch2.tex| & sample include file \\
    |cdocspt3.tex| & sample part file \\
    |cdocspt4.tex| & sample part file \\
    |cdocsdrf.tex| & sample redirection file \\
    |cdocsfn1.tex| & sample redirection file \\
    |cdocsfn2.tex| & sample redirection file \\
    |childdoc.pdf| & manual
\end{tabular}
\end{center}
%
The distribution consists of the files
|README.txt|, |childdoc.ins| and |childdoc.dtx|.
%
\begin{itemize}
\item
Run (pdf)\LaTeX{} on |childdoc.dtx|
to compile the manual |childdoc.pdf| (this file).
\item
Run \LaTeX{} on |childdoc.ins| to create the definitions file |childdoc.def|
and the sample |cdocsamp.tex| with include files
|cdocsch1.tex|, |cdocsch2.tex|, |cdocspt3.tex|, |cdocspt4.tex|,
|cdocsdrf.tex|, |cdocsfn1.tex|, |cdocsfn2.tex|.
Then copy the file |childdoc.def| to an appropriate directory of your \LaTeX{}
distribution, e.g.\ \textit{texmf-root}|/tex/latex/childdoc|.
\end{itemize}

%%%%%%%%%%%%%%%%%%%%%%%%%%%%%%%%%%%%%%%%%%%%%%%%%%%%%%%%%%%%%%%%%%%%%%%%%%%%%%%%
\subsection{Related CTAN Packages}

There are several other packages which offer a similar functionality:
%
\begin{itemize}
\item
The packages
\href{http://ctan.org/pkg/docmute}{\textsf{docmute}},
\href{http://ctan.org/pkg/includex}{\textsf{includex}} and
\href{http://ctan.org/pkg/standalone}{\textsf{standalone}}
provide commands to include only the document body of
a child file thus allowing both files to be compiled individually.
\item
The packages \href{http://ctan.org/pkg/subdocs}{\textsf{subdocs}}
and \href{http://ctan.org/pkg/subfiles}{\textsf{subfiles}}
provide structures in which the main and child documents can be
encapsulated and allowing them to be compiled individually.
The inclusion mechanism is different from the conventional |\include|.
\item
The package \href{http://ctan.org/pkg/combine}{\textsf{combine}}
is an elaborate solution to combine several documents into one.
\end{itemize}
%
See also the CTAN topic \href{http://ctan.org/topic/subdocs}{\textsf{subdocs}}
for further related packages.
The present package differs from the above solutions in that
a document structure constructed with the conventional |\include| mechanism
just needs two extra commands at the top of every file
such that all constituent files can be compiled individually.

%%%%%%%%%%%%%%%%%%%%%%%%%%%%%%%%%%%%%%%%%%%%%%%%%%%%%%%%%%%%%%%%%%%%%%%%%%%%%%%%
%\subsection{Feature Suggestions}
%
%The following is a list of features which may be useful for future
%versions of this package:
%%
%\begin{itemize}
%\item
%\ldots
%\end{itemize}

%%%%%%%%%%%%%%%%%%%%%%%%%%%%%%%%%%%%%%%%%%%%%%%%%%%%%%%%%%%%%%%%%%%%%%%%%%%%%%%%
\subsection{Revision History}

%%%%%%%%%%%%%%%%%%%%%%%%%%%%%%%%%%%%%%%%
\paragraph{v2.0:} 2018/12/30

\begin{itemize}
\item
immediate forward processing
\item
added |\childdocby| mechanism
\item
manual restructured
\end{itemize}

%%%%%%%%%%%%%%%%%%%%%%%%%%%%%%%%%%%%%%%%
\paragraph{v1.6:} 2018/01/17

\begin{itemize}
\item
application for development of include files
\item
corrections to manual
\end{itemize}

%%%%%%%%%%%%%%%%%%%%%%%%%%%%%%%%%%%%%%%%
\paragraph{v1.5:} 2017/05/21

\begin{itemize}
\item
more complete structuring introduced
\item
|\childdocof| introduced
\item
|\childdoc| renamed to |\childdocmain|
\item
|\childredirect| renamed to |\childdocforward| and |\childdocforwardprefix|
and functionality expanded
\end{itemize}

%%%%%%%%%%%%%%%%%%%%%%%%%%%%%%%%%%%%%%%%
\paragraph{v1.0:} 2017/04/27

\begin{itemize}
\item
manual and install package
\item
first version published on CTAN
\end{itemize}

%%%%%%%%%%%%%%%%%%%%%%%%%%%%%%%%%%%%%%%%
\paragraph{v0.6:} 2017/04/26

\begin{itemize}
\item
redirection mechanism added
\end{itemize}

%%%%%%%%%%%%%%%%%%%%%%%%%%%%%%%%%%%%%%%%
\paragraph{v0.5:} 2017/04/26

\begin{itemize}
\item
functionality in definition file
\end{itemize}


%%%%%%%%%%%%%%%%%%%%%%%%%%%%%%%%%%%%%%%%%%%%%%%%%%%%%%%%%%%%%%%%%%%%%%%%%%%%%%%%
%%%%%%%%%%%%%%%%%%%%%%%%%%%%%%%%%%%%%%%%%%%%%%%%%%%%%%%%%%%%%%%%%%%%%%%%%%%%%%%%
%%%%%%%%%%%%%%%%%%%%%%%%%%%%%%%%%%%%%%%%%%%%%%%%%%%%%%%%%%%%%%%%%%%%%%%%%%%%%%%%
\appendix

\settowidth\MacroIndent{\rmfamily\scriptsize 000\ }

 \DocInput{childdoc.dtx}

\end{document}
%</driver>
% \fi
%
% %%%%%%%%%%%%%%%%%%%%%%%%%%%%%%%%%%%%%%%%%%%%%%%%%%%%%%%%%%%%%%%%%%%%%%%%%%%%%%
% %%%%%%%%%%%%%%%%%%%%%%%%%%%%%%%%%%%%%%%%%%%%%%%%%%%%%%%%%%%%%%%%%%%%%%%%%%%%%%
% \section{Sample}
%\iffalse
%<*samplemain>
%\fi
%
% The following presents a sample document
% with two chapters, two parts, a title page,
% a compile flag as well as three forwarding files to set the flag.
% It consists of eight |.tex| files:
% \begin{center}
% \begin{tabular}{ll}
% |cdocsamp.tex|&main file\\
% |cdocsch1.tex|&include file for chapter 1\\
% |cdocsch2.tex|&include file for chapter 2\\
% |cdocspt3.tex|&include file for part 3\\
% |cdocspt4.tex|&include file for part 4\\
% |cdocsdrf.tex|&forwarding file for main file in draft mode\\
% |cdocsfi1.tex|&forwarding file for final version of chapter 1\\
% |cdocsfi2.tex|&forwarding file for final version of chapter 2\\
% \end{tabular}
% \end{center}
% Each of the eight files can be compiled directly by the \LaTeX{} compiler.
%
% %%%%%%%%%%%%%%%%%%%%%%%%%%%%%%%%%%%%%%
% \paragraph{Main File.}
%
% The main file is called |cdocsamp.tex|.
%
% Load the \textsf{childdoc} definitions and
% declare the filename for the main document:
%    \begin{macrocode}
\input{childdoc.def}
\childdocmain{}
%    \end{macrocode}

% Optional override for |\version| flag:
%    \begin{macrocode}
%%\ifchilddoc\else\providecommand{\version}{draft}\fi
%    \end{macrocode}

% Define the default values for the |\version| flag
% (|final| for the main file and |draft| for childs):
%    \begin{macrocode}
\ifchilddoc
\providecommand{\version}{draft}
\else
\providecommand{\version}{final}
\fi
%    \end{macrocode}

% Load the standard document class:
%    \begin{macrocode}
\documentclass[12pt]{article}
%    \end{macrocode}

% Start the document body:
%    \begin{macrocode}
\begin{document}
%    \end{macrocode}

% Declare a title page.
% Print title, part of document being processed and version flag:
%    \begin{macrocode}
\addtocounter{page}{-1}
\begin{center}
{\LARGE\bfseries{}childdoc example\par}
\vspace{1cm}
\ifchilddoc
\ifchilddocmanual part\else chapter\fi:
`\childdocname' of `\childdocjob'\par
\else
main document: `\childdocjob'\par
\fi
version: \version\par
\end{center}
\newpage
%    \end{macrocode}

% Manually include selected file,
% otherwise process as usual:
%    \begin{macrocode}
\ifchilddocmanual
\section*{part `\childdocname'}
\input{\childdocname}
\else
%    \end{macrocode}

% Include the two chapters:
%    \begin{macrocode}
\include{cdocsch1}
\include{cdocsch2}
%    \end{macrocode}

% Include the two parts unless only chapters should be displayed:
%    \begin{macrocode}
\ifchilddoc\else
\section{part three}
\input{cdocspt3}
\section{part four}
\input{cdocspt4}
\fi
%    \end{macrocode}

% Process as usual until here:
%    \begin{macrocode}
\fi
%    \end{macrocode}

% End of document body:
%    \begin{macrocode}
\end{document}
%    \end{macrocode}
%\iffalse
%</samplemain>
%\fi
%
% %%%%%%%%%%%%%%%%%%%%%%%%%%%%%%%%%%%%%%
% \paragraph{Chapter Include Files.}
%
% The include files are called |cdocsch1.tex| and |cdocsch2.tex|.
%
%\iffalse
%<*samplechap1|samplechap2>
%\fi

% Optional override for |\version| flag:
%    \begin{macrocode}
%%\providecommand{\version}{final}
%    \end{macrocode}

% Include the main document:
%    \begin{macrocode}
\input{childdoc.def}
\childdocof{cdocsamp}
%    \end{macrocode}

%\iffalse
%</samplechap1|samplechap2>
%\fi
%
%\iffalse
%<*samplechap1>
%\fi
% Some text for chapter 1:
%    \begin{macrocode}
\section{one}
some text in chapter one
%    \end{macrocode}

%\iffalse
%</samplechap1>
%\fi
% Some text for chapter 2:
%\iffalse
%<*samplechap2>
%\fi
%    \begin{macrocode}
\section{two}
more text in chapter two
%    \end{macrocode}

%\iffalse
%</samplechap2>
%\fi
%
% %%%%%%%%%%%%%%%%%%%%%%%%%%%%%%%%%%%%%%
% \paragraph{Part Include Files.}
%
% The include files are called |cdocspt3.tex| and |cdocspt4.tex|.
%
%\iffalse
%<*samplepart3|samplepart4>
%\fi

% Optional override for |\version| flag:
%    \begin{macrocode}
%%\providecommand{\version}{final}
%    \end{macrocode}

% Include the main document:
%    \begin{macrocode}
\input{childdoc.def}
\childdocby{cdocsamp}
%    \end{macrocode}

%\iffalse
%</samplepart3|samplepart4>
%\fi
%
%\iffalse
%<*samplepart3>
%\fi
% Some text for part 3:
%    \begin{macrocode}
some text in part three
%    \end{macrocode}

%\iffalse
%</samplepart3>
%\fi
% Some text for part 4:
%\iffalse
%<*samplepart4>
%\fi
%    \begin{macrocode}
more text in part four
%    \end{macrocode}

%\iffalse
%</samplepart4>
%\fi
%
% %%%%%%%%%%%%%%%%%%%%%%%%%%%%%%%%%%%%%%
% \paragraph{Forwarding for a Complete Draft.}
%
% The following forwarding file |cdocsdrf.tex|
% compiles the main document in draft mode:
%\iffalse
%<*sampledraft>
%\fi
%    \begin{macrocode}
\def\version{draft}
\input{childdoc.def}
\childdocforward{cdocsamp}
%    \end{macrocode}

%\iffalse
%</sampledraft>
%\fi
%
% %%%%%%%%%%%%%%%%%%%%%%%%%%%%%%%%%%%%%%
% \paragraph{Forwarding for Final Version of the Chapters.}
%
% The following forwarding files |cdocsfn1.tex| and |cdocsfn2.tex|
% (with identical content)
% compile the final versions of the child documents
% |cdocsch1.tex| and |cdocsch2.tex|, respectively:
%\iffalse
%<*samplefinal>
%\fi
%    \begin{macrocode}
\def\version{final}
\input{childdoc.def}
\childdocforwardprefix[cdocsamp]{cdocsfn}{cdocsch}
%    \end{macrocode}

%\iffalse
%</samplefinal>
%\fi
%
% %%%%%%%%%%%%%%%%%%%%%%%%%%%%%%%%%%%%%%
% \paragraph{Command Line Processing.}
%
% The following three command lines generate the output files
% |cdocscld|, |cdocscl1| and |cdocscl2|
% which should be identical to
% |cdocsdrf|, |cdocsch1| and |cdocsfn2|, respectively:
% \begin{center}
% \begin{tabular}{l}
% |latex -jobname cdocscld \|\\
% |  "\def\version{draft}\input{childdoc.def}\childdocforward{cdocsamp}"|\\
% |latex -jobname cdocscl1 \|\\
% |  "\input{childdoc.def}\childdocforward[cdocsamp]{cdocsch1}"|\\
% |latex -jobname cdocscl2 \|\\
% |  "\def\version{final}\input{childdoc.def}\childdocforward{cdocsch2}"|
% \end{tabular}
% \end{center}
% Note that the trailing backslash on each first line
% merely continues the input to the second line
% (for convenient cut ant paste).
% Furthermore, the command |latex| can be replaced by any
% of its alternative versions such as |pdflatex|.
%
% %%%%%%%%%%%%%%%%%%%%%%%%%%%%%%%%%%%%%%%%%%%%%%%%%%%%%%%%%%%%%%%%%%%%%%%%%%%%%%
% %%%%%%%%%%%%%%%%%%%%%%%%%%%%%%%%%%%%%%%%%%%%%%%%%%%%%%%%%%%%%%%%%%%%%%%%%%%%%%
% \section{Implementation}
%\iffalse
%<*package>
%\fi
%
% This section describes the definitions file |childdoc.def|.

% The definitions cannot be loaded using |\usepackage| or |\RequirePackage|
% which has a mechanism to prevent loading a style file more than once.
% When loading the definitions by means of |\input|
% multiple instances have to be prevented manually:
%\iffalse
%This code needs to be before the `\ProvidesFile' directive
%which is defined at the beginning of this file.
%Therefore it is also placed there and commented out here.
%</package>
%<*discard>
%\fi
%    \begin{macrocode}
\ifdefined\childdocmain\endinput\fi
%    \end{macrocode}
%\iffalse
%</discard>
%<*package>
%\fi
%
% \macro{\ifchilddoc}
% \macro{\ifchilddocmanual}
% The conditional |\ifchilddoc| tells whether a
% child (true) or main (false) document is being compiled.
% The conditional |\ifchilddocmanual| tells whether
% the |\includeonly| mechanism is used (false) or
% the selection of child files must be performed manually (true).
% The definitions initialise to false:
%    \begin{macrocode}
\newif\ifchilddoc
\newif\ifchilddocmanual
%    \end{macrocode}

% \macro{\childdocname}
% \macro{\childdocjob}
% The macro |\childdocname| stores the name of the main document
% to be compiled. The macro |\childdocjob| stores the name of
% the document on which the \LaTeX{} compiler was originally invoked.
% The content of |\jobname| cannot be compared
% to filenames specified in the source due to different catcodes.
% The following code rescans |\jobname|, stores the result
% in |\childdocname| and saves a copy in |\childdocjob|:
%    \begin{macrocode}
\edef\childdocname{\scantokens\expandafter{\jobname\noexpand}}
\let\childdocjob\childdocname
%    \end{macrocode}

% \macro{\childdocdisable}
% The macro |\childdocdisable| prevents the main file
% from being processed more than once.
% At this stage, the main document command |\childdocmain|
% is assumed to be called once again where it should do nothing.
% Any subsequent call to it should prevent
% a secondary processing of the main document
% It overwrites the forwarding commands
% |\childdocof| and |\childdocforward|
% with empty macros to prevent further inclusions of the main document:
%    \begin{macrocode}
\newcommand{\childdocdisable}
{
  \renewcommand{\childdocmain}[1]{\renewcommand{\childdocmain}[1]{\endinput}}
  \renewcommand{\childdocof}[1]{}
  \renewcommand{\childdocby}[2][]{}
  \renewcommand{\childdocforward}[2][]{}
  \renewcommand{\childdocdisable}{}
}
%    \end{macrocode}

% \macro{\childdocmain}
% The macro |\childdocmain| is to be called at the top of the main file
% with nothing or the main filename (without extension) as argument.
% First, it breaks loops.
% If the argument is not empty and does not match |\childdocname|
% (which is set by the first inclusion of |childdoc.def|),
% |\ifchilddoc| is set to true, |\includeonly| is applied to the child file
% and |\jobname| is set to the main file
% (for proper handling of |.aux| files):
%    \begin{macrocode}
\newcommand{\childdocmain}[1]
{
  \childdocdisable\childdocmain{}
  \if?#1?\else
    \begingroup
      \def\childdoctmp{#1}
      \ifx\childdoctmp\childdocname
        \def\childdoctmp{}
      \else
        \def\childdoctmp
        {
          \childdoctrue
          \includeonly{\childdocname}
          \def\childdocjob{#1}
          \def\jobname{#1}
        }
      \fi
      \expandafter
    \endgroup
    \childdoctmp
  \fi
}
%    \end{macrocode}

% \macro{\childdocof}
% The command |\childdocof| redirects
% compilation to the main file |#1|.
%    \begin{macrocode}
\newcommand{\childdocof}[1]
{
  \childdocdisable
  \childdoctrue
  \includeonly{\childdocname}
  \def\jobname{#1}
  \def\childdocjob{#1}
  \input{#1}
}
%    \end{macrocode}

% \macro{\childdocby}
% The command |\childdocby| ....
%    \begin{macrocode}
\newcommand{\childdocby}[2][]
{
  \childdocdisable
  \childdoctrue
  \childdocmanualtrue
  \if?#1?\else
    \def\jobname{#2}
  \fi
  \def\childdocjob{#2}
  \input{#2}
  \endinput
}
%    \end{macrocode}

% \macro{\childdocforward}
% The command |\childdocforward| redirects
% compilation to the main file or
% (if the optional argument is given) a child file.
% Parameters are set as if the main file
% or a child file starting with |\childdocof| was compiled.
% Then compilation is handed over to the main file:
%    \begin{macrocode}
\newcommand{\childdocforward}[2][]
{
  \begingroup
    \if?#1?
      \def\childdoctmp
      {
        \def\childdocname{#2}
        \def\childdocjob{#2}
        \def\jobname{#2}
        \input{#2}
        \endinput
      }
    \else
      \def\childdoctmp
      {
        \childdocdisable
        \def\childdocname{#2}
        \childdoctrue
        \includeonly{#2}
        \def\childdocjob{#1}
        \def\jobname{#1}
        \input{#1}
        \endinput
      }
    \fi
    \expandafter
  \endgroup
  \childdoctmp
}
%    \end{macrocode}

% \macro{\childdocforwardprefix}
% The command |\childdocforwardprefix| redirects
% compilation to the main or a child file by means of a pattern.
% The prefix |#1| in the current filename is replaced by |#2|
% and the suffix of the current filename is kept
% (it is assumed that the filename does not contain the substring `|~~~|'
% which is used as a delimiter).
% Compilation is handed over to the new file by |\childdocforward|:
%    \begin{macrocode}
\newcommand{\childdocforwardprefix}[3][]
{
  \begingroup
    \def\childdocextract #2##1~~~{\def\childdoctmp{\childdocforward[#1]{#3##1}}}
    \expandafter\childdocextract\childdocname~~~
    \expandafter
  \endgroup
  \childdoctmp
}
%    \end{macrocode}

% \macro{\childdoc}
% The deprecated macro |\childdoc| is a legacy version of |\childdocmain|:
%    \begin{macrocode}
\newcommand{\childdoc}{\childdocmain}
%    \end{macrocode}

% \macro{\childdocredirect}
% The deprecated macro |\childdocredirect| is a legacy version
% of |\childdocforward| and |\childdocforwardprefix|:
%    \begin{macrocode}
\newcommand{\childdocredirect}[2][]
{
  \begingroup
    \if?#1?
      \def\childdoctmp{\childdocforward{#2}}
    \else
      \def\childdoctmp{\childdocforwardprefix{#1}{#2}}
    \fi
    \expandafter
  \endgroup
  \childdoctmp
}
%    \end{macrocode}

%\iffalse
%</package>
%\fi
%
\endinput
\childdocforward[|\textit{main}|]{|\textit{dest}|}"|
\end{center}
%
Here \textit{target} is the name of the output file,
\textit{main} is the name of the main file
and \textit{dest} is the name of the main or child file to be processed
(all filenames without extensions).
The optional argument \textit{main} can be omitted
if \textit{main} matches \textit{dest}.
Optionally, compilation \textit{flags} can be defined via |\def| commands.
This command line makes the \TeX{} engine believe
it is compiling the file \textit{target}
whose content is specified as the latter parameter.
The provided code then forwards the processing to
\textit{main} or \textit{dest} as described in \secref{sec:forward}.

%%%%%%%%%%%%%%%%%%%%%%%%%%%%%%%%%%%%%%%%%%%%%%%%%%%%%%%%%%%%%%%%%%%%%%%%%%%%%%%%
\subsection{Include by Input}
\label{sec:input}

Including child documents by |\include| has some restrictions by design.
Most notably, the content of a child document always occupies
its own set of pages; pages cannot be shared between child documents.
Usually, this behaviour makes perfect sense
because each child document contain an essential part of the document.
However, in some situations it may be desirable to compose
a document from a collection of parts
without having mandatory page breaks between then.
For this case, the package
provides a mechanism to include parts
by |\input| which can also be processed individually.
However, by construction this mechanism
requires manual handling of the content to be output.

%%%%%%%%%%%%%%%%%%%%%%%%%%%%%%%%%%%%%%%%
\DescribeMacro{\ifchilddocmanual}
The main file should be prepared as usual, see \secref{sec:include}.
However, the document body must make a distinction
between processing of an individual part and of the main document, e.g.:
%
\begin{center}
\begin{tabular}{l}
|\ifchilddocmanual|\\
|\input{\childdocname}|\\
|\||else|\\
\textit{document body with }|\input{|\textit{part}|}|\\
|\||fi|
\end{tabular}
\end{center}
%
The conditional |\ifchilddocmanual| is true whenever
a part to be included by |\input| is being compiled,
and the name of the part is stored in |\childdocname|.

%%%%%%%%%%%%%%%%%%%%%%%%%%%%%%%%%%%%%%%%
\DescribeMacro{\childdocby}
Each part to be included by |\input| should start with:
%
\begin{center}
\begin{tabular}{l}
|% \iffalse
%
% childdoc.dtx Copyright (C) 2017-2018 Niklas Beisert
%
% This work may be distributed and/or modified under the
% conditions of the LaTeX Project Public License, either version 1.3
% of this license or (at your option) any later version.
% The latest version of this license is in
%   http://www.latex-project.org/lppl.txt
% and version 1.3 or later is part of all distributions of LaTeX
% version 2005/12/01 or later.
%
% This work has the LPPL maintenance status `maintained'.
%
% The Current Maintainer of this work is Niklas Beisert.
%
% This work consists of the files childdoc.dtx and childdoc.ins
% and the derived files childdoc.def and cdocsamp.tex with
% cdocsch1.tex, cdocsch2.tex, cdocsdrf.tex, cdocsfn1.tex, cdocsfn2.tex.
%
%<package>\ifdefined\childdocmain\endinput\fi
%<package>\ProvidesFile{childdoc.def}[2018/12/30 v2.0 child document driver]
%<samplemain>\ProvidesFile{cdocsamp.tex}[2018/12/30 v2.0 sample for childdoc]
%<*driver>
%\ProvidesFile{childdoc.drv}[2018/12/30 v2.0 childdoc reference manual file]
\PassOptionsToClass{10pt,a4paper}{article}
\documentclass{ltxdoc}

\usepackage[margin=35mm]{geometry}
\usepackage{hyperref}
\usepackage{hyperxmp}
\usepackage[usenames]{color}

\hypersetup{colorlinks=true}
\hypersetup{pdfstartview=FitH}
\hypersetup{pdfpagemode=UseNone}
\hypersetup{pdfsource={}}
\hypersetup{pdflang={en-UK}}
\hypersetup{pdfcopyright={Copyright 2017-2018 Niklas Beisert.
  This work may be distributed and/or modified under the
  conditions of the LaTeX Project Public License, either version 1.3
  of this license or (at your option) any later version.}}
\hypersetup{pdflicenseurl={http://www.latex-project.org/lppl.txt}}
\hypersetup{pdfcontactaddress={ETH Zurich, ITP, HIT K,
  Wolfgang-Pauli-Strasse 27}}
\hypersetup{pdfcontactpostcode={8093}}
\hypersetup{pdfcontactcity={Zurich}}
\hypersetup{pdfcontactcountry={Switzerland}}
\hypersetup{pdfcontactemail={nbeisert@itp.phys.ethz.ch}}
\hypersetup{pdfcontacturl={http://people.phys.ethz.ch/\xmptilde nbeisert/}}

\newcommand{\secref}[1]{\hyperref[#1]{section \ref*{#1}}}

\parskip1ex
\parindent0pt
\let\olditemize\itemize
\def\itemize{\olditemize\parskip0pt}

\begin{document}

\title{The \textsf{childdoc} Package}
\hypersetup{pdftitle={The childdoc Package}}
\author{Niklas Beisert\\[2ex]
  Institut f\"ur Theoretische Physik\\
  Eidgen\"ossische Technische Hochschule Z\"urich\\
  Wolfgang-Pauli-Strasse 27, 8093 Z\"urich, Switzerland\\[1ex]
  \href{mailto:nbeisert@itp.phys.ethz.ch}
  {\texttt{nbeisert@itp.phys.ethz.ch}}}
\hypersetup{pdfauthor={Niklas Beisert}}
\hypersetup{pdfsubject={Manual for the LaTeX2e Package childdoc}}
\date{30 December 2018, \textsf{v2.0}}
\maketitle

\begin{abstract}\noindent
\textsf{childdoc} is a \LaTeXe{} package
that enables the direct compilation
of document sections included by |\include|
to individual files.
\end{abstract}

\begingroup
\parskip0ex
\tableofcontents
\endgroup

%%%%%%%%%%%%%%%%%%%%%%%%%%%%%%%%%%%%%%%%%%%%%%%%%%%%%%%%%%%%%%%%%%%%%%%%%%%%%%%%
%%%%%%%%%%%%%%%%%%%%%%%%%%%%%%%%%%%%%%%%%%%%%%%%%%%%%%%%%%%%%%%%%%%%%%%%%%%%%%%%
\section{Introduction}

\LaTeX{} provides a mechanism to structure a large document (such as a book)
into a main file and several child files (containing the chapters)
using the |\include| command.
This mechanism is beneficial for documents
which span hundreds of pages in order to
make the source file(s) more manageable.
Moreover, compilation can be restricted to
selected child files by means of the |\includeonly| command.
The latter feature can be used to reduce the compilation time while editing
(this was significantly more useful in the earlier days of \LaTeX{})
or to generate a smaller document which is easier to navigate.
Another application of |\includeonly| is to generate
documents consisting of selected parts of the complete document.

However, there are a few drawbacks of the plain |\include| mechanism:
\begin{itemize}
\item
The child files cannot be compiled on their own,
they can only be compiled via the main file.
A naive editing environment
(such as a text editor with an option
to have the current file processed by \LaTeX)
may require one to switch to the main file before compiling;
attempting to compile the child file produces errors.
\item
The main file must be modified (each time)
to adjust the |\includeonly| command
to the present needs. This easily leaves the main file in a messy state.
\item
The generated document will always carry the filename
of the main document. This is inconvenient if
several child files are to be compiled and
to be kept for distribution.
\end{itemize}

The present package provides a simple interface
to make child files individually compilable by \LaTeX{}.
Compiling a child file then has the same effect as compiling
the main file with an |\includeonly| command
to select the appropriate child.
Moreover the generated document will carry the name of the child
rather than the main file.
This resolves all three above issues.

This feature is meant to make the editing of books,
thesis documents and lecture notes somewhat more convenient.
However, the package can also be used efficiently for
composing a series of documents (such as exercise sheets)
which are typically distributed individually.
It then assists the author in generating the individual documents
(potentially in different versions)
as well as a document containing the collected series.
Another application is in developing style files
or other kinds of included material
where compilation of the style file could redirect
to a sample or test file.

%%%%%%%%%%%%%%%%%%%%%%%%%%%%%%%%%%%%%%%%%%%%%%%%%%%%%%%%%%%%%%%%%%%%%%%%%%%%%%%%
%%%%%%%%%%%%%%%%%%%%%%%%%%%%%%%%%%%%%%%%%%%%%%%%%%%%%%%%%%%%%%%%%%%%%%%%%%%%%%%%
\section{Usage}

First of all, the package \textsf{childdoc} is \emph{not} a standard
\LaTeXe{} |.sty| style file! Therefore it needs to be invoked in
a non-standard way.

%%%%%%%%%%%%%%%%%%%%%%%%%%%%%%%%%%%%%%%%%%%%%%%%%%%%%%%%%%%%%%%%%%%%%%%%%%%%%%%%
\subsection{Included Files}
\label{sec:include}

%%%%%%%%%%%%%%%%%%%%%%%%%%%%%%%%%%%%%%%%
\DescribeMacro{\childdocmain}
To use the package, add the commands
\begin{center}
\begin{tabular}{l}
|\input{childdoc.def}|\\
|\childdocmain{}|\\
\end{tabular}
\end{center}
at the very top of the main \LaTeX{} file,
in particular \emph{before} the |\documentclass| statement!
The argument of |\childdocmain| should be left empty
(but it must be present).

%%%%%%%%%%%%%%%%%%%%%%%%%%%%%%%%%%%%%%%%
\DescribeMacro{\childdocof}
Furthermore, add the commands
\begin{center}
\begin{tabular}{l}
|\input{childdoc.def}|\\
|\childdocof{|\textit{main}|}|\\
\end{tabular}
\end{center}
at the top of every child file \textit{child}
which is included by |\include{|\textit{child}|}|
from within the main file
(or at least for those files to be compiled individually).
The argument \textit{main} must be the filename of the main file.

There are a couple of
considerations in setting up the main and child documents:

%%%%%%%%%%%%%%%%%%%%%%%%%%%%%%%%%%%%%%%%
\paragraph{Restrictions.}

Please note the following restrictions:
\begin{itemize}
\item
|\childdocmain| must be called with one argument \textit{main}
to ensure compatibility with earlier version of the package.
It must either be empty (|\childdocmain{}|)
or precisely match the filename of the main file in which it is specified.
See \secref{sec:detection} for further information.
\item
The filename \textit{main} must be specified without the |.tex| extension.
\item
The filename \textit{main} is case sensitive
(even in case-insensitive file systems)
due to internal string comparison.
\item
The argument \textit{main} should be fully expanded, it cannot be a macro.
\item
Subdirectories and special characters should be avoided in filenames.
\item
The command |\childdocmain{|\textit{main}|}| must be followed by a whitespace.
It should not be followed immediately by another command
or by a comment mark `|%|'.
This is because the \TeX{} parser reads the token immediately following
the argument of |\childdocmain| and puts it
at the beginning of every child section;
however, a white\-space is ignored.
\end{itemize}

%%%%%%%%%%%%%%%%%%%%%%%%%%%%%%%%%%%%%%%%
\paragraph{Content of Main File.}

It is advisable to place all content in the child files included by |\include|.
Any output contained in the main file will appear in all child documents
unless suppressed manually;
it cannot be suppressed automatically by the |\includeonly| directive
and thus should normally be avoided.
A method to include some content in the main file
by means of conditional processing is described in \secref{sec:conditional}.

%%%%%%%%%%%%%%%%%%%%%%%%%%%%%%%%%%%%%%%%
\paragraph{Page Numbering.}

When only a part of the document is compiled,
the appropriate numbering of pages
(as well as other status parameters)
is determined from the |.aux| files.
The latter contain information from previous passes.
However this information needs to propagate through
all intermediate child documents.
Therefore the page numbering in child documents may well
be inconsistent until the complete document is compiled at least once.

A useful (if unconventional) way to always ensure a consistent
page numbering is to restart the numbering in each child document
and denote the pages by `\textit{child}|.|\textit{page}'
where \textit{child} represents the chapter/section number of the child file.
This can be achieved by the command
|\numberwithin{page}{|\textit{child}|}|
of the \textsf{amsmath} package
where \textit{child} can be |chapter| or |section|
depending on the chosen structuring.
Alternatively, one can modify the macro |\thepage| appropriately
and reset the counter |page| at the start of each child file.

%%%%%%%%%%%%%%%%%%%%%%%%%%%%%%%%%%%%%%%%%%%%%%%%%%%%%%%%%%%%%%%%%%%%%%%%%%%%%%%%
\subsection{Conditional Processing}
\label{sec:conditional}

The package provides a mechanism to compile different versions
of a document. To customise the versions further some conditional processing
can come in handy to distinguish which version is being compiled.
The package provides two macros to describe the compilation context:

%%%%%%%%%%%%%%%%%%%%%%%%%%%%%%%%%%%%%%%%
\DescribeMacro{\ifchilddoc}
The conditional |\ifchilddoc| distinguishes between the compilation of
child documents and the main document:
%
\begin{center}
|\ifchilddoc |\textit{child-code}| |[|\||else |\textit{main-code}]| \||fi|
\end{center}

%%%%%%%%%%%%%%%%%%%%%%%%%%%%%%%%%%%%%%%%
\DescribeMacro{\childdocname}
\DescribeMacro{\childdocjob}
The macro |\childdocname| contains the filename (without extension)
of the main or child file being processed.
Note that |\childdocjob| will always contain the name of the main file.

%%%%%%%%%%%%%%%%%%%%%%%%%%%%%%%%%%%%%%%%
\paragraph{Title Page.}

Conditional processing can be used to include a title or banner page
in the main document when proper precautions are taken.
Importantly, the code in the main file should ensure that the page counter
(as well as other status parameters which are stored in the |.aux| files)
takes the same value after the conditional processing.
Otherwise the page numbers may take divergent values
depending on which part is compiled.

For example, a title page could be declared by:
%
\begin{center}
\begin{tabular}{l}
|\ifchilddoc\||else|\\
|\addtocounter{page}{-1}|\\
\textit{code for title page}\\
|\newpage|\\
|\||fi|
\end{tabular}
\end{center}
%
A banner page for the child documents can be generated by:
%
\begin{center}
\begin{tabular}{l}
|\ifchilddoc|\\
|\addtocounter{page}{-1}|\\
\textit{code for banner page}\\
|\newpage|\\
|\||fi|
\end{tabular}
\end{center}
%
Here one could write a message such as:
\begin{center}
|This is the part \childdocname{} of \childdocjob{}.|
\end{center}

%%%%%%%%%%%%%%%%%%%%%%%%%%%%%%%%%%%%%%%%%%%%%%%%%%%%%%%%%%%%%%%%%%%%%%%%%%%%%%%%
\subsection{Flags}
\label{sec:flags}

The package makes it easy to generate different versions
of the main or child documents.
To this end compilation flags can be defined
and assigned different default values.
They will be particularly useful in conjunction
with the forwarding mechanism described in \secref{sec:forward}.

For example, it may be useful to have a flag |\version|
which can be set to |draft| or |final|.
The document source will contain some conditional code
depending on the value of |\version|.
Suppose further, the flag should default to |final| for the main file
and to |draft| for child files
which is a natural assignment for editing the document.
This is achieved by placing the following code
in the preamble of the main document
(below the |\childdocmain| directive):
%
\begin{center}
\begin{tabular}{l}
|\ifchilddoc|\\
|\providecommand{\version}{draft}|\\
|\||else|\\
|\providecommand{\version}{final}|\\
|\||fi|
\end{tabular}
\end{center}
%
The definition by |\providecommand| makes sure
that previous definitions are not overwritten.
Further statements |\providecommand{\version}{...}|
can thus be added before the above code to override it.

For the main file, one might add a line
(between |\childdocmain| and the above block)
%
\begin{center}
|%\ifchilddoc\||else\providecommand{\version}{draft}\||fi|
\end{center}
%
which can be uncommented to produce a draft version.
Likewise one can add a line to the very top of a child file
(above the |\childdocof{|\textit{main}|}| directive)
%
\begin{center}
|%\providecommand{\version}{final}|
\end{center}
%
which can be uncommented to produce the final version of this child document.

%%%%%%%%%%%%%%%%%%%%%%%%%%%%%%%%%%%%%%%%%%%%%%%%%%%%%%%%%%%%%%%%%%%%%%%%%%%%%%%%
\subsection{Forwarding}
\label{sec:forward}

Different versions of the main or child documents
using compilation flags as described in \secref{sec:flags}
can be (permanently) stored in different files
for convenient compilation, viewing and distribution.
To this end, the package defines a command
to pass on compilation to a different file:

%%%%%%%%%%%%%%%%%%%%%%%%%%%%%%%%%%%%%%%%
\DescribeMacro{\childdocforward}
The command |\childdocforward| redirects processing to
another source file:
%
\begin{center}
\begin{tabular}{l}
|\input{childdoc.def}|\\
|\childdocforward[|\textit{main}|]{|\textit{dest}|}|\\
\end{tabular}
\end{center}
%
The argument \textit{dest} is the destination file
(without extension).
It should be the main file or one of the child files.
Note that further \textsf{childdoc} directives
such as |\childdocof| and |\childdocforward|
in the indicated file will be processed in this form.
The optional argument \textit{main}
passes on directly to the main file \textit{main}
while pretending to compile the child \textit{dest}.
This form behaves as if \textit{dest}
issues |\childdocof{|\textit{main}|}| right away,
and no further \textsf{childdoc} directives will be processed.

%%%%%%%%%%%%%%%%%%%%%%%%%%%%%%%%%%%%%%%%
\DescribeMacro{\...prefix}
In the alternative form |\childdocforwardprefix|,
%
\begin{center}
\begin{tabular}{l}
|\input{childdoc.def}|\\
|\childdocforwardprefix[|\textit{main}|]{|\textit{prefix}|}{|\textit{dest}|}|
\end{tabular}
\end{center}
%
the destination file is determined by a pattern
depending on the current file:
To make this work, the current file must be called
`{\textit{prefix}\hspace{0.2em}\textit{suffix}}'
with \textit{prefix} matching precisely the argument.
Processing is then passed on to the file
`{\textit{dest}\hspace{0.2em}\textit{suffix}}'.
Surely, the same effect is achieved by
directly specifying the
argument `{\textit{dest}\hspace{0.2em}\textit{suffix}}'
in the first form.
However, that requires to set up a different file
for each child. With the alternative form of the command
all these files can have exactly the same content
which simplifies setting them up and maintaining them.

For example, the following file |draft.tex|
with a compilation flag |\version| as described in \secref{sec:flags}
compiles the main document as a draft:
%
\begin{center}
\begin{tabular}{l}
|\def\version{draft}|\\
|\input{childdoc.def}|\\
|\childdocforward{|\textit{main}|}|
\end{tabular}
\end{center}
%
Likewise, the following files |final|\textit{nn}|.tex|
compile the final version of the child document
|child|\textit{nn}|.tex|:
%
\begin{center}
\begin{tabular}{l}
|\def\version{final}|\\
|\input{childdoc.def}|\\
|\childdocforwardprefix{final}{child}|
\end{tabular}
\end{center}
%

Note that when several versions of a main file and/or of each child file
are to be generated, it may be convenient to set up a |Makefile| or
shell script to automatise the process.

%%%%%%%%%%%%%%%%%%%%%%%%%%%%%%%%%%%%%%%%%%%%%%%%%%%%%%%%%%%%%%%%%%%%%%%%%%%%%%%%
\subsection{Command Line Processing}
\label{sec:commandline}

The effect of redirection files can also be achieved by invoking
the \LaTeX{} compiler with a more elaborate command line.
Most conveniently this should be done as part
of a shell script or a |Makefile|.

When using \textsf{childdoc} in the main file, the following
command lines effectively perform a redirection
(note that depending on the shell being used,
backslashes may have to be doubled: `|\|' $\to$ `|\\|'):
%
\begin{center}
|... -jobname "|\textit{target}|" |\\|"|[\textit{flags}]%
|\input{childdoc.def}\childdocforward[|\textit{main}|]{|\textit{dest}|}"|
\end{center}
%
Here \textit{target} is the name of the output file,
\textit{main} is the name of the main file
and \textit{dest} is the name of the main or child file to be processed
(all filenames without extensions).
The optional argument \textit{main} can be omitted
if \textit{main} matches \textit{dest}.
Optionally, compilation \textit{flags} can be defined via |\def| commands.
This command line makes the \TeX{} engine believe
it is compiling the file \textit{target}
whose content is specified as the latter parameter.
The provided code then forwards the processing to
\textit{main} or \textit{dest} as described in \secref{sec:forward}.

%%%%%%%%%%%%%%%%%%%%%%%%%%%%%%%%%%%%%%%%%%%%%%%%%%%%%%%%%%%%%%%%%%%%%%%%%%%%%%%%
\subsection{Include by Input}
\label{sec:input}

Including child documents by |\include| has some restrictions by design.
Most notably, the content of a child document always occupies
its own set of pages; pages cannot be shared between child documents.
Usually, this behaviour makes perfect sense
because each child document contain an essential part of the document.
However, in some situations it may be desirable to compose
a document from a collection of parts
without having mandatory page breaks between then.
For this case, the package
provides a mechanism to include parts
by |\input| which can also be processed individually.
However, by construction this mechanism
requires manual handling of the content to be output.

%%%%%%%%%%%%%%%%%%%%%%%%%%%%%%%%%%%%%%%%
\DescribeMacro{\ifchilddocmanual}
The main file should be prepared as usual, see \secref{sec:include}.
However, the document body must make a distinction
between processing of an individual part and of the main document, e.g.:
%
\begin{center}
\begin{tabular}{l}
|\ifchilddocmanual|\\
|\input{\childdocname}|\\
|\||else|\\
\textit{document body with }|\input{|\textit{part}|}|\\
|\||fi|
\end{tabular}
\end{center}
%
The conditional |\ifchilddocmanual| is true whenever
a part to be included by |\input| is being compiled,
and the name of the part is stored in |\childdocname|.

%%%%%%%%%%%%%%%%%%%%%%%%%%%%%%%%%%%%%%%%
\DescribeMacro{\childdocby}
Each part to be included by |\input| should start with:
%
\begin{center}
\begin{tabular}{l}
|\input{childdoc.def}|\\
|\childdocby{|\textit{main}|}|\\
\end{tabular}
\end{center}
%
The directive |\childdocby| is similar to |\childdocof|
described in \secref{sec:include},
but the subsequent selection of content must be done manually.
To that end, both |\ifchilddoc| and |\ifchilddocmanual|
will be true upon processing of a part,
and the name of the part is stored in |\childdocname|.
Note that |\jobname| will be set to the filename of the current part
so that each part receives an individual |.aux| file
that does not interfere with the |.aux| file(s) of the main document.
This behaviour can be altered by the alternative form
|\childdocby[*]{|\textit{main}|}| (with a non-empty optional argument)
which uses the |.aux| file of the main document
by setting |\jobname| to \textit{main}.

%%%%%%%%%%%%%%%%%%%%%%%%%%%%%%%%%%%%%%%%%%%%%%%%%%%%%%%%%%%%%%%%%%%%%%%%%%%%%%%%
\subsection{Driver Development}
\label{sec:driver}

The \textsf{childdoc} mechanism can also be use for the development
of definition files such as \LaTeX{} styles or classes.
This case differs from the above setup with multiple parts
included by |\include| in that no |\includeonly| should be invoked.
This can be achieved by starting the include file
(before |\ProvidesPackage|) with:
%
\begin{center}
\begin{tabular}{l}
|\input{childdoc.def}|\\
|\childdocforward{|\textit{main}|}|\\
\end{tabular}
\end{center}
%
or alternatively with:
%
\begin{center}
\begin{tabular}{l}
|\input{childdoc.def}|\\
|\childdocby{|\textit{main}|}|\\
\end{tabular}
\end{center}
%
Both forms have slightly different effects as described above.
The main file is prepared as usual, see \secref{sec:include}.

%%%%%%%%%%%%%%%%%%%%%%%%%%%%%%%%%%%%%%%%%%%%%%%%%%%%%%%%%%%%%%%%%%%%%%%%%%%%%%%%
\subsection{Legacy Detection}
\label{sec:detection}

The directive |\childdocmain| in the main file can detect
whether the complete document or merely a child is to be compiled
even without using the directive |\childdocof|.
This method is deprecated because it is less robust
and there is no compelling reason to use it;
it is merely provided for backward compatibility
and it may be removed in future versions.

If the detection mechanism is to be used,
it is mandatory to correctly specify
the filename of the main file as the argument of |\childdocmain|:
%
\begin{center}
\begin{tabular}{l}
|\input{childdoc.def}|\\
|\childdocmain{|\textit{main}|}|\\
\end{tabular}
\end{center}
%
If |\jobname| does not match the argument \textit{main} of |\childdocmain|,
it is assumed that |\jobname| points to the child file to be compiled.
When using |\childdocmain| with the main file specified as argument,
it suffices to start a child file
with just |\input{|\textit{main}|}|
without loading of the package and using |\childdocof|.
If instead all processing is done
with the appropriate \textsf{childdoc} directives,
the argument of \textit{main} of |\childdocmain| can be empty.

An alternative version of the command line processing described
in \secref{sec:commandline} using the detection mechanism reads:
%
\begin{center}
|... -jobname "|\textit{target}|" "|[\textit{flags}]%
[|\def\jobname{|\textit{dest}|}|]|\input{|\textit{main}|}"|
\end{center}

%%%%%%%%%%%%%%%%%%%%%%%%%%%%%%%%%%%%%%%%%%%%%%%%%%%%%%%%%%%%%%%%%%%%%%%%%%%%%%%%
\subsection{Manual Code}
\label{sec:manual}

In case one cannot be certain whether the definitions file |childdoc.def|
is installed on the target \TeX{} distribution
and one prefers not to ship it,
it is conceivable to paste a few relevant commands into the sources.

To that end, drop all statements |\input{childdoc.def}|
and perform the replacements as outlined below.
Instead of |\childdocmain{|\textit{main}|}| add the following code
to the top of the main file:
%
\begin{center}
\begin{tabular}{l}
|\||ifdefined\childdocname\endinput\||fi\newif\ifchilddoc|\\
|\edef\childdocname{\scantokens\expandafter{\jobname\noexpand}}|\\
|\def\childdocmain{|\textit{main}|}\||ifx\childdocmain\childdocname\||else|\\
|\childdoctrue\includeonly{\childdocname}\let\jobname\childdocmain\||fi|\\
\end{tabular}
\end{center}
%
Instead of |\childdocof{|\textit{main}|}| just include the main file
at the top of each child file:
%
\begin{center}
|\input{|\textit{main}|}|
\end{center}
%
A simple redirection |\childdocforward{|\textit{dest}|}| is achieved by:
%
\begin{center}
|\def\jobname{|\textit{dest}|}\input{\jobname}|
\end{center}
%
The redirection with prefix
|\childdocforwardprefix[|\textit{prefix}|]{|\textit{dest}|}|
is accomplished by:
%
\begin{center}
\begin{tabular}{l}
|{\edef\jobname{\scantokens\expandafter{\jobname\noexpand}}|\\
|\def\redirectjob |\textit{prefix}|#1~~~{\gdef\jobname{|\textit{dest}|#1}}|\\
|\expandafter\redirectjob\jobname~~~}\input{\jobname}|
\end{tabular}
\end{center}

In an alternative approach,
child documents can be compiled by a specific command line
without additional code or specific definitions:
%
\begin{center}
|... -jobname "|\textit{target}|" "|[\textit{flags}]%
|\includeonly{|\textit{dest}|}\input{|\textit{main}|}"|
\end{center}
%

%%%%%%%%%%%%%%%%%%%%%%%%%%%%%%%%%%%%%%%%%%%%%%%%%%%%%%%%%%%%%%%%%%%%%%%%%%%%%%%%
%%%%%%%%%%%%%%%%%%%%%%%%%%%%%%%%%%%%%%%%%%%%%%%%%%%%%%%%%%%%%%%%%%%%%%%%%%%%%%%%
\section{Information}

%%%%%%%%%%%%%%%%%%%%%%%%%%%%%%%%%%%%%%%%%%%%%%%%%%%%%%%%%%%%%%%%%%%%%%%%%%%%%%%%
\subsection{Copyright}

Copyright \copyright{} 2017--2018 Niklas Beisert

This work may be distributed and/or modified under the
conditions of the \LaTeX{} Project Public License, either version 1.3
of this license or (at your option) any later version.
The latest version of this license is in
  \url{http://www.latex-project.org/lppl.txt}
and version 1.3 or later is part of all distributions of \LaTeX{}
version 2005/12/01 or later.

This work has the LPPL maintenance status `maintained'.

The Current Maintainer of this work is Niklas Beisert.

This work consists of the files |README.txt|, |childdoc.ins| and |childdoc.dtx|
as well as the derived files |childdoc.def|, |cdocsamp.tex|
with |cdocsch1.tex|, |cdocsch2.tex|, |cdocspt3.tex|, |cdocspt4.tex|,
|cdocsdrf.tex|, |cdocsfn1.tex|, |cdocsfn2.tex|
as well as |childdoc.pdf|.

%%%%%%%%%%%%%%%%%%%%%%%%%%%%%%%%%%%%%%%%%%%%%%%%%%%%%%%%%%%%%%%%%%%%%%%%%%%%%%%%
\subsection{Files and Installation}

The package consists of the files:
%
\begin{center}
\begin{tabular}{ll}
    |README.txt|   & readme file \\
    |childdoc.ins| & installation file \\
    |childdoc.dtx| & source file \\
    |childdoc.def| & definition file \\
    |cdocsamp.tex| & sample main file \\
    |cdocsch1.tex| & sample include file \\
    |cdocsch2.tex| & sample include file \\
    |cdocspt3.tex| & sample part file \\
    |cdocspt4.tex| & sample part file \\
    |cdocsdrf.tex| & sample redirection file \\
    |cdocsfn1.tex| & sample redirection file \\
    |cdocsfn2.tex| & sample redirection file \\
    |childdoc.pdf| & manual
\end{tabular}
\end{center}
%
The distribution consists of the files
|README.txt|, |childdoc.ins| and |childdoc.dtx|.
%
\begin{itemize}
\item
Run (pdf)\LaTeX{} on |childdoc.dtx|
to compile the manual |childdoc.pdf| (this file).
\item
Run \LaTeX{} on |childdoc.ins| to create the definitions file |childdoc.def|
and the sample |cdocsamp.tex| with include files
|cdocsch1.tex|, |cdocsch2.tex|, |cdocspt3.tex|, |cdocspt4.tex|,
|cdocsdrf.tex|, |cdocsfn1.tex|, |cdocsfn2.tex|.
Then copy the file |childdoc.def| to an appropriate directory of your \LaTeX{}
distribution, e.g.\ \textit{texmf-root}|/tex/latex/childdoc|.
\end{itemize}

%%%%%%%%%%%%%%%%%%%%%%%%%%%%%%%%%%%%%%%%%%%%%%%%%%%%%%%%%%%%%%%%%%%%%%%%%%%%%%%%
\subsection{Related CTAN Packages}

There are several other packages which offer a similar functionality:
%
\begin{itemize}
\item
The packages
\href{http://ctan.org/pkg/docmute}{\textsf{docmute}},
\href{http://ctan.org/pkg/includex}{\textsf{includex}} and
\href{http://ctan.org/pkg/standalone}{\textsf{standalone}}
provide commands to include only the document body of
a child file thus allowing both files to be compiled individually.
\item
The packages \href{http://ctan.org/pkg/subdocs}{\textsf{subdocs}}
and \href{http://ctan.org/pkg/subfiles}{\textsf{subfiles}}
provide structures in which the main and child documents can be
encapsulated and allowing them to be compiled individually.
The inclusion mechanism is different from the conventional |\include|.
\item
The package \href{http://ctan.org/pkg/combine}{\textsf{combine}}
is an elaborate solution to combine several documents into one.
\end{itemize}
%
See also the CTAN topic \href{http://ctan.org/topic/subdocs}{\textsf{subdocs}}
for further related packages.
The present package differs from the above solutions in that
a document structure constructed with the conventional |\include| mechanism
just needs two extra commands at the top of every file
such that all constituent files can be compiled individually.

%%%%%%%%%%%%%%%%%%%%%%%%%%%%%%%%%%%%%%%%%%%%%%%%%%%%%%%%%%%%%%%%%%%%%%%%%%%%%%%%
%\subsection{Feature Suggestions}
%
%The following is a list of features which may be useful for future
%versions of this package:
%%
%\begin{itemize}
%\item
%\ldots
%\end{itemize}

%%%%%%%%%%%%%%%%%%%%%%%%%%%%%%%%%%%%%%%%%%%%%%%%%%%%%%%%%%%%%%%%%%%%%%%%%%%%%%%%
\subsection{Revision History}

%%%%%%%%%%%%%%%%%%%%%%%%%%%%%%%%%%%%%%%%
\paragraph{v2.0:} 2018/12/30

\begin{itemize}
\item
immediate forward processing
\item
added |\childdocby| mechanism
\item
manual restructured
\end{itemize}

%%%%%%%%%%%%%%%%%%%%%%%%%%%%%%%%%%%%%%%%
\paragraph{v1.6:} 2018/01/17

\begin{itemize}
\item
application for development of include files
\item
corrections to manual
\end{itemize}

%%%%%%%%%%%%%%%%%%%%%%%%%%%%%%%%%%%%%%%%
\paragraph{v1.5:} 2017/05/21

\begin{itemize}
\item
more complete structuring introduced
\item
|\childdocof| introduced
\item
|\childdoc| renamed to |\childdocmain|
\item
|\childredirect| renamed to |\childdocforward| and |\childdocforwardprefix|
and functionality expanded
\end{itemize}

%%%%%%%%%%%%%%%%%%%%%%%%%%%%%%%%%%%%%%%%
\paragraph{v1.0:} 2017/04/27

\begin{itemize}
\item
manual and install package
\item
first version published on CTAN
\end{itemize}

%%%%%%%%%%%%%%%%%%%%%%%%%%%%%%%%%%%%%%%%
\paragraph{v0.6:} 2017/04/26

\begin{itemize}
\item
redirection mechanism added
\end{itemize}

%%%%%%%%%%%%%%%%%%%%%%%%%%%%%%%%%%%%%%%%
\paragraph{v0.5:} 2017/04/26

\begin{itemize}
\item
functionality in definition file
\end{itemize}


%%%%%%%%%%%%%%%%%%%%%%%%%%%%%%%%%%%%%%%%%%%%%%%%%%%%%%%%%%%%%%%%%%%%%%%%%%%%%%%%
%%%%%%%%%%%%%%%%%%%%%%%%%%%%%%%%%%%%%%%%%%%%%%%%%%%%%%%%%%%%%%%%%%%%%%%%%%%%%%%%
%%%%%%%%%%%%%%%%%%%%%%%%%%%%%%%%%%%%%%%%%%%%%%%%%%%%%%%%%%%%%%%%%%%%%%%%%%%%%%%%
\appendix

\settowidth\MacroIndent{\rmfamily\scriptsize 000\ }

 \DocInput{childdoc.dtx}

\end{document}
%</driver>
% \fi
%
% %%%%%%%%%%%%%%%%%%%%%%%%%%%%%%%%%%%%%%%%%%%%%%%%%%%%%%%%%%%%%%%%%%%%%%%%%%%%%%
% %%%%%%%%%%%%%%%%%%%%%%%%%%%%%%%%%%%%%%%%%%%%%%%%%%%%%%%%%%%%%%%%%%%%%%%%%%%%%%
% \section{Sample}
%\iffalse
%<*samplemain>
%\fi
%
% The following presents a sample document
% with two chapters, two parts, a title page,
% a compile flag as well as three forwarding files to set the flag.
% It consists of eight |.tex| files:
% \begin{center}
% \begin{tabular}{ll}
% |cdocsamp.tex|&main file\\
% |cdocsch1.tex|&include file for chapter 1\\
% |cdocsch2.tex|&include file for chapter 2\\
% |cdocspt3.tex|&include file for part 3\\
% |cdocspt4.tex|&include file for part 4\\
% |cdocsdrf.tex|&forwarding file for main file in draft mode\\
% |cdocsfi1.tex|&forwarding file for final version of chapter 1\\
% |cdocsfi2.tex|&forwarding file for final version of chapter 2\\
% \end{tabular}
% \end{center}
% Each of the eight files can be compiled directly by the \LaTeX{} compiler.
%
% %%%%%%%%%%%%%%%%%%%%%%%%%%%%%%%%%%%%%%
% \paragraph{Main File.}
%
% The main file is called |cdocsamp.tex|.
%
% Load the \textsf{childdoc} definitions and
% declare the filename for the main document:
%    \begin{macrocode}
\input{childdoc.def}
\childdocmain{}
%    \end{macrocode}

% Optional override for |\version| flag:
%    \begin{macrocode}
%%\ifchilddoc\else\providecommand{\version}{draft}\fi
%    \end{macrocode}

% Define the default values for the |\version| flag
% (|final| for the main file and |draft| for childs):
%    \begin{macrocode}
\ifchilddoc
\providecommand{\version}{draft}
\else
\providecommand{\version}{final}
\fi
%    \end{macrocode}

% Load the standard document class:
%    \begin{macrocode}
\documentclass[12pt]{article}
%    \end{macrocode}

% Start the document body:
%    \begin{macrocode}
\begin{document}
%    \end{macrocode}

% Declare a title page.
% Print title, part of document being processed and version flag:
%    \begin{macrocode}
\addtocounter{page}{-1}
\begin{center}
{\LARGE\bfseries{}childdoc example\par}
\vspace{1cm}
\ifchilddoc
\ifchilddocmanual part\else chapter\fi:
`\childdocname' of `\childdocjob'\par
\else
main document: `\childdocjob'\par
\fi
version: \version\par
\end{center}
\newpage
%    \end{macrocode}

% Manually include selected file,
% otherwise process as usual:
%    \begin{macrocode}
\ifchilddocmanual
\section*{part `\childdocname'}
\input{\childdocname}
\else
%    \end{macrocode}

% Include the two chapters:
%    \begin{macrocode}
\include{cdocsch1}
\include{cdocsch2}
%    \end{macrocode}

% Include the two parts unless only chapters should be displayed:
%    \begin{macrocode}
\ifchilddoc\else
\section{part three}
\input{cdocspt3}
\section{part four}
\input{cdocspt4}
\fi
%    \end{macrocode}

% Process as usual until here:
%    \begin{macrocode}
\fi
%    \end{macrocode}

% End of document body:
%    \begin{macrocode}
\end{document}
%    \end{macrocode}
%\iffalse
%</samplemain>
%\fi
%
% %%%%%%%%%%%%%%%%%%%%%%%%%%%%%%%%%%%%%%
% \paragraph{Chapter Include Files.}
%
% The include files are called |cdocsch1.tex| and |cdocsch2.tex|.
%
%\iffalse
%<*samplechap1|samplechap2>
%\fi

% Optional override for |\version| flag:
%    \begin{macrocode}
%%\providecommand{\version}{final}
%    \end{macrocode}

% Include the main document:
%    \begin{macrocode}
\input{childdoc.def}
\childdocof{cdocsamp}
%    \end{macrocode}

%\iffalse
%</samplechap1|samplechap2>
%\fi
%
%\iffalse
%<*samplechap1>
%\fi
% Some text for chapter 1:
%    \begin{macrocode}
\section{one}
some text in chapter one
%    \end{macrocode}

%\iffalse
%</samplechap1>
%\fi
% Some text for chapter 2:
%\iffalse
%<*samplechap2>
%\fi
%    \begin{macrocode}
\section{two}
more text in chapter two
%    \end{macrocode}

%\iffalse
%</samplechap2>
%\fi
%
% %%%%%%%%%%%%%%%%%%%%%%%%%%%%%%%%%%%%%%
% \paragraph{Part Include Files.}
%
% The include files are called |cdocspt3.tex| and |cdocspt4.tex|.
%
%\iffalse
%<*samplepart3|samplepart4>
%\fi

% Optional override for |\version| flag:
%    \begin{macrocode}
%%\providecommand{\version}{final}
%    \end{macrocode}

% Include the main document:
%    \begin{macrocode}
\input{childdoc.def}
\childdocby{cdocsamp}
%    \end{macrocode}

%\iffalse
%</samplepart3|samplepart4>
%\fi
%
%\iffalse
%<*samplepart3>
%\fi
% Some text for part 3:
%    \begin{macrocode}
some text in part three
%    \end{macrocode}

%\iffalse
%</samplepart3>
%\fi
% Some text for part 4:
%\iffalse
%<*samplepart4>
%\fi
%    \begin{macrocode}
more text in part four
%    \end{macrocode}

%\iffalse
%</samplepart4>
%\fi
%
% %%%%%%%%%%%%%%%%%%%%%%%%%%%%%%%%%%%%%%
% \paragraph{Forwarding for a Complete Draft.}
%
% The following forwarding file |cdocsdrf.tex|
% compiles the main document in draft mode:
%\iffalse
%<*sampledraft>
%\fi
%    \begin{macrocode}
\def\version{draft}
\input{childdoc.def}
\childdocforward{cdocsamp}
%    \end{macrocode}

%\iffalse
%</sampledraft>
%\fi
%
% %%%%%%%%%%%%%%%%%%%%%%%%%%%%%%%%%%%%%%
% \paragraph{Forwarding for Final Version of the Chapters.}
%
% The following forwarding files |cdocsfn1.tex| and |cdocsfn2.tex|
% (with identical content)
% compile the final versions of the child documents
% |cdocsch1.tex| and |cdocsch2.tex|, respectively:
%\iffalse
%<*samplefinal>
%\fi
%    \begin{macrocode}
\def\version{final}
\input{childdoc.def}
\childdocforwardprefix[cdocsamp]{cdocsfn}{cdocsch}
%    \end{macrocode}

%\iffalse
%</samplefinal>
%\fi
%
% %%%%%%%%%%%%%%%%%%%%%%%%%%%%%%%%%%%%%%
% \paragraph{Command Line Processing.}
%
% The following three command lines generate the output files
% |cdocscld|, |cdocscl1| and |cdocscl2|
% which should be identical to
% |cdocsdrf|, |cdocsch1| and |cdocsfn2|, respectively:
% \begin{center}
% \begin{tabular}{l}
% |latex -jobname cdocscld \|\\
% |  "\def\version{draft}\input{childdoc.def}\childdocforward{cdocsamp}"|\\
% |latex -jobname cdocscl1 \|\\
% |  "\input{childdoc.def}\childdocforward[cdocsamp]{cdocsch1}"|\\
% |latex -jobname cdocscl2 \|\\
% |  "\def\version{final}\input{childdoc.def}\childdocforward{cdocsch2}"|
% \end{tabular}
% \end{center}
% Note that the trailing backslash on each first line
% merely continues the input to the second line
% (for convenient cut ant paste).
% Furthermore, the command |latex| can be replaced by any
% of its alternative versions such as |pdflatex|.
%
% %%%%%%%%%%%%%%%%%%%%%%%%%%%%%%%%%%%%%%%%%%%%%%%%%%%%%%%%%%%%%%%%%%%%%%%%%%%%%%
% %%%%%%%%%%%%%%%%%%%%%%%%%%%%%%%%%%%%%%%%%%%%%%%%%%%%%%%%%%%%%%%%%%%%%%%%%%%%%%
% \section{Implementation}
%\iffalse
%<*package>
%\fi
%
% This section describes the definitions file |childdoc.def|.

% The definitions cannot be loaded using |\usepackage| or |\RequirePackage|
% which has a mechanism to prevent loading a style file more than once.
% When loading the definitions by means of |\input|
% multiple instances have to be prevented manually:
%\iffalse
%This code needs to be before the `\ProvidesFile' directive
%which is defined at the beginning of this file.
%Therefore it is also placed there and commented out here.
%</package>
%<*discard>
%\fi
%    \begin{macrocode}
\ifdefined\childdocmain\endinput\fi
%    \end{macrocode}
%\iffalse
%</discard>
%<*package>
%\fi
%
% \macro{\ifchilddoc}
% \macro{\ifchilddocmanual}
% The conditional |\ifchilddoc| tells whether a
% child (true) or main (false) document is being compiled.
% The conditional |\ifchilddocmanual| tells whether
% the |\includeonly| mechanism is used (false) or
% the selection of child files must be performed manually (true).
% The definitions initialise to false:
%    \begin{macrocode}
\newif\ifchilddoc
\newif\ifchilddocmanual
%    \end{macrocode}

% \macro{\childdocname}
% \macro{\childdocjob}
% The macro |\childdocname| stores the name of the main document
% to be compiled. The macro |\childdocjob| stores the name of
% the document on which the \LaTeX{} compiler was originally invoked.
% The content of |\jobname| cannot be compared
% to filenames specified in the source due to different catcodes.
% The following code rescans |\jobname|, stores the result
% in |\childdocname| and saves a copy in |\childdocjob|:
%    \begin{macrocode}
\edef\childdocname{\scantokens\expandafter{\jobname\noexpand}}
\let\childdocjob\childdocname
%    \end{macrocode}

% \macro{\childdocdisable}
% The macro |\childdocdisable| prevents the main file
% from being processed more than once.
% At this stage, the main document command |\childdocmain|
% is assumed to be called once again where it should do nothing.
% Any subsequent call to it should prevent
% a secondary processing of the main document
% It overwrites the forwarding commands
% |\childdocof| and |\childdocforward|
% with empty macros to prevent further inclusions of the main document:
%    \begin{macrocode}
\newcommand{\childdocdisable}
{
  \renewcommand{\childdocmain}[1]{\renewcommand{\childdocmain}[1]{\endinput}}
  \renewcommand{\childdocof}[1]{}
  \renewcommand{\childdocby}[2][]{}
  \renewcommand{\childdocforward}[2][]{}
  \renewcommand{\childdocdisable}{}
}
%    \end{macrocode}

% \macro{\childdocmain}
% The macro |\childdocmain| is to be called at the top of the main file
% with nothing or the main filename (without extension) as argument.
% First, it breaks loops.
% If the argument is not empty and does not match |\childdocname|
% (which is set by the first inclusion of |childdoc.def|),
% |\ifchilddoc| is set to true, |\includeonly| is applied to the child file
% and |\jobname| is set to the main file
% (for proper handling of |.aux| files):
%    \begin{macrocode}
\newcommand{\childdocmain}[1]
{
  \childdocdisable\childdocmain{}
  \if?#1?\else
    \begingroup
      \def\childdoctmp{#1}
      \ifx\childdoctmp\childdocname
        \def\childdoctmp{}
      \else
        \def\childdoctmp
        {
          \childdoctrue
          \includeonly{\childdocname}
          \def\childdocjob{#1}
          \def\jobname{#1}
        }
      \fi
      \expandafter
    \endgroup
    \childdoctmp
  \fi
}
%    \end{macrocode}

% \macro{\childdocof}
% The command |\childdocof| redirects
% compilation to the main file |#1|.
%    \begin{macrocode}
\newcommand{\childdocof}[1]
{
  \childdocdisable
  \childdoctrue
  \includeonly{\childdocname}
  \def\jobname{#1}
  \def\childdocjob{#1}
  \input{#1}
}
%    \end{macrocode}

% \macro{\childdocby}
% The command |\childdocby| ....
%    \begin{macrocode}
\newcommand{\childdocby}[2][]
{
  \childdocdisable
  \childdoctrue
  \childdocmanualtrue
  \if?#1?\else
    \def\jobname{#2}
  \fi
  \def\childdocjob{#2}
  \input{#2}
  \endinput
}
%    \end{macrocode}

% \macro{\childdocforward}
% The command |\childdocforward| redirects
% compilation to the main file or
% (if the optional argument is given) a child file.
% Parameters are set as if the main file
% or a child file starting with |\childdocof| was compiled.
% Then compilation is handed over to the main file:
%    \begin{macrocode}
\newcommand{\childdocforward}[2][]
{
  \begingroup
    \if?#1?
      \def\childdoctmp
      {
        \def\childdocname{#2}
        \def\childdocjob{#2}
        \def\jobname{#2}
        \input{#2}
        \endinput
      }
    \else
      \def\childdoctmp
      {
        \childdocdisable
        \def\childdocname{#2}
        \childdoctrue
        \includeonly{#2}
        \def\childdocjob{#1}
        \def\jobname{#1}
        \input{#1}
        \endinput
      }
    \fi
    \expandafter
  \endgroup
  \childdoctmp
}
%    \end{macrocode}

% \macro{\childdocforwardprefix}
% The command |\childdocforwardprefix| redirects
% compilation to the main or a child file by means of a pattern.
% The prefix |#1| in the current filename is replaced by |#2|
% and the suffix of the current filename is kept
% (it is assumed that the filename does not contain the substring `|~~~|'
% which is used as a delimiter).
% Compilation is handed over to the new file by |\childdocforward|:
%    \begin{macrocode}
\newcommand{\childdocforwardprefix}[3][]
{
  \begingroup
    \def\childdocextract #2##1~~~{\def\childdoctmp{\childdocforward[#1]{#3##1}}}
    \expandafter\childdocextract\childdocname~~~
    \expandafter
  \endgroup
  \childdoctmp
}
%    \end{macrocode}

% \macro{\childdoc}
% The deprecated macro |\childdoc| is a legacy version of |\childdocmain|:
%    \begin{macrocode}
\newcommand{\childdoc}{\childdocmain}
%    \end{macrocode}

% \macro{\childdocredirect}
% The deprecated macro |\childdocredirect| is a legacy version
% of |\childdocforward| and |\childdocforwardprefix|:
%    \begin{macrocode}
\newcommand{\childdocredirect}[2][]
{
  \begingroup
    \if?#1?
      \def\childdoctmp{\childdocforward{#2}}
    \else
      \def\childdoctmp{\childdocforwardprefix{#1}{#2}}
    \fi
    \expandafter
  \endgroup
  \childdoctmp
}
%    \end{macrocode}

%\iffalse
%</package>
%\fi
%
\endinput
|\\
|\childdocby{|\textit{main}|}|\\
\end{tabular}
\end{center}
%
The directive |\childdocby| is similar to |\childdocof|
described in \secref{sec:include},
but the subsequent selection of content must be done manually.
To that end, both |\ifchilddoc| and |\ifchilddocmanual|
will be true upon processing of a part,
and the name of the part is stored in |\childdocname|.
Note that |\jobname| will be set to the filename of the current part
so that each part receives an individual |.aux| file
that does not interfere with the |.aux| file(s) of the main document.
This behaviour can be altered by the alternative form
|\childdocby[*]{|\textit{main}|}| (with a non-empty optional argument)
which uses the |.aux| file of the main document
by setting |\jobname| to \textit{main}.

%%%%%%%%%%%%%%%%%%%%%%%%%%%%%%%%%%%%%%%%%%%%%%%%%%%%%%%%%%%%%%%%%%%%%%%%%%%%%%%%
\subsection{Driver Development}
\label{sec:driver}

The \textsf{childdoc} mechanism can also be use for the development
of definition files such as \LaTeX{} styles or classes.
This case differs from the above setup with multiple parts
included by |\include| in that no |\includeonly| should be invoked.
This can be achieved by starting the include file
(before |\ProvidesPackage|) with:
%
\begin{center}
\begin{tabular}{l}
|% \iffalse
%
% childdoc.dtx Copyright (C) 2017-2018 Niklas Beisert
%
% This work may be distributed and/or modified under the
% conditions of the LaTeX Project Public License, either version 1.3
% of this license or (at your option) any later version.
% The latest version of this license is in
%   http://www.latex-project.org/lppl.txt
% and version 1.3 or later is part of all distributions of LaTeX
% version 2005/12/01 or later.
%
% This work has the LPPL maintenance status `maintained'.
%
% The Current Maintainer of this work is Niklas Beisert.
%
% This work consists of the files childdoc.dtx and childdoc.ins
% and the derived files childdoc.def and cdocsamp.tex with
% cdocsch1.tex, cdocsch2.tex, cdocsdrf.tex, cdocsfn1.tex, cdocsfn2.tex.
%
%<package>\ifdefined\childdocmain\endinput\fi
%<package>\ProvidesFile{childdoc.def}[2018/12/30 v2.0 child document driver]
%<samplemain>\ProvidesFile{cdocsamp.tex}[2018/12/30 v2.0 sample for childdoc]
%<*driver>
%\ProvidesFile{childdoc.drv}[2018/12/30 v2.0 childdoc reference manual file]
\PassOptionsToClass{10pt,a4paper}{article}
\documentclass{ltxdoc}

\usepackage[margin=35mm]{geometry}
\usepackage{hyperref}
\usepackage{hyperxmp}
\usepackage[usenames]{color}

\hypersetup{colorlinks=true}
\hypersetup{pdfstartview=FitH}
\hypersetup{pdfpagemode=UseNone}
\hypersetup{pdfsource={}}
\hypersetup{pdflang={en-UK}}
\hypersetup{pdfcopyright={Copyright 2017-2018 Niklas Beisert.
  This work may be distributed and/or modified under the
  conditions of the LaTeX Project Public License, either version 1.3
  of this license or (at your option) any later version.}}
\hypersetup{pdflicenseurl={http://www.latex-project.org/lppl.txt}}
\hypersetup{pdfcontactaddress={ETH Zurich, ITP, HIT K,
  Wolfgang-Pauli-Strasse 27}}
\hypersetup{pdfcontactpostcode={8093}}
\hypersetup{pdfcontactcity={Zurich}}
\hypersetup{pdfcontactcountry={Switzerland}}
\hypersetup{pdfcontactemail={nbeisert@itp.phys.ethz.ch}}
\hypersetup{pdfcontacturl={http://people.phys.ethz.ch/\xmptilde nbeisert/}}

\newcommand{\secref}[1]{\hyperref[#1]{section \ref*{#1}}}

\parskip1ex
\parindent0pt
\let\olditemize\itemize
\def\itemize{\olditemize\parskip0pt}

\begin{document}

\title{The \textsf{childdoc} Package}
\hypersetup{pdftitle={The childdoc Package}}
\author{Niklas Beisert\\[2ex]
  Institut f\"ur Theoretische Physik\\
  Eidgen\"ossische Technische Hochschule Z\"urich\\
  Wolfgang-Pauli-Strasse 27, 8093 Z\"urich, Switzerland\\[1ex]
  \href{mailto:nbeisert@itp.phys.ethz.ch}
  {\texttt{nbeisert@itp.phys.ethz.ch}}}
\hypersetup{pdfauthor={Niklas Beisert}}
\hypersetup{pdfsubject={Manual for the LaTeX2e Package childdoc}}
\date{30 December 2018, \textsf{v2.0}}
\maketitle

\begin{abstract}\noindent
\textsf{childdoc} is a \LaTeXe{} package
that enables the direct compilation
of document sections included by |\include|
to individual files.
\end{abstract}

\begingroup
\parskip0ex
\tableofcontents
\endgroup

%%%%%%%%%%%%%%%%%%%%%%%%%%%%%%%%%%%%%%%%%%%%%%%%%%%%%%%%%%%%%%%%%%%%%%%%%%%%%%%%
%%%%%%%%%%%%%%%%%%%%%%%%%%%%%%%%%%%%%%%%%%%%%%%%%%%%%%%%%%%%%%%%%%%%%%%%%%%%%%%%
\section{Introduction}

\LaTeX{} provides a mechanism to structure a large document (such as a book)
into a main file and several child files (containing the chapters)
using the |\include| command.
This mechanism is beneficial for documents
which span hundreds of pages in order to
make the source file(s) more manageable.
Moreover, compilation can be restricted to
selected child files by means of the |\includeonly| command.
The latter feature can be used to reduce the compilation time while editing
(this was significantly more useful in the earlier days of \LaTeX{})
or to generate a smaller document which is easier to navigate.
Another application of |\includeonly| is to generate
documents consisting of selected parts of the complete document.

However, there are a few drawbacks of the plain |\include| mechanism:
\begin{itemize}
\item
The child files cannot be compiled on their own,
they can only be compiled via the main file.
A naive editing environment
(such as a text editor with an option
to have the current file processed by \LaTeX)
may require one to switch to the main file before compiling;
attempting to compile the child file produces errors.
\item
The main file must be modified (each time)
to adjust the |\includeonly| command
to the present needs. This easily leaves the main file in a messy state.
\item
The generated document will always carry the filename
of the main document. This is inconvenient if
several child files are to be compiled and
to be kept for distribution.
\end{itemize}

The present package provides a simple interface
to make child files individually compilable by \LaTeX{}.
Compiling a child file then has the same effect as compiling
the main file with an |\includeonly| command
to select the appropriate child.
Moreover the generated document will carry the name of the child
rather than the main file.
This resolves all three above issues.

This feature is meant to make the editing of books,
thesis documents and lecture notes somewhat more convenient.
However, the package can also be used efficiently for
composing a series of documents (such as exercise sheets)
which are typically distributed individually.
It then assists the author in generating the individual documents
(potentially in different versions)
as well as a document containing the collected series.
Another application is in developing style files
or other kinds of included material
where compilation of the style file could redirect
to a sample or test file.

%%%%%%%%%%%%%%%%%%%%%%%%%%%%%%%%%%%%%%%%%%%%%%%%%%%%%%%%%%%%%%%%%%%%%%%%%%%%%%%%
%%%%%%%%%%%%%%%%%%%%%%%%%%%%%%%%%%%%%%%%%%%%%%%%%%%%%%%%%%%%%%%%%%%%%%%%%%%%%%%%
\section{Usage}

First of all, the package \textsf{childdoc} is \emph{not} a standard
\LaTeXe{} |.sty| style file! Therefore it needs to be invoked in
a non-standard way.

%%%%%%%%%%%%%%%%%%%%%%%%%%%%%%%%%%%%%%%%%%%%%%%%%%%%%%%%%%%%%%%%%%%%%%%%%%%%%%%%
\subsection{Included Files}
\label{sec:include}

%%%%%%%%%%%%%%%%%%%%%%%%%%%%%%%%%%%%%%%%
\DescribeMacro{\childdocmain}
To use the package, add the commands
\begin{center}
\begin{tabular}{l}
|\input{childdoc.def}|\\
|\childdocmain{}|\\
\end{tabular}
\end{center}
at the very top of the main \LaTeX{} file,
in particular \emph{before} the |\documentclass| statement!
The argument of |\childdocmain| should be left empty
(but it must be present).

%%%%%%%%%%%%%%%%%%%%%%%%%%%%%%%%%%%%%%%%
\DescribeMacro{\childdocof}
Furthermore, add the commands
\begin{center}
\begin{tabular}{l}
|\input{childdoc.def}|\\
|\childdocof{|\textit{main}|}|\\
\end{tabular}
\end{center}
at the top of every child file \textit{child}
which is included by |\include{|\textit{child}|}|
from within the main file
(or at least for those files to be compiled individually).
The argument \textit{main} must be the filename of the main file.

There are a couple of
considerations in setting up the main and child documents:

%%%%%%%%%%%%%%%%%%%%%%%%%%%%%%%%%%%%%%%%
\paragraph{Restrictions.}

Please note the following restrictions:
\begin{itemize}
\item
|\childdocmain| must be called with one argument \textit{main}
to ensure compatibility with earlier version of the package.
It must either be empty (|\childdocmain{}|)
or precisely match the filename of the main file in which it is specified.
See \secref{sec:detection} for further information.
\item
The filename \textit{main} must be specified without the |.tex| extension.
\item
The filename \textit{main} is case sensitive
(even in case-insensitive file systems)
due to internal string comparison.
\item
The argument \textit{main} should be fully expanded, it cannot be a macro.
\item
Subdirectories and special characters should be avoided in filenames.
\item
The command |\childdocmain{|\textit{main}|}| must be followed by a whitespace.
It should not be followed immediately by another command
or by a comment mark `|%|'.
This is because the \TeX{} parser reads the token immediately following
the argument of |\childdocmain| and puts it
at the beginning of every child section;
however, a white\-space is ignored.
\end{itemize}

%%%%%%%%%%%%%%%%%%%%%%%%%%%%%%%%%%%%%%%%
\paragraph{Content of Main File.}

It is advisable to place all content in the child files included by |\include|.
Any output contained in the main file will appear in all child documents
unless suppressed manually;
it cannot be suppressed automatically by the |\includeonly| directive
and thus should normally be avoided.
A method to include some content in the main file
by means of conditional processing is described in \secref{sec:conditional}.

%%%%%%%%%%%%%%%%%%%%%%%%%%%%%%%%%%%%%%%%
\paragraph{Page Numbering.}

When only a part of the document is compiled,
the appropriate numbering of pages
(as well as other status parameters)
is determined from the |.aux| files.
The latter contain information from previous passes.
However this information needs to propagate through
all intermediate child documents.
Therefore the page numbering in child documents may well
be inconsistent until the complete document is compiled at least once.

A useful (if unconventional) way to always ensure a consistent
page numbering is to restart the numbering in each child document
and denote the pages by `\textit{child}|.|\textit{page}'
where \textit{child} represents the chapter/section number of the child file.
This can be achieved by the command
|\numberwithin{page}{|\textit{child}|}|
of the \textsf{amsmath} package
where \textit{child} can be |chapter| or |section|
depending on the chosen structuring.
Alternatively, one can modify the macro |\thepage| appropriately
and reset the counter |page| at the start of each child file.

%%%%%%%%%%%%%%%%%%%%%%%%%%%%%%%%%%%%%%%%%%%%%%%%%%%%%%%%%%%%%%%%%%%%%%%%%%%%%%%%
\subsection{Conditional Processing}
\label{sec:conditional}

The package provides a mechanism to compile different versions
of a document. To customise the versions further some conditional processing
can come in handy to distinguish which version is being compiled.
The package provides two macros to describe the compilation context:

%%%%%%%%%%%%%%%%%%%%%%%%%%%%%%%%%%%%%%%%
\DescribeMacro{\ifchilddoc}
The conditional |\ifchilddoc| distinguishes between the compilation of
child documents and the main document:
%
\begin{center}
|\ifchilddoc |\textit{child-code}| |[|\||else |\textit{main-code}]| \||fi|
\end{center}

%%%%%%%%%%%%%%%%%%%%%%%%%%%%%%%%%%%%%%%%
\DescribeMacro{\childdocname}
\DescribeMacro{\childdocjob}
The macro |\childdocname| contains the filename (without extension)
of the main or child file being processed.
Note that |\childdocjob| will always contain the name of the main file.

%%%%%%%%%%%%%%%%%%%%%%%%%%%%%%%%%%%%%%%%
\paragraph{Title Page.}

Conditional processing can be used to include a title or banner page
in the main document when proper precautions are taken.
Importantly, the code in the main file should ensure that the page counter
(as well as other status parameters which are stored in the |.aux| files)
takes the same value after the conditional processing.
Otherwise the page numbers may take divergent values
depending on which part is compiled.

For example, a title page could be declared by:
%
\begin{center}
\begin{tabular}{l}
|\ifchilddoc\||else|\\
|\addtocounter{page}{-1}|\\
\textit{code for title page}\\
|\newpage|\\
|\||fi|
\end{tabular}
\end{center}
%
A banner page for the child documents can be generated by:
%
\begin{center}
\begin{tabular}{l}
|\ifchilddoc|\\
|\addtocounter{page}{-1}|\\
\textit{code for banner page}\\
|\newpage|\\
|\||fi|
\end{tabular}
\end{center}
%
Here one could write a message such as:
\begin{center}
|This is the part \childdocname{} of \childdocjob{}.|
\end{center}

%%%%%%%%%%%%%%%%%%%%%%%%%%%%%%%%%%%%%%%%%%%%%%%%%%%%%%%%%%%%%%%%%%%%%%%%%%%%%%%%
\subsection{Flags}
\label{sec:flags}

The package makes it easy to generate different versions
of the main or child documents.
To this end compilation flags can be defined
and assigned different default values.
They will be particularly useful in conjunction
with the forwarding mechanism described in \secref{sec:forward}.

For example, it may be useful to have a flag |\version|
which can be set to |draft| or |final|.
The document source will contain some conditional code
depending on the value of |\version|.
Suppose further, the flag should default to |final| for the main file
and to |draft| for child files
which is a natural assignment for editing the document.
This is achieved by placing the following code
in the preamble of the main document
(below the |\childdocmain| directive):
%
\begin{center}
\begin{tabular}{l}
|\ifchilddoc|\\
|\providecommand{\version}{draft}|\\
|\||else|\\
|\providecommand{\version}{final}|\\
|\||fi|
\end{tabular}
\end{center}
%
The definition by |\providecommand| makes sure
that previous definitions are not overwritten.
Further statements |\providecommand{\version}{...}|
can thus be added before the above code to override it.

For the main file, one might add a line
(between |\childdocmain| and the above block)
%
\begin{center}
|%\ifchilddoc\||else\providecommand{\version}{draft}\||fi|
\end{center}
%
which can be uncommented to produce a draft version.
Likewise one can add a line to the very top of a child file
(above the |\childdocof{|\textit{main}|}| directive)
%
\begin{center}
|%\providecommand{\version}{final}|
\end{center}
%
which can be uncommented to produce the final version of this child document.

%%%%%%%%%%%%%%%%%%%%%%%%%%%%%%%%%%%%%%%%%%%%%%%%%%%%%%%%%%%%%%%%%%%%%%%%%%%%%%%%
\subsection{Forwarding}
\label{sec:forward}

Different versions of the main or child documents
using compilation flags as described in \secref{sec:flags}
can be (permanently) stored in different files
for convenient compilation, viewing and distribution.
To this end, the package defines a command
to pass on compilation to a different file:

%%%%%%%%%%%%%%%%%%%%%%%%%%%%%%%%%%%%%%%%
\DescribeMacro{\childdocforward}
The command |\childdocforward| redirects processing to
another source file:
%
\begin{center}
\begin{tabular}{l}
|\input{childdoc.def}|\\
|\childdocforward[|\textit{main}|]{|\textit{dest}|}|\\
\end{tabular}
\end{center}
%
The argument \textit{dest} is the destination file
(without extension).
It should be the main file or one of the child files.
Note that further \textsf{childdoc} directives
such as |\childdocof| and |\childdocforward|
in the indicated file will be processed in this form.
The optional argument \textit{main}
passes on directly to the main file \textit{main}
while pretending to compile the child \textit{dest}.
This form behaves as if \textit{dest}
issues |\childdocof{|\textit{main}|}| right away,
and no further \textsf{childdoc} directives will be processed.

%%%%%%%%%%%%%%%%%%%%%%%%%%%%%%%%%%%%%%%%
\DescribeMacro{\...prefix}
In the alternative form |\childdocforwardprefix|,
%
\begin{center}
\begin{tabular}{l}
|\input{childdoc.def}|\\
|\childdocforwardprefix[|\textit{main}|]{|\textit{prefix}|}{|\textit{dest}|}|
\end{tabular}
\end{center}
%
the destination file is determined by a pattern
depending on the current file:
To make this work, the current file must be called
`{\textit{prefix}\hspace{0.2em}\textit{suffix}}'
with \textit{prefix} matching precisely the argument.
Processing is then passed on to the file
`{\textit{dest}\hspace{0.2em}\textit{suffix}}'.
Surely, the same effect is achieved by
directly specifying the
argument `{\textit{dest}\hspace{0.2em}\textit{suffix}}'
in the first form.
However, that requires to set up a different file
for each child. With the alternative form of the command
all these files can have exactly the same content
which simplifies setting them up and maintaining them.

For example, the following file |draft.tex|
with a compilation flag |\version| as described in \secref{sec:flags}
compiles the main document as a draft:
%
\begin{center}
\begin{tabular}{l}
|\def\version{draft}|\\
|\input{childdoc.def}|\\
|\childdocforward{|\textit{main}|}|
\end{tabular}
\end{center}
%
Likewise, the following files |final|\textit{nn}|.tex|
compile the final version of the child document
|child|\textit{nn}|.tex|:
%
\begin{center}
\begin{tabular}{l}
|\def\version{final}|\\
|\input{childdoc.def}|\\
|\childdocforwardprefix{final}{child}|
\end{tabular}
\end{center}
%

Note that when several versions of a main file and/or of each child file
are to be generated, it may be convenient to set up a |Makefile| or
shell script to automatise the process.

%%%%%%%%%%%%%%%%%%%%%%%%%%%%%%%%%%%%%%%%%%%%%%%%%%%%%%%%%%%%%%%%%%%%%%%%%%%%%%%%
\subsection{Command Line Processing}
\label{sec:commandline}

The effect of redirection files can also be achieved by invoking
the \LaTeX{} compiler with a more elaborate command line.
Most conveniently this should be done as part
of a shell script or a |Makefile|.

When using \textsf{childdoc} in the main file, the following
command lines effectively perform a redirection
(note that depending on the shell being used,
backslashes may have to be doubled: `|\|' $\to$ `|\\|'):
%
\begin{center}
|... -jobname "|\textit{target}|" |\\|"|[\textit{flags}]%
|\input{childdoc.def}\childdocforward[|\textit{main}|]{|\textit{dest}|}"|
\end{center}
%
Here \textit{target} is the name of the output file,
\textit{main} is the name of the main file
and \textit{dest} is the name of the main or child file to be processed
(all filenames without extensions).
The optional argument \textit{main} can be omitted
if \textit{main} matches \textit{dest}.
Optionally, compilation \textit{flags} can be defined via |\def| commands.
This command line makes the \TeX{} engine believe
it is compiling the file \textit{target}
whose content is specified as the latter parameter.
The provided code then forwards the processing to
\textit{main} or \textit{dest} as described in \secref{sec:forward}.

%%%%%%%%%%%%%%%%%%%%%%%%%%%%%%%%%%%%%%%%%%%%%%%%%%%%%%%%%%%%%%%%%%%%%%%%%%%%%%%%
\subsection{Include by Input}
\label{sec:input}

Including child documents by |\include| has some restrictions by design.
Most notably, the content of a child document always occupies
its own set of pages; pages cannot be shared between child documents.
Usually, this behaviour makes perfect sense
because each child document contain an essential part of the document.
However, in some situations it may be desirable to compose
a document from a collection of parts
without having mandatory page breaks between then.
For this case, the package
provides a mechanism to include parts
by |\input| which can also be processed individually.
However, by construction this mechanism
requires manual handling of the content to be output.

%%%%%%%%%%%%%%%%%%%%%%%%%%%%%%%%%%%%%%%%
\DescribeMacro{\ifchilddocmanual}
The main file should be prepared as usual, see \secref{sec:include}.
However, the document body must make a distinction
between processing of an individual part and of the main document, e.g.:
%
\begin{center}
\begin{tabular}{l}
|\ifchilddocmanual|\\
|\input{\childdocname}|\\
|\||else|\\
\textit{document body with }|\input{|\textit{part}|}|\\
|\||fi|
\end{tabular}
\end{center}
%
The conditional |\ifchilddocmanual| is true whenever
a part to be included by |\input| is being compiled,
and the name of the part is stored in |\childdocname|.

%%%%%%%%%%%%%%%%%%%%%%%%%%%%%%%%%%%%%%%%
\DescribeMacro{\childdocby}
Each part to be included by |\input| should start with:
%
\begin{center}
\begin{tabular}{l}
|\input{childdoc.def}|\\
|\childdocby{|\textit{main}|}|\\
\end{tabular}
\end{center}
%
The directive |\childdocby| is similar to |\childdocof|
described in \secref{sec:include},
but the subsequent selection of content must be done manually.
To that end, both |\ifchilddoc| and |\ifchilddocmanual|
will be true upon processing of a part,
and the name of the part is stored in |\childdocname|.
Note that |\jobname| will be set to the filename of the current part
so that each part receives an individual |.aux| file
that does not interfere with the |.aux| file(s) of the main document.
This behaviour can be altered by the alternative form
|\childdocby[*]{|\textit{main}|}| (with a non-empty optional argument)
which uses the |.aux| file of the main document
by setting |\jobname| to \textit{main}.

%%%%%%%%%%%%%%%%%%%%%%%%%%%%%%%%%%%%%%%%%%%%%%%%%%%%%%%%%%%%%%%%%%%%%%%%%%%%%%%%
\subsection{Driver Development}
\label{sec:driver}

The \textsf{childdoc} mechanism can also be use for the development
of definition files such as \LaTeX{} styles or classes.
This case differs from the above setup with multiple parts
included by |\include| in that no |\includeonly| should be invoked.
This can be achieved by starting the include file
(before |\ProvidesPackage|) with:
%
\begin{center}
\begin{tabular}{l}
|\input{childdoc.def}|\\
|\childdocforward{|\textit{main}|}|\\
\end{tabular}
\end{center}
%
or alternatively with:
%
\begin{center}
\begin{tabular}{l}
|\input{childdoc.def}|\\
|\childdocby{|\textit{main}|}|\\
\end{tabular}
\end{center}
%
Both forms have slightly different effects as described above.
The main file is prepared as usual, see \secref{sec:include}.

%%%%%%%%%%%%%%%%%%%%%%%%%%%%%%%%%%%%%%%%%%%%%%%%%%%%%%%%%%%%%%%%%%%%%%%%%%%%%%%%
\subsection{Legacy Detection}
\label{sec:detection}

The directive |\childdocmain| in the main file can detect
whether the complete document or merely a child is to be compiled
even without using the directive |\childdocof|.
This method is deprecated because it is less robust
and there is no compelling reason to use it;
it is merely provided for backward compatibility
and it may be removed in future versions.

If the detection mechanism is to be used,
it is mandatory to correctly specify
the filename of the main file as the argument of |\childdocmain|:
%
\begin{center}
\begin{tabular}{l}
|\input{childdoc.def}|\\
|\childdocmain{|\textit{main}|}|\\
\end{tabular}
\end{center}
%
If |\jobname| does not match the argument \textit{main} of |\childdocmain|,
it is assumed that |\jobname| points to the child file to be compiled.
When using |\childdocmain| with the main file specified as argument,
it suffices to start a child file
with just |\input{|\textit{main}|}|
without loading of the package and using |\childdocof|.
If instead all processing is done
with the appropriate \textsf{childdoc} directives,
the argument of \textit{main} of |\childdocmain| can be empty.

An alternative version of the command line processing described
in \secref{sec:commandline} using the detection mechanism reads:
%
\begin{center}
|... -jobname "|\textit{target}|" "|[\textit{flags}]%
[|\def\jobname{|\textit{dest}|}|]|\input{|\textit{main}|}"|
\end{center}

%%%%%%%%%%%%%%%%%%%%%%%%%%%%%%%%%%%%%%%%%%%%%%%%%%%%%%%%%%%%%%%%%%%%%%%%%%%%%%%%
\subsection{Manual Code}
\label{sec:manual}

In case one cannot be certain whether the definitions file |childdoc.def|
is installed on the target \TeX{} distribution
and one prefers not to ship it,
it is conceivable to paste a few relevant commands into the sources.

To that end, drop all statements |\input{childdoc.def}|
and perform the replacements as outlined below.
Instead of |\childdocmain{|\textit{main}|}| add the following code
to the top of the main file:
%
\begin{center}
\begin{tabular}{l}
|\||ifdefined\childdocname\endinput\||fi\newif\ifchilddoc|\\
|\edef\childdocname{\scantokens\expandafter{\jobname\noexpand}}|\\
|\def\childdocmain{|\textit{main}|}\||ifx\childdocmain\childdocname\||else|\\
|\childdoctrue\includeonly{\childdocname}\let\jobname\childdocmain\||fi|\\
\end{tabular}
\end{center}
%
Instead of |\childdocof{|\textit{main}|}| just include the main file
at the top of each child file:
%
\begin{center}
|\input{|\textit{main}|}|
\end{center}
%
A simple redirection |\childdocforward{|\textit{dest}|}| is achieved by:
%
\begin{center}
|\def\jobname{|\textit{dest}|}\input{\jobname}|
\end{center}
%
The redirection with prefix
|\childdocforwardprefix[|\textit{prefix}|]{|\textit{dest}|}|
is accomplished by:
%
\begin{center}
\begin{tabular}{l}
|{\edef\jobname{\scantokens\expandafter{\jobname\noexpand}}|\\
|\def\redirectjob |\textit{prefix}|#1~~~{\gdef\jobname{|\textit{dest}|#1}}|\\
|\expandafter\redirectjob\jobname~~~}\input{\jobname}|
\end{tabular}
\end{center}

In an alternative approach,
child documents can be compiled by a specific command line
without additional code or specific definitions:
%
\begin{center}
|... -jobname "|\textit{target}|" "|[\textit{flags}]%
|\includeonly{|\textit{dest}|}\input{|\textit{main}|}"|
\end{center}
%

%%%%%%%%%%%%%%%%%%%%%%%%%%%%%%%%%%%%%%%%%%%%%%%%%%%%%%%%%%%%%%%%%%%%%%%%%%%%%%%%
%%%%%%%%%%%%%%%%%%%%%%%%%%%%%%%%%%%%%%%%%%%%%%%%%%%%%%%%%%%%%%%%%%%%%%%%%%%%%%%%
\section{Information}

%%%%%%%%%%%%%%%%%%%%%%%%%%%%%%%%%%%%%%%%%%%%%%%%%%%%%%%%%%%%%%%%%%%%%%%%%%%%%%%%
\subsection{Copyright}

Copyright \copyright{} 2017--2018 Niklas Beisert

This work may be distributed and/or modified under the
conditions of the \LaTeX{} Project Public License, either version 1.3
of this license or (at your option) any later version.
The latest version of this license is in
  \url{http://www.latex-project.org/lppl.txt}
and version 1.3 or later is part of all distributions of \LaTeX{}
version 2005/12/01 or later.

This work has the LPPL maintenance status `maintained'.

The Current Maintainer of this work is Niklas Beisert.

This work consists of the files |README.txt|, |childdoc.ins| and |childdoc.dtx|
as well as the derived files |childdoc.def|, |cdocsamp.tex|
with |cdocsch1.tex|, |cdocsch2.tex|, |cdocspt3.tex|, |cdocspt4.tex|,
|cdocsdrf.tex|, |cdocsfn1.tex|, |cdocsfn2.tex|
as well as |childdoc.pdf|.

%%%%%%%%%%%%%%%%%%%%%%%%%%%%%%%%%%%%%%%%%%%%%%%%%%%%%%%%%%%%%%%%%%%%%%%%%%%%%%%%
\subsection{Files and Installation}

The package consists of the files:
%
\begin{center}
\begin{tabular}{ll}
    |README.txt|   & readme file \\
    |childdoc.ins| & installation file \\
    |childdoc.dtx| & source file \\
    |childdoc.def| & definition file \\
    |cdocsamp.tex| & sample main file \\
    |cdocsch1.tex| & sample include file \\
    |cdocsch2.tex| & sample include file \\
    |cdocspt3.tex| & sample part file \\
    |cdocspt4.tex| & sample part file \\
    |cdocsdrf.tex| & sample redirection file \\
    |cdocsfn1.tex| & sample redirection file \\
    |cdocsfn2.tex| & sample redirection file \\
    |childdoc.pdf| & manual
\end{tabular}
\end{center}
%
The distribution consists of the files
|README.txt|, |childdoc.ins| and |childdoc.dtx|.
%
\begin{itemize}
\item
Run (pdf)\LaTeX{} on |childdoc.dtx|
to compile the manual |childdoc.pdf| (this file).
\item
Run \LaTeX{} on |childdoc.ins| to create the definitions file |childdoc.def|
and the sample |cdocsamp.tex| with include files
|cdocsch1.tex|, |cdocsch2.tex|, |cdocspt3.tex|, |cdocspt4.tex|,
|cdocsdrf.tex|, |cdocsfn1.tex|, |cdocsfn2.tex|.
Then copy the file |childdoc.def| to an appropriate directory of your \LaTeX{}
distribution, e.g.\ \textit{texmf-root}|/tex/latex/childdoc|.
\end{itemize}

%%%%%%%%%%%%%%%%%%%%%%%%%%%%%%%%%%%%%%%%%%%%%%%%%%%%%%%%%%%%%%%%%%%%%%%%%%%%%%%%
\subsection{Related CTAN Packages}

There are several other packages which offer a similar functionality:
%
\begin{itemize}
\item
The packages
\href{http://ctan.org/pkg/docmute}{\textsf{docmute}},
\href{http://ctan.org/pkg/includex}{\textsf{includex}} and
\href{http://ctan.org/pkg/standalone}{\textsf{standalone}}
provide commands to include only the document body of
a child file thus allowing both files to be compiled individually.
\item
The packages \href{http://ctan.org/pkg/subdocs}{\textsf{subdocs}}
and \href{http://ctan.org/pkg/subfiles}{\textsf{subfiles}}
provide structures in which the main and child documents can be
encapsulated and allowing them to be compiled individually.
The inclusion mechanism is different from the conventional |\include|.
\item
The package \href{http://ctan.org/pkg/combine}{\textsf{combine}}
is an elaborate solution to combine several documents into one.
\end{itemize}
%
See also the CTAN topic \href{http://ctan.org/topic/subdocs}{\textsf{subdocs}}
for further related packages.
The present package differs from the above solutions in that
a document structure constructed with the conventional |\include| mechanism
just needs two extra commands at the top of every file
such that all constituent files can be compiled individually.

%%%%%%%%%%%%%%%%%%%%%%%%%%%%%%%%%%%%%%%%%%%%%%%%%%%%%%%%%%%%%%%%%%%%%%%%%%%%%%%%
%\subsection{Feature Suggestions}
%
%The following is a list of features which may be useful for future
%versions of this package:
%%
%\begin{itemize}
%\item
%\ldots
%\end{itemize}

%%%%%%%%%%%%%%%%%%%%%%%%%%%%%%%%%%%%%%%%%%%%%%%%%%%%%%%%%%%%%%%%%%%%%%%%%%%%%%%%
\subsection{Revision History}

%%%%%%%%%%%%%%%%%%%%%%%%%%%%%%%%%%%%%%%%
\paragraph{v2.0:} 2018/12/30

\begin{itemize}
\item
immediate forward processing
\item
added |\childdocby| mechanism
\item
manual restructured
\end{itemize}

%%%%%%%%%%%%%%%%%%%%%%%%%%%%%%%%%%%%%%%%
\paragraph{v1.6:} 2018/01/17

\begin{itemize}
\item
application for development of include files
\item
corrections to manual
\end{itemize}

%%%%%%%%%%%%%%%%%%%%%%%%%%%%%%%%%%%%%%%%
\paragraph{v1.5:} 2017/05/21

\begin{itemize}
\item
more complete structuring introduced
\item
|\childdocof| introduced
\item
|\childdoc| renamed to |\childdocmain|
\item
|\childredirect| renamed to |\childdocforward| and |\childdocforwardprefix|
and functionality expanded
\end{itemize}

%%%%%%%%%%%%%%%%%%%%%%%%%%%%%%%%%%%%%%%%
\paragraph{v1.0:} 2017/04/27

\begin{itemize}
\item
manual and install package
\item
first version published on CTAN
\end{itemize}

%%%%%%%%%%%%%%%%%%%%%%%%%%%%%%%%%%%%%%%%
\paragraph{v0.6:} 2017/04/26

\begin{itemize}
\item
redirection mechanism added
\end{itemize}

%%%%%%%%%%%%%%%%%%%%%%%%%%%%%%%%%%%%%%%%
\paragraph{v0.5:} 2017/04/26

\begin{itemize}
\item
functionality in definition file
\end{itemize}


%%%%%%%%%%%%%%%%%%%%%%%%%%%%%%%%%%%%%%%%%%%%%%%%%%%%%%%%%%%%%%%%%%%%%%%%%%%%%%%%
%%%%%%%%%%%%%%%%%%%%%%%%%%%%%%%%%%%%%%%%%%%%%%%%%%%%%%%%%%%%%%%%%%%%%%%%%%%%%%%%
%%%%%%%%%%%%%%%%%%%%%%%%%%%%%%%%%%%%%%%%%%%%%%%%%%%%%%%%%%%%%%%%%%%%%%%%%%%%%%%%
\appendix

\settowidth\MacroIndent{\rmfamily\scriptsize 000\ }

 \DocInput{childdoc.dtx}

\end{document}
%</driver>
% \fi
%
% %%%%%%%%%%%%%%%%%%%%%%%%%%%%%%%%%%%%%%%%%%%%%%%%%%%%%%%%%%%%%%%%%%%%%%%%%%%%%%
% %%%%%%%%%%%%%%%%%%%%%%%%%%%%%%%%%%%%%%%%%%%%%%%%%%%%%%%%%%%%%%%%%%%%%%%%%%%%%%
% \section{Sample}
%\iffalse
%<*samplemain>
%\fi
%
% The following presents a sample document
% with two chapters, two parts, a title page,
% a compile flag as well as three forwarding files to set the flag.
% It consists of eight |.tex| files:
% \begin{center}
% \begin{tabular}{ll}
% |cdocsamp.tex|&main file\\
% |cdocsch1.tex|&include file for chapter 1\\
% |cdocsch2.tex|&include file for chapter 2\\
% |cdocspt3.tex|&include file for part 3\\
% |cdocspt4.tex|&include file for part 4\\
% |cdocsdrf.tex|&forwarding file for main file in draft mode\\
% |cdocsfi1.tex|&forwarding file for final version of chapter 1\\
% |cdocsfi2.tex|&forwarding file for final version of chapter 2\\
% \end{tabular}
% \end{center}
% Each of the eight files can be compiled directly by the \LaTeX{} compiler.
%
% %%%%%%%%%%%%%%%%%%%%%%%%%%%%%%%%%%%%%%
% \paragraph{Main File.}
%
% The main file is called |cdocsamp.tex|.
%
% Load the \textsf{childdoc} definitions and
% declare the filename for the main document:
%    \begin{macrocode}
\input{childdoc.def}
\childdocmain{}
%    \end{macrocode}

% Optional override for |\version| flag:
%    \begin{macrocode}
%%\ifchilddoc\else\providecommand{\version}{draft}\fi
%    \end{macrocode}

% Define the default values for the |\version| flag
% (|final| for the main file and |draft| for childs):
%    \begin{macrocode}
\ifchilddoc
\providecommand{\version}{draft}
\else
\providecommand{\version}{final}
\fi
%    \end{macrocode}

% Load the standard document class:
%    \begin{macrocode}
\documentclass[12pt]{article}
%    \end{macrocode}

% Start the document body:
%    \begin{macrocode}
\begin{document}
%    \end{macrocode}

% Declare a title page.
% Print title, part of document being processed and version flag:
%    \begin{macrocode}
\addtocounter{page}{-1}
\begin{center}
{\LARGE\bfseries{}childdoc example\par}
\vspace{1cm}
\ifchilddoc
\ifchilddocmanual part\else chapter\fi:
`\childdocname' of `\childdocjob'\par
\else
main document: `\childdocjob'\par
\fi
version: \version\par
\end{center}
\newpage
%    \end{macrocode}

% Manually include selected file,
% otherwise process as usual:
%    \begin{macrocode}
\ifchilddocmanual
\section*{part `\childdocname'}
\input{\childdocname}
\else
%    \end{macrocode}

% Include the two chapters:
%    \begin{macrocode}
\include{cdocsch1}
\include{cdocsch2}
%    \end{macrocode}

% Include the two parts unless only chapters should be displayed:
%    \begin{macrocode}
\ifchilddoc\else
\section{part three}
\input{cdocspt3}
\section{part four}
\input{cdocspt4}
\fi
%    \end{macrocode}

% Process as usual until here:
%    \begin{macrocode}
\fi
%    \end{macrocode}

% End of document body:
%    \begin{macrocode}
\end{document}
%    \end{macrocode}
%\iffalse
%</samplemain>
%\fi
%
% %%%%%%%%%%%%%%%%%%%%%%%%%%%%%%%%%%%%%%
% \paragraph{Chapter Include Files.}
%
% The include files are called |cdocsch1.tex| and |cdocsch2.tex|.
%
%\iffalse
%<*samplechap1|samplechap2>
%\fi

% Optional override for |\version| flag:
%    \begin{macrocode}
%%\providecommand{\version}{final}
%    \end{macrocode}

% Include the main document:
%    \begin{macrocode}
\input{childdoc.def}
\childdocof{cdocsamp}
%    \end{macrocode}

%\iffalse
%</samplechap1|samplechap2>
%\fi
%
%\iffalse
%<*samplechap1>
%\fi
% Some text for chapter 1:
%    \begin{macrocode}
\section{one}
some text in chapter one
%    \end{macrocode}

%\iffalse
%</samplechap1>
%\fi
% Some text for chapter 2:
%\iffalse
%<*samplechap2>
%\fi
%    \begin{macrocode}
\section{two}
more text in chapter two
%    \end{macrocode}

%\iffalse
%</samplechap2>
%\fi
%
% %%%%%%%%%%%%%%%%%%%%%%%%%%%%%%%%%%%%%%
% \paragraph{Part Include Files.}
%
% The include files are called |cdocspt3.tex| and |cdocspt4.tex|.
%
%\iffalse
%<*samplepart3|samplepart4>
%\fi

% Optional override for |\version| flag:
%    \begin{macrocode}
%%\providecommand{\version}{final}
%    \end{macrocode}

% Include the main document:
%    \begin{macrocode}
\input{childdoc.def}
\childdocby{cdocsamp}
%    \end{macrocode}

%\iffalse
%</samplepart3|samplepart4>
%\fi
%
%\iffalse
%<*samplepart3>
%\fi
% Some text for part 3:
%    \begin{macrocode}
some text in part three
%    \end{macrocode}

%\iffalse
%</samplepart3>
%\fi
% Some text for part 4:
%\iffalse
%<*samplepart4>
%\fi
%    \begin{macrocode}
more text in part four
%    \end{macrocode}

%\iffalse
%</samplepart4>
%\fi
%
% %%%%%%%%%%%%%%%%%%%%%%%%%%%%%%%%%%%%%%
% \paragraph{Forwarding for a Complete Draft.}
%
% The following forwarding file |cdocsdrf.tex|
% compiles the main document in draft mode:
%\iffalse
%<*sampledraft>
%\fi
%    \begin{macrocode}
\def\version{draft}
\input{childdoc.def}
\childdocforward{cdocsamp}
%    \end{macrocode}

%\iffalse
%</sampledraft>
%\fi
%
% %%%%%%%%%%%%%%%%%%%%%%%%%%%%%%%%%%%%%%
% \paragraph{Forwarding for Final Version of the Chapters.}
%
% The following forwarding files |cdocsfn1.tex| and |cdocsfn2.tex|
% (with identical content)
% compile the final versions of the child documents
% |cdocsch1.tex| and |cdocsch2.tex|, respectively:
%\iffalse
%<*samplefinal>
%\fi
%    \begin{macrocode}
\def\version{final}
\input{childdoc.def}
\childdocforwardprefix[cdocsamp]{cdocsfn}{cdocsch}
%    \end{macrocode}

%\iffalse
%</samplefinal>
%\fi
%
% %%%%%%%%%%%%%%%%%%%%%%%%%%%%%%%%%%%%%%
% \paragraph{Command Line Processing.}
%
% The following three command lines generate the output files
% |cdocscld|, |cdocscl1| and |cdocscl2|
% which should be identical to
% |cdocsdrf|, |cdocsch1| and |cdocsfn2|, respectively:
% \begin{center}
% \begin{tabular}{l}
% |latex -jobname cdocscld \|\\
% |  "\def\version{draft}\input{childdoc.def}\childdocforward{cdocsamp}"|\\
% |latex -jobname cdocscl1 \|\\
% |  "\input{childdoc.def}\childdocforward[cdocsamp]{cdocsch1}"|\\
% |latex -jobname cdocscl2 \|\\
% |  "\def\version{final}\input{childdoc.def}\childdocforward{cdocsch2}"|
% \end{tabular}
% \end{center}
% Note that the trailing backslash on each first line
% merely continues the input to the second line
% (for convenient cut ant paste).
% Furthermore, the command |latex| can be replaced by any
% of its alternative versions such as |pdflatex|.
%
% %%%%%%%%%%%%%%%%%%%%%%%%%%%%%%%%%%%%%%%%%%%%%%%%%%%%%%%%%%%%%%%%%%%%%%%%%%%%%%
% %%%%%%%%%%%%%%%%%%%%%%%%%%%%%%%%%%%%%%%%%%%%%%%%%%%%%%%%%%%%%%%%%%%%%%%%%%%%%%
% \section{Implementation}
%\iffalse
%<*package>
%\fi
%
% This section describes the definitions file |childdoc.def|.

% The definitions cannot be loaded using |\usepackage| or |\RequirePackage|
% which has a mechanism to prevent loading a style file more than once.
% When loading the definitions by means of |\input|
% multiple instances have to be prevented manually:
%\iffalse
%This code needs to be before the `\ProvidesFile' directive
%which is defined at the beginning of this file.
%Therefore it is also placed there and commented out here.
%</package>
%<*discard>
%\fi
%    \begin{macrocode}
\ifdefined\childdocmain\endinput\fi
%    \end{macrocode}
%\iffalse
%</discard>
%<*package>
%\fi
%
% \macro{\ifchilddoc}
% \macro{\ifchilddocmanual}
% The conditional |\ifchilddoc| tells whether a
% child (true) or main (false) document is being compiled.
% The conditional |\ifchilddocmanual| tells whether
% the |\includeonly| mechanism is used (false) or
% the selection of child files must be performed manually (true).
% The definitions initialise to false:
%    \begin{macrocode}
\newif\ifchilddoc
\newif\ifchilddocmanual
%    \end{macrocode}

% \macro{\childdocname}
% \macro{\childdocjob}
% The macro |\childdocname| stores the name of the main document
% to be compiled. The macro |\childdocjob| stores the name of
% the document on which the \LaTeX{} compiler was originally invoked.
% The content of |\jobname| cannot be compared
% to filenames specified in the source due to different catcodes.
% The following code rescans |\jobname|, stores the result
% in |\childdocname| and saves a copy in |\childdocjob|:
%    \begin{macrocode}
\edef\childdocname{\scantokens\expandafter{\jobname\noexpand}}
\let\childdocjob\childdocname
%    \end{macrocode}

% \macro{\childdocdisable}
% The macro |\childdocdisable| prevents the main file
% from being processed more than once.
% At this stage, the main document command |\childdocmain|
% is assumed to be called once again where it should do nothing.
% Any subsequent call to it should prevent
% a secondary processing of the main document
% It overwrites the forwarding commands
% |\childdocof| and |\childdocforward|
% with empty macros to prevent further inclusions of the main document:
%    \begin{macrocode}
\newcommand{\childdocdisable}
{
  \renewcommand{\childdocmain}[1]{\renewcommand{\childdocmain}[1]{\endinput}}
  \renewcommand{\childdocof}[1]{}
  \renewcommand{\childdocby}[2][]{}
  \renewcommand{\childdocforward}[2][]{}
  \renewcommand{\childdocdisable}{}
}
%    \end{macrocode}

% \macro{\childdocmain}
% The macro |\childdocmain| is to be called at the top of the main file
% with nothing or the main filename (without extension) as argument.
% First, it breaks loops.
% If the argument is not empty and does not match |\childdocname|
% (which is set by the first inclusion of |childdoc.def|),
% |\ifchilddoc| is set to true, |\includeonly| is applied to the child file
% and |\jobname| is set to the main file
% (for proper handling of |.aux| files):
%    \begin{macrocode}
\newcommand{\childdocmain}[1]
{
  \childdocdisable\childdocmain{}
  \if?#1?\else
    \begingroup
      \def\childdoctmp{#1}
      \ifx\childdoctmp\childdocname
        \def\childdoctmp{}
      \else
        \def\childdoctmp
        {
          \childdoctrue
          \includeonly{\childdocname}
          \def\childdocjob{#1}
          \def\jobname{#1}
        }
      \fi
      \expandafter
    \endgroup
    \childdoctmp
  \fi
}
%    \end{macrocode}

% \macro{\childdocof}
% The command |\childdocof| redirects
% compilation to the main file |#1|.
%    \begin{macrocode}
\newcommand{\childdocof}[1]
{
  \childdocdisable
  \childdoctrue
  \includeonly{\childdocname}
  \def\jobname{#1}
  \def\childdocjob{#1}
  \input{#1}
}
%    \end{macrocode}

% \macro{\childdocby}
% The command |\childdocby| ....
%    \begin{macrocode}
\newcommand{\childdocby}[2][]
{
  \childdocdisable
  \childdoctrue
  \childdocmanualtrue
  \if?#1?\else
    \def\jobname{#2}
  \fi
  \def\childdocjob{#2}
  \input{#2}
  \endinput
}
%    \end{macrocode}

% \macro{\childdocforward}
% The command |\childdocforward| redirects
% compilation to the main file or
% (if the optional argument is given) a child file.
% Parameters are set as if the main file
% or a child file starting with |\childdocof| was compiled.
% Then compilation is handed over to the main file:
%    \begin{macrocode}
\newcommand{\childdocforward}[2][]
{
  \begingroup
    \if?#1?
      \def\childdoctmp
      {
        \def\childdocname{#2}
        \def\childdocjob{#2}
        \def\jobname{#2}
        \input{#2}
        \endinput
      }
    \else
      \def\childdoctmp
      {
        \childdocdisable
        \def\childdocname{#2}
        \childdoctrue
        \includeonly{#2}
        \def\childdocjob{#1}
        \def\jobname{#1}
        \input{#1}
        \endinput
      }
    \fi
    \expandafter
  \endgroup
  \childdoctmp
}
%    \end{macrocode}

% \macro{\childdocforwardprefix}
% The command |\childdocforwardprefix| redirects
% compilation to the main or a child file by means of a pattern.
% The prefix |#1| in the current filename is replaced by |#2|
% and the suffix of the current filename is kept
% (it is assumed that the filename does not contain the substring `|~~~|'
% which is used as a delimiter).
% Compilation is handed over to the new file by |\childdocforward|:
%    \begin{macrocode}
\newcommand{\childdocforwardprefix}[3][]
{
  \begingroup
    \def\childdocextract #2##1~~~{\def\childdoctmp{\childdocforward[#1]{#3##1}}}
    \expandafter\childdocextract\childdocname~~~
    \expandafter
  \endgroup
  \childdoctmp
}
%    \end{macrocode}

% \macro{\childdoc}
% The deprecated macro |\childdoc| is a legacy version of |\childdocmain|:
%    \begin{macrocode}
\newcommand{\childdoc}{\childdocmain}
%    \end{macrocode}

% \macro{\childdocredirect}
% The deprecated macro |\childdocredirect| is a legacy version
% of |\childdocforward| and |\childdocforwardprefix|:
%    \begin{macrocode}
\newcommand{\childdocredirect}[2][]
{
  \begingroup
    \if?#1?
      \def\childdoctmp{\childdocforward{#2}}
    \else
      \def\childdoctmp{\childdocforwardprefix{#1}{#2}}
    \fi
    \expandafter
  \endgroup
  \childdoctmp
}
%    \end{macrocode}

%\iffalse
%</package>
%\fi
%
\endinput
|\\
|\childdocforward{|\textit{main}|}|\\
\end{tabular}
\end{center}
%
or alternatively with:
%
\begin{center}
\begin{tabular}{l}
|% \iffalse
%
% childdoc.dtx Copyright (C) 2017-2018 Niklas Beisert
%
% This work may be distributed and/or modified under the
% conditions of the LaTeX Project Public License, either version 1.3
% of this license or (at your option) any later version.
% The latest version of this license is in
%   http://www.latex-project.org/lppl.txt
% and version 1.3 or later is part of all distributions of LaTeX
% version 2005/12/01 or later.
%
% This work has the LPPL maintenance status `maintained'.
%
% The Current Maintainer of this work is Niklas Beisert.
%
% This work consists of the files childdoc.dtx and childdoc.ins
% and the derived files childdoc.def and cdocsamp.tex with
% cdocsch1.tex, cdocsch2.tex, cdocsdrf.tex, cdocsfn1.tex, cdocsfn2.tex.
%
%<package>\ifdefined\childdocmain\endinput\fi
%<package>\ProvidesFile{childdoc.def}[2018/12/30 v2.0 child document driver]
%<samplemain>\ProvidesFile{cdocsamp.tex}[2018/12/30 v2.0 sample for childdoc]
%<*driver>
%\ProvidesFile{childdoc.drv}[2018/12/30 v2.0 childdoc reference manual file]
\PassOptionsToClass{10pt,a4paper}{article}
\documentclass{ltxdoc}

\usepackage[margin=35mm]{geometry}
\usepackage{hyperref}
\usepackage{hyperxmp}
\usepackage[usenames]{color}

\hypersetup{colorlinks=true}
\hypersetup{pdfstartview=FitH}
\hypersetup{pdfpagemode=UseNone}
\hypersetup{pdfsource={}}
\hypersetup{pdflang={en-UK}}
\hypersetup{pdfcopyright={Copyright 2017-2018 Niklas Beisert.
  This work may be distributed and/or modified under the
  conditions of the LaTeX Project Public License, either version 1.3
  of this license or (at your option) any later version.}}
\hypersetup{pdflicenseurl={http://www.latex-project.org/lppl.txt}}
\hypersetup{pdfcontactaddress={ETH Zurich, ITP, HIT K,
  Wolfgang-Pauli-Strasse 27}}
\hypersetup{pdfcontactpostcode={8093}}
\hypersetup{pdfcontactcity={Zurich}}
\hypersetup{pdfcontactcountry={Switzerland}}
\hypersetup{pdfcontactemail={nbeisert@itp.phys.ethz.ch}}
\hypersetup{pdfcontacturl={http://people.phys.ethz.ch/\xmptilde nbeisert/}}

\newcommand{\secref}[1]{\hyperref[#1]{section \ref*{#1}}}

\parskip1ex
\parindent0pt
\let\olditemize\itemize
\def\itemize{\olditemize\parskip0pt}

\begin{document}

\title{The \textsf{childdoc} Package}
\hypersetup{pdftitle={The childdoc Package}}
\author{Niklas Beisert\\[2ex]
  Institut f\"ur Theoretische Physik\\
  Eidgen\"ossische Technische Hochschule Z\"urich\\
  Wolfgang-Pauli-Strasse 27, 8093 Z\"urich, Switzerland\\[1ex]
  \href{mailto:nbeisert@itp.phys.ethz.ch}
  {\texttt{nbeisert@itp.phys.ethz.ch}}}
\hypersetup{pdfauthor={Niklas Beisert}}
\hypersetup{pdfsubject={Manual for the LaTeX2e Package childdoc}}
\date{30 December 2018, \textsf{v2.0}}
\maketitle

\begin{abstract}\noindent
\textsf{childdoc} is a \LaTeXe{} package
that enables the direct compilation
of document sections included by |\include|
to individual files.
\end{abstract}

\begingroup
\parskip0ex
\tableofcontents
\endgroup

%%%%%%%%%%%%%%%%%%%%%%%%%%%%%%%%%%%%%%%%%%%%%%%%%%%%%%%%%%%%%%%%%%%%%%%%%%%%%%%%
%%%%%%%%%%%%%%%%%%%%%%%%%%%%%%%%%%%%%%%%%%%%%%%%%%%%%%%%%%%%%%%%%%%%%%%%%%%%%%%%
\section{Introduction}

\LaTeX{} provides a mechanism to structure a large document (such as a book)
into a main file and several child files (containing the chapters)
using the |\include| command.
This mechanism is beneficial for documents
which span hundreds of pages in order to
make the source file(s) more manageable.
Moreover, compilation can be restricted to
selected child files by means of the |\includeonly| command.
The latter feature can be used to reduce the compilation time while editing
(this was significantly more useful in the earlier days of \LaTeX{})
or to generate a smaller document which is easier to navigate.
Another application of |\includeonly| is to generate
documents consisting of selected parts of the complete document.

However, there are a few drawbacks of the plain |\include| mechanism:
\begin{itemize}
\item
The child files cannot be compiled on their own,
they can only be compiled via the main file.
A naive editing environment
(such as a text editor with an option
to have the current file processed by \LaTeX)
may require one to switch to the main file before compiling;
attempting to compile the child file produces errors.
\item
The main file must be modified (each time)
to adjust the |\includeonly| command
to the present needs. This easily leaves the main file in a messy state.
\item
The generated document will always carry the filename
of the main document. This is inconvenient if
several child files are to be compiled and
to be kept for distribution.
\end{itemize}

The present package provides a simple interface
to make child files individually compilable by \LaTeX{}.
Compiling a child file then has the same effect as compiling
the main file with an |\includeonly| command
to select the appropriate child.
Moreover the generated document will carry the name of the child
rather than the main file.
This resolves all three above issues.

This feature is meant to make the editing of books,
thesis documents and lecture notes somewhat more convenient.
However, the package can also be used efficiently for
composing a series of documents (such as exercise sheets)
which are typically distributed individually.
It then assists the author in generating the individual documents
(potentially in different versions)
as well as a document containing the collected series.
Another application is in developing style files
or other kinds of included material
where compilation of the style file could redirect
to a sample or test file.

%%%%%%%%%%%%%%%%%%%%%%%%%%%%%%%%%%%%%%%%%%%%%%%%%%%%%%%%%%%%%%%%%%%%%%%%%%%%%%%%
%%%%%%%%%%%%%%%%%%%%%%%%%%%%%%%%%%%%%%%%%%%%%%%%%%%%%%%%%%%%%%%%%%%%%%%%%%%%%%%%
\section{Usage}

First of all, the package \textsf{childdoc} is \emph{not} a standard
\LaTeXe{} |.sty| style file! Therefore it needs to be invoked in
a non-standard way.

%%%%%%%%%%%%%%%%%%%%%%%%%%%%%%%%%%%%%%%%%%%%%%%%%%%%%%%%%%%%%%%%%%%%%%%%%%%%%%%%
\subsection{Included Files}
\label{sec:include}

%%%%%%%%%%%%%%%%%%%%%%%%%%%%%%%%%%%%%%%%
\DescribeMacro{\childdocmain}
To use the package, add the commands
\begin{center}
\begin{tabular}{l}
|\input{childdoc.def}|\\
|\childdocmain{}|\\
\end{tabular}
\end{center}
at the very top of the main \LaTeX{} file,
in particular \emph{before} the |\documentclass| statement!
The argument of |\childdocmain| should be left empty
(but it must be present).

%%%%%%%%%%%%%%%%%%%%%%%%%%%%%%%%%%%%%%%%
\DescribeMacro{\childdocof}
Furthermore, add the commands
\begin{center}
\begin{tabular}{l}
|\input{childdoc.def}|\\
|\childdocof{|\textit{main}|}|\\
\end{tabular}
\end{center}
at the top of every child file \textit{child}
which is included by |\include{|\textit{child}|}|
from within the main file
(or at least for those files to be compiled individually).
The argument \textit{main} must be the filename of the main file.

There are a couple of
considerations in setting up the main and child documents:

%%%%%%%%%%%%%%%%%%%%%%%%%%%%%%%%%%%%%%%%
\paragraph{Restrictions.}

Please note the following restrictions:
\begin{itemize}
\item
|\childdocmain| must be called with one argument \textit{main}
to ensure compatibility with earlier version of the package.
It must either be empty (|\childdocmain{}|)
or precisely match the filename of the main file in which it is specified.
See \secref{sec:detection} for further information.
\item
The filename \textit{main} must be specified without the |.tex| extension.
\item
The filename \textit{main} is case sensitive
(even in case-insensitive file systems)
due to internal string comparison.
\item
The argument \textit{main} should be fully expanded, it cannot be a macro.
\item
Subdirectories and special characters should be avoided in filenames.
\item
The command |\childdocmain{|\textit{main}|}| must be followed by a whitespace.
It should not be followed immediately by another command
or by a comment mark `|%|'.
This is because the \TeX{} parser reads the token immediately following
the argument of |\childdocmain| and puts it
at the beginning of every child section;
however, a white\-space is ignored.
\end{itemize}

%%%%%%%%%%%%%%%%%%%%%%%%%%%%%%%%%%%%%%%%
\paragraph{Content of Main File.}

It is advisable to place all content in the child files included by |\include|.
Any output contained in the main file will appear in all child documents
unless suppressed manually;
it cannot be suppressed automatically by the |\includeonly| directive
and thus should normally be avoided.
A method to include some content in the main file
by means of conditional processing is described in \secref{sec:conditional}.

%%%%%%%%%%%%%%%%%%%%%%%%%%%%%%%%%%%%%%%%
\paragraph{Page Numbering.}

When only a part of the document is compiled,
the appropriate numbering of pages
(as well as other status parameters)
is determined from the |.aux| files.
The latter contain information from previous passes.
However this information needs to propagate through
all intermediate child documents.
Therefore the page numbering in child documents may well
be inconsistent until the complete document is compiled at least once.

A useful (if unconventional) way to always ensure a consistent
page numbering is to restart the numbering in each child document
and denote the pages by `\textit{child}|.|\textit{page}'
where \textit{child} represents the chapter/section number of the child file.
This can be achieved by the command
|\numberwithin{page}{|\textit{child}|}|
of the \textsf{amsmath} package
where \textit{child} can be |chapter| or |section|
depending on the chosen structuring.
Alternatively, one can modify the macro |\thepage| appropriately
and reset the counter |page| at the start of each child file.

%%%%%%%%%%%%%%%%%%%%%%%%%%%%%%%%%%%%%%%%%%%%%%%%%%%%%%%%%%%%%%%%%%%%%%%%%%%%%%%%
\subsection{Conditional Processing}
\label{sec:conditional}

The package provides a mechanism to compile different versions
of a document. To customise the versions further some conditional processing
can come in handy to distinguish which version is being compiled.
The package provides two macros to describe the compilation context:

%%%%%%%%%%%%%%%%%%%%%%%%%%%%%%%%%%%%%%%%
\DescribeMacro{\ifchilddoc}
The conditional |\ifchilddoc| distinguishes between the compilation of
child documents and the main document:
%
\begin{center}
|\ifchilddoc |\textit{child-code}| |[|\||else |\textit{main-code}]| \||fi|
\end{center}

%%%%%%%%%%%%%%%%%%%%%%%%%%%%%%%%%%%%%%%%
\DescribeMacro{\childdocname}
\DescribeMacro{\childdocjob}
The macro |\childdocname| contains the filename (without extension)
of the main or child file being processed.
Note that |\childdocjob| will always contain the name of the main file.

%%%%%%%%%%%%%%%%%%%%%%%%%%%%%%%%%%%%%%%%
\paragraph{Title Page.}

Conditional processing can be used to include a title or banner page
in the main document when proper precautions are taken.
Importantly, the code in the main file should ensure that the page counter
(as well as other status parameters which are stored in the |.aux| files)
takes the same value after the conditional processing.
Otherwise the page numbers may take divergent values
depending on which part is compiled.

For example, a title page could be declared by:
%
\begin{center}
\begin{tabular}{l}
|\ifchilddoc\||else|\\
|\addtocounter{page}{-1}|\\
\textit{code for title page}\\
|\newpage|\\
|\||fi|
\end{tabular}
\end{center}
%
A banner page for the child documents can be generated by:
%
\begin{center}
\begin{tabular}{l}
|\ifchilddoc|\\
|\addtocounter{page}{-1}|\\
\textit{code for banner page}\\
|\newpage|\\
|\||fi|
\end{tabular}
\end{center}
%
Here one could write a message such as:
\begin{center}
|This is the part \childdocname{} of \childdocjob{}.|
\end{center}

%%%%%%%%%%%%%%%%%%%%%%%%%%%%%%%%%%%%%%%%%%%%%%%%%%%%%%%%%%%%%%%%%%%%%%%%%%%%%%%%
\subsection{Flags}
\label{sec:flags}

The package makes it easy to generate different versions
of the main or child documents.
To this end compilation flags can be defined
and assigned different default values.
They will be particularly useful in conjunction
with the forwarding mechanism described in \secref{sec:forward}.

For example, it may be useful to have a flag |\version|
which can be set to |draft| or |final|.
The document source will contain some conditional code
depending on the value of |\version|.
Suppose further, the flag should default to |final| for the main file
and to |draft| for child files
which is a natural assignment for editing the document.
This is achieved by placing the following code
in the preamble of the main document
(below the |\childdocmain| directive):
%
\begin{center}
\begin{tabular}{l}
|\ifchilddoc|\\
|\providecommand{\version}{draft}|\\
|\||else|\\
|\providecommand{\version}{final}|\\
|\||fi|
\end{tabular}
\end{center}
%
The definition by |\providecommand| makes sure
that previous definitions are not overwritten.
Further statements |\providecommand{\version}{...}|
can thus be added before the above code to override it.

For the main file, one might add a line
(between |\childdocmain| and the above block)
%
\begin{center}
|%\ifchilddoc\||else\providecommand{\version}{draft}\||fi|
\end{center}
%
which can be uncommented to produce a draft version.
Likewise one can add a line to the very top of a child file
(above the |\childdocof{|\textit{main}|}| directive)
%
\begin{center}
|%\providecommand{\version}{final}|
\end{center}
%
which can be uncommented to produce the final version of this child document.

%%%%%%%%%%%%%%%%%%%%%%%%%%%%%%%%%%%%%%%%%%%%%%%%%%%%%%%%%%%%%%%%%%%%%%%%%%%%%%%%
\subsection{Forwarding}
\label{sec:forward}

Different versions of the main or child documents
using compilation flags as described in \secref{sec:flags}
can be (permanently) stored in different files
for convenient compilation, viewing and distribution.
To this end, the package defines a command
to pass on compilation to a different file:

%%%%%%%%%%%%%%%%%%%%%%%%%%%%%%%%%%%%%%%%
\DescribeMacro{\childdocforward}
The command |\childdocforward| redirects processing to
another source file:
%
\begin{center}
\begin{tabular}{l}
|\input{childdoc.def}|\\
|\childdocforward[|\textit{main}|]{|\textit{dest}|}|\\
\end{tabular}
\end{center}
%
The argument \textit{dest} is the destination file
(without extension).
It should be the main file or one of the child files.
Note that further \textsf{childdoc} directives
such as |\childdocof| and |\childdocforward|
in the indicated file will be processed in this form.
The optional argument \textit{main}
passes on directly to the main file \textit{main}
while pretending to compile the child \textit{dest}.
This form behaves as if \textit{dest}
issues |\childdocof{|\textit{main}|}| right away,
and no further \textsf{childdoc} directives will be processed.

%%%%%%%%%%%%%%%%%%%%%%%%%%%%%%%%%%%%%%%%
\DescribeMacro{\...prefix}
In the alternative form |\childdocforwardprefix|,
%
\begin{center}
\begin{tabular}{l}
|\input{childdoc.def}|\\
|\childdocforwardprefix[|\textit{main}|]{|\textit{prefix}|}{|\textit{dest}|}|
\end{tabular}
\end{center}
%
the destination file is determined by a pattern
depending on the current file:
To make this work, the current file must be called
`{\textit{prefix}\hspace{0.2em}\textit{suffix}}'
with \textit{prefix} matching precisely the argument.
Processing is then passed on to the file
`{\textit{dest}\hspace{0.2em}\textit{suffix}}'.
Surely, the same effect is achieved by
directly specifying the
argument `{\textit{dest}\hspace{0.2em}\textit{suffix}}'
in the first form.
However, that requires to set up a different file
for each child. With the alternative form of the command
all these files can have exactly the same content
which simplifies setting them up and maintaining them.

For example, the following file |draft.tex|
with a compilation flag |\version| as described in \secref{sec:flags}
compiles the main document as a draft:
%
\begin{center}
\begin{tabular}{l}
|\def\version{draft}|\\
|\input{childdoc.def}|\\
|\childdocforward{|\textit{main}|}|
\end{tabular}
\end{center}
%
Likewise, the following files |final|\textit{nn}|.tex|
compile the final version of the child document
|child|\textit{nn}|.tex|:
%
\begin{center}
\begin{tabular}{l}
|\def\version{final}|\\
|\input{childdoc.def}|\\
|\childdocforwardprefix{final}{child}|
\end{tabular}
\end{center}
%

Note that when several versions of a main file and/or of each child file
are to be generated, it may be convenient to set up a |Makefile| or
shell script to automatise the process.

%%%%%%%%%%%%%%%%%%%%%%%%%%%%%%%%%%%%%%%%%%%%%%%%%%%%%%%%%%%%%%%%%%%%%%%%%%%%%%%%
\subsection{Command Line Processing}
\label{sec:commandline}

The effect of redirection files can also be achieved by invoking
the \LaTeX{} compiler with a more elaborate command line.
Most conveniently this should be done as part
of a shell script or a |Makefile|.

When using \textsf{childdoc} in the main file, the following
command lines effectively perform a redirection
(note that depending on the shell being used,
backslashes may have to be doubled: `|\|' $\to$ `|\\|'):
%
\begin{center}
|... -jobname "|\textit{target}|" |\\|"|[\textit{flags}]%
|\input{childdoc.def}\childdocforward[|\textit{main}|]{|\textit{dest}|}"|
\end{center}
%
Here \textit{target} is the name of the output file,
\textit{main} is the name of the main file
and \textit{dest} is the name of the main or child file to be processed
(all filenames without extensions).
The optional argument \textit{main} can be omitted
if \textit{main} matches \textit{dest}.
Optionally, compilation \textit{flags} can be defined via |\def| commands.
This command line makes the \TeX{} engine believe
it is compiling the file \textit{target}
whose content is specified as the latter parameter.
The provided code then forwards the processing to
\textit{main} or \textit{dest} as described in \secref{sec:forward}.

%%%%%%%%%%%%%%%%%%%%%%%%%%%%%%%%%%%%%%%%%%%%%%%%%%%%%%%%%%%%%%%%%%%%%%%%%%%%%%%%
\subsection{Include by Input}
\label{sec:input}

Including child documents by |\include| has some restrictions by design.
Most notably, the content of a child document always occupies
its own set of pages; pages cannot be shared between child documents.
Usually, this behaviour makes perfect sense
because each child document contain an essential part of the document.
However, in some situations it may be desirable to compose
a document from a collection of parts
without having mandatory page breaks between then.
For this case, the package
provides a mechanism to include parts
by |\input| which can also be processed individually.
However, by construction this mechanism
requires manual handling of the content to be output.

%%%%%%%%%%%%%%%%%%%%%%%%%%%%%%%%%%%%%%%%
\DescribeMacro{\ifchilddocmanual}
The main file should be prepared as usual, see \secref{sec:include}.
However, the document body must make a distinction
between processing of an individual part and of the main document, e.g.:
%
\begin{center}
\begin{tabular}{l}
|\ifchilddocmanual|\\
|\input{\childdocname}|\\
|\||else|\\
\textit{document body with }|\input{|\textit{part}|}|\\
|\||fi|
\end{tabular}
\end{center}
%
The conditional |\ifchilddocmanual| is true whenever
a part to be included by |\input| is being compiled,
and the name of the part is stored in |\childdocname|.

%%%%%%%%%%%%%%%%%%%%%%%%%%%%%%%%%%%%%%%%
\DescribeMacro{\childdocby}
Each part to be included by |\input| should start with:
%
\begin{center}
\begin{tabular}{l}
|\input{childdoc.def}|\\
|\childdocby{|\textit{main}|}|\\
\end{tabular}
\end{center}
%
The directive |\childdocby| is similar to |\childdocof|
described in \secref{sec:include},
but the subsequent selection of content must be done manually.
To that end, both |\ifchilddoc| and |\ifchilddocmanual|
will be true upon processing of a part,
and the name of the part is stored in |\childdocname|.
Note that |\jobname| will be set to the filename of the current part
so that each part receives an individual |.aux| file
that does not interfere with the |.aux| file(s) of the main document.
This behaviour can be altered by the alternative form
|\childdocby[*]{|\textit{main}|}| (with a non-empty optional argument)
which uses the |.aux| file of the main document
by setting |\jobname| to \textit{main}.

%%%%%%%%%%%%%%%%%%%%%%%%%%%%%%%%%%%%%%%%%%%%%%%%%%%%%%%%%%%%%%%%%%%%%%%%%%%%%%%%
\subsection{Driver Development}
\label{sec:driver}

The \textsf{childdoc} mechanism can also be use for the development
of definition files such as \LaTeX{} styles or classes.
This case differs from the above setup with multiple parts
included by |\include| in that no |\includeonly| should be invoked.
This can be achieved by starting the include file
(before |\ProvidesPackage|) with:
%
\begin{center}
\begin{tabular}{l}
|\input{childdoc.def}|\\
|\childdocforward{|\textit{main}|}|\\
\end{tabular}
\end{center}
%
or alternatively with:
%
\begin{center}
\begin{tabular}{l}
|\input{childdoc.def}|\\
|\childdocby{|\textit{main}|}|\\
\end{tabular}
\end{center}
%
Both forms have slightly different effects as described above.
The main file is prepared as usual, see \secref{sec:include}.

%%%%%%%%%%%%%%%%%%%%%%%%%%%%%%%%%%%%%%%%%%%%%%%%%%%%%%%%%%%%%%%%%%%%%%%%%%%%%%%%
\subsection{Legacy Detection}
\label{sec:detection}

The directive |\childdocmain| in the main file can detect
whether the complete document or merely a child is to be compiled
even without using the directive |\childdocof|.
This method is deprecated because it is less robust
and there is no compelling reason to use it;
it is merely provided for backward compatibility
and it may be removed in future versions.

If the detection mechanism is to be used,
it is mandatory to correctly specify
the filename of the main file as the argument of |\childdocmain|:
%
\begin{center}
\begin{tabular}{l}
|\input{childdoc.def}|\\
|\childdocmain{|\textit{main}|}|\\
\end{tabular}
\end{center}
%
If |\jobname| does not match the argument \textit{main} of |\childdocmain|,
it is assumed that |\jobname| points to the child file to be compiled.
When using |\childdocmain| with the main file specified as argument,
it suffices to start a child file
with just |\input{|\textit{main}|}|
without loading of the package and using |\childdocof|.
If instead all processing is done
with the appropriate \textsf{childdoc} directives,
the argument of \textit{main} of |\childdocmain| can be empty.

An alternative version of the command line processing described
in \secref{sec:commandline} using the detection mechanism reads:
%
\begin{center}
|... -jobname "|\textit{target}|" "|[\textit{flags}]%
[|\def\jobname{|\textit{dest}|}|]|\input{|\textit{main}|}"|
\end{center}

%%%%%%%%%%%%%%%%%%%%%%%%%%%%%%%%%%%%%%%%%%%%%%%%%%%%%%%%%%%%%%%%%%%%%%%%%%%%%%%%
\subsection{Manual Code}
\label{sec:manual}

In case one cannot be certain whether the definitions file |childdoc.def|
is installed on the target \TeX{} distribution
and one prefers not to ship it,
it is conceivable to paste a few relevant commands into the sources.

To that end, drop all statements |\input{childdoc.def}|
and perform the replacements as outlined below.
Instead of |\childdocmain{|\textit{main}|}| add the following code
to the top of the main file:
%
\begin{center}
\begin{tabular}{l}
|\||ifdefined\childdocname\endinput\||fi\newif\ifchilddoc|\\
|\edef\childdocname{\scantokens\expandafter{\jobname\noexpand}}|\\
|\def\childdocmain{|\textit{main}|}\||ifx\childdocmain\childdocname\||else|\\
|\childdoctrue\includeonly{\childdocname}\let\jobname\childdocmain\||fi|\\
\end{tabular}
\end{center}
%
Instead of |\childdocof{|\textit{main}|}| just include the main file
at the top of each child file:
%
\begin{center}
|\input{|\textit{main}|}|
\end{center}
%
A simple redirection |\childdocforward{|\textit{dest}|}| is achieved by:
%
\begin{center}
|\def\jobname{|\textit{dest}|}\input{\jobname}|
\end{center}
%
The redirection with prefix
|\childdocforwardprefix[|\textit{prefix}|]{|\textit{dest}|}|
is accomplished by:
%
\begin{center}
\begin{tabular}{l}
|{\edef\jobname{\scantokens\expandafter{\jobname\noexpand}}|\\
|\def\redirectjob |\textit{prefix}|#1~~~{\gdef\jobname{|\textit{dest}|#1}}|\\
|\expandafter\redirectjob\jobname~~~}\input{\jobname}|
\end{tabular}
\end{center}

In an alternative approach,
child documents can be compiled by a specific command line
without additional code or specific definitions:
%
\begin{center}
|... -jobname "|\textit{target}|" "|[\textit{flags}]%
|\includeonly{|\textit{dest}|}\input{|\textit{main}|}"|
\end{center}
%

%%%%%%%%%%%%%%%%%%%%%%%%%%%%%%%%%%%%%%%%%%%%%%%%%%%%%%%%%%%%%%%%%%%%%%%%%%%%%%%%
%%%%%%%%%%%%%%%%%%%%%%%%%%%%%%%%%%%%%%%%%%%%%%%%%%%%%%%%%%%%%%%%%%%%%%%%%%%%%%%%
\section{Information}

%%%%%%%%%%%%%%%%%%%%%%%%%%%%%%%%%%%%%%%%%%%%%%%%%%%%%%%%%%%%%%%%%%%%%%%%%%%%%%%%
\subsection{Copyright}

Copyright \copyright{} 2017--2018 Niklas Beisert

This work may be distributed and/or modified under the
conditions of the \LaTeX{} Project Public License, either version 1.3
of this license or (at your option) any later version.
The latest version of this license is in
  \url{http://www.latex-project.org/lppl.txt}
and version 1.3 or later is part of all distributions of \LaTeX{}
version 2005/12/01 or later.

This work has the LPPL maintenance status `maintained'.

The Current Maintainer of this work is Niklas Beisert.

This work consists of the files |README.txt|, |childdoc.ins| and |childdoc.dtx|
as well as the derived files |childdoc.def|, |cdocsamp.tex|
with |cdocsch1.tex|, |cdocsch2.tex|, |cdocspt3.tex|, |cdocspt4.tex|,
|cdocsdrf.tex|, |cdocsfn1.tex|, |cdocsfn2.tex|
as well as |childdoc.pdf|.

%%%%%%%%%%%%%%%%%%%%%%%%%%%%%%%%%%%%%%%%%%%%%%%%%%%%%%%%%%%%%%%%%%%%%%%%%%%%%%%%
\subsection{Files and Installation}

The package consists of the files:
%
\begin{center}
\begin{tabular}{ll}
    |README.txt|   & readme file \\
    |childdoc.ins| & installation file \\
    |childdoc.dtx| & source file \\
    |childdoc.def| & definition file \\
    |cdocsamp.tex| & sample main file \\
    |cdocsch1.tex| & sample include file \\
    |cdocsch2.tex| & sample include file \\
    |cdocspt3.tex| & sample part file \\
    |cdocspt4.tex| & sample part file \\
    |cdocsdrf.tex| & sample redirection file \\
    |cdocsfn1.tex| & sample redirection file \\
    |cdocsfn2.tex| & sample redirection file \\
    |childdoc.pdf| & manual
\end{tabular}
\end{center}
%
The distribution consists of the files
|README.txt|, |childdoc.ins| and |childdoc.dtx|.
%
\begin{itemize}
\item
Run (pdf)\LaTeX{} on |childdoc.dtx|
to compile the manual |childdoc.pdf| (this file).
\item
Run \LaTeX{} on |childdoc.ins| to create the definitions file |childdoc.def|
and the sample |cdocsamp.tex| with include files
|cdocsch1.tex|, |cdocsch2.tex|, |cdocspt3.tex|, |cdocspt4.tex|,
|cdocsdrf.tex|, |cdocsfn1.tex|, |cdocsfn2.tex|.
Then copy the file |childdoc.def| to an appropriate directory of your \LaTeX{}
distribution, e.g.\ \textit{texmf-root}|/tex/latex/childdoc|.
\end{itemize}

%%%%%%%%%%%%%%%%%%%%%%%%%%%%%%%%%%%%%%%%%%%%%%%%%%%%%%%%%%%%%%%%%%%%%%%%%%%%%%%%
\subsection{Related CTAN Packages}

There are several other packages which offer a similar functionality:
%
\begin{itemize}
\item
The packages
\href{http://ctan.org/pkg/docmute}{\textsf{docmute}},
\href{http://ctan.org/pkg/includex}{\textsf{includex}} and
\href{http://ctan.org/pkg/standalone}{\textsf{standalone}}
provide commands to include only the document body of
a child file thus allowing both files to be compiled individually.
\item
The packages \href{http://ctan.org/pkg/subdocs}{\textsf{subdocs}}
and \href{http://ctan.org/pkg/subfiles}{\textsf{subfiles}}
provide structures in which the main and child documents can be
encapsulated and allowing them to be compiled individually.
The inclusion mechanism is different from the conventional |\include|.
\item
The package \href{http://ctan.org/pkg/combine}{\textsf{combine}}
is an elaborate solution to combine several documents into one.
\end{itemize}
%
See also the CTAN topic \href{http://ctan.org/topic/subdocs}{\textsf{subdocs}}
for further related packages.
The present package differs from the above solutions in that
a document structure constructed with the conventional |\include| mechanism
just needs two extra commands at the top of every file
such that all constituent files can be compiled individually.

%%%%%%%%%%%%%%%%%%%%%%%%%%%%%%%%%%%%%%%%%%%%%%%%%%%%%%%%%%%%%%%%%%%%%%%%%%%%%%%%
%\subsection{Feature Suggestions}
%
%The following is a list of features which may be useful for future
%versions of this package:
%%
%\begin{itemize}
%\item
%\ldots
%\end{itemize}

%%%%%%%%%%%%%%%%%%%%%%%%%%%%%%%%%%%%%%%%%%%%%%%%%%%%%%%%%%%%%%%%%%%%%%%%%%%%%%%%
\subsection{Revision History}

%%%%%%%%%%%%%%%%%%%%%%%%%%%%%%%%%%%%%%%%
\paragraph{v2.0:} 2018/12/30

\begin{itemize}
\item
immediate forward processing
\item
added |\childdocby| mechanism
\item
manual restructured
\end{itemize}

%%%%%%%%%%%%%%%%%%%%%%%%%%%%%%%%%%%%%%%%
\paragraph{v1.6:} 2018/01/17

\begin{itemize}
\item
application for development of include files
\item
corrections to manual
\end{itemize}

%%%%%%%%%%%%%%%%%%%%%%%%%%%%%%%%%%%%%%%%
\paragraph{v1.5:} 2017/05/21

\begin{itemize}
\item
more complete structuring introduced
\item
|\childdocof| introduced
\item
|\childdoc| renamed to |\childdocmain|
\item
|\childredirect| renamed to |\childdocforward| and |\childdocforwardprefix|
and functionality expanded
\end{itemize}

%%%%%%%%%%%%%%%%%%%%%%%%%%%%%%%%%%%%%%%%
\paragraph{v1.0:} 2017/04/27

\begin{itemize}
\item
manual and install package
\item
first version published on CTAN
\end{itemize}

%%%%%%%%%%%%%%%%%%%%%%%%%%%%%%%%%%%%%%%%
\paragraph{v0.6:} 2017/04/26

\begin{itemize}
\item
redirection mechanism added
\end{itemize}

%%%%%%%%%%%%%%%%%%%%%%%%%%%%%%%%%%%%%%%%
\paragraph{v0.5:} 2017/04/26

\begin{itemize}
\item
functionality in definition file
\end{itemize}


%%%%%%%%%%%%%%%%%%%%%%%%%%%%%%%%%%%%%%%%%%%%%%%%%%%%%%%%%%%%%%%%%%%%%%%%%%%%%%%%
%%%%%%%%%%%%%%%%%%%%%%%%%%%%%%%%%%%%%%%%%%%%%%%%%%%%%%%%%%%%%%%%%%%%%%%%%%%%%%%%
%%%%%%%%%%%%%%%%%%%%%%%%%%%%%%%%%%%%%%%%%%%%%%%%%%%%%%%%%%%%%%%%%%%%%%%%%%%%%%%%
\appendix

\settowidth\MacroIndent{\rmfamily\scriptsize 000\ }

 \DocInput{childdoc.dtx}

\end{document}
%</driver>
% \fi
%
% %%%%%%%%%%%%%%%%%%%%%%%%%%%%%%%%%%%%%%%%%%%%%%%%%%%%%%%%%%%%%%%%%%%%%%%%%%%%%%
% %%%%%%%%%%%%%%%%%%%%%%%%%%%%%%%%%%%%%%%%%%%%%%%%%%%%%%%%%%%%%%%%%%%%%%%%%%%%%%
% \section{Sample}
%\iffalse
%<*samplemain>
%\fi
%
% The following presents a sample document
% with two chapters, two parts, a title page,
% a compile flag as well as three forwarding files to set the flag.
% It consists of eight |.tex| files:
% \begin{center}
% \begin{tabular}{ll}
% |cdocsamp.tex|&main file\\
% |cdocsch1.tex|&include file for chapter 1\\
% |cdocsch2.tex|&include file for chapter 2\\
% |cdocspt3.tex|&include file for part 3\\
% |cdocspt4.tex|&include file for part 4\\
% |cdocsdrf.tex|&forwarding file for main file in draft mode\\
% |cdocsfi1.tex|&forwarding file for final version of chapter 1\\
% |cdocsfi2.tex|&forwarding file for final version of chapter 2\\
% \end{tabular}
% \end{center}
% Each of the eight files can be compiled directly by the \LaTeX{} compiler.
%
% %%%%%%%%%%%%%%%%%%%%%%%%%%%%%%%%%%%%%%
% \paragraph{Main File.}
%
% The main file is called |cdocsamp.tex|.
%
% Load the \textsf{childdoc} definitions and
% declare the filename for the main document:
%    \begin{macrocode}
\input{childdoc.def}
\childdocmain{}
%    \end{macrocode}

% Optional override for |\version| flag:
%    \begin{macrocode}
%%\ifchilddoc\else\providecommand{\version}{draft}\fi
%    \end{macrocode}

% Define the default values for the |\version| flag
% (|final| for the main file and |draft| for childs):
%    \begin{macrocode}
\ifchilddoc
\providecommand{\version}{draft}
\else
\providecommand{\version}{final}
\fi
%    \end{macrocode}

% Load the standard document class:
%    \begin{macrocode}
\documentclass[12pt]{article}
%    \end{macrocode}

% Start the document body:
%    \begin{macrocode}
\begin{document}
%    \end{macrocode}

% Declare a title page.
% Print title, part of document being processed and version flag:
%    \begin{macrocode}
\addtocounter{page}{-1}
\begin{center}
{\LARGE\bfseries{}childdoc example\par}
\vspace{1cm}
\ifchilddoc
\ifchilddocmanual part\else chapter\fi:
`\childdocname' of `\childdocjob'\par
\else
main document: `\childdocjob'\par
\fi
version: \version\par
\end{center}
\newpage
%    \end{macrocode}

% Manually include selected file,
% otherwise process as usual:
%    \begin{macrocode}
\ifchilddocmanual
\section*{part `\childdocname'}
\input{\childdocname}
\else
%    \end{macrocode}

% Include the two chapters:
%    \begin{macrocode}
\include{cdocsch1}
\include{cdocsch2}
%    \end{macrocode}

% Include the two parts unless only chapters should be displayed:
%    \begin{macrocode}
\ifchilddoc\else
\section{part three}
\input{cdocspt3}
\section{part four}
\input{cdocspt4}
\fi
%    \end{macrocode}

% Process as usual until here:
%    \begin{macrocode}
\fi
%    \end{macrocode}

% End of document body:
%    \begin{macrocode}
\end{document}
%    \end{macrocode}
%\iffalse
%</samplemain>
%\fi
%
% %%%%%%%%%%%%%%%%%%%%%%%%%%%%%%%%%%%%%%
% \paragraph{Chapter Include Files.}
%
% The include files are called |cdocsch1.tex| and |cdocsch2.tex|.
%
%\iffalse
%<*samplechap1|samplechap2>
%\fi

% Optional override for |\version| flag:
%    \begin{macrocode}
%%\providecommand{\version}{final}
%    \end{macrocode}

% Include the main document:
%    \begin{macrocode}
\input{childdoc.def}
\childdocof{cdocsamp}
%    \end{macrocode}

%\iffalse
%</samplechap1|samplechap2>
%\fi
%
%\iffalse
%<*samplechap1>
%\fi
% Some text for chapter 1:
%    \begin{macrocode}
\section{one}
some text in chapter one
%    \end{macrocode}

%\iffalse
%</samplechap1>
%\fi
% Some text for chapter 2:
%\iffalse
%<*samplechap2>
%\fi
%    \begin{macrocode}
\section{two}
more text in chapter two
%    \end{macrocode}

%\iffalse
%</samplechap2>
%\fi
%
% %%%%%%%%%%%%%%%%%%%%%%%%%%%%%%%%%%%%%%
% \paragraph{Part Include Files.}
%
% The include files are called |cdocspt3.tex| and |cdocspt4.tex|.
%
%\iffalse
%<*samplepart3|samplepart4>
%\fi

% Optional override for |\version| flag:
%    \begin{macrocode}
%%\providecommand{\version}{final}
%    \end{macrocode}

% Include the main document:
%    \begin{macrocode}
\input{childdoc.def}
\childdocby{cdocsamp}
%    \end{macrocode}

%\iffalse
%</samplepart3|samplepart4>
%\fi
%
%\iffalse
%<*samplepart3>
%\fi
% Some text for part 3:
%    \begin{macrocode}
some text in part three
%    \end{macrocode}

%\iffalse
%</samplepart3>
%\fi
% Some text for part 4:
%\iffalse
%<*samplepart4>
%\fi
%    \begin{macrocode}
more text in part four
%    \end{macrocode}

%\iffalse
%</samplepart4>
%\fi
%
% %%%%%%%%%%%%%%%%%%%%%%%%%%%%%%%%%%%%%%
% \paragraph{Forwarding for a Complete Draft.}
%
% The following forwarding file |cdocsdrf.tex|
% compiles the main document in draft mode:
%\iffalse
%<*sampledraft>
%\fi
%    \begin{macrocode}
\def\version{draft}
\input{childdoc.def}
\childdocforward{cdocsamp}
%    \end{macrocode}

%\iffalse
%</sampledraft>
%\fi
%
% %%%%%%%%%%%%%%%%%%%%%%%%%%%%%%%%%%%%%%
% \paragraph{Forwarding for Final Version of the Chapters.}
%
% The following forwarding files |cdocsfn1.tex| and |cdocsfn2.tex|
% (with identical content)
% compile the final versions of the child documents
% |cdocsch1.tex| and |cdocsch2.tex|, respectively:
%\iffalse
%<*samplefinal>
%\fi
%    \begin{macrocode}
\def\version{final}
\input{childdoc.def}
\childdocforwardprefix[cdocsamp]{cdocsfn}{cdocsch}
%    \end{macrocode}

%\iffalse
%</samplefinal>
%\fi
%
% %%%%%%%%%%%%%%%%%%%%%%%%%%%%%%%%%%%%%%
% \paragraph{Command Line Processing.}
%
% The following three command lines generate the output files
% |cdocscld|, |cdocscl1| and |cdocscl2|
% which should be identical to
% |cdocsdrf|, |cdocsch1| and |cdocsfn2|, respectively:
% \begin{center}
% \begin{tabular}{l}
% |latex -jobname cdocscld \|\\
% |  "\def\version{draft}\input{childdoc.def}\childdocforward{cdocsamp}"|\\
% |latex -jobname cdocscl1 \|\\
% |  "\input{childdoc.def}\childdocforward[cdocsamp]{cdocsch1}"|\\
% |latex -jobname cdocscl2 \|\\
% |  "\def\version{final}\input{childdoc.def}\childdocforward{cdocsch2}"|
% \end{tabular}
% \end{center}
% Note that the trailing backslash on each first line
% merely continues the input to the second line
% (for convenient cut ant paste).
% Furthermore, the command |latex| can be replaced by any
% of its alternative versions such as |pdflatex|.
%
% %%%%%%%%%%%%%%%%%%%%%%%%%%%%%%%%%%%%%%%%%%%%%%%%%%%%%%%%%%%%%%%%%%%%%%%%%%%%%%
% %%%%%%%%%%%%%%%%%%%%%%%%%%%%%%%%%%%%%%%%%%%%%%%%%%%%%%%%%%%%%%%%%%%%%%%%%%%%%%
% \section{Implementation}
%\iffalse
%<*package>
%\fi
%
% This section describes the definitions file |childdoc.def|.

% The definitions cannot be loaded using |\usepackage| or |\RequirePackage|
% which has a mechanism to prevent loading a style file more than once.
% When loading the definitions by means of |\input|
% multiple instances have to be prevented manually:
%\iffalse
%This code needs to be before the `\ProvidesFile' directive
%which is defined at the beginning of this file.
%Therefore it is also placed there and commented out here.
%</package>
%<*discard>
%\fi
%    \begin{macrocode}
\ifdefined\childdocmain\endinput\fi
%    \end{macrocode}
%\iffalse
%</discard>
%<*package>
%\fi
%
% \macro{\ifchilddoc}
% \macro{\ifchilddocmanual}
% The conditional |\ifchilddoc| tells whether a
% child (true) or main (false) document is being compiled.
% The conditional |\ifchilddocmanual| tells whether
% the |\includeonly| mechanism is used (false) or
% the selection of child files must be performed manually (true).
% The definitions initialise to false:
%    \begin{macrocode}
\newif\ifchilddoc
\newif\ifchilddocmanual
%    \end{macrocode}

% \macro{\childdocname}
% \macro{\childdocjob}
% The macro |\childdocname| stores the name of the main document
% to be compiled. The macro |\childdocjob| stores the name of
% the document on which the \LaTeX{} compiler was originally invoked.
% The content of |\jobname| cannot be compared
% to filenames specified in the source due to different catcodes.
% The following code rescans |\jobname|, stores the result
% in |\childdocname| and saves a copy in |\childdocjob|:
%    \begin{macrocode}
\edef\childdocname{\scantokens\expandafter{\jobname\noexpand}}
\let\childdocjob\childdocname
%    \end{macrocode}

% \macro{\childdocdisable}
% The macro |\childdocdisable| prevents the main file
% from being processed more than once.
% At this stage, the main document command |\childdocmain|
% is assumed to be called once again where it should do nothing.
% Any subsequent call to it should prevent
% a secondary processing of the main document
% It overwrites the forwarding commands
% |\childdocof| and |\childdocforward|
% with empty macros to prevent further inclusions of the main document:
%    \begin{macrocode}
\newcommand{\childdocdisable}
{
  \renewcommand{\childdocmain}[1]{\renewcommand{\childdocmain}[1]{\endinput}}
  \renewcommand{\childdocof}[1]{}
  \renewcommand{\childdocby}[2][]{}
  \renewcommand{\childdocforward}[2][]{}
  \renewcommand{\childdocdisable}{}
}
%    \end{macrocode}

% \macro{\childdocmain}
% The macro |\childdocmain| is to be called at the top of the main file
% with nothing or the main filename (without extension) as argument.
% First, it breaks loops.
% If the argument is not empty and does not match |\childdocname|
% (which is set by the first inclusion of |childdoc.def|),
% |\ifchilddoc| is set to true, |\includeonly| is applied to the child file
% and |\jobname| is set to the main file
% (for proper handling of |.aux| files):
%    \begin{macrocode}
\newcommand{\childdocmain}[1]
{
  \childdocdisable\childdocmain{}
  \if?#1?\else
    \begingroup
      \def\childdoctmp{#1}
      \ifx\childdoctmp\childdocname
        \def\childdoctmp{}
      \else
        \def\childdoctmp
        {
          \childdoctrue
          \includeonly{\childdocname}
          \def\childdocjob{#1}
          \def\jobname{#1}
        }
      \fi
      \expandafter
    \endgroup
    \childdoctmp
  \fi
}
%    \end{macrocode}

% \macro{\childdocof}
% The command |\childdocof| redirects
% compilation to the main file |#1|.
%    \begin{macrocode}
\newcommand{\childdocof}[1]
{
  \childdocdisable
  \childdoctrue
  \includeonly{\childdocname}
  \def\jobname{#1}
  \def\childdocjob{#1}
  \input{#1}
}
%    \end{macrocode}

% \macro{\childdocby}
% The command |\childdocby| ....
%    \begin{macrocode}
\newcommand{\childdocby}[2][]
{
  \childdocdisable
  \childdoctrue
  \childdocmanualtrue
  \if?#1?\else
    \def\jobname{#2}
  \fi
  \def\childdocjob{#2}
  \input{#2}
  \endinput
}
%    \end{macrocode}

% \macro{\childdocforward}
% The command |\childdocforward| redirects
% compilation to the main file or
% (if the optional argument is given) a child file.
% Parameters are set as if the main file
% or a child file starting with |\childdocof| was compiled.
% Then compilation is handed over to the main file:
%    \begin{macrocode}
\newcommand{\childdocforward}[2][]
{
  \begingroup
    \if?#1?
      \def\childdoctmp
      {
        \def\childdocname{#2}
        \def\childdocjob{#2}
        \def\jobname{#2}
        \input{#2}
        \endinput
      }
    \else
      \def\childdoctmp
      {
        \childdocdisable
        \def\childdocname{#2}
        \childdoctrue
        \includeonly{#2}
        \def\childdocjob{#1}
        \def\jobname{#1}
        \input{#1}
        \endinput
      }
    \fi
    \expandafter
  \endgroup
  \childdoctmp
}
%    \end{macrocode}

% \macro{\childdocforwardprefix}
% The command |\childdocforwardprefix| redirects
% compilation to the main or a child file by means of a pattern.
% The prefix |#1| in the current filename is replaced by |#2|
% and the suffix of the current filename is kept
% (it is assumed that the filename does not contain the substring `|~~~|'
% which is used as a delimiter).
% Compilation is handed over to the new file by |\childdocforward|:
%    \begin{macrocode}
\newcommand{\childdocforwardprefix}[3][]
{
  \begingroup
    \def\childdocextract #2##1~~~{\def\childdoctmp{\childdocforward[#1]{#3##1}}}
    \expandafter\childdocextract\childdocname~~~
    \expandafter
  \endgroup
  \childdoctmp
}
%    \end{macrocode}

% \macro{\childdoc}
% The deprecated macro |\childdoc| is a legacy version of |\childdocmain|:
%    \begin{macrocode}
\newcommand{\childdoc}{\childdocmain}
%    \end{macrocode}

% \macro{\childdocredirect}
% The deprecated macro |\childdocredirect| is a legacy version
% of |\childdocforward| and |\childdocforwardprefix|:
%    \begin{macrocode}
\newcommand{\childdocredirect}[2][]
{
  \begingroup
    \if?#1?
      \def\childdoctmp{\childdocforward{#2}}
    \else
      \def\childdoctmp{\childdocforwardprefix{#1}{#2}}
    \fi
    \expandafter
  \endgroup
  \childdoctmp
}
%    \end{macrocode}

%\iffalse
%</package>
%\fi
%
\endinput
|\\
|\childdocby{|\textit{main}|}|\\
\end{tabular}
\end{center}
%
Both forms have slightly different effects as described above.
The main file is prepared as usual, see \secref{sec:include}.

%%%%%%%%%%%%%%%%%%%%%%%%%%%%%%%%%%%%%%%%%%%%%%%%%%%%%%%%%%%%%%%%%%%%%%%%%%%%%%%%
\subsection{Legacy Detection}
\label{sec:detection}

The directive |\childdocmain| in the main file can detect
whether the complete document or merely a child is to be compiled
even without using the directive |\childdocof|.
This method is deprecated because it is less robust
and there is no compelling reason to use it;
it is merely provided for backward compatibility
and it may be removed in future versions.

If the detection mechanism is to be used,
it is mandatory to correctly specify
the filename of the main file as the argument of |\childdocmain|:
%
\begin{center}
\begin{tabular}{l}
|% \iffalse
%
% childdoc.dtx Copyright (C) 2017-2018 Niklas Beisert
%
% This work may be distributed and/or modified under the
% conditions of the LaTeX Project Public License, either version 1.3
% of this license or (at your option) any later version.
% The latest version of this license is in
%   http://www.latex-project.org/lppl.txt
% and version 1.3 or later is part of all distributions of LaTeX
% version 2005/12/01 or later.
%
% This work has the LPPL maintenance status `maintained'.
%
% The Current Maintainer of this work is Niklas Beisert.
%
% This work consists of the files childdoc.dtx and childdoc.ins
% and the derived files childdoc.def and cdocsamp.tex with
% cdocsch1.tex, cdocsch2.tex, cdocsdrf.tex, cdocsfn1.tex, cdocsfn2.tex.
%
%<package>\ifdefined\childdocmain\endinput\fi
%<package>\ProvidesFile{childdoc.def}[2018/12/30 v2.0 child document driver]
%<samplemain>\ProvidesFile{cdocsamp.tex}[2018/12/30 v2.0 sample for childdoc]
%<*driver>
%\ProvidesFile{childdoc.drv}[2018/12/30 v2.0 childdoc reference manual file]
\PassOptionsToClass{10pt,a4paper}{article}
\documentclass{ltxdoc}

\usepackage[margin=35mm]{geometry}
\usepackage{hyperref}
\usepackage{hyperxmp}
\usepackage[usenames]{color}

\hypersetup{colorlinks=true}
\hypersetup{pdfstartview=FitH}
\hypersetup{pdfpagemode=UseNone}
\hypersetup{pdfsource={}}
\hypersetup{pdflang={en-UK}}
\hypersetup{pdfcopyright={Copyright 2017-2018 Niklas Beisert.
  This work may be distributed and/or modified under the
  conditions of the LaTeX Project Public License, either version 1.3
  of this license or (at your option) any later version.}}
\hypersetup{pdflicenseurl={http://www.latex-project.org/lppl.txt}}
\hypersetup{pdfcontactaddress={ETH Zurich, ITP, HIT K,
  Wolfgang-Pauli-Strasse 27}}
\hypersetup{pdfcontactpostcode={8093}}
\hypersetup{pdfcontactcity={Zurich}}
\hypersetup{pdfcontactcountry={Switzerland}}
\hypersetup{pdfcontactemail={nbeisert@itp.phys.ethz.ch}}
\hypersetup{pdfcontacturl={http://people.phys.ethz.ch/\xmptilde nbeisert/}}

\newcommand{\secref}[1]{\hyperref[#1]{section \ref*{#1}}}

\parskip1ex
\parindent0pt
\let\olditemize\itemize
\def\itemize{\olditemize\parskip0pt}

\begin{document}

\title{The \textsf{childdoc} Package}
\hypersetup{pdftitle={The childdoc Package}}
\author{Niklas Beisert\\[2ex]
  Institut f\"ur Theoretische Physik\\
  Eidgen\"ossische Technische Hochschule Z\"urich\\
  Wolfgang-Pauli-Strasse 27, 8093 Z\"urich, Switzerland\\[1ex]
  \href{mailto:nbeisert@itp.phys.ethz.ch}
  {\texttt{nbeisert@itp.phys.ethz.ch}}}
\hypersetup{pdfauthor={Niklas Beisert}}
\hypersetup{pdfsubject={Manual for the LaTeX2e Package childdoc}}
\date{30 December 2018, \textsf{v2.0}}
\maketitle

\begin{abstract}\noindent
\textsf{childdoc} is a \LaTeXe{} package
that enables the direct compilation
of document sections included by |\include|
to individual files.
\end{abstract}

\begingroup
\parskip0ex
\tableofcontents
\endgroup

%%%%%%%%%%%%%%%%%%%%%%%%%%%%%%%%%%%%%%%%%%%%%%%%%%%%%%%%%%%%%%%%%%%%%%%%%%%%%%%%
%%%%%%%%%%%%%%%%%%%%%%%%%%%%%%%%%%%%%%%%%%%%%%%%%%%%%%%%%%%%%%%%%%%%%%%%%%%%%%%%
\section{Introduction}

\LaTeX{} provides a mechanism to structure a large document (such as a book)
into a main file and several child files (containing the chapters)
using the |\include| command.
This mechanism is beneficial for documents
which span hundreds of pages in order to
make the source file(s) more manageable.
Moreover, compilation can be restricted to
selected child files by means of the |\includeonly| command.
The latter feature can be used to reduce the compilation time while editing
(this was significantly more useful in the earlier days of \LaTeX{})
or to generate a smaller document which is easier to navigate.
Another application of |\includeonly| is to generate
documents consisting of selected parts of the complete document.

However, there are a few drawbacks of the plain |\include| mechanism:
\begin{itemize}
\item
The child files cannot be compiled on their own,
they can only be compiled via the main file.
A naive editing environment
(such as a text editor with an option
to have the current file processed by \LaTeX)
may require one to switch to the main file before compiling;
attempting to compile the child file produces errors.
\item
The main file must be modified (each time)
to adjust the |\includeonly| command
to the present needs. This easily leaves the main file in a messy state.
\item
The generated document will always carry the filename
of the main document. This is inconvenient if
several child files are to be compiled and
to be kept for distribution.
\end{itemize}

The present package provides a simple interface
to make child files individually compilable by \LaTeX{}.
Compiling a child file then has the same effect as compiling
the main file with an |\includeonly| command
to select the appropriate child.
Moreover the generated document will carry the name of the child
rather than the main file.
This resolves all three above issues.

This feature is meant to make the editing of books,
thesis documents and lecture notes somewhat more convenient.
However, the package can also be used efficiently for
composing a series of documents (such as exercise sheets)
which are typically distributed individually.
It then assists the author in generating the individual documents
(potentially in different versions)
as well as a document containing the collected series.
Another application is in developing style files
or other kinds of included material
where compilation of the style file could redirect
to a sample or test file.

%%%%%%%%%%%%%%%%%%%%%%%%%%%%%%%%%%%%%%%%%%%%%%%%%%%%%%%%%%%%%%%%%%%%%%%%%%%%%%%%
%%%%%%%%%%%%%%%%%%%%%%%%%%%%%%%%%%%%%%%%%%%%%%%%%%%%%%%%%%%%%%%%%%%%%%%%%%%%%%%%
\section{Usage}

First of all, the package \textsf{childdoc} is \emph{not} a standard
\LaTeXe{} |.sty| style file! Therefore it needs to be invoked in
a non-standard way.

%%%%%%%%%%%%%%%%%%%%%%%%%%%%%%%%%%%%%%%%%%%%%%%%%%%%%%%%%%%%%%%%%%%%%%%%%%%%%%%%
\subsection{Included Files}
\label{sec:include}

%%%%%%%%%%%%%%%%%%%%%%%%%%%%%%%%%%%%%%%%
\DescribeMacro{\childdocmain}
To use the package, add the commands
\begin{center}
\begin{tabular}{l}
|\input{childdoc.def}|\\
|\childdocmain{}|\\
\end{tabular}
\end{center}
at the very top of the main \LaTeX{} file,
in particular \emph{before} the |\documentclass| statement!
The argument of |\childdocmain| should be left empty
(but it must be present).

%%%%%%%%%%%%%%%%%%%%%%%%%%%%%%%%%%%%%%%%
\DescribeMacro{\childdocof}
Furthermore, add the commands
\begin{center}
\begin{tabular}{l}
|\input{childdoc.def}|\\
|\childdocof{|\textit{main}|}|\\
\end{tabular}
\end{center}
at the top of every child file \textit{child}
which is included by |\include{|\textit{child}|}|
from within the main file
(or at least for those files to be compiled individually).
The argument \textit{main} must be the filename of the main file.

There are a couple of
considerations in setting up the main and child documents:

%%%%%%%%%%%%%%%%%%%%%%%%%%%%%%%%%%%%%%%%
\paragraph{Restrictions.}

Please note the following restrictions:
\begin{itemize}
\item
|\childdocmain| must be called with one argument \textit{main}
to ensure compatibility with earlier version of the package.
It must either be empty (|\childdocmain{}|)
or precisely match the filename of the main file in which it is specified.
See \secref{sec:detection} for further information.
\item
The filename \textit{main} must be specified without the |.tex| extension.
\item
The filename \textit{main} is case sensitive
(even in case-insensitive file systems)
due to internal string comparison.
\item
The argument \textit{main} should be fully expanded, it cannot be a macro.
\item
Subdirectories and special characters should be avoided in filenames.
\item
The command |\childdocmain{|\textit{main}|}| must be followed by a whitespace.
It should not be followed immediately by another command
or by a comment mark `|%|'.
This is because the \TeX{} parser reads the token immediately following
the argument of |\childdocmain| and puts it
at the beginning of every child section;
however, a white\-space is ignored.
\end{itemize}

%%%%%%%%%%%%%%%%%%%%%%%%%%%%%%%%%%%%%%%%
\paragraph{Content of Main File.}

It is advisable to place all content in the child files included by |\include|.
Any output contained in the main file will appear in all child documents
unless suppressed manually;
it cannot be suppressed automatically by the |\includeonly| directive
and thus should normally be avoided.
A method to include some content in the main file
by means of conditional processing is described in \secref{sec:conditional}.

%%%%%%%%%%%%%%%%%%%%%%%%%%%%%%%%%%%%%%%%
\paragraph{Page Numbering.}

When only a part of the document is compiled,
the appropriate numbering of pages
(as well as other status parameters)
is determined from the |.aux| files.
The latter contain information from previous passes.
However this information needs to propagate through
all intermediate child documents.
Therefore the page numbering in child documents may well
be inconsistent until the complete document is compiled at least once.

A useful (if unconventional) way to always ensure a consistent
page numbering is to restart the numbering in each child document
and denote the pages by `\textit{child}|.|\textit{page}'
where \textit{child} represents the chapter/section number of the child file.
This can be achieved by the command
|\numberwithin{page}{|\textit{child}|}|
of the \textsf{amsmath} package
where \textit{child} can be |chapter| or |section|
depending on the chosen structuring.
Alternatively, one can modify the macro |\thepage| appropriately
and reset the counter |page| at the start of each child file.

%%%%%%%%%%%%%%%%%%%%%%%%%%%%%%%%%%%%%%%%%%%%%%%%%%%%%%%%%%%%%%%%%%%%%%%%%%%%%%%%
\subsection{Conditional Processing}
\label{sec:conditional}

The package provides a mechanism to compile different versions
of a document. To customise the versions further some conditional processing
can come in handy to distinguish which version is being compiled.
The package provides two macros to describe the compilation context:

%%%%%%%%%%%%%%%%%%%%%%%%%%%%%%%%%%%%%%%%
\DescribeMacro{\ifchilddoc}
The conditional |\ifchilddoc| distinguishes between the compilation of
child documents and the main document:
%
\begin{center}
|\ifchilddoc |\textit{child-code}| |[|\||else |\textit{main-code}]| \||fi|
\end{center}

%%%%%%%%%%%%%%%%%%%%%%%%%%%%%%%%%%%%%%%%
\DescribeMacro{\childdocname}
\DescribeMacro{\childdocjob}
The macro |\childdocname| contains the filename (without extension)
of the main or child file being processed.
Note that |\childdocjob| will always contain the name of the main file.

%%%%%%%%%%%%%%%%%%%%%%%%%%%%%%%%%%%%%%%%
\paragraph{Title Page.}

Conditional processing can be used to include a title or banner page
in the main document when proper precautions are taken.
Importantly, the code in the main file should ensure that the page counter
(as well as other status parameters which are stored in the |.aux| files)
takes the same value after the conditional processing.
Otherwise the page numbers may take divergent values
depending on which part is compiled.

For example, a title page could be declared by:
%
\begin{center}
\begin{tabular}{l}
|\ifchilddoc\||else|\\
|\addtocounter{page}{-1}|\\
\textit{code for title page}\\
|\newpage|\\
|\||fi|
\end{tabular}
\end{center}
%
A banner page for the child documents can be generated by:
%
\begin{center}
\begin{tabular}{l}
|\ifchilddoc|\\
|\addtocounter{page}{-1}|\\
\textit{code for banner page}\\
|\newpage|\\
|\||fi|
\end{tabular}
\end{center}
%
Here one could write a message such as:
\begin{center}
|This is the part \childdocname{} of \childdocjob{}.|
\end{center}

%%%%%%%%%%%%%%%%%%%%%%%%%%%%%%%%%%%%%%%%%%%%%%%%%%%%%%%%%%%%%%%%%%%%%%%%%%%%%%%%
\subsection{Flags}
\label{sec:flags}

The package makes it easy to generate different versions
of the main or child documents.
To this end compilation flags can be defined
and assigned different default values.
They will be particularly useful in conjunction
with the forwarding mechanism described in \secref{sec:forward}.

For example, it may be useful to have a flag |\version|
which can be set to |draft| or |final|.
The document source will contain some conditional code
depending on the value of |\version|.
Suppose further, the flag should default to |final| for the main file
and to |draft| for child files
which is a natural assignment for editing the document.
This is achieved by placing the following code
in the preamble of the main document
(below the |\childdocmain| directive):
%
\begin{center}
\begin{tabular}{l}
|\ifchilddoc|\\
|\providecommand{\version}{draft}|\\
|\||else|\\
|\providecommand{\version}{final}|\\
|\||fi|
\end{tabular}
\end{center}
%
The definition by |\providecommand| makes sure
that previous definitions are not overwritten.
Further statements |\providecommand{\version}{...}|
can thus be added before the above code to override it.

For the main file, one might add a line
(between |\childdocmain| and the above block)
%
\begin{center}
|%\ifchilddoc\||else\providecommand{\version}{draft}\||fi|
\end{center}
%
which can be uncommented to produce a draft version.
Likewise one can add a line to the very top of a child file
(above the |\childdocof{|\textit{main}|}| directive)
%
\begin{center}
|%\providecommand{\version}{final}|
\end{center}
%
which can be uncommented to produce the final version of this child document.

%%%%%%%%%%%%%%%%%%%%%%%%%%%%%%%%%%%%%%%%%%%%%%%%%%%%%%%%%%%%%%%%%%%%%%%%%%%%%%%%
\subsection{Forwarding}
\label{sec:forward}

Different versions of the main or child documents
using compilation flags as described in \secref{sec:flags}
can be (permanently) stored in different files
for convenient compilation, viewing and distribution.
To this end, the package defines a command
to pass on compilation to a different file:

%%%%%%%%%%%%%%%%%%%%%%%%%%%%%%%%%%%%%%%%
\DescribeMacro{\childdocforward}
The command |\childdocforward| redirects processing to
another source file:
%
\begin{center}
\begin{tabular}{l}
|\input{childdoc.def}|\\
|\childdocforward[|\textit{main}|]{|\textit{dest}|}|\\
\end{tabular}
\end{center}
%
The argument \textit{dest} is the destination file
(without extension).
It should be the main file or one of the child files.
Note that further \textsf{childdoc} directives
such as |\childdocof| and |\childdocforward|
in the indicated file will be processed in this form.
The optional argument \textit{main}
passes on directly to the main file \textit{main}
while pretending to compile the child \textit{dest}.
This form behaves as if \textit{dest}
issues |\childdocof{|\textit{main}|}| right away,
and no further \textsf{childdoc} directives will be processed.

%%%%%%%%%%%%%%%%%%%%%%%%%%%%%%%%%%%%%%%%
\DescribeMacro{\...prefix}
In the alternative form |\childdocforwardprefix|,
%
\begin{center}
\begin{tabular}{l}
|\input{childdoc.def}|\\
|\childdocforwardprefix[|\textit{main}|]{|\textit{prefix}|}{|\textit{dest}|}|
\end{tabular}
\end{center}
%
the destination file is determined by a pattern
depending on the current file:
To make this work, the current file must be called
`{\textit{prefix}\hspace{0.2em}\textit{suffix}}'
with \textit{prefix} matching precisely the argument.
Processing is then passed on to the file
`{\textit{dest}\hspace{0.2em}\textit{suffix}}'.
Surely, the same effect is achieved by
directly specifying the
argument `{\textit{dest}\hspace{0.2em}\textit{suffix}}'
in the first form.
However, that requires to set up a different file
for each child. With the alternative form of the command
all these files can have exactly the same content
which simplifies setting them up and maintaining them.

For example, the following file |draft.tex|
with a compilation flag |\version| as described in \secref{sec:flags}
compiles the main document as a draft:
%
\begin{center}
\begin{tabular}{l}
|\def\version{draft}|\\
|\input{childdoc.def}|\\
|\childdocforward{|\textit{main}|}|
\end{tabular}
\end{center}
%
Likewise, the following files |final|\textit{nn}|.tex|
compile the final version of the child document
|child|\textit{nn}|.tex|:
%
\begin{center}
\begin{tabular}{l}
|\def\version{final}|\\
|\input{childdoc.def}|\\
|\childdocforwardprefix{final}{child}|
\end{tabular}
\end{center}
%

Note that when several versions of a main file and/or of each child file
are to be generated, it may be convenient to set up a |Makefile| or
shell script to automatise the process.

%%%%%%%%%%%%%%%%%%%%%%%%%%%%%%%%%%%%%%%%%%%%%%%%%%%%%%%%%%%%%%%%%%%%%%%%%%%%%%%%
\subsection{Command Line Processing}
\label{sec:commandline}

The effect of redirection files can also be achieved by invoking
the \LaTeX{} compiler with a more elaborate command line.
Most conveniently this should be done as part
of a shell script or a |Makefile|.

When using \textsf{childdoc} in the main file, the following
command lines effectively perform a redirection
(note that depending on the shell being used,
backslashes may have to be doubled: `|\|' $\to$ `|\\|'):
%
\begin{center}
|... -jobname "|\textit{target}|" |\\|"|[\textit{flags}]%
|\input{childdoc.def}\childdocforward[|\textit{main}|]{|\textit{dest}|}"|
\end{center}
%
Here \textit{target} is the name of the output file,
\textit{main} is the name of the main file
and \textit{dest} is the name of the main or child file to be processed
(all filenames without extensions).
The optional argument \textit{main} can be omitted
if \textit{main} matches \textit{dest}.
Optionally, compilation \textit{flags} can be defined via |\def| commands.
This command line makes the \TeX{} engine believe
it is compiling the file \textit{target}
whose content is specified as the latter parameter.
The provided code then forwards the processing to
\textit{main} or \textit{dest} as described in \secref{sec:forward}.

%%%%%%%%%%%%%%%%%%%%%%%%%%%%%%%%%%%%%%%%%%%%%%%%%%%%%%%%%%%%%%%%%%%%%%%%%%%%%%%%
\subsection{Include by Input}
\label{sec:input}

Including child documents by |\include| has some restrictions by design.
Most notably, the content of a child document always occupies
its own set of pages; pages cannot be shared between child documents.
Usually, this behaviour makes perfect sense
because each child document contain an essential part of the document.
However, in some situations it may be desirable to compose
a document from a collection of parts
without having mandatory page breaks between then.
For this case, the package
provides a mechanism to include parts
by |\input| which can also be processed individually.
However, by construction this mechanism
requires manual handling of the content to be output.

%%%%%%%%%%%%%%%%%%%%%%%%%%%%%%%%%%%%%%%%
\DescribeMacro{\ifchilddocmanual}
The main file should be prepared as usual, see \secref{sec:include}.
However, the document body must make a distinction
between processing of an individual part and of the main document, e.g.:
%
\begin{center}
\begin{tabular}{l}
|\ifchilddocmanual|\\
|\input{\childdocname}|\\
|\||else|\\
\textit{document body with }|\input{|\textit{part}|}|\\
|\||fi|
\end{tabular}
\end{center}
%
The conditional |\ifchilddocmanual| is true whenever
a part to be included by |\input| is being compiled,
and the name of the part is stored in |\childdocname|.

%%%%%%%%%%%%%%%%%%%%%%%%%%%%%%%%%%%%%%%%
\DescribeMacro{\childdocby}
Each part to be included by |\input| should start with:
%
\begin{center}
\begin{tabular}{l}
|\input{childdoc.def}|\\
|\childdocby{|\textit{main}|}|\\
\end{tabular}
\end{center}
%
The directive |\childdocby| is similar to |\childdocof|
described in \secref{sec:include},
but the subsequent selection of content must be done manually.
To that end, both |\ifchilddoc| and |\ifchilddocmanual|
will be true upon processing of a part,
and the name of the part is stored in |\childdocname|.
Note that |\jobname| will be set to the filename of the current part
so that each part receives an individual |.aux| file
that does not interfere with the |.aux| file(s) of the main document.
This behaviour can be altered by the alternative form
|\childdocby[*]{|\textit{main}|}| (with a non-empty optional argument)
which uses the |.aux| file of the main document
by setting |\jobname| to \textit{main}.

%%%%%%%%%%%%%%%%%%%%%%%%%%%%%%%%%%%%%%%%%%%%%%%%%%%%%%%%%%%%%%%%%%%%%%%%%%%%%%%%
\subsection{Driver Development}
\label{sec:driver}

The \textsf{childdoc} mechanism can also be use for the development
of definition files such as \LaTeX{} styles or classes.
This case differs from the above setup with multiple parts
included by |\include| in that no |\includeonly| should be invoked.
This can be achieved by starting the include file
(before |\ProvidesPackage|) with:
%
\begin{center}
\begin{tabular}{l}
|\input{childdoc.def}|\\
|\childdocforward{|\textit{main}|}|\\
\end{tabular}
\end{center}
%
or alternatively with:
%
\begin{center}
\begin{tabular}{l}
|\input{childdoc.def}|\\
|\childdocby{|\textit{main}|}|\\
\end{tabular}
\end{center}
%
Both forms have slightly different effects as described above.
The main file is prepared as usual, see \secref{sec:include}.

%%%%%%%%%%%%%%%%%%%%%%%%%%%%%%%%%%%%%%%%%%%%%%%%%%%%%%%%%%%%%%%%%%%%%%%%%%%%%%%%
\subsection{Legacy Detection}
\label{sec:detection}

The directive |\childdocmain| in the main file can detect
whether the complete document or merely a child is to be compiled
even without using the directive |\childdocof|.
This method is deprecated because it is less robust
and there is no compelling reason to use it;
it is merely provided for backward compatibility
and it may be removed in future versions.

If the detection mechanism is to be used,
it is mandatory to correctly specify
the filename of the main file as the argument of |\childdocmain|:
%
\begin{center}
\begin{tabular}{l}
|\input{childdoc.def}|\\
|\childdocmain{|\textit{main}|}|\\
\end{tabular}
\end{center}
%
If |\jobname| does not match the argument \textit{main} of |\childdocmain|,
it is assumed that |\jobname| points to the child file to be compiled.
When using |\childdocmain| with the main file specified as argument,
it suffices to start a child file
with just |\input{|\textit{main}|}|
without loading of the package and using |\childdocof|.
If instead all processing is done
with the appropriate \textsf{childdoc} directives,
the argument of \textit{main} of |\childdocmain| can be empty.

An alternative version of the command line processing described
in \secref{sec:commandline} using the detection mechanism reads:
%
\begin{center}
|... -jobname "|\textit{target}|" "|[\textit{flags}]%
[|\def\jobname{|\textit{dest}|}|]|\input{|\textit{main}|}"|
\end{center}

%%%%%%%%%%%%%%%%%%%%%%%%%%%%%%%%%%%%%%%%%%%%%%%%%%%%%%%%%%%%%%%%%%%%%%%%%%%%%%%%
\subsection{Manual Code}
\label{sec:manual}

In case one cannot be certain whether the definitions file |childdoc.def|
is installed on the target \TeX{} distribution
and one prefers not to ship it,
it is conceivable to paste a few relevant commands into the sources.

To that end, drop all statements |\input{childdoc.def}|
and perform the replacements as outlined below.
Instead of |\childdocmain{|\textit{main}|}| add the following code
to the top of the main file:
%
\begin{center}
\begin{tabular}{l}
|\||ifdefined\childdocname\endinput\||fi\newif\ifchilddoc|\\
|\edef\childdocname{\scantokens\expandafter{\jobname\noexpand}}|\\
|\def\childdocmain{|\textit{main}|}\||ifx\childdocmain\childdocname\||else|\\
|\childdoctrue\includeonly{\childdocname}\let\jobname\childdocmain\||fi|\\
\end{tabular}
\end{center}
%
Instead of |\childdocof{|\textit{main}|}| just include the main file
at the top of each child file:
%
\begin{center}
|\input{|\textit{main}|}|
\end{center}
%
A simple redirection |\childdocforward{|\textit{dest}|}| is achieved by:
%
\begin{center}
|\def\jobname{|\textit{dest}|}\input{\jobname}|
\end{center}
%
The redirection with prefix
|\childdocforwardprefix[|\textit{prefix}|]{|\textit{dest}|}|
is accomplished by:
%
\begin{center}
\begin{tabular}{l}
|{\edef\jobname{\scantokens\expandafter{\jobname\noexpand}}|\\
|\def\redirectjob |\textit{prefix}|#1~~~{\gdef\jobname{|\textit{dest}|#1}}|\\
|\expandafter\redirectjob\jobname~~~}\input{\jobname}|
\end{tabular}
\end{center}

In an alternative approach,
child documents can be compiled by a specific command line
without additional code or specific definitions:
%
\begin{center}
|... -jobname "|\textit{target}|" "|[\textit{flags}]%
|\includeonly{|\textit{dest}|}\input{|\textit{main}|}"|
\end{center}
%

%%%%%%%%%%%%%%%%%%%%%%%%%%%%%%%%%%%%%%%%%%%%%%%%%%%%%%%%%%%%%%%%%%%%%%%%%%%%%%%%
%%%%%%%%%%%%%%%%%%%%%%%%%%%%%%%%%%%%%%%%%%%%%%%%%%%%%%%%%%%%%%%%%%%%%%%%%%%%%%%%
\section{Information}

%%%%%%%%%%%%%%%%%%%%%%%%%%%%%%%%%%%%%%%%%%%%%%%%%%%%%%%%%%%%%%%%%%%%%%%%%%%%%%%%
\subsection{Copyright}

Copyright \copyright{} 2017--2018 Niklas Beisert

This work may be distributed and/or modified under the
conditions of the \LaTeX{} Project Public License, either version 1.3
of this license or (at your option) any later version.
The latest version of this license is in
  \url{http://www.latex-project.org/lppl.txt}
and version 1.3 or later is part of all distributions of \LaTeX{}
version 2005/12/01 or later.

This work has the LPPL maintenance status `maintained'.

The Current Maintainer of this work is Niklas Beisert.

This work consists of the files |README.txt|, |childdoc.ins| and |childdoc.dtx|
as well as the derived files |childdoc.def|, |cdocsamp.tex|
with |cdocsch1.tex|, |cdocsch2.tex|, |cdocspt3.tex|, |cdocspt4.tex|,
|cdocsdrf.tex|, |cdocsfn1.tex|, |cdocsfn2.tex|
as well as |childdoc.pdf|.

%%%%%%%%%%%%%%%%%%%%%%%%%%%%%%%%%%%%%%%%%%%%%%%%%%%%%%%%%%%%%%%%%%%%%%%%%%%%%%%%
\subsection{Files and Installation}

The package consists of the files:
%
\begin{center}
\begin{tabular}{ll}
    |README.txt|   & readme file \\
    |childdoc.ins| & installation file \\
    |childdoc.dtx| & source file \\
    |childdoc.def| & definition file \\
    |cdocsamp.tex| & sample main file \\
    |cdocsch1.tex| & sample include file \\
    |cdocsch2.tex| & sample include file \\
    |cdocspt3.tex| & sample part file \\
    |cdocspt4.tex| & sample part file \\
    |cdocsdrf.tex| & sample redirection file \\
    |cdocsfn1.tex| & sample redirection file \\
    |cdocsfn2.tex| & sample redirection file \\
    |childdoc.pdf| & manual
\end{tabular}
\end{center}
%
The distribution consists of the files
|README.txt|, |childdoc.ins| and |childdoc.dtx|.
%
\begin{itemize}
\item
Run (pdf)\LaTeX{} on |childdoc.dtx|
to compile the manual |childdoc.pdf| (this file).
\item
Run \LaTeX{} on |childdoc.ins| to create the definitions file |childdoc.def|
and the sample |cdocsamp.tex| with include files
|cdocsch1.tex|, |cdocsch2.tex|, |cdocspt3.tex|, |cdocspt4.tex|,
|cdocsdrf.tex|, |cdocsfn1.tex|, |cdocsfn2.tex|.
Then copy the file |childdoc.def| to an appropriate directory of your \LaTeX{}
distribution, e.g.\ \textit{texmf-root}|/tex/latex/childdoc|.
\end{itemize}

%%%%%%%%%%%%%%%%%%%%%%%%%%%%%%%%%%%%%%%%%%%%%%%%%%%%%%%%%%%%%%%%%%%%%%%%%%%%%%%%
\subsection{Related CTAN Packages}

There are several other packages which offer a similar functionality:
%
\begin{itemize}
\item
The packages
\href{http://ctan.org/pkg/docmute}{\textsf{docmute}},
\href{http://ctan.org/pkg/includex}{\textsf{includex}} and
\href{http://ctan.org/pkg/standalone}{\textsf{standalone}}
provide commands to include only the document body of
a child file thus allowing both files to be compiled individually.
\item
The packages \href{http://ctan.org/pkg/subdocs}{\textsf{subdocs}}
and \href{http://ctan.org/pkg/subfiles}{\textsf{subfiles}}
provide structures in which the main and child documents can be
encapsulated and allowing them to be compiled individually.
The inclusion mechanism is different from the conventional |\include|.
\item
The package \href{http://ctan.org/pkg/combine}{\textsf{combine}}
is an elaborate solution to combine several documents into one.
\end{itemize}
%
See also the CTAN topic \href{http://ctan.org/topic/subdocs}{\textsf{subdocs}}
for further related packages.
The present package differs from the above solutions in that
a document structure constructed with the conventional |\include| mechanism
just needs two extra commands at the top of every file
such that all constituent files can be compiled individually.

%%%%%%%%%%%%%%%%%%%%%%%%%%%%%%%%%%%%%%%%%%%%%%%%%%%%%%%%%%%%%%%%%%%%%%%%%%%%%%%%
%\subsection{Feature Suggestions}
%
%The following is a list of features which may be useful for future
%versions of this package:
%%
%\begin{itemize}
%\item
%\ldots
%\end{itemize}

%%%%%%%%%%%%%%%%%%%%%%%%%%%%%%%%%%%%%%%%%%%%%%%%%%%%%%%%%%%%%%%%%%%%%%%%%%%%%%%%
\subsection{Revision History}

%%%%%%%%%%%%%%%%%%%%%%%%%%%%%%%%%%%%%%%%
\paragraph{v2.0:} 2018/12/30

\begin{itemize}
\item
immediate forward processing
\item
added |\childdocby| mechanism
\item
manual restructured
\end{itemize}

%%%%%%%%%%%%%%%%%%%%%%%%%%%%%%%%%%%%%%%%
\paragraph{v1.6:} 2018/01/17

\begin{itemize}
\item
application for development of include files
\item
corrections to manual
\end{itemize}

%%%%%%%%%%%%%%%%%%%%%%%%%%%%%%%%%%%%%%%%
\paragraph{v1.5:} 2017/05/21

\begin{itemize}
\item
more complete structuring introduced
\item
|\childdocof| introduced
\item
|\childdoc| renamed to |\childdocmain|
\item
|\childredirect| renamed to |\childdocforward| and |\childdocforwardprefix|
and functionality expanded
\end{itemize}

%%%%%%%%%%%%%%%%%%%%%%%%%%%%%%%%%%%%%%%%
\paragraph{v1.0:} 2017/04/27

\begin{itemize}
\item
manual and install package
\item
first version published on CTAN
\end{itemize}

%%%%%%%%%%%%%%%%%%%%%%%%%%%%%%%%%%%%%%%%
\paragraph{v0.6:} 2017/04/26

\begin{itemize}
\item
redirection mechanism added
\end{itemize}

%%%%%%%%%%%%%%%%%%%%%%%%%%%%%%%%%%%%%%%%
\paragraph{v0.5:} 2017/04/26

\begin{itemize}
\item
functionality in definition file
\end{itemize}


%%%%%%%%%%%%%%%%%%%%%%%%%%%%%%%%%%%%%%%%%%%%%%%%%%%%%%%%%%%%%%%%%%%%%%%%%%%%%%%%
%%%%%%%%%%%%%%%%%%%%%%%%%%%%%%%%%%%%%%%%%%%%%%%%%%%%%%%%%%%%%%%%%%%%%%%%%%%%%%%%
%%%%%%%%%%%%%%%%%%%%%%%%%%%%%%%%%%%%%%%%%%%%%%%%%%%%%%%%%%%%%%%%%%%%%%%%%%%%%%%%
\appendix

\settowidth\MacroIndent{\rmfamily\scriptsize 000\ }

 \DocInput{childdoc.dtx}

\end{document}
%</driver>
% \fi
%
% %%%%%%%%%%%%%%%%%%%%%%%%%%%%%%%%%%%%%%%%%%%%%%%%%%%%%%%%%%%%%%%%%%%%%%%%%%%%%%
% %%%%%%%%%%%%%%%%%%%%%%%%%%%%%%%%%%%%%%%%%%%%%%%%%%%%%%%%%%%%%%%%%%%%%%%%%%%%%%
% \section{Sample}
%\iffalse
%<*samplemain>
%\fi
%
% The following presents a sample document
% with two chapters, two parts, a title page,
% a compile flag as well as three forwarding files to set the flag.
% It consists of eight |.tex| files:
% \begin{center}
% \begin{tabular}{ll}
% |cdocsamp.tex|&main file\\
% |cdocsch1.tex|&include file for chapter 1\\
% |cdocsch2.tex|&include file for chapter 2\\
% |cdocspt3.tex|&include file for part 3\\
% |cdocspt4.tex|&include file for part 4\\
% |cdocsdrf.tex|&forwarding file for main file in draft mode\\
% |cdocsfi1.tex|&forwarding file for final version of chapter 1\\
% |cdocsfi2.tex|&forwarding file for final version of chapter 2\\
% \end{tabular}
% \end{center}
% Each of the eight files can be compiled directly by the \LaTeX{} compiler.
%
% %%%%%%%%%%%%%%%%%%%%%%%%%%%%%%%%%%%%%%
% \paragraph{Main File.}
%
% The main file is called |cdocsamp.tex|.
%
% Load the \textsf{childdoc} definitions and
% declare the filename for the main document:
%    \begin{macrocode}
\input{childdoc.def}
\childdocmain{}
%    \end{macrocode}

% Optional override for |\version| flag:
%    \begin{macrocode}
%%\ifchilddoc\else\providecommand{\version}{draft}\fi
%    \end{macrocode}

% Define the default values for the |\version| flag
% (|final| for the main file and |draft| for childs):
%    \begin{macrocode}
\ifchilddoc
\providecommand{\version}{draft}
\else
\providecommand{\version}{final}
\fi
%    \end{macrocode}

% Load the standard document class:
%    \begin{macrocode}
\documentclass[12pt]{article}
%    \end{macrocode}

% Start the document body:
%    \begin{macrocode}
\begin{document}
%    \end{macrocode}

% Declare a title page.
% Print title, part of document being processed and version flag:
%    \begin{macrocode}
\addtocounter{page}{-1}
\begin{center}
{\LARGE\bfseries{}childdoc example\par}
\vspace{1cm}
\ifchilddoc
\ifchilddocmanual part\else chapter\fi:
`\childdocname' of `\childdocjob'\par
\else
main document: `\childdocjob'\par
\fi
version: \version\par
\end{center}
\newpage
%    \end{macrocode}

% Manually include selected file,
% otherwise process as usual:
%    \begin{macrocode}
\ifchilddocmanual
\section*{part `\childdocname'}
\input{\childdocname}
\else
%    \end{macrocode}

% Include the two chapters:
%    \begin{macrocode}
\include{cdocsch1}
\include{cdocsch2}
%    \end{macrocode}

% Include the two parts unless only chapters should be displayed:
%    \begin{macrocode}
\ifchilddoc\else
\section{part three}
\input{cdocspt3}
\section{part four}
\input{cdocspt4}
\fi
%    \end{macrocode}

% Process as usual until here:
%    \begin{macrocode}
\fi
%    \end{macrocode}

% End of document body:
%    \begin{macrocode}
\end{document}
%    \end{macrocode}
%\iffalse
%</samplemain>
%\fi
%
% %%%%%%%%%%%%%%%%%%%%%%%%%%%%%%%%%%%%%%
% \paragraph{Chapter Include Files.}
%
% The include files are called |cdocsch1.tex| and |cdocsch2.tex|.
%
%\iffalse
%<*samplechap1|samplechap2>
%\fi

% Optional override for |\version| flag:
%    \begin{macrocode}
%%\providecommand{\version}{final}
%    \end{macrocode}

% Include the main document:
%    \begin{macrocode}
\input{childdoc.def}
\childdocof{cdocsamp}
%    \end{macrocode}

%\iffalse
%</samplechap1|samplechap2>
%\fi
%
%\iffalse
%<*samplechap1>
%\fi
% Some text for chapter 1:
%    \begin{macrocode}
\section{one}
some text in chapter one
%    \end{macrocode}

%\iffalse
%</samplechap1>
%\fi
% Some text for chapter 2:
%\iffalse
%<*samplechap2>
%\fi
%    \begin{macrocode}
\section{two}
more text in chapter two
%    \end{macrocode}

%\iffalse
%</samplechap2>
%\fi
%
% %%%%%%%%%%%%%%%%%%%%%%%%%%%%%%%%%%%%%%
% \paragraph{Part Include Files.}
%
% The include files are called |cdocspt3.tex| and |cdocspt4.tex|.
%
%\iffalse
%<*samplepart3|samplepart4>
%\fi

% Optional override for |\version| flag:
%    \begin{macrocode}
%%\providecommand{\version}{final}
%    \end{macrocode}

% Include the main document:
%    \begin{macrocode}
\input{childdoc.def}
\childdocby{cdocsamp}
%    \end{macrocode}

%\iffalse
%</samplepart3|samplepart4>
%\fi
%
%\iffalse
%<*samplepart3>
%\fi
% Some text for part 3:
%    \begin{macrocode}
some text in part three
%    \end{macrocode}

%\iffalse
%</samplepart3>
%\fi
% Some text for part 4:
%\iffalse
%<*samplepart4>
%\fi
%    \begin{macrocode}
more text in part four
%    \end{macrocode}

%\iffalse
%</samplepart4>
%\fi
%
% %%%%%%%%%%%%%%%%%%%%%%%%%%%%%%%%%%%%%%
% \paragraph{Forwarding for a Complete Draft.}
%
% The following forwarding file |cdocsdrf.tex|
% compiles the main document in draft mode:
%\iffalse
%<*sampledraft>
%\fi
%    \begin{macrocode}
\def\version{draft}
\input{childdoc.def}
\childdocforward{cdocsamp}
%    \end{macrocode}

%\iffalse
%</sampledraft>
%\fi
%
% %%%%%%%%%%%%%%%%%%%%%%%%%%%%%%%%%%%%%%
% \paragraph{Forwarding for Final Version of the Chapters.}
%
% The following forwarding files |cdocsfn1.tex| and |cdocsfn2.tex|
% (with identical content)
% compile the final versions of the child documents
% |cdocsch1.tex| and |cdocsch2.tex|, respectively:
%\iffalse
%<*samplefinal>
%\fi
%    \begin{macrocode}
\def\version{final}
\input{childdoc.def}
\childdocforwardprefix[cdocsamp]{cdocsfn}{cdocsch}
%    \end{macrocode}

%\iffalse
%</samplefinal>
%\fi
%
% %%%%%%%%%%%%%%%%%%%%%%%%%%%%%%%%%%%%%%
% \paragraph{Command Line Processing.}
%
% The following three command lines generate the output files
% |cdocscld|, |cdocscl1| and |cdocscl2|
% which should be identical to
% |cdocsdrf|, |cdocsch1| and |cdocsfn2|, respectively:
% \begin{center}
% \begin{tabular}{l}
% |latex -jobname cdocscld \|\\
% |  "\def\version{draft}\input{childdoc.def}\childdocforward{cdocsamp}"|\\
% |latex -jobname cdocscl1 \|\\
% |  "\input{childdoc.def}\childdocforward[cdocsamp]{cdocsch1}"|\\
% |latex -jobname cdocscl2 \|\\
% |  "\def\version{final}\input{childdoc.def}\childdocforward{cdocsch2}"|
% \end{tabular}
% \end{center}
% Note that the trailing backslash on each first line
% merely continues the input to the second line
% (for convenient cut ant paste).
% Furthermore, the command |latex| can be replaced by any
% of its alternative versions such as |pdflatex|.
%
% %%%%%%%%%%%%%%%%%%%%%%%%%%%%%%%%%%%%%%%%%%%%%%%%%%%%%%%%%%%%%%%%%%%%%%%%%%%%%%
% %%%%%%%%%%%%%%%%%%%%%%%%%%%%%%%%%%%%%%%%%%%%%%%%%%%%%%%%%%%%%%%%%%%%%%%%%%%%%%
% \section{Implementation}
%\iffalse
%<*package>
%\fi
%
% This section describes the definitions file |childdoc.def|.

% The definitions cannot be loaded using |\usepackage| or |\RequirePackage|
% which has a mechanism to prevent loading a style file more than once.
% When loading the definitions by means of |\input|
% multiple instances have to be prevented manually:
%\iffalse
%This code needs to be before the `\ProvidesFile' directive
%which is defined at the beginning of this file.
%Therefore it is also placed there and commented out here.
%</package>
%<*discard>
%\fi
%    \begin{macrocode}
\ifdefined\childdocmain\endinput\fi
%    \end{macrocode}
%\iffalse
%</discard>
%<*package>
%\fi
%
% \macro{\ifchilddoc}
% \macro{\ifchilddocmanual}
% The conditional |\ifchilddoc| tells whether a
% child (true) or main (false) document is being compiled.
% The conditional |\ifchilddocmanual| tells whether
% the |\includeonly| mechanism is used (false) or
% the selection of child files must be performed manually (true).
% The definitions initialise to false:
%    \begin{macrocode}
\newif\ifchilddoc
\newif\ifchilddocmanual
%    \end{macrocode}

% \macro{\childdocname}
% \macro{\childdocjob}
% The macro |\childdocname| stores the name of the main document
% to be compiled. The macro |\childdocjob| stores the name of
% the document on which the \LaTeX{} compiler was originally invoked.
% The content of |\jobname| cannot be compared
% to filenames specified in the source due to different catcodes.
% The following code rescans |\jobname|, stores the result
% in |\childdocname| and saves a copy in |\childdocjob|:
%    \begin{macrocode}
\edef\childdocname{\scantokens\expandafter{\jobname\noexpand}}
\let\childdocjob\childdocname
%    \end{macrocode}

% \macro{\childdocdisable}
% The macro |\childdocdisable| prevents the main file
% from being processed more than once.
% At this stage, the main document command |\childdocmain|
% is assumed to be called once again where it should do nothing.
% Any subsequent call to it should prevent
% a secondary processing of the main document
% It overwrites the forwarding commands
% |\childdocof| and |\childdocforward|
% with empty macros to prevent further inclusions of the main document:
%    \begin{macrocode}
\newcommand{\childdocdisable}
{
  \renewcommand{\childdocmain}[1]{\renewcommand{\childdocmain}[1]{\endinput}}
  \renewcommand{\childdocof}[1]{}
  \renewcommand{\childdocby}[2][]{}
  \renewcommand{\childdocforward}[2][]{}
  \renewcommand{\childdocdisable}{}
}
%    \end{macrocode}

% \macro{\childdocmain}
% The macro |\childdocmain| is to be called at the top of the main file
% with nothing or the main filename (without extension) as argument.
% First, it breaks loops.
% If the argument is not empty and does not match |\childdocname|
% (which is set by the first inclusion of |childdoc.def|),
% |\ifchilddoc| is set to true, |\includeonly| is applied to the child file
% and |\jobname| is set to the main file
% (for proper handling of |.aux| files):
%    \begin{macrocode}
\newcommand{\childdocmain}[1]
{
  \childdocdisable\childdocmain{}
  \if?#1?\else
    \begingroup
      \def\childdoctmp{#1}
      \ifx\childdoctmp\childdocname
        \def\childdoctmp{}
      \else
        \def\childdoctmp
        {
          \childdoctrue
          \includeonly{\childdocname}
          \def\childdocjob{#1}
          \def\jobname{#1}
        }
      \fi
      \expandafter
    \endgroup
    \childdoctmp
  \fi
}
%    \end{macrocode}

% \macro{\childdocof}
% The command |\childdocof| redirects
% compilation to the main file |#1|.
%    \begin{macrocode}
\newcommand{\childdocof}[1]
{
  \childdocdisable
  \childdoctrue
  \includeonly{\childdocname}
  \def\jobname{#1}
  \def\childdocjob{#1}
  \input{#1}
}
%    \end{macrocode}

% \macro{\childdocby}
% The command |\childdocby| ....
%    \begin{macrocode}
\newcommand{\childdocby}[2][]
{
  \childdocdisable
  \childdoctrue
  \childdocmanualtrue
  \if?#1?\else
    \def\jobname{#2}
  \fi
  \def\childdocjob{#2}
  \input{#2}
  \endinput
}
%    \end{macrocode}

% \macro{\childdocforward}
% The command |\childdocforward| redirects
% compilation to the main file or
% (if the optional argument is given) a child file.
% Parameters are set as if the main file
% or a child file starting with |\childdocof| was compiled.
% Then compilation is handed over to the main file:
%    \begin{macrocode}
\newcommand{\childdocforward}[2][]
{
  \begingroup
    \if?#1?
      \def\childdoctmp
      {
        \def\childdocname{#2}
        \def\childdocjob{#2}
        \def\jobname{#2}
        \input{#2}
        \endinput
      }
    \else
      \def\childdoctmp
      {
        \childdocdisable
        \def\childdocname{#2}
        \childdoctrue
        \includeonly{#2}
        \def\childdocjob{#1}
        \def\jobname{#1}
        \input{#1}
        \endinput
      }
    \fi
    \expandafter
  \endgroup
  \childdoctmp
}
%    \end{macrocode}

% \macro{\childdocforwardprefix}
% The command |\childdocforwardprefix| redirects
% compilation to the main or a child file by means of a pattern.
% The prefix |#1| in the current filename is replaced by |#2|
% and the suffix of the current filename is kept
% (it is assumed that the filename does not contain the substring `|~~~|'
% which is used as a delimiter).
% Compilation is handed over to the new file by |\childdocforward|:
%    \begin{macrocode}
\newcommand{\childdocforwardprefix}[3][]
{
  \begingroup
    \def\childdocextract #2##1~~~{\def\childdoctmp{\childdocforward[#1]{#3##1}}}
    \expandafter\childdocextract\childdocname~~~
    \expandafter
  \endgroup
  \childdoctmp
}
%    \end{macrocode}

% \macro{\childdoc}
% The deprecated macro |\childdoc| is a legacy version of |\childdocmain|:
%    \begin{macrocode}
\newcommand{\childdoc}{\childdocmain}
%    \end{macrocode}

% \macro{\childdocredirect}
% The deprecated macro |\childdocredirect| is a legacy version
% of |\childdocforward| and |\childdocforwardprefix|:
%    \begin{macrocode}
\newcommand{\childdocredirect}[2][]
{
  \begingroup
    \if?#1?
      \def\childdoctmp{\childdocforward{#2}}
    \else
      \def\childdoctmp{\childdocforwardprefix{#1}{#2}}
    \fi
    \expandafter
  \endgroup
  \childdoctmp
}
%    \end{macrocode}

%\iffalse
%</package>
%\fi
%
\endinput
|\\
|\childdocmain{|\textit{main}|}|\\
\end{tabular}
\end{center}
%
If |\jobname| does not match the argument \textit{main} of |\childdocmain|,
it is assumed that |\jobname| points to the child file to be compiled.
When using |\childdocmain| with the main file specified as argument,
it suffices to start a child file
with just |\input{|\textit{main}|}|
without loading of the package and using |\childdocof|.
If instead all processing is done
with the appropriate \textsf{childdoc} directives,
the argument of \textit{main} of |\childdocmain| can be empty.

An alternative version of the command line processing described
in \secref{sec:commandline} using the detection mechanism reads:
%
\begin{center}
|... -jobname "|\textit{target}|" "|[\textit{flags}]%
[|\def\jobname{|\textit{dest}|}|]|\input{|\textit{main}|}"|
\end{center}

%%%%%%%%%%%%%%%%%%%%%%%%%%%%%%%%%%%%%%%%%%%%%%%%%%%%%%%%%%%%%%%%%%%%%%%%%%%%%%%%
\subsection{Manual Code}
\label{sec:manual}

In case one cannot be certain whether the definitions file |childdoc.def|
is installed on the target \TeX{} distribution
and one prefers not to ship it,
it is conceivable to paste a few relevant commands into the sources.

To that end, drop all statements |% \iffalse
%
% childdoc.dtx Copyright (C) 2017-2018 Niklas Beisert
%
% This work may be distributed and/or modified under the
% conditions of the LaTeX Project Public License, either version 1.3
% of this license or (at your option) any later version.
% The latest version of this license is in
%   http://www.latex-project.org/lppl.txt
% and version 1.3 or later is part of all distributions of LaTeX
% version 2005/12/01 or later.
%
% This work has the LPPL maintenance status `maintained'.
%
% The Current Maintainer of this work is Niklas Beisert.
%
% This work consists of the files childdoc.dtx and childdoc.ins
% and the derived files childdoc.def and cdocsamp.tex with
% cdocsch1.tex, cdocsch2.tex, cdocsdrf.tex, cdocsfn1.tex, cdocsfn2.tex.
%
%<package>\ifdefined\childdocmain\endinput\fi
%<package>\ProvidesFile{childdoc.def}[2018/12/30 v2.0 child document driver]
%<samplemain>\ProvidesFile{cdocsamp.tex}[2018/12/30 v2.0 sample for childdoc]
%<*driver>
%\ProvidesFile{childdoc.drv}[2018/12/30 v2.0 childdoc reference manual file]
\PassOptionsToClass{10pt,a4paper}{article}
\documentclass{ltxdoc}

\usepackage[margin=35mm]{geometry}
\usepackage{hyperref}
\usepackage{hyperxmp}
\usepackage[usenames]{color}

\hypersetup{colorlinks=true}
\hypersetup{pdfstartview=FitH}
\hypersetup{pdfpagemode=UseNone}
\hypersetup{pdfsource={}}
\hypersetup{pdflang={en-UK}}
\hypersetup{pdfcopyright={Copyright 2017-2018 Niklas Beisert.
  This work may be distributed and/or modified under the
  conditions of the LaTeX Project Public License, either version 1.3
  of this license or (at your option) any later version.}}
\hypersetup{pdflicenseurl={http://www.latex-project.org/lppl.txt}}
\hypersetup{pdfcontactaddress={ETH Zurich, ITP, HIT K,
  Wolfgang-Pauli-Strasse 27}}
\hypersetup{pdfcontactpostcode={8093}}
\hypersetup{pdfcontactcity={Zurich}}
\hypersetup{pdfcontactcountry={Switzerland}}
\hypersetup{pdfcontactemail={nbeisert@itp.phys.ethz.ch}}
\hypersetup{pdfcontacturl={http://people.phys.ethz.ch/\xmptilde nbeisert/}}

\newcommand{\secref}[1]{\hyperref[#1]{section \ref*{#1}}}

\parskip1ex
\parindent0pt
\let\olditemize\itemize
\def\itemize{\olditemize\parskip0pt}

\begin{document}

\title{The \textsf{childdoc} Package}
\hypersetup{pdftitle={The childdoc Package}}
\author{Niklas Beisert\\[2ex]
  Institut f\"ur Theoretische Physik\\
  Eidgen\"ossische Technische Hochschule Z\"urich\\
  Wolfgang-Pauli-Strasse 27, 8093 Z\"urich, Switzerland\\[1ex]
  \href{mailto:nbeisert@itp.phys.ethz.ch}
  {\texttt{nbeisert@itp.phys.ethz.ch}}}
\hypersetup{pdfauthor={Niklas Beisert}}
\hypersetup{pdfsubject={Manual for the LaTeX2e Package childdoc}}
\date{30 December 2018, \textsf{v2.0}}
\maketitle

\begin{abstract}\noindent
\textsf{childdoc} is a \LaTeXe{} package
that enables the direct compilation
of document sections included by |\include|
to individual files.
\end{abstract}

\begingroup
\parskip0ex
\tableofcontents
\endgroup

%%%%%%%%%%%%%%%%%%%%%%%%%%%%%%%%%%%%%%%%%%%%%%%%%%%%%%%%%%%%%%%%%%%%%%%%%%%%%%%%
%%%%%%%%%%%%%%%%%%%%%%%%%%%%%%%%%%%%%%%%%%%%%%%%%%%%%%%%%%%%%%%%%%%%%%%%%%%%%%%%
\section{Introduction}

\LaTeX{} provides a mechanism to structure a large document (such as a book)
into a main file and several child files (containing the chapters)
using the |\include| command.
This mechanism is beneficial for documents
which span hundreds of pages in order to
make the source file(s) more manageable.
Moreover, compilation can be restricted to
selected child files by means of the |\includeonly| command.
The latter feature can be used to reduce the compilation time while editing
(this was significantly more useful in the earlier days of \LaTeX{})
or to generate a smaller document which is easier to navigate.
Another application of |\includeonly| is to generate
documents consisting of selected parts of the complete document.

However, there are a few drawbacks of the plain |\include| mechanism:
\begin{itemize}
\item
The child files cannot be compiled on their own,
they can only be compiled via the main file.
A naive editing environment
(such as a text editor with an option
to have the current file processed by \LaTeX)
may require one to switch to the main file before compiling;
attempting to compile the child file produces errors.
\item
The main file must be modified (each time)
to adjust the |\includeonly| command
to the present needs. This easily leaves the main file in a messy state.
\item
The generated document will always carry the filename
of the main document. This is inconvenient if
several child files are to be compiled and
to be kept for distribution.
\end{itemize}

The present package provides a simple interface
to make child files individually compilable by \LaTeX{}.
Compiling a child file then has the same effect as compiling
the main file with an |\includeonly| command
to select the appropriate child.
Moreover the generated document will carry the name of the child
rather than the main file.
This resolves all three above issues.

This feature is meant to make the editing of books,
thesis documents and lecture notes somewhat more convenient.
However, the package can also be used efficiently for
composing a series of documents (such as exercise sheets)
which are typically distributed individually.
It then assists the author in generating the individual documents
(potentially in different versions)
as well as a document containing the collected series.
Another application is in developing style files
or other kinds of included material
where compilation of the style file could redirect
to a sample or test file.

%%%%%%%%%%%%%%%%%%%%%%%%%%%%%%%%%%%%%%%%%%%%%%%%%%%%%%%%%%%%%%%%%%%%%%%%%%%%%%%%
%%%%%%%%%%%%%%%%%%%%%%%%%%%%%%%%%%%%%%%%%%%%%%%%%%%%%%%%%%%%%%%%%%%%%%%%%%%%%%%%
\section{Usage}

First of all, the package \textsf{childdoc} is \emph{not} a standard
\LaTeXe{} |.sty| style file! Therefore it needs to be invoked in
a non-standard way.

%%%%%%%%%%%%%%%%%%%%%%%%%%%%%%%%%%%%%%%%%%%%%%%%%%%%%%%%%%%%%%%%%%%%%%%%%%%%%%%%
\subsection{Included Files}
\label{sec:include}

%%%%%%%%%%%%%%%%%%%%%%%%%%%%%%%%%%%%%%%%
\DescribeMacro{\childdocmain}
To use the package, add the commands
\begin{center}
\begin{tabular}{l}
|\input{childdoc.def}|\\
|\childdocmain{}|\\
\end{tabular}
\end{center}
at the very top of the main \LaTeX{} file,
in particular \emph{before} the |\documentclass| statement!
The argument of |\childdocmain| should be left empty
(but it must be present).

%%%%%%%%%%%%%%%%%%%%%%%%%%%%%%%%%%%%%%%%
\DescribeMacro{\childdocof}
Furthermore, add the commands
\begin{center}
\begin{tabular}{l}
|\input{childdoc.def}|\\
|\childdocof{|\textit{main}|}|\\
\end{tabular}
\end{center}
at the top of every child file \textit{child}
which is included by |\include{|\textit{child}|}|
from within the main file
(or at least for those files to be compiled individually).
The argument \textit{main} must be the filename of the main file.

There are a couple of
considerations in setting up the main and child documents:

%%%%%%%%%%%%%%%%%%%%%%%%%%%%%%%%%%%%%%%%
\paragraph{Restrictions.}

Please note the following restrictions:
\begin{itemize}
\item
|\childdocmain| must be called with one argument \textit{main}
to ensure compatibility with earlier version of the package.
It must either be empty (|\childdocmain{}|)
or precisely match the filename of the main file in which it is specified.
See \secref{sec:detection} for further information.
\item
The filename \textit{main} must be specified without the |.tex| extension.
\item
The filename \textit{main} is case sensitive
(even in case-insensitive file systems)
due to internal string comparison.
\item
The argument \textit{main} should be fully expanded, it cannot be a macro.
\item
Subdirectories and special characters should be avoided in filenames.
\item
The command |\childdocmain{|\textit{main}|}| must be followed by a whitespace.
It should not be followed immediately by another command
or by a comment mark `|%|'.
This is because the \TeX{} parser reads the token immediately following
the argument of |\childdocmain| and puts it
at the beginning of every child section;
however, a white\-space is ignored.
\end{itemize}

%%%%%%%%%%%%%%%%%%%%%%%%%%%%%%%%%%%%%%%%
\paragraph{Content of Main File.}

It is advisable to place all content in the child files included by |\include|.
Any output contained in the main file will appear in all child documents
unless suppressed manually;
it cannot be suppressed automatically by the |\includeonly| directive
and thus should normally be avoided.
A method to include some content in the main file
by means of conditional processing is described in \secref{sec:conditional}.

%%%%%%%%%%%%%%%%%%%%%%%%%%%%%%%%%%%%%%%%
\paragraph{Page Numbering.}

When only a part of the document is compiled,
the appropriate numbering of pages
(as well as other status parameters)
is determined from the |.aux| files.
The latter contain information from previous passes.
However this information needs to propagate through
all intermediate child documents.
Therefore the page numbering in child documents may well
be inconsistent until the complete document is compiled at least once.

A useful (if unconventional) way to always ensure a consistent
page numbering is to restart the numbering in each child document
and denote the pages by `\textit{child}|.|\textit{page}'
where \textit{child} represents the chapter/section number of the child file.
This can be achieved by the command
|\numberwithin{page}{|\textit{child}|}|
of the \textsf{amsmath} package
where \textit{child} can be |chapter| or |section|
depending on the chosen structuring.
Alternatively, one can modify the macro |\thepage| appropriately
and reset the counter |page| at the start of each child file.

%%%%%%%%%%%%%%%%%%%%%%%%%%%%%%%%%%%%%%%%%%%%%%%%%%%%%%%%%%%%%%%%%%%%%%%%%%%%%%%%
\subsection{Conditional Processing}
\label{sec:conditional}

The package provides a mechanism to compile different versions
of a document. To customise the versions further some conditional processing
can come in handy to distinguish which version is being compiled.
The package provides two macros to describe the compilation context:

%%%%%%%%%%%%%%%%%%%%%%%%%%%%%%%%%%%%%%%%
\DescribeMacro{\ifchilddoc}
The conditional |\ifchilddoc| distinguishes between the compilation of
child documents and the main document:
%
\begin{center}
|\ifchilddoc |\textit{child-code}| |[|\||else |\textit{main-code}]| \||fi|
\end{center}

%%%%%%%%%%%%%%%%%%%%%%%%%%%%%%%%%%%%%%%%
\DescribeMacro{\childdocname}
\DescribeMacro{\childdocjob}
The macro |\childdocname| contains the filename (without extension)
of the main or child file being processed.
Note that |\childdocjob| will always contain the name of the main file.

%%%%%%%%%%%%%%%%%%%%%%%%%%%%%%%%%%%%%%%%
\paragraph{Title Page.}

Conditional processing can be used to include a title or banner page
in the main document when proper precautions are taken.
Importantly, the code in the main file should ensure that the page counter
(as well as other status parameters which are stored in the |.aux| files)
takes the same value after the conditional processing.
Otherwise the page numbers may take divergent values
depending on which part is compiled.

For example, a title page could be declared by:
%
\begin{center}
\begin{tabular}{l}
|\ifchilddoc\||else|\\
|\addtocounter{page}{-1}|\\
\textit{code for title page}\\
|\newpage|\\
|\||fi|
\end{tabular}
\end{center}
%
A banner page for the child documents can be generated by:
%
\begin{center}
\begin{tabular}{l}
|\ifchilddoc|\\
|\addtocounter{page}{-1}|\\
\textit{code for banner page}\\
|\newpage|\\
|\||fi|
\end{tabular}
\end{center}
%
Here one could write a message such as:
\begin{center}
|This is the part \childdocname{} of \childdocjob{}.|
\end{center}

%%%%%%%%%%%%%%%%%%%%%%%%%%%%%%%%%%%%%%%%%%%%%%%%%%%%%%%%%%%%%%%%%%%%%%%%%%%%%%%%
\subsection{Flags}
\label{sec:flags}

The package makes it easy to generate different versions
of the main or child documents.
To this end compilation flags can be defined
and assigned different default values.
They will be particularly useful in conjunction
with the forwarding mechanism described in \secref{sec:forward}.

For example, it may be useful to have a flag |\version|
which can be set to |draft| or |final|.
The document source will contain some conditional code
depending on the value of |\version|.
Suppose further, the flag should default to |final| for the main file
and to |draft| for child files
which is a natural assignment for editing the document.
This is achieved by placing the following code
in the preamble of the main document
(below the |\childdocmain| directive):
%
\begin{center}
\begin{tabular}{l}
|\ifchilddoc|\\
|\providecommand{\version}{draft}|\\
|\||else|\\
|\providecommand{\version}{final}|\\
|\||fi|
\end{tabular}
\end{center}
%
The definition by |\providecommand| makes sure
that previous definitions are not overwritten.
Further statements |\providecommand{\version}{...}|
can thus be added before the above code to override it.

For the main file, one might add a line
(between |\childdocmain| and the above block)
%
\begin{center}
|%\ifchilddoc\||else\providecommand{\version}{draft}\||fi|
\end{center}
%
which can be uncommented to produce a draft version.
Likewise one can add a line to the very top of a child file
(above the |\childdocof{|\textit{main}|}| directive)
%
\begin{center}
|%\providecommand{\version}{final}|
\end{center}
%
which can be uncommented to produce the final version of this child document.

%%%%%%%%%%%%%%%%%%%%%%%%%%%%%%%%%%%%%%%%%%%%%%%%%%%%%%%%%%%%%%%%%%%%%%%%%%%%%%%%
\subsection{Forwarding}
\label{sec:forward}

Different versions of the main or child documents
using compilation flags as described in \secref{sec:flags}
can be (permanently) stored in different files
for convenient compilation, viewing and distribution.
To this end, the package defines a command
to pass on compilation to a different file:

%%%%%%%%%%%%%%%%%%%%%%%%%%%%%%%%%%%%%%%%
\DescribeMacro{\childdocforward}
The command |\childdocforward| redirects processing to
another source file:
%
\begin{center}
\begin{tabular}{l}
|\input{childdoc.def}|\\
|\childdocforward[|\textit{main}|]{|\textit{dest}|}|\\
\end{tabular}
\end{center}
%
The argument \textit{dest} is the destination file
(without extension).
It should be the main file or one of the child files.
Note that further \textsf{childdoc} directives
such as |\childdocof| and |\childdocforward|
in the indicated file will be processed in this form.
The optional argument \textit{main}
passes on directly to the main file \textit{main}
while pretending to compile the child \textit{dest}.
This form behaves as if \textit{dest}
issues |\childdocof{|\textit{main}|}| right away,
and no further \textsf{childdoc} directives will be processed.

%%%%%%%%%%%%%%%%%%%%%%%%%%%%%%%%%%%%%%%%
\DescribeMacro{\...prefix}
In the alternative form |\childdocforwardprefix|,
%
\begin{center}
\begin{tabular}{l}
|\input{childdoc.def}|\\
|\childdocforwardprefix[|\textit{main}|]{|\textit{prefix}|}{|\textit{dest}|}|
\end{tabular}
\end{center}
%
the destination file is determined by a pattern
depending on the current file:
To make this work, the current file must be called
`{\textit{prefix}\hspace{0.2em}\textit{suffix}}'
with \textit{prefix} matching precisely the argument.
Processing is then passed on to the file
`{\textit{dest}\hspace{0.2em}\textit{suffix}}'.
Surely, the same effect is achieved by
directly specifying the
argument `{\textit{dest}\hspace{0.2em}\textit{suffix}}'
in the first form.
However, that requires to set up a different file
for each child. With the alternative form of the command
all these files can have exactly the same content
which simplifies setting them up and maintaining them.

For example, the following file |draft.tex|
with a compilation flag |\version| as described in \secref{sec:flags}
compiles the main document as a draft:
%
\begin{center}
\begin{tabular}{l}
|\def\version{draft}|\\
|\input{childdoc.def}|\\
|\childdocforward{|\textit{main}|}|
\end{tabular}
\end{center}
%
Likewise, the following files |final|\textit{nn}|.tex|
compile the final version of the child document
|child|\textit{nn}|.tex|:
%
\begin{center}
\begin{tabular}{l}
|\def\version{final}|\\
|\input{childdoc.def}|\\
|\childdocforwardprefix{final}{child}|
\end{tabular}
\end{center}
%

Note that when several versions of a main file and/or of each child file
are to be generated, it may be convenient to set up a |Makefile| or
shell script to automatise the process.

%%%%%%%%%%%%%%%%%%%%%%%%%%%%%%%%%%%%%%%%%%%%%%%%%%%%%%%%%%%%%%%%%%%%%%%%%%%%%%%%
\subsection{Command Line Processing}
\label{sec:commandline}

The effect of redirection files can also be achieved by invoking
the \LaTeX{} compiler with a more elaborate command line.
Most conveniently this should be done as part
of a shell script or a |Makefile|.

When using \textsf{childdoc} in the main file, the following
command lines effectively perform a redirection
(note that depending on the shell being used,
backslashes may have to be doubled: `|\|' $\to$ `|\\|'):
%
\begin{center}
|... -jobname "|\textit{target}|" |\\|"|[\textit{flags}]%
|\input{childdoc.def}\childdocforward[|\textit{main}|]{|\textit{dest}|}"|
\end{center}
%
Here \textit{target} is the name of the output file,
\textit{main} is the name of the main file
and \textit{dest} is the name of the main or child file to be processed
(all filenames without extensions).
The optional argument \textit{main} can be omitted
if \textit{main} matches \textit{dest}.
Optionally, compilation \textit{flags} can be defined via |\def| commands.
This command line makes the \TeX{} engine believe
it is compiling the file \textit{target}
whose content is specified as the latter parameter.
The provided code then forwards the processing to
\textit{main} or \textit{dest} as described in \secref{sec:forward}.

%%%%%%%%%%%%%%%%%%%%%%%%%%%%%%%%%%%%%%%%%%%%%%%%%%%%%%%%%%%%%%%%%%%%%%%%%%%%%%%%
\subsection{Include by Input}
\label{sec:input}

Including child documents by |\include| has some restrictions by design.
Most notably, the content of a child document always occupies
its own set of pages; pages cannot be shared between child documents.
Usually, this behaviour makes perfect sense
because each child document contain an essential part of the document.
However, in some situations it may be desirable to compose
a document from a collection of parts
without having mandatory page breaks between then.
For this case, the package
provides a mechanism to include parts
by |\input| which can also be processed individually.
However, by construction this mechanism
requires manual handling of the content to be output.

%%%%%%%%%%%%%%%%%%%%%%%%%%%%%%%%%%%%%%%%
\DescribeMacro{\ifchilddocmanual}
The main file should be prepared as usual, see \secref{sec:include}.
However, the document body must make a distinction
between processing of an individual part and of the main document, e.g.:
%
\begin{center}
\begin{tabular}{l}
|\ifchilddocmanual|\\
|\input{\childdocname}|\\
|\||else|\\
\textit{document body with }|\input{|\textit{part}|}|\\
|\||fi|
\end{tabular}
\end{center}
%
The conditional |\ifchilddocmanual| is true whenever
a part to be included by |\input| is being compiled,
and the name of the part is stored in |\childdocname|.

%%%%%%%%%%%%%%%%%%%%%%%%%%%%%%%%%%%%%%%%
\DescribeMacro{\childdocby}
Each part to be included by |\input| should start with:
%
\begin{center}
\begin{tabular}{l}
|\input{childdoc.def}|\\
|\childdocby{|\textit{main}|}|\\
\end{tabular}
\end{center}
%
The directive |\childdocby| is similar to |\childdocof|
described in \secref{sec:include},
but the subsequent selection of content must be done manually.
To that end, both |\ifchilddoc| and |\ifchilddocmanual|
will be true upon processing of a part,
and the name of the part is stored in |\childdocname|.
Note that |\jobname| will be set to the filename of the current part
so that each part receives an individual |.aux| file
that does not interfere with the |.aux| file(s) of the main document.
This behaviour can be altered by the alternative form
|\childdocby[*]{|\textit{main}|}| (with a non-empty optional argument)
which uses the |.aux| file of the main document
by setting |\jobname| to \textit{main}.

%%%%%%%%%%%%%%%%%%%%%%%%%%%%%%%%%%%%%%%%%%%%%%%%%%%%%%%%%%%%%%%%%%%%%%%%%%%%%%%%
\subsection{Driver Development}
\label{sec:driver}

The \textsf{childdoc} mechanism can also be use for the development
of definition files such as \LaTeX{} styles or classes.
This case differs from the above setup with multiple parts
included by |\include| in that no |\includeonly| should be invoked.
This can be achieved by starting the include file
(before |\ProvidesPackage|) with:
%
\begin{center}
\begin{tabular}{l}
|\input{childdoc.def}|\\
|\childdocforward{|\textit{main}|}|\\
\end{tabular}
\end{center}
%
or alternatively with:
%
\begin{center}
\begin{tabular}{l}
|\input{childdoc.def}|\\
|\childdocby{|\textit{main}|}|\\
\end{tabular}
\end{center}
%
Both forms have slightly different effects as described above.
The main file is prepared as usual, see \secref{sec:include}.

%%%%%%%%%%%%%%%%%%%%%%%%%%%%%%%%%%%%%%%%%%%%%%%%%%%%%%%%%%%%%%%%%%%%%%%%%%%%%%%%
\subsection{Legacy Detection}
\label{sec:detection}

The directive |\childdocmain| in the main file can detect
whether the complete document or merely a child is to be compiled
even without using the directive |\childdocof|.
This method is deprecated because it is less robust
and there is no compelling reason to use it;
it is merely provided for backward compatibility
and it may be removed in future versions.

If the detection mechanism is to be used,
it is mandatory to correctly specify
the filename of the main file as the argument of |\childdocmain|:
%
\begin{center}
\begin{tabular}{l}
|\input{childdoc.def}|\\
|\childdocmain{|\textit{main}|}|\\
\end{tabular}
\end{center}
%
If |\jobname| does not match the argument \textit{main} of |\childdocmain|,
it is assumed that |\jobname| points to the child file to be compiled.
When using |\childdocmain| with the main file specified as argument,
it suffices to start a child file
with just |\input{|\textit{main}|}|
without loading of the package and using |\childdocof|.
If instead all processing is done
with the appropriate \textsf{childdoc} directives,
the argument of \textit{main} of |\childdocmain| can be empty.

An alternative version of the command line processing described
in \secref{sec:commandline} using the detection mechanism reads:
%
\begin{center}
|... -jobname "|\textit{target}|" "|[\textit{flags}]%
[|\def\jobname{|\textit{dest}|}|]|\input{|\textit{main}|}"|
\end{center}

%%%%%%%%%%%%%%%%%%%%%%%%%%%%%%%%%%%%%%%%%%%%%%%%%%%%%%%%%%%%%%%%%%%%%%%%%%%%%%%%
\subsection{Manual Code}
\label{sec:manual}

In case one cannot be certain whether the definitions file |childdoc.def|
is installed on the target \TeX{} distribution
and one prefers not to ship it,
it is conceivable to paste a few relevant commands into the sources.

To that end, drop all statements |\input{childdoc.def}|
and perform the replacements as outlined below.
Instead of |\childdocmain{|\textit{main}|}| add the following code
to the top of the main file:
%
\begin{center}
\begin{tabular}{l}
|\||ifdefined\childdocname\endinput\||fi\newif\ifchilddoc|\\
|\edef\childdocname{\scantokens\expandafter{\jobname\noexpand}}|\\
|\def\childdocmain{|\textit{main}|}\||ifx\childdocmain\childdocname\||else|\\
|\childdoctrue\includeonly{\childdocname}\let\jobname\childdocmain\||fi|\\
\end{tabular}
\end{center}
%
Instead of |\childdocof{|\textit{main}|}| just include the main file
at the top of each child file:
%
\begin{center}
|\input{|\textit{main}|}|
\end{center}
%
A simple redirection |\childdocforward{|\textit{dest}|}| is achieved by:
%
\begin{center}
|\def\jobname{|\textit{dest}|}\input{\jobname}|
\end{center}
%
The redirection with prefix
|\childdocforwardprefix[|\textit{prefix}|]{|\textit{dest}|}|
is accomplished by:
%
\begin{center}
\begin{tabular}{l}
|{\edef\jobname{\scantokens\expandafter{\jobname\noexpand}}|\\
|\def\redirectjob |\textit{prefix}|#1~~~{\gdef\jobname{|\textit{dest}|#1}}|\\
|\expandafter\redirectjob\jobname~~~}\input{\jobname}|
\end{tabular}
\end{center}

In an alternative approach,
child documents can be compiled by a specific command line
without additional code or specific definitions:
%
\begin{center}
|... -jobname "|\textit{target}|" "|[\textit{flags}]%
|\includeonly{|\textit{dest}|}\input{|\textit{main}|}"|
\end{center}
%

%%%%%%%%%%%%%%%%%%%%%%%%%%%%%%%%%%%%%%%%%%%%%%%%%%%%%%%%%%%%%%%%%%%%%%%%%%%%%%%%
%%%%%%%%%%%%%%%%%%%%%%%%%%%%%%%%%%%%%%%%%%%%%%%%%%%%%%%%%%%%%%%%%%%%%%%%%%%%%%%%
\section{Information}

%%%%%%%%%%%%%%%%%%%%%%%%%%%%%%%%%%%%%%%%%%%%%%%%%%%%%%%%%%%%%%%%%%%%%%%%%%%%%%%%
\subsection{Copyright}

Copyright \copyright{} 2017--2018 Niklas Beisert

This work may be distributed and/or modified under the
conditions of the \LaTeX{} Project Public License, either version 1.3
of this license or (at your option) any later version.
The latest version of this license is in
  \url{http://www.latex-project.org/lppl.txt}
and version 1.3 or later is part of all distributions of \LaTeX{}
version 2005/12/01 or later.

This work has the LPPL maintenance status `maintained'.

The Current Maintainer of this work is Niklas Beisert.

This work consists of the files |README.txt|, |childdoc.ins| and |childdoc.dtx|
as well as the derived files |childdoc.def|, |cdocsamp.tex|
with |cdocsch1.tex|, |cdocsch2.tex|, |cdocspt3.tex|, |cdocspt4.tex|,
|cdocsdrf.tex|, |cdocsfn1.tex|, |cdocsfn2.tex|
as well as |childdoc.pdf|.

%%%%%%%%%%%%%%%%%%%%%%%%%%%%%%%%%%%%%%%%%%%%%%%%%%%%%%%%%%%%%%%%%%%%%%%%%%%%%%%%
\subsection{Files and Installation}

The package consists of the files:
%
\begin{center}
\begin{tabular}{ll}
    |README.txt|   & readme file \\
    |childdoc.ins| & installation file \\
    |childdoc.dtx| & source file \\
    |childdoc.def| & definition file \\
    |cdocsamp.tex| & sample main file \\
    |cdocsch1.tex| & sample include file \\
    |cdocsch2.tex| & sample include file \\
    |cdocspt3.tex| & sample part file \\
    |cdocspt4.tex| & sample part file \\
    |cdocsdrf.tex| & sample redirection file \\
    |cdocsfn1.tex| & sample redirection file \\
    |cdocsfn2.tex| & sample redirection file \\
    |childdoc.pdf| & manual
\end{tabular}
\end{center}
%
The distribution consists of the files
|README.txt|, |childdoc.ins| and |childdoc.dtx|.
%
\begin{itemize}
\item
Run (pdf)\LaTeX{} on |childdoc.dtx|
to compile the manual |childdoc.pdf| (this file).
\item
Run \LaTeX{} on |childdoc.ins| to create the definitions file |childdoc.def|
and the sample |cdocsamp.tex| with include files
|cdocsch1.tex|, |cdocsch2.tex|, |cdocspt3.tex|, |cdocspt4.tex|,
|cdocsdrf.tex|, |cdocsfn1.tex|, |cdocsfn2.tex|.
Then copy the file |childdoc.def| to an appropriate directory of your \LaTeX{}
distribution, e.g.\ \textit{texmf-root}|/tex/latex/childdoc|.
\end{itemize}

%%%%%%%%%%%%%%%%%%%%%%%%%%%%%%%%%%%%%%%%%%%%%%%%%%%%%%%%%%%%%%%%%%%%%%%%%%%%%%%%
\subsection{Related CTAN Packages}

There are several other packages which offer a similar functionality:
%
\begin{itemize}
\item
The packages
\href{http://ctan.org/pkg/docmute}{\textsf{docmute}},
\href{http://ctan.org/pkg/includex}{\textsf{includex}} and
\href{http://ctan.org/pkg/standalone}{\textsf{standalone}}
provide commands to include only the document body of
a child file thus allowing both files to be compiled individually.
\item
The packages \href{http://ctan.org/pkg/subdocs}{\textsf{subdocs}}
and \href{http://ctan.org/pkg/subfiles}{\textsf{subfiles}}
provide structures in which the main and child documents can be
encapsulated and allowing them to be compiled individually.
The inclusion mechanism is different from the conventional |\include|.
\item
The package \href{http://ctan.org/pkg/combine}{\textsf{combine}}
is an elaborate solution to combine several documents into one.
\end{itemize}
%
See also the CTAN topic \href{http://ctan.org/topic/subdocs}{\textsf{subdocs}}
for further related packages.
The present package differs from the above solutions in that
a document structure constructed with the conventional |\include| mechanism
just needs two extra commands at the top of every file
such that all constituent files can be compiled individually.

%%%%%%%%%%%%%%%%%%%%%%%%%%%%%%%%%%%%%%%%%%%%%%%%%%%%%%%%%%%%%%%%%%%%%%%%%%%%%%%%
%\subsection{Feature Suggestions}
%
%The following is a list of features which may be useful for future
%versions of this package:
%%
%\begin{itemize}
%\item
%\ldots
%\end{itemize}

%%%%%%%%%%%%%%%%%%%%%%%%%%%%%%%%%%%%%%%%%%%%%%%%%%%%%%%%%%%%%%%%%%%%%%%%%%%%%%%%
\subsection{Revision History}

%%%%%%%%%%%%%%%%%%%%%%%%%%%%%%%%%%%%%%%%
\paragraph{v2.0:} 2018/12/30

\begin{itemize}
\item
immediate forward processing
\item
added |\childdocby| mechanism
\item
manual restructured
\end{itemize}

%%%%%%%%%%%%%%%%%%%%%%%%%%%%%%%%%%%%%%%%
\paragraph{v1.6:} 2018/01/17

\begin{itemize}
\item
application for development of include files
\item
corrections to manual
\end{itemize}

%%%%%%%%%%%%%%%%%%%%%%%%%%%%%%%%%%%%%%%%
\paragraph{v1.5:} 2017/05/21

\begin{itemize}
\item
more complete structuring introduced
\item
|\childdocof| introduced
\item
|\childdoc| renamed to |\childdocmain|
\item
|\childredirect| renamed to |\childdocforward| and |\childdocforwardprefix|
and functionality expanded
\end{itemize}

%%%%%%%%%%%%%%%%%%%%%%%%%%%%%%%%%%%%%%%%
\paragraph{v1.0:} 2017/04/27

\begin{itemize}
\item
manual and install package
\item
first version published on CTAN
\end{itemize}

%%%%%%%%%%%%%%%%%%%%%%%%%%%%%%%%%%%%%%%%
\paragraph{v0.6:} 2017/04/26

\begin{itemize}
\item
redirection mechanism added
\end{itemize}

%%%%%%%%%%%%%%%%%%%%%%%%%%%%%%%%%%%%%%%%
\paragraph{v0.5:} 2017/04/26

\begin{itemize}
\item
functionality in definition file
\end{itemize}


%%%%%%%%%%%%%%%%%%%%%%%%%%%%%%%%%%%%%%%%%%%%%%%%%%%%%%%%%%%%%%%%%%%%%%%%%%%%%%%%
%%%%%%%%%%%%%%%%%%%%%%%%%%%%%%%%%%%%%%%%%%%%%%%%%%%%%%%%%%%%%%%%%%%%%%%%%%%%%%%%
%%%%%%%%%%%%%%%%%%%%%%%%%%%%%%%%%%%%%%%%%%%%%%%%%%%%%%%%%%%%%%%%%%%%%%%%%%%%%%%%
\appendix

\settowidth\MacroIndent{\rmfamily\scriptsize 000\ }

 \DocInput{childdoc.dtx}

\end{document}
%</driver>
% \fi
%
% %%%%%%%%%%%%%%%%%%%%%%%%%%%%%%%%%%%%%%%%%%%%%%%%%%%%%%%%%%%%%%%%%%%%%%%%%%%%%%
% %%%%%%%%%%%%%%%%%%%%%%%%%%%%%%%%%%%%%%%%%%%%%%%%%%%%%%%%%%%%%%%%%%%%%%%%%%%%%%
% \section{Sample}
%\iffalse
%<*samplemain>
%\fi
%
% The following presents a sample document
% with two chapters, two parts, a title page,
% a compile flag as well as three forwarding files to set the flag.
% It consists of eight |.tex| files:
% \begin{center}
% \begin{tabular}{ll}
% |cdocsamp.tex|&main file\\
% |cdocsch1.tex|&include file for chapter 1\\
% |cdocsch2.tex|&include file for chapter 2\\
% |cdocspt3.tex|&include file for part 3\\
% |cdocspt4.tex|&include file for part 4\\
% |cdocsdrf.tex|&forwarding file for main file in draft mode\\
% |cdocsfi1.tex|&forwarding file for final version of chapter 1\\
% |cdocsfi2.tex|&forwarding file for final version of chapter 2\\
% \end{tabular}
% \end{center}
% Each of the eight files can be compiled directly by the \LaTeX{} compiler.
%
% %%%%%%%%%%%%%%%%%%%%%%%%%%%%%%%%%%%%%%
% \paragraph{Main File.}
%
% The main file is called |cdocsamp.tex|.
%
% Load the \textsf{childdoc} definitions and
% declare the filename for the main document:
%    \begin{macrocode}
\input{childdoc.def}
\childdocmain{}
%    \end{macrocode}

% Optional override for |\version| flag:
%    \begin{macrocode}
%%\ifchilddoc\else\providecommand{\version}{draft}\fi
%    \end{macrocode}

% Define the default values for the |\version| flag
% (|final| for the main file and |draft| for childs):
%    \begin{macrocode}
\ifchilddoc
\providecommand{\version}{draft}
\else
\providecommand{\version}{final}
\fi
%    \end{macrocode}

% Load the standard document class:
%    \begin{macrocode}
\documentclass[12pt]{article}
%    \end{macrocode}

% Start the document body:
%    \begin{macrocode}
\begin{document}
%    \end{macrocode}

% Declare a title page.
% Print title, part of document being processed and version flag:
%    \begin{macrocode}
\addtocounter{page}{-1}
\begin{center}
{\LARGE\bfseries{}childdoc example\par}
\vspace{1cm}
\ifchilddoc
\ifchilddocmanual part\else chapter\fi:
`\childdocname' of `\childdocjob'\par
\else
main document: `\childdocjob'\par
\fi
version: \version\par
\end{center}
\newpage
%    \end{macrocode}

% Manually include selected file,
% otherwise process as usual:
%    \begin{macrocode}
\ifchilddocmanual
\section*{part `\childdocname'}
\input{\childdocname}
\else
%    \end{macrocode}

% Include the two chapters:
%    \begin{macrocode}
\include{cdocsch1}
\include{cdocsch2}
%    \end{macrocode}

% Include the two parts unless only chapters should be displayed:
%    \begin{macrocode}
\ifchilddoc\else
\section{part three}
\input{cdocspt3}
\section{part four}
\input{cdocspt4}
\fi
%    \end{macrocode}

% Process as usual until here:
%    \begin{macrocode}
\fi
%    \end{macrocode}

% End of document body:
%    \begin{macrocode}
\end{document}
%    \end{macrocode}
%\iffalse
%</samplemain>
%\fi
%
% %%%%%%%%%%%%%%%%%%%%%%%%%%%%%%%%%%%%%%
% \paragraph{Chapter Include Files.}
%
% The include files are called |cdocsch1.tex| and |cdocsch2.tex|.
%
%\iffalse
%<*samplechap1|samplechap2>
%\fi

% Optional override for |\version| flag:
%    \begin{macrocode}
%%\providecommand{\version}{final}
%    \end{macrocode}

% Include the main document:
%    \begin{macrocode}
\input{childdoc.def}
\childdocof{cdocsamp}
%    \end{macrocode}

%\iffalse
%</samplechap1|samplechap2>
%\fi
%
%\iffalse
%<*samplechap1>
%\fi
% Some text for chapter 1:
%    \begin{macrocode}
\section{one}
some text in chapter one
%    \end{macrocode}

%\iffalse
%</samplechap1>
%\fi
% Some text for chapter 2:
%\iffalse
%<*samplechap2>
%\fi
%    \begin{macrocode}
\section{two}
more text in chapter two
%    \end{macrocode}

%\iffalse
%</samplechap2>
%\fi
%
% %%%%%%%%%%%%%%%%%%%%%%%%%%%%%%%%%%%%%%
% \paragraph{Part Include Files.}
%
% The include files are called |cdocspt3.tex| and |cdocspt4.tex|.
%
%\iffalse
%<*samplepart3|samplepart4>
%\fi

% Optional override for |\version| flag:
%    \begin{macrocode}
%%\providecommand{\version}{final}
%    \end{macrocode}

% Include the main document:
%    \begin{macrocode}
\input{childdoc.def}
\childdocby{cdocsamp}
%    \end{macrocode}

%\iffalse
%</samplepart3|samplepart4>
%\fi
%
%\iffalse
%<*samplepart3>
%\fi
% Some text for part 3:
%    \begin{macrocode}
some text in part three
%    \end{macrocode}

%\iffalse
%</samplepart3>
%\fi
% Some text for part 4:
%\iffalse
%<*samplepart4>
%\fi
%    \begin{macrocode}
more text in part four
%    \end{macrocode}

%\iffalse
%</samplepart4>
%\fi
%
% %%%%%%%%%%%%%%%%%%%%%%%%%%%%%%%%%%%%%%
% \paragraph{Forwarding for a Complete Draft.}
%
% The following forwarding file |cdocsdrf.tex|
% compiles the main document in draft mode:
%\iffalse
%<*sampledraft>
%\fi
%    \begin{macrocode}
\def\version{draft}
\input{childdoc.def}
\childdocforward{cdocsamp}
%    \end{macrocode}

%\iffalse
%</sampledraft>
%\fi
%
% %%%%%%%%%%%%%%%%%%%%%%%%%%%%%%%%%%%%%%
% \paragraph{Forwarding for Final Version of the Chapters.}
%
% The following forwarding files |cdocsfn1.tex| and |cdocsfn2.tex|
% (with identical content)
% compile the final versions of the child documents
% |cdocsch1.tex| and |cdocsch2.tex|, respectively:
%\iffalse
%<*samplefinal>
%\fi
%    \begin{macrocode}
\def\version{final}
\input{childdoc.def}
\childdocforwardprefix[cdocsamp]{cdocsfn}{cdocsch}
%    \end{macrocode}

%\iffalse
%</samplefinal>
%\fi
%
% %%%%%%%%%%%%%%%%%%%%%%%%%%%%%%%%%%%%%%
% \paragraph{Command Line Processing.}
%
% The following three command lines generate the output files
% |cdocscld|, |cdocscl1| and |cdocscl2|
% which should be identical to
% |cdocsdrf|, |cdocsch1| and |cdocsfn2|, respectively:
% \begin{center}
% \begin{tabular}{l}
% |latex -jobname cdocscld \|\\
% |  "\def\version{draft}\input{childdoc.def}\childdocforward{cdocsamp}"|\\
% |latex -jobname cdocscl1 \|\\
% |  "\input{childdoc.def}\childdocforward[cdocsamp]{cdocsch1}"|\\
% |latex -jobname cdocscl2 \|\\
% |  "\def\version{final}\input{childdoc.def}\childdocforward{cdocsch2}"|
% \end{tabular}
% \end{center}
% Note that the trailing backslash on each first line
% merely continues the input to the second line
% (for convenient cut ant paste).
% Furthermore, the command |latex| can be replaced by any
% of its alternative versions such as |pdflatex|.
%
% %%%%%%%%%%%%%%%%%%%%%%%%%%%%%%%%%%%%%%%%%%%%%%%%%%%%%%%%%%%%%%%%%%%%%%%%%%%%%%
% %%%%%%%%%%%%%%%%%%%%%%%%%%%%%%%%%%%%%%%%%%%%%%%%%%%%%%%%%%%%%%%%%%%%%%%%%%%%%%
% \section{Implementation}
%\iffalse
%<*package>
%\fi
%
% This section describes the definitions file |childdoc.def|.

% The definitions cannot be loaded using |\usepackage| or |\RequirePackage|
% which has a mechanism to prevent loading a style file more than once.
% When loading the definitions by means of |\input|
% multiple instances have to be prevented manually:
%\iffalse
%This code needs to be before the `\ProvidesFile' directive
%which is defined at the beginning of this file.
%Therefore it is also placed there and commented out here.
%</package>
%<*discard>
%\fi
%    \begin{macrocode}
\ifdefined\childdocmain\endinput\fi
%    \end{macrocode}
%\iffalse
%</discard>
%<*package>
%\fi
%
% \macro{\ifchilddoc}
% \macro{\ifchilddocmanual}
% The conditional |\ifchilddoc| tells whether a
% child (true) or main (false) document is being compiled.
% The conditional |\ifchilddocmanual| tells whether
% the |\includeonly| mechanism is used (false) or
% the selection of child files must be performed manually (true).
% The definitions initialise to false:
%    \begin{macrocode}
\newif\ifchilddoc
\newif\ifchilddocmanual
%    \end{macrocode}

% \macro{\childdocname}
% \macro{\childdocjob}
% The macro |\childdocname| stores the name of the main document
% to be compiled. The macro |\childdocjob| stores the name of
% the document on which the \LaTeX{} compiler was originally invoked.
% The content of |\jobname| cannot be compared
% to filenames specified in the source due to different catcodes.
% The following code rescans |\jobname|, stores the result
% in |\childdocname| and saves a copy in |\childdocjob|:
%    \begin{macrocode}
\edef\childdocname{\scantokens\expandafter{\jobname\noexpand}}
\let\childdocjob\childdocname
%    \end{macrocode}

% \macro{\childdocdisable}
% The macro |\childdocdisable| prevents the main file
% from being processed more than once.
% At this stage, the main document command |\childdocmain|
% is assumed to be called once again where it should do nothing.
% Any subsequent call to it should prevent
% a secondary processing of the main document
% It overwrites the forwarding commands
% |\childdocof| and |\childdocforward|
% with empty macros to prevent further inclusions of the main document:
%    \begin{macrocode}
\newcommand{\childdocdisable}
{
  \renewcommand{\childdocmain}[1]{\renewcommand{\childdocmain}[1]{\endinput}}
  \renewcommand{\childdocof}[1]{}
  \renewcommand{\childdocby}[2][]{}
  \renewcommand{\childdocforward}[2][]{}
  \renewcommand{\childdocdisable}{}
}
%    \end{macrocode}

% \macro{\childdocmain}
% The macro |\childdocmain| is to be called at the top of the main file
% with nothing or the main filename (without extension) as argument.
% First, it breaks loops.
% If the argument is not empty and does not match |\childdocname|
% (which is set by the first inclusion of |childdoc.def|),
% |\ifchilddoc| is set to true, |\includeonly| is applied to the child file
% and |\jobname| is set to the main file
% (for proper handling of |.aux| files):
%    \begin{macrocode}
\newcommand{\childdocmain}[1]
{
  \childdocdisable\childdocmain{}
  \if?#1?\else
    \begingroup
      \def\childdoctmp{#1}
      \ifx\childdoctmp\childdocname
        \def\childdoctmp{}
      \else
        \def\childdoctmp
        {
          \childdoctrue
          \includeonly{\childdocname}
          \def\childdocjob{#1}
          \def\jobname{#1}
        }
      \fi
      \expandafter
    \endgroup
    \childdoctmp
  \fi
}
%    \end{macrocode}

% \macro{\childdocof}
% The command |\childdocof| redirects
% compilation to the main file |#1|.
%    \begin{macrocode}
\newcommand{\childdocof}[1]
{
  \childdocdisable
  \childdoctrue
  \includeonly{\childdocname}
  \def\jobname{#1}
  \def\childdocjob{#1}
  \input{#1}
}
%    \end{macrocode}

% \macro{\childdocby}
% The command |\childdocby| ....
%    \begin{macrocode}
\newcommand{\childdocby}[2][]
{
  \childdocdisable
  \childdoctrue
  \childdocmanualtrue
  \if?#1?\else
    \def\jobname{#2}
  \fi
  \def\childdocjob{#2}
  \input{#2}
  \endinput
}
%    \end{macrocode}

% \macro{\childdocforward}
% The command |\childdocforward| redirects
% compilation to the main file or
% (if the optional argument is given) a child file.
% Parameters are set as if the main file
% or a child file starting with |\childdocof| was compiled.
% Then compilation is handed over to the main file:
%    \begin{macrocode}
\newcommand{\childdocforward}[2][]
{
  \begingroup
    \if?#1?
      \def\childdoctmp
      {
        \def\childdocname{#2}
        \def\childdocjob{#2}
        \def\jobname{#2}
        \input{#2}
        \endinput
      }
    \else
      \def\childdoctmp
      {
        \childdocdisable
        \def\childdocname{#2}
        \childdoctrue
        \includeonly{#2}
        \def\childdocjob{#1}
        \def\jobname{#1}
        \input{#1}
        \endinput
      }
    \fi
    \expandafter
  \endgroup
  \childdoctmp
}
%    \end{macrocode}

% \macro{\childdocforwardprefix}
% The command |\childdocforwardprefix| redirects
% compilation to the main or a child file by means of a pattern.
% The prefix |#1| in the current filename is replaced by |#2|
% and the suffix of the current filename is kept
% (it is assumed that the filename does not contain the substring `|~~~|'
% which is used as a delimiter).
% Compilation is handed over to the new file by |\childdocforward|:
%    \begin{macrocode}
\newcommand{\childdocforwardprefix}[3][]
{
  \begingroup
    \def\childdocextract #2##1~~~{\def\childdoctmp{\childdocforward[#1]{#3##1}}}
    \expandafter\childdocextract\childdocname~~~
    \expandafter
  \endgroup
  \childdoctmp
}
%    \end{macrocode}

% \macro{\childdoc}
% The deprecated macro |\childdoc| is a legacy version of |\childdocmain|:
%    \begin{macrocode}
\newcommand{\childdoc}{\childdocmain}
%    \end{macrocode}

% \macro{\childdocredirect}
% The deprecated macro |\childdocredirect| is a legacy version
% of |\childdocforward| and |\childdocforwardprefix|:
%    \begin{macrocode}
\newcommand{\childdocredirect}[2][]
{
  \begingroup
    \if?#1?
      \def\childdoctmp{\childdocforward{#2}}
    \else
      \def\childdoctmp{\childdocforwardprefix{#1}{#2}}
    \fi
    \expandafter
  \endgroup
  \childdoctmp
}
%    \end{macrocode}

%\iffalse
%</package>
%\fi
%
\endinput
|
and perform the replacements as outlined below.
Instead of |\childdocmain{|\textit{main}|}| add the following code
to the top of the main file:
%
\begin{center}
\begin{tabular}{l}
|\||ifdefined\childdocname\endinput\||fi\newif\ifchilddoc|\\
|\edef\childdocname{\scantokens\expandafter{\jobname\noexpand}}|\\
|\def\childdocmain{|\textit{main}|}\||ifx\childdocmain\childdocname\||else|\\
|\childdoctrue\includeonly{\childdocname}\let\jobname\childdocmain\||fi|\\
\end{tabular}
\end{center}
%
Instead of |\childdocof{|\textit{main}|}| just include the main file
at the top of each child file:
%
\begin{center}
|\input{|\textit{main}|}|
\end{center}
%
A simple redirection |\childdocforward{|\textit{dest}|}| is achieved by:
%
\begin{center}
|\def\jobname{|\textit{dest}|}\input{\jobname}|
\end{center}
%
The redirection with prefix
|\childdocforwardprefix[|\textit{prefix}|]{|\textit{dest}|}|
is accomplished by:
%
\begin{center}
\begin{tabular}{l}
|{\edef\jobname{\scantokens\expandafter{\jobname\noexpand}}|\\
|\def\redirectjob |\textit{prefix}|#1~~~{\gdef\jobname{|\textit{dest}|#1}}|\\
|\expandafter\redirectjob\jobname~~~}\input{\jobname}|
\end{tabular}
\end{center}

In an alternative approach,
child documents can be compiled by a specific command line
without additional code or specific definitions:
%
\begin{center}
|... -jobname "|\textit{target}|" "|[\textit{flags}]%
|\includeonly{|\textit{dest}|}\input{|\textit{main}|}"|
\end{center}
%

%%%%%%%%%%%%%%%%%%%%%%%%%%%%%%%%%%%%%%%%%%%%%%%%%%%%%%%%%%%%%%%%%%%%%%%%%%%%%%%%
%%%%%%%%%%%%%%%%%%%%%%%%%%%%%%%%%%%%%%%%%%%%%%%%%%%%%%%%%%%%%%%%%%%%%%%%%%%%%%%%
\section{Information}

%%%%%%%%%%%%%%%%%%%%%%%%%%%%%%%%%%%%%%%%%%%%%%%%%%%%%%%%%%%%%%%%%%%%%%%%%%%%%%%%
\subsection{Copyright}

Copyright \copyright{} 2017--2018 Niklas Beisert

This work may be distributed and/or modified under the
conditions of the \LaTeX{} Project Public License, either version 1.3
of this license or (at your option) any later version.
The latest version of this license is in
  \url{http://www.latex-project.org/lppl.txt}
and version 1.3 or later is part of all distributions of \LaTeX{}
version 2005/12/01 or later.

This work has the LPPL maintenance status `maintained'.

The Current Maintainer of this work is Niklas Beisert.

This work consists of the files |README.txt|, |childdoc.ins| and |childdoc.dtx|
as well as the derived files |childdoc.def|, |cdocsamp.tex|
with |cdocsch1.tex|, |cdocsch2.tex|, |cdocspt3.tex|, |cdocspt4.tex|,
|cdocsdrf.tex|, |cdocsfn1.tex|, |cdocsfn2.tex|
as well as |childdoc.pdf|.

%%%%%%%%%%%%%%%%%%%%%%%%%%%%%%%%%%%%%%%%%%%%%%%%%%%%%%%%%%%%%%%%%%%%%%%%%%%%%%%%
\subsection{Files and Installation}

The package consists of the files:
%
\begin{center}
\begin{tabular}{ll}
    |README.txt|   & readme file \\
    |childdoc.ins| & installation file \\
    |childdoc.dtx| & source file \\
    |childdoc.def| & definition file \\
    |cdocsamp.tex| & sample main file \\
    |cdocsch1.tex| & sample include file \\
    |cdocsch2.tex| & sample include file \\
    |cdocspt3.tex| & sample part file \\
    |cdocspt4.tex| & sample part file \\
    |cdocsdrf.tex| & sample redirection file \\
    |cdocsfn1.tex| & sample redirection file \\
    |cdocsfn2.tex| & sample redirection file \\
    |childdoc.pdf| & manual
\end{tabular}
\end{center}
%
The distribution consists of the files
|README.txt|, |childdoc.ins| and |childdoc.dtx|.
%
\begin{itemize}
\item
Run (pdf)\LaTeX{} on |childdoc.dtx|
to compile the manual |childdoc.pdf| (this file).
\item
Run \LaTeX{} on |childdoc.ins| to create the definitions file |childdoc.def|
and the sample |cdocsamp.tex| with include files
|cdocsch1.tex|, |cdocsch2.tex|, |cdocspt3.tex|, |cdocspt4.tex|,
|cdocsdrf.tex|, |cdocsfn1.tex|, |cdocsfn2.tex|.
Then copy the file |childdoc.def| to an appropriate directory of your \LaTeX{}
distribution, e.g.\ \textit{texmf-root}|/tex/latex/childdoc|.
\end{itemize}

%%%%%%%%%%%%%%%%%%%%%%%%%%%%%%%%%%%%%%%%%%%%%%%%%%%%%%%%%%%%%%%%%%%%%%%%%%%%%%%%
\subsection{Related CTAN Packages}

There are several other packages which offer a similar functionality:
%
\begin{itemize}
\item
The packages
\href{http://ctan.org/pkg/docmute}{\textsf{docmute}},
\href{http://ctan.org/pkg/includex}{\textsf{includex}} and
\href{http://ctan.org/pkg/standalone}{\textsf{standalone}}
provide commands to include only the document body of
a child file thus allowing both files to be compiled individually.
\item
The packages \href{http://ctan.org/pkg/subdocs}{\textsf{subdocs}}
and \href{http://ctan.org/pkg/subfiles}{\textsf{subfiles}}
provide structures in which the main and child documents can be
encapsulated and allowing them to be compiled individually.
The inclusion mechanism is different from the conventional |\include|.
\item
The package \href{http://ctan.org/pkg/combine}{\textsf{combine}}
is an elaborate solution to combine several documents into one.
\end{itemize}
%
See also the CTAN topic \href{http://ctan.org/topic/subdocs}{\textsf{subdocs}}
for further related packages.
The present package differs from the above solutions in that
a document structure constructed with the conventional |\include| mechanism
just needs two extra commands at the top of every file
such that all constituent files can be compiled individually.

%%%%%%%%%%%%%%%%%%%%%%%%%%%%%%%%%%%%%%%%%%%%%%%%%%%%%%%%%%%%%%%%%%%%%%%%%%%%%%%%
%\subsection{Feature Suggestions}
%
%The following is a list of features which may be useful for future
%versions of this package:
%%
%\begin{itemize}
%\item
%\ldots
%\end{itemize}

%%%%%%%%%%%%%%%%%%%%%%%%%%%%%%%%%%%%%%%%%%%%%%%%%%%%%%%%%%%%%%%%%%%%%%%%%%%%%%%%
\subsection{Revision History}

%%%%%%%%%%%%%%%%%%%%%%%%%%%%%%%%%%%%%%%%
\paragraph{v2.0:} 2018/12/30

\begin{itemize}
\item
immediate forward processing
\item
added |\childdocby| mechanism
\item
manual restructured
\end{itemize}

%%%%%%%%%%%%%%%%%%%%%%%%%%%%%%%%%%%%%%%%
\paragraph{v1.6:} 2018/01/17

\begin{itemize}
\item
application for development of include files
\item
corrections to manual
\end{itemize}

%%%%%%%%%%%%%%%%%%%%%%%%%%%%%%%%%%%%%%%%
\paragraph{v1.5:} 2017/05/21

\begin{itemize}
\item
more complete structuring introduced
\item
|\childdocof| introduced
\item
|\childdoc| renamed to |\childdocmain|
\item
|\childredirect| renamed to |\childdocforward| and |\childdocforwardprefix|
and functionality expanded
\end{itemize}

%%%%%%%%%%%%%%%%%%%%%%%%%%%%%%%%%%%%%%%%
\paragraph{v1.0:} 2017/04/27

\begin{itemize}
\item
manual and install package
\item
first version published on CTAN
\end{itemize}

%%%%%%%%%%%%%%%%%%%%%%%%%%%%%%%%%%%%%%%%
\paragraph{v0.6:} 2017/04/26

\begin{itemize}
\item
redirection mechanism added
\end{itemize}

%%%%%%%%%%%%%%%%%%%%%%%%%%%%%%%%%%%%%%%%
\paragraph{v0.5:} 2017/04/26

\begin{itemize}
\item
functionality in definition file
\end{itemize}


%%%%%%%%%%%%%%%%%%%%%%%%%%%%%%%%%%%%%%%%%%%%%%%%%%%%%%%%%%%%%%%%%%%%%%%%%%%%%%%%
%%%%%%%%%%%%%%%%%%%%%%%%%%%%%%%%%%%%%%%%%%%%%%%%%%%%%%%%%%%%%%%%%%%%%%%%%%%%%%%%
%%%%%%%%%%%%%%%%%%%%%%%%%%%%%%%%%%%%%%%%%%%%%%%%%%%%%%%%%%%%%%%%%%%%%%%%%%%%%%%%
\appendix

\settowidth\MacroIndent{\rmfamily\scriptsize 000\ }

 \DocInput{childdoc.dtx}

\end{document}
%</driver>
% \fi
%
% %%%%%%%%%%%%%%%%%%%%%%%%%%%%%%%%%%%%%%%%%%%%%%%%%%%%%%%%%%%%%%%%%%%%%%%%%%%%%%
% %%%%%%%%%%%%%%%%%%%%%%%%%%%%%%%%%%%%%%%%%%%%%%%%%%%%%%%%%%%%%%%%%%%%%%%%%%%%%%
% \section{Sample}
%\iffalse
%<*samplemain>
%\fi
%
% The following presents a sample document
% with two chapters, two parts, a title page,
% a compile flag as well as three forwarding files to set the flag.
% It consists of eight |.tex| files:
% \begin{center}
% \begin{tabular}{ll}
% |cdocsamp.tex|&main file\\
% |cdocsch1.tex|&include file for chapter 1\\
% |cdocsch2.tex|&include file for chapter 2\\
% |cdocspt3.tex|&include file for part 3\\
% |cdocspt4.tex|&include file for part 4\\
% |cdocsdrf.tex|&forwarding file for main file in draft mode\\
% |cdocsfi1.tex|&forwarding file for final version of chapter 1\\
% |cdocsfi2.tex|&forwarding file for final version of chapter 2\\
% \end{tabular}
% \end{center}
% Each of the eight files can be compiled directly by the \LaTeX{} compiler.
%
% %%%%%%%%%%%%%%%%%%%%%%%%%%%%%%%%%%%%%%
% \paragraph{Main File.}
%
% The main file is called |cdocsamp.tex|.
%
% Load the \textsf{childdoc} definitions and
% declare the filename for the main document:
%    \begin{macrocode}
% \iffalse
%
% childdoc.dtx Copyright (C) 2017-2018 Niklas Beisert
%
% This work may be distributed and/or modified under the
% conditions of the LaTeX Project Public License, either version 1.3
% of this license or (at your option) any later version.
% The latest version of this license is in
%   http://www.latex-project.org/lppl.txt
% and version 1.3 or later is part of all distributions of LaTeX
% version 2005/12/01 or later.
%
% This work has the LPPL maintenance status `maintained'.
%
% The Current Maintainer of this work is Niklas Beisert.
%
% This work consists of the files childdoc.dtx and childdoc.ins
% and the derived files childdoc.def and cdocsamp.tex with
% cdocsch1.tex, cdocsch2.tex, cdocsdrf.tex, cdocsfn1.tex, cdocsfn2.tex.
%
%<package>\ifdefined\childdocmain\endinput\fi
%<package>\ProvidesFile{childdoc.def}[2018/12/30 v2.0 child document driver]
%<samplemain>\ProvidesFile{cdocsamp.tex}[2018/12/30 v2.0 sample for childdoc]
%<*driver>
%\ProvidesFile{childdoc.drv}[2018/12/30 v2.0 childdoc reference manual file]
\PassOptionsToClass{10pt,a4paper}{article}
\documentclass{ltxdoc}

\usepackage[margin=35mm]{geometry}
\usepackage{hyperref}
\usepackage{hyperxmp}
\usepackage[usenames]{color}

\hypersetup{colorlinks=true}
\hypersetup{pdfstartview=FitH}
\hypersetup{pdfpagemode=UseNone}
\hypersetup{pdfsource={}}
\hypersetup{pdflang={en-UK}}
\hypersetup{pdfcopyright={Copyright 2017-2018 Niklas Beisert.
  This work may be distributed and/or modified under the
  conditions of the LaTeX Project Public License, either version 1.3
  of this license or (at your option) any later version.}}
\hypersetup{pdflicenseurl={http://www.latex-project.org/lppl.txt}}
\hypersetup{pdfcontactaddress={ETH Zurich, ITP, HIT K,
  Wolfgang-Pauli-Strasse 27}}
\hypersetup{pdfcontactpostcode={8093}}
\hypersetup{pdfcontactcity={Zurich}}
\hypersetup{pdfcontactcountry={Switzerland}}
\hypersetup{pdfcontactemail={nbeisert@itp.phys.ethz.ch}}
\hypersetup{pdfcontacturl={http://people.phys.ethz.ch/\xmptilde nbeisert/}}

\newcommand{\secref}[1]{\hyperref[#1]{section \ref*{#1}}}

\parskip1ex
\parindent0pt
\let\olditemize\itemize
\def\itemize{\olditemize\parskip0pt}

\begin{document}

\title{The \textsf{childdoc} Package}
\hypersetup{pdftitle={The childdoc Package}}
\author{Niklas Beisert\\[2ex]
  Institut f\"ur Theoretische Physik\\
  Eidgen\"ossische Technische Hochschule Z\"urich\\
  Wolfgang-Pauli-Strasse 27, 8093 Z\"urich, Switzerland\\[1ex]
  \href{mailto:nbeisert@itp.phys.ethz.ch}
  {\texttt{nbeisert@itp.phys.ethz.ch}}}
\hypersetup{pdfauthor={Niklas Beisert}}
\hypersetup{pdfsubject={Manual for the LaTeX2e Package childdoc}}
\date{30 December 2018, \textsf{v2.0}}
\maketitle

\begin{abstract}\noindent
\textsf{childdoc} is a \LaTeXe{} package
that enables the direct compilation
of document sections included by |\include|
to individual files.
\end{abstract}

\begingroup
\parskip0ex
\tableofcontents
\endgroup

%%%%%%%%%%%%%%%%%%%%%%%%%%%%%%%%%%%%%%%%%%%%%%%%%%%%%%%%%%%%%%%%%%%%%%%%%%%%%%%%
%%%%%%%%%%%%%%%%%%%%%%%%%%%%%%%%%%%%%%%%%%%%%%%%%%%%%%%%%%%%%%%%%%%%%%%%%%%%%%%%
\section{Introduction}

\LaTeX{} provides a mechanism to structure a large document (such as a book)
into a main file and several child files (containing the chapters)
using the |\include| command.
This mechanism is beneficial for documents
which span hundreds of pages in order to
make the source file(s) more manageable.
Moreover, compilation can be restricted to
selected child files by means of the |\includeonly| command.
The latter feature can be used to reduce the compilation time while editing
(this was significantly more useful in the earlier days of \LaTeX{})
or to generate a smaller document which is easier to navigate.
Another application of |\includeonly| is to generate
documents consisting of selected parts of the complete document.

However, there are a few drawbacks of the plain |\include| mechanism:
\begin{itemize}
\item
The child files cannot be compiled on their own,
they can only be compiled via the main file.
A naive editing environment
(such as a text editor with an option
to have the current file processed by \LaTeX)
may require one to switch to the main file before compiling;
attempting to compile the child file produces errors.
\item
The main file must be modified (each time)
to adjust the |\includeonly| command
to the present needs. This easily leaves the main file in a messy state.
\item
The generated document will always carry the filename
of the main document. This is inconvenient if
several child files are to be compiled and
to be kept for distribution.
\end{itemize}

The present package provides a simple interface
to make child files individually compilable by \LaTeX{}.
Compiling a child file then has the same effect as compiling
the main file with an |\includeonly| command
to select the appropriate child.
Moreover the generated document will carry the name of the child
rather than the main file.
This resolves all three above issues.

This feature is meant to make the editing of books,
thesis documents and lecture notes somewhat more convenient.
However, the package can also be used efficiently for
composing a series of documents (such as exercise sheets)
which are typically distributed individually.
It then assists the author in generating the individual documents
(potentially in different versions)
as well as a document containing the collected series.
Another application is in developing style files
or other kinds of included material
where compilation of the style file could redirect
to a sample or test file.

%%%%%%%%%%%%%%%%%%%%%%%%%%%%%%%%%%%%%%%%%%%%%%%%%%%%%%%%%%%%%%%%%%%%%%%%%%%%%%%%
%%%%%%%%%%%%%%%%%%%%%%%%%%%%%%%%%%%%%%%%%%%%%%%%%%%%%%%%%%%%%%%%%%%%%%%%%%%%%%%%
\section{Usage}

First of all, the package \textsf{childdoc} is \emph{not} a standard
\LaTeXe{} |.sty| style file! Therefore it needs to be invoked in
a non-standard way.

%%%%%%%%%%%%%%%%%%%%%%%%%%%%%%%%%%%%%%%%%%%%%%%%%%%%%%%%%%%%%%%%%%%%%%%%%%%%%%%%
\subsection{Included Files}
\label{sec:include}

%%%%%%%%%%%%%%%%%%%%%%%%%%%%%%%%%%%%%%%%
\DescribeMacro{\childdocmain}
To use the package, add the commands
\begin{center}
\begin{tabular}{l}
|\input{childdoc.def}|\\
|\childdocmain{}|\\
\end{tabular}
\end{center}
at the very top of the main \LaTeX{} file,
in particular \emph{before} the |\documentclass| statement!
The argument of |\childdocmain| should be left empty
(but it must be present).

%%%%%%%%%%%%%%%%%%%%%%%%%%%%%%%%%%%%%%%%
\DescribeMacro{\childdocof}
Furthermore, add the commands
\begin{center}
\begin{tabular}{l}
|\input{childdoc.def}|\\
|\childdocof{|\textit{main}|}|\\
\end{tabular}
\end{center}
at the top of every child file \textit{child}
which is included by |\include{|\textit{child}|}|
from within the main file
(or at least for those files to be compiled individually).
The argument \textit{main} must be the filename of the main file.

There are a couple of
considerations in setting up the main and child documents:

%%%%%%%%%%%%%%%%%%%%%%%%%%%%%%%%%%%%%%%%
\paragraph{Restrictions.}

Please note the following restrictions:
\begin{itemize}
\item
|\childdocmain| must be called with one argument \textit{main}
to ensure compatibility with earlier version of the package.
It must either be empty (|\childdocmain{}|)
or precisely match the filename of the main file in which it is specified.
See \secref{sec:detection} for further information.
\item
The filename \textit{main} must be specified without the |.tex| extension.
\item
The filename \textit{main} is case sensitive
(even in case-insensitive file systems)
due to internal string comparison.
\item
The argument \textit{main} should be fully expanded, it cannot be a macro.
\item
Subdirectories and special characters should be avoided in filenames.
\item
The command |\childdocmain{|\textit{main}|}| must be followed by a whitespace.
It should not be followed immediately by another command
or by a comment mark `|%|'.
This is because the \TeX{} parser reads the token immediately following
the argument of |\childdocmain| and puts it
at the beginning of every child section;
however, a white\-space is ignored.
\end{itemize}

%%%%%%%%%%%%%%%%%%%%%%%%%%%%%%%%%%%%%%%%
\paragraph{Content of Main File.}

It is advisable to place all content in the child files included by |\include|.
Any output contained in the main file will appear in all child documents
unless suppressed manually;
it cannot be suppressed automatically by the |\includeonly| directive
and thus should normally be avoided.
A method to include some content in the main file
by means of conditional processing is described in \secref{sec:conditional}.

%%%%%%%%%%%%%%%%%%%%%%%%%%%%%%%%%%%%%%%%
\paragraph{Page Numbering.}

When only a part of the document is compiled,
the appropriate numbering of pages
(as well as other status parameters)
is determined from the |.aux| files.
The latter contain information from previous passes.
However this information needs to propagate through
all intermediate child documents.
Therefore the page numbering in child documents may well
be inconsistent until the complete document is compiled at least once.

A useful (if unconventional) way to always ensure a consistent
page numbering is to restart the numbering in each child document
and denote the pages by `\textit{child}|.|\textit{page}'
where \textit{child} represents the chapter/section number of the child file.
This can be achieved by the command
|\numberwithin{page}{|\textit{child}|}|
of the \textsf{amsmath} package
where \textit{child} can be |chapter| or |section|
depending on the chosen structuring.
Alternatively, one can modify the macro |\thepage| appropriately
and reset the counter |page| at the start of each child file.

%%%%%%%%%%%%%%%%%%%%%%%%%%%%%%%%%%%%%%%%%%%%%%%%%%%%%%%%%%%%%%%%%%%%%%%%%%%%%%%%
\subsection{Conditional Processing}
\label{sec:conditional}

The package provides a mechanism to compile different versions
of a document. To customise the versions further some conditional processing
can come in handy to distinguish which version is being compiled.
The package provides two macros to describe the compilation context:

%%%%%%%%%%%%%%%%%%%%%%%%%%%%%%%%%%%%%%%%
\DescribeMacro{\ifchilddoc}
The conditional |\ifchilddoc| distinguishes between the compilation of
child documents and the main document:
%
\begin{center}
|\ifchilddoc |\textit{child-code}| |[|\||else |\textit{main-code}]| \||fi|
\end{center}

%%%%%%%%%%%%%%%%%%%%%%%%%%%%%%%%%%%%%%%%
\DescribeMacro{\childdocname}
\DescribeMacro{\childdocjob}
The macro |\childdocname| contains the filename (without extension)
of the main or child file being processed.
Note that |\childdocjob| will always contain the name of the main file.

%%%%%%%%%%%%%%%%%%%%%%%%%%%%%%%%%%%%%%%%
\paragraph{Title Page.}

Conditional processing can be used to include a title or banner page
in the main document when proper precautions are taken.
Importantly, the code in the main file should ensure that the page counter
(as well as other status parameters which are stored in the |.aux| files)
takes the same value after the conditional processing.
Otherwise the page numbers may take divergent values
depending on which part is compiled.

For example, a title page could be declared by:
%
\begin{center}
\begin{tabular}{l}
|\ifchilddoc\||else|\\
|\addtocounter{page}{-1}|\\
\textit{code for title page}\\
|\newpage|\\
|\||fi|
\end{tabular}
\end{center}
%
A banner page for the child documents can be generated by:
%
\begin{center}
\begin{tabular}{l}
|\ifchilddoc|\\
|\addtocounter{page}{-1}|\\
\textit{code for banner page}\\
|\newpage|\\
|\||fi|
\end{tabular}
\end{center}
%
Here one could write a message such as:
\begin{center}
|This is the part \childdocname{} of \childdocjob{}.|
\end{center}

%%%%%%%%%%%%%%%%%%%%%%%%%%%%%%%%%%%%%%%%%%%%%%%%%%%%%%%%%%%%%%%%%%%%%%%%%%%%%%%%
\subsection{Flags}
\label{sec:flags}

The package makes it easy to generate different versions
of the main or child documents.
To this end compilation flags can be defined
and assigned different default values.
They will be particularly useful in conjunction
with the forwarding mechanism described in \secref{sec:forward}.

For example, it may be useful to have a flag |\version|
which can be set to |draft| or |final|.
The document source will contain some conditional code
depending on the value of |\version|.
Suppose further, the flag should default to |final| for the main file
and to |draft| for child files
which is a natural assignment for editing the document.
This is achieved by placing the following code
in the preamble of the main document
(below the |\childdocmain| directive):
%
\begin{center}
\begin{tabular}{l}
|\ifchilddoc|\\
|\providecommand{\version}{draft}|\\
|\||else|\\
|\providecommand{\version}{final}|\\
|\||fi|
\end{tabular}
\end{center}
%
The definition by |\providecommand| makes sure
that previous definitions are not overwritten.
Further statements |\providecommand{\version}{...}|
can thus be added before the above code to override it.

For the main file, one might add a line
(between |\childdocmain| and the above block)
%
\begin{center}
|%\ifchilddoc\||else\providecommand{\version}{draft}\||fi|
\end{center}
%
which can be uncommented to produce a draft version.
Likewise one can add a line to the very top of a child file
(above the |\childdocof{|\textit{main}|}| directive)
%
\begin{center}
|%\providecommand{\version}{final}|
\end{center}
%
which can be uncommented to produce the final version of this child document.

%%%%%%%%%%%%%%%%%%%%%%%%%%%%%%%%%%%%%%%%%%%%%%%%%%%%%%%%%%%%%%%%%%%%%%%%%%%%%%%%
\subsection{Forwarding}
\label{sec:forward}

Different versions of the main or child documents
using compilation flags as described in \secref{sec:flags}
can be (permanently) stored in different files
for convenient compilation, viewing and distribution.
To this end, the package defines a command
to pass on compilation to a different file:

%%%%%%%%%%%%%%%%%%%%%%%%%%%%%%%%%%%%%%%%
\DescribeMacro{\childdocforward}
The command |\childdocforward| redirects processing to
another source file:
%
\begin{center}
\begin{tabular}{l}
|\input{childdoc.def}|\\
|\childdocforward[|\textit{main}|]{|\textit{dest}|}|\\
\end{tabular}
\end{center}
%
The argument \textit{dest} is the destination file
(without extension).
It should be the main file or one of the child files.
Note that further \textsf{childdoc} directives
such as |\childdocof| and |\childdocforward|
in the indicated file will be processed in this form.
The optional argument \textit{main}
passes on directly to the main file \textit{main}
while pretending to compile the child \textit{dest}.
This form behaves as if \textit{dest}
issues |\childdocof{|\textit{main}|}| right away,
and no further \textsf{childdoc} directives will be processed.

%%%%%%%%%%%%%%%%%%%%%%%%%%%%%%%%%%%%%%%%
\DescribeMacro{\...prefix}
In the alternative form |\childdocforwardprefix|,
%
\begin{center}
\begin{tabular}{l}
|\input{childdoc.def}|\\
|\childdocforwardprefix[|\textit{main}|]{|\textit{prefix}|}{|\textit{dest}|}|
\end{tabular}
\end{center}
%
the destination file is determined by a pattern
depending on the current file:
To make this work, the current file must be called
`{\textit{prefix}\hspace{0.2em}\textit{suffix}}'
with \textit{prefix} matching precisely the argument.
Processing is then passed on to the file
`{\textit{dest}\hspace{0.2em}\textit{suffix}}'.
Surely, the same effect is achieved by
directly specifying the
argument `{\textit{dest}\hspace{0.2em}\textit{suffix}}'
in the first form.
However, that requires to set up a different file
for each child. With the alternative form of the command
all these files can have exactly the same content
which simplifies setting them up and maintaining them.

For example, the following file |draft.tex|
with a compilation flag |\version| as described in \secref{sec:flags}
compiles the main document as a draft:
%
\begin{center}
\begin{tabular}{l}
|\def\version{draft}|\\
|\input{childdoc.def}|\\
|\childdocforward{|\textit{main}|}|
\end{tabular}
\end{center}
%
Likewise, the following files |final|\textit{nn}|.tex|
compile the final version of the child document
|child|\textit{nn}|.tex|:
%
\begin{center}
\begin{tabular}{l}
|\def\version{final}|\\
|\input{childdoc.def}|\\
|\childdocforwardprefix{final}{child}|
\end{tabular}
\end{center}
%

Note that when several versions of a main file and/or of each child file
are to be generated, it may be convenient to set up a |Makefile| or
shell script to automatise the process.

%%%%%%%%%%%%%%%%%%%%%%%%%%%%%%%%%%%%%%%%%%%%%%%%%%%%%%%%%%%%%%%%%%%%%%%%%%%%%%%%
\subsection{Command Line Processing}
\label{sec:commandline}

The effect of redirection files can also be achieved by invoking
the \LaTeX{} compiler with a more elaborate command line.
Most conveniently this should be done as part
of a shell script or a |Makefile|.

When using \textsf{childdoc} in the main file, the following
command lines effectively perform a redirection
(note that depending on the shell being used,
backslashes may have to be doubled: `|\|' $\to$ `|\\|'):
%
\begin{center}
|... -jobname "|\textit{target}|" |\\|"|[\textit{flags}]%
|\input{childdoc.def}\childdocforward[|\textit{main}|]{|\textit{dest}|}"|
\end{center}
%
Here \textit{target} is the name of the output file,
\textit{main} is the name of the main file
and \textit{dest} is the name of the main or child file to be processed
(all filenames without extensions).
The optional argument \textit{main} can be omitted
if \textit{main} matches \textit{dest}.
Optionally, compilation \textit{flags} can be defined via |\def| commands.
This command line makes the \TeX{} engine believe
it is compiling the file \textit{target}
whose content is specified as the latter parameter.
The provided code then forwards the processing to
\textit{main} or \textit{dest} as described in \secref{sec:forward}.

%%%%%%%%%%%%%%%%%%%%%%%%%%%%%%%%%%%%%%%%%%%%%%%%%%%%%%%%%%%%%%%%%%%%%%%%%%%%%%%%
\subsection{Include by Input}
\label{sec:input}

Including child documents by |\include| has some restrictions by design.
Most notably, the content of a child document always occupies
its own set of pages; pages cannot be shared between child documents.
Usually, this behaviour makes perfect sense
because each child document contain an essential part of the document.
However, in some situations it may be desirable to compose
a document from a collection of parts
without having mandatory page breaks between then.
For this case, the package
provides a mechanism to include parts
by |\input| which can also be processed individually.
However, by construction this mechanism
requires manual handling of the content to be output.

%%%%%%%%%%%%%%%%%%%%%%%%%%%%%%%%%%%%%%%%
\DescribeMacro{\ifchilddocmanual}
The main file should be prepared as usual, see \secref{sec:include}.
However, the document body must make a distinction
between processing of an individual part and of the main document, e.g.:
%
\begin{center}
\begin{tabular}{l}
|\ifchilddocmanual|\\
|\input{\childdocname}|\\
|\||else|\\
\textit{document body with }|\input{|\textit{part}|}|\\
|\||fi|
\end{tabular}
\end{center}
%
The conditional |\ifchilddocmanual| is true whenever
a part to be included by |\input| is being compiled,
and the name of the part is stored in |\childdocname|.

%%%%%%%%%%%%%%%%%%%%%%%%%%%%%%%%%%%%%%%%
\DescribeMacro{\childdocby}
Each part to be included by |\input| should start with:
%
\begin{center}
\begin{tabular}{l}
|\input{childdoc.def}|\\
|\childdocby{|\textit{main}|}|\\
\end{tabular}
\end{center}
%
The directive |\childdocby| is similar to |\childdocof|
described in \secref{sec:include},
but the subsequent selection of content must be done manually.
To that end, both |\ifchilddoc| and |\ifchilddocmanual|
will be true upon processing of a part,
and the name of the part is stored in |\childdocname|.
Note that |\jobname| will be set to the filename of the current part
so that each part receives an individual |.aux| file
that does not interfere with the |.aux| file(s) of the main document.
This behaviour can be altered by the alternative form
|\childdocby[*]{|\textit{main}|}| (with a non-empty optional argument)
which uses the |.aux| file of the main document
by setting |\jobname| to \textit{main}.

%%%%%%%%%%%%%%%%%%%%%%%%%%%%%%%%%%%%%%%%%%%%%%%%%%%%%%%%%%%%%%%%%%%%%%%%%%%%%%%%
\subsection{Driver Development}
\label{sec:driver}

The \textsf{childdoc} mechanism can also be use for the development
of definition files such as \LaTeX{} styles or classes.
This case differs from the above setup with multiple parts
included by |\include| in that no |\includeonly| should be invoked.
This can be achieved by starting the include file
(before |\ProvidesPackage|) with:
%
\begin{center}
\begin{tabular}{l}
|\input{childdoc.def}|\\
|\childdocforward{|\textit{main}|}|\\
\end{tabular}
\end{center}
%
or alternatively with:
%
\begin{center}
\begin{tabular}{l}
|\input{childdoc.def}|\\
|\childdocby{|\textit{main}|}|\\
\end{tabular}
\end{center}
%
Both forms have slightly different effects as described above.
The main file is prepared as usual, see \secref{sec:include}.

%%%%%%%%%%%%%%%%%%%%%%%%%%%%%%%%%%%%%%%%%%%%%%%%%%%%%%%%%%%%%%%%%%%%%%%%%%%%%%%%
\subsection{Legacy Detection}
\label{sec:detection}

The directive |\childdocmain| in the main file can detect
whether the complete document or merely a child is to be compiled
even without using the directive |\childdocof|.
This method is deprecated because it is less robust
and there is no compelling reason to use it;
it is merely provided for backward compatibility
and it may be removed in future versions.

If the detection mechanism is to be used,
it is mandatory to correctly specify
the filename of the main file as the argument of |\childdocmain|:
%
\begin{center}
\begin{tabular}{l}
|\input{childdoc.def}|\\
|\childdocmain{|\textit{main}|}|\\
\end{tabular}
\end{center}
%
If |\jobname| does not match the argument \textit{main} of |\childdocmain|,
it is assumed that |\jobname| points to the child file to be compiled.
When using |\childdocmain| with the main file specified as argument,
it suffices to start a child file
with just |\input{|\textit{main}|}|
without loading of the package and using |\childdocof|.
If instead all processing is done
with the appropriate \textsf{childdoc} directives,
the argument of \textit{main} of |\childdocmain| can be empty.

An alternative version of the command line processing described
in \secref{sec:commandline} using the detection mechanism reads:
%
\begin{center}
|... -jobname "|\textit{target}|" "|[\textit{flags}]%
[|\def\jobname{|\textit{dest}|}|]|\input{|\textit{main}|}"|
\end{center}

%%%%%%%%%%%%%%%%%%%%%%%%%%%%%%%%%%%%%%%%%%%%%%%%%%%%%%%%%%%%%%%%%%%%%%%%%%%%%%%%
\subsection{Manual Code}
\label{sec:manual}

In case one cannot be certain whether the definitions file |childdoc.def|
is installed on the target \TeX{} distribution
and one prefers not to ship it,
it is conceivable to paste a few relevant commands into the sources.

To that end, drop all statements |\input{childdoc.def}|
and perform the replacements as outlined below.
Instead of |\childdocmain{|\textit{main}|}| add the following code
to the top of the main file:
%
\begin{center}
\begin{tabular}{l}
|\||ifdefined\childdocname\endinput\||fi\newif\ifchilddoc|\\
|\edef\childdocname{\scantokens\expandafter{\jobname\noexpand}}|\\
|\def\childdocmain{|\textit{main}|}\||ifx\childdocmain\childdocname\||else|\\
|\childdoctrue\includeonly{\childdocname}\let\jobname\childdocmain\||fi|\\
\end{tabular}
\end{center}
%
Instead of |\childdocof{|\textit{main}|}| just include the main file
at the top of each child file:
%
\begin{center}
|\input{|\textit{main}|}|
\end{center}
%
A simple redirection |\childdocforward{|\textit{dest}|}| is achieved by:
%
\begin{center}
|\def\jobname{|\textit{dest}|}\input{\jobname}|
\end{center}
%
The redirection with prefix
|\childdocforwardprefix[|\textit{prefix}|]{|\textit{dest}|}|
is accomplished by:
%
\begin{center}
\begin{tabular}{l}
|{\edef\jobname{\scantokens\expandafter{\jobname\noexpand}}|\\
|\def\redirectjob |\textit{prefix}|#1~~~{\gdef\jobname{|\textit{dest}|#1}}|\\
|\expandafter\redirectjob\jobname~~~}\input{\jobname}|
\end{tabular}
\end{center}

In an alternative approach,
child documents can be compiled by a specific command line
without additional code or specific definitions:
%
\begin{center}
|... -jobname "|\textit{target}|" "|[\textit{flags}]%
|\includeonly{|\textit{dest}|}\input{|\textit{main}|}"|
\end{center}
%

%%%%%%%%%%%%%%%%%%%%%%%%%%%%%%%%%%%%%%%%%%%%%%%%%%%%%%%%%%%%%%%%%%%%%%%%%%%%%%%%
%%%%%%%%%%%%%%%%%%%%%%%%%%%%%%%%%%%%%%%%%%%%%%%%%%%%%%%%%%%%%%%%%%%%%%%%%%%%%%%%
\section{Information}

%%%%%%%%%%%%%%%%%%%%%%%%%%%%%%%%%%%%%%%%%%%%%%%%%%%%%%%%%%%%%%%%%%%%%%%%%%%%%%%%
\subsection{Copyright}

Copyright \copyright{} 2017--2018 Niklas Beisert

This work may be distributed and/or modified under the
conditions of the \LaTeX{} Project Public License, either version 1.3
of this license or (at your option) any later version.
The latest version of this license is in
  \url{http://www.latex-project.org/lppl.txt}
and version 1.3 or later is part of all distributions of \LaTeX{}
version 2005/12/01 or later.

This work has the LPPL maintenance status `maintained'.

The Current Maintainer of this work is Niklas Beisert.

This work consists of the files |README.txt|, |childdoc.ins| and |childdoc.dtx|
as well as the derived files |childdoc.def|, |cdocsamp.tex|
with |cdocsch1.tex|, |cdocsch2.tex|, |cdocspt3.tex|, |cdocspt4.tex|,
|cdocsdrf.tex|, |cdocsfn1.tex|, |cdocsfn2.tex|
as well as |childdoc.pdf|.

%%%%%%%%%%%%%%%%%%%%%%%%%%%%%%%%%%%%%%%%%%%%%%%%%%%%%%%%%%%%%%%%%%%%%%%%%%%%%%%%
\subsection{Files and Installation}

The package consists of the files:
%
\begin{center}
\begin{tabular}{ll}
    |README.txt|   & readme file \\
    |childdoc.ins| & installation file \\
    |childdoc.dtx| & source file \\
    |childdoc.def| & definition file \\
    |cdocsamp.tex| & sample main file \\
    |cdocsch1.tex| & sample include file \\
    |cdocsch2.tex| & sample include file \\
    |cdocspt3.tex| & sample part file \\
    |cdocspt4.tex| & sample part file \\
    |cdocsdrf.tex| & sample redirection file \\
    |cdocsfn1.tex| & sample redirection file \\
    |cdocsfn2.tex| & sample redirection file \\
    |childdoc.pdf| & manual
\end{tabular}
\end{center}
%
The distribution consists of the files
|README.txt|, |childdoc.ins| and |childdoc.dtx|.
%
\begin{itemize}
\item
Run (pdf)\LaTeX{} on |childdoc.dtx|
to compile the manual |childdoc.pdf| (this file).
\item
Run \LaTeX{} on |childdoc.ins| to create the definitions file |childdoc.def|
and the sample |cdocsamp.tex| with include files
|cdocsch1.tex|, |cdocsch2.tex|, |cdocspt3.tex|, |cdocspt4.tex|,
|cdocsdrf.tex|, |cdocsfn1.tex|, |cdocsfn2.tex|.
Then copy the file |childdoc.def| to an appropriate directory of your \LaTeX{}
distribution, e.g.\ \textit{texmf-root}|/tex/latex/childdoc|.
\end{itemize}

%%%%%%%%%%%%%%%%%%%%%%%%%%%%%%%%%%%%%%%%%%%%%%%%%%%%%%%%%%%%%%%%%%%%%%%%%%%%%%%%
\subsection{Related CTAN Packages}

There are several other packages which offer a similar functionality:
%
\begin{itemize}
\item
The packages
\href{http://ctan.org/pkg/docmute}{\textsf{docmute}},
\href{http://ctan.org/pkg/includex}{\textsf{includex}} and
\href{http://ctan.org/pkg/standalone}{\textsf{standalone}}
provide commands to include only the document body of
a child file thus allowing both files to be compiled individually.
\item
The packages \href{http://ctan.org/pkg/subdocs}{\textsf{subdocs}}
and \href{http://ctan.org/pkg/subfiles}{\textsf{subfiles}}
provide structures in which the main and child documents can be
encapsulated and allowing them to be compiled individually.
The inclusion mechanism is different from the conventional |\include|.
\item
The package \href{http://ctan.org/pkg/combine}{\textsf{combine}}
is an elaborate solution to combine several documents into one.
\end{itemize}
%
See also the CTAN topic \href{http://ctan.org/topic/subdocs}{\textsf{subdocs}}
for further related packages.
The present package differs from the above solutions in that
a document structure constructed with the conventional |\include| mechanism
just needs two extra commands at the top of every file
such that all constituent files can be compiled individually.

%%%%%%%%%%%%%%%%%%%%%%%%%%%%%%%%%%%%%%%%%%%%%%%%%%%%%%%%%%%%%%%%%%%%%%%%%%%%%%%%
%\subsection{Feature Suggestions}
%
%The following is a list of features which may be useful for future
%versions of this package:
%%
%\begin{itemize}
%\item
%\ldots
%\end{itemize}

%%%%%%%%%%%%%%%%%%%%%%%%%%%%%%%%%%%%%%%%%%%%%%%%%%%%%%%%%%%%%%%%%%%%%%%%%%%%%%%%
\subsection{Revision History}

%%%%%%%%%%%%%%%%%%%%%%%%%%%%%%%%%%%%%%%%
\paragraph{v2.0:} 2018/12/30

\begin{itemize}
\item
immediate forward processing
\item
added |\childdocby| mechanism
\item
manual restructured
\end{itemize}

%%%%%%%%%%%%%%%%%%%%%%%%%%%%%%%%%%%%%%%%
\paragraph{v1.6:} 2018/01/17

\begin{itemize}
\item
application for development of include files
\item
corrections to manual
\end{itemize}

%%%%%%%%%%%%%%%%%%%%%%%%%%%%%%%%%%%%%%%%
\paragraph{v1.5:} 2017/05/21

\begin{itemize}
\item
more complete structuring introduced
\item
|\childdocof| introduced
\item
|\childdoc| renamed to |\childdocmain|
\item
|\childredirect| renamed to |\childdocforward| and |\childdocforwardprefix|
and functionality expanded
\end{itemize}

%%%%%%%%%%%%%%%%%%%%%%%%%%%%%%%%%%%%%%%%
\paragraph{v1.0:} 2017/04/27

\begin{itemize}
\item
manual and install package
\item
first version published on CTAN
\end{itemize}

%%%%%%%%%%%%%%%%%%%%%%%%%%%%%%%%%%%%%%%%
\paragraph{v0.6:} 2017/04/26

\begin{itemize}
\item
redirection mechanism added
\end{itemize}

%%%%%%%%%%%%%%%%%%%%%%%%%%%%%%%%%%%%%%%%
\paragraph{v0.5:} 2017/04/26

\begin{itemize}
\item
functionality in definition file
\end{itemize}


%%%%%%%%%%%%%%%%%%%%%%%%%%%%%%%%%%%%%%%%%%%%%%%%%%%%%%%%%%%%%%%%%%%%%%%%%%%%%%%%
%%%%%%%%%%%%%%%%%%%%%%%%%%%%%%%%%%%%%%%%%%%%%%%%%%%%%%%%%%%%%%%%%%%%%%%%%%%%%%%%
%%%%%%%%%%%%%%%%%%%%%%%%%%%%%%%%%%%%%%%%%%%%%%%%%%%%%%%%%%%%%%%%%%%%%%%%%%%%%%%%
\appendix

\settowidth\MacroIndent{\rmfamily\scriptsize 000\ }

 \DocInput{childdoc.dtx}

\end{document}
%</driver>
% \fi
%
% %%%%%%%%%%%%%%%%%%%%%%%%%%%%%%%%%%%%%%%%%%%%%%%%%%%%%%%%%%%%%%%%%%%%%%%%%%%%%%
% %%%%%%%%%%%%%%%%%%%%%%%%%%%%%%%%%%%%%%%%%%%%%%%%%%%%%%%%%%%%%%%%%%%%%%%%%%%%%%
% \section{Sample}
%\iffalse
%<*samplemain>
%\fi
%
% The following presents a sample document
% with two chapters, two parts, a title page,
% a compile flag as well as three forwarding files to set the flag.
% It consists of eight |.tex| files:
% \begin{center}
% \begin{tabular}{ll}
% |cdocsamp.tex|&main file\\
% |cdocsch1.tex|&include file for chapter 1\\
% |cdocsch2.tex|&include file for chapter 2\\
% |cdocspt3.tex|&include file for part 3\\
% |cdocspt4.tex|&include file for part 4\\
% |cdocsdrf.tex|&forwarding file for main file in draft mode\\
% |cdocsfi1.tex|&forwarding file for final version of chapter 1\\
% |cdocsfi2.tex|&forwarding file for final version of chapter 2\\
% \end{tabular}
% \end{center}
% Each of the eight files can be compiled directly by the \LaTeX{} compiler.
%
% %%%%%%%%%%%%%%%%%%%%%%%%%%%%%%%%%%%%%%
% \paragraph{Main File.}
%
% The main file is called |cdocsamp.tex|.
%
% Load the \textsf{childdoc} definitions and
% declare the filename for the main document:
%    \begin{macrocode}
\input{childdoc.def}
\childdocmain{}
%    \end{macrocode}

% Optional override for |\version| flag:
%    \begin{macrocode}
%%\ifchilddoc\else\providecommand{\version}{draft}\fi
%    \end{macrocode}

% Define the default values for the |\version| flag
% (|final| for the main file and |draft| for childs):
%    \begin{macrocode}
\ifchilddoc
\providecommand{\version}{draft}
\else
\providecommand{\version}{final}
\fi
%    \end{macrocode}

% Load the standard document class:
%    \begin{macrocode}
\documentclass[12pt]{article}
%    \end{macrocode}

% Start the document body:
%    \begin{macrocode}
\begin{document}
%    \end{macrocode}

% Declare a title page.
% Print title, part of document being processed and version flag:
%    \begin{macrocode}
\addtocounter{page}{-1}
\begin{center}
{\LARGE\bfseries{}childdoc example\par}
\vspace{1cm}
\ifchilddoc
\ifchilddocmanual part\else chapter\fi:
`\childdocname' of `\childdocjob'\par
\else
main document: `\childdocjob'\par
\fi
version: \version\par
\end{center}
\newpage
%    \end{macrocode}

% Manually include selected file,
% otherwise process as usual:
%    \begin{macrocode}
\ifchilddocmanual
\section*{part `\childdocname'}
\input{\childdocname}
\else
%    \end{macrocode}

% Include the two chapters:
%    \begin{macrocode}
\include{cdocsch1}
\include{cdocsch2}
%    \end{macrocode}

% Include the two parts unless only chapters should be displayed:
%    \begin{macrocode}
\ifchilddoc\else
\section{part three}
\input{cdocspt3}
\section{part four}
\input{cdocspt4}
\fi
%    \end{macrocode}

% Process as usual until here:
%    \begin{macrocode}
\fi
%    \end{macrocode}

% End of document body:
%    \begin{macrocode}
\end{document}
%    \end{macrocode}
%\iffalse
%</samplemain>
%\fi
%
% %%%%%%%%%%%%%%%%%%%%%%%%%%%%%%%%%%%%%%
% \paragraph{Chapter Include Files.}
%
% The include files are called |cdocsch1.tex| and |cdocsch2.tex|.
%
%\iffalse
%<*samplechap1|samplechap2>
%\fi

% Optional override for |\version| flag:
%    \begin{macrocode}
%%\providecommand{\version}{final}
%    \end{macrocode}

% Include the main document:
%    \begin{macrocode}
\input{childdoc.def}
\childdocof{cdocsamp}
%    \end{macrocode}

%\iffalse
%</samplechap1|samplechap2>
%\fi
%
%\iffalse
%<*samplechap1>
%\fi
% Some text for chapter 1:
%    \begin{macrocode}
\section{one}
some text in chapter one
%    \end{macrocode}

%\iffalse
%</samplechap1>
%\fi
% Some text for chapter 2:
%\iffalse
%<*samplechap2>
%\fi
%    \begin{macrocode}
\section{two}
more text in chapter two
%    \end{macrocode}

%\iffalse
%</samplechap2>
%\fi
%
% %%%%%%%%%%%%%%%%%%%%%%%%%%%%%%%%%%%%%%
% \paragraph{Part Include Files.}
%
% The include files are called |cdocspt3.tex| and |cdocspt4.tex|.
%
%\iffalse
%<*samplepart3|samplepart4>
%\fi

% Optional override for |\version| flag:
%    \begin{macrocode}
%%\providecommand{\version}{final}
%    \end{macrocode}

% Include the main document:
%    \begin{macrocode}
\input{childdoc.def}
\childdocby{cdocsamp}
%    \end{macrocode}

%\iffalse
%</samplepart3|samplepart4>
%\fi
%
%\iffalse
%<*samplepart3>
%\fi
% Some text for part 3:
%    \begin{macrocode}
some text in part three
%    \end{macrocode}

%\iffalse
%</samplepart3>
%\fi
% Some text for part 4:
%\iffalse
%<*samplepart4>
%\fi
%    \begin{macrocode}
more text in part four
%    \end{macrocode}

%\iffalse
%</samplepart4>
%\fi
%
% %%%%%%%%%%%%%%%%%%%%%%%%%%%%%%%%%%%%%%
% \paragraph{Forwarding for a Complete Draft.}
%
% The following forwarding file |cdocsdrf.tex|
% compiles the main document in draft mode:
%\iffalse
%<*sampledraft>
%\fi
%    \begin{macrocode}
\def\version{draft}
\input{childdoc.def}
\childdocforward{cdocsamp}
%    \end{macrocode}

%\iffalse
%</sampledraft>
%\fi
%
% %%%%%%%%%%%%%%%%%%%%%%%%%%%%%%%%%%%%%%
% \paragraph{Forwarding for Final Version of the Chapters.}
%
% The following forwarding files |cdocsfn1.tex| and |cdocsfn2.tex|
% (with identical content)
% compile the final versions of the child documents
% |cdocsch1.tex| and |cdocsch2.tex|, respectively:
%\iffalse
%<*samplefinal>
%\fi
%    \begin{macrocode}
\def\version{final}
\input{childdoc.def}
\childdocforwardprefix[cdocsamp]{cdocsfn}{cdocsch}
%    \end{macrocode}

%\iffalse
%</samplefinal>
%\fi
%
% %%%%%%%%%%%%%%%%%%%%%%%%%%%%%%%%%%%%%%
% \paragraph{Command Line Processing.}
%
% The following three command lines generate the output files
% |cdocscld|, |cdocscl1| and |cdocscl2|
% which should be identical to
% |cdocsdrf|, |cdocsch1| and |cdocsfn2|, respectively:
% \begin{center}
% \begin{tabular}{l}
% |latex -jobname cdocscld \|\\
% |  "\def\version{draft}\input{childdoc.def}\childdocforward{cdocsamp}"|\\
% |latex -jobname cdocscl1 \|\\
% |  "\input{childdoc.def}\childdocforward[cdocsamp]{cdocsch1}"|\\
% |latex -jobname cdocscl2 \|\\
% |  "\def\version{final}\input{childdoc.def}\childdocforward{cdocsch2}"|
% \end{tabular}
% \end{center}
% Note that the trailing backslash on each first line
% merely continues the input to the second line
% (for convenient cut ant paste).
% Furthermore, the command |latex| can be replaced by any
% of its alternative versions such as |pdflatex|.
%
% %%%%%%%%%%%%%%%%%%%%%%%%%%%%%%%%%%%%%%%%%%%%%%%%%%%%%%%%%%%%%%%%%%%%%%%%%%%%%%
% %%%%%%%%%%%%%%%%%%%%%%%%%%%%%%%%%%%%%%%%%%%%%%%%%%%%%%%%%%%%%%%%%%%%%%%%%%%%%%
% \section{Implementation}
%\iffalse
%<*package>
%\fi
%
% This section describes the definitions file |childdoc.def|.

% The definitions cannot be loaded using |\usepackage| or |\RequirePackage|
% which has a mechanism to prevent loading a style file more than once.
% When loading the definitions by means of |\input|
% multiple instances have to be prevented manually:
%\iffalse
%This code needs to be before the `\ProvidesFile' directive
%which is defined at the beginning of this file.
%Therefore it is also placed there and commented out here.
%</package>
%<*discard>
%\fi
%    \begin{macrocode}
\ifdefined\childdocmain\endinput\fi
%    \end{macrocode}
%\iffalse
%</discard>
%<*package>
%\fi
%
% \macro{\ifchilddoc}
% \macro{\ifchilddocmanual}
% The conditional |\ifchilddoc| tells whether a
% child (true) or main (false) document is being compiled.
% The conditional |\ifchilddocmanual| tells whether
% the |\includeonly| mechanism is used (false) or
% the selection of child files must be performed manually (true).
% The definitions initialise to false:
%    \begin{macrocode}
\newif\ifchilddoc
\newif\ifchilddocmanual
%    \end{macrocode}

% \macro{\childdocname}
% \macro{\childdocjob}
% The macro |\childdocname| stores the name of the main document
% to be compiled. The macro |\childdocjob| stores the name of
% the document on which the \LaTeX{} compiler was originally invoked.
% The content of |\jobname| cannot be compared
% to filenames specified in the source due to different catcodes.
% The following code rescans |\jobname|, stores the result
% in |\childdocname| and saves a copy in |\childdocjob|:
%    \begin{macrocode}
\edef\childdocname{\scantokens\expandafter{\jobname\noexpand}}
\let\childdocjob\childdocname
%    \end{macrocode}

% \macro{\childdocdisable}
% The macro |\childdocdisable| prevents the main file
% from being processed more than once.
% At this stage, the main document command |\childdocmain|
% is assumed to be called once again where it should do nothing.
% Any subsequent call to it should prevent
% a secondary processing of the main document
% It overwrites the forwarding commands
% |\childdocof| and |\childdocforward|
% with empty macros to prevent further inclusions of the main document:
%    \begin{macrocode}
\newcommand{\childdocdisable}
{
  \renewcommand{\childdocmain}[1]{\renewcommand{\childdocmain}[1]{\endinput}}
  \renewcommand{\childdocof}[1]{}
  \renewcommand{\childdocby}[2][]{}
  \renewcommand{\childdocforward}[2][]{}
  \renewcommand{\childdocdisable}{}
}
%    \end{macrocode}

% \macro{\childdocmain}
% The macro |\childdocmain| is to be called at the top of the main file
% with nothing or the main filename (without extension) as argument.
% First, it breaks loops.
% If the argument is not empty and does not match |\childdocname|
% (which is set by the first inclusion of |childdoc.def|),
% |\ifchilddoc| is set to true, |\includeonly| is applied to the child file
% and |\jobname| is set to the main file
% (for proper handling of |.aux| files):
%    \begin{macrocode}
\newcommand{\childdocmain}[1]
{
  \childdocdisable\childdocmain{}
  \if?#1?\else
    \begingroup
      \def\childdoctmp{#1}
      \ifx\childdoctmp\childdocname
        \def\childdoctmp{}
      \else
        \def\childdoctmp
        {
          \childdoctrue
          \includeonly{\childdocname}
          \def\childdocjob{#1}
          \def\jobname{#1}
        }
      \fi
      \expandafter
    \endgroup
    \childdoctmp
  \fi
}
%    \end{macrocode}

% \macro{\childdocof}
% The command |\childdocof| redirects
% compilation to the main file |#1|.
%    \begin{macrocode}
\newcommand{\childdocof}[1]
{
  \childdocdisable
  \childdoctrue
  \includeonly{\childdocname}
  \def\jobname{#1}
  \def\childdocjob{#1}
  \input{#1}
}
%    \end{macrocode}

% \macro{\childdocby}
% The command |\childdocby| ....
%    \begin{macrocode}
\newcommand{\childdocby}[2][]
{
  \childdocdisable
  \childdoctrue
  \childdocmanualtrue
  \if?#1?\else
    \def\jobname{#2}
  \fi
  \def\childdocjob{#2}
  \input{#2}
  \endinput
}
%    \end{macrocode}

% \macro{\childdocforward}
% The command |\childdocforward| redirects
% compilation to the main file or
% (if the optional argument is given) a child file.
% Parameters are set as if the main file
% or a child file starting with |\childdocof| was compiled.
% Then compilation is handed over to the main file:
%    \begin{macrocode}
\newcommand{\childdocforward}[2][]
{
  \begingroup
    \if?#1?
      \def\childdoctmp
      {
        \def\childdocname{#2}
        \def\childdocjob{#2}
        \def\jobname{#2}
        \input{#2}
        \endinput
      }
    \else
      \def\childdoctmp
      {
        \childdocdisable
        \def\childdocname{#2}
        \childdoctrue
        \includeonly{#2}
        \def\childdocjob{#1}
        \def\jobname{#1}
        \input{#1}
        \endinput
      }
    \fi
    \expandafter
  \endgroup
  \childdoctmp
}
%    \end{macrocode}

% \macro{\childdocforwardprefix}
% The command |\childdocforwardprefix| redirects
% compilation to the main or a child file by means of a pattern.
% The prefix |#1| in the current filename is replaced by |#2|
% and the suffix of the current filename is kept
% (it is assumed that the filename does not contain the substring `|~~~|'
% which is used as a delimiter).
% Compilation is handed over to the new file by |\childdocforward|:
%    \begin{macrocode}
\newcommand{\childdocforwardprefix}[3][]
{
  \begingroup
    \def\childdocextract #2##1~~~{\def\childdoctmp{\childdocforward[#1]{#3##1}}}
    \expandafter\childdocextract\childdocname~~~
    \expandafter
  \endgroup
  \childdoctmp
}
%    \end{macrocode}

% \macro{\childdoc}
% The deprecated macro |\childdoc| is a legacy version of |\childdocmain|:
%    \begin{macrocode}
\newcommand{\childdoc}{\childdocmain}
%    \end{macrocode}

% \macro{\childdocredirect}
% The deprecated macro |\childdocredirect| is a legacy version
% of |\childdocforward| and |\childdocforwardprefix|:
%    \begin{macrocode}
\newcommand{\childdocredirect}[2][]
{
  \begingroup
    \if?#1?
      \def\childdoctmp{\childdocforward{#2}}
    \else
      \def\childdoctmp{\childdocforwardprefix{#1}{#2}}
    \fi
    \expandafter
  \endgroup
  \childdoctmp
}
%    \end{macrocode}

%\iffalse
%</package>
%\fi
%
\endinput

\childdocmain{}
%    \end{macrocode}

% Optional override for |\version| flag:
%    \begin{macrocode}
%%\ifchilddoc\else\providecommand{\version}{draft}\fi
%    \end{macrocode}

% Define the default values for the |\version| flag
% (|final| for the main file and |draft| for childs):
%    \begin{macrocode}
\ifchilddoc
\providecommand{\version}{draft}
\else
\providecommand{\version}{final}
\fi
%    \end{macrocode}

% Load the standard document class:
%    \begin{macrocode}
\documentclass[12pt]{article}
%    \end{macrocode}

% Start the document body:
%    \begin{macrocode}
\begin{document}
%    \end{macrocode}

% Declare a title page.
% Print title, part of document being processed and version flag:
%    \begin{macrocode}
\addtocounter{page}{-1}
\begin{center}
{\LARGE\bfseries{}childdoc example\par}
\vspace{1cm}
\ifchilddoc
\ifchilddocmanual part\else chapter\fi:
`\childdocname' of `\childdocjob'\par
\else
main document: `\childdocjob'\par
\fi
version: \version\par
\end{center}
\newpage
%    \end{macrocode}

% Manually include selected file,
% otherwise process as usual:
%    \begin{macrocode}
\ifchilddocmanual
\section*{part `\childdocname'}
\input{\childdocname}
\else
%    \end{macrocode}

% Include the two chapters:
%    \begin{macrocode}
\include{cdocsch1}
\include{cdocsch2}
%    \end{macrocode}

% Include the two parts unless only chapters should be displayed:
%    \begin{macrocode}
\ifchilddoc\else
\section{part three}
\input{cdocspt3}
\section{part four}
\input{cdocspt4}
\fi
%    \end{macrocode}

% Process as usual until here:
%    \begin{macrocode}
\fi
%    \end{macrocode}

% End of document body:
%    \begin{macrocode}
\end{document}
%    \end{macrocode}
%\iffalse
%</samplemain>
%\fi
%
% %%%%%%%%%%%%%%%%%%%%%%%%%%%%%%%%%%%%%%
% \paragraph{Chapter Include Files.}
%
% The include files are called |cdocsch1.tex| and |cdocsch2.tex|.
%
%\iffalse
%<*samplechap1|samplechap2>
%\fi

% Optional override for |\version| flag:
%    \begin{macrocode}
%%\providecommand{\version}{final}
%    \end{macrocode}

% Include the main document:
%    \begin{macrocode}
% \iffalse
%
% childdoc.dtx Copyright (C) 2017-2018 Niklas Beisert
%
% This work may be distributed and/or modified under the
% conditions of the LaTeX Project Public License, either version 1.3
% of this license or (at your option) any later version.
% The latest version of this license is in
%   http://www.latex-project.org/lppl.txt
% and version 1.3 or later is part of all distributions of LaTeX
% version 2005/12/01 or later.
%
% This work has the LPPL maintenance status `maintained'.
%
% The Current Maintainer of this work is Niklas Beisert.
%
% This work consists of the files childdoc.dtx and childdoc.ins
% and the derived files childdoc.def and cdocsamp.tex with
% cdocsch1.tex, cdocsch2.tex, cdocsdrf.tex, cdocsfn1.tex, cdocsfn2.tex.
%
%<package>\ifdefined\childdocmain\endinput\fi
%<package>\ProvidesFile{childdoc.def}[2018/12/30 v2.0 child document driver]
%<samplemain>\ProvidesFile{cdocsamp.tex}[2018/12/30 v2.0 sample for childdoc]
%<*driver>
%\ProvidesFile{childdoc.drv}[2018/12/30 v2.0 childdoc reference manual file]
\PassOptionsToClass{10pt,a4paper}{article}
\documentclass{ltxdoc}

\usepackage[margin=35mm]{geometry}
\usepackage{hyperref}
\usepackage{hyperxmp}
\usepackage[usenames]{color}

\hypersetup{colorlinks=true}
\hypersetup{pdfstartview=FitH}
\hypersetup{pdfpagemode=UseNone}
\hypersetup{pdfsource={}}
\hypersetup{pdflang={en-UK}}
\hypersetup{pdfcopyright={Copyright 2017-2018 Niklas Beisert.
  This work may be distributed and/or modified under the
  conditions of the LaTeX Project Public License, either version 1.3
  of this license or (at your option) any later version.}}
\hypersetup{pdflicenseurl={http://www.latex-project.org/lppl.txt}}
\hypersetup{pdfcontactaddress={ETH Zurich, ITP, HIT K,
  Wolfgang-Pauli-Strasse 27}}
\hypersetup{pdfcontactpostcode={8093}}
\hypersetup{pdfcontactcity={Zurich}}
\hypersetup{pdfcontactcountry={Switzerland}}
\hypersetup{pdfcontactemail={nbeisert@itp.phys.ethz.ch}}
\hypersetup{pdfcontacturl={http://people.phys.ethz.ch/\xmptilde nbeisert/}}

\newcommand{\secref}[1]{\hyperref[#1]{section \ref*{#1}}}

\parskip1ex
\parindent0pt
\let\olditemize\itemize
\def\itemize{\olditemize\parskip0pt}

\begin{document}

\title{The \textsf{childdoc} Package}
\hypersetup{pdftitle={The childdoc Package}}
\author{Niklas Beisert\\[2ex]
  Institut f\"ur Theoretische Physik\\
  Eidgen\"ossische Technische Hochschule Z\"urich\\
  Wolfgang-Pauli-Strasse 27, 8093 Z\"urich, Switzerland\\[1ex]
  \href{mailto:nbeisert@itp.phys.ethz.ch}
  {\texttt{nbeisert@itp.phys.ethz.ch}}}
\hypersetup{pdfauthor={Niklas Beisert}}
\hypersetup{pdfsubject={Manual for the LaTeX2e Package childdoc}}
\date{30 December 2018, \textsf{v2.0}}
\maketitle

\begin{abstract}\noindent
\textsf{childdoc} is a \LaTeXe{} package
that enables the direct compilation
of document sections included by |\include|
to individual files.
\end{abstract}

\begingroup
\parskip0ex
\tableofcontents
\endgroup

%%%%%%%%%%%%%%%%%%%%%%%%%%%%%%%%%%%%%%%%%%%%%%%%%%%%%%%%%%%%%%%%%%%%%%%%%%%%%%%%
%%%%%%%%%%%%%%%%%%%%%%%%%%%%%%%%%%%%%%%%%%%%%%%%%%%%%%%%%%%%%%%%%%%%%%%%%%%%%%%%
\section{Introduction}

\LaTeX{} provides a mechanism to structure a large document (such as a book)
into a main file and several child files (containing the chapters)
using the |\include| command.
This mechanism is beneficial for documents
which span hundreds of pages in order to
make the source file(s) more manageable.
Moreover, compilation can be restricted to
selected child files by means of the |\includeonly| command.
The latter feature can be used to reduce the compilation time while editing
(this was significantly more useful in the earlier days of \LaTeX{})
or to generate a smaller document which is easier to navigate.
Another application of |\includeonly| is to generate
documents consisting of selected parts of the complete document.

However, there are a few drawbacks of the plain |\include| mechanism:
\begin{itemize}
\item
The child files cannot be compiled on their own,
they can only be compiled via the main file.
A naive editing environment
(such as a text editor with an option
to have the current file processed by \LaTeX)
may require one to switch to the main file before compiling;
attempting to compile the child file produces errors.
\item
The main file must be modified (each time)
to adjust the |\includeonly| command
to the present needs. This easily leaves the main file in a messy state.
\item
The generated document will always carry the filename
of the main document. This is inconvenient if
several child files are to be compiled and
to be kept for distribution.
\end{itemize}

The present package provides a simple interface
to make child files individually compilable by \LaTeX{}.
Compiling a child file then has the same effect as compiling
the main file with an |\includeonly| command
to select the appropriate child.
Moreover the generated document will carry the name of the child
rather than the main file.
This resolves all three above issues.

This feature is meant to make the editing of books,
thesis documents and lecture notes somewhat more convenient.
However, the package can also be used efficiently for
composing a series of documents (such as exercise sheets)
which are typically distributed individually.
It then assists the author in generating the individual documents
(potentially in different versions)
as well as a document containing the collected series.
Another application is in developing style files
or other kinds of included material
where compilation of the style file could redirect
to a sample or test file.

%%%%%%%%%%%%%%%%%%%%%%%%%%%%%%%%%%%%%%%%%%%%%%%%%%%%%%%%%%%%%%%%%%%%%%%%%%%%%%%%
%%%%%%%%%%%%%%%%%%%%%%%%%%%%%%%%%%%%%%%%%%%%%%%%%%%%%%%%%%%%%%%%%%%%%%%%%%%%%%%%
\section{Usage}

First of all, the package \textsf{childdoc} is \emph{not} a standard
\LaTeXe{} |.sty| style file! Therefore it needs to be invoked in
a non-standard way.

%%%%%%%%%%%%%%%%%%%%%%%%%%%%%%%%%%%%%%%%%%%%%%%%%%%%%%%%%%%%%%%%%%%%%%%%%%%%%%%%
\subsection{Included Files}
\label{sec:include}

%%%%%%%%%%%%%%%%%%%%%%%%%%%%%%%%%%%%%%%%
\DescribeMacro{\childdocmain}
To use the package, add the commands
\begin{center}
\begin{tabular}{l}
|\input{childdoc.def}|\\
|\childdocmain{}|\\
\end{tabular}
\end{center}
at the very top of the main \LaTeX{} file,
in particular \emph{before} the |\documentclass| statement!
The argument of |\childdocmain| should be left empty
(but it must be present).

%%%%%%%%%%%%%%%%%%%%%%%%%%%%%%%%%%%%%%%%
\DescribeMacro{\childdocof}
Furthermore, add the commands
\begin{center}
\begin{tabular}{l}
|\input{childdoc.def}|\\
|\childdocof{|\textit{main}|}|\\
\end{tabular}
\end{center}
at the top of every child file \textit{child}
which is included by |\include{|\textit{child}|}|
from within the main file
(or at least for those files to be compiled individually).
The argument \textit{main} must be the filename of the main file.

There are a couple of
considerations in setting up the main and child documents:

%%%%%%%%%%%%%%%%%%%%%%%%%%%%%%%%%%%%%%%%
\paragraph{Restrictions.}

Please note the following restrictions:
\begin{itemize}
\item
|\childdocmain| must be called with one argument \textit{main}
to ensure compatibility with earlier version of the package.
It must either be empty (|\childdocmain{}|)
or precisely match the filename of the main file in which it is specified.
See \secref{sec:detection} for further information.
\item
The filename \textit{main} must be specified without the |.tex| extension.
\item
The filename \textit{main} is case sensitive
(even in case-insensitive file systems)
due to internal string comparison.
\item
The argument \textit{main} should be fully expanded, it cannot be a macro.
\item
Subdirectories and special characters should be avoided in filenames.
\item
The command |\childdocmain{|\textit{main}|}| must be followed by a whitespace.
It should not be followed immediately by another command
or by a comment mark `|%|'.
This is because the \TeX{} parser reads the token immediately following
the argument of |\childdocmain| and puts it
at the beginning of every child section;
however, a white\-space is ignored.
\end{itemize}

%%%%%%%%%%%%%%%%%%%%%%%%%%%%%%%%%%%%%%%%
\paragraph{Content of Main File.}

It is advisable to place all content in the child files included by |\include|.
Any output contained in the main file will appear in all child documents
unless suppressed manually;
it cannot be suppressed automatically by the |\includeonly| directive
and thus should normally be avoided.
A method to include some content in the main file
by means of conditional processing is described in \secref{sec:conditional}.

%%%%%%%%%%%%%%%%%%%%%%%%%%%%%%%%%%%%%%%%
\paragraph{Page Numbering.}

When only a part of the document is compiled,
the appropriate numbering of pages
(as well as other status parameters)
is determined from the |.aux| files.
The latter contain information from previous passes.
However this information needs to propagate through
all intermediate child documents.
Therefore the page numbering in child documents may well
be inconsistent until the complete document is compiled at least once.

A useful (if unconventional) way to always ensure a consistent
page numbering is to restart the numbering in each child document
and denote the pages by `\textit{child}|.|\textit{page}'
where \textit{child} represents the chapter/section number of the child file.
This can be achieved by the command
|\numberwithin{page}{|\textit{child}|}|
of the \textsf{amsmath} package
where \textit{child} can be |chapter| or |section|
depending on the chosen structuring.
Alternatively, one can modify the macro |\thepage| appropriately
and reset the counter |page| at the start of each child file.

%%%%%%%%%%%%%%%%%%%%%%%%%%%%%%%%%%%%%%%%%%%%%%%%%%%%%%%%%%%%%%%%%%%%%%%%%%%%%%%%
\subsection{Conditional Processing}
\label{sec:conditional}

The package provides a mechanism to compile different versions
of a document. To customise the versions further some conditional processing
can come in handy to distinguish which version is being compiled.
The package provides two macros to describe the compilation context:

%%%%%%%%%%%%%%%%%%%%%%%%%%%%%%%%%%%%%%%%
\DescribeMacro{\ifchilddoc}
The conditional |\ifchilddoc| distinguishes between the compilation of
child documents and the main document:
%
\begin{center}
|\ifchilddoc |\textit{child-code}| |[|\||else |\textit{main-code}]| \||fi|
\end{center}

%%%%%%%%%%%%%%%%%%%%%%%%%%%%%%%%%%%%%%%%
\DescribeMacro{\childdocname}
\DescribeMacro{\childdocjob}
The macro |\childdocname| contains the filename (without extension)
of the main or child file being processed.
Note that |\childdocjob| will always contain the name of the main file.

%%%%%%%%%%%%%%%%%%%%%%%%%%%%%%%%%%%%%%%%
\paragraph{Title Page.}

Conditional processing can be used to include a title or banner page
in the main document when proper precautions are taken.
Importantly, the code in the main file should ensure that the page counter
(as well as other status parameters which are stored in the |.aux| files)
takes the same value after the conditional processing.
Otherwise the page numbers may take divergent values
depending on which part is compiled.

For example, a title page could be declared by:
%
\begin{center}
\begin{tabular}{l}
|\ifchilddoc\||else|\\
|\addtocounter{page}{-1}|\\
\textit{code for title page}\\
|\newpage|\\
|\||fi|
\end{tabular}
\end{center}
%
A banner page for the child documents can be generated by:
%
\begin{center}
\begin{tabular}{l}
|\ifchilddoc|\\
|\addtocounter{page}{-1}|\\
\textit{code for banner page}\\
|\newpage|\\
|\||fi|
\end{tabular}
\end{center}
%
Here one could write a message such as:
\begin{center}
|This is the part \childdocname{} of \childdocjob{}.|
\end{center}

%%%%%%%%%%%%%%%%%%%%%%%%%%%%%%%%%%%%%%%%%%%%%%%%%%%%%%%%%%%%%%%%%%%%%%%%%%%%%%%%
\subsection{Flags}
\label{sec:flags}

The package makes it easy to generate different versions
of the main or child documents.
To this end compilation flags can be defined
and assigned different default values.
They will be particularly useful in conjunction
with the forwarding mechanism described in \secref{sec:forward}.

For example, it may be useful to have a flag |\version|
which can be set to |draft| or |final|.
The document source will contain some conditional code
depending on the value of |\version|.
Suppose further, the flag should default to |final| for the main file
and to |draft| for child files
which is a natural assignment for editing the document.
This is achieved by placing the following code
in the preamble of the main document
(below the |\childdocmain| directive):
%
\begin{center}
\begin{tabular}{l}
|\ifchilddoc|\\
|\providecommand{\version}{draft}|\\
|\||else|\\
|\providecommand{\version}{final}|\\
|\||fi|
\end{tabular}
\end{center}
%
The definition by |\providecommand| makes sure
that previous definitions are not overwritten.
Further statements |\providecommand{\version}{...}|
can thus be added before the above code to override it.

For the main file, one might add a line
(between |\childdocmain| and the above block)
%
\begin{center}
|%\ifchilddoc\||else\providecommand{\version}{draft}\||fi|
\end{center}
%
which can be uncommented to produce a draft version.
Likewise one can add a line to the very top of a child file
(above the |\childdocof{|\textit{main}|}| directive)
%
\begin{center}
|%\providecommand{\version}{final}|
\end{center}
%
which can be uncommented to produce the final version of this child document.

%%%%%%%%%%%%%%%%%%%%%%%%%%%%%%%%%%%%%%%%%%%%%%%%%%%%%%%%%%%%%%%%%%%%%%%%%%%%%%%%
\subsection{Forwarding}
\label{sec:forward}

Different versions of the main or child documents
using compilation flags as described in \secref{sec:flags}
can be (permanently) stored in different files
for convenient compilation, viewing and distribution.
To this end, the package defines a command
to pass on compilation to a different file:

%%%%%%%%%%%%%%%%%%%%%%%%%%%%%%%%%%%%%%%%
\DescribeMacro{\childdocforward}
The command |\childdocforward| redirects processing to
another source file:
%
\begin{center}
\begin{tabular}{l}
|\input{childdoc.def}|\\
|\childdocforward[|\textit{main}|]{|\textit{dest}|}|\\
\end{tabular}
\end{center}
%
The argument \textit{dest} is the destination file
(without extension).
It should be the main file or one of the child files.
Note that further \textsf{childdoc} directives
such as |\childdocof| and |\childdocforward|
in the indicated file will be processed in this form.
The optional argument \textit{main}
passes on directly to the main file \textit{main}
while pretending to compile the child \textit{dest}.
This form behaves as if \textit{dest}
issues |\childdocof{|\textit{main}|}| right away,
and no further \textsf{childdoc} directives will be processed.

%%%%%%%%%%%%%%%%%%%%%%%%%%%%%%%%%%%%%%%%
\DescribeMacro{\...prefix}
In the alternative form |\childdocforwardprefix|,
%
\begin{center}
\begin{tabular}{l}
|\input{childdoc.def}|\\
|\childdocforwardprefix[|\textit{main}|]{|\textit{prefix}|}{|\textit{dest}|}|
\end{tabular}
\end{center}
%
the destination file is determined by a pattern
depending on the current file:
To make this work, the current file must be called
`{\textit{prefix}\hspace{0.2em}\textit{suffix}}'
with \textit{prefix} matching precisely the argument.
Processing is then passed on to the file
`{\textit{dest}\hspace{0.2em}\textit{suffix}}'.
Surely, the same effect is achieved by
directly specifying the
argument `{\textit{dest}\hspace{0.2em}\textit{suffix}}'
in the first form.
However, that requires to set up a different file
for each child. With the alternative form of the command
all these files can have exactly the same content
which simplifies setting them up and maintaining them.

For example, the following file |draft.tex|
with a compilation flag |\version| as described in \secref{sec:flags}
compiles the main document as a draft:
%
\begin{center}
\begin{tabular}{l}
|\def\version{draft}|\\
|\input{childdoc.def}|\\
|\childdocforward{|\textit{main}|}|
\end{tabular}
\end{center}
%
Likewise, the following files |final|\textit{nn}|.tex|
compile the final version of the child document
|child|\textit{nn}|.tex|:
%
\begin{center}
\begin{tabular}{l}
|\def\version{final}|\\
|\input{childdoc.def}|\\
|\childdocforwardprefix{final}{child}|
\end{tabular}
\end{center}
%

Note that when several versions of a main file and/or of each child file
are to be generated, it may be convenient to set up a |Makefile| or
shell script to automatise the process.

%%%%%%%%%%%%%%%%%%%%%%%%%%%%%%%%%%%%%%%%%%%%%%%%%%%%%%%%%%%%%%%%%%%%%%%%%%%%%%%%
\subsection{Command Line Processing}
\label{sec:commandline}

The effect of redirection files can also be achieved by invoking
the \LaTeX{} compiler with a more elaborate command line.
Most conveniently this should be done as part
of a shell script or a |Makefile|.

When using \textsf{childdoc} in the main file, the following
command lines effectively perform a redirection
(note that depending on the shell being used,
backslashes may have to be doubled: `|\|' $\to$ `|\\|'):
%
\begin{center}
|... -jobname "|\textit{target}|" |\\|"|[\textit{flags}]%
|\input{childdoc.def}\childdocforward[|\textit{main}|]{|\textit{dest}|}"|
\end{center}
%
Here \textit{target} is the name of the output file,
\textit{main} is the name of the main file
and \textit{dest} is the name of the main or child file to be processed
(all filenames without extensions).
The optional argument \textit{main} can be omitted
if \textit{main} matches \textit{dest}.
Optionally, compilation \textit{flags} can be defined via |\def| commands.
This command line makes the \TeX{} engine believe
it is compiling the file \textit{target}
whose content is specified as the latter parameter.
The provided code then forwards the processing to
\textit{main} or \textit{dest} as described in \secref{sec:forward}.

%%%%%%%%%%%%%%%%%%%%%%%%%%%%%%%%%%%%%%%%%%%%%%%%%%%%%%%%%%%%%%%%%%%%%%%%%%%%%%%%
\subsection{Include by Input}
\label{sec:input}

Including child documents by |\include| has some restrictions by design.
Most notably, the content of a child document always occupies
its own set of pages; pages cannot be shared between child documents.
Usually, this behaviour makes perfect sense
because each child document contain an essential part of the document.
However, in some situations it may be desirable to compose
a document from a collection of parts
without having mandatory page breaks between then.
For this case, the package
provides a mechanism to include parts
by |\input| which can also be processed individually.
However, by construction this mechanism
requires manual handling of the content to be output.

%%%%%%%%%%%%%%%%%%%%%%%%%%%%%%%%%%%%%%%%
\DescribeMacro{\ifchilddocmanual}
The main file should be prepared as usual, see \secref{sec:include}.
However, the document body must make a distinction
between processing of an individual part and of the main document, e.g.:
%
\begin{center}
\begin{tabular}{l}
|\ifchilddocmanual|\\
|\input{\childdocname}|\\
|\||else|\\
\textit{document body with }|\input{|\textit{part}|}|\\
|\||fi|
\end{tabular}
\end{center}
%
The conditional |\ifchilddocmanual| is true whenever
a part to be included by |\input| is being compiled,
and the name of the part is stored in |\childdocname|.

%%%%%%%%%%%%%%%%%%%%%%%%%%%%%%%%%%%%%%%%
\DescribeMacro{\childdocby}
Each part to be included by |\input| should start with:
%
\begin{center}
\begin{tabular}{l}
|\input{childdoc.def}|\\
|\childdocby{|\textit{main}|}|\\
\end{tabular}
\end{center}
%
The directive |\childdocby| is similar to |\childdocof|
described in \secref{sec:include},
but the subsequent selection of content must be done manually.
To that end, both |\ifchilddoc| and |\ifchilddocmanual|
will be true upon processing of a part,
and the name of the part is stored in |\childdocname|.
Note that |\jobname| will be set to the filename of the current part
so that each part receives an individual |.aux| file
that does not interfere with the |.aux| file(s) of the main document.
This behaviour can be altered by the alternative form
|\childdocby[*]{|\textit{main}|}| (with a non-empty optional argument)
which uses the |.aux| file of the main document
by setting |\jobname| to \textit{main}.

%%%%%%%%%%%%%%%%%%%%%%%%%%%%%%%%%%%%%%%%%%%%%%%%%%%%%%%%%%%%%%%%%%%%%%%%%%%%%%%%
\subsection{Driver Development}
\label{sec:driver}

The \textsf{childdoc} mechanism can also be use for the development
of definition files such as \LaTeX{} styles or classes.
This case differs from the above setup with multiple parts
included by |\include| in that no |\includeonly| should be invoked.
This can be achieved by starting the include file
(before |\ProvidesPackage|) with:
%
\begin{center}
\begin{tabular}{l}
|\input{childdoc.def}|\\
|\childdocforward{|\textit{main}|}|\\
\end{tabular}
\end{center}
%
or alternatively with:
%
\begin{center}
\begin{tabular}{l}
|\input{childdoc.def}|\\
|\childdocby{|\textit{main}|}|\\
\end{tabular}
\end{center}
%
Both forms have slightly different effects as described above.
The main file is prepared as usual, see \secref{sec:include}.

%%%%%%%%%%%%%%%%%%%%%%%%%%%%%%%%%%%%%%%%%%%%%%%%%%%%%%%%%%%%%%%%%%%%%%%%%%%%%%%%
\subsection{Legacy Detection}
\label{sec:detection}

The directive |\childdocmain| in the main file can detect
whether the complete document or merely a child is to be compiled
even without using the directive |\childdocof|.
This method is deprecated because it is less robust
and there is no compelling reason to use it;
it is merely provided for backward compatibility
and it may be removed in future versions.

If the detection mechanism is to be used,
it is mandatory to correctly specify
the filename of the main file as the argument of |\childdocmain|:
%
\begin{center}
\begin{tabular}{l}
|\input{childdoc.def}|\\
|\childdocmain{|\textit{main}|}|\\
\end{tabular}
\end{center}
%
If |\jobname| does not match the argument \textit{main} of |\childdocmain|,
it is assumed that |\jobname| points to the child file to be compiled.
When using |\childdocmain| with the main file specified as argument,
it suffices to start a child file
with just |\input{|\textit{main}|}|
without loading of the package and using |\childdocof|.
If instead all processing is done
with the appropriate \textsf{childdoc} directives,
the argument of \textit{main} of |\childdocmain| can be empty.

An alternative version of the command line processing described
in \secref{sec:commandline} using the detection mechanism reads:
%
\begin{center}
|... -jobname "|\textit{target}|" "|[\textit{flags}]%
[|\def\jobname{|\textit{dest}|}|]|\input{|\textit{main}|}"|
\end{center}

%%%%%%%%%%%%%%%%%%%%%%%%%%%%%%%%%%%%%%%%%%%%%%%%%%%%%%%%%%%%%%%%%%%%%%%%%%%%%%%%
\subsection{Manual Code}
\label{sec:manual}

In case one cannot be certain whether the definitions file |childdoc.def|
is installed on the target \TeX{} distribution
and one prefers not to ship it,
it is conceivable to paste a few relevant commands into the sources.

To that end, drop all statements |\input{childdoc.def}|
and perform the replacements as outlined below.
Instead of |\childdocmain{|\textit{main}|}| add the following code
to the top of the main file:
%
\begin{center}
\begin{tabular}{l}
|\||ifdefined\childdocname\endinput\||fi\newif\ifchilddoc|\\
|\edef\childdocname{\scantokens\expandafter{\jobname\noexpand}}|\\
|\def\childdocmain{|\textit{main}|}\||ifx\childdocmain\childdocname\||else|\\
|\childdoctrue\includeonly{\childdocname}\let\jobname\childdocmain\||fi|\\
\end{tabular}
\end{center}
%
Instead of |\childdocof{|\textit{main}|}| just include the main file
at the top of each child file:
%
\begin{center}
|\input{|\textit{main}|}|
\end{center}
%
A simple redirection |\childdocforward{|\textit{dest}|}| is achieved by:
%
\begin{center}
|\def\jobname{|\textit{dest}|}\input{\jobname}|
\end{center}
%
The redirection with prefix
|\childdocforwardprefix[|\textit{prefix}|]{|\textit{dest}|}|
is accomplished by:
%
\begin{center}
\begin{tabular}{l}
|{\edef\jobname{\scantokens\expandafter{\jobname\noexpand}}|\\
|\def\redirectjob |\textit{prefix}|#1~~~{\gdef\jobname{|\textit{dest}|#1}}|\\
|\expandafter\redirectjob\jobname~~~}\input{\jobname}|
\end{tabular}
\end{center}

In an alternative approach,
child documents can be compiled by a specific command line
without additional code or specific definitions:
%
\begin{center}
|... -jobname "|\textit{target}|" "|[\textit{flags}]%
|\includeonly{|\textit{dest}|}\input{|\textit{main}|}"|
\end{center}
%

%%%%%%%%%%%%%%%%%%%%%%%%%%%%%%%%%%%%%%%%%%%%%%%%%%%%%%%%%%%%%%%%%%%%%%%%%%%%%%%%
%%%%%%%%%%%%%%%%%%%%%%%%%%%%%%%%%%%%%%%%%%%%%%%%%%%%%%%%%%%%%%%%%%%%%%%%%%%%%%%%
\section{Information}

%%%%%%%%%%%%%%%%%%%%%%%%%%%%%%%%%%%%%%%%%%%%%%%%%%%%%%%%%%%%%%%%%%%%%%%%%%%%%%%%
\subsection{Copyright}

Copyright \copyright{} 2017--2018 Niklas Beisert

This work may be distributed and/or modified under the
conditions of the \LaTeX{} Project Public License, either version 1.3
of this license or (at your option) any later version.
The latest version of this license is in
  \url{http://www.latex-project.org/lppl.txt}
and version 1.3 or later is part of all distributions of \LaTeX{}
version 2005/12/01 or later.

This work has the LPPL maintenance status `maintained'.

The Current Maintainer of this work is Niklas Beisert.

This work consists of the files |README.txt|, |childdoc.ins| and |childdoc.dtx|
as well as the derived files |childdoc.def|, |cdocsamp.tex|
with |cdocsch1.tex|, |cdocsch2.tex|, |cdocspt3.tex|, |cdocspt4.tex|,
|cdocsdrf.tex|, |cdocsfn1.tex|, |cdocsfn2.tex|
as well as |childdoc.pdf|.

%%%%%%%%%%%%%%%%%%%%%%%%%%%%%%%%%%%%%%%%%%%%%%%%%%%%%%%%%%%%%%%%%%%%%%%%%%%%%%%%
\subsection{Files and Installation}

The package consists of the files:
%
\begin{center}
\begin{tabular}{ll}
    |README.txt|   & readme file \\
    |childdoc.ins| & installation file \\
    |childdoc.dtx| & source file \\
    |childdoc.def| & definition file \\
    |cdocsamp.tex| & sample main file \\
    |cdocsch1.tex| & sample include file \\
    |cdocsch2.tex| & sample include file \\
    |cdocspt3.tex| & sample part file \\
    |cdocspt4.tex| & sample part file \\
    |cdocsdrf.tex| & sample redirection file \\
    |cdocsfn1.tex| & sample redirection file \\
    |cdocsfn2.tex| & sample redirection file \\
    |childdoc.pdf| & manual
\end{tabular}
\end{center}
%
The distribution consists of the files
|README.txt|, |childdoc.ins| and |childdoc.dtx|.
%
\begin{itemize}
\item
Run (pdf)\LaTeX{} on |childdoc.dtx|
to compile the manual |childdoc.pdf| (this file).
\item
Run \LaTeX{} on |childdoc.ins| to create the definitions file |childdoc.def|
and the sample |cdocsamp.tex| with include files
|cdocsch1.tex|, |cdocsch2.tex|, |cdocspt3.tex|, |cdocspt4.tex|,
|cdocsdrf.tex|, |cdocsfn1.tex|, |cdocsfn2.tex|.
Then copy the file |childdoc.def| to an appropriate directory of your \LaTeX{}
distribution, e.g.\ \textit{texmf-root}|/tex/latex/childdoc|.
\end{itemize}

%%%%%%%%%%%%%%%%%%%%%%%%%%%%%%%%%%%%%%%%%%%%%%%%%%%%%%%%%%%%%%%%%%%%%%%%%%%%%%%%
\subsection{Related CTAN Packages}

There are several other packages which offer a similar functionality:
%
\begin{itemize}
\item
The packages
\href{http://ctan.org/pkg/docmute}{\textsf{docmute}},
\href{http://ctan.org/pkg/includex}{\textsf{includex}} and
\href{http://ctan.org/pkg/standalone}{\textsf{standalone}}
provide commands to include only the document body of
a child file thus allowing both files to be compiled individually.
\item
The packages \href{http://ctan.org/pkg/subdocs}{\textsf{subdocs}}
and \href{http://ctan.org/pkg/subfiles}{\textsf{subfiles}}
provide structures in which the main and child documents can be
encapsulated and allowing them to be compiled individually.
The inclusion mechanism is different from the conventional |\include|.
\item
The package \href{http://ctan.org/pkg/combine}{\textsf{combine}}
is an elaborate solution to combine several documents into one.
\end{itemize}
%
See also the CTAN topic \href{http://ctan.org/topic/subdocs}{\textsf{subdocs}}
for further related packages.
The present package differs from the above solutions in that
a document structure constructed with the conventional |\include| mechanism
just needs two extra commands at the top of every file
such that all constituent files can be compiled individually.

%%%%%%%%%%%%%%%%%%%%%%%%%%%%%%%%%%%%%%%%%%%%%%%%%%%%%%%%%%%%%%%%%%%%%%%%%%%%%%%%
%\subsection{Feature Suggestions}
%
%The following is a list of features which may be useful for future
%versions of this package:
%%
%\begin{itemize}
%\item
%\ldots
%\end{itemize}

%%%%%%%%%%%%%%%%%%%%%%%%%%%%%%%%%%%%%%%%%%%%%%%%%%%%%%%%%%%%%%%%%%%%%%%%%%%%%%%%
\subsection{Revision History}

%%%%%%%%%%%%%%%%%%%%%%%%%%%%%%%%%%%%%%%%
\paragraph{v2.0:} 2018/12/30

\begin{itemize}
\item
immediate forward processing
\item
added |\childdocby| mechanism
\item
manual restructured
\end{itemize}

%%%%%%%%%%%%%%%%%%%%%%%%%%%%%%%%%%%%%%%%
\paragraph{v1.6:} 2018/01/17

\begin{itemize}
\item
application for development of include files
\item
corrections to manual
\end{itemize}

%%%%%%%%%%%%%%%%%%%%%%%%%%%%%%%%%%%%%%%%
\paragraph{v1.5:} 2017/05/21

\begin{itemize}
\item
more complete structuring introduced
\item
|\childdocof| introduced
\item
|\childdoc| renamed to |\childdocmain|
\item
|\childredirect| renamed to |\childdocforward| and |\childdocforwardprefix|
and functionality expanded
\end{itemize}

%%%%%%%%%%%%%%%%%%%%%%%%%%%%%%%%%%%%%%%%
\paragraph{v1.0:} 2017/04/27

\begin{itemize}
\item
manual and install package
\item
first version published on CTAN
\end{itemize}

%%%%%%%%%%%%%%%%%%%%%%%%%%%%%%%%%%%%%%%%
\paragraph{v0.6:} 2017/04/26

\begin{itemize}
\item
redirection mechanism added
\end{itemize}

%%%%%%%%%%%%%%%%%%%%%%%%%%%%%%%%%%%%%%%%
\paragraph{v0.5:} 2017/04/26

\begin{itemize}
\item
functionality in definition file
\end{itemize}


%%%%%%%%%%%%%%%%%%%%%%%%%%%%%%%%%%%%%%%%%%%%%%%%%%%%%%%%%%%%%%%%%%%%%%%%%%%%%%%%
%%%%%%%%%%%%%%%%%%%%%%%%%%%%%%%%%%%%%%%%%%%%%%%%%%%%%%%%%%%%%%%%%%%%%%%%%%%%%%%%
%%%%%%%%%%%%%%%%%%%%%%%%%%%%%%%%%%%%%%%%%%%%%%%%%%%%%%%%%%%%%%%%%%%%%%%%%%%%%%%%
\appendix

\settowidth\MacroIndent{\rmfamily\scriptsize 000\ }

 \DocInput{childdoc.dtx}

\end{document}
%</driver>
% \fi
%
% %%%%%%%%%%%%%%%%%%%%%%%%%%%%%%%%%%%%%%%%%%%%%%%%%%%%%%%%%%%%%%%%%%%%%%%%%%%%%%
% %%%%%%%%%%%%%%%%%%%%%%%%%%%%%%%%%%%%%%%%%%%%%%%%%%%%%%%%%%%%%%%%%%%%%%%%%%%%%%
% \section{Sample}
%\iffalse
%<*samplemain>
%\fi
%
% The following presents a sample document
% with two chapters, two parts, a title page,
% a compile flag as well as three forwarding files to set the flag.
% It consists of eight |.tex| files:
% \begin{center}
% \begin{tabular}{ll}
% |cdocsamp.tex|&main file\\
% |cdocsch1.tex|&include file for chapter 1\\
% |cdocsch2.tex|&include file for chapter 2\\
% |cdocspt3.tex|&include file for part 3\\
% |cdocspt4.tex|&include file for part 4\\
% |cdocsdrf.tex|&forwarding file for main file in draft mode\\
% |cdocsfi1.tex|&forwarding file for final version of chapter 1\\
% |cdocsfi2.tex|&forwarding file for final version of chapter 2\\
% \end{tabular}
% \end{center}
% Each of the eight files can be compiled directly by the \LaTeX{} compiler.
%
% %%%%%%%%%%%%%%%%%%%%%%%%%%%%%%%%%%%%%%
% \paragraph{Main File.}
%
% The main file is called |cdocsamp.tex|.
%
% Load the \textsf{childdoc} definitions and
% declare the filename for the main document:
%    \begin{macrocode}
\input{childdoc.def}
\childdocmain{}
%    \end{macrocode}

% Optional override for |\version| flag:
%    \begin{macrocode}
%%\ifchilddoc\else\providecommand{\version}{draft}\fi
%    \end{macrocode}

% Define the default values for the |\version| flag
% (|final| for the main file and |draft| for childs):
%    \begin{macrocode}
\ifchilddoc
\providecommand{\version}{draft}
\else
\providecommand{\version}{final}
\fi
%    \end{macrocode}

% Load the standard document class:
%    \begin{macrocode}
\documentclass[12pt]{article}
%    \end{macrocode}

% Start the document body:
%    \begin{macrocode}
\begin{document}
%    \end{macrocode}

% Declare a title page.
% Print title, part of document being processed and version flag:
%    \begin{macrocode}
\addtocounter{page}{-1}
\begin{center}
{\LARGE\bfseries{}childdoc example\par}
\vspace{1cm}
\ifchilddoc
\ifchilddocmanual part\else chapter\fi:
`\childdocname' of `\childdocjob'\par
\else
main document: `\childdocjob'\par
\fi
version: \version\par
\end{center}
\newpage
%    \end{macrocode}

% Manually include selected file,
% otherwise process as usual:
%    \begin{macrocode}
\ifchilddocmanual
\section*{part `\childdocname'}
\input{\childdocname}
\else
%    \end{macrocode}

% Include the two chapters:
%    \begin{macrocode}
\include{cdocsch1}
\include{cdocsch2}
%    \end{macrocode}

% Include the two parts unless only chapters should be displayed:
%    \begin{macrocode}
\ifchilddoc\else
\section{part three}
\input{cdocspt3}
\section{part four}
\input{cdocspt4}
\fi
%    \end{macrocode}

% Process as usual until here:
%    \begin{macrocode}
\fi
%    \end{macrocode}

% End of document body:
%    \begin{macrocode}
\end{document}
%    \end{macrocode}
%\iffalse
%</samplemain>
%\fi
%
% %%%%%%%%%%%%%%%%%%%%%%%%%%%%%%%%%%%%%%
% \paragraph{Chapter Include Files.}
%
% The include files are called |cdocsch1.tex| and |cdocsch2.tex|.
%
%\iffalse
%<*samplechap1|samplechap2>
%\fi

% Optional override for |\version| flag:
%    \begin{macrocode}
%%\providecommand{\version}{final}
%    \end{macrocode}

% Include the main document:
%    \begin{macrocode}
\input{childdoc.def}
\childdocof{cdocsamp}
%    \end{macrocode}

%\iffalse
%</samplechap1|samplechap2>
%\fi
%
%\iffalse
%<*samplechap1>
%\fi
% Some text for chapter 1:
%    \begin{macrocode}
\section{one}
some text in chapter one
%    \end{macrocode}

%\iffalse
%</samplechap1>
%\fi
% Some text for chapter 2:
%\iffalse
%<*samplechap2>
%\fi
%    \begin{macrocode}
\section{two}
more text in chapter two
%    \end{macrocode}

%\iffalse
%</samplechap2>
%\fi
%
% %%%%%%%%%%%%%%%%%%%%%%%%%%%%%%%%%%%%%%
% \paragraph{Part Include Files.}
%
% The include files are called |cdocspt3.tex| and |cdocspt4.tex|.
%
%\iffalse
%<*samplepart3|samplepart4>
%\fi

% Optional override for |\version| flag:
%    \begin{macrocode}
%%\providecommand{\version}{final}
%    \end{macrocode}

% Include the main document:
%    \begin{macrocode}
\input{childdoc.def}
\childdocby{cdocsamp}
%    \end{macrocode}

%\iffalse
%</samplepart3|samplepart4>
%\fi
%
%\iffalse
%<*samplepart3>
%\fi
% Some text for part 3:
%    \begin{macrocode}
some text in part three
%    \end{macrocode}

%\iffalse
%</samplepart3>
%\fi
% Some text for part 4:
%\iffalse
%<*samplepart4>
%\fi
%    \begin{macrocode}
more text in part four
%    \end{macrocode}

%\iffalse
%</samplepart4>
%\fi
%
% %%%%%%%%%%%%%%%%%%%%%%%%%%%%%%%%%%%%%%
% \paragraph{Forwarding for a Complete Draft.}
%
% The following forwarding file |cdocsdrf.tex|
% compiles the main document in draft mode:
%\iffalse
%<*sampledraft>
%\fi
%    \begin{macrocode}
\def\version{draft}
\input{childdoc.def}
\childdocforward{cdocsamp}
%    \end{macrocode}

%\iffalse
%</sampledraft>
%\fi
%
% %%%%%%%%%%%%%%%%%%%%%%%%%%%%%%%%%%%%%%
% \paragraph{Forwarding for Final Version of the Chapters.}
%
% The following forwarding files |cdocsfn1.tex| and |cdocsfn2.tex|
% (with identical content)
% compile the final versions of the child documents
% |cdocsch1.tex| and |cdocsch2.tex|, respectively:
%\iffalse
%<*samplefinal>
%\fi
%    \begin{macrocode}
\def\version{final}
\input{childdoc.def}
\childdocforwardprefix[cdocsamp]{cdocsfn}{cdocsch}
%    \end{macrocode}

%\iffalse
%</samplefinal>
%\fi
%
% %%%%%%%%%%%%%%%%%%%%%%%%%%%%%%%%%%%%%%
% \paragraph{Command Line Processing.}
%
% The following three command lines generate the output files
% |cdocscld|, |cdocscl1| and |cdocscl2|
% which should be identical to
% |cdocsdrf|, |cdocsch1| and |cdocsfn2|, respectively:
% \begin{center}
% \begin{tabular}{l}
% |latex -jobname cdocscld \|\\
% |  "\def\version{draft}\input{childdoc.def}\childdocforward{cdocsamp}"|\\
% |latex -jobname cdocscl1 \|\\
% |  "\input{childdoc.def}\childdocforward[cdocsamp]{cdocsch1}"|\\
% |latex -jobname cdocscl2 \|\\
% |  "\def\version{final}\input{childdoc.def}\childdocforward{cdocsch2}"|
% \end{tabular}
% \end{center}
% Note that the trailing backslash on each first line
% merely continues the input to the second line
% (for convenient cut ant paste).
% Furthermore, the command |latex| can be replaced by any
% of its alternative versions such as |pdflatex|.
%
% %%%%%%%%%%%%%%%%%%%%%%%%%%%%%%%%%%%%%%%%%%%%%%%%%%%%%%%%%%%%%%%%%%%%%%%%%%%%%%
% %%%%%%%%%%%%%%%%%%%%%%%%%%%%%%%%%%%%%%%%%%%%%%%%%%%%%%%%%%%%%%%%%%%%%%%%%%%%%%
% \section{Implementation}
%\iffalse
%<*package>
%\fi
%
% This section describes the definitions file |childdoc.def|.

% The definitions cannot be loaded using |\usepackage| or |\RequirePackage|
% which has a mechanism to prevent loading a style file more than once.
% When loading the definitions by means of |\input|
% multiple instances have to be prevented manually:
%\iffalse
%This code needs to be before the `\ProvidesFile' directive
%which is defined at the beginning of this file.
%Therefore it is also placed there and commented out here.
%</package>
%<*discard>
%\fi
%    \begin{macrocode}
\ifdefined\childdocmain\endinput\fi
%    \end{macrocode}
%\iffalse
%</discard>
%<*package>
%\fi
%
% \macro{\ifchilddoc}
% \macro{\ifchilddocmanual}
% The conditional |\ifchilddoc| tells whether a
% child (true) or main (false) document is being compiled.
% The conditional |\ifchilddocmanual| tells whether
% the |\includeonly| mechanism is used (false) or
% the selection of child files must be performed manually (true).
% The definitions initialise to false:
%    \begin{macrocode}
\newif\ifchilddoc
\newif\ifchilddocmanual
%    \end{macrocode}

% \macro{\childdocname}
% \macro{\childdocjob}
% The macro |\childdocname| stores the name of the main document
% to be compiled. The macro |\childdocjob| stores the name of
% the document on which the \LaTeX{} compiler was originally invoked.
% The content of |\jobname| cannot be compared
% to filenames specified in the source due to different catcodes.
% The following code rescans |\jobname|, stores the result
% in |\childdocname| and saves a copy in |\childdocjob|:
%    \begin{macrocode}
\edef\childdocname{\scantokens\expandafter{\jobname\noexpand}}
\let\childdocjob\childdocname
%    \end{macrocode}

% \macro{\childdocdisable}
% The macro |\childdocdisable| prevents the main file
% from being processed more than once.
% At this stage, the main document command |\childdocmain|
% is assumed to be called once again where it should do nothing.
% Any subsequent call to it should prevent
% a secondary processing of the main document
% It overwrites the forwarding commands
% |\childdocof| and |\childdocforward|
% with empty macros to prevent further inclusions of the main document:
%    \begin{macrocode}
\newcommand{\childdocdisable}
{
  \renewcommand{\childdocmain}[1]{\renewcommand{\childdocmain}[1]{\endinput}}
  \renewcommand{\childdocof}[1]{}
  \renewcommand{\childdocby}[2][]{}
  \renewcommand{\childdocforward}[2][]{}
  \renewcommand{\childdocdisable}{}
}
%    \end{macrocode}

% \macro{\childdocmain}
% The macro |\childdocmain| is to be called at the top of the main file
% with nothing or the main filename (without extension) as argument.
% First, it breaks loops.
% If the argument is not empty and does not match |\childdocname|
% (which is set by the first inclusion of |childdoc.def|),
% |\ifchilddoc| is set to true, |\includeonly| is applied to the child file
% and |\jobname| is set to the main file
% (for proper handling of |.aux| files):
%    \begin{macrocode}
\newcommand{\childdocmain}[1]
{
  \childdocdisable\childdocmain{}
  \if?#1?\else
    \begingroup
      \def\childdoctmp{#1}
      \ifx\childdoctmp\childdocname
        \def\childdoctmp{}
      \else
        \def\childdoctmp
        {
          \childdoctrue
          \includeonly{\childdocname}
          \def\childdocjob{#1}
          \def\jobname{#1}
        }
      \fi
      \expandafter
    \endgroup
    \childdoctmp
  \fi
}
%    \end{macrocode}

% \macro{\childdocof}
% The command |\childdocof| redirects
% compilation to the main file |#1|.
%    \begin{macrocode}
\newcommand{\childdocof}[1]
{
  \childdocdisable
  \childdoctrue
  \includeonly{\childdocname}
  \def\jobname{#1}
  \def\childdocjob{#1}
  \input{#1}
}
%    \end{macrocode}

% \macro{\childdocby}
% The command |\childdocby| ....
%    \begin{macrocode}
\newcommand{\childdocby}[2][]
{
  \childdocdisable
  \childdoctrue
  \childdocmanualtrue
  \if?#1?\else
    \def\jobname{#2}
  \fi
  \def\childdocjob{#2}
  \input{#2}
  \endinput
}
%    \end{macrocode}

% \macro{\childdocforward}
% The command |\childdocforward| redirects
% compilation to the main file or
% (if the optional argument is given) a child file.
% Parameters are set as if the main file
% or a child file starting with |\childdocof| was compiled.
% Then compilation is handed over to the main file:
%    \begin{macrocode}
\newcommand{\childdocforward}[2][]
{
  \begingroup
    \if?#1?
      \def\childdoctmp
      {
        \def\childdocname{#2}
        \def\childdocjob{#2}
        \def\jobname{#2}
        \input{#2}
        \endinput
      }
    \else
      \def\childdoctmp
      {
        \childdocdisable
        \def\childdocname{#2}
        \childdoctrue
        \includeonly{#2}
        \def\childdocjob{#1}
        \def\jobname{#1}
        \input{#1}
        \endinput
      }
    \fi
    \expandafter
  \endgroup
  \childdoctmp
}
%    \end{macrocode}

% \macro{\childdocforwardprefix}
% The command |\childdocforwardprefix| redirects
% compilation to the main or a child file by means of a pattern.
% The prefix |#1| in the current filename is replaced by |#2|
% and the suffix of the current filename is kept
% (it is assumed that the filename does not contain the substring `|~~~|'
% which is used as a delimiter).
% Compilation is handed over to the new file by |\childdocforward|:
%    \begin{macrocode}
\newcommand{\childdocforwardprefix}[3][]
{
  \begingroup
    \def\childdocextract #2##1~~~{\def\childdoctmp{\childdocforward[#1]{#3##1}}}
    \expandafter\childdocextract\childdocname~~~
    \expandafter
  \endgroup
  \childdoctmp
}
%    \end{macrocode}

% \macro{\childdoc}
% The deprecated macro |\childdoc| is a legacy version of |\childdocmain|:
%    \begin{macrocode}
\newcommand{\childdoc}{\childdocmain}
%    \end{macrocode}

% \macro{\childdocredirect}
% The deprecated macro |\childdocredirect| is a legacy version
% of |\childdocforward| and |\childdocforwardprefix|:
%    \begin{macrocode}
\newcommand{\childdocredirect}[2][]
{
  \begingroup
    \if?#1?
      \def\childdoctmp{\childdocforward{#2}}
    \else
      \def\childdoctmp{\childdocforwardprefix{#1}{#2}}
    \fi
    \expandafter
  \endgroup
  \childdoctmp
}
%    \end{macrocode}

%\iffalse
%</package>
%\fi
%
\endinput

\childdocof{cdocsamp}
%    \end{macrocode}

%\iffalse
%</samplechap1|samplechap2>
%\fi
%
%\iffalse
%<*samplechap1>
%\fi
% Some text for chapter 1:
%    \begin{macrocode}
\section{one}
some text in chapter one
%    \end{macrocode}

%\iffalse
%</samplechap1>
%\fi
% Some text for chapter 2:
%\iffalse
%<*samplechap2>
%\fi
%    \begin{macrocode}
\section{two}
more text in chapter two
%    \end{macrocode}

%\iffalse
%</samplechap2>
%\fi
%
% %%%%%%%%%%%%%%%%%%%%%%%%%%%%%%%%%%%%%%
% \paragraph{Part Include Files.}
%
% The include files are called |cdocspt3.tex| and |cdocspt4.tex|.
%
%\iffalse
%<*samplepart3|samplepart4>
%\fi

% Optional override for |\version| flag:
%    \begin{macrocode}
%%\providecommand{\version}{final}
%    \end{macrocode}

% Include the main document:
%    \begin{macrocode}
% \iffalse
%
% childdoc.dtx Copyright (C) 2017-2018 Niklas Beisert
%
% This work may be distributed and/or modified under the
% conditions of the LaTeX Project Public License, either version 1.3
% of this license or (at your option) any later version.
% The latest version of this license is in
%   http://www.latex-project.org/lppl.txt
% and version 1.3 or later is part of all distributions of LaTeX
% version 2005/12/01 or later.
%
% This work has the LPPL maintenance status `maintained'.
%
% The Current Maintainer of this work is Niklas Beisert.
%
% This work consists of the files childdoc.dtx and childdoc.ins
% and the derived files childdoc.def and cdocsamp.tex with
% cdocsch1.tex, cdocsch2.tex, cdocsdrf.tex, cdocsfn1.tex, cdocsfn2.tex.
%
%<package>\ifdefined\childdocmain\endinput\fi
%<package>\ProvidesFile{childdoc.def}[2018/12/30 v2.0 child document driver]
%<samplemain>\ProvidesFile{cdocsamp.tex}[2018/12/30 v2.0 sample for childdoc]
%<*driver>
%\ProvidesFile{childdoc.drv}[2018/12/30 v2.0 childdoc reference manual file]
\PassOptionsToClass{10pt,a4paper}{article}
\documentclass{ltxdoc}

\usepackage[margin=35mm]{geometry}
\usepackage{hyperref}
\usepackage{hyperxmp}
\usepackage[usenames]{color}

\hypersetup{colorlinks=true}
\hypersetup{pdfstartview=FitH}
\hypersetup{pdfpagemode=UseNone}
\hypersetup{pdfsource={}}
\hypersetup{pdflang={en-UK}}
\hypersetup{pdfcopyright={Copyright 2017-2018 Niklas Beisert.
  This work may be distributed and/or modified under the
  conditions of the LaTeX Project Public License, either version 1.3
  of this license or (at your option) any later version.}}
\hypersetup{pdflicenseurl={http://www.latex-project.org/lppl.txt}}
\hypersetup{pdfcontactaddress={ETH Zurich, ITP, HIT K,
  Wolfgang-Pauli-Strasse 27}}
\hypersetup{pdfcontactpostcode={8093}}
\hypersetup{pdfcontactcity={Zurich}}
\hypersetup{pdfcontactcountry={Switzerland}}
\hypersetup{pdfcontactemail={nbeisert@itp.phys.ethz.ch}}
\hypersetup{pdfcontacturl={http://people.phys.ethz.ch/\xmptilde nbeisert/}}

\newcommand{\secref}[1]{\hyperref[#1]{section \ref*{#1}}}

\parskip1ex
\parindent0pt
\let\olditemize\itemize
\def\itemize{\olditemize\parskip0pt}

\begin{document}

\title{The \textsf{childdoc} Package}
\hypersetup{pdftitle={The childdoc Package}}
\author{Niklas Beisert\\[2ex]
  Institut f\"ur Theoretische Physik\\
  Eidgen\"ossische Technische Hochschule Z\"urich\\
  Wolfgang-Pauli-Strasse 27, 8093 Z\"urich, Switzerland\\[1ex]
  \href{mailto:nbeisert@itp.phys.ethz.ch}
  {\texttt{nbeisert@itp.phys.ethz.ch}}}
\hypersetup{pdfauthor={Niklas Beisert}}
\hypersetup{pdfsubject={Manual for the LaTeX2e Package childdoc}}
\date{30 December 2018, \textsf{v2.0}}
\maketitle

\begin{abstract}\noindent
\textsf{childdoc} is a \LaTeXe{} package
that enables the direct compilation
of document sections included by |\include|
to individual files.
\end{abstract}

\begingroup
\parskip0ex
\tableofcontents
\endgroup

%%%%%%%%%%%%%%%%%%%%%%%%%%%%%%%%%%%%%%%%%%%%%%%%%%%%%%%%%%%%%%%%%%%%%%%%%%%%%%%%
%%%%%%%%%%%%%%%%%%%%%%%%%%%%%%%%%%%%%%%%%%%%%%%%%%%%%%%%%%%%%%%%%%%%%%%%%%%%%%%%
\section{Introduction}

\LaTeX{} provides a mechanism to structure a large document (such as a book)
into a main file and several child files (containing the chapters)
using the |\include| command.
This mechanism is beneficial for documents
which span hundreds of pages in order to
make the source file(s) more manageable.
Moreover, compilation can be restricted to
selected child files by means of the |\includeonly| command.
The latter feature can be used to reduce the compilation time while editing
(this was significantly more useful in the earlier days of \LaTeX{})
or to generate a smaller document which is easier to navigate.
Another application of |\includeonly| is to generate
documents consisting of selected parts of the complete document.

However, there are a few drawbacks of the plain |\include| mechanism:
\begin{itemize}
\item
The child files cannot be compiled on their own,
they can only be compiled via the main file.
A naive editing environment
(such as a text editor with an option
to have the current file processed by \LaTeX)
may require one to switch to the main file before compiling;
attempting to compile the child file produces errors.
\item
The main file must be modified (each time)
to adjust the |\includeonly| command
to the present needs. This easily leaves the main file in a messy state.
\item
The generated document will always carry the filename
of the main document. This is inconvenient if
several child files are to be compiled and
to be kept for distribution.
\end{itemize}

The present package provides a simple interface
to make child files individually compilable by \LaTeX{}.
Compiling a child file then has the same effect as compiling
the main file with an |\includeonly| command
to select the appropriate child.
Moreover the generated document will carry the name of the child
rather than the main file.
This resolves all three above issues.

This feature is meant to make the editing of books,
thesis documents and lecture notes somewhat more convenient.
However, the package can also be used efficiently for
composing a series of documents (such as exercise sheets)
which are typically distributed individually.
It then assists the author in generating the individual documents
(potentially in different versions)
as well as a document containing the collected series.
Another application is in developing style files
or other kinds of included material
where compilation of the style file could redirect
to a sample or test file.

%%%%%%%%%%%%%%%%%%%%%%%%%%%%%%%%%%%%%%%%%%%%%%%%%%%%%%%%%%%%%%%%%%%%%%%%%%%%%%%%
%%%%%%%%%%%%%%%%%%%%%%%%%%%%%%%%%%%%%%%%%%%%%%%%%%%%%%%%%%%%%%%%%%%%%%%%%%%%%%%%
\section{Usage}

First of all, the package \textsf{childdoc} is \emph{not} a standard
\LaTeXe{} |.sty| style file! Therefore it needs to be invoked in
a non-standard way.

%%%%%%%%%%%%%%%%%%%%%%%%%%%%%%%%%%%%%%%%%%%%%%%%%%%%%%%%%%%%%%%%%%%%%%%%%%%%%%%%
\subsection{Included Files}
\label{sec:include}

%%%%%%%%%%%%%%%%%%%%%%%%%%%%%%%%%%%%%%%%
\DescribeMacro{\childdocmain}
To use the package, add the commands
\begin{center}
\begin{tabular}{l}
|\input{childdoc.def}|\\
|\childdocmain{}|\\
\end{tabular}
\end{center}
at the very top of the main \LaTeX{} file,
in particular \emph{before} the |\documentclass| statement!
The argument of |\childdocmain| should be left empty
(but it must be present).

%%%%%%%%%%%%%%%%%%%%%%%%%%%%%%%%%%%%%%%%
\DescribeMacro{\childdocof}
Furthermore, add the commands
\begin{center}
\begin{tabular}{l}
|\input{childdoc.def}|\\
|\childdocof{|\textit{main}|}|\\
\end{tabular}
\end{center}
at the top of every child file \textit{child}
which is included by |\include{|\textit{child}|}|
from within the main file
(or at least for those files to be compiled individually).
The argument \textit{main} must be the filename of the main file.

There are a couple of
considerations in setting up the main and child documents:

%%%%%%%%%%%%%%%%%%%%%%%%%%%%%%%%%%%%%%%%
\paragraph{Restrictions.}

Please note the following restrictions:
\begin{itemize}
\item
|\childdocmain| must be called with one argument \textit{main}
to ensure compatibility with earlier version of the package.
It must either be empty (|\childdocmain{}|)
or precisely match the filename of the main file in which it is specified.
See \secref{sec:detection} for further information.
\item
The filename \textit{main} must be specified without the |.tex| extension.
\item
The filename \textit{main} is case sensitive
(even in case-insensitive file systems)
due to internal string comparison.
\item
The argument \textit{main} should be fully expanded, it cannot be a macro.
\item
Subdirectories and special characters should be avoided in filenames.
\item
The command |\childdocmain{|\textit{main}|}| must be followed by a whitespace.
It should not be followed immediately by another command
or by a comment mark `|%|'.
This is because the \TeX{} parser reads the token immediately following
the argument of |\childdocmain| and puts it
at the beginning of every child section;
however, a white\-space is ignored.
\end{itemize}

%%%%%%%%%%%%%%%%%%%%%%%%%%%%%%%%%%%%%%%%
\paragraph{Content of Main File.}

It is advisable to place all content in the child files included by |\include|.
Any output contained in the main file will appear in all child documents
unless suppressed manually;
it cannot be suppressed automatically by the |\includeonly| directive
and thus should normally be avoided.
A method to include some content in the main file
by means of conditional processing is described in \secref{sec:conditional}.

%%%%%%%%%%%%%%%%%%%%%%%%%%%%%%%%%%%%%%%%
\paragraph{Page Numbering.}

When only a part of the document is compiled,
the appropriate numbering of pages
(as well as other status parameters)
is determined from the |.aux| files.
The latter contain information from previous passes.
However this information needs to propagate through
all intermediate child documents.
Therefore the page numbering in child documents may well
be inconsistent until the complete document is compiled at least once.

A useful (if unconventional) way to always ensure a consistent
page numbering is to restart the numbering in each child document
and denote the pages by `\textit{child}|.|\textit{page}'
where \textit{child} represents the chapter/section number of the child file.
This can be achieved by the command
|\numberwithin{page}{|\textit{child}|}|
of the \textsf{amsmath} package
where \textit{child} can be |chapter| or |section|
depending on the chosen structuring.
Alternatively, one can modify the macro |\thepage| appropriately
and reset the counter |page| at the start of each child file.

%%%%%%%%%%%%%%%%%%%%%%%%%%%%%%%%%%%%%%%%%%%%%%%%%%%%%%%%%%%%%%%%%%%%%%%%%%%%%%%%
\subsection{Conditional Processing}
\label{sec:conditional}

The package provides a mechanism to compile different versions
of a document. To customise the versions further some conditional processing
can come in handy to distinguish which version is being compiled.
The package provides two macros to describe the compilation context:

%%%%%%%%%%%%%%%%%%%%%%%%%%%%%%%%%%%%%%%%
\DescribeMacro{\ifchilddoc}
The conditional |\ifchilddoc| distinguishes between the compilation of
child documents and the main document:
%
\begin{center}
|\ifchilddoc |\textit{child-code}| |[|\||else |\textit{main-code}]| \||fi|
\end{center}

%%%%%%%%%%%%%%%%%%%%%%%%%%%%%%%%%%%%%%%%
\DescribeMacro{\childdocname}
\DescribeMacro{\childdocjob}
The macro |\childdocname| contains the filename (without extension)
of the main or child file being processed.
Note that |\childdocjob| will always contain the name of the main file.

%%%%%%%%%%%%%%%%%%%%%%%%%%%%%%%%%%%%%%%%
\paragraph{Title Page.}

Conditional processing can be used to include a title or banner page
in the main document when proper precautions are taken.
Importantly, the code in the main file should ensure that the page counter
(as well as other status parameters which are stored in the |.aux| files)
takes the same value after the conditional processing.
Otherwise the page numbers may take divergent values
depending on which part is compiled.

For example, a title page could be declared by:
%
\begin{center}
\begin{tabular}{l}
|\ifchilddoc\||else|\\
|\addtocounter{page}{-1}|\\
\textit{code for title page}\\
|\newpage|\\
|\||fi|
\end{tabular}
\end{center}
%
A banner page for the child documents can be generated by:
%
\begin{center}
\begin{tabular}{l}
|\ifchilddoc|\\
|\addtocounter{page}{-1}|\\
\textit{code for banner page}\\
|\newpage|\\
|\||fi|
\end{tabular}
\end{center}
%
Here one could write a message such as:
\begin{center}
|This is the part \childdocname{} of \childdocjob{}.|
\end{center}

%%%%%%%%%%%%%%%%%%%%%%%%%%%%%%%%%%%%%%%%%%%%%%%%%%%%%%%%%%%%%%%%%%%%%%%%%%%%%%%%
\subsection{Flags}
\label{sec:flags}

The package makes it easy to generate different versions
of the main or child documents.
To this end compilation flags can be defined
and assigned different default values.
They will be particularly useful in conjunction
with the forwarding mechanism described in \secref{sec:forward}.

For example, it may be useful to have a flag |\version|
which can be set to |draft| or |final|.
The document source will contain some conditional code
depending on the value of |\version|.
Suppose further, the flag should default to |final| for the main file
and to |draft| for child files
which is a natural assignment for editing the document.
This is achieved by placing the following code
in the preamble of the main document
(below the |\childdocmain| directive):
%
\begin{center}
\begin{tabular}{l}
|\ifchilddoc|\\
|\providecommand{\version}{draft}|\\
|\||else|\\
|\providecommand{\version}{final}|\\
|\||fi|
\end{tabular}
\end{center}
%
The definition by |\providecommand| makes sure
that previous definitions are not overwritten.
Further statements |\providecommand{\version}{...}|
can thus be added before the above code to override it.

For the main file, one might add a line
(between |\childdocmain| and the above block)
%
\begin{center}
|%\ifchilddoc\||else\providecommand{\version}{draft}\||fi|
\end{center}
%
which can be uncommented to produce a draft version.
Likewise one can add a line to the very top of a child file
(above the |\childdocof{|\textit{main}|}| directive)
%
\begin{center}
|%\providecommand{\version}{final}|
\end{center}
%
which can be uncommented to produce the final version of this child document.

%%%%%%%%%%%%%%%%%%%%%%%%%%%%%%%%%%%%%%%%%%%%%%%%%%%%%%%%%%%%%%%%%%%%%%%%%%%%%%%%
\subsection{Forwarding}
\label{sec:forward}

Different versions of the main or child documents
using compilation flags as described in \secref{sec:flags}
can be (permanently) stored in different files
for convenient compilation, viewing and distribution.
To this end, the package defines a command
to pass on compilation to a different file:

%%%%%%%%%%%%%%%%%%%%%%%%%%%%%%%%%%%%%%%%
\DescribeMacro{\childdocforward}
The command |\childdocforward| redirects processing to
another source file:
%
\begin{center}
\begin{tabular}{l}
|\input{childdoc.def}|\\
|\childdocforward[|\textit{main}|]{|\textit{dest}|}|\\
\end{tabular}
\end{center}
%
The argument \textit{dest} is the destination file
(without extension).
It should be the main file or one of the child files.
Note that further \textsf{childdoc} directives
such as |\childdocof| and |\childdocforward|
in the indicated file will be processed in this form.
The optional argument \textit{main}
passes on directly to the main file \textit{main}
while pretending to compile the child \textit{dest}.
This form behaves as if \textit{dest}
issues |\childdocof{|\textit{main}|}| right away,
and no further \textsf{childdoc} directives will be processed.

%%%%%%%%%%%%%%%%%%%%%%%%%%%%%%%%%%%%%%%%
\DescribeMacro{\...prefix}
In the alternative form |\childdocforwardprefix|,
%
\begin{center}
\begin{tabular}{l}
|\input{childdoc.def}|\\
|\childdocforwardprefix[|\textit{main}|]{|\textit{prefix}|}{|\textit{dest}|}|
\end{tabular}
\end{center}
%
the destination file is determined by a pattern
depending on the current file:
To make this work, the current file must be called
`{\textit{prefix}\hspace{0.2em}\textit{suffix}}'
with \textit{prefix} matching precisely the argument.
Processing is then passed on to the file
`{\textit{dest}\hspace{0.2em}\textit{suffix}}'.
Surely, the same effect is achieved by
directly specifying the
argument `{\textit{dest}\hspace{0.2em}\textit{suffix}}'
in the first form.
However, that requires to set up a different file
for each child. With the alternative form of the command
all these files can have exactly the same content
which simplifies setting them up and maintaining them.

For example, the following file |draft.tex|
with a compilation flag |\version| as described in \secref{sec:flags}
compiles the main document as a draft:
%
\begin{center}
\begin{tabular}{l}
|\def\version{draft}|\\
|\input{childdoc.def}|\\
|\childdocforward{|\textit{main}|}|
\end{tabular}
\end{center}
%
Likewise, the following files |final|\textit{nn}|.tex|
compile the final version of the child document
|child|\textit{nn}|.tex|:
%
\begin{center}
\begin{tabular}{l}
|\def\version{final}|\\
|\input{childdoc.def}|\\
|\childdocforwardprefix{final}{child}|
\end{tabular}
\end{center}
%

Note that when several versions of a main file and/or of each child file
are to be generated, it may be convenient to set up a |Makefile| or
shell script to automatise the process.

%%%%%%%%%%%%%%%%%%%%%%%%%%%%%%%%%%%%%%%%%%%%%%%%%%%%%%%%%%%%%%%%%%%%%%%%%%%%%%%%
\subsection{Command Line Processing}
\label{sec:commandline}

The effect of redirection files can also be achieved by invoking
the \LaTeX{} compiler with a more elaborate command line.
Most conveniently this should be done as part
of a shell script or a |Makefile|.

When using \textsf{childdoc} in the main file, the following
command lines effectively perform a redirection
(note that depending on the shell being used,
backslashes may have to be doubled: `|\|' $\to$ `|\\|'):
%
\begin{center}
|... -jobname "|\textit{target}|" |\\|"|[\textit{flags}]%
|\input{childdoc.def}\childdocforward[|\textit{main}|]{|\textit{dest}|}"|
\end{center}
%
Here \textit{target} is the name of the output file,
\textit{main} is the name of the main file
and \textit{dest} is the name of the main or child file to be processed
(all filenames without extensions).
The optional argument \textit{main} can be omitted
if \textit{main} matches \textit{dest}.
Optionally, compilation \textit{flags} can be defined via |\def| commands.
This command line makes the \TeX{} engine believe
it is compiling the file \textit{target}
whose content is specified as the latter parameter.
The provided code then forwards the processing to
\textit{main} or \textit{dest} as described in \secref{sec:forward}.

%%%%%%%%%%%%%%%%%%%%%%%%%%%%%%%%%%%%%%%%%%%%%%%%%%%%%%%%%%%%%%%%%%%%%%%%%%%%%%%%
\subsection{Include by Input}
\label{sec:input}

Including child documents by |\include| has some restrictions by design.
Most notably, the content of a child document always occupies
its own set of pages; pages cannot be shared between child documents.
Usually, this behaviour makes perfect sense
because each child document contain an essential part of the document.
However, in some situations it may be desirable to compose
a document from a collection of parts
without having mandatory page breaks between then.
For this case, the package
provides a mechanism to include parts
by |\input| which can also be processed individually.
However, by construction this mechanism
requires manual handling of the content to be output.

%%%%%%%%%%%%%%%%%%%%%%%%%%%%%%%%%%%%%%%%
\DescribeMacro{\ifchilddocmanual}
The main file should be prepared as usual, see \secref{sec:include}.
However, the document body must make a distinction
between processing of an individual part and of the main document, e.g.:
%
\begin{center}
\begin{tabular}{l}
|\ifchilddocmanual|\\
|\input{\childdocname}|\\
|\||else|\\
\textit{document body with }|\input{|\textit{part}|}|\\
|\||fi|
\end{tabular}
\end{center}
%
The conditional |\ifchilddocmanual| is true whenever
a part to be included by |\input| is being compiled,
and the name of the part is stored in |\childdocname|.

%%%%%%%%%%%%%%%%%%%%%%%%%%%%%%%%%%%%%%%%
\DescribeMacro{\childdocby}
Each part to be included by |\input| should start with:
%
\begin{center}
\begin{tabular}{l}
|\input{childdoc.def}|\\
|\childdocby{|\textit{main}|}|\\
\end{tabular}
\end{center}
%
The directive |\childdocby| is similar to |\childdocof|
described in \secref{sec:include},
but the subsequent selection of content must be done manually.
To that end, both |\ifchilddoc| and |\ifchilddocmanual|
will be true upon processing of a part,
and the name of the part is stored in |\childdocname|.
Note that |\jobname| will be set to the filename of the current part
so that each part receives an individual |.aux| file
that does not interfere with the |.aux| file(s) of the main document.
This behaviour can be altered by the alternative form
|\childdocby[*]{|\textit{main}|}| (with a non-empty optional argument)
which uses the |.aux| file of the main document
by setting |\jobname| to \textit{main}.

%%%%%%%%%%%%%%%%%%%%%%%%%%%%%%%%%%%%%%%%%%%%%%%%%%%%%%%%%%%%%%%%%%%%%%%%%%%%%%%%
\subsection{Driver Development}
\label{sec:driver}

The \textsf{childdoc} mechanism can also be use for the development
of definition files such as \LaTeX{} styles or classes.
This case differs from the above setup with multiple parts
included by |\include| in that no |\includeonly| should be invoked.
This can be achieved by starting the include file
(before |\ProvidesPackage|) with:
%
\begin{center}
\begin{tabular}{l}
|\input{childdoc.def}|\\
|\childdocforward{|\textit{main}|}|\\
\end{tabular}
\end{center}
%
or alternatively with:
%
\begin{center}
\begin{tabular}{l}
|\input{childdoc.def}|\\
|\childdocby{|\textit{main}|}|\\
\end{tabular}
\end{center}
%
Both forms have slightly different effects as described above.
The main file is prepared as usual, see \secref{sec:include}.

%%%%%%%%%%%%%%%%%%%%%%%%%%%%%%%%%%%%%%%%%%%%%%%%%%%%%%%%%%%%%%%%%%%%%%%%%%%%%%%%
\subsection{Legacy Detection}
\label{sec:detection}

The directive |\childdocmain| in the main file can detect
whether the complete document or merely a child is to be compiled
even without using the directive |\childdocof|.
This method is deprecated because it is less robust
and there is no compelling reason to use it;
it is merely provided for backward compatibility
and it may be removed in future versions.

If the detection mechanism is to be used,
it is mandatory to correctly specify
the filename of the main file as the argument of |\childdocmain|:
%
\begin{center}
\begin{tabular}{l}
|\input{childdoc.def}|\\
|\childdocmain{|\textit{main}|}|\\
\end{tabular}
\end{center}
%
If |\jobname| does not match the argument \textit{main} of |\childdocmain|,
it is assumed that |\jobname| points to the child file to be compiled.
When using |\childdocmain| with the main file specified as argument,
it suffices to start a child file
with just |\input{|\textit{main}|}|
without loading of the package and using |\childdocof|.
If instead all processing is done
with the appropriate \textsf{childdoc} directives,
the argument of \textit{main} of |\childdocmain| can be empty.

An alternative version of the command line processing described
in \secref{sec:commandline} using the detection mechanism reads:
%
\begin{center}
|... -jobname "|\textit{target}|" "|[\textit{flags}]%
[|\def\jobname{|\textit{dest}|}|]|\input{|\textit{main}|}"|
\end{center}

%%%%%%%%%%%%%%%%%%%%%%%%%%%%%%%%%%%%%%%%%%%%%%%%%%%%%%%%%%%%%%%%%%%%%%%%%%%%%%%%
\subsection{Manual Code}
\label{sec:manual}

In case one cannot be certain whether the definitions file |childdoc.def|
is installed on the target \TeX{} distribution
and one prefers not to ship it,
it is conceivable to paste a few relevant commands into the sources.

To that end, drop all statements |\input{childdoc.def}|
and perform the replacements as outlined below.
Instead of |\childdocmain{|\textit{main}|}| add the following code
to the top of the main file:
%
\begin{center}
\begin{tabular}{l}
|\||ifdefined\childdocname\endinput\||fi\newif\ifchilddoc|\\
|\edef\childdocname{\scantokens\expandafter{\jobname\noexpand}}|\\
|\def\childdocmain{|\textit{main}|}\||ifx\childdocmain\childdocname\||else|\\
|\childdoctrue\includeonly{\childdocname}\let\jobname\childdocmain\||fi|\\
\end{tabular}
\end{center}
%
Instead of |\childdocof{|\textit{main}|}| just include the main file
at the top of each child file:
%
\begin{center}
|\input{|\textit{main}|}|
\end{center}
%
A simple redirection |\childdocforward{|\textit{dest}|}| is achieved by:
%
\begin{center}
|\def\jobname{|\textit{dest}|}\input{\jobname}|
\end{center}
%
The redirection with prefix
|\childdocforwardprefix[|\textit{prefix}|]{|\textit{dest}|}|
is accomplished by:
%
\begin{center}
\begin{tabular}{l}
|{\edef\jobname{\scantokens\expandafter{\jobname\noexpand}}|\\
|\def\redirectjob |\textit{prefix}|#1~~~{\gdef\jobname{|\textit{dest}|#1}}|\\
|\expandafter\redirectjob\jobname~~~}\input{\jobname}|
\end{tabular}
\end{center}

In an alternative approach,
child documents can be compiled by a specific command line
without additional code or specific definitions:
%
\begin{center}
|... -jobname "|\textit{target}|" "|[\textit{flags}]%
|\includeonly{|\textit{dest}|}\input{|\textit{main}|}"|
\end{center}
%

%%%%%%%%%%%%%%%%%%%%%%%%%%%%%%%%%%%%%%%%%%%%%%%%%%%%%%%%%%%%%%%%%%%%%%%%%%%%%%%%
%%%%%%%%%%%%%%%%%%%%%%%%%%%%%%%%%%%%%%%%%%%%%%%%%%%%%%%%%%%%%%%%%%%%%%%%%%%%%%%%
\section{Information}

%%%%%%%%%%%%%%%%%%%%%%%%%%%%%%%%%%%%%%%%%%%%%%%%%%%%%%%%%%%%%%%%%%%%%%%%%%%%%%%%
\subsection{Copyright}

Copyright \copyright{} 2017--2018 Niklas Beisert

This work may be distributed and/or modified under the
conditions of the \LaTeX{} Project Public License, either version 1.3
of this license or (at your option) any later version.
The latest version of this license is in
  \url{http://www.latex-project.org/lppl.txt}
and version 1.3 or later is part of all distributions of \LaTeX{}
version 2005/12/01 or later.

This work has the LPPL maintenance status `maintained'.

The Current Maintainer of this work is Niklas Beisert.

This work consists of the files |README.txt|, |childdoc.ins| and |childdoc.dtx|
as well as the derived files |childdoc.def|, |cdocsamp.tex|
with |cdocsch1.tex|, |cdocsch2.tex|, |cdocspt3.tex|, |cdocspt4.tex|,
|cdocsdrf.tex|, |cdocsfn1.tex|, |cdocsfn2.tex|
as well as |childdoc.pdf|.

%%%%%%%%%%%%%%%%%%%%%%%%%%%%%%%%%%%%%%%%%%%%%%%%%%%%%%%%%%%%%%%%%%%%%%%%%%%%%%%%
\subsection{Files and Installation}

The package consists of the files:
%
\begin{center}
\begin{tabular}{ll}
    |README.txt|   & readme file \\
    |childdoc.ins| & installation file \\
    |childdoc.dtx| & source file \\
    |childdoc.def| & definition file \\
    |cdocsamp.tex| & sample main file \\
    |cdocsch1.tex| & sample include file \\
    |cdocsch2.tex| & sample include file \\
    |cdocspt3.tex| & sample part file \\
    |cdocspt4.tex| & sample part file \\
    |cdocsdrf.tex| & sample redirection file \\
    |cdocsfn1.tex| & sample redirection file \\
    |cdocsfn2.tex| & sample redirection file \\
    |childdoc.pdf| & manual
\end{tabular}
\end{center}
%
The distribution consists of the files
|README.txt|, |childdoc.ins| and |childdoc.dtx|.
%
\begin{itemize}
\item
Run (pdf)\LaTeX{} on |childdoc.dtx|
to compile the manual |childdoc.pdf| (this file).
\item
Run \LaTeX{} on |childdoc.ins| to create the definitions file |childdoc.def|
and the sample |cdocsamp.tex| with include files
|cdocsch1.tex|, |cdocsch2.tex|, |cdocspt3.tex|, |cdocspt4.tex|,
|cdocsdrf.tex|, |cdocsfn1.tex|, |cdocsfn2.tex|.
Then copy the file |childdoc.def| to an appropriate directory of your \LaTeX{}
distribution, e.g.\ \textit{texmf-root}|/tex/latex/childdoc|.
\end{itemize}

%%%%%%%%%%%%%%%%%%%%%%%%%%%%%%%%%%%%%%%%%%%%%%%%%%%%%%%%%%%%%%%%%%%%%%%%%%%%%%%%
\subsection{Related CTAN Packages}

There are several other packages which offer a similar functionality:
%
\begin{itemize}
\item
The packages
\href{http://ctan.org/pkg/docmute}{\textsf{docmute}},
\href{http://ctan.org/pkg/includex}{\textsf{includex}} and
\href{http://ctan.org/pkg/standalone}{\textsf{standalone}}
provide commands to include only the document body of
a child file thus allowing both files to be compiled individually.
\item
The packages \href{http://ctan.org/pkg/subdocs}{\textsf{subdocs}}
and \href{http://ctan.org/pkg/subfiles}{\textsf{subfiles}}
provide structures in which the main and child documents can be
encapsulated and allowing them to be compiled individually.
The inclusion mechanism is different from the conventional |\include|.
\item
The package \href{http://ctan.org/pkg/combine}{\textsf{combine}}
is an elaborate solution to combine several documents into one.
\end{itemize}
%
See also the CTAN topic \href{http://ctan.org/topic/subdocs}{\textsf{subdocs}}
for further related packages.
The present package differs from the above solutions in that
a document structure constructed with the conventional |\include| mechanism
just needs two extra commands at the top of every file
such that all constituent files can be compiled individually.

%%%%%%%%%%%%%%%%%%%%%%%%%%%%%%%%%%%%%%%%%%%%%%%%%%%%%%%%%%%%%%%%%%%%%%%%%%%%%%%%
%\subsection{Feature Suggestions}
%
%The following is a list of features which may be useful for future
%versions of this package:
%%
%\begin{itemize}
%\item
%\ldots
%\end{itemize}

%%%%%%%%%%%%%%%%%%%%%%%%%%%%%%%%%%%%%%%%%%%%%%%%%%%%%%%%%%%%%%%%%%%%%%%%%%%%%%%%
\subsection{Revision History}

%%%%%%%%%%%%%%%%%%%%%%%%%%%%%%%%%%%%%%%%
\paragraph{v2.0:} 2018/12/30

\begin{itemize}
\item
immediate forward processing
\item
added |\childdocby| mechanism
\item
manual restructured
\end{itemize}

%%%%%%%%%%%%%%%%%%%%%%%%%%%%%%%%%%%%%%%%
\paragraph{v1.6:} 2018/01/17

\begin{itemize}
\item
application for development of include files
\item
corrections to manual
\end{itemize}

%%%%%%%%%%%%%%%%%%%%%%%%%%%%%%%%%%%%%%%%
\paragraph{v1.5:} 2017/05/21

\begin{itemize}
\item
more complete structuring introduced
\item
|\childdocof| introduced
\item
|\childdoc| renamed to |\childdocmain|
\item
|\childredirect| renamed to |\childdocforward| and |\childdocforwardprefix|
and functionality expanded
\end{itemize}

%%%%%%%%%%%%%%%%%%%%%%%%%%%%%%%%%%%%%%%%
\paragraph{v1.0:} 2017/04/27

\begin{itemize}
\item
manual and install package
\item
first version published on CTAN
\end{itemize}

%%%%%%%%%%%%%%%%%%%%%%%%%%%%%%%%%%%%%%%%
\paragraph{v0.6:} 2017/04/26

\begin{itemize}
\item
redirection mechanism added
\end{itemize}

%%%%%%%%%%%%%%%%%%%%%%%%%%%%%%%%%%%%%%%%
\paragraph{v0.5:} 2017/04/26

\begin{itemize}
\item
functionality in definition file
\end{itemize}


%%%%%%%%%%%%%%%%%%%%%%%%%%%%%%%%%%%%%%%%%%%%%%%%%%%%%%%%%%%%%%%%%%%%%%%%%%%%%%%%
%%%%%%%%%%%%%%%%%%%%%%%%%%%%%%%%%%%%%%%%%%%%%%%%%%%%%%%%%%%%%%%%%%%%%%%%%%%%%%%%
%%%%%%%%%%%%%%%%%%%%%%%%%%%%%%%%%%%%%%%%%%%%%%%%%%%%%%%%%%%%%%%%%%%%%%%%%%%%%%%%
\appendix

\settowidth\MacroIndent{\rmfamily\scriptsize 000\ }

 \DocInput{childdoc.dtx}

\end{document}
%</driver>
% \fi
%
% %%%%%%%%%%%%%%%%%%%%%%%%%%%%%%%%%%%%%%%%%%%%%%%%%%%%%%%%%%%%%%%%%%%%%%%%%%%%%%
% %%%%%%%%%%%%%%%%%%%%%%%%%%%%%%%%%%%%%%%%%%%%%%%%%%%%%%%%%%%%%%%%%%%%%%%%%%%%%%
% \section{Sample}
%\iffalse
%<*samplemain>
%\fi
%
% The following presents a sample document
% with two chapters, two parts, a title page,
% a compile flag as well as three forwarding files to set the flag.
% It consists of eight |.tex| files:
% \begin{center}
% \begin{tabular}{ll}
% |cdocsamp.tex|&main file\\
% |cdocsch1.tex|&include file for chapter 1\\
% |cdocsch2.tex|&include file for chapter 2\\
% |cdocspt3.tex|&include file for part 3\\
% |cdocspt4.tex|&include file for part 4\\
% |cdocsdrf.tex|&forwarding file for main file in draft mode\\
% |cdocsfi1.tex|&forwarding file for final version of chapter 1\\
% |cdocsfi2.tex|&forwarding file for final version of chapter 2\\
% \end{tabular}
% \end{center}
% Each of the eight files can be compiled directly by the \LaTeX{} compiler.
%
% %%%%%%%%%%%%%%%%%%%%%%%%%%%%%%%%%%%%%%
% \paragraph{Main File.}
%
% The main file is called |cdocsamp.tex|.
%
% Load the \textsf{childdoc} definitions and
% declare the filename for the main document:
%    \begin{macrocode}
\input{childdoc.def}
\childdocmain{}
%    \end{macrocode}

% Optional override for |\version| flag:
%    \begin{macrocode}
%%\ifchilddoc\else\providecommand{\version}{draft}\fi
%    \end{macrocode}

% Define the default values for the |\version| flag
% (|final| for the main file and |draft| for childs):
%    \begin{macrocode}
\ifchilddoc
\providecommand{\version}{draft}
\else
\providecommand{\version}{final}
\fi
%    \end{macrocode}

% Load the standard document class:
%    \begin{macrocode}
\documentclass[12pt]{article}
%    \end{macrocode}

% Start the document body:
%    \begin{macrocode}
\begin{document}
%    \end{macrocode}

% Declare a title page.
% Print title, part of document being processed and version flag:
%    \begin{macrocode}
\addtocounter{page}{-1}
\begin{center}
{\LARGE\bfseries{}childdoc example\par}
\vspace{1cm}
\ifchilddoc
\ifchilddocmanual part\else chapter\fi:
`\childdocname' of `\childdocjob'\par
\else
main document: `\childdocjob'\par
\fi
version: \version\par
\end{center}
\newpage
%    \end{macrocode}

% Manually include selected file,
% otherwise process as usual:
%    \begin{macrocode}
\ifchilddocmanual
\section*{part `\childdocname'}
\input{\childdocname}
\else
%    \end{macrocode}

% Include the two chapters:
%    \begin{macrocode}
\include{cdocsch1}
\include{cdocsch2}
%    \end{macrocode}

% Include the two parts unless only chapters should be displayed:
%    \begin{macrocode}
\ifchilddoc\else
\section{part three}
\input{cdocspt3}
\section{part four}
\input{cdocspt4}
\fi
%    \end{macrocode}

% Process as usual until here:
%    \begin{macrocode}
\fi
%    \end{macrocode}

% End of document body:
%    \begin{macrocode}
\end{document}
%    \end{macrocode}
%\iffalse
%</samplemain>
%\fi
%
% %%%%%%%%%%%%%%%%%%%%%%%%%%%%%%%%%%%%%%
% \paragraph{Chapter Include Files.}
%
% The include files are called |cdocsch1.tex| and |cdocsch2.tex|.
%
%\iffalse
%<*samplechap1|samplechap2>
%\fi

% Optional override for |\version| flag:
%    \begin{macrocode}
%%\providecommand{\version}{final}
%    \end{macrocode}

% Include the main document:
%    \begin{macrocode}
\input{childdoc.def}
\childdocof{cdocsamp}
%    \end{macrocode}

%\iffalse
%</samplechap1|samplechap2>
%\fi
%
%\iffalse
%<*samplechap1>
%\fi
% Some text for chapter 1:
%    \begin{macrocode}
\section{one}
some text in chapter one
%    \end{macrocode}

%\iffalse
%</samplechap1>
%\fi
% Some text for chapter 2:
%\iffalse
%<*samplechap2>
%\fi
%    \begin{macrocode}
\section{two}
more text in chapter two
%    \end{macrocode}

%\iffalse
%</samplechap2>
%\fi
%
% %%%%%%%%%%%%%%%%%%%%%%%%%%%%%%%%%%%%%%
% \paragraph{Part Include Files.}
%
% The include files are called |cdocspt3.tex| and |cdocspt4.tex|.
%
%\iffalse
%<*samplepart3|samplepart4>
%\fi

% Optional override for |\version| flag:
%    \begin{macrocode}
%%\providecommand{\version}{final}
%    \end{macrocode}

% Include the main document:
%    \begin{macrocode}
\input{childdoc.def}
\childdocby{cdocsamp}
%    \end{macrocode}

%\iffalse
%</samplepart3|samplepart4>
%\fi
%
%\iffalse
%<*samplepart3>
%\fi
% Some text for part 3:
%    \begin{macrocode}
some text in part three
%    \end{macrocode}

%\iffalse
%</samplepart3>
%\fi
% Some text for part 4:
%\iffalse
%<*samplepart4>
%\fi
%    \begin{macrocode}
more text in part four
%    \end{macrocode}

%\iffalse
%</samplepart4>
%\fi
%
% %%%%%%%%%%%%%%%%%%%%%%%%%%%%%%%%%%%%%%
% \paragraph{Forwarding for a Complete Draft.}
%
% The following forwarding file |cdocsdrf.tex|
% compiles the main document in draft mode:
%\iffalse
%<*sampledraft>
%\fi
%    \begin{macrocode}
\def\version{draft}
\input{childdoc.def}
\childdocforward{cdocsamp}
%    \end{macrocode}

%\iffalse
%</sampledraft>
%\fi
%
% %%%%%%%%%%%%%%%%%%%%%%%%%%%%%%%%%%%%%%
% \paragraph{Forwarding for Final Version of the Chapters.}
%
% The following forwarding files |cdocsfn1.tex| and |cdocsfn2.tex|
% (with identical content)
% compile the final versions of the child documents
% |cdocsch1.tex| and |cdocsch2.tex|, respectively:
%\iffalse
%<*samplefinal>
%\fi
%    \begin{macrocode}
\def\version{final}
\input{childdoc.def}
\childdocforwardprefix[cdocsamp]{cdocsfn}{cdocsch}
%    \end{macrocode}

%\iffalse
%</samplefinal>
%\fi
%
% %%%%%%%%%%%%%%%%%%%%%%%%%%%%%%%%%%%%%%
% \paragraph{Command Line Processing.}
%
% The following three command lines generate the output files
% |cdocscld|, |cdocscl1| and |cdocscl2|
% which should be identical to
% |cdocsdrf|, |cdocsch1| and |cdocsfn2|, respectively:
% \begin{center}
% \begin{tabular}{l}
% |latex -jobname cdocscld \|\\
% |  "\def\version{draft}\input{childdoc.def}\childdocforward{cdocsamp}"|\\
% |latex -jobname cdocscl1 \|\\
% |  "\input{childdoc.def}\childdocforward[cdocsamp]{cdocsch1}"|\\
% |latex -jobname cdocscl2 \|\\
% |  "\def\version{final}\input{childdoc.def}\childdocforward{cdocsch2}"|
% \end{tabular}
% \end{center}
% Note that the trailing backslash on each first line
% merely continues the input to the second line
% (for convenient cut ant paste).
% Furthermore, the command |latex| can be replaced by any
% of its alternative versions such as |pdflatex|.
%
% %%%%%%%%%%%%%%%%%%%%%%%%%%%%%%%%%%%%%%%%%%%%%%%%%%%%%%%%%%%%%%%%%%%%%%%%%%%%%%
% %%%%%%%%%%%%%%%%%%%%%%%%%%%%%%%%%%%%%%%%%%%%%%%%%%%%%%%%%%%%%%%%%%%%%%%%%%%%%%
% \section{Implementation}
%\iffalse
%<*package>
%\fi
%
% This section describes the definitions file |childdoc.def|.

% The definitions cannot be loaded using |\usepackage| or |\RequirePackage|
% which has a mechanism to prevent loading a style file more than once.
% When loading the definitions by means of |\input|
% multiple instances have to be prevented manually:
%\iffalse
%This code needs to be before the `\ProvidesFile' directive
%which is defined at the beginning of this file.
%Therefore it is also placed there and commented out here.
%</package>
%<*discard>
%\fi
%    \begin{macrocode}
\ifdefined\childdocmain\endinput\fi
%    \end{macrocode}
%\iffalse
%</discard>
%<*package>
%\fi
%
% \macro{\ifchilddoc}
% \macro{\ifchilddocmanual}
% The conditional |\ifchilddoc| tells whether a
% child (true) or main (false) document is being compiled.
% The conditional |\ifchilddocmanual| tells whether
% the |\includeonly| mechanism is used (false) or
% the selection of child files must be performed manually (true).
% The definitions initialise to false:
%    \begin{macrocode}
\newif\ifchilddoc
\newif\ifchilddocmanual
%    \end{macrocode}

% \macro{\childdocname}
% \macro{\childdocjob}
% The macro |\childdocname| stores the name of the main document
% to be compiled. The macro |\childdocjob| stores the name of
% the document on which the \LaTeX{} compiler was originally invoked.
% The content of |\jobname| cannot be compared
% to filenames specified in the source due to different catcodes.
% The following code rescans |\jobname|, stores the result
% in |\childdocname| and saves a copy in |\childdocjob|:
%    \begin{macrocode}
\edef\childdocname{\scantokens\expandafter{\jobname\noexpand}}
\let\childdocjob\childdocname
%    \end{macrocode}

% \macro{\childdocdisable}
% The macro |\childdocdisable| prevents the main file
% from being processed more than once.
% At this stage, the main document command |\childdocmain|
% is assumed to be called once again where it should do nothing.
% Any subsequent call to it should prevent
% a secondary processing of the main document
% It overwrites the forwarding commands
% |\childdocof| and |\childdocforward|
% with empty macros to prevent further inclusions of the main document:
%    \begin{macrocode}
\newcommand{\childdocdisable}
{
  \renewcommand{\childdocmain}[1]{\renewcommand{\childdocmain}[1]{\endinput}}
  \renewcommand{\childdocof}[1]{}
  \renewcommand{\childdocby}[2][]{}
  \renewcommand{\childdocforward}[2][]{}
  \renewcommand{\childdocdisable}{}
}
%    \end{macrocode}

% \macro{\childdocmain}
% The macro |\childdocmain| is to be called at the top of the main file
% with nothing or the main filename (without extension) as argument.
% First, it breaks loops.
% If the argument is not empty and does not match |\childdocname|
% (which is set by the first inclusion of |childdoc.def|),
% |\ifchilddoc| is set to true, |\includeonly| is applied to the child file
% and |\jobname| is set to the main file
% (for proper handling of |.aux| files):
%    \begin{macrocode}
\newcommand{\childdocmain}[1]
{
  \childdocdisable\childdocmain{}
  \if?#1?\else
    \begingroup
      \def\childdoctmp{#1}
      \ifx\childdoctmp\childdocname
        \def\childdoctmp{}
      \else
        \def\childdoctmp
        {
          \childdoctrue
          \includeonly{\childdocname}
          \def\childdocjob{#1}
          \def\jobname{#1}
        }
      \fi
      \expandafter
    \endgroup
    \childdoctmp
  \fi
}
%    \end{macrocode}

% \macro{\childdocof}
% The command |\childdocof| redirects
% compilation to the main file |#1|.
%    \begin{macrocode}
\newcommand{\childdocof}[1]
{
  \childdocdisable
  \childdoctrue
  \includeonly{\childdocname}
  \def\jobname{#1}
  \def\childdocjob{#1}
  \input{#1}
}
%    \end{macrocode}

% \macro{\childdocby}
% The command |\childdocby| ....
%    \begin{macrocode}
\newcommand{\childdocby}[2][]
{
  \childdocdisable
  \childdoctrue
  \childdocmanualtrue
  \if?#1?\else
    \def\jobname{#2}
  \fi
  \def\childdocjob{#2}
  \input{#2}
  \endinput
}
%    \end{macrocode}

% \macro{\childdocforward}
% The command |\childdocforward| redirects
% compilation to the main file or
% (if the optional argument is given) a child file.
% Parameters are set as if the main file
% or a child file starting with |\childdocof| was compiled.
% Then compilation is handed over to the main file:
%    \begin{macrocode}
\newcommand{\childdocforward}[2][]
{
  \begingroup
    \if?#1?
      \def\childdoctmp
      {
        \def\childdocname{#2}
        \def\childdocjob{#2}
        \def\jobname{#2}
        \input{#2}
        \endinput
      }
    \else
      \def\childdoctmp
      {
        \childdocdisable
        \def\childdocname{#2}
        \childdoctrue
        \includeonly{#2}
        \def\childdocjob{#1}
        \def\jobname{#1}
        \input{#1}
        \endinput
      }
    \fi
    \expandafter
  \endgroup
  \childdoctmp
}
%    \end{macrocode}

% \macro{\childdocforwardprefix}
% The command |\childdocforwardprefix| redirects
% compilation to the main or a child file by means of a pattern.
% The prefix |#1| in the current filename is replaced by |#2|
% and the suffix of the current filename is kept
% (it is assumed that the filename does not contain the substring `|~~~|'
% which is used as a delimiter).
% Compilation is handed over to the new file by |\childdocforward|:
%    \begin{macrocode}
\newcommand{\childdocforwardprefix}[3][]
{
  \begingroup
    \def\childdocextract #2##1~~~{\def\childdoctmp{\childdocforward[#1]{#3##1}}}
    \expandafter\childdocextract\childdocname~~~
    \expandafter
  \endgroup
  \childdoctmp
}
%    \end{macrocode}

% \macro{\childdoc}
% The deprecated macro |\childdoc| is a legacy version of |\childdocmain|:
%    \begin{macrocode}
\newcommand{\childdoc}{\childdocmain}
%    \end{macrocode}

% \macro{\childdocredirect}
% The deprecated macro |\childdocredirect| is a legacy version
% of |\childdocforward| and |\childdocforwardprefix|:
%    \begin{macrocode}
\newcommand{\childdocredirect}[2][]
{
  \begingroup
    \if?#1?
      \def\childdoctmp{\childdocforward{#2}}
    \else
      \def\childdoctmp{\childdocforwardprefix{#1}{#2}}
    \fi
    \expandafter
  \endgroup
  \childdoctmp
}
%    \end{macrocode}

%\iffalse
%</package>
%\fi
%
\endinput

\childdocby{cdocsamp}
%    \end{macrocode}

%\iffalse
%</samplepart3|samplepart4>
%\fi
%
%\iffalse
%<*samplepart3>
%\fi
% Some text for part 3:
%    \begin{macrocode}
some text in part three
%    \end{macrocode}

%\iffalse
%</samplepart3>
%\fi
% Some text for part 4:
%\iffalse
%<*samplepart4>
%\fi
%    \begin{macrocode}
more text in part four
%    \end{macrocode}

%\iffalse
%</samplepart4>
%\fi
%
% %%%%%%%%%%%%%%%%%%%%%%%%%%%%%%%%%%%%%%
% \paragraph{Forwarding for a Complete Draft.}
%
% The following forwarding file |cdocsdrf.tex|
% compiles the main document in draft mode:
%\iffalse
%<*sampledraft>
%\fi
%    \begin{macrocode}
\def\version{draft}
% \iffalse
%
% childdoc.dtx Copyright (C) 2017-2018 Niklas Beisert
%
% This work may be distributed and/or modified under the
% conditions of the LaTeX Project Public License, either version 1.3
% of this license or (at your option) any later version.
% The latest version of this license is in
%   http://www.latex-project.org/lppl.txt
% and version 1.3 or later is part of all distributions of LaTeX
% version 2005/12/01 or later.
%
% This work has the LPPL maintenance status `maintained'.
%
% The Current Maintainer of this work is Niklas Beisert.
%
% This work consists of the files childdoc.dtx and childdoc.ins
% and the derived files childdoc.def and cdocsamp.tex with
% cdocsch1.tex, cdocsch2.tex, cdocsdrf.tex, cdocsfn1.tex, cdocsfn2.tex.
%
%<package>\ifdefined\childdocmain\endinput\fi
%<package>\ProvidesFile{childdoc.def}[2018/12/30 v2.0 child document driver]
%<samplemain>\ProvidesFile{cdocsamp.tex}[2018/12/30 v2.0 sample for childdoc]
%<*driver>
%\ProvidesFile{childdoc.drv}[2018/12/30 v2.0 childdoc reference manual file]
\PassOptionsToClass{10pt,a4paper}{article}
\documentclass{ltxdoc}

\usepackage[margin=35mm]{geometry}
\usepackage{hyperref}
\usepackage{hyperxmp}
\usepackage[usenames]{color}

\hypersetup{colorlinks=true}
\hypersetup{pdfstartview=FitH}
\hypersetup{pdfpagemode=UseNone}
\hypersetup{pdfsource={}}
\hypersetup{pdflang={en-UK}}
\hypersetup{pdfcopyright={Copyright 2017-2018 Niklas Beisert.
  This work may be distributed and/or modified under the
  conditions of the LaTeX Project Public License, either version 1.3
  of this license or (at your option) any later version.}}
\hypersetup{pdflicenseurl={http://www.latex-project.org/lppl.txt}}
\hypersetup{pdfcontactaddress={ETH Zurich, ITP, HIT K,
  Wolfgang-Pauli-Strasse 27}}
\hypersetup{pdfcontactpostcode={8093}}
\hypersetup{pdfcontactcity={Zurich}}
\hypersetup{pdfcontactcountry={Switzerland}}
\hypersetup{pdfcontactemail={nbeisert@itp.phys.ethz.ch}}
\hypersetup{pdfcontacturl={http://people.phys.ethz.ch/\xmptilde nbeisert/}}

\newcommand{\secref}[1]{\hyperref[#1]{section \ref*{#1}}}

\parskip1ex
\parindent0pt
\let\olditemize\itemize
\def\itemize{\olditemize\parskip0pt}

\begin{document}

\title{The \textsf{childdoc} Package}
\hypersetup{pdftitle={The childdoc Package}}
\author{Niklas Beisert\\[2ex]
  Institut f\"ur Theoretische Physik\\
  Eidgen\"ossische Technische Hochschule Z\"urich\\
  Wolfgang-Pauli-Strasse 27, 8093 Z\"urich, Switzerland\\[1ex]
  \href{mailto:nbeisert@itp.phys.ethz.ch}
  {\texttt{nbeisert@itp.phys.ethz.ch}}}
\hypersetup{pdfauthor={Niklas Beisert}}
\hypersetup{pdfsubject={Manual for the LaTeX2e Package childdoc}}
\date{30 December 2018, \textsf{v2.0}}
\maketitle

\begin{abstract}\noindent
\textsf{childdoc} is a \LaTeXe{} package
that enables the direct compilation
of document sections included by |\include|
to individual files.
\end{abstract}

\begingroup
\parskip0ex
\tableofcontents
\endgroup

%%%%%%%%%%%%%%%%%%%%%%%%%%%%%%%%%%%%%%%%%%%%%%%%%%%%%%%%%%%%%%%%%%%%%%%%%%%%%%%%
%%%%%%%%%%%%%%%%%%%%%%%%%%%%%%%%%%%%%%%%%%%%%%%%%%%%%%%%%%%%%%%%%%%%%%%%%%%%%%%%
\section{Introduction}

\LaTeX{} provides a mechanism to structure a large document (such as a book)
into a main file and several child files (containing the chapters)
using the |\include| command.
This mechanism is beneficial for documents
which span hundreds of pages in order to
make the source file(s) more manageable.
Moreover, compilation can be restricted to
selected child files by means of the |\includeonly| command.
The latter feature can be used to reduce the compilation time while editing
(this was significantly more useful in the earlier days of \LaTeX{})
or to generate a smaller document which is easier to navigate.
Another application of |\includeonly| is to generate
documents consisting of selected parts of the complete document.

However, there are a few drawbacks of the plain |\include| mechanism:
\begin{itemize}
\item
The child files cannot be compiled on their own,
they can only be compiled via the main file.
A naive editing environment
(such as a text editor with an option
to have the current file processed by \LaTeX)
may require one to switch to the main file before compiling;
attempting to compile the child file produces errors.
\item
The main file must be modified (each time)
to adjust the |\includeonly| command
to the present needs. This easily leaves the main file in a messy state.
\item
The generated document will always carry the filename
of the main document. This is inconvenient if
several child files are to be compiled and
to be kept for distribution.
\end{itemize}

The present package provides a simple interface
to make child files individually compilable by \LaTeX{}.
Compiling a child file then has the same effect as compiling
the main file with an |\includeonly| command
to select the appropriate child.
Moreover the generated document will carry the name of the child
rather than the main file.
This resolves all three above issues.

This feature is meant to make the editing of books,
thesis documents and lecture notes somewhat more convenient.
However, the package can also be used efficiently for
composing a series of documents (such as exercise sheets)
which are typically distributed individually.
It then assists the author in generating the individual documents
(potentially in different versions)
as well as a document containing the collected series.
Another application is in developing style files
or other kinds of included material
where compilation of the style file could redirect
to a sample or test file.

%%%%%%%%%%%%%%%%%%%%%%%%%%%%%%%%%%%%%%%%%%%%%%%%%%%%%%%%%%%%%%%%%%%%%%%%%%%%%%%%
%%%%%%%%%%%%%%%%%%%%%%%%%%%%%%%%%%%%%%%%%%%%%%%%%%%%%%%%%%%%%%%%%%%%%%%%%%%%%%%%
\section{Usage}

First of all, the package \textsf{childdoc} is \emph{not} a standard
\LaTeXe{} |.sty| style file! Therefore it needs to be invoked in
a non-standard way.

%%%%%%%%%%%%%%%%%%%%%%%%%%%%%%%%%%%%%%%%%%%%%%%%%%%%%%%%%%%%%%%%%%%%%%%%%%%%%%%%
\subsection{Included Files}
\label{sec:include}

%%%%%%%%%%%%%%%%%%%%%%%%%%%%%%%%%%%%%%%%
\DescribeMacro{\childdocmain}
To use the package, add the commands
\begin{center}
\begin{tabular}{l}
|\input{childdoc.def}|\\
|\childdocmain{}|\\
\end{tabular}
\end{center}
at the very top of the main \LaTeX{} file,
in particular \emph{before} the |\documentclass| statement!
The argument of |\childdocmain| should be left empty
(but it must be present).

%%%%%%%%%%%%%%%%%%%%%%%%%%%%%%%%%%%%%%%%
\DescribeMacro{\childdocof}
Furthermore, add the commands
\begin{center}
\begin{tabular}{l}
|\input{childdoc.def}|\\
|\childdocof{|\textit{main}|}|\\
\end{tabular}
\end{center}
at the top of every child file \textit{child}
which is included by |\include{|\textit{child}|}|
from within the main file
(or at least for those files to be compiled individually).
The argument \textit{main} must be the filename of the main file.

There are a couple of
considerations in setting up the main and child documents:

%%%%%%%%%%%%%%%%%%%%%%%%%%%%%%%%%%%%%%%%
\paragraph{Restrictions.}

Please note the following restrictions:
\begin{itemize}
\item
|\childdocmain| must be called with one argument \textit{main}
to ensure compatibility with earlier version of the package.
It must either be empty (|\childdocmain{}|)
or precisely match the filename of the main file in which it is specified.
See \secref{sec:detection} for further information.
\item
The filename \textit{main} must be specified without the |.tex| extension.
\item
The filename \textit{main} is case sensitive
(even in case-insensitive file systems)
due to internal string comparison.
\item
The argument \textit{main} should be fully expanded, it cannot be a macro.
\item
Subdirectories and special characters should be avoided in filenames.
\item
The command |\childdocmain{|\textit{main}|}| must be followed by a whitespace.
It should not be followed immediately by another command
or by a comment mark `|%|'.
This is because the \TeX{} parser reads the token immediately following
the argument of |\childdocmain| and puts it
at the beginning of every child section;
however, a white\-space is ignored.
\end{itemize}

%%%%%%%%%%%%%%%%%%%%%%%%%%%%%%%%%%%%%%%%
\paragraph{Content of Main File.}

It is advisable to place all content in the child files included by |\include|.
Any output contained in the main file will appear in all child documents
unless suppressed manually;
it cannot be suppressed automatically by the |\includeonly| directive
and thus should normally be avoided.
A method to include some content in the main file
by means of conditional processing is described in \secref{sec:conditional}.

%%%%%%%%%%%%%%%%%%%%%%%%%%%%%%%%%%%%%%%%
\paragraph{Page Numbering.}

When only a part of the document is compiled,
the appropriate numbering of pages
(as well as other status parameters)
is determined from the |.aux| files.
The latter contain information from previous passes.
However this information needs to propagate through
all intermediate child documents.
Therefore the page numbering in child documents may well
be inconsistent until the complete document is compiled at least once.

A useful (if unconventional) way to always ensure a consistent
page numbering is to restart the numbering in each child document
and denote the pages by `\textit{child}|.|\textit{page}'
where \textit{child} represents the chapter/section number of the child file.
This can be achieved by the command
|\numberwithin{page}{|\textit{child}|}|
of the \textsf{amsmath} package
where \textit{child} can be |chapter| or |section|
depending on the chosen structuring.
Alternatively, one can modify the macro |\thepage| appropriately
and reset the counter |page| at the start of each child file.

%%%%%%%%%%%%%%%%%%%%%%%%%%%%%%%%%%%%%%%%%%%%%%%%%%%%%%%%%%%%%%%%%%%%%%%%%%%%%%%%
\subsection{Conditional Processing}
\label{sec:conditional}

The package provides a mechanism to compile different versions
of a document. To customise the versions further some conditional processing
can come in handy to distinguish which version is being compiled.
The package provides two macros to describe the compilation context:

%%%%%%%%%%%%%%%%%%%%%%%%%%%%%%%%%%%%%%%%
\DescribeMacro{\ifchilddoc}
The conditional |\ifchilddoc| distinguishes between the compilation of
child documents and the main document:
%
\begin{center}
|\ifchilddoc |\textit{child-code}| |[|\||else |\textit{main-code}]| \||fi|
\end{center}

%%%%%%%%%%%%%%%%%%%%%%%%%%%%%%%%%%%%%%%%
\DescribeMacro{\childdocname}
\DescribeMacro{\childdocjob}
The macro |\childdocname| contains the filename (without extension)
of the main or child file being processed.
Note that |\childdocjob| will always contain the name of the main file.

%%%%%%%%%%%%%%%%%%%%%%%%%%%%%%%%%%%%%%%%
\paragraph{Title Page.}

Conditional processing can be used to include a title or banner page
in the main document when proper precautions are taken.
Importantly, the code in the main file should ensure that the page counter
(as well as other status parameters which are stored in the |.aux| files)
takes the same value after the conditional processing.
Otherwise the page numbers may take divergent values
depending on which part is compiled.

For example, a title page could be declared by:
%
\begin{center}
\begin{tabular}{l}
|\ifchilddoc\||else|\\
|\addtocounter{page}{-1}|\\
\textit{code for title page}\\
|\newpage|\\
|\||fi|
\end{tabular}
\end{center}
%
A banner page for the child documents can be generated by:
%
\begin{center}
\begin{tabular}{l}
|\ifchilddoc|\\
|\addtocounter{page}{-1}|\\
\textit{code for banner page}\\
|\newpage|\\
|\||fi|
\end{tabular}
\end{center}
%
Here one could write a message such as:
\begin{center}
|This is the part \childdocname{} of \childdocjob{}.|
\end{center}

%%%%%%%%%%%%%%%%%%%%%%%%%%%%%%%%%%%%%%%%%%%%%%%%%%%%%%%%%%%%%%%%%%%%%%%%%%%%%%%%
\subsection{Flags}
\label{sec:flags}

The package makes it easy to generate different versions
of the main or child documents.
To this end compilation flags can be defined
and assigned different default values.
They will be particularly useful in conjunction
with the forwarding mechanism described in \secref{sec:forward}.

For example, it may be useful to have a flag |\version|
which can be set to |draft| or |final|.
The document source will contain some conditional code
depending on the value of |\version|.
Suppose further, the flag should default to |final| for the main file
and to |draft| for child files
which is a natural assignment for editing the document.
This is achieved by placing the following code
in the preamble of the main document
(below the |\childdocmain| directive):
%
\begin{center}
\begin{tabular}{l}
|\ifchilddoc|\\
|\providecommand{\version}{draft}|\\
|\||else|\\
|\providecommand{\version}{final}|\\
|\||fi|
\end{tabular}
\end{center}
%
The definition by |\providecommand| makes sure
that previous definitions are not overwritten.
Further statements |\providecommand{\version}{...}|
can thus be added before the above code to override it.

For the main file, one might add a line
(between |\childdocmain| and the above block)
%
\begin{center}
|%\ifchilddoc\||else\providecommand{\version}{draft}\||fi|
\end{center}
%
which can be uncommented to produce a draft version.
Likewise one can add a line to the very top of a child file
(above the |\childdocof{|\textit{main}|}| directive)
%
\begin{center}
|%\providecommand{\version}{final}|
\end{center}
%
which can be uncommented to produce the final version of this child document.

%%%%%%%%%%%%%%%%%%%%%%%%%%%%%%%%%%%%%%%%%%%%%%%%%%%%%%%%%%%%%%%%%%%%%%%%%%%%%%%%
\subsection{Forwarding}
\label{sec:forward}

Different versions of the main or child documents
using compilation flags as described in \secref{sec:flags}
can be (permanently) stored in different files
for convenient compilation, viewing and distribution.
To this end, the package defines a command
to pass on compilation to a different file:

%%%%%%%%%%%%%%%%%%%%%%%%%%%%%%%%%%%%%%%%
\DescribeMacro{\childdocforward}
The command |\childdocforward| redirects processing to
another source file:
%
\begin{center}
\begin{tabular}{l}
|\input{childdoc.def}|\\
|\childdocforward[|\textit{main}|]{|\textit{dest}|}|\\
\end{tabular}
\end{center}
%
The argument \textit{dest} is the destination file
(without extension).
It should be the main file or one of the child files.
Note that further \textsf{childdoc} directives
such as |\childdocof| and |\childdocforward|
in the indicated file will be processed in this form.
The optional argument \textit{main}
passes on directly to the main file \textit{main}
while pretending to compile the child \textit{dest}.
This form behaves as if \textit{dest}
issues |\childdocof{|\textit{main}|}| right away,
and no further \textsf{childdoc} directives will be processed.

%%%%%%%%%%%%%%%%%%%%%%%%%%%%%%%%%%%%%%%%
\DescribeMacro{\...prefix}
In the alternative form |\childdocforwardprefix|,
%
\begin{center}
\begin{tabular}{l}
|\input{childdoc.def}|\\
|\childdocforwardprefix[|\textit{main}|]{|\textit{prefix}|}{|\textit{dest}|}|
\end{tabular}
\end{center}
%
the destination file is determined by a pattern
depending on the current file:
To make this work, the current file must be called
`{\textit{prefix}\hspace{0.2em}\textit{suffix}}'
with \textit{prefix} matching precisely the argument.
Processing is then passed on to the file
`{\textit{dest}\hspace{0.2em}\textit{suffix}}'.
Surely, the same effect is achieved by
directly specifying the
argument `{\textit{dest}\hspace{0.2em}\textit{suffix}}'
in the first form.
However, that requires to set up a different file
for each child. With the alternative form of the command
all these files can have exactly the same content
which simplifies setting them up and maintaining them.

For example, the following file |draft.tex|
with a compilation flag |\version| as described in \secref{sec:flags}
compiles the main document as a draft:
%
\begin{center}
\begin{tabular}{l}
|\def\version{draft}|\\
|\input{childdoc.def}|\\
|\childdocforward{|\textit{main}|}|
\end{tabular}
\end{center}
%
Likewise, the following files |final|\textit{nn}|.tex|
compile the final version of the child document
|child|\textit{nn}|.tex|:
%
\begin{center}
\begin{tabular}{l}
|\def\version{final}|\\
|\input{childdoc.def}|\\
|\childdocforwardprefix{final}{child}|
\end{tabular}
\end{center}
%

Note that when several versions of a main file and/or of each child file
are to be generated, it may be convenient to set up a |Makefile| or
shell script to automatise the process.

%%%%%%%%%%%%%%%%%%%%%%%%%%%%%%%%%%%%%%%%%%%%%%%%%%%%%%%%%%%%%%%%%%%%%%%%%%%%%%%%
\subsection{Command Line Processing}
\label{sec:commandline}

The effect of redirection files can also be achieved by invoking
the \LaTeX{} compiler with a more elaborate command line.
Most conveniently this should be done as part
of a shell script or a |Makefile|.

When using \textsf{childdoc} in the main file, the following
command lines effectively perform a redirection
(note that depending on the shell being used,
backslashes may have to be doubled: `|\|' $\to$ `|\\|'):
%
\begin{center}
|... -jobname "|\textit{target}|" |\\|"|[\textit{flags}]%
|\input{childdoc.def}\childdocforward[|\textit{main}|]{|\textit{dest}|}"|
\end{center}
%
Here \textit{target} is the name of the output file,
\textit{main} is the name of the main file
and \textit{dest} is the name of the main or child file to be processed
(all filenames without extensions).
The optional argument \textit{main} can be omitted
if \textit{main} matches \textit{dest}.
Optionally, compilation \textit{flags} can be defined via |\def| commands.
This command line makes the \TeX{} engine believe
it is compiling the file \textit{target}
whose content is specified as the latter parameter.
The provided code then forwards the processing to
\textit{main} or \textit{dest} as described in \secref{sec:forward}.

%%%%%%%%%%%%%%%%%%%%%%%%%%%%%%%%%%%%%%%%%%%%%%%%%%%%%%%%%%%%%%%%%%%%%%%%%%%%%%%%
\subsection{Include by Input}
\label{sec:input}

Including child documents by |\include| has some restrictions by design.
Most notably, the content of a child document always occupies
its own set of pages; pages cannot be shared between child documents.
Usually, this behaviour makes perfect sense
because each child document contain an essential part of the document.
However, in some situations it may be desirable to compose
a document from a collection of parts
without having mandatory page breaks between then.
For this case, the package
provides a mechanism to include parts
by |\input| which can also be processed individually.
However, by construction this mechanism
requires manual handling of the content to be output.

%%%%%%%%%%%%%%%%%%%%%%%%%%%%%%%%%%%%%%%%
\DescribeMacro{\ifchilddocmanual}
The main file should be prepared as usual, see \secref{sec:include}.
However, the document body must make a distinction
between processing of an individual part and of the main document, e.g.:
%
\begin{center}
\begin{tabular}{l}
|\ifchilddocmanual|\\
|\input{\childdocname}|\\
|\||else|\\
\textit{document body with }|\input{|\textit{part}|}|\\
|\||fi|
\end{tabular}
\end{center}
%
The conditional |\ifchilddocmanual| is true whenever
a part to be included by |\input| is being compiled,
and the name of the part is stored in |\childdocname|.

%%%%%%%%%%%%%%%%%%%%%%%%%%%%%%%%%%%%%%%%
\DescribeMacro{\childdocby}
Each part to be included by |\input| should start with:
%
\begin{center}
\begin{tabular}{l}
|\input{childdoc.def}|\\
|\childdocby{|\textit{main}|}|\\
\end{tabular}
\end{center}
%
The directive |\childdocby| is similar to |\childdocof|
described in \secref{sec:include},
but the subsequent selection of content must be done manually.
To that end, both |\ifchilddoc| and |\ifchilddocmanual|
will be true upon processing of a part,
and the name of the part is stored in |\childdocname|.
Note that |\jobname| will be set to the filename of the current part
so that each part receives an individual |.aux| file
that does not interfere with the |.aux| file(s) of the main document.
This behaviour can be altered by the alternative form
|\childdocby[*]{|\textit{main}|}| (with a non-empty optional argument)
which uses the |.aux| file of the main document
by setting |\jobname| to \textit{main}.

%%%%%%%%%%%%%%%%%%%%%%%%%%%%%%%%%%%%%%%%%%%%%%%%%%%%%%%%%%%%%%%%%%%%%%%%%%%%%%%%
\subsection{Driver Development}
\label{sec:driver}

The \textsf{childdoc} mechanism can also be use for the development
of definition files such as \LaTeX{} styles or classes.
This case differs from the above setup with multiple parts
included by |\include| in that no |\includeonly| should be invoked.
This can be achieved by starting the include file
(before |\ProvidesPackage|) with:
%
\begin{center}
\begin{tabular}{l}
|\input{childdoc.def}|\\
|\childdocforward{|\textit{main}|}|\\
\end{tabular}
\end{center}
%
or alternatively with:
%
\begin{center}
\begin{tabular}{l}
|\input{childdoc.def}|\\
|\childdocby{|\textit{main}|}|\\
\end{tabular}
\end{center}
%
Both forms have slightly different effects as described above.
The main file is prepared as usual, see \secref{sec:include}.

%%%%%%%%%%%%%%%%%%%%%%%%%%%%%%%%%%%%%%%%%%%%%%%%%%%%%%%%%%%%%%%%%%%%%%%%%%%%%%%%
\subsection{Legacy Detection}
\label{sec:detection}

The directive |\childdocmain| in the main file can detect
whether the complete document or merely a child is to be compiled
even without using the directive |\childdocof|.
This method is deprecated because it is less robust
and there is no compelling reason to use it;
it is merely provided for backward compatibility
and it may be removed in future versions.

If the detection mechanism is to be used,
it is mandatory to correctly specify
the filename of the main file as the argument of |\childdocmain|:
%
\begin{center}
\begin{tabular}{l}
|\input{childdoc.def}|\\
|\childdocmain{|\textit{main}|}|\\
\end{tabular}
\end{center}
%
If |\jobname| does not match the argument \textit{main} of |\childdocmain|,
it is assumed that |\jobname| points to the child file to be compiled.
When using |\childdocmain| with the main file specified as argument,
it suffices to start a child file
with just |\input{|\textit{main}|}|
without loading of the package and using |\childdocof|.
If instead all processing is done
with the appropriate \textsf{childdoc} directives,
the argument of \textit{main} of |\childdocmain| can be empty.

An alternative version of the command line processing described
in \secref{sec:commandline} using the detection mechanism reads:
%
\begin{center}
|... -jobname "|\textit{target}|" "|[\textit{flags}]%
[|\def\jobname{|\textit{dest}|}|]|\input{|\textit{main}|}"|
\end{center}

%%%%%%%%%%%%%%%%%%%%%%%%%%%%%%%%%%%%%%%%%%%%%%%%%%%%%%%%%%%%%%%%%%%%%%%%%%%%%%%%
\subsection{Manual Code}
\label{sec:manual}

In case one cannot be certain whether the definitions file |childdoc.def|
is installed on the target \TeX{} distribution
and one prefers not to ship it,
it is conceivable to paste a few relevant commands into the sources.

To that end, drop all statements |\input{childdoc.def}|
and perform the replacements as outlined below.
Instead of |\childdocmain{|\textit{main}|}| add the following code
to the top of the main file:
%
\begin{center}
\begin{tabular}{l}
|\||ifdefined\childdocname\endinput\||fi\newif\ifchilddoc|\\
|\edef\childdocname{\scantokens\expandafter{\jobname\noexpand}}|\\
|\def\childdocmain{|\textit{main}|}\||ifx\childdocmain\childdocname\||else|\\
|\childdoctrue\includeonly{\childdocname}\let\jobname\childdocmain\||fi|\\
\end{tabular}
\end{center}
%
Instead of |\childdocof{|\textit{main}|}| just include the main file
at the top of each child file:
%
\begin{center}
|\input{|\textit{main}|}|
\end{center}
%
A simple redirection |\childdocforward{|\textit{dest}|}| is achieved by:
%
\begin{center}
|\def\jobname{|\textit{dest}|}\input{\jobname}|
\end{center}
%
The redirection with prefix
|\childdocforwardprefix[|\textit{prefix}|]{|\textit{dest}|}|
is accomplished by:
%
\begin{center}
\begin{tabular}{l}
|{\edef\jobname{\scantokens\expandafter{\jobname\noexpand}}|\\
|\def\redirectjob |\textit{prefix}|#1~~~{\gdef\jobname{|\textit{dest}|#1}}|\\
|\expandafter\redirectjob\jobname~~~}\input{\jobname}|
\end{tabular}
\end{center}

In an alternative approach,
child documents can be compiled by a specific command line
without additional code or specific definitions:
%
\begin{center}
|... -jobname "|\textit{target}|" "|[\textit{flags}]%
|\includeonly{|\textit{dest}|}\input{|\textit{main}|}"|
\end{center}
%

%%%%%%%%%%%%%%%%%%%%%%%%%%%%%%%%%%%%%%%%%%%%%%%%%%%%%%%%%%%%%%%%%%%%%%%%%%%%%%%%
%%%%%%%%%%%%%%%%%%%%%%%%%%%%%%%%%%%%%%%%%%%%%%%%%%%%%%%%%%%%%%%%%%%%%%%%%%%%%%%%
\section{Information}

%%%%%%%%%%%%%%%%%%%%%%%%%%%%%%%%%%%%%%%%%%%%%%%%%%%%%%%%%%%%%%%%%%%%%%%%%%%%%%%%
\subsection{Copyright}

Copyright \copyright{} 2017--2018 Niklas Beisert

This work may be distributed and/or modified under the
conditions of the \LaTeX{} Project Public License, either version 1.3
of this license or (at your option) any later version.
The latest version of this license is in
  \url{http://www.latex-project.org/lppl.txt}
and version 1.3 or later is part of all distributions of \LaTeX{}
version 2005/12/01 or later.

This work has the LPPL maintenance status `maintained'.

The Current Maintainer of this work is Niklas Beisert.

This work consists of the files |README.txt|, |childdoc.ins| and |childdoc.dtx|
as well as the derived files |childdoc.def|, |cdocsamp.tex|
with |cdocsch1.tex|, |cdocsch2.tex|, |cdocspt3.tex|, |cdocspt4.tex|,
|cdocsdrf.tex|, |cdocsfn1.tex|, |cdocsfn2.tex|
as well as |childdoc.pdf|.

%%%%%%%%%%%%%%%%%%%%%%%%%%%%%%%%%%%%%%%%%%%%%%%%%%%%%%%%%%%%%%%%%%%%%%%%%%%%%%%%
\subsection{Files and Installation}

The package consists of the files:
%
\begin{center}
\begin{tabular}{ll}
    |README.txt|   & readme file \\
    |childdoc.ins| & installation file \\
    |childdoc.dtx| & source file \\
    |childdoc.def| & definition file \\
    |cdocsamp.tex| & sample main file \\
    |cdocsch1.tex| & sample include file \\
    |cdocsch2.tex| & sample include file \\
    |cdocspt3.tex| & sample part file \\
    |cdocspt4.tex| & sample part file \\
    |cdocsdrf.tex| & sample redirection file \\
    |cdocsfn1.tex| & sample redirection file \\
    |cdocsfn2.tex| & sample redirection file \\
    |childdoc.pdf| & manual
\end{tabular}
\end{center}
%
The distribution consists of the files
|README.txt|, |childdoc.ins| and |childdoc.dtx|.
%
\begin{itemize}
\item
Run (pdf)\LaTeX{} on |childdoc.dtx|
to compile the manual |childdoc.pdf| (this file).
\item
Run \LaTeX{} on |childdoc.ins| to create the definitions file |childdoc.def|
and the sample |cdocsamp.tex| with include files
|cdocsch1.tex|, |cdocsch2.tex|, |cdocspt3.tex|, |cdocspt4.tex|,
|cdocsdrf.tex|, |cdocsfn1.tex|, |cdocsfn2.tex|.
Then copy the file |childdoc.def| to an appropriate directory of your \LaTeX{}
distribution, e.g.\ \textit{texmf-root}|/tex/latex/childdoc|.
\end{itemize}

%%%%%%%%%%%%%%%%%%%%%%%%%%%%%%%%%%%%%%%%%%%%%%%%%%%%%%%%%%%%%%%%%%%%%%%%%%%%%%%%
\subsection{Related CTAN Packages}

There are several other packages which offer a similar functionality:
%
\begin{itemize}
\item
The packages
\href{http://ctan.org/pkg/docmute}{\textsf{docmute}},
\href{http://ctan.org/pkg/includex}{\textsf{includex}} and
\href{http://ctan.org/pkg/standalone}{\textsf{standalone}}
provide commands to include only the document body of
a child file thus allowing both files to be compiled individually.
\item
The packages \href{http://ctan.org/pkg/subdocs}{\textsf{subdocs}}
and \href{http://ctan.org/pkg/subfiles}{\textsf{subfiles}}
provide structures in which the main and child documents can be
encapsulated and allowing them to be compiled individually.
The inclusion mechanism is different from the conventional |\include|.
\item
The package \href{http://ctan.org/pkg/combine}{\textsf{combine}}
is an elaborate solution to combine several documents into one.
\end{itemize}
%
See also the CTAN topic \href{http://ctan.org/topic/subdocs}{\textsf{subdocs}}
for further related packages.
The present package differs from the above solutions in that
a document structure constructed with the conventional |\include| mechanism
just needs two extra commands at the top of every file
such that all constituent files can be compiled individually.

%%%%%%%%%%%%%%%%%%%%%%%%%%%%%%%%%%%%%%%%%%%%%%%%%%%%%%%%%%%%%%%%%%%%%%%%%%%%%%%%
%\subsection{Feature Suggestions}
%
%The following is a list of features which may be useful for future
%versions of this package:
%%
%\begin{itemize}
%\item
%\ldots
%\end{itemize}

%%%%%%%%%%%%%%%%%%%%%%%%%%%%%%%%%%%%%%%%%%%%%%%%%%%%%%%%%%%%%%%%%%%%%%%%%%%%%%%%
\subsection{Revision History}

%%%%%%%%%%%%%%%%%%%%%%%%%%%%%%%%%%%%%%%%
\paragraph{v2.0:} 2018/12/30

\begin{itemize}
\item
immediate forward processing
\item
added |\childdocby| mechanism
\item
manual restructured
\end{itemize}

%%%%%%%%%%%%%%%%%%%%%%%%%%%%%%%%%%%%%%%%
\paragraph{v1.6:} 2018/01/17

\begin{itemize}
\item
application for development of include files
\item
corrections to manual
\end{itemize}

%%%%%%%%%%%%%%%%%%%%%%%%%%%%%%%%%%%%%%%%
\paragraph{v1.5:} 2017/05/21

\begin{itemize}
\item
more complete structuring introduced
\item
|\childdocof| introduced
\item
|\childdoc| renamed to |\childdocmain|
\item
|\childredirect| renamed to |\childdocforward| and |\childdocforwardprefix|
and functionality expanded
\end{itemize}

%%%%%%%%%%%%%%%%%%%%%%%%%%%%%%%%%%%%%%%%
\paragraph{v1.0:} 2017/04/27

\begin{itemize}
\item
manual and install package
\item
first version published on CTAN
\end{itemize}

%%%%%%%%%%%%%%%%%%%%%%%%%%%%%%%%%%%%%%%%
\paragraph{v0.6:} 2017/04/26

\begin{itemize}
\item
redirection mechanism added
\end{itemize}

%%%%%%%%%%%%%%%%%%%%%%%%%%%%%%%%%%%%%%%%
\paragraph{v0.5:} 2017/04/26

\begin{itemize}
\item
functionality in definition file
\end{itemize}


%%%%%%%%%%%%%%%%%%%%%%%%%%%%%%%%%%%%%%%%%%%%%%%%%%%%%%%%%%%%%%%%%%%%%%%%%%%%%%%%
%%%%%%%%%%%%%%%%%%%%%%%%%%%%%%%%%%%%%%%%%%%%%%%%%%%%%%%%%%%%%%%%%%%%%%%%%%%%%%%%
%%%%%%%%%%%%%%%%%%%%%%%%%%%%%%%%%%%%%%%%%%%%%%%%%%%%%%%%%%%%%%%%%%%%%%%%%%%%%%%%
\appendix

\settowidth\MacroIndent{\rmfamily\scriptsize 000\ }

 \DocInput{childdoc.dtx}

\end{document}
%</driver>
% \fi
%
% %%%%%%%%%%%%%%%%%%%%%%%%%%%%%%%%%%%%%%%%%%%%%%%%%%%%%%%%%%%%%%%%%%%%%%%%%%%%%%
% %%%%%%%%%%%%%%%%%%%%%%%%%%%%%%%%%%%%%%%%%%%%%%%%%%%%%%%%%%%%%%%%%%%%%%%%%%%%%%
% \section{Sample}
%\iffalse
%<*samplemain>
%\fi
%
% The following presents a sample document
% with two chapters, two parts, a title page,
% a compile flag as well as three forwarding files to set the flag.
% It consists of eight |.tex| files:
% \begin{center}
% \begin{tabular}{ll}
% |cdocsamp.tex|&main file\\
% |cdocsch1.tex|&include file for chapter 1\\
% |cdocsch2.tex|&include file for chapter 2\\
% |cdocspt3.tex|&include file for part 3\\
% |cdocspt4.tex|&include file for part 4\\
% |cdocsdrf.tex|&forwarding file for main file in draft mode\\
% |cdocsfi1.tex|&forwarding file for final version of chapter 1\\
% |cdocsfi2.tex|&forwarding file for final version of chapter 2\\
% \end{tabular}
% \end{center}
% Each of the eight files can be compiled directly by the \LaTeX{} compiler.
%
% %%%%%%%%%%%%%%%%%%%%%%%%%%%%%%%%%%%%%%
% \paragraph{Main File.}
%
% The main file is called |cdocsamp.tex|.
%
% Load the \textsf{childdoc} definitions and
% declare the filename for the main document:
%    \begin{macrocode}
\input{childdoc.def}
\childdocmain{}
%    \end{macrocode}

% Optional override for |\version| flag:
%    \begin{macrocode}
%%\ifchilddoc\else\providecommand{\version}{draft}\fi
%    \end{macrocode}

% Define the default values for the |\version| flag
% (|final| for the main file and |draft| for childs):
%    \begin{macrocode}
\ifchilddoc
\providecommand{\version}{draft}
\else
\providecommand{\version}{final}
\fi
%    \end{macrocode}

% Load the standard document class:
%    \begin{macrocode}
\documentclass[12pt]{article}
%    \end{macrocode}

% Start the document body:
%    \begin{macrocode}
\begin{document}
%    \end{macrocode}

% Declare a title page.
% Print title, part of document being processed and version flag:
%    \begin{macrocode}
\addtocounter{page}{-1}
\begin{center}
{\LARGE\bfseries{}childdoc example\par}
\vspace{1cm}
\ifchilddoc
\ifchilddocmanual part\else chapter\fi:
`\childdocname' of `\childdocjob'\par
\else
main document: `\childdocjob'\par
\fi
version: \version\par
\end{center}
\newpage
%    \end{macrocode}

% Manually include selected file,
% otherwise process as usual:
%    \begin{macrocode}
\ifchilddocmanual
\section*{part `\childdocname'}
\input{\childdocname}
\else
%    \end{macrocode}

% Include the two chapters:
%    \begin{macrocode}
\include{cdocsch1}
\include{cdocsch2}
%    \end{macrocode}

% Include the two parts unless only chapters should be displayed:
%    \begin{macrocode}
\ifchilddoc\else
\section{part three}
\input{cdocspt3}
\section{part four}
\input{cdocspt4}
\fi
%    \end{macrocode}

% Process as usual until here:
%    \begin{macrocode}
\fi
%    \end{macrocode}

% End of document body:
%    \begin{macrocode}
\end{document}
%    \end{macrocode}
%\iffalse
%</samplemain>
%\fi
%
% %%%%%%%%%%%%%%%%%%%%%%%%%%%%%%%%%%%%%%
% \paragraph{Chapter Include Files.}
%
% The include files are called |cdocsch1.tex| and |cdocsch2.tex|.
%
%\iffalse
%<*samplechap1|samplechap2>
%\fi

% Optional override for |\version| flag:
%    \begin{macrocode}
%%\providecommand{\version}{final}
%    \end{macrocode}

% Include the main document:
%    \begin{macrocode}
\input{childdoc.def}
\childdocof{cdocsamp}
%    \end{macrocode}

%\iffalse
%</samplechap1|samplechap2>
%\fi
%
%\iffalse
%<*samplechap1>
%\fi
% Some text for chapter 1:
%    \begin{macrocode}
\section{one}
some text in chapter one
%    \end{macrocode}

%\iffalse
%</samplechap1>
%\fi
% Some text for chapter 2:
%\iffalse
%<*samplechap2>
%\fi
%    \begin{macrocode}
\section{two}
more text in chapter two
%    \end{macrocode}

%\iffalse
%</samplechap2>
%\fi
%
% %%%%%%%%%%%%%%%%%%%%%%%%%%%%%%%%%%%%%%
% \paragraph{Part Include Files.}
%
% The include files are called |cdocspt3.tex| and |cdocspt4.tex|.
%
%\iffalse
%<*samplepart3|samplepart4>
%\fi

% Optional override for |\version| flag:
%    \begin{macrocode}
%%\providecommand{\version}{final}
%    \end{macrocode}

% Include the main document:
%    \begin{macrocode}
\input{childdoc.def}
\childdocby{cdocsamp}
%    \end{macrocode}

%\iffalse
%</samplepart3|samplepart4>
%\fi
%
%\iffalse
%<*samplepart3>
%\fi
% Some text for part 3:
%    \begin{macrocode}
some text in part three
%    \end{macrocode}

%\iffalse
%</samplepart3>
%\fi
% Some text for part 4:
%\iffalse
%<*samplepart4>
%\fi
%    \begin{macrocode}
more text in part four
%    \end{macrocode}

%\iffalse
%</samplepart4>
%\fi
%
% %%%%%%%%%%%%%%%%%%%%%%%%%%%%%%%%%%%%%%
% \paragraph{Forwarding for a Complete Draft.}
%
% The following forwarding file |cdocsdrf.tex|
% compiles the main document in draft mode:
%\iffalse
%<*sampledraft>
%\fi
%    \begin{macrocode}
\def\version{draft}
\input{childdoc.def}
\childdocforward{cdocsamp}
%    \end{macrocode}

%\iffalse
%</sampledraft>
%\fi
%
% %%%%%%%%%%%%%%%%%%%%%%%%%%%%%%%%%%%%%%
% \paragraph{Forwarding for Final Version of the Chapters.}
%
% The following forwarding files |cdocsfn1.tex| and |cdocsfn2.tex|
% (with identical content)
% compile the final versions of the child documents
% |cdocsch1.tex| and |cdocsch2.tex|, respectively:
%\iffalse
%<*samplefinal>
%\fi
%    \begin{macrocode}
\def\version{final}
\input{childdoc.def}
\childdocforwardprefix[cdocsamp]{cdocsfn}{cdocsch}
%    \end{macrocode}

%\iffalse
%</samplefinal>
%\fi
%
% %%%%%%%%%%%%%%%%%%%%%%%%%%%%%%%%%%%%%%
% \paragraph{Command Line Processing.}
%
% The following three command lines generate the output files
% |cdocscld|, |cdocscl1| and |cdocscl2|
% which should be identical to
% |cdocsdrf|, |cdocsch1| and |cdocsfn2|, respectively:
% \begin{center}
% \begin{tabular}{l}
% |latex -jobname cdocscld \|\\
% |  "\def\version{draft}\input{childdoc.def}\childdocforward{cdocsamp}"|\\
% |latex -jobname cdocscl1 \|\\
% |  "\input{childdoc.def}\childdocforward[cdocsamp]{cdocsch1}"|\\
% |latex -jobname cdocscl2 \|\\
% |  "\def\version{final}\input{childdoc.def}\childdocforward{cdocsch2}"|
% \end{tabular}
% \end{center}
% Note that the trailing backslash on each first line
% merely continues the input to the second line
% (for convenient cut ant paste).
% Furthermore, the command |latex| can be replaced by any
% of its alternative versions such as |pdflatex|.
%
% %%%%%%%%%%%%%%%%%%%%%%%%%%%%%%%%%%%%%%%%%%%%%%%%%%%%%%%%%%%%%%%%%%%%%%%%%%%%%%
% %%%%%%%%%%%%%%%%%%%%%%%%%%%%%%%%%%%%%%%%%%%%%%%%%%%%%%%%%%%%%%%%%%%%%%%%%%%%%%
% \section{Implementation}
%\iffalse
%<*package>
%\fi
%
% This section describes the definitions file |childdoc.def|.

% The definitions cannot be loaded using |\usepackage| or |\RequirePackage|
% which has a mechanism to prevent loading a style file more than once.
% When loading the definitions by means of |\input|
% multiple instances have to be prevented manually:
%\iffalse
%This code needs to be before the `\ProvidesFile' directive
%which is defined at the beginning of this file.
%Therefore it is also placed there and commented out here.
%</package>
%<*discard>
%\fi
%    \begin{macrocode}
\ifdefined\childdocmain\endinput\fi
%    \end{macrocode}
%\iffalse
%</discard>
%<*package>
%\fi
%
% \macro{\ifchilddoc}
% \macro{\ifchilddocmanual}
% The conditional |\ifchilddoc| tells whether a
% child (true) or main (false) document is being compiled.
% The conditional |\ifchilddocmanual| tells whether
% the |\includeonly| mechanism is used (false) or
% the selection of child files must be performed manually (true).
% The definitions initialise to false:
%    \begin{macrocode}
\newif\ifchilddoc
\newif\ifchilddocmanual
%    \end{macrocode}

% \macro{\childdocname}
% \macro{\childdocjob}
% The macro |\childdocname| stores the name of the main document
% to be compiled. The macro |\childdocjob| stores the name of
% the document on which the \LaTeX{} compiler was originally invoked.
% The content of |\jobname| cannot be compared
% to filenames specified in the source due to different catcodes.
% The following code rescans |\jobname|, stores the result
% in |\childdocname| and saves a copy in |\childdocjob|:
%    \begin{macrocode}
\edef\childdocname{\scantokens\expandafter{\jobname\noexpand}}
\let\childdocjob\childdocname
%    \end{macrocode}

% \macro{\childdocdisable}
% The macro |\childdocdisable| prevents the main file
% from being processed more than once.
% At this stage, the main document command |\childdocmain|
% is assumed to be called once again where it should do nothing.
% Any subsequent call to it should prevent
% a secondary processing of the main document
% It overwrites the forwarding commands
% |\childdocof| and |\childdocforward|
% with empty macros to prevent further inclusions of the main document:
%    \begin{macrocode}
\newcommand{\childdocdisable}
{
  \renewcommand{\childdocmain}[1]{\renewcommand{\childdocmain}[1]{\endinput}}
  \renewcommand{\childdocof}[1]{}
  \renewcommand{\childdocby}[2][]{}
  \renewcommand{\childdocforward}[2][]{}
  \renewcommand{\childdocdisable}{}
}
%    \end{macrocode}

% \macro{\childdocmain}
% The macro |\childdocmain| is to be called at the top of the main file
% with nothing or the main filename (without extension) as argument.
% First, it breaks loops.
% If the argument is not empty and does not match |\childdocname|
% (which is set by the first inclusion of |childdoc.def|),
% |\ifchilddoc| is set to true, |\includeonly| is applied to the child file
% and |\jobname| is set to the main file
% (for proper handling of |.aux| files):
%    \begin{macrocode}
\newcommand{\childdocmain}[1]
{
  \childdocdisable\childdocmain{}
  \if?#1?\else
    \begingroup
      \def\childdoctmp{#1}
      \ifx\childdoctmp\childdocname
        \def\childdoctmp{}
      \else
        \def\childdoctmp
        {
          \childdoctrue
          \includeonly{\childdocname}
          \def\childdocjob{#1}
          \def\jobname{#1}
        }
      \fi
      \expandafter
    \endgroup
    \childdoctmp
  \fi
}
%    \end{macrocode}

% \macro{\childdocof}
% The command |\childdocof| redirects
% compilation to the main file |#1|.
%    \begin{macrocode}
\newcommand{\childdocof}[1]
{
  \childdocdisable
  \childdoctrue
  \includeonly{\childdocname}
  \def\jobname{#1}
  \def\childdocjob{#1}
  \input{#1}
}
%    \end{macrocode}

% \macro{\childdocby}
% The command |\childdocby| ....
%    \begin{macrocode}
\newcommand{\childdocby}[2][]
{
  \childdocdisable
  \childdoctrue
  \childdocmanualtrue
  \if?#1?\else
    \def\jobname{#2}
  \fi
  \def\childdocjob{#2}
  \input{#2}
  \endinput
}
%    \end{macrocode}

% \macro{\childdocforward}
% The command |\childdocforward| redirects
% compilation to the main file or
% (if the optional argument is given) a child file.
% Parameters are set as if the main file
% or a child file starting with |\childdocof| was compiled.
% Then compilation is handed over to the main file:
%    \begin{macrocode}
\newcommand{\childdocforward}[2][]
{
  \begingroup
    \if?#1?
      \def\childdoctmp
      {
        \def\childdocname{#2}
        \def\childdocjob{#2}
        \def\jobname{#2}
        \input{#2}
        \endinput
      }
    \else
      \def\childdoctmp
      {
        \childdocdisable
        \def\childdocname{#2}
        \childdoctrue
        \includeonly{#2}
        \def\childdocjob{#1}
        \def\jobname{#1}
        \input{#1}
        \endinput
      }
    \fi
    \expandafter
  \endgroup
  \childdoctmp
}
%    \end{macrocode}

% \macro{\childdocforwardprefix}
% The command |\childdocforwardprefix| redirects
% compilation to the main or a child file by means of a pattern.
% The prefix |#1| in the current filename is replaced by |#2|
% and the suffix of the current filename is kept
% (it is assumed that the filename does not contain the substring `|~~~|'
% which is used as a delimiter).
% Compilation is handed over to the new file by |\childdocforward|:
%    \begin{macrocode}
\newcommand{\childdocforwardprefix}[3][]
{
  \begingroup
    \def\childdocextract #2##1~~~{\def\childdoctmp{\childdocforward[#1]{#3##1}}}
    \expandafter\childdocextract\childdocname~~~
    \expandafter
  \endgroup
  \childdoctmp
}
%    \end{macrocode}

% \macro{\childdoc}
% The deprecated macro |\childdoc| is a legacy version of |\childdocmain|:
%    \begin{macrocode}
\newcommand{\childdoc}{\childdocmain}
%    \end{macrocode}

% \macro{\childdocredirect}
% The deprecated macro |\childdocredirect| is a legacy version
% of |\childdocforward| and |\childdocforwardprefix|:
%    \begin{macrocode}
\newcommand{\childdocredirect}[2][]
{
  \begingroup
    \if?#1?
      \def\childdoctmp{\childdocforward{#2}}
    \else
      \def\childdoctmp{\childdocforwardprefix{#1}{#2}}
    \fi
    \expandafter
  \endgroup
  \childdoctmp
}
%    \end{macrocode}

%\iffalse
%</package>
%\fi
%
\endinput

\childdocforward{cdocsamp}
%    \end{macrocode}

%\iffalse
%</sampledraft>
%\fi
%
% %%%%%%%%%%%%%%%%%%%%%%%%%%%%%%%%%%%%%%
% \paragraph{Forwarding for Final Version of the Chapters.}
%
% The following forwarding files |cdocsfn1.tex| and |cdocsfn2.tex|
% (with identical content)
% compile the final versions of the child documents
% |cdocsch1.tex| and |cdocsch2.tex|, respectively:
%\iffalse
%<*samplefinal>
%\fi
%    \begin{macrocode}
\def\version{final}
% \iffalse
%
% childdoc.dtx Copyright (C) 2017-2018 Niklas Beisert
%
% This work may be distributed and/or modified under the
% conditions of the LaTeX Project Public License, either version 1.3
% of this license or (at your option) any later version.
% The latest version of this license is in
%   http://www.latex-project.org/lppl.txt
% and version 1.3 or later is part of all distributions of LaTeX
% version 2005/12/01 or later.
%
% This work has the LPPL maintenance status `maintained'.
%
% The Current Maintainer of this work is Niklas Beisert.
%
% This work consists of the files childdoc.dtx and childdoc.ins
% and the derived files childdoc.def and cdocsamp.tex with
% cdocsch1.tex, cdocsch2.tex, cdocsdrf.tex, cdocsfn1.tex, cdocsfn2.tex.
%
%<package>\ifdefined\childdocmain\endinput\fi
%<package>\ProvidesFile{childdoc.def}[2018/12/30 v2.0 child document driver]
%<samplemain>\ProvidesFile{cdocsamp.tex}[2018/12/30 v2.0 sample for childdoc]
%<*driver>
%\ProvidesFile{childdoc.drv}[2018/12/30 v2.0 childdoc reference manual file]
\PassOptionsToClass{10pt,a4paper}{article}
\documentclass{ltxdoc}

\usepackage[margin=35mm]{geometry}
\usepackage{hyperref}
\usepackage{hyperxmp}
\usepackage[usenames]{color}

\hypersetup{colorlinks=true}
\hypersetup{pdfstartview=FitH}
\hypersetup{pdfpagemode=UseNone}
\hypersetup{pdfsource={}}
\hypersetup{pdflang={en-UK}}
\hypersetup{pdfcopyright={Copyright 2017-2018 Niklas Beisert.
  This work may be distributed and/or modified under the
  conditions of the LaTeX Project Public License, either version 1.3
  of this license or (at your option) any later version.}}
\hypersetup{pdflicenseurl={http://www.latex-project.org/lppl.txt}}
\hypersetup{pdfcontactaddress={ETH Zurich, ITP, HIT K,
  Wolfgang-Pauli-Strasse 27}}
\hypersetup{pdfcontactpostcode={8093}}
\hypersetup{pdfcontactcity={Zurich}}
\hypersetup{pdfcontactcountry={Switzerland}}
\hypersetup{pdfcontactemail={nbeisert@itp.phys.ethz.ch}}
\hypersetup{pdfcontacturl={http://people.phys.ethz.ch/\xmptilde nbeisert/}}

\newcommand{\secref}[1]{\hyperref[#1]{section \ref*{#1}}}

\parskip1ex
\parindent0pt
\let\olditemize\itemize
\def\itemize{\olditemize\parskip0pt}

\begin{document}

\title{The \textsf{childdoc} Package}
\hypersetup{pdftitle={The childdoc Package}}
\author{Niklas Beisert\\[2ex]
  Institut f\"ur Theoretische Physik\\
  Eidgen\"ossische Technische Hochschule Z\"urich\\
  Wolfgang-Pauli-Strasse 27, 8093 Z\"urich, Switzerland\\[1ex]
  \href{mailto:nbeisert@itp.phys.ethz.ch}
  {\texttt{nbeisert@itp.phys.ethz.ch}}}
\hypersetup{pdfauthor={Niklas Beisert}}
\hypersetup{pdfsubject={Manual for the LaTeX2e Package childdoc}}
\date{30 December 2018, \textsf{v2.0}}
\maketitle

\begin{abstract}\noindent
\textsf{childdoc} is a \LaTeXe{} package
that enables the direct compilation
of document sections included by |\include|
to individual files.
\end{abstract}

\begingroup
\parskip0ex
\tableofcontents
\endgroup

%%%%%%%%%%%%%%%%%%%%%%%%%%%%%%%%%%%%%%%%%%%%%%%%%%%%%%%%%%%%%%%%%%%%%%%%%%%%%%%%
%%%%%%%%%%%%%%%%%%%%%%%%%%%%%%%%%%%%%%%%%%%%%%%%%%%%%%%%%%%%%%%%%%%%%%%%%%%%%%%%
\section{Introduction}

\LaTeX{} provides a mechanism to structure a large document (such as a book)
into a main file and several child files (containing the chapters)
using the |\include| command.
This mechanism is beneficial for documents
which span hundreds of pages in order to
make the source file(s) more manageable.
Moreover, compilation can be restricted to
selected child files by means of the |\includeonly| command.
The latter feature can be used to reduce the compilation time while editing
(this was significantly more useful in the earlier days of \LaTeX{})
or to generate a smaller document which is easier to navigate.
Another application of |\includeonly| is to generate
documents consisting of selected parts of the complete document.

However, there are a few drawbacks of the plain |\include| mechanism:
\begin{itemize}
\item
The child files cannot be compiled on their own,
they can only be compiled via the main file.
A naive editing environment
(such as a text editor with an option
to have the current file processed by \LaTeX)
may require one to switch to the main file before compiling;
attempting to compile the child file produces errors.
\item
The main file must be modified (each time)
to adjust the |\includeonly| command
to the present needs. This easily leaves the main file in a messy state.
\item
The generated document will always carry the filename
of the main document. This is inconvenient if
several child files are to be compiled and
to be kept for distribution.
\end{itemize}

The present package provides a simple interface
to make child files individually compilable by \LaTeX{}.
Compiling a child file then has the same effect as compiling
the main file with an |\includeonly| command
to select the appropriate child.
Moreover the generated document will carry the name of the child
rather than the main file.
This resolves all three above issues.

This feature is meant to make the editing of books,
thesis documents and lecture notes somewhat more convenient.
However, the package can also be used efficiently for
composing a series of documents (such as exercise sheets)
which are typically distributed individually.
It then assists the author in generating the individual documents
(potentially in different versions)
as well as a document containing the collected series.
Another application is in developing style files
or other kinds of included material
where compilation of the style file could redirect
to a sample or test file.

%%%%%%%%%%%%%%%%%%%%%%%%%%%%%%%%%%%%%%%%%%%%%%%%%%%%%%%%%%%%%%%%%%%%%%%%%%%%%%%%
%%%%%%%%%%%%%%%%%%%%%%%%%%%%%%%%%%%%%%%%%%%%%%%%%%%%%%%%%%%%%%%%%%%%%%%%%%%%%%%%
\section{Usage}

First of all, the package \textsf{childdoc} is \emph{not} a standard
\LaTeXe{} |.sty| style file! Therefore it needs to be invoked in
a non-standard way.

%%%%%%%%%%%%%%%%%%%%%%%%%%%%%%%%%%%%%%%%%%%%%%%%%%%%%%%%%%%%%%%%%%%%%%%%%%%%%%%%
\subsection{Included Files}
\label{sec:include}

%%%%%%%%%%%%%%%%%%%%%%%%%%%%%%%%%%%%%%%%
\DescribeMacro{\childdocmain}
To use the package, add the commands
\begin{center}
\begin{tabular}{l}
|\input{childdoc.def}|\\
|\childdocmain{}|\\
\end{tabular}
\end{center}
at the very top of the main \LaTeX{} file,
in particular \emph{before} the |\documentclass| statement!
The argument of |\childdocmain| should be left empty
(but it must be present).

%%%%%%%%%%%%%%%%%%%%%%%%%%%%%%%%%%%%%%%%
\DescribeMacro{\childdocof}
Furthermore, add the commands
\begin{center}
\begin{tabular}{l}
|\input{childdoc.def}|\\
|\childdocof{|\textit{main}|}|\\
\end{tabular}
\end{center}
at the top of every child file \textit{child}
which is included by |\include{|\textit{child}|}|
from within the main file
(or at least for those files to be compiled individually).
The argument \textit{main} must be the filename of the main file.

There are a couple of
considerations in setting up the main and child documents:

%%%%%%%%%%%%%%%%%%%%%%%%%%%%%%%%%%%%%%%%
\paragraph{Restrictions.}

Please note the following restrictions:
\begin{itemize}
\item
|\childdocmain| must be called with one argument \textit{main}
to ensure compatibility with earlier version of the package.
It must either be empty (|\childdocmain{}|)
or precisely match the filename of the main file in which it is specified.
See \secref{sec:detection} for further information.
\item
The filename \textit{main} must be specified without the |.tex| extension.
\item
The filename \textit{main} is case sensitive
(even in case-insensitive file systems)
due to internal string comparison.
\item
The argument \textit{main} should be fully expanded, it cannot be a macro.
\item
Subdirectories and special characters should be avoided in filenames.
\item
The command |\childdocmain{|\textit{main}|}| must be followed by a whitespace.
It should not be followed immediately by another command
or by a comment mark `|%|'.
This is because the \TeX{} parser reads the token immediately following
the argument of |\childdocmain| and puts it
at the beginning of every child section;
however, a white\-space is ignored.
\end{itemize}

%%%%%%%%%%%%%%%%%%%%%%%%%%%%%%%%%%%%%%%%
\paragraph{Content of Main File.}

It is advisable to place all content in the child files included by |\include|.
Any output contained in the main file will appear in all child documents
unless suppressed manually;
it cannot be suppressed automatically by the |\includeonly| directive
and thus should normally be avoided.
A method to include some content in the main file
by means of conditional processing is described in \secref{sec:conditional}.

%%%%%%%%%%%%%%%%%%%%%%%%%%%%%%%%%%%%%%%%
\paragraph{Page Numbering.}

When only a part of the document is compiled,
the appropriate numbering of pages
(as well as other status parameters)
is determined from the |.aux| files.
The latter contain information from previous passes.
However this information needs to propagate through
all intermediate child documents.
Therefore the page numbering in child documents may well
be inconsistent until the complete document is compiled at least once.

A useful (if unconventional) way to always ensure a consistent
page numbering is to restart the numbering in each child document
and denote the pages by `\textit{child}|.|\textit{page}'
where \textit{child} represents the chapter/section number of the child file.
This can be achieved by the command
|\numberwithin{page}{|\textit{child}|}|
of the \textsf{amsmath} package
where \textit{child} can be |chapter| or |section|
depending on the chosen structuring.
Alternatively, one can modify the macro |\thepage| appropriately
and reset the counter |page| at the start of each child file.

%%%%%%%%%%%%%%%%%%%%%%%%%%%%%%%%%%%%%%%%%%%%%%%%%%%%%%%%%%%%%%%%%%%%%%%%%%%%%%%%
\subsection{Conditional Processing}
\label{sec:conditional}

The package provides a mechanism to compile different versions
of a document. To customise the versions further some conditional processing
can come in handy to distinguish which version is being compiled.
The package provides two macros to describe the compilation context:

%%%%%%%%%%%%%%%%%%%%%%%%%%%%%%%%%%%%%%%%
\DescribeMacro{\ifchilddoc}
The conditional |\ifchilddoc| distinguishes between the compilation of
child documents and the main document:
%
\begin{center}
|\ifchilddoc |\textit{child-code}| |[|\||else |\textit{main-code}]| \||fi|
\end{center}

%%%%%%%%%%%%%%%%%%%%%%%%%%%%%%%%%%%%%%%%
\DescribeMacro{\childdocname}
\DescribeMacro{\childdocjob}
The macro |\childdocname| contains the filename (without extension)
of the main or child file being processed.
Note that |\childdocjob| will always contain the name of the main file.

%%%%%%%%%%%%%%%%%%%%%%%%%%%%%%%%%%%%%%%%
\paragraph{Title Page.}

Conditional processing can be used to include a title or banner page
in the main document when proper precautions are taken.
Importantly, the code in the main file should ensure that the page counter
(as well as other status parameters which are stored in the |.aux| files)
takes the same value after the conditional processing.
Otherwise the page numbers may take divergent values
depending on which part is compiled.

For example, a title page could be declared by:
%
\begin{center}
\begin{tabular}{l}
|\ifchilddoc\||else|\\
|\addtocounter{page}{-1}|\\
\textit{code for title page}\\
|\newpage|\\
|\||fi|
\end{tabular}
\end{center}
%
A banner page for the child documents can be generated by:
%
\begin{center}
\begin{tabular}{l}
|\ifchilddoc|\\
|\addtocounter{page}{-1}|\\
\textit{code for banner page}\\
|\newpage|\\
|\||fi|
\end{tabular}
\end{center}
%
Here one could write a message such as:
\begin{center}
|This is the part \childdocname{} of \childdocjob{}.|
\end{center}

%%%%%%%%%%%%%%%%%%%%%%%%%%%%%%%%%%%%%%%%%%%%%%%%%%%%%%%%%%%%%%%%%%%%%%%%%%%%%%%%
\subsection{Flags}
\label{sec:flags}

The package makes it easy to generate different versions
of the main or child documents.
To this end compilation flags can be defined
and assigned different default values.
They will be particularly useful in conjunction
with the forwarding mechanism described in \secref{sec:forward}.

For example, it may be useful to have a flag |\version|
which can be set to |draft| or |final|.
The document source will contain some conditional code
depending on the value of |\version|.
Suppose further, the flag should default to |final| for the main file
and to |draft| for child files
which is a natural assignment for editing the document.
This is achieved by placing the following code
in the preamble of the main document
(below the |\childdocmain| directive):
%
\begin{center}
\begin{tabular}{l}
|\ifchilddoc|\\
|\providecommand{\version}{draft}|\\
|\||else|\\
|\providecommand{\version}{final}|\\
|\||fi|
\end{tabular}
\end{center}
%
The definition by |\providecommand| makes sure
that previous definitions are not overwritten.
Further statements |\providecommand{\version}{...}|
can thus be added before the above code to override it.

For the main file, one might add a line
(between |\childdocmain| and the above block)
%
\begin{center}
|%\ifchilddoc\||else\providecommand{\version}{draft}\||fi|
\end{center}
%
which can be uncommented to produce a draft version.
Likewise one can add a line to the very top of a child file
(above the |\childdocof{|\textit{main}|}| directive)
%
\begin{center}
|%\providecommand{\version}{final}|
\end{center}
%
which can be uncommented to produce the final version of this child document.

%%%%%%%%%%%%%%%%%%%%%%%%%%%%%%%%%%%%%%%%%%%%%%%%%%%%%%%%%%%%%%%%%%%%%%%%%%%%%%%%
\subsection{Forwarding}
\label{sec:forward}

Different versions of the main or child documents
using compilation flags as described in \secref{sec:flags}
can be (permanently) stored in different files
for convenient compilation, viewing and distribution.
To this end, the package defines a command
to pass on compilation to a different file:

%%%%%%%%%%%%%%%%%%%%%%%%%%%%%%%%%%%%%%%%
\DescribeMacro{\childdocforward}
The command |\childdocforward| redirects processing to
another source file:
%
\begin{center}
\begin{tabular}{l}
|\input{childdoc.def}|\\
|\childdocforward[|\textit{main}|]{|\textit{dest}|}|\\
\end{tabular}
\end{center}
%
The argument \textit{dest} is the destination file
(without extension).
It should be the main file or one of the child files.
Note that further \textsf{childdoc} directives
such as |\childdocof| and |\childdocforward|
in the indicated file will be processed in this form.
The optional argument \textit{main}
passes on directly to the main file \textit{main}
while pretending to compile the child \textit{dest}.
This form behaves as if \textit{dest}
issues |\childdocof{|\textit{main}|}| right away,
and no further \textsf{childdoc} directives will be processed.

%%%%%%%%%%%%%%%%%%%%%%%%%%%%%%%%%%%%%%%%
\DescribeMacro{\...prefix}
In the alternative form |\childdocforwardprefix|,
%
\begin{center}
\begin{tabular}{l}
|\input{childdoc.def}|\\
|\childdocforwardprefix[|\textit{main}|]{|\textit{prefix}|}{|\textit{dest}|}|
\end{tabular}
\end{center}
%
the destination file is determined by a pattern
depending on the current file:
To make this work, the current file must be called
`{\textit{prefix}\hspace{0.2em}\textit{suffix}}'
with \textit{prefix} matching precisely the argument.
Processing is then passed on to the file
`{\textit{dest}\hspace{0.2em}\textit{suffix}}'.
Surely, the same effect is achieved by
directly specifying the
argument `{\textit{dest}\hspace{0.2em}\textit{suffix}}'
in the first form.
However, that requires to set up a different file
for each child. With the alternative form of the command
all these files can have exactly the same content
which simplifies setting them up and maintaining them.

For example, the following file |draft.tex|
with a compilation flag |\version| as described in \secref{sec:flags}
compiles the main document as a draft:
%
\begin{center}
\begin{tabular}{l}
|\def\version{draft}|\\
|\input{childdoc.def}|\\
|\childdocforward{|\textit{main}|}|
\end{tabular}
\end{center}
%
Likewise, the following files |final|\textit{nn}|.tex|
compile the final version of the child document
|child|\textit{nn}|.tex|:
%
\begin{center}
\begin{tabular}{l}
|\def\version{final}|\\
|\input{childdoc.def}|\\
|\childdocforwardprefix{final}{child}|
\end{tabular}
\end{center}
%

Note that when several versions of a main file and/or of each child file
are to be generated, it may be convenient to set up a |Makefile| or
shell script to automatise the process.

%%%%%%%%%%%%%%%%%%%%%%%%%%%%%%%%%%%%%%%%%%%%%%%%%%%%%%%%%%%%%%%%%%%%%%%%%%%%%%%%
\subsection{Command Line Processing}
\label{sec:commandline}

The effect of redirection files can also be achieved by invoking
the \LaTeX{} compiler with a more elaborate command line.
Most conveniently this should be done as part
of a shell script or a |Makefile|.

When using \textsf{childdoc} in the main file, the following
command lines effectively perform a redirection
(note that depending on the shell being used,
backslashes may have to be doubled: `|\|' $\to$ `|\\|'):
%
\begin{center}
|... -jobname "|\textit{target}|" |\\|"|[\textit{flags}]%
|\input{childdoc.def}\childdocforward[|\textit{main}|]{|\textit{dest}|}"|
\end{center}
%
Here \textit{target} is the name of the output file,
\textit{main} is the name of the main file
and \textit{dest} is the name of the main or child file to be processed
(all filenames without extensions).
The optional argument \textit{main} can be omitted
if \textit{main} matches \textit{dest}.
Optionally, compilation \textit{flags} can be defined via |\def| commands.
This command line makes the \TeX{} engine believe
it is compiling the file \textit{target}
whose content is specified as the latter parameter.
The provided code then forwards the processing to
\textit{main} or \textit{dest} as described in \secref{sec:forward}.

%%%%%%%%%%%%%%%%%%%%%%%%%%%%%%%%%%%%%%%%%%%%%%%%%%%%%%%%%%%%%%%%%%%%%%%%%%%%%%%%
\subsection{Include by Input}
\label{sec:input}

Including child documents by |\include| has some restrictions by design.
Most notably, the content of a child document always occupies
its own set of pages; pages cannot be shared between child documents.
Usually, this behaviour makes perfect sense
because each child document contain an essential part of the document.
However, in some situations it may be desirable to compose
a document from a collection of parts
without having mandatory page breaks between then.
For this case, the package
provides a mechanism to include parts
by |\input| which can also be processed individually.
However, by construction this mechanism
requires manual handling of the content to be output.

%%%%%%%%%%%%%%%%%%%%%%%%%%%%%%%%%%%%%%%%
\DescribeMacro{\ifchilddocmanual}
The main file should be prepared as usual, see \secref{sec:include}.
However, the document body must make a distinction
between processing of an individual part and of the main document, e.g.:
%
\begin{center}
\begin{tabular}{l}
|\ifchilddocmanual|\\
|\input{\childdocname}|\\
|\||else|\\
\textit{document body with }|\input{|\textit{part}|}|\\
|\||fi|
\end{tabular}
\end{center}
%
The conditional |\ifchilddocmanual| is true whenever
a part to be included by |\input| is being compiled,
and the name of the part is stored in |\childdocname|.

%%%%%%%%%%%%%%%%%%%%%%%%%%%%%%%%%%%%%%%%
\DescribeMacro{\childdocby}
Each part to be included by |\input| should start with:
%
\begin{center}
\begin{tabular}{l}
|\input{childdoc.def}|\\
|\childdocby{|\textit{main}|}|\\
\end{tabular}
\end{center}
%
The directive |\childdocby| is similar to |\childdocof|
described in \secref{sec:include},
but the subsequent selection of content must be done manually.
To that end, both |\ifchilddoc| and |\ifchilddocmanual|
will be true upon processing of a part,
and the name of the part is stored in |\childdocname|.
Note that |\jobname| will be set to the filename of the current part
so that each part receives an individual |.aux| file
that does not interfere with the |.aux| file(s) of the main document.
This behaviour can be altered by the alternative form
|\childdocby[*]{|\textit{main}|}| (with a non-empty optional argument)
which uses the |.aux| file of the main document
by setting |\jobname| to \textit{main}.

%%%%%%%%%%%%%%%%%%%%%%%%%%%%%%%%%%%%%%%%%%%%%%%%%%%%%%%%%%%%%%%%%%%%%%%%%%%%%%%%
\subsection{Driver Development}
\label{sec:driver}

The \textsf{childdoc} mechanism can also be use for the development
of definition files such as \LaTeX{} styles or classes.
This case differs from the above setup with multiple parts
included by |\include| in that no |\includeonly| should be invoked.
This can be achieved by starting the include file
(before |\ProvidesPackage|) with:
%
\begin{center}
\begin{tabular}{l}
|\input{childdoc.def}|\\
|\childdocforward{|\textit{main}|}|\\
\end{tabular}
\end{center}
%
or alternatively with:
%
\begin{center}
\begin{tabular}{l}
|\input{childdoc.def}|\\
|\childdocby{|\textit{main}|}|\\
\end{tabular}
\end{center}
%
Both forms have slightly different effects as described above.
The main file is prepared as usual, see \secref{sec:include}.

%%%%%%%%%%%%%%%%%%%%%%%%%%%%%%%%%%%%%%%%%%%%%%%%%%%%%%%%%%%%%%%%%%%%%%%%%%%%%%%%
\subsection{Legacy Detection}
\label{sec:detection}

The directive |\childdocmain| in the main file can detect
whether the complete document or merely a child is to be compiled
even without using the directive |\childdocof|.
This method is deprecated because it is less robust
and there is no compelling reason to use it;
it is merely provided for backward compatibility
and it may be removed in future versions.

If the detection mechanism is to be used,
it is mandatory to correctly specify
the filename of the main file as the argument of |\childdocmain|:
%
\begin{center}
\begin{tabular}{l}
|\input{childdoc.def}|\\
|\childdocmain{|\textit{main}|}|\\
\end{tabular}
\end{center}
%
If |\jobname| does not match the argument \textit{main} of |\childdocmain|,
it is assumed that |\jobname| points to the child file to be compiled.
When using |\childdocmain| with the main file specified as argument,
it suffices to start a child file
with just |\input{|\textit{main}|}|
without loading of the package and using |\childdocof|.
If instead all processing is done
with the appropriate \textsf{childdoc} directives,
the argument of \textit{main} of |\childdocmain| can be empty.

An alternative version of the command line processing described
in \secref{sec:commandline} using the detection mechanism reads:
%
\begin{center}
|... -jobname "|\textit{target}|" "|[\textit{flags}]%
[|\def\jobname{|\textit{dest}|}|]|\input{|\textit{main}|}"|
\end{center}

%%%%%%%%%%%%%%%%%%%%%%%%%%%%%%%%%%%%%%%%%%%%%%%%%%%%%%%%%%%%%%%%%%%%%%%%%%%%%%%%
\subsection{Manual Code}
\label{sec:manual}

In case one cannot be certain whether the definitions file |childdoc.def|
is installed on the target \TeX{} distribution
and one prefers not to ship it,
it is conceivable to paste a few relevant commands into the sources.

To that end, drop all statements |\input{childdoc.def}|
and perform the replacements as outlined below.
Instead of |\childdocmain{|\textit{main}|}| add the following code
to the top of the main file:
%
\begin{center}
\begin{tabular}{l}
|\||ifdefined\childdocname\endinput\||fi\newif\ifchilddoc|\\
|\edef\childdocname{\scantokens\expandafter{\jobname\noexpand}}|\\
|\def\childdocmain{|\textit{main}|}\||ifx\childdocmain\childdocname\||else|\\
|\childdoctrue\includeonly{\childdocname}\let\jobname\childdocmain\||fi|\\
\end{tabular}
\end{center}
%
Instead of |\childdocof{|\textit{main}|}| just include the main file
at the top of each child file:
%
\begin{center}
|\input{|\textit{main}|}|
\end{center}
%
A simple redirection |\childdocforward{|\textit{dest}|}| is achieved by:
%
\begin{center}
|\def\jobname{|\textit{dest}|}\input{\jobname}|
\end{center}
%
The redirection with prefix
|\childdocforwardprefix[|\textit{prefix}|]{|\textit{dest}|}|
is accomplished by:
%
\begin{center}
\begin{tabular}{l}
|{\edef\jobname{\scantokens\expandafter{\jobname\noexpand}}|\\
|\def\redirectjob |\textit{prefix}|#1~~~{\gdef\jobname{|\textit{dest}|#1}}|\\
|\expandafter\redirectjob\jobname~~~}\input{\jobname}|
\end{tabular}
\end{center}

In an alternative approach,
child documents can be compiled by a specific command line
without additional code or specific definitions:
%
\begin{center}
|... -jobname "|\textit{target}|" "|[\textit{flags}]%
|\includeonly{|\textit{dest}|}\input{|\textit{main}|}"|
\end{center}
%

%%%%%%%%%%%%%%%%%%%%%%%%%%%%%%%%%%%%%%%%%%%%%%%%%%%%%%%%%%%%%%%%%%%%%%%%%%%%%%%%
%%%%%%%%%%%%%%%%%%%%%%%%%%%%%%%%%%%%%%%%%%%%%%%%%%%%%%%%%%%%%%%%%%%%%%%%%%%%%%%%
\section{Information}

%%%%%%%%%%%%%%%%%%%%%%%%%%%%%%%%%%%%%%%%%%%%%%%%%%%%%%%%%%%%%%%%%%%%%%%%%%%%%%%%
\subsection{Copyright}

Copyright \copyright{} 2017--2018 Niklas Beisert

This work may be distributed and/or modified under the
conditions of the \LaTeX{} Project Public License, either version 1.3
of this license or (at your option) any later version.
The latest version of this license is in
  \url{http://www.latex-project.org/lppl.txt}
and version 1.3 or later is part of all distributions of \LaTeX{}
version 2005/12/01 or later.

This work has the LPPL maintenance status `maintained'.

The Current Maintainer of this work is Niklas Beisert.

This work consists of the files |README.txt|, |childdoc.ins| and |childdoc.dtx|
as well as the derived files |childdoc.def|, |cdocsamp.tex|
with |cdocsch1.tex|, |cdocsch2.tex|, |cdocspt3.tex|, |cdocspt4.tex|,
|cdocsdrf.tex|, |cdocsfn1.tex|, |cdocsfn2.tex|
as well as |childdoc.pdf|.

%%%%%%%%%%%%%%%%%%%%%%%%%%%%%%%%%%%%%%%%%%%%%%%%%%%%%%%%%%%%%%%%%%%%%%%%%%%%%%%%
\subsection{Files and Installation}

The package consists of the files:
%
\begin{center}
\begin{tabular}{ll}
    |README.txt|   & readme file \\
    |childdoc.ins| & installation file \\
    |childdoc.dtx| & source file \\
    |childdoc.def| & definition file \\
    |cdocsamp.tex| & sample main file \\
    |cdocsch1.tex| & sample include file \\
    |cdocsch2.tex| & sample include file \\
    |cdocspt3.tex| & sample part file \\
    |cdocspt4.tex| & sample part file \\
    |cdocsdrf.tex| & sample redirection file \\
    |cdocsfn1.tex| & sample redirection file \\
    |cdocsfn2.tex| & sample redirection file \\
    |childdoc.pdf| & manual
\end{tabular}
\end{center}
%
The distribution consists of the files
|README.txt|, |childdoc.ins| and |childdoc.dtx|.
%
\begin{itemize}
\item
Run (pdf)\LaTeX{} on |childdoc.dtx|
to compile the manual |childdoc.pdf| (this file).
\item
Run \LaTeX{} on |childdoc.ins| to create the definitions file |childdoc.def|
and the sample |cdocsamp.tex| with include files
|cdocsch1.tex|, |cdocsch2.tex|, |cdocspt3.tex|, |cdocspt4.tex|,
|cdocsdrf.tex|, |cdocsfn1.tex|, |cdocsfn2.tex|.
Then copy the file |childdoc.def| to an appropriate directory of your \LaTeX{}
distribution, e.g.\ \textit{texmf-root}|/tex/latex/childdoc|.
\end{itemize}

%%%%%%%%%%%%%%%%%%%%%%%%%%%%%%%%%%%%%%%%%%%%%%%%%%%%%%%%%%%%%%%%%%%%%%%%%%%%%%%%
\subsection{Related CTAN Packages}

There are several other packages which offer a similar functionality:
%
\begin{itemize}
\item
The packages
\href{http://ctan.org/pkg/docmute}{\textsf{docmute}},
\href{http://ctan.org/pkg/includex}{\textsf{includex}} and
\href{http://ctan.org/pkg/standalone}{\textsf{standalone}}
provide commands to include only the document body of
a child file thus allowing both files to be compiled individually.
\item
The packages \href{http://ctan.org/pkg/subdocs}{\textsf{subdocs}}
and \href{http://ctan.org/pkg/subfiles}{\textsf{subfiles}}
provide structures in which the main and child documents can be
encapsulated and allowing them to be compiled individually.
The inclusion mechanism is different from the conventional |\include|.
\item
The package \href{http://ctan.org/pkg/combine}{\textsf{combine}}
is an elaborate solution to combine several documents into one.
\end{itemize}
%
See also the CTAN topic \href{http://ctan.org/topic/subdocs}{\textsf{subdocs}}
for further related packages.
The present package differs from the above solutions in that
a document structure constructed with the conventional |\include| mechanism
just needs two extra commands at the top of every file
such that all constituent files can be compiled individually.

%%%%%%%%%%%%%%%%%%%%%%%%%%%%%%%%%%%%%%%%%%%%%%%%%%%%%%%%%%%%%%%%%%%%%%%%%%%%%%%%
%\subsection{Feature Suggestions}
%
%The following is a list of features which may be useful for future
%versions of this package:
%%
%\begin{itemize}
%\item
%\ldots
%\end{itemize}

%%%%%%%%%%%%%%%%%%%%%%%%%%%%%%%%%%%%%%%%%%%%%%%%%%%%%%%%%%%%%%%%%%%%%%%%%%%%%%%%
\subsection{Revision History}

%%%%%%%%%%%%%%%%%%%%%%%%%%%%%%%%%%%%%%%%
\paragraph{v2.0:} 2018/12/30

\begin{itemize}
\item
immediate forward processing
\item
added |\childdocby| mechanism
\item
manual restructured
\end{itemize}

%%%%%%%%%%%%%%%%%%%%%%%%%%%%%%%%%%%%%%%%
\paragraph{v1.6:} 2018/01/17

\begin{itemize}
\item
application for development of include files
\item
corrections to manual
\end{itemize}

%%%%%%%%%%%%%%%%%%%%%%%%%%%%%%%%%%%%%%%%
\paragraph{v1.5:} 2017/05/21

\begin{itemize}
\item
more complete structuring introduced
\item
|\childdocof| introduced
\item
|\childdoc| renamed to |\childdocmain|
\item
|\childredirect| renamed to |\childdocforward| and |\childdocforwardprefix|
and functionality expanded
\end{itemize}

%%%%%%%%%%%%%%%%%%%%%%%%%%%%%%%%%%%%%%%%
\paragraph{v1.0:} 2017/04/27

\begin{itemize}
\item
manual and install package
\item
first version published on CTAN
\end{itemize}

%%%%%%%%%%%%%%%%%%%%%%%%%%%%%%%%%%%%%%%%
\paragraph{v0.6:} 2017/04/26

\begin{itemize}
\item
redirection mechanism added
\end{itemize}

%%%%%%%%%%%%%%%%%%%%%%%%%%%%%%%%%%%%%%%%
\paragraph{v0.5:} 2017/04/26

\begin{itemize}
\item
functionality in definition file
\end{itemize}


%%%%%%%%%%%%%%%%%%%%%%%%%%%%%%%%%%%%%%%%%%%%%%%%%%%%%%%%%%%%%%%%%%%%%%%%%%%%%%%%
%%%%%%%%%%%%%%%%%%%%%%%%%%%%%%%%%%%%%%%%%%%%%%%%%%%%%%%%%%%%%%%%%%%%%%%%%%%%%%%%
%%%%%%%%%%%%%%%%%%%%%%%%%%%%%%%%%%%%%%%%%%%%%%%%%%%%%%%%%%%%%%%%%%%%%%%%%%%%%%%%
\appendix

\settowidth\MacroIndent{\rmfamily\scriptsize 000\ }

 \DocInput{childdoc.dtx}

\end{document}
%</driver>
% \fi
%
% %%%%%%%%%%%%%%%%%%%%%%%%%%%%%%%%%%%%%%%%%%%%%%%%%%%%%%%%%%%%%%%%%%%%%%%%%%%%%%
% %%%%%%%%%%%%%%%%%%%%%%%%%%%%%%%%%%%%%%%%%%%%%%%%%%%%%%%%%%%%%%%%%%%%%%%%%%%%%%
% \section{Sample}
%\iffalse
%<*samplemain>
%\fi
%
% The following presents a sample document
% with two chapters, two parts, a title page,
% a compile flag as well as three forwarding files to set the flag.
% It consists of eight |.tex| files:
% \begin{center}
% \begin{tabular}{ll}
% |cdocsamp.tex|&main file\\
% |cdocsch1.tex|&include file for chapter 1\\
% |cdocsch2.tex|&include file for chapter 2\\
% |cdocspt3.tex|&include file for part 3\\
% |cdocspt4.tex|&include file for part 4\\
% |cdocsdrf.tex|&forwarding file for main file in draft mode\\
% |cdocsfi1.tex|&forwarding file for final version of chapter 1\\
% |cdocsfi2.tex|&forwarding file for final version of chapter 2\\
% \end{tabular}
% \end{center}
% Each of the eight files can be compiled directly by the \LaTeX{} compiler.
%
% %%%%%%%%%%%%%%%%%%%%%%%%%%%%%%%%%%%%%%
% \paragraph{Main File.}
%
% The main file is called |cdocsamp.tex|.
%
% Load the \textsf{childdoc} definitions and
% declare the filename for the main document:
%    \begin{macrocode}
\input{childdoc.def}
\childdocmain{}
%    \end{macrocode}

% Optional override for |\version| flag:
%    \begin{macrocode}
%%\ifchilddoc\else\providecommand{\version}{draft}\fi
%    \end{macrocode}

% Define the default values for the |\version| flag
% (|final| for the main file and |draft| for childs):
%    \begin{macrocode}
\ifchilddoc
\providecommand{\version}{draft}
\else
\providecommand{\version}{final}
\fi
%    \end{macrocode}

% Load the standard document class:
%    \begin{macrocode}
\documentclass[12pt]{article}
%    \end{macrocode}

% Start the document body:
%    \begin{macrocode}
\begin{document}
%    \end{macrocode}

% Declare a title page.
% Print title, part of document being processed and version flag:
%    \begin{macrocode}
\addtocounter{page}{-1}
\begin{center}
{\LARGE\bfseries{}childdoc example\par}
\vspace{1cm}
\ifchilddoc
\ifchilddocmanual part\else chapter\fi:
`\childdocname' of `\childdocjob'\par
\else
main document: `\childdocjob'\par
\fi
version: \version\par
\end{center}
\newpage
%    \end{macrocode}

% Manually include selected file,
% otherwise process as usual:
%    \begin{macrocode}
\ifchilddocmanual
\section*{part `\childdocname'}
\input{\childdocname}
\else
%    \end{macrocode}

% Include the two chapters:
%    \begin{macrocode}
\include{cdocsch1}
\include{cdocsch2}
%    \end{macrocode}

% Include the two parts unless only chapters should be displayed:
%    \begin{macrocode}
\ifchilddoc\else
\section{part three}
\input{cdocspt3}
\section{part four}
\input{cdocspt4}
\fi
%    \end{macrocode}

% Process as usual until here:
%    \begin{macrocode}
\fi
%    \end{macrocode}

% End of document body:
%    \begin{macrocode}
\end{document}
%    \end{macrocode}
%\iffalse
%</samplemain>
%\fi
%
% %%%%%%%%%%%%%%%%%%%%%%%%%%%%%%%%%%%%%%
% \paragraph{Chapter Include Files.}
%
% The include files are called |cdocsch1.tex| and |cdocsch2.tex|.
%
%\iffalse
%<*samplechap1|samplechap2>
%\fi

% Optional override for |\version| flag:
%    \begin{macrocode}
%%\providecommand{\version}{final}
%    \end{macrocode}

% Include the main document:
%    \begin{macrocode}
\input{childdoc.def}
\childdocof{cdocsamp}
%    \end{macrocode}

%\iffalse
%</samplechap1|samplechap2>
%\fi
%
%\iffalse
%<*samplechap1>
%\fi
% Some text for chapter 1:
%    \begin{macrocode}
\section{one}
some text in chapter one
%    \end{macrocode}

%\iffalse
%</samplechap1>
%\fi
% Some text for chapter 2:
%\iffalse
%<*samplechap2>
%\fi
%    \begin{macrocode}
\section{two}
more text in chapter two
%    \end{macrocode}

%\iffalse
%</samplechap2>
%\fi
%
% %%%%%%%%%%%%%%%%%%%%%%%%%%%%%%%%%%%%%%
% \paragraph{Part Include Files.}
%
% The include files are called |cdocspt3.tex| and |cdocspt4.tex|.
%
%\iffalse
%<*samplepart3|samplepart4>
%\fi

% Optional override for |\version| flag:
%    \begin{macrocode}
%%\providecommand{\version}{final}
%    \end{macrocode}

% Include the main document:
%    \begin{macrocode}
\input{childdoc.def}
\childdocby{cdocsamp}
%    \end{macrocode}

%\iffalse
%</samplepart3|samplepart4>
%\fi
%
%\iffalse
%<*samplepart3>
%\fi
% Some text for part 3:
%    \begin{macrocode}
some text in part three
%    \end{macrocode}

%\iffalse
%</samplepart3>
%\fi
% Some text for part 4:
%\iffalse
%<*samplepart4>
%\fi
%    \begin{macrocode}
more text in part four
%    \end{macrocode}

%\iffalse
%</samplepart4>
%\fi
%
% %%%%%%%%%%%%%%%%%%%%%%%%%%%%%%%%%%%%%%
% \paragraph{Forwarding for a Complete Draft.}
%
% The following forwarding file |cdocsdrf.tex|
% compiles the main document in draft mode:
%\iffalse
%<*sampledraft>
%\fi
%    \begin{macrocode}
\def\version{draft}
\input{childdoc.def}
\childdocforward{cdocsamp}
%    \end{macrocode}

%\iffalse
%</sampledraft>
%\fi
%
% %%%%%%%%%%%%%%%%%%%%%%%%%%%%%%%%%%%%%%
% \paragraph{Forwarding for Final Version of the Chapters.}
%
% The following forwarding files |cdocsfn1.tex| and |cdocsfn2.tex|
% (with identical content)
% compile the final versions of the child documents
% |cdocsch1.tex| and |cdocsch2.tex|, respectively:
%\iffalse
%<*samplefinal>
%\fi
%    \begin{macrocode}
\def\version{final}
\input{childdoc.def}
\childdocforwardprefix[cdocsamp]{cdocsfn}{cdocsch}
%    \end{macrocode}

%\iffalse
%</samplefinal>
%\fi
%
% %%%%%%%%%%%%%%%%%%%%%%%%%%%%%%%%%%%%%%
% \paragraph{Command Line Processing.}
%
% The following three command lines generate the output files
% |cdocscld|, |cdocscl1| and |cdocscl2|
% which should be identical to
% |cdocsdrf|, |cdocsch1| and |cdocsfn2|, respectively:
% \begin{center}
% \begin{tabular}{l}
% |latex -jobname cdocscld \|\\
% |  "\def\version{draft}\input{childdoc.def}\childdocforward{cdocsamp}"|\\
% |latex -jobname cdocscl1 \|\\
% |  "\input{childdoc.def}\childdocforward[cdocsamp]{cdocsch1}"|\\
% |latex -jobname cdocscl2 \|\\
% |  "\def\version{final}\input{childdoc.def}\childdocforward{cdocsch2}"|
% \end{tabular}
% \end{center}
% Note that the trailing backslash on each first line
% merely continues the input to the second line
% (for convenient cut ant paste).
% Furthermore, the command |latex| can be replaced by any
% of its alternative versions such as |pdflatex|.
%
% %%%%%%%%%%%%%%%%%%%%%%%%%%%%%%%%%%%%%%%%%%%%%%%%%%%%%%%%%%%%%%%%%%%%%%%%%%%%%%
% %%%%%%%%%%%%%%%%%%%%%%%%%%%%%%%%%%%%%%%%%%%%%%%%%%%%%%%%%%%%%%%%%%%%%%%%%%%%%%
% \section{Implementation}
%\iffalse
%<*package>
%\fi
%
% This section describes the definitions file |childdoc.def|.

% The definitions cannot be loaded using |\usepackage| or |\RequirePackage|
% which has a mechanism to prevent loading a style file more than once.
% When loading the definitions by means of |\input|
% multiple instances have to be prevented manually:
%\iffalse
%This code needs to be before the `\ProvidesFile' directive
%which is defined at the beginning of this file.
%Therefore it is also placed there and commented out here.
%</package>
%<*discard>
%\fi
%    \begin{macrocode}
\ifdefined\childdocmain\endinput\fi
%    \end{macrocode}
%\iffalse
%</discard>
%<*package>
%\fi
%
% \macro{\ifchilddoc}
% \macro{\ifchilddocmanual}
% The conditional |\ifchilddoc| tells whether a
% child (true) or main (false) document is being compiled.
% The conditional |\ifchilddocmanual| tells whether
% the |\includeonly| mechanism is used (false) or
% the selection of child files must be performed manually (true).
% The definitions initialise to false:
%    \begin{macrocode}
\newif\ifchilddoc
\newif\ifchilddocmanual
%    \end{macrocode}

% \macro{\childdocname}
% \macro{\childdocjob}
% The macro |\childdocname| stores the name of the main document
% to be compiled. The macro |\childdocjob| stores the name of
% the document on which the \LaTeX{} compiler was originally invoked.
% The content of |\jobname| cannot be compared
% to filenames specified in the source due to different catcodes.
% The following code rescans |\jobname|, stores the result
% in |\childdocname| and saves a copy in |\childdocjob|:
%    \begin{macrocode}
\edef\childdocname{\scantokens\expandafter{\jobname\noexpand}}
\let\childdocjob\childdocname
%    \end{macrocode}

% \macro{\childdocdisable}
% The macro |\childdocdisable| prevents the main file
% from being processed more than once.
% At this stage, the main document command |\childdocmain|
% is assumed to be called once again where it should do nothing.
% Any subsequent call to it should prevent
% a secondary processing of the main document
% It overwrites the forwarding commands
% |\childdocof| and |\childdocforward|
% with empty macros to prevent further inclusions of the main document:
%    \begin{macrocode}
\newcommand{\childdocdisable}
{
  \renewcommand{\childdocmain}[1]{\renewcommand{\childdocmain}[1]{\endinput}}
  \renewcommand{\childdocof}[1]{}
  \renewcommand{\childdocby}[2][]{}
  \renewcommand{\childdocforward}[2][]{}
  \renewcommand{\childdocdisable}{}
}
%    \end{macrocode}

% \macro{\childdocmain}
% The macro |\childdocmain| is to be called at the top of the main file
% with nothing or the main filename (without extension) as argument.
% First, it breaks loops.
% If the argument is not empty and does not match |\childdocname|
% (which is set by the first inclusion of |childdoc.def|),
% |\ifchilddoc| is set to true, |\includeonly| is applied to the child file
% and |\jobname| is set to the main file
% (for proper handling of |.aux| files):
%    \begin{macrocode}
\newcommand{\childdocmain}[1]
{
  \childdocdisable\childdocmain{}
  \if?#1?\else
    \begingroup
      \def\childdoctmp{#1}
      \ifx\childdoctmp\childdocname
        \def\childdoctmp{}
      \else
        \def\childdoctmp
        {
          \childdoctrue
          \includeonly{\childdocname}
          \def\childdocjob{#1}
          \def\jobname{#1}
        }
      \fi
      \expandafter
    \endgroup
    \childdoctmp
  \fi
}
%    \end{macrocode}

% \macro{\childdocof}
% The command |\childdocof| redirects
% compilation to the main file |#1|.
%    \begin{macrocode}
\newcommand{\childdocof}[1]
{
  \childdocdisable
  \childdoctrue
  \includeonly{\childdocname}
  \def\jobname{#1}
  \def\childdocjob{#1}
  \input{#1}
}
%    \end{macrocode}

% \macro{\childdocby}
% The command |\childdocby| ....
%    \begin{macrocode}
\newcommand{\childdocby}[2][]
{
  \childdocdisable
  \childdoctrue
  \childdocmanualtrue
  \if?#1?\else
    \def\jobname{#2}
  \fi
  \def\childdocjob{#2}
  \input{#2}
  \endinput
}
%    \end{macrocode}

% \macro{\childdocforward}
% The command |\childdocforward| redirects
% compilation to the main file or
% (if the optional argument is given) a child file.
% Parameters are set as if the main file
% or a child file starting with |\childdocof| was compiled.
% Then compilation is handed over to the main file:
%    \begin{macrocode}
\newcommand{\childdocforward}[2][]
{
  \begingroup
    \if?#1?
      \def\childdoctmp
      {
        \def\childdocname{#2}
        \def\childdocjob{#2}
        \def\jobname{#2}
        \input{#2}
        \endinput
      }
    \else
      \def\childdoctmp
      {
        \childdocdisable
        \def\childdocname{#2}
        \childdoctrue
        \includeonly{#2}
        \def\childdocjob{#1}
        \def\jobname{#1}
        \input{#1}
        \endinput
      }
    \fi
    \expandafter
  \endgroup
  \childdoctmp
}
%    \end{macrocode}

% \macro{\childdocforwardprefix}
% The command |\childdocforwardprefix| redirects
% compilation to the main or a child file by means of a pattern.
% The prefix |#1| in the current filename is replaced by |#2|
% and the suffix of the current filename is kept
% (it is assumed that the filename does not contain the substring `|~~~|'
% which is used as a delimiter).
% Compilation is handed over to the new file by |\childdocforward|:
%    \begin{macrocode}
\newcommand{\childdocforwardprefix}[3][]
{
  \begingroup
    \def\childdocextract #2##1~~~{\def\childdoctmp{\childdocforward[#1]{#3##1}}}
    \expandafter\childdocextract\childdocname~~~
    \expandafter
  \endgroup
  \childdoctmp
}
%    \end{macrocode}

% \macro{\childdoc}
% The deprecated macro |\childdoc| is a legacy version of |\childdocmain|:
%    \begin{macrocode}
\newcommand{\childdoc}{\childdocmain}
%    \end{macrocode}

% \macro{\childdocredirect}
% The deprecated macro |\childdocredirect| is a legacy version
% of |\childdocforward| and |\childdocforwardprefix|:
%    \begin{macrocode}
\newcommand{\childdocredirect}[2][]
{
  \begingroup
    \if?#1?
      \def\childdoctmp{\childdocforward{#2}}
    \else
      \def\childdoctmp{\childdocforwardprefix{#1}{#2}}
    \fi
    \expandafter
  \endgroup
  \childdoctmp
}
%    \end{macrocode}

%\iffalse
%</package>
%\fi
%
\endinput

\childdocforwardprefix[cdocsamp]{cdocsfn}{cdocsch}
%    \end{macrocode}

%\iffalse
%</samplefinal>
%\fi
%
% %%%%%%%%%%%%%%%%%%%%%%%%%%%%%%%%%%%%%%
% \paragraph{Command Line Processing.}
%
% The following three command lines generate the output files
% |cdocscld|, |cdocscl1| and |cdocscl2|
% which should be identical to
% |cdocsdrf|, |cdocsch1| and |cdocsfn2|, respectively:
% \begin{center}
% \begin{tabular}{l}
% |latex -jobname cdocscld \|\\
% |  "\def\version{draft}% \iffalse
%
% childdoc.dtx Copyright (C) 2017-2018 Niklas Beisert
%
% This work may be distributed and/or modified under the
% conditions of the LaTeX Project Public License, either version 1.3
% of this license or (at your option) any later version.
% The latest version of this license is in
%   http://www.latex-project.org/lppl.txt
% and version 1.3 or later is part of all distributions of LaTeX
% version 2005/12/01 or later.
%
% This work has the LPPL maintenance status `maintained'.
%
% The Current Maintainer of this work is Niklas Beisert.
%
% This work consists of the files childdoc.dtx and childdoc.ins
% and the derived files childdoc.def and cdocsamp.tex with
% cdocsch1.tex, cdocsch2.tex, cdocsdrf.tex, cdocsfn1.tex, cdocsfn2.tex.
%
%<package>\ifdefined\childdocmain\endinput\fi
%<package>\ProvidesFile{childdoc.def}[2018/12/30 v2.0 child document driver]
%<samplemain>\ProvidesFile{cdocsamp.tex}[2018/12/30 v2.0 sample for childdoc]
%<*driver>
%\ProvidesFile{childdoc.drv}[2018/12/30 v2.0 childdoc reference manual file]
\PassOptionsToClass{10pt,a4paper}{article}
\documentclass{ltxdoc}

\usepackage[margin=35mm]{geometry}
\usepackage{hyperref}
\usepackage{hyperxmp}
\usepackage[usenames]{color}

\hypersetup{colorlinks=true}
\hypersetup{pdfstartview=FitH}
\hypersetup{pdfpagemode=UseNone}
\hypersetup{pdfsource={}}
\hypersetup{pdflang={en-UK}}
\hypersetup{pdfcopyright={Copyright 2017-2018 Niklas Beisert.
  This work may be distributed and/or modified under the
  conditions of the LaTeX Project Public License, either version 1.3
  of this license or (at your option) any later version.}}
\hypersetup{pdflicenseurl={http://www.latex-project.org/lppl.txt}}
\hypersetup{pdfcontactaddress={ETH Zurich, ITP, HIT K,
  Wolfgang-Pauli-Strasse 27}}
\hypersetup{pdfcontactpostcode={8093}}
\hypersetup{pdfcontactcity={Zurich}}
\hypersetup{pdfcontactcountry={Switzerland}}
\hypersetup{pdfcontactemail={nbeisert@itp.phys.ethz.ch}}
\hypersetup{pdfcontacturl={http://people.phys.ethz.ch/\xmptilde nbeisert/}}

\newcommand{\secref}[1]{\hyperref[#1]{section \ref*{#1}}}

\parskip1ex
\parindent0pt
\let\olditemize\itemize
\def\itemize{\olditemize\parskip0pt}

\begin{document}

\title{The \textsf{childdoc} Package}
\hypersetup{pdftitle={The childdoc Package}}
\author{Niklas Beisert\\[2ex]
  Institut f\"ur Theoretische Physik\\
  Eidgen\"ossische Technische Hochschule Z\"urich\\
  Wolfgang-Pauli-Strasse 27, 8093 Z\"urich, Switzerland\\[1ex]
  \href{mailto:nbeisert@itp.phys.ethz.ch}
  {\texttt{nbeisert@itp.phys.ethz.ch}}}
\hypersetup{pdfauthor={Niklas Beisert}}
\hypersetup{pdfsubject={Manual for the LaTeX2e Package childdoc}}
\date{30 December 2018, \textsf{v2.0}}
\maketitle

\begin{abstract}\noindent
\textsf{childdoc} is a \LaTeXe{} package
that enables the direct compilation
of document sections included by |\include|
to individual files.
\end{abstract}

\begingroup
\parskip0ex
\tableofcontents
\endgroup

%%%%%%%%%%%%%%%%%%%%%%%%%%%%%%%%%%%%%%%%%%%%%%%%%%%%%%%%%%%%%%%%%%%%%%%%%%%%%%%%
%%%%%%%%%%%%%%%%%%%%%%%%%%%%%%%%%%%%%%%%%%%%%%%%%%%%%%%%%%%%%%%%%%%%%%%%%%%%%%%%
\section{Introduction}

\LaTeX{} provides a mechanism to structure a large document (such as a book)
into a main file and several child files (containing the chapters)
using the |\include| command.
This mechanism is beneficial for documents
which span hundreds of pages in order to
make the source file(s) more manageable.
Moreover, compilation can be restricted to
selected child files by means of the |\includeonly| command.
The latter feature can be used to reduce the compilation time while editing
(this was significantly more useful in the earlier days of \LaTeX{})
or to generate a smaller document which is easier to navigate.
Another application of |\includeonly| is to generate
documents consisting of selected parts of the complete document.

However, there are a few drawbacks of the plain |\include| mechanism:
\begin{itemize}
\item
The child files cannot be compiled on their own,
they can only be compiled via the main file.
A naive editing environment
(such as a text editor with an option
to have the current file processed by \LaTeX)
may require one to switch to the main file before compiling;
attempting to compile the child file produces errors.
\item
The main file must be modified (each time)
to adjust the |\includeonly| command
to the present needs. This easily leaves the main file in a messy state.
\item
The generated document will always carry the filename
of the main document. This is inconvenient if
several child files are to be compiled and
to be kept for distribution.
\end{itemize}

The present package provides a simple interface
to make child files individually compilable by \LaTeX{}.
Compiling a child file then has the same effect as compiling
the main file with an |\includeonly| command
to select the appropriate child.
Moreover the generated document will carry the name of the child
rather than the main file.
This resolves all three above issues.

This feature is meant to make the editing of books,
thesis documents and lecture notes somewhat more convenient.
However, the package can also be used efficiently for
composing a series of documents (such as exercise sheets)
which are typically distributed individually.
It then assists the author in generating the individual documents
(potentially in different versions)
as well as a document containing the collected series.
Another application is in developing style files
or other kinds of included material
where compilation of the style file could redirect
to a sample or test file.

%%%%%%%%%%%%%%%%%%%%%%%%%%%%%%%%%%%%%%%%%%%%%%%%%%%%%%%%%%%%%%%%%%%%%%%%%%%%%%%%
%%%%%%%%%%%%%%%%%%%%%%%%%%%%%%%%%%%%%%%%%%%%%%%%%%%%%%%%%%%%%%%%%%%%%%%%%%%%%%%%
\section{Usage}

First of all, the package \textsf{childdoc} is \emph{not} a standard
\LaTeXe{} |.sty| style file! Therefore it needs to be invoked in
a non-standard way.

%%%%%%%%%%%%%%%%%%%%%%%%%%%%%%%%%%%%%%%%%%%%%%%%%%%%%%%%%%%%%%%%%%%%%%%%%%%%%%%%
\subsection{Included Files}
\label{sec:include}

%%%%%%%%%%%%%%%%%%%%%%%%%%%%%%%%%%%%%%%%
\DescribeMacro{\childdocmain}
To use the package, add the commands
\begin{center}
\begin{tabular}{l}
|\input{childdoc.def}|\\
|\childdocmain{}|\\
\end{tabular}
\end{center}
at the very top of the main \LaTeX{} file,
in particular \emph{before} the |\documentclass| statement!
The argument of |\childdocmain| should be left empty
(but it must be present).

%%%%%%%%%%%%%%%%%%%%%%%%%%%%%%%%%%%%%%%%
\DescribeMacro{\childdocof}
Furthermore, add the commands
\begin{center}
\begin{tabular}{l}
|\input{childdoc.def}|\\
|\childdocof{|\textit{main}|}|\\
\end{tabular}
\end{center}
at the top of every child file \textit{child}
which is included by |\include{|\textit{child}|}|
from within the main file
(or at least for those files to be compiled individually).
The argument \textit{main} must be the filename of the main file.

There are a couple of
considerations in setting up the main and child documents:

%%%%%%%%%%%%%%%%%%%%%%%%%%%%%%%%%%%%%%%%
\paragraph{Restrictions.}

Please note the following restrictions:
\begin{itemize}
\item
|\childdocmain| must be called with one argument \textit{main}
to ensure compatibility with earlier version of the package.
It must either be empty (|\childdocmain{}|)
or precisely match the filename of the main file in which it is specified.
See \secref{sec:detection} for further information.
\item
The filename \textit{main} must be specified without the |.tex| extension.
\item
The filename \textit{main} is case sensitive
(even in case-insensitive file systems)
due to internal string comparison.
\item
The argument \textit{main} should be fully expanded, it cannot be a macro.
\item
Subdirectories and special characters should be avoided in filenames.
\item
The command |\childdocmain{|\textit{main}|}| must be followed by a whitespace.
It should not be followed immediately by another command
or by a comment mark `|%|'.
This is because the \TeX{} parser reads the token immediately following
the argument of |\childdocmain| and puts it
at the beginning of every child section;
however, a white\-space is ignored.
\end{itemize}

%%%%%%%%%%%%%%%%%%%%%%%%%%%%%%%%%%%%%%%%
\paragraph{Content of Main File.}

It is advisable to place all content in the child files included by |\include|.
Any output contained in the main file will appear in all child documents
unless suppressed manually;
it cannot be suppressed automatically by the |\includeonly| directive
and thus should normally be avoided.
A method to include some content in the main file
by means of conditional processing is described in \secref{sec:conditional}.

%%%%%%%%%%%%%%%%%%%%%%%%%%%%%%%%%%%%%%%%
\paragraph{Page Numbering.}

When only a part of the document is compiled,
the appropriate numbering of pages
(as well as other status parameters)
is determined from the |.aux| files.
The latter contain information from previous passes.
However this information needs to propagate through
all intermediate child documents.
Therefore the page numbering in child documents may well
be inconsistent until the complete document is compiled at least once.

A useful (if unconventional) way to always ensure a consistent
page numbering is to restart the numbering in each child document
and denote the pages by `\textit{child}|.|\textit{page}'
where \textit{child} represents the chapter/section number of the child file.
This can be achieved by the command
|\numberwithin{page}{|\textit{child}|}|
of the \textsf{amsmath} package
where \textit{child} can be |chapter| or |section|
depending on the chosen structuring.
Alternatively, one can modify the macro |\thepage| appropriately
and reset the counter |page| at the start of each child file.

%%%%%%%%%%%%%%%%%%%%%%%%%%%%%%%%%%%%%%%%%%%%%%%%%%%%%%%%%%%%%%%%%%%%%%%%%%%%%%%%
\subsection{Conditional Processing}
\label{sec:conditional}

The package provides a mechanism to compile different versions
of a document. To customise the versions further some conditional processing
can come in handy to distinguish which version is being compiled.
The package provides two macros to describe the compilation context:

%%%%%%%%%%%%%%%%%%%%%%%%%%%%%%%%%%%%%%%%
\DescribeMacro{\ifchilddoc}
The conditional |\ifchilddoc| distinguishes between the compilation of
child documents and the main document:
%
\begin{center}
|\ifchilddoc |\textit{child-code}| |[|\||else |\textit{main-code}]| \||fi|
\end{center}

%%%%%%%%%%%%%%%%%%%%%%%%%%%%%%%%%%%%%%%%
\DescribeMacro{\childdocname}
\DescribeMacro{\childdocjob}
The macro |\childdocname| contains the filename (without extension)
of the main or child file being processed.
Note that |\childdocjob| will always contain the name of the main file.

%%%%%%%%%%%%%%%%%%%%%%%%%%%%%%%%%%%%%%%%
\paragraph{Title Page.}

Conditional processing can be used to include a title or banner page
in the main document when proper precautions are taken.
Importantly, the code in the main file should ensure that the page counter
(as well as other status parameters which are stored in the |.aux| files)
takes the same value after the conditional processing.
Otherwise the page numbers may take divergent values
depending on which part is compiled.

For example, a title page could be declared by:
%
\begin{center}
\begin{tabular}{l}
|\ifchilddoc\||else|\\
|\addtocounter{page}{-1}|\\
\textit{code for title page}\\
|\newpage|\\
|\||fi|
\end{tabular}
\end{center}
%
A banner page for the child documents can be generated by:
%
\begin{center}
\begin{tabular}{l}
|\ifchilddoc|\\
|\addtocounter{page}{-1}|\\
\textit{code for banner page}\\
|\newpage|\\
|\||fi|
\end{tabular}
\end{center}
%
Here one could write a message such as:
\begin{center}
|This is the part \childdocname{} of \childdocjob{}.|
\end{center}

%%%%%%%%%%%%%%%%%%%%%%%%%%%%%%%%%%%%%%%%%%%%%%%%%%%%%%%%%%%%%%%%%%%%%%%%%%%%%%%%
\subsection{Flags}
\label{sec:flags}

The package makes it easy to generate different versions
of the main or child documents.
To this end compilation flags can be defined
and assigned different default values.
They will be particularly useful in conjunction
with the forwarding mechanism described in \secref{sec:forward}.

For example, it may be useful to have a flag |\version|
which can be set to |draft| or |final|.
The document source will contain some conditional code
depending on the value of |\version|.
Suppose further, the flag should default to |final| for the main file
and to |draft| for child files
which is a natural assignment for editing the document.
This is achieved by placing the following code
in the preamble of the main document
(below the |\childdocmain| directive):
%
\begin{center}
\begin{tabular}{l}
|\ifchilddoc|\\
|\providecommand{\version}{draft}|\\
|\||else|\\
|\providecommand{\version}{final}|\\
|\||fi|
\end{tabular}
\end{center}
%
The definition by |\providecommand| makes sure
that previous definitions are not overwritten.
Further statements |\providecommand{\version}{...}|
can thus be added before the above code to override it.

For the main file, one might add a line
(between |\childdocmain| and the above block)
%
\begin{center}
|%\ifchilddoc\||else\providecommand{\version}{draft}\||fi|
\end{center}
%
which can be uncommented to produce a draft version.
Likewise one can add a line to the very top of a child file
(above the |\childdocof{|\textit{main}|}| directive)
%
\begin{center}
|%\providecommand{\version}{final}|
\end{center}
%
which can be uncommented to produce the final version of this child document.

%%%%%%%%%%%%%%%%%%%%%%%%%%%%%%%%%%%%%%%%%%%%%%%%%%%%%%%%%%%%%%%%%%%%%%%%%%%%%%%%
\subsection{Forwarding}
\label{sec:forward}

Different versions of the main or child documents
using compilation flags as described in \secref{sec:flags}
can be (permanently) stored in different files
for convenient compilation, viewing and distribution.
To this end, the package defines a command
to pass on compilation to a different file:

%%%%%%%%%%%%%%%%%%%%%%%%%%%%%%%%%%%%%%%%
\DescribeMacro{\childdocforward}
The command |\childdocforward| redirects processing to
another source file:
%
\begin{center}
\begin{tabular}{l}
|\input{childdoc.def}|\\
|\childdocforward[|\textit{main}|]{|\textit{dest}|}|\\
\end{tabular}
\end{center}
%
The argument \textit{dest} is the destination file
(without extension).
It should be the main file or one of the child files.
Note that further \textsf{childdoc} directives
such as |\childdocof| and |\childdocforward|
in the indicated file will be processed in this form.
The optional argument \textit{main}
passes on directly to the main file \textit{main}
while pretending to compile the child \textit{dest}.
This form behaves as if \textit{dest}
issues |\childdocof{|\textit{main}|}| right away,
and no further \textsf{childdoc} directives will be processed.

%%%%%%%%%%%%%%%%%%%%%%%%%%%%%%%%%%%%%%%%
\DescribeMacro{\...prefix}
In the alternative form |\childdocforwardprefix|,
%
\begin{center}
\begin{tabular}{l}
|\input{childdoc.def}|\\
|\childdocforwardprefix[|\textit{main}|]{|\textit{prefix}|}{|\textit{dest}|}|
\end{tabular}
\end{center}
%
the destination file is determined by a pattern
depending on the current file:
To make this work, the current file must be called
`{\textit{prefix}\hspace{0.2em}\textit{suffix}}'
with \textit{prefix} matching precisely the argument.
Processing is then passed on to the file
`{\textit{dest}\hspace{0.2em}\textit{suffix}}'.
Surely, the same effect is achieved by
directly specifying the
argument `{\textit{dest}\hspace{0.2em}\textit{suffix}}'
in the first form.
However, that requires to set up a different file
for each child. With the alternative form of the command
all these files can have exactly the same content
which simplifies setting them up and maintaining them.

For example, the following file |draft.tex|
with a compilation flag |\version| as described in \secref{sec:flags}
compiles the main document as a draft:
%
\begin{center}
\begin{tabular}{l}
|\def\version{draft}|\\
|\input{childdoc.def}|\\
|\childdocforward{|\textit{main}|}|
\end{tabular}
\end{center}
%
Likewise, the following files |final|\textit{nn}|.tex|
compile the final version of the child document
|child|\textit{nn}|.tex|:
%
\begin{center}
\begin{tabular}{l}
|\def\version{final}|\\
|\input{childdoc.def}|\\
|\childdocforwardprefix{final}{child}|
\end{tabular}
\end{center}
%

Note that when several versions of a main file and/or of each child file
are to be generated, it may be convenient to set up a |Makefile| or
shell script to automatise the process.

%%%%%%%%%%%%%%%%%%%%%%%%%%%%%%%%%%%%%%%%%%%%%%%%%%%%%%%%%%%%%%%%%%%%%%%%%%%%%%%%
\subsection{Command Line Processing}
\label{sec:commandline}

The effect of redirection files can also be achieved by invoking
the \LaTeX{} compiler with a more elaborate command line.
Most conveniently this should be done as part
of a shell script or a |Makefile|.

When using \textsf{childdoc} in the main file, the following
command lines effectively perform a redirection
(note that depending on the shell being used,
backslashes may have to be doubled: `|\|' $\to$ `|\\|'):
%
\begin{center}
|... -jobname "|\textit{target}|" |\\|"|[\textit{flags}]%
|\input{childdoc.def}\childdocforward[|\textit{main}|]{|\textit{dest}|}"|
\end{center}
%
Here \textit{target} is the name of the output file,
\textit{main} is the name of the main file
and \textit{dest} is the name of the main or child file to be processed
(all filenames without extensions).
The optional argument \textit{main} can be omitted
if \textit{main} matches \textit{dest}.
Optionally, compilation \textit{flags} can be defined via |\def| commands.
This command line makes the \TeX{} engine believe
it is compiling the file \textit{target}
whose content is specified as the latter parameter.
The provided code then forwards the processing to
\textit{main} or \textit{dest} as described in \secref{sec:forward}.

%%%%%%%%%%%%%%%%%%%%%%%%%%%%%%%%%%%%%%%%%%%%%%%%%%%%%%%%%%%%%%%%%%%%%%%%%%%%%%%%
\subsection{Include by Input}
\label{sec:input}

Including child documents by |\include| has some restrictions by design.
Most notably, the content of a child document always occupies
its own set of pages; pages cannot be shared between child documents.
Usually, this behaviour makes perfect sense
because each child document contain an essential part of the document.
However, in some situations it may be desirable to compose
a document from a collection of parts
without having mandatory page breaks between then.
For this case, the package
provides a mechanism to include parts
by |\input| which can also be processed individually.
However, by construction this mechanism
requires manual handling of the content to be output.

%%%%%%%%%%%%%%%%%%%%%%%%%%%%%%%%%%%%%%%%
\DescribeMacro{\ifchilddocmanual}
The main file should be prepared as usual, see \secref{sec:include}.
However, the document body must make a distinction
between processing of an individual part and of the main document, e.g.:
%
\begin{center}
\begin{tabular}{l}
|\ifchilddocmanual|\\
|\input{\childdocname}|\\
|\||else|\\
\textit{document body with }|\input{|\textit{part}|}|\\
|\||fi|
\end{tabular}
\end{center}
%
The conditional |\ifchilddocmanual| is true whenever
a part to be included by |\input| is being compiled,
and the name of the part is stored in |\childdocname|.

%%%%%%%%%%%%%%%%%%%%%%%%%%%%%%%%%%%%%%%%
\DescribeMacro{\childdocby}
Each part to be included by |\input| should start with:
%
\begin{center}
\begin{tabular}{l}
|\input{childdoc.def}|\\
|\childdocby{|\textit{main}|}|\\
\end{tabular}
\end{center}
%
The directive |\childdocby| is similar to |\childdocof|
described in \secref{sec:include},
but the subsequent selection of content must be done manually.
To that end, both |\ifchilddoc| and |\ifchilddocmanual|
will be true upon processing of a part,
and the name of the part is stored in |\childdocname|.
Note that |\jobname| will be set to the filename of the current part
so that each part receives an individual |.aux| file
that does not interfere with the |.aux| file(s) of the main document.
This behaviour can be altered by the alternative form
|\childdocby[*]{|\textit{main}|}| (with a non-empty optional argument)
which uses the |.aux| file of the main document
by setting |\jobname| to \textit{main}.

%%%%%%%%%%%%%%%%%%%%%%%%%%%%%%%%%%%%%%%%%%%%%%%%%%%%%%%%%%%%%%%%%%%%%%%%%%%%%%%%
\subsection{Driver Development}
\label{sec:driver}

The \textsf{childdoc} mechanism can also be use for the development
of definition files such as \LaTeX{} styles or classes.
This case differs from the above setup with multiple parts
included by |\include| in that no |\includeonly| should be invoked.
This can be achieved by starting the include file
(before |\ProvidesPackage|) with:
%
\begin{center}
\begin{tabular}{l}
|\input{childdoc.def}|\\
|\childdocforward{|\textit{main}|}|\\
\end{tabular}
\end{center}
%
or alternatively with:
%
\begin{center}
\begin{tabular}{l}
|\input{childdoc.def}|\\
|\childdocby{|\textit{main}|}|\\
\end{tabular}
\end{center}
%
Both forms have slightly different effects as described above.
The main file is prepared as usual, see \secref{sec:include}.

%%%%%%%%%%%%%%%%%%%%%%%%%%%%%%%%%%%%%%%%%%%%%%%%%%%%%%%%%%%%%%%%%%%%%%%%%%%%%%%%
\subsection{Legacy Detection}
\label{sec:detection}

The directive |\childdocmain| in the main file can detect
whether the complete document or merely a child is to be compiled
even without using the directive |\childdocof|.
This method is deprecated because it is less robust
and there is no compelling reason to use it;
it is merely provided for backward compatibility
and it may be removed in future versions.

If the detection mechanism is to be used,
it is mandatory to correctly specify
the filename of the main file as the argument of |\childdocmain|:
%
\begin{center}
\begin{tabular}{l}
|\input{childdoc.def}|\\
|\childdocmain{|\textit{main}|}|\\
\end{tabular}
\end{center}
%
If |\jobname| does not match the argument \textit{main} of |\childdocmain|,
it is assumed that |\jobname| points to the child file to be compiled.
When using |\childdocmain| with the main file specified as argument,
it suffices to start a child file
with just |\input{|\textit{main}|}|
without loading of the package and using |\childdocof|.
If instead all processing is done
with the appropriate \textsf{childdoc} directives,
the argument of \textit{main} of |\childdocmain| can be empty.

An alternative version of the command line processing described
in \secref{sec:commandline} using the detection mechanism reads:
%
\begin{center}
|... -jobname "|\textit{target}|" "|[\textit{flags}]%
[|\def\jobname{|\textit{dest}|}|]|\input{|\textit{main}|}"|
\end{center}

%%%%%%%%%%%%%%%%%%%%%%%%%%%%%%%%%%%%%%%%%%%%%%%%%%%%%%%%%%%%%%%%%%%%%%%%%%%%%%%%
\subsection{Manual Code}
\label{sec:manual}

In case one cannot be certain whether the definitions file |childdoc.def|
is installed on the target \TeX{} distribution
and one prefers not to ship it,
it is conceivable to paste a few relevant commands into the sources.

To that end, drop all statements |\input{childdoc.def}|
and perform the replacements as outlined below.
Instead of |\childdocmain{|\textit{main}|}| add the following code
to the top of the main file:
%
\begin{center}
\begin{tabular}{l}
|\||ifdefined\childdocname\endinput\||fi\newif\ifchilddoc|\\
|\edef\childdocname{\scantokens\expandafter{\jobname\noexpand}}|\\
|\def\childdocmain{|\textit{main}|}\||ifx\childdocmain\childdocname\||else|\\
|\childdoctrue\includeonly{\childdocname}\let\jobname\childdocmain\||fi|\\
\end{tabular}
\end{center}
%
Instead of |\childdocof{|\textit{main}|}| just include the main file
at the top of each child file:
%
\begin{center}
|\input{|\textit{main}|}|
\end{center}
%
A simple redirection |\childdocforward{|\textit{dest}|}| is achieved by:
%
\begin{center}
|\def\jobname{|\textit{dest}|}\input{\jobname}|
\end{center}
%
The redirection with prefix
|\childdocforwardprefix[|\textit{prefix}|]{|\textit{dest}|}|
is accomplished by:
%
\begin{center}
\begin{tabular}{l}
|{\edef\jobname{\scantokens\expandafter{\jobname\noexpand}}|\\
|\def\redirectjob |\textit{prefix}|#1~~~{\gdef\jobname{|\textit{dest}|#1}}|\\
|\expandafter\redirectjob\jobname~~~}\input{\jobname}|
\end{tabular}
\end{center}

In an alternative approach,
child documents can be compiled by a specific command line
without additional code or specific definitions:
%
\begin{center}
|... -jobname "|\textit{target}|" "|[\textit{flags}]%
|\includeonly{|\textit{dest}|}\input{|\textit{main}|}"|
\end{center}
%

%%%%%%%%%%%%%%%%%%%%%%%%%%%%%%%%%%%%%%%%%%%%%%%%%%%%%%%%%%%%%%%%%%%%%%%%%%%%%%%%
%%%%%%%%%%%%%%%%%%%%%%%%%%%%%%%%%%%%%%%%%%%%%%%%%%%%%%%%%%%%%%%%%%%%%%%%%%%%%%%%
\section{Information}

%%%%%%%%%%%%%%%%%%%%%%%%%%%%%%%%%%%%%%%%%%%%%%%%%%%%%%%%%%%%%%%%%%%%%%%%%%%%%%%%
\subsection{Copyright}

Copyright \copyright{} 2017--2018 Niklas Beisert

This work may be distributed and/or modified under the
conditions of the \LaTeX{} Project Public License, either version 1.3
of this license or (at your option) any later version.
The latest version of this license is in
  \url{http://www.latex-project.org/lppl.txt}
and version 1.3 or later is part of all distributions of \LaTeX{}
version 2005/12/01 or later.

This work has the LPPL maintenance status `maintained'.

The Current Maintainer of this work is Niklas Beisert.

This work consists of the files |README.txt|, |childdoc.ins| and |childdoc.dtx|
as well as the derived files |childdoc.def|, |cdocsamp.tex|
with |cdocsch1.tex|, |cdocsch2.tex|, |cdocspt3.tex|, |cdocspt4.tex|,
|cdocsdrf.tex|, |cdocsfn1.tex|, |cdocsfn2.tex|
as well as |childdoc.pdf|.

%%%%%%%%%%%%%%%%%%%%%%%%%%%%%%%%%%%%%%%%%%%%%%%%%%%%%%%%%%%%%%%%%%%%%%%%%%%%%%%%
\subsection{Files and Installation}

The package consists of the files:
%
\begin{center}
\begin{tabular}{ll}
    |README.txt|   & readme file \\
    |childdoc.ins| & installation file \\
    |childdoc.dtx| & source file \\
    |childdoc.def| & definition file \\
    |cdocsamp.tex| & sample main file \\
    |cdocsch1.tex| & sample include file \\
    |cdocsch2.tex| & sample include file \\
    |cdocspt3.tex| & sample part file \\
    |cdocspt4.tex| & sample part file \\
    |cdocsdrf.tex| & sample redirection file \\
    |cdocsfn1.tex| & sample redirection file \\
    |cdocsfn2.tex| & sample redirection file \\
    |childdoc.pdf| & manual
\end{tabular}
\end{center}
%
The distribution consists of the files
|README.txt|, |childdoc.ins| and |childdoc.dtx|.
%
\begin{itemize}
\item
Run (pdf)\LaTeX{} on |childdoc.dtx|
to compile the manual |childdoc.pdf| (this file).
\item
Run \LaTeX{} on |childdoc.ins| to create the definitions file |childdoc.def|
and the sample |cdocsamp.tex| with include files
|cdocsch1.tex|, |cdocsch2.tex|, |cdocspt3.tex|, |cdocspt4.tex|,
|cdocsdrf.tex|, |cdocsfn1.tex|, |cdocsfn2.tex|.
Then copy the file |childdoc.def| to an appropriate directory of your \LaTeX{}
distribution, e.g.\ \textit{texmf-root}|/tex/latex/childdoc|.
\end{itemize}

%%%%%%%%%%%%%%%%%%%%%%%%%%%%%%%%%%%%%%%%%%%%%%%%%%%%%%%%%%%%%%%%%%%%%%%%%%%%%%%%
\subsection{Related CTAN Packages}

There are several other packages which offer a similar functionality:
%
\begin{itemize}
\item
The packages
\href{http://ctan.org/pkg/docmute}{\textsf{docmute}},
\href{http://ctan.org/pkg/includex}{\textsf{includex}} and
\href{http://ctan.org/pkg/standalone}{\textsf{standalone}}
provide commands to include only the document body of
a child file thus allowing both files to be compiled individually.
\item
The packages \href{http://ctan.org/pkg/subdocs}{\textsf{subdocs}}
and \href{http://ctan.org/pkg/subfiles}{\textsf{subfiles}}
provide structures in which the main and child documents can be
encapsulated and allowing them to be compiled individually.
The inclusion mechanism is different from the conventional |\include|.
\item
The package \href{http://ctan.org/pkg/combine}{\textsf{combine}}
is an elaborate solution to combine several documents into one.
\end{itemize}
%
See also the CTAN topic \href{http://ctan.org/topic/subdocs}{\textsf{subdocs}}
for further related packages.
The present package differs from the above solutions in that
a document structure constructed with the conventional |\include| mechanism
just needs two extra commands at the top of every file
such that all constituent files can be compiled individually.

%%%%%%%%%%%%%%%%%%%%%%%%%%%%%%%%%%%%%%%%%%%%%%%%%%%%%%%%%%%%%%%%%%%%%%%%%%%%%%%%
%\subsection{Feature Suggestions}
%
%The following is a list of features which may be useful for future
%versions of this package:
%%
%\begin{itemize}
%\item
%\ldots
%\end{itemize}

%%%%%%%%%%%%%%%%%%%%%%%%%%%%%%%%%%%%%%%%%%%%%%%%%%%%%%%%%%%%%%%%%%%%%%%%%%%%%%%%
\subsection{Revision History}

%%%%%%%%%%%%%%%%%%%%%%%%%%%%%%%%%%%%%%%%
\paragraph{v2.0:} 2018/12/30

\begin{itemize}
\item
immediate forward processing
\item
added |\childdocby| mechanism
\item
manual restructured
\end{itemize}

%%%%%%%%%%%%%%%%%%%%%%%%%%%%%%%%%%%%%%%%
\paragraph{v1.6:} 2018/01/17

\begin{itemize}
\item
application for development of include files
\item
corrections to manual
\end{itemize}

%%%%%%%%%%%%%%%%%%%%%%%%%%%%%%%%%%%%%%%%
\paragraph{v1.5:} 2017/05/21

\begin{itemize}
\item
more complete structuring introduced
\item
|\childdocof| introduced
\item
|\childdoc| renamed to |\childdocmain|
\item
|\childredirect| renamed to |\childdocforward| and |\childdocforwardprefix|
and functionality expanded
\end{itemize}

%%%%%%%%%%%%%%%%%%%%%%%%%%%%%%%%%%%%%%%%
\paragraph{v1.0:} 2017/04/27

\begin{itemize}
\item
manual and install package
\item
first version published on CTAN
\end{itemize}

%%%%%%%%%%%%%%%%%%%%%%%%%%%%%%%%%%%%%%%%
\paragraph{v0.6:} 2017/04/26

\begin{itemize}
\item
redirection mechanism added
\end{itemize}

%%%%%%%%%%%%%%%%%%%%%%%%%%%%%%%%%%%%%%%%
\paragraph{v0.5:} 2017/04/26

\begin{itemize}
\item
functionality in definition file
\end{itemize}


%%%%%%%%%%%%%%%%%%%%%%%%%%%%%%%%%%%%%%%%%%%%%%%%%%%%%%%%%%%%%%%%%%%%%%%%%%%%%%%%
%%%%%%%%%%%%%%%%%%%%%%%%%%%%%%%%%%%%%%%%%%%%%%%%%%%%%%%%%%%%%%%%%%%%%%%%%%%%%%%%
%%%%%%%%%%%%%%%%%%%%%%%%%%%%%%%%%%%%%%%%%%%%%%%%%%%%%%%%%%%%%%%%%%%%%%%%%%%%%%%%
\appendix

\settowidth\MacroIndent{\rmfamily\scriptsize 000\ }

 \DocInput{childdoc.dtx}

\end{document}
%</driver>
% \fi
%
% %%%%%%%%%%%%%%%%%%%%%%%%%%%%%%%%%%%%%%%%%%%%%%%%%%%%%%%%%%%%%%%%%%%%%%%%%%%%%%
% %%%%%%%%%%%%%%%%%%%%%%%%%%%%%%%%%%%%%%%%%%%%%%%%%%%%%%%%%%%%%%%%%%%%%%%%%%%%%%
% \section{Sample}
%\iffalse
%<*samplemain>
%\fi
%
% The following presents a sample document
% with two chapters, two parts, a title page,
% a compile flag as well as three forwarding files to set the flag.
% It consists of eight |.tex| files:
% \begin{center}
% \begin{tabular}{ll}
% |cdocsamp.tex|&main file\\
% |cdocsch1.tex|&include file for chapter 1\\
% |cdocsch2.tex|&include file for chapter 2\\
% |cdocspt3.tex|&include file for part 3\\
% |cdocspt4.tex|&include file for part 4\\
% |cdocsdrf.tex|&forwarding file for main file in draft mode\\
% |cdocsfi1.tex|&forwarding file for final version of chapter 1\\
% |cdocsfi2.tex|&forwarding file for final version of chapter 2\\
% \end{tabular}
% \end{center}
% Each of the eight files can be compiled directly by the \LaTeX{} compiler.
%
% %%%%%%%%%%%%%%%%%%%%%%%%%%%%%%%%%%%%%%
% \paragraph{Main File.}
%
% The main file is called |cdocsamp.tex|.
%
% Load the \textsf{childdoc} definitions and
% declare the filename for the main document:
%    \begin{macrocode}
\input{childdoc.def}
\childdocmain{}
%    \end{macrocode}

% Optional override for |\version| flag:
%    \begin{macrocode}
%%\ifchilddoc\else\providecommand{\version}{draft}\fi
%    \end{macrocode}

% Define the default values for the |\version| flag
% (|final| for the main file and |draft| for childs):
%    \begin{macrocode}
\ifchilddoc
\providecommand{\version}{draft}
\else
\providecommand{\version}{final}
\fi
%    \end{macrocode}

% Load the standard document class:
%    \begin{macrocode}
\documentclass[12pt]{article}
%    \end{macrocode}

% Start the document body:
%    \begin{macrocode}
\begin{document}
%    \end{macrocode}

% Declare a title page.
% Print title, part of document being processed and version flag:
%    \begin{macrocode}
\addtocounter{page}{-1}
\begin{center}
{\LARGE\bfseries{}childdoc example\par}
\vspace{1cm}
\ifchilddoc
\ifchilddocmanual part\else chapter\fi:
`\childdocname' of `\childdocjob'\par
\else
main document: `\childdocjob'\par
\fi
version: \version\par
\end{center}
\newpage
%    \end{macrocode}

% Manually include selected file,
% otherwise process as usual:
%    \begin{macrocode}
\ifchilddocmanual
\section*{part `\childdocname'}
\input{\childdocname}
\else
%    \end{macrocode}

% Include the two chapters:
%    \begin{macrocode}
\include{cdocsch1}
\include{cdocsch2}
%    \end{macrocode}

% Include the two parts unless only chapters should be displayed:
%    \begin{macrocode}
\ifchilddoc\else
\section{part three}
\input{cdocspt3}
\section{part four}
\input{cdocspt4}
\fi
%    \end{macrocode}

% Process as usual until here:
%    \begin{macrocode}
\fi
%    \end{macrocode}

% End of document body:
%    \begin{macrocode}
\end{document}
%    \end{macrocode}
%\iffalse
%</samplemain>
%\fi
%
% %%%%%%%%%%%%%%%%%%%%%%%%%%%%%%%%%%%%%%
% \paragraph{Chapter Include Files.}
%
% The include files are called |cdocsch1.tex| and |cdocsch2.tex|.
%
%\iffalse
%<*samplechap1|samplechap2>
%\fi

% Optional override for |\version| flag:
%    \begin{macrocode}
%%\providecommand{\version}{final}
%    \end{macrocode}

% Include the main document:
%    \begin{macrocode}
\input{childdoc.def}
\childdocof{cdocsamp}
%    \end{macrocode}

%\iffalse
%</samplechap1|samplechap2>
%\fi
%
%\iffalse
%<*samplechap1>
%\fi
% Some text for chapter 1:
%    \begin{macrocode}
\section{one}
some text in chapter one
%    \end{macrocode}

%\iffalse
%</samplechap1>
%\fi
% Some text for chapter 2:
%\iffalse
%<*samplechap2>
%\fi
%    \begin{macrocode}
\section{two}
more text in chapter two
%    \end{macrocode}

%\iffalse
%</samplechap2>
%\fi
%
% %%%%%%%%%%%%%%%%%%%%%%%%%%%%%%%%%%%%%%
% \paragraph{Part Include Files.}
%
% The include files are called |cdocspt3.tex| and |cdocspt4.tex|.
%
%\iffalse
%<*samplepart3|samplepart4>
%\fi

% Optional override for |\version| flag:
%    \begin{macrocode}
%%\providecommand{\version}{final}
%    \end{macrocode}

% Include the main document:
%    \begin{macrocode}
\input{childdoc.def}
\childdocby{cdocsamp}
%    \end{macrocode}

%\iffalse
%</samplepart3|samplepart4>
%\fi
%
%\iffalse
%<*samplepart3>
%\fi
% Some text for part 3:
%    \begin{macrocode}
some text in part three
%    \end{macrocode}

%\iffalse
%</samplepart3>
%\fi
% Some text for part 4:
%\iffalse
%<*samplepart4>
%\fi
%    \begin{macrocode}
more text in part four
%    \end{macrocode}

%\iffalse
%</samplepart4>
%\fi
%
% %%%%%%%%%%%%%%%%%%%%%%%%%%%%%%%%%%%%%%
% \paragraph{Forwarding for a Complete Draft.}
%
% The following forwarding file |cdocsdrf.tex|
% compiles the main document in draft mode:
%\iffalse
%<*sampledraft>
%\fi
%    \begin{macrocode}
\def\version{draft}
\input{childdoc.def}
\childdocforward{cdocsamp}
%    \end{macrocode}

%\iffalse
%</sampledraft>
%\fi
%
% %%%%%%%%%%%%%%%%%%%%%%%%%%%%%%%%%%%%%%
% \paragraph{Forwarding for Final Version of the Chapters.}
%
% The following forwarding files |cdocsfn1.tex| and |cdocsfn2.tex|
% (with identical content)
% compile the final versions of the child documents
% |cdocsch1.tex| and |cdocsch2.tex|, respectively:
%\iffalse
%<*samplefinal>
%\fi
%    \begin{macrocode}
\def\version{final}
\input{childdoc.def}
\childdocforwardprefix[cdocsamp]{cdocsfn}{cdocsch}
%    \end{macrocode}

%\iffalse
%</samplefinal>
%\fi
%
% %%%%%%%%%%%%%%%%%%%%%%%%%%%%%%%%%%%%%%
% \paragraph{Command Line Processing.}
%
% The following three command lines generate the output files
% |cdocscld|, |cdocscl1| and |cdocscl2|
% which should be identical to
% |cdocsdrf|, |cdocsch1| and |cdocsfn2|, respectively:
% \begin{center}
% \begin{tabular}{l}
% |latex -jobname cdocscld \|\\
% |  "\def\version{draft}\input{childdoc.def}\childdocforward{cdocsamp}"|\\
% |latex -jobname cdocscl1 \|\\
% |  "\input{childdoc.def}\childdocforward[cdocsamp]{cdocsch1}"|\\
% |latex -jobname cdocscl2 \|\\
% |  "\def\version{final}\input{childdoc.def}\childdocforward{cdocsch2}"|
% \end{tabular}
% \end{center}
% Note that the trailing backslash on each first line
% merely continues the input to the second line
% (for convenient cut ant paste).
% Furthermore, the command |latex| can be replaced by any
% of its alternative versions such as |pdflatex|.
%
% %%%%%%%%%%%%%%%%%%%%%%%%%%%%%%%%%%%%%%%%%%%%%%%%%%%%%%%%%%%%%%%%%%%%%%%%%%%%%%
% %%%%%%%%%%%%%%%%%%%%%%%%%%%%%%%%%%%%%%%%%%%%%%%%%%%%%%%%%%%%%%%%%%%%%%%%%%%%%%
% \section{Implementation}
%\iffalse
%<*package>
%\fi
%
% This section describes the definitions file |childdoc.def|.

% The definitions cannot be loaded using |\usepackage| or |\RequirePackage|
% which has a mechanism to prevent loading a style file more than once.
% When loading the definitions by means of |\input|
% multiple instances have to be prevented manually:
%\iffalse
%This code needs to be before the `\ProvidesFile' directive
%which is defined at the beginning of this file.
%Therefore it is also placed there and commented out here.
%</package>
%<*discard>
%\fi
%    \begin{macrocode}
\ifdefined\childdocmain\endinput\fi
%    \end{macrocode}
%\iffalse
%</discard>
%<*package>
%\fi
%
% \macro{\ifchilddoc}
% \macro{\ifchilddocmanual}
% The conditional |\ifchilddoc| tells whether a
% child (true) or main (false) document is being compiled.
% The conditional |\ifchilddocmanual| tells whether
% the |\includeonly| mechanism is used (false) or
% the selection of child files must be performed manually (true).
% The definitions initialise to false:
%    \begin{macrocode}
\newif\ifchilddoc
\newif\ifchilddocmanual
%    \end{macrocode}

% \macro{\childdocname}
% \macro{\childdocjob}
% The macro |\childdocname| stores the name of the main document
% to be compiled. The macro |\childdocjob| stores the name of
% the document on which the \LaTeX{} compiler was originally invoked.
% The content of |\jobname| cannot be compared
% to filenames specified in the source due to different catcodes.
% The following code rescans |\jobname|, stores the result
% in |\childdocname| and saves a copy in |\childdocjob|:
%    \begin{macrocode}
\edef\childdocname{\scantokens\expandafter{\jobname\noexpand}}
\let\childdocjob\childdocname
%    \end{macrocode}

% \macro{\childdocdisable}
% The macro |\childdocdisable| prevents the main file
% from being processed more than once.
% At this stage, the main document command |\childdocmain|
% is assumed to be called once again where it should do nothing.
% Any subsequent call to it should prevent
% a secondary processing of the main document
% It overwrites the forwarding commands
% |\childdocof| and |\childdocforward|
% with empty macros to prevent further inclusions of the main document:
%    \begin{macrocode}
\newcommand{\childdocdisable}
{
  \renewcommand{\childdocmain}[1]{\renewcommand{\childdocmain}[1]{\endinput}}
  \renewcommand{\childdocof}[1]{}
  \renewcommand{\childdocby}[2][]{}
  \renewcommand{\childdocforward}[2][]{}
  \renewcommand{\childdocdisable}{}
}
%    \end{macrocode}

% \macro{\childdocmain}
% The macro |\childdocmain| is to be called at the top of the main file
% with nothing or the main filename (without extension) as argument.
% First, it breaks loops.
% If the argument is not empty and does not match |\childdocname|
% (which is set by the first inclusion of |childdoc.def|),
% |\ifchilddoc| is set to true, |\includeonly| is applied to the child file
% and |\jobname| is set to the main file
% (for proper handling of |.aux| files):
%    \begin{macrocode}
\newcommand{\childdocmain}[1]
{
  \childdocdisable\childdocmain{}
  \if?#1?\else
    \begingroup
      \def\childdoctmp{#1}
      \ifx\childdoctmp\childdocname
        \def\childdoctmp{}
      \else
        \def\childdoctmp
        {
          \childdoctrue
          \includeonly{\childdocname}
          \def\childdocjob{#1}
          \def\jobname{#1}
        }
      \fi
      \expandafter
    \endgroup
    \childdoctmp
  \fi
}
%    \end{macrocode}

% \macro{\childdocof}
% The command |\childdocof| redirects
% compilation to the main file |#1|.
%    \begin{macrocode}
\newcommand{\childdocof}[1]
{
  \childdocdisable
  \childdoctrue
  \includeonly{\childdocname}
  \def\jobname{#1}
  \def\childdocjob{#1}
  \input{#1}
}
%    \end{macrocode}

% \macro{\childdocby}
% The command |\childdocby| ....
%    \begin{macrocode}
\newcommand{\childdocby}[2][]
{
  \childdocdisable
  \childdoctrue
  \childdocmanualtrue
  \if?#1?\else
    \def\jobname{#2}
  \fi
  \def\childdocjob{#2}
  \input{#2}
  \endinput
}
%    \end{macrocode}

% \macro{\childdocforward}
% The command |\childdocforward| redirects
% compilation to the main file or
% (if the optional argument is given) a child file.
% Parameters are set as if the main file
% or a child file starting with |\childdocof| was compiled.
% Then compilation is handed over to the main file:
%    \begin{macrocode}
\newcommand{\childdocforward}[2][]
{
  \begingroup
    \if?#1?
      \def\childdoctmp
      {
        \def\childdocname{#2}
        \def\childdocjob{#2}
        \def\jobname{#2}
        \input{#2}
        \endinput
      }
    \else
      \def\childdoctmp
      {
        \childdocdisable
        \def\childdocname{#2}
        \childdoctrue
        \includeonly{#2}
        \def\childdocjob{#1}
        \def\jobname{#1}
        \input{#1}
        \endinput
      }
    \fi
    \expandafter
  \endgroup
  \childdoctmp
}
%    \end{macrocode}

% \macro{\childdocforwardprefix}
% The command |\childdocforwardprefix| redirects
% compilation to the main or a child file by means of a pattern.
% The prefix |#1| in the current filename is replaced by |#2|
% and the suffix of the current filename is kept
% (it is assumed that the filename does not contain the substring `|~~~|'
% which is used as a delimiter).
% Compilation is handed over to the new file by |\childdocforward|:
%    \begin{macrocode}
\newcommand{\childdocforwardprefix}[3][]
{
  \begingroup
    \def\childdocextract #2##1~~~{\def\childdoctmp{\childdocforward[#1]{#3##1}}}
    \expandafter\childdocextract\childdocname~~~
    \expandafter
  \endgroup
  \childdoctmp
}
%    \end{macrocode}

% \macro{\childdoc}
% The deprecated macro |\childdoc| is a legacy version of |\childdocmain|:
%    \begin{macrocode}
\newcommand{\childdoc}{\childdocmain}
%    \end{macrocode}

% \macro{\childdocredirect}
% The deprecated macro |\childdocredirect| is a legacy version
% of |\childdocforward| and |\childdocforwardprefix|:
%    \begin{macrocode}
\newcommand{\childdocredirect}[2][]
{
  \begingroup
    \if?#1?
      \def\childdoctmp{\childdocforward{#2}}
    \else
      \def\childdoctmp{\childdocforwardprefix{#1}{#2}}
    \fi
    \expandafter
  \endgroup
  \childdoctmp
}
%    \end{macrocode}

%\iffalse
%</package>
%\fi
%
\endinput
\childdocforward{cdocsamp}"|\\
% |latex -jobname cdocscl1 \|\\
% |  "% \iffalse
%
% childdoc.dtx Copyright (C) 2017-2018 Niklas Beisert
%
% This work may be distributed and/or modified under the
% conditions of the LaTeX Project Public License, either version 1.3
% of this license or (at your option) any later version.
% The latest version of this license is in
%   http://www.latex-project.org/lppl.txt
% and version 1.3 or later is part of all distributions of LaTeX
% version 2005/12/01 or later.
%
% This work has the LPPL maintenance status `maintained'.
%
% The Current Maintainer of this work is Niklas Beisert.
%
% This work consists of the files childdoc.dtx and childdoc.ins
% and the derived files childdoc.def and cdocsamp.tex with
% cdocsch1.tex, cdocsch2.tex, cdocsdrf.tex, cdocsfn1.tex, cdocsfn2.tex.
%
%<package>\ifdefined\childdocmain\endinput\fi
%<package>\ProvidesFile{childdoc.def}[2018/12/30 v2.0 child document driver]
%<samplemain>\ProvidesFile{cdocsamp.tex}[2018/12/30 v2.0 sample for childdoc]
%<*driver>
%\ProvidesFile{childdoc.drv}[2018/12/30 v2.0 childdoc reference manual file]
\PassOptionsToClass{10pt,a4paper}{article}
\documentclass{ltxdoc}

\usepackage[margin=35mm]{geometry}
\usepackage{hyperref}
\usepackage{hyperxmp}
\usepackage[usenames]{color}

\hypersetup{colorlinks=true}
\hypersetup{pdfstartview=FitH}
\hypersetup{pdfpagemode=UseNone}
\hypersetup{pdfsource={}}
\hypersetup{pdflang={en-UK}}
\hypersetup{pdfcopyright={Copyright 2017-2018 Niklas Beisert.
  This work may be distributed and/or modified under the
  conditions of the LaTeX Project Public License, either version 1.3
  of this license or (at your option) any later version.}}
\hypersetup{pdflicenseurl={http://www.latex-project.org/lppl.txt}}
\hypersetup{pdfcontactaddress={ETH Zurich, ITP, HIT K,
  Wolfgang-Pauli-Strasse 27}}
\hypersetup{pdfcontactpostcode={8093}}
\hypersetup{pdfcontactcity={Zurich}}
\hypersetup{pdfcontactcountry={Switzerland}}
\hypersetup{pdfcontactemail={nbeisert@itp.phys.ethz.ch}}
\hypersetup{pdfcontacturl={http://people.phys.ethz.ch/\xmptilde nbeisert/}}

\newcommand{\secref}[1]{\hyperref[#1]{section \ref*{#1}}}

\parskip1ex
\parindent0pt
\let\olditemize\itemize
\def\itemize{\olditemize\parskip0pt}

\begin{document}

\title{The \textsf{childdoc} Package}
\hypersetup{pdftitle={The childdoc Package}}
\author{Niklas Beisert\\[2ex]
  Institut f\"ur Theoretische Physik\\
  Eidgen\"ossische Technische Hochschule Z\"urich\\
  Wolfgang-Pauli-Strasse 27, 8093 Z\"urich, Switzerland\\[1ex]
  \href{mailto:nbeisert@itp.phys.ethz.ch}
  {\texttt{nbeisert@itp.phys.ethz.ch}}}
\hypersetup{pdfauthor={Niklas Beisert}}
\hypersetup{pdfsubject={Manual for the LaTeX2e Package childdoc}}
\date{30 December 2018, \textsf{v2.0}}
\maketitle

\begin{abstract}\noindent
\textsf{childdoc} is a \LaTeXe{} package
that enables the direct compilation
of document sections included by |\include|
to individual files.
\end{abstract}

\begingroup
\parskip0ex
\tableofcontents
\endgroup

%%%%%%%%%%%%%%%%%%%%%%%%%%%%%%%%%%%%%%%%%%%%%%%%%%%%%%%%%%%%%%%%%%%%%%%%%%%%%%%%
%%%%%%%%%%%%%%%%%%%%%%%%%%%%%%%%%%%%%%%%%%%%%%%%%%%%%%%%%%%%%%%%%%%%%%%%%%%%%%%%
\section{Introduction}

\LaTeX{} provides a mechanism to structure a large document (such as a book)
into a main file and several child files (containing the chapters)
using the |\include| command.
This mechanism is beneficial for documents
which span hundreds of pages in order to
make the source file(s) more manageable.
Moreover, compilation can be restricted to
selected child files by means of the |\includeonly| command.
The latter feature can be used to reduce the compilation time while editing
(this was significantly more useful in the earlier days of \LaTeX{})
or to generate a smaller document which is easier to navigate.
Another application of |\includeonly| is to generate
documents consisting of selected parts of the complete document.

However, there are a few drawbacks of the plain |\include| mechanism:
\begin{itemize}
\item
The child files cannot be compiled on their own,
they can only be compiled via the main file.
A naive editing environment
(such as a text editor with an option
to have the current file processed by \LaTeX)
may require one to switch to the main file before compiling;
attempting to compile the child file produces errors.
\item
The main file must be modified (each time)
to adjust the |\includeonly| command
to the present needs. This easily leaves the main file in a messy state.
\item
The generated document will always carry the filename
of the main document. This is inconvenient if
several child files are to be compiled and
to be kept for distribution.
\end{itemize}

The present package provides a simple interface
to make child files individually compilable by \LaTeX{}.
Compiling a child file then has the same effect as compiling
the main file with an |\includeonly| command
to select the appropriate child.
Moreover the generated document will carry the name of the child
rather than the main file.
This resolves all three above issues.

This feature is meant to make the editing of books,
thesis documents and lecture notes somewhat more convenient.
However, the package can also be used efficiently for
composing a series of documents (such as exercise sheets)
which are typically distributed individually.
It then assists the author in generating the individual documents
(potentially in different versions)
as well as a document containing the collected series.
Another application is in developing style files
or other kinds of included material
where compilation of the style file could redirect
to a sample or test file.

%%%%%%%%%%%%%%%%%%%%%%%%%%%%%%%%%%%%%%%%%%%%%%%%%%%%%%%%%%%%%%%%%%%%%%%%%%%%%%%%
%%%%%%%%%%%%%%%%%%%%%%%%%%%%%%%%%%%%%%%%%%%%%%%%%%%%%%%%%%%%%%%%%%%%%%%%%%%%%%%%
\section{Usage}

First of all, the package \textsf{childdoc} is \emph{not} a standard
\LaTeXe{} |.sty| style file! Therefore it needs to be invoked in
a non-standard way.

%%%%%%%%%%%%%%%%%%%%%%%%%%%%%%%%%%%%%%%%%%%%%%%%%%%%%%%%%%%%%%%%%%%%%%%%%%%%%%%%
\subsection{Included Files}
\label{sec:include}

%%%%%%%%%%%%%%%%%%%%%%%%%%%%%%%%%%%%%%%%
\DescribeMacro{\childdocmain}
To use the package, add the commands
\begin{center}
\begin{tabular}{l}
|\input{childdoc.def}|\\
|\childdocmain{}|\\
\end{tabular}
\end{center}
at the very top of the main \LaTeX{} file,
in particular \emph{before} the |\documentclass| statement!
The argument of |\childdocmain| should be left empty
(but it must be present).

%%%%%%%%%%%%%%%%%%%%%%%%%%%%%%%%%%%%%%%%
\DescribeMacro{\childdocof}
Furthermore, add the commands
\begin{center}
\begin{tabular}{l}
|\input{childdoc.def}|\\
|\childdocof{|\textit{main}|}|\\
\end{tabular}
\end{center}
at the top of every child file \textit{child}
which is included by |\include{|\textit{child}|}|
from within the main file
(or at least for those files to be compiled individually).
The argument \textit{main} must be the filename of the main file.

There are a couple of
considerations in setting up the main and child documents:

%%%%%%%%%%%%%%%%%%%%%%%%%%%%%%%%%%%%%%%%
\paragraph{Restrictions.}

Please note the following restrictions:
\begin{itemize}
\item
|\childdocmain| must be called with one argument \textit{main}
to ensure compatibility with earlier version of the package.
It must either be empty (|\childdocmain{}|)
or precisely match the filename of the main file in which it is specified.
See \secref{sec:detection} for further information.
\item
The filename \textit{main} must be specified without the |.tex| extension.
\item
The filename \textit{main} is case sensitive
(even in case-insensitive file systems)
due to internal string comparison.
\item
The argument \textit{main} should be fully expanded, it cannot be a macro.
\item
Subdirectories and special characters should be avoided in filenames.
\item
The command |\childdocmain{|\textit{main}|}| must be followed by a whitespace.
It should not be followed immediately by another command
or by a comment mark `|%|'.
This is because the \TeX{} parser reads the token immediately following
the argument of |\childdocmain| and puts it
at the beginning of every child section;
however, a white\-space is ignored.
\end{itemize}

%%%%%%%%%%%%%%%%%%%%%%%%%%%%%%%%%%%%%%%%
\paragraph{Content of Main File.}

It is advisable to place all content in the child files included by |\include|.
Any output contained in the main file will appear in all child documents
unless suppressed manually;
it cannot be suppressed automatically by the |\includeonly| directive
and thus should normally be avoided.
A method to include some content in the main file
by means of conditional processing is described in \secref{sec:conditional}.

%%%%%%%%%%%%%%%%%%%%%%%%%%%%%%%%%%%%%%%%
\paragraph{Page Numbering.}

When only a part of the document is compiled,
the appropriate numbering of pages
(as well as other status parameters)
is determined from the |.aux| files.
The latter contain information from previous passes.
However this information needs to propagate through
all intermediate child documents.
Therefore the page numbering in child documents may well
be inconsistent until the complete document is compiled at least once.

A useful (if unconventional) way to always ensure a consistent
page numbering is to restart the numbering in each child document
and denote the pages by `\textit{child}|.|\textit{page}'
where \textit{child} represents the chapter/section number of the child file.
This can be achieved by the command
|\numberwithin{page}{|\textit{child}|}|
of the \textsf{amsmath} package
where \textit{child} can be |chapter| or |section|
depending on the chosen structuring.
Alternatively, one can modify the macro |\thepage| appropriately
and reset the counter |page| at the start of each child file.

%%%%%%%%%%%%%%%%%%%%%%%%%%%%%%%%%%%%%%%%%%%%%%%%%%%%%%%%%%%%%%%%%%%%%%%%%%%%%%%%
\subsection{Conditional Processing}
\label{sec:conditional}

The package provides a mechanism to compile different versions
of a document. To customise the versions further some conditional processing
can come in handy to distinguish which version is being compiled.
The package provides two macros to describe the compilation context:

%%%%%%%%%%%%%%%%%%%%%%%%%%%%%%%%%%%%%%%%
\DescribeMacro{\ifchilddoc}
The conditional |\ifchilddoc| distinguishes between the compilation of
child documents and the main document:
%
\begin{center}
|\ifchilddoc |\textit{child-code}| |[|\||else |\textit{main-code}]| \||fi|
\end{center}

%%%%%%%%%%%%%%%%%%%%%%%%%%%%%%%%%%%%%%%%
\DescribeMacro{\childdocname}
\DescribeMacro{\childdocjob}
The macro |\childdocname| contains the filename (without extension)
of the main or child file being processed.
Note that |\childdocjob| will always contain the name of the main file.

%%%%%%%%%%%%%%%%%%%%%%%%%%%%%%%%%%%%%%%%
\paragraph{Title Page.}

Conditional processing can be used to include a title or banner page
in the main document when proper precautions are taken.
Importantly, the code in the main file should ensure that the page counter
(as well as other status parameters which are stored in the |.aux| files)
takes the same value after the conditional processing.
Otherwise the page numbers may take divergent values
depending on which part is compiled.

For example, a title page could be declared by:
%
\begin{center}
\begin{tabular}{l}
|\ifchilddoc\||else|\\
|\addtocounter{page}{-1}|\\
\textit{code for title page}\\
|\newpage|\\
|\||fi|
\end{tabular}
\end{center}
%
A banner page for the child documents can be generated by:
%
\begin{center}
\begin{tabular}{l}
|\ifchilddoc|\\
|\addtocounter{page}{-1}|\\
\textit{code for banner page}\\
|\newpage|\\
|\||fi|
\end{tabular}
\end{center}
%
Here one could write a message such as:
\begin{center}
|This is the part \childdocname{} of \childdocjob{}.|
\end{center}

%%%%%%%%%%%%%%%%%%%%%%%%%%%%%%%%%%%%%%%%%%%%%%%%%%%%%%%%%%%%%%%%%%%%%%%%%%%%%%%%
\subsection{Flags}
\label{sec:flags}

The package makes it easy to generate different versions
of the main or child documents.
To this end compilation flags can be defined
and assigned different default values.
They will be particularly useful in conjunction
with the forwarding mechanism described in \secref{sec:forward}.

For example, it may be useful to have a flag |\version|
which can be set to |draft| or |final|.
The document source will contain some conditional code
depending on the value of |\version|.
Suppose further, the flag should default to |final| for the main file
and to |draft| for child files
which is a natural assignment for editing the document.
This is achieved by placing the following code
in the preamble of the main document
(below the |\childdocmain| directive):
%
\begin{center}
\begin{tabular}{l}
|\ifchilddoc|\\
|\providecommand{\version}{draft}|\\
|\||else|\\
|\providecommand{\version}{final}|\\
|\||fi|
\end{tabular}
\end{center}
%
The definition by |\providecommand| makes sure
that previous definitions are not overwritten.
Further statements |\providecommand{\version}{...}|
can thus be added before the above code to override it.

For the main file, one might add a line
(between |\childdocmain| and the above block)
%
\begin{center}
|%\ifchilddoc\||else\providecommand{\version}{draft}\||fi|
\end{center}
%
which can be uncommented to produce a draft version.
Likewise one can add a line to the very top of a child file
(above the |\childdocof{|\textit{main}|}| directive)
%
\begin{center}
|%\providecommand{\version}{final}|
\end{center}
%
which can be uncommented to produce the final version of this child document.

%%%%%%%%%%%%%%%%%%%%%%%%%%%%%%%%%%%%%%%%%%%%%%%%%%%%%%%%%%%%%%%%%%%%%%%%%%%%%%%%
\subsection{Forwarding}
\label{sec:forward}

Different versions of the main or child documents
using compilation flags as described in \secref{sec:flags}
can be (permanently) stored in different files
for convenient compilation, viewing and distribution.
To this end, the package defines a command
to pass on compilation to a different file:

%%%%%%%%%%%%%%%%%%%%%%%%%%%%%%%%%%%%%%%%
\DescribeMacro{\childdocforward}
The command |\childdocforward| redirects processing to
another source file:
%
\begin{center}
\begin{tabular}{l}
|\input{childdoc.def}|\\
|\childdocforward[|\textit{main}|]{|\textit{dest}|}|\\
\end{tabular}
\end{center}
%
The argument \textit{dest} is the destination file
(without extension).
It should be the main file or one of the child files.
Note that further \textsf{childdoc} directives
such as |\childdocof| and |\childdocforward|
in the indicated file will be processed in this form.
The optional argument \textit{main}
passes on directly to the main file \textit{main}
while pretending to compile the child \textit{dest}.
This form behaves as if \textit{dest}
issues |\childdocof{|\textit{main}|}| right away,
and no further \textsf{childdoc} directives will be processed.

%%%%%%%%%%%%%%%%%%%%%%%%%%%%%%%%%%%%%%%%
\DescribeMacro{\...prefix}
In the alternative form |\childdocforwardprefix|,
%
\begin{center}
\begin{tabular}{l}
|\input{childdoc.def}|\\
|\childdocforwardprefix[|\textit{main}|]{|\textit{prefix}|}{|\textit{dest}|}|
\end{tabular}
\end{center}
%
the destination file is determined by a pattern
depending on the current file:
To make this work, the current file must be called
`{\textit{prefix}\hspace{0.2em}\textit{suffix}}'
with \textit{prefix} matching precisely the argument.
Processing is then passed on to the file
`{\textit{dest}\hspace{0.2em}\textit{suffix}}'.
Surely, the same effect is achieved by
directly specifying the
argument `{\textit{dest}\hspace{0.2em}\textit{suffix}}'
in the first form.
However, that requires to set up a different file
for each child. With the alternative form of the command
all these files can have exactly the same content
which simplifies setting them up and maintaining them.

For example, the following file |draft.tex|
with a compilation flag |\version| as described in \secref{sec:flags}
compiles the main document as a draft:
%
\begin{center}
\begin{tabular}{l}
|\def\version{draft}|\\
|\input{childdoc.def}|\\
|\childdocforward{|\textit{main}|}|
\end{tabular}
\end{center}
%
Likewise, the following files |final|\textit{nn}|.tex|
compile the final version of the child document
|child|\textit{nn}|.tex|:
%
\begin{center}
\begin{tabular}{l}
|\def\version{final}|\\
|\input{childdoc.def}|\\
|\childdocforwardprefix{final}{child}|
\end{tabular}
\end{center}
%

Note that when several versions of a main file and/or of each child file
are to be generated, it may be convenient to set up a |Makefile| or
shell script to automatise the process.

%%%%%%%%%%%%%%%%%%%%%%%%%%%%%%%%%%%%%%%%%%%%%%%%%%%%%%%%%%%%%%%%%%%%%%%%%%%%%%%%
\subsection{Command Line Processing}
\label{sec:commandline}

The effect of redirection files can also be achieved by invoking
the \LaTeX{} compiler with a more elaborate command line.
Most conveniently this should be done as part
of a shell script or a |Makefile|.

When using \textsf{childdoc} in the main file, the following
command lines effectively perform a redirection
(note that depending on the shell being used,
backslashes may have to be doubled: `|\|' $\to$ `|\\|'):
%
\begin{center}
|... -jobname "|\textit{target}|" |\\|"|[\textit{flags}]%
|\input{childdoc.def}\childdocforward[|\textit{main}|]{|\textit{dest}|}"|
\end{center}
%
Here \textit{target} is the name of the output file,
\textit{main} is the name of the main file
and \textit{dest} is the name of the main or child file to be processed
(all filenames without extensions).
The optional argument \textit{main} can be omitted
if \textit{main} matches \textit{dest}.
Optionally, compilation \textit{flags} can be defined via |\def| commands.
This command line makes the \TeX{} engine believe
it is compiling the file \textit{target}
whose content is specified as the latter parameter.
The provided code then forwards the processing to
\textit{main} or \textit{dest} as described in \secref{sec:forward}.

%%%%%%%%%%%%%%%%%%%%%%%%%%%%%%%%%%%%%%%%%%%%%%%%%%%%%%%%%%%%%%%%%%%%%%%%%%%%%%%%
\subsection{Include by Input}
\label{sec:input}

Including child documents by |\include| has some restrictions by design.
Most notably, the content of a child document always occupies
its own set of pages; pages cannot be shared between child documents.
Usually, this behaviour makes perfect sense
because each child document contain an essential part of the document.
However, in some situations it may be desirable to compose
a document from a collection of parts
without having mandatory page breaks between then.
For this case, the package
provides a mechanism to include parts
by |\input| which can also be processed individually.
However, by construction this mechanism
requires manual handling of the content to be output.

%%%%%%%%%%%%%%%%%%%%%%%%%%%%%%%%%%%%%%%%
\DescribeMacro{\ifchilddocmanual}
The main file should be prepared as usual, see \secref{sec:include}.
However, the document body must make a distinction
between processing of an individual part and of the main document, e.g.:
%
\begin{center}
\begin{tabular}{l}
|\ifchilddocmanual|\\
|\input{\childdocname}|\\
|\||else|\\
\textit{document body with }|\input{|\textit{part}|}|\\
|\||fi|
\end{tabular}
\end{center}
%
The conditional |\ifchilddocmanual| is true whenever
a part to be included by |\input| is being compiled,
and the name of the part is stored in |\childdocname|.

%%%%%%%%%%%%%%%%%%%%%%%%%%%%%%%%%%%%%%%%
\DescribeMacro{\childdocby}
Each part to be included by |\input| should start with:
%
\begin{center}
\begin{tabular}{l}
|\input{childdoc.def}|\\
|\childdocby{|\textit{main}|}|\\
\end{tabular}
\end{center}
%
The directive |\childdocby| is similar to |\childdocof|
described in \secref{sec:include},
but the subsequent selection of content must be done manually.
To that end, both |\ifchilddoc| and |\ifchilddocmanual|
will be true upon processing of a part,
and the name of the part is stored in |\childdocname|.
Note that |\jobname| will be set to the filename of the current part
so that each part receives an individual |.aux| file
that does not interfere with the |.aux| file(s) of the main document.
This behaviour can be altered by the alternative form
|\childdocby[*]{|\textit{main}|}| (with a non-empty optional argument)
which uses the |.aux| file of the main document
by setting |\jobname| to \textit{main}.

%%%%%%%%%%%%%%%%%%%%%%%%%%%%%%%%%%%%%%%%%%%%%%%%%%%%%%%%%%%%%%%%%%%%%%%%%%%%%%%%
\subsection{Driver Development}
\label{sec:driver}

The \textsf{childdoc} mechanism can also be use for the development
of definition files such as \LaTeX{} styles or classes.
This case differs from the above setup with multiple parts
included by |\include| in that no |\includeonly| should be invoked.
This can be achieved by starting the include file
(before |\ProvidesPackage|) with:
%
\begin{center}
\begin{tabular}{l}
|\input{childdoc.def}|\\
|\childdocforward{|\textit{main}|}|\\
\end{tabular}
\end{center}
%
or alternatively with:
%
\begin{center}
\begin{tabular}{l}
|\input{childdoc.def}|\\
|\childdocby{|\textit{main}|}|\\
\end{tabular}
\end{center}
%
Both forms have slightly different effects as described above.
The main file is prepared as usual, see \secref{sec:include}.

%%%%%%%%%%%%%%%%%%%%%%%%%%%%%%%%%%%%%%%%%%%%%%%%%%%%%%%%%%%%%%%%%%%%%%%%%%%%%%%%
\subsection{Legacy Detection}
\label{sec:detection}

The directive |\childdocmain| in the main file can detect
whether the complete document or merely a child is to be compiled
even without using the directive |\childdocof|.
This method is deprecated because it is less robust
and there is no compelling reason to use it;
it is merely provided for backward compatibility
and it may be removed in future versions.

If the detection mechanism is to be used,
it is mandatory to correctly specify
the filename of the main file as the argument of |\childdocmain|:
%
\begin{center}
\begin{tabular}{l}
|\input{childdoc.def}|\\
|\childdocmain{|\textit{main}|}|\\
\end{tabular}
\end{center}
%
If |\jobname| does not match the argument \textit{main} of |\childdocmain|,
it is assumed that |\jobname| points to the child file to be compiled.
When using |\childdocmain| with the main file specified as argument,
it suffices to start a child file
with just |\input{|\textit{main}|}|
without loading of the package and using |\childdocof|.
If instead all processing is done
with the appropriate \textsf{childdoc} directives,
the argument of \textit{main} of |\childdocmain| can be empty.

An alternative version of the command line processing described
in \secref{sec:commandline} using the detection mechanism reads:
%
\begin{center}
|... -jobname "|\textit{target}|" "|[\textit{flags}]%
[|\def\jobname{|\textit{dest}|}|]|\input{|\textit{main}|}"|
\end{center}

%%%%%%%%%%%%%%%%%%%%%%%%%%%%%%%%%%%%%%%%%%%%%%%%%%%%%%%%%%%%%%%%%%%%%%%%%%%%%%%%
\subsection{Manual Code}
\label{sec:manual}

In case one cannot be certain whether the definitions file |childdoc.def|
is installed on the target \TeX{} distribution
and one prefers not to ship it,
it is conceivable to paste a few relevant commands into the sources.

To that end, drop all statements |\input{childdoc.def}|
and perform the replacements as outlined below.
Instead of |\childdocmain{|\textit{main}|}| add the following code
to the top of the main file:
%
\begin{center}
\begin{tabular}{l}
|\||ifdefined\childdocname\endinput\||fi\newif\ifchilddoc|\\
|\edef\childdocname{\scantokens\expandafter{\jobname\noexpand}}|\\
|\def\childdocmain{|\textit{main}|}\||ifx\childdocmain\childdocname\||else|\\
|\childdoctrue\includeonly{\childdocname}\let\jobname\childdocmain\||fi|\\
\end{tabular}
\end{center}
%
Instead of |\childdocof{|\textit{main}|}| just include the main file
at the top of each child file:
%
\begin{center}
|\input{|\textit{main}|}|
\end{center}
%
A simple redirection |\childdocforward{|\textit{dest}|}| is achieved by:
%
\begin{center}
|\def\jobname{|\textit{dest}|}\input{\jobname}|
\end{center}
%
The redirection with prefix
|\childdocforwardprefix[|\textit{prefix}|]{|\textit{dest}|}|
is accomplished by:
%
\begin{center}
\begin{tabular}{l}
|{\edef\jobname{\scantokens\expandafter{\jobname\noexpand}}|\\
|\def\redirectjob |\textit{prefix}|#1~~~{\gdef\jobname{|\textit{dest}|#1}}|\\
|\expandafter\redirectjob\jobname~~~}\input{\jobname}|
\end{tabular}
\end{center}

In an alternative approach,
child documents can be compiled by a specific command line
without additional code or specific definitions:
%
\begin{center}
|... -jobname "|\textit{target}|" "|[\textit{flags}]%
|\includeonly{|\textit{dest}|}\input{|\textit{main}|}"|
\end{center}
%

%%%%%%%%%%%%%%%%%%%%%%%%%%%%%%%%%%%%%%%%%%%%%%%%%%%%%%%%%%%%%%%%%%%%%%%%%%%%%%%%
%%%%%%%%%%%%%%%%%%%%%%%%%%%%%%%%%%%%%%%%%%%%%%%%%%%%%%%%%%%%%%%%%%%%%%%%%%%%%%%%
\section{Information}

%%%%%%%%%%%%%%%%%%%%%%%%%%%%%%%%%%%%%%%%%%%%%%%%%%%%%%%%%%%%%%%%%%%%%%%%%%%%%%%%
\subsection{Copyright}

Copyright \copyright{} 2017--2018 Niklas Beisert

This work may be distributed and/or modified under the
conditions of the \LaTeX{} Project Public License, either version 1.3
of this license or (at your option) any later version.
The latest version of this license is in
  \url{http://www.latex-project.org/lppl.txt}
and version 1.3 or later is part of all distributions of \LaTeX{}
version 2005/12/01 or later.

This work has the LPPL maintenance status `maintained'.

The Current Maintainer of this work is Niklas Beisert.

This work consists of the files |README.txt|, |childdoc.ins| and |childdoc.dtx|
as well as the derived files |childdoc.def|, |cdocsamp.tex|
with |cdocsch1.tex|, |cdocsch2.tex|, |cdocspt3.tex|, |cdocspt4.tex|,
|cdocsdrf.tex|, |cdocsfn1.tex|, |cdocsfn2.tex|
as well as |childdoc.pdf|.

%%%%%%%%%%%%%%%%%%%%%%%%%%%%%%%%%%%%%%%%%%%%%%%%%%%%%%%%%%%%%%%%%%%%%%%%%%%%%%%%
\subsection{Files and Installation}

The package consists of the files:
%
\begin{center}
\begin{tabular}{ll}
    |README.txt|   & readme file \\
    |childdoc.ins| & installation file \\
    |childdoc.dtx| & source file \\
    |childdoc.def| & definition file \\
    |cdocsamp.tex| & sample main file \\
    |cdocsch1.tex| & sample include file \\
    |cdocsch2.tex| & sample include file \\
    |cdocspt3.tex| & sample part file \\
    |cdocspt4.tex| & sample part file \\
    |cdocsdrf.tex| & sample redirection file \\
    |cdocsfn1.tex| & sample redirection file \\
    |cdocsfn2.tex| & sample redirection file \\
    |childdoc.pdf| & manual
\end{tabular}
\end{center}
%
The distribution consists of the files
|README.txt|, |childdoc.ins| and |childdoc.dtx|.
%
\begin{itemize}
\item
Run (pdf)\LaTeX{} on |childdoc.dtx|
to compile the manual |childdoc.pdf| (this file).
\item
Run \LaTeX{} on |childdoc.ins| to create the definitions file |childdoc.def|
and the sample |cdocsamp.tex| with include files
|cdocsch1.tex|, |cdocsch2.tex|, |cdocspt3.tex|, |cdocspt4.tex|,
|cdocsdrf.tex|, |cdocsfn1.tex|, |cdocsfn2.tex|.
Then copy the file |childdoc.def| to an appropriate directory of your \LaTeX{}
distribution, e.g.\ \textit{texmf-root}|/tex/latex/childdoc|.
\end{itemize}

%%%%%%%%%%%%%%%%%%%%%%%%%%%%%%%%%%%%%%%%%%%%%%%%%%%%%%%%%%%%%%%%%%%%%%%%%%%%%%%%
\subsection{Related CTAN Packages}

There are several other packages which offer a similar functionality:
%
\begin{itemize}
\item
The packages
\href{http://ctan.org/pkg/docmute}{\textsf{docmute}},
\href{http://ctan.org/pkg/includex}{\textsf{includex}} and
\href{http://ctan.org/pkg/standalone}{\textsf{standalone}}
provide commands to include only the document body of
a child file thus allowing both files to be compiled individually.
\item
The packages \href{http://ctan.org/pkg/subdocs}{\textsf{subdocs}}
and \href{http://ctan.org/pkg/subfiles}{\textsf{subfiles}}
provide structures in which the main and child documents can be
encapsulated and allowing them to be compiled individually.
The inclusion mechanism is different from the conventional |\include|.
\item
The package \href{http://ctan.org/pkg/combine}{\textsf{combine}}
is an elaborate solution to combine several documents into one.
\end{itemize}
%
See also the CTAN topic \href{http://ctan.org/topic/subdocs}{\textsf{subdocs}}
for further related packages.
The present package differs from the above solutions in that
a document structure constructed with the conventional |\include| mechanism
just needs two extra commands at the top of every file
such that all constituent files can be compiled individually.

%%%%%%%%%%%%%%%%%%%%%%%%%%%%%%%%%%%%%%%%%%%%%%%%%%%%%%%%%%%%%%%%%%%%%%%%%%%%%%%%
%\subsection{Feature Suggestions}
%
%The following is a list of features which may be useful for future
%versions of this package:
%%
%\begin{itemize}
%\item
%\ldots
%\end{itemize}

%%%%%%%%%%%%%%%%%%%%%%%%%%%%%%%%%%%%%%%%%%%%%%%%%%%%%%%%%%%%%%%%%%%%%%%%%%%%%%%%
\subsection{Revision History}

%%%%%%%%%%%%%%%%%%%%%%%%%%%%%%%%%%%%%%%%
\paragraph{v2.0:} 2018/12/30

\begin{itemize}
\item
immediate forward processing
\item
added |\childdocby| mechanism
\item
manual restructured
\end{itemize}

%%%%%%%%%%%%%%%%%%%%%%%%%%%%%%%%%%%%%%%%
\paragraph{v1.6:} 2018/01/17

\begin{itemize}
\item
application for development of include files
\item
corrections to manual
\end{itemize}

%%%%%%%%%%%%%%%%%%%%%%%%%%%%%%%%%%%%%%%%
\paragraph{v1.5:} 2017/05/21

\begin{itemize}
\item
more complete structuring introduced
\item
|\childdocof| introduced
\item
|\childdoc| renamed to |\childdocmain|
\item
|\childredirect| renamed to |\childdocforward| and |\childdocforwardprefix|
and functionality expanded
\end{itemize}

%%%%%%%%%%%%%%%%%%%%%%%%%%%%%%%%%%%%%%%%
\paragraph{v1.0:} 2017/04/27

\begin{itemize}
\item
manual and install package
\item
first version published on CTAN
\end{itemize}

%%%%%%%%%%%%%%%%%%%%%%%%%%%%%%%%%%%%%%%%
\paragraph{v0.6:} 2017/04/26

\begin{itemize}
\item
redirection mechanism added
\end{itemize}

%%%%%%%%%%%%%%%%%%%%%%%%%%%%%%%%%%%%%%%%
\paragraph{v0.5:} 2017/04/26

\begin{itemize}
\item
functionality in definition file
\end{itemize}


%%%%%%%%%%%%%%%%%%%%%%%%%%%%%%%%%%%%%%%%%%%%%%%%%%%%%%%%%%%%%%%%%%%%%%%%%%%%%%%%
%%%%%%%%%%%%%%%%%%%%%%%%%%%%%%%%%%%%%%%%%%%%%%%%%%%%%%%%%%%%%%%%%%%%%%%%%%%%%%%%
%%%%%%%%%%%%%%%%%%%%%%%%%%%%%%%%%%%%%%%%%%%%%%%%%%%%%%%%%%%%%%%%%%%%%%%%%%%%%%%%
\appendix

\settowidth\MacroIndent{\rmfamily\scriptsize 000\ }

 \DocInput{childdoc.dtx}

\end{document}
%</driver>
% \fi
%
% %%%%%%%%%%%%%%%%%%%%%%%%%%%%%%%%%%%%%%%%%%%%%%%%%%%%%%%%%%%%%%%%%%%%%%%%%%%%%%
% %%%%%%%%%%%%%%%%%%%%%%%%%%%%%%%%%%%%%%%%%%%%%%%%%%%%%%%%%%%%%%%%%%%%%%%%%%%%%%
% \section{Sample}
%\iffalse
%<*samplemain>
%\fi
%
% The following presents a sample document
% with two chapters, two parts, a title page,
% a compile flag as well as three forwarding files to set the flag.
% It consists of eight |.tex| files:
% \begin{center}
% \begin{tabular}{ll}
% |cdocsamp.tex|&main file\\
% |cdocsch1.tex|&include file for chapter 1\\
% |cdocsch2.tex|&include file for chapter 2\\
% |cdocspt3.tex|&include file for part 3\\
% |cdocspt4.tex|&include file for part 4\\
% |cdocsdrf.tex|&forwarding file for main file in draft mode\\
% |cdocsfi1.tex|&forwarding file for final version of chapter 1\\
% |cdocsfi2.tex|&forwarding file for final version of chapter 2\\
% \end{tabular}
% \end{center}
% Each of the eight files can be compiled directly by the \LaTeX{} compiler.
%
% %%%%%%%%%%%%%%%%%%%%%%%%%%%%%%%%%%%%%%
% \paragraph{Main File.}
%
% The main file is called |cdocsamp.tex|.
%
% Load the \textsf{childdoc} definitions and
% declare the filename for the main document:
%    \begin{macrocode}
\input{childdoc.def}
\childdocmain{}
%    \end{macrocode}

% Optional override for |\version| flag:
%    \begin{macrocode}
%%\ifchilddoc\else\providecommand{\version}{draft}\fi
%    \end{macrocode}

% Define the default values for the |\version| flag
% (|final| for the main file and |draft| for childs):
%    \begin{macrocode}
\ifchilddoc
\providecommand{\version}{draft}
\else
\providecommand{\version}{final}
\fi
%    \end{macrocode}

% Load the standard document class:
%    \begin{macrocode}
\documentclass[12pt]{article}
%    \end{macrocode}

% Start the document body:
%    \begin{macrocode}
\begin{document}
%    \end{macrocode}

% Declare a title page.
% Print title, part of document being processed and version flag:
%    \begin{macrocode}
\addtocounter{page}{-1}
\begin{center}
{\LARGE\bfseries{}childdoc example\par}
\vspace{1cm}
\ifchilddoc
\ifchilddocmanual part\else chapter\fi:
`\childdocname' of `\childdocjob'\par
\else
main document: `\childdocjob'\par
\fi
version: \version\par
\end{center}
\newpage
%    \end{macrocode}

% Manually include selected file,
% otherwise process as usual:
%    \begin{macrocode}
\ifchilddocmanual
\section*{part `\childdocname'}
\input{\childdocname}
\else
%    \end{macrocode}

% Include the two chapters:
%    \begin{macrocode}
\include{cdocsch1}
\include{cdocsch2}
%    \end{macrocode}

% Include the two parts unless only chapters should be displayed:
%    \begin{macrocode}
\ifchilddoc\else
\section{part three}
\input{cdocspt3}
\section{part four}
\input{cdocspt4}
\fi
%    \end{macrocode}

% Process as usual until here:
%    \begin{macrocode}
\fi
%    \end{macrocode}

% End of document body:
%    \begin{macrocode}
\end{document}
%    \end{macrocode}
%\iffalse
%</samplemain>
%\fi
%
% %%%%%%%%%%%%%%%%%%%%%%%%%%%%%%%%%%%%%%
% \paragraph{Chapter Include Files.}
%
% The include files are called |cdocsch1.tex| and |cdocsch2.tex|.
%
%\iffalse
%<*samplechap1|samplechap2>
%\fi

% Optional override for |\version| flag:
%    \begin{macrocode}
%%\providecommand{\version}{final}
%    \end{macrocode}

% Include the main document:
%    \begin{macrocode}
\input{childdoc.def}
\childdocof{cdocsamp}
%    \end{macrocode}

%\iffalse
%</samplechap1|samplechap2>
%\fi
%
%\iffalse
%<*samplechap1>
%\fi
% Some text for chapter 1:
%    \begin{macrocode}
\section{one}
some text in chapter one
%    \end{macrocode}

%\iffalse
%</samplechap1>
%\fi
% Some text for chapter 2:
%\iffalse
%<*samplechap2>
%\fi
%    \begin{macrocode}
\section{two}
more text in chapter two
%    \end{macrocode}

%\iffalse
%</samplechap2>
%\fi
%
% %%%%%%%%%%%%%%%%%%%%%%%%%%%%%%%%%%%%%%
% \paragraph{Part Include Files.}
%
% The include files are called |cdocspt3.tex| and |cdocspt4.tex|.
%
%\iffalse
%<*samplepart3|samplepart4>
%\fi

% Optional override for |\version| flag:
%    \begin{macrocode}
%%\providecommand{\version}{final}
%    \end{macrocode}

% Include the main document:
%    \begin{macrocode}
\input{childdoc.def}
\childdocby{cdocsamp}
%    \end{macrocode}

%\iffalse
%</samplepart3|samplepart4>
%\fi
%
%\iffalse
%<*samplepart3>
%\fi
% Some text for part 3:
%    \begin{macrocode}
some text in part three
%    \end{macrocode}

%\iffalse
%</samplepart3>
%\fi
% Some text for part 4:
%\iffalse
%<*samplepart4>
%\fi
%    \begin{macrocode}
more text in part four
%    \end{macrocode}

%\iffalse
%</samplepart4>
%\fi
%
% %%%%%%%%%%%%%%%%%%%%%%%%%%%%%%%%%%%%%%
% \paragraph{Forwarding for a Complete Draft.}
%
% The following forwarding file |cdocsdrf.tex|
% compiles the main document in draft mode:
%\iffalse
%<*sampledraft>
%\fi
%    \begin{macrocode}
\def\version{draft}
\input{childdoc.def}
\childdocforward{cdocsamp}
%    \end{macrocode}

%\iffalse
%</sampledraft>
%\fi
%
% %%%%%%%%%%%%%%%%%%%%%%%%%%%%%%%%%%%%%%
% \paragraph{Forwarding for Final Version of the Chapters.}
%
% The following forwarding files |cdocsfn1.tex| and |cdocsfn2.tex|
% (with identical content)
% compile the final versions of the child documents
% |cdocsch1.tex| and |cdocsch2.tex|, respectively:
%\iffalse
%<*samplefinal>
%\fi
%    \begin{macrocode}
\def\version{final}
\input{childdoc.def}
\childdocforwardprefix[cdocsamp]{cdocsfn}{cdocsch}
%    \end{macrocode}

%\iffalse
%</samplefinal>
%\fi
%
% %%%%%%%%%%%%%%%%%%%%%%%%%%%%%%%%%%%%%%
% \paragraph{Command Line Processing.}
%
% The following three command lines generate the output files
% |cdocscld|, |cdocscl1| and |cdocscl2|
% which should be identical to
% |cdocsdrf|, |cdocsch1| and |cdocsfn2|, respectively:
% \begin{center}
% \begin{tabular}{l}
% |latex -jobname cdocscld \|\\
% |  "\def\version{draft}\input{childdoc.def}\childdocforward{cdocsamp}"|\\
% |latex -jobname cdocscl1 \|\\
% |  "\input{childdoc.def}\childdocforward[cdocsamp]{cdocsch1}"|\\
% |latex -jobname cdocscl2 \|\\
% |  "\def\version{final}\input{childdoc.def}\childdocforward{cdocsch2}"|
% \end{tabular}
% \end{center}
% Note that the trailing backslash on each first line
% merely continues the input to the second line
% (for convenient cut ant paste).
% Furthermore, the command |latex| can be replaced by any
% of its alternative versions such as |pdflatex|.
%
% %%%%%%%%%%%%%%%%%%%%%%%%%%%%%%%%%%%%%%%%%%%%%%%%%%%%%%%%%%%%%%%%%%%%%%%%%%%%%%
% %%%%%%%%%%%%%%%%%%%%%%%%%%%%%%%%%%%%%%%%%%%%%%%%%%%%%%%%%%%%%%%%%%%%%%%%%%%%%%
% \section{Implementation}
%\iffalse
%<*package>
%\fi
%
% This section describes the definitions file |childdoc.def|.

% The definitions cannot be loaded using |\usepackage| or |\RequirePackage|
% which has a mechanism to prevent loading a style file more than once.
% When loading the definitions by means of |\input|
% multiple instances have to be prevented manually:
%\iffalse
%This code needs to be before the `\ProvidesFile' directive
%which is defined at the beginning of this file.
%Therefore it is also placed there and commented out here.
%</package>
%<*discard>
%\fi
%    \begin{macrocode}
\ifdefined\childdocmain\endinput\fi
%    \end{macrocode}
%\iffalse
%</discard>
%<*package>
%\fi
%
% \macro{\ifchilddoc}
% \macro{\ifchilddocmanual}
% The conditional |\ifchilddoc| tells whether a
% child (true) or main (false) document is being compiled.
% The conditional |\ifchilddocmanual| tells whether
% the |\includeonly| mechanism is used (false) or
% the selection of child files must be performed manually (true).
% The definitions initialise to false:
%    \begin{macrocode}
\newif\ifchilddoc
\newif\ifchilddocmanual
%    \end{macrocode}

% \macro{\childdocname}
% \macro{\childdocjob}
% The macro |\childdocname| stores the name of the main document
% to be compiled. The macro |\childdocjob| stores the name of
% the document on which the \LaTeX{} compiler was originally invoked.
% The content of |\jobname| cannot be compared
% to filenames specified in the source due to different catcodes.
% The following code rescans |\jobname|, stores the result
% in |\childdocname| and saves a copy in |\childdocjob|:
%    \begin{macrocode}
\edef\childdocname{\scantokens\expandafter{\jobname\noexpand}}
\let\childdocjob\childdocname
%    \end{macrocode}

% \macro{\childdocdisable}
% The macro |\childdocdisable| prevents the main file
% from being processed more than once.
% At this stage, the main document command |\childdocmain|
% is assumed to be called once again where it should do nothing.
% Any subsequent call to it should prevent
% a secondary processing of the main document
% It overwrites the forwarding commands
% |\childdocof| and |\childdocforward|
% with empty macros to prevent further inclusions of the main document:
%    \begin{macrocode}
\newcommand{\childdocdisable}
{
  \renewcommand{\childdocmain}[1]{\renewcommand{\childdocmain}[1]{\endinput}}
  \renewcommand{\childdocof}[1]{}
  \renewcommand{\childdocby}[2][]{}
  \renewcommand{\childdocforward}[2][]{}
  \renewcommand{\childdocdisable}{}
}
%    \end{macrocode}

% \macro{\childdocmain}
% The macro |\childdocmain| is to be called at the top of the main file
% with nothing or the main filename (without extension) as argument.
% First, it breaks loops.
% If the argument is not empty and does not match |\childdocname|
% (which is set by the first inclusion of |childdoc.def|),
% |\ifchilddoc| is set to true, |\includeonly| is applied to the child file
% and |\jobname| is set to the main file
% (for proper handling of |.aux| files):
%    \begin{macrocode}
\newcommand{\childdocmain}[1]
{
  \childdocdisable\childdocmain{}
  \if?#1?\else
    \begingroup
      \def\childdoctmp{#1}
      \ifx\childdoctmp\childdocname
        \def\childdoctmp{}
      \else
        \def\childdoctmp
        {
          \childdoctrue
          \includeonly{\childdocname}
          \def\childdocjob{#1}
          \def\jobname{#1}
        }
      \fi
      \expandafter
    \endgroup
    \childdoctmp
  \fi
}
%    \end{macrocode}

% \macro{\childdocof}
% The command |\childdocof| redirects
% compilation to the main file |#1|.
%    \begin{macrocode}
\newcommand{\childdocof}[1]
{
  \childdocdisable
  \childdoctrue
  \includeonly{\childdocname}
  \def\jobname{#1}
  \def\childdocjob{#1}
  \input{#1}
}
%    \end{macrocode}

% \macro{\childdocby}
% The command |\childdocby| ....
%    \begin{macrocode}
\newcommand{\childdocby}[2][]
{
  \childdocdisable
  \childdoctrue
  \childdocmanualtrue
  \if?#1?\else
    \def\jobname{#2}
  \fi
  \def\childdocjob{#2}
  \input{#2}
  \endinput
}
%    \end{macrocode}

% \macro{\childdocforward}
% The command |\childdocforward| redirects
% compilation to the main file or
% (if the optional argument is given) a child file.
% Parameters are set as if the main file
% or a child file starting with |\childdocof| was compiled.
% Then compilation is handed over to the main file:
%    \begin{macrocode}
\newcommand{\childdocforward}[2][]
{
  \begingroup
    \if?#1?
      \def\childdoctmp
      {
        \def\childdocname{#2}
        \def\childdocjob{#2}
        \def\jobname{#2}
        \input{#2}
        \endinput
      }
    \else
      \def\childdoctmp
      {
        \childdocdisable
        \def\childdocname{#2}
        \childdoctrue
        \includeonly{#2}
        \def\childdocjob{#1}
        \def\jobname{#1}
        \input{#1}
        \endinput
      }
    \fi
    \expandafter
  \endgroup
  \childdoctmp
}
%    \end{macrocode}

% \macro{\childdocforwardprefix}
% The command |\childdocforwardprefix| redirects
% compilation to the main or a child file by means of a pattern.
% The prefix |#1| in the current filename is replaced by |#2|
% and the suffix of the current filename is kept
% (it is assumed that the filename does not contain the substring `|~~~|'
% which is used as a delimiter).
% Compilation is handed over to the new file by |\childdocforward|:
%    \begin{macrocode}
\newcommand{\childdocforwardprefix}[3][]
{
  \begingroup
    \def\childdocextract #2##1~~~{\def\childdoctmp{\childdocforward[#1]{#3##1}}}
    \expandafter\childdocextract\childdocname~~~
    \expandafter
  \endgroup
  \childdoctmp
}
%    \end{macrocode}

% \macro{\childdoc}
% The deprecated macro |\childdoc| is a legacy version of |\childdocmain|:
%    \begin{macrocode}
\newcommand{\childdoc}{\childdocmain}
%    \end{macrocode}

% \macro{\childdocredirect}
% The deprecated macro |\childdocredirect| is a legacy version
% of |\childdocforward| and |\childdocforwardprefix|:
%    \begin{macrocode}
\newcommand{\childdocredirect}[2][]
{
  \begingroup
    \if?#1?
      \def\childdoctmp{\childdocforward{#2}}
    \else
      \def\childdoctmp{\childdocforwardprefix{#1}{#2}}
    \fi
    \expandafter
  \endgroup
  \childdoctmp
}
%    \end{macrocode}

%\iffalse
%</package>
%\fi
%
\endinput
\childdocforward[cdocsamp]{cdocsch1}"|\\
% |latex -jobname cdocscl2 \|\\
% |  "\def\version{final}% \iffalse
%
% childdoc.dtx Copyright (C) 2017-2018 Niklas Beisert
%
% This work may be distributed and/or modified under the
% conditions of the LaTeX Project Public License, either version 1.3
% of this license or (at your option) any later version.
% The latest version of this license is in
%   http://www.latex-project.org/lppl.txt
% and version 1.3 or later is part of all distributions of LaTeX
% version 2005/12/01 or later.
%
% This work has the LPPL maintenance status `maintained'.
%
% The Current Maintainer of this work is Niklas Beisert.
%
% This work consists of the files childdoc.dtx and childdoc.ins
% and the derived files childdoc.def and cdocsamp.tex with
% cdocsch1.tex, cdocsch2.tex, cdocsdrf.tex, cdocsfn1.tex, cdocsfn2.tex.
%
%<package>\ifdefined\childdocmain\endinput\fi
%<package>\ProvidesFile{childdoc.def}[2018/12/30 v2.0 child document driver]
%<samplemain>\ProvidesFile{cdocsamp.tex}[2018/12/30 v2.0 sample for childdoc]
%<*driver>
%\ProvidesFile{childdoc.drv}[2018/12/30 v2.0 childdoc reference manual file]
\PassOptionsToClass{10pt,a4paper}{article}
\documentclass{ltxdoc}

\usepackage[margin=35mm]{geometry}
\usepackage{hyperref}
\usepackage{hyperxmp}
\usepackage[usenames]{color}

\hypersetup{colorlinks=true}
\hypersetup{pdfstartview=FitH}
\hypersetup{pdfpagemode=UseNone}
\hypersetup{pdfsource={}}
\hypersetup{pdflang={en-UK}}
\hypersetup{pdfcopyright={Copyright 2017-2018 Niklas Beisert.
  This work may be distributed and/or modified under the
  conditions of the LaTeX Project Public License, either version 1.3
  of this license or (at your option) any later version.}}
\hypersetup{pdflicenseurl={http://www.latex-project.org/lppl.txt}}
\hypersetup{pdfcontactaddress={ETH Zurich, ITP, HIT K,
  Wolfgang-Pauli-Strasse 27}}
\hypersetup{pdfcontactpostcode={8093}}
\hypersetup{pdfcontactcity={Zurich}}
\hypersetup{pdfcontactcountry={Switzerland}}
\hypersetup{pdfcontactemail={nbeisert@itp.phys.ethz.ch}}
\hypersetup{pdfcontacturl={http://people.phys.ethz.ch/\xmptilde nbeisert/}}

\newcommand{\secref}[1]{\hyperref[#1]{section \ref*{#1}}}

\parskip1ex
\parindent0pt
\let\olditemize\itemize
\def\itemize{\olditemize\parskip0pt}

\begin{document}

\title{The \textsf{childdoc} Package}
\hypersetup{pdftitle={The childdoc Package}}
\author{Niklas Beisert\\[2ex]
  Institut f\"ur Theoretische Physik\\
  Eidgen\"ossische Technische Hochschule Z\"urich\\
  Wolfgang-Pauli-Strasse 27, 8093 Z\"urich, Switzerland\\[1ex]
  \href{mailto:nbeisert@itp.phys.ethz.ch}
  {\texttt{nbeisert@itp.phys.ethz.ch}}}
\hypersetup{pdfauthor={Niklas Beisert}}
\hypersetup{pdfsubject={Manual for the LaTeX2e Package childdoc}}
\date{30 December 2018, \textsf{v2.0}}
\maketitle

\begin{abstract}\noindent
\textsf{childdoc} is a \LaTeXe{} package
that enables the direct compilation
of document sections included by |\include|
to individual files.
\end{abstract}

\begingroup
\parskip0ex
\tableofcontents
\endgroup

%%%%%%%%%%%%%%%%%%%%%%%%%%%%%%%%%%%%%%%%%%%%%%%%%%%%%%%%%%%%%%%%%%%%%%%%%%%%%%%%
%%%%%%%%%%%%%%%%%%%%%%%%%%%%%%%%%%%%%%%%%%%%%%%%%%%%%%%%%%%%%%%%%%%%%%%%%%%%%%%%
\section{Introduction}

\LaTeX{} provides a mechanism to structure a large document (such as a book)
into a main file and several child files (containing the chapters)
using the |\include| command.
This mechanism is beneficial for documents
which span hundreds of pages in order to
make the source file(s) more manageable.
Moreover, compilation can be restricted to
selected child files by means of the |\includeonly| command.
The latter feature can be used to reduce the compilation time while editing
(this was significantly more useful in the earlier days of \LaTeX{})
or to generate a smaller document which is easier to navigate.
Another application of |\includeonly| is to generate
documents consisting of selected parts of the complete document.

However, there are a few drawbacks of the plain |\include| mechanism:
\begin{itemize}
\item
The child files cannot be compiled on their own,
they can only be compiled via the main file.
A naive editing environment
(such as a text editor with an option
to have the current file processed by \LaTeX)
may require one to switch to the main file before compiling;
attempting to compile the child file produces errors.
\item
The main file must be modified (each time)
to adjust the |\includeonly| command
to the present needs. This easily leaves the main file in a messy state.
\item
The generated document will always carry the filename
of the main document. This is inconvenient if
several child files are to be compiled and
to be kept for distribution.
\end{itemize}

The present package provides a simple interface
to make child files individually compilable by \LaTeX{}.
Compiling a child file then has the same effect as compiling
the main file with an |\includeonly| command
to select the appropriate child.
Moreover the generated document will carry the name of the child
rather than the main file.
This resolves all three above issues.

This feature is meant to make the editing of books,
thesis documents and lecture notes somewhat more convenient.
However, the package can also be used efficiently for
composing a series of documents (such as exercise sheets)
which are typically distributed individually.
It then assists the author in generating the individual documents
(potentially in different versions)
as well as a document containing the collected series.
Another application is in developing style files
or other kinds of included material
where compilation of the style file could redirect
to a sample or test file.

%%%%%%%%%%%%%%%%%%%%%%%%%%%%%%%%%%%%%%%%%%%%%%%%%%%%%%%%%%%%%%%%%%%%%%%%%%%%%%%%
%%%%%%%%%%%%%%%%%%%%%%%%%%%%%%%%%%%%%%%%%%%%%%%%%%%%%%%%%%%%%%%%%%%%%%%%%%%%%%%%
\section{Usage}

First of all, the package \textsf{childdoc} is \emph{not} a standard
\LaTeXe{} |.sty| style file! Therefore it needs to be invoked in
a non-standard way.

%%%%%%%%%%%%%%%%%%%%%%%%%%%%%%%%%%%%%%%%%%%%%%%%%%%%%%%%%%%%%%%%%%%%%%%%%%%%%%%%
\subsection{Included Files}
\label{sec:include}

%%%%%%%%%%%%%%%%%%%%%%%%%%%%%%%%%%%%%%%%
\DescribeMacro{\childdocmain}
To use the package, add the commands
\begin{center}
\begin{tabular}{l}
|\input{childdoc.def}|\\
|\childdocmain{}|\\
\end{tabular}
\end{center}
at the very top of the main \LaTeX{} file,
in particular \emph{before} the |\documentclass| statement!
The argument of |\childdocmain| should be left empty
(but it must be present).

%%%%%%%%%%%%%%%%%%%%%%%%%%%%%%%%%%%%%%%%
\DescribeMacro{\childdocof}
Furthermore, add the commands
\begin{center}
\begin{tabular}{l}
|\input{childdoc.def}|\\
|\childdocof{|\textit{main}|}|\\
\end{tabular}
\end{center}
at the top of every child file \textit{child}
which is included by |\include{|\textit{child}|}|
from within the main file
(or at least for those files to be compiled individually).
The argument \textit{main} must be the filename of the main file.

There are a couple of
considerations in setting up the main and child documents:

%%%%%%%%%%%%%%%%%%%%%%%%%%%%%%%%%%%%%%%%
\paragraph{Restrictions.}

Please note the following restrictions:
\begin{itemize}
\item
|\childdocmain| must be called with one argument \textit{main}
to ensure compatibility with earlier version of the package.
It must either be empty (|\childdocmain{}|)
or precisely match the filename of the main file in which it is specified.
See \secref{sec:detection} for further information.
\item
The filename \textit{main} must be specified without the |.tex| extension.
\item
The filename \textit{main} is case sensitive
(even in case-insensitive file systems)
due to internal string comparison.
\item
The argument \textit{main} should be fully expanded, it cannot be a macro.
\item
Subdirectories and special characters should be avoided in filenames.
\item
The command |\childdocmain{|\textit{main}|}| must be followed by a whitespace.
It should not be followed immediately by another command
or by a comment mark `|%|'.
This is because the \TeX{} parser reads the token immediately following
the argument of |\childdocmain| and puts it
at the beginning of every child section;
however, a white\-space is ignored.
\end{itemize}

%%%%%%%%%%%%%%%%%%%%%%%%%%%%%%%%%%%%%%%%
\paragraph{Content of Main File.}

It is advisable to place all content in the child files included by |\include|.
Any output contained in the main file will appear in all child documents
unless suppressed manually;
it cannot be suppressed automatically by the |\includeonly| directive
and thus should normally be avoided.
A method to include some content in the main file
by means of conditional processing is described in \secref{sec:conditional}.

%%%%%%%%%%%%%%%%%%%%%%%%%%%%%%%%%%%%%%%%
\paragraph{Page Numbering.}

When only a part of the document is compiled,
the appropriate numbering of pages
(as well as other status parameters)
is determined from the |.aux| files.
The latter contain information from previous passes.
However this information needs to propagate through
all intermediate child documents.
Therefore the page numbering in child documents may well
be inconsistent until the complete document is compiled at least once.

A useful (if unconventional) way to always ensure a consistent
page numbering is to restart the numbering in each child document
and denote the pages by `\textit{child}|.|\textit{page}'
where \textit{child} represents the chapter/section number of the child file.
This can be achieved by the command
|\numberwithin{page}{|\textit{child}|}|
of the \textsf{amsmath} package
where \textit{child} can be |chapter| or |section|
depending on the chosen structuring.
Alternatively, one can modify the macro |\thepage| appropriately
and reset the counter |page| at the start of each child file.

%%%%%%%%%%%%%%%%%%%%%%%%%%%%%%%%%%%%%%%%%%%%%%%%%%%%%%%%%%%%%%%%%%%%%%%%%%%%%%%%
\subsection{Conditional Processing}
\label{sec:conditional}

The package provides a mechanism to compile different versions
of a document. To customise the versions further some conditional processing
can come in handy to distinguish which version is being compiled.
The package provides two macros to describe the compilation context:

%%%%%%%%%%%%%%%%%%%%%%%%%%%%%%%%%%%%%%%%
\DescribeMacro{\ifchilddoc}
The conditional |\ifchilddoc| distinguishes between the compilation of
child documents and the main document:
%
\begin{center}
|\ifchilddoc |\textit{child-code}| |[|\||else |\textit{main-code}]| \||fi|
\end{center}

%%%%%%%%%%%%%%%%%%%%%%%%%%%%%%%%%%%%%%%%
\DescribeMacro{\childdocname}
\DescribeMacro{\childdocjob}
The macro |\childdocname| contains the filename (without extension)
of the main or child file being processed.
Note that |\childdocjob| will always contain the name of the main file.

%%%%%%%%%%%%%%%%%%%%%%%%%%%%%%%%%%%%%%%%
\paragraph{Title Page.}

Conditional processing can be used to include a title or banner page
in the main document when proper precautions are taken.
Importantly, the code in the main file should ensure that the page counter
(as well as other status parameters which are stored in the |.aux| files)
takes the same value after the conditional processing.
Otherwise the page numbers may take divergent values
depending on which part is compiled.

For example, a title page could be declared by:
%
\begin{center}
\begin{tabular}{l}
|\ifchilddoc\||else|\\
|\addtocounter{page}{-1}|\\
\textit{code for title page}\\
|\newpage|\\
|\||fi|
\end{tabular}
\end{center}
%
A banner page for the child documents can be generated by:
%
\begin{center}
\begin{tabular}{l}
|\ifchilddoc|\\
|\addtocounter{page}{-1}|\\
\textit{code for banner page}\\
|\newpage|\\
|\||fi|
\end{tabular}
\end{center}
%
Here one could write a message such as:
\begin{center}
|This is the part \childdocname{} of \childdocjob{}.|
\end{center}

%%%%%%%%%%%%%%%%%%%%%%%%%%%%%%%%%%%%%%%%%%%%%%%%%%%%%%%%%%%%%%%%%%%%%%%%%%%%%%%%
\subsection{Flags}
\label{sec:flags}

The package makes it easy to generate different versions
of the main or child documents.
To this end compilation flags can be defined
and assigned different default values.
They will be particularly useful in conjunction
with the forwarding mechanism described in \secref{sec:forward}.

For example, it may be useful to have a flag |\version|
which can be set to |draft| or |final|.
The document source will contain some conditional code
depending on the value of |\version|.
Suppose further, the flag should default to |final| for the main file
and to |draft| for child files
which is a natural assignment for editing the document.
This is achieved by placing the following code
in the preamble of the main document
(below the |\childdocmain| directive):
%
\begin{center}
\begin{tabular}{l}
|\ifchilddoc|\\
|\providecommand{\version}{draft}|\\
|\||else|\\
|\providecommand{\version}{final}|\\
|\||fi|
\end{tabular}
\end{center}
%
The definition by |\providecommand| makes sure
that previous definitions are not overwritten.
Further statements |\providecommand{\version}{...}|
can thus be added before the above code to override it.

For the main file, one might add a line
(between |\childdocmain| and the above block)
%
\begin{center}
|%\ifchilddoc\||else\providecommand{\version}{draft}\||fi|
\end{center}
%
which can be uncommented to produce a draft version.
Likewise one can add a line to the very top of a child file
(above the |\childdocof{|\textit{main}|}| directive)
%
\begin{center}
|%\providecommand{\version}{final}|
\end{center}
%
which can be uncommented to produce the final version of this child document.

%%%%%%%%%%%%%%%%%%%%%%%%%%%%%%%%%%%%%%%%%%%%%%%%%%%%%%%%%%%%%%%%%%%%%%%%%%%%%%%%
\subsection{Forwarding}
\label{sec:forward}

Different versions of the main or child documents
using compilation flags as described in \secref{sec:flags}
can be (permanently) stored in different files
for convenient compilation, viewing and distribution.
To this end, the package defines a command
to pass on compilation to a different file:

%%%%%%%%%%%%%%%%%%%%%%%%%%%%%%%%%%%%%%%%
\DescribeMacro{\childdocforward}
The command |\childdocforward| redirects processing to
another source file:
%
\begin{center}
\begin{tabular}{l}
|\input{childdoc.def}|\\
|\childdocforward[|\textit{main}|]{|\textit{dest}|}|\\
\end{tabular}
\end{center}
%
The argument \textit{dest} is the destination file
(without extension).
It should be the main file or one of the child files.
Note that further \textsf{childdoc} directives
such as |\childdocof| and |\childdocforward|
in the indicated file will be processed in this form.
The optional argument \textit{main}
passes on directly to the main file \textit{main}
while pretending to compile the child \textit{dest}.
This form behaves as if \textit{dest}
issues |\childdocof{|\textit{main}|}| right away,
and no further \textsf{childdoc} directives will be processed.

%%%%%%%%%%%%%%%%%%%%%%%%%%%%%%%%%%%%%%%%
\DescribeMacro{\...prefix}
In the alternative form |\childdocforwardprefix|,
%
\begin{center}
\begin{tabular}{l}
|\input{childdoc.def}|\\
|\childdocforwardprefix[|\textit{main}|]{|\textit{prefix}|}{|\textit{dest}|}|
\end{tabular}
\end{center}
%
the destination file is determined by a pattern
depending on the current file:
To make this work, the current file must be called
`{\textit{prefix}\hspace{0.2em}\textit{suffix}}'
with \textit{prefix} matching precisely the argument.
Processing is then passed on to the file
`{\textit{dest}\hspace{0.2em}\textit{suffix}}'.
Surely, the same effect is achieved by
directly specifying the
argument `{\textit{dest}\hspace{0.2em}\textit{suffix}}'
in the first form.
However, that requires to set up a different file
for each child. With the alternative form of the command
all these files can have exactly the same content
which simplifies setting them up and maintaining them.

For example, the following file |draft.tex|
with a compilation flag |\version| as described in \secref{sec:flags}
compiles the main document as a draft:
%
\begin{center}
\begin{tabular}{l}
|\def\version{draft}|\\
|\input{childdoc.def}|\\
|\childdocforward{|\textit{main}|}|
\end{tabular}
\end{center}
%
Likewise, the following files |final|\textit{nn}|.tex|
compile the final version of the child document
|child|\textit{nn}|.tex|:
%
\begin{center}
\begin{tabular}{l}
|\def\version{final}|\\
|\input{childdoc.def}|\\
|\childdocforwardprefix{final}{child}|
\end{tabular}
\end{center}
%

Note that when several versions of a main file and/or of each child file
are to be generated, it may be convenient to set up a |Makefile| or
shell script to automatise the process.

%%%%%%%%%%%%%%%%%%%%%%%%%%%%%%%%%%%%%%%%%%%%%%%%%%%%%%%%%%%%%%%%%%%%%%%%%%%%%%%%
\subsection{Command Line Processing}
\label{sec:commandline}

The effect of redirection files can also be achieved by invoking
the \LaTeX{} compiler with a more elaborate command line.
Most conveniently this should be done as part
of a shell script or a |Makefile|.

When using \textsf{childdoc} in the main file, the following
command lines effectively perform a redirection
(note that depending on the shell being used,
backslashes may have to be doubled: `|\|' $\to$ `|\\|'):
%
\begin{center}
|... -jobname "|\textit{target}|" |\\|"|[\textit{flags}]%
|\input{childdoc.def}\childdocforward[|\textit{main}|]{|\textit{dest}|}"|
\end{center}
%
Here \textit{target} is the name of the output file,
\textit{main} is the name of the main file
and \textit{dest} is the name of the main or child file to be processed
(all filenames without extensions).
The optional argument \textit{main} can be omitted
if \textit{main} matches \textit{dest}.
Optionally, compilation \textit{flags} can be defined via |\def| commands.
This command line makes the \TeX{} engine believe
it is compiling the file \textit{target}
whose content is specified as the latter parameter.
The provided code then forwards the processing to
\textit{main} or \textit{dest} as described in \secref{sec:forward}.

%%%%%%%%%%%%%%%%%%%%%%%%%%%%%%%%%%%%%%%%%%%%%%%%%%%%%%%%%%%%%%%%%%%%%%%%%%%%%%%%
\subsection{Include by Input}
\label{sec:input}

Including child documents by |\include| has some restrictions by design.
Most notably, the content of a child document always occupies
its own set of pages; pages cannot be shared between child documents.
Usually, this behaviour makes perfect sense
because each child document contain an essential part of the document.
However, in some situations it may be desirable to compose
a document from a collection of parts
without having mandatory page breaks between then.
For this case, the package
provides a mechanism to include parts
by |\input| which can also be processed individually.
However, by construction this mechanism
requires manual handling of the content to be output.

%%%%%%%%%%%%%%%%%%%%%%%%%%%%%%%%%%%%%%%%
\DescribeMacro{\ifchilddocmanual}
The main file should be prepared as usual, see \secref{sec:include}.
However, the document body must make a distinction
between processing of an individual part and of the main document, e.g.:
%
\begin{center}
\begin{tabular}{l}
|\ifchilddocmanual|\\
|\input{\childdocname}|\\
|\||else|\\
\textit{document body with }|\input{|\textit{part}|}|\\
|\||fi|
\end{tabular}
\end{center}
%
The conditional |\ifchilddocmanual| is true whenever
a part to be included by |\input| is being compiled,
and the name of the part is stored in |\childdocname|.

%%%%%%%%%%%%%%%%%%%%%%%%%%%%%%%%%%%%%%%%
\DescribeMacro{\childdocby}
Each part to be included by |\input| should start with:
%
\begin{center}
\begin{tabular}{l}
|\input{childdoc.def}|\\
|\childdocby{|\textit{main}|}|\\
\end{tabular}
\end{center}
%
The directive |\childdocby| is similar to |\childdocof|
described in \secref{sec:include},
but the subsequent selection of content must be done manually.
To that end, both |\ifchilddoc| and |\ifchilddocmanual|
will be true upon processing of a part,
and the name of the part is stored in |\childdocname|.
Note that |\jobname| will be set to the filename of the current part
so that each part receives an individual |.aux| file
that does not interfere with the |.aux| file(s) of the main document.
This behaviour can be altered by the alternative form
|\childdocby[*]{|\textit{main}|}| (with a non-empty optional argument)
which uses the |.aux| file of the main document
by setting |\jobname| to \textit{main}.

%%%%%%%%%%%%%%%%%%%%%%%%%%%%%%%%%%%%%%%%%%%%%%%%%%%%%%%%%%%%%%%%%%%%%%%%%%%%%%%%
\subsection{Driver Development}
\label{sec:driver}

The \textsf{childdoc} mechanism can also be use for the development
of definition files such as \LaTeX{} styles or classes.
This case differs from the above setup with multiple parts
included by |\include| in that no |\includeonly| should be invoked.
This can be achieved by starting the include file
(before |\ProvidesPackage|) with:
%
\begin{center}
\begin{tabular}{l}
|\input{childdoc.def}|\\
|\childdocforward{|\textit{main}|}|\\
\end{tabular}
\end{center}
%
or alternatively with:
%
\begin{center}
\begin{tabular}{l}
|\input{childdoc.def}|\\
|\childdocby{|\textit{main}|}|\\
\end{tabular}
\end{center}
%
Both forms have slightly different effects as described above.
The main file is prepared as usual, see \secref{sec:include}.

%%%%%%%%%%%%%%%%%%%%%%%%%%%%%%%%%%%%%%%%%%%%%%%%%%%%%%%%%%%%%%%%%%%%%%%%%%%%%%%%
\subsection{Legacy Detection}
\label{sec:detection}

The directive |\childdocmain| in the main file can detect
whether the complete document or merely a child is to be compiled
even without using the directive |\childdocof|.
This method is deprecated because it is less robust
and there is no compelling reason to use it;
it is merely provided for backward compatibility
and it may be removed in future versions.

If the detection mechanism is to be used,
it is mandatory to correctly specify
the filename of the main file as the argument of |\childdocmain|:
%
\begin{center}
\begin{tabular}{l}
|\input{childdoc.def}|\\
|\childdocmain{|\textit{main}|}|\\
\end{tabular}
\end{center}
%
If |\jobname| does not match the argument \textit{main} of |\childdocmain|,
it is assumed that |\jobname| points to the child file to be compiled.
When using |\childdocmain| with the main file specified as argument,
it suffices to start a child file
with just |\input{|\textit{main}|}|
without loading of the package and using |\childdocof|.
If instead all processing is done
with the appropriate \textsf{childdoc} directives,
the argument of \textit{main} of |\childdocmain| can be empty.

An alternative version of the command line processing described
in \secref{sec:commandline} using the detection mechanism reads:
%
\begin{center}
|... -jobname "|\textit{target}|" "|[\textit{flags}]%
[|\def\jobname{|\textit{dest}|}|]|\input{|\textit{main}|}"|
\end{center}

%%%%%%%%%%%%%%%%%%%%%%%%%%%%%%%%%%%%%%%%%%%%%%%%%%%%%%%%%%%%%%%%%%%%%%%%%%%%%%%%
\subsection{Manual Code}
\label{sec:manual}

In case one cannot be certain whether the definitions file |childdoc.def|
is installed on the target \TeX{} distribution
and one prefers not to ship it,
it is conceivable to paste a few relevant commands into the sources.

To that end, drop all statements |\input{childdoc.def}|
and perform the replacements as outlined below.
Instead of |\childdocmain{|\textit{main}|}| add the following code
to the top of the main file:
%
\begin{center}
\begin{tabular}{l}
|\||ifdefined\childdocname\endinput\||fi\newif\ifchilddoc|\\
|\edef\childdocname{\scantokens\expandafter{\jobname\noexpand}}|\\
|\def\childdocmain{|\textit{main}|}\||ifx\childdocmain\childdocname\||else|\\
|\childdoctrue\includeonly{\childdocname}\let\jobname\childdocmain\||fi|\\
\end{tabular}
\end{center}
%
Instead of |\childdocof{|\textit{main}|}| just include the main file
at the top of each child file:
%
\begin{center}
|\input{|\textit{main}|}|
\end{center}
%
A simple redirection |\childdocforward{|\textit{dest}|}| is achieved by:
%
\begin{center}
|\def\jobname{|\textit{dest}|}\input{\jobname}|
\end{center}
%
The redirection with prefix
|\childdocforwardprefix[|\textit{prefix}|]{|\textit{dest}|}|
is accomplished by:
%
\begin{center}
\begin{tabular}{l}
|{\edef\jobname{\scantokens\expandafter{\jobname\noexpand}}|\\
|\def\redirectjob |\textit{prefix}|#1~~~{\gdef\jobname{|\textit{dest}|#1}}|\\
|\expandafter\redirectjob\jobname~~~}\input{\jobname}|
\end{tabular}
\end{center}

In an alternative approach,
child documents can be compiled by a specific command line
without additional code or specific definitions:
%
\begin{center}
|... -jobname "|\textit{target}|" "|[\textit{flags}]%
|\includeonly{|\textit{dest}|}\input{|\textit{main}|}"|
\end{center}
%

%%%%%%%%%%%%%%%%%%%%%%%%%%%%%%%%%%%%%%%%%%%%%%%%%%%%%%%%%%%%%%%%%%%%%%%%%%%%%%%%
%%%%%%%%%%%%%%%%%%%%%%%%%%%%%%%%%%%%%%%%%%%%%%%%%%%%%%%%%%%%%%%%%%%%%%%%%%%%%%%%
\section{Information}

%%%%%%%%%%%%%%%%%%%%%%%%%%%%%%%%%%%%%%%%%%%%%%%%%%%%%%%%%%%%%%%%%%%%%%%%%%%%%%%%
\subsection{Copyright}

Copyright \copyright{} 2017--2018 Niklas Beisert

This work may be distributed and/or modified under the
conditions of the \LaTeX{} Project Public License, either version 1.3
of this license or (at your option) any later version.
The latest version of this license is in
  \url{http://www.latex-project.org/lppl.txt}
and version 1.3 or later is part of all distributions of \LaTeX{}
version 2005/12/01 or later.

This work has the LPPL maintenance status `maintained'.

The Current Maintainer of this work is Niklas Beisert.

This work consists of the files |README.txt|, |childdoc.ins| and |childdoc.dtx|
as well as the derived files |childdoc.def|, |cdocsamp.tex|
with |cdocsch1.tex|, |cdocsch2.tex|, |cdocspt3.tex|, |cdocspt4.tex|,
|cdocsdrf.tex|, |cdocsfn1.tex|, |cdocsfn2.tex|
as well as |childdoc.pdf|.

%%%%%%%%%%%%%%%%%%%%%%%%%%%%%%%%%%%%%%%%%%%%%%%%%%%%%%%%%%%%%%%%%%%%%%%%%%%%%%%%
\subsection{Files and Installation}

The package consists of the files:
%
\begin{center}
\begin{tabular}{ll}
    |README.txt|   & readme file \\
    |childdoc.ins| & installation file \\
    |childdoc.dtx| & source file \\
    |childdoc.def| & definition file \\
    |cdocsamp.tex| & sample main file \\
    |cdocsch1.tex| & sample include file \\
    |cdocsch2.tex| & sample include file \\
    |cdocspt3.tex| & sample part file \\
    |cdocspt4.tex| & sample part file \\
    |cdocsdrf.tex| & sample redirection file \\
    |cdocsfn1.tex| & sample redirection file \\
    |cdocsfn2.tex| & sample redirection file \\
    |childdoc.pdf| & manual
\end{tabular}
\end{center}
%
The distribution consists of the files
|README.txt|, |childdoc.ins| and |childdoc.dtx|.
%
\begin{itemize}
\item
Run (pdf)\LaTeX{} on |childdoc.dtx|
to compile the manual |childdoc.pdf| (this file).
\item
Run \LaTeX{} on |childdoc.ins| to create the definitions file |childdoc.def|
and the sample |cdocsamp.tex| with include files
|cdocsch1.tex|, |cdocsch2.tex|, |cdocspt3.tex|, |cdocspt4.tex|,
|cdocsdrf.tex|, |cdocsfn1.tex|, |cdocsfn2.tex|.
Then copy the file |childdoc.def| to an appropriate directory of your \LaTeX{}
distribution, e.g.\ \textit{texmf-root}|/tex/latex/childdoc|.
\end{itemize}

%%%%%%%%%%%%%%%%%%%%%%%%%%%%%%%%%%%%%%%%%%%%%%%%%%%%%%%%%%%%%%%%%%%%%%%%%%%%%%%%
\subsection{Related CTAN Packages}

There are several other packages which offer a similar functionality:
%
\begin{itemize}
\item
The packages
\href{http://ctan.org/pkg/docmute}{\textsf{docmute}},
\href{http://ctan.org/pkg/includex}{\textsf{includex}} and
\href{http://ctan.org/pkg/standalone}{\textsf{standalone}}
provide commands to include only the document body of
a child file thus allowing both files to be compiled individually.
\item
The packages \href{http://ctan.org/pkg/subdocs}{\textsf{subdocs}}
and \href{http://ctan.org/pkg/subfiles}{\textsf{subfiles}}
provide structures in which the main and child documents can be
encapsulated and allowing them to be compiled individually.
The inclusion mechanism is different from the conventional |\include|.
\item
The package \href{http://ctan.org/pkg/combine}{\textsf{combine}}
is an elaborate solution to combine several documents into one.
\end{itemize}
%
See also the CTAN topic \href{http://ctan.org/topic/subdocs}{\textsf{subdocs}}
for further related packages.
The present package differs from the above solutions in that
a document structure constructed with the conventional |\include| mechanism
just needs two extra commands at the top of every file
such that all constituent files can be compiled individually.

%%%%%%%%%%%%%%%%%%%%%%%%%%%%%%%%%%%%%%%%%%%%%%%%%%%%%%%%%%%%%%%%%%%%%%%%%%%%%%%%
%\subsection{Feature Suggestions}
%
%The following is a list of features which may be useful for future
%versions of this package:
%%
%\begin{itemize}
%\item
%\ldots
%\end{itemize}

%%%%%%%%%%%%%%%%%%%%%%%%%%%%%%%%%%%%%%%%%%%%%%%%%%%%%%%%%%%%%%%%%%%%%%%%%%%%%%%%
\subsection{Revision History}

%%%%%%%%%%%%%%%%%%%%%%%%%%%%%%%%%%%%%%%%
\paragraph{v2.0:} 2018/12/30

\begin{itemize}
\item
immediate forward processing
\item
added |\childdocby| mechanism
\item
manual restructured
\end{itemize}

%%%%%%%%%%%%%%%%%%%%%%%%%%%%%%%%%%%%%%%%
\paragraph{v1.6:} 2018/01/17

\begin{itemize}
\item
application for development of include files
\item
corrections to manual
\end{itemize}

%%%%%%%%%%%%%%%%%%%%%%%%%%%%%%%%%%%%%%%%
\paragraph{v1.5:} 2017/05/21

\begin{itemize}
\item
more complete structuring introduced
\item
|\childdocof| introduced
\item
|\childdoc| renamed to |\childdocmain|
\item
|\childredirect| renamed to |\childdocforward| and |\childdocforwardprefix|
and functionality expanded
\end{itemize}

%%%%%%%%%%%%%%%%%%%%%%%%%%%%%%%%%%%%%%%%
\paragraph{v1.0:} 2017/04/27

\begin{itemize}
\item
manual and install package
\item
first version published on CTAN
\end{itemize}

%%%%%%%%%%%%%%%%%%%%%%%%%%%%%%%%%%%%%%%%
\paragraph{v0.6:} 2017/04/26

\begin{itemize}
\item
redirection mechanism added
\end{itemize}

%%%%%%%%%%%%%%%%%%%%%%%%%%%%%%%%%%%%%%%%
\paragraph{v0.5:} 2017/04/26

\begin{itemize}
\item
functionality in definition file
\end{itemize}


%%%%%%%%%%%%%%%%%%%%%%%%%%%%%%%%%%%%%%%%%%%%%%%%%%%%%%%%%%%%%%%%%%%%%%%%%%%%%%%%
%%%%%%%%%%%%%%%%%%%%%%%%%%%%%%%%%%%%%%%%%%%%%%%%%%%%%%%%%%%%%%%%%%%%%%%%%%%%%%%%
%%%%%%%%%%%%%%%%%%%%%%%%%%%%%%%%%%%%%%%%%%%%%%%%%%%%%%%%%%%%%%%%%%%%%%%%%%%%%%%%
\appendix

\settowidth\MacroIndent{\rmfamily\scriptsize 000\ }

 \DocInput{childdoc.dtx}

\end{document}
%</driver>
% \fi
%
% %%%%%%%%%%%%%%%%%%%%%%%%%%%%%%%%%%%%%%%%%%%%%%%%%%%%%%%%%%%%%%%%%%%%%%%%%%%%%%
% %%%%%%%%%%%%%%%%%%%%%%%%%%%%%%%%%%%%%%%%%%%%%%%%%%%%%%%%%%%%%%%%%%%%%%%%%%%%%%
% \section{Sample}
%\iffalse
%<*samplemain>
%\fi
%
% The following presents a sample document
% with two chapters, two parts, a title page,
% a compile flag as well as three forwarding files to set the flag.
% It consists of eight |.tex| files:
% \begin{center}
% \begin{tabular}{ll}
% |cdocsamp.tex|&main file\\
% |cdocsch1.tex|&include file for chapter 1\\
% |cdocsch2.tex|&include file for chapter 2\\
% |cdocspt3.tex|&include file for part 3\\
% |cdocspt4.tex|&include file for part 4\\
% |cdocsdrf.tex|&forwarding file for main file in draft mode\\
% |cdocsfi1.tex|&forwarding file for final version of chapter 1\\
% |cdocsfi2.tex|&forwarding file for final version of chapter 2\\
% \end{tabular}
% \end{center}
% Each of the eight files can be compiled directly by the \LaTeX{} compiler.
%
% %%%%%%%%%%%%%%%%%%%%%%%%%%%%%%%%%%%%%%
% \paragraph{Main File.}
%
% The main file is called |cdocsamp.tex|.
%
% Load the \textsf{childdoc} definitions and
% declare the filename for the main document:
%    \begin{macrocode}
\input{childdoc.def}
\childdocmain{}
%    \end{macrocode}

% Optional override for |\version| flag:
%    \begin{macrocode}
%%\ifchilddoc\else\providecommand{\version}{draft}\fi
%    \end{macrocode}

% Define the default values for the |\version| flag
% (|final| for the main file and |draft| for childs):
%    \begin{macrocode}
\ifchilddoc
\providecommand{\version}{draft}
\else
\providecommand{\version}{final}
\fi
%    \end{macrocode}

% Load the standard document class:
%    \begin{macrocode}
\documentclass[12pt]{article}
%    \end{macrocode}

% Start the document body:
%    \begin{macrocode}
\begin{document}
%    \end{macrocode}

% Declare a title page.
% Print title, part of document being processed and version flag:
%    \begin{macrocode}
\addtocounter{page}{-1}
\begin{center}
{\LARGE\bfseries{}childdoc example\par}
\vspace{1cm}
\ifchilddoc
\ifchilddocmanual part\else chapter\fi:
`\childdocname' of `\childdocjob'\par
\else
main document: `\childdocjob'\par
\fi
version: \version\par
\end{center}
\newpage
%    \end{macrocode}

% Manually include selected file,
% otherwise process as usual:
%    \begin{macrocode}
\ifchilddocmanual
\section*{part `\childdocname'}
\input{\childdocname}
\else
%    \end{macrocode}

% Include the two chapters:
%    \begin{macrocode}
\include{cdocsch1}
\include{cdocsch2}
%    \end{macrocode}

% Include the two parts unless only chapters should be displayed:
%    \begin{macrocode}
\ifchilddoc\else
\section{part three}
\input{cdocspt3}
\section{part four}
\input{cdocspt4}
\fi
%    \end{macrocode}

% Process as usual until here:
%    \begin{macrocode}
\fi
%    \end{macrocode}

% End of document body:
%    \begin{macrocode}
\end{document}
%    \end{macrocode}
%\iffalse
%</samplemain>
%\fi
%
% %%%%%%%%%%%%%%%%%%%%%%%%%%%%%%%%%%%%%%
% \paragraph{Chapter Include Files.}
%
% The include files are called |cdocsch1.tex| and |cdocsch2.tex|.
%
%\iffalse
%<*samplechap1|samplechap2>
%\fi

% Optional override for |\version| flag:
%    \begin{macrocode}
%%\providecommand{\version}{final}
%    \end{macrocode}

% Include the main document:
%    \begin{macrocode}
\input{childdoc.def}
\childdocof{cdocsamp}
%    \end{macrocode}

%\iffalse
%</samplechap1|samplechap2>
%\fi
%
%\iffalse
%<*samplechap1>
%\fi
% Some text for chapter 1:
%    \begin{macrocode}
\section{one}
some text in chapter one
%    \end{macrocode}

%\iffalse
%</samplechap1>
%\fi
% Some text for chapter 2:
%\iffalse
%<*samplechap2>
%\fi
%    \begin{macrocode}
\section{two}
more text in chapter two
%    \end{macrocode}

%\iffalse
%</samplechap2>
%\fi
%
% %%%%%%%%%%%%%%%%%%%%%%%%%%%%%%%%%%%%%%
% \paragraph{Part Include Files.}
%
% The include files are called |cdocspt3.tex| and |cdocspt4.tex|.
%
%\iffalse
%<*samplepart3|samplepart4>
%\fi

% Optional override for |\version| flag:
%    \begin{macrocode}
%%\providecommand{\version}{final}
%    \end{macrocode}

% Include the main document:
%    \begin{macrocode}
\input{childdoc.def}
\childdocby{cdocsamp}
%    \end{macrocode}

%\iffalse
%</samplepart3|samplepart4>
%\fi
%
%\iffalse
%<*samplepart3>
%\fi
% Some text for part 3:
%    \begin{macrocode}
some text in part three
%    \end{macrocode}

%\iffalse
%</samplepart3>
%\fi
% Some text for part 4:
%\iffalse
%<*samplepart4>
%\fi
%    \begin{macrocode}
more text in part four
%    \end{macrocode}

%\iffalse
%</samplepart4>
%\fi
%
% %%%%%%%%%%%%%%%%%%%%%%%%%%%%%%%%%%%%%%
% \paragraph{Forwarding for a Complete Draft.}
%
% The following forwarding file |cdocsdrf.tex|
% compiles the main document in draft mode:
%\iffalse
%<*sampledraft>
%\fi
%    \begin{macrocode}
\def\version{draft}
\input{childdoc.def}
\childdocforward{cdocsamp}
%    \end{macrocode}

%\iffalse
%</sampledraft>
%\fi
%
% %%%%%%%%%%%%%%%%%%%%%%%%%%%%%%%%%%%%%%
% \paragraph{Forwarding for Final Version of the Chapters.}
%
% The following forwarding files |cdocsfn1.tex| and |cdocsfn2.tex|
% (with identical content)
% compile the final versions of the child documents
% |cdocsch1.tex| and |cdocsch2.tex|, respectively:
%\iffalse
%<*samplefinal>
%\fi
%    \begin{macrocode}
\def\version{final}
\input{childdoc.def}
\childdocforwardprefix[cdocsamp]{cdocsfn}{cdocsch}
%    \end{macrocode}

%\iffalse
%</samplefinal>
%\fi
%
% %%%%%%%%%%%%%%%%%%%%%%%%%%%%%%%%%%%%%%
% \paragraph{Command Line Processing.}
%
% The following three command lines generate the output files
% |cdocscld|, |cdocscl1| and |cdocscl2|
% which should be identical to
% |cdocsdrf|, |cdocsch1| and |cdocsfn2|, respectively:
% \begin{center}
% \begin{tabular}{l}
% |latex -jobname cdocscld \|\\
% |  "\def\version{draft}\input{childdoc.def}\childdocforward{cdocsamp}"|\\
% |latex -jobname cdocscl1 \|\\
% |  "\input{childdoc.def}\childdocforward[cdocsamp]{cdocsch1}"|\\
% |latex -jobname cdocscl2 \|\\
% |  "\def\version{final}\input{childdoc.def}\childdocforward{cdocsch2}"|
% \end{tabular}
% \end{center}
% Note that the trailing backslash on each first line
% merely continues the input to the second line
% (for convenient cut ant paste).
% Furthermore, the command |latex| can be replaced by any
% of its alternative versions such as |pdflatex|.
%
% %%%%%%%%%%%%%%%%%%%%%%%%%%%%%%%%%%%%%%%%%%%%%%%%%%%%%%%%%%%%%%%%%%%%%%%%%%%%%%
% %%%%%%%%%%%%%%%%%%%%%%%%%%%%%%%%%%%%%%%%%%%%%%%%%%%%%%%%%%%%%%%%%%%%%%%%%%%%%%
% \section{Implementation}
%\iffalse
%<*package>
%\fi
%
% This section describes the definitions file |childdoc.def|.

% The definitions cannot be loaded using |\usepackage| or |\RequirePackage|
% which has a mechanism to prevent loading a style file more than once.
% When loading the definitions by means of |\input|
% multiple instances have to be prevented manually:
%\iffalse
%This code needs to be before the `\ProvidesFile' directive
%which is defined at the beginning of this file.
%Therefore it is also placed there and commented out here.
%</package>
%<*discard>
%\fi
%    \begin{macrocode}
\ifdefined\childdocmain\endinput\fi
%    \end{macrocode}
%\iffalse
%</discard>
%<*package>
%\fi
%
% \macro{\ifchilddoc}
% \macro{\ifchilddocmanual}
% The conditional |\ifchilddoc| tells whether a
% child (true) or main (false) document is being compiled.
% The conditional |\ifchilddocmanual| tells whether
% the |\includeonly| mechanism is used (false) or
% the selection of child files must be performed manually (true).
% The definitions initialise to false:
%    \begin{macrocode}
\newif\ifchilddoc
\newif\ifchilddocmanual
%    \end{macrocode}

% \macro{\childdocname}
% \macro{\childdocjob}
% The macro |\childdocname| stores the name of the main document
% to be compiled. The macro |\childdocjob| stores the name of
% the document on which the \LaTeX{} compiler was originally invoked.
% The content of |\jobname| cannot be compared
% to filenames specified in the source due to different catcodes.
% The following code rescans |\jobname|, stores the result
% in |\childdocname| and saves a copy in |\childdocjob|:
%    \begin{macrocode}
\edef\childdocname{\scantokens\expandafter{\jobname\noexpand}}
\let\childdocjob\childdocname
%    \end{macrocode}

% \macro{\childdocdisable}
% The macro |\childdocdisable| prevents the main file
% from being processed more than once.
% At this stage, the main document command |\childdocmain|
% is assumed to be called once again where it should do nothing.
% Any subsequent call to it should prevent
% a secondary processing of the main document
% It overwrites the forwarding commands
% |\childdocof| and |\childdocforward|
% with empty macros to prevent further inclusions of the main document:
%    \begin{macrocode}
\newcommand{\childdocdisable}
{
  \renewcommand{\childdocmain}[1]{\renewcommand{\childdocmain}[1]{\endinput}}
  \renewcommand{\childdocof}[1]{}
  \renewcommand{\childdocby}[2][]{}
  \renewcommand{\childdocforward}[2][]{}
  \renewcommand{\childdocdisable}{}
}
%    \end{macrocode}

% \macro{\childdocmain}
% The macro |\childdocmain| is to be called at the top of the main file
% with nothing or the main filename (without extension) as argument.
% First, it breaks loops.
% If the argument is not empty and does not match |\childdocname|
% (which is set by the first inclusion of |childdoc.def|),
% |\ifchilddoc| is set to true, |\includeonly| is applied to the child file
% and |\jobname| is set to the main file
% (for proper handling of |.aux| files):
%    \begin{macrocode}
\newcommand{\childdocmain}[1]
{
  \childdocdisable\childdocmain{}
  \if?#1?\else
    \begingroup
      \def\childdoctmp{#1}
      \ifx\childdoctmp\childdocname
        \def\childdoctmp{}
      \else
        \def\childdoctmp
        {
          \childdoctrue
          \includeonly{\childdocname}
          \def\childdocjob{#1}
          \def\jobname{#1}
        }
      \fi
      \expandafter
    \endgroup
    \childdoctmp
  \fi
}
%    \end{macrocode}

% \macro{\childdocof}
% The command |\childdocof| redirects
% compilation to the main file |#1|.
%    \begin{macrocode}
\newcommand{\childdocof}[1]
{
  \childdocdisable
  \childdoctrue
  \includeonly{\childdocname}
  \def\jobname{#1}
  \def\childdocjob{#1}
  \input{#1}
}
%    \end{macrocode}

% \macro{\childdocby}
% The command |\childdocby| ....
%    \begin{macrocode}
\newcommand{\childdocby}[2][]
{
  \childdocdisable
  \childdoctrue
  \childdocmanualtrue
  \if?#1?\else
    \def\jobname{#2}
  \fi
  \def\childdocjob{#2}
  \input{#2}
  \endinput
}
%    \end{macrocode}

% \macro{\childdocforward}
% The command |\childdocforward| redirects
% compilation to the main file or
% (if the optional argument is given) a child file.
% Parameters are set as if the main file
% or a child file starting with |\childdocof| was compiled.
% Then compilation is handed over to the main file:
%    \begin{macrocode}
\newcommand{\childdocforward}[2][]
{
  \begingroup
    \if?#1?
      \def\childdoctmp
      {
        \def\childdocname{#2}
        \def\childdocjob{#2}
        \def\jobname{#2}
        \input{#2}
        \endinput
      }
    \else
      \def\childdoctmp
      {
        \childdocdisable
        \def\childdocname{#2}
        \childdoctrue
        \includeonly{#2}
        \def\childdocjob{#1}
        \def\jobname{#1}
        \input{#1}
        \endinput
      }
    \fi
    \expandafter
  \endgroup
  \childdoctmp
}
%    \end{macrocode}

% \macro{\childdocforwardprefix}
% The command |\childdocforwardprefix| redirects
% compilation to the main or a child file by means of a pattern.
% The prefix |#1| in the current filename is replaced by |#2|
% and the suffix of the current filename is kept
% (it is assumed that the filename does not contain the substring `|~~~|'
% which is used as a delimiter).
% Compilation is handed over to the new file by |\childdocforward|:
%    \begin{macrocode}
\newcommand{\childdocforwardprefix}[3][]
{
  \begingroup
    \def\childdocextract #2##1~~~{\def\childdoctmp{\childdocforward[#1]{#3##1}}}
    \expandafter\childdocextract\childdocname~~~
    \expandafter
  \endgroup
  \childdoctmp
}
%    \end{macrocode}

% \macro{\childdoc}
% The deprecated macro |\childdoc| is a legacy version of |\childdocmain|:
%    \begin{macrocode}
\newcommand{\childdoc}{\childdocmain}
%    \end{macrocode}

% \macro{\childdocredirect}
% The deprecated macro |\childdocredirect| is a legacy version
% of |\childdocforward| and |\childdocforwardprefix|:
%    \begin{macrocode}
\newcommand{\childdocredirect}[2][]
{
  \begingroup
    \if?#1?
      \def\childdoctmp{\childdocforward{#2}}
    \else
      \def\childdoctmp{\childdocforwardprefix{#1}{#2}}
    \fi
    \expandafter
  \endgroup
  \childdoctmp
}
%    \end{macrocode}

%\iffalse
%</package>
%\fi
%
\endinput
\childdocforward{cdocsch2}"|
% \end{tabular}
% \end{center}
% Note that the trailing backslash on each first line
% merely continues the input to the second line
% (for convenient cut ant paste).
% Furthermore, the command |latex| can be replaced by any
% of its alternative versions such as |pdflatex|.
%
% %%%%%%%%%%%%%%%%%%%%%%%%%%%%%%%%%%%%%%%%%%%%%%%%%%%%%%%%%%%%%%%%%%%%%%%%%%%%%%
% %%%%%%%%%%%%%%%%%%%%%%%%%%%%%%%%%%%%%%%%%%%%%%%%%%%%%%%%%%%%%%%%%%%%%%%%%%%%%%
% \section{Implementation}
%\iffalse
%<*package>
%\fi
%
% This section describes the definitions file |childdoc.def|.

% The definitions cannot be loaded using |\usepackage| or |\RequirePackage|
% which has a mechanism to prevent loading a style file more than once.
% When loading the definitions by means of |\input|
% multiple instances have to be prevented manually:
%\iffalse
%This code needs to be before the `\ProvidesFile' directive
%which is defined at the beginning of this file.
%Therefore it is also placed there and commented out here.
%</package>
%<*discard>
%\fi
%    \begin{macrocode}
\ifdefined\childdocmain\endinput\fi
%    \end{macrocode}
%\iffalse
%</discard>
%<*package>
%\fi
%
% \macro{\ifchilddoc}
% \macro{\ifchilddocmanual}
% The conditional |\ifchilddoc| tells whether a
% child (true) or main (false) document is being compiled.
% The conditional |\ifchilddocmanual| tells whether
% the |\includeonly| mechanism is used (false) or
% the selection of child files must be performed manually (true).
% The definitions initialise to false:
%    \begin{macrocode}
\newif\ifchilddoc
\newif\ifchilddocmanual
%    \end{macrocode}

% \macro{\childdocname}
% \macro{\childdocjob}
% The macro |\childdocname| stores the name of the main document
% to be compiled. The macro |\childdocjob| stores the name of
% the document on which the \LaTeX{} compiler was originally invoked.
% The content of |\jobname| cannot be compared
% to filenames specified in the source due to different catcodes.
% The following code rescans |\jobname|, stores the result
% in |\childdocname| and saves a copy in |\childdocjob|:
%    \begin{macrocode}
\edef\childdocname{\scantokens\expandafter{\jobname\noexpand}}
\let\childdocjob\childdocname
%    \end{macrocode}

% \macro{\childdocdisable}
% The macro |\childdocdisable| prevents the main file
% from being processed more than once.
% At this stage, the main document command |\childdocmain|
% is assumed to be called once again where it should do nothing.
% Any subsequent call to it should prevent
% a secondary processing of the main document
% It overwrites the forwarding commands
% |\childdocof| and |\childdocforward|
% with empty macros to prevent further inclusions of the main document:
%    \begin{macrocode}
\newcommand{\childdocdisable}
{
  \renewcommand{\childdocmain}[1]{\renewcommand{\childdocmain}[1]{\endinput}}
  \renewcommand{\childdocof}[1]{}
  \renewcommand{\childdocby}[2][]{}
  \renewcommand{\childdocforward}[2][]{}
  \renewcommand{\childdocdisable}{}
}
%    \end{macrocode}

% \macro{\childdocmain}
% The macro |\childdocmain| is to be called at the top of the main file
% with nothing or the main filename (without extension) as argument.
% First, it breaks loops.
% If the argument is not empty and does not match |\childdocname|
% (which is set by the first inclusion of |childdoc.def|),
% |\ifchilddoc| is set to true, |\includeonly| is applied to the child file
% and |\jobname| is set to the main file
% (for proper handling of |.aux| files):
%    \begin{macrocode}
\newcommand{\childdocmain}[1]
{
  \childdocdisable\childdocmain{}
  \if?#1?\else
    \begingroup
      \def\childdoctmp{#1}
      \ifx\childdoctmp\childdocname
        \def\childdoctmp{}
      \else
        \def\childdoctmp
        {
          \childdoctrue
          \includeonly{\childdocname}
          \def\childdocjob{#1}
          \def\jobname{#1}
        }
      \fi
      \expandafter
    \endgroup
    \childdoctmp
  \fi
}
%    \end{macrocode}

% \macro{\childdocof}
% The command |\childdocof| redirects
% compilation to the main file |#1|.
%    \begin{macrocode}
\newcommand{\childdocof}[1]
{
  \childdocdisable
  \childdoctrue
  \includeonly{\childdocname}
  \def\jobname{#1}
  \def\childdocjob{#1}
  \input{#1}
}
%    \end{macrocode}

% \macro{\childdocby}
% The command |\childdocby| ....
%    \begin{macrocode}
\newcommand{\childdocby}[2][]
{
  \childdocdisable
  \childdoctrue
  \childdocmanualtrue
  \if?#1?\else
    \def\jobname{#2}
  \fi
  \def\childdocjob{#2}
  \input{#2}
  \endinput
}
%    \end{macrocode}

% \macro{\childdocforward}
% The command |\childdocforward| redirects
% compilation to the main file or
% (if the optional argument is given) a child file.
% Parameters are set as if the main file
% or a child file starting with |\childdocof| was compiled.
% Then compilation is handed over to the main file:
%    \begin{macrocode}
\newcommand{\childdocforward}[2][]
{
  \begingroup
    \if?#1?
      \def\childdoctmp
      {
        \def\childdocname{#2}
        \def\childdocjob{#2}
        \def\jobname{#2}
        \input{#2}
        \endinput
      }
    \else
      \def\childdoctmp
      {
        \childdocdisable
        \def\childdocname{#2}
        \childdoctrue
        \includeonly{#2}
        \def\childdocjob{#1}
        \def\jobname{#1}
        \input{#1}
        \endinput
      }
    \fi
    \expandafter
  \endgroup
  \childdoctmp
}
%    \end{macrocode}

% \macro{\childdocforwardprefix}
% The command |\childdocforwardprefix| redirects
% compilation to the main or a child file by means of a pattern.
% The prefix |#1| in the current filename is replaced by |#2|
% and the suffix of the current filename is kept
% (it is assumed that the filename does not contain the substring `|~~~|'
% which is used as a delimiter).
% Compilation is handed over to the new file by |\childdocforward|:
%    \begin{macrocode}
\newcommand{\childdocforwardprefix}[3][]
{
  \begingroup
    \def\childdocextract #2##1~~~{\def\childdoctmp{\childdocforward[#1]{#3##1}}}
    \expandafter\childdocextract\childdocname~~~
    \expandafter
  \endgroup
  \childdoctmp
}
%    \end{macrocode}

% \macro{\childdoc}
% The deprecated macro |\childdoc| is a legacy version of |\childdocmain|:
%    \begin{macrocode}
\newcommand{\childdoc}{\childdocmain}
%    \end{macrocode}

% \macro{\childdocredirect}
% The deprecated macro |\childdocredirect| is a legacy version
% of |\childdocforward| and |\childdocforwardprefix|:
%    \begin{macrocode}
\newcommand{\childdocredirect}[2][]
{
  \begingroup
    \if?#1?
      \def\childdoctmp{\childdocforward{#2}}
    \else
      \def\childdoctmp{\childdocforwardprefix{#1}{#2}}
    \fi
    \expandafter
  \endgroup
  \childdoctmp
}
%    \end{macrocode}

%\iffalse
%</package>
%\fi
%
\endinput

\childdocby{cdocsamp}
%    \end{macrocode}

%\iffalse
%</samplepart3|samplepart4>
%\fi
%
%\iffalse
%<*samplepart3>
%\fi
% Some text for part 3:
%    \begin{macrocode}
some text in part three
%    \end{macrocode}

%\iffalse
%</samplepart3>
%\fi
% Some text for part 4:
%\iffalse
%<*samplepart4>
%\fi
%    \begin{macrocode}
more text in part four
%    \end{macrocode}

%\iffalse
%</samplepart4>
%\fi
%
% %%%%%%%%%%%%%%%%%%%%%%%%%%%%%%%%%%%%%%
% \paragraph{Forwarding for a Complete Draft.}
%
% The following forwarding file |cdocsdrf.tex|
% compiles the main document in draft mode:
%\iffalse
%<*sampledraft>
%\fi
%    \begin{macrocode}
\def\version{draft}
% \iffalse
%
% childdoc.dtx Copyright (C) 2017-2018 Niklas Beisert
%
% This work may be distributed and/or modified under the
% conditions of the LaTeX Project Public License, either version 1.3
% of this license or (at your option) any later version.
% The latest version of this license is in
%   http://www.latex-project.org/lppl.txt
% and version 1.3 or later is part of all distributions of LaTeX
% version 2005/12/01 or later.
%
% This work has the LPPL maintenance status `maintained'.
%
% The Current Maintainer of this work is Niklas Beisert.
%
% This work consists of the files childdoc.dtx and childdoc.ins
% and the derived files childdoc.def and cdocsamp.tex with
% cdocsch1.tex, cdocsch2.tex, cdocsdrf.tex, cdocsfn1.tex, cdocsfn2.tex.
%
%<package>\ifdefined\childdocmain\endinput\fi
%<package>\ProvidesFile{childdoc.def}[2018/12/30 v2.0 child document driver]
%<samplemain>\ProvidesFile{cdocsamp.tex}[2018/12/30 v2.0 sample for childdoc]
%<*driver>
%\ProvidesFile{childdoc.drv}[2018/12/30 v2.0 childdoc reference manual file]
\PassOptionsToClass{10pt,a4paper}{article}
\documentclass{ltxdoc}

\usepackage[margin=35mm]{geometry}
\usepackage{hyperref}
\usepackage{hyperxmp}
\usepackage[usenames]{color}

\hypersetup{colorlinks=true}
\hypersetup{pdfstartview=FitH}
\hypersetup{pdfpagemode=UseNone}
\hypersetup{pdfsource={}}
\hypersetup{pdflang={en-UK}}
\hypersetup{pdfcopyright={Copyright 2017-2018 Niklas Beisert.
  This work may be distributed and/or modified under the
  conditions of the LaTeX Project Public License, either version 1.3
  of this license or (at your option) any later version.}}
\hypersetup{pdflicenseurl={http://www.latex-project.org/lppl.txt}}
\hypersetup{pdfcontactaddress={ETH Zurich, ITP, HIT K,
  Wolfgang-Pauli-Strasse 27}}
\hypersetup{pdfcontactpostcode={8093}}
\hypersetup{pdfcontactcity={Zurich}}
\hypersetup{pdfcontactcountry={Switzerland}}
\hypersetup{pdfcontactemail={nbeisert@itp.phys.ethz.ch}}
\hypersetup{pdfcontacturl={http://people.phys.ethz.ch/\xmptilde nbeisert/}}

\newcommand{\secref}[1]{\hyperref[#1]{section \ref*{#1}}}

\parskip1ex
\parindent0pt
\let\olditemize\itemize
\def\itemize{\olditemize\parskip0pt}

\begin{document}

\title{The \textsf{childdoc} Package}
\hypersetup{pdftitle={The childdoc Package}}
\author{Niklas Beisert\\[2ex]
  Institut f\"ur Theoretische Physik\\
  Eidgen\"ossische Technische Hochschule Z\"urich\\
  Wolfgang-Pauli-Strasse 27, 8093 Z\"urich, Switzerland\\[1ex]
  \href{mailto:nbeisert@itp.phys.ethz.ch}
  {\texttt{nbeisert@itp.phys.ethz.ch}}}
\hypersetup{pdfauthor={Niklas Beisert}}
\hypersetup{pdfsubject={Manual for the LaTeX2e Package childdoc}}
\date{30 December 2018, \textsf{v2.0}}
\maketitle

\begin{abstract}\noindent
\textsf{childdoc} is a \LaTeXe{} package
that enables the direct compilation
of document sections included by |\include|
to individual files.
\end{abstract}

\begingroup
\parskip0ex
\tableofcontents
\endgroup

%%%%%%%%%%%%%%%%%%%%%%%%%%%%%%%%%%%%%%%%%%%%%%%%%%%%%%%%%%%%%%%%%%%%%%%%%%%%%%%%
%%%%%%%%%%%%%%%%%%%%%%%%%%%%%%%%%%%%%%%%%%%%%%%%%%%%%%%%%%%%%%%%%%%%%%%%%%%%%%%%
\section{Introduction}

\LaTeX{} provides a mechanism to structure a large document (such as a book)
into a main file and several child files (containing the chapters)
using the |\include| command.
This mechanism is beneficial for documents
which span hundreds of pages in order to
make the source file(s) more manageable.
Moreover, compilation can be restricted to
selected child files by means of the |\includeonly| command.
The latter feature can be used to reduce the compilation time while editing
(this was significantly more useful in the earlier days of \LaTeX{})
or to generate a smaller document which is easier to navigate.
Another application of |\includeonly| is to generate
documents consisting of selected parts of the complete document.

However, there are a few drawbacks of the plain |\include| mechanism:
\begin{itemize}
\item
The child files cannot be compiled on their own,
they can only be compiled via the main file.
A naive editing environment
(such as a text editor with an option
to have the current file processed by \LaTeX)
may require one to switch to the main file before compiling;
attempting to compile the child file produces errors.
\item
The main file must be modified (each time)
to adjust the |\includeonly| command
to the present needs. This easily leaves the main file in a messy state.
\item
The generated document will always carry the filename
of the main document. This is inconvenient if
several child files are to be compiled and
to be kept for distribution.
\end{itemize}

The present package provides a simple interface
to make child files individually compilable by \LaTeX{}.
Compiling a child file then has the same effect as compiling
the main file with an |\includeonly| command
to select the appropriate child.
Moreover the generated document will carry the name of the child
rather than the main file.
This resolves all three above issues.

This feature is meant to make the editing of books,
thesis documents and lecture notes somewhat more convenient.
However, the package can also be used efficiently for
composing a series of documents (such as exercise sheets)
which are typically distributed individually.
It then assists the author in generating the individual documents
(potentially in different versions)
as well as a document containing the collected series.
Another application is in developing style files
or other kinds of included material
where compilation of the style file could redirect
to a sample or test file.

%%%%%%%%%%%%%%%%%%%%%%%%%%%%%%%%%%%%%%%%%%%%%%%%%%%%%%%%%%%%%%%%%%%%%%%%%%%%%%%%
%%%%%%%%%%%%%%%%%%%%%%%%%%%%%%%%%%%%%%%%%%%%%%%%%%%%%%%%%%%%%%%%%%%%%%%%%%%%%%%%
\section{Usage}

First of all, the package \textsf{childdoc} is \emph{not} a standard
\LaTeXe{} |.sty| style file! Therefore it needs to be invoked in
a non-standard way.

%%%%%%%%%%%%%%%%%%%%%%%%%%%%%%%%%%%%%%%%%%%%%%%%%%%%%%%%%%%%%%%%%%%%%%%%%%%%%%%%
\subsection{Included Files}
\label{sec:include}

%%%%%%%%%%%%%%%%%%%%%%%%%%%%%%%%%%%%%%%%
\DescribeMacro{\childdocmain}
To use the package, add the commands
\begin{center}
\begin{tabular}{l}
|% \iffalse
%
% childdoc.dtx Copyright (C) 2017-2018 Niklas Beisert
%
% This work may be distributed and/or modified under the
% conditions of the LaTeX Project Public License, either version 1.3
% of this license or (at your option) any later version.
% The latest version of this license is in
%   http://www.latex-project.org/lppl.txt
% and version 1.3 or later is part of all distributions of LaTeX
% version 2005/12/01 or later.
%
% This work has the LPPL maintenance status `maintained'.
%
% The Current Maintainer of this work is Niklas Beisert.
%
% This work consists of the files childdoc.dtx and childdoc.ins
% and the derived files childdoc.def and cdocsamp.tex with
% cdocsch1.tex, cdocsch2.tex, cdocsdrf.tex, cdocsfn1.tex, cdocsfn2.tex.
%
%<package>\ifdefined\childdocmain\endinput\fi
%<package>\ProvidesFile{childdoc.def}[2018/12/30 v2.0 child document driver]
%<samplemain>\ProvidesFile{cdocsamp.tex}[2018/12/30 v2.0 sample for childdoc]
%<*driver>
%\ProvidesFile{childdoc.drv}[2018/12/30 v2.0 childdoc reference manual file]
\PassOptionsToClass{10pt,a4paper}{article}
\documentclass{ltxdoc}

\usepackage[margin=35mm]{geometry}
\usepackage{hyperref}
\usepackage{hyperxmp}
\usepackage[usenames]{color}

\hypersetup{colorlinks=true}
\hypersetup{pdfstartview=FitH}
\hypersetup{pdfpagemode=UseNone}
\hypersetup{pdfsource={}}
\hypersetup{pdflang={en-UK}}
\hypersetup{pdfcopyright={Copyright 2017-2018 Niklas Beisert.
  This work may be distributed and/or modified under the
  conditions of the LaTeX Project Public License, either version 1.3
  of this license or (at your option) any later version.}}
\hypersetup{pdflicenseurl={http://www.latex-project.org/lppl.txt}}
\hypersetup{pdfcontactaddress={ETH Zurich, ITP, HIT K,
  Wolfgang-Pauli-Strasse 27}}
\hypersetup{pdfcontactpostcode={8093}}
\hypersetup{pdfcontactcity={Zurich}}
\hypersetup{pdfcontactcountry={Switzerland}}
\hypersetup{pdfcontactemail={nbeisert@itp.phys.ethz.ch}}
\hypersetup{pdfcontacturl={http://people.phys.ethz.ch/\xmptilde nbeisert/}}

\newcommand{\secref}[1]{\hyperref[#1]{section \ref*{#1}}}

\parskip1ex
\parindent0pt
\let\olditemize\itemize
\def\itemize{\olditemize\parskip0pt}

\begin{document}

\title{The \textsf{childdoc} Package}
\hypersetup{pdftitle={The childdoc Package}}
\author{Niklas Beisert\\[2ex]
  Institut f\"ur Theoretische Physik\\
  Eidgen\"ossische Technische Hochschule Z\"urich\\
  Wolfgang-Pauli-Strasse 27, 8093 Z\"urich, Switzerland\\[1ex]
  \href{mailto:nbeisert@itp.phys.ethz.ch}
  {\texttt{nbeisert@itp.phys.ethz.ch}}}
\hypersetup{pdfauthor={Niklas Beisert}}
\hypersetup{pdfsubject={Manual for the LaTeX2e Package childdoc}}
\date{30 December 2018, \textsf{v2.0}}
\maketitle

\begin{abstract}\noindent
\textsf{childdoc} is a \LaTeXe{} package
that enables the direct compilation
of document sections included by |\include|
to individual files.
\end{abstract}

\begingroup
\parskip0ex
\tableofcontents
\endgroup

%%%%%%%%%%%%%%%%%%%%%%%%%%%%%%%%%%%%%%%%%%%%%%%%%%%%%%%%%%%%%%%%%%%%%%%%%%%%%%%%
%%%%%%%%%%%%%%%%%%%%%%%%%%%%%%%%%%%%%%%%%%%%%%%%%%%%%%%%%%%%%%%%%%%%%%%%%%%%%%%%
\section{Introduction}

\LaTeX{} provides a mechanism to structure a large document (such as a book)
into a main file and several child files (containing the chapters)
using the |\include| command.
This mechanism is beneficial for documents
which span hundreds of pages in order to
make the source file(s) more manageable.
Moreover, compilation can be restricted to
selected child files by means of the |\includeonly| command.
The latter feature can be used to reduce the compilation time while editing
(this was significantly more useful in the earlier days of \LaTeX{})
or to generate a smaller document which is easier to navigate.
Another application of |\includeonly| is to generate
documents consisting of selected parts of the complete document.

However, there are a few drawbacks of the plain |\include| mechanism:
\begin{itemize}
\item
The child files cannot be compiled on their own,
they can only be compiled via the main file.
A naive editing environment
(such as a text editor with an option
to have the current file processed by \LaTeX)
may require one to switch to the main file before compiling;
attempting to compile the child file produces errors.
\item
The main file must be modified (each time)
to adjust the |\includeonly| command
to the present needs. This easily leaves the main file in a messy state.
\item
The generated document will always carry the filename
of the main document. This is inconvenient if
several child files are to be compiled and
to be kept for distribution.
\end{itemize}

The present package provides a simple interface
to make child files individually compilable by \LaTeX{}.
Compiling a child file then has the same effect as compiling
the main file with an |\includeonly| command
to select the appropriate child.
Moreover the generated document will carry the name of the child
rather than the main file.
This resolves all three above issues.

This feature is meant to make the editing of books,
thesis documents and lecture notes somewhat more convenient.
However, the package can also be used efficiently for
composing a series of documents (such as exercise sheets)
which are typically distributed individually.
It then assists the author in generating the individual documents
(potentially in different versions)
as well as a document containing the collected series.
Another application is in developing style files
or other kinds of included material
where compilation of the style file could redirect
to a sample or test file.

%%%%%%%%%%%%%%%%%%%%%%%%%%%%%%%%%%%%%%%%%%%%%%%%%%%%%%%%%%%%%%%%%%%%%%%%%%%%%%%%
%%%%%%%%%%%%%%%%%%%%%%%%%%%%%%%%%%%%%%%%%%%%%%%%%%%%%%%%%%%%%%%%%%%%%%%%%%%%%%%%
\section{Usage}

First of all, the package \textsf{childdoc} is \emph{not} a standard
\LaTeXe{} |.sty| style file! Therefore it needs to be invoked in
a non-standard way.

%%%%%%%%%%%%%%%%%%%%%%%%%%%%%%%%%%%%%%%%%%%%%%%%%%%%%%%%%%%%%%%%%%%%%%%%%%%%%%%%
\subsection{Included Files}
\label{sec:include}

%%%%%%%%%%%%%%%%%%%%%%%%%%%%%%%%%%%%%%%%
\DescribeMacro{\childdocmain}
To use the package, add the commands
\begin{center}
\begin{tabular}{l}
|\input{childdoc.def}|\\
|\childdocmain{}|\\
\end{tabular}
\end{center}
at the very top of the main \LaTeX{} file,
in particular \emph{before} the |\documentclass| statement!
The argument of |\childdocmain| should be left empty
(but it must be present).

%%%%%%%%%%%%%%%%%%%%%%%%%%%%%%%%%%%%%%%%
\DescribeMacro{\childdocof}
Furthermore, add the commands
\begin{center}
\begin{tabular}{l}
|\input{childdoc.def}|\\
|\childdocof{|\textit{main}|}|\\
\end{tabular}
\end{center}
at the top of every child file \textit{child}
which is included by |\include{|\textit{child}|}|
from within the main file
(or at least for those files to be compiled individually).
The argument \textit{main} must be the filename of the main file.

There are a couple of
considerations in setting up the main and child documents:

%%%%%%%%%%%%%%%%%%%%%%%%%%%%%%%%%%%%%%%%
\paragraph{Restrictions.}

Please note the following restrictions:
\begin{itemize}
\item
|\childdocmain| must be called with one argument \textit{main}
to ensure compatibility with earlier version of the package.
It must either be empty (|\childdocmain{}|)
or precisely match the filename of the main file in which it is specified.
See \secref{sec:detection} for further information.
\item
The filename \textit{main} must be specified without the |.tex| extension.
\item
The filename \textit{main} is case sensitive
(even in case-insensitive file systems)
due to internal string comparison.
\item
The argument \textit{main} should be fully expanded, it cannot be a macro.
\item
Subdirectories and special characters should be avoided in filenames.
\item
The command |\childdocmain{|\textit{main}|}| must be followed by a whitespace.
It should not be followed immediately by another command
or by a comment mark `|%|'.
This is because the \TeX{} parser reads the token immediately following
the argument of |\childdocmain| and puts it
at the beginning of every child section;
however, a white\-space is ignored.
\end{itemize}

%%%%%%%%%%%%%%%%%%%%%%%%%%%%%%%%%%%%%%%%
\paragraph{Content of Main File.}

It is advisable to place all content in the child files included by |\include|.
Any output contained in the main file will appear in all child documents
unless suppressed manually;
it cannot be suppressed automatically by the |\includeonly| directive
and thus should normally be avoided.
A method to include some content in the main file
by means of conditional processing is described in \secref{sec:conditional}.

%%%%%%%%%%%%%%%%%%%%%%%%%%%%%%%%%%%%%%%%
\paragraph{Page Numbering.}

When only a part of the document is compiled,
the appropriate numbering of pages
(as well as other status parameters)
is determined from the |.aux| files.
The latter contain information from previous passes.
However this information needs to propagate through
all intermediate child documents.
Therefore the page numbering in child documents may well
be inconsistent until the complete document is compiled at least once.

A useful (if unconventional) way to always ensure a consistent
page numbering is to restart the numbering in each child document
and denote the pages by `\textit{child}|.|\textit{page}'
where \textit{child} represents the chapter/section number of the child file.
This can be achieved by the command
|\numberwithin{page}{|\textit{child}|}|
of the \textsf{amsmath} package
where \textit{child} can be |chapter| or |section|
depending on the chosen structuring.
Alternatively, one can modify the macro |\thepage| appropriately
and reset the counter |page| at the start of each child file.

%%%%%%%%%%%%%%%%%%%%%%%%%%%%%%%%%%%%%%%%%%%%%%%%%%%%%%%%%%%%%%%%%%%%%%%%%%%%%%%%
\subsection{Conditional Processing}
\label{sec:conditional}

The package provides a mechanism to compile different versions
of a document. To customise the versions further some conditional processing
can come in handy to distinguish which version is being compiled.
The package provides two macros to describe the compilation context:

%%%%%%%%%%%%%%%%%%%%%%%%%%%%%%%%%%%%%%%%
\DescribeMacro{\ifchilddoc}
The conditional |\ifchilddoc| distinguishes between the compilation of
child documents and the main document:
%
\begin{center}
|\ifchilddoc |\textit{child-code}| |[|\||else |\textit{main-code}]| \||fi|
\end{center}

%%%%%%%%%%%%%%%%%%%%%%%%%%%%%%%%%%%%%%%%
\DescribeMacro{\childdocname}
\DescribeMacro{\childdocjob}
The macro |\childdocname| contains the filename (without extension)
of the main or child file being processed.
Note that |\childdocjob| will always contain the name of the main file.

%%%%%%%%%%%%%%%%%%%%%%%%%%%%%%%%%%%%%%%%
\paragraph{Title Page.}

Conditional processing can be used to include a title or banner page
in the main document when proper precautions are taken.
Importantly, the code in the main file should ensure that the page counter
(as well as other status parameters which are stored in the |.aux| files)
takes the same value after the conditional processing.
Otherwise the page numbers may take divergent values
depending on which part is compiled.

For example, a title page could be declared by:
%
\begin{center}
\begin{tabular}{l}
|\ifchilddoc\||else|\\
|\addtocounter{page}{-1}|\\
\textit{code for title page}\\
|\newpage|\\
|\||fi|
\end{tabular}
\end{center}
%
A banner page for the child documents can be generated by:
%
\begin{center}
\begin{tabular}{l}
|\ifchilddoc|\\
|\addtocounter{page}{-1}|\\
\textit{code for banner page}\\
|\newpage|\\
|\||fi|
\end{tabular}
\end{center}
%
Here one could write a message such as:
\begin{center}
|This is the part \childdocname{} of \childdocjob{}.|
\end{center}

%%%%%%%%%%%%%%%%%%%%%%%%%%%%%%%%%%%%%%%%%%%%%%%%%%%%%%%%%%%%%%%%%%%%%%%%%%%%%%%%
\subsection{Flags}
\label{sec:flags}

The package makes it easy to generate different versions
of the main or child documents.
To this end compilation flags can be defined
and assigned different default values.
They will be particularly useful in conjunction
with the forwarding mechanism described in \secref{sec:forward}.

For example, it may be useful to have a flag |\version|
which can be set to |draft| or |final|.
The document source will contain some conditional code
depending on the value of |\version|.
Suppose further, the flag should default to |final| for the main file
and to |draft| for child files
which is a natural assignment for editing the document.
This is achieved by placing the following code
in the preamble of the main document
(below the |\childdocmain| directive):
%
\begin{center}
\begin{tabular}{l}
|\ifchilddoc|\\
|\providecommand{\version}{draft}|\\
|\||else|\\
|\providecommand{\version}{final}|\\
|\||fi|
\end{tabular}
\end{center}
%
The definition by |\providecommand| makes sure
that previous definitions are not overwritten.
Further statements |\providecommand{\version}{...}|
can thus be added before the above code to override it.

For the main file, one might add a line
(between |\childdocmain| and the above block)
%
\begin{center}
|%\ifchilddoc\||else\providecommand{\version}{draft}\||fi|
\end{center}
%
which can be uncommented to produce a draft version.
Likewise one can add a line to the very top of a child file
(above the |\childdocof{|\textit{main}|}| directive)
%
\begin{center}
|%\providecommand{\version}{final}|
\end{center}
%
which can be uncommented to produce the final version of this child document.

%%%%%%%%%%%%%%%%%%%%%%%%%%%%%%%%%%%%%%%%%%%%%%%%%%%%%%%%%%%%%%%%%%%%%%%%%%%%%%%%
\subsection{Forwarding}
\label{sec:forward}

Different versions of the main or child documents
using compilation flags as described in \secref{sec:flags}
can be (permanently) stored in different files
for convenient compilation, viewing and distribution.
To this end, the package defines a command
to pass on compilation to a different file:

%%%%%%%%%%%%%%%%%%%%%%%%%%%%%%%%%%%%%%%%
\DescribeMacro{\childdocforward}
The command |\childdocforward| redirects processing to
another source file:
%
\begin{center}
\begin{tabular}{l}
|\input{childdoc.def}|\\
|\childdocforward[|\textit{main}|]{|\textit{dest}|}|\\
\end{tabular}
\end{center}
%
The argument \textit{dest} is the destination file
(without extension).
It should be the main file or one of the child files.
Note that further \textsf{childdoc} directives
such as |\childdocof| and |\childdocforward|
in the indicated file will be processed in this form.
The optional argument \textit{main}
passes on directly to the main file \textit{main}
while pretending to compile the child \textit{dest}.
This form behaves as if \textit{dest}
issues |\childdocof{|\textit{main}|}| right away,
and no further \textsf{childdoc} directives will be processed.

%%%%%%%%%%%%%%%%%%%%%%%%%%%%%%%%%%%%%%%%
\DescribeMacro{\...prefix}
In the alternative form |\childdocforwardprefix|,
%
\begin{center}
\begin{tabular}{l}
|\input{childdoc.def}|\\
|\childdocforwardprefix[|\textit{main}|]{|\textit{prefix}|}{|\textit{dest}|}|
\end{tabular}
\end{center}
%
the destination file is determined by a pattern
depending on the current file:
To make this work, the current file must be called
`{\textit{prefix}\hspace{0.2em}\textit{suffix}}'
with \textit{prefix} matching precisely the argument.
Processing is then passed on to the file
`{\textit{dest}\hspace{0.2em}\textit{suffix}}'.
Surely, the same effect is achieved by
directly specifying the
argument `{\textit{dest}\hspace{0.2em}\textit{suffix}}'
in the first form.
However, that requires to set up a different file
for each child. With the alternative form of the command
all these files can have exactly the same content
which simplifies setting them up and maintaining them.

For example, the following file |draft.tex|
with a compilation flag |\version| as described in \secref{sec:flags}
compiles the main document as a draft:
%
\begin{center}
\begin{tabular}{l}
|\def\version{draft}|\\
|\input{childdoc.def}|\\
|\childdocforward{|\textit{main}|}|
\end{tabular}
\end{center}
%
Likewise, the following files |final|\textit{nn}|.tex|
compile the final version of the child document
|child|\textit{nn}|.tex|:
%
\begin{center}
\begin{tabular}{l}
|\def\version{final}|\\
|\input{childdoc.def}|\\
|\childdocforwardprefix{final}{child}|
\end{tabular}
\end{center}
%

Note that when several versions of a main file and/or of each child file
are to be generated, it may be convenient to set up a |Makefile| or
shell script to automatise the process.

%%%%%%%%%%%%%%%%%%%%%%%%%%%%%%%%%%%%%%%%%%%%%%%%%%%%%%%%%%%%%%%%%%%%%%%%%%%%%%%%
\subsection{Command Line Processing}
\label{sec:commandline}

The effect of redirection files can also be achieved by invoking
the \LaTeX{} compiler with a more elaborate command line.
Most conveniently this should be done as part
of a shell script or a |Makefile|.

When using \textsf{childdoc} in the main file, the following
command lines effectively perform a redirection
(note that depending on the shell being used,
backslashes may have to be doubled: `|\|' $\to$ `|\\|'):
%
\begin{center}
|... -jobname "|\textit{target}|" |\\|"|[\textit{flags}]%
|\input{childdoc.def}\childdocforward[|\textit{main}|]{|\textit{dest}|}"|
\end{center}
%
Here \textit{target} is the name of the output file,
\textit{main} is the name of the main file
and \textit{dest} is the name of the main or child file to be processed
(all filenames without extensions).
The optional argument \textit{main} can be omitted
if \textit{main} matches \textit{dest}.
Optionally, compilation \textit{flags} can be defined via |\def| commands.
This command line makes the \TeX{} engine believe
it is compiling the file \textit{target}
whose content is specified as the latter parameter.
The provided code then forwards the processing to
\textit{main} or \textit{dest} as described in \secref{sec:forward}.

%%%%%%%%%%%%%%%%%%%%%%%%%%%%%%%%%%%%%%%%%%%%%%%%%%%%%%%%%%%%%%%%%%%%%%%%%%%%%%%%
\subsection{Include by Input}
\label{sec:input}

Including child documents by |\include| has some restrictions by design.
Most notably, the content of a child document always occupies
its own set of pages; pages cannot be shared between child documents.
Usually, this behaviour makes perfect sense
because each child document contain an essential part of the document.
However, in some situations it may be desirable to compose
a document from a collection of parts
without having mandatory page breaks between then.
For this case, the package
provides a mechanism to include parts
by |\input| which can also be processed individually.
However, by construction this mechanism
requires manual handling of the content to be output.

%%%%%%%%%%%%%%%%%%%%%%%%%%%%%%%%%%%%%%%%
\DescribeMacro{\ifchilddocmanual}
The main file should be prepared as usual, see \secref{sec:include}.
However, the document body must make a distinction
between processing of an individual part and of the main document, e.g.:
%
\begin{center}
\begin{tabular}{l}
|\ifchilddocmanual|\\
|\input{\childdocname}|\\
|\||else|\\
\textit{document body with }|\input{|\textit{part}|}|\\
|\||fi|
\end{tabular}
\end{center}
%
The conditional |\ifchilddocmanual| is true whenever
a part to be included by |\input| is being compiled,
and the name of the part is stored in |\childdocname|.

%%%%%%%%%%%%%%%%%%%%%%%%%%%%%%%%%%%%%%%%
\DescribeMacro{\childdocby}
Each part to be included by |\input| should start with:
%
\begin{center}
\begin{tabular}{l}
|\input{childdoc.def}|\\
|\childdocby{|\textit{main}|}|\\
\end{tabular}
\end{center}
%
The directive |\childdocby| is similar to |\childdocof|
described in \secref{sec:include},
but the subsequent selection of content must be done manually.
To that end, both |\ifchilddoc| and |\ifchilddocmanual|
will be true upon processing of a part,
and the name of the part is stored in |\childdocname|.
Note that |\jobname| will be set to the filename of the current part
so that each part receives an individual |.aux| file
that does not interfere with the |.aux| file(s) of the main document.
This behaviour can be altered by the alternative form
|\childdocby[*]{|\textit{main}|}| (with a non-empty optional argument)
which uses the |.aux| file of the main document
by setting |\jobname| to \textit{main}.

%%%%%%%%%%%%%%%%%%%%%%%%%%%%%%%%%%%%%%%%%%%%%%%%%%%%%%%%%%%%%%%%%%%%%%%%%%%%%%%%
\subsection{Driver Development}
\label{sec:driver}

The \textsf{childdoc} mechanism can also be use for the development
of definition files such as \LaTeX{} styles or classes.
This case differs from the above setup with multiple parts
included by |\include| in that no |\includeonly| should be invoked.
This can be achieved by starting the include file
(before |\ProvidesPackage|) with:
%
\begin{center}
\begin{tabular}{l}
|\input{childdoc.def}|\\
|\childdocforward{|\textit{main}|}|\\
\end{tabular}
\end{center}
%
or alternatively with:
%
\begin{center}
\begin{tabular}{l}
|\input{childdoc.def}|\\
|\childdocby{|\textit{main}|}|\\
\end{tabular}
\end{center}
%
Both forms have slightly different effects as described above.
The main file is prepared as usual, see \secref{sec:include}.

%%%%%%%%%%%%%%%%%%%%%%%%%%%%%%%%%%%%%%%%%%%%%%%%%%%%%%%%%%%%%%%%%%%%%%%%%%%%%%%%
\subsection{Legacy Detection}
\label{sec:detection}

The directive |\childdocmain| in the main file can detect
whether the complete document or merely a child is to be compiled
even without using the directive |\childdocof|.
This method is deprecated because it is less robust
and there is no compelling reason to use it;
it is merely provided for backward compatibility
and it may be removed in future versions.

If the detection mechanism is to be used,
it is mandatory to correctly specify
the filename of the main file as the argument of |\childdocmain|:
%
\begin{center}
\begin{tabular}{l}
|\input{childdoc.def}|\\
|\childdocmain{|\textit{main}|}|\\
\end{tabular}
\end{center}
%
If |\jobname| does not match the argument \textit{main} of |\childdocmain|,
it is assumed that |\jobname| points to the child file to be compiled.
When using |\childdocmain| with the main file specified as argument,
it suffices to start a child file
with just |\input{|\textit{main}|}|
without loading of the package and using |\childdocof|.
If instead all processing is done
with the appropriate \textsf{childdoc} directives,
the argument of \textit{main} of |\childdocmain| can be empty.

An alternative version of the command line processing described
in \secref{sec:commandline} using the detection mechanism reads:
%
\begin{center}
|... -jobname "|\textit{target}|" "|[\textit{flags}]%
[|\def\jobname{|\textit{dest}|}|]|\input{|\textit{main}|}"|
\end{center}

%%%%%%%%%%%%%%%%%%%%%%%%%%%%%%%%%%%%%%%%%%%%%%%%%%%%%%%%%%%%%%%%%%%%%%%%%%%%%%%%
\subsection{Manual Code}
\label{sec:manual}

In case one cannot be certain whether the definitions file |childdoc.def|
is installed on the target \TeX{} distribution
and one prefers not to ship it,
it is conceivable to paste a few relevant commands into the sources.

To that end, drop all statements |\input{childdoc.def}|
and perform the replacements as outlined below.
Instead of |\childdocmain{|\textit{main}|}| add the following code
to the top of the main file:
%
\begin{center}
\begin{tabular}{l}
|\||ifdefined\childdocname\endinput\||fi\newif\ifchilddoc|\\
|\edef\childdocname{\scantokens\expandafter{\jobname\noexpand}}|\\
|\def\childdocmain{|\textit{main}|}\||ifx\childdocmain\childdocname\||else|\\
|\childdoctrue\includeonly{\childdocname}\let\jobname\childdocmain\||fi|\\
\end{tabular}
\end{center}
%
Instead of |\childdocof{|\textit{main}|}| just include the main file
at the top of each child file:
%
\begin{center}
|\input{|\textit{main}|}|
\end{center}
%
A simple redirection |\childdocforward{|\textit{dest}|}| is achieved by:
%
\begin{center}
|\def\jobname{|\textit{dest}|}\input{\jobname}|
\end{center}
%
The redirection with prefix
|\childdocforwardprefix[|\textit{prefix}|]{|\textit{dest}|}|
is accomplished by:
%
\begin{center}
\begin{tabular}{l}
|{\edef\jobname{\scantokens\expandafter{\jobname\noexpand}}|\\
|\def\redirectjob |\textit{prefix}|#1~~~{\gdef\jobname{|\textit{dest}|#1}}|\\
|\expandafter\redirectjob\jobname~~~}\input{\jobname}|
\end{tabular}
\end{center}

In an alternative approach,
child documents can be compiled by a specific command line
without additional code or specific definitions:
%
\begin{center}
|... -jobname "|\textit{target}|" "|[\textit{flags}]%
|\includeonly{|\textit{dest}|}\input{|\textit{main}|}"|
\end{center}
%

%%%%%%%%%%%%%%%%%%%%%%%%%%%%%%%%%%%%%%%%%%%%%%%%%%%%%%%%%%%%%%%%%%%%%%%%%%%%%%%%
%%%%%%%%%%%%%%%%%%%%%%%%%%%%%%%%%%%%%%%%%%%%%%%%%%%%%%%%%%%%%%%%%%%%%%%%%%%%%%%%
\section{Information}

%%%%%%%%%%%%%%%%%%%%%%%%%%%%%%%%%%%%%%%%%%%%%%%%%%%%%%%%%%%%%%%%%%%%%%%%%%%%%%%%
\subsection{Copyright}

Copyright \copyright{} 2017--2018 Niklas Beisert

This work may be distributed and/or modified under the
conditions of the \LaTeX{} Project Public License, either version 1.3
of this license or (at your option) any later version.
The latest version of this license is in
  \url{http://www.latex-project.org/lppl.txt}
and version 1.3 or later is part of all distributions of \LaTeX{}
version 2005/12/01 or later.

This work has the LPPL maintenance status `maintained'.

The Current Maintainer of this work is Niklas Beisert.

This work consists of the files |README.txt|, |childdoc.ins| and |childdoc.dtx|
as well as the derived files |childdoc.def|, |cdocsamp.tex|
with |cdocsch1.tex|, |cdocsch2.tex|, |cdocspt3.tex|, |cdocspt4.tex|,
|cdocsdrf.tex|, |cdocsfn1.tex|, |cdocsfn2.tex|
as well as |childdoc.pdf|.

%%%%%%%%%%%%%%%%%%%%%%%%%%%%%%%%%%%%%%%%%%%%%%%%%%%%%%%%%%%%%%%%%%%%%%%%%%%%%%%%
\subsection{Files and Installation}

The package consists of the files:
%
\begin{center}
\begin{tabular}{ll}
    |README.txt|   & readme file \\
    |childdoc.ins| & installation file \\
    |childdoc.dtx| & source file \\
    |childdoc.def| & definition file \\
    |cdocsamp.tex| & sample main file \\
    |cdocsch1.tex| & sample include file \\
    |cdocsch2.tex| & sample include file \\
    |cdocspt3.tex| & sample part file \\
    |cdocspt4.tex| & sample part file \\
    |cdocsdrf.tex| & sample redirection file \\
    |cdocsfn1.tex| & sample redirection file \\
    |cdocsfn2.tex| & sample redirection file \\
    |childdoc.pdf| & manual
\end{tabular}
\end{center}
%
The distribution consists of the files
|README.txt|, |childdoc.ins| and |childdoc.dtx|.
%
\begin{itemize}
\item
Run (pdf)\LaTeX{} on |childdoc.dtx|
to compile the manual |childdoc.pdf| (this file).
\item
Run \LaTeX{} on |childdoc.ins| to create the definitions file |childdoc.def|
and the sample |cdocsamp.tex| with include files
|cdocsch1.tex|, |cdocsch2.tex|, |cdocspt3.tex|, |cdocspt4.tex|,
|cdocsdrf.tex|, |cdocsfn1.tex|, |cdocsfn2.tex|.
Then copy the file |childdoc.def| to an appropriate directory of your \LaTeX{}
distribution, e.g.\ \textit{texmf-root}|/tex/latex/childdoc|.
\end{itemize}

%%%%%%%%%%%%%%%%%%%%%%%%%%%%%%%%%%%%%%%%%%%%%%%%%%%%%%%%%%%%%%%%%%%%%%%%%%%%%%%%
\subsection{Related CTAN Packages}

There are several other packages which offer a similar functionality:
%
\begin{itemize}
\item
The packages
\href{http://ctan.org/pkg/docmute}{\textsf{docmute}},
\href{http://ctan.org/pkg/includex}{\textsf{includex}} and
\href{http://ctan.org/pkg/standalone}{\textsf{standalone}}
provide commands to include only the document body of
a child file thus allowing both files to be compiled individually.
\item
The packages \href{http://ctan.org/pkg/subdocs}{\textsf{subdocs}}
and \href{http://ctan.org/pkg/subfiles}{\textsf{subfiles}}
provide structures in which the main and child documents can be
encapsulated and allowing them to be compiled individually.
The inclusion mechanism is different from the conventional |\include|.
\item
The package \href{http://ctan.org/pkg/combine}{\textsf{combine}}
is an elaborate solution to combine several documents into one.
\end{itemize}
%
See also the CTAN topic \href{http://ctan.org/topic/subdocs}{\textsf{subdocs}}
for further related packages.
The present package differs from the above solutions in that
a document structure constructed with the conventional |\include| mechanism
just needs two extra commands at the top of every file
such that all constituent files can be compiled individually.

%%%%%%%%%%%%%%%%%%%%%%%%%%%%%%%%%%%%%%%%%%%%%%%%%%%%%%%%%%%%%%%%%%%%%%%%%%%%%%%%
%\subsection{Feature Suggestions}
%
%The following is a list of features which may be useful for future
%versions of this package:
%%
%\begin{itemize}
%\item
%\ldots
%\end{itemize}

%%%%%%%%%%%%%%%%%%%%%%%%%%%%%%%%%%%%%%%%%%%%%%%%%%%%%%%%%%%%%%%%%%%%%%%%%%%%%%%%
\subsection{Revision History}

%%%%%%%%%%%%%%%%%%%%%%%%%%%%%%%%%%%%%%%%
\paragraph{v2.0:} 2018/12/30

\begin{itemize}
\item
immediate forward processing
\item
added |\childdocby| mechanism
\item
manual restructured
\end{itemize}

%%%%%%%%%%%%%%%%%%%%%%%%%%%%%%%%%%%%%%%%
\paragraph{v1.6:} 2018/01/17

\begin{itemize}
\item
application for development of include files
\item
corrections to manual
\end{itemize}

%%%%%%%%%%%%%%%%%%%%%%%%%%%%%%%%%%%%%%%%
\paragraph{v1.5:} 2017/05/21

\begin{itemize}
\item
more complete structuring introduced
\item
|\childdocof| introduced
\item
|\childdoc| renamed to |\childdocmain|
\item
|\childredirect| renamed to |\childdocforward| and |\childdocforwardprefix|
and functionality expanded
\end{itemize}

%%%%%%%%%%%%%%%%%%%%%%%%%%%%%%%%%%%%%%%%
\paragraph{v1.0:} 2017/04/27

\begin{itemize}
\item
manual and install package
\item
first version published on CTAN
\end{itemize}

%%%%%%%%%%%%%%%%%%%%%%%%%%%%%%%%%%%%%%%%
\paragraph{v0.6:} 2017/04/26

\begin{itemize}
\item
redirection mechanism added
\end{itemize}

%%%%%%%%%%%%%%%%%%%%%%%%%%%%%%%%%%%%%%%%
\paragraph{v0.5:} 2017/04/26

\begin{itemize}
\item
functionality in definition file
\end{itemize}


%%%%%%%%%%%%%%%%%%%%%%%%%%%%%%%%%%%%%%%%%%%%%%%%%%%%%%%%%%%%%%%%%%%%%%%%%%%%%%%%
%%%%%%%%%%%%%%%%%%%%%%%%%%%%%%%%%%%%%%%%%%%%%%%%%%%%%%%%%%%%%%%%%%%%%%%%%%%%%%%%
%%%%%%%%%%%%%%%%%%%%%%%%%%%%%%%%%%%%%%%%%%%%%%%%%%%%%%%%%%%%%%%%%%%%%%%%%%%%%%%%
\appendix

\settowidth\MacroIndent{\rmfamily\scriptsize 000\ }

 \DocInput{childdoc.dtx}

\end{document}
%</driver>
% \fi
%
% %%%%%%%%%%%%%%%%%%%%%%%%%%%%%%%%%%%%%%%%%%%%%%%%%%%%%%%%%%%%%%%%%%%%%%%%%%%%%%
% %%%%%%%%%%%%%%%%%%%%%%%%%%%%%%%%%%%%%%%%%%%%%%%%%%%%%%%%%%%%%%%%%%%%%%%%%%%%%%
% \section{Sample}
%\iffalse
%<*samplemain>
%\fi
%
% The following presents a sample document
% with two chapters, two parts, a title page,
% a compile flag as well as three forwarding files to set the flag.
% It consists of eight |.tex| files:
% \begin{center}
% \begin{tabular}{ll}
% |cdocsamp.tex|&main file\\
% |cdocsch1.tex|&include file for chapter 1\\
% |cdocsch2.tex|&include file for chapter 2\\
% |cdocspt3.tex|&include file for part 3\\
% |cdocspt4.tex|&include file for part 4\\
% |cdocsdrf.tex|&forwarding file for main file in draft mode\\
% |cdocsfi1.tex|&forwarding file for final version of chapter 1\\
% |cdocsfi2.tex|&forwarding file for final version of chapter 2\\
% \end{tabular}
% \end{center}
% Each of the eight files can be compiled directly by the \LaTeX{} compiler.
%
% %%%%%%%%%%%%%%%%%%%%%%%%%%%%%%%%%%%%%%
% \paragraph{Main File.}
%
% The main file is called |cdocsamp.tex|.
%
% Load the \textsf{childdoc} definitions and
% declare the filename for the main document:
%    \begin{macrocode}
\input{childdoc.def}
\childdocmain{}
%    \end{macrocode}

% Optional override for |\version| flag:
%    \begin{macrocode}
%%\ifchilddoc\else\providecommand{\version}{draft}\fi
%    \end{macrocode}

% Define the default values for the |\version| flag
% (|final| for the main file and |draft| for childs):
%    \begin{macrocode}
\ifchilddoc
\providecommand{\version}{draft}
\else
\providecommand{\version}{final}
\fi
%    \end{macrocode}

% Load the standard document class:
%    \begin{macrocode}
\documentclass[12pt]{article}
%    \end{macrocode}

% Start the document body:
%    \begin{macrocode}
\begin{document}
%    \end{macrocode}

% Declare a title page.
% Print title, part of document being processed and version flag:
%    \begin{macrocode}
\addtocounter{page}{-1}
\begin{center}
{\LARGE\bfseries{}childdoc example\par}
\vspace{1cm}
\ifchilddoc
\ifchilddocmanual part\else chapter\fi:
`\childdocname' of `\childdocjob'\par
\else
main document: `\childdocjob'\par
\fi
version: \version\par
\end{center}
\newpage
%    \end{macrocode}

% Manually include selected file,
% otherwise process as usual:
%    \begin{macrocode}
\ifchilddocmanual
\section*{part `\childdocname'}
\input{\childdocname}
\else
%    \end{macrocode}

% Include the two chapters:
%    \begin{macrocode}
\include{cdocsch1}
\include{cdocsch2}
%    \end{macrocode}

% Include the two parts unless only chapters should be displayed:
%    \begin{macrocode}
\ifchilddoc\else
\section{part three}
\input{cdocspt3}
\section{part four}
\input{cdocspt4}
\fi
%    \end{macrocode}

% Process as usual until here:
%    \begin{macrocode}
\fi
%    \end{macrocode}

% End of document body:
%    \begin{macrocode}
\end{document}
%    \end{macrocode}
%\iffalse
%</samplemain>
%\fi
%
% %%%%%%%%%%%%%%%%%%%%%%%%%%%%%%%%%%%%%%
% \paragraph{Chapter Include Files.}
%
% The include files are called |cdocsch1.tex| and |cdocsch2.tex|.
%
%\iffalse
%<*samplechap1|samplechap2>
%\fi

% Optional override for |\version| flag:
%    \begin{macrocode}
%%\providecommand{\version}{final}
%    \end{macrocode}

% Include the main document:
%    \begin{macrocode}
\input{childdoc.def}
\childdocof{cdocsamp}
%    \end{macrocode}

%\iffalse
%</samplechap1|samplechap2>
%\fi
%
%\iffalse
%<*samplechap1>
%\fi
% Some text for chapter 1:
%    \begin{macrocode}
\section{one}
some text in chapter one
%    \end{macrocode}

%\iffalse
%</samplechap1>
%\fi
% Some text for chapter 2:
%\iffalse
%<*samplechap2>
%\fi
%    \begin{macrocode}
\section{two}
more text in chapter two
%    \end{macrocode}

%\iffalse
%</samplechap2>
%\fi
%
% %%%%%%%%%%%%%%%%%%%%%%%%%%%%%%%%%%%%%%
% \paragraph{Part Include Files.}
%
% The include files are called |cdocspt3.tex| and |cdocspt4.tex|.
%
%\iffalse
%<*samplepart3|samplepart4>
%\fi

% Optional override for |\version| flag:
%    \begin{macrocode}
%%\providecommand{\version}{final}
%    \end{macrocode}

% Include the main document:
%    \begin{macrocode}
\input{childdoc.def}
\childdocby{cdocsamp}
%    \end{macrocode}

%\iffalse
%</samplepart3|samplepart4>
%\fi
%
%\iffalse
%<*samplepart3>
%\fi
% Some text for part 3:
%    \begin{macrocode}
some text in part three
%    \end{macrocode}

%\iffalse
%</samplepart3>
%\fi
% Some text for part 4:
%\iffalse
%<*samplepart4>
%\fi
%    \begin{macrocode}
more text in part four
%    \end{macrocode}

%\iffalse
%</samplepart4>
%\fi
%
% %%%%%%%%%%%%%%%%%%%%%%%%%%%%%%%%%%%%%%
% \paragraph{Forwarding for a Complete Draft.}
%
% The following forwarding file |cdocsdrf.tex|
% compiles the main document in draft mode:
%\iffalse
%<*sampledraft>
%\fi
%    \begin{macrocode}
\def\version{draft}
\input{childdoc.def}
\childdocforward{cdocsamp}
%    \end{macrocode}

%\iffalse
%</sampledraft>
%\fi
%
% %%%%%%%%%%%%%%%%%%%%%%%%%%%%%%%%%%%%%%
% \paragraph{Forwarding for Final Version of the Chapters.}
%
% The following forwarding files |cdocsfn1.tex| and |cdocsfn2.tex|
% (with identical content)
% compile the final versions of the child documents
% |cdocsch1.tex| and |cdocsch2.tex|, respectively:
%\iffalse
%<*samplefinal>
%\fi
%    \begin{macrocode}
\def\version{final}
\input{childdoc.def}
\childdocforwardprefix[cdocsamp]{cdocsfn}{cdocsch}
%    \end{macrocode}

%\iffalse
%</samplefinal>
%\fi
%
% %%%%%%%%%%%%%%%%%%%%%%%%%%%%%%%%%%%%%%
% \paragraph{Command Line Processing.}
%
% The following three command lines generate the output files
% |cdocscld|, |cdocscl1| and |cdocscl2|
% which should be identical to
% |cdocsdrf|, |cdocsch1| and |cdocsfn2|, respectively:
% \begin{center}
% \begin{tabular}{l}
% |latex -jobname cdocscld \|\\
% |  "\def\version{draft}\input{childdoc.def}\childdocforward{cdocsamp}"|\\
% |latex -jobname cdocscl1 \|\\
% |  "\input{childdoc.def}\childdocforward[cdocsamp]{cdocsch1}"|\\
% |latex -jobname cdocscl2 \|\\
% |  "\def\version{final}\input{childdoc.def}\childdocforward{cdocsch2}"|
% \end{tabular}
% \end{center}
% Note that the trailing backslash on each first line
% merely continues the input to the second line
% (for convenient cut ant paste).
% Furthermore, the command |latex| can be replaced by any
% of its alternative versions such as |pdflatex|.
%
% %%%%%%%%%%%%%%%%%%%%%%%%%%%%%%%%%%%%%%%%%%%%%%%%%%%%%%%%%%%%%%%%%%%%%%%%%%%%%%
% %%%%%%%%%%%%%%%%%%%%%%%%%%%%%%%%%%%%%%%%%%%%%%%%%%%%%%%%%%%%%%%%%%%%%%%%%%%%%%
% \section{Implementation}
%\iffalse
%<*package>
%\fi
%
% This section describes the definitions file |childdoc.def|.

% The definitions cannot be loaded using |\usepackage| or |\RequirePackage|
% which has a mechanism to prevent loading a style file more than once.
% When loading the definitions by means of |\input|
% multiple instances have to be prevented manually:
%\iffalse
%This code needs to be before the `\ProvidesFile' directive
%which is defined at the beginning of this file.
%Therefore it is also placed there and commented out here.
%</package>
%<*discard>
%\fi
%    \begin{macrocode}
\ifdefined\childdocmain\endinput\fi
%    \end{macrocode}
%\iffalse
%</discard>
%<*package>
%\fi
%
% \macro{\ifchilddoc}
% \macro{\ifchilddocmanual}
% The conditional |\ifchilddoc| tells whether a
% child (true) or main (false) document is being compiled.
% The conditional |\ifchilddocmanual| tells whether
% the |\includeonly| mechanism is used (false) or
% the selection of child files must be performed manually (true).
% The definitions initialise to false:
%    \begin{macrocode}
\newif\ifchilddoc
\newif\ifchilddocmanual
%    \end{macrocode}

% \macro{\childdocname}
% \macro{\childdocjob}
% The macro |\childdocname| stores the name of the main document
% to be compiled. The macro |\childdocjob| stores the name of
% the document on which the \LaTeX{} compiler was originally invoked.
% The content of |\jobname| cannot be compared
% to filenames specified in the source due to different catcodes.
% The following code rescans |\jobname|, stores the result
% in |\childdocname| and saves a copy in |\childdocjob|:
%    \begin{macrocode}
\edef\childdocname{\scantokens\expandafter{\jobname\noexpand}}
\let\childdocjob\childdocname
%    \end{macrocode}

% \macro{\childdocdisable}
% The macro |\childdocdisable| prevents the main file
% from being processed more than once.
% At this stage, the main document command |\childdocmain|
% is assumed to be called once again where it should do nothing.
% Any subsequent call to it should prevent
% a secondary processing of the main document
% It overwrites the forwarding commands
% |\childdocof| and |\childdocforward|
% with empty macros to prevent further inclusions of the main document:
%    \begin{macrocode}
\newcommand{\childdocdisable}
{
  \renewcommand{\childdocmain}[1]{\renewcommand{\childdocmain}[1]{\endinput}}
  \renewcommand{\childdocof}[1]{}
  \renewcommand{\childdocby}[2][]{}
  \renewcommand{\childdocforward}[2][]{}
  \renewcommand{\childdocdisable}{}
}
%    \end{macrocode}

% \macro{\childdocmain}
% The macro |\childdocmain| is to be called at the top of the main file
% with nothing or the main filename (without extension) as argument.
% First, it breaks loops.
% If the argument is not empty and does not match |\childdocname|
% (which is set by the first inclusion of |childdoc.def|),
% |\ifchilddoc| is set to true, |\includeonly| is applied to the child file
% and |\jobname| is set to the main file
% (for proper handling of |.aux| files):
%    \begin{macrocode}
\newcommand{\childdocmain}[1]
{
  \childdocdisable\childdocmain{}
  \if?#1?\else
    \begingroup
      \def\childdoctmp{#1}
      \ifx\childdoctmp\childdocname
        \def\childdoctmp{}
      \else
        \def\childdoctmp
        {
          \childdoctrue
          \includeonly{\childdocname}
          \def\childdocjob{#1}
          \def\jobname{#1}
        }
      \fi
      \expandafter
    \endgroup
    \childdoctmp
  \fi
}
%    \end{macrocode}

% \macro{\childdocof}
% The command |\childdocof| redirects
% compilation to the main file |#1|.
%    \begin{macrocode}
\newcommand{\childdocof}[1]
{
  \childdocdisable
  \childdoctrue
  \includeonly{\childdocname}
  \def\jobname{#1}
  \def\childdocjob{#1}
  \input{#1}
}
%    \end{macrocode}

% \macro{\childdocby}
% The command |\childdocby| ....
%    \begin{macrocode}
\newcommand{\childdocby}[2][]
{
  \childdocdisable
  \childdoctrue
  \childdocmanualtrue
  \if?#1?\else
    \def\jobname{#2}
  \fi
  \def\childdocjob{#2}
  \input{#2}
  \endinput
}
%    \end{macrocode}

% \macro{\childdocforward}
% The command |\childdocforward| redirects
% compilation to the main file or
% (if the optional argument is given) a child file.
% Parameters are set as if the main file
% or a child file starting with |\childdocof| was compiled.
% Then compilation is handed over to the main file:
%    \begin{macrocode}
\newcommand{\childdocforward}[2][]
{
  \begingroup
    \if?#1?
      \def\childdoctmp
      {
        \def\childdocname{#2}
        \def\childdocjob{#2}
        \def\jobname{#2}
        \input{#2}
        \endinput
      }
    \else
      \def\childdoctmp
      {
        \childdocdisable
        \def\childdocname{#2}
        \childdoctrue
        \includeonly{#2}
        \def\childdocjob{#1}
        \def\jobname{#1}
        \input{#1}
        \endinput
      }
    \fi
    \expandafter
  \endgroup
  \childdoctmp
}
%    \end{macrocode}

% \macro{\childdocforwardprefix}
% The command |\childdocforwardprefix| redirects
% compilation to the main or a child file by means of a pattern.
% The prefix |#1| in the current filename is replaced by |#2|
% and the suffix of the current filename is kept
% (it is assumed that the filename does not contain the substring `|~~~|'
% which is used as a delimiter).
% Compilation is handed over to the new file by |\childdocforward|:
%    \begin{macrocode}
\newcommand{\childdocforwardprefix}[3][]
{
  \begingroup
    \def\childdocextract #2##1~~~{\def\childdoctmp{\childdocforward[#1]{#3##1}}}
    \expandafter\childdocextract\childdocname~~~
    \expandafter
  \endgroup
  \childdoctmp
}
%    \end{macrocode}

% \macro{\childdoc}
% The deprecated macro |\childdoc| is a legacy version of |\childdocmain|:
%    \begin{macrocode}
\newcommand{\childdoc}{\childdocmain}
%    \end{macrocode}

% \macro{\childdocredirect}
% The deprecated macro |\childdocredirect| is a legacy version
% of |\childdocforward| and |\childdocforwardprefix|:
%    \begin{macrocode}
\newcommand{\childdocredirect}[2][]
{
  \begingroup
    \if?#1?
      \def\childdoctmp{\childdocforward{#2}}
    \else
      \def\childdoctmp{\childdocforwardprefix{#1}{#2}}
    \fi
    \expandafter
  \endgroup
  \childdoctmp
}
%    \end{macrocode}

%\iffalse
%</package>
%\fi
%
\endinput
|\\
|\childdocmain{}|\\
\end{tabular}
\end{center}
at the very top of the main \LaTeX{} file,
in particular \emph{before} the |\documentclass| statement!
The argument of |\childdocmain| should be left empty
(but it must be present).

%%%%%%%%%%%%%%%%%%%%%%%%%%%%%%%%%%%%%%%%
\DescribeMacro{\childdocof}
Furthermore, add the commands
\begin{center}
\begin{tabular}{l}
|% \iffalse
%
% childdoc.dtx Copyright (C) 2017-2018 Niklas Beisert
%
% This work may be distributed and/or modified under the
% conditions of the LaTeX Project Public License, either version 1.3
% of this license or (at your option) any later version.
% The latest version of this license is in
%   http://www.latex-project.org/lppl.txt
% and version 1.3 or later is part of all distributions of LaTeX
% version 2005/12/01 or later.
%
% This work has the LPPL maintenance status `maintained'.
%
% The Current Maintainer of this work is Niklas Beisert.
%
% This work consists of the files childdoc.dtx and childdoc.ins
% and the derived files childdoc.def and cdocsamp.tex with
% cdocsch1.tex, cdocsch2.tex, cdocsdrf.tex, cdocsfn1.tex, cdocsfn2.tex.
%
%<package>\ifdefined\childdocmain\endinput\fi
%<package>\ProvidesFile{childdoc.def}[2018/12/30 v2.0 child document driver]
%<samplemain>\ProvidesFile{cdocsamp.tex}[2018/12/30 v2.0 sample for childdoc]
%<*driver>
%\ProvidesFile{childdoc.drv}[2018/12/30 v2.0 childdoc reference manual file]
\PassOptionsToClass{10pt,a4paper}{article}
\documentclass{ltxdoc}

\usepackage[margin=35mm]{geometry}
\usepackage{hyperref}
\usepackage{hyperxmp}
\usepackage[usenames]{color}

\hypersetup{colorlinks=true}
\hypersetup{pdfstartview=FitH}
\hypersetup{pdfpagemode=UseNone}
\hypersetup{pdfsource={}}
\hypersetup{pdflang={en-UK}}
\hypersetup{pdfcopyright={Copyright 2017-2018 Niklas Beisert.
  This work may be distributed and/or modified under the
  conditions of the LaTeX Project Public License, either version 1.3
  of this license or (at your option) any later version.}}
\hypersetup{pdflicenseurl={http://www.latex-project.org/lppl.txt}}
\hypersetup{pdfcontactaddress={ETH Zurich, ITP, HIT K,
  Wolfgang-Pauli-Strasse 27}}
\hypersetup{pdfcontactpostcode={8093}}
\hypersetup{pdfcontactcity={Zurich}}
\hypersetup{pdfcontactcountry={Switzerland}}
\hypersetup{pdfcontactemail={nbeisert@itp.phys.ethz.ch}}
\hypersetup{pdfcontacturl={http://people.phys.ethz.ch/\xmptilde nbeisert/}}

\newcommand{\secref}[1]{\hyperref[#1]{section \ref*{#1}}}

\parskip1ex
\parindent0pt
\let\olditemize\itemize
\def\itemize{\olditemize\parskip0pt}

\begin{document}

\title{The \textsf{childdoc} Package}
\hypersetup{pdftitle={The childdoc Package}}
\author{Niklas Beisert\\[2ex]
  Institut f\"ur Theoretische Physik\\
  Eidgen\"ossische Technische Hochschule Z\"urich\\
  Wolfgang-Pauli-Strasse 27, 8093 Z\"urich, Switzerland\\[1ex]
  \href{mailto:nbeisert@itp.phys.ethz.ch}
  {\texttt{nbeisert@itp.phys.ethz.ch}}}
\hypersetup{pdfauthor={Niklas Beisert}}
\hypersetup{pdfsubject={Manual for the LaTeX2e Package childdoc}}
\date{30 December 2018, \textsf{v2.0}}
\maketitle

\begin{abstract}\noindent
\textsf{childdoc} is a \LaTeXe{} package
that enables the direct compilation
of document sections included by |\include|
to individual files.
\end{abstract}

\begingroup
\parskip0ex
\tableofcontents
\endgroup

%%%%%%%%%%%%%%%%%%%%%%%%%%%%%%%%%%%%%%%%%%%%%%%%%%%%%%%%%%%%%%%%%%%%%%%%%%%%%%%%
%%%%%%%%%%%%%%%%%%%%%%%%%%%%%%%%%%%%%%%%%%%%%%%%%%%%%%%%%%%%%%%%%%%%%%%%%%%%%%%%
\section{Introduction}

\LaTeX{} provides a mechanism to structure a large document (such as a book)
into a main file and several child files (containing the chapters)
using the |\include| command.
This mechanism is beneficial for documents
which span hundreds of pages in order to
make the source file(s) more manageable.
Moreover, compilation can be restricted to
selected child files by means of the |\includeonly| command.
The latter feature can be used to reduce the compilation time while editing
(this was significantly more useful in the earlier days of \LaTeX{})
or to generate a smaller document which is easier to navigate.
Another application of |\includeonly| is to generate
documents consisting of selected parts of the complete document.

However, there are a few drawbacks of the plain |\include| mechanism:
\begin{itemize}
\item
The child files cannot be compiled on their own,
they can only be compiled via the main file.
A naive editing environment
(such as a text editor with an option
to have the current file processed by \LaTeX)
may require one to switch to the main file before compiling;
attempting to compile the child file produces errors.
\item
The main file must be modified (each time)
to adjust the |\includeonly| command
to the present needs. This easily leaves the main file in a messy state.
\item
The generated document will always carry the filename
of the main document. This is inconvenient if
several child files are to be compiled and
to be kept for distribution.
\end{itemize}

The present package provides a simple interface
to make child files individually compilable by \LaTeX{}.
Compiling a child file then has the same effect as compiling
the main file with an |\includeonly| command
to select the appropriate child.
Moreover the generated document will carry the name of the child
rather than the main file.
This resolves all three above issues.

This feature is meant to make the editing of books,
thesis documents and lecture notes somewhat more convenient.
However, the package can also be used efficiently for
composing a series of documents (such as exercise sheets)
which are typically distributed individually.
It then assists the author in generating the individual documents
(potentially in different versions)
as well as a document containing the collected series.
Another application is in developing style files
or other kinds of included material
where compilation of the style file could redirect
to a sample or test file.

%%%%%%%%%%%%%%%%%%%%%%%%%%%%%%%%%%%%%%%%%%%%%%%%%%%%%%%%%%%%%%%%%%%%%%%%%%%%%%%%
%%%%%%%%%%%%%%%%%%%%%%%%%%%%%%%%%%%%%%%%%%%%%%%%%%%%%%%%%%%%%%%%%%%%%%%%%%%%%%%%
\section{Usage}

First of all, the package \textsf{childdoc} is \emph{not} a standard
\LaTeXe{} |.sty| style file! Therefore it needs to be invoked in
a non-standard way.

%%%%%%%%%%%%%%%%%%%%%%%%%%%%%%%%%%%%%%%%%%%%%%%%%%%%%%%%%%%%%%%%%%%%%%%%%%%%%%%%
\subsection{Included Files}
\label{sec:include}

%%%%%%%%%%%%%%%%%%%%%%%%%%%%%%%%%%%%%%%%
\DescribeMacro{\childdocmain}
To use the package, add the commands
\begin{center}
\begin{tabular}{l}
|\input{childdoc.def}|\\
|\childdocmain{}|\\
\end{tabular}
\end{center}
at the very top of the main \LaTeX{} file,
in particular \emph{before} the |\documentclass| statement!
The argument of |\childdocmain| should be left empty
(but it must be present).

%%%%%%%%%%%%%%%%%%%%%%%%%%%%%%%%%%%%%%%%
\DescribeMacro{\childdocof}
Furthermore, add the commands
\begin{center}
\begin{tabular}{l}
|\input{childdoc.def}|\\
|\childdocof{|\textit{main}|}|\\
\end{tabular}
\end{center}
at the top of every child file \textit{child}
which is included by |\include{|\textit{child}|}|
from within the main file
(or at least for those files to be compiled individually).
The argument \textit{main} must be the filename of the main file.

There are a couple of
considerations in setting up the main and child documents:

%%%%%%%%%%%%%%%%%%%%%%%%%%%%%%%%%%%%%%%%
\paragraph{Restrictions.}

Please note the following restrictions:
\begin{itemize}
\item
|\childdocmain| must be called with one argument \textit{main}
to ensure compatibility with earlier version of the package.
It must either be empty (|\childdocmain{}|)
or precisely match the filename of the main file in which it is specified.
See \secref{sec:detection} for further information.
\item
The filename \textit{main} must be specified without the |.tex| extension.
\item
The filename \textit{main} is case sensitive
(even in case-insensitive file systems)
due to internal string comparison.
\item
The argument \textit{main} should be fully expanded, it cannot be a macro.
\item
Subdirectories and special characters should be avoided in filenames.
\item
The command |\childdocmain{|\textit{main}|}| must be followed by a whitespace.
It should not be followed immediately by another command
or by a comment mark `|%|'.
This is because the \TeX{} parser reads the token immediately following
the argument of |\childdocmain| and puts it
at the beginning of every child section;
however, a white\-space is ignored.
\end{itemize}

%%%%%%%%%%%%%%%%%%%%%%%%%%%%%%%%%%%%%%%%
\paragraph{Content of Main File.}

It is advisable to place all content in the child files included by |\include|.
Any output contained in the main file will appear in all child documents
unless suppressed manually;
it cannot be suppressed automatically by the |\includeonly| directive
and thus should normally be avoided.
A method to include some content in the main file
by means of conditional processing is described in \secref{sec:conditional}.

%%%%%%%%%%%%%%%%%%%%%%%%%%%%%%%%%%%%%%%%
\paragraph{Page Numbering.}

When only a part of the document is compiled,
the appropriate numbering of pages
(as well as other status parameters)
is determined from the |.aux| files.
The latter contain information from previous passes.
However this information needs to propagate through
all intermediate child documents.
Therefore the page numbering in child documents may well
be inconsistent until the complete document is compiled at least once.

A useful (if unconventional) way to always ensure a consistent
page numbering is to restart the numbering in each child document
and denote the pages by `\textit{child}|.|\textit{page}'
where \textit{child} represents the chapter/section number of the child file.
This can be achieved by the command
|\numberwithin{page}{|\textit{child}|}|
of the \textsf{amsmath} package
where \textit{child} can be |chapter| or |section|
depending on the chosen structuring.
Alternatively, one can modify the macro |\thepage| appropriately
and reset the counter |page| at the start of each child file.

%%%%%%%%%%%%%%%%%%%%%%%%%%%%%%%%%%%%%%%%%%%%%%%%%%%%%%%%%%%%%%%%%%%%%%%%%%%%%%%%
\subsection{Conditional Processing}
\label{sec:conditional}

The package provides a mechanism to compile different versions
of a document. To customise the versions further some conditional processing
can come in handy to distinguish which version is being compiled.
The package provides two macros to describe the compilation context:

%%%%%%%%%%%%%%%%%%%%%%%%%%%%%%%%%%%%%%%%
\DescribeMacro{\ifchilddoc}
The conditional |\ifchilddoc| distinguishes between the compilation of
child documents and the main document:
%
\begin{center}
|\ifchilddoc |\textit{child-code}| |[|\||else |\textit{main-code}]| \||fi|
\end{center}

%%%%%%%%%%%%%%%%%%%%%%%%%%%%%%%%%%%%%%%%
\DescribeMacro{\childdocname}
\DescribeMacro{\childdocjob}
The macro |\childdocname| contains the filename (without extension)
of the main or child file being processed.
Note that |\childdocjob| will always contain the name of the main file.

%%%%%%%%%%%%%%%%%%%%%%%%%%%%%%%%%%%%%%%%
\paragraph{Title Page.}

Conditional processing can be used to include a title or banner page
in the main document when proper precautions are taken.
Importantly, the code in the main file should ensure that the page counter
(as well as other status parameters which are stored in the |.aux| files)
takes the same value after the conditional processing.
Otherwise the page numbers may take divergent values
depending on which part is compiled.

For example, a title page could be declared by:
%
\begin{center}
\begin{tabular}{l}
|\ifchilddoc\||else|\\
|\addtocounter{page}{-1}|\\
\textit{code for title page}\\
|\newpage|\\
|\||fi|
\end{tabular}
\end{center}
%
A banner page for the child documents can be generated by:
%
\begin{center}
\begin{tabular}{l}
|\ifchilddoc|\\
|\addtocounter{page}{-1}|\\
\textit{code for banner page}\\
|\newpage|\\
|\||fi|
\end{tabular}
\end{center}
%
Here one could write a message such as:
\begin{center}
|This is the part \childdocname{} of \childdocjob{}.|
\end{center}

%%%%%%%%%%%%%%%%%%%%%%%%%%%%%%%%%%%%%%%%%%%%%%%%%%%%%%%%%%%%%%%%%%%%%%%%%%%%%%%%
\subsection{Flags}
\label{sec:flags}

The package makes it easy to generate different versions
of the main or child documents.
To this end compilation flags can be defined
and assigned different default values.
They will be particularly useful in conjunction
with the forwarding mechanism described in \secref{sec:forward}.

For example, it may be useful to have a flag |\version|
which can be set to |draft| or |final|.
The document source will contain some conditional code
depending on the value of |\version|.
Suppose further, the flag should default to |final| for the main file
and to |draft| for child files
which is a natural assignment for editing the document.
This is achieved by placing the following code
in the preamble of the main document
(below the |\childdocmain| directive):
%
\begin{center}
\begin{tabular}{l}
|\ifchilddoc|\\
|\providecommand{\version}{draft}|\\
|\||else|\\
|\providecommand{\version}{final}|\\
|\||fi|
\end{tabular}
\end{center}
%
The definition by |\providecommand| makes sure
that previous definitions are not overwritten.
Further statements |\providecommand{\version}{...}|
can thus be added before the above code to override it.

For the main file, one might add a line
(between |\childdocmain| and the above block)
%
\begin{center}
|%\ifchilddoc\||else\providecommand{\version}{draft}\||fi|
\end{center}
%
which can be uncommented to produce a draft version.
Likewise one can add a line to the very top of a child file
(above the |\childdocof{|\textit{main}|}| directive)
%
\begin{center}
|%\providecommand{\version}{final}|
\end{center}
%
which can be uncommented to produce the final version of this child document.

%%%%%%%%%%%%%%%%%%%%%%%%%%%%%%%%%%%%%%%%%%%%%%%%%%%%%%%%%%%%%%%%%%%%%%%%%%%%%%%%
\subsection{Forwarding}
\label{sec:forward}

Different versions of the main or child documents
using compilation flags as described in \secref{sec:flags}
can be (permanently) stored in different files
for convenient compilation, viewing and distribution.
To this end, the package defines a command
to pass on compilation to a different file:

%%%%%%%%%%%%%%%%%%%%%%%%%%%%%%%%%%%%%%%%
\DescribeMacro{\childdocforward}
The command |\childdocforward| redirects processing to
another source file:
%
\begin{center}
\begin{tabular}{l}
|\input{childdoc.def}|\\
|\childdocforward[|\textit{main}|]{|\textit{dest}|}|\\
\end{tabular}
\end{center}
%
The argument \textit{dest} is the destination file
(without extension).
It should be the main file or one of the child files.
Note that further \textsf{childdoc} directives
such as |\childdocof| and |\childdocforward|
in the indicated file will be processed in this form.
The optional argument \textit{main}
passes on directly to the main file \textit{main}
while pretending to compile the child \textit{dest}.
This form behaves as if \textit{dest}
issues |\childdocof{|\textit{main}|}| right away,
and no further \textsf{childdoc} directives will be processed.

%%%%%%%%%%%%%%%%%%%%%%%%%%%%%%%%%%%%%%%%
\DescribeMacro{\...prefix}
In the alternative form |\childdocforwardprefix|,
%
\begin{center}
\begin{tabular}{l}
|\input{childdoc.def}|\\
|\childdocforwardprefix[|\textit{main}|]{|\textit{prefix}|}{|\textit{dest}|}|
\end{tabular}
\end{center}
%
the destination file is determined by a pattern
depending on the current file:
To make this work, the current file must be called
`{\textit{prefix}\hspace{0.2em}\textit{suffix}}'
with \textit{prefix} matching precisely the argument.
Processing is then passed on to the file
`{\textit{dest}\hspace{0.2em}\textit{suffix}}'.
Surely, the same effect is achieved by
directly specifying the
argument `{\textit{dest}\hspace{0.2em}\textit{suffix}}'
in the first form.
However, that requires to set up a different file
for each child. With the alternative form of the command
all these files can have exactly the same content
which simplifies setting them up and maintaining them.

For example, the following file |draft.tex|
with a compilation flag |\version| as described in \secref{sec:flags}
compiles the main document as a draft:
%
\begin{center}
\begin{tabular}{l}
|\def\version{draft}|\\
|\input{childdoc.def}|\\
|\childdocforward{|\textit{main}|}|
\end{tabular}
\end{center}
%
Likewise, the following files |final|\textit{nn}|.tex|
compile the final version of the child document
|child|\textit{nn}|.tex|:
%
\begin{center}
\begin{tabular}{l}
|\def\version{final}|\\
|\input{childdoc.def}|\\
|\childdocforwardprefix{final}{child}|
\end{tabular}
\end{center}
%

Note that when several versions of a main file and/or of each child file
are to be generated, it may be convenient to set up a |Makefile| or
shell script to automatise the process.

%%%%%%%%%%%%%%%%%%%%%%%%%%%%%%%%%%%%%%%%%%%%%%%%%%%%%%%%%%%%%%%%%%%%%%%%%%%%%%%%
\subsection{Command Line Processing}
\label{sec:commandline}

The effect of redirection files can also be achieved by invoking
the \LaTeX{} compiler with a more elaborate command line.
Most conveniently this should be done as part
of a shell script or a |Makefile|.

When using \textsf{childdoc} in the main file, the following
command lines effectively perform a redirection
(note that depending on the shell being used,
backslashes may have to be doubled: `|\|' $\to$ `|\\|'):
%
\begin{center}
|... -jobname "|\textit{target}|" |\\|"|[\textit{flags}]%
|\input{childdoc.def}\childdocforward[|\textit{main}|]{|\textit{dest}|}"|
\end{center}
%
Here \textit{target} is the name of the output file,
\textit{main} is the name of the main file
and \textit{dest} is the name of the main or child file to be processed
(all filenames without extensions).
The optional argument \textit{main} can be omitted
if \textit{main} matches \textit{dest}.
Optionally, compilation \textit{flags} can be defined via |\def| commands.
This command line makes the \TeX{} engine believe
it is compiling the file \textit{target}
whose content is specified as the latter parameter.
The provided code then forwards the processing to
\textit{main} or \textit{dest} as described in \secref{sec:forward}.

%%%%%%%%%%%%%%%%%%%%%%%%%%%%%%%%%%%%%%%%%%%%%%%%%%%%%%%%%%%%%%%%%%%%%%%%%%%%%%%%
\subsection{Include by Input}
\label{sec:input}

Including child documents by |\include| has some restrictions by design.
Most notably, the content of a child document always occupies
its own set of pages; pages cannot be shared between child documents.
Usually, this behaviour makes perfect sense
because each child document contain an essential part of the document.
However, in some situations it may be desirable to compose
a document from a collection of parts
without having mandatory page breaks between then.
For this case, the package
provides a mechanism to include parts
by |\input| which can also be processed individually.
However, by construction this mechanism
requires manual handling of the content to be output.

%%%%%%%%%%%%%%%%%%%%%%%%%%%%%%%%%%%%%%%%
\DescribeMacro{\ifchilddocmanual}
The main file should be prepared as usual, see \secref{sec:include}.
However, the document body must make a distinction
between processing of an individual part and of the main document, e.g.:
%
\begin{center}
\begin{tabular}{l}
|\ifchilddocmanual|\\
|\input{\childdocname}|\\
|\||else|\\
\textit{document body with }|\input{|\textit{part}|}|\\
|\||fi|
\end{tabular}
\end{center}
%
The conditional |\ifchilddocmanual| is true whenever
a part to be included by |\input| is being compiled,
and the name of the part is stored in |\childdocname|.

%%%%%%%%%%%%%%%%%%%%%%%%%%%%%%%%%%%%%%%%
\DescribeMacro{\childdocby}
Each part to be included by |\input| should start with:
%
\begin{center}
\begin{tabular}{l}
|\input{childdoc.def}|\\
|\childdocby{|\textit{main}|}|\\
\end{tabular}
\end{center}
%
The directive |\childdocby| is similar to |\childdocof|
described in \secref{sec:include},
but the subsequent selection of content must be done manually.
To that end, both |\ifchilddoc| and |\ifchilddocmanual|
will be true upon processing of a part,
and the name of the part is stored in |\childdocname|.
Note that |\jobname| will be set to the filename of the current part
so that each part receives an individual |.aux| file
that does not interfere with the |.aux| file(s) of the main document.
This behaviour can be altered by the alternative form
|\childdocby[*]{|\textit{main}|}| (with a non-empty optional argument)
which uses the |.aux| file of the main document
by setting |\jobname| to \textit{main}.

%%%%%%%%%%%%%%%%%%%%%%%%%%%%%%%%%%%%%%%%%%%%%%%%%%%%%%%%%%%%%%%%%%%%%%%%%%%%%%%%
\subsection{Driver Development}
\label{sec:driver}

The \textsf{childdoc} mechanism can also be use for the development
of definition files such as \LaTeX{} styles or classes.
This case differs from the above setup with multiple parts
included by |\include| in that no |\includeonly| should be invoked.
This can be achieved by starting the include file
(before |\ProvidesPackage|) with:
%
\begin{center}
\begin{tabular}{l}
|\input{childdoc.def}|\\
|\childdocforward{|\textit{main}|}|\\
\end{tabular}
\end{center}
%
or alternatively with:
%
\begin{center}
\begin{tabular}{l}
|\input{childdoc.def}|\\
|\childdocby{|\textit{main}|}|\\
\end{tabular}
\end{center}
%
Both forms have slightly different effects as described above.
The main file is prepared as usual, see \secref{sec:include}.

%%%%%%%%%%%%%%%%%%%%%%%%%%%%%%%%%%%%%%%%%%%%%%%%%%%%%%%%%%%%%%%%%%%%%%%%%%%%%%%%
\subsection{Legacy Detection}
\label{sec:detection}

The directive |\childdocmain| in the main file can detect
whether the complete document or merely a child is to be compiled
even without using the directive |\childdocof|.
This method is deprecated because it is less robust
and there is no compelling reason to use it;
it is merely provided for backward compatibility
and it may be removed in future versions.

If the detection mechanism is to be used,
it is mandatory to correctly specify
the filename of the main file as the argument of |\childdocmain|:
%
\begin{center}
\begin{tabular}{l}
|\input{childdoc.def}|\\
|\childdocmain{|\textit{main}|}|\\
\end{tabular}
\end{center}
%
If |\jobname| does not match the argument \textit{main} of |\childdocmain|,
it is assumed that |\jobname| points to the child file to be compiled.
When using |\childdocmain| with the main file specified as argument,
it suffices to start a child file
with just |\input{|\textit{main}|}|
without loading of the package and using |\childdocof|.
If instead all processing is done
with the appropriate \textsf{childdoc} directives,
the argument of \textit{main} of |\childdocmain| can be empty.

An alternative version of the command line processing described
in \secref{sec:commandline} using the detection mechanism reads:
%
\begin{center}
|... -jobname "|\textit{target}|" "|[\textit{flags}]%
[|\def\jobname{|\textit{dest}|}|]|\input{|\textit{main}|}"|
\end{center}

%%%%%%%%%%%%%%%%%%%%%%%%%%%%%%%%%%%%%%%%%%%%%%%%%%%%%%%%%%%%%%%%%%%%%%%%%%%%%%%%
\subsection{Manual Code}
\label{sec:manual}

In case one cannot be certain whether the definitions file |childdoc.def|
is installed on the target \TeX{} distribution
and one prefers not to ship it,
it is conceivable to paste a few relevant commands into the sources.

To that end, drop all statements |\input{childdoc.def}|
and perform the replacements as outlined below.
Instead of |\childdocmain{|\textit{main}|}| add the following code
to the top of the main file:
%
\begin{center}
\begin{tabular}{l}
|\||ifdefined\childdocname\endinput\||fi\newif\ifchilddoc|\\
|\edef\childdocname{\scantokens\expandafter{\jobname\noexpand}}|\\
|\def\childdocmain{|\textit{main}|}\||ifx\childdocmain\childdocname\||else|\\
|\childdoctrue\includeonly{\childdocname}\let\jobname\childdocmain\||fi|\\
\end{tabular}
\end{center}
%
Instead of |\childdocof{|\textit{main}|}| just include the main file
at the top of each child file:
%
\begin{center}
|\input{|\textit{main}|}|
\end{center}
%
A simple redirection |\childdocforward{|\textit{dest}|}| is achieved by:
%
\begin{center}
|\def\jobname{|\textit{dest}|}\input{\jobname}|
\end{center}
%
The redirection with prefix
|\childdocforwardprefix[|\textit{prefix}|]{|\textit{dest}|}|
is accomplished by:
%
\begin{center}
\begin{tabular}{l}
|{\edef\jobname{\scantokens\expandafter{\jobname\noexpand}}|\\
|\def\redirectjob |\textit{prefix}|#1~~~{\gdef\jobname{|\textit{dest}|#1}}|\\
|\expandafter\redirectjob\jobname~~~}\input{\jobname}|
\end{tabular}
\end{center}

In an alternative approach,
child documents can be compiled by a specific command line
without additional code or specific definitions:
%
\begin{center}
|... -jobname "|\textit{target}|" "|[\textit{flags}]%
|\includeonly{|\textit{dest}|}\input{|\textit{main}|}"|
\end{center}
%

%%%%%%%%%%%%%%%%%%%%%%%%%%%%%%%%%%%%%%%%%%%%%%%%%%%%%%%%%%%%%%%%%%%%%%%%%%%%%%%%
%%%%%%%%%%%%%%%%%%%%%%%%%%%%%%%%%%%%%%%%%%%%%%%%%%%%%%%%%%%%%%%%%%%%%%%%%%%%%%%%
\section{Information}

%%%%%%%%%%%%%%%%%%%%%%%%%%%%%%%%%%%%%%%%%%%%%%%%%%%%%%%%%%%%%%%%%%%%%%%%%%%%%%%%
\subsection{Copyright}

Copyright \copyright{} 2017--2018 Niklas Beisert

This work may be distributed and/or modified under the
conditions of the \LaTeX{} Project Public License, either version 1.3
of this license or (at your option) any later version.
The latest version of this license is in
  \url{http://www.latex-project.org/lppl.txt}
and version 1.3 or later is part of all distributions of \LaTeX{}
version 2005/12/01 or later.

This work has the LPPL maintenance status `maintained'.

The Current Maintainer of this work is Niklas Beisert.

This work consists of the files |README.txt|, |childdoc.ins| and |childdoc.dtx|
as well as the derived files |childdoc.def|, |cdocsamp.tex|
with |cdocsch1.tex|, |cdocsch2.tex|, |cdocspt3.tex|, |cdocspt4.tex|,
|cdocsdrf.tex|, |cdocsfn1.tex|, |cdocsfn2.tex|
as well as |childdoc.pdf|.

%%%%%%%%%%%%%%%%%%%%%%%%%%%%%%%%%%%%%%%%%%%%%%%%%%%%%%%%%%%%%%%%%%%%%%%%%%%%%%%%
\subsection{Files and Installation}

The package consists of the files:
%
\begin{center}
\begin{tabular}{ll}
    |README.txt|   & readme file \\
    |childdoc.ins| & installation file \\
    |childdoc.dtx| & source file \\
    |childdoc.def| & definition file \\
    |cdocsamp.tex| & sample main file \\
    |cdocsch1.tex| & sample include file \\
    |cdocsch2.tex| & sample include file \\
    |cdocspt3.tex| & sample part file \\
    |cdocspt4.tex| & sample part file \\
    |cdocsdrf.tex| & sample redirection file \\
    |cdocsfn1.tex| & sample redirection file \\
    |cdocsfn2.tex| & sample redirection file \\
    |childdoc.pdf| & manual
\end{tabular}
\end{center}
%
The distribution consists of the files
|README.txt|, |childdoc.ins| and |childdoc.dtx|.
%
\begin{itemize}
\item
Run (pdf)\LaTeX{} on |childdoc.dtx|
to compile the manual |childdoc.pdf| (this file).
\item
Run \LaTeX{} on |childdoc.ins| to create the definitions file |childdoc.def|
and the sample |cdocsamp.tex| with include files
|cdocsch1.tex|, |cdocsch2.tex|, |cdocspt3.tex|, |cdocspt4.tex|,
|cdocsdrf.tex|, |cdocsfn1.tex|, |cdocsfn2.tex|.
Then copy the file |childdoc.def| to an appropriate directory of your \LaTeX{}
distribution, e.g.\ \textit{texmf-root}|/tex/latex/childdoc|.
\end{itemize}

%%%%%%%%%%%%%%%%%%%%%%%%%%%%%%%%%%%%%%%%%%%%%%%%%%%%%%%%%%%%%%%%%%%%%%%%%%%%%%%%
\subsection{Related CTAN Packages}

There are several other packages which offer a similar functionality:
%
\begin{itemize}
\item
The packages
\href{http://ctan.org/pkg/docmute}{\textsf{docmute}},
\href{http://ctan.org/pkg/includex}{\textsf{includex}} and
\href{http://ctan.org/pkg/standalone}{\textsf{standalone}}
provide commands to include only the document body of
a child file thus allowing both files to be compiled individually.
\item
The packages \href{http://ctan.org/pkg/subdocs}{\textsf{subdocs}}
and \href{http://ctan.org/pkg/subfiles}{\textsf{subfiles}}
provide structures in which the main and child documents can be
encapsulated and allowing them to be compiled individually.
The inclusion mechanism is different from the conventional |\include|.
\item
The package \href{http://ctan.org/pkg/combine}{\textsf{combine}}
is an elaborate solution to combine several documents into one.
\end{itemize}
%
See also the CTAN topic \href{http://ctan.org/topic/subdocs}{\textsf{subdocs}}
for further related packages.
The present package differs from the above solutions in that
a document structure constructed with the conventional |\include| mechanism
just needs two extra commands at the top of every file
such that all constituent files can be compiled individually.

%%%%%%%%%%%%%%%%%%%%%%%%%%%%%%%%%%%%%%%%%%%%%%%%%%%%%%%%%%%%%%%%%%%%%%%%%%%%%%%%
%\subsection{Feature Suggestions}
%
%The following is a list of features which may be useful for future
%versions of this package:
%%
%\begin{itemize}
%\item
%\ldots
%\end{itemize}

%%%%%%%%%%%%%%%%%%%%%%%%%%%%%%%%%%%%%%%%%%%%%%%%%%%%%%%%%%%%%%%%%%%%%%%%%%%%%%%%
\subsection{Revision History}

%%%%%%%%%%%%%%%%%%%%%%%%%%%%%%%%%%%%%%%%
\paragraph{v2.0:} 2018/12/30

\begin{itemize}
\item
immediate forward processing
\item
added |\childdocby| mechanism
\item
manual restructured
\end{itemize}

%%%%%%%%%%%%%%%%%%%%%%%%%%%%%%%%%%%%%%%%
\paragraph{v1.6:} 2018/01/17

\begin{itemize}
\item
application for development of include files
\item
corrections to manual
\end{itemize}

%%%%%%%%%%%%%%%%%%%%%%%%%%%%%%%%%%%%%%%%
\paragraph{v1.5:} 2017/05/21

\begin{itemize}
\item
more complete structuring introduced
\item
|\childdocof| introduced
\item
|\childdoc| renamed to |\childdocmain|
\item
|\childredirect| renamed to |\childdocforward| and |\childdocforwardprefix|
and functionality expanded
\end{itemize}

%%%%%%%%%%%%%%%%%%%%%%%%%%%%%%%%%%%%%%%%
\paragraph{v1.0:} 2017/04/27

\begin{itemize}
\item
manual and install package
\item
first version published on CTAN
\end{itemize}

%%%%%%%%%%%%%%%%%%%%%%%%%%%%%%%%%%%%%%%%
\paragraph{v0.6:} 2017/04/26

\begin{itemize}
\item
redirection mechanism added
\end{itemize}

%%%%%%%%%%%%%%%%%%%%%%%%%%%%%%%%%%%%%%%%
\paragraph{v0.5:} 2017/04/26

\begin{itemize}
\item
functionality in definition file
\end{itemize}


%%%%%%%%%%%%%%%%%%%%%%%%%%%%%%%%%%%%%%%%%%%%%%%%%%%%%%%%%%%%%%%%%%%%%%%%%%%%%%%%
%%%%%%%%%%%%%%%%%%%%%%%%%%%%%%%%%%%%%%%%%%%%%%%%%%%%%%%%%%%%%%%%%%%%%%%%%%%%%%%%
%%%%%%%%%%%%%%%%%%%%%%%%%%%%%%%%%%%%%%%%%%%%%%%%%%%%%%%%%%%%%%%%%%%%%%%%%%%%%%%%
\appendix

\settowidth\MacroIndent{\rmfamily\scriptsize 000\ }

 \DocInput{childdoc.dtx}

\end{document}
%</driver>
% \fi
%
% %%%%%%%%%%%%%%%%%%%%%%%%%%%%%%%%%%%%%%%%%%%%%%%%%%%%%%%%%%%%%%%%%%%%%%%%%%%%%%
% %%%%%%%%%%%%%%%%%%%%%%%%%%%%%%%%%%%%%%%%%%%%%%%%%%%%%%%%%%%%%%%%%%%%%%%%%%%%%%
% \section{Sample}
%\iffalse
%<*samplemain>
%\fi
%
% The following presents a sample document
% with two chapters, two parts, a title page,
% a compile flag as well as three forwarding files to set the flag.
% It consists of eight |.tex| files:
% \begin{center}
% \begin{tabular}{ll}
% |cdocsamp.tex|&main file\\
% |cdocsch1.tex|&include file for chapter 1\\
% |cdocsch2.tex|&include file for chapter 2\\
% |cdocspt3.tex|&include file for part 3\\
% |cdocspt4.tex|&include file for part 4\\
% |cdocsdrf.tex|&forwarding file for main file in draft mode\\
% |cdocsfi1.tex|&forwarding file for final version of chapter 1\\
% |cdocsfi2.tex|&forwarding file for final version of chapter 2\\
% \end{tabular}
% \end{center}
% Each of the eight files can be compiled directly by the \LaTeX{} compiler.
%
% %%%%%%%%%%%%%%%%%%%%%%%%%%%%%%%%%%%%%%
% \paragraph{Main File.}
%
% The main file is called |cdocsamp.tex|.
%
% Load the \textsf{childdoc} definitions and
% declare the filename for the main document:
%    \begin{macrocode}
\input{childdoc.def}
\childdocmain{}
%    \end{macrocode}

% Optional override for |\version| flag:
%    \begin{macrocode}
%%\ifchilddoc\else\providecommand{\version}{draft}\fi
%    \end{macrocode}

% Define the default values for the |\version| flag
% (|final| for the main file and |draft| for childs):
%    \begin{macrocode}
\ifchilddoc
\providecommand{\version}{draft}
\else
\providecommand{\version}{final}
\fi
%    \end{macrocode}

% Load the standard document class:
%    \begin{macrocode}
\documentclass[12pt]{article}
%    \end{macrocode}

% Start the document body:
%    \begin{macrocode}
\begin{document}
%    \end{macrocode}

% Declare a title page.
% Print title, part of document being processed and version flag:
%    \begin{macrocode}
\addtocounter{page}{-1}
\begin{center}
{\LARGE\bfseries{}childdoc example\par}
\vspace{1cm}
\ifchilddoc
\ifchilddocmanual part\else chapter\fi:
`\childdocname' of `\childdocjob'\par
\else
main document: `\childdocjob'\par
\fi
version: \version\par
\end{center}
\newpage
%    \end{macrocode}

% Manually include selected file,
% otherwise process as usual:
%    \begin{macrocode}
\ifchilddocmanual
\section*{part `\childdocname'}
\input{\childdocname}
\else
%    \end{macrocode}

% Include the two chapters:
%    \begin{macrocode}
\include{cdocsch1}
\include{cdocsch2}
%    \end{macrocode}

% Include the two parts unless only chapters should be displayed:
%    \begin{macrocode}
\ifchilddoc\else
\section{part three}
\input{cdocspt3}
\section{part four}
\input{cdocspt4}
\fi
%    \end{macrocode}

% Process as usual until here:
%    \begin{macrocode}
\fi
%    \end{macrocode}

% End of document body:
%    \begin{macrocode}
\end{document}
%    \end{macrocode}
%\iffalse
%</samplemain>
%\fi
%
% %%%%%%%%%%%%%%%%%%%%%%%%%%%%%%%%%%%%%%
% \paragraph{Chapter Include Files.}
%
% The include files are called |cdocsch1.tex| and |cdocsch2.tex|.
%
%\iffalse
%<*samplechap1|samplechap2>
%\fi

% Optional override for |\version| flag:
%    \begin{macrocode}
%%\providecommand{\version}{final}
%    \end{macrocode}

% Include the main document:
%    \begin{macrocode}
\input{childdoc.def}
\childdocof{cdocsamp}
%    \end{macrocode}

%\iffalse
%</samplechap1|samplechap2>
%\fi
%
%\iffalse
%<*samplechap1>
%\fi
% Some text for chapter 1:
%    \begin{macrocode}
\section{one}
some text in chapter one
%    \end{macrocode}

%\iffalse
%</samplechap1>
%\fi
% Some text for chapter 2:
%\iffalse
%<*samplechap2>
%\fi
%    \begin{macrocode}
\section{two}
more text in chapter two
%    \end{macrocode}

%\iffalse
%</samplechap2>
%\fi
%
% %%%%%%%%%%%%%%%%%%%%%%%%%%%%%%%%%%%%%%
% \paragraph{Part Include Files.}
%
% The include files are called |cdocspt3.tex| and |cdocspt4.tex|.
%
%\iffalse
%<*samplepart3|samplepart4>
%\fi

% Optional override for |\version| flag:
%    \begin{macrocode}
%%\providecommand{\version}{final}
%    \end{macrocode}

% Include the main document:
%    \begin{macrocode}
\input{childdoc.def}
\childdocby{cdocsamp}
%    \end{macrocode}

%\iffalse
%</samplepart3|samplepart4>
%\fi
%
%\iffalse
%<*samplepart3>
%\fi
% Some text for part 3:
%    \begin{macrocode}
some text in part three
%    \end{macrocode}

%\iffalse
%</samplepart3>
%\fi
% Some text for part 4:
%\iffalse
%<*samplepart4>
%\fi
%    \begin{macrocode}
more text in part four
%    \end{macrocode}

%\iffalse
%</samplepart4>
%\fi
%
% %%%%%%%%%%%%%%%%%%%%%%%%%%%%%%%%%%%%%%
% \paragraph{Forwarding for a Complete Draft.}
%
% The following forwarding file |cdocsdrf.tex|
% compiles the main document in draft mode:
%\iffalse
%<*sampledraft>
%\fi
%    \begin{macrocode}
\def\version{draft}
\input{childdoc.def}
\childdocforward{cdocsamp}
%    \end{macrocode}

%\iffalse
%</sampledraft>
%\fi
%
% %%%%%%%%%%%%%%%%%%%%%%%%%%%%%%%%%%%%%%
% \paragraph{Forwarding for Final Version of the Chapters.}
%
% The following forwarding files |cdocsfn1.tex| and |cdocsfn2.tex|
% (with identical content)
% compile the final versions of the child documents
% |cdocsch1.tex| and |cdocsch2.tex|, respectively:
%\iffalse
%<*samplefinal>
%\fi
%    \begin{macrocode}
\def\version{final}
\input{childdoc.def}
\childdocforwardprefix[cdocsamp]{cdocsfn}{cdocsch}
%    \end{macrocode}

%\iffalse
%</samplefinal>
%\fi
%
% %%%%%%%%%%%%%%%%%%%%%%%%%%%%%%%%%%%%%%
% \paragraph{Command Line Processing.}
%
% The following three command lines generate the output files
% |cdocscld|, |cdocscl1| and |cdocscl2|
% which should be identical to
% |cdocsdrf|, |cdocsch1| and |cdocsfn2|, respectively:
% \begin{center}
% \begin{tabular}{l}
% |latex -jobname cdocscld \|\\
% |  "\def\version{draft}\input{childdoc.def}\childdocforward{cdocsamp}"|\\
% |latex -jobname cdocscl1 \|\\
% |  "\input{childdoc.def}\childdocforward[cdocsamp]{cdocsch1}"|\\
% |latex -jobname cdocscl2 \|\\
% |  "\def\version{final}\input{childdoc.def}\childdocforward{cdocsch2}"|
% \end{tabular}
% \end{center}
% Note that the trailing backslash on each first line
% merely continues the input to the second line
% (for convenient cut ant paste).
% Furthermore, the command |latex| can be replaced by any
% of its alternative versions such as |pdflatex|.
%
% %%%%%%%%%%%%%%%%%%%%%%%%%%%%%%%%%%%%%%%%%%%%%%%%%%%%%%%%%%%%%%%%%%%%%%%%%%%%%%
% %%%%%%%%%%%%%%%%%%%%%%%%%%%%%%%%%%%%%%%%%%%%%%%%%%%%%%%%%%%%%%%%%%%%%%%%%%%%%%
% \section{Implementation}
%\iffalse
%<*package>
%\fi
%
% This section describes the definitions file |childdoc.def|.

% The definitions cannot be loaded using |\usepackage| or |\RequirePackage|
% which has a mechanism to prevent loading a style file more than once.
% When loading the definitions by means of |\input|
% multiple instances have to be prevented manually:
%\iffalse
%This code needs to be before the `\ProvidesFile' directive
%which is defined at the beginning of this file.
%Therefore it is also placed there and commented out here.
%</package>
%<*discard>
%\fi
%    \begin{macrocode}
\ifdefined\childdocmain\endinput\fi
%    \end{macrocode}
%\iffalse
%</discard>
%<*package>
%\fi
%
% \macro{\ifchilddoc}
% \macro{\ifchilddocmanual}
% The conditional |\ifchilddoc| tells whether a
% child (true) or main (false) document is being compiled.
% The conditional |\ifchilddocmanual| tells whether
% the |\includeonly| mechanism is used (false) or
% the selection of child files must be performed manually (true).
% The definitions initialise to false:
%    \begin{macrocode}
\newif\ifchilddoc
\newif\ifchilddocmanual
%    \end{macrocode}

% \macro{\childdocname}
% \macro{\childdocjob}
% The macro |\childdocname| stores the name of the main document
% to be compiled. The macro |\childdocjob| stores the name of
% the document on which the \LaTeX{} compiler was originally invoked.
% The content of |\jobname| cannot be compared
% to filenames specified in the source due to different catcodes.
% The following code rescans |\jobname|, stores the result
% in |\childdocname| and saves a copy in |\childdocjob|:
%    \begin{macrocode}
\edef\childdocname{\scantokens\expandafter{\jobname\noexpand}}
\let\childdocjob\childdocname
%    \end{macrocode}

% \macro{\childdocdisable}
% The macro |\childdocdisable| prevents the main file
% from being processed more than once.
% At this stage, the main document command |\childdocmain|
% is assumed to be called once again where it should do nothing.
% Any subsequent call to it should prevent
% a secondary processing of the main document
% It overwrites the forwarding commands
% |\childdocof| and |\childdocforward|
% with empty macros to prevent further inclusions of the main document:
%    \begin{macrocode}
\newcommand{\childdocdisable}
{
  \renewcommand{\childdocmain}[1]{\renewcommand{\childdocmain}[1]{\endinput}}
  \renewcommand{\childdocof}[1]{}
  \renewcommand{\childdocby}[2][]{}
  \renewcommand{\childdocforward}[2][]{}
  \renewcommand{\childdocdisable}{}
}
%    \end{macrocode}

% \macro{\childdocmain}
% The macro |\childdocmain| is to be called at the top of the main file
% with nothing or the main filename (without extension) as argument.
% First, it breaks loops.
% If the argument is not empty and does not match |\childdocname|
% (which is set by the first inclusion of |childdoc.def|),
% |\ifchilddoc| is set to true, |\includeonly| is applied to the child file
% and |\jobname| is set to the main file
% (for proper handling of |.aux| files):
%    \begin{macrocode}
\newcommand{\childdocmain}[1]
{
  \childdocdisable\childdocmain{}
  \if?#1?\else
    \begingroup
      \def\childdoctmp{#1}
      \ifx\childdoctmp\childdocname
        \def\childdoctmp{}
      \else
        \def\childdoctmp
        {
          \childdoctrue
          \includeonly{\childdocname}
          \def\childdocjob{#1}
          \def\jobname{#1}
        }
      \fi
      \expandafter
    \endgroup
    \childdoctmp
  \fi
}
%    \end{macrocode}

% \macro{\childdocof}
% The command |\childdocof| redirects
% compilation to the main file |#1|.
%    \begin{macrocode}
\newcommand{\childdocof}[1]
{
  \childdocdisable
  \childdoctrue
  \includeonly{\childdocname}
  \def\jobname{#1}
  \def\childdocjob{#1}
  \input{#1}
}
%    \end{macrocode}

% \macro{\childdocby}
% The command |\childdocby| ....
%    \begin{macrocode}
\newcommand{\childdocby}[2][]
{
  \childdocdisable
  \childdoctrue
  \childdocmanualtrue
  \if?#1?\else
    \def\jobname{#2}
  \fi
  \def\childdocjob{#2}
  \input{#2}
  \endinput
}
%    \end{macrocode}

% \macro{\childdocforward}
% The command |\childdocforward| redirects
% compilation to the main file or
% (if the optional argument is given) a child file.
% Parameters are set as if the main file
% or a child file starting with |\childdocof| was compiled.
% Then compilation is handed over to the main file:
%    \begin{macrocode}
\newcommand{\childdocforward}[2][]
{
  \begingroup
    \if?#1?
      \def\childdoctmp
      {
        \def\childdocname{#2}
        \def\childdocjob{#2}
        \def\jobname{#2}
        \input{#2}
        \endinput
      }
    \else
      \def\childdoctmp
      {
        \childdocdisable
        \def\childdocname{#2}
        \childdoctrue
        \includeonly{#2}
        \def\childdocjob{#1}
        \def\jobname{#1}
        \input{#1}
        \endinput
      }
    \fi
    \expandafter
  \endgroup
  \childdoctmp
}
%    \end{macrocode}

% \macro{\childdocforwardprefix}
% The command |\childdocforwardprefix| redirects
% compilation to the main or a child file by means of a pattern.
% The prefix |#1| in the current filename is replaced by |#2|
% and the suffix of the current filename is kept
% (it is assumed that the filename does not contain the substring `|~~~|'
% which is used as a delimiter).
% Compilation is handed over to the new file by |\childdocforward|:
%    \begin{macrocode}
\newcommand{\childdocforwardprefix}[3][]
{
  \begingroup
    \def\childdocextract #2##1~~~{\def\childdoctmp{\childdocforward[#1]{#3##1}}}
    \expandafter\childdocextract\childdocname~~~
    \expandafter
  \endgroup
  \childdoctmp
}
%    \end{macrocode}

% \macro{\childdoc}
% The deprecated macro |\childdoc| is a legacy version of |\childdocmain|:
%    \begin{macrocode}
\newcommand{\childdoc}{\childdocmain}
%    \end{macrocode}

% \macro{\childdocredirect}
% The deprecated macro |\childdocredirect| is a legacy version
% of |\childdocforward| and |\childdocforwardprefix|:
%    \begin{macrocode}
\newcommand{\childdocredirect}[2][]
{
  \begingroup
    \if?#1?
      \def\childdoctmp{\childdocforward{#2}}
    \else
      \def\childdoctmp{\childdocforwardprefix{#1}{#2}}
    \fi
    \expandafter
  \endgroup
  \childdoctmp
}
%    \end{macrocode}

%\iffalse
%</package>
%\fi
%
\endinput
|\\
|\childdocof{|\textit{main}|}|\\
\end{tabular}
\end{center}
at the top of every child file \textit{child}
which is included by |\include{|\textit{child}|}|
from within the main file
(or at least for those files to be compiled individually).
The argument \textit{main} must be the filename of the main file.

There are a couple of
considerations in setting up the main and child documents:

%%%%%%%%%%%%%%%%%%%%%%%%%%%%%%%%%%%%%%%%
\paragraph{Restrictions.}

Please note the following restrictions:
\begin{itemize}
\item
|\childdocmain| must be called with one argument \textit{main}
to ensure compatibility with earlier version of the package.
It must either be empty (|\childdocmain{}|)
or precisely match the filename of the main file in which it is specified.
See \secref{sec:detection} for further information.
\item
The filename \textit{main} must be specified without the |.tex| extension.
\item
The filename \textit{main} is case sensitive
(even in case-insensitive file systems)
due to internal string comparison.
\item
The argument \textit{main} should be fully expanded, it cannot be a macro.
\item
Subdirectories and special characters should be avoided in filenames.
\item
The command |\childdocmain{|\textit{main}|}| must be followed by a whitespace.
It should not be followed immediately by another command
or by a comment mark `|%|'.
This is because the \TeX{} parser reads the token immediately following
the argument of |\childdocmain| and puts it
at the beginning of every child section;
however, a white\-space is ignored.
\end{itemize}

%%%%%%%%%%%%%%%%%%%%%%%%%%%%%%%%%%%%%%%%
\paragraph{Content of Main File.}

It is advisable to place all content in the child files included by |\include|.
Any output contained in the main file will appear in all child documents
unless suppressed manually;
it cannot be suppressed automatically by the |\includeonly| directive
and thus should normally be avoided.
A method to include some content in the main file
by means of conditional processing is described in \secref{sec:conditional}.

%%%%%%%%%%%%%%%%%%%%%%%%%%%%%%%%%%%%%%%%
\paragraph{Page Numbering.}

When only a part of the document is compiled,
the appropriate numbering of pages
(as well as other status parameters)
is determined from the |.aux| files.
The latter contain information from previous passes.
However this information needs to propagate through
all intermediate child documents.
Therefore the page numbering in child documents may well
be inconsistent until the complete document is compiled at least once.

A useful (if unconventional) way to always ensure a consistent
page numbering is to restart the numbering in each child document
and denote the pages by `\textit{child}|.|\textit{page}'
where \textit{child} represents the chapter/section number of the child file.
This can be achieved by the command
|\numberwithin{page}{|\textit{child}|}|
of the \textsf{amsmath} package
where \textit{child} can be |chapter| or |section|
depending on the chosen structuring.
Alternatively, one can modify the macro |\thepage| appropriately
and reset the counter |page| at the start of each child file.

%%%%%%%%%%%%%%%%%%%%%%%%%%%%%%%%%%%%%%%%%%%%%%%%%%%%%%%%%%%%%%%%%%%%%%%%%%%%%%%%
\subsection{Conditional Processing}
\label{sec:conditional}

The package provides a mechanism to compile different versions
of a document. To customise the versions further some conditional processing
can come in handy to distinguish which version is being compiled.
The package provides two macros to describe the compilation context:

%%%%%%%%%%%%%%%%%%%%%%%%%%%%%%%%%%%%%%%%
\DescribeMacro{\ifchilddoc}
The conditional |\ifchilddoc| distinguishes between the compilation of
child documents and the main document:
%
\begin{center}
|\ifchilddoc |\textit{child-code}| |[|\||else |\textit{main-code}]| \||fi|
\end{center}

%%%%%%%%%%%%%%%%%%%%%%%%%%%%%%%%%%%%%%%%
\DescribeMacro{\childdocname}
\DescribeMacro{\childdocjob}
The macro |\childdocname| contains the filename (without extension)
of the main or child file being processed.
Note that |\childdocjob| will always contain the name of the main file.

%%%%%%%%%%%%%%%%%%%%%%%%%%%%%%%%%%%%%%%%
\paragraph{Title Page.}

Conditional processing can be used to include a title or banner page
in the main document when proper precautions are taken.
Importantly, the code in the main file should ensure that the page counter
(as well as other status parameters which are stored in the |.aux| files)
takes the same value after the conditional processing.
Otherwise the page numbers may take divergent values
depending on which part is compiled.

For example, a title page could be declared by:
%
\begin{center}
\begin{tabular}{l}
|\ifchilddoc\||else|\\
|\addtocounter{page}{-1}|\\
\textit{code for title page}\\
|\newpage|\\
|\||fi|
\end{tabular}
\end{center}
%
A banner page for the child documents can be generated by:
%
\begin{center}
\begin{tabular}{l}
|\ifchilddoc|\\
|\addtocounter{page}{-1}|\\
\textit{code for banner page}\\
|\newpage|\\
|\||fi|
\end{tabular}
\end{center}
%
Here one could write a message such as:
\begin{center}
|This is the part \childdocname{} of \childdocjob{}.|
\end{center}

%%%%%%%%%%%%%%%%%%%%%%%%%%%%%%%%%%%%%%%%%%%%%%%%%%%%%%%%%%%%%%%%%%%%%%%%%%%%%%%%
\subsection{Flags}
\label{sec:flags}

The package makes it easy to generate different versions
of the main or child documents.
To this end compilation flags can be defined
and assigned different default values.
They will be particularly useful in conjunction
with the forwarding mechanism described in \secref{sec:forward}.

For example, it may be useful to have a flag |\version|
which can be set to |draft| or |final|.
The document source will contain some conditional code
depending on the value of |\version|.
Suppose further, the flag should default to |final| for the main file
and to |draft| for child files
which is a natural assignment for editing the document.
This is achieved by placing the following code
in the preamble of the main document
(below the |\childdocmain| directive):
%
\begin{center}
\begin{tabular}{l}
|\ifchilddoc|\\
|\providecommand{\version}{draft}|\\
|\||else|\\
|\providecommand{\version}{final}|\\
|\||fi|
\end{tabular}
\end{center}
%
The definition by |\providecommand| makes sure
that previous definitions are not overwritten.
Further statements |\providecommand{\version}{...}|
can thus be added before the above code to override it.

For the main file, one might add a line
(between |\childdocmain| and the above block)
%
\begin{center}
|%\ifchilddoc\||else\providecommand{\version}{draft}\||fi|
\end{center}
%
which can be uncommented to produce a draft version.
Likewise one can add a line to the very top of a child file
(above the |\childdocof{|\textit{main}|}| directive)
%
\begin{center}
|%\providecommand{\version}{final}|
\end{center}
%
which can be uncommented to produce the final version of this child document.

%%%%%%%%%%%%%%%%%%%%%%%%%%%%%%%%%%%%%%%%%%%%%%%%%%%%%%%%%%%%%%%%%%%%%%%%%%%%%%%%
\subsection{Forwarding}
\label{sec:forward}

Different versions of the main or child documents
using compilation flags as described in \secref{sec:flags}
can be (permanently) stored in different files
for convenient compilation, viewing and distribution.
To this end, the package defines a command
to pass on compilation to a different file:

%%%%%%%%%%%%%%%%%%%%%%%%%%%%%%%%%%%%%%%%
\DescribeMacro{\childdocforward}
The command |\childdocforward| redirects processing to
another source file:
%
\begin{center}
\begin{tabular}{l}
|% \iffalse
%
% childdoc.dtx Copyright (C) 2017-2018 Niklas Beisert
%
% This work may be distributed and/or modified under the
% conditions of the LaTeX Project Public License, either version 1.3
% of this license or (at your option) any later version.
% The latest version of this license is in
%   http://www.latex-project.org/lppl.txt
% and version 1.3 or later is part of all distributions of LaTeX
% version 2005/12/01 or later.
%
% This work has the LPPL maintenance status `maintained'.
%
% The Current Maintainer of this work is Niklas Beisert.
%
% This work consists of the files childdoc.dtx and childdoc.ins
% and the derived files childdoc.def and cdocsamp.tex with
% cdocsch1.tex, cdocsch2.tex, cdocsdrf.tex, cdocsfn1.tex, cdocsfn2.tex.
%
%<package>\ifdefined\childdocmain\endinput\fi
%<package>\ProvidesFile{childdoc.def}[2018/12/30 v2.0 child document driver]
%<samplemain>\ProvidesFile{cdocsamp.tex}[2018/12/30 v2.0 sample for childdoc]
%<*driver>
%\ProvidesFile{childdoc.drv}[2018/12/30 v2.0 childdoc reference manual file]
\PassOptionsToClass{10pt,a4paper}{article}
\documentclass{ltxdoc}

\usepackage[margin=35mm]{geometry}
\usepackage{hyperref}
\usepackage{hyperxmp}
\usepackage[usenames]{color}

\hypersetup{colorlinks=true}
\hypersetup{pdfstartview=FitH}
\hypersetup{pdfpagemode=UseNone}
\hypersetup{pdfsource={}}
\hypersetup{pdflang={en-UK}}
\hypersetup{pdfcopyright={Copyright 2017-2018 Niklas Beisert.
  This work may be distributed and/or modified under the
  conditions of the LaTeX Project Public License, either version 1.3
  of this license or (at your option) any later version.}}
\hypersetup{pdflicenseurl={http://www.latex-project.org/lppl.txt}}
\hypersetup{pdfcontactaddress={ETH Zurich, ITP, HIT K,
  Wolfgang-Pauli-Strasse 27}}
\hypersetup{pdfcontactpostcode={8093}}
\hypersetup{pdfcontactcity={Zurich}}
\hypersetup{pdfcontactcountry={Switzerland}}
\hypersetup{pdfcontactemail={nbeisert@itp.phys.ethz.ch}}
\hypersetup{pdfcontacturl={http://people.phys.ethz.ch/\xmptilde nbeisert/}}

\newcommand{\secref}[1]{\hyperref[#1]{section \ref*{#1}}}

\parskip1ex
\parindent0pt
\let\olditemize\itemize
\def\itemize{\olditemize\parskip0pt}

\begin{document}

\title{The \textsf{childdoc} Package}
\hypersetup{pdftitle={The childdoc Package}}
\author{Niklas Beisert\\[2ex]
  Institut f\"ur Theoretische Physik\\
  Eidgen\"ossische Technische Hochschule Z\"urich\\
  Wolfgang-Pauli-Strasse 27, 8093 Z\"urich, Switzerland\\[1ex]
  \href{mailto:nbeisert@itp.phys.ethz.ch}
  {\texttt{nbeisert@itp.phys.ethz.ch}}}
\hypersetup{pdfauthor={Niklas Beisert}}
\hypersetup{pdfsubject={Manual for the LaTeX2e Package childdoc}}
\date{30 December 2018, \textsf{v2.0}}
\maketitle

\begin{abstract}\noindent
\textsf{childdoc} is a \LaTeXe{} package
that enables the direct compilation
of document sections included by |\include|
to individual files.
\end{abstract}

\begingroup
\parskip0ex
\tableofcontents
\endgroup

%%%%%%%%%%%%%%%%%%%%%%%%%%%%%%%%%%%%%%%%%%%%%%%%%%%%%%%%%%%%%%%%%%%%%%%%%%%%%%%%
%%%%%%%%%%%%%%%%%%%%%%%%%%%%%%%%%%%%%%%%%%%%%%%%%%%%%%%%%%%%%%%%%%%%%%%%%%%%%%%%
\section{Introduction}

\LaTeX{} provides a mechanism to structure a large document (such as a book)
into a main file and several child files (containing the chapters)
using the |\include| command.
This mechanism is beneficial for documents
which span hundreds of pages in order to
make the source file(s) more manageable.
Moreover, compilation can be restricted to
selected child files by means of the |\includeonly| command.
The latter feature can be used to reduce the compilation time while editing
(this was significantly more useful in the earlier days of \LaTeX{})
or to generate a smaller document which is easier to navigate.
Another application of |\includeonly| is to generate
documents consisting of selected parts of the complete document.

However, there are a few drawbacks of the plain |\include| mechanism:
\begin{itemize}
\item
The child files cannot be compiled on their own,
they can only be compiled via the main file.
A naive editing environment
(such as a text editor with an option
to have the current file processed by \LaTeX)
may require one to switch to the main file before compiling;
attempting to compile the child file produces errors.
\item
The main file must be modified (each time)
to adjust the |\includeonly| command
to the present needs. This easily leaves the main file in a messy state.
\item
The generated document will always carry the filename
of the main document. This is inconvenient if
several child files are to be compiled and
to be kept for distribution.
\end{itemize}

The present package provides a simple interface
to make child files individually compilable by \LaTeX{}.
Compiling a child file then has the same effect as compiling
the main file with an |\includeonly| command
to select the appropriate child.
Moreover the generated document will carry the name of the child
rather than the main file.
This resolves all three above issues.

This feature is meant to make the editing of books,
thesis documents and lecture notes somewhat more convenient.
However, the package can also be used efficiently for
composing a series of documents (such as exercise sheets)
which are typically distributed individually.
It then assists the author in generating the individual documents
(potentially in different versions)
as well as a document containing the collected series.
Another application is in developing style files
or other kinds of included material
where compilation of the style file could redirect
to a sample or test file.

%%%%%%%%%%%%%%%%%%%%%%%%%%%%%%%%%%%%%%%%%%%%%%%%%%%%%%%%%%%%%%%%%%%%%%%%%%%%%%%%
%%%%%%%%%%%%%%%%%%%%%%%%%%%%%%%%%%%%%%%%%%%%%%%%%%%%%%%%%%%%%%%%%%%%%%%%%%%%%%%%
\section{Usage}

First of all, the package \textsf{childdoc} is \emph{not} a standard
\LaTeXe{} |.sty| style file! Therefore it needs to be invoked in
a non-standard way.

%%%%%%%%%%%%%%%%%%%%%%%%%%%%%%%%%%%%%%%%%%%%%%%%%%%%%%%%%%%%%%%%%%%%%%%%%%%%%%%%
\subsection{Included Files}
\label{sec:include}

%%%%%%%%%%%%%%%%%%%%%%%%%%%%%%%%%%%%%%%%
\DescribeMacro{\childdocmain}
To use the package, add the commands
\begin{center}
\begin{tabular}{l}
|\input{childdoc.def}|\\
|\childdocmain{}|\\
\end{tabular}
\end{center}
at the very top of the main \LaTeX{} file,
in particular \emph{before} the |\documentclass| statement!
The argument of |\childdocmain| should be left empty
(but it must be present).

%%%%%%%%%%%%%%%%%%%%%%%%%%%%%%%%%%%%%%%%
\DescribeMacro{\childdocof}
Furthermore, add the commands
\begin{center}
\begin{tabular}{l}
|\input{childdoc.def}|\\
|\childdocof{|\textit{main}|}|\\
\end{tabular}
\end{center}
at the top of every child file \textit{child}
which is included by |\include{|\textit{child}|}|
from within the main file
(or at least for those files to be compiled individually).
The argument \textit{main} must be the filename of the main file.

There are a couple of
considerations in setting up the main and child documents:

%%%%%%%%%%%%%%%%%%%%%%%%%%%%%%%%%%%%%%%%
\paragraph{Restrictions.}

Please note the following restrictions:
\begin{itemize}
\item
|\childdocmain| must be called with one argument \textit{main}
to ensure compatibility with earlier version of the package.
It must either be empty (|\childdocmain{}|)
or precisely match the filename of the main file in which it is specified.
See \secref{sec:detection} for further information.
\item
The filename \textit{main} must be specified without the |.tex| extension.
\item
The filename \textit{main} is case sensitive
(even in case-insensitive file systems)
due to internal string comparison.
\item
The argument \textit{main} should be fully expanded, it cannot be a macro.
\item
Subdirectories and special characters should be avoided in filenames.
\item
The command |\childdocmain{|\textit{main}|}| must be followed by a whitespace.
It should not be followed immediately by another command
or by a comment mark `|%|'.
This is because the \TeX{} parser reads the token immediately following
the argument of |\childdocmain| and puts it
at the beginning of every child section;
however, a white\-space is ignored.
\end{itemize}

%%%%%%%%%%%%%%%%%%%%%%%%%%%%%%%%%%%%%%%%
\paragraph{Content of Main File.}

It is advisable to place all content in the child files included by |\include|.
Any output contained in the main file will appear in all child documents
unless suppressed manually;
it cannot be suppressed automatically by the |\includeonly| directive
and thus should normally be avoided.
A method to include some content in the main file
by means of conditional processing is described in \secref{sec:conditional}.

%%%%%%%%%%%%%%%%%%%%%%%%%%%%%%%%%%%%%%%%
\paragraph{Page Numbering.}

When only a part of the document is compiled,
the appropriate numbering of pages
(as well as other status parameters)
is determined from the |.aux| files.
The latter contain information from previous passes.
However this information needs to propagate through
all intermediate child documents.
Therefore the page numbering in child documents may well
be inconsistent until the complete document is compiled at least once.

A useful (if unconventional) way to always ensure a consistent
page numbering is to restart the numbering in each child document
and denote the pages by `\textit{child}|.|\textit{page}'
where \textit{child} represents the chapter/section number of the child file.
This can be achieved by the command
|\numberwithin{page}{|\textit{child}|}|
of the \textsf{amsmath} package
where \textit{child} can be |chapter| or |section|
depending on the chosen structuring.
Alternatively, one can modify the macro |\thepage| appropriately
and reset the counter |page| at the start of each child file.

%%%%%%%%%%%%%%%%%%%%%%%%%%%%%%%%%%%%%%%%%%%%%%%%%%%%%%%%%%%%%%%%%%%%%%%%%%%%%%%%
\subsection{Conditional Processing}
\label{sec:conditional}

The package provides a mechanism to compile different versions
of a document. To customise the versions further some conditional processing
can come in handy to distinguish which version is being compiled.
The package provides two macros to describe the compilation context:

%%%%%%%%%%%%%%%%%%%%%%%%%%%%%%%%%%%%%%%%
\DescribeMacro{\ifchilddoc}
The conditional |\ifchilddoc| distinguishes between the compilation of
child documents and the main document:
%
\begin{center}
|\ifchilddoc |\textit{child-code}| |[|\||else |\textit{main-code}]| \||fi|
\end{center}

%%%%%%%%%%%%%%%%%%%%%%%%%%%%%%%%%%%%%%%%
\DescribeMacro{\childdocname}
\DescribeMacro{\childdocjob}
The macro |\childdocname| contains the filename (without extension)
of the main or child file being processed.
Note that |\childdocjob| will always contain the name of the main file.

%%%%%%%%%%%%%%%%%%%%%%%%%%%%%%%%%%%%%%%%
\paragraph{Title Page.}

Conditional processing can be used to include a title or banner page
in the main document when proper precautions are taken.
Importantly, the code in the main file should ensure that the page counter
(as well as other status parameters which are stored in the |.aux| files)
takes the same value after the conditional processing.
Otherwise the page numbers may take divergent values
depending on which part is compiled.

For example, a title page could be declared by:
%
\begin{center}
\begin{tabular}{l}
|\ifchilddoc\||else|\\
|\addtocounter{page}{-1}|\\
\textit{code for title page}\\
|\newpage|\\
|\||fi|
\end{tabular}
\end{center}
%
A banner page for the child documents can be generated by:
%
\begin{center}
\begin{tabular}{l}
|\ifchilddoc|\\
|\addtocounter{page}{-1}|\\
\textit{code for banner page}\\
|\newpage|\\
|\||fi|
\end{tabular}
\end{center}
%
Here one could write a message such as:
\begin{center}
|This is the part \childdocname{} of \childdocjob{}.|
\end{center}

%%%%%%%%%%%%%%%%%%%%%%%%%%%%%%%%%%%%%%%%%%%%%%%%%%%%%%%%%%%%%%%%%%%%%%%%%%%%%%%%
\subsection{Flags}
\label{sec:flags}

The package makes it easy to generate different versions
of the main or child documents.
To this end compilation flags can be defined
and assigned different default values.
They will be particularly useful in conjunction
with the forwarding mechanism described in \secref{sec:forward}.

For example, it may be useful to have a flag |\version|
which can be set to |draft| or |final|.
The document source will contain some conditional code
depending on the value of |\version|.
Suppose further, the flag should default to |final| for the main file
and to |draft| for child files
which is a natural assignment for editing the document.
This is achieved by placing the following code
in the preamble of the main document
(below the |\childdocmain| directive):
%
\begin{center}
\begin{tabular}{l}
|\ifchilddoc|\\
|\providecommand{\version}{draft}|\\
|\||else|\\
|\providecommand{\version}{final}|\\
|\||fi|
\end{tabular}
\end{center}
%
The definition by |\providecommand| makes sure
that previous definitions are not overwritten.
Further statements |\providecommand{\version}{...}|
can thus be added before the above code to override it.

For the main file, one might add a line
(between |\childdocmain| and the above block)
%
\begin{center}
|%\ifchilddoc\||else\providecommand{\version}{draft}\||fi|
\end{center}
%
which can be uncommented to produce a draft version.
Likewise one can add a line to the very top of a child file
(above the |\childdocof{|\textit{main}|}| directive)
%
\begin{center}
|%\providecommand{\version}{final}|
\end{center}
%
which can be uncommented to produce the final version of this child document.

%%%%%%%%%%%%%%%%%%%%%%%%%%%%%%%%%%%%%%%%%%%%%%%%%%%%%%%%%%%%%%%%%%%%%%%%%%%%%%%%
\subsection{Forwarding}
\label{sec:forward}

Different versions of the main or child documents
using compilation flags as described in \secref{sec:flags}
can be (permanently) stored in different files
for convenient compilation, viewing and distribution.
To this end, the package defines a command
to pass on compilation to a different file:

%%%%%%%%%%%%%%%%%%%%%%%%%%%%%%%%%%%%%%%%
\DescribeMacro{\childdocforward}
The command |\childdocforward| redirects processing to
another source file:
%
\begin{center}
\begin{tabular}{l}
|\input{childdoc.def}|\\
|\childdocforward[|\textit{main}|]{|\textit{dest}|}|\\
\end{tabular}
\end{center}
%
The argument \textit{dest} is the destination file
(without extension).
It should be the main file or one of the child files.
Note that further \textsf{childdoc} directives
such as |\childdocof| and |\childdocforward|
in the indicated file will be processed in this form.
The optional argument \textit{main}
passes on directly to the main file \textit{main}
while pretending to compile the child \textit{dest}.
This form behaves as if \textit{dest}
issues |\childdocof{|\textit{main}|}| right away,
and no further \textsf{childdoc} directives will be processed.

%%%%%%%%%%%%%%%%%%%%%%%%%%%%%%%%%%%%%%%%
\DescribeMacro{\...prefix}
In the alternative form |\childdocforwardprefix|,
%
\begin{center}
\begin{tabular}{l}
|\input{childdoc.def}|\\
|\childdocforwardprefix[|\textit{main}|]{|\textit{prefix}|}{|\textit{dest}|}|
\end{tabular}
\end{center}
%
the destination file is determined by a pattern
depending on the current file:
To make this work, the current file must be called
`{\textit{prefix}\hspace{0.2em}\textit{suffix}}'
with \textit{prefix} matching precisely the argument.
Processing is then passed on to the file
`{\textit{dest}\hspace{0.2em}\textit{suffix}}'.
Surely, the same effect is achieved by
directly specifying the
argument `{\textit{dest}\hspace{0.2em}\textit{suffix}}'
in the first form.
However, that requires to set up a different file
for each child. With the alternative form of the command
all these files can have exactly the same content
which simplifies setting them up and maintaining them.

For example, the following file |draft.tex|
with a compilation flag |\version| as described in \secref{sec:flags}
compiles the main document as a draft:
%
\begin{center}
\begin{tabular}{l}
|\def\version{draft}|\\
|\input{childdoc.def}|\\
|\childdocforward{|\textit{main}|}|
\end{tabular}
\end{center}
%
Likewise, the following files |final|\textit{nn}|.tex|
compile the final version of the child document
|child|\textit{nn}|.tex|:
%
\begin{center}
\begin{tabular}{l}
|\def\version{final}|\\
|\input{childdoc.def}|\\
|\childdocforwardprefix{final}{child}|
\end{tabular}
\end{center}
%

Note that when several versions of a main file and/or of each child file
are to be generated, it may be convenient to set up a |Makefile| or
shell script to automatise the process.

%%%%%%%%%%%%%%%%%%%%%%%%%%%%%%%%%%%%%%%%%%%%%%%%%%%%%%%%%%%%%%%%%%%%%%%%%%%%%%%%
\subsection{Command Line Processing}
\label{sec:commandline}

The effect of redirection files can also be achieved by invoking
the \LaTeX{} compiler with a more elaborate command line.
Most conveniently this should be done as part
of a shell script or a |Makefile|.

When using \textsf{childdoc} in the main file, the following
command lines effectively perform a redirection
(note that depending on the shell being used,
backslashes may have to be doubled: `|\|' $\to$ `|\\|'):
%
\begin{center}
|... -jobname "|\textit{target}|" |\\|"|[\textit{flags}]%
|\input{childdoc.def}\childdocforward[|\textit{main}|]{|\textit{dest}|}"|
\end{center}
%
Here \textit{target} is the name of the output file,
\textit{main} is the name of the main file
and \textit{dest} is the name of the main or child file to be processed
(all filenames without extensions).
The optional argument \textit{main} can be omitted
if \textit{main} matches \textit{dest}.
Optionally, compilation \textit{flags} can be defined via |\def| commands.
This command line makes the \TeX{} engine believe
it is compiling the file \textit{target}
whose content is specified as the latter parameter.
The provided code then forwards the processing to
\textit{main} or \textit{dest} as described in \secref{sec:forward}.

%%%%%%%%%%%%%%%%%%%%%%%%%%%%%%%%%%%%%%%%%%%%%%%%%%%%%%%%%%%%%%%%%%%%%%%%%%%%%%%%
\subsection{Include by Input}
\label{sec:input}

Including child documents by |\include| has some restrictions by design.
Most notably, the content of a child document always occupies
its own set of pages; pages cannot be shared between child documents.
Usually, this behaviour makes perfect sense
because each child document contain an essential part of the document.
However, in some situations it may be desirable to compose
a document from a collection of parts
without having mandatory page breaks between then.
For this case, the package
provides a mechanism to include parts
by |\input| which can also be processed individually.
However, by construction this mechanism
requires manual handling of the content to be output.

%%%%%%%%%%%%%%%%%%%%%%%%%%%%%%%%%%%%%%%%
\DescribeMacro{\ifchilddocmanual}
The main file should be prepared as usual, see \secref{sec:include}.
However, the document body must make a distinction
between processing of an individual part and of the main document, e.g.:
%
\begin{center}
\begin{tabular}{l}
|\ifchilddocmanual|\\
|\input{\childdocname}|\\
|\||else|\\
\textit{document body with }|\input{|\textit{part}|}|\\
|\||fi|
\end{tabular}
\end{center}
%
The conditional |\ifchilddocmanual| is true whenever
a part to be included by |\input| is being compiled,
and the name of the part is stored in |\childdocname|.

%%%%%%%%%%%%%%%%%%%%%%%%%%%%%%%%%%%%%%%%
\DescribeMacro{\childdocby}
Each part to be included by |\input| should start with:
%
\begin{center}
\begin{tabular}{l}
|\input{childdoc.def}|\\
|\childdocby{|\textit{main}|}|\\
\end{tabular}
\end{center}
%
The directive |\childdocby| is similar to |\childdocof|
described in \secref{sec:include},
but the subsequent selection of content must be done manually.
To that end, both |\ifchilddoc| and |\ifchilddocmanual|
will be true upon processing of a part,
and the name of the part is stored in |\childdocname|.
Note that |\jobname| will be set to the filename of the current part
so that each part receives an individual |.aux| file
that does not interfere with the |.aux| file(s) of the main document.
This behaviour can be altered by the alternative form
|\childdocby[*]{|\textit{main}|}| (with a non-empty optional argument)
which uses the |.aux| file of the main document
by setting |\jobname| to \textit{main}.

%%%%%%%%%%%%%%%%%%%%%%%%%%%%%%%%%%%%%%%%%%%%%%%%%%%%%%%%%%%%%%%%%%%%%%%%%%%%%%%%
\subsection{Driver Development}
\label{sec:driver}

The \textsf{childdoc} mechanism can also be use for the development
of definition files such as \LaTeX{} styles or classes.
This case differs from the above setup with multiple parts
included by |\include| in that no |\includeonly| should be invoked.
This can be achieved by starting the include file
(before |\ProvidesPackage|) with:
%
\begin{center}
\begin{tabular}{l}
|\input{childdoc.def}|\\
|\childdocforward{|\textit{main}|}|\\
\end{tabular}
\end{center}
%
or alternatively with:
%
\begin{center}
\begin{tabular}{l}
|\input{childdoc.def}|\\
|\childdocby{|\textit{main}|}|\\
\end{tabular}
\end{center}
%
Both forms have slightly different effects as described above.
The main file is prepared as usual, see \secref{sec:include}.

%%%%%%%%%%%%%%%%%%%%%%%%%%%%%%%%%%%%%%%%%%%%%%%%%%%%%%%%%%%%%%%%%%%%%%%%%%%%%%%%
\subsection{Legacy Detection}
\label{sec:detection}

The directive |\childdocmain| in the main file can detect
whether the complete document or merely a child is to be compiled
even without using the directive |\childdocof|.
This method is deprecated because it is less robust
and there is no compelling reason to use it;
it is merely provided for backward compatibility
and it may be removed in future versions.

If the detection mechanism is to be used,
it is mandatory to correctly specify
the filename of the main file as the argument of |\childdocmain|:
%
\begin{center}
\begin{tabular}{l}
|\input{childdoc.def}|\\
|\childdocmain{|\textit{main}|}|\\
\end{tabular}
\end{center}
%
If |\jobname| does not match the argument \textit{main} of |\childdocmain|,
it is assumed that |\jobname| points to the child file to be compiled.
When using |\childdocmain| with the main file specified as argument,
it suffices to start a child file
with just |\input{|\textit{main}|}|
without loading of the package and using |\childdocof|.
If instead all processing is done
with the appropriate \textsf{childdoc} directives,
the argument of \textit{main} of |\childdocmain| can be empty.

An alternative version of the command line processing described
in \secref{sec:commandline} using the detection mechanism reads:
%
\begin{center}
|... -jobname "|\textit{target}|" "|[\textit{flags}]%
[|\def\jobname{|\textit{dest}|}|]|\input{|\textit{main}|}"|
\end{center}

%%%%%%%%%%%%%%%%%%%%%%%%%%%%%%%%%%%%%%%%%%%%%%%%%%%%%%%%%%%%%%%%%%%%%%%%%%%%%%%%
\subsection{Manual Code}
\label{sec:manual}

In case one cannot be certain whether the definitions file |childdoc.def|
is installed on the target \TeX{} distribution
and one prefers not to ship it,
it is conceivable to paste a few relevant commands into the sources.

To that end, drop all statements |\input{childdoc.def}|
and perform the replacements as outlined below.
Instead of |\childdocmain{|\textit{main}|}| add the following code
to the top of the main file:
%
\begin{center}
\begin{tabular}{l}
|\||ifdefined\childdocname\endinput\||fi\newif\ifchilddoc|\\
|\edef\childdocname{\scantokens\expandafter{\jobname\noexpand}}|\\
|\def\childdocmain{|\textit{main}|}\||ifx\childdocmain\childdocname\||else|\\
|\childdoctrue\includeonly{\childdocname}\let\jobname\childdocmain\||fi|\\
\end{tabular}
\end{center}
%
Instead of |\childdocof{|\textit{main}|}| just include the main file
at the top of each child file:
%
\begin{center}
|\input{|\textit{main}|}|
\end{center}
%
A simple redirection |\childdocforward{|\textit{dest}|}| is achieved by:
%
\begin{center}
|\def\jobname{|\textit{dest}|}\input{\jobname}|
\end{center}
%
The redirection with prefix
|\childdocforwardprefix[|\textit{prefix}|]{|\textit{dest}|}|
is accomplished by:
%
\begin{center}
\begin{tabular}{l}
|{\edef\jobname{\scantokens\expandafter{\jobname\noexpand}}|\\
|\def\redirectjob |\textit{prefix}|#1~~~{\gdef\jobname{|\textit{dest}|#1}}|\\
|\expandafter\redirectjob\jobname~~~}\input{\jobname}|
\end{tabular}
\end{center}

In an alternative approach,
child documents can be compiled by a specific command line
without additional code or specific definitions:
%
\begin{center}
|... -jobname "|\textit{target}|" "|[\textit{flags}]%
|\includeonly{|\textit{dest}|}\input{|\textit{main}|}"|
\end{center}
%

%%%%%%%%%%%%%%%%%%%%%%%%%%%%%%%%%%%%%%%%%%%%%%%%%%%%%%%%%%%%%%%%%%%%%%%%%%%%%%%%
%%%%%%%%%%%%%%%%%%%%%%%%%%%%%%%%%%%%%%%%%%%%%%%%%%%%%%%%%%%%%%%%%%%%%%%%%%%%%%%%
\section{Information}

%%%%%%%%%%%%%%%%%%%%%%%%%%%%%%%%%%%%%%%%%%%%%%%%%%%%%%%%%%%%%%%%%%%%%%%%%%%%%%%%
\subsection{Copyright}

Copyright \copyright{} 2017--2018 Niklas Beisert

This work may be distributed and/or modified under the
conditions of the \LaTeX{} Project Public License, either version 1.3
of this license or (at your option) any later version.
The latest version of this license is in
  \url{http://www.latex-project.org/lppl.txt}
and version 1.3 or later is part of all distributions of \LaTeX{}
version 2005/12/01 or later.

This work has the LPPL maintenance status `maintained'.

The Current Maintainer of this work is Niklas Beisert.

This work consists of the files |README.txt|, |childdoc.ins| and |childdoc.dtx|
as well as the derived files |childdoc.def|, |cdocsamp.tex|
with |cdocsch1.tex|, |cdocsch2.tex|, |cdocspt3.tex|, |cdocspt4.tex|,
|cdocsdrf.tex|, |cdocsfn1.tex|, |cdocsfn2.tex|
as well as |childdoc.pdf|.

%%%%%%%%%%%%%%%%%%%%%%%%%%%%%%%%%%%%%%%%%%%%%%%%%%%%%%%%%%%%%%%%%%%%%%%%%%%%%%%%
\subsection{Files and Installation}

The package consists of the files:
%
\begin{center}
\begin{tabular}{ll}
    |README.txt|   & readme file \\
    |childdoc.ins| & installation file \\
    |childdoc.dtx| & source file \\
    |childdoc.def| & definition file \\
    |cdocsamp.tex| & sample main file \\
    |cdocsch1.tex| & sample include file \\
    |cdocsch2.tex| & sample include file \\
    |cdocspt3.tex| & sample part file \\
    |cdocspt4.tex| & sample part file \\
    |cdocsdrf.tex| & sample redirection file \\
    |cdocsfn1.tex| & sample redirection file \\
    |cdocsfn2.tex| & sample redirection file \\
    |childdoc.pdf| & manual
\end{tabular}
\end{center}
%
The distribution consists of the files
|README.txt|, |childdoc.ins| and |childdoc.dtx|.
%
\begin{itemize}
\item
Run (pdf)\LaTeX{} on |childdoc.dtx|
to compile the manual |childdoc.pdf| (this file).
\item
Run \LaTeX{} on |childdoc.ins| to create the definitions file |childdoc.def|
and the sample |cdocsamp.tex| with include files
|cdocsch1.tex|, |cdocsch2.tex|, |cdocspt3.tex|, |cdocspt4.tex|,
|cdocsdrf.tex|, |cdocsfn1.tex|, |cdocsfn2.tex|.
Then copy the file |childdoc.def| to an appropriate directory of your \LaTeX{}
distribution, e.g.\ \textit{texmf-root}|/tex/latex/childdoc|.
\end{itemize}

%%%%%%%%%%%%%%%%%%%%%%%%%%%%%%%%%%%%%%%%%%%%%%%%%%%%%%%%%%%%%%%%%%%%%%%%%%%%%%%%
\subsection{Related CTAN Packages}

There are several other packages which offer a similar functionality:
%
\begin{itemize}
\item
The packages
\href{http://ctan.org/pkg/docmute}{\textsf{docmute}},
\href{http://ctan.org/pkg/includex}{\textsf{includex}} and
\href{http://ctan.org/pkg/standalone}{\textsf{standalone}}
provide commands to include only the document body of
a child file thus allowing both files to be compiled individually.
\item
The packages \href{http://ctan.org/pkg/subdocs}{\textsf{subdocs}}
and \href{http://ctan.org/pkg/subfiles}{\textsf{subfiles}}
provide structures in which the main and child documents can be
encapsulated and allowing them to be compiled individually.
The inclusion mechanism is different from the conventional |\include|.
\item
The package \href{http://ctan.org/pkg/combine}{\textsf{combine}}
is an elaborate solution to combine several documents into one.
\end{itemize}
%
See also the CTAN topic \href{http://ctan.org/topic/subdocs}{\textsf{subdocs}}
for further related packages.
The present package differs from the above solutions in that
a document structure constructed with the conventional |\include| mechanism
just needs two extra commands at the top of every file
such that all constituent files can be compiled individually.

%%%%%%%%%%%%%%%%%%%%%%%%%%%%%%%%%%%%%%%%%%%%%%%%%%%%%%%%%%%%%%%%%%%%%%%%%%%%%%%%
%\subsection{Feature Suggestions}
%
%The following is a list of features which may be useful for future
%versions of this package:
%%
%\begin{itemize}
%\item
%\ldots
%\end{itemize}

%%%%%%%%%%%%%%%%%%%%%%%%%%%%%%%%%%%%%%%%%%%%%%%%%%%%%%%%%%%%%%%%%%%%%%%%%%%%%%%%
\subsection{Revision History}

%%%%%%%%%%%%%%%%%%%%%%%%%%%%%%%%%%%%%%%%
\paragraph{v2.0:} 2018/12/30

\begin{itemize}
\item
immediate forward processing
\item
added |\childdocby| mechanism
\item
manual restructured
\end{itemize}

%%%%%%%%%%%%%%%%%%%%%%%%%%%%%%%%%%%%%%%%
\paragraph{v1.6:} 2018/01/17

\begin{itemize}
\item
application for development of include files
\item
corrections to manual
\end{itemize}

%%%%%%%%%%%%%%%%%%%%%%%%%%%%%%%%%%%%%%%%
\paragraph{v1.5:} 2017/05/21

\begin{itemize}
\item
more complete structuring introduced
\item
|\childdocof| introduced
\item
|\childdoc| renamed to |\childdocmain|
\item
|\childredirect| renamed to |\childdocforward| and |\childdocforwardprefix|
and functionality expanded
\end{itemize}

%%%%%%%%%%%%%%%%%%%%%%%%%%%%%%%%%%%%%%%%
\paragraph{v1.0:} 2017/04/27

\begin{itemize}
\item
manual and install package
\item
first version published on CTAN
\end{itemize}

%%%%%%%%%%%%%%%%%%%%%%%%%%%%%%%%%%%%%%%%
\paragraph{v0.6:} 2017/04/26

\begin{itemize}
\item
redirection mechanism added
\end{itemize}

%%%%%%%%%%%%%%%%%%%%%%%%%%%%%%%%%%%%%%%%
\paragraph{v0.5:} 2017/04/26

\begin{itemize}
\item
functionality in definition file
\end{itemize}


%%%%%%%%%%%%%%%%%%%%%%%%%%%%%%%%%%%%%%%%%%%%%%%%%%%%%%%%%%%%%%%%%%%%%%%%%%%%%%%%
%%%%%%%%%%%%%%%%%%%%%%%%%%%%%%%%%%%%%%%%%%%%%%%%%%%%%%%%%%%%%%%%%%%%%%%%%%%%%%%%
%%%%%%%%%%%%%%%%%%%%%%%%%%%%%%%%%%%%%%%%%%%%%%%%%%%%%%%%%%%%%%%%%%%%%%%%%%%%%%%%
\appendix

\settowidth\MacroIndent{\rmfamily\scriptsize 000\ }

 \DocInput{childdoc.dtx}

\end{document}
%</driver>
% \fi
%
% %%%%%%%%%%%%%%%%%%%%%%%%%%%%%%%%%%%%%%%%%%%%%%%%%%%%%%%%%%%%%%%%%%%%%%%%%%%%%%
% %%%%%%%%%%%%%%%%%%%%%%%%%%%%%%%%%%%%%%%%%%%%%%%%%%%%%%%%%%%%%%%%%%%%%%%%%%%%%%
% \section{Sample}
%\iffalse
%<*samplemain>
%\fi
%
% The following presents a sample document
% with two chapters, two parts, a title page,
% a compile flag as well as three forwarding files to set the flag.
% It consists of eight |.tex| files:
% \begin{center}
% \begin{tabular}{ll}
% |cdocsamp.tex|&main file\\
% |cdocsch1.tex|&include file for chapter 1\\
% |cdocsch2.tex|&include file for chapter 2\\
% |cdocspt3.tex|&include file for part 3\\
% |cdocspt4.tex|&include file for part 4\\
% |cdocsdrf.tex|&forwarding file for main file in draft mode\\
% |cdocsfi1.tex|&forwarding file for final version of chapter 1\\
% |cdocsfi2.tex|&forwarding file for final version of chapter 2\\
% \end{tabular}
% \end{center}
% Each of the eight files can be compiled directly by the \LaTeX{} compiler.
%
% %%%%%%%%%%%%%%%%%%%%%%%%%%%%%%%%%%%%%%
% \paragraph{Main File.}
%
% The main file is called |cdocsamp.tex|.
%
% Load the \textsf{childdoc} definitions and
% declare the filename for the main document:
%    \begin{macrocode}
\input{childdoc.def}
\childdocmain{}
%    \end{macrocode}

% Optional override for |\version| flag:
%    \begin{macrocode}
%%\ifchilddoc\else\providecommand{\version}{draft}\fi
%    \end{macrocode}

% Define the default values for the |\version| flag
% (|final| for the main file and |draft| for childs):
%    \begin{macrocode}
\ifchilddoc
\providecommand{\version}{draft}
\else
\providecommand{\version}{final}
\fi
%    \end{macrocode}

% Load the standard document class:
%    \begin{macrocode}
\documentclass[12pt]{article}
%    \end{macrocode}

% Start the document body:
%    \begin{macrocode}
\begin{document}
%    \end{macrocode}

% Declare a title page.
% Print title, part of document being processed and version flag:
%    \begin{macrocode}
\addtocounter{page}{-1}
\begin{center}
{\LARGE\bfseries{}childdoc example\par}
\vspace{1cm}
\ifchilddoc
\ifchilddocmanual part\else chapter\fi:
`\childdocname' of `\childdocjob'\par
\else
main document: `\childdocjob'\par
\fi
version: \version\par
\end{center}
\newpage
%    \end{macrocode}

% Manually include selected file,
% otherwise process as usual:
%    \begin{macrocode}
\ifchilddocmanual
\section*{part `\childdocname'}
\input{\childdocname}
\else
%    \end{macrocode}

% Include the two chapters:
%    \begin{macrocode}
\include{cdocsch1}
\include{cdocsch2}
%    \end{macrocode}

% Include the two parts unless only chapters should be displayed:
%    \begin{macrocode}
\ifchilddoc\else
\section{part three}
\input{cdocspt3}
\section{part four}
\input{cdocspt4}
\fi
%    \end{macrocode}

% Process as usual until here:
%    \begin{macrocode}
\fi
%    \end{macrocode}

% End of document body:
%    \begin{macrocode}
\end{document}
%    \end{macrocode}
%\iffalse
%</samplemain>
%\fi
%
% %%%%%%%%%%%%%%%%%%%%%%%%%%%%%%%%%%%%%%
% \paragraph{Chapter Include Files.}
%
% The include files are called |cdocsch1.tex| and |cdocsch2.tex|.
%
%\iffalse
%<*samplechap1|samplechap2>
%\fi

% Optional override for |\version| flag:
%    \begin{macrocode}
%%\providecommand{\version}{final}
%    \end{macrocode}

% Include the main document:
%    \begin{macrocode}
\input{childdoc.def}
\childdocof{cdocsamp}
%    \end{macrocode}

%\iffalse
%</samplechap1|samplechap2>
%\fi
%
%\iffalse
%<*samplechap1>
%\fi
% Some text for chapter 1:
%    \begin{macrocode}
\section{one}
some text in chapter one
%    \end{macrocode}

%\iffalse
%</samplechap1>
%\fi
% Some text for chapter 2:
%\iffalse
%<*samplechap2>
%\fi
%    \begin{macrocode}
\section{two}
more text in chapter two
%    \end{macrocode}

%\iffalse
%</samplechap2>
%\fi
%
% %%%%%%%%%%%%%%%%%%%%%%%%%%%%%%%%%%%%%%
% \paragraph{Part Include Files.}
%
% The include files are called |cdocspt3.tex| and |cdocspt4.tex|.
%
%\iffalse
%<*samplepart3|samplepart4>
%\fi

% Optional override for |\version| flag:
%    \begin{macrocode}
%%\providecommand{\version}{final}
%    \end{macrocode}

% Include the main document:
%    \begin{macrocode}
\input{childdoc.def}
\childdocby{cdocsamp}
%    \end{macrocode}

%\iffalse
%</samplepart3|samplepart4>
%\fi
%
%\iffalse
%<*samplepart3>
%\fi
% Some text for part 3:
%    \begin{macrocode}
some text in part three
%    \end{macrocode}

%\iffalse
%</samplepart3>
%\fi
% Some text for part 4:
%\iffalse
%<*samplepart4>
%\fi
%    \begin{macrocode}
more text in part four
%    \end{macrocode}

%\iffalse
%</samplepart4>
%\fi
%
% %%%%%%%%%%%%%%%%%%%%%%%%%%%%%%%%%%%%%%
% \paragraph{Forwarding for a Complete Draft.}
%
% The following forwarding file |cdocsdrf.tex|
% compiles the main document in draft mode:
%\iffalse
%<*sampledraft>
%\fi
%    \begin{macrocode}
\def\version{draft}
\input{childdoc.def}
\childdocforward{cdocsamp}
%    \end{macrocode}

%\iffalse
%</sampledraft>
%\fi
%
% %%%%%%%%%%%%%%%%%%%%%%%%%%%%%%%%%%%%%%
% \paragraph{Forwarding for Final Version of the Chapters.}
%
% The following forwarding files |cdocsfn1.tex| and |cdocsfn2.tex|
% (with identical content)
% compile the final versions of the child documents
% |cdocsch1.tex| and |cdocsch2.tex|, respectively:
%\iffalse
%<*samplefinal>
%\fi
%    \begin{macrocode}
\def\version{final}
\input{childdoc.def}
\childdocforwardprefix[cdocsamp]{cdocsfn}{cdocsch}
%    \end{macrocode}

%\iffalse
%</samplefinal>
%\fi
%
% %%%%%%%%%%%%%%%%%%%%%%%%%%%%%%%%%%%%%%
% \paragraph{Command Line Processing.}
%
% The following three command lines generate the output files
% |cdocscld|, |cdocscl1| and |cdocscl2|
% which should be identical to
% |cdocsdrf|, |cdocsch1| and |cdocsfn2|, respectively:
% \begin{center}
% \begin{tabular}{l}
% |latex -jobname cdocscld \|\\
% |  "\def\version{draft}\input{childdoc.def}\childdocforward{cdocsamp}"|\\
% |latex -jobname cdocscl1 \|\\
% |  "\input{childdoc.def}\childdocforward[cdocsamp]{cdocsch1}"|\\
% |latex -jobname cdocscl2 \|\\
% |  "\def\version{final}\input{childdoc.def}\childdocforward{cdocsch2}"|
% \end{tabular}
% \end{center}
% Note that the trailing backslash on each first line
% merely continues the input to the second line
% (for convenient cut ant paste).
% Furthermore, the command |latex| can be replaced by any
% of its alternative versions such as |pdflatex|.
%
% %%%%%%%%%%%%%%%%%%%%%%%%%%%%%%%%%%%%%%%%%%%%%%%%%%%%%%%%%%%%%%%%%%%%%%%%%%%%%%
% %%%%%%%%%%%%%%%%%%%%%%%%%%%%%%%%%%%%%%%%%%%%%%%%%%%%%%%%%%%%%%%%%%%%%%%%%%%%%%
% \section{Implementation}
%\iffalse
%<*package>
%\fi
%
% This section describes the definitions file |childdoc.def|.

% The definitions cannot be loaded using |\usepackage| or |\RequirePackage|
% which has a mechanism to prevent loading a style file more than once.
% When loading the definitions by means of |\input|
% multiple instances have to be prevented manually:
%\iffalse
%This code needs to be before the `\ProvidesFile' directive
%which is defined at the beginning of this file.
%Therefore it is also placed there and commented out here.
%</package>
%<*discard>
%\fi
%    \begin{macrocode}
\ifdefined\childdocmain\endinput\fi
%    \end{macrocode}
%\iffalse
%</discard>
%<*package>
%\fi
%
% \macro{\ifchilddoc}
% \macro{\ifchilddocmanual}
% The conditional |\ifchilddoc| tells whether a
% child (true) or main (false) document is being compiled.
% The conditional |\ifchilddocmanual| tells whether
% the |\includeonly| mechanism is used (false) or
% the selection of child files must be performed manually (true).
% The definitions initialise to false:
%    \begin{macrocode}
\newif\ifchilddoc
\newif\ifchilddocmanual
%    \end{macrocode}

% \macro{\childdocname}
% \macro{\childdocjob}
% The macro |\childdocname| stores the name of the main document
% to be compiled. The macro |\childdocjob| stores the name of
% the document on which the \LaTeX{} compiler was originally invoked.
% The content of |\jobname| cannot be compared
% to filenames specified in the source due to different catcodes.
% The following code rescans |\jobname|, stores the result
% in |\childdocname| and saves a copy in |\childdocjob|:
%    \begin{macrocode}
\edef\childdocname{\scantokens\expandafter{\jobname\noexpand}}
\let\childdocjob\childdocname
%    \end{macrocode}

% \macro{\childdocdisable}
% The macro |\childdocdisable| prevents the main file
% from being processed more than once.
% At this stage, the main document command |\childdocmain|
% is assumed to be called once again where it should do nothing.
% Any subsequent call to it should prevent
% a secondary processing of the main document
% It overwrites the forwarding commands
% |\childdocof| and |\childdocforward|
% with empty macros to prevent further inclusions of the main document:
%    \begin{macrocode}
\newcommand{\childdocdisable}
{
  \renewcommand{\childdocmain}[1]{\renewcommand{\childdocmain}[1]{\endinput}}
  \renewcommand{\childdocof}[1]{}
  \renewcommand{\childdocby}[2][]{}
  \renewcommand{\childdocforward}[2][]{}
  \renewcommand{\childdocdisable}{}
}
%    \end{macrocode}

% \macro{\childdocmain}
% The macro |\childdocmain| is to be called at the top of the main file
% with nothing or the main filename (without extension) as argument.
% First, it breaks loops.
% If the argument is not empty and does not match |\childdocname|
% (which is set by the first inclusion of |childdoc.def|),
% |\ifchilddoc| is set to true, |\includeonly| is applied to the child file
% and |\jobname| is set to the main file
% (for proper handling of |.aux| files):
%    \begin{macrocode}
\newcommand{\childdocmain}[1]
{
  \childdocdisable\childdocmain{}
  \if?#1?\else
    \begingroup
      \def\childdoctmp{#1}
      \ifx\childdoctmp\childdocname
        \def\childdoctmp{}
      \else
        \def\childdoctmp
        {
          \childdoctrue
          \includeonly{\childdocname}
          \def\childdocjob{#1}
          \def\jobname{#1}
        }
      \fi
      \expandafter
    \endgroup
    \childdoctmp
  \fi
}
%    \end{macrocode}

% \macro{\childdocof}
% The command |\childdocof| redirects
% compilation to the main file |#1|.
%    \begin{macrocode}
\newcommand{\childdocof}[1]
{
  \childdocdisable
  \childdoctrue
  \includeonly{\childdocname}
  \def\jobname{#1}
  \def\childdocjob{#1}
  \input{#1}
}
%    \end{macrocode}

% \macro{\childdocby}
% The command |\childdocby| ....
%    \begin{macrocode}
\newcommand{\childdocby}[2][]
{
  \childdocdisable
  \childdoctrue
  \childdocmanualtrue
  \if?#1?\else
    \def\jobname{#2}
  \fi
  \def\childdocjob{#2}
  \input{#2}
  \endinput
}
%    \end{macrocode}

% \macro{\childdocforward}
% The command |\childdocforward| redirects
% compilation to the main file or
% (if the optional argument is given) a child file.
% Parameters are set as if the main file
% or a child file starting with |\childdocof| was compiled.
% Then compilation is handed over to the main file:
%    \begin{macrocode}
\newcommand{\childdocforward}[2][]
{
  \begingroup
    \if?#1?
      \def\childdoctmp
      {
        \def\childdocname{#2}
        \def\childdocjob{#2}
        \def\jobname{#2}
        \input{#2}
        \endinput
      }
    \else
      \def\childdoctmp
      {
        \childdocdisable
        \def\childdocname{#2}
        \childdoctrue
        \includeonly{#2}
        \def\childdocjob{#1}
        \def\jobname{#1}
        \input{#1}
        \endinput
      }
    \fi
    \expandafter
  \endgroup
  \childdoctmp
}
%    \end{macrocode}

% \macro{\childdocforwardprefix}
% The command |\childdocforwardprefix| redirects
% compilation to the main or a child file by means of a pattern.
% The prefix |#1| in the current filename is replaced by |#2|
% and the suffix of the current filename is kept
% (it is assumed that the filename does not contain the substring `|~~~|'
% which is used as a delimiter).
% Compilation is handed over to the new file by |\childdocforward|:
%    \begin{macrocode}
\newcommand{\childdocforwardprefix}[3][]
{
  \begingroup
    \def\childdocextract #2##1~~~{\def\childdoctmp{\childdocforward[#1]{#3##1}}}
    \expandafter\childdocextract\childdocname~~~
    \expandafter
  \endgroup
  \childdoctmp
}
%    \end{macrocode}

% \macro{\childdoc}
% The deprecated macro |\childdoc| is a legacy version of |\childdocmain|:
%    \begin{macrocode}
\newcommand{\childdoc}{\childdocmain}
%    \end{macrocode}

% \macro{\childdocredirect}
% The deprecated macro |\childdocredirect| is a legacy version
% of |\childdocforward| and |\childdocforwardprefix|:
%    \begin{macrocode}
\newcommand{\childdocredirect}[2][]
{
  \begingroup
    \if?#1?
      \def\childdoctmp{\childdocforward{#2}}
    \else
      \def\childdoctmp{\childdocforwardprefix{#1}{#2}}
    \fi
    \expandafter
  \endgroup
  \childdoctmp
}
%    \end{macrocode}

%\iffalse
%</package>
%\fi
%
\endinput
|\\
|\childdocforward[|\textit{main}|]{|\textit{dest}|}|\\
\end{tabular}
\end{center}
%
The argument \textit{dest} is the destination file
(without extension).
It should be the main file or one of the child files.
Note that further \textsf{childdoc} directives
such as |\childdocof| and |\childdocforward|
in the indicated file will be processed in this form.
The optional argument \textit{main}
passes on directly to the main file \textit{main}
while pretending to compile the child \textit{dest}.
This form behaves as if \textit{dest}
issues |\childdocof{|\textit{main}|}| right away,
and no further \textsf{childdoc} directives will be processed.

%%%%%%%%%%%%%%%%%%%%%%%%%%%%%%%%%%%%%%%%
\DescribeMacro{\...prefix}
In the alternative form |\childdocforwardprefix|,
%
\begin{center}
\begin{tabular}{l}
|% \iffalse
%
% childdoc.dtx Copyright (C) 2017-2018 Niklas Beisert
%
% This work may be distributed and/or modified under the
% conditions of the LaTeX Project Public License, either version 1.3
% of this license or (at your option) any later version.
% The latest version of this license is in
%   http://www.latex-project.org/lppl.txt
% and version 1.3 or later is part of all distributions of LaTeX
% version 2005/12/01 or later.
%
% This work has the LPPL maintenance status `maintained'.
%
% The Current Maintainer of this work is Niklas Beisert.
%
% This work consists of the files childdoc.dtx and childdoc.ins
% and the derived files childdoc.def and cdocsamp.tex with
% cdocsch1.tex, cdocsch2.tex, cdocsdrf.tex, cdocsfn1.tex, cdocsfn2.tex.
%
%<package>\ifdefined\childdocmain\endinput\fi
%<package>\ProvidesFile{childdoc.def}[2018/12/30 v2.0 child document driver]
%<samplemain>\ProvidesFile{cdocsamp.tex}[2018/12/30 v2.0 sample for childdoc]
%<*driver>
%\ProvidesFile{childdoc.drv}[2018/12/30 v2.0 childdoc reference manual file]
\PassOptionsToClass{10pt,a4paper}{article}
\documentclass{ltxdoc}

\usepackage[margin=35mm]{geometry}
\usepackage{hyperref}
\usepackage{hyperxmp}
\usepackage[usenames]{color}

\hypersetup{colorlinks=true}
\hypersetup{pdfstartview=FitH}
\hypersetup{pdfpagemode=UseNone}
\hypersetup{pdfsource={}}
\hypersetup{pdflang={en-UK}}
\hypersetup{pdfcopyright={Copyright 2017-2018 Niklas Beisert.
  This work may be distributed and/or modified under the
  conditions of the LaTeX Project Public License, either version 1.3
  of this license or (at your option) any later version.}}
\hypersetup{pdflicenseurl={http://www.latex-project.org/lppl.txt}}
\hypersetup{pdfcontactaddress={ETH Zurich, ITP, HIT K,
  Wolfgang-Pauli-Strasse 27}}
\hypersetup{pdfcontactpostcode={8093}}
\hypersetup{pdfcontactcity={Zurich}}
\hypersetup{pdfcontactcountry={Switzerland}}
\hypersetup{pdfcontactemail={nbeisert@itp.phys.ethz.ch}}
\hypersetup{pdfcontacturl={http://people.phys.ethz.ch/\xmptilde nbeisert/}}

\newcommand{\secref}[1]{\hyperref[#1]{section \ref*{#1}}}

\parskip1ex
\parindent0pt
\let\olditemize\itemize
\def\itemize{\olditemize\parskip0pt}

\begin{document}

\title{The \textsf{childdoc} Package}
\hypersetup{pdftitle={The childdoc Package}}
\author{Niklas Beisert\\[2ex]
  Institut f\"ur Theoretische Physik\\
  Eidgen\"ossische Technische Hochschule Z\"urich\\
  Wolfgang-Pauli-Strasse 27, 8093 Z\"urich, Switzerland\\[1ex]
  \href{mailto:nbeisert@itp.phys.ethz.ch}
  {\texttt{nbeisert@itp.phys.ethz.ch}}}
\hypersetup{pdfauthor={Niklas Beisert}}
\hypersetup{pdfsubject={Manual for the LaTeX2e Package childdoc}}
\date{30 December 2018, \textsf{v2.0}}
\maketitle

\begin{abstract}\noindent
\textsf{childdoc} is a \LaTeXe{} package
that enables the direct compilation
of document sections included by |\include|
to individual files.
\end{abstract}

\begingroup
\parskip0ex
\tableofcontents
\endgroup

%%%%%%%%%%%%%%%%%%%%%%%%%%%%%%%%%%%%%%%%%%%%%%%%%%%%%%%%%%%%%%%%%%%%%%%%%%%%%%%%
%%%%%%%%%%%%%%%%%%%%%%%%%%%%%%%%%%%%%%%%%%%%%%%%%%%%%%%%%%%%%%%%%%%%%%%%%%%%%%%%
\section{Introduction}

\LaTeX{} provides a mechanism to structure a large document (such as a book)
into a main file and several child files (containing the chapters)
using the |\include| command.
This mechanism is beneficial for documents
which span hundreds of pages in order to
make the source file(s) more manageable.
Moreover, compilation can be restricted to
selected child files by means of the |\includeonly| command.
The latter feature can be used to reduce the compilation time while editing
(this was significantly more useful in the earlier days of \LaTeX{})
or to generate a smaller document which is easier to navigate.
Another application of |\includeonly| is to generate
documents consisting of selected parts of the complete document.

However, there are a few drawbacks of the plain |\include| mechanism:
\begin{itemize}
\item
The child files cannot be compiled on their own,
they can only be compiled via the main file.
A naive editing environment
(such as a text editor with an option
to have the current file processed by \LaTeX)
may require one to switch to the main file before compiling;
attempting to compile the child file produces errors.
\item
The main file must be modified (each time)
to adjust the |\includeonly| command
to the present needs. This easily leaves the main file in a messy state.
\item
The generated document will always carry the filename
of the main document. This is inconvenient if
several child files are to be compiled and
to be kept for distribution.
\end{itemize}

The present package provides a simple interface
to make child files individually compilable by \LaTeX{}.
Compiling a child file then has the same effect as compiling
the main file with an |\includeonly| command
to select the appropriate child.
Moreover the generated document will carry the name of the child
rather than the main file.
This resolves all three above issues.

This feature is meant to make the editing of books,
thesis documents and lecture notes somewhat more convenient.
However, the package can also be used efficiently for
composing a series of documents (such as exercise sheets)
which are typically distributed individually.
It then assists the author in generating the individual documents
(potentially in different versions)
as well as a document containing the collected series.
Another application is in developing style files
or other kinds of included material
where compilation of the style file could redirect
to a sample or test file.

%%%%%%%%%%%%%%%%%%%%%%%%%%%%%%%%%%%%%%%%%%%%%%%%%%%%%%%%%%%%%%%%%%%%%%%%%%%%%%%%
%%%%%%%%%%%%%%%%%%%%%%%%%%%%%%%%%%%%%%%%%%%%%%%%%%%%%%%%%%%%%%%%%%%%%%%%%%%%%%%%
\section{Usage}

First of all, the package \textsf{childdoc} is \emph{not} a standard
\LaTeXe{} |.sty| style file! Therefore it needs to be invoked in
a non-standard way.

%%%%%%%%%%%%%%%%%%%%%%%%%%%%%%%%%%%%%%%%%%%%%%%%%%%%%%%%%%%%%%%%%%%%%%%%%%%%%%%%
\subsection{Included Files}
\label{sec:include}

%%%%%%%%%%%%%%%%%%%%%%%%%%%%%%%%%%%%%%%%
\DescribeMacro{\childdocmain}
To use the package, add the commands
\begin{center}
\begin{tabular}{l}
|\input{childdoc.def}|\\
|\childdocmain{}|\\
\end{tabular}
\end{center}
at the very top of the main \LaTeX{} file,
in particular \emph{before} the |\documentclass| statement!
The argument of |\childdocmain| should be left empty
(but it must be present).

%%%%%%%%%%%%%%%%%%%%%%%%%%%%%%%%%%%%%%%%
\DescribeMacro{\childdocof}
Furthermore, add the commands
\begin{center}
\begin{tabular}{l}
|\input{childdoc.def}|\\
|\childdocof{|\textit{main}|}|\\
\end{tabular}
\end{center}
at the top of every child file \textit{child}
which is included by |\include{|\textit{child}|}|
from within the main file
(or at least for those files to be compiled individually).
The argument \textit{main} must be the filename of the main file.

There are a couple of
considerations in setting up the main and child documents:

%%%%%%%%%%%%%%%%%%%%%%%%%%%%%%%%%%%%%%%%
\paragraph{Restrictions.}

Please note the following restrictions:
\begin{itemize}
\item
|\childdocmain| must be called with one argument \textit{main}
to ensure compatibility with earlier version of the package.
It must either be empty (|\childdocmain{}|)
or precisely match the filename of the main file in which it is specified.
See \secref{sec:detection} for further information.
\item
The filename \textit{main} must be specified without the |.tex| extension.
\item
The filename \textit{main} is case sensitive
(even in case-insensitive file systems)
due to internal string comparison.
\item
The argument \textit{main} should be fully expanded, it cannot be a macro.
\item
Subdirectories and special characters should be avoided in filenames.
\item
The command |\childdocmain{|\textit{main}|}| must be followed by a whitespace.
It should not be followed immediately by another command
or by a comment mark `|%|'.
This is because the \TeX{} parser reads the token immediately following
the argument of |\childdocmain| and puts it
at the beginning of every child section;
however, a white\-space is ignored.
\end{itemize}

%%%%%%%%%%%%%%%%%%%%%%%%%%%%%%%%%%%%%%%%
\paragraph{Content of Main File.}

It is advisable to place all content in the child files included by |\include|.
Any output contained in the main file will appear in all child documents
unless suppressed manually;
it cannot be suppressed automatically by the |\includeonly| directive
and thus should normally be avoided.
A method to include some content in the main file
by means of conditional processing is described in \secref{sec:conditional}.

%%%%%%%%%%%%%%%%%%%%%%%%%%%%%%%%%%%%%%%%
\paragraph{Page Numbering.}

When only a part of the document is compiled,
the appropriate numbering of pages
(as well as other status parameters)
is determined from the |.aux| files.
The latter contain information from previous passes.
However this information needs to propagate through
all intermediate child documents.
Therefore the page numbering in child documents may well
be inconsistent until the complete document is compiled at least once.

A useful (if unconventional) way to always ensure a consistent
page numbering is to restart the numbering in each child document
and denote the pages by `\textit{child}|.|\textit{page}'
where \textit{child} represents the chapter/section number of the child file.
This can be achieved by the command
|\numberwithin{page}{|\textit{child}|}|
of the \textsf{amsmath} package
where \textit{child} can be |chapter| or |section|
depending on the chosen structuring.
Alternatively, one can modify the macro |\thepage| appropriately
and reset the counter |page| at the start of each child file.

%%%%%%%%%%%%%%%%%%%%%%%%%%%%%%%%%%%%%%%%%%%%%%%%%%%%%%%%%%%%%%%%%%%%%%%%%%%%%%%%
\subsection{Conditional Processing}
\label{sec:conditional}

The package provides a mechanism to compile different versions
of a document. To customise the versions further some conditional processing
can come in handy to distinguish which version is being compiled.
The package provides two macros to describe the compilation context:

%%%%%%%%%%%%%%%%%%%%%%%%%%%%%%%%%%%%%%%%
\DescribeMacro{\ifchilddoc}
The conditional |\ifchilddoc| distinguishes between the compilation of
child documents and the main document:
%
\begin{center}
|\ifchilddoc |\textit{child-code}| |[|\||else |\textit{main-code}]| \||fi|
\end{center}

%%%%%%%%%%%%%%%%%%%%%%%%%%%%%%%%%%%%%%%%
\DescribeMacro{\childdocname}
\DescribeMacro{\childdocjob}
The macro |\childdocname| contains the filename (without extension)
of the main or child file being processed.
Note that |\childdocjob| will always contain the name of the main file.

%%%%%%%%%%%%%%%%%%%%%%%%%%%%%%%%%%%%%%%%
\paragraph{Title Page.}

Conditional processing can be used to include a title or banner page
in the main document when proper precautions are taken.
Importantly, the code in the main file should ensure that the page counter
(as well as other status parameters which are stored in the |.aux| files)
takes the same value after the conditional processing.
Otherwise the page numbers may take divergent values
depending on which part is compiled.

For example, a title page could be declared by:
%
\begin{center}
\begin{tabular}{l}
|\ifchilddoc\||else|\\
|\addtocounter{page}{-1}|\\
\textit{code for title page}\\
|\newpage|\\
|\||fi|
\end{tabular}
\end{center}
%
A banner page for the child documents can be generated by:
%
\begin{center}
\begin{tabular}{l}
|\ifchilddoc|\\
|\addtocounter{page}{-1}|\\
\textit{code for banner page}\\
|\newpage|\\
|\||fi|
\end{tabular}
\end{center}
%
Here one could write a message such as:
\begin{center}
|This is the part \childdocname{} of \childdocjob{}.|
\end{center}

%%%%%%%%%%%%%%%%%%%%%%%%%%%%%%%%%%%%%%%%%%%%%%%%%%%%%%%%%%%%%%%%%%%%%%%%%%%%%%%%
\subsection{Flags}
\label{sec:flags}

The package makes it easy to generate different versions
of the main or child documents.
To this end compilation flags can be defined
and assigned different default values.
They will be particularly useful in conjunction
with the forwarding mechanism described in \secref{sec:forward}.

For example, it may be useful to have a flag |\version|
which can be set to |draft| or |final|.
The document source will contain some conditional code
depending on the value of |\version|.
Suppose further, the flag should default to |final| for the main file
and to |draft| for child files
which is a natural assignment for editing the document.
This is achieved by placing the following code
in the preamble of the main document
(below the |\childdocmain| directive):
%
\begin{center}
\begin{tabular}{l}
|\ifchilddoc|\\
|\providecommand{\version}{draft}|\\
|\||else|\\
|\providecommand{\version}{final}|\\
|\||fi|
\end{tabular}
\end{center}
%
The definition by |\providecommand| makes sure
that previous definitions are not overwritten.
Further statements |\providecommand{\version}{...}|
can thus be added before the above code to override it.

For the main file, one might add a line
(between |\childdocmain| and the above block)
%
\begin{center}
|%\ifchilddoc\||else\providecommand{\version}{draft}\||fi|
\end{center}
%
which can be uncommented to produce a draft version.
Likewise one can add a line to the very top of a child file
(above the |\childdocof{|\textit{main}|}| directive)
%
\begin{center}
|%\providecommand{\version}{final}|
\end{center}
%
which can be uncommented to produce the final version of this child document.

%%%%%%%%%%%%%%%%%%%%%%%%%%%%%%%%%%%%%%%%%%%%%%%%%%%%%%%%%%%%%%%%%%%%%%%%%%%%%%%%
\subsection{Forwarding}
\label{sec:forward}

Different versions of the main or child documents
using compilation flags as described in \secref{sec:flags}
can be (permanently) stored in different files
for convenient compilation, viewing and distribution.
To this end, the package defines a command
to pass on compilation to a different file:

%%%%%%%%%%%%%%%%%%%%%%%%%%%%%%%%%%%%%%%%
\DescribeMacro{\childdocforward}
The command |\childdocforward| redirects processing to
another source file:
%
\begin{center}
\begin{tabular}{l}
|\input{childdoc.def}|\\
|\childdocforward[|\textit{main}|]{|\textit{dest}|}|\\
\end{tabular}
\end{center}
%
The argument \textit{dest} is the destination file
(without extension).
It should be the main file or one of the child files.
Note that further \textsf{childdoc} directives
such as |\childdocof| and |\childdocforward|
in the indicated file will be processed in this form.
The optional argument \textit{main}
passes on directly to the main file \textit{main}
while pretending to compile the child \textit{dest}.
This form behaves as if \textit{dest}
issues |\childdocof{|\textit{main}|}| right away,
and no further \textsf{childdoc} directives will be processed.

%%%%%%%%%%%%%%%%%%%%%%%%%%%%%%%%%%%%%%%%
\DescribeMacro{\...prefix}
In the alternative form |\childdocforwardprefix|,
%
\begin{center}
\begin{tabular}{l}
|\input{childdoc.def}|\\
|\childdocforwardprefix[|\textit{main}|]{|\textit{prefix}|}{|\textit{dest}|}|
\end{tabular}
\end{center}
%
the destination file is determined by a pattern
depending on the current file:
To make this work, the current file must be called
`{\textit{prefix}\hspace{0.2em}\textit{suffix}}'
with \textit{prefix} matching precisely the argument.
Processing is then passed on to the file
`{\textit{dest}\hspace{0.2em}\textit{suffix}}'.
Surely, the same effect is achieved by
directly specifying the
argument `{\textit{dest}\hspace{0.2em}\textit{suffix}}'
in the first form.
However, that requires to set up a different file
for each child. With the alternative form of the command
all these files can have exactly the same content
which simplifies setting them up and maintaining them.

For example, the following file |draft.tex|
with a compilation flag |\version| as described in \secref{sec:flags}
compiles the main document as a draft:
%
\begin{center}
\begin{tabular}{l}
|\def\version{draft}|\\
|\input{childdoc.def}|\\
|\childdocforward{|\textit{main}|}|
\end{tabular}
\end{center}
%
Likewise, the following files |final|\textit{nn}|.tex|
compile the final version of the child document
|child|\textit{nn}|.tex|:
%
\begin{center}
\begin{tabular}{l}
|\def\version{final}|\\
|\input{childdoc.def}|\\
|\childdocforwardprefix{final}{child}|
\end{tabular}
\end{center}
%

Note that when several versions of a main file and/or of each child file
are to be generated, it may be convenient to set up a |Makefile| or
shell script to automatise the process.

%%%%%%%%%%%%%%%%%%%%%%%%%%%%%%%%%%%%%%%%%%%%%%%%%%%%%%%%%%%%%%%%%%%%%%%%%%%%%%%%
\subsection{Command Line Processing}
\label{sec:commandline}

The effect of redirection files can also be achieved by invoking
the \LaTeX{} compiler with a more elaborate command line.
Most conveniently this should be done as part
of a shell script or a |Makefile|.

When using \textsf{childdoc} in the main file, the following
command lines effectively perform a redirection
(note that depending on the shell being used,
backslashes may have to be doubled: `|\|' $\to$ `|\\|'):
%
\begin{center}
|... -jobname "|\textit{target}|" |\\|"|[\textit{flags}]%
|\input{childdoc.def}\childdocforward[|\textit{main}|]{|\textit{dest}|}"|
\end{center}
%
Here \textit{target} is the name of the output file,
\textit{main} is the name of the main file
and \textit{dest} is the name of the main or child file to be processed
(all filenames without extensions).
The optional argument \textit{main} can be omitted
if \textit{main} matches \textit{dest}.
Optionally, compilation \textit{flags} can be defined via |\def| commands.
This command line makes the \TeX{} engine believe
it is compiling the file \textit{target}
whose content is specified as the latter parameter.
The provided code then forwards the processing to
\textit{main} or \textit{dest} as described in \secref{sec:forward}.

%%%%%%%%%%%%%%%%%%%%%%%%%%%%%%%%%%%%%%%%%%%%%%%%%%%%%%%%%%%%%%%%%%%%%%%%%%%%%%%%
\subsection{Include by Input}
\label{sec:input}

Including child documents by |\include| has some restrictions by design.
Most notably, the content of a child document always occupies
its own set of pages; pages cannot be shared between child documents.
Usually, this behaviour makes perfect sense
because each child document contain an essential part of the document.
However, in some situations it may be desirable to compose
a document from a collection of parts
without having mandatory page breaks between then.
For this case, the package
provides a mechanism to include parts
by |\input| which can also be processed individually.
However, by construction this mechanism
requires manual handling of the content to be output.

%%%%%%%%%%%%%%%%%%%%%%%%%%%%%%%%%%%%%%%%
\DescribeMacro{\ifchilddocmanual}
The main file should be prepared as usual, see \secref{sec:include}.
However, the document body must make a distinction
between processing of an individual part and of the main document, e.g.:
%
\begin{center}
\begin{tabular}{l}
|\ifchilddocmanual|\\
|\input{\childdocname}|\\
|\||else|\\
\textit{document body with }|\input{|\textit{part}|}|\\
|\||fi|
\end{tabular}
\end{center}
%
The conditional |\ifchilddocmanual| is true whenever
a part to be included by |\input| is being compiled,
and the name of the part is stored in |\childdocname|.

%%%%%%%%%%%%%%%%%%%%%%%%%%%%%%%%%%%%%%%%
\DescribeMacro{\childdocby}
Each part to be included by |\input| should start with:
%
\begin{center}
\begin{tabular}{l}
|\input{childdoc.def}|\\
|\childdocby{|\textit{main}|}|\\
\end{tabular}
\end{center}
%
The directive |\childdocby| is similar to |\childdocof|
described in \secref{sec:include},
but the subsequent selection of content must be done manually.
To that end, both |\ifchilddoc| and |\ifchilddocmanual|
will be true upon processing of a part,
and the name of the part is stored in |\childdocname|.
Note that |\jobname| will be set to the filename of the current part
so that each part receives an individual |.aux| file
that does not interfere with the |.aux| file(s) of the main document.
This behaviour can be altered by the alternative form
|\childdocby[*]{|\textit{main}|}| (with a non-empty optional argument)
which uses the |.aux| file of the main document
by setting |\jobname| to \textit{main}.

%%%%%%%%%%%%%%%%%%%%%%%%%%%%%%%%%%%%%%%%%%%%%%%%%%%%%%%%%%%%%%%%%%%%%%%%%%%%%%%%
\subsection{Driver Development}
\label{sec:driver}

The \textsf{childdoc} mechanism can also be use for the development
of definition files such as \LaTeX{} styles or classes.
This case differs from the above setup with multiple parts
included by |\include| in that no |\includeonly| should be invoked.
This can be achieved by starting the include file
(before |\ProvidesPackage|) with:
%
\begin{center}
\begin{tabular}{l}
|\input{childdoc.def}|\\
|\childdocforward{|\textit{main}|}|\\
\end{tabular}
\end{center}
%
or alternatively with:
%
\begin{center}
\begin{tabular}{l}
|\input{childdoc.def}|\\
|\childdocby{|\textit{main}|}|\\
\end{tabular}
\end{center}
%
Both forms have slightly different effects as described above.
The main file is prepared as usual, see \secref{sec:include}.

%%%%%%%%%%%%%%%%%%%%%%%%%%%%%%%%%%%%%%%%%%%%%%%%%%%%%%%%%%%%%%%%%%%%%%%%%%%%%%%%
\subsection{Legacy Detection}
\label{sec:detection}

The directive |\childdocmain| in the main file can detect
whether the complete document or merely a child is to be compiled
even without using the directive |\childdocof|.
This method is deprecated because it is less robust
and there is no compelling reason to use it;
it is merely provided for backward compatibility
and it may be removed in future versions.

If the detection mechanism is to be used,
it is mandatory to correctly specify
the filename of the main file as the argument of |\childdocmain|:
%
\begin{center}
\begin{tabular}{l}
|\input{childdoc.def}|\\
|\childdocmain{|\textit{main}|}|\\
\end{tabular}
\end{center}
%
If |\jobname| does not match the argument \textit{main} of |\childdocmain|,
it is assumed that |\jobname| points to the child file to be compiled.
When using |\childdocmain| with the main file specified as argument,
it suffices to start a child file
with just |\input{|\textit{main}|}|
without loading of the package and using |\childdocof|.
If instead all processing is done
with the appropriate \textsf{childdoc} directives,
the argument of \textit{main} of |\childdocmain| can be empty.

An alternative version of the command line processing described
in \secref{sec:commandline} using the detection mechanism reads:
%
\begin{center}
|... -jobname "|\textit{target}|" "|[\textit{flags}]%
[|\def\jobname{|\textit{dest}|}|]|\input{|\textit{main}|}"|
\end{center}

%%%%%%%%%%%%%%%%%%%%%%%%%%%%%%%%%%%%%%%%%%%%%%%%%%%%%%%%%%%%%%%%%%%%%%%%%%%%%%%%
\subsection{Manual Code}
\label{sec:manual}

In case one cannot be certain whether the definitions file |childdoc.def|
is installed on the target \TeX{} distribution
and one prefers not to ship it,
it is conceivable to paste a few relevant commands into the sources.

To that end, drop all statements |\input{childdoc.def}|
and perform the replacements as outlined below.
Instead of |\childdocmain{|\textit{main}|}| add the following code
to the top of the main file:
%
\begin{center}
\begin{tabular}{l}
|\||ifdefined\childdocname\endinput\||fi\newif\ifchilddoc|\\
|\edef\childdocname{\scantokens\expandafter{\jobname\noexpand}}|\\
|\def\childdocmain{|\textit{main}|}\||ifx\childdocmain\childdocname\||else|\\
|\childdoctrue\includeonly{\childdocname}\let\jobname\childdocmain\||fi|\\
\end{tabular}
\end{center}
%
Instead of |\childdocof{|\textit{main}|}| just include the main file
at the top of each child file:
%
\begin{center}
|\input{|\textit{main}|}|
\end{center}
%
A simple redirection |\childdocforward{|\textit{dest}|}| is achieved by:
%
\begin{center}
|\def\jobname{|\textit{dest}|}\input{\jobname}|
\end{center}
%
The redirection with prefix
|\childdocforwardprefix[|\textit{prefix}|]{|\textit{dest}|}|
is accomplished by:
%
\begin{center}
\begin{tabular}{l}
|{\edef\jobname{\scantokens\expandafter{\jobname\noexpand}}|\\
|\def\redirectjob |\textit{prefix}|#1~~~{\gdef\jobname{|\textit{dest}|#1}}|\\
|\expandafter\redirectjob\jobname~~~}\input{\jobname}|
\end{tabular}
\end{center}

In an alternative approach,
child documents can be compiled by a specific command line
without additional code or specific definitions:
%
\begin{center}
|... -jobname "|\textit{target}|" "|[\textit{flags}]%
|\includeonly{|\textit{dest}|}\input{|\textit{main}|}"|
\end{center}
%

%%%%%%%%%%%%%%%%%%%%%%%%%%%%%%%%%%%%%%%%%%%%%%%%%%%%%%%%%%%%%%%%%%%%%%%%%%%%%%%%
%%%%%%%%%%%%%%%%%%%%%%%%%%%%%%%%%%%%%%%%%%%%%%%%%%%%%%%%%%%%%%%%%%%%%%%%%%%%%%%%
\section{Information}

%%%%%%%%%%%%%%%%%%%%%%%%%%%%%%%%%%%%%%%%%%%%%%%%%%%%%%%%%%%%%%%%%%%%%%%%%%%%%%%%
\subsection{Copyright}

Copyright \copyright{} 2017--2018 Niklas Beisert

This work may be distributed and/or modified under the
conditions of the \LaTeX{} Project Public License, either version 1.3
of this license or (at your option) any later version.
The latest version of this license is in
  \url{http://www.latex-project.org/lppl.txt}
and version 1.3 or later is part of all distributions of \LaTeX{}
version 2005/12/01 or later.

This work has the LPPL maintenance status `maintained'.

The Current Maintainer of this work is Niklas Beisert.

This work consists of the files |README.txt|, |childdoc.ins| and |childdoc.dtx|
as well as the derived files |childdoc.def|, |cdocsamp.tex|
with |cdocsch1.tex|, |cdocsch2.tex|, |cdocspt3.tex|, |cdocspt4.tex|,
|cdocsdrf.tex|, |cdocsfn1.tex|, |cdocsfn2.tex|
as well as |childdoc.pdf|.

%%%%%%%%%%%%%%%%%%%%%%%%%%%%%%%%%%%%%%%%%%%%%%%%%%%%%%%%%%%%%%%%%%%%%%%%%%%%%%%%
\subsection{Files and Installation}

The package consists of the files:
%
\begin{center}
\begin{tabular}{ll}
    |README.txt|   & readme file \\
    |childdoc.ins| & installation file \\
    |childdoc.dtx| & source file \\
    |childdoc.def| & definition file \\
    |cdocsamp.tex| & sample main file \\
    |cdocsch1.tex| & sample include file \\
    |cdocsch2.tex| & sample include file \\
    |cdocspt3.tex| & sample part file \\
    |cdocspt4.tex| & sample part file \\
    |cdocsdrf.tex| & sample redirection file \\
    |cdocsfn1.tex| & sample redirection file \\
    |cdocsfn2.tex| & sample redirection file \\
    |childdoc.pdf| & manual
\end{tabular}
\end{center}
%
The distribution consists of the files
|README.txt|, |childdoc.ins| and |childdoc.dtx|.
%
\begin{itemize}
\item
Run (pdf)\LaTeX{} on |childdoc.dtx|
to compile the manual |childdoc.pdf| (this file).
\item
Run \LaTeX{} on |childdoc.ins| to create the definitions file |childdoc.def|
and the sample |cdocsamp.tex| with include files
|cdocsch1.tex|, |cdocsch2.tex|, |cdocspt3.tex|, |cdocspt4.tex|,
|cdocsdrf.tex|, |cdocsfn1.tex|, |cdocsfn2.tex|.
Then copy the file |childdoc.def| to an appropriate directory of your \LaTeX{}
distribution, e.g.\ \textit{texmf-root}|/tex/latex/childdoc|.
\end{itemize}

%%%%%%%%%%%%%%%%%%%%%%%%%%%%%%%%%%%%%%%%%%%%%%%%%%%%%%%%%%%%%%%%%%%%%%%%%%%%%%%%
\subsection{Related CTAN Packages}

There are several other packages which offer a similar functionality:
%
\begin{itemize}
\item
The packages
\href{http://ctan.org/pkg/docmute}{\textsf{docmute}},
\href{http://ctan.org/pkg/includex}{\textsf{includex}} and
\href{http://ctan.org/pkg/standalone}{\textsf{standalone}}
provide commands to include only the document body of
a child file thus allowing both files to be compiled individually.
\item
The packages \href{http://ctan.org/pkg/subdocs}{\textsf{subdocs}}
and \href{http://ctan.org/pkg/subfiles}{\textsf{subfiles}}
provide structures in which the main and child documents can be
encapsulated and allowing them to be compiled individually.
The inclusion mechanism is different from the conventional |\include|.
\item
The package \href{http://ctan.org/pkg/combine}{\textsf{combine}}
is an elaborate solution to combine several documents into one.
\end{itemize}
%
See also the CTAN topic \href{http://ctan.org/topic/subdocs}{\textsf{subdocs}}
for further related packages.
The present package differs from the above solutions in that
a document structure constructed with the conventional |\include| mechanism
just needs two extra commands at the top of every file
such that all constituent files can be compiled individually.

%%%%%%%%%%%%%%%%%%%%%%%%%%%%%%%%%%%%%%%%%%%%%%%%%%%%%%%%%%%%%%%%%%%%%%%%%%%%%%%%
%\subsection{Feature Suggestions}
%
%The following is a list of features which may be useful for future
%versions of this package:
%%
%\begin{itemize}
%\item
%\ldots
%\end{itemize}

%%%%%%%%%%%%%%%%%%%%%%%%%%%%%%%%%%%%%%%%%%%%%%%%%%%%%%%%%%%%%%%%%%%%%%%%%%%%%%%%
\subsection{Revision History}

%%%%%%%%%%%%%%%%%%%%%%%%%%%%%%%%%%%%%%%%
\paragraph{v2.0:} 2018/12/30

\begin{itemize}
\item
immediate forward processing
\item
added |\childdocby| mechanism
\item
manual restructured
\end{itemize}

%%%%%%%%%%%%%%%%%%%%%%%%%%%%%%%%%%%%%%%%
\paragraph{v1.6:} 2018/01/17

\begin{itemize}
\item
application for development of include files
\item
corrections to manual
\end{itemize}

%%%%%%%%%%%%%%%%%%%%%%%%%%%%%%%%%%%%%%%%
\paragraph{v1.5:} 2017/05/21

\begin{itemize}
\item
more complete structuring introduced
\item
|\childdocof| introduced
\item
|\childdoc| renamed to |\childdocmain|
\item
|\childredirect| renamed to |\childdocforward| and |\childdocforwardprefix|
and functionality expanded
\end{itemize}

%%%%%%%%%%%%%%%%%%%%%%%%%%%%%%%%%%%%%%%%
\paragraph{v1.0:} 2017/04/27

\begin{itemize}
\item
manual and install package
\item
first version published on CTAN
\end{itemize}

%%%%%%%%%%%%%%%%%%%%%%%%%%%%%%%%%%%%%%%%
\paragraph{v0.6:} 2017/04/26

\begin{itemize}
\item
redirection mechanism added
\end{itemize}

%%%%%%%%%%%%%%%%%%%%%%%%%%%%%%%%%%%%%%%%
\paragraph{v0.5:} 2017/04/26

\begin{itemize}
\item
functionality in definition file
\end{itemize}


%%%%%%%%%%%%%%%%%%%%%%%%%%%%%%%%%%%%%%%%%%%%%%%%%%%%%%%%%%%%%%%%%%%%%%%%%%%%%%%%
%%%%%%%%%%%%%%%%%%%%%%%%%%%%%%%%%%%%%%%%%%%%%%%%%%%%%%%%%%%%%%%%%%%%%%%%%%%%%%%%
%%%%%%%%%%%%%%%%%%%%%%%%%%%%%%%%%%%%%%%%%%%%%%%%%%%%%%%%%%%%%%%%%%%%%%%%%%%%%%%%
\appendix

\settowidth\MacroIndent{\rmfamily\scriptsize 000\ }

 \DocInput{childdoc.dtx}

\end{document}
%</driver>
% \fi
%
% %%%%%%%%%%%%%%%%%%%%%%%%%%%%%%%%%%%%%%%%%%%%%%%%%%%%%%%%%%%%%%%%%%%%%%%%%%%%%%
% %%%%%%%%%%%%%%%%%%%%%%%%%%%%%%%%%%%%%%%%%%%%%%%%%%%%%%%%%%%%%%%%%%%%%%%%%%%%%%
% \section{Sample}
%\iffalse
%<*samplemain>
%\fi
%
% The following presents a sample document
% with two chapters, two parts, a title page,
% a compile flag as well as three forwarding files to set the flag.
% It consists of eight |.tex| files:
% \begin{center}
% \begin{tabular}{ll}
% |cdocsamp.tex|&main file\\
% |cdocsch1.tex|&include file for chapter 1\\
% |cdocsch2.tex|&include file for chapter 2\\
% |cdocspt3.tex|&include file for part 3\\
% |cdocspt4.tex|&include file for part 4\\
% |cdocsdrf.tex|&forwarding file for main file in draft mode\\
% |cdocsfi1.tex|&forwarding file for final version of chapter 1\\
% |cdocsfi2.tex|&forwarding file for final version of chapter 2\\
% \end{tabular}
% \end{center}
% Each of the eight files can be compiled directly by the \LaTeX{} compiler.
%
% %%%%%%%%%%%%%%%%%%%%%%%%%%%%%%%%%%%%%%
% \paragraph{Main File.}
%
% The main file is called |cdocsamp.tex|.
%
% Load the \textsf{childdoc} definitions and
% declare the filename for the main document:
%    \begin{macrocode}
\input{childdoc.def}
\childdocmain{}
%    \end{macrocode}

% Optional override for |\version| flag:
%    \begin{macrocode}
%%\ifchilddoc\else\providecommand{\version}{draft}\fi
%    \end{macrocode}

% Define the default values for the |\version| flag
% (|final| for the main file and |draft| for childs):
%    \begin{macrocode}
\ifchilddoc
\providecommand{\version}{draft}
\else
\providecommand{\version}{final}
\fi
%    \end{macrocode}

% Load the standard document class:
%    \begin{macrocode}
\documentclass[12pt]{article}
%    \end{macrocode}

% Start the document body:
%    \begin{macrocode}
\begin{document}
%    \end{macrocode}

% Declare a title page.
% Print title, part of document being processed and version flag:
%    \begin{macrocode}
\addtocounter{page}{-1}
\begin{center}
{\LARGE\bfseries{}childdoc example\par}
\vspace{1cm}
\ifchilddoc
\ifchilddocmanual part\else chapter\fi:
`\childdocname' of `\childdocjob'\par
\else
main document: `\childdocjob'\par
\fi
version: \version\par
\end{center}
\newpage
%    \end{macrocode}

% Manually include selected file,
% otherwise process as usual:
%    \begin{macrocode}
\ifchilddocmanual
\section*{part `\childdocname'}
\input{\childdocname}
\else
%    \end{macrocode}

% Include the two chapters:
%    \begin{macrocode}
\include{cdocsch1}
\include{cdocsch2}
%    \end{macrocode}

% Include the two parts unless only chapters should be displayed:
%    \begin{macrocode}
\ifchilddoc\else
\section{part three}
\input{cdocspt3}
\section{part four}
\input{cdocspt4}
\fi
%    \end{macrocode}

% Process as usual until here:
%    \begin{macrocode}
\fi
%    \end{macrocode}

% End of document body:
%    \begin{macrocode}
\end{document}
%    \end{macrocode}
%\iffalse
%</samplemain>
%\fi
%
% %%%%%%%%%%%%%%%%%%%%%%%%%%%%%%%%%%%%%%
% \paragraph{Chapter Include Files.}
%
% The include files are called |cdocsch1.tex| and |cdocsch2.tex|.
%
%\iffalse
%<*samplechap1|samplechap2>
%\fi

% Optional override for |\version| flag:
%    \begin{macrocode}
%%\providecommand{\version}{final}
%    \end{macrocode}

% Include the main document:
%    \begin{macrocode}
\input{childdoc.def}
\childdocof{cdocsamp}
%    \end{macrocode}

%\iffalse
%</samplechap1|samplechap2>
%\fi
%
%\iffalse
%<*samplechap1>
%\fi
% Some text for chapter 1:
%    \begin{macrocode}
\section{one}
some text in chapter one
%    \end{macrocode}

%\iffalse
%</samplechap1>
%\fi
% Some text for chapter 2:
%\iffalse
%<*samplechap2>
%\fi
%    \begin{macrocode}
\section{two}
more text in chapter two
%    \end{macrocode}

%\iffalse
%</samplechap2>
%\fi
%
% %%%%%%%%%%%%%%%%%%%%%%%%%%%%%%%%%%%%%%
% \paragraph{Part Include Files.}
%
% The include files are called |cdocspt3.tex| and |cdocspt4.tex|.
%
%\iffalse
%<*samplepart3|samplepart4>
%\fi

% Optional override for |\version| flag:
%    \begin{macrocode}
%%\providecommand{\version}{final}
%    \end{macrocode}

% Include the main document:
%    \begin{macrocode}
\input{childdoc.def}
\childdocby{cdocsamp}
%    \end{macrocode}

%\iffalse
%</samplepart3|samplepart4>
%\fi
%
%\iffalse
%<*samplepart3>
%\fi
% Some text for part 3:
%    \begin{macrocode}
some text in part three
%    \end{macrocode}

%\iffalse
%</samplepart3>
%\fi
% Some text for part 4:
%\iffalse
%<*samplepart4>
%\fi
%    \begin{macrocode}
more text in part four
%    \end{macrocode}

%\iffalse
%</samplepart4>
%\fi
%
% %%%%%%%%%%%%%%%%%%%%%%%%%%%%%%%%%%%%%%
% \paragraph{Forwarding for a Complete Draft.}
%
% The following forwarding file |cdocsdrf.tex|
% compiles the main document in draft mode:
%\iffalse
%<*sampledraft>
%\fi
%    \begin{macrocode}
\def\version{draft}
\input{childdoc.def}
\childdocforward{cdocsamp}
%    \end{macrocode}

%\iffalse
%</sampledraft>
%\fi
%
% %%%%%%%%%%%%%%%%%%%%%%%%%%%%%%%%%%%%%%
% \paragraph{Forwarding for Final Version of the Chapters.}
%
% The following forwarding files |cdocsfn1.tex| and |cdocsfn2.tex|
% (with identical content)
% compile the final versions of the child documents
% |cdocsch1.tex| and |cdocsch2.tex|, respectively:
%\iffalse
%<*samplefinal>
%\fi
%    \begin{macrocode}
\def\version{final}
\input{childdoc.def}
\childdocforwardprefix[cdocsamp]{cdocsfn}{cdocsch}
%    \end{macrocode}

%\iffalse
%</samplefinal>
%\fi
%
% %%%%%%%%%%%%%%%%%%%%%%%%%%%%%%%%%%%%%%
% \paragraph{Command Line Processing.}
%
% The following three command lines generate the output files
% |cdocscld|, |cdocscl1| and |cdocscl2|
% which should be identical to
% |cdocsdrf|, |cdocsch1| and |cdocsfn2|, respectively:
% \begin{center}
% \begin{tabular}{l}
% |latex -jobname cdocscld \|\\
% |  "\def\version{draft}\input{childdoc.def}\childdocforward{cdocsamp}"|\\
% |latex -jobname cdocscl1 \|\\
% |  "\input{childdoc.def}\childdocforward[cdocsamp]{cdocsch1}"|\\
% |latex -jobname cdocscl2 \|\\
% |  "\def\version{final}\input{childdoc.def}\childdocforward{cdocsch2}"|
% \end{tabular}
% \end{center}
% Note that the trailing backslash on each first line
% merely continues the input to the second line
% (for convenient cut ant paste).
% Furthermore, the command |latex| can be replaced by any
% of its alternative versions such as |pdflatex|.
%
% %%%%%%%%%%%%%%%%%%%%%%%%%%%%%%%%%%%%%%%%%%%%%%%%%%%%%%%%%%%%%%%%%%%%%%%%%%%%%%
% %%%%%%%%%%%%%%%%%%%%%%%%%%%%%%%%%%%%%%%%%%%%%%%%%%%%%%%%%%%%%%%%%%%%%%%%%%%%%%
% \section{Implementation}
%\iffalse
%<*package>
%\fi
%
% This section describes the definitions file |childdoc.def|.

% The definitions cannot be loaded using |\usepackage| or |\RequirePackage|
% which has a mechanism to prevent loading a style file more than once.
% When loading the definitions by means of |\input|
% multiple instances have to be prevented manually:
%\iffalse
%This code needs to be before the `\ProvidesFile' directive
%which is defined at the beginning of this file.
%Therefore it is also placed there and commented out here.
%</package>
%<*discard>
%\fi
%    \begin{macrocode}
\ifdefined\childdocmain\endinput\fi
%    \end{macrocode}
%\iffalse
%</discard>
%<*package>
%\fi
%
% \macro{\ifchilddoc}
% \macro{\ifchilddocmanual}
% The conditional |\ifchilddoc| tells whether a
% child (true) or main (false) document is being compiled.
% The conditional |\ifchilddocmanual| tells whether
% the |\includeonly| mechanism is used (false) or
% the selection of child files must be performed manually (true).
% The definitions initialise to false:
%    \begin{macrocode}
\newif\ifchilddoc
\newif\ifchilddocmanual
%    \end{macrocode}

% \macro{\childdocname}
% \macro{\childdocjob}
% The macro |\childdocname| stores the name of the main document
% to be compiled. The macro |\childdocjob| stores the name of
% the document on which the \LaTeX{} compiler was originally invoked.
% The content of |\jobname| cannot be compared
% to filenames specified in the source due to different catcodes.
% The following code rescans |\jobname|, stores the result
% in |\childdocname| and saves a copy in |\childdocjob|:
%    \begin{macrocode}
\edef\childdocname{\scantokens\expandafter{\jobname\noexpand}}
\let\childdocjob\childdocname
%    \end{macrocode}

% \macro{\childdocdisable}
% The macro |\childdocdisable| prevents the main file
% from being processed more than once.
% At this stage, the main document command |\childdocmain|
% is assumed to be called once again where it should do nothing.
% Any subsequent call to it should prevent
% a secondary processing of the main document
% It overwrites the forwarding commands
% |\childdocof| and |\childdocforward|
% with empty macros to prevent further inclusions of the main document:
%    \begin{macrocode}
\newcommand{\childdocdisable}
{
  \renewcommand{\childdocmain}[1]{\renewcommand{\childdocmain}[1]{\endinput}}
  \renewcommand{\childdocof}[1]{}
  \renewcommand{\childdocby}[2][]{}
  \renewcommand{\childdocforward}[2][]{}
  \renewcommand{\childdocdisable}{}
}
%    \end{macrocode}

% \macro{\childdocmain}
% The macro |\childdocmain| is to be called at the top of the main file
% with nothing or the main filename (without extension) as argument.
% First, it breaks loops.
% If the argument is not empty and does not match |\childdocname|
% (which is set by the first inclusion of |childdoc.def|),
% |\ifchilddoc| is set to true, |\includeonly| is applied to the child file
% and |\jobname| is set to the main file
% (for proper handling of |.aux| files):
%    \begin{macrocode}
\newcommand{\childdocmain}[1]
{
  \childdocdisable\childdocmain{}
  \if?#1?\else
    \begingroup
      \def\childdoctmp{#1}
      \ifx\childdoctmp\childdocname
        \def\childdoctmp{}
      \else
        \def\childdoctmp
        {
          \childdoctrue
          \includeonly{\childdocname}
          \def\childdocjob{#1}
          \def\jobname{#1}
        }
      \fi
      \expandafter
    \endgroup
    \childdoctmp
  \fi
}
%    \end{macrocode}

% \macro{\childdocof}
% The command |\childdocof| redirects
% compilation to the main file |#1|.
%    \begin{macrocode}
\newcommand{\childdocof}[1]
{
  \childdocdisable
  \childdoctrue
  \includeonly{\childdocname}
  \def\jobname{#1}
  \def\childdocjob{#1}
  \input{#1}
}
%    \end{macrocode}

% \macro{\childdocby}
% The command |\childdocby| ....
%    \begin{macrocode}
\newcommand{\childdocby}[2][]
{
  \childdocdisable
  \childdoctrue
  \childdocmanualtrue
  \if?#1?\else
    \def\jobname{#2}
  \fi
  \def\childdocjob{#2}
  \input{#2}
  \endinput
}
%    \end{macrocode}

% \macro{\childdocforward}
% The command |\childdocforward| redirects
% compilation to the main file or
% (if the optional argument is given) a child file.
% Parameters are set as if the main file
% or a child file starting with |\childdocof| was compiled.
% Then compilation is handed over to the main file:
%    \begin{macrocode}
\newcommand{\childdocforward}[2][]
{
  \begingroup
    \if?#1?
      \def\childdoctmp
      {
        \def\childdocname{#2}
        \def\childdocjob{#2}
        \def\jobname{#2}
        \input{#2}
        \endinput
      }
    \else
      \def\childdoctmp
      {
        \childdocdisable
        \def\childdocname{#2}
        \childdoctrue
        \includeonly{#2}
        \def\childdocjob{#1}
        \def\jobname{#1}
        \input{#1}
        \endinput
      }
    \fi
    \expandafter
  \endgroup
  \childdoctmp
}
%    \end{macrocode}

% \macro{\childdocforwardprefix}
% The command |\childdocforwardprefix| redirects
% compilation to the main or a child file by means of a pattern.
% The prefix |#1| in the current filename is replaced by |#2|
% and the suffix of the current filename is kept
% (it is assumed that the filename does not contain the substring `|~~~|'
% which is used as a delimiter).
% Compilation is handed over to the new file by |\childdocforward|:
%    \begin{macrocode}
\newcommand{\childdocforwardprefix}[3][]
{
  \begingroup
    \def\childdocextract #2##1~~~{\def\childdoctmp{\childdocforward[#1]{#3##1}}}
    \expandafter\childdocextract\childdocname~~~
    \expandafter
  \endgroup
  \childdoctmp
}
%    \end{macrocode}

% \macro{\childdoc}
% The deprecated macro |\childdoc| is a legacy version of |\childdocmain|:
%    \begin{macrocode}
\newcommand{\childdoc}{\childdocmain}
%    \end{macrocode}

% \macro{\childdocredirect}
% The deprecated macro |\childdocredirect| is a legacy version
% of |\childdocforward| and |\childdocforwardprefix|:
%    \begin{macrocode}
\newcommand{\childdocredirect}[2][]
{
  \begingroup
    \if?#1?
      \def\childdoctmp{\childdocforward{#2}}
    \else
      \def\childdoctmp{\childdocforwardprefix{#1}{#2}}
    \fi
    \expandafter
  \endgroup
  \childdoctmp
}
%    \end{macrocode}

%\iffalse
%</package>
%\fi
%
\endinput
|\\
|\childdocforwardprefix[|\textit{main}|]{|\textit{prefix}|}{|\textit{dest}|}|
\end{tabular}
\end{center}
%
the destination file is determined by a pattern
depending on the current file:
To make this work, the current file must be called
`{\textit{prefix}\hspace{0.2em}\textit{suffix}}'
with \textit{prefix} matching precisely the argument.
Processing is then passed on to the file
`{\textit{dest}\hspace{0.2em}\textit{suffix}}'.
Surely, the same effect is achieved by
directly specifying the
argument `{\textit{dest}\hspace{0.2em}\textit{suffix}}'
in the first form.
However, that requires to set up a different file
for each child. With the alternative form of the command
all these files can have exactly the same content
which simplifies setting them up and maintaining them.

For example, the following file |draft.tex|
with a compilation flag |\version| as described in \secref{sec:flags}
compiles the main document as a draft:
%
\begin{center}
\begin{tabular}{l}
|\def\version{draft}|\\
|% \iffalse
%
% childdoc.dtx Copyright (C) 2017-2018 Niklas Beisert
%
% This work may be distributed and/or modified under the
% conditions of the LaTeX Project Public License, either version 1.3
% of this license or (at your option) any later version.
% The latest version of this license is in
%   http://www.latex-project.org/lppl.txt
% and version 1.3 or later is part of all distributions of LaTeX
% version 2005/12/01 or later.
%
% This work has the LPPL maintenance status `maintained'.
%
% The Current Maintainer of this work is Niklas Beisert.
%
% This work consists of the files childdoc.dtx and childdoc.ins
% and the derived files childdoc.def and cdocsamp.tex with
% cdocsch1.tex, cdocsch2.tex, cdocsdrf.tex, cdocsfn1.tex, cdocsfn2.tex.
%
%<package>\ifdefined\childdocmain\endinput\fi
%<package>\ProvidesFile{childdoc.def}[2018/12/30 v2.0 child document driver]
%<samplemain>\ProvidesFile{cdocsamp.tex}[2018/12/30 v2.0 sample for childdoc]
%<*driver>
%\ProvidesFile{childdoc.drv}[2018/12/30 v2.0 childdoc reference manual file]
\PassOptionsToClass{10pt,a4paper}{article}
\documentclass{ltxdoc}

\usepackage[margin=35mm]{geometry}
\usepackage{hyperref}
\usepackage{hyperxmp}
\usepackage[usenames]{color}

\hypersetup{colorlinks=true}
\hypersetup{pdfstartview=FitH}
\hypersetup{pdfpagemode=UseNone}
\hypersetup{pdfsource={}}
\hypersetup{pdflang={en-UK}}
\hypersetup{pdfcopyright={Copyright 2017-2018 Niklas Beisert.
  This work may be distributed and/or modified under the
  conditions of the LaTeX Project Public License, either version 1.3
  of this license or (at your option) any later version.}}
\hypersetup{pdflicenseurl={http://www.latex-project.org/lppl.txt}}
\hypersetup{pdfcontactaddress={ETH Zurich, ITP, HIT K,
  Wolfgang-Pauli-Strasse 27}}
\hypersetup{pdfcontactpostcode={8093}}
\hypersetup{pdfcontactcity={Zurich}}
\hypersetup{pdfcontactcountry={Switzerland}}
\hypersetup{pdfcontactemail={nbeisert@itp.phys.ethz.ch}}
\hypersetup{pdfcontacturl={http://people.phys.ethz.ch/\xmptilde nbeisert/}}

\newcommand{\secref}[1]{\hyperref[#1]{section \ref*{#1}}}

\parskip1ex
\parindent0pt
\let\olditemize\itemize
\def\itemize{\olditemize\parskip0pt}

\begin{document}

\title{The \textsf{childdoc} Package}
\hypersetup{pdftitle={The childdoc Package}}
\author{Niklas Beisert\\[2ex]
  Institut f\"ur Theoretische Physik\\
  Eidgen\"ossische Technische Hochschule Z\"urich\\
  Wolfgang-Pauli-Strasse 27, 8093 Z\"urich, Switzerland\\[1ex]
  \href{mailto:nbeisert@itp.phys.ethz.ch}
  {\texttt{nbeisert@itp.phys.ethz.ch}}}
\hypersetup{pdfauthor={Niklas Beisert}}
\hypersetup{pdfsubject={Manual for the LaTeX2e Package childdoc}}
\date{30 December 2018, \textsf{v2.0}}
\maketitle

\begin{abstract}\noindent
\textsf{childdoc} is a \LaTeXe{} package
that enables the direct compilation
of document sections included by |\include|
to individual files.
\end{abstract}

\begingroup
\parskip0ex
\tableofcontents
\endgroup

%%%%%%%%%%%%%%%%%%%%%%%%%%%%%%%%%%%%%%%%%%%%%%%%%%%%%%%%%%%%%%%%%%%%%%%%%%%%%%%%
%%%%%%%%%%%%%%%%%%%%%%%%%%%%%%%%%%%%%%%%%%%%%%%%%%%%%%%%%%%%%%%%%%%%%%%%%%%%%%%%
\section{Introduction}

\LaTeX{} provides a mechanism to structure a large document (such as a book)
into a main file and several child files (containing the chapters)
using the |\include| command.
This mechanism is beneficial for documents
which span hundreds of pages in order to
make the source file(s) more manageable.
Moreover, compilation can be restricted to
selected child files by means of the |\includeonly| command.
The latter feature can be used to reduce the compilation time while editing
(this was significantly more useful in the earlier days of \LaTeX{})
or to generate a smaller document which is easier to navigate.
Another application of |\includeonly| is to generate
documents consisting of selected parts of the complete document.

However, there are a few drawbacks of the plain |\include| mechanism:
\begin{itemize}
\item
The child files cannot be compiled on their own,
they can only be compiled via the main file.
A naive editing environment
(such as a text editor with an option
to have the current file processed by \LaTeX)
may require one to switch to the main file before compiling;
attempting to compile the child file produces errors.
\item
The main file must be modified (each time)
to adjust the |\includeonly| command
to the present needs. This easily leaves the main file in a messy state.
\item
The generated document will always carry the filename
of the main document. This is inconvenient if
several child files are to be compiled and
to be kept for distribution.
\end{itemize}

The present package provides a simple interface
to make child files individually compilable by \LaTeX{}.
Compiling a child file then has the same effect as compiling
the main file with an |\includeonly| command
to select the appropriate child.
Moreover the generated document will carry the name of the child
rather than the main file.
This resolves all three above issues.

This feature is meant to make the editing of books,
thesis documents and lecture notes somewhat more convenient.
However, the package can also be used efficiently for
composing a series of documents (such as exercise sheets)
which are typically distributed individually.
It then assists the author in generating the individual documents
(potentially in different versions)
as well as a document containing the collected series.
Another application is in developing style files
or other kinds of included material
where compilation of the style file could redirect
to a sample or test file.

%%%%%%%%%%%%%%%%%%%%%%%%%%%%%%%%%%%%%%%%%%%%%%%%%%%%%%%%%%%%%%%%%%%%%%%%%%%%%%%%
%%%%%%%%%%%%%%%%%%%%%%%%%%%%%%%%%%%%%%%%%%%%%%%%%%%%%%%%%%%%%%%%%%%%%%%%%%%%%%%%
\section{Usage}

First of all, the package \textsf{childdoc} is \emph{not} a standard
\LaTeXe{} |.sty| style file! Therefore it needs to be invoked in
a non-standard way.

%%%%%%%%%%%%%%%%%%%%%%%%%%%%%%%%%%%%%%%%%%%%%%%%%%%%%%%%%%%%%%%%%%%%%%%%%%%%%%%%
\subsection{Included Files}
\label{sec:include}

%%%%%%%%%%%%%%%%%%%%%%%%%%%%%%%%%%%%%%%%
\DescribeMacro{\childdocmain}
To use the package, add the commands
\begin{center}
\begin{tabular}{l}
|\input{childdoc.def}|\\
|\childdocmain{}|\\
\end{tabular}
\end{center}
at the very top of the main \LaTeX{} file,
in particular \emph{before} the |\documentclass| statement!
The argument of |\childdocmain| should be left empty
(but it must be present).

%%%%%%%%%%%%%%%%%%%%%%%%%%%%%%%%%%%%%%%%
\DescribeMacro{\childdocof}
Furthermore, add the commands
\begin{center}
\begin{tabular}{l}
|\input{childdoc.def}|\\
|\childdocof{|\textit{main}|}|\\
\end{tabular}
\end{center}
at the top of every child file \textit{child}
which is included by |\include{|\textit{child}|}|
from within the main file
(or at least for those files to be compiled individually).
The argument \textit{main} must be the filename of the main file.

There are a couple of
considerations in setting up the main and child documents:

%%%%%%%%%%%%%%%%%%%%%%%%%%%%%%%%%%%%%%%%
\paragraph{Restrictions.}

Please note the following restrictions:
\begin{itemize}
\item
|\childdocmain| must be called with one argument \textit{main}
to ensure compatibility with earlier version of the package.
It must either be empty (|\childdocmain{}|)
or precisely match the filename of the main file in which it is specified.
See \secref{sec:detection} for further information.
\item
The filename \textit{main} must be specified without the |.tex| extension.
\item
The filename \textit{main} is case sensitive
(even in case-insensitive file systems)
due to internal string comparison.
\item
The argument \textit{main} should be fully expanded, it cannot be a macro.
\item
Subdirectories and special characters should be avoided in filenames.
\item
The command |\childdocmain{|\textit{main}|}| must be followed by a whitespace.
It should not be followed immediately by another command
or by a comment mark `|%|'.
This is because the \TeX{} parser reads the token immediately following
the argument of |\childdocmain| and puts it
at the beginning of every child section;
however, a white\-space is ignored.
\end{itemize}

%%%%%%%%%%%%%%%%%%%%%%%%%%%%%%%%%%%%%%%%
\paragraph{Content of Main File.}

It is advisable to place all content in the child files included by |\include|.
Any output contained in the main file will appear in all child documents
unless suppressed manually;
it cannot be suppressed automatically by the |\includeonly| directive
and thus should normally be avoided.
A method to include some content in the main file
by means of conditional processing is described in \secref{sec:conditional}.

%%%%%%%%%%%%%%%%%%%%%%%%%%%%%%%%%%%%%%%%
\paragraph{Page Numbering.}

When only a part of the document is compiled,
the appropriate numbering of pages
(as well as other status parameters)
is determined from the |.aux| files.
The latter contain information from previous passes.
However this information needs to propagate through
all intermediate child documents.
Therefore the page numbering in child documents may well
be inconsistent until the complete document is compiled at least once.

A useful (if unconventional) way to always ensure a consistent
page numbering is to restart the numbering in each child document
and denote the pages by `\textit{child}|.|\textit{page}'
where \textit{child} represents the chapter/section number of the child file.
This can be achieved by the command
|\numberwithin{page}{|\textit{child}|}|
of the \textsf{amsmath} package
where \textit{child} can be |chapter| or |section|
depending on the chosen structuring.
Alternatively, one can modify the macro |\thepage| appropriately
and reset the counter |page| at the start of each child file.

%%%%%%%%%%%%%%%%%%%%%%%%%%%%%%%%%%%%%%%%%%%%%%%%%%%%%%%%%%%%%%%%%%%%%%%%%%%%%%%%
\subsection{Conditional Processing}
\label{sec:conditional}

The package provides a mechanism to compile different versions
of a document. To customise the versions further some conditional processing
can come in handy to distinguish which version is being compiled.
The package provides two macros to describe the compilation context:

%%%%%%%%%%%%%%%%%%%%%%%%%%%%%%%%%%%%%%%%
\DescribeMacro{\ifchilddoc}
The conditional |\ifchilddoc| distinguishes between the compilation of
child documents and the main document:
%
\begin{center}
|\ifchilddoc |\textit{child-code}| |[|\||else |\textit{main-code}]| \||fi|
\end{center}

%%%%%%%%%%%%%%%%%%%%%%%%%%%%%%%%%%%%%%%%
\DescribeMacro{\childdocname}
\DescribeMacro{\childdocjob}
The macro |\childdocname| contains the filename (without extension)
of the main or child file being processed.
Note that |\childdocjob| will always contain the name of the main file.

%%%%%%%%%%%%%%%%%%%%%%%%%%%%%%%%%%%%%%%%
\paragraph{Title Page.}

Conditional processing can be used to include a title or banner page
in the main document when proper precautions are taken.
Importantly, the code in the main file should ensure that the page counter
(as well as other status parameters which are stored in the |.aux| files)
takes the same value after the conditional processing.
Otherwise the page numbers may take divergent values
depending on which part is compiled.

For example, a title page could be declared by:
%
\begin{center}
\begin{tabular}{l}
|\ifchilddoc\||else|\\
|\addtocounter{page}{-1}|\\
\textit{code for title page}\\
|\newpage|\\
|\||fi|
\end{tabular}
\end{center}
%
A banner page for the child documents can be generated by:
%
\begin{center}
\begin{tabular}{l}
|\ifchilddoc|\\
|\addtocounter{page}{-1}|\\
\textit{code for banner page}\\
|\newpage|\\
|\||fi|
\end{tabular}
\end{center}
%
Here one could write a message such as:
\begin{center}
|This is the part \childdocname{} of \childdocjob{}.|
\end{center}

%%%%%%%%%%%%%%%%%%%%%%%%%%%%%%%%%%%%%%%%%%%%%%%%%%%%%%%%%%%%%%%%%%%%%%%%%%%%%%%%
\subsection{Flags}
\label{sec:flags}

The package makes it easy to generate different versions
of the main or child documents.
To this end compilation flags can be defined
and assigned different default values.
They will be particularly useful in conjunction
with the forwarding mechanism described in \secref{sec:forward}.

For example, it may be useful to have a flag |\version|
which can be set to |draft| or |final|.
The document source will contain some conditional code
depending on the value of |\version|.
Suppose further, the flag should default to |final| for the main file
and to |draft| for child files
which is a natural assignment for editing the document.
This is achieved by placing the following code
in the preamble of the main document
(below the |\childdocmain| directive):
%
\begin{center}
\begin{tabular}{l}
|\ifchilddoc|\\
|\providecommand{\version}{draft}|\\
|\||else|\\
|\providecommand{\version}{final}|\\
|\||fi|
\end{tabular}
\end{center}
%
The definition by |\providecommand| makes sure
that previous definitions are not overwritten.
Further statements |\providecommand{\version}{...}|
can thus be added before the above code to override it.

For the main file, one might add a line
(between |\childdocmain| and the above block)
%
\begin{center}
|%\ifchilddoc\||else\providecommand{\version}{draft}\||fi|
\end{center}
%
which can be uncommented to produce a draft version.
Likewise one can add a line to the very top of a child file
(above the |\childdocof{|\textit{main}|}| directive)
%
\begin{center}
|%\providecommand{\version}{final}|
\end{center}
%
which can be uncommented to produce the final version of this child document.

%%%%%%%%%%%%%%%%%%%%%%%%%%%%%%%%%%%%%%%%%%%%%%%%%%%%%%%%%%%%%%%%%%%%%%%%%%%%%%%%
\subsection{Forwarding}
\label{sec:forward}

Different versions of the main or child documents
using compilation flags as described in \secref{sec:flags}
can be (permanently) stored in different files
for convenient compilation, viewing and distribution.
To this end, the package defines a command
to pass on compilation to a different file:

%%%%%%%%%%%%%%%%%%%%%%%%%%%%%%%%%%%%%%%%
\DescribeMacro{\childdocforward}
The command |\childdocforward| redirects processing to
another source file:
%
\begin{center}
\begin{tabular}{l}
|\input{childdoc.def}|\\
|\childdocforward[|\textit{main}|]{|\textit{dest}|}|\\
\end{tabular}
\end{center}
%
The argument \textit{dest} is the destination file
(without extension).
It should be the main file or one of the child files.
Note that further \textsf{childdoc} directives
such as |\childdocof| and |\childdocforward|
in the indicated file will be processed in this form.
The optional argument \textit{main}
passes on directly to the main file \textit{main}
while pretending to compile the child \textit{dest}.
This form behaves as if \textit{dest}
issues |\childdocof{|\textit{main}|}| right away,
and no further \textsf{childdoc} directives will be processed.

%%%%%%%%%%%%%%%%%%%%%%%%%%%%%%%%%%%%%%%%
\DescribeMacro{\...prefix}
In the alternative form |\childdocforwardprefix|,
%
\begin{center}
\begin{tabular}{l}
|\input{childdoc.def}|\\
|\childdocforwardprefix[|\textit{main}|]{|\textit{prefix}|}{|\textit{dest}|}|
\end{tabular}
\end{center}
%
the destination file is determined by a pattern
depending on the current file:
To make this work, the current file must be called
`{\textit{prefix}\hspace{0.2em}\textit{suffix}}'
with \textit{prefix} matching precisely the argument.
Processing is then passed on to the file
`{\textit{dest}\hspace{0.2em}\textit{suffix}}'.
Surely, the same effect is achieved by
directly specifying the
argument `{\textit{dest}\hspace{0.2em}\textit{suffix}}'
in the first form.
However, that requires to set up a different file
for each child. With the alternative form of the command
all these files can have exactly the same content
which simplifies setting them up and maintaining them.

For example, the following file |draft.tex|
with a compilation flag |\version| as described in \secref{sec:flags}
compiles the main document as a draft:
%
\begin{center}
\begin{tabular}{l}
|\def\version{draft}|\\
|\input{childdoc.def}|\\
|\childdocforward{|\textit{main}|}|
\end{tabular}
\end{center}
%
Likewise, the following files |final|\textit{nn}|.tex|
compile the final version of the child document
|child|\textit{nn}|.tex|:
%
\begin{center}
\begin{tabular}{l}
|\def\version{final}|\\
|\input{childdoc.def}|\\
|\childdocforwardprefix{final}{child}|
\end{tabular}
\end{center}
%

Note that when several versions of a main file and/or of each child file
are to be generated, it may be convenient to set up a |Makefile| or
shell script to automatise the process.

%%%%%%%%%%%%%%%%%%%%%%%%%%%%%%%%%%%%%%%%%%%%%%%%%%%%%%%%%%%%%%%%%%%%%%%%%%%%%%%%
\subsection{Command Line Processing}
\label{sec:commandline}

The effect of redirection files can also be achieved by invoking
the \LaTeX{} compiler with a more elaborate command line.
Most conveniently this should be done as part
of a shell script or a |Makefile|.

When using \textsf{childdoc} in the main file, the following
command lines effectively perform a redirection
(note that depending on the shell being used,
backslashes may have to be doubled: `|\|' $\to$ `|\\|'):
%
\begin{center}
|... -jobname "|\textit{target}|" |\\|"|[\textit{flags}]%
|\input{childdoc.def}\childdocforward[|\textit{main}|]{|\textit{dest}|}"|
\end{center}
%
Here \textit{target} is the name of the output file,
\textit{main} is the name of the main file
and \textit{dest} is the name of the main or child file to be processed
(all filenames without extensions).
The optional argument \textit{main} can be omitted
if \textit{main} matches \textit{dest}.
Optionally, compilation \textit{flags} can be defined via |\def| commands.
This command line makes the \TeX{} engine believe
it is compiling the file \textit{target}
whose content is specified as the latter parameter.
The provided code then forwards the processing to
\textit{main} or \textit{dest} as described in \secref{sec:forward}.

%%%%%%%%%%%%%%%%%%%%%%%%%%%%%%%%%%%%%%%%%%%%%%%%%%%%%%%%%%%%%%%%%%%%%%%%%%%%%%%%
\subsection{Include by Input}
\label{sec:input}

Including child documents by |\include| has some restrictions by design.
Most notably, the content of a child document always occupies
its own set of pages; pages cannot be shared between child documents.
Usually, this behaviour makes perfect sense
because each child document contain an essential part of the document.
However, in some situations it may be desirable to compose
a document from a collection of parts
without having mandatory page breaks between then.
For this case, the package
provides a mechanism to include parts
by |\input| which can also be processed individually.
However, by construction this mechanism
requires manual handling of the content to be output.

%%%%%%%%%%%%%%%%%%%%%%%%%%%%%%%%%%%%%%%%
\DescribeMacro{\ifchilddocmanual}
The main file should be prepared as usual, see \secref{sec:include}.
However, the document body must make a distinction
between processing of an individual part and of the main document, e.g.:
%
\begin{center}
\begin{tabular}{l}
|\ifchilddocmanual|\\
|\input{\childdocname}|\\
|\||else|\\
\textit{document body with }|\input{|\textit{part}|}|\\
|\||fi|
\end{tabular}
\end{center}
%
The conditional |\ifchilddocmanual| is true whenever
a part to be included by |\input| is being compiled,
and the name of the part is stored in |\childdocname|.

%%%%%%%%%%%%%%%%%%%%%%%%%%%%%%%%%%%%%%%%
\DescribeMacro{\childdocby}
Each part to be included by |\input| should start with:
%
\begin{center}
\begin{tabular}{l}
|\input{childdoc.def}|\\
|\childdocby{|\textit{main}|}|\\
\end{tabular}
\end{center}
%
The directive |\childdocby| is similar to |\childdocof|
described in \secref{sec:include},
but the subsequent selection of content must be done manually.
To that end, both |\ifchilddoc| and |\ifchilddocmanual|
will be true upon processing of a part,
and the name of the part is stored in |\childdocname|.
Note that |\jobname| will be set to the filename of the current part
so that each part receives an individual |.aux| file
that does not interfere with the |.aux| file(s) of the main document.
This behaviour can be altered by the alternative form
|\childdocby[*]{|\textit{main}|}| (with a non-empty optional argument)
which uses the |.aux| file of the main document
by setting |\jobname| to \textit{main}.

%%%%%%%%%%%%%%%%%%%%%%%%%%%%%%%%%%%%%%%%%%%%%%%%%%%%%%%%%%%%%%%%%%%%%%%%%%%%%%%%
\subsection{Driver Development}
\label{sec:driver}

The \textsf{childdoc} mechanism can also be use for the development
of definition files such as \LaTeX{} styles or classes.
This case differs from the above setup with multiple parts
included by |\include| in that no |\includeonly| should be invoked.
This can be achieved by starting the include file
(before |\ProvidesPackage|) with:
%
\begin{center}
\begin{tabular}{l}
|\input{childdoc.def}|\\
|\childdocforward{|\textit{main}|}|\\
\end{tabular}
\end{center}
%
or alternatively with:
%
\begin{center}
\begin{tabular}{l}
|\input{childdoc.def}|\\
|\childdocby{|\textit{main}|}|\\
\end{tabular}
\end{center}
%
Both forms have slightly different effects as described above.
The main file is prepared as usual, see \secref{sec:include}.

%%%%%%%%%%%%%%%%%%%%%%%%%%%%%%%%%%%%%%%%%%%%%%%%%%%%%%%%%%%%%%%%%%%%%%%%%%%%%%%%
\subsection{Legacy Detection}
\label{sec:detection}

The directive |\childdocmain| in the main file can detect
whether the complete document or merely a child is to be compiled
even without using the directive |\childdocof|.
This method is deprecated because it is less robust
and there is no compelling reason to use it;
it is merely provided for backward compatibility
and it may be removed in future versions.

If the detection mechanism is to be used,
it is mandatory to correctly specify
the filename of the main file as the argument of |\childdocmain|:
%
\begin{center}
\begin{tabular}{l}
|\input{childdoc.def}|\\
|\childdocmain{|\textit{main}|}|\\
\end{tabular}
\end{center}
%
If |\jobname| does not match the argument \textit{main} of |\childdocmain|,
it is assumed that |\jobname| points to the child file to be compiled.
When using |\childdocmain| with the main file specified as argument,
it suffices to start a child file
with just |\input{|\textit{main}|}|
without loading of the package and using |\childdocof|.
If instead all processing is done
with the appropriate \textsf{childdoc} directives,
the argument of \textit{main} of |\childdocmain| can be empty.

An alternative version of the command line processing described
in \secref{sec:commandline} using the detection mechanism reads:
%
\begin{center}
|... -jobname "|\textit{target}|" "|[\textit{flags}]%
[|\def\jobname{|\textit{dest}|}|]|\input{|\textit{main}|}"|
\end{center}

%%%%%%%%%%%%%%%%%%%%%%%%%%%%%%%%%%%%%%%%%%%%%%%%%%%%%%%%%%%%%%%%%%%%%%%%%%%%%%%%
\subsection{Manual Code}
\label{sec:manual}

In case one cannot be certain whether the definitions file |childdoc.def|
is installed on the target \TeX{} distribution
and one prefers not to ship it,
it is conceivable to paste a few relevant commands into the sources.

To that end, drop all statements |\input{childdoc.def}|
and perform the replacements as outlined below.
Instead of |\childdocmain{|\textit{main}|}| add the following code
to the top of the main file:
%
\begin{center}
\begin{tabular}{l}
|\||ifdefined\childdocname\endinput\||fi\newif\ifchilddoc|\\
|\edef\childdocname{\scantokens\expandafter{\jobname\noexpand}}|\\
|\def\childdocmain{|\textit{main}|}\||ifx\childdocmain\childdocname\||else|\\
|\childdoctrue\includeonly{\childdocname}\let\jobname\childdocmain\||fi|\\
\end{tabular}
\end{center}
%
Instead of |\childdocof{|\textit{main}|}| just include the main file
at the top of each child file:
%
\begin{center}
|\input{|\textit{main}|}|
\end{center}
%
A simple redirection |\childdocforward{|\textit{dest}|}| is achieved by:
%
\begin{center}
|\def\jobname{|\textit{dest}|}\input{\jobname}|
\end{center}
%
The redirection with prefix
|\childdocforwardprefix[|\textit{prefix}|]{|\textit{dest}|}|
is accomplished by:
%
\begin{center}
\begin{tabular}{l}
|{\edef\jobname{\scantokens\expandafter{\jobname\noexpand}}|\\
|\def\redirectjob |\textit{prefix}|#1~~~{\gdef\jobname{|\textit{dest}|#1}}|\\
|\expandafter\redirectjob\jobname~~~}\input{\jobname}|
\end{tabular}
\end{center}

In an alternative approach,
child documents can be compiled by a specific command line
without additional code or specific definitions:
%
\begin{center}
|... -jobname "|\textit{target}|" "|[\textit{flags}]%
|\includeonly{|\textit{dest}|}\input{|\textit{main}|}"|
\end{center}
%

%%%%%%%%%%%%%%%%%%%%%%%%%%%%%%%%%%%%%%%%%%%%%%%%%%%%%%%%%%%%%%%%%%%%%%%%%%%%%%%%
%%%%%%%%%%%%%%%%%%%%%%%%%%%%%%%%%%%%%%%%%%%%%%%%%%%%%%%%%%%%%%%%%%%%%%%%%%%%%%%%
\section{Information}

%%%%%%%%%%%%%%%%%%%%%%%%%%%%%%%%%%%%%%%%%%%%%%%%%%%%%%%%%%%%%%%%%%%%%%%%%%%%%%%%
\subsection{Copyright}

Copyright \copyright{} 2017--2018 Niklas Beisert

This work may be distributed and/or modified under the
conditions of the \LaTeX{} Project Public License, either version 1.3
of this license or (at your option) any later version.
The latest version of this license is in
  \url{http://www.latex-project.org/lppl.txt}
and version 1.3 or later is part of all distributions of \LaTeX{}
version 2005/12/01 or later.

This work has the LPPL maintenance status `maintained'.

The Current Maintainer of this work is Niklas Beisert.

This work consists of the files |README.txt|, |childdoc.ins| and |childdoc.dtx|
as well as the derived files |childdoc.def|, |cdocsamp.tex|
with |cdocsch1.tex|, |cdocsch2.tex|, |cdocspt3.tex|, |cdocspt4.tex|,
|cdocsdrf.tex|, |cdocsfn1.tex|, |cdocsfn2.tex|
as well as |childdoc.pdf|.

%%%%%%%%%%%%%%%%%%%%%%%%%%%%%%%%%%%%%%%%%%%%%%%%%%%%%%%%%%%%%%%%%%%%%%%%%%%%%%%%
\subsection{Files and Installation}

The package consists of the files:
%
\begin{center}
\begin{tabular}{ll}
    |README.txt|   & readme file \\
    |childdoc.ins| & installation file \\
    |childdoc.dtx| & source file \\
    |childdoc.def| & definition file \\
    |cdocsamp.tex| & sample main file \\
    |cdocsch1.tex| & sample include file \\
    |cdocsch2.tex| & sample include file \\
    |cdocspt3.tex| & sample part file \\
    |cdocspt4.tex| & sample part file \\
    |cdocsdrf.tex| & sample redirection file \\
    |cdocsfn1.tex| & sample redirection file \\
    |cdocsfn2.tex| & sample redirection file \\
    |childdoc.pdf| & manual
\end{tabular}
\end{center}
%
The distribution consists of the files
|README.txt|, |childdoc.ins| and |childdoc.dtx|.
%
\begin{itemize}
\item
Run (pdf)\LaTeX{} on |childdoc.dtx|
to compile the manual |childdoc.pdf| (this file).
\item
Run \LaTeX{} on |childdoc.ins| to create the definitions file |childdoc.def|
and the sample |cdocsamp.tex| with include files
|cdocsch1.tex|, |cdocsch2.tex|, |cdocspt3.tex|, |cdocspt4.tex|,
|cdocsdrf.tex|, |cdocsfn1.tex|, |cdocsfn2.tex|.
Then copy the file |childdoc.def| to an appropriate directory of your \LaTeX{}
distribution, e.g.\ \textit{texmf-root}|/tex/latex/childdoc|.
\end{itemize}

%%%%%%%%%%%%%%%%%%%%%%%%%%%%%%%%%%%%%%%%%%%%%%%%%%%%%%%%%%%%%%%%%%%%%%%%%%%%%%%%
\subsection{Related CTAN Packages}

There are several other packages which offer a similar functionality:
%
\begin{itemize}
\item
The packages
\href{http://ctan.org/pkg/docmute}{\textsf{docmute}},
\href{http://ctan.org/pkg/includex}{\textsf{includex}} and
\href{http://ctan.org/pkg/standalone}{\textsf{standalone}}
provide commands to include only the document body of
a child file thus allowing both files to be compiled individually.
\item
The packages \href{http://ctan.org/pkg/subdocs}{\textsf{subdocs}}
and \href{http://ctan.org/pkg/subfiles}{\textsf{subfiles}}
provide structures in which the main and child documents can be
encapsulated and allowing them to be compiled individually.
The inclusion mechanism is different from the conventional |\include|.
\item
The package \href{http://ctan.org/pkg/combine}{\textsf{combine}}
is an elaborate solution to combine several documents into one.
\end{itemize}
%
See also the CTAN topic \href{http://ctan.org/topic/subdocs}{\textsf{subdocs}}
for further related packages.
The present package differs from the above solutions in that
a document structure constructed with the conventional |\include| mechanism
just needs two extra commands at the top of every file
such that all constituent files can be compiled individually.

%%%%%%%%%%%%%%%%%%%%%%%%%%%%%%%%%%%%%%%%%%%%%%%%%%%%%%%%%%%%%%%%%%%%%%%%%%%%%%%%
%\subsection{Feature Suggestions}
%
%The following is a list of features which may be useful for future
%versions of this package:
%%
%\begin{itemize}
%\item
%\ldots
%\end{itemize}

%%%%%%%%%%%%%%%%%%%%%%%%%%%%%%%%%%%%%%%%%%%%%%%%%%%%%%%%%%%%%%%%%%%%%%%%%%%%%%%%
\subsection{Revision History}

%%%%%%%%%%%%%%%%%%%%%%%%%%%%%%%%%%%%%%%%
\paragraph{v2.0:} 2018/12/30

\begin{itemize}
\item
immediate forward processing
\item
added |\childdocby| mechanism
\item
manual restructured
\end{itemize}

%%%%%%%%%%%%%%%%%%%%%%%%%%%%%%%%%%%%%%%%
\paragraph{v1.6:} 2018/01/17

\begin{itemize}
\item
application for development of include files
\item
corrections to manual
\end{itemize}

%%%%%%%%%%%%%%%%%%%%%%%%%%%%%%%%%%%%%%%%
\paragraph{v1.5:} 2017/05/21

\begin{itemize}
\item
more complete structuring introduced
\item
|\childdocof| introduced
\item
|\childdoc| renamed to |\childdocmain|
\item
|\childredirect| renamed to |\childdocforward| and |\childdocforwardprefix|
and functionality expanded
\end{itemize}

%%%%%%%%%%%%%%%%%%%%%%%%%%%%%%%%%%%%%%%%
\paragraph{v1.0:} 2017/04/27

\begin{itemize}
\item
manual and install package
\item
first version published on CTAN
\end{itemize}

%%%%%%%%%%%%%%%%%%%%%%%%%%%%%%%%%%%%%%%%
\paragraph{v0.6:} 2017/04/26

\begin{itemize}
\item
redirection mechanism added
\end{itemize}

%%%%%%%%%%%%%%%%%%%%%%%%%%%%%%%%%%%%%%%%
\paragraph{v0.5:} 2017/04/26

\begin{itemize}
\item
functionality in definition file
\end{itemize}


%%%%%%%%%%%%%%%%%%%%%%%%%%%%%%%%%%%%%%%%%%%%%%%%%%%%%%%%%%%%%%%%%%%%%%%%%%%%%%%%
%%%%%%%%%%%%%%%%%%%%%%%%%%%%%%%%%%%%%%%%%%%%%%%%%%%%%%%%%%%%%%%%%%%%%%%%%%%%%%%%
%%%%%%%%%%%%%%%%%%%%%%%%%%%%%%%%%%%%%%%%%%%%%%%%%%%%%%%%%%%%%%%%%%%%%%%%%%%%%%%%
\appendix

\settowidth\MacroIndent{\rmfamily\scriptsize 000\ }

 \DocInput{childdoc.dtx}

\end{document}
%</driver>
% \fi
%
% %%%%%%%%%%%%%%%%%%%%%%%%%%%%%%%%%%%%%%%%%%%%%%%%%%%%%%%%%%%%%%%%%%%%%%%%%%%%%%
% %%%%%%%%%%%%%%%%%%%%%%%%%%%%%%%%%%%%%%%%%%%%%%%%%%%%%%%%%%%%%%%%%%%%%%%%%%%%%%
% \section{Sample}
%\iffalse
%<*samplemain>
%\fi
%
% The following presents a sample document
% with two chapters, two parts, a title page,
% a compile flag as well as three forwarding files to set the flag.
% It consists of eight |.tex| files:
% \begin{center}
% \begin{tabular}{ll}
% |cdocsamp.tex|&main file\\
% |cdocsch1.tex|&include file for chapter 1\\
% |cdocsch2.tex|&include file for chapter 2\\
% |cdocspt3.tex|&include file for part 3\\
% |cdocspt4.tex|&include file for part 4\\
% |cdocsdrf.tex|&forwarding file for main file in draft mode\\
% |cdocsfi1.tex|&forwarding file for final version of chapter 1\\
% |cdocsfi2.tex|&forwarding file for final version of chapter 2\\
% \end{tabular}
% \end{center}
% Each of the eight files can be compiled directly by the \LaTeX{} compiler.
%
% %%%%%%%%%%%%%%%%%%%%%%%%%%%%%%%%%%%%%%
% \paragraph{Main File.}
%
% The main file is called |cdocsamp.tex|.
%
% Load the \textsf{childdoc} definitions and
% declare the filename for the main document:
%    \begin{macrocode}
\input{childdoc.def}
\childdocmain{}
%    \end{macrocode}

% Optional override for |\version| flag:
%    \begin{macrocode}
%%\ifchilddoc\else\providecommand{\version}{draft}\fi
%    \end{macrocode}

% Define the default values for the |\version| flag
% (|final| for the main file and |draft| for childs):
%    \begin{macrocode}
\ifchilddoc
\providecommand{\version}{draft}
\else
\providecommand{\version}{final}
\fi
%    \end{macrocode}

% Load the standard document class:
%    \begin{macrocode}
\documentclass[12pt]{article}
%    \end{macrocode}

% Start the document body:
%    \begin{macrocode}
\begin{document}
%    \end{macrocode}

% Declare a title page.
% Print title, part of document being processed and version flag:
%    \begin{macrocode}
\addtocounter{page}{-1}
\begin{center}
{\LARGE\bfseries{}childdoc example\par}
\vspace{1cm}
\ifchilddoc
\ifchilddocmanual part\else chapter\fi:
`\childdocname' of `\childdocjob'\par
\else
main document: `\childdocjob'\par
\fi
version: \version\par
\end{center}
\newpage
%    \end{macrocode}

% Manually include selected file,
% otherwise process as usual:
%    \begin{macrocode}
\ifchilddocmanual
\section*{part `\childdocname'}
\input{\childdocname}
\else
%    \end{macrocode}

% Include the two chapters:
%    \begin{macrocode}
\include{cdocsch1}
\include{cdocsch2}
%    \end{macrocode}

% Include the two parts unless only chapters should be displayed:
%    \begin{macrocode}
\ifchilddoc\else
\section{part three}
\input{cdocspt3}
\section{part four}
\input{cdocspt4}
\fi
%    \end{macrocode}

% Process as usual until here:
%    \begin{macrocode}
\fi
%    \end{macrocode}

% End of document body:
%    \begin{macrocode}
\end{document}
%    \end{macrocode}
%\iffalse
%</samplemain>
%\fi
%
% %%%%%%%%%%%%%%%%%%%%%%%%%%%%%%%%%%%%%%
% \paragraph{Chapter Include Files.}
%
% The include files are called |cdocsch1.tex| and |cdocsch2.tex|.
%
%\iffalse
%<*samplechap1|samplechap2>
%\fi

% Optional override for |\version| flag:
%    \begin{macrocode}
%%\providecommand{\version}{final}
%    \end{macrocode}

% Include the main document:
%    \begin{macrocode}
\input{childdoc.def}
\childdocof{cdocsamp}
%    \end{macrocode}

%\iffalse
%</samplechap1|samplechap2>
%\fi
%
%\iffalse
%<*samplechap1>
%\fi
% Some text for chapter 1:
%    \begin{macrocode}
\section{one}
some text in chapter one
%    \end{macrocode}

%\iffalse
%</samplechap1>
%\fi
% Some text for chapter 2:
%\iffalse
%<*samplechap2>
%\fi
%    \begin{macrocode}
\section{two}
more text in chapter two
%    \end{macrocode}

%\iffalse
%</samplechap2>
%\fi
%
% %%%%%%%%%%%%%%%%%%%%%%%%%%%%%%%%%%%%%%
% \paragraph{Part Include Files.}
%
% The include files are called |cdocspt3.tex| and |cdocspt4.tex|.
%
%\iffalse
%<*samplepart3|samplepart4>
%\fi

% Optional override for |\version| flag:
%    \begin{macrocode}
%%\providecommand{\version}{final}
%    \end{macrocode}

% Include the main document:
%    \begin{macrocode}
\input{childdoc.def}
\childdocby{cdocsamp}
%    \end{macrocode}

%\iffalse
%</samplepart3|samplepart4>
%\fi
%
%\iffalse
%<*samplepart3>
%\fi
% Some text for part 3:
%    \begin{macrocode}
some text in part three
%    \end{macrocode}

%\iffalse
%</samplepart3>
%\fi
% Some text for part 4:
%\iffalse
%<*samplepart4>
%\fi
%    \begin{macrocode}
more text in part four
%    \end{macrocode}

%\iffalse
%</samplepart4>
%\fi
%
% %%%%%%%%%%%%%%%%%%%%%%%%%%%%%%%%%%%%%%
% \paragraph{Forwarding for a Complete Draft.}
%
% The following forwarding file |cdocsdrf.tex|
% compiles the main document in draft mode:
%\iffalse
%<*sampledraft>
%\fi
%    \begin{macrocode}
\def\version{draft}
\input{childdoc.def}
\childdocforward{cdocsamp}
%    \end{macrocode}

%\iffalse
%</sampledraft>
%\fi
%
% %%%%%%%%%%%%%%%%%%%%%%%%%%%%%%%%%%%%%%
% \paragraph{Forwarding for Final Version of the Chapters.}
%
% The following forwarding files |cdocsfn1.tex| and |cdocsfn2.tex|
% (with identical content)
% compile the final versions of the child documents
% |cdocsch1.tex| and |cdocsch2.tex|, respectively:
%\iffalse
%<*samplefinal>
%\fi
%    \begin{macrocode}
\def\version{final}
\input{childdoc.def}
\childdocforwardprefix[cdocsamp]{cdocsfn}{cdocsch}
%    \end{macrocode}

%\iffalse
%</samplefinal>
%\fi
%
% %%%%%%%%%%%%%%%%%%%%%%%%%%%%%%%%%%%%%%
% \paragraph{Command Line Processing.}
%
% The following three command lines generate the output files
% |cdocscld|, |cdocscl1| and |cdocscl2|
% which should be identical to
% |cdocsdrf|, |cdocsch1| and |cdocsfn2|, respectively:
% \begin{center}
% \begin{tabular}{l}
% |latex -jobname cdocscld \|\\
% |  "\def\version{draft}\input{childdoc.def}\childdocforward{cdocsamp}"|\\
% |latex -jobname cdocscl1 \|\\
% |  "\input{childdoc.def}\childdocforward[cdocsamp]{cdocsch1}"|\\
% |latex -jobname cdocscl2 \|\\
% |  "\def\version{final}\input{childdoc.def}\childdocforward{cdocsch2}"|
% \end{tabular}
% \end{center}
% Note that the trailing backslash on each first line
% merely continues the input to the second line
% (for convenient cut ant paste).
% Furthermore, the command |latex| can be replaced by any
% of its alternative versions such as |pdflatex|.
%
% %%%%%%%%%%%%%%%%%%%%%%%%%%%%%%%%%%%%%%%%%%%%%%%%%%%%%%%%%%%%%%%%%%%%%%%%%%%%%%
% %%%%%%%%%%%%%%%%%%%%%%%%%%%%%%%%%%%%%%%%%%%%%%%%%%%%%%%%%%%%%%%%%%%%%%%%%%%%%%
% \section{Implementation}
%\iffalse
%<*package>
%\fi
%
% This section describes the definitions file |childdoc.def|.

% The definitions cannot be loaded using |\usepackage| or |\RequirePackage|
% which has a mechanism to prevent loading a style file more than once.
% When loading the definitions by means of |\input|
% multiple instances have to be prevented manually:
%\iffalse
%This code needs to be before the `\ProvidesFile' directive
%which is defined at the beginning of this file.
%Therefore it is also placed there and commented out here.
%</package>
%<*discard>
%\fi
%    \begin{macrocode}
\ifdefined\childdocmain\endinput\fi
%    \end{macrocode}
%\iffalse
%</discard>
%<*package>
%\fi
%
% \macro{\ifchilddoc}
% \macro{\ifchilddocmanual}
% The conditional |\ifchilddoc| tells whether a
% child (true) or main (false) document is being compiled.
% The conditional |\ifchilddocmanual| tells whether
% the |\includeonly| mechanism is used (false) or
% the selection of child files must be performed manually (true).
% The definitions initialise to false:
%    \begin{macrocode}
\newif\ifchilddoc
\newif\ifchilddocmanual
%    \end{macrocode}

% \macro{\childdocname}
% \macro{\childdocjob}
% The macro |\childdocname| stores the name of the main document
% to be compiled. The macro |\childdocjob| stores the name of
% the document on which the \LaTeX{} compiler was originally invoked.
% The content of |\jobname| cannot be compared
% to filenames specified in the source due to different catcodes.
% The following code rescans |\jobname|, stores the result
% in |\childdocname| and saves a copy in |\childdocjob|:
%    \begin{macrocode}
\edef\childdocname{\scantokens\expandafter{\jobname\noexpand}}
\let\childdocjob\childdocname
%    \end{macrocode}

% \macro{\childdocdisable}
% The macro |\childdocdisable| prevents the main file
% from being processed more than once.
% At this stage, the main document command |\childdocmain|
% is assumed to be called once again where it should do nothing.
% Any subsequent call to it should prevent
% a secondary processing of the main document
% It overwrites the forwarding commands
% |\childdocof| and |\childdocforward|
% with empty macros to prevent further inclusions of the main document:
%    \begin{macrocode}
\newcommand{\childdocdisable}
{
  \renewcommand{\childdocmain}[1]{\renewcommand{\childdocmain}[1]{\endinput}}
  \renewcommand{\childdocof}[1]{}
  \renewcommand{\childdocby}[2][]{}
  \renewcommand{\childdocforward}[2][]{}
  \renewcommand{\childdocdisable}{}
}
%    \end{macrocode}

% \macro{\childdocmain}
% The macro |\childdocmain| is to be called at the top of the main file
% with nothing or the main filename (without extension) as argument.
% First, it breaks loops.
% If the argument is not empty and does not match |\childdocname|
% (which is set by the first inclusion of |childdoc.def|),
% |\ifchilddoc| is set to true, |\includeonly| is applied to the child file
% and |\jobname| is set to the main file
% (for proper handling of |.aux| files):
%    \begin{macrocode}
\newcommand{\childdocmain}[1]
{
  \childdocdisable\childdocmain{}
  \if?#1?\else
    \begingroup
      \def\childdoctmp{#1}
      \ifx\childdoctmp\childdocname
        \def\childdoctmp{}
      \else
        \def\childdoctmp
        {
          \childdoctrue
          \includeonly{\childdocname}
          \def\childdocjob{#1}
          \def\jobname{#1}
        }
      \fi
      \expandafter
    \endgroup
    \childdoctmp
  \fi
}
%    \end{macrocode}

% \macro{\childdocof}
% The command |\childdocof| redirects
% compilation to the main file |#1|.
%    \begin{macrocode}
\newcommand{\childdocof}[1]
{
  \childdocdisable
  \childdoctrue
  \includeonly{\childdocname}
  \def\jobname{#1}
  \def\childdocjob{#1}
  \input{#1}
}
%    \end{macrocode}

% \macro{\childdocby}
% The command |\childdocby| ....
%    \begin{macrocode}
\newcommand{\childdocby}[2][]
{
  \childdocdisable
  \childdoctrue
  \childdocmanualtrue
  \if?#1?\else
    \def\jobname{#2}
  \fi
  \def\childdocjob{#2}
  \input{#2}
  \endinput
}
%    \end{macrocode}

% \macro{\childdocforward}
% The command |\childdocforward| redirects
% compilation to the main file or
% (if the optional argument is given) a child file.
% Parameters are set as if the main file
% or a child file starting with |\childdocof| was compiled.
% Then compilation is handed over to the main file:
%    \begin{macrocode}
\newcommand{\childdocforward}[2][]
{
  \begingroup
    \if?#1?
      \def\childdoctmp
      {
        \def\childdocname{#2}
        \def\childdocjob{#2}
        \def\jobname{#2}
        \input{#2}
        \endinput
      }
    \else
      \def\childdoctmp
      {
        \childdocdisable
        \def\childdocname{#2}
        \childdoctrue
        \includeonly{#2}
        \def\childdocjob{#1}
        \def\jobname{#1}
        \input{#1}
        \endinput
      }
    \fi
    \expandafter
  \endgroup
  \childdoctmp
}
%    \end{macrocode}

% \macro{\childdocforwardprefix}
% The command |\childdocforwardprefix| redirects
% compilation to the main or a child file by means of a pattern.
% The prefix |#1| in the current filename is replaced by |#2|
% and the suffix of the current filename is kept
% (it is assumed that the filename does not contain the substring `|~~~|'
% which is used as a delimiter).
% Compilation is handed over to the new file by |\childdocforward|:
%    \begin{macrocode}
\newcommand{\childdocforwardprefix}[3][]
{
  \begingroup
    \def\childdocextract #2##1~~~{\def\childdoctmp{\childdocforward[#1]{#3##1}}}
    \expandafter\childdocextract\childdocname~~~
    \expandafter
  \endgroup
  \childdoctmp
}
%    \end{macrocode}

% \macro{\childdoc}
% The deprecated macro |\childdoc| is a legacy version of |\childdocmain|:
%    \begin{macrocode}
\newcommand{\childdoc}{\childdocmain}
%    \end{macrocode}

% \macro{\childdocredirect}
% The deprecated macro |\childdocredirect| is a legacy version
% of |\childdocforward| and |\childdocforwardprefix|:
%    \begin{macrocode}
\newcommand{\childdocredirect}[2][]
{
  \begingroup
    \if?#1?
      \def\childdoctmp{\childdocforward{#2}}
    \else
      \def\childdoctmp{\childdocforwardprefix{#1}{#2}}
    \fi
    \expandafter
  \endgroup
  \childdoctmp
}
%    \end{macrocode}

%\iffalse
%</package>
%\fi
%
\endinput
|\\
|\childdocforward{|\textit{main}|}|
\end{tabular}
\end{center}
%
Likewise, the following files |final|\textit{nn}|.tex|
compile the final version of the child document
|child|\textit{nn}|.tex|:
%
\begin{center}
\begin{tabular}{l}
|\def\version{final}|\\
|% \iffalse
%
% childdoc.dtx Copyright (C) 2017-2018 Niklas Beisert
%
% This work may be distributed and/or modified under the
% conditions of the LaTeX Project Public License, either version 1.3
% of this license or (at your option) any later version.
% The latest version of this license is in
%   http://www.latex-project.org/lppl.txt
% and version 1.3 or later is part of all distributions of LaTeX
% version 2005/12/01 or later.
%
% This work has the LPPL maintenance status `maintained'.
%
% The Current Maintainer of this work is Niklas Beisert.
%
% This work consists of the files childdoc.dtx and childdoc.ins
% and the derived files childdoc.def and cdocsamp.tex with
% cdocsch1.tex, cdocsch2.tex, cdocsdrf.tex, cdocsfn1.tex, cdocsfn2.tex.
%
%<package>\ifdefined\childdocmain\endinput\fi
%<package>\ProvidesFile{childdoc.def}[2018/12/30 v2.0 child document driver]
%<samplemain>\ProvidesFile{cdocsamp.tex}[2018/12/30 v2.0 sample for childdoc]
%<*driver>
%\ProvidesFile{childdoc.drv}[2018/12/30 v2.0 childdoc reference manual file]
\PassOptionsToClass{10pt,a4paper}{article}
\documentclass{ltxdoc}

\usepackage[margin=35mm]{geometry}
\usepackage{hyperref}
\usepackage{hyperxmp}
\usepackage[usenames]{color}

\hypersetup{colorlinks=true}
\hypersetup{pdfstartview=FitH}
\hypersetup{pdfpagemode=UseNone}
\hypersetup{pdfsource={}}
\hypersetup{pdflang={en-UK}}
\hypersetup{pdfcopyright={Copyright 2017-2018 Niklas Beisert.
  This work may be distributed and/or modified under the
  conditions of the LaTeX Project Public License, either version 1.3
  of this license or (at your option) any later version.}}
\hypersetup{pdflicenseurl={http://www.latex-project.org/lppl.txt}}
\hypersetup{pdfcontactaddress={ETH Zurich, ITP, HIT K,
  Wolfgang-Pauli-Strasse 27}}
\hypersetup{pdfcontactpostcode={8093}}
\hypersetup{pdfcontactcity={Zurich}}
\hypersetup{pdfcontactcountry={Switzerland}}
\hypersetup{pdfcontactemail={nbeisert@itp.phys.ethz.ch}}
\hypersetup{pdfcontacturl={http://people.phys.ethz.ch/\xmptilde nbeisert/}}

\newcommand{\secref}[1]{\hyperref[#1]{section \ref*{#1}}}

\parskip1ex
\parindent0pt
\let\olditemize\itemize
\def\itemize{\olditemize\parskip0pt}

\begin{document}

\title{The \textsf{childdoc} Package}
\hypersetup{pdftitle={The childdoc Package}}
\author{Niklas Beisert\\[2ex]
  Institut f\"ur Theoretische Physik\\
  Eidgen\"ossische Technische Hochschule Z\"urich\\
  Wolfgang-Pauli-Strasse 27, 8093 Z\"urich, Switzerland\\[1ex]
  \href{mailto:nbeisert@itp.phys.ethz.ch}
  {\texttt{nbeisert@itp.phys.ethz.ch}}}
\hypersetup{pdfauthor={Niklas Beisert}}
\hypersetup{pdfsubject={Manual for the LaTeX2e Package childdoc}}
\date{30 December 2018, \textsf{v2.0}}
\maketitle

\begin{abstract}\noindent
\textsf{childdoc} is a \LaTeXe{} package
that enables the direct compilation
of document sections included by |\include|
to individual files.
\end{abstract}

\begingroup
\parskip0ex
\tableofcontents
\endgroup

%%%%%%%%%%%%%%%%%%%%%%%%%%%%%%%%%%%%%%%%%%%%%%%%%%%%%%%%%%%%%%%%%%%%%%%%%%%%%%%%
%%%%%%%%%%%%%%%%%%%%%%%%%%%%%%%%%%%%%%%%%%%%%%%%%%%%%%%%%%%%%%%%%%%%%%%%%%%%%%%%
\section{Introduction}

\LaTeX{} provides a mechanism to structure a large document (such as a book)
into a main file and several child files (containing the chapters)
using the |\include| command.
This mechanism is beneficial for documents
which span hundreds of pages in order to
make the source file(s) more manageable.
Moreover, compilation can be restricted to
selected child files by means of the |\includeonly| command.
The latter feature can be used to reduce the compilation time while editing
(this was significantly more useful in the earlier days of \LaTeX{})
or to generate a smaller document which is easier to navigate.
Another application of |\includeonly| is to generate
documents consisting of selected parts of the complete document.

However, there are a few drawbacks of the plain |\include| mechanism:
\begin{itemize}
\item
The child files cannot be compiled on their own,
they can only be compiled via the main file.
A naive editing environment
(such as a text editor with an option
to have the current file processed by \LaTeX)
may require one to switch to the main file before compiling;
attempting to compile the child file produces errors.
\item
The main file must be modified (each time)
to adjust the |\includeonly| command
to the present needs. This easily leaves the main file in a messy state.
\item
The generated document will always carry the filename
of the main document. This is inconvenient if
several child files are to be compiled and
to be kept for distribution.
\end{itemize}

The present package provides a simple interface
to make child files individually compilable by \LaTeX{}.
Compiling a child file then has the same effect as compiling
the main file with an |\includeonly| command
to select the appropriate child.
Moreover the generated document will carry the name of the child
rather than the main file.
This resolves all three above issues.

This feature is meant to make the editing of books,
thesis documents and lecture notes somewhat more convenient.
However, the package can also be used efficiently for
composing a series of documents (such as exercise sheets)
which are typically distributed individually.
It then assists the author in generating the individual documents
(potentially in different versions)
as well as a document containing the collected series.
Another application is in developing style files
or other kinds of included material
where compilation of the style file could redirect
to a sample or test file.

%%%%%%%%%%%%%%%%%%%%%%%%%%%%%%%%%%%%%%%%%%%%%%%%%%%%%%%%%%%%%%%%%%%%%%%%%%%%%%%%
%%%%%%%%%%%%%%%%%%%%%%%%%%%%%%%%%%%%%%%%%%%%%%%%%%%%%%%%%%%%%%%%%%%%%%%%%%%%%%%%
\section{Usage}

First of all, the package \textsf{childdoc} is \emph{not} a standard
\LaTeXe{} |.sty| style file! Therefore it needs to be invoked in
a non-standard way.

%%%%%%%%%%%%%%%%%%%%%%%%%%%%%%%%%%%%%%%%%%%%%%%%%%%%%%%%%%%%%%%%%%%%%%%%%%%%%%%%
\subsection{Included Files}
\label{sec:include}

%%%%%%%%%%%%%%%%%%%%%%%%%%%%%%%%%%%%%%%%
\DescribeMacro{\childdocmain}
To use the package, add the commands
\begin{center}
\begin{tabular}{l}
|\input{childdoc.def}|\\
|\childdocmain{}|\\
\end{tabular}
\end{center}
at the very top of the main \LaTeX{} file,
in particular \emph{before} the |\documentclass| statement!
The argument of |\childdocmain| should be left empty
(but it must be present).

%%%%%%%%%%%%%%%%%%%%%%%%%%%%%%%%%%%%%%%%
\DescribeMacro{\childdocof}
Furthermore, add the commands
\begin{center}
\begin{tabular}{l}
|\input{childdoc.def}|\\
|\childdocof{|\textit{main}|}|\\
\end{tabular}
\end{center}
at the top of every child file \textit{child}
which is included by |\include{|\textit{child}|}|
from within the main file
(or at least for those files to be compiled individually).
The argument \textit{main} must be the filename of the main file.

There are a couple of
considerations in setting up the main and child documents:

%%%%%%%%%%%%%%%%%%%%%%%%%%%%%%%%%%%%%%%%
\paragraph{Restrictions.}

Please note the following restrictions:
\begin{itemize}
\item
|\childdocmain| must be called with one argument \textit{main}
to ensure compatibility with earlier version of the package.
It must either be empty (|\childdocmain{}|)
or precisely match the filename of the main file in which it is specified.
See \secref{sec:detection} for further information.
\item
The filename \textit{main} must be specified without the |.tex| extension.
\item
The filename \textit{main} is case sensitive
(even in case-insensitive file systems)
due to internal string comparison.
\item
The argument \textit{main} should be fully expanded, it cannot be a macro.
\item
Subdirectories and special characters should be avoided in filenames.
\item
The command |\childdocmain{|\textit{main}|}| must be followed by a whitespace.
It should not be followed immediately by another command
or by a comment mark `|%|'.
This is because the \TeX{} parser reads the token immediately following
the argument of |\childdocmain| and puts it
at the beginning of every child section;
however, a white\-space is ignored.
\end{itemize}

%%%%%%%%%%%%%%%%%%%%%%%%%%%%%%%%%%%%%%%%
\paragraph{Content of Main File.}

It is advisable to place all content in the child files included by |\include|.
Any output contained in the main file will appear in all child documents
unless suppressed manually;
it cannot be suppressed automatically by the |\includeonly| directive
and thus should normally be avoided.
A method to include some content in the main file
by means of conditional processing is described in \secref{sec:conditional}.

%%%%%%%%%%%%%%%%%%%%%%%%%%%%%%%%%%%%%%%%
\paragraph{Page Numbering.}

When only a part of the document is compiled,
the appropriate numbering of pages
(as well as other status parameters)
is determined from the |.aux| files.
The latter contain information from previous passes.
However this information needs to propagate through
all intermediate child documents.
Therefore the page numbering in child documents may well
be inconsistent until the complete document is compiled at least once.

A useful (if unconventional) way to always ensure a consistent
page numbering is to restart the numbering in each child document
and denote the pages by `\textit{child}|.|\textit{page}'
where \textit{child} represents the chapter/section number of the child file.
This can be achieved by the command
|\numberwithin{page}{|\textit{child}|}|
of the \textsf{amsmath} package
where \textit{child} can be |chapter| or |section|
depending on the chosen structuring.
Alternatively, one can modify the macro |\thepage| appropriately
and reset the counter |page| at the start of each child file.

%%%%%%%%%%%%%%%%%%%%%%%%%%%%%%%%%%%%%%%%%%%%%%%%%%%%%%%%%%%%%%%%%%%%%%%%%%%%%%%%
\subsection{Conditional Processing}
\label{sec:conditional}

The package provides a mechanism to compile different versions
of a document. To customise the versions further some conditional processing
can come in handy to distinguish which version is being compiled.
The package provides two macros to describe the compilation context:

%%%%%%%%%%%%%%%%%%%%%%%%%%%%%%%%%%%%%%%%
\DescribeMacro{\ifchilddoc}
The conditional |\ifchilddoc| distinguishes between the compilation of
child documents and the main document:
%
\begin{center}
|\ifchilddoc |\textit{child-code}| |[|\||else |\textit{main-code}]| \||fi|
\end{center}

%%%%%%%%%%%%%%%%%%%%%%%%%%%%%%%%%%%%%%%%
\DescribeMacro{\childdocname}
\DescribeMacro{\childdocjob}
The macro |\childdocname| contains the filename (without extension)
of the main or child file being processed.
Note that |\childdocjob| will always contain the name of the main file.

%%%%%%%%%%%%%%%%%%%%%%%%%%%%%%%%%%%%%%%%
\paragraph{Title Page.}

Conditional processing can be used to include a title or banner page
in the main document when proper precautions are taken.
Importantly, the code in the main file should ensure that the page counter
(as well as other status parameters which are stored in the |.aux| files)
takes the same value after the conditional processing.
Otherwise the page numbers may take divergent values
depending on which part is compiled.

For example, a title page could be declared by:
%
\begin{center}
\begin{tabular}{l}
|\ifchilddoc\||else|\\
|\addtocounter{page}{-1}|\\
\textit{code for title page}\\
|\newpage|\\
|\||fi|
\end{tabular}
\end{center}
%
A banner page for the child documents can be generated by:
%
\begin{center}
\begin{tabular}{l}
|\ifchilddoc|\\
|\addtocounter{page}{-1}|\\
\textit{code for banner page}\\
|\newpage|\\
|\||fi|
\end{tabular}
\end{center}
%
Here one could write a message such as:
\begin{center}
|This is the part \childdocname{} of \childdocjob{}.|
\end{center}

%%%%%%%%%%%%%%%%%%%%%%%%%%%%%%%%%%%%%%%%%%%%%%%%%%%%%%%%%%%%%%%%%%%%%%%%%%%%%%%%
\subsection{Flags}
\label{sec:flags}

The package makes it easy to generate different versions
of the main or child documents.
To this end compilation flags can be defined
and assigned different default values.
They will be particularly useful in conjunction
with the forwarding mechanism described in \secref{sec:forward}.

For example, it may be useful to have a flag |\version|
which can be set to |draft| or |final|.
The document source will contain some conditional code
depending on the value of |\version|.
Suppose further, the flag should default to |final| for the main file
and to |draft| for child files
which is a natural assignment for editing the document.
This is achieved by placing the following code
in the preamble of the main document
(below the |\childdocmain| directive):
%
\begin{center}
\begin{tabular}{l}
|\ifchilddoc|\\
|\providecommand{\version}{draft}|\\
|\||else|\\
|\providecommand{\version}{final}|\\
|\||fi|
\end{tabular}
\end{center}
%
The definition by |\providecommand| makes sure
that previous definitions are not overwritten.
Further statements |\providecommand{\version}{...}|
can thus be added before the above code to override it.

For the main file, one might add a line
(between |\childdocmain| and the above block)
%
\begin{center}
|%\ifchilddoc\||else\providecommand{\version}{draft}\||fi|
\end{center}
%
which can be uncommented to produce a draft version.
Likewise one can add a line to the very top of a child file
(above the |\childdocof{|\textit{main}|}| directive)
%
\begin{center}
|%\providecommand{\version}{final}|
\end{center}
%
which can be uncommented to produce the final version of this child document.

%%%%%%%%%%%%%%%%%%%%%%%%%%%%%%%%%%%%%%%%%%%%%%%%%%%%%%%%%%%%%%%%%%%%%%%%%%%%%%%%
\subsection{Forwarding}
\label{sec:forward}

Different versions of the main or child documents
using compilation flags as described in \secref{sec:flags}
can be (permanently) stored in different files
for convenient compilation, viewing and distribution.
To this end, the package defines a command
to pass on compilation to a different file:

%%%%%%%%%%%%%%%%%%%%%%%%%%%%%%%%%%%%%%%%
\DescribeMacro{\childdocforward}
The command |\childdocforward| redirects processing to
another source file:
%
\begin{center}
\begin{tabular}{l}
|\input{childdoc.def}|\\
|\childdocforward[|\textit{main}|]{|\textit{dest}|}|\\
\end{tabular}
\end{center}
%
The argument \textit{dest} is the destination file
(without extension).
It should be the main file or one of the child files.
Note that further \textsf{childdoc} directives
such as |\childdocof| and |\childdocforward|
in the indicated file will be processed in this form.
The optional argument \textit{main}
passes on directly to the main file \textit{main}
while pretending to compile the child \textit{dest}.
This form behaves as if \textit{dest}
issues |\childdocof{|\textit{main}|}| right away,
and no further \textsf{childdoc} directives will be processed.

%%%%%%%%%%%%%%%%%%%%%%%%%%%%%%%%%%%%%%%%
\DescribeMacro{\...prefix}
In the alternative form |\childdocforwardprefix|,
%
\begin{center}
\begin{tabular}{l}
|\input{childdoc.def}|\\
|\childdocforwardprefix[|\textit{main}|]{|\textit{prefix}|}{|\textit{dest}|}|
\end{tabular}
\end{center}
%
the destination file is determined by a pattern
depending on the current file:
To make this work, the current file must be called
`{\textit{prefix}\hspace{0.2em}\textit{suffix}}'
with \textit{prefix} matching precisely the argument.
Processing is then passed on to the file
`{\textit{dest}\hspace{0.2em}\textit{suffix}}'.
Surely, the same effect is achieved by
directly specifying the
argument `{\textit{dest}\hspace{0.2em}\textit{suffix}}'
in the first form.
However, that requires to set up a different file
for each child. With the alternative form of the command
all these files can have exactly the same content
which simplifies setting them up and maintaining them.

For example, the following file |draft.tex|
with a compilation flag |\version| as described in \secref{sec:flags}
compiles the main document as a draft:
%
\begin{center}
\begin{tabular}{l}
|\def\version{draft}|\\
|\input{childdoc.def}|\\
|\childdocforward{|\textit{main}|}|
\end{tabular}
\end{center}
%
Likewise, the following files |final|\textit{nn}|.tex|
compile the final version of the child document
|child|\textit{nn}|.tex|:
%
\begin{center}
\begin{tabular}{l}
|\def\version{final}|\\
|\input{childdoc.def}|\\
|\childdocforwardprefix{final}{child}|
\end{tabular}
\end{center}
%

Note that when several versions of a main file and/or of each child file
are to be generated, it may be convenient to set up a |Makefile| or
shell script to automatise the process.

%%%%%%%%%%%%%%%%%%%%%%%%%%%%%%%%%%%%%%%%%%%%%%%%%%%%%%%%%%%%%%%%%%%%%%%%%%%%%%%%
\subsection{Command Line Processing}
\label{sec:commandline}

The effect of redirection files can also be achieved by invoking
the \LaTeX{} compiler with a more elaborate command line.
Most conveniently this should be done as part
of a shell script or a |Makefile|.

When using \textsf{childdoc} in the main file, the following
command lines effectively perform a redirection
(note that depending on the shell being used,
backslashes may have to be doubled: `|\|' $\to$ `|\\|'):
%
\begin{center}
|... -jobname "|\textit{target}|" |\\|"|[\textit{flags}]%
|\input{childdoc.def}\childdocforward[|\textit{main}|]{|\textit{dest}|}"|
\end{center}
%
Here \textit{target} is the name of the output file,
\textit{main} is the name of the main file
and \textit{dest} is the name of the main or child file to be processed
(all filenames without extensions).
The optional argument \textit{main} can be omitted
if \textit{main} matches \textit{dest}.
Optionally, compilation \textit{flags} can be defined via |\def| commands.
This command line makes the \TeX{} engine believe
it is compiling the file \textit{target}
whose content is specified as the latter parameter.
The provided code then forwards the processing to
\textit{main} or \textit{dest} as described in \secref{sec:forward}.

%%%%%%%%%%%%%%%%%%%%%%%%%%%%%%%%%%%%%%%%%%%%%%%%%%%%%%%%%%%%%%%%%%%%%%%%%%%%%%%%
\subsection{Include by Input}
\label{sec:input}

Including child documents by |\include| has some restrictions by design.
Most notably, the content of a child document always occupies
its own set of pages; pages cannot be shared between child documents.
Usually, this behaviour makes perfect sense
because each child document contain an essential part of the document.
However, in some situations it may be desirable to compose
a document from a collection of parts
without having mandatory page breaks between then.
For this case, the package
provides a mechanism to include parts
by |\input| which can also be processed individually.
However, by construction this mechanism
requires manual handling of the content to be output.

%%%%%%%%%%%%%%%%%%%%%%%%%%%%%%%%%%%%%%%%
\DescribeMacro{\ifchilddocmanual}
The main file should be prepared as usual, see \secref{sec:include}.
However, the document body must make a distinction
between processing of an individual part and of the main document, e.g.:
%
\begin{center}
\begin{tabular}{l}
|\ifchilddocmanual|\\
|\input{\childdocname}|\\
|\||else|\\
\textit{document body with }|\input{|\textit{part}|}|\\
|\||fi|
\end{tabular}
\end{center}
%
The conditional |\ifchilddocmanual| is true whenever
a part to be included by |\input| is being compiled,
and the name of the part is stored in |\childdocname|.

%%%%%%%%%%%%%%%%%%%%%%%%%%%%%%%%%%%%%%%%
\DescribeMacro{\childdocby}
Each part to be included by |\input| should start with:
%
\begin{center}
\begin{tabular}{l}
|\input{childdoc.def}|\\
|\childdocby{|\textit{main}|}|\\
\end{tabular}
\end{center}
%
The directive |\childdocby| is similar to |\childdocof|
described in \secref{sec:include},
but the subsequent selection of content must be done manually.
To that end, both |\ifchilddoc| and |\ifchilddocmanual|
will be true upon processing of a part,
and the name of the part is stored in |\childdocname|.
Note that |\jobname| will be set to the filename of the current part
so that each part receives an individual |.aux| file
that does not interfere with the |.aux| file(s) of the main document.
This behaviour can be altered by the alternative form
|\childdocby[*]{|\textit{main}|}| (with a non-empty optional argument)
which uses the |.aux| file of the main document
by setting |\jobname| to \textit{main}.

%%%%%%%%%%%%%%%%%%%%%%%%%%%%%%%%%%%%%%%%%%%%%%%%%%%%%%%%%%%%%%%%%%%%%%%%%%%%%%%%
\subsection{Driver Development}
\label{sec:driver}

The \textsf{childdoc} mechanism can also be use for the development
of definition files such as \LaTeX{} styles or classes.
This case differs from the above setup with multiple parts
included by |\include| in that no |\includeonly| should be invoked.
This can be achieved by starting the include file
(before |\ProvidesPackage|) with:
%
\begin{center}
\begin{tabular}{l}
|\input{childdoc.def}|\\
|\childdocforward{|\textit{main}|}|\\
\end{tabular}
\end{center}
%
or alternatively with:
%
\begin{center}
\begin{tabular}{l}
|\input{childdoc.def}|\\
|\childdocby{|\textit{main}|}|\\
\end{tabular}
\end{center}
%
Both forms have slightly different effects as described above.
The main file is prepared as usual, see \secref{sec:include}.

%%%%%%%%%%%%%%%%%%%%%%%%%%%%%%%%%%%%%%%%%%%%%%%%%%%%%%%%%%%%%%%%%%%%%%%%%%%%%%%%
\subsection{Legacy Detection}
\label{sec:detection}

The directive |\childdocmain| in the main file can detect
whether the complete document or merely a child is to be compiled
even without using the directive |\childdocof|.
This method is deprecated because it is less robust
and there is no compelling reason to use it;
it is merely provided for backward compatibility
and it may be removed in future versions.

If the detection mechanism is to be used,
it is mandatory to correctly specify
the filename of the main file as the argument of |\childdocmain|:
%
\begin{center}
\begin{tabular}{l}
|\input{childdoc.def}|\\
|\childdocmain{|\textit{main}|}|\\
\end{tabular}
\end{center}
%
If |\jobname| does not match the argument \textit{main} of |\childdocmain|,
it is assumed that |\jobname| points to the child file to be compiled.
When using |\childdocmain| with the main file specified as argument,
it suffices to start a child file
with just |\input{|\textit{main}|}|
without loading of the package and using |\childdocof|.
If instead all processing is done
with the appropriate \textsf{childdoc} directives,
the argument of \textit{main} of |\childdocmain| can be empty.

An alternative version of the command line processing described
in \secref{sec:commandline} using the detection mechanism reads:
%
\begin{center}
|... -jobname "|\textit{target}|" "|[\textit{flags}]%
[|\def\jobname{|\textit{dest}|}|]|\input{|\textit{main}|}"|
\end{center}

%%%%%%%%%%%%%%%%%%%%%%%%%%%%%%%%%%%%%%%%%%%%%%%%%%%%%%%%%%%%%%%%%%%%%%%%%%%%%%%%
\subsection{Manual Code}
\label{sec:manual}

In case one cannot be certain whether the definitions file |childdoc.def|
is installed on the target \TeX{} distribution
and one prefers not to ship it,
it is conceivable to paste a few relevant commands into the sources.

To that end, drop all statements |\input{childdoc.def}|
and perform the replacements as outlined below.
Instead of |\childdocmain{|\textit{main}|}| add the following code
to the top of the main file:
%
\begin{center}
\begin{tabular}{l}
|\||ifdefined\childdocname\endinput\||fi\newif\ifchilddoc|\\
|\edef\childdocname{\scantokens\expandafter{\jobname\noexpand}}|\\
|\def\childdocmain{|\textit{main}|}\||ifx\childdocmain\childdocname\||else|\\
|\childdoctrue\includeonly{\childdocname}\let\jobname\childdocmain\||fi|\\
\end{tabular}
\end{center}
%
Instead of |\childdocof{|\textit{main}|}| just include the main file
at the top of each child file:
%
\begin{center}
|\input{|\textit{main}|}|
\end{center}
%
A simple redirection |\childdocforward{|\textit{dest}|}| is achieved by:
%
\begin{center}
|\def\jobname{|\textit{dest}|}\input{\jobname}|
\end{center}
%
The redirection with prefix
|\childdocforwardprefix[|\textit{prefix}|]{|\textit{dest}|}|
is accomplished by:
%
\begin{center}
\begin{tabular}{l}
|{\edef\jobname{\scantokens\expandafter{\jobname\noexpand}}|\\
|\def\redirectjob |\textit{prefix}|#1~~~{\gdef\jobname{|\textit{dest}|#1}}|\\
|\expandafter\redirectjob\jobname~~~}\input{\jobname}|
\end{tabular}
\end{center}

In an alternative approach,
child documents can be compiled by a specific command line
without additional code or specific definitions:
%
\begin{center}
|... -jobname "|\textit{target}|" "|[\textit{flags}]%
|\includeonly{|\textit{dest}|}\input{|\textit{main}|}"|
\end{center}
%

%%%%%%%%%%%%%%%%%%%%%%%%%%%%%%%%%%%%%%%%%%%%%%%%%%%%%%%%%%%%%%%%%%%%%%%%%%%%%%%%
%%%%%%%%%%%%%%%%%%%%%%%%%%%%%%%%%%%%%%%%%%%%%%%%%%%%%%%%%%%%%%%%%%%%%%%%%%%%%%%%
\section{Information}

%%%%%%%%%%%%%%%%%%%%%%%%%%%%%%%%%%%%%%%%%%%%%%%%%%%%%%%%%%%%%%%%%%%%%%%%%%%%%%%%
\subsection{Copyright}

Copyright \copyright{} 2017--2018 Niklas Beisert

This work may be distributed and/or modified under the
conditions of the \LaTeX{} Project Public License, either version 1.3
of this license or (at your option) any later version.
The latest version of this license is in
  \url{http://www.latex-project.org/lppl.txt}
and version 1.3 or later is part of all distributions of \LaTeX{}
version 2005/12/01 or later.

This work has the LPPL maintenance status `maintained'.

The Current Maintainer of this work is Niklas Beisert.

This work consists of the files |README.txt|, |childdoc.ins| and |childdoc.dtx|
as well as the derived files |childdoc.def|, |cdocsamp.tex|
with |cdocsch1.tex|, |cdocsch2.tex|, |cdocspt3.tex|, |cdocspt4.tex|,
|cdocsdrf.tex|, |cdocsfn1.tex|, |cdocsfn2.tex|
as well as |childdoc.pdf|.

%%%%%%%%%%%%%%%%%%%%%%%%%%%%%%%%%%%%%%%%%%%%%%%%%%%%%%%%%%%%%%%%%%%%%%%%%%%%%%%%
\subsection{Files and Installation}

The package consists of the files:
%
\begin{center}
\begin{tabular}{ll}
    |README.txt|   & readme file \\
    |childdoc.ins| & installation file \\
    |childdoc.dtx| & source file \\
    |childdoc.def| & definition file \\
    |cdocsamp.tex| & sample main file \\
    |cdocsch1.tex| & sample include file \\
    |cdocsch2.tex| & sample include file \\
    |cdocspt3.tex| & sample part file \\
    |cdocspt4.tex| & sample part file \\
    |cdocsdrf.tex| & sample redirection file \\
    |cdocsfn1.tex| & sample redirection file \\
    |cdocsfn2.tex| & sample redirection file \\
    |childdoc.pdf| & manual
\end{tabular}
\end{center}
%
The distribution consists of the files
|README.txt|, |childdoc.ins| and |childdoc.dtx|.
%
\begin{itemize}
\item
Run (pdf)\LaTeX{} on |childdoc.dtx|
to compile the manual |childdoc.pdf| (this file).
\item
Run \LaTeX{} on |childdoc.ins| to create the definitions file |childdoc.def|
and the sample |cdocsamp.tex| with include files
|cdocsch1.tex|, |cdocsch2.tex|, |cdocspt3.tex|, |cdocspt4.tex|,
|cdocsdrf.tex|, |cdocsfn1.tex|, |cdocsfn2.tex|.
Then copy the file |childdoc.def| to an appropriate directory of your \LaTeX{}
distribution, e.g.\ \textit{texmf-root}|/tex/latex/childdoc|.
\end{itemize}

%%%%%%%%%%%%%%%%%%%%%%%%%%%%%%%%%%%%%%%%%%%%%%%%%%%%%%%%%%%%%%%%%%%%%%%%%%%%%%%%
\subsection{Related CTAN Packages}

There are several other packages which offer a similar functionality:
%
\begin{itemize}
\item
The packages
\href{http://ctan.org/pkg/docmute}{\textsf{docmute}},
\href{http://ctan.org/pkg/includex}{\textsf{includex}} and
\href{http://ctan.org/pkg/standalone}{\textsf{standalone}}
provide commands to include only the document body of
a child file thus allowing both files to be compiled individually.
\item
The packages \href{http://ctan.org/pkg/subdocs}{\textsf{subdocs}}
and \href{http://ctan.org/pkg/subfiles}{\textsf{subfiles}}
provide structures in which the main and child documents can be
encapsulated and allowing them to be compiled individually.
The inclusion mechanism is different from the conventional |\include|.
\item
The package \href{http://ctan.org/pkg/combine}{\textsf{combine}}
is an elaborate solution to combine several documents into one.
\end{itemize}
%
See also the CTAN topic \href{http://ctan.org/topic/subdocs}{\textsf{subdocs}}
for further related packages.
The present package differs from the above solutions in that
a document structure constructed with the conventional |\include| mechanism
just needs two extra commands at the top of every file
such that all constituent files can be compiled individually.

%%%%%%%%%%%%%%%%%%%%%%%%%%%%%%%%%%%%%%%%%%%%%%%%%%%%%%%%%%%%%%%%%%%%%%%%%%%%%%%%
%\subsection{Feature Suggestions}
%
%The following is a list of features which may be useful for future
%versions of this package:
%%
%\begin{itemize}
%\item
%\ldots
%\end{itemize}

%%%%%%%%%%%%%%%%%%%%%%%%%%%%%%%%%%%%%%%%%%%%%%%%%%%%%%%%%%%%%%%%%%%%%%%%%%%%%%%%
\subsection{Revision History}

%%%%%%%%%%%%%%%%%%%%%%%%%%%%%%%%%%%%%%%%
\paragraph{v2.0:} 2018/12/30

\begin{itemize}
\item
immediate forward processing
\item
added |\childdocby| mechanism
\item
manual restructured
\end{itemize}

%%%%%%%%%%%%%%%%%%%%%%%%%%%%%%%%%%%%%%%%
\paragraph{v1.6:} 2018/01/17

\begin{itemize}
\item
application for development of include files
\item
corrections to manual
\end{itemize}

%%%%%%%%%%%%%%%%%%%%%%%%%%%%%%%%%%%%%%%%
\paragraph{v1.5:} 2017/05/21

\begin{itemize}
\item
more complete structuring introduced
\item
|\childdocof| introduced
\item
|\childdoc| renamed to |\childdocmain|
\item
|\childredirect| renamed to |\childdocforward| and |\childdocforwardprefix|
and functionality expanded
\end{itemize}

%%%%%%%%%%%%%%%%%%%%%%%%%%%%%%%%%%%%%%%%
\paragraph{v1.0:} 2017/04/27

\begin{itemize}
\item
manual and install package
\item
first version published on CTAN
\end{itemize}

%%%%%%%%%%%%%%%%%%%%%%%%%%%%%%%%%%%%%%%%
\paragraph{v0.6:} 2017/04/26

\begin{itemize}
\item
redirection mechanism added
\end{itemize}

%%%%%%%%%%%%%%%%%%%%%%%%%%%%%%%%%%%%%%%%
\paragraph{v0.5:} 2017/04/26

\begin{itemize}
\item
functionality in definition file
\end{itemize}


%%%%%%%%%%%%%%%%%%%%%%%%%%%%%%%%%%%%%%%%%%%%%%%%%%%%%%%%%%%%%%%%%%%%%%%%%%%%%%%%
%%%%%%%%%%%%%%%%%%%%%%%%%%%%%%%%%%%%%%%%%%%%%%%%%%%%%%%%%%%%%%%%%%%%%%%%%%%%%%%%
%%%%%%%%%%%%%%%%%%%%%%%%%%%%%%%%%%%%%%%%%%%%%%%%%%%%%%%%%%%%%%%%%%%%%%%%%%%%%%%%
\appendix

\settowidth\MacroIndent{\rmfamily\scriptsize 000\ }

 \DocInput{childdoc.dtx}

\end{document}
%</driver>
% \fi
%
% %%%%%%%%%%%%%%%%%%%%%%%%%%%%%%%%%%%%%%%%%%%%%%%%%%%%%%%%%%%%%%%%%%%%%%%%%%%%%%
% %%%%%%%%%%%%%%%%%%%%%%%%%%%%%%%%%%%%%%%%%%%%%%%%%%%%%%%%%%%%%%%%%%%%%%%%%%%%%%
% \section{Sample}
%\iffalse
%<*samplemain>
%\fi
%
% The following presents a sample document
% with two chapters, two parts, a title page,
% a compile flag as well as three forwarding files to set the flag.
% It consists of eight |.tex| files:
% \begin{center}
% \begin{tabular}{ll}
% |cdocsamp.tex|&main file\\
% |cdocsch1.tex|&include file for chapter 1\\
% |cdocsch2.tex|&include file for chapter 2\\
% |cdocspt3.tex|&include file for part 3\\
% |cdocspt4.tex|&include file for part 4\\
% |cdocsdrf.tex|&forwarding file for main file in draft mode\\
% |cdocsfi1.tex|&forwarding file for final version of chapter 1\\
% |cdocsfi2.tex|&forwarding file for final version of chapter 2\\
% \end{tabular}
% \end{center}
% Each of the eight files can be compiled directly by the \LaTeX{} compiler.
%
% %%%%%%%%%%%%%%%%%%%%%%%%%%%%%%%%%%%%%%
% \paragraph{Main File.}
%
% The main file is called |cdocsamp.tex|.
%
% Load the \textsf{childdoc} definitions and
% declare the filename for the main document:
%    \begin{macrocode}
\input{childdoc.def}
\childdocmain{}
%    \end{macrocode}

% Optional override for |\version| flag:
%    \begin{macrocode}
%%\ifchilddoc\else\providecommand{\version}{draft}\fi
%    \end{macrocode}

% Define the default values for the |\version| flag
% (|final| for the main file and |draft| for childs):
%    \begin{macrocode}
\ifchilddoc
\providecommand{\version}{draft}
\else
\providecommand{\version}{final}
\fi
%    \end{macrocode}

% Load the standard document class:
%    \begin{macrocode}
\documentclass[12pt]{article}
%    \end{macrocode}

% Start the document body:
%    \begin{macrocode}
\begin{document}
%    \end{macrocode}

% Declare a title page.
% Print title, part of document being processed and version flag:
%    \begin{macrocode}
\addtocounter{page}{-1}
\begin{center}
{\LARGE\bfseries{}childdoc example\par}
\vspace{1cm}
\ifchilddoc
\ifchilddocmanual part\else chapter\fi:
`\childdocname' of `\childdocjob'\par
\else
main document: `\childdocjob'\par
\fi
version: \version\par
\end{center}
\newpage
%    \end{macrocode}

% Manually include selected file,
% otherwise process as usual:
%    \begin{macrocode}
\ifchilddocmanual
\section*{part `\childdocname'}
\input{\childdocname}
\else
%    \end{macrocode}

% Include the two chapters:
%    \begin{macrocode}
\include{cdocsch1}
\include{cdocsch2}
%    \end{macrocode}

% Include the two parts unless only chapters should be displayed:
%    \begin{macrocode}
\ifchilddoc\else
\section{part three}
\input{cdocspt3}
\section{part four}
\input{cdocspt4}
\fi
%    \end{macrocode}

% Process as usual until here:
%    \begin{macrocode}
\fi
%    \end{macrocode}

% End of document body:
%    \begin{macrocode}
\end{document}
%    \end{macrocode}
%\iffalse
%</samplemain>
%\fi
%
% %%%%%%%%%%%%%%%%%%%%%%%%%%%%%%%%%%%%%%
% \paragraph{Chapter Include Files.}
%
% The include files are called |cdocsch1.tex| and |cdocsch2.tex|.
%
%\iffalse
%<*samplechap1|samplechap2>
%\fi

% Optional override for |\version| flag:
%    \begin{macrocode}
%%\providecommand{\version}{final}
%    \end{macrocode}

% Include the main document:
%    \begin{macrocode}
\input{childdoc.def}
\childdocof{cdocsamp}
%    \end{macrocode}

%\iffalse
%</samplechap1|samplechap2>
%\fi
%
%\iffalse
%<*samplechap1>
%\fi
% Some text for chapter 1:
%    \begin{macrocode}
\section{one}
some text in chapter one
%    \end{macrocode}

%\iffalse
%</samplechap1>
%\fi
% Some text for chapter 2:
%\iffalse
%<*samplechap2>
%\fi
%    \begin{macrocode}
\section{two}
more text in chapter two
%    \end{macrocode}

%\iffalse
%</samplechap2>
%\fi
%
% %%%%%%%%%%%%%%%%%%%%%%%%%%%%%%%%%%%%%%
% \paragraph{Part Include Files.}
%
% The include files are called |cdocspt3.tex| and |cdocspt4.tex|.
%
%\iffalse
%<*samplepart3|samplepart4>
%\fi

% Optional override for |\version| flag:
%    \begin{macrocode}
%%\providecommand{\version}{final}
%    \end{macrocode}

% Include the main document:
%    \begin{macrocode}
\input{childdoc.def}
\childdocby{cdocsamp}
%    \end{macrocode}

%\iffalse
%</samplepart3|samplepart4>
%\fi
%
%\iffalse
%<*samplepart3>
%\fi
% Some text for part 3:
%    \begin{macrocode}
some text in part three
%    \end{macrocode}

%\iffalse
%</samplepart3>
%\fi
% Some text for part 4:
%\iffalse
%<*samplepart4>
%\fi
%    \begin{macrocode}
more text in part four
%    \end{macrocode}

%\iffalse
%</samplepart4>
%\fi
%
% %%%%%%%%%%%%%%%%%%%%%%%%%%%%%%%%%%%%%%
% \paragraph{Forwarding for a Complete Draft.}
%
% The following forwarding file |cdocsdrf.tex|
% compiles the main document in draft mode:
%\iffalse
%<*sampledraft>
%\fi
%    \begin{macrocode}
\def\version{draft}
\input{childdoc.def}
\childdocforward{cdocsamp}
%    \end{macrocode}

%\iffalse
%</sampledraft>
%\fi
%
% %%%%%%%%%%%%%%%%%%%%%%%%%%%%%%%%%%%%%%
% \paragraph{Forwarding for Final Version of the Chapters.}
%
% The following forwarding files |cdocsfn1.tex| and |cdocsfn2.tex|
% (with identical content)
% compile the final versions of the child documents
% |cdocsch1.tex| and |cdocsch2.tex|, respectively:
%\iffalse
%<*samplefinal>
%\fi
%    \begin{macrocode}
\def\version{final}
\input{childdoc.def}
\childdocforwardprefix[cdocsamp]{cdocsfn}{cdocsch}
%    \end{macrocode}

%\iffalse
%</samplefinal>
%\fi
%
% %%%%%%%%%%%%%%%%%%%%%%%%%%%%%%%%%%%%%%
% \paragraph{Command Line Processing.}
%
% The following three command lines generate the output files
% |cdocscld|, |cdocscl1| and |cdocscl2|
% which should be identical to
% |cdocsdrf|, |cdocsch1| and |cdocsfn2|, respectively:
% \begin{center}
% \begin{tabular}{l}
% |latex -jobname cdocscld \|\\
% |  "\def\version{draft}\input{childdoc.def}\childdocforward{cdocsamp}"|\\
% |latex -jobname cdocscl1 \|\\
% |  "\input{childdoc.def}\childdocforward[cdocsamp]{cdocsch1}"|\\
% |latex -jobname cdocscl2 \|\\
% |  "\def\version{final}\input{childdoc.def}\childdocforward{cdocsch2}"|
% \end{tabular}
% \end{center}
% Note that the trailing backslash on each first line
% merely continues the input to the second line
% (for convenient cut ant paste).
% Furthermore, the command |latex| can be replaced by any
% of its alternative versions such as |pdflatex|.
%
% %%%%%%%%%%%%%%%%%%%%%%%%%%%%%%%%%%%%%%%%%%%%%%%%%%%%%%%%%%%%%%%%%%%%%%%%%%%%%%
% %%%%%%%%%%%%%%%%%%%%%%%%%%%%%%%%%%%%%%%%%%%%%%%%%%%%%%%%%%%%%%%%%%%%%%%%%%%%%%
% \section{Implementation}
%\iffalse
%<*package>
%\fi
%
% This section describes the definitions file |childdoc.def|.

% The definitions cannot be loaded using |\usepackage| or |\RequirePackage|
% which has a mechanism to prevent loading a style file more than once.
% When loading the definitions by means of |\input|
% multiple instances have to be prevented manually:
%\iffalse
%This code needs to be before the `\ProvidesFile' directive
%which is defined at the beginning of this file.
%Therefore it is also placed there and commented out here.
%</package>
%<*discard>
%\fi
%    \begin{macrocode}
\ifdefined\childdocmain\endinput\fi
%    \end{macrocode}
%\iffalse
%</discard>
%<*package>
%\fi
%
% \macro{\ifchilddoc}
% \macro{\ifchilddocmanual}
% The conditional |\ifchilddoc| tells whether a
% child (true) or main (false) document is being compiled.
% The conditional |\ifchilddocmanual| tells whether
% the |\includeonly| mechanism is used (false) or
% the selection of child files must be performed manually (true).
% The definitions initialise to false:
%    \begin{macrocode}
\newif\ifchilddoc
\newif\ifchilddocmanual
%    \end{macrocode}

% \macro{\childdocname}
% \macro{\childdocjob}
% The macro |\childdocname| stores the name of the main document
% to be compiled. The macro |\childdocjob| stores the name of
% the document on which the \LaTeX{} compiler was originally invoked.
% The content of |\jobname| cannot be compared
% to filenames specified in the source due to different catcodes.
% The following code rescans |\jobname|, stores the result
% in |\childdocname| and saves a copy in |\childdocjob|:
%    \begin{macrocode}
\edef\childdocname{\scantokens\expandafter{\jobname\noexpand}}
\let\childdocjob\childdocname
%    \end{macrocode}

% \macro{\childdocdisable}
% The macro |\childdocdisable| prevents the main file
% from being processed more than once.
% At this stage, the main document command |\childdocmain|
% is assumed to be called once again where it should do nothing.
% Any subsequent call to it should prevent
% a secondary processing of the main document
% It overwrites the forwarding commands
% |\childdocof| and |\childdocforward|
% with empty macros to prevent further inclusions of the main document:
%    \begin{macrocode}
\newcommand{\childdocdisable}
{
  \renewcommand{\childdocmain}[1]{\renewcommand{\childdocmain}[1]{\endinput}}
  \renewcommand{\childdocof}[1]{}
  \renewcommand{\childdocby}[2][]{}
  \renewcommand{\childdocforward}[2][]{}
  \renewcommand{\childdocdisable}{}
}
%    \end{macrocode}

% \macro{\childdocmain}
% The macro |\childdocmain| is to be called at the top of the main file
% with nothing or the main filename (without extension) as argument.
% First, it breaks loops.
% If the argument is not empty and does not match |\childdocname|
% (which is set by the first inclusion of |childdoc.def|),
% |\ifchilddoc| is set to true, |\includeonly| is applied to the child file
% and |\jobname| is set to the main file
% (for proper handling of |.aux| files):
%    \begin{macrocode}
\newcommand{\childdocmain}[1]
{
  \childdocdisable\childdocmain{}
  \if?#1?\else
    \begingroup
      \def\childdoctmp{#1}
      \ifx\childdoctmp\childdocname
        \def\childdoctmp{}
      \else
        \def\childdoctmp
        {
          \childdoctrue
          \includeonly{\childdocname}
          \def\childdocjob{#1}
          \def\jobname{#1}
        }
      \fi
      \expandafter
    \endgroup
    \childdoctmp
  \fi
}
%    \end{macrocode}

% \macro{\childdocof}
% The command |\childdocof| redirects
% compilation to the main file |#1|.
%    \begin{macrocode}
\newcommand{\childdocof}[1]
{
  \childdocdisable
  \childdoctrue
  \includeonly{\childdocname}
  \def\jobname{#1}
  \def\childdocjob{#1}
  \input{#1}
}
%    \end{macrocode}

% \macro{\childdocby}
% The command |\childdocby| ....
%    \begin{macrocode}
\newcommand{\childdocby}[2][]
{
  \childdocdisable
  \childdoctrue
  \childdocmanualtrue
  \if?#1?\else
    \def\jobname{#2}
  \fi
  \def\childdocjob{#2}
  \input{#2}
  \endinput
}
%    \end{macrocode}

% \macro{\childdocforward}
% The command |\childdocforward| redirects
% compilation to the main file or
% (if the optional argument is given) a child file.
% Parameters are set as if the main file
% or a child file starting with |\childdocof| was compiled.
% Then compilation is handed over to the main file:
%    \begin{macrocode}
\newcommand{\childdocforward}[2][]
{
  \begingroup
    \if?#1?
      \def\childdoctmp
      {
        \def\childdocname{#2}
        \def\childdocjob{#2}
        \def\jobname{#2}
        \input{#2}
        \endinput
      }
    \else
      \def\childdoctmp
      {
        \childdocdisable
        \def\childdocname{#2}
        \childdoctrue
        \includeonly{#2}
        \def\childdocjob{#1}
        \def\jobname{#1}
        \input{#1}
        \endinput
      }
    \fi
    \expandafter
  \endgroup
  \childdoctmp
}
%    \end{macrocode}

% \macro{\childdocforwardprefix}
% The command |\childdocforwardprefix| redirects
% compilation to the main or a child file by means of a pattern.
% The prefix |#1| in the current filename is replaced by |#2|
% and the suffix of the current filename is kept
% (it is assumed that the filename does not contain the substring `|~~~|'
% which is used as a delimiter).
% Compilation is handed over to the new file by |\childdocforward|:
%    \begin{macrocode}
\newcommand{\childdocforwardprefix}[3][]
{
  \begingroup
    \def\childdocextract #2##1~~~{\def\childdoctmp{\childdocforward[#1]{#3##1}}}
    \expandafter\childdocextract\childdocname~~~
    \expandafter
  \endgroup
  \childdoctmp
}
%    \end{macrocode}

% \macro{\childdoc}
% The deprecated macro |\childdoc| is a legacy version of |\childdocmain|:
%    \begin{macrocode}
\newcommand{\childdoc}{\childdocmain}
%    \end{macrocode}

% \macro{\childdocredirect}
% The deprecated macro |\childdocredirect| is a legacy version
% of |\childdocforward| and |\childdocforwardprefix|:
%    \begin{macrocode}
\newcommand{\childdocredirect}[2][]
{
  \begingroup
    \if?#1?
      \def\childdoctmp{\childdocforward{#2}}
    \else
      \def\childdoctmp{\childdocforwardprefix{#1}{#2}}
    \fi
    \expandafter
  \endgroup
  \childdoctmp
}
%    \end{macrocode}

%\iffalse
%</package>
%\fi
%
\endinput
|\\
|\childdocforwardprefix{final}{child}|
\end{tabular}
\end{center}
%

Note that when several versions of a main file and/or of each child file
are to be generated, it may be convenient to set up a |Makefile| or
shell script to automatise the process.

%%%%%%%%%%%%%%%%%%%%%%%%%%%%%%%%%%%%%%%%%%%%%%%%%%%%%%%%%%%%%%%%%%%%%%%%%%%%%%%%
\subsection{Command Line Processing}
\label{sec:commandline}

The effect of redirection files can also be achieved by invoking
the \LaTeX{} compiler with a more elaborate command line.
Most conveniently this should be done as part
of a shell script or a |Makefile|.

When using \textsf{childdoc} in the main file, the following
command lines effectively perform a redirection
(note that depending on the shell being used,
backslashes may have to be doubled: `|\|' $\to$ `|\\|'):
%
\begin{center}
|... -jobname "|\textit{target}|" |\\|"|[\textit{flags}]%
|% \iffalse
%
% childdoc.dtx Copyright (C) 2017-2018 Niklas Beisert
%
% This work may be distributed and/or modified under the
% conditions of the LaTeX Project Public License, either version 1.3
% of this license or (at your option) any later version.
% The latest version of this license is in
%   http://www.latex-project.org/lppl.txt
% and version 1.3 or later is part of all distributions of LaTeX
% version 2005/12/01 or later.
%
% This work has the LPPL maintenance status `maintained'.
%
% The Current Maintainer of this work is Niklas Beisert.
%
% This work consists of the files childdoc.dtx and childdoc.ins
% and the derived files childdoc.def and cdocsamp.tex with
% cdocsch1.tex, cdocsch2.tex, cdocsdrf.tex, cdocsfn1.tex, cdocsfn2.tex.
%
%<package>\ifdefined\childdocmain\endinput\fi
%<package>\ProvidesFile{childdoc.def}[2018/12/30 v2.0 child document driver]
%<samplemain>\ProvidesFile{cdocsamp.tex}[2018/12/30 v2.0 sample for childdoc]
%<*driver>
%\ProvidesFile{childdoc.drv}[2018/12/30 v2.0 childdoc reference manual file]
\PassOptionsToClass{10pt,a4paper}{article}
\documentclass{ltxdoc}

\usepackage[margin=35mm]{geometry}
\usepackage{hyperref}
\usepackage{hyperxmp}
\usepackage[usenames]{color}

\hypersetup{colorlinks=true}
\hypersetup{pdfstartview=FitH}
\hypersetup{pdfpagemode=UseNone}
\hypersetup{pdfsource={}}
\hypersetup{pdflang={en-UK}}
\hypersetup{pdfcopyright={Copyright 2017-2018 Niklas Beisert.
  This work may be distributed and/or modified under the
  conditions of the LaTeX Project Public License, either version 1.3
  of this license or (at your option) any later version.}}
\hypersetup{pdflicenseurl={http://www.latex-project.org/lppl.txt}}
\hypersetup{pdfcontactaddress={ETH Zurich, ITP, HIT K,
  Wolfgang-Pauli-Strasse 27}}
\hypersetup{pdfcontactpostcode={8093}}
\hypersetup{pdfcontactcity={Zurich}}
\hypersetup{pdfcontactcountry={Switzerland}}
\hypersetup{pdfcontactemail={nbeisert@itp.phys.ethz.ch}}
\hypersetup{pdfcontacturl={http://people.phys.ethz.ch/\xmptilde nbeisert/}}

\newcommand{\secref}[1]{\hyperref[#1]{section \ref*{#1}}}

\parskip1ex
\parindent0pt
\let\olditemize\itemize
\def\itemize{\olditemize\parskip0pt}

\begin{document}

\title{The \textsf{childdoc} Package}
\hypersetup{pdftitle={The childdoc Package}}
\author{Niklas Beisert\\[2ex]
  Institut f\"ur Theoretische Physik\\
  Eidgen\"ossische Technische Hochschule Z\"urich\\
  Wolfgang-Pauli-Strasse 27, 8093 Z\"urich, Switzerland\\[1ex]
  \href{mailto:nbeisert@itp.phys.ethz.ch}
  {\texttt{nbeisert@itp.phys.ethz.ch}}}
\hypersetup{pdfauthor={Niklas Beisert}}
\hypersetup{pdfsubject={Manual for the LaTeX2e Package childdoc}}
\date{30 December 2018, \textsf{v2.0}}
\maketitle

\begin{abstract}\noindent
\textsf{childdoc} is a \LaTeXe{} package
that enables the direct compilation
of document sections included by |\include|
to individual files.
\end{abstract}

\begingroup
\parskip0ex
\tableofcontents
\endgroup

%%%%%%%%%%%%%%%%%%%%%%%%%%%%%%%%%%%%%%%%%%%%%%%%%%%%%%%%%%%%%%%%%%%%%%%%%%%%%%%%
%%%%%%%%%%%%%%%%%%%%%%%%%%%%%%%%%%%%%%%%%%%%%%%%%%%%%%%%%%%%%%%%%%%%%%%%%%%%%%%%
\section{Introduction}

\LaTeX{} provides a mechanism to structure a large document (such as a book)
into a main file and several child files (containing the chapters)
using the |\include| command.
This mechanism is beneficial for documents
which span hundreds of pages in order to
make the source file(s) more manageable.
Moreover, compilation can be restricted to
selected child files by means of the |\includeonly| command.
The latter feature can be used to reduce the compilation time while editing
(this was significantly more useful in the earlier days of \LaTeX{})
or to generate a smaller document which is easier to navigate.
Another application of |\includeonly| is to generate
documents consisting of selected parts of the complete document.

However, there are a few drawbacks of the plain |\include| mechanism:
\begin{itemize}
\item
The child files cannot be compiled on their own,
they can only be compiled via the main file.
A naive editing environment
(such as a text editor with an option
to have the current file processed by \LaTeX)
may require one to switch to the main file before compiling;
attempting to compile the child file produces errors.
\item
The main file must be modified (each time)
to adjust the |\includeonly| command
to the present needs. This easily leaves the main file in a messy state.
\item
The generated document will always carry the filename
of the main document. This is inconvenient if
several child files are to be compiled and
to be kept for distribution.
\end{itemize}

The present package provides a simple interface
to make child files individually compilable by \LaTeX{}.
Compiling a child file then has the same effect as compiling
the main file with an |\includeonly| command
to select the appropriate child.
Moreover the generated document will carry the name of the child
rather than the main file.
This resolves all three above issues.

This feature is meant to make the editing of books,
thesis documents and lecture notes somewhat more convenient.
However, the package can also be used efficiently for
composing a series of documents (such as exercise sheets)
which are typically distributed individually.
It then assists the author in generating the individual documents
(potentially in different versions)
as well as a document containing the collected series.
Another application is in developing style files
or other kinds of included material
where compilation of the style file could redirect
to a sample or test file.

%%%%%%%%%%%%%%%%%%%%%%%%%%%%%%%%%%%%%%%%%%%%%%%%%%%%%%%%%%%%%%%%%%%%%%%%%%%%%%%%
%%%%%%%%%%%%%%%%%%%%%%%%%%%%%%%%%%%%%%%%%%%%%%%%%%%%%%%%%%%%%%%%%%%%%%%%%%%%%%%%
\section{Usage}

First of all, the package \textsf{childdoc} is \emph{not} a standard
\LaTeXe{} |.sty| style file! Therefore it needs to be invoked in
a non-standard way.

%%%%%%%%%%%%%%%%%%%%%%%%%%%%%%%%%%%%%%%%%%%%%%%%%%%%%%%%%%%%%%%%%%%%%%%%%%%%%%%%
\subsection{Included Files}
\label{sec:include}

%%%%%%%%%%%%%%%%%%%%%%%%%%%%%%%%%%%%%%%%
\DescribeMacro{\childdocmain}
To use the package, add the commands
\begin{center}
\begin{tabular}{l}
|\input{childdoc.def}|\\
|\childdocmain{}|\\
\end{tabular}
\end{center}
at the very top of the main \LaTeX{} file,
in particular \emph{before} the |\documentclass| statement!
The argument of |\childdocmain| should be left empty
(but it must be present).

%%%%%%%%%%%%%%%%%%%%%%%%%%%%%%%%%%%%%%%%
\DescribeMacro{\childdocof}
Furthermore, add the commands
\begin{center}
\begin{tabular}{l}
|\input{childdoc.def}|\\
|\childdocof{|\textit{main}|}|\\
\end{tabular}
\end{center}
at the top of every child file \textit{child}
which is included by |\include{|\textit{child}|}|
from within the main file
(or at least for those files to be compiled individually).
The argument \textit{main} must be the filename of the main file.

There are a couple of
considerations in setting up the main and child documents:

%%%%%%%%%%%%%%%%%%%%%%%%%%%%%%%%%%%%%%%%
\paragraph{Restrictions.}

Please note the following restrictions:
\begin{itemize}
\item
|\childdocmain| must be called with one argument \textit{main}
to ensure compatibility with earlier version of the package.
It must either be empty (|\childdocmain{}|)
or precisely match the filename of the main file in which it is specified.
See \secref{sec:detection} for further information.
\item
The filename \textit{main} must be specified without the |.tex| extension.
\item
The filename \textit{main} is case sensitive
(even in case-insensitive file systems)
due to internal string comparison.
\item
The argument \textit{main} should be fully expanded, it cannot be a macro.
\item
Subdirectories and special characters should be avoided in filenames.
\item
The command |\childdocmain{|\textit{main}|}| must be followed by a whitespace.
It should not be followed immediately by another command
or by a comment mark `|%|'.
This is because the \TeX{} parser reads the token immediately following
the argument of |\childdocmain| and puts it
at the beginning of every child section;
however, a white\-space is ignored.
\end{itemize}

%%%%%%%%%%%%%%%%%%%%%%%%%%%%%%%%%%%%%%%%
\paragraph{Content of Main File.}

It is advisable to place all content in the child files included by |\include|.
Any output contained in the main file will appear in all child documents
unless suppressed manually;
it cannot be suppressed automatically by the |\includeonly| directive
and thus should normally be avoided.
A method to include some content in the main file
by means of conditional processing is described in \secref{sec:conditional}.

%%%%%%%%%%%%%%%%%%%%%%%%%%%%%%%%%%%%%%%%
\paragraph{Page Numbering.}

When only a part of the document is compiled,
the appropriate numbering of pages
(as well as other status parameters)
is determined from the |.aux| files.
The latter contain information from previous passes.
However this information needs to propagate through
all intermediate child documents.
Therefore the page numbering in child documents may well
be inconsistent until the complete document is compiled at least once.

A useful (if unconventional) way to always ensure a consistent
page numbering is to restart the numbering in each child document
and denote the pages by `\textit{child}|.|\textit{page}'
where \textit{child} represents the chapter/section number of the child file.
This can be achieved by the command
|\numberwithin{page}{|\textit{child}|}|
of the \textsf{amsmath} package
where \textit{child} can be |chapter| or |section|
depending on the chosen structuring.
Alternatively, one can modify the macro |\thepage| appropriately
and reset the counter |page| at the start of each child file.

%%%%%%%%%%%%%%%%%%%%%%%%%%%%%%%%%%%%%%%%%%%%%%%%%%%%%%%%%%%%%%%%%%%%%%%%%%%%%%%%
\subsection{Conditional Processing}
\label{sec:conditional}

The package provides a mechanism to compile different versions
of a document. To customise the versions further some conditional processing
can come in handy to distinguish which version is being compiled.
The package provides two macros to describe the compilation context:

%%%%%%%%%%%%%%%%%%%%%%%%%%%%%%%%%%%%%%%%
\DescribeMacro{\ifchilddoc}
The conditional |\ifchilddoc| distinguishes between the compilation of
child documents and the main document:
%
\begin{center}
|\ifchilddoc |\textit{child-code}| |[|\||else |\textit{main-code}]| \||fi|
\end{center}

%%%%%%%%%%%%%%%%%%%%%%%%%%%%%%%%%%%%%%%%
\DescribeMacro{\childdocname}
\DescribeMacro{\childdocjob}
The macro |\childdocname| contains the filename (without extension)
of the main or child file being processed.
Note that |\childdocjob| will always contain the name of the main file.

%%%%%%%%%%%%%%%%%%%%%%%%%%%%%%%%%%%%%%%%
\paragraph{Title Page.}

Conditional processing can be used to include a title or banner page
in the main document when proper precautions are taken.
Importantly, the code in the main file should ensure that the page counter
(as well as other status parameters which are stored in the |.aux| files)
takes the same value after the conditional processing.
Otherwise the page numbers may take divergent values
depending on which part is compiled.

For example, a title page could be declared by:
%
\begin{center}
\begin{tabular}{l}
|\ifchilddoc\||else|\\
|\addtocounter{page}{-1}|\\
\textit{code for title page}\\
|\newpage|\\
|\||fi|
\end{tabular}
\end{center}
%
A banner page for the child documents can be generated by:
%
\begin{center}
\begin{tabular}{l}
|\ifchilddoc|\\
|\addtocounter{page}{-1}|\\
\textit{code for banner page}\\
|\newpage|\\
|\||fi|
\end{tabular}
\end{center}
%
Here one could write a message such as:
\begin{center}
|This is the part \childdocname{} of \childdocjob{}.|
\end{center}

%%%%%%%%%%%%%%%%%%%%%%%%%%%%%%%%%%%%%%%%%%%%%%%%%%%%%%%%%%%%%%%%%%%%%%%%%%%%%%%%
\subsection{Flags}
\label{sec:flags}

The package makes it easy to generate different versions
of the main or child documents.
To this end compilation flags can be defined
and assigned different default values.
They will be particularly useful in conjunction
with the forwarding mechanism described in \secref{sec:forward}.

For example, it may be useful to have a flag |\version|
which can be set to |draft| or |final|.
The document source will contain some conditional code
depending on the value of |\version|.
Suppose further, the flag should default to |final| for the main file
and to |draft| for child files
which is a natural assignment for editing the document.
This is achieved by placing the following code
in the preamble of the main document
(below the |\childdocmain| directive):
%
\begin{center}
\begin{tabular}{l}
|\ifchilddoc|\\
|\providecommand{\version}{draft}|\\
|\||else|\\
|\providecommand{\version}{final}|\\
|\||fi|
\end{tabular}
\end{center}
%
The definition by |\providecommand| makes sure
that previous definitions are not overwritten.
Further statements |\providecommand{\version}{...}|
can thus be added before the above code to override it.

For the main file, one might add a line
(between |\childdocmain| and the above block)
%
\begin{center}
|%\ifchilddoc\||else\providecommand{\version}{draft}\||fi|
\end{center}
%
which can be uncommented to produce a draft version.
Likewise one can add a line to the very top of a child file
(above the |\childdocof{|\textit{main}|}| directive)
%
\begin{center}
|%\providecommand{\version}{final}|
\end{center}
%
which can be uncommented to produce the final version of this child document.

%%%%%%%%%%%%%%%%%%%%%%%%%%%%%%%%%%%%%%%%%%%%%%%%%%%%%%%%%%%%%%%%%%%%%%%%%%%%%%%%
\subsection{Forwarding}
\label{sec:forward}

Different versions of the main or child documents
using compilation flags as described in \secref{sec:flags}
can be (permanently) stored in different files
for convenient compilation, viewing and distribution.
To this end, the package defines a command
to pass on compilation to a different file:

%%%%%%%%%%%%%%%%%%%%%%%%%%%%%%%%%%%%%%%%
\DescribeMacro{\childdocforward}
The command |\childdocforward| redirects processing to
another source file:
%
\begin{center}
\begin{tabular}{l}
|\input{childdoc.def}|\\
|\childdocforward[|\textit{main}|]{|\textit{dest}|}|\\
\end{tabular}
\end{center}
%
The argument \textit{dest} is the destination file
(without extension).
It should be the main file or one of the child files.
Note that further \textsf{childdoc} directives
such as |\childdocof| and |\childdocforward|
in the indicated file will be processed in this form.
The optional argument \textit{main}
passes on directly to the main file \textit{main}
while pretending to compile the child \textit{dest}.
This form behaves as if \textit{dest}
issues |\childdocof{|\textit{main}|}| right away,
and no further \textsf{childdoc} directives will be processed.

%%%%%%%%%%%%%%%%%%%%%%%%%%%%%%%%%%%%%%%%
\DescribeMacro{\...prefix}
In the alternative form |\childdocforwardprefix|,
%
\begin{center}
\begin{tabular}{l}
|\input{childdoc.def}|\\
|\childdocforwardprefix[|\textit{main}|]{|\textit{prefix}|}{|\textit{dest}|}|
\end{tabular}
\end{center}
%
the destination file is determined by a pattern
depending on the current file:
To make this work, the current file must be called
`{\textit{prefix}\hspace{0.2em}\textit{suffix}}'
with \textit{prefix} matching precisely the argument.
Processing is then passed on to the file
`{\textit{dest}\hspace{0.2em}\textit{suffix}}'.
Surely, the same effect is achieved by
directly specifying the
argument `{\textit{dest}\hspace{0.2em}\textit{suffix}}'
in the first form.
However, that requires to set up a different file
for each child. With the alternative form of the command
all these files can have exactly the same content
which simplifies setting them up and maintaining them.

For example, the following file |draft.tex|
with a compilation flag |\version| as described in \secref{sec:flags}
compiles the main document as a draft:
%
\begin{center}
\begin{tabular}{l}
|\def\version{draft}|\\
|\input{childdoc.def}|\\
|\childdocforward{|\textit{main}|}|
\end{tabular}
\end{center}
%
Likewise, the following files |final|\textit{nn}|.tex|
compile the final version of the child document
|child|\textit{nn}|.tex|:
%
\begin{center}
\begin{tabular}{l}
|\def\version{final}|\\
|\input{childdoc.def}|\\
|\childdocforwardprefix{final}{child}|
\end{tabular}
\end{center}
%

Note that when several versions of a main file and/or of each child file
are to be generated, it may be convenient to set up a |Makefile| or
shell script to automatise the process.

%%%%%%%%%%%%%%%%%%%%%%%%%%%%%%%%%%%%%%%%%%%%%%%%%%%%%%%%%%%%%%%%%%%%%%%%%%%%%%%%
\subsection{Command Line Processing}
\label{sec:commandline}

The effect of redirection files can also be achieved by invoking
the \LaTeX{} compiler with a more elaborate command line.
Most conveniently this should be done as part
of a shell script or a |Makefile|.

When using \textsf{childdoc} in the main file, the following
command lines effectively perform a redirection
(note that depending on the shell being used,
backslashes may have to be doubled: `|\|' $\to$ `|\\|'):
%
\begin{center}
|... -jobname "|\textit{target}|" |\\|"|[\textit{flags}]%
|\input{childdoc.def}\childdocforward[|\textit{main}|]{|\textit{dest}|}"|
\end{center}
%
Here \textit{target} is the name of the output file,
\textit{main} is the name of the main file
and \textit{dest} is the name of the main or child file to be processed
(all filenames without extensions).
The optional argument \textit{main} can be omitted
if \textit{main} matches \textit{dest}.
Optionally, compilation \textit{flags} can be defined via |\def| commands.
This command line makes the \TeX{} engine believe
it is compiling the file \textit{target}
whose content is specified as the latter parameter.
The provided code then forwards the processing to
\textit{main} or \textit{dest} as described in \secref{sec:forward}.

%%%%%%%%%%%%%%%%%%%%%%%%%%%%%%%%%%%%%%%%%%%%%%%%%%%%%%%%%%%%%%%%%%%%%%%%%%%%%%%%
\subsection{Include by Input}
\label{sec:input}

Including child documents by |\include| has some restrictions by design.
Most notably, the content of a child document always occupies
its own set of pages; pages cannot be shared between child documents.
Usually, this behaviour makes perfect sense
because each child document contain an essential part of the document.
However, in some situations it may be desirable to compose
a document from a collection of parts
without having mandatory page breaks between then.
For this case, the package
provides a mechanism to include parts
by |\input| which can also be processed individually.
However, by construction this mechanism
requires manual handling of the content to be output.

%%%%%%%%%%%%%%%%%%%%%%%%%%%%%%%%%%%%%%%%
\DescribeMacro{\ifchilddocmanual}
The main file should be prepared as usual, see \secref{sec:include}.
However, the document body must make a distinction
between processing of an individual part and of the main document, e.g.:
%
\begin{center}
\begin{tabular}{l}
|\ifchilddocmanual|\\
|\input{\childdocname}|\\
|\||else|\\
\textit{document body with }|\input{|\textit{part}|}|\\
|\||fi|
\end{tabular}
\end{center}
%
The conditional |\ifchilddocmanual| is true whenever
a part to be included by |\input| is being compiled,
and the name of the part is stored in |\childdocname|.

%%%%%%%%%%%%%%%%%%%%%%%%%%%%%%%%%%%%%%%%
\DescribeMacro{\childdocby}
Each part to be included by |\input| should start with:
%
\begin{center}
\begin{tabular}{l}
|\input{childdoc.def}|\\
|\childdocby{|\textit{main}|}|\\
\end{tabular}
\end{center}
%
The directive |\childdocby| is similar to |\childdocof|
described in \secref{sec:include},
but the subsequent selection of content must be done manually.
To that end, both |\ifchilddoc| and |\ifchilddocmanual|
will be true upon processing of a part,
and the name of the part is stored in |\childdocname|.
Note that |\jobname| will be set to the filename of the current part
so that each part receives an individual |.aux| file
that does not interfere with the |.aux| file(s) of the main document.
This behaviour can be altered by the alternative form
|\childdocby[*]{|\textit{main}|}| (with a non-empty optional argument)
which uses the |.aux| file of the main document
by setting |\jobname| to \textit{main}.

%%%%%%%%%%%%%%%%%%%%%%%%%%%%%%%%%%%%%%%%%%%%%%%%%%%%%%%%%%%%%%%%%%%%%%%%%%%%%%%%
\subsection{Driver Development}
\label{sec:driver}

The \textsf{childdoc} mechanism can also be use for the development
of definition files such as \LaTeX{} styles or classes.
This case differs from the above setup with multiple parts
included by |\include| in that no |\includeonly| should be invoked.
This can be achieved by starting the include file
(before |\ProvidesPackage|) with:
%
\begin{center}
\begin{tabular}{l}
|\input{childdoc.def}|\\
|\childdocforward{|\textit{main}|}|\\
\end{tabular}
\end{center}
%
or alternatively with:
%
\begin{center}
\begin{tabular}{l}
|\input{childdoc.def}|\\
|\childdocby{|\textit{main}|}|\\
\end{tabular}
\end{center}
%
Both forms have slightly different effects as described above.
The main file is prepared as usual, see \secref{sec:include}.

%%%%%%%%%%%%%%%%%%%%%%%%%%%%%%%%%%%%%%%%%%%%%%%%%%%%%%%%%%%%%%%%%%%%%%%%%%%%%%%%
\subsection{Legacy Detection}
\label{sec:detection}

The directive |\childdocmain| in the main file can detect
whether the complete document or merely a child is to be compiled
even without using the directive |\childdocof|.
This method is deprecated because it is less robust
and there is no compelling reason to use it;
it is merely provided for backward compatibility
and it may be removed in future versions.

If the detection mechanism is to be used,
it is mandatory to correctly specify
the filename of the main file as the argument of |\childdocmain|:
%
\begin{center}
\begin{tabular}{l}
|\input{childdoc.def}|\\
|\childdocmain{|\textit{main}|}|\\
\end{tabular}
\end{center}
%
If |\jobname| does not match the argument \textit{main} of |\childdocmain|,
it is assumed that |\jobname| points to the child file to be compiled.
When using |\childdocmain| with the main file specified as argument,
it suffices to start a child file
with just |\input{|\textit{main}|}|
without loading of the package and using |\childdocof|.
If instead all processing is done
with the appropriate \textsf{childdoc} directives,
the argument of \textit{main} of |\childdocmain| can be empty.

An alternative version of the command line processing described
in \secref{sec:commandline} using the detection mechanism reads:
%
\begin{center}
|... -jobname "|\textit{target}|" "|[\textit{flags}]%
[|\def\jobname{|\textit{dest}|}|]|\input{|\textit{main}|}"|
\end{center}

%%%%%%%%%%%%%%%%%%%%%%%%%%%%%%%%%%%%%%%%%%%%%%%%%%%%%%%%%%%%%%%%%%%%%%%%%%%%%%%%
\subsection{Manual Code}
\label{sec:manual}

In case one cannot be certain whether the definitions file |childdoc.def|
is installed on the target \TeX{} distribution
and one prefers not to ship it,
it is conceivable to paste a few relevant commands into the sources.

To that end, drop all statements |\input{childdoc.def}|
and perform the replacements as outlined below.
Instead of |\childdocmain{|\textit{main}|}| add the following code
to the top of the main file:
%
\begin{center}
\begin{tabular}{l}
|\||ifdefined\childdocname\endinput\||fi\newif\ifchilddoc|\\
|\edef\childdocname{\scantokens\expandafter{\jobname\noexpand}}|\\
|\def\childdocmain{|\textit{main}|}\||ifx\childdocmain\childdocname\||else|\\
|\childdoctrue\includeonly{\childdocname}\let\jobname\childdocmain\||fi|\\
\end{tabular}
\end{center}
%
Instead of |\childdocof{|\textit{main}|}| just include the main file
at the top of each child file:
%
\begin{center}
|\input{|\textit{main}|}|
\end{center}
%
A simple redirection |\childdocforward{|\textit{dest}|}| is achieved by:
%
\begin{center}
|\def\jobname{|\textit{dest}|}\input{\jobname}|
\end{center}
%
The redirection with prefix
|\childdocforwardprefix[|\textit{prefix}|]{|\textit{dest}|}|
is accomplished by:
%
\begin{center}
\begin{tabular}{l}
|{\edef\jobname{\scantokens\expandafter{\jobname\noexpand}}|\\
|\def\redirectjob |\textit{prefix}|#1~~~{\gdef\jobname{|\textit{dest}|#1}}|\\
|\expandafter\redirectjob\jobname~~~}\input{\jobname}|
\end{tabular}
\end{center}

In an alternative approach,
child documents can be compiled by a specific command line
without additional code or specific definitions:
%
\begin{center}
|... -jobname "|\textit{target}|" "|[\textit{flags}]%
|\includeonly{|\textit{dest}|}\input{|\textit{main}|}"|
\end{center}
%

%%%%%%%%%%%%%%%%%%%%%%%%%%%%%%%%%%%%%%%%%%%%%%%%%%%%%%%%%%%%%%%%%%%%%%%%%%%%%%%%
%%%%%%%%%%%%%%%%%%%%%%%%%%%%%%%%%%%%%%%%%%%%%%%%%%%%%%%%%%%%%%%%%%%%%%%%%%%%%%%%
\section{Information}

%%%%%%%%%%%%%%%%%%%%%%%%%%%%%%%%%%%%%%%%%%%%%%%%%%%%%%%%%%%%%%%%%%%%%%%%%%%%%%%%
\subsection{Copyright}

Copyright \copyright{} 2017--2018 Niklas Beisert

This work may be distributed and/or modified under the
conditions of the \LaTeX{} Project Public License, either version 1.3
of this license or (at your option) any later version.
The latest version of this license is in
  \url{http://www.latex-project.org/lppl.txt}
and version 1.3 or later is part of all distributions of \LaTeX{}
version 2005/12/01 or later.

This work has the LPPL maintenance status `maintained'.

The Current Maintainer of this work is Niklas Beisert.

This work consists of the files |README.txt|, |childdoc.ins| and |childdoc.dtx|
as well as the derived files |childdoc.def|, |cdocsamp.tex|
with |cdocsch1.tex|, |cdocsch2.tex|, |cdocspt3.tex|, |cdocspt4.tex|,
|cdocsdrf.tex|, |cdocsfn1.tex|, |cdocsfn2.tex|
as well as |childdoc.pdf|.

%%%%%%%%%%%%%%%%%%%%%%%%%%%%%%%%%%%%%%%%%%%%%%%%%%%%%%%%%%%%%%%%%%%%%%%%%%%%%%%%
\subsection{Files and Installation}

The package consists of the files:
%
\begin{center}
\begin{tabular}{ll}
    |README.txt|   & readme file \\
    |childdoc.ins| & installation file \\
    |childdoc.dtx| & source file \\
    |childdoc.def| & definition file \\
    |cdocsamp.tex| & sample main file \\
    |cdocsch1.tex| & sample include file \\
    |cdocsch2.tex| & sample include file \\
    |cdocspt3.tex| & sample part file \\
    |cdocspt4.tex| & sample part file \\
    |cdocsdrf.tex| & sample redirection file \\
    |cdocsfn1.tex| & sample redirection file \\
    |cdocsfn2.tex| & sample redirection file \\
    |childdoc.pdf| & manual
\end{tabular}
\end{center}
%
The distribution consists of the files
|README.txt|, |childdoc.ins| and |childdoc.dtx|.
%
\begin{itemize}
\item
Run (pdf)\LaTeX{} on |childdoc.dtx|
to compile the manual |childdoc.pdf| (this file).
\item
Run \LaTeX{} on |childdoc.ins| to create the definitions file |childdoc.def|
and the sample |cdocsamp.tex| with include files
|cdocsch1.tex|, |cdocsch2.tex|, |cdocspt3.tex|, |cdocspt4.tex|,
|cdocsdrf.tex|, |cdocsfn1.tex|, |cdocsfn2.tex|.
Then copy the file |childdoc.def| to an appropriate directory of your \LaTeX{}
distribution, e.g.\ \textit{texmf-root}|/tex/latex/childdoc|.
\end{itemize}

%%%%%%%%%%%%%%%%%%%%%%%%%%%%%%%%%%%%%%%%%%%%%%%%%%%%%%%%%%%%%%%%%%%%%%%%%%%%%%%%
\subsection{Related CTAN Packages}

There are several other packages which offer a similar functionality:
%
\begin{itemize}
\item
The packages
\href{http://ctan.org/pkg/docmute}{\textsf{docmute}},
\href{http://ctan.org/pkg/includex}{\textsf{includex}} and
\href{http://ctan.org/pkg/standalone}{\textsf{standalone}}
provide commands to include only the document body of
a child file thus allowing both files to be compiled individually.
\item
The packages \href{http://ctan.org/pkg/subdocs}{\textsf{subdocs}}
and \href{http://ctan.org/pkg/subfiles}{\textsf{subfiles}}
provide structures in which the main and child documents can be
encapsulated and allowing them to be compiled individually.
The inclusion mechanism is different from the conventional |\include|.
\item
The package \href{http://ctan.org/pkg/combine}{\textsf{combine}}
is an elaborate solution to combine several documents into one.
\end{itemize}
%
See also the CTAN topic \href{http://ctan.org/topic/subdocs}{\textsf{subdocs}}
for further related packages.
The present package differs from the above solutions in that
a document structure constructed with the conventional |\include| mechanism
just needs two extra commands at the top of every file
such that all constituent files can be compiled individually.

%%%%%%%%%%%%%%%%%%%%%%%%%%%%%%%%%%%%%%%%%%%%%%%%%%%%%%%%%%%%%%%%%%%%%%%%%%%%%%%%
%\subsection{Feature Suggestions}
%
%The following is a list of features which may be useful for future
%versions of this package:
%%
%\begin{itemize}
%\item
%\ldots
%\end{itemize}

%%%%%%%%%%%%%%%%%%%%%%%%%%%%%%%%%%%%%%%%%%%%%%%%%%%%%%%%%%%%%%%%%%%%%%%%%%%%%%%%
\subsection{Revision History}

%%%%%%%%%%%%%%%%%%%%%%%%%%%%%%%%%%%%%%%%
\paragraph{v2.0:} 2018/12/30

\begin{itemize}
\item
immediate forward processing
\item
added |\childdocby| mechanism
\item
manual restructured
\end{itemize}

%%%%%%%%%%%%%%%%%%%%%%%%%%%%%%%%%%%%%%%%
\paragraph{v1.6:} 2018/01/17

\begin{itemize}
\item
application for development of include files
\item
corrections to manual
\end{itemize}

%%%%%%%%%%%%%%%%%%%%%%%%%%%%%%%%%%%%%%%%
\paragraph{v1.5:} 2017/05/21

\begin{itemize}
\item
more complete structuring introduced
\item
|\childdocof| introduced
\item
|\childdoc| renamed to |\childdocmain|
\item
|\childredirect| renamed to |\childdocforward| and |\childdocforwardprefix|
and functionality expanded
\end{itemize}

%%%%%%%%%%%%%%%%%%%%%%%%%%%%%%%%%%%%%%%%
\paragraph{v1.0:} 2017/04/27

\begin{itemize}
\item
manual and install package
\item
first version published on CTAN
\end{itemize}

%%%%%%%%%%%%%%%%%%%%%%%%%%%%%%%%%%%%%%%%
\paragraph{v0.6:} 2017/04/26

\begin{itemize}
\item
redirection mechanism added
\end{itemize}

%%%%%%%%%%%%%%%%%%%%%%%%%%%%%%%%%%%%%%%%
\paragraph{v0.5:} 2017/04/26

\begin{itemize}
\item
functionality in definition file
\end{itemize}


%%%%%%%%%%%%%%%%%%%%%%%%%%%%%%%%%%%%%%%%%%%%%%%%%%%%%%%%%%%%%%%%%%%%%%%%%%%%%%%%
%%%%%%%%%%%%%%%%%%%%%%%%%%%%%%%%%%%%%%%%%%%%%%%%%%%%%%%%%%%%%%%%%%%%%%%%%%%%%%%%
%%%%%%%%%%%%%%%%%%%%%%%%%%%%%%%%%%%%%%%%%%%%%%%%%%%%%%%%%%%%%%%%%%%%%%%%%%%%%%%%
\appendix

\settowidth\MacroIndent{\rmfamily\scriptsize 000\ }

 \DocInput{childdoc.dtx}

\end{document}
%</driver>
% \fi
%
% %%%%%%%%%%%%%%%%%%%%%%%%%%%%%%%%%%%%%%%%%%%%%%%%%%%%%%%%%%%%%%%%%%%%%%%%%%%%%%
% %%%%%%%%%%%%%%%%%%%%%%%%%%%%%%%%%%%%%%%%%%%%%%%%%%%%%%%%%%%%%%%%%%%%%%%%%%%%%%
% \section{Sample}
%\iffalse
%<*samplemain>
%\fi
%
% The following presents a sample document
% with two chapters, two parts, a title page,
% a compile flag as well as three forwarding files to set the flag.
% It consists of eight |.tex| files:
% \begin{center}
% \begin{tabular}{ll}
% |cdocsamp.tex|&main file\\
% |cdocsch1.tex|&include file for chapter 1\\
% |cdocsch2.tex|&include file for chapter 2\\
% |cdocspt3.tex|&include file for part 3\\
% |cdocspt4.tex|&include file for part 4\\
% |cdocsdrf.tex|&forwarding file for main file in draft mode\\
% |cdocsfi1.tex|&forwarding file for final version of chapter 1\\
% |cdocsfi2.tex|&forwarding file for final version of chapter 2\\
% \end{tabular}
% \end{center}
% Each of the eight files can be compiled directly by the \LaTeX{} compiler.
%
% %%%%%%%%%%%%%%%%%%%%%%%%%%%%%%%%%%%%%%
% \paragraph{Main File.}
%
% The main file is called |cdocsamp.tex|.
%
% Load the \textsf{childdoc} definitions and
% declare the filename for the main document:
%    \begin{macrocode}
\input{childdoc.def}
\childdocmain{}
%    \end{macrocode}

% Optional override for |\version| flag:
%    \begin{macrocode}
%%\ifchilddoc\else\providecommand{\version}{draft}\fi
%    \end{macrocode}

% Define the default values for the |\version| flag
% (|final| for the main file and |draft| for childs):
%    \begin{macrocode}
\ifchilddoc
\providecommand{\version}{draft}
\else
\providecommand{\version}{final}
\fi
%    \end{macrocode}

% Load the standard document class:
%    \begin{macrocode}
\documentclass[12pt]{article}
%    \end{macrocode}

% Start the document body:
%    \begin{macrocode}
\begin{document}
%    \end{macrocode}

% Declare a title page.
% Print title, part of document being processed and version flag:
%    \begin{macrocode}
\addtocounter{page}{-1}
\begin{center}
{\LARGE\bfseries{}childdoc example\par}
\vspace{1cm}
\ifchilddoc
\ifchilddocmanual part\else chapter\fi:
`\childdocname' of `\childdocjob'\par
\else
main document: `\childdocjob'\par
\fi
version: \version\par
\end{center}
\newpage
%    \end{macrocode}

% Manually include selected file,
% otherwise process as usual:
%    \begin{macrocode}
\ifchilddocmanual
\section*{part `\childdocname'}
\input{\childdocname}
\else
%    \end{macrocode}

% Include the two chapters:
%    \begin{macrocode}
\include{cdocsch1}
\include{cdocsch2}
%    \end{macrocode}

% Include the two parts unless only chapters should be displayed:
%    \begin{macrocode}
\ifchilddoc\else
\section{part three}
\input{cdocspt3}
\section{part four}
\input{cdocspt4}
\fi
%    \end{macrocode}

% Process as usual until here:
%    \begin{macrocode}
\fi
%    \end{macrocode}

% End of document body:
%    \begin{macrocode}
\end{document}
%    \end{macrocode}
%\iffalse
%</samplemain>
%\fi
%
% %%%%%%%%%%%%%%%%%%%%%%%%%%%%%%%%%%%%%%
% \paragraph{Chapter Include Files.}
%
% The include files are called |cdocsch1.tex| and |cdocsch2.tex|.
%
%\iffalse
%<*samplechap1|samplechap2>
%\fi

% Optional override for |\version| flag:
%    \begin{macrocode}
%%\providecommand{\version}{final}
%    \end{macrocode}

% Include the main document:
%    \begin{macrocode}
\input{childdoc.def}
\childdocof{cdocsamp}
%    \end{macrocode}

%\iffalse
%</samplechap1|samplechap2>
%\fi
%
%\iffalse
%<*samplechap1>
%\fi
% Some text for chapter 1:
%    \begin{macrocode}
\section{one}
some text in chapter one
%    \end{macrocode}

%\iffalse
%</samplechap1>
%\fi
% Some text for chapter 2:
%\iffalse
%<*samplechap2>
%\fi
%    \begin{macrocode}
\section{two}
more text in chapter two
%    \end{macrocode}

%\iffalse
%</samplechap2>
%\fi
%
% %%%%%%%%%%%%%%%%%%%%%%%%%%%%%%%%%%%%%%
% \paragraph{Part Include Files.}
%
% The include files are called |cdocspt3.tex| and |cdocspt4.tex|.
%
%\iffalse
%<*samplepart3|samplepart4>
%\fi

% Optional override for |\version| flag:
%    \begin{macrocode}
%%\providecommand{\version}{final}
%    \end{macrocode}

% Include the main document:
%    \begin{macrocode}
\input{childdoc.def}
\childdocby{cdocsamp}
%    \end{macrocode}

%\iffalse
%</samplepart3|samplepart4>
%\fi
%
%\iffalse
%<*samplepart3>
%\fi
% Some text for part 3:
%    \begin{macrocode}
some text in part three
%    \end{macrocode}

%\iffalse
%</samplepart3>
%\fi
% Some text for part 4:
%\iffalse
%<*samplepart4>
%\fi
%    \begin{macrocode}
more text in part four
%    \end{macrocode}

%\iffalse
%</samplepart4>
%\fi
%
% %%%%%%%%%%%%%%%%%%%%%%%%%%%%%%%%%%%%%%
% \paragraph{Forwarding for a Complete Draft.}
%
% The following forwarding file |cdocsdrf.tex|
% compiles the main document in draft mode:
%\iffalse
%<*sampledraft>
%\fi
%    \begin{macrocode}
\def\version{draft}
\input{childdoc.def}
\childdocforward{cdocsamp}
%    \end{macrocode}

%\iffalse
%</sampledraft>
%\fi
%
% %%%%%%%%%%%%%%%%%%%%%%%%%%%%%%%%%%%%%%
% \paragraph{Forwarding for Final Version of the Chapters.}
%
% The following forwarding files |cdocsfn1.tex| and |cdocsfn2.tex|
% (with identical content)
% compile the final versions of the child documents
% |cdocsch1.tex| and |cdocsch2.tex|, respectively:
%\iffalse
%<*samplefinal>
%\fi
%    \begin{macrocode}
\def\version{final}
\input{childdoc.def}
\childdocforwardprefix[cdocsamp]{cdocsfn}{cdocsch}
%    \end{macrocode}

%\iffalse
%</samplefinal>
%\fi
%
% %%%%%%%%%%%%%%%%%%%%%%%%%%%%%%%%%%%%%%
% \paragraph{Command Line Processing.}
%
% The following three command lines generate the output files
% |cdocscld|, |cdocscl1| and |cdocscl2|
% which should be identical to
% |cdocsdrf|, |cdocsch1| and |cdocsfn2|, respectively:
% \begin{center}
% \begin{tabular}{l}
% |latex -jobname cdocscld \|\\
% |  "\def\version{draft}\input{childdoc.def}\childdocforward{cdocsamp}"|\\
% |latex -jobname cdocscl1 \|\\
% |  "\input{childdoc.def}\childdocforward[cdocsamp]{cdocsch1}"|\\
% |latex -jobname cdocscl2 \|\\
% |  "\def\version{final}\input{childdoc.def}\childdocforward{cdocsch2}"|
% \end{tabular}
% \end{center}
% Note that the trailing backslash on each first line
% merely continues the input to the second line
% (for convenient cut ant paste).
% Furthermore, the command |latex| can be replaced by any
% of its alternative versions such as |pdflatex|.
%
% %%%%%%%%%%%%%%%%%%%%%%%%%%%%%%%%%%%%%%%%%%%%%%%%%%%%%%%%%%%%%%%%%%%%%%%%%%%%%%
% %%%%%%%%%%%%%%%%%%%%%%%%%%%%%%%%%%%%%%%%%%%%%%%%%%%%%%%%%%%%%%%%%%%%%%%%%%%%%%
% \section{Implementation}
%\iffalse
%<*package>
%\fi
%
% This section describes the definitions file |childdoc.def|.

% The definitions cannot be loaded using |\usepackage| or |\RequirePackage|
% which has a mechanism to prevent loading a style file more than once.
% When loading the definitions by means of |\input|
% multiple instances have to be prevented manually:
%\iffalse
%This code needs to be before the `\ProvidesFile' directive
%which is defined at the beginning of this file.
%Therefore it is also placed there and commented out here.
%</package>
%<*discard>
%\fi
%    \begin{macrocode}
\ifdefined\childdocmain\endinput\fi
%    \end{macrocode}
%\iffalse
%</discard>
%<*package>
%\fi
%
% \macro{\ifchilddoc}
% \macro{\ifchilddocmanual}
% The conditional |\ifchilddoc| tells whether a
% child (true) or main (false) document is being compiled.
% The conditional |\ifchilddocmanual| tells whether
% the |\includeonly| mechanism is used (false) or
% the selection of child files must be performed manually (true).
% The definitions initialise to false:
%    \begin{macrocode}
\newif\ifchilddoc
\newif\ifchilddocmanual
%    \end{macrocode}

% \macro{\childdocname}
% \macro{\childdocjob}
% The macro |\childdocname| stores the name of the main document
% to be compiled. The macro |\childdocjob| stores the name of
% the document on which the \LaTeX{} compiler was originally invoked.
% The content of |\jobname| cannot be compared
% to filenames specified in the source due to different catcodes.
% The following code rescans |\jobname|, stores the result
% in |\childdocname| and saves a copy in |\childdocjob|:
%    \begin{macrocode}
\edef\childdocname{\scantokens\expandafter{\jobname\noexpand}}
\let\childdocjob\childdocname
%    \end{macrocode}

% \macro{\childdocdisable}
% The macro |\childdocdisable| prevents the main file
% from being processed more than once.
% At this stage, the main document command |\childdocmain|
% is assumed to be called once again where it should do nothing.
% Any subsequent call to it should prevent
% a secondary processing of the main document
% It overwrites the forwarding commands
% |\childdocof| and |\childdocforward|
% with empty macros to prevent further inclusions of the main document:
%    \begin{macrocode}
\newcommand{\childdocdisable}
{
  \renewcommand{\childdocmain}[1]{\renewcommand{\childdocmain}[1]{\endinput}}
  \renewcommand{\childdocof}[1]{}
  \renewcommand{\childdocby}[2][]{}
  \renewcommand{\childdocforward}[2][]{}
  \renewcommand{\childdocdisable}{}
}
%    \end{macrocode}

% \macro{\childdocmain}
% The macro |\childdocmain| is to be called at the top of the main file
% with nothing or the main filename (without extension) as argument.
% First, it breaks loops.
% If the argument is not empty and does not match |\childdocname|
% (which is set by the first inclusion of |childdoc.def|),
% |\ifchilddoc| is set to true, |\includeonly| is applied to the child file
% and |\jobname| is set to the main file
% (for proper handling of |.aux| files):
%    \begin{macrocode}
\newcommand{\childdocmain}[1]
{
  \childdocdisable\childdocmain{}
  \if?#1?\else
    \begingroup
      \def\childdoctmp{#1}
      \ifx\childdoctmp\childdocname
        \def\childdoctmp{}
      \else
        \def\childdoctmp
        {
          \childdoctrue
          \includeonly{\childdocname}
          \def\childdocjob{#1}
          \def\jobname{#1}
        }
      \fi
      \expandafter
    \endgroup
    \childdoctmp
  \fi
}
%    \end{macrocode}

% \macro{\childdocof}
% The command |\childdocof| redirects
% compilation to the main file |#1|.
%    \begin{macrocode}
\newcommand{\childdocof}[1]
{
  \childdocdisable
  \childdoctrue
  \includeonly{\childdocname}
  \def\jobname{#1}
  \def\childdocjob{#1}
  \input{#1}
}
%    \end{macrocode}

% \macro{\childdocby}
% The command |\childdocby| ....
%    \begin{macrocode}
\newcommand{\childdocby}[2][]
{
  \childdocdisable
  \childdoctrue
  \childdocmanualtrue
  \if?#1?\else
    \def\jobname{#2}
  \fi
  \def\childdocjob{#2}
  \input{#2}
  \endinput
}
%    \end{macrocode}

% \macro{\childdocforward}
% The command |\childdocforward| redirects
% compilation to the main file or
% (if the optional argument is given) a child file.
% Parameters are set as if the main file
% or a child file starting with |\childdocof| was compiled.
% Then compilation is handed over to the main file:
%    \begin{macrocode}
\newcommand{\childdocforward}[2][]
{
  \begingroup
    \if?#1?
      \def\childdoctmp
      {
        \def\childdocname{#2}
        \def\childdocjob{#2}
        \def\jobname{#2}
        \input{#2}
        \endinput
      }
    \else
      \def\childdoctmp
      {
        \childdocdisable
        \def\childdocname{#2}
        \childdoctrue
        \includeonly{#2}
        \def\childdocjob{#1}
        \def\jobname{#1}
        \input{#1}
        \endinput
      }
    \fi
    \expandafter
  \endgroup
  \childdoctmp
}
%    \end{macrocode}

% \macro{\childdocforwardprefix}
% The command |\childdocforwardprefix| redirects
% compilation to the main or a child file by means of a pattern.
% The prefix |#1| in the current filename is replaced by |#2|
% and the suffix of the current filename is kept
% (it is assumed that the filename does not contain the substring `|~~~|'
% which is used as a delimiter).
% Compilation is handed over to the new file by |\childdocforward|:
%    \begin{macrocode}
\newcommand{\childdocforwardprefix}[3][]
{
  \begingroup
    \def\childdocextract #2##1~~~{\def\childdoctmp{\childdocforward[#1]{#3##1}}}
    \expandafter\childdocextract\childdocname~~~
    \expandafter
  \endgroup
  \childdoctmp
}
%    \end{macrocode}

% \macro{\childdoc}
% The deprecated macro |\childdoc| is a legacy version of |\childdocmain|:
%    \begin{macrocode}
\newcommand{\childdoc}{\childdocmain}
%    \end{macrocode}

% \macro{\childdocredirect}
% The deprecated macro |\childdocredirect| is a legacy version
% of |\childdocforward| and |\childdocforwardprefix|:
%    \begin{macrocode}
\newcommand{\childdocredirect}[2][]
{
  \begingroup
    \if?#1?
      \def\childdoctmp{\childdocforward{#2}}
    \else
      \def\childdoctmp{\childdocforwardprefix{#1}{#2}}
    \fi
    \expandafter
  \endgroup
  \childdoctmp
}
%    \end{macrocode}

%\iffalse
%</package>
%\fi
%
\endinput
\childdocforward[|\textit{main}|]{|\textit{dest}|}"|
\end{center}
%
Here \textit{target} is the name of the output file,
\textit{main} is the name of the main file
and \textit{dest} is the name of the main or child file to be processed
(all filenames without extensions).
The optional argument \textit{main} can be omitted
if \textit{main} matches \textit{dest}.
Optionally, compilation \textit{flags} can be defined via |\def| commands.
This command line makes the \TeX{} engine believe
it is compiling the file \textit{target}
whose content is specified as the latter parameter.
The provided code then forwards the processing to
\textit{main} or \textit{dest} as described in \secref{sec:forward}.

%%%%%%%%%%%%%%%%%%%%%%%%%%%%%%%%%%%%%%%%%%%%%%%%%%%%%%%%%%%%%%%%%%%%%%%%%%%%%%%%
\subsection{Include by Input}
\label{sec:input}

Including child documents by |\include| has some restrictions by design.
Most notably, the content of a child document always occupies
its own set of pages; pages cannot be shared between child documents.
Usually, this behaviour makes perfect sense
because each child document contain an essential part of the document.
However, in some situations it may be desirable to compose
a document from a collection of parts
without having mandatory page breaks between then.
For this case, the package
provides a mechanism to include parts
by |\input| which can also be processed individually.
However, by construction this mechanism
requires manual handling of the content to be output.

%%%%%%%%%%%%%%%%%%%%%%%%%%%%%%%%%%%%%%%%
\DescribeMacro{\ifchilddocmanual}
The main file should be prepared as usual, see \secref{sec:include}.
However, the document body must make a distinction
between processing of an individual part and of the main document, e.g.:
%
\begin{center}
\begin{tabular}{l}
|\ifchilddocmanual|\\
|\input{\childdocname}|\\
|\||else|\\
\textit{document body with }|\input{|\textit{part}|}|\\
|\||fi|
\end{tabular}
\end{center}
%
The conditional |\ifchilddocmanual| is true whenever
a part to be included by |\input| is being compiled,
and the name of the part is stored in |\childdocname|.

%%%%%%%%%%%%%%%%%%%%%%%%%%%%%%%%%%%%%%%%
\DescribeMacro{\childdocby}
Each part to be included by |\input| should start with:
%
\begin{center}
\begin{tabular}{l}
|% \iffalse
%
% childdoc.dtx Copyright (C) 2017-2018 Niklas Beisert
%
% This work may be distributed and/or modified under the
% conditions of the LaTeX Project Public License, either version 1.3
% of this license or (at your option) any later version.
% The latest version of this license is in
%   http://www.latex-project.org/lppl.txt
% and version 1.3 or later is part of all distributions of LaTeX
% version 2005/12/01 or later.
%
% This work has the LPPL maintenance status `maintained'.
%
% The Current Maintainer of this work is Niklas Beisert.
%
% This work consists of the files childdoc.dtx and childdoc.ins
% and the derived files childdoc.def and cdocsamp.tex with
% cdocsch1.tex, cdocsch2.tex, cdocsdrf.tex, cdocsfn1.tex, cdocsfn2.tex.
%
%<package>\ifdefined\childdocmain\endinput\fi
%<package>\ProvidesFile{childdoc.def}[2018/12/30 v2.0 child document driver]
%<samplemain>\ProvidesFile{cdocsamp.tex}[2018/12/30 v2.0 sample for childdoc]
%<*driver>
%\ProvidesFile{childdoc.drv}[2018/12/30 v2.0 childdoc reference manual file]
\PassOptionsToClass{10pt,a4paper}{article}
\documentclass{ltxdoc}

\usepackage[margin=35mm]{geometry}
\usepackage{hyperref}
\usepackage{hyperxmp}
\usepackage[usenames]{color}

\hypersetup{colorlinks=true}
\hypersetup{pdfstartview=FitH}
\hypersetup{pdfpagemode=UseNone}
\hypersetup{pdfsource={}}
\hypersetup{pdflang={en-UK}}
\hypersetup{pdfcopyright={Copyright 2017-2018 Niklas Beisert.
  This work may be distributed and/or modified under the
  conditions of the LaTeX Project Public License, either version 1.3
  of this license or (at your option) any later version.}}
\hypersetup{pdflicenseurl={http://www.latex-project.org/lppl.txt}}
\hypersetup{pdfcontactaddress={ETH Zurich, ITP, HIT K,
  Wolfgang-Pauli-Strasse 27}}
\hypersetup{pdfcontactpostcode={8093}}
\hypersetup{pdfcontactcity={Zurich}}
\hypersetup{pdfcontactcountry={Switzerland}}
\hypersetup{pdfcontactemail={nbeisert@itp.phys.ethz.ch}}
\hypersetup{pdfcontacturl={http://people.phys.ethz.ch/\xmptilde nbeisert/}}

\newcommand{\secref}[1]{\hyperref[#1]{section \ref*{#1}}}

\parskip1ex
\parindent0pt
\let\olditemize\itemize
\def\itemize{\olditemize\parskip0pt}

\begin{document}

\title{The \textsf{childdoc} Package}
\hypersetup{pdftitle={The childdoc Package}}
\author{Niklas Beisert\\[2ex]
  Institut f\"ur Theoretische Physik\\
  Eidgen\"ossische Technische Hochschule Z\"urich\\
  Wolfgang-Pauli-Strasse 27, 8093 Z\"urich, Switzerland\\[1ex]
  \href{mailto:nbeisert@itp.phys.ethz.ch}
  {\texttt{nbeisert@itp.phys.ethz.ch}}}
\hypersetup{pdfauthor={Niklas Beisert}}
\hypersetup{pdfsubject={Manual for the LaTeX2e Package childdoc}}
\date{30 December 2018, \textsf{v2.0}}
\maketitle

\begin{abstract}\noindent
\textsf{childdoc} is a \LaTeXe{} package
that enables the direct compilation
of document sections included by |\include|
to individual files.
\end{abstract}

\begingroup
\parskip0ex
\tableofcontents
\endgroup

%%%%%%%%%%%%%%%%%%%%%%%%%%%%%%%%%%%%%%%%%%%%%%%%%%%%%%%%%%%%%%%%%%%%%%%%%%%%%%%%
%%%%%%%%%%%%%%%%%%%%%%%%%%%%%%%%%%%%%%%%%%%%%%%%%%%%%%%%%%%%%%%%%%%%%%%%%%%%%%%%
\section{Introduction}

\LaTeX{} provides a mechanism to structure a large document (such as a book)
into a main file and several child files (containing the chapters)
using the |\include| command.
This mechanism is beneficial for documents
which span hundreds of pages in order to
make the source file(s) more manageable.
Moreover, compilation can be restricted to
selected child files by means of the |\includeonly| command.
The latter feature can be used to reduce the compilation time while editing
(this was significantly more useful in the earlier days of \LaTeX{})
or to generate a smaller document which is easier to navigate.
Another application of |\includeonly| is to generate
documents consisting of selected parts of the complete document.

However, there are a few drawbacks of the plain |\include| mechanism:
\begin{itemize}
\item
The child files cannot be compiled on their own,
they can only be compiled via the main file.
A naive editing environment
(such as a text editor with an option
to have the current file processed by \LaTeX)
may require one to switch to the main file before compiling;
attempting to compile the child file produces errors.
\item
The main file must be modified (each time)
to adjust the |\includeonly| command
to the present needs. This easily leaves the main file in a messy state.
\item
The generated document will always carry the filename
of the main document. This is inconvenient if
several child files are to be compiled and
to be kept for distribution.
\end{itemize}

The present package provides a simple interface
to make child files individually compilable by \LaTeX{}.
Compiling a child file then has the same effect as compiling
the main file with an |\includeonly| command
to select the appropriate child.
Moreover the generated document will carry the name of the child
rather than the main file.
This resolves all three above issues.

This feature is meant to make the editing of books,
thesis documents and lecture notes somewhat more convenient.
However, the package can also be used efficiently for
composing a series of documents (such as exercise sheets)
which are typically distributed individually.
It then assists the author in generating the individual documents
(potentially in different versions)
as well as a document containing the collected series.
Another application is in developing style files
or other kinds of included material
where compilation of the style file could redirect
to a sample or test file.

%%%%%%%%%%%%%%%%%%%%%%%%%%%%%%%%%%%%%%%%%%%%%%%%%%%%%%%%%%%%%%%%%%%%%%%%%%%%%%%%
%%%%%%%%%%%%%%%%%%%%%%%%%%%%%%%%%%%%%%%%%%%%%%%%%%%%%%%%%%%%%%%%%%%%%%%%%%%%%%%%
\section{Usage}

First of all, the package \textsf{childdoc} is \emph{not} a standard
\LaTeXe{} |.sty| style file! Therefore it needs to be invoked in
a non-standard way.

%%%%%%%%%%%%%%%%%%%%%%%%%%%%%%%%%%%%%%%%%%%%%%%%%%%%%%%%%%%%%%%%%%%%%%%%%%%%%%%%
\subsection{Included Files}
\label{sec:include}

%%%%%%%%%%%%%%%%%%%%%%%%%%%%%%%%%%%%%%%%
\DescribeMacro{\childdocmain}
To use the package, add the commands
\begin{center}
\begin{tabular}{l}
|\input{childdoc.def}|\\
|\childdocmain{}|\\
\end{tabular}
\end{center}
at the very top of the main \LaTeX{} file,
in particular \emph{before} the |\documentclass| statement!
The argument of |\childdocmain| should be left empty
(but it must be present).

%%%%%%%%%%%%%%%%%%%%%%%%%%%%%%%%%%%%%%%%
\DescribeMacro{\childdocof}
Furthermore, add the commands
\begin{center}
\begin{tabular}{l}
|\input{childdoc.def}|\\
|\childdocof{|\textit{main}|}|\\
\end{tabular}
\end{center}
at the top of every child file \textit{child}
which is included by |\include{|\textit{child}|}|
from within the main file
(or at least for those files to be compiled individually).
The argument \textit{main} must be the filename of the main file.

There are a couple of
considerations in setting up the main and child documents:

%%%%%%%%%%%%%%%%%%%%%%%%%%%%%%%%%%%%%%%%
\paragraph{Restrictions.}

Please note the following restrictions:
\begin{itemize}
\item
|\childdocmain| must be called with one argument \textit{main}
to ensure compatibility with earlier version of the package.
It must either be empty (|\childdocmain{}|)
or precisely match the filename of the main file in which it is specified.
See \secref{sec:detection} for further information.
\item
The filename \textit{main} must be specified without the |.tex| extension.
\item
The filename \textit{main} is case sensitive
(even in case-insensitive file systems)
due to internal string comparison.
\item
The argument \textit{main} should be fully expanded, it cannot be a macro.
\item
Subdirectories and special characters should be avoided in filenames.
\item
The command |\childdocmain{|\textit{main}|}| must be followed by a whitespace.
It should not be followed immediately by another command
or by a comment mark `|%|'.
This is because the \TeX{} parser reads the token immediately following
the argument of |\childdocmain| and puts it
at the beginning of every child section;
however, a white\-space is ignored.
\end{itemize}

%%%%%%%%%%%%%%%%%%%%%%%%%%%%%%%%%%%%%%%%
\paragraph{Content of Main File.}

It is advisable to place all content in the child files included by |\include|.
Any output contained in the main file will appear in all child documents
unless suppressed manually;
it cannot be suppressed automatically by the |\includeonly| directive
and thus should normally be avoided.
A method to include some content in the main file
by means of conditional processing is described in \secref{sec:conditional}.

%%%%%%%%%%%%%%%%%%%%%%%%%%%%%%%%%%%%%%%%
\paragraph{Page Numbering.}

When only a part of the document is compiled,
the appropriate numbering of pages
(as well as other status parameters)
is determined from the |.aux| files.
The latter contain information from previous passes.
However this information needs to propagate through
all intermediate child documents.
Therefore the page numbering in child documents may well
be inconsistent until the complete document is compiled at least once.

A useful (if unconventional) way to always ensure a consistent
page numbering is to restart the numbering in each child document
and denote the pages by `\textit{child}|.|\textit{page}'
where \textit{child} represents the chapter/section number of the child file.
This can be achieved by the command
|\numberwithin{page}{|\textit{child}|}|
of the \textsf{amsmath} package
where \textit{child} can be |chapter| or |section|
depending on the chosen structuring.
Alternatively, one can modify the macro |\thepage| appropriately
and reset the counter |page| at the start of each child file.

%%%%%%%%%%%%%%%%%%%%%%%%%%%%%%%%%%%%%%%%%%%%%%%%%%%%%%%%%%%%%%%%%%%%%%%%%%%%%%%%
\subsection{Conditional Processing}
\label{sec:conditional}

The package provides a mechanism to compile different versions
of a document. To customise the versions further some conditional processing
can come in handy to distinguish which version is being compiled.
The package provides two macros to describe the compilation context:

%%%%%%%%%%%%%%%%%%%%%%%%%%%%%%%%%%%%%%%%
\DescribeMacro{\ifchilddoc}
The conditional |\ifchilddoc| distinguishes between the compilation of
child documents and the main document:
%
\begin{center}
|\ifchilddoc |\textit{child-code}| |[|\||else |\textit{main-code}]| \||fi|
\end{center}

%%%%%%%%%%%%%%%%%%%%%%%%%%%%%%%%%%%%%%%%
\DescribeMacro{\childdocname}
\DescribeMacro{\childdocjob}
The macro |\childdocname| contains the filename (without extension)
of the main or child file being processed.
Note that |\childdocjob| will always contain the name of the main file.

%%%%%%%%%%%%%%%%%%%%%%%%%%%%%%%%%%%%%%%%
\paragraph{Title Page.}

Conditional processing can be used to include a title or banner page
in the main document when proper precautions are taken.
Importantly, the code in the main file should ensure that the page counter
(as well as other status parameters which are stored in the |.aux| files)
takes the same value after the conditional processing.
Otherwise the page numbers may take divergent values
depending on which part is compiled.

For example, a title page could be declared by:
%
\begin{center}
\begin{tabular}{l}
|\ifchilddoc\||else|\\
|\addtocounter{page}{-1}|\\
\textit{code for title page}\\
|\newpage|\\
|\||fi|
\end{tabular}
\end{center}
%
A banner page for the child documents can be generated by:
%
\begin{center}
\begin{tabular}{l}
|\ifchilddoc|\\
|\addtocounter{page}{-1}|\\
\textit{code for banner page}\\
|\newpage|\\
|\||fi|
\end{tabular}
\end{center}
%
Here one could write a message such as:
\begin{center}
|This is the part \childdocname{} of \childdocjob{}.|
\end{center}

%%%%%%%%%%%%%%%%%%%%%%%%%%%%%%%%%%%%%%%%%%%%%%%%%%%%%%%%%%%%%%%%%%%%%%%%%%%%%%%%
\subsection{Flags}
\label{sec:flags}

The package makes it easy to generate different versions
of the main or child documents.
To this end compilation flags can be defined
and assigned different default values.
They will be particularly useful in conjunction
with the forwarding mechanism described in \secref{sec:forward}.

For example, it may be useful to have a flag |\version|
which can be set to |draft| or |final|.
The document source will contain some conditional code
depending on the value of |\version|.
Suppose further, the flag should default to |final| for the main file
and to |draft| for child files
which is a natural assignment for editing the document.
This is achieved by placing the following code
in the preamble of the main document
(below the |\childdocmain| directive):
%
\begin{center}
\begin{tabular}{l}
|\ifchilddoc|\\
|\providecommand{\version}{draft}|\\
|\||else|\\
|\providecommand{\version}{final}|\\
|\||fi|
\end{tabular}
\end{center}
%
The definition by |\providecommand| makes sure
that previous definitions are not overwritten.
Further statements |\providecommand{\version}{...}|
can thus be added before the above code to override it.

For the main file, one might add a line
(between |\childdocmain| and the above block)
%
\begin{center}
|%\ifchilddoc\||else\providecommand{\version}{draft}\||fi|
\end{center}
%
which can be uncommented to produce a draft version.
Likewise one can add a line to the very top of a child file
(above the |\childdocof{|\textit{main}|}| directive)
%
\begin{center}
|%\providecommand{\version}{final}|
\end{center}
%
which can be uncommented to produce the final version of this child document.

%%%%%%%%%%%%%%%%%%%%%%%%%%%%%%%%%%%%%%%%%%%%%%%%%%%%%%%%%%%%%%%%%%%%%%%%%%%%%%%%
\subsection{Forwarding}
\label{sec:forward}

Different versions of the main or child documents
using compilation flags as described in \secref{sec:flags}
can be (permanently) stored in different files
for convenient compilation, viewing and distribution.
To this end, the package defines a command
to pass on compilation to a different file:

%%%%%%%%%%%%%%%%%%%%%%%%%%%%%%%%%%%%%%%%
\DescribeMacro{\childdocforward}
The command |\childdocforward| redirects processing to
another source file:
%
\begin{center}
\begin{tabular}{l}
|\input{childdoc.def}|\\
|\childdocforward[|\textit{main}|]{|\textit{dest}|}|\\
\end{tabular}
\end{center}
%
The argument \textit{dest} is the destination file
(without extension).
It should be the main file or one of the child files.
Note that further \textsf{childdoc} directives
such as |\childdocof| and |\childdocforward|
in the indicated file will be processed in this form.
The optional argument \textit{main}
passes on directly to the main file \textit{main}
while pretending to compile the child \textit{dest}.
This form behaves as if \textit{dest}
issues |\childdocof{|\textit{main}|}| right away,
and no further \textsf{childdoc} directives will be processed.

%%%%%%%%%%%%%%%%%%%%%%%%%%%%%%%%%%%%%%%%
\DescribeMacro{\...prefix}
In the alternative form |\childdocforwardprefix|,
%
\begin{center}
\begin{tabular}{l}
|\input{childdoc.def}|\\
|\childdocforwardprefix[|\textit{main}|]{|\textit{prefix}|}{|\textit{dest}|}|
\end{tabular}
\end{center}
%
the destination file is determined by a pattern
depending on the current file:
To make this work, the current file must be called
`{\textit{prefix}\hspace{0.2em}\textit{suffix}}'
with \textit{prefix} matching precisely the argument.
Processing is then passed on to the file
`{\textit{dest}\hspace{0.2em}\textit{suffix}}'.
Surely, the same effect is achieved by
directly specifying the
argument `{\textit{dest}\hspace{0.2em}\textit{suffix}}'
in the first form.
However, that requires to set up a different file
for each child. With the alternative form of the command
all these files can have exactly the same content
which simplifies setting them up and maintaining them.

For example, the following file |draft.tex|
with a compilation flag |\version| as described in \secref{sec:flags}
compiles the main document as a draft:
%
\begin{center}
\begin{tabular}{l}
|\def\version{draft}|\\
|\input{childdoc.def}|\\
|\childdocforward{|\textit{main}|}|
\end{tabular}
\end{center}
%
Likewise, the following files |final|\textit{nn}|.tex|
compile the final version of the child document
|child|\textit{nn}|.tex|:
%
\begin{center}
\begin{tabular}{l}
|\def\version{final}|\\
|\input{childdoc.def}|\\
|\childdocforwardprefix{final}{child}|
\end{tabular}
\end{center}
%

Note that when several versions of a main file and/or of each child file
are to be generated, it may be convenient to set up a |Makefile| or
shell script to automatise the process.

%%%%%%%%%%%%%%%%%%%%%%%%%%%%%%%%%%%%%%%%%%%%%%%%%%%%%%%%%%%%%%%%%%%%%%%%%%%%%%%%
\subsection{Command Line Processing}
\label{sec:commandline}

The effect of redirection files can also be achieved by invoking
the \LaTeX{} compiler with a more elaborate command line.
Most conveniently this should be done as part
of a shell script or a |Makefile|.

When using \textsf{childdoc} in the main file, the following
command lines effectively perform a redirection
(note that depending on the shell being used,
backslashes may have to be doubled: `|\|' $\to$ `|\\|'):
%
\begin{center}
|... -jobname "|\textit{target}|" |\\|"|[\textit{flags}]%
|\input{childdoc.def}\childdocforward[|\textit{main}|]{|\textit{dest}|}"|
\end{center}
%
Here \textit{target} is the name of the output file,
\textit{main} is the name of the main file
and \textit{dest} is the name of the main or child file to be processed
(all filenames without extensions).
The optional argument \textit{main} can be omitted
if \textit{main} matches \textit{dest}.
Optionally, compilation \textit{flags} can be defined via |\def| commands.
This command line makes the \TeX{} engine believe
it is compiling the file \textit{target}
whose content is specified as the latter parameter.
The provided code then forwards the processing to
\textit{main} or \textit{dest} as described in \secref{sec:forward}.

%%%%%%%%%%%%%%%%%%%%%%%%%%%%%%%%%%%%%%%%%%%%%%%%%%%%%%%%%%%%%%%%%%%%%%%%%%%%%%%%
\subsection{Include by Input}
\label{sec:input}

Including child documents by |\include| has some restrictions by design.
Most notably, the content of a child document always occupies
its own set of pages; pages cannot be shared between child documents.
Usually, this behaviour makes perfect sense
because each child document contain an essential part of the document.
However, in some situations it may be desirable to compose
a document from a collection of parts
without having mandatory page breaks between then.
For this case, the package
provides a mechanism to include parts
by |\input| which can also be processed individually.
However, by construction this mechanism
requires manual handling of the content to be output.

%%%%%%%%%%%%%%%%%%%%%%%%%%%%%%%%%%%%%%%%
\DescribeMacro{\ifchilddocmanual}
The main file should be prepared as usual, see \secref{sec:include}.
However, the document body must make a distinction
between processing of an individual part and of the main document, e.g.:
%
\begin{center}
\begin{tabular}{l}
|\ifchilddocmanual|\\
|\input{\childdocname}|\\
|\||else|\\
\textit{document body with }|\input{|\textit{part}|}|\\
|\||fi|
\end{tabular}
\end{center}
%
The conditional |\ifchilddocmanual| is true whenever
a part to be included by |\input| is being compiled,
and the name of the part is stored in |\childdocname|.

%%%%%%%%%%%%%%%%%%%%%%%%%%%%%%%%%%%%%%%%
\DescribeMacro{\childdocby}
Each part to be included by |\input| should start with:
%
\begin{center}
\begin{tabular}{l}
|\input{childdoc.def}|\\
|\childdocby{|\textit{main}|}|\\
\end{tabular}
\end{center}
%
The directive |\childdocby| is similar to |\childdocof|
described in \secref{sec:include},
but the subsequent selection of content must be done manually.
To that end, both |\ifchilddoc| and |\ifchilddocmanual|
will be true upon processing of a part,
and the name of the part is stored in |\childdocname|.
Note that |\jobname| will be set to the filename of the current part
so that each part receives an individual |.aux| file
that does not interfere with the |.aux| file(s) of the main document.
This behaviour can be altered by the alternative form
|\childdocby[*]{|\textit{main}|}| (with a non-empty optional argument)
which uses the |.aux| file of the main document
by setting |\jobname| to \textit{main}.

%%%%%%%%%%%%%%%%%%%%%%%%%%%%%%%%%%%%%%%%%%%%%%%%%%%%%%%%%%%%%%%%%%%%%%%%%%%%%%%%
\subsection{Driver Development}
\label{sec:driver}

The \textsf{childdoc} mechanism can also be use for the development
of definition files such as \LaTeX{} styles or classes.
This case differs from the above setup with multiple parts
included by |\include| in that no |\includeonly| should be invoked.
This can be achieved by starting the include file
(before |\ProvidesPackage|) with:
%
\begin{center}
\begin{tabular}{l}
|\input{childdoc.def}|\\
|\childdocforward{|\textit{main}|}|\\
\end{tabular}
\end{center}
%
or alternatively with:
%
\begin{center}
\begin{tabular}{l}
|\input{childdoc.def}|\\
|\childdocby{|\textit{main}|}|\\
\end{tabular}
\end{center}
%
Both forms have slightly different effects as described above.
The main file is prepared as usual, see \secref{sec:include}.

%%%%%%%%%%%%%%%%%%%%%%%%%%%%%%%%%%%%%%%%%%%%%%%%%%%%%%%%%%%%%%%%%%%%%%%%%%%%%%%%
\subsection{Legacy Detection}
\label{sec:detection}

The directive |\childdocmain| in the main file can detect
whether the complete document or merely a child is to be compiled
even without using the directive |\childdocof|.
This method is deprecated because it is less robust
and there is no compelling reason to use it;
it is merely provided for backward compatibility
and it may be removed in future versions.

If the detection mechanism is to be used,
it is mandatory to correctly specify
the filename of the main file as the argument of |\childdocmain|:
%
\begin{center}
\begin{tabular}{l}
|\input{childdoc.def}|\\
|\childdocmain{|\textit{main}|}|\\
\end{tabular}
\end{center}
%
If |\jobname| does not match the argument \textit{main} of |\childdocmain|,
it is assumed that |\jobname| points to the child file to be compiled.
When using |\childdocmain| with the main file specified as argument,
it suffices to start a child file
with just |\input{|\textit{main}|}|
without loading of the package and using |\childdocof|.
If instead all processing is done
with the appropriate \textsf{childdoc} directives,
the argument of \textit{main} of |\childdocmain| can be empty.

An alternative version of the command line processing described
in \secref{sec:commandline} using the detection mechanism reads:
%
\begin{center}
|... -jobname "|\textit{target}|" "|[\textit{flags}]%
[|\def\jobname{|\textit{dest}|}|]|\input{|\textit{main}|}"|
\end{center}

%%%%%%%%%%%%%%%%%%%%%%%%%%%%%%%%%%%%%%%%%%%%%%%%%%%%%%%%%%%%%%%%%%%%%%%%%%%%%%%%
\subsection{Manual Code}
\label{sec:manual}

In case one cannot be certain whether the definitions file |childdoc.def|
is installed on the target \TeX{} distribution
and one prefers not to ship it,
it is conceivable to paste a few relevant commands into the sources.

To that end, drop all statements |\input{childdoc.def}|
and perform the replacements as outlined below.
Instead of |\childdocmain{|\textit{main}|}| add the following code
to the top of the main file:
%
\begin{center}
\begin{tabular}{l}
|\||ifdefined\childdocname\endinput\||fi\newif\ifchilddoc|\\
|\edef\childdocname{\scantokens\expandafter{\jobname\noexpand}}|\\
|\def\childdocmain{|\textit{main}|}\||ifx\childdocmain\childdocname\||else|\\
|\childdoctrue\includeonly{\childdocname}\let\jobname\childdocmain\||fi|\\
\end{tabular}
\end{center}
%
Instead of |\childdocof{|\textit{main}|}| just include the main file
at the top of each child file:
%
\begin{center}
|\input{|\textit{main}|}|
\end{center}
%
A simple redirection |\childdocforward{|\textit{dest}|}| is achieved by:
%
\begin{center}
|\def\jobname{|\textit{dest}|}\input{\jobname}|
\end{center}
%
The redirection with prefix
|\childdocforwardprefix[|\textit{prefix}|]{|\textit{dest}|}|
is accomplished by:
%
\begin{center}
\begin{tabular}{l}
|{\edef\jobname{\scantokens\expandafter{\jobname\noexpand}}|\\
|\def\redirectjob |\textit{prefix}|#1~~~{\gdef\jobname{|\textit{dest}|#1}}|\\
|\expandafter\redirectjob\jobname~~~}\input{\jobname}|
\end{tabular}
\end{center}

In an alternative approach,
child documents can be compiled by a specific command line
without additional code or specific definitions:
%
\begin{center}
|... -jobname "|\textit{target}|" "|[\textit{flags}]%
|\includeonly{|\textit{dest}|}\input{|\textit{main}|}"|
\end{center}
%

%%%%%%%%%%%%%%%%%%%%%%%%%%%%%%%%%%%%%%%%%%%%%%%%%%%%%%%%%%%%%%%%%%%%%%%%%%%%%%%%
%%%%%%%%%%%%%%%%%%%%%%%%%%%%%%%%%%%%%%%%%%%%%%%%%%%%%%%%%%%%%%%%%%%%%%%%%%%%%%%%
\section{Information}

%%%%%%%%%%%%%%%%%%%%%%%%%%%%%%%%%%%%%%%%%%%%%%%%%%%%%%%%%%%%%%%%%%%%%%%%%%%%%%%%
\subsection{Copyright}

Copyright \copyright{} 2017--2018 Niklas Beisert

This work may be distributed and/or modified under the
conditions of the \LaTeX{} Project Public License, either version 1.3
of this license or (at your option) any later version.
The latest version of this license is in
  \url{http://www.latex-project.org/lppl.txt}
and version 1.3 or later is part of all distributions of \LaTeX{}
version 2005/12/01 or later.

This work has the LPPL maintenance status `maintained'.

The Current Maintainer of this work is Niklas Beisert.

This work consists of the files |README.txt|, |childdoc.ins| and |childdoc.dtx|
as well as the derived files |childdoc.def|, |cdocsamp.tex|
with |cdocsch1.tex|, |cdocsch2.tex|, |cdocspt3.tex|, |cdocspt4.tex|,
|cdocsdrf.tex|, |cdocsfn1.tex|, |cdocsfn2.tex|
as well as |childdoc.pdf|.

%%%%%%%%%%%%%%%%%%%%%%%%%%%%%%%%%%%%%%%%%%%%%%%%%%%%%%%%%%%%%%%%%%%%%%%%%%%%%%%%
\subsection{Files and Installation}

The package consists of the files:
%
\begin{center}
\begin{tabular}{ll}
    |README.txt|   & readme file \\
    |childdoc.ins| & installation file \\
    |childdoc.dtx| & source file \\
    |childdoc.def| & definition file \\
    |cdocsamp.tex| & sample main file \\
    |cdocsch1.tex| & sample include file \\
    |cdocsch2.tex| & sample include file \\
    |cdocspt3.tex| & sample part file \\
    |cdocspt4.tex| & sample part file \\
    |cdocsdrf.tex| & sample redirection file \\
    |cdocsfn1.tex| & sample redirection file \\
    |cdocsfn2.tex| & sample redirection file \\
    |childdoc.pdf| & manual
\end{tabular}
\end{center}
%
The distribution consists of the files
|README.txt|, |childdoc.ins| and |childdoc.dtx|.
%
\begin{itemize}
\item
Run (pdf)\LaTeX{} on |childdoc.dtx|
to compile the manual |childdoc.pdf| (this file).
\item
Run \LaTeX{} on |childdoc.ins| to create the definitions file |childdoc.def|
and the sample |cdocsamp.tex| with include files
|cdocsch1.tex|, |cdocsch2.tex|, |cdocspt3.tex|, |cdocspt4.tex|,
|cdocsdrf.tex|, |cdocsfn1.tex|, |cdocsfn2.tex|.
Then copy the file |childdoc.def| to an appropriate directory of your \LaTeX{}
distribution, e.g.\ \textit{texmf-root}|/tex/latex/childdoc|.
\end{itemize}

%%%%%%%%%%%%%%%%%%%%%%%%%%%%%%%%%%%%%%%%%%%%%%%%%%%%%%%%%%%%%%%%%%%%%%%%%%%%%%%%
\subsection{Related CTAN Packages}

There are several other packages which offer a similar functionality:
%
\begin{itemize}
\item
The packages
\href{http://ctan.org/pkg/docmute}{\textsf{docmute}},
\href{http://ctan.org/pkg/includex}{\textsf{includex}} and
\href{http://ctan.org/pkg/standalone}{\textsf{standalone}}
provide commands to include only the document body of
a child file thus allowing both files to be compiled individually.
\item
The packages \href{http://ctan.org/pkg/subdocs}{\textsf{subdocs}}
and \href{http://ctan.org/pkg/subfiles}{\textsf{subfiles}}
provide structures in which the main and child documents can be
encapsulated and allowing them to be compiled individually.
The inclusion mechanism is different from the conventional |\include|.
\item
The package \href{http://ctan.org/pkg/combine}{\textsf{combine}}
is an elaborate solution to combine several documents into one.
\end{itemize}
%
See also the CTAN topic \href{http://ctan.org/topic/subdocs}{\textsf{subdocs}}
for further related packages.
The present package differs from the above solutions in that
a document structure constructed with the conventional |\include| mechanism
just needs two extra commands at the top of every file
such that all constituent files can be compiled individually.

%%%%%%%%%%%%%%%%%%%%%%%%%%%%%%%%%%%%%%%%%%%%%%%%%%%%%%%%%%%%%%%%%%%%%%%%%%%%%%%%
%\subsection{Feature Suggestions}
%
%The following is a list of features which may be useful for future
%versions of this package:
%%
%\begin{itemize}
%\item
%\ldots
%\end{itemize}

%%%%%%%%%%%%%%%%%%%%%%%%%%%%%%%%%%%%%%%%%%%%%%%%%%%%%%%%%%%%%%%%%%%%%%%%%%%%%%%%
\subsection{Revision History}

%%%%%%%%%%%%%%%%%%%%%%%%%%%%%%%%%%%%%%%%
\paragraph{v2.0:} 2018/12/30

\begin{itemize}
\item
immediate forward processing
\item
added |\childdocby| mechanism
\item
manual restructured
\end{itemize}

%%%%%%%%%%%%%%%%%%%%%%%%%%%%%%%%%%%%%%%%
\paragraph{v1.6:} 2018/01/17

\begin{itemize}
\item
application for development of include files
\item
corrections to manual
\end{itemize}

%%%%%%%%%%%%%%%%%%%%%%%%%%%%%%%%%%%%%%%%
\paragraph{v1.5:} 2017/05/21

\begin{itemize}
\item
more complete structuring introduced
\item
|\childdocof| introduced
\item
|\childdoc| renamed to |\childdocmain|
\item
|\childredirect| renamed to |\childdocforward| and |\childdocforwardprefix|
and functionality expanded
\end{itemize}

%%%%%%%%%%%%%%%%%%%%%%%%%%%%%%%%%%%%%%%%
\paragraph{v1.0:} 2017/04/27

\begin{itemize}
\item
manual and install package
\item
first version published on CTAN
\end{itemize}

%%%%%%%%%%%%%%%%%%%%%%%%%%%%%%%%%%%%%%%%
\paragraph{v0.6:} 2017/04/26

\begin{itemize}
\item
redirection mechanism added
\end{itemize}

%%%%%%%%%%%%%%%%%%%%%%%%%%%%%%%%%%%%%%%%
\paragraph{v0.5:} 2017/04/26

\begin{itemize}
\item
functionality in definition file
\end{itemize}


%%%%%%%%%%%%%%%%%%%%%%%%%%%%%%%%%%%%%%%%%%%%%%%%%%%%%%%%%%%%%%%%%%%%%%%%%%%%%%%%
%%%%%%%%%%%%%%%%%%%%%%%%%%%%%%%%%%%%%%%%%%%%%%%%%%%%%%%%%%%%%%%%%%%%%%%%%%%%%%%%
%%%%%%%%%%%%%%%%%%%%%%%%%%%%%%%%%%%%%%%%%%%%%%%%%%%%%%%%%%%%%%%%%%%%%%%%%%%%%%%%
\appendix

\settowidth\MacroIndent{\rmfamily\scriptsize 000\ }

 \DocInput{childdoc.dtx}

\end{document}
%</driver>
% \fi
%
% %%%%%%%%%%%%%%%%%%%%%%%%%%%%%%%%%%%%%%%%%%%%%%%%%%%%%%%%%%%%%%%%%%%%%%%%%%%%%%
% %%%%%%%%%%%%%%%%%%%%%%%%%%%%%%%%%%%%%%%%%%%%%%%%%%%%%%%%%%%%%%%%%%%%%%%%%%%%%%
% \section{Sample}
%\iffalse
%<*samplemain>
%\fi
%
% The following presents a sample document
% with two chapters, two parts, a title page,
% a compile flag as well as three forwarding files to set the flag.
% It consists of eight |.tex| files:
% \begin{center}
% \begin{tabular}{ll}
% |cdocsamp.tex|&main file\\
% |cdocsch1.tex|&include file for chapter 1\\
% |cdocsch2.tex|&include file for chapter 2\\
% |cdocspt3.tex|&include file for part 3\\
% |cdocspt4.tex|&include file for part 4\\
% |cdocsdrf.tex|&forwarding file for main file in draft mode\\
% |cdocsfi1.tex|&forwarding file for final version of chapter 1\\
% |cdocsfi2.tex|&forwarding file for final version of chapter 2\\
% \end{tabular}
% \end{center}
% Each of the eight files can be compiled directly by the \LaTeX{} compiler.
%
% %%%%%%%%%%%%%%%%%%%%%%%%%%%%%%%%%%%%%%
% \paragraph{Main File.}
%
% The main file is called |cdocsamp.tex|.
%
% Load the \textsf{childdoc} definitions and
% declare the filename for the main document:
%    \begin{macrocode}
\input{childdoc.def}
\childdocmain{}
%    \end{macrocode}

% Optional override for |\version| flag:
%    \begin{macrocode}
%%\ifchilddoc\else\providecommand{\version}{draft}\fi
%    \end{macrocode}

% Define the default values for the |\version| flag
% (|final| for the main file and |draft| for childs):
%    \begin{macrocode}
\ifchilddoc
\providecommand{\version}{draft}
\else
\providecommand{\version}{final}
\fi
%    \end{macrocode}

% Load the standard document class:
%    \begin{macrocode}
\documentclass[12pt]{article}
%    \end{macrocode}

% Start the document body:
%    \begin{macrocode}
\begin{document}
%    \end{macrocode}

% Declare a title page.
% Print title, part of document being processed and version flag:
%    \begin{macrocode}
\addtocounter{page}{-1}
\begin{center}
{\LARGE\bfseries{}childdoc example\par}
\vspace{1cm}
\ifchilddoc
\ifchilddocmanual part\else chapter\fi:
`\childdocname' of `\childdocjob'\par
\else
main document: `\childdocjob'\par
\fi
version: \version\par
\end{center}
\newpage
%    \end{macrocode}

% Manually include selected file,
% otherwise process as usual:
%    \begin{macrocode}
\ifchilddocmanual
\section*{part `\childdocname'}
\input{\childdocname}
\else
%    \end{macrocode}

% Include the two chapters:
%    \begin{macrocode}
\include{cdocsch1}
\include{cdocsch2}
%    \end{macrocode}

% Include the two parts unless only chapters should be displayed:
%    \begin{macrocode}
\ifchilddoc\else
\section{part three}
\input{cdocspt3}
\section{part four}
\input{cdocspt4}
\fi
%    \end{macrocode}

% Process as usual until here:
%    \begin{macrocode}
\fi
%    \end{macrocode}

% End of document body:
%    \begin{macrocode}
\end{document}
%    \end{macrocode}
%\iffalse
%</samplemain>
%\fi
%
% %%%%%%%%%%%%%%%%%%%%%%%%%%%%%%%%%%%%%%
% \paragraph{Chapter Include Files.}
%
% The include files are called |cdocsch1.tex| and |cdocsch2.tex|.
%
%\iffalse
%<*samplechap1|samplechap2>
%\fi

% Optional override for |\version| flag:
%    \begin{macrocode}
%%\providecommand{\version}{final}
%    \end{macrocode}

% Include the main document:
%    \begin{macrocode}
\input{childdoc.def}
\childdocof{cdocsamp}
%    \end{macrocode}

%\iffalse
%</samplechap1|samplechap2>
%\fi
%
%\iffalse
%<*samplechap1>
%\fi
% Some text for chapter 1:
%    \begin{macrocode}
\section{one}
some text in chapter one
%    \end{macrocode}

%\iffalse
%</samplechap1>
%\fi
% Some text for chapter 2:
%\iffalse
%<*samplechap2>
%\fi
%    \begin{macrocode}
\section{two}
more text in chapter two
%    \end{macrocode}

%\iffalse
%</samplechap2>
%\fi
%
% %%%%%%%%%%%%%%%%%%%%%%%%%%%%%%%%%%%%%%
% \paragraph{Part Include Files.}
%
% The include files are called |cdocspt3.tex| and |cdocspt4.tex|.
%
%\iffalse
%<*samplepart3|samplepart4>
%\fi

% Optional override for |\version| flag:
%    \begin{macrocode}
%%\providecommand{\version}{final}
%    \end{macrocode}

% Include the main document:
%    \begin{macrocode}
\input{childdoc.def}
\childdocby{cdocsamp}
%    \end{macrocode}

%\iffalse
%</samplepart3|samplepart4>
%\fi
%
%\iffalse
%<*samplepart3>
%\fi
% Some text for part 3:
%    \begin{macrocode}
some text in part three
%    \end{macrocode}

%\iffalse
%</samplepart3>
%\fi
% Some text for part 4:
%\iffalse
%<*samplepart4>
%\fi
%    \begin{macrocode}
more text in part four
%    \end{macrocode}

%\iffalse
%</samplepart4>
%\fi
%
% %%%%%%%%%%%%%%%%%%%%%%%%%%%%%%%%%%%%%%
% \paragraph{Forwarding for a Complete Draft.}
%
% The following forwarding file |cdocsdrf.tex|
% compiles the main document in draft mode:
%\iffalse
%<*sampledraft>
%\fi
%    \begin{macrocode}
\def\version{draft}
\input{childdoc.def}
\childdocforward{cdocsamp}
%    \end{macrocode}

%\iffalse
%</sampledraft>
%\fi
%
% %%%%%%%%%%%%%%%%%%%%%%%%%%%%%%%%%%%%%%
% \paragraph{Forwarding for Final Version of the Chapters.}
%
% The following forwarding files |cdocsfn1.tex| and |cdocsfn2.tex|
% (with identical content)
% compile the final versions of the child documents
% |cdocsch1.tex| and |cdocsch2.tex|, respectively:
%\iffalse
%<*samplefinal>
%\fi
%    \begin{macrocode}
\def\version{final}
\input{childdoc.def}
\childdocforwardprefix[cdocsamp]{cdocsfn}{cdocsch}
%    \end{macrocode}

%\iffalse
%</samplefinal>
%\fi
%
% %%%%%%%%%%%%%%%%%%%%%%%%%%%%%%%%%%%%%%
% \paragraph{Command Line Processing.}
%
% The following three command lines generate the output files
% |cdocscld|, |cdocscl1| and |cdocscl2|
% which should be identical to
% |cdocsdrf|, |cdocsch1| and |cdocsfn2|, respectively:
% \begin{center}
% \begin{tabular}{l}
% |latex -jobname cdocscld \|\\
% |  "\def\version{draft}\input{childdoc.def}\childdocforward{cdocsamp}"|\\
% |latex -jobname cdocscl1 \|\\
% |  "\input{childdoc.def}\childdocforward[cdocsamp]{cdocsch1}"|\\
% |latex -jobname cdocscl2 \|\\
% |  "\def\version{final}\input{childdoc.def}\childdocforward{cdocsch2}"|
% \end{tabular}
% \end{center}
% Note that the trailing backslash on each first line
% merely continues the input to the second line
% (for convenient cut ant paste).
% Furthermore, the command |latex| can be replaced by any
% of its alternative versions such as |pdflatex|.
%
% %%%%%%%%%%%%%%%%%%%%%%%%%%%%%%%%%%%%%%%%%%%%%%%%%%%%%%%%%%%%%%%%%%%%%%%%%%%%%%
% %%%%%%%%%%%%%%%%%%%%%%%%%%%%%%%%%%%%%%%%%%%%%%%%%%%%%%%%%%%%%%%%%%%%%%%%%%%%%%
% \section{Implementation}
%\iffalse
%<*package>
%\fi
%
% This section describes the definitions file |childdoc.def|.

% The definitions cannot be loaded using |\usepackage| or |\RequirePackage|
% which has a mechanism to prevent loading a style file more than once.
% When loading the definitions by means of |\input|
% multiple instances have to be prevented manually:
%\iffalse
%This code needs to be before the `\ProvidesFile' directive
%which is defined at the beginning of this file.
%Therefore it is also placed there and commented out here.
%</package>
%<*discard>
%\fi
%    \begin{macrocode}
\ifdefined\childdocmain\endinput\fi
%    \end{macrocode}
%\iffalse
%</discard>
%<*package>
%\fi
%
% \macro{\ifchilddoc}
% \macro{\ifchilddocmanual}
% The conditional |\ifchilddoc| tells whether a
% child (true) or main (false) document is being compiled.
% The conditional |\ifchilddocmanual| tells whether
% the |\includeonly| mechanism is used (false) or
% the selection of child files must be performed manually (true).
% The definitions initialise to false:
%    \begin{macrocode}
\newif\ifchilddoc
\newif\ifchilddocmanual
%    \end{macrocode}

% \macro{\childdocname}
% \macro{\childdocjob}
% The macro |\childdocname| stores the name of the main document
% to be compiled. The macro |\childdocjob| stores the name of
% the document on which the \LaTeX{} compiler was originally invoked.
% The content of |\jobname| cannot be compared
% to filenames specified in the source due to different catcodes.
% The following code rescans |\jobname|, stores the result
% in |\childdocname| and saves a copy in |\childdocjob|:
%    \begin{macrocode}
\edef\childdocname{\scantokens\expandafter{\jobname\noexpand}}
\let\childdocjob\childdocname
%    \end{macrocode}

% \macro{\childdocdisable}
% The macro |\childdocdisable| prevents the main file
% from being processed more than once.
% At this stage, the main document command |\childdocmain|
% is assumed to be called once again where it should do nothing.
% Any subsequent call to it should prevent
% a secondary processing of the main document
% It overwrites the forwarding commands
% |\childdocof| and |\childdocforward|
% with empty macros to prevent further inclusions of the main document:
%    \begin{macrocode}
\newcommand{\childdocdisable}
{
  \renewcommand{\childdocmain}[1]{\renewcommand{\childdocmain}[1]{\endinput}}
  \renewcommand{\childdocof}[1]{}
  \renewcommand{\childdocby}[2][]{}
  \renewcommand{\childdocforward}[2][]{}
  \renewcommand{\childdocdisable}{}
}
%    \end{macrocode}

% \macro{\childdocmain}
% The macro |\childdocmain| is to be called at the top of the main file
% with nothing or the main filename (without extension) as argument.
% First, it breaks loops.
% If the argument is not empty and does not match |\childdocname|
% (which is set by the first inclusion of |childdoc.def|),
% |\ifchilddoc| is set to true, |\includeonly| is applied to the child file
% and |\jobname| is set to the main file
% (for proper handling of |.aux| files):
%    \begin{macrocode}
\newcommand{\childdocmain}[1]
{
  \childdocdisable\childdocmain{}
  \if?#1?\else
    \begingroup
      \def\childdoctmp{#1}
      \ifx\childdoctmp\childdocname
        \def\childdoctmp{}
      \else
        \def\childdoctmp
        {
          \childdoctrue
          \includeonly{\childdocname}
          \def\childdocjob{#1}
          \def\jobname{#1}
        }
      \fi
      \expandafter
    \endgroup
    \childdoctmp
  \fi
}
%    \end{macrocode}

% \macro{\childdocof}
% The command |\childdocof| redirects
% compilation to the main file |#1|.
%    \begin{macrocode}
\newcommand{\childdocof}[1]
{
  \childdocdisable
  \childdoctrue
  \includeonly{\childdocname}
  \def\jobname{#1}
  \def\childdocjob{#1}
  \input{#1}
}
%    \end{macrocode}

% \macro{\childdocby}
% The command |\childdocby| ....
%    \begin{macrocode}
\newcommand{\childdocby}[2][]
{
  \childdocdisable
  \childdoctrue
  \childdocmanualtrue
  \if?#1?\else
    \def\jobname{#2}
  \fi
  \def\childdocjob{#2}
  \input{#2}
  \endinput
}
%    \end{macrocode}

% \macro{\childdocforward}
% The command |\childdocforward| redirects
% compilation to the main file or
% (if the optional argument is given) a child file.
% Parameters are set as if the main file
% or a child file starting with |\childdocof| was compiled.
% Then compilation is handed over to the main file:
%    \begin{macrocode}
\newcommand{\childdocforward}[2][]
{
  \begingroup
    \if?#1?
      \def\childdoctmp
      {
        \def\childdocname{#2}
        \def\childdocjob{#2}
        \def\jobname{#2}
        \input{#2}
        \endinput
      }
    \else
      \def\childdoctmp
      {
        \childdocdisable
        \def\childdocname{#2}
        \childdoctrue
        \includeonly{#2}
        \def\childdocjob{#1}
        \def\jobname{#1}
        \input{#1}
        \endinput
      }
    \fi
    \expandafter
  \endgroup
  \childdoctmp
}
%    \end{macrocode}

% \macro{\childdocforwardprefix}
% The command |\childdocforwardprefix| redirects
% compilation to the main or a child file by means of a pattern.
% The prefix |#1| in the current filename is replaced by |#2|
% and the suffix of the current filename is kept
% (it is assumed that the filename does not contain the substring `|~~~|'
% which is used as a delimiter).
% Compilation is handed over to the new file by |\childdocforward|:
%    \begin{macrocode}
\newcommand{\childdocforwardprefix}[3][]
{
  \begingroup
    \def\childdocextract #2##1~~~{\def\childdoctmp{\childdocforward[#1]{#3##1}}}
    \expandafter\childdocextract\childdocname~~~
    \expandafter
  \endgroup
  \childdoctmp
}
%    \end{macrocode}

% \macro{\childdoc}
% The deprecated macro |\childdoc| is a legacy version of |\childdocmain|:
%    \begin{macrocode}
\newcommand{\childdoc}{\childdocmain}
%    \end{macrocode}

% \macro{\childdocredirect}
% The deprecated macro |\childdocredirect| is a legacy version
% of |\childdocforward| and |\childdocforwardprefix|:
%    \begin{macrocode}
\newcommand{\childdocredirect}[2][]
{
  \begingroup
    \if?#1?
      \def\childdoctmp{\childdocforward{#2}}
    \else
      \def\childdoctmp{\childdocforwardprefix{#1}{#2}}
    \fi
    \expandafter
  \endgroup
  \childdoctmp
}
%    \end{macrocode}

%\iffalse
%</package>
%\fi
%
\endinput
|\\
|\childdocby{|\textit{main}|}|\\
\end{tabular}
\end{center}
%
The directive |\childdocby| is similar to |\childdocof|
described in \secref{sec:include},
but the subsequent selection of content must be done manually.
To that end, both |\ifchilddoc| and |\ifchilddocmanual|
will be true upon processing of a part,
and the name of the part is stored in |\childdocname|.
Note that |\jobname| will be set to the filename of the current part
so that each part receives an individual |.aux| file
that does not interfere with the |.aux| file(s) of the main document.
This behaviour can be altered by the alternative form
|\childdocby[*]{|\textit{main}|}| (with a non-empty optional argument)
which uses the |.aux| file of the main document
by setting |\jobname| to \textit{main}.

%%%%%%%%%%%%%%%%%%%%%%%%%%%%%%%%%%%%%%%%%%%%%%%%%%%%%%%%%%%%%%%%%%%%%%%%%%%%%%%%
\subsection{Driver Development}
\label{sec:driver}

The \textsf{childdoc} mechanism can also be use for the development
of definition files such as \LaTeX{} styles or classes.
This case differs from the above setup with multiple parts
included by |\include| in that no |\includeonly| should be invoked.
This can be achieved by starting the include file
(before |\ProvidesPackage|) with:
%
\begin{center}
\begin{tabular}{l}
|% \iffalse
%
% childdoc.dtx Copyright (C) 2017-2018 Niklas Beisert
%
% This work may be distributed and/or modified under the
% conditions of the LaTeX Project Public License, either version 1.3
% of this license or (at your option) any later version.
% The latest version of this license is in
%   http://www.latex-project.org/lppl.txt
% and version 1.3 or later is part of all distributions of LaTeX
% version 2005/12/01 or later.
%
% This work has the LPPL maintenance status `maintained'.
%
% The Current Maintainer of this work is Niklas Beisert.
%
% This work consists of the files childdoc.dtx and childdoc.ins
% and the derived files childdoc.def and cdocsamp.tex with
% cdocsch1.tex, cdocsch2.tex, cdocsdrf.tex, cdocsfn1.tex, cdocsfn2.tex.
%
%<package>\ifdefined\childdocmain\endinput\fi
%<package>\ProvidesFile{childdoc.def}[2018/12/30 v2.0 child document driver]
%<samplemain>\ProvidesFile{cdocsamp.tex}[2018/12/30 v2.0 sample for childdoc]
%<*driver>
%\ProvidesFile{childdoc.drv}[2018/12/30 v2.0 childdoc reference manual file]
\PassOptionsToClass{10pt,a4paper}{article}
\documentclass{ltxdoc}

\usepackage[margin=35mm]{geometry}
\usepackage{hyperref}
\usepackage{hyperxmp}
\usepackage[usenames]{color}

\hypersetup{colorlinks=true}
\hypersetup{pdfstartview=FitH}
\hypersetup{pdfpagemode=UseNone}
\hypersetup{pdfsource={}}
\hypersetup{pdflang={en-UK}}
\hypersetup{pdfcopyright={Copyright 2017-2018 Niklas Beisert.
  This work may be distributed and/or modified under the
  conditions of the LaTeX Project Public License, either version 1.3
  of this license or (at your option) any later version.}}
\hypersetup{pdflicenseurl={http://www.latex-project.org/lppl.txt}}
\hypersetup{pdfcontactaddress={ETH Zurich, ITP, HIT K,
  Wolfgang-Pauli-Strasse 27}}
\hypersetup{pdfcontactpostcode={8093}}
\hypersetup{pdfcontactcity={Zurich}}
\hypersetup{pdfcontactcountry={Switzerland}}
\hypersetup{pdfcontactemail={nbeisert@itp.phys.ethz.ch}}
\hypersetup{pdfcontacturl={http://people.phys.ethz.ch/\xmptilde nbeisert/}}

\newcommand{\secref}[1]{\hyperref[#1]{section \ref*{#1}}}

\parskip1ex
\parindent0pt
\let\olditemize\itemize
\def\itemize{\olditemize\parskip0pt}

\begin{document}

\title{The \textsf{childdoc} Package}
\hypersetup{pdftitle={The childdoc Package}}
\author{Niklas Beisert\\[2ex]
  Institut f\"ur Theoretische Physik\\
  Eidgen\"ossische Technische Hochschule Z\"urich\\
  Wolfgang-Pauli-Strasse 27, 8093 Z\"urich, Switzerland\\[1ex]
  \href{mailto:nbeisert@itp.phys.ethz.ch}
  {\texttt{nbeisert@itp.phys.ethz.ch}}}
\hypersetup{pdfauthor={Niklas Beisert}}
\hypersetup{pdfsubject={Manual for the LaTeX2e Package childdoc}}
\date{30 December 2018, \textsf{v2.0}}
\maketitle

\begin{abstract}\noindent
\textsf{childdoc} is a \LaTeXe{} package
that enables the direct compilation
of document sections included by |\include|
to individual files.
\end{abstract}

\begingroup
\parskip0ex
\tableofcontents
\endgroup

%%%%%%%%%%%%%%%%%%%%%%%%%%%%%%%%%%%%%%%%%%%%%%%%%%%%%%%%%%%%%%%%%%%%%%%%%%%%%%%%
%%%%%%%%%%%%%%%%%%%%%%%%%%%%%%%%%%%%%%%%%%%%%%%%%%%%%%%%%%%%%%%%%%%%%%%%%%%%%%%%
\section{Introduction}

\LaTeX{} provides a mechanism to structure a large document (such as a book)
into a main file and several child files (containing the chapters)
using the |\include| command.
This mechanism is beneficial for documents
which span hundreds of pages in order to
make the source file(s) more manageable.
Moreover, compilation can be restricted to
selected child files by means of the |\includeonly| command.
The latter feature can be used to reduce the compilation time while editing
(this was significantly more useful in the earlier days of \LaTeX{})
or to generate a smaller document which is easier to navigate.
Another application of |\includeonly| is to generate
documents consisting of selected parts of the complete document.

However, there are a few drawbacks of the plain |\include| mechanism:
\begin{itemize}
\item
The child files cannot be compiled on their own,
they can only be compiled via the main file.
A naive editing environment
(such as a text editor with an option
to have the current file processed by \LaTeX)
may require one to switch to the main file before compiling;
attempting to compile the child file produces errors.
\item
The main file must be modified (each time)
to adjust the |\includeonly| command
to the present needs. This easily leaves the main file in a messy state.
\item
The generated document will always carry the filename
of the main document. This is inconvenient if
several child files are to be compiled and
to be kept for distribution.
\end{itemize}

The present package provides a simple interface
to make child files individually compilable by \LaTeX{}.
Compiling a child file then has the same effect as compiling
the main file with an |\includeonly| command
to select the appropriate child.
Moreover the generated document will carry the name of the child
rather than the main file.
This resolves all three above issues.

This feature is meant to make the editing of books,
thesis documents and lecture notes somewhat more convenient.
However, the package can also be used efficiently for
composing a series of documents (such as exercise sheets)
which are typically distributed individually.
It then assists the author in generating the individual documents
(potentially in different versions)
as well as a document containing the collected series.
Another application is in developing style files
or other kinds of included material
where compilation of the style file could redirect
to a sample or test file.

%%%%%%%%%%%%%%%%%%%%%%%%%%%%%%%%%%%%%%%%%%%%%%%%%%%%%%%%%%%%%%%%%%%%%%%%%%%%%%%%
%%%%%%%%%%%%%%%%%%%%%%%%%%%%%%%%%%%%%%%%%%%%%%%%%%%%%%%%%%%%%%%%%%%%%%%%%%%%%%%%
\section{Usage}

First of all, the package \textsf{childdoc} is \emph{not} a standard
\LaTeXe{} |.sty| style file! Therefore it needs to be invoked in
a non-standard way.

%%%%%%%%%%%%%%%%%%%%%%%%%%%%%%%%%%%%%%%%%%%%%%%%%%%%%%%%%%%%%%%%%%%%%%%%%%%%%%%%
\subsection{Included Files}
\label{sec:include}

%%%%%%%%%%%%%%%%%%%%%%%%%%%%%%%%%%%%%%%%
\DescribeMacro{\childdocmain}
To use the package, add the commands
\begin{center}
\begin{tabular}{l}
|\input{childdoc.def}|\\
|\childdocmain{}|\\
\end{tabular}
\end{center}
at the very top of the main \LaTeX{} file,
in particular \emph{before} the |\documentclass| statement!
The argument of |\childdocmain| should be left empty
(but it must be present).

%%%%%%%%%%%%%%%%%%%%%%%%%%%%%%%%%%%%%%%%
\DescribeMacro{\childdocof}
Furthermore, add the commands
\begin{center}
\begin{tabular}{l}
|\input{childdoc.def}|\\
|\childdocof{|\textit{main}|}|\\
\end{tabular}
\end{center}
at the top of every child file \textit{child}
which is included by |\include{|\textit{child}|}|
from within the main file
(or at least for those files to be compiled individually).
The argument \textit{main} must be the filename of the main file.

There are a couple of
considerations in setting up the main and child documents:

%%%%%%%%%%%%%%%%%%%%%%%%%%%%%%%%%%%%%%%%
\paragraph{Restrictions.}

Please note the following restrictions:
\begin{itemize}
\item
|\childdocmain| must be called with one argument \textit{main}
to ensure compatibility with earlier version of the package.
It must either be empty (|\childdocmain{}|)
or precisely match the filename of the main file in which it is specified.
See \secref{sec:detection} for further information.
\item
The filename \textit{main} must be specified without the |.tex| extension.
\item
The filename \textit{main} is case sensitive
(even in case-insensitive file systems)
due to internal string comparison.
\item
The argument \textit{main} should be fully expanded, it cannot be a macro.
\item
Subdirectories and special characters should be avoided in filenames.
\item
The command |\childdocmain{|\textit{main}|}| must be followed by a whitespace.
It should not be followed immediately by another command
or by a comment mark `|%|'.
This is because the \TeX{} parser reads the token immediately following
the argument of |\childdocmain| and puts it
at the beginning of every child section;
however, a white\-space is ignored.
\end{itemize}

%%%%%%%%%%%%%%%%%%%%%%%%%%%%%%%%%%%%%%%%
\paragraph{Content of Main File.}

It is advisable to place all content in the child files included by |\include|.
Any output contained in the main file will appear in all child documents
unless suppressed manually;
it cannot be suppressed automatically by the |\includeonly| directive
and thus should normally be avoided.
A method to include some content in the main file
by means of conditional processing is described in \secref{sec:conditional}.

%%%%%%%%%%%%%%%%%%%%%%%%%%%%%%%%%%%%%%%%
\paragraph{Page Numbering.}

When only a part of the document is compiled,
the appropriate numbering of pages
(as well as other status parameters)
is determined from the |.aux| files.
The latter contain information from previous passes.
However this information needs to propagate through
all intermediate child documents.
Therefore the page numbering in child documents may well
be inconsistent until the complete document is compiled at least once.

A useful (if unconventional) way to always ensure a consistent
page numbering is to restart the numbering in each child document
and denote the pages by `\textit{child}|.|\textit{page}'
where \textit{child} represents the chapter/section number of the child file.
This can be achieved by the command
|\numberwithin{page}{|\textit{child}|}|
of the \textsf{amsmath} package
where \textit{child} can be |chapter| or |section|
depending on the chosen structuring.
Alternatively, one can modify the macro |\thepage| appropriately
and reset the counter |page| at the start of each child file.

%%%%%%%%%%%%%%%%%%%%%%%%%%%%%%%%%%%%%%%%%%%%%%%%%%%%%%%%%%%%%%%%%%%%%%%%%%%%%%%%
\subsection{Conditional Processing}
\label{sec:conditional}

The package provides a mechanism to compile different versions
of a document. To customise the versions further some conditional processing
can come in handy to distinguish which version is being compiled.
The package provides two macros to describe the compilation context:

%%%%%%%%%%%%%%%%%%%%%%%%%%%%%%%%%%%%%%%%
\DescribeMacro{\ifchilddoc}
The conditional |\ifchilddoc| distinguishes between the compilation of
child documents and the main document:
%
\begin{center}
|\ifchilddoc |\textit{child-code}| |[|\||else |\textit{main-code}]| \||fi|
\end{center}

%%%%%%%%%%%%%%%%%%%%%%%%%%%%%%%%%%%%%%%%
\DescribeMacro{\childdocname}
\DescribeMacro{\childdocjob}
The macro |\childdocname| contains the filename (without extension)
of the main or child file being processed.
Note that |\childdocjob| will always contain the name of the main file.

%%%%%%%%%%%%%%%%%%%%%%%%%%%%%%%%%%%%%%%%
\paragraph{Title Page.}

Conditional processing can be used to include a title or banner page
in the main document when proper precautions are taken.
Importantly, the code in the main file should ensure that the page counter
(as well as other status parameters which are stored in the |.aux| files)
takes the same value after the conditional processing.
Otherwise the page numbers may take divergent values
depending on which part is compiled.

For example, a title page could be declared by:
%
\begin{center}
\begin{tabular}{l}
|\ifchilddoc\||else|\\
|\addtocounter{page}{-1}|\\
\textit{code for title page}\\
|\newpage|\\
|\||fi|
\end{tabular}
\end{center}
%
A banner page for the child documents can be generated by:
%
\begin{center}
\begin{tabular}{l}
|\ifchilddoc|\\
|\addtocounter{page}{-1}|\\
\textit{code for banner page}\\
|\newpage|\\
|\||fi|
\end{tabular}
\end{center}
%
Here one could write a message such as:
\begin{center}
|This is the part \childdocname{} of \childdocjob{}.|
\end{center}

%%%%%%%%%%%%%%%%%%%%%%%%%%%%%%%%%%%%%%%%%%%%%%%%%%%%%%%%%%%%%%%%%%%%%%%%%%%%%%%%
\subsection{Flags}
\label{sec:flags}

The package makes it easy to generate different versions
of the main or child documents.
To this end compilation flags can be defined
and assigned different default values.
They will be particularly useful in conjunction
with the forwarding mechanism described in \secref{sec:forward}.

For example, it may be useful to have a flag |\version|
which can be set to |draft| or |final|.
The document source will contain some conditional code
depending on the value of |\version|.
Suppose further, the flag should default to |final| for the main file
and to |draft| for child files
which is a natural assignment for editing the document.
This is achieved by placing the following code
in the preamble of the main document
(below the |\childdocmain| directive):
%
\begin{center}
\begin{tabular}{l}
|\ifchilddoc|\\
|\providecommand{\version}{draft}|\\
|\||else|\\
|\providecommand{\version}{final}|\\
|\||fi|
\end{tabular}
\end{center}
%
The definition by |\providecommand| makes sure
that previous definitions are not overwritten.
Further statements |\providecommand{\version}{...}|
can thus be added before the above code to override it.

For the main file, one might add a line
(between |\childdocmain| and the above block)
%
\begin{center}
|%\ifchilddoc\||else\providecommand{\version}{draft}\||fi|
\end{center}
%
which can be uncommented to produce a draft version.
Likewise one can add a line to the very top of a child file
(above the |\childdocof{|\textit{main}|}| directive)
%
\begin{center}
|%\providecommand{\version}{final}|
\end{center}
%
which can be uncommented to produce the final version of this child document.

%%%%%%%%%%%%%%%%%%%%%%%%%%%%%%%%%%%%%%%%%%%%%%%%%%%%%%%%%%%%%%%%%%%%%%%%%%%%%%%%
\subsection{Forwarding}
\label{sec:forward}

Different versions of the main or child documents
using compilation flags as described in \secref{sec:flags}
can be (permanently) stored in different files
for convenient compilation, viewing and distribution.
To this end, the package defines a command
to pass on compilation to a different file:

%%%%%%%%%%%%%%%%%%%%%%%%%%%%%%%%%%%%%%%%
\DescribeMacro{\childdocforward}
The command |\childdocforward| redirects processing to
another source file:
%
\begin{center}
\begin{tabular}{l}
|\input{childdoc.def}|\\
|\childdocforward[|\textit{main}|]{|\textit{dest}|}|\\
\end{tabular}
\end{center}
%
The argument \textit{dest} is the destination file
(without extension).
It should be the main file or one of the child files.
Note that further \textsf{childdoc} directives
such as |\childdocof| and |\childdocforward|
in the indicated file will be processed in this form.
The optional argument \textit{main}
passes on directly to the main file \textit{main}
while pretending to compile the child \textit{dest}.
This form behaves as if \textit{dest}
issues |\childdocof{|\textit{main}|}| right away,
and no further \textsf{childdoc} directives will be processed.

%%%%%%%%%%%%%%%%%%%%%%%%%%%%%%%%%%%%%%%%
\DescribeMacro{\...prefix}
In the alternative form |\childdocforwardprefix|,
%
\begin{center}
\begin{tabular}{l}
|\input{childdoc.def}|\\
|\childdocforwardprefix[|\textit{main}|]{|\textit{prefix}|}{|\textit{dest}|}|
\end{tabular}
\end{center}
%
the destination file is determined by a pattern
depending on the current file:
To make this work, the current file must be called
`{\textit{prefix}\hspace{0.2em}\textit{suffix}}'
with \textit{prefix} matching precisely the argument.
Processing is then passed on to the file
`{\textit{dest}\hspace{0.2em}\textit{suffix}}'.
Surely, the same effect is achieved by
directly specifying the
argument `{\textit{dest}\hspace{0.2em}\textit{suffix}}'
in the first form.
However, that requires to set up a different file
for each child. With the alternative form of the command
all these files can have exactly the same content
which simplifies setting them up and maintaining them.

For example, the following file |draft.tex|
with a compilation flag |\version| as described in \secref{sec:flags}
compiles the main document as a draft:
%
\begin{center}
\begin{tabular}{l}
|\def\version{draft}|\\
|\input{childdoc.def}|\\
|\childdocforward{|\textit{main}|}|
\end{tabular}
\end{center}
%
Likewise, the following files |final|\textit{nn}|.tex|
compile the final version of the child document
|child|\textit{nn}|.tex|:
%
\begin{center}
\begin{tabular}{l}
|\def\version{final}|\\
|\input{childdoc.def}|\\
|\childdocforwardprefix{final}{child}|
\end{tabular}
\end{center}
%

Note that when several versions of a main file and/or of each child file
are to be generated, it may be convenient to set up a |Makefile| or
shell script to automatise the process.

%%%%%%%%%%%%%%%%%%%%%%%%%%%%%%%%%%%%%%%%%%%%%%%%%%%%%%%%%%%%%%%%%%%%%%%%%%%%%%%%
\subsection{Command Line Processing}
\label{sec:commandline}

The effect of redirection files can also be achieved by invoking
the \LaTeX{} compiler with a more elaborate command line.
Most conveniently this should be done as part
of a shell script or a |Makefile|.

When using \textsf{childdoc} in the main file, the following
command lines effectively perform a redirection
(note that depending on the shell being used,
backslashes may have to be doubled: `|\|' $\to$ `|\\|'):
%
\begin{center}
|... -jobname "|\textit{target}|" |\\|"|[\textit{flags}]%
|\input{childdoc.def}\childdocforward[|\textit{main}|]{|\textit{dest}|}"|
\end{center}
%
Here \textit{target} is the name of the output file,
\textit{main} is the name of the main file
and \textit{dest} is the name of the main or child file to be processed
(all filenames without extensions).
The optional argument \textit{main} can be omitted
if \textit{main} matches \textit{dest}.
Optionally, compilation \textit{flags} can be defined via |\def| commands.
This command line makes the \TeX{} engine believe
it is compiling the file \textit{target}
whose content is specified as the latter parameter.
The provided code then forwards the processing to
\textit{main} or \textit{dest} as described in \secref{sec:forward}.

%%%%%%%%%%%%%%%%%%%%%%%%%%%%%%%%%%%%%%%%%%%%%%%%%%%%%%%%%%%%%%%%%%%%%%%%%%%%%%%%
\subsection{Include by Input}
\label{sec:input}

Including child documents by |\include| has some restrictions by design.
Most notably, the content of a child document always occupies
its own set of pages; pages cannot be shared between child documents.
Usually, this behaviour makes perfect sense
because each child document contain an essential part of the document.
However, in some situations it may be desirable to compose
a document from a collection of parts
without having mandatory page breaks between then.
For this case, the package
provides a mechanism to include parts
by |\input| which can also be processed individually.
However, by construction this mechanism
requires manual handling of the content to be output.

%%%%%%%%%%%%%%%%%%%%%%%%%%%%%%%%%%%%%%%%
\DescribeMacro{\ifchilddocmanual}
The main file should be prepared as usual, see \secref{sec:include}.
However, the document body must make a distinction
between processing of an individual part and of the main document, e.g.:
%
\begin{center}
\begin{tabular}{l}
|\ifchilddocmanual|\\
|\input{\childdocname}|\\
|\||else|\\
\textit{document body with }|\input{|\textit{part}|}|\\
|\||fi|
\end{tabular}
\end{center}
%
The conditional |\ifchilddocmanual| is true whenever
a part to be included by |\input| is being compiled,
and the name of the part is stored in |\childdocname|.

%%%%%%%%%%%%%%%%%%%%%%%%%%%%%%%%%%%%%%%%
\DescribeMacro{\childdocby}
Each part to be included by |\input| should start with:
%
\begin{center}
\begin{tabular}{l}
|\input{childdoc.def}|\\
|\childdocby{|\textit{main}|}|\\
\end{tabular}
\end{center}
%
The directive |\childdocby| is similar to |\childdocof|
described in \secref{sec:include},
but the subsequent selection of content must be done manually.
To that end, both |\ifchilddoc| and |\ifchilddocmanual|
will be true upon processing of a part,
and the name of the part is stored in |\childdocname|.
Note that |\jobname| will be set to the filename of the current part
so that each part receives an individual |.aux| file
that does not interfere with the |.aux| file(s) of the main document.
This behaviour can be altered by the alternative form
|\childdocby[*]{|\textit{main}|}| (with a non-empty optional argument)
which uses the |.aux| file of the main document
by setting |\jobname| to \textit{main}.

%%%%%%%%%%%%%%%%%%%%%%%%%%%%%%%%%%%%%%%%%%%%%%%%%%%%%%%%%%%%%%%%%%%%%%%%%%%%%%%%
\subsection{Driver Development}
\label{sec:driver}

The \textsf{childdoc} mechanism can also be use for the development
of definition files such as \LaTeX{} styles or classes.
This case differs from the above setup with multiple parts
included by |\include| in that no |\includeonly| should be invoked.
This can be achieved by starting the include file
(before |\ProvidesPackage|) with:
%
\begin{center}
\begin{tabular}{l}
|\input{childdoc.def}|\\
|\childdocforward{|\textit{main}|}|\\
\end{tabular}
\end{center}
%
or alternatively with:
%
\begin{center}
\begin{tabular}{l}
|\input{childdoc.def}|\\
|\childdocby{|\textit{main}|}|\\
\end{tabular}
\end{center}
%
Both forms have slightly different effects as described above.
The main file is prepared as usual, see \secref{sec:include}.

%%%%%%%%%%%%%%%%%%%%%%%%%%%%%%%%%%%%%%%%%%%%%%%%%%%%%%%%%%%%%%%%%%%%%%%%%%%%%%%%
\subsection{Legacy Detection}
\label{sec:detection}

The directive |\childdocmain| in the main file can detect
whether the complete document or merely a child is to be compiled
even without using the directive |\childdocof|.
This method is deprecated because it is less robust
and there is no compelling reason to use it;
it is merely provided for backward compatibility
and it may be removed in future versions.

If the detection mechanism is to be used,
it is mandatory to correctly specify
the filename of the main file as the argument of |\childdocmain|:
%
\begin{center}
\begin{tabular}{l}
|\input{childdoc.def}|\\
|\childdocmain{|\textit{main}|}|\\
\end{tabular}
\end{center}
%
If |\jobname| does not match the argument \textit{main} of |\childdocmain|,
it is assumed that |\jobname| points to the child file to be compiled.
When using |\childdocmain| with the main file specified as argument,
it suffices to start a child file
with just |\input{|\textit{main}|}|
without loading of the package and using |\childdocof|.
If instead all processing is done
with the appropriate \textsf{childdoc} directives,
the argument of \textit{main} of |\childdocmain| can be empty.

An alternative version of the command line processing described
in \secref{sec:commandline} using the detection mechanism reads:
%
\begin{center}
|... -jobname "|\textit{target}|" "|[\textit{flags}]%
[|\def\jobname{|\textit{dest}|}|]|\input{|\textit{main}|}"|
\end{center}

%%%%%%%%%%%%%%%%%%%%%%%%%%%%%%%%%%%%%%%%%%%%%%%%%%%%%%%%%%%%%%%%%%%%%%%%%%%%%%%%
\subsection{Manual Code}
\label{sec:manual}

In case one cannot be certain whether the definitions file |childdoc.def|
is installed on the target \TeX{} distribution
and one prefers not to ship it,
it is conceivable to paste a few relevant commands into the sources.

To that end, drop all statements |\input{childdoc.def}|
and perform the replacements as outlined below.
Instead of |\childdocmain{|\textit{main}|}| add the following code
to the top of the main file:
%
\begin{center}
\begin{tabular}{l}
|\||ifdefined\childdocname\endinput\||fi\newif\ifchilddoc|\\
|\edef\childdocname{\scantokens\expandafter{\jobname\noexpand}}|\\
|\def\childdocmain{|\textit{main}|}\||ifx\childdocmain\childdocname\||else|\\
|\childdoctrue\includeonly{\childdocname}\let\jobname\childdocmain\||fi|\\
\end{tabular}
\end{center}
%
Instead of |\childdocof{|\textit{main}|}| just include the main file
at the top of each child file:
%
\begin{center}
|\input{|\textit{main}|}|
\end{center}
%
A simple redirection |\childdocforward{|\textit{dest}|}| is achieved by:
%
\begin{center}
|\def\jobname{|\textit{dest}|}\input{\jobname}|
\end{center}
%
The redirection with prefix
|\childdocforwardprefix[|\textit{prefix}|]{|\textit{dest}|}|
is accomplished by:
%
\begin{center}
\begin{tabular}{l}
|{\edef\jobname{\scantokens\expandafter{\jobname\noexpand}}|\\
|\def\redirectjob |\textit{prefix}|#1~~~{\gdef\jobname{|\textit{dest}|#1}}|\\
|\expandafter\redirectjob\jobname~~~}\input{\jobname}|
\end{tabular}
\end{center}

In an alternative approach,
child documents can be compiled by a specific command line
without additional code or specific definitions:
%
\begin{center}
|... -jobname "|\textit{target}|" "|[\textit{flags}]%
|\includeonly{|\textit{dest}|}\input{|\textit{main}|}"|
\end{center}
%

%%%%%%%%%%%%%%%%%%%%%%%%%%%%%%%%%%%%%%%%%%%%%%%%%%%%%%%%%%%%%%%%%%%%%%%%%%%%%%%%
%%%%%%%%%%%%%%%%%%%%%%%%%%%%%%%%%%%%%%%%%%%%%%%%%%%%%%%%%%%%%%%%%%%%%%%%%%%%%%%%
\section{Information}

%%%%%%%%%%%%%%%%%%%%%%%%%%%%%%%%%%%%%%%%%%%%%%%%%%%%%%%%%%%%%%%%%%%%%%%%%%%%%%%%
\subsection{Copyright}

Copyright \copyright{} 2017--2018 Niklas Beisert

This work may be distributed and/or modified under the
conditions of the \LaTeX{} Project Public License, either version 1.3
of this license or (at your option) any later version.
The latest version of this license is in
  \url{http://www.latex-project.org/lppl.txt}
and version 1.3 or later is part of all distributions of \LaTeX{}
version 2005/12/01 or later.

This work has the LPPL maintenance status `maintained'.

The Current Maintainer of this work is Niklas Beisert.

This work consists of the files |README.txt|, |childdoc.ins| and |childdoc.dtx|
as well as the derived files |childdoc.def|, |cdocsamp.tex|
with |cdocsch1.tex|, |cdocsch2.tex|, |cdocspt3.tex|, |cdocspt4.tex|,
|cdocsdrf.tex|, |cdocsfn1.tex|, |cdocsfn2.tex|
as well as |childdoc.pdf|.

%%%%%%%%%%%%%%%%%%%%%%%%%%%%%%%%%%%%%%%%%%%%%%%%%%%%%%%%%%%%%%%%%%%%%%%%%%%%%%%%
\subsection{Files and Installation}

The package consists of the files:
%
\begin{center}
\begin{tabular}{ll}
    |README.txt|   & readme file \\
    |childdoc.ins| & installation file \\
    |childdoc.dtx| & source file \\
    |childdoc.def| & definition file \\
    |cdocsamp.tex| & sample main file \\
    |cdocsch1.tex| & sample include file \\
    |cdocsch2.tex| & sample include file \\
    |cdocspt3.tex| & sample part file \\
    |cdocspt4.tex| & sample part file \\
    |cdocsdrf.tex| & sample redirection file \\
    |cdocsfn1.tex| & sample redirection file \\
    |cdocsfn2.tex| & sample redirection file \\
    |childdoc.pdf| & manual
\end{tabular}
\end{center}
%
The distribution consists of the files
|README.txt|, |childdoc.ins| and |childdoc.dtx|.
%
\begin{itemize}
\item
Run (pdf)\LaTeX{} on |childdoc.dtx|
to compile the manual |childdoc.pdf| (this file).
\item
Run \LaTeX{} on |childdoc.ins| to create the definitions file |childdoc.def|
and the sample |cdocsamp.tex| with include files
|cdocsch1.tex|, |cdocsch2.tex|, |cdocspt3.tex|, |cdocspt4.tex|,
|cdocsdrf.tex|, |cdocsfn1.tex|, |cdocsfn2.tex|.
Then copy the file |childdoc.def| to an appropriate directory of your \LaTeX{}
distribution, e.g.\ \textit{texmf-root}|/tex/latex/childdoc|.
\end{itemize}

%%%%%%%%%%%%%%%%%%%%%%%%%%%%%%%%%%%%%%%%%%%%%%%%%%%%%%%%%%%%%%%%%%%%%%%%%%%%%%%%
\subsection{Related CTAN Packages}

There are several other packages which offer a similar functionality:
%
\begin{itemize}
\item
The packages
\href{http://ctan.org/pkg/docmute}{\textsf{docmute}},
\href{http://ctan.org/pkg/includex}{\textsf{includex}} and
\href{http://ctan.org/pkg/standalone}{\textsf{standalone}}
provide commands to include only the document body of
a child file thus allowing both files to be compiled individually.
\item
The packages \href{http://ctan.org/pkg/subdocs}{\textsf{subdocs}}
and \href{http://ctan.org/pkg/subfiles}{\textsf{subfiles}}
provide structures in which the main and child documents can be
encapsulated and allowing them to be compiled individually.
The inclusion mechanism is different from the conventional |\include|.
\item
The package \href{http://ctan.org/pkg/combine}{\textsf{combine}}
is an elaborate solution to combine several documents into one.
\end{itemize}
%
See also the CTAN topic \href{http://ctan.org/topic/subdocs}{\textsf{subdocs}}
for further related packages.
The present package differs from the above solutions in that
a document structure constructed with the conventional |\include| mechanism
just needs two extra commands at the top of every file
such that all constituent files can be compiled individually.

%%%%%%%%%%%%%%%%%%%%%%%%%%%%%%%%%%%%%%%%%%%%%%%%%%%%%%%%%%%%%%%%%%%%%%%%%%%%%%%%
%\subsection{Feature Suggestions}
%
%The following is a list of features which may be useful for future
%versions of this package:
%%
%\begin{itemize}
%\item
%\ldots
%\end{itemize}

%%%%%%%%%%%%%%%%%%%%%%%%%%%%%%%%%%%%%%%%%%%%%%%%%%%%%%%%%%%%%%%%%%%%%%%%%%%%%%%%
\subsection{Revision History}

%%%%%%%%%%%%%%%%%%%%%%%%%%%%%%%%%%%%%%%%
\paragraph{v2.0:} 2018/12/30

\begin{itemize}
\item
immediate forward processing
\item
added |\childdocby| mechanism
\item
manual restructured
\end{itemize}

%%%%%%%%%%%%%%%%%%%%%%%%%%%%%%%%%%%%%%%%
\paragraph{v1.6:} 2018/01/17

\begin{itemize}
\item
application for development of include files
\item
corrections to manual
\end{itemize}

%%%%%%%%%%%%%%%%%%%%%%%%%%%%%%%%%%%%%%%%
\paragraph{v1.5:} 2017/05/21

\begin{itemize}
\item
more complete structuring introduced
\item
|\childdocof| introduced
\item
|\childdoc| renamed to |\childdocmain|
\item
|\childredirect| renamed to |\childdocforward| and |\childdocforwardprefix|
and functionality expanded
\end{itemize}

%%%%%%%%%%%%%%%%%%%%%%%%%%%%%%%%%%%%%%%%
\paragraph{v1.0:} 2017/04/27

\begin{itemize}
\item
manual and install package
\item
first version published on CTAN
\end{itemize}

%%%%%%%%%%%%%%%%%%%%%%%%%%%%%%%%%%%%%%%%
\paragraph{v0.6:} 2017/04/26

\begin{itemize}
\item
redirection mechanism added
\end{itemize}

%%%%%%%%%%%%%%%%%%%%%%%%%%%%%%%%%%%%%%%%
\paragraph{v0.5:} 2017/04/26

\begin{itemize}
\item
functionality in definition file
\end{itemize}


%%%%%%%%%%%%%%%%%%%%%%%%%%%%%%%%%%%%%%%%%%%%%%%%%%%%%%%%%%%%%%%%%%%%%%%%%%%%%%%%
%%%%%%%%%%%%%%%%%%%%%%%%%%%%%%%%%%%%%%%%%%%%%%%%%%%%%%%%%%%%%%%%%%%%%%%%%%%%%%%%
%%%%%%%%%%%%%%%%%%%%%%%%%%%%%%%%%%%%%%%%%%%%%%%%%%%%%%%%%%%%%%%%%%%%%%%%%%%%%%%%
\appendix

\settowidth\MacroIndent{\rmfamily\scriptsize 000\ }

 \DocInput{childdoc.dtx}

\end{document}
%</driver>
% \fi
%
% %%%%%%%%%%%%%%%%%%%%%%%%%%%%%%%%%%%%%%%%%%%%%%%%%%%%%%%%%%%%%%%%%%%%%%%%%%%%%%
% %%%%%%%%%%%%%%%%%%%%%%%%%%%%%%%%%%%%%%%%%%%%%%%%%%%%%%%%%%%%%%%%%%%%%%%%%%%%%%
% \section{Sample}
%\iffalse
%<*samplemain>
%\fi
%
% The following presents a sample document
% with two chapters, two parts, a title page,
% a compile flag as well as three forwarding files to set the flag.
% It consists of eight |.tex| files:
% \begin{center}
% \begin{tabular}{ll}
% |cdocsamp.tex|&main file\\
% |cdocsch1.tex|&include file for chapter 1\\
% |cdocsch2.tex|&include file for chapter 2\\
% |cdocspt3.tex|&include file for part 3\\
% |cdocspt4.tex|&include file for part 4\\
% |cdocsdrf.tex|&forwarding file for main file in draft mode\\
% |cdocsfi1.tex|&forwarding file for final version of chapter 1\\
% |cdocsfi2.tex|&forwarding file for final version of chapter 2\\
% \end{tabular}
% \end{center}
% Each of the eight files can be compiled directly by the \LaTeX{} compiler.
%
% %%%%%%%%%%%%%%%%%%%%%%%%%%%%%%%%%%%%%%
% \paragraph{Main File.}
%
% The main file is called |cdocsamp.tex|.
%
% Load the \textsf{childdoc} definitions and
% declare the filename for the main document:
%    \begin{macrocode}
\input{childdoc.def}
\childdocmain{}
%    \end{macrocode}

% Optional override for |\version| flag:
%    \begin{macrocode}
%%\ifchilddoc\else\providecommand{\version}{draft}\fi
%    \end{macrocode}

% Define the default values for the |\version| flag
% (|final| for the main file and |draft| for childs):
%    \begin{macrocode}
\ifchilddoc
\providecommand{\version}{draft}
\else
\providecommand{\version}{final}
\fi
%    \end{macrocode}

% Load the standard document class:
%    \begin{macrocode}
\documentclass[12pt]{article}
%    \end{macrocode}

% Start the document body:
%    \begin{macrocode}
\begin{document}
%    \end{macrocode}

% Declare a title page.
% Print title, part of document being processed and version flag:
%    \begin{macrocode}
\addtocounter{page}{-1}
\begin{center}
{\LARGE\bfseries{}childdoc example\par}
\vspace{1cm}
\ifchilddoc
\ifchilddocmanual part\else chapter\fi:
`\childdocname' of `\childdocjob'\par
\else
main document: `\childdocjob'\par
\fi
version: \version\par
\end{center}
\newpage
%    \end{macrocode}

% Manually include selected file,
% otherwise process as usual:
%    \begin{macrocode}
\ifchilddocmanual
\section*{part `\childdocname'}
\input{\childdocname}
\else
%    \end{macrocode}

% Include the two chapters:
%    \begin{macrocode}
\include{cdocsch1}
\include{cdocsch2}
%    \end{macrocode}

% Include the two parts unless only chapters should be displayed:
%    \begin{macrocode}
\ifchilddoc\else
\section{part three}
\input{cdocspt3}
\section{part four}
\input{cdocspt4}
\fi
%    \end{macrocode}

% Process as usual until here:
%    \begin{macrocode}
\fi
%    \end{macrocode}

% End of document body:
%    \begin{macrocode}
\end{document}
%    \end{macrocode}
%\iffalse
%</samplemain>
%\fi
%
% %%%%%%%%%%%%%%%%%%%%%%%%%%%%%%%%%%%%%%
% \paragraph{Chapter Include Files.}
%
% The include files are called |cdocsch1.tex| and |cdocsch2.tex|.
%
%\iffalse
%<*samplechap1|samplechap2>
%\fi

% Optional override for |\version| flag:
%    \begin{macrocode}
%%\providecommand{\version}{final}
%    \end{macrocode}

% Include the main document:
%    \begin{macrocode}
\input{childdoc.def}
\childdocof{cdocsamp}
%    \end{macrocode}

%\iffalse
%</samplechap1|samplechap2>
%\fi
%
%\iffalse
%<*samplechap1>
%\fi
% Some text for chapter 1:
%    \begin{macrocode}
\section{one}
some text in chapter one
%    \end{macrocode}

%\iffalse
%</samplechap1>
%\fi
% Some text for chapter 2:
%\iffalse
%<*samplechap2>
%\fi
%    \begin{macrocode}
\section{two}
more text in chapter two
%    \end{macrocode}

%\iffalse
%</samplechap2>
%\fi
%
% %%%%%%%%%%%%%%%%%%%%%%%%%%%%%%%%%%%%%%
% \paragraph{Part Include Files.}
%
% The include files are called |cdocspt3.tex| and |cdocspt4.tex|.
%
%\iffalse
%<*samplepart3|samplepart4>
%\fi

% Optional override for |\version| flag:
%    \begin{macrocode}
%%\providecommand{\version}{final}
%    \end{macrocode}

% Include the main document:
%    \begin{macrocode}
\input{childdoc.def}
\childdocby{cdocsamp}
%    \end{macrocode}

%\iffalse
%</samplepart3|samplepart4>
%\fi
%
%\iffalse
%<*samplepart3>
%\fi
% Some text for part 3:
%    \begin{macrocode}
some text in part three
%    \end{macrocode}

%\iffalse
%</samplepart3>
%\fi
% Some text for part 4:
%\iffalse
%<*samplepart4>
%\fi
%    \begin{macrocode}
more text in part four
%    \end{macrocode}

%\iffalse
%</samplepart4>
%\fi
%
% %%%%%%%%%%%%%%%%%%%%%%%%%%%%%%%%%%%%%%
% \paragraph{Forwarding for a Complete Draft.}
%
% The following forwarding file |cdocsdrf.tex|
% compiles the main document in draft mode:
%\iffalse
%<*sampledraft>
%\fi
%    \begin{macrocode}
\def\version{draft}
\input{childdoc.def}
\childdocforward{cdocsamp}
%    \end{macrocode}

%\iffalse
%</sampledraft>
%\fi
%
% %%%%%%%%%%%%%%%%%%%%%%%%%%%%%%%%%%%%%%
% \paragraph{Forwarding for Final Version of the Chapters.}
%
% The following forwarding files |cdocsfn1.tex| and |cdocsfn2.tex|
% (with identical content)
% compile the final versions of the child documents
% |cdocsch1.tex| and |cdocsch2.tex|, respectively:
%\iffalse
%<*samplefinal>
%\fi
%    \begin{macrocode}
\def\version{final}
\input{childdoc.def}
\childdocforwardprefix[cdocsamp]{cdocsfn}{cdocsch}
%    \end{macrocode}

%\iffalse
%</samplefinal>
%\fi
%
% %%%%%%%%%%%%%%%%%%%%%%%%%%%%%%%%%%%%%%
% \paragraph{Command Line Processing.}
%
% The following three command lines generate the output files
% |cdocscld|, |cdocscl1| and |cdocscl2|
% which should be identical to
% |cdocsdrf|, |cdocsch1| and |cdocsfn2|, respectively:
% \begin{center}
% \begin{tabular}{l}
% |latex -jobname cdocscld \|\\
% |  "\def\version{draft}\input{childdoc.def}\childdocforward{cdocsamp}"|\\
% |latex -jobname cdocscl1 \|\\
% |  "\input{childdoc.def}\childdocforward[cdocsamp]{cdocsch1}"|\\
% |latex -jobname cdocscl2 \|\\
% |  "\def\version{final}\input{childdoc.def}\childdocforward{cdocsch2}"|
% \end{tabular}
% \end{center}
% Note that the trailing backslash on each first line
% merely continues the input to the second line
% (for convenient cut ant paste).
% Furthermore, the command |latex| can be replaced by any
% of its alternative versions such as |pdflatex|.
%
% %%%%%%%%%%%%%%%%%%%%%%%%%%%%%%%%%%%%%%%%%%%%%%%%%%%%%%%%%%%%%%%%%%%%%%%%%%%%%%
% %%%%%%%%%%%%%%%%%%%%%%%%%%%%%%%%%%%%%%%%%%%%%%%%%%%%%%%%%%%%%%%%%%%%%%%%%%%%%%
% \section{Implementation}
%\iffalse
%<*package>
%\fi
%
% This section describes the definitions file |childdoc.def|.

% The definitions cannot be loaded using |\usepackage| or |\RequirePackage|
% which has a mechanism to prevent loading a style file more than once.
% When loading the definitions by means of |\input|
% multiple instances have to be prevented manually:
%\iffalse
%This code needs to be before the `\ProvidesFile' directive
%which is defined at the beginning of this file.
%Therefore it is also placed there and commented out here.
%</package>
%<*discard>
%\fi
%    \begin{macrocode}
\ifdefined\childdocmain\endinput\fi
%    \end{macrocode}
%\iffalse
%</discard>
%<*package>
%\fi
%
% \macro{\ifchilddoc}
% \macro{\ifchilddocmanual}
% The conditional |\ifchilddoc| tells whether a
% child (true) or main (false) document is being compiled.
% The conditional |\ifchilddocmanual| tells whether
% the |\includeonly| mechanism is used (false) or
% the selection of child files must be performed manually (true).
% The definitions initialise to false:
%    \begin{macrocode}
\newif\ifchilddoc
\newif\ifchilddocmanual
%    \end{macrocode}

% \macro{\childdocname}
% \macro{\childdocjob}
% The macro |\childdocname| stores the name of the main document
% to be compiled. The macro |\childdocjob| stores the name of
% the document on which the \LaTeX{} compiler was originally invoked.
% The content of |\jobname| cannot be compared
% to filenames specified in the source due to different catcodes.
% The following code rescans |\jobname|, stores the result
% in |\childdocname| and saves a copy in |\childdocjob|:
%    \begin{macrocode}
\edef\childdocname{\scantokens\expandafter{\jobname\noexpand}}
\let\childdocjob\childdocname
%    \end{macrocode}

% \macro{\childdocdisable}
% The macro |\childdocdisable| prevents the main file
% from being processed more than once.
% At this stage, the main document command |\childdocmain|
% is assumed to be called once again where it should do nothing.
% Any subsequent call to it should prevent
% a secondary processing of the main document
% It overwrites the forwarding commands
% |\childdocof| and |\childdocforward|
% with empty macros to prevent further inclusions of the main document:
%    \begin{macrocode}
\newcommand{\childdocdisable}
{
  \renewcommand{\childdocmain}[1]{\renewcommand{\childdocmain}[1]{\endinput}}
  \renewcommand{\childdocof}[1]{}
  \renewcommand{\childdocby}[2][]{}
  \renewcommand{\childdocforward}[2][]{}
  \renewcommand{\childdocdisable}{}
}
%    \end{macrocode}

% \macro{\childdocmain}
% The macro |\childdocmain| is to be called at the top of the main file
% with nothing or the main filename (without extension) as argument.
% First, it breaks loops.
% If the argument is not empty and does not match |\childdocname|
% (which is set by the first inclusion of |childdoc.def|),
% |\ifchilddoc| is set to true, |\includeonly| is applied to the child file
% and |\jobname| is set to the main file
% (for proper handling of |.aux| files):
%    \begin{macrocode}
\newcommand{\childdocmain}[1]
{
  \childdocdisable\childdocmain{}
  \if?#1?\else
    \begingroup
      \def\childdoctmp{#1}
      \ifx\childdoctmp\childdocname
        \def\childdoctmp{}
      \else
        \def\childdoctmp
        {
          \childdoctrue
          \includeonly{\childdocname}
          \def\childdocjob{#1}
          \def\jobname{#1}
        }
      \fi
      \expandafter
    \endgroup
    \childdoctmp
  \fi
}
%    \end{macrocode}

% \macro{\childdocof}
% The command |\childdocof| redirects
% compilation to the main file |#1|.
%    \begin{macrocode}
\newcommand{\childdocof}[1]
{
  \childdocdisable
  \childdoctrue
  \includeonly{\childdocname}
  \def\jobname{#1}
  \def\childdocjob{#1}
  \input{#1}
}
%    \end{macrocode}

% \macro{\childdocby}
% The command |\childdocby| ....
%    \begin{macrocode}
\newcommand{\childdocby}[2][]
{
  \childdocdisable
  \childdoctrue
  \childdocmanualtrue
  \if?#1?\else
    \def\jobname{#2}
  \fi
  \def\childdocjob{#2}
  \input{#2}
  \endinput
}
%    \end{macrocode}

% \macro{\childdocforward}
% The command |\childdocforward| redirects
% compilation to the main file or
% (if the optional argument is given) a child file.
% Parameters are set as if the main file
% or a child file starting with |\childdocof| was compiled.
% Then compilation is handed over to the main file:
%    \begin{macrocode}
\newcommand{\childdocforward}[2][]
{
  \begingroup
    \if?#1?
      \def\childdoctmp
      {
        \def\childdocname{#2}
        \def\childdocjob{#2}
        \def\jobname{#2}
        \input{#2}
        \endinput
      }
    \else
      \def\childdoctmp
      {
        \childdocdisable
        \def\childdocname{#2}
        \childdoctrue
        \includeonly{#2}
        \def\childdocjob{#1}
        \def\jobname{#1}
        \input{#1}
        \endinput
      }
    \fi
    \expandafter
  \endgroup
  \childdoctmp
}
%    \end{macrocode}

% \macro{\childdocforwardprefix}
% The command |\childdocforwardprefix| redirects
% compilation to the main or a child file by means of a pattern.
% The prefix |#1| in the current filename is replaced by |#2|
% and the suffix of the current filename is kept
% (it is assumed that the filename does not contain the substring `|~~~|'
% which is used as a delimiter).
% Compilation is handed over to the new file by |\childdocforward|:
%    \begin{macrocode}
\newcommand{\childdocforwardprefix}[3][]
{
  \begingroup
    \def\childdocextract #2##1~~~{\def\childdoctmp{\childdocforward[#1]{#3##1}}}
    \expandafter\childdocextract\childdocname~~~
    \expandafter
  \endgroup
  \childdoctmp
}
%    \end{macrocode}

% \macro{\childdoc}
% The deprecated macro |\childdoc| is a legacy version of |\childdocmain|:
%    \begin{macrocode}
\newcommand{\childdoc}{\childdocmain}
%    \end{macrocode}

% \macro{\childdocredirect}
% The deprecated macro |\childdocredirect| is a legacy version
% of |\childdocforward| and |\childdocforwardprefix|:
%    \begin{macrocode}
\newcommand{\childdocredirect}[2][]
{
  \begingroup
    \if?#1?
      \def\childdoctmp{\childdocforward{#2}}
    \else
      \def\childdoctmp{\childdocforwardprefix{#1}{#2}}
    \fi
    \expandafter
  \endgroup
  \childdoctmp
}
%    \end{macrocode}

%\iffalse
%</package>
%\fi
%
\endinput
|\\
|\childdocforward{|\textit{main}|}|\\
\end{tabular}
\end{center}
%
or alternatively with:
%
\begin{center}
\begin{tabular}{l}
|% \iffalse
%
% childdoc.dtx Copyright (C) 2017-2018 Niklas Beisert
%
% This work may be distributed and/or modified under the
% conditions of the LaTeX Project Public License, either version 1.3
% of this license or (at your option) any later version.
% The latest version of this license is in
%   http://www.latex-project.org/lppl.txt
% and version 1.3 or later is part of all distributions of LaTeX
% version 2005/12/01 or later.
%
% This work has the LPPL maintenance status `maintained'.
%
% The Current Maintainer of this work is Niklas Beisert.
%
% This work consists of the files childdoc.dtx and childdoc.ins
% and the derived files childdoc.def and cdocsamp.tex with
% cdocsch1.tex, cdocsch2.tex, cdocsdrf.tex, cdocsfn1.tex, cdocsfn2.tex.
%
%<package>\ifdefined\childdocmain\endinput\fi
%<package>\ProvidesFile{childdoc.def}[2018/12/30 v2.0 child document driver]
%<samplemain>\ProvidesFile{cdocsamp.tex}[2018/12/30 v2.0 sample for childdoc]
%<*driver>
%\ProvidesFile{childdoc.drv}[2018/12/30 v2.0 childdoc reference manual file]
\PassOptionsToClass{10pt,a4paper}{article}
\documentclass{ltxdoc}

\usepackage[margin=35mm]{geometry}
\usepackage{hyperref}
\usepackage{hyperxmp}
\usepackage[usenames]{color}

\hypersetup{colorlinks=true}
\hypersetup{pdfstartview=FitH}
\hypersetup{pdfpagemode=UseNone}
\hypersetup{pdfsource={}}
\hypersetup{pdflang={en-UK}}
\hypersetup{pdfcopyright={Copyright 2017-2018 Niklas Beisert.
  This work may be distributed and/or modified under the
  conditions of the LaTeX Project Public License, either version 1.3
  of this license or (at your option) any later version.}}
\hypersetup{pdflicenseurl={http://www.latex-project.org/lppl.txt}}
\hypersetup{pdfcontactaddress={ETH Zurich, ITP, HIT K,
  Wolfgang-Pauli-Strasse 27}}
\hypersetup{pdfcontactpostcode={8093}}
\hypersetup{pdfcontactcity={Zurich}}
\hypersetup{pdfcontactcountry={Switzerland}}
\hypersetup{pdfcontactemail={nbeisert@itp.phys.ethz.ch}}
\hypersetup{pdfcontacturl={http://people.phys.ethz.ch/\xmptilde nbeisert/}}

\newcommand{\secref}[1]{\hyperref[#1]{section \ref*{#1}}}

\parskip1ex
\parindent0pt
\let\olditemize\itemize
\def\itemize{\olditemize\parskip0pt}

\begin{document}

\title{The \textsf{childdoc} Package}
\hypersetup{pdftitle={The childdoc Package}}
\author{Niklas Beisert\\[2ex]
  Institut f\"ur Theoretische Physik\\
  Eidgen\"ossische Technische Hochschule Z\"urich\\
  Wolfgang-Pauli-Strasse 27, 8093 Z\"urich, Switzerland\\[1ex]
  \href{mailto:nbeisert@itp.phys.ethz.ch}
  {\texttt{nbeisert@itp.phys.ethz.ch}}}
\hypersetup{pdfauthor={Niklas Beisert}}
\hypersetup{pdfsubject={Manual for the LaTeX2e Package childdoc}}
\date{30 December 2018, \textsf{v2.0}}
\maketitle

\begin{abstract}\noindent
\textsf{childdoc} is a \LaTeXe{} package
that enables the direct compilation
of document sections included by |\include|
to individual files.
\end{abstract}

\begingroup
\parskip0ex
\tableofcontents
\endgroup

%%%%%%%%%%%%%%%%%%%%%%%%%%%%%%%%%%%%%%%%%%%%%%%%%%%%%%%%%%%%%%%%%%%%%%%%%%%%%%%%
%%%%%%%%%%%%%%%%%%%%%%%%%%%%%%%%%%%%%%%%%%%%%%%%%%%%%%%%%%%%%%%%%%%%%%%%%%%%%%%%
\section{Introduction}

\LaTeX{} provides a mechanism to structure a large document (such as a book)
into a main file and several child files (containing the chapters)
using the |\include| command.
This mechanism is beneficial for documents
which span hundreds of pages in order to
make the source file(s) more manageable.
Moreover, compilation can be restricted to
selected child files by means of the |\includeonly| command.
The latter feature can be used to reduce the compilation time while editing
(this was significantly more useful in the earlier days of \LaTeX{})
or to generate a smaller document which is easier to navigate.
Another application of |\includeonly| is to generate
documents consisting of selected parts of the complete document.

However, there are a few drawbacks of the plain |\include| mechanism:
\begin{itemize}
\item
The child files cannot be compiled on their own,
they can only be compiled via the main file.
A naive editing environment
(such as a text editor with an option
to have the current file processed by \LaTeX)
may require one to switch to the main file before compiling;
attempting to compile the child file produces errors.
\item
The main file must be modified (each time)
to adjust the |\includeonly| command
to the present needs. This easily leaves the main file in a messy state.
\item
The generated document will always carry the filename
of the main document. This is inconvenient if
several child files are to be compiled and
to be kept for distribution.
\end{itemize}

The present package provides a simple interface
to make child files individually compilable by \LaTeX{}.
Compiling a child file then has the same effect as compiling
the main file with an |\includeonly| command
to select the appropriate child.
Moreover the generated document will carry the name of the child
rather than the main file.
This resolves all three above issues.

This feature is meant to make the editing of books,
thesis documents and lecture notes somewhat more convenient.
However, the package can also be used efficiently for
composing a series of documents (such as exercise sheets)
which are typically distributed individually.
It then assists the author in generating the individual documents
(potentially in different versions)
as well as a document containing the collected series.
Another application is in developing style files
or other kinds of included material
where compilation of the style file could redirect
to a sample or test file.

%%%%%%%%%%%%%%%%%%%%%%%%%%%%%%%%%%%%%%%%%%%%%%%%%%%%%%%%%%%%%%%%%%%%%%%%%%%%%%%%
%%%%%%%%%%%%%%%%%%%%%%%%%%%%%%%%%%%%%%%%%%%%%%%%%%%%%%%%%%%%%%%%%%%%%%%%%%%%%%%%
\section{Usage}

First of all, the package \textsf{childdoc} is \emph{not} a standard
\LaTeXe{} |.sty| style file! Therefore it needs to be invoked in
a non-standard way.

%%%%%%%%%%%%%%%%%%%%%%%%%%%%%%%%%%%%%%%%%%%%%%%%%%%%%%%%%%%%%%%%%%%%%%%%%%%%%%%%
\subsection{Included Files}
\label{sec:include}

%%%%%%%%%%%%%%%%%%%%%%%%%%%%%%%%%%%%%%%%
\DescribeMacro{\childdocmain}
To use the package, add the commands
\begin{center}
\begin{tabular}{l}
|\input{childdoc.def}|\\
|\childdocmain{}|\\
\end{tabular}
\end{center}
at the very top of the main \LaTeX{} file,
in particular \emph{before} the |\documentclass| statement!
The argument of |\childdocmain| should be left empty
(but it must be present).

%%%%%%%%%%%%%%%%%%%%%%%%%%%%%%%%%%%%%%%%
\DescribeMacro{\childdocof}
Furthermore, add the commands
\begin{center}
\begin{tabular}{l}
|\input{childdoc.def}|\\
|\childdocof{|\textit{main}|}|\\
\end{tabular}
\end{center}
at the top of every child file \textit{child}
which is included by |\include{|\textit{child}|}|
from within the main file
(or at least for those files to be compiled individually).
The argument \textit{main} must be the filename of the main file.

There are a couple of
considerations in setting up the main and child documents:

%%%%%%%%%%%%%%%%%%%%%%%%%%%%%%%%%%%%%%%%
\paragraph{Restrictions.}

Please note the following restrictions:
\begin{itemize}
\item
|\childdocmain| must be called with one argument \textit{main}
to ensure compatibility with earlier version of the package.
It must either be empty (|\childdocmain{}|)
or precisely match the filename of the main file in which it is specified.
See \secref{sec:detection} for further information.
\item
The filename \textit{main} must be specified without the |.tex| extension.
\item
The filename \textit{main} is case sensitive
(even in case-insensitive file systems)
due to internal string comparison.
\item
The argument \textit{main} should be fully expanded, it cannot be a macro.
\item
Subdirectories and special characters should be avoided in filenames.
\item
The command |\childdocmain{|\textit{main}|}| must be followed by a whitespace.
It should not be followed immediately by another command
or by a comment mark `|%|'.
This is because the \TeX{} parser reads the token immediately following
the argument of |\childdocmain| and puts it
at the beginning of every child section;
however, a white\-space is ignored.
\end{itemize}

%%%%%%%%%%%%%%%%%%%%%%%%%%%%%%%%%%%%%%%%
\paragraph{Content of Main File.}

It is advisable to place all content in the child files included by |\include|.
Any output contained in the main file will appear in all child documents
unless suppressed manually;
it cannot be suppressed automatically by the |\includeonly| directive
and thus should normally be avoided.
A method to include some content in the main file
by means of conditional processing is described in \secref{sec:conditional}.

%%%%%%%%%%%%%%%%%%%%%%%%%%%%%%%%%%%%%%%%
\paragraph{Page Numbering.}

When only a part of the document is compiled,
the appropriate numbering of pages
(as well as other status parameters)
is determined from the |.aux| files.
The latter contain information from previous passes.
However this information needs to propagate through
all intermediate child documents.
Therefore the page numbering in child documents may well
be inconsistent until the complete document is compiled at least once.

A useful (if unconventional) way to always ensure a consistent
page numbering is to restart the numbering in each child document
and denote the pages by `\textit{child}|.|\textit{page}'
where \textit{child} represents the chapter/section number of the child file.
This can be achieved by the command
|\numberwithin{page}{|\textit{child}|}|
of the \textsf{amsmath} package
where \textit{child} can be |chapter| or |section|
depending on the chosen structuring.
Alternatively, one can modify the macro |\thepage| appropriately
and reset the counter |page| at the start of each child file.

%%%%%%%%%%%%%%%%%%%%%%%%%%%%%%%%%%%%%%%%%%%%%%%%%%%%%%%%%%%%%%%%%%%%%%%%%%%%%%%%
\subsection{Conditional Processing}
\label{sec:conditional}

The package provides a mechanism to compile different versions
of a document. To customise the versions further some conditional processing
can come in handy to distinguish which version is being compiled.
The package provides two macros to describe the compilation context:

%%%%%%%%%%%%%%%%%%%%%%%%%%%%%%%%%%%%%%%%
\DescribeMacro{\ifchilddoc}
The conditional |\ifchilddoc| distinguishes between the compilation of
child documents and the main document:
%
\begin{center}
|\ifchilddoc |\textit{child-code}| |[|\||else |\textit{main-code}]| \||fi|
\end{center}

%%%%%%%%%%%%%%%%%%%%%%%%%%%%%%%%%%%%%%%%
\DescribeMacro{\childdocname}
\DescribeMacro{\childdocjob}
The macro |\childdocname| contains the filename (without extension)
of the main or child file being processed.
Note that |\childdocjob| will always contain the name of the main file.

%%%%%%%%%%%%%%%%%%%%%%%%%%%%%%%%%%%%%%%%
\paragraph{Title Page.}

Conditional processing can be used to include a title or banner page
in the main document when proper precautions are taken.
Importantly, the code in the main file should ensure that the page counter
(as well as other status parameters which are stored in the |.aux| files)
takes the same value after the conditional processing.
Otherwise the page numbers may take divergent values
depending on which part is compiled.

For example, a title page could be declared by:
%
\begin{center}
\begin{tabular}{l}
|\ifchilddoc\||else|\\
|\addtocounter{page}{-1}|\\
\textit{code for title page}\\
|\newpage|\\
|\||fi|
\end{tabular}
\end{center}
%
A banner page for the child documents can be generated by:
%
\begin{center}
\begin{tabular}{l}
|\ifchilddoc|\\
|\addtocounter{page}{-1}|\\
\textit{code for banner page}\\
|\newpage|\\
|\||fi|
\end{tabular}
\end{center}
%
Here one could write a message such as:
\begin{center}
|This is the part \childdocname{} of \childdocjob{}.|
\end{center}

%%%%%%%%%%%%%%%%%%%%%%%%%%%%%%%%%%%%%%%%%%%%%%%%%%%%%%%%%%%%%%%%%%%%%%%%%%%%%%%%
\subsection{Flags}
\label{sec:flags}

The package makes it easy to generate different versions
of the main or child documents.
To this end compilation flags can be defined
and assigned different default values.
They will be particularly useful in conjunction
with the forwarding mechanism described in \secref{sec:forward}.

For example, it may be useful to have a flag |\version|
which can be set to |draft| or |final|.
The document source will contain some conditional code
depending on the value of |\version|.
Suppose further, the flag should default to |final| for the main file
and to |draft| for child files
which is a natural assignment for editing the document.
This is achieved by placing the following code
in the preamble of the main document
(below the |\childdocmain| directive):
%
\begin{center}
\begin{tabular}{l}
|\ifchilddoc|\\
|\providecommand{\version}{draft}|\\
|\||else|\\
|\providecommand{\version}{final}|\\
|\||fi|
\end{tabular}
\end{center}
%
The definition by |\providecommand| makes sure
that previous definitions are not overwritten.
Further statements |\providecommand{\version}{...}|
can thus be added before the above code to override it.

For the main file, one might add a line
(between |\childdocmain| and the above block)
%
\begin{center}
|%\ifchilddoc\||else\providecommand{\version}{draft}\||fi|
\end{center}
%
which can be uncommented to produce a draft version.
Likewise one can add a line to the very top of a child file
(above the |\childdocof{|\textit{main}|}| directive)
%
\begin{center}
|%\providecommand{\version}{final}|
\end{center}
%
which can be uncommented to produce the final version of this child document.

%%%%%%%%%%%%%%%%%%%%%%%%%%%%%%%%%%%%%%%%%%%%%%%%%%%%%%%%%%%%%%%%%%%%%%%%%%%%%%%%
\subsection{Forwarding}
\label{sec:forward}

Different versions of the main or child documents
using compilation flags as described in \secref{sec:flags}
can be (permanently) stored in different files
for convenient compilation, viewing and distribution.
To this end, the package defines a command
to pass on compilation to a different file:

%%%%%%%%%%%%%%%%%%%%%%%%%%%%%%%%%%%%%%%%
\DescribeMacro{\childdocforward}
The command |\childdocforward| redirects processing to
another source file:
%
\begin{center}
\begin{tabular}{l}
|\input{childdoc.def}|\\
|\childdocforward[|\textit{main}|]{|\textit{dest}|}|\\
\end{tabular}
\end{center}
%
The argument \textit{dest} is the destination file
(without extension).
It should be the main file or one of the child files.
Note that further \textsf{childdoc} directives
such as |\childdocof| and |\childdocforward|
in the indicated file will be processed in this form.
The optional argument \textit{main}
passes on directly to the main file \textit{main}
while pretending to compile the child \textit{dest}.
This form behaves as if \textit{dest}
issues |\childdocof{|\textit{main}|}| right away,
and no further \textsf{childdoc} directives will be processed.

%%%%%%%%%%%%%%%%%%%%%%%%%%%%%%%%%%%%%%%%
\DescribeMacro{\...prefix}
In the alternative form |\childdocforwardprefix|,
%
\begin{center}
\begin{tabular}{l}
|\input{childdoc.def}|\\
|\childdocforwardprefix[|\textit{main}|]{|\textit{prefix}|}{|\textit{dest}|}|
\end{tabular}
\end{center}
%
the destination file is determined by a pattern
depending on the current file:
To make this work, the current file must be called
`{\textit{prefix}\hspace{0.2em}\textit{suffix}}'
with \textit{prefix} matching precisely the argument.
Processing is then passed on to the file
`{\textit{dest}\hspace{0.2em}\textit{suffix}}'.
Surely, the same effect is achieved by
directly specifying the
argument `{\textit{dest}\hspace{0.2em}\textit{suffix}}'
in the first form.
However, that requires to set up a different file
for each child. With the alternative form of the command
all these files can have exactly the same content
which simplifies setting them up and maintaining them.

For example, the following file |draft.tex|
with a compilation flag |\version| as described in \secref{sec:flags}
compiles the main document as a draft:
%
\begin{center}
\begin{tabular}{l}
|\def\version{draft}|\\
|\input{childdoc.def}|\\
|\childdocforward{|\textit{main}|}|
\end{tabular}
\end{center}
%
Likewise, the following files |final|\textit{nn}|.tex|
compile the final version of the child document
|child|\textit{nn}|.tex|:
%
\begin{center}
\begin{tabular}{l}
|\def\version{final}|\\
|\input{childdoc.def}|\\
|\childdocforwardprefix{final}{child}|
\end{tabular}
\end{center}
%

Note that when several versions of a main file and/or of each child file
are to be generated, it may be convenient to set up a |Makefile| or
shell script to automatise the process.

%%%%%%%%%%%%%%%%%%%%%%%%%%%%%%%%%%%%%%%%%%%%%%%%%%%%%%%%%%%%%%%%%%%%%%%%%%%%%%%%
\subsection{Command Line Processing}
\label{sec:commandline}

The effect of redirection files can also be achieved by invoking
the \LaTeX{} compiler with a more elaborate command line.
Most conveniently this should be done as part
of a shell script or a |Makefile|.

When using \textsf{childdoc} in the main file, the following
command lines effectively perform a redirection
(note that depending on the shell being used,
backslashes may have to be doubled: `|\|' $\to$ `|\\|'):
%
\begin{center}
|... -jobname "|\textit{target}|" |\\|"|[\textit{flags}]%
|\input{childdoc.def}\childdocforward[|\textit{main}|]{|\textit{dest}|}"|
\end{center}
%
Here \textit{target} is the name of the output file,
\textit{main} is the name of the main file
and \textit{dest} is the name of the main or child file to be processed
(all filenames without extensions).
The optional argument \textit{main} can be omitted
if \textit{main} matches \textit{dest}.
Optionally, compilation \textit{flags} can be defined via |\def| commands.
This command line makes the \TeX{} engine believe
it is compiling the file \textit{target}
whose content is specified as the latter parameter.
The provided code then forwards the processing to
\textit{main} or \textit{dest} as described in \secref{sec:forward}.

%%%%%%%%%%%%%%%%%%%%%%%%%%%%%%%%%%%%%%%%%%%%%%%%%%%%%%%%%%%%%%%%%%%%%%%%%%%%%%%%
\subsection{Include by Input}
\label{sec:input}

Including child documents by |\include| has some restrictions by design.
Most notably, the content of a child document always occupies
its own set of pages; pages cannot be shared between child documents.
Usually, this behaviour makes perfect sense
because each child document contain an essential part of the document.
However, in some situations it may be desirable to compose
a document from a collection of parts
without having mandatory page breaks between then.
For this case, the package
provides a mechanism to include parts
by |\input| which can also be processed individually.
However, by construction this mechanism
requires manual handling of the content to be output.

%%%%%%%%%%%%%%%%%%%%%%%%%%%%%%%%%%%%%%%%
\DescribeMacro{\ifchilddocmanual}
The main file should be prepared as usual, see \secref{sec:include}.
However, the document body must make a distinction
between processing of an individual part and of the main document, e.g.:
%
\begin{center}
\begin{tabular}{l}
|\ifchilddocmanual|\\
|\input{\childdocname}|\\
|\||else|\\
\textit{document body with }|\input{|\textit{part}|}|\\
|\||fi|
\end{tabular}
\end{center}
%
The conditional |\ifchilddocmanual| is true whenever
a part to be included by |\input| is being compiled,
and the name of the part is stored in |\childdocname|.

%%%%%%%%%%%%%%%%%%%%%%%%%%%%%%%%%%%%%%%%
\DescribeMacro{\childdocby}
Each part to be included by |\input| should start with:
%
\begin{center}
\begin{tabular}{l}
|\input{childdoc.def}|\\
|\childdocby{|\textit{main}|}|\\
\end{tabular}
\end{center}
%
The directive |\childdocby| is similar to |\childdocof|
described in \secref{sec:include},
but the subsequent selection of content must be done manually.
To that end, both |\ifchilddoc| and |\ifchilddocmanual|
will be true upon processing of a part,
and the name of the part is stored in |\childdocname|.
Note that |\jobname| will be set to the filename of the current part
so that each part receives an individual |.aux| file
that does not interfere with the |.aux| file(s) of the main document.
This behaviour can be altered by the alternative form
|\childdocby[*]{|\textit{main}|}| (with a non-empty optional argument)
which uses the |.aux| file of the main document
by setting |\jobname| to \textit{main}.

%%%%%%%%%%%%%%%%%%%%%%%%%%%%%%%%%%%%%%%%%%%%%%%%%%%%%%%%%%%%%%%%%%%%%%%%%%%%%%%%
\subsection{Driver Development}
\label{sec:driver}

The \textsf{childdoc} mechanism can also be use for the development
of definition files such as \LaTeX{} styles or classes.
This case differs from the above setup with multiple parts
included by |\include| in that no |\includeonly| should be invoked.
This can be achieved by starting the include file
(before |\ProvidesPackage|) with:
%
\begin{center}
\begin{tabular}{l}
|\input{childdoc.def}|\\
|\childdocforward{|\textit{main}|}|\\
\end{tabular}
\end{center}
%
or alternatively with:
%
\begin{center}
\begin{tabular}{l}
|\input{childdoc.def}|\\
|\childdocby{|\textit{main}|}|\\
\end{tabular}
\end{center}
%
Both forms have slightly different effects as described above.
The main file is prepared as usual, see \secref{sec:include}.

%%%%%%%%%%%%%%%%%%%%%%%%%%%%%%%%%%%%%%%%%%%%%%%%%%%%%%%%%%%%%%%%%%%%%%%%%%%%%%%%
\subsection{Legacy Detection}
\label{sec:detection}

The directive |\childdocmain| in the main file can detect
whether the complete document or merely a child is to be compiled
even without using the directive |\childdocof|.
This method is deprecated because it is less robust
and there is no compelling reason to use it;
it is merely provided for backward compatibility
and it may be removed in future versions.

If the detection mechanism is to be used,
it is mandatory to correctly specify
the filename of the main file as the argument of |\childdocmain|:
%
\begin{center}
\begin{tabular}{l}
|\input{childdoc.def}|\\
|\childdocmain{|\textit{main}|}|\\
\end{tabular}
\end{center}
%
If |\jobname| does not match the argument \textit{main} of |\childdocmain|,
it is assumed that |\jobname| points to the child file to be compiled.
When using |\childdocmain| with the main file specified as argument,
it suffices to start a child file
with just |\input{|\textit{main}|}|
without loading of the package and using |\childdocof|.
If instead all processing is done
with the appropriate \textsf{childdoc} directives,
the argument of \textit{main} of |\childdocmain| can be empty.

An alternative version of the command line processing described
in \secref{sec:commandline} using the detection mechanism reads:
%
\begin{center}
|... -jobname "|\textit{target}|" "|[\textit{flags}]%
[|\def\jobname{|\textit{dest}|}|]|\input{|\textit{main}|}"|
\end{center}

%%%%%%%%%%%%%%%%%%%%%%%%%%%%%%%%%%%%%%%%%%%%%%%%%%%%%%%%%%%%%%%%%%%%%%%%%%%%%%%%
\subsection{Manual Code}
\label{sec:manual}

In case one cannot be certain whether the definitions file |childdoc.def|
is installed on the target \TeX{} distribution
and one prefers not to ship it,
it is conceivable to paste a few relevant commands into the sources.

To that end, drop all statements |\input{childdoc.def}|
and perform the replacements as outlined below.
Instead of |\childdocmain{|\textit{main}|}| add the following code
to the top of the main file:
%
\begin{center}
\begin{tabular}{l}
|\||ifdefined\childdocname\endinput\||fi\newif\ifchilddoc|\\
|\edef\childdocname{\scantokens\expandafter{\jobname\noexpand}}|\\
|\def\childdocmain{|\textit{main}|}\||ifx\childdocmain\childdocname\||else|\\
|\childdoctrue\includeonly{\childdocname}\let\jobname\childdocmain\||fi|\\
\end{tabular}
\end{center}
%
Instead of |\childdocof{|\textit{main}|}| just include the main file
at the top of each child file:
%
\begin{center}
|\input{|\textit{main}|}|
\end{center}
%
A simple redirection |\childdocforward{|\textit{dest}|}| is achieved by:
%
\begin{center}
|\def\jobname{|\textit{dest}|}\input{\jobname}|
\end{center}
%
The redirection with prefix
|\childdocforwardprefix[|\textit{prefix}|]{|\textit{dest}|}|
is accomplished by:
%
\begin{center}
\begin{tabular}{l}
|{\edef\jobname{\scantokens\expandafter{\jobname\noexpand}}|\\
|\def\redirectjob |\textit{prefix}|#1~~~{\gdef\jobname{|\textit{dest}|#1}}|\\
|\expandafter\redirectjob\jobname~~~}\input{\jobname}|
\end{tabular}
\end{center}

In an alternative approach,
child documents can be compiled by a specific command line
without additional code or specific definitions:
%
\begin{center}
|... -jobname "|\textit{target}|" "|[\textit{flags}]%
|\includeonly{|\textit{dest}|}\input{|\textit{main}|}"|
\end{center}
%

%%%%%%%%%%%%%%%%%%%%%%%%%%%%%%%%%%%%%%%%%%%%%%%%%%%%%%%%%%%%%%%%%%%%%%%%%%%%%%%%
%%%%%%%%%%%%%%%%%%%%%%%%%%%%%%%%%%%%%%%%%%%%%%%%%%%%%%%%%%%%%%%%%%%%%%%%%%%%%%%%
\section{Information}

%%%%%%%%%%%%%%%%%%%%%%%%%%%%%%%%%%%%%%%%%%%%%%%%%%%%%%%%%%%%%%%%%%%%%%%%%%%%%%%%
\subsection{Copyright}

Copyright \copyright{} 2017--2018 Niklas Beisert

This work may be distributed and/or modified under the
conditions of the \LaTeX{} Project Public License, either version 1.3
of this license or (at your option) any later version.
The latest version of this license is in
  \url{http://www.latex-project.org/lppl.txt}
and version 1.3 or later is part of all distributions of \LaTeX{}
version 2005/12/01 or later.

This work has the LPPL maintenance status `maintained'.

The Current Maintainer of this work is Niklas Beisert.

This work consists of the files |README.txt|, |childdoc.ins| and |childdoc.dtx|
as well as the derived files |childdoc.def|, |cdocsamp.tex|
with |cdocsch1.tex|, |cdocsch2.tex|, |cdocspt3.tex|, |cdocspt4.tex|,
|cdocsdrf.tex|, |cdocsfn1.tex|, |cdocsfn2.tex|
as well as |childdoc.pdf|.

%%%%%%%%%%%%%%%%%%%%%%%%%%%%%%%%%%%%%%%%%%%%%%%%%%%%%%%%%%%%%%%%%%%%%%%%%%%%%%%%
\subsection{Files and Installation}

The package consists of the files:
%
\begin{center}
\begin{tabular}{ll}
    |README.txt|   & readme file \\
    |childdoc.ins| & installation file \\
    |childdoc.dtx| & source file \\
    |childdoc.def| & definition file \\
    |cdocsamp.tex| & sample main file \\
    |cdocsch1.tex| & sample include file \\
    |cdocsch2.tex| & sample include file \\
    |cdocspt3.tex| & sample part file \\
    |cdocspt4.tex| & sample part file \\
    |cdocsdrf.tex| & sample redirection file \\
    |cdocsfn1.tex| & sample redirection file \\
    |cdocsfn2.tex| & sample redirection file \\
    |childdoc.pdf| & manual
\end{tabular}
\end{center}
%
The distribution consists of the files
|README.txt|, |childdoc.ins| and |childdoc.dtx|.
%
\begin{itemize}
\item
Run (pdf)\LaTeX{} on |childdoc.dtx|
to compile the manual |childdoc.pdf| (this file).
\item
Run \LaTeX{} on |childdoc.ins| to create the definitions file |childdoc.def|
and the sample |cdocsamp.tex| with include files
|cdocsch1.tex|, |cdocsch2.tex|, |cdocspt3.tex|, |cdocspt4.tex|,
|cdocsdrf.tex|, |cdocsfn1.tex|, |cdocsfn2.tex|.
Then copy the file |childdoc.def| to an appropriate directory of your \LaTeX{}
distribution, e.g.\ \textit{texmf-root}|/tex/latex/childdoc|.
\end{itemize}

%%%%%%%%%%%%%%%%%%%%%%%%%%%%%%%%%%%%%%%%%%%%%%%%%%%%%%%%%%%%%%%%%%%%%%%%%%%%%%%%
\subsection{Related CTAN Packages}

There are several other packages which offer a similar functionality:
%
\begin{itemize}
\item
The packages
\href{http://ctan.org/pkg/docmute}{\textsf{docmute}},
\href{http://ctan.org/pkg/includex}{\textsf{includex}} and
\href{http://ctan.org/pkg/standalone}{\textsf{standalone}}
provide commands to include only the document body of
a child file thus allowing both files to be compiled individually.
\item
The packages \href{http://ctan.org/pkg/subdocs}{\textsf{subdocs}}
and \href{http://ctan.org/pkg/subfiles}{\textsf{subfiles}}
provide structures in which the main and child documents can be
encapsulated and allowing them to be compiled individually.
The inclusion mechanism is different from the conventional |\include|.
\item
The package \href{http://ctan.org/pkg/combine}{\textsf{combine}}
is an elaborate solution to combine several documents into one.
\end{itemize}
%
See also the CTAN topic \href{http://ctan.org/topic/subdocs}{\textsf{subdocs}}
for further related packages.
The present package differs from the above solutions in that
a document structure constructed with the conventional |\include| mechanism
just needs two extra commands at the top of every file
such that all constituent files can be compiled individually.

%%%%%%%%%%%%%%%%%%%%%%%%%%%%%%%%%%%%%%%%%%%%%%%%%%%%%%%%%%%%%%%%%%%%%%%%%%%%%%%%
%\subsection{Feature Suggestions}
%
%The following is a list of features which may be useful for future
%versions of this package:
%%
%\begin{itemize}
%\item
%\ldots
%\end{itemize}

%%%%%%%%%%%%%%%%%%%%%%%%%%%%%%%%%%%%%%%%%%%%%%%%%%%%%%%%%%%%%%%%%%%%%%%%%%%%%%%%
\subsection{Revision History}

%%%%%%%%%%%%%%%%%%%%%%%%%%%%%%%%%%%%%%%%
\paragraph{v2.0:} 2018/12/30

\begin{itemize}
\item
immediate forward processing
\item
added |\childdocby| mechanism
\item
manual restructured
\end{itemize}

%%%%%%%%%%%%%%%%%%%%%%%%%%%%%%%%%%%%%%%%
\paragraph{v1.6:} 2018/01/17

\begin{itemize}
\item
application for development of include files
\item
corrections to manual
\end{itemize}

%%%%%%%%%%%%%%%%%%%%%%%%%%%%%%%%%%%%%%%%
\paragraph{v1.5:} 2017/05/21

\begin{itemize}
\item
more complete structuring introduced
\item
|\childdocof| introduced
\item
|\childdoc| renamed to |\childdocmain|
\item
|\childredirect| renamed to |\childdocforward| and |\childdocforwardprefix|
and functionality expanded
\end{itemize}

%%%%%%%%%%%%%%%%%%%%%%%%%%%%%%%%%%%%%%%%
\paragraph{v1.0:} 2017/04/27

\begin{itemize}
\item
manual and install package
\item
first version published on CTAN
\end{itemize}

%%%%%%%%%%%%%%%%%%%%%%%%%%%%%%%%%%%%%%%%
\paragraph{v0.6:} 2017/04/26

\begin{itemize}
\item
redirection mechanism added
\end{itemize}

%%%%%%%%%%%%%%%%%%%%%%%%%%%%%%%%%%%%%%%%
\paragraph{v0.5:} 2017/04/26

\begin{itemize}
\item
functionality in definition file
\end{itemize}


%%%%%%%%%%%%%%%%%%%%%%%%%%%%%%%%%%%%%%%%%%%%%%%%%%%%%%%%%%%%%%%%%%%%%%%%%%%%%%%%
%%%%%%%%%%%%%%%%%%%%%%%%%%%%%%%%%%%%%%%%%%%%%%%%%%%%%%%%%%%%%%%%%%%%%%%%%%%%%%%%
%%%%%%%%%%%%%%%%%%%%%%%%%%%%%%%%%%%%%%%%%%%%%%%%%%%%%%%%%%%%%%%%%%%%%%%%%%%%%%%%
\appendix

\settowidth\MacroIndent{\rmfamily\scriptsize 000\ }

 \DocInput{childdoc.dtx}

\end{document}
%</driver>
% \fi
%
% %%%%%%%%%%%%%%%%%%%%%%%%%%%%%%%%%%%%%%%%%%%%%%%%%%%%%%%%%%%%%%%%%%%%%%%%%%%%%%
% %%%%%%%%%%%%%%%%%%%%%%%%%%%%%%%%%%%%%%%%%%%%%%%%%%%%%%%%%%%%%%%%%%%%%%%%%%%%%%
% \section{Sample}
%\iffalse
%<*samplemain>
%\fi
%
% The following presents a sample document
% with two chapters, two parts, a title page,
% a compile flag as well as three forwarding files to set the flag.
% It consists of eight |.tex| files:
% \begin{center}
% \begin{tabular}{ll}
% |cdocsamp.tex|&main file\\
% |cdocsch1.tex|&include file for chapter 1\\
% |cdocsch2.tex|&include file for chapter 2\\
% |cdocspt3.tex|&include file for part 3\\
% |cdocspt4.tex|&include file for part 4\\
% |cdocsdrf.tex|&forwarding file for main file in draft mode\\
% |cdocsfi1.tex|&forwarding file for final version of chapter 1\\
% |cdocsfi2.tex|&forwarding file for final version of chapter 2\\
% \end{tabular}
% \end{center}
% Each of the eight files can be compiled directly by the \LaTeX{} compiler.
%
% %%%%%%%%%%%%%%%%%%%%%%%%%%%%%%%%%%%%%%
% \paragraph{Main File.}
%
% The main file is called |cdocsamp.tex|.
%
% Load the \textsf{childdoc} definitions and
% declare the filename for the main document:
%    \begin{macrocode}
\input{childdoc.def}
\childdocmain{}
%    \end{macrocode}

% Optional override for |\version| flag:
%    \begin{macrocode}
%%\ifchilddoc\else\providecommand{\version}{draft}\fi
%    \end{macrocode}

% Define the default values for the |\version| flag
% (|final| for the main file and |draft| for childs):
%    \begin{macrocode}
\ifchilddoc
\providecommand{\version}{draft}
\else
\providecommand{\version}{final}
\fi
%    \end{macrocode}

% Load the standard document class:
%    \begin{macrocode}
\documentclass[12pt]{article}
%    \end{macrocode}

% Start the document body:
%    \begin{macrocode}
\begin{document}
%    \end{macrocode}

% Declare a title page.
% Print title, part of document being processed and version flag:
%    \begin{macrocode}
\addtocounter{page}{-1}
\begin{center}
{\LARGE\bfseries{}childdoc example\par}
\vspace{1cm}
\ifchilddoc
\ifchilddocmanual part\else chapter\fi:
`\childdocname' of `\childdocjob'\par
\else
main document: `\childdocjob'\par
\fi
version: \version\par
\end{center}
\newpage
%    \end{macrocode}

% Manually include selected file,
% otherwise process as usual:
%    \begin{macrocode}
\ifchilddocmanual
\section*{part `\childdocname'}
\input{\childdocname}
\else
%    \end{macrocode}

% Include the two chapters:
%    \begin{macrocode}
\include{cdocsch1}
\include{cdocsch2}
%    \end{macrocode}

% Include the two parts unless only chapters should be displayed:
%    \begin{macrocode}
\ifchilddoc\else
\section{part three}
\input{cdocspt3}
\section{part four}
\input{cdocspt4}
\fi
%    \end{macrocode}

% Process as usual until here:
%    \begin{macrocode}
\fi
%    \end{macrocode}

% End of document body:
%    \begin{macrocode}
\end{document}
%    \end{macrocode}
%\iffalse
%</samplemain>
%\fi
%
% %%%%%%%%%%%%%%%%%%%%%%%%%%%%%%%%%%%%%%
% \paragraph{Chapter Include Files.}
%
% The include files are called |cdocsch1.tex| and |cdocsch2.tex|.
%
%\iffalse
%<*samplechap1|samplechap2>
%\fi

% Optional override for |\version| flag:
%    \begin{macrocode}
%%\providecommand{\version}{final}
%    \end{macrocode}

% Include the main document:
%    \begin{macrocode}
\input{childdoc.def}
\childdocof{cdocsamp}
%    \end{macrocode}

%\iffalse
%</samplechap1|samplechap2>
%\fi
%
%\iffalse
%<*samplechap1>
%\fi
% Some text for chapter 1:
%    \begin{macrocode}
\section{one}
some text in chapter one
%    \end{macrocode}

%\iffalse
%</samplechap1>
%\fi
% Some text for chapter 2:
%\iffalse
%<*samplechap2>
%\fi
%    \begin{macrocode}
\section{two}
more text in chapter two
%    \end{macrocode}

%\iffalse
%</samplechap2>
%\fi
%
% %%%%%%%%%%%%%%%%%%%%%%%%%%%%%%%%%%%%%%
% \paragraph{Part Include Files.}
%
% The include files are called |cdocspt3.tex| and |cdocspt4.tex|.
%
%\iffalse
%<*samplepart3|samplepart4>
%\fi

% Optional override for |\version| flag:
%    \begin{macrocode}
%%\providecommand{\version}{final}
%    \end{macrocode}

% Include the main document:
%    \begin{macrocode}
\input{childdoc.def}
\childdocby{cdocsamp}
%    \end{macrocode}

%\iffalse
%</samplepart3|samplepart4>
%\fi
%
%\iffalse
%<*samplepart3>
%\fi
% Some text for part 3:
%    \begin{macrocode}
some text in part three
%    \end{macrocode}

%\iffalse
%</samplepart3>
%\fi
% Some text for part 4:
%\iffalse
%<*samplepart4>
%\fi
%    \begin{macrocode}
more text in part four
%    \end{macrocode}

%\iffalse
%</samplepart4>
%\fi
%
% %%%%%%%%%%%%%%%%%%%%%%%%%%%%%%%%%%%%%%
% \paragraph{Forwarding for a Complete Draft.}
%
% The following forwarding file |cdocsdrf.tex|
% compiles the main document in draft mode:
%\iffalse
%<*sampledraft>
%\fi
%    \begin{macrocode}
\def\version{draft}
\input{childdoc.def}
\childdocforward{cdocsamp}
%    \end{macrocode}

%\iffalse
%</sampledraft>
%\fi
%
% %%%%%%%%%%%%%%%%%%%%%%%%%%%%%%%%%%%%%%
% \paragraph{Forwarding for Final Version of the Chapters.}
%
% The following forwarding files |cdocsfn1.tex| and |cdocsfn2.tex|
% (with identical content)
% compile the final versions of the child documents
% |cdocsch1.tex| and |cdocsch2.tex|, respectively:
%\iffalse
%<*samplefinal>
%\fi
%    \begin{macrocode}
\def\version{final}
\input{childdoc.def}
\childdocforwardprefix[cdocsamp]{cdocsfn}{cdocsch}
%    \end{macrocode}

%\iffalse
%</samplefinal>
%\fi
%
% %%%%%%%%%%%%%%%%%%%%%%%%%%%%%%%%%%%%%%
% \paragraph{Command Line Processing.}
%
% The following three command lines generate the output files
% |cdocscld|, |cdocscl1| and |cdocscl2|
% which should be identical to
% |cdocsdrf|, |cdocsch1| and |cdocsfn2|, respectively:
% \begin{center}
% \begin{tabular}{l}
% |latex -jobname cdocscld \|\\
% |  "\def\version{draft}\input{childdoc.def}\childdocforward{cdocsamp}"|\\
% |latex -jobname cdocscl1 \|\\
% |  "\input{childdoc.def}\childdocforward[cdocsamp]{cdocsch1}"|\\
% |latex -jobname cdocscl2 \|\\
% |  "\def\version{final}\input{childdoc.def}\childdocforward{cdocsch2}"|
% \end{tabular}
% \end{center}
% Note that the trailing backslash on each first line
% merely continues the input to the second line
% (for convenient cut ant paste).
% Furthermore, the command |latex| can be replaced by any
% of its alternative versions such as |pdflatex|.
%
% %%%%%%%%%%%%%%%%%%%%%%%%%%%%%%%%%%%%%%%%%%%%%%%%%%%%%%%%%%%%%%%%%%%%%%%%%%%%%%
% %%%%%%%%%%%%%%%%%%%%%%%%%%%%%%%%%%%%%%%%%%%%%%%%%%%%%%%%%%%%%%%%%%%%%%%%%%%%%%
% \section{Implementation}
%\iffalse
%<*package>
%\fi
%
% This section describes the definitions file |childdoc.def|.

% The definitions cannot be loaded using |\usepackage| or |\RequirePackage|
% which has a mechanism to prevent loading a style file more than once.
% When loading the definitions by means of |\input|
% multiple instances have to be prevented manually:
%\iffalse
%This code needs to be before the `\ProvidesFile' directive
%which is defined at the beginning of this file.
%Therefore it is also placed there and commented out here.
%</package>
%<*discard>
%\fi
%    \begin{macrocode}
\ifdefined\childdocmain\endinput\fi
%    \end{macrocode}
%\iffalse
%</discard>
%<*package>
%\fi
%
% \macro{\ifchilddoc}
% \macro{\ifchilddocmanual}
% The conditional |\ifchilddoc| tells whether a
% child (true) or main (false) document is being compiled.
% The conditional |\ifchilddocmanual| tells whether
% the |\includeonly| mechanism is used (false) or
% the selection of child files must be performed manually (true).
% The definitions initialise to false:
%    \begin{macrocode}
\newif\ifchilddoc
\newif\ifchilddocmanual
%    \end{macrocode}

% \macro{\childdocname}
% \macro{\childdocjob}
% The macro |\childdocname| stores the name of the main document
% to be compiled. The macro |\childdocjob| stores the name of
% the document on which the \LaTeX{} compiler was originally invoked.
% The content of |\jobname| cannot be compared
% to filenames specified in the source due to different catcodes.
% The following code rescans |\jobname|, stores the result
% in |\childdocname| and saves a copy in |\childdocjob|:
%    \begin{macrocode}
\edef\childdocname{\scantokens\expandafter{\jobname\noexpand}}
\let\childdocjob\childdocname
%    \end{macrocode}

% \macro{\childdocdisable}
% The macro |\childdocdisable| prevents the main file
% from being processed more than once.
% At this stage, the main document command |\childdocmain|
% is assumed to be called once again where it should do nothing.
% Any subsequent call to it should prevent
% a secondary processing of the main document
% It overwrites the forwarding commands
% |\childdocof| and |\childdocforward|
% with empty macros to prevent further inclusions of the main document:
%    \begin{macrocode}
\newcommand{\childdocdisable}
{
  \renewcommand{\childdocmain}[1]{\renewcommand{\childdocmain}[1]{\endinput}}
  \renewcommand{\childdocof}[1]{}
  \renewcommand{\childdocby}[2][]{}
  \renewcommand{\childdocforward}[2][]{}
  \renewcommand{\childdocdisable}{}
}
%    \end{macrocode}

% \macro{\childdocmain}
% The macro |\childdocmain| is to be called at the top of the main file
% with nothing or the main filename (without extension) as argument.
% First, it breaks loops.
% If the argument is not empty and does not match |\childdocname|
% (which is set by the first inclusion of |childdoc.def|),
% |\ifchilddoc| is set to true, |\includeonly| is applied to the child file
% and |\jobname| is set to the main file
% (for proper handling of |.aux| files):
%    \begin{macrocode}
\newcommand{\childdocmain}[1]
{
  \childdocdisable\childdocmain{}
  \if?#1?\else
    \begingroup
      \def\childdoctmp{#1}
      \ifx\childdoctmp\childdocname
        \def\childdoctmp{}
      \else
        \def\childdoctmp
        {
          \childdoctrue
          \includeonly{\childdocname}
          \def\childdocjob{#1}
          \def\jobname{#1}
        }
      \fi
      \expandafter
    \endgroup
    \childdoctmp
  \fi
}
%    \end{macrocode}

% \macro{\childdocof}
% The command |\childdocof| redirects
% compilation to the main file |#1|.
%    \begin{macrocode}
\newcommand{\childdocof}[1]
{
  \childdocdisable
  \childdoctrue
  \includeonly{\childdocname}
  \def\jobname{#1}
  \def\childdocjob{#1}
  \input{#1}
}
%    \end{macrocode}

% \macro{\childdocby}
% The command |\childdocby| ....
%    \begin{macrocode}
\newcommand{\childdocby}[2][]
{
  \childdocdisable
  \childdoctrue
  \childdocmanualtrue
  \if?#1?\else
    \def\jobname{#2}
  \fi
  \def\childdocjob{#2}
  \input{#2}
  \endinput
}
%    \end{macrocode}

% \macro{\childdocforward}
% The command |\childdocforward| redirects
% compilation to the main file or
% (if the optional argument is given) a child file.
% Parameters are set as if the main file
% or a child file starting with |\childdocof| was compiled.
% Then compilation is handed over to the main file:
%    \begin{macrocode}
\newcommand{\childdocforward}[2][]
{
  \begingroup
    \if?#1?
      \def\childdoctmp
      {
        \def\childdocname{#2}
        \def\childdocjob{#2}
        \def\jobname{#2}
        \input{#2}
        \endinput
      }
    \else
      \def\childdoctmp
      {
        \childdocdisable
        \def\childdocname{#2}
        \childdoctrue
        \includeonly{#2}
        \def\childdocjob{#1}
        \def\jobname{#1}
        \input{#1}
        \endinput
      }
    \fi
    \expandafter
  \endgroup
  \childdoctmp
}
%    \end{macrocode}

% \macro{\childdocforwardprefix}
% The command |\childdocforwardprefix| redirects
% compilation to the main or a child file by means of a pattern.
% The prefix |#1| in the current filename is replaced by |#2|
% and the suffix of the current filename is kept
% (it is assumed that the filename does not contain the substring `|~~~|'
% which is used as a delimiter).
% Compilation is handed over to the new file by |\childdocforward|:
%    \begin{macrocode}
\newcommand{\childdocforwardprefix}[3][]
{
  \begingroup
    \def\childdocextract #2##1~~~{\def\childdoctmp{\childdocforward[#1]{#3##1}}}
    \expandafter\childdocextract\childdocname~~~
    \expandafter
  \endgroup
  \childdoctmp
}
%    \end{macrocode}

% \macro{\childdoc}
% The deprecated macro |\childdoc| is a legacy version of |\childdocmain|:
%    \begin{macrocode}
\newcommand{\childdoc}{\childdocmain}
%    \end{macrocode}

% \macro{\childdocredirect}
% The deprecated macro |\childdocredirect| is a legacy version
% of |\childdocforward| and |\childdocforwardprefix|:
%    \begin{macrocode}
\newcommand{\childdocredirect}[2][]
{
  \begingroup
    \if?#1?
      \def\childdoctmp{\childdocforward{#2}}
    \else
      \def\childdoctmp{\childdocforwardprefix{#1}{#2}}
    \fi
    \expandafter
  \endgroup
  \childdoctmp
}
%    \end{macrocode}

%\iffalse
%</package>
%\fi
%
\endinput
|\\
|\childdocby{|\textit{main}|}|\\
\end{tabular}
\end{center}
%
Both forms have slightly different effects as described above.
The main file is prepared as usual, see \secref{sec:include}.

%%%%%%%%%%%%%%%%%%%%%%%%%%%%%%%%%%%%%%%%%%%%%%%%%%%%%%%%%%%%%%%%%%%%%%%%%%%%%%%%
\subsection{Legacy Detection}
\label{sec:detection}

The directive |\childdocmain| in the main file can detect
whether the complete document or merely a child is to be compiled
even without using the directive |\childdocof|.
This method is deprecated because it is less robust
and there is no compelling reason to use it;
it is merely provided for backward compatibility
and it may be removed in future versions.

If the detection mechanism is to be used,
it is mandatory to correctly specify
the filename of the main file as the argument of |\childdocmain|:
%
\begin{center}
\begin{tabular}{l}
|% \iffalse
%
% childdoc.dtx Copyright (C) 2017-2018 Niklas Beisert
%
% This work may be distributed and/or modified under the
% conditions of the LaTeX Project Public License, either version 1.3
% of this license or (at your option) any later version.
% The latest version of this license is in
%   http://www.latex-project.org/lppl.txt
% and version 1.3 or later is part of all distributions of LaTeX
% version 2005/12/01 or later.
%
% This work has the LPPL maintenance status `maintained'.
%
% The Current Maintainer of this work is Niklas Beisert.
%
% This work consists of the files childdoc.dtx and childdoc.ins
% and the derived files childdoc.def and cdocsamp.tex with
% cdocsch1.tex, cdocsch2.tex, cdocsdrf.tex, cdocsfn1.tex, cdocsfn2.tex.
%
%<package>\ifdefined\childdocmain\endinput\fi
%<package>\ProvidesFile{childdoc.def}[2018/12/30 v2.0 child document driver]
%<samplemain>\ProvidesFile{cdocsamp.tex}[2018/12/30 v2.0 sample for childdoc]
%<*driver>
%\ProvidesFile{childdoc.drv}[2018/12/30 v2.0 childdoc reference manual file]
\PassOptionsToClass{10pt,a4paper}{article}
\documentclass{ltxdoc}

\usepackage[margin=35mm]{geometry}
\usepackage{hyperref}
\usepackage{hyperxmp}
\usepackage[usenames]{color}

\hypersetup{colorlinks=true}
\hypersetup{pdfstartview=FitH}
\hypersetup{pdfpagemode=UseNone}
\hypersetup{pdfsource={}}
\hypersetup{pdflang={en-UK}}
\hypersetup{pdfcopyright={Copyright 2017-2018 Niklas Beisert.
  This work may be distributed and/or modified under the
  conditions of the LaTeX Project Public License, either version 1.3
  of this license or (at your option) any later version.}}
\hypersetup{pdflicenseurl={http://www.latex-project.org/lppl.txt}}
\hypersetup{pdfcontactaddress={ETH Zurich, ITP, HIT K,
  Wolfgang-Pauli-Strasse 27}}
\hypersetup{pdfcontactpostcode={8093}}
\hypersetup{pdfcontactcity={Zurich}}
\hypersetup{pdfcontactcountry={Switzerland}}
\hypersetup{pdfcontactemail={nbeisert@itp.phys.ethz.ch}}
\hypersetup{pdfcontacturl={http://people.phys.ethz.ch/\xmptilde nbeisert/}}

\newcommand{\secref}[1]{\hyperref[#1]{section \ref*{#1}}}

\parskip1ex
\parindent0pt
\let\olditemize\itemize
\def\itemize{\olditemize\parskip0pt}

\begin{document}

\title{The \textsf{childdoc} Package}
\hypersetup{pdftitle={The childdoc Package}}
\author{Niklas Beisert\\[2ex]
  Institut f\"ur Theoretische Physik\\
  Eidgen\"ossische Technische Hochschule Z\"urich\\
  Wolfgang-Pauli-Strasse 27, 8093 Z\"urich, Switzerland\\[1ex]
  \href{mailto:nbeisert@itp.phys.ethz.ch}
  {\texttt{nbeisert@itp.phys.ethz.ch}}}
\hypersetup{pdfauthor={Niklas Beisert}}
\hypersetup{pdfsubject={Manual for the LaTeX2e Package childdoc}}
\date{30 December 2018, \textsf{v2.0}}
\maketitle

\begin{abstract}\noindent
\textsf{childdoc} is a \LaTeXe{} package
that enables the direct compilation
of document sections included by |\include|
to individual files.
\end{abstract}

\begingroup
\parskip0ex
\tableofcontents
\endgroup

%%%%%%%%%%%%%%%%%%%%%%%%%%%%%%%%%%%%%%%%%%%%%%%%%%%%%%%%%%%%%%%%%%%%%%%%%%%%%%%%
%%%%%%%%%%%%%%%%%%%%%%%%%%%%%%%%%%%%%%%%%%%%%%%%%%%%%%%%%%%%%%%%%%%%%%%%%%%%%%%%
\section{Introduction}

\LaTeX{} provides a mechanism to structure a large document (such as a book)
into a main file and several child files (containing the chapters)
using the |\include| command.
This mechanism is beneficial for documents
which span hundreds of pages in order to
make the source file(s) more manageable.
Moreover, compilation can be restricted to
selected child files by means of the |\includeonly| command.
The latter feature can be used to reduce the compilation time while editing
(this was significantly more useful in the earlier days of \LaTeX{})
or to generate a smaller document which is easier to navigate.
Another application of |\includeonly| is to generate
documents consisting of selected parts of the complete document.

However, there are a few drawbacks of the plain |\include| mechanism:
\begin{itemize}
\item
The child files cannot be compiled on their own,
they can only be compiled via the main file.
A naive editing environment
(such as a text editor with an option
to have the current file processed by \LaTeX)
may require one to switch to the main file before compiling;
attempting to compile the child file produces errors.
\item
The main file must be modified (each time)
to adjust the |\includeonly| command
to the present needs. This easily leaves the main file in a messy state.
\item
The generated document will always carry the filename
of the main document. This is inconvenient if
several child files are to be compiled and
to be kept for distribution.
\end{itemize}

The present package provides a simple interface
to make child files individually compilable by \LaTeX{}.
Compiling a child file then has the same effect as compiling
the main file with an |\includeonly| command
to select the appropriate child.
Moreover the generated document will carry the name of the child
rather than the main file.
This resolves all three above issues.

This feature is meant to make the editing of books,
thesis documents and lecture notes somewhat more convenient.
However, the package can also be used efficiently for
composing a series of documents (such as exercise sheets)
which are typically distributed individually.
It then assists the author in generating the individual documents
(potentially in different versions)
as well as a document containing the collected series.
Another application is in developing style files
or other kinds of included material
where compilation of the style file could redirect
to a sample or test file.

%%%%%%%%%%%%%%%%%%%%%%%%%%%%%%%%%%%%%%%%%%%%%%%%%%%%%%%%%%%%%%%%%%%%%%%%%%%%%%%%
%%%%%%%%%%%%%%%%%%%%%%%%%%%%%%%%%%%%%%%%%%%%%%%%%%%%%%%%%%%%%%%%%%%%%%%%%%%%%%%%
\section{Usage}

First of all, the package \textsf{childdoc} is \emph{not} a standard
\LaTeXe{} |.sty| style file! Therefore it needs to be invoked in
a non-standard way.

%%%%%%%%%%%%%%%%%%%%%%%%%%%%%%%%%%%%%%%%%%%%%%%%%%%%%%%%%%%%%%%%%%%%%%%%%%%%%%%%
\subsection{Included Files}
\label{sec:include}

%%%%%%%%%%%%%%%%%%%%%%%%%%%%%%%%%%%%%%%%
\DescribeMacro{\childdocmain}
To use the package, add the commands
\begin{center}
\begin{tabular}{l}
|\input{childdoc.def}|\\
|\childdocmain{}|\\
\end{tabular}
\end{center}
at the very top of the main \LaTeX{} file,
in particular \emph{before} the |\documentclass| statement!
The argument of |\childdocmain| should be left empty
(but it must be present).

%%%%%%%%%%%%%%%%%%%%%%%%%%%%%%%%%%%%%%%%
\DescribeMacro{\childdocof}
Furthermore, add the commands
\begin{center}
\begin{tabular}{l}
|\input{childdoc.def}|\\
|\childdocof{|\textit{main}|}|\\
\end{tabular}
\end{center}
at the top of every child file \textit{child}
which is included by |\include{|\textit{child}|}|
from within the main file
(or at least for those files to be compiled individually).
The argument \textit{main} must be the filename of the main file.

There are a couple of
considerations in setting up the main and child documents:

%%%%%%%%%%%%%%%%%%%%%%%%%%%%%%%%%%%%%%%%
\paragraph{Restrictions.}

Please note the following restrictions:
\begin{itemize}
\item
|\childdocmain| must be called with one argument \textit{main}
to ensure compatibility with earlier version of the package.
It must either be empty (|\childdocmain{}|)
or precisely match the filename of the main file in which it is specified.
See \secref{sec:detection} for further information.
\item
The filename \textit{main} must be specified without the |.tex| extension.
\item
The filename \textit{main} is case sensitive
(even in case-insensitive file systems)
due to internal string comparison.
\item
The argument \textit{main} should be fully expanded, it cannot be a macro.
\item
Subdirectories and special characters should be avoided in filenames.
\item
The command |\childdocmain{|\textit{main}|}| must be followed by a whitespace.
It should not be followed immediately by another command
or by a comment mark `|%|'.
This is because the \TeX{} parser reads the token immediately following
the argument of |\childdocmain| and puts it
at the beginning of every child section;
however, a white\-space is ignored.
\end{itemize}

%%%%%%%%%%%%%%%%%%%%%%%%%%%%%%%%%%%%%%%%
\paragraph{Content of Main File.}

It is advisable to place all content in the child files included by |\include|.
Any output contained in the main file will appear in all child documents
unless suppressed manually;
it cannot be suppressed automatically by the |\includeonly| directive
and thus should normally be avoided.
A method to include some content in the main file
by means of conditional processing is described in \secref{sec:conditional}.

%%%%%%%%%%%%%%%%%%%%%%%%%%%%%%%%%%%%%%%%
\paragraph{Page Numbering.}

When only a part of the document is compiled,
the appropriate numbering of pages
(as well as other status parameters)
is determined from the |.aux| files.
The latter contain information from previous passes.
However this information needs to propagate through
all intermediate child documents.
Therefore the page numbering in child documents may well
be inconsistent until the complete document is compiled at least once.

A useful (if unconventional) way to always ensure a consistent
page numbering is to restart the numbering in each child document
and denote the pages by `\textit{child}|.|\textit{page}'
where \textit{child} represents the chapter/section number of the child file.
This can be achieved by the command
|\numberwithin{page}{|\textit{child}|}|
of the \textsf{amsmath} package
where \textit{child} can be |chapter| or |section|
depending on the chosen structuring.
Alternatively, one can modify the macro |\thepage| appropriately
and reset the counter |page| at the start of each child file.

%%%%%%%%%%%%%%%%%%%%%%%%%%%%%%%%%%%%%%%%%%%%%%%%%%%%%%%%%%%%%%%%%%%%%%%%%%%%%%%%
\subsection{Conditional Processing}
\label{sec:conditional}

The package provides a mechanism to compile different versions
of a document. To customise the versions further some conditional processing
can come in handy to distinguish which version is being compiled.
The package provides two macros to describe the compilation context:

%%%%%%%%%%%%%%%%%%%%%%%%%%%%%%%%%%%%%%%%
\DescribeMacro{\ifchilddoc}
The conditional |\ifchilddoc| distinguishes between the compilation of
child documents and the main document:
%
\begin{center}
|\ifchilddoc |\textit{child-code}| |[|\||else |\textit{main-code}]| \||fi|
\end{center}

%%%%%%%%%%%%%%%%%%%%%%%%%%%%%%%%%%%%%%%%
\DescribeMacro{\childdocname}
\DescribeMacro{\childdocjob}
The macro |\childdocname| contains the filename (without extension)
of the main or child file being processed.
Note that |\childdocjob| will always contain the name of the main file.

%%%%%%%%%%%%%%%%%%%%%%%%%%%%%%%%%%%%%%%%
\paragraph{Title Page.}

Conditional processing can be used to include a title or banner page
in the main document when proper precautions are taken.
Importantly, the code in the main file should ensure that the page counter
(as well as other status parameters which are stored in the |.aux| files)
takes the same value after the conditional processing.
Otherwise the page numbers may take divergent values
depending on which part is compiled.

For example, a title page could be declared by:
%
\begin{center}
\begin{tabular}{l}
|\ifchilddoc\||else|\\
|\addtocounter{page}{-1}|\\
\textit{code for title page}\\
|\newpage|\\
|\||fi|
\end{tabular}
\end{center}
%
A banner page for the child documents can be generated by:
%
\begin{center}
\begin{tabular}{l}
|\ifchilddoc|\\
|\addtocounter{page}{-1}|\\
\textit{code for banner page}\\
|\newpage|\\
|\||fi|
\end{tabular}
\end{center}
%
Here one could write a message such as:
\begin{center}
|This is the part \childdocname{} of \childdocjob{}.|
\end{center}

%%%%%%%%%%%%%%%%%%%%%%%%%%%%%%%%%%%%%%%%%%%%%%%%%%%%%%%%%%%%%%%%%%%%%%%%%%%%%%%%
\subsection{Flags}
\label{sec:flags}

The package makes it easy to generate different versions
of the main or child documents.
To this end compilation flags can be defined
and assigned different default values.
They will be particularly useful in conjunction
with the forwarding mechanism described in \secref{sec:forward}.

For example, it may be useful to have a flag |\version|
which can be set to |draft| or |final|.
The document source will contain some conditional code
depending on the value of |\version|.
Suppose further, the flag should default to |final| for the main file
and to |draft| for child files
which is a natural assignment for editing the document.
This is achieved by placing the following code
in the preamble of the main document
(below the |\childdocmain| directive):
%
\begin{center}
\begin{tabular}{l}
|\ifchilddoc|\\
|\providecommand{\version}{draft}|\\
|\||else|\\
|\providecommand{\version}{final}|\\
|\||fi|
\end{tabular}
\end{center}
%
The definition by |\providecommand| makes sure
that previous definitions are not overwritten.
Further statements |\providecommand{\version}{...}|
can thus be added before the above code to override it.

For the main file, one might add a line
(between |\childdocmain| and the above block)
%
\begin{center}
|%\ifchilddoc\||else\providecommand{\version}{draft}\||fi|
\end{center}
%
which can be uncommented to produce a draft version.
Likewise one can add a line to the very top of a child file
(above the |\childdocof{|\textit{main}|}| directive)
%
\begin{center}
|%\providecommand{\version}{final}|
\end{center}
%
which can be uncommented to produce the final version of this child document.

%%%%%%%%%%%%%%%%%%%%%%%%%%%%%%%%%%%%%%%%%%%%%%%%%%%%%%%%%%%%%%%%%%%%%%%%%%%%%%%%
\subsection{Forwarding}
\label{sec:forward}

Different versions of the main or child documents
using compilation flags as described in \secref{sec:flags}
can be (permanently) stored in different files
for convenient compilation, viewing and distribution.
To this end, the package defines a command
to pass on compilation to a different file:

%%%%%%%%%%%%%%%%%%%%%%%%%%%%%%%%%%%%%%%%
\DescribeMacro{\childdocforward}
The command |\childdocforward| redirects processing to
another source file:
%
\begin{center}
\begin{tabular}{l}
|\input{childdoc.def}|\\
|\childdocforward[|\textit{main}|]{|\textit{dest}|}|\\
\end{tabular}
\end{center}
%
The argument \textit{dest} is the destination file
(without extension).
It should be the main file or one of the child files.
Note that further \textsf{childdoc} directives
such as |\childdocof| and |\childdocforward|
in the indicated file will be processed in this form.
The optional argument \textit{main}
passes on directly to the main file \textit{main}
while pretending to compile the child \textit{dest}.
This form behaves as if \textit{dest}
issues |\childdocof{|\textit{main}|}| right away,
and no further \textsf{childdoc} directives will be processed.

%%%%%%%%%%%%%%%%%%%%%%%%%%%%%%%%%%%%%%%%
\DescribeMacro{\...prefix}
In the alternative form |\childdocforwardprefix|,
%
\begin{center}
\begin{tabular}{l}
|\input{childdoc.def}|\\
|\childdocforwardprefix[|\textit{main}|]{|\textit{prefix}|}{|\textit{dest}|}|
\end{tabular}
\end{center}
%
the destination file is determined by a pattern
depending on the current file:
To make this work, the current file must be called
`{\textit{prefix}\hspace{0.2em}\textit{suffix}}'
with \textit{prefix} matching precisely the argument.
Processing is then passed on to the file
`{\textit{dest}\hspace{0.2em}\textit{suffix}}'.
Surely, the same effect is achieved by
directly specifying the
argument `{\textit{dest}\hspace{0.2em}\textit{suffix}}'
in the first form.
However, that requires to set up a different file
for each child. With the alternative form of the command
all these files can have exactly the same content
which simplifies setting them up and maintaining them.

For example, the following file |draft.tex|
with a compilation flag |\version| as described in \secref{sec:flags}
compiles the main document as a draft:
%
\begin{center}
\begin{tabular}{l}
|\def\version{draft}|\\
|\input{childdoc.def}|\\
|\childdocforward{|\textit{main}|}|
\end{tabular}
\end{center}
%
Likewise, the following files |final|\textit{nn}|.tex|
compile the final version of the child document
|child|\textit{nn}|.tex|:
%
\begin{center}
\begin{tabular}{l}
|\def\version{final}|\\
|\input{childdoc.def}|\\
|\childdocforwardprefix{final}{child}|
\end{tabular}
\end{center}
%

Note that when several versions of a main file and/or of each child file
are to be generated, it may be convenient to set up a |Makefile| or
shell script to automatise the process.

%%%%%%%%%%%%%%%%%%%%%%%%%%%%%%%%%%%%%%%%%%%%%%%%%%%%%%%%%%%%%%%%%%%%%%%%%%%%%%%%
\subsection{Command Line Processing}
\label{sec:commandline}

The effect of redirection files can also be achieved by invoking
the \LaTeX{} compiler with a more elaborate command line.
Most conveniently this should be done as part
of a shell script or a |Makefile|.

When using \textsf{childdoc} in the main file, the following
command lines effectively perform a redirection
(note that depending on the shell being used,
backslashes may have to be doubled: `|\|' $\to$ `|\\|'):
%
\begin{center}
|... -jobname "|\textit{target}|" |\\|"|[\textit{flags}]%
|\input{childdoc.def}\childdocforward[|\textit{main}|]{|\textit{dest}|}"|
\end{center}
%
Here \textit{target} is the name of the output file,
\textit{main} is the name of the main file
and \textit{dest} is the name of the main or child file to be processed
(all filenames without extensions).
The optional argument \textit{main} can be omitted
if \textit{main} matches \textit{dest}.
Optionally, compilation \textit{flags} can be defined via |\def| commands.
This command line makes the \TeX{} engine believe
it is compiling the file \textit{target}
whose content is specified as the latter parameter.
The provided code then forwards the processing to
\textit{main} or \textit{dest} as described in \secref{sec:forward}.

%%%%%%%%%%%%%%%%%%%%%%%%%%%%%%%%%%%%%%%%%%%%%%%%%%%%%%%%%%%%%%%%%%%%%%%%%%%%%%%%
\subsection{Include by Input}
\label{sec:input}

Including child documents by |\include| has some restrictions by design.
Most notably, the content of a child document always occupies
its own set of pages; pages cannot be shared between child documents.
Usually, this behaviour makes perfect sense
because each child document contain an essential part of the document.
However, in some situations it may be desirable to compose
a document from a collection of parts
without having mandatory page breaks between then.
For this case, the package
provides a mechanism to include parts
by |\input| which can also be processed individually.
However, by construction this mechanism
requires manual handling of the content to be output.

%%%%%%%%%%%%%%%%%%%%%%%%%%%%%%%%%%%%%%%%
\DescribeMacro{\ifchilddocmanual}
The main file should be prepared as usual, see \secref{sec:include}.
However, the document body must make a distinction
between processing of an individual part and of the main document, e.g.:
%
\begin{center}
\begin{tabular}{l}
|\ifchilddocmanual|\\
|\input{\childdocname}|\\
|\||else|\\
\textit{document body with }|\input{|\textit{part}|}|\\
|\||fi|
\end{tabular}
\end{center}
%
The conditional |\ifchilddocmanual| is true whenever
a part to be included by |\input| is being compiled,
and the name of the part is stored in |\childdocname|.

%%%%%%%%%%%%%%%%%%%%%%%%%%%%%%%%%%%%%%%%
\DescribeMacro{\childdocby}
Each part to be included by |\input| should start with:
%
\begin{center}
\begin{tabular}{l}
|\input{childdoc.def}|\\
|\childdocby{|\textit{main}|}|\\
\end{tabular}
\end{center}
%
The directive |\childdocby| is similar to |\childdocof|
described in \secref{sec:include},
but the subsequent selection of content must be done manually.
To that end, both |\ifchilddoc| and |\ifchilddocmanual|
will be true upon processing of a part,
and the name of the part is stored in |\childdocname|.
Note that |\jobname| will be set to the filename of the current part
so that each part receives an individual |.aux| file
that does not interfere with the |.aux| file(s) of the main document.
This behaviour can be altered by the alternative form
|\childdocby[*]{|\textit{main}|}| (with a non-empty optional argument)
which uses the |.aux| file of the main document
by setting |\jobname| to \textit{main}.

%%%%%%%%%%%%%%%%%%%%%%%%%%%%%%%%%%%%%%%%%%%%%%%%%%%%%%%%%%%%%%%%%%%%%%%%%%%%%%%%
\subsection{Driver Development}
\label{sec:driver}

The \textsf{childdoc} mechanism can also be use for the development
of definition files such as \LaTeX{} styles or classes.
This case differs from the above setup with multiple parts
included by |\include| in that no |\includeonly| should be invoked.
This can be achieved by starting the include file
(before |\ProvidesPackage|) with:
%
\begin{center}
\begin{tabular}{l}
|\input{childdoc.def}|\\
|\childdocforward{|\textit{main}|}|\\
\end{tabular}
\end{center}
%
or alternatively with:
%
\begin{center}
\begin{tabular}{l}
|\input{childdoc.def}|\\
|\childdocby{|\textit{main}|}|\\
\end{tabular}
\end{center}
%
Both forms have slightly different effects as described above.
The main file is prepared as usual, see \secref{sec:include}.

%%%%%%%%%%%%%%%%%%%%%%%%%%%%%%%%%%%%%%%%%%%%%%%%%%%%%%%%%%%%%%%%%%%%%%%%%%%%%%%%
\subsection{Legacy Detection}
\label{sec:detection}

The directive |\childdocmain| in the main file can detect
whether the complete document or merely a child is to be compiled
even without using the directive |\childdocof|.
This method is deprecated because it is less robust
and there is no compelling reason to use it;
it is merely provided for backward compatibility
and it may be removed in future versions.

If the detection mechanism is to be used,
it is mandatory to correctly specify
the filename of the main file as the argument of |\childdocmain|:
%
\begin{center}
\begin{tabular}{l}
|\input{childdoc.def}|\\
|\childdocmain{|\textit{main}|}|\\
\end{tabular}
\end{center}
%
If |\jobname| does not match the argument \textit{main} of |\childdocmain|,
it is assumed that |\jobname| points to the child file to be compiled.
When using |\childdocmain| with the main file specified as argument,
it suffices to start a child file
with just |\input{|\textit{main}|}|
without loading of the package and using |\childdocof|.
If instead all processing is done
with the appropriate \textsf{childdoc} directives,
the argument of \textit{main} of |\childdocmain| can be empty.

An alternative version of the command line processing described
in \secref{sec:commandline} using the detection mechanism reads:
%
\begin{center}
|... -jobname "|\textit{target}|" "|[\textit{flags}]%
[|\def\jobname{|\textit{dest}|}|]|\input{|\textit{main}|}"|
\end{center}

%%%%%%%%%%%%%%%%%%%%%%%%%%%%%%%%%%%%%%%%%%%%%%%%%%%%%%%%%%%%%%%%%%%%%%%%%%%%%%%%
\subsection{Manual Code}
\label{sec:manual}

In case one cannot be certain whether the definitions file |childdoc.def|
is installed on the target \TeX{} distribution
and one prefers not to ship it,
it is conceivable to paste a few relevant commands into the sources.

To that end, drop all statements |\input{childdoc.def}|
and perform the replacements as outlined below.
Instead of |\childdocmain{|\textit{main}|}| add the following code
to the top of the main file:
%
\begin{center}
\begin{tabular}{l}
|\||ifdefined\childdocname\endinput\||fi\newif\ifchilddoc|\\
|\edef\childdocname{\scantokens\expandafter{\jobname\noexpand}}|\\
|\def\childdocmain{|\textit{main}|}\||ifx\childdocmain\childdocname\||else|\\
|\childdoctrue\includeonly{\childdocname}\let\jobname\childdocmain\||fi|\\
\end{tabular}
\end{center}
%
Instead of |\childdocof{|\textit{main}|}| just include the main file
at the top of each child file:
%
\begin{center}
|\input{|\textit{main}|}|
\end{center}
%
A simple redirection |\childdocforward{|\textit{dest}|}| is achieved by:
%
\begin{center}
|\def\jobname{|\textit{dest}|}\input{\jobname}|
\end{center}
%
The redirection with prefix
|\childdocforwardprefix[|\textit{prefix}|]{|\textit{dest}|}|
is accomplished by:
%
\begin{center}
\begin{tabular}{l}
|{\edef\jobname{\scantokens\expandafter{\jobname\noexpand}}|\\
|\def\redirectjob |\textit{prefix}|#1~~~{\gdef\jobname{|\textit{dest}|#1}}|\\
|\expandafter\redirectjob\jobname~~~}\input{\jobname}|
\end{tabular}
\end{center}

In an alternative approach,
child documents can be compiled by a specific command line
without additional code or specific definitions:
%
\begin{center}
|... -jobname "|\textit{target}|" "|[\textit{flags}]%
|\includeonly{|\textit{dest}|}\input{|\textit{main}|}"|
\end{center}
%

%%%%%%%%%%%%%%%%%%%%%%%%%%%%%%%%%%%%%%%%%%%%%%%%%%%%%%%%%%%%%%%%%%%%%%%%%%%%%%%%
%%%%%%%%%%%%%%%%%%%%%%%%%%%%%%%%%%%%%%%%%%%%%%%%%%%%%%%%%%%%%%%%%%%%%%%%%%%%%%%%
\section{Information}

%%%%%%%%%%%%%%%%%%%%%%%%%%%%%%%%%%%%%%%%%%%%%%%%%%%%%%%%%%%%%%%%%%%%%%%%%%%%%%%%
\subsection{Copyright}

Copyright \copyright{} 2017--2018 Niklas Beisert

This work may be distributed and/or modified under the
conditions of the \LaTeX{} Project Public License, either version 1.3
of this license or (at your option) any later version.
The latest version of this license is in
  \url{http://www.latex-project.org/lppl.txt}
and version 1.3 or later is part of all distributions of \LaTeX{}
version 2005/12/01 or later.

This work has the LPPL maintenance status `maintained'.

The Current Maintainer of this work is Niklas Beisert.

This work consists of the files |README.txt|, |childdoc.ins| and |childdoc.dtx|
as well as the derived files |childdoc.def|, |cdocsamp.tex|
with |cdocsch1.tex|, |cdocsch2.tex|, |cdocspt3.tex|, |cdocspt4.tex|,
|cdocsdrf.tex|, |cdocsfn1.tex|, |cdocsfn2.tex|
as well as |childdoc.pdf|.

%%%%%%%%%%%%%%%%%%%%%%%%%%%%%%%%%%%%%%%%%%%%%%%%%%%%%%%%%%%%%%%%%%%%%%%%%%%%%%%%
\subsection{Files and Installation}

The package consists of the files:
%
\begin{center}
\begin{tabular}{ll}
    |README.txt|   & readme file \\
    |childdoc.ins| & installation file \\
    |childdoc.dtx| & source file \\
    |childdoc.def| & definition file \\
    |cdocsamp.tex| & sample main file \\
    |cdocsch1.tex| & sample include file \\
    |cdocsch2.tex| & sample include file \\
    |cdocspt3.tex| & sample part file \\
    |cdocspt4.tex| & sample part file \\
    |cdocsdrf.tex| & sample redirection file \\
    |cdocsfn1.tex| & sample redirection file \\
    |cdocsfn2.tex| & sample redirection file \\
    |childdoc.pdf| & manual
\end{tabular}
\end{center}
%
The distribution consists of the files
|README.txt|, |childdoc.ins| and |childdoc.dtx|.
%
\begin{itemize}
\item
Run (pdf)\LaTeX{} on |childdoc.dtx|
to compile the manual |childdoc.pdf| (this file).
\item
Run \LaTeX{} on |childdoc.ins| to create the definitions file |childdoc.def|
and the sample |cdocsamp.tex| with include files
|cdocsch1.tex|, |cdocsch2.tex|, |cdocspt3.tex|, |cdocspt4.tex|,
|cdocsdrf.tex|, |cdocsfn1.tex|, |cdocsfn2.tex|.
Then copy the file |childdoc.def| to an appropriate directory of your \LaTeX{}
distribution, e.g.\ \textit{texmf-root}|/tex/latex/childdoc|.
\end{itemize}

%%%%%%%%%%%%%%%%%%%%%%%%%%%%%%%%%%%%%%%%%%%%%%%%%%%%%%%%%%%%%%%%%%%%%%%%%%%%%%%%
\subsection{Related CTAN Packages}

There are several other packages which offer a similar functionality:
%
\begin{itemize}
\item
The packages
\href{http://ctan.org/pkg/docmute}{\textsf{docmute}},
\href{http://ctan.org/pkg/includex}{\textsf{includex}} and
\href{http://ctan.org/pkg/standalone}{\textsf{standalone}}
provide commands to include only the document body of
a child file thus allowing both files to be compiled individually.
\item
The packages \href{http://ctan.org/pkg/subdocs}{\textsf{subdocs}}
and \href{http://ctan.org/pkg/subfiles}{\textsf{subfiles}}
provide structures in which the main and child documents can be
encapsulated and allowing them to be compiled individually.
The inclusion mechanism is different from the conventional |\include|.
\item
The package \href{http://ctan.org/pkg/combine}{\textsf{combine}}
is an elaborate solution to combine several documents into one.
\end{itemize}
%
See also the CTAN topic \href{http://ctan.org/topic/subdocs}{\textsf{subdocs}}
for further related packages.
The present package differs from the above solutions in that
a document structure constructed with the conventional |\include| mechanism
just needs two extra commands at the top of every file
such that all constituent files can be compiled individually.

%%%%%%%%%%%%%%%%%%%%%%%%%%%%%%%%%%%%%%%%%%%%%%%%%%%%%%%%%%%%%%%%%%%%%%%%%%%%%%%%
%\subsection{Feature Suggestions}
%
%The following is a list of features which may be useful for future
%versions of this package:
%%
%\begin{itemize}
%\item
%\ldots
%\end{itemize}

%%%%%%%%%%%%%%%%%%%%%%%%%%%%%%%%%%%%%%%%%%%%%%%%%%%%%%%%%%%%%%%%%%%%%%%%%%%%%%%%
\subsection{Revision History}

%%%%%%%%%%%%%%%%%%%%%%%%%%%%%%%%%%%%%%%%
\paragraph{v2.0:} 2018/12/30

\begin{itemize}
\item
immediate forward processing
\item
added |\childdocby| mechanism
\item
manual restructured
\end{itemize}

%%%%%%%%%%%%%%%%%%%%%%%%%%%%%%%%%%%%%%%%
\paragraph{v1.6:} 2018/01/17

\begin{itemize}
\item
application for development of include files
\item
corrections to manual
\end{itemize}

%%%%%%%%%%%%%%%%%%%%%%%%%%%%%%%%%%%%%%%%
\paragraph{v1.5:} 2017/05/21

\begin{itemize}
\item
more complete structuring introduced
\item
|\childdocof| introduced
\item
|\childdoc| renamed to |\childdocmain|
\item
|\childredirect| renamed to |\childdocforward| and |\childdocforwardprefix|
and functionality expanded
\end{itemize}

%%%%%%%%%%%%%%%%%%%%%%%%%%%%%%%%%%%%%%%%
\paragraph{v1.0:} 2017/04/27

\begin{itemize}
\item
manual and install package
\item
first version published on CTAN
\end{itemize}

%%%%%%%%%%%%%%%%%%%%%%%%%%%%%%%%%%%%%%%%
\paragraph{v0.6:} 2017/04/26

\begin{itemize}
\item
redirection mechanism added
\end{itemize}

%%%%%%%%%%%%%%%%%%%%%%%%%%%%%%%%%%%%%%%%
\paragraph{v0.5:} 2017/04/26

\begin{itemize}
\item
functionality in definition file
\end{itemize}


%%%%%%%%%%%%%%%%%%%%%%%%%%%%%%%%%%%%%%%%%%%%%%%%%%%%%%%%%%%%%%%%%%%%%%%%%%%%%%%%
%%%%%%%%%%%%%%%%%%%%%%%%%%%%%%%%%%%%%%%%%%%%%%%%%%%%%%%%%%%%%%%%%%%%%%%%%%%%%%%%
%%%%%%%%%%%%%%%%%%%%%%%%%%%%%%%%%%%%%%%%%%%%%%%%%%%%%%%%%%%%%%%%%%%%%%%%%%%%%%%%
\appendix

\settowidth\MacroIndent{\rmfamily\scriptsize 000\ }

 \DocInput{childdoc.dtx}

\end{document}
%</driver>
% \fi
%
% %%%%%%%%%%%%%%%%%%%%%%%%%%%%%%%%%%%%%%%%%%%%%%%%%%%%%%%%%%%%%%%%%%%%%%%%%%%%%%
% %%%%%%%%%%%%%%%%%%%%%%%%%%%%%%%%%%%%%%%%%%%%%%%%%%%%%%%%%%%%%%%%%%%%%%%%%%%%%%
% \section{Sample}
%\iffalse
%<*samplemain>
%\fi
%
% The following presents a sample document
% with two chapters, two parts, a title page,
% a compile flag as well as three forwarding files to set the flag.
% It consists of eight |.tex| files:
% \begin{center}
% \begin{tabular}{ll}
% |cdocsamp.tex|&main file\\
% |cdocsch1.tex|&include file for chapter 1\\
% |cdocsch2.tex|&include file for chapter 2\\
% |cdocspt3.tex|&include file for part 3\\
% |cdocspt4.tex|&include file for part 4\\
% |cdocsdrf.tex|&forwarding file for main file in draft mode\\
% |cdocsfi1.tex|&forwarding file for final version of chapter 1\\
% |cdocsfi2.tex|&forwarding file for final version of chapter 2\\
% \end{tabular}
% \end{center}
% Each of the eight files can be compiled directly by the \LaTeX{} compiler.
%
% %%%%%%%%%%%%%%%%%%%%%%%%%%%%%%%%%%%%%%
% \paragraph{Main File.}
%
% The main file is called |cdocsamp.tex|.
%
% Load the \textsf{childdoc} definitions and
% declare the filename for the main document:
%    \begin{macrocode}
\input{childdoc.def}
\childdocmain{}
%    \end{macrocode}

% Optional override for |\version| flag:
%    \begin{macrocode}
%%\ifchilddoc\else\providecommand{\version}{draft}\fi
%    \end{macrocode}

% Define the default values for the |\version| flag
% (|final| for the main file and |draft| for childs):
%    \begin{macrocode}
\ifchilddoc
\providecommand{\version}{draft}
\else
\providecommand{\version}{final}
\fi
%    \end{macrocode}

% Load the standard document class:
%    \begin{macrocode}
\documentclass[12pt]{article}
%    \end{macrocode}

% Start the document body:
%    \begin{macrocode}
\begin{document}
%    \end{macrocode}

% Declare a title page.
% Print title, part of document being processed and version flag:
%    \begin{macrocode}
\addtocounter{page}{-1}
\begin{center}
{\LARGE\bfseries{}childdoc example\par}
\vspace{1cm}
\ifchilddoc
\ifchilddocmanual part\else chapter\fi:
`\childdocname' of `\childdocjob'\par
\else
main document: `\childdocjob'\par
\fi
version: \version\par
\end{center}
\newpage
%    \end{macrocode}

% Manually include selected file,
% otherwise process as usual:
%    \begin{macrocode}
\ifchilddocmanual
\section*{part `\childdocname'}
\input{\childdocname}
\else
%    \end{macrocode}

% Include the two chapters:
%    \begin{macrocode}
\include{cdocsch1}
\include{cdocsch2}
%    \end{macrocode}

% Include the two parts unless only chapters should be displayed:
%    \begin{macrocode}
\ifchilddoc\else
\section{part three}
\input{cdocspt3}
\section{part four}
\input{cdocspt4}
\fi
%    \end{macrocode}

% Process as usual until here:
%    \begin{macrocode}
\fi
%    \end{macrocode}

% End of document body:
%    \begin{macrocode}
\end{document}
%    \end{macrocode}
%\iffalse
%</samplemain>
%\fi
%
% %%%%%%%%%%%%%%%%%%%%%%%%%%%%%%%%%%%%%%
% \paragraph{Chapter Include Files.}
%
% The include files are called |cdocsch1.tex| and |cdocsch2.tex|.
%
%\iffalse
%<*samplechap1|samplechap2>
%\fi

% Optional override for |\version| flag:
%    \begin{macrocode}
%%\providecommand{\version}{final}
%    \end{macrocode}

% Include the main document:
%    \begin{macrocode}
\input{childdoc.def}
\childdocof{cdocsamp}
%    \end{macrocode}

%\iffalse
%</samplechap1|samplechap2>
%\fi
%
%\iffalse
%<*samplechap1>
%\fi
% Some text for chapter 1:
%    \begin{macrocode}
\section{one}
some text in chapter one
%    \end{macrocode}

%\iffalse
%</samplechap1>
%\fi
% Some text for chapter 2:
%\iffalse
%<*samplechap2>
%\fi
%    \begin{macrocode}
\section{two}
more text in chapter two
%    \end{macrocode}

%\iffalse
%</samplechap2>
%\fi
%
% %%%%%%%%%%%%%%%%%%%%%%%%%%%%%%%%%%%%%%
% \paragraph{Part Include Files.}
%
% The include files are called |cdocspt3.tex| and |cdocspt4.tex|.
%
%\iffalse
%<*samplepart3|samplepart4>
%\fi

% Optional override for |\version| flag:
%    \begin{macrocode}
%%\providecommand{\version}{final}
%    \end{macrocode}

% Include the main document:
%    \begin{macrocode}
\input{childdoc.def}
\childdocby{cdocsamp}
%    \end{macrocode}

%\iffalse
%</samplepart3|samplepart4>
%\fi
%
%\iffalse
%<*samplepart3>
%\fi
% Some text for part 3:
%    \begin{macrocode}
some text in part three
%    \end{macrocode}

%\iffalse
%</samplepart3>
%\fi
% Some text for part 4:
%\iffalse
%<*samplepart4>
%\fi
%    \begin{macrocode}
more text in part four
%    \end{macrocode}

%\iffalse
%</samplepart4>
%\fi
%
% %%%%%%%%%%%%%%%%%%%%%%%%%%%%%%%%%%%%%%
% \paragraph{Forwarding for a Complete Draft.}
%
% The following forwarding file |cdocsdrf.tex|
% compiles the main document in draft mode:
%\iffalse
%<*sampledraft>
%\fi
%    \begin{macrocode}
\def\version{draft}
\input{childdoc.def}
\childdocforward{cdocsamp}
%    \end{macrocode}

%\iffalse
%</sampledraft>
%\fi
%
% %%%%%%%%%%%%%%%%%%%%%%%%%%%%%%%%%%%%%%
% \paragraph{Forwarding for Final Version of the Chapters.}
%
% The following forwarding files |cdocsfn1.tex| and |cdocsfn2.tex|
% (with identical content)
% compile the final versions of the child documents
% |cdocsch1.tex| and |cdocsch2.tex|, respectively:
%\iffalse
%<*samplefinal>
%\fi
%    \begin{macrocode}
\def\version{final}
\input{childdoc.def}
\childdocforwardprefix[cdocsamp]{cdocsfn}{cdocsch}
%    \end{macrocode}

%\iffalse
%</samplefinal>
%\fi
%
% %%%%%%%%%%%%%%%%%%%%%%%%%%%%%%%%%%%%%%
% \paragraph{Command Line Processing.}
%
% The following three command lines generate the output files
% |cdocscld|, |cdocscl1| and |cdocscl2|
% which should be identical to
% |cdocsdrf|, |cdocsch1| and |cdocsfn2|, respectively:
% \begin{center}
% \begin{tabular}{l}
% |latex -jobname cdocscld \|\\
% |  "\def\version{draft}\input{childdoc.def}\childdocforward{cdocsamp}"|\\
% |latex -jobname cdocscl1 \|\\
% |  "\input{childdoc.def}\childdocforward[cdocsamp]{cdocsch1}"|\\
% |latex -jobname cdocscl2 \|\\
% |  "\def\version{final}\input{childdoc.def}\childdocforward{cdocsch2}"|
% \end{tabular}
% \end{center}
% Note that the trailing backslash on each first line
% merely continues the input to the second line
% (for convenient cut ant paste).
% Furthermore, the command |latex| can be replaced by any
% of its alternative versions such as |pdflatex|.
%
% %%%%%%%%%%%%%%%%%%%%%%%%%%%%%%%%%%%%%%%%%%%%%%%%%%%%%%%%%%%%%%%%%%%%%%%%%%%%%%
% %%%%%%%%%%%%%%%%%%%%%%%%%%%%%%%%%%%%%%%%%%%%%%%%%%%%%%%%%%%%%%%%%%%%%%%%%%%%%%
% \section{Implementation}
%\iffalse
%<*package>
%\fi
%
% This section describes the definitions file |childdoc.def|.

% The definitions cannot be loaded using |\usepackage| or |\RequirePackage|
% which has a mechanism to prevent loading a style file more than once.
% When loading the definitions by means of |\input|
% multiple instances have to be prevented manually:
%\iffalse
%This code needs to be before the `\ProvidesFile' directive
%which is defined at the beginning of this file.
%Therefore it is also placed there and commented out here.
%</package>
%<*discard>
%\fi
%    \begin{macrocode}
\ifdefined\childdocmain\endinput\fi
%    \end{macrocode}
%\iffalse
%</discard>
%<*package>
%\fi
%
% \macro{\ifchilddoc}
% \macro{\ifchilddocmanual}
% The conditional |\ifchilddoc| tells whether a
% child (true) or main (false) document is being compiled.
% The conditional |\ifchilddocmanual| tells whether
% the |\includeonly| mechanism is used (false) or
% the selection of child files must be performed manually (true).
% The definitions initialise to false:
%    \begin{macrocode}
\newif\ifchilddoc
\newif\ifchilddocmanual
%    \end{macrocode}

% \macro{\childdocname}
% \macro{\childdocjob}
% The macro |\childdocname| stores the name of the main document
% to be compiled. The macro |\childdocjob| stores the name of
% the document on which the \LaTeX{} compiler was originally invoked.
% The content of |\jobname| cannot be compared
% to filenames specified in the source due to different catcodes.
% The following code rescans |\jobname|, stores the result
% in |\childdocname| and saves a copy in |\childdocjob|:
%    \begin{macrocode}
\edef\childdocname{\scantokens\expandafter{\jobname\noexpand}}
\let\childdocjob\childdocname
%    \end{macrocode}

% \macro{\childdocdisable}
% The macro |\childdocdisable| prevents the main file
% from being processed more than once.
% At this stage, the main document command |\childdocmain|
% is assumed to be called once again where it should do nothing.
% Any subsequent call to it should prevent
% a secondary processing of the main document
% It overwrites the forwarding commands
% |\childdocof| and |\childdocforward|
% with empty macros to prevent further inclusions of the main document:
%    \begin{macrocode}
\newcommand{\childdocdisable}
{
  \renewcommand{\childdocmain}[1]{\renewcommand{\childdocmain}[1]{\endinput}}
  \renewcommand{\childdocof}[1]{}
  \renewcommand{\childdocby}[2][]{}
  \renewcommand{\childdocforward}[2][]{}
  \renewcommand{\childdocdisable}{}
}
%    \end{macrocode}

% \macro{\childdocmain}
% The macro |\childdocmain| is to be called at the top of the main file
% with nothing or the main filename (without extension) as argument.
% First, it breaks loops.
% If the argument is not empty and does not match |\childdocname|
% (which is set by the first inclusion of |childdoc.def|),
% |\ifchilddoc| is set to true, |\includeonly| is applied to the child file
% and |\jobname| is set to the main file
% (for proper handling of |.aux| files):
%    \begin{macrocode}
\newcommand{\childdocmain}[1]
{
  \childdocdisable\childdocmain{}
  \if?#1?\else
    \begingroup
      \def\childdoctmp{#1}
      \ifx\childdoctmp\childdocname
        \def\childdoctmp{}
      \else
        \def\childdoctmp
        {
          \childdoctrue
          \includeonly{\childdocname}
          \def\childdocjob{#1}
          \def\jobname{#1}
        }
      \fi
      \expandafter
    \endgroup
    \childdoctmp
  \fi
}
%    \end{macrocode}

% \macro{\childdocof}
% The command |\childdocof| redirects
% compilation to the main file |#1|.
%    \begin{macrocode}
\newcommand{\childdocof}[1]
{
  \childdocdisable
  \childdoctrue
  \includeonly{\childdocname}
  \def\jobname{#1}
  \def\childdocjob{#1}
  \input{#1}
}
%    \end{macrocode}

% \macro{\childdocby}
% The command |\childdocby| ....
%    \begin{macrocode}
\newcommand{\childdocby}[2][]
{
  \childdocdisable
  \childdoctrue
  \childdocmanualtrue
  \if?#1?\else
    \def\jobname{#2}
  \fi
  \def\childdocjob{#2}
  \input{#2}
  \endinput
}
%    \end{macrocode}

% \macro{\childdocforward}
% The command |\childdocforward| redirects
% compilation to the main file or
% (if the optional argument is given) a child file.
% Parameters are set as if the main file
% or a child file starting with |\childdocof| was compiled.
% Then compilation is handed over to the main file:
%    \begin{macrocode}
\newcommand{\childdocforward}[2][]
{
  \begingroup
    \if?#1?
      \def\childdoctmp
      {
        \def\childdocname{#2}
        \def\childdocjob{#2}
        \def\jobname{#2}
        \input{#2}
        \endinput
      }
    \else
      \def\childdoctmp
      {
        \childdocdisable
        \def\childdocname{#2}
        \childdoctrue
        \includeonly{#2}
        \def\childdocjob{#1}
        \def\jobname{#1}
        \input{#1}
        \endinput
      }
    \fi
    \expandafter
  \endgroup
  \childdoctmp
}
%    \end{macrocode}

% \macro{\childdocforwardprefix}
% The command |\childdocforwardprefix| redirects
% compilation to the main or a child file by means of a pattern.
% The prefix |#1| in the current filename is replaced by |#2|
% and the suffix of the current filename is kept
% (it is assumed that the filename does not contain the substring `|~~~|'
% which is used as a delimiter).
% Compilation is handed over to the new file by |\childdocforward|:
%    \begin{macrocode}
\newcommand{\childdocforwardprefix}[3][]
{
  \begingroup
    \def\childdocextract #2##1~~~{\def\childdoctmp{\childdocforward[#1]{#3##1}}}
    \expandafter\childdocextract\childdocname~~~
    \expandafter
  \endgroup
  \childdoctmp
}
%    \end{macrocode}

% \macro{\childdoc}
% The deprecated macro |\childdoc| is a legacy version of |\childdocmain|:
%    \begin{macrocode}
\newcommand{\childdoc}{\childdocmain}
%    \end{macrocode}

% \macro{\childdocredirect}
% The deprecated macro |\childdocredirect| is a legacy version
% of |\childdocforward| and |\childdocforwardprefix|:
%    \begin{macrocode}
\newcommand{\childdocredirect}[2][]
{
  \begingroup
    \if?#1?
      \def\childdoctmp{\childdocforward{#2}}
    \else
      \def\childdoctmp{\childdocforwardprefix{#1}{#2}}
    \fi
    \expandafter
  \endgroup
  \childdoctmp
}
%    \end{macrocode}

%\iffalse
%</package>
%\fi
%
\endinput
|\\
|\childdocmain{|\textit{main}|}|\\
\end{tabular}
\end{center}
%
If |\jobname| does not match the argument \textit{main} of |\childdocmain|,
it is assumed that |\jobname| points to the child file to be compiled.
When using |\childdocmain| with the main file specified as argument,
it suffices to start a child file
with just |\input{|\textit{main}|}|
without loading of the package and using |\childdocof|.
If instead all processing is done
with the appropriate \textsf{childdoc} directives,
the argument of \textit{main} of |\childdocmain| can be empty.

An alternative version of the command line processing described
in \secref{sec:commandline} using the detection mechanism reads:
%
\begin{center}
|... -jobname "|\textit{target}|" "|[\textit{flags}]%
[|\def\jobname{|\textit{dest}|}|]|\input{|\textit{main}|}"|
\end{center}

%%%%%%%%%%%%%%%%%%%%%%%%%%%%%%%%%%%%%%%%%%%%%%%%%%%%%%%%%%%%%%%%%%%%%%%%%%%%%%%%
\subsection{Manual Code}
\label{sec:manual}

In case one cannot be certain whether the definitions file |childdoc.def|
is installed on the target \TeX{} distribution
and one prefers not to ship it,
it is conceivable to paste a few relevant commands into the sources.

To that end, drop all statements |% \iffalse
%
% childdoc.dtx Copyright (C) 2017-2018 Niklas Beisert
%
% This work may be distributed and/or modified under the
% conditions of the LaTeX Project Public License, either version 1.3
% of this license or (at your option) any later version.
% The latest version of this license is in
%   http://www.latex-project.org/lppl.txt
% and version 1.3 or later is part of all distributions of LaTeX
% version 2005/12/01 or later.
%
% This work has the LPPL maintenance status `maintained'.
%
% The Current Maintainer of this work is Niklas Beisert.
%
% This work consists of the files childdoc.dtx and childdoc.ins
% and the derived files childdoc.def and cdocsamp.tex with
% cdocsch1.tex, cdocsch2.tex, cdocsdrf.tex, cdocsfn1.tex, cdocsfn2.tex.
%
%<package>\ifdefined\childdocmain\endinput\fi
%<package>\ProvidesFile{childdoc.def}[2018/12/30 v2.0 child document driver]
%<samplemain>\ProvidesFile{cdocsamp.tex}[2018/12/30 v2.0 sample for childdoc]
%<*driver>
%\ProvidesFile{childdoc.drv}[2018/12/30 v2.0 childdoc reference manual file]
\PassOptionsToClass{10pt,a4paper}{article}
\documentclass{ltxdoc}

\usepackage[margin=35mm]{geometry}
\usepackage{hyperref}
\usepackage{hyperxmp}
\usepackage[usenames]{color}

\hypersetup{colorlinks=true}
\hypersetup{pdfstartview=FitH}
\hypersetup{pdfpagemode=UseNone}
\hypersetup{pdfsource={}}
\hypersetup{pdflang={en-UK}}
\hypersetup{pdfcopyright={Copyright 2017-2018 Niklas Beisert.
  This work may be distributed and/or modified under the
  conditions of the LaTeX Project Public License, either version 1.3
  of this license or (at your option) any later version.}}
\hypersetup{pdflicenseurl={http://www.latex-project.org/lppl.txt}}
\hypersetup{pdfcontactaddress={ETH Zurich, ITP, HIT K,
  Wolfgang-Pauli-Strasse 27}}
\hypersetup{pdfcontactpostcode={8093}}
\hypersetup{pdfcontactcity={Zurich}}
\hypersetup{pdfcontactcountry={Switzerland}}
\hypersetup{pdfcontactemail={nbeisert@itp.phys.ethz.ch}}
\hypersetup{pdfcontacturl={http://people.phys.ethz.ch/\xmptilde nbeisert/}}

\newcommand{\secref}[1]{\hyperref[#1]{section \ref*{#1}}}

\parskip1ex
\parindent0pt
\let\olditemize\itemize
\def\itemize{\olditemize\parskip0pt}

\begin{document}

\title{The \textsf{childdoc} Package}
\hypersetup{pdftitle={The childdoc Package}}
\author{Niklas Beisert\\[2ex]
  Institut f\"ur Theoretische Physik\\
  Eidgen\"ossische Technische Hochschule Z\"urich\\
  Wolfgang-Pauli-Strasse 27, 8093 Z\"urich, Switzerland\\[1ex]
  \href{mailto:nbeisert@itp.phys.ethz.ch}
  {\texttt{nbeisert@itp.phys.ethz.ch}}}
\hypersetup{pdfauthor={Niklas Beisert}}
\hypersetup{pdfsubject={Manual for the LaTeX2e Package childdoc}}
\date{30 December 2018, \textsf{v2.0}}
\maketitle

\begin{abstract}\noindent
\textsf{childdoc} is a \LaTeXe{} package
that enables the direct compilation
of document sections included by |\include|
to individual files.
\end{abstract}

\begingroup
\parskip0ex
\tableofcontents
\endgroup

%%%%%%%%%%%%%%%%%%%%%%%%%%%%%%%%%%%%%%%%%%%%%%%%%%%%%%%%%%%%%%%%%%%%%%%%%%%%%%%%
%%%%%%%%%%%%%%%%%%%%%%%%%%%%%%%%%%%%%%%%%%%%%%%%%%%%%%%%%%%%%%%%%%%%%%%%%%%%%%%%
\section{Introduction}

\LaTeX{} provides a mechanism to structure a large document (such as a book)
into a main file and several child files (containing the chapters)
using the |\include| command.
This mechanism is beneficial for documents
which span hundreds of pages in order to
make the source file(s) more manageable.
Moreover, compilation can be restricted to
selected child files by means of the |\includeonly| command.
The latter feature can be used to reduce the compilation time while editing
(this was significantly more useful in the earlier days of \LaTeX{})
or to generate a smaller document which is easier to navigate.
Another application of |\includeonly| is to generate
documents consisting of selected parts of the complete document.

However, there are a few drawbacks of the plain |\include| mechanism:
\begin{itemize}
\item
The child files cannot be compiled on their own,
they can only be compiled via the main file.
A naive editing environment
(such as a text editor with an option
to have the current file processed by \LaTeX)
may require one to switch to the main file before compiling;
attempting to compile the child file produces errors.
\item
The main file must be modified (each time)
to adjust the |\includeonly| command
to the present needs. This easily leaves the main file in a messy state.
\item
The generated document will always carry the filename
of the main document. This is inconvenient if
several child files are to be compiled and
to be kept for distribution.
\end{itemize}

The present package provides a simple interface
to make child files individually compilable by \LaTeX{}.
Compiling a child file then has the same effect as compiling
the main file with an |\includeonly| command
to select the appropriate child.
Moreover the generated document will carry the name of the child
rather than the main file.
This resolves all three above issues.

This feature is meant to make the editing of books,
thesis documents and lecture notes somewhat more convenient.
However, the package can also be used efficiently for
composing a series of documents (such as exercise sheets)
which are typically distributed individually.
It then assists the author in generating the individual documents
(potentially in different versions)
as well as a document containing the collected series.
Another application is in developing style files
or other kinds of included material
where compilation of the style file could redirect
to a sample or test file.

%%%%%%%%%%%%%%%%%%%%%%%%%%%%%%%%%%%%%%%%%%%%%%%%%%%%%%%%%%%%%%%%%%%%%%%%%%%%%%%%
%%%%%%%%%%%%%%%%%%%%%%%%%%%%%%%%%%%%%%%%%%%%%%%%%%%%%%%%%%%%%%%%%%%%%%%%%%%%%%%%
\section{Usage}

First of all, the package \textsf{childdoc} is \emph{not} a standard
\LaTeXe{} |.sty| style file! Therefore it needs to be invoked in
a non-standard way.

%%%%%%%%%%%%%%%%%%%%%%%%%%%%%%%%%%%%%%%%%%%%%%%%%%%%%%%%%%%%%%%%%%%%%%%%%%%%%%%%
\subsection{Included Files}
\label{sec:include}

%%%%%%%%%%%%%%%%%%%%%%%%%%%%%%%%%%%%%%%%
\DescribeMacro{\childdocmain}
To use the package, add the commands
\begin{center}
\begin{tabular}{l}
|\input{childdoc.def}|\\
|\childdocmain{}|\\
\end{tabular}
\end{center}
at the very top of the main \LaTeX{} file,
in particular \emph{before} the |\documentclass| statement!
The argument of |\childdocmain| should be left empty
(but it must be present).

%%%%%%%%%%%%%%%%%%%%%%%%%%%%%%%%%%%%%%%%
\DescribeMacro{\childdocof}
Furthermore, add the commands
\begin{center}
\begin{tabular}{l}
|\input{childdoc.def}|\\
|\childdocof{|\textit{main}|}|\\
\end{tabular}
\end{center}
at the top of every child file \textit{child}
which is included by |\include{|\textit{child}|}|
from within the main file
(or at least for those files to be compiled individually).
The argument \textit{main} must be the filename of the main file.

There are a couple of
considerations in setting up the main and child documents:

%%%%%%%%%%%%%%%%%%%%%%%%%%%%%%%%%%%%%%%%
\paragraph{Restrictions.}

Please note the following restrictions:
\begin{itemize}
\item
|\childdocmain| must be called with one argument \textit{main}
to ensure compatibility with earlier version of the package.
It must either be empty (|\childdocmain{}|)
or precisely match the filename of the main file in which it is specified.
See \secref{sec:detection} for further information.
\item
The filename \textit{main} must be specified without the |.tex| extension.
\item
The filename \textit{main} is case sensitive
(even in case-insensitive file systems)
due to internal string comparison.
\item
The argument \textit{main} should be fully expanded, it cannot be a macro.
\item
Subdirectories and special characters should be avoided in filenames.
\item
The command |\childdocmain{|\textit{main}|}| must be followed by a whitespace.
It should not be followed immediately by another command
or by a comment mark `|%|'.
This is because the \TeX{} parser reads the token immediately following
the argument of |\childdocmain| and puts it
at the beginning of every child section;
however, a white\-space is ignored.
\end{itemize}

%%%%%%%%%%%%%%%%%%%%%%%%%%%%%%%%%%%%%%%%
\paragraph{Content of Main File.}

It is advisable to place all content in the child files included by |\include|.
Any output contained in the main file will appear in all child documents
unless suppressed manually;
it cannot be suppressed automatically by the |\includeonly| directive
and thus should normally be avoided.
A method to include some content in the main file
by means of conditional processing is described in \secref{sec:conditional}.

%%%%%%%%%%%%%%%%%%%%%%%%%%%%%%%%%%%%%%%%
\paragraph{Page Numbering.}

When only a part of the document is compiled,
the appropriate numbering of pages
(as well as other status parameters)
is determined from the |.aux| files.
The latter contain information from previous passes.
However this information needs to propagate through
all intermediate child documents.
Therefore the page numbering in child documents may well
be inconsistent until the complete document is compiled at least once.

A useful (if unconventional) way to always ensure a consistent
page numbering is to restart the numbering in each child document
and denote the pages by `\textit{child}|.|\textit{page}'
where \textit{child} represents the chapter/section number of the child file.
This can be achieved by the command
|\numberwithin{page}{|\textit{child}|}|
of the \textsf{amsmath} package
where \textit{child} can be |chapter| or |section|
depending on the chosen structuring.
Alternatively, one can modify the macro |\thepage| appropriately
and reset the counter |page| at the start of each child file.

%%%%%%%%%%%%%%%%%%%%%%%%%%%%%%%%%%%%%%%%%%%%%%%%%%%%%%%%%%%%%%%%%%%%%%%%%%%%%%%%
\subsection{Conditional Processing}
\label{sec:conditional}

The package provides a mechanism to compile different versions
of a document. To customise the versions further some conditional processing
can come in handy to distinguish which version is being compiled.
The package provides two macros to describe the compilation context:

%%%%%%%%%%%%%%%%%%%%%%%%%%%%%%%%%%%%%%%%
\DescribeMacro{\ifchilddoc}
The conditional |\ifchilddoc| distinguishes between the compilation of
child documents and the main document:
%
\begin{center}
|\ifchilddoc |\textit{child-code}| |[|\||else |\textit{main-code}]| \||fi|
\end{center}

%%%%%%%%%%%%%%%%%%%%%%%%%%%%%%%%%%%%%%%%
\DescribeMacro{\childdocname}
\DescribeMacro{\childdocjob}
The macro |\childdocname| contains the filename (without extension)
of the main or child file being processed.
Note that |\childdocjob| will always contain the name of the main file.

%%%%%%%%%%%%%%%%%%%%%%%%%%%%%%%%%%%%%%%%
\paragraph{Title Page.}

Conditional processing can be used to include a title or banner page
in the main document when proper precautions are taken.
Importantly, the code in the main file should ensure that the page counter
(as well as other status parameters which are stored in the |.aux| files)
takes the same value after the conditional processing.
Otherwise the page numbers may take divergent values
depending on which part is compiled.

For example, a title page could be declared by:
%
\begin{center}
\begin{tabular}{l}
|\ifchilddoc\||else|\\
|\addtocounter{page}{-1}|\\
\textit{code for title page}\\
|\newpage|\\
|\||fi|
\end{tabular}
\end{center}
%
A banner page for the child documents can be generated by:
%
\begin{center}
\begin{tabular}{l}
|\ifchilddoc|\\
|\addtocounter{page}{-1}|\\
\textit{code for banner page}\\
|\newpage|\\
|\||fi|
\end{tabular}
\end{center}
%
Here one could write a message such as:
\begin{center}
|This is the part \childdocname{} of \childdocjob{}.|
\end{center}

%%%%%%%%%%%%%%%%%%%%%%%%%%%%%%%%%%%%%%%%%%%%%%%%%%%%%%%%%%%%%%%%%%%%%%%%%%%%%%%%
\subsection{Flags}
\label{sec:flags}

The package makes it easy to generate different versions
of the main or child documents.
To this end compilation flags can be defined
and assigned different default values.
They will be particularly useful in conjunction
with the forwarding mechanism described in \secref{sec:forward}.

For example, it may be useful to have a flag |\version|
which can be set to |draft| or |final|.
The document source will contain some conditional code
depending on the value of |\version|.
Suppose further, the flag should default to |final| for the main file
and to |draft| for child files
which is a natural assignment for editing the document.
This is achieved by placing the following code
in the preamble of the main document
(below the |\childdocmain| directive):
%
\begin{center}
\begin{tabular}{l}
|\ifchilddoc|\\
|\providecommand{\version}{draft}|\\
|\||else|\\
|\providecommand{\version}{final}|\\
|\||fi|
\end{tabular}
\end{center}
%
The definition by |\providecommand| makes sure
that previous definitions are not overwritten.
Further statements |\providecommand{\version}{...}|
can thus be added before the above code to override it.

For the main file, one might add a line
(between |\childdocmain| and the above block)
%
\begin{center}
|%\ifchilddoc\||else\providecommand{\version}{draft}\||fi|
\end{center}
%
which can be uncommented to produce a draft version.
Likewise one can add a line to the very top of a child file
(above the |\childdocof{|\textit{main}|}| directive)
%
\begin{center}
|%\providecommand{\version}{final}|
\end{center}
%
which can be uncommented to produce the final version of this child document.

%%%%%%%%%%%%%%%%%%%%%%%%%%%%%%%%%%%%%%%%%%%%%%%%%%%%%%%%%%%%%%%%%%%%%%%%%%%%%%%%
\subsection{Forwarding}
\label{sec:forward}

Different versions of the main or child documents
using compilation flags as described in \secref{sec:flags}
can be (permanently) stored in different files
for convenient compilation, viewing and distribution.
To this end, the package defines a command
to pass on compilation to a different file:

%%%%%%%%%%%%%%%%%%%%%%%%%%%%%%%%%%%%%%%%
\DescribeMacro{\childdocforward}
The command |\childdocforward| redirects processing to
another source file:
%
\begin{center}
\begin{tabular}{l}
|\input{childdoc.def}|\\
|\childdocforward[|\textit{main}|]{|\textit{dest}|}|\\
\end{tabular}
\end{center}
%
The argument \textit{dest} is the destination file
(without extension).
It should be the main file or one of the child files.
Note that further \textsf{childdoc} directives
such as |\childdocof| and |\childdocforward|
in the indicated file will be processed in this form.
The optional argument \textit{main}
passes on directly to the main file \textit{main}
while pretending to compile the child \textit{dest}.
This form behaves as if \textit{dest}
issues |\childdocof{|\textit{main}|}| right away,
and no further \textsf{childdoc} directives will be processed.

%%%%%%%%%%%%%%%%%%%%%%%%%%%%%%%%%%%%%%%%
\DescribeMacro{\...prefix}
In the alternative form |\childdocforwardprefix|,
%
\begin{center}
\begin{tabular}{l}
|\input{childdoc.def}|\\
|\childdocforwardprefix[|\textit{main}|]{|\textit{prefix}|}{|\textit{dest}|}|
\end{tabular}
\end{center}
%
the destination file is determined by a pattern
depending on the current file:
To make this work, the current file must be called
`{\textit{prefix}\hspace{0.2em}\textit{suffix}}'
with \textit{prefix} matching precisely the argument.
Processing is then passed on to the file
`{\textit{dest}\hspace{0.2em}\textit{suffix}}'.
Surely, the same effect is achieved by
directly specifying the
argument `{\textit{dest}\hspace{0.2em}\textit{suffix}}'
in the first form.
However, that requires to set up a different file
for each child. With the alternative form of the command
all these files can have exactly the same content
which simplifies setting them up and maintaining them.

For example, the following file |draft.tex|
with a compilation flag |\version| as described in \secref{sec:flags}
compiles the main document as a draft:
%
\begin{center}
\begin{tabular}{l}
|\def\version{draft}|\\
|\input{childdoc.def}|\\
|\childdocforward{|\textit{main}|}|
\end{tabular}
\end{center}
%
Likewise, the following files |final|\textit{nn}|.tex|
compile the final version of the child document
|child|\textit{nn}|.tex|:
%
\begin{center}
\begin{tabular}{l}
|\def\version{final}|\\
|\input{childdoc.def}|\\
|\childdocforwardprefix{final}{child}|
\end{tabular}
\end{center}
%

Note that when several versions of a main file and/or of each child file
are to be generated, it may be convenient to set up a |Makefile| or
shell script to automatise the process.

%%%%%%%%%%%%%%%%%%%%%%%%%%%%%%%%%%%%%%%%%%%%%%%%%%%%%%%%%%%%%%%%%%%%%%%%%%%%%%%%
\subsection{Command Line Processing}
\label{sec:commandline}

The effect of redirection files can also be achieved by invoking
the \LaTeX{} compiler with a more elaborate command line.
Most conveniently this should be done as part
of a shell script or a |Makefile|.

When using \textsf{childdoc} in the main file, the following
command lines effectively perform a redirection
(note that depending on the shell being used,
backslashes may have to be doubled: `|\|' $\to$ `|\\|'):
%
\begin{center}
|... -jobname "|\textit{target}|" |\\|"|[\textit{flags}]%
|\input{childdoc.def}\childdocforward[|\textit{main}|]{|\textit{dest}|}"|
\end{center}
%
Here \textit{target} is the name of the output file,
\textit{main} is the name of the main file
and \textit{dest} is the name of the main or child file to be processed
(all filenames without extensions).
The optional argument \textit{main} can be omitted
if \textit{main} matches \textit{dest}.
Optionally, compilation \textit{flags} can be defined via |\def| commands.
This command line makes the \TeX{} engine believe
it is compiling the file \textit{target}
whose content is specified as the latter parameter.
The provided code then forwards the processing to
\textit{main} or \textit{dest} as described in \secref{sec:forward}.

%%%%%%%%%%%%%%%%%%%%%%%%%%%%%%%%%%%%%%%%%%%%%%%%%%%%%%%%%%%%%%%%%%%%%%%%%%%%%%%%
\subsection{Include by Input}
\label{sec:input}

Including child documents by |\include| has some restrictions by design.
Most notably, the content of a child document always occupies
its own set of pages; pages cannot be shared between child documents.
Usually, this behaviour makes perfect sense
because each child document contain an essential part of the document.
However, in some situations it may be desirable to compose
a document from a collection of parts
without having mandatory page breaks between then.
For this case, the package
provides a mechanism to include parts
by |\input| which can also be processed individually.
However, by construction this mechanism
requires manual handling of the content to be output.

%%%%%%%%%%%%%%%%%%%%%%%%%%%%%%%%%%%%%%%%
\DescribeMacro{\ifchilddocmanual}
The main file should be prepared as usual, see \secref{sec:include}.
However, the document body must make a distinction
between processing of an individual part and of the main document, e.g.:
%
\begin{center}
\begin{tabular}{l}
|\ifchilddocmanual|\\
|\input{\childdocname}|\\
|\||else|\\
\textit{document body with }|\input{|\textit{part}|}|\\
|\||fi|
\end{tabular}
\end{center}
%
The conditional |\ifchilddocmanual| is true whenever
a part to be included by |\input| is being compiled,
and the name of the part is stored in |\childdocname|.

%%%%%%%%%%%%%%%%%%%%%%%%%%%%%%%%%%%%%%%%
\DescribeMacro{\childdocby}
Each part to be included by |\input| should start with:
%
\begin{center}
\begin{tabular}{l}
|\input{childdoc.def}|\\
|\childdocby{|\textit{main}|}|\\
\end{tabular}
\end{center}
%
The directive |\childdocby| is similar to |\childdocof|
described in \secref{sec:include},
but the subsequent selection of content must be done manually.
To that end, both |\ifchilddoc| and |\ifchilddocmanual|
will be true upon processing of a part,
and the name of the part is stored in |\childdocname|.
Note that |\jobname| will be set to the filename of the current part
so that each part receives an individual |.aux| file
that does not interfere with the |.aux| file(s) of the main document.
This behaviour can be altered by the alternative form
|\childdocby[*]{|\textit{main}|}| (with a non-empty optional argument)
which uses the |.aux| file of the main document
by setting |\jobname| to \textit{main}.

%%%%%%%%%%%%%%%%%%%%%%%%%%%%%%%%%%%%%%%%%%%%%%%%%%%%%%%%%%%%%%%%%%%%%%%%%%%%%%%%
\subsection{Driver Development}
\label{sec:driver}

The \textsf{childdoc} mechanism can also be use for the development
of definition files such as \LaTeX{} styles or classes.
This case differs from the above setup with multiple parts
included by |\include| in that no |\includeonly| should be invoked.
This can be achieved by starting the include file
(before |\ProvidesPackage|) with:
%
\begin{center}
\begin{tabular}{l}
|\input{childdoc.def}|\\
|\childdocforward{|\textit{main}|}|\\
\end{tabular}
\end{center}
%
or alternatively with:
%
\begin{center}
\begin{tabular}{l}
|\input{childdoc.def}|\\
|\childdocby{|\textit{main}|}|\\
\end{tabular}
\end{center}
%
Both forms have slightly different effects as described above.
The main file is prepared as usual, see \secref{sec:include}.

%%%%%%%%%%%%%%%%%%%%%%%%%%%%%%%%%%%%%%%%%%%%%%%%%%%%%%%%%%%%%%%%%%%%%%%%%%%%%%%%
\subsection{Legacy Detection}
\label{sec:detection}

The directive |\childdocmain| in the main file can detect
whether the complete document or merely a child is to be compiled
even without using the directive |\childdocof|.
This method is deprecated because it is less robust
and there is no compelling reason to use it;
it is merely provided for backward compatibility
and it may be removed in future versions.

If the detection mechanism is to be used,
it is mandatory to correctly specify
the filename of the main file as the argument of |\childdocmain|:
%
\begin{center}
\begin{tabular}{l}
|\input{childdoc.def}|\\
|\childdocmain{|\textit{main}|}|\\
\end{tabular}
\end{center}
%
If |\jobname| does not match the argument \textit{main} of |\childdocmain|,
it is assumed that |\jobname| points to the child file to be compiled.
When using |\childdocmain| with the main file specified as argument,
it suffices to start a child file
with just |\input{|\textit{main}|}|
without loading of the package and using |\childdocof|.
If instead all processing is done
with the appropriate \textsf{childdoc} directives,
the argument of \textit{main} of |\childdocmain| can be empty.

An alternative version of the command line processing described
in \secref{sec:commandline} using the detection mechanism reads:
%
\begin{center}
|... -jobname "|\textit{target}|" "|[\textit{flags}]%
[|\def\jobname{|\textit{dest}|}|]|\input{|\textit{main}|}"|
\end{center}

%%%%%%%%%%%%%%%%%%%%%%%%%%%%%%%%%%%%%%%%%%%%%%%%%%%%%%%%%%%%%%%%%%%%%%%%%%%%%%%%
\subsection{Manual Code}
\label{sec:manual}

In case one cannot be certain whether the definitions file |childdoc.def|
is installed on the target \TeX{} distribution
and one prefers not to ship it,
it is conceivable to paste a few relevant commands into the sources.

To that end, drop all statements |\input{childdoc.def}|
and perform the replacements as outlined below.
Instead of |\childdocmain{|\textit{main}|}| add the following code
to the top of the main file:
%
\begin{center}
\begin{tabular}{l}
|\||ifdefined\childdocname\endinput\||fi\newif\ifchilddoc|\\
|\edef\childdocname{\scantokens\expandafter{\jobname\noexpand}}|\\
|\def\childdocmain{|\textit{main}|}\||ifx\childdocmain\childdocname\||else|\\
|\childdoctrue\includeonly{\childdocname}\let\jobname\childdocmain\||fi|\\
\end{tabular}
\end{center}
%
Instead of |\childdocof{|\textit{main}|}| just include the main file
at the top of each child file:
%
\begin{center}
|\input{|\textit{main}|}|
\end{center}
%
A simple redirection |\childdocforward{|\textit{dest}|}| is achieved by:
%
\begin{center}
|\def\jobname{|\textit{dest}|}\input{\jobname}|
\end{center}
%
The redirection with prefix
|\childdocforwardprefix[|\textit{prefix}|]{|\textit{dest}|}|
is accomplished by:
%
\begin{center}
\begin{tabular}{l}
|{\edef\jobname{\scantokens\expandafter{\jobname\noexpand}}|\\
|\def\redirectjob |\textit{prefix}|#1~~~{\gdef\jobname{|\textit{dest}|#1}}|\\
|\expandafter\redirectjob\jobname~~~}\input{\jobname}|
\end{tabular}
\end{center}

In an alternative approach,
child documents can be compiled by a specific command line
without additional code or specific definitions:
%
\begin{center}
|... -jobname "|\textit{target}|" "|[\textit{flags}]%
|\includeonly{|\textit{dest}|}\input{|\textit{main}|}"|
\end{center}
%

%%%%%%%%%%%%%%%%%%%%%%%%%%%%%%%%%%%%%%%%%%%%%%%%%%%%%%%%%%%%%%%%%%%%%%%%%%%%%%%%
%%%%%%%%%%%%%%%%%%%%%%%%%%%%%%%%%%%%%%%%%%%%%%%%%%%%%%%%%%%%%%%%%%%%%%%%%%%%%%%%
\section{Information}

%%%%%%%%%%%%%%%%%%%%%%%%%%%%%%%%%%%%%%%%%%%%%%%%%%%%%%%%%%%%%%%%%%%%%%%%%%%%%%%%
\subsection{Copyright}

Copyright \copyright{} 2017--2018 Niklas Beisert

This work may be distributed and/or modified under the
conditions of the \LaTeX{} Project Public License, either version 1.3
of this license or (at your option) any later version.
The latest version of this license is in
  \url{http://www.latex-project.org/lppl.txt}
and version 1.3 or later is part of all distributions of \LaTeX{}
version 2005/12/01 or later.

This work has the LPPL maintenance status `maintained'.

The Current Maintainer of this work is Niklas Beisert.

This work consists of the files |README.txt|, |childdoc.ins| and |childdoc.dtx|
as well as the derived files |childdoc.def|, |cdocsamp.tex|
with |cdocsch1.tex|, |cdocsch2.tex|, |cdocspt3.tex|, |cdocspt4.tex|,
|cdocsdrf.tex|, |cdocsfn1.tex|, |cdocsfn2.tex|
as well as |childdoc.pdf|.

%%%%%%%%%%%%%%%%%%%%%%%%%%%%%%%%%%%%%%%%%%%%%%%%%%%%%%%%%%%%%%%%%%%%%%%%%%%%%%%%
\subsection{Files and Installation}

The package consists of the files:
%
\begin{center}
\begin{tabular}{ll}
    |README.txt|   & readme file \\
    |childdoc.ins| & installation file \\
    |childdoc.dtx| & source file \\
    |childdoc.def| & definition file \\
    |cdocsamp.tex| & sample main file \\
    |cdocsch1.tex| & sample include file \\
    |cdocsch2.tex| & sample include file \\
    |cdocspt3.tex| & sample part file \\
    |cdocspt4.tex| & sample part file \\
    |cdocsdrf.tex| & sample redirection file \\
    |cdocsfn1.tex| & sample redirection file \\
    |cdocsfn2.tex| & sample redirection file \\
    |childdoc.pdf| & manual
\end{tabular}
\end{center}
%
The distribution consists of the files
|README.txt|, |childdoc.ins| and |childdoc.dtx|.
%
\begin{itemize}
\item
Run (pdf)\LaTeX{} on |childdoc.dtx|
to compile the manual |childdoc.pdf| (this file).
\item
Run \LaTeX{} on |childdoc.ins| to create the definitions file |childdoc.def|
and the sample |cdocsamp.tex| with include files
|cdocsch1.tex|, |cdocsch2.tex|, |cdocspt3.tex|, |cdocspt4.tex|,
|cdocsdrf.tex|, |cdocsfn1.tex|, |cdocsfn2.tex|.
Then copy the file |childdoc.def| to an appropriate directory of your \LaTeX{}
distribution, e.g.\ \textit{texmf-root}|/tex/latex/childdoc|.
\end{itemize}

%%%%%%%%%%%%%%%%%%%%%%%%%%%%%%%%%%%%%%%%%%%%%%%%%%%%%%%%%%%%%%%%%%%%%%%%%%%%%%%%
\subsection{Related CTAN Packages}

There are several other packages which offer a similar functionality:
%
\begin{itemize}
\item
The packages
\href{http://ctan.org/pkg/docmute}{\textsf{docmute}},
\href{http://ctan.org/pkg/includex}{\textsf{includex}} and
\href{http://ctan.org/pkg/standalone}{\textsf{standalone}}
provide commands to include only the document body of
a child file thus allowing both files to be compiled individually.
\item
The packages \href{http://ctan.org/pkg/subdocs}{\textsf{subdocs}}
and \href{http://ctan.org/pkg/subfiles}{\textsf{subfiles}}
provide structures in which the main and child documents can be
encapsulated and allowing them to be compiled individually.
The inclusion mechanism is different from the conventional |\include|.
\item
The package \href{http://ctan.org/pkg/combine}{\textsf{combine}}
is an elaborate solution to combine several documents into one.
\end{itemize}
%
See also the CTAN topic \href{http://ctan.org/topic/subdocs}{\textsf{subdocs}}
for further related packages.
The present package differs from the above solutions in that
a document structure constructed with the conventional |\include| mechanism
just needs two extra commands at the top of every file
such that all constituent files can be compiled individually.

%%%%%%%%%%%%%%%%%%%%%%%%%%%%%%%%%%%%%%%%%%%%%%%%%%%%%%%%%%%%%%%%%%%%%%%%%%%%%%%%
%\subsection{Feature Suggestions}
%
%The following is a list of features which may be useful for future
%versions of this package:
%%
%\begin{itemize}
%\item
%\ldots
%\end{itemize}

%%%%%%%%%%%%%%%%%%%%%%%%%%%%%%%%%%%%%%%%%%%%%%%%%%%%%%%%%%%%%%%%%%%%%%%%%%%%%%%%
\subsection{Revision History}

%%%%%%%%%%%%%%%%%%%%%%%%%%%%%%%%%%%%%%%%
\paragraph{v2.0:} 2018/12/30

\begin{itemize}
\item
immediate forward processing
\item
added |\childdocby| mechanism
\item
manual restructured
\end{itemize}

%%%%%%%%%%%%%%%%%%%%%%%%%%%%%%%%%%%%%%%%
\paragraph{v1.6:} 2018/01/17

\begin{itemize}
\item
application for development of include files
\item
corrections to manual
\end{itemize}

%%%%%%%%%%%%%%%%%%%%%%%%%%%%%%%%%%%%%%%%
\paragraph{v1.5:} 2017/05/21

\begin{itemize}
\item
more complete structuring introduced
\item
|\childdocof| introduced
\item
|\childdoc| renamed to |\childdocmain|
\item
|\childredirect| renamed to |\childdocforward| and |\childdocforwardprefix|
and functionality expanded
\end{itemize}

%%%%%%%%%%%%%%%%%%%%%%%%%%%%%%%%%%%%%%%%
\paragraph{v1.0:} 2017/04/27

\begin{itemize}
\item
manual and install package
\item
first version published on CTAN
\end{itemize}

%%%%%%%%%%%%%%%%%%%%%%%%%%%%%%%%%%%%%%%%
\paragraph{v0.6:} 2017/04/26

\begin{itemize}
\item
redirection mechanism added
\end{itemize}

%%%%%%%%%%%%%%%%%%%%%%%%%%%%%%%%%%%%%%%%
\paragraph{v0.5:} 2017/04/26

\begin{itemize}
\item
functionality in definition file
\end{itemize}


%%%%%%%%%%%%%%%%%%%%%%%%%%%%%%%%%%%%%%%%%%%%%%%%%%%%%%%%%%%%%%%%%%%%%%%%%%%%%%%%
%%%%%%%%%%%%%%%%%%%%%%%%%%%%%%%%%%%%%%%%%%%%%%%%%%%%%%%%%%%%%%%%%%%%%%%%%%%%%%%%
%%%%%%%%%%%%%%%%%%%%%%%%%%%%%%%%%%%%%%%%%%%%%%%%%%%%%%%%%%%%%%%%%%%%%%%%%%%%%%%%
\appendix

\settowidth\MacroIndent{\rmfamily\scriptsize 000\ }

 \DocInput{childdoc.dtx}

\end{document}
%</driver>
% \fi
%
% %%%%%%%%%%%%%%%%%%%%%%%%%%%%%%%%%%%%%%%%%%%%%%%%%%%%%%%%%%%%%%%%%%%%%%%%%%%%%%
% %%%%%%%%%%%%%%%%%%%%%%%%%%%%%%%%%%%%%%%%%%%%%%%%%%%%%%%%%%%%%%%%%%%%%%%%%%%%%%
% \section{Sample}
%\iffalse
%<*samplemain>
%\fi
%
% The following presents a sample document
% with two chapters, two parts, a title page,
% a compile flag as well as three forwarding files to set the flag.
% It consists of eight |.tex| files:
% \begin{center}
% \begin{tabular}{ll}
% |cdocsamp.tex|&main file\\
% |cdocsch1.tex|&include file for chapter 1\\
% |cdocsch2.tex|&include file for chapter 2\\
% |cdocspt3.tex|&include file for part 3\\
% |cdocspt4.tex|&include file for part 4\\
% |cdocsdrf.tex|&forwarding file for main file in draft mode\\
% |cdocsfi1.tex|&forwarding file for final version of chapter 1\\
% |cdocsfi2.tex|&forwarding file for final version of chapter 2\\
% \end{tabular}
% \end{center}
% Each of the eight files can be compiled directly by the \LaTeX{} compiler.
%
% %%%%%%%%%%%%%%%%%%%%%%%%%%%%%%%%%%%%%%
% \paragraph{Main File.}
%
% The main file is called |cdocsamp.tex|.
%
% Load the \textsf{childdoc} definitions and
% declare the filename for the main document:
%    \begin{macrocode}
\input{childdoc.def}
\childdocmain{}
%    \end{macrocode}

% Optional override for |\version| flag:
%    \begin{macrocode}
%%\ifchilddoc\else\providecommand{\version}{draft}\fi
%    \end{macrocode}

% Define the default values for the |\version| flag
% (|final| for the main file and |draft| for childs):
%    \begin{macrocode}
\ifchilddoc
\providecommand{\version}{draft}
\else
\providecommand{\version}{final}
\fi
%    \end{macrocode}

% Load the standard document class:
%    \begin{macrocode}
\documentclass[12pt]{article}
%    \end{macrocode}

% Start the document body:
%    \begin{macrocode}
\begin{document}
%    \end{macrocode}

% Declare a title page.
% Print title, part of document being processed and version flag:
%    \begin{macrocode}
\addtocounter{page}{-1}
\begin{center}
{\LARGE\bfseries{}childdoc example\par}
\vspace{1cm}
\ifchilddoc
\ifchilddocmanual part\else chapter\fi:
`\childdocname' of `\childdocjob'\par
\else
main document: `\childdocjob'\par
\fi
version: \version\par
\end{center}
\newpage
%    \end{macrocode}

% Manually include selected file,
% otherwise process as usual:
%    \begin{macrocode}
\ifchilddocmanual
\section*{part `\childdocname'}
\input{\childdocname}
\else
%    \end{macrocode}

% Include the two chapters:
%    \begin{macrocode}
\include{cdocsch1}
\include{cdocsch2}
%    \end{macrocode}

% Include the two parts unless only chapters should be displayed:
%    \begin{macrocode}
\ifchilddoc\else
\section{part three}
\input{cdocspt3}
\section{part four}
\input{cdocspt4}
\fi
%    \end{macrocode}

% Process as usual until here:
%    \begin{macrocode}
\fi
%    \end{macrocode}

% End of document body:
%    \begin{macrocode}
\end{document}
%    \end{macrocode}
%\iffalse
%</samplemain>
%\fi
%
% %%%%%%%%%%%%%%%%%%%%%%%%%%%%%%%%%%%%%%
% \paragraph{Chapter Include Files.}
%
% The include files are called |cdocsch1.tex| and |cdocsch2.tex|.
%
%\iffalse
%<*samplechap1|samplechap2>
%\fi

% Optional override for |\version| flag:
%    \begin{macrocode}
%%\providecommand{\version}{final}
%    \end{macrocode}

% Include the main document:
%    \begin{macrocode}
\input{childdoc.def}
\childdocof{cdocsamp}
%    \end{macrocode}

%\iffalse
%</samplechap1|samplechap2>
%\fi
%
%\iffalse
%<*samplechap1>
%\fi
% Some text for chapter 1:
%    \begin{macrocode}
\section{one}
some text in chapter one
%    \end{macrocode}

%\iffalse
%</samplechap1>
%\fi
% Some text for chapter 2:
%\iffalse
%<*samplechap2>
%\fi
%    \begin{macrocode}
\section{two}
more text in chapter two
%    \end{macrocode}

%\iffalse
%</samplechap2>
%\fi
%
% %%%%%%%%%%%%%%%%%%%%%%%%%%%%%%%%%%%%%%
% \paragraph{Part Include Files.}
%
% The include files are called |cdocspt3.tex| and |cdocspt4.tex|.
%
%\iffalse
%<*samplepart3|samplepart4>
%\fi

% Optional override for |\version| flag:
%    \begin{macrocode}
%%\providecommand{\version}{final}
%    \end{macrocode}

% Include the main document:
%    \begin{macrocode}
\input{childdoc.def}
\childdocby{cdocsamp}
%    \end{macrocode}

%\iffalse
%</samplepart3|samplepart4>
%\fi
%
%\iffalse
%<*samplepart3>
%\fi
% Some text for part 3:
%    \begin{macrocode}
some text in part three
%    \end{macrocode}

%\iffalse
%</samplepart3>
%\fi
% Some text for part 4:
%\iffalse
%<*samplepart4>
%\fi
%    \begin{macrocode}
more text in part four
%    \end{macrocode}

%\iffalse
%</samplepart4>
%\fi
%
% %%%%%%%%%%%%%%%%%%%%%%%%%%%%%%%%%%%%%%
% \paragraph{Forwarding for a Complete Draft.}
%
% The following forwarding file |cdocsdrf.tex|
% compiles the main document in draft mode:
%\iffalse
%<*sampledraft>
%\fi
%    \begin{macrocode}
\def\version{draft}
\input{childdoc.def}
\childdocforward{cdocsamp}
%    \end{macrocode}

%\iffalse
%</sampledraft>
%\fi
%
% %%%%%%%%%%%%%%%%%%%%%%%%%%%%%%%%%%%%%%
% \paragraph{Forwarding for Final Version of the Chapters.}
%
% The following forwarding files |cdocsfn1.tex| and |cdocsfn2.tex|
% (with identical content)
% compile the final versions of the child documents
% |cdocsch1.tex| and |cdocsch2.tex|, respectively:
%\iffalse
%<*samplefinal>
%\fi
%    \begin{macrocode}
\def\version{final}
\input{childdoc.def}
\childdocforwardprefix[cdocsamp]{cdocsfn}{cdocsch}
%    \end{macrocode}

%\iffalse
%</samplefinal>
%\fi
%
% %%%%%%%%%%%%%%%%%%%%%%%%%%%%%%%%%%%%%%
% \paragraph{Command Line Processing.}
%
% The following three command lines generate the output files
% |cdocscld|, |cdocscl1| and |cdocscl2|
% which should be identical to
% |cdocsdrf|, |cdocsch1| and |cdocsfn2|, respectively:
% \begin{center}
% \begin{tabular}{l}
% |latex -jobname cdocscld \|\\
% |  "\def\version{draft}\input{childdoc.def}\childdocforward{cdocsamp}"|\\
% |latex -jobname cdocscl1 \|\\
% |  "\input{childdoc.def}\childdocforward[cdocsamp]{cdocsch1}"|\\
% |latex -jobname cdocscl2 \|\\
% |  "\def\version{final}\input{childdoc.def}\childdocforward{cdocsch2}"|
% \end{tabular}
% \end{center}
% Note that the trailing backslash on each first line
% merely continues the input to the second line
% (for convenient cut ant paste).
% Furthermore, the command |latex| can be replaced by any
% of its alternative versions such as |pdflatex|.
%
% %%%%%%%%%%%%%%%%%%%%%%%%%%%%%%%%%%%%%%%%%%%%%%%%%%%%%%%%%%%%%%%%%%%%%%%%%%%%%%
% %%%%%%%%%%%%%%%%%%%%%%%%%%%%%%%%%%%%%%%%%%%%%%%%%%%%%%%%%%%%%%%%%%%%%%%%%%%%%%
% \section{Implementation}
%\iffalse
%<*package>
%\fi
%
% This section describes the definitions file |childdoc.def|.

% The definitions cannot be loaded using |\usepackage| or |\RequirePackage|
% which has a mechanism to prevent loading a style file more than once.
% When loading the definitions by means of |\input|
% multiple instances have to be prevented manually:
%\iffalse
%This code needs to be before the `\ProvidesFile' directive
%which is defined at the beginning of this file.
%Therefore it is also placed there and commented out here.
%</package>
%<*discard>
%\fi
%    \begin{macrocode}
\ifdefined\childdocmain\endinput\fi
%    \end{macrocode}
%\iffalse
%</discard>
%<*package>
%\fi
%
% \macro{\ifchilddoc}
% \macro{\ifchilddocmanual}
% The conditional |\ifchilddoc| tells whether a
% child (true) or main (false) document is being compiled.
% The conditional |\ifchilddocmanual| tells whether
% the |\includeonly| mechanism is used (false) or
% the selection of child files must be performed manually (true).
% The definitions initialise to false:
%    \begin{macrocode}
\newif\ifchilddoc
\newif\ifchilddocmanual
%    \end{macrocode}

% \macro{\childdocname}
% \macro{\childdocjob}
% The macro |\childdocname| stores the name of the main document
% to be compiled. The macro |\childdocjob| stores the name of
% the document on which the \LaTeX{} compiler was originally invoked.
% The content of |\jobname| cannot be compared
% to filenames specified in the source due to different catcodes.
% The following code rescans |\jobname|, stores the result
% in |\childdocname| and saves a copy in |\childdocjob|:
%    \begin{macrocode}
\edef\childdocname{\scantokens\expandafter{\jobname\noexpand}}
\let\childdocjob\childdocname
%    \end{macrocode}

% \macro{\childdocdisable}
% The macro |\childdocdisable| prevents the main file
% from being processed more than once.
% At this stage, the main document command |\childdocmain|
% is assumed to be called once again where it should do nothing.
% Any subsequent call to it should prevent
% a secondary processing of the main document
% It overwrites the forwarding commands
% |\childdocof| and |\childdocforward|
% with empty macros to prevent further inclusions of the main document:
%    \begin{macrocode}
\newcommand{\childdocdisable}
{
  \renewcommand{\childdocmain}[1]{\renewcommand{\childdocmain}[1]{\endinput}}
  \renewcommand{\childdocof}[1]{}
  \renewcommand{\childdocby}[2][]{}
  \renewcommand{\childdocforward}[2][]{}
  \renewcommand{\childdocdisable}{}
}
%    \end{macrocode}

% \macro{\childdocmain}
% The macro |\childdocmain| is to be called at the top of the main file
% with nothing or the main filename (without extension) as argument.
% First, it breaks loops.
% If the argument is not empty and does not match |\childdocname|
% (which is set by the first inclusion of |childdoc.def|),
% |\ifchilddoc| is set to true, |\includeonly| is applied to the child file
% and |\jobname| is set to the main file
% (for proper handling of |.aux| files):
%    \begin{macrocode}
\newcommand{\childdocmain}[1]
{
  \childdocdisable\childdocmain{}
  \if?#1?\else
    \begingroup
      \def\childdoctmp{#1}
      \ifx\childdoctmp\childdocname
        \def\childdoctmp{}
      \else
        \def\childdoctmp
        {
          \childdoctrue
          \includeonly{\childdocname}
          \def\childdocjob{#1}
          \def\jobname{#1}
        }
      \fi
      \expandafter
    \endgroup
    \childdoctmp
  \fi
}
%    \end{macrocode}

% \macro{\childdocof}
% The command |\childdocof| redirects
% compilation to the main file |#1|.
%    \begin{macrocode}
\newcommand{\childdocof}[1]
{
  \childdocdisable
  \childdoctrue
  \includeonly{\childdocname}
  \def\jobname{#1}
  \def\childdocjob{#1}
  \input{#1}
}
%    \end{macrocode}

% \macro{\childdocby}
% The command |\childdocby| ....
%    \begin{macrocode}
\newcommand{\childdocby}[2][]
{
  \childdocdisable
  \childdoctrue
  \childdocmanualtrue
  \if?#1?\else
    \def\jobname{#2}
  \fi
  \def\childdocjob{#2}
  \input{#2}
  \endinput
}
%    \end{macrocode}

% \macro{\childdocforward}
% The command |\childdocforward| redirects
% compilation to the main file or
% (if the optional argument is given) a child file.
% Parameters are set as if the main file
% or a child file starting with |\childdocof| was compiled.
% Then compilation is handed over to the main file:
%    \begin{macrocode}
\newcommand{\childdocforward}[2][]
{
  \begingroup
    \if?#1?
      \def\childdoctmp
      {
        \def\childdocname{#2}
        \def\childdocjob{#2}
        \def\jobname{#2}
        \input{#2}
        \endinput
      }
    \else
      \def\childdoctmp
      {
        \childdocdisable
        \def\childdocname{#2}
        \childdoctrue
        \includeonly{#2}
        \def\childdocjob{#1}
        \def\jobname{#1}
        \input{#1}
        \endinput
      }
    \fi
    \expandafter
  \endgroup
  \childdoctmp
}
%    \end{macrocode}

% \macro{\childdocforwardprefix}
% The command |\childdocforwardprefix| redirects
% compilation to the main or a child file by means of a pattern.
% The prefix |#1| in the current filename is replaced by |#2|
% and the suffix of the current filename is kept
% (it is assumed that the filename does not contain the substring `|~~~|'
% which is used as a delimiter).
% Compilation is handed over to the new file by |\childdocforward|:
%    \begin{macrocode}
\newcommand{\childdocforwardprefix}[3][]
{
  \begingroup
    \def\childdocextract #2##1~~~{\def\childdoctmp{\childdocforward[#1]{#3##1}}}
    \expandafter\childdocextract\childdocname~~~
    \expandafter
  \endgroup
  \childdoctmp
}
%    \end{macrocode}

% \macro{\childdoc}
% The deprecated macro |\childdoc| is a legacy version of |\childdocmain|:
%    \begin{macrocode}
\newcommand{\childdoc}{\childdocmain}
%    \end{macrocode}

% \macro{\childdocredirect}
% The deprecated macro |\childdocredirect| is a legacy version
% of |\childdocforward| and |\childdocforwardprefix|:
%    \begin{macrocode}
\newcommand{\childdocredirect}[2][]
{
  \begingroup
    \if?#1?
      \def\childdoctmp{\childdocforward{#2}}
    \else
      \def\childdoctmp{\childdocforwardprefix{#1}{#2}}
    \fi
    \expandafter
  \endgroup
  \childdoctmp
}
%    \end{macrocode}

%\iffalse
%</package>
%\fi
%
\endinput
|
and perform the replacements as outlined below.
Instead of |\childdocmain{|\textit{main}|}| add the following code
to the top of the main file:
%
\begin{center}
\begin{tabular}{l}
|\||ifdefined\childdocname\endinput\||fi\newif\ifchilddoc|\\
|\edef\childdocname{\scantokens\expandafter{\jobname\noexpand}}|\\
|\def\childdocmain{|\textit{main}|}\||ifx\childdocmain\childdocname\||else|\\
|\childdoctrue\includeonly{\childdocname}\let\jobname\childdocmain\||fi|\\
\end{tabular}
\end{center}
%
Instead of |\childdocof{|\textit{main}|}| just include the main file
at the top of each child file:
%
\begin{center}
|\input{|\textit{main}|}|
\end{center}
%
A simple redirection |\childdocforward{|\textit{dest}|}| is achieved by:
%
\begin{center}
|\def\jobname{|\textit{dest}|}\input{\jobname}|
\end{center}
%
The redirection with prefix
|\childdocforwardprefix[|\textit{prefix}|]{|\textit{dest}|}|
is accomplished by:
%
\begin{center}
\begin{tabular}{l}
|{\edef\jobname{\scantokens\expandafter{\jobname\noexpand}}|\\
|\def\redirectjob |\textit{prefix}|#1~~~{\gdef\jobname{|\textit{dest}|#1}}|\\
|\expandafter\redirectjob\jobname~~~}\input{\jobname}|
\end{tabular}
\end{center}

In an alternative approach,
child documents can be compiled by a specific command line
without additional code or specific definitions:
%
\begin{center}
|... -jobname "|\textit{target}|" "|[\textit{flags}]%
|\includeonly{|\textit{dest}|}\input{|\textit{main}|}"|
\end{center}
%

%%%%%%%%%%%%%%%%%%%%%%%%%%%%%%%%%%%%%%%%%%%%%%%%%%%%%%%%%%%%%%%%%%%%%%%%%%%%%%%%
%%%%%%%%%%%%%%%%%%%%%%%%%%%%%%%%%%%%%%%%%%%%%%%%%%%%%%%%%%%%%%%%%%%%%%%%%%%%%%%%
\section{Information}

%%%%%%%%%%%%%%%%%%%%%%%%%%%%%%%%%%%%%%%%%%%%%%%%%%%%%%%%%%%%%%%%%%%%%%%%%%%%%%%%
\subsection{Copyright}

Copyright \copyright{} 2017--2018 Niklas Beisert

This work may be distributed and/or modified under the
conditions of the \LaTeX{} Project Public License, either version 1.3
of this license or (at your option) any later version.
The latest version of this license is in
  \url{http://www.latex-project.org/lppl.txt}
and version 1.3 or later is part of all distributions of \LaTeX{}
version 2005/12/01 or later.

This work has the LPPL maintenance status `maintained'.

The Current Maintainer of this work is Niklas Beisert.

This work consists of the files |README.txt|, |childdoc.ins| and |childdoc.dtx|
as well as the derived files |childdoc.def|, |cdocsamp.tex|
with |cdocsch1.tex|, |cdocsch2.tex|, |cdocspt3.tex|, |cdocspt4.tex|,
|cdocsdrf.tex|, |cdocsfn1.tex|, |cdocsfn2.tex|
as well as |childdoc.pdf|.

%%%%%%%%%%%%%%%%%%%%%%%%%%%%%%%%%%%%%%%%%%%%%%%%%%%%%%%%%%%%%%%%%%%%%%%%%%%%%%%%
\subsection{Files and Installation}

The package consists of the files:
%
\begin{center}
\begin{tabular}{ll}
    |README.txt|   & readme file \\
    |childdoc.ins| & installation file \\
    |childdoc.dtx| & source file \\
    |childdoc.def| & definition file \\
    |cdocsamp.tex| & sample main file \\
    |cdocsch1.tex| & sample include file \\
    |cdocsch2.tex| & sample include file \\
    |cdocspt3.tex| & sample part file \\
    |cdocspt4.tex| & sample part file \\
    |cdocsdrf.tex| & sample redirection file \\
    |cdocsfn1.tex| & sample redirection file \\
    |cdocsfn2.tex| & sample redirection file \\
    |childdoc.pdf| & manual
\end{tabular}
\end{center}
%
The distribution consists of the files
|README.txt|, |childdoc.ins| and |childdoc.dtx|.
%
\begin{itemize}
\item
Run (pdf)\LaTeX{} on |childdoc.dtx|
to compile the manual |childdoc.pdf| (this file).
\item
Run \LaTeX{} on |childdoc.ins| to create the definitions file |childdoc.def|
and the sample |cdocsamp.tex| with include files
|cdocsch1.tex|, |cdocsch2.tex|, |cdocspt3.tex|, |cdocspt4.tex|,
|cdocsdrf.tex|, |cdocsfn1.tex|, |cdocsfn2.tex|.
Then copy the file |childdoc.def| to an appropriate directory of your \LaTeX{}
distribution, e.g.\ \textit{texmf-root}|/tex/latex/childdoc|.
\end{itemize}

%%%%%%%%%%%%%%%%%%%%%%%%%%%%%%%%%%%%%%%%%%%%%%%%%%%%%%%%%%%%%%%%%%%%%%%%%%%%%%%%
\subsection{Related CTAN Packages}

There are several other packages which offer a similar functionality:
%
\begin{itemize}
\item
The packages
\href{http://ctan.org/pkg/docmute}{\textsf{docmute}},
\href{http://ctan.org/pkg/includex}{\textsf{includex}} and
\href{http://ctan.org/pkg/standalone}{\textsf{standalone}}
provide commands to include only the document body of
a child file thus allowing both files to be compiled individually.
\item
The packages \href{http://ctan.org/pkg/subdocs}{\textsf{subdocs}}
and \href{http://ctan.org/pkg/subfiles}{\textsf{subfiles}}
provide structures in which the main and child documents can be
encapsulated and allowing them to be compiled individually.
The inclusion mechanism is different from the conventional |\include|.
\item
The package \href{http://ctan.org/pkg/combine}{\textsf{combine}}
is an elaborate solution to combine several documents into one.
\end{itemize}
%
See also the CTAN topic \href{http://ctan.org/topic/subdocs}{\textsf{subdocs}}
for further related packages.
The present package differs from the above solutions in that
a document structure constructed with the conventional |\include| mechanism
just needs two extra commands at the top of every file
such that all constituent files can be compiled individually.

%%%%%%%%%%%%%%%%%%%%%%%%%%%%%%%%%%%%%%%%%%%%%%%%%%%%%%%%%%%%%%%%%%%%%%%%%%%%%%%%
%\subsection{Feature Suggestions}
%
%The following is a list of features which may be useful for future
%versions of this package:
%%
%\begin{itemize}
%\item
%\ldots
%\end{itemize}

%%%%%%%%%%%%%%%%%%%%%%%%%%%%%%%%%%%%%%%%%%%%%%%%%%%%%%%%%%%%%%%%%%%%%%%%%%%%%%%%
\subsection{Revision History}

%%%%%%%%%%%%%%%%%%%%%%%%%%%%%%%%%%%%%%%%
\paragraph{v2.0:} 2018/12/30

\begin{itemize}
\item
immediate forward processing
\item
added |\childdocby| mechanism
\item
manual restructured
\end{itemize}

%%%%%%%%%%%%%%%%%%%%%%%%%%%%%%%%%%%%%%%%
\paragraph{v1.6:} 2018/01/17

\begin{itemize}
\item
application for development of include files
\item
corrections to manual
\end{itemize}

%%%%%%%%%%%%%%%%%%%%%%%%%%%%%%%%%%%%%%%%
\paragraph{v1.5:} 2017/05/21

\begin{itemize}
\item
more complete structuring introduced
\item
|\childdocof| introduced
\item
|\childdoc| renamed to |\childdocmain|
\item
|\childredirect| renamed to |\childdocforward| and |\childdocforwardprefix|
and functionality expanded
\end{itemize}

%%%%%%%%%%%%%%%%%%%%%%%%%%%%%%%%%%%%%%%%
\paragraph{v1.0:} 2017/04/27

\begin{itemize}
\item
manual and install package
\item
first version published on CTAN
\end{itemize}

%%%%%%%%%%%%%%%%%%%%%%%%%%%%%%%%%%%%%%%%
\paragraph{v0.6:} 2017/04/26

\begin{itemize}
\item
redirection mechanism added
\end{itemize}

%%%%%%%%%%%%%%%%%%%%%%%%%%%%%%%%%%%%%%%%
\paragraph{v0.5:} 2017/04/26

\begin{itemize}
\item
functionality in definition file
\end{itemize}


%%%%%%%%%%%%%%%%%%%%%%%%%%%%%%%%%%%%%%%%%%%%%%%%%%%%%%%%%%%%%%%%%%%%%%%%%%%%%%%%
%%%%%%%%%%%%%%%%%%%%%%%%%%%%%%%%%%%%%%%%%%%%%%%%%%%%%%%%%%%%%%%%%%%%%%%%%%%%%%%%
%%%%%%%%%%%%%%%%%%%%%%%%%%%%%%%%%%%%%%%%%%%%%%%%%%%%%%%%%%%%%%%%%%%%%%%%%%%%%%%%
\appendix

\settowidth\MacroIndent{\rmfamily\scriptsize 000\ }

 \DocInput{childdoc.dtx}

\end{document}
%</driver>
% \fi
%
% %%%%%%%%%%%%%%%%%%%%%%%%%%%%%%%%%%%%%%%%%%%%%%%%%%%%%%%%%%%%%%%%%%%%%%%%%%%%%%
% %%%%%%%%%%%%%%%%%%%%%%%%%%%%%%%%%%%%%%%%%%%%%%%%%%%%%%%%%%%%%%%%%%%%%%%%%%%%%%
% \section{Sample}
%\iffalse
%<*samplemain>
%\fi
%
% The following presents a sample document
% with two chapters, two parts, a title page,
% a compile flag as well as three forwarding files to set the flag.
% It consists of eight |.tex| files:
% \begin{center}
% \begin{tabular}{ll}
% |cdocsamp.tex|&main file\\
% |cdocsch1.tex|&include file for chapter 1\\
% |cdocsch2.tex|&include file for chapter 2\\
% |cdocspt3.tex|&include file for part 3\\
% |cdocspt4.tex|&include file for part 4\\
% |cdocsdrf.tex|&forwarding file for main file in draft mode\\
% |cdocsfi1.tex|&forwarding file for final version of chapter 1\\
% |cdocsfi2.tex|&forwarding file for final version of chapter 2\\
% \end{tabular}
% \end{center}
% Each of the eight files can be compiled directly by the \LaTeX{} compiler.
%
% %%%%%%%%%%%%%%%%%%%%%%%%%%%%%%%%%%%%%%
% \paragraph{Main File.}
%
% The main file is called |cdocsamp.tex|.
%
% Load the \textsf{childdoc} definitions and
% declare the filename for the main document:
%    \begin{macrocode}
% \iffalse
%
% childdoc.dtx Copyright (C) 2017-2018 Niklas Beisert
%
% This work may be distributed and/or modified under the
% conditions of the LaTeX Project Public License, either version 1.3
% of this license or (at your option) any later version.
% The latest version of this license is in
%   http://www.latex-project.org/lppl.txt
% and version 1.3 or later is part of all distributions of LaTeX
% version 2005/12/01 or later.
%
% This work has the LPPL maintenance status `maintained'.
%
% The Current Maintainer of this work is Niklas Beisert.
%
% This work consists of the files childdoc.dtx and childdoc.ins
% and the derived files childdoc.def and cdocsamp.tex with
% cdocsch1.tex, cdocsch2.tex, cdocsdrf.tex, cdocsfn1.tex, cdocsfn2.tex.
%
%<package>\ifdefined\childdocmain\endinput\fi
%<package>\ProvidesFile{childdoc.def}[2018/12/30 v2.0 child document driver]
%<samplemain>\ProvidesFile{cdocsamp.tex}[2018/12/30 v2.0 sample for childdoc]
%<*driver>
%\ProvidesFile{childdoc.drv}[2018/12/30 v2.0 childdoc reference manual file]
\PassOptionsToClass{10pt,a4paper}{article}
\documentclass{ltxdoc}

\usepackage[margin=35mm]{geometry}
\usepackage{hyperref}
\usepackage{hyperxmp}
\usepackage[usenames]{color}

\hypersetup{colorlinks=true}
\hypersetup{pdfstartview=FitH}
\hypersetup{pdfpagemode=UseNone}
\hypersetup{pdfsource={}}
\hypersetup{pdflang={en-UK}}
\hypersetup{pdfcopyright={Copyright 2017-2018 Niklas Beisert.
  This work may be distributed and/or modified under the
  conditions of the LaTeX Project Public License, either version 1.3
  of this license or (at your option) any later version.}}
\hypersetup{pdflicenseurl={http://www.latex-project.org/lppl.txt}}
\hypersetup{pdfcontactaddress={ETH Zurich, ITP, HIT K,
  Wolfgang-Pauli-Strasse 27}}
\hypersetup{pdfcontactpostcode={8093}}
\hypersetup{pdfcontactcity={Zurich}}
\hypersetup{pdfcontactcountry={Switzerland}}
\hypersetup{pdfcontactemail={nbeisert@itp.phys.ethz.ch}}
\hypersetup{pdfcontacturl={http://people.phys.ethz.ch/\xmptilde nbeisert/}}

\newcommand{\secref}[1]{\hyperref[#1]{section \ref*{#1}}}

\parskip1ex
\parindent0pt
\let\olditemize\itemize
\def\itemize{\olditemize\parskip0pt}

\begin{document}

\title{The \textsf{childdoc} Package}
\hypersetup{pdftitle={The childdoc Package}}
\author{Niklas Beisert\\[2ex]
  Institut f\"ur Theoretische Physik\\
  Eidgen\"ossische Technische Hochschule Z\"urich\\
  Wolfgang-Pauli-Strasse 27, 8093 Z\"urich, Switzerland\\[1ex]
  \href{mailto:nbeisert@itp.phys.ethz.ch}
  {\texttt{nbeisert@itp.phys.ethz.ch}}}
\hypersetup{pdfauthor={Niklas Beisert}}
\hypersetup{pdfsubject={Manual for the LaTeX2e Package childdoc}}
\date{30 December 2018, \textsf{v2.0}}
\maketitle

\begin{abstract}\noindent
\textsf{childdoc} is a \LaTeXe{} package
that enables the direct compilation
of document sections included by |\include|
to individual files.
\end{abstract}

\begingroup
\parskip0ex
\tableofcontents
\endgroup

%%%%%%%%%%%%%%%%%%%%%%%%%%%%%%%%%%%%%%%%%%%%%%%%%%%%%%%%%%%%%%%%%%%%%%%%%%%%%%%%
%%%%%%%%%%%%%%%%%%%%%%%%%%%%%%%%%%%%%%%%%%%%%%%%%%%%%%%%%%%%%%%%%%%%%%%%%%%%%%%%
\section{Introduction}

\LaTeX{} provides a mechanism to structure a large document (such as a book)
into a main file and several child files (containing the chapters)
using the |\include| command.
This mechanism is beneficial for documents
which span hundreds of pages in order to
make the source file(s) more manageable.
Moreover, compilation can be restricted to
selected child files by means of the |\includeonly| command.
The latter feature can be used to reduce the compilation time while editing
(this was significantly more useful in the earlier days of \LaTeX{})
or to generate a smaller document which is easier to navigate.
Another application of |\includeonly| is to generate
documents consisting of selected parts of the complete document.

However, there are a few drawbacks of the plain |\include| mechanism:
\begin{itemize}
\item
The child files cannot be compiled on their own,
they can only be compiled via the main file.
A naive editing environment
(such as a text editor with an option
to have the current file processed by \LaTeX)
may require one to switch to the main file before compiling;
attempting to compile the child file produces errors.
\item
The main file must be modified (each time)
to adjust the |\includeonly| command
to the present needs. This easily leaves the main file in a messy state.
\item
The generated document will always carry the filename
of the main document. This is inconvenient if
several child files are to be compiled and
to be kept for distribution.
\end{itemize}

The present package provides a simple interface
to make child files individually compilable by \LaTeX{}.
Compiling a child file then has the same effect as compiling
the main file with an |\includeonly| command
to select the appropriate child.
Moreover the generated document will carry the name of the child
rather than the main file.
This resolves all three above issues.

This feature is meant to make the editing of books,
thesis documents and lecture notes somewhat more convenient.
However, the package can also be used efficiently for
composing a series of documents (such as exercise sheets)
which are typically distributed individually.
It then assists the author in generating the individual documents
(potentially in different versions)
as well as a document containing the collected series.
Another application is in developing style files
or other kinds of included material
where compilation of the style file could redirect
to a sample or test file.

%%%%%%%%%%%%%%%%%%%%%%%%%%%%%%%%%%%%%%%%%%%%%%%%%%%%%%%%%%%%%%%%%%%%%%%%%%%%%%%%
%%%%%%%%%%%%%%%%%%%%%%%%%%%%%%%%%%%%%%%%%%%%%%%%%%%%%%%%%%%%%%%%%%%%%%%%%%%%%%%%
\section{Usage}

First of all, the package \textsf{childdoc} is \emph{not} a standard
\LaTeXe{} |.sty| style file! Therefore it needs to be invoked in
a non-standard way.

%%%%%%%%%%%%%%%%%%%%%%%%%%%%%%%%%%%%%%%%%%%%%%%%%%%%%%%%%%%%%%%%%%%%%%%%%%%%%%%%
\subsection{Included Files}
\label{sec:include}

%%%%%%%%%%%%%%%%%%%%%%%%%%%%%%%%%%%%%%%%
\DescribeMacro{\childdocmain}
To use the package, add the commands
\begin{center}
\begin{tabular}{l}
|\input{childdoc.def}|\\
|\childdocmain{}|\\
\end{tabular}
\end{center}
at the very top of the main \LaTeX{} file,
in particular \emph{before} the |\documentclass| statement!
The argument of |\childdocmain| should be left empty
(but it must be present).

%%%%%%%%%%%%%%%%%%%%%%%%%%%%%%%%%%%%%%%%
\DescribeMacro{\childdocof}
Furthermore, add the commands
\begin{center}
\begin{tabular}{l}
|\input{childdoc.def}|\\
|\childdocof{|\textit{main}|}|\\
\end{tabular}
\end{center}
at the top of every child file \textit{child}
which is included by |\include{|\textit{child}|}|
from within the main file
(or at least for those files to be compiled individually).
The argument \textit{main} must be the filename of the main file.

There are a couple of
considerations in setting up the main and child documents:

%%%%%%%%%%%%%%%%%%%%%%%%%%%%%%%%%%%%%%%%
\paragraph{Restrictions.}

Please note the following restrictions:
\begin{itemize}
\item
|\childdocmain| must be called with one argument \textit{main}
to ensure compatibility with earlier version of the package.
It must either be empty (|\childdocmain{}|)
or precisely match the filename of the main file in which it is specified.
See \secref{sec:detection} for further information.
\item
The filename \textit{main} must be specified without the |.tex| extension.
\item
The filename \textit{main} is case sensitive
(even in case-insensitive file systems)
due to internal string comparison.
\item
The argument \textit{main} should be fully expanded, it cannot be a macro.
\item
Subdirectories and special characters should be avoided in filenames.
\item
The command |\childdocmain{|\textit{main}|}| must be followed by a whitespace.
It should not be followed immediately by another command
or by a comment mark `|%|'.
This is because the \TeX{} parser reads the token immediately following
the argument of |\childdocmain| and puts it
at the beginning of every child section;
however, a white\-space is ignored.
\end{itemize}

%%%%%%%%%%%%%%%%%%%%%%%%%%%%%%%%%%%%%%%%
\paragraph{Content of Main File.}

It is advisable to place all content in the child files included by |\include|.
Any output contained in the main file will appear in all child documents
unless suppressed manually;
it cannot be suppressed automatically by the |\includeonly| directive
and thus should normally be avoided.
A method to include some content in the main file
by means of conditional processing is described in \secref{sec:conditional}.

%%%%%%%%%%%%%%%%%%%%%%%%%%%%%%%%%%%%%%%%
\paragraph{Page Numbering.}

When only a part of the document is compiled,
the appropriate numbering of pages
(as well as other status parameters)
is determined from the |.aux| files.
The latter contain information from previous passes.
However this information needs to propagate through
all intermediate child documents.
Therefore the page numbering in child documents may well
be inconsistent until the complete document is compiled at least once.

A useful (if unconventional) way to always ensure a consistent
page numbering is to restart the numbering in each child document
and denote the pages by `\textit{child}|.|\textit{page}'
where \textit{child} represents the chapter/section number of the child file.
This can be achieved by the command
|\numberwithin{page}{|\textit{child}|}|
of the \textsf{amsmath} package
where \textit{child} can be |chapter| or |section|
depending on the chosen structuring.
Alternatively, one can modify the macro |\thepage| appropriately
and reset the counter |page| at the start of each child file.

%%%%%%%%%%%%%%%%%%%%%%%%%%%%%%%%%%%%%%%%%%%%%%%%%%%%%%%%%%%%%%%%%%%%%%%%%%%%%%%%
\subsection{Conditional Processing}
\label{sec:conditional}

The package provides a mechanism to compile different versions
of a document. To customise the versions further some conditional processing
can come in handy to distinguish which version is being compiled.
The package provides two macros to describe the compilation context:

%%%%%%%%%%%%%%%%%%%%%%%%%%%%%%%%%%%%%%%%
\DescribeMacro{\ifchilddoc}
The conditional |\ifchilddoc| distinguishes between the compilation of
child documents and the main document:
%
\begin{center}
|\ifchilddoc |\textit{child-code}| |[|\||else |\textit{main-code}]| \||fi|
\end{center}

%%%%%%%%%%%%%%%%%%%%%%%%%%%%%%%%%%%%%%%%
\DescribeMacro{\childdocname}
\DescribeMacro{\childdocjob}
The macro |\childdocname| contains the filename (without extension)
of the main or child file being processed.
Note that |\childdocjob| will always contain the name of the main file.

%%%%%%%%%%%%%%%%%%%%%%%%%%%%%%%%%%%%%%%%
\paragraph{Title Page.}

Conditional processing can be used to include a title or banner page
in the main document when proper precautions are taken.
Importantly, the code in the main file should ensure that the page counter
(as well as other status parameters which are stored in the |.aux| files)
takes the same value after the conditional processing.
Otherwise the page numbers may take divergent values
depending on which part is compiled.

For example, a title page could be declared by:
%
\begin{center}
\begin{tabular}{l}
|\ifchilddoc\||else|\\
|\addtocounter{page}{-1}|\\
\textit{code for title page}\\
|\newpage|\\
|\||fi|
\end{tabular}
\end{center}
%
A banner page for the child documents can be generated by:
%
\begin{center}
\begin{tabular}{l}
|\ifchilddoc|\\
|\addtocounter{page}{-1}|\\
\textit{code for banner page}\\
|\newpage|\\
|\||fi|
\end{tabular}
\end{center}
%
Here one could write a message such as:
\begin{center}
|This is the part \childdocname{} of \childdocjob{}.|
\end{center}

%%%%%%%%%%%%%%%%%%%%%%%%%%%%%%%%%%%%%%%%%%%%%%%%%%%%%%%%%%%%%%%%%%%%%%%%%%%%%%%%
\subsection{Flags}
\label{sec:flags}

The package makes it easy to generate different versions
of the main or child documents.
To this end compilation flags can be defined
and assigned different default values.
They will be particularly useful in conjunction
with the forwarding mechanism described in \secref{sec:forward}.

For example, it may be useful to have a flag |\version|
which can be set to |draft| or |final|.
The document source will contain some conditional code
depending on the value of |\version|.
Suppose further, the flag should default to |final| for the main file
and to |draft| for child files
which is a natural assignment for editing the document.
This is achieved by placing the following code
in the preamble of the main document
(below the |\childdocmain| directive):
%
\begin{center}
\begin{tabular}{l}
|\ifchilddoc|\\
|\providecommand{\version}{draft}|\\
|\||else|\\
|\providecommand{\version}{final}|\\
|\||fi|
\end{tabular}
\end{center}
%
The definition by |\providecommand| makes sure
that previous definitions are not overwritten.
Further statements |\providecommand{\version}{...}|
can thus be added before the above code to override it.

For the main file, one might add a line
(between |\childdocmain| and the above block)
%
\begin{center}
|%\ifchilddoc\||else\providecommand{\version}{draft}\||fi|
\end{center}
%
which can be uncommented to produce a draft version.
Likewise one can add a line to the very top of a child file
(above the |\childdocof{|\textit{main}|}| directive)
%
\begin{center}
|%\providecommand{\version}{final}|
\end{center}
%
which can be uncommented to produce the final version of this child document.

%%%%%%%%%%%%%%%%%%%%%%%%%%%%%%%%%%%%%%%%%%%%%%%%%%%%%%%%%%%%%%%%%%%%%%%%%%%%%%%%
\subsection{Forwarding}
\label{sec:forward}

Different versions of the main or child documents
using compilation flags as described in \secref{sec:flags}
can be (permanently) stored in different files
for convenient compilation, viewing and distribution.
To this end, the package defines a command
to pass on compilation to a different file:

%%%%%%%%%%%%%%%%%%%%%%%%%%%%%%%%%%%%%%%%
\DescribeMacro{\childdocforward}
The command |\childdocforward| redirects processing to
another source file:
%
\begin{center}
\begin{tabular}{l}
|\input{childdoc.def}|\\
|\childdocforward[|\textit{main}|]{|\textit{dest}|}|\\
\end{tabular}
\end{center}
%
The argument \textit{dest} is the destination file
(without extension).
It should be the main file or one of the child files.
Note that further \textsf{childdoc} directives
such as |\childdocof| and |\childdocforward|
in the indicated file will be processed in this form.
The optional argument \textit{main}
passes on directly to the main file \textit{main}
while pretending to compile the child \textit{dest}.
This form behaves as if \textit{dest}
issues |\childdocof{|\textit{main}|}| right away,
and no further \textsf{childdoc} directives will be processed.

%%%%%%%%%%%%%%%%%%%%%%%%%%%%%%%%%%%%%%%%
\DescribeMacro{\...prefix}
In the alternative form |\childdocforwardprefix|,
%
\begin{center}
\begin{tabular}{l}
|\input{childdoc.def}|\\
|\childdocforwardprefix[|\textit{main}|]{|\textit{prefix}|}{|\textit{dest}|}|
\end{tabular}
\end{center}
%
the destination file is determined by a pattern
depending on the current file:
To make this work, the current file must be called
`{\textit{prefix}\hspace{0.2em}\textit{suffix}}'
with \textit{prefix} matching precisely the argument.
Processing is then passed on to the file
`{\textit{dest}\hspace{0.2em}\textit{suffix}}'.
Surely, the same effect is achieved by
directly specifying the
argument `{\textit{dest}\hspace{0.2em}\textit{suffix}}'
in the first form.
However, that requires to set up a different file
for each child. With the alternative form of the command
all these files can have exactly the same content
which simplifies setting them up and maintaining them.

For example, the following file |draft.tex|
with a compilation flag |\version| as described in \secref{sec:flags}
compiles the main document as a draft:
%
\begin{center}
\begin{tabular}{l}
|\def\version{draft}|\\
|\input{childdoc.def}|\\
|\childdocforward{|\textit{main}|}|
\end{tabular}
\end{center}
%
Likewise, the following files |final|\textit{nn}|.tex|
compile the final version of the child document
|child|\textit{nn}|.tex|:
%
\begin{center}
\begin{tabular}{l}
|\def\version{final}|\\
|\input{childdoc.def}|\\
|\childdocforwardprefix{final}{child}|
\end{tabular}
\end{center}
%

Note that when several versions of a main file and/or of each child file
are to be generated, it may be convenient to set up a |Makefile| or
shell script to automatise the process.

%%%%%%%%%%%%%%%%%%%%%%%%%%%%%%%%%%%%%%%%%%%%%%%%%%%%%%%%%%%%%%%%%%%%%%%%%%%%%%%%
\subsection{Command Line Processing}
\label{sec:commandline}

The effect of redirection files can also be achieved by invoking
the \LaTeX{} compiler with a more elaborate command line.
Most conveniently this should be done as part
of a shell script or a |Makefile|.

When using \textsf{childdoc} in the main file, the following
command lines effectively perform a redirection
(note that depending on the shell being used,
backslashes may have to be doubled: `|\|' $\to$ `|\\|'):
%
\begin{center}
|... -jobname "|\textit{target}|" |\\|"|[\textit{flags}]%
|\input{childdoc.def}\childdocforward[|\textit{main}|]{|\textit{dest}|}"|
\end{center}
%
Here \textit{target} is the name of the output file,
\textit{main} is the name of the main file
and \textit{dest} is the name of the main or child file to be processed
(all filenames without extensions).
The optional argument \textit{main} can be omitted
if \textit{main} matches \textit{dest}.
Optionally, compilation \textit{flags} can be defined via |\def| commands.
This command line makes the \TeX{} engine believe
it is compiling the file \textit{target}
whose content is specified as the latter parameter.
The provided code then forwards the processing to
\textit{main} or \textit{dest} as described in \secref{sec:forward}.

%%%%%%%%%%%%%%%%%%%%%%%%%%%%%%%%%%%%%%%%%%%%%%%%%%%%%%%%%%%%%%%%%%%%%%%%%%%%%%%%
\subsection{Include by Input}
\label{sec:input}

Including child documents by |\include| has some restrictions by design.
Most notably, the content of a child document always occupies
its own set of pages; pages cannot be shared between child documents.
Usually, this behaviour makes perfect sense
because each child document contain an essential part of the document.
However, in some situations it may be desirable to compose
a document from a collection of parts
without having mandatory page breaks between then.
For this case, the package
provides a mechanism to include parts
by |\input| which can also be processed individually.
However, by construction this mechanism
requires manual handling of the content to be output.

%%%%%%%%%%%%%%%%%%%%%%%%%%%%%%%%%%%%%%%%
\DescribeMacro{\ifchilddocmanual}
The main file should be prepared as usual, see \secref{sec:include}.
However, the document body must make a distinction
between processing of an individual part and of the main document, e.g.:
%
\begin{center}
\begin{tabular}{l}
|\ifchilddocmanual|\\
|\input{\childdocname}|\\
|\||else|\\
\textit{document body with }|\input{|\textit{part}|}|\\
|\||fi|
\end{tabular}
\end{center}
%
The conditional |\ifchilddocmanual| is true whenever
a part to be included by |\input| is being compiled,
and the name of the part is stored in |\childdocname|.

%%%%%%%%%%%%%%%%%%%%%%%%%%%%%%%%%%%%%%%%
\DescribeMacro{\childdocby}
Each part to be included by |\input| should start with:
%
\begin{center}
\begin{tabular}{l}
|\input{childdoc.def}|\\
|\childdocby{|\textit{main}|}|\\
\end{tabular}
\end{center}
%
The directive |\childdocby| is similar to |\childdocof|
described in \secref{sec:include},
but the subsequent selection of content must be done manually.
To that end, both |\ifchilddoc| and |\ifchilddocmanual|
will be true upon processing of a part,
and the name of the part is stored in |\childdocname|.
Note that |\jobname| will be set to the filename of the current part
so that each part receives an individual |.aux| file
that does not interfere with the |.aux| file(s) of the main document.
This behaviour can be altered by the alternative form
|\childdocby[*]{|\textit{main}|}| (with a non-empty optional argument)
which uses the |.aux| file of the main document
by setting |\jobname| to \textit{main}.

%%%%%%%%%%%%%%%%%%%%%%%%%%%%%%%%%%%%%%%%%%%%%%%%%%%%%%%%%%%%%%%%%%%%%%%%%%%%%%%%
\subsection{Driver Development}
\label{sec:driver}

The \textsf{childdoc} mechanism can also be use for the development
of definition files such as \LaTeX{} styles or classes.
This case differs from the above setup with multiple parts
included by |\include| in that no |\includeonly| should be invoked.
This can be achieved by starting the include file
(before |\ProvidesPackage|) with:
%
\begin{center}
\begin{tabular}{l}
|\input{childdoc.def}|\\
|\childdocforward{|\textit{main}|}|\\
\end{tabular}
\end{center}
%
or alternatively with:
%
\begin{center}
\begin{tabular}{l}
|\input{childdoc.def}|\\
|\childdocby{|\textit{main}|}|\\
\end{tabular}
\end{center}
%
Both forms have slightly different effects as described above.
The main file is prepared as usual, see \secref{sec:include}.

%%%%%%%%%%%%%%%%%%%%%%%%%%%%%%%%%%%%%%%%%%%%%%%%%%%%%%%%%%%%%%%%%%%%%%%%%%%%%%%%
\subsection{Legacy Detection}
\label{sec:detection}

The directive |\childdocmain| in the main file can detect
whether the complete document or merely a child is to be compiled
even without using the directive |\childdocof|.
This method is deprecated because it is less robust
and there is no compelling reason to use it;
it is merely provided for backward compatibility
and it may be removed in future versions.

If the detection mechanism is to be used,
it is mandatory to correctly specify
the filename of the main file as the argument of |\childdocmain|:
%
\begin{center}
\begin{tabular}{l}
|\input{childdoc.def}|\\
|\childdocmain{|\textit{main}|}|\\
\end{tabular}
\end{center}
%
If |\jobname| does not match the argument \textit{main} of |\childdocmain|,
it is assumed that |\jobname| points to the child file to be compiled.
When using |\childdocmain| with the main file specified as argument,
it suffices to start a child file
with just |\input{|\textit{main}|}|
without loading of the package and using |\childdocof|.
If instead all processing is done
with the appropriate \textsf{childdoc} directives,
the argument of \textit{main} of |\childdocmain| can be empty.

An alternative version of the command line processing described
in \secref{sec:commandline} using the detection mechanism reads:
%
\begin{center}
|... -jobname "|\textit{target}|" "|[\textit{flags}]%
[|\def\jobname{|\textit{dest}|}|]|\input{|\textit{main}|}"|
\end{center}

%%%%%%%%%%%%%%%%%%%%%%%%%%%%%%%%%%%%%%%%%%%%%%%%%%%%%%%%%%%%%%%%%%%%%%%%%%%%%%%%
\subsection{Manual Code}
\label{sec:manual}

In case one cannot be certain whether the definitions file |childdoc.def|
is installed on the target \TeX{} distribution
and one prefers not to ship it,
it is conceivable to paste a few relevant commands into the sources.

To that end, drop all statements |\input{childdoc.def}|
and perform the replacements as outlined below.
Instead of |\childdocmain{|\textit{main}|}| add the following code
to the top of the main file:
%
\begin{center}
\begin{tabular}{l}
|\||ifdefined\childdocname\endinput\||fi\newif\ifchilddoc|\\
|\edef\childdocname{\scantokens\expandafter{\jobname\noexpand}}|\\
|\def\childdocmain{|\textit{main}|}\||ifx\childdocmain\childdocname\||else|\\
|\childdoctrue\includeonly{\childdocname}\let\jobname\childdocmain\||fi|\\
\end{tabular}
\end{center}
%
Instead of |\childdocof{|\textit{main}|}| just include the main file
at the top of each child file:
%
\begin{center}
|\input{|\textit{main}|}|
\end{center}
%
A simple redirection |\childdocforward{|\textit{dest}|}| is achieved by:
%
\begin{center}
|\def\jobname{|\textit{dest}|}\input{\jobname}|
\end{center}
%
The redirection with prefix
|\childdocforwardprefix[|\textit{prefix}|]{|\textit{dest}|}|
is accomplished by:
%
\begin{center}
\begin{tabular}{l}
|{\edef\jobname{\scantokens\expandafter{\jobname\noexpand}}|\\
|\def\redirectjob |\textit{prefix}|#1~~~{\gdef\jobname{|\textit{dest}|#1}}|\\
|\expandafter\redirectjob\jobname~~~}\input{\jobname}|
\end{tabular}
\end{center}

In an alternative approach,
child documents can be compiled by a specific command line
without additional code or specific definitions:
%
\begin{center}
|... -jobname "|\textit{target}|" "|[\textit{flags}]%
|\includeonly{|\textit{dest}|}\input{|\textit{main}|}"|
\end{center}
%

%%%%%%%%%%%%%%%%%%%%%%%%%%%%%%%%%%%%%%%%%%%%%%%%%%%%%%%%%%%%%%%%%%%%%%%%%%%%%%%%
%%%%%%%%%%%%%%%%%%%%%%%%%%%%%%%%%%%%%%%%%%%%%%%%%%%%%%%%%%%%%%%%%%%%%%%%%%%%%%%%
\section{Information}

%%%%%%%%%%%%%%%%%%%%%%%%%%%%%%%%%%%%%%%%%%%%%%%%%%%%%%%%%%%%%%%%%%%%%%%%%%%%%%%%
\subsection{Copyright}

Copyright \copyright{} 2017--2018 Niklas Beisert

This work may be distributed and/or modified under the
conditions of the \LaTeX{} Project Public License, either version 1.3
of this license or (at your option) any later version.
The latest version of this license is in
  \url{http://www.latex-project.org/lppl.txt}
and version 1.3 or later is part of all distributions of \LaTeX{}
version 2005/12/01 or later.

This work has the LPPL maintenance status `maintained'.

The Current Maintainer of this work is Niklas Beisert.

This work consists of the files |README.txt|, |childdoc.ins| and |childdoc.dtx|
as well as the derived files |childdoc.def|, |cdocsamp.tex|
with |cdocsch1.tex|, |cdocsch2.tex|, |cdocspt3.tex|, |cdocspt4.tex|,
|cdocsdrf.tex|, |cdocsfn1.tex|, |cdocsfn2.tex|
as well as |childdoc.pdf|.

%%%%%%%%%%%%%%%%%%%%%%%%%%%%%%%%%%%%%%%%%%%%%%%%%%%%%%%%%%%%%%%%%%%%%%%%%%%%%%%%
\subsection{Files and Installation}

The package consists of the files:
%
\begin{center}
\begin{tabular}{ll}
    |README.txt|   & readme file \\
    |childdoc.ins| & installation file \\
    |childdoc.dtx| & source file \\
    |childdoc.def| & definition file \\
    |cdocsamp.tex| & sample main file \\
    |cdocsch1.tex| & sample include file \\
    |cdocsch2.tex| & sample include file \\
    |cdocspt3.tex| & sample part file \\
    |cdocspt4.tex| & sample part file \\
    |cdocsdrf.tex| & sample redirection file \\
    |cdocsfn1.tex| & sample redirection file \\
    |cdocsfn2.tex| & sample redirection file \\
    |childdoc.pdf| & manual
\end{tabular}
\end{center}
%
The distribution consists of the files
|README.txt|, |childdoc.ins| and |childdoc.dtx|.
%
\begin{itemize}
\item
Run (pdf)\LaTeX{} on |childdoc.dtx|
to compile the manual |childdoc.pdf| (this file).
\item
Run \LaTeX{} on |childdoc.ins| to create the definitions file |childdoc.def|
and the sample |cdocsamp.tex| with include files
|cdocsch1.tex|, |cdocsch2.tex|, |cdocspt3.tex|, |cdocspt4.tex|,
|cdocsdrf.tex|, |cdocsfn1.tex|, |cdocsfn2.tex|.
Then copy the file |childdoc.def| to an appropriate directory of your \LaTeX{}
distribution, e.g.\ \textit{texmf-root}|/tex/latex/childdoc|.
\end{itemize}

%%%%%%%%%%%%%%%%%%%%%%%%%%%%%%%%%%%%%%%%%%%%%%%%%%%%%%%%%%%%%%%%%%%%%%%%%%%%%%%%
\subsection{Related CTAN Packages}

There are several other packages which offer a similar functionality:
%
\begin{itemize}
\item
The packages
\href{http://ctan.org/pkg/docmute}{\textsf{docmute}},
\href{http://ctan.org/pkg/includex}{\textsf{includex}} and
\href{http://ctan.org/pkg/standalone}{\textsf{standalone}}
provide commands to include only the document body of
a child file thus allowing both files to be compiled individually.
\item
The packages \href{http://ctan.org/pkg/subdocs}{\textsf{subdocs}}
and \href{http://ctan.org/pkg/subfiles}{\textsf{subfiles}}
provide structures in which the main and child documents can be
encapsulated and allowing them to be compiled individually.
The inclusion mechanism is different from the conventional |\include|.
\item
The package \href{http://ctan.org/pkg/combine}{\textsf{combine}}
is an elaborate solution to combine several documents into one.
\end{itemize}
%
See also the CTAN topic \href{http://ctan.org/topic/subdocs}{\textsf{subdocs}}
for further related packages.
The present package differs from the above solutions in that
a document structure constructed with the conventional |\include| mechanism
just needs two extra commands at the top of every file
such that all constituent files can be compiled individually.

%%%%%%%%%%%%%%%%%%%%%%%%%%%%%%%%%%%%%%%%%%%%%%%%%%%%%%%%%%%%%%%%%%%%%%%%%%%%%%%%
%\subsection{Feature Suggestions}
%
%The following is a list of features which may be useful for future
%versions of this package:
%%
%\begin{itemize}
%\item
%\ldots
%\end{itemize}

%%%%%%%%%%%%%%%%%%%%%%%%%%%%%%%%%%%%%%%%%%%%%%%%%%%%%%%%%%%%%%%%%%%%%%%%%%%%%%%%
\subsection{Revision History}

%%%%%%%%%%%%%%%%%%%%%%%%%%%%%%%%%%%%%%%%
\paragraph{v2.0:} 2018/12/30

\begin{itemize}
\item
immediate forward processing
\item
added |\childdocby| mechanism
\item
manual restructured
\end{itemize}

%%%%%%%%%%%%%%%%%%%%%%%%%%%%%%%%%%%%%%%%
\paragraph{v1.6:} 2018/01/17

\begin{itemize}
\item
application for development of include files
\item
corrections to manual
\end{itemize}

%%%%%%%%%%%%%%%%%%%%%%%%%%%%%%%%%%%%%%%%
\paragraph{v1.5:} 2017/05/21

\begin{itemize}
\item
more complete structuring introduced
\item
|\childdocof| introduced
\item
|\childdoc| renamed to |\childdocmain|
\item
|\childredirect| renamed to |\childdocforward| and |\childdocforwardprefix|
and functionality expanded
\end{itemize}

%%%%%%%%%%%%%%%%%%%%%%%%%%%%%%%%%%%%%%%%
\paragraph{v1.0:} 2017/04/27

\begin{itemize}
\item
manual and install package
\item
first version published on CTAN
\end{itemize}

%%%%%%%%%%%%%%%%%%%%%%%%%%%%%%%%%%%%%%%%
\paragraph{v0.6:} 2017/04/26

\begin{itemize}
\item
redirection mechanism added
\end{itemize}

%%%%%%%%%%%%%%%%%%%%%%%%%%%%%%%%%%%%%%%%
\paragraph{v0.5:} 2017/04/26

\begin{itemize}
\item
functionality in definition file
\end{itemize}


%%%%%%%%%%%%%%%%%%%%%%%%%%%%%%%%%%%%%%%%%%%%%%%%%%%%%%%%%%%%%%%%%%%%%%%%%%%%%%%%
%%%%%%%%%%%%%%%%%%%%%%%%%%%%%%%%%%%%%%%%%%%%%%%%%%%%%%%%%%%%%%%%%%%%%%%%%%%%%%%%
%%%%%%%%%%%%%%%%%%%%%%%%%%%%%%%%%%%%%%%%%%%%%%%%%%%%%%%%%%%%%%%%%%%%%%%%%%%%%%%%
\appendix

\settowidth\MacroIndent{\rmfamily\scriptsize 000\ }

 \DocInput{childdoc.dtx}

\end{document}
%</driver>
% \fi
%
% %%%%%%%%%%%%%%%%%%%%%%%%%%%%%%%%%%%%%%%%%%%%%%%%%%%%%%%%%%%%%%%%%%%%%%%%%%%%%%
% %%%%%%%%%%%%%%%%%%%%%%%%%%%%%%%%%%%%%%%%%%%%%%%%%%%%%%%%%%%%%%%%%%%%%%%%%%%%%%
% \section{Sample}
%\iffalse
%<*samplemain>
%\fi
%
% The following presents a sample document
% with two chapters, two parts, a title page,
% a compile flag as well as three forwarding files to set the flag.
% It consists of eight |.tex| files:
% \begin{center}
% \begin{tabular}{ll}
% |cdocsamp.tex|&main file\\
% |cdocsch1.tex|&include file for chapter 1\\
% |cdocsch2.tex|&include file for chapter 2\\
% |cdocspt3.tex|&include file for part 3\\
% |cdocspt4.tex|&include file for part 4\\
% |cdocsdrf.tex|&forwarding file for main file in draft mode\\
% |cdocsfi1.tex|&forwarding file for final version of chapter 1\\
% |cdocsfi2.tex|&forwarding file for final version of chapter 2\\
% \end{tabular}
% \end{center}
% Each of the eight files can be compiled directly by the \LaTeX{} compiler.
%
% %%%%%%%%%%%%%%%%%%%%%%%%%%%%%%%%%%%%%%
% \paragraph{Main File.}
%
% The main file is called |cdocsamp.tex|.
%
% Load the \textsf{childdoc} definitions and
% declare the filename for the main document:
%    \begin{macrocode}
\input{childdoc.def}
\childdocmain{}
%    \end{macrocode}

% Optional override for |\version| flag:
%    \begin{macrocode}
%%\ifchilddoc\else\providecommand{\version}{draft}\fi
%    \end{macrocode}

% Define the default values for the |\version| flag
% (|final| for the main file and |draft| for childs):
%    \begin{macrocode}
\ifchilddoc
\providecommand{\version}{draft}
\else
\providecommand{\version}{final}
\fi
%    \end{macrocode}

% Load the standard document class:
%    \begin{macrocode}
\documentclass[12pt]{article}
%    \end{macrocode}

% Start the document body:
%    \begin{macrocode}
\begin{document}
%    \end{macrocode}

% Declare a title page.
% Print title, part of document being processed and version flag:
%    \begin{macrocode}
\addtocounter{page}{-1}
\begin{center}
{\LARGE\bfseries{}childdoc example\par}
\vspace{1cm}
\ifchilddoc
\ifchilddocmanual part\else chapter\fi:
`\childdocname' of `\childdocjob'\par
\else
main document: `\childdocjob'\par
\fi
version: \version\par
\end{center}
\newpage
%    \end{macrocode}

% Manually include selected file,
% otherwise process as usual:
%    \begin{macrocode}
\ifchilddocmanual
\section*{part `\childdocname'}
\input{\childdocname}
\else
%    \end{macrocode}

% Include the two chapters:
%    \begin{macrocode}
\include{cdocsch1}
\include{cdocsch2}
%    \end{macrocode}

% Include the two parts unless only chapters should be displayed:
%    \begin{macrocode}
\ifchilddoc\else
\section{part three}
\input{cdocspt3}
\section{part four}
\input{cdocspt4}
\fi
%    \end{macrocode}

% Process as usual until here:
%    \begin{macrocode}
\fi
%    \end{macrocode}

% End of document body:
%    \begin{macrocode}
\end{document}
%    \end{macrocode}
%\iffalse
%</samplemain>
%\fi
%
% %%%%%%%%%%%%%%%%%%%%%%%%%%%%%%%%%%%%%%
% \paragraph{Chapter Include Files.}
%
% The include files are called |cdocsch1.tex| and |cdocsch2.tex|.
%
%\iffalse
%<*samplechap1|samplechap2>
%\fi

% Optional override for |\version| flag:
%    \begin{macrocode}
%%\providecommand{\version}{final}
%    \end{macrocode}

% Include the main document:
%    \begin{macrocode}
\input{childdoc.def}
\childdocof{cdocsamp}
%    \end{macrocode}

%\iffalse
%</samplechap1|samplechap2>
%\fi
%
%\iffalse
%<*samplechap1>
%\fi
% Some text for chapter 1:
%    \begin{macrocode}
\section{one}
some text in chapter one
%    \end{macrocode}

%\iffalse
%</samplechap1>
%\fi
% Some text for chapter 2:
%\iffalse
%<*samplechap2>
%\fi
%    \begin{macrocode}
\section{two}
more text in chapter two
%    \end{macrocode}

%\iffalse
%</samplechap2>
%\fi
%
% %%%%%%%%%%%%%%%%%%%%%%%%%%%%%%%%%%%%%%
% \paragraph{Part Include Files.}
%
% The include files are called |cdocspt3.tex| and |cdocspt4.tex|.
%
%\iffalse
%<*samplepart3|samplepart4>
%\fi

% Optional override for |\version| flag:
%    \begin{macrocode}
%%\providecommand{\version}{final}
%    \end{macrocode}

% Include the main document:
%    \begin{macrocode}
\input{childdoc.def}
\childdocby{cdocsamp}
%    \end{macrocode}

%\iffalse
%</samplepart3|samplepart4>
%\fi
%
%\iffalse
%<*samplepart3>
%\fi
% Some text for part 3:
%    \begin{macrocode}
some text in part three
%    \end{macrocode}

%\iffalse
%</samplepart3>
%\fi
% Some text for part 4:
%\iffalse
%<*samplepart4>
%\fi
%    \begin{macrocode}
more text in part four
%    \end{macrocode}

%\iffalse
%</samplepart4>
%\fi
%
% %%%%%%%%%%%%%%%%%%%%%%%%%%%%%%%%%%%%%%
% \paragraph{Forwarding for a Complete Draft.}
%
% The following forwarding file |cdocsdrf.tex|
% compiles the main document in draft mode:
%\iffalse
%<*sampledraft>
%\fi
%    \begin{macrocode}
\def\version{draft}
\input{childdoc.def}
\childdocforward{cdocsamp}
%    \end{macrocode}

%\iffalse
%</sampledraft>
%\fi
%
% %%%%%%%%%%%%%%%%%%%%%%%%%%%%%%%%%%%%%%
% \paragraph{Forwarding for Final Version of the Chapters.}
%
% The following forwarding files |cdocsfn1.tex| and |cdocsfn2.tex|
% (with identical content)
% compile the final versions of the child documents
% |cdocsch1.tex| and |cdocsch2.tex|, respectively:
%\iffalse
%<*samplefinal>
%\fi
%    \begin{macrocode}
\def\version{final}
\input{childdoc.def}
\childdocforwardprefix[cdocsamp]{cdocsfn}{cdocsch}
%    \end{macrocode}

%\iffalse
%</samplefinal>
%\fi
%
% %%%%%%%%%%%%%%%%%%%%%%%%%%%%%%%%%%%%%%
% \paragraph{Command Line Processing.}
%
% The following three command lines generate the output files
% |cdocscld|, |cdocscl1| and |cdocscl2|
% which should be identical to
% |cdocsdrf|, |cdocsch1| and |cdocsfn2|, respectively:
% \begin{center}
% \begin{tabular}{l}
% |latex -jobname cdocscld \|\\
% |  "\def\version{draft}\input{childdoc.def}\childdocforward{cdocsamp}"|\\
% |latex -jobname cdocscl1 \|\\
% |  "\input{childdoc.def}\childdocforward[cdocsamp]{cdocsch1}"|\\
% |latex -jobname cdocscl2 \|\\
% |  "\def\version{final}\input{childdoc.def}\childdocforward{cdocsch2}"|
% \end{tabular}
% \end{center}
% Note that the trailing backslash on each first line
% merely continues the input to the second line
% (for convenient cut ant paste).
% Furthermore, the command |latex| can be replaced by any
% of its alternative versions such as |pdflatex|.
%
% %%%%%%%%%%%%%%%%%%%%%%%%%%%%%%%%%%%%%%%%%%%%%%%%%%%%%%%%%%%%%%%%%%%%%%%%%%%%%%
% %%%%%%%%%%%%%%%%%%%%%%%%%%%%%%%%%%%%%%%%%%%%%%%%%%%%%%%%%%%%%%%%%%%%%%%%%%%%%%
% \section{Implementation}
%\iffalse
%<*package>
%\fi
%
% This section describes the definitions file |childdoc.def|.

% The definitions cannot be loaded using |\usepackage| or |\RequirePackage|
% which has a mechanism to prevent loading a style file more than once.
% When loading the definitions by means of |\input|
% multiple instances have to be prevented manually:
%\iffalse
%This code needs to be before the `\ProvidesFile' directive
%which is defined at the beginning of this file.
%Therefore it is also placed there and commented out here.
%</package>
%<*discard>
%\fi
%    \begin{macrocode}
\ifdefined\childdocmain\endinput\fi
%    \end{macrocode}
%\iffalse
%</discard>
%<*package>
%\fi
%
% \macro{\ifchilddoc}
% \macro{\ifchilddocmanual}
% The conditional |\ifchilddoc| tells whether a
% child (true) or main (false) document is being compiled.
% The conditional |\ifchilddocmanual| tells whether
% the |\includeonly| mechanism is used (false) or
% the selection of child files must be performed manually (true).
% The definitions initialise to false:
%    \begin{macrocode}
\newif\ifchilddoc
\newif\ifchilddocmanual
%    \end{macrocode}

% \macro{\childdocname}
% \macro{\childdocjob}
% The macro |\childdocname| stores the name of the main document
% to be compiled. The macro |\childdocjob| stores the name of
% the document on which the \LaTeX{} compiler was originally invoked.
% The content of |\jobname| cannot be compared
% to filenames specified in the source due to different catcodes.
% The following code rescans |\jobname|, stores the result
% in |\childdocname| and saves a copy in |\childdocjob|:
%    \begin{macrocode}
\edef\childdocname{\scantokens\expandafter{\jobname\noexpand}}
\let\childdocjob\childdocname
%    \end{macrocode}

% \macro{\childdocdisable}
% The macro |\childdocdisable| prevents the main file
% from being processed more than once.
% At this stage, the main document command |\childdocmain|
% is assumed to be called once again where it should do nothing.
% Any subsequent call to it should prevent
% a secondary processing of the main document
% It overwrites the forwarding commands
% |\childdocof| and |\childdocforward|
% with empty macros to prevent further inclusions of the main document:
%    \begin{macrocode}
\newcommand{\childdocdisable}
{
  \renewcommand{\childdocmain}[1]{\renewcommand{\childdocmain}[1]{\endinput}}
  \renewcommand{\childdocof}[1]{}
  \renewcommand{\childdocby}[2][]{}
  \renewcommand{\childdocforward}[2][]{}
  \renewcommand{\childdocdisable}{}
}
%    \end{macrocode}

% \macro{\childdocmain}
% The macro |\childdocmain| is to be called at the top of the main file
% with nothing or the main filename (without extension) as argument.
% First, it breaks loops.
% If the argument is not empty and does not match |\childdocname|
% (which is set by the first inclusion of |childdoc.def|),
% |\ifchilddoc| is set to true, |\includeonly| is applied to the child file
% and |\jobname| is set to the main file
% (for proper handling of |.aux| files):
%    \begin{macrocode}
\newcommand{\childdocmain}[1]
{
  \childdocdisable\childdocmain{}
  \if?#1?\else
    \begingroup
      \def\childdoctmp{#1}
      \ifx\childdoctmp\childdocname
        \def\childdoctmp{}
      \else
        \def\childdoctmp
        {
          \childdoctrue
          \includeonly{\childdocname}
          \def\childdocjob{#1}
          \def\jobname{#1}
        }
      \fi
      \expandafter
    \endgroup
    \childdoctmp
  \fi
}
%    \end{macrocode}

% \macro{\childdocof}
% The command |\childdocof| redirects
% compilation to the main file |#1|.
%    \begin{macrocode}
\newcommand{\childdocof}[1]
{
  \childdocdisable
  \childdoctrue
  \includeonly{\childdocname}
  \def\jobname{#1}
  \def\childdocjob{#1}
  \input{#1}
}
%    \end{macrocode}

% \macro{\childdocby}
% The command |\childdocby| ....
%    \begin{macrocode}
\newcommand{\childdocby}[2][]
{
  \childdocdisable
  \childdoctrue
  \childdocmanualtrue
  \if?#1?\else
    \def\jobname{#2}
  \fi
  \def\childdocjob{#2}
  \input{#2}
  \endinput
}
%    \end{macrocode}

% \macro{\childdocforward}
% The command |\childdocforward| redirects
% compilation to the main file or
% (if the optional argument is given) a child file.
% Parameters are set as if the main file
% or a child file starting with |\childdocof| was compiled.
% Then compilation is handed over to the main file:
%    \begin{macrocode}
\newcommand{\childdocforward}[2][]
{
  \begingroup
    \if?#1?
      \def\childdoctmp
      {
        \def\childdocname{#2}
        \def\childdocjob{#2}
        \def\jobname{#2}
        \input{#2}
        \endinput
      }
    \else
      \def\childdoctmp
      {
        \childdocdisable
        \def\childdocname{#2}
        \childdoctrue
        \includeonly{#2}
        \def\childdocjob{#1}
        \def\jobname{#1}
        \input{#1}
        \endinput
      }
    \fi
    \expandafter
  \endgroup
  \childdoctmp
}
%    \end{macrocode}

% \macro{\childdocforwardprefix}
% The command |\childdocforwardprefix| redirects
% compilation to the main or a child file by means of a pattern.
% The prefix |#1| in the current filename is replaced by |#2|
% and the suffix of the current filename is kept
% (it is assumed that the filename does not contain the substring `|~~~|'
% which is used as a delimiter).
% Compilation is handed over to the new file by |\childdocforward|:
%    \begin{macrocode}
\newcommand{\childdocforwardprefix}[3][]
{
  \begingroup
    \def\childdocextract #2##1~~~{\def\childdoctmp{\childdocforward[#1]{#3##1}}}
    \expandafter\childdocextract\childdocname~~~
    \expandafter
  \endgroup
  \childdoctmp
}
%    \end{macrocode}

% \macro{\childdoc}
% The deprecated macro |\childdoc| is a legacy version of |\childdocmain|:
%    \begin{macrocode}
\newcommand{\childdoc}{\childdocmain}
%    \end{macrocode}

% \macro{\childdocredirect}
% The deprecated macro |\childdocredirect| is a legacy version
% of |\childdocforward| and |\childdocforwardprefix|:
%    \begin{macrocode}
\newcommand{\childdocredirect}[2][]
{
  \begingroup
    \if?#1?
      \def\childdoctmp{\childdocforward{#2}}
    \else
      \def\childdoctmp{\childdocforwardprefix{#1}{#2}}
    \fi
    \expandafter
  \endgroup
  \childdoctmp
}
%    \end{macrocode}

%\iffalse
%</package>
%\fi
%
\endinput

\childdocmain{}
%    \end{macrocode}

% Optional override for |\version| flag:
%    \begin{macrocode}
%%\ifchilddoc\else\providecommand{\version}{draft}\fi
%    \end{macrocode}

% Define the default values for the |\version| flag
% (|final| for the main file and |draft| for childs):
%    \begin{macrocode}
\ifchilddoc
\providecommand{\version}{draft}
\else
\providecommand{\version}{final}
\fi
%    \end{macrocode}

% Load the standard document class:
%    \begin{macrocode}
\documentclass[12pt]{article}
%    \end{macrocode}

% Start the document body:
%    \begin{macrocode}
\begin{document}
%    \end{macrocode}

% Declare a title page.
% Print title, part of document being processed and version flag:
%    \begin{macrocode}
\addtocounter{page}{-1}
\begin{center}
{\LARGE\bfseries{}childdoc example\par}
\vspace{1cm}
\ifchilddoc
\ifchilddocmanual part\else chapter\fi:
`\childdocname' of `\childdocjob'\par
\else
main document: `\childdocjob'\par
\fi
version: \version\par
\end{center}
\newpage
%    \end{macrocode}

% Manually include selected file,
% otherwise process as usual:
%    \begin{macrocode}
\ifchilddocmanual
\section*{part `\childdocname'}
\input{\childdocname}
\else
%    \end{macrocode}

% Include the two chapters:
%    \begin{macrocode}
\include{cdocsch1}
\include{cdocsch2}
%    \end{macrocode}

% Include the two parts unless only chapters should be displayed:
%    \begin{macrocode}
\ifchilddoc\else
\section{part three}
\input{cdocspt3}
\section{part four}
\input{cdocspt4}
\fi
%    \end{macrocode}

% Process as usual until here:
%    \begin{macrocode}
\fi
%    \end{macrocode}

% End of document body:
%    \begin{macrocode}
\end{document}
%    \end{macrocode}
%\iffalse
%</samplemain>
%\fi
%
% %%%%%%%%%%%%%%%%%%%%%%%%%%%%%%%%%%%%%%
% \paragraph{Chapter Include Files.}
%
% The include files are called |cdocsch1.tex| and |cdocsch2.tex|.
%
%\iffalse
%<*samplechap1|samplechap2>
%\fi

% Optional override for |\version| flag:
%    \begin{macrocode}
%%\providecommand{\version}{final}
%    \end{macrocode}

% Include the main document:
%    \begin{macrocode}
% \iffalse
%
% childdoc.dtx Copyright (C) 2017-2018 Niklas Beisert
%
% This work may be distributed and/or modified under the
% conditions of the LaTeX Project Public License, either version 1.3
% of this license or (at your option) any later version.
% The latest version of this license is in
%   http://www.latex-project.org/lppl.txt
% and version 1.3 or later is part of all distributions of LaTeX
% version 2005/12/01 or later.
%
% This work has the LPPL maintenance status `maintained'.
%
% The Current Maintainer of this work is Niklas Beisert.
%
% This work consists of the files childdoc.dtx and childdoc.ins
% and the derived files childdoc.def and cdocsamp.tex with
% cdocsch1.tex, cdocsch2.tex, cdocsdrf.tex, cdocsfn1.tex, cdocsfn2.tex.
%
%<package>\ifdefined\childdocmain\endinput\fi
%<package>\ProvidesFile{childdoc.def}[2018/12/30 v2.0 child document driver]
%<samplemain>\ProvidesFile{cdocsamp.tex}[2018/12/30 v2.0 sample for childdoc]
%<*driver>
%\ProvidesFile{childdoc.drv}[2018/12/30 v2.0 childdoc reference manual file]
\PassOptionsToClass{10pt,a4paper}{article}
\documentclass{ltxdoc}

\usepackage[margin=35mm]{geometry}
\usepackage{hyperref}
\usepackage{hyperxmp}
\usepackage[usenames]{color}

\hypersetup{colorlinks=true}
\hypersetup{pdfstartview=FitH}
\hypersetup{pdfpagemode=UseNone}
\hypersetup{pdfsource={}}
\hypersetup{pdflang={en-UK}}
\hypersetup{pdfcopyright={Copyright 2017-2018 Niklas Beisert.
  This work may be distributed and/or modified under the
  conditions of the LaTeX Project Public License, either version 1.3
  of this license or (at your option) any later version.}}
\hypersetup{pdflicenseurl={http://www.latex-project.org/lppl.txt}}
\hypersetup{pdfcontactaddress={ETH Zurich, ITP, HIT K,
  Wolfgang-Pauli-Strasse 27}}
\hypersetup{pdfcontactpostcode={8093}}
\hypersetup{pdfcontactcity={Zurich}}
\hypersetup{pdfcontactcountry={Switzerland}}
\hypersetup{pdfcontactemail={nbeisert@itp.phys.ethz.ch}}
\hypersetup{pdfcontacturl={http://people.phys.ethz.ch/\xmptilde nbeisert/}}

\newcommand{\secref}[1]{\hyperref[#1]{section \ref*{#1}}}

\parskip1ex
\parindent0pt
\let\olditemize\itemize
\def\itemize{\olditemize\parskip0pt}

\begin{document}

\title{The \textsf{childdoc} Package}
\hypersetup{pdftitle={The childdoc Package}}
\author{Niklas Beisert\\[2ex]
  Institut f\"ur Theoretische Physik\\
  Eidgen\"ossische Technische Hochschule Z\"urich\\
  Wolfgang-Pauli-Strasse 27, 8093 Z\"urich, Switzerland\\[1ex]
  \href{mailto:nbeisert@itp.phys.ethz.ch}
  {\texttt{nbeisert@itp.phys.ethz.ch}}}
\hypersetup{pdfauthor={Niklas Beisert}}
\hypersetup{pdfsubject={Manual for the LaTeX2e Package childdoc}}
\date{30 December 2018, \textsf{v2.0}}
\maketitle

\begin{abstract}\noindent
\textsf{childdoc} is a \LaTeXe{} package
that enables the direct compilation
of document sections included by |\include|
to individual files.
\end{abstract}

\begingroup
\parskip0ex
\tableofcontents
\endgroup

%%%%%%%%%%%%%%%%%%%%%%%%%%%%%%%%%%%%%%%%%%%%%%%%%%%%%%%%%%%%%%%%%%%%%%%%%%%%%%%%
%%%%%%%%%%%%%%%%%%%%%%%%%%%%%%%%%%%%%%%%%%%%%%%%%%%%%%%%%%%%%%%%%%%%%%%%%%%%%%%%
\section{Introduction}

\LaTeX{} provides a mechanism to structure a large document (such as a book)
into a main file and several child files (containing the chapters)
using the |\include| command.
This mechanism is beneficial for documents
which span hundreds of pages in order to
make the source file(s) more manageable.
Moreover, compilation can be restricted to
selected child files by means of the |\includeonly| command.
The latter feature can be used to reduce the compilation time while editing
(this was significantly more useful in the earlier days of \LaTeX{})
or to generate a smaller document which is easier to navigate.
Another application of |\includeonly| is to generate
documents consisting of selected parts of the complete document.

However, there are a few drawbacks of the plain |\include| mechanism:
\begin{itemize}
\item
The child files cannot be compiled on their own,
they can only be compiled via the main file.
A naive editing environment
(such as a text editor with an option
to have the current file processed by \LaTeX)
may require one to switch to the main file before compiling;
attempting to compile the child file produces errors.
\item
The main file must be modified (each time)
to adjust the |\includeonly| command
to the present needs. This easily leaves the main file in a messy state.
\item
The generated document will always carry the filename
of the main document. This is inconvenient if
several child files are to be compiled and
to be kept for distribution.
\end{itemize}

The present package provides a simple interface
to make child files individually compilable by \LaTeX{}.
Compiling a child file then has the same effect as compiling
the main file with an |\includeonly| command
to select the appropriate child.
Moreover the generated document will carry the name of the child
rather than the main file.
This resolves all three above issues.

This feature is meant to make the editing of books,
thesis documents and lecture notes somewhat more convenient.
However, the package can also be used efficiently for
composing a series of documents (such as exercise sheets)
which are typically distributed individually.
It then assists the author in generating the individual documents
(potentially in different versions)
as well as a document containing the collected series.
Another application is in developing style files
or other kinds of included material
where compilation of the style file could redirect
to a sample or test file.

%%%%%%%%%%%%%%%%%%%%%%%%%%%%%%%%%%%%%%%%%%%%%%%%%%%%%%%%%%%%%%%%%%%%%%%%%%%%%%%%
%%%%%%%%%%%%%%%%%%%%%%%%%%%%%%%%%%%%%%%%%%%%%%%%%%%%%%%%%%%%%%%%%%%%%%%%%%%%%%%%
\section{Usage}

First of all, the package \textsf{childdoc} is \emph{not} a standard
\LaTeXe{} |.sty| style file! Therefore it needs to be invoked in
a non-standard way.

%%%%%%%%%%%%%%%%%%%%%%%%%%%%%%%%%%%%%%%%%%%%%%%%%%%%%%%%%%%%%%%%%%%%%%%%%%%%%%%%
\subsection{Included Files}
\label{sec:include}

%%%%%%%%%%%%%%%%%%%%%%%%%%%%%%%%%%%%%%%%
\DescribeMacro{\childdocmain}
To use the package, add the commands
\begin{center}
\begin{tabular}{l}
|\input{childdoc.def}|\\
|\childdocmain{}|\\
\end{tabular}
\end{center}
at the very top of the main \LaTeX{} file,
in particular \emph{before} the |\documentclass| statement!
The argument of |\childdocmain| should be left empty
(but it must be present).

%%%%%%%%%%%%%%%%%%%%%%%%%%%%%%%%%%%%%%%%
\DescribeMacro{\childdocof}
Furthermore, add the commands
\begin{center}
\begin{tabular}{l}
|\input{childdoc.def}|\\
|\childdocof{|\textit{main}|}|\\
\end{tabular}
\end{center}
at the top of every child file \textit{child}
which is included by |\include{|\textit{child}|}|
from within the main file
(or at least for those files to be compiled individually).
The argument \textit{main} must be the filename of the main file.

There are a couple of
considerations in setting up the main and child documents:

%%%%%%%%%%%%%%%%%%%%%%%%%%%%%%%%%%%%%%%%
\paragraph{Restrictions.}

Please note the following restrictions:
\begin{itemize}
\item
|\childdocmain| must be called with one argument \textit{main}
to ensure compatibility with earlier version of the package.
It must either be empty (|\childdocmain{}|)
or precisely match the filename of the main file in which it is specified.
See \secref{sec:detection} for further information.
\item
The filename \textit{main} must be specified without the |.tex| extension.
\item
The filename \textit{main} is case sensitive
(even in case-insensitive file systems)
due to internal string comparison.
\item
The argument \textit{main} should be fully expanded, it cannot be a macro.
\item
Subdirectories and special characters should be avoided in filenames.
\item
The command |\childdocmain{|\textit{main}|}| must be followed by a whitespace.
It should not be followed immediately by another command
or by a comment mark `|%|'.
This is because the \TeX{} parser reads the token immediately following
the argument of |\childdocmain| and puts it
at the beginning of every child section;
however, a white\-space is ignored.
\end{itemize}

%%%%%%%%%%%%%%%%%%%%%%%%%%%%%%%%%%%%%%%%
\paragraph{Content of Main File.}

It is advisable to place all content in the child files included by |\include|.
Any output contained in the main file will appear in all child documents
unless suppressed manually;
it cannot be suppressed automatically by the |\includeonly| directive
and thus should normally be avoided.
A method to include some content in the main file
by means of conditional processing is described in \secref{sec:conditional}.

%%%%%%%%%%%%%%%%%%%%%%%%%%%%%%%%%%%%%%%%
\paragraph{Page Numbering.}

When only a part of the document is compiled,
the appropriate numbering of pages
(as well as other status parameters)
is determined from the |.aux| files.
The latter contain information from previous passes.
However this information needs to propagate through
all intermediate child documents.
Therefore the page numbering in child documents may well
be inconsistent until the complete document is compiled at least once.

A useful (if unconventional) way to always ensure a consistent
page numbering is to restart the numbering in each child document
and denote the pages by `\textit{child}|.|\textit{page}'
where \textit{child} represents the chapter/section number of the child file.
This can be achieved by the command
|\numberwithin{page}{|\textit{child}|}|
of the \textsf{amsmath} package
where \textit{child} can be |chapter| or |section|
depending on the chosen structuring.
Alternatively, one can modify the macro |\thepage| appropriately
and reset the counter |page| at the start of each child file.

%%%%%%%%%%%%%%%%%%%%%%%%%%%%%%%%%%%%%%%%%%%%%%%%%%%%%%%%%%%%%%%%%%%%%%%%%%%%%%%%
\subsection{Conditional Processing}
\label{sec:conditional}

The package provides a mechanism to compile different versions
of a document. To customise the versions further some conditional processing
can come in handy to distinguish which version is being compiled.
The package provides two macros to describe the compilation context:

%%%%%%%%%%%%%%%%%%%%%%%%%%%%%%%%%%%%%%%%
\DescribeMacro{\ifchilddoc}
The conditional |\ifchilddoc| distinguishes between the compilation of
child documents and the main document:
%
\begin{center}
|\ifchilddoc |\textit{child-code}| |[|\||else |\textit{main-code}]| \||fi|
\end{center}

%%%%%%%%%%%%%%%%%%%%%%%%%%%%%%%%%%%%%%%%
\DescribeMacro{\childdocname}
\DescribeMacro{\childdocjob}
The macro |\childdocname| contains the filename (without extension)
of the main or child file being processed.
Note that |\childdocjob| will always contain the name of the main file.

%%%%%%%%%%%%%%%%%%%%%%%%%%%%%%%%%%%%%%%%
\paragraph{Title Page.}

Conditional processing can be used to include a title or banner page
in the main document when proper precautions are taken.
Importantly, the code in the main file should ensure that the page counter
(as well as other status parameters which are stored in the |.aux| files)
takes the same value after the conditional processing.
Otherwise the page numbers may take divergent values
depending on which part is compiled.

For example, a title page could be declared by:
%
\begin{center}
\begin{tabular}{l}
|\ifchilddoc\||else|\\
|\addtocounter{page}{-1}|\\
\textit{code for title page}\\
|\newpage|\\
|\||fi|
\end{tabular}
\end{center}
%
A banner page for the child documents can be generated by:
%
\begin{center}
\begin{tabular}{l}
|\ifchilddoc|\\
|\addtocounter{page}{-1}|\\
\textit{code for banner page}\\
|\newpage|\\
|\||fi|
\end{tabular}
\end{center}
%
Here one could write a message such as:
\begin{center}
|This is the part \childdocname{} of \childdocjob{}.|
\end{center}

%%%%%%%%%%%%%%%%%%%%%%%%%%%%%%%%%%%%%%%%%%%%%%%%%%%%%%%%%%%%%%%%%%%%%%%%%%%%%%%%
\subsection{Flags}
\label{sec:flags}

The package makes it easy to generate different versions
of the main or child documents.
To this end compilation flags can be defined
and assigned different default values.
They will be particularly useful in conjunction
with the forwarding mechanism described in \secref{sec:forward}.

For example, it may be useful to have a flag |\version|
which can be set to |draft| or |final|.
The document source will contain some conditional code
depending on the value of |\version|.
Suppose further, the flag should default to |final| for the main file
and to |draft| for child files
which is a natural assignment for editing the document.
This is achieved by placing the following code
in the preamble of the main document
(below the |\childdocmain| directive):
%
\begin{center}
\begin{tabular}{l}
|\ifchilddoc|\\
|\providecommand{\version}{draft}|\\
|\||else|\\
|\providecommand{\version}{final}|\\
|\||fi|
\end{tabular}
\end{center}
%
The definition by |\providecommand| makes sure
that previous definitions are not overwritten.
Further statements |\providecommand{\version}{...}|
can thus be added before the above code to override it.

For the main file, one might add a line
(between |\childdocmain| and the above block)
%
\begin{center}
|%\ifchilddoc\||else\providecommand{\version}{draft}\||fi|
\end{center}
%
which can be uncommented to produce a draft version.
Likewise one can add a line to the very top of a child file
(above the |\childdocof{|\textit{main}|}| directive)
%
\begin{center}
|%\providecommand{\version}{final}|
\end{center}
%
which can be uncommented to produce the final version of this child document.

%%%%%%%%%%%%%%%%%%%%%%%%%%%%%%%%%%%%%%%%%%%%%%%%%%%%%%%%%%%%%%%%%%%%%%%%%%%%%%%%
\subsection{Forwarding}
\label{sec:forward}

Different versions of the main or child documents
using compilation flags as described in \secref{sec:flags}
can be (permanently) stored in different files
for convenient compilation, viewing and distribution.
To this end, the package defines a command
to pass on compilation to a different file:

%%%%%%%%%%%%%%%%%%%%%%%%%%%%%%%%%%%%%%%%
\DescribeMacro{\childdocforward}
The command |\childdocforward| redirects processing to
another source file:
%
\begin{center}
\begin{tabular}{l}
|\input{childdoc.def}|\\
|\childdocforward[|\textit{main}|]{|\textit{dest}|}|\\
\end{tabular}
\end{center}
%
The argument \textit{dest} is the destination file
(without extension).
It should be the main file or one of the child files.
Note that further \textsf{childdoc} directives
such as |\childdocof| and |\childdocforward|
in the indicated file will be processed in this form.
The optional argument \textit{main}
passes on directly to the main file \textit{main}
while pretending to compile the child \textit{dest}.
This form behaves as if \textit{dest}
issues |\childdocof{|\textit{main}|}| right away,
and no further \textsf{childdoc} directives will be processed.

%%%%%%%%%%%%%%%%%%%%%%%%%%%%%%%%%%%%%%%%
\DescribeMacro{\...prefix}
In the alternative form |\childdocforwardprefix|,
%
\begin{center}
\begin{tabular}{l}
|\input{childdoc.def}|\\
|\childdocforwardprefix[|\textit{main}|]{|\textit{prefix}|}{|\textit{dest}|}|
\end{tabular}
\end{center}
%
the destination file is determined by a pattern
depending on the current file:
To make this work, the current file must be called
`{\textit{prefix}\hspace{0.2em}\textit{suffix}}'
with \textit{prefix} matching precisely the argument.
Processing is then passed on to the file
`{\textit{dest}\hspace{0.2em}\textit{suffix}}'.
Surely, the same effect is achieved by
directly specifying the
argument `{\textit{dest}\hspace{0.2em}\textit{suffix}}'
in the first form.
However, that requires to set up a different file
for each child. With the alternative form of the command
all these files can have exactly the same content
which simplifies setting them up and maintaining them.

For example, the following file |draft.tex|
with a compilation flag |\version| as described in \secref{sec:flags}
compiles the main document as a draft:
%
\begin{center}
\begin{tabular}{l}
|\def\version{draft}|\\
|\input{childdoc.def}|\\
|\childdocforward{|\textit{main}|}|
\end{tabular}
\end{center}
%
Likewise, the following files |final|\textit{nn}|.tex|
compile the final version of the child document
|child|\textit{nn}|.tex|:
%
\begin{center}
\begin{tabular}{l}
|\def\version{final}|\\
|\input{childdoc.def}|\\
|\childdocforwardprefix{final}{child}|
\end{tabular}
\end{center}
%

Note that when several versions of a main file and/or of each child file
are to be generated, it may be convenient to set up a |Makefile| or
shell script to automatise the process.

%%%%%%%%%%%%%%%%%%%%%%%%%%%%%%%%%%%%%%%%%%%%%%%%%%%%%%%%%%%%%%%%%%%%%%%%%%%%%%%%
\subsection{Command Line Processing}
\label{sec:commandline}

The effect of redirection files can also be achieved by invoking
the \LaTeX{} compiler with a more elaborate command line.
Most conveniently this should be done as part
of a shell script or a |Makefile|.

When using \textsf{childdoc} in the main file, the following
command lines effectively perform a redirection
(note that depending on the shell being used,
backslashes may have to be doubled: `|\|' $\to$ `|\\|'):
%
\begin{center}
|... -jobname "|\textit{target}|" |\\|"|[\textit{flags}]%
|\input{childdoc.def}\childdocforward[|\textit{main}|]{|\textit{dest}|}"|
\end{center}
%
Here \textit{target} is the name of the output file,
\textit{main} is the name of the main file
and \textit{dest} is the name of the main or child file to be processed
(all filenames without extensions).
The optional argument \textit{main} can be omitted
if \textit{main} matches \textit{dest}.
Optionally, compilation \textit{flags} can be defined via |\def| commands.
This command line makes the \TeX{} engine believe
it is compiling the file \textit{target}
whose content is specified as the latter parameter.
The provided code then forwards the processing to
\textit{main} or \textit{dest} as described in \secref{sec:forward}.

%%%%%%%%%%%%%%%%%%%%%%%%%%%%%%%%%%%%%%%%%%%%%%%%%%%%%%%%%%%%%%%%%%%%%%%%%%%%%%%%
\subsection{Include by Input}
\label{sec:input}

Including child documents by |\include| has some restrictions by design.
Most notably, the content of a child document always occupies
its own set of pages; pages cannot be shared between child documents.
Usually, this behaviour makes perfect sense
because each child document contain an essential part of the document.
However, in some situations it may be desirable to compose
a document from a collection of parts
without having mandatory page breaks between then.
For this case, the package
provides a mechanism to include parts
by |\input| which can also be processed individually.
However, by construction this mechanism
requires manual handling of the content to be output.

%%%%%%%%%%%%%%%%%%%%%%%%%%%%%%%%%%%%%%%%
\DescribeMacro{\ifchilddocmanual}
The main file should be prepared as usual, see \secref{sec:include}.
However, the document body must make a distinction
between processing of an individual part and of the main document, e.g.:
%
\begin{center}
\begin{tabular}{l}
|\ifchilddocmanual|\\
|\input{\childdocname}|\\
|\||else|\\
\textit{document body with }|\input{|\textit{part}|}|\\
|\||fi|
\end{tabular}
\end{center}
%
The conditional |\ifchilddocmanual| is true whenever
a part to be included by |\input| is being compiled,
and the name of the part is stored in |\childdocname|.

%%%%%%%%%%%%%%%%%%%%%%%%%%%%%%%%%%%%%%%%
\DescribeMacro{\childdocby}
Each part to be included by |\input| should start with:
%
\begin{center}
\begin{tabular}{l}
|\input{childdoc.def}|\\
|\childdocby{|\textit{main}|}|\\
\end{tabular}
\end{center}
%
The directive |\childdocby| is similar to |\childdocof|
described in \secref{sec:include},
but the subsequent selection of content must be done manually.
To that end, both |\ifchilddoc| and |\ifchilddocmanual|
will be true upon processing of a part,
and the name of the part is stored in |\childdocname|.
Note that |\jobname| will be set to the filename of the current part
so that each part receives an individual |.aux| file
that does not interfere with the |.aux| file(s) of the main document.
This behaviour can be altered by the alternative form
|\childdocby[*]{|\textit{main}|}| (with a non-empty optional argument)
which uses the |.aux| file of the main document
by setting |\jobname| to \textit{main}.

%%%%%%%%%%%%%%%%%%%%%%%%%%%%%%%%%%%%%%%%%%%%%%%%%%%%%%%%%%%%%%%%%%%%%%%%%%%%%%%%
\subsection{Driver Development}
\label{sec:driver}

The \textsf{childdoc} mechanism can also be use for the development
of definition files such as \LaTeX{} styles or classes.
This case differs from the above setup with multiple parts
included by |\include| in that no |\includeonly| should be invoked.
This can be achieved by starting the include file
(before |\ProvidesPackage|) with:
%
\begin{center}
\begin{tabular}{l}
|\input{childdoc.def}|\\
|\childdocforward{|\textit{main}|}|\\
\end{tabular}
\end{center}
%
or alternatively with:
%
\begin{center}
\begin{tabular}{l}
|\input{childdoc.def}|\\
|\childdocby{|\textit{main}|}|\\
\end{tabular}
\end{center}
%
Both forms have slightly different effects as described above.
The main file is prepared as usual, see \secref{sec:include}.

%%%%%%%%%%%%%%%%%%%%%%%%%%%%%%%%%%%%%%%%%%%%%%%%%%%%%%%%%%%%%%%%%%%%%%%%%%%%%%%%
\subsection{Legacy Detection}
\label{sec:detection}

The directive |\childdocmain| in the main file can detect
whether the complete document or merely a child is to be compiled
even without using the directive |\childdocof|.
This method is deprecated because it is less robust
and there is no compelling reason to use it;
it is merely provided for backward compatibility
and it may be removed in future versions.

If the detection mechanism is to be used,
it is mandatory to correctly specify
the filename of the main file as the argument of |\childdocmain|:
%
\begin{center}
\begin{tabular}{l}
|\input{childdoc.def}|\\
|\childdocmain{|\textit{main}|}|\\
\end{tabular}
\end{center}
%
If |\jobname| does not match the argument \textit{main} of |\childdocmain|,
it is assumed that |\jobname| points to the child file to be compiled.
When using |\childdocmain| with the main file specified as argument,
it suffices to start a child file
with just |\input{|\textit{main}|}|
without loading of the package and using |\childdocof|.
If instead all processing is done
with the appropriate \textsf{childdoc} directives,
the argument of \textit{main} of |\childdocmain| can be empty.

An alternative version of the command line processing described
in \secref{sec:commandline} using the detection mechanism reads:
%
\begin{center}
|... -jobname "|\textit{target}|" "|[\textit{flags}]%
[|\def\jobname{|\textit{dest}|}|]|\input{|\textit{main}|}"|
\end{center}

%%%%%%%%%%%%%%%%%%%%%%%%%%%%%%%%%%%%%%%%%%%%%%%%%%%%%%%%%%%%%%%%%%%%%%%%%%%%%%%%
\subsection{Manual Code}
\label{sec:manual}

In case one cannot be certain whether the definitions file |childdoc.def|
is installed on the target \TeX{} distribution
and one prefers not to ship it,
it is conceivable to paste a few relevant commands into the sources.

To that end, drop all statements |\input{childdoc.def}|
and perform the replacements as outlined below.
Instead of |\childdocmain{|\textit{main}|}| add the following code
to the top of the main file:
%
\begin{center}
\begin{tabular}{l}
|\||ifdefined\childdocname\endinput\||fi\newif\ifchilddoc|\\
|\edef\childdocname{\scantokens\expandafter{\jobname\noexpand}}|\\
|\def\childdocmain{|\textit{main}|}\||ifx\childdocmain\childdocname\||else|\\
|\childdoctrue\includeonly{\childdocname}\let\jobname\childdocmain\||fi|\\
\end{tabular}
\end{center}
%
Instead of |\childdocof{|\textit{main}|}| just include the main file
at the top of each child file:
%
\begin{center}
|\input{|\textit{main}|}|
\end{center}
%
A simple redirection |\childdocforward{|\textit{dest}|}| is achieved by:
%
\begin{center}
|\def\jobname{|\textit{dest}|}\input{\jobname}|
\end{center}
%
The redirection with prefix
|\childdocforwardprefix[|\textit{prefix}|]{|\textit{dest}|}|
is accomplished by:
%
\begin{center}
\begin{tabular}{l}
|{\edef\jobname{\scantokens\expandafter{\jobname\noexpand}}|\\
|\def\redirectjob |\textit{prefix}|#1~~~{\gdef\jobname{|\textit{dest}|#1}}|\\
|\expandafter\redirectjob\jobname~~~}\input{\jobname}|
\end{tabular}
\end{center}

In an alternative approach,
child documents can be compiled by a specific command line
without additional code or specific definitions:
%
\begin{center}
|... -jobname "|\textit{target}|" "|[\textit{flags}]%
|\includeonly{|\textit{dest}|}\input{|\textit{main}|}"|
\end{center}
%

%%%%%%%%%%%%%%%%%%%%%%%%%%%%%%%%%%%%%%%%%%%%%%%%%%%%%%%%%%%%%%%%%%%%%%%%%%%%%%%%
%%%%%%%%%%%%%%%%%%%%%%%%%%%%%%%%%%%%%%%%%%%%%%%%%%%%%%%%%%%%%%%%%%%%%%%%%%%%%%%%
\section{Information}

%%%%%%%%%%%%%%%%%%%%%%%%%%%%%%%%%%%%%%%%%%%%%%%%%%%%%%%%%%%%%%%%%%%%%%%%%%%%%%%%
\subsection{Copyright}

Copyright \copyright{} 2017--2018 Niklas Beisert

This work may be distributed and/or modified under the
conditions of the \LaTeX{} Project Public License, either version 1.3
of this license or (at your option) any later version.
The latest version of this license is in
  \url{http://www.latex-project.org/lppl.txt}
and version 1.3 or later is part of all distributions of \LaTeX{}
version 2005/12/01 or later.

This work has the LPPL maintenance status `maintained'.

The Current Maintainer of this work is Niklas Beisert.

This work consists of the files |README.txt|, |childdoc.ins| and |childdoc.dtx|
as well as the derived files |childdoc.def|, |cdocsamp.tex|
with |cdocsch1.tex|, |cdocsch2.tex|, |cdocspt3.tex|, |cdocspt4.tex|,
|cdocsdrf.tex|, |cdocsfn1.tex|, |cdocsfn2.tex|
as well as |childdoc.pdf|.

%%%%%%%%%%%%%%%%%%%%%%%%%%%%%%%%%%%%%%%%%%%%%%%%%%%%%%%%%%%%%%%%%%%%%%%%%%%%%%%%
\subsection{Files and Installation}

The package consists of the files:
%
\begin{center}
\begin{tabular}{ll}
    |README.txt|   & readme file \\
    |childdoc.ins| & installation file \\
    |childdoc.dtx| & source file \\
    |childdoc.def| & definition file \\
    |cdocsamp.tex| & sample main file \\
    |cdocsch1.tex| & sample include file \\
    |cdocsch2.tex| & sample include file \\
    |cdocspt3.tex| & sample part file \\
    |cdocspt4.tex| & sample part file \\
    |cdocsdrf.tex| & sample redirection file \\
    |cdocsfn1.tex| & sample redirection file \\
    |cdocsfn2.tex| & sample redirection file \\
    |childdoc.pdf| & manual
\end{tabular}
\end{center}
%
The distribution consists of the files
|README.txt|, |childdoc.ins| and |childdoc.dtx|.
%
\begin{itemize}
\item
Run (pdf)\LaTeX{} on |childdoc.dtx|
to compile the manual |childdoc.pdf| (this file).
\item
Run \LaTeX{} on |childdoc.ins| to create the definitions file |childdoc.def|
and the sample |cdocsamp.tex| with include files
|cdocsch1.tex|, |cdocsch2.tex|, |cdocspt3.tex|, |cdocspt4.tex|,
|cdocsdrf.tex|, |cdocsfn1.tex|, |cdocsfn2.tex|.
Then copy the file |childdoc.def| to an appropriate directory of your \LaTeX{}
distribution, e.g.\ \textit{texmf-root}|/tex/latex/childdoc|.
\end{itemize}

%%%%%%%%%%%%%%%%%%%%%%%%%%%%%%%%%%%%%%%%%%%%%%%%%%%%%%%%%%%%%%%%%%%%%%%%%%%%%%%%
\subsection{Related CTAN Packages}

There are several other packages which offer a similar functionality:
%
\begin{itemize}
\item
The packages
\href{http://ctan.org/pkg/docmute}{\textsf{docmute}},
\href{http://ctan.org/pkg/includex}{\textsf{includex}} and
\href{http://ctan.org/pkg/standalone}{\textsf{standalone}}
provide commands to include only the document body of
a child file thus allowing both files to be compiled individually.
\item
The packages \href{http://ctan.org/pkg/subdocs}{\textsf{subdocs}}
and \href{http://ctan.org/pkg/subfiles}{\textsf{subfiles}}
provide structures in which the main and child documents can be
encapsulated and allowing them to be compiled individually.
The inclusion mechanism is different from the conventional |\include|.
\item
The package \href{http://ctan.org/pkg/combine}{\textsf{combine}}
is an elaborate solution to combine several documents into one.
\end{itemize}
%
See also the CTAN topic \href{http://ctan.org/topic/subdocs}{\textsf{subdocs}}
for further related packages.
The present package differs from the above solutions in that
a document structure constructed with the conventional |\include| mechanism
just needs two extra commands at the top of every file
such that all constituent files can be compiled individually.

%%%%%%%%%%%%%%%%%%%%%%%%%%%%%%%%%%%%%%%%%%%%%%%%%%%%%%%%%%%%%%%%%%%%%%%%%%%%%%%%
%\subsection{Feature Suggestions}
%
%The following is a list of features which may be useful for future
%versions of this package:
%%
%\begin{itemize}
%\item
%\ldots
%\end{itemize}

%%%%%%%%%%%%%%%%%%%%%%%%%%%%%%%%%%%%%%%%%%%%%%%%%%%%%%%%%%%%%%%%%%%%%%%%%%%%%%%%
\subsection{Revision History}

%%%%%%%%%%%%%%%%%%%%%%%%%%%%%%%%%%%%%%%%
\paragraph{v2.0:} 2018/12/30

\begin{itemize}
\item
immediate forward processing
\item
added |\childdocby| mechanism
\item
manual restructured
\end{itemize}

%%%%%%%%%%%%%%%%%%%%%%%%%%%%%%%%%%%%%%%%
\paragraph{v1.6:} 2018/01/17

\begin{itemize}
\item
application for development of include files
\item
corrections to manual
\end{itemize}

%%%%%%%%%%%%%%%%%%%%%%%%%%%%%%%%%%%%%%%%
\paragraph{v1.5:} 2017/05/21

\begin{itemize}
\item
more complete structuring introduced
\item
|\childdocof| introduced
\item
|\childdoc| renamed to |\childdocmain|
\item
|\childredirect| renamed to |\childdocforward| and |\childdocforwardprefix|
and functionality expanded
\end{itemize}

%%%%%%%%%%%%%%%%%%%%%%%%%%%%%%%%%%%%%%%%
\paragraph{v1.0:} 2017/04/27

\begin{itemize}
\item
manual and install package
\item
first version published on CTAN
\end{itemize}

%%%%%%%%%%%%%%%%%%%%%%%%%%%%%%%%%%%%%%%%
\paragraph{v0.6:} 2017/04/26

\begin{itemize}
\item
redirection mechanism added
\end{itemize}

%%%%%%%%%%%%%%%%%%%%%%%%%%%%%%%%%%%%%%%%
\paragraph{v0.5:} 2017/04/26

\begin{itemize}
\item
functionality in definition file
\end{itemize}


%%%%%%%%%%%%%%%%%%%%%%%%%%%%%%%%%%%%%%%%%%%%%%%%%%%%%%%%%%%%%%%%%%%%%%%%%%%%%%%%
%%%%%%%%%%%%%%%%%%%%%%%%%%%%%%%%%%%%%%%%%%%%%%%%%%%%%%%%%%%%%%%%%%%%%%%%%%%%%%%%
%%%%%%%%%%%%%%%%%%%%%%%%%%%%%%%%%%%%%%%%%%%%%%%%%%%%%%%%%%%%%%%%%%%%%%%%%%%%%%%%
\appendix

\settowidth\MacroIndent{\rmfamily\scriptsize 000\ }

 \DocInput{childdoc.dtx}

\end{document}
%</driver>
% \fi
%
% %%%%%%%%%%%%%%%%%%%%%%%%%%%%%%%%%%%%%%%%%%%%%%%%%%%%%%%%%%%%%%%%%%%%%%%%%%%%%%
% %%%%%%%%%%%%%%%%%%%%%%%%%%%%%%%%%%%%%%%%%%%%%%%%%%%%%%%%%%%%%%%%%%%%%%%%%%%%%%
% \section{Sample}
%\iffalse
%<*samplemain>
%\fi
%
% The following presents a sample document
% with two chapters, two parts, a title page,
% a compile flag as well as three forwarding files to set the flag.
% It consists of eight |.tex| files:
% \begin{center}
% \begin{tabular}{ll}
% |cdocsamp.tex|&main file\\
% |cdocsch1.tex|&include file for chapter 1\\
% |cdocsch2.tex|&include file for chapter 2\\
% |cdocspt3.tex|&include file for part 3\\
% |cdocspt4.tex|&include file for part 4\\
% |cdocsdrf.tex|&forwarding file for main file in draft mode\\
% |cdocsfi1.tex|&forwarding file for final version of chapter 1\\
% |cdocsfi2.tex|&forwarding file for final version of chapter 2\\
% \end{tabular}
% \end{center}
% Each of the eight files can be compiled directly by the \LaTeX{} compiler.
%
% %%%%%%%%%%%%%%%%%%%%%%%%%%%%%%%%%%%%%%
% \paragraph{Main File.}
%
% The main file is called |cdocsamp.tex|.
%
% Load the \textsf{childdoc} definitions and
% declare the filename for the main document:
%    \begin{macrocode}
\input{childdoc.def}
\childdocmain{}
%    \end{macrocode}

% Optional override for |\version| flag:
%    \begin{macrocode}
%%\ifchilddoc\else\providecommand{\version}{draft}\fi
%    \end{macrocode}

% Define the default values for the |\version| flag
% (|final| for the main file and |draft| for childs):
%    \begin{macrocode}
\ifchilddoc
\providecommand{\version}{draft}
\else
\providecommand{\version}{final}
\fi
%    \end{macrocode}

% Load the standard document class:
%    \begin{macrocode}
\documentclass[12pt]{article}
%    \end{macrocode}

% Start the document body:
%    \begin{macrocode}
\begin{document}
%    \end{macrocode}

% Declare a title page.
% Print title, part of document being processed and version flag:
%    \begin{macrocode}
\addtocounter{page}{-1}
\begin{center}
{\LARGE\bfseries{}childdoc example\par}
\vspace{1cm}
\ifchilddoc
\ifchilddocmanual part\else chapter\fi:
`\childdocname' of `\childdocjob'\par
\else
main document: `\childdocjob'\par
\fi
version: \version\par
\end{center}
\newpage
%    \end{macrocode}

% Manually include selected file,
% otherwise process as usual:
%    \begin{macrocode}
\ifchilddocmanual
\section*{part `\childdocname'}
\input{\childdocname}
\else
%    \end{macrocode}

% Include the two chapters:
%    \begin{macrocode}
\include{cdocsch1}
\include{cdocsch2}
%    \end{macrocode}

% Include the two parts unless only chapters should be displayed:
%    \begin{macrocode}
\ifchilddoc\else
\section{part three}
\input{cdocspt3}
\section{part four}
\input{cdocspt4}
\fi
%    \end{macrocode}

% Process as usual until here:
%    \begin{macrocode}
\fi
%    \end{macrocode}

% End of document body:
%    \begin{macrocode}
\end{document}
%    \end{macrocode}
%\iffalse
%</samplemain>
%\fi
%
% %%%%%%%%%%%%%%%%%%%%%%%%%%%%%%%%%%%%%%
% \paragraph{Chapter Include Files.}
%
% The include files are called |cdocsch1.tex| and |cdocsch2.tex|.
%
%\iffalse
%<*samplechap1|samplechap2>
%\fi

% Optional override for |\version| flag:
%    \begin{macrocode}
%%\providecommand{\version}{final}
%    \end{macrocode}

% Include the main document:
%    \begin{macrocode}
\input{childdoc.def}
\childdocof{cdocsamp}
%    \end{macrocode}

%\iffalse
%</samplechap1|samplechap2>
%\fi
%
%\iffalse
%<*samplechap1>
%\fi
% Some text for chapter 1:
%    \begin{macrocode}
\section{one}
some text in chapter one
%    \end{macrocode}

%\iffalse
%</samplechap1>
%\fi
% Some text for chapter 2:
%\iffalse
%<*samplechap2>
%\fi
%    \begin{macrocode}
\section{two}
more text in chapter two
%    \end{macrocode}

%\iffalse
%</samplechap2>
%\fi
%
% %%%%%%%%%%%%%%%%%%%%%%%%%%%%%%%%%%%%%%
% \paragraph{Part Include Files.}
%
% The include files are called |cdocspt3.tex| and |cdocspt4.tex|.
%
%\iffalse
%<*samplepart3|samplepart4>
%\fi

% Optional override for |\version| flag:
%    \begin{macrocode}
%%\providecommand{\version}{final}
%    \end{macrocode}

% Include the main document:
%    \begin{macrocode}
\input{childdoc.def}
\childdocby{cdocsamp}
%    \end{macrocode}

%\iffalse
%</samplepart3|samplepart4>
%\fi
%
%\iffalse
%<*samplepart3>
%\fi
% Some text for part 3:
%    \begin{macrocode}
some text in part three
%    \end{macrocode}

%\iffalse
%</samplepart3>
%\fi
% Some text for part 4:
%\iffalse
%<*samplepart4>
%\fi
%    \begin{macrocode}
more text in part four
%    \end{macrocode}

%\iffalse
%</samplepart4>
%\fi
%
% %%%%%%%%%%%%%%%%%%%%%%%%%%%%%%%%%%%%%%
% \paragraph{Forwarding for a Complete Draft.}
%
% The following forwarding file |cdocsdrf.tex|
% compiles the main document in draft mode:
%\iffalse
%<*sampledraft>
%\fi
%    \begin{macrocode}
\def\version{draft}
\input{childdoc.def}
\childdocforward{cdocsamp}
%    \end{macrocode}

%\iffalse
%</sampledraft>
%\fi
%
% %%%%%%%%%%%%%%%%%%%%%%%%%%%%%%%%%%%%%%
% \paragraph{Forwarding for Final Version of the Chapters.}
%
% The following forwarding files |cdocsfn1.tex| and |cdocsfn2.tex|
% (with identical content)
% compile the final versions of the child documents
% |cdocsch1.tex| and |cdocsch2.tex|, respectively:
%\iffalse
%<*samplefinal>
%\fi
%    \begin{macrocode}
\def\version{final}
\input{childdoc.def}
\childdocforwardprefix[cdocsamp]{cdocsfn}{cdocsch}
%    \end{macrocode}

%\iffalse
%</samplefinal>
%\fi
%
% %%%%%%%%%%%%%%%%%%%%%%%%%%%%%%%%%%%%%%
% \paragraph{Command Line Processing.}
%
% The following three command lines generate the output files
% |cdocscld|, |cdocscl1| and |cdocscl2|
% which should be identical to
% |cdocsdrf|, |cdocsch1| and |cdocsfn2|, respectively:
% \begin{center}
% \begin{tabular}{l}
% |latex -jobname cdocscld \|\\
% |  "\def\version{draft}\input{childdoc.def}\childdocforward{cdocsamp}"|\\
% |latex -jobname cdocscl1 \|\\
% |  "\input{childdoc.def}\childdocforward[cdocsamp]{cdocsch1}"|\\
% |latex -jobname cdocscl2 \|\\
% |  "\def\version{final}\input{childdoc.def}\childdocforward{cdocsch2}"|
% \end{tabular}
% \end{center}
% Note that the trailing backslash on each first line
% merely continues the input to the second line
% (for convenient cut ant paste).
% Furthermore, the command |latex| can be replaced by any
% of its alternative versions such as |pdflatex|.
%
% %%%%%%%%%%%%%%%%%%%%%%%%%%%%%%%%%%%%%%%%%%%%%%%%%%%%%%%%%%%%%%%%%%%%%%%%%%%%%%
% %%%%%%%%%%%%%%%%%%%%%%%%%%%%%%%%%%%%%%%%%%%%%%%%%%%%%%%%%%%%%%%%%%%%%%%%%%%%%%
% \section{Implementation}
%\iffalse
%<*package>
%\fi
%
% This section describes the definitions file |childdoc.def|.

% The definitions cannot be loaded using |\usepackage| or |\RequirePackage|
% which has a mechanism to prevent loading a style file more than once.
% When loading the definitions by means of |\input|
% multiple instances have to be prevented manually:
%\iffalse
%This code needs to be before the `\ProvidesFile' directive
%which is defined at the beginning of this file.
%Therefore it is also placed there and commented out here.
%</package>
%<*discard>
%\fi
%    \begin{macrocode}
\ifdefined\childdocmain\endinput\fi
%    \end{macrocode}
%\iffalse
%</discard>
%<*package>
%\fi
%
% \macro{\ifchilddoc}
% \macro{\ifchilddocmanual}
% The conditional |\ifchilddoc| tells whether a
% child (true) or main (false) document is being compiled.
% The conditional |\ifchilddocmanual| tells whether
% the |\includeonly| mechanism is used (false) or
% the selection of child files must be performed manually (true).
% The definitions initialise to false:
%    \begin{macrocode}
\newif\ifchilddoc
\newif\ifchilddocmanual
%    \end{macrocode}

% \macro{\childdocname}
% \macro{\childdocjob}
% The macro |\childdocname| stores the name of the main document
% to be compiled. The macro |\childdocjob| stores the name of
% the document on which the \LaTeX{} compiler was originally invoked.
% The content of |\jobname| cannot be compared
% to filenames specified in the source due to different catcodes.
% The following code rescans |\jobname|, stores the result
% in |\childdocname| and saves a copy in |\childdocjob|:
%    \begin{macrocode}
\edef\childdocname{\scantokens\expandafter{\jobname\noexpand}}
\let\childdocjob\childdocname
%    \end{macrocode}

% \macro{\childdocdisable}
% The macro |\childdocdisable| prevents the main file
% from being processed more than once.
% At this stage, the main document command |\childdocmain|
% is assumed to be called once again where it should do nothing.
% Any subsequent call to it should prevent
% a secondary processing of the main document
% It overwrites the forwarding commands
% |\childdocof| and |\childdocforward|
% with empty macros to prevent further inclusions of the main document:
%    \begin{macrocode}
\newcommand{\childdocdisable}
{
  \renewcommand{\childdocmain}[1]{\renewcommand{\childdocmain}[1]{\endinput}}
  \renewcommand{\childdocof}[1]{}
  \renewcommand{\childdocby}[2][]{}
  \renewcommand{\childdocforward}[2][]{}
  \renewcommand{\childdocdisable}{}
}
%    \end{macrocode}

% \macro{\childdocmain}
% The macro |\childdocmain| is to be called at the top of the main file
% with nothing or the main filename (without extension) as argument.
% First, it breaks loops.
% If the argument is not empty and does not match |\childdocname|
% (which is set by the first inclusion of |childdoc.def|),
% |\ifchilddoc| is set to true, |\includeonly| is applied to the child file
% and |\jobname| is set to the main file
% (for proper handling of |.aux| files):
%    \begin{macrocode}
\newcommand{\childdocmain}[1]
{
  \childdocdisable\childdocmain{}
  \if?#1?\else
    \begingroup
      \def\childdoctmp{#1}
      \ifx\childdoctmp\childdocname
        \def\childdoctmp{}
      \else
        \def\childdoctmp
        {
          \childdoctrue
          \includeonly{\childdocname}
          \def\childdocjob{#1}
          \def\jobname{#1}
        }
      \fi
      \expandafter
    \endgroup
    \childdoctmp
  \fi
}
%    \end{macrocode}

% \macro{\childdocof}
% The command |\childdocof| redirects
% compilation to the main file |#1|.
%    \begin{macrocode}
\newcommand{\childdocof}[1]
{
  \childdocdisable
  \childdoctrue
  \includeonly{\childdocname}
  \def\jobname{#1}
  \def\childdocjob{#1}
  \input{#1}
}
%    \end{macrocode}

% \macro{\childdocby}
% The command |\childdocby| ....
%    \begin{macrocode}
\newcommand{\childdocby}[2][]
{
  \childdocdisable
  \childdoctrue
  \childdocmanualtrue
  \if?#1?\else
    \def\jobname{#2}
  \fi
  \def\childdocjob{#2}
  \input{#2}
  \endinput
}
%    \end{macrocode}

% \macro{\childdocforward}
% The command |\childdocforward| redirects
% compilation to the main file or
% (if the optional argument is given) a child file.
% Parameters are set as if the main file
% or a child file starting with |\childdocof| was compiled.
% Then compilation is handed over to the main file:
%    \begin{macrocode}
\newcommand{\childdocforward}[2][]
{
  \begingroup
    \if?#1?
      \def\childdoctmp
      {
        \def\childdocname{#2}
        \def\childdocjob{#2}
        \def\jobname{#2}
        \input{#2}
        \endinput
      }
    \else
      \def\childdoctmp
      {
        \childdocdisable
        \def\childdocname{#2}
        \childdoctrue
        \includeonly{#2}
        \def\childdocjob{#1}
        \def\jobname{#1}
        \input{#1}
        \endinput
      }
    \fi
    \expandafter
  \endgroup
  \childdoctmp
}
%    \end{macrocode}

% \macro{\childdocforwardprefix}
% The command |\childdocforwardprefix| redirects
% compilation to the main or a child file by means of a pattern.
% The prefix |#1| in the current filename is replaced by |#2|
% and the suffix of the current filename is kept
% (it is assumed that the filename does not contain the substring `|~~~|'
% which is used as a delimiter).
% Compilation is handed over to the new file by |\childdocforward|:
%    \begin{macrocode}
\newcommand{\childdocforwardprefix}[3][]
{
  \begingroup
    \def\childdocextract #2##1~~~{\def\childdoctmp{\childdocforward[#1]{#3##1}}}
    \expandafter\childdocextract\childdocname~~~
    \expandafter
  \endgroup
  \childdoctmp
}
%    \end{macrocode}

% \macro{\childdoc}
% The deprecated macro |\childdoc| is a legacy version of |\childdocmain|:
%    \begin{macrocode}
\newcommand{\childdoc}{\childdocmain}
%    \end{macrocode}

% \macro{\childdocredirect}
% The deprecated macro |\childdocredirect| is a legacy version
% of |\childdocforward| and |\childdocforwardprefix|:
%    \begin{macrocode}
\newcommand{\childdocredirect}[2][]
{
  \begingroup
    \if?#1?
      \def\childdoctmp{\childdocforward{#2}}
    \else
      \def\childdoctmp{\childdocforwardprefix{#1}{#2}}
    \fi
    \expandafter
  \endgroup
  \childdoctmp
}
%    \end{macrocode}

%\iffalse
%</package>
%\fi
%
\endinput

\childdocof{cdocsamp}
%    \end{macrocode}

%\iffalse
%</samplechap1|samplechap2>
%\fi
%
%\iffalse
%<*samplechap1>
%\fi
% Some text for chapter 1:
%    \begin{macrocode}
\section{one}
some text in chapter one
%    \end{macrocode}

%\iffalse
%</samplechap1>
%\fi
% Some text for chapter 2:
%\iffalse
%<*samplechap2>
%\fi
%    \begin{macrocode}
\section{two}
more text in chapter two
%    \end{macrocode}

%\iffalse
%</samplechap2>
%\fi
%
% %%%%%%%%%%%%%%%%%%%%%%%%%%%%%%%%%%%%%%
% \paragraph{Part Include Files.}
%
% The include files are called |cdocspt3.tex| and |cdocspt4.tex|.
%
%\iffalse
%<*samplepart3|samplepart4>
%\fi

% Optional override for |\version| flag:
%    \begin{macrocode}
%%\providecommand{\version}{final}
%    \end{macrocode}

% Include the main document:
%    \begin{macrocode}
% \iffalse
%
% childdoc.dtx Copyright (C) 2017-2018 Niklas Beisert
%
% This work may be distributed and/or modified under the
% conditions of the LaTeX Project Public License, either version 1.3
% of this license or (at your option) any later version.
% The latest version of this license is in
%   http://www.latex-project.org/lppl.txt
% and version 1.3 or later is part of all distributions of LaTeX
% version 2005/12/01 or later.
%
% This work has the LPPL maintenance status `maintained'.
%
% The Current Maintainer of this work is Niklas Beisert.
%
% This work consists of the files childdoc.dtx and childdoc.ins
% and the derived files childdoc.def and cdocsamp.tex with
% cdocsch1.tex, cdocsch2.tex, cdocsdrf.tex, cdocsfn1.tex, cdocsfn2.tex.
%
%<package>\ifdefined\childdocmain\endinput\fi
%<package>\ProvidesFile{childdoc.def}[2018/12/30 v2.0 child document driver]
%<samplemain>\ProvidesFile{cdocsamp.tex}[2018/12/30 v2.0 sample for childdoc]
%<*driver>
%\ProvidesFile{childdoc.drv}[2018/12/30 v2.0 childdoc reference manual file]
\PassOptionsToClass{10pt,a4paper}{article}
\documentclass{ltxdoc}

\usepackage[margin=35mm]{geometry}
\usepackage{hyperref}
\usepackage{hyperxmp}
\usepackage[usenames]{color}

\hypersetup{colorlinks=true}
\hypersetup{pdfstartview=FitH}
\hypersetup{pdfpagemode=UseNone}
\hypersetup{pdfsource={}}
\hypersetup{pdflang={en-UK}}
\hypersetup{pdfcopyright={Copyright 2017-2018 Niklas Beisert.
  This work may be distributed and/or modified under the
  conditions of the LaTeX Project Public License, either version 1.3
  of this license or (at your option) any later version.}}
\hypersetup{pdflicenseurl={http://www.latex-project.org/lppl.txt}}
\hypersetup{pdfcontactaddress={ETH Zurich, ITP, HIT K,
  Wolfgang-Pauli-Strasse 27}}
\hypersetup{pdfcontactpostcode={8093}}
\hypersetup{pdfcontactcity={Zurich}}
\hypersetup{pdfcontactcountry={Switzerland}}
\hypersetup{pdfcontactemail={nbeisert@itp.phys.ethz.ch}}
\hypersetup{pdfcontacturl={http://people.phys.ethz.ch/\xmptilde nbeisert/}}

\newcommand{\secref}[1]{\hyperref[#1]{section \ref*{#1}}}

\parskip1ex
\parindent0pt
\let\olditemize\itemize
\def\itemize{\olditemize\parskip0pt}

\begin{document}

\title{The \textsf{childdoc} Package}
\hypersetup{pdftitle={The childdoc Package}}
\author{Niklas Beisert\\[2ex]
  Institut f\"ur Theoretische Physik\\
  Eidgen\"ossische Technische Hochschule Z\"urich\\
  Wolfgang-Pauli-Strasse 27, 8093 Z\"urich, Switzerland\\[1ex]
  \href{mailto:nbeisert@itp.phys.ethz.ch}
  {\texttt{nbeisert@itp.phys.ethz.ch}}}
\hypersetup{pdfauthor={Niklas Beisert}}
\hypersetup{pdfsubject={Manual for the LaTeX2e Package childdoc}}
\date{30 December 2018, \textsf{v2.0}}
\maketitle

\begin{abstract}\noindent
\textsf{childdoc} is a \LaTeXe{} package
that enables the direct compilation
of document sections included by |\include|
to individual files.
\end{abstract}

\begingroup
\parskip0ex
\tableofcontents
\endgroup

%%%%%%%%%%%%%%%%%%%%%%%%%%%%%%%%%%%%%%%%%%%%%%%%%%%%%%%%%%%%%%%%%%%%%%%%%%%%%%%%
%%%%%%%%%%%%%%%%%%%%%%%%%%%%%%%%%%%%%%%%%%%%%%%%%%%%%%%%%%%%%%%%%%%%%%%%%%%%%%%%
\section{Introduction}

\LaTeX{} provides a mechanism to structure a large document (such as a book)
into a main file and several child files (containing the chapters)
using the |\include| command.
This mechanism is beneficial for documents
which span hundreds of pages in order to
make the source file(s) more manageable.
Moreover, compilation can be restricted to
selected child files by means of the |\includeonly| command.
The latter feature can be used to reduce the compilation time while editing
(this was significantly more useful in the earlier days of \LaTeX{})
or to generate a smaller document which is easier to navigate.
Another application of |\includeonly| is to generate
documents consisting of selected parts of the complete document.

However, there are a few drawbacks of the plain |\include| mechanism:
\begin{itemize}
\item
The child files cannot be compiled on their own,
they can only be compiled via the main file.
A naive editing environment
(such as a text editor with an option
to have the current file processed by \LaTeX)
may require one to switch to the main file before compiling;
attempting to compile the child file produces errors.
\item
The main file must be modified (each time)
to adjust the |\includeonly| command
to the present needs. This easily leaves the main file in a messy state.
\item
The generated document will always carry the filename
of the main document. This is inconvenient if
several child files are to be compiled and
to be kept for distribution.
\end{itemize}

The present package provides a simple interface
to make child files individually compilable by \LaTeX{}.
Compiling a child file then has the same effect as compiling
the main file with an |\includeonly| command
to select the appropriate child.
Moreover the generated document will carry the name of the child
rather than the main file.
This resolves all three above issues.

This feature is meant to make the editing of books,
thesis documents and lecture notes somewhat more convenient.
However, the package can also be used efficiently for
composing a series of documents (such as exercise sheets)
which are typically distributed individually.
It then assists the author in generating the individual documents
(potentially in different versions)
as well as a document containing the collected series.
Another application is in developing style files
or other kinds of included material
where compilation of the style file could redirect
to a sample or test file.

%%%%%%%%%%%%%%%%%%%%%%%%%%%%%%%%%%%%%%%%%%%%%%%%%%%%%%%%%%%%%%%%%%%%%%%%%%%%%%%%
%%%%%%%%%%%%%%%%%%%%%%%%%%%%%%%%%%%%%%%%%%%%%%%%%%%%%%%%%%%%%%%%%%%%%%%%%%%%%%%%
\section{Usage}

First of all, the package \textsf{childdoc} is \emph{not} a standard
\LaTeXe{} |.sty| style file! Therefore it needs to be invoked in
a non-standard way.

%%%%%%%%%%%%%%%%%%%%%%%%%%%%%%%%%%%%%%%%%%%%%%%%%%%%%%%%%%%%%%%%%%%%%%%%%%%%%%%%
\subsection{Included Files}
\label{sec:include}

%%%%%%%%%%%%%%%%%%%%%%%%%%%%%%%%%%%%%%%%
\DescribeMacro{\childdocmain}
To use the package, add the commands
\begin{center}
\begin{tabular}{l}
|\input{childdoc.def}|\\
|\childdocmain{}|\\
\end{tabular}
\end{center}
at the very top of the main \LaTeX{} file,
in particular \emph{before} the |\documentclass| statement!
The argument of |\childdocmain| should be left empty
(but it must be present).

%%%%%%%%%%%%%%%%%%%%%%%%%%%%%%%%%%%%%%%%
\DescribeMacro{\childdocof}
Furthermore, add the commands
\begin{center}
\begin{tabular}{l}
|\input{childdoc.def}|\\
|\childdocof{|\textit{main}|}|\\
\end{tabular}
\end{center}
at the top of every child file \textit{child}
which is included by |\include{|\textit{child}|}|
from within the main file
(or at least for those files to be compiled individually).
The argument \textit{main} must be the filename of the main file.

There are a couple of
considerations in setting up the main and child documents:

%%%%%%%%%%%%%%%%%%%%%%%%%%%%%%%%%%%%%%%%
\paragraph{Restrictions.}

Please note the following restrictions:
\begin{itemize}
\item
|\childdocmain| must be called with one argument \textit{main}
to ensure compatibility with earlier version of the package.
It must either be empty (|\childdocmain{}|)
or precisely match the filename of the main file in which it is specified.
See \secref{sec:detection} for further information.
\item
The filename \textit{main} must be specified without the |.tex| extension.
\item
The filename \textit{main} is case sensitive
(even in case-insensitive file systems)
due to internal string comparison.
\item
The argument \textit{main} should be fully expanded, it cannot be a macro.
\item
Subdirectories and special characters should be avoided in filenames.
\item
The command |\childdocmain{|\textit{main}|}| must be followed by a whitespace.
It should not be followed immediately by another command
or by a comment mark `|%|'.
This is because the \TeX{} parser reads the token immediately following
the argument of |\childdocmain| and puts it
at the beginning of every child section;
however, a white\-space is ignored.
\end{itemize}

%%%%%%%%%%%%%%%%%%%%%%%%%%%%%%%%%%%%%%%%
\paragraph{Content of Main File.}

It is advisable to place all content in the child files included by |\include|.
Any output contained in the main file will appear in all child documents
unless suppressed manually;
it cannot be suppressed automatically by the |\includeonly| directive
and thus should normally be avoided.
A method to include some content in the main file
by means of conditional processing is described in \secref{sec:conditional}.

%%%%%%%%%%%%%%%%%%%%%%%%%%%%%%%%%%%%%%%%
\paragraph{Page Numbering.}

When only a part of the document is compiled,
the appropriate numbering of pages
(as well as other status parameters)
is determined from the |.aux| files.
The latter contain information from previous passes.
However this information needs to propagate through
all intermediate child documents.
Therefore the page numbering in child documents may well
be inconsistent until the complete document is compiled at least once.

A useful (if unconventional) way to always ensure a consistent
page numbering is to restart the numbering in each child document
and denote the pages by `\textit{child}|.|\textit{page}'
where \textit{child} represents the chapter/section number of the child file.
This can be achieved by the command
|\numberwithin{page}{|\textit{child}|}|
of the \textsf{amsmath} package
where \textit{child} can be |chapter| or |section|
depending on the chosen structuring.
Alternatively, one can modify the macro |\thepage| appropriately
and reset the counter |page| at the start of each child file.

%%%%%%%%%%%%%%%%%%%%%%%%%%%%%%%%%%%%%%%%%%%%%%%%%%%%%%%%%%%%%%%%%%%%%%%%%%%%%%%%
\subsection{Conditional Processing}
\label{sec:conditional}

The package provides a mechanism to compile different versions
of a document. To customise the versions further some conditional processing
can come in handy to distinguish which version is being compiled.
The package provides two macros to describe the compilation context:

%%%%%%%%%%%%%%%%%%%%%%%%%%%%%%%%%%%%%%%%
\DescribeMacro{\ifchilddoc}
The conditional |\ifchilddoc| distinguishes between the compilation of
child documents and the main document:
%
\begin{center}
|\ifchilddoc |\textit{child-code}| |[|\||else |\textit{main-code}]| \||fi|
\end{center}

%%%%%%%%%%%%%%%%%%%%%%%%%%%%%%%%%%%%%%%%
\DescribeMacro{\childdocname}
\DescribeMacro{\childdocjob}
The macro |\childdocname| contains the filename (without extension)
of the main or child file being processed.
Note that |\childdocjob| will always contain the name of the main file.

%%%%%%%%%%%%%%%%%%%%%%%%%%%%%%%%%%%%%%%%
\paragraph{Title Page.}

Conditional processing can be used to include a title or banner page
in the main document when proper precautions are taken.
Importantly, the code in the main file should ensure that the page counter
(as well as other status parameters which are stored in the |.aux| files)
takes the same value after the conditional processing.
Otherwise the page numbers may take divergent values
depending on which part is compiled.

For example, a title page could be declared by:
%
\begin{center}
\begin{tabular}{l}
|\ifchilddoc\||else|\\
|\addtocounter{page}{-1}|\\
\textit{code for title page}\\
|\newpage|\\
|\||fi|
\end{tabular}
\end{center}
%
A banner page for the child documents can be generated by:
%
\begin{center}
\begin{tabular}{l}
|\ifchilddoc|\\
|\addtocounter{page}{-1}|\\
\textit{code for banner page}\\
|\newpage|\\
|\||fi|
\end{tabular}
\end{center}
%
Here one could write a message such as:
\begin{center}
|This is the part \childdocname{} of \childdocjob{}.|
\end{center}

%%%%%%%%%%%%%%%%%%%%%%%%%%%%%%%%%%%%%%%%%%%%%%%%%%%%%%%%%%%%%%%%%%%%%%%%%%%%%%%%
\subsection{Flags}
\label{sec:flags}

The package makes it easy to generate different versions
of the main or child documents.
To this end compilation flags can be defined
and assigned different default values.
They will be particularly useful in conjunction
with the forwarding mechanism described in \secref{sec:forward}.

For example, it may be useful to have a flag |\version|
which can be set to |draft| or |final|.
The document source will contain some conditional code
depending on the value of |\version|.
Suppose further, the flag should default to |final| for the main file
and to |draft| for child files
which is a natural assignment for editing the document.
This is achieved by placing the following code
in the preamble of the main document
(below the |\childdocmain| directive):
%
\begin{center}
\begin{tabular}{l}
|\ifchilddoc|\\
|\providecommand{\version}{draft}|\\
|\||else|\\
|\providecommand{\version}{final}|\\
|\||fi|
\end{tabular}
\end{center}
%
The definition by |\providecommand| makes sure
that previous definitions are not overwritten.
Further statements |\providecommand{\version}{...}|
can thus be added before the above code to override it.

For the main file, one might add a line
(between |\childdocmain| and the above block)
%
\begin{center}
|%\ifchilddoc\||else\providecommand{\version}{draft}\||fi|
\end{center}
%
which can be uncommented to produce a draft version.
Likewise one can add a line to the very top of a child file
(above the |\childdocof{|\textit{main}|}| directive)
%
\begin{center}
|%\providecommand{\version}{final}|
\end{center}
%
which can be uncommented to produce the final version of this child document.

%%%%%%%%%%%%%%%%%%%%%%%%%%%%%%%%%%%%%%%%%%%%%%%%%%%%%%%%%%%%%%%%%%%%%%%%%%%%%%%%
\subsection{Forwarding}
\label{sec:forward}

Different versions of the main or child documents
using compilation flags as described in \secref{sec:flags}
can be (permanently) stored in different files
for convenient compilation, viewing and distribution.
To this end, the package defines a command
to pass on compilation to a different file:

%%%%%%%%%%%%%%%%%%%%%%%%%%%%%%%%%%%%%%%%
\DescribeMacro{\childdocforward}
The command |\childdocforward| redirects processing to
another source file:
%
\begin{center}
\begin{tabular}{l}
|\input{childdoc.def}|\\
|\childdocforward[|\textit{main}|]{|\textit{dest}|}|\\
\end{tabular}
\end{center}
%
The argument \textit{dest} is the destination file
(without extension).
It should be the main file or one of the child files.
Note that further \textsf{childdoc} directives
such as |\childdocof| and |\childdocforward|
in the indicated file will be processed in this form.
The optional argument \textit{main}
passes on directly to the main file \textit{main}
while pretending to compile the child \textit{dest}.
This form behaves as if \textit{dest}
issues |\childdocof{|\textit{main}|}| right away,
and no further \textsf{childdoc} directives will be processed.

%%%%%%%%%%%%%%%%%%%%%%%%%%%%%%%%%%%%%%%%
\DescribeMacro{\...prefix}
In the alternative form |\childdocforwardprefix|,
%
\begin{center}
\begin{tabular}{l}
|\input{childdoc.def}|\\
|\childdocforwardprefix[|\textit{main}|]{|\textit{prefix}|}{|\textit{dest}|}|
\end{tabular}
\end{center}
%
the destination file is determined by a pattern
depending on the current file:
To make this work, the current file must be called
`{\textit{prefix}\hspace{0.2em}\textit{suffix}}'
with \textit{prefix} matching precisely the argument.
Processing is then passed on to the file
`{\textit{dest}\hspace{0.2em}\textit{suffix}}'.
Surely, the same effect is achieved by
directly specifying the
argument `{\textit{dest}\hspace{0.2em}\textit{suffix}}'
in the first form.
However, that requires to set up a different file
for each child. With the alternative form of the command
all these files can have exactly the same content
which simplifies setting them up and maintaining them.

For example, the following file |draft.tex|
with a compilation flag |\version| as described in \secref{sec:flags}
compiles the main document as a draft:
%
\begin{center}
\begin{tabular}{l}
|\def\version{draft}|\\
|\input{childdoc.def}|\\
|\childdocforward{|\textit{main}|}|
\end{tabular}
\end{center}
%
Likewise, the following files |final|\textit{nn}|.tex|
compile the final version of the child document
|child|\textit{nn}|.tex|:
%
\begin{center}
\begin{tabular}{l}
|\def\version{final}|\\
|\input{childdoc.def}|\\
|\childdocforwardprefix{final}{child}|
\end{tabular}
\end{center}
%

Note that when several versions of a main file and/or of each child file
are to be generated, it may be convenient to set up a |Makefile| or
shell script to automatise the process.

%%%%%%%%%%%%%%%%%%%%%%%%%%%%%%%%%%%%%%%%%%%%%%%%%%%%%%%%%%%%%%%%%%%%%%%%%%%%%%%%
\subsection{Command Line Processing}
\label{sec:commandline}

The effect of redirection files can also be achieved by invoking
the \LaTeX{} compiler with a more elaborate command line.
Most conveniently this should be done as part
of a shell script or a |Makefile|.

When using \textsf{childdoc} in the main file, the following
command lines effectively perform a redirection
(note that depending on the shell being used,
backslashes may have to be doubled: `|\|' $\to$ `|\\|'):
%
\begin{center}
|... -jobname "|\textit{target}|" |\\|"|[\textit{flags}]%
|\input{childdoc.def}\childdocforward[|\textit{main}|]{|\textit{dest}|}"|
\end{center}
%
Here \textit{target} is the name of the output file,
\textit{main} is the name of the main file
and \textit{dest} is the name of the main or child file to be processed
(all filenames without extensions).
The optional argument \textit{main} can be omitted
if \textit{main} matches \textit{dest}.
Optionally, compilation \textit{flags} can be defined via |\def| commands.
This command line makes the \TeX{} engine believe
it is compiling the file \textit{target}
whose content is specified as the latter parameter.
The provided code then forwards the processing to
\textit{main} or \textit{dest} as described in \secref{sec:forward}.

%%%%%%%%%%%%%%%%%%%%%%%%%%%%%%%%%%%%%%%%%%%%%%%%%%%%%%%%%%%%%%%%%%%%%%%%%%%%%%%%
\subsection{Include by Input}
\label{sec:input}

Including child documents by |\include| has some restrictions by design.
Most notably, the content of a child document always occupies
its own set of pages; pages cannot be shared between child documents.
Usually, this behaviour makes perfect sense
because each child document contain an essential part of the document.
However, in some situations it may be desirable to compose
a document from a collection of parts
without having mandatory page breaks between then.
For this case, the package
provides a mechanism to include parts
by |\input| which can also be processed individually.
However, by construction this mechanism
requires manual handling of the content to be output.

%%%%%%%%%%%%%%%%%%%%%%%%%%%%%%%%%%%%%%%%
\DescribeMacro{\ifchilddocmanual}
The main file should be prepared as usual, see \secref{sec:include}.
However, the document body must make a distinction
between processing of an individual part and of the main document, e.g.:
%
\begin{center}
\begin{tabular}{l}
|\ifchilddocmanual|\\
|\input{\childdocname}|\\
|\||else|\\
\textit{document body with }|\input{|\textit{part}|}|\\
|\||fi|
\end{tabular}
\end{center}
%
The conditional |\ifchilddocmanual| is true whenever
a part to be included by |\input| is being compiled,
and the name of the part is stored in |\childdocname|.

%%%%%%%%%%%%%%%%%%%%%%%%%%%%%%%%%%%%%%%%
\DescribeMacro{\childdocby}
Each part to be included by |\input| should start with:
%
\begin{center}
\begin{tabular}{l}
|\input{childdoc.def}|\\
|\childdocby{|\textit{main}|}|\\
\end{tabular}
\end{center}
%
The directive |\childdocby| is similar to |\childdocof|
described in \secref{sec:include},
but the subsequent selection of content must be done manually.
To that end, both |\ifchilddoc| and |\ifchilddocmanual|
will be true upon processing of a part,
and the name of the part is stored in |\childdocname|.
Note that |\jobname| will be set to the filename of the current part
so that each part receives an individual |.aux| file
that does not interfere with the |.aux| file(s) of the main document.
This behaviour can be altered by the alternative form
|\childdocby[*]{|\textit{main}|}| (with a non-empty optional argument)
which uses the |.aux| file of the main document
by setting |\jobname| to \textit{main}.

%%%%%%%%%%%%%%%%%%%%%%%%%%%%%%%%%%%%%%%%%%%%%%%%%%%%%%%%%%%%%%%%%%%%%%%%%%%%%%%%
\subsection{Driver Development}
\label{sec:driver}

The \textsf{childdoc} mechanism can also be use for the development
of definition files such as \LaTeX{} styles or classes.
This case differs from the above setup with multiple parts
included by |\include| in that no |\includeonly| should be invoked.
This can be achieved by starting the include file
(before |\ProvidesPackage|) with:
%
\begin{center}
\begin{tabular}{l}
|\input{childdoc.def}|\\
|\childdocforward{|\textit{main}|}|\\
\end{tabular}
\end{center}
%
or alternatively with:
%
\begin{center}
\begin{tabular}{l}
|\input{childdoc.def}|\\
|\childdocby{|\textit{main}|}|\\
\end{tabular}
\end{center}
%
Both forms have slightly different effects as described above.
The main file is prepared as usual, see \secref{sec:include}.

%%%%%%%%%%%%%%%%%%%%%%%%%%%%%%%%%%%%%%%%%%%%%%%%%%%%%%%%%%%%%%%%%%%%%%%%%%%%%%%%
\subsection{Legacy Detection}
\label{sec:detection}

The directive |\childdocmain| in the main file can detect
whether the complete document or merely a child is to be compiled
even without using the directive |\childdocof|.
This method is deprecated because it is less robust
and there is no compelling reason to use it;
it is merely provided for backward compatibility
and it may be removed in future versions.

If the detection mechanism is to be used,
it is mandatory to correctly specify
the filename of the main file as the argument of |\childdocmain|:
%
\begin{center}
\begin{tabular}{l}
|\input{childdoc.def}|\\
|\childdocmain{|\textit{main}|}|\\
\end{tabular}
\end{center}
%
If |\jobname| does not match the argument \textit{main} of |\childdocmain|,
it is assumed that |\jobname| points to the child file to be compiled.
When using |\childdocmain| with the main file specified as argument,
it suffices to start a child file
with just |\input{|\textit{main}|}|
without loading of the package and using |\childdocof|.
If instead all processing is done
with the appropriate \textsf{childdoc} directives,
the argument of \textit{main} of |\childdocmain| can be empty.

An alternative version of the command line processing described
in \secref{sec:commandline} using the detection mechanism reads:
%
\begin{center}
|... -jobname "|\textit{target}|" "|[\textit{flags}]%
[|\def\jobname{|\textit{dest}|}|]|\input{|\textit{main}|}"|
\end{center}

%%%%%%%%%%%%%%%%%%%%%%%%%%%%%%%%%%%%%%%%%%%%%%%%%%%%%%%%%%%%%%%%%%%%%%%%%%%%%%%%
\subsection{Manual Code}
\label{sec:manual}

In case one cannot be certain whether the definitions file |childdoc.def|
is installed on the target \TeX{} distribution
and one prefers not to ship it,
it is conceivable to paste a few relevant commands into the sources.

To that end, drop all statements |\input{childdoc.def}|
and perform the replacements as outlined below.
Instead of |\childdocmain{|\textit{main}|}| add the following code
to the top of the main file:
%
\begin{center}
\begin{tabular}{l}
|\||ifdefined\childdocname\endinput\||fi\newif\ifchilddoc|\\
|\edef\childdocname{\scantokens\expandafter{\jobname\noexpand}}|\\
|\def\childdocmain{|\textit{main}|}\||ifx\childdocmain\childdocname\||else|\\
|\childdoctrue\includeonly{\childdocname}\let\jobname\childdocmain\||fi|\\
\end{tabular}
\end{center}
%
Instead of |\childdocof{|\textit{main}|}| just include the main file
at the top of each child file:
%
\begin{center}
|\input{|\textit{main}|}|
\end{center}
%
A simple redirection |\childdocforward{|\textit{dest}|}| is achieved by:
%
\begin{center}
|\def\jobname{|\textit{dest}|}\input{\jobname}|
\end{center}
%
The redirection with prefix
|\childdocforwardprefix[|\textit{prefix}|]{|\textit{dest}|}|
is accomplished by:
%
\begin{center}
\begin{tabular}{l}
|{\edef\jobname{\scantokens\expandafter{\jobname\noexpand}}|\\
|\def\redirectjob |\textit{prefix}|#1~~~{\gdef\jobname{|\textit{dest}|#1}}|\\
|\expandafter\redirectjob\jobname~~~}\input{\jobname}|
\end{tabular}
\end{center}

In an alternative approach,
child documents can be compiled by a specific command line
without additional code or specific definitions:
%
\begin{center}
|... -jobname "|\textit{target}|" "|[\textit{flags}]%
|\includeonly{|\textit{dest}|}\input{|\textit{main}|}"|
\end{center}
%

%%%%%%%%%%%%%%%%%%%%%%%%%%%%%%%%%%%%%%%%%%%%%%%%%%%%%%%%%%%%%%%%%%%%%%%%%%%%%%%%
%%%%%%%%%%%%%%%%%%%%%%%%%%%%%%%%%%%%%%%%%%%%%%%%%%%%%%%%%%%%%%%%%%%%%%%%%%%%%%%%
\section{Information}

%%%%%%%%%%%%%%%%%%%%%%%%%%%%%%%%%%%%%%%%%%%%%%%%%%%%%%%%%%%%%%%%%%%%%%%%%%%%%%%%
\subsection{Copyright}

Copyright \copyright{} 2017--2018 Niklas Beisert

This work may be distributed and/or modified under the
conditions of the \LaTeX{} Project Public License, either version 1.3
of this license or (at your option) any later version.
The latest version of this license is in
  \url{http://www.latex-project.org/lppl.txt}
and version 1.3 or later is part of all distributions of \LaTeX{}
version 2005/12/01 or later.

This work has the LPPL maintenance status `maintained'.

The Current Maintainer of this work is Niklas Beisert.

This work consists of the files |README.txt|, |childdoc.ins| and |childdoc.dtx|
as well as the derived files |childdoc.def|, |cdocsamp.tex|
with |cdocsch1.tex|, |cdocsch2.tex|, |cdocspt3.tex|, |cdocspt4.tex|,
|cdocsdrf.tex|, |cdocsfn1.tex|, |cdocsfn2.tex|
as well as |childdoc.pdf|.

%%%%%%%%%%%%%%%%%%%%%%%%%%%%%%%%%%%%%%%%%%%%%%%%%%%%%%%%%%%%%%%%%%%%%%%%%%%%%%%%
\subsection{Files and Installation}

The package consists of the files:
%
\begin{center}
\begin{tabular}{ll}
    |README.txt|   & readme file \\
    |childdoc.ins| & installation file \\
    |childdoc.dtx| & source file \\
    |childdoc.def| & definition file \\
    |cdocsamp.tex| & sample main file \\
    |cdocsch1.tex| & sample include file \\
    |cdocsch2.tex| & sample include file \\
    |cdocspt3.tex| & sample part file \\
    |cdocspt4.tex| & sample part file \\
    |cdocsdrf.tex| & sample redirection file \\
    |cdocsfn1.tex| & sample redirection file \\
    |cdocsfn2.tex| & sample redirection file \\
    |childdoc.pdf| & manual
\end{tabular}
\end{center}
%
The distribution consists of the files
|README.txt|, |childdoc.ins| and |childdoc.dtx|.
%
\begin{itemize}
\item
Run (pdf)\LaTeX{} on |childdoc.dtx|
to compile the manual |childdoc.pdf| (this file).
\item
Run \LaTeX{} on |childdoc.ins| to create the definitions file |childdoc.def|
and the sample |cdocsamp.tex| with include files
|cdocsch1.tex|, |cdocsch2.tex|, |cdocspt3.tex|, |cdocspt4.tex|,
|cdocsdrf.tex|, |cdocsfn1.tex|, |cdocsfn2.tex|.
Then copy the file |childdoc.def| to an appropriate directory of your \LaTeX{}
distribution, e.g.\ \textit{texmf-root}|/tex/latex/childdoc|.
\end{itemize}

%%%%%%%%%%%%%%%%%%%%%%%%%%%%%%%%%%%%%%%%%%%%%%%%%%%%%%%%%%%%%%%%%%%%%%%%%%%%%%%%
\subsection{Related CTAN Packages}

There are several other packages which offer a similar functionality:
%
\begin{itemize}
\item
The packages
\href{http://ctan.org/pkg/docmute}{\textsf{docmute}},
\href{http://ctan.org/pkg/includex}{\textsf{includex}} and
\href{http://ctan.org/pkg/standalone}{\textsf{standalone}}
provide commands to include only the document body of
a child file thus allowing both files to be compiled individually.
\item
The packages \href{http://ctan.org/pkg/subdocs}{\textsf{subdocs}}
and \href{http://ctan.org/pkg/subfiles}{\textsf{subfiles}}
provide structures in which the main and child documents can be
encapsulated and allowing them to be compiled individually.
The inclusion mechanism is different from the conventional |\include|.
\item
The package \href{http://ctan.org/pkg/combine}{\textsf{combine}}
is an elaborate solution to combine several documents into one.
\end{itemize}
%
See also the CTAN topic \href{http://ctan.org/topic/subdocs}{\textsf{subdocs}}
for further related packages.
The present package differs from the above solutions in that
a document structure constructed with the conventional |\include| mechanism
just needs two extra commands at the top of every file
such that all constituent files can be compiled individually.

%%%%%%%%%%%%%%%%%%%%%%%%%%%%%%%%%%%%%%%%%%%%%%%%%%%%%%%%%%%%%%%%%%%%%%%%%%%%%%%%
%\subsection{Feature Suggestions}
%
%The following is a list of features which may be useful for future
%versions of this package:
%%
%\begin{itemize}
%\item
%\ldots
%\end{itemize}

%%%%%%%%%%%%%%%%%%%%%%%%%%%%%%%%%%%%%%%%%%%%%%%%%%%%%%%%%%%%%%%%%%%%%%%%%%%%%%%%
\subsection{Revision History}

%%%%%%%%%%%%%%%%%%%%%%%%%%%%%%%%%%%%%%%%
\paragraph{v2.0:} 2018/12/30

\begin{itemize}
\item
immediate forward processing
\item
added |\childdocby| mechanism
\item
manual restructured
\end{itemize}

%%%%%%%%%%%%%%%%%%%%%%%%%%%%%%%%%%%%%%%%
\paragraph{v1.6:} 2018/01/17

\begin{itemize}
\item
application for development of include files
\item
corrections to manual
\end{itemize}

%%%%%%%%%%%%%%%%%%%%%%%%%%%%%%%%%%%%%%%%
\paragraph{v1.5:} 2017/05/21

\begin{itemize}
\item
more complete structuring introduced
\item
|\childdocof| introduced
\item
|\childdoc| renamed to |\childdocmain|
\item
|\childredirect| renamed to |\childdocforward| and |\childdocforwardprefix|
and functionality expanded
\end{itemize}

%%%%%%%%%%%%%%%%%%%%%%%%%%%%%%%%%%%%%%%%
\paragraph{v1.0:} 2017/04/27

\begin{itemize}
\item
manual and install package
\item
first version published on CTAN
\end{itemize}

%%%%%%%%%%%%%%%%%%%%%%%%%%%%%%%%%%%%%%%%
\paragraph{v0.6:} 2017/04/26

\begin{itemize}
\item
redirection mechanism added
\end{itemize}

%%%%%%%%%%%%%%%%%%%%%%%%%%%%%%%%%%%%%%%%
\paragraph{v0.5:} 2017/04/26

\begin{itemize}
\item
functionality in definition file
\end{itemize}


%%%%%%%%%%%%%%%%%%%%%%%%%%%%%%%%%%%%%%%%%%%%%%%%%%%%%%%%%%%%%%%%%%%%%%%%%%%%%%%%
%%%%%%%%%%%%%%%%%%%%%%%%%%%%%%%%%%%%%%%%%%%%%%%%%%%%%%%%%%%%%%%%%%%%%%%%%%%%%%%%
%%%%%%%%%%%%%%%%%%%%%%%%%%%%%%%%%%%%%%%%%%%%%%%%%%%%%%%%%%%%%%%%%%%%%%%%%%%%%%%%
\appendix

\settowidth\MacroIndent{\rmfamily\scriptsize 000\ }

 \DocInput{childdoc.dtx}

\end{document}
%</driver>
% \fi
%
% %%%%%%%%%%%%%%%%%%%%%%%%%%%%%%%%%%%%%%%%%%%%%%%%%%%%%%%%%%%%%%%%%%%%%%%%%%%%%%
% %%%%%%%%%%%%%%%%%%%%%%%%%%%%%%%%%%%%%%%%%%%%%%%%%%%%%%%%%%%%%%%%%%%%%%%%%%%%%%
% \section{Sample}
%\iffalse
%<*samplemain>
%\fi
%
% The following presents a sample document
% with two chapters, two parts, a title page,
% a compile flag as well as three forwarding files to set the flag.
% It consists of eight |.tex| files:
% \begin{center}
% \begin{tabular}{ll}
% |cdocsamp.tex|&main file\\
% |cdocsch1.tex|&include file for chapter 1\\
% |cdocsch2.tex|&include file for chapter 2\\
% |cdocspt3.tex|&include file for part 3\\
% |cdocspt4.tex|&include file for part 4\\
% |cdocsdrf.tex|&forwarding file for main file in draft mode\\
% |cdocsfi1.tex|&forwarding file for final version of chapter 1\\
% |cdocsfi2.tex|&forwarding file for final version of chapter 2\\
% \end{tabular}
% \end{center}
% Each of the eight files can be compiled directly by the \LaTeX{} compiler.
%
% %%%%%%%%%%%%%%%%%%%%%%%%%%%%%%%%%%%%%%
% \paragraph{Main File.}
%
% The main file is called |cdocsamp.tex|.
%
% Load the \textsf{childdoc} definitions and
% declare the filename for the main document:
%    \begin{macrocode}
\input{childdoc.def}
\childdocmain{}
%    \end{macrocode}

% Optional override for |\version| flag:
%    \begin{macrocode}
%%\ifchilddoc\else\providecommand{\version}{draft}\fi
%    \end{macrocode}

% Define the default values for the |\version| flag
% (|final| for the main file and |draft| for childs):
%    \begin{macrocode}
\ifchilddoc
\providecommand{\version}{draft}
\else
\providecommand{\version}{final}
\fi
%    \end{macrocode}

% Load the standard document class:
%    \begin{macrocode}
\documentclass[12pt]{article}
%    \end{macrocode}

% Start the document body:
%    \begin{macrocode}
\begin{document}
%    \end{macrocode}

% Declare a title page.
% Print title, part of document being processed and version flag:
%    \begin{macrocode}
\addtocounter{page}{-1}
\begin{center}
{\LARGE\bfseries{}childdoc example\par}
\vspace{1cm}
\ifchilddoc
\ifchilddocmanual part\else chapter\fi:
`\childdocname' of `\childdocjob'\par
\else
main document: `\childdocjob'\par
\fi
version: \version\par
\end{center}
\newpage
%    \end{macrocode}

% Manually include selected file,
% otherwise process as usual:
%    \begin{macrocode}
\ifchilddocmanual
\section*{part `\childdocname'}
\input{\childdocname}
\else
%    \end{macrocode}

% Include the two chapters:
%    \begin{macrocode}
\include{cdocsch1}
\include{cdocsch2}
%    \end{macrocode}

% Include the two parts unless only chapters should be displayed:
%    \begin{macrocode}
\ifchilddoc\else
\section{part three}
\input{cdocspt3}
\section{part four}
\input{cdocspt4}
\fi
%    \end{macrocode}

% Process as usual until here:
%    \begin{macrocode}
\fi
%    \end{macrocode}

% End of document body:
%    \begin{macrocode}
\end{document}
%    \end{macrocode}
%\iffalse
%</samplemain>
%\fi
%
% %%%%%%%%%%%%%%%%%%%%%%%%%%%%%%%%%%%%%%
% \paragraph{Chapter Include Files.}
%
% The include files are called |cdocsch1.tex| and |cdocsch2.tex|.
%
%\iffalse
%<*samplechap1|samplechap2>
%\fi

% Optional override for |\version| flag:
%    \begin{macrocode}
%%\providecommand{\version}{final}
%    \end{macrocode}

% Include the main document:
%    \begin{macrocode}
\input{childdoc.def}
\childdocof{cdocsamp}
%    \end{macrocode}

%\iffalse
%</samplechap1|samplechap2>
%\fi
%
%\iffalse
%<*samplechap1>
%\fi
% Some text for chapter 1:
%    \begin{macrocode}
\section{one}
some text in chapter one
%    \end{macrocode}

%\iffalse
%</samplechap1>
%\fi
% Some text for chapter 2:
%\iffalse
%<*samplechap2>
%\fi
%    \begin{macrocode}
\section{two}
more text in chapter two
%    \end{macrocode}

%\iffalse
%</samplechap2>
%\fi
%
% %%%%%%%%%%%%%%%%%%%%%%%%%%%%%%%%%%%%%%
% \paragraph{Part Include Files.}
%
% The include files are called |cdocspt3.tex| and |cdocspt4.tex|.
%
%\iffalse
%<*samplepart3|samplepart4>
%\fi

% Optional override for |\version| flag:
%    \begin{macrocode}
%%\providecommand{\version}{final}
%    \end{macrocode}

% Include the main document:
%    \begin{macrocode}
\input{childdoc.def}
\childdocby{cdocsamp}
%    \end{macrocode}

%\iffalse
%</samplepart3|samplepart4>
%\fi
%
%\iffalse
%<*samplepart3>
%\fi
% Some text for part 3:
%    \begin{macrocode}
some text in part three
%    \end{macrocode}

%\iffalse
%</samplepart3>
%\fi
% Some text for part 4:
%\iffalse
%<*samplepart4>
%\fi
%    \begin{macrocode}
more text in part four
%    \end{macrocode}

%\iffalse
%</samplepart4>
%\fi
%
% %%%%%%%%%%%%%%%%%%%%%%%%%%%%%%%%%%%%%%
% \paragraph{Forwarding for a Complete Draft.}
%
% The following forwarding file |cdocsdrf.tex|
% compiles the main document in draft mode:
%\iffalse
%<*sampledraft>
%\fi
%    \begin{macrocode}
\def\version{draft}
\input{childdoc.def}
\childdocforward{cdocsamp}
%    \end{macrocode}

%\iffalse
%</sampledraft>
%\fi
%
% %%%%%%%%%%%%%%%%%%%%%%%%%%%%%%%%%%%%%%
% \paragraph{Forwarding for Final Version of the Chapters.}
%
% The following forwarding files |cdocsfn1.tex| and |cdocsfn2.tex|
% (with identical content)
% compile the final versions of the child documents
% |cdocsch1.tex| and |cdocsch2.tex|, respectively:
%\iffalse
%<*samplefinal>
%\fi
%    \begin{macrocode}
\def\version{final}
\input{childdoc.def}
\childdocforwardprefix[cdocsamp]{cdocsfn}{cdocsch}
%    \end{macrocode}

%\iffalse
%</samplefinal>
%\fi
%
% %%%%%%%%%%%%%%%%%%%%%%%%%%%%%%%%%%%%%%
% \paragraph{Command Line Processing.}
%
% The following three command lines generate the output files
% |cdocscld|, |cdocscl1| and |cdocscl2|
% which should be identical to
% |cdocsdrf|, |cdocsch1| and |cdocsfn2|, respectively:
% \begin{center}
% \begin{tabular}{l}
% |latex -jobname cdocscld \|\\
% |  "\def\version{draft}\input{childdoc.def}\childdocforward{cdocsamp}"|\\
% |latex -jobname cdocscl1 \|\\
% |  "\input{childdoc.def}\childdocforward[cdocsamp]{cdocsch1}"|\\
% |latex -jobname cdocscl2 \|\\
% |  "\def\version{final}\input{childdoc.def}\childdocforward{cdocsch2}"|
% \end{tabular}
% \end{center}
% Note that the trailing backslash on each first line
% merely continues the input to the second line
% (for convenient cut ant paste).
% Furthermore, the command |latex| can be replaced by any
% of its alternative versions such as |pdflatex|.
%
% %%%%%%%%%%%%%%%%%%%%%%%%%%%%%%%%%%%%%%%%%%%%%%%%%%%%%%%%%%%%%%%%%%%%%%%%%%%%%%
% %%%%%%%%%%%%%%%%%%%%%%%%%%%%%%%%%%%%%%%%%%%%%%%%%%%%%%%%%%%%%%%%%%%%%%%%%%%%%%
% \section{Implementation}
%\iffalse
%<*package>
%\fi
%
% This section describes the definitions file |childdoc.def|.

% The definitions cannot be loaded using |\usepackage| or |\RequirePackage|
% which has a mechanism to prevent loading a style file more than once.
% When loading the definitions by means of |\input|
% multiple instances have to be prevented manually:
%\iffalse
%This code needs to be before the `\ProvidesFile' directive
%which is defined at the beginning of this file.
%Therefore it is also placed there and commented out here.
%</package>
%<*discard>
%\fi
%    \begin{macrocode}
\ifdefined\childdocmain\endinput\fi
%    \end{macrocode}
%\iffalse
%</discard>
%<*package>
%\fi
%
% \macro{\ifchilddoc}
% \macro{\ifchilddocmanual}
% The conditional |\ifchilddoc| tells whether a
% child (true) or main (false) document is being compiled.
% The conditional |\ifchilddocmanual| tells whether
% the |\includeonly| mechanism is used (false) or
% the selection of child files must be performed manually (true).
% The definitions initialise to false:
%    \begin{macrocode}
\newif\ifchilddoc
\newif\ifchilddocmanual
%    \end{macrocode}

% \macro{\childdocname}
% \macro{\childdocjob}
% The macro |\childdocname| stores the name of the main document
% to be compiled. The macro |\childdocjob| stores the name of
% the document on which the \LaTeX{} compiler was originally invoked.
% The content of |\jobname| cannot be compared
% to filenames specified in the source due to different catcodes.
% The following code rescans |\jobname|, stores the result
% in |\childdocname| and saves a copy in |\childdocjob|:
%    \begin{macrocode}
\edef\childdocname{\scantokens\expandafter{\jobname\noexpand}}
\let\childdocjob\childdocname
%    \end{macrocode}

% \macro{\childdocdisable}
% The macro |\childdocdisable| prevents the main file
% from being processed more than once.
% At this stage, the main document command |\childdocmain|
% is assumed to be called once again where it should do nothing.
% Any subsequent call to it should prevent
% a secondary processing of the main document
% It overwrites the forwarding commands
% |\childdocof| and |\childdocforward|
% with empty macros to prevent further inclusions of the main document:
%    \begin{macrocode}
\newcommand{\childdocdisable}
{
  \renewcommand{\childdocmain}[1]{\renewcommand{\childdocmain}[1]{\endinput}}
  \renewcommand{\childdocof}[1]{}
  \renewcommand{\childdocby}[2][]{}
  \renewcommand{\childdocforward}[2][]{}
  \renewcommand{\childdocdisable}{}
}
%    \end{macrocode}

% \macro{\childdocmain}
% The macro |\childdocmain| is to be called at the top of the main file
% with nothing or the main filename (without extension) as argument.
% First, it breaks loops.
% If the argument is not empty and does not match |\childdocname|
% (which is set by the first inclusion of |childdoc.def|),
% |\ifchilddoc| is set to true, |\includeonly| is applied to the child file
% and |\jobname| is set to the main file
% (for proper handling of |.aux| files):
%    \begin{macrocode}
\newcommand{\childdocmain}[1]
{
  \childdocdisable\childdocmain{}
  \if?#1?\else
    \begingroup
      \def\childdoctmp{#1}
      \ifx\childdoctmp\childdocname
        \def\childdoctmp{}
      \else
        \def\childdoctmp
        {
          \childdoctrue
          \includeonly{\childdocname}
          \def\childdocjob{#1}
          \def\jobname{#1}
        }
      \fi
      \expandafter
    \endgroup
    \childdoctmp
  \fi
}
%    \end{macrocode}

% \macro{\childdocof}
% The command |\childdocof| redirects
% compilation to the main file |#1|.
%    \begin{macrocode}
\newcommand{\childdocof}[1]
{
  \childdocdisable
  \childdoctrue
  \includeonly{\childdocname}
  \def\jobname{#1}
  \def\childdocjob{#1}
  \input{#1}
}
%    \end{macrocode}

% \macro{\childdocby}
% The command |\childdocby| ....
%    \begin{macrocode}
\newcommand{\childdocby}[2][]
{
  \childdocdisable
  \childdoctrue
  \childdocmanualtrue
  \if?#1?\else
    \def\jobname{#2}
  \fi
  \def\childdocjob{#2}
  \input{#2}
  \endinput
}
%    \end{macrocode}

% \macro{\childdocforward}
% The command |\childdocforward| redirects
% compilation to the main file or
% (if the optional argument is given) a child file.
% Parameters are set as if the main file
% or a child file starting with |\childdocof| was compiled.
% Then compilation is handed over to the main file:
%    \begin{macrocode}
\newcommand{\childdocforward}[2][]
{
  \begingroup
    \if?#1?
      \def\childdoctmp
      {
        \def\childdocname{#2}
        \def\childdocjob{#2}
        \def\jobname{#2}
        \input{#2}
        \endinput
      }
    \else
      \def\childdoctmp
      {
        \childdocdisable
        \def\childdocname{#2}
        \childdoctrue
        \includeonly{#2}
        \def\childdocjob{#1}
        \def\jobname{#1}
        \input{#1}
        \endinput
      }
    \fi
    \expandafter
  \endgroup
  \childdoctmp
}
%    \end{macrocode}

% \macro{\childdocforwardprefix}
% The command |\childdocforwardprefix| redirects
% compilation to the main or a child file by means of a pattern.
% The prefix |#1| in the current filename is replaced by |#2|
% and the suffix of the current filename is kept
% (it is assumed that the filename does not contain the substring `|~~~|'
% which is used as a delimiter).
% Compilation is handed over to the new file by |\childdocforward|:
%    \begin{macrocode}
\newcommand{\childdocforwardprefix}[3][]
{
  \begingroup
    \def\childdocextract #2##1~~~{\def\childdoctmp{\childdocforward[#1]{#3##1}}}
    \expandafter\childdocextract\childdocname~~~
    \expandafter
  \endgroup
  \childdoctmp
}
%    \end{macrocode}

% \macro{\childdoc}
% The deprecated macro |\childdoc| is a legacy version of |\childdocmain|:
%    \begin{macrocode}
\newcommand{\childdoc}{\childdocmain}
%    \end{macrocode}

% \macro{\childdocredirect}
% The deprecated macro |\childdocredirect| is a legacy version
% of |\childdocforward| and |\childdocforwardprefix|:
%    \begin{macrocode}
\newcommand{\childdocredirect}[2][]
{
  \begingroup
    \if?#1?
      \def\childdoctmp{\childdocforward{#2}}
    \else
      \def\childdoctmp{\childdocforwardprefix{#1}{#2}}
    \fi
    \expandafter
  \endgroup
  \childdoctmp
}
%    \end{macrocode}

%\iffalse
%</package>
%\fi
%
\endinput

\childdocby{cdocsamp}
%    \end{macrocode}

%\iffalse
%</samplepart3|samplepart4>
%\fi
%
%\iffalse
%<*samplepart3>
%\fi
% Some text for part 3:
%    \begin{macrocode}
some text in part three
%    \end{macrocode}

%\iffalse
%</samplepart3>
%\fi
% Some text for part 4:
%\iffalse
%<*samplepart4>
%\fi
%    \begin{macrocode}
more text in part four
%    \end{macrocode}

%\iffalse
%</samplepart4>
%\fi
%
% %%%%%%%%%%%%%%%%%%%%%%%%%%%%%%%%%%%%%%
% \paragraph{Forwarding for a Complete Draft.}
%
% The following forwarding file |cdocsdrf.tex|
% compiles the main document in draft mode:
%\iffalse
%<*sampledraft>
%\fi
%    \begin{macrocode}
\def\version{draft}
% \iffalse
%
% childdoc.dtx Copyright (C) 2017-2018 Niklas Beisert
%
% This work may be distributed and/or modified under the
% conditions of the LaTeX Project Public License, either version 1.3
% of this license or (at your option) any later version.
% The latest version of this license is in
%   http://www.latex-project.org/lppl.txt
% and version 1.3 or later is part of all distributions of LaTeX
% version 2005/12/01 or later.
%
% This work has the LPPL maintenance status `maintained'.
%
% The Current Maintainer of this work is Niklas Beisert.
%
% This work consists of the files childdoc.dtx and childdoc.ins
% and the derived files childdoc.def and cdocsamp.tex with
% cdocsch1.tex, cdocsch2.tex, cdocsdrf.tex, cdocsfn1.tex, cdocsfn2.tex.
%
%<package>\ifdefined\childdocmain\endinput\fi
%<package>\ProvidesFile{childdoc.def}[2018/12/30 v2.0 child document driver]
%<samplemain>\ProvidesFile{cdocsamp.tex}[2018/12/30 v2.0 sample for childdoc]
%<*driver>
%\ProvidesFile{childdoc.drv}[2018/12/30 v2.0 childdoc reference manual file]
\PassOptionsToClass{10pt,a4paper}{article}
\documentclass{ltxdoc}

\usepackage[margin=35mm]{geometry}
\usepackage{hyperref}
\usepackage{hyperxmp}
\usepackage[usenames]{color}

\hypersetup{colorlinks=true}
\hypersetup{pdfstartview=FitH}
\hypersetup{pdfpagemode=UseNone}
\hypersetup{pdfsource={}}
\hypersetup{pdflang={en-UK}}
\hypersetup{pdfcopyright={Copyright 2017-2018 Niklas Beisert.
  This work may be distributed and/or modified under the
  conditions of the LaTeX Project Public License, either version 1.3
  of this license or (at your option) any later version.}}
\hypersetup{pdflicenseurl={http://www.latex-project.org/lppl.txt}}
\hypersetup{pdfcontactaddress={ETH Zurich, ITP, HIT K,
  Wolfgang-Pauli-Strasse 27}}
\hypersetup{pdfcontactpostcode={8093}}
\hypersetup{pdfcontactcity={Zurich}}
\hypersetup{pdfcontactcountry={Switzerland}}
\hypersetup{pdfcontactemail={nbeisert@itp.phys.ethz.ch}}
\hypersetup{pdfcontacturl={http://people.phys.ethz.ch/\xmptilde nbeisert/}}

\newcommand{\secref}[1]{\hyperref[#1]{section \ref*{#1}}}

\parskip1ex
\parindent0pt
\let\olditemize\itemize
\def\itemize{\olditemize\parskip0pt}

\begin{document}

\title{The \textsf{childdoc} Package}
\hypersetup{pdftitle={The childdoc Package}}
\author{Niklas Beisert\\[2ex]
  Institut f\"ur Theoretische Physik\\
  Eidgen\"ossische Technische Hochschule Z\"urich\\
  Wolfgang-Pauli-Strasse 27, 8093 Z\"urich, Switzerland\\[1ex]
  \href{mailto:nbeisert@itp.phys.ethz.ch}
  {\texttt{nbeisert@itp.phys.ethz.ch}}}
\hypersetup{pdfauthor={Niklas Beisert}}
\hypersetup{pdfsubject={Manual for the LaTeX2e Package childdoc}}
\date{30 December 2018, \textsf{v2.0}}
\maketitle

\begin{abstract}\noindent
\textsf{childdoc} is a \LaTeXe{} package
that enables the direct compilation
of document sections included by |\include|
to individual files.
\end{abstract}

\begingroup
\parskip0ex
\tableofcontents
\endgroup

%%%%%%%%%%%%%%%%%%%%%%%%%%%%%%%%%%%%%%%%%%%%%%%%%%%%%%%%%%%%%%%%%%%%%%%%%%%%%%%%
%%%%%%%%%%%%%%%%%%%%%%%%%%%%%%%%%%%%%%%%%%%%%%%%%%%%%%%%%%%%%%%%%%%%%%%%%%%%%%%%
\section{Introduction}

\LaTeX{} provides a mechanism to structure a large document (such as a book)
into a main file and several child files (containing the chapters)
using the |\include| command.
This mechanism is beneficial for documents
which span hundreds of pages in order to
make the source file(s) more manageable.
Moreover, compilation can be restricted to
selected child files by means of the |\includeonly| command.
The latter feature can be used to reduce the compilation time while editing
(this was significantly more useful in the earlier days of \LaTeX{})
or to generate a smaller document which is easier to navigate.
Another application of |\includeonly| is to generate
documents consisting of selected parts of the complete document.

However, there are a few drawbacks of the plain |\include| mechanism:
\begin{itemize}
\item
The child files cannot be compiled on their own,
they can only be compiled via the main file.
A naive editing environment
(such as a text editor with an option
to have the current file processed by \LaTeX)
may require one to switch to the main file before compiling;
attempting to compile the child file produces errors.
\item
The main file must be modified (each time)
to adjust the |\includeonly| command
to the present needs. This easily leaves the main file in a messy state.
\item
The generated document will always carry the filename
of the main document. This is inconvenient if
several child files are to be compiled and
to be kept for distribution.
\end{itemize}

The present package provides a simple interface
to make child files individually compilable by \LaTeX{}.
Compiling a child file then has the same effect as compiling
the main file with an |\includeonly| command
to select the appropriate child.
Moreover the generated document will carry the name of the child
rather than the main file.
This resolves all three above issues.

This feature is meant to make the editing of books,
thesis documents and lecture notes somewhat more convenient.
However, the package can also be used efficiently for
composing a series of documents (such as exercise sheets)
which are typically distributed individually.
It then assists the author in generating the individual documents
(potentially in different versions)
as well as a document containing the collected series.
Another application is in developing style files
or other kinds of included material
where compilation of the style file could redirect
to a sample or test file.

%%%%%%%%%%%%%%%%%%%%%%%%%%%%%%%%%%%%%%%%%%%%%%%%%%%%%%%%%%%%%%%%%%%%%%%%%%%%%%%%
%%%%%%%%%%%%%%%%%%%%%%%%%%%%%%%%%%%%%%%%%%%%%%%%%%%%%%%%%%%%%%%%%%%%%%%%%%%%%%%%
\section{Usage}

First of all, the package \textsf{childdoc} is \emph{not} a standard
\LaTeXe{} |.sty| style file! Therefore it needs to be invoked in
a non-standard way.

%%%%%%%%%%%%%%%%%%%%%%%%%%%%%%%%%%%%%%%%%%%%%%%%%%%%%%%%%%%%%%%%%%%%%%%%%%%%%%%%
\subsection{Included Files}
\label{sec:include}

%%%%%%%%%%%%%%%%%%%%%%%%%%%%%%%%%%%%%%%%
\DescribeMacro{\childdocmain}
To use the package, add the commands
\begin{center}
\begin{tabular}{l}
|\input{childdoc.def}|\\
|\childdocmain{}|\\
\end{tabular}
\end{center}
at the very top of the main \LaTeX{} file,
in particular \emph{before} the |\documentclass| statement!
The argument of |\childdocmain| should be left empty
(but it must be present).

%%%%%%%%%%%%%%%%%%%%%%%%%%%%%%%%%%%%%%%%
\DescribeMacro{\childdocof}
Furthermore, add the commands
\begin{center}
\begin{tabular}{l}
|\input{childdoc.def}|\\
|\childdocof{|\textit{main}|}|\\
\end{tabular}
\end{center}
at the top of every child file \textit{child}
which is included by |\include{|\textit{child}|}|
from within the main file
(or at least for those files to be compiled individually).
The argument \textit{main} must be the filename of the main file.

There are a couple of
considerations in setting up the main and child documents:

%%%%%%%%%%%%%%%%%%%%%%%%%%%%%%%%%%%%%%%%
\paragraph{Restrictions.}

Please note the following restrictions:
\begin{itemize}
\item
|\childdocmain| must be called with one argument \textit{main}
to ensure compatibility with earlier version of the package.
It must either be empty (|\childdocmain{}|)
or precisely match the filename of the main file in which it is specified.
See \secref{sec:detection} for further information.
\item
The filename \textit{main} must be specified without the |.tex| extension.
\item
The filename \textit{main} is case sensitive
(even in case-insensitive file systems)
due to internal string comparison.
\item
The argument \textit{main} should be fully expanded, it cannot be a macro.
\item
Subdirectories and special characters should be avoided in filenames.
\item
The command |\childdocmain{|\textit{main}|}| must be followed by a whitespace.
It should not be followed immediately by another command
or by a comment mark `|%|'.
This is because the \TeX{} parser reads the token immediately following
the argument of |\childdocmain| and puts it
at the beginning of every child section;
however, a white\-space is ignored.
\end{itemize}

%%%%%%%%%%%%%%%%%%%%%%%%%%%%%%%%%%%%%%%%
\paragraph{Content of Main File.}

It is advisable to place all content in the child files included by |\include|.
Any output contained in the main file will appear in all child documents
unless suppressed manually;
it cannot be suppressed automatically by the |\includeonly| directive
and thus should normally be avoided.
A method to include some content in the main file
by means of conditional processing is described in \secref{sec:conditional}.

%%%%%%%%%%%%%%%%%%%%%%%%%%%%%%%%%%%%%%%%
\paragraph{Page Numbering.}

When only a part of the document is compiled,
the appropriate numbering of pages
(as well as other status parameters)
is determined from the |.aux| files.
The latter contain information from previous passes.
However this information needs to propagate through
all intermediate child documents.
Therefore the page numbering in child documents may well
be inconsistent until the complete document is compiled at least once.

A useful (if unconventional) way to always ensure a consistent
page numbering is to restart the numbering in each child document
and denote the pages by `\textit{child}|.|\textit{page}'
where \textit{child} represents the chapter/section number of the child file.
This can be achieved by the command
|\numberwithin{page}{|\textit{child}|}|
of the \textsf{amsmath} package
where \textit{child} can be |chapter| or |section|
depending on the chosen structuring.
Alternatively, one can modify the macro |\thepage| appropriately
and reset the counter |page| at the start of each child file.

%%%%%%%%%%%%%%%%%%%%%%%%%%%%%%%%%%%%%%%%%%%%%%%%%%%%%%%%%%%%%%%%%%%%%%%%%%%%%%%%
\subsection{Conditional Processing}
\label{sec:conditional}

The package provides a mechanism to compile different versions
of a document. To customise the versions further some conditional processing
can come in handy to distinguish which version is being compiled.
The package provides two macros to describe the compilation context:

%%%%%%%%%%%%%%%%%%%%%%%%%%%%%%%%%%%%%%%%
\DescribeMacro{\ifchilddoc}
The conditional |\ifchilddoc| distinguishes between the compilation of
child documents and the main document:
%
\begin{center}
|\ifchilddoc |\textit{child-code}| |[|\||else |\textit{main-code}]| \||fi|
\end{center}

%%%%%%%%%%%%%%%%%%%%%%%%%%%%%%%%%%%%%%%%
\DescribeMacro{\childdocname}
\DescribeMacro{\childdocjob}
The macro |\childdocname| contains the filename (without extension)
of the main or child file being processed.
Note that |\childdocjob| will always contain the name of the main file.

%%%%%%%%%%%%%%%%%%%%%%%%%%%%%%%%%%%%%%%%
\paragraph{Title Page.}

Conditional processing can be used to include a title or banner page
in the main document when proper precautions are taken.
Importantly, the code in the main file should ensure that the page counter
(as well as other status parameters which are stored in the |.aux| files)
takes the same value after the conditional processing.
Otherwise the page numbers may take divergent values
depending on which part is compiled.

For example, a title page could be declared by:
%
\begin{center}
\begin{tabular}{l}
|\ifchilddoc\||else|\\
|\addtocounter{page}{-1}|\\
\textit{code for title page}\\
|\newpage|\\
|\||fi|
\end{tabular}
\end{center}
%
A banner page for the child documents can be generated by:
%
\begin{center}
\begin{tabular}{l}
|\ifchilddoc|\\
|\addtocounter{page}{-1}|\\
\textit{code for banner page}\\
|\newpage|\\
|\||fi|
\end{tabular}
\end{center}
%
Here one could write a message such as:
\begin{center}
|This is the part \childdocname{} of \childdocjob{}.|
\end{center}

%%%%%%%%%%%%%%%%%%%%%%%%%%%%%%%%%%%%%%%%%%%%%%%%%%%%%%%%%%%%%%%%%%%%%%%%%%%%%%%%
\subsection{Flags}
\label{sec:flags}

The package makes it easy to generate different versions
of the main or child documents.
To this end compilation flags can be defined
and assigned different default values.
They will be particularly useful in conjunction
with the forwarding mechanism described in \secref{sec:forward}.

For example, it may be useful to have a flag |\version|
which can be set to |draft| or |final|.
The document source will contain some conditional code
depending on the value of |\version|.
Suppose further, the flag should default to |final| for the main file
and to |draft| for child files
which is a natural assignment for editing the document.
This is achieved by placing the following code
in the preamble of the main document
(below the |\childdocmain| directive):
%
\begin{center}
\begin{tabular}{l}
|\ifchilddoc|\\
|\providecommand{\version}{draft}|\\
|\||else|\\
|\providecommand{\version}{final}|\\
|\||fi|
\end{tabular}
\end{center}
%
The definition by |\providecommand| makes sure
that previous definitions are not overwritten.
Further statements |\providecommand{\version}{...}|
can thus be added before the above code to override it.

For the main file, one might add a line
(between |\childdocmain| and the above block)
%
\begin{center}
|%\ifchilddoc\||else\providecommand{\version}{draft}\||fi|
\end{center}
%
which can be uncommented to produce a draft version.
Likewise one can add a line to the very top of a child file
(above the |\childdocof{|\textit{main}|}| directive)
%
\begin{center}
|%\providecommand{\version}{final}|
\end{center}
%
which can be uncommented to produce the final version of this child document.

%%%%%%%%%%%%%%%%%%%%%%%%%%%%%%%%%%%%%%%%%%%%%%%%%%%%%%%%%%%%%%%%%%%%%%%%%%%%%%%%
\subsection{Forwarding}
\label{sec:forward}

Different versions of the main or child documents
using compilation flags as described in \secref{sec:flags}
can be (permanently) stored in different files
for convenient compilation, viewing and distribution.
To this end, the package defines a command
to pass on compilation to a different file:

%%%%%%%%%%%%%%%%%%%%%%%%%%%%%%%%%%%%%%%%
\DescribeMacro{\childdocforward}
The command |\childdocforward| redirects processing to
another source file:
%
\begin{center}
\begin{tabular}{l}
|\input{childdoc.def}|\\
|\childdocforward[|\textit{main}|]{|\textit{dest}|}|\\
\end{tabular}
\end{center}
%
The argument \textit{dest} is the destination file
(without extension).
It should be the main file or one of the child files.
Note that further \textsf{childdoc} directives
such as |\childdocof| and |\childdocforward|
in the indicated file will be processed in this form.
The optional argument \textit{main}
passes on directly to the main file \textit{main}
while pretending to compile the child \textit{dest}.
This form behaves as if \textit{dest}
issues |\childdocof{|\textit{main}|}| right away,
and no further \textsf{childdoc} directives will be processed.

%%%%%%%%%%%%%%%%%%%%%%%%%%%%%%%%%%%%%%%%
\DescribeMacro{\...prefix}
In the alternative form |\childdocforwardprefix|,
%
\begin{center}
\begin{tabular}{l}
|\input{childdoc.def}|\\
|\childdocforwardprefix[|\textit{main}|]{|\textit{prefix}|}{|\textit{dest}|}|
\end{tabular}
\end{center}
%
the destination file is determined by a pattern
depending on the current file:
To make this work, the current file must be called
`{\textit{prefix}\hspace{0.2em}\textit{suffix}}'
with \textit{prefix} matching precisely the argument.
Processing is then passed on to the file
`{\textit{dest}\hspace{0.2em}\textit{suffix}}'.
Surely, the same effect is achieved by
directly specifying the
argument `{\textit{dest}\hspace{0.2em}\textit{suffix}}'
in the first form.
However, that requires to set up a different file
for each child. With the alternative form of the command
all these files can have exactly the same content
which simplifies setting them up and maintaining them.

For example, the following file |draft.tex|
with a compilation flag |\version| as described in \secref{sec:flags}
compiles the main document as a draft:
%
\begin{center}
\begin{tabular}{l}
|\def\version{draft}|\\
|\input{childdoc.def}|\\
|\childdocforward{|\textit{main}|}|
\end{tabular}
\end{center}
%
Likewise, the following files |final|\textit{nn}|.tex|
compile the final version of the child document
|child|\textit{nn}|.tex|:
%
\begin{center}
\begin{tabular}{l}
|\def\version{final}|\\
|\input{childdoc.def}|\\
|\childdocforwardprefix{final}{child}|
\end{tabular}
\end{center}
%

Note that when several versions of a main file and/or of each child file
are to be generated, it may be convenient to set up a |Makefile| or
shell script to automatise the process.

%%%%%%%%%%%%%%%%%%%%%%%%%%%%%%%%%%%%%%%%%%%%%%%%%%%%%%%%%%%%%%%%%%%%%%%%%%%%%%%%
\subsection{Command Line Processing}
\label{sec:commandline}

The effect of redirection files can also be achieved by invoking
the \LaTeX{} compiler with a more elaborate command line.
Most conveniently this should be done as part
of a shell script or a |Makefile|.

When using \textsf{childdoc} in the main file, the following
command lines effectively perform a redirection
(note that depending on the shell being used,
backslashes may have to be doubled: `|\|' $\to$ `|\\|'):
%
\begin{center}
|... -jobname "|\textit{target}|" |\\|"|[\textit{flags}]%
|\input{childdoc.def}\childdocforward[|\textit{main}|]{|\textit{dest}|}"|
\end{center}
%
Here \textit{target} is the name of the output file,
\textit{main} is the name of the main file
and \textit{dest} is the name of the main or child file to be processed
(all filenames without extensions).
The optional argument \textit{main} can be omitted
if \textit{main} matches \textit{dest}.
Optionally, compilation \textit{flags} can be defined via |\def| commands.
This command line makes the \TeX{} engine believe
it is compiling the file \textit{target}
whose content is specified as the latter parameter.
The provided code then forwards the processing to
\textit{main} or \textit{dest} as described in \secref{sec:forward}.

%%%%%%%%%%%%%%%%%%%%%%%%%%%%%%%%%%%%%%%%%%%%%%%%%%%%%%%%%%%%%%%%%%%%%%%%%%%%%%%%
\subsection{Include by Input}
\label{sec:input}

Including child documents by |\include| has some restrictions by design.
Most notably, the content of a child document always occupies
its own set of pages; pages cannot be shared between child documents.
Usually, this behaviour makes perfect sense
because each child document contain an essential part of the document.
However, in some situations it may be desirable to compose
a document from a collection of parts
without having mandatory page breaks between then.
For this case, the package
provides a mechanism to include parts
by |\input| which can also be processed individually.
However, by construction this mechanism
requires manual handling of the content to be output.

%%%%%%%%%%%%%%%%%%%%%%%%%%%%%%%%%%%%%%%%
\DescribeMacro{\ifchilddocmanual}
The main file should be prepared as usual, see \secref{sec:include}.
However, the document body must make a distinction
between processing of an individual part and of the main document, e.g.:
%
\begin{center}
\begin{tabular}{l}
|\ifchilddocmanual|\\
|\input{\childdocname}|\\
|\||else|\\
\textit{document body with }|\input{|\textit{part}|}|\\
|\||fi|
\end{tabular}
\end{center}
%
The conditional |\ifchilddocmanual| is true whenever
a part to be included by |\input| is being compiled,
and the name of the part is stored in |\childdocname|.

%%%%%%%%%%%%%%%%%%%%%%%%%%%%%%%%%%%%%%%%
\DescribeMacro{\childdocby}
Each part to be included by |\input| should start with:
%
\begin{center}
\begin{tabular}{l}
|\input{childdoc.def}|\\
|\childdocby{|\textit{main}|}|\\
\end{tabular}
\end{center}
%
The directive |\childdocby| is similar to |\childdocof|
described in \secref{sec:include},
but the subsequent selection of content must be done manually.
To that end, both |\ifchilddoc| and |\ifchilddocmanual|
will be true upon processing of a part,
and the name of the part is stored in |\childdocname|.
Note that |\jobname| will be set to the filename of the current part
so that each part receives an individual |.aux| file
that does not interfere with the |.aux| file(s) of the main document.
This behaviour can be altered by the alternative form
|\childdocby[*]{|\textit{main}|}| (with a non-empty optional argument)
which uses the |.aux| file of the main document
by setting |\jobname| to \textit{main}.

%%%%%%%%%%%%%%%%%%%%%%%%%%%%%%%%%%%%%%%%%%%%%%%%%%%%%%%%%%%%%%%%%%%%%%%%%%%%%%%%
\subsection{Driver Development}
\label{sec:driver}

The \textsf{childdoc} mechanism can also be use for the development
of definition files such as \LaTeX{} styles or classes.
This case differs from the above setup with multiple parts
included by |\include| in that no |\includeonly| should be invoked.
This can be achieved by starting the include file
(before |\ProvidesPackage|) with:
%
\begin{center}
\begin{tabular}{l}
|\input{childdoc.def}|\\
|\childdocforward{|\textit{main}|}|\\
\end{tabular}
\end{center}
%
or alternatively with:
%
\begin{center}
\begin{tabular}{l}
|\input{childdoc.def}|\\
|\childdocby{|\textit{main}|}|\\
\end{tabular}
\end{center}
%
Both forms have slightly different effects as described above.
The main file is prepared as usual, see \secref{sec:include}.

%%%%%%%%%%%%%%%%%%%%%%%%%%%%%%%%%%%%%%%%%%%%%%%%%%%%%%%%%%%%%%%%%%%%%%%%%%%%%%%%
\subsection{Legacy Detection}
\label{sec:detection}

The directive |\childdocmain| in the main file can detect
whether the complete document or merely a child is to be compiled
even without using the directive |\childdocof|.
This method is deprecated because it is less robust
and there is no compelling reason to use it;
it is merely provided for backward compatibility
and it may be removed in future versions.

If the detection mechanism is to be used,
it is mandatory to correctly specify
the filename of the main file as the argument of |\childdocmain|:
%
\begin{center}
\begin{tabular}{l}
|\input{childdoc.def}|\\
|\childdocmain{|\textit{main}|}|\\
\end{tabular}
\end{center}
%
If |\jobname| does not match the argument \textit{main} of |\childdocmain|,
it is assumed that |\jobname| points to the child file to be compiled.
When using |\childdocmain| with the main file specified as argument,
it suffices to start a child file
with just |\input{|\textit{main}|}|
without loading of the package and using |\childdocof|.
If instead all processing is done
with the appropriate \textsf{childdoc} directives,
the argument of \textit{main} of |\childdocmain| can be empty.

An alternative version of the command line processing described
in \secref{sec:commandline} using the detection mechanism reads:
%
\begin{center}
|... -jobname "|\textit{target}|" "|[\textit{flags}]%
[|\def\jobname{|\textit{dest}|}|]|\input{|\textit{main}|}"|
\end{center}

%%%%%%%%%%%%%%%%%%%%%%%%%%%%%%%%%%%%%%%%%%%%%%%%%%%%%%%%%%%%%%%%%%%%%%%%%%%%%%%%
\subsection{Manual Code}
\label{sec:manual}

In case one cannot be certain whether the definitions file |childdoc.def|
is installed on the target \TeX{} distribution
and one prefers not to ship it,
it is conceivable to paste a few relevant commands into the sources.

To that end, drop all statements |\input{childdoc.def}|
and perform the replacements as outlined below.
Instead of |\childdocmain{|\textit{main}|}| add the following code
to the top of the main file:
%
\begin{center}
\begin{tabular}{l}
|\||ifdefined\childdocname\endinput\||fi\newif\ifchilddoc|\\
|\edef\childdocname{\scantokens\expandafter{\jobname\noexpand}}|\\
|\def\childdocmain{|\textit{main}|}\||ifx\childdocmain\childdocname\||else|\\
|\childdoctrue\includeonly{\childdocname}\let\jobname\childdocmain\||fi|\\
\end{tabular}
\end{center}
%
Instead of |\childdocof{|\textit{main}|}| just include the main file
at the top of each child file:
%
\begin{center}
|\input{|\textit{main}|}|
\end{center}
%
A simple redirection |\childdocforward{|\textit{dest}|}| is achieved by:
%
\begin{center}
|\def\jobname{|\textit{dest}|}\input{\jobname}|
\end{center}
%
The redirection with prefix
|\childdocforwardprefix[|\textit{prefix}|]{|\textit{dest}|}|
is accomplished by:
%
\begin{center}
\begin{tabular}{l}
|{\edef\jobname{\scantokens\expandafter{\jobname\noexpand}}|\\
|\def\redirectjob |\textit{prefix}|#1~~~{\gdef\jobname{|\textit{dest}|#1}}|\\
|\expandafter\redirectjob\jobname~~~}\input{\jobname}|
\end{tabular}
\end{center}

In an alternative approach,
child documents can be compiled by a specific command line
without additional code or specific definitions:
%
\begin{center}
|... -jobname "|\textit{target}|" "|[\textit{flags}]%
|\includeonly{|\textit{dest}|}\input{|\textit{main}|}"|
\end{center}
%

%%%%%%%%%%%%%%%%%%%%%%%%%%%%%%%%%%%%%%%%%%%%%%%%%%%%%%%%%%%%%%%%%%%%%%%%%%%%%%%%
%%%%%%%%%%%%%%%%%%%%%%%%%%%%%%%%%%%%%%%%%%%%%%%%%%%%%%%%%%%%%%%%%%%%%%%%%%%%%%%%
\section{Information}

%%%%%%%%%%%%%%%%%%%%%%%%%%%%%%%%%%%%%%%%%%%%%%%%%%%%%%%%%%%%%%%%%%%%%%%%%%%%%%%%
\subsection{Copyright}

Copyright \copyright{} 2017--2018 Niklas Beisert

This work may be distributed and/or modified under the
conditions of the \LaTeX{} Project Public License, either version 1.3
of this license or (at your option) any later version.
The latest version of this license is in
  \url{http://www.latex-project.org/lppl.txt}
and version 1.3 or later is part of all distributions of \LaTeX{}
version 2005/12/01 or later.

This work has the LPPL maintenance status `maintained'.

The Current Maintainer of this work is Niklas Beisert.

This work consists of the files |README.txt|, |childdoc.ins| and |childdoc.dtx|
as well as the derived files |childdoc.def|, |cdocsamp.tex|
with |cdocsch1.tex|, |cdocsch2.tex|, |cdocspt3.tex|, |cdocspt4.tex|,
|cdocsdrf.tex|, |cdocsfn1.tex|, |cdocsfn2.tex|
as well as |childdoc.pdf|.

%%%%%%%%%%%%%%%%%%%%%%%%%%%%%%%%%%%%%%%%%%%%%%%%%%%%%%%%%%%%%%%%%%%%%%%%%%%%%%%%
\subsection{Files and Installation}

The package consists of the files:
%
\begin{center}
\begin{tabular}{ll}
    |README.txt|   & readme file \\
    |childdoc.ins| & installation file \\
    |childdoc.dtx| & source file \\
    |childdoc.def| & definition file \\
    |cdocsamp.tex| & sample main file \\
    |cdocsch1.tex| & sample include file \\
    |cdocsch2.tex| & sample include file \\
    |cdocspt3.tex| & sample part file \\
    |cdocspt4.tex| & sample part file \\
    |cdocsdrf.tex| & sample redirection file \\
    |cdocsfn1.tex| & sample redirection file \\
    |cdocsfn2.tex| & sample redirection file \\
    |childdoc.pdf| & manual
\end{tabular}
\end{center}
%
The distribution consists of the files
|README.txt|, |childdoc.ins| and |childdoc.dtx|.
%
\begin{itemize}
\item
Run (pdf)\LaTeX{} on |childdoc.dtx|
to compile the manual |childdoc.pdf| (this file).
\item
Run \LaTeX{} on |childdoc.ins| to create the definitions file |childdoc.def|
and the sample |cdocsamp.tex| with include files
|cdocsch1.tex|, |cdocsch2.tex|, |cdocspt3.tex|, |cdocspt4.tex|,
|cdocsdrf.tex|, |cdocsfn1.tex|, |cdocsfn2.tex|.
Then copy the file |childdoc.def| to an appropriate directory of your \LaTeX{}
distribution, e.g.\ \textit{texmf-root}|/tex/latex/childdoc|.
\end{itemize}

%%%%%%%%%%%%%%%%%%%%%%%%%%%%%%%%%%%%%%%%%%%%%%%%%%%%%%%%%%%%%%%%%%%%%%%%%%%%%%%%
\subsection{Related CTAN Packages}

There are several other packages which offer a similar functionality:
%
\begin{itemize}
\item
The packages
\href{http://ctan.org/pkg/docmute}{\textsf{docmute}},
\href{http://ctan.org/pkg/includex}{\textsf{includex}} and
\href{http://ctan.org/pkg/standalone}{\textsf{standalone}}
provide commands to include only the document body of
a child file thus allowing both files to be compiled individually.
\item
The packages \href{http://ctan.org/pkg/subdocs}{\textsf{subdocs}}
and \href{http://ctan.org/pkg/subfiles}{\textsf{subfiles}}
provide structures in which the main and child documents can be
encapsulated and allowing them to be compiled individually.
The inclusion mechanism is different from the conventional |\include|.
\item
The package \href{http://ctan.org/pkg/combine}{\textsf{combine}}
is an elaborate solution to combine several documents into one.
\end{itemize}
%
See also the CTAN topic \href{http://ctan.org/topic/subdocs}{\textsf{subdocs}}
for further related packages.
The present package differs from the above solutions in that
a document structure constructed with the conventional |\include| mechanism
just needs two extra commands at the top of every file
such that all constituent files can be compiled individually.

%%%%%%%%%%%%%%%%%%%%%%%%%%%%%%%%%%%%%%%%%%%%%%%%%%%%%%%%%%%%%%%%%%%%%%%%%%%%%%%%
%\subsection{Feature Suggestions}
%
%The following is a list of features which may be useful for future
%versions of this package:
%%
%\begin{itemize}
%\item
%\ldots
%\end{itemize}

%%%%%%%%%%%%%%%%%%%%%%%%%%%%%%%%%%%%%%%%%%%%%%%%%%%%%%%%%%%%%%%%%%%%%%%%%%%%%%%%
\subsection{Revision History}

%%%%%%%%%%%%%%%%%%%%%%%%%%%%%%%%%%%%%%%%
\paragraph{v2.0:} 2018/12/30

\begin{itemize}
\item
immediate forward processing
\item
added |\childdocby| mechanism
\item
manual restructured
\end{itemize}

%%%%%%%%%%%%%%%%%%%%%%%%%%%%%%%%%%%%%%%%
\paragraph{v1.6:} 2018/01/17

\begin{itemize}
\item
application for development of include files
\item
corrections to manual
\end{itemize}

%%%%%%%%%%%%%%%%%%%%%%%%%%%%%%%%%%%%%%%%
\paragraph{v1.5:} 2017/05/21

\begin{itemize}
\item
more complete structuring introduced
\item
|\childdocof| introduced
\item
|\childdoc| renamed to |\childdocmain|
\item
|\childredirect| renamed to |\childdocforward| and |\childdocforwardprefix|
and functionality expanded
\end{itemize}

%%%%%%%%%%%%%%%%%%%%%%%%%%%%%%%%%%%%%%%%
\paragraph{v1.0:} 2017/04/27

\begin{itemize}
\item
manual and install package
\item
first version published on CTAN
\end{itemize}

%%%%%%%%%%%%%%%%%%%%%%%%%%%%%%%%%%%%%%%%
\paragraph{v0.6:} 2017/04/26

\begin{itemize}
\item
redirection mechanism added
\end{itemize}

%%%%%%%%%%%%%%%%%%%%%%%%%%%%%%%%%%%%%%%%
\paragraph{v0.5:} 2017/04/26

\begin{itemize}
\item
functionality in definition file
\end{itemize}


%%%%%%%%%%%%%%%%%%%%%%%%%%%%%%%%%%%%%%%%%%%%%%%%%%%%%%%%%%%%%%%%%%%%%%%%%%%%%%%%
%%%%%%%%%%%%%%%%%%%%%%%%%%%%%%%%%%%%%%%%%%%%%%%%%%%%%%%%%%%%%%%%%%%%%%%%%%%%%%%%
%%%%%%%%%%%%%%%%%%%%%%%%%%%%%%%%%%%%%%%%%%%%%%%%%%%%%%%%%%%%%%%%%%%%%%%%%%%%%%%%
\appendix

\settowidth\MacroIndent{\rmfamily\scriptsize 000\ }

 \DocInput{childdoc.dtx}

\end{document}
%</driver>
% \fi
%
% %%%%%%%%%%%%%%%%%%%%%%%%%%%%%%%%%%%%%%%%%%%%%%%%%%%%%%%%%%%%%%%%%%%%%%%%%%%%%%
% %%%%%%%%%%%%%%%%%%%%%%%%%%%%%%%%%%%%%%%%%%%%%%%%%%%%%%%%%%%%%%%%%%%%%%%%%%%%%%
% \section{Sample}
%\iffalse
%<*samplemain>
%\fi
%
% The following presents a sample document
% with two chapters, two parts, a title page,
% a compile flag as well as three forwarding files to set the flag.
% It consists of eight |.tex| files:
% \begin{center}
% \begin{tabular}{ll}
% |cdocsamp.tex|&main file\\
% |cdocsch1.tex|&include file for chapter 1\\
% |cdocsch2.tex|&include file for chapter 2\\
% |cdocspt3.tex|&include file for part 3\\
% |cdocspt4.tex|&include file for part 4\\
% |cdocsdrf.tex|&forwarding file for main file in draft mode\\
% |cdocsfi1.tex|&forwarding file for final version of chapter 1\\
% |cdocsfi2.tex|&forwarding file for final version of chapter 2\\
% \end{tabular}
% \end{center}
% Each of the eight files can be compiled directly by the \LaTeX{} compiler.
%
% %%%%%%%%%%%%%%%%%%%%%%%%%%%%%%%%%%%%%%
% \paragraph{Main File.}
%
% The main file is called |cdocsamp.tex|.
%
% Load the \textsf{childdoc} definitions and
% declare the filename for the main document:
%    \begin{macrocode}
\input{childdoc.def}
\childdocmain{}
%    \end{macrocode}

% Optional override for |\version| flag:
%    \begin{macrocode}
%%\ifchilddoc\else\providecommand{\version}{draft}\fi
%    \end{macrocode}

% Define the default values for the |\version| flag
% (|final| for the main file and |draft| for childs):
%    \begin{macrocode}
\ifchilddoc
\providecommand{\version}{draft}
\else
\providecommand{\version}{final}
\fi
%    \end{macrocode}

% Load the standard document class:
%    \begin{macrocode}
\documentclass[12pt]{article}
%    \end{macrocode}

% Start the document body:
%    \begin{macrocode}
\begin{document}
%    \end{macrocode}

% Declare a title page.
% Print title, part of document being processed and version flag:
%    \begin{macrocode}
\addtocounter{page}{-1}
\begin{center}
{\LARGE\bfseries{}childdoc example\par}
\vspace{1cm}
\ifchilddoc
\ifchilddocmanual part\else chapter\fi:
`\childdocname' of `\childdocjob'\par
\else
main document: `\childdocjob'\par
\fi
version: \version\par
\end{center}
\newpage
%    \end{macrocode}

% Manually include selected file,
% otherwise process as usual:
%    \begin{macrocode}
\ifchilddocmanual
\section*{part `\childdocname'}
\input{\childdocname}
\else
%    \end{macrocode}

% Include the two chapters:
%    \begin{macrocode}
\include{cdocsch1}
\include{cdocsch2}
%    \end{macrocode}

% Include the two parts unless only chapters should be displayed:
%    \begin{macrocode}
\ifchilddoc\else
\section{part three}
\input{cdocspt3}
\section{part four}
\input{cdocspt4}
\fi
%    \end{macrocode}

% Process as usual until here:
%    \begin{macrocode}
\fi
%    \end{macrocode}

% End of document body:
%    \begin{macrocode}
\end{document}
%    \end{macrocode}
%\iffalse
%</samplemain>
%\fi
%
% %%%%%%%%%%%%%%%%%%%%%%%%%%%%%%%%%%%%%%
% \paragraph{Chapter Include Files.}
%
% The include files are called |cdocsch1.tex| and |cdocsch2.tex|.
%
%\iffalse
%<*samplechap1|samplechap2>
%\fi

% Optional override for |\version| flag:
%    \begin{macrocode}
%%\providecommand{\version}{final}
%    \end{macrocode}

% Include the main document:
%    \begin{macrocode}
\input{childdoc.def}
\childdocof{cdocsamp}
%    \end{macrocode}

%\iffalse
%</samplechap1|samplechap2>
%\fi
%
%\iffalse
%<*samplechap1>
%\fi
% Some text for chapter 1:
%    \begin{macrocode}
\section{one}
some text in chapter one
%    \end{macrocode}

%\iffalse
%</samplechap1>
%\fi
% Some text for chapter 2:
%\iffalse
%<*samplechap2>
%\fi
%    \begin{macrocode}
\section{two}
more text in chapter two
%    \end{macrocode}

%\iffalse
%</samplechap2>
%\fi
%
% %%%%%%%%%%%%%%%%%%%%%%%%%%%%%%%%%%%%%%
% \paragraph{Part Include Files.}
%
% The include files are called |cdocspt3.tex| and |cdocspt4.tex|.
%
%\iffalse
%<*samplepart3|samplepart4>
%\fi

% Optional override for |\version| flag:
%    \begin{macrocode}
%%\providecommand{\version}{final}
%    \end{macrocode}

% Include the main document:
%    \begin{macrocode}
\input{childdoc.def}
\childdocby{cdocsamp}
%    \end{macrocode}

%\iffalse
%</samplepart3|samplepart4>
%\fi
%
%\iffalse
%<*samplepart3>
%\fi
% Some text for part 3:
%    \begin{macrocode}
some text in part three
%    \end{macrocode}

%\iffalse
%</samplepart3>
%\fi
% Some text for part 4:
%\iffalse
%<*samplepart4>
%\fi
%    \begin{macrocode}
more text in part four
%    \end{macrocode}

%\iffalse
%</samplepart4>
%\fi
%
% %%%%%%%%%%%%%%%%%%%%%%%%%%%%%%%%%%%%%%
% \paragraph{Forwarding for a Complete Draft.}
%
% The following forwarding file |cdocsdrf.tex|
% compiles the main document in draft mode:
%\iffalse
%<*sampledraft>
%\fi
%    \begin{macrocode}
\def\version{draft}
\input{childdoc.def}
\childdocforward{cdocsamp}
%    \end{macrocode}

%\iffalse
%</sampledraft>
%\fi
%
% %%%%%%%%%%%%%%%%%%%%%%%%%%%%%%%%%%%%%%
% \paragraph{Forwarding for Final Version of the Chapters.}
%
% The following forwarding files |cdocsfn1.tex| and |cdocsfn2.tex|
% (with identical content)
% compile the final versions of the child documents
% |cdocsch1.tex| and |cdocsch2.tex|, respectively:
%\iffalse
%<*samplefinal>
%\fi
%    \begin{macrocode}
\def\version{final}
\input{childdoc.def}
\childdocforwardprefix[cdocsamp]{cdocsfn}{cdocsch}
%    \end{macrocode}

%\iffalse
%</samplefinal>
%\fi
%
% %%%%%%%%%%%%%%%%%%%%%%%%%%%%%%%%%%%%%%
% \paragraph{Command Line Processing.}
%
% The following three command lines generate the output files
% |cdocscld|, |cdocscl1| and |cdocscl2|
% which should be identical to
% |cdocsdrf|, |cdocsch1| and |cdocsfn2|, respectively:
% \begin{center}
% \begin{tabular}{l}
% |latex -jobname cdocscld \|\\
% |  "\def\version{draft}\input{childdoc.def}\childdocforward{cdocsamp}"|\\
% |latex -jobname cdocscl1 \|\\
% |  "\input{childdoc.def}\childdocforward[cdocsamp]{cdocsch1}"|\\
% |latex -jobname cdocscl2 \|\\
% |  "\def\version{final}\input{childdoc.def}\childdocforward{cdocsch2}"|
% \end{tabular}
% \end{center}
% Note that the trailing backslash on each first line
% merely continues the input to the second line
% (for convenient cut ant paste).
% Furthermore, the command |latex| can be replaced by any
% of its alternative versions such as |pdflatex|.
%
% %%%%%%%%%%%%%%%%%%%%%%%%%%%%%%%%%%%%%%%%%%%%%%%%%%%%%%%%%%%%%%%%%%%%%%%%%%%%%%
% %%%%%%%%%%%%%%%%%%%%%%%%%%%%%%%%%%%%%%%%%%%%%%%%%%%%%%%%%%%%%%%%%%%%%%%%%%%%%%
% \section{Implementation}
%\iffalse
%<*package>
%\fi
%
% This section describes the definitions file |childdoc.def|.

% The definitions cannot be loaded using |\usepackage| or |\RequirePackage|
% which has a mechanism to prevent loading a style file more than once.
% When loading the definitions by means of |\input|
% multiple instances have to be prevented manually:
%\iffalse
%This code needs to be before the `\ProvidesFile' directive
%which is defined at the beginning of this file.
%Therefore it is also placed there and commented out here.
%</package>
%<*discard>
%\fi
%    \begin{macrocode}
\ifdefined\childdocmain\endinput\fi
%    \end{macrocode}
%\iffalse
%</discard>
%<*package>
%\fi
%
% \macro{\ifchilddoc}
% \macro{\ifchilddocmanual}
% The conditional |\ifchilddoc| tells whether a
% child (true) or main (false) document is being compiled.
% The conditional |\ifchilddocmanual| tells whether
% the |\includeonly| mechanism is used (false) or
% the selection of child files must be performed manually (true).
% The definitions initialise to false:
%    \begin{macrocode}
\newif\ifchilddoc
\newif\ifchilddocmanual
%    \end{macrocode}

% \macro{\childdocname}
% \macro{\childdocjob}
% The macro |\childdocname| stores the name of the main document
% to be compiled. The macro |\childdocjob| stores the name of
% the document on which the \LaTeX{} compiler was originally invoked.
% The content of |\jobname| cannot be compared
% to filenames specified in the source due to different catcodes.
% The following code rescans |\jobname|, stores the result
% in |\childdocname| and saves a copy in |\childdocjob|:
%    \begin{macrocode}
\edef\childdocname{\scantokens\expandafter{\jobname\noexpand}}
\let\childdocjob\childdocname
%    \end{macrocode}

% \macro{\childdocdisable}
% The macro |\childdocdisable| prevents the main file
% from being processed more than once.
% At this stage, the main document command |\childdocmain|
% is assumed to be called once again where it should do nothing.
% Any subsequent call to it should prevent
% a secondary processing of the main document
% It overwrites the forwarding commands
% |\childdocof| and |\childdocforward|
% with empty macros to prevent further inclusions of the main document:
%    \begin{macrocode}
\newcommand{\childdocdisable}
{
  \renewcommand{\childdocmain}[1]{\renewcommand{\childdocmain}[1]{\endinput}}
  \renewcommand{\childdocof}[1]{}
  \renewcommand{\childdocby}[2][]{}
  \renewcommand{\childdocforward}[2][]{}
  \renewcommand{\childdocdisable}{}
}
%    \end{macrocode}

% \macro{\childdocmain}
% The macro |\childdocmain| is to be called at the top of the main file
% with nothing or the main filename (without extension) as argument.
% First, it breaks loops.
% If the argument is not empty and does not match |\childdocname|
% (which is set by the first inclusion of |childdoc.def|),
% |\ifchilddoc| is set to true, |\includeonly| is applied to the child file
% and |\jobname| is set to the main file
% (for proper handling of |.aux| files):
%    \begin{macrocode}
\newcommand{\childdocmain}[1]
{
  \childdocdisable\childdocmain{}
  \if?#1?\else
    \begingroup
      \def\childdoctmp{#1}
      \ifx\childdoctmp\childdocname
        \def\childdoctmp{}
      \else
        \def\childdoctmp
        {
          \childdoctrue
          \includeonly{\childdocname}
          \def\childdocjob{#1}
          \def\jobname{#1}
        }
      \fi
      \expandafter
    \endgroup
    \childdoctmp
  \fi
}
%    \end{macrocode}

% \macro{\childdocof}
% The command |\childdocof| redirects
% compilation to the main file |#1|.
%    \begin{macrocode}
\newcommand{\childdocof}[1]
{
  \childdocdisable
  \childdoctrue
  \includeonly{\childdocname}
  \def\jobname{#1}
  \def\childdocjob{#1}
  \input{#1}
}
%    \end{macrocode}

% \macro{\childdocby}
% The command |\childdocby| ....
%    \begin{macrocode}
\newcommand{\childdocby}[2][]
{
  \childdocdisable
  \childdoctrue
  \childdocmanualtrue
  \if?#1?\else
    \def\jobname{#2}
  \fi
  \def\childdocjob{#2}
  \input{#2}
  \endinput
}
%    \end{macrocode}

% \macro{\childdocforward}
% The command |\childdocforward| redirects
% compilation to the main file or
% (if the optional argument is given) a child file.
% Parameters are set as if the main file
% or a child file starting with |\childdocof| was compiled.
% Then compilation is handed over to the main file:
%    \begin{macrocode}
\newcommand{\childdocforward}[2][]
{
  \begingroup
    \if?#1?
      \def\childdoctmp
      {
        \def\childdocname{#2}
        \def\childdocjob{#2}
        \def\jobname{#2}
        \input{#2}
        \endinput
      }
    \else
      \def\childdoctmp
      {
        \childdocdisable
        \def\childdocname{#2}
        \childdoctrue
        \includeonly{#2}
        \def\childdocjob{#1}
        \def\jobname{#1}
        \input{#1}
        \endinput
      }
    \fi
    \expandafter
  \endgroup
  \childdoctmp
}
%    \end{macrocode}

% \macro{\childdocforwardprefix}
% The command |\childdocforwardprefix| redirects
% compilation to the main or a child file by means of a pattern.
% The prefix |#1| in the current filename is replaced by |#2|
% and the suffix of the current filename is kept
% (it is assumed that the filename does not contain the substring `|~~~|'
% which is used as a delimiter).
% Compilation is handed over to the new file by |\childdocforward|:
%    \begin{macrocode}
\newcommand{\childdocforwardprefix}[3][]
{
  \begingroup
    \def\childdocextract #2##1~~~{\def\childdoctmp{\childdocforward[#1]{#3##1}}}
    \expandafter\childdocextract\childdocname~~~
    \expandafter
  \endgroup
  \childdoctmp
}
%    \end{macrocode}

% \macro{\childdoc}
% The deprecated macro |\childdoc| is a legacy version of |\childdocmain|:
%    \begin{macrocode}
\newcommand{\childdoc}{\childdocmain}
%    \end{macrocode}

% \macro{\childdocredirect}
% The deprecated macro |\childdocredirect| is a legacy version
% of |\childdocforward| and |\childdocforwardprefix|:
%    \begin{macrocode}
\newcommand{\childdocredirect}[2][]
{
  \begingroup
    \if?#1?
      \def\childdoctmp{\childdocforward{#2}}
    \else
      \def\childdoctmp{\childdocforwardprefix{#1}{#2}}
    \fi
    \expandafter
  \endgroup
  \childdoctmp
}
%    \end{macrocode}

%\iffalse
%</package>
%\fi
%
\endinput

\childdocforward{cdocsamp}
%    \end{macrocode}

%\iffalse
%</sampledraft>
%\fi
%
% %%%%%%%%%%%%%%%%%%%%%%%%%%%%%%%%%%%%%%
% \paragraph{Forwarding for Final Version of the Chapters.}
%
% The following forwarding files |cdocsfn1.tex| and |cdocsfn2.tex|
% (with identical content)
% compile the final versions of the child documents
% |cdocsch1.tex| and |cdocsch2.tex|, respectively:
%\iffalse
%<*samplefinal>
%\fi
%    \begin{macrocode}
\def\version{final}
% \iffalse
%
% childdoc.dtx Copyright (C) 2017-2018 Niklas Beisert
%
% This work may be distributed and/or modified under the
% conditions of the LaTeX Project Public License, either version 1.3
% of this license or (at your option) any later version.
% The latest version of this license is in
%   http://www.latex-project.org/lppl.txt
% and version 1.3 or later is part of all distributions of LaTeX
% version 2005/12/01 or later.
%
% This work has the LPPL maintenance status `maintained'.
%
% The Current Maintainer of this work is Niklas Beisert.
%
% This work consists of the files childdoc.dtx and childdoc.ins
% and the derived files childdoc.def and cdocsamp.tex with
% cdocsch1.tex, cdocsch2.tex, cdocsdrf.tex, cdocsfn1.tex, cdocsfn2.tex.
%
%<package>\ifdefined\childdocmain\endinput\fi
%<package>\ProvidesFile{childdoc.def}[2018/12/30 v2.0 child document driver]
%<samplemain>\ProvidesFile{cdocsamp.tex}[2018/12/30 v2.0 sample for childdoc]
%<*driver>
%\ProvidesFile{childdoc.drv}[2018/12/30 v2.0 childdoc reference manual file]
\PassOptionsToClass{10pt,a4paper}{article}
\documentclass{ltxdoc}

\usepackage[margin=35mm]{geometry}
\usepackage{hyperref}
\usepackage{hyperxmp}
\usepackage[usenames]{color}

\hypersetup{colorlinks=true}
\hypersetup{pdfstartview=FitH}
\hypersetup{pdfpagemode=UseNone}
\hypersetup{pdfsource={}}
\hypersetup{pdflang={en-UK}}
\hypersetup{pdfcopyright={Copyright 2017-2018 Niklas Beisert.
  This work may be distributed and/or modified under the
  conditions of the LaTeX Project Public License, either version 1.3
  of this license or (at your option) any later version.}}
\hypersetup{pdflicenseurl={http://www.latex-project.org/lppl.txt}}
\hypersetup{pdfcontactaddress={ETH Zurich, ITP, HIT K,
  Wolfgang-Pauli-Strasse 27}}
\hypersetup{pdfcontactpostcode={8093}}
\hypersetup{pdfcontactcity={Zurich}}
\hypersetup{pdfcontactcountry={Switzerland}}
\hypersetup{pdfcontactemail={nbeisert@itp.phys.ethz.ch}}
\hypersetup{pdfcontacturl={http://people.phys.ethz.ch/\xmptilde nbeisert/}}

\newcommand{\secref}[1]{\hyperref[#1]{section \ref*{#1}}}

\parskip1ex
\parindent0pt
\let\olditemize\itemize
\def\itemize{\olditemize\parskip0pt}

\begin{document}

\title{The \textsf{childdoc} Package}
\hypersetup{pdftitle={The childdoc Package}}
\author{Niklas Beisert\\[2ex]
  Institut f\"ur Theoretische Physik\\
  Eidgen\"ossische Technische Hochschule Z\"urich\\
  Wolfgang-Pauli-Strasse 27, 8093 Z\"urich, Switzerland\\[1ex]
  \href{mailto:nbeisert@itp.phys.ethz.ch}
  {\texttt{nbeisert@itp.phys.ethz.ch}}}
\hypersetup{pdfauthor={Niklas Beisert}}
\hypersetup{pdfsubject={Manual for the LaTeX2e Package childdoc}}
\date{30 December 2018, \textsf{v2.0}}
\maketitle

\begin{abstract}\noindent
\textsf{childdoc} is a \LaTeXe{} package
that enables the direct compilation
of document sections included by |\include|
to individual files.
\end{abstract}

\begingroup
\parskip0ex
\tableofcontents
\endgroup

%%%%%%%%%%%%%%%%%%%%%%%%%%%%%%%%%%%%%%%%%%%%%%%%%%%%%%%%%%%%%%%%%%%%%%%%%%%%%%%%
%%%%%%%%%%%%%%%%%%%%%%%%%%%%%%%%%%%%%%%%%%%%%%%%%%%%%%%%%%%%%%%%%%%%%%%%%%%%%%%%
\section{Introduction}

\LaTeX{} provides a mechanism to structure a large document (such as a book)
into a main file and several child files (containing the chapters)
using the |\include| command.
This mechanism is beneficial for documents
which span hundreds of pages in order to
make the source file(s) more manageable.
Moreover, compilation can be restricted to
selected child files by means of the |\includeonly| command.
The latter feature can be used to reduce the compilation time while editing
(this was significantly more useful in the earlier days of \LaTeX{})
or to generate a smaller document which is easier to navigate.
Another application of |\includeonly| is to generate
documents consisting of selected parts of the complete document.

However, there are a few drawbacks of the plain |\include| mechanism:
\begin{itemize}
\item
The child files cannot be compiled on their own,
they can only be compiled via the main file.
A naive editing environment
(such as a text editor with an option
to have the current file processed by \LaTeX)
may require one to switch to the main file before compiling;
attempting to compile the child file produces errors.
\item
The main file must be modified (each time)
to adjust the |\includeonly| command
to the present needs. This easily leaves the main file in a messy state.
\item
The generated document will always carry the filename
of the main document. This is inconvenient if
several child files are to be compiled and
to be kept for distribution.
\end{itemize}

The present package provides a simple interface
to make child files individually compilable by \LaTeX{}.
Compiling a child file then has the same effect as compiling
the main file with an |\includeonly| command
to select the appropriate child.
Moreover the generated document will carry the name of the child
rather than the main file.
This resolves all three above issues.

This feature is meant to make the editing of books,
thesis documents and lecture notes somewhat more convenient.
However, the package can also be used efficiently for
composing a series of documents (such as exercise sheets)
which are typically distributed individually.
It then assists the author in generating the individual documents
(potentially in different versions)
as well as a document containing the collected series.
Another application is in developing style files
or other kinds of included material
where compilation of the style file could redirect
to a sample or test file.

%%%%%%%%%%%%%%%%%%%%%%%%%%%%%%%%%%%%%%%%%%%%%%%%%%%%%%%%%%%%%%%%%%%%%%%%%%%%%%%%
%%%%%%%%%%%%%%%%%%%%%%%%%%%%%%%%%%%%%%%%%%%%%%%%%%%%%%%%%%%%%%%%%%%%%%%%%%%%%%%%
\section{Usage}

First of all, the package \textsf{childdoc} is \emph{not} a standard
\LaTeXe{} |.sty| style file! Therefore it needs to be invoked in
a non-standard way.

%%%%%%%%%%%%%%%%%%%%%%%%%%%%%%%%%%%%%%%%%%%%%%%%%%%%%%%%%%%%%%%%%%%%%%%%%%%%%%%%
\subsection{Included Files}
\label{sec:include}

%%%%%%%%%%%%%%%%%%%%%%%%%%%%%%%%%%%%%%%%
\DescribeMacro{\childdocmain}
To use the package, add the commands
\begin{center}
\begin{tabular}{l}
|\input{childdoc.def}|\\
|\childdocmain{}|\\
\end{tabular}
\end{center}
at the very top of the main \LaTeX{} file,
in particular \emph{before} the |\documentclass| statement!
The argument of |\childdocmain| should be left empty
(but it must be present).

%%%%%%%%%%%%%%%%%%%%%%%%%%%%%%%%%%%%%%%%
\DescribeMacro{\childdocof}
Furthermore, add the commands
\begin{center}
\begin{tabular}{l}
|\input{childdoc.def}|\\
|\childdocof{|\textit{main}|}|\\
\end{tabular}
\end{center}
at the top of every child file \textit{child}
which is included by |\include{|\textit{child}|}|
from within the main file
(or at least for those files to be compiled individually).
The argument \textit{main} must be the filename of the main file.

There are a couple of
considerations in setting up the main and child documents:

%%%%%%%%%%%%%%%%%%%%%%%%%%%%%%%%%%%%%%%%
\paragraph{Restrictions.}

Please note the following restrictions:
\begin{itemize}
\item
|\childdocmain| must be called with one argument \textit{main}
to ensure compatibility with earlier version of the package.
It must either be empty (|\childdocmain{}|)
or precisely match the filename of the main file in which it is specified.
See \secref{sec:detection} for further information.
\item
The filename \textit{main} must be specified without the |.tex| extension.
\item
The filename \textit{main} is case sensitive
(even in case-insensitive file systems)
due to internal string comparison.
\item
The argument \textit{main} should be fully expanded, it cannot be a macro.
\item
Subdirectories and special characters should be avoided in filenames.
\item
The command |\childdocmain{|\textit{main}|}| must be followed by a whitespace.
It should not be followed immediately by another command
or by a comment mark `|%|'.
This is because the \TeX{} parser reads the token immediately following
the argument of |\childdocmain| and puts it
at the beginning of every child section;
however, a white\-space is ignored.
\end{itemize}

%%%%%%%%%%%%%%%%%%%%%%%%%%%%%%%%%%%%%%%%
\paragraph{Content of Main File.}

It is advisable to place all content in the child files included by |\include|.
Any output contained in the main file will appear in all child documents
unless suppressed manually;
it cannot be suppressed automatically by the |\includeonly| directive
and thus should normally be avoided.
A method to include some content in the main file
by means of conditional processing is described in \secref{sec:conditional}.

%%%%%%%%%%%%%%%%%%%%%%%%%%%%%%%%%%%%%%%%
\paragraph{Page Numbering.}

When only a part of the document is compiled,
the appropriate numbering of pages
(as well as other status parameters)
is determined from the |.aux| files.
The latter contain information from previous passes.
However this information needs to propagate through
all intermediate child documents.
Therefore the page numbering in child documents may well
be inconsistent until the complete document is compiled at least once.

A useful (if unconventional) way to always ensure a consistent
page numbering is to restart the numbering in each child document
and denote the pages by `\textit{child}|.|\textit{page}'
where \textit{child} represents the chapter/section number of the child file.
This can be achieved by the command
|\numberwithin{page}{|\textit{child}|}|
of the \textsf{amsmath} package
where \textit{child} can be |chapter| or |section|
depending on the chosen structuring.
Alternatively, one can modify the macro |\thepage| appropriately
and reset the counter |page| at the start of each child file.

%%%%%%%%%%%%%%%%%%%%%%%%%%%%%%%%%%%%%%%%%%%%%%%%%%%%%%%%%%%%%%%%%%%%%%%%%%%%%%%%
\subsection{Conditional Processing}
\label{sec:conditional}

The package provides a mechanism to compile different versions
of a document. To customise the versions further some conditional processing
can come in handy to distinguish which version is being compiled.
The package provides two macros to describe the compilation context:

%%%%%%%%%%%%%%%%%%%%%%%%%%%%%%%%%%%%%%%%
\DescribeMacro{\ifchilddoc}
The conditional |\ifchilddoc| distinguishes between the compilation of
child documents and the main document:
%
\begin{center}
|\ifchilddoc |\textit{child-code}| |[|\||else |\textit{main-code}]| \||fi|
\end{center}

%%%%%%%%%%%%%%%%%%%%%%%%%%%%%%%%%%%%%%%%
\DescribeMacro{\childdocname}
\DescribeMacro{\childdocjob}
The macro |\childdocname| contains the filename (without extension)
of the main or child file being processed.
Note that |\childdocjob| will always contain the name of the main file.

%%%%%%%%%%%%%%%%%%%%%%%%%%%%%%%%%%%%%%%%
\paragraph{Title Page.}

Conditional processing can be used to include a title or banner page
in the main document when proper precautions are taken.
Importantly, the code in the main file should ensure that the page counter
(as well as other status parameters which are stored in the |.aux| files)
takes the same value after the conditional processing.
Otherwise the page numbers may take divergent values
depending on which part is compiled.

For example, a title page could be declared by:
%
\begin{center}
\begin{tabular}{l}
|\ifchilddoc\||else|\\
|\addtocounter{page}{-1}|\\
\textit{code for title page}\\
|\newpage|\\
|\||fi|
\end{tabular}
\end{center}
%
A banner page for the child documents can be generated by:
%
\begin{center}
\begin{tabular}{l}
|\ifchilddoc|\\
|\addtocounter{page}{-1}|\\
\textit{code for banner page}\\
|\newpage|\\
|\||fi|
\end{tabular}
\end{center}
%
Here one could write a message such as:
\begin{center}
|This is the part \childdocname{} of \childdocjob{}.|
\end{center}

%%%%%%%%%%%%%%%%%%%%%%%%%%%%%%%%%%%%%%%%%%%%%%%%%%%%%%%%%%%%%%%%%%%%%%%%%%%%%%%%
\subsection{Flags}
\label{sec:flags}

The package makes it easy to generate different versions
of the main or child documents.
To this end compilation flags can be defined
and assigned different default values.
They will be particularly useful in conjunction
with the forwarding mechanism described in \secref{sec:forward}.

For example, it may be useful to have a flag |\version|
which can be set to |draft| or |final|.
The document source will contain some conditional code
depending on the value of |\version|.
Suppose further, the flag should default to |final| for the main file
and to |draft| for child files
which is a natural assignment for editing the document.
This is achieved by placing the following code
in the preamble of the main document
(below the |\childdocmain| directive):
%
\begin{center}
\begin{tabular}{l}
|\ifchilddoc|\\
|\providecommand{\version}{draft}|\\
|\||else|\\
|\providecommand{\version}{final}|\\
|\||fi|
\end{tabular}
\end{center}
%
The definition by |\providecommand| makes sure
that previous definitions are not overwritten.
Further statements |\providecommand{\version}{...}|
can thus be added before the above code to override it.

For the main file, one might add a line
(between |\childdocmain| and the above block)
%
\begin{center}
|%\ifchilddoc\||else\providecommand{\version}{draft}\||fi|
\end{center}
%
which can be uncommented to produce a draft version.
Likewise one can add a line to the very top of a child file
(above the |\childdocof{|\textit{main}|}| directive)
%
\begin{center}
|%\providecommand{\version}{final}|
\end{center}
%
which can be uncommented to produce the final version of this child document.

%%%%%%%%%%%%%%%%%%%%%%%%%%%%%%%%%%%%%%%%%%%%%%%%%%%%%%%%%%%%%%%%%%%%%%%%%%%%%%%%
\subsection{Forwarding}
\label{sec:forward}

Different versions of the main or child documents
using compilation flags as described in \secref{sec:flags}
can be (permanently) stored in different files
for convenient compilation, viewing and distribution.
To this end, the package defines a command
to pass on compilation to a different file:

%%%%%%%%%%%%%%%%%%%%%%%%%%%%%%%%%%%%%%%%
\DescribeMacro{\childdocforward}
The command |\childdocforward| redirects processing to
another source file:
%
\begin{center}
\begin{tabular}{l}
|\input{childdoc.def}|\\
|\childdocforward[|\textit{main}|]{|\textit{dest}|}|\\
\end{tabular}
\end{center}
%
The argument \textit{dest} is the destination file
(without extension).
It should be the main file or one of the child files.
Note that further \textsf{childdoc} directives
such as |\childdocof| and |\childdocforward|
in the indicated file will be processed in this form.
The optional argument \textit{main}
passes on directly to the main file \textit{main}
while pretending to compile the child \textit{dest}.
This form behaves as if \textit{dest}
issues |\childdocof{|\textit{main}|}| right away,
and no further \textsf{childdoc} directives will be processed.

%%%%%%%%%%%%%%%%%%%%%%%%%%%%%%%%%%%%%%%%
\DescribeMacro{\...prefix}
In the alternative form |\childdocforwardprefix|,
%
\begin{center}
\begin{tabular}{l}
|\input{childdoc.def}|\\
|\childdocforwardprefix[|\textit{main}|]{|\textit{prefix}|}{|\textit{dest}|}|
\end{tabular}
\end{center}
%
the destination file is determined by a pattern
depending on the current file:
To make this work, the current file must be called
`{\textit{prefix}\hspace{0.2em}\textit{suffix}}'
with \textit{prefix} matching precisely the argument.
Processing is then passed on to the file
`{\textit{dest}\hspace{0.2em}\textit{suffix}}'.
Surely, the same effect is achieved by
directly specifying the
argument `{\textit{dest}\hspace{0.2em}\textit{suffix}}'
in the first form.
However, that requires to set up a different file
for each child. With the alternative form of the command
all these files can have exactly the same content
which simplifies setting them up and maintaining them.

For example, the following file |draft.tex|
with a compilation flag |\version| as described in \secref{sec:flags}
compiles the main document as a draft:
%
\begin{center}
\begin{tabular}{l}
|\def\version{draft}|\\
|\input{childdoc.def}|\\
|\childdocforward{|\textit{main}|}|
\end{tabular}
\end{center}
%
Likewise, the following files |final|\textit{nn}|.tex|
compile the final version of the child document
|child|\textit{nn}|.tex|:
%
\begin{center}
\begin{tabular}{l}
|\def\version{final}|\\
|\input{childdoc.def}|\\
|\childdocforwardprefix{final}{child}|
\end{tabular}
\end{center}
%

Note that when several versions of a main file and/or of each child file
are to be generated, it may be convenient to set up a |Makefile| or
shell script to automatise the process.

%%%%%%%%%%%%%%%%%%%%%%%%%%%%%%%%%%%%%%%%%%%%%%%%%%%%%%%%%%%%%%%%%%%%%%%%%%%%%%%%
\subsection{Command Line Processing}
\label{sec:commandline}

The effect of redirection files can also be achieved by invoking
the \LaTeX{} compiler with a more elaborate command line.
Most conveniently this should be done as part
of a shell script or a |Makefile|.

When using \textsf{childdoc} in the main file, the following
command lines effectively perform a redirection
(note that depending on the shell being used,
backslashes may have to be doubled: `|\|' $\to$ `|\\|'):
%
\begin{center}
|... -jobname "|\textit{target}|" |\\|"|[\textit{flags}]%
|\input{childdoc.def}\childdocforward[|\textit{main}|]{|\textit{dest}|}"|
\end{center}
%
Here \textit{target} is the name of the output file,
\textit{main} is the name of the main file
and \textit{dest} is the name of the main or child file to be processed
(all filenames without extensions).
The optional argument \textit{main} can be omitted
if \textit{main} matches \textit{dest}.
Optionally, compilation \textit{flags} can be defined via |\def| commands.
This command line makes the \TeX{} engine believe
it is compiling the file \textit{target}
whose content is specified as the latter parameter.
The provided code then forwards the processing to
\textit{main} or \textit{dest} as described in \secref{sec:forward}.

%%%%%%%%%%%%%%%%%%%%%%%%%%%%%%%%%%%%%%%%%%%%%%%%%%%%%%%%%%%%%%%%%%%%%%%%%%%%%%%%
\subsection{Include by Input}
\label{sec:input}

Including child documents by |\include| has some restrictions by design.
Most notably, the content of a child document always occupies
its own set of pages; pages cannot be shared between child documents.
Usually, this behaviour makes perfect sense
because each child document contain an essential part of the document.
However, in some situations it may be desirable to compose
a document from a collection of parts
without having mandatory page breaks between then.
For this case, the package
provides a mechanism to include parts
by |\input| which can also be processed individually.
However, by construction this mechanism
requires manual handling of the content to be output.

%%%%%%%%%%%%%%%%%%%%%%%%%%%%%%%%%%%%%%%%
\DescribeMacro{\ifchilddocmanual}
The main file should be prepared as usual, see \secref{sec:include}.
However, the document body must make a distinction
between processing of an individual part and of the main document, e.g.:
%
\begin{center}
\begin{tabular}{l}
|\ifchilddocmanual|\\
|\input{\childdocname}|\\
|\||else|\\
\textit{document body with }|\input{|\textit{part}|}|\\
|\||fi|
\end{tabular}
\end{center}
%
The conditional |\ifchilddocmanual| is true whenever
a part to be included by |\input| is being compiled,
and the name of the part is stored in |\childdocname|.

%%%%%%%%%%%%%%%%%%%%%%%%%%%%%%%%%%%%%%%%
\DescribeMacro{\childdocby}
Each part to be included by |\input| should start with:
%
\begin{center}
\begin{tabular}{l}
|\input{childdoc.def}|\\
|\childdocby{|\textit{main}|}|\\
\end{tabular}
\end{center}
%
The directive |\childdocby| is similar to |\childdocof|
described in \secref{sec:include},
but the subsequent selection of content must be done manually.
To that end, both |\ifchilddoc| and |\ifchilddocmanual|
will be true upon processing of a part,
and the name of the part is stored in |\childdocname|.
Note that |\jobname| will be set to the filename of the current part
so that each part receives an individual |.aux| file
that does not interfere with the |.aux| file(s) of the main document.
This behaviour can be altered by the alternative form
|\childdocby[*]{|\textit{main}|}| (with a non-empty optional argument)
which uses the |.aux| file of the main document
by setting |\jobname| to \textit{main}.

%%%%%%%%%%%%%%%%%%%%%%%%%%%%%%%%%%%%%%%%%%%%%%%%%%%%%%%%%%%%%%%%%%%%%%%%%%%%%%%%
\subsection{Driver Development}
\label{sec:driver}

The \textsf{childdoc} mechanism can also be use for the development
of definition files such as \LaTeX{} styles or classes.
This case differs from the above setup with multiple parts
included by |\include| in that no |\includeonly| should be invoked.
This can be achieved by starting the include file
(before |\ProvidesPackage|) with:
%
\begin{center}
\begin{tabular}{l}
|\input{childdoc.def}|\\
|\childdocforward{|\textit{main}|}|\\
\end{tabular}
\end{center}
%
or alternatively with:
%
\begin{center}
\begin{tabular}{l}
|\input{childdoc.def}|\\
|\childdocby{|\textit{main}|}|\\
\end{tabular}
\end{center}
%
Both forms have slightly different effects as described above.
The main file is prepared as usual, see \secref{sec:include}.

%%%%%%%%%%%%%%%%%%%%%%%%%%%%%%%%%%%%%%%%%%%%%%%%%%%%%%%%%%%%%%%%%%%%%%%%%%%%%%%%
\subsection{Legacy Detection}
\label{sec:detection}

The directive |\childdocmain| in the main file can detect
whether the complete document or merely a child is to be compiled
even without using the directive |\childdocof|.
This method is deprecated because it is less robust
and there is no compelling reason to use it;
it is merely provided for backward compatibility
and it may be removed in future versions.

If the detection mechanism is to be used,
it is mandatory to correctly specify
the filename of the main file as the argument of |\childdocmain|:
%
\begin{center}
\begin{tabular}{l}
|\input{childdoc.def}|\\
|\childdocmain{|\textit{main}|}|\\
\end{tabular}
\end{center}
%
If |\jobname| does not match the argument \textit{main} of |\childdocmain|,
it is assumed that |\jobname| points to the child file to be compiled.
When using |\childdocmain| with the main file specified as argument,
it suffices to start a child file
with just |\input{|\textit{main}|}|
without loading of the package and using |\childdocof|.
If instead all processing is done
with the appropriate \textsf{childdoc} directives,
the argument of \textit{main} of |\childdocmain| can be empty.

An alternative version of the command line processing described
in \secref{sec:commandline} using the detection mechanism reads:
%
\begin{center}
|... -jobname "|\textit{target}|" "|[\textit{flags}]%
[|\def\jobname{|\textit{dest}|}|]|\input{|\textit{main}|}"|
\end{center}

%%%%%%%%%%%%%%%%%%%%%%%%%%%%%%%%%%%%%%%%%%%%%%%%%%%%%%%%%%%%%%%%%%%%%%%%%%%%%%%%
\subsection{Manual Code}
\label{sec:manual}

In case one cannot be certain whether the definitions file |childdoc.def|
is installed on the target \TeX{} distribution
and one prefers not to ship it,
it is conceivable to paste a few relevant commands into the sources.

To that end, drop all statements |\input{childdoc.def}|
and perform the replacements as outlined below.
Instead of |\childdocmain{|\textit{main}|}| add the following code
to the top of the main file:
%
\begin{center}
\begin{tabular}{l}
|\||ifdefined\childdocname\endinput\||fi\newif\ifchilddoc|\\
|\edef\childdocname{\scantokens\expandafter{\jobname\noexpand}}|\\
|\def\childdocmain{|\textit{main}|}\||ifx\childdocmain\childdocname\||else|\\
|\childdoctrue\includeonly{\childdocname}\let\jobname\childdocmain\||fi|\\
\end{tabular}
\end{center}
%
Instead of |\childdocof{|\textit{main}|}| just include the main file
at the top of each child file:
%
\begin{center}
|\input{|\textit{main}|}|
\end{center}
%
A simple redirection |\childdocforward{|\textit{dest}|}| is achieved by:
%
\begin{center}
|\def\jobname{|\textit{dest}|}\input{\jobname}|
\end{center}
%
The redirection with prefix
|\childdocforwardprefix[|\textit{prefix}|]{|\textit{dest}|}|
is accomplished by:
%
\begin{center}
\begin{tabular}{l}
|{\edef\jobname{\scantokens\expandafter{\jobname\noexpand}}|\\
|\def\redirectjob |\textit{prefix}|#1~~~{\gdef\jobname{|\textit{dest}|#1}}|\\
|\expandafter\redirectjob\jobname~~~}\input{\jobname}|
\end{tabular}
\end{center}

In an alternative approach,
child documents can be compiled by a specific command line
without additional code or specific definitions:
%
\begin{center}
|... -jobname "|\textit{target}|" "|[\textit{flags}]%
|\includeonly{|\textit{dest}|}\input{|\textit{main}|}"|
\end{center}
%

%%%%%%%%%%%%%%%%%%%%%%%%%%%%%%%%%%%%%%%%%%%%%%%%%%%%%%%%%%%%%%%%%%%%%%%%%%%%%%%%
%%%%%%%%%%%%%%%%%%%%%%%%%%%%%%%%%%%%%%%%%%%%%%%%%%%%%%%%%%%%%%%%%%%%%%%%%%%%%%%%
\section{Information}

%%%%%%%%%%%%%%%%%%%%%%%%%%%%%%%%%%%%%%%%%%%%%%%%%%%%%%%%%%%%%%%%%%%%%%%%%%%%%%%%
\subsection{Copyright}

Copyright \copyright{} 2017--2018 Niklas Beisert

This work may be distributed and/or modified under the
conditions of the \LaTeX{} Project Public License, either version 1.3
of this license or (at your option) any later version.
The latest version of this license is in
  \url{http://www.latex-project.org/lppl.txt}
and version 1.3 or later is part of all distributions of \LaTeX{}
version 2005/12/01 or later.

This work has the LPPL maintenance status `maintained'.

The Current Maintainer of this work is Niklas Beisert.

This work consists of the files |README.txt|, |childdoc.ins| and |childdoc.dtx|
as well as the derived files |childdoc.def|, |cdocsamp.tex|
with |cdocsch1.tex|, |cdocsch2.tex|, |cdocspt3.tex|, |cdocspt4.tex|,
|cdocsdrf.tex|, |cdocsfn1.tex|, |cdocsfn2.tex|
as well as |childdoc.pdf|.

%%%%%%%%%%%%%%%%%%%%%%%%%%%%%%%%%%%%%%%%%%%%%%%%%%%%%%%%%%%%%%%%%%%%%%%%%%%%%%%%
\subsection{Files and Installation}

The package consists of the files:
%
\begin{center}
\begin{tabular}{ll}
    |README.txt|   & readme file \\
    |childdoc.ins| & installation file \\
    |childdoc.dtx| & source file \\
    |childdoc.def| & definition file \\
    |cdocsamp.tex| & sample main file \\
    |cdocsch1.tex| & sample include file \\
    |cdocsch2.tex| & sample include file \\
    |cdocspt3.tex| & sample part file \\
    |cdocspt4.tex| & sample part file \\
    |cdocsdrf.tex| & sample redirection file \\
    |cdocsfn1.tex| & sample redirection file \\
    |cdocsfn2.tex| & sample redirection file \\
    |childdoc.pdf| & manual
\end{tabular}
\end{center}
%
The distribution consists of the files
|README.txt|, |childdoc.ins| and |childdoc.dtx|.
%
\begin{itemize}
\item
Run (pdf)\LaTeX{} on |childdoc.dtx|
to compile the manual |childdoc.pdf| (this file).
\item
Run \LaTeX{} on |childdoc.ins| to create the definitions file |childdoc.def|
and the sample |cdocsamp.tex| with include files
|cdocsch1.tex|, |cdocsch2.tex|, |cdocspt3.tex|, |cdocspt4.tex|,
|cdocsdrf.tex|, |cdocsfn1.tex|, |cdocsfn2.tex|.
Then copy the file |childdoc.def| to an appropriate directory of your \LaTeX{}
distribution, e.g.\ \textit{texmf-root}|/tex/latex/childdoc|.
\end{itemize}

%%%%%%%%%%%%%%%%%%%%%%%%%%%%%%%%%%%%%%%%%%%%%%%%%%%%%%%%%%%%%%%%%%%%%%%%%%%%%%%%
\subsection{Related CTAN Packages}

There are several other packages which offer a similar functionality:
%
\begin{itemize}
\item
The packages
\href{http://ctan.org/pkg/docmute}{\textsf{docmute}},
\href{http://ctan.org/pkg/includex}{\textsf{includex}} and
\href{http://ctan.org/pkg/standalone}{\textsf{standalone}}
provide commands to include only the document body of
a child file thus allowing both files to be compiled individually.
\item
The packages \href{http://ctan.org/pkg/subdocs}{\textsf{subdocs}}
and \href{http://ctan.org/pkg/subfiles}{\textsf{subfiles}}
provide structures in which the main and child documents can be
encapsulated and allowing them to be compiled individually.
The inclusion mechanism is different from the conventional |\include|.
\item
The package \href{http://ctan.org/pkg/combine}{\textsf{combine}}
is an elaborate solution to combine several documents into one.
\end{itemize}
%
See also the CTAN topic \href{http://ctan.org/topic/subdocs}{\textsf{subdocs}}
for further related packages.
The present package differs from the above solutions in that
a document structure constructed with the conventional |\include| mechanism
just needs two extra commands at the top of every file
such that all constituent files can be compiled individually.

%%%%%%%%%%%%%%%%%%%%%%%%%%%%%%%%%%%%%%%%%%%%%%%%%%%%%%%%%%%%%%%%%%%%%%%%%%%%%%%%
%\subsection{Feature Suggestions}
%
%The following is a list of features which may be useful for future
%versions of this package:
%%
%\begin{itemize}
%\item
%\ldots
%\end{itemize}

%%%%%%%%%%%%%%%%%%%%%%%%%%%%%%%%%%%%%%%%%%%%%%%%%%%%%%%%%%%%%%%%%%%%%%%%%%%%%%%%
\subsection{Revision History}

%%%%%%%%%%%%%%%%%%%%%%%%%%%%%%%%%%%%%%%%
\paragraph{v2.0:} 2018/12/30

\begin{itemize}
\item
immediate forward processing
\item
added |\childdocby| mechanism
\item
manual restructured
\end{itemize}

%%%%%%%%%%%%%%%%%%%%%%%%%%%%%%%%%%%%%%%%
\paragraph{v1.6:} 2018/01/17

\begin{itemize}
\item
application for development of include files
\item
corrections to manual
\end{itemize}

%%%%%%%%%%%%%%%%%%%%%%%%%%%%%%%%%%%%%%%%
\paragraph{v1.5:} 2017/05/21

\begin{itemize}
\item
more complete structuring introduced
\item
|\childdocof| introduced
\item
|\childdoc| renamed to |\childdocmain|
\item
|\childredirect| renamed to |\childdocforward| and |\childdocforwardprefix|
and functionality expanded
\end{itemize}

%%%%%%%%%%%%%%%%%%%%%%%%%%%%%%%%%%%%%%%%
\paragraph{v1.0:} 2017/04/27

\begin{itemize}
\item
manual and install package
\item
first version published on CTAN
\end{itemize}

%%%%%%%%%%%%%%%%%%%%%%%%%%%%%%%%%%%%%%%%
\paragraph{v0.6:} 2017/04/26

\begin{itemize}
\item
redirection mechanism added
\end{itemize}

%%%%%%%%%%%%%%%%%%%%%%%%%%%%%%%%%%%%%%%%
\paragraph{v0.5:} 2017/04/26

\begin{itemize}
\item
functionality in definition file
\end{itemize}


%%%%%%%%%%%%%%%%%%%%%%%%%%%%%%%%%%%%%%%%%%%%%%%%%%%%%%%%%%%%%%%%%%%%%%%%%%%%%%%%
%%%%%%%%%%%%%%%%%%%%%%%%%%%%%%%%%%%%%%%%%%%%%%%%%%%%%%%%%%%%%%%%%%%%%%%%%%%%%%%%
%%%%%%%%%%%%%%%%%%%%%%%%%%%%%%%%%%%%%%%%%%%%%%%%%%%%%%%%%%%%%%%%%%%%%%%%%%%%%%%%
\appendix

\settowidth\MacroIndent{\rmfamily\scriptsize 000\ }

 \DocInput{childdoc.dtx}

\end{document}
%</driver>
% \fi
%
% %%%%%%%%%%%%%%%%%%%%%%%%%%%%%%%%%%%%%%%%%%%%%%%%%%%%%%%%%%%%%%%%%%%%%%%%%%%%%%
% %%%%%%%%%%%%%%%%%%%%%%%%%%%%%%%%%%%%%%%%%%%%%%%%%%%%%%%%%%%%%%%%%%%%%%%%%%%%%%
% \section{Sample}
%\iffalse
%<*samplemain>
%\fi
%
% The following presents a sample document
% with two chapters, two parts, a title page,
% a compile flag as well as three forwarding files to set the flag.
% It consists of eight |.tex| files:
% \begin{center}
% \begin{tabular}{ll}
% |cdocsamp.tex|&main file\\
% |cdocsch1.tex|&include file for chapter 1\\
% |cdocsch2.tex|&include file for chapter 2\\
% |cdocspt3.tex|&include file for part 3\\
% |cdocspt4.tex|&include file for part 4\\
% |cdocsdrf.tex|&forwarding file for main file in draft mode\\
% |cdocsfi1.tex|&forwarding file for final version of chapter 1\\
% |cdocsfi2.tex|&forwarding file for final version of chapter 2\\
% \end{tabular}
% \end{center}
% Each of the eight files can be compiled directly by the \LaTeX{} compiler.
%
% %%%%%%%%%%%%%%%%%%%%%%%%%%%%%%%%%%%%%%
% \paragraph{Main File.}
%
% The main file is called |cdocsamp.tex|.
%
% Load the \textsf{childdoc} definitions and
% declare the filename for the main document:
%    \begin{macrocode}
\input{childdoc.def}
\childdocmain{}
%    \end{macrocode}

% Optional override for |\version| flag:
%    \begin{macrocode}
%%\ifchilddoc\else\providecommand{\version}{draft}\fi
%    \end{macrocode}

% Define the default values for the |\version| flag
% (|final| for the main file and |draft| for childs):
%    \begin{macrocode}
\ifchilddoc
\providecommand{\version}{draft}
\else
\providecommand{\version}{final}
\fi
%    \end{macrocode}

% Load the standard document class:
%    \begin{macrocode}
\documentclass[12pt]{article}
%    \end{macrocode}

% Start the document body:
%    \begin{macrocode}
\begin{document}
%    \end{macrocode}

% Declare a title page.
% Print title, part of document being processed and version flag:
%    \begin{macrocode}
\addtocounter{page}{-1}
\begin{center}
{\LARGE\bfseries{}childdoc example\par}
\vspace{1cm}
\ifchilddoc
\ifchilddocmanual part\else chapter\fi:
`\childdocname' of `\childdocjob'\par
\else
main document: `\childdocjob'\par
\fi
version: \version\par
\end{center}
\newpage
%    \end{macrocode}

% Manually include selected file,
% otherwise process as usual:
%    \begin{macrocode}
\ifchilddocmanual
\section*{part `\childdocname'}
\input{\childdocname}
\else
%    \end{macrocode}

% Include the two chapters:
%    \begin{macrocode}
\include{cdocsch1}
\include{cdocsch2}
%    \end{macrocode}

% Include the two parts unless only chapters should be displayed:
%    \begin{macrocode}
\ifchilddoc\else
\section{part three}
\input{cdocspt3}
\section{part four}
\input{cdocspt4}
\fi
%    \end{macrocode}

% Process as usual until here:
%    \begin{macrocode}
\fi
%    \end{macrocode}

% End of document body:
%    \begin{macrocode}
\end{document}
%    \end{macrocode}
%\iffalse
%</samplemain>
%\fi
%
% %%%%%%%%%%%%%%%%%%%%%%%%%%%%%%%%%%%%%%
% \paragraph{Chapter Include Files.}
%
% The include files are called |cdocsch1.tex| and |cdocsch2.tex|.
%
%\iffalse
%<*samplechap1|samplechap2>
%\fi

% Optional override for |\version| flag:
%    \begin{macrocode}
%%\providecommand{\version}{final}
%    \end{macrocode}

% Include the main document:
%    \begin{macrocode}
\input{childdoc.def}
\childdocof{cdocsamp}
%    \end{macrocode}

%\iffalse
%</samplechap1|samplechap2>
%\fi
%
%\iffalse
%<*samplechap1>
%\fi
% Some text for chapter 1:
%    \begin{macrocode}
\section{one}
some text in chapter one
%    \end{macrocode}

%\iffalse
%</samplechap1>
%\fi
% Some text for chapter 2:
%\iffalse
%<*samplechap2>
%\fi
%    \begin{macrocode}
\section{two}
more text in chapter two
%    \end{macrocode}

%\iffalse
%</samplechap2>
%\fi
%
% %%%%%%%%%%%%%%%%%%%%%%%%%%%%%%%%%%%%%%
% \paragraph{Part Include Files.}
%
% The include files are called |cdocspt3.tex| and |cdocspt4.tex|.
%
%\iffalse
%<*samplepart3|samplepart4>
%\fi

% Optional override for |\version| flag:
%    \begin{macrocode}
%%\providecommand{\version}{final}
%    \end{macrocode}

% Include the main document:
%    \begin{macrocode}
\input{childdoc.def}
\childdocby{cdocsamp}
%    \end{macrocode}

%\iffalse
%</samplepart3|samplepart4>
%\fi
%
%\iffalse
%<*samplepart3>
%\fi
% Some text for part 3:
%    \begin{macrocode}
some text in part three
%    \end{macrocode}

%\iffalse
%</samplepart3>
%\fi
% Some text for part 4:
%\iffalse
%<*samplepart4>
%\fi
%    \begin{macrocode}
more text in part four
%    \end{macrocode}

%\iffalse
%</samplepart4>
%\fi
%
% %%%%%%%%%%%%%%%%%%%%%%%%%%%%%%%%%%%%%%
% \paragraph{Forwarding for a Complete Draft.}
%
% The following forwarding file |cdocsdrf.tex|
% compiles the main document in draft mode:
%\iffalse
%<*sampledraft>
%\fi
%    \begin{macrocode}
\def\version{draft}
\input{childdoc.def}
\childdocforward{cdocsamp}
%    \end{macrocode}

%\iffalse
%</sampledraft>
%\fi
%
% %%%%%%%%%%%%%%%%%%%%%%%%%%%%%%%%%%%%%%
% \paragraph{Forwarding for Final Version of the Chapters.}
%
% The following forwarding files |cdocsfn1.tex| and |cdocsfn2.tex|
% (with identical content)
% compile the final versions of the child documents
% |cdocsch1.tex| and |cdocsch2.tex|, respectively:
%\iffalse
%<*samplefinal>
%\fi
%    \begin{macrocode}
\def\version{final}
\input{childdoc.def}
\childdocforwardprefix[cdocsamp]{cdocsfn}{cdocsch}
%    \end{macrocode}

%\iffalse
%</samplefinal>
%\fi
%
% %%%%%%%%%%%%%%%%%%%%%%%%%%%%%%%%%%%%%%
% \paragraph{Command Line Processing.}
%
% The following three command lines generate the output files
% |cdocscld|, |cdocscl1| and |cdocscl2|
% which should be identical to
% |cdocsdrf|, |cdocsch1| and |cdocsfn2|, respectively:
% \begin{center}
% \begin{tabular}{l}
% |latex -jobname cdocscld \|\\
% |  "\def\version{draft}\input{childdoc.def}\childdocforward{cdocsamp}"|\\
% |latex -jobname cdocscl1 \|\\
% |  "\input{childdoc.def}\childdocforward[cdocsamp]{cdocsch1}"|\\
% |latex -jobname cdocscl2 \|\\
% |  "\def\version{final}\input{childdoc.def}\childdocforward{cdocsch2}"|
% \end{tabular}
% \end{center}
% Note that the trailing backslash on each first line
% merely continues the input to the second line
% (for convenient cut ant paste).
% Furthermore, the command |latex| can be replaced by any
% of its alternative versions such as |pdflatex|.
%
% %%%%%%%%%%%%%%%%%%%%%%%%%%%%%%%%%%%%%%%%%%%%%%%%%%%%%%%%%%%%%%%%%%%%%%%%%%%%%%
% %%%%%%%%%%%%%%%%%%%%%%%%%%%%%%%%%%%%%%%%%%%%%%%%%%%%%%%%%%%%%%%%%%%%%%%%%%%%%%
% \section{Implementation}
%\iffalse
%<*package>
%\fi
%
% This section describes the definitions file |childdoc.def|.

% The definitions cannot be loaded using |\usepackage| or |\RequirePackage|
% which has a mechanism to prevent loading a style file more than once.
% When loading the definitions by means of |\input|
% multiple instances have to be prevented manually:
%\iffalse
%This code needs to be before the `\ProvidesFile' directive
%which is defined at the beginning of this file.
%Therefore it is also placed there and commented out here.
%</package>
%<*discard>
%\fi
%    \begin{macrocode}
\ifdefined\childdocmain\endinput\fi
%    \end{macrocode}
%\iffalse
%</discard>
%<*package>
%\fi
%
% \macro{\ifchilddoc}
% \macro{\ifchilddocmanual}
% The conditional |\ifchilddoc| tells whether a
% child (true) or main (false) document is being compiled.
% The conditional |\ifchilddocmanual| tells whether
% the |\includeonly| mechanism is used (false) or
% the selection of child files must be performed manually (true).
% The definitions initialise to false:
%    \begin{macrocode}
\newif\ifchilddoc
\newif\ifchilddocmanual
%    \end{macrocode}

% \macro{\childdocname}
% \macro{\childdocjob}
% The macro |\childdocname| stores the name of the main document
% to be compiled. The macro |\childdocjob| stores the name of
% the document on which the \LaTeX{} compiler was originally invoked.
% The content of |\jobname| cannot be compared
% to filenames specified in the source due to different catcodes.
% The following code rescans |\jobname|, stores the result
% in |\childdocname| and saves a copy in |\childdocjob|:
%    \begin{macrocode}
\edef\childdocname{\scantokens\expandafter{\jobname\noexpand}}
\let\childdocjob\childdocname
%    \end{macrocode}

% \macro{\childdocdisable}
% The macro |\childdocdisable| prevents the main file
% from being processed more than once.
% At this stage, the main document command |\childdocmain|
% is assumed to be called once again where it should do nothing.
% Any subsequent call to it should prevent
% a secondary processing of the main document
% It overwrites the forwarding commands
% |\childdocof| and |\childdocforward|
% with empty macros to prevent further inclusions of the main document:
%    \begin{macrocode}
\newcommand{\childdocdisable}
{
  \renewcommand{\childdocmain}[1]{\renewcommand{\childdocmain}[1]{\endinput}}
  \renewcommand{\childdocof}[1]{}
  \renewcommand{\childdocby}[2][]{}
  \renewcommand{\childdocforward}[2][]{}
  \renewcommand{\childdocdisable}{}
}
%    \end{macrocode}

% \macro{\childdocmain}
% The macro |\childdocmain| is to be called at the top of the main file
% with nothing or the main filename (without extension) as argument.
% First, it breaks loops.
% If the argument is not empty and does not match |\childdocname|
% (which is set by the first inclusion of |childdoc.def|),
% |\ifchilddoc| is set to true, |\includeonly| is applied to the child file
% and |\jobname| is set to the main file
% (for proper handling of |.aux| files):
%    \begin{macrocode}
\newcommand{\childdocmain}[1]
{
  \childdocdisable\childdocmain{}
  \if?#1?\else
    \begingroup
      \def\childdoctmp{#1}
      \ifx\childdoctmp\childdocname
        \def\childdoctmp{}
      \else
        \def\childdoctmp
        {
          \childdoctrue
          \includeonly{\childdocname}
          \def\childdocjob{#1}
          \def\jobname{#1}
        }
      \fi
      \expandafter
    \endgroup
    \childdoctmp
  \fi
}
%    \end{macrocode}

% \macro{\childdocof}
% The command |\childdocof| redirects
% compilation to the main file |#1|.
%    \begin{macrocode}
\newcommand{\childdocof}[1]
{
  \childdocdisable
  \childdoctrue
  \includeonly{\childdocname}
  \def\jobname{#1}
  \def\childdocjob{#1}
  \input{#1}
}
%    \end{macrocode}

% \macro{\childdocby}
% The command |\childdocby| ....
%    \begin{macrocode}
\newcommand{\childdocby}[2][]
{
  \childdocdisable
  \childdoctrue
  \childdocmanualtrue
  \if?#1?\else
    \def\jobname{#2}
  \fi
  \def\childdocjob{#2}
  \input{#2}
  \endinput
}
%    \end{macrocode}

% \macro{\childdocforward}
% The command |\childdocforward| redirects
% compilation to the main file or
% (if the optional argument is given) a child file.
% Parameters are set as if the main file
% or a child file starting with |\childdocof| was compiled.
% Then compilation is handed over to the main file:
%    \begin{macrocode}
\newcommand{\childdocforward}[2][]
{
  \begingroup
    \if?#1?
      \def\childdoctmp
      {
        \def\childdocname{#2}
        \def\childdocjob{#2}
        \def\jobname{#2}
        \input{#2}
        \endinput
      }
    \else
      \def\childdoctmp
      {
        \childdocdisable
        \def\childdocname{#2}
        \childdoctrue
        \includeonly{#2}
        \def\childdocjob{#1}
        \def\jobname{#1}
        \input{#1}
        \endinput
      }
    \fi
    \expandafter
  \endgroup
  \childdoctmp
}
%    \end{macrocode}

% \macro{\childdocforwardprefix}
% The command |\childdocforwardprefix| redirects
% compilation to the main or a child file by means of a pattern.
% The prefix |#1| in the current filename is replaced by |#2|
% and the suffix of the current filename is kept
% (it is assumed that the filename does not contain the substring `|~~~|'
% which is used as a delimiter).
% Compilation is handed over to the new file by |\childdocforward|:
%    \begin{macrocode}
\newcommand{\childdocforwardprefix}[3][]
{
  \begingroup
    \def\childdocextract #2##1~~~{\def\childdoctmp{\childdocforward[#1]{#3##1}}}
    \expandafter\childdocextract\childdocname~~~
    \expandafter
  \endgroup
  \childdoctmp
}
%    \end{macrocode}

% \macro{\childdoc}
% The deprecated macro |\childdoc| is a legacy version of |\childdocmain|:
%    \begin{macrocode}
\newcommand{\childdoc}{\childdocmain}
%    \end{macrocode}

% \macro{\childdocredirect}
% The deprecated macro |\childdocredirect| is a legacy version
% of |\childdocforward| and |\childdocforwardprefix|:
%    \begin{macrocode}
\newcommand{\childdocredirect}[2][]
{
  \begingroup
    \if?#1?
      \def\childdoctmp{\childdocforward{#2}}
    \else
      \def\childdoctmp{\childdocforwardprefix{#1}{#2}}
    \fi
    \expandafter
  \endgroup
  \childdoctmp
}
%    \end{macrocode}

%\iffalse
%</package>
%\fi
%
\endinput

\childdocforwardprefix[cdocsamp]{cdocsfn}{cdocsch}
%    \end{macrocode}

%\iffalse
%</samplefinal>
%\fi
%
% %%%%%%%%%%%%%%%%%%%%%%%%%%%%%%%%%%%%%%
% \paragraph{Command Line Processing.}
%
% The following three command lines generate the output files
% |cdocscld|, |cdocscl1| and |cdocscl2|
% which should be identical to
% |cdocsdrf|, |cdocsch1| and |cdocsfn2|, respectively:
% \begin{center}
% \begin{tabular}{l}
% |latex -jobname cdocscld \|\\
% |  "\def\version{draft}% \iffalse
%
% childdoc.dtx Copyright (C) 2017-2018 Niklas Beisert
%
% This work may be distributed and/or modified under the
% conditions of the LaTeX Project Public License, either version 1.3
% of this license or (at your option) any later version.
% The latest version of this license is in
%   http://www.latex-project.org/lppl.txt
% and version 1.3 or later is part of all distributions of LaTeX
% version 2005/12/01 or later.
%
% This work has the LPPL maintenance status `maintained'.
%
% The Current Maintainer of this work is Niklas Beisert.
%
% This work consists of the files childdoc.dtx and childdoc.ins
% and the derived files childdoc.def and cdocsamp.tex with
% cdocsch1.tex, cdocsch2.tex, cdocsdrf.tex, cdocsfn1.tex, cdocsfn2.tex.
%
%<package>\ifdefined\childdocmain\endinput\fi
%<package>\ProvidesFile{childdoc.def}[2018/12/30 v2.0 child document driver]
%<samplemain>\ProvidesFile{cdocsamp.tex}[2018/12/30 v2.0 sample for childdoc]
%<*driver>
%\ProvidesFile{childdoc.drv}[2018/12/30 v2.0 childdoc reference manual file]
\PassOptionsToClass{10pt,a4paper}{article}
\documentclass{ltxdoc}

\usepackage[margin=35mm]{geometry}
\usepackage{hyperref}
\usepackage{hyperxmp}
\usepackage[usenames]{color}

\hypersetup{colorlinks=true}
\hypersetup{pdfstartview=FitH}
\hypersetup{pdfpagemode=UseNone}
\hypersetup{pdfsource={}}
\hypersetup{pdflang={en-UK}}
\hypersetup{pdfcopyright={Copyright 2017-2018 Niklas Beisert.
  This work may be distributed and/or modified under the
  conditions of the LaTeX Project Public License, either version 1.3
  of this license or (at your option) any later version.}}
\hypersetup{pdflicenseurl={http://www.latex-project.org/lppl.txt}}
\hypersetup{pdfcontactaddress={ETH Zurich, ITP, HIT K,
  Wolfgang-Pauli-Strasse 27}}
\hypersetup{pdfcontactpostcode={8093}}
\hypersetup{pdfcontactcity={Zurich}}
\hypersetup{pdfcontactcountry={Switzerland}}
\hypersetup{pdfcontactemail={nbeisert@itp.phys.ethz.ch}}
\hypersetup{pdfcontacturl={http://people.phys.ethz.ch/\xmptilde nbeisert/}}

\newcommand{\secref}[1]{\hyperref[#1]{section \ref*{#1}}}

\parskip1ex
\parindent0pt
\let\olditemize\itemize
\def\itemize{\olditemize\parskip0pt}

\begin{document}

\title{The \textsf{childdoc} Package}
\hypersetup{pdftitle={The childdoc Package}}
\author{Niklas Beisert\\[2ex]
  Institut f\"ur Theoretische Physik\\
  Eidgen\"ossische Technische Hochschule Z\"urich\\
  Wolfgang-Pauli-Strasse 27, 8093 Z\"urich, Switzerland\\[1ex]
  \href{mailto:nbeisert@itp.phys.ethz.ch}
  {\texttt{nbeisert@itp.phys.ethz.ch}}}
\hypersetup{pdfauthor={Niklas Beisert}}
\hypersetup{pdfsubject={Manual for the LaTeX2e Package childdoc}}
\date{30 December 2018, \textsf{v2.0}}
\maketitle

\begin{abstract}\noindent
\textsf{childdoc} is a \LaTeXe{} package
that enables the direct compilation
of document sections included by |\include|
to individual files.
\end{abstract}

\begingroup
\parskip0ex
\tableofcontents
\endgroup

%%%%%%%%%%%%%%%%%%%%%%%%%%%%%%%%%%%%%%%%%%%%%%%%%%%%%%%%%%%%%%%%%%%%%%%%%%%%%%%%
%%%%%%%%%%%%%%%%%%%%%%%%%%%%%%%%%%%%%%%%%%%%%%%%%%%%%%%%%%%%%%%%%%%%%%%%%%%%%%%%
\section{Introduction}

\LaTeX{} provides a mechanism to structure a large document (such as a book)
into a main file and several child files (containing the chapters)
using the |\include| command.
This mechanism is beneficial for documents
which span hundreds of pages in order to
make the source file(s) more manageable.
Moreover, compilation can be restricted to
selected child files by means of the |\includeonly| command.
The latter feature can be used to reduce the compilation time while editing
(this was significantly more useful in the earlier days of \LaTeX{})
or to generate a smaller document which is easier to navigate.
Another application of |\includeonly| is to generate
documents consisting of selected parts of the complete document.

However, there are a few drawbacks of the plain |\include| mechanism:
\begin{itemize}
\item
The child files cannot be compiled on their own,
they can only be compiled via the main file.
A naive editing environment
(such as a text editor with an option
to have the current file processed by \LaTeX)
may require one to switch to the main file before compiling;
attempting to compile the child file produces errors.
\item
The main file must be modified (each time)
to adjust the |\includeonly| command
to the present needs. This easily leaves the main file in a messy state.
\item
The generated document will always carry the filename
of the main document. This is inconvenient if
several child files are to be compiled and
to be kept for distribution.
\end{itemize}

The present package provides a simple interface
to make child files individually compilable by \LaTeX{}.
Compiling a child file then has the same effect as compiling
the main file with an |\includeonly| command
to select the appropriate child.
Moreover the generated document will carry the name of the child
rather than the main file.
This resolves all three above issues.

This feature is meant to make the editing of books,
thesis documents and lecture notes somewhat more convenient.
However, the package can also be used efficiently for
composing a series of documents (such as exercise sheets)
which are typically distributed individually.
It then assists the author in generating the individual documents
(potentially in different versions)
as well as a document containing the collected series.
Another application is in developing style files
or other kinds of included material
where compilation of the style file could redirect
to a sample or test file.

%%%%%%%%%%%%%%%%%%%%%%%%%%%%%%%%%%%%%%%%%%%%%%%%%%%%%%%%%%%%%%%%%%%%%%%%%%%%%%%%
%%%%%%%%%%%%%%%%%%%%%%%%%%%%%%%%%%%%%%%%%%%%%%%%%%%%%%%%%%%%%%%%%%%%%%%%%%%%%%%%
\section{Usage}

First of all, the package \textsf{childdoc} is \emph{not} a standard
\LaTeXe{} |.sty| style file! Therefore it needs to be invoked in
a non-standard way.

%%%%%%%%%%%%%%%%%%%%%%%%%%%%%%%%%%%%%%%%%%%%%%%%%%%%%%%%%%%%%%%%%%%%%%%%%%%%%%%%
\subsection{Included Files}
\label{sec:include}

%%%%%%%%%%%%%%%%%%%%%%%%%%%%%%%%%%%%%%%%
\DescribeMacro{\childdocmain}
To use the package, add the commands
\begin{center}
\begin{tabular}{l}
|\input{childdoc.def}|\\
|\childdocmain{}|\\
\end{tabular}
\end{center}
at the very top of the main \LaTeX{} file,
in particular \emph{before} the |\documentclass| statement!
The argument of |\childdocmain| should be left empty
(but it must be present).

%%%%%%%%%%%%%%%%%%%%%%%%%%%%%%%%%%%%%%%%
\DescribeMacro{\childdocof}
Furthermore, add the commands
\begin{center}
\begin{tabular}{l}
|\input{childdoc.def}|\\
|\childdocof{|\textit{main}|}|\\
\end{tabular}
\end{center}
at the top of every child file \textit{child}
which is included by |\include{|\textit{child}|}|
from within the main file
(or at least for those files to be compiled individually).
The argument \textit{main} must be the filename of the main file.

There are a couple of
considerations in setting up the main and child documents:

%%%%%%%%%%%%%%%%%%%%%%%%%%%%%%%%%%%%%%%%
\paragraph{Restrictions.}

Please note the following restrictions:
\begin{itemize}
\item
|\childdocmain| must be called with one argument \textit{main}
to ensure compatibility with earlier version of the package.
It must either be empty (|\childdocmain{}|)
or precisely match the filename of the main file in which it is specified.
See \secref{sec:detection} for further information.
\item
The filename \textit{main} must be specified without the |.tex| extension.
\item
The filename \textit{main} is case sensitive
(even in case-insensitive file systems)
due to internal string comparison.
\item
The argument \textit{main} should be fully expanded, it cannot be a macro.
\item
Subdirectories and special characters should be avoided in filenames.
\item
The command |\childdocmain{|\textit{main}|}| must be followed by a whitespace.
It should not be followed immediately by another command
or by a comment mark `|%|'.
This is because the \TeX{} parser reads the token immediately following
the argument of |\childdocmain| and puts it
at the beginning of every child section;
however, a white\-space is ignored.
\end{itemize}

%%%%%%%%%%%%%%%%%%%%%%%%%%%%%%%%%%%%%%%%
\paragraph{Content of Main File.}

It is advisable to place all content in the child files included by |\include|.
Any output contained in the main file will appear in all child documents
unless suppressed manually;
it cannot be suppressed automatically by the |\includeonly| directive
and thus should normally be avoided.
A method to include some content in the main file
by means of conditional processing is described in \secref{sec:conditional}.

%%%%%%%%%%%%%%%%%%%%%%%%%%%%%%%%%%%%%%%%
\paragraph{Page Numbering.}

When only a part of the document is compiled,
the appropriate numbering of pages
(as well as other status parameters)
is determined from the |.aux| files.
The latter contain information from previous passes.
However this information needs to propagate through
all intermediate child documents.
Therefore the page numbering in child documents may well
be inconsistent until the complete document is compiled at least once.

A useful (if unconventional) way to always ensure a consistent
page numbering is to restart the numbering in each child document
and denote the pages by `\textit{child}|.|\textit{page}'
where \textit{child} represents the chapter/section number of the child file.
This can be achieved by the command
|\numberwithin{page}{|\textit{child}|}|
of the \textsf{amsmath} package
where \textit{child} can be |chapter| or |section|
depending on the chosen structuring.
Alternatively, one can modify the macro |\thepage| appropriately
and reset the counter |page| at the start of each child file.

%%%%%%%%%%%%%%%%%%%%%%%%%%%%%%%%%%%%%%%%%%%%%%%%%%%%%%%%%%%%%%%%%%%%%%%%%%%%%%%%
\subsection{Conditional Processing}
\label{sec:conditional}

The package provides a mechanism to compile different versions
of a document. To customise the versions further some conditional processing
can come in handy to distinguish which version is being compiled.
The package provides two macros to describe the compilation context:

%%%%%%%%%%%%%%%%%%%%%%%%%%%%%%%%%%%%%%%%
\DescribeMacro{\ifchilddoc}
The conditional |\ifchilddoc| distinguishes between the compilation of
child documents and the main document:
%
\begin{center}
|\ifchilddoc |\textit{child-code}| |[|\||else |\textit{main-code}]| \||fi|
\end{center}

%%%%%%%%%%%%%%%%%%%%%%%%%%%%%%%%%%%%%%%%
\DescribeMacro{\childdocname}
\DescribeMacro{\childdocjob}
The macro |\childdocname| contains the filename (without extension)
of the main or child file being processed.
Note that |\childdocjob| will always contain the name of the main file.

%%%%%%%%%%%%%%%%%%%%%%%%%%%%%%%%%%%%%%%%
\paragraph{Title Page.}

Conditional processing can be used to include a title or banner page
in the main document when proper precautions are taken.
Importantly, the code in the main file should ensure that the page counter
(as well as other status parameters which are stored in the |.aux| files)
takes the same value after the conditional processing.
Otherwise the page numbers may take divergent values
depending on which part is compiled.

For example, a title page could be declared by:
%
\begin{center}
\begin{tabular}{l}
|\ifchilddoc\||else|\\
|\addtocounter{page}{-1}|\\
\textit{code for title page}\\
|\newpage|\\
|\||fi|
\end{tabular}
\end{center}
%
A banner page for the child documents can be generated by:
%
\begin{center}
\begin{tabular}{l}
|\ifchilddoc|\\
|\addtocounter{page}{-1}|\\
\textit{code for banner page}\\
|\newpage|\\
|\||fi|
\end{tabular}
\end{center}
%
Here one could write a message such as:
\begin{center}
|This is the part \childdocname{} of \childdocjob{}.|
\end{center}

%%%%%%%%%%%%%%%%%%%%%%%%%%%%%%%%%%%%%%%%%%%%%%%%%%%%%%%%%%%%%%%%%%%%%%%%%%%%%%%%
\subsection{Flags}
\label{sec:flags}

The package makes it easy to generate different versions
of the main or child documents.
To this end compilation flags can be defined
and assigned different default values.
They will be particularly useful in conjunction
with the forwarding mechanism described in \secref{sec:forward}.

For example, it may be useful to have a flag |\version|
which can be set to |draft| or |final|.
The document source will contain some conditional code
depending on the value of |\version|.
Suppose further, the flag should default to |final| for the main file
and to |draft| for child files
which is a natural assignment for editing the document.
This is achieved by placing the following code
in the preamble of the main document
(below the |\childdocmain| directive):
%
\begin{center}
\begin{tabular}{l}
|\ifchilddoc|\\
|\providecommand{\version}{draft}|\\
|\||else|\\
|\providecommand{\version}{final}|\\
|\||fi|
\end{tabular}
\end{center}
%
The definition by |\providecommand| makes sure
that previous definitions are not overwritten.
Further statements |\providecommand{\version}{...}|
can thus be added before the above code to override it.

For the main file, one might add a line
(between |\childdocmain| and the above block)
%
\begin{center}
|%\ifchilddoc\||else\providecommand{\version}{draft}\||fi|
\end{center}
%
which can be uncommented to produce a draft version.
Likewise one can add a line to the very top of a child file
(above the |\childdocof{|\textit{main}|}| directive)
%
\begin{center}
|%\providecommand{\version}{final}|
\end{center}
%
which can be uncommented to produce the final version of this child document.

%%%%%%%%%%%%%%%%%%%%%%%%%%%%%%%%%%%%%%%%%%%%%%%%%%%%%%%%%%%%%%%%%%%%%%%%%%%%%%%%
\subsection{Forwarding}
\label{sec:forward}

Different versions of the main or child documents
using compilation flags as described in \secref{sec:flags}
can be (permanently) stored in different files
for convenient compilation, viewing and distribution.
To this end, the package defines a command
to pass on compilation to a different file:

%%%%%%%%%%%%%%%%%%%%%%%%%%%%%%%%%%%%%%%%
\DescribeMacro{\childdocforward}
The command |\childdocforward| redirects processing to
another source file:
%
\begin{center}
\begin{tabular}{l}
|\input{childdoc.def}|\\
|\childdocforward[|\textit{main}|]{|\textit{dest}|}|\\
\end{tabular}
\end{center}
%
The argument \textit{dest} is the destination file
(without extension).
It should be the main file or one of the child files.
Note that further \textsf{childdoc} directives
such as |\childdocof| and |\childdocforward|
in the indicated file will be processed in this form.
The optional argument \textit{main}
passes on directly to the main file \textit{main}
while pretending to compile the child \textit{dest}.
This form behaves as if \textit{dest}
issues |\childdocof{|\textit{main}|}| right away,
and no further \textsf{childdoc} directives will be processed.

%%%%%%%%%%%%%%%%%%%%%%%%%%%%%%%%%%%%%%%%
\DescribeMacro{\...prefix}
In the alternative form |\childdocforwardprefix|,
%
\begin{center}
\begin{tabular}{l}
|\input{childdoc.def}|\\
|\childdocforwardprefix[|\textit{main}|]{|\textit{prefix}|}{|\textit{dest}|}|
\end{tabular}
\end{center}
%
the destination file is determined by a pattern
depending on the current file:
To make this work, the current file must be called
`{\textit{prefix}\hspace{0.2em}\textit{suffix}}'
with \textit{prefix} matching precisely the argument.
Processing is then passed on to the file
`{\textit{dest}\hspace{0.2em}\textit{suffix}}'.
Surely, the same effect is achieved by
directly specifying the
argument `{\textit{dest}\hspace{0.2em}\textit{suffix}}'
in the first form.
However, that requires to set up a different file
for each child. With the alternative form of the command
all these files can have exactly the same content
which simplifies setting them up and maintaining them.

For example, the following file |draft.tex|
with a compilation flag |\version| as described in \secref{sec:flags}
compiles the main document as a draft:
%
\begin{center}
\begin{tabular}{l}
|\def\version{draft}|\\
|\input{childdoc.def}|\\
|\childdocforward{|\textit{main}|}|
\end{tabular}
\end{center}
%
Likewise, the following files |final|\textit{nn}|.tex|
compile the final version of the child document
|child|\textit{nn}|.tex|:
%
\begin{center}
\begin{tabular}{l}
|\def\version{final}|\\
|\input{childdoc.def}|\\
|\childdocforwardprefix{final}{child}|
\end{tabular}
\end{center}
%

Note that when several versions of a main file and/or of each child file
are to be generated, it may be convenient to set up a |Makefile| or
shell script to automatise the process.

%%%%%%%%%%%%%%%%%%%%%%%%%%%%%%%%%%%%%%%%%%%%%%%%%%%%%%%%%%%%%%%%%%%%%%%%%%%%%%%%
\subsection{Command Line Processing}
\label{sec:commandline}

The effect of redirection files can also be achieved by invoking
the \LaTeX{} compiler with a more elaborate command line.
Most conveniently this should be done as part
of a shell script or a |Makefile|.

When using \textsf{childdoc} in the main file, the following
command lines effectively perform a redirection
(note that depending on the shell being used,
backslashes may have to be doubled: `|\|' $\to$ `|\\|'):
%
\begin{center}
|... -jobname "|\textit{target}|" |\\|"|[\textit{flags}]%
|\input{childdoc.def}\childdocforward[|\textit{main}|]{|\textit{dest}|}"|
\end{center}
%
Here \textit{target} is the name of the output file,
\textit{main} is the name of the main file
and \textit{dest} is the name of the main or child file to be processed
(all filenames without extensions).
The optional argument \textit{main} can be omitted
if \textit{main} matches \textit{dest}.
Optionally, compilation \textit{flags} can be defined via |\def| commands.
This command line makes the \TeX{} engine believe
it is compiling the file \textit{target}
whose content is specified as the latter parameter.
The provided code then forwards the processing to
\textit{main} or \textit{dest} as described in \secref{sec:forward}.

%%%%%%%%%%%%%%%%%%%%%%%%%%%%%%%%%%%%%%%%%%%%%%%%%%%%%%%%%%%%%%%%%%%%%%%%%%%%%%%%
\subsection{Include by Input}
\label{sec:input}

Including child documents by |\include| has some restrictions by design.
Most notably, the content of a child document always occupies
its own set of pages; pages cannot be shared between child documents.
Usually, this behaviour makes perfect sense
because each child document contain an essential part of the document.
However, in some situations it may be desirable to compose
a document from a collection of parts
without having mandatory page breaks between then.
For this case, the package
provides a mechanism to include parts
by |\input| which can also be processed individually.
However, by construction this mechanism
requires manual handling of the content to be output.

%%%%%%%%%%%%%%%%%%%%%%%%%%%%%%%%%%%%%%%%
\DescribeMacro{\ifchilddocmanual}
The main file should be prepared as usual, see \secref{sec:include}.
However, the document body must make a distinction
between processing of an individual part and of the main document, e.g.:
%
\begin{center}
\begin{tabular}{l}
|\ifchilddocmanual|\\
|\input{\childdocname}|\\
|\||else|\\
\textit{document body with }|\input{|\textit{part}|}|\\
|\||fi|
\end{tabular}
\end{center}
%
The conditional |\ifchilddocmanual| is true whenever
a part to be included by |\input| is being compiled,
and the name of the part is stored in |\childdocname|.

%%%%%%%%%%%%%%%%%%%%%%%%%%%%%%%%%%%%%%%%
\DescribeMacro{\childdocby}
Each part to be included by |\input| should start with:
%
\begin{center}
\begin{tabular}{l}
|\input{childdoc.def}|\\
|\childdocby{|\textit{main}|}|\\
\end{tabular}
\end{center}
%
The directive |\childdocby| is similar to |\childdocof|
described in \secref{sec:include},
but the subsequent selection of content must be done manually.
To that end, both |\ifchilddoc| and |\ifchilddocmanual|
will be true upon processing of a part,
and the name of the part is stored in |\childdocname|.
Note that |\jobname| will be set to the filename of the current part
so that each part receives an individual |.aux| file
that does not interfere with the |.aux| file(s) of the main document.
This behaviour can be altered by the alternative form
|\childdocby[*]{|\textit{main}|}| (with a non-empty optional argument)
which uses the |.aux| file of the main document
by setting |\jobname| to \textit{main}.

%%%%%%%%%%%%%%%%%%%%%%%%%%%%%%%%%%%%%%%%%%%%%%%%%%%%%%%%%%%%%%%%%%%%%%%%%%%%%%%%
\subsection{Driver Development}
\label{sec:driver}

The \textsf{childdoc} mechanism can also be use for the development
of definition files such as \LaTeX{} styles or classes.
This case differs from the above setup with multiple parts
included by |\include| in that no |\includeonly| should be invoked.
This can be achieved by starting the include file
(before |\ProvidesPackage|) with:
%
\begin{center}
\begin{tabular}{l}
|\input{childdoc.def}|\\
|\childdocforward{|\textit{main}|}|\\
\end{tabular}
\end{center}
%
or alternatively with:
%
\begin{center}
\begin{tabular}{l}
|\input{childdoc.def}|\\
|\childdocby{|\textit{main}|}|\\
\end{tabular}
\end{center}
%
Both forms have slightly different effects as described above.
The main file is prepared as usual, see \secref{sec:include}.

%%%%%%%%%%%%%%%%%%%%%%%%%%%%%%%%%%%%%%%%%%%%%%%%%%%%%%%%%%%%%%%%%%%%%%%%%%%%%%%%
\subsection{Legacy Detection}
\label{sec:detection}

The directive |\childdocmain| in the main file can detect
whether the complete document or merely a child is to be compiled
even without using the directive |\childdocof|.
This method is deprecated because it is less robust
and there is no compelling reason to use it;
it is merely provided for backward compatibility
and it may be removed in future versions.

If the detection mechanism is to be used,
it is mandatory to correctly specify
the filename of the main file as the argument of |\childdocmain|:
%
\begin{center}
\begin{tabular}{l}
|\input{childdoc.def}|\\
|\childdocmain{|\textit{main}|}|\\
\end{tabular}
\end{center}
%
If |\jobname| does not match the argument \textit{main} of |\childdocmain|,
it is assumed that |\jobname| points to the child file to be compiled.
When using |\childdocmain| with the main file specified as argument,
it suffices to start a child file
with just |\input{|\textit{main}|}|
without loading of the package and using |\childdocof|.
If instead all processing is done
with the appropriate \textsf{childdoc} directives,
the argument of \textit{main} of |\childdocmain| can be empty.

An alternative version of the command line processing described
in \secref{sec:commandline} using the detection mechanism reads:
%
\begin{center}
|... -jobname "|\textit{target}|" "|[\textit{flags}]%
[|\def\jobname{|\textit{dest}|}|]|\input{|\textit{main}|}"|
\end{center}

%%%%%%%%%%%%%%%%%%%%%%%%%%%%%%%%%%%%%%%%%%%%%%%%%%%%%%%%%%%%%%%%%%%%%%%%%%%%%%%%
\subsection{Manual Code}
\label{sec:manual}

In case one cannot be certain whether the definitions file |childdoc.def|
is installed on the target \TeX{} distribution
and one prefers not to ship it,
it is conceivable to paste a few relevant commands into the sources.

To that end, drop all statements |\input{childdoc.def}|
and perform the replacements as outlined below.
Instead of |\childdocmain{|\textit{main}|}| add the following code
to the top of the main file:
%
\begin{center}
\begin{tabular}{l}
|\||ifdefined\childdocname\endinput\||fi\newif\ifchilddoc|\\
|\edef\childdocname{\scantokens\expandafter{\jobname\noexpand}}|\\
|\def\childdocmain{|\textit{main}|}\||ifx\childdocmain\childdocname\||else|\\
|\childdoctrue\includeonly{\childdocname}\let\jobname\childdocmain\||fi|\\
\end{tabular}
\end{center}
%
Instead of |\childdocof{|\textit{main}|}| just include the main file
at the top of each child file:
%
\begin{center}
|\input{|\textit{main}|}|
\end{center}
%
A simple redirection |\childdocforward{|\textit{dest}|}| is achieved by:
%
\begin{center}
|\def\jobname{|\textit{dest}|}\input{\jobname}|
\end{center}
%
The redirection with prefix
|\childdocforwardprefix[|\textit{prefix}|]{|\textit{dest}|}|
is accomplished by:
%
\begin{center}
\begin{tabular}{l}
|{\edef\jobname{\scantokens\expandafter{\jobname\noexpand}}|\\
|\def\redirectjob |\textit{prefix}|#1~~~{\gdef\jobname{|\textit{dest}|#1}}|\\
|\expandafter\redirectjob\jobname~~~}\input{\jobname}|
\end{tabular}
\end{center}

In an alternative approach,
child documents can be compiled by a specific command line
without additional code or specific definitions:
%
\begin{center}
|... -jobname "|\textit{target}|" "|[\textit{flags}]%
|\includeonly{|\textit{dest}|}\input{|\textit{main}|}"|
\end{center}
%

%%%%%%%%%%%%%%%%%%%%%%%%%%%%%%%%%%%%%%%%%%%%%%%%%%%%%%%%%%%%%%%%%%%%%%%%%%%%%%%%
%%%%%%%%%%%%%%%%%%%%%%%%%%%%%%%%%%%%%%%%%%%%%%%%%%%%%%%%%%%%%%%%%%%%%%%%%%%%%%%%
\section{Information}

%%%%%%%%%%%%%%%%%%%%%%%%%%%%%%%%%%%%%%%%%%%%%%%%%%%%%%%%%%%%%%%%%%%%%%%%%%%%%%%%
\subsection{Copyright}

Copyright \copyright{} 2017--2018 Niklas Beisert

This work may be distributed and/or modified under the
conditions of the \LaTeX{} Project Public License, either version 1.3
of this license or (at your option) any later version.
The latest version of this license is in
  \url{http://www.latex-project.org/lppl.txt}
and version 1.3 or later is part of all distributions of \LaTeX{}
version 2005/12/01 or later.

This work has the LPPL maintenance status `maintained'.

The Current Maintainer of this work is Niklas Beisert.

This work consists of the files |README.txt|, |childdoc.ins| and |childdoc.dtx|
as well as the derived files |childdoc.def|, |cdocsamp.tex|
with |cdocsch1.tex|, |cdocsch2.tex|, |cdocspt3.tex|, |cdocspt4.tex|,
|cdocsdrf.tex|, |cdocsfn1.tex|, |cdocsfn2.tex|
as well as |childdoc.pdf|.

%%%%%%%%%%%%%%%%%%%%%%%%%%%%%%%%%%%%%%%%%%%%%%%%%%%%%%%%%%%%%%%%%%%%%%%%%%%%%%%%
\subsection{Files and Installation}

The package consists of the files:
%
\begin{center}
\begin{tabular}{ll}
    |README.txt|   & readme file \\
    |childdoc.ins| & installation file \\
    |childdoc.dtx| & source file \\
    |childdoc.def| & definition file \\
    |cdocsamp.tex| & sample main file \\
    |cdocsch1.tex| & sample include file \\
    |cdocsch2.tex| & sample include file \\
    |cdocspt3.tex| & sample part file \\
    |cdocspt4.tex| & sample part file \\
    |cdocsdrf.tex| & sample redirection file \\
    |cdocsfn1.tex| & sample redirection file \\
    |cdocsfn2.tex| & sample redirection file \\
    |childdoc.pdf| & manual
\end{tabular}
\end{center}
%
The distribution consists of the files
|README.txt|, |childdoc.ins| and |childdoc.dtx|.
%
\begin{itemize}
\item
Run (pdf)\LaTeX{} on |childdoc.dtx|
to compile the manual |childdoc.pdf| (this file).
\item
Run \LaTeX{} on |childdoc.ins| to create the definitions file |childdoc.def|
and the sample |cdocsamp.tex| with include files
|cdocsch1.tex|, |cdocsch2.tex|, |cdocspt3.tex|, |cdocspt4.tex|,
|cdocsdrf.tex|, |cdocsfn1.tex|, |cdocsfn2.tex|.
Then copy the file |childdoc.def| to an appropriate directory of your \LaTeX{}
distribution, e.g.\ \textit{texmf-root}|/tex/latex/childdoc|.
\end{itemize}

%%%%%%%%%%%%%%%%%%%%%%%%%%%%%%%%%%%%%%%%%%%%%%%%%%%%%%%%%%%%%%%%%%%%%%%%%%%%%%%%
\subsection{Related CTAN Packages}

There are several other packages which offer a similar functionality:
%
\begin{itemize}
\item
The packages
\href{http://ctan.org/pkg/docmute}{\textsf{docmute}},
\href{http://ctan.org/pkg/includex}{\textsf{includex}} and
\href{http://ctan.org/pkg/standalone}{\textsf{standalone}}
provide commands to include only the document body of
a child file thus allowing both files to be compiled individually.
\item
The packages \href{http://ctan.org/pkg/subdocs}{\textsf{subdocs}}
and \href{http://ctan.org/pkg/subfiles}{\textsf{subfiles}}
provide structures in which the main and child documents can be
encapsulated and allowing them to be compiled individually.
The inclusion mechanism is different from the conventional |\include|.
\item
The package \href{http://ctan.org/pkg/combine}{\textsf{combine}}
is an elaborate solution to combine several documents into one.
\end{itemize}
%
See also the CTAN topic \href{http://ctan.org/topic/subdocs}{\textsf{subdocs}}
for further related packages.
The present package differs from the above solutions in that
a document structure constructed with the conventional |\include| mechanism
just needs two extra commands at the top of every file
such that all constituent files can be compiled individually.

%%%%%%%%%%%%%%%%%%%%%%%%%%%%%%%%%%%%%%%%%%%%%%%%%%%%%%%%%%%%%%%%%%%%%%%%%%%%%%%%
%\subsection{Feature Suggestions}
%
%The following is a list of features which may be useful for future
%versions of this package:
%%
%\begin{itemize}
%\item
%\ldots
%\end{itemize}

%%%%%%%%%%%%%%%%%%%%%%%%%%%%%%%%%%%%%%%%%%%%%%%%%%%%%%%%%%%%%%%%%%%%%%%%%%%%%%%%
\subsection{Revision History}

%%%%%%%%%%%%%%%%%%%%%%%%%%%%%%%%%%%%%%%%
\paragraph{v2.0:} 2018/12/30

\begin{itemize}
\item
immediate forward processing
\item
added |\childdocby| mechanism
\item
manual restructured
\end{itemize}

%%%%%%%%%%%%%%%%%%%%%%%%%%%%%%%%%%%%%%%%
\paragraph{v1.6:} 2018/01/17

\begin{itemize}
\item
application for development of include files
\item
corrections to manual
\end{itemize}

%%%%%%%%%%%%%%%%%%%%%%%%%%%%%%%%%%%%%%%%
\paragraph{v1.5:} 2017/05/21

\begin{itemize}
\item
more complete structuring introduced
\item
|\childdocof| introduced
\item
|\childdoc| renamed to |\childdocmain|
\item
|\childredirect| renamed to |\childdocforward| and |\childdocforwardprefix|
and functionality expanded
\end{itemize}

%%%%%%%%%%%%%%%%%%%%%%%%%%%%%%%%%%%%%%%%
\paragraph{v1.0:} 2017/04/27

\begin{itemize}
\item
manual and install package
\item
first version published on CTAN
\end{itemize}

%%%%%%%%%%%%%%%%%%%%%%%%%%%%%%%%%%%%%%%%
\paragraph{v0.6:} 2017/04/26

\begin{itemize}
\item
redirection mechanism added
\end{itemize}

%%%%%%%%%%%%%%%%%%%%%%%%%%%%%%%%%%%%%%%%
\paragraph{v0.5:} 2017/04/26

\begin{itemize}
\item
functionality in definition file
\end{itemize}


%%%%%%%%%%%%%%%%%%%%%%%%%%%%%%%%%%%%%%%%%%%%%%%%%%%%%%%%%%%%%%%%%%%%%%%%%%%%%%%%
%%%%%%%%%%%%%%%%%%%%%%%%%%%%%%%%%%%%%%%%%%%%%%%%%%%%%%%%%%%%%%%%%%%%%%%%%%%%%%%%
%%%%%%%%%%%%%%%%%%%%%%%%%%%%%%%%%%%%%%%%%%%%%%%%%%%%%%%%%%%%%%%%%%%%%%%%%%%%%%%%
\appendix

\settowidth\MacroIndent{\rmfamily\scriptsize 000\ }

 \DocInput{childdoc.dtx}

\end{document}
%</driver>
% \fi
%
% %%%%%%%%%%%%%%%%%%%%%%%%%%%%%%%%%%%%%%%%%%%%%%%%%%%%%%%%%%%%%%%%%%%%%%%%%%%%%%
% %%%%%%%%%%%%%%%%%%%%%%%%%%%%%%%%%%%%%%%%%%%%%%%%%%%%%%%%%%%%%%%%%%%%%%%%%%%%%%
% \section{Sample}
%\iffalse
%<*samplemain>
%\fi
%
% The following presents a sample document
% with two chapters, two parts, a title page,
% a compile flag as well as three forwarding files to set the flag.
% It consists of eight |.tex| files:
% \begin{center}
% \begin{tabular}{ll}
% |cdocsamp.tex|&main file\\
% |cdocsch1.tex|&include file for chapter 1\\
% |cdocsch2.tex|&include file for chapter 2\\
% |cdocspt3.tex|&include file for part 3\\
% |cdocspt4.tex|&include file for part 4\\
% |cdocsdrf.tex|&forwarding file for main file in draft mode\\
% |cdocsfi1.tex|&forwarding file for final version of chapter 1\\
% |cdocsfi2.tex|&forwarding file for final version of chapter 2\\
% \end{tabular}
% \end{center}
% Each of the eight files can be compiled directly by the \LaTeX{} compiler.
%
% %%%%%%%%%%%%%%%%%%%%%%%%%%%%%%%%%%%%%%
% \paragraph{Main File.}
%
% The main file is called |cdocsamp.tex|.
%
% Load the \textsf{childdoc} definitions and
% declare the filename for the main document:
%    \begin{macrocode}
\input{childdoc.def}
\childdocmain{}
%    \end{macrocode}

% Optional override for |\version| flag:
%    \begin{macrocode}
%%\ifchilddoc\else\providecommand{\version}{draft}\fi
%    \end{macrocode}

% Define the default values for the |\version| flag
% (|final| for the main file and |draft| for childs):
%    \begin{macrocode}
\ifchilddoc
\providecommand{\version}{draft}
\else
\providecommand{\version}{final}
\fi
%    \end{macrocode}

% Load the standard document class:
%    \begin{macrocode}
\documentclass[12pt]{article}
%    \end{macrocode}

% Start the document body:
%    \begin{macrocode}
\begin{document}
%    \end{macrocode}

% Declare a title page.
% Print title, part of document being processed and version flag:
%    \begin{macrocode}
\addtocounter{page}{-1}
\begin{center}
{\LARGE\bfseries{}childdoc example\par}
\vspace{1cm}
\ifchilddoc
\ifchilddocmanual part\else chapter\fi:
`\childdocname' of `\childdocjob'\par
\else
main document: `\childdocjob'\par
\fi
version: \version\par
\end{center}
\newpage
%    \end{macrocode}

% Manually include selected file,
% otherwise process as usual:
%    \begin{macrocode}
\ifchilddocmanual
\section*{part `\childdocname'}
\input{\childdocname}
\else
%    \end{macrocode}

% Include the two chapters:
%    \begin{macrocode}
\include{cdocsch1}
\include{cdocsch2}
%    \end{macrocode}

% Include the two parts unless only chapters should be displayed:
%    \begin{macrocode}
\ifchilddoc\else
\section{part three}
\input{cdocspt3}
\section{part four}
\input{cdocspt4}
\fi
%    \end{macrocode}

% Process as usual until here:
%    \begin{macrocode}
\fi
%    \end{macrocode}

% End of document body:
%    \begin{macrocode}
\end{document}
%    \end{macrocode}
%\iffalse
%</samplemain>
%\fi
%
% %%%%%%%%%%%%%%%%%%%%%%%%%%%%%%%%%%%%%%
% \paragraph{Chapter Include Files.}
%
% The include files are called |cdocsch1.tex| and |cdocsch2.tex|.
%
%\iffalse
%<*samplechap1|samplechap2>
%\fi

% Optional override for |\version| flag:
%    \begin{macrocode}
%%\providecommand{\version}{final}
%    \end{macrocode}

% Include the main document:
%    \begin{macrocode}
\input{childdoc.def}
\childdocof{cdocsamp}
%    \end{macrocode}

%\iffalse
%</samplechap1|samplechap2>
%\fi
%
%\iffalse
%<*samplechap1>
%\fi
% Some text for chapter 1:
%    \begin{macrocode}
\section{one}
some text in chapter one
%    \end{macrocode}

%\iffalse
%</samplechap1>
%\fi
% Some text for chapter 2:
%\iffalse
%<*samplechap2>
%\fi
%    \begin{macrocode}
\section{two}
more text in chapter two
%    \end{macrocode}

%\iffalse
%</samplechap2>
%\fi
%
% %%%%%%%%%%%%%%%%%%%%%%%%%%%%%%%%%%%%%%
% \paragraph{Part Include Files.}
%
% The include files are called |cdocspt3.tex| and |cdocspt4.tex|.
%
%\iffalse
%<*samplepart3|samplepart4>
%\fi

% Optional override for |\version| flag:
%    \begin{macrocode}
%%\providecommand{\version}{final}
%    \end{macrocode}

% Include the main document:
%    \begin{macrocode}
\input{childdoc.def}
\childdocby{cdocsamp}
%    \end{macrocode}

%\iffalse
%</samplepart3|samplepart4>
%\fi
%
%\iffalse
%<*samplepart3>
%\fi
% Some text for part 3:
%    \begin{macrocode}
some text in part three
%    \end{macrocode}

%\iffalse
%</samplepart3>
%\fi
% Some text for part 4:
%\iffalse
%<*samplepart4>
%\fi
%    \begin{macrocode}
more text in part four
%    \end{macrocode}

%\iffalse
%</samplepart4>
%\fi
%
% %%%%%%%%%%%%%%%%%%%%%%%%%%%%%%%%%%%%%%
% \paragraph{Forwarding for a Complete Draft.}
%
% The following forwarding file |cdocsdrf.tex|
% compiles the main document in draft mode:
%\iffalse
%<*sampledraft>
%\fi
%    \begin{macrocode}
\def\version{draft}
\input{childdoc.def}
\childdocforward{cdocsamp}
%    \end{macrocode}

%\iffalse
%</sampledraft>
%\fi
%
% %%%%%%%%%%%%%%%%%%%%%%%%%%%%%%%%%%%%%%
% \paragraph{Forwarding for Final Version of the Chapters.}
%
% The following forwarding files |cdocsfn1.tex| and |cdocsfn2.tex|
% (with identical content)
% compile the final versions of the child documents
% |cdocsch1.tex| and |cdocsch2.tex|, respectively:
%\iffalse
%<*samplefinal>
%\fi
%    \begin{macrocode}
\def\version{final}
\input{childdoc.def}
\childdocforwardprefix[cdocsamp]{cdocsfn}{cdocsch}
%    \end{macrocode}

%\iffalse
%</samplefinal>
%\fi
%
% %%%%%%%%%%%%%%%%%%%%%%%%%%%%%%%%%%%%%%
% \paragraph{Command Line Processing.}
%
% The following three command lines generate the output files
% |cdocscld|, |cdocscl1| and |cdocscl2|
% which should be identical to
% |cdocsdrf|, |cdocsch1| and |cdocsfn2|, respectively:
% \begin{center}
% \begin{tabular}{l}
% |latex -jobname cdocscld \|\\
% |  "\def\version{draft}\input{childdoc.def}\childdocforward{cdocsamp}"|\\
% |latex -jobname cdocscl1 \|\\
% |  "\input{childdoc.def}\childdocforward[cdocsamp]{cdocsch1}"|\\
% |latex -jobname cdocscl2 \|\\
% |  "\def\version{final}\input{childdoc.def}\childdocforward{cdocsch2}"|
% \end{tabular}
% \end{center}
% Note that the trailing backslash on each first line
% merely continues the input to the second line
% (for convenient cut ant paste).
% Furthermore, the command |latex| can be replaced by any
% of its alternative versions such as |pdflatex|.
%
% %%%%%%%%%%%%%%%%%%%%%%%%%%%%%%%%%%%%%%%%%%%%%%%%%%%%%%%%%%%%%%%%%%%%%%%%%%%%%%
% %%%%%%%%%%%%%%%%%%%%%%%%%%%%%%%%%%%%%%%%%%%%%%%%%%%%%%%%%%%%%%%%%%%%%%%%%%%%%%
% \section{Implementation}
%\iffalse
%<*package>
%\fi
%
% This section describes the definitions file |childdoc.def|.

% The definitions cannot be loaded using |\usepackage| or |\RequirePackage|
% which has a mechanism to prevent loading a style file more than once.
% When loading the definitions by means of |\input|
% multiple instances have to be prevented manually:
%\iffalse
%This code needs to be before the `\ProvidesFile' directive
%which is defined at the beginning of this file.
%Therefore it is also placed there and commented out here.
%</package>
%<*discard>
%\fi
%    \begin{macrocode}
\ifdefined\childdocmain\endinput\fi
%    \end{macrocode}
%\iffalse
%</discard>
%<*package>
%\fi
%
% \macro{\ifchilddoc}
% \macro{\ifchilddocmanual}
% The conditional |\ifchilddoc| tells whether a
% child (true) or main (false) document is being compiled.
% The conditional |\ifchilddocmanual| tells whether
% the |\includeonly| mechanism is used (false) or
% the selection of child files must be performed manually (true).
% The definitions initialise to false:
%    \begin{macrocode}
\newif\ifchilddoc
\newif\ifchilddocmanual
%    \end{macrocode}

% \macro{\childdocname}
% \macro{\childdocjob}
% The macro |\childdocname| stores the name of the main document
% to be compiled. The macro |\childdocjob| stores the name of
% the document on which the \LaTeX{} compiler was originally invoked.
% The content of |\jobname| cannot be compared
% to filenames specified in the source due to different catcodes.
% The following code rescans |\jobname|, stores the result
% in |\childdocname| and saves a copy in |\childdocjob|:
%    \begin{macrocode}
\edef\childdocname{\scantokens\expandafter{\jobname\noexpand}}
\let\childdocjob\childdocname
%    \end{macrocode}

% \macro{\childdocdisable}
% The macro |\childdocdisable| prevents the main file
% from being processed more than once.
% At this stage, the main document command |\childdocmain|
% is assumed to be called once again where it should do nothing.
% Any subsequent call to it should prevent
% a secondary processing of the main document
% It overwrites the forwarding commands
% |\childdocof| and |\childdocforward|
% with empty macros to prevent further inclusions of the main document:
%    \begin{macrocode}
\newcommand{\childdocdisable}
{
  \renewcommand{\childdocmain}[1]{\renewcommand{\childdocmain}[1]{\endinput}}
  \renewcommand{\childdocof}[1]{}
  \renewcommand{\childdocby}[2][]{}
  \renewcommand{\childdocforward}[2][]{}
  \renewcommand{\childdocdisable}{}
}
%    \end{macrocode}

% \macro{\childdocmain}
% The macro |\childdocmain| is to be called at the top of the main file
% with nothing or the main filename (without extension) as argument.
% First, it breaks loops.
% If the argument is not empty and does not match |\childdocname|
% (which is set by the first inclusion of |childdoc.def|),
% |\ifchilddoc| is set to true, |\includeonly| is applied to the child file
% and |\jobname| is set to the main file
% (for proper handling of |.aux| files):
%    \begin{macrocode}
\newcommand{\childdocmain}[1]
{
  \childdocdisable\childdocmain{}
  \if?#1?\else
    \begingroup
      \def\childdoctmp{#1}
      \ifx\childdoctmp\childdocname
        \def\childdoctmp{}
      \else
        \def\childdoctmp
        {
          \childdoctrue
          \includeonly{\childdocname}
          \def\childdocjob{#1}
          \def\jobname{#1}
        }
      \fi
      \expandafter
    \endgroup
    \childdoctmp
  \fi
}
%    \end{macrocode}

% \macro{\childdocof}
% The command |\childdocof| redirects
% compilation to the main file |#1|.
%    \begin{macrocode}
\newcommand{\childdocof}[1]
{
  \childdocdisable
  \childdoctrue
  \includeonly{\childdocname}
  \def\jobname{#1}
  \def\childdocjob{#1}
  \input{#1}
}
%    \end{macrocode}

% \macro{\childdocby}
% The command |\childdocby| ....
%    \begin{macrocode}
\newcommand{\childdocby}[2][]
{
  \childdocdisable
  \childdoctrue
  \childdocmanualtrue
  \if?#1?\else
    \def\jobname{#2}
  \fi
  \def\childdocjob{#2}
  \input{#2}
  \endinput
}
%    \end{macrocode}

% \macro{\childdocforward}
% The command |\childdocforward| redirects
% compilation to the main file or
% (if the optional argument is given) a child file.
% Parameters are set as if the main file
% or a child file starting with |\childdocof| was compiled.
% Then compilation is handed over to the main file:
%    \begin{macrocode}
\newcommand{\childdocforward}[2][]
{
  \begingroup
    \if?#1?
      \def\childdoctmp
      {
        \def\childdocname{#2}
        \def\childdocjob{#2}
        \def\jobname{#2}
        \input{#2}
        \endinput
      }
    \else
      \def\childdoctmp
      {
        \childdocdisable
        \def\childdocname{#2}
        \childdoctrue
        \includeonly{#2}
        \def\childdocjob{#1}
        \def\jobname{#1}
        \input{#1}
        \endinput
      }
    \fi
    \expandafter
  \endgroup
  \childdoctmp
}
%    \end{macrocode}

% \macro{\childdocforwardprefix}
% The command |\childdocforwardprefix| redirects
% compilation to the main or a child file by means of a pattern.
% The prefix |#1| in the current filename is replaced by |#2|
% and the suffix of the current filename is kept
% (it is assumed that the filename does not contain the substring `|~~~|'
% which is used as a delimiter).
% Compilation is handed over to the new file by |\childdocforward|:
%    \begin{macrocode}
\newcommand{\childdocforwardprefix}[3][]
{
  \begingroup
    \def\childdocextract #2##1~~~{\def\childdoctmp{\childdocforward[#1]{#3##1}}}
    \expandafter\childdocextract\childdocname~~~
    \expandafter
  \endgroup
  \childdoctmp
}
%    \end{macrocode}

% \macro{\childdoc}
% The deprecated macro |\childdoc| is a legacy version of |\childdocmain|:
%    \begin{macrocode}
\newcommand{\childdoc}{\childdocmain}
%    \end{macrocode}

% \macro{\childdocredirect}
% The deprecated macro |\childdocredirect| is a legacy version
% of |\childdocforward| and |\childdocforwardprefix|:
%    \begin{macrocode}
\newcommand{\childdocredirect}[2][]
{
  \begingroup
    \if?#1?
      \def\childdoctmp{\childdocforward{#2}}
    \else
      \def\childdoctmp{\childdocforwardprefix{#1}{#2}}
    \fi
    \expandafter
  \endgroup
  \childdoctmp
}
%    \end{macrocode}

%\iffalse
%</package>
%\fi
%
\endinput
\childdocforward{cdocsamp}"|\\
% |latex -jobname cdocscl1 \|\\
% |  "% \iffalse
%
% childdoc.dtx Copyright (C) 2017-2018 Niklas Beisert
%
% This work may be distributed and/or modified under the
% conditions of the LaTeX Project Public License, either version 1.3
% of this license or (at your option) any later version.
% The latest version of this license is in
%   http://www.latex-project.org/lppl.txt
% and version 1.3 or later is part of all distributions of LaTeX
% version 2005/12/01 or later.
%
% This work has the LPPL maintenance status `maintained'.
%
% The Current Maintainer of this work is Niklas Beisert.
%
% This work consists of the files childdoc.dtx and childdoc.ins
% and the derived files childdoc.def and cdocsamp.tex with
% cdocsch1.tex, cdocsch2.tex, cdocsdrf.tex, cdocsfn1.tex, cdocsfn2.tex.
%
%<package>\ifdefined\childdocmain\endinput\fi
%<package>\ProvidesFile{childdoc.def}[2018/12/30 v2.0 child document driver]
%<samplemain>\ProvidesFile{cdocsamp.tex}[2018/12/30 v2.0 sample for childdoc]
%<*driver>
%\ProvidesFile{childdoc.drv}[2018/12/30 v2.0 childdoc reference manual file]
\PassOptionsToClass{10pt,a4paper}{article}
\documentclass{ltxdoc}

\usepackage[margin=35mm]{geometry}
\usepackage{hyperref}
\usepackage{hyperxmp}
\usepackage[usenames]{color}

\hypersetup{colorlinks=true}
\hypersetup{pdfstartview=FitH}
\hypersetup{pdfpagemode=UseNone}
\hypersetup{pdfsource={}}
\hypersetup{pdflang={en-UK}}
\hypersetup{pdfcopyright={Copyright 2017-2018 Niklas Beisert.
  This work may be distributed and/or modified under the
  conditions of the LaTeX Project Public License, either version 1.3
  of this license or (at your option) any later version.}}
\hypersetup{pdflicenseurl={http://www.latex-project.org/lppl.txt}}
\hypersetup{pdfcontactaddress={ETH Zurich, ITP, HIT K,
  Wolfgang-Pauli-Strasse 27}}
\hypersetup{pdfcontactpostcode={8093}}
\hypersetup{pdfcontactcity={Zurich}}
\hypersetup{pdfcontactcountry={Switzerland}}
\hypersetup{pdfcontactemail={nbeisert@itp.phys.ethz.ch}}
\hypersetup{pdfcontacturl={http://people.phys.ethz.ch/\xmptilde nbeisert/}}

\newcommand{\secref}[1]{\hyperref[#1]{section \ref*{#1}}}

\parskip1ex
\parindent0pt
\let\olditemize\itemize
\def\itemize{\olditemize\parskip0pt}

\begin{document}

\title{The \textsf{childdoc} Package}
\hypersetup{pdftitle={The childdoc Package}}
\author{Niklas Beisert\\[2ex]
  Institut f\"ur Theoretische Physik\\
  Eidgen\"ossische Technische Hochschule Z\"urich\\
  Wolfgang-Pauli-Strasse 27, 8093 Z\"urich, Switzerland\\[1ex]
  \href{mailto:nbeisert@itp.phys.ethz.ch}
  {\texttt{nbeisert@itp.phys.ethz.ch}}}
\hypersetup{pdfauthor={Niklas Beisert}}
\hypersetup{pdfsubject={Manual for the LaTeX2e Package childdoc}}
\date{30 December 2018, \textsf{v2.0}}
\maketitle

\begin{abstract}\noindent
\textsf{childdoc} is a \LaTeXe{} package
that enables the direct compilation
of document sections included by |\include|
to individual files.
\end{abstract}

\begingroup
\parskip0ex
\tableofcontents
\endgroup

%%%%%%%%%%%%%%%%%%%%%%%%%%%%%%%%%%%%%%%%%%%%%%%%%%%%%%%%%%%%%%%%%%%%%%%%%%%%%%%%
%%%%%%%%%%%%%%%%%%%%%%%%%%%%%%%%%%%%%%%%%%%%%%%%%%%%%%%%%%%%%%%%%%%%%%%%%%%%%%%%
\section{Introduction}

\LaTeX{} provides a mechanism to structure a large document (such as a book)
into a main file and several child files (containing the chapters)
using the |\include| command.
This mechanism is beneficial for documents
which span hundreds of pages in order to
make the source file(s) more manageable.
Moreover, compilation can be restricted to
selected child files by means of the |\includeonly| command.
The latter feature can be used to reduce the compilation time while editing
(this was significantly more useful in the earlier days of \LaTeX{})
or to generate a smaller document which is easier to navigate.
Another application of |\includeonly| is to generate
documents consisting of selected parts of the complete document.

However, there are a few drawbacks of the plain |\include| mechanism:
\begin{itemize}
\item
The child files cannot be compiled on their own,
they can only be compiled via the main file.
A naive editing environment
(such as a text editor with an option
to have the current file processed by \LaTeX)
may require one to switch to the main file before compiling;
attempting to compile the child file produces errors.
\item
The main file must be modified (each time)
to adjust the |\includeonly| command
to the present needs. This easily leaves the main file in a messy state.
\item
The generated document will always carry the filename
of the main document. This is inconvenient if
several child files are to be compiled and
to be kept for distribution.
\end{itemize}

The present package provides a simple interface
to make child files individually compilable by \LaTeX{}.
Compiling a child file then has the same effect as compiling
the main file with an |\includeonly| command
to select the appropriate child.
Moreover the generated document will carry the name of the child
rather than the main file.
This resolves all three above issues.

This feature is meant to make the editing of books,
thesis documents and lecture notes somewhat more convenient.
However, the package can also be used efficiently for
composing a series of documents (such as exercise sheets)
which are typically distributed individually.
It then assists the author in generating the individual documents
(potentially in different versions)
as well as a document containing the collected series.
Another application is in developing style files
or other kinds of included material
where compilation of the style file could redirect
to a sample or test file.

%%%%%%%%%%%%%%%%%%%%%%%%%%%%%%%%%%%%%%%%%%%%%%%%%%%%%%%%%%%%%%%%%%%%%%%%%%%%%%%%
%%%%%%%%%%%%%%%%%%%%%%%%%%%%%%%%%%%%%%%%%%%%%%%%%%%%%%%%%%%%%%%%%%%%%%%%%%%%%%%%
\section{Usage}

First of all, the package \textsf{childdoc} is \emph{not} a standard
\LaTeXe{} |.sty| style file! Therefore it needs to be invoked in
a non-standard way.

%%%%%%%%%%%%%%%%%%%%%%%%%%%%%%%%%%%%%%%%%%%%%%%%%%%%%%%%%%%%%%%%%%%%%%%%%%%%%%%%
\subsection{Included Files}
\label{sec:include}

%%%%%%%%%%%%%%%%%%%%%%%%%%%%%%%%%%%%%%%%
\DescribeMacro{\childdocmain}
To use the package, add the commands
\begin{center}
\begin{tabular}{l}
|\input{childdoc.def}|\\
|\childdocmain{}|\\
\end{tabular}
\end{center}
at the very top of the main \LaTeX{} file,
in particular \emph{before} the |\documentclass| statement!
The argument of |\childdocmain| should be left empty
(but it must be present).

%%%%%%%%%%%%%%%%%%%%%%%%%%%%%%%%%%%%%%%%
\DescribeMacro{\childdocof}
Furthermore, add the commands
\begin{center}
\begin{tabular}{l}
|\input{childdoc.def}|\\
|\childdocof{|\textit{main}|}|\\
\end{tabular}
\end{center}
at the top of every child file \textit{child}
which is included by |\include{|\textit{child}|}|
from within the main file
(or at least for those files to be compiled individually).
The argument \textit{main} must be the filename of the main file.

There are a couple of
considerations in setting up the main and child documents:

%%%%%%%%%%%%%%%%%%%%%%%%%%%%%%%%%%%%%%%%
\paragraph{Restrictions.}

Please note the following restrictions:
\begin{itemize}
\item
|\childdocmain| must be called with one argument \textit{main}
to ensure compatibility with earlier version of the package.
It must either be empty (|\childdocmain{}|)
or precisely match the filename of the main file in which it is specified.
See \secref{sec:detection} for further information.
\item
The filename \textit{main} must be specified without the |.tex| extension.
\item
The filename \textit{main} is case sensitive
(even in case-insensitive file systems)
due to internal string comparison.
\item
The argument \textit{main} should be fully expanded, it cannot be a macro.
\item
Subdirectories and special characters should be avoided in filenames.
\item
The command |\childdocmain{|\textit{main}|}| must be followed by a whitespace.
It should not be followed immediately by another command
or by a comment mark `|%|'.
This is because the \TeX{} parser reads the token immediately following
the argument of |\childdocmain| and puts it
at the beginning of every child section;
however, a white\-space is ignored.
\end{itemize}

%%%%%%%%%%%%%%%%%%%%%%%%%%%%%%%%%%%%%%%%
\paragraph{Content of Main File.}

It is advisable to place all content in the child files included by |\include|.
Any output contained in the main file will appear in all child documents
unless suppressed manually;
it cannot be suppressed automatically by the |\includeonly| directive
and thus should normally be avoided.
A method to include some content in the main file
by means of conditional processing is described in \secref{sec:conditional}.

%%%%%%%%%%%%%%%%%%%%%%%%%%%%%%%%%%%%%%%%
\paragraph{Page Numbering.}

When only a part of the document is compiled,
the appropriate numbering of pages
(as well as other status parameters)
is determined from the |.aux| files.
The latter contain information from previous passes.
However this information needs to propagate through
all intermediate child documents.
Therefore the page numbering in child documents may well
be inconsistent until the complete document is compiled at least once.

A useful (if unconventional) way to always ensure a consistent
page numbering is to restart the numbering in each child document
and denote the pages by `\textit{child}|.|\textit{page}'
where \textit{child} represents the chapter/section number of the child file.
This can be achieved by the command
|\numberwithin{page}{|\textit{child}|}|
of the \textsf{amsmath} package
where \textit{child} can be |chapter| or |section|
depending on the chosen structuring.
Alternatively, one can modify the macro |\thepage| appropriately
and reset the counter |page| at the start of each child file.

%%%%%%%%%%%%%%%%%%%%%%%%%%%%%%%%%%%%%%%%%%%%%%%%%%%%%%%%%%%%%%%%%%%%%%%%%%%%%%%%
\subsection{Conditional Processing}
\label{sec:conditional}

The package provides a mechanism to compile different versions
of a document. To customise the versions further some conditional processing
can come in handy to distinguish which version is being compiled.
The package provides two macros to describe the compilation context:

%%%%%%%%%%%%%%%%%%%%%%%%%%%%%%%%%%%%%%%%
\DescribeMacro{\ifchilddoc}
The conditional |\ifchilddoc| distinguishes between the compilation of
child documents and the main document:
%
\begin{center}
|\ifchilddoc |\textit{child-code}| |[|\||else |\textit{main-code}]| \||fi|
\end{center}

%%%%%%%%%%%%%%%%%%%%%%%%%%%%%%%%%%%%%%%%
\DescribeMacro{\childdocname}
\DescribeMacro{\childdocjob}
The macro |\childdocname| contains the filename (without extension)
of the main or child file being processed.
Note that |\childdocjob| will always contain the name of the main file.

%%%%%%%%%%%%%%%%%%%%%%%%%%%%%%%%%%%%%%%%
\paragraph{Title Page.}

Conditional processing can be used to include a title or banner page
in the main document when proper precautions are taken.
Importantly, the code in the main file should ensure that the page counter
(as well as other status parameters which are stored in the |.aux| files)
takes the same value after the conditional processing.
Otherwise the page numbers may take divergent values
depending on which part is compiled.

For example, a title page could be declared by:
%
\begin{center}
\begin{tabular}{l}
|\ifchilddoc\||else|\\
|\addtocounter{page}{-1}|\\
\textit{code for title page}\\
|\newpage|\\
|\||fi|
\end{tabular}
\end{center}
%
A banner page for the child documents can be generated by:
%
\begin{center}
\begin{tabular}{l}
|\ifchilddoc|\\
|\addtocounter{page}{-1}|\\
\textit{code for banner page}\\
|\newpage|\\
|\||fi|
\end{tabular}
\end{center}
%
Here one could write a message such as:
\begin{center}
|This is the part \childdocname{} of \childdocjob{}.|
\end{center}

%%%%%%%%%%%%%%%%%%%%%%%%%%%%%%%%%%%%%%%%%%%%%%%%%%%%%%%%%%%%%%%%%%%%%%%%%%%%%%%%
\subsection{Flags}
\label{sec:flags}

The package makes it easy to generate different versions
of the main or child documents.
To this end compilation flags can be defined
and assigned different default values.
They will be particularly useful in conjunction
with the forwarding mechanism described in \secref{sec:forward}.

For example, it may be useful to have a flag |\version|
which can be set to |draft| or |final|.
The document source will contain some conditional code
depending on the value of |\version|.
Suppose further, the flag should default to |final| for the main file
and to |draft| for child files
which is a natural assignment for editing the document.
This is achieved by placing the following code
in the preamble of the main document
(below the |\childdocmain| directive):
%
\begin{center}
\begin{tabular}{l}
|\ifchilddoc|\\
|\providecommand{\version}{draft}|\\
|\||else|\\
|\providecommand{\version}{final}|\\
|\||fi|
\end{tabular}
\end{center}
%
The definition by |\providecommand| makes sure
that previous definitions are not overwritten.
Further statements |\providecommand{\version}{...}|
can thus be added before the above code to override it.

For the main file, one might add a line
(between |\childdocmain| and the above block)
%
\begin{center}
|%\ifchilddoc\||else\providecommand{\version}{draft}\||fi|
\end{center}
%
which can be uncommented to produce a draft version.
Likewise one can add a line to the very top of a child file
(above the |\childdocof{|\textit{main}|}| directive)
%
\begin{center}
|%\providecommand{\version}{final}|
\end{center}
%
which can be uncommented to produce the final version of this child document.

%%%%%%%%%%%%%%%%%%%%%%%%%%%%%%%%%%%%%%%%%%%%%%%%%%%%%%%%%%%%%%%%%%%%%%%%%%%%%%%%
\subsection{Forwarding}
\label{sec:forward}

Different versions of the main or child documents
using compilation flags as described in \secref{sec:flags}
can be (permanently) stored in different files
for convenient compilation, viewing and distribution.
To this end, the package defines a command
to pass on compilation to a different file:

%%%%%%%%%%%%%%%%%%%%%%%%%%%%%%%%%%%%%%%%
\DescribeMacro{\childdocforward}
The command |\childdocforward| redirects processing to
another source file:
%
\begin{center}
\begin{tabular}{l}
|\input{childdoc.def}|\\
|\childdocforward[|\textit{main}|]{|\textit{dest}|}|\\
\end{tabular}
\end{center}
%
The argument \textit{dest} is the destination file
(without extension).
It should be the main file or one of the child files.
Note that further \textsf{childdoc} directives
such as |\childdocof| and |\childdocforward|
in the indicated file will be processed in this form.
The optional argument \textit{main}
passes on directly to the main file \textit{main}
while pretending to compile the child \textit{dest}.
This form behaves as if \textit{dest}
issues |\childdocof{|\textit{main}|}| right away,
and no further \textsf{childdoc} directives will be processed.

%%%%%%%%%%%%%%%%%%%%%%%%%%%%%%%%%%%%%%%%
\DescribeMacro{\...prefix}
In the alternative form |\childdocforwardprefix|,
%
\begin{center}
\begin{tabular}{l}
|\input{childdoc.def}|\\
|\childdocforwardprefix[|\textit{main}|]{|\textit{prefix}|}{|\textit{dest}|}|
\end{tabular}
\end{center}
%
the destination file is determined by a pattern
depending on the current file:
To make this work, the current file must be called
`{\textit{prefix}\hspace{0.2em}\textit{suffix}}'
with \textit{prefix} matching precisely the argument.
Processing is then passed on to the file
`{\textit{dest}\hspace{0.2em}\textit{suffix}}'.
Surely, the same effect is achieved by
directly specifying the
argument `{\textit{dest}\hspace{0.2em}\textit{suffix}}'
in the first form.
However, that requires to set up a different file
for each child. With the alternative form of the command
all these files can have exactly the same content
which simplifies setting them up and maintaining them.

For example, the following file |draft.tex|
with a compilation flag |\version| as described in \secref{sec:flags}
compiles the main document as a draft:
%
\begin{center}
\begin{tabular}{l}
|\def\version{draft}|\\
|\input{childdoc.def}|\\
|\childdocforward{|\textit{main}|}|
\end{tabular}
\end{center}
%
Likewise, the following files |final|\textit{nn}|.tex|
compile the final version of the child document
|child|\textit{nn}|.tex|:
%
\begin{center}
\begin{tabular}{l}
|\def\version{final}|\\
|\input{childdoc.def}|\\
|\childdocforwardprefix{final}{child}|
\end{tabular}
\end{center}
%

Note that when several versions of a main file and/or of each child file
are to be generated, it may be convenient to set up a |Makefile| or
shell script to automatise the process.

%%%%%%%%%%%%%%%%%%%%%%%%%%%%%%%%%%%%%%%%%%%%%%%%%%%%%%%%%%%%%%%%%%%%%%%%%%%%%%%%
\subsection{Command Line Processing}
\label{sec:commandline}

The effect of redirection files can also be achieved by invoking
the \LaTeX{} compiler with a more elaborate command line.
Most conveniently this should be done as part
of a shell script or a |Makefile|.

When using \textsf{childdoc} in the main file, the following
command lines effectively perform a redirection
(note that depending on the shell being used,
backslashes may have to be doubled: `|\|' $\to$ `|\\|'):
%
\begin{center}
|... -jobname "|\textit{target}|" |\\|"|[\textit{flags}]%
|\input{childdoc.def}\childdocforward[|\textit{main}|]{|\textit{dest}|}"|
\end{center}
%
Here \textit{target} is the name of the output file,
\textit{main} is the name of the main file
and \textit{dest} is the name of the main or child file to be processed
(all filenames without extensions).
The optional argument \textit{main} can be omitted
if \textit{main} matches \textit{dest}.
Optionally, compilation \textit{flags} can be defined via |\def| commands.
This command line makes the \TeX{} engine believe
it is compiling the file \textit{target}
whose content is specified as the latter parameter.
The provided code then forwards the processing to
\textit{main} or \textit{dest} as described in \secref{sec:forward}.

%%%%%%%%%%%%%%%%%%%%%%%%%%%%%%%%%%%%%%%%%%%%%%%%%%%%%%%%%%%%%%%%%%%%%%%%%%%%%%%%
\subsection{Include by Input}
\label{sec:input}

Including child documents by |\include| has some restrictions by design.
Most notably, the content of a child document always occupies
its own set of pages; pages cannot be shared between child documents.
Usually, this behaviour makes perfect sense
because each child document contain an essential part of the document.
However, in some situations it may be desirable to compose
a document from a collection of parts
without having mandatory page breaks between then.
For this case, the package
provides a mechanism to include parts
by |\input| which can also be processed individually.
However, by construction this mechanism
requires manual handling of the content to be output.

%%%%%%%%%%%%%%%%%%%%%%%%%%%%%%%%%%%%%%%%
\DescribeMacro{\ifchilddocmanual}
The main file should be prepared as usual, see \secref{sec:include}.
However, the document body must make a distinction
between processing of an individual part and of the main document, e.g.:
%
\begin{center}
\begin{tabular}{l}
|\ifchilddocmanual|\\
|\input{\childdocname}|\\
|\||else|\\
\textit{document body with }|\input{|\textit{part}|}|\\
|\||fi|
\end{tabular}
\end{center}
%
The conditional |\ifchilddocmanual| is true whenever
a part to be included by |\input| is being compiled,
and the name of the part is stored in |\childdocname|.

%%%%%%%%%%%%%%%%%%%%%%%%%%%%%%%%%%%%%%%%
\DescribeMacro{\childdocby}
Each part to be included by |\input| should start with:
%
\begin{center}
\begin{tabular}{l}
|\input{childdoc.def}|\\
|\childdocby{|\textit{main}|}|\\
\end{tabular}
\end{center}
%
The directive |\childdocby| is similar to |\childdocof|
described in \secref{sec:include},
but the subsequent selection of content must be done manually.
To that end, both |\ifchilddoc| and |\ifchilddocmanual|
will be true upon processing of a part,
and the name of the part is stored in |\childdocname|.
Note that |\jobname| will be set to the filename of the current part
so that each part receives an individual |.aux| file
that does not interfere with the |.aux| file(s) of the main document.
This behaviour can be altered by the alternative form
|\childdocby[*]{|\textit{main}|}| (with a non-empty optional argument)
which uses the |.aux| file of the main document
by setting |\jobname| to \textit{main}.

%%%%%%%%%%%%%%%%%%%%%%%%%%%%%%%%%%%%%%%%%%%%%%%%%%%%%%%%%%%%%%%%%%%%%%%%%%%%%%%%
\subsection{Driver Development}
\label{sec:driver}

The \textsf{childdoc} mechanism can also be use for the development
of definition files such as \LaTeX{} styles or classes.
This case differs from the above setup with multiple parts
included by |\include| in that no |\includeonly| should be invoked.
This can be achieved by starting the include file
(before |\ProvidesPackage|) with:
%
\begin{center}
\begin{tabular}{l}
|\input{childdoc.def}|\\
|\childdocforward{|\textit{main}|}|\\
\end{tabular}
\end{center}
%
or alternatively with:
%
\begin{center}
\begin{tabular}{l}
|\input{childdoc.def}|\\
|\childdocby{|\textit{main}|}|\\
\end{tabular}
\end{center}
%
Both forms have slightly different effects as described above.
The main file is prepared as usual, see \secref{sec:include}.

%%%%%%%%%%%%%%%%%%%%%%%%%%%%%%%%%%%%%%%%%%%%%%%%%%%%%%%%%%%%%%%%%%%%%%%%%%%%%%%%
\subsection{Legacy Detection}
\label{sec:detection}

The directive |\childdocmain| in the main file can detect
whether the complete document or merely a child is to be compiled
even without using the directive |\childdocof|.
This method is deprecated because it is less robust
and there is no compelling reason to use it;
it is merely provided for backward compatibility
and it may be removed in future versions.

If the detection mechanism is to be used,
it is mandatory to correctly specify
the filename of the main file as the argument of |\childdocmain|:
%
\begin{center}
\begin{tabular}{l}
|\input{childdoc.def}|\\
|\childdocmain{|\textit{main}|}|\\
\end{tabular}
\end{center}
%
If |\jobname| does not match the argument \textit{main} of |\childdocmain|,
it is assumed that |\jobname| points to the child file to be compiled.
When using |\childdocmain| with the main file specified as argument,
it suffices to start a child file
with just |\input{|\textit{main}|}|
without loading of the package and using |\childdocof|.
If instead all processing is done
with the appropriate \textsf{childdoc} directives,
the argument of \textit{main} of |\childdocmain| can be empty.

An alternative version of the command line processing described
in \secref{sec:commandline} using the detection mechanism reads:
%
\begin{center}
|... -jobname "|\textit{target}|" "|[\textit{flags}]%
[|\def\jobname{|\textit{dest}|}|]|\input{|\textit{main}|}"|
\end{center}

%%%%%%%%%%%%%%%%%%%%%%%%%%%%%%%%%%%%%%%%%%%%%%%%%%%%%%%%%%%%%%%%%%%%%%%%%%%%%%%%
\subsection{Manual Code}
\label{sec:manual}

In case one cannot be certain whether the definitions file |childdoc.def|
is installed on the target \TeX{} distribution
and one prefers not to ship it,
it is conceivable to paste a few relevant commands into the sources.

To that end, drop all statements |\input{childdoc.def}|
and perform the replacements as outlined below.
Instead of |\childdocmain{|\textit{main}|}| add the following code
to the top of the main file:
%
\begin{center}
\begin{tabular}{l}
|\||ifdefined\childdocname\endinput\||fi\newif\ifchilddoc|\\
|\edef\childdocname{\scantokens\expandafter{\jobname\noexpand}}|\\
|\def\childdocmain{|\textit{main}|}\||ifx\childdocmain\childdocname\||else|\\
|\childdoctrue\includeonly{\childdocname}\let\jobname\childdocmain\||fi|\\
\end{tabular}
\end{center}
%
Instead of |\childdocof{|\textit{main}|}| just include the main file
at the top of each child file:
%
\begin{center}
|\input{|\textit{main}|}|
\end{center}
%
A simple redirection |\childdocforward{|\textit{dest}|}| is achieved by:
%
\begin{center}
|\def\jobname{|\textit{dest}|}\input{\jobname}|
\end{center}
%
The redirection with prefix
|\childdocforwardprefix[|\textit{prefix}|]{|\textit{dest}|}|
is accomplished by:
%
\begin{center}
\begin{tabular}{l}
|{\edef\jobname{\scantokens\expandafter{\jobname\noexpand}}|\\
|\def\redirectjob |\textit{prefix}|#1~~~{\gdef\jobname{|\textit{dest}|#1}}|\\
|\expandafter\redirectjob\jobname~~~}\input{\jobname}|
\end{tabular}
\end{center}

In an alternative approach,
child documents can be compiled by a specific command line
without additional code or specific definitions:
%
\begin{center}
|... -jobname "|\textit{target}|" "|[\textit{flags}]%
|\includeonly{|\textit{dest}|}\input{|\textit{main}|}"|
\end{center}
%

%%%%%%%%%%%%%%%%%%%%%%%%%%%%%%%%%%%%%%%%%%%%%%%%%%%%%%%%%%%%%%%%%%%%%%%%%%%%%%%%
%%%%%%%%%%%%%%%%%%%%%%%%%%%%%%%%%%%%%%%%%%%%%%%%%%%%%%%%%%%%%%%%%%%%%%%%%%%%%%%%
\section{Information}

%%%%%%%%%%%%%%%%%%%%%%%%%%%%%%%%%%%%%%%%%%%%%%%%%%%%%%%%%%%%%%%%%%%%%%%%%%%%%%%%
\subsection{Copyright}

Copyright \copyright{} 2017--2018 Niklas Beisert

This work may be distributed and/or modified under the
conditions of the \LaTeX{} Project Public License, either version 1.3
of this license or (at your option) any later version.
The latest version of this license is in
  \url{http://www.latex-project.org/lppl.txt}
and version 1.3 or later is part of all distributions of \LaTeX{}
version 2005/12/01 or later.

This work has the LPPL maintenance status `maintained'.

The Current Maintainer of this work is Niklas Beisert.

This work consists of the files |README.txt|, |childdoc.ins| and |childdoc.dtx|
as well as the derived files |childdoc.def|, |cdocsamp.tex|
with |cdocsch1.tex|, |cdocsch2.tex|, |cdocspt3.tex|, |cdocspt4.tex|,
|cdocsdrf.tex|, |cdocsfn1.tex|, |cdocsfn2.tex|
as well as |childdoc.pdf|.

%%%%%%%%%%%%%%%%%%%%%%%%%%%%%%%%%%%%%%%%%%%%%%%%%%%%%%%%%%%%%%%%%%%%%%%%%%%%%%%%
\subsection{Files and Installation}

The package consists of the files:
%
\begin{center}
\begin{tabular}{ll}
    |README.txt|   & readme file \\
    |childdoc.ins| & installation file \\
    |childdoc.dtx| & source file \\
    |childdoc.def| & definition file \\
    |cdocsamp.tex| & sample main file \\
    |cdocsch1.tex| & sample include file \\
    |cdocsch2.tex| & sample include file \\
    |cdocspt3.tex| & sample part file \\
    |cdocspt4.tex| & sample part file \\
    |cdocsdrf.tex| & sample redirection file \\
    |cdocsfn1.tex| & sample redirection file \\
    |cdocsfn2.tex| & sample redirection file \\
    |childdoc.pdf| & manual
\end{tabular}
\end{center}
%
The distribution consists of the files
|README.txt|, |childdoc.ins| and |childdoc.dtx|.
%
\begin{itemize}
\item
Run (pdf)\LaTeX{} on |childdoc.dtx|
to compile the manual |childdoc.pdf| (this file).
\item
Run \LaTeX{} on |childdoc.ins| to create the definitions file |childdoc.def|
and the sample |cdocsamp.tex| with include files
|cdocsch1.tex|, |cdocsch2.tex|, |cdocspt3.tex|, |cdocspt4.tex|,
|cdocsdrf.tex|, |cdocsfn1.tex|, |cdocsfn2.tex|.
Then copy the file |childdoc.def| to an appropriate directory of your \LaTeX{}
distribution, e.g.\ \textit{texmf-root}|/tex/latex/childdoc|.
\end{itemize}

%%%%%%%%%%%%%%%%%%%%%%%%%%%%%%%%%%%%%%%%%%%%%%%%%%%%%%%%%%%%%%%%%%%%%%%%%%%%%%%%
\subsection{Related CTAN Packages}

There are several other packages which offer a similar functionality:
%
\begin{itemize}
\item
The packages
\href{http://ctan.org/pkg/docmute}{\textsf{docmute}},
\href{http://ctan.org/pkg/includex}{\textsf{includex}} and
\href{http://ctan.org/pkg/standalone}{\textsf{standalone}}
provide commands to include only the document body of
a child file thus allowing both files to be compiled individually.
\item
The packages \href{http://ctan.org/pkg/subdocs}{\textsf{subdocs}}
and \href{http://ctan.org/pkg/subfiles}{\textsf{subfiles}}
provide structures in which the main and child documents can be
encapsulated and allowing them to be compiled individually.
The inclusion mechanism is different from the conventional |\include|.
\item
The package \href{http://ctan.org/pkg/combine}{\textsf{combine}}
is an elaborate solution to combine several documents into one.
\end{itemize}
%
See also the CTAN topic \href{http://ctan.org/topic/subdocs}{\textsf{subdocs}}
for further related packages.
The present package differs from the above solutions in that
a document structure constructed with the conventional |\include| mechanism
just needs two extra commands at the top of every file
such that all constituent files can be compiled individually.

%%%%%%%%%%%%%%%%%%%%%%%%%%%%%%%%%%%%%%%%%%%%%%%%%%%%%%%%%%%%%%%%%%%%%%%%%%%%%%%%
%\subsection{Feature Suggestions}
%
%The following is a list of features which may be useful for future
%versions of this package:
%%
%\begin{itemize}
%\item
%\ldots
%\end{itemize}

%%%%%%%%%%%%%%%%%%%%%%%%%%%%%%%%%%%%%%%%%%%%%%%%%%%%%%%%%%%%%%%%%%%%%%%%%%%%%%%%
\subsection{Revision History}

%%%%%%%%%%%%%%%%%%%%%%%%%%%%%%%%%%%%%%%%
\paragraph{v2.0:} 2018/12/30

\begin{itemize}
\item
immediate forward processing
\item
added |\childdocby| mechanism
\item
manual restructured
\end{itemize}

%%%%%%%%%%%%%%%%%%%%%%%%%%%%%%%%%%%%%%%%
\paragraph{v1.6:} 2018/01/17

\begin{itemize}
\item
application for development of include files
\item
corrections to manual
\end{itemize}

%%%%%%%%%%%%%%%%%%%%%%%%%%%%%%%%%%%%%%%%
\paragraph{v1.5:} 2017/05/21

\begin{itemize}
\item
more complete structuring introduced
\item
|\childdocof| introduced
\item
|\childdoc| renamed to |\childdocmain|
\item
|\childredirect| renamed to |\childdocforward| and |\childdocforwardprefix|
and functionality expanded
\end{itemize}

%%%%%%%%%%%%%%%%%%%%%%%%%%%%%%%%%%%%%%%%
\paragraph{v1.0:} 2017/04/27

\begin{itemize}
\item
manual and install package
\item
first version published on CTAN
\end{itemize}

%%%%%%%%%%%%%%%%%%%%%%%%%%%%%%%%%%%%%%%%
\paragraph{v0.6:} 2017/04/26

\begin{itemize}
\item
redirection mechanism added
\end{itemize}

%%%%%%%%%%%%%%%%%%%%%%%%%%%%%%%%%%%%%%%%
\paragraph{v0.5:} 2017/04/26

\begin{itemize}
\item
functionality in definition file
\end{itemize}


%%%%%%%%%%%%%%%%%%%%%%%%%%%%%%%%%%%%%%%%%%%%%%%%%%%%%%%%%%%%%%%%%%%%%%%%%%%%%%%%
%%%%%%%%%%%%%%%%%%%%%%%%%%%%%%%%%%%%%%%%%%%%%%%%%%%%%%%%%%%%%%%%%%%%%%%%%%%%%%%%
%%%%%%%%%%%%%%%%%%%%%%%%%%%%%%%%%%%%%%%%%%%%%%%%%%%%%%%%%%%%%%%%%%%%%%%%%%%%%%%%
\appendix

\settowidth\MacroIndent{\rmfamily\scriptsize 000\ }

 \DocInput{childdoc.dtx}

\end{document}
%</driver>
% \fi
%
% %%%%%%%%%%%%%%%%%%%%%%%%%%%%%%%%%%%%%%%%%%%%%%%%%%%%%%%%%%%%%%%%%%%%%%%%%%%%%%
% %%%%%%%%%%%%%%%%%%%%%%%%%%%%%%%%%%%%%%%%%%%%%%%%%%%%%%%%%%%%%%%%%%%%%%%%%%%%%%
% \section{Sample}
%\iffalse
%<*samplemain>
%\fi
%
% The following presents a sample document
% with two chapters, two parts, a title page,
% a compile flag as well as three forwarding files to set the flag.
% It consists of eight |.tex| files:
% \begin{center}
% \begin{tabular}{ll}
% |cdocsamp.tex|&main file\\
% |cdocsch1.tex|&include file for chapter 1\\
% |cdocsch2.tex|&include file for chapter 2\\
% |cdocspt3.tex|&include file for part 3\\
% |cdocspt4.tex|&include file for part 4\\
% |cdocsdrf.tex|&forwarding file for main file in draft mode\\
% |cdocsfi1.tex|&forwarding file for final version of chapter 1\\
% |cdocsfi2.tex|&forwarding file for final version of chapter 2\\
% \end{tabular}
% \end{center}
% Each of the eight files can be compiled directly by the \LaTeX{} compiler.
%
% %%%%%%%%%%%%%%%%%%%%%%%%%%%%%%%%%%%%%%
% \paragraph{Main File.}
%
% The main file is called |cdocsamp.tex|.
%
% Load the \textsf{childdoc} definitions and
% declare the filename for the main document:
%    \begin{macrocode}
\input{childdoc.def}
\childdocmain{}
%    \end{macrocode}

% Optional override for |\version| flag:
%    \begin{macrocode}
%%\ifchilddoc\else\providecommand{\version}{draft}\fi
%    \end{macrocode}

% Define the default values for the |\version| flag
% (|final| for the main file and |draft| for childs):
%    \begin{macrocode}
\ifchilddoc
\providecommand{\version}{draft}
\else
\providecommand{\version}{final}
\fi
%    \end{macrocode}

% Load the standard document class:
%    \begin{macrocode}
\documentclass[12pt]{article}
%    \end{macrocode}

% Start the document body:
%    \begin{macrocode}
\begin{document}
%    \end{macrocode}

% Declare a title page.
% Print title, part of document being processed and version flag:
%    \begin{macrocode}
\addtocounter{page}{-1}
\begin{center}
{\LARGE\bfseries{}childdoc example\par}
\vspace{1cm}
\ifchilddoc
\ifchilddocmanual part\else chapter\fi:
`\childdocname' of `\childdocjob'\par
\else
main document: `\childdocjob'\par
\fi
version: \version\par
\end{center}
\newpage
%    \end{macrocode}

% Manually include selected file,
% otherwise process as usual:
%    \begin{macrocode}
\ifchilddocmanual
\section*{part `\childdocname'}
\input{\childdocname}
\else
%    \end{macrocode}

% Include the two chapters:
%    \begin{macrocode}
\include{cdocsch1}
\include{cdocsch2}
%    \end{macrocode}

% Include the two parts unless only chapters should be displayed:
%    \begin{macrocode}
\ifchilddoc\else
\section{part three}
\input{cdocspt3}
\section{part four}
\input{cdocspt4}
\fi
%    \end{macrocode}

% Process as usual until here:
%    \begin{macrocode}
\fi
%    \end{macrocode}

% End of document body:
%    \begin{macrocode}
\end{document}
%    \end{macrocode}
%\iffalse
%</samplemain>
%\fi
%
% %%%%%%%%%%%%%%%%%%%%%%%%%%%%%%%%%%%%%%
% \paragraph{Chapter Include Files.}
%
% The include files are called |cdocsch1.tex| and |cdocsch2.tex|.
%
%\iffalse
%<*samplechap1|samplechap2>
%\fi

% Optional override for |\version| flag:
%    \begin{macrocode}
%%\providecommand{\version}{final}
%    \end{macrocode}

% Include the main document:
%    \begin{macrocode}
\input{childdoc.def}
\childdocof{cdocsamp}
%    \end{macrocode}

%\iffalse
%</samplechap1|samplechap2>
%\fi
%
%\iffalse
%<*samplechap1>
%\fi
% Some text for chapter 1:
%    \begin{macrocode}
\section{one}
some text in chapter one
%    \end{macrocode}

%\iffalse
%</samplechap1>
%\fi
% Some text for chapter 2:
%\iffalse
%<*samplechap2>
%\fi
%    \begin{macrocode}
\section{two}
more text in chapter two
%    \end{macrocode}

%\iffalse
%</samplechap2>
%\fi
%
% %%%%%%%%%%%%%%%%%%%%%%%%%%%%%%%%%%%%%%
% \paragraph{Part Include Files.}
%
% The include files are called |cdocspt3.tex| and |cdocspt4.tex|.
%
%\iffalse
%<*samplepart3|samplepart4>
%\fi

% Optional override for |\version| flag:
%    \begin{macrocode}
%%\providecommand{\version}{final}
%    \end{macrocode}

% Include the main document:
%    \begin{macrocode}
\input{childdoc.def}
\childdocby{cdocsamp}
%    \end{macrocode}

%\iffalse
%</samplepart3|samplepart4>
%\fi
%
%\iffalse
%<*samplepart3>
%\fi
% Some text for part 3:
%    \begin{macrocode}
some text in part three
%    \end{macrocode}

%\iffalse
%</samplepart3>
%\fi
% Some text for part 4:
%\iffalse
%<*samplepart4>
%\fi
%    \begin{macrocode}
more text in part four
%    \end{macrocode}

%\iffalse
%</samplepart4>
%\fi
%
% %%%%%%%%%%%%%%%%%%%%%%%%%%%%%%%%%%%%%%
% \paragraph{Forwarding for a Complete Draft.}
%
% The following forwarding file |cdocsdrf.tex|
% compiles the main document in draft mode:
%\iffalse
%<*sampledraft>
%\fi
%    \begin{macrocode}
\def\version{draft}
\input{childdoc.def}
\childdocforward{cdocsamp}
%    \end{macrocode}

%\iffalse
%</sampledraft>
%\fi
%
% %%%%%%%%%%%%%%%%%%%%%%%%%%%%%%%%%%%%%%
% \paragraph{Forwarding for Final Version of the Chapters.}
%
% The following forwarding files |cdocsfn1.tex| and |cdocsfn2.tex|
% (with identical content)
% compile the final versions of the child documents
% |cdocsch1.tex| and |cdocsch2.tex|, respectively:
%\iffalse
%<*samplefinal>
%\fi
%    \begin{macrocode}
\def\version{final}
\input{childdoc.def}
\childdocforwardprefix[cdocsamp]{cdocsfn}{cdocsch}
%    \end{macrocode}

%\iffalse
%</samplefinal>
%\fi
%
% %%%%%%%%%%%%%%%%%%%%%%%%%%%%%%%%%%%%%%
% \paragraph{Command Line Processing.}
%
% The following three command lines generate the output files
% |cdocscld|, |cdocscl1| and |cdocscl2|
% which should be identical to
% |cdocsdrf|, |cdocsch1| and |cdocsfn2|, respectively:
% \begin{center}
% \begin{tabular}{l}
% |latex -jobname cdocscld \|\\
% |  "\def\version{draft}\input{childdoc.def}\childdocforward{cdocsamp}"|\\
% |latex -jobname cdocscl1 \|\\
% |  "\input{childdoc.def}\childdocforward[cdocsamp]{cdocsch1}"|\\
% |latex -jobname cdocscl2 \|\\
% |  "\def\version{final}\input{childdoc.def}\childdocforward{cdocsch2}"|
% \end{tabular}
% \end{center}
% Note that the trailing backslash on each first line
% merely continues the input to the second line
% (for convenient cut ant paste).
% Furthermore, the command |latex| can be replaced by any
% of its alternative versions such as |pdflatex|.
%
% %%%%%%%%%%%%%%%%%%%%%%%%%%%%%%%%%%%%%%%%%%%%%%%%%%%%%%%%%%%%%%%%%%%%%%%%%%%%%%
% %%%%%%%%%%%%%%%%%%%%%%%%%%%%%%%%%%%%%%%%%%%%%%%%%%%%%%%%%%%%%%%%%%%%%%%%%%%%%%
% \section{Implementation}
%\iffalse
%<*package>
%\fi
%
% This section describes the definitions file |childdoc.def|.

% The definitions cannot be loaded using |\usepackage| or |\RequirePackage|
% which has a mechanism to prevent loading a style file more than once.
% When loading the definitions by means of |\input|
% multiple instances have to be prevented manually:
%\iffalse
%This code needs to be before the `\ProvidesFile' directive
%which is defined at the beginning of this file.
%Therefore it is also placed there and commented out here.
%</package>
%<*discard>
%\fi
%    \begin{macrocode}
\ifdefined\childdocmain\endinput\fi
%    \end{macrocode}
%\iffalse
%</discard>
%<*package>
%\fi
%
% \macro{\ifchilddoc}
% \macro{\ifchilddocmanual}
% The conditional |\ifchilddoc| tells whether a
% child (true) or main (false) document is being compiled.
% The conditional |\ifchilddocmanual| tells whether
% the |\includeonly| mechanism is used (false) or
% the selection of child files must be performed manually (true).
% The definitions initialise to false:
%    \begin{macrocode}
\newif\ifchilddoc
\newif\ifchilddocmanual
%    \end{macrocode}

% \macro{\childdocname}
% \macro{\childdocjob}
% The macro |\childdocname| stores the name of the main document
% to be compiled. The macro |\childdocjob| stores the name of
% the document on which the \LaTeX{} compiler was originally invoked.
% The content of |\jobname| cannot be compared
% to filenames specified in the source due to different catcodes.
% The following code rescans |\jobname|, stores the result
% in |\childdocname| and saves a copy in |\childdocjob|:
%    \begin{macrocode}
\edef\childdocname{\scantokens\expandafter{\jobname\noexpand}}
\let\childdocjob\childdocname
%    \end{macrocode}

% \macro{\childdocdisable}
% The macro |\childdocdisable| prevents the main file
% from being processed more than once.
% At this stage, the main document command |\childdocmain|
% is assumed to be called once again where it should do nothing.
% Any subsequent call to it should prevent
% a secondary processing of the main document
% It overwrites the forwarding commands
% |\childdocof| and |\childdocforward|
% with empty macros to prevent further inclusions of the main document:
%    \begin{macrocode}
\newcommand{\childdocdisable}
{
  \renewcommand{\childdocmain}[1]{\renewcommand{\childdocmain}[1]{\endinput}}
  \renewcommand{\childdocof}[1]{}
  \renewcommand{\childdocby}[2][]{}
  \renewcommand{\childdocforward}[2][]{}
  \renewcommand{\childdocdisable}{}
}
%    \end{macrocode}

% \macro{\childdocmain}
% The macro |\childdocmain| is to be called at the top of the main file
% with nothing or the main filename (without extension) as argument.
% First, it breaks loops.
% If the argument is not empty and does not match |\childdocname|
% (which is set by the first inclusion of |childdoc.def|),
% |\ifchilddoc| is set to true, |\includeonly| is applied to the child file
% and |\jobname| is set to the main file
% (for proper handling of |.aux| files):
%    \begin{macrocode}
\newcommand{\childdocmain}[1]
{
  \childdocdisable\childdocmain{}
  \if?#1?\else
    \begingroup
      \def\childdoctmp{#1}
      \ifx\childdoctmp\childdocname
        \def\childdoctmp{}
      \else
        \def\childdoctmp
        {
          \childdoctrue
          \includeonly{\childdocname}
          \def\childdocjob{#1}
          \def\jobname{#1}
        }
      \fi
      \expandafter
    \endgroup
    \childdoctmp
  \fi
}
%    \end{macrocode}

% \macro{\childdocof}
% The command |\childdocof| redirects
% compilation to the main file |#1|.
%    \begin{macrocode}
\newcommand{\childdocof}[1]
{
  \childdocdisable
  \childdoctrue
  \includeonly{\childdocname}
  \def\jobname{#1}
  \def\childdocjob{#1}
  \input{#1}
}
%    \end{macrocode}

% \macro{\childdocby}
% The command |\childdocby| ....
%    \begin{macrocode}
\newcommand{\childdocby}[2][]
{
  \childdocdisable
  \childdoctrue
  \childdocmanualtrue
  \if?#1?\else
    \def\jobname{#2}
  \fi
  \def\childdocjob{#2}
  \input{#2}
  \endinput
}
%    \end{macrocode}

% \macro{\childdocforward}
% The command |\childdocforward| redirects
% compilation to the main file or
% (if the optional argument is given) a child file.
% Parameters are set as if the main file
% or a child file starting with |\childdocof| was compiled.
% Then compilation is handed over to the main file:
%    \begin{macrocode}
\newcommand{\childdocforward}[2][]
{
  \begingroup
    \if?#1?
      \def\childdoctmp
      {
        \def\childdocname{#2}
        \def\childdocjob{#2}
        \def\jobname{#2}
        \input{#2}
        \endinput
      }
    \else
      \def\childdoctmp
      {
        \childdocdisable
        \def\childdocname{#2}
        \childdoctrue
        \includeonly{#2}
        \def\childdocjob{#1}
        \def\jobname{#1}
        \input{#1}
        \endinput
      }
    \fi
    \expandafter
  \endgroup
  \childdoctmp
}
%    \end{macrocode}

% \macro{\childdocforwardprefix}
% The command |\childdocforwardprefix| redirects
% compilation to the main or a child file by means of a pattern.
% The prefix |#1| in the current filename is replaced by |#2|
% and the suffix of the current filename is kept
% (it is assumed that the filename does not contain the substring `|~~~|'
% which is used as a delimiter).
% Compilation is handed over to the new file by |\childdocforward|:
%    \begin{macrocode}
\newcommand{\childdocforwardprefix}[3][]
{
  \begingroup
    \def\childdocextract #2##1~~~{\def\childdoctmp{\childdocforward[#1]{#3##1}}}
    \expandafter\childdocextract\childdocname~~~
    \expandafter
  \endgroup
  \childdoctmp
}
%    \end{macrocode}

% \macro{\childdoc}
% The deprecated macro |\childdoc| is a legacy version of |\childdocmain|:
%    \begin{macrocode}
\newcommand{\childdoc}{\childdocmain}
%    \end{macrocode}

% \macro{\childdocredirect}
% The deprecated macro |\childdocredirect| is a legacy version
% of |\childdocforward| and |\childdocforwardprefix|:
%    \begin{macrocode}
\newcommand{\childdocredirect}[2][]
{
  \begingroup
    \if?#1?
      \def\childdoctmp{\childdocforward{#2}}
    \else
      \def\childdoctmp{\childdocforwardprefix{#1}{#2}}
    \fi
    \expandafter
  \endgroup
  \childdoctmp
}
%    \end{macrocode}

%\iffalse
%</package>
%\fi
%
\endinput
\childdocforward[cdocsamp]{cdocsch1}"|\\
% |latex -jobname cdocscl2 \|\\
% |  "\def\version{final}% \iffalse
%
% childdoc.dtx Copyright (C) 2017-2018 Niklas Beisert
%
% This work may be distributed and/or modified under the
% conditions of the LaTeX Project Public License, either version 1.3
% of this license or (at your option) any later version.
% The latest version of this license is in
%   http://www.latex-project.org/lppl.txt
% and version 1.3 or later is part of all distributions of LaTeX
% version 2005/12/01 or later.
%
% This work has the LPPL maintenance status `maintained'.
%
% The Current Maintainer of this work is Niklas Beisert.
%
% This work consists of the files childdoc.dtx and childdoc.ins
% and the derived files childdoc.def and cdocsamp.tex with
% cdocsch1.tex, cdocsch2.tex, cdocsdrf.tex, cdocsfn1.tex, cdocsfn2.tex.
%
%<package>\ifdefined\childdocmain\endinput\fi
%<package>\ProvidesFile{childdoc.def}[2018/12/30 v2.0 child document driver]
%<samplemain>\ProvidesFile{cdocsamp.tex}[2018/12/30 v2.0 sample for childdoc]
%<*driver>
%\ProvidesFile{childdoc.drv}[2018/12/30 v2.0 childdoc reference manual file]
\PassOptionsToClass{10pt,a4paper}{article}
\documentclass{ltxdoc}

\usepackage[margin=35mm]{geometry}
\usepackage{hyperref}
\usepackage{hyperxmp}
\usepackage[usenames]{color}

\hypersetup{colorlinks=true}
\hypersetup{pdfstartview=FitH}
\hypersetup{pdfpagemode=UseNone}
\hypersetup{pdfsource={}}
\hypersetup{pdflang={en-UK}}
\hypersetup{pdfcopyright={Copyright 2017-2018 Niklas Beisert.
  This work may be distributed and/or modified under the
  conditions of the LaTeX Project Public License, either version 1.3
  of this license or (at your option) any later version.}}
\hypersetup{pdflicenseurl={http://www.latex-project.org/lppl.txt}}
\hypersetup{pdfcontactaddress={ETH Zurich, ITP, HIT K,
  Wolfgang-Pauli-Strasse 27}}
\hypersetup{pdfcontactpostcode={8093}}
\hypersetup{pdfcontactcity={Zurich}}
\hypersetup{pdfcontactcountry={Switzerland}}
\hypersetup{pdfcontactemail={nbeisert@itp.phys.ethz.ch}}
\hypersetup{pdfcontacturl={http://people.phys.ethz.ch/\xmptilde nbeisert/}}

\newcommand{\secref}[1]{\hyperref[#1]{section \ref*{#1}}}

\parskip1ex
\parindent0pt
\let\olditemize\itemize
\def\itemize{\olditemize\parskip0pt}

\begin{document}

\title{The \textsf{childdoc} Package}
\hypersetup{pdftitle={The childdoc Package}}
\author{Niklas Beisert\\[2ex]
  Institut f\"ur Theoretische Physik\\
  Eidgen\"ossische Technische Hochschule Z\"urich\\
  Wolfgang-Pauli-Strasse 27, 8093 Z\"urich, Switzerland\\[1ex]
  \href{mailto:nbeisert@itp.phys.ethz.ch}
  {\texttt{nbeisert@itp.phys.ethz.ch}}}
\hypersetup{pdfauthor={Niklas Beisert}}
\hypersetup{pdfsubject={Manual for the LaTeX2e Package childdoc}}
\date{30 December 2018, \textsf{v2.0}}
\maketitle

\begin{abstract}\noindent
\textsf{childdoc} is a \LaTeXe{} package
that enables the direct compilation
of document sections included by |\include|
to individual files.
\end{abstract}

\begingroup
\parskip0ex
\tableofcontents
\endgroup

%%%%%%%%%%%%%%%%%%%%%%%%%%%%%%%%%%%%%%%%%%%%%%%%%%%%%%%%%%%%%%%%%%%%%%%%%%%%%%%%
%%%%%%%%%%%%%%%%%%%%%%%%%%%%%%%%%%%%%%%%%%%%%%%%%%%%%%%%%%%%%%%%%%%%%%%%%%%%%%%%
\section{Introduction}

\LaTeX{} provides a mechanism to structure a large document (such as a book)
into a main file and several child files (containing the chapters)
using the |\include| command.
This mechanism is beneficial for documents
which span hundreds of pages in order to
make the source file(s) more manageable.
Moreover, compilation can be restricted to
selected child files by means of the |\includeonly| command.
The latter feature can be used to reduce the compilation time while editing
(this was significantly more useful in the earlier days of \LaTeX{})
or to generate a smaller document which is easier to navigate.
Another application of |\includeonly| is to generate
documents consisting of selected parts of the complete document.

However, there are a few drawbacks of the plain |\include| mechanism:
\begin{itemize}
\item
The child files cannot be compiled on their own,
they can only be compiled via the main file.
A naive editing environment
(such as a text editor with an option
to have the current file processed by \LaTeX)
may require one to switch to the main file before compiling;
attempting to compile the child file produces errors.
\item
The main file must be modified (each time)
to adjust the |\includeonly| command
to the present needs. This easily leaves the main file in a messy state.
\item
The generated document will always carry the filename
of the main document. This is inconvenient if
several child files are to be compiled and
to be kept for distribution.
\end{itemize}

The present package provides a simple interface
to make child files individually compilable by \LaTeX{}.
Compiling a child file then has the same effect as compiling
the main file with an |\includeonly| command
to select the appropriate child.
Moreover the generated document will carry the name of the child
rather than the main file.
This resolves all three above issues.

This feature is meant to make the editing of books,
thesis documents and lecture notes somewhat more convenient.
However, the package can also be used efficiently for
composing a series of documents (such as exercise sheets)
which are typically distributed individually.
It then assists the author in generating the individual documents
(potentially in different versions)
as well as a document containing the collected series.
Another application is in developing style files
or other kinds of included material
where compilation of the style file could redirect
to a sample or test file.

%%%%%%%%%%%%%%%%%%%%%%%%%%%%%%%%%%%%%%%%%%%%%%%%%%%%%%%%%%%%%%%%%%%%%%%%%%%%%%%%
%%%%%%%%%%%%%%%%%%%%%%%%%%%%%%%%%%%%%%%%%%%%%%%%%%%%%%%%%%%%%%%%%%%%%%%%%%%%%%%%
\section{Usage}

First of all, the package \textsf{childdoc} is \emph{not} a standard
\LaTeXe{} |.sty| style file! Therefore it needs to be invoked in
a non-standard way.

%%%%%%%%%%%%%%%%%%%%%%%%%%%%%%%%%%%%%%%%%%%%%%%%%%%%%%%%%%%%%%%%%%%%%%%%%%%%%%%%
\subsection{Included Files}
\label{sec:include}

%%%%%%%%%%%%%%%%%%%%%%%%%%%%%%%%%%%%%%%%
\DescribeMacro{\childdocmain}
To use the package, add the commands
\begin{center}
\begin{tabular}{l}
|\input{childdoc.def}|\\
|\childdocmain{}|\\
\end{tabular}
\end{center}
at the very top of the main \LaTeX{} file,
in particular \emph{before} the |\documentclass| statement!
The argument of |\childdocmain| should be left empty
(but it must be present).

%%%%%%%%%%%%%%%%%%%%%%%%%%%%%%%%%%%%%%%%
\DescribeMacro{\childdocof}
Furthermore, add the commands
\begin{center}
\begin{tabular}{l}
|\input{childdoc.def}|\\
|\childdocof{|\textit{main}|}|\\
\end{tabular}
\end{center}
at the top of every child file \textit{child}
which is included by |\include{|\textit{child}|}|
from within the main file
(or at least for those files to be compiled individually).
The argument \textit{main} must be the filename of the main file.

There are a couple of
considerations in setting up the main and child documents:

%%%%%%%%%%%%%%%%%%%%%%%%%%%%%%%%%%%%%%%%
\paragraph{Restrictions.}

Please note the following restrictions:
\begin{itemize}
\item
|\childdocmain| must be called with one argument \textit{main}
to ensure compatibility with earlier version of the package.
It must either be empty (|\childdocmain{}|)
or precisely match the filename of the main file in which it is specified.
See \secref{sec:detection} for further information.
\item
The filename \textit{main} must be specified without the |.tex| extension.
\item
The filename \textit{main} is case sensitive
(even in case-insensitive file systems)
due to internal string comparison.
\item
The argument \textit{main} should be fully expanded, it cannot be a macro.
\item
Subdirectories and special characters should be avoided in filenames.
\item
The command |\childdocmain{|\textit{main}|}| must be followed by a whitespace.
It should not be followed immediately by another command
or by a comment mark `|%|'.
This is because the \TeX{} parser reads the token immediately following
the argument of |\childdocmain| and puts it
at the beginning of every child section;
however, a white\-space is ignored.
\end{itemize}

%%%%%%%%%%%%%%%%%%%%%%%%%%%%%%%%%%%%%%%%
\paragraph{Content of Main File.}

It is advisable to place all content in the child files included by |\include|.
Any output contained in the main file will appear in all child documents
unless suppressed manually;
it cannot be suppressed automatically by the |\includeonly| directive
and thus should normally be avoided.
A method to include some content in the main file
by means of conditional processing is described in \secref{sec:conditional}.

%%%%%%%%%%%%%%%%%%%%%%%%%%%%%%%%%%%%%%%%
\paragraph{Page Numbering.}

When only a part of the document is compiled,
the appropriate numbering of pages
(as well as other status parameters)
is determined from the |.aux| files.
The latter contain information from previous passes.
However this information needs to propagate through
all intermediate child documents.
Therefore the page numbering in child documents may well
be inconsistent until the complete document is compiled at least once.

A useful (if unconventional) way to always ensure a consistent
page numbering is to restart the numbering in each child document
and denote the pages by `\textit{child}|.|\textit{page}'
where \textit{child} represents the chapter/section number of the child file.
This can be achieved by the command
|\numberwithin{page}{|\textit{child}|}|
of the \textsf{amsmath} package
where \textit{child} can be |chapter| or |section|
depending on the chosen structuring.
Alternatively, one can modify the macro |\thepage| appropriately
and reset the counter |page| at the start of each child file.

%%%%%%%%%%%%%%%%%%%%%%%%%%%%%%%%%%%%%%%%%%%%%%%%%%%%%%%%%%%%%%%%%%%%%%%%%%%%%%%%
\subsection{Conditional Processing}
\label{sec:conditional}

The package provides a mechanism to compile different versions
of a document. To customise the versions further some conditional processing
can come in handy to distinguish which version is being compiled.
The package provides two macros to describe the compilation context:

%%%%%%%%%%%%%%%%%%%%%%%%%%%%%%%%%%%%%%%%
\DescribeMacro{\ifchilddoc}
The conditional |\ifchilddoc| distinguishes between the compilation of
child documents and the main document:
%
\begin{center}
|\ifchilddoc |\textit{child-code}| |[|\||else |\textit{main-code}]| \||fi|
\end{center}

%%%%%%%%%%%%%%%%%%%%%%%%%%%%%%%%%%%%%%%%
\DescribeMacro{\childdocname}
\DescribeMacro{\childdocjob}
The macro |\childdocname| contains the filename (without extension)
of the main or child file being processed.
Note that |\childdocjob| will always contain the name of the main file.

%%%%%%%%%%%%%%%%%%%%%%%%%%%%%%%%%%%%%%%%
\paragraph{Title Page.}

Conditional processing can be used to include a title or banner page
in the main document when proper precautions are taken.
Importantly, the code in the main file should ensure that the page counter
(as well as other status parameters which are stored in the |.aux| files)
takes the same value after the conditional processing.
Otherwise the page numbers may take divergent values
depending on which part is compiled.

For example, a title page could be declared by:
%
\begin{center}
\begin{tabular}{l}
|\ifchilddoc\||else|\\
|\addtocounter{page}{-1}|\\
\textit{code for title page}\\
|\newpage|\\
|\||fi|
\end{tabular}
\end{center}
%
A banner page for the child documents can be generated by:
%
\begin{center}
\begin{tabular}{l}
|\ifchilddoc|\\
|\addtocounter{page}{-1}|\\
\textit{code for banner page}\\
|\newpage|\\
|\||fi|
\end{tabular}
\end{center}
%
Here one could write a message such as:
\begin{center}
|This is the part \childdocname{} of \childdocjob{}.|
\end{center}

%%%%%%%%%%%%%%%%%%%%%%%%%%%%%%%%%%%%%%%%%%%%%%%%%%%%%%%%%%%%%%%%%%%%%%%%%%%%%%%%
\subsection{Flags}
\label{sec:flags}

The package makes it easy to generate different versions
of the main or child documents.
To this end compilation flags can be defined
and assigned different default values.
They will be particularly useful in conjunction
with the forwarding mechanism described in \secref{sec:forward}.

For example, it may be useful to have a flag |\version|
which can be set to |draft| or |final|.
The document source will contain some conditional code
depending on the value of |\version|.
Suppose further, the flag should default to |final| for the main file
and to |draft| for child files
which is a natural assignment for editing the document.
This is achieved by placing the following code
in the preamble of the main document
(below the |\childdocmain| directive):
%
\begin{center}
\begin{tabular}{l}
|\ifchilddoc|\\
|\providecommand{\version}{draft}|\\
|\||else|\\
|\providecommand{\version}{final}|\\
|\||fi|
\end{tabular}
\end{center}
%
The definition by |\providecommand| makes sure
that previous definitions are not overwritten.
Further statements |\providecommand{\version}{...}|
can thus be added before the above code to override it.

For the main file, one might add a line
(between |\childdocmain| and the above block)
%
\begin{center}
|%\ifchilddoc\||else\providecommand{\version}{draft}\||fi|
\end{center}
%
which can be uncommented to produce a draft version.
Likewise one can add a line to the very top of a child file
(above the |\childdocof{|\textit{main}|}| directive)
%
\begin{center}
|%\providecommand{\version}{final}|
\end{center}
%
which can be uncommented to produce the final version of this child document.

%%%%%%%%%%%%%%%%%%%%%%%%%%%%%%%%%%%%%%%%%%%%%%%%%%%%%%%%%%%%%%%%%%%%%%%%%%%%%%%%
\subsection{Forwarding}
\label{sec:forward}

Different versions of the main or child documents
using compilation flags as described in \secref{sec:flags}
can be (permanently) stored in different files
for convenient compilation, viewing and distribution.
To this end, the package defines a command
to pass on compilation to a different file:

%%%%%%%%%%%%%%%%%%%%%%%%%%%%%%%%%%%%%%%%
\DescribeMacro{\childdocforward}
The command |\childdocforward| redirects processing to
another source file:
%
\begin{center}
\begin{tabular}{l}
|\input{childdoc.def}|\\
|\childdocforward[|\textit{main}|]{|\textit{dest}|}|\\
\end{tabular}
\end{center}
%
The argument \textit{dest} is the destination file
(without extension).
It should be the main file or one of the child files.
Note that further \textsf{childdoc} directives
such as |\childdocof| and |\childdocforward|
in the indicated file will be processed in this form.
The optional argument \textit{main}
passes on directly to the main file \textit{main}
while pretending to compile the child \textit{dest}.
This form behaves as if \textit{dest}
issues |\childdocof{|\textit{main}|}| right away,
and no further \textsf{childdoc} directives will be processed.

%%%%%%%%%%%%%%%%%%%%%%%%%%%%%%%%%%%%%%%%
\DescribeMacro{\...prefix}
In the alternative form |\childdocforwardprefix|,
%
\begin{center}
\begin{tabular}{l}
|\input{childdoc.def}|\\
|\childdocforwardprefix[|\textit{main}|]{|\textit{prefix}|}{|\textit{dest}|}|
\end{tabular}
\end{center}
%
the destination file is determined by a pattern
depending on the current file:
To make this work, the current file must be called
`{\textit{prefix}\hspace{0.2em}\textit{suffix}}'
with \textit{prefix} matching precisely the argument.
Processing is then passed on to the file
`{\textit{dest}\hspace{0.2em}\textit{suffix}}'.
Surely, the same effect is achieved by
directly specifying the
argument `{\textit{dest}\hspace{0.2em}\textit{suffix}}'
in the first form.
However, that requires to set up a different file
for each child. With the alternative form of the command
all these files can have exactly the same content
which simplifies setting them up and maintaining them.

For example, the following file |draft.tex|
with a compilation flag |\version| as described in \secref{sec:flags}
compiles the main document as a draft:
%
\begin{center}
\begin{tabular}{l}
|\def\version{draft}|\\
|\input{childdoc.def}|\\
|\childdocforward{|\textit{main}|}|
\end{tabular}
\end{center}
%
Likewise, the following files |final|\textit{nn}|.tex|
compile the final version of the child document
|child|\textit{nn}|.tex|:
%
\begin{center}
\begin{tabular}{l}
|\def\version{final}|\\
|\input{childdoc.def}|\\
|\childdocforwardprefix{final}{child}|
\end{tabular}
\end{center}
%

Note that when several versions of a main file and/or of each child file
are to be generated, it may be convenient to set up a |Makefile| or
shell script to automatise the process.

%%%%%%%%%%%%%%%%%%%%%%%%%%%%%%%%%%%%%%%%%%%%%%%%%%%%%%%%%%%%%%%%%%%%%%%%%%%%%%%%
\subsection{Command Line Processing}
\label{sec:commandline}

The effect of redirection files can also be achieved by invoking
the \LaTeX{} compiler with a more elaborate command line.
Most conveniently this should be done as part
of a shell script or a |Makefile|.

When using \textsf{childdoc} in the main file, the following
command lines effectively perform a redirection
(note that depending on the shell being used,
backslashes may have to be doubled: `|\|' $\to$ `|\\|'):
%
\begin{center}
|... -jobname "|\textit{target}|" |\\|"|[\textit{flags}]%
|\input{childdoc.def}\childdocforward[|\textit{main}|]{|\textit{dest}|}"|
\end{center}
%
Here \textit{target} is the name of the output file,
\textit{main} is the name of the main file
and \textit{dest} is the name of the main or child file to be processed
(all filenames without extensions).
The optional argument \textit{main} can be omitted
if \textit{main} matches \textit{dest}.
Optionally, compilation \textit{flags} can be defined via |\def| commands.
This command line makes the \TeX{} engine believe
it is compiling the file \textit{target}
whose content is specified as the latter parameter.
The provided code then forwards the processing to
\textit{main} or \textit{dest} as described in \secref{sec:forward}.

%%%%%%%%%%%%%%%%%%%%%%%%%%%%%%%%%%%%%%%%%%%%%%%%%%%%%%%%%%%%%%%%%%%%%%%%%%%%%%%%
\subsection{Include by Input}
\label{sec:input}

Including child documents by |\include| has some restrictions by design.
Most notably, the content of a child document always occupies
its own set of pages; pages cannot be shared between child documents.
Usually, this behaviour makes perfect sense
because each child document contain an essential part of the document.
However, in some situations it may be desirable to compose
a document from a collection of parts
without having mandatory page breaks between then.
For this case, the package
provides a mechanism to include parts
by |\input| which can also be processed individually.
However, by construction this mechanism
requires manual handling of the content to be output.

%%%%%%%%%%%%%%%%%%%%%%%%%%%%%%%%%%%%%%%%
\DescribeMacro{\ifchilddocmanual}
The main file should be prepared as usual, see \secref{sec:include}.
However, the document body must make a distinction
between processing of an individual part and of the main document, e.g.:
%
\begin{center}
\begin{tabular}{l}
|\ifchilddocmanual|\\
|\input{\childdocname}|\\
|\||else|\\
\textit{document body with }|\input{|\textit{part}|}|\\
|\||fi|
\end{tabular}
\end{center}
%
The conditional |\ifchilddocmanual| is true whenever
a part to be included by |\input| is being compiled,
and the name of the part is stored in |\childdocname|.

%%%%%%%%%%%%%%%%%%%%%%%%%%%%%%%%%%%%%%%%
\DescribeMacro{\childdocby}
Each part to be included by |\input| should start with:
%
\begin{center}
\begin{tabular}{l}
|\input{childdoc.def}|\\
|\childdocby{|\textit{main}|}|\\
\end{tabular}
\end{center}
%
The directive |\childdocby| is similar to |\childdocof|
described in \secref{sec:include},
but the subsequent selection of content must be done manually.
To that end, both |\ifchilddoc| and |\ifchilddocmanual|
will be true upon processing of a part,
and the name of the part is stored in |\childdocname|.
Note that |\jobname| will be set to the filename of the current part
so that each part receives an individual |.aux| file
that does not interfere with the |.aux| file(s) of the main document.
This behaviour can be altered by the alternative form
|\childdocby[*]{|\textit{main}|}| (with a non-empty optional argument)
which uses the |.aux| file of the main document
by setting |\jobname| to \textit{main}.

%%%%%%%%%%%%%%%%%%%%%%%%%%%%%%%%%%%%%%%%%%%%%%%%%%%%%%%%%%%%%%%%%%%%%%%%%%%%%%%%
\subsection{Driver Development}
\label{sec:driver}

The \textsf{childdoc} mechanism can also be use for the development
of definition files such as \LaTeX{} styles or classes.
This case differs from the above setup with multiple parts
included by |\include| in that no |\includeonly| should be invoked.
This can be achieved by starting the include file
(before |\ProvidesPackage|) with:
%
\begin{center}
\begin{tabular}{l}
|\input{childdoc.def}|\\
|\childdocforward{|\textit{main}|}|\\
\end{tabular}
\end{center}
%
or alternatively with:
%
\begin{center}
\begin{tabular}{l}
|\input{childdoc.def}|\\
|\childdocby{|\textit{main}|}|\\
\end{tabular}
\end{center}
%
Both forms have slightly different effects as described above.
The main file is prepared as usual, see \secref{sec:include}.

%%%%%%%%%%%%%%%%%%%%%%%%%%%%%%%%%%%%%%%%%%%%%%%%%%%%%%%%%%%%%%%%%%%%%%%%%%%%%%%%
\subsection{Legacy Detection}
\label{sec:detection}

The directive |\childdocmain| in the main file can detect
whether the complete document or merely a child is to be compiled
even without using the directive |\childdocof|.
This method is deprecated because it is less robust
and there is no compelling reason to use it;
it is merely provided for backward compatibility
and it may be removed in future versions.

If the detection mechanism is to be used,
it is mandatory to correctly specify
the filename of the main file as the argument of |\childdocmain|:
%
\begin{center}
\begin{tabular}{l}
|\input{childdoc.def}|\\
|\childdocmain{|\textit{main}|}|\\
\end{tabular}
\end{center}
%
If |\jobname| does not match the argument \textit{main} of |\childdocmain|,
it is assumed that |\jobname| points to the child file to be compiled.
When using |\childdocmain| with the main file specified as argument,
it suffices to start a child file
with just |\input{|\textit{main}|}|
without loading of the package and using |\childdocof|.
If instead all processing is done
with the appropriate \textsf{childdoc} directives,
the argument of \textit{main} of |\childdocmain| can be empty.

An alternative version of the command line processing described
in \secref{sec:commandline} using the detection mechanism reads:
%
\begin{center}
|... -jobname "|\textit{target}|" "|[\textit{flags}]%
[|\def\jobname{|\textit{dest}|}|]|\input{|\textit{main}|}"|
\end{center}

%%%%%%%%%%%%%%%%%%%%%%%%%%%%%%%%%%%%%%%%%%%%%%%%%%%%%%%%%%%%%%%%%%%%%%%%%%%%%%%%
\subsection{Manual Code}
\label{sec:manual}

In case one cannot be certain whether the definitions file |childdoc.def|
is installed on the target \TeX{} distribution
and one prefers not to ship it,
it is conceivable to paste a few relevant commands into the sources.

To that end, drop all statements |\input{childdoc.def}|
and perform the replacements as outlined below.
Instead of |\childdocmain{|\textit{main}|}| add the following code
to the top of the main file:
%
\begin{center}
\begin{tabular}{l}
|\||ifdefined\childdocname\endinput\||fi\newif\ifchilddoc|\\
|\edef\childdocname{\scantokens\expandafter{\jobname\noexpand}}|\\
|\def\childdocmain{|\textit{main}|}\||ifx\childdocmain\childdocname\||else|\\
|\childdoctrue\includeonly{\childdocname}\let\jobname\childdocmain\||fi|\\
\end{tabular}
\end{center}
%
Instead of |\childdocof{|\textit{main}|}| just include the main file
at the top of each child file:
%
\begin{center}
|\input{|\textit{main}|}|
\end{center}
%
A simple redirection |\childdocforward{|\textit{dest}|}| is achieved by:
%
\begin{center}
|\def\jobname{|\textit{dest}|}\input{\jobname}|
\end{center}
%
The redirection with prefix
|\childdocforwardprefix[|\textit{prefix}|]{|\textit{dest}|}|
is accomplished by:
%
\begin{center}
\begin{tabular}{l}
|{\edef\jobname{\scantokens\expandafter{\jobname\noexpand}}|\\
|\def\redirectjob |\textit{prefix}|#1~~~{\gdef\jobname{|\textit{dest}|#1}}|\\
|\expandafter\redirectjob\jobname~~~}\input{\jobname}|
\end{tabular}
\end{center}

In an alternative approach,
child documents can be compiled by a specific command line
without additional code or specific definitions:
%
\begin{center}
|... -jobname "|\textit{target}|" "|[\textit{flags}]%
|\includeonly{|\textit{dest}|}\input{|\textit{main}|}"|
\end{center}
%

%%%%%%%%%%%%%%%%%%%%%%%%%%%%%%%%%%%%%%%%%%%%%%%%%%%%%%%%%%%%%%%%%%%%%%%%%%%%%%%%
%%%%%%%%%%%%%%%%%%%%%%%%%%%%%%%%%%%%%%%%%%%%%%%%%%%%%%%%%%%%%%%%%%%%%%%%%%%%%%%%
\section{Information}

%%%%%%%%%%%%%%%%%%%%%%%%%%%%%%%%%%%%%%%%%%%%%%%%%%%%%%%%%%%%%%%%%%%%%%%%%%%%%%%%
\subsection{Copyright}

Copyright \copyright{} 2017--2018 Niklas Beisert

This work may be distributed and/or modified under the
conditions of the \LaTeX{} Project Public License, either version 1.3
of this license or (at your option) any later version.
The latest version of this license is in
  \url{http://www.latex-project.org/lppl.txt}
and version 1.3 or later is part of all distributions of \LaTeX{}
version 2005/12/01 or later.

This work has the LPPL maintenance status `maintained'.

The Current Maintainer of this work is Niklas Beisert.

This work consists of the files |README.txt|, |childdoc.ins| and |childdoc.dtx|
as well as the derived files |childdoc.def|, |cdocsamp.tex|
with |cdocsch1.tex|, |cdocsch2.tex|, |cdocspt3.tex|, |cdocspt4.tex|,
|cdocsdrf.tex|, |cdocsfn1.tex|, |cdocsfn2.tex|
as well as |childdoc.pdf|.

%%%%%%%%%%%%%%%%%%%%%%%%%%%%%%%%%%%%%%%%%%%%%%%%%%%%%%%%%%%%%%%%%%%%%%%%%%%%%%%%
\subsection{Files and Installation}

The package consists of the files:
%
\begin{center}
\begin{tabular}{ll}
    |README.txt|   & readme file \\
    |childdoc.ins| & installation file \\
    |childdoc.dtx| & source file \\
    |childdoc.def| & definition file \\
    |cdocsamp.tex| & sample main file \\
    |cdocsch1.tex| & sample include file \\
    |cdocsch2.tex| & sample include file \\
    |cdocspt3.tex| & sample part file \\
    |cdocspt4.tex| & sample part file \\
    |cdocsdrf.tex| & sample redirection file \\
    |cdocsfn1.tex| & sample redirection file \\
    |cdocsfn2.tex| & sample redirection file \\
    |childdoc.pdf| & manual
\end{tabular}
\end{center}
%
The distribution consists of the files
|README.txt|, |childdoc.ins| and |childdoc.dtx|.
%
\begin{itemize}
\item
Run (pdf)\LaTeX{} on |childdoc.dtx|
to compile the manual |childdoc.pdf| (this file).
\item
Run \LaTeX{} on |childdoc.ins| to create the definitions file |childdoc.def|
and the sample |cdocsamp.tex| with include files
|cdocsch1.tex|, |cdocsch2.tex|, |cdocspt3.tex|, |cdocspt4.tex|,
|cdocsdrf.tex|, |cdocsfn1.tex|, |cdocsfn2.tex|.
Then copy the file |childdoc.def| to an appropriate directory of your \LaTeX{}
distribution, e.g.\ \textit{texmf-root}|/tex/latex/childdoc|.
\end{itemize}

%%%%%%%%%%%%%%%%%%%%%%%%%%%%%%%%%%%%%%%%%%%%%%%%%%%%%%%%%%%%%%%%%%%%%%%%%%%%%%%%
\subsection{Related CTAN Packages}

There are several other packages which offer a similar functionality:
%
\begin{itemize}
\item
The packages
\href{http://ctan.org/pkg/docmute}{\textsf{docmute}},
\href{http://ctan.org/pkg/includex}{\textsf{includex}} and
\href{http://ctan.org/pkg/standalone}{\textsf{standalone}}
provide commands to include only the document body of
a child file thus allowing both files to be compiled individually.
\item
The packages \href{http://ctan.org/pkg/subdocs}{\textsf{subdocs}}
and \href{http://ctan.org/pkg/subfiles}{\textsf{subfiles}}
provide structures in which the main and child documents can be
encapsulated and allowing them to be compiled individually.
The inclusion mechanism is different from the conventional |\include|.
\item
The package \href{http://ctan.org/pkg/combine}{\textsf{combine}}
is an elaborate solution to combine several documents into one.
\end{itemize}
%
See also the CTAN topic \href{http://ctan.org/topic/subdocs}{\textsf{subdocs}}
for further related packages.
The present package differs from the above solutions in that
a document structure constructed with the conventional |\include| mechanism
just needs two extra commands at the top of every file
such that all constituent files can be compiled individually.

%%%%%%%%%%%%%%%%%%%%%%%%%%%%%%%%%%%%%%%%%%%%%%%%%%%%%%%%%%%%%%%%%%%%%%%%%%%%%%%%
%\subsection{Feature Suggestions}
%
%The following is a list of features which may be useful for future
%versions of this package:
%%
%\begin{itemize}
%\item
%\ldots
%\end{itemize}

%%%%%%%%%%%%%%%%%%%%%%%%%%%%%%%%%%%%%%%%%%%%%%%%%%%%%%%%%%%%%%%%%%%%%%%%%%%%%%%%
\subsection{Revision History}

%%%%%%%%%%%%%%%%%%%%%%%%%%%%%%%%%%%%%%%%
\paragraph{v2.0:} 2018/12/30

\begin{itemize}
\item
immediate forward processing
\item
added |\childdocby| mechanism
\item
manual restructured
\end{itemize}

%%%%%%%%%%%%%%%%%%%%%%%%%%%%%%%%%%%%%%%%
\paragraph{v1.6:} 2018/01/17

\begin{itemize}
\item
application for development of include files
\item
corrections to manual
\end{itemize}

%%%%%%%%%%%%%%%%%%%%%%%%%%%%%%%%%%%%%%%%
\paragraph{v1.5:} 2017/05/21

\begin{itemize}
\item
more complete structuring introduced
\item
|\childdocof| introduced
\item
|\childdoc| renamed to |\childdocmain|
\item
|\childredirect| renamed to |\childdocforward| and |\childdocforwardprefix|
and functionality expanded
\end{itemize}

%%%%%%%%%%%%%%%%%%%%%%%%%%%%%%%%%%%%%%%%
\paragraph{v1.0:} 2017/04/27

\begin{itemize}
\item
manual and install package
\item
first version published on CTAN
\end{itemize}

%%%%%%%%%%%%%%%%%%%%%%%%%%%%%%%%%%%%%%%%
\paragraph{v0.6:} 2017/04/26

\begin{itemize}
\item
redirection mechanism added
\end{itemize}

%%%%%%%%%%%%%%%%%%%%%%%%%%%%%%%%%%%%%%%%
\paragraph{v0.5:} 2017/04/26

\begin{itemize}
\item
functionality in definition file
\end{itemize}


%%%%%%%%%%%%%%%%%%%%%%%%%%%%%%%%%%%%%%%%%%%%%%%%%%%%%%%%%%%%%%%%%%%%%%%%%%%%%%%%
%%%%%%%%%%%%%%%%%%%%%%%%%%%%%%%%%%%%%%%%%%%%%%%%%%%%%%%%%%%%%%%%%%%%%%%%%%%%%%%%
%%%%%%%%%%%%%%%%%%%%%%%%%%%%%%%%%%%%%%%%%%%%%%%%%%%%%%%%%%%%%%%%%%%%%%%%%%%%%%%%
\appendix

\settowidth\MacroIndent{\rmfamily\scriptsize 000\ }

 \DocInput{childdoc.dtx}

\end{document}
%</driver>
% \fi
%
% %%%%%%%%%%%%%%%%%%%%%%%%%%%%%%%%%%%%%%%%%%%%%%%%%%%%%%%%%%%%%%%%%%%%%%%%%%%%%%
% %%%%%%%%%%%%%%%%%%%%%%%%%%%%%%%%%%%%%%%%%%%%%%%%%%%%%%%%%%%%%%%%%%%%%%%%%%%%%%
% \section{Sample}
%\iffalse
%<*samplemain>
%\fi
%
% The following presents a sample document
% with two chapters, two parts, a title page,
% a compile flag as well as three forwarding files to set the flag.
% It consists of eight |.tex| files:
% \begin{center}
% \begin{tabular}{ll}
% |cdocsamp.tex|&main file\\
% |cdocsch1.tex|&include file for chapter 1\\
% |cdocsch2.tex|&include file for chapter 2\\
% |cdocspt3.tex|&include file for part 3\\
% |cdocspt4.tex|&include file for part 4\\
% |cdocsdrf.tex|&forwarding file for main file in draft mode\\
% |cdocsfi1.tex|&forwarding file for final version of chapter 1\\
% |cdocsfi2.tex|&forwarding file for final version of chapter 2\\
% \end{tabular}
% \end{center}
% Each of the eight files can be compiled directly by the \LaTeX{} compiler.
%
% %%%%%%%%%%%%%%%%%%%%%%%%%%%%%%%%%%%%%%
% \paragraph{Main File.}
%
% The main file is called |cdocsamp.tex|.
%
% Load the \textsf{childdoc} definitions and
% declare the filename for the main document:
%    \begin{macrocode}
\input{childdoc.def}
\childdocmain{}
%    \end{macrocode}

% Optional override for |\version| flag:
%    \begin{macrocode}
%%\ifchilddoc\else\providecommand{\version}{draft}\fi
%    \end{macrocode}

% Define the default values for the |\version| flag
% (|final| for the main file and |draft| for childs):
%    \begin{macrocode}
\ifchilddoc
\providecommand{\version}{draft}
\else
\providecommand{\version}{final}
\fi
%    \end{macrocode}

% Load the standard document class:
%    \begin{macrocode}
\documentclass[12pt]{article}
%    \end{macrocode}

% Start the document body:
%    \begin{macrocode}
\begin{document}
%    \end{macrocode}

% Declare a title page.
% Print title, part of document being processed and version flag:
%    \begin{macrocode}
\addtocounter{page}{-1}
\begin{center}
{\LARGE\bfseries{}childdoc example\par}
\vspace{1cm}
\ifchilddoc
\ifchilddocmanual part\else chapter\fi:
`\childdocname' of `\childdocjob'\par
\else
main document: `\childdocjob'\par
\fi
version: \version\par
\end{center}
\newpage
%    \end{macrocode}

% Manually include selected file,
% otherwise process as usual:
%    \begin{macrocode}
\ifchilddocmanual
\section*{part `\childdocname'}
\input{\childdocname}
\else
%    \end{macrocode}

% Include the two chapters:
%    \begin{macrocode}
\include{cdocsch1}
\include{cdocsch2}
%    \end{macrocode}

% Include the two parts unless only chapters should be displayed:
%    \begin{macrocode}
\ifchilddoc\else
\section{part three}
\input{cdocspt3}
\section{part four}
\input{cdocspt4}
\fi
%    \end{macrocode}

% Process as usual until here:
%    \begin{macrocode}
\fi
%    \end{macrocode}

% End of document body:
%    \begin{macrocode}
\end{document}
%    \end{macrocode}
%\iffalse
%</samplemain>
%\fi
%
% %%%%%%%%%%%%%%%%%%%%%%%%%%%%%%%%%%%%%%
% \paragraph{Chapter Include Files.}
%
% The include files are called |cdocsch1.tex| and |cdocsch2.tex|.
%
%\iffalse
%<*samplechap1|samplechap2>
%\fi

% Optional override for |\version| flag:
%    \begin{macrocode}
%%\providecommand{\version}{final}
%    \end{macrocode}

% Include the main document:
%    \begin{macrocode}
\input{childdoc.def}
\childdocof{cdocsamp}
%    \end{macrocode}

%\iffalse
%</samplechap1|samplechap2>
%\fi
%
%\iffalse
%<*samplechap1>
%\fi
% Some text for chapter 1:
%    \begin{macrocode}
\section{one}
some text in chapter one
%    \end{macrocode}

%\iffalse
%</samplechap1>
%\fi
% Some text for chapter 2:
%\iffalse
%<*samplechap2>
%\fi
%    \begin{macrocode}
\section{two}
more text in chapter two
%    \end{macrocode}

%\iffalse
%</samplechap2>
%\fi
%
% %%%%%%%%%%%%%%%%%%%%%%%%%%%%%%%%%%%%%%
% \paragraph{Part Include Files.}
%
% The include files are called |cdocspt3.tex| and |cdocspt4.tex|.
%
%\iffalse
%<*samplepart3|samplepart4>
%\fi

% Optional override for |\version| flag:
%    \begin{macrocode}
%%\providecommand{\version}{final}
%    \end{macrocode}

% Include the main document:
%    \begin{macrocode}
\input{childdoc.def}
\childdocby{cdocsamp}
%    \end{macrocode}

%\iffalse
%</samplepart3|samplepart4>
%\fi
%
%\iffalse
%<*samplepart3>
%\fi
% Some text for part 3:
%    \begin{macrocode}
some text in part three
%    \end{macrocode}

%\iffalse
%</samplepart3>
%\fi
% Some text for part 4:
%\iffalse
%<*samplepart4>
%\fi
%    \begin{macrocode}
more text in part four
%    \end{macrocode}

%\iffalse
%</samplepart4>
%\fi
%
% %%%%%%%%%%%%%%%%%%%%%%%%%%%%%%%%%%%%%%
% \paragraph{Forwarding for a Complete Draft.}
%
% The following forwarding file |cdocsdrf.tex|
% compiles the main document in draft mode:
%\iffalse
%<*sampledraft>
%\fi
%    \begin{macrocode}
\def\version{draft}
\input{childdoc.def}
\childdocforward{cdocsamp}
%    \end{macrocode}

%\iffalse
%</sampledraft>
%\fi
%
% %%%%%%%%%%%%%%%%%%%%%%%%%%%%%%%%%%%%%%
% \paragraph{Forwarding for Final Version of the Chapters.}
%
% The following forwarding files |cdocsfn1.tex| and |cdocsfn2.tex|
% (with identical content)
% compile the final versions of the child documents
% |cdocsch1.tex| and |cdocsch2.tex|, respectively:
%\iffalse
%<*samplefinal>
%\fi
%    \begin{macrocode}
\def\version{final}
\input{childdoc.def}
\childdocforwardprefix[cdocsamp]{cdocsfn}{cdocsch}
%    \end{macrocode}

%\iffalse
%</samplefinal>
%\fi
%
% %%%%%%%%%%%%%%%%%%%%%%%%%%%%%%%%%%%%%%
% \paragraph{Command Line Processing.}
%
% The following three command lines generate the output files
% |cdocscld|, |cdocscl1| and |cdocscl2|
% which should be identical to
% |cdocsdrf|, |cdocsch1| and |cdocsfn2|, respectively:
% \begin{center}
% \begin{tabular}{l}
% |latex -jobname cdocscld \|\\
% |  "\def\version{draft}\input{childdoc.def}\childdocforward{cdocsamp}"|\\
% |latex -jobname cdocscl1 \|\\
% |  "\input{childdoc.def}\childdocforward[cdocsamp]{cdocsch1}"|\\
% |latex -jobname cdocscl2 \|\\
% |  "\def\version{final}\input{childdoc.def}\childdocforward{cdocsch2}"|
% \end{tabular}
% \end{center}
% Note that the trailing backslash on each first line
% merely continues the input to the second line
% (for convenient cut ant paste).
% Furthermore, the command |latex| can be replaced by any
% of its alternative versions such as |pdflatex|.
%
% %%%%%%%%%%%%%%%%%%%%%%%%%%%%%%%%%%%%%%%%%%%%%%%%%%%%%%%%%%%%%%%%%%%%%%%%%%%%%%
% %%%%%%%%%%%%%%%%%%%%%%%%%%%%%%%%%%%%%%%%%%%%%%%%%%%%%%%%%%%%%%%%%%%%%%%%%%%%%%
% \section{Implementation}
%\iffalse
%<*package>
%\fi
%
% This section describes the definitions file |childdoc.def|.

% The definitions cannot be loaded using |\usepackage| or |\RequirePackage|
% which has a mechanism to prevent loading a style file more than once.
% When loading the definitions by means of |\input|
% multiple instances have to be prevented manually:
%\iffalse
%This code needs to be before the `\ProvidesFile' directive
%which is defined at the beginning of this file.
%Therefore it is also placed there and commented out here.
%</package>
%<*discard>
%\fi
%    \begin{macrocode}
\ifdefined\childdocmain\endinput\fi
%    \end{macrocode}
%\iffalse
%</discard>
%<*package>
%\fi
%
% \macro{\ifchilddoc}
% \macro{\ifchilddocmanual}
% The conditional |\ifchilddoc| tells whether a
% child (true) or main (false) document is being compiled.
% The conditional |\ifchilddocmanual| tells whether
% the |\includeonly| mechanism is used (false) or
% the selection of child files must be performed manually (true).
% The definitions initialise to false:
%    \begin{macrocode}
\newif\ifchilddoc
\newif\ifchilddocmanual
%    \end{macrocode}

% \macro{\childdocname}
% \macro{\childdocjob}
% The macro |\childdocname| stores the name of the main document
% to be compiled. The macro |\childdocjob| stores the name of
% the document on which the \LaTeX{} compiler was originally invoked.
% The content of |\jobname| cannot be compared
% to filenames specified in the source due to different catcodes.
% The following code rescans |\jobname|, stores the result
% in |\childdocname| and saves a copy in |\childdocjob|:
%    \begin{macrocode}
\edef\childdocname{\scantokens\expandafter{\jobname\noexpand}}
\let\childdocjob\childdocname
%    \end{macrocode}

% \macro{\childdocdisable}
% The macro |\childdocdisable| prevents the main file
% from being processed more than once.
% At this stage, the main document command |\childdocmain|
% is assumed to be called once again where it should do nothing.
% Any subsequent call to it should prevent
% a secondary processing of the main document
% It overwrites the forwarding commands
% |\childdocof| and |\childdocforward|
% with empty macros to prevent further inclusions of the main document:
%    \begin{macrocode}
\newcommand{\childdocdisable}
{
  \renewcommand{\childdocmain}[1]{\renewcommand{\childdocmain}[1]{\endinput}}
  \renewcommand{\childdocof}[1]{}
  \renewcommand{\childdocby}[2][]{}
  \renewcommand{\childdocforward}[2][]{}
  \renewcommand{\childdocdisable}{}
}
%    \end{macrocode}

% \macro{\childdocmain}
% The macro |\childdocmain| is to be called at the top of the main file
% with nothing or the main filename (without extension) as argument.
% First, it breaks loops.
% If the argument is not empty and does not match |\childdocname|
% (which is set by the first inclusion of |childdoc.def|),
% |\ifchilddoc| is set to true, |\includeonly| is applied to the child file
% and |\jobname| is set to the main file
% (for proper handling of |.aux| files):
%    \begin{macrocode}
\newcommand{\childdocmain}[1]
{
  \childdocdisable\childdocmain{}
  \if?#1?\else
    \begingroup
      \def\childdoctmp{#1}
      \ifx\childdoctmp\childdocname
        \def\childdoctmp{}
      \else
        \def\childdoctmp
        {
          \childdoctrue
          \includeonly{\childdocname}
          \def\childdocjob{#1}
          \def\jobname{#1}
        }
      \fi
      \expandafter
    \endgroup
    \childdoctmp
  \fi
}
%    \end{macrocode}

% \macro{\childdocof}
% The command |\childdocof| redirects
% compilation to the main file |#1|.
%    \begin{macrocode}
\newcommand{\childdocof}[1]
{
  \childdocdisable
  \childdoctrue
  \includeonly{\childdocname}
  \def\jobname{#1}
  \def\childdocjob{#1}
  \input{#1}
}
%    \end{macrocode}

% \macro{\childdocby}
% The command |\childdocby| ....
%    \begin{macrocode}
\newcommand{\childdocby}[2][]
{
  \childdocdisable
  \childdoctrue
  \childdocmanualtrue
  \if?#1?\else
    \def\jobname{#2}
  \fi
  \def\childdocjob{#2}
  \input{#2}
  \endinput
}
%    \end{macrocode}

% \macro{\childdocforward}
% The command |\childdocforward| redirects
% compilation to the main file or
% (if the optional argument is given) a child file.
% Parameters are set as if the main file
% or a child file starting with |\childdocof| was compiled.
% Then compilation is handed over to the main file:
%    \begin{macrocode}
\newcommand{\childdocforward}[2][]
{
  \begingroup
    \if?#1?
      \def\childdoctmp
      {
        \def\childdocname{#2}
        \def\childdocjob{#2}
        \def\jobname{#2}
        \input{#2}
        \endinput
      }
    \else
      \def\childdoctmp
      {
        \childdocdisable
        \def\childdocname{#2}
        \childdoctrue
        \includeonly{#2}
        \def\childdocjob{#1}
        \def\jobname{#1}
        \input{#1}
        \endinput
      }
    \fi
    \expandafter
  \endgroup
  \childdoctmp
}
%    \end{macrocode}

% \macro{\childdocforwardprefix}
% The command |\childdocforwardprefix| redirects
% compilation to the main or a child file by means of a pattern.
% The prefix |#1| in the current filename is replaced by |#2|
% and the suffix of the current filename is kept
% (it is assumed that the filename does not contain the substring `|~~~|'
% which is used as a delimiter).
% Compilation is handed over to the new file by |\childdocforward|:
%    \begin{macrocode}
\newcommand{\childdocforwardprefix}[3][]
{
  \begingroup
    \def\childdocextract #2##1~~~{\def\childdoctmp{\childdocforward[#1]{#3##1}}}
    \expandafter\childdocextract\childdocname~~~
    \expandafter
  \endgroup
  \childdoctmp
}
%    \end{macrocode}

% \macro{\childdoc}
% The deprecated macro |\childdoc| is a legacy version of |\childdocmain|:
%    \begin{macrocode}
\newcommand{\childdoc}{\childdocmain}
%    \end{macrocode}

% \macro{\childdocredirect}
% The deprecated macro |\childdocredirect| is a legacy version
% of |\childdocforward| and |\childdocforwardprefix|:
%    \begin{macrocode}
\newcommand{\childdocredirect}[2][]
{
  \begingroup
    \if?#1?
      \def\childdoctmp{\childdocforward{#2}}
    \else
      \def\childdoctmp{\childdocforwardprefix{#1}{#2}}
    \fi
    \expandafter
  \endgroup
  \childdoctmp
}
%    \end{macrocode}

%\iffalse
%</package>
%\fi
%
\endinput
\childdocforward{cdocsch2}"|
% \end{tabular}
% \end{center}
% Note that the trailing backslash on each first line
% merely continues the input to the second line
% (for convenient cut ant paste).
% Furthermore, the command |latex| can be replaced by any
% of its alternative versions such as |pdflatex|.
%
% %%%%%%%%%%%%%%%%%%%%%%%%%%%%%%%%%%%%%%%%%%%%%%%%%%%%%%%%%%%%%%%%%%%%%%%%%%%%%%
% %%%%%%%%%%%%%%%%%%%%%%%%%%%%%%%%%%%%%%%%%%%%%%%%%%%%%%%%%%%%%%%%%%%%%%%%%%%%%%
% \section{Implementation}
%\iffalse
%<*package>
%\fi
%
% This section describes the definitions file |childdoc.def|.

% The definitions cannot be loaded using |\usepackage| or |\RequirePackage|
% which has a mechanism to prevent loading a style file more than once.
% When loading the definitions by means of |\input|
% multiple instances have to be prevented manually:
%\iffalse
%This code needs to be before the `\ProvidesFile' directive
%which is defined at the beginning of this file.
%Therefore it is also placed there and commented out here.
%</package>
%<*discard>
%\fi
%    \begin{macrocode}
\ifdefined\childdocmain\endinput\fi
%    \end{macrocode}
%\iffalse
%</discard>
%<*package>
%\fi
%
% \macro{\ifchilddoc}
% \macro{\ifchilddocmanual}
% The conditional |\ifchilddoc| tells whether a
% child (true) or main (false) document is being compiled.
% The conditional |\ifchilddocmanual| tells whether
% the |\includeonly| mechanism is used (false) or
% the selection of child files must be performed manually (true).
% The definitions initialise to false:
%    \begin{macrocode}
\newif\ifchilddoc
\newif\ifchilddocmanual
%    \end{macrocode}

% \macro{\childdocname}
% \macro{\childdocjob}
% The macro |\childdocname| stores the name of the main document
% to be compiled. The macro |\childdocjob| stores the name of
% the document on which the \LaTeX{} compiler was originally invoked.
% The content of |\jobname| cannot be compared
% to filenames specified in the source due to different catcodes.
% The following code rescans |\jobname|, stores the result
% in |\childdocname| and saves a copy in |\childdocjob|:
%    \begin{macrocode}
\edef\childdocname{\scantokens\expandafter{\jobname\noexpand}}
\let\childdocjob\childdocname
%    \end{macrocode}

% \macro{\childdocdisable}
% The macro |\childdocdisable| prevents the main file
% from being processed more than once.
% At this stage, the main document command |\childdocmain|
% is assumed to be called once again where it should do nothing.
% Any subsequent call to it should prevent
% a secondary processing of the main document
% It overwrites the forwarding commands
% |\childdocof| and |\childdocforward|
% with empty macros to prevent further inclusions of the main document:
%    \begin{macrocode}
\newcommand{\childdocdisable}
{
  \renewcommand{\childdocmain}[1]{\renewcommand{\childdocmain}[1]{\endinput}}
  \renewcommand{\childdocof}[1]{}
  \renewcommand{\childdocby}[2][]{}
  \renewcommand{\childdocforward}[2][]{}
  \renewcommand{\childdocdisable}{}
}
%    \end{macrocode}

% \macro{\childdocmain}
% The macro |\childdocmain| is to be called at the top of the main file
% with nothing or the main filename (without extension) as argument.
% First, it breaks loops.
% If the argument is not empty and does not match |\childdocname|
% (which is set by the first inclusion of |childdoc.def|),
% |\ifchilddoc| is set to true, |\includeonly| is applied to the child file
% and |\jobname| is set to the main file
% (for proper handling of |.aux| files):
%    \begin{macrocode}
\newcommand{\childdocmain}[1]
{
  \childdocdisable\childdocmain{}
  \if?#1?\else
    \begingroup
      \def\childdoctmp{#1}
      \ifx\childdoctmp\childdocname
        \def\childdoctmp{}
      \else
        \def\childdoctmp
        {
          \childdoctrue
          \includeonly{\childdocname}
          \def\childdocjob{#1}
          \def\jobname{#1}
        }
      \fi
      \expandafter
    \endgroup
    \childdoctmp
  \fi
}
%    \end{macrocode}

% \macro{\childdocof}
% The command |\childdocof| redirects
% compilation to the main file |#1|.
%    \begin{macrocode}
\newcommand{\childdocof}[1]
{
  \childdocdisable
  \childdoctrue
  \includeonly{\childdocname}
  \def\jobname{#1}
  \def\childdocjob{#1}
  \input{#1}
}
%    \end{macrocode}

% \macro{\childdocby}
% The command |\childdocby| ....
%    \begin{macrocode}
\newcommand{\childdocby}[2][]
{
  \childdocdisable
  \childdoctrue
  \childdocmanualtrue
  \if?#1?\else
    \def\jobname{#2}
  \fi
  \def\childdocjob{#2}
  \input{#2}
  \endinput
}
%    \end{macrocode}

% \macro{\childdocforward}
% The command |\childdocforward| redirects
% compilation to the main file or
% (if the optional argument is given) a child file.
% Parameters are set as if the main file
% or a child file starting with |\childdocof| was compiled.
% Then compilation is handed over to the main file:
%    \begin{macrocode}
\newcommand{\childdocforward}[2][]
{
  \begingroup
    \if?#1?
      \def\childdoctmp
      {
        \def\childdocname{#2}
        \def\childdocjob{#2}
        \def\jobname{#2}
        \input{#2}
        \endinput
      }
    \else
      \def\childdoctmp
      {
        \childdocdisable
        \def\childdocname{#2}
        \childdoctrue
        \includeonly{#2}
        \def\childdocjob{#1}
        \def\jobname{#1}
        \input{#1}
        \endinput
      }
    \fi
    \expandafter
  \endgroup
  \childdoctmp
}
%    \end{macrocode}

% \macro{\childdocforwardprefix}
% The command |\childdocforwardprefix| redirects
% compilation to the main or a child file by means of a pattern.
% The prefix |#1| in the current filename is replaced by |#2|
% and the suffix of the current filename is kept
% (it is assumed that the filename does not contain the substring `|~~~|'
% which is used as a delimiter).
% Compilation is handed over to the new file by |\childdocforward|:
%    \begin{macrocode}
\newcommand{\childdocforwardprefix}[3][]
{
  \begingroup
    \def\childdocextract #2##1~~~{\def\childdoctmp{\childdocforward[#1]{#3##1}}}
    \expandafter\childdocextract\childdocname~~~
    \expandafter
  \endgroup
  \childdoctmp
}
%    \end{macrocode}

% \macro{\childdoc}
% The deprecated macro |\childdoc| is a legacy version of |\childdocmain|:
%    \begin{macrocode}
\newcommand{\childdoc}{\childdocmain}
%    \end{macrocode}

% \macro{\childdocredirect}
% The deprecated macro |\childdocredirect| is a legacy version
% of |\childdocforward| and |\childdocforwardprefix|:
%    \begin{macrocode}
\newcommand{\childdocredirect}[2][]
{
  \begingroup
    \if?#1?
      \def\childdoctmp{\childdocforward{#2}}
    \else
      \def\childdoctmp{\childdocforwardprefix{#1}{#2}}
    \fi
    \expandafter
  \endgroup
  \childdoctmp
}
%    \end{macrocode}

%\iffalse
%</package>
%\fi
%
\endinput

\childdocforward{cdocsamp}
%    \end{macrocode}

%\iffalse
%</sampledraft>
%\fi
%
% %%%%%%%%%%%%%%%%%%%%%%%%%%%%%%%%%%%%%%
% \paragraph{Forwarding for Final Version of the Chapters.}
%
% The following forwarding files |cdocsfn1.tex| and |cdocsfn2.tex|
% (with identical content)
% compile the final versions of the child documents
% |cdocsch1.tex| and |cdocsch2.tex|, respectively:
%\iffalse
%<*samplefinal>
%\fi
%    \begin{macrocode}
\def\version{final}
% \iffalse
%
% childdoc.dtx Copyright (C) 2017-2018 Niklas Beisert
%
% This work may be distributed and/or modified under the
% conditions of the LaTeX Project Public License, either version 1.3
% of this license or (at your option) any later version.
% The latest version of this license is in
%   http://www.latex-project.org/lppl.txt
% and version 1.3 or later is part of all distributions of LaTeX
% version 2005/12/01 or later.
%
% This work has the LPPL maintenance status `maintained'.
%
% The Current Maintainer of this work is Niklas Beisert.
%
% This work consists of the files childdoc.dtx and childdoc.ins
% and the derived files childdoc.def and cdocsamp.tex with
% cdocsch1.tex, cdocsch2.tex, cdocsdrf.tex, cdocsfn1.tex, cdocsfn2.tex.
%
%<package>\ifdefined\childdocmain\endinput\fi
%<package>\ProvidesFile{childdoc.def}[2018/12/30 v2.0 child document driver]
%<samplemain>\ProvidesFile{cdocsamp.tex}[2018/12/30 v2.0 sample for childdoc]
%<*driver>
%\ProvidesFile{childdoc.drv}[2018/12/30 v2.0 childdoc reference manual file]
\PassOptionsToClass{10pt,a4paper}{article}
\documentclass{ltxdoc}

\usepackage[margin=35mm]{geometry}
\usepackage{hyperref}
\usepackage{hyperxmp}
\usepackage[usenames]{color}

\hypersetup{colorlinks=true}
\hypersetup{pdfstartview=FitH}
\hypersetup{pdfpagemode=UseNone}
\hypersetup{pdfsource={}}
\hypersetup{pdflang={en-UK}}
\hypersetup{pdfcopyright={Copyright 2017-2018 Niklas Beisert.
  This work may be distributed and/or modified under the
  conditions of the LaTeX Project Public License, either version 1.3
  of this license or (at your option) any later version.}}
\hypersetup{pdflicenseurl={http://www.latex-project.org/lppl.txt}}
\hypersetup{pdfcontactaddress={ETH Zurich, ITP, HIT K,
  Wolfgang-Pauli-Strasse 27}}
\hypersetup{pdfcontactpostcode={8093}}
\hypersetup{pdfcontactcity={Zurich}}
\hypersetup{pdfcontactcountry={Switzerland}}
\hypersetup{pdfcontactemail={nbeisert@itp.phys.ethz.ch}}
\hypersetup{pdfcontacturl={http://people.phys.ethz.ch/\xmptilde nbeisert/}}

\newcommand{\secref}[1]{\hyperref[#1]{section \ref*{#1}}}

\parskip1ex
\parindent0pt
\let\olditemize\itemize
\def\itemize{\olditemize\parskip0pt}

\begin{document}

\title{The \textsf{childdoc} Package}
\hypersetup{pdftitle={The childdoc Package}}
\author{Niklas Beisert\\[2ex]
  Institut f\"ur Theoretische Physik\\
  Eidgen\"ossische Technische Hochschule Z\"urich\\
  Wolfgang-Pauli-Strasse 27, 8093 Z\"urich, Switzerland\\[1ex]
  \href{mailto:nbeisert@itp.phys.ethz.ch}
  {\texttt{nbeisert@itp.phys.ethz.ch}}}
\hypersetup{pdfauthor={Niklas Beisert}}
\hypersetup{pdfsubject={Manual for the LaTeX2e Package childdoc}}
\date{30 December 2018, \textsf{v2.0}}
\maketitle

\begin{abstract}\noindent
\textsf{childdoc} is a \LaTeXe{} package
that enables the direct compilation
of document sections included by |\include|
to individual files.
\end{abstract}

\begingroup
\parskip0ex
\tableofcontents
\endgroup

%%%%%%%%%%%%%%%%%%%%%%%%%%%%%%%%%%%%%%%%%%%%%%%%%%%%%%%%%%%%%%%%%%%%%%%%%%%%%%%%
%%%%%%%%%%%%%%%%%%%%%%%%%%%%%%%%%%%%%%%%%%%%%%%%%%%%%%%%%%%%%%%%%%%%%%%%%%%%%%%%
\section{Introduction}

\LaTeX{} provides a mechanism to structure a large document (such as a book)
into a main file and several child files (containing the chapters)
using the |\include| command.
This mechanism is beneficial for documents
which span hundreds of pages in order to
make the source file(s) more manageable.
Moreover, compilation can be restricted to
selected child files by means of the |\includeonly| command.
The latter feature can be used to reduce the compilation time while editing
(this was significantly more useful in the earlier days of \LaTeX{})
or to generate a smaller document which is easier to navigate.
Another application of |\includeonly| is to generate
documents consisting of selected parts of the complete document.

However, there are a few drawbacks of the plain |\include| mechanism:
\begin{itemize}
\item
The child files cannot be compiled on their own,
they can only be compiled via the main file.
A naive editing environment
(such as a text editor with an option
to have the current file processed by \LaTeX)
may require one to switch to the main file before compiling;
attempting to compile the child file produces errors.
\item
The main file must be modified (each time)
to adjust the |\includeonly| command
to the present needs. This easily leaves the main file in a messy state.
\item
The generated document will always carry the filename
of the main document. This is inconvenient if
several child files are to be compiled and
to be kept for distribution.
\end{itemize}

The present package provides a simple interface
to make child files individually compilable by \LaTeX{}.
Compiling a child file then has the same effect as compiling
the main file with an |\includeonly| command
to select the appropriate child.
Moreover the generated document will carry the name of the child
rather than the main file.
This resolves all three above issues.

This feature is meant to make the editing of books,
thesis documents and lecture notes somewhat more convenient.
However, the package can also be used efficiently for
composing a series of documents (such as exercise sheets)
which are typically distributed individually.
It then assists the author in generating the individual documents
(potentially in different versions)
as well as a document containing the collected series.
Another application is in developing style files
or other kinds of included material
where compilation of the style file could redirect
to a sample or test file.

%%%%%%%%%%%%%%%%%%%%%%%%%%%%%%%%%%%%%%%%%%%%%%%%%%%%%%%%%%%%%%%%%%%%%%%%%%%%%%%%
%%%%%%%%%%%%%%%%%%%%%%%%%%%%%%%%%%%%%%%%%%%%%%%%%%%%%%%%%%%%%%%%%%%%%%%%%%%%%%%%
\section{Usage}

First of all, the package \textsf{childdoc} is \emph{not} a standard
\LaTeXe{} |.sty| style file! Therefore it needs to be invoked in
a non-standard way.

%%%%%%%%%%%%%%%%%%%%%%%%%%%%%%%%%%%%%%%%%%%%%%%%%%%%%%%%%%%%%%%%%%%%%%%%%%%%%%%%
\subsection{Included Files}
\label{sec:include}

%%%%%%%%%%%%%%%%%%%%%%%%%%%%%%%%%%%%%%%%
\DescribeMacro{\childdocmain}
To use the package, add the commands
\begin{center}
\begin{tabular}{l}
|% \iffalse
%
% childdoc.dtx Copyright (C) 2017-2018 Niklas Beisert
%
% This work may be distributed and/or modified under the
% conditions of the LaTeX Project Public License, either version 1.3
% of this license or (at your option) any later version.
% The latest version of this license is in
%   http://www.latex-project.org/lppl.txt
% and version 1.3 or later is part of all distributions of LaTeX
% version 2005/12/01 or later.
%
% This work has the LPPL maintenance status `maintained'.
%
% The Current Maintainer of this work is Niklas Beisert.
%
% This work consists of the files childdoc.dtx and childdoc.ins
% and the derived files childdoc.def and cdocsamp.tex with
% cdocsch1.tex, cdocsch2.tex, cdocsdrf.tex, cdocsfn1.tex, cdocsfn2.tex.
%
%<package>\ifdefined\childdocmain\endinput\fi
%<package>\ProvidesFile{childdoc.def}[2018/12/30 v2.0 child document driver]
%<samplemain>\ProvidesFile{cdocsamp.tex}[2018/12/30 v2.0 sample for childdoc]
%<*driver>
%\ProvidesFile{childdoc.drv}[2018/12/30 v2.0 childdoc reference manual file]
\PassOptionsToClass{10pt,a4paper}{article}
\documentclass{ltxdoc}

\usepackage[margin=35mm]{geometry}
\usepackage{hyperref}
\usepackage{hyperxmp}
\usepackage[usenames]{color}

\hypersetup{colorlinks=true}
\hypersetup{pdfstartview=FitH}
\hypersetup{pdfpagemode=UseNone}
\hypersetup{pdfsource={}}
\hypersetup{pdflang={en-UK}}
\hypersetup{pdfcopyright={Copyright 2017-2018 Niklas Beisert.
  This work may be distributed and/or modified under the
  conditions of the LaTeX Project Public License, either version 1.3
  of this license or (at your option) any later version.}}
\hypersetup{pdflicenseurl={http://www.latex-project.org/lppl.txt}}
\hypersetup{pdfcontactaddress={ETH Zurich, ITP, HIT K,
  Wolfgang-Pauli-Strasse 27}}
\hypersetup{pdfcontactpostcode={8093}}
\hypersetup{pdfcontactcity={Zurich}}
\hypersetup{pdfcontactcountry={Switzerland}}
\hypersetup{pdfcontactemail={nbeisert@itp.phys.ethz.ch}}
\hypersetup{pdfcontacturl={http://people.phys.ethz.ch/\xmptilde nbeisert/}}

\newcommand{\secref}[1]{\hyperref[#1]{section \ref*{#1}}}

\parskip1ex
\parindent0pt
\let\olditemize\itemize
\def\itemize{\olditemize\parskip0pt}

\begin{document}

\title{The \textsf{childdoc} Package}
\hypersetup{pdftitle={The childdoc Package}}
\author{Niklas Beisert\\[2ex]
  Institut f\"ur Theoretische Physik\\
  Eidgen\"ossische Technische Hochschule Z\"urich\\
  Wolfgang-Pauli-Strasse 27, 8093 Z\"urich, Switzerland\\[1ex]
  \href{mailto:nbeisert@itp.phys.ethz.ch}
  {\texttt{nbeisert@itp.phys.ethz.ch}}}
\hypersetup{pdfauthor={Niklas Beisert}}
\hypersetup{pdfsubject={Manual for the LaTeX2e Package childdoc}}
\date{30 December 2018, \textsf{v2.0}}
\maketitle

\begin{abstract}\noindent
\textsf{childdoc} is a \LaTeXe{} package
that enables the direct compilation
of document sections included by |\include|
to individual files.
\end{abstract}

\begingroup
\parskip0ex
\tableofcontents
\endgroup

%%%%%%%%%%%%%%%%%%%%%%%%%%%%%%%%%%%%%%%%%%%%%%%%%%%%%%%%%%%%%%%%%%%%%%%%%%%%%%%%
%%%%%%%%%%%%%%%%%%%%%%%%%%%%%%%%%%%%%%%%%%%%%%%%%%%%%%%%%%%%%%%%%%%%%%%%%%%%%%%%
\section{Introduction}

\LaTeX{} provides a mechanism to structure a large document (such as a book)
into a main file and several child files (containing the chapters)
using the |\include| command.
This mechanism is beneficial for documents
which span hundreds of pages in order to
make the source file(s) more manageable.
Moreover, compilation can be restricted to
selected child files by means of the |\includeonly| command.
The latter feature can be used to reduce the compilation time while editing
(this was significantly more useful in the earlier days of \LaTeX{})
or to generate a smaller document which is easier to navigate.
Another application of |\includeonly| is to generate
documents consisting of selected parts of the complete document.

However, there are a few drawbacks of the plain |\include| mechanism:
\begin{itemize}
\item
The child files cannot be compiled on their own,
they can only be compiled via the main file.
A naive editing environment
(such as a text editor with an option
to have the current file processed by \LaTeX)
may require one to switch to the main file before compiling;
attempting to compile the child file produces errors.
\item
The main file must be modified (each time)
to adjust the |\includeonly| command
to the present needs. This easily leaves the main file in a messy state.
\item
The generated document will always carry the filename
of the main document. This is inconvenient if
several child files are to be compiled and
to be kept for distribution.
\end{itemize}

The present package provides a simple interface
to make child files individually compilable by \LaTeX{}.
Compiling a child file then has the same effect as compiling
the main file with an |\includeonly| command
to select the appropriate child.
Moreover the generated document will carry the name of the child
rather than the main file.
This resolves all three above issues.

This feature is meant to make the editing of books,
thesis documents and lecture notes somewhat more convenient.
However, the package can also be used efficiently for
composing a series of documents (such as exercise sheets)
which are typically distributed individually.
It then assists the author in generating the individual documents
(potentially in different versions)
as well as a document containing the collected series.
Another application is in developing style files
or other kinds of included material
where compilation of the style file could redirect
to a sample or test file.

%%%%%%%%%%%%%%%%%%%%%%%%%%%%%%%%%%%%%%%%%%%%%%%%%%%%%%%%%%%%%%%%%%%%%%%%%%%%%%%%
%%%%%%%%%%%%%%%%%%%%%%%%%%%%%%%%%%%%%%%%%%%%%%%%%%%%%%%%%%%%%%%%%%%%%%%%%%%%%%%%
\section{Usage}

First of all, the package \textsf{childdoc} is \emph{not} a standard
\LaTeXe{} |.sty| style file! Therefore it needs to be invoked in
a non-standard way.

%%%%%%%%%%%%%%%%%%%%%%%%%%%%%%%%%%%%%%%%%%%%%%%%%%%%%%%%%%%%%%%%%%%%%%%%%%%%%%%%
\subsection{Included Files}
\label{sec:include}

%%%%%%%%%%%%%%%%%%%%%%%%%%%%%%%%%%%%%%%%
\DescribeMacro{\childdocmain}
To use the package, add the commands
\begin{center}
\begin{tabular}{l}
|\input{childdoc.def}|\\
|\childdocmain{}|\\
\end{tabular}
\end{center}
at the very top of the main \LaTeX{} file,
in particular \emph{before} the |\documentclass| statement!
The argument of |\childdocmain| should be left empty
(but it must be present).

%%%%%%%%%%%%%%%%%%%%%%%%%%%%%%%%%%%%%%%%
\DescribeMacro{\childdocof}
Furthermore, add the commands
\begin{center}
\begin{tabular}{l}
|\input{childdoc.def}|\\
|\childdocof{|\textit{main}|}|\\
\end{tabular}
\end{center}
at the top of every child file \textit{child}
which is included by |\include{|\textit{child}|}|
from within the main file
(or at least for those files to be compiled individually).
The argument \textit{main} must be the filename of the main file.

There are a couple of
considerations in setting up the main and child documents:

%%%%%%%%%%%%%%%%%%%%%%%%%%%%%%%%%%%%%%%%
\paragraph{Restrictions.}

Please note the following restrictions:
\begin{itemize}
\item
|\childdocmain| must be called with one argument \textit{main}
to ensure compatibility with earlier version of the package.
It must either be empty (|\childdocmain{}|)
or precisely match the filename of the main file in which it is specified.
See \secref{sec:detection} for further information.
\item
The filename \textit{main} must be specified without the |.tex| extension.
\item
The filename \textit{main} is case sensitive
(even in case-insensitive file systems)
due to internal string comparison.
\item
The argument \textit{main} should be fully expanded, it cannot be a macro.
\item
Subdirectories and special characters should be avoided in filenames.
\item
The command |\childdocmain{|\textit{main}|}| must be followed by a whitespace.
It should not be followed immediately by another command
or by a comment mark `|%|'.
This is because the \TeX{} parser reads the token immediately following
the argument of |\childdocmain| and puts it
at the beginning of every child section;
however, a white\-space is ignored.
\end{itemize}

%%%%%%%%%%%%%%%%%%%%%%%%%%%%%%%%%%%%%%%%
\paragraph{Content of Main File.}

It is advisable to place all content in the child files included by |\include|.
Any output contained in the main file will appear in all child documents
unless suppressed manually;
it cannot be suppressed automatically by the |\includeonly| directive
and thus should normally be avoided.
A method to include some content in the main file
by means of conditional processing is described in \secref{sec:conditional}.

%%%%%%%%%%%%%%%%%%%%%%%%%%%%%%%%%%%%%%%%
\paragraph{Page Numbering.}

When only a part of the document is compiled,
the appropriate numbering of pages
(as well as other status parameters)
is determined from the |.aux| files.
The latter contain information from previous passes.
However this information needs to propagate through
all intermediate child documents.
Therefore the page numbering in child documents may well
be inconsistent until the complete document is compiled at least once.

A useful (if unconventional) way to always ensure a consistent
page numbering is to restart the numbering in each child document
and denote the pages by `\textit{child}|.|\textit{page}'
where \textit{child} represents the chapter/section number of the child file.
This can be achieved by the command
|\numberwithin{page}{|\textit{child}|}|
of the \textsf{amsmath} package
where \textit{child} can be |chapter| or |section|
depending on the chosen structuring.
Alternatively, one can modify the macro |\thepage| appropriately
and reset the counter |page| at the start of each child file.

%%%%%%%%%%%%%%%%%%%%%%%%%%%%%%%%%%%%%%%%%%%%%%%%%%%%%%%%%%%%%%%%%%%%%%%%%%%%%%%%
\subsection{Conditional Processing}
\label{sec:conditional}

The package provides a mechanism to compile different versions
of a document. To customise the versions further some conditional processing
can come in handy to distinguish which version is being compiled.
The package provides two macros to describe the compilation context:

%%%%%%%%%%%%%%%%%%%%%%%%%%%%%%%%%%%%%%%%
\DescribeMacro{\ifchilddoc}
The conditional |\ifchilddoc| distinguishes between the compilation of
child documents and the main document:
%
\begin{center}
|\ifchilddoc |\textit{child-code}| |[|\||else |\textit{main-code}]| \||fi|
\end{center}

%%%%%%%%%%%%%%%%%%%%%%%%%%%%%%%%%%%%%%%%
\DescribeMacro{\childdocname}
\DescribeMacro{\childdocjob}
The macro |\childdocname| contains the filename (without extension)
of the main or child file being processed.
Note that |\childdocjob| will always contain the name of the main file.

%%%%%%%%%%%%%%%%%%%%%%%%%%%%%%%%%%%%%%%%
\paragraph{Title Page.}

Conditional processing can be used to include a title or banner page
in the main document when proper precautions are taken.
Importantly, the code in the main file should ensure that the page counter
(as well as other status parameters which are stored in the |.aux| files)
takes the same value after the conditional processing.
Otherwise the page numbers may take divergent values
depending on which part is compiled.

For example, a title page could be declared by:
%
\begin{center}
\begin{tabular}{l}
|\ifchilddoc\||else|\\
|\addtocounter{page}{-1}|\\
\textit{code for title page}\\
|\newpage|\\
|\||fi|
\end{tabular}
\end{center}
%
A banner page for the child documents can be generated by:
%
\begin{center}
\begin{tabular}{l}
|\ifchilddoc|\\
|\addtocounter{page}{-1}|\\
\textit{code for banner page}\\
|\newpage|\\
|\||fi|
\end{tabular}
\end{center}
%
Here one could write a message such as:
\begin{center}
|This is the part \childdocname{} of \childdocjob{}.|
\end{center}

%%%%%%%%%%%%%%%%%%%%%%%%%%%%%%%%%%%%%%%%%%%%%%%%%%%%%%%%%%%%%%%%%%%%%%%%%%%%%%%%
\subsection{Flags}
\label{sec:flags}

The package makes it easy to generate different versions
of the main or child documents.
To this end compilation flags can be defined
and assigned different default values.
They will be particularly useful in conjunction
with the forwarding mechanism described in \secref{sec:forward}.

For example, it may be useful to have a flag |\version|
which can be set to |draft| or |final|.
The document source will contain some conditional code
depending on the value of |\version|.
Suppose further, the flag should default to |final| for the main file
and to |draft| for child files
which is a natural assignment for editing the document.
This is achieved by placing the following code
in the preamble of the main document
(below the |\childdocmain| directive):
%
\begin{center}
\begin{tabular}{l}
|\ifchilddoc|\\
|\providecommand{\version}{draft}|\\
|\||else|\\
|\providecommand{\version}{final}|\\
|\||fi|
\end{tabular}
\end{center}
%
The definition by |\providecommand| makes sure
that previous definitions are not overwritten.
Further statements |\providecommand{\version}{...}|
can thus be added before the above code to override it.

For the main file, one might add a line
(between |\childdocmain| and the above block)
%
\begin{center}
|%\ifchilddoc\||else\providecommand{\version}{draft}\||fi|
\end{center}
%
which can be uncommented to produce a draft version.
Likewise one can add a line to the very top of a child file
(above the |\childdocof{|\textit{main}|}| directive)
%
\begin{center}
|%\providecommand{\version}{final}|
\end{center}
%
which can be uncommented to produce the final version of this child document.

%%%%%%%%%%%%%%%%%%%%%%%%%%%%%%%%%%%%%%%%%%%%%%%%%%%%%%%%%%%%%%%%%%%%%%%%%%%%%%%%
\subsection{Forwarding}
\label{sec:forward}

Different versions of the main or child documents
using compilation flags as described in \secref{sec:flags}
can be (permanently) stored in different files
for convenient compilation, viewing and distribution.
To this end, the package defines a command
to pass on compilation to a different file:

%%%%%%%%%%%%%%%%%%%%%%%%%%%%%%%%%%%%%%%%
\DescribeMacro{\childdocforward}
The command |\childdocforward| redirects processing to
another source file:
%
\begin{center}
\begin{tabular}{l}
|\input{childdoc.def}|\\
|\childdocforward[|\textit{main}|]{|\textit{dest}|}|\\
\end{tabular}
\end{center}
%
The argument \textit{dest} is the destination file
(without extension).
It should be the main file or one of the child files.
Note that further \textsf{childdoc} directives
such as |\childdocof| and |\childdocforward|
in the indicated file will be processed in this form.
The optional argument \textit{main}
passes on directly to the main file \textit{main}
while pretending to compile the child \textit{dest}.
This form behaves as if \textit{dest}
issues |\childdocof{|\textit{main}|}| right away,
and no further \textsf{childdoc} directives will be processed.

%%%%%%%%%%%%%%%%%%%%%%%%%%%%%%%%%%%%%%%%
\DescribeMacro{\...prefix}
In the alternative form |\childdocforwardprefix|,
%
\begin{center}
\begin{tabular}{l}
|\input{childdoc.def}|\\
|\childdocforwardprefix[|\textit{main}|]{|\textit{prefix}|}{|\textit{dest}|}|
\end{tabular}
\end{center}
%
the destination file is determined by a pattern
depending on the current file:
To make this work, the current file must be called
`{\textit{prefix}\hspace{0.2em}\textit{suffix}}'
with \textit{prefix} matching precisely the argument.
Processing is then passed on to the file
`{\textit{dest}\hspace{0.2em}\textit{suffix}}'.
Surely, the same effect is achieved by
directly specifying the
argument `{\textit{dest}\hspace{0.2em}\textit{suffix}}'
in the first form.
However, that requires to set up a different file
for each child. With the alternative form of the command
all these files can have exactly the same content
which simplifies setting them up and maintaining them.

For example, the following file |draft.tex|
with a compilation flag |\version| as described in \secref{sec:flags}
compiles the main document as a draft:
%
\begin{center}
\begin{tabular}{l}
|\def\version{draft}|\\
|\input{childdoc.def}|\\
|\childdocforward{|\textit{main}|}|
\end{tabular}
\end{center}
%
Likewise, the following files |final|\textit{nn}|.tex|
compile the final version of the child document
|child|\textit{nn}|.tex|:
%
\begin{center}
\begin{tabular}{l}
|\def\version{final}|\\
|\input{childdoc.def}|\\
|\childdocforwardprefix{final}{child}|
\end{tabular}
\end{center}
%

Note that when several versions of a main file and/or of each child file
are to be generated, it may be convenient to set up a |Makefile| or
shell script to automatise the process.

%%%%%%%%%%%%%%%%%%%%%%%%%%%%%%%%%%%%%%%%%%%%%%%%%%%%%%%%%%%%%%%%%%%%%%%%%%%%%%%%
\subsection{Command Line Processing}
\label{sec:commandline}

The effect of redirection files can also be achieved by invoking
the \LaTeX{} compiler with a more elaborate command line.
Most conveniently this should be done as part
of a shell script or a |Makefile|.

When using \textsf{childdoc} in the main file, the following
command lines effectively perform a redirection
(note that depending on the shell being used,
backslashes may have to be doubled: `|\|' $\to$ `|\\|'):
%
\begin{center}
|... -jobname "|\textit{target}|" |\\|"|[\textit{flags}]%
|\input{childdoc.def}\childdocforward[|\textit{main}|]{|\textit{dest}|}"|
\end{center}
%
Here \textit{target} is the name of the output file,
\textit{main} is the name of the main file
and \textit{dest} is the name of the main or child file to be processed
(all filenames without extensions).
The optional argument \textit{main} can be omitted
if \textit{main} matches \textit{dest}.
Optionally, compilation \textit{flags} can be defined via |\def| commands.
This command line makes the \TeX{} engine believe
it is compiling the file \textit{target}
whose content is specified as the latter parameter.
The provided code then forwards the processing to
\textit{main} or \textit{dest} as described in \secref{sec:forward}.

%%%%%%%%%%%%%%%%%%%%%%%%%%%%%%%%%%%%%%%%%%%%%%%%%%%%%%%%%%%%%%%%%%%%%%%%%%%%%%%%
\subsection{Include by Input}
\label{sec:input}

Including child documents by |\include| has some restrictions by design.
Most notably, the content of a child document always occupies
its own set of pages; pages cannot be shared between child documents.
Usually, this behaviour makes perfect sense
because each child document contain an essential part of the document.
However, in some situations it may be desirable to compose
a document from a collection of parts
without having mandatory page breaks between then.
For this case, the package
provides a mechanism to include parts
by |\input| which can also be processed individually.
However, by construction this mechanism
requires manual handling of the content to be output.

%%%%%%%%%%%%%%%%%%%%%%%%%%%%%%%%%%%%%%%%
\DescribeMacro{\ifchilddocmanual}
The main file should be prepared as usual, see \secref{sec:include}.
However, the document body must make a distinction
between processing of an individual part and of the main document, e.g.:
%
\begin{center}
\begin{tabular}{l}
|\ifchilddocmanual|\\
|\input{\childdocname}|\\
|\||else|\\
\textit{document body with }|\input{|\textit{part}|}|\\
|\||fi|
\end{tabular}
\end{center}
%
The conditional |\ifchilddocmanual| is true whenever
a part to be included by |\input| is being compiled,
and the name of the part is stored in |\childdocname|.

%%%%%%%%%%%%%%%%%%%%%%%%%%%%%%%%%%%%%%%%
\DescribeMacro{\childdocby}
Each part to be included by |\input| should start with:
%
\begin{center}
\begin{tabular}{l}
|\input{childdoc.def}|\\
|\childdocby{|\textit{main}|}|\\
\end{tabular}
\end{center}
%
The directive |\childdocby| is similar to |\childdocof|
described in \secref{sec:include},
but the subsequent selection of content must be done manually.
To that end, both |\ifchilddoc| and |\ifchilddocmanual|
will be true upon processing of a part,
and the name of the part is stored in |\childdocname|.
Note that |\jobname| will be set to the filename of the current part
so that each part receives an individual |.aux| file
that does not interfere with the |.aux| file(s) of the main document.
This behaviour can be altered by the alternative form
|\childdocby[*]{|\textit{main}|}| (with a non-empty optional argument)
which uses the |.aux| file of the main document
by setting |\jobname| to \textit{main}.

%%%%%%%%%%%%%%%%%%%%%%%%%%%%%%%%%%%%%%%%%%%%%%%%%%%%%%%%%%%%%%%%%%%%%%%%%%%%%%%%
\subsection{Driver Development}
\label{sec:driver}

The \textsf{childdoc} mechanism can also be use for the development
of definition files such as \LaTeX{} styles or classes.
This case differs from the above setup with multiple parts
included by |\include| in that no |\includeonly| should be invoked.
This can be achieved by starting the include file
(before |\ProvidesPackage|) with:
%
\begin{center}
\begin{tabular}{l}
|\input{childdoc.def}|\\
|\childdocforward{|\textit{main}|}|\\
\end{tabular}
\end{center}
%
or alternatively with:
%
\begin{center}
\begin{tabular}{l}
|\input{childdoc.def}|\\
|\childdocby{|\textit{main}|}|\\
\end{tabular}
\end{center}
%
Both forms have slightly different effects as described above.
The main file is prepared as usual, see \secref{sec:include}.

%%%%%%%%%%%%%%%%%%%%%%%%%%%%%%%%%%%%%%%%%%%%%%%%%%%%%%%%%%%%%%%%%%%%%%%%%%%%%%%%
\subsection{Legacy Detection}
\label{sec:detection}

The directive |\childdocmain| in the main file can detect
whether the complete document or merely a child is to be compiled
even without using the directive |\childdocof|.
This method is deprecated because it is less robust
and there is no compelling reason to use it;
it is merely provided for backward compatibility
and it may be removed in future versions.

If the detection mechanism is to be used,
it is mandatory to correctly specify
the filename of the main file as the argument of |\childdocmain|:
%
\begin{center}
\begin{tabular}{l}
|\input{childdoc.def}|\\
|\childdocmain{|\textit{main}|}|\\
\end{tabular}
\end{center}
%
If |\jobname| does not match the argument \textit{main} of |\childdocmain|,
it is assumed that |\jobname| points to the child file to be compiled.
When using |\childdocmain| with the main file specified as argument,
it suffices to start a child file
with just |\input{|\textit{main}|}|
without loading of the package and using |\childdocof|.
If instead all processing is done
with the appropriate \textsf{childdoc} directives,
the argument of \textit{main} of |\childdocmain| can be empty.

An alternative version of the command line processing described
in \secref{sec:commandline} using the detection mechanism reads:
%
\begin{center}
|... -jobname "|\textit{target}|" "|[\textit{flags}]%
[|\def\jobname{|\textit{dest}|}|]|\input{|\textit{main}|}"|
\end{center}

%%%%%%%%%%%%%%%%%%%%%%%%%%%%%%%%%%%%%%%%%%%%%%%%%%%%%%%%%%%%%%%%%%%%%%%%%%%%%%%%
\subsection{Manual Code}
\label{sec:manual}

In case one cannot be certain whether the definitions file |childdoc.def|
is installed on the target \TeX{} distribution
and one prefers not to ship it,
it is conceivable to paste a few relevant commands into the sources.

To that end, drop all statements |\input{childdoc.def}|
and perform the replacements as outlined below.
Instead of |\childdocmain{|\textit{main}|}| add the following code
to the top of the main file:
%
\begin{center}
\begin{tabular}{l}
|\||ifdefined\childdocname\endinput\||fi\newif\ifchilddoc|\\
|\edef\childdocname{\scantokens\expandafter{\jobname\noexpand}}|\\
|\def\childdocmain{|\textit{main}|}\||ifx\childdocmain\childdocname\||else|\\
|\childdoctrue\includeonly{\childdocname}\let\jobname\childdocmain\||fi|\\
\end{tabular}
\end{center}
%
Instead of |\childdocof{|\textit{main}|}| just include the main file
at the top of each child file:
%
\begin{center}
|\input{|\textit{main}|}|
\end{center}
%
A simple redirection |\childdocforward{|\textit{dest}|}| is achieved by:
%
\begin{center}
|\def\jobname{|\textit{dest}|}\input{\jobname}|
\end{center}
%
The redirection with prefix
|\childdocforwardprefix[|\textit{prefix}|]{|\textit{dest}|}|
is accomplished by:
%
\begin{center}
\begin{tabular}{l}
|{\edef\jobname{\scantokens\expandafter{\jobname\noexpand}}|\\
|\def\redirectjob |\textit{prefix}|#1~~~{\gdef\jobname{|\textit{dest}|#1}}|\\
|\expandafter\redirectjob\jobname~~~}\input{\jobname}|
\end{tabular}
\end{center}

In an alternative approach,
child documents can be compiled by a specific command line
without additional code or specific definitions:
%
\begin{center}
|... -jobname "|\textit{target}|" "|[\textit{flags}]%
|\includeonly{|\textit{dest}|}\input{|\textit{main}|}"|
\end{center}
%

%%%%%%%%%%%%%%%%%%%%%%%%%%%%%%%%%%%%%%%%%%%%%%%%%%%%%%%%%%%%%%%%%%%%%%%%%%%%%%%%
%%%%%%%%%%%%%%%%%%%%%%%%%%%%%%%%%%%%%%%%%%%%%%%%%%%%%%%%%%%%%%%%%%%%%%%%%%%%%%%%
\section{Information}

%%%%%%%%%%%%%%%%%%%%%%%%%%%%%%%%%%%%%%%%%%%%%%%%%%%%%%%%%%%%%%%%%%%%%%%%%%%%%%%%
\subsection{Copyright}

Copyright \copyright{} 2017--2018 Niklas Beisert

This work may be distributed and/or modified under the
conditions of the \LaTeX{} Project Public License, either version 1.3
of this license or (at your option) any later version.
The latest version of this license is in
  \url{http://www.latex-project.org/lppl.txt}
and version 1.3 or later is part of all distributions of \LaTeX{}
version 2005/12/01 or later.

This work has the LPPL maintenance status `maintained'.

The Current Maintainer of this work is Niklas Beisert.

This work consists of the files |README.txt|, |childdoc.ins| and |childdoc.dtx|
as well as the derived files |childdoc.def|, |cdocsamp.tex|
with |cdocsch1.tex|, |cdocsch2.tex|, |cdocspt3.tex|, |cdocspt4.tex|,
|cdocsdrf.tex|, |cdocsfn1.tex|, |cdocsfn2.tex|
as well as |childdoc.pdf|.

%%%%%%%%%%%%%%%%%%%%%%%%%%%%%%%%%%%%%%%%%%%%%%%%%%%%%%%%%%%%%%%%%%%%%%%%%%%%%%%%
\subsection{Files and Installation}

The package consists of the files:
%
\begin{center}
\begin{tabular}{ll}
    |README.txt|   & readme file \\
    |childdoc.ins| & installation file \\
    |childdoc.dtx| & source file \\
    |childdoc.def| & definition file \\
    |cdocsamp.tex| & sample main file \\
    |cdocsch1.tex| & sample include file \\
    |cdocsch2.tex| & sample include file \\
    |cdocspt3.tex| & sample part file \\
    |cdocspt4.tex| & sample part file \\
    |cdocsdrf.tex| & sample redirection file \\
    |cdocsfn1.tex| & sample redirection file \\
    |cdocsfn2.tex| & sample redirection file \\
    |childdoc.pdf| & manual
\end{tabular}
\end{center}
%
The distribution consists of the files
|README.txt|, |childdoc.ins| and |childdoc.dtx|.
%
\begin{itemize}
\item
Run (pdf)\LaTeX{} on |childdoc.dtx|
to compile the manual |childdoc.pdf| (this file).
\item
Run \LaTeX{} on |childdoc.ins| to create the definitions file |childdoc.def|
and the sample |cdocsamp.tex| with include files
|cdocsch1.tex|, |cdocsch2.tex|, |cdocspt3.tex|, |cdocspt4.tex|,
|cdocsdrf.tex|, |cdocsfn1.tex|, |cdocsfn2.tex|.
Then copy the file |childdoc.def| to an appropriate directory of your \LaTeX{}
distribution, e.g.\ \textit{texmf-root}|/tex/latex/childdoc|.
\end{itemize}

%%%%%%%%%%%%%%%%%%%%%%%%%%%%%%%%%%%%%%%%%%%%%%%%%%%%%%%%%%%%%%%%%%%%%%%%%%%%%%%%
\subsection{Related CTAN Packages}

There are several other packages which offer a similar functionality:
%
\begin{itemize}
\item
The packages
\href{http://ctan.org/pkg/docmute}{\textsf{docmute}},
\href{http://ctan.org/pkg/includex}{\textsf{includex}} and
\href{http://ctan.org/pkg/standalone}{\textsf{standalone}}
provide commands to include only the document body of
a child file thus allowing both files to be compiled individually.
\item
The packages \href{http://ctan.org/pkg/subdocs}{\textsf{subdocs}}
and \href{http://ctan.org/pkg/subfiles}{\textsf{subfiles}}
provide structures in which the main and child documents can be
encapsulated and allowing them to be compiled individually.
The inclusion mechanism is different from the conventional |\include|.
\item
The package \href{http://ctan.org/pkg/combine}{\textsf{combine}}
is an elaborate solution to combine several documents into one.
\end{itemize}
%
See also the CTAN topic \href{http://ctan.org/topic/subdocs}{\textsf{subdocs}}
for further related packages.
The present package differs from the above solutions in that
a document structure constructed with the conventional |\include| mechanism
just needs two extra commands at the top of every file
such that all constituent files can be compiled individually.

%%%%%%%%%%%%%%%%%%%%%%%%%%%%%%%%%%%%%%%%%%%%%%%%%%%%%%%%%%%%%%%%%%%%%%%%%%%%%%%%
%\subsection{Feature Suggestions}
%
%The following is a list of features which may be useful for future
%versions of this package:
%%
%\begin{itemize}
%\item
%\ldots
%\end{itemize}

%%%%%%%%%%%%%%%%%%%%%%%%%%%%%%%%%%%%%%%%%%%%%%%%%%%%%%%%%%%%%%%%%%%%%%%%%%%%%%%%
\subsection{Revision History}

%%%%%%%%%%%%%%%%%%%%%%%%%%%%%%%%%%%%%%%%
\paragraph{v2.0:} 2018/12/30

\begin{itemize}
\item
immediate forward processing
\item
added |\childdocby| mechanism
\item
manual restructured
\end{itemize}

%%%%%%%%%%%%%%%%%%%%%%%%%%%%%%%%%%%%%%%%
\paragraph{v1.6:} 2018/01/17

\begin{itemize}
\item
application for development of include files
\item
corrections to manual
\end{itemize}

%%%%%%%%%%%%%%%%%%%%%%%%%%%%%%%%%%%%%%%%
\paragraph{v1.5:} 2017/05/21

\begin{itemize}
\item
more complete structuring introduced
\item
|\childdocof| introduced
\item
|\childdoc| renamed to |\childdocmain|
\item
|\childredirect| renamed to |\childdocforward| and |\childdocforwardprefix|
and functionality expanded
\end{itemize}

%%%%%%%%%%%%%%%%%%%%%%%%%%%%%%%%%%%%%%%%
\paragraph{v1.0:} 2017/04/27

\begin{itemize}
\item
manual and install package
\item
first version published on CTAN
\end{itemize}

%%%%%%%%%%%%%%%%%%%%%%%%%%%%%%%%%%%%%%%%
\paragraph{v0.6:} 2017/04/26

\begin{itemize}
\item
redirection mechanism added
\end{itemize}

%%%%%%%%%%%%%%%%%%%%%%%%%%%%%%%%%%%%%%%%
\paragraph{v0.5:} 2017/04/26

\begin{itemize}
\item
functionality in definition file
\end{itemize}


%%%%%%%%%%%%%%%%%%%%%%%%%%%%%%%%%%%%%%%%%%%%%%%%%%%%%%%%%%%%%%%%%%%%%%%%%%%%%%%%
%%%%%%%%%%%%%%%%%%%%%%%%%%%%%%%%%%%%%%%%%%%%%%%%%%%%%%%%%%%%%%%%%%%%%%%%%%%%%%%%
%%%%%%%%%%%%%%%%%%%%%%%%%%%%%%%%%%%%%%%%%%%%%%%%%%%%%%%%%%%%%%%%%%%%%%%%%%%%%%%%
\appendix

\settowidth\MacroIndent{\rmfamily\scriptsize 000\ }

 \DocInput{childdoc.dtx}

\end{document}
%</driver>
% \fi
%
% %%%%%%%%%%%%%%%%%%%%%%%%%%%%%%%%%%%%%%%%%%%%%%%%%%%%%%%%%%%%%%%%%%%%%%%%%%%%%%
% %%%%%%%%%%%%%%%%%%%%%%%%%%%%%%%%%%%%%%%%%%%%%%%%%%%%%%%%%%%%%%%%%%%%%%%%%%%%%%
% \section{Sample}
%\iffalse
%<*samplemain>
%\fi
%
% The following presents a sample document
% with two chapters, two parts, a title page,
% a compile flag as well as three forwarding files to set the flag.
% It consists of eight |.tex| files:
% \begin{center}
% \begin{tabular}{ll}
% |cdocsamp.tex|&main file\\
% |cdocsch1.tex|&include file for chapter 1\\
% |cdocsch2.tex|&include file for chapter 2\\
% |cdocspt3.tex|&include file for part 3\\
% |cdocspt4.tex|&include file for part 4\\
% |cdocsdrf.tex|&forwarding file for main file in draft mode\\
% |cdocsfi1.tex|&forwarding file for final version of chapter 1\\
% |cdocsfi2.tex|&forwarding file for final version of chapter 2\\
% \end{tabular}
% \end{center}
% Each of the eight files can be compiled directly by the \LaTeX{} compiler.
%
% %%%%%%%%%%%%%%%%%%%%%%%%%%%%%%%%%%%%%%
% \paragraph{Main File.}
%
% The main file is called |cdocsamp.tex|.
%
% Load the \textsf{childdoc} definitions and
% declare the filename for the main document:
%    \begin{macrocode}
\input{childdoc.def}
\childdocmain{}
%    \end{macrocode}

% Optional override for |\version| flag:
%    \begin{macrocode}
%%\ifchilddoc\else\providecommand{\version}{draft}\fi
%    \end{macrocode}

% Define the default values for the |\version| flag
% (|final| for the main file and |draft| for childs):
%    \begin{macrocode}
\ifchilddoc
\providecommand{\version}{draft}
\else
\providecommand{\version}{final}
\fi
%    \end{macrocode}

% Load the standard document class:
%    \begin{macrocode}
\documentclass[12pt]{article}
%    \end{macrocode}

% Start the document body:
%    \begin{macrocode}
\begin{document}
%    \end{macrocode}

% Declare a title page.
% Print title, part of document being processed and version flag:
%    \begin{macrocode}
\addtocounter{page}{-1}
\begin{center}
{\LARGE\bfseries{}childdoc example\par}
\vspace{1cm}
\ifchilddoc
\ifchilddocmanual part\else chapter\fi:
`\childdocname' of `\childdocjob'\par
\else
main document: `\childdocjob'\par
\fi
version: \version\par
\end{center}
\newpage
%    \end{macrocode}

% Manually include selected file,
% otherwise process as usual:
%    \begin{macrocode}
\ifchilddocmanual
\section*{part `\childdocname'}
\input{\childdocname}
\else
%    \end{macrocode}

% Include the two chapters:
%    \begin{macrocode}
\include{cdocsch1}
\include{cdocsch2}
%    \end{macrocode}

% Include the two parts unless only chapters should be displayed:
%    \begin{macrocode}
\ifchilddoc\else
\section{part three}
\input{cdocspt3}
\section{part four}
\input{cdocspt4}
\fi
%    \end{macrocode}

% Process as usual until here:
%    \begin{macrocode}
\fi
%    \end{macrocode}

% End of document body:
%    \begin{macrocode}
\end{document}
%    \end{macrocode}
%\iffalse
%</samplemain>
%\fi
%
% %%%%%%%%%%%%%%%%%%%%%%%%%%%%%%%%%%%%%%
% \paragraph{Chapter Include Files.}
%
% The include files are called |cdocsch1.tex| and |cdocsch2.tex|.
%
%\iffalse
%<*samplechap1|samplechap2>
%\fi

% Optional override for |\version| flag:
%    \begin{macrocode}
%%\providecommand{\version}{final}
%    \end{macrocode}

% Include the main document:
%    \begin{macrocode}
\input{childdoc.def}
\childdocof{cdocsamp}
%    \end{macrocode}

%\iffalse
%</samplechap1|samplechap2>
%\fi
%
%\iffalse
%<*samplechap1>
%\fi
% Some text for chapter 1:
%    \begin{macrocode}
\section{one}
some text in chapter one
%    \end{macrocode}

%\iffalse
%</samplechap1>
%\fi
% Some text for chapter 2:
%\iffalse
%<*samplechap2>
%\fi
%    \begin{macrocode}
\section{two}
more text in chapter two
%    \end{macrocode}

%\iffalse
%</samplechap2>
%\fi
%
% %%%%%%%%%%%%%%%%%%%%%%%%%%%%%%%%%%%%%%
% \paragraph{Part Include Files.}
%
% The include files are called |cdocspt3.tex| and |cdocspt4.tex|.
%
%\iffalse
%<*samplepart3|samplepart4>
%\fi

% Optional override for |\version| flag:
%    \begin{macrocode}
%%\providecommand{\version}{final}
%    \end{macrocode}

% Include the main document:
%    \begin{macrocode}
\input{childdoc.def}
\childdocby{cdocsamp}
%    \end{macrocode}

%\iffalse
%</samplepart3|samplepart4>
%\fi
%
%\iffalse
%<*samplepart3>
%\fi
% Some text for part 3:
%    \begin{macrocode}
some text in part three
%    \end{macrocode}

%\iffalse
%</samplepart3>
%\fi
% Some text for part 4:
%\iffalse
%<*samplepart4>
%\fi
%    \begin{macrocode}
more text in part four
%    \end{macrocode}

%\iffalse
%</samplepart4>
%\fi
%
% %%%%%%%%%%%%%%%%%%%%%%%%%%%%%%%%%%%%%%
% \paragraph{Forwarding for a Complete Draft.}
%
% The following forwarding file |cdocsdrf.tex|
% compiles the main document in draft mode:
%\iffalse
%<*sampledraft>
%\fi
%    \begin{macrocode}
\def\version{draft}
\input{childdoc.def}
\childdocforward{cdocsamp}
%    \end{macrocode}

%\iffalse
%</sampledraft>
%\fi
%
% %%%%%%%%%%%%%%%%%%%%%%%%%%%%%%%%%%%%%%
% \paragraph{Forwarding for Final Version of the Chapters.}
%
% The following forwarding files |cdocsfn1.tex| and |cdocsfn2.tex|
% (with identical content)
% compile the final versions of the child documents
% |cdocsch1.tex| and |cdocsch2.tex|, respectively:
%\iffalse
%<*samplefinal>
%\fi
%    \begin{macrocode}
\def\version{final}
\input{childdoc.def}
\childdocforwardprefix[cdocsamp]{cdocsfn}{cdocsch}
%    \end{macrocode}

%\iffalse
%</samplefinal>
%\fi
%
% %%%%%%%%%%%%%%%%%%%%%%%%%%%%%%%%%%%%%%
% \paragraph{Command Line Processing.}
%
% The following three command lines generate the output files
% |cdocscld|, |cdocscl1| and |cdocscl2|
% which should be identical to
% |cdocsdrf|, |cdocsch1| and |cdocsfn2|, respectively:
% \begin{center}
% \begin{tabular}{l}
% |latex -jobname cdocscld \|\\
% |  "\def\version{draft}\input{childdoc.def}\childdocforward{cdocsamp}"|\\
% |latex -jobname cdocscl1 \|\\
% |  "\input{childdoc.def}\childdocforward[cdocsamp]{cdocsch1}"|\\
% |latex -jobname cdocscl2 \|\\
% |  "\def\version{final}\input{childdoc.def}\childdocforward{cdocsch2}"|
% \end{tabular}
% \end{center}
% Note that the trailing backslash on each first line
% merely continues the input to the second line
% (for convenient cut ant paste).
% Furthermore, the command |latex| can be replaced by any
% of its alternative versions such as |pdflatex|.
%
% %%%%%%%%%%%%%%%%%%%%%%%%%%%%%%%%%%%%%%%%%%%%%%%%%%%%%%%%%%%%%%%%%%%%%%%%%%%%%%
% %%%%%%%%%%%%%%%%%%%%%%%%%%%%%%%%%%%%%%%%%%%%%%%%%%%%%%%%%%%%%%%%%%%%%%%%%%%%%%
% \section{Implementation}
%\iffalse
%<*package>
%\fi
%
% This section describes the definitions file |childdoc.def|.

% The definitions cannot be loaded using |\usepackage| or |\RequirePackage|
% which has a mechanism to prevent loading a style file more than once.
% When loading the definitions by means of |\input|
% multiple instances have to be prevented manually:
%\iffalse
%This code needs to be before the `\ProvidesFile' directive
%which is defined at the beginning of this file.
%Therefore it is also placed there and commented out here.
%</package>
%<*discard>
%\fi
%    \begin{macrocode}
\ifdefined\childdocmain\endinput\fi
%    \end{macrocode}
%\iffalse
%</discard>
%<*package>
%\fi
%
% \macro{\ifchilddoc}
% \macro{\ifchilddocmanual}
% The conditional |\ifchilddoc| tells whether a
% child (true) or main (false) document is being compiled.
% The conditional |\ifchilddocmanual| tells whether
% the |\includeonly| mechanism is used (false) or
% the selection of child files must be performed manually (true).
% The definitions initialise to false:
%    \begin{macrocode}
\newif\ifchilddoc
\newif\ifchilddocmanual
%    \end{macrocode}

% \macro{\childdocname}
% \macro{\childdocjob}
% The macro |\childdocname| stores the name of the main document
% to be compiled. The macro |\childdocjob| stores the name of
% the document on which the \LaTeX{} compiler was originally invoked.
% The content of |\jobname| cannot be compared
% to filenames specified in the source due to different catcodes.
% The following code rescans |\jobname|, stores the result
% in |\childdocname| and saves a copy in |\childdocjob|:
%    \begin{macrocode}
\edef\childdocname{\scantokens\expandafter{\jobname\noexpand}}
\let\childdocjob\childdocname
%    \end{macrocode}

% \macro{\childdocdisable}
% The macro |\childdocdisable| prevents the main file
% from being processed more than once.
% At this stage, the main document command |\childdocmain|
% is assumed to be called once again where it should do nothing.
% Any subsequent call to it should prevent
% a secondary processing of the main document
% It overwrites the forwarding commands
% |\childdocof| and |\childdocforward|
% with empty macros to prevent further inclusions of the main document:
%    \begin{macrocode}
\newcommand{\childdocdisable}
{
  \renewcommand{\childdocmain}[1]{\renewcommand{\childdocmain}[1]{\endinput}}
  \renewcommand{\childdocof}[1]{}
  \renewcommand{\childdocby}[2][]{}
  \renewcommand{\childdocforward}[2][]{}
  \renewcommand{\childdocdisable}{}
}
%    \end{macrocode}

% \macro{\childdocmain}
% The macro |\childdocmain| is to be called at the top of the main file
% with nothing or the main filename (without extension) as argument.
% First, it breaks loops.
% If the argument is not empty and does not match |\childdocname|
% (which is set by the first inclusion of |childdoc.def|),
% |\ifchilddoc| is set to true, |\includeonly| is applied to the child file
% and |\jobname| is set to the main file
% (for proper handling of |.aux| files):
%    \begin{macrocode}
\newcommand{\childdocmain}[1]
{
  \childdocdisable\childdocmain{}
  \if?#1?\else
    \begingroup
      \def\childdoctmp{#1}
      \ifx\childdoctmp\childdocname
        \def\childdoctmp{}
      \else
        \def\childdoctmp
        {
          \childdoctrue
          \includeonly{\childdocname}
          \def\childdocjob{#1}
          \def\jobname{#1}
        }
      \fi
      \expandafter
    \endgroup
    \childdoctmp
  \fi
}
%    \end{macrocode}

% \macro{\childdocof}
% The command |\childdocof| redirects
% compilation to the main file |#1|.
%    \begin{macrocode}
\newcommand{\childdocof}[1]
{
  \childdocdisable
  \childdoctrue
  \includeonly{\childdocname}
  \def\jobname{#1}
  \def\childdocjob{#1}
  \input{#1}
}
%    \end{macrocode}

% \macro{\childdocby}
% The command |\childdocby| ....
%    \begin{macrocode}
\newcommand{\childdocby}[2][]
{
  \childdocdisable
  \childdoctrue
  \childdocmanualtrue
  \if?#1?\else
    \def\jobname{#2}
  \fi
  \def\childdocjob{#2}
  \input{#2}
  \endinput
}
%    \end{macrocode}

% \macro{\childdocforward}
% The command |\childdocforward| redirects
% compilation to the main file or
% (if the optional argument is given) a child file.
% Parameters are set as if the main file
% or a child file starting with |\childdocof| was compiled.
% Then compilation is handed over to the main file:
%    \begin{macrocode}
\newcommand{\childdocforward}[2][]
{
  \begingroup
    \if?#1?
      \def\childdoctmp
      {
        \def\childdocname{#2}
        \def\childdocjob{#2}
        \def\jobname{#2}
        \input{#2}
        \endinput
      }
    \else
      \def\childdoctmp
      {
        \childdocdisable
        \def\childdocname{#2}
        \childdoctrue
        \includeonly{#2}
        \def\childdocjob{#1}
        \def\jobname{#1}
        \input{#1}
        \endinput
      }
    \fi
    \expandafter
  \endgroup
  \childdoctmp
}
%    \end{macrocode}

% \macro{\childdocforwardprefix}
% The command |\childdocforwardprefix| redirects
% compilation to the main or a child file by means of a pattern.
% The prefix |#1| in the current filename is replaced by |#2|
% and the suffix of the current filename is kept
% (it is assumed that the filename does not contain the substring `|~~~|'
% which is used as a delimiter).
% Compilation is handed over to the new file by |\childdocforward|:
%    \begin{macrocode}
\newcommand{\childdocforwardprefix}[3][]
{
  \begingroup
    \def\childdocextract #2##1~~~{\def\childdoctmp{\childdocforward[#1]{#3##1}}}
    \expandafter\childdocextract\childdocname~~~
    \expandafter
  \endgroup
  \childdoctmp
}
%    \end{macrocode}

% \macro{\childdoc}
% The deprecated macro |\childdoc| is a legacy version of |\childdocmain|:
%    \begin{macrocode}
\newcommand{\childdoc}{\childdocmain}
%    \end{macrocode}

% \macro{\childdocredirect}
% The deprecated macro |\childdocredirect| is a legacy version
% of |\childdocforward| and |\childdocforwardprefix|:
%    \begin{macrocode}
\newcommand{\childdocredirect}[2][]
{
  \begingroup
    \if?#1?
      \def\childdoctmp{\childdocforward{#2}}
    \else
      \def\childdoctmp{\childdocforwardprefix{#1}{#2}}
    \fi
    \expandafter
  \endgroup
  \childdoctmp
}
%    \end{macrocode}

%\iffalse
%</package>
%\fi
%
\endinput
|\\
|\childdocmain{}|\\
\end{tabular}
\end{center}
at the very top of the main \LaTeX{} file,
in particular \emph{before} the |\documentclass| statement!
The argument of |\childdocmain| should be left empty
(but it must be present).

%%%%%%%%%%%%%%%%%%%%%%%%%%%%%%%%%%%%%%%%
\DescribeMacro{\childdocof}
Furthermore, add the commands
\begin{center}
\begin{tabular}{l}
|% \iffalse
%
% childdoc.dtx Copyright (C) 2017-2018 Niklas Beisert
%
% This work may be distributed and/or modified under the
% conditions of the LaTeX Project Public License, either version 1.3
% of this license or (at your option) any later version.
% The latest version of this license is in
%   http://www.latex-project.org/lppl.txt
% and version 1.3 or later is part of all distributions of LaTeX
% version 2005/12/01 or later.
%
% This work has the LPPL maintenance status `maintained'.
%
% The Current Maintainer of this work is Niklas Beisert.
%
% This work consists of the files childdoc.dtx and childdoc.ins
% and the derived files childdoc.def and cdocsamp.tex with
% cdocsch1.tex, cdocsch2.tex, cdocsdrf.tex, cdocsfn1.tex, cdocsfn2.tex.
%
%<package>\ifdefined\childdocmain\endinput\fi
%<package>\ProvidesFile{childdoc.def}[2018/12/30 v2.0 child document driver]
%<samplemain>\ProvidesFile{cdocsamp.tex}[2018/12/30 v2.0 sample for childdoc]
%<*driver>
%\ProvidesFile{childdoc.drv}[2018/12/30 v2.0 childdoc reference manual file]
\PassOptionsToClass{10pt,a4paper}{article}
\documentclass{ltxdoc}

\usepackage[margin=35mm]{geometry}
\usepackage{hyperref}
\usepackage{hyperxmp}
\usepackage[usenames]{color}

\hypersetup{colorlinks=true}
\hypersetup{pdfstartview=FitH}
\hypersetup{pdfpagemode=UseNone}
\hypersetup{pdfsource={}}
\hypersetup{pdflang={en-UK}}
\hypersetup{pdfcopyright={Copyright 2017-2018 Niklas Beisert.
  This work may be distributed and/or modified under the
  conditions of the LaTeX Project Public License, either version 1.3
  of this license or (at your option) any later version.}}
\hypersetup{pdflicenseurl={http://www.latex-project.org/lppl.txt}}
\hypersetup{pdfcontactaddress={ETH Zurich, ITP, HIT K,
  Wolfgang-Pauli-Strasse 27}}
\hypersetup{pdfcontactpostcode={8093}}
\hypersetup{pdfcontactcity={Zurich}}
\hypersetup{pdfcontactcountry={Switzerland}}
\hypersetup{pdfcontactemail={nbeisert@itp.phys.ethz.ch}}
\hypersetup{pdfcontacturl={http://people.phys.ethz.ch/\xmptilde nbeisert/}}

\newcommand{\secref}[1]{\hyperref[#1]{section \ref*{#1}}}

\parskip1ex
\parindent0pt
\let\olditemize\itemize
\def\itemize{\olditemize\parskip0pt}

\begin{document}

\title{The \textsf{childdoc} Package}
\hypersetup{pdftitle={The childdoc Package}}
\author{Niklas Beisert\\[2ex]
  Institut f\"ur Theoretische Physik\\
  Eidgen\"ossische Technische Hochschule Z\"urich\\
  Wolfgang-Pauli-Strasse 27, 8093 Z\"urich, Switzerland\\[1ex]
  \href{mailto:nbeisert@itp.phys.ethz.ch}
  {\texttt{nbeisert@itp.phys.ethz.ch}}}
\hypersetup{pdfauthor={Niklas Beisert}}
\hypersetup{pdfsubject={Manual for the LaTeX2e Package childdoc}}
\date{30 December 2018, \textsf{v2.0}}
\maketitle

\begin{abstract}\noindent
\textsf{childdoc} is a \LaTeXe{} package
that enables the direct compilation
of document sections included by |\include|
to individual files.
\end{abstract}

\begingroup
\parskip0ex
\tableofcontents
\endgroup

%%%%%%%%%%%%%%%%%%%%%%%%%%%%%%%%%%%%%%%%%%%%%%%%%%%%%%%%%%%%%%%%%%%%%%%%%%%%%%%%
%%%%%%%%%%%%%%%%%%%%%%%%%%%%%%%%%%%%%%%%%%%%%%%%%%%%%%%%%%%%%%%%%%%%%%%%%%%%%%%%
\section{Introduction}

\LaTeX{} provides a mechanism to structure a large document (such as a book)
into a main file and several child files (containing the chapters)
using the |\include| command.
This mechanism is beneficial for documents
which span hundreds of pages in order to
make the source file(s) more manageable.
Moreover, compilation can be restricted to
selected child files by means of the |\includeonly| command.
The latter feature can be used to reduce the compilation time while editing
(this was significantly more useful in the earlier days of \LaTeX{})
or to generate a smaller document which is easier to navigate.
Another application of |\includeonly| is to generate
documents consisting of selected parts of the complete document.

However, there are a few drawbacks of the plain |\include| mechanism:
\begin{itemize}
\item
The child files cannot be compiled on their own,
they can only be compiled via the main file.
A naive editing environment
(such as a text editor with an option
to have the current file processed by \LaTeX)
may require one to switch to the main file before compiling;
attempting to compile the child file produces errors.
\item
The main file must be modified (each time)
to adjust the |\includeonly| command
to the present needs. This easily leaves the main file in a messy state.
\item
The generated document will always carry the filename
of the main document. This is inconvenient if
several child files are to be compiled and
to be kept for distribution.
\end{itemize}

The present package provides a simple interface
to make child files individually compilable by \LaTeX{}.
Compiling a child file then has the same effect as compiling
the main file with an |\includeonly| command
to select the appropriate child.
Moreover the generated document will carry the name of the child
rather than the main file.
This resolves all three above issues.

This feature is meant to make the editing of books,
thesis documents and lecture notes somewhat more convenient.
However, the package can also be used efficiently for
composing a series of documents (such as exercise sheets)
which are typically distributed individually.
It then assists the author in generating the individual documents
(potentially in different versions)
as well as a document containing the collected series.
Another application is in developing style files
or other kinds of included material
where compilation of the style file could redirect
to a sample or test file.

%%%%%%%%%%%%%%%%%%%%%%%%%%%%%%%%%%%%%%%%%%%%%%%%%%%%%%%%%%%%%%%%%%%%%%%%%%%%%%%%
%%%%%%%%%%%%%%%%%%%%%%%%%%%%%%%%%%%%%%%%%%%%%%%%%%%%%%%%%%%%%%%%%%%%%%%%%%%%%%%%
\section{Usage}

First of all, the package \textsf{childdoc} is \emph{not} a standard
\LaTeXe{} |.sty| style file! Therefore it needs to be invoked in
a non-standard way.

%%%%%%%%%%%%%%%%%%%%%%%%%%%%%%%%%%%%%%%%%%%%%%%%%%%%%%%%%%%%%%%%%%%%%%%%%%%%%%%%
\subsection{Included Files}
\label{sec:include}

%%%%%%%%%%%%%%%%%%%%%%%%%%%%%%%%%%%%%%%%
\DescribeMacro{\childdocmain}
To use the package, add the commands
\begin{center}
\begin{tabular}{l}
|\input{childdoc.def}|\\
|\childdocmain{}|\\
\end{tabular}
\end{center}
at the very top of the main \LaTeX{} file,
in particular \emph{before} the |\documentclass| statement!
The argument of |\childdocmain| should be left empty
(but it must be present).

%%%%%%%%%%%%%%%%%%%%%%%%%%%%%%%%%%%%%%%%
\DescribeMacro{\childdocof}
Furthermore, add the commands
\begin{center}
\begin{tabular}{l}
|\input{childdoc.def}|\\
|\childdocof{|\textit{main}|}|\\
\end{tabular}
\end{center}
at the top of every child file \textit{child}
which is included by |\include{|\textit{child}|}|
from within the main file
(or at least for those files to be compiled individually).
The argument \textit{main} must be the filename of the main file.

There are a couple of
considerations in setting up the main and child documents:

%%%%%%%%%%%%%%%%%%%%%%%%%%%%%%%%%%%%%%%%
\paragraph{Restrictions.}

Please note the following restrictions:
\begin{itemize}
\item
|\childdocmain| must be called with one argument \textit{main}
to ensure compatibility with earlier version of the package.
It must either be empty (|\childdocmain{}|)
or precisely match the filename of the main file in which it is specified.
See \secref{sec:detection} for further information.
\item
The filename \textit{main} must be specified without the |.tex| extension.
\item
The filename \textit{main} is case sensitive
(even in case-insensitive file systems)
due to internal string comparison.
\item
The argument \textit{main} should be fully expanded, it cannot be a macro.
\item
Subdirectories and special characters should be avoided in filenames.
\item
The command |\childdocmain{|\textit{main}|}| must be followed by a whitespace.
It should not be followed immediately by another command
or by a comment mark `|%|'.
This is because the \TeX{} parser reads the token immediately following
the argument of |\childdocmain| and puts it
at the beginning of every child section;
however, a white\-space is ignored.
\end{itemize}

%%%%%%%%%%%%%%%%%%%%%%%%%%%%%%%%%%%%%%%%
\paragraph{Content of Main File.}

It is advisable to place all content in the child files included by |\include|.
Any output contained in the main file will appear in all child documents
unless suppressed manually;
it cannot be suppressed automatically by the |\includeonly| directive
and thus should normally be avoided.
A method to include some content in the main file
by means of conditional processing is described in \secref{sec:conditional}.

%%%%%%%%%%%%%%%%%%%%%%%%%%%%%%%%%%%%%%%%
\paragraph{Page Numbering.}

When only a part of the document is compiled,
the appropriate numbering of pages
(as well as other status parameters)
is determined from the |.aux| files.
The latter contain information from previous passes.
However this information needs to propagate through
all intermediate child documents.
Therefore the page numbering in child documents may well
be inconsistent until the complete document is compiled at least once.

A useful (if unconventional) way to always ensure a consistent
page numbering is to restart the numbering in each child document
and denote the pages by `\textit{child}|.|\textit{page}'
where \textit{child} represents the chapter/section number of the child file.
This can be achieved by the command
|\numberwithin{page}{|\textit{child}|}|
of the \textsf{amsmath} package
where \textit{child} can be |chapter| or |section|
depending on the chosen structuring.
Alternatively, one can modify the macro |\thepage| appropriately
and reset the counter |page| at the start of each child file.

%%%%%%%%%%%%%%%%%%%%%%%%%%%%%%%%%%%%%%%%%%%%%%%%%%%%%%%%%%%%%%%%%%%%%%%%%%%%%%%%
\subsection{Conditional Processing}
\label{sec:conditional}

The package provides a mechanism to compile different versions
of a document. To customise the versions further some conditional processing
can come in handy to distinguish which version is being compiled.
The package provides two macros to describe the compilation context:

%%%%%%%%%%%%%%%%%%%%%%%%%%%%%%%%%%%%%%%%
\DescribeMacro{\ifchilddoc}
The conditional |\ifchilddoc| distinguishes between the compilation of
child documents and the main document:
%
\begin{center}
|\ifchilddoc |\textit{child-code}| |[|\||else |\textit{main-code}]| \||fi|
\end{center}

%%%%%%%%%%%%%%%%%%%%%%%%%%%%%%%%%%%%%%%%
\DescribeMacro{\childdocname}
\DescribeMacro{\childdocjob}
The macro |\childdocname| contains the filename (without extension)
of the main or child file being processed.
Note that |\childdocjob| will always contain the name of the main file.

%%%%%%%%%%%%%%%%%%%%%%%%%%%%%%%%%%%%%%%%
\paragraph{Title Page.}

Conditional processing can be used to include a title or banner page
in the main document when proper precautions are taken.
Importantly, the code in the main file should ensure that the page counter
(as well as other status parameters which are stored in the |.aux| files)
takes the same value after the conditional processing.
Otherwise the page numbers may take divergent values
depending on which part is compiled.

For example, a title page could be declared by:
%
\begin{center}
\begin{tabular}{l}
|\ifchilddoc\||else|\\
|\addtocounter{page}{-1}|\\
\textit{code for title page}\\
|\newpage|\\
|\||fi|
\end{tabular}
\end{center}
%
A banner page for the child documents can be generated by:
%
\begin{center}
\begin{tabular}{l}
|\ifchilddoc|\\
|\addtocounter{page}{-1}|\\
\textit{code for banner page}\\
|\newpage|\\
|\||fi|
\end{tabular}
\end{center}
%
Here one could write a message such as:
\begin{center}
|This is the part \childdocname{} of \childdocjob{}.|
\end{center}

%%%%%%%%%%%%%%%%%%%%%%%%%%%%%%%%%%%%%%%%%%%%%%%%%%%%%%%%%%%%%%%%%%%%%%%%%%%%%%%%
\subsection{Flags}
\label{sec:flags}

The package makes it easy to generate different versions
of the main or child documents.
To this end compilation flags can be defined
and assigned different default values.
They will be particularly useful in conjunction
with the forwarding mechanism described in \secref{sec:forward}.

For example, it may be useful to have a flag |\version|
which can be set to |draft| or |final|.
The document source will contain some conditional code
depending on the value of |\version|.
Suppose further, the flag should default to |final| for the main file
and to |draft| for child files
which is a natural assignment for editing the document.
This is achieved by placing the following code
in the preamble of the main document
(below the |\childdocmain| directive):
%
\begin{center}
\begin{tabular}{l}
|\ifchilddoc|\\
|\providecommand{\version}{draft}|\\
|\||else|\\
|\providecommand{\version}{final}|\\
|\||fi|
\end{tabular}
\end{center}
%
The definition by |\providecommand| makes sure
that previous definitions are not overwritten.
Further statements |\providecommand{\version}{...}|
can thus be added before the above code to override it.

For the main file, one might add a line
(between |\childdocmain| and the above block)
%
\begin{center}
|%\ifchilddoc\||else\providecommand{\version}{draft}\||fi|
\end{center}
%
which can be uncommented to produce a draft version.
Likewise one can add a line to the very top of a child file
(above the |\childdocof{|\textit{main}|}| directive)
%
\begin{center}
|%\providecommand{\version}{final}|
\end{center}
%
which can be uncommented to produce the final version of this child document.

%%%%%%%%%%%%%%%%%%%%%%%%%%%%%%%%%%%%%%%%%%%%%%%%%%%%%%%%%%%%%%%%%%%%%%%%%%%%%%%%
\subsection{Forwarding}
\label{sec:forward}

Different versions of the main or child documents
using compilation flags as described in \secref{sec:flags}
can be (permanently) stored in different files
for convenient compilation, viewing and distribution.
To this end, the package defines a command
to pass on compilation to a different file:

%%%%%%%%%%%%%%%%%%%%%%%%%%%%%%%%%%%%%%%%
\DescribeMacro{\childdocforward}
The command |\childdocforward| redirects processing to
another source file:
%
\begin{center}
\begin{tabular}{l}
|\input{childdoc.def}|\\
|\childdocforward[|\textit{main}|]{|\textit{dest}|}|\\
\end{tabular}
\end{center}
%
The argument \textit{dest} is the destination file
(without extension).
It should be the main file or one of the child files.
Note that further \textsf{childdoc} directives
such as |\childdocof| and |\childdocforward|
in the indicated file will be processed in this form.
The optional argument \textit{main}
passes on directly to the main file \textit{main}
while pretending to compile the child \textit{dest}.
This form behaves as if \textit{dest}
issues |\childdocof{|\textit{main}|}| right away,
and no further \textsf{childdoc} directives will be processed.

%%%%%%%%%%%%%%%%%%%%%%%%%%%%%%%%%%%%%%%%
\DescribeMacro{\...prefix}
In the alternative form |\childdocforwardprefix|,
%
\begin{center}
\begin{tabular}{l}
|\input{childdoc.def}|\\
|\childdocforwardprefix[|\textit{main}|]{|\textit{prefix}|}{|\textit{dest}|}|
\end{tabular}
\end{center}
%
the destination file is determined by a pattern
depending on the current file:
To make this work, the current file must be called
`{\textit{prefix}\hspace{0.2em}\textit{suffix}}'
with \textit{prefix} matching precisely the argument.
Processing is then passed on to the file
`{\textit{dest}\hspace{0.2em}\textit{suffix}}'.
Surely, the same effect is achieved by
directly specifying the
argument `{\textit{dest}\hspace{0.2em}\textit{suffix}}'
in the first form.
However, that requires to set up a different file
for each child. With the alternative form of the command
all these files can have exactly the same content
which simplifies setting them up and maintaining them.

For example, the following file |draft.tex|
with a compilation flag |\version| as described in \secref{sec:flags}
compiles the main document as a draft:
%
\begin{center}
\begin{tabular}{l}
|\def\version{draft}|\\
|\input{childdoc.def}|\\
|\childdocforward{|\textit{main}|}|
\end{tabular}
\end{center}
%
Likewise, the following files |final|\textit{nn}|.tex|
compile the final version of the child document
|child|\textit{nn}|.tex|:
%
\begin{center}
\begin{tabular}{l}
|\def\version{final}|\\
|\input{childdoc.def}|\\
|\childdocforwardprefix{final}{child}|
\end{tabular}
\end{center}
%

Note that when several versions of a main file and/or of each child file
are to be generated, it may be convenient to set up a |Makefile| or
shell script to automatise the process.

%%%%%%%%%%%%%%%%%%%%%%%%%%%%%%%%%%%%%%%%%%%%%%%%%%%%%%%%%%%%%%%%%%%%%%%%%%%%%%%%
\subsection{Command Line Processing}
\label{sec:commandline}

The effect of redirection files can also be achieved by invoking
the \LaTeX{} compiler with a more elaborate command line.
Most conveniently this should be done as part
of a shell script or a |Makefile|.

When using \textsf{childdoc} in the main file, the following
command lines effectively perform a redirection
(note that depending on the shell being used,
backslashes may have to be doubled: `|\|' $\to$ `|\\|'):
%
\begin{center}
|... -jobname "|\textit{target}|" |\\|"|[\textit{flags}]%
|\input{childdoc.def}\childdocforward[|\textit{main}|]{|\textit{dest}|}"|
\end{center}
%
Here \textit{target} is the name of the output file,
\textit{main} is the name of the main file
and \textit{dest} is the name of the main or child file to be processed
(all filenames without extensions).
The optional argument \textit{main} can be omitted
if \textit{main} matches \textit{dest}.
Optionally, compilation \textit{flags} can be defined via |\def| commands.
This command line makes the \TeX{} engine believe
it is compiling the file \textit{target}
whose content is specified as the latter parameter.
The provided code then forwards the processing to
\textit{main} or \textit{dest} as described in \secref{sec:forward}.

%%%%%%%%%%%%%%%%%%%%%%%%%%%%%%%%%%%%%%%%%%%%%%%%%%%%%%%%%%%%%%%%%%%%%%%%%%%%%%%%
\subsection{Include by Input}
\label{sec:input}

Including child documents by |\include| has some restrictions by design.
Most notably, the content of a child document always occupies
its own set of pages; pages cannot be shared between child documents.
Usually, this behaviour makes perfect sense
because each child document contain an essential part of the document.
However, in some situations it may be desirable to compose
a document from a collection of parts
without having mandatory page breaks between then.
For this case, the package
provides a mechanism to include parts
by |\input| which can also be processed individually.
However, by construction this mechanism
requires manual handling of the content to be output.

%%%%%%%%%%%%%%%%%%%%%%%%%%%%%%%%%%%%%%%%
\DescribeMacro{\ifchilddocmanual}
The main file should be prepared as usual, see \secref{sec:include}.
However, the document body must make a distinction
between processing of an individual part and of the main document, e.g.:
%
\begin{center}
\begin{tabular}{l}
|\ifchilddocmanual|\\
|\input{\childdocname}|\\
|\||else|\\
\textit{document body with }|\input{|\textit{part}|}|\\
|\||fi|
\end{tabular}
\end{center}
%
The conditional |\ifchilddocmanual| is true whenever
a part to be included by |\input| is being compiled,
and the name of the part is stored in |\childdocname|.

%%%%%%%%%%%%%%%%%%%%%%%%%%%%%%%%%%%%%%%%
\DescribeMacro{\childdocby}
Each part to be included by |\input| should start with:
%
\begin{center}
\begin{tabular}{l}
|\input{childdoc.def}|\\
|\childdocby{|\textit{main}|}|\\
\end{tabular}
\end{center}
%
The directive |\childdocby| is similar to |\childdocof|
described in \secref{sec:include},
but the subsequent selection of content must be done manually.
To that end, both |\ifchilddoc| and |\ifchilddocmanual|
will be true upon processing of a part,
and the name of the part is stored in |\childdocname|.
Note that |\jobname| will be set to the filename of the current part
so that each part receives an individual |.aux| file
that does not interfere with the |.aux| file(s) of the main document.
This behaviour can be altered by the alternative form
|\childdocby[*]{|\textit{main}|}| (with a non-empty optional argument)
which uses the |.aux| file of the main document
by setting |\jobname| to \textit{main}.

%%%%%%%%%%%%%%%%%%%%%%%%%%%%%%%%%%%%%%%%%%%%%%%%%%%%%%%%%%%%%%%%%%%%%%%%%%%%%%%%
\subsection{Driver Development}
\label{sec:driver}

The \textsf{childdoc} mechanism can also be use for the development
of definition files such as \LaTeX{} styles or classes.
This case differs from the above setup with multiple parts
included by |\include| in that no |\includeonly| should be invoked.
This can be achieved by starting the include file
(before |\ProvidesPackage|) with:
%
\begin{center}
\begin{tabular}{l}
|\input{childdoc.def}|\\
|\childdocforward{|\textit{main}|}|\\
\end{tabular}
\end{center}
%
or alternatively with:
%
\begin{center}
\begin{tabular}{l}
|\input{childdoc.def}|\\
|\childdocby{|\textit{main}|}|\\
\end{tabular}
\end{center}
%
Both forms have slightly different effects as described above.
The main file is prepared as usual, see \secref{sec:include}.

%%%%%%%%%%%%%%%%%%%%%%%%%%%%%%%%%%%%%%%%%%%%%%%%%%%%%%%%%%%%%%%%%%%%%%%%%%%%%%%%
\subsection{Legacy Detection}
\label{sec:detection}

The directive |\childdocmain| in the main file can detect
whether the complete document or merely a child is to be compiled
even without using the directive |\childdocof|.
This method is deprecated because it is less robust
and there is no compelling reason to use it;
it is merely provided for backward compatibility
and it may be removed in future versions.

If the detection mechanism is to be used,
it is mandatory to correctly specify
the filename of the main file as the argument of |\childdocmain|:
%
\begin{center}
\begin{tabular}{l}
|\input{childdoc.def}|\\
|\childdocmain{|\textit{main}|}|\\
\end{tabular}
\end{center}
%
If |\jobname| does not match the argument \textit{main} of |\childdocmain|,
it is assumed that |\jobname| points to the child file to be compiled.
When using |\childdocmain| with the main file specified as argument,
it suffices to start a child file
with just |\input{|\textit{main}|}|
without loading of the package and using |\childdocof|.
If instead all processing is done
with the appropriate \textsf{childdoc} directives,
the argument of \textit{main} of |\childdocmain| can be empty.

An alternative version of the command line processing described
in \secref{sec:commandline} using the detection mechanism reads:
%
\begin{center}
|... -jobname "|\textit{target}|" "|[\textit{flags}]%
[|\def\jobname{|\textit{dest}|}|]|\input{|\textit{main}|}"|
\end{center}

%%%%%%%%%%%%%%%%%%%%%%%%%%%%%%%%%%%%%%%%%%%%%%%%%%%%%%%%%%%%%%%%%%%%%%%%%%%%%%%%
\subsection{Manual Code}
\label{sec:manual}

In case one cannot be certain whether the definitions file |childdoc.def|
is installed on the target \TeX{} distribution
and one prefers not to ship it,
it is conceivable to paste a few relevant commands into the sources.

To that end, drop all statements |\input{childdoc.def}|
and perform the replacements as outlined below.
Instead of |\childdocmain{|\textit{main}|}| add the following code
to the top of the main file:
%
\begin{center}
\begin{tabular}{l}
|\||ifdefined\childdocname\endinput\||fi\newif\ifchilddoc|\\
|\edef\childdocname{\scantokens\expandafter{\jobname\noexpand}}|\\
|\def\childdocmain{|\textit{main}|}\||ifx\childdocmain\childdocname\||else|\\
|\childdoctrue\includeonly{\childdocname}\let\jobname\childdocmain\||fi|\\
\end{tabular}
\end{center}
%
Instead of |\childdocof{|\textit{main}|}| just include the main file
at the top of each child file:
%
\begin{center}
|\input{|\textit{main}|}|
\end{center}
%
A simple redirection |\childdocforward{|\textit{dest}|}| is achieved by:
%
\begin{center}
|\def\jobname{|\textit{dest}|}\input{\jobname}|
\end{center}
%
The redirection with prefix
|\childdocforwardprefix[|\textit{prefix}|]{|\textit{dest}|}|
is accomplished by:
%
\begin{center}
\begin{tabular}{l}
|{\edef\jobname{\scantokens\expandafter{\jobname\noexpand}}|\\
|\def\redirectjob |\textit{prefix}|#1~~~{\gdef\jobname{|\textit{dest}|#1}}|\\
|\expandafter\redirectjob\jobname~~~}\input{\jobname}|
\end{tabular}
\end{center}

In an alternative approach,
child documents can be compiled by a specific command line
without additional code or specific definitions:
%
\begin{center}
|... -jobname "|\textit{target}|" "|[\textit{flags}]%
|\includeonly{|\textit{dest}|}\input{|\textit{main}|}"|
\end{center}
%

%%%%%%%%%%%%%%%%%%%%%%%%%%%%%%%%%%%%%%%%%%%%%%%%%%%%%%%%%%%%%%%%%%%%%%%%%%%%%%%%
%%%%%%%%%%%%%%%%%%%%%%%%%%%%%%%%%%%%%%%%%%%%%%%%%%%%%%%%%%%%%%%%%%%%%%%%%%%%%%%%
\section{Information}

%%%%%%%%%%%%%%%%%%%%%%%%%%%%%%%%%%%%%%%%%%%%%%%%%%%%%%%%%%%%%%%%%%%%%%%%%%%%%%%%
\subsection{Copyright}

Copyright \copyright{} 2017--2018 Niklas Beisert

This work may be distributed and/or modified under the
conditions of the \LaTeX{} Project Public License, either version 1.3
of this license or (at your option) any later version.
The latest version of this license is in
  \url{http://www.latex-project.org/lppl.txt}
and version 1.3 or later is part of all distributions of \LaTeX{}
version 2005/12/01 or later.

This work has the LPPL maintenance status `maintained'.

The Current Maintainer of this work is Niklas Beisert.

This work consists of the files |README.txt|, |childdoc.ins| and |childdoc.dtx|
as well as the derived files |childdoc.def|, |cdocsamp.tex|
with |cdocsch1.tex|, |cdocsch2.tex|, |cdocspt3.tex|, |cdocspt4.tex|,
|cdocsdrf.tex|, |cdocsfn1.tex|, |cdocsfn2.tex|
as well as |childdoc.pdf|.

%%%%%%%%%%%%%%%%%%%%%%%%%%%%%%%%%%%%%%%%%%%%%%%%%%%%%%%%%%%%%%%%%%%%%%%%%%%%%%%%
\subsection{Files and Installation}

The package consists of the files:
%
\begin{center}
\begin{tabular}{ll}
    |README.txt|   & readme file \\
    |childdoc.ins| & installation file \\
    |childdoc.dtx| & source file \\
    |childdoc.def| & definition file \\
    |cdocsamp.tex| & sample main file \\
    |cdocsch1.tex| & sample include file \\
    |cdocsch2.tex| & sample include file \\
    |cdocspt3.tex| & sample part file \\
    |cdocspt4.tex| & sample part file \\
    |cdocsdrf.tex| & sample redirection file \\
    |cdocsfn1.tex| & sample redirection file \\
    |cdocsfn2.tex| & sample redirection file \\
    |childdoc.pdf| & manual
\end{tabular}
\end{center}
%
The distribution consists of the files
|README.txt|, |childdoc.ins| and |childdoc.dtx|.
%
\begin{itemize}
\item
Run (pdf)\LaTeX{} on |childdoc.dtx|
to compile the manual |childdoc.pdf| (this file).
\item
Run \LaTeX{} on |childdoc.ins| to create the definitions file |childdoc.def|
and the sample |cdocsamp.tex| with include files
|cdocsch1.tex|, |cdocsch2.tex|, |cdocspt3.tex|, |cdocspt4.tex|,
|cdocsdrf.tex|, |cdocsfn1.tex|, |cdocsfn2.tex|.
Then copy the file |childdoc.def| to an appropriate directory of your \LaTeX{}
distribution, e.g.\ \textit{texmf-root}|/tex/latex/childdoc|.
\end{itemize}

%%%%%%%%%%%%%%%%%%%%%%%%%%%%%%%%%%%%%%%%%%%%%%%%%%%%%%%%%%%%%%%%%%%%%%%%%%%%%%%%
\subsection{Related CTAN Packages}

There are several other packages which offer a similar functionality:
%
\begin{itemize}
\item
The packages
\href{http://ctan.org/pkg/docmute}{\textsf{docmute}},
\href{http://ctan.org/pkg/includex}{\textsf{includex}} and
\href{http://ctan.org/pkg/standalone}{\textsf{standalone}}
provide commands to include only the document body of
a child file thus allowing both files to be compiled individually.
\item
The packages \href{http://ctan.org/pkg/subdocs}{\textsf{subdocs}}
and \href{http://ctan.org/pkg/subfiles}{\textsf{subfiles}}
provide structures in which the main and child documents can be
encapsulated and allowing them to be compiled individually.
The inclusion mechanism is different from the conventional |\include|.
\item
The package \href{http://ctan.org/pkg/combine}{\textsf{combine}}
is an elaborate solution to combine several documents into one.
\end{itemize}
%
See also the CTAN topic \href{http://ctan.org/topic/subdocs}{\textsf{subdocs}}
for further related packages.
The present package differs from the above solutions in that
a document structure constructed with the conventional |\include| mechanism
just needs two extra commands at the top of every file
such that all constituent files can be compiled individually.

%%%%%%%%%%%%%%%%%%%%%%%%%%%%%%%%%%%%%%%%%%%%%%%%%%%%%%%%%%%%%%%%%%%%%%%%%%%%%%%%
%\subsection{Feature Suggestions}
%
%The following is a list of features which may be useful for future
%versions of this package:
%%
%\begin{itemize}
%\item
%\ldots
%\end{itemize}

%%%%%%%%%%%%%%%%%%%%%%%%%%%%%%%%%%%%%%%%%%%%%%%%%%%%%%%%%%%%%%%%%%%%%%%%%%%%%%%%
\subsection{Revision History}

%%%%%%%%%%%%%%%%%%%%%%%%%%%%%%%%%%%%%%%%
\paragraph{v2.0:} 2018/12/30

\begin{itemize}
\item
immediate forward processing
\item
added |\childdocby| mechanism
\item
manual restructured
\end{itemize}

%%%%%%%%%%%%%%%%%%%%%%%%%%%%%%%%%%%%%%%%
\paragraph{v1.6:} 2018/01/17

\begin{itemize}
\item
application for development of include files
\item
corrections to manual
\end{itemize}

%%%%%%%%%%%%%%%%%%%%%%%%%%%%%%%%%%%%%%%%
\paragraph{v1.5:} 2017/05/21

\begin{itemize}
\item
more complete structuring introduced
\item
|\childdocof| introduced
\item
|\childdoc| renamed to |\childdocmain|
\item
|\childredirect| renamed to |\childdocforward| and |\childdocforwardprefix|
and functionality expanded
\end{itemize}

%%%%%%%%%%%%%%%%%%%%%%%%%%%%%%%%%%%%%%%%
\paragraph{v1.0:} 2017/04/27

\begin{itemize}
\item
manual and install package
\item
first version published on CTAN
\end{itemize}

%%%%%%%%%%%%%%%%%%%%%%%%%%%%%%%%%%%%%%%%
\paragraph{v0.6:} 2017/04/26

\begin{itemize}
\item
redirection mechanism added
\end{itemize}

%%%%%%%%%%%%%%%%%%%%%%%%%%%%%%%%%%%%%%%%
\paragraph{v0.5:} 2017/04/26

\begin{itemize}
\item
functionality in definition file
\end{itemize}


%%%%%%%%%%%%%%%%%%%%%%%%%%%%%%%%%%%%%%%%%%%%%%%%%%%%%%%%%%%%%%%%%%%%%%%%%%%%%%%%
%%%%%%%%%%%%%%%%%%%%%%%%%%%%%%%%%%%%%%%%%%%%%%%%%%%%%%%%%%%%%%%%%%%%%%%%%%%%%%%%
%%%%%%%%%%%%%%%%%%%%%%%%%%%%%%%%%%%%%%%%%%%%%%%%%%%%%%%%%%%%%%%%%%%%%%%%%%%%%%%%
\appendix

\settowidth\MacroIndent{\rmfamily\scriptsize 000\ }

 \DocInput{childdoc.dtx}

\end{document}
%</driver>
% \fi
%
% %%%%%%%%%%%%%%%%%%%%%%%%%%%%%%%%%%%%%%%%%%%%%%%%%%%%%%%%%%%%%%%%%%%%%%%%%%%%%%
% %%%%%%%%%%%%%%%%%%%%%%%%%%%%%%%%%%%%%%%%%%%%%%%%%%%%%%%%%%%%%%%%%%%%%%%%%%%%%%
% \section{Sample}
%\iffalse
%<*samplemain>
%\fi
%
% The following presents a sample document
% with two chapters, two parts, a title page,
% a compile flag as well as three forwarding files to set the flag.
% It consists of eight |.tex| files:
% \begin{center}
% \begin{tabular}{ll}
% |cdocsamp.tex|&main file\\
% |cdocsch1.tex|&include file for chapter 1\\
% |cdocsch2.tex|&include file for chapter 2\\
% |cdocspt3.tex|&include file for part 3\\
% |cdocspt4.tex|&include file for part 4\\
% |cdocsdrf.tex|&forwarding file for main file in draft mode\\
% |cdocsfi1.tex|&forwarding file for final version of chapter 1\\
% |cdocsfi2.tex|&forwarding file for final version of chapter 2\\
% \end{tabular}
% \end{center}
% Each of the eight files can be compiled directly by the \LaTeX{} compiler.
%
% %%%%%%%%%%%%%%%%%%%%%%%%%%%%%%%%%%%%%%
% \paragraph{Main File.}
%
% The main file is called |cdocsamp.tex|.
%
% Load the \textsf{childdoc} definitions and
% declare the filename for the main document:
%    \begin{macrocode}
\input{childdoc.def}
\childdocmain{}
%    \end{macrocode}

% Optional override for |\version| flag:
%    \begin{macrocode}
%%\ifchilddoc\else\providecommand{\version}{draft}\fi
%    \end{macrocode}

% Define the default values for the |\version| flag
% (|final| for the main file and |draft| for childs):
%    \begin{macrocode}
\ifchilddoc
\providecommand{\version}{draft}
\else
\providecommand{\version}{final}
\fi
%    \end{macrocode}

% Load the standard document class:
%    \begin{macrocode}
\documentclass[12pt]{article}
%    \end{macrocode}

% Start the document body:
%    \begin{macrocode}
\begin{document}
%    \end{macrocode}

% Declare a title page.
% Print title, part of document being processed and version flag:
%    \begin{macrocode}
\addtocounter{page}{-1}
\begin{center}
{\LARGE\bfseries{}childdoc example\par}
\vspace{1cm}
\ifchilddoc
\ifchilddocmanual part\else chapter\fi:
`\childdocname' of `\childdocjob'\par
\else
main document: `\childdocjob'\par
\fi
version: \version\par
\end{center}
\newpage
%    \end{macrocode}

% Manually include selected file,
% otherwise process as usual:
%    \begin{macrocode}
\ifchilddocmanual
\section*{part `\childdocname'}
\input{\childdocname}
\else
%    \end{macrocode}

% Include the two chapters:
%    \begin{macrocode}
\include{cdocsch1}
\include{cdocsch2}
%    \end{macrocode}

% Include the two parts unless only chapters should be displayed:
%    \begin{macrocode}
\ifchilddoc\else
\section{part three}
\input{cdocspt3}
\section{part four}
\input{cdocspt4}
\fi
%    \end{macrocode}

% Process as usual until here:
%    \begin{macrocode}
\fi
%    \end{macrocode}

% End of document body:
%    \begin{macrocode}
\end{document}
%    \end{macrocode}
%\iffalse
%</samplemain>
%\fi
%
% %%%%%%%%%%%%%%%%%%%%%%%%%%%%%%%%%%%%%%
% \paragraph{Chapter Include Files.}
%
% The include files are called |cdocsch1.tex| and |cdocsch2.tex|.
%
%\iffalse
%<*samplechap1|samplechap2>
%\fi

% Optional override for |\version| flag:
%    \begin{macrocode}
%%\providecommand{\version}{final}
%    \end{macrocode}

% Include the main document:
%    \begin{macrocode}
\input{childdoc.def}
\childdocof{cdocsamp}
%    \end{macrocode}

%\iffalse
%</samplechap1|samplechap2>
%\fi
%
%\iffalse
%<*samplechap1>
%\fi
% Some text for chapter 1:
%    \begin{macrocode}
\section{one}
some text in chapter one
%    \end{macrocode}

%\iffalse
%</samplechap1>
%\fi
% Some text for chapter 2:
%\iffalse
%<*samplechap2>
%\fi
%    \begin{macrocode}
\section{two}
more text in chapter two
%    \end{macrocode}

%\iffalse
%</samplechap2>
%\fi
%
% %%%%%%%%%%%%%%%%%%%%%%%%%%%%%%%%%%%%%%
% \paragraph{Part Include Files.}
%
% The include files are called |cdocspt3.tex| and |cdocspt4.tex|.
%
%\iffalse
%<*samplepart3|samplepart4>
%\fi

% Optional override for |\version| flag:
%    \begin{macrocode}
%%\providecommand{\version}{final}
%    \end{macrocode}

% Include the main document:
%    \begin{macrocode}
\input{childdoc.def}
\childdocby{cdocsamp}
%    \end{macrocode}

%\iffalse
%</samplepart3|samplepart4>
%\fi
%
%\iffalse
%<*samplepart3>
%\fi
% Some text for part 3:
%    \begin{macrocode}
some text in part three
%    \end{macrocode}

%\iffalse
%</samplepart3>
%\fi
% Some text for part 4:
%\iffalse
%<*samplepart4>
%\fi
%    \begin{macrocode}
more text in part four
%    \end{macrocode}

%\iffalse
%</samplepart4>
%\fi
%
% %%%%%%%%%%%%%%%%%%%%%%%%%%%%%%%%%%%%%%
% \paragraph{Forwarding for a Complete Draft.}
%
% The following forwarding file |cdocsdrf.tex|
% compiles the main document in draft mode:
%\iffalse
%<*sampledraft>
%\fi
%    \begin{macrocode}
\def\version{draft}
\input{childdoc.def}
\childdocforward{cdocsamp}
%    \end{macrocode}

%\iffalse
%</sampledraft>
%\fi
%
% %%%%%%%%%%%%%%%%%%%%%%%%%%%%%%%%%%%%%%
% \paragraph{Forwarding for Final Version of the Chapters.}
%
% The following forwarding files |cdocsfn1.tex| and |cdocsfn2.tex|
% (with identical content)
% compile the final versions of the child documents
% |cdocsch1.tex| and |cdocsch2.tex|, respectively:
%\iffalse
%<*samplefinal>
%\fi
%    \begin{macrocode}
\def\version{final}
\input{childdoc.def}
\childdocforwardprefix[cdocsamp]{cdocsfn}{cdocsch}
%    \end{macrocode}

%\iffalse
%</samplefinal>
%\fi
%
% %%%%%%%%%%%%%%%%%%%%%%%%%%%%%%%%%%%%%%
% \paragraph{Command Line Processing.}
%
% The following three command lines generate the output files
% |cdocscld|, |cdocscl1| and |cdocscl2|
% which should be identical to
% |cdocsdrf|, |cdocsch1| and |cdocsfn2|, respectively:
% \begin{center}
% \begin{tabular}{l}
% |latex -jobname cdocscld \|\\
% |  "\def\version{draft}\input{childdoc.def}\childdocforward{cdocsamp}"|\\
% |latex -jobname cdocscl1 \|\\
% |  "\input{childdoc.def}\childdocforward[cdocsamp]{cdocsch1}"|\\
% |latex -jobname cdocscl2 \|\\
% |  "\def\version{final}\input{childdoc.def}\childdocforward{cdocsch2}"|
% \end{tabular}
% \end{center}
% Note that the trailing backslash on each first line
% merely continues the input to the second line
% (for convenient cut ant paste).
% Furthermore, the command |latex| can be replaced by any
% of its alternative versions such as |pdflatex|.
%
% %%%%%%%%%%%%%%%%%%%%%%%%%%%%%%%%%%%%%%%%%%%%%%%%%%%%%%%%%%%%%%%%%%%%%%%%%%%%%%
% %%%%%%%%%%%%%%%%%%%%%%%%%%%%%%%%%%%%%%%%%%%%%%%%%%%%%%%%%%%%%%%%%%%%%%%%%%%%%%
% \section{Implementation}
%\iffalse
%<*package>
%\fi
%
% This section describes the definitions file |childdoc.def|.

% The definitions cannot be loaded using |\usepackage| or |\RequirePackage|
% which has a mechanism to prevent loading a style file more than once.
% When loading the definitions by means of |\input|
% multiple instances have to be prevented manually:
%\iffalse
%This code needs to be before the `\ProvidesFile' directive
%which is defined at the beginning of this file.
%Therefore it is also placed there and commented out here.
%</package>
%<*discard>
%\fi
%    \begin{macrocode}
\ifdefined\childdocmain\endinput\fi
%    \end{macrocode}
%\iffalse
%</discard>
%<*package>
%\fi
%
% \macro{\ifchilddoc}
% \macro{\ifchilddocmanual}
% The conditional |\ifchilddoc| tells whether a
% child (true) or main (false) document is being compiled.
% The conditional |\ifchilddocmanual| tells whether
% the |\includeonly| mechanism is used (false) or
% the selection of child files must be performed manually (true).
% The definitions initialise to false:
%    \begin{macrocode}
\newif\ifchilddoc
\newif\ifchilddocmanual
%    \end{macrocode}

% \macro{\childdocname}
% \macro{\childdocjob}
% The macro |\childdocname| stores the name of the main document
% to be compiled. The macro |\childdocjob| stores the name of
% the document on which the \LaTeX{} compiler was originally invoked.
% The content of |\jobname| cannot be compared
% to filenames specified in the source due to different catcodes.
% The following code rescans |\jobname|, stores the result
% in |\childdocname| and saves a copy in |\childdocjob|:
%    \begin{macrocode}
\edef\childdocname{\scantokens\expandafter{\jobname\noexpand}}
\let\childdocjob\childdocname
%    \end{macrocode}

% \macro{\childdocdisable}
% The macro |\childdocdisable| prevents the main file
% from being processed more than once.
% At this stage, the main document command |\childdocmain|
% is assumed to be called once again where it should do nothing.
% Any subsequent call to it should prevent
% a secondary processing of the main document
% It overwrites the forwarding commands
% |\childdocof| and |\childdocforward|
% with empty macros to prevent further inclusions of the main document:
%    \begin{macrocode}
\newcommand{\childdocdisable}
{
  \renewcommand{\childdocmain}[1]{\renewcommand{\childdocmain}[1]{\endinput}}
  \renewcommand{\childdocof}[1]{}
  \renewcommand{\childdocby}[2][]{}
  \renewcommand{\childdocforward}[2][]{}
  \renewcommand{\childdocdisable}{}
}
%    \end{macrocode}

% \macro{\childdocmain}
% The macro |\childdocmain| is to be called at the top of the main file
% with nothing or the main filename (without extension) as argument.
% First, it breaks loops.
% If the argument is not empty and does not match |\childdocname|
% (which is set by the first inclusion of |childdoc.def|),
% |\ifchilddoc| is set to true, |\includeonly| is applied to the child file
% and |\jobname| is set to the main file
% (for proper handling of |.aux| files):
%    \begin{macrocode}
\newcommand{\childdocmain}[1]
{
  \childdocdisable\childdocmain{}
  \if?#1?\else
    \begingroup
      \def\childdoctmp{#1}
      \ifx\childdoctmp\childdocname
        \def\childdoctmp{}
      \else
        \def\childdoctmp
        {
          \childdoctrue
          \includeonly{\childdocname}
          \def\childdocjob{#1}
          \def\jobname{#1}
        }
      \fi
      \expandafter
    \endgroup
    \childdoctmp
  \fi
}
%    \end{macrocode}

% \macro{\childdocof}
% The command |\childdocof| redirects
% compilation to the main file |#1|.
%    \begin{macrocode}
\newcommand{\childdocof}[1]
{
  \childdocdisable
  \childdoctrue
  \includeonly{\childdocname}
  \def\jobname{#1}
  \def\childdocjob{#1}
  \input{#1}
}
%    \end{macrocode}

% \macro{\childdocby}
% The command |\childdocby| ....
%    \begin{macrocode}
\newcommand{\childdocby}[2][]
{
  \childdocdisable
  \childdoctrue
  \childdocmanualtrue
  \if?#1?\else
    \def\jobname{#2}
  \fi
  \def\childdocjob{#2}
  \input{#2}
  \endinput
}
%    \end{macrocode}

% \macro{\childdocforward}
% The command |\childdocforward| redirects
% compilation to the main file or
% (if the optional argument is given) a child file.
% Parameters are set as if the main file
% or a child file starting with |\childdocof| was compiled.
% Then compilation is handed over to the main file:
%    \begin{macrocode}
\newcommand{\childdocforward}[2][]
{
  \begingroup
    \if?#1?
      \def\childdoctmp
      {
        \def\childdocname{#2}
        \def\childdocjob{#2}
        \def\jobname{#2}
        \input{#2}
        \endinput
      }
    \else
      \def\childdoctmp
      {
        \childdocdisable
        \def\childdocname{#2}
        \childdoctrue
        \includeonly{#2}
        \def\childdocjob{#1}
        \def\jobname{#1}
        \input{#1}
        \endinput
      }
    \fi
    \expandafter
  \endgroup
  \childdoctmp
}
%    \end{macrocode}

% \macro{\childdocforwardprefix}
% The command |\childdocforwardprefix| redirects
% compilation to the main or a child file by means of a pattern.
% The prefix |#1| in the current filename is replaced by |#2|
% and the suffix of the current filename is kept
% (it is assumed that the filename does not contain the substring `|~~~|'
% which is used as a delimiter).
% Compilation is handed over to the new file by |\childdocforward|:
%    \begin{macrocode}
\newcommand{\childdocforwardprefix}[3][]
{
  \begingroup
    \def\childdocextract #2##1~~~{\def\childdoctmp{\childdocforward[#1]{#3##1}}}
    \expandafter\childdocextract\childdocname~~~
    \expandafter
  \endgroup
  \childdoctmp
}
%    \end{macrocode}

% \macro{\childdoc}
% The deprecated macro |\childdoc| is a legacy version of |\childdocmain|:
%    \begin{macrocode}
\newcommand{\childdoc}{\childdocmain}
%    \end{macrocode}

% \macro{\childdocredirect}
% The deprecated macro |\childdocredirect| is a legacy version
% of |\childdocforward| and |\childdocforwardprefix|:
%    \begin{macrocode}
\newcommand{\childdocredirect}[2][]
{
  \begingroup
    \if?#1?
      \def\childdoctmp{\childdocforward{#2}}
    \else
      \def\childdoctmp{\childdocforwardprefix{#1}{#2}}
    \fi
    \expandafter
  \endgroup
  \childdoctmp
}
%    \end{macrocode}

%\iffalse
%</package>
%\fi
%
\endinput
|\\
|\childdocof{|\textit{main}|}|\\
\end{tabular}
\end{center}
at the top of every child file \textit{child}
which is included by |\include{|\textit{child}|}|
from within the main file
(or at least for those files to be compiled individually).
The argument \textit{main} must be the filename of the main file.

There are a couple of
considerations in setting up the main and child documents:

%%%%%%%%%%%%%%%%%%%%%%%%%%%%%%%%%%%%%%%%
\paragraph{Restrictions.}

Please note the following restrictions:
\begin{itemize}
\item
|\childdocmain| must be called with one argument \textit{main}
to ensure compatibility with earlier version of the package.
It must either be empty (|\childdocmain{}|)
or precisely match the filename of the main file in which it is specified.
See \secref{sec:detection} for further information.
\item
The filename \textit{main} must be specified without the |.tex| extension.
\item
The filename \textit{main} is case sensitive
(even in case-insensitive file systems)
due to internal string comparison.
\item
The argument \textit{main} should be fully expanded, it cannot be a macro.
\item
Subdirectories and special characters should be avoided in filenames.
\item
The command |\childdocmain{|\textit{main}|}| must be followed by a whitespace.
It should not be followed immediately by another command
or by a comment mark `|%|'.
This is because the \TeX{} parser reads the token immediately following
the argument of |\childdocmain| and puts it
at the beginning of every child section;
however, a white\-space is ignored.
\end{itemize}

%%%%%%%%%%%%%%%%%%%%%%%%%%%%%%%%%%%%%%%%
\paragraph{Content of Main File.}

It is advisable to place all content in the child files included by |\include|.
Any output contained in the main file will appear in all child documents
unless suppressed manually;
it cannot be suppressed automatically by the |\includeonly| directive
and thus should normally be avoided.
A method to include some content in the main file
by means of conditional processing is described in \secref{sec:conditional}.

%%%%%%%%%%%%%%%%%%%%%%%%%%%%%%%%%%%%%%%%
\paragraph{Page Numbering.}

When only a part of the document is compiled,
the appropriate numbering of pages
(as well as other status parameters)
is determined from the |.aux| files.
The latter contain information from previous passes.
However this information needs to propagate through
all intermediate child documents.
Therefore the page numbering in child documents may well
be inconsistent until the complete document is compiled at least once.

A useful (if unconventional) way to always ensure a consistent
page numbering is to restart the numbering in each child document
and denote the pages by `\textit{child}|.|\textit{page}'
where \textit{child} represents the chapter/section number of the child file.
This can be achieved by the command
|\numberwithin{page}{|\textit{child}|}|
of the \textsf{amsmath} package
where \textit{child} can be |chapter| or |section|
depending on the chosen structuring.
Alternatively, one can modify the macro |\thepage| appropriately
and reset the counter |page| at the start of each child file.

%%%%%%%%%%%%%%%%%%%%%%%%%%%%%%%%%%%%%%%%%%%%%%%%%%%%%%%%%%%%%%%%%%%%%%%%%%%%%%%%
\subsection{Conditional Processing}
\label{sec:conditional}

The package provides a mechanism to compile different versions
of a document. To customise the versions further some conditional processing
can come in handy to distinguish which version is being compiled.
The package provides two macros to describe the compilation context:

%%%%%%%%%%%%%%%%%%%%%%%%%%%%%%%%%%%%%%%%
\DescribeMacro{\ifchilddoc}
The conditional |\ifchilddoc| distinguishes between the compilation of
child documents and the main document:
%
\begin{center}
|\ifchilddoc |\textit{child-code}| |[|\||else |\textit{main-code}]| \||fi|
\end{center}

%%%%%%%%%%%%%%%%%%%%%%%%%%%%%%%%%%%%%%%%
\DescribeMacro{\childdocname}
\DescribeMacro{\childdocjob}
The macro |\childdocname| contains the filename (without extension)
of the main or child file being processed.
Note that |\childdocjob| will always contain the name of the main file.

%%%%%%%%%%%%%%%%%%%%%%%%%%%%%%%%%%%%%%%%
\paragraph{Title Page.}

Conditional processing can be used to include a title or banner page
in the main document when proper precautions are taken.
Importantly, the code in the main file should ensure that the page counter
(as well as other status parameters which are stored in the |.aux| files)
takes the same value after the conditional processing.
Otherwise the page numbers may take divergent values
depending on which part is compiled.

For example, a title page could be declared by:
%
\begin{center}
\begin{tabular}{l}
|\ifchilddoc\||else|\\
|\addtocounter{page}{-1}|\\
\textit{code for title page}\\
|\newpage|\\
|\||fi|
\end{tabular}
\end{center}
%
A banner page for the child documents can be generated by:
%
\begin{center}
\begin{tabular}{l}
|\ifchilddoc|\\
|\addtocounter{page}{-1}|\\
\textit{code for banner page}\\
|\newpage|\\
|\||fi|
\end{tabular}
\end{center}
%
Here one could write a message such as:
\begin{center}
|This is the part \childdocname{} of \childdocjob{}.|
\end{center}

%%%%%%%%%%%%%%%%%%%%%%%%%%%%%%%%%%%%%%%%%%%%%%%%%%%%%%%%%%%%%%%%%%%%%%%%%%%%%%%%
\subsection{Flags}
\label{sec:flags}

The package makes it easy to generate different versions
of the main or child documents.
To this end compilation flags can be defined
and assigned different default values.
They will be particularly useful in conjunction
with the forwarding mechanism described in \secref{sec:forward}.

For example, it may be useful to have a flag |\version|
which can be set to |draft| or |final|.
The document source will contain some conditional code
depending on the value of |\version|.
Suppose further, the flag should default to |final| for the main file
and to |draft| for child files
which is a natural assignment for editing the document.
This is achieved by placing the following code
in the preamble of the main document
(below the |\childdocmain| directive):
%
\begin{center}
\begin{tabular}{l}
|\ifchilddoc|\\
|\providecommand{\version}{draft}|\\
|\||else|\\
|\providecommand{\version}{final}|\\
|\||fi|
\end{tabular}
\end{center}
%
The definition by |\providecommand| makes sure
that previous definitions are not overwritten.
Further statements |\providecommand{\version}{...}|
can thus be added before the above code to override it.

For the main file, one might add a line
(between |\childdocmain| and the above block)
%
\begin{center}
|%\ifchilddoc\||else\providecommand{\version}{draft}\||fi|
\end{center}
%
which can be uncommented to produce a draft version.
Likewise one can add a line to the very top of a child file
(above the |\childdocof{|\textit{main}|}| directive)
%
\begin{center}
|%\providecommand{\version}{final}|
\end{center}
%
which can be uncommented to produce the final version of this child document.

%%%%%%%%%%%%%%%%%%%%%%%%%%%%%%%%%%%%%%%%%%%%%%%%%%%%%%%%%%%%%%%%%%%%%%%%%%%%%%%%
\subsection{Forwarding}
\label{sec:forward}

Different versions of the main or child documents
using compilation flags as described in \secref{sec:flags}
can be (permanently) stored in different files
for convenient compilation, viewing and distribution.
To this end, the package defines a command
to pass on compilation to a different file:

%%%%%%%%%%%%%%%%%%%%%%%%%%%%%%%%%%%%%%%%
\DescribeMacro{\childdocforward}
The command |\childdocforward| redirects processing to
another source file:
%
\begin{center}
\begin{tabular}{l}
|% \iffalse
%
% childdoc.dtx Copyright (C) 2017-2018 Niklas Beisert
%
% This work may be distributed and/or modified under the
% conditions of the LaTeX Project Public License, either version 1.3
% of this license or (at your option) any later version.
% The latest version of this license is in
%   http://www.latex-project.org/lppl.txt
% and version 1.3 or later is part of all distributions of LaTeX
% version 2005/12/01 or later.
%
% This work has the LPPL maintenance status `maintained'.
%
% The Current Maintainer of this work is Niklas Beisert.
%
% This work consists of the files childdoc.dtx and childdoc.ins
% and the derived files childdoc.def and cdocsamp.tex with
% cdocsch1.tex, cdocsch2.tex, cdocsdrf.tex, cdocsfn1.tex, cdocsfn2.tex.
%
%<package>\ifdefined\childdocmain\endinput\fi
%<package>\ProvidesFile{childdoc.def}[2018/12/30 v2.0 child document driver]
%<samplemain>\ProvidesFile{cdocsamp.tex}[2018/12/30 v2.0 sample for childdoc]
%<*driver>
%\ProvidesFile{childdoc.drv}[2018/12/30 v2.0 childdoc reference manual file]
\PassOptionsToClass{10pt,a4paper}{article}
\documentclass{ltxdoc}

\usepackage[margin=35mm]{geometry}
\usepackage{hyperref}
\usepackage{hyperxmp}
\usepackage[usenames]{color}

\hypersetup{colorlinks=true}
\hypersetup{pdfstartview=FitH}
\hypersetup{pdfpagemode=UseNone}
\hypersetup{pdfsource={}}
\hypersetup{pdflang={en-UK}}
\hypersetup{pdfcopyright={Copyright 2017-2018 Niklas Beisert.
  This work may be distributed and/or modified under the
  conditions of the LaTeX Project Public License, either version 1.3
  of this license or (at your option) any later version.}}
\hypersetup{pdflicenseurl={http://www.latex-project.org/lppl.txt}}
\hypersetup{pdfcontactaddress={ETH Zurich, ITP, HIT K,
  Wolfgang-Pauli-Strasse 27}}
\hypersetup{pdfcontactpostcode={8093}}
\hypersetup{pdfcontactcity={Zurich}}
\hypersetup{pdfcontactcountry={Switzerland}}
\hypersetup{pdfcontactemail={nbeisert@itp.phys.ethz.ch}}
\hypersetup{pdfcontacturl={http://people.phys.ethz.ch/\xmptilde nbeisert/}}

\newcommand{\secref}[1]{\hyperref[#1]{section \ref*{#1}}}

\parskip1ex
\parindent0pt
\let\olditemize\itemize
\def\itemize{\olditemize\parskip0pt}

\begin{document}

\title{The \textsf{childdoc} Package}
\hypersetup{pdftitle={The childdoc Package}}
\author{Niklas Beisert\\[2ex]
  Institut f\"ur Theoretische Physik\\
  Eidgen\"ossische Technische Hochschule Z\"urich\\
  Wolfgang-Pauli-Strasse 27, 8093 Z\"urich, Switzerland\\[1ex]
  \href{mailto:nbeisert@itp.phys.ethz.ch}
  {\texttt{nbeisert@itp.phys.ethz.ch}}}
\hypersetup{pdfauthor={Niklas Beisert}}
\hypersetup{pdfsubject={Manual for the LaTeX2e Package childdoc}}
\date{30 December 2018, \textsf{v2.0}}
\maketitle

\begin{abstract}\noindent
\textsf{childdoc} is a \LaTeXe{} package
that enables the direct compilation
of document sections included by |\include|
to individual files.
\end{abstract}

\begingroup
\parskip0ex
\tableofcontents
\endgroup

%%%%%%%%%%%%%%%%%%%%%%%%%%%%%%%%%%%%%%%%%%%%%%%%%%%%%%%%%%%%%%%%%%%%%%%%%%%%%%%%
%%%%%%%%%%%%%%%%%%%%%%%%%%%%%%%%%%%%%%%%%%%%%%%%%%%%%%%%%%%%%%%%%%%%%%%%%%%%%%%%
\section{Introduction}

\LaTeX{} provides a mechanism to structure a large document (such as a book)
into a main file and several child files (containing the chapters)
using the |\include| command.
This mechanism is beneficial for documents
which span hundreds of pages in order to
make the source file(s) more manageable.
Moreover, compilation can be restricted to
selected child files by means of the |\includeonly| command.
The latter feature can be used to reduce the compilation time while editing
(this was significantly more useful in the earlier days of \LaTeX{})
or to generate a smaller document which is easier to navigate.
Another application of |\includeonly| is to generate
documents consisting of selected parts of the complete document.

However, there are a few drawbacks of the plain |\include| mechanism:
\begin{itemize}
\item
The child files cannot be compiled on their own,
they can only be compiled via the main file.
A naive editing environment
(such as a text editor with an option
to have the current file processed by \LaTeX)
may require one to switch to the main file before compiling;
attempting to compile the child file produces errors.
\item
The main file must be modified (each time)
to adjust the |\includeonly| command
to the present needs. This easily leaves the main file in a messy state.
\item
The generated document will always carry the filename
of the main document. This is inconvenient if
several child files are to be compiled and
to be kept for distribution.
\end{itemize}

The present package provides a simple interface
to make child files individually compilable by \LaTeX{}.
Compiling a child file then has the same effect as compiling
the main file with an |\includeonly| command
to select the appropriate child.
Moreover the generated document will carry the name of the child
rather than the main file.
This resolves all three above issues.

This feature is meant to make the editing of books,
thesis documents and lecture notes somewhat more convenient.
However, the package can also be used efficiently for
composing a series of documents (such as exercise sheets)
which are typically distributed individually.
It then assists the author in generating the individual documents
(potentially in different versions)
as well as a document containing the collected series.
Another application is in developing style files
or other kinds of included material
where compilation of the style file could redirect
to a sample or test file.

%%%%%%%%%%%%%%%%%%%%%%%%%%%%%%%%%%%%%%%%%%%%%%%%%%%%%%%%%%%%%%%%%%%%%%%%%%%%%%%%
%%%%%%%%%%%%%%%%%%%%%%%%%%%%%%%%%%%%%%%%%%%%%%%%%%%%%%%%%%%%%%%%%%%%%%%%%%%%%%%%
\section{Usage}

First of all, the package \textsf{childdoc} is \emph{not} a standard
\LaTeXe{} |.sty| style file! Therefore it needs to be invoked in
a non-standard way.

%%%%%%%%%%%%%%%%%%%%%%%%%%%%%%%%%%%%%%%%%%%%%%%%%%%%%%%%%%%%%%%%%%%%%%%%%%%%%%%%
\subsection{Included Files}
\label{sec:include}

%%%%%%%%%%%%%%%%%%%%%%%%%%%%%%%%%%%%%%%%
\DescribeMacro{\childdocmain}
To use the package, add the commands
\begin{center}
\begin{tabular}{l}
|\input{childdoc.def}|\\
|\childdocmain{}|\\
\end{tabular}
\end{center}
at the very top of the main \LaTeX{} file,
in particular \emph{before} the |\documentclass| statement!
The argument of |\childdocmain| should be left empty
(but it must be present).

%%%%%%%%%%%%%%%%%%%%%%%%%%%%%%%%%%%%%%%%
\DescribeMacro{\childdocof}
Furthermore, add the commands
\begin{center}
\begin{tabular}{l}
|\input{childdoc.def}|\\
|\childdocof{|\textit{main}|}|\\
\end{tabular}
\end{center}
at the top of every child file \textit{child}
which is included by |\include{|\textit{child}|}|
from within the main file
(or at least for those files to be compiled individually).
The argument \textit{main} must be the filename of the main file.

There are a couple of
considerations in setting up the main and child documents:

%%%%%%%%%%%%%%%%%%%%%%%%%%%%%%%%%%%%%%%%
\paragraph{Restrictions.}

Please note the following restrictions:
\begin{itemize}
\item
|\childdocmain| must be called with one argument \textit{main}
to ensure compatibility with earlier version of the package.
It must either be empty (|\childdocmain{}|)
or precisely match the filename of the main file in which it is specified.
See \secref{sec:detection} for further information.
\item
The filename \textit{main} must be specified without the |.tex| extension.
\item
The filename \textit{main} is case sensitive
(even in case-insensitive file systems)
due to internal string comparison.
\item
The argument \textit{main} should be fully expanded, it cannot be a macro.
\item
Subdirectories and special characters should be avoided in filenames.
\item
The command |\childdocmain{|\textit{main}|}| must be followed by a whitespace.
It should not be followed immediately by another command
or by a comment mark `|%|'.
This is because the \TeX{} parser reads the token immediately following
the argument of |\childdocmain| and puts it
at the beginning of every child section;
however, a white\-space is ignored.
\end{itemize}

%%%%%%%%%%%%%%%%%%%%%%%%%%%%%%%%%%%%%%%%
\paragraph{Content of Main File.}

It is advisable to place all content in the child files included by |\include|.
Any output contained in the main file will appear in all child documents
unless suppressed manually;
it cannot be suppressed automatically by the |\includeonly| directive
and thus should normally be avoided.
A method to include some content in the main file
by means of conditional processing is described in \secref{sec:conditional}.

%%%%%%%%%%%%%%%%%%%%%%%%%%%%%%%%%%%%%%%%
\paragraph{Page Numbering.}

When only a part of the document is compiled,
the appropriate numbering of pages
(as well as other status parameters)
is determined from the |.aux| files.
The latter contain information from previous passes.
However this information needs to propagate through
all intermediate child documents.
Therefore the page numbering in child documents may well
be inconsistent until the complete document is compiled at least once.

A useful (if unconventional) way to always ensure a consistent
page numbering is to restart the numbering in each child document
and denote the pages by `\textit{child}|.|\textit{page}'
where \textit{child} represents the chapter/section number of the child file.
This can be achieved by the command
|\numberwithin{page}{|\textit{child}|}|
of the \textsf{amsmath} package
where \textit{child} can be |chapter| or |section|
depending on the chosen structuring.
Alternatively, one can modify the macro |\thepage| appropriately
and reset the counter |page| at the start of each child file.

%%%%%%%%%%%%%%%%%%%%%%%%%%%%%%%%%%%%%%%%%%%%%%%%%%%%%%%%%%%%%%%%%%%%%%%%%%%%%%%%
\subsection{Conditional Processing}
\label{sec:conditional}

The package provides a mechanism to compile different versions
of a document. To customise the versions further some conditional processing
can come in handy to distinguish which version is being compiled.
The package provides two macros to describe the compilation context:

%%%%%%%%%%%%%%%%%%%%%%%%%%%%%%%%%%%%%%%%
\DescribeMacro{\ifchilddoc}
The conditional |\ifchilddoc| distinguishes between the compilation of
child documents and the main document:
%
\begin{center}
|\ifchilddoc |\textit{child-code}| |[|\||else |\textit{main-code}]| \||fi|
\end{center}

%%%%%%%%%%%%%%%%%%%%%%%%%%%%%%%%%%%%%%%%
\DescribeMacro{\childdocname}
\DescribeMacro{\childdocjob}
The macro |\childdocname| contains the filename (without extension)
of the main or child file being processed.
Note that |\childdocjob| will always contain the name of the main file.

%%%%%%%%%%%%%%%%%%%%%%%%%%%%%%%%%%%%%%%%
\paragraph{Title Page.}

Conditional processing can be used to include a title or banner page
in the main document when proper precautions are taken.
Importantly, the code in the main file should ensure that the page counter
(as well as other status parameters which are stored in the |.aux| files)
takes the same value after the conditional processing.
Otherwise the page numbers may take divergent values
depending on which part is compiled.

For example, a title page could be declared by:
%
\begin{center}
\begin{tabular}{l}
|\ifchilddoc\||else|\\
|\addtocounter{page}{-1}|\\
\textit{code for title page}\\
|\newpage|\\
|\||fi|
\end{tabular}
\end{center}
%
A banner page for the child documents can be generated by:
%
\begin{center}
\begin{tabular}{l}
|\ifchilddoc|\\
|\addtocounter{page}{-1}|\\
\textit{code for banner page}\\
|\newpage|\\
|\||fi|
\end{tabular}
\end{center}
%
Here one could write a message such as:
\begin{center}
|This is the part \childdocname{} of \childdocjob{}.|
\end{center}

%%%%%%%%%%%%%%%%%%%%%%%%%%%%%%%%%%%%%%%%%%%%%%%%%%%%%%%%%%%%%%%%%%%%%%%%%%%%%%%%
\subsection{Flags}
\label{sec:flags}

The package makes it easy to generate different versions
of the main or child documents.
To this end compilation flags can be defined
and assigned different default values.
They will be particularly useful in conjunction
with the forwarding mechanism described in \secref{sec:forward}.

For example, it may be useful to have a flag |\version|
which can be set to |draft| or |final|.
The document source will contain some conditional code
depending on the value of |\version|.
Suppose further, the flag should default to |final| for the main file
and to |draft| for child files
which is a natural assignment for editing the document.
This is achieved by placing the following code
in the preamble of the main document
(below the |\childdocmain| directive):
%
\begin{center}
\begin{tabular}{l}
|\ifchilddoc|\\
|\providecommand{\version}{draft}|\\
|\||else|\\
|\providecommand{\version}{final}|\\
|\||fi|
\end{tabular}
\end{center}
%
The definition by |\providecommand| makes sure
that previous definitions are not overwritten.
Further statements |\providecommand{\version}{...}|
can thus be added before the above code to override it.

For the main file, one might add a line
(between |\childdocmain| and the above block)
%
\begin{center}
|%\ifchilddoc\||else\providecommand{\version}{draft}\||fi|
\end{center}
%
which can be uncommented to produce a draft version.
Likewise one can add a line to the very top of a child file
(above the |\childdocof{|\textit{main}|}| directive)
%
\begin{center}
|%\providecommand{\version}{final}|
\end{center}
%
which can be uncommented to produce the final version of this child document.

%%%%%%%%%%%%%%%%%%%%%%%%%%%%%%%%%%%%%%%%%%%%%%%%%%%%%%%%%%%%%%%%%%%%%%%%%%%%%%%%
\subsection{Forwarding}
\label{sec:forward}

Different versions of the main or child documents
using compilation flags as described in \secref{sec:flags}
can be (permanently) stored in different files
for convenient compilation, viewing and distribution.
To this end, the package defines a command
to pass on compilation to a different file:

%%%%%%%%%%%%%%%%%%%%%%%%%%%%%%%%%%%%%%%%
\DescribeMacro{\childdocforward}
The command |\childdocforward| redirects processing to
another source file:
%
\begin{center}
\begin{tabular}{l}
|\input{childdoc.def}|\\
|\childdocforward[|\textit{main}|]{|\textit{dest}|}|\\
\end{tabular}
\end{center}
%
The argument \textit{dest} is the destination file
(without extension).
It should be the main file or one of the child files.
Note that further \textsf{childdoc} directives
such as |\childdocof| and |\childdocforward|
in the indicated file will be processed in this form.
The optional argument \textit{main}
passes on directly to the main file \textit{main}
while pretending to compile the child \textit{dest}.
This form behaves as if \textit{dest}
issues |\childdocof{|\textit{main}|}| right away,
and no further \textsf{childdoc} directives will be processed.

%%%%%%%%%%%%%%%%%%%%%%%%%%%%%%%%%%%%%%%%
\DescribeMacro{\...prefix}
In the alternative form |\childdocforwardprefix|,
%
\begin{center}
\begin{tabular}{l}
|\input{childdoc.def}|\\
|\childdocforwardprefix[|\textit{main}|]{|\textit{prefix}|}{|\textit{dest}|}|
\end{tabular}
\end{center}
%
the destination file is determined by a pattern
depending on the current file:
To make this work, the current file must be called
`{\textit{prefix}\hspace{0.2em}\textit{suffix}}'
with \textit{prefix} matching precisely the argument.
Processing is then passed on to the file
`{\textit{dest}\hspace{0.2em}\textit{suffix}}'.
Surely, the same effect is achieved by
directly specifying the
argument `{\textit{dest}\hspace{0.2em}\textit{suffix}}'
in the first form.
However, that requires to set up a different file
for each child. With the alternative form of the command
all these files can have exactly the same content
which simplifies setting them up and maintaining them.

For example, the following file |draft.tex|
with a compilation flag |\version| as described in \secref{sec:flags}
compiles the main document as a draft:
%
\begin{center}
\begin{tabular}{l}
|\def\version{draft}|\\
|\input{childdoc.def}|\\
|\childdocforward{|\textit{main}|}|
\end{tabular}
\end{center}
%
Likewise, the following files |final|\textit{nn}|.tex|
compile the final version of the child document
|child|\textit{nn}|.tex|:
%
\begin{center}
\begin{tabular}{l}
|\def\version{final}|\\
|\input{childdoc.def}|\\
|\childdocforwardprefix{final}{child}|
\end{tabular}
\end{center}
%

Note that when several versions of a main file and/or of each child file
are to be generated, it may be convenient to set up a |Makefile| or
shell script to automatise the process.

%%%%%%%%%%%%%%%%%%%%%%%%%%%%%%%%%%%%%%%%%%%%%%%%%%%%%%%%%%%%%%%%%%%%%%%%%%%%%%%%
\subsection{Command Line Processing}
\label{sec:commandline}

The effect of redirection files can also be achieved by invoking
the \LaTeX{} compiler with a more elaborate command line.
Most conveniently this should be done as part
of a shell script or a |Makefile|.

When using \textsf{childdoc} in the main file, the following
command lines effectively perform a redirection
(note that depending on the shell being used,
backslashes may have to be doubled: `|\|' $\to$ `|\\|'):
%
\begin{center}
|... -jobname "|\textit{target}|" |\\|"|[\textit{flags}]%
|\input{childdoc.def}\childdocforward[|\textit{main}|]{|\textit{dest}|}"|
\end{center}
%
Here \textit{target} is the name of the output file,
\textit{main} is the name of the main file
and \textit{dest} is the name of the main or child file to be processed
(all filenames without extensions).
The optional argument \textit{main} can be omitted
if \textit{main} matches \textit{dest}.
Optionally, compilation \textit{flags} can be defined via |\def| commands.
This command line makes the \TeX{} engine believe
it is compiling the file \textit{target}
whose content is specified as the latter parameter.
The provided code then forwards the processing to
\textit{main} or \textit{dest} as described in \secref{sec:forward}.

%%%%%%%%%%%%%%%%%%%%%%%%%%%%%%%%%%%%%%%%%%%%%%%%%%%%%%%%%%%%%%%%%%%%%%%%%%%%%%%%
\subsection{Include by Input}
\label{sec:input}

Including child documents by |\include| has some restrictions by design.
Most notably, the content of a child document always occupies
its own set of pages; pages cannot be shared between child documents.
Usually, this behaviour makes perfect sense
because each child document contain an essential part of the document.
However, in some situations it may be desirable to compose
a document from a collection of parts
without having mandatory page breaks between then.
For this case, the package
provides a mechanism to include parts
by |\input| which can also be processed individually.
However, by construction this mechanism
requires manual handling of the content to be output.

%%%%%%%%%%%%%%%%%%%%%%%%%%%%%%%%%%%%%%%%
\DescribeMacro{\ifchilddocmanual}
The main file should be prepared as usual, see \secref{sec:include}.
However, the document body must make a distinction
between processing of an individual part and of the main document, e.g.:
%
\begin{center}
\begin{tabular}{l}
|\ifchilddocmanual|\\
|\input{\childdocname}|\\
|\||else|\\
\textit{document body with }|\input{|\textit{part}|}|\\
|\||fi|
\end{tabular}
\end{center}
%
The conditional |\ifchilddocmanual| is true whenever
a part to be included by |\input| is being compiled,
and the name of the part is stored in |\childdocname|.

%%%%%%%%%%%%%%%%%%%%%%%%%%%%%%%%%%%%%%%%
\DescribeMacro{\childdocby}
Each part to be included by |\input| should start with:
%
\begin{center}
\begin{tabular}{l}
|\input{childdoc.def}|\\
|\childdocby{|\textit{main}|}|\\
\end{tabular}
\end{center}
%
The directive |\childdocby| is similar to |\childdocof|
described in \secref{sec:include},
but the subsequent selection of content must be done manually.
To that end, both |\ifchilddoc| and |\ifchilddocmanual|
will be true upon processing of a part,
and the name of the part is stored in |\childdocname|.
Note that |\jobname| will be set to the filename of the current part
so that each part receives an individual |.aux| file
that does not interfere with the |.aux| file(s) of the main document.
This behaviour can be altered by the alternative form
|\childdocby[*]{|\textit{main}|}| (with a non-empty optional argument)
which uses the |.aux| file of the main document
by setting |\jobname| to \textit{main}.

%%%%%%%%%%%%%%%%%%%%%%%%%%%%%%%%%%%%%%%%%%%%%%%%%%%%%%%%%%%%%%%%%%%%%%%%%%%%%%%%
\subsection{Driver Development}
\label{sec:driver}

The \textsf{childdoc} mechanism can also be use for the development
of definition files such as \LaTeX{} styles or classes.
This case differs from the above setup with multiple parts
included by |\include| in that no |\includeonly| should be invoked.
This can be achieved by starting the include file
(before |\ProvidesPackage|) with:
%
\begin{center}
\begin{tabular}{l}
|\input{childdoc.def}|\\
|\childdocforward{|\textit{main}|}|\\
\end{tabular}
\end{center}
%
or alternatively with:
%
\begin{center}
\begin{tabular}{l}
|\input{childdoc.def}|\\
|\childdocby{|\textit{main}|}|\\
\end{tabular}
\end{center}
%
Both forms have slightly different effects as described above.
The main file is prepared as usual, see \secref{sec:include}.

%%%%%%%%%%%%%%%%%%%%%%%%%%%%%%%%%%%%%%%%%%%%%%%%%%%%%%%%%%%%%%%%%%%%%%%%%%%%%%%%
\subsection{Legacy Detection}
\label{sec:detection}

The directive |\childdocmain| in the main file can detect
whether the complete document or merely a child is to be compiled
even without using the directive |\childdocof|.
This method is deprecated because it is less robust
and there is no compelling reason to use it;
it is merely provided for backward compatibility
and it may be removed in future versions.

If the detection mechanism is to be used,
it is mandatory to correctly specify
the filename of the main file as the argument of |\childdocmain|:
%
\begin{center}
\begin{tabular}{l}
|\input{childdoc.def}|\\
|\childdocmain{|\textit{main}|}|\\
\end{tabular}
\end{center}
%
If |\jobname| does not match the argument \textit{main} of |\childdocmain|,
it is assumed that |\jobname| points to the child file to be compiled.
When using |\childdocmain| with the main file specified as argument,
it suffices to start a child file
with just |\input{|\textit{main}|}|
without loading of the package and using |\childdocof|.
If instead all processing is done
with the appropriate \textsf{childdoc} directives,
the argument of \textit{main} of |\childdocmain| can be empty.

An alternative version of the command line processing described
in \secref{sec:commandline} using the detection mechanism reads:
%
\begin{center}
|... -jobname "|\textit{target}|" "|[\textit{flags}]%
[|\def\jobname{|\textit{dest}|}|]|\input{|\textit{main}|}"|
\end{center}

%%%%%%%%%%%%%%%%%%%%%%%%%%%%%%%%%%%%%%%%%%%%%%%%%%%%%%%%%%%%%%%%%%%%%%%%%%%%%%%%
\subsection{Manual Code}
\label{sec:manual}

In case one cannot be certain whether the definitions file |childdoc.def|
is installed on the target \TeX{} distribution
and one prefers not to ship it,
it is conceivable to paste a few relevant commands into the sources.

To that end, drop all statements |\input{childdoc.def}|
and perform the replacements as outlined below.
Instead of |\childdocmain{|\textit{main}|}| add the following code
to the top of the main file:
%
\begin{center}
\begin{tabular}{l}
|\||ifdefined\childdocname\endinput\||fi\newif\ifchilddoc|\\
|\edef\childdocname{\scantokens\expandafter{\jobname\noexpand}}|\\
|\def\childdocmain{|\textit{main}|}\||ifx\childdocmain\childdocname\||else|\\
|\childdoctrue\includeonly{\childdocname}\let\jobname\childdocmain\||fi|\\
\end{tabular}
\end{center}
%
Instead of |\childdocof{|\textit{main}|}| just include the main file
at the top of each child file:
%
\begin{center}
|\input{|\textit{main}|}|
\end{center}
%
A simple redirection |\childdocforward{|\textit{dest}|}| is achieved by:
%
\begin{center}
|\def\jobname{|\textit{dest}|}\input{\jobname}|
\end{center}
%
The redirection with prefix
|\childdocforwardprefix[|\textit{prefix}|]{|\textit{dest}|}|
is accomplished by:
%
\begin{center}
\begin{tabular}{l}
|{\edef\jobname{\scantokens\expandafter{\jobname\noexpand}}|\\
|\def\redirectjob |\textit{prefix}|#1~~~{\gdef\jobname{|\textit{dest}|#1}}|\\
|\expandafter\redirectjob\jobname~~~}\input{\jobname}|
\end{tabular}
\end{center}

In an alternative approach,
child documents can be compiled by a specific command line
without additional code or specific definitions:
%
\begin{center}
|... -jobname "|\textit{target}|" "|[\textit{flags}]%
|\includeonly{|\textit{dest}|}\input{|\textit{main}|}"|
\end{center}
%

%%%%%%%%%%%%%%%%%%%%%%%%%%%%%%%%%%%%%%%%%%%%%%%%%%%%%%%%%%%%%%%%%%%%%%%%%%%%%%%%
%%%%%%%%%%%%%%%%%%%%%%%%%%%%%%%%%%%%%%%%%%%%%%%%%%%%%%%%%%%%%%%%%%%%%%%%%%%%%%%%
\section{Information}

%%%%%%%%%%%%%%%%%%%%%%%%%%%%%%%%%%%%%%%%%%%%%%%%%%%%%%%%%%%%%%%%%%%%%%%%%%%%%%%%
\subsection{Copyright}

Copyright \copyright{} 2017--2018 Niklas Beisert

This work may be distributed and/or modified under the
conditions of the \LaTeX{} Project Public License, either version 1.3
of this license or (at your option) any later version.
The latest version of this license is in
  \url{http://www.latex-project.org/lppl.txt}
and version 1.3 or later is part of all distributions of \LaTeX{}
version 2005/12/01 or later.

This work has the LPPL maintenance status `maintained'.

The Current Maintainer of this work is Niklas Beisert.

This work consists of the files |README.txt|, |childdoc.ins| and |childdoc.dtx|
as well as the derived files |childdoc.def|, |cdocsamp.tex|
with |cdocsch1.tex|, |cdocsch2.tex|, |cdocspt3.tex|, |cdocspt4.tex|,
|cdocsdrf.tex|, |cdocsfn1.tex|, |cdocsfn2.tex|
as well as |childdoc.pdf|.

%%%%%%%%%%%%%%%%%%%%%%%%%%%%%%%%%%%%%%%%%%%%%%%%%%%%%%%%%%%%%%%%%%%%%%%%%%%%%%%%
\subsection{Files and Installation}

The package consists of the files:
%
\begin{center}
\begin{tabular}{ll}
    |README.txt|   & readme file \\
    |childdoc.ins| & installation file \\
    |childdoc.dtx| & source file \\
    |childdoc.def| & definition file \\
    |cdocsamp.tex| & sample main file \\
    |cdocsch1.tex| & sample include file \\
    |cdocsch2.tex| & sample include file \\
    |cdocspt3.tex| & sample part file \\
    |cdocspt4.tex| & sample part file \\
    |cdocsdrf.tex| & sample redirection file \\
    |cdocsfn1.tex| & sample redirection file \\
    |cdocsfn2.tex| & sample redirection file \\
    |childdoc.pdf| & manual
\end{tabular}
\end{center}
%
The distribution consists of the files
|README.txt|, |childdoc.ins| and |childdoc.dtx|.
%
\begin{itemize}
\item
Run (pdf)\LaTeX{} on |childdoc.dtx|
to compile the manual |childdoc.pdf| (this file).
\item
Run \LaTeX{} on |childdoc.ins| to create the definitions file |childdoc.def|
and the sample |cdocsamp.tex| with include files
|cdocsch1.tex|, |cdocsch2.tex|, |cdocspt3.tex|, |cdocspt4.tex|,
|cdocsdrf.tex|, |cdocsfn1.tex|, |cdocsfn2.tex|.
Then copy the file |childdoc.def| to an appropriate directory of your \LaTeX{}
distribution, e.g.\ \textit{texmf-root}|/tex/latex/childdoc|.
\end{itemize}

%%%%%%%%%%%%%%%%%%%%%%%%%%%%%%%%%%%%%%%%%%%%%%%%%%%%%%%%%%%%%%%%%%%%%%%%%%%%%%%%
\subsection{Related CTAN Packages}

There are several other packages which offer a similar functionality:
%
\begin{itemize}
\item
The packages
\href{http://ctan.org/pkg/docmute}{\textsf{docmute}},
\href{http://ctan.org/pkg/includex}{\textsf{includex}} and
\href{http://ctan.org/pkg/standalone}{\textsf{standalone}}
provide commands to include only the document body of
a child file thus allowing both files to be compiled individually.
\item
The packages \href{http://ctan.org/pkg/subdocs}{\textsf{subdocs}}
and \href{http://ctan.org/pkg/subfiles}{\textsf{subfiles}}
provide structures in which the main and child documents can be
encapsulated and allowing them to be compiled individually.
The inclusion mechanism is different from the conventional |\include|.
\item
The package \href{http://ctan.org/pkg/combine}{\textsf{combine}}
is an elaborate solution to combine several documents into one.
\end{itemize}
%
See also the CTAN topic \href{http://ctan.org/topic/subdocs}{\textsf{subdocs}}
for further related packages.
The present package differs from the above solutions in that
a document structure constructed with the conventional |\include| mechanism
just needs two extra commands at the top of every file
such that all constituent files can be compiled individually.

%%%%%%%%%%%%%%%%%%%%%%%%%%%%%%%%%%%%%%%%%%%%%%%%%%%%%%%%%%%%%%%%%%%%%%%%%%%%%%%%
%\subsection{Feature Suggestions}
%
%The following is a list of features which may be useful for future
%versions of this package:
%%
%\begin{itemize}
%\item
%\ldots
%\end{itemize}

%%%%%%%%%%%%%%%%%%%%%%%%%%%%%%%%%%%%%%%%%%%%%%%%%%%%%%%%%%%%%%%%%%%%%%%%%%%%%%%%
\subsection{Revision History}

%%%%%%%%%%%%%%%%%%%%%%%%%%%%%%%%%%%%%%%%
\paragraph{v2.0:} 2018/12/30

\begin{itemize}
\item
immediate forward processing
\item
added |\childdocby| mechanism
\item
manual restructured
\end{itemize}

%%%%%%%%%%%%%%%%%%%%%%%%%%%%%%%%%%%%%%%%
\paragraph{v1.6:} 2018/01/17

\begin{itemize}
\item
application for development of include files
\item
corrections to manual
\end{itemize}

%%%%%%%%%%%%%%%%%%%%%%%%%%%%%%%%%%%%%%%%
\paragraph{v1.5:} 2017/05/21

\begin{itemize}
\item
more complete structuring introduced
\item
|\childdocof| introduced
\item
|\childdoc| renamed to |\childdocmain|
\item
|\childredirect| renamed to |\childdocforward| and |\childdocforwardprefix|
and functionality expanded
\end{itemize}

%%%%%%%%%%%%%%%%%%%%%%%%%%%%%%%%%%%%%%%%
\paragraph{v1.0:} 2017/04/27

\begin{itemize}
\item
manual and install package
\item
first version published on CTAN
\end{itemize}

%%%%%%%%%%%%%%%%%%%%%%%%%%%%%%%%%%%%%%%%
\paragraph{v0.6:} 2017/04/26

\begin{itemize}
\item
redirection mechanism added
\end{itemize}

%%%%%%%%%%%%%%%%%%%%%%%%%%%%%%%%%%%%%%%%
\paragraph{v0.5:} 2017/04/26

\begin{itemize}
\item
functionality in definition file
\end{itemize}


%%%%%%%%%%%%%%%%%%%%%%%%%%%%%%%%%%%%%%%%%%%%%%%%%%%%%%%%%%%%%%%%%%%%%%%%%%%%%%%%
%%%%%%%%%%%%%%%%%%%%%%%%%%%%%%%%%%%%%%%%%%%%%%%%%%%%%%%%%%%%%%%%%%%%%%%%%%%%%%%%
%%%%%%%%%%%%%%%%%%%%%%%%%%%%%%%%%%%%%%%%%%%%%%%%%%%%%%%%%%%%%%%%%%%%%%%%%%%%%%%%
\appendix

\settowidth\MacroIndent{\rmfamily\scriptsize 000\ }

 \DocInput{childdoc.dtx}

\end{document}
%</driver>
% \fi
%
% %%%%%%%%%%%%%%%%%%%%%%%%%%%%%%%%%%%%%%%%%%%%%%%%%%%%%%%%%%%%%%%%%%%%%%%%%%%%%%
% %%%%%%%%%%%%%%%%%%%%%%%%%%%%%%%%%%%%%%%%%%%%%%%%%%%%%%%%%%%%%%%%%%%%%%%%%%%%%%
% \section{Sample}
%\iffalse
%<*samplemain>
%\fi
%
% The following presents a sample document
% with two chapters, two parts, a title page,
% a compile flag as well as three forwarding files to set the flag.
% It consists of eight |.tex| files:
% \begin{center}
% \begin{tabular}{ll}
% |cdocsamp.tex|&main file\\
% |cdocsch1.tex|&include file for chapter 1\\
% |cdocsch2.tex|&include file for chapter 2\\
% |cdocspt3.tex|&include file for part 3\\
% |cdocspt4.tex|&include file for part 4\\
% |cdocsdrf.tex|&forwarding file for main file in draft mode\\
% |cdocsfi1.tex|&forwarding file for final version of chapter 1\\
% |cdocsfi2.tex|&forwarding file for final version of chapter 2\\
% \end{tabular}
% \end{center}
% Each of the eight files can be compiled directly by the \LaTeX{} compiler.
%
% %%%%%%%%%%%%%%%%%%%%%%%%%%%%%%%%%%%%%%
% \paragraph{Main File.}
%
% The main file is called |cdocsamp.tex|.
%
% Load the \textsf{childdoc} definitions and
% declare the filename for the main document:
%    \begin{macrocode}
\input{childdoc.def}
\childdocmain{}
%    \end{macrocode}

% Optional override for |\version| flag:
%    \begin{macrocode}
%%\ifchilddoc\else\providecommand{\version}{draft}\fi
%    \end{macrocode}

% Define the default values for the |\version| flag
% (|final| for the main file and |draft| for childs):
%    \begin{macrocode}
\ifchilddoc
\providecommand{\version}{draft}
\else
\providecommand{\version}{final}
\fi
%    \end{macrocode}

% Load the standard document class:
%    \begin{macrocode}
\documentclass[12pt]{article}
%    \end{macrocode}

% Start the document body:
%    \begin{macrocode}
\begin{document}
%    \end{macrocode}

% Declare a title page.
% Print title, part of document being processed and version flag:
%    \begin{macrocode}
\addtocounter{page}{-1}
\begin{center}
{\LARGE\bfseries{}childdoc example\par}
\vspace{1cm}
\ifchilddoc
\ifchilddocmanual part\else chapter\fi:
`\childdocname' of `\childdocjob'\par
\else
main document: `\childdocjob'\par
\fi
version: \version\par
\end{center}
\newpage
%    \end{macrocode}

% Manually include selected file,
% otherwise process as usual:
%    \begin{macrocode}
\ifchilddocmanual
\section*{part `\childdocname'}
\input{\childdocname}
\else
%    \end{macrocode}

% Include the two chapters:
%    \begin{macrocode}
\include{cdocsch1}
\include{cdocsch2}
%    \end{macrocode}

% Include the two parts unless only chapters should be displayed:
%    \begin{macrocode}
\ifchilddoc\else
\section{part three}
\input{cdocspt3}
\section{part four}
\input{cdocspt4}
\fi
%    \end{macrocode}

% Process as usual until here:
%    \begin{macrocode}
\fi
%    \end{macrocode}

% End of document body:
%    \begin{macrocode}
\end{document}
%    \end{macrocode}
%\iffalse
%</samplemain>
%\fi
%
% %%%%%%%%%%%%%%%%%%%%%%%%%%%%%%%%%%%%%%
% \paragraph{Chapter Include Files.}
%
% The include files are called |cdocsch1.tex| and |cdocsch2.tex|.
%
%\iffalse
%<*samplechap1|samplechap2>
%\fi

% Optional override for |\version| flag:
%    \begin{macrocode}
%%\providecommand{\version}{final}
%    \end{macrocode}

% Include the main document:
%    \begin{macrocode}
\input{childdoc.def}
\childdocof{cdocsamp}
%    \end{macrocode}

%\iffalse
%</samplechap1|samplechap2>
%\fi
%
%\iffalse
%<*samplechap1>
%\fi
% Some text for chapter 1:
%    \begin{macrocode}
\section{one}
some text in chapter one
%    \end{macrocode}

%\iffalse
%</samplechap1>
%\fi
% Some text for chapter 2:
%\iffalse
%<*samplechap2>
%\fi
%    \begin{macrocode}
\section{two}
more text in chapter two
%    \end{macrocode}

%\iffalse
%</samplechap2>
%\fi
%
% %%%%%%%%%%%%%%%%%%%%%%%%%%%%%%%%%%%%%%
% \paragraph{Part Include Files.}
%
% The include files are called |cdocspt3.tex| and |cdocspt4.tex|.
%
%\iffalse
%<*samplepart3|samplepart4>
%\fi

% Optional override for |\version| flag:
%    \begin{macrocode}
%%\providecommand{\version}{final}
%    \end{macrocode}

% Include the main document:
%    \begin{macrocode}
\input{childdoc.def}
\childdocby{cdocsamp}
%    \end{macrocode}

%\iffalse
%</samplepart3|samplepart4>
%\fi
%
%\iffalse
%<*samplepart3>
%\fi
% Some text for part 3:
%    \begin{macrocode}
some text in part three
%    \end{macrocode}

%\iffalse
%</samplepart3>
%\fi
% Some text for part 4:
%\iffalse
%<*samplepart4>
%\fi
%    \begin{macrocode}
more text in part four
%    \end{macrocode}

%\iffalse
%</samplepart4>
%\fi
%
% %%%%%%%%%%%%%%%%%%%%%%%%%%%%%%%%%%%%%%
% \paragraph{Forwarding for a Complete Draft.}
%
% The following forwarding file |cdocsdrf.tex|
% compiles the main document in draft mode:
%\iffalse
%<*sampledraft>
%\fi
%    \begin{macrocode}
\def\version{draft}
\input{childdoc.def}
\childdocforward{cdocsamp}
%    \end{macrocode}

%\iffalse
%</sampledraft>
%\fi
%
% %%%%%%%%%%%%%%%%%%%%%%%%%%%%%%%%%%%%%%
% \paragraph{Forwarding for Final Version of the Chapters.}
%
% The following forwarding files |cdocsfn1.tex| and |cdocsfn2.tex|
% (with identical content)
% compile the final versions of the child documents
% |cdocsch1.tex| and |cdocsch2.tex|, respectively:
%\iffalse
%<*samplefinal>
%\fi
%    \begin{macrocode}
\def\version{final}
\input{childdoc.def}
\childdocforwardprefix[cdocsamp]{cdocsfn}{cdocsch}
%    \end{macrocode}

%\iffalse
%</samplefinal>
%\fi
%
% %%%%%%%%%%%%%%%%%%%%%%%%%%%%%%%%%%%%%%
% \paragraph{Command Line Processing.}
%
% The following three command lines generate the output files
% |cdocscld|, |cdocscl1| and |cdocscl2|
% which should be identical to
% |cdocsdrf|, |cdocsch1| and |cdocsfn2|, respectively:
% \begin{center}
% \begin{tabular}{l}
% |latex -jobname cdocscld \|\\
% |  "\def\version{draft}\input{childdoc.def}\childdocforward{cdocsamp}"|\\
% |latex -jobname cdocscl1 \|\\
% |  "\input{childdoc.def}\childdocforward[cdocsamp]{cdocsch1}"|\\
% |latex -jobname cdocscl2 \|\\
% |  "\def\version{final}\input{childdoc.def}\childdocforward{cdocsch2}"|
% \end{tabular}
% \end{center}
% Note that the trailing backslash on each first line
% merely continues the input to the second line
% (for convenient cut ant paste).
% Furthermore, the command |latex| can be replaced by any
% of its alternative versions such as |pdflatex|.
%
% %%%%%%%%%%%%%%%%%%%%%%%%%%%%%%%%%%%%%%%%%%%%%%%%%%%%%%%%%%%%%%%%%%%%%%%%%%%%%%
% %%%%%%%%%%%%%%%%%%%%%%%%%%%%%%%%%%%%%%%%%%%%%%%%%%%%%%%%%%%%%%%%%%%%%%%%%%%%%%
% \section{Implementation}
%\iffalse
%<*package>
%\fi
%
% This section describes the definitions file |childdoc.def|.

% The definitions cannot be loaded using |\usepackage| or |\RequirePackage|
% which has a mechanism to prevent loading a style file more than once.
% When loading the definitions by means of |\input|
% multiple instances have to be prevented manually:
%\iffalse
%This code needs to be before the `\ProvidesFile' directive
%which is defined at the beginning of this file.
%Therefore it is also placed there and commented out here.
%</package>
%<*discard>
%\fi
%    \begin{macrocode}
\ifdefined\childdocmain\endinput\fi
%    \end{macrocode}
%\iffalse
%</discard>
%<*package>
%\fi
%
% \macro{\ifchilddoc}
% \macro{\ifchilddocmanual}
% The conditional |\ifchilddoc| tells whether a
% child (true) or main (false) document is being compiled.
% The conditional |\ifchilddocmanual| tells whether
% the |\includeonly| mechanism is used (false) or
% the selection of child files must be performed manually (true).
% The definitions initialise to false:
%    \begin{macrocode}
\newif\ifchilddoc
\newif\ifchilddocmanual
%    \end{macrocode}

% \macro{\childdocname}
% \macro{\childdocjob}
% The macro |\childdocname| stores the name of the main document
% to be compiled. The macro |\childdocjob| stores the name of
% the document on which the \LaTeX{} compiler was originally invoked.
% The content of |\jobname| cannot be compared
% to filenames specified in the source due to different catcodes.
% The following code rescans |\jobname|, stores the result
% in |\childdocname| and saves a copy in |\childdocjob|:
%    \begin{macrocode}
\edef\childdocname{\scantokens\expandafter{\jobname\noexpand}}
\let\childdocjob\childdocname
%    \end{macrocode}

% \macro{\childdocdisable}
% The macro |\childdocdisable| prevents the main file
% from being processed more than once.
% At this stage, the main document command |\childdocmain|
% is assumed to be called once again where it should do nothing.
% Any subsequent call to it should prevent
% a secondary processing of the main document
% It overwrites the forwarding commands
% |\childdocof| and |\childdocforward|
% with empty macros to prevent further inclusions of the main document:
%    \begin{macrocode}
\newcommand{\childdocdisable}
{
  \renewcommand{\childdocmain}[1]{\renewcommand{\childdocmain}[1]{\endinput}}
  \renewcommand{\childdocof}[1]{}
  \renewcommand{\childdocby}[2][]{}
  \renewcommand{\childdocforward}[2][]{}
  \renewcommand{\childdocdisable}{}
}
%    \end{macrocode}

% \macro{\childdocmain}
% The macro |\childdocmain| is to be called at the top of the main file
% with nothing or the main filename (without extension) as argument.
% First, it breaks loops.
% If the argument is not empty and does not match |\childdocname|
% (which is set by the first inclusion of |childdoc.def|),
% |\ifchilddoc| is set to true, |\includeonly| is applied to the child file
% and |\jobname| is set to the main file
% (for proper handling of |.aux| files):
%    \begin{macrocode}
\newcommand{\childdocmain}[1]
{
  \childdocdisable\childdocmain{}
  \if?#1?\else
    \begingroup
      \def\childdoctmp{#1}
      \ifx\childdoctmp\childdocname
        \def\childdoctmp{}
      \else
        \def\childdoctmp
        {
          \childdoctrue
          \includeonly{\childdocname}
          \def\childdocjob{#1}
          \def\jobname{#1}
        }
      \fi
      \expandafter
    \endgroup
    \childdoctmp
  \fi
}
%    \end{macrocode}

% \macro{\childdocof}
% The command |\childdocof| redirects
% compilation to the main file |#1|.
%    \begin{macrocode}
\newcommand{\childdocof}[1]
{
  \childdocdisable
  \childdoctrue
  \includeonly{\childdocname}
  \def\jobname{#1}
  \def\childdocjob{#1}
  \input{#1}
}
%    \end{macrocode}

% \macro{\childdocby}
% The command |\childdocby| ....
%    \begin{macrocode}
\newcommand{\childdocby}[2][]
{
  \childdocdisable
  \childdoctrue
  \childdocmanualtrue
  \if?#1?\else
    \def\jobname{#2}
  \fi
  \def\childdocjob{#2}
  \input{#2}
  \endinput
}
%    \end{macrocode}

% \macro{\childdocforward}
% The command |\childdocforward| redirects
% compilation to the main file or
% (if the optional argument is given) a child file.
% Parameters are set as if the main file
% or a child file starting with |\childdocof| was compiled.
% Then compilation is handed over to the main file:
%    \begin{macrocode}
\newcommand{\childdocforward}[2][]
{
  \begingroup
    \if?#1?
      \def\childdoctmp
      {
        \def\childdocname{#2}
        \def\childdocjob{#2}
        \def\jobname{#2}
        \input{#2}
        \endinput
      }
    \else
      \def\childdoctmp
      {
        \childdocdisable
        \def\childdocname{#2}
        \childdoctrue
        \includeonly{#2}
        \def\childdocjob{#1}
        \def\jobname{#1}
        \input{#1}
        \endinput
      }
    \fi
    \expandafter
  \endgroup
  \childdoctmp
}
%    \end{macrocode}

% \macro{\childdocforwardprefix}
% The command |\childdocforwardprefix| redirects
% compilation to the main or a child file by means of a pattern.
% The prefix |#1| in the current filename is replaced by |#2|
% and the suffix of the current filename is kept
% (it is assumed that the filename does not contain the substring `|~~~|'
% which is used as a delimiter).
% Compilation is handed over to the new file by |\childdocforward|:
%    \begin{macrocode}
\newcommand{\childdocforwardprefix}[3][]
{
  \begingroup
    \def\childdocextract #2##1~~~{\def\childdoctmp{\childdocforward[#1]{#3##1}}}
    \expandafter\childdocextract\childdocname~~~
    \expandafter
  \endgroup
  \childdoctmp
}
%    \end{macrocode}

% \macro{\childdoc}
% The deprecated macro |\childdoc| is a legacy version of |\childdocmain|:
%    \begin{macrocode}
\newcommand{\childdoc}{\childdocmain}
%    \end{macrocode}

% \macro{\childdocredirect}
% The deprecated macro |\childdocredirect| is a legacy version
% of |\childdocforward| and |\childdocforwardprefix|:
%    \begin{macrocode}
\newcommand{\childdocredirect}[2][]
{
  \begingroup
    \if?#1?
      \def\childdoctmp{\childdocforward{#2}}
    \else
      \def\childdoctmp{\childdocforwardprefix{#1}{#2}}
    \fi
    \expandafter
  \endgroup
  \childdoctmp
}
%    \end{macrocode}

%\iffalse
%</package>
%\fi
%
\endinput
|\\
|\childdocforward[|\textit{main}|]{|\textit{dest}|}|\\
\end{tabular}
\end{center}
%
The argument \textit{dest} is the destination file
(without extension).
It should be the main file or one of the child files.
Note that further \textsf{childdoc} directives
such as |\childdocof| and |\childdocforward|
in the indicated file will be processed in this form.
The optional argument \textit{main}
passes on directly to the main file \textit{main}
while pretending to compile the child \textit{dest}.
This form behaves as if \textit{dest}
issues |\childdocof{|\textit{main}|}| right away,
and no further \textsf{childdoc} directives will be processed.

%%%%%%%%%%%%%%%%%%%%%%%%%%%%%%%%%%%%%%%%
\DescribeMacro{\...prefix}
In the alternative form |\childdocforwardprefix|,
%
\begin{center}
\begin{tabular}{l}
|% \iffalse
%
% childdoc.dtx Copyright (C) 2017-2018 Niklas Beisert
%
% This work may be distributed and/or modified under the
% conditions of the LaTeX Project Public License, either version 1.3
% of this license or (at your option) any later version.
% The latest version of this license is in
%   http://www.latex-project.org/lppl.txt
% and version 1.3 or later is part of all distributions of LaTeX
% version 2005/12/01 or later.
%
% This work has the LPPL maintenance status `maintained'.
%
% The Current Maintainer of this work is Niklas Beisert.
%
% This work consists of the files childdoc.dtx and childdoc.ins
% and the derived files childdoc.def and cdocsamp.tex with
% cdocsch1.tex, cdocsch2.tex, cdocsdrf.tex, cdocsfn1.tex, cdocsfn2.tex.
%
%<package>\ifdefined\childdocmain\endinput\fi
%<package>\ProvidesFile{childdoc.def}[2018/12/30 v2.0 child document driver]
%<samplemain>\ProvidesFile{cdocsamp.tex}[2018/12/30 v2.0 sample for childdoc]
%<*driver>
%\ProvidesFile{childdoc.drv}[2018/12/30 v2.0 childdoc reference manual file]
\PassOptionsToClass{10pt,a4paper}{article}
\documentclass{ltxdoc}

\usepackage[margin=35mm]{geometry}
\usepackage{hyperref}
\usepackage{hyperxmp}
\usepackage[usenames]{color}

\hypersetup{colorlinks=true}
\hypersetup{pdfstartview=FitH}
\hypersetup{pdfpagemode=UseNone}
\hypersetup{pdfsource={}}
\hypersetup{pdflang={en-UK}}
\hypersetup{pdfcopyright={Copyright 2017-2018 Niklas Beisert.
  This work may be distributed and/or modified under the
  conditions of the LaTeX Project Public License, either version 1.3
  of this license or (at your option) any later version.}}
\hypersetup{pdflicenseurl={http://www.latex-project.org/lppl.txt}}
\hypersetup{pdfcontactaddress={ETH Zurich, ITP, HIT K,
  Wolfgang-Pauli-Strasse 27}}
\hypersetup{pdfcontactpostcode={8093}}
\hypersetup{pdfcontactcity={Zurich}}
\hypersetup{pdfcontactcountry={Switzerland}}
\hypersetup{pdfcontactemail={nbeisert@itp.phys.ethz.ch}}
\hypersetup{pdfcontacturl={http://people.phys.ethz.ch/\xmptilde nbeisert/}}

\newcommand{\secref}[1]{\hyperref[#1]{section \ref*{#1}}}

\parskip1ex
\parindent0pt
\let\olditemize\itemize
\def\itemize{\olditemize\parskip0pt}

\begin{document}

\title{The \textsf{childdoc} Package}
\hypersetup{pdftitle={The childdoc Package}}
\author{Niklas Beisert\\[2ex]
  Institut f\"ur Theoretische Physik\\
  Eidgen\"ossische Technische Hochschule Z\"urich\\
  Wolfgang-Pauli-Strasse 27, 8093 Z\"urich, Switzerland\\[1ex]
  \href{mailto:nbeisert@itp.phys.ethz.ch}
  {\texttt{nbeisert@itp.phys.ethz.ch}}}
\hypersetup{pdfauthor={Niklas Beisert}}
\hypersetup{pdfsubject={Manual for the LaTeX2e Package childdoc}}
\date{30 December 2018, \textsf{v2.0}}
\maketitle

\begin{abstract}\noindent
\textsf{childdoc} is a \LaTeXe{} package
that enables the direct compilation
of document sections included by |\include|
to individual files.
\end{abstract}

\begingroup
\parskip0ex
\tableofcontents
\endgroup

%%%%%%%%%%%%%%%%%%%%%%%%%%%%%%%%%%%%%%%%%%%%%%%%%%%%%%%%%%%%%%%%%%%%%%%%%%%%%%%%
%%%%%%%%%%%%%%%%%%%%%%%%%%%%%%%%%%%%%%%%%%%%%%%%%%%%%%%%%%%%%%%%%%%%%%%%%%%%%%%%
\section{Introduction}

\LaTeX{} provides a mechanism to structure a large document (such as a book)
into a main file and several child files (containing the chapters)
using the |\include| command.
This mechanism is beneficial for documents
which span hundreds of pages in order to
make the source file(s) more manageable.
Moreover, compilation can be restricted to
selected child files by means of the |\includeonly| command.
The latter feature can be used to reduce the compilation time while editing
(this was significantly more useful in the earlier days of \LaTeX{})
or to generate a smaller document which is easier to navigate.
Another application of |\includeonly| is to generate
documents consisting of selected parts of the complete document.

However, there are a few drawbacks of the plain |\include| mechanism:
\begin{itemize}
\item
The child files cannot be compiled on their own,
they can only be compiled via the main file.
A naive editing environment
(such as a text editor with an option
to have the current file processed by \LaTeX)
may require one to switch to the main file before compiling;
attempting to compile the child file produces errors.
\item
The main file must be modified (each time)
to adjust the |\includeonly| command
to the present needs. This easily leaves the main file in a messy state.
\item
The generated document will always carry the filename
of the main document. This is inconvenient if
several child files are to be compiled and
to be kept for distribution.
\end{itemize}

The present package provides a simple interface
to make child files individually compilable by \LaTeX{}.
Compiling a child file then has the same effect as compiling
the main file with an |\includeonly| command
to select the appropriate child.
Moreover the generated document will carry the name of the child
rather than the main file.
This resolves all three above issues.

This feature is meant to make the editing of books,
thesis documents and lecture notes somewhat more convenient.
However, the package can also be used efficiently for
composing a series of documents (such as exercise sheets)
which are typically distributed individually.
It then assists the author in generating the individual documents
(potentially in different versions)
as well as a document containing the collected series.
Another application is in developing style files
or other kinds of included material
where compilation of the style file could redirect
to a sample or test file.

%%%%%%%%%%%%%%%%%%%%%%%%%%%%%%%%%%%%%%%%%%%%%%%%%%%%%%%%%%%%%%%%%%%%%%%%%%%%%%%%
%%%%%%%%%%%%%%%%%%%%%%%%%%%%%%%%%%%%%%%%%%%%%%%%%%%%%%%%%%%%%%%%%%%%%%%%%%%%%%%%
\section{Usage}

First of all, the package \textsf{childdoc} is \emph{not} a standard
\LaTeXe{} |.sty| style file! Therefore it needs to be invoked in
a non-standard way.

%%%%%%%%%%%%%%%%%%%%%%%%%%%%%%%%%%%%%%%%%%%%%%%%%%%%%%%%%%%%%%%%%%%%%%%%%%%%%%%%
\subsection{Included Files}
\label{sec:include}

%%%%%%%%%%%%%%%%%%%%%%%%%%%%%%%%%%%%%%%%
\DescribeMacro{\childdocmain}
To use the package, add the commands
\begin{center}
\begin{tabular}{l}
|\input{childdoc.def}|\\
|\childdocmain{}|\\
\end{tabular}
\end{center}
at the very top of the main \LaTeX{} file,
in particular \emph{before} the |\documentclass| statement!
The argument of |\childdocmain| should be left empty
(but it must be present).

%%%%%%%%%%%%%%%%%%%%%%%%%%%%%%%%%%%%%%%%
\DescribeMacro{\childdocof}
Furthermore, add the commands
\begin{center}
\begin{tabular}{l}
|\input{childdoc.def}|\\
|\childdocof{|\textit{main}|}|\\
\end{tabular}
\end{center}
at the top of every child file \textit{child}
which is included by |\include{|\textit{child}|}|
from within the main file
(or at least for those files to be compiled individually).
The argument \textit{main} must be the filename of the main file.

There are a couple of
considerations in setting up the main and child documents:

%%%%%%%%%%%%%%%%%%%%%%%%%%%%%%%%%%%%%%%%
\paragraph{Restrictions.}

Please note the following restrictions:
\begin{itemize}
\item
|\childdocmain| must be called with one argument \textit{main}
to ensure compatibility with earlier version of the package.
It must either be empty (|\childdocmain{}|)
or precisely match the filename of the main file in which it is specified.
See \secref{sec:detection} for further information.
\item
The filename \textit{main} must be specified without the |.tex| extension.
\item
The filename \textit{main} is case sensitive
(even in case-insensitive file systems)
due to internal string comparison.
\item
The argument \textit{main} should be fully expanded, it cannot be a macro.
\item
Subdirectories and special characters should be avoided in filenames.
\item
The command |\childdocmain{|\textit{main}|}| must be followed by a whitespace.
It should not be followed immediately by another command
or by a comment mark `|%|'.
This is because the \TeX{} parser reads the token immediately following
the argument of |\childdocmain| and puts it
at the beginning of every child section;
however, a white\-space is ignored.
\end{itemize}

%%%%%%%%%%%%%%%%%%%%%%%%%%%%%%%%%%%%%%%%
\paragraph{Content of Main File.}

It is advisable to place all content in the child files included by |\include|.
Any output contained in the main file will appear in all child documents
unless suppressed manually;
it cannot be suppressed automatically by the |\includeonly| directive
and thus should normally be avoided.
A method to include some content in the main file
by means of conditional processing is described in \secref{sec:conditional}.

%%%%%%%%%%%%%%%%%%%%%%%%%%%%%%%%%%%%%%%%
\paragraph{Page Numbering.}

When only a part of the document is compiled,
the appropriate numbering of pages
(as well as other status parameters)
is determined from the |.aux| files.
The latter contain information from previous passes.
However this information needs to propagate through
all intermediate child documents.
Therefore the page numbering in child documents may well
be inconsistent until the complete document is compiled at least once.

A useful (if unconventional) way to always ensure a consistent
page numbering is to restart the numbering in each child document
and denote the pages by `\textit{child}|.|\textit{page}'
where \textit{child} represents the chapter/section number of the child file.
This can be achieved by the command
|\numberwithin{page}{|\textit{child}|}|
of the \textsf{amsmath} package
where \textit{child} can be |chapter| or |section|
depending on the chosen structuring.
Alternatively, one can modify the macro |\thepage| appropriately
and reset the counter |page| at the start of each child file.

%%%%%%%%%%%%%%%%%%%%%%%%%%%%%%%%%%%%%%%%%%%%%%%%%%%%%%%%%%%%%%%%%%%%%%%%%%%%%%%%
\subsection{Conditional Processing}
\label{sec:conditional}

The package provides a mechanism to compile different versions
of a document. To customise the versions further some conditional processing
can come in handy to distinguish which version is being compiled.
The package provides two macros to describe the compilation context:

%%%%%%%%%%%%%%%%%%%%%%%%%%%%%%%%%%%%%%%%
\DescribeMacro{\ifchilddoc}
The conditional |\ifchilddoc| distinguishes between the compilation of
child documents and the main document:
%
\begin{center}
|\ifchilddoc |\textit{child-code}| |[|\||else |\textit{main-code}]| \||fi|
\end{center}

%%%%%%%%%%%%%%%%%%%%%%%%%%%%%%%%%%%%%%%%
\DescribeMacro{\childdocname}
\DescribeMacro{\childdocjob}
The macro |\childdocname| contains the filename (without extension)
of the main or child file being processed.
Note that |\childdocjob| will always contain the name of the main file.

%%%%%%%%%%%%%%%%%%%%%%%%%%%%%%%%%%%%%%%%
\paragraph{Title Page.}

Conditional processing can be used to include a title or banner page
in the main document when proper precautions are taken.
Importantly, the code in the main file should ensure that the page counter
(as well as other status parameters which are stored in the |.aux| files)
takes the same value after the conditional processing.
Otherwise the page numbers may take divergent values
depending on which part is compiled.

For example, a title page could be declared by:
%
\begin{center}
\begin{tabular}{l}
|\ifchilddoc\||else|\\
|\addtocounter{page}{-1}|\\
\textit{code for title page}\\
|\newpage|\\
|\||fi|
\end{tabular}
\end{center}
%
A banner page for the child documents can be generated by:
%
\begin{center}
\begin{tabular}{l}
|\ifchilddoc|\\
|\addtocounter{page}{-1}|\\
\textit{code for banner page}\\
|\newpage|\\
|\||fi|
\end{tabular}
\end{center}
%
Here one could write a message such as:
\begin{center}
|This is the part \childdocname{} of \childdocjob{}.|
\end{center}

%%%%%%%%%%%%%%%%%%%%%%%%%%%%%%%%%%%%%%%%%%%%%%%%%%%%%%%%%%%%%%%%%%%%%%%%%%%%%%%%
\subsection{Flags}
\label{sec:flags}

The package makes it easy to generate different versions
of the main or child documents.
To this end compilation flags can be defined
and assigned different default values.
They will be particularly useful in conjunction
with the forwarding mechanism described in \secref{sec:forward}.

For example, it may be useful to have a flag |\version|
which can be set to |draft| or |final|.
The document source will contain some conditional code
depending on the value of |\version|.
Suppose further, the flag should default to |final| for the main file
and to |draft| for child files
which is a natural assignment for editing the document.
This is achieved by placing the following code
in the preamble of the main document
(below the |\childdocmain| directive):
%
\begin{center}
\begin{tabular}{l}
|\ifchilddoc|\\
|\providecommand{\version}{draft}|\\
|\||else|\\
|\providecommand{\version}{final}|\\
|\||fi|
\end{tabular}
\end{center}
%
The definition by |\providecommand| makes sure
that previous definitions are not overwritten.
Further statements |\providecommand{\version}{...}|
can thus be added before the above code to override it.

For the main file, one might add a line
(between |\childdocmain| and the above block)
%
\begin{center}
|%\ifchilddoc\||else\providecommand{\version}{draft}\||fi|
\end{center}
%
which can be uncommented to produce a draft version.
Likewise one can add a line to the very top of a child file
(above the |\childdocof{|\textit{main}|}| directive)
%
\begin{center}
|%\providecommand{\version}{final}|
\end{center}
%
which can be uncommented to produce the final version of this child document.

%%%%%%%%%%%%%%%%%%%%%%%%%%%%%%%%%%%%%%%%%%%%%%%%%%%%%%%%%%%%%%%%%%%%%%%%%%%%%%%%
\subsection{Forwarding}
\label{sec:forward}

Different versions of the main or child documents
using compilation flags as described in \secref{sec:flags}
can be (permanently) stored in different files
for convenient compilation, viewing and distribution.
To this end, the package defines a command
to pass on compilation to a different file:

%%%%%%%%%%%%%%%%%%%%%%%%%%%%%%%%%%%%%%%%
\DescribeMacro{\childdocforward}
The command |\childdocforward| redirects processing to
another source file:
%
\begin{center}
\begin{tabular}{l}
|\input{childdoc.def}|\\
|\childdocforward[|\textit{main}|]{|\textit{dest}|}|\\
\end{tabular}
\end{center}
%
The argument \textit{dest} is the destination file
(without extension).
It should be the main file or one of the child files.
Note that further \textsf{childdoc} directives
such as |\childdocof| and |\childdocforward|
in the indicated file will be processed in this form.
The optional argument \textit{main}
passes on directly to the main file \textit{main}
while pretending to compile the child \textit{dest}.
This form behaves as if \textit{dest}
issues |\childdocof{|\textit{main}|}| right away,
and no further \textsf{childdoc} directives will be processed.

%%%%%%%%%%%%%%%%%%%%%%%%%%%%%%%%%%%%%%%%
\DescribeMacro{\...prefix}
In the alternative form |\childdocforwardprefix|,
%
\begin{center}
\begin{tabular}{l}
|\input{childdoc.def}|\\
|\childdocforwardprefix[|\textit{main}|]{|\textit{prefix}|}{|\textit{dest}|}|
\end{tabular}
\end{center}
%
the destination file is determined by a pattern
depending on the current file:
To make this work, the current file must be called
`{\textit{prefix}\hspace{0.2em}\textit{suffix}}'
with \textit{prefix} matching precisely the argument.
Processing is then passed on to the file
`{\textit{dest}\hspace{0.2em}\textit{suffix}}'.
Surely, the same effect is achieved by
directly specifying the
argument `{\textit{dest}\hspace{0.2em}\textit{suffix}}'
in the first form.
However, that requires to set up a different file
for each child. With the alternative form of the command
all these files can have exactly the same content
which simplifies setting them up and maintaining them.

For example, the following file |draft.tex|
with a compilation flag |\version| as described in \secref{sec:flags}
compiles the main document as a draft:
%
\begin{center}
\begin{tabular}{l}
|\def\version{draft}|\\
|\input{childdoc.def}|\\
|\childdocforward{|\textit{main}|}|
\end{tabular}
\end{center}
%
Likewise, the following files |final|\textit{nn}|.tex|
compile the final version of the child document
|child|\textit{nn}|.tex|:
%
\begin{center}
\begin{tabular}{l}
|\def\version{final}|\\
|\input{childdoc.def}|\\
|\childdocforwardprefix{final}{child}|
\end{tabular}
\end{center}
%

Note that when several versions of a main file and/or of each child file
are to be generated, it may be convenient to set up a |Makefile| or
shell script to automatise the process.

%%%%%%%%%%%%%%%%%%%%%%%%%%%%%%%%%%%%%%%%%%%%%%%%%%%%%%%%%%%%%%%%%%%%%%%%%%%%%%%%
\subsection{Command Line Processing}
\label{sec:commandline}

The effect of redirection files can also be achieved by invoking
the \LaTeX{} compiler with a more elaborate command line.
Most conveniently this should be done as part
of a shell script or a |Makefile|.

When using \textsf{childdoc} in the main file, the following
command lines effectively perform a redirection
(note that depending on the shell being used,
backslashes may have to be doubled: `|\|' $\to$ `|\\|'):
%
\begin{center}
|... -jobname "|\textit{target}|" |\\|"|[\textit{flags}]%
|\input{childdoc.def}\childdocforward[|\textit{main}|]{|\textit{dest}|}"|
\end{center}
%
Here \textit{target} is the name of the output file,
\textit{main} is the name of the main file
and \textit{dest} is the name of the main or child file to be processed
(all filenames without extensions).
The optional argument \textit{main} can be omitted
if \textit{main} matches \textit{dest}.
Optionally, compilation \textit{flags} can be defined via |\def| commands.
This command line makes the \TeX{} engine believe
it is compiling the file \textit{target}
whose content is specified as the latter parameter.
The provided code then forwards the processing to
\textit{main} or \textit{dest} as described in \secref{sec:forward}.

%%%%%%%%%%%%%%%%%%%%%%%%%%%%%%%%%%%%%%%%%%%%%%%%%%%%%%%%%%%%%%%%%%%%%%%%%%%%%%%%
\subsection{Include by Input}
\label{sec:input}

Including child documents by |\include| has some restrictions by design.
Most notably, the content of a child document always occupies
its own set of pages; pages cannot be shared between child documents.
Usually, this behaviour makes perfect sense
because each child document contain an essential part of the document.
However, in some situations it may be desirable to compose
a document from a collection of parts
without having mandatory page breaks between then.
For this case, the package
provides a mechanism to include parts
by |\input| which can also be processed individually.
However, by construction this mechanism
requires manual handling of the content to be output.

%%%%%%%%%%%%%%%%%%%%%%%%%%%%%%%%%%%%%%%%
\DescribeMacro{\ifchilddocmanual}
The main file should be prepared as usual, see \secref{sec:include}.
However, the document body must make a distinction
between processing of an individual part and of the main document, e.g.:
%
\begin{center}
\begin{tabular}{l}
|\ifchilddocmanual|\\
|\input{\childdocname}|\\
|\||else|\\
\textit{document body with }|\input{|\textit{part}|}|\\
|\||fi|
\end{tabular}
\end{center}
%
The conditional |\ifchilddocmanual| is true whenever
a part to be included by |\input| is being compiled,
and the name of the part is stored in |\childdocname|.

%%%%%%%%%%%%%%%%%%%%%%%%%%%%%%%%%%%%%%%%
\DescribeMacro{\childdocby}
Each part to be included by |\input| should start with:
%
\begin{center}
\begin{tabular}{l}
|\input{childdoc.def}|\\
|\childdocby{|\textit{main}|}|\\
\end{tabular}
\end{center}
%
The directive |\childdocby| is similar to |\childdocof|
described in \secref{sec:include},
but the subsequent selection of content must be done manually.
To that end, both |\ifchilddoc| and |\ifchilddocmanual|
will be true upon processing of a part,
and the name of the part is stored in |\childdocname|.
Note that |\jobname| will be set to the filename of the current part
so that each part receives an individual |.aux| file
that does not interfere with the |.aux| file(s) of the main document.
This behaviour can be altered by the alternative form
|\childdocby[*]{|\textit{main}|}| (with a non-empty optional argument)
which uses the |.aux| file of the main document
by setting |\jobname| to \textit{main}.

%%%%%%%%%%%%%%%%%%%%%%%%%%%%%%%%%%%%%%%%%%%%%%%%%%%%%%%%%%%%%%%%%%%%%%%%%%%%%%%%
\subsection{Driver Development}
\label{sec:driver}

The \textsf{childdoc} mechanism can also be use for the development
of definition files such as \LaTeX{} styles or classes.
This case differs from the above setup with multiple parts
included by |\include| in that no |\includeonly| should be invoked.
This can be achieved by starting the include file
(before |\ProvidesPackage|) with:
%
\begin{center}
\begin{tabular}{l}
|\input{childdoc.def}|\\
|\childdocforward{|\textit{main}|}|\\
\end{tabular}
\end{center}
%
or alternatively with:
%
\begin{center}
\begin{tabular}{l}
|\input{childdoc.def}|\\
|\childdocby{|\textit{main}|}|\\
\end{tabular}
\end{center}
%
Both forms have slightly different effects as described above.
The main file is prepared as usual, see \secref{sec:include}.

%%%%%%%%%%%%%%%%%%%%%%%%%%%%%%%%%%%%%%%%%%%%%%%%%%%%%%%%%%%%%%%%%%%%%%%%%%%%%%%%
\subsection{Legacy Detection}
\label{sec:detection}

The directive |\childdocmain| in the main file can detect
whether the complete document or merely a child is to be compiled
even without using the directive |\childdocof|.
This method is deprecated because it is less robust
and there is no compelling reason to use it;
it is merely provided for backward compatibility
and it may be removed in future versions.

If the detection mechanism is to be used,
it is mandatory to correctly specify
the filename of the main file as the argument of |\childdocmain|:
%
\begin{center}
\begin{tabular}{l}
|\input{childdoc.def}|\\
|\childdocmain{|\textit{main}|}|\\
\end{tabular}
\end{center}
%
If |\jobname| does not match the argument \textit{main} of |\childdocmain|,
it is assumed that |\jobname| points to the child file to be compiled.
When using |\childdocmain| with the main file specified as argument,
it suffices to start a child file
with just |\input{|\textit{main}|}|
without loading of the package and using |\childdocof|.
If instead all processing is done
with the appropriate \textsf{childdoc} directives,
the argument of \textit{main} of |\childdocmain| can be empty.

An alternative version of the command line processing described
in \secref{sec:commandline} using the detection mechanism reads:
%
\begin{center}
|... -jobname "|\textit{target}|" "|[\textit{flags}]%
[|\def\jobname{|\textit{dest}|}|]|\input{|\textit{main}|}"|
\end{center}

%%%%%%%%%%%%%%%%%%%%%%%%%%%%%%%%%%%%%%%%%%%%%%%%%%%%%%%%%%%%%%%%%%%%%%%%%%%%%%%%
\subsection{Manual Code}
\label{sec:manual}

In case one cannot be certain whether the definitions file |childdoc.def|
is installed on the target \TeX{} distribution
and one prefers not to ship it,
it is conceivable to paste a few relevant commands into the sources.

To that end, drop all statements |\input{childdoc.def}|
and perform the replacements as outlined below.
Instead of |\childdocmain{|\textit{main}|}| add the following code
to the top of the main file:
%
\begin{center}
\begin{tabular}{l}
|\||ifdefined\childdocname\endinput\||fi\newif\ifchilddoc|\\
|\edef\childdocname{\scantokens\expandafter{\jobname\noexpand}}|\\
|\def\childdocmain{|\textit{main}|}\||ifx\childdocmain\childdocname\||else|\\
|\childdoctrue\includeonly{\childdocname}\let\jobname\childdocmain\||fi|\\
\end{tabular}
\end{center}
%
Instead of |\childdocof{|\textit{main}|}| just include the main file
at the top of each child file:
%
\begin{center}
|\input{|\textit{main}|}|
\end{center}
%
A simple redirection |\childdocforward{|\textit{dest}|}| is achieved by:
%
\begin{center}
|\def\jobname{|\textit{dest}|}\input{\jobname}|
\end{center}
%
The redirection with prefix
|\childdocforwardprefix[|\textit{prefix}|]{|\textit{dest}|}|
is accomplished by:
%
\begin{center}
\begin{tabular}{l}
|{\edef\jobname{\scantokens\expandafter{\jobname\noexpand}}|\\
|\def\redirectjob |\textit{prefix}|#1~~~{\gdef\jobname{|\textit{dest}|#1}}|\\
|\expandafter\redirectjob\jobname~~~}\input{\jobname}|
\end{tabular}
\end{center}

In an alternative approach,
child documents can be compiled by a specific command line
without additional code or specific definitions:
%
\begin{center}
|... -jobname "|\textit{target}|" "|[\textit{flags}]%
|\includeonly{|\textit{dest}|}\input{|\textit{main}|}"|
\end{center}
%

%%%%%%%%%%%%%%%%%%%%%%%%%%%%%%%%%%%%%%%%%%%%%%%%%%%%%%%%%%%%%%%%%%%%%%%%%%%%%%%%
%%%%%%%%%%%%%%%%%%%%%%%%%%%%%%%%%%%%%%%%%%%%%%%%%%%%%%%%%%%%%%%%%%%%%%%%%%%%%%%%
\section{Information}

%%%%%%%%%%%%%%%%%%%%%%%%%%%%%%%%%%%%%%%%%%%%%%%%%%%%%%%%%%%%%%%%%%%%%%%%%%%%%%%%
\subsection{Copyright}

Copyright \copyright{} 2017--2018 Niklas Beisert

This work may be distributed and/or modified under the
conditions of the \LaTeX{} Project Public License, either version 1.3
of this license or (at your option) any later version.
The latest version of this license is in
  \url{http://www.latex-project.org/lppl.txt}
and version 1.3 or later is part of all distributions of \LaTeX{}
version 2005/12/01 or later.

This work has the LPPL maintenance status `maintained'.

The Current Maintainer of this work is Niklas Beisert.

This work consists of the files |README.txt|, |childdoc.ins| and |childdoc.dtx|
as well as the derived files |childdoc.def|, |cdocsamp.tex|
with |cdocsch1.tex|, |cdocsch2.tex|, |cdocspt3.tex|, |cdocspt4.tex|,
|cdocsdrf.tex|, |cdocsfn1.tex|, |cdocsfn2.tex|
as well as |childdoc.pdf|.

%%%%%%%%%%%%%%%%%%%%%%%%%%%%%%%%%%%%%%%%%%%%%%%%%%%%%%%%%%%%%%%%%%%%%%%%%%%%%%%%
\subsection{Files and Installation}

The package consists of the files:
%
\begin{center}
\begin{tabular}{ll}
    |README.txt|   & readme file \\
    |childdoc.ins| & installation file \\
    |childdoc.dtx| & source file \\
    |childdoc.def| & definition file \\
    |cdocsamp.tex| & sample main file \\
    |cdocsch1.tex| & sample include file \\
    |cdocsch2.tex| & sample include file \\
    |cdocspt3.tex| & sample part file \\
    |cdocspt4.tex| & sample part file \\
    |cdocsdrf.tex| & sample redirection file \\
    |cdocsfn1.tex| & sample redirection file \\
    |cdocsfn2.tex| & sample redirection file \\
    |childdoc.pdf| & manual
\end{tabular}
\end{center}
%
The distribution consists of the files
|README.txt|, |childdoc.ins| and |childdoc.dtx|.
%
\begin{itemize}
\item
Run (pdf)\LaTeX{} on |childdoc.dtx|
to compile the manual |childdoc.pdf| (this file).
\item
Run \LaTeX{} on |childdoc.ins| to create the definitions file |childdoc.def|
and the sample |cdocsamp.tex| with include files
|cdocsch1.tex|, |cdocsch2.tex|, |cdocspt3.tex|, |cdocspt4.tex|,
|cdocsdrf.tex|, |cdocsfn1.tex|, |cdocsfn2.tex|.
Then copy the file |childdoc.def| to an appropriate directory of your \LaTeX{}
distribution, e.g.\ \textit{texmf-root}|/tex/latex/childdoc|.
\end{itemize}

%%%%%%%%%%%%%%%%%%%%%%%%%%%%%%%%%%%%%%%%%%%%%%%%%%%%%%%%%%%%%%%%%%%%%%%%%%%%%%%%
\subsection{Related CTAN Packages}

There are several other packages which offer a similar functionality:
%
\begin{itemize}
\item
The packages
\href{http://ctan.org/pkg/docmute}{\textsf{docmute}},
\href{http://ctan.org/pkg/includex}{\textsf{includex}} and
\href{http://ctan.org/pkg/standalone}{\textsf{standalone}}
provide commands to include only the document body of
a child file thus allowing both files to be compiled individually.
\item
The packages \href{http://ctan.org/pkg/subdocs}{\textsf{subdocs}}
and \href{http://ctan.org/pkg/subfiles}{\textsf{subfiles}}
provide structures in which the main and child documents can be
encapsulated and allowing them to be compiled individually.
The inclusion mechanism is different from the conventional |\include|.
\item
The package \href{http://ctan.org/pkg/combine}{\textsf{combine}}
is an elaborate solution to combine several documents into one.
\end{itemize}
%
See also the CTAN topic \href{http://ctan.org/topic/subdocs}{\textsf{subdocs}}
for further related packages.
The present package differs from the above solutions in that
a document structure constructed with the conventional |\include| mechanism
just needs two extra commands at the top of every file
such that all constituent files can be compiled individually.

%%%%%%%%%%%%%%%%%%%%%%%%%%%%%%%%%%%%%%%%%%%%%%%%%%%%%%%%%%%%%%%%%%%%%%%%%%%%%%%%
%\subsection{Feature Suggestions}
%
%The following is a list of features which may be useful for future
%versions of this package:
%%
%\begin{itemize}
%\item
%\ldots
%\end{itemize}

%%%%%%%%%%%%%%%%%%%%%%%%%%%%%%%%%%%%%%%%%%%%%%%%%%%%%%%%%%%%%%%%%%%%%%%%%%%%%%%%
\subsection{Revision History}

%%%%%%%%%%%%%%%%%%%%%%%%%%%%%%%%%%%%%%%%
\paragraph{v2.0:} 2018/12/30

\begin{itemize}
\item
immediate forward processing
\item
added |\childdocby| mechanism
\item
manual restructured
\end{itemize}

%%%%%%%%%%%%%%%%%%%%%%%%%%%%%%%%%%%%%%%%
\paragraph{v1.6:} 2018/01/17

\begin{itemize}
\item
application for development of include files
\item
corrections to manual
\end{itemize}

%%%%%%%%%%%%%%%%%%%%%%%%%%%%%%%%%%%%%%%%
\paragraph{v1.5:} 2017/05/21

\begin{itemize}
\item
more complete structuring introduced
\item
|\childdocof| introduced
\item
|\childdoc| renamed to |\childdocmain|
\item
|\childredirect| renamed to |\childdocforward| and |\childdocforwardprefix|
and functionality expanded
\end{itemize}

%%%%%%%%%%%%%%%%%%%%%%%%%%%%%%%%%%%%%%%%
\paragraph{v1.0:} 2017/04/27

\begin{itemize}
\item
manual and install package
\item
first version published on CTAN
\end{itemize}

%%%%%%%%%%%%%%%%%%%%%%%%%%%%%%%%%%%%%%%%
\paragraph{v0.6:} 2017/04/26

\begin{itemize}
\item
redirection mechanism added
\end{itemize}

%%%%%%%%%%%%%%%%%%%%%%%%%%%%%%%%%%%%%%%%
\paragraph{v0.5:} 2017/04/26

\begin{itemize}
\item
functionality in definition file
\end{itemize}


%%%%%%%%%%%%%%%%%%%%%%%%%%%%%%%%%%%%%%%%%%%%%%%%%%%%%%%%%%%%%%%%%%%%%%%%%%%%%%%%
%%%%%%%%%%%%%%%%%%%%%%%%%%%%%%%%%%%%%%%%%%%%%%%%%%%%%%%%%%%%%%%%%%%%%%%%%%%%%%%%
%%%%%%%%%%%%%%%%%%%%%%%%%%%%%%%%%%%%%%%%%%%%%%%%%%%%%%%%%%%%%%%%%%%%%%%%%%%%%%%%
\appendix

\settowidth\MacroIndent{\rmfamily\scriptsize 000\ }

 \DocInput{childdoc.dtx}

\end{document}
%</driver>
% \fi
%
% %%%%%%%%%%%%%%%%%%%%%%%%%%%%%%%%%%%%%%%%%%%%%%%%%%%%%%%%%%%%%%%%%%%%%%%%%%%%%%
% %%%%%%%%%%%%%%%%%%%%%%%%%%%%%%%%%%%%%%%%%%%%%%%%%%%%%%%%%%%%%%%%%%%%%%%%%%%%%%
% \section{Sample}
%\iffalse
%<*samplemain>
%\fi
%
% The following presents a sample document
% with two chapters, two parts, a title page,
% a compile flag as well as three forwarding files to set the flag.
% It consists of eight |.tex| files:
% \begin{center}
% \begin{tabular}{ll}
% |cdocsamp.tex|&main file\\
% |cdocsch1.tex|&include file for chapter 1\\
% |cdocsch2.tex|&include file for chapter 2\\
% |cdocspt3.tex|&include file for part 3\\
% |cdocspt4.tex|&include file for part 4\\
% |cdocsdrf.tex|&forwarding file for main file in draft mode\\
% |cdocsfi1.tex|&forwarding file for final version of chapter 1\\
% |cdocsfi2.tex|&forwarding file for final version of chapter 2\\
% \end{tabular}
% \end{center}
% Each of the eight files can be compiled directly by the \LaTeX{} compiler.
%
% %%%%%%%%%%%%%%%%%%%%%%%%%%%%%%%%%%%%%%
% \paragraph{Main File.}
%
% The main file is called |cdocsamp.tex|.
%
% Load the \textsf{childdoc} definitions and
% declare the filename for the main document:
%    \begin{macrocode}
\input{childdoc.def}
\childdocmain{}
%    \end{macrocode}

% Optional override for |\version| flag:
%    \begin{macrocode}
%%\ifchilddoc\else\providecommand{\version}{draft}\fi
%    \end{macrocode}

% Define the default values for the |\version| flag
% (|final| for the main file and |draft| for childs):
%    \begin{macrocode}
\ifchilddoc
\providecommand{\version}{draft}
\else
\providecommand{\version}{final}
\fi
%    \end{macrocode}

% Load the standard document class:
%    \begin{macrocode}
\documentclass[12pt]{article}
%    \end{macrocode}

% Start the document body:
%    \begin{macrocode}
\begin{document}
%    \end{macrocode}

% Declare a title page.
% Print title, part of document being processed and version flag:
%    \begin{macrocode}
\addtocounter{page}{-1}
\begin{center}
{\LARGE\bfseries{}childdoc example\par}
\vspace{1cm}
\ifchilddoc
\ifchilddocmanual part\else chapter\fi:
`\childdocname' of `\childdocjob'\par
\else
main document: `\childdocjob'\par
\fi
version: \version\par
\end{center}
\newpage
%    \end{macrocode}

% Manually include selected file,
% otherwise process as usual:
%    \begin{macrocode}
\ifchilddocmanual
\section*{part `\childdocname'}
\input{\childdocname}
\else
%    \end{macrocode}

% Include the two chapters:
%    \begin{macrocode}
\include{cdocsch1}
\include{cdocsch2}
%    \end{macrocode}

% Include the two parts unless only chapters should be displayed:
%    \begin{macrocode}
\ifchilddoc\else
\section{part three}
\input{cdocspt3}
\section{part four}
\input{cdocspt4}
\fi
%    \end{macrocode}

% Process as usual until here:
%    \begin{macrocode}
\fi
%    \end{macrocode}

% End of document body:
%    \begin{macrocode}
\end{document}
%    \end{macrocode}
%\iffalse
%</samplemain>
%\fi
%
% %%%%%%%%%%%%%%%%%%%%%%%%%%%%%%%%%%%%%%
% \paragraph{Chapter Include Files.}
%
% The include files are called |cdocsch1.tex| and |cdocsch2.tex|.
%
%\iffalse
%<*samplechap1|samplechap2>
%\fi

% Optional override for |\version| flag:
%    \begin{macrocode}
%%\providecommand{\version}{final}
%    \end{macrocode}

% Include the main document:
%    \begin{macrocode}
\input{childdoc.def}
\childdocof{cdocsamp}
%    \end{macrocode}

%\iffalse
%</samplechap1|samplechap2>
%\fi
%
%\iffalse
%<*samplechap1>
%\fi
% Some text for chapter 1:
%    \begin{macrocode}
\section{one}
some text in chapter one
%    \end{macrocode}

%\iffalse
%</samplechap1>
%\fi
% Some text for chapter 2:
%\iffalse
%<*samplechap2>
%\fi
%    \begin{macrocode}
\section{two}
more text in chapter two
%    \end{macrocode}

%\iffalse
%</samplechap2>
%\fi
%
% %%%%%%%%%%%%%%%%%%%%%%%%%%%%%%%%%%%%%%
% \paragraph{Part Include Files.}
%
% The include files are called |cdocspt3.tex| and |cdocspt4.tex|.
%
%\iffalse
%<*samplepart3|samplepart4>
%\fi

% Optional override for |\version| flag:
%    \begin{macrocode}
%%\providecommand{\version}{final}
%    \end{macrocode}

% Include the main document:
%    \begin{macrocode}
\input{childdoc.def}
\childdocby{cdocsamp}
%    \end{macrocode}

%\iffalse
%</samplepart3|samplepart4>
%\fi
%
%\iffalse
%<*samplepart3>
%\fi
% Some text for part 3:
%    \begin{macrocode}
some text in part three
%    \end{macrocode}

%\iffalse
%</samplepart3>
%\fi
% Some text for part 4:
%\iffalse
%<*samplepart4>
%\fi
%    \begin{macrocode}
more text in part four
%    \end{macrocode}

%\iffalse
%</samplepart4>
%\fi
%
% %%%%%%%%%%%%%%%%%%%%%%%%%%%%%%%%%%%%%%
% \paragraph{Forwarding for a Complete Draft.}
%
% The following forwarding file |cdocsdrf.tex|
% compiles the main document in draft mode:
%\iffalse
%<*sampledraft>
%\fi
%    \begin{macrocode}
\def\version{draft}
\input{childdoc.def}
\childdocforward{cdocsamp}
%    \end{macrocode}

%\iffalse
%</sampledraft>
%\fi
%
% %%%%%%%%%%%%%%%%%%%%%%%%%%%%%%%%%%%%%%
% \paragraph{Forwarding for Final Version of the Chapters.}
%
% The following forwarding files |cdocsfn1.tex| and |cdocsfn2.tex|
% (with identical content)
% compile the final versions of the child documents
% |cdocsch1.tex| and |cdocsch2.tex|, respectively:
%\iffalse
%<*samplefinal>
%\fi
%    \begin{macrocode}
\def\version{final}
\input{childdoc.def}
\childdocforwardprefix[cdocsamp]{cdocsfn}{cdocsch}
%    \end{macrocode}

%\iffalse
%</samplefinal>
%\fi
%
% %%%%%%%%%%%%%%%%%%%%%%%%%%%%%%%%%%%%%%
% \paragraph{Command Line Processing.}
%
% The following three command lines generate the output files
% |cdocscld|, |cdocscl1| and |cdocscl2|
% which should be identical to
% |cdocsdrf|, |cdocsch1| and |cdocsfn2|, respectively:
% \begin{center}
% \begin{tabular}{l}
% |latex -jobname cdocscld \|\\
% |  "\def\version{draft}\input{childdoc.def}\childdocforward{cdocsamp}"|\\
% |latex -jobname cdocscl1 \|\\
% |  "\input{childdoc.def}\childdocforward[cdocsamp]{cdocsch1}"|\\
% |latex -jobname cdocscl2 \|\\
% |  "\def\version{final}\input{childdoc.def}\childdocforward{cdocsch2}"|
% \end{tabular}
% \end{center}
% Note that the trailing backslash on each first line
% merely continues the input to the second line
% (for convenient cut ant paste).
% Furthermore, the command |latex| can be replaced by any
% of its alternative versions such as |pdflatex|.
%
% %%%%%%%%%%%%%%%%%%%%%%%%%%%%%%%%%%%%%%%%%%%%%%%%%%%%%%%%%%%%%%%%%%%%%%%%%%%%%%
% %%%%%%%%%%%%%%%%%%%%%%%%%%%%%%%%%%%%%%%%%%%%%%%%%%%%%%%%%%%%%%%%%%%%%%%%%%%%%%
% \section{Implementation}
%\iffalse
%<*package>
%\fi
%
% This section describes the definitions file |childdoc.def|.

% The definitions cannot be loaded using |\usepackage| or |\RequirePackage|
% which has a mechanism to prevent loading a style file more than once.
% When loading the definitions by means of |\input|
% multiple instances have to be prevented manually:
%\iffalse
%This code needs to be before the `\ProvidesFile' directive
%which is defined at the beginning of this file.
%Therefore it is also placed there and commented out here.
%</package>
%<*discard>
%\fi
%    \begin{macrocode}
\ifdefined\childdocmain\endinput\fi
%    \end{macrocode}
%\iffalse
%</discard>
%<*package>
%\fi
%
% \macro{\ifchilddoc}
% \macro{\ifchilddocmanual}
% The conditional |\ifchilddoc| tells whether a
% child (true) or main (false) document is being compiled.
% The conditional |\ifchilddocmanual| tells whether
% the |\includeonly| mechanism is used (false) or
% the selection of child files must be performed manually (true).
% The definitions initialise to false:
%    \begin{macrocode}
\newif\ifchilddoc
\newif\ifchilddocmanual
%    \end{macrocode}

% \macro{\childdocname}
% \macro{\childdocjob}
% The macro |\childdocname| stores the name of the main document
% to be compiled. The macro |\childdocjob| stores the name of
% the document on which the \LaTeX{} compiler was originally invoked.
% The content of |\jobname| cannot be compared
% to filenames specified in the source due to different catcodes.
% The following code rescans |\jobname|, stores the result
% in |\childdocname| and saves a copy in |\childdocjob|:
%    \begin{macrocode}
\edef\childdocname{\scantokens\expandafter{\jobname\noexpand}}
\let\childdocjob\childdocname
%    \end{macrocode}

% \macro{\childdocdisable}
% The macro |\childdocdisable| prevents the main file
% from being processed more than once.
% At this stage, the main document command |\childdocmain|
% is assumed to be called once again where it should do nothing.
% Any subsequent call to it should prevent
% a secondary processing of the main document
% It overwrites the forwarding commands
% |\childdocof| and |\childdocforward|
% with empty macros to prevent further inclusions of the main document:
%    \begin{macrocode}
\newcommand{\childdocdisable}
{
  \renewcommand{\childdocmain}[1]{\renewcommand{\childdocmain}[1]{\endinput}}
  \renewcommand{\childdocof}[1]{}
  \renewcommand{\childdocby}[2][]{}
  \renewcommand{\childdocforward}[2][]{}
  \renewcommand{\childdocdisable}{}
}
%    \end{macrocode}

% \macro{\childdocmain}
% The macro |\childdocmain| is to be called at the top of the main file
% with nothing or the main filename (without extension) as argument.
% First, it breaks loops.
% If the argument is not empty and does not match |\childdocname|
% (which is set by the first inclusion of |childdoc.def|),
% |\ifchilddoc| is set to true, |\includeonly| is applied to the child file
% and |\jobname| is set to the main file
% (for proper handling of |.aux| files):
%    \begin{macrocode}
\newcommand{\childdocmain}[1]
{
  \childdocdisable\childdocmain{}
  \if?#1?\else
    \begingroup
      \def\childdoctmp{#1}
      \ifx\childdoctmp\childdocname
        \def\childdoctmp{}
      \else
        \def\childdoctmp
        {
          \childdoctrue
          \includeonly{\childdocname}
          \def\childdocjob{#1}
          \def\jobname{#1}
        }
      \fi
      \expandafter
    \endgroup
    \childdoctmp
  \fi
}
%    \end{macrocode}

% \macro{\childdocof}
% The command |\childdocof| redirects
% compilation to the main file |#1|.
%    \begin{macrocode}
\newcommand{\childdocof}[1]
{
  \childdocdisable
  \childdoctrue
  \includeonly{\childdocname}
  \def\jobname{#1}
  \def\childdocjob{#1}
  \input{#1}
}
%    \end{macrocode}

% \macro{\childdocby}
% The command |\childdocby| ....
%    \begin{macrocode}
\newcommand{\childdocby}[2][]
{
  \childdocdisable
  \childdoctrue
  \childdocmanualtrue
  \if?#1?\else
    \def\jobname{#2}
  \fi
  \def\childdocjob{#2}
  \input{#2}
  \endinput
}
%    \end{macrocode}

% \macro{\childdocforward}
% The command |\childdocforward| redirects
% compilation to the main file or
% (if the optional argument is given) a child file.
% Parameters are set as if the main file
% or a child file starting with |\childdocof| was compiled.
% Then compilation is handed over to the main file:
%    \begin{macrocode}
\newcommand{\childdocforward}[2][]
{
  \begingroup
    \if?#1?
      \def\childdoctmp
      {
        \def\childdocname{#2}
        \def\childdocjob{#2}
        \def\jobname{#2}
        \input{#2}
        \endinput
      }
    \else
      \def\childdoctmp
      {
        \childdocdisable
        \def\childdocname{#2}
        \childdoctrue
        \includeonly{#2}
        \def\childdocjob{#1}
        \def\jobname{#1}
        \input{#1}
        \endinput
      }
    \fi
    \expandafter
  \endgroup
  \childdoctmp
}
%    \end{macrocode}

% \macro{\childdocforwardprefix}
% The command |\childdocforwardprefix| redirects
% compilation to the main or a child file by means of a pattern.
% The prefix |#1| in the current filename is replaced by |#2|
% and the suffix of the current filename is kept
% (it is assumed that the filename does not contain the substring `|~~~|'
% which is used as a delimiter).
% Compilation is handed over to the new file by |\childdocforward|:
%    \begin{macrocode}
\newcommand{\childdocforwardprefix}[3][]
{
  \begingroup
    \def\childdocextract #2##1~~~{\def\childdoctmp{\childdocforward[#1]{#3##1}}}
    \expandafter\childdocextract\childdocname~~~
    \expandafter
  \endgroup
  \childdoctmp
}
%    \end{macrocode}

% \macro{\childdoc}
% The deprecated macro |\childdoc| is a legacy version of |\childdocmain|:
%    \begin{macrocode}
\newcommand{\childdoc}{\childdocmain}
%    \end{macrocode}

% \macro{\childdocredirect}
% The deprecated macro |\childdocredirect| is a legacy version
% of |\childdocforward| and |\childdocforwardprefix|:
%    \begin{macrocode}
\newcommand{\childdocredirect}[2][]
{
  \begingroup
    \if?#1?
      \def\childdoctmp{\childdocforward{#2}}
    \else
      \def\childdoctmp{\childdocforwardprefix{#1}{#2}}
    \fi
    \expandafter
  \endgroup
  \childdoctmp
}
%    \end{macrocode}

%\iffalse
%</package>
%\fi
%
\endinput
|\\
|\childdocforwardprefix[|\textit{main}|]{|\textit{prefix}|}{|\textit{dest}|}|
\end{tabular}
\end{center}
%
the destination file is determined by a pattern
depending on the current file:
To make this work, the current file must be called
`{\textit{prefix}\hspace{0.2em}\textit{suffix}}'
with \textit{prefix} matching precisely the argument.
Processing is then passed on to the file
`{\textit{dest}\hspace{0.2em}\textit{suffix}}'.
Surely, the same effect is achieved by
directly specifying the
argument `{\textit{dest}\hspace{0.2em}\textit{suffix}}'
in the first form.
However, that requires to set up a different file
for each child. With the alternative form of the command
all these files can have exactly the same content
which simplifies setting them up and maintaining them.

For example, the following file |draft.tex|
with a compilation flag |\version| as described in \secref{sec:flags}
compiles the main document as a draft:
%
\begin{center}
\begin{tabular}{l}
|\def\version{draft}|\\
|% \iffalse
%
% childdoc.dtx Copyright (C) 2017-2018 Niklas Beisert
%
% This work may be distributed and/or modified under the
% conditions of the LaTeX Project Public License, either version 1.3
% of this license or (at your option) any later version.
% The latest version of this license is in
%   http://www.latex-project.org/lppl.txt
% and version 1.3 or later is part of all distributions of LaTeX
% version 2005/12/01 or later.
%
% This work has the LPPL maintenance status `maintained'.
%
% The Current Maintainer of this work is Niklas Beisert.
%
% This work consists of the files childdoc.dtx and childdoc.ins
% and the derived files childdoc.def and cdocsamp.tex with
% cdocsch1.tex, cdocsch2.tex, cdocsdrf.tex, cdocsfn1.tex, cdocsfn2.tex.
%
%<package>\ifdefined\childdocmain\endinput\fi
%<package>\ProvidesFile{childdoc.def}[2018/12/30 v2.0 child document driver]
%<samplemain>\ProvidesFile{cdocsamp.tex}[2018/12/30 v2.0 sample for childdoc]
%<*driver>
%\ProvidesFile{childdoc.drv}[2018/12/30 v2.0 childdoc reference manual file]
\PassOptionsToClass{10pt,a4paper}{article}
\documentclass{ltxdoc}

\usepackage[margin=35mm]{geometry}
\usepackage{hyperref}
\usepackage{hyperxmp}
\usepackage[usenames]{color}

\hypersetup{colorlinks=true}
\hypersetup{pdfstartview=FitH}
\hypersetup{pdfpagemode=UseNone}
\hypersetup{pdfsource={}}
\hypersetup{pdflang={en-UK}}
\hypersetup{pdfcopyright={Copyright 2017-2018 Niklas Beisert.
  This work may be distributed and/or modified under the
  conditions of the LaTeX Project Public License, either version 1.3
  of this license or (at your option) any later version.}}
\hypersetup{pdflicenseurl={http://www.latex-project.org/lppl.txt}}
\hypersetup{pdfcontactaddress={ETH Zurich, ITP, HIT K,
  Wolfgang-Pauli-Strasse 27}}
\hypersetup{pdfcontactpostcode={8093}}
\hypersetup{pdfcontactcity={Zurich}}
\hypersetup{pdfcontactcountry={Switzerland}}
\hypersetup{pdfcontactemail={nbeisert@itp.phys.ethz.ch}}
\hypersetup{pdfcontacturl={http://people.phys.ethz.ch/\xmptilde nbeisert/}}

\newcommand{\secref}[1]{\hyperref[#1]{section \ref*{#1}}}

\parskip1ex
\parindent0pt
\let\olditemize\itemize
\def\itemize{\olditemize\parskip0pt}

\begin{document}

\title{The \textsf{childdoc} Package}
\hypersetup{pdftitle={The childdoc Package}}
\author{Niklas Beisert\\[2ex]
  Institut f\"ur Theoretische Physik\\
  Eidgen\"ossische Technische Hochschule Z\"urich\\
  Wolfgang-Pauli-Strasse 27, 8093 Z\"urich, Switzerland\\[1ex]
  \href{mailto:nbeisert@itp.phys.ethz.ch}
  {\texttt{nbeisert@itp.phys.ethz.ch}}}
\hypersetup{pdfauthor={Niklas Beisert}}
\hypersetup{pdfsubject={Manual for the LaTeX2e Package childdoc}}
\date{30 December 2018, \textsf{v2.0}}
\maketitle

\begin{abstract}\noindent
\textsf{childdoc} is a \LaTeXe{} package
that enables the direct compilation
of document sections included by |\include|
to individual files.
\end{abstract}

\begingroup
\parskip0ex
\tableofcontents
\endgroup

%%%%%%%%%%%%%%%%%%%%%%%%%%%%%%%%%%%%%%%%%%%%%%%%%%%%%%%%%%%%%%%%%%%%%%%%%%%%%%%%
%%%%%%%%%%%%%%%%%%%%%%%%%%%%%%%%%%%%%%%%%%%%%%%%%%%%%%%%%%%%%%%%%%%%%%%%%%%%%%%%
\section{Introduction}

\LaTeX{} provides a mechanism to structure a large document (such as a book)
into a main file and several child files (containing the chapters)
using the |\include| command.
This mechanism is beneficial for documents
which span hundreds of pages in order to
make the source file(s) more manageable.
Moreover, compilation can be restricted to
selected child files by means of the |\includeonly| command.
The latter feature can be used to reduce the compilation time while editing
(this was significantly more useful in the earlier days of \LaTeX{})
or to generate a smaller document which is easier to navigate.
Another application of |\includeonly| is to generate
documents consisting of selected parts of the complete document.

However, there are a few drawbacks of the plain |\include| mechanism:
\begin{itemize}
\item
The child files cannot be compiled on their own,
they can only be compiled via the main file.
A naive editing environment
(such as a text editor with an option
to have the current file processed by \LaTeX)
may require one to switch to the main file before compiling;
attempting to compile the child file produces errors.
\item
The main file must be modified (each time)
to adjust the |\includeonly| command
to the present needs. This easily leaves the main file in a messy state.
\item
The generated document will always carry the filename
of the main document. This is inconvenient if
several child files are to be compiled and
to be kept for distribution.
\end{itemize}

The present package provides a simple interface
to make child files individually compilable by \LaTeX{}.
Compiling a child file then has the same effect as compiling
the main file with an |\includeonly| command
to select the appropriate child.
Moreover the generated document will carry the name of the child
rather than the main file.
This resolves all three above issues.

This feature is meant to make the editing of books,
thesis documents and lecture notes somewhat more convenient.
However, the package can also be used efficiently for
composing a series of documents (such as exercise sheets)
which are typically distributed individually.
It then assists the author in generating the individual documents
(potentially in different versions)
as well as a document containing the collected series.
Another application is in developing style files
or other kinds of included material
where compilation of the style file could redirect
to a sample or test file.

%%%%%%%%%%%%%%%%%%%%%%%%%%%%%%%%%%%%%%%%%%%%%%%%%%%%%%%%%%%%%%%%%%%%%%%%%%%%%%%%
%%%%%%%%%%%%%%%%%%%%%%%%%%%%%%%%%%%%%%%%%%%%%%%%%%%%%%%%%%%%%%%%%%%%%%%%%%%%%%%%
\section{Usage}

First of all, the package \textsf{childdoc} is \emph{not} a standard
\LaTeXe{} |.sty| style file! Therefore it needs to be invoked in
a non-standard way.

%%%%%%%%%%%%%%%%%%%%%%%%%%%%%%%%%%%%%%%%%%%%%%%%%%%%%%%%%%%%%%%%%%%%%%%%%%%%%%%%
\subsection{Included Files}
\label{sec:include}

%%%%%%%%%%%%%%%%%%%%%%%%%%%%%%%%%%%%%%%%
\DescribeMacro{\childdocmain}
To use the package, add the commands
\begin{center}
\begin{tabular}{l}
|\input{childdoc.def}|\\
|\childdocmain{}|\\
\end{tabular}
\end{center}
at the very top of the main \LaTeX{} file,
in particular \emph{before} the |\documentclass| statement!
The argument of |\childdocmain| should be left empty
(but it must be present).

%%%%%%%%%%%%%%%%%%%%%%%%%%%%%%%%%%%%%%%%
\DescribeMacro{\childdocof}
Furthermore, add the commands
\begin{center}
\begin{tabular}{l}
|\input{childdoc.def}|\\
|\childdocof{|\textit{main}|}|\\
\end{tabular}
\end{center}
at the top of every child file \textit{child}
which is included by |\include{|\textit{child}|}|
from within the main file
(or at least for those files to be compiled individually).
The argument \textit{main} must be the filename of the main file.

There are a couple of
considerations in setting up the main and child documents:

%%%%%%%%%%%%%%%%%%%%%%%%%%%%%%%%%%%%%%%%
\paragraph{Restrictions.}

Please note the following restrictions:
\begin{itemize}
\item
|\childdocmain| must be called with one argument \textit{main}
to ensure compatibility with earlier version of the package.
It must either be empty (|\childdocmain{}|)
or precisely match the filename of the main file in which it is specified.
See \secref{sec:detection} for further information.
\item
The filename \textit{main} must be specified without the |.tex| extension.
\item
The filename \textit{main} is case sensitive
(even in case-insensitive file systems)
due to internal string comparison.
\item
The argument \textit{main} should be fully expanded, it cannot be a macro.
\item
Subdirectories and special characters should be avoided in filenames.
\item
The command |\childdocmain{|\textit{main}|}| must be followed by a whitespace.
It should not be followed immediately by another command
or by a comment mark `|%|'.
This is because the \TeX{} parser reads the token immediately following
the argument of |\childdocmain| and puts it
at the beginning of every child section;
however, a white\-space is ignored.
\end{itemize}

%%%%%%%%%%%%%%%%%%%%%%%%%%%%%%%%%%%%%%%%
\paragraph{Content of Main File.}

It is advisable to place all content in the child files included by |\include|.
Any output contained in the main file will appear in all child documents
unless suppressed manually;
it cannot be suppressed automatically by the |\includeonly| directive
and thus should normally be avoided.
A method to include some content in the main file
by means of conditional processing is described in \secref{sec:conditional}.

%%%%%%%%%%%%%%%%%%%%%%%%%%%%%%%%%%%%%%%%
\paragraph{Page Numbering.}

When only a part of the document is compiled,
the appropriate numbering of pages
(as well as other status parameters)
is determined from the |.aux| files.
The latter contain information from previous passes.
However this information needs to propagate through
all intermediate child documents.
Therefore the page numbering in child documents may well
be inconsistent until the complete document is compiled at least once.

A useful (if unconventional) way to always ensure a consistent
page numbering is to restart the numbering in each child document
and denote the pages by `\textit{child}|.|\textit{page}'
where \textit{child} represents the chapter/section number of the child file.
This can be achieved by the command
|\numberwithin{page}{|\textit{child}|}|
of the \textsf{amsmath} package
where \textit{child} can be |chapter| or |section|
depending on the chosen structuring.
Alternatively, one can modify the macro |\thepage| appropriately
and reset the counter |page| at the start of each child file.

%%%%%%%%%%%%%%%%%%%%%%%%%%%%%%%%%%%%%%%%%%%%%%%%%%%%%%%%%%%%%%%%%%%%%%%%%%%%%%%%
\subsection{Conditional Processing}
\label{sec:conditional}

The package provides a mechanism to compile different versions
of a document. To customise the versions further some conditional processing
can come in handy to distinguish which version is being compiled.
The package provides two macros to describe the compilation context:

%%%%%%%%%%%%%%%%%%%%%%%%%%%%%%%%%%%%%%%%
\DescribeMacro{\ifchilddoc}
The conditional |\ifchilddoc| distinguishes between the compilation of
child documents and the main document:
%
\begin{center}
|\ifchilddoc |\textit{child-code}| |[|\||else |\textit{main-code}]| \||fi|
\end{center}

%%%%%%%%%%%%%%%%%%%%%%%%%%%%%%%%%%%%%%%%
\DescribeMacro{\childdocname}
\DescribeMacro{\childdocjob}
The macro |\childdocname| contains the filename (without extension)
of the main or child file being processed.
Note that |\childdocjob| will always contain the name of the main file.

%%%%%%%%%%%%%%%%%%%%%%%%%%%%%%%%%%%%%%%%
\paragraph{Title Page.}

Conditional processing can be used to include a title or banner page
in the main document when proper precautions are taken.
Importantly, the code in the main file should ensure that the page counter
(as well as other status parameters which are stored in the |.aux| files)
takes the same value after the conditional processing.
Otherwise the page numbers may take divergent values
depending on which part is compiled.

For example, a title page could be declared by:
%
\begin{center}
\begin{tabular}{l}
|\ifchilddoc\||else|\\
|\addtocounter{page}{-1}|\\
\textit{code for title page}\\
|\newpage|\\
|\||fi|
\end{tabular}
\end{center}
%
A banner page for the child documents can be generated by:
%
\begin{center}
\begin{tabular}{l}
|\ifchilddoc|\\
|\addtocounter{page}{-1}|\\
\textit{code for banner page}\\
|\newpage|\\
|\||fi|
\end{tabular}
\end{center}
%
Here one could write a message such as:
\begin{center}
|This is the part \childdocname{} of \childdocjob{}.|
\end{center}

%%%%%%%%%%%%%%%%%%%%%%%%%%%%%%%%%%%%%%%%%%%%%%%%%%%%%%%%%%%%%%%%%%%%%%%%%%%%%%%%
\subsection{Flags}
\label{sec:flags}

The package makes it easy to generate different versions
of the main or child documents.
To this end compilation flags can be defined
and assigned different default values.
They will be particularly useful in conjunction
with the forwarding mechanism described in \secref{sec:forward}.

For example, it may be useful to have a flag |\version|
which can be set to |draft| or |final|.
The document source will contain some conditional code
depending on the value of |\version|.
Suppose further, the flag should default to |final| for the main file
and to |draft| for child files
which is a natural assignment for editing the document.
This is achieved by placing the following code
in the preamble of the main document
(below the |\childdocmain| directive):
%
\begin{center}
\begin{tabular}{l}
|\ifchilddoc|\\
|\providecommand{\version}{draft}|\\
|\||else|\\
|\providecommand{\version}{final}|\\
|\||fi|
\end{tabular}
\end{center}
%
The definition by |\providecommand| makes sure
that previous definitions are not overwritten.
Further statements |\providecommand{\version}{...}|
can thus be added before the above code to override it.

For the main file, one might add a line
(between |\childdocmain| and the above block)
%
\begin{center}
|%\ifchilddoc\||else\providecommand{\version}{draft}\||fi|
\end{center}
%
which can be uncommented to produce a draft version.
Likewise one can add a line to the very top of a child file
(above the |\childdocof{|\textit{main}|}| directive)
%
\begin{center}
|%\providecommand{\version}{final}|
\end{center}
%
which can be uncommented to produce the final version of this child document.

%%%%%%%%%%%%%%%%%%%%%%%%%%%%%%%%%%%%%%%%%%%%%%%%%%%%%%%%%%%%%%%%%%%%%%%%%%%%%%%%
\subsection{Forwarding}
\label{sec:forward}

Different versions of the main or child documents
using compilation flags as described in \secref{sec:flags}
can be (permanently) stored in different files
for convenient compilation, viewing and distribution.
To this end, the package defines a command
to pass on compilation to a different file:

%%%%%%%%%%%%%%%%%%%%%%%%%%%%%%%%%%%%%%%%
\DescribeMacro{\childdocforward}
The command |\childdocforward| redirects processing to
another source file:
%
\begin{center}
\begin{tabular}{l}
|\input{childdoc.def}|\\
|\childdocforward[|\textit{main}|]{|\textit{dest}|}|\\
\end{tabular}
\end{center}
%
The argument \textit{dest} is the destination file
(without extension).
It should be the main file or one of the child files.
Note that further \textsf{childdoc} directives
such as |\childdocof| and |\childdocforward|
in the indicated file will be processed in this form.
The optional argument \textit{main}
passes on directly to the main file \textit{main}
while pretending to compile the child \textit{dest}.
This form behaves as if \textit{dest}
issues |\childdocof{|\textit{main}|}| right away,
and no further \textsf{childdoc} directives will be processed.

%%%%%%%%%%%%%%%%%%%%%%%%%%%%%%%%%%%%%%%%
\DescribeMacro{\...prefix}
In the alternative form |\childdocforwardprefix|,
%
\begin{center}
\begin{tabular}{l}
|\input{childdoc.def}|\\
|\childdocforwardprefix[|\textit{main}|]{|\textit{prefix}|}{|\textit{dest}|}|
\end{tabular}
\end{center}
%
the destination file is determined by a pattern
depending on the current file:
To make this work, the current file must be called
`{\textit{prefix}\hspace{0.2em}\textit{suffix}}'
with \textit{prefix} matching precisely the argument.
Processing is then passed on to the file
`{\textit{dest}\hspace{0.2em}\textit{suffix}}'.
Surely, the same effect is achieved by
directly specifying the
argument `{\textit{dest}\hspace{0.2em}\textit{suffix}}'
in the first form.
However, that requires to set up a different file
for each child. With the alternative form of the command
all these files can have exactly the same content
which simplifies setting them up and maintaining them.

For example, the following file |draft.tex|
with a compilation flag |\version| as described in \secref{sec:flags}
compiles the main document as a draft:
%
\begin{center}
\begin{tabular}{l}
|\def\version{draft}|\\
|\input{childdoc.def}|\\
|\childdocforward{|\textit{main}|}|
\end{tabular}
\end{center}
%
Likewise, the following files |final|\textit{nn}|.tex|
compile the final version of the child document
|child|\textit{nn}|.tex|:
%
\begin{center}
\begin{tabular}{l}
|\def\version{final}|\\
|\input{childdoc.def}|\\
|\childdocforwardprefix{final}{child}|
\end{tabular}
\end{center}
%

Note that when several versions of a main file and/or of each child file
are to be generated, it may be convenient to set up a |Makefile| or
shell script to automatise the process.

%%%%%%%%%%%%%%%%%%%%%%%%%%%%%%%%%%%%%%%%%%%%%%%%%%%%%%%%%%%%%%%%%%%%%%%%%%%%%%%%
\subsection{Command Line Processing}
\label{sec:commandline}

The effect of redirection files can also be achieved by invoking
the \LaTeX{} compiler with a more elaborate command line.
Most conveniently this should be done as part
of a shell script or a |Makefile|.

When using \textsf{childdoc} in the main file, the following
command lines effectively perform a redirection
(note that depending on the shell being used,
backslashes may have to be doubled: `|\|' $\to$ `|\\|'):
%
\begin{center}
|... -jobname "|\textit{target}|" |\\|"|[\textit{flags}]%
|\input{childdoc.def}\childdocforward[|\textit{main}|]{|\textit{dest}|}"|
\end{center}
%
Here \textit{target} is the name of the output file,
\textit{main} is the name of the main file
and \textit{dest} is the name of the main or child file to be processed
(all filenames without extensions).
The optional argument \textit{main} can be omitted
if \textit{main} matches \textit{dest}.
Optionally, compilation \textit{flags} can be defined via |\def| commands.
This command line makes the \TeX{} engine believe
it is compiling the file \textit{target}
whose content is specified as the latter parameter.
The provided code then forwards the processing to
\textit{main} or \textit{dest} as described in \secref{sec:forward}.

%%%%%%%%%%%%%%%%%%%%%%%%%%%%%%%%%%%%%%%%%%%%%%%%%%%%%%%%%%%%%%%%%%%%%%%%%%%%%%%%
\subsection{Include by Input}
\label{sec:input}

Including child documents by |\include| has some restrictions by design.
Most notably, the content of a child document always occupies
its own set of pages; pages cannot be shared between child documents.
Usually, this behaviour makes perfect sense
because each child document contain an essential part of the document.
However, in some situations it may be desirable to compose
a document from a collection of parts
without having mandatory page breaks between then.
For this case, the package
provides a mechanism to include parts
by |\input| which can also be processed individually.
However, by construction this mechanism
requires manual handling of the content to be output.

%%%%%%%%%%%%%%%%%%%%%%%%%%%%%%%%%%%%%%%%
\DescribeMacro{\ifchilddocmanual}
The main file should be prepared as usual, see \secref{sec:include}.
However, the document body must make a distinction
between processing of an individual part and of the main document, e.g.:
%
\begin{center}
\begin{tabular}{l}
|\ifchilddocmanual|\\
|\input{\childdocname}|\\
|\||else|\\
\textit{document body with }|\input{|\textit{part}|}|\\
|\||fi|
\end{tabular}
\end{center}
%
The conditional |\ifchilddocmanual| is true whenever
a part to be included by |\input| is being compiled,
and the name of the part is stored in |\childdocname|.

%%%%%%%%%%%%%%%%%%%%%%%%%%%%%%%%%%%%%%%%
\DescribeMacro{\childdocby}
Each part to be included by |\input| should start with:
%
\begin{center}
\begin{tabular}{l}
|\input{childdoc.def}|\\
|\childdocby{|\textit{main}|}|\\
\end{tabular}
\end{center}
%
The directive |\childdocby| is similar to |\childdocof|
described in \secref{sec:include},
but the subsequent selection of content must be done manually.
To that end, both |\ifchilddoc| and |\ifchilddocmanual|
will be true upon processing of a part,
and the name of the part is stored in |\childdocname|.
Note that |\jobname| will be set to the filename of the current part
so that each part receives an individual |.aux| file
that does not interfere with the |.aux| file(s) of the main document.
This behaviour can be altered by the alternative form
|\childdocby[*]{|\textit{main}|}| (with a non-empty optional argument)
which uses the |.aux| file of the main document
by setting |\jobname| to \textit{main}.

%%%%%%%%%%%%%%%%%%%%%%%%%%%%%%%%%%%%%%%%%%%%%%%%%%%%%%%%%%%%%%%%%%%%%%%%%%%%%%%%
\subsection{Driver Development}
\label{sec:driver}

The \textsf{childdoc} mechanism can also be use for the development
of definition files such as \LaTeX{} styles or classes.
This case differs from the above setup with multiple parts
included by |\include| in that no |\includeonly| should be invoked.
This can be achieved by starting the include file
(before |\ProvidesPackage|) with:
%
\begin{center}
\begin{tabular}{l}
|\input{childdoc.def}|\\
|\childdocforward{|\textit{main}|}|\\
\end{tabular}
\end{center}
%
or alternatively with:
%
\begin{center}
\begin{tabular}{l}
|\input{childdoc.def}|\\
|\childdocby{|\textit{main}|}|\\
\end{tabular}
\end{center}
%
Both forms have slightly different effects as described above.
The main file is prepared as usual, see \secref{sec:include}.

%%%%%%%%%%%%%%%%%%%%%%%%%%%%%%%%%%%%%%%%%%%%%%%%%%%%%%%%%%%%%%%%%%%%%%%%%%%%%%%%
\subsection{Legacy Detection}
\label{sec:detection}

The directive |\childdocmain| in the main file can detect
whether the complete document or merely a child is to be compiled
even without using the directive |\childdocof|.
This method is deprecated because it is less robust
and there is no compelling reason to use it;
it is merely provided for backward compatibility
and it may be removed in future versions.

If the detection mechanism is to be used,
it is mandatory to correctly specify
the filename of the main file as the argument of |\childdocmain|:
%
\begin{center}
\begin{tabular}{l}
|\input{childdoc.def}|\\
|\childdocmain{|\textit{main}|}|\\
\end{tabular}
\end{center}
%
If |\jobname| does not match the argument \textit{main} of |\childdocmain|,
it is assumed that |\jobname| points to the child file to be compiled.
When using |\childdocmain| with the main file specified as argument,
it suffices to start a child file
with just |\input{|\textit{main}|}|
without loading of the package and using |\childdocof|.
If instead all processing is done
with the appropriate \textsf{childdoc} directives,
the argument of \textit{main} of |\childdocmain| can be empty.

An alternative version of the command line processing described
in \secref{sec:commandline} using the detection mechanism reads:
%
\begin{center}
|... -jobname "|\textit{target}|" "|[\textit{flags}]%
[|\def\jobname{|\textit{dest}|}|]|\input{|\textit{main}|}"|
\end{center}

%%%%%%%%%%%%%%%%%%%%%%%%%%%%%%%%%%%%%%%%%%%%%%%%%%%%%%%%%%%%%%%%%%%%%%%%%%%%%%%%
\subsection{Manual Code}
\label{sec:manual}

In case one cannot be certain whether the definitions file |childdoc.def|
is installed on the target \TeX{} distribution
and one prefers not to ship it,
it is conceivable to paste a few relevant commands into the sources.

To that end, drop all statements |\input{childdoc.def}|
and perform the replacements as outlined below.
Instead of |\childdocmain{|\textit{main}|}| add the following code
to the top of the main file:
%
\begin{center}
\begin{tabular}{l}
|\||ifdefined\childdocname\endinput\||fi\newif\ifchilddoc|\\
|\edef\childdocname{\scantokens\expandafter{\jobname\noexpand}}|\\
|\def\childdocmain{|\textit{main}|}\||ifx\childdocmain\childdocname\||else|\\
|\childdoctrue\includeonly{\childdocname}\let\jobname\childdocmain\||fi|\\
\end{tabular}
\end{center}
%
Instead of |\childdocof{|\textit{main}|}| just include the main file
at the top of each child file:
%
\begin{center}
|\input{|\textit{main}|}|
\end{center}
%
A simple redirection |\childdocforward{|\textit{dest}|}| is achieved by:
%
\begin{center}
|\def\jobname{|\textit{dest}|}\input{\jobname}|
\end{center}
%
The redirection with prefix
|\childdocforwardprefix[|\textit{prefix}|]{|\textit{dest}|}|
is accomplished by:
%
\begin{center}
\begin{tabular}{l}
|{\edef\jobname{\scantokens\expandafter{\jobname\noexpand}}|\\
|\def\redirectjob |\textit{prefix}|#1~~~{\gdef\jobname{|\textit{dest}|#1}}|\\
|\expandafter\redirectjob\jobname~~~}\input{\jobname}|
\end{tabular}
\end{center}

In an alternative approach,
child documents can be compiled by a specific command line
without additional code or specific definitions:
%
\begin{center}
|... -jobname "|\textit{target}|" "|[\textit{flags}]%
|\includeonly{|\textit{dest}|}\input{|\textit{main}|}"|
\end{center}
%

%%%%%%%%%%%%%%%%%%%%%%%%%%%%%%%%%%%%%%%%%%%%%%%%%%%%%%%%%%%%%%%%%%%%%%%%%%%%%%%%
%%%%%%%%%%%%%%%%%%%%%%%%%%%%%%%%%%%%%%%%%%%%%%%%%%%%%%%%%%%%%%%%%%%%%%%%%%%%%%%%
\section{Information}

%%%%%%%%%%%%%%%%%%%%%%%%%%%%%%%%%%%%%%%%%%%%%%%%%%%%%%%%%%%%%%%%%%%%%%%%%%%%%%%%
\subsection{Copyright}

Copyright \copyright{} 2017--2018 Niklas Beisert

This work may be distributed and/or modified under the
conditions of the \LaTeX{} Project Public License, either version 1.3
of this license or (at your option) any later version.
The latest version of this license is in
  \url{http://www.latex-project.org/lppl.txt}
and version 1.3 or later is part of all distributions of \LaTeX{}
version 2005/12/01 or later.

This work has the LPPL maintenance status `maintained'.

The Current Maintainer of this work is Niklas Beisert.

This work consists of the files |README.txt|, |childdoc.ins| and |childdoc.dtx|
as well as the derived files |childdoc.def|, |cdocsamp.tex|
with |cdocsch1.tex|, |cdocsch2.tex|, |cdocspt3.tex|, |cdocspt4.tex|,
|cdocsdrf.tex|, |cdocsfn1.tex|, |cdocsfn2.tex|
as well as |childdoc.pdf|.

%%%%%%%%%%%%%%%%%%%%%%%%%%%%%%%%%%%%%%%%%%%%%%%%%%%%%%%%%%%%%%%%%%%%%%%%%%%%%%%%
\subsection{Files and Installation}

The package consists of the files:
%
\begin{center}
\begin{tabular}{ll}
    |README.txt|   & readme file \\
    |childdoc.ins| & installation file \\
    |childdoc.dtx| & source file \\
    |childdoc.def| & definition file \\
    |cdocsamp.tex| & sample main file \\
    |cdocsch1.tex| & sample include file \\
    |cdocsch2.tex| & sample include file \\
    |cdocspt3.tex| & sample part file \\
    |cdocspt4.tex| & sample part file \\
    |cdocsdrf.tex| & sample redirection file \\
    |cdocsfn1.tex| & sample redirection file \\
    |cdocsfn2.tex| & sample redirection file \\
    |childdoc.pdf| & manual
\end{tabular}
\end{center}
%
The distribution consists of the files
|README.txt|, |childdoc.ins| and |childdoc.dtx|.
%
\begin{itemize}
\item
Run (pdf)\LaTeX{} on |childdoc.dtx|
to compile the manual |childdoc.pdf| (this file).
\item
Run \LaTeX{} on |childdoc.ins| to create the definitions file |childdoc.def|
and the sample |cdocsamp.tex| with include files
|cdocsch1.tex|, |cdocsch2.tex|, |cdocspt3.tex|, |cdocspt4.tex|,
|cdocsdrf.tex|, |cdocsfn1.tex|, |cdocsfn2.tex|.
Then copy the file |childdoc.def| to an appropriate directory of your \LaTeX{}
distribution, e.g.\ \textit{texmf-root}|/tex/latex/childdoc|.
\end{itemize}

%%%%%%%%%%%%%%%%%%%%%%%%%%%%%%%%%%%%%%%%%%%%%%%%%%%%%%%%%%%%%%%%%%%%%%%%%%%%%%%%
\subsection{Related CTAN Packages}

There are several other packages which offer a similar functionality:
%
\begin{itemize}
\item
The packages
\href{http://ctan.org/pkg/docmute}{\textsf{docmute}},
\href{http://ctan.org/pkg/includex}{\textsf{includex}} and
\href{http://ctan.org/pkg/standalone}{\textsf{standalone}}
provide commands to include only the document body of
a child file thus allowing both files to be compiled individually.
\item
The packages \href{http://ctan.org/pkg/subdocs}{\textsf{subdocs}}
and \href{http://ctan.org/pkg/subfiles}{\textsf{subfiles}}
provide structures in which the main and child documents can be
encapsulated and allowing them to be compiled individually.
The inclusion mechanism is different from the conventional |\include|.
\item
The package \href{http://ctan.org/pkg/combine}{\textsf{combine}}
is an elaborate solution to combine several documents into one.
\end{itemize}
%
See also the CTAN topic \href{http://ctan.org/topic/subdocs}{\textsf{subdocs}}
for further related packages.
The present package differs from the above solutions in that
a document structure constructed with the conventional |\include| mechanism
just needs two extra commands at the top of every file
such that all constituent files can be compiled individually.

%%%%%%%%%%%%%%%%%%%%%%%%%%%%%%%%%%%%%%%%%%%%%%%%%%%%%%%%%%%%%%%%%%%%%%%%%%%%%%%%
%\subsection{Feature Suggestions}
%
%The following is a list of features which may be useful for future
%versions of this package:
%%
%\begin{itemize}
%\item
%\ldots
%\end{itemize}

%%%%%%%%%%%%%%%%%%%%%%%%%%%%%%%%%%%%%%%%%%%%%%%%%%%%%%%%%%%%%%%%%%%%%%%%%%%%%%%%
\subsection{Revision History}

%%%%%%%%%%%%%%%%%%%%%%%%%%%%%%%%%%%%%%%%
\paragraph{v2.0:} 2018/12/30

\begin{itemize}
\item
immediate forward processing
\item
added |\childdocby| mechanism
\item
manual restructured
\end{itemize}

%%%%%%%%%%%%%%%%%%%%%%%%%%%%%%%%%%%%%%%%
\paragraph{v1.6:} 2018/01/17

\begin{itemize}
\item
application for development of include files
\item
corrections to manual
\end{itemize}

%%%%%%%%%%%%%%%%%%%%%%%%%%%%%%%%%%%%%%%%
\paragraph{v1.5:} 2017/05/21

\begin{itemize}
\item
more complete structuring introduced
\item
|\childdocof| introduced
\item
|\childdoc| renamed to |\childdocmain|
\item
|\childredirect| renamed to |\childdocforward| and |\childdocforwardprefix|
and functionality expanded
\end{itemize}

%%%%%%%%%%%%%%%%%%%%%%%%%%%%%%%%%%%%%%%%
\paragraph{v1.0:} 2017/04/27

\begin{itemize}
\item
manual and install package
\item
first version published on CTAN
\end{itemize}

%%%%%%%%%%%%%%%%%%%%%%%%%%%%%%%%%%%%%%%%
\paragraph{v0.6:} 2017/04/26

\begin{itemize}
\item
redirection mechanism added
\end{itemize}

%%%%%%%%%%%%%%%%%%%%%%%%%%%%%%%%%%%%%%%%
\paragraph{v0.5:} 2017/04/26

\begin{itemize}
\item
functionality in definition file
\end{itemize}


%%%%%%%%%%%%%%%%%%%%%%%%%%%%%%%%%%%%%%%%%%%%%%%%%%%%%%%%%%%%%%%%%%%%%%%%%%%%%%%%
%%%%%%%%%%%%%%%%%%%%%%%%%%%%%%%%%%%%%%%%%%%%%%%%%%%%%%%%%%%%%%%%%%%%%%%%%%%%%%%%
%%%%%%%%%%%%%%%%%%%%%%%%%%%%%%%%%%%%%%%%%%%%%%%%%%%%%%%%%%%%%%%%%%%%%%%%%%%%%%%%
\appendix

\settowidth\MacroIndent{\rmfamily\scriptsize 000\ }

 \DocInput{childdoc.dtx}

\end{document}
%</driver>
% \fi
%
% %%%%%%%%%%%%%%%%%%%%%%%%%%%%%%%%%%%%%%%%%%%%%%%%%%%%%%%%%%%%%%%%%%%%%%%%%%%%%%
% %%%%%%%%%%%%%%%%%%%%%%%%%%%%%%%%%%%%%%%%%%%%%%%%%%%%%%%%%%%%%%%%%%%%%%%%%%%%%%
% \section{Sample}
%\iffalse
%<*samplemain>
%\fi
%
% The following presents a sample document
% with two chapters, two parts, a title page,
% a compile flag as well as three forwarding files to set the flag.
% It consists of eight |.tex| files:
% \begin{center}
% \begin{tabular}{ll}
% |cdocsamp.tex|&main file\\
% |cdocsch1.tex|&include file for chapter 1\\
% |cdocsch2.tex|&include file for chapter 2\\
% |cdocspt3.tex|&include file for part 3\\
% |cdocspt4.tex|&include file for part 4\\
% |cdocsdrf.tex|&forwarding file for main file in draft mode\\
% |cdocsfi1.tex|&forwarding file for final version of chapter 1\\
% |cdocsfi2.tex|&forwarding file for final version of chapter 2\\
% \end{tabular}
% \end{center}
% Each of the eight files can be compiled directly by the \LaTeX{} compiler.
%
% %%%%%%%%%%%%%%%%%%%%%%%%%%%%%%%%%%%%%%
% \paragraph{Main File.}
%
% The main file is called |cdocsamp.tex|.
%
% Load the \textsf{childdoc} definitions and
% declare the filename for the main document:
%    \begin{macrocode}
\input{childdoc.def}
\childdocmain{}
%    \end{macrocode}

% Optional override for |\version| flag:
%    \begin{macrocode}
%%\ifchilddoc\else\providecommand{\version}{draft}\fi
%    \end{macrocode}

% Define the default values for the |\version| flag
% (|final| for the main file and |draft| for childs):
%    \begin{macrocode}
\ifchilddoc
\providecommand{\version}{draft}
\else
\providecommand{\version}{final}
\fi
%    \end{macrocode}

% Load the standard document class:
%    \begin{macrocode}
\documentclass[12pt]{article}
%    \end{macrocode}

% Start the document body:
%    \begin{macrocode}
\begin{document}
%    \end{macrocode}

% Declare a title page.
% Print title, part of document being processed and version flag:
%    \begin{macrocode}
\addtocounter{page}{-1}
\begin{center}
{\LARGE\bfseries{}childdoc example\par}
\vspace{1cm}
\ifchilddoc
\ifchilddocmanual part\else chapter\fi:
`\childdocname' of `\childdocjob'\par
\else
main document: `\childdocjob'\par
\fi
version: \version\par
\end{center}
\newpage
%    \end{macrocode}

% Manually include selected file,
% otherwise process as usual:
%    \begin{macrocode}
\ifchilddocmanual
\section*{part `\childdocname'}
\input{\childdocname}
\else
%    \end{macrocode}

% Include the two chapters:
%    \begin{macrocode}
\include{cdocsch1}
\include{cdocsch2}
%    \end{macrocode}

% Include the two parts unless only chapters should be displayed:
%    \begin{macrocode}
\ifchilddoc\else
\section{part three}
\input{cdocspt3}
\section{part four}
\input{cdocspt4}
\fi
%    \end{macrocode}

% Process as usual until here:
%    \begin{macrocode}
\fi
%    \end{macrocode}

% End of document body:
%    \begin{macrocode}
\end{document}
%    \end{macrocode}
%\iffalse
%</samplemain>
%\fi
%
% %%%%%%%%%%%%%%%%%%%%%%%%%%%%%%%%%%%%%%
% \paragraph{Chapter Include Files.}
%
% The include files are called |cdocsch1.tex| and |cdocsch2.tex|.
%
%\iffalse
%<*samplechap1|samplechap2>
%\fi

% Optional override for |\version| flag:
%    \begin{macrocode}
%%\providecommand{\version}{final}
%    \end{macrocode}

% Include the main document:
%    \begin{macrocode}
\input{childdoc.def}
\childdocof{cdocsamp}
%    \end{macrocode}

%\iffalse
%</samplechap1|samplechap2>
%\fi
%
%\iffalse
%<*samplechap1>
%\fi
% Some text for chapter 1:
%    \begin{macrocode}
\section{one}
some text in chapter one
%    \end{macrocode}

%\iffalse
%</samplechap1>
%\fi
% Some text for chapter 2:
%\iffalse
%<*samplechap2>
%\fi
%    \begin{macrocode}
\section{two}
more text in chapter two
%    \end{macrocode}

%\iffalse
%</samplechap2>
%\fi
%
% %%%%%%%%%%%%%%%%%%%%%%%%%%%%%%%%%%%%%%
% \paragraph{Part Include Files.}
%
% The include files are called |cdocspt3.tex| and |cdocspt4.tex|.
%
%\iffalse
%<*samplepart3|samplepart4>
%\fi

% Optional override for |\version| flag:
%    \begin{macrocode}
%%\providecommand{\version}{final}
%    \end{macrocode}

% Include the main document:
%    \begin{macrocode}
\input{childdoc.def}
\childdocby{cdocsamp}
%    \end{macrocode}

%\iffalse
%</samplepart3|samplepart4>
%\fi
%
%\iffalse
%<*samplepart3>
%\fi
% Some text for part 3:
%    \begin{macrocode}
some text in part three
%    \end{macrocode}

%\iffalse
%</samplepart3>
%\fi
% Some text for part 4:
%\iffalse
%<*samplepart4>
%\fi
%    \begin{macrocode}
more text in part four
%    \end{macrocode}

%\iffalse
%</samplepart4>
%\fi
%
% %%%%%%%%%%%%%%%%%%%%%%%%%%%%%%%%%%%%%%
% \paragraph{Forwarding for a Complete Draft.}
%
% The following forwarding file |cdocsdrf.tex|
% compiles the main document in draft mode:
%\iffalse
%<*sampledraft>
%\fi
%    \begin{macrocode}
\def\version{draft}
\input{childdoc.def}
\childdocforward{cdocsamp}
%    \end{macrocode}

%\iffalse
%</sampledraft>
%\fi
%
% %%%%%%%%%%%%%%%%%%%%%%%%%%%%%%%%%%%%%%
% \paragraph{Forwarding for Final Version of the Chapters.}
%
% The following forwarding files |cdocsfn1.tex| and |cdocsfn2.tex|
% (with identical content)
% compile the final versions of the child documents
% |cdocsch1.tex| and |cdocsch2.tex|, respectively:
%\iffalse
%<*samplefinal>
%\fi
%    \begin{macrocode}
\def\version{final}
\input{childdoc.def}
\childdocforwardprefix[cdocsamp]{cdocsfn}{cdocsch}
%    \end{macrocode}

%\iffalse
%</samplefinal>
%\fi
%
% %%%%%%%%%%%%%%%%%%%%%%%%%%%%%%%%%%%%%%
% \paragraph{Command Line Processing.}
%
% The following three command lines generate the output files
% |cdocscld|, |cdocscl1| and |cdocscl2|
% which should be identical to
% |cdocsdrf|, |cdocsch1| and |cdocsfn2|, respectively:
% \begin{center}
% \begin{tabular}{l}
% |latex -jobname cdocscld \|\\
% |  "\def\version{draft}\input{childdoc.def}\childdocforward{cdocsamp}"|\\
% |latex -jobname cdocscl1 \|\\
% |  "\input{childdoc.def}\childdocforward[cdocsamp]{cdocsch1}"|\\
% |latex -jobname cdocscl2 \|\\
% |  "\def\version{final}\input{childdoc.def}\childdocforward{cdocsch2}"|
% \end{tabular}
% \end{center}
% Note that the trailing backslash on each first line
% merely continues the input to the second line
% (for convenient cut ant paste).
% Furthermore, the command |latex| can be replaced by any
% of its alternative versions such as |pdflatex|.
%
% %%%%%%%%%%%%%%%%%%%%%%%%%%%%%%%%%%%%%%%%%%%%%%%%%%%%%%%%%%%%%%%%%%%%%%%%%%%%%%
% %%%%%%%%%%%%%%%%%%%%%%%%%%%%%%%%%%%%%%%%%%%%%%%%%%%%%%%%%%%%%%%%%%%%%%%%%%%%%%
% \section{Implementation}
%\iffalse
%<*package>
%\fi
%
% This section describes the definitions file |childdoc.def|.

% The definitions cannot be loaded using |\usepackage| or |\RequirePackage|
% which has a mechanism to prevent loading a style file more than once.
% When loading the definitions by means of |\input|
% multiple instances have to be prevented manually:
%\iffalse
%This code needs to be before the `\ProvidesFile' directive
%which is defined at the beginning of this file.
%Therefore it is also placed there and commented out here.
%</package>
%<*discard>
%\fi
%    \begin{macrocode}
\ifdefined\childdocmain\endinput\fi
%    \end{macrocode}
%\iffalse
%</discard>
%<*package>
%\fi
%
% \macro{\ifchilddoc}
% \macro{\ifchilddocmanual}
% The conditional |\ifchilddoc| tells whether a
% child (true) or main (false) document is being compiled.
% The conditional |\ifchilddocmanual| tells whether
% the |\includeonly| mechanism is used (false) or
% the selection of child files must be performed manually (true).
% The definitions initialise to false:
%    \begin{macrocode}
\newif\ifchilddoc
\newif\ifchilddocmanual
%    \end{macrocode}

% \macro{\childdocname}
% \macro{\childdocjob}
% The macro |\childdocname| stores the name of the main document
% to be compiled. The macro |\childdocjob| stores the name of
% the document on which the \LaTeX{} compiler was originally invoked.
% The content of |\jobname| cannot be compared
% to filenames specified in the source due to different catcodes.
% The following code rescans |\jobname|, stores the result
% in |\childdocname| and saves a copy in |\childdocjob|:
%    \begin{macrocode}
\edef\childdocname{\scantokens\expandafter{\jobname\noexpand}}
\let\childdocjob\childdocname
%    \end{macrocode}

% \macro{\childdocdisable}
% The macro |\childdocdisable| prevents the main file
% from being processed more than once.
% At this stage, the main document command |\childdocmain|
% is assumed to be called once again where it should do nothing.
% Any subsequent call to it should prevent
% a secondary processing of the main document
% It overwrites the forwarding commands
% |\childdocof| and |\childdocforward|
% with empty macros to prevent further inclusions of the main document:
%    \begin{macrocode}
\newcommand{\childdocdisable}
{
  \renewcommand{\childdocmain}[1]{\renewcommand{\childdocmain}[1]{\endinput}}
  \renewcommand{\childdocof}[1]{}
  \renewcommand{\childdocby}[2][]{}
  \renewcommand{\childdocforward}[2][]{}
  \renewcommand{\childdocdisable}{}
}
%    \end{macrocode}

% \macro{\childdocmain}
% The macro |\childdocmain| is to be called at the top of the main file
% with nothing or the main filename (without extension) as argument.
% First, it breaks loops.
% If the argument is not empty and does not match |\childdocname|
% (which is set by the first inclusion of |childdoc.def|),
% |\ifchilddoc| is set to true, |\includeonly| is applied to the child file
% and |\jobname| is set to the main file
% (for proper handling of |.aux| files):
%    \begin{macrocode}
\newcommand{\childdocmain}[1]
{
  \childdocdisable\childdocmain{}
  \if?#1?\else
    \begingroup
      \def\childdoctmp{#1}
      \ifx\childdoctmp\childdocname
        \def\childdoctmp{}
      \else
        \def\childdoctmp
        {
          \childdoctrue
          \includeonly{\childdocname}
          \def\childdocjob{#1}
          \def\jobname{#1}
        }
      \fi
      \expandafter
    \endgroup
    \childdoctmp
  \fi
}
%    \end{macrocode}

% \macro{\childdocof}
% The command |\childdocof| redirects
% compilation to the main file |#1|.
%    \begin{macrocode}
\newcommand{\childdocof}[1]
{
  \childdocdisable
  \childdoctrue
  \includeonly{\childdocname}
  \def\jobname{#1}
  \def\childdocjob{#1}
  \input{#1}
}
%    \end{macrocode}

% \macro{\childdocby}
% The command |\childdocby| ....
%    \begin{macrocode}
\newcommand{\childdocby}[2][]
{
  \childdocdisable
  \childdoctrue
  \childdocmanualtrue
  \if?#1?\else
    \def\jobname{#2}
  \fi
  \def\childdocjob{#2}
  \input{#2}
  \endinput
}
%    \end{macrocode}

% \macro{\childdocforward}
% The command |\childdocforward| redirects
% compilation to the main file or
% (if the optional argument is given) a child file.
% Parameters are set as if the main file
% or a child file starting with |\childdocof| was compiled.
% Then compilation is handed over to the main file:
%    \begin{macrocode}
\newcommand{\childdocforward}[2][]
{
  \begingroup
    \if?#1?
      \def\childdoctmp
      {
        \def\childdocname{#2}
        \def\childdocjob{#2}
        \def\jobname{#2}
        \input{#2}
        \endinput
      }
    \else
      \def\childdoctmp
      {
        \childdocdisable
        \def\childdocname{#2}
        \childdoctrue
        \includeonly{#2}
        \def\childdocjob{#1}
        \def\jobname{#1}
        \input{#1}
        \endinput
      }
    \fi
    \expandafter
  \endgroup
  \childdoctmp
}
%    \end{macrocode}

% \macro{\childdocforwardprefix}
% The command |\childdocforwardprefix| redirects
% compilation to the main or a child file by means of a pattern.
% The prefix |#1| in the current filename is replaced by |#2|
% and the suffix of the current filename is kept
% (it is assumed that the filename does not contain the substring `|~~~|'
% which is used as a delimiter).
% Compilation is handed over to the new file by |\childdocforward|:
%    \begin{macrocode}
\newcommand{\childdocforwardprefix}[3][]
{
  \begingroup
    \def\childdocextract #2##1~~~{\def\childdoctmp{\childdocforward[#1]{#3##1}}}
    \expandafter\childdocextract\childdocname~~~
    \expandafter
  \endgroup
  \childdoctmp
}
%    \end{macrocode}

% \macro{\childdoc}
% The deprecated macro |\childdoc| is a legacy version of |\childdocmain|:
%    \begin{macrocode}
\newcommand{\childdoc}{\childdocmain}
%    \end{macrocode}

% \macro{\childdocredirect}
% The deprecated macro |\childdocredirect| is a legacy version
% of |\childdocforward| and |\childdocforwardprefix|:
%    \begin{macrocode}
\newcommand{\childdocredirect}[2][]
{
  \begingroup
    \if?#1?
      \def\childdoctmp{\childdocforward{#2}}
    \else
      \def\childdoctmp{\childdocforwardprefix{#1}{#2}}
    \fi
    \expandafter
  \endgroup
  \childdoctmp
}
%    \end{macrocode}

%\iffalse
%</package>
%\fi
%
\endinput
|\\
|\childdocforward{|\textit{main}|}|
\end{tabular}
\end{center}
%
Likewise, the following files |final|\textit{nn}|.tex|
compile the final version of the child document
|child|\textit{nn}|.tex|:
%
\begin{center}
\begin{tabular}{l}
|\def\version{final}|\\
|% \iffalse
%
% childdoc.dtx Copyright (C) 2017-2018 Niklas Beisert
%
% This work may be distributed and/or modified under the
% conditions of the LaTeX Project Public License, either version 1.3
% of this license or (at your option) any later version.
% The latest version of this license is in
%   http://www.latex-project.org/lppl.txt
% and version 1.3 or later is part of all distributions of LaTeX
% version 2005/12/01 or later.
%
% This work has the LPPL maintenance status `maintained'.
%
% The Current Maintainer of this work is Niklas Beisert.
%
% This work consists of the files childdoc.dtx and childdoc.ins
% and the derived files childdoc.def and cdocsamp.tex with
% cdocsch1.tex, cdocsch2.tex, cdocsdrf.tex, cdocsfn1.tex, cdocsfn2.tex.
%
%<package>\ifdefined\childdocmain\endinput\fi
%<package>\ProvidesFile{childdoc.def}[2018/12/30 v2.0 child document driver]
%<samplemain>\ProvidesFile{cdocsamp.tex}[2018/12/30 v2.0 sample for childdoc]
%<*driver>
%\ProvidesFile{childdoc.drv}[2018/12/30 v2.0 childdoc reference manual file]
\PassOptionsToClass{10pt,a4paper}{article}
\documentclass{ltxdoc}

\usepackage[margin=35mm]{geometry}
\usepackage{hyperref}
\usepackage{hyperxmp}
\usepackage[usenames]{color}

\hypersetup{colorlinks=true}
\hypersetup{pdfstartview=FitH}
\hypersetup{pdfpagemode=UseNone}
\hypersetup{pdfsource={}}
\hypersetup{pdflang={en-UK}}
\hypersetup{pdfcopyright={Copyright 2017-2018 Niklas Beisert.
  This work may be distributed and/or modified under the
  conditions of the LaTeX Project Public License, either version 1.3
  of this license or (at your option) any later version.}}
\hypersetup{pdflicenseurl={http://www.latex-project.org/lppl.txt}}
\hypersetup{pdfcontactaddress={ETH Zurich, ITP, HIT K,
  Wolfgang-Pauli-Strasse 27}}
\hypersetup{pdfcontactpostcode={8093}}
\hypersetup{pdfcontactcity={Zurich}}
\hypersetup{pdfcontactcountry={Switzerland}}
\hypersetup{pdfcontactemail={nbeisert@itp.phys.ethz.ch}}
\hypersetup{pdfcontacturl={http://people.phys.ethz.ch/\xmptilde nbeisert/}}

\newcommand{\secref}[1]{\hyperref[#1]{section \ref*{#1}}}

\parskip1ex
\parindent0pt
\let\olditemize\itemize
\def\itemize{\olditemize\parskip0pt}

\begin{document}

\title{The \textsf{childdoc} Package}
\hypersetup{pdftitle={The childdoc Package}}
\author{Niklas Beisert\\[2ex]
  Institut f\"ur Theoretische Physik\\
  Eidgen\"ossische Technische Hochschule Z\"urich\\
  Wolfgang-Pauli-Strasse 27, 8093 Z\"urich, Switzerland\\[1ex]
  \href{mailto:nbeisert@itp.phys.ethz.ch}
  {\texttt{nbeisert@itp.phys.ethz.ch}}}
\hypersetup{pdfauthor={Niklas Beisert}}
\hypersetup{pdfsubject={Manual for the LaTeX2e Package childdoc}}
\date{30 December 2018, \textsf{v2.0}}
\maketitle

\begin{abstract}\noindent
\textsf{childdoc} is a \LaTeXe{} package
that enables the direct compilation
of document sections included by |\include|
to individual files.
\end{abstract}

\begingroup
\parskip0ex
\tableofcontents
\endgroup

%%%%%%%%%%%%%%%%%%%%%%%%%%%%%%%%%%%%%%%%%%%%%%%%%%%%%%%%%%%%%%%%%%%%%%%%%%%%%%%%
%%%%%%%%%%%%%%%%%%%%%%%%%%%%%%%%%%%%%%%%%%%%%%%%%%%%%%%%%%%%%%%%%%%%%%%%%%%%%%%%
\section{Introduction}

\LaTeX{} provides a mechanism to structure a large document (such as a book)
into a main file and several child files (containing the chapters)
using the |\include| command.
This mechanism is beneficial for documents
which span hundreds of pages in order to
make the source file(s) more manageable.
Moreover, compilation can be restricted to
selected child files by means of the |\includeonly| command.
The latter feature can be used to reduce the compilation time while editing
(this was significantly more useful in the earlier days of \LaTeX{})
or to generate a smaller document which is easier to navigate.
Another application of |\includeonly| is to generate
documents consisting of selected parts of the complete document.

However, there are a few drawbacks of the plain |\include| mechanism:
\begin{itemize}
\item
The child files cannot be compiled on their own,
they can only be compiled via the main file.
A naive editing environment
(such as a text editor with an option
to have the current file processed by \LaTeX)
may require one to switch to the main file before compiling;
attempting to compile the child file produces errors.
\item
The main file must be modified (each time)
to adjust the |\includeonly| command
to the present needs. This easily leaves the main file in a messy state.
\item
The generated document will always carry the filename
of the main document. This is inconvenient if
several child files are to be compiled and
to be kept for distribution.
\end{itemize}

The present package provides a simple interface
to make child files individually compilable by \LaTeX{}.
Compiling a child file then has the same effect as compiling
the main file with an |\includeonly| command
to select the appropriate child.
Moreover the generated document will carry the name of the child
rather than the main file.
This resolves all three above issues.

This feature is meant to make the editing of books,
thesis documents and lecture notes somewhat more convenient.
However, the package can also be used efficiently for
composing a series of documents (such as exercise sheets)
which are typically distributed individually.
It then assists the author in generating the individual documents
(potentially in different versions)
as well as a document containing the collected series.
Another application is in developing style files
or other kinds of included material
where compilation of the style file could redirect
to a sample or test file.

%%%%%%%%%%%%%%%%%%%%%%%%%%%%%%%%%%%%%%%%%%%%%%%%%%%%%%%%%%%%%%%%%%%%%%%%%%%%%%%%
%%%%%%%%%%%%%%%%%%%%%%%%%%%%%%%%%%%%%%%%%%%%%%%%%%%%%%%%%%%%%%%%%%%%%%%%%%%%%%%%
\section{Usage}

First of all, the package \textsf{childdoc} is \emph{not} a standard
\LaTeXe{} |.sty| style file! Therefore it needs to be invoked in
a non-standard way.

%%%%%%%%%%%%%%%%%%%%%%%%%%%%%%%%%%%%%%%%%%%%%%%%%%%%%%%%%%%%%%%%%%%%%%%%%%%%%%%%
\subsection{Included Files}
\label{sec:include}

%%%%%%%%%%%%%%%%%%%%%%%%%%%%%%%%%%%%%%%%
\DescribeMacro{\childdocmain}
To use the package, add the commands
\begin{center}
\begin{tabular}{l}
|\input{childdoc.def}|\\
|\childdocmain{}|\\
\end{tabular}
\end{center}
at the very top of the main \LaTeX{} file,
in particular \emph{before} the |\documentclass| statement!
The argument of |\childdocmain| should be left empty
(but it must be present).

%%%%%%%%%%%%%%%%%%%%%%%%%%%%%%%%%%%%%%%%
\DescribeMacro{\childdocof}
Furthermore, add the commands
\begin{center}
\begin{tabular}{l}
|\input{childdoc.def}|\\
|\childdocof{|\textit{main}|}|\\
\end{tabular}
\end{center}
at the top of every child file \textit{child}
which is included by |\include{|\textit{child}|}|
from within the main file
(or at least for those files to be compiled individually).
The argument \textit{main} must be the filename of the main file.

There are a couple of
considerations in setting up the main and child documents:

%%%%%%%%%%%%%%%%%%%%%%%%%%%%%%%%%%%%%%%%
\paragraph{Restrictions.}

Please note the following restrictions:
\begin{itemize}
\item
|\childdocmain| must be called with one argument \textit{main}
to ensure compatibility with earlier version of the package.
It must either be empty (|\childdocmain{}|)
or precisely match the filename of the main file in which it is specified.
See \secref{sec:detection} for further information.
\item
The filename \textit{main} must be specified without the |.tex| extension.
\item
The filename \textit{main} is case sensitive
(even in case-insensitive file systems)
due to internal string comparison.
\item
The argument \textit{main} should be fully expanded, it cannot be a macro.
\item
Subdirectories and special characters should be avoided in filenames.
\item
The command |\childdocmain{|\textit{main}|}| must be followed by a whitespace.
It should not be followed immediately by another command
or by a comment mark `|%|'.
This is because the \TeX{} parser reads the token immediately following
the argument of |\childdocmain| and puts it
at the beginning of every child section;
however, a white\-space is ignored.
\end{itemize}

%%%%%%%%%%%%%%%%%%%%%%%%%%%%%%%%%%%%%%%%
\paragraph{Content of Main File.}

It is advisable to place all content in the child files included by |\include|.
Any output contained in the main file will appear in all child documents
unless suppressed manually;
it cannot be suppressed automatically by the |\includeonly| directive
and thus should normally be avoided.
A method to include some content in the main file
by means of conditional processing is described in \secref{sec:conditional}.

%%%%%%%%%%%%%%%%%%%%%%%%%%%%%%%%%%%%%%%%
\paragraph{Page Numbering.}

When only a part of the document is compiled,
the appropriate numbering of pages
(as well as other status parameters)
is determined from the |.aux| files.
The latter contain information from previous passes.
However this information needs to propagate through
all intermediate child documents.
Therefore the page numbering in child documents may well
be inconsistent until the complete document is compiled at least once.

A useful (if unconventional) way to always ensure a consistent
page numbering is to restart the numbering in each child document
and denote the pages by `\textit{child}|.|\textit{page}'
where \textit{child} represents the chapter/section number of the child file.
This can be achieved by the command
|\numberwithin{page}{|\textit{child}|}|
of the \textsf{amsmath} package
where \textit{child} can be |chapter| or |section|
depending on the chosen structuring.
Alternatively, one can modify the macro |\thepage| appropriately
and reset the counter |page| at the start of each child file.

%%%%%%%%%%%%%%%%%%%%%%%%%%%%%%%%%%%%%%%%%%%%%%%%%%%%%%%%%%%%%%%%%%%%%%%%%%%%%%%%
\subsection{Conditional Processing}
\label{sec:conditional}

The package provides a mechanism to compile different versions
of a document. To customise the versions further some conditional processing
can come in handy to distinguish which version is being compiled.
The package provides two macros to describe the compilation context:

%%%%%%%%%%%%%%%%%%%%%%%%%%%%%%%%%%%%%%%%
\DescribeMacro{\ifchilddoc}
The conditional |\ifchilddoc| distinguishes between the compilation of
child documents and the main document:
%
\begin{center}
|\ifchilddoc |\textit{child-code}| |[|\||else |\textit{main-code}]| \||fi|
\end{center}

%%%%%%%%%%%%%%%%%%%%%%%%%%%%%%%%%%%%%%%%
\DescribeMacro{\childdocname}
\DescribeMacro{\childdocjob}
The macro |\childdocname| contains the filename (without extension)
of the main or child file being processed.
Note that |\childdocjob| will always contain the name of the main file.

%%%%%%%%%%%%%%%%%%%%%%%%%%%%%%%%%%%%%%%%
\paragraph{Title Page.}

Conditional processing can be used to include a title or banner page
in the main document when proper precautions are taken.
Importantly, the code in the main file should ensure that the page counter
(as well as other status parameters which are stored in the |.aux| files)
takes the same value after the conditional processing.
Otherwise the page numbers may take divergent values
depending on which part is compiled.

For example, a title page could be declared by:
%
\begin{center}
\begin{tabular}{l}
|\ifchilddoc\||else|\\
|\addtocounter{page}{-1}|\\
\textit{code for title page}\\
|\newpage|\\
|\||fi|
\end{tabular}
\end{center}
%
A banner page for the child documents can be generated by:
%
\begin{center}
\begin{tabular}{l}
|\ifchilddoc|\\
|\addtocounter{page}{-1}|\\
\textit{code for banner page}\\
|\newpage|\\
|\||fi|
\end{tabular}
\end{center}
%
Here one could write a message such as:
\begin{center}
|This is the part \childdocname{} of \childdocjob{}.|
\end{center}

%%%%%%%%%%%%%%%%%%%%%%%%%%%%%%%%%%%%%%%%%%%%%%%%%%%%%%%%%%%%%%%%%%%%%%%%%%%%%%%%
\subsection{Flags}
\label{sec:flags}

The package makes it easy to generate different versions
of the main or child documents.
To this end compilation flags can be defined
and assigned different default values.
They will be particularly useful in conjunction
with the forwarding mechanism described in \secref{sec:forward}.

For example, it may be useful to have a flag |\version|
which can be set to |draft| or |final|.
The document source will contain some conditional code
depending on the value of |\version|.
Suppose further, the flag should default to |final| for the main file
and to |draft| for child files
which is a natural assignment for editing the document.
This is achieved by placing the following code
in the preamble of the main document
(below the |\childdocmain| directive):
%
\begin{center}
\begin{tabular}{l}
|\ifchilddoc|\\
|\providecommand{\version}{draft}|\\
|\||else|\\
|\providecommand{\version}{final}|\\
|\||fi|
\end{tabular}
\end{center}
%
The definition by |\providecommand| makes sure
that previous definitions are not overwritten.
Further statements |\providecommand{\version}{...}|
can thus be added before the above code to override it.

For the main file, one might add a line
(between |\childdocmain| and the above block)
%
\begin{center}
|%\ifchilddoc\||else\providecommand{\version}{draft}\||fi|
\end{center}
%
which can be uncommented to produce a draft version.
Likewise one can add a line to the very top of a child file
(above the |\childdocof{|\textit{main}|}| directive)
%
\begin{center}
|%\providecommand{\version}{final}|
\end{center}
%
which can be uncommented to produce the final version of this child document.

%%%%%%%%%%%%%%%%%%%%%%%%%%%%%%%%%%%%%%%%%%%%%%%%%%%%%%%%%%%%%%%%%%%%%%%%%%%%%%%%
\subsection{Forwarding}
\label{sec:forward}

Different versions of the main or child documents
using compilation flags as described in \secref{sec:flags}
can be (permanently) stored in different files
for convenient compilation, viewing and distribution.
To this end, the package defines a command
to pass on compilation to a different file:

%%%%%%%%%%%%%%%%%%%%%%%%%%%%%%%%%%%%%%%%
\DescribeMacro{\childdocforward}
The command |\childdocforward| redirects processing to
another source file:
%
\begin{center}
\begin{tabular}{l}
|\input{childdoc.def}|\\
|\childdocforward[|\textit{main}|]{|\textit{dest}|}|\\
\end{tabular}
\end{center}
%
The argument \textit{dest} is the destination file
(without extension).
It should be the main file or one of the child files.
Note that further \textsf{childdoc} directives
such as |\childdocof| and |\childdocforward|
in the indicated file will be processed in this form.
The optional argument \textit{main}
passes on directly to the main file \textit{main}
while pretending to compile the child \textit{dest}.
This form behaves as if \textit{dest}
issues |\childdocof{|\textit{main}|}| right away,
and no further \textsf{childdoc} directives will be processed.

%%%%%%%%%%%%%%%%%%%%%%%%%%%%%%%%%%%%%%%%
\DescribeMacro{\...prefix}
In the alternative form |\childdocforwardprefix|,
%
\begin{center}
\begin{tabular}{l}
|\input{childdoc.def}|\\
|\childdocforwardprefix[|\textit{main}|]{|\textit{prefix}|}{|\textit{dest}|}|
\end{tabular}
\end{center}
%
the destination file is determined by a pattern
depending on the current file:
To make this work, the current file must be called
`{\textit{prefix}\hspace{0.2em}\textit{suffix}}'
with \textit{prefix} matching precisely the argument.
Processing is then passed on to the file
`{\textit{dest}\hspace{0.2em}\textit{suffix}}'.
Surely, the same effect is achieved by
directly specifying the
argument `{\textit{dest}\hspace{0.2em}\textit{suffix}}'
in the first form.
However, that requires to set up a different file
for each child. With the alternative form of the command
all these files can have exactly the same content
which simplifies setting them up and maintaining them.

For example, the following file |draft.tex|
with a compilation flag |\version| as described in \secref{sec:flags}
compiles the main document as a draft:
%
\begin{center}
\begin{tabular}{l}
|\def\version{draft}|\\
|\input{childdoc.def}|\\
|\childdocforward{|\textit{main}|}|
\end{tabular}
\end{center}
%
Likewise, the following files |final|\textit{nn}|.tex|
compile the final version of the child document
|child|\textit{nn}|.tex|:
%
\begin{center}
\begin{tabular}{l}
|\def\version{final}|\\
|\input{childdoc.def}|\\
|\childdocforwardprefix{final}{child}|
\end{tabular}
\end{center}
%

Note that when several versions of a main file and/or of each child file
are to be generated, it may be convenient to set up a |Makefile| or
shell script to automatise the process.

%%%%%%%%%%%%%%%%%%%%%%%%%%%%%%%%%%%%%%%%%%%%%%%%%%%%%%%%%%%%%%%%%%%%%%%%%%%%%%%%
\subsection{Command Line Processing}
\label{sec:commandline}

The effect of redirection files can also be achieved by invoking
the \LaTeX{} compiler with a more elaborate command line.
Most conveniently this should be done as part
of a shell script or a |Makefile|.

When using \textsf{childdoc} in the main file, the following
command lines effectively perform a redirection
(note that depending on the shell being used,
backslashes may have to be doubled: `|\|' $\to$ `|\\|'):
%
\begin{center}
|... -jobname "|\textit{target}|" |\\|"|[\textit{flags}]%
|\input{childdoc.def}\childdocforward[|\textit{main}|]{|\textit{dest}|}"|
\end{center}
%
Here \textit{target} is the name of the output file,
\textit{main} is the name of the main file
and \textit{dest} is the name of the main or child file to be processed
(all filenames without extensions).
The optional argument \textit{main} can be omitted
if \textit{main} matches \textit{dest}.
Optionally, compilation \textit{flags} can be defined via |\def| commands.
This command line makes the \TeX{} engine believe
it is compiling the file \textit{target}
whose content is specified as the latter parameter.
The provided code then forwards the processing to
\textit{main} or \textit{dest} as described in \secref{sec:forward}.

%%%%%%%%%%%%%%%%%%%%%%%%%%%%%%%%%%%%%%%%%%%%%%%%%%%%%%%%%%%%%%%%%%%%%%%%%%%%%%%%
\subsection{Include by Input}
\label{sec:input}

Including child documents by |\include| has some restrictions by design.
Most notably, the content of a child document always occupies
its own set of pages; pages cannot be shared between child documents.
Usually, this behaviour makes perfect sense
because each child document contain an essential part of the document.
However, in some situations it may be desirable to compose
a document from a collection of parts
without having mandatory page breaks between then.
For this case, the package
provides a mechanism to include parts
by |\input| which can also be processed individually.
However, by construction this mechanism
requires manual handling of the content to be output.

%%%%%%%%%%%%%%%%%%%%%%%%%%%%%%%%%%%%%%%%
\DescribeMacro{\ifchilddocmanual}
The main file should be prepared as usual, see \secref{sec:include}.
However, the document body must make a distinction
between processing of an individual part and of the main document, e.g.:
%
\begin{center}
\begin{tabular}{l}
|\ifchilddocmanual|\\
|\input{\childdocname}|\\
|\||else|\\
\textit{document body with }|\input{|\textit{part}|}|\\
|\||fi|
\end{tabular}
\end{center}
%
The conditional |\ifchilddocmanual| is true whenever
a part to be included by |\input| is being compiled,
and the name of the part is stored in |\childdocname|.

%%%%%%%%%%%%%%%%%%%%%%%%%%%%%%%%%%%%%%%%
\DescribeMacro{\childdocby}
Each part to be included by |\input| should start with:
%
\begin{center}
\begin{tabular}{l}
|\input{childdoc.def}|\\
|\childdocby{|\textit{main}|}|\\
\end{tabular}
\end{center}
%
The directive |\childdocby| is similar to |\childdocof|
described in \secref{sec:include},
but the subsequent selection of content must be done manually.
To that end, both |\ifchilddoc| and |\ifchilddocmanual|
will be true upon processing of a part,
and the name of the part is stored in |\childdocname|.
Note that |\jobname| will be set to the filename of the current part
so that each part receives an individual |.aux| file
that does not interfere with the |.aux| file(s) of the main document.
This behaviour can be altered by the alternative form
|\childdocby[*]{|\textit{main}|}| (with a non-empty optional argument)
which uses the |.aux| file of the main document
by setting |\jobname| to \textit{main}.

%%%%%%%%%%%%%%%%%%%%%%%%%%%%%%%%%%%%%%%%%%%%%%%%%%%%%%%%%%%%%%%%%%%%%%%%%%%%%%%%
\subsection{Driver Development}
\label{sec:driver}

The \textsf{childdoc} mechanism can also be use for the development
of definition files such as \LaTeX{} styles or classes.
This case differs from the above setup with multiple parts
included by |\include| in that no |\includeonly| should be invoked.
This can be achieved by starting the include file
(before |\ProvidesPackage|) with:
%
\begin{center}
\begin{tabular}{l}
|\input{childdoc.def}|\\
|\childdocforward{|\textit{main}|}|\\
\end{tabular}
\end{center}
%
or alternatively with:
%
\begin{center}
\begin{tabular}{l}
|\input{childdoc.def}|\\
|\childdocby{|\textit{main}|}|\\
\end{tabular}
\end{center}
%
Both forms have slightly different effects as described above.
The main file is prepared as usual, see \secref{sec:include}.

%%%%%%%%%%%%%%%%%%%%%%%%%%%%%%%%%%%%%%%%%%%%%%%%%%%%%%%%%%%%%%%%%%%%%%%%%%%%%%%%
\subsection{Legacy Detection}
\label{sec:detection}

The directive |\childdocmain| in the main file can detect
whether the complete document or merely a child is to be compiled
even without using the directive |\childdocof|.
This method is deprecated because it is less robust
and there is no compelling reason to use it;
it is merely provided for backward compatibility
and it may be removed in future versions.

If the detection mechanism is to be used,
it is mandatory to correctly specify
the filename of the main file as the argument of |\childdocmain|:
%
\begin{center}
\begin{tabular}{l}
|\input{childdoc.def}|\\
|\childdocmain{|\textit{main}|}|\\
\end{tabular}
\end{center}
%
If |\jobname| does not match the argument \textit{main} of |\childdocmain|,
it is assumed that |\jobname| points to the child file to be compiled.
When using |\childdocmain| with the main file specified as argument,
it suffices to start a child file
with just |\input{|\textit{main}|}|
without loading of the package and using |\childdocof|.
If instead all processing is done
with the appropriate \textsf{childdoc} directives,
the argument of \textit{main} of |\childdocmain| can be empty.

An alternative version of the command line processing described
in \secref{sec:commandline} using the detection mechanism reads:
%
\begin{center}
|... -jobname "|\textit{target}|" "|[\textit{flags}]%
[|\def\jobname{|\textit{dest}|}|]|\input{|\textit{main}|}"|
\end{center}

%%%%%%%%%%%%%%%%%%%%%%%%%%%%%%%%%%%%%%%%%%%%%%%%%%%%%%%%%%%%%%%%%%%%%%%%%%%%%%%%
\subsection{Manual Code}
\label{sec:manual}

In case one cannot be certain whether the definitions file |childdoc.def|
is installed on the target \TeX{} distribution
and one prefers not to ship it,
it is conceivable to paste a few relevant commands into the sources.

To that end, drop all statements |\input{childdoc.def}|
and perform the replacements as outlined below.
Instead of |\childdocmain{|\textit{main}|}| add the following code
to the top of the main file:
%
\begin{center}
\begin{tabular}{l}
|\||ifdefined\childdocname\endinput\||fi\newif\ifchilddoc|\\
|\edef\childdocname{\scantokens\expandafter{\jobname\noexpand}}|\\
|\def\childdocmain{|\textit{main}|}\||ifx\childdocmain\childdocname\||else|\\
|\childdoctrue\includeonly{\childdocname}\let\jobname\childdocmain\||fi|\\
\end{tabular}
\end{center}
%
Instead of |\childdocof{|\textit{main}|}| just include the main file
at the top of each child file:
%
\begin{center}
|\input{|\textit{main}|}|
\end{center}
%
A simple redirection |\childdocforward{|\textit{dest}|}| is achieved by:
%
\begin{center}
|\def\jobname{|\textit{dest}|}\input{\jobname}|
\end{center}
%
The redirection with prefix
|\childdocforwardprefix[|\textit{prefix}|]{|\textit{dest}|}|
is accomplished by:
%
\begin{center}
\begin{tabular}{l}
|{\edef\jobname{\scantokens\expandafter{\jobname\noexpand}}|\\
|\def\redirectjob |\textit{prefix}|#1~~~{\gdef\jobname{|\textit{dest}|#1}}|\\
|\expandafter\redirectjob\jobname~~~}\input{\jobname}|
\end{tabular}
\end{center}

In an alternative approach,
child documents can be compiled by a specific command line
without additional code or specific definitions:
%
\begin{center}
|... -jobname "|\textit{target}|" "|[\textit{flags}]%
|\includeonly{|\textit{dest}|}\input{|\textit{main}|}"|
\end{center}
%

%%%%%%%%%%%%%%%%%%%%%%%%%%%%%%%%%%%%%%%%%%%%%%%%%%%%%%%%%%%%%%%%%%%%%%%%%%%%%%%%
%%%%%%%%%%%%%%%%%%%%%%%%%%%%%%%%%%%%%%%%%%%%%%%%%%%%%%%%%%%%%%%%%%%%%%%%%%%%%%%%
\section{Information}

%%%%%%%%%%%%%%%%%%%%%%%%%%%%%%%%%%%%%%%%%%%%%%%%%%%%%%%%%%%%%%%%%%%%%%%%%%%%%%%%
\subsection{Copyright}

Copyright \copyright{} 2017--2018 Niklas Beisert

This work may be distributed and/or modified under the
conditions of the \LaTeX{} Project Public License, either version 1.3
of this license or (at your option) any later version.
The latest version of this license is in
  \url{http://www.latex-project.org/lppl.txt}
and version 1.3 or later is part of all distributions of \LaTeX{}
version 2005/12/01 or later.

This work has the LPPL maintenance status `maintained'.

The Current Maintainer of this work is Niklas Beisert.

This work consists of the files |README.txt|, |childdoc.ins| and |childdoc.dtx|
as well as the derived files |childdoc.def|, |cdocsamp.tex|
with |cdocsch1.tex|, |cdocsch2.tex|, |cdocspt3.tex|, |cdocspt4.tex|,
|cdocsdrf.tex|, |cdocsfn1.tex|, |cdocsfn2.tex|
as well as |childdoc.pdf|.

%%%%%%%%%%%%%%%%%%%%%%%%%%%%%%%%%%%%%%%%%%%%%%%%%%%%%%%%%%%%%%%%%%%%%%%%%%%%%%%%
\subsection{Files and Installation}

The package consists of the files:
%
\begin{center}
\begin{tabular}{ll}
    |README.txt|   & readme file \\
    |childdoc.ins| & installation file \\
    |childdoc.dtx| & source file \\
    |childdoc.def| & definition file \\
    |cdocsamp.tex| & sample main file \\
    |cdocsch1.tex| & sample include file \\
    |cdocsch2.tex| & sample include file \\
    |cdocspt3.tex| & sample part file \\
    |cdocspt4.tex| & sample part file \\
    |cdocsdrf.tex| & sample redirection file \\
    |cdocsfn1.tex| & sample redirection file \\
    |cdocsfn2.tex| & sample redirection file \\
    |childdoc.pdf| & manual
\end{tabular}
\end{center}
%
The distribution consists of the files
|README.txt|, |childdoc.ins| and |childdoc.dtx|.
%
\begin{itemize}
\item
Run (pdf)\LaTeX{} on |childdoc.dtx|
to compile the manual |childdoc.pdf| (this file).
\item
Run \LaTeX{} on |childdoc.ins| to create the definitions file |childdoc.def|
and the sample |cdocsamp.tex| with include files
|cdocsch1.tex|, |cdocsch2.tex|, |cdocspt3.tex|, |cdocspt4.tex|,
|cdocsdrf.tex|, |cdocsfn1.tex|, |cdocsfn2.tex|.
Then copy the file |childdoc.def| to an appropriate directory of your \LaTeX{}
distribution, e.g.\ \textit{texmf-root}|/tex/latex/childdoc|.
\end{itemize}

%%%%%%%%%%%%%%%%%%%%%%%%%%%%%%%%%%%%%%%%%%%%%%%%%%%%%%%%%%%%%%%%%%%%%%%%%%%%%%%%
\subsection{Related CTAN Packages}

There are several other packages which offer a similar functionality:
%
\begin{itemize}
\item
The packages
\href{http://ctan.org/pkg/docmute}{\textsf{docmute}},
\href{http://ctan.org/pkg/includex}{\textsf{includex}} and
\href{http://ctan.org/pkg/standalone}{\textsf{standalone}}
provide commands to include only the document body of
a child file thus allowing both files to be compiled individually.
\item
The packages \href{http://ctan.org/pkg/subdocs}{\textsf{subdocs}}
and \href{http://ctan.org/pkg/subfiles}{\textsf{subfiles}}
provide structures in which the main and child documents can be
encapsulated and allowing them to be compiled individually.
The inclusion mechanism is different from the conventional |\include|.
\item
The package \href{http://ctan.org/pkg/combine}{\textsf{combine}}
is an elaborate solution to combine several documents into one.
\end{itemize}
%
See also the CTAN topic \href{http://ctan.org/topic/subdocs}{\textsf{subdocs}}
for further related packages.
The present package differs from the above solutions in that
a document structure constructed with the conventional |\include| mechanism
just needs two extra commands at the top of every file
such that all constituent files can be compiled individually.

%%%%%%%%%%%%%%%%%%%%%%%%%%%%%%%%%%%%%%%%%%%%%%%%%%%%%%%%%%%%%%%%%%%%%%%%%%%%%%%%
%\subsection{Feature Suggestions}
%
%The following is a list of features which may be useful for future
%versions of this package:
%%
%\begin{itemize}
%\item
%\ldots
%\end{itemize}

%%%%%%%%%%%%%%%%%%%%%%%%%%%%%%%%%%%%%%%%%%%%%%%%%%%%%%%%%%%%%%%%%%%%%%%%%%%%%%%%
\subsection{Revision History}

%%%%%%%%%%%%%%%%%%%%%%%%%%%%%%%%%%%%%%%%
\paragraph{v2.0:} 2018/12/30

\begin{itemize}
\item
immediate forward processing
\item
added |\childdocby| mechanism
\item
manual restructured
\end{itemize}

%%%%%%%%%%%%%%%%%%%%%%%%%%%%%%%%%%%%%%%%
\paragraph{v1.6:} 2018/01/17

\begin{itemize}
\item
application for development of include files
\item
corrections to manual
\end{itemize}

%%%%%%%%%%%%%%%%%%%%%%%%%%%%%%%%%%%%%%%%
\paragraph{v1.5:} 2017/05/21

\begin{itemize}
\item
more complete structuring introduced
\item
|\childdocof| introduced
\item
|\childdoc| renamed to |\childdocmain|
\item
|\childredirect| renamed to |\childdocforward| and |\childdocforwardprefix|
and functionality expanded
\end{itemize}

%%%%%%%%%%%%%%%%%%%%%%%%%%%%%%%%%%%%%%%%
\paragraph{v1.0:} 2017/04/27

\begin{itemize}
\item
manual and install package
\item
first version published on CTAN
\end{itemize}

%%%%%%%%%%%%%%%%%%%%%%%%%%%%%%%%%%%%%%%%
\paragraph{v0.6:} 2017/04/26

\begin{itemize}
\item
redirection mechanism added
\end{itemize}

%%%%%%%%%%%%%%%%%%%%%%%%%%%%%%%%%%%%%%%%
\paragraph{v0.5:} 2017/04/26

\begin{itemize}
\item
functionality in definition file
\end{itemize}


%%%%%%%%%%%%%%%%%%%%%%%%%%%%%%%%%%%%%%%%%%%%%%%%%%%%%%%%%%%%%%%%%%%%%%%%%%%%%%%%
%%%%%%%%%%%%%%%%%%%%%%%%%%%%%%%%%%%%%%%%%%%%%%%%%%%%%%%%%%%%%%%%%%%%%%%%%%%%%%%%
%%%%%%%%%%%%%%%%%%%%%%%%%%%%%%%%%%%%%%%%%%%%%%%%%%%%%%%%%%%%%%%%%%%%%%%%%%%%%%%%
\appendix

\settowidth\MacroIndent{\rmfamily\scriptsize 000\ }

 \DocInput{childdoc.dtx}

\end{document}
%</driver>
% \fi
%
% %%%%%%%%%%%%%%%%%%%%%%%%%%%%%%%%%%%%%%%%%%%%%%%%%%%%%%%%%%%%%%%%%%%%%%%%%%%%%%
% %%%%%%%%%%%%%%%%%%%%%%%%%%%%%%%%%%%%%%%%%%%%%%%%%%%%%%%%%%%%%%%%%%%%%%%%%%%%%%
% \section{Sample}
%\iffalse
%<*samplemain>
%\fi
%
% The following presents a sample document
% with two chapters, two parts, a title page,
% a compile flag as well as three forwarding files to set the flag.
% It consists of eight |.tex| files:
% \begin{center}
% \begin{tabular}{ll}
% |cdocsamp.tex|&main file\\
% |cdocsch1.tex|&include file for chapter 1\\
% |cdocsch2.tex|&include file for chapter 2\\
% |cdocspt3.tex|&include file for part 3\\
% |cdocspt4.tex|&include file for part 4\\
% |cdocsdrf.tex|&forwarding file for main file in draft mode\\
% |cdocsfi1.tex|&forwarding file for final version of chapter 1\\
% |cdocsfi2.tex|&forwarding file for final version of chapter 2\\
% \end{tabular}
% \end{center}
% Each of the eight files can be compiled directly by the \LaTeX{} compiler.
%
% %%%%%%%%%%%%%%%%%%%%%%%%%%%%%%%%%%%%%%
% \paragraph{Main File.}
%
% The main file is called |cdocsamp.tex|.
%
% Load the \textsf{childdoc} definitions and
% declare the filename for the main document:
%    \begin{macrocode}
\input{childdoc.def}
\childdocmain{}
%    \end{macrocode}

% Optional override for |\version| flag:
%    \begin{macrocode}
%%\ifchilddoc\else\providecommand{\version}{draft}\fi
%    \end{macrocode}

% Define the default values for the |\version| flag
% (|final| for the main file and |draft| for childs):
%    \begin{macrocode}
\ifchilddoc
\providecommand{\version}{draft}
\else
\providecommand{\version}{final}
\fi
%    \end{macrocode}

% Load the standard document class:
%    \begin{macrocode}
\documentclass[12pt]{article}
%    \end{macrocode}

% Start the document body:
%    \begin{macrocode}
\begin{document}
%    \end{macrocode}

% Declare a title page.
% Print title, part of document being processed and version flag:
%    \begin{macrocode}
\addtocounter{page}{-1}
\begin{center}
{\LARGE\bfseries{}childdoc example\par}
\vspace{1cm}
\ifchilddoc
\ifchilddocmanual part\else chapter\fi:
`\childdocname' of `\childdocjob'\par
\else
main document: `\childdocjob'\par
\fi
version: \version\par
\end{center}
\newpage
%    \end{macrocode}

% Manually include selected file,
% otherwise process as usual:
%    \begin{macrocode}
\ifchilddocmanual
\section*{part `\childdocname'}
\input{\childdocname}
\else
%    \end{macrocode}

% Include the two chapters:
%    \begin{macrocode}
\include{cdocsch1}
\include{cdocsch2}
%    \end{macrocode}

% Include the two parts unless only chapters should be displayed:
%    \begin{macrocode}
\ifchilddoc\else
\section{part three}
\input{cdocspt3}
\section{part four}
\input{cdocspt4}
\fi
%    \end{macrocode}

% Process as usual until here:
%    \begin{macrocode}
\fi
%    \end{macrocode}

% End of document body:
%    \begin{macrocode}
\end{document}
%    \end{macrocode}
%\iffalse
%</samplemain>
%\fi
%
% %%%%%%%%%%%%%%%%%%%%%%%%%%%%%%%%%%%%%%
% \paragraph{Chapter Include Files.}
%
% The include files are called |cdocsch1.tex| and |cdocsch2.tex|.
%
%\iffalse
%<*samplechap1|samplechap2>
%\fi

% Optional override for |\version| flag:
%    \begin{macrocode}
%%\providecommand{\version}{final}
%    \end{macrocode}

% Include the main document:
%    \begin{macrocode}
\input{childdoc.def}
\childdocof{cdocsamp}
%    \end{macrocode}

%\iffalse
%</samplechap1|samplechap2>
%\fi
%
%\iffalse
%<*samplechap1>
%\fi
% Some text for chapter 1:
%    \begin{macrocode}
\section{one}
some text in chapter one
%    \end{macrocode}

%\iffalse
%</samplechap1>
%\fi
% Some text for chapter 2:
%\iffalse
%<*samplechap2>
%\fi
%    \begin{macrocode}
\section{two}
more text in chapter two
%    \end{macrocode}

%\iffalse
%</samplechap2>
%\fi
%
% %%%%%%%%%%%%%%%%%%%%%%%%%%%%%%%%%%%%%%
% \paragraph{Part Include Files.}
%
% The include files are called |cdocspt3.tex| and |cdocspt4.tex|.
%
%\iffalse
%<*samplepart3|samplepart4>
%\fi

% Optional override for |\version| flag:
%    \begin{macrocode}
%%\providecommand{\version}{final}
%    \end{macrocode}

% Include the main document:
%    \begin{macrocode}
\input{childdoc.def}
\childdocby{cdocsamp}
%    \end{macrocode}

%\iffalse
%</samplepart3|samplepart4>
%\fi
%
%\iffalse
%<*samplepart3>
%\fi
% Some text for part 3:
%    \begin{macrocode}
some text in part three
%    \end{macrocode}

%\iffalse
%</samplepart3>
%\fi
% Some text for part 4:
%\iffalse
%<*samplepart4>
%\fi
%    \begin{macrocode}
more text in part four
%    \end{macrocode}

%\iffalse
%</samplepart4>
%\fi
%
% %%%%%%%%%%%%%%%%%%%%%%%%%%%%%%%%%%%%%%
% \paragraph{Forwarding for a Complete Draft.}
%
% The following forwarding file |cdocsdrf.tex|
% compiles the main document in draft mode:
%\iffalse
%<*sampledraft>
%\fi
%    \begin{macrocode}
\def\version{draft}
\input{childdoc.def}
\childdocforward{cdocsamp}
%    \end{macrocode}

%\iffalse
%</sampledraft>
%\fi
%
% %%%%%%%%%%%%%%%%%%%%%%%%%%%%%%%%%%%%%%
% \paragraph{Forwarding for Final Version of the Chapters.}
%
% The following forwarding files |cdocsfn1.tex| and |cdocsfn2.tex|
% (with identical content)
% compile the final versions of the child documents
% |cdocsch1.tex| and |cdocsch2.tex|, respectively:
%\iffalse
%<*samplefinal>
%\fi
%    \begin{macrocode}
\def\version{final}
\input{childdoc.def}
\childdocforwardprefix[cdocsamp]{cdocsfn}{cdocsch}
%    \end{macrocode}

%\iffalse
%</samplefinal>
%\fi
%
% %%%%%%%%%%%%%%%%%%%%%%%%%%%%%%%%%%%%%%
% \paragraph{Command Line Processing.}
%
% The following three command lines generate the output files
% |cdocscld|, |cdocscl1| and |cdocscl2|
% which should be identical to
% |cdocsdrf|, |cdocsch1| and |cdocsfn2|, respectively:
% \begin{center}
% \begin{tabular}{l}
% |latex -jobname cdocscld \|\\
% |  "\def\version{draft}\input{childdoc.def}\childdocforward{cdocsamp}"|\\
% |latex -jobname cdocscl1 \|\\
% |  "\input{childdoc.def}\childdocforward[cdocsamp]{cdocsch1}"|\\
% |latex -jobname cdocscl2 \|\\
% |  "\def\version{final}\input{childdoc.def}\childdocforward{cdocsch2}"|
% \end{tabular}
% \end{center}
% Note that the trailing backslash on each first line
% merely continues the input to the second line
% (for convenient cut ant paste).
% Furthermore, the command |latex| can be replaced by any
% of its alternative versions such as |pdflatex|.
%
% %%%%%%%%%%%%%%%%%%%%%%%%%%%%%%%%%%%%%%%%%%%%%%%%%%%%%%%%%%%%%%%%%%%%%%%%%%%%%%
% %%%%%%%%%%%%%%%%%%%%%%%%%%%%%%%%%%%%%%%%%%%%%%%%%%%%%%%%%%%%%%%%%%%%%%%%%%%%%%
% \section{Implementation}
%\iffalse
%<*package>
%\fi
%
% This section describes the definitions file |childdoc.def|.

% The definitions cannot be loaded using |\usepackage| or |\RequirePackage|
% which has a mechanism to prevent loading a style file more than once.
% When loading the definitions by means of |\input|
% multiple instances have to be prevented manually:
%\iffalse
%This code needs to be before the `\ProvidesFile' directive
%which is defined at the beginning of this file.
%Therefore it is also placed there and commented out here.
%</package>
%<*discard>
%\fi
%    \begin{macrocode}
\ifdefined\childdocmain\endinput\fi
%    \end{macrocode}
%\iffalse
%</discard>
%<*package>
%\fi
%
% \macro{\ifchilddoc}
% \macro{\ifchilddocmanual}
% The conditional |\ifchilddoc| tells whether a
% child (true) or main (false) document is being compiled.
% The conditional |\ifchilddocmanual| tells whether
% the |\includeonly| mechanism is used (false) or
% the selection of child files must be performed manually (true).
% The definitions initialise to false:
%    \begin{macrocode}
\newif\ifchilddoc
\newif\ifchilddocmanual
%    \end{macrocode}

% \macro{\childdocname}
% \macro{\childdocjob}
% The macro |\childdocname| stores the name of the main document
% to be compiled. The macro |\childdocjob| stores the name of
% the document on which the \LaTeX{} compiler was originally invoked.
% The content of |\jobname| cannot be compared
% to filenames specified in the source due to different catcodes.
% The following code rescans |\jobname|, stores the result
% in |\childdocname| and saves a copy in |\childdocjob|:
%    \begin{macrocode}
\edef\childdocname{\scantokens\expandafter{\jobname\noexpand}}
\let\childdocjob\childdocname
%    \end{macrocode}

% \macro{\childdocdisable}
% The macro |\childdocdisable| prevents the main file
% from being processed more than once.
% At this stage, the main document command |\childdocmain|
% is assumed to be called once again where it should do nothing.
% Any subsequent call to it should prevent
% a secondary processing of the main document
% It overwrites the forwarding commands
% |\childdocof| and |\childdocforward|
% with empty macros to prevent further inclusions of the main document:
%    \begin{macrocode}
\newcommand{\childdocdisable}
{
  \renewcommand{\childdocmain}[1]{\renewcommand{\childdocmain}[1]{\endinput}}
  \renewcommand{\childdocof}[1]{}
  \renewcommand{\childdocby}[2][]{}
  \renewcommand{\childdocforward}[2][]{}
  \renewcommand{\childdocdisable}{}
}
%    \end{macrocode}

% \macro{\childdocmain}
% The macro |\childdocmain| is to be called at the top of the main file
% with nothing or the main filename (without extension) as argument.
% First, it breaks loops.
% If the argument is not empty and does not match |\childdocname|
% (which is set by the first inclusion of |childdoc.def|),
% |\ifchilddoc| is set to true, |\includeonly| is applied to the child file
% and |\jobname| is set to the main file
% (for proper handling of |.aux| files):
%    \begin{macrocode}
\newcommand{\childdocmain}[1]
{
  \childdocdisable\childdocmain{}
  \if?#1?\else
    \begingroup
      \def\childdoctmp{#1}
      \ifx\childdoctmp\childdocname
        \def\childdoctmp{}
      \else
        \def\childdoctmp
        {
          \childdoctrue
          \includeonly{\childdocname}
          \def\childdocjob{#1}
          \def\jobname{#1}
        }
      \fi
      \expandafter
    \endgroup
    \childdoctmp
  \fi
}
%    \end{macrocode}

% \macro{\childdocof}
% The command |\childdocof| redirects
% compilation to the main file |#1|.
%    \begin{macrocode}
\newcommand{\childdocof}[1]
{
  \childdocdisable
  \childdoctrue
  \includeonly{\childdocname}
  \def\jobname{#1}
  \def\childdocjob{#1}
  \input{#1}
}
%    \end{macrocode}

% \macro{\childdocby}
% The command |\childdocby| ....
%    \begin{macrocode}
\newcommand{\childdocby}[2][]
{
  \childdocdisable
  \childdoctrue
  \childdocmanualtrue
  \if?#1?\else
    \def\jobname{#2}
  \fi
  \def\childdocjob{#2}
  \input{#2}
  \endinput
}
%    \end{macrocode}

% \macro{\childdocforward}
% The command |\childdocforward| redirects
% compilation to the main file or
% (if the optional argument is given) a child file.
% Parameters are set as if the main file
% or a child file starting with |\childdocof| was compiled.
% Then compilation is handed over to the main file:
%    \begin{macrocode}
\newcommand{\childdocforward}[2][]
{
  \begingroup
    \if?#1?
      \def\childdoctmp
      {
        \def\childdocname{#2}
        \def\childdocjob{#2}
        \def\jobname{#2}
        \input{#2}
        \endinput
      }
    \else
      \def\childdoctmp
      {
        \childdocdisable
        \def\childdocname{#2}
        \childdoctrue
        \includeonly{#2}
        \def\childdocjob{#1}
        \def\jobname{#1}
        \input{#1}
        \endinput
      }
    \fi
    \expandafter
  \endgroup
  \childdoctmp
}
%    \end{macrocode}

% \macro{\childdocforwardprefix}
% The command |\childdocforwardprefix| redirects
% compilation to the main or a child file by means of a pattern.
% The prefix |#1| in the current filename is replaced by |#2|
% and the suffix of the current filename is kept
% (it is assumed that the filename does not contain the substring `|~~~|'
% which is used as a delimiter).
% Compilation is handed over to the new file by |\childdocforward|:
%    \begin{macrocode}
\newcommand{\childdocforwardprefix}[3][]
{
  \begingroup
    \def\childdocextract #2##1~~~{\def\childdoctmp{\childdocforward[#1]{#3##1}}}
    \expandafter\childdocextract\childdocname~~~
    \expandafter
  \endgroup
  \childdoctmp
}
%    \end{macrocode}

% \macro{\childdoc}
% The deprecated macro |\childdoc| is a legacy version of |\childdocmain|:
%    \begin{macrocode}
\newcommand{\childdoc}{\childdocmain}
%    \end{macrocode}

% \macro{\childdocredirect}
% The deprecated macro |\childdocredirect| is a legacy version
% of |\childdocforward| and |\childdocforwardprefix|:
%    \begin{macrocode}
\newcommand{\childdocredirect}[2][]
{
  \begingroup
    \if?#1?
      \def\childdoctmp{\childdocforward{#2}}
    \else
      \def\childdoctmp{\childdocforwardprefix{#1}{#2}}
    \fi
    \expandafter
  \endgroup
  \childdoctmp
}
%    \end{macrocode}

%\iffalse
%</package>
%\fi
%
\endinput
|\\
|\childdocforwardprefix{final}{child}|
\end{tabular}
\end{center}
%

Note that when several versions of a main file and/or of each child file
are to be generated, it may be convenient to set up a |Makefile| or
shell script to automatise the process.

%%%%%%%%%%%%%%%%%%%%%%%%%%%%%%%%%%%%%%%%%%%%%%%%%%%%%%%%%%%%%%%%%%%%%%%%%%%%%%%%
\subsection{Command Line Processing}
\label{sec:commandline}

The effect of redirection files can also be achieved by invoking
the \LaTeX{} compiler with a more elaborate command line.
Most conveniently this should be done as part
of a shell script or a |Makefile|.

When using \textsf{childdoc} in the main file, the following
command lines effectively perform a redirection
(note that depending on the shell being used,
backslashes may have to be doubled: `|\|' $\to$ `|\\|'):
%
\begin{center}
|... -jobname "|\textit{target}|" |\\|"|[\textit{flags}]%
|% \iffalse
%
% childdoc.dtx Copyright (C) 2017-2018 Niklas Beisert
%
% This work may be distributed and/or modified under the
% conditions of the LaTeX Project Public License, either version 1.3
% of this license or (at your option) any later version.
% The latest version of this license is in
%   http://www.latex-project.org/lppl.txt
% and version 1.3 or later is part of all distributions of LaTeX
% version 2005/12/01 or later.
%
% This work has the LPPL maintenance status `maintained'.
%
% The Current Maintainer of this work is Niklas Beisert.
%
% This work consists of the files childdoc.dtx and childdoc.ins
% and the derived files childdoc.def and cdocsamp.tex with
% cdocsch1.tex, cdocsch2.tex, cdocsdrf.tex, cdocsfn1.tex, cdocsfn2.tex.
%
%<package>\ifdefined\childdocmain\endinput\fi
%<package>\ProvidesFile{childdoc.def}[2018/12/30 v2.0 child document driver]
%<samplemain>\ProvidesFile{cdocsamp.tex}[2018/12/30 v2.0 sample for childdoc]
%<*driver>
%\ProvidesFile{childdoc.drv}[2018/12/30 v2.0 childdoc reference manual file]
\PassOptionsToClass{10pt,a4paper}{article}
\documentclass{ltxdoc}

\usepackage[margin=35mm]{geometry}
\usepackage{hyperref}
\usepackage{hyperxmp}
\usepackage[usenames]{color}

\hypersetup{colorlinks=true}
\hypersetup{pdfstartview=FitH}
\hypersetup{pdfpagemode=UseNone}
\hypersetup{pdfsource={}}
\hypersetup{pdflang={en-UK}}
\hypersetup{pdfcopyright={Copyright 2017-2018 Niklas Beisert.
  This work may be distributed and/or modified under the
  conditions of the LaTeX Project Public License, either version 1.3
  of this license or (at your option) any later version.}}
\hypersetup{pdflicenseurl={http://www.latex-project.org/lppl.txt}}
\hypersetup{pdfcontactaddress={ETH Zurich, ITP, HIT K,
  Wolfgang-Pauli-Strasse 27}}
\hypersetup{pdfcontactpostcode={8093}}
\hypersetup{pdfcontactcity={Zurich}}
\hypersetup{pdfcontactcountry={Switzerland}}
\hypersetup{pdfcontactemail={nbeisert@itp.phys.ethz.ch}}
\hypersetup{pdfcontacturl={http://people.phys.ethz.ch/\xmptilde nbeisert/}}

\newcommand{\secref}[1]{\hyperref[#1]{section \ref*{#1}}}

\parskip1ex
\parindent0pt
\let\olditemize\itemize
\def\itemize{\olditemize\parskip0pt}

\begin{document}

\title{The \textsf{childdoc} Package}
\hypersetup{pdftitle={The childdoc Package}}
\author{Niklas Beisert\\[2ex]
  Institut f\"ur Theoretische Physik\\
  Eidgen\"ossische Technische Hochschule Z\"urich\\
  Wolfgang-Pauli-Strasse 27, 8093 Z\"urich, Switzerland\\[1ex]
  \href{mailto:nbeisert@itp.phys.ethz.ch}
  {\texttt{nbeisert@itp.phys.ethz.ch}}}
\hypersetup{pdfauthor={Niklas Beisert}}
\hypersetup{pdfsubject={Manual for the LaTeX2e Package childdoc}}
\date{30 December 2018, \textsf{v2.0}}
\maketitle

\begin{abstract}\noindent
\textsf{childdoc} is a \LaTeXe{} package
that enables the direct compilation
of document sections included by |\include|
to individual files.
\end{abstract}

\begingroup
\parskip0ex
\tableofcontents
\endgroup

%%%%%%%%%%%%%%%%%%%%%%%%%%%%%%%%%%%%%%%%%%%%%%%%%%%%%%%%%%%%%%%%%%%%%%%%%%%%%%%%
%%%%%%%%%%%%%%%%%%%%%%%%%%%%%%%%%%%%%%%%%%%%%%%%%%%%%%%%%%%%%%%%%%%%%%%%%%%%%%%%
\section{Introduction}

\LaTeX{} provides a mechanism to structure a large document (such as a book)
into a main file and several child files (containing the chapters)
using the |\include| command.
This mechanism is beneficial for documents
which span hundreds of pages in order to
make the source file(s) more manageable.
Moreover, compilation can be restricted to
selected child files by means of the |\includeonly| command.
The latter feature can be used to reduce the compilation time while editing
(this was significantly more useful in the earlier days of \LaTeX{})
or to generate a smaller document which is easier to navigate.
Another application of |\includeonly| is to generate
documents consisting of selected parts of the complete document.

However, there are a few drawbacks of the plain |\include| mechanism:
\begin{itemize}
\item
The child files cannot be compiled on their own,
they can only be compiled via the main file.
A naive editing environment
(such as a text editor with an option
to have the current file processed by \LaTeX)
may require one to switch to the main file before compiling;
attempting to compile the child file produces errors.
\item
The main file must be modified (each time)
to adjust the |\includeonly| command
to the present needs. This easily leaves the main file in a messy state.
\item
The generated document will always carry the filename
of the main document. This is inconvenient if
several child files are to be compiled and
to be kept for distribution.
\end{itemize}

The present package provides a simple interface
to make child files individually compilable by \LaTeX{}.
Compiling a child file then has the same effect as compiling
the main file with an |\includeonly| command
to select the appropriate child.
Moreover the generated document will carry the name of the child
rather than the main file.
This resolves all three above issues.

This feature is meant to make the editing of books,
thesis documents and lecture notes somewhat more convenient.
However, the package can also be used efficiently for
composing a series of documents (such as exercise sheets)
which are typically distributed individually.
It then assists the author in generating the individual documents
(potentially in different versions)
as well as a document containing the collected series.
Another application is in developing style files
or other kinds of included material
where compilation of the style file could redirect
to a sample or test file.

%%%%%%%%%%%%%%%%%%%%%%%%%%%%%%%%%%%%%%%%%%%%%%%%%%%%%%%%%%%%%%%%%%%%%%%%%%%%%%%%
%%%%%%%%%%%%%%%%%%%%%%%%%%%%%%%%%%%%%%%%%%%%%%%%%%%%%%%%%%%%%%%%%%%%%%%%%%%%%%%%
\section{Usage}

First of all, the package \textsf{childdoc} is \emph{not} a standard
\LaTeXe{} |.sty| style file! Therefore it needs to be invoked in
a non-standard way.

%%%%%%%%%%%%%%%%%%%%%%%%%%%%%%%%%%%%%%%%%%%%%%%%%%%%%%%%%%%%%%%%%%%%%%%%%%%%%%%%
\subsection{Included Files}
\label{sec:include}

%%%%%%%%%%%%%%%%%%%%%%%%%%%%%%%%%%%%%%%%
\DescribeMacro{\childdocmain}
To use the package, add the commands
\begin{center}
\begin{tabular}{l}
|\input{childdoc.def}|\\
|\childdocmain{}|\\
\end{tabular}
\end{center}
at the very top of the main \LaTeX{} file,
in particular \emph{before} the |\documentclass| statement!
The argument of |\childdocmain| should be left empty
(but it must be present).

%%%%%%%%%%%%%%%%%%%%%%%%%%%%%%%%%%%%%%%%
\DescribeMacro{\childdocof}
Furthermore, add the commands
\begin{center}
\begin{tabular}{l}
|\input{childdoc.def}|\\
|\childdocof{|\textit{main}|}|\\
\end{tabular}
\end{center}
at the top of every child file \textit{child}
which is included by |\include{|\textit{child}|}|
from within the main file
(or at least for those files to be compiled individually).
The argument \textit{main} must be the filename of the main file.

There are a couple of
considerations in setting up the main and child documents:

%%%%%%%%%%%%%%%%%%%%%%%%%%%%%%%%%%%%%%%%
\paragraph{Restrictions.}

Please note the following restrictions:
\begin{itemize}
\item
|\childdocmain| must be called with one argument \textit{main}
to ensure compatibility with earlier version of the package.
It must either be empty (|\childdocmain{}|)
or precisely match the filename of the main file in which it is specified.
See \secref{sec:detection} for further information.
\item
The filename \textit{main} must be specified without the |.tex| extension.
\item
The filename \textit{main} is case sensitive
(even in case-insensitive file systems)
due to internal string comparison.
\item
The argument \textit{main} should be fully expanded, it cannot be a macro.
\item
Subdirectories and special characters should be avoided in filenames.
\item
The command |\childdocmain{|\textit{main}|}| must be followed by a whitespace.
It should not be followed immediately by another command
or by a comment mark `|%|'.
This is because the \TeX{} parser reads the token immediately following
the argument of |\childdocmain| and puts it
at the beginning of every child section;
however, a white\-space is ignored.
\end{itemize}

%%%%%%%%%%%%%%%%%%%%%%%%%%%%%%%%%%%%%%%%
\paragraph{Content of Main File.}

It is advisable to place all content in the child files included by |\include|.
Any output contained in the main file will appear in all child documents
unless suppressed manually;
it cannot be suppressed automatically by the |\includeonly| directive
and thus should normally be avoided.
A method to include some content in the main file
by means of conditional processing is described in \secref{sec:conditional}.

%%%%%%%%%%%%%%%%%%%%%%%%%%%%%%%%%%%%%%%%
\paragraph{Page Numbering.}

When only a part of the document is compiled,
the appropriate numbering of pages
(as well as other status parameters)
is determined from the |.aux| files.
The latter contain information from previous passes.
However this information needs to propagate through
all intermediate child documents.
Therefore the page numbering in child documents may well
be inconsistent until the complete document is compiled at least once.

A useful (if unconventional) way to always ensure a consistent
page numbering is to restart the numbering in each child document
and denote the pages by `\textit{child}|.|\textit{page}'
where \textit{child} represents the chapter/section number of the child file.
This can be achieved by the command
|\numberwithin{page}{|\textit{child}|}|
of the \textsf{amsmath} package
where \textit{child} can be |chapter| or |section|
depending on the chosen structuring.
Alternatively, one can modify the macro |\thepage| appropriately
and reset the counter |page| at the start of each child file.

%%%%%%%%%%%%%%%%%%%%%%%%%%%%%%%%%%%%%%%%%%%%%%%%%%%%%%%%%%%%%%%%%%%%%%%%%%%%%%%%
\subsection{Conditional Processing}
\label{sec:conditional}

The package provides a mechanism to compile different versions
of a document. To customise the versions further some conditional processing
can come in handy to distinguish which version is being compiled.
The package provides two macros to describe the compilation context:

%%%%%%%%%%%%%%%%%%%%%%%%%%%%%%%%%%%%%%%%
\DescribeMacro{\ifchilddoc}
The conditional |\ifchilddoc| distinguishes between the compilation of
child documents and the main document:
%
\begin{center}
|\ifchilddoc |\textit{child-code}| |[|\||else |\textit{main-code}]| \||fi|
\end{center}

%%%%%%%%%%%%%%%%%%%%%%%%%%%%%%%%%%%%%%%%
\DescribeMacro{\childdocname}
\DescribeMacro{\childdocjob}
The macro |\childdocname| contains the filename (without extension)
of the main or child file being processed.
Note that |\childdocjob| will always contain the name of the main file.

%%%%%%%%%%%%%%%%%%%%%%%%%%%%%%%%%%%%%%%%
\paragraph{Title Page.}

Conditional processing can be used to include a title or banner page
in the main document when proper precautions are taken.
Importantly, the code in the main file should ensure that the page counter
(as well as other status parameters which are stored in the |.aux| files)
takes the same value after the conditional processing.
Otherwise the page numbers may take divergent values
depending on which part is compiled.

For example, a title page could be declared by:
%
\begin{center}
\begin{tabular}{l}
|\ifchilddoc\||else|\\
|\addtocounter{page}{-1}|\\
\textit{code for title page}\\
|\newpage|\\
|\||fi|
\end{tabular}
\end{center}
%
A banner page for the child documents can be generated by:
%
\begin{center}
\begin{tabular}{l}
|\ifchilddoc|\\
|\addtocounter{page}{-1}|\\
\textit{code for banner page}\\
|\newpage|\\
|\||fi|
\end{tabular}
\end{center}
%
Here one could write a message such as:
\begin{center}
|This is the part \childdocname{} of \childdocjob{}.|
\end{center}

%%%%%%%%%%%%%%%%%%%%%%%%%%%%%%%%%%%%%%%%%%%%%%%%%%%%%%%%%%%%%%%%%%%%%%%%%%%%%%%%
\subsection{Flags}
\label{sec:flags}

The package makes it easy to generate different versions
of the main or child documents.
To this end compilation flags can be defined
and assigned different default values.
They will be particularly useful in conjunction
with the forwarding mechanism described in \secref{sec:forward}.

For example, it may be useful to have a flag |\version|
which can be set to |draft| or |final|.
The document source will contain some conditional code
depending on the value of |\version|.
Suppose further, the flag should default to |final| for the main file
and to |draft| for child files
which is a natural assignment for editing the document.
This is achieved by placing the following code
in the preamble of the main document
(below the |\childdocmain| directive):
%
\begin{center}
\begin{tabular}{l}
|\ifchilddoc|\\
|\providecommand{\version}{draft}|\\
|\||else|\\
|\providecommand{\version}{final}|\\
|\||fi|
\end{tabular}
\end{center}
%
The definition by |\providecommand| makes sure
that previous definitions are not overwritten.
Further statements |\providecommand{\version}{...}|
can thus be added before the above code to override it.

For the main file, one might add a line
(between |\childdocmain| and the above block)
%
\begin{center}
|%\ifchilddoc\||else\providecommand{\version}{draft}\||fi|
\end{center}
%
which can be uncommented to produce a draft version.
Likewise one can add a line to the very top of a child file
(above the |\childdocof{|\textit{main}|}| directive)
%
\begin{center}
|%\providecommand{\version}{final}|
\end{center}
%
which can be uncommented to produce the final version of this child document.

%%%%%%%%%%%%%%%%%%%%%%%%%%%%%%%%%%%%%%%%%%%%%%%%%%%%%%%%%%%%%%%%%%%%%%%%%%%%%%%%
\subsection{Forwarding}
\label{sec:forward}

Different versions of the main or child documents
using compilation flags as described in \secref{sec:flags}
can be (permanently) stored in different files
for convenient compilation, viewing and distribution.
To this end, the package defines a command
to pass on compilation to a different file:

%%%%%%%%%%%%%%%%%%%%%%%%%%%%%%%%%%%%%%%%
\DescribeMacro{\childdocforward}
The command |\childdocforward| redirects processing to
another source file:
%
\begin{center}
\begin{tabular}{l}
|\input{childdoc.def}|\\
|\childdocforward[|\textit{main}|]{|\textit{dest}|}|\\
\end{tabular}
\end{center}
%
The argument \textit{dest} is the destination file
(without extension).
It should be the main file or one of the child files.
Note that further \textsf{childdoc} directives
such as |\childdocof| and |\childdocforward|
in the indicated file will be processed in this form.
The optional argument \textit{main}
passes on directly to the main file \textit{main}
while pretending to compile the child \textit{dest}.
This form behaves as if \textit{dest}
issues |\childdocof{|\textit{main}|}| right away,
and no further \textsf{childdoc} directives will be processed.

%%%%%%%%%%%%%%%%%%%%%%%%%%%%%%%%%%%%%%%%
\DescribeMacro{\...prefix}
In the alternative form |\childdocforwardprefix|,
%
\begin{center}
\begin{tabular}{l}
|\input{childdoc.def}|\\
|\childdocforwardprefix[|\textit{main}|]{|\textit{prefix}|}{|\textit{dest}|}|
\end{tabular}
\end{center}
%
the destination file is determined by a pattern
depending on the current file:
To make this work, the current file must be called
`{\textit{prefix}\hspace{0.2em}\textit{suffix}}'
with \textit{prefix} matching precisely the argument.
Processing is then passed on to the file
`{\textit{dest}\hspace{0.2em}\textit{suffix}}'.
Surely, the same effect is achieved by
directly specifying the
argument `{\textit{dest}\hspace{0.2em}\textit{suffix}}'
in the first form.
However, that requires to set up a different file
for each child. With the alternative form of the command
all these files can have exactly the same content
which simplifies setting them up and maintaining them.

For example, the following file |draft.tex|
with a compilation flag |\version| as described in \secref{sec:flags}
compiles the main document as a draft:
%
\begin{center}
\begin{tabular}{l}
|\def\version{draft}|\\
|\input{childdoc.def}|\\
|\childdocforward{|\textit{main}|}|
\end{tabular}
\end{center}
%
Likewise, the following files |final|\textit{nn}|.tex|
compile the final version of the child document
|child|\textit{nn}|.tex|:
%
\begin{center}
\begin{tabular}{l}
|\def\version{final}|\\
|\input{childdoc.def}|\\
|\childdocforwardprefix{final}{child}|
\end{tabular}
\end{center}
%

Note that when several versions of a main file and/or of each child file
are to be generated, it may be convenient to set up a |Makefile| or
shell script to automatise the process.

%%%%%%%%%%%%%%%%%%%%%%%%%%%%%%%%%%%%%%%%%%%%%%%%%%%%%%%%%%%%%%%%%%%%%%%%%%%%%%%%
\subsection{Command Line Processing}
\label{sec:commandline}

The effect of redirection files can also be achieved by invoking
the \LaTeX{} compiler with a more elaborate command line.
Most conveniently this should be done as part
of a shell script or a |Makefile|.

When using \textsf{childdoc} in the main file, the following
command lines effectively perform a redirection
(note that depending on the shell being used,
backslashes may have to be doubled: `|\|' $\to$ `|\\|'):
%
\begin{center}
|... -jobname "|\textit{target}|" |\\|"|[\textit{flags}]%
|\input{childdoc.def}\childdocforward[|\textit{main}|]{|\textit{dest}|}"|
\end{center}
%
Here \textit{target} is the name of the output file,
\textit{main} is the name of the main file
and \textit{dest} is the name of the main or child file to be processed
(all filenames without extensions).
The optional argument \textit{main} can be omitted
if \textit{main} matches \textit{dest}.
Optionally, compilation \textit{flags} can be defined via |\def| commands.
This command line makes the \TeX{} engine believe
it is compiling the file \textit{target}
whose content is specified as the latter parameter.
The provided code then forwards the processing to
\textit{main} or \textit{dest} as described in \secref{sec:forward}.

%%%%%%%%%%%%%%%%%%%%%%%%%%%%%%%%%%%%%%%%%%%%%%%%%%%%%%%%%%%%%%%%%%%%%%%%%%%%%%%%
\subsection{Include by Input}
\label{sec:input}

Including child documents by |\include| has some restrictions by design.
Most notably, the content of a child document always occupies
its own set of pages; pages cannot be shared between child documents.
Usually, this behaviour makes perfect sense
because each child document contain an essential part of the document.
However, in some situations it may be desirable to compose
a document from a collection of parts
without having mandatory page breaks between then.
For this case, the package
provides a mechanism to include parts
by |\input| which can also be processed individually.
However, by construction this mechanism
requires manual handling of the content to be output.

%%%%%%%%%%%%%%%%%%%%%%%%%%%%%%%%%%%%%%%%
\DescribeMacro{\ifchilddocmanual}
The main file should be prepared as usual, see \secref{sec:include}.
However, the document body must make a distinction
between processing of an individual part and of the main document, e.g.:
%
\begin{center}
\begin{tabular}{l}
|\ifchilddocmanual|\\
|\input{\childdocname}|\\
|\||else|\\
\textit{document body with }|\input{|\textit{part}|}|\\
|\||fi|
\end{tabular}
\end{center}
%
The conditional |\ifchilddocmanual| is true whenever
a part to be included by |\input| is being compiled,
and the name of the part is stored in |\childdocname|.

%%%%%%%%%%%%%%%%%%%%%%%%%%%%%%%%%%%%%%%%
\DescribeMacro{\childdocby}
Each part to be included by |\input| should start with:
%
\begin{center}
\begin{tabular}{l}
|\input{childdoc.def}|\\
|\childdocby{|\textit{main}|}|\\
\end{tabular}
\end{center}
%
The directive |\childdocby| is similar to |\childdocof|
described in \secref{sec:include},
but the subsequent selection of content must be done manually.
To that end, both |\ifchilddoc| and |\ifchilddocmanual|
will be true upon processing of a part,
and the name of the part is stored in |\childdocname|.
Note that |\jobname| will be set to the filename of the current part
so that each part receives an individual |.aux| file
that does not interfere with the |.aux| file(s) of the main document.
This behaviour can be altered by the alternative form
|\childdocby[*]{|\textit{main}|}| (with a non-empty optional argument)
which uses the |.aux| file of the main document
by setting |\jobname| to \textit{main}.

%%%%%%%%%%%%%%%%%%%%%%%%%%%%%%%%%%%%%%%%%%%%%%%%%%%%%%%%%%%%%%%%%%%%%%%%%%%%%%%%
\subsection{Driver Development}
\label{sec:driver}

The \textsf{childdoc} mechanism can also be use for the development
of definition files such as \LaTeX{} styles or classes.
This case differs from the above setup with multiple parts
included by |\include| in that no |\includeonly| should be invoked.
This can be achieved by starting the include file
(before |\ProvidesPackage|) with:
%
\begin{center}
\begin{tabular}{l}
|\input{childdoc.def}|\\
|\childdocforward{|\textit{main}|}|\\
\end{tabular}
\end{center}
%
or alternatively with:
%
\begin{center}
\begin{tabular}{l}
|\input{childdoc.def}|\\
|\childdocby{|\textit{main}|}|\\
\end{tabular}
\end{center}
%
Both forms have slightly different effects as described above.
The main file is prepared as usual, see \secref{sec:include}.

%%%%%%%%%%%%%%%%%%%%%%%%%%%%%%%%%%%%%%%%%%%%%%%%%%%%%%%%%%%%%%%%%%%%%%%%%%%%%%%%
\subsection{Legacy Detection}
\label{sec:detection}

The directive |\childdocmain| in the main file can detect
whether the complete document or merely a child is to be compiled
even without using the directive |\childdocof|.
This method is deprecated because it is less robust
and there is no compelling reason to use it;
it is merely provided for backward compatibility
and it may be removed in future versions.

If the detection mechanism is to be used,
it is mandatory to correctly specify
the filename of the main file as the argument of |\childdocmain|:
%
\begin{center}
\begin{tabular}{l}
|\input{childdoc.def}|\\
|\childdocmain{|\textit{main}|}|\\
\end{tabular}
\end{center}
%
If |\jobname| does not match the argument \textit{main} of |\childdocmain|,
it is assumed that |\jobname| points to the child file to be compiled.
When using |\childdocmain| with the main file specified as argument,
it suffices to start a child file
with just |\input{|\textit{main}|}|
without loading of the package and using |\childdocof|.
If instead all processing is done
with the appropriate \textsf{childdoc} directives,
the argument of \textit{main} of |\childdocmain| can be empty.

An alternative version of the command line processing described
in \secref{sec:commandline} using the detection mechanism reads:
%
\begin{center}
|... -jobname "|\textit{target}|" "|[\textit{flags}]%
[|\def\jobname{|\textit{dest}|}|]|\input{|\textit{main}|}"|
\end{center}

%%%%%%%%%%%%%%%%%%%%%%%%%%%%%%%%%%%%%%%%%%%%%%%%%%%%%%%%%%%%%%%%%%%%%%%%%%%%%%%%
\subsection{Manual Code}
\label{sec:manual}

In case one cannot be certain whether the definitions file |childdoc.def|
is installed on the target \TeX{} distribution
and one prefers not to ship it,
it is conceivable to paste a few relevant commands into the sources.

To that end, drop all statements |\input{childdoc.def}|
and perform the replacements as outlined below.
Instead of |\childdocmain{|\textit{main}|}| add the following code
to the top of the main file:
%
\begin{center}
\begin{tabular}{l}
|\||ifdefined\childdocname\endinput\||fi\newif\ifchilddoc|\\
|\edef\childdocname{\scantokens\expandafter{\jobname\noexpand}}|\\
|\def\childdocmain{|\textit{main}|}\||ifx\childdocmain\childdocname\||else|\\
|\childdoctrue\includeonly{\childdocname}\let\jobname\childdocmain\||fi|\\
\end{tabular}
\end{center}
%
Instead of |\childdocof{|\textit{main}|}| just include the main file
at the top of each child file:
%
\begin{center}
|\input{|\textit{main}|}|
\end{center}
%
A simple redirection |\childdocforward{|\textit{dest}|}| is achieved by:
%
\begin{center}
|\def\jobname{|\textit{dest}|}\input{\jobname}|
\end{center}
%
The redirection with prefix
|\childdocforwardprefix[|\textit{prefix}|]{|\textit{dest}|}|
is accomplished by:
%
\begin{center}
\begin{tabular}{l}
|{\edef\jobname{\scantokens\expandafter{\jobname\noexpand}}|\\
|\def\redirectjob |\textit{prefix}|#1~~~{\gdef\jobname{|\textit{dest}|#1}}|\\
|\expandafter\redirectjob\jobname~~~}\input{\jobname}|
\end{tabular}
\end{center}

In an alternative approach,
child documents can be compiled by a specific command line
without additional code or specific definitions:
%
\begin{center}
|... -jobname "|\textit{target}|" "|[\textit{flags}]%
|\includeonly{|\textit{dest}|}\input{|\textit{main}|}"|
\end{center}
%

%%%%%%%%%%%%%%%%%%%%%%%%%%%%%%%%%%%%%%%%%%%%%%%%%%%%%%%%%%%%%%%%%%%%%%%%%%%%%%%%
%%%%%%%%%%%%%%%%%%%%%%%%%%%%%%%%%%%%%%%%%%%%%%%%%%%%%%%%%%%%%%%%%%%%%%%%%%%%%%%%
\section{Information}

%%%%%%%%%%%%%%%%%%%%%%%%%%%%%%%%%%%%%%%%%%%%%%%%%%%%%%%%%%%%%%%%%%%%%%%%%%%%%%%%
\subsection{Copyright}

Copyright \copyright{} 2017--2018 Niklas Beisert

This work may be distributed and/or modified under the
conditions of the \LaTeX{} Project Public License, either version 1.3
of this license or (at your option) any later version.
The latest version of this license is in
  \url{http://www.latex-project.org/lppl.txt}
and version 1.3 or later is part of all distributions of \LaTeX{}
version 2005/12/01 or later.

This work has the LPPL maintenance status `maintained'.

The Current Maintainer of this work is Niklas Beisert.

This work consists of the files |README.txt|, |childdoc.ins| and |childdoc.dtx|
as well as the derived files |childdoc.def|, |cdocsamp.tex|
with |cdocsch1.tex|, |cdocsch2.tex|, |cdocspt3.tex|, |cdocspt4.tex|,
|cdocsdrf.tex|, |cdocsfn1.tex|, |cdocsfn2.tex|
as well as |childdoc.pdf|.

%%%%%%%%%%%%%%%%%%%%%%%%%%%%%%%%%%%%%%%%%%%%%%%%%%%%%%%%%%%%%%%%%%%%%%%%%%%%%%%%
\subsection{Files and Installation}

The package consists of the files:
%
\begin{center}
\begin{tabular}{ll}
    |README.txt|   & readme file \\
    |childdoc.ins| & installation file \\
    |childdoc.dtx| & source file \\
    |childdoc.def| & definition file \\
    |cdocsamp.tex| & sample main file \\
    |cdocsch1.tex| & sample include file \\
    |cdocsch2.tex| & sample include file \\
    |cdocspt3.tex| & sample part file \\
    |cdocspt4.tex| & sample part file \\
    |cdocsdrf.tex| & sample redirection file \\
    |cdocsfn1.tex| & sample redirection file \\
    |cdocsfn2.tex| & sample redirection file \\
    |childdoc.pdf| & manual
\end{tabular}
\end{center}
%
The distribution consists of the files
|README.txt|, |childdoc.ins| and |childdoc.dtx|.
%
\begin{itemize}
\item
Run (pdf)\LaTeX{} on |childdoc.dtx|
to compile the manual |childdoc.pdf| (this file).
\item
Run \LaTeX{} on |childdoc.ins| to create the definitions file |childdoc.def|
and the sample |cdocsamp.tex| with include files
|cdocsch1.tex|, |cdocsch2.tex|, |cdocspt3.tex|, |cdocspt4.tex|,
|cdocsdrf.tex|, |cdocsfn1.tex|, |cdocsfn2.tex|.
Then copy the file |childdoc.def| to an appropriate directory of your \LaTeX{}
distribution, e.g.\ \textit{texmf-root}|/tex/latex/childdoc|.
\end{itemize}

%%%%%%%%%%%%%%%%%%%%%%%%%%%%%%%%%%%%%%%%%%%%%%%%%%%%%%%%%%%%%%%%%%%%%%%%%%%%%%%%
\subsection{Related CTAN Packages}

There are several other packages which offer a similar functionality:
%
\begin{itemize}
\item
The packages
\href{http://ctan.org/pkg/docmute}{\textsf{docmute}},
\href{http://ctan.org/pkg/includex}{\textsf{includex}} and
\href{http://ctan.org/pkg/standalone}{\textsf{standalone}}
provide commands to include only the document body of
a child file thus allowing both files to be compiled individually.
\item
The packages \href{http://ctan.org/pkg/subdocs}{\textsf{subdocs}}
and \href{http://ctan.org/pkg/subfiles}{\textsf{subfiles}}
provide structures in which the main and child documents can be
encapsulated and allowing them to be compiled individually.
The inclusion mechanism is different from the conventional |\include|.
\item
The package \href{http://ctan.org/pkg/combine}{\textsf{combine}}
is an elaborate solution to combine several documents into one.
\end{itemize}
%
See also the CTAN topic \href{http://ctan.org/topic/subdocs}{\textsf{subdocs}}
for further related packages.
The present package differs from the above solutions in that
a document structure constructed with the conventional |\include| mechanism
just needs two extra commands at the top of every file
such that all constituent files can be compiled individually.

%%%%%%%%%%%%%%%%%%%%%%%%%%%%%%%%%%%%%%%%%%%%%%%%%%%%%%%%%%%%%%%%%%%%%%%%%%%%%%%%
%\subsection{Feature Suggestions}
%
%The following is a list of features which may be useful for future
%versions of this package:
%%
%\begin{itemize}
%\item
%\ldots
%\end{itemize}

%%%%%%%%%%%%%%%%%%%%%%%%%%%%%%%%%%%%%%%%%%%%%%%%%%%%%%%%%%%%%%%%%%%%%%%%%%%%%%%%
\subsection{Revision History}

%%%%%%%%%%%%%%%%%%%%%%%%%%%%%%%%%%%%%%%%
\paragraph{v2.0:} 2018/12/30

\begin{itemize}
\item
immediate forward processing
\item
added |\childdocby| mechanism
\item
manual restructured
\end{itemize}

%%%%%%%%%%%%%%%%%%%%%%%%%%%%%%%%%%%%%%%%
\paragraph{v1.6:} 2018/01/17

\begin{itemize}
\item
application for development of include files
\item
corrections to manual
\end{itemize}

%%%%%%%%%%%%%%%%%%%%%%%%%%%%%%%%%%%%%%%%
\paragraph{v1.5:} 2017/05/21

\begin{itemize}
\item
more complete structuring introduced
\item
|\childdocof| introduced
\item
|\childdoc| renamed to |\childdocmain|
\item
|\childredirect| renamed to |\childdocforward| and |\childdocforwardprefix|
and functionality expanded
\end{itemize}

%%%%%%%%%%%%%%%%%%%%%%%%%%%%%%%%%%%%%%%%
\paragraph{v1.0:} 2017/04/27

\begin{itemize}
\item
manual and install package
\item
first version published on CTAN
\end{itemize}

%%%%%%%%%%%%%%%%%%%%%%%%%%%%%%%%%%%%%%%%
\paragraph{v0.6:} 2017/04/26

\begin{itemize}
\item
redirection mechanism added
\end{itemize}

%%%%%%%%%%%%%%%%%%%%%%%%%%%%%%%%%%%%%%%%
\paragraph{v0.5:} 2017/04/26

\begin{itemize}
\item
functionality in definition file
\end{itemize}


%%%%%%%%%%%%%%%%%%%%%%%%%%%%%%%%%%%%%%%%%%%%%%%%%%%%%%%%%%%%%%%%%%%%%%%%%%%%%%%%
%%%%%%%%%%%%%%%%%%%%%%%%%%%%%%%%%%%%%%%%%%%%%%%%%%%%%%%%%%%%%%%%%%%%%%%%%%%%%%%%
%%%%%%%%%%%%%%%%%%%%%%%%%%%%%%%%%%%%%%%%%%%%%%%%%%%%%%%%%%%%%%%%%%%%%%%%%%%%%%%%
\appendix

\settowidth\MacroIndent{\rmfamily\scriptsize 000\ }

 \DocInput{childdoc.dtx}

\end{document}
%</driver>
% \fi
%
% %%%%%%%%%%%%%%%%%%%%%%%%%%%%%%%%%%%%%%%%%%%%%%%%%%%%%%%%%%%%%%%%%%%%%%%%%%%%%%
% %%%%%%%%%%%%%%%%%%%%%%%%%%%%%%%%%%%%%%%%%%%%%%%%%%%%%%%%%%%%%%%%%%%%%%%%%%%%%%
% \section{Sample}
%\iffalse
%<*samplemain>
%\fi
%
% The following presents a sample document
% with two chapters, two parts, a title page,
% a compile flag as well as three forwarding files to set the flag.
% It consists of eight |.tex| files:
% \begin{center}
% \begin{tabular}{ll}
% |cdocsamp.tex|&main file\\
% |cdocsch1.tex|&include file for chapter 1\\
% |cdocsch2.tex|&include file for chapter 2\\
% |cdocspt3.tex|&include file for part 3\\
% |cdocspt4.tex|&include file for part 4\\
% |cdocsdrf.tex|&forwarding file for main file in draft mode\\
% |cdocsfi1.tex|&forwarding file for final version of chapter 1\\
% |cdocsfi2.tex|&forwarding file for final version of chapter 2\\
% \end{tabular}
% \end{center}
% Each of the eight files can be compiled directly by the \LaTeX{} compiler.
%
% %%%%%%%%%%%%%%%%%%%%%%%%%%%%%%%%%%%%%%
% \paragraph{Main File.}
%
% The main file is called |cdocsamp.tex|.
%
% Load the \textsf{childdoc} definitions and
% declare the filename for the main document:
%    \begin{macrocode}
\input{childdoc.def}
\childdocmain{}
%    \end{macrocode}

% Optional override for |\version| flag:
%    \begin{macrocode}
%%\ifchilddoc\else\providecommand{\version}{draft}\fi
%    \end{macrocode}

% Define the default values for the |\version| flag
% (|final| for the main file and |draft| for childs):
%    \begin{macrocode}
\ifchilddoc
\providecommand{\version}{draft}
\else
\providecommand{\version}{final}
\fi
%    \end{macrocode}

% Load the standard document class:
%    \begin{macrocode}
\documentclass[12pt]{article}
%    \end{macrocode}

% Start the document body:
%    \begin{macrocode}
\begin{document}
%    \end{macrocode}

% Declare a title page.
% Print title, part of document being processed and version flag:
%    \begin{macrocode}
\addtocounter{page}{-1}
\begin{center}
{\LARGE\bfseries{}childdoc example\par}
\vspace{1cm}
\ifchilddoc
\ifchilddocmanual part\else chapter\fi:
`\childdocname' of `\childdocjob'\par
\else
main document: `\childdocjob'\par
\fi
version: \version\par
\end{center}
\newpage
%    \end{macrocode}

% Manually include selected file,
% otherwise process as usual:
%    \begin{macrocode}
\ifchilddocmanual
\section*{part `\childdocname'}
\input{\childdocname}
\else
%    \end{macrocode}

% Include the two chapters:
%    \begin{macrocode}
\include{cdocsch1}
\include{cdocsch2}
%    \end{macrocode}

% Include the two parts unless only chapters should be displayed:
%    \begin{macrocode}
\ifchilddoc\else
\section{part three}
\input{cdocspt3}
\section{part four}
\input{cdocspt4}
\fi
%    \end{macrocode}

% Process as usual until here:
%    \begin{macrocode}
\fi
%    \end{macrocode}

% End of document body:
%    \begin{macrocode}
\end{document}
%    \end{macrocode}
%\iffalse
%</samplemain>
%\fi
%
% %%%%%%%%%%%%%%%%%%%%%%%%%%%%%%%%%%%%%%
% \paragraph{Chapter Include Files.}
%
% The include files are called |cdocsch1.tex| and |cdocsch2.tex|.
%
%\iffalse
%<*samplechap1|samplechap2>
%\fi

% Optional override for |\version| flag:
%    \begin{macrocode}
%%\providecommand{\version}{final}
%    \end{macrocode}

% Include the main document:
%    \begin{macrocode}
\input{childdoc.def}
\childdocof{cdocsamp}
%    \end{macrocode}

%\iffalse
%</samplechap1|samplechap2>
%\fi
%
%\iffalse
%<*samplechap1>
%\fi
% Some text for chapter 1:
%    \begin{macrocode}
\section{one}
some text in chapter one
%    \end{macrocode}

%\iffalse
%</samplechap1>
%\fi
% Some text for chapter 2:
%\iffalse
%<*samplechap2>
%\fi
%    \begin{macrocode}
\section{two}
more text in chapter two
%    \end{macrocode}

%\iffalse
%</samplechap2>
%\fi
%
% %%%%%%%%%%%%%%%%%%%%%%%%%%%%%%%%%%%%%%
% \paragraph{Part Include Files.}
%
% The include files are called |cdocspt3.tex| and |cdocspt4.tex|.
%
%\iffalse
%<*samplepart3|samplepart4>
%\fi

% Optional override for |\version| flag:
%    \begin{macrocode}
%%\providecommand{\version}{final}
%    \end{macrocode}

% Include the main document:
%    \begin{macrocode}
\input{childdoc.def}
\childdocby{cdocsamp}
%    \end{macrocode}

%\iffalse
%</samplepart3|samplepart4>
%\fi
%
%\iffalse
%<*samplepart3>
%\fi
% Some text for part 3:
%    \begin{macrocode}
some text in part three
%    \end{macrocode}

%\iffalse
%</samplepart3>
%\fi
% Some text for part 4:
%\iffalse
%<*samplepart4>
%\fi
%    \begin{macrocode}
more text in part four
%    \end{macrocode}

%\iffalse
%</samplepart4>
%\fi
%
% %%%%%%%%%%%%%%%%%%%%%%%%%%%%%%%%%%%%%%
% \paragraph{Forwarding for a Complete Draft.}
%
% The following forwarding file |cdocsdrf.tex|
% compiles the main document in draft mode:
%\iffalse
%<*sampledraft>
%\fi
%    \begin{macrocode}
\def\version{draft}
\input{childdoc.def}
\childdocforward{cdocsamp}
%    \end{macrocode}

%\iffalse
%</sampledraft>
%\fi
%
% %%%%%%%%%%%%%%%%%%%%%%%%%%%%%%%%%%%%%%
% \paragraph{Forwarding for Final Version of the Chapters.}
%
% The following forwarding files |cdocsfn1.tex| and |cdocsfn2.tex|
% (with identical content)
% compile the final versions of the child documents
% |cdocsch1.tex| and |cdocsch2.tex|, respectively:
%\iffalse
%<*samplefinal>
%\fi
%    \begin{macrocode}
\def\version{final}
\input{childdoc.def}
\childdocforwardprefix[cdocsamp]{cdocsfn}{cdocsch}
%    \end{macrocode}

%\iffalse
%</samplefinal>
%\fi
%
% %%%%%%%%%%%%%%%%%%%%%%%%%%%%%%%%%%%%%%
% \paragraph{Command Line Processing.}
%
% The following three command lines generate the output files
% |cdocscld|, |cdocscl1| and |cdocscl2|
% which should be identical to
% |cdocsdrf|, |cdocsch1| and |cdocsfn2|, respectively:
% \begin{center}
% \begin{tabular}{l}
% |latex -jobname cdocscld \|\\
% |  "\def\version{draft}\input{childdoc.def}\childdocforward{cdocsamp}"|\\
% |latex -jobname cdocscl1 \|\\
% |  "\input{childdoc.def}\childdocforward[cdocsamp]{cdocsch1}"|\\
% |latex -jobname cdocscl2 \|\\
% |  "\def\version{final}\input{childdoc.def}\childdocforward{cdocsch2}"|
% \end{tabular}
% \end{center}
% Note that the trailing backslash on each first line
% merely continues the input to the second line
% (for convenient cut ant paste).
% Furthermore, the command |latex| can be replaced by any
% of its alternative versions such as |pdflatex|.
%
% %%%%%%%%%%%%%%%%%%%%%%%%%%%%%%%%%%%%%%%%%%%%%%%%%%%%%%%%%%%%%%%%%%%%%%%%%%%%%%
% %%%%%%%%%%%%%%%%%%%%%%%%%%%%%%%%%%%%%%%%%%%%%%%%%%%%%%%%%%%%%%%%%%%%%%%%%%%%%%
% \section{Implementation}
%\iffalse
%<*package>
%\fi
%
% This section describes the definitions file |childdoc.def|.

% The definitions cannot be loaded using |\usepackage| or |\RequirePackage|
% which has a mechanism to prevent loading a style file more than once.
% When loading the definitions by means of |\input|
% multiple instances have to be prevented manually:
%\iffalse
%This code needs to be before the `\ProvidesFile' directive
%which is defined at the beginning of this file.
%Therefore it is also placed there and commented out here.
%</package>
%<*discard>
%\fi
%    \begin{macrocode}
\ifdefined\childdocmain\endinput\fi
%    \end{macrocode}
%\iffalse
%</discard>
%<*package>
%\fi
%
% \macro{\ifchilddoc}
% \macro{\ifchilddocmanual}
% The conditional |\ifchilddoc| tells whether a
% child (true) or main (false) document is being compiled.
% The conditional |\ifchilddocmanual| tells whether
% the |\includeonly| mechanism is used (false) or
% the selection of child files must be performed manually (true).
% The definitions initialise to false:
%    \begin{macrocode}
\newif\ifchilddoc
\newif\ifchilddocmanual
%    \end{macrocode}

% \macro{\childdocname}
% \macro{\childdocjob}
% The macro |\childdocname| stores the name of the main document
% to be compiled. The macro |\childdocjob| stores the name of
% the document on which the \LaTeX{} compiler was originally invoked.
% The content of |\jobname| cannot be compared
% to filenames specified in the source due to different catcodes.
% The following code rescans |\jobname|, stores the result
% in |\childdocname| and saves a copy in |\childdocjob|:
%    \begin{macrocode}
\edef\childdocname{\scantokens\expandafter{\jobname\noexpand}}
\let\childdocjob\childdocname
%    \end{macrocode}

% \macro{\childdocdisable}
% The macro |\childdocdisable| prevents the main file
% from being processed more than once.
% At this stage, the main document command |\childdocmain|
% is assumed to be called once again where it should do nothing.
% Any subsequent call to it should prevent
% a secondary processing of the main document
% It overwrites the forwarding commands
% |\childdocof| and |\childdocforward|
% with empty macros to prevent further inclusions of the main document:
%    \begin{macrocode}
\newcommand{\childdocdisable}
{
  \renewcommand{\childdocmain}[1]{\renewcommand{\childdocmain}[1]{\endinput}}
  \renewcommand{\childdocof}[1]{}
  \renewcommand{\childdocby}[2][]{}
  \renewcommand{\childdocforward}[2][]{}
  \renewcommand{\childdocdisable}{}
}
%    \end{macrocode}

% \macro{\childdocmain}
% The macro |\childdocmain| is to be called at the top of the main file
% with nothing or the main filename (without extension) as argument.
% First, it breaks loops.
% If the argument is not empty and does not match |\childdocname|
% (which is set by the first inclusion of |childdoc.def|),
% |\ifchilddoc| is set to true, |\includeonly| is applied to the child file
% and |\jobname| is set to the main file
% (for proper handling of |.aux| files):
%    \begin{macrocode}
\newcommand{\childdocmain}[1]
{
  \childdocdisable\childdocmain{}
  \if?#1?\else
    \begingroup
      \def\childdoctmp{#1}
      \ifx\childdoctmp\childdocname
        \def\childdoctmp{}
      \else
        \def\childdoctmp
        {
          \childdoctrue
          \includeonly{\childdocname}
          \def\childdocjob{#1}
          \def\jobname{#1}
        }
      \fi
      \expandafter
    \endgroup
    \childdoctmp
  \fi
}
%    \end{macrocode}

% \macro{\childdocof}
% The command |\childdocof| redirects
% compilation to the main file |#1|.
%    \begin{macrocode}
\newcommand{\childdocof}[1]
{
  \childdocdisable
  \childdoctrue
  \includeonly{\childdocname}
  \def\jobname{#1}
  \def\childdocjob{#1}
  \input{#1}
}
%    \end{macrocode}

% \macro{\childdocby}
% The command |\childdocby| ....
%    \begin{macrocode}
\newcommand{\childdocby}[2][]
{
  \childdocdisable
  \childdoctrue
  \childdocmanualtrue
  \if?#1?\else
    \def\jobname{#2}
  \fi
  \def\childdocjob{#2}
  \input{#2}
  \endinput
}
%    \end{macrocode}

% \macro{\childdocforward}
% The command |\childdocforward| redirects
% compilation to the main file or
% (if the optional argument is given) a child file.
% Parameters are set as if the main file
% or a child file starting with |\childdocof| was compiled.
% Then compilation is handed over to the main file:
%    \begin{macrocode}
\newcommand{\childdocforward}[2][]
{
  \begingroup
    \if?#1?
      \def\childdoctmp
      {
        \def\childdocname{#2}
        \def\childdocjob{#2}
        \def\jobname{#2}
        \input{#2}
        \endinput
      }
    \else
      \def\childdoctmp
      {
        \childdocdisable
        \def\childdocname{#2}
        \childdoctrue
        \includeonly{#2}
        \def\childdocjob{#1}
        \def\jobname{#1}
        \input{#1}
        \endinput
      }
    \fi
    \expandafter
  \endgroup
  \childdoctmp
}
%    \end{macrocode}

% \macro{\childdocforwardprefix}
% The command |\childdocforwardprefix| redirects
% compilation to the main or a child file by means of a pattern.
% The prefix |#1| in the current filename is replaced by |#2|
% and the suffix of the current filename is kept
% (it is assumed that the filename does not contain the substring `|~~~|'
% which is used as a delimiter).
% Compilation is handed over to the new file by |\childdocforward|:
%    \begin{macrocode}
\newcommand{\childdocforwardprefix}[3][]
{
  \begingroup
    \def\childdocextract #2##1~~~{\def\childdoctmp{\childdocforward[#1]{#3##1}}}
    \expandafter\childdocextract\childdocname~~~
    \expandafter
  \endgroup
  \childdoctmp
}
%    \end{macrocode}

% \macro{\childdoc}
% The deprecated macro |\childdoc| is a legacy version of |\childdocmain|:
%    \begin{macrocode}
\newcommand{\childdoc}{\childdocmain}
%    \end{macrocode}

% \macro{\childdocredirect}
% The deprecated macro |\childdocredirect| is a legacy version
% of |\childdocforward| and |\childdocforwardprefix|:
%    \begin{macrocode}
\newcommand{\childdocredirect}[2][]
{
  \begingroup
    \if?#1?
      \def\childdoctmp{\childdocforward{#2}}
    \else
      \def\childdoctmp{\childdocforwardprefix{#1}{#2}}
    \fi
    \expandafter
  \endgroup
  \childdoctmp
}
%    \end{macrocode}

%\iffalse
%</package>
%\fi
%
\endinput
\childdocforward[|\textit{main}|]{|\textit{dest}|}"|
\end{center}
%
Here \textit{target} is the name of the output file,
\textit{main} is the name of the main file
and \textit{dest} is the name of the main or child file to be processed
(all filenames without extensions).
The optional argument \textit{main} can be omitted
if \textit{main} matches \textit{dest}.
Optionally, compilation \textit{flags} can be defined via |\def| commands.
This command line makes the \TeX{} engine believe
it is compiling the file \textit{target}
whose content is specified as the latter parameter.
The provided code then forwards the processing to
\textit{main} or \textit{dest} as described in \secref{sec:forward}.

%%%%%%%%%%%%%%%%%%%%%%%%%%%%%%%%%%%%%%%%%%%%%%%%%%%%%%%%%%%%%%%%%%%%%%%%%%%%%%%%
\subsection{Include by Input}
\label{sec:input}

Including child documents by |\include| has some restrictions by design.
Most notably, the content of a child document always occupies
its own set of pages; pages cannot be shared between child documents.
Usually, this behaviour makes perfect sense
because each child document contain an essential part of the document.
However, in some situations it may be desirable to compose
a document from a collection of parts
without having mandatory page breaks between then.
For this case, the package
provides a mechanism to include parts
by |\input| which can also be processed individually.
However, by construction this mechanism
requires manual handling of the content to be output.

%%%%%%%%%%%%%%%%%%%%%%%%%%%%%%%%%%%%%%%%
\DescribeMacro{\ifchilddocmanual}
The main file should be prepared as usual, see \secref{sec:include}.
However, the document body must make a distinction
between processing of an individual part and of the main document, e.g.:
%
\begin{center}
\begin{tabular}{l}
|\ifchilddocmanual|\\
|\input{\childdocname}|\\
|\||else|\\
\textit{document body with }|\input{|\textit{part}|}|\\
|\||fi|
\end{tabular}
\end{center}
%
The conditional |\ifchilddocmanual| is true whenever
a part to be included by |\input| is being compiled,
and the name of the part is stored in |\childdocname|.

%%%%%%%%%%%%%%%%%%%%%%%%%%%%%%%%%%%%%%%%
\DescribeMacro{\childdocby}
Each part to be included by |\input| should start with:
%
\begin{center}
\begin{tabular}{l}
|% \iffalse
%
% childdoc.dtx Copyright (C) 2017-2018 Niklas Beisert
%
% This work may be distributed and/or modified under the
% conditions of the LaTeX Project Public License, either version 1.3
% of this license or (at your option) any later version.
% The latest version of this license is in
%   http://www.latex-project.org/lppl.txt
% and version 1.3 or later is part of all distributions of LaTeX
% version 2005/12/01 or later.
%
% This work has the LPPL maintenance status `maintained'.
%
% The Current Maintainer of this work is Niklas Beisert.
%
% This work consists of the files childdoc.dtx and childdoc.ins
% and the derived files childdoc.def and cdocsamp.tex with
% cdocsch1.tex, cdocsch2.tex, cdocsdrf.tex, cdocsfn1.tex, cdocsfn2.tex.
%
%<package>\ifdefined\childdocmain\endinput\fi
%<package>\ProvidesFile{childdoc.def}[2018/12/30 v2.0 child document driver]
%<samplemain>\ProvidesFile{cdocsamp.tex}[2018/12/30 v2.0 sample for childdoc]
%<*driver>
%\ProvidesFile{childdoc.drv}[2018/12/30 v2.0 childdoc reference manual file]
\PassOptionsToClass{10pt,a4paper}{article}
\documentclass{ltxdoc}

\usepackage[margin=35mm]{geometry}
\usepackage{hyperref}
\usepackage{hyperxmp}
\usepackage[usenames]{color}

\hypersetup{colorlinks=true}
\hypersetup{pdfstartview=FitH}
\hypersetup{pdfpagemode=UseNone}
\hypersetup{pdfsource={}}
\hypersetup{pdflang={en-UK}}
\hypersetup{pdfcopyright={Copyright 2017-2018 Niklas Beisert.
  This work may be distributed and/or modified under the
  conditions of the LaTeX Project Public License, either version 1.3
  of this license or (at your option) any later version.}}
\hypersetup{pdflicenseurl={http://www.latex-project.org/lppl.txt}}
\hypersetup{pdfcontactaddress={ETH Zurich, ITP, HIT K,
  Wolfgang-Pauli-Strasse 27}}
\hypersetup{pdfcontactpostcode={8093}}
\hypersetup{pdfcontactcity={Zurich}}
\hypersetup{pdfcontactcountry={Switzerland}}
\hypersetup{pdfcontactemail={nbeisert@itp.phys.ethz.ch}}
\hypersetup{pdfcontacturl={http://people.phys.ethz.ch/\xmptilde nbeisert/}}

\newcommand{\secref}[1]{\hyperref[#1]{section \ref*{#1}}}

\parskip1ex
\parindent0pt
\let\olditemize\itemize
\def\itemize{\olditemize\parskip0pt}

\begin{document}

\title{The \textsf{childdoc} Package}
\hypersetup{pdftitle={The childdoc Package}}
\author{Niklas Beisert\\[2ex]
  Institut f\"ur Theoretische Physik\\
  Eidgen\"ossische Technische Hochschule Z\"urich\\
  Wolfgang-Pauli-Strasse 27, 8093 Z\"urich, Switzerland\\[1ex]
  \href{mailto:nbeisert@itp.phys.ethz.ch}
  {\texttt{nbeisert@itp.phys.ethz.ch}}}
\hypersetup{pdfauthor={Niklas Beisert}}
\hypersetup{pdfsubject={Manual for the LaTeX2e Package childdoc}}
\date{30 December 2018, \textsf{v2.0}}
\maketitle

\begin{abstract}\noindent
\textsf{childdoc} is a \LaTeXe{} package
that enables the direct compilation
of document sections included by |\include|
to individual files.
\end{abstract}

\begingroup
\parskip0ex
\tableofcontents
\endgroup

%%%%%%%%%%%%%%%%%%%%%%%%%%%%%%%%%%%%%%%%%%%%%%%%%%%%%%%%%%%%%%%%%%%%%%%%%%%%%%%%
%%%%%%%%%%%%%%%%%%%%%%%%%%%%%%%%%%%%%%%%%%%%%%%%%%%%%%%%%%%%%%%%%%%%%%%%%%%%%%%%
\section{Introduction}

\LaTeX{} provides a mechanism to structure a large document (such as a book)
into a main file and several child files (containing the chapters)
using the |\include| command.
This mechanism is beneficial for documents
which span hundreds of pages in order to
make the source file(s) more manageable.
Moreover, compilation can be restricted to
selected child files by means of the |\includeonly| command.
The latter feature can be used to reduce the compilation time while editing
(this was significantly more useful in the earlier days of \LaTeX{})
or to generate a smaller document which is easier to navigate.
Another application of |\includeonly| is to generate
documents consisting of selected parts of the complete document.

However, there are a few drawbacks of the plain |\include| mechanism:
\begin{itemize}
\item
The child files cannot be compiled on their own,
they can only be compiled via the main file.
A naive editing environment
(such as a text editor with an option
to have the current file processed by \LaTeX)
may require one to switch to the main file before compiling;
attempting to compile the child file produces errors.
\item
The main file must be modified (each time)
to adjust the |\includeonly| command
to the present needs. This easily leaves the main file in a messy state.
\item
The generated document will always carry the filename
of the main document. This is inconvenient if
several child files are to be compiled and
to be kept for distribution.
\end{itemize}

The present package provides a simple interface
to make child files individually compilable by \LaTeX{}.
Compiling a child file then has the same effect as compiling
the main file with an |\includeonly| command
to select the appropriate child.
Moreover the generated document will carry the name of the child
rather than the main file.
This resolves all three above issues.

This feature is meant to make the editing of books,
thesis documents and lecture notes somewhat more convenient.
However, the package can also be used efficiently for
composing a series of documents (such as exercise sheets)
which are typically distributed individually.
It then assists the author in generating the individual documents
(potentially in different versions)
as well as a document containing the collected series.
Another application is in developing style files
or other kinds of included material
where compilation of the style file could redirect
to a sample or test file.

%%%%%%%%%%%%%%%%%%%%%%%%%%%%%%%%%%%%%%%%%%%%%%%%%%%%%%%%%%%%%%%%%%%%%%%%%%%%%%%%
%%%%%%%%%%%%%%%%%%%%%%%%%%%%%%%%%%%%%%%%%%%%%%%%%%%%%%%%%%%%%%%%%%%%%%%%%%%%%%%%
\section{Usage}

First of all, the package \textsf{childdoc} is \emph{not} a standard
\LaTeXe{} |.sty| style file! Therefore it needs to be invoked in
a non-standard way.

%%%%%%%%%%%%%%%%%%%%%%%%%%%%%%%%%%%%%%%%%%%%%%%%%%%%%%%%%%%%%%%%%%%%%%%%%%%%%%%%
\subsection{Included Files}
\label{sec:include}

%%%%%%%%%%%%%%%%%%%%%%%%%%%%%%%%%%%%%%%%
\DescribeMacro{\childdocmain}
To use the package, add the commands
\begin{center}
\begin{tabular}{l}
|\input{childdoc.def}|\\
|\childdocmain{}|\\
\end{tabular}
\end{center}
at the very top of the main \LaTeX{} file,
in particular \emph{before} the |\documentclass| statement!
The argument of |\childdocmain| should be left empty
(but it must be present).

%%%%%%%%%%%%%%%%%%%%%%%%%%%%%%%%%%%%%%%%
\DescribeMacro{\childdocof}
Furthermore, add the commands
\begin{center}
\begin{tabular}{l}
|\input{childdoc.def}|\\
|\childdocof{|\textit{main}|}|\\
\end{tabular}
\end{center}
at the top of every child file \textit{child}
which is included by |\include{|\textit{child}|}|
from within the main file
(or at least for those files to be compiled individually).
The argument \textit{main} must be the filename of the main file.

There are a couple of
considerations in setting up the main and child documents:

%%%%%%%%%%%%%%%%%%%%%%%%%%%%%%%%%%%%%%%%
\paragraph{Restrictions.}

Please note the following restrictions:
\begin{itemize}
\item
|\childdocmain| must be called with one argument \textit{main}
to ensure compatibility with earlier version of the package.
It must either be empty (|\childdocmain{}|)
or precisely match the filename of the main file in which it is specified.
See \secref{sec:detection} for further information.
\item
The filename \textit{main} must be specified without the |.tex| extension.
\item
The filename \textit{main} is case sensitive
(even in case-insensitive file systems)
due to internal string comparison.
\item
The argument \textit{main} should be fully expanded, it cannot be a macro.
\item
Subdirectories and special characters should be avoided in filenames.
\item
The command |\childdocmain{|\textit{main}|}| must be followed by a whitespace.
It should not be followed immediately by another command
or by a comment mark `|%|'.
This is because the \TeX{} parser reads the token immediately following
the argument of |\childdocmain| and puts it
at the beginning of every child section;
however, a white\-space is ignored.
\end{itemize}

%%%%%%%%%%%%%%%%%%%%%%%%%%%%%%%%%%%%%%%%
\paragraph{Content of Main File.}

It is advisable to place all content in the child files included by |\include|.
Any output contained in the main file will appear in all child documents
unless suppressed manually;
it cannot be suppressed automatically by the |\includeonly| directive
and thus should normally be avoided.
A method to include some content in the main file
by means of conditional processing is described in \secref{sec:conditional}.

%%%%%%%%%%%%%%%%%%%%%%%%%%%%%%%%%%%%%%%%
\paragraph{Page Numbering.}

When only a part of the document is compiled,
the appropriate numbering of pages
(as well as other status parameters)
is determined from the |.aux| files.
The latter contain information from previous passes.
However this information needs to propagate through
all intermediate child documents.
Therefore the page numbering in child documents may well
be inconsistent until the complete document is compiled at least once.

A useful (if unconventional) way to always ensure a consistent
page numbering is to restart the numbering in each child document
and denote the pages by `\textit{child}|.|\textit{page}'
where \textit{child} represents the chapter/section number of the child file.
This can be achieved by the command
|\numberwithin{page}{|\textit{child}|}|
of the \textsf{amsmath} package
where \textit{child} can be |chapter| or |section|
depending on the chosen structuring.
Alternatively, one can modify the macro |\thepage| appropriately
and reset the counter |page| at the start of each child file.

%%%%%%%%%%%%%%%%%%%%%%%%%%%%%%%%%%%%%%%%%%%%%%%%%%%%%%%%%%%%%%%%%%%%%%%%%%%%%%%%
\subsection{Conditional Processing}
\label{sec:conditional}

The package provides a mechanism to compile different versions
of a document. To customise the versions further some conditional processing
can come in handy to distinguish which version is being compiled.
The package provides two macros to describe the compilation context:

%%%%%%%%%%%%%%%%%%%%%%%%%%%%%%%%%%%%%%%%
\DescribeMacro{\ifchilddoc}
The conditional |\ifchilddoc| distinguishes between the compilation of
child documents and the main document:
%
\begin{center}
|\ifchilddoc |\textit{child-code}| |[|\||else |\textit{main-code}]| \||fi|
\end{center}

%%%%%%%%%%%%%%%%%%%%%%%%%%%%%%%%%%%%%%%%
\DescribeMacro{\childdocname}
\DescribeMacro{\childdocjob}
The macro |\childdocname| contains the filename (without extension)
of the main or child file being processed.
Note that |\childdocjob| will always contain the name of the main file.

%%%%%%%%%%%%%%%%%%%%%%%%%%%%%%%%%%%%%%%%
\paragraph{Title Page.}

Conditional processing can be used to include a title or banner page
in the main document when proper precautions are taken.
Importantly, the code in the main file should ensure that the page counter
(as well as other status parameters which are stored in the |.aux| files)
takes the same value after the conditional processing.
Otherwise the page numbers may take divergent values
depending on which part is compiled.

For example, a title page could be declared by:
%
\begin{center}
\begin{tabular}{l}
|\ifchilddoc\||else|\\
|\addtocounter{page}{-1}|\\
\textit{code for title page}\\
|\newpage|\\
|\||fi|
\end{tabular}
\end{center}
%
A banner page for the child documents can be generated by:
%
\begin{center}
\begin{tabular}{l}
|\ifchilddoc|\\
|\addtocounter{page}{-1}|\\
\textit{code for banner page}\\
|\newpage|\\
|\||fi|
\end{tabular}
\end{center}
%
Here one could write a message such as:
\begin{center}
|This is the part \childdocname{} of \childdocjob{}.|
\end{center}

%%%%%%%%%%%%%%%%%%%%%%%%%%%%%%%%%%%%%%%%%%%%%%%%%%%%%%%%%%%%%%%%%%%%%%%%%%%%%%%%
\subsection{Flags}
\label{sec:flags}

The package makes it easy to generate different versions
of the main or child documents.
To this end compilation flags can be defined
and assigned different default values.
They will be particularly useful in conjunction
with the forwarding mechanism described in \secref{sec:forward}.

For example, it may be useful to have a flag |\version|
which can be set to |draft| or |final|.
The document source will contain some conditional code
depending on the value of |\version|.
Suppose further, the flag should default to |final| for the main file
and to |draft| for child files
which is a natural assignment for editing the document.
This is achieved by placing the following code
in the preamble of the main document
(below the |\childdocmain| directive):
%
\begin{center}
\begin{tabular}{l}
|\ifchilddoc|\\
|\providecommand{\version}{draft}|\\
|\||else|\\
|\providecommand{\version}{final}|\\
|\||fi|
\end{tabular}
\end{center}
%
The definition by |\providecommand| makes sure
that previous definitions are not overwritten.
Further statements |\providecommand{\version}{...}|
can thus be added before the above code to override it.

For the main file, one might add a line
(between |\childdocmain| and the above block)
%
\begin{center}
|%\ifchilddoc\||else\providecommand{\version}{draft}\||fi|
\end{center}
%
which can be uncommented to produce a draft version.
Likewise one can add a line to the very top of a child file
(above the |\childdocof{|\textit{main}|}| directive)
%
\begin{center}
|%\providecommand{\version}{final}|
\end{center}
%
which can be uncommented to produce the final version of this child document.

%%%%%%%%%%%%%%%%%%%%%%%%%%%%%%%%%%%%%%%%%%%%%%%%%%%%%%%%%%%%%%%%%%%%%%%%%%%%%%%%
\subsection{Forwarding}
\label{sec:forward}

Different versions of the main or child documents
using compilation flags as described in \secref{sec:flags}
can be (permanently) stored in different files
for convenient compilation, viewing and distribution.
To this end, the package defines a command
to pass on compilation to a different file:

%%%%%%%%%%%%%%%%%%%%%%%%%%%%%%%%%%%%%%%%
\DescribeMacro{\childdocforward}
The command |\childdocforward| redirects processing to
another source file:
%
\begin{center}
\begin{tabular}{l}
|\input{childdoc.def}|\\
|\childdocforward[|\textit{main}|]{|\textit{dest}|}|\\
\end{tabular}
\end{center}
%
The argument \textit{dest} is the destination file
(without extension).
It should be the main file or one of the child files.
Note that further \textsf{childdoc} directives
such as |\childdocof| and |\childdocforward|
in the indicated file will be processed in this form.
The optional argument \textit{main}
passes on directly to the main file \textit{main}
while pretending to compile the child \textit{dest}.
This form behaves as if \textit{dest}
issues |\childdocof{|\textit{main}|}| right away,
and no further \textsf{childdoc} directives will be processed.

%%%%%%%%%%%%%%%%%%%%%%%%%%%%%%%%%%%%%%%%
\DescribeMacro{\...prefix}
In the alternative form |\childdocforwardprefix|,
%
\begin{center}
\begin{tabular}{l}
|\input{childdoc.def}|\\
|\childdocforwardprefix[|\textit{main}|]{|\textit{prefix}|}{|\textit{dest}|}|
\end{tabular}
\end{center}
%
the destination file is determined by a pattern
depending on the current file:
To make this work, the current file must be called
`{\textit{prefix}\hspace{0.2em}\textit{suffix}}'
with \textit{prefix} matching precisely the argument.
Processing is then passed on to the file
`{\textit{dest}\hspace{0.2em}\textit{suffix}}'.
Surely, the same effect is achieved by
directly specifying the
argument `{\textit{dest}\hspace{0.2em}\textit{suffix}}'
in the first form.
However, that requires to set up a different file
for each child. With the alternative form of the command
all these files can have exactly the same content
which simplifies setting them up and maintaining them.

For example, the following file |draft.tex|
with a compilation flag |\version| as described in \secref{sec:flags}
compiles the main document as a draft:
%
\begin{center}
\begin{tabular}{l}
|\def\version{draft}|\\
|\input{childdoc.def}|\\
|\childdocforward{|\textit{main}|}|
\end{tabular}
\end{center}
%
Likewise, the following files |final|\textit{nn}|.tex|
compile the final version of the child document
|child|\textit{nn}|.tex|:
%
\begin{center}
\begin{tabular}{l}
|\def\version{final}|\\
|\input{childdoc.def}|\\
|\childdocforwardprefix{final}{child}|
\end{tabular}
\end{center}
%

Note that when several versions of a main file and/or of each child file
are to be generated, it may be convenient to set up a |Makefile| or
shell script to automatise the process.

%%%%%%%%%%%%%%%%%%%%%%%%%%%%%%%%%%%%%%%%%%%%%%%%%%%%%%%%%%%%%%%%%%%%%%%%%%%%%%%%
\subsection{Command Line Processing}
\label{sec:commandline}

The effect of redirection files can also be achieved by invoking
the \LaTeX{} compiler with a more elaborate command line.
Most conveniently this should be done as part
of a shell script or a |Makefile|.

When using \textsf{childdoc} in the main file, the following
command lines effectively perform a redirection
(note that depending on the shell being used,
backslashes may have to be doubled: `|\|' $\to$ `|\\|'):
%
\begin{center}
|... -jobname "|\textit{target}|" |\\|"|[\textit{flags}]%
|\input{childdoc.def}\childdocforward[|\textit{main}|]{|\textit{dest}|}"|
\end{center}
%
Here \textit{target} is the name of the output file,
\textit{main} is the name of the main file
and \textit{dest} is the name of the main or child file to be processed
(all filenames without extensions).
The optional argument \textit{main} can be omitted
if \textit{main} matches \textit{dest}.
Optionally, compilation \textit{flags} can be defined via |\def| commands.
This command line makes the \TeX{} engine believe
it is compiling the file \textit{target}
whose content is specified as the latter parameter.
The provided code then forwards the processing to
\textit{main} or \textit{dest} as described in \secref{sec:forward}.

%%%%%%%%%%%%%%%%%%%%%%%%%%%%%%%%%%%%%%%%%%%%%%%%%%%%%%%%%%%%%%%%%%%%%%%%%%%%%%%%
\subsection{Include by Input}
\label{sec:input}

Including child documents by |\include| has some restrictions by design.
Most notably, the content of a child document always occupies
its own set of pages; pages cannot be shared between child documents.
Usually, this behaviour makes perfect sense
because each child document contain an essential part of the document.
However, in some situations it may be desirable to compose
a document from a collection of parts
without having mandatory page breaks between then.
For this case, the package
provides a mechanism to include parts
by |\input| which can also be processed individually.
However, by construction this mechanism
requires manual handling of the content to be output.

%%%%%%%%%%%%%%%%%%%%%%%%%%%%%%%%%%%%%%%%
\DescribeMacro{\ifchilddocmanual}
The main file should be prepared as usual, see \secref{sec:include}.
However, the document body must make a distinction
between processing of an individual part and of the main document, e.g.:
%
\begin{center}
\begin{tabular}{l}
|\ifchilddocmanual|\\
|\input{\childdocname}|\\
|\||else|\\
\textit{document body with }|\input{|\textit{part}|}|\\
|\||fi|
\end{tabular}
\end{center}
%
The conditional |\ifchilddocmanual| is true whenever
a part to be included by |\input| is being compiled,
and the name of the part is stored in |\childdocname|.

%%%%%%%%%%%%%%%%%%%%%%%%%%%%%%%%%%%%%%%%
\DescribeMacro{\childdocby}
Each part to be included by |\input| should start with:
%
\begin{center}
\begin{tabular}{l}
|\input{childdoc.def}|\\
|\childdocby{|\textit{main}|}|\\
\end{tabular}
\end{center}
%
The directive |\childdocby| is similar to |\childdocof|
described in \secref{sec:include},
but the subsequent selection of content must be done manually.
To that end, both |\ifchilddoc| and |\ifchilddocmanual|
will be true upon processing of a part,
and the name of the part is stored in |\childdocname|.
Note that |\jobname| will be set to the filename of the current part
so that each part receives an individual |.aux| file
that does not interfere with the |.aux| file(s) of the main document.
This behaviour can be altered by the alternative form
|\childdocby[*]{|\textit{main}|}| (with a non-empty optional argument)
which uses the |.aux| file of the main document
by setting |\jobname| to \textit{main}.

%%%%%%%%%%%%%%%%%%%%%%%%%%%%%%%%%%%%%%%%%%%%%%%%%%%%%%%%%%%%%%%%%%%%%%%%%%%%%%%%
\subsection{Driver Development}
\label{sec:driver}

The \textsf{childdoc} mechanism can also be use for the development
of definition files such as \LaTeX{} styles or classes.
This case differs from the above setup with multiple parts
included by |\include| in that no |\includeonly| should be invoked.
This can be achieved by starting the include file
(before |\ProvidesPackage|) with:
%
\begin{center}
\begin{tabular}{l}
|\input{childdoc.def}|\\
|\childdocforward{|\textit{main}|}|\\
\end{tabular}
\end{center}
%
or alternatively with:
%
\begin{center}
\begin{tabular}{l}
|\input{childdoc.def}|\\
|\childdocby{|\textit{main}|}|\\
\end{tabular}
\end{center}
%
Both forms have slightly different effects as described above.
The main file is prepared as usual, see \secref{sec:include}.

%%%%%%%%%%%%%%%%%%%%%%%%%%%%%%%%%%%%%%%%%%%%%%%%%%%%%%%%%%%%%%%%%%%%%%%%%%%%%%%%
\subsection{Legacy Detection}
\label{sec:detection}

The directive |\childdocmain| in the main file can detect
whether the complete document or merely a child is to be compiled
even without using the directive |\childdocof|.
This method is deprecated because it is less robust
and there is no compelling reason to use it;
it is merely provided for backward compatibility
and it may be removed in future versions.

If the detection mechanism is to be used,
it is mandatory to correctly specify
the filename of the main file as the argument of |\childdocmain|:
%
\begin{center}
\begin{tabular}{l}
|\input{childdoc.def}|\\
|\childdocmain{|\textit{main}|}|\\
\end{tabular}
\end{center}
%
If |\jobname| does not match the argument \textit{main} of |\childdocmain|,
it is assumed that |\jobname| points to the child file to be compiled.
When using |\childdocmain| with the main file specified as argument,
it suffices to start a child file
with just |\input{|\textit{main}|}|
without loading of the package and using |\childdocof|.
If instead all processing is done
with the appropriate \textsf{childdoc} directives,
the argument of \textit{main} of |\childdocmain| can be empty.

An alternative version of the command line processing described
in \secref{sec:commandline} using the detection mechanism reads:
%
\begin{center}
|... -jobname "|\textit{target}|" "|[\textit{flags}]%
[|\def\jobname{|\textit{dest}|}|]|\input{|\textit{main}|}"|
\end{center}

%%%%%%%%%%%%%%%%%%%%%%%%%%%%%%%%%%%%%%%%%%%%%%%%%%%%%%%%%%%%%%%%%%%%%%%%%%%%%%%%
\subsection{Manual Code}
\label{sec:manual}

In case one cannot be certain whether the definitions file |childdoc.def|
is installed on the target \TeX{} distribution
and one prefers not to ship it,
it is conceivable to paste a few relevant commands into the sources.

To that end, drop all statements |\input{childdoc.def}|
and perform the replacements as outlined below.
Instead of |\childdocmain{|\textit{main}|}| add the following code
to the top of the main file:
%
\begin{center}
\begin{tabular}{l}
|\||ifdefined\childdocname\endinput\||fi\newif\ifchilddoc|\\
|\edef\childdocname{\scantokens\expandafter{\jobname\noexpand}}|\\
|\def\childdocmain{|\textit{main}|}\||ifx\childdocmain\childdocname\||else|\\
|\childdoctrue\includeonly{\childdocname}\let\jobname\childdocmain\||fi|\\
\end{tabular}
\end{center}
%
Instead of |\childdocof{|\textit{main}|}| just include the main file
at the top of each child file:
%
\begin{center}
|\input{|\textit{main}|}|
\end{center}
%
A simple redirection |\childdocforward{|\textit{dest}|}| is achieved by:
%
\begin{center}
|\def\jobname{|\textit{dest}|}\input{\jobname}|
\end{center}
%
The redirection with prefix
|\childdocforwardprefix[|\textit{prefix}|]{|\textit{dest}|}|
is accomplished by:
%
\begin{center}
\begin{tabular}{l}
|{\edef\jobname{\scantokens\expandafter{\jobname\noexpand}}|\\
|\def\redirectjob |\textit{prefix}|#1~~~{\gdef\jobname{|\textit{dest}|#1}}|\\
|\expandafter\redirectjob\jobname~~~}\input{\jobname}|
\end{tabular}
\end{center}

In an alternative approach,
child documents can be compiled by a specific command line
without additional code or specific definitions:
%
\begin{center}
|... -jobname "|\textit{target}|" "|[\textit{flags}]%
|\includeonly{|\textit{dest}|}\input{|\textit{main}|}"|
\end{center}
%

%%%%%%%%%%%%%%%%%%%%%%%%%%%%%%%%%%%%%%%%%%%%%%%%%%%%%%%%%%%%%%%%%%%%%%%%%%%%%%%%
%%%%%%%%%%%%%%%%%%%%%%%%%%%%%%%%%%%%%%%%%%%%%%%%%%%%%%%%%%%%%%%%%%%%%%%%%%%%%%%%
\section{Information}

%%%%%%%%%%%%%%%%%%%%%%%%%%%%%%%%%%%%%%%%%%%%%%%%%%%%%%%%%%%%%%%%%%%%%%%%%%%%%%%%
\subsection{Copyright}

Copyright \copyright{} 2017--2018 Niklas Beisert

This work may be distributed and/or modified under the
conditions of the \LaTeX{} Project Public License, either version 1.3
of this license or (at your option) any later version.
The latest version of this license is in
  \url{http://www.latex-project.org/lppl.txt}
and version 1.3 or later is part of all distributions of \LaTeX{}
version 2005/12/01 or later.

This work has the LPPL maintenance status `maintained'.

The Current Maintainer of this work is Niklas Beisert.

This work consists of the files |README.txt|, |childdoc.ins| and |childdoc.dtx|
as well as the derived files |childdoc.def|, |cdocsamp.tex|
with |cdocsch1.tex|, |cdocsch2.tex|, |cdocspt3.tex|, |cdocspt4.tex|,
|cdocsdrf.tex|, |cdocsfn1.tex|, |cdocsfn2.tex|
as well as |childdoc.pdf|.

%%%%%%%%%%%%%%%%%%%%%%%%%%%%%%%%%%%%%%%%%%%%%%%%%%%%%%%%%%%%%%%%%%%%%%%%%%%%%%%%
\subsection{Files and Installation}

The package consists of the files:
%
\begin{center}
\begin{tabular}{ll}
    |README.txt|   & readme file \\
    |childdoc.ins| & installation file \\
    |childdoc.dtx| & source file \\
    |childdoc.def| & definition file \\
    |cdocsamp.tex| & sample main file \\
    |cdocsch1.tex| & sample include file \\
    |cdocsch2.tex| & sample include file \\
    |cdocspt3.tex| & sample part file \\
    |cdocspt4.tex| & sample part file \\
    |cdocsdrf.tex| & sample redirection file \\
    |cdocsfn1.tex| & sample redirection file \\
    |cdocsfn2.tex| & sample redirection file \\
    |childdoc.pdf| & manual
\end{tabular}
\end{center}
%
The distribution consists of the files
|README.txt|, |childdoc.ins| and |childdoc.dtx|.
%
\begin{itemize}
\item
Run (pdf)\LaTeX{} on |childdoc.dtx|
to compile the manual |childdoc.pdf| (this file).
\item
Run \LaTeX{} on |childdoc.ins| to create the definitions file |childdoc.def|
and the sample |cdocsamp.tex| with include files
|cdocsch1.tex|, |cdocsch2.tex|, |cdocspt3.tex|, |cdocspt4.tex|,
|cdocsdrf.tex|, |cdocsfn1.tex|, |cdocsfn2.tex|.
Then copy the file |childdoc.def| to an appropriate directory of your \LaTeX{}
distribution, e.g.\ \textit{texmf-root}|/tex/latex/childdoc|.
\end{itemize}

%%%%%%%%%%%%%%%%%%%%%%%%%%%%%%%%%%%%%%%%%%%%%%%%%%%%%%%%%%%%%%%%%%%%%%%%%%%%%%%%
\subsection{Related CTAN Packages}

There are several other packages which offer a similar functionality:
%
\begin{itemize}
\item
The packages
\href{http://ctan.org/pkg/docmute}{\textsf{docmute}},
\href{http://ctan.org/pkg/includex}{\textsf{includex}} and
\href{http://ctan.org/pkg/standalone}{\textsf{standalone}}
provide commands to include only the document body of
a child file thus allowing both files to be compiled individually.
\item
The packages \href{http://ctan.org/pkg/subdocs}{\textsf{subdocs}}
and \href{http://ctan.org/pkg/subfiles}{\textsf{subfiles}}
provide structures in which the main and child documents can be
encapsulated and allowing them to be compiled individually.
The inclusion mechanism is different from the conventional |\include|.
\item
The package \href{http://ctan.org/pkg/combine}{\textsf{combine}}
is an elaborate solution to combine several documents into one.
\end{itemize}
%
See also the CTAN topic \href{http://ctan.org/topic/subdocs}{\textsf{subdocs}}
for further related packages.
The present package differs from the above solutions in that
a document structure constructed with the conventional |\include| mechanism
just needs two extra commands at the top of every file
such that all constituent files can be compiled individually.

%%%%%%%%%%%%%%%%%%%%%%%%%%%%%%%%%%%%%%%%%%%%%%%%%%%%%%%%%%%%%%%%%%%%%%%%%%%%%%%%
%\subsection{Feature Suggestions}
%
%The following is a list of features which may be useful for future
%versions of this package:
%%
%\begin{itemize}
%\item
%\ldots
%\end{itemize}

%%%%%%%%%%%%%%%%%%%%%%%%%%%%%%%%%%%%%%%%%%%%%%%%%%%%%%%%%%%%%%%%%%%%%%%%%%%%%%%%
\subsection{Revision History}

%%%%%%%%%%%%%%%%%%%%%%%%%%%%%%%%%%%%%%%%
\paragraph{v2.0:} 2018/12/30

\begin{itemize}
\item
immediate forward processing
\item
added |\childdocby| mechanism
\item
manual restructured
\end{itemize}

%%%%%%%%%%%%%%%%%%%%%%%%%%%%%%%%%%%%%%%%
\paragraph{v1.6:} 2018/01/17

\begin{itemize}
\item
application for development of include files
\item
corrections to manual
\end{itemize}

%%%%%%%%%%%%%%%%%%%%%%%%%%%%%%%%%%%%%%%%
\paragraph{v1.5:} 2017/05/21

\begin{itemize}
\item
more complete structuring introduced
\item
|\childdocof| introduced
\item
|\childdoc| renamed to |\childdocmain|
\item
|\childredirect| renamed to |\childdocforward| and |\childdocforwardprefix|
and functionality expanded
\end{itemize}

%%%%%%%%%%%%%%%%%%%%%%%%%%%%%%%%%%%%%%%%
\paragraph{v1.0:} 2017/04/27

\begin{itemize}
\item
manual and install package
\item
first version published on CTAN
\end{itemize}

%%%%%%%%%%%%%%%%%%%%%%%%%%%%%%%%%%%%%%%%
\paragraph{v0.6:} 2017/04/26

\begin{itemize}
\item
redirection mechanism added
\end{itemize}

%%%%%%%%%%%%%%%%%%%%%%%%%%%%%%%%%%%%%%%%
\paragraph{v0.5:} 2017/04/26

\begin{itemize}
\item
functionality in definition file
\end{itemize}


%%%%%%%%%%%%%%%%%%%%%%%%%%%%%%%%%%%%%%%%%%%%%%%%%%%%%%%%%%%%%%%%%%%%%%%%%%%%%%%%
%%%%%%%%%%%%%%%%%%%%%%%%%%%%%%%%%%%%%%%%%%%%%%%%%%%%%%%%%%%%%%%%%%%%%%%%%%%%%%%%
%%%%%%%%%%%%%%%%%%%%%%%%%%%%%%%%%%%%%%%%%%%%%%%%%%%%%%%%%%%%%%%%%%%%%%%%%%%%%%%%
\appendix

\settowidth\MacroIndent{\rmfamily\scriptsize 000\ }

 \DocInput{childdoc.dtx}

\end{document}
%</driver>
% \fi
%
% %%%%%%%%%%%%%%%%%%%%%%%%%%%%%%%%%%%%%%%%%%%%%%%%%%%%%%%%%%%%%%%%%%%%%%%%%%%%%%
% %%%%%%%%%%%%%%%%%%%%%%%%%%%%%%%%%%%%%%%%%%%%%%%%%%%%%%%%%%%%%%%%%%%%%%%%%%%%%%
% \section{Sample}
%\iffalse
%<*samplemain>
%\fi
%
% The following presents a sample document
% with two chapters, two parts, a title page,
% a compile flag as well as three forwarding files to set the flag.
% It consists of eight |.tex| files:
% \begin{center}
% \begin{tabular}{ll}
% |cdocsamp.tex|&main file\\
% |cdocsch1.tex|&include file for chapter 1\\
% |cdocsch2.tex|&include file for chapter 2\\
% |cdocspt3.tex|&include file for part 3\\
% |cdocspt4.tex|&include file for part 4\\
% |cdocsdrf.tex|&forwarding file for main file in draft mode\\
% |cdocsfi1.tex|&forwarding file for final version of chapter 1\\
% |cdocsfi2.tex|&forwarding file for final version of chapter 2\\
% \end{tabular}
% \end{center}
% Each of the eight files can be compiled directly by the \LaTeX{} compiler.
%
% %%%%%%%%%%%%%%%%%%%%%%%%%%%%%%%%%%%%%%
% \paragraph{Main File.}
%
% The main file is called |cdocsamp.tex|.
%
% Load the \textsf{childdoc} definitions and
% declare the filename for the main document:
%    \begin{macrocode}
\input{childdoc.def}
\childdocmain{}
%    \end{macrocode}

% Optional override for |\version| flag:
%    \begin{macrocode}
%%\ifchilddoc\else\providecommand{\version}{draft}\fi
%    \end{macrocode}

% Define the default values for the |\version| flag
% (|final| for the main file and |draft| for childs):
%    \begin{macrocode}
\ifchilddoc
\providecommand{\version}{draft}
\else
\providecommand{\version}{final}
\fi
%    \end{macrocode}

% Load the standard document class:
%    \begin{macrocode}
\documentclass[12pt]{article}
%    \end{macrocode}

% Start the document body:
%    \begin{macrocode}
\begin{document}
%    \end{macrocode}

% Declare a title page.
% Print title, part of document being processed and version flag:
%    \begin{macrocode}
\addtocounter{page}{-1}
\begin{center}
{\LARGE\bfseries{}childdoc example\par}
\vspace{1cm}
\ifchilddoc
\ifchilddocmanual part\else chapter\fi:
`\childdocname' of `\childdocjob'\par
\else
main document: `\childdocjob'\par
\fi
version: \version\par
\end{center}
\newpage
%    \end{macrocode}

% Manually include selected file,
% otherwise process as usual:
%    \begin{macrocode}
\ifchilddocmanual
\section*{part `\childdocname'}
\input{\childdocname}
\else
%    \end{macrocode}

% Include the two chapters:
%    \begin{macrocode}
\include{cdocsch1}
\include{cdocsch2}
%    \end{macrocode}

% Include the two parts unless only chapters should be displayed:
%    \begin{macrocode}
\ifchilddoc\else
\section{part three}
\input{cdocspt3}
\section{part four}
\input{cdocspt4}
\fi
%    \end{macrocode}

% Process as usual until here:
%    \begin{macrocode}
\fi
%    \end{macrocode}

% End of document body:
%    \begin{macrocode}
\end{document}
%    \end{macrocode}
%\iffalse
%</samplemain>
%\fi
%
% %%%%%%%%%%%%%%%%%%%%%%%%%%%%%%%%%%%%%%
% \paragraph{Chapter Include Files.}
%
% The include files are called |cdocsch1.tex| and |cdocsch2.tex|.
%
%\iffalse
%<*samplechap1|samplechap2>
%\fi

% Optional override for |\version| flag:
%    \begin{macrocode}
%%\providecommand{\version}{final}
%    \end{macrocode}

% Include the main document:
%    \begin{macrocode}
\input{childdoc.def}
\childdocof{cdocsamp}
%    \end{macrocode}

%\iffalse
%</samplechap1|samplechap2>
%\fi
%
%\iffalse
%<*samplechap1>
%\fi
% Some text for chapter 1:
%    \begin{macrocode}
\section{one}
some text in chapter one
%    \end{macrocode}

%\iffalse
%</samplechap1>
%\fi
% Some text for chapter 2:
%\iffalse
%<*samplechap2>
%\fi
%    \begin{macrocode}
\section{two}
more text in chapter two
%    \end{macrocode}

%\iffalse
%</samplechap2>
%\fi
%
% %%%%%%%%%%%%%%%%%%%%%%%%%%%%%%%%%%%%%%
% \paragraph{Part Include Files.}
%
% The include files are called |cdocspt3.tex| and |cdocspt4.tex|.
%
%\iffalse
%<*samplepart3|samplepart4>
%\fi

% Optional override for |\version| flag:
%    \begin{macrocode}
%%\providecommand{\version}{final}
%    \end{macrocode}

% Include the main document:
%    \begin{macrocode}
\input{childdoc.def}
\childdocby{cdocsamp}
%    \end{macrocode}

%\iffalse
%</samplepart3|samplepart4>
%\fi
%
%\iffalse
%<*samplepart3>
%\fi
% Some text for part 3:
%    \begin{macrocode}
some text in part three
%    \end{macrocode}

%\iffalse
%</samplepart3>
%\fi
% Some text for part 4:
%\iffalse
%<*samplepart4>
%\fi
%    \begin{macrocode}
more text in part four
%    \end{macrocode}

%\iffalse
%</samplepart4>
%\fi
%
% %%%%%%%%%%%%%%%%%%%%%%%%%%%%%%%%%%%%%%
% \paragraph{Forwarding for a Complete Draft.}
%
% The following forwarding file |cdocsdrf.tex|
% compiles the main document in draft mode:
%\iffalse
%<*sampledraft>
%\fi
%    \begin{macrocode}
\def\version{draft}
\input{childdoc.def}
\childdocforward{cdocsamp}
%    \end{macrocode}

%\iffalse
%</sampledraft>
%\fi
%
% %%%%%%%%%%%%%%%%%%%%%%%%%%%%%%%%%%%%%%
% \paragraph{Forwarding for Final Version of the Chapters.}
%
% The following forwarding files |cdocsfn1.tex| and |cdocsfn2.tex|
% (with identical content)
% compile the final versions of the child documents
% |cdocsch1.tex| and |cdocsch2.tex|, respectively:
%\iffalse
%<*samplefinal>
%\fi
%    \begin{macrocode}
\def\version{final}
\input{childdoc.def}
\childdocforwardprefix[cdocsamp]{cdocsfn}{cdocsch}
%    \end{macrocode}

%\iffalse
%</samplefinal>
%\fi
%
% %%%%%%%%%%%%%%%%%%%%%%%%%%%%%%%%%%%%%%
% \paragraph{Command Line Processing.}
%
% The following three command lines generate the output files
% |cdocscld|, |cdocscl1| and |cdocscl2|
% which should be identical to
% |cdocsdrf|, |cdocsch1| and |cdocsfn2|, respectively:
% \begin{center}
% \begin{tabular}{l}
% |latex -jobname cdocscld \|\\
% |  "\def\version{draft}\input{childdoc.def}\childdocforward{cdocsamp}"|\\
% |latex -jobname cdocscl1 \|\\
% |  "\input{childdoc.def}\childdocforward[cdocsamp]{cdocsch1}"|\\
% |latex -jobname cdocscl2 \|\\
% |  "\def\version{final}\input{childdoc.def}\childdocforward{cdocsch2}"|
% \end{tabular}
% \end{center}
% Note that the trailing backslash on each first line
% merely continues the input to the second line
% (for convenient cut ant paste).
% Furthermore, the command |latex| can be replaced by any
% of its alternative versions such as |pdflatex|.
%
% %%%%%%%%%%%%%%%%%%%%%%%%%%%%%%%%%%%%%%%%%%%%%%%%%%%%%%%%%%%%%%%%%%%%%%%%%%%%%%
% %%%%%%%%%%%%%%%%%%%%%%%%%%%%%%%%%%%%%%%%%%%%%%%%%%%%%%%%%%%%%%%%%%%%%%%%%%%%%%
% \section{Implementation}
%\iffalse
%<*package>
%\fi
%
% This section describes the definitions file |childdoc.def|.

% The definitions cannot be loaded using |\usepackage| or |\RequirePackage|
% which has a mechanism to prevent loading a style file more than once.
% When loading the definitions by means of |\input|
% multiple instances have to be prevented manually:
%\iffalse
%This code needs to be before the `\ProvidesFile' directive
%which is defined at the beginning of this file.
%Therefore it is also placed there and commented out here.
%</package>
%<*discard>
%\fi
%    \begin{macrocode}
\ifdefined\childdocmain\endinput\fi
%    \end{macrocode}
%\iffalse
%</discard>
%<*package>
%\fi
%
% \macro{\ifchilddoc}
% \macro{\ifchilddocmanual}
% The conditional |\ifchilddoc| tells whether a
% child (true) or main (false) document is being compiled.
% The conditional |\ifchilddocmanual| tells whether
% the |\includeonly| mechanism is used (false) or
% the selection of child files must be performed manually (true).
% The definitions initialise to false:
%    \begin{macrocode}
\newif\ifchilddoc
\newif\ifchilddocmanual
%    \end{macrocode}

% \macro{\childdocname}
% \macro{\childdocjob}
% The macro |\childdocname| stores the name of the main document
% to be compiled. The macro |\childdocjob| stores the name of
% the document on which the \LaTeX{} compiler was originally invoked.
% The content of |\jobname| cannot be compared
% to filenames specified in the source due to different catcodes.
% The following code rescans |\jobname|, stores the result
% in |\childdocname| and saves a copy in |\childdocjob|:
%    \begin{macrocode}
\edef\childdocname{\scantokens\expandafter{\jobname\noexpand}}
\let\childdocjob\childdocname
%    \end{macrocode}

% \macro{\childdocdisable}
% The macro |\childdocdisable| prevents the main file
% from being processed more than once.
% At this stage, the main document command |\childdocmain|
% is assumed to be called once again where it should do nothing.
% Any subsequent call to it should prevent
% a secondary processing of the main document
% It overwrites the forwarding commands
% |\childdocof| and |\childdocforward|
% with empty macros to prevent further inclusions of the main document:
%    \begin{macrocode}
\newcommand{\childdocdisable}
{
  \renewcommand{\childdocmain}[1]{\renewcommand{\childdocmain}[1]{\endinput}}
  \renewcommand{\childdocof}[1]{}
  \renewcommand{\childdocby}[2][]{}
  \renewcommand{\childdocforward}[2][]{}
  \renewcommand{\childdocdisable}{}
}
%    \end{macrocode}

% \macro{\childdocmain}
% The macro |\childdocmain| is to be called at the top of the main file
% with nothing or the main filename (without extension) as argument.
% First, it breaks loops.
% If the argument is not empty and does not match |\childdocname|
% (which is set by the first inclusion of |childdoc.def|),
% |\ifchilddoc| is set to true, |\includeonly| is applied to the child file
% and |\jobname| is set to the main file
% (for proper handling of |.aux| files):
%    \begin{macrocode}
\newcommand{\childdocmain}[1]
{
  \childdocdisable\childdocmain{}
  \if?#1?\else
    \begingroup
      \def\childdoctmp{#1}
      \ifx\childdoctmp\childdocname
        \def\childdoctmp{}
      \else
        \def\childdoctmp
        {
          \childdoctrue
          \includeonly{\childdocname}
          \def\childdocjob{#1}
          \def\jobname{#1}
        }
      \fi
      \expandafter
    \endgroup
    \childdoctmp
  \fi
}
%    \end{macrocode}

% \macro{\childdocof}
% The command |\childdocof| redirects
% compilation to the main file |#1|.
%    \begin{macrocode}
\newcommand{\childdocof}[1]
{
  \childdocdisable
  \childdoctrue
  \includeonly{\childdocname}
  \def\jobname{#1}
  \def\childdocjob{#1}
  \input{#1}
}
%    \end{macrocode}

% \macro{\childdocby}
% The command |\childdocby| ....
%    \begin{macrocode}
\newcommand{\childdocby}[2][]
{
  \childdocdisable
  \childdoctrue
  \childdocmanualtrue
  \if?#1?\else
    \def\jobname{#2}
  \fi
  \def\childdocjob{#2}
  \input{#2}
  \endinput
}
%    \end{macrocode}

% \macro{\childdocforward}
% The command |\childdocforward| redirects
% compilation to the main file or
% (if the optional argument is given) a child file.
% Parameters are set as if the main file
% or a child file starting with |\childdocof| was compiled.
% Then compilation is handed over to the main file:
%    \begin{macrocode}
\newcommand{\childdocforward}[2][]
{
  \begingroup
    \if?#1?
      \def\childdoctmp
      {
        \def\childdocname{#2}
        \def\childdocjob{#2}
        \def\jobname{#2}
        \input{#2}
        \endinput
      }
    \else
      \def\childdoctmp
      {
        \childdocdisable
        \def\childdocname{#2}
        \childdoctrue
        \includeonly{#2}
        \def\childdocjob{#1}
        \def\jobname{#1}
        \input{#1}
        \endinput
      }
    \fi
    \expandafter
  \endgroup
  \childdoctmp
}
%    \end{macrocode}

% \macro{\childdocforwardprefix}
% The command |\childdocforwardprefix| redirects
% compilation to the main or a child file by means of a pattern.
% The prefix |#1| in the current filename is replaced by |#2|
% and the suffix of the current filename is kept
% (it is assumed that the filename does not contain the substring `|~~~|'
% which is used as a delimiter).
% Compilation is handed over to the new file by |\childdocforward|:
%    \begin{macrocode}
\newcommand{\childdocforwardprefix}[3][]
{
  \begingroup
    \def\childdocextract #2##1~~~{\def\childdoctmp{\childdocforward[#1]{#3##1}}}
    \expandafter\childdocextract\childdocname~~~
    \expandafter
  \endgroup
  \childdoctmp
}
%    \end{macrocode}

% \macro{\childdoc}
% The deprecated macro |\childdoc| is a legacy version of |\childdocmain|:
%    \begin{macrocode}
\newcommand{\childdoc}{\childdocmain}
%    \end{macrocode}

% \macro{\childdocredirect}
% The deprecated macro |\childdocredirect| is a legacy version
% of |\childdocforward| and |\childdocforwardprefix|:
%    \begin{macrocode}
\newcommand{\childdocredirect}[2][]
{
  \begingroup
    \if?#1?
      \def\childdoctmp{\childdocforward{#2}}
    \else
      \def\childdoctmp{\childdocforwardprefix{#1}{#2}}
    \fi
    \expandafter
  \endgroup
  \childdoctmp
}
%    \end{macrocode}

%\iffalse
%</package>
%\fi
%
\endinput
|\\
|\childdocby{|\textit{main}|}|\\
\end{tabular}
\end{center}
%
The directive |\childdocby| is similar to |\childdocof|
described in \secref{sec:include},
but the subsequent selection of content must be done manually.
To that end, both |\ifchilddoc| and |\ifchilddocmanual|
will be true upon processing of a part,
and the name of the part is stored in |\childdocname|.
Note that |\jobname| will be set to the filename of the current part
so that each part receives an individual |.aux| file
that does not interfere with the |.aux| file(s) of the main document.
This behaviour can be altered by the alternative form
|\childdocby[*]{|\textit{main}|}| (with a non-empty optional argument)
which uses the |.aux| file of the main document
by setting |\jobname| to \textit{main}.

%%%%%%%%%%%%%%%%%%%%%%%%%%%%%%%%%%%%%%%%%%%%%%%%%%%%%%%%%%%%%%%%%%%%%%%%%%%%%%%%
\subsection{Driver Development}
\label{sec:driver}

The \textsf{childdoc} mechanism can also be use for the development
of definition files such as \LaTeX{} styles or classes.
This case differs from the above setup with multiple parts
included by |\include| in that no |\includeonly| should be invoked.
This can be achieved by starting the include file
(before |\ProvidesPackage|) with:
%
\begin{center}
\begin{tabular}{l}
|% \iffalse
%
% childdoc.dtx Copyright (C) 2017-2018 Niklas Beisert
%
% This work may be distributed and/or modified under the
% conditions of the LaTeX Project Public License, either version 1.3
% of this license or (at your option) any later version.
% The latest version of this license is in
%   http://www.latex-project.org/lppl.txt
% and version 1.3 or later is part of all distributions of LaTeX
% version 2005/12/01 or later.
%
% This work has the LPPL maintenance status `maintained'.
%
% The Current Maintainer of this work is Niklas Beisert.
%
% This work consists of the files childdoc.dtx and childdoc.ins
% and the derived files childdoc.def and cdocsamp.tex with
% cdocsch1.tex, cdocsch2.tex, cdocsdrf.tex, cdocsfn1.tex, cdocsfn2.tex.
%
%<package>\ifdefined\childdocmain\endinput\fi
%<package>\ProvidesFile{childdoc.def}[2018/12/30 v2.0 child document driver]
%<samplemain>\ProvidesFile{cdocsamp.tex}[2018/12/30 v2.0 sample for childdoc]
%<*driver>
%\ProvidesFile{childdoc.drv}[2018/12/30 v2.0 childdoc reference manual file]
\PassOptionsToClass{10pt,a4paper}{article}
\documentclass{ltxdoc}

\usepackage[margin=35mm]{geometry}
\usepackage{hyperref}
\usepackage{hyperxmp}
\usepackage[usenames]{color}

\hypersetup{colorlinks=true}
\hypersetup{pdfstartview=FitH}
\hypersetup{pdfpagemode=UseNone}
\hypersetup{pdfsource={}}
\hypersetup{pdflang={en-UK}}
\hypersetup{pdfcopyright={Copyright 2017-2018 Niklas Beisert.
  This work may be distributed and/or modified under the
  conditions of the LaTeX Project Public License, either version 1.3
  of this license or (at your option) any later version.}}
\hypersetup{pdflicenseurl={http://www.latex-project.org/lppl.txt}}
\hypersetup{pdfcontactaddress={ETH Zurich, ITP, HIT K,
  Wolfgang-Pauli-Strasse 27}}
\hypersetup{pdfcontactpostcode={8093}}
\hypersetup{pdfcontactcity={Zurich}}
\hypersetup{pdfcontactcountry={Switzerland}}
\hypersetup{pdfcontactemail={nbeisert@itp.phys.ethz.ch}}
\hypersetup{pdfcontacturl={http://people.phys.ethz.ch/\xmptilde nbeisert/}}

\newcommand{\secref}[1]{\hyperref[#1]{section \ref*{#1}}}

\parskip1ex
\parindent0pt
\let\olditemize\itemize
\def\itemize{\olditemize\parskip0pt}

\begin{document}

\title{The \textsf{childdoc} Package}
\hypersetup{pdftitle={The childdoc Package}}
\author{Niklas Beisert\\[2ex]
  Institut f\"ur Theoretische Physik\\
  Eidgen\"ossische Technische Hochschule Z\"urich\\
  Wolfgang-Pauli-Strasse 27, 8093 Z\"urich, Switzerland\\[1ex]
  \href{mailto:nbeisert@itp.phys.ethz.ch}
  {\texttt{nbeisert@itp.phys.ethz.ch}}}
\hypersetup{pdfauthor={Niklas Beisert}}
\hypersetup{pdfsubject={Manual for the LaTeX2e Package childdoc}}
\date{30 December 2018, \textsf{v2.0}}
\maketitle

\begin{abstract}\noindent
\textsf{childdoc} is a \LaTeXe{} package
that enables the direct compilation
of document sections included by |\include|
to individual files.
\end{abstract}

\begingroup
\parskip0ex
\tableofcontents
\endgroup

%%%%%%%%%%%%%%%%%%%%%%%%%%%%%%%%%%%%%%%%%%%%%%%%%%%%%%%%%%%%%%%%%%%%%%%%%%%%%%%%
%%%%%%%%%%%%%%%%%%%%%%%%%%%%%%%%%%%%%%%%%%%%%%%%%%%%%%%%%%%%%%%%%%%%%%%%%%%%%%%%
\section{Introduction}

\LaTeX{} provides a mechanism to structure a large document (such as a book)
into a main file and several child files (containing the chapters)
using the |\include| command.
This mechanism is beneficial for documents
which span hundreds of pages in order to
make the source file(s) more manageable.
Moreover, compilation can be restricted to
selected child files by means of the |\includeonly| command.
The latter feature can be used to reduce the compilation time while editing
(this was significantly more useful in the earlier days of \LaTeX{})
or to generate a smaller document which is easier to navigate.
Another application of |\includeonly| is to generate
documents consisting of selected parts of the complete document.

However, there are a few drawbacks of the plain |\include| mechanism:
\begin{itemize}
\item
The child files cannot be compiled on their own,
they can only be compiled via the main file.
A naive editing environment
(such as a text editor with an option
to have the current file processed by \LaTeX)
may require one to switch to the main file before compiling;
attempting to compile the child file produces errors.
\item
The main file must be modified (each time)
to adjust the |\includeonly| command
to the present needs. This easily leaves the main file in a messy state.
\item
The generated document will always carry the filename
of the main document. This is inconvenient if
several child files are to be compiled and
to be kept for distribution.
\end{itemize}

The present package provides a simple interface
to make child files individually compilable by \LaTeX{}.
Compiling a child file then has the same effect as compiling
the main file with an |\includeonly| command
to select the appropriate child.
Moreover the generated document will carry the name of the child
rather than the main file.
This resolves all three above issues.

This feature is meant to make the editing of books,
thesis documents and lecture notes somewhat more convenient.
However, the package can also be used efficiently for
composing a series of documents (such as exercise sheets)
which are typically distributed individually.
It then assists the author in generating the individual documents
(potentially in different versions)
as well as a document containing the collected series.
Another application is in developing style files
or other kinds of included material
where compilation of the style file could redirect
to a sample or test file.

%%%%%%%%%%%%%%%%%%%%%%%%%%%%%%%%%%%%%%%%%%%%%%%%%%%%%%%%%%%%%%%%%%%%%%%%%%%%%%%%
%%%%%%%%%%%%%%%%%%%%%%%%%%%%%%%%%%%%%%%%%%%%%%%%%%%%%%%%%%%%%%%%%%%%%%%%%%%%%%%%
\section{Usage}

First of all, the package \textsf{childdoc} is \emph{not} a standard
\LaTeXe{} |.sty| style file! Therefore it needs to be invoked in
a non-standard way.

%%%%%%%%%%%%%%%%%%%%%%%%%%%%%%%%%%%%%%%%%%%%%%%%%%%%%%%%%%%%%%%%%%%%%%%%%%%%%%%%
\subsection{Included Files}
\label{sec:include}

%%%%%%%%%%%%%%%%%%%%%%%%%%%%%%%%%%%%%%%%
\DescribeMacro{\childdocmain}
To use the package, add the commands
\begin{center}
\begin{tabular}{l}
|\input{childdoc.def}|\\
|\childdocmain{}|\\
\end{tabular}
\end{center}
at the very top of the main \LaTeX{} file,
in particular \emph{before} the |\documentclass| statement!
The argument of |\childdocmain| should be left empty
(but it must be present).

%%%%%%%%%%%%%%%%%%%%%%%%%%%%%%%%%%%%%%%%
\DescribeMacro{\childdocof}
Furthermore, add the commands
\begin{center}
\begin{tabular}{l}
|\input{childdoc.def}|\\
|\childdocof{|\textit{main}|}|\\
\end{tabular}
\end{center}
at the top of every child file \textit{child}
which is included by |\include{|\textit{child}|}|
from within the main file
(or at least for those files to be compiled individually).
The argument \textit{main} must be the filename of the main file.

There are a couple of
considerations in setting up the main and child documents:

%%%%%%%%%%%%%%%%%%%%%%%%%%%%%%%%%%%%%%%%
\paragraph{Restrictions.}

Please note the following restrictions:
\begin{itemize}
\item
|\childdocmain| must be called with one argument \textit{main}
to ensure compatibility with earlier version of the package.
It must either be empty (|\childdocmain{}|)
or precisely match the filename of the main file in which it is specified.
See \secref{sec:detection} for further information.
\item
The filename \textit{main} must be specified without the |.tex| extension.
\item
The filename \textit{main} is case sensitive
(even in case-insensitive file systems)
due to internal string comparison.
\item
The argument \textit{main} should be fully expanded, it cannot be a macro.
\item
Subdirectories and special characters should be avoided in filenames.
\item
The command |\childdocmain{|\textit{main}|}| must be followed by a whitespace.
It should not be followed immediately by another command
or by a comment mark `|%|'.
This is because the \TeX{} parser reads the token immediately following
the argument of |\childdocmain| and puts it
at the beginning of every child section;
however, a white\-space is ignored.
\end{itemize}

%%%%%%%%%%%%%%%%%%%%%%%%%%%%%%%%%%%%%%%%
\paragraph{Content of Main File.}

It is advisable to place all content in the child files included by |\include|.
Any output contained in the main file will appear in all child documents
unless suppressed manually;
it cannot be suppressed automatically by the |\includeonly| directive
and thus should normally be avoided.
A method to include some content in the main file
by means of conditional processing is described in \secref{sec:conditional}.

%%%%%%%%%%%%%%%%%%%%%%%%%%%%%%%%%%%%%%%%
\paragraph{Page Numbering.}

When only a part of the document is compiled,
the appropriate numbering of pages
(as well as other status parameters)
is determined from the |.aux| files.
The latter contain information from previous passes.
However this information needs to propagate through
all intermediate child documents.
Therefore the page numbering in child documents may well
be inconsistent until the complete document is compiled at least once.

A useful (if unconventional) way to always ensure a consistent
page numbering is to restart the numbering in each child document
and denote the pages by `\textit{child}|.|\textit{page}'
where \textit{child} represents the chapter/section number of the child file.
This can be achieved by the command
|\numberwithin{page}{|\textit{child}|}|
of the \textsf{amsmath} package
where \textit{child} can be |chapter| or |section|
depending on the chosen structuring.
Alternatively, one can modify the macro |\thepage| appropriately
and reset the counter |page| at the start of each child file.

%%%%%%%%%%%%%%%%%%%%%%%%%%%%%%%%%%%%%%%%%%%%%%%%%%%%%%%%%%%%%%%%%%%%%%%%%%%%%%%%
\subsection{Conditional Processing}
\label{sec:conditional}

The package provides a mechanism to compile different versions
of a document. To customise the versions further some conditional processing
can come in handy to distinguish which version is being compiled.
The package provides two macros to describe the compilation context:

%%%%%%%%%%%%%%%%%%%%%%%%%%%%%%%%%%%%%%%%
\DescribeMacro{\ifchilddoc}
The conditional |\ifchilddoc| distinguishes between the compilation of
child documents and the main document:
%
\begin{center}
|\ifchilddoc |\textit{child-code}| |[|\||else |\textit{main-code}]| \||fi|
\end{center}

%%%%%%%%%%%%%%%%%%%%%%%%%%%%%%%%%%%%%%%%
\DescribeMacro{\childdocname}
\DescribeMacro{\childdocjob}
The macro |\childdocname| contains the filename (without extension)
of the main or child file being processed.
Note that |\childdocjob| will always contain the name of the main file.

%%%%%%%%%%%%%%%%%%%%%%%%%%%%%%%%%%%%%%%%
\paragraph{Title Page.}

Conditional processing can be used to include a title or banner page
in the main document when proper precautions are taken.
Importantly, the code in the main file should ensure that the page counter
(as well as other status parameters which are stored in the |.aux| files)
takes the same value after the conditional processing.
Otherwise the page numbers may take divergent values
depending on which part is compiled.

For example, a title page could be declared by:
%
\begin{center}
\begin{tabular}{l}
|\ifchilddoc\||else|\\
|\addtocounter{page}{-1}|\\
\textit{code for title page}\\
|\newpage|\\
|\||fi|
\end{tabular}
\end{center}
%
A banner page for the child documents can be generated by:
%
\begin{center}
\begin{tabular}{l}
|\ifchilddoc|\\
|\addtocounter{page}{-1}|\\
\textit{code for banner page}\\
|\newpage|\\
|\||fi|
\end{tabular}
\end{center}
%
Here one could write a message such as:
\begin{center}
|This is the part \childdocname{} of \childdocjob{}.|
\end{center}

%%%%%%%%%%%%%%%%%%%%%%%%%%%%%%%%%%%%%%%%%%%%%%%%%%%%%%%%%%%%%%%%%%%%%%%%%%%%%%%%
\subsection{Flags}
\label{sec:flags}

The package makes it easy to generate different versions
of the main or child documents.
To this end compilation flags can be defined
and assigned different default values.
They will be particularly useful in conjunction
with the forwarding mechanism described in \secref{sec:forward}.

For example, it may be useful to have a flag |\version|
which can be set to |draft| or |final|.
The document source will contain some conditional code
depending on the value of |\version|.
Suppose further, the flag should default to |final| for the main file
and to |draft| for child files
which is a natural assignment for editing the document.
This is achieved by placing the following code
in the preamble of the main document
(below the |\childdocmain| directive):
%
\begin{center}
\begin{tabular}{l}
|\ifchilddoc|\\
|\providecommand{\version}{draft}|\\
|\||else|\\
|\providecommand{\version}{final}|\\
|\||fi|
\end{tabular}
\end{center}
%
The definition by |\providecommand| makes sure
that previous definitions are not overwritten.
Further statements |\providecommand{\version}{...}|
can thus be added before the above code to override it.

For the main file, one might add a line
(between |\childdocmain| and the above block)
%
\begin{center}
|%\ifchilddoc\||else\providecommand{\version}{draft}\||fi|
\end{center}
%
which can be uncommented to produce a draft version.
Likewise one can add a line to the very top of a child file
(above the |\childdocof{|\textit{main}|}| directive)
%
\begin{center}
|%\providecommand{\version}{final}|
\end{center}
%
which can be uncommented to produce the final version of this child document.

%%%%%%%%%%%%%%%%%%%%%%%%%%%%%%%%%%%%%%%%%%%%%%%%%%%%%%%%%%%%%%%%%%%%%%%%%%%%%%%%
\subsection{Forwarding}
\label{sec:forward}

Different versions of the main or child documents
using compilation flags as described in \secref{sec:flags}
can be (permanently) stored in different files
for convenient compilation, viewing and distribution.
To this end, the package defines a command
to pass on compilation to a different file:

%%%%%%%%%%%%%%%%%%%%%%%%%%%%%%%%%%%%%%%%
\DescribeMacro{\childdocforward}
The command |\childdocforward| redirects processing to
another source file:
%
\begin{center}
\begin{tabular}{l}
|\input{childdoc.def}|\\
|\childdocforward[|\textit{main}|]{|\textit{dest}|}|\\
\end{tabular}
\end{center}
%
The argument \textit{dest} is the destination file
(without extension).
It should be the main file or one of the child files.
Note that further \textsf{childdoc} directives
such as |\childdocof| and |\childdocforward|
in the indicated file will be processed in this form.
The optional argument \textit{main}
passes on directly to the main file \textit{main}
while pretending to compile the child \textit{dest}.
This form behaves as if \textit{dest}
issues |\childdocof{|\textit{main}|}| right away,
and no further \textsf{childdoc} directives will be processed.

%%%%%%%%%%%%%%%%%%%%%%%%%%%%%%%%%%%%%%%%
\DescribeMacro{\...prefix}
In the alternative form |\childdocforwardprefix|,
%
\begin{center}
\begin{tabular}{l}
|\input{childdoc.def}|\\
|\childdocforwardprefix[|\textit{main}|]{|\textit{prefix}|}{|\textit{dest}|}|
\end{tabular}
\end{center}
%
the destination file is determined by a pattern
depending on the current file:
To make this work, the current file must be called
`{\textit{prefix}\hspace{0.2em}\textit{suffix}}'
with \textit{prefix} matching precisely the argument.
Processing is then passed on to the file
`{\textit{dest}\hspace{0.2em}\textit{suffix}}'.
Surely, the same effect is achieved by
directly specifying the
argument `{\textit{dest}\hspace{0.2em}\textit{suffix}}'
in the first form.
However, that requires to set up a different file
for each child. With the alternative form of the command
all these files can have exactly the same content
which simplifies setting them up and maintaining them.

For example, the following file |draft.tex|
with a compilation flag |\version| as described in \secref{sec:flags}
compiles the main document as a draft:
%
\begin{center}
\begin{tabular}{l}
|\def\version{draft}|\\
|\input{childdoc.def}|\\
|\childdocforward{|\textit{main}|}|
\end{tabular}
\end{center}
%
Likewise, the following files |final|\textit{nn}|.tex|
compile the final version of the child document
|child|\textit{nn}|.tex|:
%
\begin{center}
\begin{tabular}{l}
|\def\version{final}|\\
|\input{childdoc.def}|\\
|\childdocforwardprefix{final}{child}|
\end{tabular}
\end{center}
%

Note that when several versions of a main file and/or of each child file
are to be generated, it may be convenient to set up a |Makefile| or
shell script to automatise the process.

%%%%%%%%%%%%%%%%%%%%%%%%%%%%%%%%%%%%%%%%%%%%%%%%%%%%%%%%%%%%%%%%%%%%%%%%%%%%%%%%
\subsection{Command Line Processing}
\label{sec:commandline}

The effect of redirection files can also be achieved by invoking
the \LaTeX{} compiler with a more elaborate command line.
Most conveniently this should be done as part
of a shell script or a |Makefile|.

When using \textsf{childdoc} in the main file, the following
command lines effectively perform a redirection
(note that depending on the shell being used,
backslashes may have to be doubled: `|\|' $\to$ `|\\|'):
%
\begin{center}
|... -jobname "|\textit{target}|" |\\|"|[\textit{flags}]%
|\input{childdoc.def}\childdocforward[|\textit{main}|]{|\textit{dest}|}"|
\end{center}
%
Here \textit{target} is the name of the output file,
\textit{main} is the name of the main file
and \textit{dest} is the name of the main or child file to be processed
(all filenames without extensions).
The optional argument \textit{main} can be omitted
if \textit{main} matches \textit{dest}.
Optionally, compilation \textit{flags} can be defined via |\def| commands.
This command line makes the \TeX{} engine believe
it is compiling the file \textit{target}
whose content is specified as the latter parameter.
The provided code then forwards the processing to
\textit{main} or \textit{dest} as described in \secref{sec:forward}.

%%%%%%%%%%%%%%%%%%%%%%%%%%%%%%%%%%%%%%%%%%%%%%%%%%%%%%%%%%%%%%%%%%%%%%%%%%%%%%%%
\subsection{Include by Input}
\label{sec:input}

Including child documents by |\include| has some restrictions by design.
Most notably, the content of a child document always occupies
its own set of pages; pages cannot be shared between child documents.
Usually, this behaviour makes perfect sense
because each child document contain an essential part of the document.
However, in some situations it may be desirable to compose
a document from a collection of parts
without having mandatory page breaks between then.
For this case, the package
provides a mechanism to include parts
by |\input| which can also be processed individually.
However, by construction this mechanism
requires manual handling of the content to be output.

%%%%%%%%%%%%%%%%%%%%%%%%%%%%%%%%%%%%%%%%
\DescribeMacro{\ifchilddocmanual}
The main file should be prepared as usual, see \secref{sec:include}.
However, the document body must make a distinction
between processing of an individual part and of the main document, e.g.:
%
\begin{center}
\begin{tabular}{l}
|\ifchilddocmanual|\\
|\input{\childdocname}|\\
|\||else|\\
\textit{document body with }|\input{|\textit{part}|}|\\
|\||fi|
\end{tabular}
\end{center}
%
The conditional |\ifchilddocmanual| is true whenever
a part to be included by |\input| is being compiled,
and the name of the part is stored in |\childdocname|.

%%%%%%%%%%%%%%%%%%%%%%%%%%%%%%%%%%%%%%%%
\DescribeMacro{\childdocby}
Each part to be included by |\input| should start with:
%
\begin{center}
\begin{tabular}{l}
|\input{childdoc.def}|\\
|\childdocby{|\textit{main}|}|\\
\end{tabular}
\end{center}
%
The directive |\childdocby| is similar to |\childdocof|
described in \secref{sec:include},
but the subsequent selection of content must be done manually.
To that end, both |\ifchilddoc| and |\ifchilddocmanual|
will be true upon processing of a part,
and the name of the part is stored in |\childdocname|.
Note that |\jobname| will be set to the filename of the current part
so that each part receives an individual |.aux| file
that does not interfere with the |.aux| file(s) of the main document.
This behaviour can be altered by the alternative form
|\childdocby[*]{|\textit{main}|}| (with a non-empty optional argument)
which uses the |.aux| file of the main document
by setting |\jobname| to \textit{main}.

%%%%%%%%%%%%%%%%%%%%%%%%%%%%%%%%%%%%%%%%%%%%%%%%%%%%%%%%%%%%%%%%%%%%%%%%%%%%%%%%
\subsection{Driver Development}
\label{sec:driver}

The \textsf{childdoc} mechanism can also be use for the development
of definition files such as \LaTeX{} styles or classes.
This case differs from the above setup with multiple parts
included by |\include| in that no |\includeonly| should be invoked.
This can be achieved by starting the include file
(before |\ProvidesPackage|) with:
%
\begin{center}
\begin{tabular}{l}
|\input{childdoc.def}|\\
|\childdocforward{|\textit{main}|}|\\
\end{tabular}
\end{center}
%
or alternatively with:
%
\begin{center}
\begin{tabular}{l}
|\input{childdoc.def}|\\
|\childdocby{|\textit{main}|}|\\
\end{tabular}
\end{center}
%
Both forms have slightly different effects as described above.
The main file is prepared as usual, see \secref{sec:include}.

%%%%%%%%%%%%%%%%%%%%%%%%%%%%%%%%%%%%%%%%%%%%%%%%%%%%%%%%%%%%%%%%%%%%%%%%%%%%%%%%
\subsection{Legacy Detection}
\label{sec:detection}

The directive |\childdocmain| in the main file can detect
whether the complete document or merely a child is to be compiled
even without using the directive |\childdocof|.
This method is deprecated because it is less robust
and there is no compelling reason to use it;
it is merely provided for backward compatibility
and it may be removed in future versions.

If the detection mechanism is to be used,
it is mandatory to correctly specify
the filename of the main file as the argument of |\childdocmain|:
%
\begin{center}
\begin{tabular}{l}
|\input{childdoc.def}|\\
|\childdocmain{|\textit{main}|}|\\
\end{tabular}
\end{center}
%
If |\jobname| does not match the argument \textit{main} of |\childdocmain|,
it is assumed that |\jobname| points to the child file to be compiled.
When using |\childdocmain| with the main file specified as argument,
it suffices to start a child file
with just |\input{|\textit{main}|}|
without loading of the package and using |\childdocof|.
If instead all processing is done
with the appropriate \textsf{childdoc} directives,
the argument of \textit{main} of |\childdocmain| can be empty.

An alternative version of the command line processing described
in \secref{sec:commandline} using the detection mechanism reads:
%
\begin{center}
|... -jobname "|\textit{target}|" "|[\textit{flags}]%
[|\def\jobname{|\textit{dest}|}|]|\input{|\textit{main}|}"|
\end{center}

%%%%%%%%%%%%%%%%%%%%%%%%%%%%%%%%%%%%%%%%%%%%%%%%%%%%%%%%%%%%%%%%%%%%%%%%%%%%%%%%
\subsection{Manual Code}
\label{sec:manual}

In case one cannot be certain whether the definitions file |childdoc.def|
is installed on the target \TeX{} distribution
and one prefers not to ship it,
it is conceivable to paste a few relevant commands into the sources.

To that end, drop all statements |\input{childdoc.def}|
and perform the replacements as outlined below.
Instead of |\childdocmain{|\textit{main}|}| add the following code
to the top of the main file:
%
\begin{center}
\begin{tabular}{l}
|\||ifdefined\childdocname\endinput\||fi\newif\ifchilddoc|\\
|\edef\childdocname{\scantokens\expandafter{\jobname\noexpand}}|\\
|\def\childdocmain{|\textit{main}|}\||ifx\childdocmain\childdocname\||else|\\
|\childdoctrue\includeonly{\childdocname}\let\jobname\childdocmain\||fi|\\
\end{tabular}
\end{center}
%
Instead of |\childdocof{|\textit{main}|}| just include the main file
at the top of each child file:
%
\begin{center}
|\input{|\textit{main}|}|
\end{center}
%
A simple redirection |\childdocforward{|\textit{dest}|}| is achieved by:
%
\begin{center}
|\def\jobname{|\textit{dest}|}\input{\jobname}|
\end{center}
%
The redirection with prefix
|\childdocforwardprefix[|\textit{prefix}|]{|\textit{dest}|}|
is accomplished by:
%
\begin{center}
\begin{tabular}{l}
|{\edef\jobname{\scantokens\expandafter{\jobname\noexpand}}|\\
|\def\redirectjob |\textit{prefix}|#1~~~{\gdef\jobname{|\textit{dest}|#1}}|\\
|\expandafter\redirectjob\jobname~~~}\input{\jobname}|
\end{tabular}
\end{center}

In an alternative approach,
child documents can be compiled by a specific command line
without additional code or specific definitions:
%
\begin{center}
|... -jobname "|\textit{target}|" "|[\textit{flags}]%
|\includeonly{|\textit{dest}|}\input{|\textit{main}|}"|
\end{center}
%

%%%%%%%%%%%%%%%%%%%%%%%%%%%%%%%%%%%%%%%%%%%%%%%%%%%%%%%%%%%%%%%%%%%%%%%%%%%%%%%%
%%%%%%%%%%%%%%%%%%%%%%%%%%%%%%%%%%%%%%%%%%%%%%%%%%%%%%%%%%%%%%%%%%%%%%%%%%%%%%%%
\section{Information}

%%%%%%%%%%%%%%%%%%%%%%%%%%%%%%%%%%%%%%%%%%%%%%%%%%%%%%%%%%%%%%%%%%%%%%%%%%%%%%%%
\subsection{Copyright}

Copyright \copyright{} 2017--2018 Niklas Beisert

This work may be distributed and/or modified under the
conditions of the \LaTeX{} Project Public License, either version 1.3
of this license or (at your option) any later version.
The latest version of this license is in
  \url{http://www.latex-project.org/lppl.txt}
and version 1.3 or later is part of all distributions of \LaTeX{}
version 2005/12/01 or later.

This work has the LPPL maintenance status `maintained'.

The Current Maintainer of this work is Niklas Beisert.

This work consists of the files |README.txt|, |childdoc.ins| and |childdoc.dtx|
as well as the derived files |childdoc.def|, |cdocsamp.tex|
with |cdocsch1.tex|, |cdocsch2.tex|, |cdocspt3.tex|, |cdocspt4.tex|,
|cdocsdrf.tex|, |cdocsfn1.tex|, |cdocsfn2.tex|
as well as |childdoc.pdf|.

%%%%%%%%%%%%%%%%%%%%%%%%%%%%%%%%%%%%%%%%%%%%%%%%%%%%%%%%%%%%%%%%%%%%%%%%%%%%%%%%
\subsection{Files and Installation}

The package consists of the files:
%
\begin{center}
\begin{tabular}{ll}
    |README.txt|   & readme file \\
    |childdoc.ins| & installation file \\
    |childdoc.dtx| & source file \\
    |childdoc.def| & definition file \\
    |cdocsamp.tex| & sample main file \\
    |cdocsch1.tex| & sample include file \\
    |cdocsch2.tex| & sample include file \\
    |cdocspt3.tex| & sample part file \\
    |cdocspt4.tex| & sample part file \\
    |cdocsdrf.tex| & sample redirection file \\
    |cdocsfn1.tex| & sample redirection file \\
    |cdocsfn2.tex| & sample redirection file \\
    |childdoc.pdf| & manual
\end{tabular}
\end{center}
%
The distribution consists of the files
|README.txt|, |childdoc.ins| and |childdoc.dtx|.
%
\begin{itemize}
\item
Run (pdf)\LaTeX{} on |childdoc.dtx|
to compile the manual |childdoc.pdf| (this file).
\item
Run \LaTeX{} on |childdoc.ins| to create the definitions file |childdoc.def|
and the sample |cdocsamp.tex| with include files
|cdocsch1.tex|, |cdocsch2.tex|, |cdocspt3.tex|, |cdocspt4.tex|,
|cdocsdrf.tex|, |cdocsfn1.tex|, |cdocsfn2.tex|.
Then copy the file |childdoc.def| to an appropriate directory of your \LaTeX{}
distribution, e.g.\ \textit{texmf-root}|/tex/latex/childdoc|.
\end{itemize}

%%%%%%%%%%%%%%%%%%%%%%%%%%%%%%%%%%%%%%%%%%%%%%%%%%%%%%%%%%%%%%%%%%%%%%%%%%%%%%%%
\subsection{Related CTAN Packages}

There are several other packages which offer a similar functionality:
%
\begin{itemize}
\item
The packages
\href{http://ctan.org/pkg/docmute}{\textsf{docmute}},
\href{http://ctan.org/pkg/includex}{\textsf{includex}} and
\href{http://ctan.org/pkg/standalone}{\textsf{standalone}}
provide commands to include only the document body of
a child file thus allowing both files to be compiled individually.
\item
The packages \href{http://ctan.org/pkg/subdocs}{\textsf{subdocs}}
and \href{http://ctan.org/pkg/subfiles}{\textsf{subfiles}}
provide structures in which the main and child documents can be
encapsulated and allowing them to be compiled individually.
The inclusion mechanism is different from the conventional |\include|.
\item
The package \href{http://ctan.org/pkg/combine}{\textsf{combine}}
is an elaborate solution to combine several documents into one.
\end{itemize}
%
See also the CTAN topic \href{http://ctan.org/topic/subdocs}{\textsf{subdocs}}
for further related packages.
The present package differs from the above solutions in that
a document structure constructed with the conventional |\include| mechanism
just needs two extra commands at the top of every file
such that all constituent files can be compiled individually.

%%%%%%%%%%%%%%%%%%%%%%%%%%%%%%%%%%%%%%%%%%%%%%%%%%%%%%%%%%%%%%%%%%%%%%%%%%%%%%%%
%\subsection{Feature Suggestions}
%
%The following is a list of features which may be useful for future
%versions of this package:
%%
%\begin{itemize}
%\item
%\ldots
%\end{itemize}

%%%%%%%%%%%%%%%%%%%%%%%%%%%%%%%%%%%%%%%%%%%%%%%%%%%%%%%%%%%%%%%%%%%%%%%%%%%%%%%%
\subsection{Revision History}

%%%%%%%%%%%%%%%%%%%%%%%%%%%%%%%%%%%%%%%%
\paragraph{v2.0:} 2018/12/30

\begin{itemize}
\item
immediate forward processing
\item
added |\childdocby| mechanism
\item
manual restructured
\end{itemize}

%%%%%%%%%%%%%%%%%%%%%%%%%%%%%%%%%%%%%%%%
\paragraph{v1.6:} 2018/01/17

\begin{itemize}
\item
application for development of include files
\item
corrections to manual
\end{itemize}

%%%%%%%%%%%%%%%%%%%%%%%%%%%%%%%%%%%%%%%%
\paragraph{v1.5:} 2017/05/21

\begin{itemize}
\item
more complete structuring introduced
\item
|\childdocof| introduced
\item
|\childdoc| renamed to |\childdocmain|
\item
|\childredirect| renamed to |\childdocforward| and |\childdocforwardprefix|
and functionality expanded
\end{itemize}

%%%%%%%%%%%%%%%%%%%%%%%%%%%%%%%%%%%%%%%%
\paragraph{v1.0:} 2017/04/27

\begin{itemize}
\item
manual and install package
\item
first version published on CTAN
\end{itemize}

%%%%%%%%%%%%%%%%%%%%%%%%%%%%%%%%%%%%%%%%
\paragraph{v0.6:} 2017/04/26

\begin{itemize}
\item
redirection mechanism added
\end{itemize}

%%%%%%%%%%%%%%%%%%%%%%%%%%%%%%%%%%%%%%%%
\paragraph{v0.5:} 2017/04/26

\begin{itemize}
\item
functionality in definition file
\end{itemize}


%%%%%%%%%%%%%%%%%%%%%%%%%%%%%%%%%%%%%%%%%%%%%%%%%%%%%%%%%%%%%%%%%%%%%%%%%%%%%%%%
%%%%%%%%%%%%%%%%%%%%%%%%%%%%%%%%%%%%%%%%%%%%%%%%%%%%%%%%%%%%%%%%%%%%%%%%%%%%%%%%
%%%%%%%%%%%%%%%%%%%%%%%%%%%%%%%%%%%%%%%%%%%%%%%%%%%%%%%%%%%%%%%%%%%%%%%%%%%%%%%%
\appendix

\settowidth\MacroIndent{\rmfamily\scriptsize 000\ }

 \DocInput{childdoc.dtx}

\end{document}
%</driver>
% \fi
%
% %%%%%%%%%%%%%%%%%%%%%%%%%%%%%%%%%%%%%%%%%%%%%%%%%%%%%%%%%%%%%%%%%%%%%%%%%%%%%%
% %%%%%%%%%%%%%%%%%%%%%%%%%%%%%%%%%%%%%%%%%%%%%%%%%%%%%%%%%%%%%%%%%%%%%%%%%%%%%%
% \section{Sample}
%\iffalse
%<*samplemain>
%\fi
%
% The following presents a sample document
% with two chapters, two parts, a title page,
% a compile flag as well as three forwarding files to set the flag.
% It consists of eight |.tex| files:
% \begin{center}
% \begin{tabular}{ll}
% |cdocsamp.tex|&main file\\
% |cdocsch1.tex|&include file for chapter 1\\
% |cdocsch2.tex|&include file for chapter 2\\
% |cdocspt3.tex|&include file for part 3\\
% |cdocspt4.tex|&include file for part 4\\
% |cdocsdrf.tex|&forwarding file for main file in draft mode\\
% |cdocsfi1.tex|&forwarding file for final version of chapter 1\\
% |cdocsfi2.tex|&forwarding file for final version of chapter 2\\
% \end{tabular}
% \end{center}
% Each of the eight files can be compiled directly by the \LaTeX{} compiler.
%
% %%%%%%%%%%%%%%%%%%%%%%%%%%%%%%%%%%%%%%
% \paragraph{Main File.}
%
% The main file is called |cdocsamp.tex|.
%
% Load the \textsf{childdoc} definitions and
% declare the filename for the main document:
%    \begin{macrocode}
\input{childdoc.def}
\childdocmain{}
%    \end{macrocode}

% Optional override for |\version| flag:
%    \begin{macrocode}
%%\ifchilddoc\else\providecommand{\version}{draft}\fi
%    \end{macrocode}

% Define the default values for the |\version| flag
% (|final| for the main file and |draft| for childs):
%    \begin{macrocode}
\ifchilddoc
\providecommand{\version}{draft}
\else
\providecommand{\version}{final}
\fi
%    \end{macrocode}

% Load the standard document class:
%    \begin{macrocode}
\documentclass[12pt]{article}
%    \end{macrocode}

% Start the document body:
%    \begin{macrocode}
\begin{document}
%    \end{macrocode}

% Declare a title page.
% Print title, part of document being processed and version flag:
%    \begin{macrocode}
\addtocounter{page}{-1}
\begin{center}
{\LARGE\bfseries{}childdoc example\par}
\vspace{1cm}
\ifchilddoc
\ifchilddocmanual part\else chapter\fi:
`\childdocname' of `\childdocjob'\par
\else
main document: `\childdocjob'\par
\fi
version: \version\par
\end{center}
\newpage
%    \end{macrocode}

% Manually include selected file,
% otherwise process as usual:
%    \begin{macrocode}
\ifchilddocmanual
\section*{part `\childdocname'}
\input{\childdocname}
\else
%    \end{macrocode}

% Include the two chapters:
%    \begin{macrocode}
\include{cdocsch1}
\include{cdocsch2}
%    \end{macrocode}

% Include the two parts unless only chapters should be displayed:
%    \begin{macrocode}
\ifchilddoc\else
\section{part three}
\input{cdocspt3}
\section{part four}
\input{cdocspt4}
\fi
%    \end{macrocode}

% Process as usual until here:
%    \begin{macrocode}
\fi
%    \end{macrocode}

% End of document body:
%    \begin{macrocode}
\end{document}
%    \end{macrocode}
%\iffalse
%</samplemain>
%\fi
%
% %%%%%%%%%%%%%%%%%%%%%%%%%%%%%%%%%%%%%%
% \paragraph{Chapter Include Files.}
%
% The include files are called |cdocsch1.tex| and |cdocsch2.tex|.
%
%\iffalse
%<*samplechap1|samplechap2>
%\fi

% Optional override for |\version| flag:
%    \begin{macrocode}
%%\providecommand{\version}{final}
%    \end{macrocode}

% Include the main document:
%    \begin{macrocode}
\input{childdoc.def}
\childdocof{cdocsamp}
%    \end{macrocode}

%\iffalse
%</samplechap1|samplechap2>
%\fi
%
%\iffalse
%<*samplechap1>
%\fi
% Some text for chapter 1:
%    \begin{macrocode}
\section{one}
some text in chapter one
%    \end{macrocode}

%\iffalse
%</samplechap1>
%\fi
% Some text for chapter 2:
%\iffalse
%<*samplechap2>
%\fi
%    \begin{macrocode}
\section{two}
more text in chapter two
%    \end{macrocode}

%\iffalse
%</samplechap2>
%\fi
%
% %%%%%%%%%%%%%%%%%%%%%%%%%%%%%%%%%%%%%%
% \paragraph{Part Include Files.}
%
% The include files are called |cdocspt3.tex| and |cdocspt4.tex|.
%
%\iffalse
%<*samplepart3|samplepart4>
%\fi

% Optional override for |\version| flag:
%    \begin{macrocode}
%%\providecommand{\version}{final}
%    \end{macrocode}

% Include the main document:
%    \begin{macrocode}
\input{childdoc.def}
\childdocby{cdocsamp}
%    \end{macrocode}

%\iffalse
%</samplepart3|samplepart4>
%\fi
%
%\iffalse
%<*samplepart3>
%\fi
% Some text for part 3:
%    \begin{macrocode}
some text in part three
%    \end{macrocode}

%\iffalse
%</samplepart3>
%\fi
% Some text for part 4:
%\iffalse
%<*samplepart4>
%\fi
%    \begin{macrocode}
more text in part four
%    \end{macrocode}

%\iffalse
%</samplepart4>
%\fi
%
% %%%%%%%%%%%%%%%%%%%%%%%%%%%%%%%%%%%%%%
% \paragraph{Forwarding for a Complete Draft.}
%
% The following forwarding file |cdocsdrf.tex|
% compiles the main document in draft mode:
%\iffalse
%<*sampledraft>
%\fi
%    \begin{macrocode}
\def\version{draft}
\input{childdoc.def}
\childdocforward{cdocsamp}
%    \end{macrocode}

%\iffalse
%</sampledraft>
%\fi
%
% %%%%%%%%%%%%%%%%%%%%%%%%%%%%%%%%%%%%%%
% \paragraph{Forwarding for Final Version of the Chapters.}
%
% The following forwarding files |cdocsfn1.tex| and |cdocsfn2.tex|
% (with identical content)
% compile the final versions of the child documents
% |cdocsch1.tex| and |cdocsch2.tex|, respectively:
%\iffalse
%<*samplefinal>
%\fi
%    \begin{macrocode}
\def\version{final}
\input{childdoc.def}
\childdocforwardprefix[cdocsamp]{cdocsfn}{cdocsch}
%    \end{macrocode}

%\iffalse
%</samplefinal>
%\fi
%
% %%%%%%%%%%%%%%%%%%%%%%%%%%%%%%%%%%%%%%
% \paragraph{Command Line Processing.}
%
% The following three command lines generate the output files
% |cdocscld|, |cdocscl1| and |cdocscl2|
% which should be identical to
% |cdocsdrf|, |cdocsch1| and |cdocsfn2|, respectively:
% \begin{center}
% \begin{tabular}{l}
% |latex -jobname cdocscld \|\\
% |  "\def\version{draft}\input{childdoc.def}\childdocforward{cdocsamp}"|\\
% |latex -jobname cdocscl1 \|\\
% |  "\input{childdoc.def}\childdocforward[cdocsamp]{cdocsch1}"|\\
% |latex -jobname cdocscl2 \|\\
% |  "\def\version{final}\input{childdoc.def}\childdocforward{cdocsch2}"|
% \end{tabular}
% \end{center}
% Note that the trailing backslash on each first line
% merely continues the input to the second line
% (for convenient cut ant paste).
% Furthermore, the command |latex| can be replaced by any
% of its alternative versions such as |pdflatex|.
%
% %%%%%%%%%%%%%%%%%%%%%%%%%%%%%%%%%%%%%%%%%%%%%%%%%%%%%%%%%%%%%%%%%%%%%%%%%%%%%%
% %%%%%%%%%%%%%%%%%%%%%%%%%%%%%%%%%%%%%%%%%%%%%%%%%%%%%%%%%%%%%%%%%%%%%%%%%%%%%%
% \section{Implementation}
%\iffalse
%<*package>
%\fi
%
% This section describes the definitions file |childdoc.def|.

% The definitions cannot be loaded using |\usepackage| or |\RequirePackage|
% which has a mechanism to prevent loading a style file more than once.
% When loading the definitions by means of |\input|
% multiple instances have to be prevented manually:
%\iffalse
%This code needs to be before the `\ProvidesFile' directive
%which is defined at the beginning of this file.
%Therefore it is also placed there and commented out here.
%</package>
%<*discard>
%\fi
%    \begin{macrocode}
\ifdefined\childdocmain\endinput\fi
%    \end{macrocode}
%\iffalse
%</discard>
%<*package>
%\fi
%
% \macro{\ifchilddoc}
% \macro{\ifchilddocmanual}
% The conditional |\ifchilddoc| tells whether a
% child (true) or main (false) document is being compiled.
% The conditional |\ifchilddocmanual| tells whether
% the |\includeonly| mechanism is used (false) or
% the selection of child files must be performed manually (true).
% The definitions initialise to false:
%    \begin{macrocode}
\newif\ifchilddoc
\newif\ifchilddocmanual
%    \end{macrocode}

% \macro{\childdocname}
% \macro{\childdocjob}
% The macro |\childdocname| stores the name of the main document
% to be compiled. The macro |\childdocjob| stores the name of
% the document on which the \LaTeX{} compiler was originally invoked.
% The content of |\jobname| cannot be compared
% to filenames specified in the source due to different catcodes.
% The following code rescans |\jobname|, stores the result
% in |\childdocname| and saves a copy in |\childdocjob|:
%    \begin{macrocode}
\edef\childdocname{\scantokens\expandafter{\jobname\noexpand}}
\let\childdocjob\childdocname
%    \end{macrocode}

% \macro{\childdocdisable}
% The macro |\childdocdisable| prevents the main file
% from being processed more than once.
% At this stage, the main document command |\childdocmain|
% is assumed to be called once again where it should do nothing.
% Any subsequent call to it should prevent
% a secondary processing of the main document
% It overwrites the forwarding commands
% |\childdocof| and |\childdocforward|
% with empty macros to prevent further inclusions of the main document:
%    \begin{macrocode}
\newcommand{\childdocdisable}
{
  \renewcommand{\childdocmain}[1]{\renewcommand{\childdocmain}[1]{\endinput}}
  \renewcommand{\childdocof}[1]{}
  \renewcommand{\childdocby}[2][]{}
  \renewcommand{\childdocforward}[2][]{}
  \renewcommand{\childdocdisable}{}
}
%    \end{macrocode}

% \macro{\childdocmain}
% The macro |\childdocmain| is to be called at the top of the main file
% with nothing or the main filename (without extension) as argument.
% First, it breaks loops.
% If the argument is not empty and does not match |\childdocname|
% (which is set by the first inclusion of |childdoc.def|),
% |\ifchilddoc| is set to true, |\includeonly| is applied to the child file
% and |\jobname| is set to the main file
% (for proper handling of |.aux| files):
%    \begin{macrocode}
\newcommand{\childdocmain}[1]
{
  \childdocdisable\childdocmain{}
  \if?#1?\else
    \begingroup
      \def\childdoctmp{#1}
      \ifx\childdoctmp\childdocname
        \def\childdoctmp{}
      \else
        \def\childdoctmp
        {
          \childdoctrue
          \includeonly{\childdocname}
          \def\childdocjob{#1}
          \def\jobname{#1}
        }
      \fi
      \expandafter
    \endgroup
    \childdoctmp
  \fi
}
%    \end{macrocode}

% \macro{\childdocof}
% The command |\childdocof| redirects
% compilation to the main file |#1|.
%    \begin{macrocode}
\newcommand{\childdocof}[1]
{
  \childdocdisable
  \childdoctrue
  \includeonly{\childdocname}
  \def\jobname{#1}
  \def\childdocjob{#1}
  \input{#1}
}
%    \end{macrocode}

% \macro{\childdocby}
% The command |\childdocby| ....
%    \begin{macrocode}
\newcommand{\childdocby}[2][]
{
  \childdocdisable
  \childdoctrue
  \childdocmanualtrue
  \if?#1?\else
    \def\jobname{#2}
  \fi
  \def\childdocjob{#2}
  \input{#2}
  \endinput
}
%    \end{macrocode}

% \macro{\childdocforward}
% The command |\childdocforward| redirects
% compilation to the main file or
% (if the optional argument is given) a child file.
% Parameters are set as if the main file
% or a child file starting with |\childdocof| was compiled.
% Then compilation is handed over to the main file:
%    \begin{macrocode}
\newcommand{\childdocforward}[2][]
{
  \begingroup
    \if?#1?
      \def\childdoctmp
      {
        \def\childdocname{#2}
        \def\childdocjob{#2}
        \def\jobname{#2}
        \input{#2}
        \endinput
      }
    \else
      \def\childdoctmp
      {
        \childdocdisable
        \def\childdocname{#2}
        \childdoctrue
        \includeonly{#2}
        \def\childdocjob{#1}
        \def\jobname{#1}
        \input{#1}
        \endinput
      }
    \fi
    \expandafter
  \endgroup
  \childdoctmp
}
%    \end{macrocode}

% \macro{\childdocforwardprefix}
% The command |\childdocforwardprefix| redirects
% compilation to the main or a child file by means of a pattern.
% The prefix |#1| in the current filename is replaced by |#2|
% and the suffix of the current filename is kept
% (it is assumed that the filename does not contain the substring `|~~~|'
% which is used as a delimiter).
% Compilation is handed over to the new file by |\childdocforward|:
%    \begin{macrocode}
\newcommand{\childdocforwardprefix}[3][]
{
  \begingroup
    \def\childdocextract #2##1~~~{\def\childdoctmp{\childdocforward[#1]{#3##1}}}
    \expandafter\childdocextract\childdocname~~~
    \expandafter
  \endgroup
  \childdoctmp
}
%    \end{macrocode}

% \macro{\childdoc}
% The deprecated macro |\childdoc| is a legacy version of |\childdocmain|:
%    \begin{macrocode}
\newcommand{\childdoc}{\childdocmain}
%    \end{macrocode}

% \macro{\childdocredirect}
% The deprecated macro |\childdocredirect| is a legacy version
% of |\childdocforward| and |\childdocforwardprefix|:
%    \begin{macrocode}
\newcommand{\childdocredirect}[2][]
{
  \begingroup
    \if?#1?
      \def\childdoctmp{\childdocforward{#2}}
    \else
      \def\childdoctmp{\childdocforwardprefix{#1}{#2}}
    \fi
    \expandafter
  \endgroup
  \childdoctmp
}
%    \end{macrocode}

%\iffalse
%</package>
%\fi
%
\endinput
|\\
|\childdocforward{|\textit{main}|}|\\
\end{tabular}
\end{center}
%
or alternatively with:
%
\begin{center}
\begin{tabular}{l}
|% \iffalse
%
% childdoc.dtx Copyright (C) 2017-2018 Niklas Beisert
%
% This work may be distributed and/or modified under the
% conditions of the LaTeX Project Public License, either version 1.3
% of this license or (at your option) any later version.
% The latest version of this license is in
%   http://www.latex-project.org/lppl.txt
% and version 1.3 or later is part of all distributions of LaTeX
% version 2005/12/01 or later.
%
% This work has the LPPL maintenance status `maintained'.
%
% The Current Maintainer of this work is Niklas Beisert.
%
% This work consists of the files childdoc.dtx and childdoc.ins
% and the derived files childdoc.def and cdocsamp.tex with
% cdocsch1.tex, cdocsch2.tex, cdocsdrf.tex, cdocsfn1.tex, cdocsfn2.tex.
%
%<package>\ifdefined\childdocmain\endinput\fi
%<package>\ProvidesFile{childdoc.def}[2018/12/30 v2.0 child document driver]
%<samplemain>\ProvidesFile{cdocsamp.tex}[2018/12/30 v2.0 sample for childdoc]
%<*driver>
%\ProvidesFile{childdoc.drv}[2018/12/30 v2.0 childdoc reference manual file]
\PassOptionsToClass{10pt,a4paper}{article}
\documentclass{ltxdoc}

\usepackage[margin=35mm]{geometry}
\usepackage{hyperref}
\usepackage{hyperxmp}
\usepackage[usenames]{color}

\hypersetup{colorlinks=true}
\hypersetup{pdfstartview=FitH}
\hypersetup{pdfpagemode=UseNone}
\hypersetup{pdfsource={}}
\hypersetup{pdflang={en-UK}}
\hypersetup{pdfcopyright={Copyright 2017-2018 Niklas Beisert.
  This work may be distributed and/or modified under the
  conditions of the LaTeX Project Public License, either version 1.3
  of this license or (at your option) any later version.}}
\hypersetup{pdflicenseurl={http://www.latex-project.org/lppl.txt}}
\hypersetup{pdfcontactaddress={ETH Zurich, ITP, HIT K,
  Wolfgang-Pauli-Strasse 27}}
\hypersetup{pdfcontactpostcode={8093}}
\hypersetup{pdfcontactcity={Zurich}}
\hypersetup{pdfcontactcountry={Switzerland}}
\hypersetup{pdfcontactemail={nbeisert@itp.phys.ethz.ch}}
\hypersetup{pdfcontacturl={http://people.phys.ethz.ch/\xmptilde nbeisert/}}

\newcommand{\secref}[1]{\hyperref[#1]{section \ref*{#1}}}

\parskip1ex
\parindent0pt
\let\olditemize\itemize
\def\itemize{\olditemize\parskip0pt}

\begin{document}

\title{The \textsf{childdoc} Package}
\hypersetup{pdftitle={The childdoc Package}}
\author{Niklas Beisert\\[2ex]
  Institut f\"ur Theoretische Physik\\
  Eidgen\"ossische Technische Hochschule Z\"urich\\
  Wolfgang-Pauli-Strasse 27, 8093 Z\"urich, Switzerland\\[1ex]
  \href{mailto:nbeisert@itp.phys.ethz.ch}
  {\texttt{nbeisert@itp.phys.ethz.ch}}}
\hypersetup{pdfauthor={Niklas Beisert}}
\hypersetup{pdfsubject={Manual for the LaTeX2e Package childdoc}}
\date{30 December 2018, \textsf{v2.0}}
\maketitle

\begin{abstract}\noindent
\textsf{childdoc} is a \LaTeXe{} package
that enables the direct compilation
of document sections included by |\include|
to individual files.
\end{abstract}

\begingroup
\parskip0ex
\tableofcontents
\endgroup

%%%%%%%%%%%%%%%%%%%%%%%%%%%%%%%%%%%%%%%%%%%%%%%%%%%%%%%%%%%%%%%%%%%%%%%%%%%%%%%%
%%%%%%%%%%%%%%%%%%%%%%%%%%%%%%%%%%%%%%%%%%%%%%%%%%%%%%%%%%%%%%%%%%%%%%%%%%%%%%%%
\section{Introduction}

\LaTeX{} provides a mechanism to structure a large document (such as a book)
into a main file and several child files (containing the chapters)
using the |\include| command.
This mechanism is beneficial for documents
which span hundreds of pages in order to
make the source file(s) more manageable.
Moreover, compilation can be restricted to
selected child files by means of the |\includeonly| command.
The latter feature can be used to reduce the compilation time while editing
(this was significantly more useful in the earlier days of \LaTeX{})
or to generate a smaller document which is easier to navigate.
Another application of |\includeonly| is to generate
documents consisting of selected parts of the complete document.

However, there are a few drawbacks of the plain |\include| mechanism:
\begin{itemize}
\item
The child files cannot be compiled on their own,
they can only be compiled via the main file.
A naive editing environment
(such as a text editor with an option
to have the current file processed by \LaTeX)
may require one to switch to the main file before compiling;
attempting to compile the child file produces errors.
\item
The main file must be modified (each time)
to adjust the |\includeonly| command
to the present needs. This easily leaves the main file in a messy state.
\item
The generated document will always carry the filename
of the main document. This is inconvenient if
several child files are to be compiled and
to be kept for distribution.
\end{itemize}

The present package provides a simple interface
to make child files individually compilable by \LaTeX{}.
Compiling a child file then has the same effect as compiling
the main file with an |\includeonly| command
to select the appropriate child.
Moreover the generated document will carry the name of the child
rather than the main file.
This resolves all three above issues.

This feature is meant to make the editing of books,
thesis documents and lecture notes somewhat more convenient.
However, the package can also be used efficiently for
composing a series of documents (such as exercise sheets)
which are typically distributed individually.
It then assists the author in generating the individual documents
(potentially in different versions)
as well as a document containing the collected series.
Another application is in developing style files
or other kinds of included material
where compilation of the style file could redirect
to a sample or test file.

%%%%%%%%%%%%%%%%%%%%%%%%%%%%%%%%%%%%%%%%%%%%%%%%%%%%%%%%%%%%%%%%%%%%%%%%%%%%%%%%
%%%%%%%%%%%%%%%%%%%%%%%%%%%%%%%%%%%%%%%%%%%%%%%%%%%%%%%%%%%%%%%%%%%%%%%%%%%%%%%%
\section{Usage}

First of all, the package \textsf{childdoc} is \emph{not} a standard
\LaTeXe{} |.sty| style file! Therefore it needs to be invoked in
a non-standard way.

%%%%%%%%%%%%%%%%%%%%%%%%%%%%%%%%%%%%%%%%%%%%%%%%%%%%%%%%%%%%%%%%%%%%%%%%%%%%%%%%
\subsection{Included Files}
\label{sec:include}

%%%%%%%%%%%%%%%%%%%%%%%%%%%%%%%%%%%%%%%%
\DescribeMacro{\childdocmain}
To use the package, add the commands
\begin{center}
\begin{tabular}{l}
|\input{childdoc.def}|\\
|\childdocmain{}|\\
\end{tabular}
\end{center}
at the very top of the main \LaTeX{} file,
in particular \emph{before} the |\documentclass| statement!
The argument of |\childdocmain| should be left empty
(but it must be present).

%%%%%%%%%%%%%%%%%%%%%%%%%%%%%%%%%%%%%%%%
\DescribeMacro{\childdocof}
Furthermore, add the commands
\begin{center}
\begin{tabular}{l}
|\input{childdoc.def}|\\
|\childdocof{|\textit{main}|}|\\
\end{tabular}
\end{center}
at the top of every child file \textit{child}
which is included by |\include{|\textit{child}|}|
from within the main file
(or at least for those files to be compiled individually).
The argument \textit{main} must be the filename of the main file.

There are a couple of
considerations in setting up the main and child documents:

%%%%%%%%%%%%%%%%%%%%%%%%%%%%%%%%%%%%%%%%
\paragraph{Restrictions.}

Please note the following restrictions:
\begin{itemize}
\item
|\childdocmain| must be called with one argument \textit{main}
to ensure compatibility with earlier version of the package.
It must either be empty (|\childdocmain{}|)
or precisely match the filename of the main file in which it is specified.
See \secref{sec:detection} for further information.
\item
The filename \textit{main} must be specified without the |.tex| extension.
\item
The filename \textit{main} is case sensitive
(even in case-insensitive file systems)
due to internal string comparison.
\item
The argument \textit{main} should be fully expanded, it cannot be a macro.
\item
Subdirectories and special characters should be avoided in filenames.
\item
The command |\childdocmain{|\textit{main}|}| must be followed by a whitespace.
It should not be followed immediately by another command
or by a comment mark `|%|'.
This is because the \TeX{} parser reads the token immediately following
the argument of |\childdocmain| and puts it
at the beginning of every child section;
however, a white\-space is ignored.
\end{itemize}

%%%%%%%%%%%%%%%%%%%%%%%%%%%%%%%%%%%%%%%%
\paragraph{Content of Main File.}

It is advisable to place all content in the child files included by |\include|.
Any output contained in the main file will appear in all child documents
unless suppressed manually;
it cannot be suppressed automatically by the |\includeonly| directive
and thus should normally be avoided.
A method to include some content in the main file
by means of conditional processing is described in \secref{sec:conditional}.

%%%%%%%%%%%%%%%%%%%%%%%%%%%%%%%%%%%%%%%%
\paragraph{Page Numbering.}

When only a part of the document is compiled,
the appropriate numbering of pages
(as well as other status parameters)
is determined from the |.aux| files.
The latter contain information from previous passes.
However this information needs to propagate through
all intermediate child documents.
Therefore the page numbering in child documents may well
be inconsistent until the complete document is compiled at least once.

A useful (if unconventional) way to always ensure a consistent
page numbering is to restart the numbering in each child document
and denote the pages by `\textit{child}|.|\textit{page}'
where \textit{child} represents the chapter/section number of the child file.
This can be achieved by the command
|\numberwithin{page}{|\textit{child}|}|
of the \textsf{amsmath} package
where \textit{child} can be |chapter| or |section|
depending on the chosen structuring.
Alternatively, one can modify the macro |\thepage| appropriately
and reset the counter |page| at the start of each child file.

%%%%%%%%%%%%%%%%%%%%%%%%%%%%%%%%%%%%%%%%%%%%%%%%%%%%%%%%%%%%%%%%%%%%%%%%%%%%%%%%
\subsection{Conditional Processing}
\label{sec:conditional}

The package provides a mechanism to compile different versions
of a document. To customise the versions further some conditional processing
can come in handy to distinguish which version is being compiled.
The package provides two macros to describe the compilation context:

%%%%%%%%%%%%%%%%%%%%%%%%%%%%%%%%%%%%%%%%
\DescribeMacro{\ifchilddoc}
The conditional |\ifchilddoc| distinguishes between the compilation of
child documents and the main document:
%
\begin{center}
|\ifchilddoc |\textit{child-code}| |[|\||else |\textit{main-code}]| \||fi|
\end{center}

%%%%%%%%%%%%%%%%%%%%%%%%%%%%%%%%%%%%%%%%
\DescribeMacro{\childdocname}
\DescribeMacro{\childdocjob}
The macro |\childdocname| contains the filename (without extension)
of the main or child file being processed.
Note that |\childdocjob| will always contain the name of the main file.

%%%%%%%%%%%%%%%%%%%%%%%%%%%%%%%%%%%%%%%%
\paragraph{Title Page.}

Conditional processing can be used to include a title or banner page
in the main document when proper precautions are taken.
Importantly, the code in the main file should ensure that the page counter
(as well as other status parameters which are stored in the |.aux| files)
takes the same value after the conditional processing.
Otherwise the page numbers may take divergent values
depending on which part is compiled.

For example, a title page could be declared by:
%
\begin{center}
\begin{tabular}{l}
|\ifchilddoc\||else|\\
|\addtocounter{page}{-1}|\\
\textit{code for title page}\\
|\newpage|\\
|\||fi|
\end{tabular}
\end{center}
%
A banner page for the child documents can be generated by:
%
\begin{center}
\begin{tabular}{l}
|\ifchilddoc|\\
|\addtocounter{page}{-1}|\\
\textit{code for banner page}\\
|\newpage|\\
|\||fi|
\end{tabular}
\end{center}
%
Here one could write a message such as:
\begin{center}
|This is the part \childdocname{} of \childdocjob{}.|
\end{center}

%%%%%%%%%%%%%%%%%%%%%%%%%%%%%%%%%%%%%%%%%%%%%%%%%%%%%%%%%%%%%%%%%%%%%%%%%%%%%%%%
\subsection{Flags}
\label{sec:flags}

The package makes it easy to generate different versions
of the main or child documents.
To this end compilation flags can be defined
and assigned different default values.
They will be particularly useful in conjunction
with the forwarding mechanism described in \secref{sec:forward}.

For example, it may be useful to have a flag |\version|
which can be set to |draft| or |final|.
The document source will contain some conditional code
depending on the value of |\version|.
Suppose further, the flag should default to |final| for the main file
and to |draft| for child files
which is a natural assignment for editing the document.
This is achieved by placing the following code
in the preamble of the main document
(below the |\childdocmain| directive):
%
\begin{center}
\begin{tabular}{l}
|\ifchilddoc|\\
|\providecommand{\version}{draft}|\\
|\||else|\\
|\providecommand{\version}{final}|\\
|\||fi|
\end{tabular}
\end{center}
%
The definition by |\providecommand| makes sure
that previous definitions are not overwritten.
Further statements |\providecommand{\version}{...}|
can thus be added before the above code to override it.

For the main file, one might add a line
(between |\childdocmain| and the above block)
%
\begin{center}
|%\ifchilddoc\||else\providecommand{\version}{draft}\||fi|
\end{center}
%
which can be uncommented to produce a draft version.
Likewise one can add a line to the very top of a child file
(above the |\childdocof{|\textit{main}|}| directive)
%
\begin{center}
|%\providecommand{\version}{final}|
\end{center}
%
which can be uncommented to produce the final version of this child document.

%%%%%%%%%%%%%%%%%%%%%%%%%%%%%%%%%%%%%%%%%%%%%%%%%%%%%%%%%%%%%%%%%%%%%%%%%%%%%%%%
\subsection{Forwarding}
\label{sec:forward}

Different versions of the main or child documents
using compilation flags as described in \secref{sec:flags}
can be (permanently) stored in different files
for convenient compilation, viewing and distribution.
To this end, the package defines a command
to pass on compilation to a different file:

%%%%%%%%%%%%%%%%%%%%%%%%%%%%%%%%%%%%%%%%
\DescribeMacro{\childdocforward}
The command |\childdocforward| redirects processing to
another source file:
%
\begin{center}
\begin{tabular}{l}
|\input{childdoc.def}|\\
|\childdocforward[|\textit{main}|]{|\textit{dest}|}|\\
\end{tabular}
\end{center}
%
The argument \textit{dest} is the destination file
(without extension).
It should be the main file or one of the child files.
Note that further \textsf{childdoc} directives
such as |\childdocof| and |\childdocforward|
in the indicated file will be processed in this form.
The optional argument \textit{main}
passes on directly to the main file \textit{main}
while pretending to compile the child \textit{dest}.
This form behaves as if \textit{dest}
issues |\childdocof{|\textit{main}|}| right away,
and no further \textsf{childdoc} directives will be processed.

%%%%%%%%%%%%%%%%%%%%%%%%%%%%%%%%%%%%%%%%
\DescribeMacro{\...prefix}
In the alternative form |\childdocforwardprefix|,
%
\begin{center}
\begin{tabular}{l}
|\input{childdoc.def}|\\
|\childdocforwardprefix[|\textit{main}|]{|\textit{prefix}|}{|\textit{dest}|}|
\end{tabular}
\end{center}
%
the destination file is determined by a pattern
depending on the current file:
To make this work, the current file must be called
`{\textit{prefix}\hspace{0.2em}\textit{suffix}}'
with \textit{prefix} matching precisely the argument.
Processing is then passed on to the file
`{\textit{dest}\hspace{0.2em}\textit{suffix}}'.
Surely, the same effect is achieved by
directly specifying the
argument `{\textit{dest}\hspace{0.2em}\textit{suffix}}'
in the first form.
However, that requires to set up a different file
for each child. With the alternative form of the command
all these files can have exactly the same content
which simplifies setting them up and maintaining them.

For example, the following file |draft.tex|
with a compilation flag |\version| as described in \secref{sec:flags}
compiles the main document as a draft:
%
\begin{center}
\begin{tabular}{l}
|\def\version{draft}|\\
|\input{childdoc.def}|\\
|\childdocforward{|\textit{main}|}|
\end{tabular}
\end{center}
%
Likewise, the following files |final|\textit{nn}|.tex|
compile the final version of the child document
|child|\textit{nn}|.tex|:
%
\begin{center}
\begin{tabular}{l}
|\def\version{final}|\\
|\input{childdoc.def}|\\
|\childdocforwardprefix{final}{child}|
\end{tabular}
\end{center}
%

Note that when several versions of a main file and/or of each child file
are to be generated, it may be convenient to set up a |Makefile| or
shell script to automatise the process.

%%%%%%%%%%%%%%%%%%%%%%%%%%%%%%%%%%%%%%%%%%%%%%%%%%%%%%%%%%%%%%%%%%%%%%%%%%%%%%%%
\subsection{Command Line Processing}
\label{sec:commandline}

The effect of redirection files can also be achieved by invoking
the \LaTeX{} compiler with a more elaborate command line.
Most conveniently this should be done as part
of a shell script or a |Makefile|.

When using \textsf{childdoc} in the main file, the following
command lines effectively perform a redirection
(note that depending on the shell being used,
backslashes may have to be doubled: `|\|' $\to$ `|\\|'):
%
\begin{center}
|... -jobname "|\textit{target}|" |\\|"|[\textit{flags}]%
|\input{childdoc.def}\childdocforward[|\textit{main}|]{|\textit{dest}|}"|
\end{center}
%
Here \textit{target} is the name of the output file,
\textit{main} is the name of the main file
and \textit{dest} is the name of the main or child file to be processed
(all filenames without extensions).
The optional argument \textit{main} can be omitted
if \textit{main} matches \textit{dest}.
Optionally, compilation \textit{flags} can be defined via |\def| commands.
This command line makes the \TeX{} engine believe
it is compiling the file \textit{target}
whose content is specified as the latter parameter.
The provided code then forwards the processing to
\textit{main} or \textit{dest} as described in \secref{sec:forward}.

%%%%%%%%%%%%%%%%%%%%%%%%%%%%%%%%%%%%%%%%%%%%%%%%%%%%%%%%%%%%%%%%%%%%%%%%%%%%%%%%
\subsection{Include by Input}
\label{sec:input}

Including child documents by |\include| has some restrictions by design.
Most notably, the content of a child document always occupies
its own set of pages; pages cannot be shared between child documents.
Usually, this behaviour makes perfect sense
because each child document contain an essential part of the document.
However, in some situations it may be desirable to compose
a document from a collection of parts
without having mandatory page breaks between then.
For this case, the package
provides a mechanism to include parts
by |\input| which can also be processed individually.
However, by construction this mechanism
requires manual handling of the content to be output.

%%%%%%%%%%%%%%%%%%%%%%%%%%%%%%%%%%%%%%%%
\DescribeMacro{\ifchilddocmanual}
The main file should be prepared as usual, see \secref{sec:include}.
However, the document body must make a distinction
between processing of an individual part and of the main document, e.g.:
%
\begin{center}
\begin{tabular}{l}
|\ifchilddocmanual|\\
|\input{\childdocname}|\\
|\||else|\\
\textit{document body with }|\input{|\textit{part}|}|\\
|\||fi|
\end{tabular}
\end{center}
%
The conditional |\ifchilddocmanual| is true whenever
a part to be included by |\input| is being compiled,
and the name of the part is stored in |\childdocname|.

%%%%%%%%%%%%%%%%%%%%%%%%%%%%%%%%%%%%%%%%
\DescribeMacro{\childdocby}
Each part to be included by |\input| should start with:
%
\begin{center}
\begin{tabular}{l}
|\input{childdoc.def}|\\
|\childdocby{|\textit{main}|}|\\
\end{tabular}
\end{center}
%
The directive |\childdocby| is similar to |\childdocof|
described in \secref{sec:include},
but the subsequent selection of content must be done manually.
To that end, both |\ifchilddoc| and |\ifchilddocmanual|
will be true upon processing of a part,
and the name of the part is stored in |\childdocname|.
Note that |\jobname| will be set to the filename of the current part
so that each part receives an individual |.aux| file
that does not interfere with the |.aux| file(s) of the main document.
This behaviour can be altered by the alternative form
|\childdocby[*]{|\textit{main}|}| (with a non-empty optional argument)
which uses the |.aux| file of the main document
by setting |\jobname| to \textit{main}.

%%%%%%%%%%%%%%%%%%%%%%%%%%%%%%%%%%%%%%%%%%%%%%%%%%%%%%%%%%%%%%%%%%%%%%%%%%%%%%%%
\subsection{Driver Development}
\label{sec:driver}

The \textsf{childdoc} mechanism can also be use for the development
of definition files such as \LaTeX{} styles or classes.
This case differs from the above setup with multiple parts
included by |\include| in that no |\includeonly| should be invoked.
This can be achieved by starting the include file
(before |\ProvidesPackage|) with:
%
\begin{center}
\begin{tabular}{l}
|\input{childdoc.def}|\\
|\childdocforward{|\textit{main}|}|\\
\end{tabular}
\end{center}
%
or alternatively with:
%
\begin{center}
\begin{tabular}{l}
|\input{childdoc.def}|\\
|\childdocby{|\textit{main}|}|\\
\end{tabular}
\end{center}
%
Both forms have slightly different effects as described above.
The main file is prepared as usual, see \secref{sec:include}.

%%%%%%%%%%%%%%%%%%%%%%%%%%%%%%%%%%%%%%%%%%%%%%%%%%%%%%%%%%%%%%%%%%%%%%%%%%%%%%%%
\subsection{Legacy Detection}
\label{sec:detection}

The directive |\childdocmain| in the main file can detect
whether the complete document or merely a child is to be compiled
even without using the directive |\childdocof|.
This method is deprecated because it is less robust
and there is no compelling reason to use it;
it is merely provided for backward compatibility
and it may be removed in future versions.

If the detection mechanism is to be used,
it is mandatory to correctly specify
the filename of the main file as the argument of |\childdocmain|:
%
\begin{center}
\begin{tabular}{l}
|\input{childdoc.def}|\\
|\childdocmain{|\textit{main}|}|\\
\end{tabular}
\end{center}
%
If |\jobname| does not match the argument \textit{main} of |\childdocmain|,
it is assumed that |\jobname| points to the child file to be compiled.
When using |\childdocmain| with the main file specified as argument,
it suffices to start a child file
with just |\input{|\textit{main}|}|
without loading of the package and using |\childdocof|.
If instead all processing is done
with the appropriate \textsf{childdoc} directives,
the argument of \textit{main} of |\childdocmain| can be empty.

An alternative version of the command line processing described
in \secref{sec:commandline} using the detection mechanism reads:
%
\begin{center}
|... -jobname "|\textit{target}|" "|[\textit{flags}]%
[|\def\jobname{|\textit{dest}|}|]|\input{|\textit{main}|}"|
\end{center}

%%%%%%%%%%%%%%%%%%%%%%%%%%%%%%%%%%%%%%%%%%%%%%%%%%%%%%%%%%%%%%%%%%%%%%%%%%%%%%%%
\subsection{Manual Code}
\label{sec:manual}

In case one cannot be certain whether the definitions file |childdoc.def|
is installed on the target \TeX{} distribution
and one prefers not to ship it,
it is conceivable to paste a few relevant commands into the sources.

To that end, drop all statements |\input{childdoc.def}|
and perform the replacements as outlined below.
Instead of |\childdocmain{|\textit{main}|}| add the following code
to the top of the main file:
%
\begin{center}
\begin{tabular}{l}
|\||ifdefined\childdocname\endinput\||fi\newif\ifchilddoc|\\
|\edef\childdocname{\scantokens\expandafter{\jobname\noexpand}}|\\
|\def\childdocmain{|\textit{main}|}\||ifx\childdocmain\childdocname\||else|\\
|\childdoctrue\includeonly{\childdocname}\let\jobname\childdocmain\||fi|\\
\end{tabular}
\end{center}
%
Instead of |\childdocof{|\textit{main}|}| just include the main file
at the top of each child file:
%
\begin{center}
|\input{|\textit{main}|}|
\end{center}
%
A simple redirection |\childdocforward{|\textit{dest}|}| is achieved by:
%
\begin{center}
|\def\jobname{|\textit{dest}|}\input{\jobname}|
\end{center}
%
The redirection with prefix
|\childdocforwardprefix[|\textit{prefix}|]{|\textit{dest}|}|
is accomplished by:
%
\begin{center}
\begin{tabular}{l}
|{\edef\jobname{\scantokens\expandafter{\jobname\noexpand}}|\\
|\def\redirectjob |\textit{prefix}|#1~~~{\gdef\jobname{|\textit{dest}|#1}}|\\
|\expandafter\redirectjob\jobname~~~}\input{\jobname}|
\end{tabular}
\end{center}

In an alternative approach,
child documents can be compiled by a specific command line
without additional code or specific definitions:
%
\begin{center}
|... -jobname "|\textit{target}|" "|[\textit{flags}]%
|\includeonly{|\textit{dest}|}\input{|\textit{main}|}"|
\end{center}
%

%%%%%%%%%%%%%%%%%%%%%%%%%%%%%%%%%%%%%%%%%%%%%%%%%%%%%%%%%%%%%%%%%%%%%%%%%%%%%%%%
%%%%%%%%%%%%%%%%%%%%%%%%%%%%%%%%%%%%%%%%%%%%%%%%%%%%%%%%%%%%%%%%%%%%%%%%%%%%%%%%
\section{Information}

%%%%%%%%%%%%%%%%%%%%%%%%%%%%%%%%%%%%%%%%%%%%%%%%%%%%%%%%%%%%%%%%%%%%%%%%%%%%%%%%
\subsection{Copyright}

Copyright \copyright{} 2017--2018 Niklas Beisert

This work may be distributed and/or modified under the
conditions of the \LaTeX{} Project Public License, either version 1.3
of this license or (at your option) any later version.
The latest version of this license is in
  \url{http://www.latex-project.org/lppl.txt}
and version 1.3 or later is part of all distributions of \LaTeX{}
version 2005/12/01 or later.

This work has the LPPL maintenance status `maintained'.

The Current Maintainer of this work is Niklas Beisert.

This work consists of the files |README.txt|, |childdoc.ins| and |childdoc.dtx|
as well as the derived files |childdoc.def|, |cdocsamp.tex|
with |cdocsch1.tex|, |cdocsch2.tex|, |cdocspt3.tex|, |cdocspt4.tex|,
|cdocsdrf.tex|, |cdocsfn1.tex|, |cdocsfn2.tex|
as well as |childdoc.pdf|.

%%%%%%%%%%%%%%%%%%%%%%%%%%%%%%%%%%%%%%%%%%%%%%%%%%%%%%%%%%%%%%%%%%%%%%%%%%%%%%%%
\subsection{Files and Installation}

The package consists of the files:
%
\begin{center}
\begin{tabular}{ll}
    |README.txt|   & readme file \\
    |childdoc.ins| & installation file \\
    |childdoc.dtx| & source file \\
    |childdoc.def| & definition file \\
    |cdocsamp.tex| & sample main file \\
    |cdocsch1.tex| & sample include file \\
    |cdocsch2.tex| & sample include file \\
    |cdocspt3.tex| & sample part file \\
    |cdocspt4.tex| & sample part file \\
    |cdocsdrf.tex| & sample redirection file \\
    |cdocsfn1.tex| & sample redirection file \\
    |cdocsfn2.tex| & sample redirection file \\
    |childdoc.pdf| & manual
\end{tabular}
\end{center}
%
The distribution consists of the files
|README.txt|, |childdoc.ins| and |childdoc.dtx|.
%
\begin{itemize}
\item
Run (pdf)\LaTeX{} on |childdoc.dtx|
to compile the manual |childdoc.pdf| (this file).
\item
Run \LaTeX{} on |childdoc.ins| to create the definitions file |childdoc.def|
and the sample |cdocsamp.tex| with include files
|cdocsch1.tex|, |cdocsch2.tex|, |cdocspt3.tex|, |cdocspt4.tex|,
|cdocsdrf.tex|, |cdocsfn1.tex|, |cdocsfn2.tex|.
Then copy the file |childdoc.def| to an appropriate directory of your \LaTeX{}
distribution, e.g.\ \textit{texmf-root}|/tex/latex/childdoc|.
\end{itemize}

%%%%%%%%%%%%%%%%%%%%%%%%%%%%%%%%%%%%%%%%%%%%%%%%%%%%%%%%%%%%%%%%%%%%%%%%%%%%%%%%
\subsection{Related CTAN Packages}

There are several other packages which offer a similar functionality:
%
\begin{itemize}
\item
The packages
\href{http://ctan.org/pkg/docmute}{\textsf{docmute}},
\href{http://ctan.org/pkg/includex}{\textsf{includex}} and
\href{http://ctan.org/pkg/standalone}{\textsf{standalone}}
provide commands to include only the document body of
a child file thus allowing both files to be compiled individually.
\item
The packages \href{http://ctan.org/pkg/subdocs}{\textsf{subdocs}}
and \href{http://ctan.org/pkg/subfiles}{\textsf{subfiles}}
provide structures in which the main and child documents can be
encapsulated and allowing them to be compiled individually.
The inclusion mechanism is different from the conventional |\include|.
\item
The package \href{http://ctan.org/pkg/combine}{\textsf{combine}}
is an elaborate solution to combine several documents into one.
\end{itemize}
%
See also the CTAN topic \href{http://ctan.org/topic/subdocs}{\textsf{subdocs}}
for further related packages.
The present package differs from the above solutions in that
a document structure constructed with the conventional |\include| mechanism
just needs two extra commands at the top of every file
such that all constituent files can be compiled individually.

%%%%%%%%%%%%%%%%%%%%%%%%%%%%%%%%%%%%%%%%%%%%%%%%%%%%%%%%%%%%%%%%%%%%%%%%%%%%%%%%
%\subsection{Feature Suggestions}
%
%The following is a list of features which may be useful for future
%versions of this package:
%%
%\begin{itemize}
%\item
%\ldots
%\end{itemize}

%%%%%%%%%%%%%%%%%%%%%%%%%%%%%%%%%%%%%%%%%%%%%%%%%%%%%%%%%%%%%%%%%%%%%%%%%%%%%%%%
\subsection{Revision History}

%%%%%%%%%%%%%%%%%%%%%%%%%%%%%%%%%%%%%%%%
\paragraph{v2.0:} 2018/12/30

\begin{itemize}
\item
immediate forward processing
\item
added |\childdocby| mechanism
\item
manual restructured
\end{itemize}

%%%%%%%%%%%%%%%%%%%%%%%%%%%%%%%%%%%%%%%%
\paragraph{v1.6:} 2018/01/17

\begin{itemize}
\item
application for development of include files
\item
corrections to manual
\end{itemize}

%%%%%%%%%%%%%%%%%%%%%%%%%%%%%%%%%%%%%%%%
\paragraph{v1.5:} 2017/05/21

\begin{itemize}
\item
more complete structuring introduced
\item
|\childdocof| introduced
\item
|\childdoc| renamed to |\childdocmain|
\item
|\childredirect| renamed to |\childdocforward| and |\childdocforwardprefix|
and functionality expanded
\end{itemize}

%%%%%%%%%%%%%%%%%%%%%%%%%%%%%%%%%%%%%%%%
\paragraph{v1.0:} 2017/04/27

\begin{itemize}
\item
manual and install package
\item
first version published on CTAN
\end{itemize}

%%%%%%%%%%%%%%%%%%%%%%%%%%%%%%%%%%%%%%%%
\paragraph{v0.6:} 2017/04/26

\begin{itemize}
\item
redirection mechanism added
\end{itemize}

%%%%%%%%%%%%%%%%%%%%%%%%%%%%%%%%%%%%%%%%
\paragraph{v0.5:} 2017/04/26

\begin{itemize}
\item
functionality in definition file
\end{itemize}


%%%%%%%%%%%%%%%%%%%%%%%%%%%%%%%%%%%%%%%%%%%%%%%%%%%%%%%%%%%%%%%%%%%%%%%%%%%%%%%%
%%%%%%%%%%%%%%%%%%%%%%%%%%%%%%%%%%%%%%%%%%%%%%%%%%%%%%%%%%%%%%%%%%%%%%%%%%%%%%%%
%%%%%%%%%%%%%%%%%%%%%%%%%%%%%%%%%%%%%%%%%%%%%%%%%%%%%%%%%%%%%%%%%%%%%%%%%%%%%%%%
\appendix

\settowidth\MacroIndent{\rmfamily\scriptsize 000\ }

 \DocInput{childdoc.dtx}

\end{document}
%</driver>
% \fi
%
% %%%%%%%%%%%%%%%%%%%%%%%%%%%%%%%%%%%%%%%%%%%%%%%%%%%%%%%%%%%%%%%%%%%%%%%%%%%%%%
% %%%%%%%%%%%%%%%%%%%%%%%%%%%%%%%%%%%%%%%%%%%%%%%%%%%%%%%%%%%%%%%%%%%%%%%%%%%%%%
% \section{Sample}
%\iffalse
%<*samplemain>
%\fi
%
% The following presents a sample document
% with two chapters, two parts, a title page,
% a compile flag as well as three forwarding files to set the flag.
% It consists of eight |.tex| files:
% \begin{center}
% \begin{tabular}{ll}
% |cdocsamp.tex|&main file\\
% |cdocsch1.tex|&include file for chapter 1\\
% |cdocsch2.tex|&include file for chapter 2\\
% |cdocspt3.tex|&include file for part 3\\
% |cdocspt4.tex|&include file for part 4\\
% |cdocsdrf.tex|&forwarding file for main file in draft mode\\
% |cdocsfi1.tex|&forwarding file for final version of chapter 1\\
% |cdocsfi2.tex|&forwarding file for final version of chapter 2\\
% \end{tabular}
% \end{center}
% Each of the eight files can be compiled directly by the \LaTeX{} compiler.
%
% %%%%%%%%%%%%%%%%%%%%%%%%%%%%%%%%%%%%%%
% \paragraph{Main File.}
%
% The main file is called |cdocsamp.tex|.
%
% Load the \textsf{childdoc} definitions and
% declare the filename for the main document:
%    \begin{macrocode}
\input{childdoc.def}
\childdocmain{}
%    \end{macrocode}

% Optional override for |\version| flag:
%    \begin{macrocode}
%%\ifchilddoc\else\providecommand{\version}{draft}\fi
%    \end{macrocode}

% Define the default values for the |\version| flag
% (|final| for the main file and |draft| for childs):
%    \begin{macrocode}
\ifchilddoc
\providecommand{\version}{draft}
\else
\providecommand{\version}{final}
\fi
%    \end{macrocode}

% Load the standard document class:
%    \begin{macrocode}
\documentclass[12pt]{article}
%    \end{macrocode}

% Start the document body:
%    \begin{macrocode}
\begin{document}
%    \end{macrocode}

% Declare a title page.
% Print title, part of document being processed and version flag:
%    \begin{macrocode}
\addtocounter{page}{-1}
\begin{center}
{\LARGE\bfseries{}childdoc example\par}
\vspace{1cm}
\ifchilddoc
\ifchilddocmanual part\else chapter\fi:
`\childdocname' of `\childdocjob'\par
\else
main document: `\childdocjob'\par
\fi
version: \version\par
\end{center}
\newpage
%    \end{macrocode}

% Manually include selected file,
% otherwise process as usual:
%    \begin{macrocode}
\ifchilddocmanual
\section*{part `\childdocname'}
\input{\childdocname}
\else
%    \end{macrocode}

% Include the two chapters:
%    \begin{macrocode}
\include{cdocsch1}
\include{cdocsch2}
%    \end{macrocode}

% Include the two parts unless only chapters should be displayed:
%    \begin{macrocode}
\ifchilddoc\else
\section{part three}
\input{cdocspt3}
\section{part four}
\input{cdocspt4}
\fi
%    \end{macrocode}

% Process as usual until here:
%    \begin{macrocode}
\fi
%    \end{macrocode}

% End of document body:
%    \begin{macrocode}
\end{document}
%    \end{macrocode}
%\iffalse
%</samplemain>
%\fi
%
% %%%%%%%%%%%%%%%%%%%%%%%%%%%%%%%%%%%%%%
% \paragraph{Chapter Include Files.}
%
% The include files are called |cdocsch1.tex| and |cdocsch2.tex|.
%
%\iffalse
%<*samplechap1|samplechap2>
%\fi

% Optional override for |\version| flag:
%    \begin{macrocode}
%%\providecommand{\version}{final}
%    \end{macrocode}

% Include the main document:
%    \begin{macrocode}
\input{childdoc.def}
\childdocof{cdocsamp}
%    \end{macrocode}

%\iffalse
%</samplechap1|samplechap2>
%\fi
%
%\iffalse
%<*samplechap1>
%\fi
% Some text for chapter 1:
%    \begin{macrocode}
\section{one}
some text in chapter one
%    \end{macrocode}

%\iffalse
%</samplechap1>
%\fi
% Some text for chapter 2:
%\iffalse
%<*samplechap2>
%\fi
%    \begin{macrocode}
\section{two}
more text in chapter two
%    \end{macrocode}

%\iffalse
%</samplechap2>
%\fi
%
% %%%%%%%%%%%%%%%%%%%%%%%%%%%%%%%%%%%%%%
% \paragraph{Part Include Files.}
%
% The include files are called |cdocspt3.tex| and |cdocspt4.tex|.
%
%\iffalse
%<*samplepart3|samplepart4>
%\fi

% Optional override for |\version| flag:
%    \begin{macrocode}
%%\providecommand{\version}{final}
%    \end{macrocode}

% Include the main document:
%    \begin{macrocode}
\input{childdoc.def}
\childdocby{cdocsamp}
%    \end{macrocode}

%\iffalse
%</samplepart3|samplepart4>
%\fi
%
%\iffalse
%<*samplepart3>
%\fi
% Some text for part 3:
%    \begin{macrocode}
some text in part three
%    \end{macrocode}

%\iffalse
%</samplepart3>
%\fi
% Some text for part 4:
%\iffalse
%<*samplepart4>
%\fi
%    \begin{macrocode}
more text in part four
%    \end{macrocode}

%\iffalse
%</samplepart4>
%\fi
%
% %%%%%%%%%%%%%%%%%%%%%%%%%%%%%%%%%%%%%%
% \paragraph{Forwarding for a Complete Draft.}
%
% The following forwarding file |cdocsdrf.tex|
% compiles the main document in draft mode:
%\iffalse
%<*sampledraft>
%\fi
%    \begin{macrocode}
\def\version{draft}
\input{childdoc.def}
\childdocforward{cdocsamp}
%    \end{macrocode}

%\iffalse
%</sampledraft>
%\fi
%
% %%%%%%%%%%%%%%%%%%%%%%%%%%%%%%%%%%%%%%
% \paragraph{Forwarding for Final Version of the Chapters.}
%
% The following forwarding files |cdocsfn1.tex| and |cdocsfn2.tex|
% (with identical content)
% compile the final versions of the child documents
% |cdocsch1.tex| and |cdocsch2.tex|, respectively:
%\iffalse
%<*samplefinal>
%\fi
%    \begin{macrocode}
\def\version{final}
\input{childdoc.def}
\childdocforwardprefix[cdocsamp]{cdocsfn}{cdocsch}
%    \end{macrocode}

%\iffalse
%</samplefinal>
%\fi
%
% %%%%%%%%%%%%%%%%%%%%%%%%%%%%%%%%%%%%%%
% \paragraph{Command Line Processing.}
%
% The following three command lines generate the output files
% |cdocscld|, |cdocscl1| and |cdocscl2|
% which should be identical to
% |cdocsdrf|, |cdocsch1| and |cdocsfn2|, respectively:
% \begin{center}
% \begin{tabular}{l}
% |latex -jobname cdocscld \|\\
% |  "\def\version{draft}\input{childdoc.def}\childdocforward{cdocsamp}"|\\
% |latex -jobname cdocscl1 \|\\
% |  "\input{childdoc.def}\childdocforward[cdocsamp]{cdocsch1}"|\\
% |latex -jobname cdocscl2 \|\\
% |  "\def\version{final}\input{childdoc.def}\childdocforward{cdocsch2}"|
% \end{tabular}
% \end{center}
% Note that the trailing backslash on each first line
% merely continues the input to the second line
% (for convenient cut ant paste).
% Furthermore, the command |latex| can be replaced by any
% of its alternative versions such as |pdflatex|.
%
% %%%%%%%%%%%%%%%%%%%%%%%%%%%%%%%%%%%%%%%%%%%%%%%%%%%%%%%%%%%%%%%%%%%%%%%%%%%%%%
% %%%%%%%%%%%%%%%%%%%%%%%%%%%%%%%%%%%%%%%%%%%%%%%%%%%%%%%%%%%%%%%%%%%%%%%%%%%%%%
% \section{Implementation}
%\iffalse
%<*package>
%\fi
%
% This section describes the definitions file |childdoc.def|.

% The definitions cannot be loaded using |\usepackage| or |\RequirePackage|
% which has a mechanism to prevent loading a style file more than once.
% When loading the definitions by means of |\input|
% multiple instances have to be prevented manually:
%\iffalse
%This code needs to be before the `\ProvidesFile' directive
%which is defined at the beginning of this file.
%Therefore it is also placed there and commented out here.
%</package>
%<*discard>
%\fi
%    \begin{macrocode}
\ifdefined\childdocmain\endinput\fi
%    \end{macrocode}
%\iffalse
%</discard>
%<*package>
%\fi
%
% \macro{\ifchilddoc}
% \macro{\ifchilddocmanual}
% The conditional |\ifchilddoc| tells whether a
% child (true) or main (false) document is being compiled.
% The conditional |\ifchilddocmanual| tells whether
% the |\includeonly| mechanism is used (false) or
% the selection of child files must be performed manually (true).
% The definitions initialise to false:
%    \begin{macrocode}
\newif\ifchilddoc
\newif\ifchilddocmanual
%    \end{macrocode}

% \macro{\childdocname}
% \macro{\childdocjob}
% The macro |\childdocname| stores the name of the main document
% to be compiled. The macro |\childdocjob| stores the name of
% the document on which the \LaTeX{} compiler was originally invoked.
% The content of |\jobname| cannot be compared
% to filenames specified in the source due to different catcodes.
% The following code rescans |\jobname|, stores the result
% in |\childdocname| and saves a copy in |\childdocjob|:
%    \begin{macrocode}
\edef\childdocname{\scantokens\expandafter{\jobname\noexpand}}
\let\childdocjob\childdocname
%    \end{macrocode}

% \macro{\childdocdisable}
% The macro |\childdocdisable| prevents the main file
% from being processed more than once.
% At this stage, the main document command |\childdocmain|
% is assumed to be called once again where it should do nothing.
% Any subsequent call to it should prevent
% a secondary processing of the main document
% It overwrites the forwarding commands
% |\childdocof| and |\childdocforward|
% with empty macros to prevent further inclusions of the main document:
%    \begin{macrocode}
\newcommand{\childdocdisable}
{
  \renewcommand{\childdocmain}[1]{\renewcommand{\childdocmain}[1]{\endinput}}
  \renewcommand{\childdocof}[1]{}
  \renewcommand{\childdocby}[2][]{}
  \renewcommand{\childdocforward}[2][]{}
  \renewcommand{\childdocdisable}{}
}
%    \end{macrocode}

% \macro{\childdocmain}
% The macro |\childdocmain| is to be called at the top of the main file
% with nothing or the main filename (without extension) as argument.
% First, it breaks loops.
% If the argument is not empty and does not match |\childdocname|
% (which is set by the first inclusion of |childdoc.def|),
% |\ifchilddoc| is set to true, |\includeonly| is applied to the child file
% and |\jobname| is set to the main file
% (for proper handling of |.aux| files):
%    \begin{macrocode}
\newcommand{\childdocmain}[1]
{
  \childdocdisable\childdocmain{}
  \if?#1?\else
    \begingroup
      \def\childdoctmp{#1}
      \ifx\childdoctmp\childdocname
        \def\childdoctmp{}
      \else
        \def\childdoctmp
        {
          \childdoctrue
          \includeonly{\childdocname}
          \def\childdocjob{#1}
          \def\jobname{#1}
        }
      \fi
      \expandafter
    \endgroup
    \childdoctmp
  \fi
}
%    \end{macrocode}

% \macro{\childdocof}
% The command |\childdocof| redirects
% compilation to the main file |#1|.
%    \begin{macrocode}
\newcommand{\childdocof}[1]
{
  \childdocdisable
  \childdoctrue
  \includeonly{\childdocname}
  \def\jobname{#1}
  \def\childdocjob{#1}
  \input{#1}
}
%    \end{macrocode}

% \macro{\childdocby}
% The command |\childdocby| ....
%    \begin{macrocode}
\newcommand{\childdocby}[2][]
{
  \childdocdisable
  \childdoctrue
  \childdocmanualtrue
  \if?#1?\else
    \def\jobname{#2}
  \fi
  \def\childdocjob{#2}
  \input{#2}
  \endinput
}
%    \end{macrocode}

% \macro{\childdocforward}
% The command |\childdocforward| redirects
% compilation to the main file or
% (if the optional argument is given) a child file.
% Parameters are set as if the main file
% or a child file starting with |\childdocof| was compiled.
% Then compilation is handed over to the main file:
%    \begin{macrocode}
\newcommand{\childdocforward}[2][]
{
  \begingroup
    \if?#1?
      \def\childdoctmp
      {
        \def\childdocname{#2}
        \def\childdocjob{#2}
        \def\jobname{#2}
        \input{#2}
        \endinput
      }
    \else
      \def\childdoctmp
      {
        \childdocdisable
        \def\childdocname{#2}
        \childdoctrue
        \includeonly{#2}
        \def\childdocjob{#1}
        \def\jobname{#1}
        \input{#1}
        \endinput
      }
    \fi
    \expandafter
  \endgroup
  \childdoctmp
}
%    \end{macrocode}

% \macro{\childdocforwardprefix}
% The command |\childdocforwardprefix| redirects
% compilation to the main or a child file by means of a pattern.
% The prefix |#1| in the current filename is replaced by |#2|
% and the suffix of the current filename is kept
% (it is assumed that the filename does not contain the substring `|~~~|'
% which is used as a delimiter).
% Compilation is handed over to the new file by |\childdocforward|:
%    \begin{macrocode}
\newcommand{\childdocforwardprefix}[3][]
{
  \begingroup
    \def\childdocextract #2##1~~~{\def\childdoctmp{\childdocforward[#1]{#3##1}}}
    \expandafter\childdocextract\childdocname~~~
    \expandafter
  \endgroup
  \childdoctmp
}
%    \end{macrocode}

% \macro{\childdoc}
% The deprecated macro |\childdoc| is a legacy version of |\childdocmain|:
%    \begin{macrocode}
\newcommand{\childdoc}{\childdocmain}
%    \end{macrocode}

% \macro{\childdocredirect}
% The deprecated macro |\childdocredirect| is a legacy version
% of |\childdocforward| and |\childdocforwardprefix|:
%    \begin{macrocode}
\newcommand{\childdocredirect}[2][]
{
  \begingroup
    \if?#1?
      \def\childdoctmp{\childdocforward{#2}}
    \else
      \def\childdoctmp{\childdocforwardprefix{#1}{#2}}
    \fi
    \expandafter
  \endgroup
  \childdoctmp
}
%    \end{macrocode}

%\iffalse
%</package>
%\fi
%
\endinput
|\\
|\childdocby{|\textit{main}|}|\\
\end{tabular}
\end{center}
%
Both forms have slightly different effects as described above.
The main file is prepared as usual, see \secref{sec:include}.

%%%%%%%%%%%%%%%%%%%%%%%%%%%%%%%%%%%%%%%%%%%%%%%%%%%%%%%%%%%%%%%%%%%%%%%%%%%%%%%%
\subsection{Legacy Detection}
\label{sec:detection}

The directive |\childdocmain| in the main file can detect
whether the complete document or merely a child is to be compiled
even without using the directive |\childdocof|.
This method is deprecated because it is less robust
and there is no compelling reason to use it;
it is merely provided for backward compatibility
and it may be removed in future versions.

If the detection mechanism is to be used,
it is mandatory to correctly specify
the filename of the main file as the argument of |\childdocmain|:
%
\begin{center}
\begin{tabular}{l}
|% \iffalse
%
% childdoc.dtx Copyright (C) 2017-2018 Niklas Beisert
%
% This work may be distributed and/or modified under the
% conditions of the LaTeX Project Public License, either version 1.3
% of this license or (at your option) any later version.
% The latest version of this license is in
%   http://www.latex-project.org/lppl.txt
% and version 1.3 or later is part of all distributions of LaTeX
% version 2005/12/01 or later.
%
% This work has the LPPL maintenance status `maintained'.
%
% The Current Maintainer of this work is Niklas Beisert.
%
% This work consists of the files childdoc.dtx and childdoc.ins
% and the derived files childdoc.def and cdocsamp.tex with
% cdocsch1.tex, cdocsch2.tex, cdocsdrf.tex, cdocsfn1.tex, cdocsfn2.tex.
%
%<package>\ifdefined\childdocmain\endinput\fi
%<package>\ProvidesFile{childdoc.def}[2018/12/30 v2.0 child document driver]
%<samplemain>\ProvidesFile{cdocsamp.tex}[2018/12/30 v2.0 sample for childdoc]
%<*driver>
%\ProvidesFile{childdoc.drv}[2018/12/30 v2.0 childdoc reference manual file]
\PassOptionsToClass{10pt,a4paper}{article}
\documentclass{ltxdoc}

\usepackage[margin=35mm]{geometry}
\usepackage{hyperref}
\usepackage{hyperxmp}
\usepackage[usenames]{color}

\hypersetup{colorlinks=true}
\hypersetup{pdfstartview=FitH}
\hypersetup{pdfpagemode=UseNone}
\hypersetup{pdfsource={}}
\hypersetup{pdflang={en-UK}}
\hypersetup{pdfcopyright={Copyright 2017-2018 Niklas Beisert.
  This work may be distributed and/or modified under the
  conditions of the LaTeX Project Public License, either version 1.3
  of this license or (at your option) any later version.}}
\hypersetup{pdflicenseurl={http://www.latex-project.org/lppl.txt}}
\hypersetup{pdfcontactaddress={ETH Zurich, ITP, HIT K,
  Wolfgang-Pauli-Strasse 27}}
\hypersetup{pdfcontactpostcode={8093}}
\hypersetup{pdfcontactcity={Zurich}}
\hypersetup{pdfcontactcountry={Switzerland}}
\hypersetup{pdfcontactemail={nbeisert@itp.phys.ethz.ch}}
\hypersetup{pdfcontacturl={http://people.phys.ethz.ch/\xmptilde nbeisert/}}

\newcommand{\secref}[1]{\hyperref[#1]{section \ref*{#1}}}

\parskip1ex
\parindent0pt
\let\olditemize\itemize
\def\itemize{\olditemize\parskip0pt}

\begin{document}

\title{The \textsf{childdoc} Package}
\hypersetup{pdftitle={The childdoc Package}}
\author{Niklas Beisert\\[2ex]
  Institut f\"ur Theoretische Physik\\
  Eidgen\"ossische Technische Hochschule Z\"urich\\
  Wolfgang-Pauli-Strasse 27, 8093 Z\"urich, Switzerland\\[1ex]
  \href{mailto:nbeisert@itp.phys.ethz.ch}
  {\texttt{nbeisert@itp.phys.ethz.ch}}}
\hypersetup{pdfauthor={Niklas Beisert}}
\hypersetup{pdfsubject={Manual for the LaTeX2e Package childdoc}}
\date{30 December 2018, \textsf{v2.0}}
\maketitle

\begin{abstract}\noindent
\textsf{childdoc} is a \LaTeXe{} package
that enables the direct compilation
of document sections included by |\include|
to individual files.
\end{abstract}

\begingroup
\parskip0ex
\tableofcontents
\endgroup

%%%%%%%%%%%%%%%%%%%%%%%%%%%%%%%%%%%%%%%%%%%%%%%%%%%%%%%%%%%%%%%%%%%%%%%%%%%%%%%%
%%%%%%%%%%%%%%%%%%%%%%%%%%%%%%%%%%%%%%%%%%%%%%%%%%%%%%%%%%%%%%%%%%%%%%%%%%%%%%%%
\section{Introduction}

\LaTeX{} provides a mechanism to structure a large document (such as a book)
into a main file and several child files (containing the chapters)
using the |\include| command.
This mechanism is beneficial for documents
which span hundreds of pages in order to
make the source file(s) more manageable.
Moreover, compilation can be restricted to
selected child files by means of the |\includeonly| command.
The latter feature can be used to reduce the compilation time while editing
(this was significantly more useful in the earlier days of \LaTeX{})
or to generate a smaller document which is easier to navigate.
Another application of |\includeonly| is to generate
documents consisting of selected parts of the complete document.

However, there are a few drawbacks of the plain |\include| mechanism:
\begin{itemize}
\item
The child files cannot be compiled on their own,
they can only be compiled via the main file.
A naive editing environment
(such as a text editor with an option
to have the current file processed by \LaTeX)
may require one to switch to the main file before compiling;
attempting to compile the child file produces errors.
\item
The main file must be modified (each time)
to adjust the |\includeonly| command
to the present needs. This easily leaves the main file in a messy state.
\item
The generated document will always carry the filename
of the main document. This is inconvenient if
several child files are to be compiled and
to be kept for distribution.
\end{itemize}

The present package provides a simple interface
to make child files individually compilable by \LaTeX{}.
Compiling a child file then has the same effect as compiling
the main file with an |\includeonly| command
to select the appropriate child.
Moreover the generated document will carry the name of the child
rather than the main file.
This resolves all three above issues.

This feature is meant to make the editing of books,
thesis documents and lecture notes somewhat more convenient.
However, the package can also be used efficiently for
composing a series of documents (such as exercise sheets)
which are typically distributed individually.
It then assists the author in generating the individual documents
(potentially in different versions)
as well as a document containing the collected series.
Another application is in developing style files
or other kinds of included material
where compilation of the style file could redirect
to a sample or test file.

%%%%%%%%%%%%%%%%%%%%%%%%%%%%%%%%%%%%%%%%%%%%%%%%%%%%%%%%%%%%%%%%%%%%%%%%%%%%%%%%
%%%%%%%%%%%%%%%%%%%%%%%%%%%%%%%%%%%%%%%%%%%%%%%%%%%%%%%%%%%%%%%%%%%%%%%%%%%%%%%%
\section{Usage}

First of all, the package \textsf{childdoc} is \emph{not} a standard
\LaTeXe{} |.sty| style file! Therefore it needs to be invoked in
a non-standard way.

%%%%%%%%%%%%%%%%%%%%%%%%%%%%%%%%%%%%%%%%%%%%%%%%%%%%%%%%%%%%%%%%%%%%%%%%%%%%%%%%
\subsection{Included Files}
\label{sec:include}

%%%%%%%%%%%%%%%%%%%%%%%%%%%%%%%%%%%%%%%%
\DescribeMacro{\childdocmain}
To use the package, add the commands
\begin{center}
\begin{tabular}{l}
|\input{childdoc.def}|\\
|\childdocmain{}|\\
\end{tabular}
\end{center}
at the very top of the main \LaTeX{} file,
in particular \emph{before} the |\documentclass| statement!
The argument of |\childdocmain| should be left empty
(but it must be present).

%%%%%%%%%%%%%%%%%%%%%%%%%%%%%%%%%%%%%%%%
\DescribeMacro{\childdocof}
Furthermore, add the commands
\begin{center}
\begin{tabular}{l}
|\input{childdoc.def}|\\
|\childdocof{|\textit{main}|}|\\
\end{tabular}
\end{center}
at the top of every child file \textit{child}
which is included by |\include{|\textit{child}|}|
from within the main file
(or at least for those files to be compiled individually).
The argument \textit{main} must be the filename of the main file.

There are a couple of
considerations in setting up the main and child documents:

%%%%%%%%%%%%%%%%%%%%%%%%%%%%%%%%%%%%%%%%
\paragraph{Restrictions.}

Please note the following restrictions:
\begin{itemize}
\item
|\childdocmain| must be called with one argument \textit{main}
to ensure compatibility with earlier version of the package.
It must either be empty (|\childdocmain{}|)
or precisely match the filename of the main file in which it is specified.
See \secref{sec:detection} for further information.
\item
The filename \textit{main} must be specified without the |.tex| extension.
\item
The filename \textit{main} is case sensitive
(even in case-insensitive file systems)
due to internal string comparison.
\item
The argument \textit{main} should be fully expanded, it cannot be a macro.
\item
Subdirectories and special characters should be avoided in filenames.
\item
The command |\childdocmain{|\textit{main}|}| must be followed by a whitespace.
It should not be followed immediately by another command
or by a comment mark `|%|'.
This is because the \TeX{} parser reads the token immediately following
the argument of |\childdocmain| and puts it
at the beginning of every child section;
however, a white\-space is ignored.
\end{itemize}

%%%%%%%%%%%%%%%%%%%%%%%%%%%%%%%%%%%%%%%%
\paragraph{Content of Main File.}

It is advisable to place all content in the child files included by |\include|.
Any output contained in the main file will appear in all child documents
unless suppressed manually;
it cannot be suppressed automatically by the |\includeonly| directive
and thus should normally be avoided.
A method to include some content in the main file
by means of conditional processing is described in \secref{sec:conditional}.

%%%%%%%%%%%%%%%%%%%%%%%%%%%%%%%%%%%%%%%%
\paragraph{Page Numbering.}

When only a part of the document is compiled,
the appropriate numbering of pages
(as well as other status parameters)
is determined from the |.aux| files.
The latter contain information from previous passes.
However this information needs to propagate through
all intermediate child documents.
Therefore the page numbering in child documents may well
be inconsistent until the complete document is compiled at least once.

A useful (if unconventional) way to always ensure a consistent
page numbering is to restart the numbering in each child document
and denote the pages by `\textit{child}|.|\textit{page}'
where \textit{child} represents the chapter/section number of the child file.
This can be achieved by the command
|\numberwithin{page}{|\textit{child}|}|
of the \textsf{amsmath} package
where \textit{child} can be |chapter| or |section|
depending on the chosen structuring.
Alternatively, one can modify the macro |\thepage| appropriately
and reset the counter |page| at the start of each child file.

%%%%%%%%%%%%%%%%%%%%%%%%%%%%%%%%%%%%%%%%%%%%%%%%%%%%%%%%%%%%%%%%%%%%%%%%%%%%%%%%
\subsection{Conditional Processing}
\label{sec:conditional}

The package provides a mechanism to compile different versions
of a document. To customise the versions further some conditional processing
can come in handy to distinguish which version is being compiled.
The package provides two macros to describe the compilation context:

%%%%%%%%%%%%%%%%%%%%%%%%%%%%%%%%%%%%%%%%
\DescribeMacro{\ifchilddoc}
The conditional |\ifchilddoc| distinguishes between the compilation of
child documents and the main document:
%
\begin{center}
|\ifchilddoc |\textit{child-code}| |[|\||else |\textit{main-code}]| \||fi|
\end{center}

%%%%%%%%%%%%%%%%%%%%%%%%%%%%%%%%%%%%%%%%
\DescribeMacro{\childdocname}
\DescribeMacro{\childdocjob}
The macro |\childdocname| contains the filename (without extension)
of the main or child file being processed.
Note that |\childdocjob| will always contain the name of the main file.

%%%%%%%%%%%%%%%%%%%%%%%%%%%%%%%%%%%%%%%%
\paragraph{Title Page.}

Conditional processing can be used to include a title or banner page
in the main document when proper precautions are taken.
Importantly, the code in the main file should ensure that the page counter
(as well as other status parameters which are stored in the |.aux| files)
takes the same value after the conditional processing.
Otherwise the page numbers may take divergent values
depending on which part is compiled.

For example, a title page could be declared by:
%
\begin{center}
\begin{tabular}{l}
|\ifchilddoc\||else|\\
|\addtocounter{page}{-1}|\\
\textit{code for title page}\\
|\newpage|\\
|\||fi|
\end{tabular}
\end{center}
%
A banner page for the child documents can be generated by:
%
\begin{center}
\begin{tabular}{l}
|\ifchilddoc|\\
|\addtocounter{page}{-1}|\\
\textit{code for banner page}\\
|\newpage|\\
|\||fi|
\end{tabular}
\end{center}
%
Here one could write a message such as:
\begin{center}
|This is the part \childdocname{} of \childdocjob{}.|
\end{center}

%%%%%%%%%%%%%%%%%%%%%%%%%%%%%%%%%%%%%%%%%%%%%%%%%%%%%%%%%%%%%%%%%%%%%%%%%%%%%%%%
\subsection{Flags}
\label{sec:flags}

The package makes it easy to generate different versions
of the main or child documents.
To this end compilation flags can be defined
and assigned different default values.
They will be particularly useful in conjunction
with the forwarding mechanism described in \secref{sec:forward}.

For example, it may be useful to have a flag |\version|
which can be set to |draft| or |final|.
The document source will contain some conditional code
depending on the value of |\version|.
Suppose further, the flag should default to |final| for the main file
and to |draft| for child files
which is a natural assignment for editing the document.
This is achieved by placing the following code
in the preamble of the main document
(below the |\childdocmain| directive):
%
\begin{center}
\begin{tabular}{l}
|\ifchilddoc|\\
|\providecommand{\version}{draft}|\\
|\||else|\\
|\providecommand{\version}{final}|\\
|\||fi|
\end{tabular}
\end{center}
%
The definition by |\providecommand| makes sure
that previous definitions are not overwritten.
Further statements |\providecommand{\version}{...}|
can thus be added before the above code to override it.

For the main file, one might add a line
(between |\childdocmain| and the above block)
%
\begin{center}
|%\ifchilddoc\||else\providecommand{\version}{draft}\||fi|
\end{center}
%
which can be uncommented to produce a draft version.
Likewise one can add a line to the very top of a child file
(above the |\childdocof{|\textit{main}|}| directive)
%
\begin{center}
|%\providecommand{\version}{final}|
\end{center}
%
which can be uncommented to produce the final version of this child document.

%%%%%%%%%%%%%%%%%%%%%%%%%%%%%%%%%%%%%%%%%%%%%%%%%%%%%%%%%%%%%%%%%%%%%%%%%%%%%%%%
\subsection{Forwarding}
\label{sec:forward}

Different versions of the main or child documents
using compilation flags as described in \secref{sec:flags}
can be (permanently) stored in different files
for convenient compilation, viewing and distribution.
To this end, the package defines a command
to pass on compilation to a different file:

%%%%%%%%%%%%%%%%%%%%%%%%%%%%%%%%%%%%%%%%
\DescribeMacro{\childdocforward}
The command |\childdocforward| redirects processing to
another source file:
%
\begin{center}
\begin{tabular}{l}
|\input{childdoc.def}|\\
|\childdocforward[|\textit{main}|]{|\textit{dest}|}|\\
\end{tabular}
\end{center}
%
The argument \textit{dest} is the destination file
(without extension).
It should be the main file or one of the child files.
Note that further \textsf{childdoc} directives
such as |\childdocof| and |\childdocforward|
in the indicated file will be processed in this form.
The optional argument \textit{main}
passes on directly to the main file \textit{main}
while pretending to compile the child \textit{dest}.
This form behaves as if \textit{dest}
issues |\childdocof{|\textit{main}|}| right away,
and no further \textsf{childdoc} directives will be processed.

%%%%%%%%%%%%%%%%%%%%%%%%%%%%%%%%%%%%%%%%
\DescribeMacro{\...prefix}
In the alternative form |\childdocforwardprefix|,
%
\begin{center}
\begin{tabular}{l}
|\input{childdoc.def}|\\
|\childdocforwardprefix[|\textit{main}|]{|\textit{prefix}|}{|\textit{dest}|}|
\end{tabular}
\end{center}
%
the destination file is determined by a pattern
depending on the current file:
To make this work, the current file must be called
`{\textit{prefix}\hspace{0.2em}\textit{suffix}}'
with \textit{prefix} matching precisely the argument.
Processing is then passed on to the file
`{\textit{dest}\hspace{0.2em}\textit{suffix}}'.
Surely, the same effect is achieved by
directly specifying the
argument `{\textit{dest}\hspace{0.2em}\textit{suffix}}'
in the first form.
However, that requires to set up a different file
for each child. With the alternative form of the command
all these files can have exactly the same content
which simplifies setting them up and maintaining them.

For example, the following file |draft.tex|
with a compilation flag |\version| as described in \secref{sec:flags}
compiles the main document as a draft:
%
\begin{center}
\begin{tabular}{l}
|\def\version{draft}|\\
|\input{childdoc.def}|\\
|\childdocforward{|\textit{main}|}|
\end{tabular}
\end{center}
%
Likewise, the following files |final|\textit{nn}|.tex|
compile the final version of the child document
|child|\textit{nn}|.tex|:
%
\begin{center}
\begin{tabular}{l}
|\def\version{final}|\\
|\input{childdoc.def}|\\
|\childdocforwardprefix{final}{child}|
\end{tabular}
\end{center}
%

Note that when several versions of a main file and/or of each child file
are to be generated, it may be convenient to set up a |Makefile| or
shell script to automatise the process.

%%%%%%%%%%%%%%%%%%%%%%%%%%%%%%%%%%%%%%%%%%%%%%%%%%%%%%%%%%%%%%%%%%%%%%%%%%%%%%%%
\subsection{Command Line Processing}
\label{sec:commandline}

The effect of redirection files can also be achieved by invoking
the \LaTeX{} compiler with a more elaborate command line.
Most conveniently this should be done as part
of a shell script or a |Makefile|.

When using \textsf{childdoc} in the main file, the following
command lines effectively perform a redirection
(note that depending on the shell being used,
backslashes may have to be doubled: `|\|' $\to$ `|\\|'):
%
\begin{center}
|... -jobname "|\textit{target}|" |\\|"|[\textit{flags}]%
|\input{childdoc.def}\childdocforward[|\textit{main}|]{|\textit{dest}|}"|
\end{center}
%
Here \textit{target} is the name of the output file,
\textit{main} is the name of the main file
and \textit{dest} is the name of the main or child file to be processed
(all filenames without extensions).
The optional argument \textit{main} can be omitted
if \textit{main} matches \textit{dest}.
Optionally, compilation \textit{flags} can be defined via |\def| commands.
This command line makes the \TeX{} engine believe
it is compiling the file \textit{target}
whose content is specified as the latter parameter.
The provided code then forwards the processing to
\textit{main} or \textit{dest} as described in \secref{sec:forward}.

%%%%%%%%%%%%%%%%%%%%%%%%%%%%%%%%%%%%%%%%%%%%%%%%%%%%%%%%%%%%%%%%%%%%%%%%%%%%%%%%
\subsection{Include by Input}
\label{sec:input}

Including child documents by |\include| has some restrictions by design.
Most notably, the content of a child document always occupies
its own set of pages; pages cannot be shared between child documents.
Usually, this behaviour makes perfect sense
because each child document contain an essential part of the document.
However, in some situations it may be desirable to compose
a document from a collection of parts
without having mandatory page breaks between then.
For this case, the package
provides a mechanism to include parts
by |\input| which can also be processed individually.
However, by construction this mechanism
requires manual handling of the content to be output.

%%%%%%%%%%%%%%%%%%%%%%%%%%%%%%%%%%%%%%%%
\DescribeMacro{\ifchilddocmanual}
The main file should be prepared as usual, see \secref{sec:include}.
However, the document body must make a distinction
between processing of an individual part and of the main document, e.g.:
%
\begin{center}
\begin{tabular}{l}
|\ifchilddocmanual|\\
|\input{\childdocname}|\\
|\||else|\\
\textit{document body with }|\input{|\textit{part}|}|\\
|\||fi|
\end{tabular}
\end{center}
%
The conditional |\ifchilddocmanual| is true whenever
a part to be included by |\input| is being compiled,
and the name of the part is stored in |\childdocname|.

%%%%%%%%%%%%%%%%%%%%%%%%%%%%%%%%%%%%%%%%
\DescribeMacro{\childdocby}
Each part to be included by |\input| should start with:
%
\begin{center}
\begin{tabular}{l}
|\input{childdoc.def}|\\
|\childdocby{|\textit{main}|}|\\
\end{tabular}
\end{center}
%
The directive |\childdocby| is similar to |\childdocof|
described in \secref{sec:include},
but the subsequent selection of content must be done manually.
To that end, both |\ifchilddoc| and |\ifchilddocmanual|
will be true upon processing of a part,
and the name of the part is stored in |\childdocname|.
Note that |\jobname| will be set to the filename of the current part
so that each part receives an individual |.aux| file
that does not interfere with the |.aux| file(s) of the main document.
This behaviour can be altered by the alternative form
|\childdocby[*]{|\textit{main}|}| (with a non-empty optional argument)
which uses the |.aux| file of the main document
by setting |\jobname| to \textit{main}.

%%%%%%%%%%%%%%%%%%%%%%%%%%%%%%%%%%%%%%%%%%%%%%%%%%%%%%%%%%%%%%%%%%%%%%%%%%%%%%%%
\subsection{Driver Development}
\label{sec:driver}

The \textsf{childdoc} mechanism can also be use for the development
of definition files such as \LaTeX{} styles or classes.
This case differs from the above setup with multiple parts
included by |\include| in that no |\includeonly| should be invoked.
This can be achieved by starting the include file
(before |\ProvidesPackage|) with:
%
\begin{center}
\begin{tabular}{l}
|\input{childdoc.def}|\\
|\childdocforward{|\textit{main}|}|\\
\end{tabular}
\end{center}
%
or alternatively with:
%
\begin{center}
\begin{tabular}{l}
|\input{childdoc.def}|\\
|\childdocby{|\textit{main}|}|\\
\end{tabular}
\end{center}
%
Both forms have slightly different effects as described above.
The main file is prepared as usual, see \secref{sec:include}.

%%%%%%%%%%%%%%%%%%%%%%%%%%%%%%%%%%%%%%%%%%%%%%%%%%%%%%%%%%%%%%%%%%%%%%%%%%%%%%%%
\subsection{Legacy Detection}
\label{sec:detection}

The directive |\childdocmain| in the main file can detect
whether the complete document or merely a child is to be compiled
even without using the directive |\childdocof|.
This method is deprecated because it is less robust
and there is no compelling reason to use it;
it is merely provided for backward compatibility
and it may be removed in future versions.

If the detection mechanism is to be used,
it is mandatory to correctly specify
the filename of the main file as the argument of |\childdocmain|:
%
\begin{center}
\begin{tabular}{l}
|\input{childdoc.def}|\\
|\childdocmain{|\textit{main}|}|\\
\end{tabular}
\end{center}
%
If |\jobname| does not match the argument \textit{main} of |\childdocmain|,
it is assumed that |\jobname| points to the child file to be compiled.
When using |\childdocmain| with the main file specified as argument,
it suffices to start a child file
with just |\input{|\textit{main}|}|
without loading of the package and using |\childdocof|.
If instead all processing is done
with the appropriate \textsf{childdoc} directives,
the argument of \textit{main} of |\childdocmain| can be empty.

An alternative version of the command line processing described
in \secref{sec:commandline} using the detection mechanism reads:
%
\begin{center}
|... -jobname "|\textit{target}|" "|[\textit{flags}]%
[|\def\jobname{|\textit{dest}|}|]|\input{|\textit{main}|}"|
\end{center}

%%%%%%%%%%%%%%%%%%%%%%%%%%%%%%%%%%%%%%%%%%%%%%%%%%%%%%%%%%%%%%%%%%%%%%%%%%%%%%%%
\subsection{Manual Code}
\label{sec:manual}

In case one cannot be certain whether the definitions file |childdoc.def|
is installed on the target \TeX{} distribution
and one prefers not to ship it,
it is conceivable to paste a few relevant commands into the sources.

To that end, drop all statements |\input{childdoc.def}|
and perform the replacements as outlined below.
Instead of |\childdocmain{|\textit{main}|}| add the following code
to the top of the main file:
%
\begin{center}
\begin{tabular}{l}
|\||ifdefined\childdocname\endinput\||fi\newif\ifchilddoc|\\
|\edef\childdocname{\scantokens\expandafter{\jobname\noexpand}}|\\
|\def\childdocmain{|\textit{main}|}\||ifx\childdocmain\childdocname\||else|\\
|\childdoctrue\includeonly{\childdocname}\let\jobname\childdocmain\||fi|\\
\end{tabular}
\end{center}
%
Instead of |\childdocof{|\textit{main}|}| just include the main file
at the top of each child file:
%
\begin{center}
|\input{|\textit{main}|}|
\end{center}
%
A simple redirection |\childdocforward{|\textit{dest}|}| is achieved by:
%
\begin{center}
|\def\jobname{|\textit{dest}|}\input{\jobname}|
\end{center}
%
The redirection with prefix
|\childdocforwardprefix[|\textit{prefix}|]{|\textit{dest}|}|
is accomplished by:
%
\begin{center}
\begin{tabular}{l}
|{\edef\jobname{\scantokens\expandafter{\jobname\noexpand}}|\\
|\def\redirectjob |\textit{prefix}|#1~~~{\gdef\jobname{|\textit{dest}|#1}}|\\
|\expandafter\redirectjob\jobname~~~}\input{\jobname}|
\end{tabular}
\end{center}

In an alternative approach,
child documents can be compiled by a specific command line
without additional code or specific definitions:
%
\begin{center}
|... -jobname "|\textit{target}|" "|[\textit{flags}]%
|\includeonly{|\textit{dest}|}\input{|\textit{main}|}"|
\end{center}
%

%%%%%%%%%%%%%%%%%%%%%%%%%%%%%%%%%%%%%%%%%%%%%%%%%%%%%%%%%%%%%%%%%%%%%%%%%%%%%%%%
%%%%%%%%%%%%%%%%%%%%%%%%%%%%%%%%%%%%%%%%%%%%%%%%%%%%%%%%%%%%%%%%%%%%%%%%%%%%%%%%
\section{Information}

%%%%%%%%%%%%%%%%%%%%%%%%%%%%%%%%%%%%%%%%%%%%%%%%%%%%%%%%%%%%%%%%%%%%%%%%%%%%%%%%
\subsection{Copyright}

Copyright \copyright{} 2017--2018 Niklas Beisert

This work may be distributed and/or modified under the
conditions of the \LaTeX{} Project Public License, either version 1.3
of this license or (at your option) any later version.
The latest version of this license is in
  \url{http://www.latex-project.org/lppl.txt}
and version 1.3 or later is part of all distributions of \LaTeX{}
version 2005/12/01 or later.

This work has the LPPL maintenance status `maintained'.

The Current Maintainer of this work is Niklas Beisert.

This work consists of the files |README.txt|, |childdoc.ins| and |childdoc.dtx|
as well as the derived files |childdoc.def|, |cdocsamp.tex|
with |cdocsch1.tex|, |cdocsch2.tex|, |cdocspt3.tex|, |cdocspt4.tex|,
|cdocsdrf.tex|, |cdocsfn1.tex|, |cdocsfn2.tex|
as well as |childdoc.pdf|.

%%%%%%%%%%%%%%%%%%%%%%%%%%%%%%%%%%%%%%%%%%%%%%%%%%%%%%%%%%%%%%%%%%%%%%%%%%%%%%%%
\subsection{Files and Installation}

The package consists of the files:
%
\begin{center}
\begin{tabular}{ll}
    |README.txt|   & readme file \\
    |childdoc.ins| & installation file \\
    |childdoc.dtx| & source file \\
    |childdoc.def| & definition file \\
    |cdocsamp.tex| & sample main file \\
    |cdocsch1.tex| & sample include file \\
    |cdocsch2.tex| & sample include file \\
    |cdocspt3.tex| & sample part file \\
    |cdocspt4.tex| & sample part file \\
    |cdocsdrf.tex| & sample redirection file \\
    |cdocsfn1.tex| & sample redirection file \\
    |cdocsfn2.tex| & sample redirection file \\
    |childdoc.pdf| & manual
\end{tabular}
\end{center}
%
The distribution consists of the files
|README.txt|, |childdoc.ins| and |childdoc.dtx|.
%
\begin{itemize}
\item
Run (pdf)\LaTeX{} on |childdoc.dtx|
to compile the manual |childdoc.pdf| (this file).
\item
Run \LaTeX{} on |childdoc.ins| to create the definitions file |childdoc.def|
and the sample |cdocsamp.tex| with include files
|cdocsch1.tex|, |cdocsch2.tex|, |cdocspt3.tex|, |cdocspt4.tex|,
|cdocsdrf.tex|, |cdocsfn1.tex|, |cdocsfn2.tex|.
Then copy the file |childdoc.def| to an appropriate directory of your \LaTeX{}
distribution, e.g.\ \textit{texmf-root}|/tex/latex/childdoc|.
\end{itemize}

%%%%%%%%%%%%%%%%%%%%%%%%%%%%%%%%%%%%%%%%%%%%%%%%%%%%%%%%%%%%%%%%%%%%%%%%%%%%%%%%
\subsection{Related CTAN Packages}

There are several other packages which offer a similar functionality:
%
\begin{itemize}
\item
The packages
\href{http://ctan.org/pkg/docmute}{\textsf{docmute}},
\href{http://ctan.org/pkg/includex}{\textsf{includex}} and
\href{http://ctan.org/pkg/standalone}{\textsf{standalone}}
provide commands to include only the document body of
a child file thus allowing both files to be compiled individually.
\item
The packages \href{http://ctan.org/pkg/subdocs}{\textsf{subdocs}}
and \href{http://ctan.org/pkg/subfiles}{\textsf{subfiles}}
provide structures in which the main and child documents can be
encapsulated and allowing them to be compiled individually.
The inclusion mechanism is different from the conventional |\include|.
\item
The package \href{http://ctan.org/pkg/combine}{\textsf{combine}}
is an elaborate solution to combine several documents into one.
\end{itemize}
%
See also the CTAN topic \href{http://ctan.org/topic/subdocs}{\textsf{subdocs}}
for further related packages.
The present package differs from the above solutions in that
a document structure constructed with the conventional |\include| mechanism
just needs two extra commands at the top of every file
such that all constituent files can be compiled individually.

%%%%%%%%%%%%%%%%%%%%%%%%%%%%%%%%%%%%%%%%%%%%%%%%%%%%%%%%%%%%%%%%%%%%%%%%%%%%%%%%
%\subsection{Feature Suggestions}
%
%The following is a list of features which may be useful for future
%versions of this package:
%%
%\begin{itemize}
%\item
%\ldots
%\end{itemize}

%%%%%%%%%%%%%%%%%%%%%%%%%%%%%%%%%%%%%%%%%%%%%%%%%%%%%%%%%%%%%%%%%%%%%%%%%%%%%%%%
\subsection{Revision History}

%%%%%%%%%%%%%%%%%%%%%%%%%%%%%%%%%%%%%%%%
\paragraph{v2.0:} 2018/12/30

\begin{itemize}
\item
immediate forward processing
\item
added |\childdocby| mechanism
\item
manual restructured
\end{itemize}

%%%%%%%%%%%%%%%%%%%%%%%%%%%%%%%%%%%%%%%%
\paragraph{v1.6:} 2018/01/17

\begin{itemize}
\item
application for development of include files
\item
corrections to manual
\end{itemize}

%%%%%%%%%%%%%%%%%%%%%%%%%%%%%%%%%%%%%%%%
\paragraph{v1.5:} 2017/05/21

\begin{itemize}
\item
more complete structuring introduced
\item
|\childdocof| introduced
\item
|\childdoc| renamed to |\childdocmain|
\item
|\childredirect| renamed to |\childdocforward| and |\childdocforwardprefix|
and functionality expanded
\end{itemize}

%%%%%%%%%%%%%%%%%%%%%%%%%%%%%%%%%%%%%%%%
\paragraph{v1.0:} 2017/04/27

\begin{itemize}
\item
manual and install package
\item
first version published on CTAN
\end{itemize}

%%%%%%%%%%%%%%%%%%%%%%%%%%%%%%%%%%%%%%%%
\paragraph{v0.6:} 2017/04/26

\begin{itemize}
\item
redirection mechanism added
\end{itemize}

%%%%%%%%%%%%%%%%%%%%%%%%%%%%%%%%%%%%%%%%
\paragraph{v0.5:} 2017/04/26

\begin{itemize}
\item
functionality in definition file
\end{itemize}


%%%%%%%%%%%%%%%%%%%%%%%%%%%%%%%%%%%%%%%%%%%%%%%%%%%%%%%%%%%%%%%%%%%%%%%%%%%%%%%%
%%%%%%%%%%%%%%%%%%%%%%%%%%%%%%%%%%%%%%%%%%%%%%%%%%%%%%%%%%%%%%%%%%%%%%%%%%%%%%%%
%%%%%%%%%%%%%%%%%%%%%%%%%%%%%%%%%%%%%%%%%%%%%%%%%%%%%%%%%%%%%%%%%%%%%%%%%%%%%%%%
\appendix

\settowidth\MacroIndent{\rmfamily\scriptsize 000\ }

 \DocInput{childdoc.dtx}

\end{document}
%</driver>
% \fi
%
% %%%%%%%%%%%%%%%%%%%%%%%%%%%%%%%%%%%%%%%%%%%%%%%%%%%%%%%%%%%%%%%%%%%%%%%%%%%%%%
% %%%%%%%%%%%%%%%%%%%%%%%%%%%%%%%%%%%%%%%%%%%%%%%%%%%%%%%%%%%%%%%%%%%%%%%%%%%%%%
% \section{Sample}
%\iffalse
%<*samplemain>
%\fi
%
% The following presents a sample document
% with two chapters, two parts, a title page,
% a compile flag as well as three forwarding files to set the flag.
% It consists of eight |.tex| files:
% \begin{center}
% \begin{tabular}{ll}
% |cdocsamp.tex|&main file\\
% |cdocsch1.tex|&include file for chapter 1\\
% |cdocsch2.tex|&include file for chapter 2\\
% |cdocspt3.tex|&include file for part 3\\
% |cdocspt4.tex|&include file for part 4\\
% |cdocsdrf.tex|&forwarding file for main file in draft mode\\
% |cdocsfi1.tex|&forwarding file for final version of chapter 1\\
% |cdocsfi2.tex|&forwarding file for final version of chapter 2\\
% \end{tabular}
% \end{center}
% Each of the eight files can be compiled directly by the \LaTeX{} compiler.
%
% %%%%%%%%%%%%%%%%%%%%%%%%%%%%%%%%%%%%%%
% \paragraph{Main File.}
%
% The main file is called |cdocsamp.tex|.
%
% Load the \textsf{childdoc} definitions and
% declare the filename for the main document:
%    \begin{macrocode}
\input{childdoc.def}
\childdocmain{}
%    \end{macrocode}

% Optional override for |\version| flag:
%    \begin{macrocode}
%%\ifchilddoc\else\providecommand{\version}{draft}\fi
%    \end{macrocode}

% Define the default values for the |\version| flag
% (|final| for the main file and |draft| for childs):
%    \begin{macrocode}
\ifchilddoc
\providecommand{\version}{draft}
\else
\providecommand{\version}{final}
\fi
%    \end{macrocode}

% Load the standard document class:
%    \begin{macrocode}
\documentclass[12pt]{article}
%    \end{macrocode}

% Start the document body:
%    \begin{macrocode}
\begin{document}
%    \end{macrocode}

% Declare a title page.
% Print title, part of document being processed and version flag:
%    \begin{macrocode}
\addtocounter{page}{-1}
\begin{center}
{\LARGE\bfseries{}childdoc example\par}
\vspace{1cm}
\ifchilddoc
\ifchilddocmanual part\else chapter\fi:
`\childdocname' of `\childdocjob'\par
\else
main document: `\childdocjob'\par
\fi
version: \version\par
\end{center}
\newpage
%    \end{macrocode}

% Manually include selected file,
% otherwise process as usual:
%    \begin{macrocode}
\ifchilddocmanual
\section*{part `\childdocname'}
\input{\childdocname}
\else
%    \end{macrocode}

% Include the two chapters:
%    \begin{macrocode}
\include{cdocsch1}
\include{cdocsch2}
%    \end{macrocode}

% Include the two parts unless only chapters should be displayed:
%    \begin{macrocode}
\ifchilddoc\else
\section{part three}
\input{cdocspt3}
\section{part four}
\input{cdocspt4}
\fi
%    \end{macrocode}

% Process as usual until here:
%    \begin{macrocode}
\fi
%    \end{macrocode}

% End of document body:
%    \begin{macrocode}
\end{document}
%    \end{macrocode}
%\iffalse
%</samplemain>
%\fi
%
% %%%%%%%%%%%%%%%%%%%%%%%%%%%%%%%%%%%%%%
% \paragraph{Chapter Include Files.}
%
% The include files are called |cdocsch1.tex| and |cdocsch2.tex|.
%
%\iffalse
%<*samplechap1|samplechap2>
%\fi

% Optional override for |\version| flag:
%    \begin{macrocode}
%%\providecommand{\version}{final}
%    \end{macrocode}

% Include the main document:
%    \begin{macrocode}
\input{childdoc.def}
\childdocof{cdocsamp}
%    \end{macrocode}

%\iffalse
%</samplechap1|samplechap2>
%\fi
%
%\iffalse
%<*samplechap1>
%\fi
% Some text for chapter 1:
%    \begin{macrocode}
\section{one}
some text in chapter one
%    \end{macrocode}

%\iffalse
%</samplechap1>
%\fi
% Some text for chapter 2:
%\iffalse
%<*samplechap2>
%\fi
%    \begin{macrocode}
\section{two}
more text in chapter two
%    \end{macrocode}

%\iffalse
%</samplechap2>
%\fi
%
% %%%%%%%%%%%%%%%%%%%%%%%%%%%%%%%%%%%%%%
% \paragraph{Part Include Files.}
%
% The include files are called |cdocspt3.tex| and |cdocspt4.tex|.
%
%\iffalse
%<*samplepart3|samplepart4>
%\fi

% Optional override for |\version| flag:
%    \begin{macrocode}
%%\providecommand{\version}{final}
%    \end{macrocode}

% Include the main document:
%    \begin{macrocode}
\input{childdoc.def}
\childdocby{cdocsamp}
%    \end{macrocode}

%\iffalse
%</samplepart3|samplepart4>
%\fi
%
%\iffalse
%<*samplepart3>
%\fi
% Some text for part 3:
%    \begin{macrocode}
some text in part three
%    \end{macrocode}

%\iffalse
%</samplepart3>
%\fi
% Some text for part 4:
%\iffalse
%<*samplepart4>
%\fi
%    \begin{macrocode}
more text in part four
%    \end{macrocode}

%\iffalse
%</samplepart4>
%\fi
%
% %%%%%%%%%%%%%%%%%%%%%%%%%%%%%%%%%%%%%%
% \paragraph{Forwarding for a Complete Draft.}
%
% The following forwarding file |cdocsdrf.tex|
% compiles the main document in draft mode:
%\iffalse
%<*sampledraft>
%\fi
%    \begin{macrocode}
\def\version{draft}
\input{childdoc.def}
\childdocforward{cdocsamp}
%    \end{macrocode}

%\iffalse
%</sampledraft>
%\fi
%
% %%%%%%%%%%%%%%%%%%%%%%%%%%%%%%%%%%%%%%
% \paragraph{Forwarding for Final Version of the Chapters.}
%
% The following forwarding files |cdocsfn1.tex| and |cdocsfn2.tex|
% (with identical content)
% compile the final versions of the child documents
% |cdocsch1.tex| and |cdocsch2.tex|, respectively:
%\iffalse
%<*samplefinal>
%\fi
%    \begin{macrocode}
\def\version{final}
\input{childdoc.def}
\childdocforwardprefix[cdocsamp]{cdocsfn}{cdocsch}
%    \end{macrocode}

%\iffalse
%</samplefinal>
%\fi
%
% %%%%%%%%%%%%%%%%%%%%%%%%%%%%%%%%%%%%%%
% \paragraph{Command Line Processing.}
%
% The following three command lines generate the output files
% |cdocscld|, |cdocscl1| and |cdocscl2|
% which should be identical to
% |cdocsdrf|, |cdocsch1| and |cdocsfn2|, respectively:
% \begin{center}
% \begin{tabular}{l}
% |latex -jobname cdocscld \|\\
% |  "\def\version{draft}\input{childdoc.def}\childdocforward{cdocsamp}"|\\
% |latex -jobname cdocscl1 \|\\
% |  "\input{childdoc.def}\childdocforward[cdocsamp]{cdocsch1}"|\\
% |latex -jobname cdocscl2 \|\\
% |  "\def\version{final}\input{childdoc.def}\childdocforward{cdocsch2}"|
% \end{tabular}
% \end{center}
% Note that the trailing backslash on each first line
% merely continues the input to the second line
% (for convenient cut ant paste).
% Furthermore, the command |latex| can be replaced by any
% of its alternative versions such as |pdflatex|.
%
% %%%%%%%%%%%%%%%%%%%%%%%%%%%%%%%%%%%%%%%%%%%%%%%%%%%%%%%%%%%%%%%%%%%%%%%%%%%%%%
% %%%%%%%%%%%%%%%%%%%%%%%%%%%%%%%%%%%%%%%%%%%%%%%%%%%%%%%%%%%%%%%%%%%%%%%%%%%%%%
% \section{Implementation}
%\iffalse
%<*package>
%\fi
%
% This section describes the definitions file |childdoc.def|.

% The definitions cannot be loaded using |\usepackage| or |\RequirePackage|
% which has a mechanism to prevent loading a style file more than once.
% When loading the definitions by means of |\input|
% multiple instances have to be prevented manually:
%\iffalse
%This code needs to be before the `\ProvidesFile' directive
%which is defined at the beginning of this file.
%Therefore it is also placed there and commented out here.
%</package>
%<*discard>
%\fi
%    \begin{macrocode}
\ifdefined\childdocmain\endinput\fi
%    \end{macrocode}
%\iffalse
%</discard>
%<*package>
%\fi
%
% \macro{\ifchilddoc}
% \macro{\ifchilddocmanual}
% The conditional |\ifchilddoc| tells whether a
% child (true) or main (false) document is being compiled.
% The conditional |\ifchilddocmanual| tells whether
% the |\includeonly| mechanism is used (false) or
% the selection of child files must be performed manually (true).
% The definitions initialise to false:
%    \begin{macrocode}
\newif\ifchilddoc
\newif\ifchilddocmanual
%    \end{macrocode}

% \macro{\childdocname}
% \macro{\childdocjob}
% The macro |\childdocname| stores the name of the main document
% to be compiled. The macro |\childdocjob| stores the name of
% the document on which the \LaTeX{} compiler was originally invoked.
% The content of |\jobname| cannot be compared
% to filenames specified in the source due to different catcodes.
% The following code rescans |\jobname|, stores the result
% in |\childdocname| and saves a copy in |\childdocjob|:
%    \begin{macrocode}
\edef\childdocname{\scantokens\expandafter{\jobname\noexpand}}
\let\childdocjob\childdocname
%    \end{macrocode}

% \macro{\childdocdisable}
% The macro |\childdocdisable| prevents the main file
% from being processed more than once.
% At this stage, the main document command |\childdocmain|
% is assumed to be called once again where it should do nothing.
% Any subsequent call to it should prevent
% a secondary processing of the main document
% It overwrites the forwarding commands
% |\childdocof| and |\childdocforward|
% with empty macros to prevent further inclusions of the main document:
%    \begin{macrocode}
\newcommand{\childdocdisable}
{
  \renewcommand{\childdocmain}[1]{\renewcommand{\childdocmain}[1]{\endinput}}
  \renewcommand{\childdocof}[1]{}
  \renewcommand{\childdocby}[2][]{}
  \renewcommand{\childdocforward}[2][]{}
  \renewcommand{\childdocdisable}{}
}
%    \end{macrocode}

% \macro{\childdocmain}
% The macro |\childdocmain| is to be called at the top of the main file
% with nothing or the main filename (without extension) as argument.
% First, it breaks loops.
% If the argument is not empty and does not match |\childdocname|
% (which is set by the first inclusion of |childdoc.def|),
% |\ifchilddoc| is set to true, |\includeonly| is applied to the child file
% and |\jobname| is set to the main file
% (for proper handling of |.aux| files):
%    \begin{macrocode}
\newcommand{\childdocmain}[1]
{
  \childdocdisable\childdocmain{}
  \if?#1?\else
    \begingroup
      \def\childdoctmp{#1}
      \ifx\childdoctmp\childdocname
        \def\childdoctmp{}
      \else
        \def\childdoctmp
        {
          \childdoctrue
          \includeonly{\childdocname}
          \def\childdocjob{#1}
          \def\jobname{#1}
        }
      \fi
      \expandafter
    \endgroup
    \childdoctmp
  \fi
}
%    \end{macrocode}

% \macro{\childdocof}
% The command |\childdocof| redirects
% compilation to the main file |#1|.
%    \begin{macrocode}
\newcommand{\childdocof}[1]
{
  \childdocdisable
  \childdoctrue
  \includeonly{\childdocname}
  \def\jobname{#1}
  \def\childdocjob{#1}
  \input{#1}
}
%    \end{macrocode}

% \macro{\childdocby}
% The command |\childdocby| ....
%    \begin{macrocode}
\newcommand{\childdocby}[2][]
{
  \childdocdisable
  \childdoctrue
  \childdocmanualtrue
  \if?#1?\else
    \def\jobname{#2}
  \fi
  \def\childdocjob{#2}
  \input{#2}
  \endinput
}
%    \end{macrocode}

% \macro{\childdocforward}
% The command |\childdocforward| redirects
% compilation to the main file or
% (if the optional argument is given) a child file.
% Parameters are set as if the main file
% or a child file starting with |\childdocof| was compiled.
% Then compilation is handed over to the main file:
%    \begin{macrocode}
\newcommand{\childdocforward}[2][]
{
  \begingroup
    \if?#1?
      \def\childdoctmp
      {
        \def\childdocname{#2}
        \def\childdocjob{#2}
        \def\jobname{#2}
        \input{#2}
        \endinput
      }
    \else
      \def\childdoctmp
      {
        \childdocdisable
        \def\childdocname{#2}
        \childdoctrue
        \includeonly{#2}
        \def\childdocjob{#1}
        \def\jobname{#1}
        \input{#1}
        \endinput
      }
    \fi
    \expandafter
  \endgroup
  \childdoctmp
}
%    \end{macrocode}

% \macro{\childdocforwardprefix}
% The command |\childdocforwardprefix| redirects
% compilation to the main or a child file by means of a pattern.
% The prefix |#1| in the current filename is replaced by |#2|
% and the suffix of the current filename is kept
% (it is assumed that the filename does not contain the substring `|~~~|'
% which is used as a delimiter).
% Compilation is handed over to the new file by |\childdocforward|:
%    \begin{macrocode}
\newcommand{\childdocforwardprefix}[3][]
{
  \begingroup
    \def\childdocextract #2##1~~~{\def\childdoctmp{\childdocforward[#1]{#3##1}}}
    \expandafter\childdocextract\childdocname~~~
    \expandafter
  \endgroup
  \childdoctmp
}
%    \end{macrocode}

% \macro{\childdoc}
% The deprecated macro |\childdoc| is a legacy version of |\childdocmain|:
%    \begin{macrocode}
\newcommand{\childdoc}{\childdocmain}
%    \end{macrocode}

% \macro{\childdocredirect}
% The deprecated macro |\childdocredirect| is a legacy version
% of |\childdocforward| and |\childdocforwardprefix|:
%    \begin{macrocode}
\newcommand{\childdocredirect}[2][]
{
  \begingroup
    \if?#1?
      \def\childdoctmp{\childdocforward{#2}}
    \else
      \def\childdoctmp{\childdocforwardprefix{#1}{#2}}
    \fi
    \expandafter
  \endgroup
  \childdoctmp
}
%    \end{macrocode}

%\iffalse
%</package>
%\fi
%
\endinput
|\\
|\childdocmain{|\textit{main}|}|\\
\end{tabular}
\end{center}
%
If |\jobname| does not match the argument \textit{main} of |\childdocmain|,
it is assumed that |\jobname| points to the child file to be compiled.
When using |\childdocmain| with the main file specified as argument,
it suffices to start a child file
with just |\input{|\textit{main}|}|
without loading of the package and using |\childdocof|.
If instead all processing is done
with the appropriate \textsf{childdoc} directives,
the argument of \textit{main} of |\childdocmain| can be empty.

An alternative version of the command line processing described
in \secref{sec:commandline} using the detection mechanism reads:
%
\begin{center}
|... -jobname "|\textit{target}|" "|[\textit{flags}]%
[|\def\jobname{|\textit{dest}|}|]|\input{|\textit{main}|}"|
\end{center}

%%%%%%%%%%%%%%%%%%%%%%%%%%%%%%%%%%%%%%%%%%%%%%%%%%%%%%%%%%%%%%%%%%%%%%%%%%%%%%%%
\subsection{Manual Code}
\label{sec:manual}

In case one cannot be certain whether the definitions file |childdoc.def|
is installed on the target \TeX{} distribution
and one prefers not to ship it,
it is conceivable to paste a few relevant commands into the sources.

To that end, drop all statements |% \iffalse
%
% childdoc.dtx Copyright (C) 2017-2018 Niklas Beisert
%
% This work may be distributed and/or modified under the
% conditions of the LaTeX Project Public License, either version 1.3
% of this license or (at your option) any later version.
% The latest version of this license is in
%   http://www.latex-project.org/lppl.txt
% and version 1.3 or later is part of all distributions of LaTeX
% version 2005/12/01 or later.
%
% This work has the LPPL maintenance status `maintained'.
%
% The Current Maintainer of this work is Niklas Beisert.
%
% This work consists of the files childdoc.dtx and childdoc.ins
% and the derived files childdoc.def and cdocsamp.tex with
% cdocsch1.tex, cdocsch2.tex, cdocsdrf.tex, cdocsfn1.tex, cdocsfn2.tex.
%
%<package>\ifdefined\childdocmain\endinput\fi
%<package>\ProvidesFile{childdoc.def}[2018/12/30 v2.0 child document driver]
%<samplemain>\ProvidesFile{cdocsamp.tex}[2018/12/30 v2.0 sample for childdoc]
%<*driver>
%\ProvidesFile{childdoc.drv}[2018/12/30 v2.0 childdoc reference manual file]
\PassOptionsToClass{10pt,a4paper}{article}
\documentclass{ltxdoc}

\usepackage[margin=35mm]{geometry}
\usepackage{hyperref}
\usepackage{hyperxmp}
\usepackage[usenames]{color}

\hypersetup{colorlinks=true}
\hypersetup{pdfstartview=FitH}
\hypersetup{pdfpagemode=UseNone}
\hypersetup{pdfsource={}}
\hypersetup{pdflang={en-UK}}
\hypersetup{pdfcopyright={Copyright 2017-2018 Niklas Beisert.
  This work may be distributed and/or modified under the
  conditions of the LaTeX Project Public License, either version 1.3
  of this license or (at your option) any later version.}}
\hypersetup{pdflicenseurl={http://www.latex-project.org/lppl.txt}}
\hypersetup{pdfcontactaddress={ETH Zurich, ITP, HIT K,
  Wolfgang-Pauli-Strasse 27}}
\hypersetup{pdfcontactpostcode={8093}}
\hypersetup{pdfcontactcity={Zurich}}
\hypersetup{pdfcontactcountry={Switzerland}}
\hypersetup{pdfcontactemail={nbeisert@itp.phys.ethz.ch}}
\hypersetup{pdfcontacturl={http://people.phys.ethz.ch/\xmptilde nbeisert/}}

\newcommand{\secref}[1]{\hyperref[#1]{section \ref*{#1}}}

\parskip1ex
\parindent0pt
\let\olditemize\itemize
\def\itemize{\olditemize\parskip0pt}

\begin{document}

\title{The \textsf{childdoc} Package}
\hypersetup{pdftitle={The childdoc Package}}
\author{Niklas Beisert\\[2ex]
  Institut f\"ur Theoretische Physik\\
  Eidgen\"ossische Technische Hochschule Z\"urich\\
  Wolfgang-Pauli-Strasse 27, 8093 Z\"urich, Switzerland\\[1ex]
  \href{mailto:nbeisert@itp.phys.ethz.ch}
  {\texttt{nbeisert@itp.phys.ethz.ch}}}
\hypersetup{pdfauthor={Niklas Beisert}}
\hypersetup{pdfsubject={Manual for the LaTeX2e Package childdoc}}
\date{30 December 2018, \textsf{v2.0}}
\maketitle

\begin{abstract}\noindent
\textsf{childdoc} is a \LaTeXe{} package
that enables the direct compilation
of document sections included by |\include|
to individual files.
\end{abstract}

\begingroup
\parskip0ex
\tableofcontents
\endgroup

%%%%%%%%%%%%%%%%%%%%%%%%%%%%%%%%%%%%%%%%%%%%%%%%%%%%%%%%%%%%%%%%%%%%%%%%%%%%%%%%
%%%%%%%%%%%%%%%%%%%%%%%%%%%%%%%%%%%%%%%%%%%%%%%%%%%%%%%%%%%%%%%%%%%%%%%%%%%%%%%%
\section{Introduction}

\LaTeX{} provides a mechanism to structure a large document (such as a book)
into a main file and several child files (containing the chapters)
using the |\include| command.
This mechanism is beneficial for documents
which span hundreds of pages in order to
make the source file(s) more manageable.
Moreover, compilation can be restricted to
selected child files by means of the |\includeonly| command.
The latter feature can be used to reduce the compilation time while editing
(this was significantly more useful in the earlier days of \LaTeX{})
or to generate a smaller document which is easier to navigate.
Another application of |\includeonly| is to generate
documents consisting of selected parts of the complete document.

However, there are a few drawbacks of the plain |\include| mechanism:
\begin{itemize}
\item
The child files cannot be compiled on their own,
they can only be compiled via the main file.
A naive editing environment
(such as a text editor with an option
to have the current file processed by \LaTeX)
may require one to switch to the main file before compiling;
attempting to compile the child file produces errors.
\item
The main file must be modified (each time)
to adjust the |\includeonly| command
to the present needs. This easily leaves the main file in a messy state.
\item
The generated document will always carry the filename
of the main document. This is inconvenient if
several child files are to be compiled and
to be kept for distribution.
\end{itemize}

The present package provides a simple interface
to make child files individually compilable by \LaTeX{}.
Compiling a child file then has the same effect as compiling
the main file with an |\includeonly| command
to select the appropriate child.
Moreover the generated document will carry the name of the child
rather than the main file.
This resolves all three above issues.

This feature is meant to make the editing of books,
thesis documents and lecture notes somewhat more convenient.
However, the package can also be used efficiently for
composing a series of documents (such as exercise sheets)
which are typically distributed individually.
It then assists the author in generating the individual documents
(potentially in different versions)
as well as a document containing the collected series.
Another application is in developing style files
or other kinds of included material
where compilation of the style file could redirect
to a sample or test file.

%%%%%%%%%%%%%%%%%%%%%%%%%%%%%%%%%%%%%%%%%%%%%%%%%%%%%%%%%%%%%%%%%%%%%%%%%%%%%%%%
%%%%%%%%%%%%%%%%%%%%%%%%%%%%%%%%%%%%%%%%%%%%%%%%%%%%%%%%%%%%%%%%%%%%%%%%%%%%%%%%
\section{Usage}

First of all, the package \textsf{childdoc} is \emph{not} a standard
\LaTeXe{} |.sty| style file! Therefore it needs to be invoked in
a non-standard way.

%%%%%%%%%%%%%%%%%%%%%%%%%%%%%%%%%%%%%%%%%%%%%%%%%%%%%%%%%%%%%%%%%%%%%%%%%%%%%%%%
\subsection{Included Files}
\label{sec:include}

%%%%%%%%%%%%%%%%%%%%%%%%%%%%%%%%%%%%%%%%
\DescribeMacro{\childdocmain}
To use the package, add the commands
\begin{center}
\begin{tabular}{l}
|\input{childdoc.def}|\\
|\childdocmain{}|\\
\end{tabular}
\end{center}
at the very top of the main \LaTeX{} file,
in particular \emph{before} the |\documentclass| statement!
The argument of |\childdocmain| should be left empty
(but it must be present).

%%%%%%%%%%%%%%%%%%%%%%%%%%%%%%%%%%%%%%%%
\DescribeMacro{\childdocof}
Furthermore, add the commands
\begin{center}
\begin{tabular}{l}
|\input{childdoc.def}|\\
|\childdocof{|\textit{main}|}|\\
\end{tabular}
\end{center}
at the top of every child file \textit{child}
which is included by |\include{|\textit{child}|}|
from within the main file
(or at least for those files to be compiled individually).
The argument \textit{main} must be the filename of the main file.

There are a couple of
considerations in setting up the main and child documents:

%%%%%%%%%%%%%%%%%%%%%%%%%%%%%%%%%%%%%%%%
\paragraph{Restrictions.}

Please note the following restrictions:
\begin{itemize}
\item
|\childdocmain| must be called with one argument \textit{main}
to ensure compatibility with earlier version of the package.
It must either be empty (|\childdocmain{}|)
or precisely match the filename of the main file in which it is specified.
See \secref{sec:detection} for further information.
\item
The filename \textit{main} must be specified without the |.tex| extension.
\item
The filename \textit{main} is case sensitive
(even in case-insensitive file systems)
due to internal string comparison.
\item
The argument \textit{main} should be fully expanded, it cannot be a macro.
\item
Subdirectories and special characters should be avoided in filenames.
\item
The command |\childdocmain{|\textit{main}|}| must be followed by a whitespace.
It should not be followed immediately by another command
or by a comment mark `|%|'.
This is because the \TeX{} parser reads the token immediately following
the argument of |\childdocmain| and puts it
at the beginning of every child section;
however, a white\-space is ignored.
\end{itemize}

%%%%%%%%%%%%%%%%%%%%%%%%%%%%%%%%%%%%%%%%
\paragraph{Content of Main File.}

It is advisable to place all content in the child files included by |\include|.
Any output contained in the main file will appear in all child documents
unless suppressed manually;
it cannot be suppressed automatically by the |\includeonly| directive
and thus should normally be avoided.
A method to include some content in the main file
by means of conditional processing is described in \secref{sec:conditional}.

%%%%%%%%%%%%%%%%%%%%%%%%%%%%%%%%%%%%%%%%
\paragraph{Page Numbering.}

When only a part of the document is compiled,
the appropriate numbering of pages
(as well as other status parameters)
is determined from the |.aux| files.
The latter contain information from previous passes.
However this information needs to propagate through
all intermediate child documents.
Therefore the page numbering in child documents may well
be inconsistent until the complete document is compiled at least once.

A useful (if unconventional) way to always ensure a consistent
page numbering is to restart the numbering in each child document
and denote the pages by `\textit{child}|.|\textit{page}'
where \textit{child} represents the chapter/section number of the child file.
This can be achieved by the command
|\numberwithin{page}{|\textit{child}|}|
of the \textsf{amsmath} package
where \textit{child} can be |chapter| or |section|
depending on the chosen structuring.
Alternatively, one can modify the macro |\thepage| appropriately
and reset the counter |page| at the start of each child file.

%%%%%%%%%%%%%%%%%%%%%%%%%%%%%%%%%%%%%%%%%%%%%%%%%%%%%%%%%%%%%%%%%%%%%%%%%%%%%%%%
\subsection{Conditional Processing}
\label{sec:conditional}

The package provides a mechanism to compile different versions
of a document. To customise the versions further some conditional processing
can come in handy to distinguish which version is being compiled.
The package provides two macros to describe the compilation context:

%%%%%%%%%%%%%%%%%%%%%%%%%%%%%%%%%%%%%%%%
\DescribeMacro{\ifchilddoc}
The conditional |\ifchilddoc| distinguishes between the compilation of
child documents and the main document:
%
\begin{center}
|\ifchilddoc |\textit{child-code}| |[|\||else |\textit{main-code}]| \||fi|
\end{center}

%%%%%%%%%%%%%%%%%%%%%%%%%%%%%%%%%%%%%%%%
\DescribeMacro{\childdocname}
\DescribeMacro{\childdocjob}
The macro |\childdocname| contains the filename (without extension)
of the main or child file being processed.
Note that |\childdocjob| will always contain the name of the main file.

%%%%%%%%%%%%%%%%%%%%%%%%%%%%%%%%%%%%%%%%
\paragraph{Title Page.}

Conditional processing can be used to include a title or banner page
in the main document when proper precautions are taken.
Importantly, the code in the main file should ensure that the page counter
(as well as other status parameters which are stored in the |.aux| files)
takes the same value after the conditional processing.
Otherwise the page numbers may take divergent values
depending on which part is compiled.

For example, a title page could be declared by:
%
\begin{center}
\begin{tabular}{l}
|\ifchilddoc\||else|\\
|\addtocounter{page}{-1}|\\
\textit{code for title page}\\
|\newpage|\\
|\||fi|
\end{tabular}
\end{center}
%
A banner page for the child documents can be generated by:
%
\begin{center}
\begin{tabular}{l}
|\ifchilddoc|\\
|\addtocounter{page}{-1}|\\
\textit{code for banner page}\\
|\newpage|\\
|\||fi|
\end{tabular}
\end{center}
%
Here one could write a message such as:
\begin{center}
|This is the part \childdocname{} of \childdocjob{}.|
\end{center}

%%%%%%%%%%%%%%%%%%%%%%%%%%%%%%%%%%%%%%%%%%%%%%%%%%%%%%%%%%%%%%%%%%%%%%%%%%%%%%%%
\subsection{Flags}
\label{sec:flags}

The package makes it easy to generate different versions
of the main or child documents.
To this end compilation flags can be defined
and assigned different default values.
They will be particularly useful in conjunction
with the forwarding mechanism described in \secref{sec:forward}.

For example, it may be useful to have a flag |\version|
which can be set to |draft| or |final|.
The document source will contain some conditional code
depending on the value of |\version|.
Suppose further, the flag should default to |final| for the main file
and to |draft| for child files
which is a natural assignment for editing the document.
This is achieved by placing the following code
in the preamble of the main document
(below the |\childdocmain| directive):
%
\begin{center}
\begin{tabular}{l}
|\ifchilddoc|\\
|\providecommand{\version}{draft}|\\
|\||else|\\
|\providecommand{\version}{final}|\\
|\||fi|
\end{tabular}
\end{center}
%
The definition by |\providecommand| makes sure
that previous definitions are not overwritten.
Further statements |\providecommand{\version}{...}|
can thus be added before the above code to override it.

For the main file, one might add a line
(between |\childdocmain| and the above block)
%
\begin{center}
|%\ifchilddoc\||else\providecommand{\version}{draft}\||fi|
\end{center}
%
which can be uncommented to produce a draft version.
Likewise one can add a line to the very top of a child file
(above the |\childdocof{|\textit{main}|}| directive)
%
\begin{center}
|%\providecommand{\version}{final}|
\end{center}
%
which can be uncommented to produce the final version of this child document.

%%%%%%%%%%%%%%%%%%%%%%%%%%%%%%%%%%%%%%%%%%%%%%%%%%%%%%%%%%%%%%%%%%%%%%%%%%%%%%%%
\subsection{Forwarding}
\label{sec:forward}

Different versions of the main or child documents
using compilation flags as described in \secref{sec:flags}
can be (permanently) stored in different files
for convenient compilation, viewing and distribution.
To this end, the package defines a command
to pass on compilation to a different file:

%%%%%%%%%%%%%%%%%%%%%%%%%%%%%%%%%%%%%%%%
\DescribeMacro{\childdocforward}
The command |\childdocforward| redirects processing to
another source file:
%
\begin{center}
\begin{tabular}{l}
|\input{childdoc.def}|\\
|\childdocforward[|\textit{main}|]{|\textit{dest}|}|\\
\end{tabular}
\end{center}
%
The argument \textit{dest} is the destination file
(without extension).
It should be the main file or one of the child files.
Note that further \textsf{childdoc} directives
such as |\childdocof| and |\childdocforward|
in the indicated file will be processed in this form.
The optional argument \textit{main}
passes on directly to the main file \textit{main}
while pretending to compile the child \textit{dest}.
This form behaves as if \textit{dest}
issues |\childdocof{|\textit{main}|}| right away,
and no further \textsf{childdoc} directives will be processed.

%%%%%%%%%%%%%%%%%%%%%%%%%%%%%%%%%%%%%%%%
\DescribeMacro{\...prefix}
In the alternative form |\childdocforwardprefix|,
%
\begin{center}
\begin{tabular}{l}
|\input{childdoc.def}|\\
|\childdocforwardprefix[|\textit{main}|]{|\textit{prefix}|}{|\textit{dest}|}|
\end{tabular}
\end{center}
%
the destination file is determined by a pattern
depending on the current file:
To make this work, the current file must be called
`{\textit{prefix}\hspace{0.2em}\textit{suffix}}'
with \textit{prefix} matching precisely the argument.
Processing is then passed on to the file
`{\textit{dest}\hspace{0.2em}\textit{suffix}}'.
Surely, the same effect is achieved by
directly specifying the
argument `{\textit{dest}\hspace{0.2em}\textit{suffix}}'
in the first form.
However, that requires to set up a different file
for each child. With the alternative form of the command
all these files can have exactly the same content
which simplifies setting them up and maintaining them.

For example, the following file |draft.tex|
with a compilation flag |\version| as described in \secref{sec:flags}
compiles the main document as a draft:
%
\begin{center}
\begin{tabular}{l}
|\def\version{draft}|\\
|\input{childdoc.def}|\\
|\childdocforward{|\textit{main}|}|
\end{tabular}
\end{center}
%
Likewise, the following files |final|\textit{nn}|.tex|
compile the final version of the child document
|child|\textit{nn}|.tex|:
%
\begin{center}
\begin{tabular}{l}
|\def\version{final}|\\
|\input{childdoc.def}|\\
|\childdocforwardprefix{final}{child}|
\end{tabular}
\end{center}
%

Note that when several versions of a main file and/or of each child file
are to be generated, it may be convenient to set up a |Makefile| or
shell script to automatise the process.

%%%%%%%%%%%%%%%%%%%%%%%%%%%%%%%%%%%%%%%%%%%%%%%%%%%%%%%%%%%%%%%%%%%%%%%%%%%%%%%%
\subsection{Command Line Processing}
\label{sec:commandline}

The effect of redirection files can also be achieved by invoking
the \LaTeX{} compiler with a more elaborate command line.
Most conveniently this should be done as part
of a shell script or a |Makefile|.

When using \textsf{childdoc} in the main file, the following
command lines effectively perform a redirection
(note that depending on the shell being used,
backslashes may have to be doubled: `|\|' $\to$ `|\\|'):
%
\begin{center}
|... -jobname "|\textit{target}|" |\\|"|[\textit{flags}]%
|\input{childdoc.def}\childdocforward[|\textit{main}|]{|\textit{dest}|}"|
\end{center}
%
Here \textit{target} is the name of the output file,
\textit{main} is the name of the main file
and \textit{dest} is the name of the main or child file to be processed
(all filenames without extensions).
The optional argument \textit{main} can be omitted
if \textit{main} matches \textit{dest}.
Optionally, compilation \textit{flags} can be defined via |\def| commands.
This command line makes the \TeX{} engine believe
it is compiling the file \textit{target}
whose content is specified as the latter parameter.
The provided code then forwards the processing to
\textit{main} or \textit{dest} as described in \secref{sec:forward}.

%%%%%%%%%%%%%%%%%%%%%%%%%%%%%%%%%%%%%%%%%%%%%%%%%%%%%%%%%%%%%%%%%%%%%%%%%%%%%%%%
\subsection{Include by Input}
\label{sec:input}

Including child documents by |\include| has some restrictions by design.
Most notably, the content of a child document always occupies
its own set of pages; pages cannot be shared between child documents.
Usually, this behaviour makes perfect sense
because each child document contain an essential part of the document.
However, in some situations it may be desirable to compose
a document from a collection of parts
without having mandatory page breaks between then.
For this case, the package
provides a mechanism to include parts
by |\input| which can also be processed individually.
However, by construction this mechanism
requires manual handling of the content to be output.

%%%%%%%%%%%%%%%%%%%%%%%%%%%%%%%%%%%%%%%%
\DescribeMacro{\ifchilddocmanual}
The main file should be prepared as usual, see \secref{sec:include}.
However, the document body must make a distinction
between processing of an individual part and of the main document, e.g.:
%
\begin{center}
\begin{tabular}{l}
|\ifchilddocmanual|\\
|\input{\childdocname}|\\
|\||else|\\
\textit{document body with }|\input{|\textit{part}|}|\\
|\||fi|
\end{tabular}
\end{center}
%
The conditional |\ifchilddocmanual| is true whenever
a part to be included by |\input| is being compiled,
and the name of the part is stored in |\childdocname|.

%%%%%%%%%%%%%%%%%%%%%%%%%%%%%%%%%%%%%%%%
\DescribeMacro{\childdocby}
Each part to be included by |\input| should start with:
%
\begin{center}
\begin{tabular}{l}
|\input{childdoc.def}|\\
|\childdocby{|\textit{main}|}|\\
\end{tabular}
\end{center}
%
The directive |\childdocby| is similar to |\childdocof|
described in \secref{sec:include},
but the subsequent selection of content must be done manually.
To that end, both |\ifchilddoc| and |\ifchilddocmanual|
will be true upon processing of a part,
and the name of the part is stored in |\childdocname|.
Note that |\jobname| will be set to the filename of the current part
so that each part receives an individual |.aux| file
that does not interfere with the |.aux| file(s) of the main document.
This behaviour can be altered by the alternative form
|\childdocby[*]{|\textit{main}|}| (with a non-empty optional argument)
which uses the |.aux| file of the main document
by setting |\jobname| to \textit{main}.

%%%%%%%%%%%%%%%%%%%%%%%%%%%%%%%%%%%%%%%%%%%%%%%%%%%%%%%%%%%%%%%%%%%%%%%%%%%%%%%%
\subsection{Driver Development}
\label{sec:driver}

The \textsf{childdoc} mechanism can also be use for the development
of definition files such as \LaTeX{} styles or classes.
This case differs from the above setup with multiple parts
included by |\include| in that no |\includeonly| should be invoked.
This can be achieved by starting the include file
(before |\ProvidesPackage|) with:
%
\begin{center}
\begin{tabular}{l}
|\input{childdoc.def}|\\
|\childdocforward{|\textit{main}|}|\\
\end{tabular}
\end{center}
%
or alternatively with:
%
\begin{center}
\begin{tabular}{l}
|\input{childdoc.def}|\\
|\childdocby{|\textit{main}|}|\\
\end{tabular}
\end{center}
%
Both forms have slightly different effects as described above.
The main file is prepared as usual, see \secref{sec:include}.

%%%%%%%%%%%%%%%%%%%%%%%%%%%%%%%%%%%%%%%%%%%%%%%%%%%%%%%%%%%%%%%%%%%%%%%%%%%%%%%%
\subsection{Legacy Detection}
\label{sec:detection}

The directive |\childdocmain| in the main file can detect
whether the complete document or merely a child is to be compiled
even without using the directive |\childdocof|.
This method is deprecated because it is less robust
and there is no compelling reason to use it;
it is merely provided for backward compatibility
and it may be removed in future versions.

If the detection mechanism is to be used,
it is mandatory to correctly specify
the filename of the main file as the argument of |\childdocmain|:
%
\begin{center}
\begin{tabular}{l}
|\input{childdoc.def}|\\
|\childdocmain{|\textit{main}|}|\\
\end{tabular}
\end{center}
%
If |\jobname| does not match the argument \textit{main} of |\childdocmain|,
it is assumed that |\jobname| points to the child file to be compiled.
When using |\childdocmain| with the main file specified as argument,
it suffices to start a child file
with just |\input{|\textit{main}|}|
without loading of the package and using |\childdocof|.
If instead all processing is done
with the appropriate \textsf{childdoc} directives,
the argument of \textit{main} of |\childdocmain| can be empty.

An alternative version of the command line processing described
in \secref{sec:commandline} using the detection mechanism reads:
%
\begin{center}
|... -jobname "|\textit{target}|" "|[\textit{flags}]%
[|\def\jobname{|\textit{dest}|}|]|\input{|\textit{main}|}"|
\end{center}

%%%%%%%%%%%%%%%%%%%%%%%%%%%%%%%%%%%%%%%%%%%%%%%%%%%%%%%%%%%%%%%%%%%%%%%%%%%%%%%%
\subsection{Manual Code}
\label{sec:manual}

In case one cannot be certain whether the definitions file |childdoc.def|
is installed on the target \TeX{} distribution
and one prefers not to ship it,
it is conceivable to paste a few relevant commands into the sources.

To that end, drop all statements |\input{childdoc.def}|
and perform the replacements as outlined below.
Instead of |\childdocmain{|\textit{main}|}| add the following code
to the top of the main file:
%
\begin{center}
\begin{tabular}{l}
|\||ifdefined\childdocname\endinput\||fi\newif\ifchilddoc|\\
|\edef\childdocname{\scantokens\expandafter{\jobname\noexpand}}|\\
|\def\childdocmain{|\textit{main}|}\||ifx\childdocmain\childdocname\||else|\\
|\childdoctrue\includeonly{\childdocname}\let\jobname\childdocmain\||fi|\\
\end{tabular}
\end{center}
%
Instead of |\childdocof{|\textit{main}|}| just include the main file
at the top of each child file:
%
\begin{center}
|\input{|\textit{main}|}|
\end{center}
%
A simple redirection |\childdocforward{|\textit{dest}|}| is achieved by:
%
\begin{center}
|\def\jobname{|\textit{dest}|}\input{\jobname}|
\end{center}
%
The redirection with prefix
|\childdocforwardprefix[|\textit{prefix}|]{|\textit{dest}|}|
is accomplished by:
%
\begin{center}
\begin{tabular}{l}
|{\edef\jobname{\scantokens\expandafter{\jobname\noexpand}}|\\
|\def\redirectjob |\textit{prefix}|#1~~~{\gdef\jobname{|\textit{dest}|#1}}|\\
|\expandafter\redirectjob\jobname~~~}\input{\jobname}|
\end{tabular}
\end{center}

In an alternative approach,
child documents can be compiled by a specific command line
without additional code or specific definitions:
%
\begin{center}
|... -jobname "|\textit{target}|" "|[\textit{flags}]%
|\includeonly{|\textit{dest}|}\input{|\textit{main}|}"|
\end{center}
%

%%%%%%%%%%%%%%%%%%%%%%%%%%%%%%%%%%%%%%%%%%%%%%%%%%%%%%%%%%%%%%%%%%%%%%%%%%%%%%%%
%%%%%%%%%%%%%%%%%%%%%%%%%%%%%%%%%%%%%%%%%%%%%%%%%%%%%%%%%%%%%%%%%%%%%%%%%%%%%%%%
\section{Information}

%%%%%%%%%%%%%%%%%%%%%%%%%%%%%%%%%%%%%%%%%%%%%%%%%%%%%%%%%%%%%%%%%%%%%%%%%%%%%%%%
\subsection{Copyright}

Copyright \copyright{} 2017--2018 Niklas Beisert

This work may be distributed and/or modified under the
conditions of the \LaTeX{} Project Public License, either version 1.3
of this license or (at your option) any later version.
The latest version of this license is in
  \url{http://www.latex-project.org/lppl.txt}
and version 1.3 or later is part of all distributions of \LaTeX{}
version 2005/12/01 or later.

This work has the LPPL maintenance status `maintained'.

The Current Maintainer of this work is Niklas Beisert.

This work consists of the files |README.txt|, |childdoc.ins| and |childdoc.dtx|
as well as the derived files |childdoc.def|, |cdocsamp.tex|
with |cdocsch1.tex|, |cdocsch2.tex|, |cdocspt3.tex|, |cdocspt4.tex|,
|cdocsdrf.tex|, |cdocsfn1.tex|, |cdocsfn2.tex|
as well as |childdoc.pdf|.

%%%%%%%%%%%%%%%%%%%%%%%%%%%%%%%%%%%%%%%%%%%%%%%%%%%%%%%%%%%%%%%%%%%%%%%%%%%%%%%%
\subsection{Files and Installation}

The package consists of the files:
%
\begin{center}
\begin{tabular}{ll}
    |README.txt|   & readme file \\
    |childdoc.ins| & installation file \\
    |childdoc.dtx| & source file \\
    |childdoc.def| & definition file \\
    |cdocsamp.tex| & sample main file \\
    |cdocsch1.tex| & sample include file \\
    |cdocsch2.tex| & sample include file \\
    |cdocspt3.tex| & sample part file \\
    |cdocspt4.tex| & sample part file \\
    |cdocsdrf.tex| & sample redirection file \\
    |cdocsfn1.tex| & sample redirection file \\
    |cdocsfn2.tex| & sample redirection file \\
    |childdoc.pdf| & manual
\end{tabular}
\end{center}
%
The distribution consists of the files
|README.txt|, |childdoc.ins| and |childdoc.dtx|.
%
\begin{itemize}
\item
Run (pdf)\LaTeX{} on |childdoc.dtx|
to compile the manual |childdoc.pdf| (this file).
\item
Run \LaTeX{} on |childdoc.ins| to create the definitions file |childdoc.def|
and the sample |cdocsamp.tex| with include files
|cdocsch1.tex|, |cdocsch2.tex|, |cdocspt3.tex|, |cdocspt4.tex|,
|cdocsdrf.tex|, |cdocsfn1.tex|, |cdocsfn2.tex|.
Then copy the file |childdoc.def| to an appropriate directory of your \LaTeX{}
distribution, e.g.\ \textit{texmf-root}|/tex/latex/childdoc|.
\end{itemize}

%%%%%%%%%%%%%%%%%%%%%%%%%%%%%%%%%%%%%%%%%%%%%%%%%%%%%%%%%%%%%%%%%%%%%%%%%%%%%%%%
\subsection{Related CTAN Packages}

There are several other packages which offer a similar functionality:
%
\begin{itemize}
\item
The packages
\href{http://ctan.org/pkg/docmute}{\textsf{docmute}},
\href{http://ctan.org/pkg/includex}{\textsf{includex}} and
\href{http://ctan.org/pkg/standalone}{\textsf{standalone}}
provide commands to include only the document body of
a child file thus allowing both files to be compiled individually.
\item
The packages \href{http://ctan.org/pkg/subdocs}{\textsf{subdocs}}
and \href{http://ctan.org/pkg/subfiles}{\textsf{subfiles}}
provide structures in which the main and child documents can be
encapsulated and allowing them to be compiled individually.
The inclusion mechanism is different from the conventional |\include|.
\item
The package \href{http://ctan.org/pkg/combine}{\textsf{combine}}
is an elaborate solution to combine several documents into one.
\end{itemize}
%
See also the CTAN topic \href{http://ctan.org/topic/subdocs}{\textsf{subdocs}}
for further related packages.
The present package differs from the above solutions in that
a document structure constructed with the conventional |\include| mechanism
just needs two extra commands at the top of every file
such that all constituent files can be compiled individually.

%%%%%%%%%%%%%%%%%%%%%%%%%%%%%%%%%%%%%%%%%%%%%%%%%%%%%%%%%%%%%%%%%%%%%%%%%%%%%%%%
%\subsection{Feature Suggestions}
%
%The following is a list of features which may be useful for future
%versions of this package:
%%
%\begin{itemize}
%\item
%\ldots
%\end{itemize}

%%%%%%%%%%%%%%%%%%%%%%%%%%%%%%%%%%%%%%%%%%%%%%%%%%%%%%%%%%%%%%%%%%%%%%%%%%%%%%%%
\subsection{Revision History}

%%%%%%%%%%%%%%%%%%%%%%%%%%%%%%%%%%%%%%%%
\paragraph{v2.0:} 2018/12/30

\begin{itemize}
\item
immediate forward processing
\item
added |\childdocby| mechanism
\item
manual restructured
\end{itemize}

%%%%%%%%%%%%%%%%%%%%%%%%%%%%%%%%%%%%%%%%
\paragraph{v1.6:} 2018/01/17

\begin{itemize}
\item
application for development of include files
\item
corrections to manual
\end{itemize}

%%%%%%%%%%%%%%%%%%%%%%%%%%%%%%%%%%%%%%%%
\paragraph{v1.5:} 2017/05/21

\begin{itemize}
\item
more complete structuring introduced
\item
|\childdocof| introduced
\item
|\childdoc| renamed to |\childdocmain|
\item
|\childredirect| renamed to |\childdocforward| and |\childdocforwardprefix|
and functionality expanded
\end{itemize}

%%%%%%%%%%%%%%%%%%%%%%%%%%%%%%%%%%%%%%%%
\paragraph{v1.0:} 2017/04/27

\begin{itemize}
\item
manual and install package
\item
first version published on CTAN
\end{itemize}

%%%%%%%%%%%%%%%%%%%%%%%%%%%%%%%%%%%%%%%%
\paragraph{v0.6:} 2017/04/26

\begin{itemize}
\item
redirection mechanism added
\end{itemize}

%%%%%%%%%%%%%%%%%%%%%%%%%%%%%%%%%%%%%%%%
\paragraph{v0.5:} 2017/04/26

\begin{itemize}
\item
functionality in definition file
\end{itemize}


%%%%%%%%%%%%%%%%%%%%%%%%%%%%%%%%%%%%%%%%%%%%%%%%%%%%%%%%%%%%%%%%%%%%%%%%%%%%%%%%
%%%%%%%%%%%%%%%%%%%%%%%%%%%%%%%%%%%%%%%%%%%%%%%%%%%%%%%%%%%%%%%%%%%%%%%%%%%%%%%%
%%%%%%%%%%%%%%%%%%%%%%%%%%%%%%%%%%%%%%%%%%%%%%%%%%%%%%%%%%%%%%%%%%%%%%%%%%%%%%%%
\appendix

\settowidth\MacroIndent{\rmfamily\scriptsize 000\ }

 \DocInput{childdoc.dtx}

\end{document}
%</driver>
% \fi
%
% %%%%%%%%%%%%%%%%%%%%%%%%%%%%%%%%%%%%%%%%%%%%%%%%%%%%%%%%%%%%%%%%%%%%%%%%%%%%%%
% %%%%%%%%%%%%%%%%%%%%%%%%%%%%%%%%%%%%%%%%%%%%%%%%%%%%%%%%%%%%%%%%%%%%%%%%%%%%%%
% \section{Sample}
%\iffalse
%<*samplemain>
%\fi
%
% The following presents a sample document
% with two chapters, two parts, a title page,
% a compile flag as well as three forwarding files to set the flag.
% It consists of eight |.tex| files:
% \begin{center}
% \begin{tabular}{ll}
% |cdocsamp.tex|&main file\\
% |cdocsch1.tex|&include file for chapter 1\\
% |cdocsch2.tex|&include file for chapter 2\\
% |cdocspt3.tex|&include file for part 3\\
% |cdocspt4.tex|&include file for part 4\\
% |cdocsdrf.tex|&forwarding file for main file in draft mode\\
% |cdocsfi1.tex|&forwarding file for final version of chapter 1\\
% |cdocsfi2.tex|&forwarding file for final version of chapter 2\\
% \end{tabular}
% \end{center}
% Each of the eight files can be compiled directly by the \LaTeX{} compiler.
%
% %%%%%%%%%%%%%%%%%%%%%%%%%%%%%%%%%%%%%%
% \paragraph{Main File.}
%
% The main file is called |cdocsamp.tex|.
%
% Load the \textsf{childdoc} definitions and
% declare the filename for the main document:
%    \begin{macrocode}
\input{childdoc.def}
\childdocmain{}
%    \end{macrocode}

% Optional override for |\version| flag:
%    \begin{macrocode}
%%\ifchilddoc\else\providecommand{\version}{draft}\fi
%    \end{macrocode}

% Define the default values for the |\version| flag
% (|final| for the main file and |draft| for childs):
%    \begin{macrocode}
\ifchilddoc
\providecommand{\version}{draft}
\else
\providecommand{\version}{final}
\fi
%    \end{macrocode}

% Load the standard document class:
%    \begin{macrocode}
\documentclass[12pt]{article}
%    \end{macrocode}

% Start the document body:
%    \begin{macrocode}
\begin{document}
%    \end{macrocode}

% Declare a title page.
% Print title, part of document being processed and version flag:
%    \begin{macrocode}
\addtocounter{page}{-1}
\begin{center}
{\LARGE\bfseries{}childdoc example\par}
\vspace{1cm}
\ifchilddoc
\ifchilddocmanual part\else chapter\fi:
`\childdocname' of `\childdocjob'\par
\else
main document: `\childdocjob'\par
\fi
version: \version\par
\end{center}
\newpage
%    \end{macrocode}

% Manually include selected file,
% otherwise process as usual:
%    \begin{macrocode}
\ifchilddocmanual
\section*{part `\childdocname'}
\input{\childdocname}
\else
%    \end{macrocode}

% Include the two chapters:
%    \begin{macrocode}
\include{cdocsch1}
\include{cdocsch2}
%    \end{macrocode}

% Include the two parts unless only chapters should be displayed:
%    \begin{macrocode}
\ifchilddoc\else
\section{part three}
\input{cdocspt3}
\section{part four}
\input{cdocspt4}
\fi
%    \end{macrocode}

% Process as usual until here:
%    \begin{macrocode}
\fi
%    \end{macrocode}

% End of document body:
%    \begin{macrocode}
\end{document}
%    \end{macrocode}
%\iffalse
%</samplemain>
%\fi
%
% %%%%%%%%%%%%%%%%%%%%%%%%%%%%%%%%%%%%%%
% \paragraph{Chapter Include Files.}
%
% The include files are called |cdocsch1.tex| and |cdocsch2.tex|.
%
%\iffalse
%<*samplechap1|samplechap2>
%\fi

% Optional override for |\version| flag:
%    \begin{macrocode}
%%\providecommand{\version}{final}
%    \end{macrocode}

% Include the main document:
%    \begin{macrocode}
\input{childdoc.def}
\childdocof{cdocsamp}
%    \end{macrocode}

%\iffalse
%</samplechap1|samplechap2>
%\fi
%
%\iffalse
%<*samplechap1>
%\fi
% Some text for chapter 1:
%    \begin{macrocode}
\section{one}
some text in chapter one
%    \end{macrocode}

%\iffalse
%</samplechap1>
%\fi
% Some text for chapter 2:
%\iffalse
%<*samplechap2>
%\fi
%    \begin{macrocode}
\section{two}
more text in chapter two
%    \end{macrocode}

%\iffalse
%</samplechap2>
%\fi
%
% %%%%%%%%%%%%%%%%%%%%%%%%%%%%%%%%%%%%%%
% \paragraph{Part Include Files.}
%
% The include files are called |cdocspt3.tex| and |cdocspt4.tex|.
%
%\iffalse
%<*samplepart3|samplepart4>
%\fi

% Optional override for |\version| flag:
%    \begin{macrocode}
%%\providecommand{\version}{final}
%    \end{macrocode}

% Include the main document:
%    \begin{macrocode}
\input{childdoc.def}
\childdocby{cdocsamp}
%    \end{macrocode}

%\iffalse
%</samplepart3|samplepart4>
%\fi
%
%\iffalse
%<*samplepart3>
%\fi
% Some text for part 3:
%    \begin{macrocode}
some text in part three
%    \end{macrocode}

%\iffalse
%</samplepart3>
%\fi
% Some text for part 4:
%\iffalse
%<*samplepart4>
%\fi
%    \begin{macrocode}
more text in part four
%    \end{macrocode}

%\iffalse
%</samplepart4>
%\fi
%
% %%%%%%%%%%%%%%%%%%%%%%%%%%%%%%%%%%%%%%
% \paragraph{Forwarding for a Complete Draft.}
%
% The following forwarding file |cdocsdrf.tex|
% compiles the main document in draft mode:
%\iffalse
%<*sampledraft>
%\fi
%    \begin{macrocode}
\def\version{draft}
\input{childdoc.def}
\childdocforward{cdocsamp}
%    \end{macrocode}

%\iffalse
%</sampledraft>
%\fi
%
% %%%%%%%%%%%%%%%%%%%%%%%%%%%%%%%%%%%%%%
% \paragraph{Forwarding for Final Version of the Chapters.}
%
% The following forwarding files |cdocsfn1.tex| and |cdocsfn2.tex|
% (with identical content)
% compile the final versions of the child documents
% |cdocsch1.tex| and |cdocsch2.tex|, respectively:
%\iffalse
%<*samplefinal>
%\fi
%    \begin{macrocode}
\def\version{final}
\input{childdoc.def}
\childdocforwardprefix[cdocsamp]{cdocsfn}{cdocsch}
%    \end{macrocode}

%\iffalse
%</samplefinal>
%\fi
%
% %%%%%%%%%%%%%%%%%%%%%%%%%%%%%%%%%%%%%%
% \paragraph{Command Line Processing.}
%
% The following three command lines generate the output files
% |cdocscld|, |cdocscl1| and |cdocscl2|
% which should be identical to
% |cdocsdrf|, |cdocsch1| and |cdocsfn2|, respectively:
% \begin{center}
% \begin{tabular}{l}
% |latex -jobname cdocscld \|\\
% |  "\def\version{draft}\input{childdoc.def}\childdocforward{cdocsamp}"|\\
% |latex -jobname cdocscl1 \|\\
% |  "\input{childdoc.def}\childdocforward[cdocsamp]{cdocsch1}"|\\
% |latex -jobname cdocscl2 \|\\
% |  "\def\version{final}\input{childdoc.def}\childdocforward{cdocsch2}"|
% \end{tabular}
% \end{center}
% Note that the trailing backslash on each first line
% merely continues the input to the second line
% (for convenient cut ant paste).
% Furthermore, the command |latex| can be replaced by any
% of its alternative versions such as |pdflatex|.
%
% %%%%%%%%%%%%%%%%%%%%%%%%%%%%%%%%%%%%%%%%%%%%%%%%%%%%%%%%%%%%%%%%%%%%%%%%%%%%%%
% %%%%%%%%%%%%%%%%%%%%%%%%%%%%%%%%%%%%%%%%%%%%%%%%%%%%%%%%%%%%%%%%%%%%%%%%%%%%%%
% \section{Implementation}
%\iffalse
%<*package>
%\fi
%
% This section describes the definitions file |childdoc.def|.

% The definitions cannot be loaded using |\usepackage| or |\RequirePackage|
% which has a mechanism to prevent loading a style file more than once.
% When loading the definitions by means of |\input|
% multiple instances have to be prevented manually:
%\iffalse
%This code needs to be before the `\ProvidesFile' directive
%which is defined at the beginning of this file.
%Therefore it is also placed there and commented out here.
%</package>
%<*discard>
%\fi
%    \begin{macrocode}
\ifdefined\childdocmain\endinput\fi
%    \end{macrocode}
%\iffalse
%</discard>
%<*package>
%\fi
%
% \macro{\ifchilddoc}
% \macro{\ifchilddocmanual}
% The conditional |\ifchilddoc| tells whether a
% child (true) or main (false) document is being compiled.
% The conditional |\ifchilddocmanual| tells whether
% the |\includeonly| mechanism is used (false) or
% the selection of child files must be performed manually (true).
% The definitions initialise to false:
%    \begin{macrocode}
\newif\ifchilddoc
\newif\ifchilddocmanual
%    \end{macrocode}

% \macro{\childdocname}
% \macro{\childdocjob}
% The macro |\childdocname| stores the name of the main document
% to be compiled. The macro |\childdocjob| stores the name of
% the document on which the \LaTeX{} compiler was originally invoked.
% The content of |\jobname| cannot be compared
% to filenames specified in the source due to different catcodes.
% The following code rescans |\jobname|, stores the result
% in |\childdocname| and saves a copy in |\childdocjob|:
%    \begin{macrocode}
\edef\childdocname{\scantokens\expandafter{\jobname\noexpand}}
\let\childdocjob\childdocname
%    \end{macrocode}

% \macro{\childdocdisable}
% The macro |\childdocdisable| prevents the main file
% from being processed more than once.
% At this stage, the main document command |\childdocmain|
% is assumed to be called once again where it should do nothing.
% Any subsequent call to it should prevent
% a secondary processing of the main document
% It overwrites the forwarding commands
% |\childdocof| and |\childdocforward|
% with empty macros to prevent further inclusions of the main document:
%    \begin{macrocode}
\newcommand{\childdocdisable}
{
  \renewcommand{\childdocmain}[1]{\renewcommand{\childdocmain}[1]{\endinput}}
  \renewcommand{\childdocof}[1]{}
  \renewcommand{\childdocby}[2][]{}
  \renewcommand{\childdocforward}[2][]{}
  \renewcommand{\childdocdisable}{}
}
%    \end{macrocode}

% \macro{\childdocmain}
% The macro |\childdocmain| is to be called at the top of the main file
% with nothing or the main filename (without extension) as argument.
% First, it breaks loops.
% If the argument is not empty and does not match |\childdocname|
% (which is set by the first inclusion of |childdoc.def|),
% |\ifchilddoc| is set to true, |\includeonly| is applied to the child file
% and |\jobname| is set to the main file
% (for proper handling of |.aux| files):
%    \begin{macrocode}
\newcommand{\childdocmain}[1]
{
  \childdocdisable\childdocmain{}
  \if?#1?\else
    \begingroup
      \def\childdoctmp{#1}
      \ifx\childdoctmp\childdocname
        \def\childdoctmp{}
      \else
        \def\childdoctmp
        {
          \childdoctrue
          \includeonly{\childdocname}
          \def\childdocjob{#1}
          \def\jobname{#1}
        }
      \fi
      \expandafter
    \endgroup
    \childdoctmp
  \fi
}
%    \end{macrocode}

% \macro{\childdocof}
% The command |\childdocof| redirects
% compilation to the main file |#1|.
%    \begin{macrocode}
\newcommand{\childdocof}[1]
{
  \childdocdisable
  \childdoctrue
  \includeonly{\childdocname}
  \def\jobname{#1}
  \def\childdocjob{#1}
  \input{#1}
}
%    \end{macrocode}

% \macro{\childdocby}
% The command |\childdocby| ....
%    \begin{macrocode}
\newcommand{\childdocby}[2][]
{
  \childdocdisable
  \childdoctrue
  \childdocmanualtrue
  \if?#1?\else
    \def\jobname{#2}
  \fi
  \def\childdocjob{#2}
  \input{#2}
  \endinput
}
%    \end{macrocode}

% \macro{\childdocforward}
% The command |\childdocforward| redirects
% compilation to the main file or
% (if the optional argument is given) a child file.
% Parameters are set as if the main file
% or a child file starting with |\childdocof| was compiled.
% Then compilation is handed over to the main file:
%    \begin{macrocode}
\newcommand{\childdocforward}[2][]
{
  \begingroup
    \if?#1?
      \def\childdoctmp
      {
        \def\childdocname{#2}
        \def\childdocjob{#2}
        \def\jobname{#2}
        \input{#2}
        \endinput
      }
    \else
      \def\childdoctmp
      {
        \childdocdisable
        \def\childdocname{#2}
        \childdoctrue
        \includeonly{#2}
        \def\childdocjob{#1}
        \def\jobname{#1}
        \input{#1}
        \endinput
      }
    \fi
    \expandafter
  \endgroup
  \childdoctmp
}
%    \end{macrocode}

% \macro{\childdocforwardprefix}
% The command |\childdocforwardprefix| redirects
% compilation to the main or a child file by means of a pattern.
% The prefix |#1| in the current filename is replaced by |#2|
% and the suffix of the current filename is kept
% (it is assumed that the filename does not contain the substring `|~~~|'
% which is used as a delimiter).
% Compilation is handed over to the new file by |\childdocforward|:
%    \begin{macrocode}
\newcommand{\childdocforwardprefix}[3][]
{
  \begingroup
    \def\childdocextract #2##1~~~{\def\childdoctmp{\childdocforward[#1]{#3##1}}}
    \expandafter\childdocextract\childdocname~~~
    \expandafter
  \endgroup
  \childdoctmp
}
%    \end{macrocode}

% \macro{\childdoc}
% The deprecated macro |\childdoc| is a legacy version of |\childdocmain|:
%    \begin{macrocode}
\newcommand{\childdoc}{\childdocmain}
%    \end{macrocode}

% \macro{\childdocredirect}
% The deprecated macro |\childdocredirect| is a legacy version
% of |\childdocforward| and |\childdocforwardprefix|:
%    \begin{macrocode}
\newcommand{\childdocredirect}[2][]
{
  \begingroup
    \if?#1?
      \def\childdoctmp{\childdocforward{#2}}
    \else
      \def\childdoctmp{\childdocforwardprefix{#1}{#2}}
    \fi
    \expandafter
  \endgroup
  \childdoctmp
}
%    \end{macrocode}

%\iffalse
%</package>
%\fi
%
\endinput
|
and perform the replacements as outlined below.
Instead of |\childdocmain{|\textit{main}|}| add the following code
to the top of the main file:
%
\begin{center}
\begin{tabular}{l}
|\||ifdefined\childdocname\endinput\||fi\newif\ifchilddoc|\\
|\edef\childdocname{\scantokens\expandafter{\jobname\noexpand}}|\\
|\def\childdocmain{|\textit{main}|}\||ifx\childdocmain\childdocname\||else|\\
|\childdoctrue\includeonly{\childdocname}\let\jobname\childdocmain\||fi|\\
\end{tabular}
\end{center}
%
Instead of |\childdocof{|\textit{main}|}| just include the main file
at the top of each child file:
%
\begin{center}
|\input{|\textit{main}|}|
\end{center}
%
A simple redirection |\childdocforward{|\textit{dest}|}| is achieved by:
%
\begin{center}
|\def\jobname{|\textit{dest}|}\input{\jobname}|
\end{center}
%
The redirection with prefix
|\childdocforwardprefix[|\textit{prefix}|]{|\textit{dest}|}|
is accomplished by:
%
\begin{center}
\begin{tabular}{l}
|{\edef\jobname{\scantokens\expandafter{\jobname\noexpand}}|\\
|\def\redirectjob |\textit{prefix}|#1~~~{\gdef\jobname{|\textit{dest}|#1}}|\\
|\expandafter\redirectjob\jobname~~~}\input{\jobname}|
\end{tabular}
\end{center}

In an alternative approach,
child documents can be compiled by a specific command line
without additional code or specific definitions:
%
\begin{center}
|... -jobname "|\textit{target}|" "|[\textit{flags}]%
|\includeonly{|\textit{dest}|}\input{|\textit{main}|}"|
\end{center}
%

%%%%%%%%%%%%%%%%%%%%%%%%%%%%%%%%%%%%%%%%%%%%%%%%%%%%%%%%%%%%%%%%%%%%%%%%%%%%%%%%
%%%%%%%%%%%%%%%%%%%%%%%%%%%%%%%%%%%%%%%%%%%%%%%%%%%%%%%%%%%%%%%%%%%%%%%%%%%%%%%%
\section{Information}

%%%%%%%%%%%%%%%%%%%%%%%%%%%%%%%%%%%%%%%%%%%%%%%%%%%%%%%%%%%%%%%%%%%%%%%%%%%%%%%%
\subsection{Copyright}

Copyright \copyright{} 2017--2018 Niklas Beisert

This work may be distributed and/or modified under the
conditions of the \LaTeX{} Project Public License, either version 1.3
of this license or (at your option) any later version.
The latest version of this license is in
  \url{http://www.latex-project.org/lppl.txt}
and version 1.3 or later is part of all distributions of \LaTeX{}
version 2005/12/01 or later.

This work has the LPPL maintenance status `maintained'.

The Current Maintainer of this work is Niklas Beisert.

This work consists of the files |README.txt|, |childdoc.ins| and |childdoc.dtx|
as well as the derived files |childdoc.def|, |cdocsamp.tex|
with |cdocsch1.tex|, |cdocsch2.tex|, |cdocspt3.tex|, |cdocspt4.tex|,
|cdocsdrf.tex|, |cdocsfn1.tex|, |cdocsfn2.tex|
as well as |childdoc.pdf|.

%%%%%%%%%%%%%%%%%%%%%%%%%%%%%%%%%%%%%%%%%%%%%%%%%%%%%%%%%%%%%%%%%%%%%%%%%%%%%%%%
\subsection{Files and Installation}

The package consists of the files:
%
\begin{center}
\begin{tabular}{ll}
    |README.txt|   & readme file \\
    |childdoc.ins| & installation file \\
    |childdoc.dtx| & source file \\
    |childdoc.def| & definition file \\
    |cdocsamp.tex| & sample main file \\
    |cdocsch1.tex| & sample include file \\
    |cdocsch2.tex| & sample include file \\
    |cdocspt3.tex| & sample part file \\
    |cdocspt4.tex| & sample part file \\
    |cdocsdrf.tex| & sample redirection file \\
    |cdocsfn1.tex| & sample redirection file \\
    |cdocsfn2.tex| & sample redirection file \\
    |childdoc.pdf| & manual
\end{tabular}
\end{center}
%
The distribution consists of the files
|README.txt|, |childdoc.ins| and |childdoc.dtx|.
%
\begin{itemize}
\item
Run (pdf)\LaTeX{} on |childdoc.dtx|
to compile the manual |childdoc.pdf| (this file).
\item
Run \LaTeX{} on |childdoc.ins| to create the definitions file |childdoc.def|
and the sample |cdocsamp.tex| with include files
|cdocsch1.tex|, |cdocsch2.tex|, |cdocspt3.tex|, |cdocspt4.tex|,
|cdocsdrf.tex|, |cdocsfn1.tex|, |cdocsfn2.tex|.
Then copy the file |childdoc.def| to an appropriate directory of your \LaTeX{}
distribution, e.g.\ \textit{texmf-root}|/tex/latex/childdoc|.
\end{itemize}

%%%%%%%%%%%%%%%%%%%%%%%%%%%%%%%%%%%%%%%%%%%%%%%%%%%%%%%%%%%%%%%%%%%%%%%%%%%%%%%%
\subsection{Related CTAN Packages}

There are several other packages which offer a similar functionality:
%
\begin{itemize}
\item
The packages
\href{http://ctan.org/pkg/docmute}{\textsf{docmute}},
\href{http://ctan.org/pkg/includex}{\textsf{includex}} and
\href{http://ctan.org/pkg/standalone}{\textsf{standalone}}
provide commands to include only the document body of
a child file thus allowing both files to be compiled individually.
\item
The packages \href{http://ctan.org/pkg/subdocs}{\textsf{subdocs}}
and \href{http://ctan.org/pkg/subfiles}{\textsf{subfiles}}
provide structures in which the main and child documents can be
encapsulated and allowing them to be compiled individually.
The inclusion mechanism is different from the conventional |\include|.
\item
The package \href{http://ctan.org/pkg/combine}{\textsf{combine}}
is an elaborate solution to combine several documents into one.
\end{itemize}
%
See also the CTAN topic \href{http://ctan.org/topic/subdocs}{\textsf{subdocs}}
for further related packages.
The present package differs from the above solutions in that
a document structure constructed with the conventional |\include| mechanism
just needs two extra commands at the top of every file
such that all constituent files can be compiled individually.

%%%%%%%%%%%%%%%%%%%%%%%%%%%%%%%%%%%%%%%%%%%%%%%%%%%%%%%%%%%%%%%%%%%%%%%%%%%%%%%%
%\subsection{Feature Suggestions}
%
%The following is a list of features which may be useful for future
%versions of this package:
%%
%\begin{itemize}
%\item
%\ldots
%\end{itemize}

%%%%%%%%%%%%%%%%%%%%%%%%%%%%%%%%%%%%%%%%%%%%%%%%%%%%%%%%%%%%%%%%%%%%%%%%%%%%%%%%
\subsection{Revision History}

%%%%%%%%%%%%%%%%%%%%%%%%%%%%%%%%%%%%%%%%
\paragraph{v2.0:} 2018/12/30

\begin{itemize}
\item
immediate forward processing
\item
added |\childdocby| mechanism
\item
manual restructured
\end{itemize}

%%%%%%%%%%%%%%%%%%%%%%%%%%%%%%%%%%%%%%%%
\paragraph{v1.6:} 2018/01/17

\begin{itemize}
\item
application for development of include files
\item
corrections to manual
\end{itemize}

%%%%%%%%%%%%%%%%%%%%%%%%%%%%%%%%%%%%%%%%
\paragraph{v1.5:} 2017/05/21

\begin{itemize}
\item
more complete structuring introduced
\item
|\childdocof| introduced
\item
|\childdoc| renamed to |\childdocmain|
\item
|\childredirect| renamed to |\childdocforward| and |\childdocforwardprefix|
and functionality expanded
\end{itemize}

%%%%%%%%%%%%%%%%%%%%%%%%%%%%%%%%%%%%%%%%
\paragraph{v1.0:} 2017/04/27

\begin{itemize}
\item
manual and install package
\item
first version published on CTAN
\end{itemize}

%%%%%%%%%%%%%%%%%%%%%%%%%%%%%%%%%%%%%%%%
\paragraph{v0.6:} 2017/04/26

\begin{itemize}
\item
redirection mechanism added
\end{itemize}

%%%%%%%%%%%%%%%%%%%%%%%%%%%%%%%%%%%%%%%%
\paragraph{v0.5:} 2017/04/26

\begin{itemize}
\item
functionality in definition file
\end{itemize}


%%%%%%%%%%%%%%%%%%%%%%%%%%%%%%%%%%%%%%%%%%%%%%%%%%%%%%%%%%%%%%%%%%%%%%%%%%%%%%%%
%%%%%%%%%%%%%%%%%%%%%%%%%%%%%%%%%%%%%%%%%%%%%%%%%%%%%%%%%%%%%%%%%%%%%%%%%%%%%%%%
%%%%%%%%%%%%%%%%%%%%%%%%%%%%%%%%%%%%%%%%%%%%%%%%%%%%%%%%%%%%%%%%%%%%%%%%%%%%%%%%
\appendix

\settowidth\MacroIndent{\rmfamily\scriptsize 000\ }

 \DocInput{childdoc.dtx}

\end{document}
%</driver>
% \fi
%
% %%%%%%%%%%%%%%%%%%%%%%%%%%%%%%%%%%%%%%%%%%%%%%%%%%%%%%%%%%%%%%%%%%%%%%%%%%%%%%
% %%%%%%%%%%%%%%%%%%%%%%%%%%%%%%%%%%%%%%%%%%%%%%%%%%%%%%%%%%%%%%%%%%%%%%%%%%%%%%
% \section{Sample}
%\iffalse
%<*samplemain>
%\fi
%
% The following presents a sample document
% with two chapters, two parts, a title page,
% a compile flag as well as three forwarding files to set the flag.
% It consists of eight |.tex| files:
% \begin{center}
% \begin{tabular}{ll}
% |cdocsamp.tex|&main file\\
% |cdocsch1.tex|&include file for chapter 1\\
% |cdocsch2.tex|&include file for chapter 2\\
% |cdocspt3.tex|&include file for part 3\\
% |cdocspt4.tex|&include file for part 4\\
% |cdocsdrf.tex|&forwarding file for main file in draft mode\\
% |cdocsfi1.tex|&forwarding file for final version of chapter 1\\
% |cdocsfi2.tex|&forwarding file for final version of chapter 2\\
% \end{tabular}
% \end{center}
% Each of the eight files can be compiled directly by the \LaTeX{} compiler.
%
% %%%%%%%%%%%%%%%%%%%%%%%%%%%%%%%%%%%%%%
% \paragraph{Main File.}
%
% The main file is called |cdocsamp.tex|.
%
% Load the \textsf{childdoc} definitions and
% declare the filename for the main document:
%    \begin{macrocode}
% \iffalse
%
% childdoc.dtx Copyright (C) 2017-2018 Niklas Beisert
%
% This work may be distributed and/or modified under the
% conditions of the LaTeX Project Public License, either version 1.3
% of this license or (at your option) any later version.
% The latest version of this license is in
%   http://www.latex-project.org/lppl.txt
% and version 1.3 or later is part of all distributions of LaTeX
% version 2005/12/01 or later.
%
% This work has the LPPL maintenance status `maintained'.
%
% The Current Maintainer of this work is Niklas Beisert.
%
% This work consists of the files childdoc.dtx and childdoc.ins
% and the derived files childdoc.def and cdocsamp.tex with
% cdocsch1.tex, cdocsch2.tex, cdocsdrf.tex, cdocsfn1.tex, cdocsfn2.tex.
%
%<package>\ifdefined\childdocmain\endinput\fi
%<package>\ProvidesFile{childdoc.def}[2018/12/30 v2.0 child document driver]
%<samplemain>\ProvidesFile{cdocsamp.tex}[2018/12/30 v2.0 sample for childdoc]
%<*driver>
%\ProvidesFile{childdoc.drv}[2018/12/30 v2.0 childdoc reference manual file]
\PassOptionsToClass{10pt,a4paper}{article}
\documentclass{ltxdoc}

\usepackage[margin=35mm]{geometry}
\usepackage{hyperref}
\usepackage{hyperxmp}
\usepackage[usenames]{color}

\hypersetup{colorlinks=true}
\hypersetup{pdfstartview=FitH}
\hypersetup{pdfpagemode=UseNone}
\hypersetup{pdfsource={}}
\hypersetup{pdflang={en-UK}}
\hypersetup{pdfcopyright={Copyright 2017-2018 Niklas Beisert.
  This work may be distributed and/or modified under the
  conditions of the LaTeX Project Public License, either version 1.3
  of this license or (at your option) any later version.}}
\hypersetup{pdflicenseurl={http://www.latex-project.org/lppl.txt}}
\hypersetup{pdfcontactaddress={ETH Zurich, ITP, HIT K,
  Wolfgang-Pauli-Strasse 27}}
\hypersetup{pdfcontactpostcode={8093}}
\hypersetup{pdfcontactcity={Zurich}}
\hypersetup{pdfcontactcountry={Switzerland}}
\hypersetup{pdfcontactemail={nbeisert@itp.phys.ethz.ch}}
\hypersetup{pdfcontacturl={http://people.phys.ethz.ch/\xmptilde nbeisert/}}

\newcommand{\secref}[1]{\hyperref[#1]{section \ref*{#1}}}

\parskip1ex
\parindent0pt
\let\olditemize\itemize
\def\itemize{\olditemize\parskip0pt}

\begin{document}

\title{The \textsf{childdoc} Package}
\hypersetup{pdftitle={The childdoc Package}}
\author{Niklas Beisert\\[2ex]
  Institut f\"ur Theoretische Physik\\
  Eidgen\"ossische Technische Hochschule Z\"urich\\
  Wolfgang-Pauli-Strasse 27, 8093 Z\"urich, Switzerland\\[1ex]
  \href{mailto:nbeisert@itp.phys.ethz.ch}
  {\texttt{nbeisert@itp.phys.ethz.ch}}}
\hypersetup{pdfauthor={Niklas Beisert}}
\hypersetup{pdfsubject={Manual for the LaTeX2e Package childdoc}}
\date{30 December 2018, \textsf{v2.0}}
\maketitle

\begin{abstract}\noindent
\textsf{childdoc} is a \LaTeXe{} package
that enables the direct compilation
of document sections included by |\include|
to individual files.
\end{abstract}

\begingroup
\parskip0ex
\tableofcontents
\endgroup

%%%%%%%%%%%%%%%%%%%%%%%%%%%%%%%%%%%%%%%%%%%%%%%%%%%%%%%%%%%%%%%%%%%%%%%%%%%%%%%%
%%%%%%%%%%%%%%%%%%%%%%%%%%%%%%%%%%%%%%%%%%%%%%%%%%%%%%%%%%%%%%%%%%%%%%%%%%%%%%%%
\section{Introduction}

\LaTeX{} provides a mechanism to structure a large document (such as a book)
into a main file and several child files (containing the chapters)
using the |\include| command.
This mechanism is beneficial for documents
which span hundreds of pages in order to
make the source file(s) more manageable.
Moreover, compilation can be restricted to
selected child files by means of the |\includeonly| command.
The latter feature can be used to reduce the compilation time while editing
(this was significantly more useful in the earlier days of \LaTeX{})
or to generate a smaller document which is easier to navigate.
Another application of |\includeonly| is to generate
documents consisting of selected parts of the complete document.

However, there are a few drawbacks of the plain |\include| mechanism:
\begin{itemize}
\item
The child files cannot be compiled on their own,
they can only be compiled via the main file.
A naive editing environment
(such as a text editor with an option
to have the current file processed by \LaTeX)
may require one to switch to the main file before compiling;
attempting to compile the child file produces errors.
\item
The main file must be modified (each time)
to adjust the |\includeonly| command
to the present needs. This easily leaves the main file in a messy state.
\item
The generated document will always carry the filename
of the main document. This is inconvenient if
several child files are to be compiled and
to be kept for distribution.
\end{itemize}

The present package provides a simple interface
to make child files individually compilable by \LaTeX{}.
Compiling a child file then has the same effect as compiling
the main file with an |\includeonly| command
to select the appropriate child.
Moreover the generated document will carry the name of the child
rather than the main file.
This resolves all three above issues.

This feature is meant to make the editing of books,
thesis documents and lecture notes somewhat more convenient.
However, the package can also be used efficiently for
composing a series of documents (such as exercise sheets)
which are typically distributed individually.
It then assists the author in generating the individual documents
(potentially in different versions)
as well as a document containing the collected series.
Another application is in developing style files
or other kinds of included material
where compilation of the style file could redirect
to a sample or test file.

%%%%%%%%%%%%%%%%%%%%%%%%%%%%%%%%%%%%%%%%%%%%%%%%%%%%%%%%%%%%%%%%%%%%%%%%%%%%%%%%
%%%%%%%%%%%%%%%%%%%%%%%%%%%%%%%%%%%%%%%%%%%%%%%%%%%%%%%%%%%%%%%%%%%%%%%%%%%%%%%%
\section{Usage}

First of all, the package \textsf{childdoc} is \emph{not} a standard
\LaTeXe{} |.sty| style file! Therefore it needs to be invoked in
a non-standard way.

%%%%%%%%%%%%%%%%%%%%%%%%%%%%%%%%%%%%%%%%%%%%%%%%%%%%%%%%%%%%%%%%%%%%%%%%%%%%%%%%
\subsection{Included Files}
\label{sec:include}

%%%%%%%%%%%%%%%%%%%%%%%%%%%%%%%%%%%%%%%%
\DescribeMacro{\childdocmain}
To use the package, add the commands
\begin{center}
\begin{tabular}{l}
|\input{childdoc.def}|\\
|\childdocmain{}|\\
\end{tabular}
\end{center}
at the very top of the main \LaTeX{} file,
in particular \emph{before} the |\documentclass| statement!
The argument of |\childdocmain| should be left empty
(but it must be present).

%%%%%%%%%%%%%%%%%%%%%%%%%%%%%%%%%%%%%%%%
\DescribeMacro{\childdocof}
Furthermore, add the commands
\begin{center}
\begin{tabular}{l}
|\input{childdoc.def}|\\
|\childdocof{|\textit{main}|}|\\
\end{tabular}
\end{center}
at the top of every child file \textit{child}
which is included by |\include{|\textit{child}|}|
from within the main file
(or at least for those files to be compiled individually).
The argument \textit{main} must be the filename of the main file.

There are a couple of
considerations in setting up the main and child documents:

%%%%%%%%%%%%%%%%%%%%%%%%%%%%%%%%%%%%%%%%
\paragraph{Restrictions.}

Please note the following restrictions:
\begin{itemize}
\item
|\childdocmain| must be called with one argument \textit{main}
to ensure compatibility with earlier version of the package.
It must either be empty (|\childdocmain{}|)
or precisely match the filename of the main file in which it is specified.
See \secref{sec:detection} for further information.
\item
The filename \textit{main} must be specified without the |.tex| extension.
\item
The filename \textit{main} is case sensitive
(even in case-insensitive file systems)
due to internal string comparison.
\item
The argument \textit{main} should be fully expanded, it cannot be a macro.
\item
Subdirectories and special characters should be avoided in filenames.
\item
The command |\childdocmain{|\textit{main}|}| must be followed by a whitespace.
It should not be followed immediately by another command
or by a comment mark `|%|'.
This is because the \TeX{} parser reads the token immediately following
the argument of |\childdocmain| and puts it
at the beginning of every child section;
however, a white\-space is ignored.
\end{itemize}

%%%%%%%%%%%%%%%%%%%%%%%%%%%%%%%%%%%%%%%%
\paragraph{Content of Main File.}

It is advisable to place all content in the child files included by |\include|.
Any output contained in the main file will appear in all child documents
unless suppressed manually;
it cannot be suppressed automatically by the |\includeonly| directive
and thus should normally be avoided.
A method to include some content in the main file
by means of conditional processing is described in \secref{sec:conditional}.

%%%%%%%%%%%%%%%%%%%%%%%%%%%%%%%%%%%%%%%%
\paragraph{Page Numbering.}

When only a part of the document is compiled,
the appropriate numbering of pages
(as well as other status parameters)
is determined from the |.aux| files.
The latter contain information from previous passes.
However this information needs to propagate through
all intermediate child documents.
Therefore the page numbering in child documents may well
be inconsistent until the complete document is compiled at least once.

A useful (if unconventional) way to always ensure a consistent
page numbering is to restart the numbering in each child document
and denote the pages by `\textit{child}|.|\textit{page}'
where \textit{child} represents the chapter/section number of the child file.
This can be achieved by the command
|\numberwithin{page}{|\textit{child}|}|
of the \textsf{amsmath} package
where \textit{child} can be |chapter| or |section|
depending on the chosen structuring.
Alternatively, one can modify the macro |\thepage| appropriately
and reset the counter |page| at the start of each child file.

%%%%%%%%%%%%%%%%%%%%%%%%%%%%%%%%%%%%%%%%%%%%%%%%%%%%%%%%%%%%%%%%%%%%%%%%%%%%%%%%
\subsection{Conditional Processing}
\label{sec:conditional}

The package provides a mechanism to compile different versions
of a document. To customise the versions further some conditional processing
can come in handy to distinguish which version is being compiled.
The package provides two macros to describe the compilation context:

%%%%%%%%%%%%%%%%%%%%%%%%%%%%%%%%%%%%%%%%
\DescribeMacro{\ifchilddoc}
The conditional |\ifchilddoc| distinguishes between the compilation of
child documents and the main document:
%
\begin{center}
|\ifchilddoc |\textit{child-code}| |[|\||else |\textit{main-code}]| \||fi|
\end{center}

%%%%%%%%%%%%%%%%%%%%%%%%%%%%%%%%%%%%%%%%
\DescribeMacro{\childdocname}
\DescribeMacro{\childdocjob}
The macro |\childdocname| contains the filename (without extension)
of the main or child file being processed.
Note that |\childdocjob| will always contain the name of the main file.

%%%%%%%%%%%%%%%%%%%%%%%%%%%%%%%%%%%%%%%%
\paragraph{Title Page.}

Conditional processing can be used to include a title or banner page
in the main document when proper precautions are taken.
Importantly, the code in the main file should ensure that the page counter
(as well as other status parameters which are stored in the |.aux| files)
takes the same value after the conditional processing.
Otherwise the page numbers may take divergent values
depending on which part is compiled.

For example, a title page could be declared by:
%
\begin{center}
\begin{tabular}{l}
|\ifchilddoc\||else|\\
|\addtocounter{page}{-1}|\\
\textit{code for title page}\\
|\newpage|\\
|\||fi|
\end{tabular}
\end{center}
%
A banner page for the child documents can be generated by:
%
\begin{center}
\begin{tabular}{l}
|\ifchilddoc|\\
|\addtocounter{page}{-1}|\\
\textit{code for banner page}\\
|\newpage|\\
|\||fi|
\end{tabular}
\end{center}
%
Here one could write a message such as:
\begin{center}
|This is the part \childdocname{} of \childdocjob{}.|
\end{center}

%%%%%%%%%%%%%%%%%%%%%%%%%%%%%%%%%%%%%%%%%%%%%%%%%%%%%%%%%%%%%%%%%%%%%%%%%%%%%%%%
\subsection{Flags}
\label{sec:flags}

The package makes it easy to generate different versions
of the main or child documents.
To this end compilation flags can be defined
and assigned different default values.
They will be particularly useful in conjunction
with the forwarding mechanism described in \secref{sec:forward}.

For example, it may be useful to have a flag |\version|
which can be set to |draft| or |final|.
The document source will contain some conditional code
depending on the value of |\version|.
Suppose further, the flag should default to |final| for the main file
and to |draft| for child files
which is a natural assignment for editing the document.
This is achieved by placing the following code
in the preamble of the main document
(below the |\childdocmain| directive):
%
\begin{center}
\begin{tabular}{l}
|\ifchilddoc|\\
|\providecommand{\version}{draft}|\\
|\||else|\\
|\providecommand{\version}{final}|\\
|\||fi|
\end{tabular}
\end{center}
%
The definition by |\providecommand| makes sure
that previous definitions are not overwritten.
Further statements |\providecommand{\version}{...}|
can thus be added before the above code to override it.

For the main file, one might add a line
(between |\childdocmain| and the above block)
%
\begin{center}
|%\ifchilddoc\||else\providecommand{\version}{draft}\||fi|
\end{center}
%
which can be uncommented to produce a draft version.
Likewise one can add a line to the very top of a child file
(above the |\childdocof{|\textit{main}|}| directive)
%
\begin{center}
|%\providecommand{\version}{final}|
\end{center}
%
which can be uncommented to produce the final version of this child document.

%%%%%%%%%%%%%%%%%%%%%%%%%%%%%%%%%%%%%%%%%%%%%%%%%%%%%%%%%%%%%%%%%%%%%%%%%%%%%%%%
\subsection{Forwarding}
\label{sec:forward}

Different versions of the main or child documents
using compilation flags as described in \secref{sec:flags}
can be (permanently) stored in different files
for convenient compilation, viewing and distribution.
To this end, the package defines a command
to pass on compilation to a different file:

%%%%%%%%%%%%%%%%%%%%%%%%%%%%%%%%%%%%%%%%
\DescribeMacro{\childdocforward}
The command |\childdocforward| redirects processing to
another source file:
%
\begin{center}
\begin{tabular}{l}
|\input{childdoc.def}|\\
|\childdocforward[|\textit{main}|]{|\textit{dest}|}|\\
\end{tabular}
\end{center}
%
The argument \textit{dest} is the destination file
(without extension).
It should be the main file or one of the child files.
Note that further \textsf{childdoc} directives
such as |\childdocof| and |\childdocforward|
in the indicated file will be processed in this form.
The optional argument \textit{main}
passes on directly to the main file \textit{main}
while pretending to compile the child \textit{dest}.
This form behaves as if \textit{dest}
issues |\childdocof{|\textit{main}|}| right away,
and no further \textsf{childdoc} directives will be processed.

%%%%%%%%%%%%%%%%%%%%%%%%%%%%%%%%%%%%%%%%
\DescribeMacro{\...prefix}
In the alternative form |\childdocforwardprefix|,
%
\begin{center}
\begin{tabular}{l}
|\input{childdoc.def}|\\
|\childdocforwardprefix[|\textit{main}|]{|\textit{prefix}|}{|\textit{dest}|}|
\end{tabular}
\end{center}
%
the destination file is determined by a pattern
depending on the current file:
To make this work, the current file must be called
`{\textit{prefix}\hspace{0.2em}\textit{suffix}}'
with \textit{prefix} matching precisely the argument.
Processing is then passed on to the file
`{\textit{dest}\hspace{0.2em}\textit{suffix}}'.
Surely, the same effect is achieved by
directly specifying the
argument `{\textit{dest}\hspace{0.2em}\textit{suffix}}'
in the first form.
However, that requires to set up a different file
for each child. With the alternative form of the command
all these files can have exactly the same content
which simplifies setting them up and maintaining them.

For example, the following file |draft.tex|
with a compilation flag |\version| as described in \secref{sec:flags}
compiles the main document as a draft:
%
\begin{center}
\begin{tabular}{l}
|\def\version{draft}|\\
|\input{childdoc.def}|\\
|\childdocforward{|\textit{main}|}|
\end{tabular}
\end{center}
%
Likewise, the following files |final|\textit{nn}|.tex|
compile the final version of the child document
|child|\textit{nn}|.tex|:
%
\begin{center}
\begin{tabular}{l}
|\def\version{final}|\\
|\input{childdoc.def}|\\
|\childdocforwardprefix{final}{child}|
\end{tabular}
\end{center}
%

Note that when several versions of a main file and/or of each child file
are to be generated, it may be convenient to set up a |Makefile| or
shell script to automatise the process.

%%%%%%%%%%%%%%%%%%%%%%%%%%%%%%%%%%%%%%%%%%%%%%%%%%%%%%%%%%%%%%%%%%%%%%%%%%%%%%%%
\subsection{Command Line Processing}
\label{sec:commandline}

The effect of redirection files can also be achieved by invoking
the \LaTeX{} compiler with a more elaborate command line.
Most conveniently this should be done as part
of a shell script or a |Makefile|.

When using \textsf{childdoc} in the main file, the following
command lines effectively perform a redirection
(note that depending on the shell being used,
backslashes may have to be doubled: `|\|' $\to$ `|\\|'):
%
\begin{center}
|... -jobname "|\textit{target}|" |\\|"|[\textit{flags}]%
|\input{childdoc.def}\childdocforward[|\textit{main}|]{|\textit{dest}|}"|
\end{center}
%
Here \textit{target} is the name of the output file,
\textit{main} is the name of the main file
and \textit{dest} is the name of the main or child file to be processed
(all filenames without extensions).
The optional argument \textit{main} can be omitted
if \textit{main} matches \textit{dest}.
Optionally, compilation \textit{flags} can be defined via |\def| commands.
This command line makes the \TeX{} engine believe
it is compiling the file \textit{target}
whose content is specified as the latter parameter.
The provided code then forwards the processing to
\textit{main} or \textit{dest} as described in \secref{sec:forward}.

%%%%%%%%%%%%%%%%%%%%%%%%%%%%%%%%%%%%%%%%%%%%%%%%%%%%%%%%%%%%%%%%%%%%%%%%%%%%%%%%
\subsection{Include by Input}
\label{sec:input}

Including child documents by |\include| has some restrictions by design.
Most notably, the content of a child document always occupies
its own set of pages; pages cannot be shared between child documents.
Usually, this behaviour makes perfect sense
because each child document contain an essential part of the document.
However, in some situations it may be desirable to compose
a document from a collection of parts
without having mandatory page breaks between then.
For this case, the package
provides a mechanism to include parts
by |\input| which can also be processed individually.
However, by construction this mechanism
requires manual handling of the content to be output.

%%%%%%%%%%%%%%%%%%%%%%%%%%%%%%%%%%%%%%%%
\DescribeMacro{\ifchilddocmanual}
The main file should be prepared as usual, see \secref{sec:include}.
However, the document body must make a distinction
between processing of an individual part and of the main document, e.g.:
%
\begin{center}
\begin{tabular}{l}
|\ifchilddocmanual|\\
|\input{\childdocname}|\\
|\||else|\\
\textit{document body with }|\input{|\textit{part}|}|\\
|\||fi|
\end{tabular}
\end{center}
%
The conditional |\ifchilddocmanual| is true whenever
a part to be included by |\input| is being compiled,
and the name of the part is stored in |\childdocname|.

%%%%%%%%%%%%%%%%%%%%%%%%%%%%%%%%%%%%%%%%
\DescribeMacro{\childdocby}
Each part to be included by |\input| should start with:
%
\begin{center}
\begin{tabular}{l}
|\input{childdoc.def}|\\
|\childdocby{|\textit{main}|}|\\
\end{tabular}
\end{center}
%
The directive |\childdocby| is similar to |\childdocof|
described in \secref{sec:include},
but the subsequent selection of content must be done manually.
To that end, both |\ifchilddoc| and |\ifchilddocmanual|
will be true upon processing of a part,
and the name of the part is stored in |\childdocname|.
Note that |\jobname| will be set to the filename of the current part
so that each part receives an individual |.aux| file
that does not interfere with the |.aux| file(s) of the main document.
This behaviour can be altered by the alternative form
|\childdocby[*]{|\textit{main}|}| (with a non-empty optional argument)
which uses the |.aux| file of the main document
by setting |\jobname| to \textit{main}.

%%%%%%%%%%%%%%%%%%%%%%%%%%%%%%%%%%%%%%%%%%%%%%%%%%%%%%%%%%%%%%%%%%%%%%%%%%%%%%%%
\subsection{Driver Development}
\label{sec:driver}

The \textsf{childdoc} mechanism can also be use for the development
of definition files such as \LaTeX{} styles or classes.
This case differs from the above setup with multiple parts
included by |\include| in that no |\includeonly| should be invoked.
This can be achieved by starting the include file
(before |\ProvidesPackage|) with:
%
\begin{center}
\begin{tabular}{l}
|\input{childdoc.def}|\\
|\childdocforward{|\textit{main}|}|\\
\end{tabular}
\end{center}
%
or alternatively with:
%
\begin{center}
\begin{tabular}{l}
|\input{childdoc.def}|\\
|\childdocby{|\textit{main}|}|\\
\end{tabular}
\end{center}
%
Both forms have slightly different effects as described above.
The main file is prepared as usual, see \secref{sec:include}.

%%%%%%%%%%%%%%%%%%%%%%%%%%%%%%%%%%%%%%%%%%%%%%%%%%%%%%%%%%%%%%%%%%%%%%%%%%%%%%%%
\subsection{Legacy Detection}
\label{sec:detection}

The directive |\childdocmain| in the main file can detect
whether the complete document or merely a child is to be compiled
even without using the directive |\childdocof|.
This method is deprecated because it is less robust
and there is no compelling reason to use it;
it is merely provided for backward compatibility
and it may be removed in future versions.

If the detection mechanism is to be used,
it is mandatory to correctly specify
the filename of the main file as the argument of |\childdocmain|:
%
\begin{center}
\begin{tabular}{l}
|\input{childdoc.def}|\\
|\childdocmain{|\textit{main}|}|\\
\end{tabular}
\end{center}
%
If |\jobname| does not match the argument \textit{main} of |\childdocmain|,
it is assumed that |\jobname| points to the child file to be compiled.
When using |\childdocmain| with the main file specified as argument,
it suffices to start a child file
with just |\input{|\textit{main}|}|
without loading of the package and using |\childdocof|.
If instead all processing is done
with the appropriate \textsf{childdoc} directives,
the argument of \textit{main} of |\childdocmain| can be empty.

An alternative version of the command line processing described
in \secref{sec:commandline} using the detection mechanism reads:
%
\begin{center}
|... -jobname "|\textit{target}|" "|[\textit{flags}]%
[|\def\jobname{|\textit{dest}|}|]|\input{|\textit{main}|}"|
\end{center}

%%%%%%%%%%%%%%%%%%%%%%%%%%%%%%%%%%%%%%%%%%%%%%%%%%%%%%%%%%%%%%%%%%%%%%%%%%%%%%%%
\subsection{Manual Code}
\label{sec:manual}

In case one cannot be certain whether the definitions file |childdoc.def|
is installed on the target \TeX{} distribution
and one prefers not to ship it,
it is conceivable to paste a few relevant commands into the sources.

To that end, drop all statements |\input{childdoc.def}|
and perform the replacements as outlined below.
Instead of |\childdocmain{|\textit{main}|}| add the following code
to the top of the main file:
%
\begin{center}
\begin{tabular}{l}
|\||ifdefined\childdocname\endinput\||fi\newif\ifchilddoc|\\
|\edef\childdocname{\scantokens\expandafter{\jobname\noexpand}}|\\
|\def\childdocmain{|\textit{main}|}\||ifx\childdocmain\childdocname\||else|\\
|\childdoctrue\includeonly{\childdocname}\let\jobname\childdocmain\||fi|\\
\end{tabular}
\end{center}
%
Instead of |\childdocof{|\textit{main}|}| just include the main file
at the top of each child file:
%
\begin{center}
|\input{|\textit{main}|}|
\end{center}
%
A simple redirection |\childdocforward{|\textit{dest}|}| is achieved by:
%
\begin{center}
|\def\jobname{|\textit{dest}|}\input{\jobname}|
\end{center}
%
The redirection with prefix
|\childdocforwardprefix[|\textit{prefix}|]{|\textit{dest}|}|
is accomplished by:
%
\begin{center}
\begin{tabular}{l}
|{\edef\jobname{\scantokens\expandafter{\jobname\noexpand}}|\\
|\def\redirectjob |\textit{prefix}|#1~~~{\gdef\jobname{|\textit{dest}|#1}}|\\
|\expandafter\redirectjob\jobname~~~}\input{\jobname}|
\end{tabular}
\end{center}

In an alternative approach,
child documents can be compiled by a specific command line
without additional code or specific definitions:
%
\begin{center}
|... -jobname "|\textit{target}|" "|[\textit{flags}]%
|\includeonly{|\textit{dest}|}\input{|\textit{main}|}"|
\end{center}
%

%%%%%%%%%%%%%%%%%%%%%%%%%%%%%%%%%%%%%%%%%%%%%%%%%%%%%%%%%%%%%%%%%%%%%%%%%%%%%%%%
%%%%%%%%%%%%%%%%%%%%%%%%%%%%%%%%%%%%%%%%%%%%%%%%%%%%%%%%%%%%%%%%%%%%%%%%%%%%%%%%
\section{Information}

%%%%%%%%%%%%%%%%%%%%%%%%%%%%%%%%%%%%%%%%%%%%%%%%%%%%%%%%%%%%%%%%%%%%%%%%%%%%%%%%
\subsection{Copyright}

Copyright \copyright{} 2017--2018 Niklas Beisert

This work may be distributed and/or modified under the
conditions of the \LaTeX{} Project Public License, either version 1.3
of this license or (at your option) any later version.
The latest version of this license is in
  \url{http://www.latex-project.org/lppl.txt}
and version 1.3 or later is part of all distributions of \LaTeX{}
version 2005/12/01 or later.

This work has the LPPL maintenance status `maintained'.

The Current Maintainer of this work is Niklas Beisert.

This work consists of the files |README.txt|, |childdoc.ins| and |childdoc.dtx|
as well as the derived files |childdoc.def|, |cdocsamp.tex|
with |cdocsch1.tex|, |cdocsch2.tex|, |cdocspt3.tex|, |cdocspt4.tex|,
|cdocsdrf.tex|, |cdocsfn1.tex|, |cdocsfn2.tex|
as well as |childdoc.pdf|.

%%%%%%%%%%%%%%%%%%%%%%%%%%%%%%%%%%%%%%%%%%%%%%%%%%%%%%%%%%%%%%%%%%%%%%%%%%%%%%%%
\subsection{Files and Installation}

The package consists of the files:
%
\begin{center}
\begin{tabular}{ll}
    |README.txt|   & readme file \\
    |childdoc.ins| & installation file \\
    |childdoc.dtx| & source file \\
    |childdoc.def| & definition file \\
    |cdocsamp.tex| & sample main file \\
    |cdocsch1.tex| & sample include file \\
    |cdocsch2.tex| & sample include file \\
    |cdocspt3.tex| & sample part file \\
    |cdocspt4.tex| & sample part file \\
    |cdocsdrf.tex| & sample redirection file \\
    |cdocsfn1.tex| & sample redirection file \\
    |cdocsfn2.tex| & sample redirection file \\
    |childdoc.pdf| & manual
\end{tabular}
\end{center}
%
The distribution consists of the files
|README.txt|, |childdoc.ins| and |childdoc.dtx|.
%
\begin{itemize}
\item
Run (pdf)\LaTeX{} on |childdoc.dtx|
to compile the manual |childdoc.pdf| (this file).
\item
Run \LaTeX{} on |childdoc.ins| to create the definitions file |childdoc.def|
and the sample |cdocsamp.tex| with include files
|cdocsch1.tex|, |cdocsch2.tex|, |cdocspt3.tex|, |cdocspt4.tex|,
|cdocsdrf.tex|, |cdocsfn1.tex|, |cdocsfn2.tex|.
Then copy the file |childdoc.def| to an appropriate directory of your \LaTeX{}
distribution, e.g.\ \textit{texmf-root}|/tex/latex/childdoc|.
\end{itemize}

%%%%%%%%%%%%%%%%%%%%%%%%%%%%%%%%%%%%%%%%%%%%%%%%%%%%%%%%%%%%%%%%%%%%%%%%%%%%%%%%
\subsection{Related CTAN Packages}

There are several other packages which offer a similar functionality:
%
\begin{itemize}
\item
The packages
\href{http://ctan.org/pkg/docmute}{\textsf{docmute}},
\href{http://ctan.org/pkg/includex}{\textsf{includex}} and
\href{http://ctan.org/pkg/standalone}{\textsf{standalone}}
provide commands to include only the document body of
a child file thus allowing both files to be compiled individually.
\item
The packages \href{http://ctan.org/pkg/subdocs}{\textsf{subdocs}}
and \href{http://ctan.org/pkg/subfiles}{\textsf{subfiles}}
provide structures in which the main and child documents can be
encapsulated and allowing them to be compiled individually.
The inclusion mechanism is different from the conventional |\include|.
\item
The package \href{http://ctan.org/pkg/combine}{\textsf{combine}}
is an elaborate solution to combine several documents into one.
\end{itemize}
%
See also the CTAN topic \href{http://ctan.org/topic/subdocs}{\textsf{subdocs}}
for further related packages.
The present package differs from the above solutions in that
a document structure constructed with the conventional |\include| mechanism
just needs two extra commands at the top of every file
such that all constituent files can be compiled individually.

%%%%%%%%%%%%%%%%%%%%%%%%%%%%%%%%%%%%%%%%%%%%%%%%%%%%%%%%%%%%%%%%%%%%%%%%%%%%%%%%
%\subsection{Feature Suggestions}
%
%The following is a list of features which may be useful for future
%versions of this package:
%%
%\begin{itemize}
%\item
%\ldots
%\end{itemize}

%%%%%%%%%%%%%%%%%%%%%%%%%%%%%%%%%%%%%%%%%%%%%%%%%%%%%%%%%%%%%%%%%%%%%%%%%%%%%%%%
\subsection{Revision History}

%%%%%%%%%%%%%%%%%%%%%%%%%%%%%%%%%%%%%%%%
\paragraph{v2.0:} 2018/12/30

\begin{itemize}
\item
immediate forward processing
\item
added |\childdocby| mechanism
\item
manual restructured
\end{itemize}

%%%%%%%%%%%%%%%%%%%%%%%%%%%%%%%%%%%%%%%%
\paragraph{v1.6:} 2018/01/17

\begin{itemize}
\item
application for development of include files
\item
corrections to manual
\end{itemize}

%%%%%%%%%%%%%%%%%%%%%%%%%%%%%%%%%%%%%%%%
\paragraph{v1.5:} 2017/05/21

\begin{itemize}
\item
more complete structuring introduced
\item
|\childdocof| introduced
\item
|\childdoc| renamed to |\childdocmain|
\item
|\childredirect| renamed to |\childdocforward| and |\childdocforwardprefix|
and functionality expanded
\end{itemize}

%%%%%%%%%%%%%%%%%%%%%%%%%%%%%%%%%%%%%%%%
\paragraph{v1.0:} 2017/04/27

\begin{itemize}
\item
manual and install package
\item
first version published on CTAN
\end{itemize}

%%%%%%%%%%%%%%%%%%%%%%%%%%%%%%%%%%%%%%%%
\paragraph{v0.6:} 2017/04/26

\begin{itemize}
\item
redirection mechanism added
\end{itemize}

%%%%%%%%%%%%%%%%%%%%%%%%%%%%%%%%%%%%%%%%
\paragraph{v0.5:} 2017/04/26

\begin{itemize}
\item
functionality in definition file
\end{itemize}


%%%%%%%%%%%%%%%%%%%%%%%%%%%%%%%%%%%%%%%%%%%%%%%%%%%%%%%%%%%%%%%%%%%%%%%%%%%%%%%%
%%%%%%%%%%%%%%%%%%%%%%%%%%%%%%%%%%%%%%%%%%%%%%%%%%%%%%%%%%%%%%%%%%%%%%%%%%%%%%%%
%%%%%%%%%%%%%%%%%%%%%%%%%%%%%%%%%%%%%%%%%%%%%%%%%%%%%%%%%%%%%%%%%%%%%%%%%%%%%%%%
\appendix

\settowidth\MacroIndent{\rmfamily\scriptsize 000\ }

 \DocInput{childdoc.dtx}

\end{document}
%</driver>
% \fi
%
% %%%%%%%%%%%%%%%%%%%%%%%%%%%%%%%%%%%%%%%%%%%%%%%%%%%%%%%%%%%%%%%%%%%%%%%%%%%%%%
% %%%%%%%%%%%%%%%%%%%%%%%%%%%%%%%%%%%%%%%%%%%%%%%%%%%%%%%%%%%%%%%%%%%%%%%%%%%%%%
% \section{Sample}
%\iffalse
%<*samplemain>
%\fi
%
% The following presents a sample document
% with two chapters, two parts, a title page,
% a compile flag as well as three forwarding files to set the flag.
% It consists of eight |.tex| files:
% \begin{center}
% \begin{tabular}{ll}
% |cdocsamp.tex|&main file\\
% |cdocsch1.tex|&include file for chapter 1\\
% |cdocsch2.tex|&include file for chapter 2\\
% |cdocspt3.tex|&include file for part 3\\
% |cdocspt4.tex|&include file for part 4\\
% |cdocsdrf.tex|&forwarding file for main file in draft mode\\
% |cdocsfi1.tex|&forwarding file for final version of chapter 1\\
% |cdocsfi2.tex|&forwarding file for final version of chapter 2\\
% \end{tabular}
% \end{center}
% Each of the eight files can be compiled directly by the \LaTeX{} compiler.
%
% %%%%%%%%%%%%%%%%%%%%%%%%%%%%%%%%%%%%%%
% \paragraph{Main File.}
%
% The main file is called |cdocsamp.tex|.
%
% Load the \textsf{childdoc} definitions and
% declare the filename for the main document:
%    \begin{macrocode}
\input{childdoc.def}
\childdocmain{}
%    \end{macrocode}

% Optional override for |\version| flag:
%    \begin{macrocode}
%%\ifchilddoc\else\providecommand{\version}{draft}\fi
%    \end{macrocode}

% Define the default values for the |\version| flag
% (|final| for the main file and |draft| for childs):
%    \begin{macrocode}
\ifchilddoc
\providecommand{\version}{draft}
\else
\providecommand{\version}{final}
\fi
%    \end{macrocode}

% Load the standard document class:
%    \begin{macrocode}
\documentclass[12pt]{article}
%    \end{macrocode}

% Start the document body:
%    \begin{macrocode}
\begin{document}
%    \end{macrocode}

% Declare a title page.
% Print title, part of document being processed and version flag:
%    \begin{macrocode}
\addtocounter{page}{-1}
\begin{center}
{\LARGE\bfseries{}childdoc example\par}
\vspace{1cm}
\ifchilddoc
\ifchilddocmanual part\else chapter\fi:
`\childdocname' of `\childdocjob'\par
\else
main document: `\childdocjob'\par
\fi
version: \version\par
\end{center}
\newpage
%    \end{macrocode}

% Manually include selected file,
% otherwise process as usual:
%    \begin{macrocode}
\ifchilddocmanual
\section*{part `\childdocname'}
\input{\childdocname}
\else
%    \end{macrocode}

% Include the two chapters:
%    \begin{macrocode}
\include{cdocsch1}
\include{cdocsch2}
%    \end{macrocode}

% Include the two parts unless only chapters should be displayed:
%    \begin{macrocode}
\ifchilddoc\else
\section{part three}
\input{cdocspt3}
\section{part four}
\input{cdocspt4}
\fi
%    \end{macrocode}

% Process as usual until here:
%    \begin{macrocode}
\fi
%    \end{macrocode}

% End of document body:
%    \begin{macrocode}
\end{document}
%    \end{macrocode}
%\iffalse
%</samplemain>
%\fi
%
% %%%%%%%%%%%%%%%%%%%%%%%%%%%%%%%%%%%%%%
% \paragraph{Chapter Include Files.}
%
% The include files are called |cdocsch1.tex| and |cdocsch2.tex|.
%
%\iffalse
%<*samplechap1|samplechap2>
%\fi

% Optional override for |\version| flag:
%    \begin{macrocode}
%%\providecommand{\version}{final}
%    \end{macrocode}

% Include the main document:
%    \begin{macrocode}
\input{childdoc.def}
\childdocof{cdocsamp}
%    \end{macrocode}

%\iffalse
%</samplechap1|samplechap2>
%\fi
%
%\iffalse
%<*samplechap1>
%\fi
% Some text for chapter 1:
%    \begin{macrocode}
\section{one}
some text in chapter one
%    \end{macrocode}

%\iffalse
%</samplechap1>
%\fi
% Some text for chapter 2:
%\iffalse
%<*samplechap2>
%\fi
%    \begin{macrocode}
\section{two}
more text in chapter two
%    \end{macrocode}

%\iffalse
%</samplechap2>
%\fi
%
% %%%%%%%%%%%%%%%%%%%%%%%%%%%%%%%%%%%%%%
% \paragraph{Part Include Files.}
%
% The include files are called |cdocspt3.tex| and |cdocspt4.tex|.
%
%\iffalse
%<*samplepart3|samplepart4>
%\fi

% Optional override for |\version| flag:
%    \begin{macrocode}
%%\providecommand{\version}{final}
%    \end{macrocode}

% Include the main document:
%    \begin{macrocode}
\input{childdoc.def}
\childdocby{cdocsamp}
%    \end{macrocode}

%\iffalse
%</samplepart3|samplepart4>
%\fi
%
%\iffalse
%<*samplepart3>
%\fi
% Some text for part 3:
%    \begin{macrocode}
some text in part three
%    \end{macrocode}

%\iffalse
%</samplepart3>
%\fi
% Some text for part 4:
%\iffalse
%<*samplepart4>
%\fi
%    \begin{macrocode}
more text in part four
%    \end{macrocode}

%\iffalse
%</samplepart4>
%\fi
%
% %%%%%%%%%%%%%%%%%%%%%%%%%%%%%%%%%%%%%%
% \paragraph{Forwarding for a Complete Draft.}
%
% The following forwarding file |cdocsdrf.tex|
% compiles the main document in draft mode:
%\iffalse
%<*sampledraft>
%\fi
%    \begin{macrocode}
\def\version{draft}
\input{childdoc.def}
\childdocforward{cdocsamp}
%    \end{macrocode}

%\iffalse
%</sampledraft>
%\fi
%
% %%%%%%%%%%%%%%%%%%%%%%%%%%%%%%%%%%%%%%
% \paragraph{Forwarding for Final Version of the Chapters.}
%
% The following forwarding files |cdocsfn1.tex| and |cdocsfn2.tex|
% (with identical content)
% compile the final versions of the child documents
% |cdocsch1.tex| and |cdocsch2.tex|, respectively:
%\iffalse
%<*samplefinal>
%\fi
%    \begin{macrocode}
\def\version{final}
\input{childdoc.def}
\childdocforwardprefix[cdocsamp]{cdocsfn}{cdocsch}
%    \end{macrocode}

%\iffalse
%</samplefinal>
%\fi
%
% %%%%%%%%%%%%%%%%%%%%%%%%%%%%%%%%%%%%%%
% \paragraph{Command Line Processing.}
%
% The following three command lines generate the output files
% |cdocscld|, |cdocscl1| and |cdocscl2|
% which should be identical to
% |cdocsdrf|, |cdocsch1| and |cdocsfn2|, respectively:
% \begin{center}
% \begin{tabular}{l}
% |latex -jobname cdocscld \|\\
% |  "\def\version{draft}\input{childdoc.def}\childdocforward{cdocsamp}"|\\
% |latex -jobname cdocscl1 \|\\
% |  "\input{childdoc.def}\childdocforward[cdocsamp]{cdocsch1}"|\\
% |latex -jobname cdocscl2 \|\\
% |  "\def\version{final}\input{childdoc.def}\childdocforward{cdocsch2}"|
% \end{tabular}
% \end{center}
% Note that the trailing backslash on each first line
% merely continues the input to the second line
% (for convenient cut ant paste).
% Furthermore, the command |latex| can be replaced by any
% of its alternative versions such as |pdflatex|.
%
% %%%%%%%%%%%%%%%%%%%%%%%%%%%%%%%%%%%%%%%%%%%%%%%%%%%%%%%%%%%%%%%%%%%%%%%%%%%%%%
% %%%%%%%%%%%%%%%%%%%%%%%%%%%%%%%%%%%%%%%%%%%%%%%%%%%%%%%%%%%%%%%%%%%%%%%%%%%%%%
% \section{Implementation}
%\iffalse
%<*package>
%\fi
%
% This section describes the definitions file |childdoc.def|.

% The definitions cannot be loaded using |\usepackage| or |\RequirePackage|
% which has a mechanism to prevent loading a style file more than once.
% When loading the definitions by means of |\input|
% multiple instances have to be prevented manually:
%\iffalse
%This code needs to be before the `\ProvidesFile' directive
%which is defined at the beginning of this file.
%Therefore it is also placed there and commented out here.
%</package>
%<*discard>
%\fi
%    \begin{macrocode}
\ifdefined\childdocmain\endinput\fi
%    \end{macrocode}
%\iffalse
%</discard>
%<*package>
%\fi
%
% \macro{\ifchilddoc}
% \macro{\ifchilddocmanual}
% The conditional |\ifchilddoc| tells whether a
% child (true) or main (false) document is being compiled.
% The conditional |\ifchilddocmanual| tells whether
% the |\includeonly| mechanism is used (false) or
% the selection of child files must be performed manually (true).
% The definitions initialise to false:
%    \begin{macrocode}
\newif\ifchilddoc
\newif\ifchilddocmanual
%    \end{macrocode}

% \macro{\childdocname}
% \macro{\childdocjob}
% The macro |\childdocname| stores the name of the main document
% to be compiled. The macro |\childdocjob| stores the name of
% the document on which the \LaTeX{} compiler was originally invoked.
% The content of |\jobname| cannot be compared
% to filenames specified in the source due to different catcodes.
% The following code rescans |\jobname|, stores the result
% in |\childdocname| and saves a copy in |\childdocjob|:
%    \begin{macrocode}
\edef\childdocname{\scantokens\expandafter{\jobname\noexpand}}
\let\childdocjob\childdocname
%    \end{macrocode}

% \macro{\childdocdisable}
% The macro |\childdocdisable| prevents the main file
% from being processed more than once.
% At this stage, the main document command |\childdocmain|
% is assumed to be called once again where it should do nothing.
% Any subsequent call to it should prevent
% a secondary processing of the main document
% It overwrites the forwarding commands
% |\childdocof| and |\childdocforward|
% with empty macros to prevent further inclusions of the main document:
%    \begin{macrocode}
\newcommand{\childdocdisable}
{
  \renewcommand{\childdocmain}[1]{\renewcommand{\childdocmain}[1]{\endinput}}
  \renewcommand{\childdocof}[1]{}
  \renewcommand{\childdocby}[2][]{}
  \renewcommand{\childdocforward}[2][]{}
  \renewcommand{\childdocdisable}{}
}
%    \end{macrocode}

% \macro{\childdocmain}
% The macro |\childdocmain| is to be called at the top of the main file
% with nothing or the main filename (without extension) as argument.
% First, it breaks loops.
% If the argument is not empty and does not match |\childdocname|
% (which is set by the first inclusion of |childdoc.def|),
% |\ifchilddoc| is set to true, |\includeonly| is applied to the child file
% and |\jobname| is set to the main file
% (for proper handling of |.aux| files):
%    \begin{macrocode}
\newcommand{\childdocmain}[1]
{
  \childdocdisable\childdocmain{}
  \if?#1?\else
    \begingroup
      \def\childdoctmp{#1}
      \ifx\childdoctmp\childdocname
        \def\childdoctmp{}
      \else
        \def\childdoctmp
        {
          \childdoctrue
          \includeonly{\childdocname}
          \def\childdocjob{#1}
          \def\jobname{#1}
        }
      \fi
      \expandafter
    \endgroup
    \childdoctmp
  \fi
}
%    \end{macrocode}

% \macro{\childdocof}
% The command |\childdocof| redirects
% compilation to the main file |#1|.
%    \begin{macrocode}
\newcommand{\childdocof}[1]
{
  \childdocdisable
  \childdoctrue
  \includeonly{\childdocname}
  \def\jobname{#1}
  \def\childdocjob{#1}
  \input{#1}
}
%    \end{macrocode}

% \macro{\childdocby}
% The command |\childdocby| ....
%    \begin{macrocode}
\newcommand{\childdocby}[2][]
{
  \childdocdisable
  \childdoctrue
  \childdocmanualtrue
  \if?#1?\else
    \def\jobname{#2}
  \fi
  \def\childdocjob{#2}
  \input{#2}
  \endinput
}
%    \end{macrocode}

% \macro{\childdocforward}
% The command |\childdocforward| redirects
% compilation to the main file or
% (if the optional argument is given) a child file.
% Parameters are set as if the main file
% or a child file starting with |\childdocof| was compiled.
% Then compilation is handed over to the main file:
%    \begin{macrocode}
\newcommand{\childdocforward}[2][]
{
  \begingroup
    \if?#1?
      \def\childdoctmp
      {
        \def\childdocname{#2}
        \def\childdocjob{#2}
        \def\jobname{#2}
        \input{#2}
        \endinput
      }
    \else
      \def\childdoctmp
      {
        \childdocdisable
        \def\childdocname{#2}
        \childdoctrue
        \includeonly{#2}
        \def\childdocjob{#1}
        \def\jobname{#1}
        \input{#1}
        \endinput
      }
    \fi
    \expandafter
  \endgroup
  \childdoctmp
}
%    \end{macrocode}

% \macro{\childdocforwardprefix}
% The command |\childdocforwardprefix| redirects
% compilation to the main or a child file by means of a pattern.
% The prefix |#1| in the current filename is replaced by |#2|
% and the suffix of the current filename is kept
% (it is assumed that the filename does not contain the substring `|~~~|'
% which is used as a delimiter).
% Compilation is handed over to the new file by |\childdocforward|:
%    \begin{macrocode}
\newcommand{\childdocforwardprefix}[3][]
{
  \begingroup
    \def\childdocextract #2##1~~~{\def\childdoctmp{\childdocforward[#1]{#3##1}}}
    \expandafter\childdocextract\childdocname~~~
    \expandafter
  \endgroup
  \childdoctmp
}
%    \end{macrocode}

% \macro{\childdoc}
% The deprecated macro |\childdoc| is a legacy version of |\childdocmain|:
%    \begin{macrocode}
\newcommand{\childdoc}{\childdocmain}
%    \end{macrocode}

% \macro{\childdocredirect}
% The deprecated macro |\childdocredirect| is a legacy version
% of |\childdocforward| and |\childdocforwardprefix|:
%    \begin{macrocode}
\newcommand{\childdocredirect}[2][]
{
  \begingroup
    \if?#1?
      \def\childdoctmp{\childdocforward{#2}}
    \else
      \def\childdoctmp{\childdocforwardprefix{#1}{#2}}
    \fi
    \expandafter
  \endgroup
  \childdoctmp
}
%    \end{macrocode}

%\iffalse
%</package>
%\fi
%
\endinput

\childdocmain{}
%    \end{macrocode}

% Optional override for |\version| flag:
%    \begin{macrocode}
%%\ifchilddoc\else\providecommand{\version}{draft}\fi
%    \end{macrocode}

% Define the default values for the |\version| flag
% (|final| for the main file and |draft| for childs):
%    \begin{macrocode}
\ifchilddoc
\providecommand{\version}{draft}
\else
\providecommand{\version}{final}
\fi
%    \end{macrocode}

% Load the standard document class:
%    \begin{macrocode}
\documentclass[12pt]{article}
%    \end{macrocode}

% Start the document body:
%    \begin{macrocode}
\begin{document}
%    \end{macrocode}

% Declare a title page.
% Print title, part of document being processed and version flag:
%    \begin{macrocode}
\addtocounter{page}{-1}
\begin{center}
{\LARGE\bfseries{}childdoc example\par}
\vspace{1cm}
\ifchilddoc
\ifchilddocmanual part\else chapter\fi:
`\childdocname' of `\childdocjob'\par
\else
main document: `\childdocjob'\par
\fi
version: \version\par
\end{center}
\newpage
%    \end{macrocode}

% Manually include selected file,
% otherwise process as usual:
%    \begin{macrocode}
\ifchilddocmanual
\section*{part `\childdocname'}
\input{\childdocname}
\else
%    \end{macrocode}

% Include the two chapters:
%    \begin{macrocode}
\include{cdocsch1}
\include{cdocsch2}
%    \end{macrocode}

% Include the two parts unless only chapters should be displayed:
%    \begin{macrocode}
\ifchilddoc\else
\section{part three}
\input{cdocspt3}
\section{part four}
\input{cdocspt4}
\fi
%    \end{macrocode}

% Process as usual until here:
%    \begin{macrocode}
\fi
%    \end{macrocode}

% End of document body:
%    \begin{macrocode}
\end{document}
%    \end{macrocode}
%\iffalse
%</samplemain>
%\fi
%
% %%%%%%%%%%%%%%%%%%%%%%%%%%%%%%%%%%%%%%
% \paragraph{Chapter Include Files.}
%
% The include files are called |cdocsch1.tex| and |cdocsch2.tex|.
%
%\iffalse
%<*samplechap1|samplechap2>
%\fi

% Optional override for |\version| flag:
%    \begin{macrocode}
%%\providecommand{\version}{final}
%    \end{macrocode}

% Include the main document:
%    \begin{macrocode}
% \iffalse
%
% childdoc.dtx Copyright (C) 2017-2018 Niklas Beisert
%
% This work may be distributed and/or modified under the
% conditions of the LaTeX Project Public License, either version 1.3
% of this license or (at your option) any later version.
% The latest version of this license is in
%   http://www.latex-project.org/lppl.txt
% and version 1.3 or later is part of all distributions of LaTeX
% version 2005/12/01 or later.
%
% This work has the LPPL maintenance status `maintained'.
%
% The Current Maintainer of this work is Niklas Beisert.
%
% This work consists of the files childdoc.dtx and childdoc.ins
% and the derived files childdoc.def and cdocsamp.tex with
% cdocsch1.tex, cdocsch2.tex, cdocsdrf.tex, cdocsfn1.tex, cdocsfn2.tex.
%
%<package>\ifdefined\childdocmain\endinput\fi
%<package>\ProvidesFile{childdoc.def}[2018/12/30 v2.0 child document driver]
%<samplemain>\ProvidesFile{cdocsamp.tex}[2018/12/30 v2.0 sample for childdoc]
%<*driver>
%\ProvidesFile{childdoc.drv}[2018/12/30 v2.0 childdoc reference manual file]
\PassOptionsToClass{10pt,a4paper}{article}
\documentclass{ltxdoc}

\usepackage[margin=35mm]{geometry}
\usepackage{hyperref}
\usepackage{hyperxmp}
\usepackage[usenames]{color}

\hypersetup{colorlinks=true}
\hypersetup{pdfstartview=FitH}
\hypersetup{pdfpagemode=UseNone}
\hypersetup{pdfsource={}}
\hypersetup{pdflang={en-UK}}
\hypersetup{pdfcopyright={Copyright 2017-2018 Niklas Beisert.
  This work may be distributed and/or modified under the
  conditions of the LaTeX Project Public License, either version 1.3
  of this license or (at your option) any later version.}}
\hypersetup{pdflicenseurl={http://www.latex-project.org/lppl.txt}}
\hypersetup{pdfcontactaddress={ETH Zurich, ITP, HIT K,
  Wolfgang-Pauli-Strasse 27}}
\hypersetup{pdfcontactpostcode={8093}}
\hypersetup{pdfcontactcity={Zurich}}
\hypersetup{pdfcontactcountry={Switzerland}}
\hypersetup{pdfcontactemail={nbeisert@itp.phys.ethz.ch}}
\hypersetup{pdfcontacturl={http://people.phys.ethz.ch/\xmptilde nbeisert/}}

\newcommand{\secref}[1]{\hyperref[#1]{section \ref*{#1}}}

\parskip1ex
\parindent0pt
\let\olditemize\itemize
\def\itemize{\olditemize\parskip0pt}

\begin{document}

\title{The \textsf{childdoc} Package}
\hypersetup{pdftitle={The childdoc Package}}
\author{Niklas Beisert\\[2ex]
  Institut f\"ur Theoretische Physik\\
  Eidgen\"ossische Technische Hochschule Z\"urich\\
  Wolfgang-Pauli-Strasse 27, 8093 Z\"urich, Switzerland\\[1ex]
  \href{mailto:nbeisert@itp.phys.ethz.ch}
  {\texttt{nbeisert@itp.phys.ethz.ch}}}
\hypersetup{pdfauthor={Niklas Beisert}}
\hypersetup{pdfsubject={Manual for the LaTeX2e Package childdoc}}
\date{30 December 2018, \textsf{v2.0}}
\maketitle

\begin{abstract}\noindent
\textsf{childdoc} is a \LaTeXe{} package
that enables the direct compilation
of document sections included by |\include|
to individual files.
\end{abstract}

\begingroup
\parskip0ex
\tableofcontents
\endgroup

%%%%%%%%%%%%%%%%%%%%%%%%%%%%%%%%%%%%%%%%%%%%%%%%%%%%%%%%%%%%%%%%%%%%%%%%%%%%%%%%
%%%%%%%%%%%%%%%%%%%%%%%%%%%%%%%%%%%%%%%%%%%%%%%%%%%%%%%%%%%%%%%%%%%%%%%%%%%%%%%%
\section{Introduction}

\LaTeX{} provides a mechanism to structure a large document (such as a book)
into a main file and several child files (containing the chapters)
using the |\include| command.
This mechanism is beneficial for documents
which span hundreds of pages in order to
make the source file(s) more manageable.
Moreover, compilation can be restricted to
selected child files by means of the |\includeonly| command.
The latter feature can be used to reduce the compilation time while editing
(this was significantly more useful in the earlier days of \LaTeX{})
or to generate a smaller document which is easier to navigate.
Another application of |\includeonly| is to generate
documents consisting of selected parts of the complete document.

However, there are a few drawbacks of the plain |\include| mechanism:
\begin{itemize}
\item
The child files cannot be compiled on their own,
they can only be compiled via the main file.
A naive editing environment
(such as a text editor with an option
to have the current file processed by \LaTeX)
may require one to switch to the main file before compiling;
attempting to compile the child file produces errors.
\item
The main file must be modified (each time)
to adjust the |\includeonly| command
to the present needs. This easily leaves the main file in a messy state.
\item
The generated document will always carry the filename
of the main document. This is inconvenient if
several child files are to be compiled and
to be kept for distribution.
\end{itemize}

The present package provides a simple interface
to make child files individually compilable by \LaTeX{}.
Compiling a child file then has the same effect as compiling
the main file with an |\includeonly| command
to select the appropriate child.
Moreover the generated document will carry the name of the child
rather than the main file.
This resolves all three above issues.

This feature is meant to make the editing of books,
thesis documents and lecture notes somewhat more convenient.
However, the package can also be used efficiently for
composing a series of documents (such as exercise sheets)
which are typically distributed individually.
It then assists the author in generating the individual documents
(potentially in different versions)
as well as a document containing the collected series.
Another application is in developing style files
or other kinds of included material
where compilation of the style file could redirect
to a sample or test file.

%%%%%%%%%%%%%%%%%%%%%%%%%%%%%%%%%%%%%%%%%%%%%%%%%%%%%%%%%%%%%%%%%%%%%%%%%%%%%%%%
%%%%%%%%%%%%%%%%%%%%%%%%%%%%%%%%%%%%%%%%%%%%%%%%%%%%%%%%%%%%%%%%%%%%%%%%%%%%%%%%
\section{Usage}

First of all, the package \textsf{childdoc} is \emph{not} a standard
\LaTeXe{} |.sty| style file! Therefore it needs to be invoked in
a non-standard way.

%%%%%%%%%%%%%%%%%%%%%%%%%%%%%%%%%%%%%%%%%%%%%%%%%%%%%%%%%%%%%%%%%%%%%%%%%%%%%%%%
\subsection{Included Files}
\label{sec:include}

%%%%%%%%%%%%%%%%%%%%%%%%%%%%%%%%%%%%%%%%
\DescribeMacro{\childdocmain}
To use the package, add the commands
\begin{center}
\begin{tabular}{l}
|\input{childdoc.def}|\\
|\childdocmain{}|\\
\end{tabular}
\end{center}
at the very top of the main \LaTeX{} file,
in particular \emph{before} the |\documentclass| statement!
The argument of |\childdocmain| should be left empty
(but it must be present).

%%%%%%%%%%%%%%%%%%%%%%%%%%%%%%%%%%%%%%%%
\DescribeMacro{\childdocof}
Furthermore, add the commands
\begin{center}
\begin{tabular}{l}
|\input{childdoc.def}|\\
|\childdocof{|\textit{main}|}|\\
\end{tabular}
\end{center}
at the top of every child file \textit{child}
which is included by |\include{|\textit{child}|}|
from within the main file
(or at least for those files to be compiled individually).
The argument \textit{main} must be the filename of the main file.

There are a couple of
considerations in setting up the main and child documents:

%%%%%%%%%%%%%%%%%%%%%%%%%%%%%%%%%%%%%%%%
\paragraph{Restrictions.}

Please note the following restrictions:
\begin{itemize}
\item
|\childdocmain| must be called with one argument \textit{main}
to ensure compatibility with earlier version of the package.
It must either be empty (|\childdocmain{}|)
or precisely match the filename of the main file in which it is specified.
See \secref{sec:detection} for further information.
\item
The filename \textit{main} must be specified without the |.tex| extension.
\item
The filename \textit{main} is case sensitive
(even in case-insensitive file systems)
due to internal string comparison.
\item
The argument \textit{main} should be fully expanded, it cannot be a macro.
\item
Subdirectories and special characters should be avoided in filenames.
\item
The command |\childdocmain{|\textit{main}|}| must be followed by a whitespace.
It should not be followed immediately by another command
or by a comment mark `|%|'.
This is because the \TeX{} parser reads the token immediately following
the argument of |\childdocmain| and puts it
at the beginning of every child section;
however, a white\-space is ignored.
\end{itemize}

%%%%%%%%%%%%%%%%%%%%%%%%%%%%%%%%%%%%%%%%
\paragraph{Content of Main File.}

It is advisable to place all content in the child files included by |\include|.
Any output contained in the main file will appear in all child documents
unless suppressed manually;
it cannot be suppressed automatically by the |\includeonly| directive
and thus should normally be avoided.
A method to include some content in the main file
by means of conditional processing is described in \secref{sec:conditional}.

%%%%%%%%%%%%%%%%%%%%%%%%%%%%%%%%%%%%%%%%
\paragraph{Page Numbering.}

When only a part of the document is compiled,
the appropriate numbering of pages
(as well as other status parameters)
is determined from the |.aux| files.
The latter contain information from previous passes.
However this information needs to propagate through
all intermediate child documents.
Therefore the page numbering in child documents may well
be inconsistent until the complete document is compiled at least once.

A useful (if unconventional) way to always ensure a consistent
page numbering is to restart the numbering in each child document
and denote the pages by `\textit{child}|.|\textit{page}'
where \textit{child} represents the chapter/section number of the child file.
This can be achieved by the command
|\numberwithin{page}{|\textit{child}|}|
of the \textsf{amsmath} package
where \textit{child} can be |chapter| or |section|
depending on the chosen structuring.
Alternatively, one can modify the macro |\thepage| appropriately
and reset the counter |page| at the start of each child file.

%%%%%%%%%%%%%%%%%%%%%%%%%%%%%%%%%%%%%%%%%%%%%%%%%%%%%%%%%%%%%%%%%%%%%%%%%%%%%%%%
\subsection{Conditional Processing}
\label{sec:conditional}

The package provides a mechanism to compile different versions
of a document. To customise the versions further some conditional processing
can come in handy to distinguish which version is being compiled.
The package provides two macros to describe the compilation context:

%%%%%%%%%%%%%%%%%%%%%%%%%%%%%%%%%%%%%%%%
\DescribeMacro{\ifchilddoc}
The conditional |\ifchilddoc| distinguishes between the compilation of
child documents and the main document:
%
\begin{center}
|\ifchilddoc |\textit{child-code}| |[|\||else |\textit{main-code}]| \||fi|
\end{center}

%%%%%%%%%%%%%%%%%%%%%%%%%%%%%%%%%%%%%%%%
\DescribeMacro{\childdocname}
\DescribeMacro{\childdocjob}
The macro |\childdocname| contains the filename (without extension)
of the main or child file being processed.
Note that |\childdocjob| will always contain the name of the main file.

%%%%%%%%%%%%%%%%%%%%%%%%%%%%%%%%%%%%%%%%
\paragraph{Title Page.}

Conditional processing can be used to include a title or banner page
in the main document when proper precautions are taken.
Importantly, the code in the main file should ensure that the page counter
(as well as other status parameters which are stored in the |.aux| files)
takes the same value after the conditional processing.
Otherwise the page numbers may take divergent values
depending on which part is compiled.

For example, a title page could be declared by:
%
\begin{center}
\begin{tabular}{l}
|\ifchilddoc\||else|\\
|\addtocounter{page}{-1}|\\
\textit{code for title page}\\
|\newpage|\\
|\||fi|
\end{tabular}
\end{center}
%
A banner page for the child documents can be generated by:
%
\begin{center}
\begin{tabular}{l}
|\ifchilddoc|\\
|\addtocounter{page}{-1}|\\
\textit{code for banner page}\\
|\newpage|\\
|\||fi|
\end{tabular}
\end{center}
%
Here one could write a message such as:
\begin{center}
|This is the part \childdocname{} of \childdocjob{}.|
\end{center}

%%%%%%%%%%%%%%%%%%%%%%%%%%%%%%%%%%%%%%%%%%%%%%%%%%%%%%%%%%%%%%%%%%%%%%%%%%%%%%%%
\subsection{Flags}
\label{sec:flags}

The package makes it easy to generate different versions
of the main or child documents.
To this end compilation flags can be defined
and assigned different default values.
They will be particularly useful in conjunction
with the forwarding mechanism described in \secref{sec:forward}.

For example, it may be useful to have a flag |\version|
which can be set to |draft| or |final|.
The document source will contain some conditional code
depending on the value of |\version|.
Suppose further, the flag should default to |final| for the main file
and to |draft| for child files
which is a natural assignment for editing the document.
This is achieved by placing the following code
in the preamble of the main document
(below the |\childdocmain| directive):
%
\begin{center}
\begin{tabular}{l}
|\ifchilddoc|\\
|\providecommand{\version}{draft}|\\
|\||else|\\
|\providecommand{\version}{final}|\\
|\||fi|
\end{tabular}
\end{center}
%
The definition by |\providecommand| makes sure
that previous definitions are not overwritten.
Further statements |\providecommand{\version}{...}|
can thus be added before the above code to override it.

For the main file, one might add a line
(between |\childdocmain| and the above block)
%
\begin{center}
|%\ifchilddoc\||else\providecommand{\version}{draft}\||fi|
\end{center}
%
which can be uncommented to produce a draft version.
Likewise one can add a line to the very top of a child file
(above the |\childdocof{|\textit{main}|}| directive)
%
\begin{center}
|%\providecommand{\version}{final}|
\end{center}
%
which can be uncommented to produce the final version of this child document.

%%%%%%%%%%%%%%%%%%%%%%%%%%%%%%%%%%%%%%%%%%%%%%%%%%%%%%%%%%%%%%%%%%%%%%%%%%%%%%%%
\subsection{Forwarding}
\label{sec:forward}

Different versions of the main or child documents
using compilation flags as described in \secref{sec:flags}
can be (permanently) stored in different files
for convenient compilation, viewing and distribution.
To this end, the package defines a command
to pass on compilation to a different file:

%%%%%%%%%%%%%%%%%%%%%%%%%%%%%%%%%%%%%%%%
\DescribeMacro{\childdocforward}
The command |\childdocforward| redirects processing to
another source file:
%
\begin{center}
\begin{tabular}{l}
|\input{childdoc.def}|\\
|\childdocforward[|\textit{main}|]{|\textit{dest}|}|\\
\end{tabular}
\end{center}
%
The argument \textit{dest} is the destination file
(without extension).
It should be the main file or one of the child files.
Note that further \textsf{childdoc} directives
such as |\childdocof| and |\childdocforward|
in the indicated file will be processed in this form.
The optional argument \textit{main}
passes on directly to the main file \textit{main}
while pretending to compile the child \textit{dest}.
This form behaves as if \textit{dest}
issues |\childdocof{|\textit{main}|}| right away,
and no further \textsf{childdoc} directives will be processed.

%%%%%%%%%%%%%%%%%%%%%%%%%%%%%%%%%%%%%%%%
\DescribeMacro{\...prefix}
In the alternative form |\childdocforwardprefix|,
%
\begin{center}
\begin{tabular}{l}
|\input{childdoc.def}|\\
|\childdocforwardprefix[|\textit{main}|]{|\textit{prefix}|}{|\textit{dest}|}|
\end{tabular}
\end{center}
%
the destination file is determined by a pattern
depending on the current file:
To make this work, the current file must be called
`{\textit{prefix}\hspace{0.2em}\textit{suffix}}'
with \textit{prefix} matching precisely the argument.
Processing is then passed on to the file
`{\textit{dest}\hspace{0.2em}\textit{suffix}}'.
Surely, the same effect is achieved by
directly specifying the
argument `{\textit{dest}\hspace{0.2em}\textit{suffix}}'
in the first form.
However, that requires to set up a different file
for each child. With the alternative form of the command
all these files can have exactly the same content
which simplifies setting them up and maintaining them.

For example, the following file |draft.tex|
with a compilation flag |\version| as described in \secref{sec:flags}
compiles the main document as a draft:
%
\begin{center}
\begin{tabular}{l}
|\def\version{draft}|\\
|\input{childdoc.def}|\\
|\childdocforward{|\textit{main}|}|
\end{tabular}
\end{center}
%
Likewise, the following files |final|\textit{nn}|.tex|
compile the final version of the child document
|child|\textit{nn}|.tex|:
%
\begin{center}
\begin{tabular}{l}
|\def\version{final}|\\
|\input{childdoc.def}|\\
|\childdocforwardprefix{final}{child}|
\end{tabular}
\end{center}
%

Note that when several versions of a main file and/or of each child file
are to be generated, it may be convenient to set up a |Makefile| or
shell script to automatise the process.

%%%%%%%%%%%%%%%%%%%%%%%%%%%%%%%%%%%%%%%%%%%%%%%%%%%%%%%%%%%%%%%%%%%%%%%%%%%%%%%%
\subsection{Command Line Processing}
\label{sec:commandline}

The effect of redirection files can also be achieved by invoking
the \LaTeX{} compiler with a more elaborate command line.
Most conveniently this should be done as part
of a shell script or a |Makefile|.

When using \textsf{childdoc} in the main file, the following
command lines effectively perform a redirection
(note that depending on the shell being used,
backslashes may have to be doubled: `|\|' $\to$ `|\\|'):
%
\begin{center}
|... -jobname "|\textit{target}|" |\\|"|[\textit{flags}]%
|\input{childdoc.def}\childdocforward[|\textit{main}|]{|\textit{dest}|}"|
\end{center}
%
Here \textit{target} is the name of the output file,
\textit{main} is the name of the main file
and \textit{dest} is the name of the main or child file to be processed
(all filenames without extensions).
The optional argument \textit{main} can be omitted
if \textit{main} matches \textit{dest}.
Optionally, compilation \textit{flags} can be defined via |\def| commands.
This command line makes the \TeX{} engine believe
it is compiling the file \textit{target}
whose content is specified as the latter parameter.
The provided code then forwards the processing to
\textit{main} or \textit{dest} as described in \secref{sec:forward}.

%%%%%%%%%%%%%%%%%%%%%%%%%%%%%%%%%%%%%%%%%%%%%%%%%%%%%%%%%%%%%%%%%%%%%%%%%%%%%%%%
\subsection{Include by Input}
\label{sec:input}

Including child documents by |\include| has some restrictions by design.
Most notably, the content of a child document always occupies
its own set of pages; pages cannot be shared between child documents.
Usually, this behaviour makes perfect sense
because each child document contain an essential part of the document.
However, in some situations it may be desirable to compose
a document from a collection of parts
without having mandatory page breaks between then.
For this case, the package
provides a mechanism to include parts
by |\input| which can also be processed individually.
However, by construction this mechanism
requires manual handling of the content to be output.

%%%%%%%%%%%%%%%%%%%%%%%%%%%%%%%%%%%%%%%%
\DescribeMacro{\ifchilddocmanual}
The main file should be prepared as usual, see \secref{sec:include}.
However, the document body must make a distinction
between processing of an individual part and of the main document, e.g.:
%
\begin{center}
\begin{tabular}{l}
|\ifchilddocmanual|\\
|\input{\childdocname}|\\
|\||else|\\
\textit{document body with }|\input{|\textit{part}|}|\\
|\||fi|
\end{tabular}
\end{center}
%
The conditional |\ifchilddocmanual| is true whenever
a part to be included by |\input| is being compiled,
and the name of the part is stored in |\childdocname|.

%%%%%%%%%%%%%%%%%%%%%%%%%%%%%%%%%%%%%%%%
\DescribeMacro{\childdocby}
Each part to be included by |\input| should start with:
%
\begin{center}
\begin{tabular}{l}
|\input{childdoc.def}|\\
|\childdocby{|\textit{main}|}|\\
\end{tabular}
\end{center}
%
The directive |\childdocby| is similar to |\childdocof|
described in \secref{sec:include},
but the subsequent selection of content must be done manually.
To that end, both |\ifchilddoc| and |\ifchilddocmanual|
will be true upon processing of a part,
and the name of the part is stored in |\childdocname|.
Note that |\jobname| will be set to the filename of the current part
so that each part receives an individual |.aux| file
that does not interfere with the |.aux| file(s) of the main document.
This behaviour can be altered by the alternative form
|\childdocby[*]{|\textit{main}|}| (with a non-empty optional argument)
which uses the |.aux| file of the main document
by setting |\jobname| to \textit{main}.

%%%%%%%%%%%%%%%%%%%%%%%%%%%%%%%%%%%%%%%%%%%%%%%%%%%%%%%%%%%%%%%%%%%%%%%%%%%%%%%%
\subsection{Driver Development}
\label{sec:driver}

The \textsf{childdoc} mechanism can also be use for the development
of definition files such as \LaTeX{} styles or classes.
This case differs from the above setup with multiple parts
included by |\include| in that no |\includeonly| should be invoked.
This can be achieved by starting the include file
(before |\ProvidesPackage|) with:
%
\begin{center}
\begin{tabular}{l}
|\input{childdoc.def}|\\
|\childdocforward{|\textit{main}|}|\\
\end{tabular}
\end{center}
%
or alternatively with:
%
\begin{center}
\begin{tabular}{l}
|\input{childdoc.def}|\\
|\childdocby{|\textit{main}|}|\\
\end{tabular}
\end{center}
%
Both forms have slightly different effects as described above.
The main file is prepared as usual, see \secref{sec:include}.

%%%%%%%%%%%%%%%%%%%%%%%%%%%%%%%%%%%%%%%%%%%%%%%%%%%%%%%%%%%%%%%%%%%%%%%%%%%%%%%%
\subsection{Legacy Detection}
\label{sec:detection}

The directive |\childdocmain| in the main file can detect
whether the complete document or merely a child is to be compiled
even without using the directive |\childdocof|.
This method is deprecated because it is less robust
and there is no compelling reason to use it;
it is merely provided for backward compatibility
and it may be removed in future versions.

If the detection mechanism is to be used,
it is mandatory to correctly specify
the filename of the main file as the argument of |\childdocmain|:
%
\begin{center}
\begin{tabular}{l}
|\input{childdoc.def}|\\
|\childdocmain{|\textit{main}|}|\\
\end{tabular}
\end{center}
%
If |\jobname| does not match the argument \textit{main} of |\childdocmain|,
it is assumed that |\jobname| points to the child file to be compiled.
When using |\childdocmain| with the main file specified as argument,
it suffices to start a child file
with just |\input{|\textit{main}|}|
without loading of the package and using |\childdocof|.
If instead all processing is done
with the appropriate \textsf{childdoc} directives,
the argument of \textit{main} of |\childdocmain| can be empty.

An alternative version of the command line processing described
in \secref{sec:commandline} using the detection mechanism reads:
%
\begin{center}
|... -jobname "|\textit{target}|" "|[\textit{flags}]%
[|\def\jobname{|\textit{dest}|}|]|\input{|\textit{main}|}"|
\end{center}

%%%%%%%%%%%%%%%%%%%%%%%%%%%%%%%%%%%%%%%%%%%%%%%%%%%%%%%%%%%%%%%%%%%%%%%%%%%%%%%%
\subsection{Manual Code}
\label{sec:manual}

In case one cannot be certain whether the definitions file |childdoc.def|
is installed on the target \TeX{} distribution
and one prefers not to ship it,
it is conceivable to paste a few relevant commands into the sources.

To that end, drop all statements |\input{childdoc.def}|
and perform the replacements as outlined below.
Instead of |\childdocmain{|\textit{main}|}| add the following code
to the top of the main file:
%
\begin{center}
\begin{tabular}{l}
|\||ifdefined\childdocname\endinput\||fi\newif\ifchilddoc|\\
|\edef\childdocname{\scantokens\expandafter{\jobname\noexpand}}|\\
|\def\childdocmain{|\textit{main}|}\||ifx\childdocmain\childdocname\||else|\\
|\childdoctrue\includeonly{\childdocname}\let\jobname\childdocmain\||fi|\\
\end{tabular}
\end{center}
%
Instead of |\childdocof{|\textit{main}|}| just include the main file
at the top of each child file:
%
\begin{center}
|\input{|\textit{main}|}|
\end{center}
%
A simple redirection |\childdocforward{|\textit{dest}|}| is achieved by:
%
\begin{center}
|\def\jobname{|\textit{dest}|}\input{\jobname}|
\end{center}
%
The redirection with prefix
|\childdocforwardprefix[|\textit{prefix}|]{|\textit{dest}|}|
is accomplished by:
%
\begin{center}
\begin{tabular}{l}
|{\edef\jobname{\scantokens\expandafter{\jobname\noexpand}}|\\
|\def\redirectjob |\textit{prefix}|#1~~~{\gdef\jobname{|\textit{dest}|#1}}|\\
|\expandafter\redirectjob\jobname~~~}\input{\jobname}|
\end{tabular}
\end{center}

In an alternative approach,
child documents can be compiled by a specific command line
without additional code or specific definitions:
%
\begin{center}
|... -jobname "|\textit{target}|" "|[\textit{flags}]%
|\includeonly{|\textit{dest}|}\input{|\textit{main}|}"|
\end{center}
%

%%%%%%%%%%%%%%%%%%%%%%%%%%%%%%%%%%%%%%%%%%%%%%%%%%%%%%%%%%%%%%%%%%%%%%%%%%%%%%%%
%%%%%%%%%%%%%%%%%%%%%%%%%%%%%%%%%%%%%%%%%%%%%%%%%%%%%%%%%%%%%%%%%%%%%%%%%%%%%%%%
\section{Information}

%%%%%%%%%%%%%%%%%%%%%%%%%%%%%%%%%%%%%%%%%%%%%%%%%%%%%%%%%%%%%%%%%%%%%%%%%%%%%%%%
\subsection{Copyright}

Copyright \copyright{} 2017--2018 Niklas Beisert

This work may be distributed and/or modified under the
conditions of the \LaTeX{} Project Public License, either version 1.3
of this license or (at your option) any later version.
The latest version of this license is in
  \url{http://www.latex-project.org/lppl.txt}
and version 1.3 or later is part of all distributions of \LaTeX{}
version 2005/12/01 or later.

This work has the LPPL maintenance status `maintained'.

The Current Maintainer of this work is Niklas Beisert.

This work consists of the files |README.txt|, |childdoc.ins| and |childdoc.dtx|
as well as the derived files |childdoc.def|, |cdocsamp.tex|
with |cdocsch1.tex|, |cdocsch2.tex|, |cdocspt3.tex|, |cdocspt4.tex|,
|cdocsdrf.tex|, |cdocsfn1.tex|, |cdocsfn2.tex|
as well as |childdoc.pdf|.

%%%%%%%%%%%%%%%%%%%%%%%%%%%%%%%%%%%%%%%%%%%%%%%%%%%%%%%%%%%%%%%%%%%%%%%%%%%%%%%%
\subsection{Files and Installation}

The package consists of the files:
%
\begin{center}
\begin{tabular}{ll}
    |README.txt|   & readme file \\
    |childdoc.ins| & installation file \\
    |childdoc.dtx| & source file \\
    |childdoc.def| & definition file \\
    |cdocsamp.tex| & sample main file \\
    |cdocsch1.tex| & sample include file \\
    |cdocsch2.tex| & sample include file \\
    |cdocspt3.tex| & sample part file \\
    |cdocspt4.tex| & sample part file \\
    |cdocsdrf.tex| & sample redirection file \\
    |cdocsfn1.tex| & sample redirection file \\
    |cdocsfn2.tex| & sample redirection file \\
    |childdoc.pdf| & manual
\end{tabular}
\end{center}
%
The distribution consists of the files
|README.txt|, |childdoc.ins| and |childdoc.dtx|.
%
\begin{itemize}
\item
Run (pdf)\LaTeX{} on |childdoc.dtx|
to compile the manual |childdoc.pdf| (this file).
\item
Run \LaTeX{} on |childdoc.ins| to create the definitions file |childdoc.def|
and the sample |cdocsamp.tex| with include files
|cdocsch1.tex|, |cdocsch2.tex|, |cdocspt3.tex|, |cdocspt4.tex|,
|cdocsdrf.tex|, |cdocsfn1.tex|, |cdocsfn2.tex|.
Then copy the file |childdoc.def| to an appropriate directory of your \LaTeX{}
distribution, e.g.\ \textit{texmf-root}|/tex/latex/childdoc|.
\end{itemize}

%%%%%%%%%%%%%%%%%%%%%%%%%%%%%%%%%%%%%%%%%%%%%%%%%%%%%%%%%%%%%%%%%%%%%%%%%%%%%%%%
\subsection{Related CTAN Packages}

There are several other packages which offer a similar functionality:
%
\begin{itemize}
\item
The packages
\href{http://ctan.org/pkg/docmute}{\textsf{docmute}},
\href{http://ctan.org/pkg/includex}{\textsf{includex}} and
\href{http://ctan.org/pkg/standalone}{\textsf{standalone}}
provide commands to include only the document body of
a child file thus allowing both files to be compiled individually.
\item
The packages \href{http://ctan.org/pkg/subdocs}{\textsf{subdocs}}
and \href{http://ctan.org/pkg/subfiles}{\textsf{subfiles}}
provide structures in which the main and child documents can be
encapsulated and allowing them to be compiled individually.
The inclusion mechanism is different from the conventional |\include|.
\item
The package \href{http://ctan.org/pkg/combine}{\textsf{combine}}
is an elaborate solution to combine several documents into one.
\end{itemize}
%
See also the CTAN topic \href{http://ctan.org/topic/subdocs}{\textsf{subdocs}}
for further related packages.
The present package differs from the above solutions in that
a document structure constructed with the conventional |\include| mechanism
just needs two extra commands at the top of every file
such that all constituent files can be compiled individually.

%%%%%%%%%%%%%%%%%%%%%%%%%%%%%%%%%%%%%%%%%%%%%%%%%%%%%%%%%%%%%%%%%%%%%%%%%%%%%%%%
%\subsection{Feature Suggestions}
%
%The following is a list of features which may be useful for future
%versions of this package:
%%
%\begin{itemize}
%\item
%\ldots
%\end{itemize}

%%%%%%%%%%%%%%%%%%%%%%%%%%%%%%%%%%%%%%%%%%%%%%%%%%%%%%%%%%%%%%%%%%%%%%%%%%%%%%%%
\subsection{Revision History}

%%%%%%%%%%%%%%%%%%%%%%%%%%%%%%%%%%%%%%%%
\paragraph{v2.0:} 2018/12/30

\begin{itemize}
\item
immediate forward processing
\item
added |\childdocby| mechanism
\item
manual restructured
\end{itemize}

%%%%%%%%%%%%%%%%%%%%%%%%%%%%%%%%%%%%%%%%
\paragraph{v1.6:} 2018/01/17

\begin{itemize}
\item
application for development of include files
\item
corrections to manual
\end{itemize}

%%%%%%%%%%%%%%%%%%%%%%%%%%%%%%%%%%%%%%%%
\paragraph{v1.5:} 2017/05/21

\begin{itemize}
\item
more complete structuring introduced
\item
|\childdocof| introduced
\item
|\childdoc| renamed to |\childdocmain|
\item
|\childredirect| renamed to |\childdocforward| and |\childdocforwardprefix|
and functionality expanded
\end{itemize}

%%%%%%%%%%%%%%%%%%%%%%%%%%%%%%%%%%%%%%%%
\paragraph{v1.0:} 2017/04/27

\begin{itemize}
\item
manual and install package
\item
first version published on CTAN
\end{itemize}

%%%%%%%%%%%%%%%%%%%%%%%%%%%%%%%%%%%%%%%%
\paragraph{v0.6:} 2017/04/26

\begin{itemize}
\item
redirection mechanism added
\end{itemize}

%%%%%%%%%%%%%%%%%%%%%%%%%%%%%%%%%%%%%%%%
\paragraph{v0.5:} 2017/04/26

\begin{itemize}
\item
functionality in definition file
\end{itemize}


%%%%%%%%%%%%%%%%%%%%%%%%%%%%%%%%%%%%%%%%%%%%%%%%%%%%%%%%%%%%%%%%%%%%%%%%%%%%%%%%
%%%%%%%%%%%%%%%%%%%%%%%%%%%%%%%%%%%%%%%%%%%%%%%%%%%%%%%%%%%%%%%%%%%%%%%%%%%%%%%%
%%%%%%%%%%%%%%%%%%%%%%%%%%%%%%%%%%%%%%%%%%%%%%%%%%%%%%%%%%%%%%%%%%%%%%%%%%%%%%%%
\appendix

\settowidth\MacroIndent{\rmfamily\scriptsize 000\ }

 \DocInput{childdoc.dtx}

\end{document}
%</driver>
% \fi
%
% %%%%%%%%%%%%%%%%%%%%%%%%%%%%%%%%%%%%%%%%%%%%%%%%%%%%%%%%%%%%%%%%%%%%%%%%%%%%%%
% %%%%%%%%%%%%%%%%%%%%%%%%%%%%%%%%%%%%%%%%%%%%%%%%%%%%%%%%%%%%%%%%%%%%%%%%%%%%%%
% \section{Sample}
%\iffalse
%<*samplemain>
%\fi
%
% The following presents a sample document
% with two chapters, two parts, a title page,
% a compile flag as well as three forwarding files to set the flag.
% It consists of eight |.tex| files:
% \begin{center}
% \begin{tabular}{ll}
% |cdocsamp.tex|&main file\\
% |cdocsch1.tex|&include file for chapter 1\\
% |cdocsch2.tex|&include file for chapter 2\\
% |cdocspt3.tex|&include file for part 3\\
% |cdocspt4.tex|&include file for part 4\\
% |cdocsdrf.tex|&forwarding file for main file in draft mode\\
% |cdocsfi1.tex|&forwarding file for final version of chapter 1\\
% |cdocsfi2.tex|&forwarding file for final version of chapter 2\\
% \end{tabular}
% \end{center}
% Each of the eight files can be compiled directly by the \LaTeX{} compiler.
%
% %%%%%%%%%%%%%%%%%%%%%%%%%%%%%%%%%%%%%%
% \paragraph{Main File.}
%
% The main file is called |cdocsamp.tex|.
%
% Load the \textsf{childdoc} definitions and
% declare the filename for the main document:
%    \begin{macrocode}
\input{childdoc.def}
\childdocmain{}
%    \end{macrocode}

% Optional override for |\version| flag:
%    \begin{macrocode}
%%\ifchilddoc\else\providecommand{\version}{draft}\fi
%    \end{macrocode}

% Define the default values for the |\version| flag
% (|final| for the main file and |draft| for childs):
%    \begin{macrocode}
\ifchilddoc
\providecommand{\version}{draft}
\else
\providecommand{\version}{final}
\fi
%    \end{macrocode}

% Load the standard document class:
%    \begin{macrocode}
\documentclass[12pt]{article}
%    \end{macrocode}

% Start the document body:
%    \begin{macrocode}
\begin{document}
%    \end{macrocode}

% Declare a title page.
% Print title, part of document being processed and version flag:
%    \begin{macrocode}
\addtocounter{page}{-1}
\begin{center}
{\LARGE\bfseries{}childdoc example\par}
\vspace{1cm}
\ifchilddoc
\ifchilddocmanual part\else chapter\fi:
`\childdocname' of `\childdocjob'\par
\else
main document: `\childdocjob'\par
\fi
version: \version\par
\end{center}
\newpage
%    \end{macrocode}

% Manually include selected file,
% otherwise process as usual:
%    \begin{macrocode}
\ifchilddocmanual
\section*{part `\childdocname'}
\input{\childdocname}
\else
%    \end{macrocode}

% Include the two chapters:
%    \begin{macrocode}
\include{cdocsch1}
\include{cdocsch2}
%    \end{macrocode}

% Include the two parts unless only chapters should be displayed:
%    \begin{macrocode}
\ifchilddoc\else
\section{part three}
\input{cdocspt3}
\section{part four}
\input{cdocspt4}
\fi
%    \end{macrocode}

% Process as usual until here:
%    \begin{macrocode}
\fi
%    \end{macrocode}

% End of document body:
%    \begin{macrocode}
\end{document}
%    \end{macrocode}
%\iffalse
%</samplemain>
%\fi
%
% %%%%%%%%%%%%%%%%%%%%%%%%%%%%%%%%%%%%%%
% \paragraph{Chapter Include Files.}
%
% The include files are called |cdocsch1.tex| and |cdocsch2.tex|.
%
%\iffalse
%<*samplechap1|samplechap2>
%\fi

% Optional override for |\version| flag:
%    \begin{macrocode}
%%\providecommand{\version}{final}
%    \end{macrocode}

% Include the main document:
%    \begin{macrocode}
\input{childdoc.def}
\childdocof{cdocsamp}
%    \end{macrocode}

%\iffalse
%</samplechap1|samplechap2>
%\fi
%
%\iffalse
%<*samplechap1>
%\fi
% Some text for chapter 1:
%    \begin{macrocode}
\section{one}
some text in chapter one
%    \end{macrocode}

%\iffalse
%</samplechap1>
%\fi
% Some text for chapter 2:
%\iffalse
%<*samplechap2>
%\fi
%    \begin{macrocode}
\section{two}
more text in chapter two
%    \end{macrocode}

%\iffalse
%</samplechap2>
%\fi
%
% %%%%%%%%%%%%%%%%%%%%%%%%%%%%%%%%%%%%%%
% \paragraph{Part Include Files.}
%
% The include files are called |cdocspt3.tex| and |cdocspt4.tex|.
%
%\iffalse
%<*samplepart3|samplepart4>
%\fi

% Optional override for |\version| flag:
%    \begin{macrocode}
%%\providecommand{\version}{final}
%    \end{macrocode}

% Include the main document:
%    \begin{macrocode}
\input{childdoc.def}
\childdocby{cdocsamp}
%    \end{macrocode}

%\iffalse
%</samplepart3|samplepart4>
%\fi
%
%\iffalse
%<*samplepart3>
%\fi
% Some text for part 3:
%    \begin{macrocode}
some text in part three
%    \end{macrocode}

%\iffalse
%</samplepart3>
%\fi
% Some text for part 4:
%\iffalse
%<*samplepart4>
%\fi
%    \begin{macrocode}
more text in part four
%    \end{macrocode}

%\iffalse
%</samplepart4>
%\fi
%
% %%%%%%%%%%%%%%%%%%%%%%%%%%%%%%%%%%%%%%
% \paragraph{Forwarding for a Complete Draft.}
%
% The following forwarding file |cdocsdrf.tex|
% compiles the main document in draft mode:
%\iffalse
%<*sampledraft>
%\fi
%    \begin{macrocode}
\def\version{draft}
\input{childdoc.def}
\childdocforward{cdocsamp}
%    \end{macrocode}

%\iffalse
%</sampledraft>
%\fi
%
% %%%%%%%%%%%%%%%%%%%%%%%%%%%%%%%%%%%%%%
% \paragraph{Forwarding for Final Version of the Chapters.}
%
% The following forwarding files |cdocsfn1.tex| and |cdocsfn2.tex|
% (with identical content)
% compile the final versions of the child documents
% |cdocsch1.tex| and |cdocsch2.tex|, respectively:
%\iffalse
%<*samplefinal>
%\fi
%    \begin{macrocode}
\def\version{final}
\input{childdoc.def}
\childdocforwardprefix[cdocsamp]{cdocsfn}{cdocsch}
%    \end{macrocode}

%\iffalse
%</samplefinal>
%\fi
%
% %%%%%%%%%%%%%%%%%%%%%%%%%%%%%%%%%%%%%%
% \paragraph{Command Line Processing.}
%
% The following three command lines generate the output files
% |cdocscld|, |cdocscl1| and |cdocscl2|
% which should be identical to
% |cdocsdrf|, |cdocsch1| and |cdocsfn2|, respectively:
% \begin{center}
% \begin{tabular}{l}
% |latex -jobname cdocscld \|\\
% |  "\def\version{draft}\input{childdoc.def}\childdocforward{cdocsamp}"|\\
% |latex -jobname cdocscl1 \|\\
% |  "\input{childdoc.def}\childdocforward[cdocsamp]{cdocsch1}"|\\
% |latex -jobname cdocscl2 \|\\
% |  "\def\version{final}\input{childdoc.def}\childdocforward{cdocsch2}"|
% \end{tabular}
% \end{center}
% Note that the trailing backslash on each first line
% merely continues the input to the second line
% (for convenient cut ant paste).
% Furthermore, the command |latex| can be replaced by any
% of its alternative versions such as |pdflatex|.
%
% %%%%%%%%%%%%%%%%%%%%%%%%%%%%%%%%%%%%%%%%%%%%%%%%%%%%%%%%%%%%%%%%%%%%%%%%%%%%%%
% %%%%%%%%%%%%%%%%%%%%%%%%%%%%%%%%%%%%%%%%%%%%%%%%%%%%%%%%%%%%%%%%%%%%%%%%%%%%%%
% \section{Implementation}
%\iffalse
%<*package>
%\fi
%
% This section describes the definitions file |childdoc.def|.

% The definitions cannot be loaded using |\usepackage| or |\RequirePackage|
% which has a mechanism to prevent loading a style file more than once.
% When loading the definitions by means of |\input|
% multiple instances have to be prevented manually:
%\iffalse
%This code needs to be before the `\ProvidesFile' directive
%which is defined at the beginning of this file.
%Therefore it is also placed there and commented out here.
%</package>
%<*discard>
%\fi
%    \begin{macrocode}
\ifdefined\childdocmain\endinput\fi
%    \end{macrocode}
%\iffalse
%</discard>
%<*package>
%\fi
%
% \macro{\ifchilddoc}
% \macro{\ifchilddocmanual}
% The conditional |\ifchilddoc| tells whether a
% child (true) or main (false) document is being compiled.
% The conditional |\ifchilddocmanual| tells whether
% the |\includeonly| mechanism is used (false) or
% the selection of child files must be performed manually (true).
% The definitions initialise to false:
%    \begin{macrocode}
\newif\ifchilddoc
\newif\ifchilddocmanual
%    \end{macrocode}

% \macro{\childdocname}
% \macro{\childdocjob}
% The macro |\childdocname| stores the name of the main document
% to be compiled. The macro |\childdocjob| stores the name of
% the document on which the \LaTeX{} compiler was originally invoked.
% The content of |\jobname| cannot be compared
% to filenames specified in the source due to different catcodes.
% The following code rescans |\jobname|, stores the result
% in |\childdocname| and saves a copy in |\childdocjob|:
%    \begin{macrocode}
\edef\childdocname{\scantokens\expandafter{\jobname\noexpand}}
\let\childdocjob\childdocname
%    \end{macrocode}

% \macro{\childdocdisable}
% The macro |\childdocdisable| prevents the main file
% from being processed more than once.
% At this stage, the main document command |\childdocmain|
% is assumed to be called once again where it should do nothing.
% Any subsequent call to it should prevent
% a secondary processing of the main document
% It overwrites the forwarding commands
% |\childdocof| and |\childdocforward|
% with empty macros to prevent further inclusions of the main document:
%    \begin{macrocode}
\newcommand{\childdocdisable}
{
  \renewcommand{\childdocmain}[1]{\renewcommand{\childdocmain}[1]{\endinput}}
  \renewcommand{\childdocof}[1]{}
  \renewcommand{\childdocby}[2][]{}
  \renewcommand{\childdocforward}[2][]{}
  \renewcommand{\childdocdisable}{}
}
%    \end{macrocode}

% \macro{\childdocmain}
% The macro |\childdocmain| is to be called at the top of the main file
% with nothing or the main filename (without extension) as argument.
% First, it breaks loops.
% If the argument is not empty and does not match |\childdocname|
% (which is set by the first inclusion of |childdoc.def|),
% |\ifchilddoc| is set to true, |\includeonly| is applied to the child file
% and |\jobname| is set to the main file
% (for proper handling of |.aux| files):
%    \begin{macrocode}
\newcommand{\childdocmain}[1]
{
  \childdocdisable\childdocmain{}
  \if?#1?\else
    \begingroup
      \def\childdoctmp{#1}
      \ifx\childdoctmp\childdocname
        \def\childdoctmp{}
      \else
        \def\childdoctmp
        {
          \childdoctrue
          \includeonly{\childdocname}
          \def\childdocjob{#1}
          \def\jobname{#1}
        }
      \fi
      \expandafter
    \endgroup
    \childdoctmp
  \fi
}
%    \end{macrocode}

% \macro{\childdocof}
% The command |\childdocof| redirects
% compilation to the main file |#1|.
%    \begin{macrocode}
\newcommand{\childdocof}[1]
{
  \childdocdisable
  \childdoctrue
  \includeonly{\childdocname}
  \def\jobname{#1}
  \def\childdocjob{#1}
  \input{#1}
}
%    \end{macrocode}

% \macro{\childdocby}
% The command |\childdocby| ....
%    \begin{macrocode}
\newcommand{\childdocby}[2][]
{
  \childdocdisable
  \childdoctrue
  \childdocmanualtrue
  \if?#1?\else
    \def\jobname{#2}
  \fi
  \def\childdocjob{#2}
  \input{#2}
  \endinput
}
%    \end{macrocode}

% \macro{\childdocforward}
% The command |\childdocforward| redirects
% compilation to the main file or
% (if the optional argument is given) a child file.
% Parameters are set as if the main file
% or a child file starting with |\childdocof| was compiled.
% Then compilation is handed over to the main file:
%    \begin{macrocode}
\newcommand{\childdocforward}[2][]
{
  \begingroup
    \if?#1?
      \def\childdoctmp
      {
        \def\childdocname{#2}
        \def\childdocjob{#2}
        \def\jobname{#2}
        \input{#2}
        \endinput
      }
    \else
      \def\childdoctmp
      {
        \childdocdisable
        \def\childdocname{#2}
        \childdoctrue
        \includeonly{#2}
        \def\childdocjob{#1}
        \def\jobname{#1}
        \input{#1}
        \endinput
      }
    \fi
    \expandafter
  \endgroup
  \childdoctmp
}
%    \end{macrocode}

% \macro{\childdocforwardprefix}
% The command |\childdocforwardprefix| redirects
% compilation to the main or a child file by means of a pattern.
% The prefix |#1| in the current filename is replaced by |#2|
% and the suffix of the current filename is kept
% (it is assumed that the filename does not contain the substring `|~~~|'
% which is used as a delimiter).
% Compilation is handed over to the new file by |\childdocforward|:
%    \begin{macrocode}
\newcommand{\childdocforwardprefix}[3][]
{
  \begingroup
    \def\childdocextract #2##1~~~{\def\childdoctmp{\childdocforward[#1]{#3##1}}}
    \expandafter\childdocextract\childdocname~~~
    \expandafter
  \endgroup
  \childdoctmp
}
%    \end{macrocode}

% \macro{\childdoc}
% The deprecated macro |\childdoc| is a legacy version of |\childdocmain|:
%    \begin{macrocode}
\newcommand{\childdoc}{\childdocmain}
%    \end{macrocode}

% \macro{\childdocredirect}
% The deprecated macro |\childdocredirect| is a legacy version
% of |\childdocforward| and |\childdocforwardprefix|:
%    \begin{macrocode}
\newcommand{\childdocredirect}[2][]
{
  \begingroup
    \if?#1?
      \def\childdoctmp{\childdocforward{#2}}
    \else
      \def\childdoctmp{\childdocforwardprefix{#1}{#2}}
    \fi
    \expandafter
  \endgroup
  \childdoctmp
}
%    \end{macrocode}

%\iffalse
%</package>
%\fi
%
\endinput

\childdocof{cdocsamp}
%    \end{macrocode}

%\iffalse
%</samplechap1|samplechap2>
%\fi
%
%\iffalse
%<*samplechap1>
%\fi
% Some text for chapter 1:
%    \begin{macrocode}
\section{one}
some text in chapter one
%    \end{macrocode}

%\iffalse
%</samplechap1>
%\fi
% Some text for chapter 2:
%\iffalse
%<*samplechap2>
%\fi
%    \begin{macrocode}
\section{two}
more text in chapter two
%    \end{macrocode}

%\iffalse
%</samplechap2>
%\fi
%
% %%%%%%%%%%%%%%%%%%%%%%%%%%%%%%%%%%%%%%
% \paragraph{Part Include Files.}
%
% The include files are called |cdocspt3.tex| and |cdocspt4.tex|.
%
%\iffalse
%<*samplepart3|samplepart4>
%\fi

% Optional override for |\version| flag:
%    \begin{macrocode}
%%\providecommand{\version}{final}
%    \end{macrocode}

% Include the main document:
%    \begin{macrocode}
% \iffalse
%
% childdoc.dtx Copyright (C) 2017-2018 Niklas Beisert
%
% This work may be distributed and/or modified under the
% conditions of the LaTeX Project Public License, either version 1.3
% of this license or (at your option) any later version.
% The latest version of this license is in
%   http://www.latex-project.org/lppl.txt
% and version 1.3 or later is part of all distributions of LaTeX
% version 2005/12/01 or later.
%
% This work has the LPPL maintenance status `maintained'.
%
% The Current Maintainer of this work is Niklas Beisert.
%
% This work consists of the files childdoc.dtx and childdoc.ins
% and the derived files childdoc.def and cdocsamp.tex with
% cdocsch1.tex, cdocsch2.tex, cdocsdrf.tex, cdocsfn1.tex, cdocsfn2.tex.
%
%<package>\ifdefined\childdocmain\endinput\fi
%<package>\ProvidesFile{childdoc.def}[2018/12/30 v2.0 child document driver]
%<samplemain>\ProvidesFile{cdocsamp.tex}[2018/12/30 v2.0 sample for childdoc]
%<*driver>
%\ProvidesFile{childdoc.drv}[2018/12/30 v2.0 childdoc reference manual file]
\PassOptionsToClass{10pt,a4paper}{article}
\documentclass{ltxdoc}

\usepackage[margin=35mm]{geometry}
\usepackage{hyperref}
\usepackage{hyperxmp}
\usepackage[usenames]{color}

\hypersetup{colorlinks=true}
\hypersetup{pdfstartview=FitH}
\hypersetup{pdfpagemode=UseNone}
\hypersetup{pdfsource={}}
\hypersetup{pdflang={en-UK}}
\hypersetup{pdfcopyright={Copyright 2017-2018 Niklas Beisert.
  This work may be distributed and/or modified under the
  conditions of the LaTeX Project Public License, either version 1.3
  of this license or (at your option) any later version.}}
\hypersetup{pdflicenseurl={http://www.latex-project.org/lppl.txt}}
\hypersetup{pdfcontactaddress={ETH Zurich, ITP, HIT K,
  Wolfgang-Pauli-Strasse 27}}
\hypersetup{pdfcontactpostcode={8093}}
\hypersetup{pdfcontactcity={Zurich}}
\hypersetup{pdfcontactcountry={Switzerland}}
\hypersetup{pdfcontactemail={nbeisert@itp.phys.ethz.ch}}
\hypersetup{pdfcontacturl={http://people.phys.ethz.ch/\xmptilde nbeisert/}}

\newcommand{\secref}[1]{\hyperref[#1]{section \ref*{#1}}}

\parskip1ex
\parindent0pt
\let\olditemize\itemize
\def\itemize{\olditemize\parskip0pt}

\begin{document}

\title{The \textsf{childdoc} Package}
\hypersetup{pdftitle={The childdoc Package}}
\author{Niklas Beisert\\[2ex]
  Institut f\"ur Theoretische Physik\\
  Eidgen\"ossische Technische Hochschule Z\"urich\\
  Wolfgang-Pauli-Strasse 27, 8093 Z\"urich, Switzerland\\[1ex]
  \href{mailto:nbeisert@itp.phys.ethz.ch}
  {\texttt{nbeisert@itp.phys.ethz.ch}}}
\hypersetup{pdfauthor={Niklas Beisert}}
\hypersetup{pdfsubject={Manual for the LaTeX2e Package childdoc}}
\date{30 December 2018, \textsf{v2.0}}
\maketitle

\begin{abstract}\noindent
\textsf{childdoc} is a \LaTeXe{} package
that enables the direct compilation
of document sections included by |\include|
to individual files.
\end{abstract}

\begingroup
\parskip0ex
\tableofcontents
\endgroup

%%%%%%%%%%%%%%%%%%%%%%%%%%%%%%%%%%%%%%%%%%%%%%%%%%%%%%%%%%%%%%%%%%%%%%%%%%%%%%%%
%%%%%%%%%%%%%%%%%%%%%%%%%%%%%%%%%%%%%%%%%%%%%%%%%%%%%%%%%%%%%%%%%%%%%%%%%%%%%%%%
\section{Introduction}

\LaTeX{} provides a mechanism to structure a large document (such as a book)
into a main file and several child files (containing the chapters)
using the |\include| command.
This mechanism is beneficial for documents
which span hundreds of pages in order to
make the source file(s) more manageable.
Moreover, compilation can be restricted to
selected child files by means of the |\includeonly| command.
The latter feature can be used to reduce the compilation time while editing
(this was significantly more useful in the earlier days of \LaTeX{})
or to generate a smaller document which is easier to navigate.
Another application of |\includeonly| is to generate
documents consisting of selected parts of the complete document.

However, there are a few drawbacks of the plain |\include| mechanism:
\begin{itemize}
\item
The child files cannot be compiled on their own,
they can only be compiled via the main file.
A naive editing environment
(such as a text editor with an option
to have the current file processed by \LaTeX)
may require one to switch to the main file before compiling;
attempting to compile the child file produces errors.
\item
The main file must be modified (each time)
to adjust the |\includeonly| command
to the present needs. This easily leaves the main file in a messy state.
\item
The generated document will always carry the filename
of the main document. This is inconvenient if
several child files are to be compiled and
to be kept for distribution.
\end{itemize}

The present package provides a simple interface
to make child files individually compilable by \LaTeX{}.
Compiling a child file then has the same effect as compiling
the main file with an |\includeonly| command
to select the appropriate child.
Moreover the generated document will carry the name of the child
rather than the main file.
This resolves all three above issues.

This feature is meant to make the editing of books,
thesis documents and lecture notes somewhat more convenient.
However, the package can also be used efficiently for
composing a series of documents (such as exercise sheets)
which are typically distributed individually.
It then assists the author in generating the individual documents
(potentially in different versions)
as well as a document containing the collected series.
Another application is in developing style files
or other kinds of included material
where compilation of the style file could redirect
to a sample or test file.

%%%%%%%%%%%%%%%%%%%%%%%%%%%%%%%%%%%%%%%%%%%%%%%%%%%%%%%%%%%%%%%%%%%%%%%%%%%%%%%%
%%%%%%%%%%%%%%%%%%%%%%%%%%%%%%%%%%%%%%%%%%%%%%%%%%%%%%%%%%%%%%%%%%%%%%%%%%%%%%%%
\section{Usage}

First of all, the package \textsf{childdoc} is \emph{not} a standard
\LaTeXe{} |.sty| style file! Therefore it needs to be invoked in
a non-standard way.

%%%%%%%%%%%%%%%%%%%%%%%%%%%%%%%%%%%%%%%%%%%%%%%%%%%%%%%%%%%%%%%%%%%%%%%%%%%%%%%%
\subsection{Included Files}
\label{sec:include}

%%%%%%%%%%%%%%%%%%%%%%%%%%%%%%%%%%%%%%%%
\DescribeMacro{\childdocmain}
To use the package, add the commands
\begin{center}
\begin{tabular}{l}
|\input{childdoc.def}|\\
|\childdocmain{}|\\
\end{tabular}
\end{center}
at the very top of the main \LaTeX{} file,
in particular \emph{before} the |\documentclass| statement!
The argument of |\childdocmain| should be left empty
(but it must be present).

%%%%%%%%%%%%%%%%%%%%%%%%%%%%%%%%%%%%%%%%
\DescribeMacro{\childdocof}
Furthermore, add the commands
\begin{center}
\begin{tabular}{l}
|\input{childdoc.def}|\\
|\childdocof{|\textit{main}|}|\\
\end{tabular}
\end{center}
at the top of every child file \textit{child}
which is included by |\include{|\textit{child}|}|
from within the main file
(or at least for those files to be compiled individually).
The argument \textit{main} must be the filename of the main file.

There are a couple of
considerations in setting up the main and child documents:

%%%%%%%%%%%%%%%%%%%%%%%%%%%%%%%%%%%%%%%%
\paragraph{Restrictions.}

Please note the following restrictions:
\begin{itemize}
\item
|\childdocmain| must be called with one argument \textit{main}
to ensure compatibility with earlier version of the package.
It must either be empty (|\childdocmain{}|)
or precisely match the filename of the main file in which it is specified.
See \secref{sec:detection} for further information.
\item
The filename \textit{main} must be specified without the |.tex| extension.
\item
The filename \textit{main} is case sensitive
(even in case-insensitive file systems)
due to internal string comparison.
\item
The argument \textit{main} should be fully expanded, it cannot be a macro.
\item
Subdirectories and special characters should be avoided in filenames.
\item
The command |\childdocmain{|\textit{main}|}| must be followed by a whitespace.
It should not be followed immediately by another command
or by a comment mark `|%|'.
This is because the \TeX{} parser reads the token immediately following
the argument of |\childdocmain| and puts it
at the beginning of every child section;
however, a white\-space is ignored.
\end{itemize}

%%%%%%%%%%%%%%%%%%%%%%%%%%%%%%%%%%%%%%%%
\paragraph{Content of Main File.}

It is advisable to place all content in the child files included by |\include|.
Any output contained in the main file will appear in all child documents
unless suppressed manually;
it cannot be suppressed automatically by the |\includeonly| directive
and thus should normally be avoided.
A method to include some content in the main file
by means of conditional processing is described in \secref{sec:conditional}.

%%%%%%%%%%%%%%%%%%%%%%%%%%%%%%%%%%%%%%%%
\paragraph{Page Numbering.}

When only a part of the document is compiled,
the appropriate numbering of pages
(as well as other status parameters)
is determined from the |.aux| files.
The latter contain information from previous passes.
However this information needs to propagate through
all intermediate child documents.
Therefore the page numbering in child documents may well
be inconsistent until the complete document is compiled at least once.

A useful (if unconventional) way to always ensure a consistent
page numbering is to restart the numbering in each child document
and denote the pages by `\textit{child}|.|\textit{page}'
where \textit{child} represents the chapter/section number of the child file.
This can be achieved by the command
|\numberwithin{page}{|\textit{child}|}|
of the \textsf{amsmath} package
where \textit{child} can be |chapter| or |section|
depending on the chosen structuring.
Alternatively, one can modify the macro |\thepage| appropriately
and reset the counter |page| at the start of each child file.

%%%%%%%%%%%%%%%%%%%%%%%%%%%%%%%%%%%%%%%%%%%%%%%%%%%%%%%%%%%%%%%%%%%%%%%%%%%%%%%%
\subsection{Conditional Processing}
\label{sec:conditional}

The package provides a mechanism to compile different versions
of a document. To customise the versions further some conditional processing
can come in handy to distinguish which version is being compiled.
The package provides two macros to describe the compilation context:

%%%%%%%%%%%%%%%%%%%%%%%%%%%%%%%%%%%%%%%%
\DescribeMacro{\ifchilddoc}
The conditional |\ifchilddoc| distinguishes between the compilation of
child documents and the main document:
%
\begin{center}
|\ifchilddoc |\textit{child-code}| |[|\||else |\textit{main-code}]| \||fi|
\end{center}

%%%%%%%%%%%%%%%%%%%%%%%%%%%%%%%%%%%%%%%%
\DescribeMacro{\childdocname}
\DescribeMacro{\childdocjob}
The macro |\childdocname| contains the filename (without extension)
of the main or child file being processed.
Note that |\childdocjob| will always contain the name of the main file.

%%%%%%%%%%%%%%%%%%%%%%%%%%%%%%%%%%%%%%%%
\paragraph{Title Page.}

Conditional processing can be used to include a title or banner page
in the main document when proper precautions are taken.
Importantly, the code in the main file should ensure that the page counter
(as well as other status parameters which are stored in the |.aux| files)
takes the same value after the conditional processing.
Otherwise the page numbers may take divergent values
depending on which part is compiled.

For example, a title page could be declared by:
%
\begin{center}
\begin{tabular}{l}
|\ifchilddoc\||else|\\
|\addtocounter{page}{-1}|\\
\textit{code for title page}\\
|\newpage|\\
|\||fi|
\end{tabular}
\end{center}
%
A banner page for the child documents can be generated by:
%
\begin{center}
\begin{tabular}{l}
|\ifchilddoc|\\
|\addtocounter{page}{-1}|\\
\textit{code for banner page}\\
|\newpage|\\
|\||fi|
\end{tabular}
\end{center}
%
Here one could write a message such as:
\begin{center}
|This is the part \childdocname{} of \childdocjob{}.|
\end{center}

%%%%%%%%%%%%%%%%%%%%%%%%%%%%%%%%%%%%%%%%%%%%%%%%%%%%%%%%%%%%%%%%%%%%%%%%%%%%%%%%
\subsection{Flags}
\label{sec:flags}

The package makes it easy to generate different versions
of the main or child documents.
To this end compilation flags can be defined
and assigned different default values.
They will be particularly useful in conjunction
with the forwarding mechanism described in \secref{sec:forward}.

For example, it may be useful to have a flag |\version|
which can be set to |draft| or |final|.
The document source will contain some conditional code
depending on the value of |\version|.
Suppose further, the flag should default to |final| for the main file
and to |draft| for child files
which is a natural assignment for editing the document.
This is achieved by placing the following code
in the preamble of the main document
(below the |\childdocmain| directive):
%
\begin{center}
\begin{tabular}{l}
|\ifchilddoc|\\
|\providecommand{\version}{draft}|\\
|\||else|\\
|\providecommand{\version}{final}|\\
|\||fi|
\end{tabular}
\end{center}
%
The definition by |\providecommand| makes sure
that previous definitions are not overwritten.
Further statements |\providecommand{\version}{...}|
can thus be added before the above code to override it.

For the main file, one might add a line
(between |\childdocmain| and the above block)
%
\begin{center}
|%\ifchilddoc\||else\providecommand{\version}{draft}\||fi|
\end{center}
%
which can be uncommented to produce a draft version.
Likewise one can add a line to the very top of a child file
(above the |\childdocof{|\textit{main}|}| directive)
%
\begin{center}
|%\providecommand{\version}{final}|
\end{center}
%
which can be uncommented to produce the final version of this child document.

%%%%%%%%%%%%%%%%%%%%%%%%%%%%%%%%%%%%%%%%%%%%%%%%%%%%%%%%%%%%%%%%%%%%%%%%%%%%%%%%
\subsection{Forwarding}
\label{sec:forward}

Different versions of the main or child documents
using compilation flags as described in \secref{sec:flags}
can be (permanently) stored in different files
for convenient compilation, viewing and distribution.
To this end, the package defines a command
to pass on compilation to a different file:

%%%%%%%%%%%%%%%%%%%%%%%%%%%%%%%%%%%%%%%%
\DescribeMacro{\childdocforward}
The command |\childdocforward| redirects processing to
another source file:
%
\begin{center}
\begin{tabular}{l}
|\input{childdoc.def}|\\
|\childdocforward[|\textit{main}|]{|\textit{dest}|}|\\
\end{tabular}
\end{center}
%
The argument \textit{dest} is the destination file
(without extension).
It should be the main file or one of the child files.
Note that further \textsf{childdoc} directives
such as |\childdocof| and |\childdocforward|
in the indicated file will be processed in this form.
The optional argument \textit{main}
passes on directly to the main file \textit{main}
while pretending to compile the child \textit{dest}.
This form behaves as if \textit{dest}
issues |\childdocof{|\textit{main}|}| right away,
and no further \textsf{childdoc} directives will be processed.

%%%%%%%%%%%%%%%%%%%%%%%%%%%%%%%%%%%%%%%%
\DescribeMacro{\...prefix}
In the alternative form |\childdocforwardprefix|,
%
\begin{center}
\begin{tabular}{l}
|\input{childdoc.def}|\\
|\childdocforwardprefix[|\textit{main}|]{|\textit{prefix}|}{|\textit{dest}|}|
\end{tabular}
\end{center}
%
the destination file is determined by a pattern
depending on the current file:
To make this work, the current file must be called
`{\textit{prefix}\hspace{0.2em}\textit{suffix}}'
with \textit{prefix} matching precisely the argument.
Processing is then passed on to the file
`{\textit{dest}\hspace{0.2em}\textit{suffix}}'.
Surely, the same effect is achieved by
directly specifying the
argument `{\textit{dest}\hspace{0.2em}\textit{suffix}}'
in the first form.
However, that requires to set up a different file
for each child. With the alternative form of the command
all these files can have exactly the same content
which simplifies setting them up and maintaining them.

For example, the following file |draft.tex|
with a compilation flag |\version| as described in \secref{sec:flags}
compiles the main document as a draft:
%
\begin{center}
\begin{tabular}{l}
|\def\version{draft}|\\
|\input{childdoc.def}|\\
|\childdocforward{|\textit{main}|}|
\end{tabular}
\end{center}
%
Likewise, the following files |final|\textit{nn}|.tex|
compile the final version of the child document
|child|\textit{nn}|.tex|:
%
\begin{center}
\begin{tabular}{l}
|\def\version{final}|\\
|\input{childdoc.def}|\\
|\childdocforwardprefix{final}{child}|
\end{tabular}
\end{center}
%

Note that when several versions of a main file and/or of each child file
are to be generated, it may be convenient to set up a |Makefile| or
shell script to automatise the process.

%%%%%%%%%%%%%%%%%%%%%%%%%%%%%%%%%%%%%%%%%%%%%%%%%%%%%%%%%%%%%%%%%%%%%%%%%%%%%%%%
\subsection{Command Line Processing}
\label{sec:commandline}

The effect of redirection files can also be achieved by invoking
the \LaTeX{} compiler with a more elaborate command line.
Most conveniently this should be done as part
of a shell script or a |Makefile|.

When using \textsf{childdoc} in the main file, the following
command lines effectively perform a redirection
(note that depending on the shell being used,
backslashes may have to be doubled: `|\|' $\to$ `|\\|'):
%
\begin{center}
|... -jobname "|\textit{target}|" |\\|"|[\textit{flags}]%
|\input{childdoc.def}\childdocforward[|\textit{main}|]{|\textit{dest}|}"|
\end{center}
%
Here \textit{target} is the name of the output file,
\textit{main} is the name of the main file
and \textit{dest} is the name of the main or child file to be processed
(all filenames without extensions).
The optional argument \textit{main} can be omitted
if \textit{main} matches \textit{dest}.
Optionally, compilation \textit{flags} can be defined via |\def| commands.
This command line makes the \TeX{} engine believe
it is compiling the file \textit{target}
whose content is specified as the latter parameter.
The provided code then forwards the processing to
\textit{main} or \textit{dest} as described in \secref{sec:forward}.

%%%%%%%%%%%%%%%%%%%%%%%%%%%%%%%%%%%%%%%%%%%%%%%%%%%%%%%%%%%%%%%%%%%%%%%%%%%%%%%%
\subsection{Include by Input}
\label{sec:input}

Including child documents by |\include| has some restrictions by design.
Most notably, the content of a child document always occupies
its own set of pages; pages cannot be shared between child documents.
Usually, this behaviour makes perfect sense
because each child document contain an essential part of the document.
However, in some situations it may be desirable to compose
a document from a collection of parts
without having mandatory page breaks between then.
For this case, the package
provides a mechanism to include parts
by |\input| which can also be processed individually.
However, by construction this mechanism
requires manual handling of the content to be output.

%%%%%%%%%%%%%%%%%%%%%%%%%%%%%%%%%%%%%%%%
\DescribeMacro{\ifchilddocmanual}
The main file should be prepared as usual, see \secref{sec:include}.
However, the document body must make a distinction
between processing of an individual part and of the main document, e.g.:
%
\begin{center}
\begin{tabular}{l}
|\ifchilddocmanual|\\
|\input{\childdocname}|\\
|\||else|\\
\textit{document body with }|\input{|\textit{part}|}|\\
|\||fi|
\end{tabular}
\end{center}
%
The conditional |\ifchilddocmanual| is true whenever
a part to be included by |\input| is being compiled,
and the name of the part is stored in |\childdocname|.

%%%%%%%%%%%%%%%%%%%%%%%%%%%%%%%%%%%%%%%%
\DescribeMacro{\childdocby}
Each part to be included by |\input| should start with:
%
\begin{center}
\begin{tabular}{l}
|\input{childdoc.def}|\\
|\childdocby{|\textit{main}|}|\\
\end{tabular}
\end{center}
%
The directive |\childdocby| is similar to |\childdocof|
described in \secref{sec:include},
but the subsequent selection of content must be done manually.
To that end, both |\ifchilddoc| and |\ifchilddocmanual|
will be true upon processing of a part,
and the name of the part is stored in |\childdocname|.
Note that |\jobname| will be set to the filename of the current part
so that each part receives an individual |.aux| file
that does not interfere with the |.aux| file(s) of the main document.
This behaviour can be altered by the alternative form
|\childdocby[*]{|\textit{main}|}| (with a non-empty optional argument)
which uses the |.aux| file of the main document
by setting |\jobname| to \textit{main}.

%%%%%%%%%%%%%%%%%%%%%%%%%%%%%%%%%%%%%%%%%%%%%%%%%%%%%%%%%%%%%%%%%%%%%%%%%%%%%%%%
\subsection{Driver Development}
\label{sec:driver}

The \textsf{childdoc} mechanism can also be use for the development
of definition files such as \LaTeX{} styles or classes.
This case differs from the above setup with multiple parts
included by |\include| in that no |\includeonly| should be invoked.
This can be achieved by starting the include file
(before |\ProvidesPackage|) with:
%
\begin{center}
\begin{tabular}{l}
|\input{childdoc.def}|\\
|\childdocforward{|\textit{main}|}|\\
\end{tabular}
\end{center}
%
or alternatively with:
%
\begin{center}
\begin{tabular}{l}
|\input{childdoc.def}|\\
|\childdocby{|\textit{main}|}|\\
\end{tabular}
\end{center}
%
Both forms have slightly different effects as described above.
The main file is prepared as usual, see \secref{sec:include}.

%%%%%%%%%%%%%%%%%%%%%%%%%%%%%%%%%%%%%%%%%%%%%%%%%%%%%%%%%%%%%%%%%%%%%%%%%%%%%%%%
\subsection{Legacy Detection}
\label{sec:detection}

The directive |\childdocmain| in the main file can detect
whether the complete document or merely a child is to be compiled
even without using the directive |\childdocof|.
This method is deprecated because it is less robust
and there is no compelling reason to use it;
it is merely provided for backward compatibility
and it may be removed in future versions.

If the detection mechanism is to be used,
it is mandatory to correctly specify
the filename of the main file as the argument of |\childdocmain|:
%
\begin{center}
\begin{tabular}{l}
|\input{childdoc.def}|\\
|\childdocmain{|\textit{main}|}|\\
\end{tabular}
\end{center}
%
If |\jobname| does not match the argument \textit{main} of |\childdocmain|,
it is assumed that |\jobname| points to the child file to be compiled.
When using |\childdocmain| with the main file specified as argument,
it suffices to start a child file
with just |\input{|\textit{main}|}|
without loading of the package and using |\childdocof|.
If instead all processing is done
with the appropriate \textsf{childdoc} directives,
the argument of \textit{main} of |\childdocmain| can be empty.

An alternative version of the command line processing described
in \secref{sec:commandline} using the detection mechanism reads:
%
\begin{center}
|... -jobname "|\textit{target}|" "|[\textit{flags}]%
[|\def\jobname{|\textit{dest}|}|]|\input{|\textit{main}|}"|
\end{center}

%%%%%%%%%%%%%%%%%%%%%%%%%%%%%%%%%%%%%%%%%%%%%%%%%%%%%%%%%%%%%%%%%%%%%%%%%%%%%%%%
\subsection{Manual Code}
\label{sec:manual}

In case one cannot be certain whether the definitions file |childdoc.def|
is installed on the target \TeX{} distribution
and one prefers not to ship it,
it is conceivable to paste a few relevant commands into the sources.

To that end, drop all statements |\input{childdoc.def}|
and perform the replacements as outlined below.
Instead of |\childdocmain{|\textit{main}|}| add the following code
to the top of the main file:
%
\begin{center}
\begin{tabular}{l}
|\||ifdefined\childdocname\endinput\||fi\newif\ifchilddoc|\\
|\edef\childdocname{\scantokens\expandafter{\jobname\noexpand}}|\\
|\def\childdocmain{|\textit{main}|}\||ifx\childdocmain\childdocname\||else|\\
|\childdoctrue\includeonly{\childdocname}\let\jobname\childdocmain\||fi|\\
\end{tabular}
\end{center}
%
Instead of |\childdocof{|\textit{main}|}| just include the main file
at the top of each child file:
%
\begin{center}
|\input{|\textit{main}|}|
\end{center}
%
A simple redirection |\childdocforward{|\textit{dest}|}| is achieved by:
%
\begin{center}
|\def\jobname{|\textit{dest}|}\input{\jobname}|
\end{center}
%
The redirection with prefix
|\childdocforwardprefix[|\textit{prefix}|]{|\textit{dest}|}|
is accomplished by:
%
\begin{center}
\begin{tabular}{l}
|{\edef\jobname{\scantokens\expandafter{\jobname\noexpand}}|\\
|\def\redirectjob |\textit{prefix}|#1~~~{\gdef\jobname{|\textit{dest}|#1}}|\\
|\expandafter\redirectjob\jobname~~~}\input{\jobname}|
\end{tabular}
\end{center}

In an alternative approach,
child documents can be compiled by a specific command line
without additional code or specific definitions:
%
\begin{center}
|... -jobname "|\textit{target}|" "|[\textit{flags}]%
|\includeonly{|\textit{dest}|}\input{|\textit{main}|}"|
\end{center}
%

%%%%%%%%%%%%%%%%%%%%%%%%%%%%%%%%%%%%%%%%%%%%%%%%%%%%%%%%%%%%%%%%%%%%%%%%%%%%%%%%
%%%%%%%%%%%%%%%%%%%%%%%%%%%%%%%%%%%%%%%%%%%%%%%%%%%%%%%%%%%%%%%%%%%%%%%%%%%%%%%%
\section{Information}

%%%%%%%%%%%%%%%%%%%%%%%%%%%%%%%%%%%%%%%%%%%%%%%%%%%%%%%%%%%%%%%%%%%%%%%%%%%%%%%%
\subsection{Copyright}

Copyright \copyright{} 2017--2018 Niklas Beisert

This work may be distributed and/or modified under the
conditions of the \LaTeX{} Project Public License, either version 1.3
of this license or (at your option) any later version.
The latest version of this license is in
  \url{http://www.latex-project.org/lppl.txt}
and version 1.3 or later is part of all distributions of \LaTeX{}
version 2005/12/01 or later.

This work has the LPPL maintenance status `maintained'.

The Current Maintainer of this work is Niklas Beisert.

This work consists of the files |README.txt|, |childdoc.ins| and |childdoc.dtx|
as well as the derived files |childdoc.def|, |cdocsamp.tex|
with |cdocsch1.tex|, |cdocsch2.tex|, |cdocspt3.tex|, |cdocspt4.tex|,
|cdocsdrf.tex|, |cdocsfn1.tex|, |cdocsfn2.tex|
as well as |childdoc.pdf|.

%%%%%%%%%%%%%%%%%%%%%%%%%%%%%%%%%%%%%%%%%%%%%%%%%%%%%%%%%%%%%%%%%%%%%%%%%%%%%%%%
\subsection{Files and Installation}

The package consists of the files:
%
\begin{center}
\begin{tabular}{ll}
    |README.txt|   & readme file \\
    |childdoc.ins| & installation file \\
    |childdoc.dtx| & source file \\
    |childdoc.def| & definition file \\
    |cdocsamp.tex| & sample main file \\
    |cdocsch1.tex| & sample include file \\
    |cdocsch2.tex| & sample include file \\
    |cdocspt3.tex| & sample part file \\
    |cdocspt4.tex| & sample part file \\
    |cdocsdrf.tex| & sample redirection file \\
    |cdocsfn1.tex| & sample redirection file \\
    |cdocsfn2.tex| & sample redirection file \\
    |childdoc.pdf| & manual
\end{tabular}
\end{center}
%
The distribution consists of the files
|README.txt|, |childdoc.ins| and |childdoc.dtx|.
%
\begin{itemize}
\item
Run (pdf)\LaTeX{} on |childdoc.dtx|
to compile the manual |childdoc.pdf| (this file).
\item
Run \LaTeX{} on |childdoc.ins| to create the definitions file |childdoc.def|
and the sample |cdocsamp.tex| with include files
|cdocsch1.tex|, |cdocsch2.tex|, |cdocspt3.tex|, |cdocspt4.tex|,
|cdocsdrf.tex|, |cdocsfn1.tex|, |cdocsfn2.tex|.
Then copy the file |childdoc.def| to an appropriate directory of your \LaTeX{}
distribution, e.g.\ \textit{texmf-root}|/tex/latex/childdoc|.
\end{itemize}

%%%%%%%%%%%%%%%%%%%%%%%%%%%%%%%%%%%%%%%%%%%%%%%%%%%%%%%%%%%%%%%%%%%%%%%%%%%%%%%%
\subsection{Related CTAN Packages}

There are several other packages which offer a similar functionality:
%
\begin{itemize}
\item
The packages
\href{http://ctan.org/pkg/docmute}{\textsf{docmute}},
\href{http://ctan.org/pkg/includex}{\textsf{includex}} and
\href{http://ctan.org/pkg/standalone}{\textsf{standalone}}
provide commands to include only the document body of
a child file thus allowing both files to be compiled individually.
\item
The packages \href{http://ctan.org/pkg/subdocs}{\textsf{subdocs}}
and \href{http://ctan.org/pkg/subfiles}{\textsf{subfiles}}
provide structures in which the main and child documents can be
encapsulated and allowing them to be compiled individually.
The inclusion mechanism is different from the conventional |\include|.
\item
The package \href{http://ctan.org/pkg/combine}{\textsf{combine}}
is an elaborate solution to combine several documents into one.
\end{itemize}
%
See also the CTAN topic \href{http://ctan.org/topic/subdocs}{\textsf{subdocs}}
for further related packages.
The present package differs from the above solutions in that
a document structure constructed with the conventional |\include| mechanism
just needs two extra commands at the top of every file
such that all constituent files can be compiled individually.

%%%%%%%%%%%%%%%%%%%%%%%%%%%%%%%%%%%%%%%%%%%%%%%%%%%%%%%%%%%%%%%%%%%%%%%%%%%%%%%%
%\subsection{Feature Suggestions}
%
%The following is a list of features which may be useful for future
%versions of this package:
%%
%\begin{itemize}
%\item
%\ldots
%\end{itemize}

%%%%%%%%%%%%%%%%%%%%%%%%%%%%%%%%%%%%%%%%%%%%%%%%%%%%%%%%%%%%%%%%%%%%%%%%%%%%%%%%
\subsection{Revision History}

%%%%%%%%%%%%%%%%%%%%%%%%%%%%%%%%%%%%%%%%
\paragraph{v2.0:} 2018/12/30

\begin{itemize}
\item
immediate forward processing
\item
added |\childdocby| mechanism
\item
manual restructured
\end{itemize}

%%%%%%%%%%%%%%%%%%%%%%%%%%%%%%%%%%%%%%%%
\paragraph{v1.6:} 2018/01/17

\begin{itemize}
\item
application for development of include files
\item
corrections to manual
\end{itemize}

%%%%%%%%%%%%%%%%%%%%%%%%%%%%%%%%%%%%%%%%
\paragraph{v1.5:} 2017/05/21

\begin{itemize}
\item
more complete structuring introduced
\item
|\childdocof| introduced
\item
|\childdoc| renamed to |\childdocmain|
\item
|\childredirect| renamed to |\childdocforward| and |\childdocforwardprefix|
and functionality expanded
\end{itemize}

%%%%%%%%%%%%%%%%%%%%%%%%%%%%%%%%%%%%%%%%
\paragraph{v1.0:} 2017/04/27

\begin{itemize}
\item
manual and install package
\item
first version published on CTAN
\end{itemize}

%%%%%%%%%%%%%%%%%%%%%%%%%%%%%%%%%%%%%%%%
\paragraph{v0.6:} 2017/04/26

\begin{itemize}
\item
redirection mechanism added
\end{itemize}

%%%%%%%%%%%%%%%%%%%%%%%%%%%%%%%%%%%%%%%%
\paragraph{v0.5:} 2017/04/26

\begin{itemize}
\item
functionality in definition file
\end{itemize}


%%%%%%%%%%%%%%%%%%%%%%%%%%%%%%%%%%%%%%%%%%%%%%%%%%%%%%%%%%%%%%%%%%%%%%%%%%%%%%%%
%%%%%%%%%%%%%%%%%%%%%%%%%%%%%%%%%%%%%%%%%%%%%%%%%%%%%%%%%%%%%%%%%%%%%%%%%%%%%%%%
%%%%%%%%%%%%%%%%%%%%%%%%%%%%%%%%%%%%%%%%%%%%%%%%%%%%%%%%%%%%%%%%%%%%%%%%%%%%%%%%
\appendix

\settowidth\MacroIndent{\rmfamily\scriptsize 000\ }

 \DocInput{childdoc.dtx}

\end{document}
%</driver>
% \fi
%
% %%%%%%%%%%%%%%%%%%%%%%%%%%%%%%%%%%%%%%%%%%%%%%%%%%%%%%%%%%%%%%%%%%%%%%%%%%%%%%
% %%%%%%%%%%%%%%%%%%%%%%%%%%%%%%%%%%%%%%%%%%%%%%%%%%%%%%%%%%%%%%%%%%%%%%%%%%%%%%
% \section{Sample}
%\iffalse
%<*samplemain>
%\fi
%
% The following presents a sample document
% with two chapters, two parts, a title page,
% a compile flag as well as three forwarding files to set the flag.
% It consists of eight |.tex| files:
% \begin{center}
% \begin{tabular}{ll}
% |cdocsamp.tex|&main file\\
% |cdocsch1.tex|&include file for chapter 1\\
% |cdocsch2.tex|&include file for chapter 2\\
% |cdocspt3.tex|&include file for part 3\\
% |cdocspt4.tex|&include file for part 4\\
% |cdocsdrf.tex|&forwarding file for main file in draft mode\\
% |cdocsfi1.tex|&forwarding file for final version of chapter 1\\
% |cdocsfi2.tex|&forwarding file for final version of chapter 2\\
% \end{tabular}
% \end{center}
% Each of the eight files can be compiled directly by the \LaTeX{} compiler.
%
% %%%%%%%%%%%%%%%%%%%%%%%%%%%%%%%%%%%%%%
% \paragraph{Main File.}
%
% The main file is called |cdocsamp.tex|.
%
% Load the \textsf{childdoc} definitions and
% declare the filename for the main document:
%    \begin{macrocode}
\input{childdoc.def}
\childdocmain{}
%    \end{macrocode}

% Optional override for |\version| flag:
%    \begin{macrocode}
%%\ifchilddoc\else\providecommand{\version}{draft}\fi
%    \end{macrocode}

% Define the default values for the |\version| flag
% (|final| for the main file and |draft| for childs):
%    \begin{macrocode}
\ifchilddoc
\providecommand{\version}{draft}
\else
\providecommand{\version}{final}
\fi
%    \end{macrocode}

% Load the standard document class:
%    \begin{macrocode}
\documentclass[12pt]{article}
%    \end{macrocode}

% Start the document body:
%    \begin{macrocode}
\begin{document}
%    \end{macrocode}

% Declare a title page.
% Print title, part of document being processed and version flag:
%    \begin{macrocode}
\addtocounter{page}{-1}
\begin{center}
{\LARGE\bfseries{}childdoc example\par}
\vspace{1cm}
\ifchilddoc
\ifchilddocmanual part\else chapter\fi:
`\childdocname' of `\childdocjob'\par
\else
main document: `\childdocjob'\par
\fi
version: \version\par
\end{center}
\newpage
%    \end{macrocode}

% Manually include selected file,
% otherwise process as usual:
%    \begin{macrocode}
\ifchilddocmanual
\section*{part `\childdocname'}
\input{\childdocname}
\else
%    \end{macrocode}

% Include the two chapters:
%    \begin{macrocode}
\include{cdocsch1}
\include{cdocsch2}
%    \end{macrocode}

% Include the two parts unless only chapters should be displayed:
%    \begin{macrocode}
\ifchilddoc\else
\section{part three}
\input{cdocspt3}
\section{part four}
\input{cdocspt4}
\fi
%    \end{macrocode}

% Process as usual until here:
%    \begin{macrocode}
\fi
%    \end{macrocode}

% End of document body:
%    \begin{macrocode}
\end{document}
%    \end{macrocode}
%\iffalse
%</samplemain>
%\fi
%
% %%%%%%%%%%%%%%%%%%%%%%%%%%%%%%%%%%%%%%
% \paragraph{Chapter Include Files.}
%
% The include files are called |cdocsch1.tex| and |cdocsch2.tex|.
%
%\iffalse
%<*samplechap1|samplechap2>
%\fi

% Optional override for |\version| flag:
%    \begin{macrocode}
%%\providecommand{\version}{final}
%    \end{macrocode}

% Include the main document:
%    \begin{macrocode}
\input{childdoc.def}
\childdocof{cdocsamp}
%    \end{macrocode}

%\iffalse
%</samplechap1|samplechap2>
%\fi
%
%\iffalse
%<*samplechap1>
%\fi
% Some text for chapter 1:
%    \begin{macrocode}
\section{one}
some text in chapter one
%    \end{macrocode}

%\iffalse
%</samplechap1>
%\fi
% Some text for chapter 2:
%\iffalse
%<*samplechap2>
%\fi
%    \begin{macrocode}
\section{two}
more text in chapter two
%    \end{macrocode}

%\iffalse
%</samplechap2>
%\fi
%
% %%%%%%%%%%%%%%%%%%%%%%%%%%%%%%%%%%%%%%
% \paragraph{Part Include Files.}
%
% The include files are called |cdocspt3.tex| and |cdocspt4.tex|.
%
%\iffalse
%<*samplepart3|samplepart4>
%\fi

% Optional override for |\version| flag:
%    \begin{macrocode}
%%\providecommand{\version}{final}
%    \end{macrocode}

% Include the main document:
%    \begin{macrocode}
\input{childdoc.def}
\childdocby{cdocsamp}
%    \end{macrocode}

%\iffalse
%</samplepart3|samplepart4>
%\fi
%
%\iffalse
%<*samplepart3>
%\fi
% Some text for part 3:
%    \begin{macrocode}
some text in part three
%    \end{macrocode}

%\iffalse
%</samplepart3>
%\fi
% Some text for part 4:
%\iffalse
%<*samplepart4>
%\fi
%    \begin{macrocode}
more text in part four
%    \end{macrocode}

%\iffalse
%</samplepart4>
%\fi
%
% %%%%%%%%%%%%%%%%%%%%%%%%%%%%%%%%%%%%%%
% \paragraph{Forwarding for a Complete Draft.}
%
% The following forwarding file |cdocsdrf.tex|
% compiles the main document in draft mode:
%\iffalse
%<*sampledraft>
%\fi
%    \begin{macrocode}
\def\version{draft}
\input{childdoc.def}
\childdocforward{cdocsamp}
%    \end{macrocode}

%\iffalse
%</sampledraft>
%\fi
%
% %%%%%%%%%%%%%%%%%%%%%%%%%%%%%%%%%%%%%%
% \paragraph{Forwarding for Final Version of the Chapters.}
%
% The following forwarding files |cdocsfn1.tex| and |cdocsfn2.tex|
% (with identical content)
% compile the final versions of the child documents
% |cdocsch1.tex| and |cdocsch2.tex|, respectively:
%\iffalse
%<*samplefinal>
%\fi
%    \begin{macrocode}
\def\version{final}
\input{childdoc.def}
\childdocforwardprefix[cdocsamp]{cdocsfn}{cdocsch}
%    \end{macrocode}

%\iffalse
%</samplefinal>
%\fi
%
% %%%%%%%%%%%%%%%%%%%%%%%%%%%%%%%%%%%%%%
% \paragraph{Command Line Processing.}
%
% The following three command lines generate the output files
% |cdocscld|, |cdocscl1| and |cdocscl2|
% which should be identical to
% |cdocsdrf|, |cdocsch1| and |cdocsfn2|, respectively:
% \begin{center}
% \begin{tabular}{l}
% |latex -jobname cdocscld \|\\
% |  "\def\version{draft}\input{childdoc.def}\childdocforward{cdocsamp}"|\\
% |latex -jobname cdocscl1 \|\\
% |  "\input{childdoc.def}\childdocforward[cdocsamp]{cdocsch1}"|\\
% |latex -jobname cdocscl2 \|\\
% |  "\def\version{final}\input{childdoc.def}\childdocforward{cdocsch2}"|
% \end{tabular}
% \end{center}
% Note that the trailing backslash on each first line
% merely continues the input to the second line
% (for convenient cut ant paste).
% Furthermore, the command |latex| can be replaced by any
% of its alternative versions such as |pdflatex|.
%
% %%%%%%%%%%%%%%%%%%%%%%%%%%%%%%%%%%%%%%%%%%%%%%%%%%%%%%%%%%%%%%%%%%%%%%%%%%%%%%
% %%%%%%%%%%%%%%%%%%%%%%%%%%%%%%%%%%%%%%%%%%%%%%%%%%%%%%%%%%%%%%%%%%%%%%%%%%%%%%
% \section{Implementation}
%\iffalse
%<*package>
%\fi
%
% This section describes the definitions file |childdoc.def|.

% The definitions cannot be loaded using |\usepackage| or |\RequirePackage|
% which has a mechanism to prevent loading a style file more than once.
% When loading the definitions by means of |\input|
% multiple instances have to be prevented manually:
%\iffalse
%This code needs to be before the `\ProvidesFile' directive
%which is defined at the beginning of this file.
%Therefore it is also placed there and commented out here.
%</package>
%<*discard>
%\fi
%    \begin{macrocode}
\ifdefined\childdocmain\endinput\fi
%    \end{macrocode}
%\iffalse
%</discard>
%<*package>
%\fi
%
% \macro{\ifchilddoc}
% \macro{\ifchilddocmanual}
% The conditional |\ifchilddoc| tells whether a
% child (true) or main (false) document is being compiled.
% The conditional |\ifchilddocmanual| tells whether
% the |\includeonly| mechanism is used (false) or
% the selection of child files must be performed manually (true).
% The definitions initialise to false:
%    \begin{macrocode}
\newif\ifchilddoc
\newif\ifchilddocmanual
%    \end{macrocode}

% \macro{\childdocname}
% \macro{\childdocjob}
% The macro |\childdocname| stores the name of the main document
% to be compiled. The macro |\childdocjob| stores the name of
% the document on which the \LaTeX{} compiler was originally invoked.
% The content of |\jobname| cannot be compared
% to filenames specified in the source due to different catcodes.
% The following code rescans |\jobname|, stores the result
% in |\childdocname| and saves a copy in |\childdocjob|:
%    \begin{macrocode}
\edef\childdocname{\scantokens\expandafter{\jobname\noexpand}}
\let\childdocjob\childdocname
%    \end{macrocode}

% \macro{\childdocdisable}
% The macro |\childdocdisable| prevents the main file
% from being processed more than once.
% At this stage, the main document command |\childdocmain|
% is assumed to be called once again where it should do nothing.
% Any subsequent call to it should prevent
% a secondary processing of the main document
% It overwrites the forwarding commands
% |\childdocof| and |\childdocforward|
% with empty macros to prevent further inclusions of the main document:
%    \begin{macrocode}
\newcommand{\childdocdisable}
{
  \renewcommand{\childdocmain}[1]{\renewcommand{\childdocmain}[1]{\endinput}}
  \renewcommand{\childdocof}[1]{}
  \renewcommand{\childdocby}[2][]{}
  \renewcommand{\childdocforward}[2][]{}
  \renewcommand{\childdocdisable}{}
}
%    \end{macrocode}

% \macro{\childdocmain}
% The macro |\childdocmain| is to be called at the top of the main file
% with nothing or the main filename (without extension) as argument.
% First, it breaks loops.
% If the argument is not empty and does not match |\childdocname|
% (which is set by the first inclusion of |childdoc.def|),
% |\ifchilddoc| is set to true, |\includeonly| is applied to the child file
% and |\jobname| is set to the main file
% (for proper handling of |.aux| files):
%    \begin{macrocode}
\newcommand{\childdocmain}[1]
{
  \childdocdisable\childdocmain{}
  \if?#1?\else
    \begingroup
      \def\childdoctmp{#1}
      \ifx\childdoctmp\childdocname
        \def\childdoctmp{}
      \else
        \def\childdoctmp
        {
          \childdoctrue
          \includeonly{\childdocname}
          \def\childdocjob{#1}
          \def\jobname{#1}
        }
      \fi
      \expandafter
    \endgroup
    \childdoctmp
  \fi
}
%    \end{macrocode}

% \macro{\childdocof}
% The command |\childdocof| redirects
% compilation to the main file |#1|.
%    \begin{macrocode}
\newcommand{\childdocof}[1]
{
  \childdocdisable
  \childdoctrue
  \includeonly{\childdocname}
  \def\jobname{#1}
  \def\childdocjob{#1}
  \input{#1}
}
%    \end{macrocode}

% \macro{\childdocby}
% The command |\childdocby| ....
%    \begin{macrocode}
\newcommand{\childdocby}[2][]
{
  \childdocdisable
  \childdoctrue
  \childdocmanualtrue
  \if?#1?\else
    \def\jobname{#2}
  \fi
  \def\childdocjob{#2}
  \input{#2}
  \endinput
}
%    \end{macrocode}

% \macro{\childdocforward}
% The command |\childdocforward| redirects
% compilation to the main file or
% (if the optional argument is given) a child file.
% Parameters are set as if the main file
% or a child file starting with |\childdocof| was compiled.
% Then compilation is handed over to the main file:
%    \begin{macrocode}
\newcommand{\childdocforward}[2][]
{
  \begingroup
    \if?#1?
      \def\childdoctmp
      {
        \def\childdocname{#2}
        \def\childdocjob{#2}
        \def\jobname{#2}
        \input{#2}
        \endinput
      }
    \else
      \def\childdoctmp
      {
        \childdocdisable
        \def\childdocname{#2}
        \childdoctrue
        \includeonly{#2}
        \def\childdocjob{#1}
        \def\jobname{#1}
        \input{#1}
        \endinput
      }
    \fi
    \expandafter
  \endgroup
  \childdoctmp
}
%    \end{macrocode}

% \macro{\childdocforwardprefix}
% The command |\childdocforwardprefix| redirects
% compilation to the main or a child file by means of a pattern.
% The prefix |#1| in the current filename is replaced by |#2|
% and the suffix of the current filename is kept
% (it is assumed that the filename does not contain the substring `|~~~|'
% which is used as a delimiter).
% Compilation is handed over to the new file by |\childdocforward|:
%    \begin{macrocode}
\newcommand{\childdocforwardprefix}[3][]
{
  \begingroup
    \def\childdocextract #2##1~~~{\def\childdoctmp{\childdocforward[#1]{#3##1}}}
    \expandafter\childdocextract\childdocname~~~
    \expandafter
  \endgroup
  \childdoctmp
}
%    \end{macrocode}

% \macro{\childdoc}
% The deprecated macro |\childdoc| is a legacy version of |\childdocmain|:
%    \begin{macrocode}
\newcommand{\childdoc}{\childdocmain}
%    \end{macrocode}

% \macro{\childdocredirect}
% The deprecated macro |\childdocredirect| is a legacy version
% of |\childdocforward| and |\childdocforwardprefix|:
%    \begin{macrocode}
\newcommand{\childdocredirect}[2][]
{
  \begingroup
    \if?#1?
      \def\childdoctmp{\childdocforward{#2}}
    \else
      \def\childdoctmp{\childdocforwardprefix{#1}{#2}}
    \fi
    \expandafter
  \endgroup
  \childdoctmp
}
%    \end{macrocode}

%\iffalse
%</package>
%\fi
%
\endinput

\childdocby{cdocsamp}
%    \end{macrocode}

%\iffalse
%</samplepart3|samplepart4>
%\fi
%
%\iffalse
%<*samplepart3>
%\fi
% Some text for part 3:
%    \begin{macrocode}
some text in part three
%    \end{macrocode}

%\iffalse
%</samplepart3>
%\fi
% Some text for part 4:
%\iffalse
%<*samplepart4>
%\fi
%    \begin{macrocode}
more text in part four
%    \end{macrocode}

%\iffalse
%</samplepart4>
%\fi
%
% %%%%%%%%%%%%%%%%%%%%%%%%%%%%%%%%%%%%%%
% \paragraph{Forwarding for a Complete Draft.}
%
% The following forwarding file |cdocsdrf.tex|
% compiles the main document in draft mode:
%\iffalse
%<*sampledraft>
%\fi
%    \begin{macrocode}
\def\version{draft}
% \iffalse
%
% childdoc.dtx Copyright (C) 2017-2018 Niklas Beisert
%
% This work may be distributed and/or modified under the
% conditions of the LaTeX Project Public License, either version 1.3
% of this license or (at your option) any later version.
% The latest version of this license is in
%   http://www.latex-project.org/lppl.txt
% and version 1.3 or later is part of all distributions of LaTeX
% version 2005/12/01 or later.
%
% This work has the LPPL maintenance status `maintained'.
%
% The Current Maintainer of this work is Niklas Beisert.
%
% This work consists of the files childdoc.dtx and childdoc.ins
% and the derived files childdoc.def and cdocsamp.tex with
% cdocsch1.tex, cdocsch2.tex, cdocsdrf.tex, cdocsfn1.tex, cdocsfn2.tex.
%
%<package>\ifdefined\childdocmain\endinput\fi
%<package>\ProvidesFile{childdoc.def}[2018/12/30 v2.0 child document driver]
%<samplemain>\ProvidesFile{cdocsamp.tex}[2018/12/30 v2.0 sample for childdoc]
%<*driver>
%\ProvidesFile{childdoc.drv}[2018/12/30 v2.0 childdoc reference manual file]
\PassOptionsToClass{10pt,a4paper}{article}
\documentclass{ltxdoc}

\usepackage[margin=35mm]{geometry}
\usepackage{hyperref}
\usepackage{hyperxmp}
\usepackage[usenames]{color}

\hypersetup{colorlinks=true}
\hypersetup{pdfstartview=FitH}
\hypersetup{pdfpagemode=UseNone}
\hypersetup{pdfsource={}}
\hypersetup{pdflang={en-UK}}
\hypersetup{pdfcopyright={Copyright 2017-2018 Niklas Beisert.
  This work may be distributed and/or modified under the
  conditions of the LaTeX Project Public License, either version 1.3
  of this license or (at your option) any later version.}}
\hypersetup{pdflicenseurl={http://www.latex-project.org/lppl.txt}}
\hypersetup{pdfcontactaddress={ETH Zurich, ITP, HIT K,
  Wolfgang-Pauli-Strasse 27}}
\hypersetup{pdfcontactpostcode={8093}}
\hypersetup{pdfcontactcity={Zurich}}
\hypersetup{pdfcontactcountry={Switzerland}}
\hypersetup{pdfcontactemail={nbeisert@itp.phys.ethz.ch}}
\hypersetup{pdfcontacturl={http://people.phys.ethz.ch/\xmptilde nbeisert/}}

\newcommand{\secref}[1]{\hyperref[#1]{section \ref*{#1}}}

\parskip1ex
\parindent0pt
\let\olditemize\itemize
\def\itemize{\olditemize\parskip0pt}

\begin{document}

\title{The \textsf{childdoc} Package}
\hypersetup{pdftitle={The childdoc Package}}
\author{Niklas Beisert\\[2ex]
  Institut f\"ur Theoretische Physik\\
  Eidgen\"ossische Technische Hochschule Z\"urich\\
  Wolfgang-Pauli-Strasse 27, 8093 Z\"urich, Switzerland\\[1ex]
  \href{mailto:nbeisert@itp.phys.ethz.ch}
  {\texttt{nbeisert@itp.phys.ethz.ch}}}
\hypersetup{pdfauthor={Niklas Beisert}}
\hypersetup{pdfsubject={Manual for the LaTeX2e Package childdoc}}
\date{30 December 2018, \textsf{v2.0}}
\maketitle

\begin{abstract}\noindent
\textsf{childdoc} is a \LaTeXe{} package
that enables the direct compilation
of document sections included by |\include|
to individual files.
\end{abstract}

\begingroup
\parskip0ex
\tableofcontents
\endgroup

%%%%%%%%%%%%%%%%%%%%%%%%%%%%%%%%%%%%%%%%%%%%%%%%%%%%%%%%%%%%%%%%%%%%%%%%%%%%%%%%
%%%%%%%%%%%%%%%%%%%%%%%%%%%%%%%%%%%%%%%%%%%%%%%%%%%%%%%%%%%%%%%%%%%%%%%%%%%%%%%%
\section{Introduction}

\LaTeX{} provides a mechanism to structure a large document (such as a book)
into a main file and several child files (containing the chapters)
using the |\include| command.
This mechanism is beneficial for documents
which span hundreds of pages in order to
make the source file(s) more manageable.
Moreover, compilation can be restricted to
selected child files by means of the |\includeonly| command.
The latter feature can be used to reduce the compilation time while editing
(this was significantly more useful in the earlier days of \LaTeX{})
or to generate a smaller document which is easier to navigate.
Another application of |\includeonly| is to generate
documents consisting of selected parts of the complete document.

However, there are a few drawbacks of the plain |\include| mechanism:
\begin{itemize}
\item
The child files cannot be compiled on their own,
they can only be compiled via the main file.
A naive editing environment
(such as a text editor with an option
to have the current file processed by \LaTeX)
may require one to switch to the main file before compiling;
attempting to compile the child file produces errors.
\item
The main file must be modified (each time)
to adjust the |\includeonly| command
to the present needs. This easily leaves the main file in a messy state.
\item
The generated document will always carry the filename
of the main document. This is inconvenient if
several child files are to be compiled and
to be kept for distribution.
\end{itemize}

The present package provides a simple interface
to make child files individually compilable by \LaTeX{}.
Compiling a child file then has the same effect as compiling
the main file with an |\includeonly| command
to select the appropriate child.
Moreover the generated document will carry the name of the child
rather than the main file.
This resolves all three above issues.

This feature is meant to make the editing of books,
thesis documents and lecture notes somewhat more convenient.
However, the package can also be used efficiently for
composing a series of documents (such as exercise sheets)
which are typically distributed individually.
It then assists the author in generating the individual documents
(potentially in different versions)
as well as a document containing the collected series.
Another application is in developing style files
or other kinds of included material
where compilation of the style file could redirect
to a sample or test file.

%%%%%%%%%%%%%%%%%%%%%%%%%%%%%%%%%%%%%%%%%%%%%%%%%%%%%%%%%%%%%%%%%%%%%%%%%%%%%%%%
%%%%%%%%%%%%%%%%%%%%%%%%%%%%%%%%%%%%%%%%%%%%%%%%%%%%%%%%%%%%%%%%%%%%%%%%%%%%%%%%
\section{Usage}

First of all, the package \textsf{childdoc} is \emph{not} a standard
\LaTeXe{} |.sty| style file! Therefore it needs to be invoked in
a non-standard way.

%%%%%%%%%%%%%%%%%%%%%%%%%%%%%%%%%%%%%%%%%%%%%%%%%%%%%%%%%%%%%%%%%%%%%%%%%%%%%%%%
\subsection{Included Files}
\label{sec:include}

%%%%%%%%%%%%%%%%%%%%%%%%%%%%%%%%%%%%%%%%
\DescribeMacro{\childdocmain}
To use the package, add the commands
\begin{center}
\begin{tabular}{l}
|\input{childdoc.def}|\\
|\childdocmain{}|\\
\end{tabular}
\end{center}
at the very top of the main \LaTeX{} file,
in particular \emph{before} the |\documentclass| statement!
The argument of |\childdocmain| should be left empty
(but it must be present).

%%%%%%%%%%%%%%%%%%%%%%%%%%%%%%%%%%%%%%%%
\DescribeMacro{\childdocof}
Furthermore, add the commands
\begin{center}
\begin{tabular}{l}
|\input{childdoc.def}|\\
|\childdocof{|\textit{main}|}|\\
\end{tabular}
\end{center}
at the top of every child file \textit{child}
which is included by |\include{|\textit{child}|}|
from within the main file
(or at least for those files to be compiled individually).
The argument \textit{main} must be the filename of the main file.

There are a couple of
considerations in setting up the main and child documents:

%%%%%%%%%%%%%%%%%%%%%%%%%%%%%%%%%%%%%%%%
\paragraph{Restrictions.}

Please note the following restrictions:
\begin{itemize}
\item
|\childdocmain| must be called with one argument \textit{main}
to ensure compatibility with earlier version of the package.
It must either be empty (|\childdocmain{}|)
or precisely match the filename of the main file in which it is specified.
See \secref{sec:detection} for further information.
\item
The filename \textit{main} must be specified without the |.tex| extension.
\item
The filename \textit{main} is case sensitive
(even in case-insensitive file systems)
due to internal string comparison.
\item
The argument \textit{main} should be fully expanded, it cannot be a macro.
\item
Subdirectories and special characters should be avoided in filenames.
\item
The command |\childdocmain{|\textit{main}|}| must be followed by a whitespace.
It should not be followed immediately by another command
or by a comment mark `|%|'.
This is because the \TeX{} parser reads the token immediately following
the argument of |\childdocmain| and puts it
at the beginning of every child section;
however, a white\-space is ignored.
\end{itemize}

%%%%%%%%%%%%%%%%%%%%%%%%%%%%%%%%%%%%%%%%
\paragraph{Content of Main File.}

It is advisable to place all content in the child files included by |\include|.
Any output contained in the main file will appear in all child documents
unless suppressed manually;
it cannot be suppressed automatically by the |\includeonly| directive
and thus should normally be avoided.
A method to include some content in the main file
by means of conditional processing is described in \secref{sec:conditional}.

%%%%%%%%%%%%%%%%%%%%%%%%%%%%%%%%%%%%%%%%
\paragraph{Page Numbering.}

When only a part of the document is compiled,
the appropriate numbering of pages
(as well as other status parameters)
is determined from the |.aux| files.
The latter contain information from previous passes.
However this information needs to propagate through
all intermediate child documents.
Therefore the page numbering in child documents may well
be inconsistent until the complete document is compiled at least once.

A useful (if unconventional) way to always ensure a consistent
page numbering is to restart the numbering in each child document
and denote the pages by `\textit{child}|.|\textit{page}'
where \textit{child} represents the chapter/section number of the child file.
This can be achieved by the command
|\numberwithin{page}{|\textit{child}|}|
of the \textsf{amsmath} package
where \textit{child} can be |chapter| or |section|
depending on the chosen structuring.
Alternatively, one can modify the macro |\thepage| appropriately
and reset the counter |page| at the start of each child file.

%%%%%%%%%%%%%%%%%%%%%%%%%%%%%%%%%%%%%%%%%%%%%%%%%%%%%%%%%%%%%%%%%%%%%%%%%%%%%%%%
\subsection{Conditional Processing}
\label{sec:conditional}

The package provides a mechanism to compile different versions
of a document. To customise the versions further some conditional processing
can come in handy to distinguish which version is being compiled.
The package provides two macros to describe the compilation context:

%%%%%%%%%%%%%%%%%%%%%%%%%%%%%%%%%%%%%%%%
\DescribeMacro{\ifchilddoc}
The conditional |\ifchilddoc| distinguishes between the compilation of
child documents and the main document:
%
\begin{center}
|\ifchilddoc |\textit{child-code}| |[|\||else |\textit{main-code}]| \||fi|
\end{center}

%%%%%%%%%%%%%%%%%%%%%%%%%%%%%%%%%%%%%%%%
\DescribeMacro{\childdocname}
\DescribeMacro{\childdocjob}
The macro |\childdocname| contains the filename (without extension)
of the main or child file being processed.
Note that |\childdocjob| will always contain the name of the main file.

%%%%%%%%%%%%%%%%%%%%%%%%%%%%%%%%%%%%%%%%
\paragraph{Title Page.}

Conditional processing can be used to include a title or banner page
in the main document when proper precautions are taken.
Importantly, the code in the main file should ensure that the page counter
(as well as other status parameters which are stored in the |.aux| files)
takes the same value after the conditional processing.
Otherwise the page numbers may take divergent values
depending on which part is compiled.

For example, a title page could be declared by:
%
\begin{center}
\begin{tabular}{l}
|\ifchilddoc\||else|\\
|\addtocounter{page}{-1}|\\
\textit{code for title page}\\
|\newpage|\\
|\||fi|
\end{tabular}
\end{center}
%
A banner page for the child documents can be generated by:
%
\begin{center}
\begin{tabular}{l}
|\ifchilddoc|\\
|\addtocounter{page}{-1}|\\
\textit{code for banner page}\\
|\newpage|\\
|\||fi|
\end{tabular}
\end{center}
%
Here one could write a message such as:
\begin{center}
|This is the part \childdocname{} of \childdocjob{}.|
\end{center}

%%%%%%%%%%%%%%%%%%%%%%%%%%%%%%%%%%%%%%%%%%%%%%%%%%%%%%%%%%%%%%%%%%%%%%%%%%%%%%%%
\subsection{Flags}
\label{sec:flags}

The package makes it easy to generate different versions
of the main or child documents.
To this end compilation flags can be defined
and assigned different default values.
They will be particularly useful in conjunction
with the forwarding mechanism described in \secref{sec:forward}.

For example, it may be useful to have a flag |\version|
which can be set to |draft| or |final|.
The document source will contain some conditional code
depending on the value of |\version|.
Suppose further, the flag should default to |final| for the main file
and to |draft| for child files
which is a natural assignment for editing the document.
This is achieved by placing the following code
in the preamble of the main document
(below the |\childdocmain| directive):
%
\begin{center}
\begin{tabular}{l}
|\ifchilddoc|\\
|\providecommand{\version}{draft}|\\
|\||else|\\
|\providecommand{\version}{final}|\\
|\||fi|
\end{tabular}
\end{center}
%
The definition by |\providecommand| makes sure
that previous definitions are not overwritten.
Further statements |\providecommand{\version}{...}|
can thus be added before the above code to override it.

For the main file, one might add a line
(between |\childdocmain| and the above block)
%
\begin{center}
|%\ifchilddoc\||else\providecommand{\version}{draft}\||fi|
\end{center}
%
which can be uncommented to produce a draft version.
Likewise one can add a line to the very top of a child file
(above the |\childdocof{|\textit{main}|}| directive)
%
\begin{center}
|%\providecommand{\version}{final}|
\end{center}
%
which can be uncommented to produce the final version of this child document.

%%%%%%%%%%%%%%%%%%%%%%%%%%%%%%%%%%%%%%%%%%%%%%%%%%%%%%%%%%%%%%%%%%%%%%%%%%%%%%%%
\subsection{Forwarding}
\label{sec:forward}

Different versions of the main or child documents
using compilation flags as described in \secref{sec:flags}
can be (permanently) stored in different files
for convenient compilation, viewing and distribution.
To this end, the package defines a command
to pass on compilation to a different file:

%%%%%%%%%%%%%%%%%%%%%%%%%%%%%%%%%%%%%%%%
\DescribeMacro{\childdocforward}
The command |\childdocforward| redirects processing to
another source file:
%
\begin{center}
\begin{tabular}{l}
|\input{childdoc.def}|\\
|\childdocforward[|\textit{main}|]{|\textit{dest}|}|\\
\end{tabular}
\end{center}
%
The argument \textit{dest} is the destination file
(without extension).
It should be the main file or one of the child files.
Note that further \textsf{childdoc} directives
such as |\childdocof| and |\childdocforward|
in the indicated file will be processed in this form.
The optional argument \textit{main}
passes on directly to the main file \textit{main}
while pretending to compile the child \textit{dest}.
This form behaves as if \textit{dest}
issues |\childdocof{|\textit{main}|}| right away,
and no further \textsf{childdoc} directives will be processed.

%%%%%%%%%%%%%%%%%%%%%%%%%%%%%%%%%%%%%%%%
\DescribeMacro{\...prefix}
In the alternative form |\childdocforwardprefix|,
%
\begin{center}
\begin{tabular}{l}
|\input{childdoc.def}|\\
|\childdocforwardprefix[|\textit{main}|]{|\textit{prefix}|}{|\textit{dest}|}|
\end{tabular}
\end{center}
%
the destination file is determined by a pattern
depending on the current file:
To make this work, the current file must be called
`{\textit{prefix}\hspace{0.2em}\textit{suffix}}'
with \textit{prefix} matching precisely the argument.
Processing is then passed on to the file
`{\textit{dest}\hspace{0.2em}\textit{suffix}}'.
Surely, the same effect is achieved by
directly specifying the
argument `{\textit{dest}\hspace{0.2em}\textit{suffix}}'
in the first form.
However, that requires to set up a different file
for each child. With the alternative form of the command
all these files can have exactly the same content
which simplifies setting them up and maintaining them.

For example, the following file |draft.tex|
with a compilation flag |\version| as described in \secref{sec:flags}
compiles the main document as a draft:
%
\begin{center}
\begin{tabular}{l}
|\def\version{draft}|\\
|\input{childdoc.def}|\\
|\childdocforward{|\textit{main}|}|
\end{tabular}
\end{center}
%
Likewise, the following files |final|\textit{nn}|.tex|
compile the final version of the child document
|child|\textit{nn}|.tex|:
%
\begin{center}
\begin{tabular}{l}
|\def\version{final}|\\
|\input{childdoc.def}|\\
|\childdocforwardprefix{final}{child}|
\end{tabular}
\end{center}
%

Note that when several versions of a main file and/or of each child file
are to be generated, it may be convenient to set up a |Makefile| or
shell script to automatise the process.

%%%%%%%%%%%%%%%%%%%%%%%%%%%%%%%%%%%%%%%%%%%%%%%%%%%%%%%%%%%%%%%%%%%%%%%%%%%%%%%%
\subsection{Command Line Processing}
\label{sec:commandline}

The effect of redirection files can also be achieved by invoking
the \LaTeX{} compiler with a more elaborate command line.
Most conveniently this should be done as part
of a shell script or a |Makefile|.

When using \textsf{childdoc} in the main file, the following
command lines effectively perform a redirection
(note that depending on the shell being used,
backslashes may have to be doubled: `|\|' $\to$ `|\\|'):
%
\begin{center}
|... -jobname "|\textit{target}|" |\\|"|[\textit{flags}]%
|\input{childdoc.def}\childdocforward[|\textit{main}|]{|\textit{dest}|}"|
\end{center}
%
Here \textit{target} is the name of the output file,
\textit{main} is the name of the main file
and \textit{dest} is the name of the main or child file to be processed
(all filenames without extensions).
The optional argument \textit{main} can be omitted
if \textit{main} matches \textit{dest}.
Optionally, compilation \textit{flags} can be defined via |\def| commands.
This command line makes the \TeX{} engine believe
it is compiling the file \textit{target}
whose content is specified as the latter parameter.
The provided code then forwards the processing to
\textit{main} or \textit{dest} as described in \secref{sec:forward}.

%%%%%%%%%%%%%%%%%%%%%%%%%%%%%%%%%%%%%%%%%%%%%%%%%%%%%%%%%%%%%%%%%%%%%%%%%%%%%%%%
\subsection{Include by Input}
\label{sec:input}

Including child documents by |\include| has some restrictions by design.
Most notably, the content of a child document always occupies
its own set of pages; pages cannot be shared between child documents.
Usually, this behaviour makes perfect sense
because each child document contain an essential part of the document.
However, in some situations it may be desirable to compose
a document from a collection of parts
without having mandatory page breaks between then.
For this case, the package
provides a mechanism to include parts
by |\input| which can also be processed individually.
However, by construction this mechanism
requires manual handling of the content to be output.

%%%%%%%%%%%%%%%%%%%%%%%%%%%%%%%%%%%%%%%%
\DescribeMacro{\ifchilddocmanual}
The main file should be prepared as usual, see \secref{sec:include}.
However, the document body must make a distinction
between processing of an individual part and of the main document, e.g.:
%
\begin{center}
\begin{tabular}{l}
|\ifchilddocmanual|\\
|\input{\childdocname}|\\
|\||else|\\
\textit{document body with }|\input{|\textit{part}|}|\\
|\||fi|
\end{tabular}
\end{center}
%
The conditional |\ifchilddocmanual| is true whenever
a part to be included by |\input| is being compiled,
and the name of the part is stored in |\childdocname|.

%%%%%%%%%%%%%%%%%%%%%%%%%%%%%%%%%%%%%%%%
\DescribeMacro{\childdocby}
Each part to be included by |\input| should start with:
%
\begin{center}
\begin{tabular}{l}
|\input{childdoc.def}|\\
|\childdocby{|\textit{main}|}|\\
\end{tabular}
\end{center}
%
The directive |\childdocby| is similar to |\childdocof|
described in \secref{sec:include},
but the subsequent selection of content must be done manually.
To that end, both |\ifchilddoc| and |\ifchilddocmanual|
will be true upon processing of a part,
and the name of the part is stored in |\childdocname|.
Note that |\jobname| will be set to the filename of the current part
so that each part receives an individual |.aux| file
that does not interfere with the |.aux| file(s) of the main document.
This behaviour can be altered by the alternative form
|\childdocby[*]{|\textit{main}|}| (with a non-empty optional argument)
which uses the |.aux| file of the main document
by setting |\jobname| to \textit{main}.

%%%%%%%%%%%%%%%%%%%%%%%%%%%%%%%%%%%%%%%%%%%%%%%%%%%%%%%%%%%%%%%%%%%%%%%%%%%%%%%%
\subsection{Driver Development}
\label{sec:driver}

The \textsf{childdoc} mechanism can also be use for the development
of definition files such as \LaTeX{} styles or classes.
This case differs from the above setup with multiple parts
included by |\include| in that no |\includeonly| should be invoked.
This can be achieved by starting the include file
(before |\ProvidesPackage|) with:
%
\begin{center}
\begin{tabular}{l}
|\input{childdoc.def}|\\
|\childdocforward{|\textit{main}|}|\\
\end{tabular}
\end{center}
%
or alternatively with:
%
\begin{center}
\begin{tabular}{l}
|\input{childdoc.def}|\\
|\childdocby{|\textit{main}|}|\\
\end{tabular}
\end{center}
%
Both forms have slightly different effects as described above.
The main file is prepared as usual, see \secref{sec:include}.

%%%%%%%%%%%%%%%%%%%%%%%%%%%%%%%%%%%%%%%%%%%%%%%%%%%%%%%%%%%%%%%%%%%%%%%%%%%%%%%%
\subsection{Legacy Detection}
\label{sec:detection}

The directive |\childdocmain| in the main file can detect
whether the complete document or merely a child is to be compiled
even without using the directive |\childdocof|.
This method is deprecated because it is less robust
and there is no compelling reason to use it;
it is merely provided for backward compatibility
and it may be removed in future versions.

If the detection mechanism is to be used,
it is mandatory to correctly specify
the filename of the main file as the argument of |\childdocmain|:
%
\begin{center}
\begin{tabular}{l}
|\input{childdoc.def}|\\
|\childdocmain{|\textit{main}|}|\\
\end{tabular}
\end{center}
%
If |\jobname| does not match the argument \textit{main} of |\childdocmain|,
it is assumed that |\jobname| points to the child file to be compiled.
When using |\childdocmain| with the main file specified as argument,
it suffices to start a child file
with just |\input{|\textit{main}|}|
without loading of the package and using |\childdocof|.
If instead all processing is done
with the appropriate \textsf{childdoc} directives,
the argument of \textit{main} of |\childdocmain| can be empty.

An alternative version of the command line processing described
in \secref{sec:commandline} using the detection mechanism reads:
%
\begin{center}
|... -jobname "|\textit{target}|" "|[\textit{flags}]%
[|\def\jobname{|\textit{dest}|}|]|\input{|\textit{main}|}"|
\end{center}

%%%%%%%%%%%%%%%%%%%%%%%%%%%%%%%%%%%%%%%%%%%%%%%%%%%%%%%%%%%%%%%%%%%%%%%%%%%%%%%%
\subsection{Manual Code}
\label{sec:manual}

In case one cannot be certain whether the definitions file |childdoc.def|
is installed on the target \TeX{} distribution
and one prefers not to ship it,
it is conceivable to paste a few relevant commands into the sources.

To that end, drop all statements |\input{childdoc.def}|
and perform the replacements as outlined below.
Instead of |\childdocmain{|\textit{main}|}| add the following code
to the top of the main file:
%
\begin{center}
\begin{tabular}{l}
|\||ifdefined\childdocname\endinput\||fi\newif\ifchilddoc|\\
|\edef\childdocname{\scantokens\expandafter{\jobname\noexpand}}|\\
|\def\childdocmain{|\textit{main}|}\||ifx\childdocmain\childdocname\||else|\\
|\childdoctrue\includeonly{\childdocname}\let\jobname\childdocmain\||fi|\\
\end{tabular}
\end{center}
%
Instead of |\childdocof{|\textit{main}|}| just include the main file
at the top of each child file:
%
\begin{center}
|\input{|\textit{main}|}|
\end{center}
%
A simple redirection |\childdocforward{|\textit{dest}|}| is achieved by:
%
\begin{center}
|\def\jobname{|\textit{dest}|}\input{\jobname}|
\end{center}
%
The redirection with prefix
|\childdocforwardprefix[|\textit{prefix}|]{|\textit{dest}|}|
is accomplished by:
%
\begin{center}
\begin{tabular}{l}
|{\edef\jobname{\scantokens\expandafter{\jobname\noexpand}}|\\
|\def\redirectjob |\textit{prefix}|#1~~~{\gdef\jobname{|\textit{dest}|#1}}|\\
|\expandafter\redirectjob\jobname~~~}\input{\jobname}|
\end{tabular}
\end{center}

In an alternative approach,
child documents can be compiled by a specific command line
without additional code or specific definitions:
%
\begin{center}
|... -jobname "|\textit{target}|" "|[\textit{flags}]%
|\includeonly{|\textit{dest}|}\input{|\textit{main}|}"|
\end{center}
%

%%%%%%%%%%%%%%%%%%%%%%%%%%%%%%%%%%%%%%%%%%%%%%%%%%%%%%%%%%%%%%%%%%%%%%%%%%%%%%%%
%%%%%%%%%%%%%%%%%%%%%%%%%%%%%%%%%%%%%%%%%%%%%%%%%%%%%%%%%%%%%%%%%%%%%%%%%%%%%%%%
\section{Information}

%%%%%%%%%%%%%%%%%%%%%%%%%%%%%%%%%%%%%%%%%%%%%%%%%%%%%%%%%%%%%%%%%%%%%%%%%%%%%%%%
\subsection{Copyright}

Copyright \copyright{} 2017--2018 Niklas Beisert

This work may be distributed and/or modified under the
conditions of the \LaTeX{} Project Public License, either version 1.3
of this license or (at your option) any later version.
The latest version of this license is in
  \url{http://www.latex-project.org/lppl.txt}
and version 1.3 or later is part of all distributions of \LaTeX{}
version 2005/12/01 or later.

This work has the LPPL maintenance status `maintained'.

The Current Maintainer of this work is Niklas Beisert.

This work consists of the files |README.txt|, |childdoc.ins| and |childdoc.dtx|
as well as the derived files |childdoc.def|, |cdocsamp.tex|
with |cdocsch1.tex|, |cdocsch2.tex|, |cdocspt3.tex|, |cdocspt4.tex|,
|cdocsdrf.tex|, |cdocsfn1.tex|, |cdocsfn2.tex|
as well as |childdoc.pdf|.

%%%%%%%%%%%%%%%%%%%%%%%%%%%%%%%%%%%%%%%%%%%%%%%%%%%%%%%%%%%%%%%%%%%%%%%%%%%%%%%%
\subsection{Files and Installation}

The package consists of the files:
%
\begin{center}
\begin{tabular}{ll}
    |README.txt|   & readme file \\
    |childdoc.ins| & installation file \\
    |childdoc.dtx| & source file \\
    |childdoc.def| & definition file \\
    |cdocsamp.tex| & sample main file \\
    |cdocsch1.tex| & sample include file \\
    |cdocsch2.tex| & sample include file \\
    |cdocspt3.tex| & sample part file \\
    |cdocspt4.tex| & sample part file \\
    |cdocsdrf.tex| & sample redirection file \\
    |cdocsfn1.tex| & sample redirection file \\
    |cdocsfn2.tex| & sample redirection file \\
    |childdoc.pdf| & manual
\end{tabular}
\end{center}
%
The distribution consists of the files
|README.txt|, |childdoc.ins| and |childdoc.dtx|.
%
\begin{itemize}
\item
Run (pdf)\LaTeX{} on |childdoc.dtx|
to compile the manual |childdoc.pdf| (this file).
\item
Run \LaTeX{} on |childdoc.ins| to create the definitions file |childdoc.def|
and the sample |cdocsamp.tex| with include files
|cdocsch1.tex|, |cdocsch2.tex|, |cdocspt3.tex|, |cdocspt4.tex|,
|cdocsdrf.tex|, |cdocsfn1.tex|, |cdocsfn2.tex|.
Then copy the file |childdoc.def| to an appropriate directory of your \LaTeX{}
distribution, e.g.\ \textit{texmf-root}|/tex/latex/childdoc|.
\end{itemize}

%%%%%%%%%%%%%%%%%%%%%%%%%%%%%%%%%%%%%%%%%%%%%%%%%%%%%%%%%%%%%%%%%%%%%%%%%%%%%%%%
\subsection{Related CTAN Packages}

There are several other packages which offer a similar functionality:
%
\begin{itemize}
\item
The packages
\href{http://ctan.org/pkg/docmute}{\textsf{docmute}},
\href{http://ctan.org/pkg/includex}{\textsf{includex}} and
\href{http://ctan.org/pkg/standalone}{\textsf{standalone}}
provide commands to include only the document body of
a child file thus allowing both files to be compiled individually.
\item
The packages \href{http://ctan.org/pkg/subdocs}{\textsf{subdocs}}
and \href{http://ctan.org/pkg/subfiles}{\textsf{subfiles}}
provide structures in which the main and child documents can be
encapsulated and allowing them to be compiled individually.
The inclusion mechanism is different from the conventional |\include|.
\item
The package \href{http://ctan.org/pkg/combine}{\textsf{combine}}
is an elaborate solution to combine several documents into one.
\end{itemize}
%
See also the CTAN topic \href{http://ctan.org/topic/subdocs}{\textsf{subdocs}}
for further related packages.
The present package differs from the above solutions in that
a document structure constructed with the conventional |\include| mechanism
just needs two extra commands at the top of every file
such that all constituent files can be compiled individually.

%%%%%%%%%%%%%%%%%%%%%%%%%%%%%%%%%%%%%%%%%%%%%%%%%%%%%%%%%%%%%%%%%%%%%%%%%%%%%%%%
%\subsection{Feature Suggestions}
%
%The following is a list of features which may be useful for future
%versions of this package:
%%
%\begin{itemize}
%\item
%\ldots
%\end{itemize}

%%%%%%%%%%%%%%%%%%%%%%%%%%%%%%%%%%%%%%%%%%%%%%%%%%%%%%%%%%%%%%%%%%%%%%%%%%%%%%%%
\subsection{Revision History}

%%%%%%%%%%%%%%%%%%%%%%%%%%%%%%%%%%%%%%%%
\paragraph{v2.0:} 2018/12/30

\begin{itemize}
\item
immediate forward processing
\item
added |\childdocby| mechanism
\item
manual restructured
\end{itemize}

%%%%%%%%%%%%%%%%%%%%%%%%%%%%%%%%%%%%%%%%
\paragraph{v1.6:} 2018/01/17

\begin{itemize}
\item
application for development of include files
\item
corrections to manual
\end{itemize}

%%%%%%%%%%%%%%%%%%%%%%%%%%%%%%%%%%%%%%%%
\paragraph{v1.5:} 2017/05/21

\begin{itemize}
\item
more complete structuring introduced
\item
|\childdocof| introduced
\item
|\childdoc| renamed to |\childdocmain|
\item
|\childredirect| renamed to |\childdocforward| and |\childdocforwardprefix|
and functionality expanded
\end{itemize}

%%%%%%%%%%%%%%%%%%%%%%%%%%%%%%%%%%%%%%%%
\paragraph{v1.0:} 2017/04/27

\begin{itemize}
\item
manual and install package
\item
first version published on CTAN
\end{itemize}

%%%%%%%%%%%%%%%%%%%%%%%%%%%%%%%%%%%%%%%%
\paragraph{v0.6:} 2017/04/26

\begin{itemize}
\item
redirection mechanism added
\end{itemize}

%%%%%%%%%%%%%%%%%%%%%%%%%%%%%%%%%%%%%%%%
\paragraph{v0.5:} 2017/04/26

\begin{itemize}
\item
functionality in definition file
\end{itemize}


%%%%%%%%%%%%%%%%%%%%%%%%%%%%%%%%%%%%%%%%%%%%%%%%%%%%%%%%%%%%%%%%%%%%%%%%%%%%%%%%
%%%%%%%%%%%%%%%%%%%%%%%%%%%%%%%%%%%%%%%%%%%%%%%%%%%%%%%%%%%%%%%%%%%%%%%%%%%%%%%%
%%%%%%%%%%%%%%%%%%%%%%%%%%%%%%%%%%%%%%%%%%%%%%%%%%%%%%%%%%%%%%%%%%%%%%%%%%%%%%%%
\appendix

\settowidth\MacroIndent{\rmfamily\scriptsize 000\ }

 \DocInput{childdoc.dtx}

\end{document}
%</driver>
% \fi
%
% %%%%%%%%%%%%%%%%%%%%%%%%%%%%%%%%%%%%%%%%%%%%%%%%%%%%%%%%%%%%%%%%%%%%%%%%%%%%%%
% %%%%%%%%%%%%%%%%%%%%%%%%%%%%%%%%%%%%%%%%%%%%%%%%%%%%%%%%%%%%%%%%%%%%%%%%%%%%%%
% \section{Sample}
%\iffalse
%<*samplemain>
%\fi
%
% The following presents a sample document
% with two chapters, two parts, a title page,
% a compile flag as well as three forwarding files to set the flag.
% It consists of eight |.tex| files:
% \begin{center}
% \begin{tabular}{ll}
% |cdocsamp.tex|&main file\\
% |cdocsch1.tex|&include file for chapter 1\\
% |cdocsch2.tex|&include file for chapter 2\\
% |cdocspt3.tex|&include file for part 3\\
% |cdocspt4.tex|&include file for part 4\\
% |cdocsdrf.tex|&forwarding file for main file in draft mode\\
% |cdocsfi1.tex|&forwarding file for final version of chapter 1\\
% |cdocsfi2.tex|&forwarding file for final version of chapter 2\\
% \end{tabular}
% \end{center}
% Each of the eight files can be compiled directly by the \LaTeX{} compiler.
%
% %%%%%%%%%%%%%%%%%%%%%%%%%%%%%%%%%%%%%%
% \paragraph{Main File.}
%
% The main file is called |cdocsamp.tex|.
%
% Load the \textsf{childdoc} definitions and
% declare the filename for the main document:
%    \begin{macrocode}
\input{childdoc.def}
\childdocmain{}
%    \end{macrocode}

% Optional override for |\version| flag:
%    \begin{macrocode}
%%\ifchilddoc\else\providecommand{\version}{draft}\fi
%    \end{macrocode}

% Define the default values for the |\version| flag
% (|final| for the main file and |draft| for childs):
%    \begin{macrocode}
\ifchilddoc
\providecommand{\version}{draft}
\else
\providecommand{\version}{final}
\fi
%    \end{macrocode}

% Load the standard document class:
%    \begin{macrocode}
\documentclass[12pt]{article}
%    \end{macrocode}

% Start the document body:
%    \begin{macrocode}
\begin{document}
%    \end{macrocode}

% Declare a title page.
% Print title, part of document being processed and version flag:
%    \begin{macrocode}
\addtocounter{page}{-1}
\begin{center}
{\LARGE\bfseries{}childdoc example\par}
\vspace{1cm}
\ifchilddoc
\ifchilddocmanual part\else chapter\fi:
`\childdocname' of `\childdocjob'\par
\else
main document: `\childdocjob'\par
\fi
version: \version\par
\end{center}
\newpage
%    \end{macrocode}

% Manually include selected file,
% otherwise process as usual:
%    \begin{macrocode}
\ifchilddocmanual
\section*{part `\childdocname'}
\input{\childdocname}
\else
%    \end{macrocode}

% Include the two chapters:
%    \begin{macrocode}
\include{cdocsch1}
\include{cdocsch2}
%    \end{macrocode}

% Include the two parts unless only chapters should be displayed:
%    \begin{macrocode}
\ifchilddoc\else
\section{part three}
\input{cdocspt3}
\section{part four}
\input{cdocspt4}
\fi
%    \end{macrocode}

% Process as usual until here:
%    \begin{macrocode}
\fi
%    \end{macrocode}

% End of document body:
%    \begin{macrocode}
\end{document}
%    \end{macrocode}
%\iffalse
%</samplemain>
%\fi
%
% %%%%%%%%%%%%%%%%%%%%%%%%%%%%%%%%%%%%%%
% \paragraph{Chapter Include Files.}
%
% The include files are called |cdocsch1.tex| and |cdocsch2.tex|.
%
%\iffalse
%<*samplechap1|samplechap2>
%\fi

% Optional override for |\version| flag:
%    \begin{macrocode}
%%\providecommand{\version}{final}
%    \end{macrocode}

% Include the main document:
%    \begin{macrocode}
\input{childdoc.def}
\childdocof{cdocsamp}
%    \end{macrocode}

%\iffalse
%</samplechap1|samplechap2>
%\fi
%
%\iffalse
%<*samplechap1>
%\fi
% Some text for chapter 1:
%    \begin{macrocode}
\section{one}
some text in chapter one
%    \end{macrocode}

%\iffalse
%</samplechap1>
%\fi
% Some text for chapter 2:
%\iffalse
%<*samplechap2>
%\fi
%    \begin{macrocode}
\section{two}
more text in chapter two
%    \end{macrocode}

%\iffalse
%</samplechap2>
%\fi
%
% %%%%%%%%%%%%%%%%%%%%%%%%%%%%%%%%%%%%%%
% \paragraph{Part Include Files.}
%
% The include files are called |cdocspt3.tex| and |cdocspt4.tex|.
%
%\iffalse
%<*samplepart3|samplepart4>
%\fi

% Optional override for |\version| flag:
%    \begin{macrocode}
%%\providecommand{\version}{final}
%    \end{macrocode}

% Include the main document:
%    \begin{macrocode}
\input{childdoc.def}
\childdocby{cdocsamp}
%    \end{macrocode}

%\iffalse
%</samplepart3|samplepart4>
%\fi
%
%\iffalse
%<*samplepart3>
%\fi
% Some text for part 3:
%    \begin{macrocode}
some text in part three
%    \end{macrocode}

%\iffalse
%</samplepart3>
%\fi
% Some text for part 4:
%\iffalse
%<*samplepart4>
%\fi
%    \begin{macrocode}
more text in part four
%    \end{macrocode}

%\iffalse
%</samplepart4>
%\fi
%
% %%%%%%%%%%%%%%%%%%%%%%%%%%%%%%%%%%%%%%
% \paragraph{Forwarding for a Complete Draft.}
%
% The following forwarding file |cdocsdrf.tex|
% compiles the main document in draft mode:
%\iffalse
%<*sampledraft>
%\fi
%    \begin{macrocode}
\def\version{draft}
\input{childdoc.def}
\childdocforward{cdocsamp}
%    \end{macrocode}

%\iffalse
%</sampledraft>
%\fi
%
% %%%%%%%%%%%%%%%%%%%%%%%%%%%%%%%%%%%%%%
% \paragraph{Forwarding for Final Version of the Chapters.}
%
% The following forwarding files |cdocsfn1.tex| and |cdocsfn2.tex|
% (with identical content)
% compile the final versions of the child documents
% |cdocsch1.tex| and |cdocsch2.tex|, respectively:
%\iffalse
%<*samplefinal>
%\fi
%    \begin{macrocode}
\def\version{final}
\input{childdoc.def}
\childdocforwardprefix[cdocsamp]{cdocsfn}{cdocsch}
%    \end{macrocode}

%\iffalse
%</samplefinal>
%\fi
%
% %%%%%%%%%%%%%%%%%%%%%%%%%%%%%%%%%%%%%%
% \paragraph{Command Line Processing.}
%
% The following three command lines generate the output files
% |cdocscld|, |cdocscl1| and |cdocscl2|
% which should be identical to
% |cdocsdrf|, |cdocsch1| and |cdocsfn2|, respectively:
% \begin{center}
% \begin{tabular}{l}
% |latex -jobname cdocscld \|\\
% |  "\def\version{draft}\input{childdoc.def}\childdocforward{cdocsamp}"|\\
% |latex -jobname cdocscl1 \|\\
% |  "\input{childdoc.def}\childdocforward[cdocsamp]{cdocsch1}"|\\
% |latex -jobname cdocscl2 \|\\
% |  "\def\version{final}\input{childdoc.def}\childdocforward{cdocsch2}"|
% \end{tabular}
% \end{center}
% Note that the trailing backslash on each first line
% merely continues the input to the second line
% (for convenient cut ant paste).
% Furthermore, the command |latex| can be replaced by any
% of its alternative versions such as |pdflatex|.
%
% %%%%%%%%%%%%%%%%%%%%%%%%%%%%%%%%%%%%%%%%%%%%%%%%%%%%%%%%%%%%%%%%%%%%%%%%%%%%%%
% %%%%%%%%%%%%%%%%%%%%%%%%%%%%%%%%%%%%%%%%%%%%%%%%%%%%%%%%%%%%%%%%%%%%%%%%%%%%%%
% \section{Implementation}
%\iffalse
%<*package>
%\fi
%
% This section describes the definitions file |childdoc.def|.

% The definitions cannot be loaded using |\usepackage| or |\RequirePackage|
% which has a mechanism to prevent loading a style file more than once.
% When loading the definitions by means of |\input|
% multiple instances have to be prevented manually:
%\iffalse
%This code needs to be before the `\ProvidesFile' directive
%which is defined at the beginning of this file.
%Therefore it is also placed there and commented out here.
%</package>
%<*discard>
%\fi
%    \begin{macrocode}
\ifdefined\childdocmain\endinput\fi
%    \end{macrocode}
%\iffalse
%</discard>
%<*package>
%\fi
%
% \macro{\ifchilddoc}
% \macro{\ifchilddocmanual}
% The conditional |\ifchilddoc| tells whether a
% child (true) or main (false) document is being compiled.
% The conditional |\ifchilddocmanual| tells whether
% the |\includeonly| mechanism is used (false) or
% the selection of child files must be performed manually (true).
% The definitions initialise to false:
%    \begin{macrocode}
\newif\ifchilddoc
\newif\ifchilddocmanual
%    \end{macrocode}

% \macro{\childdocname}
% \macro{\childdocjob}
% The macro |\childdocname| stores the name of the main document
% to be compiled. The macro |\childdocjob| stores the name of
% the document on which the \LaTeX{} compiler was originally invoked.
% The content of |\jobname| cannot be compared
% to filenames specified in the source due to different catcodes.
% The following code rescans |\jobname|, stores the result
% in |\childdocname| and saves a copy in |\childdocjob|:
%    \begin{macrocode}
\edef\childdocname{\scantokens\expandafter{\jobname\noexpand}}
\let\childdocjob\childdocname
%    \end{macrocode}

% \macro{\childdocdisable}
% The macro |\childdocdisable| prevents the main file
% from being processed more than once.
% At this stage, the main document command |\childdocmain|
% is assumed to be called once again where it should do nothing.
% Any subsequent call to it should prevent
% a secondary processing of the main document
% It overwrites the forwarding commands
% |\childdocof| and |\childdocforward|
% with empty macros to prevent further inclusions of the main document:
%    \begin{macrocode}
\newcommand{\childdocdisable}
{
  \renewcommand{\childdocmain}[1]{\renewcommand{\childdocmain}[1]{\endinput}}
  \renewcommand{\childdocof}[1]{}
  \renewcommand{\childdocby}[2][]{}
  \renewcommand{\childdocforward}[2][]{}
  \renewcommand{\childdocdisable}{}
}
%    \end{macrocode}

% \macro{\childdocmain}
% The macro |\childdocmain| is to be called at the top of the main file
% with nothing or the main filename (without extension) as argument.
% First, it breaks loops.
% If the argument is not empty and does not match |\childdocname|
% (which is set by the first inclusion of |childdoc.def|),
% |\ifchilddoc| is set to true, |\includeonly| is applied to the child file
% and |\jobname| is set to the main file
% (for proper handling of |.aux| files):
%    \begin{macrocode}
\newcommand{\childdocmain}[1]
{
  \childdocdisable\childdocmain{}
  \if?#1?\else
    \begingroup
      \def\childdoctmp{#1}
      \ifx\childdoctmp\childdocname
        \def\childdoctmp{}
      \else
        \def\childdoctmp
        {
          \childdoctrue
          \includeonly{\childdocname}
          \def\childdocjob{#1}
          \def\jobname{#1}
        }
      \fi
      \expandafter
    \endgroup
    \childdoctmp
  \fi
}
%    \end{macrocode}

% \macro{\childdocof}
% The command |\childdocof| redirects
% compilation to the main file |#1|.
%    \begin{macrocode}
\newcommand{\childdocof}[1]
{
  \childdocdisable
  \childdoctrue
  \includeonly{\childdocname}
  \def\jobname{#1}
  \def\childdocjob{#1}
  \input{#1}
}
%    \end{macrocode}

% \macro{\childdocby}
% The command |\childdocby| ....
%    \begin{macrocode}
\newcommand{\childdocby}[2][]
{
  \childdocdisable
  \childdoctrue
  \childdocmanualtrue
  \if?#1?\else
    \def\jobname{#2}
  \fi
  \def\childdocjob{#2}
  \input{#2}
  \endinput
}
%    \end{macrocode}

% \macro{\childdocforward}
% The command |\childdocforward| redirects
% compilation to the main file or
% (if the optional argument is given) a child file.
% Parameters are set as if the main file
% or a child file starting with |\childdocof| was compiled.
% Then compilation is handed over to the main file:
%    \begin{macrocode}
\newcommand{\childdocforward}[2][]
{
  \begingroup
    \if?#1?
      \def\childdoctmp
      {
        \def\childdocname{#2}
        \def\childdocjob{#2}
        \def\jobname{#2}
        \input{#2}
        \endinput
      }
    \else
      \def\childdoctmp
      {
        \childdocdisable
        \def\childdocname{#2}
        \childdoctrue
        \includeonly{#2}
        \def\childdocjob{#1}
        \def\jobname{#1}
        \input{#1}
        \endinput
      }
    \fi
    \expandafter
  \endgroup
  \childdoctmp
}
%    \end{macrocode}

% \macro{\childdocforwardprefix}
% The command |\childdocforwardprefix| redirects
% compilation to the main or a child file by means of a pattern.
% The prefix |#1| in the current filename is replaced by |#2|
% and the suffix of the current filename is kept
% (it is assumed that the filename does not contain the substring `|~~~|'
% which is used as a delimiter).
% Compilation is handed over to the new file by |\childdocforward|:
%    \begin{macrocode}
\newcommand{\childdocforwardprefix}[3][]
{
  \begingroup
    \def\childdocextract #2##1~~~{\def\childdoctmp{\childdocforward[#1]{#3##1}}}
    \expandafter\childdocextract\childdocname~~~
    \expandafter
  \endgroup
  \childdoctmp
}
%    \end{macrocode}

% \macro{\childdoc}
% The deprecated macro |\childdoc| is a legacy version of |\childdocmain|:
%    \begin{macrocode}
\newcommand{\childdoc}{\childdocmain}
%    \end{macrocode}

% \macro{\childdocredirect}
% The deprecated macro |\childdocredirect| is a legacy version
% of |\childdocforward| and |\childdocforwardprefix|:
%    \begin{macrocode}
\newcommand{\childdocredirect}[2][]
{
  \begingroup
    \if?#1?
      \def\childdoctmp{\childdocforward{#2}}
    \else
      \def\childdoctmp{\childdocforwardprefix{#1}{#2}}
    \fi
    \expandafter
  \endgroup
  \childdoctmp
}
%    \end{macrocode}

%\iffalse
%</package>
%\fi
%
\endinput

\childdocforward{cdocsamp}
%    \end{macrocode}

%\iffalse
%</sampledraft>
%\fi
%
% %%%%%%%%%%%%%%%%%%%%%%%%%%%%%%%%%%%%%%
% \paragraph{Forwarding for Final Version of the Chapters.}
%
% The following forwarding files |cdocsfn1.tex| and |cdocsfn2.tex|
% (with identical content)
% compile the final versions of the child documents
% |cdocsch1.tex| and |cdocsch2.tex|, respectively:
%\iffalse
%<*samplefinal>
%\fi
%    \begin{macrocode}
\def\version{final}
% \iffalse
%
% childdoc.dtx Copyright (C) 2017-2018 Niklas Beisert
%
% This work may be distributed and/or modified under the
% conditions of the LaTeX Project Public License, either version 1.3
% of this license or (at your option) any later version.
% The latest version of this license is in
%   http://www.latex-project.org/lppl.txt
% and version 1.3 or later is part of all distributions of LaTeX
% version 2005/12/01 or later.
%
% This work has the LPPL maintenance status `maintained'.
%
% The Current Maintainer of this work is Niklas Beisert.
%
% This work consists of the files childdoc.dtx and childdoc.ins
% and the derived files childdoc.def and cdocsamp.tex with
% cdocsch1.tex, cdocsch2.tex, cdocsdrf.tex, cdocsfn1.tex, cdocsfn2.tex.
%
%<package>\ifdefined\childdocmain\endinput\fi
%<package>\ProvidesFile{childdoc.def}[2018/12/30 v2.0 child document driver]
%<samplemain>\ProvidesFile{cdocsamp.tex}[2018/12/30 v2.0 sample for childdoc]
%<*driver>
%\ProvidesFile{childdoc.drv}[2018/12/30 v2.0 childdoc reference manual file]
\PassOptionsToClass{10pt,a4paper}{article}
\documentclass{ltxdoc}

\usepackage[margin=35mm]{geometry}
\usepackage{hyperref}
\usepackage{hyperxmp}
\usepackage[usenames]{color}

\hypersetup{colorlinks=true}
\hypersetup{pdfstartview=FitH}
\hypersetup{pdfpagemode=UseNone}
\hypersetup{pdfsource={}}
\hypersetup{pdflang={en-UK}}
\hypersetup{pdfcopyright={Copyright 2017-2018 Niklas Beisert.
  This work may be distributed and/or modified under the
  conditions of the LaTeX Project Public License, either version 1.3
  of this license or (at your option) any later version.}}
\hypersetup{pdflicenseurl={http://www.latex-project.org/lppl.txt}}
\hypersetup{pdfcontactaddress={ETH Zurich, ITP, HIT K,
  Wolfgang-Pauli-Strasse 27}}
\hypersetup{pdfcontactpostcode={8093}}
\hypersetup{pdfcontactcity={Zurich}}
\hypersetup{pdfcontactcountry={Switzerland}}
\hypersetup{pdfcontactemail={nbeisert@itp.phys.ethz.ch}}
\hypersetup{pdfcontacturl={http://people.phys.ethz.ch/\xmptilde nbeisert/}}

\newcommand{\secref}[1]{\hyperref[#1]{section \ref*{#1}}}

\parskip1ex
\parindent0pt
\let\olditemize\itemize
\def\itemize{\olditemize\parskip0pt}

\begin{document}

\title{The \textsf{childdoc} Package}
\hypersetup{pdftitle={The childdoc Package}}
\author{Niklas Beisert\\[2ex]
  Institut f\"ur Theoretische Physik\\
  Eidgen\"ossische Technische Hochschule Z\"urich\\
  Wolfgang-Pauli-Strasse 27, 8093 Z\"urich, Switzerland\\[1ex]
  \href{mailto:nbeisert@itp.phys.ethz.ch}
  {\texttt{nbeisert@itp.phys.ethz.ch}}}
\hypersetup{pdfauthor={Niklas Beisert}}
\hypersetup{pdfsubject={Manual for the LaTeX2e Package childdoc}}
\date{30 December 2018, \textsf{v2.0}}
\maketitle

\begin{abstract}\noindent
\textsf{childdoc} is a \LaTeXe{} package
that enables the direct compilation
of document sections included by |\include|
to individual files.
\end{abstract}

\begingroup
\parskip0ex
\tableofcontents
\endgroup

%%%%%%%%%%%%%%%%%%%%%%%%%%%%%%%%%%%%%%%%%%%%%%%%%%%%%%%%%%%%%%%%%%%%%%%%%%%%%%%%
%%%%%%%%%%%%%%%%%%%%%%%%%%%%%%%%%%%%%%%%%%%%%%%%%%%%%%%%%%%%%%%%%%%%%%%%%%%%%%%%
\section{Introduction}

\LaTeX{} provides a mechanism to structure a large document (such as a book)
into a main file and several child files (containing the chapters)
using the |\include| command.
This mechanism is beneficial for documents
which span hundreds of pages in order to
make the source file(s) more manageable.
Moreover, compilation can be restricted to
selected child files by means of the |\includeonly| command.
The latter feature can be used to reduce the compilation time while editing
(this was significantly more useful in the earlier days of \LaTeX{})
or to generate a smaller document which is easier to navigate.
Another application of |\includeonly| is to generate
documents consisting of selected parts of the complete document.

However, there are a few drawbacks of the plain |\include| mechanism:
\begin{itemize}
\item
The child files cannot be compiled on their own,
they can only be compiled via the main file.
A naive editing environment
(such as a text editor with an option
to have the current file processed by \LaTeX)
may require one to switch to the main file before compiling;
attempting to compile the child file produces errors.
\item
The main file must be modified (each time)
to adjust the |\includeonly| command
to the present needs. This easily leaves the main file in a messy state.
\item
The generated document will always carry the filename
of the main document. This is inconvenient if
several child files are to be compiled and
to be kept for distribution.
\end{itemize}

The present package provides a simple interface
to make child files individually compilable by \LaTeX{}.
Compiling a child file then has the same effect as compiling
the main file with an |\includeonly| command
to select the appropriate child.
Moreover the generated document will carry the name of the child
rather than the main file.
This resolves all three above issues.

This feature is meant to make the editing of books,
thesis documents and lecture notes somewhat more convenient.
However, the package can also be used efficiently for
composing a series of documents (such as exercise sheets)
which are typically distributed individually.
It then assists the author in generating the individual documents
(potentially in different versions)
as well as a document containing the collected series.
Another application is in developing style files
or other kinds of included material
where compilation of the style file could redirect
to a sample or test file.

%%%%%%%%%%%%%%%%%%%%%%%%%%%%%%%%%%%%%%%%%%%%%%%%%%%%%%%%%%%%%%%%%%%%%%%%%%%%%%%%
%%%%%%%%%%%%%%%%%%%%%%%%%%%%%%%%%%%%%%%%%%%%%%%%%%%%%%%%%%%%%%%%%%%%%%%%%%%%%%%%
\section{Usage}

First of all, the package \textsf{childdoc} is \emph{not} a standard
\LaTeXe{} |.sty| style file! Therefore it needs to be invoked in
a non-standard way.

%%%%%%%%%%%%%%%%%%%%%%%%%%%%%%%%%%%%%%%%%%%%%%%%%%%%%%%%%%%%%%%%%%%%%%%%%%%%%%%%
\subsection{Included Files}
\label{sec:include}

%%%%%%%%%%%%%%%%%%%%%%%%%%%%%%%%%%%%%%%%
\DescribeMacro{\childdocmain}
To use the package, add the commands
\begin{center}
\begin{tabular}{l}
|\input{childdoc.def}|\\
|\childdocmain{}|\\
\end{tabular}
\end{center}
at the very top of the main \LaTeX{} file,
in particular \emph{before} the |\documentclass| statement!
The argument of |\childdocmain| should be left empty
(but it must be present).

%%%%%%%%%%%%%%%%%%%%%%%%%%%%%%%%%%%%%%%%
\DescribeMacro{\childdocof}
Furthermore, add the commands
\begin{center}
\begin{tabular}{l}
|\input{childdoc.def}|\\
|\childdocof{|\textit{main}|}|\\
\end{tabular}
\end{center}
at the top of every child file \textit{child}
which is included by |\include{|\textit{child}|}|
from within the main file
(or at least for those files to be compiled individually).
The argument \textit{main} must be the filename of the main file.

There are a couple of
considerations in setting up the main and child documents:

%%%%%%%%%%%%%%%%%%%%%%%%%%%%%%%%%%%%%%%%
\paragraph{Restrictions.}

Please note the following restrictions:
\begin{itemize}
\item
|\childdocmain| must be called with one argument \textit{main}
to ensure compatibility with earlier version of the package.
It must either be empty (|\childdocmain{}|)
or precisely match the filename of the main file in which it is specified.
See \secref{sec:detection} for further information.
\item
The filename \textit{main} must be specified without the |.tex| extension.
\item
The filename \textit{main} is case sensitive
(even in case-insensitive file systems)
due to internal string comparison.
\item
The argument \textit{main} should be fully expanded, it cannot be a macro.
\item
Subdirectories and special characters should be avoided in filenames.
\item
The command |\childdocmain{|\textit{main}|}| must be followed by a whitespace.
It should not be followed immediately by another command
or by a comment mark `|%|'.
This is because the \TeX{} parser reads the token immediately following
the argument of |\childdocmain| and puts it
at the beginning of every child section;
however, a white\-space is ignored.
\end{itemize}

%%%%%%%%%%%%%%%%%%%%%%%%%%%%%%%%%%%%%%%%
\paragraph{Content of Main File.}

It is advisable to place all content in the child files included by |\include|.
Any output contained in the main file will appear in all child documents
unless suppressed manually;
it cannot be suppressed automatically by the |\includeonly| directive
and thus should normally be avoided.
A method to include some content in the main file
by means of conditional processing is described in \secref{sec:conditional}.

%%%%%%%%%%%%%%%%%%%%%%%%%%%%%%%%%%%%%%%%
\paragraph{Page Numbering.}

When only a part of the document is compiled,
the appropriate numbering of pages
(as well as other status parameters)
is determined from the |.aux| files.
The latter contain information from previous passes.
However this information needs to propagate through
all intermediate child documents.
Therefore the page numbering in child documents may well
be inconsistent until the complete document is compiled at least once.

A useful (if unconventional) way to always ensure a consistent
page numbering is to restart the numbering in each child document
and denote the pages by `\textit{child}|.|\textit{page}'
where \textit{child} represents the chapter/section number of the child file.
This can be achieved by the command
|\numberwithin{page}{|\textit{child}|}|
of the \textsf{amsmath} package
where \textit{child} can be |chapter| or |section|
depending on the chosen structuring.
Alternatively, one can modify the macro |\thepage| appropriately
and reset the counter |page| at the start of each child file.

%%%%%%%%%%%%%%%%%%%%%%%%%%%%%%%%%%%%%%%%%%%%%%%%%%%%%%%%%%%%%%%%%%%%%%%%%%%%%%%%
\subsection{Conditional Processing}
\label{sec:conditional}

The package provides a mechanism to compile different versions
of a document. To customise the versions further some conditional processing
can come in handy to distinguish which version is being compiled.
The package provides two macros to describe the compilation context:

%%%%%%%%%%%%%%%%%%%%%%%%%%%%%%%%%%%%%%%%
\DescribeMacro{\ifchilddoc}
The conditional |\ifchilddoc| distinguishes between the compilation of
child documents and the main document:
%
\begin{center}
|\ifchilddoc |\textit{child-code}| |[|\||else |\textit{main-code}]| \||fi|
\end{center}

%%%%%%%%%%%%%%%%%%%%%%%%%%%%%%%%%%%%%%%%
\DescribeMacro{\childdocname}
\DescribeMacro{\childdocjob}
The macro |\childdocname| contains the filename (without extension)
of the main or child file being processed.
Note that |\childdocjob| will always contain the name of the main file.

%%%%%%%%%%%%%%%%%%%%%%%%%%%%%%%%%%%%%%%%
\paragraph{Title Page.}

Conditional processing can be used to include a title or banner page
in the main document when proper precautions are taken.
Importantly, the code in the main file should ensure that the page counter
(as well as other status parameters which are stored in the |.aux| files)
takes the same value after the conditional processing.
Otherwise the page numbers may take divergent values
depending on which part is compiled.

For example, a title page could be declared by:
%
\begin{center}
\begin{tabular}{l}
|\ifchilddoc\||else|\\
|\addtocounter{page}{-1}|\\
\textit{code for title page}\\
|\newpage|\\
|\||fi|
\end{tabular}
\end{center}
%
A banner page for the child documents can be generated by:
%
\begin{center}
\begin{tabular}{l}
|\ifchilddoc|\\
|\addtocounter{page}{-1}|\\
\textit{code for banner page}\\
|\newpage|\\
|\||fi|
\end{tabular}
\end{center}
%
Here one could write a message such as:
\begin{center}
|This is the part \childdocname{} of \childdocjob{}.|
\end{center}

%%%%%%%%%%%%%%%%%%%%%%%%%%%%%%%%%%%%%%%%%%%%%%%%%%%%%%%%%%%%%%%%%%%%%%%%%%%%%%%%
\subsection{Flags}
\label{sec:flags}

The package makes it easy to generate different versions
of the main or child documents.
To this end compilation flags can be defined
and assigned different default values.
They will be particularly useful in conjunction
with the forwarding mechanism described in \secref{sec:forward}.

For example, it may be useful to have a flag |\version|
which can be set to |draft| or |final|.
The document source will contain some conditional code
depending on the value of |\version|.
Suppose further, the flag should default to |final| for the main file
and to |draft| for child files
which is a natural assignment for editing the document.
This is achieved by placing the following code
in the preamble of the main document
(below the |\childdocmain| directive):
%
\begin{center}
\begin{tabular}{l}
|\ifchilddoc|\\
|\providecommand{\version}{draft}|\\
|\||else|\\
|\providecommand{\version}{final}|\\
|\||fi|
\end{tabular}
\end{center}
%
The definition by |\providecommand| makes sure
that previous definitions are not overwritten.
Further statements |\providecommand{\version}{...}|
can thus be added before the above code to override it.

For the main file, one might add a line
(between |\childdocmain| and the above block)
%
\begin{center}
|%\ifchilddoc\||else\providecommand{\version}{draft}\||fi|
\end{center}
%
which can be uncommented to produce a draft version.
Likewise one can add a line to the very top of a child file
(above the |\childdocof{|\textit{main}|}| directive)
%
\begin{center}
|%\providecommand{\version}{final}|
\end{center}
%
which can be uncommented to produce the final version of this child document.

%%%%%%%%%%%%%%%%%%%%%%%%%%%%%%%%%%%%%%%%%%%%%%%%%%%%%%%%%%%%%%%%%%%%%%%%%%%%%%%%
\subsection{Forwarding}
\label{sec:forward}

Different versions of the main or child documents
using compilation flags as described in \secref{sec:flags}
can be (permanently) stored in different files
for convenient compilation, viewing and distribution.
To this end, the package defines a command
to pass on compilation to a different file:

%%%%%%%%%%%%%%%%%%%%%%%%%%%%%%%%%%%%%%%%
\DescribeMacro{\childdocforward}
The command |\childdocforward| redirects processing to
another source file:
%
\begin{center}
\begin{tabular}{l}
|\input{childdoc.def}|\\
|\childdocforward[|\textit{main}|]{|\textit{dest}|}|\\
\end{tabular}
\end{center}
%
The argument \textit{dest} is the destination file
(without extension).
It should be the main file or one of the child files.
Note that further \textsf{childdoc} directives
such as |\childdocof| and |\childdocforward|
in the indicated file will be processed in this form.
The optional argument \textit{main}
passes on directly to the main file \textit{main}
while pretending to compile the child \textit{dest}.
This form behaves as if \textit{dest}
issues |\childdocof{|\textit{main}|}| right away,
and no further \textsf{childdoc} directives will be processed.

%%%%%%%%%%%%%%%%%%%%%%%%%%%%%%%%%%%%%%%%
\DescribeMacro{\...prefix}
In the alternative form |\childdocforwardprefix|,
%
\begin{center}
\begin{tabular}{l}
|\input{childdoc.def}|\\
|\childdocforwardprefix[|\textit{main}|]{|\textit{prefix}|}{|\textit{dest}|}|
\end{tabular}
\end{center}
%
the destination file is determined by a pattern
depending on the current file:
To make this work, the current file must be called
`{\textit{prefix}\hspace{0.2em}\textit{suffix}}'
with \textit{prefix} matching precisely the argument.
Processing is then passed on to the file
`{\textit{dest}\hspace{0.2em}\textit{suffix}}'.
Surely, the same effect is achieved by
directly specifying the
argument `{\textit{dest}\hspace{0.2em}\textit{suffix}}'
in the first form.
However, that requires to set up a different file
for each child. With the alternative form of the command
all these files can have exactly the same content
which simplifies setting them up and maintaining them.

For example, the following file |draft.tex|
with a compilation flag |\version| as described in \secref{sec:flags}
compiles the main document as a draft:
%
\begin{center}
\begin{tabular}{l}
|\def\version{draft}|\\
|\input{childdoc.def}|\\
|\childdocforward{|\textit{main}|}|
\end{tabular}
\end{center}
%
Likewise, the following files |final|\textit{nn}|.tex|
compile the final version of the child document
|child|\textit{nn}|.tex|:
%
\begin{center}
\begin{tabular}{l}
|\def\version{final}|\\
|\input{childdoc.def}|\\
|\childdocforwardprefix{final}{child}|
\end{tabular}
\end{center}
%

Note that when several versions of a main file and/or of each child file
are to be generated, it may be convenient to set up a |Makefile| or
shell script to automatise the process.

%%%%%%%%%%%%%%%%%%%%%%%%%%%%%%%%%%%%%%%%%%%%%%%%%%%%%%%%%%%%%%%%%%%%%%%%%%%%%%%%
\subsection{Command Line Processing}
\label{sec:commandline}

The effect of redirection files can also be achieved by invoking
the \LaTeX{} compiler with a more elaborate command line.
Most conveniently this should be done as part
of a shell script or a |Makefile|.

When using \textsf{childdoc} in the main file, the following
command lines effectively perform a redirection
(note that depending on the shell being used,
backslashes may have to be doubled: `|\|' $\to$ `|\\|'):
%
\begin{center}
|... -jobname "|\textit{target}|" |\\|"|[\textit{flags}]%
|\input{childdoc.def}\childdocforward[|\textit{main}|]{|\textit{dest}|}"|
\end{center}
%
Here \textit{target} is the name of the output file,
\textit{main} is the name of the main file
and \textit{dest} is the name of the main or child file to be processed
(all filenames without extensions).
The optional argument \textit{main} can be omitted
if \textit{main} matches \textit{dest}.
Optionally, compilation \textit{flags} can be defined via |\def| commands.
This command line makes the \TeX{} engine believe
it is compiling the file \textit{target}
whose content is specified as the latter parameter.
The provided code then forwards the processing to
\textit{main} or \textit{dest} as described in \secref{sec:forward}.

%%%%%%%%%%%%%%%%%%%%%%%%%%%%%%%%%%%%%%%%%%%%%%%%%%%%%%%%%%%%%%%%%%%%%%%%%%%%%%%%
\subsection{Include by Input}
\label{sec:input}

Including child documents by |\include| has some restrictions by design.
Most notably, the content of a child document always occupies
its own set of pages; pages cannot be shared between child documents.
Usually, this behaviour makes perfect sense
because each child document contain an essential part of the document.
However, in some situations it may be desirable to compose
a document from a collection of parts
without having mandatory page breaks between then.
For this case, the package
provides a mechanism to include parts
by |\input| which can also be processed individually.
However, by construction this mechanism
requires manual handling of the content to be output.

%%%%%%%%%%%%%%%%%%%%%%%%%%%%%%%%%%%%%%%%
\DescribeMacro{\ifchilddocmanual}
The main file should be prepared as usual, see \secref{sec:include}.
However, the document body must make a distinction
between processing of an individual part and of the main document, e.g.:
%
\begin{center}
\begin{tabular}{l}
|\ifchilddocmanual|\\
|\input{\childdocname}|\\
|\||else|\\
\textit{document body with }|\input{|\textit{part}|}|\\
|\||fi|
\end{tabular}
\end{center}
%
The conditional |\ifchilddocmanual| is true whenever
a part to be included by |\input| is being compiled,
and the name of the part is stored in |\childdocname|.

%%%%%%%%%%%%%%%%%%%%%%%%%%%%%%%%%%%%%%%%
\DescribeMacro{\childdocby}
Each part to be included by |\input| should start with:
%
\begin{center}
\begin{tabular}{l}
|\input{childdoc.def}|\\
|\childdocby{|\textit{main}|}|\\
\end{tabular}
\end{center}
%
The directive |\childdocby| is similar to |\childdocof|
described in \secref{sec:include},
but the subsequent selection of content must be done manually.
To that end, both |\ifchilddoc| and |\ifchilddocmanual|
will be true upon processing of a part,
and the name of the part is stored in |\childdocname|.
Note that |\jobname| will be set to the filename of the current part
so that each part receives an individual |.aux| file
that does not interfere with the |.aux| file(s) of the main document.
This behaviour can be altered by the alternative form
|\childdocby[*]{|\textit{main}|}| (with a non-empty optional argument)
which uses the |.aux| file of the main document
by setting |\jobname| to \textit{main}.

%%%%%%%%%%%%%%%%%%%%%%%%%%%%%%%%%%%%%%%%%%%%%%%%%%%%%%%%%%%%%%%%%%%%%%%%%%%%%%%%
\subsection{Driver Development}
\label{sec:driver}

The \textsf{childdoc} mechanism can also be use for the development
of definition files such as \LaTeX{} styles or classes.
This case differs from the above setup with multiple parts
included by |\include| in that no |\includeonly| should be invoked.
This can be achieved by starting the include file
(before |\ProvidesPackage|) with:
%
\begin{center}
\begin{tabular}{l}
|\input{childdoc.def}|\\
|\childdocforward{|\textit{main}|}|\\
\end{tabular}
\end{center}
%
or alternatively with:
%
\begin{center}
\begin{tabular}{l}
|\input{childdoc.def}|\\
|\childdocby{|\textit{main}|}|\\
\end{tabular}
\end{center}
%
Both forms have slightly different effects as described above.
The main file is prepared as usual, see \secref{sec:include}.

%%%%%%%%%%%%%%%%%%%%%%%%%%%%%%%%%%%%%%%%%%%%%%%%%%%%%%%%%%%%%%%%%%%%%%%%%%%%%%%%
\subsection{Legacy Detection}
\label{sec:detection}

The directive |\childdocmain| in the main file can detect
whether the complete document or merely a child is to be compiled
even without using the directive |\childdocof|.
This method is deprecated because it is less robust
and there is no compelling reason to use it;
it is merely provided for backward compatibility
and it may be removed in future versions.

If the detection mechanism is to be used,
it is mandatory to correctly specify
the filename of the main file as the argument of |\childdocmain|:
%
\begin{center}
\begin{tabular}{l}
|\input{childdoc.def}|\\
|\childdocmain{|\textit{main}|}|\\
\end{tabular}
\end{center}
%
If |\jobname| does not match the argument \textit{main} of |\childdocmain|,
it is assumed that |\jobname| points to the child file to be compiled.
When using |\childdocmain| with the main file specified as argument,
it suffices to start a child file
with just |\input{|\textit{main}|}|
without loading of the package and using |\childdocof|.
If instead all processing is done
with the appropriate \textsf{childdoc} directives,
the argument of \textit{main} of |\childdocmain| can be empty.

An alternative version of the command line processing described
in \secref{sec:commandline} using the detection mechanism reads:
%
\begin{center}
|... -jobname "|\textit{target}|" "|[\textit{flags}]%
[|\def\jobname{|\textit{dest}|}|]|\input{|\textit{main}|}"|
\end{center}

%%%%%%%%%%%%%%%%%%%%%%%%%%%%%%%%%%%%%%%%%%%%%%%%%%%%%%%%%%%%%%%%%%%%%%%%%%%%%%%%
\subsection{Manual Code}
\label{sec:manual}

In case one cannot be certain whether the definitions file |childdoc.def|
is installed on the target \TeX{} distribution
and one prefers not to ship it,
it is conceivable to paste a few relevant commands into the sources.

To that end, drop all statements |\input{childdoc.def}|
and perform the replacements as outlined below.
Instead of |\childdocmain{|\textit{main}|}| add the following code
to the top of the main file:
%
\begin{center}
\begin{tabular}{l}
|\||ifdefined\childdocname\endinput\||fi\newif\ifchilddoc|\\
|\edef\childdocname{\scantokens\expandafter{\jobname\noexpand}}|\\
|\def\childdocmain{|\textit{main}|}\||ifx\childdocmain\childdocname\||else|\\
|\childdoctrue\includeonly{\childdocname}\let\jobname\childdocmain\||fi|\\
\end{tabular}
\end{center}
%
Instead of |\childdocof{|\textit{main}|}| just include the main file
at the top of each child file:
%
\begin{center}
|\input{|\textit{main}|}|
\end{center}
%
A simple redirection |\childdocforward{|\textit{dest}|}| is achieved by:
%
\begin{center}
|\def\jobname{|\textit{dest}|}\input{\jobname}|
\end{center}
%
The redirection with prefix
|\childdocforwardprefix[|\textit{prefix}|]{|\textit{dest}|}|
is accomplished by:
%
\begin{center}
\begin{tabular}{l}
|{\edef\jobname{\scantokens\expandafter{\jobname\noexpand}}|\\
|\def\redirectjob |\textit{prefix}|#1~~~{\gdef\jobname{|\textit{dest}|#1}}|\\
|\expandafter\redirectjob\jobname~~~}\input{\jobname}|
\end{tabular}
\end{center}

In an alternative approach,
child documents can be compiled by a specific command line
without additional code or specific definitions:
%
\begin{center}
|... -jobname "|\textit{target}|" "|[\textit{flags}]%
|\includeonly{|\textit{dest}|}\input{|\textit{main}|}"|
\end{center}
%

%%%%%%%%%%%%%%%%%%%%%%%%%%%%%%%%%%%%%%%%%%%%%%%%%%%%%%%%%%%%%%%%%%%%%%%%%%%%%%%%
%%%%%%%%%%%%%%%%%%%%%%%%%%%%%%%%%%%%%%%%%%%%%%%%%%%%%%%%%%%%%%%%%%%%%%%%%%%%%%%%
\section{Information}

%%%%%%%%%%%%%%%%%%%%%%%%%%%%%%%%%%%%%%%%%%%%%%%%%%%%%%%%%%%%%%%%%%%%%%%%%%%%%%%%
\subsection{Copyright}

Copyright \copyright{} 2017--2018 Niklas Beisert

This work may be distributed and/or modified under the
conditions of the \LaTeX{} Project Public License, either version 1.3
of this license or (at your option) any later version.
The latest version of this license is in
  \url{http://www.latex-project.org/lppl.txt}
and version 1.3 or later is part of all distributions of \LaTeX{}
version 2005/12/01 or later.

This work has the LPPL maintenance status `maintained'.

The Current Maintainer of this work is Niklas Beisert.

This work consists of the files |README.txt|, |childdoc.ins| and |childdoc.dtx|
as well as the derived files |childdoc.def|, |cdocsamp.tex|
with |cdocsch1.tex|, |cdocsch2.tex|, |cdocspt3.tex|, |cdocspt4.tex|,
|cdocsdrf.tex|, |cdocsfn1.tex|, |cdocsfn2.tex|
as well as |childdoc.pdf|.

%%%%%%%%%%%%%%%%%%%%%%%%%%%%%%%%%%%%%%%%%%%%%%%%%%%%%%%%%%%%%%%%%%%%%%%%%%%%%%%%
\subsection{Files and Installation}

The package consists of the files:
%
\begin{center}
\begin{tabular}{ll}
    |README.txt|   & readme file \\
    |childdoc.ins| & installation file \\
    |childdoc.dtx| & source file \\
    |childdoc.def| & definition file \\
    |cdocsamp.tex| & sample main file \\
    |cdocsch1.tex| & sample include file \\
    |cdocsch2.tex| & sample include file \\
    |cdocspt3.tex| & sample part file \\
    |cdocspt4.tex| & sample part file \\
    |cdocsdrf.tex| & sample redirection file \\
    |cdocsfn1.tex| & sample redirection file \\
    |cdocsfn2.tex| & sample redirection file \\
    |childdoc.pdf| & manual
\end{tabular}
\end{center}
%
The distribution consists of the files
|README.txt|, |childdoc.ins| and |childdoc.dtx|.
%
\begin{itemize}
\item
Run (pdf)\LaTeX{} on |childdoc.dtx|
to compile the manual |childdoc.pdf| (this file).
\item
Run \LaTeX{} on |childdoc.ins| to create the definitions file |childdoc.def|
and the sample |cdocsamp.tex| with include files
|cdocsch1.tex|, |cdocsch2.tex|, |cdocspt3.tex|, |cdocspt4.tex|,
|cdocsdrf.tex|, |cdocsfn1.tex|, |cdocsfn2.tex|.
Then copy the file |childdoc.def| to an appropriate directory of your \LaTeX{}
distribution, e.g.\ \textit{texmf-root}|/tex/latex/childdoc|.
\end{itemize}

%%%%%%%%%%%%%%%%%%%%%%%%%%%%%%%%%%%%%%%%%%%%%%%%%%%%%%%%%%%%%%%%%%%%%%%%%%%%%%%%
\subsection{Related CTAN Packages}

There are several other packages which offer a similar functionality:
%
\begin{itemize}
\item
The packages
\href{http://ctan.org/pkg/docmute}{\textsf{docmute}},
\href{http://ctan.org/pkg/includex}{\textsf{includex}} and
\href{http://ctan.org/pkg/standalone}{\textsf{standalone}}
provide commands to include only the document body of
a child file thus allowing both files to be compiled individually.
\item
The packages \href{http://ctan.org/pkg/subdocs}{\textsf{subdocs}}
and \href{http://ctan.org/pkg/subfiles}{\textsf{subfiles}}
provide structures in which the main and child documents can be
encapsulated and allowing them to be compiled individually.
The inclusion mechanism is different from the conventional |\include|.
\item
The package \href{http://ctan.org/pkg/combine}{\textsf{combine}}
is an elaborate solution to combine several documents into one.
\end{itemize}
%
See also the CTAN topic \href{http://ctan.org/topic/subdocs}{\textsf{subdocs}}
for further related packages.
The present package differs from the above solutions in that
a document structure constructed with the conventional |\include| mechanism
just needs two extra commands at the top of every file
such that all constituent files can be compiled individually.

%%%%%%%%%%%%%%%%%%%%%%%%%%%%%%%%%%%%%%%%%%%%%%%%%%%%%%%%%%%%%%%%%%%%%%%%%%%%%%%%
%\subsection{Feature Suggestions}
%
%The following is a list of features which may be useful for future
%versions of this package:
%%
%\begin{itemize}
%\item
%\ldots
%\end{itemize}

%%%%%%%%%%%%%%%%%%%%%%%%%%%%%%%%%%%%%%%%%%%%%%%%%%%%%%%%%%%%%%%%%%%%%%%%%%%%%%%%
\subsection{Revision History}

%%%%%%%%%%%%%%%%%%%%%%%%%%%%%%%%%%%%%%%%
\paragraph{v2.0:} 2018/12/30

\begin{itemize}
\item
immediate forward processing
\item
added |\childdocby| mechanism
\item
manual restructured
\end{itemize}

%%%%%%%%%%%%%%%%%%%%%%%%%%%%%%%%%%%%%%%%
\paragraph{v1.6:} 2018/01/17

\begin{itemize}
\item
application for development of include files
\item
corrections to manual
\end{itemize}

%%%%%%%%%%%%%%%%%%%%%%%%%%%%%%%%%%%%%%%%
\paragraph{v1.5:} 2017/05/21

\begin{itemize}
\item
more complete structuring introduced
\item
|\childdocof| introduced
\item
|\childdoc| renamed to |\childdocmain|
\item
|\childredirect| renamed to |\childdocforward| and |\childdocforwardprefix|
and functionality expanded
\end{itemize}

%%%%%%%%%%%%%%%%%%%%%%%%%%%%%%%%%%%%%%%%
\paragraph{v1.0:} 2017/04/27

\begin{itemize}
\item
manual and install package
\item
first version published on CTAN
\end{itemize}

%%%%%%%%%%%%%%%%%%%%%%%%%%%%%%%%%%%%%%%%
\paragraph{v0.6:} 2017/04/26

\begin{itemize}
\item
redirection mechanism added
\end{itemize}

%%%%%%%%%%%%%%%%%%%%%%%%%%%%%%%%%%%%%%%%
\paragraph{v0.5:} 2017/04/26

\begin{itemize}
\item
functionality in definition file
\end{itemize}


%%%%%%%%%%%%%%%%%%%%%%%%%%%%%%%%%%%%%%%%%%%%%%%%%%%%%%%%%%%%%%%%%%%%%%%%%%%%%%%%
%%%%%%%%%%%%%%%%%%%%%%%%%%%%%%%%%%%%%%%%%%%%%%%%%%%%%%%%%%%%%%%%%%%%%%%%%%%%%%%%
%%%%%%%%%%%%%%%%%%%%%%%%%%%%%%%%%%%%%%%%%%%%%%%%%%%%%%%%%%%%%%%%%%%%%%%%%%%%%%%%
\appendix

\settowidth\MacroIndent{\rmfamily\scriptsize 000\ }

 \DocInput{childdoc.dtx}

\end{document}
%</driver>
% \fi
%
% %%%%%%%%%%%%%%%%%%%%%%%%%%%%%%%%%%%%%%%%%%%%%%%%%%%%%%%%%%%%%%%%%%%%%%%%%%%%%%
% %%%%%%%%%%%%%%%%%%%%%%%%%%%%%%%%%%%%%%%%%%%%%%%%%%%%%%%%%%%%%%%%%%%%%%%%%%%%%%
% \section{Sample}
%\iffalse
%<*samplemain>
%\fi
%
% The following presents a sample document
% with two chapters, two parts, a title page,
% a compile flag as well as three forwarding files to set the flag.
% It consists of eight |.tex| files:
% \begin{center}
% \begin{tabular}{ll}
% |cdocsamp.tex|&main file\\
% |cdocsch1.tex|&include file for chapter 1\\
% |cdocsch2.tex|&include file for chapter 2\\
% |cdocspt3.tex|&include file for part 3\\
% |cdocspt4.tex|&include file for part 4\\
% |cdocsdrf.tex|&forwarding file for main file in draft mode\\
% |cdocsfi1.tex|&forwarding file for final version of chapter 1\\
% |cdocsfi2.tex|&forwarding file for final version of chapter 2\\
% \end{tabular}
% \end{center}
% Each of the eight files can be compiled directly by the \LaTeX{} compiler.
%
% %%%%%%%%%%%%%%%%%%%%%%%%%%%%%%%%%%%%%%
% \paragraph{Main File.}
%
% The main file is called |cdocsamp.tex|.
%
% Load the \textsf{childdoc} definitions and
% declare the filename for the main document:
%    \begin{macrocode}
\input{childdoc.def}
\childdocmain{}
%    \end{macrocode}

% Optional override for |\version| flag:
%    \begin{macrocode}
%%\ifchilddoc\else\providecommand{\version}{draft}\fi
%    \end{macrocode}

% Define the default values for the |\version| flag
% (|final| for the main file and |draft| for childs):
%    \begin{macrocode}
\ifchilddoc
\providecommand{\version}{draft}
\else
\providecommand{\version}{final}
\fi
%    \end{macrocode}

% Load the standard document class:
%    \begin{macrocode}
\documentclass[12pt]{article}
%    \end{macrocode}

% Start the document body:
%    \begin{macrocode}
\begin{document}
%    \end{macrocode}

% Declare a title page.
% Print title, part of document being processed and version flag:
%    \begin{macrocode}
\addtocounter{page}{-1}
\begin{center}
{\LARGE\bfseries{}childdoc example\par}
\vspace{1cm}
\ifchilddoc
\ifchilddocmanual part\else chapter\fi:
`\childdocname' of `\childdocjob'\par
\else
main document: `\childdocjob'\par
\fi
version: \version\par
\end{center}
\newpage
%    \end{macrocode}

% Manually include selected file,
% otherwise process as usual:
%    \begin{macrocode}
\ifchilddocmanual
\section*{part `\childdocname'}
\input{\childdocname}
\else
%    \end{macrocode}

% Include the two chapters:
%    \begin{macrocode}
\include{cdocsch1}
\include{cdocsch2}
%    \end{macrocode}

% Include the two parts unless only chapters should be displayed:
%    \begin{macrocode}
\ifchilddoc\else
\section{part three}
\input{cdocspt3}
\section{part four}
\input{cdocspt4}
\fi
%    \end{macrocode}

% Process as usual until here:
%    \begin{macrocode}
\fi
%    \end{macrocode}

% End of document body:
%    \begin{macrocode}
\end{document}
%    \end{macrocode}
%\iffalse
%</samplemain>
%\fi
%
% %%%%%%%%%%%%%%%%%%%%%%%%%%%%%%%%%%%%%%
% \paragraph{Chapter Include Files.}
%
% The include files are called |cdocsch1.tex| and |cdocsch2.tex|.
%
%\iffalse
%<*samplechap1|samplechap2>
%\fi

% Optional override for |\version| flag:
%    \begin{macrocode}
%%\providecommand{\version}{final}
%    \end{macrocode}

% Include the main document:
%    \begin{macrocode}
\input{childdoc.def}
\childdocof{cdocsamp}
%    \end{macrocode}

%\iffalse
%</samplechap1|samplechap2>
%\fi
%
%\iffalse
%<*samplechap1>
%\fi
% Some text for chapter 1:
%    \begin{macrocode}
\section{one}
some text in chapter one
%    \end{macrocode}

%\iffalse
%</samplechap1>
%\fi
% Some text for chapter 2:
%\iffalse
%<*samplechap2>
%\fi
%    \begin{macrocode}
\section{two}
more text in chapter two
%    \end{macrocode}

%\iffalse
%</samplechap2>
%\fi
%
% %%%%%%%%%%%%%%%%%%%%%%%%%%%%%%%%%%%%%%
% \paragraph{Part Include Files.}
%
% The include files are called |cdocspt3.tex| and |cdocspt4.tex|.
%
%\iffalse
%<*samplepart3|samplepart4>
%\fi

% Optional override for |\version| flag:
%    \begin{macrocode}
%%\providecommand{\version}{final}
%    \end{macrocode}

% Include the main document:
%    \begin{macrocode}
\input{childdoc.def}
\childdocby{cdocsamp}
%    \end{macrocode}

%\iffalse
%</samplepart3|samplepart4>
%\fi
%
%\iffalse
%<*samplepart3>
%\fi
% Some text for part 3:
%    \begin{macrocode}
some text in part three
%    \end{macrocode}

%\iffalse
%</samplepart3>
%\fi
% Some text for part 4:
%\iffalse
%<*samplepart4>
%\fi
%    \begin{macrocode}
more text in part four
%    \end{macrocode}

%\iffalse
%</samplepart4>
%\fi
%
% %%%%%%%%%%%%%%%%%%%%%%%%%%%%%%%%%%%%%%
% \paragraph{Forwarding for a Complete Draft.}
%
% The following forwarding file |cdocsdrf.tex|
% compiles the main document in draft mode:
%\iffalse
%<*sampledraft>
%\fi
%    \begin{macrocode}
\def\version{draft}
\input{childdoc.def}
\childdocforward{cdocsamp}
%    \end{macrocode}

%\iffalse
%</sampledraft>
%\fi
%
% %%%%%%%%%%%%%%%%%%%%%%%%%%%%%%%%%%%%%%
% \paragraph{Forwarding for Final Version of the Chapters.}
%
% The following forwarding files |cdocsfn1.tex| and |cdocsfn2.tex|
% (with identical content)
% compile the final versions of the child documents
% |cdocsch1.tex| and |cdocsch2.tex|, respectively:
%\iffalse
%<*samplefinal>
%\fi
%    \begin{macrocode}
\def\version{final}
\input{childdoc.def}
\childdocforwardprefix[cdocsamp]{cdocsfn}{cdocsch}
%    \end{macrocode}

%\iffalse
%</samplefinal>
%\fi
%
% %%%%%%%%%%%%%%%%%%%%%%%%%%%%%%%%%%%%%%
% \paragraph{Command Line Processing.}
%
% The following three command lines generate the output files
% |cdocscld|, |cdocscl1| and |cdocscl2|
% which should be identical to
% |cdocsdrf|, |cdocsch1| and |cdocsfn2|, respectively:
% \begin{center}
% \begin{tabular}{l}
% |latex -jobname cdocscld \|\\
% |  "\def\version{draft}\input{childdoc.def}\childdocforward{cdocsamp}"|\\
% |latex -jobname cdocscl1 \|\\
% |  "\input{childdoc.def}\childdocforward[cdocsamp]{cdocsch1}"|\\
% |latex -jobname cdocscl2 \|\\
% |  "\def\version{final}\input{childdoc.def}\childdocforward{cdocsch2}"|
% \end{tabular}
% \end{center}
% Note that the trailing backslash on each first line
% merely continues the input to the second line
% (for convenient cut ant paste).
% Furthermore, the command |latex| can be replaced by any
% of its alternative versions such as |pdflatex|.
%
% %%%%%%%%%%%%%%%%%%%%%%%%%%%%%%%%%%%%%%%%%%%%%%%%%%%%%%%%%%%%%%%%%%%%%%%%%%%%%%
% %%%%%%%%%%%%%%%%%%%%%%%%%%%%%%%%%%%%%%%%%%%%%%%%%%%%%%%%%%%%%%%%%%%%%%%%%%%%%%
% \section{Implementation}
%\iffalse
%<*package>
%\fi
%
% This section describes the definitions file |childdoc.def|.

% The definitions cannot be loaded using |\usepackage| or |\RequirePackage|
% which has a mechanism to prevent loading a style file more than once.
% When loading the definitions by means of |\input|
% multiple instances have to be prevented manually:
%\iffalse
%This code needs to be before the `\ProvidesFile' directive
%which is defined at the beginning of this file.
%Therefore it is also placed there and commented out here.
%</package>
%<*discard>
%\fi
%    \begin{macrocode}
\ifdefined\childdocmain\endinput\fi
%    \end{macrocode}
%\iffalse
%</discard>
%<*package>
%\fi
%
% \macro{\ifchilddoc}
% \macro{\ifchilddocmanual}
% The conditional |\ifchilddoc| tells whether a
% child (true) or main (false) document is being compiled.
% The conditional |\ifchilddocmanual| tells whether
% the |\includeonly| mechanism is used (false) or
% the selection of child files must be performed manually (true).
% The definitions initialise to false:
%    \begin{macrocode}
\newif\ifchilddoc
\newif\ifchilddocmanual
%    \end{macrocode}

% \macro{\childdocname}
% \macro{\childdocjob}
% The macro |\childdocname| stores the name of the main document
% to be compiled. The macro |\childdocjob| stores the name of
% the document on which the \LaTeX{} compiler was originally invoked.
% The content of |\jobname| cannot be compared
% to filenames specified in the source due to different catcodes.
% The following code rescans |\jobname|, stores the result
% in |\childdocname| and saves a copy in |\childdocjob|:
%    \begin{macrocode}
\edef\childdocname{\scantokens\expandafter{\jobname\noexpand}}
\let\childdocjob\childdocname
%    \end{macrocode}

% \macro{\childdocdisable}
% The macro |\childdocdisable| prevents the main file
% from being processed more than once.
% At this stage, the main document command |\childdocmain|
% is assumed to be called once again where it should do nothing.
% Any subsequent call to it should prevent
% a secondary processing of the main document
% It overwrites the forwarding commands
% |\childdocof| and |\childdocforward|
% with empty macros to prevent further inclusions of the main document:
%    \begin{macrocode}
\newcommand{\childdocdisable}
{
  \renewcommand{\childdocmain}[1]{\renewcommand{\childdocmain}[1]{\endinput}}
  \renewcommand{\childdocof}[1]{}
  \renewcommand{\childdocby}[2][]{}
  \renewcommand{\childdocforward}[2][]{}
  \renewcommand{\childdocdisable}{}
}
%    \end{macrocode}

% \macro{\childdocmain}
% The macro |\childdocmain| is to be called at the top of the main file
% with nothing or the main filename (without extension) as argument.
% First, it breaks loops.
% If the argument is not empty and does not match |\childdocname|
% (which is set by the first inclusion of |childdoc.def|),
% |\ifchilddoc| is set to true, |\includeonly| is applied to the child file
% and |\jobname| is set to the main file
% (for proper handling of |.aux| files):
%    \begin{macrocode}
\newcommand{\childdocmain}[1]
{
  \childdocdisable\childdocmain{}
  \if?#1?\else
    \begingroup
      \def\childdoctmp{#1}
      \ifx\childdoctmp\childdocname
        \def\childdoctmp{}
      \else
        \def\childdoctmp
        {
          \childdoctrue
          \includeonly{\childdocname}
          \def\childdocjob{#1}
          \def\jobname{#1}
        }
      \fi
      \expandafter
    \endgroup
    \childdoctmp
  \fi
}
%    \end{macrocode}

% \macro{\childdocof}
% The command |\childdocof| redirects
% compilation to the main file |#1|.
%    \begin{macrocode}
\newcommand{\childdocof}[1]
{
  \childdocdisable
  \childdoctrue
  \includeonly{\childdocname}
  \def\jobname{#1}
  \def\childdocjob{#1}
  \input{#1}
}
%    \end{macrocode}

% \macro{\childdocby}
% The command |\childdocby| ....
%    \begin{macrocode}
\newcommand{\childdocby}[2][]
{
  \childdocdisable
  \childdoctrue
  \childdocmanualtrue
  \if?#1?\else
    \def\jobname{#2}
  \fi
  \def\childdocjob{#2}
  \input{#2}
  \endinput
}
%    \end{macrocode}

% \macro{\childdocforward}
% The command |\childdocforward| redirects
% compilation to the main file or
% (if the optional argument is given) a child file.
% Parameters are set as if the main file
% or a child file starting with |\childdocof| was compiled.
% Then compilation is handed over to the main file:
%    \begin{macrocode}
\newcommand{\childdocforward}[2][]
{
  \begingroup
    \if?#1?
      \def\childdoctmp
      {
        \def\childdocname{#2}
        \def\childdocjob{#2}
        \def\jobname{#2}
        \input{#2}
        \endinput
      }
    \else
      \def\childdoctmp
      {
        \childdocdisable
        \def\childdocname{#2}
        \childdoctrue
        \includeonly{#2}
        \def\childdocjob{#1}
        \def\jobname{#1}
        \input{#1}
        \endinput
      }
    \fi
    \expandafter
  \endgroup
  \childdoctmp
}
%    \end{macrocode}

% \macro{\childdocforwardprefix}
% The command |\childdocforwardprefix| redirects
% compilation to the main or a child file by means of a pattern.
% The prefix |#1| in the current filename is replaced by |#2|
% and the suffix of the current filename is kept
% (it is assumed that the filename does not contain the substring `|~~~|'
% which is used as a delimiter).
% Compilation is handed over to the new file by |\childdocforward|:
%    \begin{macrocode}
\newcommand{\childdocforwardprefix}[3][]
{
  \begingroup
    \def\childdocextract #2##1~~~{\def\childdoctmp{\childdocforward[#1]{#3##1}}}
    \expandafter\childdocextract\childdocname~~~
    \expandafter
  \endgroup
  \childdoctmp
}
%    \end{macrocode}

% \macro{\childdoc}
% The deprecated macro |\childdoc| is a legacy version of |\childdocmain|:
%    \begin{macrocode}
\newcommand{\childdoc}{\childdocmain}
%    \end{macrocode}

% \macro{\childdocredirect}
% The deprecated macro |\childdocredirect| is a legacy version
% of |\childdocforward| and |\childdocforwardprefix|:
%    \begin{macrocode}
\newcommand{\childdocredirect}[2][]
{
  \begingroup
    \if?#1?
      \def\childdoctmp{\childdocforward{#2}}
    \else
      \def\childdoctmp{\childdocforwardprefix{#1}{#2}}
    \fi
    \expandafter
  \endgroup
  \childdoctmp
}
%    \end{macrocode}

%\iffalse
%</package>
%\fi
%
\endinput

\childdocforwardprefix[cdocsamp]{cdocsfn}{cdocsch}
%    \end{macrocode}

%\iffalse
%</samplefinal>
%\fi
%
% %%%%%%%%%%%%%%%%%%%%%%%%%%%%%%%%%%%%%%
% \paragraph{Command Line Processing.}
%
% The following three command lines generate the output files
% |cdocscld|, |cdocscl1| and |cdocscl2|
% which should be identical to
% |cdocsdrf|, |cdocsch1| and |cdocsfn2|, respectively:
% \begin{center}
% \begin{tabular}{l}
% |latex -jobname cdocscld \|\\
% |  "\def\version{draft}% \iffalse
%
% childdoc.dtx Copyright (C) 2017-2018 Niklas Beisert
%
% This work may be distributed and/or modified under the
% conditions of the LaTeX Project Public License, either version 1.3
% of this license or (at your option) any later version.
% The latest version of this license is in
%   http://www.latex-project.org/lppl.txt
% and version 1.3 or later is part of all distributions of LaTeX
% version 2005/12/01 or later.
%
% This work has the LPPL maintenance status `maintained'.
%
% The Current Maintainer of this work is Niklas Beisert.
%
% This work consists of the files childdoc.dtx and childdoc.ins
% and the derived files childdoc.def and cdocsamp.tex with
% cdocsch1.tex, cdocsch2.tex, cdocsdrf.tex, cdocsfn1.tex, cdocsfn2.tex.
%
%<package>\ifdefined\childdocmain\endinput\fi
%<package>\ProvidesFile{childdoc.def}[2018/12/30 v2.0 child document driver]
%<samplemain>\ProvidesFile{cdocsamp.tex}[2018/12/30 v2.0 sample for childdoc]
%<*driver>
%\ProvidesFile{childdoc.drv}[2018/12/30 v2.0 childdoc reference manual file]
\PassOptionsToClass{10pt,a4paper}{article}
\documentclass{ltxdoc}

\usepackage[margin=35mm]{geometry}
\usepackage{hyperref}
\usepackage{hyperxmp}
\usepackage[usenames]{color}

\hypersetup{colorlinks=true}
\hypersetup{pdfstartview=FitH}
\hypersetup{pdfpagemode=UseNone}
\hypersetup{pdfsource={}}
\hypersetup{pdflang={en-UK}}
\hypersetup{pdfcopyright={Copyright 2017-2018 Niklas Beisert.
  This work may be distributed and/or modified under the
  conditions of the LaTeX Project Public License, either version 1.3
  of this license or (at your option) any later version.}}
\hypersetup{pdflicenseurl={http://www.latex-project.org/lppl.txt}}
\hypersetup{pdfcontactaddress={ETH Zurich, ITP, HIT K,
  Wolfgang-Pauli-Strasse 27}}
\hypersetup{pdfcontactpostcode={8093}}
\hypersetup{pdfcontactcity={Zurich}}
\hypersetup{pdfcontactcountry={Switzerland}}
\hypersetup{pdfcontactemail={nbeisert@itp.phys.ethz.ch}}
\hypersetup{pdfcontacturl={http://people.phys.ethz.ch/\xmptilde nbeisert/}}

\newcommand{\secref}[1]{\hyperref[#1]{section \ref*{#1}}}

\parskip1ex
\parindent0pt
\let\olditemize\itemize
\def\itemize{\olditemize\parskip0pt}

\begin{document}

\title{The \textsf{childdoc} Package}
\hypersetup{pdftitle={The childdoc Package}}
\author{Niklas Beisert\\[2ex]
  Institut f\"ur Theoretische Physik\\
  Eidgen\"ossische Technische Hochschule Z\"urich\\
  Wolfgang-Pauli-Strasse 27, 8093 Z\"urich, Switzerland\\[1ex]
  \href{mailto:nbeisert@itp.phys.ethz.ch}
  {\texttt{nbeisert@itp.phys.ethz.ch}}}
\hypersetup{pdfauthor={Niklas Beisert}}
\hypersetup{pdfsubject={Manual for the LaTeX2e Package childdoc}}
\date{30 December 2018, \textsf{v2.0}}
\maketitle

\begin{abstract}\noindent
\textsf{childdoc} is a \LaTeXe{} package
that enables the direct compilation
of document sections included by |\include|
to individual files.
\end{abstract}

\begingroup
\parskip0ex
\tableofcontents
\endgroup

%%%%%%%%%%%%%%%%%%%%%%%%%%%%%%%%%%%%%%%%%%%%%%%%%%%%%%%%%%%%%%%%%%%%%%%%%%%%%%%%
%%%%%%%%%%%%%%%%%%%%%%%%%%%%%%%%%%%%%%%%%%%%%%%%%%%%%%%%%%%%%%%%%%%%%%%%%%%%%%%%
\section{Introduction}

\LaTeX{} provides a mechanism to structure a large document (such as a book)
into a main file and several child files (containing the chapters)
using the |\include| command.
This mechanism is beneficial for documents
which span hundreds of pages in order to
make the source file(s) more manageable.
Moreover, compilation can be restricted to
selected child files by means of the |\includeonly| command.
The latter feature can be used to reduce the compilation time while editing
(this was significantly more useful in the earlier days of \LaTeX{})
or to generate a smaller document which is easier to navigate.
Another application of |\includeonly| is to generate
documents consisting of selected parts of the complete document.

However, there are a few drawbacks of the plain |\include| mechanism:
\begin{itemize}
\item
The child files cannot be compiled on their own,
they can only be compiled via the main file.
A naive editing environment
(such as a text editor with an option
to have the current file processed by \LaTeX)
may require one to switch to the main file before compiling;
attempting to compile the child file produces errors.
\item
The main file must be modified (each time)
to adjust the |\includeonly| command
to the present needs. This easily leaves the main file in a messy state.
\item
The generated document will always carry the filename
of the main document. This is inconvenient if
several child files are to be compiled and
to be kept for distribution.
\end{itemize}

The present package provides a simple interface
to make child files individually compilable by \LaTeX{}.
Compiling a child file then has the same effect as compiling
the main file with an |\includeonly| command
to select the appropriate child.
Moreover the generated document will carry the name of the child
rather than the main file.
This resolves all three above issues.

This feature is meant to make the editing of books,
thesis documents and lecture notes somewhat more convenient.
However, the package can also be used efficiently for
composing a series of documents (such as exercise sheets)
which are typically distributed individually.
It then assists the author in generating the individual documents
(potentially in different versions)
as well as a document containing the collected series.
Another application is in developing style files
or other kinds of included material
where compilation of the style file could redirect
to a sample or test file.

%%%%%%%%%%%%%%%%%%%%%%%%%%%%%%%%%%%%%%%%%%%%%%%%%%%%%%%%%%%%%%%%%%%%%%%%%%%%%%%%
%%%%%%%%%%%%%%%%%%%%%%%%%%%%%%%%%%%%%%%%%%%%%%%%%%%%%%%%%%%%%%%%%%%%%%%%%%%%%%%%
\section{Usage}

First of all, the package \textsf{childdoc} is \emph{not} a standard
\LaTeXe{} |.sty| style file! Therefore it needs to be invoked in
a non-standard way.

%%%%%%%%%%%%%%%%%%%%%%%%%%%%%%%%%%%%%%%%%%%%%%%%%%%%%%%%%%%%%%%%%%%%%%%%%%%%%%%%
\subsection{Included Files}
\label{sec:include}

%%%%%%%%%%%%%%%%%%%%%%%%%%%%%%%%%%%%%%%%
\DescribeMacro{\childdocmain}
To use the package, add the commands
\begin{center}
\begin{tabular}{l}
|\input{childdoc.def}|\\
|\childdocmain{}|\\
\end{tabular}
\end{center}
at the very top of the main \LaTeX{} file,
in particular \emph{before} the |\documentclass| statement!
The argument of |\childdocmain| should be left empty
(but it must be present).

%%%%%%%%%%%%%%%%%%%%%%%%%%%%%%%%%%%%%%%%
\DescribeMacro{\childdocof}
Furthermore, add the commands
\begin{center}
\begin{tabular}{l}
|\input{childdoc.def}|\\
|\childdocof{|\textit{main}|}|\\
\end{tabular}
\end{center}
at the top of every child file \textit{child}
which is included by |\include{|\textit{child}|}|
from within the main file
(or at least for those files to be compiled individually).
The argument \textit{main} must be the filename of the main file.

There are a couple of
considerations in setting up the main and child documents:

%%%%%%%%%%%%%%%%%%%%%%%%%%%%%%%%%%%%%%%%
\paragraph{Restrictions.}

Please note the following restrictions:
\begin{itemize}
\item
|\childdocmain| must be called with one argument \textit{main}
to ensure compatibility with earlier version of the package.
It must either be empty (|\childdocmain{}|)
or precisely match the filename of the main file in which it is specified.
See \secref{sec:detection} for further information.
\item
The filename \textit{main} must be specified without the |.tex| extension.
\item
The filename \textit{main} is case sensitive
(even in case-insensitive file systems)
due to internal string comparison.
\item
The argument \textit{main} should be fully expanded, it cannot be a macro.
\item
Subdirectories and special characters should be avoided in filenames.
\item
The command |\childdocmain{|\textit{main}|}| must be followed by a whitespace.
It should not be followed immediately by another command
or by a comment mark `|%|'.
This is because the \TeX{} parser reads the token immediately following
the argument of |\childdocmain| and puts it
at the beginning of every child section;
however, a white\-space is ignored.
\end{itemize}

%%%%%%%%%%%%%%%%%%%%%%%%%%%%%%%%%%%%%%%%
\paragraph{Content of Main File.}

It is advisable to place all content in the child files included by |\include|.
Any output contained in the main file will appear in all child documents
unless suppressed manually;
it cannot be suppressed automatically by the |\includeonly| directive
and thus should normally be avoided.
A method to include some content in the main file
by means of conditional processing is described in \secref{sec:conditional}.

%%%%%%%%%%%%%%%%%%%%%%%%%%%%%%%%%%%%%%%%
\paragraph{Page Numbering.}

When only a part of the document is compiled,
the appropriate numbering of pages
(as well as other status parameters)
is determined from the |.aux| files.
The latter contain information from previous passes.
However this information needs to propagate through
all intermediate child documents.
Therefore the page numbering in child documents may well
be inconsistent until the complete document is compiled at least once.

A useful (if unconventional) way to always ensure a consistent
page numbering is to restart the numbering in each child document
and denote the pages by `\textit{child}|.|\textit{page}'
where \textit{child} represents the chapter/section number of the child file.
This can be achieved by the command
|\numberwithin{page}{|\textit{child}|}|
of the \textsf{amsmath} package
where \textit{child} can be |chapter| or |section|
depending on the chosen structuring.
Alternatively, one can modify the macro |\thepage| appropriately
and reset the counter |page| at the start of each child file.

%%%%%%%%%%%%%%%%%%%%%%%%%%%%%%%%%%%%%%%%%%%%%%%%%%%%%%%%%%%%%%%%%%%%%%%%%%%%%%%%
\subsection{Conditional Processing}
\label{sec:conditional}

The package provides a mechanism to compile different versions
of a document. To customise the versions further some conditional processing
can come in handy to distinguish which version is being compiled.
The package provides two macros to describe the compilation context:

%%%%%%%%%%%%%%%%%%%%%%%%%%%%%%%%%%%%%%%%
\DescribeMacro{\ifchilddoc}
The conditional |\ifchilddoc| distinguishes between the compilation of
child documents and the main document:
%
\begin{center}
|\ifchilddoc |\textit{child-code}| |[|\||else |\textit{main-code}]| \||fi|
\end{center}

%%%%%%%%%%%%%%%%%%%%%%%%%%%%%%%%%%%%%%%%
\DescribeMacro{\childdocname}
\DescribeMacro{\childdocjob}
The macro |\childdocname| contains the filename (without extension)
of the main or child file being processed.
Note that |\childdocjob| will always contain the name of the main file.

%%%%%%%%%%%%%%%%%%%%%%%%%%%%%%%%%%%%%%%%
\paragraph{Title Page.}

Conditional processing can be used to include a title or banner page
in the main document when proper precautions are taken.
Importantly, the code in the main file should ensure that the page counter
(as well as other status parameters which are stored in the |.aux| files)
takes the same value after the conditional processing.
Otherwise the page numbers may take divergent values
depending on which part is compiled.

For example, a title page could be declared by:
%
\begin{center}
\begin{tabular}{l}
|\ifchilddoc\||else|\\
|\addtocounter{page}{-1}|\\
\textit{code for title page}\\
|\newpage|\\
|\||fi|
\end{tabular}
\end{center}
%
A banner page for the child documents can be generated by:
%
\begin{center}
\begin{tabular}{l}
|\ifchilddoc|\\
|\addtocounter{page}{-1}|\\
\textit{code for banner page}\\
|\newpage|\\
|\||fi|
\end{tabular}
\end{center}
%
Here one could write a message such as:
\begin{center}
|This is the part \childdocname{} of \childdocjob{}.|
\end{center}

%%%%%%%%%%%%%%%%%%%%%%%%%%%%%%%%%%%%%%%%%%%%%%%%%%%%%%%%%%%%%%%%%%%%%%%%%%%%%%%%
\subsection{Flags}
\label{sec:flags}

The package makes it easy to generate different versions
of the main or child documents.
To this end compilation flags can be defined
and assigned different default values.
They will be particularly useful in conjunction
with the forwarding mechanism described in \secref{sec:forward}.

For example, it may be useful to have a flag |\version|
which can be set to |draft| or |final|.
The document source will contain some conditional code
depending on the value of |\version|.
Suppose further, the flag should default to |final| for the main file
and to |draft| for child files
which is a natural assignment for editing the document.
This is achieved by placing the following code
in the preamble of the main document
(below the |\childdocmain| directive):
%
\begin{center}
\begin{tabular}{l}
|\ifchilddoc|\\
|\providecommand{\version}{draft}|\\
|\||else|\\
|\providecommand{\version}{final}|\\
|\||fi|
\end{tabular}
\end{center}
%
The definition by |\providecommand| makes sure
that previous definitions are not overwritten.
Further statements |\providecommand{\version}{...}|
can thus be added before the above code to override it.

For the main file, one might add a line
(between |\childdocmain| and the above block)
%
\begin{center}
|%\ifchilddoc\||else\providecommand{\version}{draft}\||fi|
\end{center}
%
which can be uncommented to produce a draft version.
Likewise one can add a line to the very top of a child file
(above the |\childdocof{|\textit{main}|}| directive)
%
\begin{center}
|%\providecommand{\version}{final}|
\end{center}
%
which can be uncommented to produce the final version of this child document.

%%%%%%%%%%%%%%%%%%%%%%%%%%%%%%%%%%%%%%%%%%%%%%%%%%%%%%%%%%%%%%%%%%%%%%%%%%%%%%%%
\subsection{Forwarding}
\label{sec:forward}

Different versions of the main or child documents
using compilation flags as described in \secref{sec:flags}
can be (permanently) stored in different files
for convenient compilation, viewing and distribution.
To this end, the package defines a command
to pass on compilation to a different file:

%%%%%%%%%%%%%%%%%%%%%%%%%%%%%%%%%%%%%%%%
\DescribeMacro{\childdocforward}
The command |\childdocforward| redirects processing to
another source file:
%
\begin{center}
\begin{tabular}{l}
|\input{childdoc.def}|\\
|\childdocforward[|\textit{main}|]{|\textit{dest}|}|\\
\end{tabular}
\end{center}
%
The argument \textit{dest} is the destination file
(without extension).
It should be the main file or one of the child files.
Note that further \textsf{childdoc} directives
such as |\childdocof| and |\childdocforward|
in the indicated file will be processed in this form.
The optional argument \textit{main}
passes on directly to the main file \textit{main}
while pretending to compile the child \textit{dest}.
This form behaves as if \textit{dest}
issues |\childdocof{|\textit{main}|}| right away,
and no further \textsf{childdoc} directives will be processed.

%%%%%%%%%%%%%%%%%%%%%%%%%%%%%%%%%%%%%%%%
\DescribeMacro{\...prefix}
In the alternative form |\childdocforwardprefix|,
%
\begin{center}
\begin{tabular}{l}
|\input{childdoc.def}|\\
|\childdocforwardprefix[|\textit{main}|]{|\textit{prefix}|}{|\textit{dest}|}|
\end{tabular}
\end{center}
%
the destination file is determined by a pattern
depending on the current file:
To make this work, the current file must be called
`{\textit{prefix}\hspace{0.2em}\textit{suffix}}'
with \textit{prefix} matching precisely the argument.
Processing is then passed on to the file
`{\textit{dest}\hspace{0.2em}\textit{suffix}}'.
Surely, the same effect is achieved by
directly specifying the
argument `{\textit{dest}\hspace{0.2em}\textit{suffix}}'
in the first form.
However, that requires to set up a different file
for each child. With the alternative form of the command
all these files can have exactly the same content
which simplifies setting them up and maintaining them.

For example, the following file |draft.tex|
with a compilation flag |\version| as described in \secref{sec:flags}
compiles the main document as a draft:
%
\begin{center}
\begin{tabular}{l}
|\def\version{draft}|\\
|\input{childdoc.def}|\\
|\childdocforward{|\textit{main}|}|
\end{tabular}
\end{center}
%
Likewise, the following files |final|\textit{nn}|.tex|
compile the final version of the child document
|child|\textit{nn}|.tex|:
%
\begin{center}
\begin{tabular}{l}
|\def\version{final}|\\
|\input{childdoc.def}|\\
|\childdocforwardprefix{final}{child}|
\end{tabular}
\end{center}
%

Note that when several versions of a main file and/or of each child file
are to be generated, it may be convenient to set up a |Makefile| or
shell script to automatise the process.

%%%%%%%%%%%%%%%%%%%%%%%%%%%%%%%%%%%%%%%%%%%%%%%%%%%%%%%%%%%%%%%%%%%%%%%%%%%%%%%%
\subsection{Command Line Processing}
\label{sec:commandline}

The effect of redirection files can also be achieved by invoking
the \LaTeX{} compiler with a more elaborate command line.
Most conveniently this should be done as part
of a shell script or a |Makefile|.

When using \textsf{childdoc} in the main file, the following
command lines effectively perform a redirection
(note that depending on the shell being used,
backslashes may have to be doubled: `|\|' $\to$ `|\\|'):
%
\begin{center}
|... -jobname "|\textit{target}|" |\\|"|[\textit{flags}]%
|\input{childdoc.def}\childdocforward[|\textit{main}|]{|\textit{dest}|}"|
\end{center}
%
Here \textit{target} is the name of the output file,
\textit{main} is the name of the main file
and \textit{dest} is the name of the main or child file to be processed
(all filenames without extensions).
The optional argument \textit{main} can be omitted
if \textit{main} matches \textit{dest}.
Optionally, compilation \textit{flags} can be defined via |\def| commands.
This command line makes the \TeX{} engine believe
it is compiling the file \textit{target}
whose content is specified as the latter parameter.
The provided code then forwards the processing to
\textit{main} or \textit{dest} as described in \secref{sec:forward}.

%%%%%%%%%%%%%%%%%%%%%%%%%%%%%%%%%%%%%%%%%%%%%%%%%%%%%%%%%%%%%%%%%%%%%%%%%%%%%%%%
\subsection{Include by Input}
\label{sec:input}

Including child documents by |\include| has some restrictions by design.
Most notably, the content of a child document always occupies
its own set of pages; pages cannot be shared between child documents.
Usually, this behaviour makes perfect sense
because each child document contain an essential part of the document.
However, in some situations it may be desirable to compose
a document from a collection of parts
without having mandatory page breaks between then.
For this case, the package
provides a mechanism to include parts
by |\input| which can also be processed individually.
However, by construction this mechanism
requires manual handling of the content to be output.

%%%%%%%%%%%%%%%%%%%%%%%%%%%%%%%%%%%%%%%%
\DescribeMacro{\ifchilddocmanual}
The main file should be prepared as usual, see \secref{sec:include}.
However, the document body must make a distinction
between processing of an individual part and of the main document, e.g.:
%
\begin{center}
\begin{tabular}{l}
|\ifchilddocmanual|\\
|\input{\childdocname}|\\
|\||else|\\
\textit{document body with }|\input{|\textit{part}|}|\\
|\||fi|
\end{tabular}
\end{center}
%
The conditional |\ifchilddocmanual| is true whenever
a part to be included by |\input| is being compiled,
and the name of the part is stored in |\childdocname|.

%%%%%%%%%%%%%%%%%%%%%%%%%%%%%%%%%%%%%%%%
\DescribeMacro{\childdocby}
Each part to be included by |\input| should start with:
%
\begin{center}
\begin{tabular}{l}
|\input{childdoc.def}|\\
|\childdocby{|\textit{main}|}|\\
\end{tabular}
\end{center}
%
The directive |\childdocby| is similar to |\childdocof|
described in \secref{sec:include},
but the subsequent selection of content must be done manually.
To that end, both |\ifchilddoc| and |\ifchilddocmanual|
will be true upon processing of a part,
and the name of the part is stored in |\childdocname|.
Note that |\jobname| will be set to the filename of the current part
so that each part receives an individual |.aux| file
that does not interfere with the |.aux| file(s) of the main document.
This behaviour can be altered by the alternative form
|\childdocby[*]{|\textit{main}|}| (with a non-empty optional argument)
which uses the |.aux| file of the main document
by setting |\jobname| to \textit{main}.

%%%%%%%%%%%%%%%%%%%%%%%%%%%%%%%%%%%%%%%%%%%%%%%%%%%%%%%%%%%%%%%%%%%%%%%%%%%%%%%%
\subsection{Driver Development}
\label{sec:driver}

The \textsf{childdoc} mechanism can also be use for the development
of definition files such as \LaTeX{} styles or classes.
This case differs from the above setup with multiple parts
included by |\include| in that no |\includeonly| should be invoked.
This can be achieved by starting the include file
(before |\ProvidesPackage|) with:
%
\begin{center}
\begin{tabular}{l}
|\input{childdoc.def}|\\
|\childdocforward{|\textit{main}|}|\\
\end{tabular}
\end{center}
%
or alternatively with:
%
\begin{center}
\begin{tabular}{l}
|\input{childdoc.def}|\\
|\childdocby{|\textit{main}|}|\\
\end{tabular}
\end{center}
%
Both forms have slightly different effects as described above.
The main file is prepared as usual, see \secref{sec:include}.

%%%%%%%%%%%%%%%%%%%%%%%%%%%%%%%%%%%%%%%%%%%%%%%%%%%%%%%%%%%%%%%%%%%%%%%%%%%%%%%%
\subsection{Legacy Detection}
\label{sec:detection}

The directive |\childdocmain| in the main file can detect
whether the complete document or merely a child is to be compiled
even without using the directive |\childdocof|.
This method is deprecated because it is less robust
and there is no compelling reason to use it;
it is merely provided for backward compatibility
and it may be removed in future versions.

If the detection mechanism is to be used,
it is mandatory to correctly specify
the filename of the main file as the argument of |\childdocmain|:
%
\begin{center}
\begin{tabular}{l}
|\input{childdoc.def}|\\
|\childdocmain{|\textit{main}|}|\\
\end{tabular}
\end{center}
%
If |\jobname| does not match the argument \textit{main} of |\childdocmain|,
it is assumed that |\jobname| points to the child file to be compiled.
When using |\childdocmain| with the main file specified as argument,
it suffices to start a child file
with just |\input{|\textit{main}|}|
without loading of the package and using |\childdocof|.
If instead all processing is done
with the appropriate \textsf{childdoc} directives,
the argument of \textit{main} of |\childdocmain| can be empty.

An alternative version of the command line processing described
in \secref{sec:commandline} using the detection mechanism reads:
%
\begin{center}
|... -jobname "|\textit{target}|" "|[\textit{flags}]%
[|\def\jobname{|\textit{dest}|}|]|\input{|\textit{main}|}"|
\end{center}

%%%%%%%%%%%%%%%%%%%%%%%%%%%%%%%%%%%%%%%%%%%%%%%%%%%%%%%%%%%%%%%%%%%%%%%%%%%%%%%%
\subsection{Manual Code}
\label{sec:manual}

In case one cannot be certain whether the definitions file |childdoc.def|
is installed on the target \TeX{} distribution
and one prefers not to ship it,
it is conceivable to paste a few relevant commands into the sources.

To that end, drop all statements |\input{childdoc.def}|
and perform the replacements as outlined below.
Instead of |\childdocmain{|\textit{main}|}| add the following code
to the top of the main file:
%
\begin{center}
\begin{tabular}{l}
|\||ifdefined\childdocname\endinput\||fi\newif\ifchilddoc|\\
|\edef\childdocname{\scantokens\expandafter{\jobname\noexpand}}|\\
|\def\childdocmain{|\textit{main}|}\||ifx\childdocmain\childdocname\||else|\\
|\childdoctrue\includeonly{\childdocname}\let\jobname\childdocmain\||fi|\\
\end{tabular}
\end{center}
%
Instead of |\childdocof{|\textit{main}|}| just include the main file
at the top of each child file:
%
\begin{center}
|\input{|\textit{main}|}|
\end{center}
%
A simple redirection |\childdocforward{|\textit{dest}|}| is achieved by:
%
\begin{center}
|\def\jobname{|\textit{dest}|}\input{\jobname}|
\end{center}
%
The redirection with prefix
|\childdocforwardprefix[|\textit{prefix}|]{|\textit{dest}|}|
is accomplished by:
%
\begin{center}
\begin{tabular}{l}
|{\edef\jobname{\scantokens\expandafter{\jobname\noexpand}}|\\
|\def\redirectjob |\textit{prefix}|#1~~~{\gdef\jobname{|\textit{dest}|#1}}|\\
|\expandafter\redirectjob\jobname~~~}\input{\jobname}|
\end{tabular}
\end{center}

In an alternative approach,
child documents can be compiled by a specific command line
without additional code or specific definitions:
%
\begin{center}
|... -jobname "|\textit{target}|" "|[\textit{flags}]%
|\includeonly{|\textit{dest}|}\input{|\textit{main}|}"|
\end{center}
%

%%%%%%%%%%%%%%%%%%%%%%%%%%%%%%%%%%%%%%%%%%%%%%%%%%%%%%%%%%%%%%%%%%%%%%%%%%%%%%%%
%%%%%%%%%%%%%%%%%%%%%%%%%%%%%%%%%%%%%%%%%%%%%%%%%%%%%%%%%%%%%%%%%%%%%%%%%%%%%%%%
\section{Information}

%%%%%%%%%%%%%%%%%%%%%%%%%%%%%%%%%%%%%%%%%%%%%%%%%%%%%%%%%%%%%%%%%%%%%%%%%%%%%%%%
\subsection{Copyright}

Copyright \copyright{} 2017--2018 Niklas Beisert

This work may be distributed and/or modified under the
conditions of the \LaTeX{} Project Public License, either version 1.3
of this license or (at your option) any later version.
The latest version of this license is in
  \url{http://www.latex-project.org/lppl.txt}
and version 1.3 or later is part of all distributions of \LaTeX{}
version 2005/12/01 or later.

This work has the LPPL maintenance status `maintained'.

The Current Maintainer of this work is Niklas Beisert.

This work consists of the files |README.txt|, |childdoc.ins| and |childdoc.dtx|
as well as the derived files |childdoc.def|, |cdocsamp.tex|
with |cdocsch1.tex|, |cdocsch2.tex|, |cdocspt3.tex|, |cdocspt4.tex|,
|cdocsdrf.tex|, |cdocsfn1.tex|, |cdocsfn2.tex|
as well as |childdoc.pdf|.

%%%%%%%%%%%%%%%%%%%%%%%%%%%%%%%%%%%%%%%%%%%%%%%%%%%%%%%%%%%%%%%%%%%%%%%%%%%%%%%%
\subsection{Files and Installation}

The package consists of the files:
%
\begin{center}
\begin{tabular}{ll}
    |README.txt|   & readme file \\
    |childdoc.ins| & installation file \\
    |childdoc.dtx| & source file \\
    |childdoc.def| & definition file \\
    |cdocsamp.tex| & sample main file \\
    |cdocsch1.tex| & sample include file \\
    |cdocsch2.tex| & sample include file \\
    |cdocspt3.tex| & sample part file \\
    |cdocspt4.tex| & sample part file \\
    |cdocsdrf.tex| & sample redirection file \\
    |cdocsfn1.tex| & sample redirection file \\
    |cdocsfn2.tex| & sample redirection file \\
    |childdoc.pdf| & manual
\end{tabular}
\end{center}
%
The distribution consists of the files
|README.txt|, |childdoc.ins| and |childdoc.dtx|.
%
\begin{itemize}
\item
Run (pdf)\LaTeX{} on |childdoc.dtx|
to compile the manual |childdoc.pdf| (this file).
\item
Run \LaTeX{} on |childdoc.ins| to create the definitions file |childdoc.def|
and the sample |cdocsamp.tex| with include files
|cdocsch1.tex|, |cdocsch2.tex|, |cdocspt3.tex|, |cdocspt4.tex|,
|cdocsdrf.tex|, |cdocsfn1.tex|, |cdocsfn2.tex|.
Then copy the file |childdoc.def| to an appropriate directory of your \LaTeX{}
distribution, e.g.\ \textit{texmf-root}|/tex/latex/childdoc|.
\end{itemize}

%%%%%%%%%%%%%%%%%%%%%%%%%%%%%%%%%%%%%%%%%%%%%%%%%%%%%%%%%%%%%%%%%%%%%%%%%%%%%%%%
\subsection{Related CTAN Packages}

There are several other packages which offer a similar functionality:
%
\begin{itemize}
\item
The packages
\href{http://ctan.org/pkg/docmute}{\textsf{docmute}},
\href{http://ctan.org/pkg/includex}{\textsf{includex}} and
\href{http://ctan.org/pkg/standalone}{\textsf{standalone}}
provide commands to include only the document body of
a child file thus allowing both files to be compiled individually.
\item
The packages \href{http://ctan.org/pkg/subdocs}{\textsf{subdocs}}
and \href{http://ctan.org/pkg/subfiles}{\textsf{subfiles}}
provide structures in which the main and child documents can be
encapsulated and allowing them to be compiled individually.
The inclusion mechanism is different from the conventional |\include|.
\item
The package \href{http://ctan.org/pkg/combine}{\textsf{combine}}
is an elaborate solution to combine several documents into one.
\end{itemize}
%
See also the CTAN topic \href{http://ctan.org/topic/subdocs}{\textsf{subdocs}}
for further related packages.
The present package differs from the above solutions in that
a document structure constructed with the conventional |\include| mechanism
just needs two extra commands at the top of every file
such that all constituent files can be compiled individually.

%%%%%%%%%%%%%%%%%%%%%%%%%%%%%%%%%%%%%%%%%%%%%%%%%%%%%%%%%%%%%%%%%%%%%%%%%%%%%%%%
%\subsection{Feature Suggestions}
%
%The following is a list of features which may be useful for future
%versions of this package:
%%
%\begin{itemize}
%\item
%\ldots
%\end{itemize}

%%%%%%%%%%%%%%%%%%%%%%%%%%%%%%%%%%%%%%%%%%%%%%%%%%%%%%%%%%%%%%%%%%%%%%%%%%%%%%%%
\subsection{Revision History}

%%%%%%%%%%%%%%%%%%%%%%%%%%%%%%%%%%%%%%%%
\paragraph{v2.0:} 2018/12/30

\begin{itemize}
\item
immediate forward processing
\item
added |\childdocby| mechanism
\item
manual restructured
\end{itemize}

%%%%%%%%%%%%%%%%%%%%%%%%%%%%%%%%%%%%%%%%
\paragraph{v1.6:} 2018/01/17

\begin{itemize}
\item
application for development of include files
\item
corrections to manual
\end{itemize}

%%%%%%%%%%%%%%%%%%%%%%%%%%%%%%%%%%%%%%%%
\paragraph{v1.5:} 2017/05/21

\begin{itemize}
\item
more complete structuring introduced
\item
|\childdocof| introduced
\item
|\childdoc| renamed to |\childdocmain|
\item
|\childredirect| renamed to |\childdocforward| and |\childdocforwardprefix|
and functionality expanded
\end{itemize}

%%%%%%%%%%%%%%%%%%%%%%%%%%%%%%%%%%%%%%%%
\paragraph{v1.0:} 2017/04/27

\begin{itemize}
\item
manual and install package
\item
first version published on CTAN
\end{itemize}

%%%%%%%%%%%%%%%%%%%%%%%%%%%%%%%%%%%%%%%%
\paragraph{v0.6:} 2017/04/26

\begin{itemize}
\item
redirection mechanism added
\end{itemize}

%%%%%%%%%%%%%%%%%%%%%%%%%%%%%%%%%%%%%%%%
\paragraph{v0.5:} 2017/04/26

\begin{itemize}
\item
functionality in definition file
\end{itemize}


%%%%%%%%%%%%%%%%%%%%%%%%%%%%%%%%%%%%%%%%%%%%%%%%%%%%%%%%%%%%%%%%%%%%%%%%%%%%%%%%
%%%%%%%%%%%%%%%%%%%%%%%%%%%%%%%%%%%%%%%%%%%%%%%%%%%%%%%%%%%%%%%%%%%%%%%%%%%%%%%%
%%%%%%%%%%%%%%%%%%%%%%%%%%%%%%%%%%%%%%%%%%%%%%%%%%%%%%%%%%%%%%%%%%%%%%%%%%%%%%%%
\appendix

\settowidth\MacroIndent{\rmfamily\scriptsize 000\ }

 \DocInput{childdoc.dtx}

\end{document}
%</driver>
% \fi
%
% %%%%%%%%%%%%%%%%%%%%%%%%%%%%%%%%%%%%%%%%%%%%%%%%%%%%%%%%%%%%%%%%%%%%%%%%%%%%%%
% %%%%%%%%%%%%%%%%%%%%%%%%%%%%%%%%%%%%%%%%%%%%%%%%%%%%%%%%%%%%%%%%%%%%%%%%%%%%%%
% \section{Sample}
%\iffalse
%<*samplemain>
%\fi
%
% The following presents a sample document
% with two chapters, two parts, a title page,
% a compile flag as well as three forwarding files to set the flag.
% It consists of eight |.tex| files:
% \begin{center}
% \begin{tabular}{ll}
% |cdocsamp.tex|&main file\\
% |cdocsch1.tex|&include file for chapter 1\\
% |cdocsch2.tex|&include file for chapter 2\\
% |cdocspt3.tex|&include file for part 3\\
% |cdocspt4.tex|&include file for part 4\\
% |cdocsdrf.tex|&forwarding file for main file in draft mode\\
% |cdocsfi1.tex|&forwarding file for final version of chapter 1\\
% |cdocsfi2.tex|&forwarding file for final version of chapter 2\\
% \end{tabular}
% \end{center}
% Each of the eight files can be compiled directly by the \LaTeX{} compiler.
%
% %%%%%%%%%%%%%%%%%%%%%%%%%%%%%%%%%%%%%%
% \paragraph{Main File.}
%
% The main file is called |cdocsamp.tex|.
%
% Load the \textsf{childdoc} definitions and
% declare the filename for the main document:
%    \begin{macrocode}
\input{childdoc.def}
\childdocmain{}
%    \end{macrocode}

% Optional override for |\version| flag:
%    \begin{macrocode}
%%\ifchilddoc\else\providecommand{\version}{draft}\fi
%    \end{macrocode}

% Define the default values for the |\version| flag
% (|final| for the main file and |draft| for childs):
%    \begin{macrocode}
\ifchilddoc
\providecommand{\version}{draft}
\else
\providecommand{\version}{final}
\fi
%    \end{macrocode}

% Load the standard document class:
%    \begin{macrocode}
\documentclass[12pt]{article}
%    \end{macrocode}

% Start the document body:
%    \begin{macrocode}
\begin{document}
%    \end{macrocode}

% Declare a title page.
% Print title, part of document being processed and version flag:
%    \begin{macrocode}
\addtocounter{page}{-1}
\begin{center}
{\LARGE\bfseries{}childdoc example\par}
\vspace{1cm}
\ifchilddoc
\ifchilddocmanual part\else chapter\fi:
`\childdocname' of `\childdocjob'\par
\else
main document: `\childdocjob'\par
\fi
version: \version\par
\end{center}
\newpage
%    \end{macrocode}

% Manually include selected file,
% otherwise process as usual:
%    \begin{macrocode}
\ifchilddocmanual
\section*{part `\childdocname'}
\input{\childdocname}
\else
%    \end{macrocode}

% Include the two chapters:
%    \begin{macrocode}
\include{cdocsch1}
\include{cdocsch2}
%    \end{macrocode}

% Include the two parts unless only chapters should be displayed:
%    \begin{macrocode}
\ifchilddoc\else
\section{part three}
\input{cdocspt3}
\section{part four}
\input{cdocspt4}
\fi
%    \end{macrocode}

% Process as usual until here:
%    \begin{macrocode}
\fi
%    \end{macrocode}

% End of document body:
%    \begin{macrocode}
\end{document}
%    \end{macrocode}
%\iffalse
%</samplemain>
%\fi
%
% %%%%%%%%%%%%%%%%%%%%%%%%%%%%%%%%%%%%%%
% \paragraph{Chapter Include Files.}
%
% The include files are called |cdocsch1.tex| and |cdocsch2.tex|.
%
%\iffalse
%<*samplechap1|samplechap2>
%\fi

% Optional override for |\version| flag:
%    \begin{macrocode}
%%\providecommand{\version}{final}
%    \end{macrocode}

% Include the main document:
%    \begin{macrocode}
\input{childdoc.def}
\childdocof{cdocsamp}
%    \end{macrocode}

%\iffalse
%</samplechap1|samplechap2>
%\fi
%
%\iffalse
%<*samplechap1>
%\fi
% Some text for chapter 1:
%    \begin{macrocode}
\section{one}
some text in chapter one
%    \end{macrocode}

%\iffalse
%</samplechap1>
%\fi
% Some text for chapter 2:
%\iffalse
%<*samplechap2>
%\fi
%    \begin{macrocode}
\section{two}
more text in chapter two
%    \end{macrocode}

%\iffalse
%</samplechap2>
%\fi
%
% %%%%%%%%%%%%%%%%%%%%%%%%%%%%%%%%%%%%%%
% \paragraph{Part Include Files.}
%
% The include files are called |cdocspt3.tex| and |cdocspt4.tex|.
%
%\iffalse
%<*samplepart3|samplepart4>
%\fi

% Optional override for |\version| flag:
%    \begin{macrocode}
%%\providecommand{\version}{final}
%    \end{macrocode}

% Include the main document:
%    \begin{macrocode}
\input{childdoc.def}
\childdocby{cdocsamp}
%    \end{macrocode}

%\iffalse
%</samplepart3|samplepart4>
%\fi
%
%\iffalse
%<*samplepart3>
%\fi
% Some text for part 3:
%    \begin{macrocode}
some text in part three
%    \end{macrocode}

%\iffalse
%</samplepart3>
%\fi
% Some text for part 4:
%\iffalse
%<*samplepart4>
%\fi
%    \begin{macrocode}
more text in part four
%    \end{macrocode}

%\iffalse
%</samplepart4>
%\fi
%
% %%%%%%%%%%%%%%%%%%%%%%%%%%%%%%%%%%%%%%
% \paragraph{Forwarding for a Complete Draft.}
%
% The following forwarding file |cdocsdrf.tex|
% compiles the main document in draft mode:
%\iffalse
%<*sampledraft>
%\fi
%    \begin{macrocode}
\def\version{draft}
\input{childdoc.def}
\childdocforward{cdocsamp}
%    \end{macrocode}

%\iffalse
%</sampledraft>
%\fi
%
% %%%%%%%%%%%%%%%%%%%%%%%%%%%%%%%%%%%%%%
% \paragraph{Forwarding for Final Version of the Chapters.}
%
% The following forwarding files |cdocsfn1.tex| and |cdocsfn2.tex|
% (with identical content)
% compile the final versions of the child documents
% |cdocsch1.tex| and |cdocsch2.tex|, respectively:
%\iffalse
%<*samplefinal>
%\fi
%    \begin{macrocode}
\def\version{final}
\input{childdoc.def}
\childdocforwardprefix[cdocsamp]{cdocsfn}{cdocsch}
%    \end{macrocode}

%\iffalse
%</samplefinal>
%\fi
%
% %%%%%%%%%%%%%%%%%%%%%%%%%%%%%%%%%%%%%%
% \paragraph{Command Line Processing.}
%
% The following three command lines generate the output files
% |cdocscld|, |cdocscl1| and |cdocscl2|
% which should be identical to
% |cdocsdrf|, |cdocsch1| and |cdocsfn2|, respectively:
% \begin{center}
% \begin{tabular}{l}
% |latex -jobname cdocscld \|\\
% |  "\def\version{draft}\input{childdoc.def}\childdocforward{cdocsamp}"|\\
% |latex -jobname cdocscl1 \|\\
% |  "\input{childdoc.def}\childdocforward[cdocsamp]{cdocsch1}"|\\
% |latex -jobname cdocscl2 \|\\
% |  "\def\version{final}\input{childdoc.def}\childdocforward{cdocsch2}"|
% \end{tabular}
% \end{center}
% Note that the trailing backslash on each first line
% merely continues the input to the second line
% (for convenient cut ant paste).
% Furthermore, the command |latex| can be replaced by any
% of its alternative versions such as |pdflatex|.
%
% %%%%%%%%%%%%%%%%%%%%%%%%%%%%%%%%%%%%%%%%%%%%%%%%%%%%%%%%%%%%%%%%%%%%%%%%%%%%%%
% %%%%%%%%%%%%%%%%%%%%%%%%%%%%%%%%%%%%%%%%%%%%%%%%%%%%%%%%%%%%%%%%%%%%%%%%%%%%%%
% \section{Implementation}
%\iffalse
%<*package>
%\fi
%
% This section describes the definitions file |childdoc.def|.

% The definitions cannot be loaded using |\usepackage| or |\RequirePackage|
% which has a mechanism to prevent loading a style file more than once.
% When loading the definitions by means of |\input|
% multiple instances have to be prevented manually:
%\iffalse
%This code needs to be before the `\ProvidesFile' directive
%which is defined at the beginning of this file.
%Therefore it is also placed there and commented out here.
%</package>
%<*discard>
%\fi
%    \begin{macrocode}
\ifdefined\childdocmain\endinput\fi
%    \end{macrocode}
%\iffalse
%</discard>
%<*package>
%\fi
%
% \macro{\ifchilddoc}
% \macro{\ifchilddocmanual}
% The conditional |\ifchilddoc| tells whether a
% child (true) or main (false) document is being compiled.
% The conditional |\ifchilddocmanual| tells whether
% the |\includeonly| mechanism is used (false) or
% the selection of child files must be performed manually (true).
% The definitions initialise to false:
%    \begin{macrocode}
\newif\ifchilddoc
\newif\ifchilddocmanual
%    \end{macrocode}

% \macro{\childdocname}
% \macro{\childdocjob}
% The macro |\childdocname| stores the name of the main document
% to be compiled. The macro |\childdocjob| stores the name of
% the document on which the \LaTeX{} compiler was originally invoked.
% The content of |\jobname| cannot be compared
% to filenames specified in the source due to different catcodes.
% The following code rescans |\jobname|, stores the result
% in |\childdocname| and saves a copy in |\childdocjob|:
%    \begin{macrocode}
\edef\childdocname{\scantokens\expandafter{\jobname\noexpand}}
\let\childdocjob\childdocname
%    \end{macrocode}

% \macro{\childdocdisable}
% The macro |\childdocdisable| prevents the main file
% from being processed more than once.
% At this stage, the main document command |\childdocmain|
% is assumed to be called once again where it should do nothing.
% Any subsequent call to it should prevent
% a secondary processing of the main document
% It overwrites the forwarding commands
% |\childdocof| and |\childdocforward|
% with empty macros to prevent further inclusions of the main document:
%    \begin{macrocode}
\newcommand{\childdocdisable}
{
  \renewcommand{\childdocmain}[1]{\renewcommand{\childdocmain}[1]{\endinput}}
  \renewcommand{\childdocof}[1]{}
  \renewcommand{\childdocby}[2][]{}
  \renewcommand{\childdocforward}[2][]{}
  \renewcommand{\childdocdisable}{}
}
%    \end{macrocode}

% \macro{\childdocmain}
% The macro |\childdocmain| is to be called at the top of the main file
% with nothing or the main filename (without extension) as argument.
% First, it breaks loops.
% If the argument is not empty and does not match |\childdocname|
% (which is set by the first inclusion of |childdoc.def|),
% |\ifchilddoc| is set to true, |\includeonly| is applied to the child file
% and |\jobname| is set to the main file
% (for proper handling of |.aux| files):
%    \begin{macrocode}
\newcommand{\childdocmain}[1]
{
  \childdocdisable\childdocmain{}
  \if?#1?\else
    \begingroup
      \def\childdoctmp{#1}
      \ifx\childdoctmp\childdocname
        \def\childdoctmp{}
      \else
        \def\childdoctmp
        {
          \childdoctrue
          \includeonly{\childdocname}
          \def\childdocjob{#1}
          \def\jobname{#1}
        }
      \fi
      \expandafter
    \endgroup
    \childdoctmp
  \fi
}
%    \end{macrocode}

% \macro{\childdocof}
% The command |\childdocof| redirects
% compilation to the main file |#1|.
%    \begin{macrocode}
\newcommand{\childdocof}[1]
{
  \childdocdisable
  \childdoctrue
  \includeonly{\childdocname}
  \def\jobname{#1}
  \def\childdocjob{#1}
  \input{#1}
}
%    \end{macrocode}

% \macro{\childdocby}
% The command |\childdocby| ....
%    \begin{macrocode}
\newcommand{\childdocby}[2][]
{
  \childdocdisable
  \childdoctrue
  \childdocmanualtrue
  \if?#1?\else
    \def\jobname{#2}
  \fi
  \def\childdocjob{#2}
  \input{#2}
  \endinput
}
%    \end{macrocode}

% \macro{\childdocforward}
% The command |\childdocforward| redirects
% compilation to the main file or
% (if the optional argument is given) a child file.
% Parameters are set as if the main file
% or a child file starting with |\childdocof| was compiled.
% Then compilation is handed over to the main file:
%    \begin{macrocode}
\newcommand{\childdocforward}[2][]
{
  \begingroup
    \if?#1?
      \def\childdoctmp
      {
        \def\childdocname{#2}
        \def\childdocjob{#2}
        \def\jobname{#2}
        \input{#2}
        \endinput
      }
    \else
      \def\childdoctmp
      {
        \childdocdisable
        \def\childdocname{#2}
        \childdoctrue
        \includeonly{#2}
        \def\childdocjob{#1}
        \def\jobname{#1}
        \input{#1}
        \endinput
      }
    \fi
    \expandafter
  \endgroup
  \childdoctmp
}
%    \end{macrocode}

% \macro{\childdocforwardprefix}
% The command |\childdocforwardprefix| redirects
% compilation to the main or a child file by means of a pattern.
% The prefix |#1| in the current filename is replaced by |#2|
% and the suffix of the current filename is kept
% (it is assumed that the filename does not contain the substring `|~~~|'
% which is used as a delimiter).
% Compilation is handed over to the new file by |\childdocforward|:
%    \begin{macrocode}
\newcommand{\childdocforwardprefix}[3][]
{
  \begingroup
    \def\childdocextract #2##1~~~{\def\childdoctmp{\childdocforward[#1]{#3##1}}}
    \expandafter\childdocextract\childdocname~~~
    \expandafter
  \endgroup
  \childdoctmp
}
%    \end{macrocode}

% \macro{\childdoc}
% The deprecated macro |\childdoc| is a legacy version of |\childdocmain|:
%    \begin{macrocode}
\newcommand{\childdoc}{\childdocmain}
%    \end{macrocode}

% \macro{\childdocredirect}
% The deprecated macro |\childdocredirect| is a legacy version
% of |\childdocforward| and |\childdocforwardprefix|:
%    \begin{macrocode}
\newcommand{\childdocredirect}[2][]
{
  \begingroup
    \if?#1?
      \def\childdoctmp{\childdocforward{#2}}
    \else
      \def\childdoctmp{\childdocforwardprefix{#1}{#2}}
    \fi
    \expandafter
  \endgroup
  \childdoctmp
}
%    \end{macrocode}

%\iffalse
%</package>
%\fi
%
\endinput
\childdocforward{cdocsamp}"|\\
% |latex -jobname cdocscl1 \|\\
% |  "% \iffalse
%
% childdoc.dtx Copyright (C) 2017-2018 Niklas Beisert
%
% This work may be distributed and/or modified under the
% conditions of the LaTeX Project Public License, either version 1.3
% of this license or (at your option) any later version.
% The latest version of this license is in
%   http://www.latex-project.org/lppl.txt
% and version 1.3 or later is part of all distributions of LaTeX
% version 2005/12/01 or later.
%
% This work has the LPPL maintenance status `maintained'.
%
% The Current Maintainer of this work is Niklas Beisert.
%
% This work consists of the files childdoc.dtx and childdoc.ins
% and the derived files childdoc.def and cdocsamp.tex with
% cdocsch1.tex, cdocsch2.tex, cdocsdrf.tex, cdocsfn1.tex, cdocsfn2.tex.
%
%<package>\ifdefined\childdocmain\endinput\fi
%<package>\ProvidesFile{childdoc.def}[2018/12/30 v2.0 child document driver]
%<samplemain>\ProvidesFile{cdocsamp.tex}[2018/12/30 v2.0 sample for childdoc]
%<*driver>
%\ProvidesFile{childdoc.drv}[2018/12/30 v2.0 childdoc reference manual file]
\PassOptionsToClass{10pt,a4paper}{article}
\documentclass{ltxdoc}

\usepackage[margin=35mm]{geometry}
\usepackage{hyperref}
\usepackage{hyperxmp}
\usepackage[usenames]{color}

\hypersetup{colorlinks=true}
\hypersetup{pdfstartview=FitH}
\hypersetup{pdfpagemode=UseNone}
\hypersetup{pdfsource={}}
\hypersetup{pdflang={en-UK}}
\hypersetup{pdfcopyright={Copyright 2017-2018 Niklas Beisert.
  This work may be distributed and/or modified under the
  conditions of the LaTeX Project Public License, either version 1.3
  of this license or (at your option) any later version.}}
\hypersetup{pdflicenseurl={http://www.latex-project.org/lppl.txt}}
\hypersetup{pdfcontactaddress={ETH Zurich, ITP, HIT K,
  Wolfgang-Pauli-Strasse 27}}
\hypersetup{pdfcontactpostcode={8093}}
\hypersetup{pdfcontactcity={Zurich}}
\hypersetup{pdfcontactcountry={Switzerland}}
\hypersetup{pdfcontactemail={nbeisert@itp.phys.ethz.ch}}
\hypersetup{pdfcontacturl={http://people.phys.ethz.ch/\xmptilde nbeisert/}}

\newcommand{\secref}[1]{\hyperref[#1]{section \ref*{#1}}}

\parskip1ex
\parindent0pt
\let\olditemize\itemize
\def\itemize{\olditemize\parskip0pt}

\begin{document}

\title{The \textsf{childdoc} Package}
\hypersetup{pdftitle={The childdoc Package}}
\author{Niklas Beisert\\[2ex]
  Institut f\"ur Theoretische Physik\\
  Eidgen\"ossische Technische Hochschule Z\"urich\\
  Wolfgang-Pauli-Strasse 27, 8093 Z\"urich, Switzerland\\[1ex]
  \href{mailto:nbeisert@itp.phys.ethz.ch}
  {\texttt{nbeisert@itp.phys.ethz.ch}}}
\hypersetup{pdfauthor={Niklas Beisert}}
\hypersetup{pdfsubject={Manual for the LaTeX2e Package childdoc}}
\date{30 December 2018, \textsf{v2.0}}
\maketitle

\begin{abstract}\noindent
\textsf{childdoc} is a \LaTeXe{} package
that enables the direct compilation
of document sections included by |\include|
to individual files.
\end{abstract}

\begingroup
\parskip0ex
\tableofcontents
\endgroup

%%%%%%%%%%%%%%%%%%%%%%%%%%%%%%%%%%%%%%%%%%%%%%%%%%%%%%%%%%%%%%%%%%%%%%%%%%%%%%%%
%%%%%%%%%%%%%%%%%%%%%%%%%%%%%%%%%%%%%%%%%%%%%%%%%%%%%%%%%%%%%%%%%%%%%%%%%%%%%%%%
\section{Introduction}

\LaTeX{} provides a mechanism to structure a large document (such as a book)
into a main file and several child files (containing the chapters)
using the |\include| command.
This mechanism is beneficial for documents
which span hundreds of pages in order to
make the source file(s) more manageable.
Moreover, compilation can be restricted to
selected child files by means of the |\includeonly| command.
The latter feature can be used to reduce the compilation time while editing
(this was significantly more useful in the earlier days of \LaTeX{})
or to generate a smaller document which is easier to navigate.
Another application of |\includeonly| is to generate
documents consisting of selected parts of the complete document.

However, there are a few drawbacks of the plain |\include| mechanism:
\begin{itemize}
\item
The child files cannot be compiled on their own,
they can only be compiled via the main file.
A naive editing environment
(such as a text editor with an option
to have the current file processed by \LaTeX)
may require one to switch to the main file before compiling;
attempting to compile the child file produces errors.
\item
The main file must be modified (each time)
to adjust the |\includeonly| command
to the present needs. This easily leaves the main file in a messy state.
\item
The generated document will always carry the filename
of the main document. This is inconvenient if
several child files are to be compiled and
to be kept for distribution.
\end{itemize}

The present package provides a simple interface
to make child files individually compilable by \LaTeX{}.
Compiling a child file then has the same effect as compiling
the main file with an |\includeonly| command
to select the appropriate child.
Moreover the generated document will carry the name of the child
rather than the main file.
This resolves all three above issues.

This feature is meant to make the editing of books,
thesis documents and lecture notes somewhat more convenient.
However, the package can also be used efficiently for
composing a series of documents (such as exercise sheets)
which are typically distributed individually.
It then assists the author in generating the individual documents
(potentially in different versions)
as well as a document containing the collected series.
Another application is in developing style files
or other kinds of included material
where compilation of the style file could redirect
to a sample or test file.

%%%%%%%%%%%%%%%%%%%%%%%%%%%%%%%%%%%%%%%%%%%%%%%%%%%%%%%%%%%%%%%%%%%%%%%%%%%%%%%%
%%%%%%%%%%%%%%%%%%%%%%%%%%%%%%%%%%%%%%%%%%%%%%%%%%%%%%%%%%%%%%%%%%%%%%%%%%%%%%%%
\section{Usage}

First of all, the package \textsf{childdoc} is \emph{not} a standard
\LaTeXe{} |.sty| style file! Therefore it needs to be invoked in
a non-standard way.

%%%%%%%%%%%%%%%%%%%%%%%%%%%%%%%%%%%%%%%%%%%%%%%%%%%%%%%%%%%%%%%%%%%%%%%%%%%%%%%%
\subsection{Included Files}
\label{sec:include}

%%%%%%%%%%%%%%%%%%%%%%%%%%%%%%%%%%%%%%%%
\DescribeMacro{\childdocmain}
To use the package, add the commands
\begin{center}
\begin{tabular}{l}
|\input{childdoc.def}|\\
|\childdocmain{}|\\
\end{tabular}
\end{center}
at the very top of the main \LaTeX{} file,
in particular \emph{before} the |\documentclass| statement!
The argument of |\childdocmain| should be left empty
(but it must be present).

%%%%%%%%%%%%%%%%%%%%%%%%%%%%%%%%%%%%%%%%
\DescribeMacro{\childdocof}
Furthermore, add the commands
\begin{center}
\begin{tabular}{l}
|\input{childdoc.def}|\\
|\childdocof{|\textit{main}|}|\\
\end{tabular}
\end{center}
at the top of every child file \textit{child}
which is included by |\include{|\textit{child}|}|
from within the main file
(or at least for those files to be compiled individually).
The argument \textit{main} must be the filename of the main file.

There are a couple of
considerations in setting up the main and child documents:

%%%%%%%%%%%%%%%%%%%%%%%%%%%%%%%%%%%%%%%%
\paragraph{Restrictions.}

Please note the following restrictions:
\begin{itemize}
\item
|\childdocmain| must be called with one argument \textit{main}
to ensure compatibility with earlier version of the package.
It must either be empty (|\childdocmain{}|)
or precisely match the filename of the main file in which it is specified.
See \secref{sec:detection} for further information.
\item
The filename \textit{main} must be specified without the |.tex| extension.
\item
The filename \textit{main} is case sensitive
(even in case-insensitive file systems)
due to internal string comparison.
\item
The argument \textit{main} should be fully expanded, it cannot be a macro.
\item
Subdirectories and special characters should be avoided in filenames.
\item
The command |\childdocmain{|\textit{main}|}| must be followed by a whitespace.
It should not be followed immediately by another command
or by a comment mark `|%|'.
This is because the \TeX{} parser reads the token immediately following
the argument of |\childdocmain| and puts it
at the beginning of every child section;
however, a white\-space is ignored.
\end{itemize}

%%%%%%%%%%%%%%%%%%%%%%%%%%%%%%%%%%%%%%%%
\paragraph{Content of Main File.}

It is advisable to place all content in the child files included by |\include|.
Any output contained in the main file will appear in all child documents
unless suppressed manually;
it cannot be suppressed automatically by the |\includeonly| directive
and thus should normally be avoided.
A method to include some content in the main file
by means of conditional processing is described in \secref{sec:conditional}.

%%%%%%%%%%%%%%%%%%%%%%%%%%%%%%%%%%%%%%%%
\paragraph{Page Numbering.}

When only a part of the document is compiled,
the appropriate numbering of pages
(as well as other status parameters)
is determined from the |.aux| files.
The latter contain information from previous passes.
However this information needs to propagate through
all intermediate child documents.
Therefore the page numbering in child documents may well
be inconsistent until the complete document is compiled at least once.

A useful (if unconventional) way to always ensure a consistent
page numbering is to restart the numbering in each child document
and denote the pages by `\textit{child}|.|\textit{page}'
where \textit{child} represents the chapter/section number of the child file.
This can be achieved by the command
|\numberwithin{page}{|\textit{child}|}|
of the \textsf{amsmath} package
where \textit{child} can be |chapter| or |section|
depending on the chosen structuring.
Alternatively, one can modify the macro |\thepage| appropriately
and reset the counter |page| at the start of each child file.

%%%%%%%%%%%%%%%%%%%%%%%%%%%%%%%%%%%%%%%%%%%%%%%%%%%%%%%%%%%%%%%%%%%%%%%%%%%%%%%%
\subsection{Conditional Processing}
\label{sec:conditional}

The package provides a mechanism to compile different versions
of a document. To customise the versions further some conditional processing
can come in handy to distinguish which version is being compiled.
The package provides two macros to describe the compilation context:

%%%%%%%%%%%%%%%%%%%%%%%%%%%%%%%%%%%%%%%%
\DescribeMacro{\ifchilddoc}
The conditional |\ifchilddoc| distinguishes between the compilation of
child documents and the main document:
%
\begin{center}
|\ifchilddoc |\textit{child-code}| |[|\||else |\textit{main-code}]| \||fi|
\end{center}

%%%%%%%%%%%%%%%%%%%%%%%%%%%%%%%%%%%%%%%%
\DescribeMacro{\childdocname}
\DescribeMacro{\childdocjob}
The macro |\childdocname| contains the filename (without extension)
of the main or child file being processed.
Note that |\childdocjob| will always contain the name of the main file.

%%%%%%%%%%%%%%%%%%%%%%%%%%%%%%%%%%%%%%%%
\paragraph{Title Page.}

Conditional processing can be used to include a title or banner page
in the main document when proper precautions are taken.
Importantly, the code in the main file should ensure that the page counter
(as well as other status parameters which are stored in the |.aux| files)
takes the same value after the conditional processing.
Otherwise the page numbers may take divergent values
depending on which part is compiled.

For example, a title page could be declared by:
%
\begin{center}
\begin{tabular}{l}
|\ifchilddoc\||else|\\
|\addtocounter{page}{-1}|\\
\textit{code for title page}\\
|\newpage|\\
|\||fi|
\end{tabular}
\end{center}
%
A banner page for the child documents can be generated by:
%
\begin{center}
\begin{tabular}{l}
|\ifchilddoc|\\
|\addtocounter{page}{-1}|\\
\textit{code for banner page}\\
|\newpage|\\
|\||fi|
\end{tabular}
\end{center}
%
Here one could write a message such as:
\begin{center}
|This is the part \childdocname{} of \childdocjob{}.|
\end{center}

%%%%%%%%%%%%%%%%%%%%%%%%%%%%%%%%%%%%%%%%%%%%%%%%%%%%%%%%%%%%%%%%%%%%%%%%%%%%%%%%
\subsection{Flags}
\label{sec:flags}

The package makes it easy to generate different versions
of the main or child documents.
To this end compilation flags can be defined
and assigned different default values.
They will be particularly useful in conjunction
with the forwarding mechanism described in \secref{sec:forward}.

For example, it may be useful to have a flag |\version|
which can be set to |draft| or |final|.
The document source will contain some conditional code
depending on the value of |\version|.
Suppose further, the flag should default to |final| for the main file
and to |draft| for child files
which is a natural assignment for editing the document.
This is achieved by placing the following code
in the preamble of the main document
(below the |\childdocmain| directive):
%
\begin{center}
\begin{tabular}{l}
|\ifchilddoc|\\
|\providecommand{\version}{draft}|\\
|\||else|\\
|\providecommand{\version}{final}|\\
|\||fi|
\end{tabular}
\end{center}
%
The definition by |\providecommand| makes sure
that previous definitions are not overwritten.
Further statements |\providecommand{\version}{...}|
can thus be added before the above code to override it.

For the main file, one might add a line
(between |\childdocmain| and the above block)
%
\begin{center}
|%\ifchilddoc\||else\providecommand{\version}{draft}\||fi|
\end{center}
%
which can be uncommented to produce a draft version.
Likewise one can add a line to the very top of a child file
(above the |\childdocof{|\textit{main}|}| directive)
%
\begin{center}
|%\providecommand{\version}{final}|
\end{center}
%
which can be uncommented to produce the final version of this child document.

%%%%%%%%%%%%%%%%%%%%%%%%%%%%%%%%%%%%%%%%%%%%%%%%%%%%%%%%%%%%%%%%%%%%%%%%%%%%%%%%
\subsection{Forwarding}
\label{sec:forward}

Different versions of the main or child documents
using compilation flags as described in \secref{sec:flags}
can be (permanently) stored in different files
for convenient compilation, viewing and distribution.
To this end, the package defines a command
to pass on compilation to a different file:

%%%%%%%%%%%%%%%%%%%%%%%%%%%%%%%%%%%%%%%%
\DescribeMacro{\childdocforward}
The command |\childdocforward| redirects processing to
another source file:
%
\begin{center}
\begin{tabular}{l}
|\input{childdoc.def}|\\
|\childdocforward[|\textit{main}|]{|\textit{dest}|}|\\
\end{tabular}
\end{center}
%
The argument \textit{dest} is the destination file
(without extension).
It should be the main file or one of the child files.
Note that further \textsf{childdoc} directives
such as |\childdocof| and |\childdocforward|
in the indicated file will be processed in this form.
The optional argument \textit{main}
passes on directly to the main file \textit{main}
while pretending to compile the child \textit{dest}.
This form behaves as if \textit{dest}
issues |\childdocof{|\textit{main}|}| right away,
and no further \textsf{childdoc} directives will be processed.

%%%%%%%%%%%%%%%%%%%%%%%%%%%%%%%%%%%%%%%%
\DescribeMacro{\...prefix}
In the alternative form |\childdocforwardprefix|,
%
\begin{center}
\begin{tabular}{l}
|\input{childdoc.def}|\\
|\childdocforwardprefix[|\textit{main}|]{|\textit{prefix}|}{|\textit{dest}|}|
\end{tabular}
\end{center}
%
the destination file is determined by a pattern
depending on the current file:
To make this work, the current file must be called
`{\textit{prefix}\hspace{0.2em}\textit{suffix}}'
with \textit{prefix} matching precisely the argument.
Processing is then passed on to the file
`{\textit{dest}\hspace{0.2em}\textit{suffix}}'.
Surely, the same effect is achieved by
directly specifying the
argument `{\textit{dest}\hspace{0.2em}\textit{suffix}}'
in the first form.
However, that requires to set up a different file
for each child. With the alternative form of the command
all these files can have exactly the same content
which simplifies setting them up and maintaining them.

For example, the following file |draft.tex|
with a compilation flag |\version| as described in \secref{sec:flags}
compiles the main document as a draft:
%
\begin{center}
\begin{tabular}{l}
|\def\version{draft}|\\
|\input{childdoc.def}|\\
|\childdocforward{|\textit{main}|}|
\end{tabular}
\end{center}
%
Likewise, the following files |final|\textit{nn}|.tex|
compile the final version of the child document
|child|\textit{nn}|.tex|:
%
\begin{center}
\begin{tabular}{l}
|\def\version{final}|\\
|\input{childdoc.def}|\\
|\childdocforwardprefix{final}{child}|
\end{tabular}
\end{center}
%

Note that when several versions of a main file and/or of each child file
are to be generated, it may be convenient to set up a |Makefile| or
shell script to automatise the process.

%%%%%%%%%%%%%%%%%%%%%%%%%%%%%%%%%%%%%%%%%%%%%%%%%%%%%%%%%%%%%%%%%%%%%%%%%%%%%%%%
\subsection{Command Line Processing}
\label{sec:commandline}

The effect of redirection files can also be achieved by invoking
the \LaTeX{} compiler with a more elaborate command line.
Most conveniently this should be done as part
of a shell script or a |Makefile|.

When using \textsf{childdoc} in the main file, the following
command lines effectively perform a redirection
(note that depending on the shell being used,
backslashes may have to be doubled: `|\|' $\to$ `|\\|'):
%
\begin{center}
|... -jobname "|\textit{target}|" |\\|"|[\textit{flags}]%
|\input{childdoc.def}\childdocforward[|\textit{main}|]{|\textit{dest}|}"|
\end{center}
%
Here \textit{target} is the name of the output file,
\textit{main} is the name of the main file
and \textit{dest} is the name of the main or child file to be processed
(all filenames without extensions).
The optional argument \textit{main} can be omitted
if \textit{main} matches \textit{dest}.
Optionally, compilation \textit{flags} can be defined via |\def| commands.
This command line makes the \TeX{} engine believe
it is compiling the file \textit{target}
whose content is specified as the latter parameter.
The provided code then forwards the processing to
\textit{main} or \textit{dest} as described in \secref{sec:forward}.

%%%%%%%%%%%%%%%%%%%%%%%%%%%%%%%%%%%%%%%%%%%%%%%%%%%%%%%%%%%%%%%%%%%%%%%%%%%%%%%%
\subsection{Include by Input}
\label{sec:input}

Including child documents by |\include| has some restrictions by design.
Most notably, the content of a child document always occupies
its own set of pages; pages cannot be shared between child documents.
Usually, this behaviour makes perfect sense
because each child document contain an essential part of the document.
However, in some situations it may be desirable to compose
a document from a collection of parts
without having mandatory page breaks between then.
For this case, the package
provides a mechanism to include parts
by |\input| which can also be processed individually.
However, by construction this mechanism
requires manual handling of the content to be output.

%%%%%%%%%%%%%%%%%%%%%%%%%%%%%%%%%%%%%%%%
\DescribeMacro{\ifchilddocmanual}
The main file should be prepared as usual, see \secref{sec:include}.
However, the document body must make a distinction
between processing of an individual part and of the main document, e.g.:
%
\begin{center}
\begin{tabular}{l}
|\ifchilddocmanual|\\
|\input{\childdocname}|\\
|\||else|\\
\textit{document body with }|\input{|\textit{part}|}|\\
|\||fi|
\end{tabular}
\end{center}
%
The conditional |\ifchilddocmanual| is true whenever
a part to be included by |\input| is being compiled,
and the name of the part is stored in |\childdocname|.

%%%%%%%%%%%%%%%%%%%%%%%%%%%%%%%%%%%%%%%%
\DescribeMacro{\childdocby}
Each part to be included by |\input| should start with:
%
\begin{center}
\begin{tabular}{l}
|\input{childdoc.def}|\\
|\childdocby{|\textit{main}|}|\\
\end{tabular}
\end{center}
%
The directive |\childdocby| is similar to |\childdocof|
described in \secref{sec:include},
but the subsequent selection of content must be done manually.
To that end, both |\ifchilddoc| and |\ifchilddocmanual|
will be true upon processing of a part,
and the name of the part is stored in |\childdocname|.
Note that |\jobname| will be set to the filename of the current part
so that each part receives an individual |.aux| file
that does not interfere with the |.aux| file(s) of the main document.
This behaviour can be altered by the alternative form
|\childdocby[*]{|\textit{main}|}| (with a non-empty optional argument)
which uses the |.aux| file of the main document
by setting |\jobname| to \textit{main}.

%%%%%%%%%%%%%%%%%%%%%%%%%%%%%%%%%%%%%%%%%%%%%%%%%%%%%%%%%%%%%%%%%%%%%%%%%%%%%%%%
\subsection{Driver Development}
\label{sec:driver}

The \textsf{childdoc} mechanism can also be use for the development
of definition files such as \LaTeX{} styles or classes.
This case differs from the above setup with multiple parts
included by |\include| in that no |\includeonly| should be invoked.
This can be achieved by starting the include file
(before |\ProvidesPackage|) with:
%
\begin{center}
\begin{tabular}{l}
|\input{childdoc.def}|\\
|\childdocforward{|\textit{main}|}|\\
\end{tabular}
\end{center}
%
or alternatively with:
%
\begin{center}
\begin{tabular}{l}
|\input{childdoc.def}|\\
|\childdocby{|\textit{main}|}|\\
\end{tabular}
\end{center}
%
Both forms have slightly different effects as described above.
The main file is prepared as usual, see \secref{sec:include}.

%%%%%%%%%%%%%%%%%%%%%%%%%%%%%%%%%%%%%%%%%%%%%%%%%%%%%%%%%%%%%%%%%%%%%%%%%%%%%%%%
\subsection{Legacy Detection}
\label{sec:detection}

The directive |\childdocmain| in the main file can detect
whether the complete document or merely a child is to be compiled
even without using the directive |\childdocof|.
This method is deprecated because it is less robust
and there is no compelling reason to use it;
it is merely provided for backward compatibility
and it may be removed in future versions.

If the detection mechanism is to be used,
it is mandatory to correctly specify
the filename of the main file as the argument of |\childdocmain|:
%
\begin{center}
\begin{tabular}{l}
|\input{childdoc.def}|\\
|\childdocmain{|\textit{main}|}|\\
\end{tabular}
\end{center}
%
If |\jobname| does not match the argument \textit{main} of |\childdocmain|,
it is assumed that |\jobname| points to the child file to be compiled.
When using |\childdocmain| with the main file specified as argument,
it suffices to start a child file
with just |\input{|\textit{main}|}|
without loading of the package and using |\childdocof|.
If instead all processing is done
with the appropriate \textsf{childdoc} directives,
the argument of \textit{main} of |\childdocmain| can be empty.

An alternative version of the command line processing described
in \secref{sec:commandline} using the detection mechanism reads:
%
\begin{center}
|... -jobname "|\textit{target}|" "|[\textit{flags}]%
[|\def\jobname{|\textit{dest}|}|]|\input{|\textit{main}|}"|
\end{center}

%%%%%%%%%%%%%%%%%%%%%%%%%%%%%%%%%%%%%%%%%%%%%%%%%%%%%%%%%%%%%%%%%%%%%%%%%%%%%%%%
\subsection{Manual Code}
\label{sec:manual}

In case one cannot be certain whether the definitions file |childdoc.def|
is installed on the target \TeX{} distribution
and one prefers not to ship it,
it is conceivable to paste a few relevant commands into the sources.

To that end, drop all statements |\input{childdoc.def}|
and perform the replacements as outlined below.
Instead of |\childdocmain{|\textit{main}|}| add the following code
to the top of the main file:
%
\begin{center}
\begin{tabular}{l}
|\||ifdefined\childdocname\endinput\||fi\newif\ifchilddoc|\\
|\edef\childdocname{\scantokens\expandafter{\jobname\noexpand}}|\\
|\def\childdocmain{|\textit{main}|}\||ifx\childdocmain\childdocname\||else|\\
|\childdoctrue\includeonly{\childdocname}\let\jobname\childdocmain\||fi|\\
\end{tabular}
\end{center}
%
Instead of |\childdocof{|\textit{main}|}| just include the main file
at the top of each child file:
%
\begin{center}
|\input{|\textit{main}|}|
\end{center}
%
A simple redirection |\childdocforward{|\textit{dest}|}| is achieved by:
%
\begin{center}
|\def\jobname{|\textit{dest}|}\input{\jobname}|
\end{center}
%
The redirection with prefix
|\childdocforwardprefix[|\textit{prefix}|]{|\textit{dest}|}|
is accomplished by:
%
\begin{center}
\begin{tabular}{l}
|{\edef\jobname{\scantokens\expandafter{\jobname\noexpand}}|\\
|\def\redirectjob |\textit{prefix}|#1~~~{\gdef\jobname{|\textit{dest}|#1}}|\\
|\expandafter\redirectjob\jobname~~~}\input{\jobname}|
\end{tabular}
\end{center}

In an alternative approach,
child documents can be compiled by a specific command line
without additional code or specific definitions:
%
\begin{center}
|... -jobname "|\textit{target}|" "|[\textit{flags}]%
|\includeonly{|\textit{dest}|}\input{|\textit{main}|}"|
\end{center}
%

%%%%%%%%%%%%%%%%%%%%%%%%%%%%%%%%%%%%%%%%%%%%%%%%%%%%%%%%%%%%%%%%%%%%%%%%%%%%%%%%
%%%%%%%%%%%%%%%%%%%%%%%%%%%%%%%%%%%%%%%%%%%%%%%%%%%%%%%%%%%%%%%%%%%%%%%%%%%%%%%%
\section{Information}

%%%%%%%%%%%%%%%%%%%%%%%%%%%%%%%%%%%%%%%%%%%%%%%%%%%%%%%%%%%%%%%%%%%%%%%%%%%%%%%%
\subsection{Copyright}

Copyright \copyright{} 2017--2018 Niklas Beisert

This work may be distributed and/or modified under the
conditions of the \LaTeX{} Project Public License, either version 1.3
of this license or (at your option) any later version.
The latest version of this license is in
  \url{http://www.latex-project.org/lppl.txt}
and version 1.3 or later is part of all distributions of \LaTeX{}
version 2005/12/01 or later.

This work has the LPPL maintenance status `maintained'.

The Current Maintainer of this work is Niklas Beisert.

This work consists of the files |README.txt|, |childdoc.ins| and |childdoc.dtx|
as well as the derived files |childdoc.def|, |cdocsamp.tex|
with |cdocsch1.tex|, |cdocsch2.tex|, |cdocspt3.tex|, |cdocspt4.tex|,
|cdocsdrf.tex|, |cdocsfn1.tex|, |cdocsfn2.tex|
as well as |childdoc.pdf|.

%%%%%%%%%%%%%%%%%%%%%%%%%%%%%%%%%%%%%%%%%%%%%%%%%%%%%%%%%%%%%%%%%%%%%%%%%%%%%%%%
\subsection{Files and Installation}

The package consists of the files:
%
\begin{center}
\begin{tabular}{ll}
    |README.txt|   & readme file \\
    |childdoc.ins| & installation file \\
    |childdoc.dtx| & source file \\
    |childdoc.def| & definition file \\
    |cdocsamp.tex| & sample main file \\
    |cdocsch1.tex| & sample include file \\
    |cdocsch2.tex| & sample include file \\
    |cdocspt3.tex| & sample part file \\
    |cdocspt4.tex| & sample part file \\
    |cdocsdrf.tex| & sample redirection file \\
    |cdocsfn1.tex| & sample redirection file \\
    |cdocsfn2.tex| & sample redirection file \\
    |childdoc.pdf| & manual
\end{tabular}
\end{center}
%
The distribution consists of the files
|README.txt|, |childdoc.ins| and |childdoc.dtx|.
%
\begin{itemize}
\item
Run (pdf)\LaTeX{} on |childdoc.dtx|
to compile the manual |childdoc.pdf| (this file).
\item
Run \LaTeX{} on |childdoc.ins| to create the definitions file |childdoc.def|
and the sample |cdocsamp.tex| with include files
|cdocsch1.tex|, |cdocsch2.tex|, |cdocspt3.tex|, |cdocspt4.tex|,
|cdocsdrf.tex|, |cdocsfn1.tex|, |cdocsfn2.tex|.
Then copy the file |childdoc.def| to an appropriate directory of your \LaTeX{}
distribution, e.g.\ \textit{texmf-root}|/tex/latex/childdoc|.
\end{itemize}

%%%%%%%%%%%%%%%%%%%%%%%%%%%%%%%%%%%%%%%%%%%%%%%%%%%%%%%%%%%%%%%%%%%%%%%%%%%%%%%%
\subsection{Related CTAN Packages}

There are several other packages which offer a similar functionality:
%
\begin{itemize}
\item
The packages
\href{http://ctan.org/pkg/docmute}{\textsf{docmute}},
\href{http://ctan.org/pkg/includex}{\textsf{includex}} and
\href{http://ctan.org/pkg/standalone}{\textsf{standalone}}
provide commands to include only the document body of
a child file thus allowing both files to be compiled individually.
\item
The packages \href{http://ctan.org/pkg/subdocs}{\textsf{subdocs}}
and \href{http://ctan.org/pkg/subfiles}{\textsf{subfiles}}
provide structures in which the main and child documents can be
encapsulated and allowing them to be compiled individually.
The inclusion mechanism is different from the conventional |\include|.
\item
The package \href{http://ctan.org/pkg/combine}{\textsf{combine}}
is an elaborate solution to combine several documents into one.
\end{itemize}
%
See also the CTAN topic \href{http://ctan.org/topic/subdocs}{\textsf{subdocs}}
for further related packages.
The present package differs from the above solutions in that
a document structure constructed with the conventional |\include| mechanism
just needs two extra commands at the top of every file
such that all constituent files can be compiled individually.

%%%%%%%%%%%%%%%%%%%%%%%%%%%%%%%%%%%%%%%%%%%%%%%%%%%%%%%%%%%%%%%%%%%%%%%%%%%%%%%%
%\subsection{Feature Suggestions}
%
%The following is a list of features which may be useful for future
%versions of this package:
%%
%\begin{itemize}
%\item
%\ldots
%\end{itemize}

%%%%%%%%%%%%%%%%%%%%%%%%%%%%%%%%%%%%%%%%%%%%%%%%%%%%%%%%%%%%%%%%%%%%%%%%%%%%%%%%
\subsection{Revision History}

%%%%%%%%%%%%%%%%%%%%%%%%%%%%%%%%%%%%%%%%
\paragraph{v2.0:} 2018/12/30

\begin{itemize}
\item
immediate forward processing
\item
added |\childdocby| mechanism
\item
manual restructured
\end{itemize}

%%%%%%%%%%%%%%%%%%%%%%%%%%%%%%%%%%%%%%%%
\paragraph{v1.6:} 2018/01/17

\begin{itemize}
\item
application for development of include files
\item
corrections to manual
\end{itemize}

%%%%%%%%%%%%%%%%%%%%%%%%%%%%%%%%%%%%%%%%
\paragraph{v1.5:} 2017/05/21

\begin{itemize}
\item
more complete structuring introduced
\item
|\childdocof| introduced
\item
|\childdoc| renamed to |\childdocmain|
\item
|\childredirect| renamed to |\childdocforward| and |\childdocforwardprefix|
and functionality expanded
\end{itemize}

%%%%%%%%%%%%%%%%%%%%%%%%%%%%%%%%%%%%%%%%
\paragraph{v1.0:} 2017/04/27

\begin{itemize}
\item
manual and install package
\item
first version published on CTAN
\end{itemize}

%%%%%%%%%%%%%%%%%%%%%%%%%%%%%%%%%%%%%%%%
\paragraph{v0.6:} 2017/04/26

\begin{itemize}
\item
redirection mechanism added
\end{itemize}

%%%%%%%%%%%%%%%%%%%%%%%%%%%%%%%%%%%%%%%%
\paragraph{v0.5:} 2017/04/26

\begin{itemize}
\item
functionality in definition file
\end{itemize}


%%%%%%%%%%%%%%%%%%%%%%%%%%%%%%%%%%%%%%%%%%%%%%%%%%%%%%%%%%%%%%%%%%%%%%%%%%%%%%%%
%%%%%%%%%%%%%%%%%%%%%%%%%%%%%%%%%%%%%%%%%%%%%%%%%%%%%%%%%%%%%%%%%%%%%%%%%%%%%%%%
%%%%%%%%%%%%%%%%%%%%%%%%%%%%%%%%%%%%%%%%%%%%%%%%%%%%%%%%%%%%%%%%%%%%%%%%%%%%%%%%
\appendix

\settowidth\MacroIndent{\rmfamily\scriptsize 000\ }

 \DocInput{childdoc.dtx}

\end{document}
%</driver>
% \fi
%
% %%%%%%%%%%%%%%%%%%%%%%%%%%%%%%%%%%%%%%%%%%%%%%%%%%%%%%%%%%%%%%%%%%%%%%%%%%%%%%
% %%%%%%%%%%%%%%%%%%%%%%%%%%%%%%%%%%%%%%%%%%%%%%%%%%%%%%%%%%%%%%%%%%%%%%%%%%%%%%
% \section{Sample}
%\iffalse
%<*samplemain>
%\fi
%
% The following presents a sample document
% with two chapters, two parts, a title page,
% a compile flag as well as three forwarding files to set the flag.
% It consists of eight |.tex| files:
% \begin{center}
% \begin{tabular}{ll}
% |cdocsamp.tex|&main file\\
% |cdocsch1.tex|&include file for chapter 1\\
% |cdocsch2.tex|&include file for chapter 2\\
% |cdocspt3.tex|&include file for part 3\\
% |cdocspt4.tex|&include file for part 4\\
% |cdocsdrf.tex|&forwarding file for main file in draft mode\\
% |cdocsfi1.tex|&forwarding file for final version of chapter 1\\
% |cdocsfi2.tex|&forwarding file for final version of chapter 2\\
% \end{tabular}
% \end{center}
% Each of the eight files can be compiled directly by the \LaTeX{} compiler.
%
% %%%%%%%%%%%%%%%%%%%%%%%%%%%%%%%%%%%%%%
% \paragraph{Main File.}
%
% The main file is called |cdocsamp.tex|.
%
% Load the \textsf{childdoc} definitions and
% declare the filename for the main document:
%    \begin{macrocode}
\input{childdoc.def}
\childdocmain{}
%    \end{macrocode}

% Optional override for |\version| flag:
%    \begin{macrocode}
%%\ifchilddoc\else\providecommand{\version}{draft}\fi
%    \end{macrocode}

% Define the default values for the |\version| flag
% (|final| for the main file and |draft| for childs):
%    \begin{macrocode}
\ifchilddoc
\providecommand{\version}{draft}
\else
\providecommand{\version}{final}
\fi
%    \end{macrocode}

% Load the standard document class:
%    \begin{macrocode}
\documentclass[12pt]{article}
%    \end{macrocode}

% Start the document body:
%    \begin{macrocode}
\begin{document}
%    \end{macrocode}

% Declare a title page.
% Print title, part of document being processed and version flag:
%    \begin{macrocode}
\addtocounter{page}{-1}
\begin{center}
{\LARGE\bfseries{}childdoc example\par}
\vspace{1cm}
\ifchilddoc
\ifchilddocmanual part\else chapter\fi:
`\childdocname' of `\childdocjob'\par
\else
main document: `\childdocjob'\par
\fi
version: \version\par
\end{center}
\newpage
%    \end{macrocode}

% Manually include selected file,
% otherwise process as usual:
%    \begin{macrocode}
\ifchilddocmanual
\section*{part `\childdocname'}
\input{\childdocname}
\else
%    \end{macrocode}

% Include the two chapters:
%    \begin{macrocode}
\include{cdocsch1}
\include{cdocsch2}
%    \end{macrocode}

% Include the two parts unless only chapters should be displayed:
%    \begin{macrocode}
\ifchilddoc\else
\section{part three}
\input{cdocspt3}
\section{part four}
\input{cdocspt4}
\fi
%    \end{macrocode}

% Process as usual until here:
%    \begin{macrocode}
\fi
%    \end{macrocode}

% End of document body:
%    \begin{macrocode}
\end{document}
%    \end{macrocode}
%\iffalse
%</samplemain>
%\fi
%
% %%%%%%%%%%%%%%%%%%%%%%%%%%%%%%%%%%%%%%
% \paragraph{Chapter Include Files.}
%
% The include files are called |cdocsch1.tex| and |cdocsch2.tex|.
%
%\iffalse
%<*samplechap1|samplechap2>
%\fi

% Optional override for |\version| flag:
%    \begin{macrocode}
%%\providecommand{\version}{final}
%    \end{macrocode}

% Include the main document:
%    \begin{macrocode}
\input{childdoc.def}
\childdocof{cdocsamp}
%    \end{macrocode}

%\iffalse
%</samplechap1|samplechap2>
%\fi
%
%\iffalse
%<*samplechap1>
%\fi
% Some text for chapter 1:
%    \begin{macrocode}
\section{one}
some text in chapter one
%    \end{macrocode}

%\iffalse
%</samplechap1>
%\fi
% Some text for chapter 2:
%\iffalse
%<*samplechap2>
%\fi
%    \begin{macrocode}
\section{two}
more text in chapter two
%    \end{macrocode}

%\iffalse
%</samplechap2>
%\fi
%
% %%%%%%%%%%%%%%%%%%%%%%%%%%%%%%%%%%%%%%
% \paragraph{Part Include Files.}
%
% The include files are called |cdocspt3.tex| and |cdocspt4.tex|.
%
%\iffalse
%<*samplepart3|samplepart4>
%\fi

% Optional override for |\version| flag:
%    \begin{macrocode}
%%\providecommand{\version}{final}
%    \end{macrocode}

% Include the main document:
%    \begin{macrocode}
\input{childdoc.def}
\childdocby{cdocsamp}
%    \end{macrocode}

%\iffalse
%</samplepart3|samplepart4>
%\fi
%
%\iffalse
%<*samplepart3>
%\fi
% Some text for part 3:
%    \begin{macrocode}
some text in part three
%    \end{macrocode}

%\iffalse
%</samplepart3>
%\fi
% Some text for part 4:
%\iffalse
%<*samplepart4>
%\fi
%    \begin{macrocode}
more text in part four
%    \end{macrocode}

%\iffalse
%</samplepart4>
%\fi
%
% %%%%%%%%%%%%%%%%%%%%%%%%%%%%%%%%%%%%%%
% \paragraph{Forwarding for a Complete Draft.}
%
% The following forwarding file |cdocsdrf.tex|
% compiles the main document in draft mode:
%\iffalse
%<*sampledraft>
%\fi
%    \begin{macrocode}
\def\version{draft}
\input{childdoc.def}
\childdocforward{cdocsamp}
%    \end{macrocode}

%\iffalse
%</sampledraft>
%\fi
%
% %%%%%%%%%%%%%%%%%%%%%%%%%%%%%%%%%%%%%%
% \paragraph{Forwarding for Final Version of the Chapters.}
%
% The following forwarding files |cdocsfn1.tex| and |cdocsfn2.tex|
% (with identical content)
% compile the final versions of the child documents
% |cdocsch1.tex| and |cdocsch2.tex|, respectively:
%\iffalse
%<*samplefinal>
%\fi
%    \begin{macrocode}
\def\version{final}
\input{childdoc.def}
\childdocforwardprefix[cdocsamp]{cdocsfn}{cdocsch}
%    \end{macrocode}

%\iffalse
%</samplefinal>
%\fi
%
% %%%%%%%%%%%%%%%%%%%%%%%%%%%%%%%%%%%%%%
% \paragraph{Command Line Processing.}
%
% The following three command lines generate the output files
% |cdocscld|, |cdocscl1| and |cdocscl2|
% which should be identical to
% |cdocsdrf|, |cdocsch1| and |cdocsfn2|, respectively:
% \begin{center}
% \begin{tabular}{l}
% |latex -jobname cdocscld \|\\
% |  "\def\version{draft}\input{childdoc.def}\childdocforward{cdocsamp}"|\\
% |latex -jobname cdocscl1 \|\\
% |  "\input{childdoc.def}\childdocforward[cdocsamp]{cdocsch1}"|\\
% |latex -jobname cdocscl2 \|\\
% |  "\def\version{final}\input{childdoc.def}\childdocforward{cdocsch2}"|
% \end{tabular}
% \end{center}
% Note that the trailing backslash on each first line
% merely continues the input to the second line
% (for convenient cut ant paste).
% Furthermore, the command |latex| can be replaced by any
% of its alternative versions such as |pdflatex|.
%
% %%%%%%%%%%%%%%%%%%%%%%%%%%%%%%%%%%%%%%%%%%%%%%%%%%%%%%%%%%%%%%%%%%%%%%%%%%%%%%
% %%%%%%%%%%%%%%%%%%%%%%%%%%%%%%%%%%%%%%%%%%%%%%%%%%%%%%%%%%%%%%%%%%%%%%%%%%%%%%
% \section{Implementation}
%\iffalse
%<*package>
%\fi
%
% This section describes the definitions file |childdoc.def|.

% The definitions cannot be loaded using |\usepackage| or |\RequirePackage|
% which has a mechanism to prevent loading a style file more than once.
% When loading the definitions by means of |\input|
% multiple instances have to be prevented manually:
%\iffalse
%This code needs to be before the `\ProvidesFile' directive
%which is defined at the beginning of this file.
%Therefore it is also placed there and commented out here.
%</package>
%<*discard>
%\fi
%    \begin{macrocode}
\ifdefined\childdocmain\endinput\fi
%    \end{macrocode}
%\iffalse
%</discard>
%<*package>
%\fi
%
% \macro{\ifchilddoc}
% \macro{\ifchilddocmanual}
% The conditional |\ifchilddoc| tells whether a
% child (true) or main (false) document is being compiled.
% The conditional |\ifchilddocmanual| tells whether
% the |\includeonly| mechanism is used (false) or
% the selection of child files must be performed manually (true).
% The definitions initialise to false:
%    \begin{macrocode}
\newif\ifchilddoc
\newif\ifchilddocmanual
%    \end{macrocode}

% \macro{\childdocname}
% \macro{\childdocjob}
% The macro |\childdocname| stores the name of the main document
% to be compiled. The macro |\childdocjob| stores the name of
% the document on which the \LaTeX{} compiler was originally invoked.
% The content of |\jobname| cannot be compared
% to filenames specified in the source due to different catcodes.
% The following code rescans |\jobname|, stores the result
% in |\childdocname| and saves a copy in |\childdocjob|:
%    \begin{macrocode}
\edef\childdocname{\scantokens\expandafter{\jobname\noexpand}}
\let\childdocjob\childdocname
%    \end{macrocode}

% \macro{\childdocdisable}
% The macro |\childdocdisable| prevents the main file
% from being processed more than once.
% At this stage, the main document command |\childdocmain|
% is assumed to be called once again where it should do nothing.
% Any subsequent call to it should prevent
% a secondary processing of the main document
% It overwrites the forwarding commands
% |\childdocof| and |\childdocforward|
% with empty macros to prevent further inclusions of the main document:
%    \begin{macrocode}
\newcommand{\childdocdisable}
{
  \renewcommand{\childdocmain}[1]{\renewcommand{\childdocmain}[1]{\endinput}}
  \renewcommand{\childdocof}[1]{}
  \renewcommand{\childdocby}[2][]{}
  \renewcommand{\childdocforward}[2][]{}
  \renewcommand{\childdocdisable}{}
}
%    \end{macrocode}

% \macro{\childdocmain}
% The macro |\childdocmain| is to be called at the top of the main file
% with nothing or the main filename (without extension) as argument.
% First, it breaks loops.
% If the argument is not empty and does not match |\childdocname|
% (which is set by the first inclusion of |childdoc.def|),
% |\ifchilddoc| is set to true, |\includeonly| is applied to the child file
% and |\jobname| is set to the main file
% (for proper handling of |.aux| files):
%    \begin{macrocode}
\newcommand{\childdocmain}[1]
{
  \childdocdisable\childdocmain{}
  \if?#1?\else
    \begingroup
      \def\childdoctmp{#1}
      \ifx\childdoctmp\childdocname
        \def\childdoctmp{}
      \else
        \def\childdoctmp
        {
          \childdoctrue
          \includeonly{\childdocname}
          \def\childdocjob{#1}
          \def\jobname{#1}
        }
      \fi
      \expandafter
    \endgroup
    \childdoctmp
  \fi
}
%    \end{macrocode}

% \macro{\childdocof}
% The command |\childdocof| redirects
% compilation to the main file |#1|.
%    \begin{macrocode}
\newcommand{\childdocof}[1]
{
  \childdocdisable
  \childdoctrue
  \includeonly{\childdocname}
  \def\jobname{#1}
  \def\childdocjob{#1}
  \input{#1}
}
%    \end{macrocode}

% \macro{\childdocby}
% The command |\childdocby| ....
%    \begin{macrocode}
\newcommand{\childdocby}[2][]
{
  \childdocdisable
  \childdoctrue
  \childdocmanualtrue
  \if?#1?\else
    \def\jobname{#2}
  \fi
  \def\childdocjob{#2}
  \input{#2}
  \endinput
}
%    \end{macrocode}

% \macro{\childdocforward}
% The command |\childdocforward| redirects
% compilation to the main file or
% (if the optional argument is given) a child file.
% Parameters are set as if the main file
% or a child file starting with |\childdocof| was compiled.
% Then compilation is handed over to the main file:
%    \begin{macrocode}
\newcommand{\childdocforward}[2][]
{
  \begingroup
    \if?#1?
      \def\childdoctmp
      {
        \def\childdocname{#2}
        \def\childdocjob{#2}
        \def\jobname{#2}
        \input{#2}
        \endinput
      }
    \else
      \def\childdoctmp
      {
        \childdocdisable
        \def\childdocname{#2}
        \childdoctrue
        \includeonly{#2}
        \def\childdocjob{#1}
        \def\jobname{#1}
        \input{#1}
        \endinput
      }
    \fi
    \expandafter
  \endgroup
  \childdoctmp
}
%    \end{macrocode}

% \macro{\childdocforwardprefix}
% The command |\childdocforwardprefix| redirects
% compilation to the main or a child file by means of a pattern.
% The prefix |#1| in the current filename is replaced by |#2|
% and the suffix of the current filename is kept
% (it is assumed that the filename does not contain the substring `|~~~|'
% which is used as a delimiter).
% Compilation is handed over to the new file by |\childdocforward|:
%    \begin{macrocode}
\newcommand{\childdocforwardprefix}[3][]
{
  \begingroup
    \def\childdocextract #2##1~~~{\def\childdoctmp{\childdocforward[#1]{#3##1}}}
    \expandafter\childdocextract\childdocname~~~
    \expandafter
  \endgroup
  \childdoctmp
}
%    \end{macrocode}

% \macro{\childdoc}
% The deprecated macro |\childdoc| is a legacy version of |\childdocmain|:
%    \begin{macrocode}
\newcommand{\childdoc}{\childdocmain}
%    \end{macrocode}

% \macro{\childdocredirect}
% The deprecated macro |\childdocredirect| is a legacy version
% of |\childdocforward| and |\childdocforwardprefix|:
%    \begin{macrocode}
\newcommand{\childdocredirect}[2][]
{
  \begingroup
    \if?#1?
      \def\childdoctmp{\childdocforward{#2}}
    \else
      \def\childdoctmp{\childdocforwardprefix{#1}{#2}}
    \fi
    \expandafter
  \endgroup
  \childdoctmp
}
%    \end{macrocode}

%\iffalse
%</package>
%\fi
%
\endinput
\childdocforward[cdocsamp]{cdocsch1}"|\\
% |latex -jobname cdocscl2 \|\\
% |  "\def\version{final}% \iffalse
%
% childdoc.dtx Copyright (C) 2017-2018 Niklas Beisert
%
% This work may be distributed and/or modified under the
% conditions of the LaTeX Project Public License, either version 1.3
% of this license or (at your option) any later version.
% The latest version of this license is in
%   http://www.latex-project.org/lppl.txt
% and version 1.3 or later is part of all distributions of LaTeX
% version 2005/12/01 or later.
%
% This work has the LPPL maintenance status `maintained'.
%
% The Current Maintainer of this work is Niklas Beisert.
%
% This work consists of the files childdoc.dtx and childdoc.ins
% and the derived files childdoc.def and cdocsamp.tex with
% cdocsch1.tex, cdocsch2.tex, cdocsdrf.tex, cdocsfn1.tex, cdocsfn2.tex.
%
%<package>\ifdefined\childdocmain\endinput\fi
%<package>\ProvidesFile{childdoc.def}[2018/12/30 v2.0 child document driver]
%<samplemain>\ProvidesFile{cdocsamp.tex}[2018/12/30 v2.0 sample for childdoc]
%<*driver>
%\ProvidesFile{childdoc.drv}[2018/12/30 v2.0 childdoc reference manual file]
\PassOptionsToClass{10pt,a4paper}{article}
\documentclass{ltxdoc}

\usepackage[margin=35mm]{geometry}
\usepackage{hyperref}
\usepackage{hyperxmp}
\usepackage[usenames]{color}

\hypersetup{colorlinks=true}
\hypersetup{pdfstartview=FitH}
\hypersetup{pdfpagemode=UseNone}
\hypersetup{pdfsource={}}
\hypersetup{pdflang={en-UK}}
\hypersetup{pdfcopyright={Copyright 2017-2018 Niklas Beisert.
  This work may be distributed and/or modified under the
  conditions of the LaTeX Project Public License, either version 1.3
  of this license or (at your option) any later version.}}
\hypersetup{pdflicenseurl={http://www.latex-project.org/lppl.txt}}
\hypersetup{pdfcontactaddress={ETH Zurich, ITP, HIT K,
  Wolfgang-Pauli-Strasse 27}}
\hypersetup{pdfcontactpostcode={8093}}
\hypersetup{pdfcontactcity={Zurich}}
\hypersetup{pdfcontactcountry={Switzerland}}
\hypersetup{pdfcontactemail={nbeisert@itp.phys.ethz.ch}}
\hypersetup{pdfcontacturl={http://people.phys.ethz.ch/\xmptilde nbeisert/}}

\newcommand{\secref}[1]{\hyperref[#1]{section \ref*{#1}}}

\parskip1ex
\parindent0pt
\let\olditemize\itemize
\def\itemize{\olditemize\parskip0pt}

\begin{document}

\title{The \textsf{childdoc} Package}
\hypersetup{pdftitle={The childdoc Package}}
\author{Niklas Beisert\\[2ex]
  Institut f\"ur Theoretische Physik\\
  Eidgen\"ossische Technische Hochschule Z\"urich\\
  Wolfgang-Pauli-Strasse 27, 8093 Z\"urich, Switzerland\\[1ex]
  \href{mailto:nbeisert@itp.phys.ethz.ch}
  {\texttt{nbeisert@itp.phys.ethz.ch}}}
\hypersetup{pdfauthor={Niklas Beisert}}
\hypersetup{pdfsubject={Manual for the LaTeX2e Package childdoc}}
\date{30 December 2018, \textsf{v2.0}}
\maketitle

\begin{abstract}\noindent
\textsf{childdoc} is a \LaTeXe{} package
that enables the direct compilation
of document sections included by |\include|
to individual files.
\end{abstract}

\begingroup
\parskip0ex
\tableofcontents
\endgroup

%%%%%%%%%%%%%%%%%%%%%%%%%%%%%%%%%%%%%%%%%%%%%%%%%%%%%%%%%%%%%%%%%%%%%%%%%%%%%%%%
%%%%%%%%%%%%%%%%%%%%%%%%%%%%%%%%%%%%%%%%%%%%%%%%%%%%%%%%%%%%%%%%%%%%%%%%%%%%%%%%
\section{Introduction}

\LaTeX{} provides a mechanism to structure a large document (such as a book)
into a main file and several child files (containing the chapters)
using the |\include| command.
This mechanism is beneficial for documents
which span hundreds of pages in order to
make the source file(s) more manageable.
Moreover, compilation can be restricted to
selected child files by means of the |\includeonly| command.
The latter feature can be used to reduce the compilation time while editing
(this was significantly more useful in the earlier days of \LaTeX{})
or to generate a smaller document which is easier to navigate.
Another application of |\includeonly| is to generate
documents consisting of selected parts of the complete document.

However, there are a few drawbacks of the plain |\include| mechanism:
\begin{itemize}
\item
The child files cannot be compiled on their own,
they can only be compiled via the main file.
A naive editing environment
(such as a text editor with an option
to have the current file processed by \LaTeX)
may require one to switch to the main file before compiling;
attempting to compile the child file produces errors.
\item
The main file must be modified (each time)
to adjust the |\includeonly| command
to the present needs. This easily leaves the main file in a messy state.
\item
The generated document will always carry the filename
of the main document. This is inconvenient if
several child files are to be compiled and
to be kept for distribution.
\end{itemize}

The present package provides a simple interface
to make child files individually compilable by \LaTeX{}.
Compiling a child file then has the same effect as compiling
the main file with an |\includeonly| command
to select the appropriate child.
Moreover the generated document will carry the name of the child
rather than the main file.
This resolves all three above issues.

This feature is meant to make the editing of books,
thesis documents and lecture notes somewhat more convenient.
However, the package can also be used efficiently for
composing a series of documents (such as exercise sheets)
which are typically distributed individually.
It then assists the author in generating the individual documents
(potentially in different versions)
as well as a document containing the collected series.
Another application is in developing style files
or other kinds of included material
where compilation of the style file could redirect
to a sample or test file.

%%%%%%%%%%%%%%%%%%%%%%%%%%%%%%%%%%%%%%%%%%%%%%%%%%%%%%%%%%%%%%%%%%%%%%%%%%%%%%%%
%%%%%%%%%%%%%%%%%%%%%%%%%%%%%%%%%%%%%%%%%%%%%%%%%%%%%%%%%%%%%%%%%%%%%%%%%%%%%%%%
\section{Usage}

First of all, the package \textsf{childdoc} is \emph{not} a standard
\LaTeXe{} |.sty| style file! Therefore it needs to be invoked in
a non-standard way.

%%%%%%%%%%%%%%%%%%%%%%%%%%%%%%%%%%%%%%%%%%%%%%%%%%%%%%%%%%%%%%%%%%%%%%%%%%%%%%%%
\subsection{Included Files}
\label{sec:include}

%%%%%%%%%%%%%%%%%%%%%%%%%%%%%%%%%%%%%%%%
\DescribeMacro{\childdocmain}
To use the package, add the commands
\begin{center}
\begin{tabular}{l}
|\input{childdoc.def}|\\
|\childdocmain{}|\\
\end{tabular}
\end{center}
at the very top of the main \LaTeX{} file,
in particular \emph{before} the |\documentclass| statement!
The argument of |\childdocmain| should be left empty
(but it must be present).

%%%%%%%%%%%%%%%%%%%%%%%%%%%%%%%%%%%%%%%%
\DescribeMacro{\childdocof}
Furthermore, add the commands
\begin{center}
\begin{tabular}{l}
|\input{childdoc.def}|\\
|\childdocof{|\textit{main}|}|\\
\end{tabular}
\end{center}
at the top of every child file \textit{child}
which is included by |\include{|\textit{child}|}|
from within the main file
(or at least for those files to be compiled individually).
The argument \textit{main} must be the filename of the main file.

There are a couple of
considerations in setting up the main and child documents:

%%%%%%%%%%%%%%%%%%%%%%%%%%%%%%%%%%%%%%%%
\paragraph{Restrictions.}

Please note the following restrictions:
\begin{itemize}
\item
|\childdocmain| must be called with one argument \textit{main}
to ensure compatibility with earlier version of the package.
It must either be empty (|\childdocmain{}|)
or precisely match the filename of the main file in which it is specified.
See \secref{sec:detection} for further information.
\item
The filename \textit{main} must be specified without the |.tex| extension.
\item
The filename \textit{main} is case sensitive
(even in case-insensitive file systems)
due to internal string comparison.
\item
The argument \textit{main} should be fully expanded, it cannot be a macro.
\item
Subdirectories and special characters should be avoided in filenames.
\item
The command |\childdocmain{|\textit{main}|}| must be followed by a whitespace.
It should not be followed immediately by another command
or by a comment mark `|%|'.
This is because the \TeX{} parser reads the token immediately following
the argument of |\childdocmain| and puts it
at the beginning of every child section;
however, a white\-space is ignored.
\end{itemize}

%%%%%%%%%%%%%%%%%%%%%%%%%%%%%%%%%%%%%%%%
\paragraph{Content of Main File.}

It is advisable to place all content in the child files included by |\include|.
Any output contained in the main file will appear in all child documents
unless suppressed manually;
it cannot be suppressed automatically by the |\includeonly| directive
and thus should normally be avoided.
A method to include some content in the main file
by means of conditional processing is described in \secref{sec:conditional}.

%%%%%%%%%%%%%%%%%%%%%%%%%%%%%%%%%%%%%%%%
\paragraph{Page Numbering.}

When only a part of the document is compiled,
the appropriate numbering of pages
(as well as other status parameters)
is determined from the |.aux| files.
The latter contain information from previous passes.
However this information needs to propagate through
all intermediate child documents.
Therefore the page numbering in child documents may well
be inconsistent until the complete document is compiled at least once.

A useful (if unconventional) way to always ensure a consistent
page numbering is to restart the numbering in each child document
and denote the pages by `\textit{child}|.|\textit{page}'
where \textit{child} represents the chapter/section number of the child file.
This can be achieved by the command
|\numberwithin{page}{|\textit{child}|}|
of the \textsf{amsmath} package
where \textit{child} can be |chapter| or |section|
depending on the chosen structuring.
Alternatively, one can modify the macro |\thepage| appropriately
and reset the counter |page| at the start of each child file.

%%%%%%%%%%%%%%%%%%%%%%%%%%%%%%%%%%%%%%%%%%%%%%%%%%%%%%%%%%%%%%%%%%%%%%%%%%%%%%%%
\subsection{Conditional Processing}
\label{sec:conditional}

The package provides a mechanism to compile different versions
of a document. To customise the versions further some conditional processing
can come in handy to distinguish which version is being compiled.
The package provides two macros to describe the compilation context:

%%%%%%%%%%%%%%%%%%%%%%%%%%%%%%%%%%%%%%%%
\DescribeMacro{\ifchilddoc}
The conditional |\ifchilddoc| distinguishes between the compilation of
child documents and the main document:
%
\begin{center}
|\ifchilddoc |\textit{child-code}| |[|\||else |\textit{main-code}]| \||fi|
\end{center}

%%%%%%%%%%%%%%%%%%%%%%%%%%%%%%%%%%%%%%%%
\DescribeMacro{\childdocname}
\DescribeMacro{\childdocjob}
The macro |\childdocname| contains the filename (without extension)
of the main or child file being processed.
Note that |\childdocjob| will always contain the name of the main file.

%%%%%%%%%%%%%%%%%%%%%%%%%%%%%%%%%%%%%%%%
\paragraph{Title Page.}

Conditional processing can be used to include a title or banner page
in the main document when proper precautions are taken.
Importantly, the code in the main file should ensure that the page counter
(as well as other status parameters which are stored in the |.aux| files)
takes the same value after the conditional processing.
Otherwise the page numbers may take divergent values
depending on which part is compiled.

For example, a title page could be declared by:
%
\begin{center}
\begin{tabular}{l}
|\ifchilddoc\||else|\\
|\addtocounter{page}{-1}|\\
\textit{code for title page}\\
|\newpage|\\
|\||fi|
\end{tabular}
\end{center}
%
A banner page for the child documents can be generated by:
%
\begin{center}
\begin{tabular}{l}
|\ifchilddoc|\\
|\addtocounter{page}{-1}|\\
\textit{code for banner page}\\
|\newpage|\\
|\||fi|
\end{tabular}
\end{center}
%
Here one could write a message such as:
\begin{center}
|This is the part \childdocname{} of \childdocjob{}.|
\end{center}

%%%%%%%%%%%%%%%%%%%%%%%%%%%%%%%%%%%%%%%%%%%%%%%%%%%%%%%%%%%%%%%%%%%%%%%%%%%%%%%%
\subsection{Flags}
\label{sec:flags}

The package makes it easy to generate different versions
of the main or child documents.
To this end compilation flags can be defined
and assigned different default values.
They will be particularly useful in conjunction
with the forwarding mechanism described in \secref{sec:forward}.

For example, it may be useful to have a flag |\version|
which can be set to |draft| or |final|.
The document source will contain some conditional code
depending on the value of |\version|.
Suppose further, the flag should default to |final| for the main file
and to |draft| for child files
which is a natural assignment for editing the document.
This is achieved by placing the following code
in the preamble of the main document
(below the |\childdocmain| directive):
%
\begin{center}
\begin{tabular}{l}
|\ifchilddoc|\\
|\providecommand{\version}{draft}|\\
|\||else|\\
|\providecommand{\version}{final}|\\
|\||fi|
\end{tabular}
\end{center}
%
The definition by |\providecommand| makes sure
that previous definitions are not overwritten.
Further statements |\providecommand{\version}{...}|
can thus be added before the above code to override it.

For the main file, one might add a line
(between |\childdocmain| and the above block)
%
\begin{center}
|%\ifchilddoc\||else\providecommand{\version}{draft}\||fi|
\end{center}
%
which can be uncommented to produce a draft version.
Likewise one can add a line to the very top of a child file
(above the |\childdocof{|\textit{main}|}| directive)
%
\begin{center}
|%\providecommand{\version}{final}|
\end{center}
%
which can be uncommented to produce the final version of this child document.

%%%%%%%%%%%%%%%%%%%%%%%%%%%%%%%%%%%%%%%%%%%%%%%%%%%%%%%%%%%%%%%%%%%%%%%%%%%%%%%%
\subsection{Forwarding}
\label{sec:forward}

Different versions of the main or child documents
using compilation flags as described in \secref{sec:flags}
can be (permanently) stored in different files
for convenient compilation, viewing and distribution.
To this end, the package defines a command
to pass on compilation to a different file:

%%%%%%%%%%%%%%%%%%%%%%%%%%%%%%%%%%%%%%%%
\DescribeMacro{\childdocforward}
The command |\childdocforward| redirects processing to
another source file:
%
\begin{center}
\begin{tabular}{l}
|\input{childdoc.def}|\\
|\childdocforward[|\textit{main}|]{|\textit{dest}|}|\\
\end{tabular}
\end{center}
%
The argument \textit{dest} is the destination file
(without extension).
It should be the main file or one of the child files.
Note that further \textsf{childdoc} directives
such as |\childdocof| and |\childdocforward|
in the indicated file will be processed in this form.
The optional argument \textit{main}
passes on directly to the main file \textit{main}
while pretending to compile the child \textit{dest}.
This form behaves as if \textit{dest}
issues |\childdocof{|\textit{main}|}| right away,
and no further \textsf{childdoc} directives will be processed.

%%%%%%%%%%%%%%%%%%%%%%%%%%%%%%%%%%%%%%%%
\DescribeMacro{\...prefix}
In the alternative form |\childdocforwardprefix|,
%
\begin{center}
\begin{tabular}{l}
|\input{childdoc.def}|\\
|\childdocforwardprefix[|\textit{main}|]{|\textit{prefix}|}{|\textit{dest}|}|
\end{tabular}
\end{center}
%
the destination file is determined by a pattern
depending on the current file:
To make this work, the current file must be called
`{\textit{prefix}\hspace{0.2em}\textit{suffix}}'
with \textit{prefix} matching precisely the argument.
Processing is then passed on to the file
`{\textit{dest}\hspace{0.2em}\textit{suffix}}'.
Surely, the same effect is achieved by
directly specifying the
argument `{\textit{dest}\hspace{0.2em}\textit{suffix}}'
in the first form.
However, that requires to set up a different file
for each child. With the alternative form of the command
all these files can have exactly the same content
which simplifies setting them up and maintaining them.

For example, the following file |draft.tex|
with a compilation flag |\version| as described in \secref{sec:flags}
compiles the main document as a draft:
%
\begin{center}
\begin{tabular}{l}
|\def\version{draft}|\\
|\input{childdoc.def}|\\
|\childdocforward{|\textit{main}|}|
\end{tabular}
\end{center}
%
Likewise, the following files |final|\textit{nn}|.tex|
compile the final version of the child document
|child|\textit{nn}|.tex|:
%
\begin{center}
\begin{tabular}{l}
|\def\version{final}|\\
|\input{childdoc.def}|\\
|\childdocforwardprefix{final}{child}|
\end{tabular}
\end{center}
%

Note that when several versions of a main file and/or of each child file
are to be generated, it may be convenient to set up a |Makefile| or
shell script to automatise the process.

%%%%%%%%%%%%%%%%%%%%%%%%%%%%%%%%%%%%%%%%%%%%%%%%%%%%%%%%%%%%%%%%%%%%%%%%%%%%%%%%
\subsection{Command Line Processing}
\label{sec:commandline}

The effect of redirection files can also be achieved by invoking
the \LaTeX{} compiler with a more elaborate command line.
Most conveniently this should be done as part
of a shell script or a |Makefile|.

When using \textsf{childdoc} in the main file, the following
command lines effectively perform a redirection
(note that depending on the shell being used,
backslashes may have to be doubled: `|\|' $\to$ `|\\|'):
%
\begin{center}
|... -jobname "|\textit{target}|" |\\|"|[\textit{flags}]%
|\input{childdoc.def}\childdocforward[|\textit{main}|]{|\textit{dest}|}"|
\end{center}
%
Here \textit{target} is the name of the output file,
\textit{main} is the name of the main file
and \textit{dest} is the name of the main or child file to be processed
(all filenames without extensions).
The optional argument \textit{main} can be omitted
if \textit{main} matches \textit{dest}.
Optionally, compilation \textit{flags} can be defined via |\def| commands.
This command line makes the \TeX{} engine believe
it is compiling the file \textit{target}
whose content is specified as the latter parameter.
The provided code then forwards the processing to
\textit{main} or \textit{dest} as described in \secref{sec:forward}.

%%%%%%%%%%%%%%%%%%%%%%%%%%%%%%%%%%%%%%%%%%%%%%%%%%%%%%%%%%%%%%%%%%%%%%%%%%%%%%%%
\subsection{Include by Input}
\label{sec:input}

Including child documents by |\include| has some restrictions by design.
Most notably, the content of a child document always occupies
its own set of pages; pages cannot be shared between child documents.
Usually, this behaviour makes perfect sense
because each child document contain an essential part of the document.
However, in some situations it may be desirable to compose
a document from a collection of parts
without having mandatory page breaks between then.
For this case, the package
provides a mechanism to include parts
by |\input| which can also be processed individually.
However, by construction this mechanism
requires manual handling of the content to be output.

%%%%%%%%%%%%%%%%%%%%%%%%%%%%%%%%%%%%%%%%
\DescribeMacro{\ifchilddocmanual}
The main file should be prepared as usual, see \secref{sec:include}.
However, the document body must make a distinction
between processing of an individual part and of the main document, e.g.:
%
\begin{center}
\begin{tabular}{l}
|\ifchilddocmanual|\\
|\input{\childdocname}|\\
|\||else|\\
\textit{document body with }|\input{|\textit{part}|}|\\
|\||fi|
\end{tabular}
\end{center}
%
The conditional |\ifchilddocmanual| is true whenever
a part to be included by |\input| is being compiled,
and the name of the part is stored in |\childdocname|.

%%%%%%%%%%%%%%%%%%%%%%%%%%%%%%%%%%%%%%%%
\DescribeMacro{\childdocby}
Each part to be included by |\input| should start with:
%
\begin{center}
\begin{tabular}{l}
|\input{childdoc.def}|\\
|\childdocby{|\textit{main}|}|\\
\end{tabular}
\end{center}
%
The directive |\childdocby| is similar to |\childdocof|
described in \secref{sec:include},
but the subsequent selection of content must be done manually.
To that end, both |\ifchilddoc| and |\ifchilddocmanual|
will be true upon processing of a part,
and the name of the part is stored in |\childdocname|.
Note that |\jobname| will be set to the filename of the current part
so that each part receives an individual |.aux| file
that does not interfere with the |.aux| file(s) of the main document.
This behaviour can be altered by the alternative form
|\childdocby[*]{|\textit{main}|}| (with a non-empty optional argument)
which uses the |.aux| file of the main document
by setting |\jobname| to \textit{main}.

%%%%%%%%%%%%%%%%%%%%%%%%%%%%%%%%%%%%%%%%%%%%%%%%%%%%%%%%%%%%%%%%%%%%%%%%%%%%%%%%
\subsection{Driver Development}
\label{sec:driver}

The \textsf{childdoc} mechanism can also be use for the development
of definition files such as \LaTeX{} styles or classes.
This case differs from the above setup with multiple parts
included by |\include| in that no |\includeonly| should be invoked.
This can be achieved by starting the include file
(before |\ProvidesPackage|) with:
%
\begin{center}
\begin{tabular}{l}
|\input{childdoc.def}|\\
|\childdocforward{|\textit{main}|}|\\
\end{tabular}
\end{center}
%
or alternatively with:
%
\begin{center}
\begin{tabular}{l}
|\input{childdoc.def}|\\
|\childdocby{|\textit{main}|}|\\
\end{tabular}
\end{center}
%
Both forms have slightly different effects as described above.
The main file is prepared as usual, see \secref{sec:include}.

%%%%%%%%%%%%%%%%%%%%%%%%%%%%%%%%%%%%%%%%%%%%%%%%%%%%%%%%%%%%%%%%%%%%%%%%%%%%%%%%
\subsection{Legacy Detection}
\label{sec:detection}

The directive |\childdocmain| in the main file can detect
whether the complete document or merely a child is to be compiled
even without using the directive |\childdocof|.
This method is deprecated because it is less robust
and there is no compelling reason to use it;
it is merely provided for backward compatibility
and it may be removed in future versions.

If the detection mechanism is to be used,
it is mandatory to correctly specify
the filename of the main file as the argument of |\childdocmain|:
%
\begin{center}
\begin{tabular}{l}
|\input{childdoc.def}|\\
|\childdocmain{|\textit{main}|}|\\
\end{tabular}
\end{center}
%
If |\jobname| does not match the argument \textit{main} of |\childdocmain|,
it is assumed that |\jobname| points to the child file to be compiled.
When using |\childdocmain| with the main file specified as argument,
it suffices to start a child file
with just |\input{|\textit{main}|}|
without loading of the package and using |\childdocof|.
If instead all processing is done
with the appropriate \textsf{childdoc} directives,
the argument of \textit{main} of |\childdocmain| can be empty.

An alternative version of the command line processing described
in \secref{sec:commandline} using the detection mechanism reads:
%
\begin{center}
|... -jobname "|\textit{target}|" "|[\textit{flags}]%
[|\def\jobname{|\textit{dest}|}|]|\input{|\textit{main}|}"|
\end{center}

%%%%%%%%%%%%%%%%%%%%%%%%%%%%%%%%%%%%%%%%%%%%%%%%%%%%%%%%%%%%%%%%%%%%%%%%%%%%%%%%
\subsection{Manual Code}
\label{sec:manual}

In case one cannot be certain whether the definitions file |childdoc.def|
is installed on the target \TeX{} distribution
and one prefers not to ship it,
it is conceivable to paste a few relevant commands into the sources.

To that end, drop all statements |\input{childdoc.def}|
and perform the replacements as outlined below.
Instead of |\childdocmain{|\textit{main}|}| add the following code
to the top of the main file:
%
\begin{center}
\begin{tabular}{l}
|\||ifdefined\childdocname\endinput\||fi\newif\ifchilddoc|\\
|\edef\childdocname{\scantokens\expandafter{\jobname\noexpand}}|\\
|\def\childdocmain{|\textit{main}|}\||ifx\childdocmain\childdocname\||else|\\
|\childdoctrue\includeonly{\childdocname}\let\jobname\childdocmain\||fi|\\
\end{tabular}
\end{center}
%
Instead of |\childdocof{|\textit{main}|}| just include the main file
at the top of each child file:
%
\begin{center}
|\input{|\textit{main}|}|
\end{center}
%
A simple redirection |\childdocforward{|\textit{dest}|}| is achieved by:
%
\begin{center}
|\def\jobname{|\textit{dest}|}\input{\jobname}|
\end{center}
%
The redirection with prefix
|\childdocforwardprefix[|\textit{prefix}|]{|\textit{dest}|}|
is accomplished by:
%
\begin{center}
\begin{tabular}{l}
|{\edef\jobname{\scantokens\expandafter{\jobname\noexpand}}|\\
|\def\redirectjob |\textit{prefix}|#1~~~{\gdef\jobname{|\textit{dest}|#1}}|\\
|\expandafter\redirectjob\jobname~~~}\input{\jobname}|
\end{tabular}
\end{center}

In an alternative approach,
child documents can be compiled by a specific command line
without additional code or specific definitions:
%
\begin{center}
|... -jobname "|\textit{target}|" "|[\textit{flags}]%
|\includeonly{|\textit{dest}|}\input{|\textit{main}|}"|
\end{center}
%

%%%%%%%%%%%%%%%%%%%%%%%%%%%%%%%%%%%%%%%%%%%%%%%%%%%%%%%%%%%%%%%%%%%%%%%%%%%%%%%%
%%%%%%%%%%%%%%%%%%%%%%%%%%%%%%%%%%%%%%%%%%%%%%%%%%%%%%%%%%%%%%%%%%%%%%%%%%%%%%%%
\section{Information}

%%%%%%%%%%%%%%%%%%%%%%%%%%%%%%%%%%%%%%%%%%%%%%%%%%%%%%%%%%%%%%%%%%%%%%%%%%%%%%%%
\subsection{Copyright}

Copyright \copyright{} 2017--2018 Niklas Beisert

This work may be distributed and/or modified under the
conditions of the \LaTeX{} Project Public License, either version 1.3
of this license or (at your option) any later version.
The latest version of this license is in
  \url{http://www.latex-project.org/lppl.txt}
and version 1.3 or later is part of all distributions of \LaTeX{}
version 2005/12/01 or later.

This work has the LPPL maintenance status `maintained'.

The Current Maintainer of this work is Niklas Beisert.

This work consists of the files |README.txt|, |childdoc.ins| and |childdoc.dtx|
as well as the derived files |childdoc.def|, |cdocsamp.tex|
with |cdocsch1.tex|, |cdocsch2.tex|, |cdocspt3.tex|, |cdocspt4.tex|,
|cdocsdrf.tex|, |cdocsfn1.tex|, |cdocsfn2.tex|
as well as |childdoc.pdf|.

%%%%%%%%%%%%%%%%%%%%%%%%%%%%%%%%%%%%%%%%%%%%%%%%%%%%%%%%%%%%%%%%%%%%%%%%%%%%%%%%
\subsection{Files and Installation}

The package consists of the files:
%
\begin{center}
\begin{tabular}{ll}
    |README.txt|   & readme file \\
    |childdoc.ins| & installation file \\
    |childdoc.dtx| & source file \\
    |childdoc.def| & definition file \\
    |cdocsamp.tex| & sample main file \\
    |cdocsch1.tex| & sample include file \\
    |cdocsch2.tex| & sample include file \\
    |cdocspt3.tex| & sample part file \\
    |cdocspt4.tex| & sample part file \\
    |cdocsdrf.tex| & sample redirection file \\
    |cdocsfn1.tex| & sample redirection file \\
    |cdocsfn2.tex| & sample redirection file \\
    |childdoc.pdf| & manual
\end{tabular}
\end{center}
%
The distribution consists of the files
|README.txt|, |childdoc.ins| and |childdoc.dtx|.
%
\begin{itemize}
\item
Run (pdf)\LaTeX{} on |childdoc.dtx|
to compile the manual |childdoc.pdf| (this file).
\item
Run \LaTeX{} on |childdoc.ins| to create the definitions file |childdoc.def|
and the sample |cdocsamp.tex| with include files
|cdocsch1.tex|, |cdocsch2.tex|, |cdocspt3.tex|, |cdocspt4.tex|,
|cdocsdrf.tex|, |cdocsfn1.tex|, |cdocsfn2.tex|.
Then copy the file |childdoc.def| to an appropriate directory of your \LaTeX{}
distribution, e.g.\ \textit{texmf-root}|/tex/latex/childdoc|.
\end{itemize}

%%%%%%%%%%%%%%%%%%%%%%%%%%%%%%%%%%%%%%%%%%%%%%%%%%%%%%%%%%%%%%%%%%%%%%%%%%%%%%%%
\subsection{Related CTAN Packages}

There are several other packages which offer a similar functionality:
%
\begin{itemize}
\item
The packages
\href{http://ctan.org/pkg/docmute}{\textsf{docmute}},
\href{http://ctan.org/pkg/includex}{\textsf{includex}} and
\href{http://ctan.org/pkg/standalone}{\textsf{standalone}}
provide commands to include only the document body of
a child file thus allowing both files to be compiled individually.
\item
The packages \href{http://ctan.org/pkg/subdocs}{\textsf{subdocs}}
and \href{http://ctan.org/pkg/subfiles}{\textsf{subfiles}}
provide structures in which the main and child documents can be
encapsulated and allowing them to be compiled individually.
The inclusion mechanism is different from the conventional |\include|.
\item
The package \href{http://ctan.org/pkg/combine}{\textsf{combine}}
is an elaborate solution to combine several documents into one.
\end{itemize}
%
See also the CTAN topic \href{http://ctan.org/topic/subdocs}{\textsf{subdocs}}
for further related packages.
The present package differs from the above solutions in that
a document structure constructed with the conventional |\include| mechanism
just needs two extra commands at the top of every file
such that all constituent files can be compiled individually.

%%%%%%%%%%%%%%%%%%%%%%%%%%%%%%%%%%%%%%%%%%%%%%%%%%%%%%%%%%%%%%%%%%%%%%%%%%%%%%%%
%\subsection{Feature Suggestions}
%
%The following is a list of features which may be useful for future
%versions of this package:
%%
%\begin{itemize}
%\item
%\ldots
%\end{itemize}

%%%%%%%%%%%%%%%%%%%%%%%%%%%%%%%%%%%%%%%%%%%%%%%%%%%%%%%%%%%%%%%%%%%%%%%%%%%%%%%%
\subsection{Revision History}

%%%%%%%%%%%%%%%%%%%%%%%%%%%%%%%%%%%%%%%%
\paragraph{v2.0:} 2018/12/30

\begin{itemize}
\item
immediate forward processing
\item
added |\childdocby| mechanism
\item
manual restructured
\end{itemize}

%%%%%%%%%%%%%%%%%%%%%%%%%%%%%%%%%%%%%%%%
\paragraph{v1.6:} 2018/01/17

\begin{itemize}
\item
application for development of include files
\item
corrections to manual
\end{itemize}

%%%%%%%%%%%%%%%%%%%%%%%%%%%%%%%%%%%%%%%%
\paragraph{v1.5:} 2017/05/21

\begin{itemize}
\item
more complete structuring introduced
\item
|\childdocof| introduced
\item
|\childdoc| renamed to |\childdocmain|
\item
|\childredirect| renamed to |\childdocforward| and |\childdocforwardprefix|
and functionality expanded
\end{itemize}

%%%%%%%%%%%%%%%%%%%%%%%%%%%%%%%%%%%%%%%%
\paragraph{v1.0:} 2017/04/27

\begin{itemize}
\item
manual and install package
\item
first version published on CTAN
\end{itemize}

%%%%%%%%%%%%%%%%%%%%%%%%%%%%%%%%%%%%%%%%
\paragraph{v0.6:} 2017/04/26

\begin{itemize}
\item
redirection mechanism added
\end{itemize}

%%%%%%%%%%%%%%%%%%%%%%%%%%%%%%%%%%%%%%%%
\paragraph{v0.5:} 2017/04/26

\begin{itemize}
\item
functionality in definition file
\end{itemize}


%%%%%%%%%%%%%%%%%%%%%%%%%%%%%%%%%%%%%%%%%%%%%%%%%%%%%%%%%%%%%%%%%%%%%%%%%%%%%%%%
%%%%%%%%%%%%%%%%%%%%%%%%%%%%%%%%%%%%%%%%%%%%%%%%%%%%%%%%%%%%%%%%%%%%%%%%%%%%%%%%
%%%%%%%%%%%%%%%%%%%%%%%%%%%%%%%%%%%%%%%%%%%%%%%%%%%%%%%%%%%%%%%%%%%%%%%%%%%%%%%%
\appendix

\settowidth\MacroIndent{\rmfamily\scriptsize 000\ }

 \DocInput{childdoc.dtx}

\end{document}
%</driver>
% \fi
%
% %%%%%%%%%%%%%%%%%%%%%%%%%%%%%%%%%%%%%%%%%%%%%%%%%%%%%%%%%%%%%%%%%%%%%%%%%%%%%%
% %%%%%%%%%%%%%%%%%%%%%%%%%%%%%%%%%%%%%%%%%%%%%%%%%%%%%%%%%%%%%%%%%%%%%%%%%%%%%%
% \section{Sample}
%\iffalse
%<*samplemain>
%\fi
%
% The following presents a sample document
% with two chapters, two parts, a title page,
% a compile flag as well as three forwarding files to set the flag.
% It consists of eight |.tex| files:
% \begin{center}
% \begin{tabular}{ll}
% |cdocsamp.tex|&main file\\
% |cdocsch1.tex|&include file for chapter 1\\
% |cdocsch2.tex|&include file for chapter 2\\
% |cdocspt3.tex|&include file for part 3\\
% |cdocspt4.tex|&include file for part 4\\
% |cdocsdrf.tex|&forwarding file for main file in draft mode\\
% |cdocsfi1.tex|&forwarding file for final version of chapter 1\\
% |cdocsfi2.tex|&forwarding file for final version of chapter 2\\
% \end{tabular}
% \end{center}
% Each of the eight files can be compiled directly by the \LaTeX{} compiler.
%
% %%%%%%%%%%%%%%%%%%%%%%%%%%%%%%%%%%%%%%
% \paragraph{Main File.}
%
% The main file is called |cdocsamp.tex|.
%
% Load the \textsf{childdoc} definitions and
% declare the filename for the main document:
%    \begin{macrocode}
\input{childdoc.def}
\childdocmain{}
%    \end{macrocode}

% Optional override for |\version| flag:
%    \begin{macrocode}
%%\ifchilddoc\else\providecommand{\version}{draft}\fi
%    \end{macrocode}

% Define the default values for the |\version| flag
% (|final| for the main file and |draft| for childs):
%    \begin{macrocode}
\ifchilddoc
\providecommand{\version}{draft}
\else
\providecommand{\version}{final}
\fi
%    \end{macrocode}

% Load the standard document class:
%    \begin{macrocode}
\documentclass[12pt]{article}
%    \end{macrocode}

% Start the document body:
%    \begin{macrocode}
\begin{document}
%    \end{macrocode}

% Declare a title page.
% Print title, part of document being processed and version flag:
%    \begin{macrocode}
\addtocounter{page}{-1}
\begin{center}
{\LARGE\bfseries{}childdoc example\par}
\vspace{1cm}
\ifchilddoc
\ifchilddocmanual part\else chapter\fi:
`\childdocname' of `\childdocjob'\par
\else
main document: `\childdocjob'\par
\fi
version: \version\par
\end{center}
\newpage
%    \end{macrocode}

% Manually include selected file,
% otherwise process as usual:
%    \begin{macrocode}
\ifchilddocmanual
\section*{part `\childdocname'}
\input{\childdocname}
\else
%    \end{macrocode}

% Include the two chapters:
%    \begin{macrocode}
\include{cdocsch1}
\include{cdocsch2}
%    \end{macrocode}

% Include the two parts unless only chapters should be displayed:
%    \begin{macrocode}
\ifchilddoc\else
\section{part three}
\input{cdocspt3}
\section{part four}
\input{cdocspt4}
\fi
%    \end{macrocode}

% Process as usual until here:
%    \begin{macrocode}
\fi
%    \end{macrocode}

% End of document body:
%    \begin{macrocode}
\end{document}
%    \end{macrocode}
%\iffalse
%</samplemain>
%\fi
%
% %%%%%%%%%%%%%%%%%%%%%%%%%%%%%%%%%%%%%%
% \paragraph{Chapter Include Files.}
%
% The include files are called |cdocsch1.tex| and |cdocsch2.tex|.
%
%\iffalse
%<*samplechap1|samplechap2>
%\fi

% Optional override for |\version| flag:
%    \begin{macrocode}
%%\providecommand{\version}{final}
%    \end{macrocode}

% Include the main document:
%    \begin{macrocode}
\input{childdoc.def}
\childdocof{cdocsamp}
%    \end{macrocode}

%\iffalse
%</samplechap1|samplechap2>
%\fi
%
%\iffalse
%<*samplechap1>
%\fi
% Some text for chapter 1:
%    \begin{macrocode}
\section{one}
some text in chapter one
%    \end{macrocode}

%\iffalse
%</samplechap1>
%\fi
% Some text for chapter 2:
%\iffalse
%<*samplechap2>
%\fi
%    \begin{macrocode}
\section{two}
more text in chapter two
%    \end{macrocode}

%\iffalse
%</samplechap2>
%\fi
%
% %%%%%%%%%%%%%%%%%%%%%%%%%%%%%%%%%%%%%%
% \paragraph{Part Include Files.}
%
% The include files are called |cdocspt3.tex| and |cdocspt4.tex|.
%
%\iffalse
%<*samplepart3|samplepart4>
%\fi

% Optional override for |\version| flag:
%    \begin{macrocode}
%%\providecommand{\version}{final}
%    \end{macrocode}

% Include the main document:
%    \begin{macrocode}
\input{childdoc.def}
\childdocby{cdocsamp}
%    \end{macrocode}

%\iffalse
%</samplepart3|samplepart4>
%\fi
%
%\iffalse
%<*samplepart3>
%\fi
% Some text for part 3:
%    \begin{macrocode}
some text in part three
%    \end{macrocode}

%\iffalse
%</samplepart3>
%\fi
% Some text for part 4:
%\iffalse
%<*samplepart4>
%\fi
%    \begin{macrocode}
more text in part four
%    \end{macrocode}

%\iffalse
%</samplepart4>
%\fi
%
% %%%%%%%%%%%%%%%%%%%%%%%%%%%%%%%%%%%%%%
% \paragraph{Forwarding for a Complete Draft.}
%
% The following forwarding file |cdocsdrf.tex|
% compiles the main document in draft mode:
%\iffalse
%<*sampledraft>
%\fi
%    \begin{macrocode}
\def\version{draft}
\input{childdoc.def}
\childdocforward{cdocsamp}
%    \end{macrocode}

%\iffalse
%</sampledraft>
%\fi
%
% %%%%%%%%%%%%%%%%%%%%%%%%%%%%%%%%%%%%%%
% \paragraph{Forwarding for Final Version of the Chapters.}
%
% The following forwarding files |cdocsfn1.tex| and |cdocsfn2.tex|
% (with identical content)
% compile the final versions of the child documents
% |cdocsch1.tex| and |cdocsch2.tex|, respectively:
%\iffalse
%<*samplefinal>
%\fi
%    \begin{macrocode}
\def\version{final}
\input{childdoc.def}
\childdocforwardprefix[cdocsamp]{cdocsfn}{cdocsch}
%    \end{macrocode}

%\iffalse
%</samplefinal>
%\fi
%
% %%%%%%%%%%%%%%%%%%%%%%%%%%%%%%%%%%%%%%
% \paragraph{Command Line Processing.}
%
% The following three command lines generate the output files
% |cdocscld|, |cdocscl1| and |cdocscl2|
% which should be identical to
% |cdocsdrf|, |cdocsch1| and |cdocsfn2|, respectively:
% \begin{center}
% \begin{tabular}{l}
% |latex -jobname cdocscld \|\\
% |  "\def\version{draft}\input{childdoc.def}\childdocforward{cdocsamp}"|\\
% |latex -jobname cdocscl1 \|\\
% |  "\input{childdoc.def}\childdocforward[cdocsamp]{cdocsch1}"|\\
% |latex -jobname cdocscl2 \|\\
% |  "\def\version{final}\input{childdoc.def}\childdocforward{cdocsch2}"|
% \end{tabular}
% \end{center}
% Note that the trailing backslash on each first line
% merely continues the input to the second line
% (for convenient cut ant paste).
% Furthermore, the command |latex| can be replaced by any
% of its alternative versions such as |pdflatex|.
%
% %%%%%%%%%%%%%%%%%%%%%%%%%%%%%%%%%%%%%%%%%%%%%%%%%%%%%%%%%%%%%%%%%%%%%%%%%%%%%%
% %%%%%%%%%%%%%%%%%%%%%%%%%%%%%%%%%%%%%%%%%%%%%%%%%%%%%%%%%%%%%%%%%%%%%%%%%%%%%%
% \section{Implementation}
%\iffalse
%<*package>
%\fi
%
% This section describes the definitions file |childdoc.def|.

% The definitions cannot be loaded using |\usepackage| or |\RequirePackage|
% which has a mechanism to prevent loading a style file more than once.
% When loading the definitions by means of |\input|
% multiple instances have to be prevented manually:
%\iffalse
%This code needs to be before the `\ProvidesFile' directive
%which is defined at the beginning of this file.
%Therefore it is also placed there and commented out here.
%</package>
%<*discard>
%\fi
%    \begin{macrocode}
\ifdefined\childdocmain\endinput\fi
%    \end{macrocode}
%\iffalse
%</discard>
%<*package>
%\fi
%
% \macro{\ifchilddoc}
% \macro{\ifchilddocmanual}
% The conditional |\ifchilddoc| tells whether a
% child (true) or main (false) document is being compiled.
% The conditional |\ifchilddocmanual| tells whether
% the |\includeonly| mechanism is used (false) or
% the selection of child files must be performed manually (true).
% The definitions initialise to false:
%    \begin{macrocode}
\newif\ifchilddoc
\newif\ifchilddocmanual
%    \end{macrocode}

% \macro{\childdocname}
% \macro{\childdocjob}
% The macro |\childdocname| stores the name of the main document
% to be compiled. The macro |\childdocjob| stores the name of
% the document on which the \LaTeX{} compiler was originally invoked.
% The content of |\jobname| cannot be compared
% to filenames specified in the source due to different catcodes.
% The following code rescans |\jobname|, stores the result
% in |\childdocname| and saves a copy in |\childdocjob|:
%    \begin{macrocode}
\edef\childdocname{\scantokens\expandafter{\jobname\noexpand}}
\let\childdocjob\childdocname
%    \end{macrocode}

% \macro{\childdocdisable}
% The macro |\childdocdisable| prevents the main file
% from being processed more than once.
% At this stage, the main document command |\childdocmain|
% is assumed to be called once again where it should do nothing.
% Any subsequent call to it should prevent
% a secondary processing of the main document
% It overwrites the forwarding commands
% |\childdocof| and |\childdocforward|
% with empty macros to prevent further inclusions of the main document:
%    \begin{macrocode}
\newcommand{\childdocdisable}
{
  \renewcommand{\childdocmain}[1]{\renewcommand{\childdocmain}[1]{\endinput}}
  \renewcommand{\childdocof}[1]{}
  \renewcommand{\childdocby}[2][]{}
  \renewcommand{\childdocforward}[2][]{}
  \renewcommand{\childdocdisable}{}
}
%    \end{macrocode}

% \macro{\childdocmain}
% The macro |\childdocmain| is to be called at the top of the main file
% with nothing or the main filename (without extension) as argument.
% First, it breaks loops.
% If the argument is not empty and does not match |\childdocname|
% (which is set by the first inclusion of |childdoc.def|),
% |\ifchilddoc| is set to true, |\includeonly| is applied to the child file
% and |\jobname| is set to the main file
% (for proper handling of |.aux| files):
%    \begin{macrocode}
\newcommand{\childdocmain}[1]
{
  \childdocdisable\childdocmain{}
  \if?#1?\else
    \begingroup
      \def\childdoctmp{#1}
      \ifx\childdoctmp\childdocname
        \def\childdoctmp{}
      \else
        \def\childdoctmp
        {
          \childdoctrue
          \includeonly{\childdocname}
          \def\childdocjob{#1}
          \def\jobname{#1}
        }
      \fi
      \expandafter
    \endgroup
    \childdoctmp
  \fi
}
%    \end{macrocode}

% \macro{\childdocof}
% The command |\childdocof| redirects
% compilation to the main file |#1|.
%    \begin{macrocode}
\newcommand{\childdocof}[1]
{
  \childdocdisable
  \childdoctrue
  \includeonly{\childdocname}
  \def\jobname{#1}
  \def\childdocjob{#1}
  \input{#1}
}
%    \end{macrocode}

% \macro{\childdocby}
% The command |\childdocby| ....
%    \begin{macrocode}
\newcommand{\childdocby}[2][]
{
  \childdocdisable
  \childdoctrue
  \childdocmanualtrue
  \if?#1?\else
    \def\jobname{#2}
  \fi
  \def\childdocjob{#2}
  \input{#2}
  \endinput
}
%    \end{macrocode}

% \macro{\childdocforward}
% The command |\childdocforward| redirects
% compilation to the main file or
% (if the optional argument is given) a child file.
% Parameters are set as if the main file
% or a child file starting with |\childdocof| was compiled.
% Then compilation is handed over to the main file:
%    \begin{macrocode}
\newcommand{\childdocforward}[2][]
{
  \begingroup
    \if?#1?
      \def\childdoctmp
      {
        \def\childdocname{#2}
        \def\childdocjob{#2}
        \def\jobname{#2}
        \input{#2}
        \endinput
      }
    \else
      \def\childdoctmp
      {
        \childdocdisable
        \def\childdocname{#2}
        \childdoctrue
        \includeonly{#2}
        \def\childdocjob{#1}
        \def\jobname{#1}
        \input{#1}
        \endinput
      }
    \fi
    \expandafter
  \endgroup
  \childdoctmp
}
%    \end{macrocode}

% \macro{\childdocforwardprefix}
% The command |\childdocforwardprefix| redirects
% compilation to the main or a child file by means of a pattern.
% The prefix |#1| in the current filename is replaced by |#2|
% and the suffix of the current filename is kept
% (it is assumed that the filename does not contain the substring `|~~~|'
% which is used as a delimiter).
% Compilation is handed over to the new file by |\childdocforward|:
%    \begin{macrocode}
\newcommand{\childdocforwardprefix}[3][]
{
  \begingroup
    \def\childdocextract #2##1~~~{\def\childdoctmp{\childdocforward[#1]{#3##1}}}
    \expandafter\childdocextract\childdocname~~~
    \expandafter
  \endgroup
  \childdoctmp
}
%    \end{macrocode}

% \macro{\childdoc}
% The deprecated macro |\childdoc| is a legacy version of |\childdocmain|:
%    \begin{macrocode}
\newcommand{\childdoc}{\childdocmain}
%    \end{macrocode}

% \macro{\childdocredirect}
% The deprecated macro |\childdocredirect| is a legacy version
% of |\childdocforward| and |\childdocforwardprefix|:
%    \begin{macrocode}
\newcommand{\childdocredirect}[2][]
{
  \begingroup
    \if?#1?
      \def\childdoctmp{\childdocforward{#2}}
    \else
      \def\childdoctmp{\childdocforwardprefix{#1}{#2}}
    \fi
    \expandafter
  \endgroup
  \childdoctmp
}
%    \end{macrocode}

%\iffalse
%</package>
%\fi
%
\endinput
\childdocforward{cdocsch2}"|
% \end{tabular}
% \end{center}
% Note that the trailing backslash on each first line
% merely continues the input to the second line
% (for convenient cut ant paste).
% Furthermore, the command |latex| can be replaced by any
% of its alternative versions such as |pdflatex|.
%
% %%%%%%%%%%%%%%%%%%%%%%%%%%%%%%%%%%%%%%%%%%%%%%%%%%%%%%%%%%%%%%%%%%%%%%%%%%%%%%
% %%%%%%%%%%%%%%%%%%%%%%%%%%%%%%%%%%%%%%%%%%%%%%%%%%%%%%%%%%%%%%%%%%%%%%%%%%%%%%
% \section{Implementation}
%\iffalse
%<*package>
%\fi
%
% This section describes the definitions file |childdoc.def|.

% The definitions cannot be loaded using |\usepackage| or |\RequirePackage|
% which has a mechanism to prevent loading a style file more than once.
% When loading the definitions by means of |\input|
% multiple instances have to be prevented manually:
%\iffalse
%This code needs to be before the `\ProvidesFile' directive
%which is defined at the beginning of this file.
%Therefore it is also placed there and commented out here.
%</package>
%<*discard>
%\fi
%    \begin{macrocode}
\ifdefined\childdocmain\endinput\fi
%    \end{macrocode}
%\iffalse
%</discard>
%<*package>
%\fi
%
% \macro{\ifchilddoc}
% \macro{\ifchilddocmanual}
% The conditional |\ifchilddoc| tells whether a
% child (true) or main (false) document is being compiled.
% The conditional |\ifchilddocmanual| tells whether
% the |\includeonly| mechanism is used (false) or
% the selection of child files must be performed manually (true).
% The definitions initialise to false:
%    \begin{macrocode}
\newif\ifchilddoc
\newif\ifchilddocmanual
%    \end{macrocode}

% \macro{\childdocname}
% \macro{\childdocjob}
% The macro |\childdocname| stores the name of the main document
% to be compiled. The macro |\childdocjob| stores the name of
% the document on which the \LaTeX{} compiler was originally invoked.
% The content of |\jobname| cannot be compared
% to filenames specified in the source due to different catcodes.
% The following code rescans |\jobname|, stores the result
% in |\childdocname| and saves a copy in |\childdocjob|:
%    \begin{macrocode}
\edef\childdocname{\scantokens\expandafter{\jobname\noexpand}}
\let\childdocjob\childdocname
%    \end{macrocode}

% \macro{\childdocdisable}
% The macro |\childdocdisable| prevents the main file
% from being processed more than once.
% At this stage, the main document command |\childdocmain|
% is assumed to be called once again where it should do nothing.
% Any subsequent call to it should prevent
% a secondary processing of the main document
% It overwrites the forwarding commands
% |\childdocof| and |\childdocforward|
% with empty macros to prevent further inclusions of the main document:
%    \begin{macrocode}
\newcommand{\childdocdisable}
{
  \renewcommand{\childdocmain}[1]{\renewcommand{\childdocmain}[1]{\endinput}}
  \renewcommand{\childdocof}[1]{}
  \renewcommand{\childdocby}[2][]{}
  \renewcommand{\childdocforward}[2][]{}
  \renewcommand{\childdocdisable}{}
}
%    \end{macrocode}

% \macro{\childdocmain}
% The macro |\childdocmain| is to be called at the top of the main file
% with nothing or the main filename (without extension) as argument.
% First, it breaks loops.
% If the argument is not empty and does not match |\childdocname|
% (which is set by the first inclusion of |childdoc.def|),
% |\ifchilddoc| is set to true, |\includeonly| is applied to the child file
% and |\jobname| is set to the main file
% (for proper handling of |.aux| files):
%    \begin{macrocode}
\newcommand{\childdocmain}[1]
{
  \childdocdisable\childdocmain{}
  \if?#1?\else
    \begingroup
      \def\childdoctmp{#1}
      \ifx\childdoctmp\childdocname
        \def\childdoctmp{}
      \else
        \def\childdoctmp
        {
          \childdoctrue
          \includeonly{\childdocname}
          \def\childdocjob{#1}
          \def\jobname{#1}
        }
      \fi
      \expandafter
    \endgroup
    \childdoctmp
  \fi
}
%    \end{macrocode}

% \macro{\childdocof}
% The command |\childdocof| redirects
% compilation to the main file |#1|.
%    \begin{macrocode}
\newcommand{\childdocof}[1]
{
  \childdocdisable
  \childdoctrue
  \includeonly{\childdocname}
  \def\jobname{#1}
  \def\childdocjob{#1}
  \input{#1}
}
%    \end{macrocode}

% \macro{\childdocby}
% The command |\childdocby| ....
%    \begin{macrocode}
\newcommand{\childdocby}[2][]
{
  \childdocdisable
  \childdoctrue
  \childdocmanualtrue
  \if?#1?\else
    \def\jobname{#2}
  \fi
  \def\childdocjob{#2}
  \input{#2}
  \endinput
}
%    \end{macrocode}

% \macro{\childdocforward}
% The command |\childdocforward| redirects
% compilation to the main file or
% (if the optional argument is given) a child file.
% Parameters are set as if the main file
% or a child file starting with |\childdocof| was compiled.
% Then compilation is handed over to the main file:
%    \begin{macrocode}
\newcommand{\childdocforward}[2][]
{
  \begingroup
    \if?#1?
      \def\childdoctmp
      {
        \def\childdocname{#2}
        \def\childdocjob{#2}
        \def\jobname{#2}
        \input{#2}
        \endinput
      }
    \else
      \def\childdoctmp
      {
        \childdocdisable
        \def\childdocname{#2}
        \childdoctrue
        \includeonly{#2}
        \def\childdocjob{#1}
        \def\jobname{#1}
        \input{#1}
        \endinput
      }
    \fi
    \expandafter
  \endgroup
  \childdoctmp
}
%    \end{macrocode}

% \macro{\childdocforwardprefix}
% The command |\childdocforwardprefix| redirects
% compilation to the main or a child file by means of a pattern.
% The prefix |#1| in the current filename is replaced by |#2|
% and the suffix of the current filename is kept
% (it is assumed that the filename does not contain the substring `|~~~|'
% which is used as a delimiter).
% Compilation is handed over to the new file by |\childdocforward|:
%    \begin{macrocode}
\newcommand{\childdocforwardprefix}[3][]
{
  \begingroup
    \def\childdocextract #2##1~~~{\def\childdoctmp{\childdocforward[#1]{#3##1}}}
    \expandafter\childdocextract\childdocname~~~
    \expandafter
  \endgroup
  \childdoctmp
}
%    \end{macrocode}

% \macro{\childdoc}
% The deprecated macro |\childdoc| is a legacy version of |\childdocmain|:
%    \begin{macrocode}
\newcommand{\childdoc}{\childdocmain}
%    \end{macrocode}

% \macro{\childdocredirect}
% The deprecated macro |\childdocredirect| is a legacy version
% of |\childdocforward| and |\childdocforwardprefix|:
%    \begin{macrocode}
\newcommand{\childdocredirect}[2][]
{
  \begingroup
    \if?#1?
      \def\childdoctmp{\childdocforward{#2}}
    \else
      \def\childdoctmp{\childdocforwardprefix{#1}{#2}}
    \fi
    \expandafter
  \endgroup
  \childdoctmp
}
%    \end{macrocode}

%\iffalse
%</package>
%\fi
%
\endinput

\childdocforwardprefix[cdocsamp]{cdocsfn}{cdocsch}
%    \end{macrocode}

%\iffalse
%</samplefinal>
%\fi
%
% %%%%%%%%%%%%%%%%%%%%%%%%%%%%%%%%%%%%%%
% \paragraph{Command Line Processing.}
%
% The following three command lines generate the output files
% |cdocscld|, |cdocscl1| and |cdocscl2|
% which should be identical to
% |cdocsdrf|, |cdocsch1| and |cdocsfn2|, respectively:
% \begin{center}
% \begin{tabular}{l}
% |latex -jobname cdocscld \|\\
% |  "\def\version{draft}% \iffalse
%
% childdoc.dtx Copyright (C) 2017-2018 Niklas Beisert
%
% This work may be distributed and/or modified under the
% conditions of the LaTeX Project Public License, either version 1.3
% of this license or (at your option) any later version.
% The latest version of this license is in
%   http://www.latex-project.org/lppl.txt
% and version 1.3 or later is part of all distributions of LaTeX
% version 2005/12/01 or later.
%
% This work has the LPPL maintenance status `maintained'.
%
% The Current Maintainer of this work is Niklas Beisert.
%
% This work consists of the files childdoc.dtx and childdoc.ins
% and the derived files childdoc.def and cdocsamp.tex with
% cdocsch1.tex, cdocsch2.tex, cdocsdrf.tex, cdocsfn1.tex, cdocsfn2.tex.
%
%<package>\ifdefined\childdocmain\endinput\fi
%<package>\ProvidesFile{childdoc.def}[2018/12/30 v2.0 child document driver]
%<samplemain>\ProvidesFile{cdocsamp.tex}[2018/12/30 v2.0 sample for childdoc]
%<*driver>
%\ProvidesFile{childdoc.drv}[2018/12/30 v2.0 childdoc reference manual file]
\PassOptionsToClass{10pt,a4paper}{article}
\documentclass{ltxdoc}

\usepackage[margin=35mm]{geometry}
\usepackage{hyperref}
\usepackage{hyperxmp}
\usepackage[usenames]{color}

\hypersetup{colorlinks=true}
\hypersetup{pdfstartview=FitH}
\hypersetup{pdfpagemode=UseNone}
\hypersetup{pdfsource={}}
\hypersetup{pdflang={en-UK}}
\hypersetup{pdfcopyright={Copyright 2017-2018 Niklas Beisert.
  This work may be distributed and/or modified under the
  conditions of the LaTeX Project Public License, either version 1.3
  of this license or (at your option) any later version.}}
\hypersetup{pdflicenseurl={http://www.latex-project.org/lppl.txt}}
\hypersetup{pdfcontactaddress={ETH Zurich, ITP, HIT K,
  Wolfgang-Pauli-Strasse 27}}
\hypersetup{pdfcontactpostcode={8093}}
\hypersetup{pdfcontactcity={Zurich}}
\hypersetup{pdfcontactcountry={Switzerland}}
\hypersetup{pdfcontactemail={nbeisert@itp.phys.ethz.ch}}
\hypersetup{pdfcontacturl={http://people.phys.ethz.ch/\xmptilde nbeisert/}}

\newcommand{\secref}[1]{\hyperref[#1]{section \ref*{#1}}}

\parskip1ex
\parindent0pt
\let\olditemize\itemize
\def\itemize{\olditemize\parskip0pt}

\begin{document}

\title{The \textsf{childdoc} Package}
\hypersetup{pdftitle={The childdoc Package}}
\author{Niklas Beisert\\[2ex]
  Institut f\"ur Theoretische Physik\\
  Eidgen\"ossische Technische Hochschule Z\"urich\\
  Wolfgang-Pauli-Strasse 27, 8093 Z\"urich, Switzerland\\[1ex]
  \href{mailto:nbeisert@itp.phys.ethz.ch}
  {\texttt{nbeisert@itp.phys.ethz.ch}}}
\hypersetup{pdfauthor={Niklas Beisert}}
\hypersetup{pdfsubject={Manual for the LaTeX2e Package childdoc}}
\date{30 December 2018, \textsf{v2.0}}
\maketitle

\begin{abstract}\noindent
\textsf{childdoc} is a \LaTeXe{} package
that enables the direct compilation
of document sections included by |\include|
to individual files.
\end{abstract}

\begingroup
\parskip0ex
\tableofcontents
\endgroup

%%%%%%%%%%%%%%%%%%%%%%%%%%%%%%%%%%%%%%%%%%%%%%%%%%%%%%%%%%%%%%%%%%%%%%%%%%%%%%%%
%%%%%%%%%%%%%%%%%%%%%%%%%%%%%%%%%%%%%%%%%%%%%%%%%%%%%%%%%%%%%%%%%%%%%%%%%%%%%%%%
\section{Introduction}

\LaTeX{} provides a mechanism to structure a large document (such as a book)
into a main file and several child files (containing the chapters)
using the |\include| command.
This mechanism is beneficial for documents
which span hundreds of pages in order to
make the source file(s) more manageable.
Moreover, compilation can be restricted to
selected child files by means of the |\includeonly| command.
The latter feature can be used to reduce the compilation time while editing
(this was significantly more useful in the earlier days of \LaTeX{})
or to generate a smaller document which is easier to navigate.
Another application of |\includeonly| is to generate
documents consisting of selected parts of the complete document.

However, there are a few drawbacks of the plain |\include| mechanism:
\begin{itemize}
\item
The child files cannot be compiled on their own,
they can only be compiled via the main file.
A naive editing environment
(such as a text editor with an option
to have the current file processed by \LaTeX)
may require one to switch to the main file before compiling;
attempting to compile the child file produces errors.
\item
The main file must be modified (each time)
to adjust the |\includeonly| command
to the present needs. This easily leaves the main file in a messy state.
\item
The generated document will always carry the filename
of the main document. This is inconvenient if
several child files are to be compiled and
to be kept for distribution.
\end{itemize}

The present package provides a simple interface
to make child files individually compilable by \LaTeX{}.
Compiling a child file then has the same effect as compiling
the main file with an |\includeonly| command
to select the appropriate child.
Moreover the generated document will carry the name of the child
rather than the main file.
This resolves all three above issues.

This feature is meant to make the editing of books,
thesis documents and lecture notes somewhat more convenient.
However, the package can also be used efficiently for
composing a series of documents (such as exercise sheets)
which are typically distributed individually.
It then assists the author in generating the individual documents
(potentially in different versions)
as well as a document containing the collected series.
Another application is in developing style files
or other kinds of included material
where compilation of the style file could redirect
to a sample or test file.

%%%%%%%%%%%%%%%%%%%%%%%%%%%%%%%%%%%%%%%%%%%%%%%%%%%%%%%%%%%%%%%%%%%%%%%%%%%%%%%%
%%%%%%%%%%%%%%%%%%%%%%%%%%%%%%%%%%%%%%%%%%%%%%%%%%%%%%%%%%%%%%%%%%%%%%%%%%%%%%%%
\section{Usage}

First of all, the package \textsf{childdoc} is \emph{not} a standard
\LaTeXe{} |.sty| style file! Therefore it needs to be invoked in
a non-standard way.

%%%%%%%%%%%%%%%%%%%%%%%%%%%%%%%%%%%%%%%%%%%%%%%%%%%%%%%%%%%%%%%%%%%%%%%%%%%%%%%%
\subsection{Included Files}
\label{sec:include}

%%%%%%%%%%%%%%%%%%%%%%%%%%%%%%%%%%%%%%%%
\DescribeMacro{\childdocmain}
To use the package, add the commands
\begin{center}
\begin{tabular}{l}
|% \iffalse
%
% childdoc.dtx Copyright (C) 2017-2018 Niklas Beisert
%
% This work may be distributed and/or modified under the
% conditions of the LaTeX Project Public License, either version 1.3
% of this license or (at your option) any later version.
% The latest version of this license is in
%   http://www.latex-project.org/lppl.txt
% and version 1.3 or later is part of all distributions of LaTeX
% version 2005/12/01 or later.
%
% This work has the LPPL maintenance status `maintained'.
%
% The Current Maintainer of this work is Niklas Beisert.
%
% This work consists of the files childdoc.dtx and childdoc.ins
% and the derived files childdoc.def and cdocsamp.tex with
% cdocsch1.tex, cdocsch2.tex, cdocsdrf.tex, cdocsfn1.tex, cdocsfn2.tex.
%
%<package>\ifdefined\childdocmain\endinput\fi
%<package>\ProvidesFile{childdoc.def}[2018/12/30 v2.0 child document driver]
%<samplemain>\ProvidesFile{cdocsamp.tex}[2018/12/30 v2.0 sample for childdoc]
%<*driver>
%\ProvidesFile{childdoc.drv}[2018/12/30 v2.0 childdoc reference manual file]
\PassOptionsToClass{10pt,a4paper}{article}
\documentclass{ltxdoc}

\usepackage[margin=35mm]{geometry}
\usepackage{hyperref}
\usepackage{hyperxmp}
\usepackage[usenames]{color}

\hypersetup{colorlinks=true}
\hypersetup{pdfstartview=FitH}
\hypersetup{pdfpagemode=UseNone}
\hypersetup{pdfsource={}}
\hypersetup{pdflang={en-UK}}
\hypersetup{pdfcopyright={Copyright 2017-2018 Niklas Beisert.
  This work may be distributed and/or modified under the
  conditions of the LaTeX Project Public License, either version 1.3
  of this license or (at your option) any later version.}}
\hypersetup{pdflicenseurl={http://www.latex-project.org/lppl.txt}}
\hypersetup{pdfcontactaddress={ETH Zurich, ITP, HIT K,
  Wolfgang-Pauli-Strasse 27}}
\hypersetup{pdfcontactpostcode={8093}}
\hypersetup{pdfcontactcity={Zurich}}
\hypersetup{pdfcontactcountry={Switzerland}}
\hypersetup{pdfcontactemail={nbeisert@itp.phys.ethz.ch}}
\hypersetup{pdfcontacturl={http://people.phys.ethz.ch/\xmptilde nbeisert/}}

\newcommand{\secref}[1]{\hyperref[#1]{section \ref*{#1}}}

\parskip1ex
\parindent0pt
\let\olditemize\itemize
\def\itemize{\olditemize\parskip0pt}

\begin{document}

\title{The \textsf{childdoc} Package}
\hypersetup{pdftitle={The childdoc Package}}
\author{Niklas Beisert\\[2ex]
  Institut f\"ur Theoretische Physik\\
  Eidgen\"ossische Technische Hochschule Z\"urich\\
  Wolfgang-Pauli-Strasse 27, 8093 Z\"urich, Switzerland\\[1ex]
  \href{mailto:nbeisert@itp.phys.ethz.ch}
  {\texttt{nbeisert@itp.phys.ethz.ch}}}
\hypersetup{pdfauthor={Niklas Beisert}}
\hypersetup{pdfsubject={Manual for the LaTeX2e Package childdoc}}
\date{30 December 2018, \textsf{v2.0}}
\maketitle

\begin{abstract}\noindent
\textsf{childdoc} is a \LaTeXe{} package
that enables the direct compilation
of document sections included by |\include|
to individual files.
\end{abstract}

\begingroup
\parskip0ex
\tableofcontents
\endgroup

%%%%%%%%%%%%%%%%%%%%%%%%%%%%%%%%%%%%%%%%%%%%%%%%%%%%%%%%%%%%%%%%%%%%%%%%%%%%%%%%
%%%%%%%%%%%%%%%%%%%%%%%%%%%%%%%%%%%%%%%%%%%%%%%%%%%%%%%%%%%%%%%%%%%%%%%%%%%%%%%%
\section{Introduction}

\LaTeX{} provides a mechanism to structure a large document (such as a book)
into a main file and several child files (containing the chapters)
using the |\include| command.
This mechanism is beneficial for documents
which span hundreds of pages in order to
make the source file(s) more manageable.
Moreover, compilation can be restricted to
selected child files by means of the |\includeonly| command.
The latter feature can be used to reduce the compilation time while editing
(this was significantly more useful in the earlier days of \LaTeX{})
or to generate a smaller document which is easier to navigate.
Another application of |\includeonly| is to generate
documents consisting of selected parts of the complete document.

However, there are a few drawbacks of the plain |\include| mechanism:
\begin{itemize}
\item
The child files cannot be compiled on their own,
they can only be compiled via the main file.
A naive editing environment
(such as a text editor with an option
to have the current file processed by \LaTeX)
may require one to switch to the main file before compiling;
attempting to compile the child file produces errors.
\item
The main file must be modified (each time)
to adjust the |\includeonly| command
to the present needs. This easily leaves the main file in a messy state.
\item
The generated document will always carry the filename
of the main document. This is inconvenient if
several child files are to be compiled and
to be kept for distribution.
\end{itemize}

The present package provides a simple interface
to make child files individually compilable by \LaTeX{}.
Compiling a child file then has the same effect as compiling
the main file with an |\includeonly| command
to select the appropriate child.
Moreover the generated document will carry the name of the child
rather than the main file.
This resolves all three above issues.

This feature is meant to make the editing of books,
thesis documents and lecture notes somewhat more convenient.
However, the package can also be used efficiently for
composing a series of documents (such as exercise sheets)
which are typically distributed individually.
It then assists the author in generating the individual documents
(potentially in different versions)
as well as a document containing the collected series.
Another application is in developing style files
or other kinds of included material
where compilation of the style file could redirect
to a sample or test file.

%%%%%%%%%%%%%%%%%%%%%%%%%%%%%%%%%%%%%%%%%%%%%%%%%%%%%%%%%%%%%%%%%%%%%%%%%%%%%%%%
%%%%%%%%%%%%%%%%%%%%%%%%%%%%%%%%%%%%%%%%%%%%%%%%%%%%%%%%%%%%%%%%%%%%%%%%%%%%%%%%
\section{Usage}

First of all, the package \textsf{childdoc} is \emph{not} a standard
\LaTeXe{} |.sty| style file! Therefore it needs to be invoked in
a non-standard way.

%%%%%%%%%%%%%%%%%%%%%%%%%%%%%%%%%%%%%%%%%%%%%%%%%%%%%%%%%%%%%%%%%%%%%%%%%%%%%%%%
\subsection{Included Files}
\label{sec:include}

%%%%%%%%%%%%%%%%%%%%%%%%%%%%%%%%%%%%%%%%
\DescribeMacro{\childdocmain}
To use the package, add the commands
\begin{center}
\begin{tabular}{l}
|\input{childdoc.def}|\\
|\childdocmain{}|\\
\end{tabular}
\end{center}
at the very top of the main \LaTeX{} file,
in particular \emph{before} the |\documentclass| statement!
The argument of |\childdocmain| should be left empty
(but it must be present).

%%%%%%%%%%%%%%%%%%%%%%%%%%%%%%%%%%%%%%%%
\DescribeMacro{\childdocof}
Furthermore, add the commands
\begin{center}
\begin{tabular}{l}
|\input{childdoc.def}|\\
|\childdocof{|\textit{main}|}|\\
\end{tabular}
\end{center}
at the top of every child file \textit{child}
which is included by |\include{|\textit{child}|}|
from within the main file
(or at least for those files to be compiled individually).
The argument \textit{main} must be the filename of the main file.

There are a couple of
considerations in setting up the main and child documents:

%%%%%%%%%%%%%%%%%%%%%%%%%%%%%%%%%%%%%%%%
\paragraph{Restrictions.}

Please note the following restrictions:
\begin{itemize}
\item
|\childdocmain| must be called with one argument \textit{main}
to ensure compatibility with earlier version of the package.
It must either be empty (|\childdocmain{}|)
or precisely match the filename of the main file in which it is specified.
See \secref{sec:detection} for further information.
\item
The filename \textit{main} must be specified without the |.tex| extension.
\item
The filename \textit{main} is case sensitive
(even in case-insensitive file systems)
due to internal string comparison.
\item
The argument \textit{main} should be fully expanded, it cannot be a macro.
\item
Subdirectories and special characters should be avoided in filenames.
\item
The command |\childdocmain{|\textit{main}|}| must be followed by a whitespace.
It should not be followed immediately by another command
or by a comment mark `|%|'.
This is because the \TeX{} parser reads the token immediately following
the argument of |\childdocmain| and puts it
at the beginning of every child section;
however, a white\-space is ignored.
\end{itemize}

%%%%%%%%%%%%%%%%%%%%%%%%%%%%%%%%%%%%%%%%
\paragraph{Content of Main File.}

It is advisable to place all content in the child files included by |\include|.
Any output contained in the main file will appear in all child documents
unless suppressed manually;
it cannot be suppressed automatically by the |\includeonly| directive
and thus should normally be avoided.
A method to include some content in the main file
by means of conditional processing is described in \secref{sec:conditional}.

%%%%%%%%%%%%%%%%%%%%%%%%%%%%%%%%%%%%%%%%
\paragraph{Page Numbering.}

When only a part of the document is compiled,
the appropriate numbering of pages
(as well as other status parameters)
is determined from the |.aux| files.
The latter contain information from previous passes.
However this information needs to propagate through
all intermediate child documents.
Therefore the page numbering in child documents may well
be inconsistent until the complete document is compiled at least once.

A useful (if unconventional) way to always ensure a consistent
page numbering is to restart the numbering in each child document
and denote the pages by `\textit{child}|.|\textit{page}'
where \textit{child} represents the chapter/section number of the child file.
This can be achieved by the command
|\numberwithin{page}{|\textit{child}|}|
of the \textsf{amsmath} package
where \textit{child} can be |chapter| or |section|
depending on the chosen structuring.
Alternatively, one can modify the macro |\thepage| appropriately
and reset the counter |page| at the start of each child file.

%%%%%%%%%%%%%%%%%%%%%%%%%%%%%%%%%%%%%%%%%%%%%%%%%%%%%%%%%%%%%%%%%%%%%%%%%%%%%%%%
\subsection{Conditional Processing}
\label{sec:conditional}

The package provides a mechanism to compile different versions
of a document. To customise the versions further some conditional processing
can come in handy to distinguish which version is being compiled.
The package provides two macros to describe the compilation context:

%%%%%%%%%%%%%%%%%%%%%%%%%%%%%%%%%%%%%%%%
\DescribeMacro{\ifchilddoc}
The conditional |\ifchilddoc| distinguishes between the compilation of
child documents and the main document:
%
\begin{center}
|\ifchilddoc |\textit{child-code}| |[|\||else |\textit{main-code}]| \||fi|
\end{center}

%%%%%%%%%%%%%%%%%%%%%%%%%%%%%%%%%%%%%%%%
\DescribeMacro{\childdocname}
\DescribeMacro{\childdocjob}
The macro |\childdocname| contains the filename (without extension)
of the main or child file being processed.
Note that |\childdocjob| will always contain the name of the main file.

%%%%%%%%%%%%%%%%%%%%%%%%%%%%%%%%%%%%%%%%
\paragraph{Title Page.}

Conditional processing can be used to include a title or banner page
in the main document when proper precautions are taken.
Importantly, the code in the main file should ensure that the page counter
(as well as other status parameters which are stored in the |.aux| files)
takes the same value after the conditional processing.
Otherwise the page numbers may take divergent values
depending on which part is compiled.

For example, a title page could be declared by:
%
\begin{center}
\begin{tabular}{l}
|\ifchilddoc\||else|\\
|\addtocounter{page}{-1}|\\
\textit{code for title page}\\
|\newpage|\\
|\||fi|
\end{tabular}
\end{center}
%
A banner page for the child documents can be generated by:
%
\begin{center}
\begin{tabular}{l}
|\ifchilddoc|\\
|\addtocounter{page}{-1}|\\
\textit{code for banner page}\\
|\newpage|\\
|\||fi|
\end{tabular}
\end{center}
%
Here one could write a message such as:
\begin{center}
|This is the part \childdocname{} of \childdocjob{}.|
\end{center}

%%%%%%%%%%%%%%%%%%%%%%%%%%%%%%%%%%%%%%%%%%%%%%%%%%%%%%%%%%%%%%%%%%%%%%%%%%%%%%%%
\subsection{Flags}
\label{sec:flags}

The package makes it easy to generate different versions
of the main or child documents.
To this end compilation flags can be defined
and assigned different default values.
They will be particularly useful in conjunction
with the forwarding mechanism described in \secref{sec:forward}.

For example, it may be useful to have a flag |\version|
which can be set to |draft| or |final|.
The document source will contain some conditional code
depending on the value of |\version|.
Suppose further, the flag should default to |final| for the main file
and to |draft| for child files
which is a natural assignment for editing the document.
This is achieved by placing the following code
in the preamble of the main document
(below the |\childdocmain| directive):
%
\begin{center}
\begin{tabular}{l}
|\ifchilddoc|\\
|\providecommand{\version}{draft}|\\
|\||else|\\
|\providecommand{\version}{final}|\\
|\||fi|
\end{tabular}
\end{center}
%
The definition by |\providecommand| makes sure
that previous definitions are not overwritten.
Further statements |\providecommand{\version}{...}|
can thus be added before the above code to override it.

For the main file, one might add a line
(between |\childdocmain| and the above block)
%
\begin{center}
|%\ifchilddoc\||else\providecommand{\version}{draft}\||fi|
\end{center}
%
which can be uncommented to produce a draft version.
Likewise one can add a line to the very top of a child file
(above the |\childdocof{|\textit{main}|}| directive)
%
\begin{center}
|%\providecommand{\version}{final}|
\end{center}
%
which can be uncommented to produce the final version of this child document.

%%%%%%%%%%%%%%%%%%%%%%%%%%%%%%%%%%%%%%%%%%%%%%%%%%%%%%%%%%%%%%%%%%%%%%%%%%%%%%%%
\subsection{Forwarding}
\label{sec:forward}

Different versions of the main or child documents
using compilation flags as described in \secref{sec:flags}
can be (permanently) stored in different files
for convenient compilation, viewing and distribution.
To this end, the package defines a command
to pass on compilation to a different file:

%%%%%%%%%%%%%%%%%%%%%%%%%%%%%%%%%%%%%%%%
\DescribeMacro{\childdocforward}
The command |\childdocforward| redirects processing to
another source file:
%
\begin{center}
\begin{tabular}{l}
|\input{childdoc.def}|\\
|\childdocforward[|\textit{main}|]{|\textit{dest}|}|\\
\end{tabular}
\end{center}
%
The argument \textit{dest} is the destination file
(without extension).
It should be the main file or one of the child files.
Note that further \textsf{childdoc} directives
such as |\childdocof| and |\childdocforward|
in the indicated file will be processed in this form.
The optional argument \textit{main}
passes on directly to the main file \textit{main}
while pretending to compile the child \textit{dest}.
This form behaves as if \textit{dest}
issues |\childdocof{|\textit{main}|}| right away,
and no further \textsf{childdoc} directives will be processed.

%%%%%%%%%%%%%%%%%%%%%%%%%%%%%%%%%%%%%%%%
\DescribeMacro{\...prefix}
In the alternative form |\childdocforwardprefix|,
%
\begin{center}
\begin{tabular}{l}
|\input{childdoc.def}|\\
|\childdocforwardprefix[|\textit{main}|]{|\textit{prefix}|}{|\textit{dest}|}|
\end{tabular}
\end{center}
%
the destination file is determined by a pattern
depending on the current file:
To make this work, the current file must be called
`{\textit{prefix}\hspace{0.2em}\textit{suffix}}'
with \textit{prefix} matching precisely the argument.
Processing is then passed on to the file
`{\textit{dest}\hspace{0.2em}\textit{suffix}}'.
Surely, the same effect is achieved by
directly specifying the
argument `{\textit{dest}\hspace{0.2em}\textit{suffix}}'
in the first form.
However, that requires to set up a different file
for each child. With the alternative form of the command
all these files can have exactly the same content
which simplifies setting them up and maintaining them.

For example, the following file |draft.tex|
with a compilation flag |\version| as described in \secref{sec:flags}
compiles the main document as a draft:
%
\begin{center}
\begin{tabular}{l}
|\def\version{draft}|\\
|\input{childdoc.def}|\\
|\childdocforward{|\textit{main}|}|
\end{tabular}
\end{center}
%
Likewise, the following files |final|\textit{nn}|.tex|
compile the final version of the child document
|child|\textit{nn}|.tex|:
%
\begin{center}
\begin{tabular}{l}
|\def\version{final}|\\
|\input{childdoc.def}|\\
|\childdocforwardprefix{final}{child}|
\end{tabular}
\end{center}
%

Note that when several versions of a main file and/or of each child file
are to be generated, it may be convenient to set up a |Makefile| or
shell script to automatise the process.

%%%%%%%%%%%%%%%%%%%%%%%%%%%%%%%%%%%%%%%%%%%%%%%%%%%%%%%%%%%%%%%%%%%%%%%%%%%%%%%%
\subsection{Command Line Processing}
\label{sec:commandline}

The effect of redirection files can also be achieved by invoking
the \LaTeX{} compiler with a more elaborate command line.
Most conveniently this should be done as part
of a shell script or a |Makefile|.

When using \textsf{childdoc} in the main file, the following
command lines effectively perform a redirection
(note that depending on the shell being used,
backslashes may have to be doubled: `|\|' $\to$ `|\\|'):
%
\begin{center}
|... -jobname "|\textit{target}|" |\\|"|[\textit{flags}]%
|\input{childdoc.def}\childdocforward[|\textit{main}|]{|\textit{dest}|}"|
\end{center}
%
Here \textit{target} is the name of the output file,
\textit{main} is the name of the main file
and \textit{dest} is the name of the main or child file to be processed
(all filenames without extensions).
The optional argument \textit{main} can be omitted
if \textit{main} matches \textit{dest}.
Optionally, compilation \textit{flags} can be defined via |\def| commands.
This command line makes the \TeX{} engine believe
it is compiling the file \textit{target}
whose content is specified as the latter parameter.
The provided code then forwards the processing to
\textit{main} or \textit{dest} as described in \secref{sec:forward}.

%%%%%%%%%%%%%%%%%%%%%%%%%%%%%%%%%%%%%%%%%%%%%%%%%%%%%%%%%%%%%%%%%%%%%%%%%%%%%%%%
\subsection{Include by Input}
\label{sec:input}

Including child documents by |\include| has some restrictions by design.
Most notably, the content of a child document always occupies
its own set of pages; pages cannot be shared between child documents.
Usually, this behaviour makes perfect sense
because each child document contain an essential part of the document.
However, in some situations it may be desirable to compose
a document from a collection of parts
without having mandatory page breaks between then.
For this case, the package
provides a mechanism to include parts
by |\input| which can also be processed individually.
However, by construction this mechanism
requires manual handling of the content to be output.

%%%%%%%%%%%%%%%%%%%%%%%%%%%%%%%%%%%%%%%%
\DescribeMacro{\ifchilddocmanual}
The main file should be prepared as usual, see \secref{sec:include}.
However, the document body must make a distinction
between processing of an individual part and of the main document, e.g.:
%
\begin{center}
\begin{tabular}{l}
|\ifchilddocmanual|\\
|\input{\childdocname}|\\
|\||else|\\
\textit{document body with }|\input{|\textit{part}|}|\\
|\||fi|
\end{tabular}
\end{center}
%
The conditional |\ifchilddocmanual| is true whenever
a part to be included by |\input| is being compiled,
and the name of the part is stored in |\childdocname|.

%%%%%%%%%%%%%%%%%%%%%%%%%%%%%%%%%%%%%%%%
\DescribeMacro{\childdocby}
Each part to be included by |\input| should start with:
%
\begin{center}
\begin{tabular}{l}
|\input{childdoc.def}|\\
|\childdocby{|\textit{main}|}|\\
\end{tabular}
\end{center}
%
The directive |\childdocby| is similar to |\childdocof|
described in \secref{sec:include},
but the subsequent selection of content must be done manually.
To that end, both |\ifchilddoc| and |\ifchilddocmanual|
will be true upon processing of a part,
and the name of the part is stored in |\childdocname|.
Note that |\jobname| will be set to the filename of the current part
so that each part receives an individual |.aux| file
that does not interfere with the |.aux| file(s) of the main document.
This behaviour can be altered by the alternative form
|\childdocby[*]{|\textit{main}|}| (with a non-empty optional argument)
which uses the |.aux| file of the main document
by setting |\jobname| to \textit{main}.

%%%%%%%%%%%%%%%%%%%%%%%%%%%%%%%%%%%%%%%%%%%%%%%%%%%%%%%%%%%%%%%%%%%%%%%%%%%%%%%%
\subsection{Driver Development}
\label{sec:driver}

The \textsf{childdoc} mechanism can also be use for the development
of definition files such as \LaTeX{} styles or classes.
This case differs from the above setup with multiple parts
included by |\include| in that no |\includeonly| should be invoked.
This can be achieved by starting the include file
(before |\ProvidesPackage|) with:
%
\begin{center}
\begin{tabular}{l}
|\input{childdoc.def}|\\
|\childdocforward{|\textit{main}|}|\\
\end{tabular}
\end{center}
%
or alternatively with:
%
\begin{center}
\begin{tabular}{l}
|\input{childdoc.def}|\\
|\childdocby{|\textit{main}|}|\\
\end{tabular}
\end{center}
%
Both forms have slightly different effects as described above.
The main file is prepared as usual, see \secref{sec:include}.

%%%%%%%%%%%%%%%%%%%%%%%%%%%%%%%%%%%%%%%%%%%%%%%%%%%%%%%%%%%%%%%%%%%%%%%%%%%%%%%%
\subsection{Legacy Detection}
\label{sec:detection}

The directive |\childdocmain| in the main file can detect
whether the complete document or merely a child is to be compiled
even without using the directive |\childdocof|.
This method is deprecated because it is less robust
and there is no compelling reason to use it;
it is merely provided for backward compatibility
and it may be removed in future versions.

If the detection mechanism is to be used,
it is mandatory to correctly specify
the filename of the main file as the argument of |\childdocmain|:
%
\begin{center}
\begin{tabular}{l}
|\input{childdoc.def}|\\
|\childdocmain{|\textit{main}|}|\\
\end{tabular}
\end{center}
%
If |\jobname| does not match the argument \textit{main} of |\childdocmain|,
it is assumed that |\jobname| points to the child file to be compiled.
When using |\childdocmain| with the main file specified as argument,
it suffices to start a child file
with just |\input{|\textit{main}|}|
without loading of the package and using |\childdocof|.
If instead all processing is done
with the appropriate \textsf{childdoc} directives,
the argument of \textit{main} of |\childdocmain| can be empty.

An alternative version of the command line processing described
in \secref{sec:commandline} using the detection mechanism reads:
%
\begin{center}
|... -jobname "|\textit{target}|" "|[\textit{flags}]%
[|\def\jobname{|\textit{dest}|}|]|\input{|\textit{main}|}"|
\end{center}

%%%%%%%%%%%%%%%%%%%%%%%%%%%%%%%%%%%%%%%%%%%%%%%%%%%%%%%%%%%%%%%%%%%%%%%%%%%%%%%%
\subsection{Manual Code}
\label{sec:manual}

In case one cannot be certain whether the definitions file |childdoc.def|
is installed on the target \TeX{} distribution
and one prefers not to ship it,
it is conceivable to paste a few relevant commands into the sources.

To that end, drop all statements |\input{childdoc.def}|
and perform the replacements as outlined below.
Instead of |\childdocmain{|\textit{main}|}| add the following code
to the top of the main file:
%
\begin{center}
\begin{tabular}{l}
|\||ifdefined\childdocname\endinput\||fi\newif\ifchilddoc|\\
|\edef\childdocname{\scantokens\expandafter{\jobname\noexpand}}|\\
|\def\childdocmain{|\textit{main}|}\||ifx\childdocmain\childdocname\||else|\\
|\childdoctrue\includeonly{\childdocname}\let\jobname\childdocmain\||fi|\\
\end{tabular}
\end{center}
%
Instead of |\childdocof{|\textit{main}|}| just include the main file
at the top of each child file:
%
\begin{center}
|\input{|\textit{main}|}|
\end{center}
%
A simple redirection |\childdocforward{|\textit{dest}|}| is achieved by:
%
\begin{center}
|\def\jobname{|\textit{dest}|}\input{\jobname}|
\end{center}
%
The redirection with prefix
|\childdocforwardprefix[|\textit{prefix}|]{|\textit{dest}|}|
is accomplished by:
%
\begin{center}
\begin{tabular}{l}
|{\edef\jobname{\scantokens\expandafter{\jobname\noexpand}}|\\
|\def\redirectjob |\textit{prefix}|#1~~~{\gdef\jobname{|\textit{dest}|#1}}|\\
|\expandafter\redirectjob\jobname~~~}\input{\jobname}|
\end{tabular}
\end{center}

In an alternative approach,
child documents can be compiled by a specific command line
without additional code or specific definitions:
%
\begin{center}
|... -jobname "|\textit{target}|" "|[\textit{flags}]%
|\includeonly{|\textit{dest}|}\input{|\textit{main}|}"|
\end{center}
%

%%%%%%%%%%%%%%%%%%%%%%%%%%%%%%%%%%%%%%%%%%%%%%%%%%%%%%%%%%%%%%%%%%%%%%%%%%%%%%%%
%%%%%%%%%%%%%%%%%%%%%%%%%%%%%%%%%%%%%%%%%%%%%%%%%%%%%%%%%%%%%%%%%%%%%%%%%%%%%%%%
\section{Information}

%%%%%%%%%%%%%%%%%%%%%%%%%%%%%%%%%%%%%%%%%%%%%%%%%%%%%%%%%%%%%%%%%%%%%%%%%%%%%%%%
\subsection{Copyright}

Copyright \copyright{} 2017--2018 Niklas Beisert

This work may be distributed and/or modified under the
conditions of the \LaTeX{} Project Public License, either version 1.3
of this license or (at your option) any later version.
The latest version of this license is in
  \url{http://www.latex-project.org/lppl.txt}
and version 1.3 or later is part of all distributions of \LaTeX{}
version 2005/12/01 or later.

This work has the LPPL maintenance status `maintained'.

The Current Maintainer of this work is Niklas Beisert.

This work consists of the files |README.txt|, |childdoc.ins| and |childdoc.dtx|
as well as the derived files |childdoc.def|, |cdocsamp.tex|
with |cdocsch1.tex|, |cdocsch2.tex|, |cdocspt3.tex|, |cdocspt4.tex|,
|cdocsdrf.tex|, |cdocsfn1.tex|, |cdocsfn2.tex|
as well as |childdoc.pdf|.

%%%%%%%%%%%%%%%%%%%%%%%%%%%%%%%%%%%%%%%%%%%%%%%%%%%%%%%%%%%%%%%%%%%%%%%%%%%%%%%%
\subsection{Files and Installation}

The package consists of the files:
%
\begin{center}
\begin{tabular}{ll}
    |README.txt|   & readme file \\
    |childdoc.ins| & installation file \\
    |childdoc.dtx| & source file \\
    |childdoc.def| & definition file \\
    |cdocsamp.tex| & sample main file \\
    |cdocsch1.tex| & sample include file \\
    |cdocsch2.tex| & sample include file \\
    |cdocspt3.tex| & sample part file \\
    |cdocspt4.tex| & sample part file \\
    |cdocsdrf.tex| & sample redirection file \\
    |cdocsfn1.tex| & sample redirection file \\
    |cdocsfn2.tex| & sample redirection file \\
    |childdoc.pdf| & manual
\end{tabular}
\end{center}
%
The distribution consists of the files
|README.txt|, |childdoc.ins| and |childdoc.dtx|.
%
\begin{itemize}
\item
Run (pdf)\LaTeX{} on |childdoc.dtx|
to compile the manual |childdoc.pdf| (this file).
\item
Run \LaTeX{} on |childdoc.ins| to create the definitions file |childdoc.def|
and the sample |cdocsamp.tex| with include files
|cdocsch1.tex|, |cdocsch2.tex|, |cdocspt3.tex|, |cdocspt4.tex|,
|cdocsdrf.tex|, |cdocsfn1.tex|, |cdocsfn2.tex|.
Then copy the file |childdoc.def| to an appropriate directory of your \LaTeX{}
distribution, e.g.\ \textit{texmf-root}|/tex/latex/childdoc|.
\end{itemize}

%%%%%%%%%%%%%%%%%%%%%%%%%%%%%%%%%%%%%%%%%%%%%%%%%%%%%%%%%%%%%%%%%%%%%%%%%%%%%%%%
\subsection{Related CTAN Packages}

There are several other packages which offer a similar functionality:
%
\begin{itemize}
\item
The packages
\href{http://ctan.org/pkg/docmute}{\textsf{docmute}},
\href{http://ctan.org/pkg/includex}{\textsf{includex}} and
\href{http://ctan.org/pkg/standalone}{\textsf{standalone}}
provide commands to include only the document body of
a child file thus allowing both files to be compiled individually.
\item
The packages \href{http://ctan.org/pkg/subdocs}{\textsf{subdocs}}
and \href{http://ctan.org/pkg/subfiles}{\textsf{subfiles}}
provide structures in which the main and child documents can be
encapsulated and allowing them to be compiled individually.
The inclusion mechanism is different from the conventional |\include|.
\item
The package \href{http://ctan.org/pkg/combine}{\textsf{combine}}
is an elaborate solution to combine several documents into one.
\end{itemize}
%
See also the CTAN topic \href{http://ctan.org/topic/subdocs}{\textsf{subdocs}}
for further related packages.
The present package differs from the above solutions in that
a document structure constructed with the conventional |\include| mechanism
just needs two extra commands at the top of every file
such that all constituent files can be compiled individually.

%%%%%%%%%%%%%%%%%%%%%%%%%%%%%%%%%%%%%%%%%%%%%%%%%%%%%%%%%%%%%%%%%%%%%%%%%%%%%%%%
%\subsection{Feature Suggestions}
%
%The following is a list of features which may be useful for future
%versions of this package:
%%
%\begin{itemize}
%\item
%\ldots
%\end{itemize}

%%%%%%%%%%%%%%%%%%%%%%%%%%%%%%%%%%%%%%%%%%%%%%%%%%%%%%%%%%%%%%%%%%%%%%%%%%%%%%%%
\subsection{Revision History}

%%%%%%%%%%%%%%%%%%%%%%%%%%%%%%%%%%%%%%%%
\paragraph{v2.0:} 2018/12/30

\begin{itemize}
\item
immediate forward processing
\item
added |\childdocby| mechanism
\item
manual restructured
\end{itemize}

%%%%%%%%%%%%%%%%%%%%%%%%%%%%%%%%%%%%%%%%
\paragraph{v1.6:} 2018/01/17

\begin{itemize}
\item
application for development of include files
\item
corrections to manual
\end{itemize}

%%%%%%%%%%%%%%%%%%%%%%%%%%%%%%%%%%%%%%%%
\paragraph{v1.5:} 2017/05/21

\begin{itemize}
\item
more complete structuring introduced
\item
|\childdocof| introduced
\item
|\childdoc| renamed to |\childdocmain|
\item
|\childredirect| renamed to |\childdocforward| and |\childdocforwardprefix|
and functionality expanded
\end{itemize}

%%%%%%%%%%%%%%%%%%%%%%%%%%%%%%%%%%%%%%%%
\paragraph{v1.0:} 2017/04/27

\begin{itemize}
\item
manual and install package
\item
first version published on CTAN
\end{itemize}

%%%%%%%%%%%%%%%%%%%%%%%%%%%%%%%%%%%%%%%%
\paragraph{v0.6:} 2017/04/26

\begin{itemize}
\item
redirection mechanism added
\end{itemize}

%%%%%%%%%%%%%%%%%%%%%%%%%%%%%%%%%%%%%%%%
\paragraph{v0.5:} 2017/04/26

\begin{itemize}
\item
functionality in definition file
\end{itemize}


%%%%%%%%%%%%%%%%%%%%%%%%%%%%%%%%%%%%%%%%%%%%%%%%%%%%%%%%%%%%%%%%%%%%%%%%%%%%%%%%
%%%%%%%%%%%%%%%%%%%%%%%%%%%%%%%%%%%%%%%%%%%%%%%%%%%%%%%%%%%%%%%%%%%%%%%%%%%%%%%%
%%%%%%%%%%%%%%%%%%%%%%%%%%%%%%%%%%%%%%%%%%%%%%%%%%%%%%%%%%%%%%%%%%%%%%%%%%%%%%%%
\appendix

\settowidth\MacroIndent{\rmfamily\scriptsize 000\ }

 \DocInput{childdoc.dtx}

\end{document}
%</driver>
% \fi
%
% %%%%%%%%%%%%%%%%%%%%%%%%%%%%%%%%%%%%%%%%%%%%%%%%%%%%%%%%%%%%%%%%%%%%%%%%%%%%%%
% %%%%%%%%%%%%%%%%%%%%%%%%%%%%%%%%%%%%%%%%%%%%%%%%%%%%%%%%%%%%%%%%%%%%%%%%%%%%%%
% \section{Sample}
%\iffalse
%<*samplemain>
%\fi
%
% The following presents a sample document
% with two chapters, two parts, a title page,
% a compile flag as well as three forwarding files to set the flag.
% It consists of eight |.tex| files:
% \begin{center}
% \begin{tabular}{ll}
% |cdocsamp.tex|&main file\\
% |cdocsch1.tex|&include file for chapter 1\\
% |cdocsch2.tex|&include file for chapter 2\\
% |cdocspt3.tex|&include file for part 3\\
% |cdocspt4.tex|&include file for part 4\\
% |cdocsdrf.tex|&forwarding file for main file in draft mode\\
% |cdocsfi1.tex|&forwarding file for final version of chapter 1\\
% |cdocsfi2.tex|&forwarding file for final version of chapter 2\\
% \end{tabular}
% \end{center}
% Each of the eight files can be compiled directly by the \LaTeX{} compiler.
%
% %%%%%%%%%%%%%%%%%%%%%%%%%%%%%%%%%%%%%%
% \paragraph{Main File.}
%
% The main file is called |cdocsamp.tex|.
%
% Load the \textsf{childdoc} definitions and
% declare the filename for the main document:
%    \begin{macrocode}
\input{childdoc.def}
\childdocmain{}
%    \end{macrocode}

% Optional override for |\version| flag:
%    \begin{macrocode}
%%\ifchilddoc\else\providecommand{\version}{draft}\fi
%    \end{macrocode}

% Define the default values for the |\version| flag
% (|final| for the main file and |draft| for childs):
%    \begin{macrocode}
\ifchilddoc
\providecommand{\version}{draft}
\else
\providecommand{\version}{final}
\fi
%    \end{macrocode}

% Load the standard document class:
%    \begin{macrocode}
\documentclass[12pt]{article}
%    \end{macrocode}

% Start the document body:
%    \begin{macrocode}
\begin{document}
%    \end{macrocode}

% Declare a title page.
% Print title, part of document being processed and version flag:
%    \begin{macrocode}
\addtocounter{page}{-1}
\begin{center}
{\LARGE\bfseries{}childdoc example\par}
\vspace{1cm}
\ifchilddoc
\ifchilddocmanual part\else chapter\fi:
`\childdocname' of `\childdocjob'\par
\else
main document: `\childdocjob'\par
\fi
version: \version\par
\end{center}
\newpage
%    \end{macrocode}

% Manually include selected file,
% otherwise process as usual:
%    \begin{macrocode}
\ifchilddocmanual
\section*{part `\childdocname'}
\input{\childdocname}
\else
%    \end{macrocode}

% Include the two chapters:
%    \begin{macrocode}
\include{cdocsch1}
\include{cdocsch2}
%    \end{macrocode}

% Include the two parts unless only chapters should be displayed:
%    \begin{macrocode}
\ifchilddoc\else
\section{part three}
\input{cdocspt3}
\section{part four}
\input{cdocspt4}
\fi
%    \end{macrocode}

% Process as usual until here:
%    \begin{macrocode}
\fi
%    \end{macrocode}

% End of document body:
%    \begin{macrocode}
\end{document}
%    \end{macrocode}
%\iffalse
%</samplemain>
%\fi
%
% %%%%%%%%%%%%%%%%%%%%%%%%%%%%%%%%%%%%%%
% \paragraph{Chapter Include Files.}
%
% The include files are called |cdocsch1.tex| and |cdocsch2.tex|.
%
%\iffalse
%<*samplechap1|samplechap2>
%\fi

% Optional override for |\version| flag:
%    \begin{macrocode}
%%\providecommand{\version}{final}
%    \end{macrocode}

% Include the main document:
%    \begin{macrocode}
\input{childdoc.def}
\childdocof{cdocsamp}
%    \end{macrocode}

%\iffalse
%</samplechap1|samplechap2>
%\fi
%
%\iffalse
%<*samplechap1>
%\fi
% Some text for chapter 1:
%    \begin{macrocode}
\section{one}
some text in chapter one
%    \end{macrocode}

%\iffalse
%</samplechap1>
%\fi
% Some text for chapter 2:
%\iffalse
%<*samplechap2>
%\fi
%    \begin{macrocode}
\section{two}
more text in chapter two
%    \end{macrocode}

%\iffalse
%</samplechap2>
%\fi
%
% %%%%%%%%%%%%%%%%%%%%%%%%%%%%%%%%%%%%%%
% \paragraph{Part Include Files.}
%
% The include files are called |cdocspt3.tex| and |cdocspt4.tex|.
%
%\iffalse
%<*samplepart3|samplepart4>
%\fi

% Optional override for |\version| flag:
%    \begin{macrocode}
%%\providecommand{\version}{final}
%    \end{macrocode}

% Include the main document:
%    \begin{macrocode}
\input{childdoc.def}
\childdocby{cdocsamp}
%    \end{macrocode}

%\iffalse
%</samplepart3|samplepart4>
%\fi
%
%\iffalse
%<*samplepart3>
%\fi
% Some text for part 3:
%    \begin{macrocode}
some text in part three
%    \end{macrocode}

%\iffalse
%</samplepart3>
%\fi
% Some text for part 4:
%\iffalse
%<*samplepart4>
%\fi
%    \begin{macrocode}
more text in part four
%    \end{macrocode}

%\iffalse
%</samplepart4>
%\fi
%
% %%%%%%%%%%%%%%%%%%%%%%%%%%%%%%%%%%%%%%
% \paragraph{Forwarding for a Complete Draft.}
%
% The following forwarding file |cdocsdrf.tex|
% compiles the main document in draft mode:
%\iffalse
%<*sampledraft>
%\fi
%    \begin{macrocode}
\def\version{draft}
\input{childdoc.def}
\childdocforward{cdocsamp}
%    \end{macrocode}

%\iffalse
%</sampledraft>
%\fi
%
% %%%%%%%%%%%%%%%%%%%%%%%%%%%%%%%%%%%%%%
% \paragraph{Forwarding for Final Version of the Chapters.}
%
% The following forwarding files |cdocsfn1.tex| and |cdocsfn2.tex|
% (with identical content)
% compile the final versions of the child documents
% |cdocsch1.tex| and |cdocsch2.tex|, respectively:
%\iffalse
%<*samplefinal>
%\fi
%    \begin{macrocode}
\def\version{final}
\input{childdoc.def}
\childdocforwardprefix[cdocsamp]{cdocsfn}{cdocsch}
%    \end{macrocode}

%\iffalse
%</samplefinal>
%\fi
%
% %%%%%%%%%%%%%%%%%%%%%%%%%%%%%%%%%%%%%%
% \paragraph{Command Line Processing.}
%
% The following three command lines generate the output files
% |cdocscld|, |cdocscl1| and |cdocscl2|
% which should be identical to
% |cdocsdrf|, |cdocsch1| and |cdocsfn2|, respectively:
% \begin{center}
% \begin{tabular}{l}
% |latex -jobname cdocscld \|\\
% |  "\def\version{draft}\input{childdoc.def}\childdocforward{cdocsamp}"|\\
% |latex -jobname cdocscl1 \|\\
% |  "\input{childdoc.def}\childdocforward[cdocsamp]{cdocsch1}"|\\
% |latex -jobname cdocscl2 \|\\
% |  "\def\version{final}\input{childdoc.def}\childdocforward{cdocsch2}"|
% \end{tabular}
% \end{center}
% Note that the trailing backslash on each first line
% merely continues the input to the second line
% (for convenient cut ant paste).
% Furthermore, the command |latex| can be replaced by any
% of its alternative versions such as |pdflatex|.
%
% %%%%%%%%%%%%%%%%%%%%%%%%%%%%%%%%%%%%%%%%%%%%%%%%%%%%%%%%%%%%%%%%%%%%%%%%%%%%%%
% %%%%%%%%%%%%%%%%%%%%%%%%%%%%%%%%%%%%%%%%%%%%%%%%%%%%%%%%%%%%%%%%%%%%%%%%%%%%%%
% \section{Implementation}
%\iffalse
%<*package>
%\fi
%
% This section describes the definitions file |childdoc.def|.

% The definitions cannot be loaded using |\usepackage| or |\RequirePackage|
% which has a mechanism to prevent loading a style file more than once.
% When loading the definitions by means of |\input|
% multiple instances have to be prevented manually:
%\iffalse
%This code needs to be before the `\ProvidesFile' directive
%which is defined at the beginning of this file.
%Therefore it is also placed there and commented out here.
%</package>
%<*discard>
%\fi
%    \begin{macrocode}
\ifdefined\childdocmain\endinput\fi
%    \end{macrocode}
%\iffalse
%</discard>
%<*package>
%\fi
%
% \macro{\ifchilddoc}
% \macro{\ifchilddocmanual}
% The conditional |\ifchilddoc| tells whether a
% child (true) or main (false) document is being compiled.
% The conditional |\ifchilddocmanual| tells whether
% the |\includeonly| mechanism is used (false) or
% the selection of child files must be performed manually (true).
% The definitions initialise to false:
%    \begin{macrocode}
\newif\ifchilddoc
\newif\ifchilddocmanual
%    \end{macrocode}

% \macro{\childdocname}
% \macro{\childdocjob}
% The macro |\childdocname| stores the name of the main document
% to be compiled. The macro |\childdocjob| stores the name of
% the document on which the \LaTeX{} compiler was originally invoked.
% The content of |\jobname| cannot be compared
% to filenames specified in the source due to different catcodes.
% The following code rescans |\jobname|, stores the result
% in |\childdocname| and saves a copy in |\childdocjob|:
%    \begin{macrocode}
\edef\childdocname{\scantokens\expandafter{\jobname\noexpand}}
\let\childdocjob\childdocname
%    \end{macrocode}

% \macro{\childdocdisable}
% The macro |\childdocdisable| prevents the main file
% from being processed more than once.
% At this stage, the main document command |\childdocmain|
% is assumed to be called once again where it should do nothing.
% Any subsequent call to it should prevent
% a secondary processing of the main document
% It overwrites the forwarding commands
% |\childdocof| and |\childdocforward|
% with empty macros to prevent further inclusions of the main document:
%    \begin{macrocode}
\newcommand{\childdocdisable}
{
  \renewcommand{\childdocmain}[1]{\renewcommand{\childdocmain}[1]{\endinput}}
  \renewcommand{\childdocof}[1]{}
  \renewcommand{\childdocby}[2][]{}
  \renewcommand{\childdocforward}[2][]{}
  \renewcommand{\childdocdisable}{}
}
%    \end{macrocode}

% \macro{\childdocmain}
% The macro |\childdocmain| is to be called at the top of the main file
% with nothing or the main filename (without extension) as argument.
% First, it breaks loops.
% If the argument is not empty and does not match |\childdocname|
% (which is set by the first inclusion of |childdoc.def|),
% |\ifchilddoc| is set to true, |\includeonly| is applied to the child file
% and |\jobname| is set to the main file
% (for proper handling of |.aux| files):
%    \begin{macrocode}
\newcommand{\childdocmain}[1]
{
  \childdocdisable\childdocmain{}
  \if?#1?\else
    \begingroup
      \def\childdoctmp{#1}
      \ifx\childdoctmp\childdocname
        \def\childdoctmp{}
      \else
        \def\childdoctmp
        {
          \childdoctrue
          \includeonly{\childdocname}
          \def\childdocjob{#1}
          \def\jobname{#1}
        }
      \fi
      \expandafter
    \endgroup
    \childdoctmp
  \fi
}
%    \end{macrocode}

% \macro{\childdocof}
% The command |\childdocof| redirects
% compilation to the main file |#1|.
%    \begin{macrocode}
\newcommand{\childdocof}[1]
{
  \childdocdisable
  \childdoctrue
  \includeonly{\childdocname}
  \def\jobname{#1}
  \def\childdocjob{#1}
  \input{#1}
}
%    \end{macrocode}

% \macro{\childdocby}
% The command |\childdocby| ....
%    \begin{macrocode}
\newcommand{\childdocby}[2][]
{
  \childdocdisable
  \childdoctrue
  \childdocmanualtrue
  \if?#1?\else
    \def\jobname{#2}
  \fi
  \def\childdocjob{#2}
  \input{#2}
  \endinput
}
%    \end{macrocode}

% \macro{\childdocforward}
% The command |\childdocforward| redirects
% compilation to the main file or
% (if the optional argument is given) a child file.
% Parameters are set as if the main file
% or a child file starting with |\childdocof| was compiled.
% Then compilation is handed over to the main file:
%    \begin{macrocode}
\newcommand{\childdocforward}[2][]
{
  \begingroup
    \if?#1?
      \def\childdoctmp
      {
        \def\childdocname{#2}
        \def\childdocjob{#2}
        \def\jobname{#2}
        \input{#2}
        \endinput
      }
    \else
      \def\childdoctmp
      {
        \childdocdisable
        \def\childdocname{#2}
        \childdoctrue
        \includeonly{#2}
        \def\childdocjob{#1}
        \def\jobname{#1}
        \input{#1}
        \endinput
      }
    \fi
    \expandafter
  \endgroup
  \childdoctmp
}
%    \end{macrocode}

% \macro{\childdocforwardprefix}
% The command |\childdocforwardprefix| redirects
% compilation to the main or a child file by means of a pattern.
% The prefix |#1| in the current filename is replaced by |#2|
% and the suffix of the current filename is kept
% (it is assumed that the filename does not contain the substring `|~~~|'
% which is used as a delimiter).
% Compilation is handed over to the new file by |\childdocforward|:
%    \begin{macrocode}
\newcommand{\childdocforwardprefix}[3][]
{
  \begingroup
    \def\childdocextract #2##1~~~{\def\childdoctmp{\childdocforward[#1]{#3##1}}}
    \expandafter\childdocextract\childdocname~~~
    \expandafter
  \endgroup
  \childdoctmp
}
%    \end{macrocode}

% \macro{\childdoc}
% The deprecated macro |\childdoc| is a legacy version of |\childdocmain|:
%    \begin{macrocode}
\newcommand{\childdoc}{\childdocmain}
%    \end{macrocode}

% \macro{\childdocredirect}
% The deprecated macro |\childdocredirect| is a legacy version
% of |\childdocforward| and |\childdocforwardprefix|:
%    \begin{macrocode}
\newcommand{\childdocredirect}[2][]
{
  \begingroup
    \if?#1?
      \def\childdoctmp{\childdocforward{#2}}
    \else
      \def\childdoctmp{\childdocforwardprefix{#1}{#2}}
    \fi
    \expandafter
  \endgroup
  \childdoctmp
}
%    \end{macrocode}

%\iffalse
%</package>
%\fi
%
\endinput
|\\
|\childdocmain{}|\\
\end{tabular}
\end{center}
at the very top of the main \LaTeX{} file,
in particular \emph{before} the |\documentclass| statement!
The argument of |\childdocmain| should be left empty
(but it must be present).

%%%%%%%%%%%%%%%%%%%%%%%%%%%%%%%%%%%%%%%%
\DescribeMacro{\childdocof}
Furthermore, add the commands
\begin{center}
\begin{tabular}{l}
|% \iffalse
%
% childdoc.dtx Copyright (C) 2017-2018 Niklas Beisert
%
% This work may be distributed and/or modified under the
% conditions of the LaTeX Project Public License, either version 1.3
% of this license or (at your option) any later version.
% The latest version of this license is in
%   http://www.latex-project.org/lppl.txt
% and version 1.3 or later is part of all distributions of LaTeX
% version 2005/12/01 or later.
%
% This work has the LPPL maintenance status `maintained'.
%
% The Current Maintainer of this work is Niklas Beisert.
%
% This work consists of the files childdoc.dtx and childdoc.ins
% and the derived files childdoc.def and cdocsamp.tex with
% cdocsch1.tex, cdocsch2.tex, cdocsdrf.tex, cdocsfn1.tex, cdocsfn2.tex.
%
%<package>\ifdefined\childdocmain\endinput\fi
%<package>\ProvidesFile{childdoc.def}[2018/12/30 v2.0 child document driver]
%<samplemain>\ProvidesFile{cdocsamp.tex}[2018/12/30 v2.0 sample for childdoc]
%<*driver>
%\ProvidesFile{childdoc.drv}[2018/12/30 v2.0 childdoc reference manual file]
\PassOptionsToClass{10pt,a4paper}{article}
\documentclass{ltxdoc}

\usepackage[margin=35mm]{geometry}
\usepackage{hyperref}
\usepackage{hyperxmp}
\usepackage[usenames]{color}

\hypersetup{colorlinks=true}
\hypersetup{pdfstartview=FitH}
\hypersetup{pdfpagemode=UseNone}
\hypersetup{pdfsource={}}
\hypersetup{pdflang={en-UK}}
\hypersetup{pdfcopyright={Copyright 2017-2018 Niklas Beisert.
  This work may be distributed and/or modified under the
  conditions of the LaTeX Project Public License, either version 1.3
  of this license or (at your option) any later version.}}
\hypersetup{pdflicenseurl={http://www.latex-project.org/lppl.txt}}
\hypersetup{pdfcontactaddress={ETH Zurich, ITP, HIT K,
  Wolfgang-Pauli-Strasse 27}}
\hypersetup{pdfcontactpostcode={8093}}
\hypersetup{pdfcontactcity={Zurich}}
\hypersetup{pdfcontactcountry={Switzerland}}
\hypersetup{pdfcontactemail={nbeisert@itp.phys.ethz.ch}}
\hypersetup{pdfcontacturl={http://people.phys.ethz.ch/\xmptilde nbeisert/}}

\newcommand{\secref}[1]{\hyperref[#1]{section \ref*{#1}}}

\parskip1ex
\parindent0pt
\let\olditemize\itemize
\def\itemize{\olditemize\parskip0pt}

\begin{document}

\title{The \textsf{childdoc} Package}
\hypersetup{pdftitle={The childdoc Package}}
\author{Niklas Beisert\\[2ex]
  Institut f\"ur Theoretische Physik\\
  Eidgen\"ossische Technische Hochschule Z\"urich\\
  Wolfgang-Pauli-Strasse 27, 8093 Z\"urich, Switzerland\\[1ex]
  \href{mailto:nbeisert@itp.phys.ethz.ch}
  {\texttt{nbeisert@itp.phys.ethz.ch}}}
\hypersetup{pdfauthor={Niklas Beisert}}
\hypersetup{pdfsubject={Manual for the LaTeX2e Package childdoc}}
\date{30 December 2018, \textsf{v2.0}}
\maketitle

\begin{abstract}\noindent
\textsf{childdoc} is a \LaTeXe{} package
that enables the direct compilation
of document sections included by |\include|
to individual files.
\end{abstract}

\begingroup
\parskip0ex
\tableofcontents
\endgroup

%%%%%%%%%%%%%%%%%%%%%%%%%%%%%%%%%%%%%%%%%%%%%%%%%%%%%%%%%%%%%%%%%%%%%%%%%%%%%%%%
%%%%%%%%%%%%%%%%%%%%%%%%%%%%%%%%%%%%%%%%%%%%%%%%%%%%%%%%%%%%%%%%%%%%%%%%%%%%%%%%
\section{Introduction}

\LaTeX{} provides a mechanism to structure a large document (such as a book)
into a main file and several child files (containing the chapters)
using the |\include| command.
This mechanism is beneficial for documents
which span hundreds of pages in order to
make the source file(s) more manageable.
Moreover, compilation can be restricted to
selected child files by means of the |\includeonly| command.
The latter feature can be used to reduce the compilation time while editing
(this was significantly more useful in the earlier days of \LaTeX{})
or to generate a smaller document which is easier to navigate.
Another application of |\includeonly| is to generate
documents consisting of selected parts of the complete document.

However, there are a few drawbacks of the plain |\include| mechanism:
\begin{itemize}
\item
The child files cannot be compiled on their own,
they can only be compiled via the main file.
A naive editing environment
(such as a text editor with an option
to have the current file processed by \LaTeX)
may require one to switch to the main file before compiling;
attempting to compile the child file produces errors.
\item
The main file must be modified (each time)
to adjust the |\includeonly| command
to the present needs. This easily leaves the main file in a messy state.
\item
The generated document will always carry the filename
of the main document. This is inconvenient if
several child files are to be compiled and
to be kept for distribution.
\end{itemize}

The present package provides a simple interface
to make child files individually compilable by \LaTeX{}.
Compiling a child file then has the same effect as compiling
the main file with an |\includeonly| command
to select the appropriate child.
Moreover the generated document will carry the name of the child
rather than the main file.
This resolves all three above issues.

This feature is meant to make the editing of books,
thesis documents and lecture notes somewhat more convenient.
However, the package can also be used efficiently for
composing a series of documents (such as exercise sheets)
which are typically distributed individually.
It then assists the author in generating the individual documents
(potentially in different versions)
as well as a document containing the collected series.
Another application is in developing style files
or other kinds of included material
where compilation of the style file could redirect
to a sample or test file.

%%%%%%%%%%%%%%%%%%%%%%%%%%%%%%%%%%%%%%%%%%%%%%%%%%%%%%%%%%%%%%%%%%%%%%%%%%%%%%%%
%%%%%%%%%%%%%%%%%%%%%%%%%%%%%%%%%%%%%%%%%%%%%%%%%%%%%%%%%%%%%%%%%%%%%%%%%%%%%%%%
\section{Usage}

First of all, the package \textsf{childdoc} is \emph{not} a standard
\LaTeXe{} |.sty| style file! Therefore it needs to be invoked in
a non-standard way.

%%%%%%%%%%%%%%%%%%%%%%%%%%%%%%%%%%%%%%%%%%%%%%%%%%%%%%%%%%%%%%%%%%%%%%%%%%%%%%%%
\subsection{Included Files}
\label{sec:include}

%%%%%%%%%%%%%%%%%%%%%%%%%%%%%%%%%%%%%%%%
\DescribeMacro{\childdocmain}
To use the package, add the commands
\begin{center}
\begin{tabular}{l}
|\input{childdoc.def}|\\
|\childdocmain{}|\\
\end{tabular}
\end{center}
at the very top of the main \LaTeX{} file,
in particular \emph{before} the |\documentclass| statement!
The argument of |\childdocmain| should be left empty
(but it must be present).

%%%%%%%%%%%%%%%%%%%%%%%%%%%%%%%%%%%%%%%%
\DescribeMacro{\childdocof}
Furthermore, add the commands
\begin{center}
\begin{tabular}{l}
|\input{childdoc.def}|\\
|\childdocof{|\textit{main}|}|\\
\end{tabular}
\end{center}
at the top of every child file \textit{child}
which is included by |\include{|\textit{child}|}|
from within the main file
(or at least for those files to be compiled individually).
The argument \textit{main} must be the filename of the main file.

There are a couple of
considerations in setting up the main and child documents:

%%%%%%%%%%%%%%%%%%%%%%%%%%%%%%%%%%%%%%%%
\paragraph{Restrictions.}

Please note the following restrictions:
\begin{itemize}
\item
|\childdocmain| must be called with one argument \textit{main}
to ensure compatibility with earlier version of the package.
It must either be empty (|\childdocmain{}|)
or precisely match the filename of the main file in which it is specified.
See \secref{sec:detection} for further information.
\item
The filename \textit{main} must be specified without the |.tex| extension.
\item
The filename \textit{main} is case sensitive
(even in case-insensitive file systems)
due to internal string comparison.
\item
The argument \textit{main} should be fully expanded, it cannot be a macro.
\item
Subdirectories and special characters should be avoided in filenames.
\item
The command |\childdocmain{|\textit{main}|}| must be followed by a whitespace.
It should not be followed immediately by another command
or by a comment mark `|%|'.
This is because the \TeX{} parser reads the token immediately following
the argument of |\childdocmain| and puts it
at the beginning of every child section;
however, a white\-space is ignored.
\end{itemize}

%%%%%%%%%%%%%%%%%%%%%%%%%%%%%%%%%%%%%%%%
\paragraph{Content of Main File.}

It is advisable to place all content in the child files included by |\include|.
Any output contained in the main file will appear in all child documents
unless suppressed manually;
it cannot be suppressed automatically by the |\includeonly| directive
and thus should normally be avoided.
A method to include some content in the main file
by means of conditional processing is described in \secref{sec:conditional}.

%%%%%%%%%%%%%%%%%%%%%%%%%%%%%%%%%%%%%%%%
\paragraph{Page Numbering.}

When only a part of the document is compiled,
the appropriate numbering of pages
(as well as other status parameters)
is determined from the |.aux| files.
The latter contain information from previous passes.
However this information needs to propagate through
all intermediate child documents.
Therefore the page numbering in child documents may well
be inconsistent until the complete document is compiled at least once.

A useful (if unconventional) way to always ensure a consistent
page numbering is to restart the numbering in each child document
and denote the pages by `\textit{child}|.|\textit{page}'
where \textit{child} represents the chapter/section number of the child file.
This can be achieved by the command
|\numberwithin{page}{|\textit{child}|}|
of the \textsf{amsmath} package
where \textit{child} can be |chapter| or |section|
depending on the chosen structuring.
Alternatively, one can modify the macro |\thepage| appropriately
and reset the counter |page| at the start of each child file.

%%%%%%%%%%%%%%%%%%%%%%%%%%%%%%%%%%%%%%%%%%%%%%%%%%%%%%%%%%%%%%%%%%%%%%%%%%%%%%%%
\subsection{Conditional Processing}
\label{sec:conditional}

The package provides a mechanism to compile different versions
of a document. To customise the versions further some conditional processing
can come in handy to distinguish which version is being compiled.
The package provides two macros to describe the compilation context:

%%%%%%%%%%%%%%%%%%%%%%%%%%%%%%%%%%%%%%%%
\DescribeMacro{\ifchilddoc}
The conditional |\ifchilddoc| distinguishes between the compilation of
child documents and the main document:
%
\begin{center}
|\ifchilddoc |\textit{child-code}| |[|\||else |\textit{main-code}]| \||fi|
\end{center}

%%%%%%%%%%%%%%%%%%%%%%%%%%%%%%%%%%%%%%%%
\DescribeMacro{\childdocname}
\DescribeMacro{\childdocjob}
The macro |\childdocname| contains the filename (without extension)
of the main or child file being processed.
Note that |\childdocjob| will always contain the name of the main file.

%%%%%%%%%%%%%%%%%%%%%%%%%%%%%%%%%%%%%%%%
\paragraph{Title Page.}

Conditional processing can be used to include a title or banner page
in the main document when proper precautions are taken.
Importantly, the code in the main file should ensure that the page counter
(as well as other status parameters which are stored in the |.aux| files)
takes the same value after the conditional processing.
Otherwise the page numbers may take divergent values
depending on which part is compiled.

For example, a title page could be declared by:
%
\begin{center}
\begin{tabular}{l}
|\ifchilddoc\||else|\\
|\addtocounter{page}{-1}|\\
\textit{code for title page}\\
|\newpage|\\
|\||fi|
\end{tabular}
\end{center}
%
A banner page for the child documents can be generated by:
%
\begin{center}
\begin{tabular}{l}
|\ifchilddoc|\\
|\addtocounter{page}{-1}|\\
\textit{code for banner page}\\
|\newpage|\\
|\||fi|
\end{tabular}
\end{center}
%
Here one could write a message such as:
\begin{center}
|This is the part \childdocname{} of \childdocjob{}.|
\end{center}

%%%%%%%%%%%%%%%%%%%%%%%%%%%%%%%%%%%%%%%%%%%%%%%%%%%%%%%%%%%%%%%%%%%%%%%%%%%%%%%%
\subsection{Flags}
\label{sec:flags}

The package makes it easy to generate different versions
of the main or child documents.
To this end compilation flags can be defined
and assigned different default values.
They will be particularly useful in conjunction
with the forwarding mechanism described in \secref{sec:forward}.

For example, it may be useful to have a flag |\version|
which can be set to |draft| or |final|.
The document source will contain some conditional code
depending on the value of |\version|.
Suppose further, the flag should default to |final| for the main file
and to |draft| for child files
which is a natural assignment for editing the document.
This is achieved by placing the following code
in the preamble of the main document
(below the |\childdocmain| directive):
%
\begin{center}
\begin{tabular}{l}
|\ifchilddoc|\\
|\providecommand{\version}{draft}|\\
|\||else|\\
|\providecommand{\version}{final}|\\
|\||fi|
\end{tabular}
\end{center}
%
The definition by |\providecommand| makes sure
that previous definitions are not overwritten.
Further statements |\providecommand{\version}{...}|
can thus be added before the above code to override it.

For the main file, one might add a line
(between |\childdocmain| and the above block)
%
\begin{center}
|%\ifchilddoc\||else\providecommand{\version}{draft}\||fi|
\end{center}
%
which can be uncommented to produce a draft version.
Likewise one can add a line to the very top of a child file
(above the |\childdocof{|\textit{main}|}| directive)
%
\begin{center}
|%\providecommand{\version}{final}|
\end{center}
%
which can be uncommented to produce the final version of this child document.

%%%%%%%%%%%%%%%%%%%%%%%%%%%%%%%%%%%%%%%%%%%%%%%%%%%%%%%%%%%%%%%%%%%%%%%%%%%%%%%%
\subsection{Forwarding}
\label{sec:forward}

Different versions of the main or child documents
using compilation flags as described in \secref{sec:flags}
can be (permanently) stored in different files
for convenient compilation, viewing and distribution.
To this end, the package defines a command
to pass on compilation to a different file:

%%%%%%%%%%%%%%%%%%%%%%%%%%%%%%%%%%%%%%%%
\DescribeMacro{\childdocforward}
The command |\childdocforward| redirects processing to
another source file:
%
\begin{center}
\begin{tabular}{l}
|\input{childdoc.def}|\\
|\childdocforward[|\textit{main}|]{|\textit{dest}|}|\\
\end{tabular}
\end{center}
%
The argument \textit{dest} is the destination file
(without extension).
It should be the main file or one of the child files.
Note that further \textsf{childdoc} directives
such as |\childdocof| and |\childdocforward|
in the indicated file will be processed in this form.
The optional argument \textit{main}
passes on directly to the main file \textit{main}
while pretending to compile the child \textit{dest}.
This form behaves as if \textit{dest}
issues |\childdocof{|\textit{main}|}| right away,
and no further \textsf{childdoc} directives will be processed.

%%%%%%%%%%%%%%%%%%%%%%%%%%%%%%%%%%%%%%%%
\DescribeMacro{\...prefix}
In the alternative form |\childdocforwardprefix|,
%
\begin{center}
\begin{tabular}{l}
|\input{childdoc.def}|\\
|\childdocforwardprefix[|\textit{main}|]{|\textit{prefix}|}{|\textit{dest}|}|
\end{tabular}
\end{center}
%
the destination file is determined by a pattern
depending on the current file:
To make this work, the current file must be called
`{\textit{prefix}\hspace{0.2em}\textit{suffix}}'
with \textit{prefix} matching precisely the argument.
Processing is then passed on to the file
`{\textit{dest}\hspace{0.2em}\textit{suffix}}'.
Surely, the same effect is achieved by
directly specifying the
argument `{\textit{dest}\hspace{0.2em}\textit{suffix}}'
in the first form.
However, that requires to set up a different file
for each child. With the alternative form of the command
all these files can have exactly the same content
which simplifies setting them up and maintaining them.

For example, the following file |draft.tex|
with a compilation flag |\version| as described in \secref{sec:flags}
compiles the main document as a draft:
%
\begin{center}
\begin{tabular}{l}
|\def\version{draft}|\\
|\input{childdoc.def}|\\
|\childdocforward{|\textit{main}|}|
\end{tabular}
\end{center}
%
Likewise, the following files |final|\textit{nn}|.tex|
compile the final version of the child document
|child|\textit{nn}|.tex|:
%
\begin{center}
\begin{tabular}{l}
|\def\version{final}|\\
|\input{childdoc.def}|\\
|\childdocforwardprefix{final}{child}|
\end{tabular}
\end{center}
%

Note that when several versions of a main file and/or of each child file
are to be generated, it may be convenient to set up a |Makefile| or
shell script to automatise the process.

%%%%%%%%%%%%%%%%%%%%%%%%%%%%%%%%%%%%%%%%%%%%%%%%%%%%%%%%%%%%%%%%%%%%%%%%%%%%%%%%
\subsection{Command Line Processing}
\label{sec:commandline}

The effect of redirection files can also be achieved by invoking
the \LaTeX{} compiler with a more elaborate command line.
Most conveniently this should be done as part
of a shell script or a |Makefile|.

When using \textsf{childdoc} in the main file, the following
command lines effectively perform a redirection
(note that depending on the shell being used,
backslashes may have to be doubled: `|\|' $\to$ `|\\|'):
%
\begin{center}
|... -jobname "|\textit{target}|" |\\|"|[\textit{flags}]%
|\input{childdoc.def}\childdocforward[|\textit{main}|]{|\textit{dest}|}"|
\end{center}
%
Here \textit{target} is the name of the output file,
\textit{main} is the name of the main file
and \textit{dest} is the name of the main or child file to be processed
(all filenames without extensions).
The optional argument \textit{main} can be omitted
if \textit{main} matches \textit{dest}.
Optionally, compilation \textit{flags} can be defined via |\def| commands.
This command line makes the \TeX{} engine believe
it is compiling the file \textit{target}
whose content is specified as the latter parameter.
The provided code then forwards the processing to
\textit{main} or \textit{dest} as described in \secref{sec:forward}.

%%%%%%%%%%%%%%%%%%%%%%%%%%%%%%%%%%%%%%%%%%%%%%%%%%%%%%%%%%%%%%%%%%%%%%%%%%%%%%%%
\subsection{Include by Input}
\label{sec:input}

Including child documents by |\include| has some restrictions by design.
Most notably, the content of a child document always occupies
its own set of pages; pages cannot be shared between child documents.
Usually, this behaviour makes perfect sense
because each child document contain an essential part of the document.
However, in some situations it may be desirable to compose
a document from a collection of parts
without having mandatory page breaks between then.
For this case, the package
provides a mechanism to include parts
by |\input| which can also be processed individually.
However, by construction this mechanism
requires manual handling of the content to be output.

%%%%%%%%%%%%%%%%%%%%%%%%%%%%%%%%%%%%%%%%
\DescribeMacro{\ifchilddocmanual}
The main file should be prepared as usual, see \secref{sec:include}.
However, the document body must make a distinction
between processing of an individual part and of the main document, e.g.:
%
\begin{center}
\begin{tabular}{l}
|\ifchilddocmanual|\\
|\input{\childdocname}|\\
|\||else|\\
\textit{document body with }|\input{|\textit{part}|}|\\
|\||fi|
\end{tabular}
\end{center}
%
The conditional |\ifchilddocmanual| is true whenever
a part to be included by |\input| is being compiled,
and the name of the part is stored in |\childdocname|.

%%%%%%%%%%%%%%%%%%%%%%%%%%%%%%%%%%%%%%%%
\DescribeMacro{\childdocby}
Each part to be included by |\input| should start with:
%
\begin{center}
\begin{tabular}{l}
|\input{childdoc.def}|\\
|\childdocby{|\textit{main}|}|\\
\end{tabular}
\end{center}
%
The directive |\childdocby| is similar to |\childdocof|
described in \secref{sec:include},
but the subsequent selection of content must be done manually.
To that end, both |\ifchilddoc| and |\ifchilddocmanual|
will be true upon processing of a part,
and the name of the part is stored in |\childdocname|.
Note that |\jobname| will be set to the filename of the current part
so that each part receives an individual |.aux| file
that does not interfere with the |.aux| file(s) of the main document.
This behaviour can be altered by the alternative form
|\childdocby[*]{|\textit{main}|}| (with a non-empty optional argument)
which uses the |.aux| file of the main document
by setting |\jobname| to \textit{main}.

%%%%%%%%%%%%%%%%%%%%%%%%%%%%%%%%%%%%%%%%%%%%%%%%%%%%%%%%%%%%%%%%%%%%%%%%%%%%%%%%
\subsection{Driver Development}
\label{sec:driver}

The \textsf{childdoc} mechanism can also be use for the development
of definition files such as \LaTeX{} styles or classes.
This case differs from the above setup with multiple parts
included by |\include| in that no |\includeonly| should be invoked.
This can be achieved by starting the include file
(before |\ProvidesPackage|) with:
%
\begin{center}
\begin{tabular}{l}
|\input{childdoc.def}|\\
|\childdocforward{|\textit{main}|}|\\
\end{tabular}
\end{center}
%
or alternatively with:
%
\begin{center}
\begin{tabular}{l}
|\input{childdoc.def}|\\
|\childdocby{|\textit{main}|}|\\
\end{tabular}
\end{center}
%
Both forms have slightly different effects as described above.
The main file is prepared as usual, see \secref{sec:include}.

%%%%%%%%%%%%%%%%%%%%%%%%%%%%%%%%%%%%%%%%%%%%%%%%%%%%%%%%%%%%%%%%%%%%%%%%%%%%%%%%
\subsection{Legacy Detection}
\label{sec:detection}

The directive |\childdocmain| in the main file can detect
whether the complete document or merely a child is to be compiled
even without using the directive |\childdocof|.
This method is deprecated because it is less robust
and there is no compelling reason to use it;
it is merely provided for backward compatibility
and it may be removed in future versions.

If the detection mechanism is to be used,
it is mandatory to correctly specify
the filename of the main file as the argument of |\childdocmain|:
%
\begin{center}
\begin{tabular}{l}
|\input{childdoc.def}|\\
|\childdocmain{|\textit{main}|}|\\
\end{tabular}
\end{center}
%
If |\jobname| does not match the argument \textit{main} of |\childdocmain|,
it is assumed that |\jobname| points to the child file to be compiled.
When using |\childdocmain| with the main file specified as argument,
it suffices to start a child file
with just |\input{|\textit{main}|}|
without loading of the package and using |\childdocof|.
If instead all processing is done
with the appropriate \textsf{childdoc} directives,
the argument of \textit{main} of |\childdocmain| can be empty.

An alternative version of the command line processing described
in \secref{sec:commandline} using the detection mechanism reads:
%
\begin{center}
|... -jobname "|\textit{target}|" "|[\textit{flags}]%
[|\def\jobname{|\textit{dest}|}|]|\input{|\textit{main}|}"|
\end{center}

%%%%%%%%%%%%%%%%%%%%%%%%%%%%%%%%%%%%%%%%%%%%%%%%%%%%%%%%%%%%%%%%%%%%%%%%%%%%%%%%
\subsection{Manual Code}
\label{sec:manual}

In case one cannot be certain whether the definitions file |childdoc.def|
is installed on the target \TeX{} distribution
and one prefers not to ship it,
it is conceivable to paste a few relevant commands into the sources.

To that end, drop all statements |\input{childdoc.def}|
and perform the replacements as outlined below.
Instead of |\childdocmain{|\textit{main}|}| add the following code
to the top of the main file:
%
\begin{center}
\begin{tabular}{l}
|\||ifdefined\childdocname\endinput\||fi\newif\ifchilddoc|\\
|\edef\childdocname{\scantokens\expandafter{\jobname\noexpand}}|\\
|\def\childdocmain{|\textit{main}|}\||ifx\childdocmain\childdocname\||else|\\
|\childdoctrue\includeonly{\childdocname}\let\jobname\childdocmain\||fi|\\
\end{tabular}
\end{center}
%
Instead of |\childdocof{|\textit{main}|}| just include the main file
at the top of each child file:
%
\begin{center}
|\input{|\textit{main}|}|
\end{center}
%
A simple redirection |\childdocforward{|\textit{dest}|}| is achieved by:
%
\begin{center}
|\def\jobname{|\textit{dest}|}\input{\jobname}|
\end{center}
%
The redirection with prefix
|\childdocforwardprefix[|\textit{prefix}|]{|\textit{dest}|}|
is accomplished by:
%
\begin{center}
\begin{tabular}{l}
|{\edef\jobname{\scantokens\expandafter{\jobname\noexpand}}|\\
|\def\redirectjob |\textit{prefix}|#1~~~{\gdef\jobname{|\textit{dest}|#1}}|\\
|\expandafter\redirectjob\jobname~~~}\input{\jobname}|
\end{tabular}
\end{center}

In an alternative approach,
child documents can be compiled by a specific command line
without additional code or specific definitions:
%
\begin{center}
|... -jobname "|\textit{target}|" "|[\textit{flags}]%
|\includeonly{|\textit{dest}|}\input{|\textit{main}|}"|
\end{center}
%

%%%%%%%%%%%%%%%%%%%%%%%%%%%%%%%%%%%%%%%%%%%%%%%%%%%%%%%%%%%%%%%%%%%%%%%%%%%%%%%%
%%%%%%%%%%%%%%%%%%%%%%%%%%%%%%%%%%%%%%%%%%%%%%%%%%%%%%%%%%%%%%%%%%%%%%%%%%%%%%%%
\section{Information}

%%%%%%%%%%%%%%%%%%%%%%%%%%%%%%%%%%%%%%%%%%%%%%%%%%%%%%%%%%%%%%%%%%%%%%%%%%%%%%%%
\subsection{Copyright}

Copyright \copyright{} 2017--2018 Niklas Beisert

This work may be distributed and/or modified under the
conditions of the \LaTeX{} Project Public License, either version 1.3
of this license or (at your option) any later version.
The latest version of this license is in
  \url{http://www.latex-project.org/lppl.txt}
and version 1.3 or later is part of all distributions of \LaTeX{}
version 2005/12/01 or later.

This work has the LPPL maintenance status `maintained'.

The Current Maintainer of this work is Niklas Beisert.

This work consists of the files |README.txt|, |childdoc.ins| and |childdoc.dtx|
as well as the derived files |childdoc.def|, |cdocsamp.tex|
with |cdocsch1.tex|, |cdocsch2.tex|, |cdocspt3.tex|, |cdocspt4.tex|,
|cdocsdrf.tex|, |cdocsfn1.tex|, |cdocsfn2.tex|
as well as |childdoc.pdf|.

%%%%%%%%%%%%%%%%%%%%%%%%%%%%%%%%%%%%%%%%%%%%%%%%%%%%%%%%%%%%%%%%%%%%%%%%%%%%%%%%
\subsection{Files and Installation}

The package consists of the files:
%
\begin{center}
\begin{tabular}{ll}
    |README.txt|   & readme file \\
    |childdoc.ins| & installation file \\
    |childdoc.dtx| & source file \\
    |childdoc.def| & definition file \\
    |cdocsamp.tex| & sample main file \\
    |cdocsch1.tex| & sample include file \\
    |cdocsch2.tex| & sample include file \\
    |cdocspt3.tex| & sample part file \\
    |cdocspt4.tex| & sample part file \\
    |cdocsdrf.tex| & sample redirection file \\
    |cdocsfn1.tex| & sample redirection file \\
    |cdocsfn2.tex| & sample redirection file \\
    |childdoc.pdf| & manual
\end{tabular}
\end{center}
%
The distribution consists of the files
|README.txt|, |childdoc.ins| and |childdoc.dtx|.
%
\begin{itemize}
\item
Run (pdf)\LaTeX{} on |childdoc.dtx|
to compile the manual |childdoc.pdf| (this file).
\item
Run \LaTeX{} on |childdoc.ins| to create the definitions file |childdoc.def|
and the sample |cdocsamp.tex| with include files
|cdocsch1.tex|, |cdocsch2.tex|, |cdocspt3.tex|, |cdocspt4.tex|,
|cdocsdrf.tex|, |cdocsfn1.tex|, |cdocsfn2.tex|.
Then copy the file |childdoc.def| to an appropriate directory of your \LaTeX{}
distribution, e.g.\ \textit{texmf-root}|/tex/latex/childdoc|.
\end{itemize}

%%%%%%%%%%%%%%%%%%%%%%%%%%%%%%%%%%%%%%%%%%%%%%%%%%%%%%%%%%%%%%%%%%%%%%%%%%%%%%%%
\subsection{Related CTAN Packages}

There are several other packages which offer a similar functionality:
%
\begin{itemize}
\item
The packages
\href{http://ctan.org/pkg/docmute}{\textsf{docmute}},
\href{http://ctan.org/pkg/includex}{\textsf{includex}} and
\href{http://ctan.org/pkg/standalone}{\textsf{standalone}}
provide commands to include only the document body of
a child file thus allowing both files to be compiled individually.
\item
The packages \href{http://ctan.org/pkg/subdocs}{\textsf{subdocs}}
and \href{http://ctan.org/pkg/subfiles}{\textsf{subfiles}}
provide structures in which the main and child documents can be
encapsulated and allowing them to be compiled individually.
The inclusion mechanism is different from the conventional |\include|.
\item
The package \href{http://ctan.org/pkg/combine}{\textsf{combine}}
is an elaborate solution to combine several documents into one.
\end{itemize}
%
See also the CTAN topic \href{http://ctan.org/topic/subdocs}{\textsf{subdocs}}
for further related packages.
The present package differs from the above solutions in that
a document structure constructed with the conventional |\include| mechanism
just needs two extra commands at the top of every file
such that all constituent files can be compiled individually.

%%%%%%%%%%%%%%%%%%%%%%%%%%%%%%%%%%%%%%%%%%%%%%%%%%%%%%%%%%%%%%%%%%%%%%%%%%%%%%%%
%\subsection{Feature Suggestions}
%
%The following is a list of features which may be useful for future
%versions of this package:
%%
%\begin{itemize}
%\item
%\ldots
%\end{itemize}

%%%%%%%%%%%%%%%%%%%%%%%%%%%%%%%%%%%%%%%%%%%%%%%%%%%%%%%%%%%%%%%%%%%%%%%%%%%%%%%%
\subsection{Revision History}

%%%%%%%%%%%%%%%%%%%%%%%%%%%%%%%%%%%%%%%%
\paragraph{v2.0:} 2018/12/30

\begin{itemize}
\item
immediate forward processing
\item
added |\childdocby| mechanism
\item
manual restructured
\end{itemize}

%%%%%%%%%%%%%%%%%%%%%%%%%%%%%%%%%%%%%%%%
\paragraph{v1.6:} 2018/01/17

\begin{itemize}
\item
application for development of include files
\item
corrections to manual
\end{itemize}

%%%%%%%%%%%%%%%%%%%%%%%%%%%%%%%%%%%%%%%%
\paragraph{v1.5:} 2017/05/21

\begin{itemize}
\item
more complete structuring introduced
\item
|\childdocof| introduced
\item
|\childdoc| renamed to |\childdocmain|
\item
|\childredirect| renamed to |\childdocforward| and |\childdocforwardprefix|
and functionality expanded
\end{itemize}

%%%%%%%%%%%%%%%%%%%%%%%%%%%%%%%%%%%%%%%%
\paragraph{v1.0:} 2017/04/27

\begin{itemize}
\item
manual and install package
\item
first version published on CTAN
\end{itemize}

%%%%%%%%%%%%%%%%%%%%%%%%%%%%%%%%%%%%%%%%
\paragraph{v0.6:} 2017/04/26

\begin{itemize}
\item
redirection mechanism added
\end{itemize}

%%%%%%%%%%%%%%%%%%%%%%%%%%%%%%%%%%%%%%%%
\paragraph{v0.5:} 2017/04/26

\begin{itemize}
\item
functionality in definition file
\end{itemize}


%%%%%%%%%%%%%%%%%%%%%%%%%%%%%%%%%%%%%%%%%%%%%%%%%%%%%%%%%%%%%%%%%%%%%%%%%%%%%%%%
%%%%%%%%%%%%%%%%%%%%%%%%%%%%%%%%%%%%%%%%%%%%%%%%%%%%%%%%%%%%%%%%%%%%%%%%%%%%%%%%
%%%%%%%%%%%%%%%%%%%%%%%%%%%%%%%%%%%%%%%%%%%%%%%%%%%%%%%%%%%%%%%%%%%%%%%%%%%%%%%%
\appendix

\settowidth\MacroIndent{\rmfamily\scriptsize 000\ }

 \DocInput{childdoc.dtx}

\end{document}
%</driver>
% \fi
%
% %%%%%%%%%%%%%%%%%%%%%%%%%%%%%%%%%%%%%%%%%%%%%%%%%%%%%%%%%%%%%%%%%%%%%%%%%%%%%%
% %%%%%%%%%%%%%%%%%%%%%%%%%%%%%%%%%%%%%%%%%%%%%%%%%%%%%%%%%%%%%%%%%%%%%%%%%%%%%%
% \section{Sample}
%\iffalse
%<*samplemain>
%\fi
%
% The following presents a sample document
% with two chapters, two parts, a title page,
% a compile flag as well as three forwarding files to set the flag.
% It consists of eight |.tex| files:
% \begin{center}
% \begin{tabular}{ll}
% |cdocsamp.tex|&main file\\
% |cdocsch1.tex|&include file for chapter 1\\
% |cdocsch2.tex|&include file for chapter 2\\
% |cdocspt3.tex|&include file for part 3\\
% |cdocspt4.tex|&include file for part 4\\
% |cdocsdrf.tex|&forwarding file for main file in draft mode\\
% |cdocsfi1.tex|&forwarding file for final version of chapter 1\\
% |cdocsfi2.tex|&forwarding file for final version of chapter 2\\
% \end{tabular}
% \end{center}
% Each of the eight files can be compiled directly by the \LaTeX{} compiler.
%
% %%%%%%%%%%%%%%%%%%%%%%%%%%%%%%%%%%%%%%
% \paragraph{Main File.}
%
% The main file is called |cdocsamp.tex|.
%
% Load the \textsf{childdoc} definitions and
% declare the filename for the main document:
%    \begin{macrocode}
\input{childdoc.def}
\childdocmain{}
%    \end{macrocode}

% Optional override for |\version| flag:
%    \begin{macrocode}
%%\ifchilddoc\else\providecommand{\version}{draft}\fi
%    \end{macrocode}

% Define the default values for the |\version| flag
% (|final| for the main file and |draft| for childs):
%    \begin{macrocode}
\ifchilddoc
\providecommand{\version}{draft}
\else
\providecommand{\version}{final}
\fi
%    \end{macrocode}

% Load the standard document class:
%    \begin{macrocode}
\documentclass[12pt]{article}
%    \end{macrocode}

% Start the document body:
%    \begin{macrocode}
\begin{document}
%    \end{macrocode}

% Declare a title page.
% Print title, part of document being processed and version flag:
%    \begin{macrocode}
\addtocounter{page}{-1}
\begin{center}
{\LARGE\bfseries{}childdoc example\par}
\vspace{1cm}
\ifchilddoc
\ifchilddocmanual part\else chapter\fi:
`\childdocname' of `\childdocjob'\par
\else
main document: `\childdocjob'\par
\fi
version: \version\par
\end{center}
\newpage
%    \end{macrocode}

% Manually include selected file,
% otherwise process as usual:
%    \begin{macrocode}
\ifchilddocmanual
\section*{part `\childdocname'}
\input{\childdocname}
\else
%    \end{macrocode}

% Include the two chapters:
%    \begin{macrocode}
\include{cdocsch1}
\include{cdocsch2}
%    \end{macrocode}

% Include the two parts unless only chapters should be displayed:
%    \begin{macrocode}
\ifchilddoc\else
\section{part three}
\input{cdocspt3}
\section{part four}
\input{cdocspt4}
\fi
%    \end{macrocode}

% Process as usual until here:
%    \begin{macrocode}
\fi
%    \end{macrocode}

% End of document body:
%    \begin{macrocode}
\end{document}
%    \end{macrocode}
%\iffalse
%</samplemain>
%\fi
%
% %%%%%%%%%%%%%%%%%%%%%%%%%%%%%%%%%%%%%%
% \paragraph{Chapter Include Files.}
%
% The include files are called |cdocsch1.tex| and |cdocsch2.tex|.
%
%\iffalse
%<*samplechap1|samplechap2>
%\fi

% Optional override for |\version| flag:
%    \begin{macrocode}
%%\providecommand{\version}{final}
%    \end{macrocode}

% Include the main document:
%    \begin{macrocode}
\input{childdoc.def}
\childdocof{cdocsamp}
%    \end{macrocode}

%\iffalse
%</samplechap1|samplechap2>
%\fi
%
%\iffalse
%<*samplechap1>
%\fi
% Some text for chapter 1:
%    \begin{macrocode}
\section{one}
some text in chapter one
%    \end{macrocode}

%\iffalse
%</samplechap1>
%\fi
% Some text for chapter 2:
%\iffalse
%<*samplechap2>
%\fi
%    \begin{macrocode}
\section{two}
more text in chapter two
%    \end{macrocode}

%\iffalse
%</samplechap2>
%\fi
%
% %%%%%%%%%%%%%%%%%%%%%%%%%%%%%%%%%%%%%%
% \paragraph{Part Include Files.}
%
% The include files are called |cdocspt3.tex| and |cdocspt4.tex|.
%
%\iffalse
%<*samplepart3|samplepart4>
%\fi

% Optional override for |\version| flag:
%    \begin{macrocode}
%%\providecommand{\version}{final}
%    \end{macrocode}

% Include the main document:
%    \begin{macrocode}
\input{childdoc.def}
\childdocby{cdocsamp}
%    \end{macrocode}

%\iffalse
%</samplepart3|samplepart4>
%\fi
%
%\iffalse
%<*samplepart3>
%\fi
% Some text for part 3:
%    \begin{macrocode}
some text in part three
%    \end{macrocode}

%\iffalse
%</samplepart3>
%\fi
% Some text for part 4:
%\iffalse
%<*samplepart4>
%\fi
%    \begin{macrocode}
more text in part four
%    \end{macrocode}

%\iffalse
%</samplepart4>
%\fi
%
% %%%%%%%%%%%%%%%%%%%%%%%%%%%%%%%%%%%%%%
% \paragraph{Forwarding for a Complete Draft.}
%
% The following forwarding file |cdocsdrf.tex|
% compiles the main document in draft mode:
%\iffalse
%<*sampledraft>
%\fi
%    \begin{macrocode}
\def\version{draft}
\input{childdoc.def}
\childdocforward{cdocsamp}
%    \end{macrocode}

%\iffalse
%</sampledraft>
%\fi
%
% %%%%%%%%%%%%%%%%%%%%%%%%%%%%%%%%%%%%%%
% \paragraph{Forwarding for Final Version of the Chapters.}
%
% The following forwarding files |cdocsfn1.tex| and |cdocsfn2.tex|
% (with identical content)
% compile the final versions of the child documents
% |cdocsch1.tex| and |cdocsch2.tex|, respectively:
%\iffalse
%<*samplefinal>
%\fi
%    \begin{macrocode}
\def\version{final}
\input{childdoc.def}
\childdocforwardprefix[cdocsamp]{cdocsfn}{cdocsch}
%    \end{macrocode}

%\iffalse
%</samplefinal>
%\fi
%
% %%%%%%%%%%%%%%%%%%%%%%%%%%%%%%%%%%%%%%
% \paragraph{Command Line Processing.}
%
% The following three command lines generate the output files
% |cdocscld|, |cdocscl1| and |cdocscl2|
% which should be identical to
% |cdocsdrf|, |cdocsch1| and |cdocsfn2|, respectively:
% \begin{center}
% \begin{tabular}{l}
% |latex -jobname cdocscld \|\\
% |  "\def\version{draft}\input{childdoc.def}\childdocforward{cdocsamp}"|\\
% |latex -jobname cdocscl1 \|\\
% |  "\input{childdoc.def}\childdocforward[cdocsamp]{cdocsch1}"|\\
% |latex -jobname cdocscl2 \|\\
% |  "\def\version{final}\input{childdoc.def}\childdocforward{cdocsch2}"|
% \end{tabular}
% \end{center}
% Note that the trailing backslash on each first line
% merely continues the input to the second line
% (for convenient cut ant paste).
% Furthermore, the command |latex| can be replaced by any
% of its alternative versions such as |pdflatex|.
%
% %%%%%%%%%%%%%%%%%%%%%%%%%%%%%%%%%%%%%%%%%%%%%%%%%%%%%%%%%%%%%%%%%%%%%%%%%%%%%%
% %%%%%%%%%%%%%%%%%%%%%%%%%%%%%%%%%%%%%%%%%%%%%%%%%%%%%%%%%%%%%%%%%%%%%%%%%%%%%%
% \section{Implementation}
%\iffalse
%<*package>
%\fi
%
% This section describes the definitions file |childdoc.def|.

% The definitions cannot be loaded using |\usepackage| or |\RequirePackage|
% which has a mechanism to prevent loading a style file more than once.
% When loading the definitions by means of |\input|
% multiple instances have to be prevented manually:
%\iffalse
%This code needs to be before the `\ProvidesFile' directive
%which is defined at the beginning of this file.
%Therefore it is also placed there and commented out here.
%</package>
%<*discard>
%\fi
%    \begin{macrocode}
\ifdefined\childdocmain\endinput\fi
%    \end{macrocode}
%\iffalse
%</discard>
%<*package>
%\fi
%
% \macro{\ifchilddoc}
% \macro{\ifchilddocmanual}
% The conditional |\ifchilddoc| tells whether a
% child (true) or main (false) document is being compiled.
% The conditional |\ifchilddocmanual| tells whether
% the |\includeonly| mechanism is used (false) or
% the selection of child files must be performed manually (true).
% The definitions initialise to false:
%    \begin{macrocode}
\newif\ifchilddoc
\newif\ifchilddocmanual
%    \end{macrocode}

% \macro{\childdocname}
% \macro{\childdocjob}
% The macro |\childdocname| stores the name of the main document
% to be compiled. The macro |\childdocjob| stores the name of
% the document on which the \LaTeX{} compiler was originally invoked.
% The content of |\jobname| cannot be compared
% to filenames specified in the source due to different catcodes.
% The following code rescans |\jobname|, stores the result
% in |\childdocname| and saves a copy in |\childdocjob|:
%    \begin{macrocode}
\edef\childdocname{\scantokens\expandafter{\jobname\noexpand}}
\let\childdocjob\childdocname
%    \end{macrocode}

% \macro{\childdocdisable}
% The macro |\childdocdisable| prevents the main file
% from being processed more than once.
% At this stage, the main document command |\childdocmain|
% is assumed to be called once again where it should do nothing.
% Any subsequent call to it should prevent
% a secondary processing of the main document
% It overwrites the forwarding commands
% |\childdocof| and |\childdocforward|
% with empty macros to prevent further inclusions of the main document:
%    \begin{macrocode}
\newcommand{\childdocdisable}
{
  \renewcommand{\childdocmain}[1]{\renewcommand{\childdocmain}[1]{\endinput}}
  \renewcommand{\childdocof}[1]{}
  \renewcommand{\childdocby}[2][]{}
  \renewcommand{\childdocforward}[2][]{}
  \renewcommand{\childdocdisable}{}
}
%    \end{macrocode}

% \macro{\childdocmain}
% The macro |\childdocmain| is to be called at the top of the main file
% with nothing or the main filename (without extension) as argument.
% First, it breaks loops.
% If the argument is not empty and does not match |\childdocname|
% (which is set by the first inclusion of |childdoc.def|),
% |\ifchilddoc| is set to true, |\includeonly| is applied to the child file
% and |\jobname| is set to the main file
% (for proper handling of |.aux| files):
%    \begin{macrocode}
\newcommand{\childdocmain}[1]
{
  \childdocdisable\childdocmain{}
  \if?#1?\else
    \begingroup
      \def\childdoctmp{#1}
      \ifx\childdoctmp\childdocname
        \def\childdoctmp{}
      \else
        \def\childdoctmp
        {
          \childdoctrue
          \includeonly{\childdocname}
          \def\childdocjob{#1}
          \def\jobname{#1}
        }
      \fi
      \expandafter
    \endgroup
    \childdoctmp
  \fi
}
%    \end{macrocode}

% \macro{\childdocof}
% The command |\childdocof| redirects
% compilation to the main file |#1|.
%    \begin{macrocode}
\newcommand{\childdocof}[1]
{
  \childdocdisable
  \childdoctrue
  \includeonly{\childdocname}
  \def\jobname{#1}
  \def\childdocjob{#1}
  \input{#1}
}
%    \end{macrocode}

% \macro{\childdocby}
% The command |\childdocby| ....
%    \begin{macrocode}
\newcommand{\childdocby}[2][]
{
  \childdocdisable
  \childdoctrue
  \childdocmanualtrue
  \if?#1?\else
    \def\jobname{#2}
  \fi
  \def\childdocjob{#2}
  \input{#2}
  \endinput
}
%    \end{macrocode}

% \macro{\childdocforward}
% The command |\childdocforward| redirects
% compilation to the main file or
% (if the optional argument is given) a child file.
% Parameters are set as if the main file
% or a child file starting with |\childdocof| was compiled.
% Then compilation is handed over to the main file:
%    \begin{macrocode}
\newcommand{\childdocforward}[2][]
{
  \begingroup
    \if?#1?
      \def\childdoctmp
      {
        \def\childdocname{#2}
        \def\childdocjob{#2}
        \def\jobname{#2}
        \input{#2}
        \endinput
      }
    \else
      \def\childdoctmp
      {
        \childdocdisable
        \def\childdocname{#2}
        \childdoctrue
        \includeonly{#2}
        \def\childdocjob{#1}
        \def\jobname{#1}
        \input{#1}
        \endinput
      }
    \fi
    \expandafter
  \endgroup
  \childdoctmp
}
%    \end{macrocode}

% \macro{\childdocforwardprefix}
% The command |\childdocforwardprefix| redirects
% compilation to the main or a child file by means of a pattern.
% The prefix |#1| in the current filename is replaced by |#2|
% and the suffix of the current filename is kept
% (it is assumed that the filename does not contain the substring `|~~~|'
% which is used as a delimiter).
% Compilation is handed over to the new file by |\childdocforward|:
%    \begin{macrocode}
\newcommand{\childdocforwardprefix}[3][]
{
  \begingroup
    \def\childdocextract #2##1~~~{\def\childdoctmp{\childdocforward[#1]{#3##1}}}
    \expandafter\childdocextract\childdocname~~~
    \expandafter
  \endgroup
  \childdoctmp
}
%    \end{macrocode}

% \macro{\childdoc}
% The deprecated macro |\childdoc| is a legacy version of |\childdocmain|:
%    \begin{macrocode}
\newcommand{\childdoc}{\childdocmain}
%    \end{macrocode}

% \macro{\childdocredirect}
% The deprecated macro |\childdocredirect| is a legacy version
% of |\childdocforward| and |\childdocforwardprefix|:
%    \begin{macrocode}
\newcommand{\childdocredirect}[2][]
{
  \begingroup
    \if?#1?
      \def\childdoctmp{\childdocforward{#2}}
    \else
      \def\childdoctmp{\childdocforwardprefix{#1}{#2}}
    \fi
    \expandafter
  \endgroup
  \childdoctmp
}
%    \end{macrocode}

%\iffalse
%</package>
%\fi
%
\endinput
|\\
|\childdocof{|\textit{main}|}|\\
\end{tabular}
\end{center}
at the top of every child file \textit{child}
which is included by |\include{|\textit{child}|}|
from within the main file
(or at least for those files to be compiled individually).
The argument \textit{main} must be the filename of the main file.

There are a couple of
considerations in setting up the main and child documents:

%%%%%%%%%%%%%%%%%%%%%%%%%%%%%%%%%%%%%%%%
\paragraph{Restrictions.}

Please note the following restrictions:
\begin{itemize}
\item
|\childdocmain| must be called with one argument \textit{main}
to ensure compatibility with earlier version of the package.
It must either be empty (|\childdocmain{}|)
or precisely match the filename of the main file in which it is specified.
See \secref{sec:detection} for further information.
\item
The filename \textit{main} must be specified without the |.tex| extension.
\item
The filename \textit{main} is case sensitive
(even in case-insensitive file systems)
due to internal string comparison.
\item
The argument \textit{main} should be fully expanded, it cannot be a macro.
\item
Subdirectories and special characters should be avoided in filenames.
\item
The command |\childdocmain{|\textit{main}|}| must be followed by a whitespace.
It should not be followed immediately by another command
or by a comment mark `|%|'.
This is because the \TeX{} parser reads the token immediately following
the argument of |\childdocmain| and puts it
at the beginning of every child section;
however, a white\-space is ignored.
\end{itemize}

%%%%%%%%%%%%%%%%%%%%%%%%%%%%%%%%%%%%%%%%
\paragraph{Content of Main File.}

It is advisable to place all content in the child files included by |\include|.
Any output contained in the main file will appear in all child documents
unless suppressed manually;
it cannot be suppressed automatically by the |\includeonly| directive
and thus should normally be avoided.
A method to include some content in the main file
by means of conditional processing is described in \secref{sec:conditional}.

%%%%%%%%%%%%%%%%%%%%%%%%%%%%%%%%%%%%%%%%
\paragraph{Page Numbering.}

When only a part of the document is compiled,
the appropriate numbering of pages
(as well as other status parameters)
is determined from the |.aux| files.
The latter contain information from previous passes.
However this information needs to propagate through
all intermediate child documents.
Therefore the page numbering in child documents may well
be inconsistent until the complete document is compiled at least once.

A useful (if unconventional) way to always ensure a consistent
page numbering is to restart the numbering in each child document
and denote the pages by `\textit{child}|.|\textit{page}'
where \textit{child} represents the chapter/section number of the child file.
This can be achieved by the command
|\numberwithin{page}{|\textit{child}|}|
of the \textsf{amsmath} package
where \textit{child} can be |chapter| or |section|
depending on the chosen structuring.
Alternatively, one can modify the macro |\thepage| appropriately
and reset the counter |page| at the start of each child file.

%%%%%%%%%%%%%%%%%%%%%%%%%%%%%%%%%%%%%%%%%%%%%%%%%%%%%%%%%%%%%%%%%%%%%%%%%%%%%%%%
\subsection{Conditional Processing}
\label{sec:conditional}

The package provides a mechanism to compile different versions
of a document. To customise the versions further some conditional processing
can come in handy to distinguish which version is being compiled.
The package provides two macros to describe the compilation context:

%%%%%%%%%%%%%%%%%%%%%%%%%%%%%%%%%%%%%%%%
\DescribeMacro{\ifchilddoc}
The conditional |\ifchilddoc| distinguishes between the compilation of
child documents and the main document:
%
\begin{center}
|\ifchilddoc |\textit{child-code}| |[|\||else |\textit{main-code}]| \||fi|
\end{center}

%%%%%%%%%%%%%%%%%%%%%%%%%%%%%%%%%%%%%%%%
\DescribeMacro{\childdocname}
\DescribeMacro{\childdocjob}
The macro |\childdocname| contains the filename (without extension)
of the main or child file being processed.
Note that |\childdocjob| will always contain the name of the main file.

%%%%%%%%%%%%%%%%%%%%%%%%%%%%%%%%%%%%%%%%
\paragraph{Title Page.}

Conditional processing can be used to include a title or banner page
in the main document when proper precautions are taken.
Importantly, the code in the main file should ensure that the page counter
(as well as other status parameters which are stored in the |.aux| files)
takes the same value after the conditional processing.
Otherwise the page numbers may take divergent values
depending on which part is compiled.

For example, a title page could be declared by:
%
\begin{center}
\begin{tabular}{l}
|\ifchilddoc\||else|\\
|\addtocounter{page}{-1}|\\
\textit{code for title page}\\
|\newpage|\\
|\||fi|
\end{tabular}
\end{center}
%
A banner page for the child documents can be generated by:
%
\begin{center}
\begin{tabular}{l}
|\ifchilddoc|\\
|\addtocounter{page}{-1}|\\
\textit{code for banner page}\\
|\newpage|\\
|\||fi|
\end{tabular}
\end{center}
%
Here one could write a message such as:
\begin{center}
|This is the part \childdocname{} of \childdocjob{}.|
\end{center}

%%%%%%%%%%%%%%%%%%%%%%%%%%%%%%%%%%%%%%%%%%%%%%%%%%%%%%%%%%%%%%%%%%%%%%%%%%%%%%%%
\subsection{Flags}
\label{sec:flags}

The package makes it easy to generate different versions
of the main or child documents.
To this end compilation flags can be defined
and assigned different default values.
They will be particularly useful in conjunction
with the forwarding mechanism described in \secref{sec:forward}.

For example, it may be useful to have a flag |\version|
which can be set to |draft| or |final|.
The document source will contain some conditional code
depending on the value of |\version|.
Suppose further, the flag should default to |final| for the main file
and to |draft| for child files
which is a natural assignment for editing the document.
This is achieved by placing the following code
in the preamble of the main document
(below the |\childdocmain| directive):
%
\begin{center}
\begin{tabular}{l}
|\ifchilddoc|\\
|\providecommand{\version}{draft}|\\
|\||else|\\
|\providecommand{\version}{final}|\\
|\||fi|
\end{tabular}
\end{center}
%
The definition by |\providecommand| makes sure
that previous definitions are not overwritten.
Further statements |\providecommand{\version}{...}|
can thus be added before the above code to override it.

For the main file, one might add a line
(between |\childdocmain| and the above block)
%
\begin{center}
|%\ifchilddoc\||else\providecommand{\version}{draft}\||fi|
\end{center}
%
which can be uncommented to produce a draft version.
Likewise one can add a line to the very top of a child file
(above the |\childdocof{|\textit{main}|}| directive)
%
\begin{center}
|%\providecommand{\version}{final}|
\end{center}
%
which can be uncommented to produce the final version of this child document.

%%%%%%%%%%%%%%%%%%%%%%%%%%%%%%%%%%%%%%%%%%%%%%%%%%%%%%%%%%%%%%%%%%%%%%%%%%%%%%%%
\subsection{Forwarding}
\label{sec:forward}

Different versions of the main or child documents
using compilation flags as described in \secref{sec:flags}
can be (permanently) stored in different files
for convenient compilation, viewing and distribution.
To this end, the package defines a command
to pass on compilation to a different file:

%%%%%%%%%%%%%%%%%%%%%%%%%%%%%%%%%%%%%%%%
\DescribeMacro{\childdocforward}
The command |\childdocforward| redirects processing to
another source file:
%
\begin{center}
\begin{tabular}{l}
|% \iffalse
%
% childdoc.dtx Copyright (C) 2017-2018 Niklas Beisert
%
% This work may be distributed and/or modified under the
% conditions of the LaTeX Project Public License, either version 1.3
% of this license or (at your option) any later version.
% The latest version of this license is in
%   http://www.latex-project.org/lppl.txt
% and version 1.3 or later is part of all distributions of LaTeX
% version 2005/12/01 or later.
%
% This work has the LPPL maintenance status `maintained'.
%
% The Current Maintainer of this work is Niklas Beisert.
%
% This work consists of the files childdoc.dtx and childdoc.ins
% and the derived files childdoc.def and cdocsamp.tex with
% cdocsch1.tex, cdocsch2.tex, cdocsdrf.tex, cdocsfn1.tex, cdocsfn2.tex.
%
%<package>\ifdefined\childdocmain\endinput\fi
%<package>\ProvidesFile{childdoc.def}[2018/12/30 v2.0 child document driver]
%<samplemain>\ProvidesFile{cdocsamp.tex}[2018/12/30 v2.0 sample for childdoc]
%<*driver>
%\ProvidesFile{childdoc.drv}[2018/12/30 v2.0 childdoc reference manual file]
\PassOptionsToClass{10pt,a4paper}{article}
\documentclass{ltxdoc}

\usepackage[margin=35mm]{geometry}
\usepackage{hyperref}
\usepackage{hyperxmp}
\usepackage[usenames]{color}

\hypersetup{colorlinks=true}
\hypersetup{pdfstartview=FitH}
\hypersetup{pdfpagemode=UseNone}
\hypersetup{pdfsource={}}
\hypersetup{pdflang={en-UK}}
\hypersetup{pdfcopyright={Copyright 2017-2018 Niklas Beisert.
  This work may be distributed and/or modified under the
  conditions of the LaTeX Project Public License, either version 1.3
  of this license or (at your option) any later version.}}
\hypersetup{pdflicenseurl={http://www.latex-project.org/lppl.txt}}
\hypersetup{pdfcontactaddress={ETH Zurich, ITP, HIT K,
  Wolfgang-Pauli-Strasse 27}}
\hypersetup{pdfcontactpostcode={8093}}
\hypersetup{pdfcontactcity={Zurich}}
\hypersetup{pdfcontactcountry={Switzerland}}
\hypersetup{pdfcontactemail={nbeisert@itp.phys.ethz.ch}}
\hypersetup{pdfcontacturl={http://people.phys.ethz.ch/\xmptilde nbeisert/}}

\newcommand{\secref}[1]{\hyperref[#1]{section \ref*{#1}}}

\parskip1ex
\parindent0pt
\let\olditemize\itemize
\def\itemize{\olditemize\parskip0pt}

\begin{document}

\title{The \textsf{childdoc} Package}
\hypersetup{pdftitle={The childdoc Package}}
\author{Niklas Beisert\\[2ex]
  Institut f\"ur Theoretische Physik\\
  Eidgen\"ossische Technische Hochschule Z\"urich\\
  Wolfgang-Pauli-Strasse 27, 8093 Z\"urich, Switzerland\\[1ex]
  \href{mailto:nbeisert@itp.phys.ethz.ch}
  {\texttt{nbeisert@itp.phys.ethz.ch}}}
\hypersetup{pdfauthor={Niklas Beisert}}
\hypersetup{pdfsubject={Manual for the LaTeX2e Package childdoc}}
\date{30 December 2018, \textsf{v2.0}}
\maketitle

\begin{abstract}\noindent
\textsf{childdoc} is a \LaTeXe{} package
that enables the direct compilation
of document sections included by |\include|
to individual files.
\end{abstract}

\begingroup
\parskip0ex
\tableofcontents
\endgroup

%%%%%%%%%%%%%%%%%%%%%%%%%%%%%%%%%%%%%%%%%%%%%%%%%%%%%%%%%%%%%%%%%%%%%%%%%%%%%%%%
%%%%%%%%%%%%%%%%%%%%%%%%%%%%%%%%%%%%%%%%%%%%%%%%%%%%%%%%%%%%%%%%%%%%%%%%%%%%%%%%
\section{Introduction}

\LaTeX{} provides a mechanism to structure a large document (such as a book)
into a main file and several child files (containing the chapters)
using the |\include| command.
This mechanism is beneficial for documents
which span hundreds of pages in order to
make the source file(s) more manageable.
Moreover, compilation can be restricted to
selected child files by means of the |\includeonly| command.
The latter feature can be used to reduce the compilation time while editing
(this was significantly more useful in the earlier days of \LaTeX{})
or to generate a smaller document which is easier to navigate.
Another application of |\includeonly| is to generate
documents consisting of selected parts of the complete document.

However, there are a few drawbacks of the plain |\include| mechanism:
\begin{itemize}
\item
The child files cannot be compiled on their own,
they can only be compiled via the main file.
A naive editing environment
(such as a text editor with an option
to have the current file processed by \LaTeX)
may require one to switch to the main file before compiling;
attempting to compile the child file produces errors.
\item
The main file must be modified (each time)
to adjust the |\includeonly| command
to the present needs. This easily leaves the main file in a messy state.
\item
The generated document will always carry the filename
of the main document. This is inconvenient if
several child files are to be compiled and
to be kept for distribution.
\end{itemize}

The present package provides a simple interface
to make child files individually compilable by \LaTeX{}.
Compiling a child file then has the same effect as compiling
the main file with an |\includeonly| command
to select the appropriate child.
Moreover the generated document will carry the name of the child
rather than the main file.
This resolves all three above issues.

This feature is meant to make the editing of books,
thesis documents and lecture notes somewhat more convenient.
However, the package can also be used efficiently for
composing a series of documents (such as exercise sheets)
which are typically distributed individually.
It then assists the author in generating the individual documents
(potentially in different versions)
as well as a document containing the collected series.
Another application is in developing style files
or other kinds of included material
where compilation of the style file could redirect
to a sample or test file.

%%%%%%%%%%%%%%%%%%%%%%%%%%%%%%%%%%%%%%%%%%%%%%%%%%%%%%%%%%%%%%%%%%%%%%%%%%%%%%%%
%%%%%%%%%%%%%%%%%%%%%%%%%%%%%%%%%%%%%%%%%%%%%%%%%%%%%%%%%%%%%%%%%%%%%%%%%%%%%%%%
\section{Usage}

First of all, the package \textsf{childdoc} is \emph{not} a standard
\LaTeXe{} |.sty| style file! Therefore it needs to be invoked in
a non-standard way.

%%%%%%%%%%%%%%%%%%%%%%%%%%%%%%%%%%%%%%%%%%%%%%%%%%%%%%%%%%%%%%%%%%%%%%%%%%%%%%%%
\subsection{Included Files}
\label{sec:include}

%%%%%%%%%%%%%%%%%%%%%%%%%%%%%%%%%%%%%%%%
\DescribeMacro{\childdocmain}
To use the package, add the commands
\begin{center}
\begin{tabular}{l}
|\input{childdoc.def}|\\
|\childdocmain{}|\\
\end{tabular}
\end{center}
at the very top of the main \LaTeX{} file,
in particular \emph{before} the |\documentclass| statement!
The argument of |\childdocmain| should be left empty
(but it must be present).

%%%%%%%%%%%%%%%%%%%%%%%%%%%%%%%%%%%%%%%%
\DescribeMacro{\childdocof}
Furthermore, add the commands
\begin{center}
\begin{tabular}{l}
|\input{childdoc.def}|\\
|\childdocof{|\textit{main}|}|\\
\end{tabular}
\end{center}
at the top of every child file \textit{child}
which is included by |\include{|\textit{child}|}|
from within the main file
(or at least for those files to be compiled individually).
The argument \textit{main} must be the filename of the main file.

There are a couple of
considerations in setting up the main and child documents:

%%%%%%%%%%%%%%%%%%%%%%%%%%%%%%%%%%%%%%%%
\paragraph{Restrictions.}

Please note the following restrictions:
\begin{itemize}
\item
|\childdocmain| must be called with one argument \textit{main}
to ensure compatibility with earlier version of the package.
It must either be empty (|\childdocmain{}|)
or precisely match the filename of the main file in which it is specified.
See \secref{sec:detection} for further information.
\item
The filename \textit{main} must be specified without the |.tex| extension.
\item
The filename \textit{main} is case sensitive
(even in case-insensitive file systems)
due to internal string comparison.
\item
The argument \textit{main} should be fully expanded, it cannot be a macro.
\item
Subdirectories and special characters should be avoided in filenames.
\item
The command |\childdocmain{|\textit{main}|}| must be followed by a whitespace.
It should not be followed immediately by another command
or by a comment mark `|%|'.
This is because the \TeX{} parser reads the token immediately following
the argument of |\childdocmain| and puts it
at the beginning of every child section;
however, a white\-space is ignored.
\end{itemize}

%%%%%%%%%%%%%%%%%%%%%%%%%%%%%%%%%%%%%%%%
\paragraph{Content of Main File.}

It is advisable to place all content in the child files included by |\include|.
Any output contained in the main file will appear in all child documents
unless suppressed manually;
it cannot be suppressed automatically by the |\includeonly| directive
and thus should normally be avoided.
A method to include some content in the main file
by means of conditional processing is described in \secref{sec:conditional}.

%%%%%%%%%%%%%%%%%%%%%%%%%%%%%%%%%%%%%%%%
\paragraph{Page Numbering.}

When only a part of the document is compiled,
the appropriate numbering of pages
(as well as other status parameters)
is determined from the |.aux| files.
The latter contain information from previous passes.
However this information needs to propagate through
all intermediate child documents.
Therefore the page numbering in child documents may well
be inconsistent until the complete document is compiled at least once.

A useful (if unconventional) way to always ensure a consistent
page numbering is to restart the numbering in each child document
and denote the pages by `\textit{child}|.|\textit{page}'
where \textit{child} represents the chapter/section number of the child file.
This can be achieved by the command
|\numberwithin{page}{|\textit{child}|}|
of the \textsf{amsmath} package
where \textit{child} can be |chapter| or |section|
depending on the chosen structuring.
Alternatively, one can modify the macro |\thepage| appropriately
and reset the counter |page| at the start of each child file.

%%%%%%%%%%%%%%%%%%%%%%%%%%%%%%%%%%%%%%%%%%%%%%%%%%%%%%%%%%%%%%%%%%%%%%%%%%%%%%%%
\subsection{Conditional Processing}
\label{sec:conditional}

The package provides a mechanism to compile different versions
of a document. To customise the versions further some conditional processing
can come in handy to distinguish which version is being compiled.
The package provides two macros to describe the compilation context:

%%%%%%%%%%%%%%%%%%%%%%%%%%%%%%%%%%%%%%%%
\DescribeMacro{\ifchilddoc}
The conditional |\ifchilddoc| distinguishes between the compilation of
child documents and the main document:
%
\begin{center}
|\ifchilddoc |\textit{child-code}| |[|\||else |\textit{main-code}]| \||fi|
\end{center}

%%%%%%%%%%%%%%%%%%%%%%%%%%%%%%%%%%%%%%%%
\DescribeMacro{\childdocname}
\DescribeMacro{\childdocjob}
The macro |\childdocname| contains the filename (without extension)
of the main or child file being processed.
Note that |\childdocjob| will always contain the name of the main file.

%%%%%%%%%%%%%%%%%%%%%%%%%%%%%%%%%%%%%%%%
\paragraph{Title Page.}

Conditional processing can be used to include a title or banner page
in the main document when proper precautions are taken.
Importantly, the code in the main file should ensure that the page counter
(as well as other status parameters which are stored in the |.aux| files)
takes the same value after the conditional processing.
Otherwise the page numbers may take divergent values
depending on which part is compiled.

For example, a title page could be declared by:
%
\begin{center}
\begin{tabular}{l}
|\ifchilddoc\||else|\\
|\addtocounter{page}{-1}|\\
\textit{code for title page}\\
|\newpage|\\
|\||fi|
\end{tabular}
\end{center}
%
A banner page for the child documents can be generated by:
%
\begin{center}
\begin{tabular}{l}
|\ifchilddoc|\\
|\addtocounter{page}{-1}|\\
\textit{code for banner page}\\
|\newpage|\\
|\||fi|
\end{tabular}
\end{center}
%
Here one could write a message such as:
\begin{center}
|This is the part \childdocname{} of \childdocjob{}.|
\end{center}

%%%%%%%%%%%%%%%%%%%%%%%%%%%%%%%%%%%%%%%%%%%%%%%%%%%%%%%%%%%%%%%%%%%%%%%%%%%%%%%%
\subsection{Flags}
\label{sec:flags}

The package makes it easy to generate different versions
of the main or child documents.
To this end compilation flags can be defined
and assigned different default values.
They will be particularly useful in conjunction
with the forwarding mechanism described in \secref{sec:forward}.

For example, it may be useful to have a flag |\version|
which can be set to |draft| or |final|.
The document source will contain some conditional code
depending on the value of |\version|.
Suppose further, the flag should default to |final| for the main file
and to |draft| for child files
which is a natural assignment for editing the document.
This is achieved by placing the following code
in the preamble of the main document
(below the |\childdocmain| directive):
%
\begin{center}
\begin{tabular}{l}
|\ifchilddoc|\\
|\providecommand{\version}{draft}|\\
|\||else|\\
|\providecommand{\version}{final}|\\
|\||fi|
\end{tabular}
\end{center}
%
The definition by |\providecommand| makes sure
that previous definitions are not overwritten.
Further statements |\providecommand{\version}{...}|
can thus be added before the above code to override it.

For the main file, one might add a line
(between |\childdocmain| and the above block)
%
\begin{center}
|%\ifchilddoc\||else\providecommand{\version}{draft}\||fi|
\end{center}
%
which can be uncommented to produce a draft version.
Likewise one can add a line to the very top of a child file
(above the |\childdocof{|\textit{main}|}| directive)
%
\begin{center}
|%\providecommand{\version}{final}|
\end{center}
%
which can be uncommented to produce the final version of this child document.

%%%%%%%%%%%%%%%%%%%%%%%%%%%%%%%%%%%%%%%%%%%%%%%%%%%%%%%%%%%%%%%%%%%%%%%%%%%%%%%%
\subsection{Forwarding}
\label{sec:forward}

Different versions of the main or child documents
using compilation flags as described in \secref{sec:flags}
can be (permanently) stored in different files
for convenient compilation, viewing and distribution.
To this end, the package defines a command
to pass on compilation to a different file:

%%%%%%%%%%%%%%%%%%%%%%%%%%%%%%%%%%%%%%%%
\DescribeMacro{\childdocforward}
The command |\childdocforward| redirects processing to
another source file:
%
\begin{center}
\begin{tabular}{l}
|\input{childdoc.def}|\\
|\childdocforward[|\textit{main}|]{|\textit{dest}|}|\\
\end{tabular}
\end{center}
%
The argument \textit{dest} is the destination file
(without extension).
It should be the main file or one of the child files.
Note that further \textsf{childdoc} directives
such as |\childdocof| and |\childdocforward|
in the indicated file will be processed in this form.
The optional argument \textit{main}
passes on directly to the main file \textit{main}
while pretending to compile the child \textit{dest}.
This form behaves as if \textit{dest}
issues |\childdocof{|\textit{main}|}| right away,
and no further \textsf{childdoc} directives will be processed.

%%%%%%%%%%%%%%%%%%%%%%%%%%%%%%%%%%%%%%%%
\DescribeMacro{\...prefix}
In the alternative form |\childdocforwardprefix|,
%
\begin{center}
\begin{tabular}{l}
|\input{childdoc.def}|\\
|\childdocforwardprefix[|\textit{main}|]{|\textit{prefix}|}{|\textit{dest}|}|
\end{tabular}
\end{center}
%
the destination file is determined by a pattern
depending on the current file:
To make this work, the current file must be called
`{\textit{prefix}\hspace{0.2em}\textit{suffix}}'
with \textit{prefix} matching precisely the argument.
Processing is then passed on to the file
`{\textit{dest}\hspace{0.2em}\textit{suffix}}'.
Surely, the same effect is achieved by
directly specifying the
argument `{\textit{dest}\hspace{0.2em}\textit{suffix}}'
in the first form.
However, that requires to set up a different file
for each child. With the alternative form of the command
all these files can have exactly the same content
which simplifies setting them up and maintaining them.

For example, the following file |draft.tex|
with a compilation flag |\version| as described in \secref{sec:flags}
compiles the main document as a draft:
%
\begin{center}
\begin{tabular}{l}
|\def\version{draft}|\\
|\input{childdoc.def}|\\
|\childdocforward{|\textit{main}|}|
\end{tabular}
\end{center}
%
Likewise, the following files |final|\textit{nn}|.tex|
compile the final version of the child document
|child|\textit{nn}|.tex|:
%
\begin{center}
\begin{tabular}{l}
|\def\version{final}|\\
|\input{childdoc.def}|\\
|\childdocforwardprefix{final}{child}|
\end{tabular}
\end{center}
%

Note that when several versions of a main file and/or of each child file
are to be generated, it may be convenient to set up a |Makefile| or
shell script to automatise the process.

%%%%%%%%%%%%%%%%%%%%%%%%%%%%%%%%%%%%%%%%%%%%%%%%%%%%%%%%%%%%%%%%%%%%%%%%%%%%%%%%
\subsection{Command Line Processing}
\label{sec:commandline}

The effect of redirection files can also be achieved by invoking
the \LaTeX{} compiler with a more elaborate command line.
Most conveniently this should be done as part
of a shell script or a |Makefile|.

When using \textsf{childdoc} in the main file, the following
command lines effectively perform a redirection
(note that depending on the shell being used,
backslashes may have to be doubled: `|\|' $\to$ `|\\|'):
%
\begin{center}
|... -jobname "|\textit{target}|" |\\|"|[\textit{flags}]%
|\input{childdoc.def}\childdocforward[|\textit{main}|]{|\textit{dest}|}"|
\end{center}
%
Here \textit{target} is the name of the output file,
\textit{main} is the name of the main file
and \textit{dest} is the name of the main or child file to be processed
(all filenames without extensions).
The optional argument \textit{main} can be omitted
if \textit{main} matches \textit{dest}.
Optionally, compilation \textit{flags} can be defined via |\def| commands.
This command line makes the \TeX{} engine believe
it is compiling the file \textit{target}
whose content is specified as the latter parameter.
The provided code then forwards the processing to
\textit{main} or \textit{dest} as described in \secref{sec:forward}.

%%%%%%%%%%%%%%%%%%%%%%%%%%%%%%%%%%%%%%%%%%%%%%%%%%%%%%%%%%%%%%%%%%%%%%%%%%%%%%%%
\subsection{Include by Input}
\label{sec:input}

Including child documents by |\include| has some restrictions by design.
Most notably, the content of a child document always occupies
its own set of pages; pages cannot be shared between child documents.
Usually, this behaviour makes perfect sense
because each child document contain an essential part of the document.
However, in some situations it may be desirable to compose
a document from a collection of parts
without having mandatory page breaks between then.
For this case, the package
provides a mechanism to include parts
by |\input| which can also be processed individually.
However, by construction this mechanism
requires manual handling of the content to be output.

%%%%%%%%%%%%%%%%%%%%%%%%%%%%%%%%%%%%%%%%
\DescribeMacro{\ifchilddocmanual}
The main file should be prepared as usual, see \secref{sec:include}.
However, the document body must make a distinction
between processing of an individual part and of the main document, e.g.:
%
\begin{center}
\begin{tabular}{l}
|\ifchilddocmanual|\\
|\input{\childdocname}|\\
|\||else|\\
\textit{document body with }|\input{|\textit{part}|}|\\
|\||fi|
\end{tabular}
\end{center}
%
The conditional |\ifchilddocmanual| is true whenever
a part to be included by |\input| is being compiled,
and the name of the part is stored in |\childdocname|.

%%%%%%%%%%%%%%%%%%%%%%%%%%%%%%%%%%%%%%%%
\DescribeMacro{\childdocby}
Each part to be included by |\input| should start with:
%
\begin{center}
\begin{tabular}{l}
|\input{childdoc.def}|\\
|\childdocby{|\textit{main}|}|\\
\end{tabular}
\end{center}
%
The directive |\childdocby| is similar to |\childdocof|
described in \secref{sec:include},
but the subsequent selection of content must be done manually.
To that end, both |\ifchilddoc| and |\ifchilddocmanual|
will be true upon processing of a part,
and the name of the part is stored in |\childdocname|.
Note that |\jobname| will be set to the filename of the current part
so that each part receives an individual |.aux| file
that does not interfere with the |.aux| file(s) of the main document.
This behaviour can be altered by the alternative form
|\childdocby[*]{|\textit{main}|}| (with a non-empty optional argument)
which uses the |.aux| file of the main document
by setting |\jobname| to \textit{main}.

%%%%%%%%%%%%%%%%%%%%%%%%%%%%%%%%%%%%%%%%%%%%%%%%%%%%%%%%%%%%%%%%%%%%%%%%%%%%%%%%
\subsection{Driver Development}
\label{sec:driver}

The \textsf{childdoc} mechanism can also be use for the development
of definition files such as \LaTeX{} styles or classes.
This case differs from the above setup with multiple parts
included by |\include| in that no |\includeonly| should be invoked.
This can be achieved by starting the include file
(before |\ProvidesPackage|) with:
%
\begin{center}
\begin{tabular}{l}
|\input{childdoc.def}|\\
|\childdocforward{|\textit{main}|}|\\
\end{tabular}
\end{center}
%
or alternatively with:
%
\begin{center}
\begin{tabular}{l}
|\input{childdoc.def}|\\
|\childdocby{|\textit{main}|}|\\
\end{tabular}
\end{center}
%
Both forms have slightly different effects as described above.
The main file is prepared as usual, see \secref{sec:include}.

%%%%%%%%%%%%%%%%%%%%%%%%%%%%%%%%%%%%%%%%%%%%%%%%%%%%%%%%%%%%%%%%%%%%%%%%%%%%%%%%
\subsection{Legacy Detection}
\label{sec:detection}

The directive |\childdocmain| in the main file can detect
whether the complete document or merely a child is to be compiled
even without using the directive |\childdocof|.
This method is deprecated because it is less robust
and there is no compelling reason to use it;
it is merely provided for backward compatibility
and it may be removed in future versions.

If the detection mechanism is to be used,
it is mandatory to correctly specify
the filename of the main file as the argument of |\childdocmain|:
%
\begin{center}
\begin{tabular}{l}
|\input{childdoc.def}|\\
|\childdocmain{|\textit{main}|}|\\
\end{tabular}
\end{center}
%
If |\jobname| does not match the argument \textit{main} of |\childdocmain|,
it is assumed that |\jobname| points to the child file to be compiled.
When using |\childdocmain| with the main file specified as argument,
it suffices to start a child file
with just |\input{|\textit{main}|}|
without loading of the package and using |\childdocof|.
If instead all processing is done
with the appropriate \textsf{childdoc} directives,
the argument of \textit{main} of |\childdocmain| can be empty.

An alternative version of the command line processing described
in \secref{sec:commandline} using the detection mechanism reads:
%
\begin{center}
|... -jobname "|\textit{target}|" "|[\textit{flags}]%
[|\def\jobname{|\textit{dest}|}|]|\input{|\textit{main}|}"|
\end{center}

%%%%%%%%%%%%%%%%%%%%%%%%%%%%%%%%%%%%%%%%%%%%%%%%%%%%%%%%%%%%%%%%%%%%%%%%%%%%%%%%
\subsection{Manual Code}
\label{sec:manual}

In case one cannot be certain whether the definitions file |childdoc.def|
is installed on the target \TeX{} distribution
and one prefers not to ship it,
it is conceivable to paste a few relevant commands into the sources.

To that end, drop all statements |\input{childdoc.def}|
and perform the replacements as outlined below.
Instead of |\childdocmain{|\textit{main}|}| add the following code
to the top of the main file:
%
\begin{center}
\begin{tabular}{l}
|\||ifdefined\childdocname\endinput\||fi\newif\ifchilddoc|\\
|\edef\childdocname{\scantokens\expandafter{\jobname\noexpand}}|\\
|\def\childdocmain{|\textit{main}|}\||ifx\childdocmain\childdocname\||else|\\
|\childdoctrue\includeonly{\childdocname}\let\jobname\childdocmain\||fi|\\
\end{tabular}
\end{center}
%
Instead of |\childdocof{|\textit{main}|}| just include the main file
at the top of each child file:
%
\begin{center}
|\input{|\textit{main}|}|
\end{center}
%
A simple redirection |\childdocforward{|\textit{dest}|}| is achieved by:
%
\begin{center}
|\def\jobname{|\textit{dest}|}\input{\jobname}|
\end{center}
%
The redirection with prefix
|\childdocforwardprefix[|\textit{prefix}|]{|\textit{dest}|}|
is accomplished by:
%
\begin{center}
\begin{tabular}{l}
|{\edef\jobname{\scantokens\expandafter{\jobname\noexpand}}|\\
|\def\redirectjob |\textit{prefix}|#1~~~{\gdef\jobname{|\textit{dest}|#1}}|\\
|\expandafter\redirectjob\jobname~~~}\input{\jobname}|
\end{tabular}
\end{center}

In an alternative approach,
child documents can be compiled by a specific command line
without additional code or specific definitions:
%
\begin{center}
|... -jobname "|\textit{target}|" "|[\textit{flags}]%
|\includeonly{|\textit{dest}|}\input{|\textit{main}|}"|
\end{center}
%

%%%%%%%%%%%%%%%%%%%%%%%%%%%%%%%%%%%%%%%%%%%%%%%%%%%%%%%%%%%%%%%%%%%%%%%%%%%%%%%%
%%%%%%%%%%%%%%%%%%%%%%%%%%%%%%%%%%%%%%%%%%%%%%%%%%%%%%%%%%%%%%%%%%%%%%%%%%%%%%%%
\section{Information}

%%%%%%%%%%%%%%%%%%%%%%%%%%%%%%%%%%%%%%%%%%%%%%%%%%%%%%%%%%%%%%%%%%%%%%%%%%%%%%%%
\subsection{Copyright}

Copyright \copyright{} 2017--2018 Niklas Beisert

This work may be distributed and/or modified under the
conditions of the \LaTeX{} Project Public License, either version 1.3
of this license or (at your option) any later version.
The latest version of this license is in
  \url{http://www.latex-project.org/lppl.txt}
and version 1.3 or later is part of all distributions of \LaTeX{}
version 2005/12/01 or later.

This work has the LPPL maintenance status `maintained'.

The Current Maintainer of this work is Niklas Beisert.

This work consists of the files |README.txt|, |childdoc.ins| and |childdoc.dtx|
as well as the derived files |childdoc.def|, |cdocsamp.tex|
with |cdocsch1.tex|, |cdocsch2.tex|, |cdocspt3.tex|, |cdocspt4.tex|,
|cdocsdrf.tex|, |cdocsfn1.tex|, |cdocsfn2.tex|
as well as |childdoc.pdf|.

%%%%%%%%%%%%%%%%%%%%%%%%%%%%%%%%%%%%%%%%%%%%%%%%%%%%%%%%%%%%%%%%%%%%%%%%%%%%%%%%
\subsection{Files and Installation}

The package consists of the files:
%
\begin{center}
\begin{tabular}{ll}
    |README.txt|   & readme file \\
    |childdoc.ins| & installation file \\
    |childdoc.dtx| & source file \\
    |childdoc.def| & definition file \\
    |cdocsamp.tex| & sample main file \\
    |cdocsch1.tex| & sample include file \\
    |cdocsch2.tex| & sample include file \\
    |cdocspt3.tex| & sample part file \\
    |cdocspt4.tex| & sample part file \\
    |cdocsdrf.tex| & sample redirection file \\
    |cdocsfn1.tex| & sample redirection file \\
    |cdocsfn2.tex| & sample redirection file \\
    |childdoc.pdf| & manual
\end{tabular}
\end{center}
%
The distribution consists of the files
|README.txt|, |childdoc.ins| and |childdoc.dtx|.
%
\begin{itemize}
\item
Run (pdf)\LaTeX{} on |childdoc.dtx|
to compile the manual |childdoc.pdf| (this file).
\item
Run \LaTeX{} on |childdoc.ins| to create the definitions file |childdoc.def|
and the sample |cdocsamp.tex| with include files
|cdocsch1.tex|, |cdocsch2.tex|, |cdocspt3.tex|, |cdocspt4.tex|,
|cdocsdrf.tex|, |cdocsfn1.tex|, |cdocsfn2.tex|.
Then copy the file |childdoc.def| to an appropriate directory of your \LaTeX{}
distribution, e.g.\ \textit{texmf-root}|/tex/latex/childdoc|.
\end{itemize}

%%%%%%%%%%%%%%%%%%%%%%%%%%%%%%%%%%%%%%%%%%%%%%%%%%%%%%%%%%%%%%%%%%%%%%%%%%%%%%%%
\subsection{Related CTAN Packages}

There are several other packages which offer a similar functionality:
%
\begin{itemize}
\item
The packages
\href{http://ctan.org/pkg/docmute}{\textsf{docmute}},
\href{http://ctan.org/pkg/includex}{\textsf{includex}} and
\href{http://ctan.org/pkg/standalone}{\textsf{standalone}}
provide commands to include only the document body of
a child file thus allowing both files to be compiled individually.
\item
The packages \href{http://ctan.org/pkg/subdocs}{\textsf{subdocs}}
and \href{http://ctan.org/pkg/subfiles}{\textsf{subfiles}}
provide structures in which the main and child documents can be
encapsulated and allowing them to be compiled individually.
The inclusion mechanism is different from the conventional |\include|.
\item
The package \href{http://ctan.org/pkg/combine}{\textsf{combine}}
is an elaborate solution to combine several documents into one.
\end{itemize}
%
See also the CTAN topic \href{http://ctan.org/topic/subdocs}{\textsf{subdocs}}
for further related packages.
The present package differs from the above solutions in that
a document structure constructed with the conventional |\include| mechanism
just needs two extra commands at the top of every file
such that all constituent files can be compiled individually.

%%%%%%%%%%%%%%%%%%%%%%%%%%%%%%%%%%%%%%%%%%%%%%%%%%%%%%%%%%%%%%%%%%%%%%%%%%%%%%%%
%\subsection{Feature Suggestions}
%
%The following is a list of features which may be useful for future
%versions of this package:
%%
%\begin{itemize}
%\item
%\ldots
%\end{itemize}

%%%%%%%%%%%%%%%%%%%%%%%%%%%%%%%%%%%%%%%%%%%%%%%%%%%%%%%%%%%%%%%%%%%%%%%%%%%%%%%%
\subsection{Revision History}

%%%%%%%%%%%%%%%%%%%%%%%%%%%%%%%%%%%%%%%%
\paragraph{v2.0:} 2018/12/30

\begin{itemize}
\item
immediate forward processing
\item
added |\childdocby| mechanism
\item
manual restructured
\end{itemize}

%%%%%%%%%%%%%%%%%%%%%%%%%%%%%%%%%%%%%%%%
\paragraph{v1.6:} 2018/01/17

\begin{itemize}
\item
application for development of include files
\item
corrections to manual
\end{itemize}

%%%%%%%%%%%%%%%%%%%%%%%%%%%%%%%%%%%%%%%%
\paragraph{v1.5:} 2017/05/21

\begin{itemize}
\item
more complete structuring introduced
\item
|\childdocof| introduced
\item
|\childdoc| renamed to |\childdocmain|
\item
|\childredirect| renamed to |\childdocforward| and |\childdocforwardprefix|
and functionality expanded
\end{itemize}

%%%%%%%%%%%%%%%%%%%%%%%%%%%%%%%%%%%%%%%%
\paragraph{v1.0:} 2017/04/27

\begin{itemize}
\item
manual and install package
\item
first version published on CTAN
\end{itemize}

%%%%%%%%%%%%%%%%%%%%%%%%%%%%%%%%%%%%%%%%
\paragraph{v0.6:} 2017/04/26

\begin{itemize}
\item
redirection mechanism added
\end{itemize}

%%%%%%%%%%%%%%%%%%%%%%%%%%%%%%%%%%%%%%%%
\paragraph{v0.5:} 2017/04/26

\begin{itemize}
\item
functionality in definition file
\end{itemize}


%%%%%%%%%%%%%%%%%%%%%%%%%%%%%%%%%%%%%%%%%%%%%%%%%%%%%%%%%%%%%%%%%%%%%%%%%%%%%%%%
%%%%%%%%%%%%%%%%%%%%%%%%%%%%%%%%%%%%%%%%%%%%%%%%%%%%%%%%%%%%%%%%%%%%%%%%%%%%%%%%
%%%%%%%%%%%%%%%%%%%%%%%%%%%%%%%%%%%%%%%%%%%%%%%%%%%%%%%%%%%%%%%%%%%%%%%%%%%%%%%%
\appendix

\settowidth\MacroIndent{\rmfamily\scriptsize 000\ }

 \DocInput{childdoc.dtx}

\end{document}
%</driver>
% \fi
%
% %%%%%%%%%%%%%%%%%%%%%%%%%%%%%%%%%%%%%%%%%%%%%%%%%%%%%%%%%%%%%%%%%%%%%%%%%%%%%%
% %%%%%%%%%%%%%%%%%%%%%%%%%%%%%%%%%%%%%%%%%%%%%%%%%%%%%%%%%%%%%%%%%%%%%%%%%%%%%%
% \section{Sample}
%\iffalse
%<*samplemain>
%\fi
%
% The following presents a sample document
% with two chapters, two parts, a title page,
% a compile flag as well as three forwarding files to set the flag.
% It consists of eight |.tex| files:
% \begin{center}
% \begin{tabular}{ll}
% |cdocsamp.tex|&main file\\
% |cdocsch1.tex|&include file for chapter 1\\
% |cdocsch2.tex|&include file for chapter 2\\
% |cdocspt3.tex|&include file for part 3\\
% |cdocspt4.tex|&include file for part 4\\
% |cdocsdrf.tex|&forwarding file for main file in draft mode\\
% |cdocsfi1.tex|&forwarding file for final version of chapter 1\\
% |cdocsfi2.tex|&forwarding file for final version of chapter 2\\
% \end{tabular}
% \end{center}
% Each of the eight files can be compiled directly by the \LaTeX{} compiler.
%
% %%%%%%%%%%%%%%%%%%%%%%%%%%%%%%%%%%%%%%
% \paragraph{Main File.}
%
% The main file is called |cdocsamp.tex|.
%
% Load the \textsf{childdoc} definitions and
% declare the filename for the main document:
%    \begin{macrocode}
\input{childdoc.def}
\childdocmain{}
%    \end{macrocode}

% Optional override for |\version| flag:
%    \begin{macrocode}
%%\ifchilddoc\else\providecommand{\version}{draft}\fi
%    \end{macrocode}

% Define the default values for the |\version| flag
% (|final| for the main file and |draft| for childs):
%    \begin{macrocode}
\ifchilddoc
\providecommand{\version}{draft}
\else
\providecommand{\version}{final}
\fi
%    \end{macrocode}

% Load the standard document class:
%    \begin{macrocode}
\documentclass[12pt]{article}
%    \end{macrocode}

% Start the document body:
%    \begin{macrocode}
\begin{document}
%    \end{macrocode}

% Declare a title page.
% Print title, part of document being processed and version flag:
%    \begin{macrocode}
\addtocounter{page}{-1}
\begin{center}
{\LARGE\bfseries{}childdoc example\par}
\vspace{1cm}
\ifchilddoc
\ifchilddocmanual part\else chapter\fi:
`\childdocname' of `\childdocjob'\par
\else
main document: `\childdocjob'\par
\fi
version: \version\par
\end{center}
\newpage
%    \end{macrocode}

% Manually include selected file,
% otherwise process as usual:
%    \begin{macrocode}
\ifchilddocmanual
\section*{part `\childdocname'}
\input{\childdocname}
\else
%    \end{macrocode}

% Include the two chapters:
%    \begin{macrocode}
\include{cdocsch1}
\include{cdocsch2}
%    \end{macrocode}

% Include the two parts unless only chapters should be displayed:
%    \begin{macrocode}
\ifchilddoc\else
\section{part three}
\input{cdocspt3}
\section{part four}
\input{cdocspt4}
\fi
%    \end{macrocode}

% Process as usual until here:
%    \begin{macrocode}
\fi
%    \end{macrocode}

% End of document body:
%    \begin{macrocode}
\end{document}
%    \end{macrocode}
%\iffalse
%</samplemain>
%\fi
%
% %%%%%%%%%%%%%%%%%%%%%%%%%%%%%%%%%%%%%%
% \paragraph{Chapter Include Files.}
%
% The include files are called |cdocsch1.tex| and |cdocsch2.tex|.
%
%\iffalse
%<*samplechap1|samplechap2>
%\fi

% Optional override for |\version| flag:
%    \begin{macrocode}
%%\providecommand{\version}{final}
%    \end{macrocode}

% Include the main document:
%    \begin{macrocode}
\input{childdoc.def}
\childdocof{cdocsamp}
%    \end{macrocode}

%\iffalse
%</samplechap1|samplechap2>
%\fi
%
%\iffalse
%<*samplechap1>
%\fi
% Some text for chapter 1:
%    \begin{macrocode}
\section{one}
some text in chapter one
%    \end{macrocode}

%\iffalse
%</samplechap1>
%\fi
% Some text for chapter 2:
%\iffalse
%<*samplechap2>
%\fi
%    \begin{macrocode}
\section{two}
more text in chapter two
%    \end{macrocode}

%\iffalse
%</samplechap2>
%\fi
%
% %%%%%%%%%%%%%%%%%%%%%%%%%%%%%%%%%%%%%%
% \paragraph{Part Include Files.}
%
% The include files are called |cdocspt3.tex| and |cdocspt4.tex|.
%
%\iffalse
%<*samplepart3|samplepart4>
%\fi

% Optional override for |\version| flag:
%    \begin{macrocode}
%%\providecommand{\version}{final}
%    \end{macrocode}

% Include the main document:
%    \begin{macrocode}
\input{childdoc.def}
\childdocby{cdocsamp}
%    \end{macrocode}

%\iffalse
%</samplepart3|samplepart4>
%\fi
%
%\iffalse
%<*samplepart3>
%\fi
% Some text for part 3:
%    \begin{macrocode}
some text in part three
%    \end{macrocode}

%\iffalse
%</samplepart3>
%\fi
% Some text for part 4:
%\iffalse
%<*samplepart4>
%\fi
%    \begin{macrocode}
more text in part four
%    \end{macrocode}

%\iffalse
%</samplepart4>
%\fi
%
% %%%%%%%%%%%%%%%%%%%%%%%%%%%%%%%%%%%%%%
% \paragraph{Forwarding for a Complete Draft.}
%
% The following forwarding file |cdocsdrf.tex|
% compiles the main document in draft mode:
%\iffalse
%<*sampledraft>
%\fi
%    \begin{macrocode}
\def\version{draft}
\input{childdoc.def}
\childdocforward{cdocsamp}
%    \end{macrocode}

%\iffalse
%</sampledraft>
%\fi
%
% %%%%%%%%%%%%%%%%%%%%%%%%%%%%%%%%%%%%%%
% \paragraph{Forwarding for Final Version of the Chapters.}
%
% The following forwarding files |cdocsfn1.tex| and |cdocsfn2.tex|
% (with identical content)
% compile the final versions of the child documents
% |cdocsch1.tex| and |cdocsch2.tex|, respectively:
%\iffalse
%<*samplefinal>
%\fi
%    \begin{macrocode}
\def\version{final}
\input{childdoc.def}
\childdocforwardprefix[cdocsamp]{cdocsfn}{cdocsch}
%    \end{macrocode}

%\iffalse
%</samplefinal>
%\fi
%
% %%%%%%%%%%%%%%%%%%%%%%%%%%%%%%%%%%%%%%
% \paragraph{Command Line Processing.}
%
% The following three command lines generate the output files
% |cdocscld|, |cdocscl1| and |cdocscl2|
% which should be identical to
% |cdocsdrf|, |cdocsch1| and |cdocsfn2|, respectively:
% \begin{center}
% \begin{tabular}{l}
% |latex -jobname cdocscld \|\\
% |  "\def\version{draft}\input{childdoc.def}\childdocforward{cdocsamp}"|\\
% |latex -jobname cdocscl1 \|\\
% |  "\input{childdoc.def}\childdocforward[cdocsamp]{cdocsch1}"|\\
% |latex -jobname cdocscl2 \|\\
% |  "\def\version{final}\input{childdoc.def}\childdocforward{cdocsch2}"|
% \end{tabular}
% \end{center}
% Note that the trailing backslash on each first line
% merely continues the input to the second line
% (for convenient cut ant paste).
% Furthermore, the command |latex| can be replaced by any
% of its alternative versions such as |pdflatex|.
%
% %%%%%%%%%%%%%%%%%%%%%%%%%%%%%%%%%%%%%%%%%%%%%%%%%%%%%%%%%%%%%%%%%%%%%%%%%%%%%%
% %%%%%%%%%%%%%%%%%%%%%%%%%%%%%%%%%%%%%%%%%%%%%%%%%%%%%%%%%%%%%%%%%%%%%%%%%%%%%%
% \section{Implementation}
%\iffalse
%<*package>
%\fi
%
% This section describes the definitions file |childdoc.def|.

% The definitions cannot be loaded using |\usepackage| or |\RequirePackage|
% which has a mechanism to prevent loading a style file more than once.
% When loading the definitions by means of |\input|
% multiple instances have to be prevented manually:
%\iffalse
%This code needs to be before the `\ProvidesFile' directive
%which is defined at the beginning of this file.
%Therefore it is also placed there and commented out here.
%</package>
%<*discard>
%\fi
%    \begin{macrocode}
\ifdefined\childdocmain\endinput\fi
%    \end{macrocode}
%\iffalse
%</discard>
%<*package>
%\fi
%
% \macro{\ifchilddoc}
% \macro{\ifchilddocmanual}
% The conditional |\ifchilddoc| tells whether a
% child (true) or main (false) document is being compiled.
% The conditional |\ifchilddocmanual| tells whether
% the |\includeonly| mechanism is used (false) or
% the selection of child files must be performed manually (true).
% The definitions initialise to false:
%    \begin{macrocode}
\newif\ifchilddoc
\newif\ifchilddocmanual
%    \end{macrocode}

% \macro{\childdocname}
% \macro{\childdocjob}
% The macro |\childdocname| stores the name of the main document
% to be compiled. The macro |\childdocjob| stores the name of
% the document on which the \LaTeX{} compiler was originally invoked.
% The content of |\jobname| cannot be compared
% to filenames specified in the source due to different catcodes.
% The following code rescans |\jobname|, stores the result
% in |\childdocname| and saves a copy in |\childdocjob|:
%    \begin{macrocode}
\edef\childdocname{\scantokens\expandafter{\jobname\noexpand}}
\let\childdocjob\childdocname
%    \end{macrocode}

% \macro{\childdocdisable}
% The macro |\childdocdisable| prevents the main file
% from being processed more than once.
% At this stage, the main document command |\childdocmain|
% is assumed to be called once again where it should do nothing.
% Any subsequent call to it should prevent
% a secondary processing of the main document
% It overwrites the forwarding commands
% |\childdocof| and |\childdocforward|
% with empty macros to prevent further inclusions of the main document:
%    \begin{macrocode}
\newcommand{\childdocdisable}
{
  \renewcommand{\childdocmain}[1]{\renewcommand{\childdocmain}[1]{\endinput}}
  \renewcommand{\childdocof}[1]{}
  \renewcommand{\childdocby}[2][]{}
  \renewcommand{\childdocforward}[2][]{}
  \renewcommand{\childdocdisable}{}
}
%    \end{macrocode}

% \macro{\childdocmain}
% The macro |\childdocmain| is to be called at the top of the main file
% with nothing or the main filename (without extension) as argument.
% First, it breaks loops.
% If the argument is not empty and does not match |\childdocname|
% (which is set by the first inclusion of |childdoc.def|),
% |\ifchilddoc| is set to true, |\includeonly| is applied to the child file
% and |\jobname| is set to the main file
% (for proper handling of |.aux| files):
%    \begin{macrocode}
\newcommand{\childdocmain}[1]
{
  \childdocdisable\childdocmain{}
  \if?#1?\else
    \begingroup
      \def\childdoctmp{#1}
      \ifx\childdoctmp\childdocname
        \def\childdoctmp{}
      \else
        \def\childdoctmp
        {
          \childdoctrue
          \includeonly{\childdocname}
          \def\childdocjob{#1}
          \def\jobname{#1}
        }
      \fi
      \expandafter
    \endgroup
    \childdoctmp
  \fi
}
%    \end{macrocode}

% \macro{\childdocof}
% The command |\childdocof| redirects
% compilation to the main file |#1|.
%    \begin{macrocode}
\newcommand{\childdocof}[1]
{
  \childdocdisable
  \childdoctrue
  \includeonly{\childdocname}
  \def\jobname{#1}
  \def\childdocjob{#1}
  \input{#1}
}
%    \end{macrocode}

% \macro{\childdocby}
% The command |\childdocby| ....
%    \begin{macrocode}
\newcommand{\childdocby}[2][]
{
  \childdocdisable
  \childdoctrue
  \childdocmanualtrue
  \if?#1?\else
    \def\jobname{#2}
  \fi
  \def\childdocjob{#2}
  \input{#2}
  \endinput
}
%    \end{macrocode}

% \macro{\childdocforward}
% The command |\childdocforward| redirects
% compilation to the main file or
% (if the optional argument is given) a child file.
% Parameters are set as if the main file
% or a child file starting with |\childdocof| was compiled.
% Then compilation is handed over to the main file:
%    \begin{macrocode}
\newcommand{\childdocforward}[2][]
{
  \begingroup
    \if?#1?
      \def\childdoctmp
      {
        \def\childdocname{#2}
        \def\childdocjob{#2}
        \def\jobname{#2}
        \input{#2}
        \endinput
      }
    \else
      \def\childdoctmp
      {
        \childdocdisable
        \def\childdocname{#2}
        \childdoctrue
        \includeonly{#2}
        \def\childdocjob{#1}
        \def\jobname{#1}
        \input{#1}
        \endinput
      }
    \fi
    \expandafter
  \endgroup
  \childdoctmp
}
%    \end{macrocode}

% \macro{\childdocforwardprefix}
% The command |\childdocforwardprefix| redirects
% compilation to the main or a child file by means of a pattern.
% The prefix |#1| in the current filename is replaced by |#2|
% and the suffix of the current filename is kept
% (it is assumed that the filename does not contain the substring `|~~~|'
% which is used as a delimiter).
% Compilation is handed over to the new file by |\childdocforward|:
%    \begin{macrocode}
\newcommand{\childdocforwardprefix}[3][]
{
  \begingroup
    \def\childdocextract #2##1~~~{\def\childdoctmp{\childdocforward[#1]{#3##1}}}
    \expandafter\childdocextract\childdocname~~~
    \expandafter
  \endgroup
  \childdoctmp
}
%    \end{macrocode}

% \macro{\childdoc}
% The deprecated macro |\childdoc| is a legacy version of |\childdocmain|:
%    \begin{macrocode}
\newcommand{\childdoc}{\childdocmain}
%    \end{macrocode}

% \macro{\childdocredirect}
% The deprecated macro |\childdocredirect| is a legacy version
% of |\childdocforward| and |\childdocforwardprefix|:
%    \begin{macrocode}
\newcommand{\childdocredirect}[2][]
{
  \begingroup
    \if?#1?
      \def\childdoctmp{\childdocforward{#2}}
    \else
      \def\childdoctmp{\childdocforwardprefix{#1}{#2}}
    \fi
    \expandafter
  \endgroup
  \childdoctmp
}
%    \end{macrocode}

%\iffalse
%</package>
%\fi
%
\endinput
|\\
|\childdocforward[|\textit{main}|]{|\textit{dest}|}|\\
\end{tabular}
\end{center}
%
The argument \textit{dest} is the destination file
(without extension).
It should be the main file or one of the child files.
Note that further \textsf{childdoc} directives
such as |\childdocof| and |\childdocforward|
in the indicated file will be processed in this form.
The optional argument \textit{main}
passes on directly to the main file \textit{main}
while pretending to compile the child \textit{dest}.
This form behaves as if \textit{dest}
issues |\childdocof{|\textit{main}|}| right away,
and no further \textsf{childdoc} directives will be processed.

%%%%%%%%%%%%%%%%%%%%%%%%%%%%%%%%%%%%%%%%
\DescribeMacro{\...prefix}
In the alternative form |\childdocforwardprefix|,
%
\begin{center}
\begin{tabular}{l}
|% \iffalse
%
% childdoc.dtx Copyright (C) 2017-2018 Niklas Beisert
%
% This work may be distributed and/or modified under the
% conditions of the LaTeX Project Public License, either version 1.3
% of this license or (at your option) any later version.
% The latest version of this license is in
%   http://www.latex-project.org/lppl.txt
% and version 1.3 or later is part of all distributions of LaTeX
% version 2005/12/01 or later.
%
% This work has the LPPL maintenance status `maintained'.
%
% The Current Maintainer of this work is Niklas Beisert.
%
% This work consists of the files childdoc.dtx and childdoc.ins
% and the derived files childdoc.def and cdocsamp.tex with
% cdocsch1.tex, cdocsch2.tex, cdocsdrf.tex, cdocsfn1.tex, cdocsfn2.tex.
%
%<package>\ifdefined\childdocmain\endinput\fi
%<package>\ProvidesFile{childdoc.def}[2018/12/30 v2.0 child document driver]
%<samplemain>\ProvidesFile{cdocsamp.tex}[2018/12/30 v2.0 sample for childdoc]
%<*driver>
%\ProvidesFile{childdoc.drv}[2018/12/30 v2.0 childdoc reference manual file]
\PassOptionsToClass{10pt,a4paper}{article}
\documentclass{ltxdoc}

\usepackage[margin=35mm]{geometry}
\usepackage{hyperref}
\usepackage{hyperxmp}
\usepackage[usenames]{color}

\hypersetup{colorlinks=true}
\hypersetup{pdfstartview=FitH}
\hypersetup{pdfpagemode=UseNone}
\hypersetup{pdfsource={}}
\hypersetup{pdflang={en-UK}}
\hypersetup{pdfcopyright={Copyright 2017-2018 Niklas Beisert.
  This work may be distributed and/or modified under the
  conditions of the LaTeX Project Public License, either version 1.3
  of this license or (at your option) any later version.}}
\hypersetup{pdflicenseurl={http://www.latex-project.org/lppl.txt}}
\hypersetup{pdfcontactaddress={ETH Zurich, ITP, HIT K,
  Wolfgang-Pauli-Strasse 27}}
\hypersetup{pdfcontactpostcode={8093}}
\hypersetup{pdfcontactcity={Zurich}}
\hypersetup{pdfcontactcountry={Switzerland}}
\hypersetup{pdfcontactemail={nbeisert@itp.phys.ethz.ch}}
\hypersetup{pdfcontacturl={http://people.phys.ethz.ch/\xmptilde nbeisert/}}

\newcommand{\secref}[1]{\hyperref[#1]{section \ref*{#1}}}

\parskip1ex
\parindent0pt
\let\olditemize\itemize
\def\itemize{\olditemize\parskip0pt}

\begin{document}

\title{The \textsf{childdoc} Package}
\hypersetup{pdftitle={The childdoc Package}}
\author{Niklas Beisert\\[2ex]
  Institut f\"ur Theoretische Physik\\
  Eidgen\"ossische Technische Hochschule Z\"urich\\
  Wolfgang-Pauli-Strasse 27, 8093 Z\"urich, Switzerland\\[1ex]
  \href{mailto:nbeisert@itp.phys.ethz.ch}
  {\texttt{nbeisert@itp.phys.ethz.ch}}}
\hypersetup{pdfauthor={Niklas Beisert}}
\hypersetup{pdfsubject={Manual for the LaTeX2e Package childdoc}}
\date{30 December 2018, \textsf{v2.0}}
\maketitle

\begin{abstract}\noindent
\textsf{childdoc} is a \LaTeXe{} package
that enables the direct compilation
of document sections included by |\include|
to individual files.
\end{abstract}

\begingroup
\parskip0ex
\tableofcontents
\endgroup

%%%%%%%%%%%%%%%%%%%%%%%%%%%%%%%%%%%%%%%%%%%%%%%%%%%%%%%%%%%%%%%%%%%%%%%%%%%%%%%%
%%%%%%%%%%%%%%%%%%%%%%%%%%%%%%%%%%%%%%%%%%%%%%%%%%%%%%%%%%%%%%%%%%%%%%%%%%%%%%%%
\section{Introduction}

\LaTeX{} provides a mechanism to structure a large document (such as a book)
into a main file and several child files (containing the chapters)
using the |\include| command.
This mechanism is beneficial for documents
which span hundreds of pages in order to
make the source file(s) more manageable.
Moreover, compilation can be restricted to
selected child files by means of the |\includeonly| command.
The latter feature can be used to reduce the compilation time while editing
(this was significantly more useful in the earlier days of \LaTeX{})
or to generate a smaller document which is easier to navigate.
Another application of |\includeonly| is to generate
documents consisting of selected parts of the complete document.

However, there are a few drawbacks of the plain |\include| mechanism:
\begin{itemize}
\item
The child files cannot be compiled on their own,
they can only be compiled via the main file.
A naive editing environment
(such as a text editor with an option
to have the current file processed by \LaTeX)
may require one to switch to the main file before compiling;
attempting to compile the child file produces errors.
\item
The main file must be modified (each time)
to adjust the |\includeonly| command
to the present needs. This easily leaves the main file in a messy state.
\item
The generated document will always carry the filename
of the main document. This is inconvenient if
several child files are to be compiled and
to be kept for distribution.
\end{itemize}

The present package provides a simple interface
to make child files individually compilable by \LaTeX{}.
Compiling a child file then has the same effect as compiling
the main file with an |\includeonly| command
to select the appropriate child.
Moreover the generated document will carry the name of the child
rather than the main file.
This resolves all three above issues.

This feature is meant to make the editing of books,
thesis documents and lecture notes somewhat more convenient.
However, the package can also be used efficiently for
composing a series of documents (such as exercise sheets)
which are typically distributed individually.
It then assists the author in generating the individual documents
(potentially in different versions)
as well as a document containing the collected series.
Another application is in developing style files
or other kinds of included material
where compilation of the style file could redirect
to a sample or test file.

%%%%%%%%%%%%%%%%%%%%%%%%%%%%%%%%%%%%%%%%%%%%%%%%%%%%%%%%%%%%%%%%%%%%%%%%%%%%%%%%
%%%%%%%%%%%%%%%%%%%%%%%%%%%%%%%%%%%%%%%%%%%%%%%%%%%%%%%%%%%%%%%%%%%%%%%%%%%%%%%%
\section{Usage}

First of all, the package \textsf{childdoc} is \emph{not} a standard
\LaTeXe{} |.sty| style file! Therefore it needs to be invoked in
a non-standard way.

%%%%%%%%%%%%%%%%%%%%%%%%%%%%%%%%%%%%%%%%%%%%%%%%%%%%%%%%%%%%%%%%%%%%%%%%%%%%%%%%
\subsection{Included Files}
\label{sec:include}

%%%%%%%%%%%%%%%%%%%%%%%%%%%%%%%%%%%%%%%%
\DescribeMacro{\childdocmain}
To use the package, add the commands
\begin{center}
\begin{tabular}{l}
|\input{childdoc.def}|\\
|\childdocmain{}|\\
\end{tabular}
\end{center}
at the very top of the main \LaTeX{} file,
in particular \emph{before} the |\documentclass| statement!
The argument of |\childdocmain| should be left empty
(but it must be present).

%%%%%%%%%%%%%%%%%%%%%%%%%%%%%%%%%%%%%%%%
\DescribeMacro{\childdocof}
Furthermore, add the commands
\begin{center}
\begin{tabular}{l}
|\input{childdoc.def}|\\
|\childdocof{|\textit{main}|}|\\
\end{tabular}
\end{center}
at the top of every child file \textit{child}
which is included by |\include{|\textit{child}|}|
from within the main file
(or at least for those files to be compiled individually).
The argument \textit{main} must be the filename of the main file.

There are a couple of
considerations in setting up the main and child documents:

%%%%%%%%%%%%%%%%%%%%%%%%%%%%%%%%%%%%%%%%
\paragraph{Restrictions.}

Please note the following restrictions:
\begin{itemize}
\item
|\childdocmain| must be called with one argument \textit{main}
to ensure compatibility with earlier version of the package.
It must either be empty (|\childdocmain{}|)
or precisely match the filename of the main file in which it is specified.
See \secref{sec:detection} for further information.
\item
The filename \textit{main} must be specified without the |.tex| extension.
\item
The filename \textit{main} is case sensitive
(even in case-insensitive file systems)
due to internal string comparison.
\item
The argument \textit{main} should be fully expanded, it cannot be a macro.
\item
Subdirectories and special characters should be avoided in filenames.
\item
The command |\childdocmain{|\textit{main}|}| must be followed by a whitespace.
It should not be followed immediately by another command
or by a comment mark `|%|'.
This is because the \TeX{} parser reads the token immediately following
the argument of |\childdocmain| and puts it
at the beginning of every child section;
however, a white\-space is ignored.
\end{itemize}

%%%%%%%%%%%%%%%%%%%%%%%%%%%%%%%%%%%%%%%%
\paragraph{Content of Main File.}

It is advisable to place all content in the child files included by |\include|.
Any output contained in the main file will appear in all child documents
unless suppressed manually;
it cannot be suppressed automatically by the |\includeonly| directive
and thus should normally be avoided.
A method to include some content in the main file
by means of conditional processing is described in \secref{sec:conditional}.

%%%%%%%%%%%%%%%%%%%%%%%%%%%%%%%%%%%%%%%%
\paragraph{Page Numbering.}

When only a part of the document is compiled,
the appropriate numbering of pages
(as well as other status parameters)
is determined from the |.aux| files.
The latter contain information from previous passes.
However this information needs to propagate through
all intermediate child documents.
Therefore the page numbering in child documents may well
be inconsistent until the complete document is compiled at least once.

A useful (if unconventional) way to always ensure a consistent
page numbering is to restart the numbering in each child document
and denote the pages by `\textit{child}|.|\textit{page}'
where \textit{child} represents the chapter/section number of the child file.
This can be achieved by the command
|\numberwithin{page}{|\textit{child}|}|
of the \textsf{amsmath} package
where \textit{child} can be |chapter| or |section|
depending on the chosen structuring.
Alternatively, one can modify the macro |\thepage| appropriately
and reset the counter |page| at the start of each child file.

%%%%%%%%%%%%%%%%%%%%%%%%%%%%%%%%%%%%%%%%%%%%%%%%%%%%%%%%%%%%%%%%%%%%%%%%%%%%%%%%
\subsection{Conditional Processing}
\label{sec:conditional}

The package provides a mechanism to compile different versions
of a document. To customise the versions further some conditional processing
can come in handy to distinguish which version is being compiled.
The package provides two macros to describe the compilation context:

%%%%%%%%%%%%%%%%%%%%%%%%%%%%%%%%%%%%%%%%
\DescribeMacro{\ifchilddoc}
The conditional |\ifchilddoc| distinguishes between the compilation of
child documents and the main document:
%
\begin{center}
|\ifchilddoc |\textit{child-code}| |[|\||else |\textit{main-code}]| \||fi|
\end{center}

%%%%%%%%%%%%%%%%%%%%%%%%%%%%%%%%%%%%%%%%
\DescribeMacro{\childdocname}
\DescribeMacro{\childdocjob}
The macro |\childdocname| contains the filename (without extension)
of the main or child file being processed.
Note that |\childdocjob| will always contain the name of the main file.

%%%%%%%%%%%%%%%%%%%%%%%%%%%%%%%%%%%%%%%%
\paragraph{Title Page.}

Conditional processing can be used to include a title or banner page
in the main document when proper precautions are taken.
Importantly, the code in the main file should ensure that the page counter
(as well as other status parameters which are stored in the |.aux| files)
takes the same value after the conditional processing.
Otherwise the page numbers may take divergent values
depending on which part is compiled.

For example, a title page could be declared by:
%
\begin{center}
\begin{tabular}{l}
|\ifchilddoc\||else|\\
|\addtocounter{page}{-1}|\\
\textit{code for title page}\\
|\newpage|\\
|\||fi|
\end{tabular}
\end{center}
%
A banner page for the child documents can be generated by:
%
\begin{center}
\begin{tabular}{l}
|\ifchilddoc|\\
|\addtocounter{page}{-1}|\\
\textit{code for banner page}\\
|\newpage|\\
|\||fi|
\end{tabular}
\end{center}
%
Here one could write a message such as:
\begin{center}
|This is the part \childdocname{} of \childdocjob{}.|
\end{center}

%%%%%%%%%%%%%%%%%%%%%%%%%%%%%%%%%%%%%%%%%%%%%%%%%%%%%%%%%%%%%%%%%%%%%%%%%%%%%%%%
\subsection{Flags}
\label{sec:flags}

The package makes it easy to generate different versions
of the main or child documents.
To this end compilation flags can be defined
and assigned different default values.
They will be particularly useful in conjunction
with the forwarding mechanism described in \secref{sec:forward}.

For example, it may be useful to have a flag |\version|
which can be set to |draft| or |final|.
The document source will contain some conditional code
depending on the value of |\version|.
Suppose further, the flag should default to |final| for the main file
and to |draft| for child files
which is a natural assignment for editing the document.
This is achieved by placing the following code
in the preamble of the main document
(below the |\childdocmain| directive):
%
\begin{center}
\begin{tabular}{l}
|\ifchilddoc|\\
|\providecommand{\version}{draft}|\\
|\||else|\\
|\providecommand{\version}{final}|\\
|\||fi|
\end{tabular}
\end{center}
%
The definition by |\providecommand| makes sure
that previous definitions are not overwritten.
Further statements |\providecommand{\version}{...}|
can thus be added before the above code to override it.

For the main file, one might add a line
(between |\childdocmain| and the above block)
%
\begin{center}
|%\ifchilddoc\||else\providecommand{\version}{draft}\||fi|
\end{center}
%
which can be uncommented to produce a draft version.
Likewise one can add a line to the very top of a child file
(above the |\childdocof{|\textit{main}|}| directive)
%
\begin{center}
|%\providecommand{\version}{final}|
\end{center}
%
which can be uncommented to produce the final version of this child document.

%%%%%%%%%%%%%%%%%%%%%%%%%%%%%%%%%%%%%%%%%%%%%%%%%%%%%%%%%%%%%%%%%%%%%%%%%%%%%%%%
\subsection{Forwarding}
\label{sec:forward}

Different versions of the main or child documents
using compilation flags as described in \secref{sec:flags}
can be (permanently) stored in different files
for convenient compilation, viewing and distribution.
To this end, the package defines a command
to pass on compilation to a different file:

%%%%%%%%%%%%%%%%%%%%%%%%%%%%%%%%%%%%%%%%
\DescribeMacro{\childdocforward}
The command |\childdocforward| redirects processing to
another source file:
%
\begin{center}
\begin{tabular}{l}
|\input{childdoc.def}|\\
|\childdocforward[|\textit{main}|]{|\textit{dest}|}|\\
\end{tabular}
\end{center}
%
The argument \textit{dest} is the destination file
(without extension).
It should be the main file or one of the child files.
Note that further \textsf{childdoc} directives
such as |\childdocof| and |\childdocforward|
in the indicated file will be processed in this form.
The optional argument \textit{main}
passes on directly to the main file \textit{main}
while pretending to compile the child \textit{dest}.
This form behaves as if \textit{dest}
issues |\childdocof{|\textit{main}|}| right away,
and no further \textsf{childdoc} directives will be processed.

%%%%%%%%%%%%%%%%%%%%%%%%%%%%%%%%%%%%%%%%
\DescribeMacro{\...prefix}
In the alternative form |\childdocforwardprefix|,
%
\begin{center}
\begin{tabular}{l}
|\input{childdoc.def}|\\
|\childdocforwardprefix[|\textit{main}|]{|\textit{prefix}|}{|\textit{dest}|}|
\end{tabular}
\end{center}
%
the destination file is determined by a pattern
depending on the current file:
To make this work, the current file must be called
`{\textit{prefix}\hspace{0.2em}\textit{suffix}}'
with \textit{prefix} matching precisely the argument.
Processing is then passed on to the file
`{\textit{dest}\hspace{0.2em}\textit{suffix}}'.
Surely, the same effect is achieved by
directly specifying the
argument `{\textit{dest}\hspace{0.2em}\textit{suffix}}'
in the first form.
However, that requires to set up a different file
for each child. With the alternative form of the command
all these files can have exactly the same content
which simplifies setting them up and maintaining them.

For example, the following file |draft.tex|
with a compilation flag |\version| as described in \secref{sec:flags}
compiles the main document as a draft:
%
\begin{center}
\begin{tabular}{l}
|\def\version{draft}|\\
|\input{childdoc.def}|\\
|\childdocforward{|\textit{main}|}|
\end{tabular}
\end{center}
%
Likewise, the following files |final|\textit{nn}|.tex|
compile the final version of the child document
|child|\textit{nn}|.tex|:
%
\begin{center}
\begin{tabular}{l}
|\def\version{final}|\\
|\input{childdoc.def}|\\
|\childdocforwardprefix{final}{child}|
\end{tabular}
\end{center}
%

Note that when several versions of a main file and/or of each child file
are to be generated, it may be convenient to set up a |Makefile| or
shell script to automatise the process.

%%%%%%%%%%%%%%%%%%%%%%%%%%%%%%%%%%%%%%%%%%%%%%%%%%%%%%%%%%%%%%%%%%%%%%%%%%%%%%%%
\subsection{Command Line Processing}
\label{sec:commandline}

The effect of redirection files can also be achieved by invoking
the \LaTeX{} compiler with a more elaborate command line.
Most conveniently this should be done as part
of a shell script or a |Makefile|.

When using \textsf{childdoc} in the main file, the following
command lines effectively perform a redirection
(note that depending on the shell being used,
backslashes may have to be doubled: `|\|' $\to$ `|\\|'):
%
\begin{center}
|... -jobname "|\textit{target}|" |\\|"|[\textit{flags}]%
|\input{childdoc.def}\childdocforward[|\textit{main}|]{|\textit{dest}|}"|
\end{center}
%
Here \textit{target} is the name of the output file,
\textit{main} is the name of the main file
and \textit{dest} is the name of the main or child file to be processed
(all filenames without extensions).
The optional argument \textit{main} can be omitted
if \textit{main} matches \textit{dest}.
Optionally, compilation \textit{flags} can be defined via |\def| commands.
This command line makes the \TeX{} engine believe
it is compiling the file \textit{target}
whose content is specified as the latter parameter.
The provided code then forwards the processing to
\textit{main} or \textit{dest} as described in \secref{sec:forward}.

%%%%%%%%%%%%%%%%%%%%%%%%%%%%%%%%%%%%%%%%%%%%%%%%%%%%%%%%%%%%%%%%%%%%%%%%%%%%%%%%
\subsection{Include by Input}
\label{sec:input}

Including child documents by |\include| has some restrictions by design.
Most notably, the content of a child document always occupies
its own set of pages; pages cannot be shared between child documents.
Usually, this behaviour makes perfect sense
because each child document contain an essential part of the document.
However, in some situations it may be desirable to compose
a document from a collection of parts
without having mandatory page breaks between then.
For this case, the package
provides a mechanism to include parts
by |\input| which can also be processed individually.
However, by construction this mechanism
requires manual handling of the content to be output.

%%%%%%%%%%%%%%%%%%%%%%%%%%%%%%%%%%%%%%%%
\DescribeMacro{\ifchilddocmanual}
The main file should be prepared as usual, see \secref{sec:include}.
However, the document body must make a distinction
between processing of an individual part and of the main document, e.g.:
%
\begin{center}
\begin{tabular}{l}
|\ifchilddocmanual|\\
|\input{\childdocname}|\\
|\||else|\\
\textit{document body with }|\input{|\textit{part}|}|\\
|\||fi|
\end{tabular}
\end{center}
%
The conditional |\ifchilddocmanual| is true whenever
a part to be included by |\input| is being compiled,
and the name of the part is stored in |\childdocname|.

%%%%%%%%%%%%%%%%%%%%%%%%%%%%%%%%%%%%%%%%
\DescribeMacro{\childdocby}
Each part to be included by |\input| should start with:
%
\begin{center}
\begin{tabular}{l}
|\input{childdoc.def}|\\
|\childdocby{|\textit{main}|}|\\
\end{tabular}
\end{center}
%
The directive |\childdocby| is similar to |\childdocof|
described in \secref{sec:include},
but the subsequent selection of content must be done manually.
To that end, both |\ifchilddoc| and |\ifchilddocmanual|
will be true upon processing of a part,
and the name of the part is stored in |\childdocname|.
Note that |\jobname| will be set to the filename of the current part
so that each part receives an individual |.aux| file
that does not interfere with the |.aux| file(s) of the main document.
This behaviour can be altered by the alternative form
|\childdocby[*]{|\textit{main}|}| (with a non-empty optional argument)
which uses the |.aux| file of the main document
by setting |\jobname| to \textit{main}.

%%%%%%%%%%%%%%%%%%%%%%%%%%%%%%%%%%%%%%%%%%%%%%%%%%%%%%%%%%%%%%%%%%%%%%%%%%%%%%%%
\subsection{Driver Development}
\label{sec:driver}

The \textsf{childdoc} mechanism can also be use for the development
of definition files such as \LaTeX{} styles or classes.
This case differs from the above setup with multiple parts
included by |\include| in that no |\includeonly| should be invoked.
This can be achieved by starting the include file
(before |\ProvidesPackage|) with:
%
\begin{center}
\begin{tabular}{l}
|\input{childdoc.def}|\\
|\childdocforward{|\textit{main}|}|\\
\end{tabular}
\end{center}
%
or alternatively with:
%
\begin{center}
\begin{tabular}{l}
|\input{childdoc.def}|\\
|\childdocby{|\textit{main}|}|\\
\end{tabular}
\end{center}
%
Both forms have slightly different effects as described above.
The main file is prepared as usual, see \secref{sec:include}.

%%%%%%%%%%%%%%%%%%%%%%%%%%%%%%%%%%%%%%%%%%%%%%%%%%%%%%%%%%%%%%%%%%%%%%%%%%%%%%%%
\subsection{Legacy Detection}
\label{sec:detection}

The directive |\childdocmain| in the main file can detect
whether the complete document or merely a child is to be compiled
even without using the directive |\childdocof|.
This method is deprecated because it is less robust
and there is no compelling reason to use it;
it is merely provided for backward compatibility
and it may be removed in future versions.

If the detection mechanism is to be used,
it is mandatory to correctly specify
the filename of the main file as the argument of |\childdocmain|:
%
\begin{center}
\begin{tabular}{l}
|\input{childdoc.def}|\\
|\childdocmain{|\textit{main}|}|\\
\end{tabular}
\end{center}
%
If |\jobname| does not match the argument \textit{main} of |\childdocmain|,
it is assumed that |\jobname| points to the child file to be compiled.
When using |\childdocmain| with the main file specified as argument,
it suffices to start a child file
with just |\input{|\textit{main}|}|
without loading of the package and using |\childdocof|.
If instead all processing is done
with the appropriate \textsf{childdoc} directives,
the argument of \textit{main} of |\childdocmain| can be empty.

An alternative version of the command line processing described
in \secref{sec:commandline} using the detection mechanism reads:
%
\begin{center}
|... -jobname "|\textit{target}|" "|[\textit{flags}]%
[|\def\jobname{|\textit{dest}|}|]|\input{|\textit{main}|}"|
\end{center}

%%%%%%%%%%%%%%%%%%%%%%%%%%%%%%%%%%%%%%%%%%%%%%%%%%%%%%%%%%%%%%%%%%%%%%%%%%%%%%%%
\subsection{Manual Code}
\label{sec:manual}

In case one cannot be certain whether the definitions file |childdoc.def|
is installed on the target \TeX{} distribution
and one prefers not to ship it,
it is conceivable to paste a few relevant commands into the sources.

To that end, drop all statements |\input{childdoc.def}|
and perform the replacements as outlined below.
Instead of |\childdocmain{|\textit{main}|}| add the following code
to the top of the main file:
%
\begin{center}
\begin{tabular}{l}
|\||ifdefined\childdocname\endinput\||fi\newif\ifchilddoc|\\
|\edef\childdocname{\scantokens\expandafter{\jobname\noexpand}}|\\
|\def\childdocmain{|\textit{main}|}\||ifx\childdocmain\childdocname\||else|\\
|\childdoctrue\includeonly{\childdocname}\let\jobname\childdocmain\||fi|\\
\end{tabular}
\end{center}
%
Instead of |\childdocof{|\textit{main}|}| just include the main file
at the top of each child file:
%
\begin{center}
|\input{|\textit{main}|}|
\end{center}
%
A simple redirection |\childdocforward{|\textit{dest}|}| is achieved by:
%
\begin{center}
|\def\jobname{|\textit{dest}|}\input{\jobname}|
\end{center}
%
The redirection with prefix
|\childdocforwardprefix[|\textit{prefix}|]{|\textit{dest}|}|
is accomplished by:
%
\begin{center}
\begin{tabular}{l}
|{\edef\jobname{\scantokens\expandafter{\jobname\noexpand}}|\\
|\def\redirectjob |\textit{prefix}|#1~~~{\gdef\jobname{|\textit{dest}|#1}}|\\
|\expandafter\redirectjob\jobname~~~}\input{\jobname}|
\end{tabular}
\end{center}

In an alternative approach,
child documents can be compiled by a specific command line
without additional code or specific definitions:
%
\begin{center}
|... -jobname "|\textit{target}|" "|[\textit{flags}]%
|\includeonly{|\textit{dest}|}\input{|\textit{main}|}"|
\end{center}
%

%%%%%%%%%%%%%%%%%%%%%%%%%%%%%%%%%%%%%%%%%%%%%%%%%%%%%%%%%%%%%%%%%%%%%%%%%%%%%%%%
%%%%%%%%%%%%%%%%%%%%%%%%%%%%%%%%%%%%%%%%%%%%%%%%%%%%%%%%%%%%%%%%%%%%%%%%%%%%%%%%
\section{Information}

%%%%%%%%%%%%%%%%%%%%%%%%%%%%%%%%%%%%%%%%%%%%%%%%%%%%%%%%%%%%%%%%%%%%%%%%%%%%%%%%
\subsection{Copyright}

Copyright \copyright{} 2017--2018 Niklas Beisert

This work may be distributed and/or modified under the
conditions of the \LaTeX{} Project Public License, either version 1.3
of this license or (at your option) any later version.
The latest version of this license is in
  \url{http://www.latex-project.org/lppl.txt}
and version 1.3 or later is part of all distributions of \LaTeX{}
version 2005/12/01 or later.

This work has the LPPL maintenance status `maintained'.

The Current Maintainer of this work is Niklas Beisert.

This work consists of the files |README.txt|, |childdoc.ins| and |childdoc.dtx|
as well as the derived files |childdoc.def|, |cdocsamp.tex|
with |cdocsch1.tex|, |cdocsch2.tex|, |cdocspt3.tex|, |cdocspt4.tex|,
|cdocsdrf.tex|, |cdocsfn1.tex|, |cdocsfn2.tex|
as well as |childdoc.pdf|.

%%%%%%%%%%%%%%%%%%%%%%%%%%%%%%%%%%%%%%%%%%%%%%%%%%%%%%%%%%%%%%%%%%%%%%%%%%%%%%%%
\subsection{Files and Installation}

The package consists of the files:
%
\begin{center}
\begin{tabular}{ll}
    |README.txt|   & readme file \\
    |childdoc.ins| & installation file \\
    |childdoc.dtx| & source file \\
    |childdoc.def| & definition file \\
    |cdocsamp.tex| & sample main file \\
    |cdocsch1.tex| & sample include file \\
    |cdocsch2.tex| & sample include file \\
    |cdocspt3.tex| & sample part file \\
    |cdocspt4.tex| & sample part file \\
    |cdocsdrf.tex| & sample redirection file \\
    |cdocsfn1.tex| & sample redirection file \\
    |cdocsfn2.tex| & sample redirection file \\
    |childdoc.pdf| & manual
\end{tabular}
\end{center}
%
The distribution consists of the files
|README.txt|, |childdoc.ins| and |childdoc.dtx|.
%
\begin{itemize}
\item
Run (pdf)\LaTeX{} on |childdoc.dtx|
to compile the manual |childdoc.pdf| (this file).
\item
Run \LaTeX{} on |childdoc.ins| to create the definitions file |childdoc.def|
and the sample |cdocsamp.tex| with include files
|cdocsch1.tex|, |cdocsch2.tex|, |cdocspt3.tex|, |cdocspt4.tex|,
|cdocsdrf.tex|, |cdocsfn1.tex|, |cdocsfn2.tex|.
Then copy the file |childdoc.def| to an appropriate directory of your \LaTeX{}
distribution, e.g.\ \textit{texmf-root}|/tex/latex/childdoc|.
\end{itemize}

%%%%%%%%%%%%%%%%%%%%%%%%%%%%%%%%%%%%%%%%%%%%%%%%%%%%%%%%%%%%%%%%%%%%%%%%%%%%%%%%
\subsection{Related CTAN Packages}

There are several other packages which offer a similar functionality:
%
\begin{itemize}
\item
The packages
\href{http://ctan.org/pkg/docmute}{\textsf{docmute}},
\href{http://ctan.org/pkg/includex}{\textsf{includex}} and
\href{http://ctan.org/pkg/standalone}{\textsf{standalone}}
provide commands to include only the document body of
a child file thus allowing both files to be compiled individually.
\item
The packages \href{http://ctan.org/pkg/subdocs}{\textsf{subdocs}}
and \href{http://ctan.org/pkg/subfiles}{\textsf{subfiles}}
provide structures in which the main and child documents can be
encapsulated and allowing them to be compiled individually.
The inclusion mechanism is different from the conventional |\include|.
\item
The package \href{http://ctan.org/pkg/combine}{\textsf{combine}}
is an elaborate solution to combine several documents into one.
\end{itemize}
%
See also the CTAN topic \href{http://ctan.org/topic/subdocs}{\textsf{subdocs}}
for further related packages.
The present package differs from the above solutions in that
a document structure constructed with the conventional |\include| mechanism
just needs two extra commands at the top of every file
such that all constituent files can be compiled individually.

%%%%%%%%%%%%%%%%%%%%%%%%%%%%%%%%%%%%%%%%%%%%%%%%%%%%%%%%%%%%%%%%%%%%%%%%%%%%%%%%
%\subsection{Feature Suggestions}
%
%The following is a list of features which may be useful for future
%versions of this package:
%%
%\begin{itemize}
%\item
%\ldots
%\end{itemize}

%%%%%%%%%%%%%%%%%%%%%%%%%%%%%%%%%%%%%%%%%%%%%%%%%%%%%%%%%%%%%%%%%%%%%%%%%%%%%%%%
\subsection{Revision History}

%%%%%%%%%%%%%%%%%%%%%%%%%%%%%%%%%%%%%%%%
\paragraph{v2.0:} 2018/12/30

\begin{itemize}
\item
immediate forward processing
\item
added |\childdocby| mechanism
\item
manual restructured
\end{itemize}

%%%%%%%%%%%%%%%%%%%%%%%%%%%%%%%%%%%%%%%%
\paragraph{v1.6:} 2018/01/17

\begin{itemize}
\item
application for development of include files
\item
corrections to manual
\end{itemize}

%%%%%%%%%%%%%%%%%%%%%%%%%%%%%%%%%%%%%%%%
\paragraph{v1.5:} 2017/05/21

\begin{itemize}
\item
more complete structuring introduced
\item
|\childdocof| introduced
\item
|\childdoc| renamed to |\childdocmain|
\item
|\childredirect| renamed to |\childdocforward| and |\childdocforwardprefix|
and functionality expanded
\end{itemize}

%%%%%%%%%%%%%%%%%%%%%%%%%%%%%%%%%%%%%%%%
\paragraph{v1.0:} 2017/04/27

\begin{itemize}
\item
manual and install package
\item
first version published on CTAN
\end{itemize}

%%%%%%%%%%%%%%%%%%%%%%%%%%%%%%%%%%%%%%%%
\paragraph{v0.6:} 2017/04/26

\begin{itemize}
\item
redirection mechanism added
\end{itemize}

%%%%%%%%%%%%%%%%%%%%%%%%%%%%%%%%%%%%%%%%
\paragraph{v0.5:} 2017/04/26

\begin{itemize}
\item
functionality in definition file
\end{itemize}


%%%%%%%%%%%%%%%%%%%%%%%%%%%%%%%%%%%%%%%%%%%%%%%%%%%%%%%%%%%%%%%%%%%%%%%%%%%%%%%%
%%%%%%%%%%%%%%%%%%%%%%%%%%%%%%%%%%%%%%%%%%%%%%%%%%%%%%%%%%%%%%%%%%%%%%%%%%%%%%%%
%%%%%%%%%%%%%%%%%%%%%%%%%%%%%%%%%%%%%%%%%%%%%%%%%%%%%%%%%%%%%%%%%%%%%%%%%%%%%%%%
\appendix

\settowidth\MacroIndent{\rmfamily\scriptsize 000\ }

 \DocInput{childdoc.dtx}

\end{document}
%</driver>
% \fi
%
% %%%%%%%%%%%%%%%%%%%%%%%%%%%%%%%%%%%%%%%%%%%%%%%%%%%%%%%%%%%%%%%%%%%%%%%%%%%%%%
% %%%%%%%%%%%%%%%%%%%%%%%%%%%%%%%%%%%%%%%%%%%%%%%%%%%%%%%%%%%%%%%%%%%%%%%%%%%%%%
% \section{Sample}
%\iffalse
%<*samplemain>
%\fi
%
% The following presents a sample document
% with two chapters, two parts, a title page,
% a compile flag as well as three forwarding files to set the flag.
% It consists of eight |.tex| files:
% \begin{center}
% \begin{tabular}{ll}
% |cdocsamp.tex|&main file\\
% |cdocsch1.tex|&include file for chapter 1\\
% |cdocsch2.tex|&include file for chapter 2\\
% |cdocspt3.tex|&include file for part 3\\
% |cdocspt4.tex|&include file for part 4\\
% |cdocsdrf.tex|&forwarding file for main file in draft mode\\
% |cdocsfi1.tex|&forwarding file for final version of chapter 1\\
% |cdocsfi2.tex|&forwarding file for final version of chapter 2\\
% \end{tabular}
% \end{center}
% Each of the eight files can be compiled directly by the \LaTeX{} compiler.
%
% %%%%%%%%%%%%%%%%%%%%%%%%%%%%%%%%%%%%%%
% \paragraph{Main File.}
%
% The main file is called |cdocsamp.tex|.
%
% Load the \textsf{childdoc} definitions and
% declare the filename for the main document:
%    \begin{macrocode}
\input{childdoc.def}
\childdocmain{}
%    \end{macrocode}

% Optional override for |\version| flag:
%    \begin{macrocode}
%%\ifchilddoc\else\providecommand{\version}{draft}\fi
%    \end{macrocode}

% Define the default values for the |\version| flag
% (|final| for the main file and |draft| for childs):
%    \begin{macrocode}
\ifchilddoc
\providecommand{\version}{draft}
\else
\providecommand{\version}{final}
\fi
%    \end{macrocode}

% Load the standard document class:
%    \begin{macrocode}
\documentclass[12pt]{article}
%    \end{macrocode}

% Start the document body:
%    \begin{macrocode}
\begin{document}
%    \end{macrocode}

% Declare a title page.
% Print title, part of document being processed and version flag:
%    \begin{macrocode}
\addtocounter{page}{-1}
\begin{center}
{\LARGE\bfseries{}childdoc example\par}
\vspace{1cm}
\ifchilddoc
\ifchilddocmanual part\else chapter\fi:
`\childdocname' of `\childdocjob'\par
\else
main document: `\childdocjob'\par
\fi
version: \version\par
\end{center}
\newpage
%    \end{macrocode}

% Manually include selected file,
% otherwise process as usual:
%    \begin{macrocode}
\ifchilddocmanual
\section*{part `\childdocname'}
\input{\childdocname}
\else
%    \end{macrocode}

% Include the two chapters:
%    \begin{macrocode}
\include{cdocsch1}
\include{cdocsch2}
%    \end{macrocode}

% Include the two parts unless only chapters should be displayed:
%    \begin{macrocode}
\ifchilddoc\else
\section{part three}
\input{cdocspt3}
\section{part four}
\input{cdocspt4}
\fi
%    \end{macrocode}

% Process as usual until here:
%    \begin{macrocode}
\fi
%    \end{macrocode}

% End of document body:
%    \begin{macrocode}
\end{document}
%    \end{macrocode}
%\iffalse
%</samplemain>
%\fi
%
% %%%%%%%%%%%%%%%%%%%%%%%%%%%%%%%%%%%%%%
% \paragraph{Chapter Include Files.}
%
% The include files are called |cdocsch1.tex| and |cdocsch2.tex|.
%
%\iffalse
%<*samplechap1|samplechap2>
%\fi

% Optional override for |\version| flag:
%    \begin{macrocode}
%%\providecommand{\version}{final}
%    \end{macrocode}

% Include the main document:
%    \begin{macrocode}
\input{childdoc.def}
\childdocof{cdocsamp}
%    \end{macrocode}

%\iffalse
%</samplechap1|samplechap2>
%\fi
%
%\iffalse
%<*samplechap1>
%\fi
% Some text for chapter 1:
%    \begin{macrocode}
\section{one}
some text in chapter one
%    \end{macrocode}

%\iffalse
%</samplechap1>
%\fi
% Some text for chapter 2:
%\iffalse
%<*samplechap2>
%\fi
%    \begin{macrocode}
\section{two}
more text in chapter two
%    \end{macrocode}

%\iffalse
%</samplechap2>
%\fi
%
% %%%%%%%%%%%%%%%%%%%%%%%%%%%%%%%%%%%%%%
% \paragraph{Part Include Files.}
%
% The include files are called |cdocspt3.tex| and |cdocspt4.tex|.
%
%\iffalse
%<*samplepart3|samplepart4>
%\fi

% Optional override for |\version| flag:
%    \begin{macrocode}
%%\providecommand{\version}{final}
%    \end{macrocode}

% Include the main document:
%    \begin{macrocode}
\input{childdoc.def}
\childdocby{cdocsamp}
%    \end{macrocode}

%\iffalse
%</samplepart3|samplepart4>
%\fi
%
%\iffalse
%<*samplepart3>
%\fi
% Some text for part 3:
%    \begin{macrocode}
some text in part three
%    \end{macrocode}

%\iffalse
%</samplepart3>
%\fi
% Some text for part 4:
%\iffalse
%<*samplepart4>
%\fi
%    \begin{macrocode}
more text in part four
%    \end{macrocode}

%\iffalse
%</samplepart4>
%\fi
%
% %%%%%%%%%%%%%%%%%%%%%%%%%%%%%%%%%%%%%%
% \paragraph{Forwarding for a Complete Draft.}
%
% The following forwarding file |cdocsdrf.tex|
% compiles the main document in draft mode:
%\iffalse
%<*sampledraft>
%\fi
%    \begin{macrocode}
\def\version{draft}
\input{childdoc.def}
\childdocforward{cdocsamp}
%    \end{macrocode}

%\iffalse
%</sampledraft>
%\fi
%
% %%%%%%%%%%%%%%%%%%%%%%%%%%%%%%%%%%%%%%
% \paragraph{Forwarding for Final Version of the Chapters.}
%
% The following forwarding files |cdocsfn1.tex| and |cdocsfn2.tex|
% (with identical content)
% compile the final versions of the child documents
% |cdocsch1.tex| and |cdocsch2.tex|, respectively:
%\iffalse
%<*samplefinal>
%\fi
%    \begin{macrocode}
\def\version{final}
\input{childdoc.def}
\childdocforwardprefix[cdocsamp]{cdocsfn}{cdocsch}
%    \end{macrocode}

%\iffalse
%</samplefinal>
%\fi
%
% %%%%%%%%%%%%%%%%%%%%%%%%%%%%%%%%%%%%%%
% \paragraph{Command Line Processing.}
%
% The following three command lines generate the output files
% |cdocscld|, |cdocscl1| and |cdocscl2|
% which should be identical to
% |cdocsdrf|, |cdocsch1| and |cdocsfn2|, respectively:
% \begin{center}
% \begin{tabular}{l}
% |latex -jobname cdocscld \|\\
% |  "\def\version{draft}\input{childdoc.def}\childdocforward{cdocsamp}"|\\
% |latex -jobname cdocscl1 \|\\
% |  "\input{childdoc.def}\childdocforward[cdocsamp]{cdocsch1}"|\\
% |latex -jobname cdocscl2 \|\\
% |  "\def\version{final}\input{childdoc.def}\childdocforward{cdocsch2}"|
% \end{tabular}
% \end{center}
% Note that the trailing backslash on each first line
% merely continues the input to the second line
% (for convenient cut ant paste).
% Furthermore, the command |latex| can be replaced by any
% of its alternative versions such as |pdflatex|.
%
% %%%%%%%%%%%%%%%%%%%%%%%%%%%%%%%%%%%%%%%%%%%%%%%%%%%%%%%%%%%%%%%%%%%%%%%%%%%%%%
% %%%%%%%%%%%%%%%%%%%%%%%%%%%%%%%%%%%%%%%%%%%%%%%%%%%%%%%%%%%%%%%%%%%%%%%%%%%%%%
% \section{Implementation}
%\iffalse
%<*package>
%\fi
%
% This section describes the definitions file |childdoc.def|.

% The definitions cannot be loaded using |\usepackage| or |\RequirePackage|
% which has a mechanism to prevent loading a style file more than once.
% When loading the definitions by means of |\input|
% multiple instances have to be prevented manually:
%\iffalse
%This code needs to be before the `\ProvidesFile' directive
%which is defined at the beginning of this file.
%Therefore it is also placed there and commented out here.
%</package>
%<*discard>
%\fi
%    \begin{macrocode}
\ifdefined\childdocmain\endinput\fi
%    \end{macrocode}
%\iffalse
%</discard>
%<*package>
%\fi
%
% \macro{\ifchilddoc}
% \macro{\ifchilddocmanual}
% The conditional |\ifchilddoc| tells whether a
% child (true) or main (false) document is being compiled.
% The conditional |\ifchilddocmanual| tells whether
% the |\includeonly| mechanism is used (false) or
% the selection of child files must be performed manually (true).
% The definitions initialise to false:
%    \begin{macrocode}
\newif\ifchilddoc
\newif\ifchilddocmanual
%    \end{macrocode}

% \macro{\childdocname}
% \macro{\childdocjob}
% The macro |\childdocname| stores the name of the main document
% to be compiled. The macro |\childdocjob| stores the name of
% the document on which the \LaTeX{} compiler was originally invoked.
% The content of |\jobname| cannot be compared
% to filenames specified in the source due to different catcodes.
% The following code rescans |\jobname|, stores the result
% in |\childdocname| and saves a copy in |\childdocjob|:
%    \begin{macrocode}
\edef\childdocname{\scantokens\expandafter{\jobname\noexpand}}
\let\childdocjob\childdocname
%    \end{macrocode}

% \macro{\childdocdisable}
% The macro |\childdocdisable| prevents the main file
% from being processed more than once.
% At this stage, the main document command |\childdocmain|
% is assumed to be called once again where it should do nothing.
% Any subsequent call to it should prevent
% a secondary processing of the main document
% It overwrites the forwarding commands
% |\childdocof| and |\childdocforward|
% with empty macros to prevent further inclusions of the main document:
%    \begin{macrocode}
\newcommand{\childdocdisable}
{
  \renewcommand{\childdocmain}[1]{\renewcommand{\childdocmain}[1]{\endinput}}
  \renewcommand{\childdocof}[1]{}
  \renewcommand{\childdocby}[2][]{}
  \renewcommand{\childdocforward}[2][]{}
  \renewcommand{\childdocdisable}{}
}
%    \end{macrocode}

% \macro{\childdocmain}
% The macro |\childdocmain| is to be called at the top of the main file
% with nothing or the main filename (without extension) as argument.
% First, it breaks loops.
% If the argument is not empty and does not match |\childdocname|
% (which is set by the first inclusion of |childdoc.def|),
% |\ifchilddoc| is set to true, |\includeonly| is applied to the child file
% and |\jobname| is set to the main file
% (for proper handling of |.aux| files):
%    \begin{macrocode}
\newcommand{\childdocmain}[1]
{
  \childdocdisable\childdocmain{}
  \if?#1?\else
    \begingroup
      \def\childdoctmp{#1}
      \ifx\childdoctmp\childdocname
        \def\childdoctmp{}
      \else
        \def\childdoctmp
        {
          \childdoctrue
          \includeonly{\childdocname}
          \def\childdocjob{#1}
          \def\jobname{#1}
        }
      \fi
      \expandafter
    \endgroup
    \childdoctmp
  \fi
}
%    \end{macrocode}

% \macro{\childdocof}
% The command |\childdocof| redirects
% compilation to the main file |#1|.
%    \begin{macrocode}
\newcommand{\childdocof}[1]
{
  \childdocdisable
  \childdoctrue
  \includeonly{\childdocname}
  \def\jobname{#1}
  \def\childdocjob{#1}
  \input{#1}
}
%    \end{macrocode}

% \macro{\childdocby}
% The command |\childdocby| ....
%    \begin{macrocode}
\newcommand{\childdocby}[2][]
{
  \childdocdisable
  \childdoctrue
  \childdocmanualtrue
  \if?#1?\else
    \def\jobname{#2}
  \fi
  \def\childdocjob{#2}
  \input{#2}
  \endinput
}
%    \end{macrocode}

% \macro{\childdocforward}
% The command |\childdocforward| redirects
% compilation to the main file or
% (if the optional argument is given) a child file.
% Parameters are set as if the main file
% or a child file starting with |\childdocof| was compiled.
% Then compilation is handed over to the main file:
%    \begin{macrocode}
\newcommand{\childdocforward}[2][]
{
  \begingroup
    \if?#1?
      \def\childdoctmp
      {
        \def\childdocname{#2}
        \def\childdocjob{#2}
        \def\jobname{#2}
        \input{#2}
        \endinput
      }
    \else
      \def\childdoctmp
      {
        \childdocdisable
        \def\childdocname{#2}
        \childdoctrue
        \includeonly{#2}
        \def\childdocjob{#1}
        \def\jobname{#1}
        \input{#1}
        \endinput
      }
    \fi
    \expandafter
  \endgroup
  \childdoctmp
}
%    \end{macrocode}

% \macro{\childdocforwardprefix}
% The command |\childdocforwardprefix| redirects
% compilation to the main or a child file by means of a pattern.
% The prefix |#1| in the current filename is replaced by |#2|
% and the suffix of the current filename is kept
% (it is assumed that the filename does not contain the substring `|~~~|'
% which is used as a delimiter).
% Compilation is handed over to the new file by |\childdocforward|:
%    \begin{macrocode}
\newcommand{\childdocforwardprefix}[3][]
{
  \begingroup
    \def\childdocextract #2##1~~~{\def\childdoctmp{\childdocforward[#1]{#3##1}}}
    \expandafter\childdocextract\childdocname~~~
    \expandafter
  \endgroup
  \childdoctmp
}
%    \end{macrocode}

% \macro{\childdoc}
% The deprecated macro |\childdoc| is a legacy version of |\childdocmain|:
%    \begin{macrocode}
\newcommand{\childdoc}{\childdocmain}
%    \end{macrocode}

% \macro{\childdocredirect}
% The deprecated macro |\childdocredirect| is a legacy version
% of |\childdocforward| and |\childdocforwardprefix|:
%    \begin{macrocode}
\newcommand{\childdocredirect}[2][]
{
  \begingroup
    \if?#1?
      \def\childdoctmp{\childdocforward{#2}}
    \else
      \def\childdoctmp{\childdocforwardprefix{#1}{#2}}
    \fi
    \expandafter
  \endgroup
  \childdoctmp
}
%    \end{macrocode}

%\iffalse
%</package>
%\fi
%
\endinput
|\\
|\childdocforwardprefix[|\textit{main}|]{|\textit{prefix}|}{|\textit{dest}|}|
\end{tabular}
\end{center}
%
the destination file is determined by a pattern
depending on the current file:
To make this work, the current file must be called
`{\textit{prefix}\hspace{0.2em}\textit{suffix}}'
with \textit{prefix} matching precisely the argument.
Processing is then passed on to the file
`{\textit{dest}\hspace{0.2em}\textit{suffix}}'.
Surely, the same effect is achieved by
directly specifying the
argument `{\textit{dest}\hspace{0.2em}\textit{suffix}}'
in the first form.
However, that requires to set up a different file
for each child. With the alternative form of the command
all these files can have exactly the same content
which simplifies setting them up and maintaining them.

For example, the following file |draft.tex|
with a compilation flag |\version| as described in \secref{sec:flags}
compiles the main document as a draft:
%
\begin{center}
\begin{tabular}{l}
|\def\version{draft}|\\
|% \iffalse
%
% childdoc.dtx Copyright (C) 2017-2018 Niklas Beisert
%
% This work may be distributed and/or modified under the
% conditions of the LaTeX Project Public License, either version 1.3
% of this license or (at your option) any later version.
% The latest version of this license is in
%   http://www.latex-project.org/lppl.txt
% and version 1.3 or later is part of all distributions of LaTeX
% version 2005/12/01 or later.
%
% This work has the LPPL maintenance status `maintained'.
%
% The Current Maintainer of this work is Niklas Beisert.
%
% This work consists of the files childdoc.dtx and childdoc.ins
% and the derived files childdoc.def and cdocsamp.tex with
% cdocsch1.tex, cdocsch2.tex, cdocsdrf.tex, cdocsfn1.tex, cdocsfn2.tex.
%
%<package>\ifdefined\childdocmain\endinput\fi
%<package>\ProvidesFile{childdoc.def}[2018/12/30 v2.0 child document driver]
%<samplemain>\ProvidesFile{cdocsamp.tex}[2018/12/30 v2.0 sample for childdoc]
%<*driver>
%\ProvidesFile{childdoc.drv}[2018/12/30 v2.0 childdoc reference manual file]
\PassOptionsToClass{10pt,a4paper}{article}
\documentclass{ltxdoc}

\usepackage[margin=35mm]{geometry}
\usepackage{hyperref}
\usepackage{hyperxmp}
\usepackage[usenames]{color}

\hypersetup{colorlinks=true}
\hypersetup{pdfstartview=FitH}
\hypersetup{pdfpagemode=UseNone}
\hypersetup{pdfsource={}}
\hypersetup{pdflang={en-UK}}
\hypersetup{pdfcopyright={Copyright 2017-2018 Niklas Beisert.
  This work may be distributed and/or modified under the
  conditions of the LaTeX Project Public License, either version 1.3
  of this license or (at your option) any later version.}}
\hypersetup{pdflicenseurl={http://www.latex-project.org/lppl.txt}}
\hypersetup{pdfcontactaddress={ETH Zurich, ITP, HIT K,
  Wolfgang-Pauli-Strasse 27}}
\hypersetup{pdfcontactpostcode={8093}}
\hypersetup{pdfcontactcity={Zurich}}
\hypersetup{pdfcontactcountry={Switzerland}}
\hypersetup{pdfcontactemail={nbeisert@itp.phys.ethz.ch}}
\hypersetup{pdfcontacturl={http://people.phys.ethz.ch/\xmptilde nbeisert/}}

\newcommand{\secref}[1]{\hyperref[#1]{section \ref*{#1}}}

\parskip1ex
\parindent0pt
\let\olditemize\itemize
\def\itemize{\olditemize\parskip0pt}

\begin{document}

\title{The \textsf{childdoc} Package}
\hypersetup{pdftitle={The childdoc Package}}
\author{Niklas Beisert\\[2ex]
  Institut f\"ur Theoretische Physik\\
  Eidgen\"ossische Technische Hochschule Z\"urich\\
  Wolfgang-Pauli-Strasse 27, 8093 Z\"urich, Switzerland\\[1ex]
  \href{mailto:nbeisert@itp.phys.ethz.ch}
  {\texttt{nbeisert@itp.phys.ethz.ch}}}
\hypersetup{pdfauthor={Niklas Beisert}}
\hypersetup{pdfsubject={Manual for the LaTeX2e Package childdoc}}
\date{30 December 2018, \textsf{v2.0}}
\maketitle

\begin{abstract}\noindent
\textsf{childdoc} is a \LaTeXe{} package
that enables the direct compilation
of document sections included by |\include|
to individual files.
\end{abstract}

\begingroup
\parskip0ex
\tableofcontents
\endgroup

%%%%%%%%%%%%%%%%%%%%%%%%%%%%%%%%%%%%%%%%%%%%%%%%%%%%%%%%%%%%%%%%%%%%%%%%%%%%%%%%
%%%%%%%%%%%%%%%%%%%%%%%%%%%%%%%%%%%%%%%%%%%%%%%%%%%%%%%%%%%%%%%%%%%%%%%%%%%%%%%%
\section{Introduction}

\LaTeX{} provides a mechanism to structure a large document (such as a book)
into a main file and several child files (containing the chapters)
using the |\include| command.
This mechanism is beneficial for documents
which span hundreds of pages in order to
make the source file(s) more manageable.
Moreover, compilation can be restricted to
selected child files by means of the |\includeonly| command.
The latter feature can be used to reduce the compilation time while editing
(this was significantly more useful in the earlier days of \LaTeX{})
or to generate a smaller document which is easier to navigate.
Another application of |\includeonly| is to generate
documents consisting of selected parts of the complete document.

However, there are a few drawbacks of the plain |\include| mechanism:
\begin{itemize}
\item
The child files cannot be compiled on their own,
they can only be compiled via the main file.
A naive editing environment
(such as a text editor with an option
to have the current file processed by \LaTeX)
may require one to switch to the main file before compiling;
attempting to compile the child file produces errors.
\item
The main file must be modified (each time)
to adjust the |\includeonly| command
to the present needs. This easily leaves the main file in a messy state.
\item
The generated document will always carry the filename
of the main document. This is inconvenient if
several child files are to be compiled and
to be kept for distribution.
\end{itemize}

The present package provides a simple interface
to make child files individually compilable by \LaTeX{}.
Compiling a child file then has the same effect as compiling
the main file with an |\includeonly| command
to select the appropriate child.
Moreover the generated document will carry the name of the child
rather than the main file.
This resolves all three above issues.

This feature is meant to make the editing of books,
thesis documents and lecture notes somewhat more convenient.
However, the package can also be used efficiently for
composing a series of documents (such as exercise sheets)
which are typically distributed individually.
It then assists the author in generating the individual documents
(potentially in different versions)
as well as a document containing the collected series.
Another application is in developing style files
or other kinds of included material
where compilation of the style file could redirect
to a sample or test file.

%%%%%%%%%%%%%%%%%%%%%%%%%%%%%%%%%%%%%%%%%%%%%%%%%%%%%%%%%%%%%%%%%%%%%%%%%%%%%%%%
%%%%%%%%%%%%%%%%%%%%%%%%%%%%%%%%%%%%%%%%%%%%%%%%%%%%%%%%%%%%%%%%%%%%%%%%%%%%%%%%
\section{Usage}

First of all, the package \textsf{childdoc} is \emph{not} a standard
\LaTeXe{} |.sty| style file! Therefore it needs to be invoked in
a non-standard way.

%%%%%%%%%%%%%%%%%%%%%%%%%%%%%%%%%%%%%%%%%%%%%%%%%%%%%%%%%%%%%%%%%%%%%%%%%%%%%%%%
\subsection{Included Files}
\label{sec:include}

%%%%%%%%%%%%%%%%%%%%%%%%%%%%%%%%%%%%%%%%
\DescribeMacro{\childdocmain}
To use the package, add the commands
\begin{center}
\begin{tabular}{l}
|\input{childdoc.def}|\\
|\childdocmain{}|\\
\end{tabular}
\end{center}
at the very top of the main \LaTeX{} file,
in particular \emph{before} the |\documentclass| statement!
The argument of |\childdocmain| should be left empty
(but it must be present).

%%%%%%%%%%%%%%%%%%%%%%%%%%%%%%%%%%%%%%%%
\DescribeMacro{\childdocof}
Furthermore, add the commands
\begin{center}
\begin{tabular}{l}
|\input{childdoc.def}|\\
|\childdocof{|\textit{main}|}|\\
\end{tabular}
\end{center}
at the top of every child file \textit{child}
which is included by |\include{|\textit{child}|}|
from within the main file
(or at least for those files to be compiled individually).
The argument \textit{main} must be the filename of the main file.

There are a couple of
considerations in setting up the main and child documents:

%%%%%%%%%%%%%%%%%%%%%%%%%%%%%%%%%%%%%%%%
\paragraph{Restrictions.}

Please note the following restrictions:
\begin{itemize}
\item
|\childdocmain| must be called with one argument \textit{main}
to ensure compatibility with earlier version of the package.
It must either be empty (|\childdocmain{}|)
or precisely match the filename of the main file in which it is specified.
See \secref{sec:detection} for further information.
\item
The filename \textit{main} must be specified without the |.tex| extension.
\item
The filename \textit{main} is case sensitive
(even in case-insensitive file systems)
due to internal string comparison.
\item
The argument \textit{main} should be fully expanded, it cannot be a macro.
\item
Subdirectories and special characters should be avoided in filenames.
\item
The command |\childdocmain{|\textit{main}|}| must be followed by a whitespace.
It should not be followed immediately by another command
or by a comment mark `|%|'.
This is because the \TeX{} parser reads the token immediately following
the argument of |\childdocmain| and puts it
at the beginning of every child section;
however, a white\-space is ignored.
\end{itemize}

%%%%%%%%%%%%%%%%%%%%%%%%%%%%%%%%%%%%%%%%
\paragraph{Content of Main File.}

It is advisable to place all content in the child files included by |\include|.
Any output contained in the main file will appear in all child documents
unless suppressed manually;
it cannot be suppressed automatically by the |\includeonly| directive
and thus should normally be avoided.
A method to include some content in the main file
by means of conditional processing is described in \secref{sec:conditional}.

%%%%%%%%%%%%%%%%%%%%%%%%%%%%%%%%%%%%%%%%
\paragraph{Page Numbering.}

When only a part of the document is compiled,
the appropriate numbering of pages
(as well as other status parameters)
is determined from the |.aux| files.
The latter contain information from previous passes.
However this information needs to propagate through
all intermediate child documents.
Therefore the page numbering in child documents may well
be inconsistent until the complete document is compiled at least once.

A useful (if unconventional) way to always ensure a consistent
page numbering is to restart the numbering in each child document
and denote the pages by `\textit{child}|.|\textit{page}'
where \textit{child} represents the chapter/section number of the child file.
This can be achieved by the command
|\numberwithin{page}{|\textit{child}|}|
of the \textsf{amsmath} package
where \textit{child} can be |chapter| or |section|
depending on the chosen structuring.
Alternatively, one can modify the macro |\thepage| appropriately
and reset the counter |page| at the start of each child file.

%%%%%%%%%%%%%%%%%%%%%%%%%%%%%%%%%%%%%%%%%%%%%%%%%%%%%%%%%%%%%%%%%%%%%%%%%%%%%%%%
\subsection{Conditional Processing}
\label{sec:conditional}

The package provides a mechanism to compile different versions
of a document. To customise the versions further some conditional processing
can come in handy to distinguish which version is being compiled.
The package provides two macros to describe the compilation context:

%%%%%%%%%%%%%%%%%%%%%%%%%%%%%%%%%%%%%%%%
\DescribeMacro{\ifchilddoc}
The conditional |\ifchilddoc| distinguishes between the compilation of
child documents and the main document:
%
\begin{center}
|\ifchilddoc |\textit{child-code}| |[|\||else |\textit{main-code}]| \||fi|
\end{center}

%%%%%%%%%%%%%%%%%%%%%%%%%%%%%%%%%%%%%%%%
\DescribeMacro{\childdocname}
\DescribeMacro{\childdocjob}
The macro |\childdocname| contains the filename (without extension)
of the main or child file being processed.
Note that |\childdocjob| will always contain the name of the main file.

%%%%%%%%%%%%%%%%%%%%%%%%%%%%%%%%%%%%%%%%
\paragraph{Title Page.}

Conditional processing can be used to include a title or banner page
in the main document when proper precautions are taken.
Importantly, the code in the main file should ensure that the page counter
(as well as other status parameters which are stored in the |.aux| files)
takes the same value after the conditional processing.
Otherwise the page numbers may take divergent values
depending on which part is compiled.

For example, a title page could be declared by:
%
\begin{center}
\begin{tabular}{l}
|\ifchilddoc\||else|\\
|\addtocounter{page}{-1}|\\
\textit{code for title page}\\
|\newpage|\\
|\||fi|
\end{tabular}
\end{center}
%
A banner page for the child documents can be generated by:
%
\begin{center}
\begin{tabular}{l}
|\ifchilddoc|\\
|\addtocounter{page}{-1}|\\
\textit{code for banner page}\\
|\newpage|\\
|\||fi|
\end{tabular}
\end{center}
%
Here one could write a message such as:
\begin{center}
|This is the part \childdocname{} of \childdocjob{}.|
\end{center}

%%%%%%%%%%%%%%%%%%%%%%%%%%%%%%%%%%%%%%%%%%%%%%%%%%%%%%%%%%%%%%%%%%%%%%%%%%%%%%%%
\subsection{Flags}
\label{sec:flags}

The package makes it easy to generate different versions
of the main or child documents.
To this end compilation flags can be defined
and assigned different default values.
They will be particularly useful in conjunction
with the forwarding mechanism described in \secref{sec:forward}.

For example, it may be useful to have a flag |\version|
which can be set to |draft| or |final|.
The document source will contain some conditional code
depending on the value of |\version|.
Suppose further, the flag should default to |final| for the main file
and to |draft| for child files
which is a natural assignment for editing the document.
This is achieved by placing the following code
in the preamble of the main document
(below the |\childdocmain| directive):
%
\begin{center}
\begin{tabular}{l}
|\ifchilddoc|\\
|\providecommand{\version}{draft}|\\
|\||else|\\
|\providecommand{\version}{final}|\\
|\||fi|
\end{tabular}
\end{center}
%
The definition by |\providecommand| makes sure
that previous definitions are not overwritten.
Further statements |\providecommand{\version}{...}|
can thus be added before the above code to override it.

For the main file, one might add a line
(between |\childdocmain| and the above block)
%
\begin{center}
|%\ifchilddoc\||else\providecommand{\version}{draft}\||fi|
\end{center}
%
which can be uncommented to produce a draft version.
Likewise one can add a line to the very top of a child file
(above the |\childdocof{|\textit{main}|}| directive)
%
\begin{center}
|%\providecommand{\version}{final}|
\end{center}
%
which can be uncommented to produce the final version of this child document.

%%%%%%%%%%%%%%%%%%%%%%%%%%%%%%%%%%%%%%%%%%%%%%%%%%%%%%%%%%%%%%%%%%%%%%%%%%%%%%%%
\subsection{Forwarding}
\label{sec:forward}

Different versions of the main or child documents
using compilation flags as described in \secref{sec:flags}
can be (permanently) stored in different files
for convenient compilation, viewing and distribution.
To this end, the package defines a command
to pass on compilation to a different file:

%%%%%%%%%%%%%%%%%%%%%%%%%%%%%%%%%%%%%%%%
\DescribeMacro{\childdocforward}
The command |\childdocforward| redirects processing to
another source file:
%
\begin{center}
\begin{tabular}{l}
|\input{childdoc.def}|\\
|\childdocforward[|\textit{main}|]{|\textit{dest}|}|\\
\end{tabular}
\end{center}
%
The argument \textit{dest} is the destination file
(without extension).
It should be the main file or one of the child files.
Note that further \textsf{childdoc} directives
such as |\childdocof| and |\childdocforward|
in the indicated file will be processed in this form.
The optional argument \textit{main}
passes on directly to the main file \textit{main}
while pretending to compile the child \textit{dest}.
This form behaves as if \textit{dest}
issues |\childdocof{|\textit{main}|}| right away,
and no further \textsf{childdoc} directives will be processed.

%%%%%%%%%%%%%%%%%%%%%%%%%%%%%%%%%%%%%%%%
\DescribeMacro{\...prefix}
In the alternative form |\childdocforwardprefix|,
%
\begin{center}
\begin{tabular}{l}
|\input{childdoc.def}|\\
|\childdocforwardprefix[|\textit{main}|]{|\textit{prefix}|}{|\textit{dest}|}|
\end{tabular}
\end{center}
%
the destination file is determined by a pattern
depending on the current file:
To make this work, the current file must be called
`{\textit{prefix}\hspace{0.2em}\textit{suffix}}'
with \textit{prefix} matching precisely the argument.
Processing is then passed on to the file
`{\textit{dest}\hspace{0.2em}\textit{suffix}}'.
Surely, the same effect is achieved by
directly specifying the
argument `{\textit{dest}\hspace{0.2em}\textit{suffix}}'
in the first form.
However, that requires to set up a different file
for each child. With the alternative form of the command
all these files can have exactly the same content
which simplifies setting them up and maintaining them.

For example, the following file |draft.tex|
with a compilation flag |\version| as described in \secref{sec:flags}
compiles the main document as a draft:
%
\begin{center}
\begin{tabular}{l}
|\def\version{draft}|\\
|\input{childdoc.def}|\\
|\childdocforward{|\textit{main}|}|
\end{tabular}
\end{center}
%
Likewise, the following files |final|\textit{nn}|.tex|
compile the final version of the child document
|child|\textit{nn}|.tex|:
%
\begin{center}
\begin{tabular}{l}
|\def\version{final}|\\
|\input{childdoc.def}|\\
|\childdocforwardprefix{final}{child}|
\end{tabular}
\end{center}
%

Note that when several versions of a main file and/or of each child file
are to be generated, it may be convenient to set up a |Makefile| or
shell script to automatise the process.

%%%%%%%%%%%%%%%%%%%%%%%%%%%%%%%%%%%%%%%%%%%%%%%%%%%%%%%%%%%%%%%%%%%%%%%%%%%%%%%%
\subsection{Command Line Processing}
\label{sec:commandline}

The effect of redirection files can also be achieved by invoking
the \LaTeX{} compiler with a more elaborate command line.
Most conveniently this should be done as part
of a shell script or a |Makefile|.

When using \textsf{childdoc} in the main file, the following
command lines effectively perform a redirection
(note that depending on the shell being used,
backslashes may have to be doubled: `|\|' $\to$ `|\\|'):
%
\begin{center}
|... -jobname "|\textit{target}|" |\\|"|[\textit{flags}]%
|\input{childdoc.def}\childdocforward[|\textit{main}|]{|\textit{dest}|}"|
\end{center}
%
Here \textit{target} is the name of the output file,
\textit{main} is the name of the main file
and \textit{dest} is the name of the main or child file to be processed
(all filenames without extensions).
The optional argument \textit{main} can be omitted
if \textit{main} matches \textit{dest}.
Optionally, compilation \textit{flags} can be defined via |\def| commands.
This command line makes the \TeX{} engine believe
it is compiling the file \textit{target}
whose content is specified as the latter parameter.
The provided code then forwards the processing to
\textit{main} or \textit{dest} as described in \secref{sec:forward}.

%%%%%%%%%%%%%%%%%%%%%%%%%%%%%%%%%%%%%%%%%%%%%%%%%%%%%%%%%%%%%%%%%%%%%%%%%%%%%%%%
\subsection{Include by Input}
\label{sec:input}

Including child documents by |\include| has some restrictions by design.
Most notably, the content of a child document always occupies
its own set of pages; pages cannot be shared between child documents.
Usually, this behaviour makes perfect sense
because each child document contain an essential part of the document.
However, in some situations it may be desirable to compose
a document from a collection of parts
without having mandatory page breaks between then.
For this case, the package
provides a mechanism to include parts
by |\input| which can also be processed individually.
However, by construction this mechanism
requires manual handling of the content to be output.

%%%%%%%%%%%%%%%%%%%%%%%%%%%%%%%%%%%%%%%%
\DescribeMacro{\ifchilddocmanual}
The main file should be prepared as usual, see \secref{sec:include}.
However, the document body must make a distinction
between processing of an individual part and of the main document, e.g.:
%
\begin{center}
\begin{tabular}{l}
|\ifchilddocmanual|\\
|\input{\childdocname}|\\
|\||else|\\
\textit{document body with }|\input{|\textit{part}|}|\\
|\||fi|
\end{tabular}
\end{center}
%
The conditional |\ifchilddocmanual| is true whenever
a part to be included by |\input| is being compiled,
and the name of the part is stored in |\childdocname|.

%%%%%%%%%%%%%%%%%%%%%%%%%%%%%%%%%%%%%%%%
\DescribeMacro{\childdocby}
Each part to be included by |\input| should start with:
%
\begin{center}
\begin{tabular}{l}
|\input{childdoc.def}|\\
|\childdocby{|\textit{main}|}|\\
\end{tabular}
\end{center}
%
The directive |\childdocby| is similar to |\childdocof|
described in \secref{sec:include},
but the subsequent selection of content must be done manually.
To that end, both |\ifchilddoc| and |\ifchilddocmanual|
will be true upon processing of a part,
and the name of the part is stored in |\childdocname|.
Note that |\jobname| will be set to the filename of the current part
so that each part receives an individual |.aux| file
that does not interfere with the |.aux| file(s) of the main document.
This behaviour can be altered by the alternative form
|\childdocby[*]{|\textit{main}|}| (with a non-empty optional argument)
which uses the |.aux| file of the main document
by setting |\jobname| to \textit{main}.

%%%%%%%%%%%%%%%%%%%%%%%%%%%%%%%%%%%%%%%%%%%%%%%%%%%%%%%%%%%%%%%%%%%%%%%%%%%%%%%%
\subsection{Driver Development}
\label{sec:driver}

The \textsf{childdoc} mechanism can also be use for the development
of definition files such as \LaTeX{} styles or classes.
This case differs from the above setup with multiple parts
included by |\include| in that no |\includeonly| should be invoked.
This can be achieved by starting the include file
(before |\ProvidesPackage|) with:
%
\begin{center}
\begin{tabular}{l}
|\input{childdoc.def}|\\
|\childdocforward{|\textit{main}|}|\\
\end{tabular}
\end{center}
%
or alternatively with:
%
\begin{center}
\begin{tabular}{l}
|\input{childdoc.def}|\\
|\childdocby{|\textit{main}|}|\\
\end{tabular}
\end{center}
%
Both forms have slightly different effects as described above.
The main file is prepared as usual, see \secref{sec:include}.

%%%%%%%%%%%%%%%%%%%%%%%%%%%%%%%%%%%%%%%%%%%%%%%%%%%%%%%%%%%%%%%%%%%%%%%%%%%%%%%%
\subsection{Legacy Detection}
\label{sec:detection}

The directive |\childdocmain| in the main file can detect
whether the complete document or merely a child is to be compiled
even without using the directive |\childdocof|.
This method is deprecated because it is less robust
and there is no compelling reason to use it;
it is merely provided for backward compatibility
and it may be removed in future versions.

If the detection mechanism is to be used,
it is mandatory to correctly specify
the filename of the main file as the argument of |\childdocmain|:
%
\begin{center}
\begin{tabular}{l}
|\input{childdoc.def}|\\
|\childdocmain{|\textit{main}|}|\\
\end{tabular}
\end{center}
%
If |\jobname| does not match the argument \textit{main} of |\childdocmain|,
it is assumed that |\jobname| points to the child file to be compiled.
When using |\childdocmain| with the main file specified as argument,
it suffices to start a child file
with just |\input{|\textit{main}|}|
without loading of the package and using |\childdocof|.
If instead all processing is done
with the appropriate \textsf{childdoc} directives,
the argument of \textit{main} of |\childdocmain| can be empty.

An alternative version of the command line processing described
in \secref{sec:commandline} using the detection mechanism reads:
%
\begin{center}
|... -jobname "|\textit{target}|" "|[\textit{flags}]%
[|\def\jobname{|\textit{dest}|}|]|\input{|\textit{main}|}"|
\end{center}

%%%%%%%%%%%%%%%%%%%%%%%%%%%%%%%%%%%%%%%%%%%%%%%%%%%%%%%%%%%%%%%%%%%%%%%%%%%%%%%%
\subsection{Manual Code}
\label{sec:manual}

In case one cannot be certain whether the definitions file |childdoc.def|
is installed on the target \TeX{} distribution
and one prefers not to ship it,
it is conceivable to paste a few relevant commands into the sources.

To that end, drop all statements |\input{childdoc.def}|
and perform the replacements as outlined below.
Instead of |\childdocmain{|\textit{main}|}| add the following code
to the top of the main file:
%
\begin{center}
\begin{tabular}{l}
|\||ifdefined\childdocname\endinput\||fi\newif\ifchilddoc|\\
|\edef\childdocname{\scantokens\expandafter{\jobname\noexpand}}|\\
|\def\childdocmain{|\textit{main}|}\||ifx\childdocmain\childdocname\||else|\\
|\childdoctrue\includeonly{\childdocname}\let\jobname\childdocmain\||fi|\\
\end{tabular}
\end{center}
%
Instead of |\childdocof{|\textit{main}|}| just include the main file
at the top of each child file:
%
\begin{center}
|\input{|\textit{main}|}|
\end{center}
%
A simple redirection |\childdocforward{|\textit{dest}|}| is achieved by:
%
\begin{center}
|\def\jobname{|\textit{dest}|}\input{\jobname}|
\end{center}
%
The redirection with prefix
|\childdocforwardprefix[|\textit{prefix}|]{|\textit{dest}|}|
is accomplished by:
%
\begin{center}
\begin{tabular}{l}
|{\edef\jobname{\scantokens\expandafter{\jobname\noexpand}}|\\
|\def\redirectjob |\textit{prefix}|#1~~~{\gdef\jobname{|\textit{dest}|#1}}|\\
|\expandafter\redirectjob\jobname~~~}\input{\jobname}|
\end{tabular}
\end{center}

In an alternative approach,
child documents can be compiled by a specific command line
without additional code or specific definitions:
%
\begin{center}
|... -jobname "|\textit{target}|" "|[\textit{flags}]%
|\includeonly{|\textit{dest}|}\input{|\textit{main}|}"|
\end{center}
%

%%%%%%%%%%%%%%%%%%%%%%%%%%%%%%%%%%%%%%%%%%%%%%%%%%%%%%%%%%%%%%%%%%%%%%%%%%%%%%%%
%%%%%%%%%%%%%%%%%%%%%%%%%%%%%%%%%%%%%%%%%%%%%%%%%%%%%%%%%%%%%%%%%%%%%%%%%%%%%%%%
\section{Information}

%%%%%%%%%%%%%%%%%%%%%%%%%%%%%%%%%%%%%%%%%%%%%%%%%%%%%%%%%%%%%%%%%%%%%%%%%%%%%%%%
\subsection{Copyright}

Copyright \copyright{} 2017--2018 Niklas Beisert

This work may be distributed and/or modified under the
conditions of the \LaTeX{} Project Public License, either version 1.3
of this license or (at your option) any later version.
The latest version of this license is in
  \url{http://www.latex-project.org/lppl.txt}
and version 1.3 or later is part of all distributions of \LaTeX{}
version 2005/12/01 or later.

This work has the LPPL maintenance status `maintained'.

The Current Maintainer of this work is Niklas Beisert.

This work consists of the files |README.txt|, |childdoc.ins| and |childdoc.dtx|
as well as the derived files |childdoc.def|, |cdocsamp.tex|
with |cdocsch1.tex|, |cdocsch2.tex|, |cdocspt3.tex|, |cdocspt4.tex|,
|cdocsdrf.tex|, |cdocsfn1.tex|, |cdocsfn2.tex|
as well as |childdoc.pdf|.

%%%%%%%%%%%%%%%%%%%%%%%%%%%%%%%%%%%%%%%%%%%%%%%%%%%%%%%%%%%%%%%%%%%%%%%%%%%%%%%%
\subsection{Files and Installation}

The package consists of the files:
%
\begin{center}
\begin{tabular}{ll}
    |README.txt|   & readme file \\
    |childdoc.ins| & installation file \\
    |childdoc.dtx| & source file \\
    |childdoc.def| & definition file \\
    |cdocsamp.tex| & sample main file \\
    |cdocsch1.tex| & sample include file \\
    |cdocsch2.tex| & sample include file \\
    |cdocspt3.tex| & sample part file \\
    |cdocspt4.tex| & sample part file \\
    |cdocsdrf.tex| & sample redirection file \\
    |cdocsfn1.tex| & sample redirection file \\
    |cdocsfn2.tex| & sample redirection file \\
    |childdoc.pdf| & manual
\end{tabular}
\end{center}
%
The distribution consists of the files
|README.txt|, |childdoc.ins| and |childdoc.dtx|.
%
\begin{itemize}
\item
Run (pdf)\LaTeX{} on |childdoc.dtx|
to compile the manual |childdoc.pdf| (this file).
\item
Run \LaTeX{} on |childdoc.ins| to create the definitions file |childdoc.def|
and the sample |cdocsamp.tex| with include files
|cdocsch1.tex|, |cdocsch2.tex|, |cdocspt3.tex|, |cdocspt4.tex|,
|cdocsdrf.tex|, |cdocsfn1.tex|, |cdocsfn2.tex|.
Then copy the file |childdoc.def| to an appropriate directory of your \LaTeX{}
distribution, e.g.\ \textit{texmf-root}|/tex/latex/childdoc|.
\end{itemize}

%%%%%%%%%%%%%%%%%%%%%%%%%%%%%%%%%%%%%%%%%%%%%%%%%%%%%%%%%%%%%%%%%%%%%%%%%%%%%%%%
\subsection{Related CTAN Packages}

There are several other packages which offer a similar functionality:
%
\begin{itemize}
\item
The packages
\href{http://ctan.org/pkg/docmute}{\textsf{docmute}},
\href{http://ctan.org/pkg/includex}{\textsf{includex}} and
\href{http://ctan.org/pkg/standalone}{\textsf{standalone}}
provide commands to include only the document body of
a child file thus allowing both files to be compiled individually.
\item
The packages \href{http://ctan.org/pkg/subdocs}{\textsf{subdocs}}
and \href{http://ctan.org/pkg/subfiles}{\textsf{subfiles}}
provide structures in which the main and child documents can be
encapsulated and allowing them to be compiled individually.
The inclusion mechanism is different from the conventional |\include|.
\item
The package \href{http://ctan.org/pkg/combine}{\textsf{combine}}
is an elaborate solution to combine several documents into one.
\end{itemize}
%
See also the CTAN topic \href{http://ctan.org/topic/subdocs}{\textsf{subdocs}}
for further related packages.
The present package differs from the above solutions in that
a document structure constructed with the conventional |\include| mechanism
just needs two extra commands at the top of every file
such that all constituent files can be compiled individually.

%%%%%%%%%%%%%%%%%%%%%%%%%%%%%%%%%%%%%%%%%%%%%%%%%%%%%%%%%%%%%%%%%%%%%%%%%%%%%%%%
%\subsection{Feature Suggestions}
%
%The following is a list of features which may be useful for future
%versions of this package:
%%
%\begin{itemize}
%\item
%\ldots
%\end{itemize}

%%%%%%%%%%%%%%%%%%%%%%%%%%%%%%%%%%%%%%%%%%%%%%%%%%%%%%%%%%%%%%%%%%%%%%%%%%%%%%%%
\subsection{Revision History}

%%%%%%%%%%%%%%%%%%%%%%%%%%%%%%%%%%%%%%%%
\paragraph{v2.0:} 2018/12/30

\begin{itemize}
\item
immediate forward processing
\item
added |\childdocby| mechanism
\item
manual restructured
\end{itemize}

%%%%%%%%%%%%%%%%%%%%%%%%%%%%%%%%%%%%%%%%
\paragraph{v1.6:} 2018/01/17

\begin{itemize}
\item
application for development of include files
\item
corrections to manual
\end{itemize}

%%%%%%%%%%%%%%%%%%%%%%%%%%%%%%%%%%%%%%%%
\paragraph{v1.5:} 2017/05/21

\begin{itemize}
\item
more complete structuring introduced
\item
|\childdocof| introduced
\item
|\childdoc| renamed to |\childdocmain|
\item
|\childredirect| renamed to |\childdocforward| and |\childdocforwardprefix|
and functionality expanded
\end{itemize}

%%%%%%%%%%%%%%%%%%%%%%%%%%%%%%%%%%%%%%%%
\paragraph{v1.0:} 2017/04/27

\begin{itemize}
\item
manual and install package
\item
first version published on CTAN
\end{itemize}

%%%%%%%%%%%%%%%%%%%%%%%%%%%%%%%%%%%%%%%%
\paragraph{v0.6:} 2017/04/26

\begin{itemize}
\item
redirection mechanism added
\end{itemize}

%%%%%%%%%%%%%%%%%%%%%%%%%%%%%%%%%%%%%%%%
\paragraph{v0.5:} 2017/04/26

\begin{itemize}
\item
functionality in definition file
\end{itemize}


%%%%%%%%%%%%%%%%%%%%%%%%%%%%%%%%%%%%%%%%%%%%%%%%%%%%%%%%%%%%%%%%%%%%%%%%%%%%%%%%
%%%%%%%%%%%%%%%%%%%%%%%%%%%%%%%%%%%%%%%%%%%%%%%%%%%%%%%%%%%%%%%%%%%%%%%%%%%%%%%%
%%%%%%%%%%%%%%%%%%%%%%%%%%%%%%%%%%%%%%%%%%%%%%%%%%%%%%%%%%%%%%%%%%%%%%%%%%%%%%%%
\appendix

\settowidth\MacroIndent{\rmfamily\scriptsize 000\ }

 \DocInput{childdoc.dtx}

\end{document}
%</driver>
% \fi
%
% %%%%%%%%%%%%%%%%%%%%%%%%%%%%%%%%%%%%%%%%%%%%%%%%%%%%%%%%%%%%%%%%%%%%%%%%%%%%%%
% %%%%%%%%%%%%%%%%%%%%%%%%%%%%%%%%%%%%%%%%%%%%%%%%%%%%%%%%%%%%%%%%%%%%%%%%%%%%%%
% \section{Sample}
%\iffalse
%<*samplemain>
%\fi
%
% The following presents a sample document
% with two chapters, two parts, a title page,
% a compile flag as well as three forwarding files to set the flag.
% It consists of eight |.tex| files:
% \begin{center}
% \begin{tabular}{ll}
% |cdocsamp.tex|&main file\\
% |cdocsch1.tex|&include file for chapter 1\\
% |cdocsch2.tex|&include file for chapter 2\\
% |cdocspt3.tex|&include file for part 3\\
% |cdocspt4.tex|&include file for part 4\\
% |cdocsdrf.tex|&forwarding file for main file in draft mode\\
% |cdocsfi1.tex|&forwarding file for final version of chapter 1\\
% |cdocsfi2.tex|&forwarding file for final version of chapter 2\\
% \end{tabular}
% \end{center}
% Each of the eight files can be compiled directly by the \LaTeX{} compiler.
%
% %%%%%%%%%%%%%%%%%%%%%%%%%%%%%%%%%%%%%%
% \paragraph{Main File.}
%
% The main file is called |cdocsamp.tex|.
%
% Load the \textsf{childdoc} definitions and
% declare the filename for the main document:
%    \begin{macrocode}
\input{childdoc.def}
\childdocmain{}
%    \end{macrocode}

% Optional override for |\version| flag:
%    \begin{macrocode}
%%\ifchilddoc\else\providecommand{\version}{draft}\fi
%    \end{macrocode}

% Define the default values for the |\version| flag
% (|final| for the main file and |draft| for childs):
%    \begin{macrocode}
\ifchilddoc
\providecommand{\version}{draft}
\else
\providecommand{\version}{final}
\fi
%    \end{macrocode}

% Load the standard document class:
%    \begin{macrocode}
\documentclass[12pt]{article}
%    \end{macrocode}

% Start the document body:
%    \begin{macrocode}
\begin{document}
%    \end{macrocode}

% Declare a title page.
% Print title, part of document being processed and version flag:
%    \begin{macrocode}
\addtocounter{page}{-1}
\begin{center}
{\LARGE\bfseries{}childdoc example\par}
\vspace{1cm}
\ifchilddoc
\ifchilddocmanual part\else chapter\fi:
`\childdocname' of `\childdocjob'\par
\else
main document: `\childdocjob'\par
\fi
version: \version\par
\end{center}
\newpage
%    \end{macrocode}

% Manually include selected file,
% otherwise process as usual:
%    \begin{macrocode}
\ifchilddocmanual
\section*{part `\childdocname'}
\input{\childdocname}
\else
%    \end{macrocode}

% Include the two chapters:
%    \begin{macrocode}
\include{cdocsch1}
\include{cdocsch2}
%    \end{macrocode}

% Include the two parts unless only chapters should be displayed:
%    \begin{macrocode}
\ifchilddoc\else
\section{part three}
\input{cdocspt3}
\section{part four}
\input{cdocspt4}
\fi
%    \end{macrocode}

% Process as usual until here:
%    \begin{macrocode}
\fi
%    \end{macrocode}

% End of document body:
%    \begin{macrocode}
\end{document}
%    \end{macrocode}
%\iffalse
%</samplemain>
%\fi
%
% %%%%%%%%%%%%%%%%%%%%%%%%%%%%%%%%%%%%%%
% \paragraph{Chapter Include Files.}
%
% The include files are called |cdocsch1.tex| and |cdocsch2.tex|.
%
%\iffalse
%<*samplechap1|samplechap2>
%\fi

% Optional override for |\version| flag:
%    \begin{macrocode}
%%\providecommand{\version}{final}
%    \end{macrocode}

% Include the main document:
%    \begin{macrocode}
\input{childdoc.def}
\childdocof{cdocsamp}
%    \end{macrocode}

%\iffalse
%</samplechap1|samplechap2>
%\fi
%
%\iffalse
%<*samplechap1>
%\fi
% Some text for chapter 1:
%    \begin{macrocode}
\section{one}
some text in chapter one
%    \end{macrocode}

%\iffalse
%</samplechap1>
%\fi
% Some text for chapter 2:
%\iffalse
%<*samplechap2>
%\fi
%    \begin{macrocode}
\section{two}
more text in chapter two
%    \end{macrocode}

%\iffalse
%</samplechap2>
%\fi
%
% %%%%%%%%%%%%%%%%%%%%%%%%%%%%%%%%%%%%%%
% \paragraph{Part Include Files.}
%
% The include files are called |cdocspt3.tex| and |cdocspt4.tex|.
%
%\iffalse
%<*samplepart3|samplepart4>
%\fi

% Optional override for |\version| flag:
%    \begin{macrocode}
%%\providecommand{\version}{final}
%    \end{macrocode}

% Include the main document:
%    \begin{macrocode}
\input{childdoc.def}
\childdocby{cdocsamp}
%    \end{macrocode}

%\iffalse
%</samplepart3|samplepart4>
%\fi
%
%\iffalse
%<*samplepart3>
%\fi
% Some text for part 3:
%    \begin{macrocode}
some text in part three
%    \end{macrocode}

%\iffalse
%</samplepart3>
%\fi
% Some text for part 4:
%\iffalse
%<*samplepart4>
%\fi
%    \begin{macrocode}
more text in part four
%    \end{macrocode}

%\iffalse
%</samplepart4>
%\fi
%
% %%%%%%%%%%%%%%%%%%%%%%%%%%%%%%%%%%%%%%
% \paragraph{Forwarding for a Complete Draft.}
%
% The following forwarding file |cdocsdrf.tex|
% compiles the main document in draft mode:
%\iffalse
%<*sampledraft>
%\fi
%    \begin{macrocode}
\def\version{draft}
\input{childdoc.def}
\childdocforward{cdocsamp}
%    \end{macrocode}

%\iffalse
%</sampledraft>
%\fi
%
% %%%%%%%%%%%%%%%%%%%%%%%%%%%%%%%%%%%%%%
% \paragraph{Forwarding for Final Version of the Chapters.}
%
% The following forwarding files |cdocsfn1.tex| and |cdocsfn2.tex|
% (with identical content)
% compile the final versions of the child documents
% |cdocsch1.tex| and |cdocsch2.tex|, respectively:
%\iffalse
%<*samplefinal>
%\fi
%    \begin{macrocode}
\def\version{final}
\input{childdoc.def}
\childdocforwardprefix[cdocsamp]{cdocsfn}{cdocsch}
%    \end{macrocode}

%\iffalse
%</samplefinal>
%\fi
%
% %%%%%%%%%%%%%%%%%%%%%%%%%%%%%%%%%%%%%%
% \paragraph{Command Line Processing.}
%
% The following three command lines generate the output files
% |cdocscld|, |cdocscl1| and |cdocscl2|
% which should be identical to
% |cdocsdrf|, |cdocsch1| and |cdocsfn2|, respectively:
% \begin{center}
% \begin{tabular}{l}
% |latex -jobname cdocscld \|\\
% |  "\def\version{draft}\input{childdoc.def}\childdocforward{cdocsamp}"|\\
% |latex -jobname cdocscl1 \|\\
% |  "\input{childdoc.def}\childdocforward[cdocsamp]{cdocsch1}"|\\
% |latex -jobname cdocscl2 \|\\
% |  "\def\version{final}\input{childdoc.def}\childdocforward{cdocsch2}"|
% \end{tabular}
% \end{center}
% Note that the trailing backslash on each first line
% merely continues the input to the second line
% (for convenient cut ant paste).
% Furthermore, the command |latex| can be replaced by any
% of its alternative versions such as |pdflatex|.
%
% %%%%%%%%%%%%%%%%%%%%%%%%%%%%%%%%%%%%%%%%%%%%%%%%%%%%%%%%%%%%%%%%%%%%%%%%%%%%%%
% %%%%%%%%%%%%%%%%%%%%%%%%%%%%%%%%%%%%%%%%%%%%%%%%%%%%%%%%%%%%%%%%%%%%%%%%%%%%%%
% \section{Implementation}
%\iffalse
%<*package>
%\fi
%
% This section describes the definitions file |childdoc.def|.

% The definitions cannot be loaded using |\usepackage| or |\RequirePackage|
% which has a mechanism to prevent loading a style file more than once.
% When loading the definitions by means of |\input|
% multiple instances have to be prevented manually:
%\iffalse
%This code needs to be before the `\ProvidesFile' directive
%which is defined at the beginning of this file.
%Therefore it is also placed there and commented out here.
%</package>
%<*discard>
%\fi
%    \begin{macrocode}
\ifdefined\childdocmain\endinput\fi
%    \end{macrocode}
%\iffalse
%</discard>
%<*package>
%\fi
%
% \macro{\ifchilddoc}
% \macro{\ifchilddocmanual}
% The conditional |\ifchilddoc| tells whether a
% child (true) or main (false) document is being compiled.
% The conditional |\ifchilddocmanual| tells whether
% the |\includeonly| mechanism is used (false) or
% the selection of child files must be performed manually (true).
% The definitions initialise to false:
%    \begin{macrocode}
\newif\ifchilddoc
\newif\ifchilddocmanual
%    \end{macrocode}

% \macro{\childdocname}
% \macro{\childdocjob}
% The macro |\childdocname| stores the name of the main document
% to be compiled. The macro |\childdocjob| stores the name of
% the document on which the \LaTeX{} compiler was originally invoked.
% The content of |\jobname| cannot be compared
% to filenames specified in the source due to different catcodes.
% The following code rescans |\jobname|, stores the result
% in |\childdocname| and saves a copy in |\childdocjob|:
%    \begin{macrocode}
\edef\childdocname{\scantokens\expandafter{\jobname\noexpand}}
\let\childdocjob\childdocname
%    \end{macrocode}

% \macro{\childdocdisable}
% The macro |\childdocdisable| prevents the main file
% from being processed more than once.
% At this stage, the main document command |\childdocmain|
% is assumed to be called once again where it should do nothing.
% Any subsequent call to it should prevent
% a secondary processing of the main document
% It overwrites the forwarding commands
% |\childdocof| and |\childdocforward|
% with empty macros to prevent further inclusions of the main document:
%    \begin{macrocode}
\newcommand{\childdocdisable}
{
  \renewcommand{\childdocmain}[1]{\renewcommand{\childdocmain}[1]{\endinput}}
  \renewcommand{\childdocof}[1]{}
  \renewcommand{\childdocby}[2][]{}
  \renewcommand{\childdocforward}[2][]{}
  \renewcommand{\childdocdisable}{}
}
%    \end{macrocode}

% \macro{\childdocmain}
% The macro |\childdocmain| is to be called at the top of the main file
% with nothing or the main filename (without extension) as argument.
% First, it breaks loops.
% If the argument is not empty and does not match |\childdocname|
% (which is set by the first inclusion of |childdoc.def|),
% |\ifchilddoc| is set to true, |\includeonly| is applied to the child file
% and |\jobname| is set to the main file
% (for proper handling of |.aux| files):
%    \begin{macrocode}
\newcommand{\childdocmain}[1]
{
  \childdocdisable\childdocmain{}
  \if?#1?\else
    \begingroup
      \def\childdoctmp{#1}
      \ifx\childdoctmp\childdocname
        \def\childdoctmp{}
      \else
        \def\childdoctmp
        {
          \childdoctrue
          \includeonly{\childdocname}
          \def\childdocjob{#1}
          \def\jobname{#1}
        }
      \fi
      \expandafter
    \endgroup
    \childdoctmp
  \fi
}
%    \end{macrocode}

% \macro{\childdocof}
% The command |\childdocof| redirects
% compilation to the main file |#1|.
%    \begin{macrocode}
\newcommand{\childdocof}[1]
{
  \childdocdisable
  \childdoctrue
  \includeonly{\childdocname}
  \def\jobname{#1}
  \def\childdocjob{#1}
  \input{#1}
}
%    \end{macrocode}

% \macro{\childdocby}
% The command |\childdocby| ....
%    \begin{macrocode}
\newcommand{\childdocby}[2][]
{
  \childdocdisable
  \childdoctrue
  \childdocmanualtrue
  \if?#1?\else
    \def\jobname{#2}
  \fi
  \def\childdocjob{#2}
  \input{#2}
  \endinput
}
%    \end{macrocode}

% \macro{\childdocforward}
% The command |\childdocforward| redirects
% compilation to the main file or
% (if the optional argument is given) a child file.
% Parameters are set as if the main file
% or a child file starting with |\childdocof| was compiled.
% Then compilation is handed over to the main file:
%    \begin{macrocode}
\newcommand{\childdocforward}[2][]
{
  \begingroup
    \if?#1?
      \def\childdoctmp
      {
        \def\childdocname{#2}
        \def\childdocjob{#2}
        \def\jobname{#2}
        \input{#2}
        \endinput
      }
    \else
      \def\childdoctmp
      {
        \childdocdisable
        \def\childdocname{#2}
        \childdoctrue
        \includeonly{#2}
        \def\childdocjob{#1}
        \def\jobname{#1}
        \input{#1}
        \endinput
      }
    \fi
    \expandafter
  \endgroup
  \childdoctmp
}
%    \end{macrocode}

% \macro{\childdocforwardprefix}
% The command |\childdocforwardprefix| redirects
% compilation to the main or a child file by means of a pattern.
% The prefix |#1| in the current filename is replaced by |#2|
% and the suffix of the current filename is kept
% (it is assumed that the filename does not contain the substring `|~~~|'
% which is used as a delimiter).
% Compilation is handed over to the new file by |\childdocforward|:
%    \begin{macrocode}
\newcommand{\childdocforwardprefix}[3][]
{
  \begingroup
    \def\childdocextract #2##1~~~{\def\childdoctmp{\childdocforward[#1]{#3##1}}}
    \expandafter\childdocextract\childdocname~~~
    \expandafter
  \endgroup
  \childdoctmp
}
%    \end{macrocode}

% \macro{\childdoc}
% The deprecated macro |\childdoc| is a legacy version of |\childdocmain|:
%    \begin{macrocode}
\newcommand{\childdoc}{\childdocmain}
%    \end{macrocode}

% \macro{\childdocredirect}
% The deprecated macro |\childdocredirect| is a legacy version
% of |\childdocforward| and |\childdocforwardprefix|:
%    \begin{macrocode}
\newcommand{\childdocredirect}[2][]
{
  \begingroup
    \if?#1?
      \def\childdoctmp{\childdocforward{#2}}
    \else
      \def\childdoctmp{\childdocforwardprefix{#1}{#2}}
    \fi
    \expandafter
  \endgroup
  \childdoctmp
}
%    \end{macrocode}

%\iffalse
%</package>
%\fi
%
\endinput
|\\
|\childdocforward{|\textit{main}|}|
\end{tabular}
\end{center}
%
Likewise, the following files |final|\textit{nn}|.tex|
compile the final version of the child document
|child|\textit{nn}|.tex|:
%
\begin{center}
\begin{tabular}{l}
|\def\version{final}|\\
|% \iffalse
%
% childdoc.dtx Copyright (C) 2017-2018 Niklas Beisert
%
% This work may be distributed and/or modified under the
% conditions of the LaTeX Project Public License, either version 1.3
% of this license or (at your option) any later version.
% The latest version of this license is in
%   http://www.latex-project.org/lppl.txt
% and version 1.3 or later is part of all distributions of LaTeX
% version 2005/12/01 or later.
%
% This work has the LPPL maintenance status `maintained'.
%
% The Current Maintainer of this work is Niklas Beisert.
%
% This work consists of the files childdoc.dtx and childdoc.ins
% and the derived files childdoc.def and cdocsamp.tex with
% cdocsch1.tex, cdocsch2.tex, cdocsdrf.tex, cdocsfn1.tex, cdocsfn2.tex.
%
%<package>\ifdefined\childdocmain\endinput\fi
%<package>\ProvidesFile{childdoc.def}[2018/12/30 v2.0 child document driver]
%<samplemain>\ProvidesFile{cdocsamp.tex}[2018/12/30 v2.0 sample for childdoc]
%<*driver>
%\ProvidesFile{childdoc.drv}[2018/12/30 v2.0 childdoc reference manual file]
\PassOptionsToClass{10pt,a4paper}{article}
\documentclass{ltxdoc}

\usepackage[margin=35mm]{geometry}
\usepackage{hyperref}
\usepackage{hyperxmp}
\usepackage[usenames]{color}

\hypersetup{colorlinks=true}
\hypersetup{pdfstartview=FitH}
\hypersetup{pdfpagemode=UseNone}
\hypersetup{pdfsource={}}
\hypersetup{pdflang={en-UK}}
\hypersetup{pdfcopyright={Copyright 2017-2018 Niklas Beisert.
  This work may be distributed and/or modified under the
  conditions of the LaTeX Project Public License, either version 1.3
  of this license or (at your option) any later version.}}
\hypersetup{pdflicenseurl={http://www.latex-project.org/lppl.txt}}
\hypersetup{pdfcontactaddress={ETH Zurich, ITP, HIT K,
  Wolfgang-Pauli-Strasse 27}}
\hypersetup{pdfcontactpostcode={8093}}
\hypersetup{pdfcontactcity={Zurich}}
\hypersetup{pdfcontactcountry={Switzerland}}
\hypersetup{pdfcontactemail={nbeisert@itp.phys.ethz.ch}}
\hypersetup{pdfcontacturl={http://people.phys.ethz.ch/\xmptilde nbeisert/}}

\newcommand{\secref}[1]{\hyperref[#1]{section \ref*{#1}}}

\parskip1ex
\parindent0pt
\let\olditemize\itemize
\def\itemize{\olditemize\parskip0pt}

\begin{document}

\title{The \textsf{childdoc} Package}
\hypersetup{pdftitle={The childdoc Package}}
\author{Niklas Beisert\\[2ex]
  Institut f\"ur Theoretische Physik\\
  Eidgen\"ossische Technische Hochschule Z\"urich\\
  Wolfgang-Pauli-Strasse 27, 8093 Z\"urich, Switzerland\\[1ex]
  \href{mailto:nbeisert@itp.phys.ethz.ch}
  {\texttt{nbeisert@itp.phys.ethz.ch}}}
\hypersetup{pdfauthor={Niklas Beisert}}
\hypersetup{pdfsubject={Manual for the LaTeX2e Package childdoc}}
\date{30 December 2018, \textsf{v2.0}}
\maketitle

\begin{abstract}\noindent
\textsf{childdoc} is a \LaTeXe{} package
that enables the direct compilation
of document sections included by |\include|
to individual files.
\end{abstract}

\begingroup
\parskip0ex
\tableofcontents
\endgroup

%%%%%%%%%%%%%%%%%%%%%%%%%%%%%%%%%%%%%%%%%%%%%%%%%%%%%%%%%%%%%%%%%%%%%%%%%%%%%%%%
%%%%%%%%%%%%%%%%%%%%%%%%%%%%%%%%%%%%%%%%%%%%%%%%%%%%%%%%%%%%%%%%%%%%%%%%%%%%%%%%
\section{Introduction}

\LaTeX{} provides a mechanism to structure a large document (such as a book)
into a main file and several child files (containing the chapters)
using the |\include| command.
This mechanism is beneficial for documents
which span hundreds of pages in order to
make the source file(s) more manageable.
Moreover, compilation can be restricted to
selected child files by means of the |\includeonly| command.
The latter feature can be used to reduce the compilation time while editing
(this was significantly more useful in the earlier days of \LaTeX{})
or to generate a smaller document which is easier to navigate.
Another application of |\includeonly| is to generate
documents consisting of selected parts of the complete document.

However, there are a few drawbacks of the plain |\include| mechanism:
\begin{itemize}
\item
The child files cannot be compiled on their own,
they can only be compiled via the main file.
A naive editing environment
(such as a text editor with an option
to have the current file processed by \LaTeX)
may require one to switch to the main file before compiling;
attempting to compile the child file produces errors.
\item
The main file must be modified (each time)
to adjust the |\includeonly| command
to the present needs. This easily leaves the main file in a messy state.
\item
The generated document will always carry the filename
of the main document. This is inconvenient if
several child files are to be compiled and
to be kept for distribution.
\end{itemize}

The present package provides a simple interface
to make child files individually compilable by \LaTeX{}.
Compiling a child file then has the same effect as compiling
the main file with an |\includeonly| command
to select the appropriate child.
Moreover the generated document will carry the name of the child
rather than the main file.
This resolves all three above issues.

This feature is meant to make the editing of books,
thesis documents and lecture notes somewhat more convenient.
However, the package can also be used efficiently for
composing a series of documents (such as exercise sheets)
which are typically distributed individually.
It then assists the author in generating the individual documents
(potentially in different versions)
as well as a document containing the collected series.
Another application is in developing style files
or other kinds of included material
where compilation of the style file could redirect
to a sample or test file.

%%%%%%%%%%%%%%%%%%%%%%%%%%%%%%%%%%%%%%%%%%%%%%%%%%%%%%%%%%%%%%%%%%%%%%%%%%%%%%%%
%%%%%%%%%%%%%%%%%%%%%%%%%%%%%%%%%%%%%%%%%%%%%%%%%%%%%%%%%%%%%%%%%%%%%%%%%%%%%%%%
\section{Usage}

First of all, the package \textsf{childdoc} is \emph{not} a standard
\LaTeXe{} |.sty| style file! Therefore it needs to be invoked in
a non-standard way.

%%%%%%%%%%%%%%%%%%%%%%%%%%%%%%%%%%%%%%%%%%%%%%%%%%%%%%%%%%%%%%%%%%%%%%%%%%%%%%%%
\subsection{Included Files}
\label{sec:include}

%%%%%%%%%%%%%%%%%%%%%%%%%%%%%%%%%%%%%%%%
\DescribeMacro{\childdocmain}
To use the package, add the commands
\begin{center}
\begin{tabular}{l}
|\input{childdoc.def}|\\
|\childdocmain{}|\\
\end{tabular}
\end{center}
at the very top of the main \LaTeX{} file,
in particular \emph{before} the |\documentclass| statement!
The argument of |\childdocmain| should be left empty
(but it must be present).

%%%%%%%%%%%%%%%%%%%%%%%%%%%%%%%%%%%%%%%%
\DescribeMacro{\childdocof}
Furthermore, add the commands
\begin{center}
\begin{tabular}{l}
|\input{childdoc.def}|\\
|\childdocof{|\textit{main}|}|\\
\end{tabular}
\end{center}
at the top of every child file \textit{child}
which is included by |\include{|\textit{child}|}|
from within the main file
(or at least for those files to be compiled individually).
The argument \textit{main} must be the filename of the main file.

There are a couple of
considerations in setting up the main and child documents:

%%%%%%%%%%%%%%%%%%%%%%%%%%%%%%%%%%%%%%%%
\paragraph{Restrictions.}

Please note the following restrictions:
\begin{itemize}
\item
|\childdocmain| must be called with one argument \textit{main}
to ensure compatibility with earlier version of the package.
It must either be empty (|\childdocmain{}|)
or precisely match the filename of the main file in which it is specified.
See \secref{sec:detection} for further information.
\item
The filename \textit{main} must be specified without the |.tex| extension.
\item
The filename \textit{main} is case sensitive
(even in case-insensitive file systems)
due to internal string comparison.
\item
The argument \textit{main} should be fully expanded, it cannot be a macro.
\item
Subdirectories and special characters should be avoided in filenames.
\item
The command |\childdocmain{|\textit{main}|}| must be followed by a whitespace.
It should not be followed immediately by another command
or by a comment mark `|%|'.
This is because the \TeX{} parser reads the token immediately following
the argument of |\childdocmain| and puts it
at the beginning of every child section;
however, a white\-space is ignored.
\end{itemize}

%%%%%%%%%%%%%%%%%%%%%%%%%%%%%%%%%%%%%%%%
\paragraph{Content of Main File.}

It is advisable to place all content in the child files included by |\include|.
Any output contained in the main file will appear in all child documents
unless suppressed manually;
it cannot be suppressed automatically by the |\includeonly| directive
and thus should normally be avoided.
A method to include some content in the main file
by means of conditional processing is described in \secref{sec:conditional}.

%%%%%%%%%%%%%%%%%%%%%%%%%%%%%%%%%%%%%%%%
\paragraph{Page Numbering.}

When only a part of the document is compiled,
the appropriate numbering of pages
(as well as other status parameters)
is determined from the |.aux| files.
The latter contain information from previous passes.
However this information needs to propagate through
all intermediate child documents.
Therefore the page numbering in child documents may well
be inconsistent until the complete document is compiled at least once.

A useful (if unconventional) way to always ensure a consistent
page numbering is to restart the numbering in each child document
and denote the pages by `\textit{child}|.|\textit{page}'
where \textit{child} represents the chapter/section number of the child file.
This can be achieved by the command
|\numberwithin{page}{|\textit{child}|}|
of the \textsf{amsmath} package
where \textit{child} can be |chapter| or |section|
depending on the chosen structuring.
Alternatively, one can modify the macro |\thepage| appropriately
and reset the counter |page| at the start of each child file.

%%%%%%%%%%%%%%%%%%%%%%%%%%%%%%%%%%%%%%%%%%%%%%%%%%%%%%%%%%%%%%%%%%%%%%%%%%%%%%%%
\subsection{Conditional Processing}
\label{sec:conditional}

The package provides a mechanism to compile different versions
of a document. To customise the versions further some conditional processing
can come in handy to distinguish which version is being compiled.
The package provides two macros to describe the compilation context:

%%%%%%%%%%%%%%%%%%%%%%%%%%%%%%%%%%%%%%%%
\DescribeMacro{\ifchilddoc}
The conditional |\ifchilddoc| distinguishes between the compilation of
child documents and the main document:
%
\begin{center}
|\ifchilddoc |\textit{child-code}| |[|\||else |\textit{main-code}]| \||fi|
\end{center}

%%%%%%%%%%%%%%%%%%%%%%%%%%%%%%%%%%%%%%%%
\DescribeMacro{\childdocname}
\DescribeMacro{\childdocjob}
The macro |\childdocname| contains the filename (without extension)
of the main or child file being processed.
Note that |\childdocjob| will always contain the name of the main file.

%%%%%%%%%%%%%%%%%%%%%%%%%%%%%%%%%%%%%%%%
\paragraph{Title Page.}

Conditional processing can be used to include a title or banner page
in the main document when proper precautions are taken.
Importantly, the code in the main file should ensure that the page counter
(as well as other status parameters which are stored in the |.aux| files)
takes the same value after the conditional processing.
Otherwise the page numbers may take divergent values
depending on which part is compiled.

For example, a title page could be declared by:
%
\begin{center}
\begin{tabular}{l}
|\ifchilddoc\||else|\\
|\addtocounter{page}{-1}|\\
\textit{code for title page}\\
|\newpage|\\
|\||fi|
\end{tabular}
\end{center}
%
A banner page for the child documents can be generated by:
%
\begin{center}
\begin{tabular}{l}
|\ifchilddoc|\\
|\addtocounter{page}{-1}|\\
\textit{code for banner page}\\
|\newpage|\\
|\||fi|
\end{tabular}
\end{center}
%
Here one could write a message such as:
\begin{center}
|This is the part \childdocname{} of \childdocjob{}.|
\end{center}

%%%%%%%%%%%%%%%%%%%%%%%%%%%%%%%%%%%%%%%%%%%%%%%%%%%%%%%%%%%%%%%%%%%%%%%%%%%%%%%%
\subsection{Flags}
\label{sec:flags}

The package makes it easy to generate different versions
of the main or child documents.
To this end compilation flags can be defined
and assigned different default values.
They will be particularly useful in conjunction
with the forwarding mechanism described in \secref{sec:forward}.

For example, it may be useful to have a flag |\version|
which can be set to |draft| or |final|.
The document source will contain some conditional code
depending on the value of |\version|.
Suppose further, the flag should default to |final| for the main file
and to |draft| for child files
which is a natural assignment for editing the document.
This is achieved by placing the following code
in the preamble of the main document
(below the |\childdocmain| directive):
%
\begin{center}
\begin{tabular}{l}
|\ifchilddoc|\\
|\providecommand{\version}{draft}|\\
|\||else|\\
|\providecommand{\version}{final}|\\
|\||fi|
\end{tabular}
\end{center}
%
The definition by |\providecommand| makes sure
that previous definitions are not overwritten.
Further statements |\providecommand{\version}{...}|
can thus be added before the above code to override it.

For the main file, one might add a line
(between |\childdocmain| and the above block)
%
\begin{center}
|%\ifchilddoc\||else\providecommand{\version}{draft}\||fi|
\end{center}
%
which can be uncommented to produce a draft version.
Likewise one can add a line to the very top of a child file
(above the |\childdocof{|\textit{main}|}| directive)
%
\begin{center}
|%\providecommand{\version}{final}|
\end{center}
%
which can be uncommented to produce the final version of this child document.

%%%%%%%%%%%%%%%%%%%%%%%%%%%%%%%%%%%%%%%%%%%%%%%%%%%%%%%%%%%%%%%%%%%%%%%%%%%%%%%%
\subsection{Forwarding}
\label{sec:forward}

Different versions of the main or child documents
using compilation flags as described in \secref{sec:flags}
can be (permanently) stored in different files
for convenient compilation, viewing and distribution.
To this end, the package defines a command
to pass on compilation to a different file:

%%%%%%%%%%%%%%%%%%%%%%%%%%%%%%%%%%%%%%%%
\DescribeMacro{\childdocforward}
The command |\childdocforward| redirects processing to
another source file:
%
\begin{center}
\begin{tabular}{l}
|\input{childdoc.def}|\\
|\childdocforward[|\textit{main}|]{|\textit{dest}|}|\\
\end{tabular}
\end{center}
%
The argument \textit{dest} is the destination file
(without extension).
It should be the main file or one of the child files.
Note that further \textsf{childdoc} directives
such as |\childdocof| and |\childdocforward|
in the indicated file will be processed in this form.
The optional argument \textit{main}
passes on directly to the main file \textit{main}
while pretending to compile the child \textit{dest}.
This form behaves as if \textit{dest}
issues |\childdocof{|\textit{main}|}| right away,
and no further \textsf{childdoc} directives will be processed.

%%%%%%%%%%%%%%%%%%%%%%%%%%%%%%%%%%%%%%%%
\DescribeMacro{\...prefix}
In the alternative form |\childdocforwardprefix|,
%
\begin{center}
\begin{tabular}{l}
|\input{childdoc.def}|\\
|\childdocforwardprefix[|\textit{main}|]{|\textit{prefix}|}{|\textit{dest}|}|
\end{tabular}
\end{center}
%
the destination file is determined by a pattern
depending on the current file:
To make this work, the current file must be called
`{\textit{prefix}\hspace{0.2em}\textit{suffix}}'
with \textit{prefix} matching precisely the argument.
Processing is then passed on to the file
`{\textit{dest}\hspace{0.2em}\textit{suffix}}'.
Surely, the same effect is achieved by
directly specifying the
argument `{\textit{dest}\hspace{0.2em}\textit{suffix}}'
in the first form.
However, that requires to set up a different file
for each child. With the alternative form of the command
all these files can have exactly the same content
which simplifies setting them up and maintaining them.

For example, the following file |draft.tex|
with a compilation flag |\version| as described in \secref{sec:flags}
compiles the main document as a draft:
%
\begin{center}
\begin{tabular}{l}
|\def\version{draft}|\\
|\input{childdoc.def}|\\
|\childdocforward{|\textit{main}|}|
\end{tabular}
\end{center}
%
Likewise, the following files |final|\textit{nn}|.tex|
compile the final version of the child document
|child|\textit{nn}|.tex|:
%
\begin{center}
\begin{tabular}{l}
|\def\version{final}|\\
|\input{childdoc.def}|\\
|\childdocforwardprefix{final}{child}|
\end{tabular}
\end{center}
%

Note that when several versions of a main file and/or of each child file
are to be generated, it may be convenient to set up a |Makefile| or
shell script to automatise the process.

%%%%%%%%%%%%%%%%%%%%%%%%%%%%%%%%%%%%%%%%%%%%%%%%%%%%%%%%%%%%%%%%%%%%%%%%%%%%%%%%
\subsection{Command Line Processing}
\label{sec:commandline}

The effect of redirection files can also be achieved by invoking
the \LaTeX{} compiler with a more elaborate command line.
Most conveniently this should be done as part
of a shell script or a |Makefile|.

When using \textsf{childdoc} in the main file, the following
command lines effectively perform a redirection
(note that depending on the shell being used,
backslashes may have to be doubled: `|\|' $\to$ `|\\|'):
%
\begin{center}
|... -jobname "|\textit{target}|" |\\|"|[\textit{flags}]%
|\input{childdoc.def}\childdocforward[|\textit{main}|]{|\textit{dest}|}"|
\end{center}
%
Here \textit{target} is the name of the output file,
\textit{main} is the name of the main file
and \textit{dest} is the name of the main or child file to be processed
(all filenames without extensions).
The optional argument \textit{main} can be omitted
if \textit{main} matches \textit{dest}.
Optionally, compilation \textit{flags} can be defined via |\def| commands.
This command line makes the \TeX{} engine believe
it is compiling the file \textit{target}
whose content is specified as the latter parameter.
The provided code then forwards the processing to
\textit{main} or \textit{dest} as described in \secref{sec:forward}.

%%%%%%%%%%%%%%%%%%%%%%%%%%%%%%%%%%%%%%%%%%%%%%%%%%%%%%%%%%%%%%%%%%%%%%%%%%%%%%%%
\subsection{Include by Input}
\label{sec:input}

Including child documents by |\include| has some restrictions by design.
Most notably, the content of a child document always occupies
its own set of pages; pages cannot be shared between child documents.
Usually, this behaviour makes perfect sense
because each child document contain an essential part of the document.
However, in some situations it may be desirable to compose
a document from a collection of parts
without having mandatory page breaks between then.
For this case, the package
provides a mechanism to include parts
by |\input| which can also be processed individually.
However, by construction this mechanism
requires manual handling of the content to be output.

%%%%%%%%%%%%%%%%%%%%%%%%%%%%%%%%%%%%%%%%
\DescribeMacro{\ifchilddocmanual}
The main file should be prepared as usual, see \secref{sec:include}.
However, the document body must make a distinction
between processing of an individual part and of the main document, e.g.:
%
\begin{center}
\begin{tabular}{l}
|\ifchilddocmanual|\\
|\input{\childdocname}|\\
|\||else|\\
\textit{document body with }|\input{|\textit{part}|}|\\
|\||fi|
\end{tabular}
\end{center}
%
The conditional |\ifchilddocmanual| is true whenever
a part to be included by |\input| is being compiled,
and the name of the part is stored in |\childdocname|.

%%%%%%%%%%%%%%%%%%%%%%%%%%%%%%%%%%%%%%%%
\DescribeMacro{\childdocby}
Each part to be included by |\input| should start with:
%
\begin{center}
\begin{tabular}{l}
|\input{childdoc.def}|\\
|\childdocby{|\textit{main}|}|\\
\end{tabular}
\end{center}
%
The directive |\childdocby| is similar to |\childdocof|
described in \secref{sec:include},
but the subsequent selection of content must be done manually.
To that end, both |\ifchilddoc| and |\ifchilddocmanual|
will be true upon processing of a part,
and the name of the part is stored in |\childdocname|.
Note that |\jobname| will be set to the filename of the current part
so that each part receives an individual |.aux| file
that does not interfere with the |.aux| file(s) of the main document.
This behaviour can be altered by the alternative form
|\childdocby[*]{|\textit{main}|}| (with a non-empty optional argument)
which uses the |.aux| file of the main document
by setting |\jobname| to \textit{main}.

%%%%%%%%%%%%%%%%%%%%%%%%%%%%%%%%%%%%%%%%%%%%%%%%%%%%%%%%%%%%%%%%%%%%%%%%%%%%%%%%
\subsection{Driver Development}
\label{sec:driver}

The \textsf{childdoc} mechanism can also be use for the development
of definition files such as \LaTeX{} styles or classes.
This case differs from the above setup with multiple parts
included by |\include| in that no |\includeonly| should be invoked.
This can be achieved by starting the include file
(before |\ProvidesPackage|) with:
%
\begin{center}
\begin{tabular}{l}
|\input{childdoc.def}|\\
|\childdocforward{|\textit{main}|}|\\
\end{tabular}
\end{center}
%
or alternatively with:
%
\begin{center}
\begin{tabular}{l}
|\input{childdoc.def}|\\
|\childdocby{|\textit{main}|}|\\
\end{tabular}
\end{center}
%
Both forms have slightly different effects as described above.
The main file is prepared as usual, see \secref{sec:include}.

%%%%%%%%%%%%%%%%%%%%%%%%%%%%%%%%%%%%%%%%%%%%%%%%%%%%%%%%%%%%%%%%%%%%%%%%%%%%%%%%
\subsection{Legacy Detection}
\label{sec:detection}

The directive |\childdocmain| in the main file can detect
whether the complete document or merely a child is to be compiled
even without using the directive |\childdocof|.
This method is deprecated because it is less robust
and there is no compelling reason to use it;
it is merely provided for backward compatibility
and it may be removed in future versions.

If the detection mechanism is to be used,
it is mandatory to correctly specify
the filename of the main file as the argument of |\childdocmain|:
%
\begin{center}
\begin{tabular}{l}
|\input{childdoc.def}|\\
|\childdocmain{|\textit{main}|}|\\
\end{tabular}
\end{center}
%
If |\jobname| does not match the argument \textit{main} of |\childdocmain|,
it is assumed that |\jobname| points to the child file to be compiled.
When using |\childdocmain| with the main file specified as argument,
it suffices to start a child file
with just |\input{|\textit{main}|}|
without loading of the package and using |\childdocof|.
If instead all processing is done
with the appropriate \textsf{childdoc} directives,
the argument of \textit{main} of |\childdocmain| can be empty.

An alternative version of the command line processing described
in \secref{sec:commandline} using the detection mechanism reads:
%
\begin{center}
|... -jobname "|\textit{target}|" "|[\textit{flags}]%
[|\def\jobname{|\textit{dest}|}|]|\input{|\textit{main}|}"|
\end{center}

%%%%%%%%%%%%%%%%%%%%%%%%%%%%%%%%%%%%%%%%%%%%%%%%%%%%%%%%%%%%%%%%%%%%%%%%%%%%%%%%
\subsection{Manual Code}
\label{sec:manual}

In case one cannot be certain whether the definitions file |childdoc.def|
is installed on the target \TeX{} distribution
and one prefers not to ship it,
it is conceivable to paste a few relevant commands into the sources.

To that end, drop all statements |\input{childdoc.def}|
and perform the replacements as outlined below.
Instead of |\childdocmain{|\textit{main}|}| add the following code
to the top of the main file:
%
\begin{center}
\begin{tabular}{l}
|\||ifdefined\childdocname\endinput\||fi\newif\ifchilddoc|\\
|\edef\childdocname{\scantokens\expandafter{\jobname\noexpand}}|\\
|\def\childdocmain{|\textit{main}|}\||ifx\childdocmain\childdocname\||else|\\
|\childdoctrue\includeonly{\childdocname}\let\jobname\childdocmain\||fi|\\
\end{tabular}
\end{center}
%
Instead of |\childdocof{|\textit{main}|}| just include the main file
at the top of each child file:
%
\begin{center}
|\input{|\textit{main}|}|
\end{center}
%
A simple redirection |\childdocforward{|\textit{dest}|}| is achieved by:
%
\begin{center}
|\def\jobname{|\textit{dest}|}\input{\jobname}|
\end{center}
%
The redirection with prefix
|\childdocforwardprefix[|\textit{prefix}|]{|\textit{dest}|}|
is accomplished by:
%
\begin{center}
\begin{tabular}{l}
|{\edef\jobname{\scantokens\expandafter{\jobname\noexpand}}|\\
|\def\redirectjob |\textit{prefix}|#1~~~{\gdef\jobname{|\textit{dest}|#1}}|\\
|\expandafter\redirectjob\jobname~~~}\input{\jobname}|
\end{tabular}
\end{center}

In an alternative approach,
child documents can be compiled by a specific command line
without additional code or specific definitions:
%
\begin{center}
|... -jobname "|\textit{target}|" "|[\textit{flags}]%
|\includeonly{|\textit{dest}|}\input{|\textit{main}|}"|
\end{center}
%

%%%%%%%%%%%%%%%%%%%%%%%%%%%%%%%%%%%%%%%%%%%%%%%%%%%%%%%%%%%%%%%%%%%%%%%%%%%%%%%%
%%%%%%%%%%%%%%%%%%%%%%%%%%%%%%%%%%%%%%%%%%%%%%%%%%%%%%%%%%%%%%%%%%%%%%%%%%%%%%%%
\section{Information}

%%%%%%%%%%%%%%%%%%%%%%%%%%%%%%%%%%%%%%%%%%%%%%%%%%%%%%%%%%%%%%%%%%%%%%%%%%%%%%%%
\subsection{Copyright}

Copyright \copyright{} 2017--2018 Niklas Beisert

This work may be distributed and/or modified under the
conditions of the \LaTeX{} Project Public License, either version 1.3
of this license or (at your option) any later version.
The latest version of this license is in
  \url{http://www.latex-project.org/lppl.txt}
and version 1.3 or later is part of all distributions of \LaTeX{}
version 2005/12/01 or later.

This work has the LPPL maintenance status `maintained'.

The Current Maintainer of this work is Niklas Beisert.

This work consists of the files |README.txt|, |childdoc.ins| and |childdoc.dtx|
as well as the derived files |childdoc.def|, |cdocsamp.tex|
with |cdocsch1.tex|, |cdocsch2.tex|, |cdocspt3.tex|, |cdocspt4.tex|,
|cdocsdrf.tex|, |cdocsfn1.tex|, |cdocsfn2.tex|
as well as |childdoc.pdf|.

%%%%%%%%%%%%%%%%%%%%%%%%%%%%%%%%%%%%%%%%%%%%%%%%%%%%%%%%%%%%%%%%%%%%%%%%%%%%%%%%
\subsection{Files and Installation}

The package consists of the files:
%
\begin{center}
\begin{tabular}{ll}
    |README.txt|   & readme file \\
    |childdoc.ins| & installation file \\
    |childdoc.dtx| & source file \\
    |childdoc.def| & definition file \\
    |cdocsamp.tex| & sample main file \\
    |cdocsch1.tex| & sample include file \\
    |cdocsch2.tex| & sample include file \\
    |cdocspt3.tex| & sample part file \\
    |cdocspt4.tex| & sample part file \\
    |cdocsdrf.tex| & sample redirection file \\
    |cdocsfn1.tex| & sample redirection file \\
    |cdocsfn2.tex| & sample redirection file \\
    |childdoc.pdf| & manual
\end{tabular}
\end{center}
%
The distribution consists of the files
|README.txt|, |childdoc.ins| and |childdoc.dtx|.
%
\begin{itemize}
\item
Run (pdf)\LaTeX{} on |childdoc.dtx|
to compile the manual |childdoc.pdf| (this file).
\item
Run \LaTeX{} on |childdoc.ins| to create the definitions file |childdoc.def|
and the sample |cdocsamp.tex| with include files
|cdocsch1.tex|, |cdocsch2.tex|, |cdocspt3.tex|, |cdocspt4.tex|,
|cdocsdrf.tex|, |cdocsfn1.tex|, |cdocsfn2.tex|.
Then copy the file |childdoc.def| to an appropriate directory of your \LaTeX{}
distribution, e.g.\ \textit{texmf-root}|/tex/latex/childdoc|.
\end{itemize}

%%%%%%%%%%%%%%%%%%%%%%%%%%%%%%%%%%%%%%%%%%%%%%%%%%%%%%%%%%%%%%%%%%%%%%%%%%%%%%%%
\subsection{Related CTAN Packages}

There are several other packages which offer a similar functionality:
%
\begin{itemize}
\item
The packages
\href{http://ctan.org/pkg/docmute}{\textsf{docmute}},
\href{http://ctan.org/pkg/includex}{\textsf{includex}} and
\href{http://ctan.org/pkg/standalone}{\textsf{standalone}}
provide commands to include only the document body of
a child file thus allowing both files to be compiled individually.
\item
The packages \href{http://ctan.org/pkg/subdocs}{\textsf{subdocs}}
and \href{http://ctan.org/pkg/subfiles}{\textsf{subfiles}}
provide structures in which the main and child documents can be
encapsulated and allowing them to be compiled individually.
The inclusion mechanism is different from the conventional |\include|.
\item
The package \href{http://ctan.org/pkg/combine}{\textsf{combine}}
is an elaborate solution to combine several documents into one.
\end{itemize}
%
See also the CTAN topic \href{http://ctan.org/topic/subdocs}{\textsf{subdocs}}
for further related packages.
The present package differs from the above solutions in that
a document structure constructed with the conventional |\include| mechanism
just needs two extra commands at the top of every file
such that all constituent files can be compiled individually.

%%%%%%%%%%%%%%%%%%%%%%%%%%%%%%%%%%%%%%%%%%%%%%%%%%%%%%%%%%%%%%%%%%%%%%%%%%%%%%%%
%\subsection{Feature Suggestions}
%
%The following is a list of features which may be useful for future
%versions of this package:
%%
%\begin{itemize}
%\item
%\ldots
%\end{itemize}

%%%%%%%%%%%%%%%%%%%%%%%%%%%%%%%%%%%%%%%%%%%%%%%%%%%%%%%%%%%%%%%%%%%%%%%%%%%%%%%%
\subsection{Revision History}

%%%%%%%%%%%%%%%%%%%%%%%%%%%%%%%%%%%%%%%%
\paragraph{v2.0:} 2018/12/30

\begin{itemize}
\item
immediate forward processing
\item
added |\childdocby| mechanism
\item
manual restructured
\end{itemize}

%%%%%%%%%%%%%%%%%%%%%%%%%%%%%%%%%%%%%%%%
\paragraph{v1.6:} 2018/01/17

\begin{itemize}
\item
application for development of include files
\item
corrections to manual
\end{itemize}

%%%%%%%%%%%%%%%%%%%%%%%%%%%%%%%%%%%%%%%%
\paragraph{v1.5:} 2017/05/21

\begin{itemize}
\item
more complete structuring introduced
\item
|\childdocof| introduced
\item
|\childdoc| renamed to |\childdocmain|
\item
|\childredirect| renamed to |\childdocforward| and |\childdocforwardprefix|
and functionality expanded
\end{itemize}

%%%%%%%%%%%%%%%%%%%%%%%%%%%%%%%%%%%%%%%%
\paragraph{v1.0:} 2017/04/27

\begin{itemize}
\item
manual and install package
\item
first version published on CTAN
\end{itemize}

%%%%%%%%%%%%%%%%%%%%%%%%%%%%%%%%%%%%%%%%
\paragraph{v0.6:} 2017/04/26

\begin{itemize}
\item
redirection mechanism added
\end{itemize}

%%%%%%%%%%%%%%%%%%%%%%%%%%%%%%%%%%%%%%%%
\paragraph{v0.5:} 2017/04/26

\begin{itemize}
\item
functionality in definition file
\end{itemize}


%%%%%%%%%%%%%%%%%%%%%%%%%%%%%%%%%%%%%%%%%%%%%%%%%%%%%%%%%%%%%%%%%%%%%%%%%%%%%%%%
%%%%%%%%%%%%%%%%%%%%%%%%%%%%%%%%%%%%%%%%%%%%%%%%%%%%%%%%%%%%%%%%%%%%%%%%%%%%%%%%
%%%%%%%%%%%%%%%%%%%%%%%%%%%%%%%%%%%%%%%%%%%%%%%%%%%%%%%%%%%%%%%%%%%%%%%%%%%%%%%%
\appendix

\settowidth\MacroIndent{\rmfamily\scriptsize 000\ }

 \DocInput{childdoc.dtx}

\end{document}
%</driver>
% \fi
%
% %%%%%%%%%%%%%%%%%%%%%%%%%%%%%%%%%%%%%%%%%%%%%%%%%%%%%%%%%%%%%%%%%%%%%%%%%%%%%%
% %%%%%%%%%%%%%%%%%%%%%%%%%%%%%%%%%%%%%%%%%%%%%%%%%%%%%%%%%%%%%%%%%%%%%%%%%%%%%%
% \section{Sample}
%\iffalse
%<*samplemain>
%\fi
%
% The following presents a sample document
% with two chapters, two parts, a title page,
% a compile flag as well as three forwarding files to set the flag.
% It consists of eight |.tex| files:
% \begin{center}
% \begin{tabular}{ll}
% |cdocsamp.tex|&main file\\
% |cdocsch1.tex|&include file for chapter 1\\
% |cdocsch2.tex|&include file for chapter 2\\
% |cdocspt3.tex|&include file for part 3\\
% |cdocspt4.tex|&include file for part 4\\
% |cdocsdrf.tex|&forwarding file for main file in draft mode\\
% |cdocsfi1.tex|&forwarding file for final version of chapter 1\\
% |cdocsfi2.tex|&forwarding file for final version of chapter 2\\
% \end{tabular}
% \end{center}
% Each of the eight files can be compiled directly by the \LaTeX{} compiler.
%
% %%%%%%%%%%%%%%%%%%%%%%%%%%%%%%%%%%%%%%
% \paragraph{Main File.}
%
% The main file is called |cdocsamp.tex|.
%
% Load the \textsf{childdoc} definitions and
% declare the filename for the main document:
%    \begin{macrocode}
\input{childdoc.def}
\childdocmain{}
%    \end{macrocode}

% Optional override for |\version| flag:
%    \begin{macrocode}
%%\ifchilddoc\else\providecommand{\version}{draft}\fi
%    \end{macrocode}

% Define the default values for the |\version| flag
% (|final| for the main file and |draft| for childs):
%    \begin{macrocode}
\ifchilddoc
\providecommand{\version}{draft}
\else
\providecommand{\version}{final}
\fi
%    \end{macrocode}

% Load the standard document class:
%    \begin{macrocode}
\documentclass[12pt]{article}
%    \end{macrocode}

% Start the document body:
%    \begin{macrocode}
\begin{document}
%    \end{macrocode}

% Declare a title page.
% Print title, part of document being processed and version flag:
%    \begin{macrocode}
\addtocounter{page}{-1}
\begin{center}
{\LARGE\bfseries{}childdoc example\par}
\vspace{1cm}
\ifchilddoc
\ifchilddocmanual part\else chapter\fi:
`\childdocname' of `\childdocjob'\par
\else
main document: `\childdocjob'\par
\fi
version: \version\par
\end{center}
\newpage
%    \end{macrocode}

% Manually include selected file,
% otherwise process as usual:
%    \begin{macrocode}
\ifchilddocmanual
\section*{part `\childdocname'}
\input{\childdocname}
\else
%    \end{macrocode}

% Include the two chapters:
%    \begin{macrocode}
\include{cdocsch1}
\include{cdocsch2}
%    \end{macrocode}

% Include the two parts unless only chapters should be displayed:
%    \begin{macrocode}
\ifchilddoc\else
\section{part three}
\input{cdocspt3}
\section{part four}
\input{cdocspt4}
\fi
%    \end{macrocode}

% Process as usual until here:
%    \begin{macrocode}
\fi
%    \end{macrocode}

% End of document body:
%    \begin{macrocode}
\end{document}
%    \end{macrocode}
%\iffalse
%</samplemain>
%\fi
%
% %%%%%%%%%%%%%%%%%%%%%%%%%%%%%%%%%%%%%%
% \paragraph{Chapter Include Files.}
%
% The include files are called |cdocsch1.tex| and |cdocsch2.tex|.
%
%\iffalse
%<*samplechap1|samplechap2>
%\fi

% Optional override for |\version| flag:
%    \begin{macrocode}
%%\providecommand{\version}{final}
%    \end{macrocode}

% Include the main document:
%    \begin{macrocode}
\input{childdoc.def}
\childdocof{cdocsamp}
%    \end{macrocode}

%\iffalse
%</samplechap1|samplechap2>
%\fi
%
%\iffalse
%<*samplechap1>
%\fi
% Some text for chapter 1:
%    \begin{macrocode}
\section{one}
some text in chapter one
%    \end{macrocode}

%\iffalse
%</samplechap1>
%\fi
% Some text for chapter 2:
%\iffalse
%<*samplechap2>
%\fi
%    \begin{macrocode}
\section{two}
more text in chapter two
%    \end{macrocode}

%\iffalse
%</samplechap2>
%\fi
%
% %%%%%%%%%%%%%%%%%%%%%%%%%%%%%%%%%%%%%%
% \paragraph{Part Include Files.}
%
% The include files are called |cdocspt3.tex| and |cdocspt4.tex|.
%
%\iffalse
%<*samplepart3|samplepart4>
%\fi

% Optional override for |\version| flag:
%    \begin{macrocode}
%%\providecommand{\version}{final}
%    \end{macrocode}

% Include the main document:
%    \begin{macrocode}
\input{childdoc.def}
\childdocby{cdocsamp}
%    \end{macrocode}

%\iffalse
%</samplepart3|samplepart4>
%\fi
%
%\iffalse
%<*samplepart3>
%\fi
% Some text for part 3:
%    \begin{macrocode}
some text in part three
%    \end{macrocode}

%\iffalse
%</samplepart3>
%\fi
% Some text for part 4:
%\iffalse
%<*samplepart4>
%\fi
%    \begin{macrocode}
more text in part four
%    \end{macrocode}

%\iffalse
%</samplepart4>
%\fi
%
% %%%%%%%%%%%%%%%%%%%%%%%%%%%%%%%%%%%%%%
% \paragraph{Forwarding for a Complete Draft.}
%
% The following forwarding file |cdocsdrf.tex|
% compiles the main document in draft mode:
%\iffalse
%<*sampledraft>
%\fi
%    \begin{macrocode}
\def\version{draft}
\input{childdoc.def}
\childdocforward{cdocsamp}
%    \end{macrocode}

%\iffalse
%</sampledraft>
%\fi
%
% %%%%%%%%%%%%%%%%%%%%%%%%%%%%%%%%%%%%%%
% \paragraph{Forwarding for Final Version of the Chapters.}
%
% The following forwarding files |cdocsfn1.tex| and |cdocsfn2.tex|
% (with identical content)
% compile the final versions of the child documents
% |cdocsch1.tex| and |cdocsch2.tex|, respectively:
%\iffalse
%<*samplefinal>
%\fi
%    \begin{macrocode}
\def\version{final}
\input{childdoc.def}
\childdocforwardprefix[cdocsamp]{cdocsfn}{cdocsch}
%    \end{macrocode}

%\iffalse
%</samplefinal>
%\fi
%
% %%%%%%%%%%%%%%%%%%%%%%%%%%%%%%%%%%%%%%
% \paragraph{Command Line Processing.}
%
% The following three command lines generate the output files
% |cdocscld|, |cdocscl1| and |cdocscl2|
% which should be identical to
% |cdocsdrf|, |cdocsch1| and |cdocsfn2|, respectively:
% \begin{center}
% \begin{tabular}{l}
% |latex -jobname cdocscld \|\\
% |  "\def\version{draft}\input{childdoc.def}\childdocforward{cdocsamp}"|\\
% |latex -jobname cdocscl1 \|\\
% |  "\input{childdoc.def}\childdocforward[cdocsamp]{cdocsch1}"|\\
% |latex -jobname cdocscl2 \|\\
% |  "\def\version{final}\input{childdoc.def}\childdocforward{cdocsch2}"|
% \end{tabular}
% \end{center}
% Note that the trailing backslash on each first line
% merely continues the input to the second line
% (for convenient cut ant paste).
% Furthermore, the command |latex| can be replaced by any
% of its alternative versions such as |pdflatex|.
%
% %%%%%%%%%%%%%%%%%%%%%%%%%%%%%%%%%%%%%%%%%%%%%%%%%%%%%%%%%%%%%%%%%%%%%%%%%%%%%%
% %%%%%%%%%%%%%%%%%%%%%%%%%%%%%%%%%%%%%%%%%%%%%%%%%%%%%%%%%%%%%%%%%%%%%%%%%%%%%%
% \section{Implementation}
%\iffalse
%<*package>
%\fi
%
% This section describes the definitions file |childdoc.def|.

% The definitions cannot be loaded using |\usepackage| or |\RequirePackage|
% which has a mechanism to prevent loading a style file more than once.
% When loading the definitions by means of |\input|
% multiple instances have to be prevented manually:
%\iffalse
%This code needs to be before the `\ProvidesFile' directive
%which is defined at the beginning of this file.
%Therefore it is also placed there and commented out here.
%</package>
%<*discard>
%\fi
%    \begin{macrocode}
\ifdefined\childdocmain\endinput\fi
%    \end{macrocode}
%\iffalse
%</discard>
%<*package>
%\fi
%
% \macro{\ifchilddoc}
% \macro{\ifchilddocmanual}
% The conditional |\ifchilddoc| tells whether a
% child (true) or main (false) document is being compiled.
% The conditional |\ifchilddocmanual| tells whether
% the |\includeonly| mechanism is used (false) or
% the selection of child files must be performed manually (true).
% The definitions initialise to false:
%    \begin{macrocode}
\newif\ifchilddoc
\newif\ifchilddocmanual
%    \end{macrocode}

% \macro{\childdocname}
% \macro{\childdocjob}
% The macro |\childdocname| stores the name of the main document
% to be compiled. The macro |\childdocjob| stores the name of
% the document on which the \LaTeX{} compiler was originally invoked.
% The content of |\jobname| cannot be compared
% to filenames specified in the source due to different catcodes.
% The following code rescans |\jobname|, stores the result
% in |\childdocname| and saves a copy in |\childdocjob|:
%    \begin{macrocode}
\edef\childdocname{\scantokens\expandafter{\jobname\noexpand}}
\let\childdocjob\childdocname
%    \end{macrocode}

% \macro{\childdocdisable}
% The macro |\childdocdisable| prevents the main file
% from being processed more than once.
% At this stage, the main document command |\childdocmain|
% is assumed to be called once again where it should do nothing.
% Any subsequent call to it should prevent
% a secondary processing of the main document
% It overwrites the forwarding commands
% |\childdocof| and |\childdocforward|
% with empty macros to prevent further inclusions of the main document:
%    \begin{macrocode}
\newcommand{\childdocdisable}
{
  \renewcommand{\childdocmain}[1]{\renewcommand{\childdocmain}[1]{\endinput}}
  \renewcommand{\childdocof}[1]{}
  \renewcommand{\childdocby}[2][]{}
  \renewcommand{\childdocforward}[2][]{}
  \renewcommand{\childdocdisable}{}
}
%    \end{macrocode}

% \macro{\childdocmain}
% The macro |\childdocmain| is to be called at the top of the main file
% with nothing or the main filename (without extension) as argument.
% First, it breaks loops.
% If the argument is not empty and does not match |\childdocname|
% (which is set by the first inclusion of |childdoc.def|),
% |\ifchilddoc| is set to true, |\includeonly| is applied to the child file
% and |\jobname| is set to the main file
% (for proper handling of |.aux| files):
%    \begin{macrocode}
\newcommand{\childdocmain}[1]
{
  \childdocdisable\childdocmain{}
  \if?#1?\else
    \begingroup
      \def\childdoctmp{#1}
      \ifx\childdoctmp\childdocname
        \def\childdoctmp{}
      \else
        \def\childdoctmp
        {
          \childdoctrue
          \includeonly{\childdocname}
          \def\childdocjob{#1}
          \def\jobname{#1}
        }
      \fi
      \expandafter
    \endgroup
    \childdoctmp
  \fi
}
%    \end{macrocode}

% \macro{\childdocof}
% The command |\childdocof| redirects
% compilation to the main file |#1|.
%    \begin{macrocode}
\newcommand{\childdocof}[1]
{
  \childdocdisable
  \childdoctrue
  \includeonly{\childdocname}
  \def\jobname{#1}
  \def\childdocjob{#1}
  \input{#1}
}
%    \end{macrocode}

% \macro{\childdocby}
% The command |\childdocby| ....
%    \begin{macrocode}
\newcommand{\childdocby}[2][]
{
  \childdocdisable
  \childdoctrue
  \childdocmanualtrue
  \if?#1?\else
    \def\jobname{#2}
  \fi
  \def\childdocjob{#2}
  \input{#2}
  \endinput
}
%    \end{macrocode}

% \macro{\childdocforward}
% The command |\childdocforward| redirects
% compilation to the main file or
% (if the optional argument is given) a child file.
% Parameters are set as if the main file
% or a child file starting with |\childdocof| was compiled.
% Then compilation is handed over to the main file:
%    \begin{macrocode}
\newcommand{\childdocforward}[2][]
{
  \begingroup
    \if?#1?
      \def\childdoctmp
      {
        \def\childdocname{#2}
        \def\childdocjob{#2}
        \def\jobname{#2}
        \input{#2}
        \endinput
      }
    \else
      \def\childdoctmp
      {
        \childdocdisable
        \def\childdocname{#2}
        \childdoctrue
        \includeonly{#2}
        \def\childdocjob{#1}
        \def\jobname{#1}
        \input{#1}
        \endinput
      }
    \fi
    \expandafter
  \endgroup
  \childdoctmp
}
%    \end{macrocode}

% \macro{\childdocforwardprefix}
% The command |\childdocforwardprefix| redirects
% compilation to the main or a child file by means of a pattern.
% The prefix |#1| in the current filename is replaced by |#2|
% and the suffix of the current filename is kept
% (it is assumed that the filename does not contain the substring `|~~~|'
% which is used as a delimiter).
% Compilation is handed over to the new file by |\childdocforward|:
%    \begin{macrocode}
\newcommand{\childdocforwardprefix}[3][]
{
  \begingroup
    \def\childdocextract #2##1~~~{\def\childdoctmp{\childdocforward[#1]{#3##1}}}
    \expandafter\childdocextract\childdocname~~~
    \expandafter
  \endgroup
  \childdoctmp
}
%    \end{macrocode}

% \macro{\childdoc}
% The deprecated macro |\childdoc| is a legacy version of |\childdocmain|:
%    \begin{macrocode}
\newcommand{\childdoc}{\childdocmain}
%    \end{macrocode}

% \macro{\childdocredirect}
% The deprecated macro |\childdocredirect| is a legacy version
% of |\childdocforward| and |\childdocforwardprefix|:
%    \begin{macrocode}
\newcommand{\childdocredirect}[2][]
{
  \begingroup
    \if?#1?
      \def\childdoctmp{\childdocforward{#2}}
    \else
      \def\childdoctmp{\childdocforwardprefix{#1}{#2}}
    \fi
    \expandafter
  \endgroup
  \childdoctmp
}
%    \end{macrocode}

%\iffalse
%</package>
%\fi
%
\endinput
|\\
|\childdocforwardprefix{final}{child}|
\end{tabular}
\end{center}
%

Note that when several versions of a main file and/or of each child file
are to be generated, it may be convenient to set up a |Makefile| or
shell script to automatise the process.

%%%%%%%%%%%%%%%%%%%%%%%%%%%%%%%%%%%%%%%%%%%%%%%%%%%%%%%%%%%%%%%%%%%%%%%%%%%%%%%%
\subsection{Command Line Processing}
\label{sec:commandline}

The effect of redirection files can also be achieved by invoking
the \LaTeX{} compiler with a more elaborate command line.
Most conveniently this should be done as part
of a shell script or a |Makefile|.

When using \textsf{childdoc} in the main file, the following
command lines effectively perform a redirection
(note that depending on the shell being used,
backslashes may have to be doubled: `|\|' $\to$ `|\\|'):
%
\begin{center}
|... -jobname "|\textit{target}|" |\\|"|[\textit{flags}]%
|% \iffalse
%
% childdoc.dtx Copyright (C) 2017-2018 Niklas Beisert
%
% This work may be distributed and/or modified under the
% conditions of the LaTeX Project Public License, either version 1.3
% of this license or (at your option) any later version.
% The latest version of this license is in
%   http://www.latex-project.org/lppl.txt
% and version 1.3 or later is part of all distributions of LaTeX
% version 2005/12/01 or later.
%
% This work has the LPPL maintenance status `maintained'.
%
% The Current Maintainer of this work is Niklas Beisert.
%
% This work consists of the files childdoc.dtx and childdoc.ins
% and the derived files childdoc.def and cdocsamp.tex with
% cdocsch1.tex, cdocsch2.tex, cdocsdrf.tex, cdocsfn1.tex, cdocsfn2.tex.
%
%<package>\ifdefined\childdocmain\endinput\fi
%<package>\ProvidesFile{childdoc.def}[2018/12/30 v2.0 child document driver]
%<samplemain>\ProvidesFile{cdocsamp.tex}[2018/12/30 v2.0 sample for childdoc]
%<*driver>
%\ProvidesFile{childdoc.drv}[2018/12/30 v2.0 childdoc reference manual file]
\PassOptionsToClass{10pt,a4paper}{article}
\documentclass{ltxdoc}

\usepackage[margin=35mm]{geometry}
\usepackage{hyperref}
\usepackage{hyperxmp}
\usepackage[usenames]{color}

\hypersetup{colorlinks=true}
\hypersetup{pdfstartview=FitH}
\hypersetup{pdfpagemode=UseNone}
\hypersetup{pdfsource={}}
\hypersetup{pdflang={en-UK}}
\hypersetup{pdfcopyright={Copyright 2017-2018 Niklas Beisert.
  This work may be distributed and/or modified under the
  conditions of the LaTeX Project Public License, either version 1.3
  of this license or (at your option) any later version.}}
\hypersetup{pdflicenseurl={http://www.latex-project.org/lppl.txt}}
\hypersetup{pdfcontactaddress={ETH Zurich, ITP, HIT K,
  Wolfgang-Pauli-Strasse 27}}
\hypersetup{pdfcontactpostcode={8093}}
\hypersetup{pdfcontactcity={Zurich}}
\hypersetup{pdfcontactcountry={Switzerland}}
\hypersetup{pdfcontactemail={nbeisert@itp.phys.ethz.ch}}
\hypersetup{pdfcontacturl={http://people.phys.ethz.ch/\xmptilde nbeisert/}}

\newcommand{\secref}[1]{\hyperref[#1]{section \ref*{#1}}}

\parskip1ex
\parindent0pt
\let\olditemize\itemize
\def\itemize{\olditemize\parskip0pt}

\begin{document}

\title{The \textsf{childdoc} Package}
\hypersetup{pdftitle={The childdoc Package}}
\author{Niklas Beisert\\[2ex]
  Institut f\"ur Theoretische Physik\\
  Eidgen\"ossische Technische Hochschule Z\"urich\\
  Wolfgang-Pauli-Strasse 27, 8093 Z\"urich, Switzerland\\[1ex]
  \href{mailto:nbeisert@itp.phys.ethz.ch}
  {\texttt{nbeisert@itp.phys.ethz.ch}}}
\hypersetup{pdfauthor={Niklas Beisert}}
\hypersetup{pdfsubject={Manual for the LaTeX2e Package childdoc}}
\date{30 December 2018, \textsf{v2.0}}
\maketitle

\begin{abstract}\noindent
\textsf{childdoc} is a \LaTeXe{} package
that enables the direct compilation
of document sections included by |\include|
to individual files.
\end{abstract}

\begingroup
\parskip0ex
\tableofcontents
\endgroup

%%%%%%%%%%%%%%%%%%%%%%%%%%%%%%%%%%%%%%%%%%%%%%%%%%%%%%%%%%%%%%%%%%%%%%%%%%%%%%%%
%%%%%%%%%%%%%%%%%%%%%%%%%%%%%%%%%%%%%%%%%%%%%%%%%%%%%%%%%%%%%%%%%%%%%%%%%%%%%%%%
\section{Introduction}

\LaTeX{} provides a mechanism to structure a large document (such as a book)
into a main file and several child files (containing the chapters)
using the |\include| command.
This mechanism is beneficial for documents
which span hundreds of pages in order to
make the source file(s) more manageable.
Moreover, compilation can be restricted to
selected child files by means of the |\includeonly| command.
The latter feature can be used to reduce the compilation time while editing
(this was significantly more useful in the earlier days of \LaTeX{})
or to generate a smaller document which is easier to navigate.
Another application of |\includeonly| is to generate
documents consisting of selected parts of the complete document.

However, there are a few drawbacks of the plain |\include| mechanism:
\begin{itemize}
\item
The child files cannot be compiled on their own,
they can only be compiled via the main file.
A naive editing environment
(such as a text editor with an option
to have the current file processed by \LaTeX)
may require one to switch to the main file before compiling;
attempting to compile the child file produces errors.
\item
The main file must be modified (each time)
to adjust the |\includeonly| command
to the present needs. This easily leaves the main file in a messy state.
\item
The generated document will always carry the filename
of the main document. This is inconvenient if
several child files are to be compiled and
to be kept for distribution.
\end{itemize}

The present package provides a simple interface
to make child files individually compilable by \LaTeX{}.
Compiling a child file then has the same effect as compiling
the main file with an |\includeonly| command
to select the appropriate child.
Moreover the generated document will carry the name of the child
rather than the main file.
This resolves all three above issues.

This feature is meant to make the editing of books,
thesis documents and lecture notes somewhat more convenient.
However, the package can also be used efficiently for
composing a series of documents (such as exercise sheets)
which are typically distributed individually.
It then assists the author in generating the individual documents
(potentially in different versions)
as well as a document containing the collected series.
Another application is in developing style files
or other kinds of included material
where compilation of the style file could redirect
to a sample or test file.

%%%%%%%%%%%%%%%%%%%%%%%%%%%%%%%%%%%%%%%%%%%%%%%%%%%%%%%%%%%%%%%%%%%%%%%%%%%%%%%%
%%%%%%%%%%%%%%%%%%%%%%%%%%%%%%%%%%%%%%%%%%%%%%%%%%%%%%%%%%%%%%%%%%%%%%%%%%%%%%%%
\section{Usage}

First of all, the package \textsf{childdoc} is \emph{not} a standard
\LaTeXe{} |.sty| style file! Therefore it needs to be invoked in
a non-standard way.

%%%%%%%%%%%%%%%%%%%%%%%%%%%%%%%%%%%%%%%%%%%%%%%%%%%%%%%%%%%%%%%%%%%%%%%%%%%%%%%%
\subsection{Included Files}
\label{sec:include}

%%%%%%%%%%%%%%%%%%%%%%%%%%%%%%%%%%%%%%%%
\DescribeMacro{\childdocmain}
To use the package, add the commands
\begin{center}
\begin{tabular}{l}
|\input{childdoc.def}|\\
|\childdocmain{}|\\
\end{tabular}
\end{center}
at the very top of the main \LaTeX{} file,
in particular \emph{before} the |\documentclass| statement!
The argument of |\childdocmain| should be left empty
(but it must be present).

%%%%%%%%%%%%%%%%%%%%%%%%%%%%%%%%%%%%%%%%
\DescribeMacro{\childdocof}
Furthermore, add the commands
\begin{center}
\begin{tabular}{l}
|\input{childdoc.def}|\\
|\childdocof{|\textit{main}|}|\\
\end{tabular}
\end{center}
at the top of every child file \textit{child}
which is included by |\include{|\textit{child}|}|
from within the main file
(or at least for those files to be compiled individually).
The argument \textit{main} must be the filename of the main file.

There are a couple of
considerations in setting up the main and child documents:

%%%%%%%%%%%%%%%%%%%%%%%%%%%%%%%%%%%%%%%%
\paragraph{Restrictions.}

Please note the following restrictions:
\begin{itemize}
\item
|\childdocmain| must be called with one argument \textit{main}
to ensure compatibility with earlier version of the package.
It must either be empty (|\childdocmain{}|)
or precisely match the filename of the main file in which it is specified.
See \secref{sec:detection} for further information.
\item
The filename \textit{main} must be specified without the |.tex| extension.
\item
The filename \textit{main} is case sensitive
(even in case-insensitive file systems)
due to internal string comparison.
\item
The argument \textit{main} should be fully expanded, it cannot be a macro.
\item
Subdirectories and special characters should be avoided in filenames.
\item
The command |\childdocmain{|\textit{main}|}| must be followed by a whitespace.
It should not be followed immediately by another command
or by a comment mark `|%|'.
This is because the \TeX{} parser reads the token immediately following
the argument of |\childdocmain| and puts it
at the beginning of every child section;
however, a white\-space is ignored.
\end{itemize}

%%%%%%%%%%%%%%%%%%%%%%%%%%%%%%%%%%%%%%%%
\paragraph{Content of Main File.}

It is advisable to place all content in the child files included by |\include|.
Any output contained in the main file will appear in all child documents
unless suppressed manually;
it cannot be suppressed automatically by the |\includeonly| directive
and thus should normally be avoided.
A method to include some content in the main file
by means of conditional processing is described in \secref{sec:conditional}.

%%%%%%%%%%%%%%%%%%%%%%%%%%%%%%%%%%%%%%%%
\paragraph{Page Numbering.}

When only a part of the document is compiled,
the appropriate numbering of pages
(as well as other status parameters)
is determined from the |.aux| files.
The latter contain information from previous passes.
However this information needs to propagate through
all intermediate child documents.
Therefore the page numbering in child documents may well
be inconsistent until the complete document is compiled at least once.

A useful (if unconventional) way to always ensure a consistent
page numbering is to restart the numbering in each child document
and denote the pages by `\textit{child}|.|\textit{page}'
where \textit{child} represents the chapter/section number of the child file.
This can be achieved by the command
|\numberwithin{page}{|\textit{child}|}|
of the \textsf{amsmath} package
where \textit{child} can be |chapter| or |section|
depending on the chosen structuring.
Alternatively, one can modify the macro |\thepage| appropriately
and reset the counter |page| at the start of each child file.

%%%%%%%%%%%%%%%%%%%%%%%%%%%%%%%%%%%%%%%%%%%%%%%%%%%%%%%%%%%%%%%%%%%%%%%%%%%%%%%%
\subsection{Conditional Processing}
\label{sec:conditional}

The package provides a mechanism to compile different versions
of a document. To customise the versions further some conditional processing
can come in handy to distinguish which version is being compiled.
The package provides two macros to describe the compilation context:

%%%%%%%%%%%%%%%%%%%%%%%%%%%%%%%%%%%%%%%%
\DescribeMacro{\ifchilddoc}
The conditional |\ifchilddoc| distinguishes between the compilation of
child documents and the main document:
%
\begin{center}
|\ifchilddoc |\textit{child-code}| |[|\||else |\textit{main-code}]| \||fi|
\end{center}

%%%%%%%%%%%%%%%%%%%%%%%%%%%%%%%%%%%%%%%%
\DescribeMacro{\childdocname}
\DescribeMacro{\childdocjob}
The macro |\childdocname| contains the filename (without extension)
of the main or child file being processed.
Note that |\childdocjob| will always contain the name of the main file.

%%%%%%%%%%%%%%%%%%%%%%%%%%%%%%%%%%%%%%%%
\paragraph{Title Page.}

Conditional processing can be used to include a title or banner page
in the main document when proper precautions are taken.
Importantly, the code in the main file should ensure that the page counter
(as well as other status parameters which are stored in the |.aux| files)
takes the same value after the conditional processing.
Otherwise the page numbers may take divergent values
depending on which part is compiled.

For example, a title page could be declared by:
%
\begin{center}
\begin{tabular}{l}
|\ifchilddoc\||else|\\
|\addtocounter{page}{-1}|\\
\textit{code for title page}\\
|\newpage|\\
|\||fi|
\end{tabular}
\end{center}
%
A banner page for the child documents can be generated by:
%
\begin{center}
\begin{tabular}{l}
|\ifchilddoc|\\
|\addtocounter{page}{-1}|\\
\textit{code for banner page}\\
|\newpage|\\
|\||fi|
\end{tabular}
\end{center}
%
Here one could write a message such as:
\begin{center}
|This is the part \childdocname{} of \childdocjob{}.|
\end{center}

%%%%%%%%%%%%%%%%%%%%%%%%%%%%%%%%%%%%%%%%%%%%%%%%%%%%%%%%%%%%%%%%%%%%%%%%%%%%%%%%
\subsection{Flags}
\label{sec:flags}

The package makes it easy to generate different versions
of the main or child documents.
To this end compilation flags can be defined
and assigned different default values.
They will be particularly useful in conjunction
with the forwarding mechanism described in \secref{sec:forward}.

For example, it may be useful to have a flag |\version|
which can be set to |draft| or |final|.
The document source will contain some conditional code
depending on the value of |\version|.
Suppose further, the flag should default to |final| for the main file
and to |draft| for child files
which is a natural assignment for editing the document.
This is achieved by placing the following code
in the preamble of the main document
(below the |\childdocmain| directive):
%
\begin{center}
\begin{tabular}{l}
|\ifchilddoc|\\
|\providecommand{\version}{draft}|\\
|\||else|\\
|\providecommand{\version}{final}|\\
|\||fi|
\end{tabular}
\end{center}
%
The definition by |\providecommand| makes sure
that previous definitions are not overwritten.
Further statements |\providecommand{\version}{...}|
can thus be added before the above code to override it.

For the main file, one might add a line
(between |\childdocmain| and the above block)
%
\begin{center}
|%\ifchilddoc\||else\providecommand{\version}{draft}\||fi|
\end{center}
%
which can be uncommented to produce a draft version.
Likewise one can add a line to the very top of a child file
(above the |\childdocof{|\textit{main}|}| directive)
%
\begin{center}
|%\providecommand{\version}{final}|
\end{center}
%
which can be uncommented to produce the final version of this child document.

%%%%%%%%%%%%%%%%%%%%%%%%%%%%%%%%%%%%%%%%%%%%%%%%%%%%%%%%%%%%%%%%%%%%%%%%%%%%%%%%
\subsection{Forwarding}
\label{sec:forward}

Different versions of the main or child documents
using compilation flags as described in \secref{sec:flags}
can be (permanently) stored in different files
for convenient compilation, viewing and distribution.
To this end, the package defines a command
to pass on compilation to a different file:

%%%%%%%%%%%%%%%%%%%%%%%%%%%%%%%%%%%%%%%%
\DescribeMacro{\childdocforward}
The command |\childdocforward| redirects processing to
another source file:
%
\begin{center}
\begin{tabular}{l}
|\input{childdoc.def}|\\
|\childdocforward[|\textit{main}|]{|\textit{dest}|}|\\
\end{tabular}
\end{center}
%
The argument \textit{dest} is the destination file
(without extension).
It should be the main file or one of the child files.
Note that further \textsf{childdoc} directives
such as |\childdocof| and |\childdocforward|
in the indicated file will be processed in this form.
The optional argument \textit{main}
passes on directly to the main file \textit{main}
while pretending to compile the child \textit{dest}.
This form behaves as if \textit{dest}
issues |\childdocof{|\textit{main}|}| right away,
and no further \textsf{childdoc} directives will be processed.

%%%%%%%%%%%%%%%%%%%%%%%%%%%%%%%%%%%%%%%%
\DescribeMacro{\...prefix}
In the alternative form |\childdocforwardprefix|,
%
\begin{center}
\begin{tabular}{l}
|\input{childdoc.def}|\\
|\childdocforwardprefix[|\textit{main}|]{|\textit{prefix}|}{|\textit{dest}|}|
\end{tabular}
\end{center}
%
the destination file is determined by a pattern
depending on the current file:
To make this work, the current file must be called
`{\textit{prefix}\hspace{0.2em}\textit{suffix}}'
with \textit{prefix} matching precisely the argument.
Processing is then passed on to the file
`{\textit{dest}\hspace{0.2em}\textit{suffix}}'.
Surely, the same effect is achieved by
directly specifying the
argument `{\textit{dest}\hspace{0.2em}\textit{suffix}}'
in the first form.
However, that requires to set up a different file
for each child. With the alternative form of the command
all these files can have exactly the same content
which simplifies setting them up and maintaining them.

For example, the following file |draft.tex|
with a compilation flag |\version| as described in \secref{sec:flags}
compiles the main document as a draft:
%
\begin{center}
\begin{tabular}{l}
|\def\version{draft}|\\
|\input{childdoc.def}|\\
|\childdocforward{|\textit{main}|}|
\end{tabular}
\end{center}
%
Likewise, the following files |final|\textit{nn}|.tex|
compile the final version of the child document
|child|\textit{nn}|.tex|:
%
\begin{center}
\begin{tabular}{l}
|\def\version{final}|\\
|\input{childdoc.def}|\\
|\childdocforwardprefix{final}{child}|
\end{tabular}
\end{center}
%

Note that when several versions of a main file and/or of each child file
are to be generated, it may be convenient to set up a |Makefile| or
shell script to automatise the process.

%%%%%%%%%%%%%%%%%%%%%%%%%%%%%%%%%%%%%%%%%%%%%%%%%%%%%%%%%%%%%%%%%%%%%%%%%%%%%%%%
\subsection{Command Line Processing}
\label{sec:commandline}

The effect of redirection files can also be achieved by invoking
the \LaTeX{} compiler with a more elaborate command line.
Most conveniently this should be done as part
of a shell script or a |Makefile|.

When using \textsf{childdoc} in the main file, the following
command lines effectively perform a redirection
(note that depending on the shell being used,
backslashes may have to be doubled: `|\|' $\to$ `|\\|'):
%
\begin{center}
|... -jobname "|\textit{target}|" |\\|"|[\textit{flags}]%
|\input{childdoc.def}\childdocforward[|\textit{main}|]{|\textit{dest}|}"|
\end{center}
%
Here \textit{target} is the name of the output file,
\textit{main} is the name of the main file
and \textit{dest} is the name of the main or child file to be processed
(all filenames without extensions).
The optional argument \textit{main} can be omitted
if \textit{main} matches \textit{dest}.
Optionally, compilation \textit{flags} can be defined via |\def| commands.
This command line makes the \TeX{} engine believe
it is compiling the file \textit{target}
whose content is specified as the latter parameter.
The provided code then forwards the processing to
\textit{main} or \textit{dest} as described in \secref{sec:forward}.

%%%%%%%%%%%%%%%%%%%%%%%%%%%%%%%%%%%%%%%%%%%%%%%%%%%%%%%%%%%%%%%%%%%%%%%%%%%%%%%%
\subsection{Include by Input}
\label{sec:input}

Including child documents by |\include| has some restrictions by design.
Most notably, the content of a child document always occupies
its own set of pages; pages cannot be shared between child documents.
Usually, this behaviour makes perfect sense
because each child document contain an essential part of the document.
However, in some situations it may be desirable to compose
a document from a collection of parts
without having mandatory page breaks between then.
For this case, the package
provides a mechanism to include parts
by |\input| which can also be processed individually.
However, by construction this mechanism
requires manual handling of the content to be output.

%%%%%%%%%%%%%%%%%%%%%%%%%%%%%%%%%%%%%%%%
\DescribeMacro{\ifchilddocmanual}
The main file should be prepared as usual, see \secref{sec:include}.
However, the document body must make a distinction
between processing of an individual part and of the main document, e.g.:
%
\begin{center}
\begin{tabular}{l}
|\ifchilddocmanual|\\
|\input{\childdocname}|\\
|\||else|\\
\textit{document body with }|\input{|\textit{part}|}|\\
|\||fi|
\end{tabular}
\end{center}
%
The conditional |\ifchilddocmanual| is true whenever
a part to be included by |\input| is being compiled,
and the name of the part is stored in |\childdocname|.

%%%%%%%%%%%%%%%%%%%%%%%%%%%%%%%%%%%%%%%%
\DescribeMacro{\childdocby}
Each part to be included by |\input| should start with:
%
\begin{center}
\begin{tabular}{l}
|\input{childdoc.def}|\\
|\childdocby{|\textit{main}|}|\\
\end{tabular}
\end{center}
%
The directive |\childdocby| is similar to |\childdocof|
described in \secref{sec:include},
but the subsequent selection of content must be done manually.
To that end, both |\ifchilddoc| and |\ifchilddocmanual|
will be true upon processing of a part,
and the name of the part is stored in |\childdocname|.
Note that |\jobname| will be set to the filename of the current part
so that each part receives an individual |.aux| file
that does not interfere with the |.aux| file(s) of the main document.
This behaviour can be altered by the alternative form
|\childdocby[*]{|\textit{main}|}| (with a non-empty optional argument)
which uses the |.aux| file of the main document
by setting |\jobname| to \textit{main}.

%%%%%%%%%%%%%%%%%%%%%%%%%%%%%%%%%%%%%%%%%%%%%%%%%%%%%%%%%%%%%%%%%%%%%%%%%%%%%%%%
\subsection{Driver Development}
\label{sec:driver}

The \textsf{childdoc} mechanism can also be use for the development
of definition files such as \LaTeX{} styles or classes.
This case differs from the above setup with multiple parts
included by |\include| in that no |\includeonly| should be invoked.
This can be achieved by starting the include file
(before |\ProvidesPackage|) with:
%
\begin{center}
\begin{tabular}{l}
|\input{childdoc.def}|\\
|\childdocforward{|\textit{main}|}|\\
\end{tabular}
\end{center}
%
or alternatively with:
%
\begin{center}
\begin{tabular}{l}
|\input{childdoc.def}|\\
|\childdocby{|\textit{main}|}|\\
\end{tabular}
\end{center}
%
Both forms have slightly different effects as described above.
The main file is prepared as usual, see \secref{sec:include}.

%%%%%%%%%%%%%%%%%%%%%%%%%%%%%%%%%%%%%%%%%%%%%%%%%%%%%%%%%%%%%%%%%%%%%%%%%%%%%%%%
\subsection{Legacy Detection}
\label{sec:detection}

The directive |\childdocmain| in the main file can detect
whether the complete document or merely a child is to be compiled
even without using the directive |\childdocof|.
This method is deprecated because it is less robust
and there is no compelling reason to use it;
it is merely provided for backward compatibility
and it may be removed in future versions.

If the detection mechanism is to be used,
it is mandatory to correctly specify
the filename of the main file as the argument of |\childdocmain|:
%
\begin{center}
\begin{tabular}{l}
|\input{childdoc.def}|\\
|\childdocmain{|\textit{main}|}|\\
\end{tabular}
\end{center}
%
If |\jobname| does not match the argument \textit{main} of |\childdocmain|,
it is assumed that |\jobname| points to the child file to be compiled.
When using |\childdocmain| with the main file specified as argument,
it suffices to start a child file
with just |\input{|\textit{main}|}|
without loading of the package and using |\childdocof|.
If instead all processing is done
with the appropriate \textsf{childdoc} directives,
the argument of \textit{main} of |\childdocmain| can be empty.

An alternative version of the command line processing described
in \secref{sec:commandline} using the detection mechanism reads:
%
\begin{center}
|... -jobname "|\textit{target}|" "|[\textit{flags}]%
[|\def\jobname{|\textit{dest}|}|]|\input{|\textit{main}|}"|
\end{center}

%%%%%%%%%%%%%%%%%%%%%%%%%%%%%%%%%%%%%%%%%%%%%%%%%%%%%%%%%%%%%%%%%%%%%%%%%%%%%%%%
\subsection{Manual Code}
\label{sec:manual}

In case one cannot be certain whether the definitions file |childdoc.def|
is installed on the target \TeX{} distribution
and one prefers not to ship it,
it is conceivable to paste a few relevant commands into the sources.

To that end, drop all statements |\input{childdoc.def}|
and perform the replacements as outlined below.
Instead of |\childdocmain{|\textit{main}|}| add the following code
to the top of the main file:
%
\begin{center}
\begin{tabular}{l}
|\||ifdefined\childdocname\endinput\||fi\newif\ifchilddoc|\\
|\edef\childdocname{\scantokens\expandafter{\jobname\noexpand}}|\\
|\def\childdocmain{|\textit{main}|}\||ifx\childdocmain\childdocname\||else|\\
|\childdoctrue\includeonly{\childdocname}\let\jobname\childdocmain\||fi|\\
\end{tabular}
\end{center}
%
Instead of |\childdocof{|\textit{main}|}| just include the main file
at the top of each child file:
%
\begin{center}
|\input{|\textit{main}|}|
\end{center}
%
A simple redirection |\childdocforward{|\textit{dest}|}| is achieved by:
%
\begin{center}
|\def\jobname{|\textit{dest}|}\input{\jobname}|
\end{center}
%
The redirection with prefix
|\childdocforwardprefix[|\textit{prefix}|]{|\textit{dest}|}|
is accomplished by:
%
\begin{center}
\begin{tabular}{l}
|{\edef\jobname{\scantokens\expandafter{\jobname\noexpand}}|\\
|\def\redirectjob |\textit{prefix}|#1~~~{\gdef\jobname{|\textit{dest}|#1}}|\\
|\expandafter\redirectjob\jobname~~~}\input{\jobname}|
\end{tabular}
\end{center}

In an alternative approach,
child documents can be compiled by a specific command line
without additional code or specific definitions:
%
\begin{center}
|... -jobname "|\textit{target}|" "|[\textit{flags}]%
|\includeonly{|\textit{dest}|}\input{|\textit{main}|}"|
\end{center}
%

%%%%%%%%%%%%%%%%%%%%%%%%%%%%%%%%%%%%%%%%%%%%%%%%%%%%%%%%%%%%%%%%%%%%%%%%%%%%%%%%
%%%%%%%%%%%%%%%%%%%%%%%%%%%%%%%%%%%%%%%%%%%%%%%%%%%%%%%%%%%%%%%%%%%%%%%%%%%%%%%%
\section{Information}

%%%%%%%%%%%%%%%%%%%%%%%%%%%%%%%%%%%%%%%%%%%%%%%%%%%%%%%%%%%%%%%%%%%%%%%%%%%%%%%%
\subsection{Copyright}

Copyright \copyright{} 2017--2018 Niklas Beisert

This work may be distributed and/or modified under the
conditions of the \LaTeX{} Project Public License, either version 1.3
of this license or (at your option) any later version.
The latest version of this license is in
  \url{http://www.latex-project.org/lppl.txt}
and version 1.3 or later is part of all distributions of \LaTeX{}
version 2005/12/01 or later.

This work has the LPPL maintenance status `maintained'.

The Current Maintainer of this work is Niklas Beisert.

This work consists of the files |README.txt|, |childdoc.ins| and |childdoc.dtx|
as well as the derived files |childdoc.def|, |cdocsamp.tex|
with |cdocsch1.tex|, |cdocsch2.tex|, |cdocspt3.tex|, |cdocspt4.tex|,
|cdocsdrf.tex|, |cdocsfn1.tex|, |cdocsfn2.tex|
as well as |childdoc.pdf|.

%%%%%%%%%%%%%%%%%%%%%%%%%%%%%%%%%%%%%%%%%%%%%%%%%%%%%%%%%%%%%%%%%%%%%%%%%%%%%%%%
\subsection{Files and Installation}

The package consists of the files:
%
\begin{center}
\begin{tabular}{ll}
    |README.txt|   & readme file \\
    |childdoc.ins| & installation file \\
    |childdoc.dtx| & source file \\
    |childdoc.def| & definition file \\
    |cdocsamp.tex| & sample main file \\
    |cdocsch1.tex| & sample include file \\
    |cdocsch2.tex| & sample include file \\
    |cdocspt3.tex| & sample part file \\
    |cdocspt4.tex| & sample part file \\
    |cdocsdrf.tex| & sample redirection file \\
    |cdocsfn1.tex| & sample redirection file \\
    |cdocsfn2.tex| & sample redirection file \\
    |childdoc.pdf| & manual
\end{tabular}
\end{center}
%
The distribution consists of the files
|README.txt|, |childdoc.ins| and |childdoc.dtx|.
%
\begin{itemize}
\item
Run (pdf)\LaTeX{} on |childdoc.dtx|
to compile the manual |childdoc.pdf| (this file).
\item
Run \LaTeX{} on |childdoc.ins| to create the definitions file |childdoc.def|
and the sample |cdocsamp.tex| with include files
|cdocsch1.tex|, |cdocsch2.tex|, |cdocspt3.tex|, |cdocspt4.tex|,
|cdocsdrf.tex|, |cdocsfn1.tex|, |cdocsfn2.tex|.
Then copy the file |childdoc.def| to an appropriate directory of your \LaTeX{}
distribution, e.g.\ \textit{texmf-root}|/tex/latex/childdoc|.
\end{itemize}

%%%%%%%%%%%%%%%%%%%%%%%%%%%%%%%%%%%%%%%%%%%%%%%%%%%%%%%%%%%%%%%%%%%%%%%%%%%%%%%%
\subsection{Related CTAN Packages}

There are several other packages which offer a similar functionality:
%
\begin{itemize}
\item
The packages
\href{http://ctan.org/pkg/docmute}{\textsf{docmute}},
\href{http://ctan.org/pkg/includex}{\textsf{includex}} and
\href{http://ctan.org/pkg/standalone}{\textsf{standalone}}
provide commands to include only the document body of
a child file thus allowing both files to be compiled individually.
\item
The packages \href{http://ctan.org/pkg/subdocs}{\textsf{subdocs}}
and \href{http://ctan.org/pkg/subfiles}{\textsf{subfiles}}
provide structures in which the main and child documents can be
encapsulated and allowing them to be compiled individually.
The inclusion mechanism is different from the conventional |\include|.
\item
The package \href{http://ctan.org/pkg/combine}{\textsf{combine}}
is an elaborate solution to combine several documents into one.
\end{itemize}
%
See also the CTAN topic \href{http://ctan.org/topic/subdocs}{\textsf{subdocs}}
for further related packages.
The present package differs from the above solutions in that
a document structure constructed with the conventional |\include| mechanism
just needs two extra commands at the top of every file
such that all constituent files can be compiled individually.

%%%%%%%%%%%%%%%%%%%%%%%%%%%%%%%%%%%%%%%%%%%%%%%%%%%%%%%%%%%%%%%%%%%%%%%%%%%%%%%%
%\subsection{Feature Suggestions}
%
%The following is a list of features which may be useful for future
%versions of this package:
%%
%\begin{itemize}
%\item
%\ldots
%\end{itemize}

%%%%%%%%%%%%%%%%%%%%%%%%%%%%%%%%%%%%%%%%%%%%%%%%%%%%%%%%%%%%%%%%%%%%%%%%%%%%%%%%
\subsection{Revision History}

%%%%%%%%%%%%%%%%%%%%%%%%%%%%%%%%%%%%%%%%
\paragraph{v2.0:} 2018/12/30

\begin{itemize}
\item
immediate forward processing
\item
added |\childdocby| mechanism
\item
manual restructured
\end{itemize}

%%%%%%%%%%%%%%%%%%%%%%%%%%%%%%%%%%%%%%%%
\paragraph{v1.6:} 2018/01/17

\begin{itemize}
\item
application for development of include files
\item
corrections to manual
\end{itemize}

%%%%%%%%%%%%%%%%%%%%%%%%%%%%%%%%%%%%%%%%
\paragraph{v1.5:} 2017/05/21

\begin{itemize}
\item
more complete structuring introduced
\item
|\childdocof| introduced
\item
|\childdoc| renamed to |\childdocmain|
\item
|\childredirect| renamed to |\childdocforward| and |\childdocforwardprefix|
and functionality expanded
\end{itemize}

%%%%%%%%%%%%%%%%%%%%%%%%%%%%%%%%%%%%%%%%
\paragraph{v1.0:} 2017/04/27

\begin{itemize}
\item
manual and install package
\item
first version published on CTAN
\end{itemize}

%%%%%%%%%%%%%%%%%%%%%%%%%%%%%%%%%%%%%%%%
\paragraph{v0.6:} 2017/04/26

\begin{itemize}
\item
redirection mechanism added
\end{itemize}

%%%%%%%%%%%%%%%%%%%%%%%%%%%%%%%%%%%%%%%%
\paragraph{v0.5:} 2017/04/26

\begin{itemize}
\item
functionality in definition file
\end{itemize}


%%%%%%%%%%%%%%%%%%%%%%%%%%%%%%%%%%%%%%%%%%%%%%%%%%%%%%%%%%%%%%%%%%%%%%%%%%%%%%%%
%%%%%%%%%%%%%%%%%%%%%%%%%%%%%%%%%%%%%%%%%%%%%%%%%%%%%%%%%%%%%%%%%%%%%%%%%%%%%%%%
%%%%%%%%%%%%%%%%%%%%%%%%%%%%%%%%%%%%%%%%%%%%%%%%%%%%%%%%%%%%%%%%%%%%%%%%%%%%%%%%
\appendix

\settowidth\MacroIndent{\rmfamily\scriptsize 000\ }

 \DocInput{childdoc.dtx}

\end{document}
%</driver>
% \fi
%
% %%%%%%%%%%%%%%%%%%%%%%%%%%%%%%%%%%%%%%%%%%%%%%%%%%%%%%%%%%%%%%%%%%%%%%%%%%%%%%
% %%%%%%%%%%%%%%%%%%%%%%%%%%%%%%%%%%%%%%%%%%%%%%%%%%%%%%%%%%%%%%%%%%%%%%%%%%%%%%
% \section{Sample}
%\iffalse
%<*samplemain>
%\fi
%
% The following presents a sample document
% with two chapters, two parts, a title page,
% a compile flag as well as three forwarding files to set the flag.
% It consists of eight |.tex| files:
% \begin{center}
% \begin{tabular}{ll}
% |cdocsamp.tex|&main file\\
% |cdocsch1.tex|&include file for chapter 1\\
% |cdocsch2.tex|&include file for chapter 2\\
% |cdocspt3.tex|&include file for part 3\\
% |cdocspt4.tex|&include file for part 4\\
% |cdocsdrf.tex|&forwarding file for main file in draft mode\\
% |cdocsfi1.tex|&forwarding file for final version of chapter 1\\
% |cdocsfi2.tex|&forwarding file for final version of chapter 2\\
% \end{tabular}
% \end{center}
% Each of the eight files can be compiled directly by the \LaTeX{} compiler.
%
% %%%%%%%%%%%%%%%%%%%%%%%%%%%%%%%%%%%%%%
% \paragraph{Main File.}
%
% The main file is called |cdocsamp.tex|.
%
% Load the \textsf{childdoc} definitions and
% declare the filename for the main document:
%    \begin{macrocode}
\input{childdoc.def}
\childdocmain{}
%    \end{macrocode}

% Optional override for |\version| flag:
%    \begin{macrocode}
%%\ifchilddoc\else\providecommand{\version}{draft}\fi
%    \end{macrocode}

% Define the default values for the |\version| flag
% (|final| for the main file and |draft| for childs):
%    \begin{macrocode}
\ifchilddoc
\providecommand{\version}{draft}
\else
\providecommand{\version}{final}
\fi
%    \end{macrocode}

% Load the standard document class:
%    \begin{macrocode}
\documentclass[12pt]{article}
%    \end{macrocode}

% Start the document body:
%    \begin{macrocode}
\begin{document}
%    \end{macrocode}

% Declare a title page.
% Print title, part of document being processed and version flag:
%    \begin{macrocode}
\addtocounter{page}{-1}
\begin{center}
{\LARGE\bfseries{}childdoc example\par}
\vspace{1cm}
\ifchilddoc
\ifchilddocmanual part\else chapter\fi:
`\childdocname' of `\childdocjob'\par
\else
main document: `\childdocjob'\par
\fi
version: \version\par
\end{center}
\newpage
%    \end{macrocode}

% Manually include selected file,
% otherwise process as usual:
%    \begin{macrocode}
\ifchilddocmanual
\section*{part `\childdocname'}
\input{\childdocname}
\else
%    \end{macrocode}

% Include the two chapters:
%    \begin{macrocode}
\include{cdocsch1}
\include{cdocsch2}
%    \end{macrocode}

% Include the two parts unless only chapters should be displayed:
%    \begin{macrocode}
\ifchilddoc\else
\section{part three}
\input{cdocspt3}
\section{part four}
\input{cdocspt4}
\fi
%    \end{macrocode}

% Process as usual until here:
%    \begin{macrocode}
\fi
%    \end{macrocode}

% End of document body:
%    \begin{macrocode}
\end{document}
%    \end{macrocode}
%\iffalse
%</samplemain>
%\fi
%
% %%%%%%%%%%%%%%%%%%%%%%%%%%%%%%%%%%%%%%
% \paragraph{Chapter Include Files.}
%
% The include files are called |cdocsch1.tex| and |cdocsch2.tex|.
%
%\iffalse
%<*samplechap1|samplechap2>
%\fi

% Optional override for |\version| flag:
%    \begin{macrocode}
%%\providecommand{\version}{final}
%    \end{macrocode}

% Include the main document:
%    \begin{macrocode}
\input{childdoc.def}
\childdocof{cdocsamp}
%    \end{macrocode}

%\iffalse
%</samplechap1|samplechap2>
%\fi
%
%\iffalse
%<*samplechap1>
%\fi
% Some text for chapter 1:
%    \begin{macrocode}
\section{one}
some text in chapter one
%    \end{macrocode}

%\iffalse
%</samplechap1>
%\fi
% Some text for chapter 2:
%\iffalse
%<*samplechap2>
%\fi
%    \begin{macrocode}
\section{two}
more text in chapter two
%    \end{macrocode}

%\iffalse
%</samplechap2>
%\fi
%
% %%%%%%%%%%%%%%%%%%%%%%%%%%%%%%%%%%%%%%
% \paragraph{Part Include Files.}
%
% The include files are called |cdocspt3.tex| and |cdocspt4.tex|.
%
%\iffalse
%<*samplepart3|samplepart4>
%\fi

% Optional override for |\version| flag:
%    \begin{macrocode}
%%\providecommand{\version}{final}
%    \end{macrocode}

% Include the main document:
%    \begin{macrocode}
\input{childdoc.def}
\childdocby{cdocsamp}
%    \end{macrocode}

%\iffalse
%</samplepart3|samplepart4>
%\fi
%
%\iffalse
%<*samplepart3>
%\fi
% Some text for part 3:
%    \begin{macrocode}
some text in part three
%    \end{macrocode}

%\iffalse
%</samplepart3>
%\fi
% Some text for part 4:
%\iffalse
%<*samplepart4>
%\fi
%    \begin{macrocode}
more text in part four
%    \end{macrocode}

%\iffalse
%</samplepart4>
%\fi
%
% %%%%%%%%%%%%%%%%%%%%%%%%%%%%%%%%%%%%%%
% \paragraph{Forwarding for a Complete Draft.}
%
% The following forwarding file |cdocsdrf.tex|
% compiles the main document in draft mode:
%\iffalse
%<*sampledraft>
%\fi
%    \begin{macrocode}
\def\version{draft}
\input{childdoc.def}
\childdocforward{cdocsamp}
%    \end{macrocode}

%\iffalse
%</sampledraft>
%\fi
%
% %%%%%%%%%%%%%%%%%%%%%%%%%%%%%%%%%%%%%%
% \paragraph{Forwarding for Final Version of the Chapters.}
%
% The following forwarding files |cdocsfn1.tex| and |cdocsfn2.tex|
% (with identical content)
% compile the final versions of the child documents
% |cdocsch1.tex| and |cdocsch2.tex|, respectively:
%\iffalse
%<*samplefinal>
%\fi
%    \begin{macrocode}
\def\version{final}
\input{childdoc.def}
\childdocforwardprefix[cdocsamp]{cdocsfn}{cdocsch}
%    \end{macrocode}

%\iffalse
%</samplefinal>
%\fi
%
% %%%%%%%%%%%%%%%%%%%%%%%%%%%%%%%%%%%%%%
% \paragraph{Command Line Processing.}
%
% The following three command lines generate the output files
% |cdocscld|, |cdocscl1| and |cdocscl2|
% which should be identical to
% |cdocsdrf|, |cdocsch1| and |cdocsfn2|, respectively:
% \begin{center}
% \begin{tabular}{l}
% |latex -jobname cdocscld \|\\
% |  "\def\version{draft}\input{childdoc.def}\childdocforward{cdocsamp}"|\\
% |latex -jobname cdocscl1 \|\\
% |  "\input{childdoc.def}\childdocforward[cdocsamp]{cdocsch1}"|\\
% |latex -jobname cdocscl2 \|\\
% |  "\def\version{final}\input{childdoc.def}\childdocforward{cdocsch2}"|
% \end{tabular}
% \end{center}
% Note that the trailing backslash on each first line
% merely continues the input to the second line
% (for convenient cut ant paste).
% Furthermore, the command |latex| can be replaced by any
% of its alternative versions such as |pdflatex|.
%
% %%%%%%%%%%%%%%%%%%%%%%%%%%%%%%%%%%%%%%%%%%%%%%%%%%%%%%%%%%%%%%%%%%%%%%%%%%%%%%
% %%%%%%%%%%%%%%%%%%%%%%%%%%%%%%%%%%%%%%%%%%%%%%%%%%%%%%%%%%%%%%%%%%%%%%%%%%%%%%
% \section{Implementation}
%\iffalse
%<*package>
%\fi
%
% This section describes the definitions file |childdoc.def|.

% The definitions cannot be loaded using |\usepackage| or |\RequirePackage|
% which has a mechanism to prevent loading a style file more than once.
% When loading the definitions by means of |\input|
% multiple instances have to be prevented manually:
%\iffalse
%This code needs to be before the `\ProvidesFile' directive
%which is defined at the beginning of this file.
%Therefore it is also placed there and commented out here.
%</package>
%<*discard>
%\fi
%    \begin{macrocode}
\ifdefined\childdocmain\endinput\fi
%    \end{macrocode}
%\iffalse
%</discard>
%<*package>
%\fi
%
% \macro{\ifchilddoc}
% \macro{\ifchilddocmanual}
% The conditional |\ifchilddoc| tells whether a
% child (true) or main (false) document is being compiled.
% The conditional |\ifchilddocmanual| tells whether
% the |\includeonly| mechanism is used (false) or
% the selection of child files must be performed manually (true).
% The definitions initialise to false:
%    \begin{macrocode}
\newif\ifchilddoc
\newif\ifchilddocmanual
%    \end{macrocode}

% \macro{\childdocname}
% \macro{\childdocjob}
% The macro |\childdocname| stores the name of the main document
% to be compiled. The macro |\childdocjob| stores the name of
% the document on which the \LaTeX{} compiler was originally invoked.
% The content of |\jobname| cannot be compared
% to filenames specified in the source due to different catcodes.
% The following code rescans |\jobname|, stores the result
% in |\childdocname| and saves a copy in |\childdocjob|:
%    \begin{macrocode}
\edef\childdocname{\scantokens\expandafter{\jobname\noexpand}}
\let\childdocjob\childdocname
%    \end{macrocode}

% \macro{\childdocdisable}
% The macro |\childdocdisable| prevents the main file
% from being processed more than once.
% At this stage, the main document command |\childdocmain|
% is assumed to be called once again where it should do nothing.
% Any subsequent call to it should prevent
% a secondary processing of the main document
% It overwrites the forwarding commands
% |\childdocof| and |\childdocforward|
% with empty macros to prevent further inclusions of the main document:
%    \begin{macrocode}
\newcommand{\childdocdisable}
{
  \renewcommand{\childdocmain}[1]{\renewcommand{\childdocmain}[1]{\endinput}}
  \renewcommand{\childdocof}[1]{}
  \renewcommand{\childdocby}[2][]{}
  \renewcommand{\childdocforward}[2][]{}
  \renewcommand{\childdocdisable}{}
}
%    \end{macrocode}

% \macro{\childdocmain}
% The macro |\childdocmain| is to be called at the top of the main file
% with nothing or the main filename (without extension) as argument.
% First, it breaks loops.
% If the argument is not empty and does not match |\childdocname|
% (which is set by the first inclusion of |childdoc.def|),
% |\ifchilddoc| is set to true, |\includeonly| is applied to the child file
% and |\jobname| is set to the main file
% (for proper handling of |.aux| files):
%    \begin{macrocode}
\newcommand{\childdocmain}[1]
{
  \childdocdisable\childdocmain{}
  \if?#1?\else
    \begingroup
      \def\childdoctmp{#1}
      \ifx\childdoctmp\childdocname
        \def\childdoctmp{}
      \else
        \def\childdoctmp
        {
          \childdoctrue
          \includeonly{\childdocname}
          \def\childdocjob{#1}
          \def\jobname{#1}
        }
      \fi
      \expandafter
    \endgroup
    \childdoctmp
  \fi
}
%    \end{macrocode}

% \macro{\childdocof}
% The command |\childdocof| redirects
% compilation to the main file |#1|.
%    \begin{macrocode}
\newcommand{\childdocof}[1]
{
  \childdocdisable
  \childdoctrue
  \includeonly{\childdocname}
  \def\jobname{#1}
  \def\childdocjob{#1}
  \input{#1}
}
%    \end{macrocode}

% \macro{\childdocby}
% The command |\childdocby| ....
%    \begin{macrocode}
\newcommand{\childdocby}[2][]
{
  \childdocdisable
  \childdoctrue
  \childdocmanualtrue
  \if?#1?\else
    \def\jobname{#2}
  \fi
  \def\childdocjob{#2}
  \input{#2}
  \endinput
}
%    \end{macrocode}

% \macro{\childdocforward}
% The command |\childdocforward| redirects
% compilation to the main file or
% (if the optional argument is given) a child file.
% Parameters are set as if the main file
% or a child file starting with |\childdocof| was compiled.
% Then compilation is handed over to the main file:
%    \begin{macrocode}
\newcommand{\childdocforward}[2][]
{
  \begingroup
    \if?#1?
      \def\childdoctmp
      {
        \def\childdocname{#2}
        \def\childdocjob{#2}
        \def\jobname{#2}
        \input{#2}
        \endinput
      }
    \else
      \def\childdoctmp
      {
        \childdocdisable
        \def\childdocname{#2}
        \childdoctrue
        \includeonly{#2}
        \def\childdocjob{#1}
        \def\jobname{#1}
        \input{#1}
        \endinput
      }
    \fi
    \expandafter
  \endgroup
  \childdoctmp
}
%    \end{macrocode}

% \macro{\childdocforwardprefix}
% The command |\childdocforwardprefix| redirects
% compilation to the main or a child file by means of a pattern.
% The prefix |#1| in the current filename is replaced by |#2|
% and the suffix of the current filename is kept
% (it is assumed that the filename does not contain the substring `|~~~|'
% which is used as a delimiter).
% Compilation is handed over to the new file by |\childdocforward|:
%    \begin{macrocode}
\newcommand{\childdocforwardprefix}[3][]
{
  \begingroup
    \def\childdocextract #2##1~~~{\def\childdoctmp{\childdocforward[#1]{#3##1}}}
    \expandafter\childdocextract\childdocname~~~
    \expandafter
  \endgroup
  \childdoctmp
}
%    \end{macrocode}

% \macro{\childdoc}
% The deprecated macro |\childdoc| is a legacy version of |\childdocmain|:
%    \begin{macrocode}
\newcommand{\childdoc}{\childdocmain}
%    \end{macrocode}

% \macro{\childdocredirect}
% The deprecated macro |\childdocredirect| is a legacy version
% of |\childdocforward| and |\childdocforwardprefix|:
%    \begin{macrocode}
\newcommand{\childdocredirect}[2][]
{
  \begingroup
    \if?#1?
      \def\childdoctmp{\childdocforward{#2}}
    \else
      \def\childdoctmp{\childdocforwardprefix{#1}{#2}}
    \fi
    \expandafter
  \endgroup
  \childdoctmp
}
%    \end{macrocode}

%\iffalse
%</package>
%\fi
%
\endinput
\childdocforward[|\textit{main}|]{|\textit{dest}|}"|
\end{center}
%
Here \textit{target} is the name of the output file,
\textit{main} is the name of the main file
and \textit{dest} is the name of the main or child file to be processed
(all filenames without extensions).
The optional argument \textit{main} can be omitted
if \textit{main} matches \textit{dest}.
Optionally, compilation \textit{flags} can be defined via |\def| commands.
This command line makes the \TeX{} engine believe
it is compiling the file \textit{target}
whose content is specified as the latter parameter.
The provided code then forwards the processing to
\textit{main} or \textit{dest} as described in \secref{sec:forward}.

%%%%%%%%%%%%%%%%%%%%%%%%%%%%%%%%%%%%%%%%%%%%%%%%%%%%%%%%%%%%%%%%%%%%%%%%%%%%%%%%
\subsection{Include by Input}
\label{sec:input}

Including child documents by |\include| has some restrictions by design.
Most notably, the content of a child document always occupies
its own set of pages; pages cannot be shared between child documents.
Usually, this behaviour makes perfect sense
because each child document contain an essential part of the document.
However, in some situations it may be desirable to compose
a document from a collection of parts
without having mandatory page breaks between then.
For this case, the package
provides a mechanism to include parts
by |\input| which can also be processed individually.
However, by construction this mechanism
requires manual handling of the content to be output.

%%%%%%%%%%%%%%%%%%%%%%%%%%%%%%%%%%%%%%%%
\DescribeMacro{\ifchilddocmanual}
The main file should be prepared as usual, see \secref{sec:include}.
However, the document body must make a distinction
between processing of an individual part and of the main document, e.g.:
%
\begin{center}
\begin{tabular}{l}
|\ifchilddocmanual|\\
|\input{\childdocname}|\\
|\||else|\\
\textit{document body with }|\input{|\textit{part}|}|\\
|\||fi|
\end{tabular}
\end{center}
%
The conditional |\ifchilddocmanual| is true whenever
a part to be included by |\input| is being compiled,
and the name of the part is stored in |\childdocname|.

%%%%%%%%%%%%%%%%%%%%%%%%%%%%%%%%%%%%%%%%
\DescribeMacro{\childdocby}
Each part to be included by |\input| should start with:
%
\begin{center}
\begin{tabular}{l}
|% \iffalse
%
% childdoc.dtx Copyright (C) 2017-2018 Niklas Beisert
%
% This work may be distributed and/or modified under the
% conditions of the LaTeX Project Public License, either version 1.3
% of this license or (at your option) any later version.
% The latest version of this license is in
%   http://www.latex-project.org/lppl.txt
% and version 1.3 or later is part of all distributions of LaTeX
% version 2005/12/01 or later.
%
% This work has the LPPL maintenance status `maintained'.
%
% The Current Maintainer of this work is Niklas Beisert.
%
% This work consists of the files childdoc.dtx and childdoc.ins
% and the derived files childdoc.def and cdocsamp.tex with
% cdocsch1.tex, cdocsch2.tex, cdocsdrf.tex, cdocsfn1.tex, cdocsfn2.tex.
%
%<package>\ifdefined\childdocmain\endinput\fi
%<package>\ProvidesFile{childdoc.def}[2018/12/30 v2.0 child document driver]
%<samplemain>\ProvidesFile{cdocsamp.tex}[2018/12/30 v2.0 sample for childdoc]
%<*driver>
%\ProvidesFile{childdoc.drv}[2018/12/30 v2.0 childdoc reference manual file]
\PassOptionsToClass{10pt,a4paper}{article}
\documentclass{ltxdoc}

\usepackage[margin=35mm]{geometry}
\usepackage{hyperref}
\usepackage{hyperxmp}
\usepackage[usenames]{color}

\hypersetup{colorlinks=true}
\hypersetup{pdfstartview=FitH}
\hypersetup{pdfpagemode=UseNone}
\hypersetup{pdfsource={}}
\hypersetup{pdflang={en-UK}}
\hypersetup{pdfcopyright={Copyright 2017-2018 Niklas Beisert.
  This work may be distributed and/or modified under the
  conditions of the LaTeX Project Public License, either version 1.3
  of this license or (at your option) any later version.}}
\hypersetup{pdflicenseurl={http://www.latex-project.org/lppl.txt}}
\hypersetup{pdfcontactaddress={ETH Zurich, ITP, HIT K,
  Wolfgang-Pauli-Strasse 27}}
\hypersetup{pdfcontactpostcode={8093}}
\hypersetup{pdfcontactcity={Zurich}}
\hypersetup{pdfcontactcountry={Switzerland}}
\hypersetup{pdfcontactemail={nbeisert@itp.phys.ethz.ch}}
\hypersetup{pdfcontacturl={http://people.phys.ethz.ch/\xmptilde nbeisert/}}

\newcommand{\secref}[1]{\hyperref[#1]{section \ref*{#1}}}

\parskip1ex
\parindent0pt
\let\olditemize\itemize
\def\itemize{\olditemize\parskip0pt}

\begin{document}

\title{The \textsf{childdoc} Package}
\hypersetup{pdftitle={The childdoc Package}}
\author{Niklas Beisert\\[2ex]
  Institut f\"ur Theoretische Physik\\
  Eidgen\"ossische Technische Hochschule Z\"urich\\
  Wolfgang-Pauli-Strasse 27, 8093 Z\"urich, Switzerland\\[1ex]
  \href{mailto:nbeisert@itp.phys.ethz.ch}
  {\texttt{nbeisert@itp.phys.ethz.ch}}}
\hypersetup{pdfauthor={Niklas Beisert}}
\hypersetup{pdfsubject={Manual for the LaTeX2e Package childdoc}}
\date{30 December 2018, \textsf{v2.0}}
\maketitle

\begin{abstract}\noindent
\textsf{childdoc} is a \LaTeXe{} package
that enables the direct compilation
of document sections included by |\include|
to individual files.
\end{abstract}

\begingroup
\parskip0ex
\tableofcontents
\endgroup

%%%%%%%%%%%%%%%%%%%%%%%%%%%%%%%%%%%%%%%%%%%%%%%%%%%%%%%%%%%%%%%%%%%%%%%%%%%%%%%%
%%%%%%%%%%%%%%%%%%%%%%%%%%%%%%%%%%%%%%%%%%%%%%%%%%%%%%%%%%%%%%%%%%%%%%%%%%%%%%%%
\section{Introduction}

\LaTeX{} provides a mechanism to structure a large document (such as a book)
into a main file and several child files (containing the chapters)
using the |\include| command.
This mechanism is beneficial for documents
which span hundreds of pages in order to
make the source file(s) more manageable.
Moreover, compilation can be restricted to
selected child files by means of the |\includeonly| command.
The latter feature can be used to reduce the compilation time while editing
(this was significantly more useful in the earlier days of \LaTeX{})
or to generate a smaller document which is easier to navigate.
Another application of |\includeonly| is to generate
documents consisting of selected parts of the complete document.

However, there are a few drawbacks of the plain |\include| mechanism:
\begin{itemize}
\item
The child files cannot be compiled on their own,
they can only be compiled via the main file.
A naive editing environment
(such as a text editor with an option
to have the current file processed by \LaTeX)
may require one to switch to the main file before compiling;
attempting to compile the child file produces errors.
\item
The main file must be modified (each time)
to adjust the |\includeonly| command
to the present needs. This easily leaves the main file in a messy state.
\item
The generated document will always carry the filename
of the main document. This is inconvenient if
several child files are to be compiled and
to be kept for distribution.
\end{itemize}

The present package provides a simple interface
to make child files individually compilable by \LaTeX{}.
Compiling a child file then has the same effect as compiling
the main file with an |\includeonly| command
to select the appropriate child.
Moreover the generated document will carry the name of the child
rather than the main file.
This resolves all three above issues.

This feature is meant to make the editing of books,
thesis documents and lecture notes somewhat more convenient.
However, the package can also be used efficiently for
composing a series of documents (such as exercise sheets)
which are typically distributed individually.
It then assists the author in generating the individual documents
(potentially in different versions)
as well as a document containing the collected series.
Another application is in developing style files
or other kinds of included material
where compilation of the style file could redirect
to a sample or test file.

%%%%%%%%%%%%%%%%%%%%%%%%%%%%%%%%%%%%%%%%%%%%%%%%%%%%%%%%%%%%%%%%%%%%%%%%%%%%%%%%
%%%%%%%%%%%%%%%%%%%%%%%%%%%%%%%%%%%%%%%%%%%%%%%%%%%%%%%%%%%%%%%%%%%%%%%%%%%%%%%%
\section{Usage}

First of all, the package \textsf{childdoc} is \emph{not} a standard
\LaTeXe{} |.sty| style file! Therefore it needs to be invoked in
a non-standard way.

%%%%%%%%%%%%%%%%%%%%%%%%%%%%%%%%%%%%%%%%%%%%%%%%%%%%%%%%%%%%%%%%%%%%%%%%%%%%%%%%
\subsection{Included Files}
\label{sec:include}

%%%%%%%%%%%%%%%%%%%%%%%%%%%%%%%%%%%%%%%%
\DescribeMacro{\childdocmain}
To use the package, add the commands
\begin{center}
\begin{tabular}{l}
|\input{childdoc.def}|\\
|\childdocmain{}|\\
\end{tabular}
\end{center}
at the very top of the main \LaTeX{} file,
in particular \emph{before} the |\documentclass| statement!
The argument of |\childdocmain| should be left empty
(but it must be present).

%%%%%%%%%%%%%%%%%%%%%%%%%%%%%%%%%%%%%%%%
\DescribeMacro{\childdocof}
Furthermore, add the commands
\begin{center}
\begin{tabular}{l}
|\input{childdoc.def}|\\
|\childdocof{|\textit{main}|}|\\
\end{tabular}
\end{center}
at the top of every child file \textit{child}
which is included by |\include{|\textit{child}|}|
from within the main file
(or at least for those files to be compiled individually).
The argument \textit{main} must be the filename of the main file.

There are a couple of
considerations in setting up the main and child documents:

%%%%%%%%%%%%%%%%%%%%%%%%%%%%%%%%%%%%%%%%
\paragraph{Restrictions.}

Please note the following restrictions:
\begin{itemize}
\item
|\childdocmain| must be called with one argument \textit{main}
to ensure compatibility with earlier version of the package.
It must either be empty (|\childdocmain{}|)
or precisely match the filename of the main file in which it is specified.
See \secref{sec:detection} for further information.
\item
The filename \textit{main} must be specified without the |.tex| extension.
\item
The filename \textit{main} is case sensitive
(even in case-insensitive file systems)
due to internal string comparison.
\item
The argument \textit{main} should be fully expanded, it cannot be a macro.
\item
Subdirectories and special characters should be avoided in filenames.
\item
The command |\childdocmain{|\textit{main}|}| must be followed by a whitespace.
It should not be followed immediately by another command
or by a comment mark `|%|'.
This is because the \TeX{} parser reads the token immediately following
the argument of |\childdocmain| and puts it
at the beginning of every child section;
however, a white\-space is ignored.
\end{itemize}

%%%%%%%%%%%%%%%%%%%%%%%%%%%%%%%%%%%%%%%%
\paragraph{Content of Main File.}

It is advisable to place all content in the child files included by |\include|.
Any output contained in the main file will appear in all child documents
unless suppressed manually;
it cannot be suppressed automatically by the |\includeonly| directive
and thus should normally be avoided.
A method to include some content in the main file
by means of conditional processing is described in \secref{sec:conditional}.

%%%%%%%%%%%%%%%%%%%%%%%%%%%%%%%%%%%%%%%%
\paragraph{Page Numbering.}

When only a part of the document is compiled,
the appropriate numbering of pages
(as well as other status parameters)
is determined from the |.aux| files.
The latter contain information from previous passes.
However this information needs to propagate through
all intermediate child documents.
Therefore the page numbering in child documents may well
be inconsistent until the complete document is compiled at least once.

A useful (if unconventional) way to always ensure a consistent
page numbering is to restart the numbering in each child document
and denote the pages by `\textit{child}|.|\textit{page}'
where \textit{child} represents the chapter/section number of the child file.
This can be achieved by the command
|\numberwithin{page}{|\textit{child}|}|
of the \textsf{amsmath} package
where \textit{child} can be |chapter| or |section|
depending on the chosen structuring.
Alternatively, one can modify the macro |\thepage| appropriately
and reset the counter |page| at the start of each child file.

%%%%%%%%%%%%%%%%%%%%%%%%%%%%%%%%%%%%%%%%%%%%%%%%%%%%%%%%%%%%%%%%%%%%%%%%%%%%%%%%
\subsection{Conditional Processing}
\label{sec:conditional}

The package provides a mechanism to compile different versions
of a document. To customise the versions further some conditional processing
can come in handy to distinguish which version is being compiled.
The package provides two macros to describe the compilation context:

%%%%%%%%%%%%%%%%%%%%%%%%%%%%%%%%%%%%%%%%
\DescribeMacro{\ifchilddoc}
The conditional |\ifchilddoc| distinguishes between the compilation of
child documents and the main document:
%
\begin{center}
|\ifchilddoc |\textit{child-code}| |[|\||else |\textit{main-code}]| \||fi|
\end{center}

%%%%%%%%%%%%%%%%%%%%%%%%%%%%%%%%%%%%%%%%
\DescribeMacro{\childdocname}
\DescribeMacro{\childdocjob}
The macro |\childdocname| contains the filename (without extension)
of the main or child file being processed.
Note that |\childdocjob| will always contain the name of the main file.

%%%%%%%%%%%%%%%%%%%%%%%%%%%%%%%%%%%%%%%%
\paragraph{Title Page.}

Conditional processing can be used to include a title or banner page
in the main document when proper precautions are taken.
Importantly, the code in the main file should ensure that the page counter
(as well as other status parameters which are stored in the |.aux| files)
takes the same value after the conditional processing.
Otherwise the page numbers may take divergent values
depending on which part is compiled.

For example, a title page could be declared by:
%
\begin{center}
\begin{tabular}{l}
|\ifchilddoc\||else|\\
|\addtocounter{page}{-1}|\\
\textit{code for title page}\\
|\newpage|\\
|\||fi|
\end{tabular}
\end{center}
%
A banner page for the child documents can be generated by:
%
\begin{center}
\begin{tabular}{l}
|\ifchilddoc|\\
|\addtocounter{page}{-1}|\\
\textit{code for banner page}\\
|\newpage|\\
|\||fi|
\end{tabular}
\end{center}
%
Here one could write a message such as:
\begin{center}
|This is the part \childdocname{} of \childdocjob{}.|
\end{center}

%%%%%%%%%%%%%%%%%%%%%%%%%%%%%%%%%%%%%%%%%%%%%%%%%%%%%%%%%%%%%%%%%%%%%%%%%%%%%%%%
\subsection{Flags}
\label{sec:flags}

The package makes it easy to generate different versions
of the main or child documents.
To this end compilation flags can be defined
and assigned different default values.
They will be particularly useful in conjunction
with the forwarding mechanism described in \secref{sec:forward}.

For example, it may be useful to have a flag |\version|
which can be set to |draft| or |final|.
The document source will contain some conditional code
depending on the value of |\version|.
Suppose further, the flag should default to |final| for the main file
and to |draft| for child files
which is a natural assignment for editing the document.
This is achieved by placing the following code
in the preamble of the main document
(below the |\childdocmain| directive):
%
\begin{center}
\begin{tabular}{l}
|\ifchilddoc|\\
|\providecommand{\version}{draft}|\\
|\||else|\\
|\providecommand{\version}{final}|\\
|\||fi|
\end{tabular}
\end{center}
%
The definition by |\providecommand| makes sure
that previous definitions are not overwritten.
Further statements |\providecommand{\version}{...}|
can thus be added before the above code to override it.

For the main file, one might add a line
(between |\childdocmain| and the above block)
%
\begin{center}
|%\ifchilddoc\||else\providecommand{\version}{draft}\||fi|
\end{center}
%
which can be uncommented to produce a draft version.
Likewise one can add a line to the very top of a child file
(above the |\childdocof{|\textit{main}|}| directive)
%
\begin{center}
|%\providecommand{\version}{final}|
\end{center}
%
which can be uncommented to produce the final version of this child document.

%%%%%%%%%%%%%%%%%%%%%%%%%%%%%%%%%%%%%%%%%%%%%%%%%%%%%%%%%%%%%%%%%%%%%%%%%%%%%%%%
\subsection{Forwarding}
\label{sec:forward}

Different versions of the main or child documents
using compilation flags as described in \secref{sec:flags}
can be (permanently) stored in different files
for convenient compilation, viewing and distribution.
To this end, the package defines a command
to pass on compilation to a different file:

%%%%%%%%%%%%%%%%%%%%%%%%%%%%%%%%%%%%%%%%
\DescribeMacro{\childdocforward}
The command |\childdocforward| redirects processing to
another source file:
%
\begin{center}
\begin{tabular}{l}
|\input{childdoc.def}|\\
|\childdocforward[|\textit{main}|]{|\textit{dest}|}|\\
\end{tabular}
\end{center}
%
The argument \textit{dest} is the destination file
(without extension).
It should be the main file or one of the child files.
Note that further \textsf{childdoc} directives
such as |\childdocof| and |\childdocforward|
in the indicated file will be processed in this form.
The optional argument \textit{main}
passes on directly to the main file \textit{main}
while pretending to compile the child \textit{dest}.
This form behaves as if \textit{dest}
issues |\childdocof{|\textit{main}|}| right away,
and no further \textsf{childdoc} directives will be processed.

%%%%%%%%%%%%%%%%%%%%%%%%%%%%%%%%%%%%%%%%
\DescribeMacro{\...prefix}
In the alternative form |\childdocforwardprefix|,
%
\begin{center}
\begin{tabular}{l}
|\input{childdoc.def}|\\
|\childdocforwardprefix[|\textit{main}|]{|\textit{prefix}|}{|\textit{dest}|}|
\end{tabular}
\end{center}
%
the destination file is determined by a pattern
depending on the current file:
To make this work, the current file must be called
`{\textit{prefix}\hspace{0.2em}\textit{suffix}}'
with \textit{prefix} matching precisely the argument.
Processing is then passed on to the file
`{\textit{dest}\hspace{0.2em}\textit{suffix}}'.
Surely, the same effect is achieved by
directly specifying the
argument `{\textit{dest}\hspace{0.2em}\textit{suffix}}'
in the first form.
However, that requires to set up a different file
for each child. With the alternative form of the command
all these files can have exactly the same content
which simplifies setting them up and maintaining them.

For example, the following file |draft.tex|
with a compilation flag |\version| as described in \secref{sec:flags}
compiles the main document as a draft:
%
\begin{center}
\begin{tabular}{l}
|\def\version{draft}|\\
|\input{childdoc.def}|\\
|\childdocforward{|\textit{main}|}|
\end{tabular}
\end{center}
%
Likewise, the following files |final|\textit{nn}|.tex|
compile the final version of the child document
|child|\textit{nn}|.tex|:
%
\begin{center}
\begin{tabular}{l}
|\def\version{final}|\\
|\input{childdoc.def}|\\
|\childdocforwardprefix{final}{child}|
\end{tabular}
\end{center}
%

Note that when several versions of a main file and/or of each child file
are to be generated, it may be convenient to set up a |Makefile| or
shell script to automatise the process.

%%%%%%%%%%%%%%%%%%%%%%%%%%%%%%%%%%%%%%%%%%%%%%%%%%%%%%%%%%%%%%%%%%%%%%%%%%%%%%%%
\subsection{Command Line Processing}
\label{sec:commandline}

The effect of redirection files can also be achieved by invoking
the \LaTeX{} compiler with a more elaborate command line.
Most conveniently this should be done as part
of a shell script or a |Makefile|.

When using \textsf{childdoc} in the main file, the following
command lines effectively perform a redirection
(note that depending on the shell being used,
backslashes may have to be doubled: `|\|' $\to$ `|\\|'):
%
\begin{center}
|... -jobname "|\textit{target}|" |\\|"|[\textit{flags}]%
|\input{childdoc.def}\childdocforward[|\textit{main}|]{|\textit{dest}|}"|
\end{center}
%
Here \textit{target} is the name of the output file,
\textit{main} is the name of the main file
and \textit{dest} is the name of the main or child file to be processed
(all filenames without extensions).
The optional argument \textit{main} can be omitted
if \textit{main} matches \textit{dest}.
Optionally, compilation \textit{flags} can be defined via |\def| commands.
This command line makes the \TeX{} engine believe
it is compiling the file \textit{target}
whose content is specified as the latter parameter.
The provided code then forwards the processing to
\textit{main} or \textit{dest} as described in \secref{sec:forward}.

%%%%%%%%%%%%%%%%%%%%%%%%%%%%%%%%%%%%%%%%%%%%%%%%%%%%%%%%%%%%%%%%%%%%%%%%%%%%%%%%
\subsection{Include by Input}
\label{sec:input}

Including child documents by |\include| has some restrictions by design.
Most notably, the content of a child document always occupies
its own set of pages; pages cannot be shared between child documents.
Usually, this behaviour makes perfect sense
because each child document contain an essential part of the document.
However, in some situations it may be desirable to compose
a document from a collection of parts
without having mandatory page breaks between then.
For this case, the package
provides a mechanism to include parts
by |\input| which can also be processed individually.
However, by construction this mechanism
requires manual handling of the content to be output.

%%%%%%%%%%%%%%%%%%%%%%%%%%%%%%%%%%%%%%%%
\DescribeMacro{\ifchilddocmanual}
The main file should be prepared as usual, see \secref{sec:include}.
However, the document body must make a distinction
between processing of an individual part and of the main document, e.g.:
%
\begin{center}
\begin{tabular}{l}
|\ifchilddocmanual|\\
|\input{\childdocname}|\\
|\||else|\\
\textit{document body with }|\input{|\textit{part}|}|\\
|\||fi|
\end{tabular}
\end{center}
%
The conditional |\ifchilddocmanual| is true whenever
a part to be included by |\input| is being compiled,
and the name of the part is stored in |\childdocname|.

%%%%%%%%%%%%%%%%%%%%%%%%%%%%%%%%%%%%%%%%
\DescribeMacro{\childdocby}
Each part to be included by |\input| should start with:
%
\begin{center}
\begin{tabular}{l}
|\input{childdoc.def}|\\
|\childdocby{|\textit{main}|}|\\
\end{tabular}
\end{center}
%
The directive |\childdocby| is similar to |\childdocof|
described in \secref{sec:include},
but the subsequent selection of content must be done manually.
To that end, both |\ifchilddoc| and |\ifchilddocmanual|
will be true upon processing of a part,
and the name of the part is stored in |\childdocname|.
Note that |\jobname| will be set to the filename of the current part
so that each part receives an individual |.aux| file
that does not interfere with the |.aux| file(s) of the main document.
This behaviour can be altered by the alternative form
|\childdocby[*]{|\textit{main}|}| (with a non-empty optional argument)
which uses the |.aux| file of the main document
by setting |\jobname| to \textit{main}.

%%%%%%%%%%%%%%%%%%%%%%%%%%%%%%%%%%%%%%%%%%%%%%%%%%%%%%%%%%%%%%%%%%%%%%%%%%%%%%%%
\subsection{Driver Development}
\label{sec:driver}

The \textsf{childdoc} mechanism can also be use for the development
of definition files such as \LaTeX{} styles or classes.
This case differs from the above setup with multiple parts
included by |\include| in that no |\includeonly| should be invoked.
This can be achieved by starting the include file
(before |\ProvidesPackage|) with:
%
\begin{center}
\begin{tabular}{l}
|\input{childdoc.def}|\\
|\childdocforward{|\textit{main}|}|\\
\end{tabular}
\end{center}
%
or alternatively with:
%
\begin{center}
\begin{tabular}{l}
|\input{childdoc.def}|\\
|\childdocby{|\textit{main}|}|\\
\end{tabular}
\end{center}
%
Both forms have slightly different effects as described above.
The main file is prepared as usual, see \secref{sec:include}.

%%%%%%%%%%%%%%%%%%%%%%%%%%%%%%%%%%%%%%%%%%%%%%%%%%%%%%%%%%%%%%%%%%%%%%%%%%%%%%%%
\subsection{Legacy Detection}
\label{sec:detection}

The directive |\childdocmain| in the main file can detect
whether the complete document or merely a child is to be compiled
even without using the directive |\childdocof|.
This method is deprecated because it is less robust
and there is no compelling reason to use it;
it is merely provided for backward compatibility
and it may be removed in future versions.

If the detection mechanism is to be used,
it is mandatory to correctly specify
the filename of the main file as the argument of |\childdocmain|:
%
\begin{center}
\begin{tabular}{l}
|\input{childdoc.def}|\\
|\childdocmain{|\textit{main}|}|\\
\end{tabular}
\end{center}
%
If |\jobname| does not match the argument \textit{main} of |\childdocmain|,
it is assumed that |\jobname| points to the child file to be compiled.
When using |\childdocmain| with the main file specified as argument,
it suffices to start a child file
with just |\input{|\textit{main}|}|
without loading of the package and using |\childdocof|.
If instead all processing is done
with the appropriate \textsf{childdoc} directives,
the argument of \textit{main} of |\childdocmain| can be empty.

An alternative version of the command line processing described
in \secref{sec:commandline} using the detection mechanism reads:
%
\begin{center}
|... -jobname "|\textit{target}|" "|[\textit{flags}]%
[|\def\jobname{|\textit{dest}|}|]|\input{|\textit{main}|}"|
\end{center}

%%%%%%%%%%%%%%%%%%%%%%%%%%%%%%%%%%%%%%%%%%%%%%%%%%%%%%%%%%%%%%%%%%%%%%%%%%%%%%%%
\subsection{Manual Code}
\label{sec:manual}

In case one cannot be certain whether the definitions file |childdoc.def|
is installed on the target \TeX{} distribution
and one prefers not to ship it,
it is conceivable to paste a few relevant commands into the sources.

To that end, drop all statements |\input{childdoc.def}|
and perform the replacements as outlined below.
Instead of |\childdocmain{|\textit{main}|}| add the following code
to the top of the main file:
%
\begin{center}
\begin{tabular}{l}
|\||ifdefined\childdocname\endinput\||fi\newif\ifchilddoc|\\
|\edef\childdocname{\scantokens\expandafter{\jobname\noexpand}}|\\
|\def\childdocmain{|\textit{main}|}\||ifx\childdocmain\childdocname\||else|\\
|\childdoctrue\includeonly{\childdocname}\let\jobname\childdocmain\||fi|\\
\end{tabular}
\end{center}
%
Instead of |\childdocof{|\textit{main}|}| just include the main file
at the top of each child file:
%
\begin{center}
|\input{|\textit{main}|}|
\end{center}
%
A simple redirection |\childdocforward{|\textit{dest}|}| is achieved by:
%
\begin{center}
|\def\jobname{|\textit{dest}|}\input{\jobname}|
\end{center}
%
The redirection with prefix
|\childdocforwardprefix[|\textit{prefix}|]{|\textit{dest}|}|
is accomplished by:
%
\begin{center}
\begin{tabular}{l}
|{\edef\jobname{\scantokens\expandafter{\jobname\noexpand}}|\\
|\def\redirectjob |\textit{prefix}|#1~~~{\gdef\jobname{|\textit{dest}|#1}}|\\
|\expandafter\redirectjob\jobname~~~}\input{\jobname}|
\end{tabular}
\end{center}

In an alternative approach,
child documents can be compiled by a specific command line
without additional code or specific definitions:
%
\begin{center}
|... -jobname "|\textit{target}|" "|[\textit{flags}]%
|\includeonly{|\textit{dest}|}\input{|\textit{main}|}"|
\end{center}
%

%%%%%%%%%%%%%%%%%%%%%%%%%%%%%%%%%%%%%%%%%%%%%%%%%%%%%%%%%%%%%%%%%%%%%%%%%%%%%%%%
%%%%%%%%%%%%%%%%%%%%%%%%%%%%%%%%%%%%%%%%%%%%%%%%%%%%%%%%%%%%%%%%%%%%%%%%%%%%%%%%
\section{Information}

%%%%%%%%%%%%%%%%%%%%%%%%%%%%%%%%%%%%%%%%%%%%%%%%%%%%%%%%%%%%%%%%%%%%%%%%%%%%%%%%
\subsection{Copyright}

Copyright \copyright{} 2017--2018 Niklas Beisert

This work may be distributed and/or modified under the
conditions of the \LaTeX{} Project Public License, either version 1.3
of this license or (at your option) any later version.
The latest version of this license is in
  \url{http://www.latex-project.org/lppl.txt}
and version 1.3 or later is part of all distributions of \LaTeX{}
version 2005/12/01 or later.

This work has the LPPL maintenance status `maintained'.

The Current Maintainer of this work is Niklas Beisert.

This work consists of the files |README.txt|, |childdoc.ins| and |childdoc.dtx|
as well as the derived files |childdoc.def|, |cdocsamp.tex|
with |cdocsch1.tex|, |cdocsch2.tex|, |cdocspt3.tex|, |cdocspt4.tex|,
|cdocsdrf.tex|, |cdocsfn1.tex|, |cdocsfn2.tex|
as well as |childdoc.pdf|.

%%%%%%%%%%%%%%%%%%%%%%%%%%%%%%%%%%%%%%%%%%%%%%%%%%%%%%%%%%%%%%%%%%%%%%%%%%%%%%%%
\subsection{Files and Installation}

The package consists of the files:
%
\begin{center}
\begin{tabular}{ll}
    |README.txt|   & readme file \\
    |childdoc.ins| & installation file \\
    |childdoc.dtx| & source file \\
    |childdoc.def| & definition file \\
    |cdocsamp.tex| & sample main file \\
    |cdocsch1.tex| & sample include file \\
    |cdocsch2.tex| & sample include file \\
    |cdocspt3.tex| & sample part file \\
    |cdocspt4.tex| & sample part file \\
    |cdocsdrf.tex| & sample redirection file \\
    |cdocsfn1.tex| & sample redirection file \\
    |cdocsfn2.tex| & sample redirection file \\
    |childdoc.pdf| & manual
\end{tabular}
\end{center}
%
The distribution consists of the files
|README.txt|, |childdoc.ins| and |childdoc.dtx|.
%
\begin{itemize}
\item
Run (pdf)\LaTeX{} on |childdoc.dtx|
to compile the manual |childdoc.pdf| (this file).
\item
Run \LaTeX{} on |childdoc.ins| to create the definitions file |childdoc.def|
and the sample |cdocsamp.tex| with include files
|cdocsch1.tex|, |cdocsch2.tex|, |cdocspt3.tex|, |cdocspt4.tex|,
|cdocsdrf.tex|, |cdocsfn1.tex|, |cdocsfn2.tex|.
Then copy the file |childdoc.def| to an appropriate directory of your \LaTeX{}
distribution, e.g.\ \textit{texmf-root}|/tex/latex/childdoc|.
\end{itemize}

%%%%%%%%%%%%%%%%%%%%%%%%%%%%%%%%%%%%%%%%%%%%%%%%%%%%%%%%%%%%%%%%%%%%%%%%%%%%%%%%
\subsection{Related CTAN Packages}

There are several other packages which offer a similar functionality:
%
\begin{itemize}
\item
The packages
\href{http://ctan.org/pkg/docmute}{\textsf{docmute}},
\href{http://ctan.org/pkg/includex}{\textsf{includex}} and
\href{http://ctan.org/pkg/standalone}{\textsf{standalone}}
provide commands to include only the document body of
a child file thus allowing both files to be compiled individually.
\item
The packages \href{http://ctan.org/pkg/subdocs}{\textsf{subdocs}}
and \href{http://ctan.org/pkg/subfiles}{\textsf{subfiles}}
provide structures in which the main and child documents can be
encapsulated and allowing them to be compiled individually.
The inclusion mechanism is different from the conventional |\include|.
\item
The package \href{http://ctan.org/pkg/combine}{\textsf{combine}}
is an elaborate solution to combine several documents into one.
\end{itemize}
%
See also the CTAN topic \href{http://ctan.org/topic/subdocs}{\textsf{subdocs}}
for further related packages.
The present package differs from the above solutions in that
a document structure constructed with the conventional |\include| mechanism
just needs two extra commands at the top of every file
such that all constituent files can be compiled individually.

%%%%%%%%%%%%%%%%%%%%%%%%%%%%%%%%%%%%%%%%%%%%%%%%%%%%%%%%%%%%%%%%%%%%%%%%%%%%%%%%
%\subsection{Feature Suggestions}
%
%The following is a list of features which may be useful for future
%versions of this package:
%%
%\begin{itemize}
%\item
%\ldots
%\end{itemize}

%%%%%%%%%%%%%%%%%%%%%%%%%%%%%%%%%%%%%%%%%%%%%%%%%%%%%%%%%%%%%%%%%%%%%%%%%%%%%%%%
\subsection{Revision History}

%%%%%%%%%%%%%%%%%%%%%%%%%%%%%%%%%%%%%%%%
\paragraph{v2.0:} 2018/12/30

\begin{itemize}
\item
immediate forward processing
\item
added |\childdocby| mechanism
\item
manual restructured
\end{itemize}

%%%%%%%%%%%%%%%%%%%%%%%%%%%%%%%%%%%%%%%%
\paragraph{v1.6:} 2018/01/17

\begin{itemize}
\item
application for development of include files
\item
corrections to manual
\end{itemize}

%%%%%%%%%%%%%%%%%%%%%%%%%%%%%%%%%%%%%%%%
\paragraph{v1.5:} 2017/05/21

\begin{itemize}
\item
more complete structuring introduced
\item
|\childdocof| introduced
\item
|\childdoc| renamed to |\childdocmain|
\item
|\childredirect| renamed to |\childdocforward| and |\childdocforwardprefix|
and functionality expanded
\end{itemize}

%%%%%%%%%%%%%%%%%%%%%%%%%%%%%%%%%%%%%%%%
\paragraph{v1.0:} 2017/04/27

\begin{itemize}
\item
manual and install package
\item
first version published on CTAN
\end{itemize}

%%%%%%%%%%%%%%%%%%%%%%%%%%%%%%%%%%%%%%%%
\paragraph{v0.6:} 2017/04/26

\begin{itemize}
\item
redirection mechanism added
\end{itemize}

%%%%%%%%%%%%%%%%%%%%%%%%%%%%%%%%%%%%%%%%
\paragraph{v0.5:} 2017/04/26

\begin{itemize}
\item
functionality in definition file
\end{itemize}


%%%%%%%%%%%%%%%%%%%%%%%%%%%%%%%%%%%%%%%%%%%%%%%%%%%%%%%%%%%%%%%%%%%%%%%%%%%%%%%%
%%%%%%%%%%%%%%%%%%%%%%%%%%%%%%%%%%%%%%%%%%%%%%%%%%%%%%%%%%%%%%%%%%%%%%%%%%%%%%%%
%%%%%%%%%%%%%%%%%%%%%%%%%%%%%%%%%%%%%%%%%%%%%%%%%%%%%%%%%%%%%%%%%%%%%%%%%%%%%%%%
\appendix

\settowidth\MacroIndent{\rmfamily\scriptsize 000\ }

 \DocInput{childdoc.dtx}

\end{document}
%</driver>
% \fi
%
% %%%%%%%%%%%%%%%%%%%%%%%%%%%%%%%%%%%%%%%%%%%%%%%%%%%%%%%%%%%%%%%%%%%%%%%%%%%%%%
% %%%%%%%%%%%%%%%%%%%%%%%%%%%%%%%%%%%%%%%%%%%%%%%%%%%%%%%%%%%%%%%%%%%%%%%%%%%%%%
% \section{Sample}
%\iffalse
%<*samplemain>
%\fi
%
% The following presents a sample document
% with two chapters, two parts, a title page,
% a compile flag as well as three forwarding files to set the flag.
% It consists of eight |.tex| files:
% \begin{center}
% \begin{tabular}{ll}
% |cdocsamp.tex|&main file\\
% |cdocsch1.tex|&include file for chapter 1\\
% |cdocsch2.tex|&include file for chapter 2\\
% |cdocspt3.tex|&include file for part 3\\
% |cdocspt4.tex|&include file for part 4\\
% |cdocsdrf.tex|&forwarding file for main file in draft mode\\
% |cdocsfi1.tex|&forwarding file for final version of chapter 1\\
% |cdocsfi2.tex|&forwarding file for final version of chapter 2\\
% \end{tabular}
% \end{center}
% Each of the eight files can be compiled directly by the \LaTeX{} compiler.
%
% %%%%%%%%%%%%%%%%%%%%%%%%%%%%%%%%%%%%%%
% \paragraph{Main File.}
%
% The main file is called |cdocsamp.tex|.
%
% Load the \textsf{childdoc} definitions and
% declare the filename for the main document:
%    \begin{macrocode}
\input{childdoc.def}
\childdocmain{}
%    \end{macrocode}

% Optional override for |\version| flag:
%    \begin{macrocode}
%%\ifchilddoc\else\providecommand{\version}{draft}\fi
%    \end{macrocode}

% Define the default values for the |\version| flag
% (|final| for the main file and |draft| for childs):
%    \begin{macrocode}
\ifchilddoc
\providecommand{\version}{draft}
\else
\providecommand{\version}{final}
\fi
%    \end{macrocode}

% Load the standard document class:
%    \begin{macrocode}
\documentclass[12pt]{article}
%    \end{macrocode}

% Start the document body:
%    \begin{macrocode}
\begin{document}
%    \end{macrocode}

% Declare a title page.
% Print title, part of document being processed and version flag:
%    \begin{macrocode}
\addtocounter{page}{-1}
\begin{center}
{\LARGE\bfseries{}childdoc example\par}
\vspace{1cm}
\ifchilddoc
\ifchilddocmanual part\else chapter\fi:
`\childdocname' of `\childdocjob'\par
\else
main document: `\childdocjob'\par
\fi
version: \version\par
\end{center}
\newpage
%    \end{macrocode}

% Manually include selected file,
% otherwise process as usual:
%    \begin{macrocode}
\ifchilddocmanual
\section*{part `\childdocname'}
\input{\childdocname}
\else
%    \end{macrocode}

% Include the two chapters:
%    \begin{macrocode}
\include{cdocsch1}
\include{cdocsch2}
%    \end{macrocode}

% Include the two parts unless only chapters should be displayed:
%    \begin{macrocode}
\ifchilddoc\else
\section{part three}
\input{cdocspt3}
\section{part four}
\input{cdocspt4}
\fi
%    \end{macrocode}

% Process as usual until here:
%    \begin{macrocode}
\fi
%    \end{macrocode}

% End of document body:
%    \begin{macrocode}
\end{document}
%    \end{macrocode}
%\iffalse
%</samplemain>
%\fi
%
% %%%%%%%%%%%%%%%%%%%%%%%%%%%%%%%%%%%%%%
% \paragraph{Chapter Include Files.}
%
% The include files are called |cdocsch1.tex| and |cdocsch2.tex|.
%
%\iffalse
%<*samplechap1|samplechap2>
%\fi

% Optional override for |\version| flag:
%    \begin{macrocode}
%%\providecommand{\version}{final}
%    \end{macrocode}

% Include the main document:
%    \begin{macrocode}
\input{childdoc.def}
\childdocof{cdocsamp}
%    \end{macrocode}

%\iffalse
%</samplechap1|samplechap2>
%\fi
%
%\iffalse
%<*samplechap1>
%\fi
% Some text for chapter 1:
%    \begin{macrocode}
\section{one}
some text in chapter one
%    \end{macrocode}

%\iffalse
%</samplechap1>
%\fi
% Some text for chapter 2:
%\iffalse
%<*samplechap2>
%\fi
%    \begin{macrocode}
\section{two}
more text in chapter two
%    \end{macrocode}

%\iffalse
%</samplechap2>
%\fi
%
% %%%%%%%%%%%%%%%%%%%%%%%%%%%%%%%%%%%%%%
% \paragraph{Part Include Files.}
%
% The include files are called |cdocspt3.tex| and |cdocspt4.tex|.
%
%\iffalse
%<*samplepart3|samplepart4>
%\fi

% Optional override for |\version| flag:
%    \begin{macrocode}
%%\providecommand{\version}{final}
%    \end{macrocode}

% Include the main document:
%    \begin{macrocode}
\input{childdoc.def}
\childdocby{cdocsamp}
%    \end{macrocode}

%\iffalse
%</samplepart3|samplepart4>
%\fi
%
%\iffalse
%<*samplepart3>
%\fi
% Some text for part 3:
%    \begin{macrocode}
some text in part three
%    \end{macrocode}

%\iffalse
%</samplepart3>
%\fi
% Some text for part 4:
%\iffalse
%<*samplepart4>
%\fi
%    \begin{macrocode}
more text in part four
%    \end{macrocode}

%\iffalse
%</samplepart4>
%\fi
%
% %%%%%%%%%%%%%%%%%%%%%%%%%%%%%%%%%%%%%%
% \paragraph{Forwarding for a Complete Draft.}
%
% The following forwarding file |cdocsdrf.tex|
% compiles the main document in draft mode:
%\iffalse
%<*sampledraft>
%\fi
%    \begin{macrocode}
\def\version{draft}
\input{childdoc.def}
\childdocforward{cdocsamp}
%    \end{macrocode}

%\iffalse
%</sampledraft>
%\fi
%
% %%%%%%%%%%%%%%%%%%%%%%%%%%%%%%%%%%%%%%
% \paragraph{Forwarding for Final Version of the Chapters.}
%
% The following forwarding files |cdocsfn1.tex| and |cdocsfn2.tex|
% (with identical content)
% compile the final versions of the child documents
% |cdocsch1.tex| and |cdocsch2.tex|, respectively:
%\iffalse
%<*samplefinal>
%\fi
%    \begin{macrocode}
\def\version{final}
\input{childdoc.def}
\childdocforwardprefix[cdocsamp]{cdocsfn}{cdocsch}
%    \end{macrocode}

%\iffalse
%</samplefinal>
%\fi
%
% %%%%%%%%%%%%%%%%%%%%%%%%%%%%%%%%%%%%%%
% \paragraph{Command Line Processing.}
%
% The following three command lines generate the output files
% |cdocscld|, |cdocscl1| and |cdocscl2|
% which should be identical to
% |cdocsdrf|, |cdocsch1| and |cdocsfn2|, respectively:
% \begin{center}
% \begin{tabular}{l}
% |latex -jobname cdocscld \|\\
% |  "\def\version{draft}\input{childdoc.def}\childdocforward{cdocsamp}"|\\
% |latex -jobname cdocscl1 \|\\
% |  "\input{childdoc.def}\childdocforward[cdocsamp]{cdocsch1}"|\\
% |latex -jobname cdocscl2 \|\\
% |  "\def\version{final}\input{childdoc.def}\childdocforward{cdocsch2}"|
% \end{tabular}
% \end{center}
% Note that the trailing backslash on each first line
% merely continues the input to the second line
% (for convenient cut ant paste).
% Furthermore, the command |latex| can be replaced by any
% of its alternative versions such as |pdflatex|.
%
% %%%%%%%%%%%%%%%%%%%%%%%%%%%%%%%%%%%%%%%%%%%%%%%%%%%%%%%%%%%%%%%%%%%%%%%%%%%%%%
% %%%%%%%%%%%%%%%%%%%%%%%%%%%%%%%%%%%%%%%%%%%%%%%%%%%%%%%%%%%%%%%%%%%%%%%%%%%%%%
% \section{Implementation}
%\iffalse
%<*package>
%\fi
%
% This section describes the definitions file |childdoc.def|.

% The definitions cannot be loaded using |\usepackage| or |\RequirePackage|
% which has a mechanism to prevent loading a style file more than once.
% When loading the definitions by means of |\input|
% multiple instances have to be prevented manually:
%\iffalse
%This code needs to be before the `\ProvidesFile' directive
%which is defined at the beginning of this file.
%Therefore it is also placed there and commented out here.
%</package>
%<*discard>
%\fi
%    \begin{macrocode}
\ifdefined\childdocmain\endinput\fi
%    \end{macrocode}
%\iffalse
%</discard>
%<*package>
%\fi
%
% \macro{\ifchilddoc}
% \macro{\ifchilddocmanual}
% The conditional |\ifchilddoc| tells whether a
% child (true) or main (false) document is being compiled.
% The conditional |\ifchilddocmanual| tells whether
% the |\includeonly| mechanism is used (false) or
% the selection of child files must be performed manually (true).
% The definitions initialise to false:
%    \begin{macrocode}
\newif\ifchilddoc
\newif\ifchilddocmanual
%    \end{macrocode}

% \macro{\childdocname}
% \macro{\childdocjob}
% The macro |\childdocname| stores the name of the main document
% to be compiled. The macro |\childdocjob| stores the name of
% the document on which the \LaTeX{} compiler was originally invoked.
% The content of |\jobname| cannot be compared
% to filenames specified in the source due to different catcodes.
% The following code rescans |\jobname|, stores the result
% in |\childdocname| and saves a copy in |\childdocjob|:
%    \begin{macrocode}
\edef\childdocname{\scantokens\expandafter{\jobname\noexpand}}
\let\childdocjob\childdocname
%    \end{macrocode}

% \macro{\childdocdisable}
% The macro |\childdocdisable| prevents the main file
% from being processed more than once.
% At this stage, the main document command |\childdocmain|
% is assumed to be called once again where it should do nothing.
% Any subsequent call to it should prevent
% a secondary processing of the main document
% It overwrites the forwarding commands
% |\childdocof| and |\childdocforward|
% with empty macros to prevent further inclusions of the main document:
%    \begin{macrocode}
\newcommand{\childdocdisable}
{
  \renewcommand{\childdocmain}[1]{\renewcommand{\childdocmain}[1]{\endinput}}
  \renewcommand{\childdocof}[1]{}
  \renewcommand{\childdocby}[2][]{}
  \renewcommand{\childdocforward}[2][]{}
  \renewcommand{\childdocdisable}{}
}
%    \end{macrocode}

% \macro{\childdocmain}
% The macro |\childdocmain| is to be called at the top of the main file
% with nothing or the main filename (without extension) as argument.
% First, it breaks loops.
% If the argument is not empty and does not match |\childdocname|
% (which is set by the first inclusion of |childdoc.def|),
% |\ifchilddoc| is set to true, |\includeonly| is applied to the child file
% and |\jobname| is set to the main file
% (for proper handling of |.aux| files):
%    \begin{macrocode}
\newcommand{\childdocmain}[1]
{
  \childdocdisable\childdocmain{}
  \if?#1?\else
    \begingroup
      \def\childdoctmp{#1}
      \ifx\childdoctmp\childdocname
        \def\childdoctmp{}
      \else
        \def\childdoctmp
        {
          \childdoctrue
          \includeonly{\childdocname}
          \def\childdocjob{#1}
          \def\jobname{#1}
        }
      \fi
      \expandafter
    \endgroup
    \childdoctmp
  \fi
}
%    \end{macrocode}

% \macro{\childdocof}
% The command |\childdocof| redirects
% compilation to the main file |#1|.
%    \begin{macrocode}
\newcommand{\childdocof}[1]
{
  \childdocdisable
  \childdoctrue
  \includeonly{\childdocname}
  \def\jobname{#1}
  \def\childdocjob{#1}
  \input{#1}
}
%    \end{macrocode}

% \macro{\childdocby}
% The command |\childdocby| ....
%    \begin{macrocode}
\newcommand{\childdocby}[2][]
{
  \childdocdisable
  \childdoctrue
  \childdocmanualtrue
  \if?#1?\else
    \def\jobname{#2}
  \fi
  \def\childdocjob{#2}
  \input{#2}
  \endinput
}
%    \end{macrocode}

% \macro{\childdocforward}
% The command |\childdocforward| redirects
% compilation to the main file or
% (if the optional argument is given) a child file.
% Parameters are set as if the main file
% or a child file starting with |\childdocof| was compiled.
% Then compilation is handed over to the main file:
%    \begin{macrocode}
\newcommand{\childdocforward}[2][]
{
  \begingroup
    \if?#1?
      \def\childdoctmp
      {
        \def\childdocname{#2}
        \def\childdocjob{#2}
        \def\jobname{#2}
        \input{#2}
        \endinput
      }
    \else
      \def\childdoctmp
      {
        \childdocdisable
        \def\childdocname{#2}
        \childdoctrue
        \includeonly{#2}
        \def\childdocjob{#1}
        \def\jobname{#1}
        \input{#1}
        \endinput
      }
    \fi
    \expandafter
  \endgroup
  \childdoctmp
}
%    \end{macrocode}

% \macro{\childdocforwardprefix}
% The command |\childdocforwardprefix| redirects
% compilation to the main or a child file by means of a pattern.
% The prefix |#1| in the current filename is replaced by |#2|
% and the suffix of the current filename is kept
% (it is assumed that the filename does not contain the substring `|~~~|'
% which is used as a delimiter).
% Compilation is handed over to the new file by |\childdocforward|:
%    \begin{macrocode}
\newcommand{\childdocforwardprefix}[3][]
{
  \begingroup
    \def\childdocextract #2##1~~~{\def\childdoctmp{\childdocforward[#1]{#3##1}}}
    \expandafter\childdocextract\childdocname~~~
    \expandafter
  \endgroup
  \childdoctmp
}
%    \end{macrocode}

% \macro{\childdoc}
% The deprecated macro |\childdoc| is a legacy version of |\childdocmain|:
%    \begin{macrocode}
\newcommand{\childdoc}{\childdocmain}
%    \end{macrocode}

% \macro{\childdocredirect}
% The deprecated macro |\childdocredirect| is a legacy version
% of |\childdocforward| and |\childdocforwardprefix|:
%    \begin{macrocode}
\newcommand{\childdocredirect}[2][]
{
  \begingroup
    \if?#1?
      \def\childdoctmp{\childdocforward{#2}}
    \else
      \def\childdoctmp{\childdocforwardprefix{#1}{#2}}
    \fi
    \expandafter
  \endgroup
  \childdoctmp
}
%    \end{macrocode}

%\iffalse
%</package>
%\fi
%
\endinput
|\\
|\childdocby{|\textit{main}|}|\\
\end{tabular}
\end{center}
%
The directive |\childdocby| is similar to |\childdocof|
described in \secref{sec:include},
but the subsequent selection of content must be done manually.
To that end, both |\ifchilddoc| and |\ifchilddocmanual|
will be true upon processing of a part,
and the name of the part is stored in |\childdocname|.
Note that |\jobname| will be set to the filename of the current part
so that each part receives an individual |.aux| file
that does not interfere with the |.aux| file(s) of the main document.
This behaviour can be altered by the alternative form
|\childdocby[*]{|\textit{main}|}| (with a non-empty optional argument)
which uses the |.aux| file of the main document
by setting |\jobname| to \textit{main}.

%%%%%%%%%%%%%%%%%%%%%%%%%%%%%%%%%%%%%%%%%%%%%%%%%%%%%%%%%%%%%%%%%%%%%%%%%%%%%%%%
\subsection{Driver Development}
\label{sec:driver}

The \textsf{childdoc} mechanism can also be use for the development
of definition files such as \LaTeX{} styles or classes.
This case differs from the above setup with multiple parts
included by |\include| in that no |\includeonly| should be invoked.
This can be achieved by starting the include file
(before |\ProvidesPackage|) with:
%
\begin{center}
\begin{tabular}{l}
|% \iffalse
%
% childdoc.dtx Copyright (C) 2017-2018 Niklas Beisert
%
% This work may be distributed and/or modified under the
% conditions of the LaTeX Project Public License, either version 1.3
% of this license or (at your option) any later version.
% The latest version of this license is in
%   http://www.latex-project.org/lppl.txt
% and version 1.3 or later is part of all distributions of LaTeX
% version 2005/12/01 or later.
%
% This work has the LPPL maintenance status `maintained'.
%
% The Current Maintainer of this work is Niklas Beisert.
%
% This work consists of the files childdoc.dtx and childdoc.ins
% and the derived files childdoc.def and cdocsamp.tex with
% cdocsch1.tex, cdocsch2.tex, cdocsdrf.tex, cdocsfn1.tex, cdocsfn2.tex.
%
%<package>\ifdefined\childdocmain\endinput\fi
%<package>\ProvidesFile{childdoc.def}[2018/12/30 v2.0 child document driver]
%<samplemain>\ProvidesFile{cdocsamp.tex}[2018/12/30 v2.0 sample for childdoc]
%<*driver>
%\ProvidesFile{childdoc.drv}[2018/12/30 v2.0 childdoc reference manual file]
\PassOptionsToClass{10pt,a4paper}{article}
\documentclass{ltxdoc}

\usepackage[margin=35mm]{geometry}
\usepackage{hyperref}
\usepackage{hyperxmp}
\usepackage[usenames]{color}

\hypersetup{colorlinks=true}
\hypersetup{pdfstartview=FitH}
\hypersetup{pdfpagemode=UseNone}
\hypersetup{pdfsource={}}
\hypersetup{pdflang={en-UK}}
\hypersetup{pdfcopyright={Copyright 2017-2018 Niklas Beisert.
  This work may be distributed and/or modified under the
  conditions of the LaTeX Project Public License, either version 1.3
  of this license or (at your option) any later version.}}
\hypersetup{pdflicenseurl={http://www.latex-project.org/lppl.txt}}
\hypersetup{pdfcontactaddress={ETH Zurich, ITP, HIT K,
  Wolfgang-Pauli-Strasse 27}}
\hypersetup{pdfcontactpostcode={8093}}
\hypersetup{pdfcontactcity={Zurich}}
\hypersetup{pdfcontactcountry={Switzerland}}
\hypersetup{pdfcontactemail={nbeisert@itp.phys.ethz.ch}}
\hypersetup{pdfcontacturl={http://people.phys.ethz.ch/\xmptilde nbeisert/}}

\newcommand{\secref}[1]{\hyperref[#1]{section \ref*{#1}}}

\parskip1ex
\parindent0pt
\let\olditemize\itemize
\def\itemize{\olditemize\parskip0pt}

\begin{document}

\title{The \textsf{childdoc} Package}
\hypersetup{pdftitle={The childdoc Package}}
\author{Niklas Beisert\\[2ex]
  Institut f\"ur Theoretische Physik\\
  Eidgen\"ossische Technische Hochschule Z\"urich\\
  Wolfgang-Pauli-Strasse 27, 8093 Z\"urich, Switzerland\\[1ex]
  \href{mailto:nbeisert@itp.phys.ethz.ch}
  {\texttt{nbeisert@itp.phys.ethz.ch}}}
\hypersetup{pdfauthor={Niklas Beisert}}
\hypersetup{pdfsubject={Manual for the LaTeX2e Package childdoc}}
\date{30 December 2018, \textsf{v2.0}}
\maketitle

\begin{abstract}\noindent
\textsf{childdoc} is a \LaTeXe{} package
that enables the direct compilation
of document sections included by |\include|
to individual files.
\end{abstract}

\begingroup
\parskip0ex
\tableofcontents
\endgroup

%%%%%%%%%%%%%%%%%%%%%%%%%%%%%%%%%%%%%%%%%%%%%%%%%%%%%%%%%%%%%%%%%%%%%%%%%%%%%%%%
%%%%%%%%%%%%%%%%%%%%%%%%%%%%%%%%%%%%%%%%%%%%%%%%%%%%%%%%%%%%%%%%%%%%%%%%%%%%%%%%
\section{Introduction}

\LaTeX{} provides a mechanism to structure a large document (such as a book)
into a main file and several child files (containing the chapters)
using the |\include| command.
This mechanism is beneficial for documents
which span hundreds of pages in order to
make the source file(s) more manageable.
Moreover, compilation can be restricted to
selected child files by means of the |\includeonly| command.
The latter feature can be used to reduce the compilation time while editing
(this was significantly more useful in the earlier days of \LaTeX{})
or to generate a smaller document which is easier to navigate.
Another application of |\includeonly| is to generate
documents consisting of selected parts of the complete document.

However, there are a few drawbacks of the plain |\include| mechanism:
\begin{itemize}
\item
The child files cannot be compiled on their own,
they can only be compiled via the main file.
A naive editing environment
(such as a text editor with an option
to have the current file processed by \LaTeX)
may require one to switch to the main file before compiling;
attempting to compile the child file produces errors.
\item
The main file must be modified (each time)
to adjust the |\includeonly| command
to the present needs. This easily leaves the main file in a messy state.
\item
The generated document will always carry the filename
of the main document. This is inconvenient if
several child files are to be compiled and
to be kept for distribution.
\end{itemize}

The present package provides a simple interface
to make child files individually compilable by \LaTeX{}.
Compiling a child file then has the same effect as compiling
the main file with an |\includeonly| command
to select the appropriate child.
Moreover the generated document will carry the name of the child
rather than the main file.
This resolves all three above issues.

This feature is meant to make the editing of books,
thesis documents and lecture notes somewhat more convenient.
However, the package can also be used efficiently for
composing a series of documents (such as exercise sheets)
which are typically distributed individually.
It then assists the author in generating the individual documents
(potentially in different versions)
as well as a document containing the collected series.
Another application is in developing style files
or other kinds of included material
where compilation of the style file could redirect
to a sample or test file.

%%%%%%%%%%%%%%%%%%%%%%%%%%%%%%%%%%%%%%%%%%%%%%%%%%%%%%%%%%%%%%%%%%%%%%%%%%%%%%%%
%%%%%%%%%%%%%%%%%%%%%%%%%%%%%%%%%%%%%%%%%%%%%%%%%%%%%%%%%%%%%%%%%%%%%%%%%%%%%%%%
\section{Usage}

First of all, the package \textsf{childdoc} is \emph{not} a standard
\LaTeXe{} |.sty| style file! Therefore it needs to be invoked in
a non-standard way.

%%%%%%%%%%%%%%%%%%%%%%%%%%%%%%%%%%%%%%%%%%%%%%%%%%%%%%%%%%%%%%%%%%%%%%%%%%%%%%%%
\subsection{Included Files}
\label{sec:include}

%%%%%%%%%%%%%%%%%%%%%%%%%%%%%%%%%%%%%%%%
\DescribeMacro{\childdocmain}
To use the package, add the commands
\begin{center}
\begin{tabular}{l}
|\input{childdoc.def}|\\
|\childdocmain{}|\\
\end{tabular}
\end{center}
at the very top of the main \LaTeX{} file,
in particular \emph{before} the |\documentclass| statement!
The argument of |\childdocmain| should be left empty
(but it must be present).

%%%%%%%%%%%%%%%%%%%%%%%%%%%%%%%%%%%%%%%%
\DescribeMacro{\childdocof}
Furthermore, add the commands
\begin{center}
\begin{tabular}{l}
|\input{childdoc.def}|\\
|\childdocof{|\textit{main}|}|\\
\end{tabular}
\end{center}
at the top of every child file \textit{child}
which is included by |\include{|\textit{child}|}|
from within the main file
(or at least for those files to be compiled individually).
The argument \textit{main} must be the filename of the main file.

There are a couple of
considerations in setting up the main and child documents:

%%%%%%%%%%%%%%%%%%%%%%%%%%%%%%%%%%%%%%%%
\paragraph{Restrictions.}

Please note the following restrictions:
\begin{itemize}
\item
|\childdocmain| must be called with one argument \textit{main}
to ensure compatibility with earlier version of the package.
It must either be empty (|\childdocmain{}|)
or precisely match the filename of the main file in which it is specified.
See \secref{sec:detection} for further information.
\item
The filename \textit{main} must be specified without the |.tex| extension.
\item
The filename \textit{main} is case sensitive
(even in case-insensitive file systems)
due to internal string comparison.
\item
The argument \textit{main} should be fully expanded, it cannot be a macro.
\item
Subdirectories and special characters should be avoided in filenames.
\item
The command |\childdocmain{|\textit{main}|}| must be followed by a whitespace.
It should not be followed immediately by another command
or by a comment mark `|%|'.
This is because the \TeX{} parser reads the token immediately following
the argument of |\childdocmain| and puts it
at the beginning of every child section;
however, a white\-space is ignored.
\end{itemize}

%%%%%%%%%%%%%%%%%%%%%%%%%%%%%%%%%%%%%%%%
\paragraph{Content of Main File.}

It is advisable to place all content in the child files included by |\include|.
Any output contained in the main file will appear in all child documents
unless suppressed manually;
it cannot be suppressed automatically by the |\includeonly| directive
and thus should normally be avoided.
A method to include some content in the main file
by means of conditional processing is described in \secref{sec:conditional}.

%%%%%%%%%%%%%%%%%%%%%%%%%%%%%%%%%%%%%%%%
\paragraph{Page Numbering.}

When only a part of the document is compiled,
the appropriate numbering of pages
(as well as other status parameters)
is determined from the |.aux| files.
The latter contain information from previous passes.
However this information needs to propagate through
all intermediate child documents.
Therefore the page numbering in child documents may well
be inconsistent until the complete document is compiled at least once.

A useful (if unconventional) way to always ensure a consistent
page numbering is to restart the numbering in each child document
and denote the pages by `\textit{child}|.|\textit{page}'
where \textit{child} represents the chapter/section number of the child file.
This can be achieved by the command
|\numberwithin{page}{|\textit{child}|}|
of the \textsf{amsmath} package
where \textit{child} can be |chapter| or |section|
depending on the chosen structuring.
Alternatively, one can modify the macro |\thepage| appropriately
and reset the counter |page| at the start of each child file.

%%%%%%%%%%%%%%%%%%%%%%%%%%%%%%%%%%%%%%%%%%%%%%%%%%%%%%%%%%%%%%%%%%%%%%%%%%%%%%%%
\subsection{Conditional Processing}
\label{sec:conditional}

The package provides a mechanism to compile different versions
of a document. To customise the versions further some conditional processing
can come in handy to distinguish which version is being compiled.
The package provides two macros to describe the compilation context:

%%%%%%%%%%%%%%%%%%%%%%%%%%%%%%%%%%%%%%%%
\DescribeMacro{\ifchilddoc}
The conditional |\ifchilddoc| distinguishes between the compilation of
child documents and the main document:
%
\begin{center}
|\ifchilddoc |\textit{child-code}| |[|\||else |\textit{main-code}]| \||fi|
\end{center}

%%%%%%%%%%%%%%%%%%%%%%%%%%%%%%%%%%%%%%%%
\DescribeMacro{\childdocname}
\DescribeMacro{\childdocjob}
The macro |\childdocname| contains the filename (without extension)
of the main or child file being processed.
Note that |\childdocjob| will always contain the name of the main file.

%%%%%%%%%%%%%%%%%%%%%%%%%%%%%%%%%%%%%%%%
\paragraph{Title Page.}

Conditional processing can be used to include a title or banner page
in the main document when proper precautions are taken.
Importantly, the code in the main file should ensure that the page counter
(as well as other status parameters which are stored in the |.aux| files)
takes the same value after the conditional processing.
Otherwise the page numbers may take divergent values
depending on which part is compiled.

For example, a title page could be declared by:
%
\begin{center}
\begin{tabular}{l}
|\ifchilddoc\||else|\\
|\addtocounter{page}{-1}|\\
\textit{code for title page}\\
|\newpage|\\
|\||fi|
\end{tabular}
\end{center}
%
A banner page for the child documents can be generated by:
%
\begin{center}
\begin{tabular}{l}
|\ifchilddoc|\\
|\addtocounter{page}{-1}|\\
\textit{code for banner page}\\
|\newpage|\\
|\||fi|
\end{tabular}
\end{center}
%
Here one could write a message such as:
\begin{center}
|This is the part \childdocname{} of \childdocjob{}.|
\end{center}

%%%%%%%%%%%%%%%%%%%%%%%%%%%%%%%%%%%%%%%%%%%%%%%%%%%%%%%%%%%%%%%%%%%%%%%%%%%%%%%%
\subsection{Flags}
\label{sec:flags}

The package makes it easy to generate different versions
of the main or child documents.
To this end compilation flags can be defined
and assigned different default values.
They will be particularly useful in conjunction
with the forwarding mechanism described in \secref{sec:forward}.

For example, it may be useful to have a flag |\version|
which can be set to |draft| or |final|.
The document source will contain some conditional code
depending on the value of |\version|.
Suppose further, the flag should default to |final| for the main file
and to |draft| for child files
which is a natural assignment for editing the document.
This is achieved by placing the following code
in the preamble of the main document
(below the |\childdocmain| directive):
%
\begin{center}
\begin{tabular}{l}
|\ifchilddoc|\\
|\providecommand{\version}{draft}|\\
|\||else|\\
|\providecommand{\version}{final}|\\
|\||fi|
\end{tabular}
\end{center}
%
The definition by |\providecommand| makes sure
that previous definitions are not overwritten.
Further statements |\providecommand{\version}{...}|
can thus be added before the above code to override it.

For the main file, one might add a line
(between |\childdocmain| and the above block)
%
\begin{center}
|%\ifchilddoc\||else\providecommand{\version}{draft}\||fi|
\end{center}
%
which can be uncommented to produce a draft version.
Likewise one can add a line to the very top of a child file
(above the |\childdocof{|\textit{main}|}| directive)
%
\begin{center}
|%\providecommand{\version}{final}|
\end{center}
%
which can be uncommented to produce the final version of this child document.

%%%%%%%%%%%%%%%%%%%%%%%%%%%%%%%%%%%%%%%%%%%%%%%%%%%%%%%%%%%%%%%%%%%%%%%%%%%%%%%%
\subsection{Forwarding}
\label{sec:forward}

Different versions of the main or child documents
using compilation flags as described in \secref{sec:flags}
can be (permanently) stored in different files
for convenient compilation, viewing and distribution.
To this end, the package defines a command
to pass on compilation to a different file:

%%%%%%%%%%%%%%%%%%%%%%%%%%%%%%%%%%%%%%%%
\DescribeMacro{\childdocforward}
The command |\childdocforward| redirects processing to
another source file:
%
\begin{center}
\begin{tabular}{l}
|\input{childdoc.def}|\\
|\childdocforward[|\textit{main}|]{|\textit{dest}|}|\\
\end{tabular}
\end{center}
%
The argument \textit{dest} is the destination file
(without extension).
It should be the main file or one of the child files.
Note that further \textsf{childdoc} directives
such as |\childdocof| and |\childdocforward|
in the indicated file will be processed in this form.
The optional argument \textit{main}
passes on directly to the main file \textit{main}
while pretending to compile the child \textit{dest}.
This form behaves as if \textit{dest}
issues |\childdocof{|\textit{main}|}| right away,
and no further \textsf{childdoc} directives will be processed.

%%%%%%%%%%%%%%%%%%%%%%%%%%%%%%%%%%%%%%%%
\DescribeMacro{\...prefix}
In the alternative form |\childdocforwardprefix|,
%
\begin{center}
\begin{tabular}{l}
|\input{childdoc.def}|\\
|\childdocforwardprefix[|\textit{main}|]{|\textit{prefix}|}{|\textit{dest}|}|
\end{tabular}
\end{center}
%
the destination file is determined by a pattern
depending on the current file:
To make this work, the current file must be called
`{\textit{prefix}\hspace{0.2em}\textit{suffix}}'
with \textit{prefix} matching precisely the argument.
Processing is then passed on to the file
`{\textit{dest}\hspace{0.2em}\textit{suffix}}'.
Surely, the same effect is achieved by
directly specifying the
argument `{\textit{dest}\hspace{0.2em}\textit{suffix}}'
in the first form.
However, that requires to set up a different file
for each child. With the alternative form of the command
all these files can have exactly the same content
which simplifies setting them up and maintaining them.

For example, the following file |draft.tex|
with a compilation flag |\version| as described in \secref{sec:flags}
compiles the main document as a draft:
%
\begin{center}
\begin{tabular}{l}
|\def\version{draft}|\\
|\input{childdoc.def}|\\
|\childdocforward{|\textit{main}|}|
\end{tabular}
\end{center}
%
Likewise, the following files |final|\textit{nn}|.tex|
compile the final version of the child document
|child|\textit{nn}|.tex|:
%
\begin{center}
\begin{tabular}{l}
|\def\version{final}|\\
|\input{childdoc.def}|\\
|\childdocforwardprefix{final}{child}|
\end{tabular}
\end{center}
%

Note that when several versions of a main file and/or of each child file
are to be generated, it may be convenient to set up a |Makefile| or
shell script to automatise the process.

%%%%%%%%%%%%%%%%%%%%%%%%%%%%%%%%%%%%%%%%%%%%%%%%%%%%%%%%%%%%%%%%%%%%%%%%%%%%%%%%
\subsection{Command Line Processing}
\label{sec:commandline}

The effect of redirection files can also be achieved by invoking
the \LaTeX{} compiler with a more elaborate command line.
Most conveniently this should be done as part
of a shell script or a |Makefile|.

When using \textsf{childdoc} in the main file, the following
command lines effectively perform a redirection
(note that depending on the shell being used,
backslashes may have to be doubled: `|\|' $\to$ `|\\|'):
%
\begin{center}
|... -jobname "|\textit{target}|" |\\|"|[\textit{flags}]%
|\input{childdoc.def}\childdocforward[|\textit{main}|]{|\textit{dest}|}"|
\end{center}
%
Here \textit{target} is the name of the output file,
\textit{main} is the name of the main file
and \textit{dest} is the name of the main or child file to be processed
(all filenames without extensions).
The optional argument \textit{main} can be omitted
if \textit{main} matches \textit{dest}.
Optionally, compilation \textit{flags} can be defined via |\def| commands.
This command line makes the \TeX{} engine believe
it is compiling the file \textit{target}
whose content is specified as the latter parameter.
The provided code then forwards the processing to
\textit{main} or \textit{dest} as described in \secref{sec:forward}.

%%%%%%%%%%%%%%%%%%%%%%%%%%%%%%%%%%%%%%%%%%%%%%%%%%%%%%%%%%%%%%%%%%%%%%%%%%%%%%%%
\subsection{Include by Input}
\label{sec:input}

Including child documents by |\include| has some restrictions by design.
Most notably, the content of a child document always occupies
its own set of pages; pages cannot be shared between child documents.
Usually, this behaviour makes perfect sense
because each child document contain an essential part of the document.
However, in some situations it may be desirable to compose
a document from a collection of parts
without having mandatory page breaks between then.
For this case, the package
provides a mechanism to include parts
by |\input| which can also be processed individually.
However, by construction this mechanism
requires manual handling of the content to be output.

%%%%%%%%%%%%%%%%%%%%%%%%%%%%%%%%%%%%%%%%
\DescribeMacro{\ifchilddocmanual}
The main file should be prepared as usual, see \secref{sec:include}.
However, the document body must make a distinction
between processing of an individual part and of the main document, e.g.:
%
\begin{center}
\begin{tabular}{l}
|\ifchilddocmanual|\\
|\input{\childdocname}|\\
|\||else|\\
\textit{document body with }|\input{|\textit{part}|}|\\
|\||fi|
\end{tabular}
\end{center}
%
The conditional |\ifchilddocmanual| is true whenever
a part to be included by |\input| is being compiled,
and the name of the part is stored in |\childdocname|.

%%%%%%%%%%%%%%%%%%%%%%%%%%%%%%%%%%%%%%%%
\DescribeMacro{\childdocby}
Each part to be included by |\input| should start with:
%
\begin{center}
\begin{tabular}{l}
|\input{childdoc.def}|\\
|\childdocby{|\textit{main}|}|\\
\end{tabular}
\end{center}
%
The directive |\childdocby| is similar to |\childdocof|
described in \secref{sec:include},
but the subsequent selection of content must be done manually.
To that end, both |\ifchilddoc| and |\ifchilddocmanual|
will be true upon processing of a part,
and the name of the part is stored in |\childdocname|.
Note that |\jobname| will be set to the filename of the current part
so that each part receives an individual |.aux| file
that does not interfere with the |.aux| file(s) of the main document.
This behaviour can be altered by the alternative form
|\childdocby[*]{|\textit{main}|}| (with a non-empty optional argument)
which uses the |.aux| file of the main document
by setting |\jobname| to \textit{main}.

%%%%%%%%%%%%%%%%%%%%%%%%%%%%%%%%%%%%%%%%%%%%%%%%%%%%%%%%%%%%%%%%%%%%%%%%%%%%%%%%
\subsection{Driver Development}
\label{sec:driver}

The \textsf{childdoc} mechanism can also be use for the development
of definition files such as \LaTeX{} styles or classes.
This case differs from the above setup with multiple parts
included by |\include| in that no |\includeonly| should be invoked.
This can be achieved by starting the include file
(before |\ProvidesPackage|) with:
%
\begin{center}
\begin{tabular}{l}
|\input{childdoc.def}|\\
|\childdocforward{|\textit{main}|}|\\
\end{tabular}
\end{center}
%
or alternatively with:
%
\begin{center}
\begin{tabular}{l}
|\input{childdoc.def}|\\
|\childdocby{|\textit{main}|}|\\
\end{tabular}
\end{center}
%
Both forms have slightly different effects as described above.
The main file is prepared as usual, see \secref{sec:include}.

%%%%%%%%%%%%%%%%%%%%%%%%%%%%%%%%%%%%%%%%%%%%%%%%%%%%%%%%%%%%%%%%%%%%%%%%%%%%%%%%
\subsection{Legacy Detection}
\label{sec:detection}

The directive |\childdocmain| in the main file can detect
whether the complete document or merely a child is to be compiled
even without using the directive |\childdocof|.
This method is deprecated because it is less robust
and there is no compelling reason to use it;
it is merely provided for backward compatibility
and it may be removed in future versions.

If the detection mechanism is to be used,
it is mandatory to correctly specify
the filename of the main file as the argument of |\childdocmain|:
%
\begin{center}
\begin{tabular}{l}
|\input{childdoc.def}|\\
|\childdocmain{|\textit{main}|}|\\
\end{tabular}
\end{center}
%
If |\jobname| does not match the argument \textit{main} of |\childdocmain|,
it is assumed that |\jobname| points to the child file to be compiled.
When using |\childdocmain| with the main file specified as argument,
it suffices to start a child file
with just |\input{|\textit{main}|}|
without loading of the package and using |\childdocof|.
If instead all processing is done
with the appropriate \textsf{childdoc} directives,
the argument of \textit{main} of |\childdocmain| can be empty.

An alternative version of the command line processing described
in \secref{sec:commandline} using the detection mechanism reads:
%
\begin{center}
|... -jobname "|\textit{target}|" "|[\textit{flags}]%
[|\def\jobname{|\textit{dest}|}|]|\input{|\textit{main}|}"|
\end{center}

%%%%%%%%%%%%%%%%%%%%%%%%%%%%%%%%%%%%%%%%%%%%%%%%%%%%%%%%%%%%%%%%%%%%%%%%%%%%%%%%
\subsection{Manual Code}
\label{sec:manual}

In case one cannot be certain whether the definitions file |childdoc.def|
is installed on the target \TeX{} distribution
and one prefers not to ship it,
it is conceivable to paste a few relevant commands into the sources.

To that end, drop all statements |\input{childdoc.def}|
and perform the replacements as outlined below.
Instead of |\childdocmain{|\textit{main}|}| add the following code
to the top of the main file:
%
\begin{center}
\begin{tabular}{l}
|\||ifdefined\childdocname\endinput\||fi\newif\ifchilddoc|\\
|\edef\childdocname{\scantokens\expandafter{\jobname\noexpand}}|\\
|\def\childdocmain{|\textit{main}|}\||ifx\childdocmain\childdocname\||else|\\
|\childdoctrue\includeonly{\childdocname}\let\jobname\childdocmain\||fi|\\
\end{tabular}
\end{center}
%
Instead of |\childdocof{|\textit{main}|}| just include the main file
at the top of each child file:
%
\begin{center}
|\input{|\textit{main}|}|
\end{center}
%
A simple redirection |\childdocforward{|\textit{dest}|}| is achieved by:
%
\begin{center}
|\def\jobname{|\textit{dest}|}\input{\jobname}|
\end{center}
%
The redirection with prefix
|\childdocforwardprefix[|\textit{prefix}|]{|\textit{dest}|}|
is accomplished by:
%
\begin{center}
\begin{tabular}{l}
|{\edef\jobname{\scantokens\expandafter{\jobname\noexpand}}|\\
|\def\redirectjob |\textit{prefix}|#1~~~{\gdef\jobname{|\textit{dest}|#1}}|\\
|\expandafter\redirectjob\jobname~~~}\input{\jobname}|
\end{tabular}
\end{center}

In an alternative approach,
child documents can be compiled by a specific command line
without additional code or specific definitions:
%
\begin{center}
|... -jobname "|\textit{target}|" "|[\textit{flags}]%
|\includeonly{|\textit{dest}|}\input{|\textit{main}|}"|
\end{center}
%

%%%%%%%%%%%%%%%%%%%%%%%%%%%%%%%%%%%%%%%%%%%%%%%%%%%%%%%%%%%%%%%%%%%%%%%%%%%%%%%%
%%%%%%%%%%%%%%%%%%%%%%%%%%%%%%%%%%%%%%%%%%%%%%%%%%%%%%%%%%%%%%%%%%%%%%%%%%%%%%%%
\section{Information}

%%%%%%%%%%%%%%%%%%%%%%%%%%%%%%%%%%%%%%%%%%%%%%%%%%%%%%%%%%%%%%%%%%%%%%%%%%%%%%%%
\subsection{Copyright}

Copyright \copyright{} 2017--2018 Niklas Beisert

This work may be distributed and/or modified under the
conditions of the \LaTeX{} Project Public License, either version 1.3
of this license or (at your option) any later version.
The latest version of this license is in
  \url{http://www.latex-project.org/lppl.txt}
and version 1.3 or later is part of all distributions of \LaTeX{}
version 2005/12/01 or later.

This work has the LPPL maintenance status `maintained'.

The Current Maintainer of this work is Niklas Beisert.

This work consists of the files |README.txt|, |childdoc.ins| and |childdoc.dtx|
as well as the derived files |childdoc.def|, |cdocsamp.tex|
with |cdocsch1.tex|, |cdocsch2.tex|, |cdocspt3.tex|, |cdocspt4.tex|,
|cdocsdrf.tex|, |cdocsfn1.tex|, |cdocsfn2.tex|
as well as |childdoc.pdf|.

%%%%%%%%%%%%%%%%%%%%%%%%%%%%%%%%%%%%%%%%%%%%%%%%%%%%%%%%%%%%%%%%%%%%%%%%%%%%%%%%
\subsection{Files and Installation}

The package consists of the files:
%
\begin{center}
\begin{tabular}{ll}
    |README.txt|   & readme file \\
    |childdoc.ins| & installation file \\
    |childdoc.dtx| & source file \\
    |childdoc.def| & definition file \\
    |cdocsamp.tex| & sample main file \\
    |cdocsch1.tex| & sample include file \\
    |cdocsch2.tex| & sample include file \\
    |cdocspt3.tex| & sample part file \\
    |cdocspt4.tex| & sample part file \\
    |cdocsdrf.tex| & sample redirection file \\
    |cdocsfn1.tex| & sample redirection file \\
    |cdocsfn2.tex| & sample redirection file \\
    |childdoc.pdf| & manual
\end{tabular}
\end{center}
%
The distribution consists of the files
|README.txt|, |childdoc.ins| and |childdoc.dtx|.
%
\begin{itemize}
\item
Run (pdf)\LaTeX{} on |childdoc.dtx|
to compile the manual |childdoc.pdf| (this file).
\item
Run \LaTeX{} on |childdoc.ins| to create the definitions file |childdoc.def|
and the sample |cdocsamp.tex| with include files
|cdocsch1.tex|, |cdocsch2.tex|, |cdocspt3.tex|, |cdocspt4.tex|,
|cdocsdrf.tex|, |cdocsfn1.tex|, |cdocsfn2.tex|.
Then copy the file |childdoc.def| to an appropriate directory of your \LaTeX{}
distribution, e.g.\ \textit{texmf-root}|/tex/latex/childdoc|.
\end{itemize}

%%%%%%%%%%%%%%%%%%%%%%%%%%%%%%%%%%%%%%%%%%%%%%%%%%%%%%%%%%%%%%%%%%%%%%%%%%%%%%%%
\subsection{Related CTAN Packages}

There are several other packages which offer a similar functionality:
%
\begin{itemize}
\item
The packages
\href{http://ctan.org/pkg/docmute}{\textsf{docmute}},
\href{http://ctan.org/pkg/includex}{\textsf{includex}} and
\href{http://ctan.org/pkg/standalone}{\textsf{standalone}}
provide commands to include only the document body of
a child file thus allowing both files to be compiled individually.
\item
The packages \href{http://ctan.org/pkg/subdocs}{\textsf{subdocs}}
and \href{http://ctan.org/pkg/subfiles}{\textsf{subfiles}}
provide structures in which the main and child documents can be
encapsulated and allowing them to be compiled individually.
The inclusion mechanism is different from the conventional |\include|.
\item
The package \href{http://ctan.org/pkg/combine}{\textsf{combine}}
is an elaborate solution to combine several documents into one.
\end{itemize}
%
See also the CTAN topic \href{http://ctan.org/topic/subdocs}{\textsf{subdocs}}
for further related packages.
The present package differs from the above solutions in that
a document structure constructed with the conventional |\include| mechanism
just needs two extra commands at the top of every file
such that all constituent files can be compiled individually.

%%%%%%%%%%%%%%%%%%%%%%%%%%%%%%%%%%%%%%%%%%%%%%%%%%%%%%%%%%%%%%%%%%%%%%%%%%%%%%%%
%\subsection{Feature Suggestions}
%
%The following is a list of features which may be useful for future
%versions of this package:
%%
%\begin{itemize}
%\item
%\ldots
%\end{itemize}

%%%%%%%%%%%%%%%%%%%%%%%%%%%%%%%%%%%%%%%%%%%%%%%%%%%%%%%%%%%%%%%%%%%%%%%%%%%%%%%%
\subsection{Revision History}

%%%%%%%%%%%%%%%%%%%%%%%%%%%%%%%%%%%%%%%%
\paragraph{v2.0:} 2018/12/30

\begin{itemize}
\item
immediate forward processing
\item
added |\childdocby| mechanism
\item
manual restructured
\end{itemize}

%%%%%%%%%%%%%%%%%%%%%%%%%%%%%%%%%%%%%%%%
\paragraph{v1.6:} 2018/01/17

\begin{itemize}
\item
application for development of include files
\item
corrections to manual
\end{itemize}

%%%%%%%%%%%%%%%%%%%%%%%%%%%%%%%%%%%%%%%%
\paragraph{v1.5:} 2017/05/21

\begin{itemize}
\item
more complete structuring introduced
\item
|\childdocof| introduced
\item
|\childdoc| renamed to |\childdocmain|
\item
|\childredirect| renamed to |\childdocforward| and |\childdocforwardprefix|
and functionality expanded
\end{itemize}

%%%%%%%%%%%%%%%%%%%%%%%%%%%%%%%%%%%%%%%%
\paragraph{v1.0:} 2017/04/27

\begin{itemize}
\item
manual and install package
\item
first version published on CTAN
\end{itemize}

%%%%%%%%%%%%%%%%%%%%%%%%%%%%%%%%%%%%%%%%
\paragraph{v0.6:} 2017/04/26

\begin{itemize}
\item
redirection mechanism added
\end{itemize}

%%%%%%%%%%%%%%%%%%%%%%%%%%%%%%%%%%%%%%%%
\paragraph{v0.5:} 2017/04/26

\begin{itemize}
\item
functionality in definition file
\end{itemize}


%%%%%%%%%%%%%%%%%%%%%%%%%%%%%%%%%%%%%%%%%%%%%%%%%%%%%%%%%%%%%%%%%%%%%%%%%%%%%%%%
%%%%%%%%%%%%%%%%%%%%%%%%%%%%%%%%%%%%%%%%%%%%%%%%%%%%%%%%%%%%%%%%%%%%%%%%%%%%%%%%
%%%%%%%%%%%%%%%%%%%%%%%%%%%%%%%%%%%%%%%%%%%%%%%%%%%%%%%%%%%%%%%%%%%%%%%%%%%%%%%%
\appendix

\settowidth\MacroIndent{\rmfamily\scriptsize 000\ }

 \DocInput{childdoc.dtx}

\end{document}
%</driver>
% \fi
%
% %%%%%%%%%%%%%%%%%%%%%%%%%%%%%%%%%%%%%%%%%%%%%%%%%%%%%%%%%%%%%%%%%%%%%%%%%%%%%%
% %%%%%%%%%%%%%%%%%%%%%%%%%%%%%%%%%%%%%%%%%%%%%%%%%%%%%%%%%%%%%%%%%%%%%%%%%%%%%%
% \section{Sample}
%\iffalse
%<*samplemain>
%\fi
%
% The following presents a sample document
% with two chapters, two parts, a title page,
% a compile flag as well as three forwarding files to set the flag.
% It consists of eight |.tex| files:
% \begin{center}
% \begin{tabular}{ll}
% |cdocsamp.tex|&main file\\
% |cdocsch1.tex|&include file for chapter 1\\
% |cdocsch2.tex|&include file for chapter 2\\
% |cdocspt3.tex|&include file for part 3\\
% |cdocspt4.tex|&include file for part 4\\
% |cdocsdrf.tex|&forwarding file for main file in draft mode\\
% |cdocsfi1.tex|&forwarding file for final version of chapter 1\\
% |cdocsfi2.tex|&forwarding file for final version of chapter 2\\
% \end{tabular}
% \end{center}
% Each of the eight files can be compiled directly by the \LaTeX{} compiler.
%
% %%%%%%%%%%%%%%%%%%%%%%%%%%%%%%%%%%%%%%
% \paragraph{Main File.}
%
% The main file is called |cdocsamp.tex|.
%
% Load the \textsf{childdoc} definitions and
% declare the filename for the main document:
%    \begin{macrocode}
\input{childdoc.def}
\childdocmain{}
%    \end{macrocode}

% Optional override for |\version| flag:
%    \begin{macrocode}
%%\ifchilddoc\else\providecommand{\version}{draft}\fi
%    \end{macrocode}

% Define the default values for the |\version| flag
% (|final| for the main file and |draft| for childs):
%    \begin{macrocode}
\ifchilddoc
\providecommand{\version}{draft}
\else
\providecommand{\version}{final}
\fi
%    \end{macrocode}

% Load the standard document class:
%    \begin{macrocode}
\documentclass[12pt]{article}
%    \end{macrocode}

% Start the document body:
%    \begin{macrocode}
\begin{document}
%    \end{macrocode}

% Declare a title page.
% Print title, part of document being processed and version flag:
%    \begin{macrocode}
\addtocounter{page}{-1}
\begin{center}
{\LARGE\bfseries{}childdoc example\par}
\vspace{1cm}
\ifchilddoc
\ifchilddocmanual part\else chapter\fi:
`\childdocname' of `\childdocjob'\par
\else
main document: `\childdocjob'\par
\fi
version: \version\par
\end{center}
\newpage
%    \end{macrocode}

% Manually include selected file,
% otherwise process as usual:
%    \begin{macrocode}
\ifchilddocmanual
\section*{part `\childdocname'}
\input{\childdocname}
\else
%    \end{macrocode}

% Include the two chapters:
%    \begin{macrocode}
\include{cdocsch1}
\include{cdocsch2}
%    \end{macrocode}

% Include the two parts unless only chapters should be displayed:
%    \begin{macrocode}
\ifchilddoc\else
\section{part three}
\input{cdocspt3}
\section{part four}
\input{cdocspt4}
\fi
%    \end{macrocode}

% Process as usual until here:
%    \begin{macrocode}
\fi
%    \end{macrocode}

% End of document body:
%    \begin{macrocode}
\end{document}
%    \end{macrocode}
%\iffalse
%</samplemain>
%\fi
%
% %%%%%%%%%%%%%%%%%%%%%%%%%%%%%%%%%%%%%%
% \paragraph{Chapter Include Files.}
%
% The include files are called |cdocsch1.tex| and |cdocsch2.tex|.
%
%\iffalse
%<*samplechap1|samplechap2>
%\fi

% Optional override for |\version| flag:
%    \begin{macrocode}
%%\providecommand{\version}{final}
%    \end{macrocode}

% Include the main document:
%    \begin{macrocode}
\input{childdoc.def}
\childdocof{cdocsamp}
%    \end{macrocode}

%\iffalse
%</samplechap1|samplechap2>
%\fi
%
%\iffalse
%<*samplechap1>
%\fi
% Some text for chapter 1:
%    \begin{macrocode}
\section{one}
some text in chapter one
%    \end{macrocode}

%\iffalse
%</samplechap1>
%\fi
% Some text for chapter 2:
%\iffalse
%<*samplechap2>
%\fi
%    \begin{macrocode}
\section{two}
more text in chapter two
%    \end{macrocode}

%\iffalse
%</samplechap2>
%\fi
%
% %%%%%%%%%%%%%%%%%%%%%%%%%%%%%%%%%%%%%%
% \paragraph{Part Include Files.}
%
% The include files are called |cdocspt3.tex| and |cdocspt4.tex|.
%
%\iffalse
%<*samplepart3|samplepart4>
%\fi

% Optional override for |\version| flag:
%    \begin{macrocode}
%%\providecommand{\version}{final}
%    \end{macrocode}

% Include the main document:
%    \begin{macrocode}
\input{childdoc.def}
\childdocby{cdocsamp}
%    \end{macrocode}

%\iffalse
%</samplepart3|samplepart4>
%\fi
%
%\iffalse
%<*samplepart3>
%\fi
% Some text for part 3:
%    \begin{macrocode}
some text in part three
%    \end{macrocode}

%\iffalse
%</samplepart3>
%\fi
% Some text for part 4:
%\iffalse
%<*samplepart4>
%\fi
%    \begin{macrocode}
more text in part four
%    \end{macrocode}

%\iffalse
%</samplepart4>
%\fi
%
% %%%%%%%%%%%%%%%%%%%%%%%%%%%%%%%%%%%%%%
% \paragraph{Forwarding for a Complete Draft.}
%
% The following forwarding file |cdocsdrf.tex|
% compiles the main document in draft mode:
%\iffalse
%<*sampledraft>
%\fi
%    \begin{macrocode}
\def\version{draft}
\input{childdoc.def}
\childdocforward{cdocsamp}
%    \end{macrocode}

%\iffalse
%</sampledraft>
%\fi
%
% %%%%%%%%%%%%%%%%%%%%%%%%%%%%%%%%%%%%%%
% \paragraph{Forwarding for Final Version of the Chapters.}
%
% The following forwarding files |cdocsfn1.tex| and |cdocsfn2.tex|
% (with identical content)
% compile the final versions of the child documents
% |cdocsch1.tex| and |cdocsch2.tex|, respectively:
%\iffalse
%<*samplefinal>
%\fi
%    \begin{macrocode}
\def\version{final}
\input{childdoc.def}
\childdocforwardprefix[cdocsamp]{cdocsfn}{cdocsch}
%    \end{macrocode}

%\iffalse
%</samplefinal>
%\fi
%
% %%%%%%%%%%%%%%%%%%%%%%%%%%%%%%%%%%%%%%
% \paragraph{Command Line Processing.}
%
% The following three command lines generate the output files
% |cdocscld|, |cdocscl1| and |cdocscl2|
% which should be identical to
% |cdocsdrf|, |cdocsch1| and |cdocsfn2|, respectively:
% \begin{center}
% \begin{tabular}{l}
% |latex -jobname cdocscld \|\\
% |  "\def\version{draft}\input{childdoc.def}\childdocforward{cdocsamp}"|\\
% |latex -jobname cdocscl1 \|\\
% |  "\input{childdoc.def}\childdocforward[cdocsamp]{cdocsch1}"|\\
% |latex -jobname cdocscl2 \|\\
% |  "\def\version{final}\input{childdoc.def}\childdocforward{cdocsch2}"|
% \end{tabular}
% \end{center}
% Note that the trailing backslash on each first line
% merely continues the input to the second line
% (for convenient cut ant paste).
% Furthermore, the command |latex| can be replaced by any
% of its alternative versions such as |pdflatex|.
%
% %%%%%%%%%%%%%%%%%%%%%%%%%%%%%%%%%%%%%%%%%%%%%%%%%%%%%%%%%%%%%%%%%%%%%%%%%%%%%%
% %%%%%%%%%%%%%%%%%%%%%%%%%%%%%%%%%%%%%%%%%%%%%%%%%%%%%%%%%%%%%%%%%%%%%%%%%%%%%%
% \section{Implementation}
%\iffalse
%<*package>
%\fi
%
% This section describes the definitions file |childdoc.def|.

% The definitions cannot be loaded using |\usepackage| or |\RequirePackage|
% which has a mechanism to prevent loading a style file more than once.
% When loading the definitions by means of |\input|
% multiple instances have to be prevented manually:
%\iffalse
%This code needs to be before the `\ProvidesFile' directive
%which is defined at the beginning of this file.
%Therefore it is also placed there and commented out here.
%</package>
%<*discard>
%\fi
%    \begin{macrocode}
\ifdefined\childdocmain\endinput\fi
%    \end{macrocode}
%\iffalse
%</discard>
%<*package>
%\fi
%
% \macro{\ifchilddoc}
% \macro{\ifchilddocmanual}
% The conditional |\ifchilddoc| tells whether a
% child (true) or main (false) document is being compiled.
% The conditional |\ifchilddocmanual| tells whether
% the |\includeonly| mechanism is used (false) or
% the selection of child files must be performed manually (true).
% The definitions initialise to false:
%    \begin{macrocode}
\newif\ifchilddoc
\newif\ifchilddocmanual
%    \end{macrocode}

% \macro{\childdocname}
% \macro{\childdocjob}
% The macro |\childdocname| stores the name of the main document
% to be compiled. The macro |\childdocjob| stores the name of
% the document on which the \LaTeX{} compiler was originally invoked.
% The content of |\jobname| cannot be compared
% to filenames specified in the source due to different catcodes.
% The following code rescans |\jobname|, stores the result
% in |\childdocname| and saves a copy in |\childdocjob|:
%    \begin{macrocode}
\edef\childdocname{\scantokens\expandafter{\jobname\noexpand}}
\let\childdocjob\childdocname
%    \end{macrocode}

% \macro{\childdocdisable}
% The macro |\childdocdisable| prevents the main file
% from being processed more than once.
% At this stage, the main document command |\childdocmain|
% is assumed to be called once again where it should do nothing.
% Any subsequent call to it should prevent
% a secondary processing of the main document
% It overwrites the forwarding commands
% |\childdocof| and |\childdocforward|
% with empty macros to prevent further inclusions of the main document:
%    \begin{macrocode}
\newcommand{\childdocdisable}
{
  \renewcommand{\childdocmain}[1]{\renewcommand{\childdocmain}[1]{\endinput}}
  \renewcommand{\childdocof}[1]{}
  \renewcommand{\childdocby}[2][]{}
  \renewcommand{\childdocforward}[2][]{}
  \renewcommand{\childdocdisable}{}
}
%    \end{macrocode}

% \macro{\childdocmain}
% The macro |\childdocmain| is to be called at the top of the main file
% with nothing or the main filename (without extension) as argument.
% First, it breaks loops.
% If the argument is not empty and does not match |\childdocname|
% (which is set by the first inclusion of |childdoc.def|),
% |\ifchilddoc| is set to true, |\includeonly| is applied to the child file
% and |\jobname| is set to the main file
% (for proper handling of |.aux| files):
%    \begin{macrocode}
\newcommand{\childdocmain}[1]
{
  \childdocdisable\childdocmain{}
  \if?#1?\else
    \begingroup
      \def\childdoctmp{#1}
      \ifx\childdoctmp\childdocname
        \def\childdoctmp{}
      \else
        \def\childdoctmp
        {
          \childdoctrue
          \includeonly{\childdocname}
          \def\childdocjob{#1}
          \def\jobname{#1}
        }
      \fi
      \expandafter
    \endgroup
    \childdoctmp
  \fi
}
%    \end{macrocode}

% \macro{\childdocof}
% The command |\childdocof| redirects
% compilation to the main file |#1|.
%    \begin{macrocode}
\newcommand{\childdocof}[1]
{
  \childdocdisable
  \childdoctrue
  \includeonly{\childdocname}
  \def\jobname{#1}
  \def\childdocjob{#1}
  \input{#1}
}
%    \end{macrocode}

% \macro{\childdocby}
% The command |\childdocby| ....
%    \begin{macrocode}
\newcommand{\childdocby}[2][]
{
  \childdocdisable
  \childdoctrue
  \childdocmanualtrue
  \if?#1?\else
    \def\jobname{#2}
  \fi
  \def\childdocjob{#2}
  \input{#2}
  \endinput
}
%    \end{macrocode}

% \macro{\childdocforward}
% The command |\childdocforward| redirects
% compilation to the main file or
% (if the optional argument is given) a child file.
% Parameters are set as if the main file
% or a child file starting with |\childdocof| was compiled.
% Then compilation is handed over to the main file:
%    \begin{macrocode}
\newcommand{\childdocforward}[2][]
{
  \begingroup
    \if?#1?
      \def\childdoctmp
      {
        \def\childdocname{#2}
        \def\childdocjob{#2}
        \def\jobname{#2}
        \input{#2}
        \endinput
      }
    \else
      \def\childdoctmp
      {
        \childdocdisable
        \def\childdocname{#2}
        \childdoctrue
        \includeonly{#2}
        \def\childdocjob{#1}
        \def\jobname{#1}
        \input{#1}
        \endinput
      }
    \fi
    \expandafter
  \endgroup
  \childdoctmp
}
%    \end{macrocode}

% \macro{\childdocforwardprefix}
% The command |\childdocforwardprefix| redirects
% compilation to the main or a child file by means of a pattern.
% The prefix |#1| in the current filename is replaced by |#2|
% and the suffix of the current filename is kept
% (it is assumed that the filename does not contain the substring `|~~~|'
% which is used as a delimiter).
% Compilation is handed over to the new file by |\childdocforward|:
%    \begin{macrocode}
\newcommand{\childdocforwardprefix}[3][]
{
  \begingroup
    \def\childdocextract #2##1~~~{\def\childdoctmp{\childdocforward[#1]{#3##1}}}
    \expandafter\childdocextract\childdocname~~~
    \expandafter
  \endgroup
  \childdoctmp
}
%    \end{macrocode}

% \macro{\childdoc}
% The deprecated macro |\childdoc| is a legacy version of |\childdocmain|:
%    \begin{macrocode}
\newcommand{\childdoc}{\childdocmain}
%    \end{macrocode}

% \macro{\childdocredirect}
% The deprecated macro |\childdocredirect| is a legacy version
% of |\childdocforward| and |\childdocforwardprefix|:
%    \begin{macrocode}
\newcommand{\childdocredirect}[2][]
{
  \begingroup
    \if?#1?
      \def\childdoctmp{\childdocforward{#2}}
    \else
      \def\childdoctmp{\childdocforwardprefix{#1}{#2}}
    \fi
    \expandafter
  \endgroup
  \childdoctmp
}
%    \end{macrocode}

%\iffalse
%</package>
%\fi
%
\endinput
|\\
|\childdocforward{|\textit{main}|}|\\
\end{tabular}
\end{center}
%
or alternatively with:
%
\begin{center}
\begin{tabular}{l}
|% \iffalse
%
% childdoc.dtx Copyright (C) 2017-2018 Niklas Beisert
%
% This work may be distributed and/or modified under the
% conditions of the LaTeX Project Public License, either version 1.3
% of this license or (at your option) any later version.
% The latest version of this license is in
%   http://www.latex-project.org/lppl.txt
% and version 1.3 or later is part of all distributions of LaTeX
% version 2005/12/01 or later.
%
% This work has the LPPL maintenance status `maintained'.
%
% The Current Maintainer of this work is Niklas Beisert.
%
% This work consists of the files childdoc.dtx and childdoc.ins
% and the derived files childdoc.def and cdocsamp.tex with
% cdocsch1.tex, cdocsch2.tex, cdocsdrf.tex, cdocsfn1.tex, cdocsfn2.tex.
%
%<package>\ifdefined\childdocmain\endinput\fi
%<package>\ProvidesFile{childdoc.def}[2018/12/30 v2.0 child document driver]
%<samplemain>\ProvidesFile{cdocsamp.tex}[2018/12/30 v2.0 sample for childdoc]
%<*driver>
%\ProvidesFile{childdoc.drv}[2018/12/30 v2.0 childdoc reference manual file]
\PassOptionsToClass{10pt,a4paper}{article}
\documentclass{ltxdoc}

\usepackage[margin=35mm]{geometry}
\usepackage{hyperref}
\usepackage{hyperxmp}
\usepackage[usenames]{color}

\hypersetup{colorlinks=true}
\hypersetup{pdfstartview=FitH}
\hypersetup{pdfpagemode=UseNone}
\hypersetup{pdfsource={}}
\hypersetup{pdflang={en-UK}}
\hypersetup{pdfcopyright={Copyright 2017-2018 Niklas Beisert.
  This work may be distributed and/or modified under the
  conditions of the LaTeX Project Public License, either version 1.3
  of this license or (at your option) any later version.}}
\hypersetup{pdflicenseurl={http://www.latex-project.org/lppl.txt}}
\hypersetup{pdfcontactaddress={ETH Zurich, ITP, HIT K,
  Wolfgang-Pauli-Strasse 27}}
\hypersetup{pdfcontactpostcode={8093}}
\hypersetup{pdfcontactcity={Zurich}}
\hypersetup{pdfcontactcountry={Switzerland}}
\hypersetup{pdfcontactemail={nbeisert@itp.phys.ethz.ch}}
\hypersetup{pdfcontacturl={http://people.phys.ethz.ch/\xmptilde nbeisert/}}

\newcommand{\secref}[1]{\hyperref[#1]{section \ref*{#1}}}

\parskip1ex
\parindent0pt
\let\olditemize\itemize
\def\itemize{\olditemize\parskip0pt}

\begin{document}

\title{The \textsf{childdoc} Package}
\hypersetup{pdftitle={The childdoc Package}}
\author{Niklas Beisert\\[2ex]
  Institut f\"ur Theoretische Physik\\
  Eidgen\"ossische Technische Hochschule Z\"urich\\
  Wolfgang-Pauli-Strasse 27, 8093 Z\"urich, Switzerland\\[1ex]
  \href{mailto:nbeisert@itp.phys.ethz.ch}
  {\texttt{nbeisert@itp.phys.ethz.ch}}}
\hypersetup{pdfauthor={Niklas Beisert}}
\hypersetup{pdfsubject={Manual for the LaTeX2e Package childdoc}}
\date{30 December 2018, \textsf{v2.0}}
\maketitle

\begin{abstract}\noindent
\textsf{childdoc} is a \LaTeXe{} package
that enables the direct compilation
of document sections included by |\include|
to individual files.
\end{abstract}

\begingroup
\parskip0ex
\tableofcontents
\endgroup

%%%%%%%%%%%%%%%%%%%%%%%%%%%%%%%%%%%%%%%%%%%%%%%%%%%%%%%%%%%%%%%%%%%%%%%%%%%%%%%%
%%%%%%%%%%%%%%%%%%%%%%%%%%%%%%%%%%%%%%%%%%%%%%%%%%%%%%%%%%%%%%%%%%%%%%%%%%%%%%%%
\section{Introduction}

\LaTeX{} provides a mechanism to structure a large document (such as a book)
into a main file and several child files (containing the chapters)
using the |\include| command.
This mechanism is beneficial for documents
which span hundreds of pages in order to
make the source file(s) more manageable.
Moreover, compilation can be restricted to
selected child files by means of the |\includeonly| command.
The latter feature can be used to reduce the compilation time while editing
(this was significantly more useful in the earlier days of \LaTeX{})
or to generate a smaller document which is easier to navigate.
Another application of |\includeonly| is to generate
documents consisting of selected parts of the complete document.

However, there are a few drawbacks of the plain |\include| mechanism:
\begin{itemize}
\item
The child files cannot be compiled on their own,
they can only be compiled via the main file.
A naive editing environment
(such as a text editor with an option
to have the current file processed by \LaTeX)
may require one to switch to the main file before compiling;
attempting to compile the child file produces errors.
\item
The main file must be modified (each time)
to adjust the |\includeonly| command
to the present needs. This easily leaves the main file in a messy state.
\item
The generated document will always carry the filename
of the main document. This is inconvenient if
several child files are to be compiled and
to be kept for distribution.
\end{itemize}

The present package provides a simple interface
to make child files individually compilable by \LaTeX{}.
Compiling a child file then has the same effect as compiling
the main file with an |\includeonly| command
to select the appropriate child.
Moreover the generated document will carry the name of the child
rather than the main file.
This resolves all three above issues.

This feature is meant to make the editing of books,
thesis documents and lecture notes somewhat more convenient.
However, the package can also be used efficiently for
composing a series of documents (such as exercise sheets)
which are typically distributed individually.
It then assists the author in generating the individual documents
(potentially in different versions)
as well as a document containing the collected series.
Another application is in developing style files
or other kinds of included material
where compilation of the style file could redirect
to a sample or test file.

%%%%%%%%%%%%%%%%%%%%%%%%%%%%%%%%%%%%%%%%%%%%%%%%%%%%%%%%%%%%%%%%%%%%%%%%%%%%%%%%
%%%%%%%%%%%%%%%%%%%%%%%%%%%%%%%%%%%%%%%%%%%%%%%%%%%%%%%%%%%%%%%%%%%%%%%%%%%%%%%%
\section{Usage}

First of all, the package \textsf{childdoc} is \emph{not} a standard
\LaTeXe{} |.sty| style file! Therefore it needs to be invoked in
a non-standard way.

%%%%%%%%%%%%%%%%%%%%%%%%%%%%%%%%%%%%%%%%%%%%%%%%%%%%%%%%%%%%%%%%%%%%%%%%%%%%%%%%
\subsection{Included Files}
\label{sec:include}

%%%%%%%%%%%%%%%%%%%%%%%%%%%%%%%%%%%%%%%%
\DescribeMacro{\childdocmain}
To use the package, add the commands
\begin{center}
\begin{tabular}{l}
|\input{childdoc.def}|\\
|\childdocmain{}|\\
\end{tabular}
\end{center}
at the very top of the main \LaTeX{} file,
in particular \emph{before} the |\documentclass| statement!
The argument of |\childdocmain| should be left empty
(but it must be present).

%%%%%%%%%%%%%%%%%%%%%%%%%%%%%%%%%%%%%%%%
\DescribeMacro{\childdocof}
Furthermore, add the commands
\begin{center}
\begin{tabular}{l}
|\input{childdoc.def}|\\
|\childdocof{|\textit{main}|}|\\
\end{tabular}
\end{center}
at the top of every child file \textit{child}
which is included by |\include{|\textit{child}|}|
from within the main file
(or at least for those files to be compiled individually).
The argument \textit{main} must be the filename of the main file.

There are a couple of
considerations in setting up the main and child documents:

%%%%%%%%%%%%%%%%%%%%%%%%%%%%%%%%%%%%%%%%
\paragraph{Restrictions.}

Please note the following restrictions:
\begin{itemize}
\item
|\childdocmain| must be called with one argument \textit{main}
to ensure compatibility with earlier version of the package.
It must either be empty (|\childdocmain{}|)
or precisely match the filename of the main file in which it is specified.
See \secref{sec:detection} for further information.
\item
The filename \textit{main} must be specified without the |.tex| extension.
\item
The filename \textit{main} is case sensitive
(even in case-insensitive file systems)
due to internal string comparison.
\item
The argument \textit{main} should be fully expanded, it cannot be a macro.
\item
Subdirectories and special characters should be avoided in filenames.
\item
The command |\childdocmain{|\textit{main}|}| must be followed by a whitespace.
It should not be followed immediately by another command
or by a comment mark `|%|'.
This is because the \TeX{} parser reads the token immediately following
the argument of |\childdocmain| and puts it
at the beginning of every child section;
however, a white\-space is ignored.
\end{itemize}

%%%%%%%%%%%%%%%%%%%%%%%%%%%%%%%%%%%%%%%%
\paragraph{Content of Main File.}

It is advisable to place all content in the child files included by |\include|.
Any output contained in the main file will appear in all child documents
unless suppressed manually;
it cannot be suppressed automatically by the |\includeonly| directive
and thus should normally be avoided.
A method to include some content in the main file
by means of conditional processing is described in \secref{sec:conditional}.

%%%%%%%%%%%%%%%%%%%%%%%%%%%%%%%%%%%%%%%%
\paragraph{Page Numbering.}

When only a part of the document is compiled,
the appropriate numbering of pages
(as well as other status parameters)
is determined from the |.aux| files.
The latter contain information from previous passes.
However this information needs to propagate through
all intermediate child documents.
Therefore the page numbering in child documents may well
be inconsistent until the complete document is compiled at least once.

A useful (if unconventional) way to always ensure a consistent
page numbering is to restart the numbering in each child document
and denote the pages by `\textit{child}|.|\textit{page}'
where \textit{child} represents the chapter/section number of the child file.
This can be achieved by the command
|\numberwithin{page}{|\textit{child}|}|
of the \textsf{amsmath} package
where \textit{child} can be |chapter| or |section|
depending on the chosen structuring.
Alternatively, one can modify the macro |\thepage| appropriately
and reset the counter |page| at the start of each child file.

%%%%%%%%%%%%%%%%%%%%%%%%%%%%%%%%%%%%%%%%%%%%%%%%%%%%%%%%%%%%%%%%%%%%%%%%%%%%%%%%
\subsection{Conditional Processing}
\label{sec:conditional}

The package provides a mechanism to compile different versions
of a document. To customise the versions further some conditional processing
can come in handy to distinguish which version is being compiled.
The package provides two macros to describe the compilation context:

%%%%%%%%%%%%%%%%%%%%%%%%%%%%%%%%%%%%%%%%
\DescribeMacro{\ifchilddoc}
The conditional |\ifchilddoc| distinguishes between the compilation of
child documents and the main document:
%
\begin{center}
|\ifchilddoc |\textit{child-code}| |[|\||else |\textit{main-code}]| \||fi|
\end{center}

%%%%%%%%%%%%%%%%%%%%%%%%%%%%%%%%%%%%%%%%
\DescribeMacro{\childdocname}
\DescribeMacro{\childdocjob}
The macro |\childdocname| contains the filename (without extension)
of the main or child file being processed.
Note that |\childdocjob| will always contain the name of the main file.

%%%%%%%%%%%%%%%%%%%%%%%%%%%%%%%%%%%%%%%%
\paragraph{Title Page.}

Conditional processing can be used to include a title or banner page
in the main document when proper precautions are taken.
Importantly, the code in the main file should ensure that the page counter
(as well as other status parameters which are stored in the |.aux| files)
takes the same value after the conditional processing.
Otherwise the page numbers may take divergent values
depending on which part is compiled.

For example, a title page could be declared by:
%
\begin{center}
\begin{tabular}{l}
|\ifchilddoc\||else|\\
|\addtocounter{page}{-1}|\\
\textit{code for title page}\\
|\newpage|\\
|\||fi|
\end{tabular}
\end{center}
%
A banner page for the child documents can be generated by:
%
\begin{center}
\begin{tabular}{l}
|\ifchilddoc|\\
|\addtocounter{page}{-1}|\\
\textit{code for banner page}\\
|\newpage|\\
|\||fi|
\end{tabular}
\end{center}
%
Here one could write a message such as:
\begin{center}
|This is the part \childdocname{} of \childdocjob{}.|
\end{center}

%%%%%%%%%%%%%%%%%%%%%%%%%%%%%%%%%%%%%%%%%%%%%%%%%%%%%%%%%%%%%%%%%%%%%%%%%%%%%%%%
\subsection{Flags}
\label{sec:flags}

The package makes it easy to generate different versions
of the main or child documents.
To this end compilation flags can be defined
and assigned different default values.
They will be particularly useful in conjunction
with the forwarding mechanism described in \secref{sec:forward}.

For example, it may be useful to have a flag |\version|
which can be set to |draft| or |final|.
The document source will contain some conditional code
depending on the value of |\version|.
Suppose further, the flag should default to |final| for the main file
and to |draft| for child files
which is a natural assignment for editing the document.
This is achieved by placing the following code
in the preamble of the main document
(below the |\childdocmain| directive):
%
\begin{center}
\begin{tabular}{l}
|\ifchilddoc|\\
|\providecommand{\version}{draft}|\\
|\||else|\\
|\providecommand{\version}{final}|\\
|\||fi|
\end{tabular}
\end{center}
%
The definition by |\providecommand| makes sure
that previous definitions are not overwritten.
Further statements |\providecommand{\version}{...}|
can thus be added before the above code to override it.

For the main file, one might add a line
(between |\childdocmain| and the above block)
%
\begin{center}
|%\ifchilddoc\||else\providecommand{\version}{draft}\||fi|
\end{center}
%
which can be uncommented to produce a draft version.
Likewise one can add a line to the very top of a child file
(above the |\childdocof{|\textit{main}|}| directive)
%
\begin{center}
|%\providecommand{\version}{final}|
\end{center}
%
which can be uncommented to produce the final version of this child document.

%%%%%%%%%%%%%%%%%%%%%%%%%%%%%%%%%%%%%%%%%%%%%%%%%%%%%%%%%%%%%%%%%%%%%%%%%%%%%%%%
\subsection{Forwarding}
\label{sec:forward}

Different versions of the main or child documents
using compilation flags as described in \secref{sec:flags}
can be (permanently) stored in different files
for convenient compilation, viewing and distribution.
To this end, the package defines a command
to pass on compilation to a different file:

%%%%%%%%%%%%%%%%%%%%%%%%%%%%%%%%%%%%%%%%
\DescribeMacro{\childdocforward}
The command |\childdocforward| redirects processing to
another source file:
%
\begin{center}
\begin{tabular}{l}
|\input{childdoc.def}|\\
|\childdocforward[|\textit{main}|]{|\textit{dest}|}|\\
\end{tabular}
\end{center}
%
The argument \textit{dest} is the destination file
(without extension).
It should be the main file or one of the child files.
Note that further \textsf{childdoc} directives
such as |\childdocof| and |\childdocforward|
in the indicated file will be processed in this form.
The optional argument \textit{main}
passes on directly to the main file \textit{main}
while pretending to compile the child \textit{dest}.
This form behaves as if \textit{dest}
issues |\childdocof{|\textit{main}|}| right away,
and no further \textsf{childdoc} directives will be processed.

%%%%%%%%%%%%%%%%%%%%%%%%%%%%%%%%%%%%%%%%
\DescribeMacro{\...prefix}
In the alternative form |\childdocforwardprefix|,
%
\begin{center}
\begin{tabular}{l}
|\input{childdoc.def}|\\
|\childdocforwardprefix[|\textit{main}|]{|\textit{prefix}|}{|\textit{dest}|}|
\end{tabular}
\end{center}
%
the destination file is determined by a pattern
depending on the current file:
To make this work, the current file must be called
`{\textit{prefix}\hspace{0.2em}\textit{suffix}}'
with \textit{prefix} matching precisely the argument.
Processing is then passed on to the file
`{\textit{dest}\hspace{0.2em}\textit{suffix}}'.
Surely, the same effect is achieved by
directly specifying the
argument `{\textit{dest}\hspace{0.2em}\textit{suffix}}'
in the first form.
However, that requires to set up a different file
for each child. With the alternative form of the command
all these files can have exactly the same content
which simplifies setting them up and maintaining them.

For example, the following file |draft.tex|
with a compilation flag |\version| as described in \secref{sec:flags}
compiles the main document as a draft:
%
\begin{center}
\begin{tabular}{l}
|\def\version{draft}|\\
|\input{childdoc.def}|\\
|\childdocforward{|\textit{main}|}|
\end{tabular}
\end{center}
%
Likewise, the following files |final|\textit{nn}|.tex|
compile the final version of the child document
|child|\textit{nn}|.tex|:
%
\begin{center}
\begin{tabular}{l}
|\def\version{final}|\\
|\input{childdoc.def}|\\
|\childdocforwardprefix{final}{child}|
\end{tabular}
\end{center}
%

Note that when several versions of a main file and/or of each child file
are to be generated, it may be convenient to set up a |Makefile| or
shell script to automatise the process.

%%%%%%%%%%%%%%%%%%%%%%%%%%%%%%%%%%%%%%%%%%%%%%%%%%%%%%%%%%%%%%%%%%%%%%%%%%%%%%%%
\subsection{Command Line Processing}
\label{sec:commandline}

The effect of redirection files can also be achieved by invoking
the \LaTeX{} compiler with a more elaborate command line.
Most conveniently this should be done as part
of a shell script or a |Makefile|.

When using \textsf{childdoc} in the main file, the following
command lines effectively perform a redirection
(note that depending on the shell being used,
backslashes may have to be doubled: `|\|' $\to$ `|\\|'):
%
\begin{center}
|... -jobname "|\textit{target}|" |\\|"|[\textit{flags}]%
|\input{childdoc.def}\childdocforward[|\textit{main}|]{|\textit{dest}|}"|
\end{center}
%
Here \textit{target} is the name of the output file,
\textit{main} is the name of the main file
and \textit{dest} is the name of the main or child file to be processed
(all filenames without extensions).
The optional argument \textit{main} can be omitted
if \textit{main} matches \textit{dest}.
Optionally, compilation \textit{flags} can be defined via |\def| commands.
This command line makes the \TeX{} engine believe
it is compiling the file \textit{target}
whose content is specified as the latter parameter.
The provided code then forwards the processing to
\textit{main} or \textit{dest} as described in \secref{sec:forward}.

%%%%%%%%%%%%%%%%%%%%%%%%%%%%%%%%%%%%%%%%%%%%%%%%%%%%%%%%%%%%%%%%%%%%%%%%%%%%%%%%
\subsection{Include by Input}
\label{sec:input}

Including child documents by |\include| has some restrictions by design.
Most notably, the content of a child document always occupies
its own set of pages; pages cannot be shared between child documents.
Usually, this behaviour makes perfect sense
because each child document contain an essential part of the document.
However, in some situations it may be desirable to compose
a document from a collection of parts
without having mandatory page breaks between then.
For this case, the package
provides a mechanism to include parts
by |\input| which can also be processed individually.
However, by construction this mechanism
requires manual handling of the content to be output.

%%%%%%%%%%%%%%%%%%%%%%%%%%%%%%%%%%%%%%%%
\DescribeMacro{\ifchilddocmanual}
The main file should be prepared as usual, see \secref{sec:include}.
However, the document body must make a distinction
between processing of an individual part and of the main document, e.g.:
%
\begin{center}
\begin{tabular}{l}
|\ifchilddocmanual|\\
|\input{\childdocname}|\\
|\||else|\\
\textit{document body with }|\input{|\textit{part}|}|\\
|\||fi|
\end{tabular}
\end{center}
%
The conditional |\ifchilddocmanual| is true whenever
a part to be included by |\input| is being compiled,
and the name of the part is stored in |\childdocname|.

%%%%%%%%%%%%%%%%%%%%%%%%%%%%%%%%%%%%%%%%
\DescribeMacro{\childdocby}
Each part to be included by |\input| should start with:
%
\begin{center}
\begin{tabular}{l}
|\input{childdoc.def}|\\
|\childdocby{|\textit{main}|}|\\
\end{tabular}
\end{center}
%
The directive |\childdocby| is similar to |\childdocof|
described in \secref{sec:include},
but the subsequent selection of content must be done manually.
To that end, both |\ifchilddoc| and |\ifchilddocmanual|
will be true upon processing of a part,
and the name of the part is stored in |\childdocname|.
Note that |\jobname| will be set to the filename of the current part
so that each part receives an individual |.aux| file
that does not interfere with the |.aux| file(s) of the main document.
This behaviour can be altered by the alternative form
|\childdocby[*]{|\textit{main}|}| (with a non-empty optional argument)
which uses the |.aux| file of the main document
by setting |\jobname| to \textit{main}.

%%%%%%%%%%%%%%%%%%%%%%%%%%%%%%%%%%%%%%%%%%%%%%%%%%%%%%%%%%%%%%%%%%%%%%%%%%%%%%%%
\subsection{Driver Development}
\label{sec:driver}

The \textsf{childdoc} mechanism can also be use for the development
of definition files such as \LaTeX{} styles or classes.
This case differs from the above setup with multiple parts
included by |\include| in that no |\includeonly| should be invoked.
This can be achieved by starting the include file
(before |\ProvidesPackage|) with:
%
\begin{center}
\begin{tabular}{l}
|\input{childdoc.def}|\\
|\childdocforward{|\textit{main}|}|\\
\end{tabular}
\end{center}
%
or alternatively with:
%
\begin{center}
\begin{tabular}{l}
|\input{childdoc.def}|\\
|\childdocby{|\textit{main}|}|\\
\end{tabular}
\end{center}
%
Both forms have slightly different effects as described above.
The main file is prepared as usual, see \secref{sec:include}.

%%%%%%%%%%%%%%%%%%%%%%%%%%%%%%%%%%%%%%%%%%%%%%%%%%%%%%%%%%%%%%%%%%%%%%%%%%%%%%%%
\subsection{Legacy Detection}
\label{sec:detection}

The directive |\childdocmain| in the main file can detect
whether the complete document or merely a child is to be compiled
even without using the directive |\childdocof|.
This method is deprecated because it is less robust
and there is no compelling reason to use it;
it is merely provided for backward compatibility
and it may be removed in future versions.

If the detection mechanism is to be used,
it is mandatory to correctly specify
the filename of the main file as the argument of |\childdocmain|:
%
\begin{center}
\begin{tabular}{l}
|\input{childdoc.def}|\\
|\childdocmain{|\textit{main}|}|\\
\end{tabular}
\end{center}
%
If |\jobname| does not match the argument \textit{main} of |\childdocmain|,
it is assumed that |\jobname| points to the child file to be compiled.
When using |\childdocmain| with the main file specified as argument,
it suffices to start a child file
with just |\input{|\textit{main}|}|
without loading of the package and using |\childdocof|.
If instead all processing is done
with the appropriate \textsf{childdoc} directives,
the argument of \textit{main} of |\childdocmain| can be empty.

An alternative version of the command line processing described
in \secref{sec:commandline} using the detection mechanism reads:
%
\begin{center}
|... -jobname "|\textit{target}|" "|[\textit{flags}]%
[|\def\jobname{|\textit{dest}|}|]|\input{|\textit{main}|}"|
\end{center}

%%%%%%%%%%%%%%%%%%%%%%%%%%%%%%%%%%%%%%%%%%%%%%%%%%%%%%%%%%%%%%%%%%%%%%%%%%%%%%%%
\subsection{Manual Code}
\label{sec:manual}

In case one cannot be certain whether the definitions file |childdoc.def|
is installed on the target \TeX{} distribution
and one prefers not to ship it,
it is conceivable to paste a few relevant commands into the sources.

To that end, drop all statements |\input{childdoc.def}|
and perform the replacements as outlined below.
Instead of |\childdocmain{|\textit{main}|}| add the following code
to the top of the main file:
%
\begin{center}
\begin{tabular}{l}
|\||ifdefined\childdocname\endinput\||fi\newif\ifchilddoc|\\
|\edef\childdocname{\scantokens\expandafter{\jobname\noexpand}}|\\
|\def\childdocmain{|\textit{main}|}\||ifx\childdocmain\childdocname\||else|\\
|\childdoctrue\includeonly{\childdocname}\let\jobname\childdocmain\||fi|\\
\end{tabular}
\end{center}
%
Instead of |\childdocof{|\textit{main}|}| just include the main file
at the top of each child file:
%
\begin{center}
|\input{|\textit{main}|}|
\end{center}
%
A simple redirection |\childdocforward{|\textit{dest}|}| is achieved by:
%
\begin{center}
|\def\jobname{|\textit{dest}|}\input{\jobname}|
\end{center}
%
The redirection with prefix
|\childdocforwardprefix[|\textit{prefix}|]{|\textit{dest}|}|
is accomplished by:
%
\begin{center}
\begin{tabular}{l}
|{\edef\jobname{\scantokens\expandafter{\jobname\noexpand}}|\\
|\def\redirectjob |\textit{prefix}|#1~~~{\gdef\jobname{|\textit{dest}|#1}}|\\
|\expandafter\redirectjob\jobname~~~}\input{\jobname}|
\end{tabular}
\end{center}

In an alternative approach,
child documents can be compiled by a specific command line
without additional code or specific definitions:
%
\begin{center}
|... -jobname "|\textit{target}|" "|[\textit{flags}]%
|\includeonly{|\textit{dest}|}\input{|\textit{main}|}"|
\end{center}
%

%%%%%%%%%%%%%%%%%%%%%%%%%%%%%%%%%%%%%%%%%%%%%%%%%%%%%%%%%%%%%%%%%%%%%%%%%%%%%%%%
%%%%%%%%%%%%%%%%%%%%%%%%%%%%%%%%%%%%%%%%%%%%%%%%%%%%%%%%%%%%%%%%%%%%%%%%%%%%%%%%
\section{Information}

%%%%%%%%%%%%%%%%%%%%%%%%%%%%%%%%%%%%%%%%%%%%%%%%%%%%%%%%%%%%%%%%%%%%%%%%%%%%%%%%
\subsection{Copyright}

Copyright \copyright{} 2017--2018 Niklas Beisert

This work may be distributed and/or modified under the
conditions of the \LaTeX{} Project Public License, either version 1.3
of this license or (at your option) any later version.
The latest version of this license is in
  \url{http://www.latex-project.org/lppl.txt}
and version 1.3 or later is part of all distributions of \LaTeX{}
version 2005/12/01 or later.

This work has the LPPL maintenance status `maintained'.

The Current Maintainer of this work is Niklas Beisert.

This work consists of the files |README.txt|, |childdoc.ins| and |childdoc.dtx|
as well as the derived files |childdoc.def|, |cdocsamp.tex|
with |cdocsch1.tex|, |cdocsch2.tex|, |cdocspt3.tex|, |cdocspt4.tex|,
|cdocsdrf.tex|, |cdocsfn1.tex|, |cdocsfn2.tex|
as well as |childdoc.pdf|.

%%%%%%%%%%%%%%%%%%%%%%%%%%%%%%%%%%%%%%%%%%%%%%%%%%%%%%%%%%%%%%%%%%%%%%%%%%%%%%%%
\subsection{Files and Installation}

The package consists of the files:
%
\begin{center}
\begin{tabular}{ll}
    |README.txt|   & readme file \\
    |childdoc.ins| & installation file \\
    |childdoc.dtx| & source file \\
    |childdoc.def| & definition file \\
    |cdocsamp.tex| & sample main file \\
    |cdocsch1.tex| & sample include file \\
    |cdocsch2.tex| & sample include file \\
    |cdocspt3.tex| & sample part file \\
    |cdocspt4.tex| & sample part file \\
    |cdocsdrf.tex| & sample redirection file \\
    |cdocsfn1.tex| & sample redirection file \\
    |cdocsfn2.tex| & sample redirection file \\
    |childdoc.pdf| & manual
\end{tabular}
\end{center}
%
The distribution consists of the files
|README.txt|, |childdoc.ins| and |childdoc.dtx|.
%
\begin{itemize}
\item
Run (pdf)\LaTeX{} on |childdoc.dtx|
to compile the manual |childdoc.pdf| (this file).
\item
Run \LaTeX{} on |childdoc.ins| to create the definitions file |childdoc.def|
and the sample |cdocsamp.tex| with include files
|cdocsch1.tex|, |cdocsch2.tex|, |cdocspt3.tex|, |cdocspt4.tex|,
|cdocsdrf.tex|, |cdocsfn1.tex|, |cdocsfn2.tex|.
Then copy the file |childdoc.def| to an appropriate directory of your \LaTeX{}
distribution, e.g.\ \textit{texmf-root}|/tex/latex/childdoc|.
\end{itemize}

%%%%%%%%%%%%%%%%%%%%%%%%%%%%%%%%%%%%%%%%%%%%%%%%%%%%%%%%%%%%%%%%%%%%%%%%%%%%%%%%
\subsection{Related CTAN Packages}

There are several other packages which offer a similar functionality:
%
\begin{itemize}
\item
The packages
\href{http://ctan.org/pkg/docmute}{\textsf{docmute}},
\href{http://ctan.org/pkg/includex}{\textsf{includex}} and
\href{http://ctan.org/pkg/standalone}{\textsf{standalone}}
provide commands to include only the document body of
a child file thus allowing both files to be compiled individually.
\item
The packages \href{http://ctan.org/pkg/subdocs}{\textsf{subdocs}}
and \href{http://ctan.org/pkg/subfiles}{\textsf{subfiles}}
provide structures in which the main and child documents can be
encapsulated and allowing them to be compiled individually.
The inclusion mechanism is different from the conventional |\include|.
\item
The package \href{http://ctan.org/pkg/combine}{\textsf{combine}}
is an elaborate solution to combine several documents into one.
\end{itemize}
%
See also the CTAN topic \href{http://ctan.org/topic/subdocs}{\textsf{subdocs}}
for further related packages.
The present package differs from the above solutions in that
a document structure constructed with the conventional |\include| mechanism
just needs two extra commands at the top of every file
such that all constituent files can be compiled individually.

%%%%%%%%%%%%%%%%%%%%%%%%%%%%%%%%%%%%%%%%%%%%%%%%%%%%%%%%%%%%%%%%%%%%%%%%%%%%%%%%
%\subsection{Feature Suggestions}
%
%The following is a list of features which may be useful for future
%versions of this package:
%%
%\begin{itemize}
%\item
%\ldots
%\end{itemize}

%%%%%%%%%%%%%%%%%%%%%%%%%%%%%%%%%%%%%%%%%%%%%%%%%%%%%%%%%%%%%%%%%%%%%%%%%%%%%%%%
\subsection{Revision History}

%%%%%%%%%%%%%%%%%%%%%%%%%%%%%%%%%%%%%%%%
\paragraph{v2.0:} 2018/12/30

\begin{itemize}
\item
immediate forward processing
\item
added |\childdocby| mechanism
\item
manual restructured
\end{itemize}

%%%%%%%%%%%%%%%%%%%%%%%%%%%%%%%%%%%%%%%%
\paragraph{v1.6:} 2018/01/17

\begin{itemize}
\item
application for development of include files
\item
corrections to manual
\end{itemize}

%%%%%%%%%%%%%%%%%%%%%%%%%%%%%%%%%%%%%%%%
\paragraph{v1.5:} 2017/05/21

\begin{itemize}
\item
more complete structuring introduced
\item
|\childdocof| introduced
\item
|\childdoc| renamed to |\childdocmain|
\item
|\childredirect| renamed to |\childdocforward| and |\childdocforwardprefix|
and functionality expanded
\end{itemize}

%%%%%%%%%%%%%%%%%%%%%%%%%%%%%%%%%%%%%%%%
\paragraph{v1.0:} 2017/04/27

\begin{itemize}
\item
manual and install package
\item
first version published on CTAN
\end{itemize}

%%%%%%%%%%%%%%%%%%%%%%%%%%%%%%%%%%%%%%%%
\paragraph{v0.6:} 2017/04/26

\begin{itemize}
\item
redirection mechanism added
\end{itemize}

%%%%%%%%%%%%%%%%%%%%%%%%%%%%%%%%%%%%%%%%
\paragraph{v0.5:} 2017/04/26

\begin{itemize}
\item
functionality in definition file
\end{itemize}


%%%%%%%%%%%%%%%%%%%%%%%%%%%%%%%%%%%%%%%%%%%%%%%%%%%%%%%%%%%%%%%%%%%%%%%%%%%%%%%%
%%%%%%%%%%%%%%%%%%%%%%%%%%%%%%%%%%%%%%%%%%%%%%%%%%%%%%%%%%%%%%%%%%%%%%%%%%%%%%%%
%%%%%%%%%%%%%%%%%%%%%%%%%%%%%%%%%%%%%%%%%%%%%%%%%%%%%%%%%%%%%%%%%%%%%%%%%%%%%%%%
\appendix

\settowidth\MacroIndent{\rmfamily\scriptsize 000\ }

 \DocInput{childdoc.dtx}

\end{document}
%</driver>
% \fi
%
% %%%%%%%%%%%%%%%%%%%%%%%%%%%%%%%%%%%%%%%%%%%%%%%%%%%%%%%%%%%%%%%%%%%%%%%%%%%%%%
% %%%%%%%%%%%%%%%%%%%%%%%%%%%%%%%%%%%%%%%%%%%%%%%%%%%%%%%%%%%%%%%%%%%%%%%%%%%%%%
% \section{Sample}
%\iffalse
%<*samplemain>
%\fi
%
% The following presents a sample document
% with two chapters, two parts, a title page,
% a compile flag as well as three forwarding files to set the flag.
% It consists of eight |.tex| files:
% \begin{center}
% \begin{tabular}{ll}
% |cdocsamp.tex|&main file\\
% |cdocsch1.tex|&include file for chapter 1\\
% |cdocsch2.tex|&include file for chapter 2\\
% |cdocspt3.tex|&include file for part 3\\
% |cdocspt4.tex|&include file for part 4\\
% |cdocsdrf.tex|&forwarding file for main file in draft mode\\
% |cdocsfi1.tex|&forwarding file for final version of chapter 1\\
% |cdocsfi2.tex|&forwarding file for final version of chapter 2\\
% \end{tabular}
% \end{center}
% Each of the eight files can be compiled directly by the \LaTeX{} compiler.
%
% %%%%%%%%%%%%%%%%%%%%%%%%%%%%%%%%%%%%%%
% \paragraph{Main File.}
%
% The main file is called |cdocsamp.tex|.
%
% Load the \textsf{childdoc} definitions and
% declare the filename for the main document:
%    \begin{macrocode}
\input{childdoc.def}
\childdocmain{}
%    \end{macrocode}

% Optional override for |\version| flag:
%    \begin{macrocode}
%%\ifchilddoc\else\providecommand{\version}{draft}\fi
%    \end{macrocode}

% Define the default values for the |\version| flag
% (|final| for the main file and |draft| for childs):
%    \begin{macrocode}
\ifchilddoc
\providecommand{\version}{draft}
\else
\providecommand{\version}{final}
\fi
%    \end{macrocode}

% Load the standard document class:
%    \begin{macrocode}
\documentclass[12pt]{article}
%    \end{macrocode}

% Start the document body:
%    \begin{macrocode}
\begin{document}
%    \end{macrocode}

% Declare a title page.
% Print title, part of document being processed and version flag:
%    \begin{macrocode}
\addtocounter{page}{-1}
\begin{center}
{\LARGE\bfseries{}childdoc example\par}
\vspace{1cm}
\ifchilddoc
\ifchilddocmanual part\else chapter\fi:
`\childdocname' of `\childdocjob'\par
\else
main document: `\childdocjob'\par
\fi
version: \version\par
\end{center}
\newpage
%    \end{macrocode}

% Manually include selected file,
% otherwise process as usual:
%    \begin{macrocode}
\ifchilddocmanual
\section*{part `\childdocname'}
\input{\childdocname}
\else
%    \end{macrocode}

% Include the two chapters:
%    \begin{macrocode}
\include{cdocsch1}
\include{cdocsch2}
%    \end{macrocode}

% Include the two parts unless only chapters should be displayed:
%    \begin{macrocode}
\ifchilddoc\else
\section{part three}
\input{cdocspt3}
\section{part four}
\input{cdocspt4}
\fi
%    \end{macrocode}

% Process as usual until here:
%    \begin{macrocode}
\fi
%    \end{macrocode}

% End of document body:
%    \begin{macrocode}
\end{document}
%    \end{macrocode}
%\iffalse
%</samplemain>
%\fi
%
% %%%%%%%%%%%%%%%%%%%%%%%%%%%%%%%%%%%%%%
% \paragraph{Chapter Include Files.}
%
% The include files are called |cdocsch1.tex| and |cdocsch2.tex|.
%
%\iffalse
%<*samplechap1|samplechap2>
%\fi

% Optional override for |\version| flag:
%    \begin{macrocode}
%%\providecommand{\version}{final}
%    \end{macrocode}

% Include the main document:
%    \begin{macrocode}
\input{childdoc.def}
\childdocof{cdocsamp}
%    \end{macrocode}

%\iffalse
%</samplechap1|samplechap2>
%\fi
%
%\iffalse
%<*samplechap1>
%\fi
% Some text for chapter 1:
%    \begin{macrocode}
\section{one}
some text in chapter one
%    \end{macrocode}

%\iffalse
%</samplechap1>
%\fi
% Some text for chapter 2:
%\iffalse
%<*samplechap2>
%\fi
%    \begin{macrocode}
\section{two}
more text in chapter two
%    \end{macrocode}

%\iffalse
%</samplechap2>
%\fi
%
% %%%%%%%%%%%%%%%%%%%%%%%%%%%%%%%%%%%%%%
% \paragraph{Part Include Files.}
%
% The include files are called |cdocspt3.tex| and |cdocspt4.tex|.
%
%\iffalse
%<*samplepart3|samplepart4>
%\fi

% Optional override for |\version| flag:
%    \begin{macrocode}
%%\providecommand{\version}{final}
%    \end{macrocode}

% Include the main document:
%    \begin{macrocode}
\input{childdoc.def}
\childdocby{cdocsamp}
%    \end{macrocode}

%\iffalse
%</samplepart3|samplepart4>
%\fi
%
%\iffalse
%<*samplepart3>
%\fi
% Some text for part 3:
%    \begin{macrocode}
some text in part three
%    \end{macrocode}

%\iffalse
%</samplepart3>
%\fi
% Some text for part 4:
%\iffalse
%<*samplepart4>
%\fi
%    \begin{macrocode}
more text in part four
%    \end{macrocode}

%\iffalse
%</samplepart4>
%\fi
%
% %%%%%%%%%%%%%%%%%%%%%%%%%%%%%%%%%%%%%%
% \paragraph{Forwarding for a Complete Draft.}
%
% The following forwarding file |cdocsdrf.tex|
% compiles the main document in draft mode:
%\iffalse
%<*sampledraft>
%\fi
%    \begin{macrocode}
\def\version{draft}
\input{childdoc.def}
\childdocforward{cdocsamp}
%    \end{macrocode}

%\iffalse
%</sampledraft>
%\fi
%
% %%%%%%%%%%%%%%%%%%%%%%%%%%%%%%%%%%%%%%
% \paragraph{Forwarding for Final Version of the Chapters.}
%
% The following forwarding files |cdocsfn1.tex| and |cdocsfn2.tex|
% (with identical content)
% compile the final versions of the child documents
% |cdocsch1.tex| and |cdocsch2.tex|, respectively:
%\iffalse
%<*samplefinal>
%\fi
%    \begin{macrocode}
\def\version{final}
\input{childdoc.def}
\childdocforwardprefix[cdocsamp]{cdocsfn}{cdocsch}
%    \end{macrocode}

%\iffalse
%</samplefinal>
%\fi
%
% %%%%%%%%%%%%%%%%%%%%%%%%%%%%%%%%%%%%%%
% \paragraph{Command Line Processing.}
%
% The following three command lines generate the output files
% |cdocscld|, |cdocscl1| and |cdocscl2|
% which should be identical to
% |cdocsdrf|, |cdocsch1| and |cdocsfn2|, respectively:
% \begin{center}
% \begin{tabular}{l}
% |latex -jobname cdocscld \|\\
% |  "\def\version{draft}\input{childdoc.def}\childdocforward{cdocsamp}"|\\
% |latex -jobname cdocscl1 \|\\
% |  "\input{childdoc.def}\childdocforward[cdocsamp]{cdocsch1}"|\\
% |latex -jobname cdocscl2 \|\\
% |  "\def\version{final}\input{childdoc.def}\childdocforward{cdocsch2}"|
% \end{tabular}
% \end{center}
% Note that the trailing backslash on each first line
% merely continues the input to the second line
% (for convenient cut ant paste).
% Furthermore, the command |latex| can be replaced by any
% of its alternative versions such as |pdflatex|.
%
% %%%%%%%%%%%%%%%%%%%%%%%%%%%%%%%%%%%%%%%%%%%%%%%%%%%%%%%%%%%%%%%%%%%%%%%%%%%%%%
% %%%%%%%%%%%%%%%%%%%%%%%%%%%%%%%%%%%%%%%%%%%%%%%%%%%%%%%%%%%%%%%%%%%%%%%%%%%%%%
% \section{Implementation}
%\iffalse
%<*package>
%\fi
%
% This section describes the definitions file |childdoc.def|.

% The definitions cannot be loaded using |\usepackage| or |\RequirePackage|
% which has a mechanism to prevent loading a style file more than once.
% When loading the definitions by means of |\input|
% multiple instances have to be prevented manually:
%\iffalse
%This code needs to be before the `\ProvidesFile' directive
%which is defined at the beginning of this file.
%Therefore it is also placed there and commented out here.
%</package>
%<*discard>
%\fi
%    \begin{macrocode}
\ifdefined\childdocmain\endinput\fi
%    \end{macrocode}
%\iffalse
%</discard>
%<*package>
%\fi
%
% \macro{\ifchilddoc}
% \macro{\ifchilddocmanual}
% The conditional |\ifchilddoc| tells whether a
% child (true) or main (false) document is being compiled.
% The conditional |\ifchilddocmanual| tells whether
% the |\includeonly| mechanism is used (false) or
% the selection of child files must be performed manually (true).
% The definitions initialise to false:
%    \begin{macrocode}
\newif\ifchilddoc
\newif\ifchilddocmanual
%    \end{macrocode}

% \macro{\childdocname}
% \macro{\childdocjob}
% The macro |\childdocname| stores the name of the main document
% to be compiled. The macro |\childdocjob| stores the name of
% the document on which the \LaTeX{} compiler was originally invoked.
% The content of |\jobname| cannot be compared
% to filenames specified in the source due to different catcodes.
% The following code rescans |\jobname|, stores the result
% in |\childdocname| and saves a copy in |\childdocjob|:
%    \begin{macrocode}
\edef\childdocname{\scantokens\expandafter{\jobname\noexpand}}
\let\childdocjob\childdocname
%    \end{macrocode}

% \macro{\childdocdisable}
% The macro |\childdocdisable| prevents the main file
% from being processed more than once.
% At this stage, the main document command |\childdocmain|
% is assumed to be called once again where it should do nothing.
% Any subsequent call to it should prevent
% a secondary processing of the main document
% It overwrites the forwarding commands
% |\childdocof| and |\childdocforward|
% with empty macros to prevent further inclusions of the main document:
%    \begin{macrocode}
\newcommand{\childdocdisable}
{
  \renewcommand{\childdocmain}[1]{\renewcommand{\childdocmain}[1]{\endinput}}
  \renewcommand{\childdocof}[1]{}
  \renewcommand{\childdocby}[2][]{}
  \renewcommand{\childdocforward}[2][]{}
  \renewcommand{\childdocdisable}{}
}
%    \end{macrocode}

% \macro{\childdocmain}
% The macro |\childdocmain| is to be called at the top of the main file
% with nothing or the main filename (without extension) as argument.
% First, it breaks loops.
% If the argument is not empty and does not match |\childdocname|
% (which is set by the first inclusion of |childdoc.def|),
% |\ifchilddoc| is set to true, |\includeonly| is applied to the child file
% and |\jobname| is set to the main file
% (for proper handling of |.aux| files):
%    \begin{macrocode}
\newcommand{\childdocmain}[1]
{
  \childdocdisable\childdocmain{}
  \if?#1?\else
    \begingroup
      \def\childdoctmp{#1}
      \ifx\childdoctmp\childdocname
        \def\childdoctmp{}
      \else
        \def\childdoctmp
        {
          \childdoctrue
          \includeonly{\childdocname}
          \def\childdocjob{#1}
          \def\jobname{#1}
        }
      \fi
      \expandafter
    \endgroup
    \childdoctmp
  \fi
}
%    \end{macrocode}

% \macro{\childdocof}
% The command |\childdocof| redirects
% compilation to the main file |#1|.
%    \begin{macrocode}
\newcommand{\childdocof}[1]
{
  \childdocdisable
  \childdoctrue
  \includeonly{\childdocname}
  \def\jobname{#1}
  \def\childdocjob{#1}
  \input{#1}
}
%    \end{macrocode}

% \macro{\childdocby}
% The command |\childdocby| ....
%    \begin{macrocode}
\newcommand{\childdocby}[2][]
{
  \childdocdisable
  \childdoctrue
  \childdocmanualtrue
  \if?#1?\else
    \def\jobname{#2}
  \fi
  \def\childdocjob{#2}
  \input{#2}
  \endinput
}
%    \end{macrocode}

% \macro{\childdocforward}
% The command |\childdocforward| redirects
% compilation to the main file or
% (if the optional argument is given) a child file.
% Parameters are set as if the main file
% or a child file starting with |\childdocof| was compiled.
% Then compilation is handed over to the main file:
%    \begin{macrocode}
\newcommand{\childdocforward}[2][]
{
  \begingroup
    \if?#1?
      \def\childdoctmp
      {
        \def\childdocname{#2}
        \def\childdocjob{#2}
        \def\jobname{#2}
        \input{#2}
        \endinput
      }
    \else
      \def\childdoctmp
      {
        \childdocdisable
        \def\childdocname{#2}
        \childdoctrue
        \includeonly{#2}
        \def\childdocjob{#1}
        \def\jobname{#1}
        \input{#1}
        \endinput
      }
    \fi
    \expandafter
  \endgroup
  \childdoctmp
}
%    \end{macrocode}

% \macro{\childdocforwardprefix}
% The command |\childdocforwardprefix| redirects
% compilation to the main or a child file by means of a pattern.
% The prefix |#1| in the current filename is replaced by |#2|
% and the suffix of the current filename is kept
% (it is assumed that the filename does not contain the substring `|~~~|'
% which is used as a delimiter).
% Compilation is handed over to the new file by |\childdocforward|:
%    \begin{macrocode}
\newcommand{\childdocforwardprefix}[3][]
{
  \begingroup
    \def\childdocextract #2##1~~~{\def\childdoctmp{\childdocforward[#1]{#3##1}}}
    \expandafter\childdocextract\childdocname~~~
    \expandafter
  \endgroup
  \childdoctmp
}
%    \end{macrocode}

% \macro{\childdoc}
% The deprecated macro |\childdoc| is a legacy version of |\childdocmain|:
%    \begin{macrocode}
\newcommand{\childdoc}{\childdocmain}
%    \end{macrocode}

% \macro{\childdocredirect}
% The deprecated macro |\childdocredirect| is a legacy version
% of |\childdocforward| and |\childdocforwardprefix|:
%    \begin{macrocode}
\newcommand{\childdocredirect}[2][]
{
  \begingroup
    \if?#1?
      \def\childdoctmp{\childdocforward{#2}}
    \else
      \def\childdoctmp{\childdocforwardprefix{#1}{#2}}
    \fi
    \expandafter
  \endgroup
  \childdoctmp
}
%    \end{macrocode}

%\iffalse
%</package>
%\fi
%
\endinput
|\\
|\childdocby{|\textit{main}|}|\\
\end{tabular}
\end{center}
%
Both forms have slightly different effects as described above.
The main file is prepared as usual, see \secref{sec:include}.

%%%%%%%%%%%%%%%%%%%%%%%%%%%%%%%%%%%%%%%%%%%%%%%%%%%%%%%%%%%%%%%%%%%%%%%%%%%%%%%%
\subsection{Legacy Detection}
\label{sec:detection}

The directive |\childdocmain| in the main file can detect
whether the complete document or merely a child is to be compiled
even without using the directive |\childdocof|.
This method is deprecated because it is less robust
and there is no compelling reason to use it;
it is merely provided for backward compatibility
and it may be removed in future versions.

If the detection mechanism is to be used,
it is mandatory to correctly specify
the filename of the main file as the argument of |\childdocmain|:
%
\begin{center}
\begin{tabular}{l}
|% \iffalse
%
% childdoc.dtx Copyright (C) 2017-2018 Niklas Beisert
%
% This work may be distributed and/or modified under the
% conditions of the LaTeX Project Public License, either version 1.3
% of this license or (at your option) any later version.
% The latest version of this license is in
%   http://www.latex-project.org/lppl.txt
% and version 1.3 or later is part of all distributions of LaTeX
% version 2005/12/01 or later.
%
% This work has the LPPL maintenance status `maintained'.
%
% The Current Maintainer of this work is Niklas Beisert.
%
% This work consists of the files childdoc.dtx and childdoc.ins
% and the derived files childdoc.def and cdocsamp.tex with
% cdocsch1.tex, cdocsch2.tex, cdocsdrf.tex, cdocsfn1.tex, cdocsfn2.tex.
%
%<package>\ifdefined\childdocmain\endinput\fi
%<package>\ProvidesFile{childdoc.def}[2018/12/30 v2.0 child document driver]
%<samplemain>\ProvidesFile{cdocsamp.tex}[2018/12/30 v2.0 sample for childdoc]
%<*driver>
%\ProvidesFile{childdoc.drv}[2018/12/30 v2.0 childdoc reference manual file]
\PassOptionsToClass{10pt,a4paper}{article}
\documentclass{ltxdoc}

\usepackage[margin=35mm]{geometry}
\usepackage{hyperref}
\usepackage{hyperxmp}
\usepackage[usenames]{color}

\hypersetup{colorlinks=true}
\hypersetup{pdfstartview=FitH}
\hypersetup{pdfpagemode=UseNone}
\hypersetup{pdfsource={}}
\hypersetup{pdflang={en-UK}}
\hypersetup{pdfcopyright={Copyright 2017-2018 Niklas Beisert.
  This work may be distributed and/or modified under the
  conditions of the LaTeX Project Public License, either version 1.3
  of this license or (at your option) any later version.}}
\hypersetup{pdflicenseurl={http://www.latex-project.org/lppl.txt}}
\hypersetup{pdfcontactaddress={ETH Zurich, ITP, HIT K,
  Wolfgang-Pauli-Strasse 27}}
\hypersetup{pdfcontactpostcode={8093}}
\hypersetup{pdfcontactcity={Zurich}}
\hypersetup{pdfcontactcountry={Switzerland}}
\hypersetup{pdfcontactemail={nbeisert@itp.phys.ethz.ch}}
\hypersetup{pdfcontacturl={http://people.phys.ethz.ch/\xmptilde nbeisert/}}

\newcommand{\secref}[1]{\hyperref[#1]{section \ref*{#1}}}

\parskip1ex
\parindent0pt
\let\olditemize\itemize
\def\itemize{\olditemize\parskip0pt}

\begin{document}

\title{The \textsf{childdoc} Package}
\hypersetup{pdftitle={The childdoc Package}}
\author{Niklas Beisert\\[2ex]
  Institut f\"ur Theoretische Physik\\
  Eidgen\"ossische Technische Hochschule Z\"urich\\
  Wolfgang-Pauli-Strasse 27, 8093 Z\"urich, Switzerland\\[1ex]
  \href{mailto:nbeisert@itp.phys.ethz.ch}
  {\texttt{nbeisert@itp.phys.ethz.ch}}}
\hypersetup{pdfauthor={Niklas Beisert}}
\hypersetup{pdfsubject={Manual for the LaTeX2e Package childdoc}}
\date{30 December 2018, \textsf{v2.0}}
\maketitle

\begin{abstract}\noindent
\textsf{childdoc} is a \LaTeXe{} package
that enables the direct compilation
of document sections included by |\include|
to individual files.
\end{abstract}

\begingroup
\parskip0ex
\tableofcontents
\endgroup

%%%%%%%%%%%%%%%%%%%%%%%%%%%%%%%%%%%%%%%%%%%%%%%%%%%%%%%%%%%%%%%%%%%%%%%%%%%%%%%%
%%%%%%%%%%%%%%%%%%%%%%%%%%%%%%%%%%%%%%%%%%%%%%%%%%%%%%%%%%%%%%%%%%%%%%%%%%%%%%%%
\section{Introduction}

\LaTeX{} provides a mechanism to structure a large document (such as a book)
into a main file and several child files (containing the chapters)
using the |\include| command.
This mechanism is beneficial for documents
which span hundreds of pages in order to
make the source file(s) more manageable.
Moreover, compilation can be restricted to
selected child files by means of the |\includeonly| command.
The latter feature can be used to reduce the compilation time while editing
(this was significantly more useful in the earlier days of \LaTeX{})
or to generate a smaller document which is easier to navigate.
Another application of |\includeonly| is to generate
documents consisting of selected parts of the complete document.

However, there are a few drawbacks of the plain |\include| mechanism:
\begin{itemize}
\item
The child files cannot be compiled on their own,
they can only be compiled via the main file.
A naive editing environment
(such as a text editor with an option
to have the current file processed by \LaTeX)
may require one to switch to the main file before compiling;
attempting to compile the child file produces errors.
\item
The main file must be modified (each time)
to adjust the |\includeonly| command
to the present needs. This easily leaves the main file in a messy state.
\item
The generated document will always carry the filename
of the main document. This is inconvenient if
several child files are to be compiled and
to be kept for distribution.
\end{itemize}

The present package provides a simple interface
to make child files individually compilable by \LaTeX{}.
Compiling a child file then has the same effect as compiling
the main file with an |\includeonly| command
to select the appropriate child.
Moreover the generated document will carry the name of the child
rather than the main file.
This resolves all three above issues.

This feature is meant to make the editing of books,
thesis documents and lecture notes somewhat more convenient.
However, the package can also be used efficiently for
composing a series of documents (such as exercise sheets)
which are typically distributed individually.
It then assists the author in generating the individual documents
(potentially in different versions)
as well as a document containing the collected series.
Another application is in developing style files
or other kinds of included material
where compilation of the style file could redirect
to a sample or test file.

%%%%%%%%%%%%%%%%%%%%%%%%%%%%%%%%%%%%%%%%%%%%%%%%%%%%%%%%%%%%%%%%%%%%%%%%%%%%%%%%
%%%%%%%%%%%%%%%%%%%%%%%%%%%%%%%%%%%%%%%%%%%%%%%%%%%%%%%%%%%%%%%%%%%%%%%%%%%%%%%%
\section{Usage}

First of all, the package \textsf{childdoc} is \emph{not} a standard
\LaTeXe{} |.sty| style file! Therefore it needs to be invoked in
a non-standard way.

%%%%%%%%%%%%%%%%%%%%%%%%%%%%%%%%%%%%%%%%%%%%%%%%%%%%%%%%%%%%%%%%%%%%%%%%%%%%%%%%
\subsection{Included Files}
\label{sec:include}

%%%%%%%%%%%%%%%%%%%%%%%%%%%%%%%%%%%%%%%%
\DescribeMacro{\childdocmain}
To use the package, add the commands
\begin{center}
\begin{tabular}{l}
|\input{childdoc.def}|\\
|\childdocmain{}|\\
\end{tabular}
\end{center}
at the very top of the main \LaTeX{} file,
in particular \emph{before} the |\documentclass| statement!
The argument of |\childdocmain| should be left empty
(but it must be present).

%%%%%%%%%%%%%%%%%%%%%%%%%%%%%%%%%%%%%%%%
\DescribeMacro{\childdocof}
Furthermore, add the commands
\begin{center}
\begin{tabular}{l}
|\input{childdoc.def}|\\
|\childdocof{|\textit{main}|}|\\
\end{tabular}
\end{center}
at the top of every child file \textit{child}
which is included by |\include{|\textit{child}|}|
from within the main file
(or at least for those files to be compiled individually).
The argument \textit{main} must be the filename of the main file.

There are a couple of
considerations in setting up the main and child documents:

%%%%%%%%%%%%%%%%%%%%%%%%%%%%%%%%%%%%%%%%
\paragraph{Restrictions.}

Please note the following restrictions:
\begin{itemize}
\item
|\childdocmain| must be called with one argument \textit{main}
to ensure compatibility with earlier version of the package.
It must either be empty (|\childdocmain{}|)
or precisely match the filename of the main file in which it is specified.
See \secref{sec:detection} for further information.
\item
The filename \textit{main} must be specified without the |.tex| extension.
\item
The filename \textit{main} is case sensitive
(even in case-insensitive file systems)
due to internal string comparison.
\item
The argument \textit{main} should be fully expanded, it cannot be a macro.
\item
Subdirectories and special characters should be avoided in filenames.
\item
The command |\childdocmain{|\textit{main}|}| must be followed by a whitespace.
It should not be followed immediately by another command
or by a comment mark `|%|'.
This is because the \TeX{} parser reads the token immediately following
the argument of |\childdocmain| and puts it
at the beginning of every child section;
however, a white\-space is ignored.
\end{itemize}

%%%%%%%%%%%%%%%%%%%%%%%%%%%%%%%%%%%%%%%%
\paragraph{Content of Main File.}

It is advisable to place all content in the child files included by |\include|.
Any output contained in the main file will appear in all child documents
unless suppressed manually;
it cannot be suppressed automatically by the |\includeonly| directive
and thus should normally be avoided.
A method to include some content in the main file
by means of conditional processing is described in \secref{sec:conditional}.

%%%%%%%%%%%%%%%%%%%%%%%%%%%%%%%%%%%%%%%%
\paragraph{Page Numbering.}

When only a part of the document is compiled,
the appropriate numbering of pages
(as well as other status parameters)
is determined from the |.aux| files.
The latter contain information from previous passes.
However this information needs to propagate through
all intermediate child documents.
Therefore the page numbering in child documents may well
be inconsistent until the complete document is compiled at least once.

A useful (if unconventional) way to always ensure a consistent
page numbering is to restart the numbering in each child document
and denote the pages by `\textit{child}|.|\textit{page}'
where \textit{child} represents the chapter/section number of the child file.
This can be achieved by the command
|\numberwithin{page}{|\textit{child}|}|
of the \textsf{amsmath} package
where \textit{child} can be |chapter| or |section|
depending on the chosen structuring.
Alternatively, one can modify the macro |\thepage| appropriately
and reset the counter |page| at the start of each child file.

%%%%%%%%%%%%%%%%%%%%%%%%%%%%%%%%%%%%%%%%%%%%%%%%%%%%%%%%%%%%%%%%%%%%%%%%%%%%%%%%
\subsection{Conditional Processing}
\label{sec:conditional}

The package provides a mechanism to compile different versions
of a document. To customise the versions further some conditional processing
can come in handy to distinguish which version is being compiled.
The package provides two macros to describe the compilation context:

%%%%%%%%%%%%%%%%%%%%%%%%%%%%%%%%%%%%%%%%
\DescribeMacro{\ifchilddoc}
The conditional |\ifchilddoc| distinguishes between the compilation of
child documents and the main document:
%
\begin{center}
|\ifchilddoc |\textit{child-code}| |[|\||else |\textit{main-code}]| \||fi|
\end{center}

%%%%%%%%%%%%%%%%%%%%%%%%%%%%%%%%%%%%%%%%
\DescribeMacro{\childdocname}
\DescribeMacro{\childdocjob}
The macro |\childdocname| contains the filename (without extension)
of the main or child file being processed.
Note that |\childdocjob| will always contain the name of the main file.

%%%%%%%%%%%%%%%%%%%%%%%%%%%%%%%%%%%%%%%%
\paragraph{Title Page.}

Conditional processing can be used to include a title or banner page
in the main document when proper precautions are taken.
Importantly, the code in the main file should ensure that the page counter
(as well as other status parameters which are stored in the |.aux| files)
takes the same value after the conditional processing.
Otherwise the page numbers may take divergent values
depending on which part is compiled.

For example, a title page could be declared by:
%
\begin{center}
\begin{tabular}{l}
|\ifchilddoc\||else|\\
|\addtocounter{page}{-1}|\\
\textit{code for title page}\\
|\newpage|\\
|\||fi|
\end{tabular}
\end{center}
%
A banner page for the child documents can be generated by:
%
\begin{center}
\begin{tabular}{l}
|\ifchilddoc|\\
|\addtocounter{page}{-1}|\\
\textit{code for banner page}\\
|\newpage|\\
|\||fi|
\end{tabular}
\end{center}
%
Here one could write a message such as:
\begin{center}
|This is the part \childdocname{} of \childdocjob{}.|
\end{center}

%%%%%%%%%%%%%%%%%%%%%%%%%%%%%%%%%%%%%%%%%%%%%%%%%%%%%%%%%%%%%%%%%%%%%%%%%%%%%%%%
\subsection{Flags}
\label{sec:flags}

The package makes it easy to generate different versions
of the main or child documents.
To this end compilation flags can be defined
and assigned different default values.
They will be particularly useful in conjunction
with the forwarding mechanism described in \secref{sec:forward}.

For example, it may be useful to have a flag |\version|
which can be set to |draft| or |final|.
The document source will contain some conditional code
depending on the value of |\version|.
Suppose further, the flag should default to |final| for the main file
and to |draft| for child files
which is a natural assignment for editing the document.
This is achieved by placing the following code
in the preamble of the main document
(below the |\childdocmain| directive):
%
\begin{center}
\begin{tabular}{l}
|\ifchilddoc|\\
|\providecommand{\version}{draft}|\\
|\||else|\\
|\providecommand{\version}{final}|\\
|\||fi|
\end{tabular}
\end{center}
%
The definition by |\providecommand| makes sure
that previous definitions are not overwritten.
Further statements |\providecommand{\version}{...}|
can thus be added before the above code to override it.

For the main file, one might add a line
(between |\childdocmain| and the above block)
%
\begin{center}
|%\ifchilddoc\||else\providecommand{\version}{draft}\||fi|
\end{center}
%
which can be uncommented to produce a draft version.
Likewise one can add a line to the very top of a child file
(above the |\childdocof{|\textit{main}|}| directive)
%
\begin{center}
|%\providecommand{\version}{final}|
\end{center}
%
which can be uncommented to produce the final version of this child document.

%%%%%%%%%%%%%%%%%%%%%%%%%%%%%%%%%%%%%%%%%%%%%%%%%%%%%%%%%%%%%%%%%%%%%%%%%%%%%%%%
\subsection{Forwarding}
\label{sec:forward}

Different versions of the main or child documents
using compilation flags as described in \secref{sec:flags}
can be (permanently) stored in different files
for convenient compilation, viewing and distribution.
To this end, the package defines a command
to pass on compilation to a different file:

%%%%%%%%%%%%%%%%%%%%%%%%%%%%%%%%%%%%%%%%
\DescribeMacro{\childdocforward}
The command |\childdocforward| redirects processing to
another source file:
%
\begin{center}
\begin{tabular}{l}
|\input{childdoc.def}|\\
|\childdocforward[|\textit{main}|]{|\textit{dest}|}|\\
\end{tabular}
\end{center}
%
The argument \textit{dest} is the destination file
(without extension).
It should be the main file or one of the child files.
Note that further \textsf{childdoc} directives
such as |\childdocof| and |\childdocforward|
in the indicated file will be processed in this form.
The optional argument \textit{main}
passes on directly to the main file \textit{main}
while pretending to compile the child \textit{dest}.
This form behaves as if \textit{dest}
issues |\childdocof{|\textit{main}|}| right away,
and no further \textsf{childdoc} directives will be processed.

%%%%%%%%%%%%%%%%%%%%%%%%%%%%%%%%%%%%%%%%
\DescribeMacro{\...prefix}
In the alternative form |\childdocforwardprefix|,
%
\begin{center}
\begin{tabular}{l}
|\input{childdoc.def}|\\
|\childdocforwardprefix[|\textit{main}|]{|\textit{prefix}|}{|\textit{dest}|}|
\end{tabular}
\end{center}
%
the destination file is determined by a pattern
depending on the current file:
To make this work, the current file must be called
`{\textit{prefix}\hspace{0.2em}\textit{suffix}}'
with \textit{prefix} matching precisely the argument.
Processing is then passed on to the file
`{\textit{dest}\hspace{0.2em}\textit{suffix}}'.
Surely, the same effect is achieved by
directly specifying the
argument `{\textit{dest}\hspace{0.2em}\textit{suffix}}'
in the first form.
However, that requires to set up a different file
for each child. With the alternative form of the command
all these files can have exactly the same content
which simplifies setting them up and maintaining them.

For example, the following file |draft.tex|
with a compilation flag |\version| as described in \secref{sec:flags}
compiles the main document as a draft:
%
\begin{center}
\begin{tabular}{l}
|\def\version{draft}|\\
|\input{childdoc.def}|\\
|\childdocforward{|\textit{main}|}|
\end{tabular}
\end{center}
%
Likewise, the following files |final|\textit{nn}|.tex|
compile the final version of the child document
|child|\textit{nn}|.tex|:
%
\begin{center}
\begin{tabular}{l}
|\def\version{final}|\\
|\input{childdoc.def}|\\
|\childdocforwardprefix{final}{child}|
\end{tabular}
\end{center}
%

Note that when several versions of a main file and/or of each child file
are to be generated, it may be convenient to set up a |Makefile| or
shell script to automatise the process.

%%%%%%%%%%%%%%%%%%%%%%%%%%%%%%%%%%%%%%%%%%%%%%%%%%%%%%%%%%%%%%%%%%%%%%%%%%%%%%%%
\subsection{Command Line Processing}
\label{sec:commandline}

The effect of redirection files can also be achieved by invoking
the \LaTeX{} compiler with a more elaborate command line.
Most conveniently this should be done as part
of a shell script or a |Makefile|.

When using \textsf{childdoc} in the main file, the following
command lines effectively perform a redirection
(note that depending on the shell being used,
backslashes may have to be doubled: `|\|' $\to$ `|\\|'):
%
\begin{center}
|... -jobname "|\textit{target}|" |\\|"|[\textit{flags}]%
|\input{childdoc.def}\childdocforward[|\textit{main}|]{|\textit{dest}|}"|
\end{center}
%
Here \textit{target} is the name of the output file,
\textit{main} is the name of the main file
and \textit{dest} is the name of the main or child file to be processed
(all filenames without extensions).
The optional argument \textit{main} can be omitted
if \textit{main} matches \textit{dest}.
Optionally, compilation \textit{flags} can be defined via |\def| commands.
This command line makes the \TeX{} engine believe
it is compiling the file \textit{target}
whose content is specified as the latter parameter.
The provided code then forwards the processing to
\textit{main} or \textit{dest} as described in \secref{sec:forward}.

%%%%%%%%%%%%%%%%%%%%%%%%%%%%%%%%%%%%%%%%%%%%%%%%%%%%%%%%%%%%%%%%%%%%%%%%%%%%%%%%
\subsection{Include by Input}
\label{sec:input}

Including child documents by |\include| has some restrictions by design.
Most notably, the content of a child document always occupies
its own set of pages; pages cannot be shared between child documents.
Usually, this behaviour makes perfect sense
because each child document contain an essential part of the document.
However, in some situations it may be desirable to compose
a document from a collection of parts
without having mandatory page breaks between then.
For this case, the package
provides a mechanism to include parts
by |\input| which can also be processed individually.
However, by construction this mechanism
requires manual handling of the content to be output.

%%%%%%%%%%%%%%%%%%%%%%%%%%%%%%%%%%%%%%%%
\DescribeMacro{\ifchilddocmanual}
The main file should be prepared as usual, see \secref{sec:include}.
However, the document body must make a distinction
between processing of an individual part and of the main document, e.g.:
%
\begin{center}
\begin{tabular}{l}
|\ifchilddocmanual|\\
|\input{\childdocname}|\\
|\||else|\\
\textit{document body with }|\input{|\textit{part}|}|\\
|\||fi|
\end{tabular}
\end{center}
%
The conditional |\ifchilddocmanual| is true whenever
a part to be included by |\input| is being compiled,
and the name of the part is stored in |\childdocname|.

%%%%%%%%%%%%%%%%%%%%%%%%%%%%%%%%%%%%%%%%
\DescribeMacro{\childdocby}
Each part to be included by |\input| should start with:
%
\begin{center}
\begin{tabular}{l}
|\input{childdoc.def}|\\
|\childdocby{|\textit{main}|}|\\
\end{tabular}
\end{center}
%
The directive |\childdocby| is similar to |\childdocof|
described in \secref{sec:include},
but the subsequent selection of content must be done manually.
To that end, both |\ifchilddoc| and |\ifchilddocmanual|
will be true upon processing of a part,
and the name of the part is stored in |\childdocname|.
Note that |\jobname| will be set to the filename of the current part
so that each part receives an individual |.aux| file
that does not interfere with the |.aux| file(s) of the main document.
This behaviour can be altered by the alternative form
|\childdocby[*]{|\textit{main}|}| (with a non-empty optional argument)
which uses the |.aux| file of the main document
by setting |\jobname| to \textit{main}.

%%%%%%%%%%%%%%%%%%%%%%%%%%%%%%%%%%%%%%%%%%%%%%%%%%%%%%%%%%%%%%%%%%%%%%%%%%%%%%%%
\subsection{Driver Development}
\label{sec:driver}

The \textsf{childdoc} mechanism can also be use for the development
of definition files such as \LaTeX{} styles or classes.
This case differs from the above setup with multiple parts
included by |\include| in that no |\includeonly| should be invoked.
This can be achieved by starting the include file
(before |\ProvidesPackage|) with:
%
\begin{center}
\begin{tabular}{l}
|\input{childdoc.def}|\\
|\childdocforward{|\textit{main}|}|\\
\end{tabular}
\end{center}
%
or alternatively with:
%
\begin{center}
\begin{tabular}{l}
|\input{childdoc.def}|\\
|\childdocby{|\textit{main}|}|\\
\end{tabular}
\end{center}
%
Both forms have slightly different effects as described above.
The main file is prepared as usual, see \secref{sec:include}.

%%%%%%%%%%%%%%%%%%%%%%%%%%%%%%%%%%%%%%%%%%%%%%%%%%%%%%%%%%%%%%%%%%%%%%%%%%%%%%%%
\subsection{Legacy Detection}
\label{sec:detection}

The directive |\childdocmain| in the main file can detect
whether the complete document or merely a child is to be compiled
even without using the directive |\childdocof|.
This method is deprecated because it is less robust
and there is no compelling reason to use it;
it is merely provided for backward compatibility
and it may be removed in future versions.

If the detection mechanism is to be used,
it is mandatory to correctly specify
the filename of the main file as the argument of |\childdocmain|:
%
\begin{center}
\begin{tabular}{l}
|\input{childdoc.def}|\\
|\childdocmain{|\textit{main}|}|\\
\end{tabular}
\end{center}
%
If |\jobname| does not match the argument \textit{main} of |\childdocmain|,
it is assumed that |\jobname| points to the child file to be compiled.
When using |\childdocmain| with the main file specified as argument,
it suffices to start a child file
with just |\input{|\textit{main}|}|
without loading of the package and using |\childdocof|.
If instead all processing is done
with the appropriate \textsf{childdoc} directives,
the argument of \textit{main} of |\childdocmain| can be empty.

An alternative version of the command line processing described
in \secref{sec:commandline} using the detection mechanism reads:
%
\begin{center}
|... -jobname "|\textit{target}|" "|[\textit{flags}]%
[|\def\jobname{|\textit{dest}|}|]|\input{|\textit{main}|}"|
\end{center}

%%%%%%%%%%%%%%%%%%%%%%%%%%%%%%%%%%%%%%%%%%%%%%%%%%%%%%%%%%%%%%%%%%%%%%%%%%%%%%%%
\subsection{Manual Code}
\label{sec:manual}

In case one cannot be certain whether the definitions file |childdoc.def|
is installed on the target \TeX{} distribution
and one prefers not to ship it,
it is conceivable to paste a few relevant commands into the sources.

To that end, drop all statements |\input{childdoc.def}|
and perform the replacements as outlined below.
Instead of |\childdocmain{|\textit{main}|}| add the following code
to the top of the main file:
%
\begin{center}
\begin{tabular}{l}
|\||ifdefined\childdocname\endinput\||fi\newif\ifchilddoc|\\
|\edef\childdocname{\scantokens\expandafter{\jobname\noexpand}}|\\
|\def\childdocmain{|\textit{main}|}\||ifx\childdocmain\childdocname\||else|\\
|\childdoctrue\includeonly{\childdocname}\let\jobname\childdocmain\||fi|\\
\end{tabular}
\end{center}
%
Instead of |\childdocof{|\textit{main}|}| just include the main file
at the top of each child file:
%
\begin{center}
|\input{|\textit{main}|}|
\end{center}
%
A simple redirection |\childdocforward{|\textit{dest}|}| is achieved by:
%
\begin{center}
|\def\jobname{|\textit{dest}|}\input{\jobname}|
\end{center}
%
The redirection with prefix
|\childdocforwardprefix[|\textit{prefix}|]{|\textit{dest}|}|
is accomplished by:
%
\begin{center}
\begin{tabular}{l}
|{\edef\jobname{\scantokens\expandafter{\jobname\noexpand}}|\\
|\def\redirectjob |\textit{prefix}|#1~~~{\gdef\jobname{|\textit{dest}|#1}}|\\
|\expandafter\redirectjob\jobname~~~}\input{\jobname}|
\end{tabular}
\end{center}

In an alternative approach,
child documents can be compiled by a specific command line
without additional code or specific definitions:
%
\begin{center}
|... -jobname "|\textit{target}|" "|[\textit{flags}]%
|\includeonly{|\textit{dest}|}\input{|\textit{main}|}"|
\end{center}
%

%%%%%%%%%%%%%%%%%%%%%%%%%%%%%%%%%%%%%%%%%%%%%%%%%%%%%%%%%%%%%%%%%%%%%%%%%%%%%%%%
%%%%%%%%%%%%%%%%%%%%%%%%%%%%%%%%%%%%%%%%%%%%%%%%%%%%%%%%%%%%%%%%%%%%%%%%%%%%%%%%
\section{Information}

%%%%%%%%%%%%%%%%%%%%%%%%%%%%%%%%%%%%%%%%%%%%%%%%%%%%%%%%%%%%%%%%%%%%%%%%%%%%%%%%
\subsection{Copyright}

Copyright \copyright{} 2017--2018 Niklas Beisert

This work may be distributed and/or modified under the
conditions of the \LaTeX{} Project Public License, either version 1.3
of this license or (at your option) any later version.
The latest version of this license is in
  \url{http://www.latex-project.org/lppl.txt}
and version 1.3 or later is part of all distributions of \LaTeX{}
version 2005/12/01 or later.

This work has the LPPL maintenance status `maintained'.

The Current Maintainer of this work is Niklas Beisert.

This work consists of the files |README.txt|, |childdoc.ins| and |childdoc.dtx|
as well as the derived files |childdoc.def|, |cdocsamp.tex|
with |cdocsch1.tex|, |cdocsch2.tex|, |cdocspt3.tex|, |cdocspt4.tex|,
|cdocsdrf.tex|, |cdocsfn1.tex|, |cdocsfn2.tex|
as well as |childdoc.pdf|.

%%%%%%%%%%%%%%%%%%%%%%%%%%%%%%%%%%%%%%%%%%%%%%%%%%%%%%%%%%%%%%%%%%%%%%%%%%%%%%%%
\subsection{Files and Installation}

The package consists of the files:
%
\begin{center}
\begin{tabular}{ll}
    |README.txt|   & readme file \\
    |childdoc.ins| & installation file \\
    |childdoc.dtx| & source file \\
    |childdoc.def| & definition file \\
    |cdocsamp.tex| & sample main file \\
    |cdocsch1.tex| & sample include file \\
    |cdocsch2.tex| & sample include file \\
    |cdocspt3.tex| & sample part file \\
    |cdocspt4.tex| & sample part file \\
    |cdocsdrf.tex| & sample redirection file \\
    |cdocsfn1.tex| & sample redirection file \\
    |cdocsfn2.tex| & sample redirection file \\
    |childdoc.pdf| & manual
\end{tabular}
\end{center}
%
The distribution consists of the files
|README.txt|, |childdoc.ins| and |childdoc.dtx|.
%
\begin{itemize}
\item
Run (pdf)\LaTeX{} on |childdoc.dtx|
to compile the manual |childdoc.pdf| (this file).
\item
Run \LaTeX{} on |childdoc.ins| to create the definitions file |childdoc.def|
and the sample |cdocsamp.tex| with include files
|cdocsch1.tex|, |cdocsch2.tex|, |cdocspt3.tex|, |cdocspt4.tex|,
|cdocsdrf.tex|, |cdocsfn1.tex|, |cdocsfn2.tex|.
Then copy the file |childdoc.def| to an appropriate directory of your \LaTeX{}
distribution, e.g.\ \textit{texmf-root}|/tex/latex/childdoc|.
\end{itemize}

%%%%%%%%%%%%%%%%%%%%%%%%%%%%%%%%%%%%%%%%%%%%%%%%%%%%%%%%%%%%%%%%%%%%%%%%%%%%%%%%
\subsection{Related CTAN Packages}

There are several other packages which offer a similar functionality:
%
\begin{itemize}
\item
The packages
\href{http://ctan.org/pkg/docmute}{\textsf{docmute}},
\href{http://ctan.org/pkg/includex}{\textsf{includex}} and
\href{http://ctan.org/pkg/standalone}{\textsf{standalone}}
provide commands to include only the document body of
a child file thus allowing both files to be compiled individually.
\item
The packages \href{http://ctan.org/pkg/subdocs}{\textsf{subdocs}}
and \href{http://ctan.org/pkg/subfiles}{\textsf{subfiles}}
provide structures in which the main and child documents can be
encapsulated and allowing them to be compiled individually.
The inclusion mechanism is different from the conventional |\include|.
\item
The package \href{http://ctan.org/pkg/combine}{\textsf{combine}}
is an elaborate solution to combine several documents into one.
\end{itemize}
%
See also the CTAN topic \href{http://ctan.org/topic/subdocs}{\textsf{subdocs}}
for further related packages.
The present package differs from the above solutions in that
a document structure constructed with the conventional |\include| mechanism
just needs two extra commands at the top of every file
such that all constituent files can be compiled individually.

%%%%%%%%%%%%%%%%%%%%%%%%%%%%%%%%%%%%%%%%%%%%%%%%%%%%%%%%%%%%%%%%%%%%%%%%%%%%%%%%
%\subsection{Feature Suggestions}
%
%The following is a list of features which may be useful for future
%versions of this package:
%%
%\begin{itemize}
%\item
%\ldots
%\end{itemize}

%%%%%%%%%%%%%%%%%%%%%%%%%%%%%%%%%%%%%%%%%%%%%%%%%%%%%%%%%%%%%%%%%%%%%%%%%%%%%%%%
\subsection{Revision History}

%%%%%%%%%%%%%%%%%%%%%%%%%%%%%%%%%%%%%%%%
\paragraph{v2.0:} 2018/12/30

\begin{itemize}
\item
immediate forward processing
\item
added |\childdocby| mechanism
\item
manual restructured
\end{itemize}

%%%%%%%%%%%%%%%%%%%%%%%%%%%%%%%%%%%%%%%%
\paragraph{v1.6:} 2018/01/17

\begin{itemize}
\item
application for development of include files
\item
corrections to manual
\end{itemize}

%%%%%%%%%%%%%%%%%%%%%%%%%%%%%%%%%%%%%%%%
\paragraph{v1.5:} 2017/05/21

\begin{itemize}
\item
more complete structuring introduced
\item
|\childdocof| introduced
\item
|\childdoc| renamed to |\childdocmain|
\item
|\childredirect| renamed to |\childdocforward| and |\childdocforwardprefix|
and functionality expanded
\end{itemize}

%%%%%%%%%%%%%%%%%%%%%%%%%%%%%%%%%%%%%%%%
\paragraph{v1.0:} 2017/04/27

\begin{itemize}
\item
manual and install package
\item
first version published on CTAN
\end{itemize}

%%%%%%%%%%%%%%%%%%%%%%%%%%%%%%%%%%%%%%%%
\paragraph{v0.6:} 2017/04/26

\begin{itemize}
\item
redirection mechanism added
\end{itemize}

%%%%%%%%%%%%%%%%%%%%%%%%%%%%%%%%%%%%%%%%
\paragraph{v0.5:} 2017/04/26

\begin{itemize}
\item
functionality in definition file
\end{itemize}


%%%%%%%%%%%%%%%%%%%%%%%%%%%%%%%%%%%%%%%%%%%%%%%%%%%%%%%%%%%%%%%%%%%%%%%%%%%%%%%%
%%%%%%%%%%%%%%%%%%%%%%%%%%%%%%%%%%%%%%%%%%%%%%%%%%%%%%%%%%%%%%%%%%%%%%%%%%%%%%%%
%%%%%%%%%%%%%%%%%%%%%%%%%%%%%%%%%%%%%%%%%%%%%%%%%%%%%%%%%%%%%%%%%%%%%%%%%%%%%%%%
\appendix

\settowidth\MacroIndent{\rmfamily\scriptsize 000\ }

 \DocInput{childdoc.dtx}

\end{document}
%</driver>
% \fi
%
% %%%%%%%%%%%%%%%%%%%%%%%%%%%%%%%%%%%%%%%%%%%%%%%%%%%%%%%%%%%%%%%%%%%%%%%%%%%%%%
% %%%%%%%%%%%%%%%%%%%%%%%%%%%%%%%%%%%%%%%%%%%%%%%%%%%%%%%%%%%%%%%%%%%%%%%%%%%%%%
% \section{Sample}
%\iffalse
%<*samplemain>
%\fi
%
% The following presents a sample document
% with two chapters, two parts, a title page,
% a compile flag as well as three forwarding files to set the flag.
% It consists of eight |.tex| files:
% \begin{center}
% \begin{tabular}{ll}
% |cdocsamp.tex|&main file\\
% |cdocsch1.tex|&include file for chapter 1\\
% |cdocsch2.tex|&include file for chapter 2\\
% |cdocspt3.tex|&include file for part 3\\
% |cdocspt4.tex|&include file for part 4\\
% |cdocsdrf.tex|&forwarding file for main file in draft mode\\
% |cdocsfi1.tex|&forwarding file for final version of chapter 1\\
% |cdocsfi2.tex|&forwarding file for final version of chapter 2\\
% \end{tabular}
% \end{center}
% Each of the eight files can be compiled directly by the \LaTeX{} compiler.
%
% %%%%%%%%%%%%%%%%%%%%%%%%%%%%%%%%%%%%%%
% \paragraph{Main File.}
%
% The main file is called |cdocsamp.tex|.
%
% Load the \textsf{childdoc} definitions and
% declare the filename for the main document:
%    \begin{macrocode}
\input{childdoc.def}
\childdocmain{}
%    \end{macrocode}

% Optional override for |\version| flag:
%    \begin{macrocode}
%%\ifchilddoc\else\providecommand{\version}{draft}\fi
%    \end{macrocode}

% Define the default values for the |\version| flag
% (|final| for the main file and |draft| for childs):
%    \begin{macrocode}
\ifchilddoc
\providecommand{\version}{draft}
\else
\providecommand{\version}{final}
\fi
%    \end{macrocode}

% Load the standard document class:
%    \begin{macrocode}
\documentclass[12pt]{article}
%    \end{macrocode}

% Start the document body:
%    \begin{macrocode}
\begin{document}
%    \end{macrocode}

% Declare a title page.
% Print title, part of document being processed and version flag:
%    \begin{macrocode}
\addtocounter{page}{-1}
\begin{center}
{\LARGE\bfseries{}childdoc example\par}
\vspace{1cm}
\ifchilddoc
\ifchilddocmanual part\else chapter\fi:
`\childdocname' of `\childdocjob'\par
\else
main document: `\childdocjob'\par
\fi
version: \version\par
\end{center}
\newpage
%    \end{macrocode}

% Manually include selected file,
% otherwise process as usual:
%    \begin{macrocode}
\ifchilddocmanual
\section*{part `\childdocname'}
\input{\childdocname}
\else
%    \end{macrocode}

% Include the two chapters:
%    \begin{macrocode}
\include{cdocsch1}
\include{cdocsch2}
%    \end{macrocode}

% Include the two parts unless only chapters should be displayed:
%    \begin{macrocode}
\ifchilddoc\else
\section{part three}
\input{cdocspt3}
\section{part four}
\input{cdocspt4}
\fi
%    \end{macrocode}

% Process as usual until here:
%    \begin{macrocode}
\fi
%    \end{macrocode}

% End of document body:
%    \begin{macrocode}
\end{document}
%    \end{macrocode}
%\iffalse
%</samplemain>
%\fi
%
% %%%%%%%%%%%%%%%%%%%%%%%%%%%%%%%%%%%%%%
% \paragraph{Chapter Include Files.}
%
% The include files are called |cdocsch1.tex| and |cdocsch2.tex|.
%
%\iffalse
%<*samplechap1|samplechap2>
%\fi

% Optional override for |\version| flag:
%    \begin{macrocode}
%%\providecommand{\version}{final}
%    \end{macrocode}

% Include the main document:
%    \begin{macrocode}
\input{childdoc.def}
\childdocof{cdocsamp}
%    \end{macrocode}

%\iffalse
%</samplechap1|samplechap2>
%\fi
%
%\iffalse
%<*samplechap1>
%\fi
% Some text for chapter 1:
%    \begin{macrocode}
\section{one}
some text in chapter one
%    \end{macrocode}

%\iffalse
%</samplechap1>
%\fi
% Some text for chapter 2:
%\iffalse
%<*samplechap2>
%\fi
%    \begin{macrocode}
\section{two}
more text in chapter two
%    \end{macrocode}

%\iffalse
%</samplechap2>
%\fi
%
% %%%%%%%%%%%%%%%%%%%%%%%%%%%%%%%%%%%%%%
% \paragraph{Part Include Files.}
%
% The include files are called |cdocspt3.tex| and |cdocspt4.tex|.
%
%\iffalse
%<*samplepart3|samplepart4>
%\fi

% Optional override for |\version| flag:
%    \begin{macrocode}
%%\providecommand{\version}{final}
%    \end{macrocode}

% Include the main document:
%    \begin{macrocode}
\input{childdoc.def}
\childdocby{cdocsamp}
%    \end{macrocode}

%\iffalse
%</samplepart3|samplepart4>
%\fi
%
%\iffalse
%<*samplepart3>
%\fi
% Some text for part 3:
%    \begin{macrocode}
some text in part three
%    \end{macrocode}

%\iffalse
%</samplepart3>
%\fi
% Some text for part 4:
%\iffalse
%<*samplepart4>
%\fi
%    \begin{macrocode}
more text in part four
%    \end{macrocode}

%\iffalse
%</samplepart4>
%\fi
%
% %%%%%%%%%%%%%%%%%%%%%%%%%%%%%%%%%%%%%%
% \paragraph{Forwarding for a Complete Draft.}
%
% The following forwarding file |cdocsdrf.tex|
% compiles the main document in draft mode:
%\iffalse
%<*sampledraft>
%\fi
%    \begin{macrocode}
\def\version{draft}
\input{childdoc.def}
\childdocforward{cdocsamp}
%    \end{macrocode}

%\iffalse
%</sampledraft>
%\fi
%
% %%%%%%%%%%%%%%%%%%%%%%%%%%%%%%%%%%%%%%
% \paragraph{Forwarding for Final Version of the Chapters.}
%
% The following forwarding files |cdocsfn1.tex| and |cdocsfn2.tex|
% (with identical content)
% compile the final versions of the child documents
% |cdocsch1.tex| and |cdocsch2.tex|, respectively:
%\iffalse
%<*samplefinal>
%\fi
%    \begin{macrocode}
\def\version{final}
\input{childdoc.def}
\childdocforwardprefix[cdocsamp]{cdocsfn}{cdocsch}
%    \end{macrocode}

%\iffalse
%</samplefinal>
%\fi
%
% %%%%%%%%%%%%%%%%%%%%%%%%%%%%%%%%%%%%%%
% \paragraph{Command Line Processing.}
%
% The following three command lines generate the output files
% |cdocscld|, |cdocscl1| and |cdocscl2|
% which should be identical to
% |cdocsdrf|, |cdocsch1| and |cdocsfn2|, respectively:
% \begin{center}
% \begin{tabular}{l}
% |latex -jobname cdocscld \|\\
% |  "\def\version{draft}\input{childdoc.def}\childdocforward{cdocsamp}"|\\
% |latex -jobname cdocscl1 \|\\
% |  "\input{childdoc.def}\childdocforward[cdocsamp]{cdocsch1}"|\\
% |latex -jobname cdocscl2 \|\\
% |  "\def\version{final}\input{childdoc.def}\childdocforward{cdocsch2}"|
% \end{tabular}
% \end{center}
% Note that the trailing backslash on each first line
% merely continues the input to the second line
% (for convenient cut ant paste).
% Furthermore, the command |latex| can be replaced by any
% of its alternative versions such as |pdflatex|.
%
% %%%%%%%%%%%%%%%%%%%%%%%%%%%%%%%%%%%%%%%%%%%%%%%%%%%%%%%%%%%%%%%%%%%%%%%%%%%%%%
% %%%%%%%%%%%%%%%%%%%%%%%%%%%%%%%%%%%%%%%%%%%%%%%%%%%%%%%%%%%%%%%%%%%%%%%%%%%%%%
% \section{Implementation}
%\iffalse
%<*package>
%\fi
%
% This section describes the definitions file |childdoc.def|.

% The definitions cannot be loaded using |\usepackage| or |\RequirePackage|
% which has a mechanism to prevent loading a style file more than once.
% When loading the definitions by means of |\input|
% multiple instances have to be prevented manually:
%\iffalse
%This code needs to be before the `\ProvidesFile' directive
%which is defined at the beginning of this file.
%Therefore it is also placed there and commented out here.
%</package>
%<*discard>
%\fi
%    \begin{macrocode}
\ifdefined\childdocmain\endinput\fi
%    \end{macrocode}
%\iffalse
%</discard>
%<*package>
%\fi
%
% \macro{\ifchilddoc}
% \macro{\ifchilddocmanual}
% The conditional |\ifchilddoc| tells whether a
% child (true) or main (false) document is being compiled.
% The conditional |\ifchilddocmanual| tells whether
% the |\includeonly| mechanism is used (false) or
% the selection of child files must be performed manually (true).
% The definitions initialise to false:
%    \begin{macrocode}
\newif\ifchilddoc
\newif\ifchilddocmanual
%    \end{macrocode}

% \macro{\childdocname}
% \macro{\childdocjob}
% The macro |\childdocname| stores the name of the main document
% to be compiled. The macro |\childdocjob| stores the name of
% the document on which the \LaTeX{} compiler was originally invoked.
% The content of |\jobname| cannot be compared
% to filenames specified in the source due to different catcodes.
% The following code rescans |\jobname|, stores the result
% in |\childdocname| and saves a copy in |\childdocjob|:
%    \begin{macrocode}
\edef\childdocname{\scantokens\expandafter{\jobname\noexpand}}
\let\childdocjob\childdocname
%    \end{macrocode}

% \macro{\childdocdisable}
% The macro |\childdocdisable| prevents the main file
% from being processed more than once.
% At this stage, the main document command |\childdocmain|
% is assumed to be called once again where it should do nothing.
% Any subsequent call to it should prevent
% a secondary processing of the main document
% It overwrites the forwarding commands
% |\childdocof| and |\childdocforward|
% with empty macros to prevent further inclusions of the main document:
%    \begin{macrocode}
\newcommand{\childdocdisable}
{
  \renewcommand{\childdocmain}[1]{\renewcommand{\childdocmain}[1]{\endinput}}
  \renewcommand{\childdocof}[1]{}
  \renewcommand{\childdocby}[2][]{}
  \renewcommand{\childdocforward}[2][]{}
  \renewcommand{\childdocdisable}{}
}
%    \end{macrocode}

% \macro{\childdocmain}
% The macro |\childdocmain| is to be called at the top of the main file
% with nothing or the main filename (without extension) as argument.
% First, it breaks loops.
% If the argument is not empty and does not match |\childdocname|
% (which is set by the first inclusion of |childdoc.def|),
% |\ifchilddoc| is set to true, |\includeonly| is applied to the child file
% and |\jobname| is set to the main file
% (for proper handling of |.aux| files):
%    \begin{macrocode}
\newcommand{\childdocmain}[1]
{
  \childdocdisable\childdocmain{}
  \if?#1?\else
    \begingroup
      \def\childdoctmp{#1}
      \ifx\childdoctmp\childdocname
        \def\childdoctmp{}
      \else
        \def\childdoctmp
        {
          \childdoctrue
          \includeonly{\childdocname}
          \def\childdocjob{#1}
          \def\jobname{#1}
        }
      \fi
      \expandafter
    \endgroup
    \childdoctmp
  \fi
}
%    \end{macrocode}

% \macro{\childdocof}
% The command |\childdocof| redirects
% compilation to the main file |#1|.
%    \begin{macrocode}
\newcommand{\childdocof}[1]
{
  \childdocdisable
  \childdoctrue
  \includeonly{\childdocname}
  \def\jobname{#1}
  \def\childdocjob{#1}
  \input{#1}
}
%    \end{macrocode}

% \macro{\childdocby}
% The command |\childdocby| ....
%    \begin{macrocode}
\newcommand{\childdocby}[2][]
{
  \childdocdisable
  \childdoctrue
  \childdocmanualtrue
  \if?#1?\else
    \def\jobname{#2}
  \fi
  \def\childdocjob{#2}
  \input{#2}
  \endinput
}
%    \end{macrocode}

% \macro{\childdocforward}
% The command |\childdocforward| redirects
% compilation to the main file or
% (if the optional argument is given) a child file.
% Parameters are set as if the main file
% or a child file starting with |\childdocof| was compiled.
% Then compilation is handed over to the main file:
%    \begin{macrocode}
\newcommand{\childdocforward}[2][]
{
  \begingroup
    \if?#1?
      \def\childdoctmp
      {
        \def\childdocname{#2}
        \def\childdocjob{#2}
        \def\jobname{#2}
        \input{#2}
        \endinput
      }
    \else
      \def\childdoctmp
      {
        \childdocdisable
        \def\childdocname{#2}
        \childdoctrue
        \includeonly{#2}
        \def\childdocjob{#1}
        \def\jobname{#1}
        \input{#1}
        \endinput
      }
    \fi
    \expandafter
  \endgroup
  \childdoctmp
}
%    \end{macrocode}

% \macro{\childdocforwardprefix}
% The command |\childdocforwardprefix| redirects
% compilation to the main or a child file by means of a pattern.
% The prefix |#1| in the current filename is replaced by |#2|
% and the suffix of the current filename is kept
% (it is assumed that the filename does not contain the substring `|~~~|'
% which is used as a delimiter).
% Compilation is handed over to the new file by |\childdocforward|:
%    \begin{macrocode}
\newcommand{\childdocforwardprefix}[3][]
{
  \begingroup
    \def\childdocextract #2##1~~~{\def\childdoctmp{\childdocforward[#1]{#3##1}}}
    \expandafter\childdocextract\childdocname~~~
    \expandafter
  \endgroup
  \childdoctmp
}
%    \end{macrocode}

% \macro{\childdoc}
% The deprecated macro |\childdoc| is a legacy version of |\childdocmain|:
%    \begin{macrocode}
\newcommand{\childdoc}{\childdocmain}
%    \end{macrocode}

% \macro{\childdocredirect}
% The deprecated macro |\childdocredirect| is a legacy version
% of |\childdocforward| and |\childdocforwardprefix|:
%    \begin{macrocode}
\newcommand{\childdocredirect}[2][]
{
  \begingroup
    \if?#1?
      \def\childdoctmp{\childdocforward{#2}}
    \else
      \def\childdoctmp{\childdocforwardprefix{#1}{#2}}
    \fi
    \expandafter
  \endgroup
  \childdoctmp
}
%    \end{macrocode}

%\iffalse
%</package>
%\fi
%
\endinput
|\\
|\childdocmain{|\textit{main}|}|\\
\end{tabular}
\end{center}
%
If |\jobname| does not match the argument \textit{main} of |\childdocmain|,
it is assumed that |\jobname| points to the child file to be compiled.
When using |\childdocmain| with the main file specified as argument,
it suffices to start a child file
with just |\input{|\textit{main}|}|
without loading of the package and using |\childdocof|.
If instead all processing is done
with the appropriate \textsf{childdoc} directives,
the argument of \textit{main} of |\childdocmain| can be empty.

An alternative version of the command line processing described
in \secref{sec:commandline} using the detection mechanism reads:
%
\begin{center}
|... -jobname "|\textit{target}|" "|[\textit{flags}]%
[|\def\jobname{|\textit{dest}|}|]|\input{|\textit{main}|}"|
\end{center}

%%%%%%%%%%%%%%%%%%%%%%%%%%%%%%%%%%%%%%%%%%%%%%%%%%%%%%%%%%%%%%%%%%%%%%%%%%%%%%%%
\subsection{Manual Code}
\label{sec:manual}

In case one cannot be certain whether the definitions file |childdoc.def|
is installed on the target \TeX{} distribution
and one prefers not to ship it,
it is conceivable to paste a few relevant commands into the sources.

To that end, drop all statements |% \iffalse
%
% childdoc.dtx Copyright (C) 2017-2018 Niklas Beisert
%
% This work may be distributed and/or modified under the
% conditions of the LaTeX Project Public License, either version 1.3
% of this license or (at your option) any later version.
% The latest version of this license is in
%   http://www.latex-project.org/lppl.txt
% and version 1.3 or later is part of all distributions of LaTeX
% version 2005/12/01 or later.
%
% This work has the LPPL maintenance status `maintained'.
%
% The Current Maintainer of this work is Niklas Beisert.
%
% This work consists of the files childdoc.dtx and childdoc.ins
% and the derived files childdoc.def and cdocsamp.tex with
% cdocsch1.tex, cdocsch2.tex, cdocsdrf.tex, cdocsfn1.tex, cdocsfn2.tex.
%
%<package>\ifdefined\childdocmain\endinput\fi
%<package>\ProvidesFile{childdoc.def}[2018/12/30 v2.0 child document driver]
%<samplemain>\ProvidesFile{cdocsamp.tex}[2018/12/30 v2.0 sample for childdoc]
%<*driver>
%\ProvidesFile{childdoc.drv}[2018/12/30 v2.0 childdoc reference manual file]
\PassOptionsToClass{10pt,a4paper}{article}
\documentclass{ltxdoc}

\usepackage[margin=35mm]{geometry}
\usepackage{hyperref}
\usepackage{hyperxmp}
\usepackage[usenames]{color}

\hypersetup{colorlinks=true}
\hypersetup{pdfstartview=FitH}
\hypersetup{pdfpagemode=UseNone}
\hypersetup{pdfsource={}}
\hypersetup{pdflang={en-UK}}
\hypersetup{pdfcopyright={Copyright 2017-2018 Niklas Beisert.
  This work may be distributed and/or modified under the
  conditions of the LaTeX Project Public License, either version 1.3
  of this license or (at your option) any later version.}}
\hypersetup{pdflicenseurl={http://www.latex-project.org/lppl.txt}}
\hypersetup{pdfcontactaddress={ETH Zurich, ITP, HIT K,
  Wolfgang-Pauli-Strasse 27}}
\hypersetup{pdfcontactpostcode={8093}}
\hypersetup{pdfcontactcity={Zurich}}
\hypersetup{pdfcontactcountry={Switzerland}}
\hypersetup{pdfcontactemail={nbeisert@itp.phys.ethz.ch}}
\hypersetup{pdfcontacturl={http://people.phys.ethz.ch/\xmptilde nbeisert/}}

\newcommand{\secref}[1]{\hyperref[#1]{section \ref*{#1}}}

\parskip1ex
\parindent0pt
\let\olditemize\itemize
\def\itemize{\olditemize\parskip0pt}

\begin{document}

\title{The \textsf{childdoc} Package}
\hypersetup{pdftitle={The childdoc Package}}
\author{Niklas Beisert\\[2ex]
  Institut f\"ur Theoretische Physik\\
  Eidgen\"ossische Technische Hochschule Z\"urich\\
  Wolfgang-Pauli-Strasse 27, 8093 Z\"urich, Switzerland\\[1ex]
  \href{mailto:nbeisert@itp.phys.ethz.ch}
  {\texttt{nbeisert@itp.phys.ethz.ch}}}
\hypersetup{pdfauthor={Niklas Beisert}}
\hypersetup{pdfsubject={Manual for the LaTeX2e Package childdoc}}
\date{30 December 2018, \textsf{v2.0}}
\maketitle

\begin{abstract}\noindent
\textsf{childdoc} is a \LaTeXe{} package
that enables the direct compilation
of document sections included by |\include|
to individual files.
\end{abstract}

\begingroup
\parskip0ex
\tableofcontents
\endgroup

%%%%%%%%%%%%%%%%%%%%%%%%%%%%%%%%%%%%%%%%%%%%%%%%%%%%%%%%%%%%%%%%%%%%%%%%%%%%%%%%
%%%%%%%%%%%%%%%%%%%%%%%%%%%%%%%%%%%%%%%%%%%%%%%%%%%%%%%%%%%%%%%%%%%%%%%%%%%%%%%%
\section{Introduction}

\LaTeX{} provides a mechanism to structure a large document (such as a book)
into a main file and several child files (containing the chapters)
using the |\include| command.
This mechanism is beneficial for documents
which span hundreds of pages in order to
make the source file(s) more manageable.
Moreover, compilation can be restricted to
selected child files by means of the |\includeonly| command.
The latter feature can be used to reduce the compilation time while editing
(this was significantly more useful in the earlier days of \LaTeX{})
or to generate a smaller document which is easier to navigate.
Another application of |\includeonly| is to generate
documents consisting of selected parts of the complete document.

However, there are a few drawbacks of the plain |\include| mechanism:
\begin{itemize}
\item
The child files cannot be compiled on their own,
they can only be compiled via the main file.
A naive editing environment
(such as a text editor with an option
to have the current file processed by \LaTeX)
may require one to switch to the main file before compiling;
attempting to compile the child file produces errors.
\item
The main file must be modified (each time)
to adjust the |\includeonly| command
to the present needs. This easily leaves the main file in a messy state.
\item
The generated document will always carry the filename
of the main document. This is inconvenient if
several child files are to be compiled and
to be kept for distribution.
\end{itemize}

The present package provides a simple interface
to make child files individually compilable by \LaTeX{}.
Compiling a child file then has the same effect as compiling
the main file with an |\includeonly| command
to select the appropriate child.
Moreover the generated document will carry the name of the child
rather than the main file.
This resolves all three above issues.

This feature is meant to make the editing of books,
thesis documents and lecture notes somewhat more convenient.
However, the package can also be used efficiently for
composing a series of documents (such as exercise sheets)
which are typically distributed individually.
It then assists the author in generating the individual documents
(potentially in different versions)
as well as a document containing the collected series.
Another application is in developing style files
or other kinds of included material
where compilation of the style file could redirect
to a sample or test file.

%%%%%%%%%%%%%%%%%%%%%%%%%%%%%%%%%%%%%%%%%%%%%%%%%%%%%%%%%%%%%%%%%%%%%%%%%%%%%%%%
%%%%%%%%%%%%%%%%%%%%%%%%%%%%%%%%%%%%%%%%%%%%%%%%%%%%%%%%%%%%%%%%%%%%%%%%%%%%%%%%
\section{Usage}

First of all, the package \textsf{childdoc} is \emph{not} a standard
\LaTeXe{} |.sty| style file! Therefore it needs to be invoked in
a non-standard way.

%%%%%%%%%%%%%%%%%%%%%%%%%%%%%%%%%%%%%%%%%%%%%%%%%%%%%%%%%%%%%%%%%%%%%%%%%%%%%%%%
\subsection{Included Files}
\label{sec:include}

%%%%%%%%%%%%%%%%%%%%%%%%%%%%%%%%%%%%%%%%
\DescribeMacro{\childdocmain}
To use the package, add the commands
\begin{center}
\begin{tabular}{l}
|\input{childdoc.def}|\\
|\childdocmain{}|\\
\end{tabular}
\end{center}
at the very top of the main \LaTeX{} file,
in particular \emph{before} the |\documentclass| statement!
The argument of |\childdocmain| should be left empty
(but it must be present).

%%%%%%%%%%%%%%%%%%%%%%%%%%%%%%%%%%%%%%%%
\DescribeMacro{\childdocof}
Furthermore, add the commands
\begin{center}
\begin{tabular}{l}
|\input{childdoc.def}|\\
|\childdocof{|\textit{main}|}|\\
\end{tabular}
\end{center}
at the top of every child file \textit{child}
which is included by |\include{|\textit{child}|}|
from within the main file
(or at least for those files to be compiled individually).
The argument \textit{main} must be the filename of the main file.

There are a couple of
considerations in setting up the main and child documents:

%%%%%%%%%%%%%%%%%%%%%%%%%%%%%%%%%%%%%%%%
\paragraph{Restrictions.}

Please note the following restrictions:
\begin{itemize}
\item
|\childdocmain| must be called with one argument \textit{main}
to ensure compatibility with earlier version of the package.
It must either be empty (|\childdocmain{}|)
or precisely match the filename of the main file in which it is specified.
See \secref{sec:detection} for further information.
\item
The filename \textit{main} must be specified without the |.tex| extension.
\item
The filename \textit{main} is case sensitive
(even in case-insensitive file systems)
due to internal string comparison.
\item
The argument \textit{main} should be fully expanded, it cannot be a macro.
\item
Subdirectories and special characters should be avoided in filenames.
\item
The command |\childdocmain{|\textit{main}|}| must be followed by a whitespace.
It should not be followed immediately by another command
or by a comment mark `|%|'.
This is because the \TeX{} parser reads the token immediately following
the argument of |\childdocmain| and puts it
at the beginning of every child section;
however, a white\-space is ignored.
\end{itemize}

%%%%%%%%%%%%%%%%%%%%%%%%%%%%%%%%%%%%%%%%
\paragraph{Content of Main File.}

It is advisable to place all content in the child files included by |\include|.
Any output contained in the main file will appear in all child documents
unless suppressed manually;
it cannot be suppressed automatically by the |\includeonly| directive
and thus should normally be avoided.
A method to include some content in the main file
by means of conditional processing is described in \secref{sec:conditional}.

%%%%%%%%%%%%%%%%%%%%%%%%%%%%%%%%%%%%%%%%
\paragraph{Page Numbering.}

When only a part of the document is compiled,
the appropriate numbering of pages
(as well as other status parameters)
is determined from the |.aux| files.
The latter contain information from previous passes.
However this information needs to propagate through
all intermediate child documents.
Therefore the page numbering in child documents may well
be inconsistent until the complete document is compiled at least once.

A useful (if unconventional) way to always ensure a consistent
page numbering is to restart the numbering in each child document
and denote the pages by `\textit{child}|.|\textit{page}'
where \textit{child} represents the chapter/section number of the child file.
This can be achieved by the command
|\numberwithin{page}{|\textit{child}|}|
of the \textsf{amsmath} package
where \textit{child} can be |chapter| or |section|
depending on the chosen structuring.
Alternatively, one can modify the macro |\thepage| appropriately
and reset the counter |page| at the start of each child file.

%%%%%%%%%%%%%%%%%%%%%%%%%%%%%%%%%%%%%%%%%%%%%%%%%%%%%%%%%%%%%%%%%%%%%%%%%%%%%%%%
\subsection{Conditional Processing}
\label{sec:conditional}

The package provides a mechanism to compile different versions
of a document. To customise the versions further some conditional processing
can come in handy to distinguish which version is being compiled.
The package provides two macros to describe the compilation context:

%%%%%%%%%%%%%%%%%%%%%%%%%%%%%%%%%%%%%%%%
\DescribeMacro{\ifchilddoc}
The conditional |\ifchilddoc| distinguishes between the compilation of
child documents and the main document:
%
\begin{center}
|\ifchilddoc |\textit{child-code}| |[|\||else |\textit{main-code}]| \||fi|
\end{center}

%%%%%%%%%%%%%%%%%%%%%%%%%%%%%%%%%%%%%%%%
\DescribeMacro{\childdocname}
\DescribeMacro{\childdocjob}
The macro |\childdocname| contains the filename (without extension)
of the main or child file being processed.
Note that |\childdocjob| will always contain the name of the main file.

%%%%%%%%%%%%%%%%%%%%%%%%%%%%%%%%%%%%%%%%
\paragraph{Title Page.}

Conditional processing can be used to include a title or banner page
in the main document when proper precautions are taken.
Importantly, the code in the main file should ensure that the page counter
(as well as other status parameters which are stored in the |.aux| files)
takes the same value after the conditional processing.
Otherwise the page numbers may take divergent values
depending on which part is compiled.

For example, a title page could be declared by:
%
\begin{center}
\begin{tabular}{l}
|\ifchilddoc\||else|\\
|\addtocounter{page}{-1}|\\
\textit{code for title page}\\
|\newpage|\\
|\||fi|
\end{tabular}
\end{center}
%
A banner page for the child documents can be generated by:
%
\begin{center}
\begin{tabular}{l}
|\ifchilddoc|\\
|\addtocounter{page}{-1}|\\
\textit{code for banner page}\\
|\newpage|\\
|\||fi|
\end{tabular}
\end{center}
%
Here one could write a message such as:
\begin{center}
|This is the part \childdocname{} of \childdocjob{}.|
\end{center}

%%%%%%%%%%%%%%%%%%%%%%%%%%%%%%%%%%%%%%%%%%%%%%%%%%%%%%%%%%%%%%%%%%%%%%%%%%%%%%%%
\subsection{Flags}
\label{sec:flags}

The package makes it easy to generate different versions
of the main or child documents.
To this end compilation flags can be defined
and assigned different default values.
They will be particularly useful in conjunction
with the forwarding mechanism described in \secref{sec:forward}.

For example, it may be useful to have a flag |\version|
which can be set to |draft| or |final|.
The document source will contain some conditional code
depending on the value of |\version|.
Suppose further, the flag should default to |final| for the main file
and to |draft| for child files
which is a natural assignment for editing the document.
This is achieved by placing the following code
in the preamble of the main document
(below the |\childdocmain| directive):
%
\begin{center}
\begin{tabular}{l}
|\ifchilddoc|\\
|\providecommand{\version}{draft}|\\
|\||else|\\
|\providecommand{\version}{final}|\\
|\||fi|
\end{tabular}
\end{center}
%
The definition by |\providecommand| makes sure
that previous definitions are not overwritten.
Further statements |\providecommand{\version}{...}|
can thus be added before the above code to override it.

For the main file, one might add a line
(between |\childdocmain| and the above block)
%
\begin{center}
|%\ifchilddoc\||else\providecommand{\version}{draft}\||fi|
\end{center}
%
which can be uncommented to produce a draft version.
Likewise one can add a line to the very top of a child file
(above the |\childdocof{|\textit{main}|}| directive)
%
\begin{center}
|%\providecommand{\version}{final}|
\end{center}
%
which can be uncommented to produce the final version of this child document.

%%%%%%%%%%%%%%%%%%%%%%%%%%%%%%%%%%%%%%%%%%%%%%%%%%%%%%%%%%%%%%%%%%%%%%%%%%%%%%%%
\subsection{Forwarding}
\label{sec:forward}

Different versions of the main or child documents
using compilation flags as described in \secref{sec:flags}
can be (permanently) stored in different files
for convenient compilation, viewing and distribution.
To this end, the package defines a command
to pass on compilation to a different file:

%%%%%%%%%%%%%%%%%%%%%%%%%%%%%%%%%%%%%%%%
\DescribeMacro{\childdocforward}
The command |\childdocforward| redirects processing to
another source file:
%
\begin{center}
\begin{tabular}{l}
|\input{childdoc.def}|\\
|\childdocforward[|\textit{main}|]{|\textit{dest}|}|\\
\end{tabular}
\end{center}
%
The argument \textit{dest} is the destination file
(without extension).
It should be the main file or one of the child files.
Note that further \textsf{childdoc} directives
such as |\childdocof| and |\childdocforward|
in the indicated file will be processed in this form.
The optional argument \textit{main}
passes on directly to the main file \textit{main}
while pretending to compile the child \textit{dest}.
This form behaves as if \textit{dest}
issues |\childdocof{|\textit{main}|}| right away,
and no further \textsf{childdoc} directives will be processed.

%%%%%%%%%%%%%%%%%%%%%%%%%%%%%%%%%%%%%%%%
\DescribeMacro{\...prefix}
In the alternative form |\childdocforwardprefix|,
%
\begin{center}
\begin{tabular}{l}
|\input{childdoc.def}|\\
|\childdocforwardprefix[|\textit{main}|]{|\textit{prefix}|}{|\textit{dest}|}|
\end{tabular}
\end{center}
%
the destination file is determined by a pattern
depending on the current file:
To make this work, the current file must be called
`{\textit{prefix}\hspace{0.2em}\textit{suffix}}'
with \textit{prefix} matching precisely the argument.
Processing is then passed on to the file
`{\textit{dest}\hspace{0.2em}\textit{suffix}}'.
Surely, the same effect is achieved by
directly specifying the
argument `{\textit{dest}\hspace{0.2em}\textit{suffix}}'
in the first form.
However, that requires to set up a different file
for each child. With the alternative form of the command
all these files can have exactly the same content
which simplifies setting them up and maintaining them.

For example, the following file |draft.tex|
with a compilation flag |\version| as described in \secref{sec:flags}
compiles the main document as a draft:
%
\begin{center}
\begin{tabular}{l}
|\def\version{draft}|\\
|\input{childdoc.def}|\\
|\childdocforward{|\textit{main}|}|
\end{tabular}
\end{center}
%
Likewise, the following files |final|\textit{nn}|.tex|
compile the final version of the child document
|child|\textit{nn}|.tex|:
%
\begin{center}
\begin{tabular}{l}
|\def\version{final}|\\
|\input{childdoc.def}|\\
|\childdocforwardprefix{final}{child}|
\end{tabular}
\end{center}
%

Note that when several versions of a main file and/or of each child file
are to be generated, it may be convenient to set up a |Makefile| or
shell script to automatise the process.

%%%%%%%%%%%%%%%%%%%%%%%%%%%%%%%%%%%%%%%%%%%%%%%%%%%%%%%%%%%%%%%%%%%%%%%%%%%%%%%%
\subsection{Command Line Processing}
\label{sec:commandline}

The effect of redirection files can also be achieved by invoking
the \LaTeX{} compiler with a more elaborate command line.
Most conveniently this should be done as part
of a shell script or a |Makefile|.

When using \textsf{childdoc} in the main file, the following
command lines effectively perform a redirection
(note that depending on the shell being used,
backslashes may have to be doubled: `|\|' $\to$ `|\\|'):
%
\begin{center}
|... -jobname "|\textit{target}|" |\\|"|[\textit{flags}]%
|\input{childdoc.def}\childdocforward[|\textit{main}|]{|\textit{dest}|}"|
\end{center}
%
Here \textit{target} is the name of the output file,
\textit{main} is the name of the main file
and \textit{dest} is the name of the main or child file to be processed
(all filenames without extensions).
The optional argument \textit{main} can be omitted
if \textit{main} matches \textit{dest}.
Optionally, compilation \textit{flags} can be defined via |\def| commands.
This command line makes the \TeX{} engine believe
it is compiling the file \textit{target}
whose content is specified as the latter parameter.
The provided code then forwards the processing to
\textit{main} or \textit{dest} as described in \secref{sec:forward}.

%%%%%%%%%%%%%%%%%%%%%%%%%%%%%%%%%%%%%%%%%%%%%%%%%%%%%%%%%%%%%%%%%%%%%%%%%%%%%%%%
\subsection{Include by Input}
\label{sec:input}

Including child documents by |\include| has some restrictions by design.
Most notably, the content of a child document always occupies
its own set of pages; pages cannot be shared between child documents.
Usually, this behaviour makes perfect sense
because each child document contain an essential part of the document.
However, in some situations it may be desirable to compose
a document from a collection of parts
without having mandatory page breaks between then.
For this case, the package
provides a mechanism to include parts
by |\input| which can also be processed individually.
However, by construction this mechanism
requires manual handling of the content to be output.

%%%%%%%%%%%%%%%%%%%%%%%%%%%%%%%%%%%%%%%%
\DescribeMacro{\ifchilddocmanual}
The main file should be prepared as usual, see \secref{sec:include}.
However, the document body must make a distinction
between processing of an individual part and of the main document, e.g.:
%
\begin{center}
\begin{tabular}{l}
|\ifchilddocmanual|\\
|\input{\childdocname}|\\
|\||else|\\
\textit{document body with }|\input{|\textit{part}|}|\\
|\||fi|
\end{tabular}
\end{center}
%
The conditional |\ifchilddocmanual| is true whenever
a part to be included by |\input| is being compiled,
and the name of the part is stored in |\childdocname|.

%%%%%%%%%%%%%%%%%%%%%%%%%%%%%%%%%%%%%%%%
\DescribeMacro{\childdocby}
Each part to be included by |\input| should start with:
%
\begin{center}
\begin{tabular}{l}
|\input{childdoc.def}|\\
|\childdocby{|\textit{main}|}|\\
\end{tabular}
\end{center}
%
The directive |\childdocby| is similar to |\childdocof|
described in \secref{sec:include},
but the subsequent selection of content must be done manually.
To that end, both |\ifchilddoc| and |\ifchilddocmanual|
will be true upon processing of a part,
and the name of the part is stored in |\childdocname|.
Note that |\jobname| will be set to the filename of the current part
so that each part receives an individual |.aux| file
that does not interfere with the |.aux| file(s) of the main document.
This behaviour can be altered by the alternative form
|\childdocby[*]{|\textit{main}|}| (with a non-empty optional argument)
which uses the |.aux| file of the main document
by setting |\jobname| to \textit{main}.

%%%%%%%%%%%%%%%%%%%%%%%%%%%%%%%%%%%%%%%%%%%%%%%%%%%%%%%%%%%%%%%%%%%%%%%%%%%%%%%%
\subsection{Driver Development}
\label{sec:driver}

The \textsf{childdoc} mechanism can also be use for the development
of definition files such as \LaTeX{} styles or classes.
This case differs from the above setup with multiple parts
included by |\include| in that no |\includeonly| should be invoked.
This can be achieved by starting the include file
(before |\ProvidesPackage|) with:
%
\begin{center}
\begin{tabular}{l}
|\input{childdoc.def}|\\
|\childdocforward{|\textit{main}|}|\\
\end{tabular}
\end{center}
%
or alternatively with:
%
\begin{center}
\begin{tabular}{l}
|\input{childdoc.def}|\\
|\childdocby{|\textit{main}|}|\\
\end{tabular}
\end{center}
%
Both forms have slightly different effects as described above.
The main file is prepared as usual, see \secref{sec:include}.

%%%%%%%%%%%%%%%%%%%%%%%%%%%%%%%%%%%%%%%%%%%%%%%%%%%%%%%%%%%%%%%%%%%%%%%%%%%%%%%%
\subsection{Legacy Detection}
\label{sec:detection}

The directive |\childdocmain| in the main file can detect
whether the complete document or merely a child is to be compiled
even without using the directive |\childdocof|.
This method is deprecated because it is less robust
and there is no compelling reason to use it;
it is merely provided for backward compatibility
and it may be removed in future versions.

If the detection mechanism is to be used,
it is mandatory to correctly specify
the filename of the main file as the argument of |\childdocmain|:
%
\begin{center}
\begin{tabular}{l}
|\input{childdoc.def}|\\
|\childdocmain{|\textit{main}|}|\\
\end{tabular}
\end{center}
%
If |\jobname| does not match the argument \textit{main} of |\childdocmain|,
it is assumed that |\jobname| points to the child file to be compiled.
When using |\childdocmain| with the main file specified as argument,
it suffices to start a child file
with just |\input{|\textit{main}|}|
without loading of the package and using |\childdocof|.
If instead all processing is done
with the appropriate \textsf{childdoc} directives,
the argument of \textit{main} of |\childdocmain| can be empty.

An alternative version of the command line processing described
in \secref{sec:commandline} using the detection mechanism reads:
%
\begin{center}
|... -jobname "|\textit{target}|" "|[\textit{flags}]%
[|\def\jobname{|\textit{dest}|}|]|\input{|\textit{main}|}"|
\end{center}

%%%%%%%%%%%%%%%%%%%%%%%%%%%%%%%%%%%%%%%%%%%%%%%%%%%%%%%%%%%%%%%%%%%%%%%%%%%%%%%%
\subsection{Manual Code}
\label{sec:manual}

In case one cannot be certain whether the definitions file |childdoc.def|
is installed on the target \TeX{} distribution
and one prefers not to ship it,
it is conceivable to paste a few relevant commands into the sources.

To that end, drop all statements |\input{childdoc.def}|
and perform the replacements as outlined below.
Instead of |\childdocmain{|\textit{main}|}| add the following code
to the top of the main file:
%
\begin{center}
\begin{tabular}{l}
|\||ifdefined\childdocname\endinput\||fi\newif\ifchilddoc|\\
|\edef\childdocname{\scantokens\expandafter{\jobname\noexpand}}|\\
|\def\childdocmain{|\textit{main}|}\||ifx\childdocmain\childdocname\||else|\\
|\childdoctrue\includeonly{\childdocname}\let\jobname\childdocmain\||fi|\\
\end{tabular}
\end{center}
%
Instead of |\childdocof{|\textit{main}|}| just include the main file
at the top of each child file:
%
\begin{center}
|\input{|\textit{main}|}|
\end{center}
%
A simple redirection |\childdocforward{|\textit{dest}|}| is achieved by:
%
\begin{center}
|\def\jobname{|\textit{dest}|}\input{\jobname}|
\end{center}
%
The redirection with prefix
|\childdocforwardprefix[|\textit{prefix}|]{|\textit{dest}|}|
is accomplished by:
%
\begin{center}
\begin{tabular}{l}
|{\edef\jobname{\scantokens\expandafter{\jobname\noexpand}}|\\
|\def\redirectjob |\textit{prefix}|#1~~~{\gdef\jobname{|\textit{dest}|#1}}|\\
|\expandafter\redirectjob\jobname~~~}\input{\jobname}|
\end{tabular}
\end{center}

In an alternative approach,
child documents can be compiled by a specific command line
without additional code or specific definitions:
%
\begin{center}
|... -jobname "|\textit{target}|" "|[\textit{flags}]%
|\includeonly{|\textit{dest}|}\input{|\textit{main}|}"|
\end{center}
%

%%%%%%%%%%%%%%%%%%%%%%%%%%%%%%%%%%%%%%%%%%%%%%%%%%%%%%%%%%%%%%%%%%%%%%%%%%%%%%%%
%%%%%%%%%%%%%%%%%%%%%%%%%%%%%%%%%%%%%%%%%%%%%%%%%%%%%%%%%%%%%%%%%%%%%%%%%%%%%%%%
\section{Information}

%%%%%%%%%%%%%%%%%%%%%%%%%%%%%%%%%%%%%%%%%%%%%%%%%%%%%%%%%%%%%%%%%%%%%%%%%%%%%%%%
\subsection{Copyright}

Copyright \copyright{} 2017--2018 Niklas Beisert

This work may be distributed and/or modified under the
conditions of the \LaTeX{} Project Public License, either version 1.3
of this license or (at your option) any later version.
The latest version of this license is in
  \url{http://www.latex-project.org/lppl.txt}
and version 1.3 or later is part of all distributions of \LaTeX{}
version 2005/12/01 or later.

This work has the LPPL maintenance status `maintained'.

The Current Maintainer of this work is Niklas Beisert.

This work consists of the files |README.txt|, |childdoc.ins| and |childdoc.dtx|
as well as the derived files |childdoc.def|, |cdocsamp.tex|
with |cdocsch1.tex|, |cdocsch2.tex|, |cdocspt3.tex|, |cdocspt4.tex|,
|cdocsdrf.tex|, |cdocsfn1.tex|, |cdocsfn2.tex|
as well as |childdoc.pdf|.

%%%%%%%%%%%%%%%%%%%%%%%%%%%%%%%%%%%%%%%%%%%%%%%%%%%%%%%%%%%%%%%%%%%%%%%%%%%%%%%%
\subsection{Files and Installation}

The package consists of the files:
%
\begin{center}
\begin{tabular}{ll}
    |README.txt|   & readme file \\
    |childdoc.ins| & installation file \\
    |childdoc.dtx| & source file \\
    |childdoc.def| & definition file \\
    |cdocsamp.tex| & sample main file \\
    |cdocsch1.tex| & sample include file \\
    |cdocsch2.tex| & sample include file \\
    |cdocspt3.tex| & sample part file \\
    |cdocspt4.tex| & sample part file \\
    |cdocsdrf.tex| & sample redirection file \\
    |cdocsfn1.tex| & sample redirection file \\
    |cdocsfn2.tex| & sample redirection file \\
    |childdoc.pdf| & manual
\end{tabular}
\end{center}
%
The distribution consists of the files
|README.txt|, |childdoc.ins| and |childdoc.dtx|.
%
\begin{itemize}
\item
Run (pdf)\LaTeX{} on |childdoc.dtx|
to compile the manual |childdoc.pdf| (this file).
\item
Run \LaTeX{} on |childdoc.ins| to create the definitions file |childdoc.def|
and the sample |cdocsamp.tex| with include files
|cdocsch1.tex|, |cdocsch2.tex|, |cdocspt3.tex|, |cdocspt4.tex|,
|cdocsdrf.tex|, |cdocsfn1.tex|, |cdocsfn2.tex|.
Then copy the file |childdoc.def| to an appropriate directory of your \LaTeX{}
distribution, e.g.\ \textit{texmf-root}|/tex/latex/childdoc|.
\end{itemize}

%%%%%%%%%%%%%%%%%%%%%%%%%%%%%%%%%%%%%%%%%%%%%%%%%%%%%%%%%%%%%%%%%%%%%%%%%%%%%%%%
\subsection{Related CTAN Packages}

There are several other packages which offer a similar functionality:
%
\begin{itemize}
\item
The packages
\href{http://ctan.org/pkg/docmute}{\textsf{docmute}},
\href{http://ctan.org/pkg/includex}{\textsf{includex}} and
\href{http://ctan.org/pkg/standalone}{\textsf{standalone}}
provide commands to include only the document body of
a child file thus allowing both files to be compiled individually.
\item
The packages \href{http://ctan.org/pkg/subdocs}{\textsf{subdocs}}
and \href{http://ctan.org/pkg/subfiles}{\textsf{subfiles}}
provide structures in which the main and child documents can be
encapsulated and allowing them to be compiled individually.
The inclusion mechanism is different from the conventional |\include|.
\item
The package \href{http://ctan.org/pkg/combine}{\textsf{combine}}
is an elaborate solution to combine several documents into one.
\end{itemize}
%
See also the CTAN topic \href{http://ctan.org/topic/subdocs}{\textsf{subdocs}}
for further related packages.
The present package differs from the above solutions in that
a document structure constructed with the conventional |\include| mechanism
just needs two extra commands at the top of every file
such that all constituent files can be compiled individually.

%%%%%%%%%%%%%%%%%%%%%%%%%%%%%%%%%%%%%%%%%%%%%%%%%%%%%%%%%%%%%%%%%%%%%%%%%%%%%%%%
%\subsection{Feature Suggestions}
%
%The following is a list of features which may be useful for future
%versions of this package:
%%
%\begin{itemize}
%\item
%\ldots
%\end{itemize}

%%%%%%%%%%%%%%%%%%%%%%%%%%%%%%%%%%%%%%%%%%%%%%%%%%%%%%%%%%%%%%%%%%%%%%%%%%%%%%%%
\subsection{Revision History}

%%%%%%%%%%%%%%%%%%%%%%%%%%%%%%%%%%%%%%%%
\paragraph{v2.0:} 2018/12/30

\begin{itemize}
\item
immediate forward processing
\item
added |\childdocby| mechanism
\item
manual restructured
\end{itemize}

%%%%%%%%%%%%%%%%%%%%%%%%%%%%%%%%%%%%%%%%
\paragraph{v1.6:} 2018/01/17

\begin{itemize}
\item
application for development of include files
\item
corrections to manual
\end{itemize}

%%%%%%%%%%%%%%%%%%%%%%%%%%%%%%%%%%%%%%%%
\paragraph{v1.5:} 2017/05/21

\begin{itemize}
\item
more complete structuring introduced
\item
|\childdocof| introduced
\item
|\childdoc| renamed to |\childdocmain|
\item
|\childredirect| renamed to |\childdocforward| and |\childdocforwardprefix|
and functionality expanded
\end{itemize}

%%%%%%%%%%%%%%%%%%%%%%%%%%%%%%%%%%%%%%%%
\paragraph{v1.0:} 2017/04/27

\begin{itemize}
\item
manual and install package
\item
first version published on CTAN
\end{itemize}

%%%%%%%%%%%%%%%%%%%%%%%%%%%%%%%%%%%%%%%%
\paragraph{v0.6:} 2017/04/26

\begin{itemize}
\item
redirection mechanism added
\end{itemize}

%%%%%%%%%%%%%%%%%%%%%%%%%%%%%%%%%%%%%%%%
\paragraph{v0.5:} 2017/04/26

\begin{itemize}
\item
functionality in definition file
\end{itemize}


%%%%%%%%%%%%%%%%%%%%%%%%%%%%%%%%%%%%%%%%%%%%%%%%%%%%%%%%%%%%%%%%%%%%%%%%%%%%%%%%
%%%%%%%%%%%%%%%%%%%%%%%%%%%%%%%%%%%%%%%%%%%%%%%%%%%%%%%%%%%%%%%%%%%%%%%%%%%%%%%%
%%%%%%%%%%%%%%%%%%%%%%%%%%%%%%%%%%%%%%%%%%%%%%%%%%%%%%%%%%%%%%%%%%%%%%%%%%%%%%%%
\appendix

\settowidth\MacroIndent{\rmfamily\scriptsize 000\ }

 \DocInput{childdoc.dtx}

\end{document}
%</driver>
% \fi
%
% %%%%%%%%%%%%%%%%%%%%%%%%%%%%%%%%%%%%%%%%%%%%%%%%%%%%%%%%%%%%%%%%%%%%%%%%%%%%%%
% %%%%%%%%%%%%%%%%%%%%%%%%%%%%%%%%%%%%%%%%%%%%%%%%%%%%%%%%%%%%%%%%%%%%%%%%%%%%%%
% \section{Sample}
%\iffalse
%<*samplemain>
%\fi
%
% The following presents a sample document
% with two chapters, two parts, a title page,
% a compile flag as well as three forwarding files to set the flag.
% It consists of eight |.tex| files:
% \begin{center}
% \begin{tabular}{ll}
% |cdocsamp.tex|&main file\\
% |cdocsch1.tex|&include file for chapter 1\\
% |cdocsch2.tex|&include file for chapter 2\\
% |cdocspt3.tex|&include file for part 3\\
% |cdocspt4.tex|&include file for part 4\\
% |cdocsdrf.tex|&forwarding file for main file in draft mode\\
% |cdocsfi1.tex|&forwarding file for final version of chapter 1\\
% |cdocsfi2.tex|&forwarding file for final version of chapter 2\\
% \end{tabular}
% \end{center}
% Each of the eight files can be compiled directly by the \LaTeX{} compiler.
%
% %%%%%%%%%%%%%%%%%%%%%%%%%%%%%%%%%%%%%%
% \paragraph{Main File.}
%
% The main file is called |cdocsamp.tex|.
%
% Load the \textsf{childdoc} definitions and
% declare the filename for the main document:
%    \begin{macrocode}
\input{childdoc.def}
\childdocmain{}
%    \end{macrocode}

% Optional override for |\version| flag:
%    \begin{macrocode}
%%\ifchilddoc\else\providecommand{\version}{draft}\fi
%    \end{macrocode}

% Define the default values for the |\version| flag
% (|final| for the main file and |draft| for childs):
%    \begin{macrocode}
\ifchilddoc
\providecommand{\version}{draft}
\else
\providecommand{\version}{final}
\fi
%    \end{macrocode}

% Load the standard document class:
%    \begin{macrocode}
\documentclass[12pt]{article}
%    \end{macrocode}

% Start the document body:
%    \begin{macrocode}
\begin{document}
%    \end{macrocode}

% Declare a title page.
% Print title, part of document being processed and version flag:
%    \begin{macrocode}
\addtocounter{page}{-1}
\begin{center}
{\LARGE\bfseries{}childdoc example\par}
\vspace{1cm}
\ifchilddoc
\ifchilddocmanual part\else chapter\fi:
`\childdocname' of `\childdocjob'\par
\else
main document: `\childdocjob'\par
\fi
version: \version\par
\end{center}
\newpage
%    \end{macrocode}

% Manually include selected file,
% otherwise process as usual:
%    \begin{macrocode}
\ifchilddocmanual
\section*{part `\childdocname'}
\input{\childdocname}
\else
%    \end{macrocode}

% Include the two chapters:
%    \begin{macrocode}
\include{cdocsch1}
\include{cdocsch2}
%    \end{macrocode}

% Include the two parts unless only chapters should be displayed:
%    \begin{macrocode}
\ifchilddoc\else
\section{part three}
\input{cdocspt3}
\section{part four}
\input{cdocspt4}
\fi
%    \end{macrocode}

% Process as usual until here:
%    \begin{macrocode}
\fi
%    \end{macrocode}

% End of document body:
%    \begin{macrocode}
\end{document}
%    \end{macrocode}
%\iffalse
%</samplemain>
%\fi
%
% %%%%%%%%%%%%%%%%%%%%%%%%%%%%%%%%%%%%%%
% \paragraph{Chapter Include Files.}
%
% The include files are called |cdocsch1.tex| and |cdocsch2.tex|.
%
%\iffalse
%<*samplechap1|samplechap2>
%\fi

% Optional override for |\version| flag:
%    \begin{macrocode}
%%\providecommand{\version}{final}
%    \end{macrocode}

% Include the main document:
%    \begin{macrocode}
\input{childdoc.def}
\childdocof{cdocsamp}
%    \end{macrocode}

%\iffalse
%</samplechap1|samplechap2>
%\fi
%
%\iffalse
%<*samplechap1>
%\fi
% Some text for chapter 1:
%    \begin{macrocode}
\section{one}
some text in chapter one
%    \end{macrocode}

%\iffalse
%</samplechap1>
%\fi
% Some text for chapter 2:
%\iffalse
%<*samplechap2>
%\fi
%    \begin{macrocode}
\section{two}
more text in chapter two
%    \end{macrocode}

%\iffalse
%</samplechap2>
%\fi
%
% %%%%%%%%%%%%%%%%%%%%%%%%%%%%%%%%%%%%%%
% \paragraph{Part Include Files.}
%
% The include files are called |cdocspt3.tex| and |cdocspt4.tex|.
%
%\iffalse
%<*samplepart3|samplepart4>
%\fi

% Optional override for |\version| flag:
%    \begin{macrocode}
%%\providecommand{\version}{final}
%    \end{macrocode}

% Include the main document:
%    \begin{macrocode}
\input{childdoc.def}
\childdocby{cdocsamp}
%    \end{macrocode}

%\iffalse
%</samplepart3|samplepart4>
%\fi
%
%\iffalse
%<*samplepart3>
%\fi
% Some text for part 3:
%    \begin{macrocode}
some text in part three
%    \end{macrocode}

%\iffalse
%</samplepart3>
%\fi
% Some text for part 4:
%\iffalse
%<*samplepart4>
%\fi
%    \begin{macrocode}
more text in part four
%    \end{macrocode}

%\iffalse
%</samplepart4>
%\fi
%
% %%%%%%%%%%%%%%%%%%%%%%%%%%%%%%%%%%%%%%
% \paragraph{Forwarding for a Complete Draft.}
%
% The following forwarding file |cdocsdrf.tex|
% compiles the main document in draft mode:
%\iffalse
%<*sampledraft>
%\fi
%    \begin{macrocode}
\def\version{draft}
\input{childdoc.def}
\childdocforward{cdocsamp}
%    \end{macrocode}

%\iffalse
%</sampledraft>
%\fi
%
% %%%%%%%%%%%%%%%%%%%%%%%%%%%%%%%%%%%%%%
% \paragraph{Forwarding for Final Version of the Chapters.}
%
% The following forwarding files |cdocsfn1.tex| and |cdocsfn2.tex|
% (with identical content)
% compile the final versions of the child documents
% |cdocsch1.tex| and |cdocsch2.tex|, respectively:
%\iffalse
%<*samplefinal>
%\fi
%    \begin{macrocode}
\def\version{final}
\input{childdoc.def}
\childdocforwardprefix[cdocsamp]{cdocsfn}{cdocsch}
%    \end{macrocode}

%\iffalse
%</samplefinal>
%\fi
%
% %%%%%%%%%%%%%%%%%%%%%%%%%%%%%%%%%%%%%%
% \paragraph{Command Line Processing.}
%
% The following three command lines generate the output files
% |cdocscld|, |cdocscl1| and |cdocscl2|
% which should be identical to
% |cdocsdrf|, |cdocsch1| and |cdocsfn2|, respectively:
% \begin{center}
% \begin{tabular}{l}
% |latex -jobname cdocscld \|\\
% |  "\def\version{draft}\input{childdoc.def}\childdocforward{cdocsamp}"|\\
% |latex -jobname cdocscl1 \|\\
% |  "\input{childdoc.def}\childdocforward[cdocsamp]{cdocsch1}"|\\
% |latex -jobname cdocscl2 \|\\
% |  "\def\version{final}\input{childdoc.def}\childdocforward{cdocsch2}"|
% \end{tabular}
% \end{center}
% Note that the trailing backslash on each first line
% merely continues the input to the second line
% (for convenient cut ant paste).
% Furthermore, the command |latex| can be replaced by any
% of its alternative versions such as |pdflatex|.
%
% %%%%%%%%%%%%%%%%%%%%%%%%%%%%%%%%%%%%%%%%%%%%%%%%%%%%%%%%%%%%%%%%%%%%%%%%%%%%%%
% %%%%%%%%%%%%%%%%%%%%%%%%%%%%%%%%%%%%%%%%%%%%%%%%%%%%%%%%%%%%%%%%%%%%%%%%%%%%%%
% \section{Implementation}
%\iffalse
%<*package>
%\fi
%
% This section describes the definitions file |childdoc.def|.

% The definitions cannot be loaded using |\usepackage| or |\RequirePackage|
% which has a mechanism to prevent loading a style file more than once.
% When loading the definitions by means of |\input|
% multiple instances have to be prevented manually:
%\iffalse
%This code needs to be before the `\ProvidesFile' directive
%which is defined at the beginning of this file.
%Therefore it is also placed there and commented out here.
%</package>
%<*discard>
%\fi
%    \begin{macrocode}
\ifdefined\childdocmain\endinput\fi
%    \end{macrocode}
%\iffalse
%</discard>
%<*package>
%\fi
%
% \macro{\ifchilddoc}
% \macro{\ifchilddocmanual}
% The conditional |\ifchilddoc| tells whether a
% child (true) or main (false) document is being compiled.
% The conditional |\ifchilddocmanual| tells whether
% the |\includeonly| mechanism is used (false) or
% the selection of child files must be performed manually (true).
% The definitions initialise to false:
%    \begin{macrocode}
\newif\ifchilddoc
\newif\ifchilddocmanual
%    \end{macrocode}

% \macro{\childdocname}
% \macro{\childdocjob}
% The macro |\childdocname| stores the name of the main document
% to be compiled. The macro |\childdocjob| stores the name of
% the document on which the \LaTeX{} compiler was originally invoked.
% The content of |\jobname| cannot be compared
% to filenames specified in the source due to different catcodes.
% The following code rescans |\jobname|, stores the result
% in |\childdocname| and saves a copy in |\childdocjob|:
%    \begin{macrocode}
\edef\childdocname{\scantokens\expandafter{\jobname\noexpand}}
\let\childdocjob\childdocname
%    \end{macrocode}

% \macro{\childdocdisable}
% The macro |\childdocdisable| prevents the main file
% from being processed more than once.
% At this stage, the main document command |\childdocmain|
% is assumed to be called once again where it should do nothing.
% Any subsequent call to it should prevent
% a secondary processing of the main document
% It overwrites the forwarding commands
% |\childdocof| and |\childdocforward|
% with empty macros to prevent further inclusions of the main document:
%    \begin{macrocode}
\newcommand{\childdocdisable}
{
  \renewcommand{\childdocmain}[1]{\renewcommand{\childdocmain}[1]{\endinput}}
  \renewcommand{\childdocof}[1]{}
  \renewcommand{\childdocby}[2][]{}
  \renewcommand{\childdocforward}[2][]{}
  \renewcommand{\childdocdisable}{}
}
%    \end{macrocode}

% \macro{\childdocmain}
% The macro |\childdocmain| is to be called at the top of the main file
% with nothing or the main filename (without extension) as argument.
% First, it breaks loops.
% If the argument is not empty and does not match |\childdocname|
% (which is set by the first inclusion of |childdoc.def|),
% |\ifchilddoc| is set to true, |\includeonly| is applied to the child file
% and |\jobname| is set to the main file
% (for proper handling of |.aux| files):
%    \begin{macrocode}
\newcommand{\childdocmain}[1]
{
  \childdocdisable\childdocmain{}
  \if?#1?\else
    \begingroup
      \def\childdoctmp{#1}
      \ifx\childdoctmp\childdocname
        \def\childdoctmp{}
      \else
        \def\childdoctmp
        {
          \childdoctrue
          \includeonly{\childdocname}
          \def\childdocjob{#1}
          \def\jobname{#1}
        }
      \fi
      \expandafter
    \endgroup
    \childdoctmp
  \fi
}
%    \end{macrocode}

% \macro{\childdocof}
% The command |\childdocof| redirects
% compilation to the main file |#1|.
%    \begin{macrocode}
\newcommand{\childdocof}[1]
{
  \childdocdisable
  \childdoctrue
  \includeonly{\childdocname}
  \def\jobname{#1}
  \def\childdocjob{#1}
  \input{#1}
}
%    \end{macrocode}

% \macro{\childdocby}
% The command |\childdocby| ....
%    \begin{macrocode}
\newcommand{\childdocby}[2][]
{
  \childdocdisable
  \childdoctrue
  \childdocmanualtrue
  \if?#1?\else
    \def\jobname{#2}
  \fi
  \def\childdocjob{#2}
  \input{#2}
  \endinput
}
%    \end{macrocode}

% \macro{\childdocforward}
% The command |\childdocforward| redirects
% compilation to the main file or
% (if the optional argument is given) a child file.
% Parameters are set as if the main file
% or a child file starting with |\childdocof| was compiled.
% Then compilation is handed over to the main file:
%    \begin{macrocode}
\newcommand{\childdocforward}[2][]
{
  \begingroup
    \if?#1?
      \def\childdoctmp
      {
        \def\childdocname{#2}
        \def\childdocjob{#2}
        \def\jobname{#2}
        \input{#2}
        \endinput
      }
    \else
      \def\childdoctmp
      {
        \childdocdisable
        \def\childdocname{#2}
        \childdoctrue
        \includeonly{#2}
        \def\childdocjob{#1}
        \def\jobname{#1}
        \input{#1}
        \endinput
      }
    \fi
    \expandafter
  \endgroup
  \childdoctmp
}
%    \end{macrocode}

% \macro{\childdocforwardprefix}
% The command |\childdocforwardprefix| redirects
% compilation to the main or a child file by means of a pattern.
% The prefix |#1| in the current filename is replaced by |#2|
% and the suffix of the current filename is kept
% (it is assumed that the filename does not contain the substring `|~~~|'
% which is used as a delimiter).
% Compilation is handed over to the new file by |\childdocforward|:
%    \begin{macrocode}
\newcommand{\childdocforwardprefix}[3][]
{
  \begingroup
    \def\childdocextract #2##1~~~{\def\childdoctmp{\childdocforward[#1]{#3##1}}}
    \expandafter\childdocextract\childdocname~~~
    \expandafter
  \endgroup
  \childdoctmp
}
%    \end{macrocode}

% \macro{\childdoc}
% The deprecated macro |\childdoc| is a legacy version of |\childdocmain|:
%    \begin{macrocode}
\newcommand{\childdoc}{\childdocmain}
%    \end{macrocode}

% \macro{\childdocredirect}
% The deprecated macro |\childdocredirect| is a legacy version
% of |\childdocforward| and |\childdocforwardprefix|:
%    \begin{macrocode}
\newcommand{\childdocredirect}[2][]
{
  \begingroup
    \if?#1?
      \def\childdoctmp{\childdocforward{#2}}
    \else
      \def\childdoctmp{\childdocforwardprefix{#1}{#2}}
    \fi
    \expandafter
  \endgroup
  \childdoctmp
}
%    \end{macrocode}

%\iffalse
%</package>
%\fi
%
\endinput
|
and perform the replacements as outlined below.
Instead of |\childdocmain{|\textit{main}|}| add the following code
to the top of the main file:
%
\begin{center}
\begin{tabular}{l}
|\||ifdefined\childdocname\endinput\||fi\newif\ifchilddoc|\\
|\edef\childdocname{\scantokens\expandafter{\jobname\noexpand}}|\\
|\def\childdocmain{|\textit{main}|}\||ifx\childdocmain\childdocname\||else|\\
|\childdoctrue\includeonly{\childdocname}\let\jobname\childdocmain\||fi|\\
\end{tabular}
\end{center}
%
Instead of |\childdocof{|\textit{main}|}| just include the main file
at the top of each child file:
%
\begin{center}
|\input{|\textit{main}|}|
\end{center}
%
A simple redirection |\childdocforward{|\textit{dest}|}| is achieved by:
%
\begin{center}
|\def\jobname{|\textit{dest}|}\input{\jobname}|
\end{center}
%
The redirection with prefix
|\childdocforwardprefix[|\textit{prefix}|]{|\textit{dest}|}|
is accomplished by:
%
\begin{center}
\begin{tabular}{l}
|{\edef\jobname{\scantokens\expandafter{\jobname\noexpand}}|\\
|\def\redirectjob |\textit{prefix}|#1~~~{\gdef\jobname{|\textit{dest}|#1}}|\\
|\expandafter\redirectjob\jobname~~~}\input{\jobname}|
\end{tabular}
\end{center}

In an alternative approach,
child documents can be compiled by a specific command line
without additional code or specific definitions:
%
\begin{center}
|... -jobname "|\textit{target}|" "|[\textit{flags}]%
|\includeonly{|\textit{dest}|}\input{|\textit{main}|}"|
\end{center}
%

%%%%%%%%%%%%%%%%%%%%%%%%%%%%%%%%%%%%%%%%%%%%%%%%%%%%%%%%%%%%%%%%%%%%%%%%%%%%%%%%
%%%%%%%%%%%%%%%%%%%%%%%%%%%%%%%%%%%%%%%%%%%%%%%%%%%%%%%%%%%%%%%%%%%%%%%%%%%%%%%%
\section{Information}

%%%%%%%%%%%%%%%%%%%%%%%%%%%%%%%%%%%%%%%%%%%%%%%%%%%%%%%%%%%%%%%%%%%%%%%%%%%%%%%%
\subsection{Copyright}

Copyright \copyright{} 2017--2018 Niklas Beisert

This work may be distributed and/or modified under the
conditions of the \LaTeX{} Project Public License, either version 1.3
of this license or (at your option) any later version.
The latest version of this license is in
  \url{http://www.latex-project.org/lppl.txt}
and version 1.3 or later is part of all distributions of \LaTeX{}
version 2005/12/01 or later.

This work has the LPPL maintenance status `maintained'.

The Current Maintainer of this work is Niklas Beisert.

This work consists of the files |README.txt|, |childdoc.ins| and |childdoc.dtx|
as well as the derived files |childdoc.def|, |cdocsamp.tex|
with |cdocsch1.tex|, |cdocsch2.tex|, |cdocspt3.tex|, |cdocspt4.tex|,
|cdocsdrf.tex|, |cdocsfn1.tex|, |cdocsfn2.tex|
as well as |childdoc.pdf|.

%%%%%%%%%%%%%%%%%%%%%%%%%%%%%%%%%%%%%%%%%%%%%%%%%%%%%%%%%%%%%%%%%%%%%%%%%%%%%%%%
\subsection{Files and Installation}

The package consists of the files:
%
\begin{center}
\begin{tabular}{ll}
    |README.txt|   & readme file \\
    |childdoc.ins| & installation file \\
    |childdoc.dtx| & source file \\
    |childdoc.def| & definition file \\
    |cdocsamp.tex| & sample main file \\
    |cdocsch1.tex| & sample include file \\
    |cdocsch2.tex| & sample include file \\
    |cdocspt3.tex| & sample part file \\
    |cdocspt4.tex| & sample part file \\
    |cdocsdrf.tex| & sample redirection file \\
    |cdocsfn1.tex| & sample redirection file \\
    |cdocsfn2.tex| & sample redirection file \\
    |childdoc.pdf| & manual
\end{tabular}
\end{center}
%
The distribution consists of the files
|README.txt|, |childdoc.ins| and |childdoc.dtx|.
%
\begin{itemize}
\item
Run (pdf)\LaTeX{} on |childdoc.dtx|
to compile the manual |childdoc.pdf| (this file).
\item
Run \LaTeX{} on |childdoc.ins| to create the definitions file |childdoc.def|
and the sample |cdocsamp.tex| with include files
|cdocsch1.tex|, |cdocsch2.tex|, |cdocspt3.tex|, |cdocspt4.tex|,
|cdocsdrf.tex|, |cdocsfn1.tex|, |cdocsfn2.tex|.
Then copy the file |childdoc.def| to an appropriate directory of your \LaTeX{}
distribution, e.g.\ \textit{texmf-root}|/tex/latex/childdoc|.
\end{itemize}

%%%%%%%%%%%%%%%%%%%%%%%%%%%%%%%%%%%%%%%%%%%%%%%%%%%%%%%%%%%%%%%%%%%%%%%%%%%%%%%%
\subsection{Related CTAN Packages}

There are several other packages which offer a similar functionality:
%
\begin{itemize}
\item
The packages
\href{http://ctan.org/pkg/docmute}{\textsf{docmute}},
\href{http://ctan.org/pkg/includex}{\textsf{includex}} and
\href{http://ctan.org/pkg/standalone}{\textsf{standalone}}
provide commands to include only the document body of
a child file thus allowing both files to be compiled individually.
\item
The packages \href{http://ctan.org/pkg/subdocs}{\textsf{subdocs}}
and \href{http://ctan.org/pkg/subfiles}{\textsf{subfiles}}
provide structures in which the main and child documents can be
encapsulated and allowing them to be compiled individually.
The inclusion mechanism is different from the conventional |\include|.
\item
The package \href{http://ctan.org/pkg/combine}{\textsf{combine}}
is an elaborate solution to combine several documents into one.
\end{itemize}
%
See also the CTAN topic \href{http://ctan.org/topic/subdocs}{\textsf{subdocs}}
for further related packages.
The present package differs from the above solutions in that
a document structure constructed with the conventional |\include| mechanism
just needs two extra commands at the top of every file
such that all constituent files can be compiled individually.

%%%%%%%%%%%%%%%%%%%%%%%%%%%%%%%%%%%%%%%%%%%%%%%%%%%%%%%%%%%%%%%%%%%%%%%%%%%%%%%%
%\subsection{Feature Suggestions}
%
%The following is a list of features which may be useful for future
%versions of this package:
%%
%\begin{itemize}
%\item
%\ldots
%\end{itemize}

%%%%%%%%%%%%%%%%%%%%%%%%%%%%%%%%%%%%%%%%%%%%%%%%%%%%%%%%%%%%%%%%%%%%%%%%%%%%%%%%
\subsection{Revision History}

%%%%%%%%%%%%%%%%%%%%%%%%%%%%%%%%%%%%%%%%
\paragraph{v2.0:} 2018/12/30

\begin{itemize}
\item
immediate forward processing
\item
added |\childdocby| mechanism
\item
manual restructured
\end{itemize}

%%%%%%%%%%%%%%%%%%%%%%%%%%%%%%%%%%%%%%%%
\paragraph{v1.6:} 2018/01/17

\begin{itemize}
\item
application for development of include files
\item
corrections to manual
\end{itemize}

%%%%%%%%%%%%%%%%%%%%%%%%%%%%%%%%%%%%%%%%
\paragraph{v1.5:} 2017/05/21

\begin{itemize}
\item
more complete structuring introduced
\item
|\childdocof| introduced
\item
|\childdoc| renamed to |\childdocmain|
\item
|\childredirect| renamed to |\childdocforward| and |\childdocforwardprefix|
and functionality expanded
\end{itemize}

%%%%%%%%%%%%%%%%%%%%%%%%%%%%%%%%%%%%%%%%
\paragraph{v1.0:} 2017/04/27

\begin{itemize}
\item
manual and install package
\item
first version published on CTAN
\end{itemize}

%%%%%%%%%%%%%%%%%%%%%%%%%%%%%%%%%%%%%%%%
\paragraph{v0.6:} 2017/04/26

\begin{itemize}
\item
redirection mechanism added
\end{itemize}

%%%%%%%%%%%%%%%%%%%%%%%%%%%%%%%%%%%%%%%%
\paragraph{v0.5:} 2017/04/26

\begin{itemize}
\item
functionality in definition file
\end{itemize}


%%%%%%%%%%%%%%%%%%%%%%%%%%%%%%%%%%%%%%%%%%%%%%%%%%%%%%%%%%%%%%%%%%%%%%%%%%%%%%%%
%%%%%%%%%%%%%%%%%%%%%%%%%%%%%%%%%%%%%%%%%%%%%%%%%%%%%%%%%%%%%%%%%%%%%%%%%%%%%%%%
%%%%%%%%%%%%%%%%%%%%%%%%%%%%%%%%%%%%%%%%%%%%%%%%%%%%%%%%%%%%%%%%%%%%%%%%%%%%%%%%
\appendix

\settowidth\MacroIndent{\rmfamily\scriptsize 000\ }

 \DocInput{childdoc.dtx}

\end{document}
%</driver>
% \fi
%
% %%%%%%%%%%%%%%%%%%%%%%%%%%%%%%%%%%%%%%%%%%%%%%%%%%%%%%%%%%%%%%%%%%%%%%%%%%%%%%
% %%%%%%%%%%%%%%%%%%%%%%%%%%%%%%%%%%%%%%%%%%%%%%%%%%%%%%%%%%%%%%%%%%%%%%%%%%%%%%
% \section{Sample}
%\iffalse
%<*samplemain>
%\fi
%
% The following presents a sample document
% with two chapters, two parts, a title page,
% a compile flag as well as three forwarding files to set the flag.
% It consists of eight |.tex| files:
% \begin{center}
% \begin{tabular}{ll}
% |cdocsamp.tex|&main file\\
% |cdocsch1.tex|&include file for chapter 1\\
% |cdocsch2.tex|&include file for chapter 2\\
% |cdocspt3.tex|&include file for part 3\\
% |cdocspt4.tex|&include file for part 4\\
% |cdocsdrf.tex|&forwarding file for main file in draft mode\\
% |cdocsfi1.tex|&forwarding file for final version of chapter 1\\
% |cdocsfi2.tex|&forwarding file for final version of chapter 2\\
% \end{tabular}
% \end{center}
% Each of the eight files can be compiled directly by the \LaTeX{} compiler.
%
% %%%%%%%%%%%%%%%%%%%%%%%%%%%%%%%%%%%%%%
% \paragraph{Main File.}
%
% The main file is called |cdocsamp.tex|.
%
% Load the \textsf{childdoc} definitions and
% declare the filename for the main document:
%    \begin{macrocode}
% \iffalse
%
% childdoc.dtx Copyright (C) 2017-2018 Niklas Beisert
%
% This work may be distributed and/or modified under the
% conditions of the LaTeX Project Public License, either version 1.3
% of this license or (at your option) any later version.
% The latest version of this license is in
%   http://www.latex-project.org/lppl.txt
% and version 1.3 or later is part of all distributions of LaTeX
% version 2005/12/01 or later.
%
% This work has the LPPL maintenance status `maintained'.
%
% The Current Maintainer of this work is Niklas Beisert.
%
% This work consists of the files childdoc.dtx and childdoc.ins
% and the derived files childdoc.def and cdocsamp.tex with
% cdocsch1.tex, cdocsch2.tex, cdocsdrf.tex, cdocsfn1.tex, cdocsfn2.tex.
%
%<package>\ifdefined\childdocmain\endinput\fi
%<package>\ProvidesFile{childdoc.def}[2018/12/30 v2.0 child document driver]
%<samplemain>\ProvidesFile{cdocsamp.tex}[2018/12/30 v2.0 sample for childdoc]
%<*driver>
%\ProvidesFile{childdoc.drv}[2018/12/30 v2.0 childdoc reference manual file]
\PassOptionsToClass{10pt,a4paper}{article}
\documentclass{ltxdoc}

\usepackage[margin=35mm]{geometry}
\usepackage{hyperref}
\usepackage{hyperxmp}
\usepackage[usenames]{color}

\hypersetup{colorlinks=true}
\hypersetup{pdfstartview=FitH}
\hypersetup{pdfpagemode=UseNone}
\hypersetup{pdfsource={}}
\hypersetup{pdflang={en-UK}}
\hypersetup{pdfcopyright={Copyright 2017-2018 Niklas Beisert.
  This work may be distributed and/or modified under the
  conditions of the LaTeX Project Public License, either version 1.3
  of this license or (at your option) any later version.}}
\hypersetup{pdflicenseurl={http://www.latex-project.org/lppl.txt}}
\hypersetup{pdfcontactaddress={ETH Zurich, ITP, HIT K,
  Wolfgang-Pauli-Strasse 27}}
\hypersetup{pdfcontactpostcode={8093}}
\hypersetup{pdfcontactcity={Zurich}}
\hypersetup{pdfcontactcountry={Switzerland}}
\hypersetup{pdfcontactemail={nbeisert@itp.phys.ethz.ch}}
\hypersetup{pdfcontacturl={http://people.phys.ethz.ch/\xmptilde nbeisert/}}

\newcommand{\secref}[1]{\hyperref[#1]{section \ref*{#1}}}

\parskip1ex
\parindent0pt
\let\olditemize\itemize
\def\itemize{\olditemize\parskip0pt}

\begin{document}

\title{The \textsf{childdoc} Package}
\hypersetup{pdftitle={The childdoc Package}}
\author{Niklas Beisert\\[2ex]
  Institut f\"ur Theoretische Physik\\
  Eidgen\"ossische Technische Hochschule Z\"urich\\
  Wolfgang-Pauli-Strasse 27, 8093 Z\"urich, Switzerland\\[1ex]
  \href{mailto:nbeisert@itp.phys.ethz.ch}
  {\texttt{nbeisert@itp.phys.ethz.ch}}}
\hypersetup{pdfauthor={Niklas Beisert}}
\hypersetup{pdfsubject={Manual for the LaTeX2e Package childdoc}}
\date{30 December 2018, \textsf{v2.0}}
\maketitle

\begin{abstract}\noindent
\textsf{childdoc} is a \LaTeXe{} package
that enables the direct compilation
of document sections included by |\include|
to individual files.
\end{abstract}

\begingroup
\parskip0ex
\tableofcontents
\endgroup

%%%%%%%%%%%%%%%%%%%%%%%%%%%%%%%%%%%%%%%%%%%%%%%%%%%%%%%%%%%%%%%%%%%%%%%%%%%%%%%%
%%%%%%%%%%%%%%%%%%%%%%%%%%%%%%%%%%%%%%%%%%%%%%%%%%%%%%%%%%%%%%%%%%%%%%%%%%%%%%%%
\section{Introduction}

\LaTeX{} provides a mechanism to structure a large document (such as a book)
into a main file and several child files (containing the chapters)
using the |\include| command.
This mechanism is beneficial for documents
which span hundreds of pages in order to
make the source file(s) more manageable.
Moreover, compilation can be restricted to
selected child files by means of the |\includeonly| command.
The latter feature can be used to reduce the compilation time while editing
(this was significantly more useful in the earlier days of \LaTeX{})
or to generate a smaller document which is easier to navigate.
Another application of |\includeonly| is to generate
documents consisting of selected parts of the complete document.

However, there are a few drawbacks of the plain |\include| mechanism:
\begin{itemize}
\item
The child files cannot be compiled on their own,
they can only be compiled via the main file.
A naive editing environment
(such as a text editor with an option
to have the current file processed by \LaTeX)
may require one to switch to the main file before compiling;
attempting to compile the child file produces errors.
\item
The main file must be modified (each time)
to adjust the |\includeonly| command
to the present needs. This easily leaves the main file in a messy state.
\item
The generated document will always carry the filename
of the main document. This is inconvenient if
several child files are to be compiled and
to be kept for distribution.
\end{itemize}

The present package provides a simple interface
to make child files individually compilable by \LaTeX{}.
Compiling a child file then has the same effect as compiling
the main file with an |\includeonly| command
to select the appropriate child.
Moreover the generated document will carry the name of the child
rather than the main file.
This resolves all three above issues.

This feature is meant to make the editing of books,
thesis documents and lecture notes somewhat more convenient.
However, the package can also be used efficiently for
composing a series of documents (such as exercise sheets)
which are typically distributed individually.
It then assists the author in generating the individual documents
(potentially in different versions)
as well as a document containing the collected series.
Another application is in developing style files
or other kinds of included material
where compilation of the style file could redirect
to a sample or test file.

%%%%%%%%%%%%%%%%%%%%%%%%%%%%%%%%%%%%%%%%%%%%%%%%%%%%%%%%%%%%%%%%%%%%%%%%%%%%%%%%
%%%%%%%%%%%%%%%%%%%%%%%%%%%%%%%%%%%%%%%%%%%%%%%%%%%%%%%%%%%%%%%%%%%%%%%%%%%%%%%%
\section{Usage}

First of all, the package \textsf{childdoc} is \emph{not} a standard
\LaTeXe{} |.sty| style file! Therefore it needs to be invoked in
a non-standard way.

%%%%%%%%%%%%%%%%%%%%%%%%%%%%%%%%%%%%%%%%%%%%%%%%%%%%%%%%%%%%%%%%%%%%%%%%%%%%%%%%
\subsection{Included Files}
\label{sec:include}

%%%%%%%%%%%%%%%%%%%%%%%%%%%%%%%%%%%%%%%%
\DescribeMacro{\childdocmain}
To use the package, add the commands
\begin{center}
\begin{tabular}{l}
|\input{childdoc.def}|\\
|\childdocmain{}|\\
\end{tabular}
\end{center}
at the very top of the main \LaTeX{} file,
in particular \emph{before} the |\documentclass| statement!
The argument of |\childdocmain| should be left empty
(but it must be present).

%%%%%%%%%%%%%%%%%%%%%%%%%%%%%%%%%%%%%%%%
\DescribeMacro{\childdocof}
Furthermore, add the commands
\begin{center}
\begin{tabular}{l}
|\input{childdoc.def}|\\
|\childdocof{|\textit{main}|}|\\
\end{tabular}
\end{center}
at the top of every child file \textit{child}
which is included by |\include{|\textit{child}|}|
from within the main file
(or at least for those files to be compiled individually).
The argument \textit{main} must be the filename of the main file.

There are a couple of
considerations in setting up the main and child documents:

%%%%%%%%%%%%%%%%%%%%%%%%%%%%%%%%%%%%%%%%
\paragraph{Restrictions.}

Please note the following restrictions:
\begin{itemize}
\item
|\childdocmain| must be called with one argument \textit{main}
to ensure compatibility with earlier version of the package.
It must either be empty (|\childdocmain{}|)
or precisely match the filename of the main file in which it is specified.
See \secref{sec:detection} for further information.
\item
The filename \textit{main} must be specified without the |.tex| extension.
\item
The filename \textit{main} is case sensitive
(even in case-insensitive file systems)
due to internal string comparison.
\item
The argument \textit{main} should be fully expanded, it cannot be a macro.
\item
Subdirectories and special characters should be avoided in filenames.
\item
The command |\childdocmain{|\textit{main}|}| must be followed by a whitespace.
It should not be followed immediately by another command
or by a comment mark `|%|'.
This is because the \TeX{} parser reads the token immediately following
the argument of |\childdocmain| and puts it
at the beginning of every child section;
however, a white\-space is ignored.
\end{itemize}

%%%%%%%%%%%%%%%%%%%%%%%%%%%%%%%%%%%%%%%%
\paragraph{Content of Main File.}

It is advisable to place all content in the child files included by |\include|.
Any output contained in the main file will appear in all child documents
unless suppressed manually;
it cannot be suppressed automatically by the |\includeonly| directive
and thus should normally be avoided.
A method to include some content in the main file
by means of conditional processing is described in \secref{sec:conditional}.

%%%%%%%%%%%%%%%%%%%%%%%%%%%%%%%%%%%%%%%%
\paragraph{Page Numbering.}

When only a part of the document is compiled,
the appropriate numbering of pages
(as well as other status parameters)
is determined from the |.aux| files.
The latter contain information from previous passes.
However this information needs to propagate through
all intermediate child documents.
Therefore the page numbering in child documents may well
be inconsistent until the complete document is compiled at least once.

A useful (if unconventional) way to always ensure a consistent
page numbering is to restart the numbering in each child document
and denote the pages by `\textit{child}|.|\textit{page}'
where \textit{child} represents the chapter/section number of the child file.
This can be achieved by the command
|\numberwithin{page}{|\textit{child}|}|
of the \textsf{amsmath} package
where \textit{child} can be |chapter| or |section|
depending on the chosen structuring.
Alternatively, one can modify the macro |\thepage| appropriately
and reset the counter |page| at the start of each child file.

%%%%%%%%%%%%%%%%%%%%%%%%%%%%%%%%%%%%%%%%%%%%%%%%%%%%%%%%%%%%%%%%%%%%%%%%%%%%%%%%
\subsection{Conditional Processing}
\label{sec:conditional}

The package provides a mechanism to compile different versions
of a document. To customise the versions further some conditional processing
can come in handy to distinguish which version is being compiled.
The package provides two macros to describe the compilation context:

%%%%%%%%%%%%%%%%%%%%%%%%%%%%%%%%%%%%%%%%
\DescribeMacro{\ifchilddoc}
The conditional |\ifchilddoc| distinguishes between the compilation of
child documents and the main document:
%
\begin{center}
|\ifchilddoc |\textit{child-code}| |[|\||else |\textit{main-code}]| \||fi|
\end{center}

%%%%%%%%%%%%%%%%%%%%%%%%%%%%%%%%%%%%%%%%
\DescribeMacro{\childdocname}
\DescribeMacro{\childdocjob}
The macro |\childdocname| contains the filename (without extension)
of the main or child file being processed.
Note that |\childdocjob| will always contain the name of the main file.

%%%%%%%%%%%%%%%%%%%%%%%%%%%%%%%%%%%%%%%%
\paragraph{Title Page.}

Conditional processing can be used to include a title or banner page
in the main document when proper precautions are taken.
Importantly, the code in the main file should ensure that the page counter
(as well as other status parameters which are stored in the |.aux| files)
takes the same value after the conditional processing.
Otherwise the page numbers may take divergent values
depending on which part is compiled.

For example, a title page could be declared by:
%
\begin{center}
\begin{tabular}{l}
|\ifchilddoc\||else|\\
|\addtocounter{page}{-1}|\\
\textit{code for title page}\\
|\newpage|\\
|\||fi|
\end{tabular}
\end{center}
%
A banner page for the child documents can be generated by:
%
\begin{center}
\begin{tabular}{l}
|\ifchilddoc|\\
|\addtocounter{page}{-1}|\\
\textit{code for banner page}\\
|\newpage|\\
|\||fi|
\end{tabular}
\end{center}
%
Here one could write a message such as:
\begin{center}
|This is the part \childdocname{} of \childdocjob{}.|
\end{center}

%%%%%%%%%%%%%%%%%%%%%%%%%%%%%%%%%%%%%%%%%%%%%%%%%%%%%%%%%%%%%%%%%%%%%%%%%%%%%%%%
\subsection{Flags}
\label{sec:flags}

The package makes it easy to generate different versions
of the main or child documents.
To this end compilation flags can be defined
and assigned different default values.
They will be particularly useful in conjunction
with the forwarding mechanism described in \secref{sec:forward}.

For example, it may be useful to have a flag |\version|
which can be set to |draft| or |final|.
The document source will contain some conditional code
depending on the value of |\version|.
Suppose further, the flag should default to |final| for the main file
and to |draft| for child files
which is a natural assignment for editing the document.
This is achieved by placing the following code
in the preamble of the main document
(below the |\childdocmain| directive):
%
\begin{center}
\begin{tabular}{l}
|\ifchilddoc|\\
|\providecommand{\version}{draft}|\\
|\||else|\\
|\providecommand{\version}{final}|\\
|\||fi|
\end{tabular}
\end{center}
%
The definition by |\providecommand| makes sure
that previous definitions are not overwritten.
Further statements |\providecommand{\version}{...}|
can thus be added before the above code to override it.

For the main file, one might add a line
(between |\childdocmain| and the above block)
%
\begin{center}
|%\ifchilddoc\||else\providecommand{\version}{draft}\||fi|
\end{center}
%
which can be uncommented to produce a draft version.
Likewise one can add a line to the very top of a child file
(above the |\childdocof{|\textit{main}|}| directive)
%
\begin{center}
|%\providecommand{\version}{final}|
\end{center}
%
which can be uncommented to produce the final version of this child document.

%%%%%%%%%%%%%%%%%%%%%%%%%%%%%%%%%%%%%%%%%%%%%%%%%%%%%%%%%%%%%%%%%%%%%%%%%%%%%%%%
\subsection{Forwarding}
\label{sec:forward}

Different versions of the main or child documents
using compilation flags as described in \secref{sec:flags}
can be (permanently) stored in different files
for convenient compilation, viewing and distribution.
To this end, the package defines a command
to pass on compilation to a different file:

%%%%%%%%%%%%%%%%%%%%%%%%%%%%%%%%%%%%%%%%
\DescribeMacro{\childdocforward}
The command |\childdocforward| redirects processing to
another source file:
%
\begin{center}
\begin{tabular}{l}
|\input{childdoc.def}|\\
|\childdocforward[|\textit{main}|]{|\textit{dest}|}|\\
\end{tabular}
\end{center}
%
The argument \textit{dest} is the destination file
(without extension).
It should be the main file or one of the child files.
Note that further \textsf{childdoc} directives
such as |\childdocof| and |\childdocforward|
in the indicated file will be processed in this form.
The optional argument \textit{main}
passes on directly to the main file \textit{main}
while pretending to compile the child \textit{dest}.
This form behaves as if \textit{dest}
issues |\childdocof{|\textit{main}|}| right away,
and no further \textsf{childdoc} directives will be processed.

%%%%%%%%%%%%%%%%%%%%%%%%%%%%%%%%%%%%%%%%
\DescribeMacro{\...prefix}
In the alternative form |\childdocforwardprefix|,
%
\begin{center}
\begin{tabular}{l}
|\input{childdoc.def}|\\
|\childdocforwardprefix[|\textit{main}|]{|\textit{prefix}|}{|\textit{dest}|}|
\end{tabular}
\end{center}
%
the destination file is determined by a pattern
depending on the current file:
To make this work, the current file must be called
`{\textit{prefix}\hspace{0.2em}\textit{suffix}}'
with \textit{prefix} matching precisely the argument.
Processing is then passed on to the file
`{\textit{dest}\hspace{0.2em}\textit{suffix}}'.
Surely, the same effect is achieved by
directly specifying the
argument `{\textit{dest}\hspace{0.2em}\textit{suffix}}'
in the first form.
However, that requires to set up a different file
for each child. With the alternative form of the command
all these files can have exactly the same content
which simplifies setting them up and maintaining them.

For example, the following file |draft.tex|
with a compilation flag |\version| as described in \secref{sec:flags}
compiles the main document as a draft:
%
\begin{center}
\begin{tabular}{l}
|\def\version{draft}|\\
|\input{childdoc.def}|\\
|\childdocforward{|\textit{main}|}|
\end{tabular}
\end{center}
%
Likewise, the following files |final|\textit{nn}|.tex|
compile the final version of the child document
|child|\textit{nn}|.tex|:
%
\begin{center}
\begin{tabular}{l}
|\def\version{final}|\\
|\input{childdoc.def}|\\
|\childdocforwardprefix{final}{child}|
\end{tabular}
\end{center}
%

Note that when several versions of a main file and/or of each child file
are to be generated, it may be convenient to set up a |Makefile| or
shell script to automatise the process.

%%%%%%%%%%%%%%%%%%%%%%%%%%%%%%%%%%%%%%%%%%%%%%%%%%%%%%%%%%%%%%%%%%%%%%%%%%%%%%%%
\subsection{Command Line Processing}
\label{sec:commandline}

The effect of redirection files can also be achieved by invoking
the \LaTeX{} compiler with a more elaborate command line.
Most conveniently this should be done as part
of a shell script or a |Makefile|.

When using \textsf{childdoc} in the main file, the following
command lines effectively perform a redirection
(note that depending on the shell being used,
backslashes may have to be doubled: `|\|' $\to$ `|\\|'):
%
\begin{center}
|... -jobname "|\textit{target}|" |\\|"|[\textit{flags}]%
|\input{childdoc.def}\childdocforward[|\textit{main}|]{|\textit{dest}|}"|
\end{center}
%
Here \textit{target} is the name of the output file,
\textit{main} is the name of the main file
and \textit{dest} is the name of the main or child file to be processed
(all filenames without extensions).
The optional argument \textit{main} can be omitted
if \textit{main} matches \textit{dest}.
Optionally, compilation \textit{flags} can be defined via |\def| commands.
This command line makes the \TeX{} engine believe
it is compiling the file \textit{target}
whose content is specified as the latter parameter.
The provided code then forwards the processing to
\textit{main} or \textit{dest} as described in \secref{sec:forward}.

%%%%%%%%%%%%%%%%%%%%%%%%%%%%%%%%%%%%%%%%%%%%%%%%%%%%%%%%%%%%%%%%%%%%%%%%%%%%%%%%
\subsection{Include by Input}
\label{sec:input}

Including child documents by |\include| has some restrictions by design.
Most notably, the content of a child document always occupies
its own set of pages; pages cannot be shared between child documents.
Usually, this behaviour makes perfect sense
because each child document contain an essential part of the document.
However, in some situations it may be desirable to compose
a document from a collection of parts
without having mandatory page breaks between then.
For this case, the package
provides a mechanism to include parts
by |\input| which can also be processed individually.
However, by construction this mechanism
requires manual handling of the content to be output.

%%%%%%%%%%%%%%%%%%%%%%%%%%%%%%%%%%%%%%%%
\DescribeMacro{\ifchilddocmanual}
The main file should be prepared as usual, see \secref{sec:include}.
However, the document body must make a distinction
between processing of an individual part and of the main document, e.g.:
%
\begin{center}
\begin{tabular}{l}
|\ifchilddocmanual|\\
|\input{\childdocname}|\\
|\||else|\\
\textit{document body with }|\input{|\textit{part}|}|\\
|\||fi|
\end{tabular}
\end{center}
%
The conditional |\ifchilddocmanual| is true whenever
a part to be included by |\input| is being compiled,
and the name of the part is stored in |\childdocname|.

%%%%%%%%%%%%%%%%%%%%%%%%%%%%%%%%%%%%%%%%
\DescribeMacro{\childdocby}
Each part to be included by |\input| should start with:
%
\begin{center}
\begin{tabular}{l}
|\input{childdoc.def}|\\
|\childdocby{|\textit{main}|}|\\
\end{tabular}
\end{center}
%
The directive |\childdocby| is similar to |\childdocof|
described in \secref{sec:include},
but the subsequent selection of content must be done manually.
To that end, both |\ifchilddoc| and |\ifchilddocmanual|
will be true upon processing of a part,
and the name of the part is stored in |\childdocname|.
Note that |\jobname| will be set to the filename of the current part
so that each part receives an individual |.aux| file
that does not interfere with the |.aux| file(s) of the main document.
This behaviour can be altered by the alternative form
|\childdocby[*]{|\textit{main}|}| (with a non-empty optional argument)
which uses the |.aux| file of the main document
by setting |\jobname| to \textit{main}.

%%%%%%%%%%%%%%%%%%%%%%%%%%%%%%%%%%%%%%%%%%%%%%%%%%%%%%%%%%%%%%%%%%%%%%%%%%%%%%%%
\subsection{Driver Development}
\label{sec:driver}

The \textsf{childdoc} mechanism can also be use for the development
of definition files such as \LaTeX{} styles or classes.
This case differs from the above setup with multiple parts
included by |\include| in that no |\includeonly| should be invoked.
This can be achieved by starting the include file
(before |\ProvidesPackage|) with:
%
\begin{center}
\begin{tabular}{l}
|\input{childdoc.def}|\\
|\childdocforward{|\textit{main}|}|\\
\end{tabular}
\end{center}
%
or alternatively with:
%
\begin{center}
\begin{tabular}{l}
|\input{childdoc.def}|\\
|\childdocby{|\textit{main}|}|\\
\end{tabular}
\end{center}
%
Both forms have slightly different effects as described above.
The main file is prepared as usual, see \secref{sec:include}.

%%%%%%%%%%%%%%%%%%%%%%%%%%%%%%%%%%%%%%%%%%%%%%%%%%%%%%%%%%%%%%%%%%%%%%%%%%%%%%%%
\subsection{Legacy Detection}
\label{sec:detection}

The directive |\childdocmain| in the main file can detect
whether the complete document or merely a child is to be compiled
even without using the directive |\childdocof|.
This method is deprecated because it is less robust
and there is no compelling reason to use it;
it is merely provided for backward compatibility
and it may be removed in future versions.

If the detection mechanism is to be used,
it is mandatory to correctly specify
the filename of the main file as the argument of |\childdocmain|:
%
\begin{center}
\begin{tabular}{l}
|\input{childdoc.def}|\\
|\childdocmain{|\textit{main}|}|\\
\end{tabular}
\end{center}
%
If |\jobname| does not match the argument \textit{main} of |\childdocmain|,
it is assumed that |\jobname| points to the child file to be compiled.
When using |\childdocmain| with the main file specified as argument,
it suffices to start a child file
with just |\input{|\textit{main}|}|
without loading of the package and using |\childdocof|.
If instead all processing is done
with the appropriate \textsf{childdoc} directives,
the argument of \textit{main} of |\childdocmain| can be empty.

An alternative version of the command line processing described
in \secref{sec:commandline} using the detection mechanism reads:
%
\begin{center}
|... -jobname "|\textit{target}|" "|[\textit{flags}]%
[|\def\jobname{|\textit{dest}|}|]|\input{|\textit{main}|}"|
\end{center}

%%%%%%%%%%%%%%%%%%%%%%%%%%%%%%%%%%%%%%%%%%%%%%%%%%%%%%%%%%%%%%%%%%%%%%%%%%%%%%%%
\subsection{Manual Code}
\label{sec:manual}

In case one cannot be certain whether the definitions file |childdoc.def|
is installed on the target \TeX{} distribution
and one prefers not to ship it,
it is conceivable to paste a few relevant commands into the sources.

To that end, drop all statements |\input{childdoc.def}|
and perform the replacements as outlined below.
Instead of |\childdocmain{|\textit{main}|}| add the following code
to the top of the main file:
%
\begin{center}
\begin{tabular}{l}
|\||ifdefined\childdocname\endinput\||fi\newif\ifchilddoc|\\
|\edef\childdocname{\scantokens\expandafter{\jobname\noexpand}}|\\
|\def\childdocmain{|\textit{main}|}\||ifx\childdocmain\childdocname\||else|\\
|\childdoctrue\includeonly{\childdocname}\let\jobname\childdocmain\||fi|\\
\end{tabular}
\end{center}
%
Instead of |\childdocof{|\textit{main}|}| just include the main file
at the top of each child file:
%
\begin{center}
|\input{|\textit{main}|}|
\end{center}
%
A simple redirection |\childdocforward{|\textit{dest}|}| is achieved by:
%
\begin{center}
|\def\jobname{|\textit{dest}|}\input{\jobname}|
\end{center}
%
The redirection with prefix
|\childdocforwardprefix[|\textit{prefix}|]{|\textit{dest}|}|
is accomplished by:
%
\begin{center}
\begin{tabular}{l}
|{\edef\jobname{\scantokens\expandafter{\jobname\noexpand}}|\\
|\def\redirectjob |\textit{prefix}|#1~~~{\gdef\jobname{|\textit{dest}|#1}}|\\
|\expandafter\redirectjob\jobname~~~}\input{\jobname}|
\end{tabular}
\end{center}

In an alternative approach,
child documents can be compiled by a specific command line
without additional code or specific definitions:
%
\begin{center}
|... -jobname "|\textit{target}|" "|[\textit{flags}]%
|\includeonly{|\textit{dest}|}\input{|\textit{main}|}"|
\end{center}
%

%%%%%%%%%%%%%%%%%%%%%%%%%%%%%%%%%%%%%%%%%%%%%%%%%%%%%%%%%%%%%%%%%%%%%%%%%%%%%%%%
%%%%%%%%%%%%%%%%%%%%%%%%%%%%%%%%%%%%%%%%%%%%%%%%%%%%%%%%%%%%%%%%%%%%%%%%%%%%%%%%
\section{Information}

%%%%%%%%%%%%%%%%%%%%%%%%%%%%%%%%%%%%%%%%%%%%%%%%%%%%%%%%%%%%%%%%%%%%%%%%%%%%%%%%
\subsection{Copyright}

Copyright \copyright{} 2017--2018 Niklas Beisert

This work may be distributed and/or modified under the
conditions of the \LaTeX{} Project Public License, either version 1.3
of this license or (at your option) any later version.
The latest version of this license is in
  \url{http://www.latex-project.org/lppl.txt}
and version 1.3 or later is part of all distributions of \LaTeX{}
version 2005/12/01 or later.

This work has the LPPL maintenance status `maintained'.

The Current Maintainer of this work is Niklas Beisert.

This work consists of the files |README.txt|, |childdoc.ins| and |childdoc.dtx|
as well as the derived files |childdoc.def|, |cdocsamp.tex|
with |cdocsch1.tex|, |cdocsch2.tex|, |cdocspt3.tex|, |cdocspt4.tex|,
|cdocsdrf.tex|, |cdocsfn1.tex|, |cdocsfn2.tex|
as well as |childdoc.pdf|.

%%%%%%%%%%%%%%%%%%%%%%%%%%%%%%%%%%%%%%%%%%%%%%%%%%%%%%%%%%%%%%%%%%%%%%%%%%%%%%%%
\subsection{Files and Installation}

The package consists of the files:
%
\begin{center}
\begin{tabular}{ll}
    |README.txt|   & readme file \\
    |childdoc.ins| & installation file \\
    |childdoc.dtx| & source file \\
    |childdoc.def| & definition file \\
    |cdocsamp.tex| & sample main file \\
    |cdocsch1.tex| & sample include file \\
    |cdocsch2.tex| & sample include file \\
    |cdocspt3.tex| & sample part file \\
    |cdocspt4.tex| & sample part file \\
    |cdocsdrf.tex| & sample redirection file \\
    |cdocsfn1.tex| & sample redirection file \\
    |cdocsfn2.tex| & sample redirection file \\
    |childdoc.pdf| & manual
\end{tabular}
\end{center}
%
The distribution consists of the files
|README.txt|, |childdoc.ins| and |childdoc.dtx|.
%
\begin{itemize}
\item
Run (pdf)\LaTeX{} on |childdoc.dtx|
to compile the manual |childdoc.pdf| (this file).
\item
Run \LaTeX{} on |childdoc.ins| to create the definitions file |childdoc.def|
and the sample |cdocsamp.tex| with include files
|cdocsch1.tex|, |cdocsch2.tex|, |cdocspt3.tex|, |cdocspt4.tex|,
|cdocsdrf.tex|, |cdocsfn1.tex|, |cdocsfn2.tex|.
Then copy the file |childdoc.def| to an appropriate directory of your \LaTeX{}
distribution, e.g.\ \textit{texmf-root}|/tex/latex/childdoc|.
\end{itemize}

%%%%%%%%%%%%%%%%%%%%%%%%%%%%%%%%%%%%%%%%%%%%%%%%%%%%%%%%%%%%%%%%%%%%%%%%%%%%%%%%
\subsection{Related CTAN Packages}

There are several other packages which offer a similar functionality:
%
\begin{itemize}
\item
The packages
\href{http://ctan.org/pkg/docmute}{\textsf{docmute}},
\href{http://ctan.org/pkg/includex}{\textsf{includex}} and
\href{http://ctan.org/pkg/standalone}{\textsf{standalone}}
provide commands to include only the document body of
a child file thus allowing both files to be compiled individually.
\item
The packages \href{http://ctan.org/pkg/subdocs}{\textsf{subdocs}}
and \href{http://ctan.org/pkg/subfiles}{\textsf{subfiles}}
provide structures in which the main and child documents can be
encapsulated and allowing them to be compiled individually.
The inclusion mechanism is different from the conventional |\include|.
\item
The package \href{http://ctan.org/pkg/combine}{\textsf{combine}}
is an elaborate solution to combine several documents into one.
\end{itemize}
%
See also the CTAN topic \href{http://ctan.org/topic/subdocs}{\textsf{subdocs}}
for further related packages.
The present package differs from the above solutions in that
a document structure constructed with the conventional |\include| mechanism
just needs two extra commands at the top of every file
such that all constituent files can be compiled individually.

%%%%%%%%%%%%%%%%%%%%%%%%%%%%%%%%%%%%%%%%%%%%%%%%%%%%%%%%%%%%%%%%%%%%%%%%%%%%%%%%
%\subsection{Feature Suggestions}
%
%The following is a list of features which may be useful for future
%versions of this package:
%%
%\begin{itemize}
%\item
%\ldots
%\end{itemize}

%%%%%%%%%%%%%%%%%%%%%%%%%%%%%%%%%%%%%%%%%%%%%%%%%%%%%%%%%%%%%%%%%%%%%%%%%%%%%%%%
\subsection{Revision History}

%%%%%%%%%%%%%%%%%%%%%%%%%%%%%%%%%%%%%%%%
\paragraph{v2.0:} 2018/12/30

\begin{itemize}
\item
immediate forward processing
\item
added |\childdocby| mechanism
\item
manual restructured
\end{itemize}

%%%%%%%%%%%%%%%%%%%%%%%%%%%%%%%%%%%%%%%%
\paragraph{v1.6:} 2018/01/17

\begin{itemize}
\item
application for development of include files
\item
corrections to manual
\end{itemize}

%%%%%%%%%%%%%%%%%%%%%%%%%%%%%%%%%%%%%%%%
\paragraph{v1.5:} 2017/05/21

\begin{itemize}
\item
more complete structuring introduced
\item
|\childdocof| introduced
\item
|\childdoc| renamed to |\childdocmain|
\item
|\childredirect| renamed to |\childdocforward| and |\childdocforwardprefix|
and functionality expanded
\end{itemize}

%%%%%%%%%%%%%%%%%%%%%%%%%%%%%%%%%%%%%%%%
\paragraph{v1.0:} 2017/04/27

\begin{itemize}
\item
manual and install package
\item
first version published on CTAN
\end{itemize}

%%%%%%%%%%%%%%%%%%%%%%%%%%%%%%%%%%%%%%%%
\paragraph{v0.6:} 2017/04/26

\begin{itemize}
\item
redirection mechanism added
\end{itemize}

%%%%%%%%%%%%%%%%%%%%%%%%%%%%%%%%%%%%%%%%
\paragraph{v0.5:} 2017/04/26

\begin{itemize}
\item
functionality in definition file
\end{itemize}


%%%%%%%%%%%%%%%%%%%%%%%%%%%%%%%%%%%%%%%%%%%%%%%%%%%%%%%%%%%%%%%%%%%%%%%%%%%%%%%%
%%%%%%%%%%%%%%%%%%%%%%%%%%%%%%%%%%%%%%%%%%%%%%%%%%%%%%%%%%%%%%%%%%%%%%%%%%%%%%%%
%%%%%%%%%%%%%%%%%%%%%%%%%%%%%%%%%%%%%%%%%%%%%%%%%%%%%%%%%%%%%%%%%%%%%%%%%%%%%%%%
\appendix

\settowidth\MacroIndent{\rmfamily\scriptsize 000\ }

 \DocInput{childdoc.dtx}

\end{document}
%</driver>
% \fi
%
% %%%%%%%%%%%%%%%%%%%%%%%%%%%%%%%%%%%%%%%%%%%%%%%%%%%%%%%%%%%%%%%%%%%%%%%%%%%%%%
% %%%%%%%%%%%%%%%%%%%%%%%%%%%%%%%%%%%%%%%%%%%%%%%%%%%%%%%%%%%%%%%%%%%%%%%%%%%%%%
% \section{Sample}
%\iffalse
%<*samplemain>
%\fi
%
% The following presents a sample document
% with two chapters, two parts, a title page,
% a compile flag as well as three forwarding files to set the flag.
% It consists of eight |.tex| files:
% \begin{center}
% \begin{tabular}{ll}
% |cdocsamp.tex|&main file\\
% |cdocsch1.tex|&include file for chapter 1\\
% |cdocsch2.tex|&include file for chapter 2\\
% |cdocspt3.tex|&include file for part 3\\
% |cdocspt4.tex|&include file for part 4\\
% |cdocsdrf.tex|&forwarding file for main file in draft mode\\
% |cdocsfi1.tex|&forwarding file for final version of chapter 1\\
% |cdocsfi2.tex|&forwarding file for final version of chapter 2\\
% \end{tabular}
% \end{center}
% Each of the eight files can be compiled directly by the \LaTeX{} compiler.
%
% %%%%%%%%%%%%%%%%%%%%%%%%%%%%%%%%%%%%%%
% \paragraph{Main File.}
%
% The main file is called |cdocsamp.tex|.
%
% Load the \textsf{childdoc} definitions and
% declare the filename for the main document:
%    \begin{macrocode}
\input{childdoc.def}
\childdocmain{}
%    \end{macrocode}

% Optional override for |\version| flag:
%    \begin{macrocode}
%%\ifchilddoc\else\providecommand{\version}{draft}\fi
%    \end{macrocode}

% Define the default values for the |\version| flag
% (|final| for the main file and |draft| for childs):
%    \begin{macrocode}
\ifchilddoc
\providecommand{\version}{draft}
\else
\providecommand{\version}{final}
\fi
%    \end{macrocode}

% Load the standard document class:
%    \begin{macrocode}
\documentclass[12pt]{article}
%    \end{macrocode}

% Start the document body:
%    \begin{macrocode}
\begin{document}
%    \end{macrocode}

% Declare a title page.
% Print title, part of document being processed and version flag:
%    \begin{macrocode}
\addtocounter{page}{-1}
\begin{center}
{\LARGE\bfseries{}childdoc example\par}
\vspace{1cm}
\ifchilddoc
\ifchilddocmanual part\else chapter\fi:
`\childdocname' of `\childdocjob'\par
\else
main document: `\childdocjob'\par
\fi
version: \version\par
\end{center}
\newpage
%    \end{macrocode}

% Manually include selected file,
% otherwise process as usual:
%    \begin{macrocode}
\ifchilddocmanual
\section*{part `\childdocname'}
\input{\childdocname}
\else
%    \end{macrocode}

% Include the two chapters:
%    \begin{macrocode}
\include{cdocsch1}
\include{cdocsch2}
%    \end{macrocode}

% Include the two parts unless only chapters should be displayed:
%    \begin{macrocode}
\ifchilddoc\else
\section{part three}
\input{cdocspt3}
\section{part four}
\input{cdocspt4}
\fi
%    \end{macrocode}

% Process as usual until here:
%    \begin{macrocode}
\fi
%    \end{macrocode}

% End of document body:
%    \begin{macrocode}
\end{document}
%    \end{macrocode}
%\iffalse
%</samplemain>
%\fi
%
% %%%%%%%%%%%%%%%%%%%%%%%%%%%%%%%%%%%%%%
% \paragraph{Chapter Include Files.}
%
% The include files are called |cdocsch1.tex| and |cdocsch2.tex|.
%
%\iffalse
%<*samplechap1|samplechap2>
%\fi

% Optional override for |\version| flag:
%    \begin{macrocode}
%%\providecommand{\version}{final}
%    \end{macrocode}

% Include the main document:
%    \begin{macrocode}
\input{childdoc.def}
\childdocof{cdocsamp}
%    \end{macrocode}

%\iffalse
%</samplechap1|samplechap2>
%\fi
%
%\iffalse
%<*samplechap1>
%\fi
% Some text for chapter 1:
%    \begin{macrocode}
\section{one}
some text in chapter one
%    \end{macrocode}

%\iffalse
%</samplechap1>
%\fi
% Some text for chapter 2:
%\iffalse
%<*samplechap2>
%\fi
%    \begin{macrocode}
\section{two}
more text in chapter two
%    \end{macrocode}

%\iffalse
%</samplechap2>
%\fi
%
% %%%%%%%%%%%%%%%%%%%%%%%%%%%%%%%%%%%%%%
% \paragraph{Part Include Files.}
%
% The include files are called |cdocspt3.tex| and |cdocspt4.tex|.
%
%\iffalse
%<*samplepart3|samplepart4>
%\fi

% Optional override for |\version| flag:
%    \begin{macrocode}
%%\providecommand{\version}{final}
%    \end{macrocode}

% Include the main document:
%    \begin{macrocode}
\input{childdoc.def}
\childdocby{cdocsamp}
%    \end{macrocode}

%\iffalse
%</samplepart3|samplepart4>
%\fi
%
%\iffalse
%<*samplepart3>
%\fi
% Some text for part 3:
%    \begin{macrocode}
some text in part three
%    \end{macrocode}

%\iffalse
%</samplepart3>
%\fi
% Some text for part 4:
%\iffalse
%<*samplepart4>
%\fi
%    \begin{macrocode}
more text in part four
%    \end{macrocode}

%\iffalse
%</samplepart4>
%\fi
%
% %%%%%%%%%%%%%%%%%%%%%%%%%%%%%%%%%%%%%%
% \paragraph{Forwarding for a Complete Draft.}
%
% The following forwarding file |cdocsdrf.tex|
% compiles the main document in draft mode:
%\iffalse
%<*sampledraft>
%\fi
%    \begin{macrocode}
\def\version{draft}
\input{childdoc.def}
\childdocforward{cdocsamp}
%    \end{macrocode}

%\iffalse
%</sampledraft>
%\fi
%
% %%%%%%%%%%%%%%%%%%%%%%%%%%%%%%%%%%%%%%
% \paragraph{Forwarding for Final Version of the Chapters.}
%
% The following forwarding files |cdocsfn1.tex| and |cdocsfn2.tex|
% (with identical content)
% compile the final versions of the child documents
% |cdocsch1.tex| and |cdocsch2.tex|, respectively:
%\iffalse
%<*samplefinal>
%\fi
%    \begin{macrocode}
\def\version{final}
\input{childdoc.def}
\childdocforwardprefix[cdocsamp]{cdocsfn}{cdocsch}
%    \end{macrocode}

%\iffalse
%</samplefinal>
%\fi
%
% %%%%%%%%%%%%%%%%%%%%%%%%%%%%%%%%%%%%%%
% \paragraph{Command Line Processing.}
%
% The following three command lines generate the output files
% |cdocscld|, |cdocscl1| and |cdocscl2|
% which should be identical to
% |cdocsdrf|, |cdocsch1| and |cdocsfn2|, respectively:
% \begin{center}
% \begin{tabular}{l}
% |latex -jobname cdocscld \|\\
% |  "\def\version{draft}\input{childdoc.def}\childdocforward{cdocsamp}"|\\
% |latex -jobname cdocscl1 \|\\
% |  "\input{childdoc.def}\childdocforward[cdocsamp]{cdocsch1}"|\\
% |latex -jobname cdocscl2 \|\\
% |  "\def\version{final}\input{childdoc.def}\childdocforward{cdocsch2}"|
% \end{tabular}
% \end{center}
% Note that the trailing backslash on each first line
% merely continues the input to the second line
% (for convenient cut ant paste).
% Furthermore, the command |latex| can be replaced by any
% of its alternative versions such as |pdflatex|.
%
% %%%%%%%%%%%%%%%%%%%%%%%%%%%%%%%%%%%%%%%%%%%%%%%%%%%%%%%%%%%%%%%%%%%%%%%%%%%%%%
% %%%%%%%%%%%%%%%%%%%%%%%%%%%%%%%%%%%%%%%%%%%%%%%%%%%%%%%%%%%%%%%%%%%%%%%%%%%%%%
% \section{Implementation}
%\iffalse
%<*package>
%\fi
%
% This section describes the definitions file |childdoc.def|.

% The definitions cannot be loaded using |\usepackage| or |\RequirePackage|
% which has a mechanism to prevent loading a style file more than once.
% When loading the definitions by means of |\input|
% multiple instances have to be prevented manually:
%\iffalse
%This code needs to be before the `\ProvidesFile' directive
%which is defined at the beginning of this file.
%Therefore it is also placed there and commented out here.
%</package>
%<*discard>
%\fi
%    \begin{macrocode}
\ifdefined\childdocmain\endinput\fi
%    \end{macrocode}
%\iffalse
%</discard>
%<*package>
%\fi
%
% \macro{\ifchilddoc}
% \macro{\ifchilddocmanual}
% The conditional |\ifchilddoc| tells whether a
% child (true) or main (false) document is being compiled.
% The conditional |\ifchilddocmanual| tells whether
% the |\includeonly| mechanism is used (false) or
% the selection of child files must be performed manually (true).
% The definitions initialise to false:
%    \begin{macrocode}
\newif\ifchilddoc
\newif\ifchilddocmanual
%    \end{macrocode}

% \macro{\childdocname}
% \macro{\childdocjob}
% The macro |\childdocname| stores the name of the main document
% to be compiled. The macro |\childdocjob| stores the name of
% the document on which the \LaTeX{} compiler was originally invoked.
% The content of |\jobname| cannot be compared
% to filenames specified in the source due to different catcodes.
% The following code rescans |\jobname|, stores the result
% in |\childdocname| and saves a copy in |\childdocjob|:
%    \begin{macrocode}
\edef\childdocname{\scantokens\expandafter{\jobname\noexpand}}
\let\childdocjob\childdocname
%    \end{macrocode}

% \macro{\childdocdisable}
% The macro |\childdocdisable| prevents the main file
% from being processed more than once.
% At this stage, the main document command |\childdocmain|
% is assumed to be called once again where it should do nothing.
% Any subsequent call to it should prevent
% a secondary processing of the main document
% It overwrites the forwarding commands
% |\childdocof| and |\childdocforward|
% with empty macros to prevent further inclusions of the main document:
%    \begin{macrocode}
\newcommand{\childdocdisable}
{
  \renewcommand{\childdocmain}[1]{\renewcommand{\childdocmain}[1]{\endinput}}
  \renewcommand{\childdocof}[1]{}
  \renewcommand{\childdocby}[2][]{}
  \renewcommand{\childdocforward}[2][]{}
  \renewcommand{\childdocdisable}{}
}
%    \end{macrocode}

% \macro{\childdocmain}
% The macro |\childdocmain| is to be called at the top of the main file
% with nothing or the main filename (without extension) as argument.
% First, it breaks loops.
% If the argument is not empty and does not match |\childdocname|
% (which is set by the first inclusion of |childdoc.def|),
% |\ifchilddoc| is set to true, |\includeonly| is applied to the child file
% and |\jobname| is set to the main file
% (for proper handling of |.aux| files):
%    \begin{macrocode}
\newcommand{\childdocmain}[1]
{
  \childdocdisable\childdocmain{}
  \if?#1?\else
    \begingroup
      \def\childdoctmp{#1}
      \ifx\childdoctmp\childdocname
        \def\childdoctmp{}
      \else
        \def\childdoctmp
        {
          \childdoctrue
          \includeonly{\childdocname}
          \def\childdocjob{#1}
          \def\jobname{#1}
        }
      \fi
      \expandafter
    \endgroup
    \childdoctmp
  \fi
}
%    \end{macrocode}

% \macro{\childdocof}
% The command |\childdocof| redirects
% compilation to the main file |#1|.
%    \begin{macrocode}
\newcommand{\childdocof}[1]
{
  \childdocdisable
  \childdoctrue
  \includeonly{\childdocname}
  \def\jobname{#1}
  \def\childdocjob{#1}
  \input{#1}
}
%    \end{macrocode}

% \macro{\childdocby}
% The command |\childdocby| ....
%    \begin{macrocode}
\newcommand{\childdocby}[2][]
{
  \childdocdisable
  \childdoctrue
  \childdocmanualtrue
  \if?#1?\else
    \def\jobname{#2}
  \fi
  \def\childdocjob{#2}
  \input{#2}
  \endinput
}
%    \end{macrocode}

% \macro{\childdocforward}
% The command |\childdocforward| redirects
% compilation to the main file or
% (if the optional argument is given) a child file.
% Parameters are set as if the main file
% or a child file starting with |\childdocof| was compiled.
% Then compilation is handed over to the main file:
%    \begin{macrocode}
\newcommand{\childdocforward}[2][]
{
  \begingroup
    \if?#1?
      \def\childdoctmp
      {
        \def\childdocname{#2}
        \def\childdocjob{#2}
        \def\jobname{#2}
        \input{#2}
        \endinput
      }
    \else
      \def\childdoctmp
      {
        \childdocdisable
        \def\childdocname{#2}
        \childdoctrue
        \includeonly{#2}
        \def\childdocjob{#1}
        \def\jobname{#1}
        \input{#1}
        \endinput
      }
    \fi
    \expandafter
  \endgroup
  \childdoctmp
}
%    \end{macrocode}

% \macro{\childdocforwardprefix}
% The command |\childdocforwardprefix| redirects
% compilation to the main or a child file by means of a pattern.
% The prefix |#1| in the current filename is replaced by |#2|
% and the suffix of the current filename is kept
% (it is assumed that the filename does not contain the substring `|~~~|'
% which is used as a delimiter).
% Compilation is handed over to the new file by |\childdocforward|:
%    \begin{macrocode}
\newcommand{\childdocforwardprefix}[3][]
{
  \begingroup
    \def\childdocextract #2##1~~~{\def\childdoctmp{\childdocforward[#1]{#3##1}}}
    \expandafter\childdocextract\childdocname~~~
    \expandafter
  \endgroup
  \childdoctmp
}
%    \end{macrocode}

% \macro{\childdoc}
% The deprecated macro |\childdoc| is a legacy version of |\childdocmain|:
%    \begin{macrocode}
\newcommand{\childdoc}{\childdocmain}
%    \end{macrocode}

% \macro{\childdocredirect}
% The deprecated macro |\childdocredirect| is a legacy version
% of |\childdocforward| and |\childdocforwardprefix|:
%    \begin{macrocode}
\newcommand{\childdocredirect}[2][]
{
  \begingroup
    \if?#1?
      \def\childdoctmp{\childdocforward{#2}}
    \else
      \def\childdoctmp{\childdocforwardprefix{#1}{#2}}
    \fi
    \expandafter
  \endgroup
  \childdoctmp
}
%    \end{macrocode}

%\iffalse
%</package>
%\fi
%
\endinput

\childdocmain{}
%    \end{macrocode}

% Optional override for |\version| flag:
%    \begin{macrocode}
%%\ifchilddoc\else\providecommand{\version}{draft}\fi
%    \end{macrocode}

% Define the default values for the |\version| flag
% (|final| for the main file and |draft| for childs):
%    \begin{macrocode}
\ifchilddoc
\providecommand{\version}{draft}
\else
\providecommand{\version}{final}
\fi
%    \end{macrocode}

% Load the standard document class:
%    \begin{macrocode}
\documentclass[12pt]{article}
%    \end{macrocode}

% Start the document body:
%    \begin{macrocode}
\begin{document}
%    \end{macrocode}

% Declare a title page.
% Print title, part of document being processed and version flag:
%    \begin{macrocode}
\addtocounter{page}{-1}
\begin{center}
{\LARGE\bfseries{}childdoc example\par}
\vspace{1cm}
\ifchilddoc
\ifchilddocmanual part\else chapter\fi:
`\childdocname' of `\childdocjob'\par
\else
main document: `\childdocjob'\par
\fi
version: \version\par
\end{center}
\newpage
%    \end{macrocode}

% Manually include selected file,
% otherwise process as usual:
%    \begin{macrocode}
\ifchilddocmanual
\section*{part `\childdocname'}
\input{\childdocname}
\else
%    \end{macrocode}

% Include the two chapters:
%    \begin{macrocode}
\include{cdocsch1}
\include{cdocsch2}
%    \end{macrocode}

% Include the two parts unless only chapters should be displayed:
%    \begin{macrocode}
\ifchilddoc\else
\section{part three}
\input{cdocspt3}
\section{part four}
\input{cdocspt4}
\fi
%    \end{macrocode}

% Process as usual until here:
%    \begin{macrocode}
\fi
%    \end{macrocode}

% End of document body:
%    \begin{macrocode}
\end{document}
%    \end{macrocode}
%\iffalse
%</samplemain>
%\fi
%
% %%%%%%%%%%%%%%%%%%%%%%%%%%%%%%%%%%%%%%
% \paragraph{Chapter Include Files.}
%
% The include files are called |cdocsch1.tex| and |cdocsch2.tex|.
%
%\iffalse
%<*samplechap1|samplechap2>
%\fi

% Optional override for |\version| flag:
%    \begin{macrocode}
%%\providecommand{\version}{final}
%    \end{macrocode}

% Include the main document:
%    \begin{macrocode}
% \iffalse
%
% childdoc.dtx Copyright (C) 2017-2018 Niklas Beisert
%
% This work may be distributed and/or modified under the
% conditions of the LaTeX Project Public License, either version 1.3
% of this license or (at your option) any later version.
% The latest version of this license is in
%   http://www.latex-project.org/lppl.txt
% and version 1.3 or later is part of all distributions of LaTeX
% version 2005/12/01 or later.
%
% This work has the LPPL maintenance status `maintained'.
%
% The Current Maintainer of this work is Niklas Beisert.
%
% This work consists of the files childdoc.dtx and childdoc.ins
% and the derived files childdoc.def and cdocsamp.tex with
% cdocsch1.tex, cdocsch2.tex, cdocsdrf.tex, cdocsfn1.tex, cdocsfn2.tex.
%
%<package>\ifdefined\childdocmain\endinput\fi
%<package>\ProvidesFile{childdoc.def}[2018/12/30 v2.0 child document driver]
%<samplemain>\ProvidesFile{cdocsamp.tex}[2018/12/30 v2.0 sample for childdoc]
%<*driver>
%\ProvidesFile{childdoc.drv}[2018/12/30 v2.0 childdoc reference manual file]
\PassOptionsToClass{10pt,a4paper}{article}
\documentclass{ltxdoc}

\usepackage[margin=35mm]{geometry}
\usepackage{hyperref}
\usepackage{hyperxmp}
\usepackage[usenames]{color}

\hypersetup{colorlinks=true}
\hypersetup{pdfstartview=FitH}
\hypersetup{pdfpagemode=UseNone}
\hypersetup{pdfsource={}}
\hypersetup{pdflang={en-UK}}
\hypersetup{pdfcopyright={Copyright 2017-2018 Niklas Beisert.
  This work may be distributed and/or modified under the
  conditions of the LaTeX Project Public License, either version 1.3
  of this license or (at your option) any later version.}}
\hypersetup{pdflicenseurl={http://www.latex-project.org/lppl.txt}}
\hypersetup{pdfcontactaddress={ETH Zurich, ITP, HIT K,
  Wolfgang-Pauli-Strasse 27}}
\hypersetup{pdfcontactpostcode={8093}}
\hypersetup{pdfcontactcity={Zurich}}
\hypersetup{pdfcontactcountry={Switzerland}}
\hypersetup{pdfcontactemail={nbeisert@itp.phys.ethz.ch}}
\hypersetup{pdfcontacturl={http://people.phys.ethz.ch/\xmptilde nbeisert/}}

\newcommand{\secref}[1]{\hyperref[#1]{section \ref*{#1}}}

\parskip1ex
\parindent0pt
\let\olditemize\itemize
\def\itemize{\olditemize\parskip0pt}

\begin{document}

\title{The \textsf{childdoc} Package}
\hypersetup{pdftitle={The childdoc Package}}
\author{Niklas Beisert\\[2ex]
  Institut f\"ur Theoretische Physik\\
  Eidgen\"ossische Technische Hochschule Z\"urich\\
  Wolfgang-Pauli-Strasse 27, 8093 Z\"urich, Switzerland\\[1ex]
  \href{mailto:nbeisert@itp.phys.ethz.ch}
  {\texttt{nbeisert@itp.phys.ethz.ch}}}
\hypersetup{pdfauthor={Niklas Beisert}}
\hypersetup{pdfsubject={Manual for the LaTeX2e Package childdoc}}
\date{30 December 2018, \textsf{v2.0}}
\maketitle

\begin{abstract}\noindent
\textsf{childdoc} is a \LaTeXe{} package
that enables the direct compilation
of document sections included by |\include|
to individual files.
\end{abstract}

\begingroup
\parskip0ex
\tableofcontents
\endgroup

%%%%%%%%%%%%%%%%%%%%%%%%%%%%%%%%%%%%%%%%%%%%%%%%%%%%%%%%%%%%%%%%%%%%%%%%%%%%%%%%
%%%%%%%%%%%%%%%%%%%%%%%%%%%%%%%%%%%%%%%%%%%%%%%%%%%%%%%%%%%%%%%%%%%%%%%%%%%%%%%%
\section{Introduction}

\LaTeX{} provides a mechanism to structure a large document (such as a book)
into a main file and several child files (containing the chapters)
using the |\include| command.
This mechanism is beneficial for documents
which span hundreds of pages in order to
make the source file(s) more manageable.
Moreover, compilation can be restricted to
selected child files by means of the |\includeonly| command.
The latter feature can be used to reduce the compilation time while editing
(this was significantly more useful in the earlier days of \LaTeX{})
or to generate a smaller document which is easier to navigate.
Another application of |\includeonly| is to generate
documents consisting of selected parts of the complete document.

However, there are a few drawbacks of the plain |\include| mechanism:
\begin{itemize}
\item
The child files cannot be compiled on their own,
they can only be compiled via the main file.
A naive editing environment
(such as a text editor with an option
to have the current file processed by \LaTeX)
may require one to switch to the main file before compiling;
attempting to compile the child file produces errors.
\item
The main file must be modified (each time)
to adjust the |\includeonly| command
to the present needs. This easily leaves the main file in a messy state.
\item
The generated document will always carry the filename
of the main document. This is inconvenient if
several child files are to be compiled and
to be kept for distribution.
\end{itemize}

The present package provides a simple interface
to make child files individually compilable by \LaTeX{}.
Compiling a child file then has the same effect as compiling
the main file with an |\includeonly| command
to select the appropriate child.
Moreover the generated document will carry the name of the child
rather than the main file.
This resolves all three above issues.

This feature is meant to make the editing of books,
thesis documents and lecture notes somewhat more convenient.
However, the package can also be used efficiently for
composing a series of documents (such as exercise sheets)
which are typically distributed individually.
It then assists the author in generating the individual documents
(potentially in different versions)
as well as a document containing the collected series.
Another application is in developing style files
or other kinds of included material
where compilation of the style file could redirect
to a sample or test file.

%%%%%%%%%%%%%%%%%%%%%%%%%%%%%%%%%%%%%%%%%%%%%%%%%%%%%%%%%%%%%%%%%%%%%%%%%%%%%%%%
%%%%%%%%%%%%%%%%%%%%%%%%%%%%%%%%%%%%%%%%%%%%%%%%%%%%%%%%%%%%%%%%%%%%%%%%%%%%%%%%
\section{Usage}

First of all, the package \textsf{childdoc} is \emph{not} a standard
\LaTeXe{} |.sty| style file! Therefore it needs to be invoked in
a non-standard way.

%%%%%%%%%%%%%%%%%%%%%%%%%%%%%%%%%%%%%%%%%%%%%%%%%%%%%%%%%%%%%%%%%%%%%%%%%%%%%%%%
\subsection{Included Files}
\label{sec:include}

%%%%%%%%%%%%%%%%%%%%%%%%%%%%%%%%%%%%%%%%
\DescribeMacro{\childdocmain}
To use the package, add the commands
\begin{center}
\begin{tabular}{l}
|\input{childdoc.def}|\\
|\childdocmain{}|\\
\end{tabular}
\end{center}
at the very top of the main \LaTeX{} file,
in particular \emph{before} the |\documentclass| statement!
The argument of |\childdocmain| should be left empty
(but it must be present).

%%%%%%%%%%%%%%%%%%%%%%%%%%%%%%%%%%%%%%%%
\DescribeMacro{\childdocof}
Furthermore, add the commands
\begin{center}
\begin{tabular}{l}
|\input{childdoc.def}|\\
|\childdocof{|\textit{main}|}|\\
\end{tabular}
\end{center}
at the top of every child file \textit{child}
which is included by |\include{|\textit{child}|}|
from within the main file
(or at least for those files to be compiled individually).
The argument \textit{main} must be the filename of the main file.

There are a couple of
considerations in setting up the main and child documents:

%%%%%%%%%%%%%%%%%%%%%%%%%%%%%%%%%%%%%%%%
\paragraph{Restrictions.}

Please note the following restrictions:
\begin{itemize}
\item
|\childdocmain| must be called with one argument \textit{main}
to ensure compatibility with earlier version of the package.
It must either be empty (|\childdocmain{}|)
or precisely match the filename of the main file in which it is specified.
See \secref{sec:detection} for further information.
\item
The filename \textit{main} must be specified without the |.tex| extension.
\item
The filename \textit{main} is case sensitive
(even in case-insensitive file systems)
due to internal string comparison.
\item
The argument \textit{main} should be fully expanded, it cannot be a macro.
\item
Subdirectories and special characters should be avoided in filenames.
\item
The command |\childdocmain{|\textit{main}|}| must be followed by a whitespace.
It should not be followed immediately by another command
or by a comment mark `|%|'.
This is because the \TeX{} parser reads the token immediately following
the argument of |\childdocmain| and puts it
at the beginning of every child section;
however, a white\-space is ignored.
\end{itemize}

%%%%%%%%%%%%%%%%%%%%%%%%%%%%%%%%%%%%%%%%
\paragraph{Content of Main File.}

It is advisable to place all content in the child files included by |\include|.
Any output contained in the main file will appear in all child documents
unless suppressed manually;
it cannot be suppressed automatically by the |\includeonly| directive
and thus should normally be avoided.
A method to include some content in the main file
by means of conditional processing is described in \secref{sec:conditional}.

%%%%%%%%%%%%%%%%%%%%%%%%%%%%%%%%%%%%%%%%
\paragraph{Page Numbering.}

When only a part of the document is compiled,
the appropriate numbering of pages
(as well as other status parameters)
is determined from the |.aux| files.
The latter contain information from previous passes.
However this information needs to propagate through
all intermediate child documents.
Therefore the page numbering in child documents may well
be inconsistent until the complete document is compiled at least once.

A useful (if unconventional) way to always ensure a consistent
page numbering is to restart the numbering in each child document
and denote the pages by `\textit{child}|.|\textit{page}'
where \textit{child} represents the chapter/section number of the child file.
This can be achieved by the command
|\numberwithin{page}{|\textit{child}|}|
of the \textsf{amsmath} package
where \textit{child} can be |chapter| or |section|
depending on the chosen structuring.
Alternatively, one can modify the macro |\thepage| appropriately
and reset the counter |page| at the start of each child file.

%%%%%%%%%%%%%%%%%%%%%%%%%%%%%%%%%%%%%%%%%%%%%%%%%%%%%%%%%%%%%%%%%%%%%%%%%%%%%%%%
\subsection{Conditional Processing}
\label{sec:conditional}

The package provides a mechanism to compile different versions
of a document. To customise the versions further some conditional processing
can come in handy to distinguish which version is being compiled.
The package provides two macros to describe the compilation context:

%%%%%%%%%%%%%%%%%%%%%%%%%%%%%%%%%%%%%%%%
\DescribeMacro{\ifchilddoc}
The conditional |\ifchilddoc| distinguishes between the compilation of
child documents and the main document:
%
\begin{center}
|\ifchilddoc |\textit{child-code}| |[|\||else |\textit{main-code}]| \||fi|
\end{center}

%%%%%%%%%%%%%%%%%%%%%%%%%%%%%%%%%%%%%%%%
\DescribeMacro{\childdocname}
\DescribeMacro{\childdocjob}
The macro |\childdocname| contains the filename (without extension)
of the main or child file being processed.
Note that |\childdocjob| will always contain the name of the main file.

%%%%%%%%%%%%%%%%%%%%%%%%%%%%%%%%%%%%%%%%
\paragraph{Title Page.}

Conditional processing can be used to include a title or banner page
in the main document when proper precautions are taken.
Importantly, the code in the main file should ensure that the page counter
(as well as other status parameters which are stored in the |.aux| files)
takes the same value after the conditional processing.
Otherwise the page numbers may take divergent values
depending on which part is compiled.

For example, a title page could be declared by:
%
\begin{center}
\begin{tabular}{l}
|\ifchilddoc\||else|\\
|\addtocounter{page}{-1}|\\
\textit{code for title page}\\
|\newpage|\\
|\||fi|
\end{tabular}
\end{center}
%
A banner page for the child documents can be generated by:
%
\begin{center}
\begin{tabular}{l}
|\ifchilddoc|\\
|\addtocounter{page}{-1}|\\
\textit{code for banner page}\\
|\newpage|\\
|\||fi|
\end{tabular}
\end{center}
%
Here one could write a message such as:
\begin{center}
|This is the part \childdocname{} of \childdocjob{}.|
\end{center}

%%%%%%%%%%%%%%%%%%%%%%%%%%%%%%%%%%%%%%%%%%%%%%%%%%%%%%%%%%%%%%%%%%%%%%%%%%%%%%%%
\subsection{Flags}
\label{sec:flags}

The package makes it easy to generate different versions
of the main or child documents.
To this end compilation flags can be defined
and assigned different default values.
They will be particularly useful in conjunction
with the forwarding mechanism described in \secref{sec:forward}.

For example, it may be useful to have a flag |\version|
which can be set to |draft| or |final|.
The document source will contain some conditional code
depending on the value of |\version|.
Suppose further, the flag should default to |final| for the main file
and to |draft| for child files
which is a natural assignment for editing the document.
This is achieved by placing the following code
in the preamble of the main document
(below the |\childdocmain| directive):
%
\begin{center}
\begin{tabular}{l}
|\ifchilddoc|\\
|\providecommand{\version}{draft}|\\
|\||else|\\
|\providecommand{\version}{final}|\\
|\||fi|
\end{tabular}
\end{center}
%
The definition by |\providecommand| makes sure
that previous definitions are not overwritten.
Further statements |\providecommand{\version}{...}|
can thus be added before the above code to override it.

For the main file, one might add a line
(between |\childdocmain| and the above block)
%
\begin{center}
|%\ifchilddoc\||else\providecommand{\version}{draft}\||fi|
\end{center}
%
which can be uncommented to produce a draft version.
Likewise one can add a line to the very top of a child file
(above the |\childdocof{|\textit{main}|}| directive)
%
\begin{center}
|%\providecommand{\version}{final}|
\end{center}
%
which can be uncommented to produce the final version of this child document.

%%%%%%%%%%%%%%%%%%%%%%%%%%%%%%%%%%%%%%%%%%%%%%%%%%%%%%%%%%%%%%%%%%%%%%%%%%%%%%%%
\subsection{Forwarding}
\label{sec:forward}

Different versions of the main or child documents
using compilation flags as described in \secref{sec:flags}
can be (permanently) stored in different files
for convenient compilation, viewing and distribution.
To this end, the package defines a command
to pass on compilation to a different file:

%%%%%%%%%%%%%%%%%%%%%%%%%%%%%%%%%%%%%%%%
\DescribeMacro{\childdocforward}
The command |\childdocforward| redirects processing to
another source file:
%
\begin{center}
\begin{tabular}{l}
|\input{childdoc.def}|\\
|\childdocforward[|\textit{main}|]{|\textit{dest}|}|\\
\end{tabular}
\end{center}
%
The argument \textit{dest} is the destination file
(without extension).
It should be the main file or one of the child files.
Note that further \textsf{childdoc} directives
such as |\childdocof| and |\childdocforward|
in the indicated file will be processed in this form.
The optional argument \textit{main}
passes on directly to the main file \textit{main}
while pretending to compile the child \textit{dest}.
This form behaves as if \textit{dest}
issues |\childdocof{|\textit{main}|}| right away,
and no further \textsf{childdoc} directives will be processed.

%%%%%%%%%%%%%%%%%%%%%%%%%%%%%%%%%%%%%%%%
\DescribeMacro{\...prefix}
In the alternative form |\childdocforwardprefix|,
%
\begin{center}
\begin{tabular}{l}
|\input{childdoc.def}|\\
|\childdocforwardprefix[|\textit{main}|]{|\textit{prefix}|}{|\textit{dest}|}|
\end{tabular}
\end{center}
%
the destination file is determined by a pattern
depending on the current file:
To make this work, the current file must be called
`{\textit{prefix}\hspace{0.2em}\textit{suffix}}'
with \textit{prefix} matching precisely the argument.
Processing is then passed on to the file
`{\textit{dest}\hspace{0.2em}\textit{suffix}}'.
Surely, the same effect is achieved by
directly specifying the
argument `{\textit{dest}\hspace{0.2em}\textit{suffix}}'
in the first form.
However, that requires to set up a different file
for each child. With the alternative form of the command
all these files can have exactly the same content
which simplifies setting them up and maintaining them.

For example, the following file |draft.tex|
with a compilation flag |\version| as described in \secref{sec:flags}
compiles the main document as a draft:
%
\begin{center}
\begin{tabular}{l}
|\def\version{draft}|\\
|\input{childdoc.def}|\\
|\childdocforward{|\textit{main}|}|
\end{tabular}
\end{center}
%
Likewise, the following files |final|\textit{nn}|.tex|
compile the final version of the child document
|child|\textit{nn}|.tex|:
%
\begin{center}
\begin{tabular}{l}
|\def\version{final}|\\
|\input{childdoc.def}|\\
|\childdocforwardprefix{final}{child}|
\end{tabular}
\end{center}
%

Note that when several versions of a main file and/or of each child file
are to be generated, it may be convenient to set up a |Makefile| or
shell script to automatise the process.

%%%%%%%%%%%%%%%%%%%%%%%%%%%%%%%%%%%%%%%%%%%%%%%%%%%%%%%%%%%%%%%%%%%%%%%%%%%%%%%%
\subsection{Command Line Processing}
\label{sec:commandline}

The effect of redirection files can also be achieved by invoking
the \LaTeX{} compiler with a more elaborate command line.
Most conveniently this should be done as part
of a shell script or a |Makefile|.

When using \textsf{childdoc} in the main file, the following
command lines effectively perform a redirection
(note that depending on the shell being used,
backslashes may have to be doubled: `|\|' $\to$ `|\\|'):
%
\begin{center}
|... -jobname "|\textit{target}|" |\\|"|[\textit{flags}]%
|\input{childdoc.def}\childdocforward[|\textit{main}|]{|\textit{dest}|}"|
\end{center}
%
Here \textit{target} is the name of the output file,
\textit{main} is the name of the main file
and \textit{dest} is the name of the main or child file to be processed
(all filenames without extensions).
The optional argument \textit{main} can be omitted
if \textit{main} matches \textit{dest}.
Optionally, compilation \textit{flags} can be defined via |\def| commands.
This command line makes the \TeX{} engine believe
it is compiling the file \textit{target}
whose content is specified as the latter parameter.
The provided code then forwards the processing to
\textit{main} or \textit{dest} as described in \secref{sec:forward}.

%%%%%%%%%%%%%%%%%%%%%%%%%%%%%%%%%%%%%%%%%%%%%%%%%%%%%%%%%%%%%%%%%%%%%%%%%%%%%%%%
\subsection{Include by Input}
\label{sec:input}

Including child documents by |\include| has some restrictions by design.
Most notably, the content of a child document always occupies
its own set of pages; pages cannot be shared between child documents.
Usually, this behaviour makes perfect sense
because each child document contain an essential part of the document.
However, in some situations it may be desirable to compose
a document from a collection of parts
without having mandatory page breaks between then.
For this case, the package
provides a mechanism to include parts
by |\input| which can also be processed individually.
However, by construction this mechanism
requires manual handling of the content to be output.

%%%%%%%%%%%%%%%%%%%%%%%%%%%%%%%%%%%%%%%%
\DescribeMacro{\ifchilddocmanual}
The main file should be prepared as usual, see \secref{sec:include}.
However, the document body must make a distinction
between processing of an individual part and of the main document, e.g.:
%
\begin{center}
\begin{tabular}{l}
|\ifchilddocmanual|\\
|\input{\childdocname}|\\
|\||else|\\
\textit{document body with }|\input{|\textit{part}|}|\\
|\||fi|
\end{tabular}
\end{center}
%
The conditional |\ifchilddocmanual| is true whenever
a part to be included by |\input| is being compiled,
and the name of the part is stored in |\childdocname|.

%%%%%%%%%%%%%%%%%%%%%%%%%%%%%%%%%%%%%%%%
\DescribeMacro{\childdocby}
Each part to be included by |\input| should start with:
%
\begin{center}
\begin{tabular}{l}
|\input{childdoc.def}|\\
|\childdocby{|\textit{main}|}|\\
\end{tabular}
\end{center}
%
The directive |\childdocby| is similar to |\childdocof|
described in \secref{sec:include},
but the subsequent selection of content must be done manually.
To that end, both |\ifchilddoc| and |\ifchilddocmanual|
will be true upon processing of a part,
and the name of the part is stored in |\childdocname|.
Note that |\jobname| will be set to the filename of the current part
so that each part receives an individual |.aux| file
that does not interfere with the |.aux| file(s) of the main document.
This behaviour can be altered by the alternative form
|\childdocby[*]{|\textit{main}|}| (with a non-empty optional argument)
which uses the |.aux| file of the main document
by setting |\jobname| to \textit{main}.

%%%%%%%%%%%%%%%%%%%%%%%%%%%%%%%%%%%%%%%%%%%%%%%%%%%%%%%%%%%%%%%%%%%%%%%%%%%%%%%%
\subsection{Driver Development}
\label{sec:driver}

The \textsf{childdoc} mechanism can also be use for the development
of definition files such as \LaTeX{} styles or classes.
This case differs from the above setup with multiple parts
included by |\include| in that no |\includeonly| should be invoked.
This can be achieved by starting the include file
(before |\ProvidesPackage|) with:
%
\begin{center}
\begin{tabular}{l}
|\input{childdoc.def}|\\
|\childdocforward{|\textit{main}|}|\\
\end{tabular}
\end{center}
%
or alternatively with:
%
\begin{center}
\begin{tabular}{l}
|\input{childdoc.def}|\\
|\childdocby{|\textit{main}|}|\\
\end{tabular}
\end{center}
%
Both forms have slightly different effects as described above.
The main file is prepared as usual, see \secref{sec:include}.

%%%%%%%%%%%%%%%%%%%%%%%%%%%%%%%%%%%%%%%%%%%%%%%%%%%%%%%%%%%%%%%%%%%%%%%%%%%%%%%%
\subsection{Legacy Detection}
\label{sec:detection}

The directive |\childdocmain| in the main file can detect
whether the complete document or merely a child is to be compiled
even without using the directive |\childdocof|.
This method is deprecated because it is less robust
and there is no compelling reason to use it;
it is merely provided for backward compatibility
and it may be removed in future versions.

If the detection mechanism is to be used,
it is mandatory to correctly specify
the filename of the main file as the argument of |\childdocmain|:
%
\begin{center}
\begin{tabular}{l}
|\input{childdoc.def}|\\
|\childdocmain{|\textit{main}|}|\\
\end{tabular}
\end{center}
%
If |\jobname| does not match the argument \textit{main} of |\childdocmain|,
it is assumed that |\jobname| points to the child file to be compiled.
When using |\childdocmain| with the main file specified as argument,
it suffices to start a child file
with just |\input{|\textit{main}|}|
without loading of the package and using |\childdocof|.
If instead all processing is done
with the appropriate \textsf{childdoc} directives,
the argument of \textit{main} of |\childdocmain| can be empty.

An alternative version of the command line processing described
in \secref{sec:commandline} using the detection mechanism reads:
%
\begin{center}
|... -jobname "|\textit{target}|" "|[\textit{flags}]%
[|\def\jobname{|\textit{dest}|}|]|\input{|\textit{main}|}"|
\end{center}

%%%%%%%%%%%%%%%%%%%%%%%%%%%%%%%%%%%%%%%%%%%%%%%%%%%%%%%%%%%%%%%%%%%%%%%%%%%%%%%%
\subsection{Manual Code}
\label{sec:manual}

In case one cannot be certain whether the definitions file |childdoc.def|
is installed on the target \TeX{} distribution
and one prefers not to ship it,
it is conceivable to paste a few relevant commands into the sources.

To that end, drop all statements |\input{childdoc.def}|
and perform the replacements as outlined below.
Instead of |\childdocmain{|\textit{main}|}| add the following code
to the top of the main file:
%
\begin{center}
\begin{tabular}{l}
|\||ifdefined\childdocname\endinput\||fi\newif\ifchilddoc|\\
|\edef\childdocname{\scantokens\expandafter{\jobname\noexpand}}|\\
|\def\childdocmain{|\textit{main}|}\||ifx\childdocmain\childdocname\||else|\\
|\childdoctrue\includeonly{\childdocname}\let\jobname\childdocmain\||fi|\\
\end{tabular}
\end{center}
%
Instead of |\childdocof{|\textit{main}|}| just include the main file
at the top of each child file:
%
\begin{center}
|\input{|\textit{main}|}|
\end{center}
%
A simple redirection |\childdocforward{|\textit{dest}|}| is achieved by:
%
\begin{center}
|\def\jobname{|\textit{dest}|}\input{\jobname}|
\end{center}
%
The redirection with prefix
|\childdocforwardprefix[|\textit{prefix}|]{|\textit{dest}|}|
is accomplished by:
%
\begin{center}
\begin{tabular}{l}
|{\edef\jobname{\scantokens\expandafter{\jobname\noexpand}}|\\
|\def\redirectjob |\textit{prefix}|#1~~~{\gdef\jobname{|\textit{dest}|#1}}|\\
|\expandafter\redirectjob\jobname~~~}\input{\jobname}|
\end{tabular}
\end{center}

In an alternative approach,
child documents can be compiled by a specific command line
without additional code or specific definitions:
%
\begin{center}
|... -jobname "|\textit{target}|" "|[\textit{flags}]%
|\includeonly{|\textit{dest}|}\input{|\textit{main}|}"|
\end{center}
%

%%%%%%%%%%%%%%%%%%%%%%%%%%%%%%%%%%%%%%%%%%%%%%%%%%%%%%%%%%%%%%%%%%%%%%%%%%%%%%%%
%%%%%%%%%%%%%%%%%%%%%%%%%%%%%%%%%%%%%%%%%%%%%%%%%%%%%%%%%%%%%%%%%%%%%%%%%%%%%%%%
\section{Information}

%%%%%%%%%%%%%%%%%%%%%%%%%%%%%%%%%%%%%%%%%%%%%%%%%%%%%%%%%%%%%%%%%%%%%%%%%%%%%%%%
\subsection{Copyright}

Copyright \copyright{} 2017--2018 Niklas Beisert

This work may be distributed and/or modified under the
conditions of the \LaTeX{} Project Public License, either version 1.3
of this license or (at your option) any later version.
The latest version of this license is in
  \url{http://www.latex-project.org/lppl.txt}
and version 1.3 or later is part of all distributions of \LaTeX{}
version 2005/12/01 or later.

This work has the LPPL maintenance status `maintained'.

The Current Maintainer of this work is Niklas Beisert.

This work consists of the files |README.txt|, |childdoc.ins| and |childdoc.dtx|
as well as the derived files |childdoc.def|, |cdocsamp.tex|
with |cdocsch1.tex|, |cdocsch2.tex|, |cdocspt3.tex|, |cdocspt4.tex|,
|cdocsdrf.tex|, |cdocsfn1.tex|, |cdocsfn2.tex|
as well as |childdoc.pdf|.

%%%%%%%%%%%%%%%%%%%%%%%%%%%%%%%%%%%%%%%%%%%%%%%%%%%%%%%%%%%%%%%%%%%%%%%%%%%%%%%%
\subsection{Files and Installation}

The package consists of the files:
%
\begin{center}
\begin{tabular}{ll}
    |README.txt|   & readme file \\
    |childdoc.ins| & installation file \\
    |childdoc.dtx| & source file \\
    |childdoc.def| & definition file \\
    |cdocsamp.tex| & sample main file \\
    |cdocsch1.tex| & sample include file \\
    |cdocsch2.tex| & sample include file \\
    |cdocspt3.tex| & sample part file \\
    |cdocspt4.tex| & sample part file \\
    |cdocsdrf.tex| & sample redirection file \\
    |cdocsfn1.tex| & sample redirection file \\
    |cdocsfn2.tex| & sample redirection file \\
    |childdoc.pdf| & manual
\end{tabular}
\end{center}
%
The distribution consists of the files
|README.txt|, |childdoc.ins| and |childdoc.dtx|.
%
\begin{itemize}
\item
Run (pdf)\LaTeX{} on |childdoc.dtx|
to compile the manual |childdoc.pdf| (this file).
\item
Run \LaTeX{} on |childdoc.ins| to create the definitions file |childdoc.def|
and the sample |cdocsamp.tex| with include files
|cdocsch1.tex|, |cdocsch2.tex|, |cdocspt3.tex|, |cdocspt4.tex|,
|cdocsdrf.tex|, |cdocsfn1.tex|, |cdocsfn2.tex|.
Then copy the file |childdoc.def| to an appropriate directory of your \LaTeX{}
distribution, e.g.\ \textit{texmf-root}|/tex/latex/childdoc|.
\end{itemize}

%%%%%%%%%%%%%%%%%%%%%%%%%%%%%%%%%%%%%%%%%%%%%%%%%%%%%%%%%%%%%%%%%%%%%%%%%%%%%%%%
\subsection{Related CTAN Packages}

There are several other packages which offer a similar functionality:
%
\begin{itemize}
\item
The packages
\href{http://ctan.org/pkg/docmute}{\textsf{docmute}},
\href{http://ctan.org/pkg/includex}{\textsf{includex}} and
\href{http://ctan.org/pkg/standalone}{\textsf{standalone}}
provide commands to include only the document body of
a child file thus allowing both files to be compiled individually.
\item
The packages \href{http://ctan.org/pkg/subdocs}{\textsf{subdocs}}
and \href{http://ctan.org/pkg/subfiles}{\textsf{subfiles}}
provide structures in which the main and child documents can be
encapsulated and allowing them to be compiled individually.
The inclusion mechanism is different from the conventional |\include|.
\item
The package \href{http://ctan.org/pkg/combine}{\textsf{combine}}
is an elaborate solution to combine several documents into one.
\end{itemize}
%
See also the CTAN topic \href{http://ctan.org/topic/subdocs}{\textsf{subdocs}}
for further related packages.
The present package differs from the above solutions in that
a document structure constructed with the conventional |\include| mechanism
just needs two extra commands at the top of every file
such that all constituent files can be compiled individually.

%%%%%%%%%%%%%%%%%%%%%%%%%%%%%%%%%%%%%%%%%%%%%%%%%%%%%%%%%%%%%%%%%%%%%%%%%%%%%%%%
%\subsection{Feature Suggestions}
%
%The following is a list of features which may be useful for future
%versions of this package:
%%
%\begin{itemize}
%\item
%\ldots
%\end{itemize}

%%%%%%%%%%%%%%%%%%%%%%%%%%%%%%%%%%%%%%%%%%%%%%%%%%%%%%%%%%%%%%%%%%%%%%%%%%%%%%%%
\subsection{Revision History}

%%%%%%%%%%%%%%%%%%%%%%%%%%%%%%%%%%%%%%%%
\paragraph{v2.0:} 2018/12/30

\begin{itemize}
\item
immediate forward processing
\item
added |\childdocby| mechanism
\item
manual restructured
\end{itemize}

%%%%%%%%%%%%%%%%%%%%%%%%%%%%%%%%%%%%%%%%
\paragraph{v1.6:} 2018/01/17

\begin{itemize}
\item
application for development of include files
\item
corrections to manual
\end{itemize}

%%%%%%%%%%%%%%%%%%%%%%%%%%%%%%%%%%%%%%%%
\paragraph{v1.5:} 2017/05/21

\begin{itemize}
\item
more complete structuring introduced
\item
|\childdocof| introduced
\item
|\childdoc| renamed to |\childdocmain|
\item
|\childredirect| renamed to |\childdocforward| and |\childdocforwardprefix|
and functionality expanded
\end{itemize}

%%%%%%%%%%%%%%%%%%%%%%%%%%%%%%%%%%%%%%%%
\paragraph{v1.0:} 2017/04/27

\begin{itemize}
\item
manual and install package
\item
first version published on CTAN
\end{itemize}

%%%%%%%%%%%%%%%%%%%%%%%%%%%%%%%%%%%%%%%%
\paragraph{v0.6:} 2017/04/26

\begin{itemize}
\item
redirection mechanism added
\end{itemize}

%%%%%%%%%%%%%%%%%%%%%%%%%%%%%%%%%%%%%%%%
\paragraph{v0.5:} 2017/04/26

\begin{itemize}
\item
functionality in definition file
\end{itemize}


%%%%%%%%%%%%%%%%%%%%%%%%%%%%%%%%%%%%%%%%%%%%%%%%%%%%%%%%%%%%%%%%%%%%%%%%%%%%%%%%
%%%%%%%%%%%%%%%%%%%%%%%%%%%%%%%%%%%%%%%%%%%%%%%%%%%%%%%%%%%%%%%%%%%%%%%%%%%%%%%%
%%%%%%%%%%%%%%%%%%%%%%%%%%%%%%%%%%%%%%%%%%%%%%%%%%%%%%%%%%%%%%%%%%%%%%%%%%%%%%%%
\appendix

\settowidth\MacroIndent{\rmfamily\scriptsize 000\ }

 \DocInput{childdoc.dtx}

\end{document}
%</driver>
% \fi
%
% %%%%%%%%%%%%%%%%%%%%%%%%%%%%%%%%%%%%%%%%%%%%%%%%%%%%%%%%%%%%%%%%%%%%%%%%%%%%%%
% %%%%%%%%%%%%%%%%%%%%%%%%%%%%%%%%%%%%%%%%%%%%%%%%%%%%%%%%%%%%%%%%%%%%%%%%%%%%%%
% \section{Sample}
%\iffalse
%<*samplemain>
%\fi
%
% The following presents a sample document
% with two chapters, two parts, a title page,
% a compile flag as well as three forwarding files to set the flag.
% It consists of eight |.tex| files:
% \begin{center}
% \begin{tabular}{ll}
% |cdocsamp.tex|&main file\\
% |cdocsch1.tex|&include file for chapter 1\\
% |cdocsch2.tex|&include file for chapter 2\\
% |cdocspt3.tex|&include file for part 3\\
% |cdocspt4.tex|&include file for part 4\\
% |cdocsdrf.tex|&forwarding file for main file in draft mode\\
% |cdocsfi1.tex|&forwarding file for final version of chapter 1\\
% |cdocsfi2.tex|&forwarding file for final version of chapter 2\\
% \end{tabular}
% \end{center}
% Each of the eight files can be compiled directly by the \LaTeX{} compiler.
%
% %%%%%%%%%%%%%%%%%%%%%%%%%%%%%%%%%%%%%%
% \paragraph{Main File.}
%
% The main file is called |cdocsamp.tex|.
%
% Load the \textsf{childdoc} definitions and
% declare the filename for the main document:
%    \begin{macrocode}
\input{childdoc.def}
\childdocmain{}
%    \end{macrocode}

% Optional override for |\version| flag:
%    \begin{macrocode}
%%\ifchilddoc\else\providecommand{\version}{draft}\fi
%    \end{macrocode}

% Define the default values for the |\version| flag
% (|final| for the main file and |draft| for childs):
%    \begin{macrocode}
\ifchilddoc
\providecommand{\version}{draft}
\else
\providecommand{\version}{final}
\fi
%    \end{macrocode}

% Load the standard document class:
%    \begin{macrocode}
\documentclass[12pt]{article}
%    \end{macrocode}

% Start the document body:
%    \begin{macrocode}
\begin{document}
%    \end{macrocode}

% Declare a title page.
% Print title, part of document being processed and version flag:
%    \begin{macrocode}
\addtocounter{page}{-1}
\begin{center}
{\LARGE\bfseries{}childdoc example\par}
\vspace{1cm}
\ifchilddoc
\ifchilddocmanual part\else chapter\fi:
`\childdocname' of `\childdocjob'\par
\else
main document: `\childdocjob'\par
\fi
version: \version\par
\end{center}
\newpage
%    \end{macrocode}

% Manually include selected file,
% otherwise process as usual:
%    \begin{macrocode}
\ifchilddocmanual
\section*{part `\childdocname'}
\input{\childdocname}
\else
%    \end{macrocode}

% Include the two chapters:
%    \begin{macrocode}
\include{cdocsch1}
\include{cdocsch2}
%    \end{macrocode}

% Include the two parts unless only chapters should be displayed:
%    \begin{macrocode}
\ifchilddoc\else
\section{part three}
\input{cdocspt3}
\section{part four}
\input{cdocspt4}
\fi
%    \end{macrocode}

% Process as usual until here:
%    \begin{macrocode}
\fi
%    \end{macrocode}

% End of document body:
%    \begin{macrocode}
\end{document}
%    \end{macrocode}
%\iffalse
%</samplemain>
%\fi
%
% %%%%%%%%%%%%%%%%%%%%%%%%%%%%%%%%%%%%%%
% \paragraph{Chapter Include Files.}
%
% The include files are called |cdocsch1.tex| and |cdocsch2.tex|.
%
%\iffalse
%<*samplechap1|samplechap2>
%\fi

% Optional override for |\version| flag:
%    \begin{macrocode}
%%\providecommand{\version}{final}
%    \end{macrocode}

% Include the main document:
%    \begin{macrocode}
\input{childdoc.def}
\childdocof{cdocsamp}
%    \end{macrocode}

%\iffalse
%</samplechap1|samplechap2>
%\fi
%
%\iffalse
%<*samplechap1>
%\fi
% Some text for chapter 1:
%    \begin{macrocode}
\section{one}
some text in chapter one
%    \end{macrocode}

%\iffalse
%</samplechap1>
%\fi
% Some text for chapter 2:
%\iffalse
%<*samplechap2>
%\fi
%    \begin{macrocode}
\section{two}
more text in chapter two
%    \end{macrocode}

%\iffalse
%</samplechap2>
%\fi
%
% %%%%%%%%%%%%%%%%%%%%%%%%%%%%%%%%%%%%%%
% \paragraph{Part Include Files.}
%
% The include files are called |cdocspt3.tex| and |cdocspt4.tex|.
%
%\iffalse
%<*samplepart3|samplepart4>
%\fi

% Optional override for |\version| flag:
%    \begin{macrocode}
%%\providecommand{\version}{final}
%    \end{macrocode}

% Include the main document:
%    \begin{macrocode}
\input{childdoc.def}
\childdocby{cdocsamp}
%    \end{macrocode}

%\iffalse
%</samplepart3|samplepart4>
%\fi
%
%\iffalse
%<*samplepart3>
%\fi
% Some text for part 3:
%    \begin{macrocode}
some text in part three
%    \end{macrocode}

%\iffalse
%</samplepart3>
%\fi
% Some text for part 4:
%\iffalse
%<*samplepart4>
%\fi
%    \begin{macrocode}
more text in part four
%    \end{macrocode}

%\iffalse
%</samplepart4>
%\fi
%
% %%%%%%%%%%%%%%%%%%%%%%%%%%%%%%%%%%%%%%
% \paragraph{Forwarding for a Complete Draft.}
%
% The following forwarding file |cdocsdrf.tex|
% compiles the main document in draft mode:
%\iffalse
%<*sampledraft>
%\fi
%    \begin{macrocode}
\def\version{draft}
\input{childdoc.def}
\childdocforward{cdocsamp}
%    \end{macrocode}

%\iffalse
%</sampledraft>
%\fi
%
% %%%%%%%%%%%%%%%%%%%%%%%%%%%%%%%%%%%%%%
% \paragraph{Forwarding for Final Version of the Chapters.}
%
% The following forwarding files |cdocsfn1.tex| and |cdocsfn2.tex|
% (with identical content)
% compile the final versions of the child documents
% |cdocsch1.tex| and |cdocsch2.tex|, respectively:
%\iffalse
%<*samplefinal>
%\fi
%    \begin{macrocode}
\def\version{final}
\input{childdoc.def}
\childdocforwardprefix[cdocsamp]{cdocsfn}{cdocsch}
%    \end{macrocode}

%\iffalse
%</samplefinal>
%\fi
%
% %%%%%%%%%%%%%%%%%%%%%%%%%%%%%%%%%%%%%%
% \paragraph{Command Line Processing.}
%
% The following three command lines generate the output files
% |cdocscld|, |cdocscl1| and |cdocscl2|
% which should be identical to
% |cdocsdrf|, |cdocsch1| and |cdocsfn2|, respectively:
% \begin{center}
% \begin{tabular}{l}
% |latex -jobname cdocscld \|\\
% |  "\def\version{draft}\input{childdoc.def}\childdocforward{cdocsamp}"|\\
% |latex -jobname cdocscl1 \|\\
% |  "\input{childdoc.def}\childdocforward[cdocsamp]{cdocsch1}"|\\
% |latex -jobname cdocscl2 \|\\
% |  "\def\version{final}\input{childdoc.def}\childdocforward{cdocsch2}"|
% \end{tabular}
% \end{center}
% Note that the trailing backslash on each first line
% merely continues the input to the second line
% (for convenient cut ant paste).
% Furthermore, the command |latex| can be replaced by any
% of its alternative versions such as |pdflatex|.
%
% %%%%%%%%%%%%%%%%%%%%%%%%%%%%%%%%%%%%%%%%%%%%%%%%%%%%%%%%%%%%%%%%%%%%%%%%%%%%%%
% %%%%%%%%%%%%%%%%%%%%%%%%%%%%%%%%%%%%%%%%%%%%%%%%%%%%%%%%%%%%%%%%%%%%%%%%%%%%%%
% \section{Implementation}
%\iffalse
%<*package>
%\fi
%
% This section describes the definitions file |childdoc.def|.

% The definitions cannot be loaded using |\usepackage| or |\RequirePackage|
% which has a mechanism to prevent loading a style file more than once.
% When loading the definitions by means of |\input|
% multiple instances have to be prevented manually:
%\iffalse
%This code needs to be before the `\ProvidesFile' directive
%which is defined at the beginning of this file.
%Therefore it is also placed there and commented out here.
%</package>
%<*discard>
%\fi
%    \begin{macrocode}
\ifdefined\childdocmain\endinput\fi
%    \end{macrocode}
%\iffalse
%</discard>
%<*package>
%\fi
%
% \macro{\ifchilddoc}
% \macro{\ifchilddocmanual}
% The conditional |\ifchilddoc| tells whether a
% child (true) or main (false) document is being compiled.
% The conditional |\ifchilddocmanual| tells whether
% the |\includeonly| mechanism is used (false) or
% the selection of child files must be performed manually (true).
% The definitions initialise to false:
%    \begin{macrocode}
\newif\ifchilddoc
\newif\ifchilddocmanual
%    \end{macrocode}

% \macro{\childdocname}
% \macro{\childdocjob}
% The macro |\childdocname| stores the name of the main document
% to be compiled. The macro |\childdocjob| stores the name of
% the document on which the \LaTeX{} compiler was originally invoked.
% The content of |\jobname| cannot be compared
% to filenames specified in the source due to different catcodes.
% The following code rescans |\jobname|, stores the result
% in |\childdocname| and saves a copy in |\childdocjob|:
%    \begin{macrocode}
\edef\childdocname{\scantokens\expandafter{\jobname\noexpand}}
\let\childdocjob\childdocname
%    \end{macrocode}

% \macro{\childdocdisable}
% The macro |\childdocdisable| prevents the main file
% from being processed more than once.
% At this stage, the main document command |\childdocmain|
% is assumed to be called once again where it should do nothing.
% Any subsequent call to it should prevent
% a secondary processing of the main document
% It overwrites the forwarding commands
% |\childdocof| and |\childdocforward|
% with empty macros to prevent further inclusions of the main document:
%    \begin{macrocode}
\newcommand{\childdocdisable}
{
  \renewcommand{\childdocmain}[1]{\renewcommand{\childdocmain}[1]{\endinput}}
  \renewcommand{\childdocof}[1]{}
  \renewcommand{\childdocby}[2][]{}
  \renewcommand{\childdocforward}[2][]{}
  \renewcommand{\childdocdisable}{}
}
%    \end{macrocode}

% \macro{\childdocmain}
% The macro |\childdocmain| is to be called at the top of the main file
% with nothing or the main filename (without extension) as argument.
% First, it breaks loops.
% If the argument is not empty and does not match |\childdocname|
% (which is set by the first inclusion of |childdoc.def|),
% |\ifchilddoc| is set to true, |\includeonly| is applied to the child file
% and |\jobname| is set to the main file
% (for proper handling of |.aux| files):
%    \begin{macrocode}
\newcommand{\childdocmain}[1]
{
  \childdocdisable\childdocmain{}
  \if?#1?\else
    \begingroup
      \def\childdoctmp{#1}
      \ifx\childdoctmp\childdocname
        \def\childdoctmp{}
      \else
        \def\childdoctmp
        {
          \childdoctrue
          \includeonly{\childdocname}
          \def\childdocjob{#1}
          \def\jobname{#1}
        }
      \fi
      \expandafter
    \endgroup
    \childdoctmp
  \fi
}
%    \end{macrocode}

% \macro{\childdocof}
% The command |\childdocof| redirects
% compilation to the main file |#1|.
%    \begin{macrocode}
\newcommand{\childdocof}[1]
{
  \childdocdisable
  \childdoctrue
  \includeonly{\childdocname}
  \def\jobname{#1}
  \def\childdocjob{#1}
  \input{#1}
}
%    \end{macrocode}

% \macro{\childdocby}
% The command |\childdocby| ....
%    \begin{macrocode}
\newcommand{\childdocby}[2][]
{
  \childdocdisable
  \childdoctrue
  \childdocmanualtrue
  \if?#1?\else
    \def\jobname{#2}
  \fi
  \def\childdocjob{#2}
  \input{#2}
  \endinput
}
%    \end{macrocode}

% \macro{\childdocforward}
% The command |\childdocforward| redirects
% compilation to the main file or
% (if the optional argument is given) a child file.
% Parameters are set as if the main file
% or a child file starting with |\childdocof| was compiled.
% Then compilation is handed over to the main file:
%    \begin{macrocode}
\newcommand{\childdocforward}[2][]
{
  \begingroup
    \if?#1?
      \def\childdoctmp
      {
        \def\childdocname{#2}
        \def\childdocjob{#2}
        \def\jobname{#2}
        \input{#2}
        \endinput
      }
    \else
      \def\childdoctmp
      {
        \childdocdisable
        \def\childdocname{#2}
        \childdoctrue
        \includeonly{#2}
        \def\childdocjob{#1}
        \def\jobname{#1}
        \input{#1}
        \endinput
      }
    \fi
    \expandafter
  \endgroup
  \childdoctmp
}
%    \end{macrocode}

% \macro{\childdocforwardprefix}
% The command |\childdocforwardprefix| redirects
% compilation to the main or a child file by means of a pattern.
% The prefix |#1| in the current filename is replaced by |#2|
% and the suffix of the current filename is kept
% (it is assumed that the filename does not contain the substring `|~~~|'
% which is used as a delimiter).
% Compilation is handed over to the new file by |\childdocforward|:
%    \begin{macrocode}
\newcommand{\childdocforwardprefix}[3][]
{
  \begingroup
    \def\childdocextract #2##1~~~{\def\childdoctmp{\childdocforward[#1]{#3##1}}}
    \expandafter\childdocextract\childdocname~~~
    \expandafter
  \endgroup
  \childdoctmp
}
%    \end{macrocode}

% \macro{\childdoc}
% The deprecated macro |\childdoc| is a legacy version of |\childdocmain|:
%    \begin{macrocode}
\newcommand{\childdoc}{\childdocmain}
%    \end{macrocode}

% \macro{\childdocredirect}
% The deprecated macro |\childdocredirect| is a legacy version
% of |\childdocforward| and |\childdocforwardprefix|:
%    \begin{macrocode}
\newcommand{\childdocredirect}[2][]
{
  \begingroup
    \if?#1?
      \def\childdoctmp{\childdocforward{#2}}
    \else
      \def\childdoctmp{\childdocforwardprefix{#1}{#2}}
    \fi
    \expandafter
  \endgroup
  \childdoctmp
}
%    \end{macrocode}

%\iffalse
%</package>
%\fi
%
\endinput

\childdocof{cdocsamp}
%    \end{macrocode}

%\iffalse
%</samplechap1|samplechap2>
%\fi
%
%\iffalse
%<*samplechap1>
%\fi
% Some text for chapter 1:
%    \begin{macrocode}
\section{one}
some text in chapter one
%    \end{macrocode}

%\iffalse
%</samplechap1>
%\fi
% Some text for chapter 2:
%\iffalse
%<*samplechap2>
%\fi
%    \begin{macrocode}
\section{two}
more text in chapter two
%    \end{macrocode}

%\iffalse
%</samplechap2>
%\fi
%
% %%%%%%%%%%%%%%%%%%%%%%%%%%%%%%%%%%%%%%
% \paragraph{Part Include Files.}
%
% The include files are called |cdocspt3.tex| and |cdocspt4.tex|.
%
%\iffalse
%<*samplepart3|samplepart4>
%\fi

% Optional override for |\version| flag:
%    \begin{macrocode}
%%\providecommand{\version}{final}
%    \end{macrocode}

% Include the main document:
%    \begin{macrocode}
% \iffalse
%
% childdoc.dtx Copyright (C) 2017-2018 Niklas Beisert
%
% This work may be distributed and/or modified under the
% conditions of the LaTeX Project Public License, either version 1.3
% of this license or (at your option) any later version.
% The latest version of this license is in
%   http://www.latex-project.org/lppl.txt
% and version 1.3 or later is part of all distributions of LaTeX
% version 2005/12/01 or later.
%
% This work has the LPPL maintenance status `maintained'.
%
% The Current Maintainer of this work is Niklas Beisert.
%
% This work consists of the files childdoc.dtx and childdoc.ins
% and the derived files childdoc.def and cdocsamp.tex with
% cdocsch1.tex, cdocsch2.tex, cdocsdrf.tex, cdocsfn1.tex, cdocsfn2.tex.
%
%<package>\ifdefined\childdocmain\endinput\fi
%<package>\ProvidesFile{childdoc.def}[2018/12/30 v2.0 child document driver]
%<samplemain>\ProvidesFile{cdocsamp.tex}[2018/12/30 v2.0 sample for childdoc]
%<*driver>
%\ProvidesFile{childdoc.drv}[2018/12/30 v2.0 childdoc reference manual file]
\PassOptionsToClass{10pt,a4paper}{article}
\documentclass{ltxdoc}

\usepackage[margin=35mm]{geometry}
\usepackage{hyperref}
\usepackage{hyperxmp}
\usepackage[usenames]{color}

\hypersetup{colorlinks=true}
\hypersetup{pdfstartview=FitH}
\hypersetup{pdfpagemode=UseNone}
\hypersetup{pdfsource={}}
\hypersetup{pdflang={en-UK}}
\hypersetup{pdfcopyright={Copyright 2017-2018 Niklas Beisert.
  This work may be distributed and/or modified under the
  conditions of the LaTeX Project Public License, either version 1.3
  of this license or (at your option) any later version.}}
\hypersetup{pdflicenseurl={http://www.latex-project.org/lppl.txt}}
\hypersetup{pdfcontactaddress={ETH Zurich, ITP, HIT K,
  Wolfgang-Pauli-Strasse 27}}
\hypersetup{pdfcontactpostcode={8093}}
\hypersetup{pdfcontactcity={Zurich}}
\hypersetup{pdfcontactcountry={Switzerland}}
\hypersetup{pdfcontactemail={nbeisert@itp.phys.ethz.ch}}
\hypersetup{pdfcontacturl={http://people.phys.ethz.ch/\xmptilde nbeisert/}}

\newcommand{\secref}[1]{\hyperref[#1]{section \ref*{#1}}}

\parskip1ex
\parindent0pt
\let\olditemize\itemize
\def\itemize{\olditemize\parskip0pt}

\begin{document}

\title{The \textsf{childdoc} Package}
\hypersetup{pdftitle={The childdoc Package}}
\author{Niklas Beisert\\[2ex]
  Institut f\"ur Theoretische Physik\\
  Eidgen\"ossische Technische Hochschule Z\"urich\\
  Wolfgang-Pauli-Strasse 27, 8093 Z\"urich, Switzerland\\[1ex]
  \href{mailto:nbeisert@itp.phys.ethz.ch}
  {\texttt{nbeisert@itp.phys.ethz.ch}}}
\hypersetup{pdfauthor={Niklas Beisert}}
\hypersetup{pdfsubject={Manual for the LaTeX2e Package childdoc}}
\date{30 December 2018, \textsf{v2.0}}
\maketitle

\begin{abstract}\noindent
\textsf{childdoc} is a \LaTeXe{} package
that enables the direct compilation
of document sections included by |\include|
to individual files.
\end{abstract}

\begingroup
\parskip0ex
\tableofcontents
\endgroup

%%%%%%%%%%%%%%%%%%%%%%%%%%%%%%%%%%%%%%%%%%%%%%%%%%%%%%%%%%%%%%%%%%%%%%%%%%%%%%%%
%%%%%%%%%%%%%%%%%%%%%%%%%%%%%%%%%%%%%%%%%%%%%%%%%%%%%%%%%%%%%%%%%%%%%%%%%%%%%%%%
\section{Introduction}

\LaTeX{} provides a mechanism to structure a large document (such as a book)
into a main file and several child files (containing the chapters)
using the |\include| command.
This mechanism is beneficial for documents
which span hundreds of pages in order to
make the source file(s) more manageable.
Moreover, compilation can be restricted to
selected child files by means of the |\includeonly| command.
The latter feature can be used to reduce the compilation time while editing
(this was significantly more useful in the earlier days of \LaTeX{})
or to generate a smaller document which is easier to navigate.
Another application of |\includeonly| is to generate
documents consisting of selected parts of the complete document.

However, there are a few drawbacks of the plain |\include| mechanism:
\begin{itemize}
\item
The child files cannot be compiled on their own,
they can only be compiled via the main file.
A naive editing environment
(such as a text editor with an option
to have the current file processed by \LaTeX)
may require one to switch to the main file before compiling;
attempting to compile the child file produces errors.
\item
The main file must be modified (each time)
to adjust the |\includeonly| command
to the present needs. This easily leaves the main file in a messy state.
\item
The generated document will always carry the filename
of the main document. This is inconvenient if
several child files are to be compiled and
to be kept for distribution.
\end{itemize}

The present package provides a simple interface
to make child files individually compilable by \LaTeX{}.
Compiling a child file then has the same effect as compiling
the main file with an |\includeonly| command
to select the appropriate child.
Moreover the generated document will carry the name of the child
rather than the main file.
This resolves all three above issues.

This feature is meant to make the editing of books,
thesis documents and lecture notes somewhat more convenient.
However, the package can also be used efficiently for
composing a series of documents (such as exercise sheets)
which are typically distributed individually.
It then assists the author in generating the individual documents
(potentially in different versions)
as well as a document containing the collected series.
Another application is in developing style files
or other kinds of included material
where compilation of the style file could redirect
to a sample or test file.

%%%%%%%%%%%%%%%%%%%%%%%%%%%%%%%%%%%%%%%%%%%%%%%%%%%%%%%%%%%%%%%%%%%%%%%%%%%%%%%%
%%%%%%%%%%%%%%%%%%%%%%%%%%%%%%%%%%%%%%%%%%%%%%%%%%%%%%%%%%%%%%%%%%%%%%%%%%%%%%%%
\section{Usage}

First of all, the package \textsf{childdoc} is \emph{not} a standard
\LaTeXe{} |.sty| style file! Therefore it needs to be invoked in
a non-standard way.

%%%%%%%%%%%%%%%%%%%%%%%%%%%%%%%%%%%%%%%%%%%%%%%%%%%%%%%%%%%%%%%%%%%%%%%%%%%%%%%%
\subsection{Included Files}
\label{sec:include}

%%%%%%%%%%%%%%%%%%%%%%%%%%%%%%%%%%%%%%%%
\DescribeMacro{\childdocmain}
To use the package, add the commands
\begin{center}
\begin{tabular}{l}
|\input{childdoc.def}|\\
|\childdocmain{}|\\
\end{tabular}
\end{center}
at the very top of the main \LaTeX{} file,
in particular \emph{before} the |\documentclass| statement!
The argument of |\childdocmain| should be left empty
(but it must be present).

%%%%%%%%%%%%%%%%%%%%%%%%%%%%%%%%%%%%%%%%
\DescribeMacro{\childdocof}
Furthermore, add the commands
\begin{center}
\begin{tabular}{l}
|\input{childdoc.def}|\\
|\childdocof{|\textit{main}|}|\\
\end{tabular}
\end{center}
at the top of every child file \textit{child}
which is included by |\include{|\textit{child}|}|
from within the main file
(or at least for those files to be compiled individually).
The argument \textit{main} must be the filename of the main file.

There are a couple of
considerations in setting up the main and child documents:

%%%%%%%%%%%%%%%%%%%%%%%%%%%%%%%%%%%%%%%%
\paragraph{Restrictions.}

Please note the following restrictions:
\begin{itemize}
\item
|\childdocmain| must be called with one argument \textit{main}
to ensure compatibility with earlier version of the package.
It must either be empty (|\childdocmain{}|)
or precisely match the filename of the main file in which it is specified.
See \secref{sec:detection} for further information.
\item
The filename \textit{main} must be specified without the |.tex| extension.
\item
The filename \textit{main} is case sensitive
(even in case-insensitive file systems)
due to internal string comparison.
\item
The argument \textit{main} should be fully expanded, it cannot be a macro.
\item
Subdirectories and special characters should be avoided in filenames.
\item
The command |\childdocmain{|\textit{main}|}| must be followed by a whitespace.
It should not be followed immediately by another command
or by a comment mark `|%|'.
This is because the \TeX{} parser reads the token immediately following
the argument of |\childdocmain| and puts it
at the beginning of every child section;
however, a white\-space is ignored.
\end{itemize}

%%%%%%%%%%%%%%%%%%%%%%%%%%%%%%%%%%%%%%%%
\paragraph{Content of Main File.}

It is advisable to place all content in the child files included by |\include|.
Any output contained in the main file will appear in all child documents
unless suppressed manually;
it cannot be suppressed automatically by the |\includeonly| directive
and thus should normally be avoided.
A method to include some content in the main file
by means of conditional processing is described in \secref{sec:conditional}.

%%%%%%%%%%%%%%%%%%%%%%%%%%%%%%%%%%%%%%%%
\paragraph{Page Numbering.}

When only a part of the document is compiled,
the appropriate numbering of pages
(as well as other status parameters)
is determined from the |.aux| files.
The latter contain information from previous passes.
However this information needs to propagate through
all intermediate child documents.
Therefore the page numbering in child documents may well
be inconsistent until the complete document is compiled at least once.

A useful (if unconventional) way to always ensure a consistent
page numbering is to restart the numbering in each child document
and denote the pages by `\textit{child}|.|\textit{page}'
where \textit{child} represents the chapter/section number of the child file.
This can be achieved by the command
|\numberwithin{page}{|\textit{child}|}|
of the \textsf{amsmath} package
where \textit{child} can be |chapter| or |section|
depending on the chosen structuring.
Alternatively, one can modify the macro |\thepage| appropriately
and reset the counter |page| at the start of each child file.

%%%%%%%%%%%%%%%%%%%%%%%%%%%%%%%%%%%%%%%%%%%%%%%%%%%%%%%%%%%%%%%%%%%%%%%%%%%%%%%%
\subsection{Conditional Processing}
\label{sec:conditional}

The package provides a mechanism to compile different versions
of a document. To customise the versions further some conditional processing
can come in handy to distinguish which version is being compiled.
The package provides two macros to describe the compilation context:

%%%%%%%%%%%%%%%%%%%%%%%%%%%%%%%%%%%%%%%%
\DescribeMacro{\ifchilddoc}
The conditional |\ifchilddoc| distinguishes between the compilation of
child documents and the main document:
%
\begin{center}
|\ifchilddoc |\textit{child-code}| |[|\||else |\textit{main-code}]| \||fi|
\end{center}

%%%%%%%%%%%%%%%%%%%%%%%%%%%%%%%%%%%%%%%%
\DescribeMacro{\childdocname}
\DescribeMacro{\childdocjob}
The macro |\childdocname| contains the filename (without extension)
of the main or child file being processed.
Note that |\childdocjob| will always contain the name of the main file.

%%%%%%%%%%%%%%%%%%%%%%%%%%%%%%%%%%%%%%%%
\paragraph{Title Page.}

Conditional processing can be used to include a title or banner page
in the main document when proper precautions are taken.
Importantly, the code in the main file should ensure that the page counter
(as well as other status parameters which are stored in the |.aux| files)
takes the same value after the conditional processing.
Otherwise the page numbers may take divergent values
depending on which part is compiled.

For example, a title page could be declared by:
%
\begin{center}
\begin{tabular}{l}
|\ifchilddoc\||else|\\
|\addtocounter{page}{-1}|\\
\textit{code for title page}\\
|\newpage|\\
|\||fi|
\end{tabular}
\end{center}
%
A banner page for the child documents can be generated by:
%
\begin{center}
\begin{tabular}{l}
|\ifchilddoc|\\
|\addtocounter{page}{-1}|\\
\textit{code for banner page}\\
|\newpage|\\
|\||fi|
\end{tabular}
\end{center}
%
Here one could write a message such as:
\begin{center}
|This is the part \childdocname{} of \childdocjob{}.|
\end{center}

%%%%%%%%%%%%%%%%%%%%%%%%%%%%%%%%%%%%%%%%%%%%%%%%%%%%%%%%%%%%%%%%%%%%%%%%%%%%%%%%
\subsection{Flags}
\label{sec:flags}

The package makes it easy to generate different versions
of the main or child documents.
To this end compilation flags can be defined
and assigned different default values.
They will be particularly useful in conjunction
with the forwarding mechanism described in \secref{sec:forward}.

For example, it may be useful to have a flag |\version|
which can be set to |draft| or |final|.
The document source will contain some conditional code
depending on the value of |\version|.
Suppose further, the flag should default to |final| for the main file
and to |draft| for child files
which is a natural assignment for editing the document.
This is achieved by placing the following code
in the preamble of the main document
(below the |\childdocmain| directive):
%
\begin{center}
\begin{tabular}{l}
|\ifchilddoc|\\
|\providecommand{\version}{draft}|\\
|\||else|\\
|\providecommand{\version}{final}|\\
|\||fi|
\end{tabular}
\end{center}
%
The definition by |\providecommand| makes sure
that previous definitions are not overwritten.
Further statements |\providecommand{\version}{...}|
can thus be added before the above code to override it.

For the main file, one might add a line
(between |\childdocmain| and the above block)
%
\begin{center}
|%\ifchilddoc\||else\providecommand{\version}{draft}\||fi|
\end{center}
%
which can be uncommented to produce a draft version.
Likewise one can add a line to the very top of a child file
(above the |\childdocof{|\textit{main}|}| directive)
%
\begin{center}
|%\providecommand{\version}{final}|
\end{center}
%
which can be uncommented to produce the final version of this child document.

%%%%%%%%%%%%%%%%%%%%%%%%%%%%%%%%%%%%%%%%%%%%%%%%%%%%%%%%%%%%%%%%%%%%%%%%%%%%%%%%
\subsection{Forwarding}
\label{sec:forward}

Different versions of the main or child documents
using compilation flags as described in \secref{sec:flags}
can be (permanently) stored in different files
for convenient compilation, viewing and distribution.
To this end, the package defines a command
to pass on compilation to a different file:

%%%%%%%%%%%%%%%%%%%%%%%%%%%%%%%%%%%%%%%%
\DescribeMacro{\childdocforward}
The command |\childdocforward| redirects processing to
another source file:
%
\begin{center}
\begin{tabular}{l}
|\input{childdoc.def}|\\
|\childdocforward[|\textit{main}|]{|\textit{dest}|}|\\
\end{tabular}
\end{center}
%
The argument \textit{dest} is the destination file
(without extension).
It should be the main file or one of the child files.
Note that further \textsf{childdoc} directives
such as |\childdocof| and |\childdocforward|
in the indicated file will be processed in this form.
The optional argument \textit{main}
passes on directly to the main file \textit{main}
while pretending to compile the child \textit{dest}.
This form behaves as if \textit{dest}
issues |\childdocof{|\textit{main}|}| right away,
and no further \textsf{childdoc} directives will be processed.

%%%%%%%%%%%%%%%%%%%%%%%%%%%%%%%%%%%%%%%%
\DescribeMacro{\...prefix}
In the alternative form |\childdocforwardprefix|,
%
\begin{center}
\begin{tabular}{l}
|\input{childdoc.def}|\\
|\childdocforwardprefix[|\textit{main}|]{|\textit{prefix}|}{|\textit{dest}|}|
\end{tabular}
\end{center}
%
the destination file is determined by a pattern
depending on the current file:
To make this work, the current file must be called
`{\textit{prefix}\hspace{0.2em}\textit{suffix}}'
with \textit{prefix} matching precisely the argument.
Processing is then passed on to the file
`{\textit{dest}\hspace{0.2em}\textit{suffix}}'.
Surely, the same effect is achieved by
directly specifying the
argument `{\textit{dest}\hspace{0.2em}\textit{suffix}}'
in the first form.
However, that requires to set up a different file
for each child. With the alternative form of the command
all these files can have exactly the same content
which simplifies setting them up and maintaining them.

For example, the following file |draft.tex|
with a compilation flag |\version| as described in \secref{sec:flags}
compiles the main document as a draft:
%
\begin{center}
\begin{tabular}{l}
|\def\version{draft}|\\
|\input{childdoc.def}|\\
|\childdocforward{|\textit{main}|}|
\end{tabular}
\end{center}
%
Likewise, the following files |final|\textit{nn}|.tex|
compile the final version of the child document
|child|\textit{nn}|.tex|:
%
\begin{center}
\begin{tabular}{l}
|\def\version{final}|\\
|\input{childdoc.def}|\\
|\childdocforwardprefix{final}{child}|
\end{tabular}
\end{center}
%

Note that when several versions of a main file and/or of each child file
are to be generated, it may be convenient to set up a |Makefile| or
shell script to automatise the process.

%%%%%%%%%%%%%%%%%%%%%%%%%%%%%%%%%%%%%%%%%%%%%%%%%%%%%%%%%%%%%%%%%%%%%%%%%%%%%%%%
\subsection{Command Line Processing}
\label{sec:commandline}

The effect of redirection files can also be achieved by invoking
the \LaTeX{} compiler with a more elaborate command line.
Most conveniently this should be done as part
of a shell script or a |Makefile|.

When using \textsf{childdoc} in the main file, the following
command lines effectively perform a redirection
(note that depending on the shell being used,
backslashes may have to be doubled: `|\|' $\to$ `|\\|'):
%
\begin{center}
|... -jobname "|\textit{target}|" |\\|"|[\textit{flags}]%
|\input{childdoc.def}\childdocforward[|\textit{main}|]{|\textit{dest}|}"|
\end{center}
%
Here \textit{target} is the name of the output file,
\textit{main} is the name of the main file
and \textit{dest} is the name of the main or child file to be processed
(all filenames without extensions).
The optional argument \textit{main} can be omitted
if \textit{main} matches \textit{dest}.
Optionally, compilation \textit{flags} can be defined via |\def| commands.
This command line makes the \TeX{} engine believe
it is compiling the file \textit{target}
whose content is specified as the latter parameter.
The provided code then forwards the processing to
\textit{main} or \textit{dest} as described in \secref{sec:forward}.

%%%%%%%%%%%%%%%%%%%%%%%%%%%%%%%%%%%%%%%%%%%%%%%%%%%%%%%%%%%%%%%%%%%%%%%%%%%%%%%%
\subsection{Include by Input}
\label{sec:input}

Including child documents by |\include| has some restrictions by design.
Most notably, the content of a child document always occupies
its own set of pages; pages cannot be shared between child documents.
Usually, this behaviour makes perfect sense
because each child document contain an essential part of the document.
However, in some situations it may be desirable to compose
a document from a collection of parts
without having mandatory page breaks between then.
For this case, the package
provides a mechanism to include parts
by |\input| which can also be processed individually.
However, by construction this mechanism
requires manual handling of the content to be output.

%%%%%%%%%%%%%%%%%%%%%%%%%%%%%%%%%%%%%%%%
\DescribeMacro{\ifchilddocmanual}
The main file should be prepared as usual, see \secref{sec:include}.
However, the document body must make a distinction
between processing of an individual part and of the main document, e.g.:
%
\begin{center}
\begin{tabular}{l}
|\ifchilddocmanual|\\
|\input{\childdocname}|\\
|\||else|\\
\textit{document body with }|\input{|\textit{part}|}|\\
|\||fi|
\end{tabular}
\end{center}
%
The conditional |\ifchilddocmanual| is true whenever
a part to be included by |\input| is being compiled,
and the name of the part is stored in |\childdocname|.

%%%%%%%%%%%%%%%%%%%%%%%%%%%%%%%%%%%%%%%%
\DescribeMacro{\childdocby}
Each part to be included by |\input| should start with:
%
\begin{center}
\begin{tabular}{l}
|\input{childdoc.def}|\\
|\childdocby{|\textit{main}|}|\\
\end{tabular}
\end{center}
%
The directive |\childdocby| is similar to |\childdocof|
described in \secref{sec:include},
but the subsequent selection of content must be done manually.
To that end, both |\ifchilddoc| and |\ifchilddocmanual|
will be true upon processing of a part,
and the name of the part is stored in |\childdocname|.
Note that |\jobname| will be set to the filename of the current part
so that each part receives an individual |.aux| file
that does not interfere with the |.aux| file(s) of the main document.
This behaviour can be altered by the alternative form
|\childdocby[*]{|\textit{main}|}| (with a non-empty optional argument)
which uses the |.aux| file of the main document
by setting |\jobname| to \textit{main}.

%%%%%%%%%%%%%%%%%%%%%%%%%%%%%%%%%%%%%%%%%%%%%%%%%%%%%%%%%%%%%%%%%%%%%%%%%%%%%%%%
\subsection{Driver Development}
\label{sec:driver}

The \textsf{childdoc} mechanism can also be use for the development
of definition files such as \LaTeX{} styles or classes.
This case differs from the above setup with multiple parts
included by |\include| in that no |\includeonly| should be invoked.
This can be achieved by starting the include file
(before |\ProvidesPackage|) with:
%
\begin{center}
\begin{tabular}{l}
|\input{childdoc.def}|\\
|\childdocforward{|\textit{main}|}|\\
\end{tabular}
\end{center}
%
or alternatively with:
%
\begin{center}
\begin{tabular}{l}
|\input{childdoc.def}|\\
|\childdocby{|\textit{main}|}|\\
\end{tabular}
\end{center}
%
Both forms have slightly different effects as described above.
The main file is prepared as usual, see \secref{sec:include}.

%%%%%%%%%%%%%%%%%%%%%%%%%%%%%%%%%%%%%%%%%%%%%%%%%%%%%%%%%%%%%%%%%%%%%%%%%%%%%%%%
\subsection{Legacy Detection}
\label{sec:detection}

The directive |\childdocmain| in the main file can detect
whether the complete document or merely a child is to be compiled
even without using the directive |\childdocof|.
This method is deprecated because it is less robust
and there is no compelling reason to use it;
it is merely provided for backward compatibility
and it may be removed in future versions.

If the detection mechanism is to be used,
it is mandatory to correctly specify
the filename of the main file as the argument of |\childdocmain|:
%
\begin{center}
\begin{tabular}{l}
|\input{childdoc.def}|\\
|\childdocmain{|\textit{main}|}|\\
\end{tabular}
\end{center}
%
If |\jobname| does not match the argument \textit{main} of |\childdocmain|,
it is assumed that |\jobname| points to the child file to be compiled.
When using |\childdocmain| with the main file specified as argument,
it suffices to start a child file
with just |\input{|\textit{main}|}|
without loading of the package and using |\childdocof|.
If instead all processing is done
with the appropriate \textsf{childdoc} directives,
the argument of \textit{main} of |\childdocmain| can be empty.

An alternative version of the command line processing described
in \secref{sec:commandline} using the detection mechanism reads:
%
\begin{center}
|... -jobname "|\textit{target}|" "|[\textit{flags}]%
[|\def\jobname{|\textit{dest}|}|]|\input{|\textit{main}|}"|
\end{center}

%%%%%%%%%%%%%%%%%%%%%%%%%%%%%%%%%%%%%%%%%%%%%%%%%%%%%%%%%%%%%%%%%%%%%%%%%%%%%%%%
\subsection{Manual Code}
\label{sec:manual}

In case one cannot be certain whether the definitions file |childdoc.def|
is installed on the target \TeX{} distribution
and one prefers not to ship it,
it is conceivable to paste a few relevant commands into the sources.

To that end, drop all statements |\input{childdoc.def}|
and perform the replacements as outlined below.
Instead of |\childdocmain{|\textit{main}|}| add the following code
to the top of the main file:
%
\begin{center}
\begin{tabular}{l}
|\||ifdefined\childdocname\endinput\||fi\newif\ifchilddoc|\\
|\edef\childdocname{\scantokens\expandafter{\jobname\noexpand}}|\\
|\def\childdocmain{|\textit{main}|}\||ifx\childdocmain\childdocname\||else|\\
|\childdoctrue\includeonly{\childdocname}\let\jobname\childdocmain\||fi|\\
\end{tabular}
\end{center}
%
Instead of |\childdocof{|\textit{main}|}| just include the main file
at the top of each child file:
%
\begin{center}
|\input{|\textit{main}|}|
\end{center}
%
A simple redirection |\childdocforward{|\textit{dest}|}| is achieved by:
%
\begin{center}
|\def\jobname{|\textit{dest}|}\input{\jobname}|
\end{center}
%
The redirection with prefix
|\childdocforwardprefix[|\textit{prefix}|]{|\textit{dest}|}|
is accomplished by:
%
\begin{center}
\begin{tabular}{l}
|{\edef\jobname{\scantokens\expandafter{\jobname\noexpand}}|\\
|\def\redirectjob |\textit{prefix}|#1~~~{\gdef\jobname{|\textit{dest}|#1}}|\\
|\expandafter\redirectjob\jobname~~~}\input{\jobname}|
\end{tabular}
\end{center}

In an alternative approach,
child documents can be compiled by a specific command line
without additional code or specific definitions:
%
\begin{center}
|... -jobname "|\textit{target}|" "|[\textit{flags}]%
|\includeonly{|\textit{dest}|}\input{|\textit{main}|}"|
\end{center}
%

%%%%%%%%%%%%%%%%%%%%%%%%%%%%%%%%%%%%%%%%%%%%%%%%%%%%%%%%%%%%%%%%%%%%%%%%%%%%%%%%
%%%%%%%%%%%%%%%%%%%%%%%%%%%%%%%%%%%%%%%%%%%%%%%%%%%%%%%%%%%%%%%%%%%%%%%%%%%%%%%%
\section{Information}

%%%%%%%%%%%%%%%%%%%%%%%%%%%%%%%%%%%%%%%%%%%%%%%%%%%%%%%%%%%%%%%%%%%%%%%%%%%%%%%%
\subsection{Copyright}

Copyright \copyright{} 2017--2018 Niklas Beisert

This work may be distributed and/or modified under the
conditions of the \LaTeX{} Project Public License, either version 1.3
of this license or (at your option) any later version.
The latest version of this license is in
  \url{http://www.latex-project.org/lppl.txt}
and version 1.3 or later is part of all distributions of \LaTeX{}
version 2005/12/01 or later.

This work has the LPPL maintenance status `maintained'.

The Current Maintainer of this work is Niklas Beisert.

This work consists of the files |README.txt|, |childdoc.ins| and |childdoc.dtx|
as well as the derived files |childdoc.def|, |cdocsamp.tex|
with |cdocsch1.tex|, |cdocsch2.tex|, |cdocspt3.tex|, |cdocspt4.tex|,
|cdocsdrf.tex|, |cdocsfn1.tex|, |cdocsfn2.tex|
as well as |childdoc.pdf|.

%%%%%%%%%%%%%%%%%%%%%%%%%%%%%%%%%%%%%%%%%%%%%%%%%%%%%%%%%%%%%%%%%%%%%%%%%%%%%%%%
\subsection{Files and Installation}

The package consists of the files:
%
\begin{center}
\begin{tabular}{ll}
    |README.txt|   & readme file \\
    |childdoc.ins| & installation file \\
    |childdoc.dtx| & source file \\
    |childdoc.def| & definition file \\
    |cdocsamp.tex| & sample main file \\
    |cdocsch1.tex| & sample include file \\
    |cdocsch2.tex| & sample include file \\
    |cdocspt3.tex| & sample part file \\
    |cdocspt4.tex| & sample part file \\
    |cdocsdrf.tex| & sample redirection file \\
    |cdocsfn1.tex| & sample redirection file \\
    |cdocsfn2.tex| & sample redirection file \\
    |childdoc.pdf| & manual
\end{tabular}
\end{center}
%
The distribution consists of the files
|README.txt|, |childdoc.ins| and |childdoc.dtx|.
%
\begin{itemize}
\item
Run (pdf)\LaTeX{} on |childdoc.dtx|
to compile the manual |childdoc.pdf| (this file).
\item
Run \LaTeX{} on |childdoc.ins| to create the definitions file |childdoc.def|
and the sample |cdocsamp.tex| with include files
|cdocsch1.tex|, |cdocsch2.tex|, |cdocspt3.tex|, |cdocspt4.tex|,
|cdocsdrf.tex|, |cdocsfn1.tex|, |cdocsfn2.tex|.
Then copy the file |childdoc.def| to an appropriate directory of your \LaTeX{}
distribution, e.g.\ \textit{texmf-root}|/tex/latex/childdoc|.
\end{itemize}

%%%%%%%%%%%%%%%%%%%%%%%%%%%%%%%%%%%%%%%%%%%%%%%%%%%%%%%%%%%%%%%%%%%%%%%%%%%%%%%%
\subsection{Related CTAN Packages}

There are several other packages which offer a similar functionality:
%
\begin{itemize}
\item
The packages
\href{http://ctan.org/pkg/docmute}{\textsf{docmute}},
\href{http://ctan.org/pkg/includex}{\textsf{includex}} and
\href{http://ctan.org/pkg/standalone}{\textsf{standalone}}
provide commands to include only the document body of
a child file thus allowing both files to be compiled individually.
\item
The packages \href{http://ctan.org/pkg/subdocs}{\textsf{subdocs}}
and \href{http://ctan.org/pkg/subfiles}{\textsf{subfiles}}
provide structures in which the main and child documents can be
encapsulated and allowing them to be compiled individually.
The inclusion mechanism is different from the conventional |\include|.
\item
The package \href{http://ctan.org/pkg/combine}{\textsf{combine}}
is an elaborate solution to combine several documents into one.
\end{itemize}
%
See also the CTAN topic \href{http://ctan.org/topic/subdocs}{\textsf{subdocs}}
for further related packages.
The present package differs from the above solutions in that
a document structure constructed with the conventional |\include| mechanism
just needs two extra commands at the top of every file
such that all constituent files can be compiled individually.

%%%%%%%%%%%%%%%%%%%%%%%%%%%%%%%%%%%%%%%%%%%%%%%%%%%%%%%%%%%%%%%%%%%%%%%%%%%%%%%%
%\subsection{Feature Suggestions}
%
%The following is a list of features which may be useful for future
%versions of this package:
%%
%\begin{itemize}
%\item
%\ldots
%\end{itemize}

%%%%%%%%%%%%%%%%%%%%%%%%%%%%%%%%%%%%%%%%%%%%%%%%%%%%%%%%%%%%%%%%%%%%%%%%%%%%%%%%
\subsection{Revision History}

%%%%%%%%%%%%%%%%%%%%%%%%%%%%%%%%%%%%%%%%
\paragraph{v2.0:} 2018/12/30

\begin{itemize}
\item
immediate forward processing
\item
added |\childdocby| mechanism
\item
manual restructured
\end{itemize}

%%%%%%%%%%%%%%%%%%%%%%%%%%%%%%%%%%%%%%%%
\paragraph{v1.6:} 2018/01/17

\begin{itemize}
\item
application for development of include files
\item
corrections to manual
\end{itemize}

%%%%%%%%%%%%%%%%%%%%%%%%%%%%%%%%%%%%%%%%
\paragraph{v1.5:} 2017/05/21

\begin{itemize}
\item
more complete structuring introduced
\item
|\childdocof| introduced
\item
|\childdoc| renamed to |\childdocmain|
\item
|\childredirect| renamed to |\childdocforward| and |\childdocforwardprefix|
and functionality expanded
\end{itemize}

%%%%%%%%%%%%%%%%%%%%%%%%%%%%%%%%%%%%%%%%
\paragraph{v1.0:} 2017/04/27

\begin{itemize}
\item
manual and install package
\item
first version published on CTAN
\end{itemize}

%%%%%%%%%%%%%%%%%%%%%%%%%%%%%%%%%%%%%%%%
\paragraph{v0.6:} 2017/04/26

\begin{itemize}
\item
redirection mechanism added
\end{itemize}

%%%%%%%%%%%%%%%%%%%%%%%%%%%%%%%%%%%%%%%%
\paragraph{v0.5:} 2017/04/26

\begin{itemize}
\item
functionality in definition file
\end{itemize}


%%%%%%%%%%%%%%%%%%%%%%%%%%%%%%%%%%%%%%%%%%%%%%%%%%%%%%%%%%%%%%%%%%%%%%%%%%%%%%%%
%%%%%%%%%%%%%%%%%%%%%%%%%%%%%%%%%%%%%%%%%%%%%%%%%%%%%%%%%%%%%%%%%%%%%%%%%%%%%%%%
%%%%%%%%%%%%%%%%%%%%%%%%%%%%%%%%%%%%%%%%%%%%%%%%%%%%%%%%%%%%%%%%%%%%%%%%%%%%%%%%
\appendix

\settowidth\MacroIndent{\rmfamily\scriptsize 000\ }

 \DocInput{childdoc.dtx}

\end{document}
%</driver>
% \fi
%
% %%%%%%%%%%%%%%%%%%%%%%%%%%%%%%%%%%%%%%%%%%%%%%%%%%%%%%%%%%%%%%%%%%%%%%%%%%%%%%
% %%%%%%%%%%%%%%%%%%%%%%%%%%%%%%%%%%%%%%%%%%%%%%%%%%%%%%%%%%%%%%%%%%%%%%%%%%%%%%
% \section{Sample}
%\iffalse
%<*samplemain>
%\fi
%
% The following presents a sample document
% with two chapters, two parts, a title page,
% a compile flag as well as three forwarding files to set the flag.
% It consists of eight |.tex| files:
% \begin{center}
% \begin{tabular}{ll}
% |cdocsamp.tex|&main file\\
% |cdocsch1.tex|&include file for chapter 1\\
% |cdocsch2.tex|&include file for chapter 2\\
% |cdocspt3.tex|&include file for part 3\\
% |cdocspt4.tex|&include file for part 4\\
% |cdocsdrf.tex|&forwarding file for main file in draft mode\\
% |cdocsfi1.tex|&forwarding file for final version of chapter 1\\
% |cdocsfi2.tex|&forwarding file for final version of chapter 2\\
% \end{tabular}
% \end{center}
% Each of the eight files can be compiled directly by the \LaTeX{} compiler.
%
% %%%%%%%%%%%%%%%%%%%%%%%%%%%%%%%%%%%%%%
% \paragraph{Main File.}
%
% The main file is called |cdocsamp.tex|.
%
% Load the \textsf{childdoc} definitions and
% declare the filename for the main document:
%    \begin{macrocode}
\input{childdoc.def}
\childdocmain{}
%    \end{macrocode}

% Optional override for |\version| flag:
%    \begin{macrocode}
%%\ifchilddoc\else\providecommand{\version}{draft}\fi
%    \end{macrocode}

% Define the default values for the |\version| flag
% (|final| for the main file and |draft| for childs):
%    \begin{macrocode}
\ifchilddoc
\providecommand{\version}{draft}
\else
\providecommand{\version}{final}
\fi
%    \end{macrocode}

% Load the standard document class:
%    \begin{macrocode}
\documentclass[12pt]{article}
%    \end{macrocode}

% Start the document body:
%    \begin{macrocode}
\begin{document}
%    \end{macrocode}

% Declare a title page.
% Print title, part of document being processed and version flag:
%    \begin{macrocode}
\addtocounter{page}{-1}
\begin{center}
{\LARGE\bfseries{}childdoc example\par}
\vspace{1cm}
\ifchilddoc
\ifchilddocmanual part\else chapter\fi:
`\childdocname' of `\childdocjob'\par
\else
main document: `\childdocjob'\par
\fi
version: \version\par
\end{center}
\newpage
%    \end{macrocode}

% Manually include selected file,
% otherwise process as usual:
%    \begin{macrocode}
\ifchilddocmanual
\section*{part `\childdocname'}
\input{\childdocname}
\else
%    \end{macrocode}

% Include the two chapters:
%    \begin{macrocode}
\include{cdocsch1}
\include{cdocsch2}
%    \end{macrocode}

% Include the two parts unless only chapters should be displayed:
%    \begin{macrocode}
\ifchilddoc\else
\section{part three}
\input{cdocspt3}
\section{part four}
\input{cdocspt4}
\fi
%    \end{macrocode}

% Process as usual until here:
%    \begin{macrocode}
\fi
%    \end{macrocode}

% End of document body:
%    \begin{macrocode}
\end{document}
%    \end{macrocode}
%\iffalse
%</samplemain>
%\fi
%
% %%%%%%%%%%%%%%%%%%%%%%%%%%%%%%%%%%%%%%
% \paragraph{Chapter Include Files.}
%
% The include files are called |cdocsch1.tex| and |cdocsch2.tex|.
%
%\iffalse
%<*samplechap1|samplechap2>
%\fi

% Optional override for |\version| flag:
%    \begin{macrocode}
%%\providecommand{\version}{final}
%    \end{macrocode}

% Include the main document:
%    \begin{macrocode}
\input{childdoc.def}
\childdocof{cdocsamp}
%    \end{macrocode}

%\iffalse
%</samplechap1|samplechap2>
%\fi
%
%\iffalse
%<*samplechap1>
%\fi
% Some text for chapter 1:
%    \begin{macrocode}
\section{one}
some text in chapter one
%    \end{macrocode}

%\iffalse
%</samplechap1>
%\fi
% Some text for chapter 2:
%\iffalse
%<*samplechap2>
%\fi
%    \begin{macrocode}
\section{two}
more text in chapter two
%    \end{macrocode}

%\iffalse
%</samplechap2>
%\fi
%
% %%%%%%%%%%%%%%%%%%%%%%%%%%%%%%%%%%%%%%
% \paragraph{Part Include Files.}
%
% The include files are called |cdocspt3.tex| and |cdocspt4.tex|.
%
%\iffalse
%<*samplepart3|samplepart4>
%\fi

% Optional override for |\version| flag:
%    \begin{macrocode}
%%\providecommand{\version}{final}
%    \end{macrocode}

% Include the main document:
%    \begin{macrocode}
\input{childdoc.def}
\childdocby{cdocsamp}
%    \end{macrocode}

%\iffalse
%</samplepart3|samplepart4>
%\fi
%
%\iffalse
%<*samplepart3>
%\fi
% Some text for part 3:
%    \begin{macrocode}
some text in part three
%    \end{macrocode}

%\iffalse
%</samplepart3>
%\fi
% Some text for part 4:
%\iffalse
%<*samplepart4>
%\fi
%    \begin{macrocode}
more text in part four
%    \end{macrocode}

%\iffalse
%</samplepart4>
%\fi
%
% %%%%%%%%%%%%%%%%%%%%%%%%%%%%%%%%%%%%%%
% \paragraph{Forwarding for a Complete Draft.}
%
% The following forwarding file |cdocsdrf.tex|
% compiles the main document in draft mode:
%\iffalse
%<*sampledraft>
%\fi
%    \begin{macrocode}
\def\version{draft}
\input{childdoc.def}
\childdocforward{cdocsamp}
%    \end{macrocode}

%\iffalse
%</sampledraft>
%\fi
%
% %%%%%%%%%%%%%%%%%%%%%%%%%%%%%%%%%%%%%%
% \paragraph{Forwarding for Final Version of the Chapters.}
%
% The following forwarding files |cdocsfn1.tex| and |cdocsfn2.tex|
% (with identical content)
% compile the final versions of the child documents
% |cdocsch1.tex| and |cdocsch2.tex|, respectively:
%\iffalse
%<*samplefinal>
%\fi
%    \begin{macrocode}
\def\version{final}
\input{childdoc.def}
\childdocforwardprefix[cdocsamp]{cdocsfn}{cdocsch}
%    \end{macrocode}

%\iffalse
%</samplefinal>
%\fi
%
% %%%%%%%%%%%%%%%%%%%%%%%%%%%%%%%%%%%%%%
% \paragraph{Command Line Processing.}
%
% The following three command lines generate the output files
% |cdocscld|, |cdocscl1| and |cdocscl2|
% which should be identical to
% |cdocsdrf|, |cdocsch1| and |cdocsfn2|, respectively:
% \begin{center}
% \begin{tabular}{l}
% |latex -jobname cdocscld \|\\
% |  "\def\version{draft}\input{childdoc.def}\childdocforward{cdocsamp}"|\\
% |latex -jobname cdocscl1 \|\\
% |  "\input{childdoc.def}\childdocforward[cdocsamp]{cdocsch1}"|\\
% |latex -jobname cdocscl2 \|\\
% |  "\def\version{final}\input{childdoc.def}\childdocforward{cdocsch2}"|
% \end{tabular}
% \end{center}
% Note that the trailing backslash on each first line
% merely continues the input to the second line
% (for convenient cut ant paste).
% Furthermore, the command |latex| can be replaced by any
% of its alternative versions such as |pdflatex|.
%
% %%%%%%%%%%%%%%%%%%%%%%%%%%%%%%%%%%%%%%%%%%%%%%%%%%%%%%%%%%%%%%%%%%%%%%%%%%%%%%
% %%%%%%%%%%%%%%%%%%%%%%%%%%%%%%%%%%%%%%%%%%%%%%%%%%%%%%%%%%%%%%%%%%%%%%%%%%%%%%
% \section{Implementation}
%\iffalse
%<*package>
%\fi
%
% This section describes the definitions file |childdoc.def|.

% The definitions cannot be loaded using |\usepackage| or |\RequirePackage|
% which has a mechanism to prevent loading a style file more than once.
% When loading the definitions by means of |\input|
% multiple instances have to be prevented manually:
%\iffalse
%This code needs to be before the `\ProvidesFile' directive
%which is defined at the beginning of this file.
%Therefore it is also placed there and commented out here.
%</package>
%<*discard>
%\fi
%    \begin{macrocode}
\ifdefined\childdocmain\endinput\fi
%    \end{macrocode}
%\iffalse
%</discard>
%<*package>
%\fi
%
% \macro{\ifchilddoc}
% \macro{\ifchilddocmanual}
% The conditional |\ifchilddoc| tells whether a
% child (true) or main (false) document is being compiled.
% The conditional |\ifchilddocmanual| tells whether
% the |\includeonly| mechanism is used (false) or
% the selection of child files must be performed manually (true).
% The definitions initialise to false:
%    \begin{macrocode}
\newif\ifchilddoc
\newif\ifchilddocmanual
%    \end{macrocode}

% \macro{\childdocname}
% \macro{\childdocjob}
% The macro |\childdocname| stores the name of the main document
% to be compiled. The macro |\childdocjob| stores the name of
% the document on which the \LaTeX{} compiler was originally invoked.
% The content of |\jobname| cannot be compared
% to filenames specified in the source due to different catcodes.
% The following code rescans |\jobname|, stores the result
% in |\childdocname| and saves a copy in |\childdocjob|:
%    \begin{macrocode}
\edef\childdocname{\scantokens\expandafter{\jobname\noexpand}}
\let\childdocjob\childdocname
%    \end{macrocode}

% \macro{\childdocdisable}
% The macro |\childdocdisable| prevents the main file
% from being processed more than once.
% At this stage, the main document command |\childdocmain|
% is assumed to be called once again where it should do nothing.
% Any subsequent call to it should prevent
% a secondary processing of the main document
% It overwrites the forwarding commands
% |\childdocof| and |\childdocforward|
% with empty macros to prevent further inclusions of the main document:
%    \begin{macrocode}
\newcommand{\childdocdisable}
{
  \renewcommand{\childdocmain}[1]{\renewcommand{\childdocmain}[1]{\endinput}}
  \renewcommand{\childdocof}[1]{}
  \renewcommand{\childdocby}[2][]{}
  \renewcommand{\childdocforward}[2][]{}
  \renewcommand{\childdocdisable}{}
}
%    \end{macrocode}

% \macro{\childdocmain}
% The macro |\childdocmain| is to be called at the top of the main file
% with nothing or the main filename (without extension) as argument.
% First, it breaks loops.
% If the argument is not empty and does not match |\childdocname|
% (which is set by the first inclusion of |childdoc.def|),
% |\ifchilddoc| is set to true, |\includeonly| is applied to the child file
% and |\jobname| is set to the main file
% (for proper handling of |.aux| files):
%    \begin{macrocode}
\newcommand{\childdocmain}[1]
{
  \childdocdisable\childdocmain{}
  \if?#1?\else
    \begingroup
      \def\childdoctmp{#1}
      \ifx\childdoctmp\childdocname
        \def\childdoctmp{}
      \else
        \def\childdoctmp
        {
          \childdoctrue
          \includeonly{\childdocname}
          \def\childdocjob{#1}
          \def\jobname{#1}
        }
      \fi
      \expandafter
    \endgroup
    \childdoctmp
  \fi
}
%    \end{macrocode}

% \macro{\childdocof}
% The command |\childdocof| redirects
% compilation to the main file |#1|.
%    \begin{macrocode}
\newcommand{\childdocof}[1]
{
  \childdocdisable
  \childdoctrue
  \includeonly{\childdocname}
  \def\jobname{#1}
  \def\childdocjob{#1}
  \input{#1}
}
%    \end{macrocode}

% \macro{\childdocby}
% The command |\childdocby| ....
%    \begin{macrocode}
\newcommand{\childdocby}[2][]
{
  \childdocdisable
  \childdoctrue
  \childdocmanualtrue
  \if?#1?\else
    \def\jobname{#2}
  \fi
  \def\childdocjob{#2}
  \input{#2}
  \endinput
}
%    \end{macrocode}

% \macro{\childdocforward}
% The command |\childdocforward| redirects
% compilation to the main file or
% (if the optional argument is given) a child file.
% Parameters are set as if the main file
% or a child file starting with |\childdocof| was compiled.
% Then compilation is handed over to the main file:
%    \begin{macrocode}
\newcommand{\childdocforward}[2][]
{
  \begingroup
    \if?#1?
      \def\childdoctmp
      {
        \def\childdocname{#2}
        \def\childdocjob{#2}
        \def\jobname{#2}
        \input{#2}
        \endinput
      }
    \else
      \def\childdoctmp
      {
        \childdocdisable
        \def\childdocname{#2}
        \childdoctrue
        \includeonly{#2}
        \def\childdocjob{#1}
        \def\jobname{#1}
        \input{#1}
        \endinput
      }
    \fi
    \expandafter
  \endgroup
  \childdoctmp
}
%    \end{macrocode}

% \macro{\childdocforwardprefix}
% The command |\childdocforwardprefix| redirects
% compilation to the main or a child file by means of a pattern.
% The prefix |#1| in the current filename is replaced by |#2|
% and the suffix of the current filename is kept
% (it is assumed that the filename does not contain the substring `|~~~|'
% which is used as a delimiter).
% Compilation is handed over to the new file by |\childdocforward|:
%    \begin{macrocode}
\newcommand{\childdocforwardprefix}[3][]
{
  \begingroup
    \def\childdocextract #2##1~~~{\def\childdoctmp{\childdocforward[#1]{#3##1}}}
    \expandafter\childdocextract\childdocname~~~
    \expandafter
  \endgroup
  \childdoctmp
}
%    \end{macrocode}

% \macro{\childdoc}
% The deprecated macro |\childdoc| is a legacy version of |\childdocmain|:
%    \begin{macrocode}
\newcommand{\childdoc}{\childdocmain}
%    \end{macrocode}

% \macro{\childdocredirect}
% The deprecated macro |\childdocredirect| is a legacy version
% of |\childdocforward| and |\childdocforwardprefix|:
%    \begin{macrocode}
\newcommand{\childdocredirect}[2][]
{
  \begingroup
    \if?#1?
      \def\childdoctmp{\childdocforward{#2}}
    \else
      \def\childdoctmp{\childdocforwardprefix{#1}{#2}}
    \fi
    \expandafter
  \endgroup
  \childdoctmp
}
%    \end{macrocode}

%\iffalse
%</package>
%\fi
%
\endinput

\childdocby{cdocsamp}
%    \end{macrocode}

%\iffalse
%</samplepart3|samplepart4>
%\fi
%
%\iffalse
%<*samplepart3>
%\fi
% Some text for part 3:
%    \begin{macrocode}
some text in part three
%    \end{macrocode}

%\iffalse
%</samplepart3>
%\fi
% Some text for part 4:
%\iffalse
%<*samplepart4>
%\fi
%    \begin{macrocode}
more text in part four
%    \end{macrocode}

%\iffalse
%</samplepart4>
%\fi
%
% %%%%%%%%%%%%%%%%%%%%%%%%%%%%%%%%%%%%%%
% \paragraph{Forwarding for a Complete Draft.}
%
% The following forwarding file |cdocsdrf.tex|
% compiles the main document in draft mode:
%\iffalse
%<*sampledraft>
%\fi
%    \begin{macrocode}
\def\version{draft}
% \iffalse
%
% childdoc.dtx Copyright (C) 2017-2018 Niklas Beisert
%
% This work may be distributed and/or modified under the
% conditions of the LaTeX Project Public License, either version 1.3
% of this license or (at your option) any later version.
% The latest version of this license is in
%   http://www.latex-project.org/lppl.txt
% and version 1.3 or later is part of all distributions of LaTeX
% version 2005/12/01 or later.
%
% This work has the LPPL maintenance status `maintained'.
%
% The Current Maintainer of this work is Niklas Beisert.
%
% This work consists of the files childdoc.dtx and childdoc.ins
% and the derived files childdoc.def and cdocsamp.tex with
% cdocsch1.tex, cdocsch2.tex, cdocsdrf.tex, cdocsfn1.tex, cdocsfn2.tex.
%
%<package>\ifdefined\childdocmain\endinput\fi
%<package>\ProvidesFile{childdoc.def}[2018/12/30 v2.0 child document driver]
%<samplemain>\ProvidesFile{cdocsamp.tex}[2018/12/30 v2.0 sample for childdoc]
%<*driver>
%\ProvidesFile{childdoc.drv}[2018/12/30 v2.0 childdoc reference manual file]
\PassOptionsToClass{10pt,a4paper}{article}
\documentclass{ltxdoc}

\usepackage[margin=35mm]{geometry}
\usepackage{hyperref}
\usepackage{hyperxmp}
\usepackage[usenames]{color}

\hypersetup{colorlinks=true}
\hypersetup{pdfstartview=FitH}
\hypersetup{pdfpagemode=UseNone}
\hypersetup{pdfsource={}}
\hypersetup{pdflang={en-UK}}
\hypersetup{pdfcopyright={Copyright 2017-2018 Niklas Beisert.
  This work may be distributed and/or modified under the
  conditions of the LaTeX Project Public License, either version 1.3
  of this license or (at your option) any later version.}}
\hypersetup{pdflicenseurl={http://www.latex-project.org/lppl.txt}}
\hypersetup{pdfcontactaddress={ETH Zurich, ITP, HIT K,
  Wolfgang-Pauli-Strasse 27}}
\hypersetup{pdfcontactpostcode={8093}}
\hypersetup{pdfcontactcity={Zurich}}
\hypersetup{pdfcontactcountry={Switzerland}}
\hypersetup{pdfcontactemail={nbeisert@itp.phys.ethz.ch}}
\hypersetup{pdfcontacturl={http://people.phys.ethz.ch/\xmptilde nbeisert/}}

\newcommand{\secref}[1]{\hyperref[#1]{section \ref*{#1}}}

\parskip1ex
\parindent0pt
\let\olditemize\itemize
\def\itemize{\olditemize\parskip0pt}

\begin{document}

\title{The \textsf{childdoc} Package}
\hypersetup{pdftitle={The childdoc Package}}
\author{Niklas Beisert\\[2ex]
  Institut f\"ur Theoretische Physik\\
  Eidgen\"ossische Technische Hochschule Z\"urich\\
  Wolfgang-Pauli-Strasse 27, 8093 Z\"urich, Switzerland\\[1ex]
  \href{mailto:nbeisert@itp.phys.ethz.ch}
  {\texttt{nbeisert@itp.phys.ethz.ch}}}
\hypersetup{pdfauthor={Niklas Beisert}}
\hypersetup{pdfsubject={Manual for the LaTeX2e Package childdoc}}
\date{30 December 2018, \textsf{v2.0}}
\maketitle

\begin{abstract}\noindent
\textsf{childdoc} is a \LaTeXe{} package
that enables the direct compilation
of document sections included by |\include|
to individual files.
\end{abstract}

\begingroup
\parskip0ex
\tableofcontents
\endgroup

%%%%%%%%%%%%%%%%%%%%%%%%%%%%%%%%%%%%%%%%%%%%%%%%%%%%%%%%%%%%%%%%%%%%%%%%%%%%%%%%
%%%%%%%%%%%%%%%%%%%%%%%%%%%%%%%%%%%%%%%%%%%%%%%%%%%%%%%%%%%%%%%%%%%%%%%%%%%%%%%%
\section{Introduction}

\LaTeX{} provides a mechanism to structure a large document (such as a book)
into a main file and several child files (containing the chapters)
using the |\include| command.
This mechanism is beneficial for documents
which span hundreds of pages in order to
make the source file(s) more manageable.
Moreover, compilation can be restricted to
selected child files by means of the |\includeonly| command.
The latter feature can be used to reduce the compilation time while editing
(this was significantly more useful in the earlier days of \LaTeX{})
or to generate a smaller document which is easier to navigate.
Another application of |\includeonly| is to generate
documents consisting of selected parts of the complete document.

However, there are a few drawbacks of the plain |\include| mechanism:
\begin{itemize}
\item
The child files cannot be compiled on their own,
they can only be compiled via the main file.
A naive editing environment
(such as a text editor with an option
to have the current file processed by \LaTeX)
may require one to switch to the main file before compiling;
attempting to compile the child file produces errors.
\item
The main file must be modified (each time)
to adjust the |\includeonly| command
to the present needs. This easily leaves the main file in a messy state.
\item
The generated document will always carry the filename
of the main document. This is inconvenient if
several child files are to be compiled and
to be kept for distribution.
\end{itemize}

The present package provides a simple interface
to make child files individually compilable by \LaTeX{}.
Compiling a child file then has the same effect as compiling
the main file with an |\includeonly| command
to select the appropriate child.
Moreover the generated document will carry the name of the child
rather than the main file.
This resolves all three above issues.

This feature is meant to make the editing of books,
thesis documents and lecture notes somewhat more convenient.
However, the package can also be used efficiently for
composing a series of documents (such as exercise sheets)
which are typically distributed individually.
It then assists the author in generating the individual documents
(potentially in different versions)
as well as a document containing the collected series.
Another application is in developing style files
or other kinds of included material
where compilation of the style file could redirect
to a sample or test file.

%%%%%%%%%%%%%%%%%%%%%%%%%%%%%%%%%%%%%%%%%%%%%%%%%%%%%%%%%%%%%%%%%%%%%%%%%%%%%%%%
%%%%%%%%%%%%%%%%%%%%%%%%%%%%%%%%%%%%%%%%%%%%%%%%%%%%%%%%%%%%%%%%%%%%%%%%%%%%%%%%
\section{Usage}

First of all, the package \textsf{childdoc} is \emph{not} a standard
\LaTeXe{} |.sty| style file! Therefore it needs to be invoked in
a non-standard way.

%%%%%%%%%%%%%%%%%%%%%%%%%%%%%%%%%%%%%%%%%%%%%%%%%%%%%%%%%%%%%%%%%%%%%%%%%%%%%%%%
\subsection{Included Files}
\label{sec:include}

%%%%%%%%%%%%%%%%%%%%%%%%%%%%%%%%%%%%%%%%
\DescribeMacro{\childdocmain}
To use the package, add the commands
\begin{center}
\begin{tabular}{l}
|\input{childdoc.def}|\\
|\childdocmain{}|\\
\end{tabular}
\end{center}
at the very top of the main \LaTeX{} file,
in particular \emph{before} the |\documentclass| statement!
The argument of |\childdocmain| should be left empty
(but it must be present).

%%%%%%%%%%%%%%%%%%%%%%%%%%%%%%%%%%%%%%%%
\DescribeMacro{\childdocof}
Furthermore, add the commands
\begin{center}
\begin{tabular}{l}
|\input{childdoc.def}|\\
|\childdocof{|\textit{main}|}|\\
\end{tabular}
\end{center}
at the top of every child file \textit{child}
which is included by |\include{|\textit{child}|}|
from within the main file
(or at least for those files to be compiled individually).
The argument \textit{main} must be the filename of the main file.

There are a couple of
considerations in setting up the main and child documents:

%%%%%%%%%%%%%%%%%%%%%%%%%%%%%%%%%%%%%%%%
\paragraph{Restrictions.}

Please note the following restrictions:
\begin{itemize}
\item
|\childdocmain| must be called with one argument \textit{main}
to ensure compatibility with earlier version of the package.
It must either be empty (|\childdocmain{}|)
or precisely match the filename of the main file in which it is specified.
See \secref{sec:detection} for further information.
\item
The filename \textit{main} must be specified without the |.tex| extension.
\item
The filename \textit{main} is case sensitive
(even in case-insensitive file systems)
due to internal string comparison.
\item
The argument \textit{main} should be fully expanded, it cannot be a macro.
\item
Subdirectories and special characters should be avoided in filenames.
\item
The command |\childdocmain{|\textit{main}|}| must be followed by a whitespace.
It should not be followed immediately by another command
or by a comment mark `|%|'.
This is because the \TeX{} parser reads the token immediately following
the argument of |\childdocmain| and puts it
at the beginning of every child section;
however, a white\-space is ignored.
\end{itemize}

%%%%%%%%%%%%%%%%%%%%%%%%%%%%%%%%%%%%%%%%
\paragraph{Content of Main File.}

It is advisable to place all content in the child files included by |\include|.
Any output contained in the main file will appear in all child documents
unless suppressed manually;
it cannot be suppressed automatically by the |\includeonly| directive
and thus should normally be avoided.
A method to include some content in the main file
by means of conditional processing is described in \secref{sec:conditional}.

%%%%%%%%%%%%%%%%%%%%%%%%%%%%%%%%%%%%%%%%
\paragraph{Page Numbering.}

When only a part of the document is compiled,
the appropriate numbering of pages
(as well as other status parameters)
is determined from the |.aux| files.
The latter contain information from previous passes.
However this information needs to propagate through
all intermediate child documents.
Therefore the page numbering in child documents may well
be inconsistent until the complete document is compiled at least once.

A useful (if unconventional) way to always ensure a consistent
page numbering is to restart the numbering in each child document
and denote the pages by `\textit{child}|.|\textit{page}'
where \textit{child} represents the chapter/section number of the child file.
This can be achieved by the command
|\numberwithin{page}{|\textit{child}|}|
of the \textsf{amsmath} package
where \textit{child} can be |chapter| or |section|
depending on the chosen structuring.
Alternatively, one can modify the macro |\thepage| appropriately
and reset the counter |page| at the start of each child file.

%%%%%%%%%%%%%%%%%%%%%%%%%%%%%%%%%%%%%%%%%%%%%%%%%%%%%%%%%%%%%%%%%%%%%%%%%%%%%%%%
\subsection{Conditional Processing}
\label{sec:conditional}

The package provides a mechanism to compile different versions
of a document. To customise the versions further some conditional processing
can come in handy to distinguish which version is being compiled.
The package provides two macros to describe the compilation context:

%%%%%%%%%%%%%%%%%%%%%%%%%%%%%%%%%%%%%%%%
\DescribeMacro{\ifchilddoc}
The conditional |\ifchilddoc| distinguishes between the compilation of
child documents and the main document:
%
\begin{center}
|\ifchilddoc |\textit{child-code}| |[|\||else |\textit{main-code}]| \||fi|
\end{center}

%%%%%%%%%%%%%%%%%%%%%%%%%%%%%%%%%%%%%%%%
\DescribeMacro{\childdocname}
\DescribeMacro{\childdocjob}
The macro |\childdocname| contains the filename (without extension)
of the main or child file being processed.
Note that |\childdocjob| will always contain the name of the main file.

%%%%%%%%%%%%%%%%%%%%%%%%%%%%%%%%%%%%%%%%
\paragraph{Title Page.}

Conditional processing can be used to include a title or banner page
in the main document when proper precautions are taken.
Importantly, the code in the main file should ensure that the page counter
(as well as other status parameters which are stored in the |.aux| files)
takes the same value after the conditional processing.
Otherwise the page numbers may take divergent values
depending on which part is compiled.

For example, a title page could be declared by:
%
\begin{center}
\begin{tabular}{l}
|\ifchilddoc\||else|\\
|\addtocounter{page}{-1}|\\
\textit{code for title page}\\
|\newpage|\\
|\||fi|
\end{tabular}
\end{center}
%
A banner page for the child documents can be generated by:
%
\begin{center}
\begin{tabular}{l}
|\ifchilddoc|\\
|\addtocounter{page}{-1}|\\
\textit{code for banner page}\\
|\newpage|\\
|\||fi|
\end{tabular}
\end{center}
%
Here one could write a message such as:
\begin{center}
|This is the part \childdocname{} of \childdocjob{}.|
\end{center}

%%%%%%%%%%%%%%%%%%%%%%%%%%%%%%%%%%%%%%%%%%%%%%%%%%%%%%%%%%%%%%%%%%%%%%%%%%%%%%%%
\subsection{Flags}
\label{sec:flags}

The package makes it easy to generate different versions
of the main or child documents.
To this end compilation flags can be defined
and assigned different default values.
They will be particularly useful in conjunction
with the forwarding mechanism described in \secref{sec:forward}.

For example, it may be useful to have a flag |\version|
which can be set to |draft| or |final|.
The document source will contain some conditional code
depending on the value of |\version|.
Suppose further, the flag should default to |final| for the main file
and to |draft| for child files
which is a natural assignment for editing the document.
This is achieved by placing the following code
in the preamble of the main document
(below the |\childdocmain| directive):
%
\begin{center}
\begin{tabular}{l}
|\ifchilddoc|\\
|\providecommand{\version}{draft}|\\
|\||else|\\
|\providecommand{\version}{final}|\\
|\||fi|
\end{tabular}
\end{center}
%
The definition by |\providecommand| makes sure
that previous definitions are not overwritten.
Further statements |\providecommand{\version}{...}|
can thus be added before the above code to override it.

For the main file, one might add a line
(between |\childdocmain| and the above block)
%
\begin{center}
|%\ifchilddoc\||else\providecommand{\version}{draft}\||fi|
\end{center}
%
which can be uncommented to produce a draft version.
Likewise one can add a line to the very top of a child file
(above the |\childdocof{|\textit{main}|}| directive)
%
\begin{center}
|%\providecommand{\version}{final}|
\end{center}
%
which can be uncommented to produce the final version of this child document.

%%%%%%%%%%%%%%%%%%%%%%%%%%%%%%%%%%%%%%%%%%%%%%%%%%%%%%%%%%%%%%%%%%%%%%%%%%%%%%%%
\subsection{Forwarding}
\label{sec:forward}

Different versions of the main or child documents
using compilation flags as described in \secref{sec:flags}
can be (permanently) stored in different files
for convenient compilation, viewing and distribution.
To this end, the package defines a command
to pass on compilation to a different file:

%%%%%%%%%%%%%%%%%%%%%%%%%%%%%%%%%%%%%%%%
\DescribeMacro{\childdocforward}
The command |\childdocforward| redirects processing to
another source file:
%
\begin{center}
\begin{tabular}{l}
|\input{childdoc.def}|\\
|\childdocforward[|\textit{main}|]{|\textit{dest}|}|\\
\end{tabular}
\end{center}
%
The argument \textit{dest} is the destination file
(without extension).
It should be the main file or one of the child files.
Note that further \textsf{childdoc} directives
such as |\childdocof| and |\childdocforward|
in the indicated file will be processed in this form.
The optional argument \textit{main}
passes on directly to the main file \textit{main}
while pretending to compile the child \textit{dest}.
This form behaves as if \textit{dest}
issues |\childdocof{|\textit{main}|}| right away,
and no further \textsf{childdoc} directives will be processed.

%%%%%%%%%%%%%%%%%%%%%%%%%%%%%%%%%%%%%%%%
\DescribeMacro{\...prefix}
In the alternative form |\childdocforwardprefix|,
%
\begin{center}
\begin{tabular}{l}
|\input{childdoc.def}|\\
|\childdocforwardprefix[|\textit{main}|]{|\textit{prefix}|}{|\textit{dest}|}|
\end{tabular}
\end{center}
%
the destination file is determined by a pattern
depending on the current file:
To make this work, the current file must be called
`{\textit{prefix}\hspace{0.2em}\textit{suffix}}'
with \textit{prefix} matching precisely the argument.
Processing is then passed on to the file
`{\textit{dest}\hspace{0.2em}\textit{suffix}}'.
Surely, the same effect is achieved by
directly specifying the
argument `{\textit{dest}\hspace{0.2em}\textit{suffix}}'
in the first form.
However, that requires to set up a different file
for each child. With the alternative form of the command
all these files can have exactly the same content
which simplifies setting them up and maintaining them.

For example, the following file |draft.tex|
with a compilation flag |\version| as described in \secref{sec:flags}
compiles the main document as a draft:
%
\begin{center}
\begin{tabular}{l}
|\def\version{draft}|\\
|\input{childdoc.def}|\\
|\childdocforward{|\textit{main}|}|
\end{tabular}
\end{center}
%
Likewise, the following files |final|\textit{nn}|.tex|
compile the final version of the child document
|child|\textit{nn}|.tex|:
%
\begin{center}
\begin{tabular}{l}
|\def\version{final}|\\
|\input{childdoc.def}|\\
|\childdocforwardprefix{final}{child}|
\end{tabular}
\end{center}
%

Note that when several versions of a main file and/or of each child file
are to be generated, it may be convenient to set up a |Makefile| or
shell script to automatise the process.

%%%%%%%%%%%%%%%%%%%%%%%%%%%%%%%%%%%%%%%%%%%%%%%%%%%%%%%%%%%%%%%%%%%%%%%%%%%%%%%%
\subsection{Command Line Processing}
\label{sec:commandline}

The effect of redirection files can also be achieved by invoking
the \LaTeX{} compiler with a more elaborate command line.
Most conveniently this should be done as part
of a shell script or a |Makefile|.

When using \textsf{childdoc} in the main file, the following
command lines effectively perform a redirection
(note that depending on the shell being used,
backslashes may have to be doubled: `|\|' $\to$ `|\\|'):
%
\begin{center}
|... -jobname "|\textit{target}|" |\\|"|[\textit{flags}]%
|\input{childdoc.def}\childdocforward[|\textit{main}|]{|\textit{dest}|}"|
\end{center}
%
Here \textit{target} is the name of the output file,
\textit{main} is the name of the main file
and \textit{dest} is the name of the main or child file to be processed
(all filenames without extensions).
The optional argument \textit{main} can be omitted
if \textit{main} matches \textit{dest}.
Optionally, compilation \textit{flags} can be defined via |\def| commands.
This command line makes the \TeX{} engine believe
it is compiling the file \textit{target}
whose content is specified as the latter parameter.
The provided code then forwards the processing to
\textit{main} or \textit{dest} as described in \secref{sec:forward}.

%%%%%%%%%%%%%%%%%%%%%%%%%%%%%%%%%%%%%%%%%%%%%%%%%%%%%%%%%%%%%%%%%%%%%%%%%%%%%%%%
\subsection{Include by Input}
\label{sec:input}

Including child documents by |\include| has some restrictions by design.
Most notably, the content of a child document always occupies
its own set of pages; pages cannot be shared between child documents.
Usually, this behaviour makes perfect sense
because each child document contain an essential part of the document.
However, in some situations it may be desirable to compose
a document from a collection of parts
without having mandatory page breaks between then.
For this case, the package
provides a mechanism to include parts
by |\input| which can also be processed individually.
However, by construction this mechanism
requires manual handling of the content to be output.

%%%%%%%%%%%%%%%%%%%%%%%%%%%%%%%%%%%%%%%%
\DescribeMacro{\ifchilddocmanual}
The main file should be prepared as usual, see \secref{sec:include}.
However, the document body must make a distinction
between processing of an individual part and of the main document, e.g.:
%
\begin{center}
\begin{tabular}{l}
|\ifchilddocmanual|\\
|\input{\childdocname}|\\
|\||else|\\
\textit{document body with }|\input{|\textit{part}|}|\\
|\||fi|
\end{tabular}
\end{center}
%
The conditional |\ifchilddocmanual| is true whenever
a part to be included by |\input| is being compiled,
and the name of the part is stored in |\childdocname|.

%%%%%%%%%%%%%%%%%%%%%%%%%%%%%%%%%%%%%%%%
\DescribeMacro{\childdocby}
Each part to be included by |\input| should start with:
%
\begin{center}
\begin{tabular}{l}
|\input{childdoc.def}|\\
|\childdocby{|\textit{main}|}|\\
\end{tabular}
\end{center}
%
The directive |\childdocby| is similar to |\childdocof|
described in \secref{sec:include},
but the subsequent selection of content must be done manually.
To that end, both |\ifchilddoc| and |\ifchilddocmanual|
will be true upon processing of a part,
and the name of the part is stored in |\childdocname|.
Note that |\jobname| will be set to the filename of the current part
so that each part receives an individual |.aux| file
that does not interfere with the |.aux| file(s) of the main document.
This behaviour can be altered by the alternative form
|\childdocby[*]{|\textit{main}|}| (with a non-empty optional argument)
which uses the |.aux| file of the main document
by setting |\jobname| to \textit{main}.

%%%%%%%%%%%%%%%%%%%%%%%%%%%%%%%%%%%%%%%%%%%%%%%%%%%%%%%%%%%%%%%%%%%%%%%%%%%%%%%%
\subsection{Driver Development}
\label{sec:driver}

The \textsf{childdoc} mechanism can also be use for the development
of definition files such as \LaTeX{} styles or classes.
This case differs from the above setup with multiple parts
included by |\include| in that no |\includeonly| should be invoked.
This can be achieved by starting the include file
(before |\ProvidesPackage|) with:
%
\begin{center}
\begin{tabular}{l}
|\input{childdoc.def}|\\
|\childdocforward{|\textit{main}|}|\\
\end{tabular}
\end{center}
%
or alternatively with:
%
\begin{center}
\begin{tabular}{l}
|\input{childdoc.def}|\\
|\childdocby{|\textit{main}|}|\\
\end{tabular}
\end{center}
%
Both forms have slightly different effects as described above.
The main file is prepared as usual, see \secref{sec:include}.

%%%%%%%%%%%%%%%%%%%%%%%%%%%%%%%%%%%%%%%%%%%%%%%%%%%%%%%%%%%%%%%%%%%%%%%%%%%%%%%%
\subsection{Legacy Detection}
\label{sec:detection}

The directive |\childdocmain| in the main file can detect
whether the complete document or merely a child is to be compiled
even without using the directive |\childdocof|.
This method is deprecated because it is less robust
and there is no compelling reason to use it;
it is merely provided for backward compatibility
and it may be removed in future versions.

If the detection mechanism is to be used,
it is mandatory to correctly specify
the filename of the main file as the argument of |\childdocmain|:
%
\begin{center}
\begin{tabular}{l}
|\input{childdoc.def}|\\
|\childdocmain{|\textit{main}|}|\\
\end{tabular}
\end{center}
%
If |\jobname| does not match the argument \textit{main} of |\childdocmain|,
it is assumed that |\jobname| points to the child file to be compiled.
When using |\childdocmain| with the main file specified as argument,
it suffices to start a child file
with just |\input{|\textit{main}|}|
without loading of the package and using |\childdocof|.
If instead all processing is done
with the appropriate \textsf{childdoc} directives,
the argument of \textit{main} of |\childdocmain| can be empty.

An alternative version of the command line processing described
in \secref{sec:commandline} using the detection mechanism reads:
%
\begin{center}
|... -jobname "|\textit{target}|" "|[\textit{flags}]%
[|\def\jobname{|\textit{dest}|}|]|\input{|\textit{main}|}"|
\end{center}

%%%%%%%%%%%%%%%%%%%%%%%%%%%%%%%%%%%%%%%%%%%%%%%%%%%%%%%%%%%%%%%%%%%%%%%%%%%%%%%%
\subsection{Manual Code}
\label{sec:manual}

In case one cannot be certain whether the definitions file |childdoc.def|
is installed on the target \TeX{} distribution
and one prefers not to ship it,
it is conceivable to paste a few relevant commands into the sources.

To that end, drop all statements |\input{childdoc.def}|
and perform the replacements as outlined below.
Instead of |\childdocmain{|\textit{main}|}| add the following code
to the top of the main file:
%
\begin{center}
\begin{tabular}{l}
|\||ifdefined\childdocname\endinput\||fi\newif\ifchilddoc|\\
|\edef\childdocname{\scantokens\expandafter{\jobname\noexpand}}|\\
|\def\childdocmain{|\textit{main}|}\||ifx\childdocmain\childdocname\||else|\\
|\childdoctrue\includeonly{\childdocname}\let\jobname\childdocmain\||fi|\\
\end{tabular}
\end{center}
%
Instead of |\childdocof{|\textit{main}|}| just include the main file
at the top of each child file:
%
\begin{center}
|\input{|\textit{main}|}|
\end{center}
%
A simple redirection |\childdocforward{|\textit{dest}|}| is achieved by:
%
\begin{center}
|\def\jobname{|\textit{dest}|}\input{\jobname}|
\end{center}
%
The redirection with prefix
|\childdocforwardprefix[|\textit{prefix}|]{|\textit{dest}|}|
is accomplished by:
%
\begin{center}
\begin{tabular}{l}
|{\edef\jobname{\scantokens\expandafter{\jobname\noexpand}}|\\
|\def\redirectjob |\textit{prefix}|#1~~~{\gdef\jobname{|\textit{dest}|#1}}|\\
|\expandafter\redirectjob\jobname~~~}\input{\jobname}|
\end{tabular}
\end{center}

In an alternative approach,
child documents can be compiled by a specific command line
without additional code or specific definitions:
%
\begin{center}
|... -jobname "|\textit{target}|" "|[\textit{flags}]%
|\includeonly{|\textit{dest}|}\input{|\textit{main}|}"|
\end{center}
%

%%%%%%%%%%%%%%%%%%%%%%%%%%%%%%%%%%%%%%%%%%%%%%%%%%%%%%%%%%%%%%%%%%%%%%%%%%%%%%%%
%%%%%%%%%%%%%%%%%%%%%%%%%%%%%%%%%%%%%%%%%%%%%%%%%%%%%%%%%%%%%%%%%%%%%%%%%%%%%%%%
\section{Information}

%%%%%%%%%%%%%%%%%%%%%%%%%%%%%%%%%%%%%%%%%%%%%%%%%%%%%%%%%%%%%%%%%%%%%%%%%%%%%%%%
\subsection{Copyright}

Copyright \copyright{} 2017--2018 Niklas Beisert

This work may be distributed and/or modified under the
conditions of the \LaTeX{} Project Public License, either version 1.3
of this license or (at your option) any later version.
The latest version of this license is in
  \url{http://www.latex-project.org/lppl.txt}
and version 1.3 or later is part of all distributions of \LaTeX{}
version 2005/12/01 or later.

This work has the LPPL maintenance status `maintained'.

The Current Maintainer of this work is Niklas Beisert.

This work consists of the files |README.txt|, |childdoc.ins| and |childdoc.dtx|
as well as the derived files |childdoc.def|, |cdocsamp.tex|
with |cdocsch1.tex|, |cdocsch2.tex|, |cdocspt3.tex|, |cdocspt4.tex|,
|cdocsdrf.tex|, |cdocsfn1.tex|, |cdocsfn2.tex|
as well as |childdoc.pdf|.

%%%%%%%%%%%%%%%%%%%%%%%%%%%%%%%%%%%%%%%%%%%%%%%%%%%%%%%%%%%%%%%%%%%%%%%%%%%%%%%%
\subsection{Files and Installation}

The package consists of the files:
%
\begin{center}
\begin{tabular}{ll}
    |README.txt|   & readme file \\
    |childdoc.ins| & installation file \\
    |childdoc.dtx| & source file \\
    |childdoc.def| & definition file \\
    |cdocsamp.tex| & sample main file \\
    |cdocsch1.tex| & sample include file \\
    |cdocsch2.tex| & sample include file \\
    |cdocspt3.tex| & sample part file \\
    |cdocspt4.tex| & sample part file \\
    |cdocsdrf.tex| & sample redirection file \\
    |cdocsfn1.tex| & sample redirection file \\
    |cdocsfn2.tex| & sample redirection file \\
    |childdoc.pdf| & manual
\end{tabular}
\end{center}
%
The distribution consists of the files
|README.txt|, |childdoc.ins| and |childdoc.dtx|.
%
\begin{itemize}
\item
Run (pdf)\LaTeX{} on |childdoc.dtx|
to compile the manual |childdoc.pdf| (this file).
\item
Run \LaTeX{} on |childdoc.ins| to create the definitions file |childdoc.def|
and the sample |cdocsamp.tex| with include files
|cdocsch1.tex|, |cdocsch2.tex|, |cdocspt3.tex|, |cdocspt4.tex|,
|cdocsdrf.tex|, |cdocsfn1.tex|, |cdocsfn2.tex|.
Then copy the file |childdoc.def| to an appropriate directory of your \LaTeX{}
distribution, e.g.\ \textit{texmf-root}|/tex/latex/childdoc|.
\end{itemize}

%%%%%%%%%%%%%%%%%%%%%%%%%%%%%%%%%%%%%%%%%%%%%%%%%%%%%%%%%%%%%%%%%%%%%%%%%%%%%%%%
\subsection{Related CTAN Packages}

There are several other packages which offer a similar functionality:
%
\begin{itemize}
\item
The packages
\href{http://ctan.org/pkg/docmute}{\textsf{docmute}},
\href{http://ctan.org/pkg/includex}{\textsf{includex}} and
\href{http://ctan.org/pkg/standalone}{\textsf{standalone}}
provide commands to include only the document body of
a child file thus allowing both files to be compiled individually.
\item
The packages \href{http://ctan.org/pkg/subdocs}{\textsf{subdocs}}
and \href{http://ctan.org/pkg/subfiles}{\textsf{subfiles}}
provide structures in which the main and child documents can be
encapsulated and allowing them to be compiled individually.
The inclusion mechanism is different from the conventional |\include|.
\item
The package \href{http://ctan.org/pkg/combine}{\textsf{combine}}
is an elaborate solution to combine several documents into one.
\end{itemize}
%
See also the CTAN topic \href{http://ctan.org/topic/subdocs}{\textsf{subdocs}}
for further related packages.
The present package differs from the above solutions in that
a document structure constructed with the conventional |\include| mechanism
just needs two extra commands at the top of every file
such that all constituent files can be compiled individually.

%%%%%%%%%%%%%%%%%%%%%%%%%%%%%%%%%%%%%%%%%%%%%%%%%%%%%%%%%%%%%%%%%%%%%%%%%%%%%%%%
%\subsection{Feature Suggestions}
%
%The following is a list of features which may be useful for future
%versions of this package:
%%
%\begin{itemize}
%\item
%\ldots
%\end{itemize}

%%%%%%%%%%%%%%%%%%%%%%%%%%%%%%%%%%%%%%%%%%%%%%%%%%%%%%%%%%%%%%%%%%%%%%%%%%%%%%%%
\subsection{Revision History}

%%%%%%%%%%%%%%%%%%%%%%%%%%%%%%%%%%%%%%%%
\paragraph{v2.0:} 2018/12/30

\begin{itemize}
\item
immediate forward processing
\item
added |\childdocby| mechanism
\item
manual restructured
\end{itemize}

%%%%%%%%%%%%%%%%%%%%%%%%%%%%%%%%%%%%%%%%
\paragraph{v1.6:} 2018/01/17

\begin{itemize}
\item
application for development of include files
\item
corrections to manual
\end{itemize}

%%%%%%%%%%%%%%%%%%%%%%%%%%%%%%%%%%%%%%%%
\paragraph{v1.5:} 2017/05/21

\begin{itemize}
\item
more complete structuring introduced
\item
|\childdocof| introduced
\item
|\childdoc| renamed to |\childdocmain|
\item
|\childredirect| renamed to |\childdocforward| and |\childdocforwardprefix|
and functionality expanded
\end{itemize}

%%%%%%%%%%%%%%%%%%%%%%%%%%%%%%%%%%%%%%%%
\paragraph{v1.0:} 2017/04/27

\begin{itemize}
\item
manual and install package
\item
first version published on CTAN
\end{itemize}

%%%%%%%%%%%%%%%%%%%%%%%%%%%%%%%%%%%%%%%%
\paragraph{v0.6:} 2017/04/26

\begin{itemize}
\item
redirection mechanism added
\end{itemize}

%%%%%%%%%%%%%%%%%%%%%%%%%%%%%%%%%%%%%%%%
\paragraph{v0.5:} 2017/04/26

\begin{itemize}
\item
functionality in definition file
\end{itemize}


%%%%%%%%%%%%%%%%%%%%%%%%%%%%%%%%%%%%%%%%%%%%%%%%%%%%%%%%%%%%%%%%%%%%%%%%%%%%%%%%
%%%%%%%%%%%%%%%%%%%%%%%%%%%%%%%%%%%%%%%%%%%%%%%%%%%%%%%%%%%%%%%%%%%%%%%%%%%%%%%%
%%%%%%%%%%%%%%%%%%%%%%%%%%%%%%%%%%%%%%%%%%%%%%%%%%%%%%%%%%%%%%%%%%%%%%%%%%%%%%%%
\appendix

\settowidth\MacroIndent{\rmfamily\scriptsize 000\ }

 \DocInput{childdoc.dtx}

\end{document}
%</driver>
% \fi
%
% %%%%%%%%%%%%%%%%%%%%%%%%%%%%%%%%%%%%%%%%%%%%%%%%%%%%%%%%%%%%%%%%%%%%%%%%%%%%%%
% %%%%%%%%%%%%%%%%%%%%%%%%%%%%%%%%%%%%%%%%%%%%%%%%%%%%%%%%%%%%%%%%%%%%%%%%%%%%%%
% \section{Sample}
%\iffalse
%<*samplemain>
%\fi
%
% The following presents a sample document
% with two chapters, two parts, a title page,
% a compile flag as well as three forwarding files to set the flag.
% It consists of eight |.tex| files:
% \begin{center}
% \begin{tabular}{ll}
% |cdocsamp.tex|&main file\\
% |cdocsch1.tex|&include file for chapter 1\\
% |cdocsch2.tex|&include file for chapter 2\\
% |cdocspt3.tex|&include file for part 3\\
% |cdocspt4.tex|&include file for part 4\\
% |cdocsdrf.tex|&forwarding file for main file in draft mode\\
% |cdocsfi1.tex|&forwarding file for final version of chapter 1\\
% |cdocsfi2.tex|&forwarding file for final version of chapter 2\\
% \end{tabular}
% \end{center}
% Each of the eight files can be compiled directly by the \LaTeX{} compiler.
%
% %%%%%%%%%%%%%%%%%%%%%%%%%%%%%%%%%%%%%%
% \paragraph{Main File.}
%
% The main file is called |cdocsamp.tex|.
%
% Load the \textsf{childdoc} definitions and
% declare the filename for the main document:
%    \begin{macrocode}
\input{childdoc.def}
\childdocmain{}
%    \end{macrocode}

% Optional override for |\version| flag:
%    \begin{macrocode}
%%\ifchilddoc\else\providecommand{\version}{draft}\fi
%    \end{macrocode}

% Define the default values for the |\version| flag
% (|final| for the main file and |draft| for childs):
%    \begin{macrocode}
\ifchilddoc
\providecommand{\version}{draft}
\else
\providecommand{\version}{final}
\fi
%    \end{macrocode}

% Load the standard document class:
%    \begin{macrocode}
\documentclass[12pt]{article}
%    \end{macrocode}

% Start the document body:
%    \begin{macrocode}
\begin{document}
%    \end{macrocode}

% Declare a title page.
% Print title, part of document being processed and version flag:
%    \begin{macrocode}
\addtocounter{page}{-1}
\begin{center}
{\LARGE\bfseries{}childdoc example\par}
\vspace{1cm}
\ifchilddoc
\ifchilddocmanual part\else chapter\fi:
`\childdocname' of `\childdocjob'\par
\else
main document: `\childdocjob'\par
\fi
version: \version\par
\end{center}
\newpage
%    \end{macrocode}

% Manually include selected file,
% otherwise process as usual:
%    \begin{macrocode}
\ifchilddocmanual
\section*{part `\childdocname'}
\input{\childdocname}
\else
%    \end{macrocode}

% Include the two chapters:
%    \begin{macrocode}
\include{cdocsch1}
\include{cdocsch2}
%    \end{macrocode}

% Include the two parts unless only chapters should be displayed:
%    \begin{macrocode}
\ifchilddoc\else
\section{part three}
\input{cdocspt3}
\section{part four}
\input{cdocspt4}
\fi
%    \end{macrocode}

% Process as usual until here:
%    \begin{macrocode}
\fi
%    \end{macrocode}

% End of document body:
%    \begin{macrocode}
\end{document}
%    \end{macrocode}
%\iffalse
%</samplemain>
%\fi
%
% %%%%%%%%%%%%%%%%%%%%%%%%%%%%%%%%%%%%%%
% \paragraph{Chapter Include Files.}
%
% The include files are called |cdocsch1.tex| and |cdocsch2.tex|.
%
%\iffalse
%<*samplechap1|samplechap2>
%\fi

% Optional override for |\version| flag:
%    \begin{macrocode}
%%\providecommand{\version}{final}
%    \end{macrocode}

% Include the main document:
%    \begin{macrocode}
\input{childdoc.def}
\childdocof{cdocsamp}
%    \end{macrocode}

%\iffalse
%</samplechap1|samplechap2>
%\fi
%
%\iffalse
%<*samplechap1>
%\fi
% Some text for chapter 1:
%    \begin{macrocode}
\section{one}
some text in chapter one
%    \end{macrocode}

%\iffalse
%</samplechap1>
%\fi
% Some text for chapter 2:
%\iffalse
%<*samplechap2>
%\fi
%    \begin{macrocode}
\section{two}
more text in chapter two
%    \end{macrocode}

%\iffalse
%</samplechap2>
%\fi
%
% %%%%%%%%%%%%%%%%%%%%%%%%%%%%%%%%%%%%%%
% \paragraph{Part Include Files.}
%
% The include files are called |cdocspt3.tex| and |cdocspt4.tex|.
%
%\iffalse
%<*samplepart3|samplepart4>
%\fi

% Optional override for |\version| flag:
%    \begin{macrocode}
%%\providecommand{\version}{final}
%    \end{macrocode}

% Include the main document:
%    \begin{macrocode}
\input{childdoc.def}
\childdocby{cdocsamp}
%    \end{macrocode}

%\iffalse
%</samplepart3|samplepart4>
%\fi
%
%\iffalse
%<*samplepart3>
%\fi
% Some text for part 3:
%    \begin{macrocode}
some text in part three
%    \end{macrocode}

%\iffalse
%</samplepart3>
%\fi
% Some text for part 4:
%\iffalse
%<*samplepart4>
%\fi
%    \begin{macrocode}
more text in part four
%    \end{macrocode}

%\iffalse
%</samplepart4>
%\fi
%
% %%%%%%%%%%%%%%%%%%%%%%%%%%%%%%%%%%%%%%
% \paragraph{Forwarding for a Complete Draft.}
%
% The following forwarding file |cdocsdrf.tex|
% compiles the main document in draft mode:
%\iffalse
%<*sampledraft>
%\fi
%    \begin{macrocode}
\def\version{draft}
\input{childdoc.def}
\childdocforward{cdocsamp}
%    \end{macrocode}

%\iffalse
%</sampledraft>
%\fi
%
% %%%%%%%%%%%%%%%%%%%%%%%%%%%%%%%%%%%%%%
% \paragraph{Forwarding for Final Version of the Chapters.}
%
% The following forwarding files |cdocsfn1.tex| and |cdocsfn2.tex|
% (with identical content)
% compile the final versions of the child documents
% |cdocsch1.tex| and |cdocsch2.tex|, respectively:
%\iffalse
%<*samplefinal>
%\fi
%    \begin{macrocode}
\def\version{final}
\input{childdoc.def}
\childdocforwardprefix[cdocsamp]{cdocsfn}{cdocsch}
%    \end{macrocode}

%\iffalse
%</samplefinal>
%\fi
%
% %%%%%%%%%%%%%%%%%%%%%%%%%%%%%%%%%%%%%%
% \paragraph{Command Line Processing.}
%
% The following three command lines generate the output files
% |cdocscld|, |cdocscl1| and |cdocscl2|
% which should be identical to
% |cdocsdrf|, |cdocsch1| and |cdocsfn2|, respectively:
% \begin{center}
% \begin{tabular}{l}
% |latex -jobname cdocscld \|\\
% |  "\def\version{draft}\input{childdoc.def}\childdocforward{cdocsamp}"|\\
% |latex -jobname cdocscl1 \|\\
% |  "\input{childdoc.def}\childdocforward[cdocsamp]{cdocsch1}"|\\
% |latex -jobname cdocscl2 \|\\
% |  "\def\version{final}\input{childdoc.def}\childdocforward{cdocsch2}"|
% \end{tabular}
% \end{center}
% Note that the trailing backslash on each first line
% merely continues the input to the second line
% (for convenient cut ant paste).
% Furthermore, the command |latex| can be replaced by any
% of its alternative versions such as |pdflatex|.
%
% %%%%%%%%%%%%%%%%%%%%%%%%%%%%%%%%%%%%%%%%%%%%%%%%%%%%%%%%%%%%%%%%%%%%%%%%%%%%%%
% %%%%%%%%%%%%%%%%%%%%%%%%%%%%%%%%%%%%%%%%%%%%%%%%%%%%%%%%%%%%%%%%%%%%%%%%%%%%%%
% \section{Implementation}
%\iffalse
%<*package>
%\fi
%
% This section describes the definitions file |childdoc.def|.

% The definitions cannot be loaded using |\usepackage| or |\RequirePackage|
% which has a mechanism to prevent loading a style file more than once.
% When loading the definitions by means of |\input|
% multiple instances have to be prevented manually:
%\iffalse
%This code needs to be before the `\ProvidesFile' directive
%which is defined at the beginning of this file.
%Therefore it is also placed there and commented out here.
%</package>
%<*discard>
%\fi
%    \begin{macrocode}
\ifdefined\childdocmain\endinput\fi
%    \end{macrocode}
%\iffalse
%</discard>
%<*package>
%\fi
%
% \macro{\ifchilddoc}
% \macro{\ifchilddocmanual}
% The conditional |\ifchilddoc| tells whether a
% child (true) or main (false) document is being compiled.
% The conditional |\ifchilddocmanual| tells whether
% the |\includeonly| mechanism is used (false) or
% the selection of child files must be performed manually (true).
% The definitions initialise to false:
%    \begin{macrocode}
\newif\ifchilddoc
\newif\ifchilddocmanual
%    \end{macrocode}

% \macro{\childdocname}
% \macro{\childdocjob}
% The macro |\childdocname| stores the name of the main document
% to be compiled. The macro |\childdocjob| stores the name of
% the document on which the \LaTeX{} compiler was originally invoked.
% The content of |\jobname| cannot be compared
% to filenames specified in the source due to different catcodes.
% The following code rescans |\jobname|, stores the result
% in |\childdocname| and saves a copy in |\childdocjob|:
%    \begin{macrocode}
\edef\childdocname{\scantokens\expandafter{\jobname\noexpand}}
\let\childdocjob\childdocname
%    \end{macrocode}

% \macro{\childdocdisable}
% The macro |\childdocdisable| prevents the main file
% from being processed more than once.
% At this stage, the main document command |\childdocmain|
% is assumed to be called once again where it should do nothing.
% Any subsequent call to it should prevent
% a secondary processing of the main document
% It overwrites the forwarding commands
% |\childdocof| and |\childdocforward|
% with empty macros to prevent further inclusions of the main document:
%    \begin{macrocode}
\newcommand{\childdocdisable}
{
  \renewcommand{\childdocmain}[1]{\renewcommand{\childdocmain}[1]{\endinput}}
  \renewcommand{\childdocof}[1]{}
  \renewcommand{\childdocby}[2][]{}
  \renewcommand{\childdocforward}[2][]{}
  \renewcommand{\childdocdisable}{}
}
%    \end{macrocode}

% \macro{\childdocmain}
% The macro |\childdocmain| is to be called at the top of the main file
% with nothing or the main filename (without extension) as argument.
% First, it breaks loops.
% If the argument is not empty and does not match |\childdocname|
% (which is set by the first inclusion of |childdoc.def|),
% |\ifchilddoc| is set to true, |\includeonly| is applied to the child file
% and |\jobname| is set to the main file
% (for proper handling of |.aux| files):
%    \begin{macrocode}
\newcommand{\childdocmain}[1]
{
  \childdocdisable\childdocmain{}
  \if?#1?\else
    \begingroup
      \def\childdoctmp{#1}
      \ifx\childdoctmp\childdocname
        \def\childdoctmp{}
      \else
        \def\childdoctmp
        {
          \childdoctrue
          \includeonly{\childdocname}
          \def\childdocjob{#1}
          \def\jobname{#1}
        }
      \fi
      \expandafter
    \endgroup
    \childdoctmp
  \fi
}
%    \end{macrocode}

% \macro{\childdocof}
% The command |\childdocof| redirects
% compilation to the main file |#1|.
%    \begin{macrocode}
\newcommand{\childdocof}[1]
{
  \childdocdisable
  \childdoctrue
  \includeonly{\childdocname}
  \def\jobname{#1}
  \def\childdocjob{#1}
  \input{#1}
}
%    \end{macrocode}

% \macro{\childdocby}
% The command |\childdocby| ....
%    \begin{macrocode}
\newcommand{\childdocby}[2][]
{
  \childdocdisable
  \childdoctrue
  \childdocmanualtrue
  \if?#1?\else
    \def\jobname{#2}
  \fi
  \def\childdocjob{#2}
  \input{#2}
  \endinput
}
%    \end{macrocode}

% \macro{\childdocforward}
% The command |\childdocforward| redirects
% compilation to the main file or
% (if the optional argument is given) a child file.
% Parameters are set as if the main file
% or a child file starting with |\childdocof| was compiled.
% Then compilation is handed over to the main file:
%    \begin{macrocode}
\newcommand{\childdocforward}[2][]
{
  \begingroup
    \if?#1?
      \def\childdoctmp
      {
        \def\childdocname{#2}
        \def\childdocjob{#2}
        \def\jobname{#2}
        \input{#2}
        \endinput
      }
    \else
      \def\childdoctmp
      {
        \childdocdisable
        \def\childdocname{#2}
        \childdoctrue
        \includeonly{#2}
        \def\childdocjob{#1}
        \def\jobname{#1}
        \input{#1}
        \endinput
      }
    \fi
    \expandafter
  \endgroup
  \childdoctmp
}
%    \end{macrocode}

% \macro{\childdocforwardprefix}
% The command |\childdocforwardprefix| redirects
% compilation to the main or a child file by means of a pattern.
% The prefix |#1| in the current filename is replaced by |#2|
% and the suffix of the current filename is kept
% (it is assumed that the filename does not contain the substring `|~~~|'
% which is used as a delimiter).
% Compilation is handed over to the new file by |\childdocforward|:
%    \begin{macrocode}
\newcommand{\childdocforwardprefix}[3][]
{
  \begingroup
    \def\childdocextract #2##1~~~{\def\childdoctmp{\childdocforward[#1]{#3##1}}}
    \expandafter\childdocextract\childdocname~~~
    \expandafter
  \endgroup
  \childdoctmp
}
%    \end{macrocode}

% \macro{\childdoc}
% The deprecated macro |\childdoc| is a legacy version of |\childdocmain|:
%    \begin{macrocode}
\newcommand{\childdoc}{\childdocmain}
%    \end{macrocode}

% \macro{\childdocredirect}
% The deprecated macro |\childdocredirect| is a legacy version
% of |\childdocforward| and |\childdocforwardprefix|:
%    \begin{macrocode}
\newcommand{\childdocredirect}[2][]
{
  \begingroup
    \if?#1?
      \def\childdoctmp{\childdocforward{#2}}
    \else
      \def\childdoctmp{\childdocforwardprefix{#1}{#2}}
    \fi
    \expandafter
  \endgroup
  \childdoctmp
}
%    \end{macrocode}

%\iffalse
%</package>
%\fi
%
\endinput

\childdocforward{cdocsamp}
%    \end{macrocode}

%\iffalse
%</sampledraft>
%\fi
%
% %%%%%%%%%%%%%%%%%%%%%%%%%%%%%%%%%%%%%%
% \paragraph{Forwarding for Final Version of the Chapters.}
%
% The following forwarding files |cdocsfn1.tex| and |cdocsfn2.tex|
% (with identical content)
% compile the final versions of the child documents
% |cdocsch1.tex| and |cdocsch2.tex|, respectively:
%\iffalse
%<*samplefinal>
%\fi
%    \begin{macrocode}
\def\version{final}
% \iffalse
%
% childdoc.dtx Copyright (C) 2017-2018 Niklas Beisert
%
% This work may be distributed and/or modified under the
% conditions of the LaTeX Project Public License, either version 1.3
% of this license or (at your option) any later version.
% The latest version of this license is in
%   http://www.latex-project.org/lppl.txt
% and version 1.3 or later is part of all distributions of LaTeX
% version 2005/12/01 or later.
%
% This work has the LPPL maintenance status `maintained'.
%
% The Current Maintainer of this work is Niklas Beisert.
%
% This work consists of the files childdoc.dtx and childdoc.ins
% and the derived files childdoc.def and cdocsamp.tex with
% cdocsch1.tex, cdocsch2.tex, cdocsdrf.tex, cdocsfn1.tex, cdocsfn2.tex.
%
%<package>\ifdefined\childdocmain\endinput\fi
%<package>\ProvidesFile{childdoc.def}[2018/12/30 v2.0 child document driver]
%<samplemain>\ProvidesFile{cdocsamp.tex}[2018/12/30 v2.0 sample for childdoc]
%<*driver>
%\ProvidesFile{childdoc.drv}[2018/12/30 v2.0 childdoc reference manual file]
\PassOptionsToClass{10pt,a4paper}{article}
\documentclass{ltxdoc}

\usepackage[margin=35mm]{geometry}
\usepackage{hyperref}
\usepackage{hyperxmp}
\usepackage[usenames]{color}

\hypersetup{colorlinks=true}
\hypersetup{pdfstartview=FitH}
\hypersetup{pdfpagemode=UseNone}
\hypersetup{pdfsource={}}
\hypersetup{pdflang={en-UK}}
\hypersetup{pdfcopyright={Copyright 2017-2018 Niklas Beisert.
  This work may be distributed and/or modified under the
  conditions of the LaTeX Project Public License, either version 1.3
  of this license or (at your option) any later version.}}
\hypersetup{pdflicenseurl={http://www.latex-project.org/lppl.txt}}
\hypersetup{pdfcontactaddress={ETH Zurich, ITP, HIT K,
  Wolfgang-Pauli-Strasse 27}}
\hypersetup{pdfcontactpostcode={8093}}
\hypersetup{pdfcontactcity={Zurich}}
\hypersetup{pdfcontactcountry={Switzerland}}
\hypersetup{pdfcontactemail={nbeisert@itp.phys.ethz.ch}}
\hypersetup{pdfcontacturl={http://people.phys.ethz.ch/\xmptilde nbeisert/}}

\newcommand{\secref}[1]{\hyperref[#1]{section \ref*{#1}}}

\parskip1ex
\parindent0pt
\let\olditemize\itemize
\def\itemize{\olditemize\parskip0pt}

\begin{document}

\title{The \textsf{childdoc} Package}
\hypersetup{pdftitle={The childdoc Package}}
\author{Niklas Beisert\\[2ex]
  Institut f\"ur Theoretische Physik\\
  Eidgen\"ossische Technische Hochschule Z\"urich\\
  Wolfgang-Pauli-Strasse 27, 8093 Z\"urich, Switzerland\\[1ex]
  \href{mailto:nbeisert@itp.phys.ethz.ch}
  {\texttt{nbeisert@itp.phys.ethz.ch}}}
\hypersetup{pdfauthor={Niklas Beisert}}
\hypersetup{pdfsubject={Manual for the LaTeX2e Package childdoc}}
\date{30 December 2018, \textsf{v2.0}}
\maketitle

\begin{abstract}\noindent
\textsf{childdoc} is a \LaTeXe{} package
that enables the direct compilation
of document sections included by |\include|
to individual files.
\end{abstract}

\begingroup
\parskip0ex
\tableofcontents
\endgroup

%%%%%%%%%%%%%%%%%%%%%%%%%%%%%%%%%%%%%%%%%%%%%%%%%%%%%%%%%%%%%%%%%%%%%%%%%%%%%%%%
%%%%%%%%%%%%%%%%%%%%%%%%%%%%%%%%%%%%%%%%%%%%%%%%%%%%%%%%%%%%%%%%%%%%%%%%%%%%%%%%
\section{Introduction}

\LaTeX{} provides a mechanism to structure a large document (such as a book)
into a main file and several child files (containing the chapters)
using the |\include| command.
This mechanism is beneficial for documents
which span hundreds of pages in order to
make the source file(s) more manageable.
Moreover, compilation can be restricted to
selected child files by means of the |\includeonly| command.
The latter feature can be used to reduce the compilation time while editing
(this was significantly more useful in the earlier days of \LaTeX{})
or to generate a smaller document which is easier to navigate.
Another application of |\includeonly| is to generate
documents consisting of selected parts of the complete document.

However, there are a few drawbacks of the plain |\include| mechanism:
\begin{itemize}
\item
The child files cannot be compiled on their own,
they can only be compiled via the main file.
A naive editing environment
(such as a text editor with an option
to have the current file processed by \LaTeX)
may require one to switch to the main file before compiling;
attempting to compile the child file produces errors.
\item
The main file must be modified (each time)
to adjust the |\includeonly| command
to the present needs. This easily leaves the main file in a messy state.
\item
The generated document will always carry the filename
of the main document. This is inconvenient if
several child files are to be compiled and
to be kept for distribution.
\end{itemize}

The present package provides a simple interface
to make child files individually compilable by \LaTeX{}.
Compiling a child file then has the same effect as compiling
the main file with an |\includeonly| command
to select the appropriate child.
Moreover the generated document will carry the name of the child
rather than the main file.
This resolves all three above issues.

This feature is meant to make the editing of books,
thesis documents and lecture notes somewhat more convenient.
However, the package can also be used efficiently for
composing a series of documents (such as exercise sheets)
which are typically distributed individually.
It then assists the author in generating the individual documents
(potentially in different versions)
as well as a document containing the collected series.
Another application is in developing style files
or other kinds of included material
where compilation of the style file could redirect
to a sample or test file.

%%%%%%%%%%%%%%%%%%%%%%%%%%%%%%%%%%%%%%%%%%%%%%%%%%%%%%%%%%%%%%%%%%%%%%%%%%%%%%%%
%%%%%%%%%%%%%%%%%%%%%%%%%%%%%%%%%%%%%%%%%%%%%%%%%%%%%%%%%%%%%%%%%%%%%%%%%%%%%%%%
\section{Usage}

First of all, the package \textsf{childdoc} is \emph{not} a standard
\LaTeXe{} |.sty| style file! Therefore it needs to be invoked in
a non-standard way.

%%%%%%%%%%%%%%%%%%%%%%%%%%%%%%%%%%%%%%%%%%%%%%%%%%%%%%%%%%%%%%%%%%%%%%%%%%%%%%%%
\subsection{Included Files}
\label{sec:include}

%%%%%%%%%%%%%%%%%%%%%%%%%%%%%%%%%%%%%%%%
\DescribeMacro{\childdocmain}
To use the package, add the commands
\begin{center}
\begin{tabular}{l}
|\input{childdoc.def}|\\
|\childdocmain{}|\\
\end{tabular}
\end{center}
at the very top of the main \LaTeX{} file,
in particular \emph{before} the |\documentclass| statement!
The argument of |\childdocmain| should be left empty
(but it must be present).

%%%%%%%%%%%%%%%%%%%%%%%%%%%%%%%%%%%%%%%%
\DescribeMacro{\childdocof}
Furthermore, add the commands
\begin{center}
\begin{tabular}{l}
|\input{childdoc.def}|\\
|\childdocof{|\textit{main}|}|\\
\end{tabular}
\end{center}
at the top of every child file \textit{child}
which is included by |\include{|\textit{child}|}|
from within the main file
(or at least for those files to be compiled individually).
The argument \textit{main} must be the filename of the main file.

There are a couple of
considerations in setting up the main and child documents:

%%%%%%%%%%%%%%%%%%%%%%%%%%%%%%%%%%%%%%%%
\paragraph{Restrictions.}

Please note the following restrictions:
\begin{itemize}
\item
|\childdocmain| must be called with one argument \textit{main}
to ensure compatibility with earlier version of the package.
It must either be empty (|\childdocmain{}|)
or precisely match the filename of the main file in which it is specified.
See \secref{sec:detection} for further information.
\item
The filename \textit{main} must be specified without the |.tex| extension.
\item
The filename \textit{main} is case sensitive
(even in case-insensitive file systems)
due to internal string comparison.
\item
The argument \textit{main} should be fully expanded, it cannot be a macro.
\item
Subdirectories and special characters should be avoided in filenames.
\item
The command |\childdocmain{|\textit{main}|}| must be followed by a whitespace.
It should not be followed immediately by another command
or by a comment mark `|%|'.
This is because the \TeX{} parser reads the token immediately following
the argument of |\childdocmain| and puts it
at the beginning of every child section;
however, a white\-space is ignored.
\end{itemize}

%%%%%%%%%%%%%%%%%%%%%%%%%%%%%%%%%%%%%%%%
\paragraph{Content of Main File.}

It is advisable to place all content in the child files included by |\include|.
Any output contained in the main file will appear in all child documents
unless suppressed manually;
it cannot be suppressed automatically by the |\includeonly| directive
and thus should normally be avoided.
A method to include some content in the main file
by means of conditional processing is described in \secref{sec:conditional}.

%%%%%%%%%%%%%%%%%%%%%%%%%%%%%%%%%%%%%%%%
\paragraph{Page Numbering.}

When only a part of the document is compiled,
the appropriate numbering of pages
(as well as other status parameters)
is determined from the |.aux| files.
The latter contain information from previous passes.
However this information needs to propagate through
all intermediate child documents.
Therefore the page numbering in child documents may well
be inconsistent until the complete document is compiled at least once.

A useful (if unconventional) way to always ensure a consistent
page numbering is to restart the numbering in each child document
and denote the pages by `\textit{child}|.|\textit{page}'
where \textit{child} represents the chapter/section number of the child file.
This can be achieved by the command
|\numberwithin{page}{|\textit{child}|}|
of the \textsf{amsmath} package
where \textit{child} can be |chapter| or |section|
depending on the chosen structuring.
Alternatively, one can modify the macro |\thepage| appropriately
and reset the counter |page| at the start of each child file.

%%%%%%%%%%%%%%%%%%%%%%%%%%%%%%%%%%%%%%%%%%%%%%%%%%%%%%%%%%%%%%%%%%%%%%%%%%%%%%%%
\subsection{Conditional Processing}
\label{sec:conditional}

The package provides a mechanism to compile different versions
of a document. To customise the versions further some conditional processing
can come in handy to distinguish which version is being compiled.
The package provides two macros to describe the compilation context:

%%%%%%%%%%%%%%%%%%%%%%%%%%%%%%%%%%%%%%%%
\DescribeMacro{\ifchilddoc}
The conditional |\ifchilddoc| distinguishes between the compilation of
child documents and the main document:
%
\begin{center}
|\ifchilddoc |\textit{child-code}| |[|\||else |\textit{main-code}]| \||fi|
\end{center}

%%%%%%%%%%%%%%%%%%%%%%%%%%%%%%%%%%%%%%%%
\DescribeMacro{\childdocname}
\DescribeMacro{\childdocjob}
The macro |\childdocname| contains the filename (without extension)
of the main or child file being processed.
Note that |\childdocjob| will always contain the name of the main file.

%%%%%%%%%%%%%%%%%%%%%%%%%%%%%%%%%%%%%%%%
\paragraph{Title Page.}

Conditional processing can be used to include a title or banner page
in the main document when proper precautions are taken.
Importantly, the code in the main file should ensure that the page counter
(as well as other status parameters which are stored in the |.aux| files)
takes the same value after the conditional processing.
Otherwise the page numbers may take divergent values
depending on which part is compiled.

For example, a title page could be declared by:
%
\begin{center}
\begin{tabular}{l}
|\ifchilddoc\||else|\\
|\addtocounter{page}{-1}|\\
\textit{code for title page}\\
|\newpage|\\
|\||fi|
\end{tabular}
\end{center}
%
A banner page for the child documents can be generated by:
%
\begin{center}
\begin{tabular}{l}
|\ifchilddoc|\\
|\addtocounter{page}{-1}|\\
\textit{code for banner page}\\
|\newpage|\\
|\||fi|
\end{tabular}
\end{center}
%
Here one could write a message such as:
\begin{center}
|This is the part \childdocname{} of \childdocjob{}.|
\end{center}

%%%%%%%%%%%%%%%%%%%%%%%%%%%%%%%%%%%%%%%%%%%%%%%%%%%%%%%%%%%%%%%%%%%%%%%%%%%%%%%%
\subsection{Flags}
\label{sec:flags}

The package makes it easy to generate different versions
of the main or child documents.
To this end compilation flags can be defined
and assigned different default values.
They will be particularly useful in conjunction
with the forwarding mechanism described in \secref{sec:forward}.

For example, it may be useful to have a flag |\version|
which can be set to |draft| or |final|.
The document source will contain some conditional code
depending on the value of |\version|.
Suppose further, the flag should default to |final| for the main file
and to |draft| for child files
which is a natural assignment for editing the document.
This is achieved by placing the following code
in the preamble of the main document
(below the |\childdocmain| directive):
%
\begin{center}
\begin{tabular}{l}
|\ifchilddoc|\\
|\providecommand{\version}{draft}|\\
|\||else|\\
|\providecommand{\version}{final}|\\
|\||fi|
\end{tabular}
\end{center}
%
The definition by |\providecommand| makes sure
that previous definitions are not overwritten.
Further statements |\providecommand{\version}{...}|
can thus be added before the above code to override it.

For the main file, one might add a line
(between |\childdocmain| and the above block)
%
\begin{center}
|%\ifchilddoc\||else\providecommand{\version}{draft}\||fi|
\end{center}
%
which can be uncommented to produce a draft version.
Likewise one can add a line to the very top of a child file
(above the |\childdocof{|\textit{main}|}| directive)
%
\begin{center}
|%\providecommand{\version}{final}|
\end{center}
%
which can be uncommented to produce the final version of this child document.

%%%%%%%%%%%%%%%%%%%%%%%%%%%%%%%%%%%%%%%%%%%%%%%%%%%%%%%%%%%%%%%%%%%%%%%%%%%%%%%%
\subsection{Forwarding}
\label{sec:forward}

Different versions of the main or child documents
using compilation flags as described in \secref{sec:flags}
can be (permanently) stored in different files
for convenient compilation, viewing and distribution.
To this end, the package defines a command
to pass on compilation to a different file:

%%%%%%%%%%%%%%%%%%%%%%%%%%%%%%%%%%%%%%%%
\DescribeMacro{\childdocforward}
The command |\childdocforward| redirects processing to
another source file:
%
\begin{center}
\begin{tabular}{l}
|\input{childdoc.def}|\\
|\childdocforward[|\textit{main}|]{|\textit{dest}|}|\\
\end{tabular}
\end{center}
%
The argument \textit{dest} is the destination file
(without extension).
It should be the main file or one of the child files.
Note that further \textsf{childdoc} directives
such as |\childdocof| and |\childdocforward|
in the indicated file will be processed in this form.
The optional argument \textit{main}
passes on directly to the main file \textit{main}
while pretending to compile the child \textit{dest}.
This form behaves as if \textit{dest}
issues |\childdocof{|\textit{main}|}| right away,
and no further \textsf{childdoc} directives will be processed.

%%%%%%%%%%%%%%%%%%%%%%%%%%%%%%%%%%%%%%%%
\DescribeMacro{\...prefix}
In the alternative form |\childdocforwardprefix|,
%
\begin{center}
\begin{tabular}{l}
|\input{childdoc.def}|\\
|\childdocforwardprefix[|\textit{main}|]{|\textit{prefix}|}{|\textit{dest}|}|
\end{tabular}
\end{center}
%
the destination file is determined by a pattern
depending on the current file:
To make this work, the current file must be called
`{\textit{prefix}\hspace{0.2em}\textit{suffix}}'
with \textit{prefix} matching precisely the argument.
Processing is then passed on to the file
`{\textit{dest}\hspace{0.2em}\textit{suffix}}'.
Surely, the same effect is achieved by
directly specifying the
argument `{\textit{dest}\hspace{0.2em}\textit{suffix}}'
in the first form.
However, that requires to set up a different file
for each child. With the alternative form of the command
all these files can have exactly the same content
which simplifies setting them up and maintaining them.

For example, the following file |draft.tex|
with a compilation flag |\version| as described in \secref{sec:flags}
compiles the main document as a draft:
%
\begin{center}
\begin{tabular}{l}
|\def\version{draft}|\\
|\input{childdoc.def}|\\
|\childdocforward{|\textit{main}|}|
\end{tabular}
\end{center}
%
Likewise, the following files |final|\textit{nn}|.tex|
compile the final version of the child document
|child|\textit{nn}|.tex|:
%
\begin{center}
\begin{tabular}{l}
|\def\version{final}|\\
|\input{childdoc.def}|\\
|\childdocforwardprefix{final}{child}|
\end{tabular}
\end{center}
%

Note that when several versions of a main file and/or of each child file
are to be generated, it may be convenient to set up a |Makefile| or
shell script to automatise the process.

%%%%%%%%%%%%%%%%%%%%%%%%%%%%%%%%%%%%%%%%%%%%%%%%%%%%%%%%%%%%%%%%%%%%%%%%%%%%%%%%
\subsection{Command Line Processing}
\label{sec:commandline}

The effect of redirection files can also be achieved by invoking
the \LaTeX{} compiler with a more elaborate command line.
Most conveniently this should be done as part
of a shell script or a |Makefile|.

When using \textsf{childdoc} in the main file, the following
command lines effectively perform a redirection
(note that depending on the shell being used,
backslashes may have to be doubled: `|\|' $\to$ `|\\|'):
%
\begin{center}
|... -jobname "|\textit{target}|" |\\|"|[\textit{flags}]%
|\input{childdoc.def}\childdocforward[|\textit{main}|]{|\textit{dest}|}"|
\end{center}
%
Here \textit{target} is the name of the output file,
\textit{main} is the name of the main file
and \textit{dest} is the name of the main or child file to be processed
(all filenames without extensions).
The optional argument \textit{main} can be omitted
if \textit{main} matches \textit{dest}.
Optionally, compilation \textit{flags} can be defined via |\def| commands.
This command line makes the \TeX{} engine believe
it is compiling the file \textit{target}
whose content is specified as the latter parameter.
The provided code then forwards the processing to
\textit{main} or \textit{dest} as described in \secref{sec:forward}.

%%%%%%%%%%%%%%%%%%%%%%%%%%%%%%%%%%%%%%%%%%%%%%%%%%%%%%%%%%%%%%%%%%%%%%%%%%%%%%%%
\subsection{Include by Input}
\label{sec:input}

Including child documents by |\include| has some restrictions by design.
Most notably, the content of a child document always occupies
its own set of pages; pages cannot be shared between child documents.
Usually, this behaviour makes perfect sense
because each child document contain an essential part of the document.
However, in some situations it may be desirable to compose
a document from a collection of parts
without having mandatory page breaks between then.
For this case, the package
provides a mechanism to include parts
by |\input| which can also be processed individually.
However, by construction this mechanism
requires manual handling of the content to be output.

%%%%%%%%%%%%%%%%%%%%%%%%%%%%%%%%%%%%%%%%
\DescribeMacro{\ifchilddocmanual}
The main file should be prepared as usual, see \secref{sec:include}.
However, the document body must make a distinction
between processing of an individual part and of the main document, e.g.:
%
\begin{center}
\begin{tabular}{l}
|\ifchilddocmanual|\\
|\input{\childdocname}|\\
|\||else|\\
\textit{document body with }|\input{|\textit{part}|}|\\
|\||fi|
\end{tabular}
\end{center}
%
The conditional |\ifchilddocmanual| is true whenever
a part to be included by |\input| is being compiled,
and the name of the part is stored in |\childdocname|.

%%%%%%%%%%%%%%%%%%%%%%%%%%%%%%%%%%%%%%%%
\DescribeMacro{\childdocby}
Each part to be included by |\input| should start with:
%
\begin{center}
\begin{tabular}{l}
|\input{childdoc.def}|\\
|\childdocby{|\textit{main}|}|\\
\end{tabular}
\end{center}
%
The directive |\childdocby| is similar to |\childdocof|
described in \secref{sec:include},
but the subsequent selection of content must be done manually.
To that end, both |\ifchilddoc| and |\ifchilddocmanual|
will be true upon processing of a part,
and the name of the part is stored in |\childdocname|.
Note that |\jobname| will be set to the filename of the current part
so that each part receives an individual |.aux| file
that does not interfere with the |.aux| file(s) of the main document.
This behaviour can be altered by the alternative form
|\childdocby[*]{|\textit{main}|}| (with a non-empty optional argument)
which uses the |.aux| file of the main document
by setting |\jobname| to \textit{main}.

%%%%%%%%%%%%%%%%%%%%%%%%%%%%%%%%%%%%%%%%%%%%%%%%%%%%%%%%%%%%%%%%%%%%%%%%%%%%%%%%
\subsection{Driver Development}
\label{sec:driver}

The \textsf{childdoc} mechanism can also be use for the development
of definition files such as \LaTeX{} styles or classes.
This case differs from the above setup with multiple parts
included by |\include| in that no |\includeonly| should be invoked.
This can be achieved by starting the include file
(before |\ProvidesPackage|) with:
%
\begin{center}
\begin{tabular}{l}
|\input{childdoc.def}|\\
|\childdocforward{|\textit{main}|}|\\
\end{tabular}
\end{center}
%
or alternatively with:
%
\begin{center}
\begin{tabular}{l}
|\input{childdoc.def}|\\
|\childdocby{|\textit{main}|}|\\
\end{tabular}
\end{center}
%
Both forms have slightly different effects as described above.
The main file is prepared as usual, see \secref{sec:include}.

%%%%%%%%%%%%%%%%%%%%%%%%%%%%%%%%%%%%%%%%%%%%%%%%%%%%%%%%%%%%%%%%%%%%%%%%%%%%%%%%
\subsection{Legacy Detection}
\label{sec:detection}

The directive |\childdocmain| in the main file can detect
whether the complete document or merely a child is to be compiled
even without using the directive |\childdocof|.
This method is deprecated because it is less robust
and there is no compelling reason to use it;
it is merely provided for backward compatibility
and it may be removed in future versions.

If the detection mechanism is to be used,
it is mandatory to correctly specify
the filename of the main file as the argument of |\childdocmain|:
%
\begin{center}
\begin{tabular}{l}
|\input{childdoc.def}|\\
|\childdocmain{|\textit{main}|}|\\
\end{tabular}
\end{center}
%
If |\jobname| does not match the argument \textit{main} of |\childdocmain|,
it is assumed that |\jobname| points to the child file to be compiled.
When using |\childdocmain| with the main file specified as argument,
it suffices to start a child file
with just |\input{|\textit{main}|}|
without loading of the package and using |\childdocof|.
If instead all processing is done
with the appropriate \textsf{childdoc} directives,
the argument of \textit{main} of |\childdocmain| can be empty.

An alternative version of the command line processing described
in \secref{sec:commandline} using the detection mechanism reads:
%
\begin{center}
|... -jobname "|\textit{target}|" "|[\textit{flags}]%
[|\def\jobname{|\textit{dest}|}|]|\input{|\textit{main}|}"|
\end{center}

%%%%%%%%%%%%%%%%%%%%%%%%%%%%%%%%%%%%%%%%%%%%%%%%%%%%%%%%%%%%%%%%%%%%%%%%%%%%%%%%
\subsection{Manual Code}
\label{sec:manual}

In case one cannot be certain whether the definitions file |childdoc.def|
is installed on the target \TeX{} distribution
and one prefers not to ship it,
it is conceivable to paste a few relevant commands into the sources.

To that end, drop all statements |\input{childdoc.def}|
and perform the replacements as outlined below.
Instead of |\childdocmain{|\textit{main}|}| add the following code
to the top of the main file:
%
\begin{center}
\begin{tabular}{l}
|\||ifdefined\childdocname\endinput\||fi\newif\ifchilddoc|\\
|\edef\childdocname{\scantokens\expandafter{\jobname\noexpand}}|\\
|\def\childdocmain{|\textit{main}|}\||ifx\childdocmain\childdocname\||else|\\
|\childdoctrue\includeonly{\childdocname}\let\jobname\childdocmain\||fi|\\
\end{tabular}
\end{center}
%
Instead of |\childdocof{|\textit{main}|}| just include the main file
at the top of each child file:
%
\begin{center}
|\input{|\textit{main}|}|
\end{center}
%
A simple redirection |\childdocforward{|\textit{dest}|}| is achieved by:
%
\begin{center}
|\def\jobname{|\textit{dest}|}\input{\jobname}|
\end{center}
%
The redirection with prefix
|\childdocforwardprefix[|\textit{prefix}|]{|\textit{dest}|}|
is accomplished by:
%
\begin{center}
\begin{tabular}{l}
|{\edef\jobname{\scantokens\expandafter{\jobname\noexpand}}|\\
|\def\redirectjob |\textit{prefix}|#1~~~{\gdef\jobname{|\textit{dest}|#1}}|\\
|\expandafter\redirectjob\jobname~~~}\input{\jobname}|
\end{tabular}
\end{center}

In an alternative approach,
child documents can be compiled by a specific command line
without additional code or specific definitions:
%
\begin{center}
|... -jobname "|\textit{target}|" "|[\textit{flags}]%
|\includeonly{|\textit{dest}|}\input{|\textit{main}|}"|
\end{center}
%

%%%%%%%%%%%%%%%%%%%%%%%%%%%%%%%%%%%%%%%%%%%%%%%%%%%%%%%%%%%%%%%%%%%%%%%%%%%%%%%%
%%%%%%%%%%%%%%%%%%%%%%%%%%%%%%%%%%%%%%%%%%%%%%%%%%%%%%%%%%%%%%%%%%%%%%%%%%%%%%%%
\section{Information}

%%%%%%%%%%%%%%%%%%%%%%%%%%%%%%%%%%%%%%%%%%%%%%%%%%%%%%%%%%%%%%%%%%%%%%%%%%%%%%%%
\subsection{Copyright}

Copyright \copyright{} 2017--2018 Niklas Beisert

This work may be distributed and/or modified under the
conditions of the \LaTeX{} Project Public License, either version 1.3
of this license or (at your option) any later version.
The latest version of this license is in
  \url{http://www.latex-project.org/lppl.txt}
and version 1.3 or later is part of all distributions of \LaTeX{}
version 2005/12/01 or later.

This work has the LPPL maintenance status `maintained'.

The Current Maintainer of this work is Niklas Beisert.

This work consists of the files |README.txt|, |childdoc.ins| and |childdoc.dtx|
as well as the derived files |childdoc.def|, |cdocsamp.tex|
with |cdocsch1.tex|, |cdocsch2.tex|, |cdocspt3.tex|, |cdocspt4.tex|,
|cdocsdrf.tex|, |cdocsfn1.tex|, |cdocsfn2.tex|
as well as |childdoc.pdf|.

%%%%%%%%%%%%%%%%%%%%%%%%%%%%%%%%%%%%%%%%%%%%%%%%%%%%%%%%%%%%%%%%%%%%%%%%%%%%%%%%
\subsection{Files and Installation}

The package consists of the files:
%
\begin{center}
\begin{tabular}{ll}
    |README.txt|   & readme file \\
    |childdoc.ins| & installation file \\
    |childdoc.dtx| & source file \\
    |childdoc.def| & definition file \\
    |cdocsamp.tex| & sample main file \\
    |cdocsch1.tex| & sample include file \\
    |cdocsch2.tex| & sample include file \\
    |cdocspt3.tex| & sample part file \\
    |cdocspt4.tex| & sample part file \\
    |cdocsdrf.tex| & sample redirection file \\
    |cdocsfn1.tex| & sample redirection file \\
    |cdocsfn2.tex| & sample redirection file \\
    |childdoc.pdf| & manual
\end{tabular}
\end{center}
%
The distribution consists of the files
|README.txt|, |childdoc.ins| and |childdoc.dtx|.
%
\begin{itemize}
\item
Run (pdf)\LaTeX{} on |childdoc.dtx|
to compile the manual |childdoc.pdf| (this file).
\item
Run \LaTeX{} on |childdoc.ins| to create the definitions file |childdoc.def|
and the sample |cdocsamp.tex| with include files
|cdocsch1.tex|, |cdocsch2.tex|, |cdocspt3.tex|, |cdocspt4.tex|,
|cdocsdrf.tex|, |cdocsfn1.tex|, |cdocsfn2.tex|.
Then copy the file |childdoc.def| to an appropriate directory of your \LaTeX{}
distribution, e.g.\ \textit{texmf-root}|/tex/latex/childdoc|.
\end{itemize}

%%%%%%%%%%%%%%%%%%%%%%%%%%%%%%%%%%%%%%%%%%%%%%%%%%%%%%%%%%%%%%%%%%%%%%%%%%%%%%%%
\subsection{Related CTAN Packages}

There are several other packages which offer a similar functionality:
%
\begin{itemize}
\item
The packages
\href{http://ctan.org/pkg/docmute}{\textsf{docmute}},
\href{http://ctan.org/pkg/includex}{\textsf{includex}} and
\href{http://ctan.org/pkg/standalone}{\textsf{standalone}}
provide commands to include only the document body of
a child file thus allowing both files to be compiled individually.
\item
The packages \href{http://ctan.org/pkg/subdocs}{\textsf{subdocs}}
and \href{http://ctan.org/pkg/subfiles}{\textsf{subfiles}}
provide structures in which the main and child documents can be
encapsulated and allowing them to be compiled individually.
The inclusion mechanism is different from the conventional |\include|.
\item
The package \href{http://ctan.org/pkg/combine}{\textsf{combine}}
is an elaborate solution to combine several documents into one.
\end{itemize}
%
See also the CTAN topic \href{http://ctan.org/topic/subdocs}{\textsf{subdocs}}
for further related packages.
The present package differs from the above solutions in that
a document structure constructed with the conventional |\include| mechanism
just needs two extra commands at the top of every file
such that all constituent files can be compiled individually.

%%%%%%%%%%%%%%%%%%%%%%%%%%%%%%%%%%%%%%%%%%%%%%%%%%%%%%%%%%%%%%%%%%%%%%%%%%%%%%%%
%\subsection{Feature Suggestions}
%
%The following is a list of features which may be useful for future
%versions of this package:
%%
%\begin{itemize}
%\item
%\ldots
%\end{itemize}

%%%%%%%%%%%%%%%%%%%%%%%%%%%%%%%%%%%%%%%%%%%%%%%%%%%%%%%%%%%%%%%%%%%%%%%%%%%%%%%%
\subsection{Revision History}

%%%%%%%%%%%%%%%%%%%%%%%%%%%%%%%%%%%%%%%%
\paragraph{v2.0:} 2018/12/30

\begin{itemize}
\item
immediate forward processing
\item
added |\childdocby| mechanism
\item
manual restructured
\end{itemize}

%%%%%%%%%%%%%%%%%%%%%%%%%%%%%%%%%%%%%%%%
\paragraph{v1.6:} 2018/01/17

\begin{itemize}
\item
application for development of include files
\item
corrections to manual
\end{itemize}

%%%%%%%%%%%%%%%%%%%%%%%%%%%%%%%%%%%%%%%%
\paragraph{v1.5:} 2017/05/21

\begin{itemize}
\item
more complete structuring introduced
\item
|\childdocof| introduced
\item
|\childdoc| renamed to |\childdocmain|
\item
|\childredirect| renamed to |\childdocforward| and |\childdocforwardprefix|
and functionality expanded
\end{itemize}

%%%%%%%%%%%%%%%%%%%%%%%%%%%%%%%%%%%%%%%%
\paragraph{v1.0:} 2017/04/27

\begin{itemize}
\item
manual and install package
\item
first version published on CTAN
\end{itemize}

%%%%%%%%%%%%%%%%%%%%%%%%%%%%%%%%%%%%%%%%
\paragraph{v0.6:} 2017/04/26

\begin{itemize}
\item
redirection mechanism added
\end{itemize}

%%%%%%%%%%%%%%%%%%%%%%%%%%%%%%%%%%%%%%%%
\paragraph{v0.5:} 2017/04/26

\begin{itemize}
\item
functionality in definition file
\end{itemize}


%%%%%%%%%%%%%%%%%%%%%%%%%%%%%%%%%%%%%%%%%%%%%%%%%%%%%%%%%%%%%%%%%%%%%%%%%%%%%%%%
%%%%%%%%%%%%%%%%%%%%%%%%%%%%%%%%%%%%%%%%%%%%%%%%%%%%%%%%%%%%%%%%%%%%%%%%%%%%%%%%
%%%%%%%%%%%%%%%%%%%%%%%%%%%%%%%%%%%%%%%%%%%%%%%%%%%%%%%%%%%%%%%%%%%%%%%%%%%%%%%%
\appendix

\settowidth\MacroIndent{\rmfamily\scriptsize 000\ }

 \DocInput{childdoc.dtx}

\end{document}
%</driver>
% \fi
%
% %%%%%%%%%%%%%%%%%%%%%%%%%%%%%%%%%%%%%%%%%%%%%%%%%%%%%%%%%%%%%%%%%%%%%%%%%%%%%%
% %%%%%%%%%%%%%%%%%%%%%%%%%%%%%%%%%%%%%%%%%%%%%%%%%%%%%%%%%%%%%%%%%%%%%%%%%%%%%%
% \section{Sample}
%\iffalse
%<*samplemain>
%\fi
%
% The following presents a sample document
% with two chapters, two parts, a title page,
% a compile flag as well as three forwarding files to set the flag.
% It consists of eight |.tex| files:
% \begin{center}
% \begin{tabular}{ll}
% |cdocsamp.tex|&main file\\
% |cdocsch1.tex|&include file for chapter 1\\
% |cdocsch2.tex|&include file for chapter 2\\
% |cdocspt3.tex|&include file for part 3\\
% |cdocspt4.tex|&include file for part 4\\
% |cdocsdrf.tex|&forwarding file for main file in draft mode\\
% |cdocsfi1.tex|&forwarding file for final version of chapter 1\\
% |cdocsfi2.tex|&forwarding file for final version of chapter 2\\
% \end{tabular}
% \end{center}
% Each of the eight files can be compiled directly by the \LaTeX{} compiler.
%
% %%%%%%%%%%%%%%%%%%%%%%%%%%%%%%%%%%%%%%
% \paragraph{Main File.}
%
% The main file is called |cdocsamp.tex|.
%
% Load the \textsf{childdoc} definitions and
% declare the filename for the main document:
%    \begin{macrocode}
\input{childdoc.def}
\childdocmain{}
%    \end{macrocode}

% Optional override for |\version| flag:
%    \begin{macrocode}
%%\ifchilddoc\else\providecommand{\version}{draft}\fi
%    \end{macrocode}

% Define the default values for the |\version| flag
% (|final| for the main file and |draft| for childs):
%    \begin{macrocode}
\ifchilddoc
\providecommand{\version}{draft}
\else
\providecommand{\version}{final}
\fi
%    \end{macrocode}

% Load the standard document class:
%    \begin{macrocode}
\documentclass[12pt]{article}
%    \end{macrocode}

% Start the document body:
%    \begin{macrocode}
\begin{document}
%    \end{macrocode}

% Declare a title page.
% Print title, part of document being processed and version flag:
%    \begin{macrocode}
\addtocounter{page}{-1}
\begin{center}
{\LARGE\bfseries{}childdoc example\par}
\vspace{1cm}
\ifchilddoc
\ifchilddocmanual part\else chapter\fi:
`\childdocname' of `\childdocjob'\par
\else
main document: `\childdocjob'\par
\fi
version: \version\par
\end{center}
\newpage
%    \end{macrocode}

% Manually include selected file,
% otherwise process as usual:
%    \begin{macrocode}
\ifchilddocmanual
\section*{part `\childdocname'}
\input{\childdocname}
\else
%    \end{macrocode}

% Include the two chapters:
%    \begin{macrocode}
\include{cdocsch1}
\include{cdocsch2}
%    \end{macrocode}

% Include the two parts unless only chapters should be displayed:
%    \begin{macrocode}
\ifchilddoc\else
\section{part three}
\input{cdocspt3}
\section{part four}
\input{cdocspt4}
\fi
%    \end{macrocode}

% Process as usual until here:
%    \begin{macrocode}
\fi
%    \end{macrocode}

% End of document body:
%    \begin{macrocode}
\end{document}
%    \end{macrocode}
%\iffalse
%</samplemain>
%\fi
%
% %%%%%%%%%%%%%%%%%%%%%%%%%%%%%%%%%%%%%%
% \paragraph{Chapter Include Files.}
%
% The include files are called |cdocsch1.tex| and |cdocsch2.tex|.
%
%\iffalse
%<*samplechap1|samplechap2>
%\fi

% Optional override for |\version| flag:
%    \begin{macrocode}
%%\providecommand{\version}{final}
%    \end{macrocode}

% Include the main document:
%    \begin{macrocode}
\input{childdoc.def}
\childdocof{cdocsamp}
%    \end{macrocode}

%\iffalse
%</samplechap1|samplechap2>
%\fi
%
%\iffalse
%<*samplechap1>
%\fi
% Some text for chapter 1:
%    \begin{macrocode}
\section{one}
some text in chapter one
%    \end{macrocode}

%\iffalse
%</samplechap1>
%\fi
% Some text for chapter 2:
%\iffalse
%<*samplechap2>
%\fi
%    \begin{macrocode}
\section{two}
more text in chapter two
%    \end{macrocode}

%\iffalse
%</samplechap2>
%\fi
%
% %%%%%%%%%%%%%%%%%%%%%%%%%%%%%%%%%%%%%%
% \paragraph{Part Include Files.}
%
% The include files are called |cdocspt3.tex| and |cdocspt4.tex|.
%
%\iffalse
%<*samplepart3|samplepart4>
%\fi

% Optional override for |\version| flag:
%    \begin{macrocode}
%%\providecommand{\version}{final}
%    \end{macrocode}

% Include the main document:
%    \begin{macrocode}
\input{childdoc.def}
\childdocby{cdocsamp}
%    \end{macrocode}

%\iffalse
%</samplepart3|samplepart4>
%\fi
%
%\iffalse
%<*samplepart3>
%\fi
% Some text for part 3:
%    \begin{macrocode}
some text in part three
%    \end{macrocode}

%\iffalse
%</samplepart3>
%\fi
% Some text for part 4:
%\iffalse
%<*samplepart4>
%\fi
%    \begin{macrocode}
more text in part four
%    \end{macrocode}

%\iffalse
%</samplepart4>
%\fi
%
% %%%%%%%%%%%%%%%%%%%%%%%%%%%%%%%%%%%%%%
% \paragraph{Forwarding for a Complete Draft.}
%
% The following forwarding file |cdocsdrf.tex|
% compiles the main document in draft mode:
%\iffalse
%<*sampledraft>
%\fi
%    \begin{macrocode}
\def\version{draft}
\input{childdoc.def}
\childdocforward{cdocsamp}
%    \end{macrocode}

%\iffalse
%</sampledraft>
%\fi
%
% %%%%%%%%%%%%%%%%%%%%%%%%%%%%%%%%%%%%%%
% \paragraph{Forwarding for Final Version of the Chapters.}
%
% The following forwarding files |cdocsfn1.tex| and |cdocsfn2.tex|
% (with identical content)
% compile the final versions of the child documents
% |cdocsch1.tex| and |cdocsch2.tex|, respectively:
%\iffalse
%<*samplefinal>
%\fi
%    \begin{macrocode}
\def\version{final}
\input{childdoc.def}
\childdocforwardprefix[cdocsamp]{cdocsfn}{cdocsch}
%    \end{macrocode}

%\iffalse
%</samplefinal>
%\fi
%
% %%%%%%%%%%%%%%%%%%%%%%%%%%%%%%%%%%%%%%
% \paragraph{Command Line Processing.}
%
% The following three command lines generate the output files
% |cdocscld|, |cdocscl1| and |cdocscl2|
% which should be identical to
% |cdocsdrf|, |cdocsch1| and |cdocsfn2|, respectively:
% \begin{center}
% \begin{tabular}{l}
% |latex -jobname cdocscld \|\\
% |  "\def\version{draft}\input{childdoc.def}\childdocforward{cdocsamp}"|\\
% |latex -jobname cdocscl1 \|\\
% |  "\input{childdoc.def}\childdocforward[cdocsamp]{cdocsch1}"|\\
% |latex -jobname cdocscl2 \|\\
% |  "\def\version{final}\input{childdoc.def}\childdocforward{cdocsch2}"|
% \end{tabular}
% \end{center}
% Note that the trailing backslash on each first line
% merely continues the input to the second line
% (for convenient cut ant paste).
% Furthermore, the command |latex| can be replaced by any
% of its alternative versions such as |pdflatex|.
%
% %%%%%%%%%%%%%%%%%%%%%%%%%%%%%%%%%%%%%%%%%%%%%%%%%%%%%%%%%%%%%%%%%%%%%%%%%%%%%%
% %%%%%%%%%%%%%%%%%%%%%%%%%%%%%%%%%%%%%%%%%%%%%%%%%%%%%%%%%%%%%%%%%%%%%%%%%%%%%%
% \section{Implementation}
%\iffalse
%<*package>
%\fi
%
% This section describes the definitions file |childdoc.def|.

% The definitions cannot be loaded using |\usepackage| or |\RequirePackage|
% which has a mechanism to prevent loading a style file more than once.
% When loading the definitions by means of |\input|
% multiple instances have to be prevented manually:
%\iffalse
%This code needs to be before the `\ProvidesFile' directive
%which is defined at the beginning of this file.
%Therefore it is also placed there and commented out here.
%</package>
%<*discard>
%\fi
%    \begin{macrocode}
\ifdefined\childdocmain\endinput\fi
%    \end{macrocode}
%\iffalse
%</discard>
%<*package>
%\fi
%
% \macro{\ifchilddoc}
% \macro{\ifchilddocmanual}
% The conditional |\ifchilddoc| tells whether a
% child (true) or main (false) document is being compiled.
% The conditional |\ifchilddocmanual| tells whether
% the |\includeonly| mechanism is used (false) or
% the selection of child files must be performed manually (true).
% The definitions initialise to false:
%    \begin{macrocode}
\newif\ifchilddoc
\newif\ifchilddocmanual
%    \end{macrocode}

% \macro{\childdocname}
% \macro{\childdocjob}
% The macro |\childdocname| stores the name of the main document
% to be compiled. The macro |\childdocjob| stores the name of
% the document on which the \LaTeX{} compiler was originally invoked.
% The content of |\jobname| cannot be compared
% to filenames specified in the source due to different catcodes.
% The following code rescans |\jobname|, stores the result
% in |\childdocname| and saves a copy in |\childdocjob|:
%    \begin{macrocode}
\edef\childdocname{\scantokens\expandafter{\jobname\noexpand}}
\let\childdocjob\childdocname
%    \end{macrocode}

% \macro{\childdocdisable}
% The macro |\childdocdisable| prevents the main file
% from being processed more than once.
% At this stage, the main document command |\childdocmain|
% is assumed to be called once again where it should do nothing.
% Any subsequent call to it should prevent
% a secondary processing of the main document
% It overwrites the forwarding commands
% |\childdocof| and |\childdocforward|
% with empty macros to prevent further inclusions of the main document:
%    \begin{macrocode}
\newcommand{\childdocdisable}
{
  \renewcommand{\childdocmain}[1]{\renewcommand{\childdocmain}[1]{\endinput}}
  \renewcommand{\childdocof}[1]{}
  \renewcommand{\childdocby}[2][]{}
  \renewcommand{\childdocforward}[2][]{}
  \renewcommand{\childdocdisable}{}
}
%    \end{macrocode}

% \macro{\childdocmain}
% The macro |\childdocmain| is to be called at the top of the main file
% with nothing or the main filename (without extension) as argument.
% First, it breaks loops.
% If the argument is not empty and does not match |\childdocname|
% (which is set by the first inclusion of |childdoc.def|),
% |\ifchilddoc| is set to true, |\includeonly| is applied to the child file
% and |\jobname| is set to the main file
% (for proper handling of |.aux| files):
%    \begin{macrocode}
\newcommand{\childdocmain}[1]
{
  \childdocdisable\childdocmain{}
  \if?#1?\else
    \begingroup
      \def\childdoctmp{#1}
      \ifx\childdoctmp\childdocname
        \def\childdoctmp{}
      \else
        \def\childdoctmp
        {
          \childdoctrue
          \includeonly{\childdocname}
          \def\childdocjob{#1}
          \def\jobname{#1}
        }
      \fi
      \expandafter
    \endgroup
    \childdoctmp
  \fi
}
%    \end{macrocode}

% \macro{\childdocof}
% The command |\childdocof| redirects
% compilation to the main file |#1|.
%    \begin{macrocode}
\newcommand{\childdocof}[1]
{
  \childdocdisable
  \childdoctrue
  \includeonly{\childdocname}
  \def\jobname{#1}
  \def\childdocjob{#1}
  \input{#1}
}
%    \end{macrocode}

% \macro{\childdocby}
% The command |\childdocby| ....
%    \begin{macrocode}
\newcommand{\childdocby}[2][]
{
  \childdocdisable
  \childdoctrue
  \childdocmanualtrue
  \if?#1?\else
    \def\jobname{#2}
  \fi
  \def\childdocjob{#2}
  \input{#2}
  \endinput
}
%    \end{macrocode}

% \macro{\childdocforward}
% The command |\childdocforward| redirects
% compilation to the main file or
% (if the optional argument is given) a child file.
% Parameters are set as if the main file
% or a child file starting with |\childdocof| was compiled.
% Then compilation is handed over to the main file:
%    \begin{macrocode}
\newcommand{\childdocforward}[2][]
{
  \begingroup
    \if?#1?
      \def\childdoctmp
      {
        \def\childdocname{#2}
        \def\childdocjob{#2}
        \def\jobname{#2}
        \input{#2}
        \endinput
      }
    \else
      \def\childdoctmp
      {
        \childdocdisable
        \def\childdocname{#2}
        \childdoctrue
        \includeonly{#2}
        \def\childdocjob{#1}
        \def\jobname{#1}
        \input{#1}
        \endinput
      }
    \fi
    \expandafter
  \endgroup
  \childdoctmp
}
%    \end{macrocode}

% \macro{\childdocforwardprefix}
% The command |\childdocforwardprefix| redirects
% compilation to the main or a child file by means of a pattern.
% The prefix |#1| in the current filename is replaced by |#2|
% and the suffix of the current filename is kept
% (it is assumed that the filename does not contain the substring `|~~~|'
% which is used as a delimiter).
% Compilation is handed over to the new file by |\childdocforward|:
%    \begin{macrocode}
\newcommand{\childdocforwardprefix}[3][]
{
  \begingroup
    \def\childdocextract #2##1~~~{\def\childdoctmp{\childdocforward[#1]{#3##1}}}
    \expandafter\childdocextract\childdocname~~~
    \expandafter
  \endgroup
  \childdoctmp
}
%    \end{macrocode}

% \macro{\childdoc}
% The deprecated macro |\childdoc| is a legacy version of |\childdocmain|:
%    \begin{macrocode}
\newcommand{\childdoc}{\childdocmain}
%    \end{macrocode}

% \macro{\childdocredirect}
% The deprecated macro |\childdocredirect| is a legacy version
% of |\childdocforward| and |\childdocforwardprefix|:
%    \begin{macrocode}
\newcommand{\childdocredirect}[2][]
{
  \begingroup
    \if?#1?
      \def\childdoctmp{\childdocforward{#2}}
    \else
      \def\childdoctmp{\childdocforwardprefix{#1}{#2}}
    \fi
    \expandafter
  \endgroup
  \childdoctmp
}
%    \end{macrocode}

%\iffalse
%</package>
%\fi
%
\endinput

\childdocforwardprefix[cdocsamp]{cdocsfn}{cdocsch}
%    \end{macrocode}

%\iffalse
%</samplefinal>
%\fi
%
% %%%%%%%%%%%%%%%%%%%%%%%%%%%%%%%%%%%%%%
% \paragraph{Command Line Processing.}
%
% The following three command lines generate the output files
% |cdocscld|, |cdocscl1| and |cdocscl2|
% which should be identical to
% |cdocsdrf|, |cdocsch1| and |cdocsfn2|, respectively:
% \begin{center}
% \begin{tabular}{l}
% |latex -jobname cdocscld \|\\
% |  "\def\version{draft}% \iffalse
%
% childdoc.dtx Copyright (C) 2017-2018 Niklas Beisert
%
% This work may be distributed and/or modified under the
% conditions of the LaTeX Project Public License, either version 1.3
% of this license or (at your option) any later version.
% The latest version of this license is in
%   http://www.latex-project.org/lppl.txt
% and version 1.3 or later is part of all distributions of LaTeX
% version 2005/12/01 or later.
%
% This work has the LPPL maintenance status `maintained'.
%
% The Current Maintainer of this work is Niklas Beisert.
%
% This work consists of the files childdoc.dtx and childdoc.ins
% and the derived files childdoc.def and cdocsamp.tex with
% cdocsch1.tex, cdocsch2.tex, cdocsdrf.tex, cdocsfn1.tex, cdocsfn2.tex.
%
%<package>\ifdefined\childdocmain\endinput\fi
%<package>\ProvidesFile{childdoc.def}[2018/12/30 v2.0 child document driver]
%<samplemain>\ProvidesFile{cdocsamp.tex}[2018/12/30 v2.0 sample for childdoc]
%<*driver>
%\ProvidesFile{childdoc.drv}[2018/12/30 v2.0 childdoc reference manual file]
\PassOptionsToClass{10pt,a4paper}{article}
\documentclass{ltxdoc}

\usepackage[margin=35mm]{geometry}
\usepackage{hyperref}
\usepackage{hyperxmp}
\usepackage[usenames]{color}

\hypersetup{colorlinks=true}
\hypersetup{pdfstartview=FitH}
\hypersetup{pdfpagemode=UseNone}
\hypersetup{pdfsource={}}
\hypersetup{pdflang={en-UK}}
\hypersetup{pdfcopyright={Copyright 2017-2018 Niklas Beisert.
  This work may be distributed and/or modified under the
  conditions of the LaTeX Project Public License, either version 1.3
  of this license or (at your option) any later version.}}
\hypersetup{pdflicenseurl={http://www.latex-project.org/lppl.txt}}
\hypersetup{pdfcontactaddress={ETH Zurich, ITP, HIT K,
  Wolfgang-Pauli-Strasse 27}}
\hypersetup{pdfcontactpostcode={8093}}
\hypersetup{pdfcontactcity={Zurich}}
\hypersetup{pdfcontactcountry={Switzerland}}
\hypersetup{pdfcontactemail={nbeisert@itp.phys.ethz.ch}}
\hypersetup{pdfcontacturl={http://people.phys.ethz.ch/\xmptilde nbeisert/}}

\newcommand{\secref}[1]{\hyperref[#1]{section \ref*{#1}}}

\parskip1ex
\parindent0pt
\let\olditemize\itemize
\def\itemize{\olditemize\parskip0pt}

\begin{document}

\title{The \textsf{childdoc} Package}
\hypersetup{pdftitle={The childdoc Package}}
\author{Niklas Beisert\\[2ex]
  Institut f\"ur Theoretische Physik\\
  Eidgen\"ossische Technische Hochschule Z\"urich\\
  Wolfgang-Pauli-Strasse 27, 8093 Z\"urich, Switzerland\\[1ex]
  \href{mailto:nbeisert@itp.phys.ethz.ch}
  {\texttt{nbeisert@itp.phys.ethz.ch}}}
\hypersetup{pdfauthor={Niklas Beisert}}
\hypersetup{pdfsubject={Manual for the LaTeX2e Package childdoc}}
\date{30 December 2018, \textsf{v2.0}}
\maketitle

\begin{abstract}\noindent
\textsf{childdoc} is a \LaTeXe{} package
that enables the direct compilation
of document sections included by |\include|
to individual files.
\end{abstract}

\begingroup
\parskip0ex
\tableofcontents
\endgroup

%%%%%%%%%%%%%%%%%%%%%%%%%%%%%%%%%%%%%%%%%%%%%%%%%%%%%%%%%%%%%%%%%%%%%%%%%%%%%%%%
%%%%%%%%%%%%%%%%%%%%%%%%%%%%%%%%%%%%%%%%%%%%%%%%%%%%%%%%%%%%%%%%%%%%%%%%%%%%%%%%
\section{Introduction}

\LaTeX{} provides a mechanism to structure a large document (such as a book)
into a main file and several child files (containing the chapters)
using the |\include| command.
This mechanism is beneficial for documents
which span hundreds of pages in order to
make the source file(s) more manageable.
Moreover, compilation can be restricted to
selected child files by means of the |\includeonly| command.
The latter feature can be used to reduce the compilation time while editing
(this was significantly more useful in the earlier days of \LaTeX{})
or to generate a smaller document which is easier to navigate.
Another application of |\includeonly| is to generate
documents consisting of selected parts of the complete document.

However, there are a few drawbacks of the plain |\include| mechanism:
\begin{itemize}
\item
The child files cannot be compiled on their own,
they can only be compiled via the main file.
A naive editing environment
(such as a text editor with an option
to have the current file processed by \LaTeX)
may require one to switch to the main file before compiling;
attempting to compile the child file produces errors.
\item
The main file must be modified (each time)
to adjust the |\includeonly| command
to the present needs. This easily leaves the main file in a messy state.
\item
The generated document will always carry the filename
of the main document. This is inconvenient if
several child files are to be compiled and
to be kept for distribution.
\end{itemize}

The present package provides a simple interface
to make child files individually compilable by \LaTeX{}.
Compiling a child file then has the same effect as compiling
the main file with an |\includeonly| command
to select the appropriate child.
Moreover the generated document will carry the name of the child
rather than the main file.
This resolves all three above issues.

This feature is meant to make the editing of books,
thesis documents and lecture notes somewhat more convenient.
However, the package can also be used efficiently for
composing a series of documents (such as exercise sheets)
which are typically distributed individually.
It then assists the author in generating the individual documents
(potentially in different versions)
as well as a document containing the collected series.
Another application is in developing style files
or other kinds of included material
where compilation of the style file could redirect
to a sample or test file.

%%%%%%%%%%%%%%%%%%%%%%%%%%%%%%%%%%%%%%%%%%%%%%%%%%%%%%%%%%%%%%%%%%%%%%%%%%%%%%%%
%%%%%%%%%%%%%%%%%%%%%%%%%%%%%%%%%%%%%%%%%%%%%%%%%%%%%%%%%%%%%%%%%%%%%%%%%%%%%%%%
\section{Usage}

First of all, the package \textsf{childdoc} is \emph{not} a standard
\LaTeXe{} |.sty| style file! Therefore it needs to be invoked in
a non-standard way.

%%%%%%%%%%%%%%%%%%%%%%%%%%%%%%%%%%%%%%%%%%%%%%%%%%%%%%%%%%%%%%%%%%%%%%%%%%%%%%%%
\subsection{Included Files}
\label{sec:include}

%%%%%%%%%%%%%%%%%%%%%%%%%%%%%%%%%%%%%%%%
\DescribeMacro{\childdocmain}
To use the package, add the commands
\begin{center}
\begin{tabular}{l}
|\input{childdoc.def}|\\
|\childdocmain{}|\\
\end{tabular}
\end{center}
at the very top of the main \LaTeX{} file,
in particular \emph{before} the |\documentclass| statement!
The argument of |\childdocmain| should be left empty
(but it must be present).

%%%%%%%%%%%%%%%%%%%%%%%%%%%%%%%%%%%%%%%%
\DescribeMacro{\childdocof}
Furthermore, add the commands
\begin{center}
\begin{tabular}{l}
|\input{childdoc.def}|\\
|\childdocof{|\textit{main}|}|\\
\end{tabular}
\end{center}
at the top of every child file \textit{child}
which is included by |\include{|\textit{child}|}|
from within the main file
(or at least for those files to be compiled individually).
The argument \textit{main} must be the filename of the main file.

There are a couple of
considerations in setting up the main and child documents:

%%%%%%%%%%%%%%%%%%%%%%%%%%%%%%%%%%%%%%%%
\paragraph{Restrictions.}

Please note the following restrictions:
\begin{itemize}
\item
|\childdocmain| must be called with one argument \textit{main}
to ensure compatibility with earlier version of the package.
It must either be empty (|\childdocmain{}|)
or precisely match the filename of the main file in which it is specified.
See \secref{sec:detection} for further information.
\item
The filename \textit{main} must be specified without the |.tex| extension.
\item
The filename \textit{main} is case sensitive
(even in case-insensitive file systems)
due to internal string comparison.
\item
The argument \textit{main} should be fully expanded, it cannot be a macro.
\item
Subdirectories and special characters should be avoided in filenames.
\item
The command |\childdocmain{|\textit{main}|}| must be followed by a whitespace.
It should not be followed immediately by another command
or by a comment mark `|%|'.
This is because the \TeX{} parser reads the token immediately following
the argument of |\childdocmain| and puts it
at the beginning of every child section;
however, a white\-space is ignored.
\end{itemize}

%%%%%%%%%%%%%%%%%%%%%%%%%%%%%%%%%%%%%%%%
\paragraph{Content of Main File.}

It is advisable to place all content in the child files included by |\include|.
Any output contained in the main file will appear in all child documents
unless suppressed manually;
it cannot be suppressed automatically by the |\includeonly| directive
and thus should normally be avoided.
A method to include some content in the main file
by means of conditional processing is described in \secref{sec:conditional}.

%%%%%%%%%%%%%%%%%%%%%%%%%%%%%%%%%%%%%%%%
\paragraph{Page Numbering.}

When only a part of the document is compiled,
the appropriate numbering of pages
(as well as other status parameters)
is determined from the |.aux| files.
The latter contain information from previous passes.
However this information needs to propagate through
all intermediate child documents.
Therefore the page numbering in child documents may well
be inconsistent until the complete document is compiled at least once.

A useful (if unconventional) way to always ensure a consistent
page numbering is to restart the numbering in each child document
and denote the pages by `\textit{child}|.|\textit{page}'
where \textit{child} represents the chapter/section number of the child file.
This can be achieved by the command
|\numberwithin{page}{|\textit{child}|}|
of the \textsf{amsmath} package
where \textit{child} can be |chapter| or |section|
depending on the chosen structuring.
Alternatively, one can modify the macro |\thepage| appropriately
and reset the counter |page| at the start of each child file.

%%%%%%%%%%%%%%%%%%%%%%%%%%%%%%%%%%%%%%%%%%%%%%%%%%%%%%%%%%%%%%%%%%%%%%%%%%%%%%%%
\subsection{Conditional Processing}
\label{sec:conditional}

The package provides a mechanism to compile different versions
of a document. To customise the versions further some conditional processing
can come in handy to distinguish which version is being compiled.
The package provides two macros to describe the compilation context:

%%%%%%%%%%%%%%%%%%%%%%%%%%%%%%%%%%%%%%%%
\DescribeMacro{\ifchilddoc}
The conditional |\ifchilddoc| distinguishes between the compilation of
child documents and the main document:
%
\begin{center}
|\ifchilddoc |\textit{child-code}| |[|\||else |\textit{main-code}]| \||fi|
\end{center}

%%%%%%%%%%%%%%%%%%%%%%%%%%%%%%%%%%%%%%%%
\DescribeMacro{\childdocname}
\DescribeMacro{\childdocjob}
The macro |\childdocname| contains the filename (without extension)
of the main or child file being processed.
Note that |\childdocjob| will always contain the name of the main file.

%%%%%%%%%%%%%%%%%%%%%%%%%%%%%%%%%%%%%%%%
\paragraph{Title Page.}

Conditional processing can be used to include a title or banner page
in the main document when proper precautions are taken.
Importantly, the code in the main file should ensure that the page counter
(as well as other status parameters which are stored in the |.aux| files)
takes the same value after the conditional processing.
Otherwise the page numbers may take divergent values
depending on which part is compiled.

For example, a title page could be declared by:
%
\begin{center}
\begin{tabular}{l}
|\ifchilddoc\||else|\\
|\addtocounter{page}{-1}|\\
\textit{code for title page}\\
|\newpage|\\
|\||fi|
\end{tabular}
\end{center}
%
A banner page for the child documents can be generated by:
%
\begin{center}
\begin{tabular}{l}
|\ifchilddoc|\\
|\addtocounter{page}{-1}|\\
\textit{code for banner page}\\
|\newpage|\\
|\||fi|
\end{tabular}
\end{center}
%
Here one could write a message such as:
\begin{center}
|This is the part \childdocname{} of \childdocjob{}.|
\end{center}

%%%%%%%%%%%%%%%%%%%%%%%%%%%%%%%%%%%%%%%%%%%%%%%%%%%%%%%%%%%%%%%%%%%%%%%%%%%%%%%%
\subsection{Flags}
\label{sec:flags}

The package makes it easy to generate different versions
of the main or child documents.
To this end compilation flags can be defined
and assigned different default values.
They will be particularly useful in conjunction
with the forwarding mechanism described in \secref{sec:forward}.

For example, it may be useful to have a flag |\version|
which can be set to |draft| or |final|.
The document source will contain some conditional code
depending on the value of |\version|.
Suppose further, the flag should default to |final| for the main file
and to |draft| for child files
which is a natural assignment for editing the document.
This is achieved by placing the following code
in the preamble of the main document
(below the |\childdocmain| directive):
%
\begin{center}
\begin{tabular}{l}
|\ifchilddoc|\\
|\providecommand{\version}{draft}|\\
|\||else|\\
|\providecommand{\version}{final}|\\
|\||fi|
\end{tabular}
\end{center}
%
The definition by |\providecommand| makes sure
that previous definitions are not overwritten.
Further statements |\providecommand{\version}{...}|
can thus be added before the above code to override it.

For the main file, one might add a line
(between |\childdocmain| and the above block)
%
\begin{center}
|%\ifchilddoc\||else\providecommand{\version}{draft}\||fi|
\end{center}
%
which can be uncommented to produce a draft version.
Likewise one can add a line to the very top of a child file
(above the |\childdocof{|\textit{main}|}| directive)
%
\begin{center}
|%\providecommand{\version}{final}|
\end{center}
%
which can be uncommented to produce the final version of this child document.

%%%%%%%%%%%%%%%%%%%%%%%%%%%%%%%%%%%%%%%%%%%%%%%%%%%%%%%%%%%%%%%%%%%%%%%%%%%%%%%%
\subsection{Forwarding}
\label{sec:forward}

Different versions of the main or child documents
using compilation flags as described in \secref{sec:flags}
can be (permanently) stored in different files
for convenient compilation, viewing and distribution.
To this end, the package defines a command
to pass on compilation to a different file:

%%%%%%%%%%%%%%%%%%%%%%%%%%%%%%%%%%%%%%%%
\DescribeMacro{\childdocforward}
The command |\childdocforward| redirects processing to
another source file:
%
\begin{center}
\begin{tabular}{l}
|\input{childdoc.def}|\\
|\childdocforward[|\textit{main}|]{|\textit{dest}|}|\\
\end{tabular}
\end{center}
%
The argument \textit{dest} is the destination file
(without extension).
It should be the main file or one of the child files.
Note that further \textsf{childdoc} directives
such as |\childdocof| and |\childdocforward|
in the indicated file will be processed in this form.
The optional argument \textit{main}
passes on directly to the main file \textit{main}
while pretending to compile the child \textit{dest}.
This form behaves as if \textit{dest}
issues |\childdocof{|\textit{main}|}| right away,
and no further \textsf{childdoc} directives will be processed.

%%%%%%%%%%%%%%%%%%%%%%%%%%%%%%%%%%%%%%%%
\DescribeMacro{\...prefix}
In the alternative form |\childdocforwardprefix|,
%
\begin{center}
\begin{tabular}{l}
|\input{childdoc.def}|\\
|\childdocforwardprefix[|\textit{main}|]{|\textit{prefix}|}{|\textit{dest}|}|
\end{tabular}
\end{center}
%
the destination file is determined by a pattern
depending on the current file:
To make this work, the current file must be called
`{\textit{prefix}\hspace{0.2em}\textit{suffix}}'
with \textit{prefix} matching precisely the argument.
Processing is then passed on to the file
`{\textit{dest}\hspace{0.2em}\textit{suffix}}'.
Surely, the same effect is achieved by
directly specifying the
argument `{\textit{dest}\hspace{0.2em}\textit{suffix}}'
in the first form.
However, that requires to set up a different file
for each child. With the alternative form of the command
all these files can have exactly the same content
which simplifies setting them up and maintaining them.

For example, the following file |draft.tex|
with a compilation flag |\version| as described in \secref{sec:flags}
compiles the main document as a draft:
%
\begin{center}
\begin{tabular}{l}
|\def\version{draft}|\\
|\input{childdoc.def}|\\
|\childdocforward{|\textit{main}|}|
\end{tabular}
\end{center}
%
Likewise, the following files |final|\textit{nn}|.tex|
compile the final version of the child document
|child|\textit{nn}|.tex|:
%
\begin{center}
\begin{tabular}{l}
|\def\version{final}|\\
|\input{childdoc.def}|\\
|\childdocforwardprefix{final}{child}|
\end{tabular}
\end{center}
%

Note that when several versions of a main file and/or of each child file
are to be generated, it may be convenient to set up a |Makefile| or
shell script to automatise the process.

%%%%%%%%%%%%%%%%%%%%%%%%%%%%%%%%%%%%%%%%%%%%%%%%%%%%%%%%%%%%%%%%%%%%%%%%%%%%%%%%
\subsection{Command Line Processing}
\label{sec:commandline}

The effect of redirection files can also be achieved by invoking
the \LaTeX{} compiler with a more elaborate command line.
Most conveniently this should be done as part
of a shell script or a |Makefile|.

When using \textsf{childdoc} in the main file, the following
command lines effectively perform a redirection
(note that depending on the shell being used,
backslashes may have to be doubled: `|\|' $\to$ `|\\|'):
%
\begin{center}
|... -jobname "|\textit{target}|" |\\|"|[\textit{flags}]%
|\input{childdoc.def}\childdocforward[|\textit{main}|]{|\textit{dest}|}"|
\end{center}
%
Here \textit{target} is the name of the output file,
\textit{main} is the name of the main file
and \textit{dest} is the name of the main or child file to be processed
(all filenames without extensions).
The optional argument \textit{main} can be omitted
if \textit{main} matches \textit{dest}.
Optionally, compilation \textit{flags} can be defined via |\def| commands.
This command line makes the \TeX{} engine believe
it is compiling the file \textit{target}
whose content is specified as the latter parameter.
The provided code then forwards the processing to
\textit{main} or \textit{dest} as described in \secref{sec:forward}.

%%%%%%%%%%%%%%%%%%%%%%%%%%%%%%%%%%%%%%%%%%%%%%%%%%%%%%%%%%%%%%%%%%%%%%%%%%%%%%%%
\subsection{Include by Input}
\label{sec:input}

Including child documents by |\include| has some restrictions by design.
Most notably, the content of a child document always occupies
its own set of pages; pages cannot be shared between child documents.
Usually, this behaviour makes perfect sense
because each child document contain an essential part of the document.
However, in some situations it may be desirable to compose
a document from a collection of parts
without having mandatory page breaks between then.
For this case, the package
provides a mechanism to include parts
by |\input| which can also be processed individually.
However, by construction this mechanism
requires manual handling of the content to be output.

%%%%%%%%%%%%%%%%%%%%%%%%%%%%%%%%%%%%%%%%
\DescribeMacro{\ifchilddocmanual}
The main file should be prepared as usual, see \secref{sec:include}.
However, the document body must make a distinction
between processing of an individual part and of the main document, e.g.:
%
\begin{center}
\begin{tabular}{l}
|\ifchilddocmanual|\\
|\input{\childdocname}|\\
|\||else|\\
\textit{document body with }|\input{|\textit{part}|}|\\
|\||fi|
\end{tabular}
\end{center}
%
The conditional |\ifchilddocmanual| is true whenever
a part to be included by |\input| is being compiled,
and the name of the part is stored in |\childdocname|.

%%%%%%%%%%%%%%%%%%%%%%%%%%%%%%%%%%%%%%%%
\DescribeMacro{\childdocby}
Each part to be included by |\input| should start with:
%
\begin{center}
\begin{tabular}{l}
|\input{childdoc.def}|\\
|\childdocby{|\textit{main}|}|\\
\end{tabular}
\end{center}
%
The directive |\childdocby| is similar to |\childdocof|
described in \secref{sec:include},
but the subsequent selection of content must be done manually.
To that end, both |\ifchilddoc| and |\ifchilddocmanual|
will be true upon processing of a part,
and the name of the part is stored in |\childdocname|.
Note that |\jobname| will be set to the filename of the current part
so that each part receives an individual |.aux| file
that does not interfere with the |.aux| file(s) of the main document.
This behaviour can be altered by the alternative form
|\childdocby[*]{|\textit{main}|}| (with a non-empty optional argument)
which uses the |.aux| file of the main document
by setting |\jobname| to \textit{main}.

%%%%%%%%%%%%%%%%%%%%%%%%%%%%%%%%%%%%%%%%%%%%%%%%%%%%%%%%%%%%%%%%%%%%%%%%%%%%%%%%
\subsection{Driver Development}
\label{sec:driver}

The \textsf{childdoc} mechanism can also be use for the development
of definition files such as \LaTeX{} styles or classes.
This case differs from the above setup with multiple parts
included by |\include| in that no |\includeonly| should be invoked.
This can be achieved by starting the include file
(before |\ProvidesPackage|) with:
%
\begin{center}
\begin{tabular}{l}
|\input{childdoc.def}|\\
|\childdocforward{|\textit{main}|}|\\
\end{tabular}
\end{center}
%
or alternatively with:
%
\begin{center}
\begin{tabular}{l}
|\input{childdoc.def}|\\
|\childdocby{|\textit{main}|}|\\
\end{tabular}
\end{center}
%
Both forms have slightly different effects as described above.
The main file is prepared as usual, see \secref{sec:include}.

%%%%%%%%%%%%%%%%%%%%%%%%%%%%%%%%%%%%%%%%%%%%%%%%%%%%%%%%%%%%%%%%%%%%%%%%%%%%%%%%
\subsection{Legacy Detection}
\label{sec:detection}

The directive |\childdocmain| in the main file can detect
whether the complete document or merely a child is to be compiled
even without using the directive |\childdocof|.
This method is deprecated because it is less robust
and there is no compelling reason to use it;
it is merely provided for backward compatibility
and it may be removed in future versions.

If the detection mechanism is to be used,
it is mandatory to correctly specify
the filename of the main file as the argument of |\childdocmain|:
%
\begin{center}
\begin{tabular}{l}
|\input{childdoc.def}|\\
|\childdocmain{|\textit{main}|}|\\
\end{tabular}
\end{center}
%
If |\jobname| does not match the argument \textit{main} of |\childdocmain|,
it is assumed that |\jobname| points to the child file to be compiled.
When using |\childdocmain| with the main file specified as argument,
it suffices to start a child file
with just |\input{|\textit{main}|}|
without loading of the package and using |\childdocof|.
If instead all processing is done
with the appropriate \textsf{childdoc} directives,
the argument of \textit{main} of |\childdocmain| can be empty.

An alternative version of the command line processing described
in \secref{sec:commandline} using the detection mechanism reads:
%
\begin{center}
|... -jobname "|\textit{target}|" "|[\textit{flags}]%
[|\def\jobname{|\textit{dest}|}|]|\input{|\textit{main}|}"|
\end{center}

%%%%%%%%%%%%%%%%%%%%%%%%%%%%%%%%%%%%%%%%%%%%%%%%%%%%%%%%%%%%%%%%%%%%%%%%%%%%%%%%
\subsection{Manual Code}
\label{sec:manual}

In case one cannot be certain whether the definitions file |childdoc.def|
is installed on the target \TeX{} distribution
and one prefers not to ship it,
it is conceivable to paste a few relevant commands into the sources.

To that end, drop all statements |\input{childdoc.def}|
and perform the replacements as outlined below.
Instead of |\childdocmain{|\textit{main}|}| add the following code
to the top of the main file:
%
\begin{center}
\begin{tabular}{l}
|\||ifdefined\childdocname\endinput\||fi\newif\ifchilddoc|\\
|\edef\childdocname{\scantokens\expandafter{\jobname\noexpand}}|\\
|\def\childdocmain{|\textit{main}|}\||ifx\childdocmain\childdocname\||else|\\
|\childdoctrue\includeonly{\childdocname}\let\jobname\childdocmain\||fi|\\
\end{tabular}
\end{center}
%
Instead of |\childdocof{|\textit{main}|}| just include the main file
at the top of each child file:
%
\begin{center}
|\input{|\textit{main}|}|
\end{center}
%
A simple redirection |\childdocforward{|\textit{dest}|}| is achieved by:
%
\begin{center}
|\def\jobname{|\textit{dest}|}\input{\jobname}|
\end{center}
%
The redirection with prefix
|\childdocforwardprefix[|\textit{prefix}|]{|\textit{dest}|}|
is accomplished by:
%
\begin{center}
\begin{tabular}{l}
|{\edef\jobname{\scantokens\expandafter{\jobname\noexpand}}|\\
|\def\redirectjob |\textit{prefix}|#1~~~{\gdef\jobname{|\textit{dest}|#1}}|\\
|\expandafter\redirectjob\jobname~~~}\input{\jobname}|
\end{tabular}
\end{center}

In an alternative approach,
child documents can be compiled by a specific command line
without additional code or specific definitions:
%
\begin{center}
|... -jobname "|\textit{target}|" "|[\textit{flags}]%
|\includeonly{|\textit{dest}|}\input{|\textit{main}|}"|
\end{center}
%

%%%%%%%%%%%%%%%%%%%%%%%%%%%%%%%%%%%%%%%%%%%%%%%%%%%%%%%%%%%%%%%%%%%%%%%%%%%%%%%%
%%%%%%%%%%%%%%%%%%%%%%%%%%%%%%%%%%%%%%%%%%%%%%%%%%%%%%%%%%%%%%%%%%%%%%%%%%%%%%%%
\section{Information}

%%%%%%%%%%%%%%%%%%%%%%%%%%%%%%%%%%%%%%%%%%%%%%%%%%%%%%%%%%%%%%%%%%%%%%%%%%%%%%%%
\subsection{Copyright}

Copyright \copyright{} 2017--2018 Niklas Beisert

This work may be distributed and/or modified under the
conditions of the \LaTeX{} Project Public License, either version 1.3
of this license or (at your option) any later version.
The latest version of this license is in
  \url{http://www.latex-project.org/lppl.txt}
and version 1.3 or later is part of all distributions of \LaTeX{}
version 2005/12/01 or later.

This work has the LPPL maintenance status `maintained'.

The Current Maintainer of this work is Niklas Beisert.

This work consists of the files |README.txt|, |childdoc.ins| and |childdoc.dtx|
as well as the derived files |childdoc.def|, |cdocsamp.tex|
with |cdocsch1.tex|, |cdocsch2.tex|, |cdocspt3.tex|, |cdocspt4.tex|,
|cdocsdrf.tex|, |cdocsfn1.tex|, |cdocsfn2.tex|
as well as |childdoc.pdf|.

%%%%%%%%%%%%%%%%%%%%%%%%%%%%%%%%%%%%%%%%%%%%%%%%%%%%%%%%%%%%%%%%%%%%%%%%%%%%%%%%
\subsection{Files and Installation}

The package consists of the files:
%
\begin{center}
\begin{tabular}{ll}
    |README.txt|   & readme file \\
    |childdoc.ins| & installation file \\
    |childdoc.dtx| & source file \\
    |childdoc.def| & definition file \\
    |cdocsamp.tex| & sample main file \\
    |cdocsch1.tex| & sample include file \\
    |cdocsch2.tex| & sample include file \\
    |cdocspt3.tex| & sample part file \\
    |cdocspt4.tex| & sample part file \\
    |cdocsdrf.tex| & sample redirection file \\
    |cdocsfn1.tex| & sample redirection file \\
    |cdocsfn2.tex| & sample redirection file \\
    |childdoc.pdf| & manual
\end{tabular}
\end{center}
%
The distribution consists of the files
|README.txt|, |childdoc.ins| and |childdoc.dtx|.
%
\begin{itemize}
\item
Run (pdf)\LaTeX{} on |childdoc.dtx|
to compile the manual |childdoc.pdf| (this file).
\item
Run \LaTeX{} on |childdoc.ins| to create the definitions file |childdoc.def|
and the sample |cdocsamp.tex| with include files
|cdocsch1.tex|, |cdocsch2.tex|, |cdocspt3.tex|, |cdocspt4.tex|,
|cdocsdrf.tex|, |cdocsfn1.tex|, |cdocsfn2.tex|.
Then copy the file |childdoc.def| to an appropriate directory of your \LaTeX{}
distribution, e.g.\ \textit{texmf-root}|/tex/latex/childdoc|.
\end{itemize}

%%%%%%%%%%%%%%%%%%%%%%%%%%%%%%%%%%%%%%%%%%%%%%%%%%%%%%%%%%%%%%%%%%%%%%%%%%%%%%%%
\subsection{Related CTAN Packages}

There are several other packages which offer a similar functionality:
%
\begin{itemize}
\item
The packages
\href{http://ctan.org/pkg/docmute}{\textsf{docmute}},
\href{http://ctan.org/pkg/includex}{\textsf{includex}} and
\href{http://ctan.org/pkg/standalone}{\textsf{standalone}}
provide commands to include only the document body of
a child file thus allowing both files to be compiled individually.
\item
The packages \href{http://ctan.org/pkg/subdocs}{\textsf{subdocs}}
and \href{http://ctan.org/pkg/subfiles}{\textsf{subfiles}}
provide structures in which the main and child documents can be
encapsulated and allowing them to be compiled individually.
The inclusion mechanism is different from the conventional |\include|.
\item
The package \href{http://ctan.org/pkg/combine}{\textsf{combine}}
is an elaborate solution to combine several documents into one.
\end{itemize}
%
See also the CTAN topic \href{http://ctan.org/topic/subdocs}{\textsf{subdocs}}
for further related packages.
The present package differs from the above solutions in that
a document structure constructed with the conventional |\include| mechanism
just needs two extra commands at the top of every file
such that all constituent files can be compiled individually.

%%%%%%%%%%%%%%%%%%%%%%%%%%%%%%%%%%%%%%%%%%%%%%%%%%%%%%%%%%%%%%%%%%%%%%%%%%%%%%%%
%\subsection{Feature Suggestions}
%
%The following is a list of features which may be useful for future
%versions of this package:
%%
%\begin{itemize}
%\item
%\ldots
%\end{itemize}

%%%%%%%%%%%%%%%%%%%%%%%%%%%%%%%%%%%%%%%%%%%%%%%%%%%%%%%%%%%%%%%%%%%%%%%%%%%%%%%%
\subsection{Revision History}

%%%%%%%%%%%%%%%%%%%%%%%%%%%%%%%%%%%%%%%%
\paragraph{v2.0:} 2018/12/30

\begin{itemize}
\item
immediate forward processing
\item
added |\childdocby| mechanism
\item
manual restructured
\end{itemize}

%%%%%%%%%%%%%%%%%%%%%%%%%%%%%%%%%%%%%%%%
\paragraph{v1.6:} 2018/01/17

\begin{itemize}
\item
application for development of include files
\item
corrections to manual
\end{itemize}

%%%%%%%%%%%%%%%%%%%%%%%%%%%%%%%%%%%%%%%%
\paragraph{v1.5:} 2017/05/21

\begin{itemize}
\item
more complete structuring introduced
\item
|\childdocof| introduced
\item
|\childdoc| renamed to |\childdocmain|
\item
|\childredirect| renamed to |\childdocforward| and |\childdocforwardprefix|
and functionality expanded
\end{itemize}

%%%%%%%%%%%%%%%%%%%%%%%%%%%%%%%%%%%%%%%%
\paragraph{v1.0:} 2017/04/27

\begin{itemize}
\item
manual and install package
\item
first version published on CTAN
\end{itemize}

%%%%%%%%%%%%%%%%%%%%%%%%%%%%%%%%%%%%%%%%
\paragraph{v0.6:} 2017/04/26

\begin{itemize}
\item
redirection mechanism added
\end{itemize}

%%%%%%%%%%%%%%%%%%%%%%%%%%%%%%%%%%%%%%%%
\paragraph{v0.5:} 2017/04/26

\begin{itemize}
\item
functionality in definition file
\end{itemize}


%%%%%%%%%%%%%%%%%%%%%%%%%%%%%%%%%%%%%%%%%%%%%%%%%%%%%%%%%%%%%%%%%%%%%%%%%%%%%%%%
%%%%%%%%%%%%%%%%%%%%%%%%%%%%%%%%%%%%%%%%%%%%%%%%%%%%%%%%%%%%%%%%%%%%%%%%%%%%%%%%
%%%%%%%%%%%%%%%%%%%%%%%%%%%%%%%%%%%%%%%%%%%%%%%%%%%%%%%%%%%%%%%%%%%%%%%%%%%%%%%%
\appendix

\settowidth\MacroIndent{\rmfamily\scriptsize 000\ }

 \DocInput{childdoc.dtx}

\end{document}
%</driver>
% \fi
%
% %%%%%%%%%%%%%%%%%%%%%%%%%%%%%%%%%%%%%%%%%%%%%%%%%%%%%%%%%%%%%%%%%%%%%%%%%%%%%%
% %%%%%%%%%%%%%%%%%%%%%%%%%%%%%%%%%%%%%%%%%%%%%%%%%%%%%%%%%%%%%%%%%%%%%%%%%%%%%%
% \section{Sample}
%\iffalse
%<*samplemain>
%\fi
%
% The following presents a sample document
% with two chapters, two parts, a title page,
% a compile flag as well as three forwarding files to set the flag.
% It consists of eight |.tex| files:
% \begin{center}
% \begin{tabular}{ll}
% |cdocsamp.tex|&main file\\
% |cdocsch1.tex|&include file for chapter 1\\
% |cdocsch2.tex|&include file for chapter 2\\
% |cdocspt3.tex|&include file for part 3\\
% |cdocspt4.tex|&include file for part 4\\
% |cdocsdrf.tex|&forwarding file for main file in draft mode\\
% |cdocsfi1.tex|&forwarding file for final version of chapter 1\\
% |cdocsfi2.tex|&forwarding file for final version of chapter 2\\
% \end{tabular}
% \end{center}
% Each of the eight files can be compiled directly by the \LaTeX{} compiler.
%
% %%%%%%%%%%%%%%%%%%%%%%%%%%%%%%%%%%%%%%
% \paragraph{Main File.}
%
% The main file is called |cdocsamp.tex|.
%
% Load the \textsf{childdoc} definitions and
% declare the filename for the main document:
%    \begin{macrocode}
\input{childdoc.def}
\childdocmain{}
%    \end{macrocode}

% Optional override for |\version| flag:
%    \begin{macrocode}
%%\ifchilddoc\else\providecommand{\version}{draft}\fi
%    \end{macrocode}

% Define the default values for the |\version| flag
% (|final| for the main file and |draft| for childs):
%    \begin{macrocode}
\ifchilddoc
\providecommand{\version}{draft}
\else
\providecommand{\version}{final}
\fi
%    \end{macrocode}

% Load the standard document class:
%    \begin{macrocode}
\documentclass[12pt]{article}
%    \end{macrocode}

% Start the document body:
%    \begin{macrocode}
\begin{document}
%    \end{macrocode}

% Declare a title page.
% Print title, part of document being processed and version flag:
%    \begin{macrocode}
\addtocounter{page}{-1}
\begin{center}
{\LARGE\bfseries{}childdoc example\par}
\vspace{1cm}
\ifchilddoc
\ifchilddocmanual part\else chapter\fi:
`\childdocname' of `\childdocjob'\par
\else
main document: `\childdocjob'\par
\fi
version: \version\par
\end{center}
\newpage
%    \end{macrocode}

% Manually include selected file,
% otherwise process as usual:
%    \begin{macrocode}
\ifchilddocmanual
\section*{part `\childdocname'}
\input{\childdocname}
\else
%    \end{macrocode}

% Include the two chapters:
%    \begin{macrocode}
\include{cdocsch1}
\include{cdocsch2}
%    \end{macrocode}

% Include the two parts unless only chapters should be displayed:
%    \begin{macrocode}
\ifchilddoc\else
\section{part three}
\input{cdocspt3}
\section{part four}
\input{cdocspt4}
\fi
%    \end{macrocode}

% Process as usual until here:
%    \begin{macrocode}
\fi
%    \end{macrocode}

% End of document body:
%    \begin{macrocode}
\end{document}
%    \end{macrocode}
%\iffalse
%</samplemain>
%\fi
%
% %%%%%%%%%%%%%%%%%%%%%%%%%%%%%%%%%%%%%%
% \paragraph{Chapter Include Files.}
%
% The include files are called |cdocsch1.tex| and |cdocsch2.tex|.
%
%\iffalse
%<*samplechap1|samplechap2>
%\fi

% Optional override for |\version| flag:
%    \begin{macrocode}
%%\providecommand{\version}{final}
%    \end{macrocode}

% Include the main document:
%    \begin{macrocode}
\input{childdoc.def}
\childdocof{cdocsamp}
%    \end{macrocode}

%\iffalse
%</samplechap1|samplechap2>
%\fi
%
%\iffalse
%<*samplechap1>
%\fi
% Some text for chapter 1:
%    \begin{macrocode}
\section{one}
some text in chapter one
%    \end{macrocode}

%\iffalse
%</samplechap1>
%\fi
% Some text for chapter 2:
%\iffalse
%<*samplechap2>
%\fi
%    \begin{macrocode}
\section{two}
more text in chapter two
%    \end{macrocode}

%\iffalse
%</samplechap2>
%\fi
%
% %%%%%%%%%%%%%%%%%%%%%%%%%%%%%%%%%%%%%%
% \paragraph{Part Include Files.}
%
% The include files are called |cdocspt3.tex| and |cdocspt4.tex|.
%
%\iffalse
%<*samplepart3|samplepart4>
%\fi

% Optional override for |\version| flag:
%    \begin{macrocode}
%%\providecommand{\version}{final}
%    \end{macrocode}

% Include the main document:
%    \begin{macrocode}
\input{childdoc.def}
\childdocby{cdocsamp}
%    \end{macrocode}

%\iffalse
%</samplepart3|samplepart4>
%\fi
%
%\iffalse
%<*samplepart3>
%\fi
% Some text for part 3:
%    \begin{macrocode}
some text in part three
%    \end{macrocode}

%\iffalse
%</samplepart3>
%\fi
% Some text for part 4:
%\iffalse
%<*samplepart4>
%\fi
%    \begin{macrocode}
more text in part four
%    \end{macrocode}

%\iffalse
%</samplepart4>
%\fi
%
% %%%%%%%%%%%%%%%%%%%%%%%%%%%%%%%%%%%%%%
% \paragraph{Forwarding for a Complete Draft.}
%
% The following forwarding file |cdocsdrf.tex|
% compiles the main document in draft mode:
%\iffalse
%<*sampledraft>
%\fi
%    \begin{macrocode}
\def\version{draft}
\input{childdoc.def}
\childdocforward{cdocsamp}
%    \end{macrocode}

%\iffalse
%</sampledraft>
%\fi
%
% %%%%%%%%%%%%%%%%%%%%%%%%%%%%%%%%%%%%%%
% \paragraph{Forwarding for Final Version of the Chapters.}
%
% The following forwarding files |cdocsfn1.tex| and |cdocsfn2.tex|
% (with identical content)
% compile the final versions of the child documents
% |cdocsch1.tex| and |cdocsch2.tex|, respectively:
%\iffalse
%<*samplefinal>
%\fi
%    \begin{macrocode}
\def\version{final}
\input{childdoc.def}
\childdocforwardprefix[cdocsamp]{cdocsfn}{cdocsch}
%    \end{macrocode}

%\iffalse
%</samplefinal>
%\fi
%
% %%%%%%%%%%%%%%%%%%%%%%%%%%%%%%%%%%%%%%
% \paragraph{Command Line Processing.}
%
% The following three command lines generate the output files
% |cdocscld|, |cdocscl1| and |cdocscl2|
% which should be identical to
% |cdocsdrf|, |cdocsch1| and |cdocsfn2|, respectively:
% \begin{center}
% \begin{tabular}{l}
% |latex -jobname cdocscld \|\\
% |  "\def\version{draft}\input{childdoc.def}\childdocforward{cdocsamp}"|\\
% |latex -jobname cdocscl1 \|\\
% |  "\input{childdoc.def}\childdocforward[cdocsamp]{cdocsch1}"|\\
% |latex -jobname cdocscl2 \|\\
% |  "\def\version{final}\input{childdoc.def}\childdocforward{cdocsch2}"|
% \end{tabular}
% \end{center}
% Note that the trailing backslash on each first line
% merely continues the input to the second line
% (for convenient cut ant paste).
% Furthermore, the command |latex| can be replaced by any
% of its alternative versions such as |pdflatex|.
%
% %%%%%%%%%%%%%%%%%%%%%%%%%%%%%%%%%%%%%%%%%%%%%%%%%%%%%%%%%%%%%%%%%%%%%%%%%%%%%%
% %%%%%%%%%%%%%%%%%%%%%%%%%%%%%%%%%%%%%%%%%%%%%%%%%%%%%%%%%%%%%%%%%%%%%%%%%%%%%%
% \section{Implementation}
%\iffalse
%<*package>
%\fi
%
% This section describes the definitions file |childdoc.def|.

% The definitions cannot be loaded using |\usepackage| or |\RequirePackage|
% which has a mechanism to prevent loading a style file more than once.
% When loading the definitions by means of |\input|
% multiple instances have to be prevented manually:
%\iffalse
%This code needs to be before the `\ProvidesFile' directive
%which is defined at the beginning of this file.
%Therefore it is also placed there and commented out here.
%</package>
%<*discard>
%\fi
%    \begin{macrocode}
\ifdefined\childdocmain\endinput\fi
%    \end{macrocode}
%\iffalse
%</discard>
%<*package>
%\fi
%
% \macro{\ifchilddoc}
% \macro{\ifchilddocmanual}
% The conditional |\ifchilddoc| tells whether a
% child (true) or main (false) document is being compiled.
% The conditional |\ifchilddocmanual| tells whether
% the |\includeonly| mechanism is used (false) or
% the selection of child files must be performed manually (true).
% The definitions initialise to false:
%    \begin{macrocode}
\newif\ifchilddoc
\newif\ifchilddocmanual
%    \end{macrocode}

% \macro{\childdocname}
% \macro{\childdocjob}
% The macro |\childdocname| stores the name of the main document
% to be compiled. The macro |\childdocjob| stores the name of
% the document on which the \LaTeX{} compiler was originally invoked.
% The content of |\jobname| cannot be compared
% to filenames specified in the source due to different catcodes.
% The following code rescans |\jobname|, stores the result
% in |\childdocname| and saves a copy in |\childdocjob|:
%    \begin{macrocode}
\edef\childdocname{\scantokens\expandafter{\jobname\noexpand}}
\let\childdocjob\childdocname
%    \end{macrocode}

% \macro{\childdocdisable}
% The macro |\childdocdisable| prevents the main file
% from being processed more than once.
% At this stage, the main document command |\childdocmain|
% is assumed to be called once again where it should do nothing.
% Any subsequent call to it should prevent
% a secondary processing of the main document
% It overwrites the forwarding commands
% |\childdocof| and |\childdocforward|
% with empty macros to prevent further inclusions of the main document:
%    \begin{macrocode}
\newcommand{\childdocdisable}
{
  \renewcommand{\childdocmain}[1]{\renewcommand{\childdocmain}[1]{\endinput}}
  \renewcommand{\childdocof}[1]{}
  \renewcommand{\childdocby}[2][]{}
  \renewcommand{\childdocforward}[2][]{}
  \renewcommand{\childdocdisable}{}
}
%    \end{macrocode}

% \macro{\childdocmain}
% The macro |\childdocmain| is to be called at the top of the main file
% with nothing or the main filename (without extension) as argument.
% First, it breaks loops.
% If the argument is not empty and does not match |\childdocname|
% (which is set by the first inclusion of |childdoc.def|),
% |\ifchilddoc| is set to true, |\includeonly| is applied to the child file
% and |\jobname| is set to the main file
% (for proper handling of |.aux| files):
%    \begin{macrocode}
\newcommand{\childdocmain}[1]
{
  \childdocdisable\childdocmain{}
  \if?#1?\else
    \begingroup
      \def\childdoctmp{#1}
      \ifx\childdoctmp\childdocname
        \def\childdoctmp{}
      \else
        \def\childdoctmp
        {
          \childdoctrue
          \includeonly{\childdocname}
          \def\childdocjob{#1}
          \def\jobname{#1}
        }
      \fi
      \expandafter
    \endgroup
    \childdoctmp
  \fi
}
%    \end{macrocode}

% \macro{\childdocof}
% The command |\childdocof| redirects
% compilation to the main file |#1|.
%    \begin{macrocode}
\newcommand{\childdocof}[1]
{
  \childdocdisable
  \childdoctrue
  \includeonly{\childdocname}
  \def\jobname{#1}
  \def\childdocjob{#1}
  \input{#1}
}
%    \end{macrocode}

% \macro{\childdocby}
% The command |\childdocby| ....
%    \begin{macrocode}
\newcommand{\childdocby}[2][]
{
  \childdocdisable
  \childdoctrue
  \childdocmanualtrue
  \if?#1?\else
    \def\jobname{#2}
  \fi
  \def\childdocjob{#2}
  \input{#2}
  \endinput
}
%    \end{macrocode}

% \macro{\childdocforward}
% The command |\childdocforward| redirects
% compilation to the main file or
% (if the optional argument is given) a child file.
% Parameters are set as if the main file
% or a child file starting with |\childdocof| was compiled.
% Then compilation is handed over to the main file:
%    \begin{macrocode}
\newcommand{\childdocforward}[2][]
{
  \begingroup
    \if?#1?
      \def\childdoctmp
      {
        \def\childdocname{#2}
        \def\childdocjob{#2}
        \def\jobname{#2}
        \input{#2}
        \endinput
      }
    \else
      \def\childdoctmp
      {
        \childdocdisable
        \def\childdocname{#2}
        \childdoctrue
        \includeonly{#2}
        \def\childdocjob{#1}
        \def\jobname{#1}
        \input{#1}
        \endinput
      }
    \fi
    \expandafter
  \endgroup
  \childdoctmp
}
%    \end{macrocode}

% \macro{\childdocforwardprefix}
% The command |\childdocforwardprefix| redirects
% compilation to the main or a child file by means of a pattern.
% The prefix |#1| in the current filename is replaced by |#2|
% and the suffix of the current filename is kept
% (it is assumed that the filename does not contain the substring `|~~~|'
% which is used as a delimiter).
% Compilation is handed over to the new file by |\childdocforward|:
%    \begin{macrocode}
\newcommand{\childdocforwardprefix}[3][]
{
  \begingroup
    \def\childdocextract #2##1~~~{\def\childdoctmp{\childdocforward[#1]{#3##1}}}
    \expandafter\childdocextract\childdocname~~~
    \expandafter
  \endgroup
  \childdoctmp
}
%    \end{macrocode}

% \macro{\childdoc}
% The deprecated macro |\childdoc| is a legacy version of |\childdocmain|:
%    \begin{macrocode}
\newcommand{\childdoc}{\childdocmain}
%    \end{macrocode}

% \macro{\childdocredirect}
% The deprecated macro |\childdocredirect| is a legacy version
% of |\childdocforward| and |\childdocforwardprefix|:
%    \begin{macrocode}
\newcommand{\childdocredirect}[2][]
{
  \begingroup
    \if?#1?
      \def\childdoctmp{\childdocforward{#2}}
    \else
      \def\childdoctmp{\childdocforwardprefix{#1}{#2}}
    \fi
    \expandafter
  \endgroup
  \childdoctmp
}
%    \end{macrocode}

%\iffalse
%</package>
%\fi
%
\endinput
\childdocforward{cdocsamp}"|\\
% |latex -jobname cdocscl1 \|\\
% |  "% \iffalse
%
% childdoc.dtx Copyright (C) 2017-2018 Niklas Beisert
%
% This work may be distributed and/or modified under the
% conditions of the LaTeX Project Public License, either version 1.3
% of this license or (at your option) any later version.
% The latest version of this license is in
%   http://www.latex-project.org/lppl.txt
% and version 1.3 or later is part of all distributions of LaTeX
% version 2005/12/01 or later.
%
% This work has the LPPL maintenance status `maintained'.
%
% The Current Maintainer of this work is Niklas Beisert.
%
% This work consists of the files childdoc.dtx and childdoc.ins
% and the derived files childdoc.def and cdocsamp.tex with
% cdocsch1.tex, cdocsch2.tex, cdocsdrf.tex, cdocsfn1.tex, cdocsfn2.tex.
%
%<package>\ifdefined\childdocmain\endinput\fi
%<package>\ProvidesFile{childdoc.def}[2018/12/30 v2.0 child document driver]
%<samplemain>\ProvidesFile{cdocsamp.tex}[2018/12/30 v2.0 sample for childdoc]
%<*driver>
%\ProvidesFile{childdoc.drv}[2018/12/30 v2.0 childdoc reference manual file]
\PassOptionsToClass{10pt,a4paper}{article}
\documentclass{ltxdoc}

\usepackage[margin=35mm]{geometry}
\usepackage{hyperref}
\usepackage{hyperxmp}
\usepackage[usenames]{color}

\hypersetup{colorlinks=true}
\hypersetup{pdfstartview=FitH}
\hypersetup{pdfpagemode=UseNone}
\hypersetup{pdfsource={}}
\hypersetup{pdflang={en-UK}}
\hypersetup{pdfcopyright={Copyright 2017-2018 Niklas Beisert.
  This work may be distributed and/or modified under the
  conditions of the LaTeX Project Public License, either version 1.3
  of this license or (at your option) any later version.}}
\hypersetup{pdflicenseurl={http://www.latex-project.org/lppl.txt}}
\hypersetup{pdfcontactaddress={ETH Zurich, ITP, HIT K,
  Wolfgang-Pauli-Strasse 27}}
\hypersetup{pdfcontactpostcode={8093}}
\hypersetup{pdfcontactcity={Zurich}}
\hypersetup{pdfcontactcountry={Switzerland}}
\hypersetup{pdfcontactemail={nbeisert@itp.phys.ethz.ch}}
\hypersetup{pdfcontacturl={http://people.phys.ethz.ch/\xmptilde nbeisert/}}

\newcommand{\secref}[1]{\hyperref[#1]{section \ref*{#1}}}

\parskip1ex
\parindent0pt
\let\olditemize\itemize
\def\itemize{\olditemize\parskip0pt}

\begin{document}

\title{The \textsf{childdoc} Package}
\hypersetup{pdftitle={The childdoc Package}}
\author{Niklas Beisert\\[2ex]
  Institut f\"ur Theoretische Physik\\
  Eidgen\"ossische Technische Hochschule Z\"urich\\
  Wolfgang-Pauli-Strasse 27, 8093 Z\"urich, Switzerland\\[1ex]
  \href{mailto:nbeisert@itp.phys.ethz.ch}
  {\texttt{nbeisert@itp.phys.ethz.ch}}}
\hypersetup{pdfauthor={Niklas Beisert}}
\hypersetup{pdfsubject={Manual for the LaTeX2e Package childdoc}}
\date{30 December 2018, \textsf{v2.0}}
\maketitle

\begin{abstract}\noindent
\textsf{childdoc} is a \LaTeXe{} package
that enables the direct compilation
of document sections included by |\include|
to individual files.
\end{abstract}

\begingroup
\parskip0ex
\tableofcontents
\endgroup

%%%%%%%%%%%%%%%%%%%%%%%%%%%%%%%%%%%%%%%%%%%%%%%%%%%%%%%%%%%%%%%%%%%%%%%%%%%%%%%%
%%%%%%%%%%%%%%%%%%%%%%%%%%%%%%%%%%%%%%%%%%%%%%%%%%%%%%%%%%%%%%%%%%%%%%%%%%%%%%%%
\section{Introduction}

\LaTeX{} provides a mechanism to structure a large document (such as a book)
into a main file and several child files (containing the chapters)
using the |\include| command.
This mechanism is beneficial for documents
which span hundreds of pages in order to
make the source file(s) more manageable.
Moreover, compilation can be restricted to
selected child files by means of the |\includeonly| command.
The latter feature can be used to reduce the compilation time while editing
(this was significantly more useful in the earlier days of \LaTeX{})
or to generate a smaller document which is easier to navigate.
Another application of |\includeonly| is to generate
documents consisting of selected parts of the complete document.

However, there are a few drawbacks of the plain |\include| mechanism:
\begin{itemize}
\item
The child files cannot be compiled on their own,
they can only be compiled via the main file.
A naive editing environment
(such as a text editor with an option
to have the current file processed by \LaTeX)
may require one to switch to the main file before compiling;
attempting to compile the child file produces errors.
\item
The main file must be modified (each time)
to adjust the |\includeonly| command
to the present needs. This easily leaves the main file in a messy state.
\item
The generated document will always carry the filename
of the main document. This is inconvenient if
several child files are to be compiled and
to be kept for distribution.
\end{itemize}

The present package provides a simple interface
to make child files individually compilable by \LaTeX{}.
Compiling a child file then has the same effect as compiling
the main file with an |\includeonly| command
to select the appropriate child.
Moreover the generated document will carry the name of the child
rather than the main file.
This resolves all three above issues.

This feature is meant to make the editing of books,
thesis documents and lecture notes somewhat more convenient.
However, the package can also be used efficiently for
composing a series of documents (such as exercise sheets)
which are typically distributed individually.
It then assists the author in generating the individual documents
(potentially in different versions)
as well as a document containing the collected series.
Another application is in developing style files
or other kinds of included material
where compilation of the style file could redirect
to a sample or test file.

%%%%%%%%%%%%%%%%%%%%%%%%%%%%%%%%%%%%%%%%%%%%%%%%%%%%%%%%%%%%%%%%%%%%%%%%%%%%%%%%
%%%%%%%%%%%%%%%%%%%%%%%%%%%%%%%%%%%%%%%%%%%%%%%%%%%%%%%%%%%%%%%%%%%%%%%%%%%%%%%%
\section{Usage}

First of all, the package \textsf{childdoc} is \emph{not} a standard
\LaTeXe{} |.sty| style file! Therefore it needs to be invoked in
a non-standard way.

%%%%%%%%%%%%%%%%%%%%%%%%%%%%%%%%%%%%%%%%%%%%%%%%%%%%%%%%%%%%%%%%%%%%%%%%%%%%%%%%
\subsection{Included Files}
\label{sec:include}

%%%%%%%%%%%%%%%%%%%%%%%%%%%%%%%%%%%%%%%%
\DescribeMacro{\childdocmain}
To use the package, add the commands
\begin{center}
\begin{tabular}{l}
|\input{childdoc.def}|\\
|\childdocmain{}|\\
\end{tabular}
\end{center}
at the very top of the main \LaTeX{} file,
in particular \emph{before} the |\documentclass| statement!
The argument of |\childdocmain| should be left empty
(but it must be present).

%%%%%%%%%%%%%%%%%%%%%%%%%%%%%%%%%%%%%%%%
\DescribeMacro{\childdocof}
Furthermore, add the commands
\begin{center}
\begin{tabular}{l}
|\input{childdoc.def}|\\
|\childdocof{|\textit{main}|}|\\
\end{tabular}
\end{center}
at the top of every child file \textit{child}
which is included by |\include{|\textit{child}|}|
from within the main file
(or at least for those files to be compiled individually).
The argument \textit{main} must be the filename of the main file.

There are a couple of
considerations in setting up the main and child documents:

%%%%%%%%%%%%%%%%%%%%%%%%%%%%%%%%%%%%%%%%
\paragraph{Restrictions.}

Please note the following restrictions:
\begin{itemize}
\item
|\childdocmain| must be called with one argument \textit{main}
to ensure compatibility with earlier version of the package.
It must either be empty (|\childdocmain{}|)
or precisely match the filename of the main file in which it is specified.
See \secref{sec:detection} for further information.
\item
The filename \textit{main} must be specified without the |.tex| extension.
\item
The filename \textit{main} is case sensitive
(even in case-insensitive file systems)
due to internal string comparison.
\item
The argument \textit{main} should be fully expanded, it cannot be a macro.
\item
Subdirectories and special characters should be avoided in filenames.
\item
The command |\childdocmain{|\textit{main}|}| must be followed by a whitespace.
It should not be followed immediately by another command
or by a comment mark `|%|'.
This is because the \TeX{} parser reads the token immediately following
the argument of |\childdocmain| and puts it
at the beginning of every child section;
however, a white\-space is ignored.
\end{itemize}

%%%%%%%%%%%%%%%%%%%%%%%%%%%%%%%%%%%%%%%%
\paragraph{Content of Main File.}

It is advisable to place all content in the child files included by |\include|.
Any output contained in the main file will appear in all child documents
unless suppressed manually;
it cannot be suppressed automatically by the |\includeonly| directive
and thus should normally be avoided.
A method to include some content in the main file
by means of conditional processing is described in \secref{sec:conditional}.

%%%%%%%%%%%%%%%%%%%%%%%%%%%%%%%%%%%%%%%%
\paragraph{Page Numbering.}

When only a part of the document is compiled,
the appropriate numbering of pages
(as well as other status parameters)
is determined from the |.aux| files.
The latter contain information from previous passes.
However this information needs to propagate through
all intermediate child documents.
Therefore the page numbering in child documents may well
be inconsistent until the complete document is compiled at least once.

A useful (if unconventional) way to always ensure a consistent
page numbering is to restart the numbering in each child document
and denote the pages by `\textit{child}|.|\textit{page}'
where \textit{child} represents the chapter/section number of the child file.
This can be achieved by the command
|\numberwithin{page}{|\textit{child}|}|
of the \textsf{amsmath} package
where \textit{child} can be |chapter| or |section|
depending on the chosen structuring.
Alternatively, one can modify the macro |\thepage| appropriately
and reset the counter |page| at the start of each child file.

%%%%%%%%%%%%%%%%%%%%%%%%%%%%%%%%%%%%%%%%%%%%%%%%%%%%%%%%%%%%%%%%%%%%%%%%%%%%%%%%
\subsection{Conditional Processing}
\label{sec:conditional}

The package provides a mechanism to compile different versions
of a document. To customise the versions further some conditional processing
can come in handy to distinguish which version is being compiled.
The package provides two macros to describe the compilation context:

%%%%%%%%%%%%%%%%%%%%%%%%%%%%%%%%%%%%%%%%
\DescribeMacro{\ifchilddoc}
The conditional |\ifchilddoc| distinguishes between the compilation of
child documents and the main document:
%
\begin{center}
|\ifchilddoc |\textit{child-code}| |[|\||else |\textit{main-code}]| \||fi|
\end{center}

%%%%%%%%%%%%%%%%%%%%%%%%%%%%%%%%%%%%%%%%
\DescribeMacro{\childdocname}
\DescribeMacro{\childdocjob}
The macro |\childdocname| contains the filename (without extension)
of the main or child file being processed.
Note that |\childdocjob| will always contain the name of the main file.

%%%%%%%%%%%%%%%%%%%%%%%%%%%%%%%%%%%%%%%%
\paragraph{Title Page.}

Conditional processing can be used to include a title or banner page
in the main document when proper precautions are taken.
Importantly, the code in the main file should ensure that the page counter
(as well as other status parameters which are stored in the |.aux| files)
takes the same value after the conditional processing.
Otherwise the page numbers may take divergent values
depending on which part is compiled.

For example, a title page could be declared by:
%
\begin{center}
\begin{tabular}{l}
|\ifchilddoc\||else|\\
|\addtocounter{page}{-1}|\\
\textit{code for title page}\\
|\newpage|\\
|\||fi|
\end{tabular}
\end{center}
%
A banner page for the child documents can be generated by:
%
\begin{center}
\begin{tabular}{l}
|\ifchilddoc|\\
|\addtocounter{page}{-1}|\\
\textit{code for banner page}\\
|\newpage|\\
|\||fi|
\end{tabular}
\end{center}
%
Here one could write a message such as:
\begin{center}
|This is the part \childdocname{} of \childdocjob{}.|
\end{center}

%%%%%%%%%%%%%%%%%%%%%%%%%%%%%%%%%%%%%%%%%%%%%%%%%%%%%%%%%%%%%%%%%%%%%%%%%%%%%%%%
\subsection{Flags}
\label{sec:flags}

The package makes it easy to generate different versions
of the main or child documents.
To this end compilation flags can be defined
and assigned different default values.
They will be particularly useful in conjunction
with the forwarding mechanism described in \secref{sec:forward}.

For example, it may be useful to have a flag |\version|
which can be set to |draft| or |final|.
The document source will contain some conditional code
depending on the value of |\version|.
Suppose further, the flag should default to |final| for the main file
and to |draft| for child files
which is a natural assignment for editing the document.
This is achieved by placing the following code
in the preamble of the main document
(below the |\childdocmain| directive):
%
\begin{center}
\begin{tabular}{l}
|\ifchilddoc|\\
|\providecommand{\version}{draft}|\\
|\||else|\\
|\providecommand{\version}{final}|\\
|\||fi|
\end{tabular}
\end{center}
%
The definition by |\providecommand| makes sure
that previous definitions are not overwritten.
Further statements |\providecommand{\version}{...}|
can thus be added before the above code to override it.

For the main file, one might add a line
(between |\childdocmain| and the above block)
%
\begin{center}
|%\ifchilddoc\||else\providecommand{\version}{draft}\||fi|
\end{center}
%
which can be uncommented to produce a draft version.
Likewise one can add a line to the very top of a child file
(above the |\childdocof{|\textit{main}|}| directive)
%
\begin{center}
|%\providecommand{\version}{final}|
\end{center}
%
which can be uncommented to produce the final version of this child document.

%%%%%%%%%%%%%%%%%%%%%%%%%%%%%%%%%%%%%%%%%%%%%%%%%%%%%%%%%%%%%%%%%%%%%%%%%%%%%%%%
\subsection{Forwarding}
\label{sec:forward}

Different versions of the main or child documents
using compilation flags as described in \secref{sec:flags}
can be (permanently) stored in different files
for convenient compilation, viewing and distribution.
To this end, the package defines a command
to pass on compilation to a different file:

%%%%%%%%%%%%%%%%%%%%%%%%%%%%%%%%%%%%%%%%
\DescribeMacro{\childdocforward}
The command |\childdocforward| redirects processing to
another source file:
%
\begin{center}
\begin{tabular}{l}
|\input{childdoc.def}|\\
|\childdocforward[|\textit{main}|]{|\textit{dest}|}|\\
\end{tabular}
\end{center}
%
The argument \textit{dest} is the destination file
(without extension).
It should be the main file or one of the child files.
Note that further \textsf{childdoc} directives
such as |\childdocof| and |\childdocforward|
in the indicated file will be processed in this form.
The optional argument \textit{main}
passes on directly to the main file \textit{main}
while pretending to compile the child \textit{dest}.
This form behaves as if \textit{dest}
issues |\childdocof{|\textit{main}|}| right away,
and no further \textsf{childdoc} directives will be processed.

%%%%%%%%%%%%%%%%%%%%%%%%%%%%%%%%%%%%%%%%
\DescribeMacro{\...prefix}
In the alternative form |\childdocforwardprefix|,
%
\begin{center}
\begin{tabular}{l}
|\input{childdoc.def}|\\
|\childdocforwardprefix[|\textit{main}|]{|\textit{prefix}|}{|\textit{dest}|}|
\end{tabular}
\end{center}
%
the destination file is determined by a pattern
depending on the current file:
To make this work, the current file must be called
`{\textit{prefix}\hspace{0.2em}\textit{suffix}}'
with \textit{prefix} matching precisely the argument.
Processing is then passed on to the file
`{\textit{dest}\hspace{0.2em}\textit{suffix}}'.
Surely, the same effect is achieved by
directly specifying the
argument `{\textit{dest}\hspace{0.2em}\textit{suffix}}'
in the first form.
However, that requires to set up a different file
for each child. With the alternative form of the command
all these files can have exactly the same content
which simplifies setting them up and maintaining them.

For example, the following file |draft.tex|
with a compilation flag |\version| as described in \secref{sec:flags}
compiles the main document as a draft:
%
\begin{center}
\begin{tabular}{l}
|\def\version{draft}|\\
|\input{childdoc.def}|\\
|\childdocforward{|\textit{main}|}|
\end{tabular}
\end{center}
%
Likewise, the following files |final|\textit{nn}|.tex|
compile the final version of the child document
|child|\textit{nn}|.tex|:
%
\begin{center}
\begin{tabular}{l}
|\def\version{final}|\\
|\input{childdoc.def}|\\
|\childdocforwardprefix{final}{child}|
\end{tabular}
\end{center}
%

Note that when several versions of a main file and/or of each child file
are to be generated, it may be convenient to set up a |Makefile| or
shell script to automatise the process.

%%%%%%%%%%%%%%%%%%%%%%%%%%%%%%%%%%%%%%%%%%%%%%%%%%%%%%%%%%%%%%%%%%%%%%%%%%%%%%%%
\subsection{Command Line Processing}
\label{sec:commandline}

The effect of redirection files can also be achieved by invoking
the \LaTeX{} compiler with a more elaborate command line.
Most conveniently this should be done as part
of a shell script or a |Makefile|.

When using \textsf{childdoc} in the main file, the following
command lines effectively perform a redirection
(note that depending on the shell being used,
backslashes may have to be doubled: `|\|' $\to$ `|\\|'):
%
\begin{center}
|... -jobname "|\textit{target}|" |\\|"|[\textit{flags}]%
|\input{childdoc.def}\childdocforward[|\textit{main}|]{|\textit{dest}|}"|
\end{center}
%
Here \textit{target} is the name of the output file,
\textit{main} is the name of the main file
and \textit{dest} is the name of the main or child file to be processed
(all filenames without extensions).
The optional argument \textit{main} can be omitted
if \textit{main} matches \textit{dest}.
Optionally, compilation \textit{flags} can be defined via |\def| commands.
This command line makes the \TeX{} engine believe
it is compiling the file \textit{target}
whose content is specified as the latter parameter.
The provided code then forwards the processing to
\textit{main} or \textit{dest} as described in \secref{sec:forward}.

%%%%%%%%%%%%%%%%%%%%%%%%%%%%%%%%%%%%%%%%%%%%%%%%%%%%%%%%%%%%%%%%%%%%%%%%%%%%%%%%
\subsection{Include by Input}
\label{sec:input}

Including child documents by |\include| has some restrictions by design.
Most notably, the content of a child document always occupies
its own set of pages; pages cannot be shared between child documents.
Usually, this behaviour makes perfect sense
because each child document contain an essential part of the document.
However, in some situations it may be desirable to compose
a document from a collection of parts
without having mandatory page breaks between then.
For this case, the package
provides a mechanism to include parts
by |\input| which can also be processed individually.
However, by construction this mechanism
requires manual handling of the content to be output.

%%%%%%%%%%%%%%%%%%%%%%%%%%%%%%%%%%%%%%%%
\DescribeMacro{\ifchilddocmanual}
The main file should be prepared as usual, see \secref{sec:include}.
However, the document body must make a distinction
between processing of an individual part and of the main document, e.g.:
%
\begin{center}
\begin{tabular}{l}
|\ifchilddocmanual|\\
|\input{\childdocname}|\\
|\||else|\\
\textit{document body with }|\input{|\textit{part}|}|\\
|\||fi|
\end{tabular}
\end{center}
%
The conditional |\ifchilddocmanual| is true whenever
a part to be included by |\input| is being compiled,
and the name of the part is stored in |\childdocname|.

%%%%%%%%%%%%%%%%%%%%%%%%%%%%%%%%%%%%%%%%
\DescribeMacro{\childdocby}
Each part to be included by |\input| should start with:
%
\begin{center}
\begin{tabular}{l}
|\input{childdoc.def}|\\
|\childdocby{|\textit{main}|}|\\
\end{tabular}
\end{center}
%
The directive |\childdocby| is similar to |\childdocof|
described in \secref{sec:include},
but the subsequent selection of content must be done manually.
To that end, both |\ifchilddoc| and |\ifchilddocmanual|
will be true upon processing of a part,
and the name of the part is stored in |\childdocname|.
Note that |\jobname| will be set to the filename of the current part
so that each part receives an individual |.aux| file
that does not interfere with the |.aux| file(s) of the main document.
This behaviour can be altered by the alternative form
|\childdocby[*]{|\textit{main}|}| (with a non-empty optional argument)
which uses the |.aux| file of the main document
by setting |\jobname| to \textit{main}.

%%%%%%%%%%%%%%%%%%%%%%%%%%%%%%%%%%%%%%%%%%%%%%%%%%%%%%%%%%%%%%%%%%%%%%%%%%%%%%%%
\subsection{Driver Development}
\label{sec:driver}

The \textsf{childdoc} mechanism can also be use for the development
of definition files such as \LaTeX{} styles or classes.
This case differs from the above setup with multiple parts
included by |\include| in that no |\includeonly| should be invoked.
This can be achieved by starting the include file
(before |\ProvidesPackage|) with:
%
\begin{center}
\begin{tabular}{l}
|\input{childdoc.def}|\\
|\childdocforward{|\textit{main}|}|\\
\end{tabular}
\end{center}
%
or alternatively with:
%
\begin{center}
\begin{tabular}{l}
|\input{childdoc.def}|\\
|\childdocby{|\textit{main}|}|\\
\end{tabular}
\end{center}
%
Both forms have slightly different effects as described above.
The main file is prepared as usual, see \secref{sec:include}.

%%%%%%%%%%%%%%%%%%%%%%%%%%%%%%%%%%%%%%%%%%%%%%%%%%%%%%%%%%%%%%%%%%%%%%%%%%%%%%%%
\subsection{Legacy Detection}
\label{sec:detection}

The directive |\childdocmain| in the main file can detect
whether the complete document or merely a child is to be compiled
even without using the directive |\childdocof|.
This method is deprecated because it is less robust
and there is no compelling reason to use it;
it is merely provided for backward compatibility
and it may be removed in future versions.

If the detection mechanism is to be used,
it is mandatory to correctly specify
the filename of the main file as the argument of |\childdocmain|:
%
\begin{center}
\begin{tabular}{l}
|\input{childdoc.def}|\\
|\childdocmain{|\textit{main}|}|\\
\end{tabular}
\end{center}
%
If |\jobname| does not match the argument \textit{main} of |\childdocmain|,
it is assumed that |\jobname| points to the child file to be compiled.
When using |\childdocmain| with the main file specified as argument,
it suffices to start a child file
with just |\input{|\textit{main}|}|
without loading of the package and using |\childdocof|.
If instead all processing is done
with the appropriate \textsf{childdoc} directives,
the argument of \textit{main} of |\childdocmain| can be empty.

An alternative version of the command line processing described
in \secref{sec:commandline} using the detection mechanism reads:
%
\begin{center}
|... -jobname "|\textit{target}|" "|[\textit{flags}]%
[|\def\jobname{|\textit{dest}|}|]|\input{|\textit{main}|}"|
\end{center}

%%%%%%%%%%%%%%%%%%%%%%%%%%%%%%%%%%%%%%%%%%%%%%%%%%%%%%%%%%%%%%%%%%%%%%%%%%%%%%%%
\subsection{Manual Code}
\label{sec:manual}

In case one cannot be certain whether the definitions file |childdoc.def|
is installed on the target \TeX{} distribution
and one prefers not to ship it,
it is conceivable to paste a few relevant commands into the sources.

To that end, drop all statements |\input{childdoc.def}|
and perform the replacements as outlined below.
Instead of |\childdocmain{|\textit{main}|}| add the following code
to the top of the main file:
%
\begin{center}
\begin{tabular}{l}
|\||ifdefined\childdocname\endinput\||fi\newif\ifchilddoc|\\
|\edef\childdocname{\scantokens\expandafter{\jobname\noexpand}}|\\
|\def\childdocmain{|\textit{main}|}\||ifx\childdocmain\childdocname\||else|\\
|\childdoctrue\includeonly{\childdocname}\let\jobname\childdocmain\||fi|\\
\end{tabular}
\end{center}
%
Instead of |\childdocof{|\textit{main}|}| just include the main file
at the top of each child file:
%
\begin{center}
|\input{|\textit{main}|}|
\end{center}
%
A simple redirection |\childdocforward{|\textit{dest}|}| is achieved by:
%
\begin{center}
|\def\jobname{|\textit{dest}|}\input{\jobname}|
\end{center}
%
The redirection with prefix
|\childdocforwardprefix[|\textit{prefix}|]{|\textit{dest}|}|
is accomplished by:
%
\begin{center}
\begin{tabular}{l}
|{\edef\jobname{\scantokens\expandafter{\jobname\noexpand}}|\\
|\def\redirectjob |\textit{prefix}|#1~~~{\gdef\jobname{|\textit{dest}|#1}}|\\
|\expandafter\redirectjob\jobname~~~}\input{\jobname}|
\end{tabular}
\end{center}

In an alternative approach,
child documents can be compiled by a specific command line
without additional code or specific definitions:
%
\begin{center}
|... -jobname "|\textit{target}|" "|[\textit{flags}]%
|\includeonly{|\textit{dest}|}\input{|\textit{main}|}"|
\end{center}
%

%%%%%%%%%%%%%%%%%%%%%%%%%%%%%%%%%%%%%%%%%%%%%%%%%%%%%%%%%%%%%%%%%%%%%%%%%%%%%%%%
%%%%%%%%%%%%%%%%%%%%%%%%%%%%%%%%%%%%%%%%%%%%%%%%%%%%%%%%%%%%%%%%%%%%%%%%%%%%%%%%
\section{Information}

%%%%%%%%%%%%%%%%%%%%%%%%%%%%%%%%%%%%%%%%%%%%%%%%%%%%%%%%%%%%%%%%%%%%%%%%%%%%%%%%
\subsection{Copyright}

Copyright \copyright{} 2017--2018 Niklas Beisert

This work may be distributed and/or modified under the
conditions of the \LaTeX{} Project Public License, either version 1.3
of this license or (at your option) any later version.
The latest version of this license is in
  \url{http://www.latex-project.org/lppl.txt}
and version 1.3 or later is part of all distributions of \LaTeX{}
version 2005/12/01 or later.

This work has the LPPL maintenance status `maintained'.

The Current Maintainer of this work is Niklas Beisert.

This work consists of the files |README.txt|, |childdoc.ins| and |childdoc.dtx|
as well as the derived files |childdoc.def|, |cdocsamp.tex|
with |cdocsch1.tex|, |cdocsch2.tex|, |cdocspt3.tex|, |cdocspt4.tex|,
|cdocsdrf.tex|, |cdocsfn1.tex|, |cdocsfn2.tex|
as well as |childdoc.pdf|.

%%%%%%%%%%%%%%%%%%%%%%%%%%%%%%%%%%%%%%%%%%%%%%%%%%%%%%%%%%%%%%%%%%%%%%%%%%%%%%%%
\subsection{Files and Installation}

The package consists of the files:
%
\begin{center}
\begin{tabular}{ll}
    |README.txt|   & readme file \\
    |childdoc.ins| & installation file \\
    |childdoc.dtx| & source file \\
    |childdoc.def| & definition file \\
    |cdocsamp.tex| & sample main file \\
    |cdocsch1.tex| & sample include file \\
    |cdocsch2.tex| & sample include file \\
    |cdocspt3.tex| & sample part file \\
    |cdocspt4.tex| & sample part file \\
    |cdocsdrf.tex| & sample redirection file \\
    |cdocsfn1.tex| & sample redirection file \\
    |cdocsfn2.tex| & sample redirection file \\
    |childdoc.pdf| & manual
\end{tabular}
\end{center}
%
The distribution consists of the files
|README.txt|, |childdoc.ins| and |childdoc.dtx|.
%
\begin{itemize}
\item
Run (pdf)\LaTeX{} on |childdoc.dtx|
to compile the manual |childdoc.pdf| (this file).
\item
Run \LaTeX{} on |childdoc.ins| to create the definitions file |childdoc.def|
and the sample |cdocsamp.tex| with include files
|cdocsch1.tex|, |cdocsch2.tex|, |cdocspt3.tex|, |cdocspt4.tex|,
|cdocsdrf.tex|, |cdocsfn1.tex|, |cdocsfn2.tex|.
Then copy the file |childdoc.def| to an appropriate directory of your \LaTeX{}
distribution, e.g.\ \textit{texmf-root}|/tex/latex/childdoc|.
\end{itemize}

%%%%%%%%%%%%%%%%%%%%%%%%%%%%%%%%%%%%%%%%%%%%%%%%%%%%%%%%%%%%%%%%%%%%%%%%%%%%%%%%
\subsection{Related CTAN Packages}

There are several other packages which offer a similar functionality:
%
\begin{itemize}
\item
The packages
\href{http://ctan.org/pkg/docmute}{\textsf{docmute}},
\href{http://ctan.org/pkg/includex}{\textsf{includex}} and
\href{http://ctan.org/pkg/standalone}{\textsf{standalone}}
provide commands to include only the document body of
a child file thus allowing both files to be compiled individually.
\item
The packages \href{http://ctan.org/pkg/subdocs}{\textsf{subdocs}}
and \href{http://ctan.org/pkg/subfiles}{\textsf{subfiles}}
provide structures in which the main and child documents can be
encapsulated and allowing them to be compiled individually.
The inclusion mechanism is different from the conventional |\include|.
\item
The package \href{http://ctan.org/pkg/combine}{\textsf{combine}}
is an elaborate solution to combine several documents into one.
\end{itemize}
%
See also the CTAN topic \href{http://ctan.org/topic/subdocs}{\textsf{subdocs}}
for further related packages.
The present package differs from the above solutions in that
a document structure constructed with the conventional |\include| mechanism
just needs two extra commands at the top of every file
such that all constituent files can be compiled individually.

%%%%%%%%%%%%%%%%%%%%%%%%%%%%%%%%%%%%%%%%%%%%%%%%%%%%%%%%%%%%%%%%%%%%%%%%%%%%%%%%
%\subsection{Feature Suggestions}
%
%The following is a list of features which may be useful for future
%versions of this package:
%%
%\begin{itemize}
%\item
%\ldots
%\end{itemize}

%%%%%%%%%%%%%%%%%%%%%%%%%%%%%%%%%%%%%%%%%%%%%%%%%%%%%%%%%%%%%%%%%%%%%%%%%%%%%%%%
\subsection{Revision History}

%%%%%%%%%%%%%%%%%%%%%%%%%%%%%%%%%%%%%%%%
\paragraph{v2.0:} 2018/12/30

\begin{itemize}
\item
immediate forward processing
\item
added |\childdocby| mechanism
\item
manual restructured
\end{itemize}

%%%%%%%%%%%%%%%%%%%%%%%%%%%%%%%%%%%%%%%%
\paragraph{v1.6:} 2018/01/17

\begin{itemize}
\item
application for development of include files
\item
corrections to manual
\end{itemize}

%%%%%%%%%%%%%%%%%%%%%%%%%%%%%%%%%%%%%%%%
\paragraph{v1.5:} 2017/05/21

\begin{itemize}
\item
more complete structuring introduced
\item
|\childdocof| introduced
\item
|\childdoc| renamed to |\childdocmain|
\item
|\childredirect| renamed to |\childdocforward| and |\childdocforwardprefix|
and functionality expanded
\end{itemize}

%%%%%%%%%%%%%%%%%%%%%%%%%%%%%%%%%%%%%%%%
\paragraph{v1.0:} 2017/04/27

\begin{itemize}
\item
manual and install package
\item
first version published on CTAN
\end{itemize}

%%%%%%%%%%%%%%%%%%%%%%%%%%%%%%%%%%%%%%%%
\paragraph{v0.6:} 2017/04/26

\begin{itemize}
\item
redirection mechanism added
\end{itemize}

%%%%%%%%%%%%%%%%%%%%%%%%%%%%%%%%%%%%%%%%
\paragraph{v0.5:} 2017/04/26

\begin{itemize}
\item
functionality in definition file
\end{itemize}


%%%%%%%%%%%%%%%%%%%%%%%%%%%%%%%%%%%%%%%%%%%%%%%%%%%%%%%%%%%%%%%%%%%%%%%%%%%%%%%%
%%%%%%%%%%%%%%%%%%%%%%%%%%%%%%%%%%%%%%%%%%%%%%%%%%%%%%%%%%%%%%%%%%%%%%%%%%%%%%%%
%%%%%%%%%%%%%%%%%%%%%%%%%%%%%%%%%%%%%%%%%%%%%%%%%%%%%%%%%%%%%%%%%%%%%%%%%%%%%%%%
\appendix

\settowidth\MacroIndent{\rmfamily\scriptsize 000\ }

 \DocInput{childdoc.dtx}

\end{document}
%</driver>
% \fi
%
% %%%%%%%%%%%%%%%%%%%%%%%%%%%%%%%%%%%%%%%%%%%%%%%%%%%%%%%%%%%%%%%%%%%%%%%%%%%%%%
% %%%%%%%%%%%%%%%%%%%%%%%%%%%%%%%%%%%%%%%%%%%%%%%%%%%%%%%%%%%%%%%%%%%%%%%%%%%%%%
% \section{Sample}
%\iffalse
%<*samplemain>
%\fi
%
% The following presents a sample document
% with two chapters, two parts, a title page,
% a compile flag as well as three forwarding files to set the flag.
% It consists of eight |.tex| files:
% \begin{center}
% \begin{tabular}{ll}
% |cdocsamp.tex|&main file\\
% |cdocsch1.tex|&include file for chapter 1\\
% |cdocsch2.tex|&include file for chapter 2\\
% |cdocspt3.tex|&include file for part 3\\
% |cdocspt4.tex|&include file for part 4\\
% |cdocsdrf.tex|&forwarding file for main file in draft mode\\
% |cdocsfi1.tex|&forwarding file for final version of chapter 1\\
% |cdocsfi2.tex|&forwarding file for final version of chapter 2\\
% \end{tabular}
% \end{center}
% Each of the eight files can be compiled directly by the \LaTeX{} compiler.
%
% %%%%%%%%%%%%%%%%%%%%%%%%%%%%%%%%%%%%%%
% \paragraph{Main File.}
%
% The main file is called |cdocsamp.tex|.
%
% Load the \textsf{childdoc} definitions and
% declare the filename for the main document:
%    \begin{macrocode}
\input{childdoc.def}
\childdocmain{}
%    \end{macrocode}

% Optional override for |\version| flag:
%    \begin{macrocode}
%%\ifchilddoc\else\providecommand{\version}{draft}\fi
%    \end{macrocode}

% Define the default values for the |\version| flag
% (|final| for the main file and |draft| for childs):
%    \begin{macrocode}
\ifchilddoc
\providecommand{\version}{draft}
\else
\providecommand{\version}{final}
\fi
%    \end{macrocode}

% Load the standard document class:
%    \begin{macrocode}
\documentclass[12pt]{article}
%    \end{macrocode}

% Start the document body:
%    \begin{macrocode}
\begin{document}
%    \end{macrocode}

% Declare a title page.
% Print title, part of document being processed and version flag:
%    \begin{macrocode}
\addtocounter{page}{-1}
\begin{center}
{\LARGE\bfseries{}childdoc example\par}
\vspace{1cm}
\ifchilddoc
\ifchilddocmanual part\else chapter\fi:
`\childdocname' of `\childdocjob'\par
\else
main document: `\childdocjob'\par
\fi
version: \version\par
\end{center}
\newpage
%    \end{macrocode}

% Manually include selected file,
% otherwise process as usual:
%    \begin{macrocode}
\ifchilddocmanual
\section*{part `\childdocname'}
\input{\childdocname}
\else
%    \end{macrocode}

% Include the two chapters:
%    \begin{macrocode}
\include{cdocsch1}
\include{cdocsch2}
%    \end{macrocode}

% Include the two parts unless only chapters should be displayed:
%    \begin{macrocode}
\ifchilddoc\else
\section{part three}
\input{cdocspt3}
\section{part four}
\input{cdocspt4}
\fi
%    \end{macrocode}

% Process as usual until here:
%    \begin{macrocode}
\fi
%    \end{macrocode}

% End of document body:
%    \begin{macrocode}
\end{document}
%    \end{macrocode}
%\iffalse
%</samplemain>
%\fi
%
% %%%%%%%%%%%%%%%%%%%%%%%%%%%%%%%%%%%%%%
% \paragraph{Chapter Include Files.}
%
% The include files are called |cdocsch1.tex| and |cdocsch2.tex|.
%
%\iffalse
%<*samplechap1|samplechap2>
%\fi

% Optional override for |\version| flag:
%    \begin{macrocode}
%%\providecommand{\version}{final}
%    \end{macrocode}

% Include the main document:
%    \begin{macrocode}
\input{childdoc.def}
\childdocof{cdocsamp}
%    \end{macrocode}

%\iffalse
%</samplechap1|samplechap2>
%\fi
%
%\iffalse
%<*samplechap1>
%\fi
% Some text for chapter 1:
%    \begin{macrocode}
\section{one}
some text in chapter one
%    \end{macrocode}

%\iffalse
%</samplechap1>
%\fi
% Some text for chapter 2:
%\iffalse
%<*samplechap2>
%\fi
%    \begin{macrocode}
\section{two}
more text in chapter two
%    \end{macrocode}

%\iffalse
%</samplechap2>
%\fi
%
% %%%%%%%%%%%%%%%%%%%%%%%%%%%%%%%%%%%%%%
% \paragraph{Part Include Files.}
%
% The include files are called |cdocspt3.tex| and |cdocspt4.tex|.
%
%\iffalse
%<*samplepart3|samplepart4>
%\fi

% Optional override for |\version| flag:
%    \begin{macrocode}
%%\providecommand{\version}{final}
%    \end{macrocode}

% Include the main document:
%    \begin{macrocode}
\input{childdoc.def}
\childdocby{cdocsamp}
%    \end{macrocode}

%\iffalse
%</samplepart3|samplepart4>
%\fi
%
%\iffalse
%<*samplepart3>
%\fi
% Some text for part 3:
%    \begin{macrocode}
some text in part three
%    \end{macrocode}

%\iffalse
%</samplepart3>
%\fi
% Some text for part 4:
%\iffalse
%<*samplepart4>
%\fi
%    \begin{macrocode}
more text in part four
%    \end{macrocode}

%\iffalse
%</samplepart4>
%\fi
%
% %%%%%%%%%%%%%%%%%%%%%%%%%%%%%%%%%%%%%%
% \paragraph{Forwarding for a Complete Draft.}
%
% The following forwarding file |cdocsdrf.tex|
% compiles the main document in draft mode:
%\iffalse
%<*sampledraft>
%\fi
%    \begin{macrocode}
\def\version{draft}
\input{childdoc.def}
\childdocforward{cdocsamp}
%    \end{macrocode}

%\iffalse
%</sampledraft>
%\fi
%
% %%%%%%%%%%%%%%%%%%%%%%%%%%%%%%%%%%%%%%
% \paragraph{Forwarding for Final Version of the Chapters.}
%
% The following forwarding files |cdocsfn1.tex| and |cdocsfn2.tex|
% (with identical content)
% compile the final versions of the child documents
% |cdocsch1.tex| and |cdocsch2.tex|, respectively:
%\iffalse
%<*samplefinal>
%\fi
%    \begin{macrocode}
\def\version{final}
\input{childdoc.def}
\childdocforwardprefix[cdocsamp]{cdocsfn}{cdocsch}
%    \end{macrocode}

%\iffalse
%</samplefinal>
%\fi
%
% %%%%%%%%%%%%%%%%%%%%%%%%%%%%%%%%%%%%%%
% \paragraph{Command Line Processing.}
%
% The following three command lines generate the output files
% |cdocscld|, |cdocscl1| and |cdocscl2|
% which should be identical to
% |cdocsdrf|, |cdocsch1| and |cdocsfn2|, respectively:
% \begin{center}
% \begin{tabular}{l}
% |latex -jobname cdocscld \|\\
% |  "\def\version{draft}\input{childdoc.def}\childdocforward{cdocsamp}"|\\
% |latex -jobname cdocscl1 \|\\
% |  "\input{childdoc.def}\childdocforward[cdocsamp]{cdocsch1}"|\\
% |latex -jobname cdocscl2 \|\\
% |  "\def\version{final}\input{childdoc.def}\childdocforward{cdocsch2}"|
% \end{tabular}
% \end{center}
% Note that the trailing backslash on each first line
% merely continues the input to the second line
% (for convenient cut ant paste).
% Furthermore, the command |latex| can be replaced by any
% of its alternative versions such as |pdflatex|.
%
% %%%%%%%%%%%%%%%%%%%%%%%%%%%%%%%%%%%%%%%%%%%%%%%%%%%%%%%%%%%%%%%%%%%%%%%%%%%%%%
% %%%%%%%%%%%%%%%%%%%%%%%%%%%%%%%%%%%%%%%%%%%%%%%%%%%%%%%%%%%%%%%%%%%%%%%%%%%%%%
% \section{Implementation}
%\iffalse
%<*package>
%\fi
%
% This section describes the definitions file |childdoc.def|.

% The definitions cannot be loaded using |\usepackage| or |\RequirePackage|
% which has a mechanism to prevent loading a style file more than once.
% When loading the definitions by means of |\input|
% multiple instances have to be prevented manually:
%\iffalse
%This code needs to be before the `\ProvidesFile' directive
%which is defined at the beginning of this file.
%Therefore it is also placed there and commented out here.
%</package>
%<*discard>
%\fi
%    \begin{macrocode}
\ifdefined\childdocmain\endinput\fi
%    \end{macrocode}
%\iffalse
%</discard>
%<*package>
%\fi
%
% \macro{\ifchilddoc}
% \macro{\ifchilddocmanual}
% The conditional |\ifchilddoc| tells whether a
% child (true) or main (false) document is being compiled.
% The conditional |\ifchilddocmanual| tells whether
% the |\includeonly| mechanism is used (false) or
% the selection of child files must be performed manually (true).
% The definitions initialise to false:
%    \begin{macrocode}
\newif\ifchilddoc
\newif\ifchilddocmanual
%    \end{macrocode}

% \macro{\childdocname}
% \macro{\childdocjob}
% The macro |\childdocname| stores the name of the main document
% to be compiled. The macro |\childdocjob| stores the name of
% the document on which the \LaTeX{} compiler was originally invoked.
% The content of |\jobname| cannot be compared
% to filenames specified in the source due to different catcodes.
% The following code rescans |\jobname|, stores the result
% in |\childdocname| and saves a copy in |\childdocjob|:
%    \begin{macrocode}
\edef\childdocname{\scantokens\expandafter{\jobname\noexpand}}
\let\childdocjob\childdocname
%    \end{macrocode}

% \macro{\childdocdisable}
% The macro |\childdocdisable| prevents the main file
% from being processed more than once.
% At this stage, the main document command |\childdocmain|
% is assumed to be called once again where it should do nothing.
% Any subsequent call to it should prevent
% a secondary processing of the main document
% It overwrites the forwarding commands
% |\childdocof| and |\childdocforward|
% with empty macros to prevent further inclusions of the main document:
%    \begin{macrocode}
\newcommand{\childdocdisable}
{
  \renewcommand{\childdocmain}[1]{\renewcommand{\childdocmain}[1]{\endinput}}
  \renewcommand{\childdocof}[1]{}
  \renewcommand{\childdocby}[2][]{}
  \renewcommand{\childdocforward}[2][]{}
  \renewcommand{\childdocdisable}{}
}
%    \end{macrocode}

% \macro{\childdocmain}
% The macro |\childdocmain| is to be called at the top of the main file
% with nothing or the main filename (without extension) as argument.
% First, it breaks loops.
% If the argument is not empty and does not match |\childdocname|
% (which is set by the first inclusion of |childdoc.def|),
% |\ifchilddoc| is set to true, |\includeonly| is applied to the child file
% and |\jobname| is set to the main file
% (for proper handling of |.aux| files):
%    \begin{macrocode}
\newcommand{\childdocmain}[1]
{
  \childdocdisable\childdocmain{}
  \if?#1?\else
    \begingroup
      \def\childdoctmp{#1}
      \ifx\childdoctmp\childdocname
        \def\childdoctmp{}
      \else
        \def\childdoctmp
        {
          \childdoctrue
          \includeonly{\childdocname}
          \def\childdocjob{#1}
          \def\jobname{#1}
        }
      \fi
      \expandafter
    \endgroup
    \childdoctmp
  \fi
}
%    \end{macrocode}

% \macro{\childdocof}
% The command |\childdocof| redirects
% compilation to the main file |#1|.
%    \begin{macrocode}
\newcommand{\childdocof}[1]
{
  \childdocdisable
  \childdoctrue
  \includeonly{\childdocname}
  \def\jobname{#1}
  \def\childdocjob{#1}
  \input{#1}
}
%    \end{macrocode}

% \macro{\childdocby}
% The command |\childdocby| ....
%    \begin{macrocode}
\newcommand{\childdocby}[2][]
{
  \childdocdisable
  \childdoctrue
  \childdocmanualtrue
  \if?#1?\else
    \def\jobname{#2}
  \fi
  \def\childdocjob{#2}
  \input{#2}
  \endinput
}
%    \end{macrocode}

% \macro{\childdocforward}
% The command |\childdocforward| redirects
% compilation to the main file or
% (if the optional argument is given) a child file.
% Parameters are set as if the main file
% or a child file starting with |\childdocof| was compiled.
% Then compilation is handed over to the main file:
%    \begin{macrocode}
\newcommand{\childdocforward}[2][]
{
  \begingroup
    \if?#1?
      \def\childdoctmp
      {
        \def\childdocname{#2}
        \def\childdocjob{#2}
        \def\jobname{#2}
        \input{#2}
        \endinput
      }
    \else
      \def\childdoctmp
      {
        \childdocdisable
        \def\childdocname{#2}
        \childdoctrue
        \includeonly{#2}
        \def\childdocjob{#1}
        \def\jobname{#1}
        \input{#1}
        \endinput
      }
    \fi
    \expandafter
  \endgroup
  \childdoctmp
}
%    \end{macrocode}

% \macro{\childdocforwardprefix}
% The command |\childdocforwardprefix| redirects
% compilation to the main or a child file by means of a pattern.
% The prefix |#1| in the current filename is replaced by |#2|
% and the suffix of the current filename is kept
% (it is assumed that the filename does not contain the substring `|~~~|'
% which is used as a delimiter).
% Compilation is handed over to the new file by |\childdocforward|:
%    \begin{macrocode}
\newcommand{\childdocforwardprefix}[3][]
{
  \begingroup
    \def\childdocextract #2##1~~~{\def\childdoctmp{\childdocforward[#1]{#3##1}}}
    \expandafter\childdocextract\childdocname~~~
    \expandafter
  \endgroup
  \childdoctmp
}
%    \end{macrocode}

% \macro{\childdoc}
% The deprecated macro |\childdoc| is a legacy version of |\childdocmain|:
%    \begin{macrocode}
\newcommand{\childdoc}{\childdocmain}
%    \end{macrocode}

% \macro{\childdocredirect}
% The deprecated macro |\childdocredirect| is a legacy version
% of |\childdocforward| and |\childdocforwardprefix|:
%    \begin{macrocode}
\newcommand{\childdocredirect}[2][]
{
  \begingroup
    \if?#1?
      \def\childdoctmp{\childdocforward{#2}}
    \else
      \def\childdoctmp{\childdocforwardprefix{#1}{#2}}
    \fi
    \expandafter
  \endgroup
  \childdoctmp
}
%    \end{macrocode}

%\iffalse
%</package>
%\fi
%
\endinput
\childdocforward[cdocsamp]{cdocsch1}"|\\
% |latex -jobname cdocscl2 \|\\
% |  "\def\version{final}% \iffalse
%
% childdoc.dtx Copyright (C) 2017-2018 Niklas Beisert
%
% This work may be distributed and/or modified under the
% conditions of the LaTeX Project Public License, either version 1.3
% of this license or (at your option) any later version.
% The latest version of this license is in
%   http://www.latex-project.org/lppl.txt
% and version 1.3 or later is part of all distributions of LaTeX
% version 2005/12/01 or later.
%
% This work has the LPPL maintenance status `maintained'.
%
% The Current Maintainer of this work is Niklas Beisert.
%
% This work consists of the files childdoc.dtx and childdoc.ins
% and the derived files childdoc.def and cdocsamp.tex with
% cdocsch1.tex, cdocsch2.tex, cdocsdrf.tex, cdocsfn1.tex, cdocsfn2.tex.
%
%<package>\ifdefined\childdocmain\endinput\fi
%<package>\ProvidesFile{childdoc.def}[2018/12/30 v2.0 child document driver]
%<samplemain>\ProvidesFile{cdocsamp.tex}[2018/12/30 v2.0 sample for childdoc]
%<*driver>
%\ProvidesFile{childdoc.drv}[2018/12/30 v2.0 childdoc reference manual file]
\PassOptionsToClass{10pt,a4paper}{article}
\documentclass{ltxdoc}

\usepackage[margin=35mm]{geometry}
\usepackage{hyperref}
\usepackage{hyperxmp}
\usepackage[usenames]{color}

\hypersetup{colorlinks=true}
\hypersetup{pdfstartview=FitH}
\hypersetup{pdfpagemode=UseNone}
\hypersetup{pdfsource={}}
\hypersetup{pdflang={en-UK}}
\hypersetup{pdfcopyright={Copyright 2017-2018 Niklas Beisert.
  This work may be distributed and/or modified under the
  conditions of the LaTeX Project Public License, either version 1.3
  of this license or (at your option) any later version.}}
\hypersetup{pdflicenseurl={http://www.latex-project.org/lppl.txt}}
\hypersetup{pdfcontactaddress={ETH Zurich, ITP, HIT K,
  Wolfgang-Pauli-Strasse 27}}
\hypersetup{pdfcontactpostcode={8093}}
\hypersetup{pdfcontactcity={Zurich}}
\hypersetup{pdfcontactcountry={Switzerland}}
\hypersetup{pdfcontactemail={nbeisert@itp.phys.ethz.ch}}
\hypersetup{pdfcontacturl={http://people.phys.ethz.ch/\xmptilde nbeisert/}}

\newcommand{\secref}[1]{\hyperref[#1]{section \ref*{#1}}}

\parskip1ex
\parindent0pt
\let\olditemize\itemize
\def\itemize{\olditemize\parskip0pt}

\begin{document}

\title{The \textsf{childdoc} Package}
\hypersetup{pdftitle={The childdoc Package}}
\author{Niklas Beisert\\[2ex]
  Institut f\"ur Theoretische Physik\\
  Eidgen\"ossische Technische Hochschule Z\"urich\\
  Wolfgang-Pauli-Strasse 27, 8093 Z\"urich, Switzerland\\[1ex]
  \href{mailto:nbeisert@itp.phys.ethz.ch}
  {\texttt{nbeisert@itp.phys.ethz.ch}}}
\hypersetup{pdfauthor={Niklas Beisert}}
\hypersetup{pdfsubject={Manual for the LaTeX2e Package childdoc}}
\date{30 December 2018, \textsf{v2.0}}
\maketitle

\begin{abstract}\noindent
\textsf{childdoc} is a \LaTeXe{} package
that enables the direct compilation
of document sections included by |\include|
to individual files.
\end{abstract}

\begingroup
\parskip0ex
\tableofcontents
\endgroup

%%%%%%%%%%%%%%%%%%%%%%%%%%%%%%%%%%%%%%%%%%%%%%%%%%%%%%%%%%%%%%%%%%%%%%%%%%%%%%%%
%%%%%%%%%%%%%%%%%%%%%%%%%%%%%%%%%%%%%%%%%%%%%%%%%%%%%%%%%%%%%%%%%%%%%%%%%%%%%%%%
\section{Introduction}

\LaTeX{} provides a mechanism to structure a large document (such as a book)
into a main file and several child files (containing the chapters)
using the |\include| command.
This mechanism is beneficial for documents
which span hundreds of pages in order to
make the source file(s) more manageable.
Moreover, compilation can be restricted to
selected child files by means of the |\includeonly| command.
The latter feature can be used to reduce the compilation time while editing
(this was significantly more useful in the earlier days of \LaTeX{})
or to generate a smaller document which is easier to navigate.
Another application of |\includeonly| is to generate
documents consisting of selected parts of the complete document.

However, there are a few drawbacks of the plain |\include| mechanism:
\begin{itemize}
\item
The child files cannot be compiled on their own,
they can only be compiled via the main file.
A naive editing environment
(such as a text editor with an option
to have the current file processed by \LaTeX)
may require one to switch to the main file before compiling;
attempting to compile the child file produces errors.
\item
The main file must be modified (each time)
to adjust the |\includeonly| command
to the present needs. This easily leaves the main file in a messy state.
\item
The generated document will always carry the filename
of the main document. This is inconvenient if
several child files are to be compiled and
to be kept for distribution.
\end{itemize}

The present package provides a simple interface
to make child files individually compilable by \LaTeX{}.
Compiling a child file then has the same effect as compiling
the main file with an |\includeonly| command
to select the appropriate child.
Moreover the generated document will carry the name of the child
rather than the main file.
This resolves all three above issues.

This feature is meant to make the editing of books,
thesis documents and lecture notes somewhat more convenient.
However, the package can also be used efficiently for
composing a series of documents (such as exercise sheets)
which are typically distributed individually.
It then assists the author in generating the individual documents
(potentially in different versions)
as well as a document containing the collected series.
Another application is in developing style files
or other kinds of included material
where compilation of the style file could redirect
to a sample or test file.

%%%%%%%%%%%%%%%%%%%%%%%%%%%%%%%%%%%%%%%%%%%%%%%%%%%%%%%%%%%%%%%%%%%%%%%%%%%%%%%%
%%%%%%%%%%%%%%%%%%%%%%%%%%%%%%%%%%%%%%%%%%%%%%%%%%%%%%%%%%%%%%%%%%%%%%%%%%%%%%%%
\section{Usage}

First of all, the package \textsf{childdoc} is \emph{not} a standard
\LaTeXe{} |.sty| style file! Therefore it needs to be invoked in
a non-standard way.

%%%%%%%%%%%%%%%%%%%%%%%%%%%%%%%%%%%%%%%%%%%%%%%%%%%%%%%%%%%%%%%%%%%%%%%%%%%%%%%%
\subsection{Included Files}
\label{sec:include}

%%%%%%%%%%%%%%%%%%%%%%%%%%%%%%%%%%%%%%%%
\DescribeMacro{\childdocmain}
To use the package, add the commands
\begin{center}
\begin{tabular}{l}
|\input{childdoc.def}|\\
|\childdocmain{}|\\
\end{tabular}
\end{center}
at the very top of the main \LaTeX{} file,
in particular \emph{before} the |\documentclass| statement!
The argument of |\childdocmain| should be left empty
(but it must be present).

%%%%%%%%%%%%%%%%%%%%%%%%%%%%%%%%%%%%%%%%
\DescribeMacro{\childdocof}
Furthermore, add the commands
\begin{center}
\begin{tabular}{l}
|\input{childdoc.def}|\\
|\childdocof{|\textit{main}|}|\\
\end{tabular}
\end{center}
at the top of every child file \textit{child}
which is included by |\include{|\textit{child}|}|
from within the main file
(or at least for those files to be compiled individually).
The argument \textit{main} must be the filename of the main file.

There are a couple of
considerations in setting up the main and child documents:

%%%%%%%%%%%%%%%%%%%%%%%%%%%%%%%%%%%%%%%%
\paragraph{Restrictions.}

Please note the following restrictions:
\begin{itemize}
\item
|\childdocmain| must be called with one argument \textit{main}
to ensure compatibility with earlier version of the package.
It must either be empty (|\childdocmain{}|)
or precisely match the filename of the main file in which it is specified.
See \secref{sec:detection} for further information.
\item
The filename \textit{main} must be specified without the |.tex| extension.
\item
The filename \textit{main} is case sensitive
(even in case-insensitive file systems)
due to internal string comparison.
\item
The argument \textit{main} should be fully expanded, it cannot be a macro.
\item
Subdirectories and special characters should be avoided in filenames.
\item
The command |\childdocmain{|\textit{main}|}| must be followed by a whitespace.
It should not be followed immediately by another command
or by a comment mark `|%|'.
This is because the \TeX{} parser reads the token immediately following
the argument of |\childdocmain| and puts it
at the beginning of every child section;
however, a white\-space is ignored.
\end{itemize}

%%%%%%%%%%%%%%%%%%%%%%%%%%%%%%%%%%%%%%%%
\paragraph{Content of Main File.}

It is advisable to place all content in the child files included by |\include|.
Any output contained in the main file will appear in all child documents
unless suppressed manually;
it cannot be suppressed automatically by the |\includeonly| directive
and thus should normally be avoided.
A method to include some content in the main file
by means of conditional processing is described in \secref{sec:conditional}.

%%%%%%%%%%%%%%%%%%%%%%%%%%%%%%%%%%%%%%%%
\paragraph{Page Numbering.}

When only a part of the document is compiled,
the appropriate numbering of pages
(as well as other status parameters)
is determined from the |.aux| files.
The latter contain information from previous passes.
However this information needs to propagate through
all intermediate child documents.
Therefore the page numbering in child documents may well
be inconsistent until the complete document is compiled at least once.

A useful (if unconventional) way to always ensure a consistent
page numbering is to restart the numbering in each child document
and denote the pages by `\textit{child}|.|\textit{page}'
where \textit{child} represents the chapter/section number of the child file.
This can be achieved by the command
|\numberwithin{page}{|\textit{child}|}|
of the \textsf{amsmath} package
where \textit{child} can be |chapter| or |section|
depending on the chosen structuring.
Alternatively, one can modify the macro |\thepage| appropriately
and reset the counter |page| at the start of each child file.

%%%%%%%%%%%%%%%%%%%%%%%%%%%%%%%%%%%%%%%%%%%%%%%%%%%%%%%%%%%%%%%%%%%%%%%%%%%%%%%%
\subsection{Conditional Processing}
\label{sec:conditional}

The package provides a mechanism to compile different versions
of a document. To customise the versions further some conditional processing
can come in handy to distinguish which version is being compiled.
The package provides two macros to describe the compilation context:

%%%%%%%%%%%%%%%%%%%%%%%%%%%%%%%%%%%%%%%%
\DescribeMacro{\ifchilddoc}
The conditional |\ifchilddoc| distinguishes between the compilation of
child documents and the main document:
%
\begin{center}
|\ifchilddoc |\textit{child-code}| |[|\||else |\textit{main-code}]| \||fi|
\end{center}

%%%%%%%%%%%%%%%%%%%%%%%%%%%%%%%%%%%%%%%%
\DescribeMacro{\childdocname}
\DescribeMacro{\childdocjob}
The macro |\childdocname| contains the filename (without extension)
of the main or child file being processed.
Note that |\childdocjob| will always contain the name of the main file.

%%%%%%%%%%%%%%%%%%%%%%%%%%%%%%%%%%%%%%%%
\paragraph{Title Page.}

Conditional processing can be used to include a title or banner page
in the main document when proper precautions are taken.
Importantly, the code in the main file should ensure that the page counter
(as well as other status parameters which are stored in the |.aux| files)
takes the same value after the conditional processing.
Otherwise the page numbers may take divergent values
depending on which part is compiled.

For example, a title page could be declared by:
%
\begin{center}
\begin{tabular}{l}
|\ifchilddoc\||else|\\
|\addtocounter{page}{-1}|\\
\textit{code for title page}\\
|\newpage|\\
|\||fi|
\end{tabular}
\end{center}
%
A banner page for the child documents can be generated by:
%
\begin{center}
\begin{tabular}{l}
|\ifchilddoc|\\
|\addtocounter{page}{-1}|\\
\textit{code for banner page}\\
|\newpage|\\
|\||fi|
\end{tabular}
\end{center}
%
Here one could write a message such as:
\begin{center}
|This is the part \childdocname{} of \childdocjob{}.|
\end{center}

%%%%%%%%%%%%%%%%%%%%%%%%%%%%%%%%%%%%%%%%%%%%%%%%%%%%%%%%%%%%%%%%%%%%%%%%%%%%%%%%
\subsection{Flags}
\label{sec:flags}

The package makes it easy to generate different versions
of the main or child documents.
To this end compilation flags can be defined
and assigned different default values.
They will be particularly useful in conjunction
with the forwarding mechanism described in \secref{sec:forward}.

For example, it may be useful to have a flag |\version|
which can be set to |draft| or |final|.
The document source will contain some conditional code
depending on the value of |\version|.
Suppose further, the flag should default to |final| for the main file
and to |draft| for child files
which is a natural assignment for editing the document.
This is achieved by placing the following code
in the preamble of the main document
(below the |\childdocmain| directive):
%
\begin{center}
\begin{tabular}{l}
|\ifchilddoc|\\
|\providecommand{\version}{draft}|\\
|\||else|\\
|\providecommand{\version}{final}|\\
|\||fi|
\end{tabular}
\end{center}
%
The definition by |\providecommand| makes sure
that previous definitions are not overwritten.
Further statements |\providecommand{\version}{...}|
can thus be added before the above code to override it.

For the main file, one might add a line
(between |\childdocmain| and the above block)
%
\begin{center}
|%\ifchilddoc\||else\providecommand{\version}{draft}\||fi|
\end{center}
%
which can be uncommented to produce a draft version.
Likewise one can add a line to the very top of a child file
(above the |\childdocof{|\textit{main}|}| directive)
%
\begin{center}
|%\providecommand{\version}{final}|
\end{center}
%
which can be uncommented to produce the final version of this child document.

%%%%%%%%%%%%%%%%%%%%%%%%%%%%%%%%%%%%%%%%%%%%%%%%%%%%%%%%%%%%%%%%%%%%%%%%%%%%%%%%
\subsection{Forwarding}
\label{sec:forward}

Different versions of the main or child documents
using compilation flags as described in \secref{sec:flags}
can be (permanently) stored in different files
for convenient compilation, viewing and distribution.
To this end, the package defines a command
to pass on compilation to a different file:

%%%%%%%%%%%%%%%%%%%%%%%%%%%%%%%%%%%%%%%%
\DescribeMacro{\childdocforward}
The command |\childdocforward| redirects processing to
another source file:
%
\begin{center}
\begin{tabular}{l}
|\input{childdoc.def}|\\
|\childdocforward[|\textit{main}|]{|\textit{dest}|}|\\
\end{tabular}
\end{center}
%
The argument \textit{dest} is the destination file
(without extension).
It should be the main file or one of the child files.
Note that further \textsf{childdoc} directives
such as |\childdocof| and |\childdocforward|
in the indicated file will be processed in this form.
The optional argument \textit{main}
passes on directly to the main file \textit{main}
while pretending to compile the child \textit{dest}.
This form behaves as if \textit{dest}
issues |\childdocof{|\textit{main}|}| right away,
and no further \textsf{childdoc} directives will be processed.

%%%%%%%%%%%%%%%%%%%%%%%%%%%%%%%%%%%%%%%%
\DescribeMacro{\...prefix}
In the alternative form |\childdocforwardprefix|,
%
\begin{center}
\begin{tabular}{l}
|\input{childdoc.def}|\\
|\childdocforwardprefix[|\textit{main}|]{|\textit{prefix}|}{|\textit{dest}|}|
\end{tabular}
\end{center}
%
the destination file is determined by a pattern
depending on the current file:
To make this work, the current file must be called
`{\textit{prefix}\hspace{0.2em}\textit{suffix}}'
with \textit{prefix} matching precisely the argument.
Processing is then passed on to the file
`{\textit{dest}\hspace{0.2em}\textit{suffix}}'.
Surely, the same effect is achieved by
directly specifying the
argument `{\textit{dest}\hspace{0.2em}\textit{suffix}}'
in the first form.
However, that requires to set up a different file
for each child. With the alternative form of the command
all these files can have exactly the same content
which simplifies setting them up and maintaining them.

For example, the following file |draft.tex|
with a compilation flag |\version| as described in \secref{sec:flags}
compiles the main document as a draft:
%
\begin{center}
\begin{tabular}{l}
|\def\version{draft}|\\
|\input{childdoc.def}|\\
|\childdocforward{|\textit{main}|}|
\end{tabular}
\end{center}
%
Likewise, the following files |final|\textit{nn}|.tex|
compile the final version of the child document
|child|\textit{nn}|.tex|:
%
\begin{center}
\begin{tabular}{l}
|\def\version{final}|\\
|\input{childdoc.def}|\\
|\childdocforwardprefix{final}{child}|
\end{tabular}
\end{center}
%

Note that when several versions of a main file and/or of each child file
are to be generated, it may be convenient to set up a |Makefile| or
shell script to automatise the process.

%%%%%%%%%%%%%%%%%%%%%%%%%%%%%%%%%%%%%%%%%%%%%%%%%%%%%%%%%%%%%%%%%%%%%%%%%%%%%%%%
\subsection{Command Line Processing}
\label{sec:commandline}

The effect of redirection files can also be achieved by invoking
the \LaTeX{} compiler with a more elaborate command line.
Most conveniently this should be done as part
of a shell script or a |Makefile|.

When using \textsf{childdoc} in the main file, the following
command lines effectively perform a redirection
(note that depending on the shell being used,
backslashes may have to be doubled: `|\|' $\to$ `|\\|'):
%
\begin{center}
|... -jobname "|\textit{target}|" |\\|"|[\textit{flags}]%
|\input{childdoc.def}\childdocforward[|\textit{main}|]{|\textit{dest}|}"|
\end{center}
%
Here \textit{target} is the name of the output file,
\textit{main} is the name of the main file
and \textit{dest} is the name of the main or child file to be processed
(all filenames without extensions).
The optional argument \textit{main} can be omitted
if \textit{main} matches \textit{dest}.
Optionally, compilation \textit{flags} can be defined via |\def| commands.
This command line makes the \TeX{} engine believe
it is compiling the file \textit{target}
whose content is specified as the latter parameter.
The provided code then forwards the processing to
\textit{main} or \textit{dest} as described in \secref{sec:forward}.

%%%%%%%%%%%%%%%%%%%%%%%%%%%%%%%%%%%%%%%%%%%%%%%%%%%%%%%%%%%%%%%%%%%%%%%%%%%%%%%%
\subsection{Include by Input}
\label{sec:input}

Including child documents by |\include| has some restrictions by design.
Most notably, the content of a child document always occupies
its own set of pages; pages cannot be shared between child documents.
Usually, this behaviour makes perfect sense
because each child document contain an essential part of the document.
However, in some situations it may be desirable to compose
a document from a collection of parts
without having mandatory page breaks between then.
For this case, the package
provides a mechanism to include parts
by |\input| which can also be processed individually.
However, by construction this mechanism
requires manual handling of the content to be output.

%%%%%%%%%%%%%%%%%%%%%%%%%%%%%%%%%%%%%%%%
\DescribeMacro{\ifchilddocmanual}
The main file should be prepared as usual, see \secref{sec:include}.
However, the document body must make a distinction
between processing of an individual part and of the main document, e.g.:
%
\begin{center}
\begin{tabular}{l}
|\ifchilddocmanual|\\
|\input{\childdocname}|\\
|\||else|\\
\textit{document body with }|\input{|\textit{part}|}|\\
|\||fi|
\end{tabular}
\end{center}
%
The conditional |\ifchilddocmanual| is true whenever
a part to be included by |\input| is being compiled,
and the name of the part is stored in |\childdocname|.

%%%%%%%%%%%%%%%%%%%%%%%%%%%%%%%%%%%%%%%%
\DescribeMacro{\childdocby}
Each part to be included by |\input| should start with:
%
\begin{center}
\begin{tabular}{l}
|\input{childdoc.def}|\\
|\childdocby{|\textit{main}|}|\\
\end{tabular}
\end{center}
%
The directive |\childdocby| is similar to |\childdocof|
described in \secref{sec:include},
but the subsequent selection of content must be done manually.
To that end, both |\ifchilddoc| and |\ifchilddocmanual|
will be true upon processing of a part,
and the name of the part is stored in |\childdocname|.
Note that |\jobname| will be set to the filename of the current part
so that each part receives an individual |.aux| file
that does not interfere with the |.aux| file(s) of the main document.
This behaviour can be altered by the alternative form
|\childdocby[*]{|\textit{main}|}| (with a non-empty optional argument)
which uses the |.aux| file of the main document
by setting |\jobname| to \textit{main}.

%%%%%%%%%%%%%%%%%%%%%%%%%%%%%%%%%%%%%%%%%%%%%%%%%%%%%%%%%%%%%%%%%%%%%%%%%%%%%%%%
\subsection{Driver Development}
\label{sec:driver}

The \textsf{childdoc} mechanism can also be use for the development
of definition files such as \LaTeX{} styles or classes.
This case differs from the above setup with multiple parts
included by |\include| in that no |\includeonly| should be invoked.
This can be achieved by starting the include file
(before |\ProvidesPackage|) with:
%
\begin{center}
\begin{tabular}{l}
|\input{childdoc.def}|\\
|\childdocforward{|\textit{main}|}|\\
\end{tabular}
\end{center}
%
or alternatively with:
%
\begin{center}
\begin{tabular}{l}
|\input{childdoc.def}|\\
|\childdocby{|\textit{main}|}|\\
\end{tabular}
\end{center}
%
Both forms have slightly different effects as described above.
The main file is prepared as usual, see \secref{sec:include}.

%%%%%%%%%%%%%%%%%%%%%%%%%%%%%%%%%%%%%%%%%%%%%%%%%%%%%%%%%%%%%%%%%%%%%%%%%%%%%%%%
\subsection{Legacy Detection}
\label{sec:detection}

The directive |\childdocmain| in the main file can detect
whether the complete document or merely a child is to be compiled
even without using the directive |\childdocof|.
This method is deprecated because it is less robust
and there is no compelling reason to use it;
it is merely provided for backward compatibility
and it may be removed in future versions.

If the detection mechanism is to be used,
it is mandatory to correctly specify
the filename of the main file as the argument of |\childdocmain|:
%
\begin{center}
\begin{tabular}{l}
|\input{childdoc.def}|\\
|\childdocmain{|\textit{main}|}|\\
\end{tabular}
\end{center}
%
If |\jobname| does not match the argument \textit{main} of |\childdocmain|,
it is assumed that |\jobname| points to the child file to be compiled.
When using |\childdocmain| with the main file specified as argument,
it suffices to start a child file
with just |\input{|\textit{main}|}|
without loading of the package and using |\childdocof|.
If instead all processing is done
with the appropriate \textsf{childdoc} directives,
the argument of \textit{main} of |\childdocmain| can be empty.

An alternative version of the command line processing described
in \secref{sec:commandline} using the detection mechanism reads:
%
\begin{center}
|... -jobname "|\textit{target}|" "|[\textit{flags}]%
[|\def\jobname{|\textit{dest}|}|]|\input{|\textit{main}|}"|
\end{center}

%%%%%%%%%%%%%%%%%%%%%%%%%%%%%%%%%%%%%%%%%%%%%%%%%%%%%%%%%%%%%%%%%%%%%%%%%%%%%%%%
\subsection{Manual Code}
\label{sec:manual}

In case one cannot be certain whether the definitions file |childdoc.def|
is installed on the target \TeX{} distribution
and one prefers not to ship it,
it is conceivable to paste a few relevant commands into the sources.

To that end, drop all statements |\input{childdoc.def}|
and perform the replacements as outlined below.
Instead of |\childdocmain{|\textit{main}|}| add the following code
to the top of the main file:
%
\begin{center}
\begin{tabular}{l}
|\||ifdefined\childdocname\endinput\||fi\newif\ifchilddoc|\\
|\edef\childdocname{\scantokens\expandafter{\jobname\noexpand}}|\\
|\def\childdocmain{|\textit{main}|}\||ifx\childdocmain\childdocname\||else|\\
|\childdoctrue\includeonly{\childdocname}\let\jobname\childdocmain\||fi|\\
\end{tabular}
\end{center}
%
Instead of |\childdocof{|\textit{main}|}| just include the main file
at the top of each child file:
%
\begin{center}
|\input{|\textit{main}|}|
\end{center}
%
A simple redirection |\childdocforward{|\textit{dest}|}| is achieved by:
%
\begin{center}
|\def\jobname{|\textit{dest}|}\input{\jobname}|
\end{center}
%
The redirection with prefix
|\childdocforwardprefix[|\textit{prefix}|]{|\textit{dest}|}|
is accomplished by:
%
\begin{center}
\begin{tabular}{l}
|{\edef\jobname{\scantokens\expandafter{\jobname\noexpand}}|\\
|\def\redirectjob |\textit{prefix}|#1~~~{\gdef\jobname{|\textit{dest}|#1}}|\\
|\expandafter\redirectjob\jobname~~~}\input{\jobname}|
\end{tabular}
\end{center}

In an alternative approach,
child documents can be compiled by a specific command line
without additional code or specific definitions:
%
\begin{center}
|... -jobname "|\textit{target}|" "|[\textit{flags}]%
|\includeonly{|\textit{dest}|}\input{|\textit{main}|}"|
\end{center}
%

%%%%%%%%%%%%%%%%%%%%%%%%%%%%%%%%%%%%%%%%%%%%%%%%%%%%%%%%%%%%%%%%%%%%%%%%%%%%%%%%
%%%%%%%%%%%%%%%%%%%%%%%%%%%%%%%%%%%%%%%%%%%%%%%%%%%%%%%%%%%%%%%%%%%%%%%%%%%%%%%%
\section{Information}

%%%%%%%%%%%%%%%%%%%%%%%%%%%%%%%%%%%%%%%%%%%%%%%%%%%%%%%%%%%%%%%%%%%%%%%%%%%%%%%%
\subsection{Copyright}

Copyright \copyright{} 2017--2018 Niklas Beisert

This work may be distributed and/or modified under the
conditions of the \LaTeX{} Project Public License, either version 1.3
of this license or (at your option) any later version.
The latest version of this license is in
  \url{http://www.latex-project.org/lppl.txt}
and version 1.3 or later is part of all distributions of \LaTeX{}
version 2005/12/01 or later.

This work has the LPPL maintenance status `maintained'.

The Current Maintainer of this work is Niklas Beisert.

This work consists of the files |README.txt|, |childdoc.ins| and |childdoc.dtx|
as well as the derived files |childdoc.def|, |cdocsamp.tex|
with |cdocsch1.tex|, |cdocsch2.tex|, |cdocspt3.tex|, |cdocspt4.tex|,
|cdocsdrf.tex|, |cdocsfn1.tex|, |cdocsfn2.tex|
as well as |childdoc.pdf|.

%%%%%%%%%%%%%%%%%%%%%%%%%%%%%%%%%%%%%%%%%%%%%%%%%%%%%%%%%%%%%%%%%%%%%%%%%%%%%%%%
\subsection{Files and Installation}

The package consists of the files:
%
\begin{center}
\begin{tabular}{ll}
    |README.txt|   & readme file \\
    |childdoc.ins| & installation file \\
    |childdoc.dtx| & source file \\
    |childdoc.def| & definition file \\
    |cdocsamp.tex| & sample main file \\
    |cdocsch1.tex| & sample include file \\
    |cdocsch2.tex| & sample include file \\
    |cdocspt3.tex| & sample part file \\
    |cdocspt4.tex| & sample part file \\
    |cdocsdrf.tex| & sample redirection file \\
    |cdocsfn1.tex| & sample redirection file \\
    |cdocsfn2.tex| & sample redirection file \\
    |childdoc.pdf| & manual
\end{tabular}
\end{center}
%
The distribution consists of the files
|README.txt|, |childdoc.ins| and |childdoc.dtx|.
%
\begin{itemize}
\item
Run (pdf)\LaTeX{} on |childdoc.dtx|
to compile the manual |childdoc.pdf| (this file).
\item
Run \LaTeX{} on |childdoc.ins| to create the definitions file |childdoc.def|
and the sample |cdocsamp.tex| with include files
|cdocsch1.tex|, |cdocsch2.tex|, |cdocspt3.tex|, |cdocspt4.tex|,
|cdocsdrf.tex|, |cdocsfn1.tex|, |cdocsfn2.tex|.
Then copy the file |childdoc.def| to an appropriate directory of your \LaTeX{}
distribution, e.g.\ \textit{texmf-root}|/tex/latex/childdoc|.
\end{itemize}

%%%%%%%%%%%%%%%%%%%%%%%%%%%%%%%%%%%%%%%%%%%%%%%%%%%%%%%%%%%%%%%%%%%%%%%%%%%%%%%%
\subsection{Related CTAN Packages}

There are several other packages which offer a similar functionality:
%
\begin{itemize}
\item
The packages
\href{http://ctan.org/pkg/docmute}{\textsf{docmute}},
\href{http://ctan.org/pkg/includex}{\textsf{includex}} and
\href{http://ctan.org/pkg/standalone}{\textsf{standalone}}
provide commands to include only the document body of
a child file thus allowing both files to be compiled individually.
\item
The packages \href{http://ctan.org/pkg/subdocs}{\textsf{subdocs}}
and \href{http://ctan.org/pkg/subfiles}{\textsf{subfiles}}
provide structures in which the main and child documents can be
encapsulated and allowing them to be compiled individually.
The inclusion mechanism is different from the conventional |\include|.
\item
The package \href{http://ctan.org/pkg/combine}{\textsf{combine}}
is an elaborate solution to combine several documents into one.
\end{itemize}
%
See also the CTAN topic \href{http://ctan.org/topic/subdocs}{\textsf{subdocs}}
for further related packages.
The present package differs from the above solutions in that
a document structure constructed with the conventional |\include| mechanism
just needs two extra commands at the top of every file
such that all constituent files can be compiled individually.

%%%%%%%%%%%%%%%%%%%%%%%%%%%%%%%%%%%%%%%%%%%%%%%%%%%%%%%%%%%%%%%%%%%%%%%%%%%%%%%%
%\subsection{Feature Suggestions}
%
%The following is a list of features which may be useful for future
%versions of this package:
%%
%\begin{itemize}
%\item
%\ldots
%\end{itemize}

%%%%%%%%%%%%%%%%%%%%%%%%%%%%%%%%%%%%%%%%%%%%%%%%%%%%%%%%%%%%%%%%%%%%%%%%%%%%%%%%
\subsection{Revision History}

%%%%%%%%%%%%%%%%%%%%%%%%%%%%%%%%%%%%%%%%
\paragraph{v2.0:} 2018/12/30

\begin{itemize}
\item
immediate forward processing
\item
added |\childdocby| mechanism
\item
manual restructured
\end{itemize}

%%%%%%%%%%%%%%%%%%%%%%%%%%%%%%%%%%%%%%%%
\paragraph{v1.6:} 2018/01/17

\begin{itemize}
\item
application for development of include files
\item
corrections to manual
\end{itemize}

%%%%%%%%%%%%%%%%%%%%%%%%%%%%%%%%%%%%%%%%
\paragraph{v1.5:} 2017/05/21

\begin{itemize}
\item
more complete structuring introduced
\item
|\childdocof| introduced
\item
|\childdoc| renamed to |\childdocmain|
\item
|\childredirect| renamed to |\childdocforward| and |\childdocforwardprefix|
and functionality expanded
\end{itemize}

%%%%%%%%%%%%%%%%%%%%%%%%%%%%%%%%%%%%%%%%
\paragraph{v1.0:} 2017/04/27

\begin{itemize}
\item
manual and install package
\item
first version published on CTAN
\end{itemize}

%%%%%%%%%%%%%%%%%%%%%%%%%%%%%%%%%%%%%%%%
\paragraph{v0.6:} 2017/04/26

\begin{itemize}
\item
redirection mechanism added
\end{itemize}

%%%%%%%%%%%%%%%%%%%%%%%%%%%%%%%%%%%%%%%%
\paragraph{v0.5:} 2017/04/26

\begin{itemize}
\item
functionality in definition file
\end{itemize}


%%%%%%%%%%%%%%%%%%%%%%%%%%%%%%%%%%%%%%%%%%%%%%%%%%%%%%%%%%%%%%%%%%%%%%%%%%%%%%%%
%%%%%%%%%%%%%%%%%%%%%%%%%%%%%%%%%%%%%%%%%%%%%%%%%%%%%%%%%%%%%%%%%%%%%%%%%%%%%%%%
%%%%%%%%%%%%%%%%%%%%%%%%%%%%%%%%%%%%%%%%%%%%%%%%%%%%%%%%%%%%%%%%%%%%%%%%%%%%%%%%
\appendix

\settowidth\MacroIndent{\rmfamily\scriptsize 000\ }

 \DocInput{childdoc.dtx}

\end{document}
%</driver>
% \fi
%
% %%%%%%%%%%%%%%%%%%%%%%%%%%%%%%%%%%%%%%%%%%%%%%%%%%%%%%%%%%%%%%%%%%%%%%%%%%%%%%
% %%%%%%%%%%%%%%%%%%%%%%%%%%%%%%%%%%%%%%%%%%%%%%%%%%%%%%%%%%%%%%%%%%%%%%%%%%%%%%
% \section{Sample}
%\iffalse
%<*samplemain>
%\fi
%
% The following presents a sample document
% with two chapters, two parts, a title page,
% a compile flag as well as three forwarding files to set the flag.
% It consists of eight |.tex| files:
% \begin{center}
% \begin{tabular}{ll}
% |cdocsamp.tex|&main file\\
% |cdocsch1.tex|&include file for chapter 1\\
% |cdocsch2.tex|&include file for chapter 2\\
% |cdocspt3.tex|&include file for part 3\\
% |cdocspt4.tex|&include file for part 4\\
% |cdocsdrf.tex|&forwarding file for main file in draft mode\\
% |cdocsfi1.tex|&forwarding file for final version of chapter 1\\
% |cdocsfi2.tex|&forwarding file for final version of chapter 2\\
% \end{tabular}
% \end{center}
% Each of the eight files can be compiled directly by the \LaTeX{} compiler.
%
% %%%%%%%%%%%%%%%%%%%%%%%%%%%%%%%%%%%%%%
% \paragraph{Main File.}
%
% The main file is called |cdocsamp.tex|.
%
% Load the \textsf{childdoc} definitions and
% declare the filename for the main document:
%    \begin{macrocode}
\input{childdoc.def}
\childdocmain{}
%    \end{macrocode}

% Optional override for |\version| flag:
%    \begin{macrocode}
%%\ifchilddoc\else\providecommand{\version}{draft}\fi
%    \end{macrocode}

% Define the default values for the |\version| flag
% (|final| for the main file and |draft| for childs):
%    \begin{macrocode}
\ifchilddoc
\providecommand{\version}{draft}
\else
\providecommand{\version}{final}
\fi
%    \end{macrocode}

% Load the standard document class:
%    \begin{macrocode}
\documentclass[12pt]{article}
%    \end{macrocode}

% Start the document body:
%    \begin{macrocode}
\begin{document}
%    \end{macrocode}

% Declare a title page.
% Print title, part of document being processed and version flag:
%    \begin{macrocode}
\addtocounter{page}{-1}
\begin{center}
{\LARGE\bfseries{}childdoc example\par}
\vspace{1cm}
\ifchilddoc
\ifchilddocmanual part\else chapter\fi:
`\childdocname' of `\childdocjob'\par
\else
main document: `\childdocjob'\par
\fi
version: \version\par
\end{center}
\newpage
%    \end{macrocode}

% Manually include selected file,
% otherwise process as usual:
%    \begin{macrocode}
\ifchilddocmanual
\section*{part `\childdocname'}
\input{\childdocname}
\else
%    \end{macrocode}

% Include the two chapters:
%    \begin{macrocode}
\include{cdocsch1}
\include{cdocsch2}
%    \end{macrocode}

% Include the two parts unless only chapters should be displayed:
%    \begin{macrocode}
\ifchilddoc\else
\section{part three}
\input{cdocspt3}
\section{part four}
\input{cdocspt4}
\fi
%    \end{macrocode}

% Process as usual until here:
%    \begin{macrocode}
\fi
%    \end{macrocode}

% End of document body:
%    \begin{macrocode}
\end{document}
%    \end{macrocode}
%\iffalse
%</samplemain>
%\fi
%
% %%%%%%%%%%%%%%%%%%%%%%%%%%%%%%%%%%%%%%
% \paragraph{Chapter Include Files.}
%
% The include files are called |cdocsch1.tex| and |cdocsch2.tex|.
%
%\iffalse
%<*samplechap1|samplechap2>
%\fi

% Optional override for |\version| flag:
%    \begin{macrocode}
%%\providecommand{\version}{final}
%    \end{macrocode}

% Include the main document:
%    \begin{macrocode}
\input{childdoc.def}
\childdocof{cdocsamp}
%    \end{macrocode}

%\iffalse
%</samplechap1|samplechap2>
%\fi
%
%\iffalse
%<*samplechap1>
%\fi
% Some text for chapter 1:
%    \begin{macrocode}
\section{one}
some text in chapter one
%    \end{macrocode}

%\iffalse
%</samplechap1>
%\fi
% Some text for chapter 2:
%\iffalse
%<*samplechap2>
%\fi
%    \begin{macrocode}
\section{two}
more text in chapter two
%    \end{macrocode}

%\iffalse
%</samplechap2>
%\fi
%
% %%%%%%%%%%%%%%%%%%%%%%%%%%%%%%%%%%%%%%
% \paragraph{Part Include Files.}
%
% The include files are called |cdocspt3.tex| and |cdocspt4.tex|.
%
%\iffalse
%<*samplepart3|samplepart4>
%\fi

% Optional override for |\version| flag:
%    \begin{macrocode}
%%\providecommand{\version}{final}
%    \end{macrocode}

% Include the main document:
%    \begin{macrocode}
\input{childdoc.def}
\childdocby{cdocsamp}
%    \end{macrocode}

%\iffalse
%</samplepart3|samplepart4>
%\fi
%
%\iffalse
%<*samplepart3>
%\fi
% Some text for part 3:
%    \begin{macrocode}
some text in part three
%    \end{macrocode}

%\iffalse
%</samplepart3>
%\fi
% Some text for part 4:
%\iffalse
%<*samplepart4>
%\fi
%    \begin{macrocode}
more text in part four
%    \end{macrocode}

%\iffalse
%</samplepart4>
%\fi
%
% %%%%%%%%%%%%%%%%%%%%%%%%%%%%%%%%%%%%%%
% \paragraph{Forwarding for a Complete Draft.}
%
% The following forwarding file |cdocsdrf.tex|
% compiles the main document in draft mode:
%\iffalse
%<*sampledraft>
%\fi
%    \begin{macrocode}
\def\version{draft}
\input{childdoc.def}
\childdocforward{cdocsamp}
%    \end{macrocode}

%\iffalse
%</sampledraft>
%\fi
%
% %%%%%%%%%%%%%%%%%%%%%%%%%%%%%%%%%%%%%%
% \paragraph{Forwarding for Final Version of the Chapters.}
%
% The following forwarding files |cdocsfn1.tex| and |cdocsfn2.tex|
% (with identical content)
% compile the final versions of the child documents
% |cdocsch1.tex| and |cdocsch2.tex|, respectively:
%\iffalse
%<*samplefinal>
%\fi
%    \begin{macrocode}
\def\version{final}
\input{childdoc.def}
\childdocforwardprefix[cdocsamp]{cdocsfn}{cdocsch}
%    \end{macrocode}

%\iffalse
%</samplefinal>
%\fi
%
% %%%%%%%%%%%%%%%%%%%%%%%%%%%%%%%%%%%%%%
% \paragraph{Command Line Processing.}
%
% The following three command lines generate the output files
% |cdocscld|, |cdocscl1| and |cdocscl2|
% which should be identical to
% |cdocsdrf|, |cdocsch1| and |cdocsfn2|, respectively:
% \begin{center}
% \begin{tabular}{l}
% |latex -jobname cdocscld \|\\
% |  "\def\version{draft}\input{childdoc.def}\childdocforward{cdocsamp}"|\\
% |latex -jobname cdocscl1 \|\\
% |  "\input{childdoc.def}\childdocforward[cdocsamp]{cdocsch1}"|\\
% |latex -jobname cdocscl2 \|\\
% |  "\def\version{final}\input{childdoc.def}\childdocforward{cdocsch2}"|
% \end{tabular}
% \end{center}
% Note that the trailing backslash on each first line
% merely continues the input to the second line
% (for convenient cut ant paste).
% Furthermore, the command |latex| can be replaced by any
% of its alternative versions such as |pdflatex|.
%
% %%%%%%%%%%%%%%%%%%%%%%%%%%%%%%%%%%%%%%%%%%%%%%%%%%%%%%%%%%%%%%%%%%%%%%%%%%%%%%
% %%%%%%%%%%%%%%%%%%%%%%%%%%%%%%%%%%%%%%%%%%%%%%%%%%%%%%%%%%%%%%%%%%%%%%%%%%%%%%
% \section{Implementation}
%\iffalse
%<*package>
%\fi
%
% This section describes the definitions file |childdoc.def|.

% The definitions cannot be loaded using |\usepackage| or |\RequirePackage|
% which has a mechanism to prevent loading a style file more than once.
% When loading the definitions by means of |\input|
% multiple instances have to be prevented manually:
%\iffalse
%This code needs to be before the `\ProvidesFile' directive
%which is defined at the beginning of this file.
%Therefore it is also placed there and commented out here.
%</package>
%<*discard>
%\fi
%    \begin{macrocode}
\ifdefined\childdocmain\endinput\fi
%    \end{macrocode}
%\iffalse
%</discard>
%<*package>
%\fi
%
% \macro{\ifchilddoc}
% \macro{\ifchilddocmanual}
% The conditional |\ifchilddoc| tells whether a
% child (true) or main (false) document is being compiled.
% The conditional |\ifchilddocmanual| tells whether
% the |\includeonly| mechanism is used (false) or
% the selection of child files must be performed manually (true).
% The definitions initialise to false:
%    \begin{macrocode}
\newif\ifchilddoc
\newif\ifchilddocmanual
%    \end{macrocode}

% \macro{\childdocname}
% \macro{\childdocjob}
% The macro |\childdocname| stores the name of the main document
% to be compiled. The macro |\childdocjob| stores the name of
% the document on which the \LaTeX{} compiler was originally invoked.
% The content of |\jobname| cannot be compared
% to filenames specified in the source due to different catcodes.
% The following code rescans |\jobname|, stores the result
% in |\childdocname| and saves a copy in |\childdocjob|:
%    \begin{macrocode}
\edef\childdocname{\scantokens\expandafter{\jobname\noexpand}}
\let\childdocjob\childdocname
%    \end{macrocode}

% \macro{\childdocdisable}
% The macro |\childdocdisable| prevents the main file
% from being processed more than once.
% At this stage, the main document command |\childdocmain|
% is assumed to be called once again where it should do nothing.
% Any subsequent call to it should prevent
% a secondary processing of the main document
% It overwrites the forwarding commands
% |\childdocof| and |\childdocforward|
% with empty macros to prevent further inclusions of the main document:
%    \begin{macrocode}
\newcommand{\childdocdisable}
{
  \renewcommand{\childdocmain}[1]{\renewcommand{\childdocmain}[1]{\endinput}}
  \renewcommand{\childdocof}[1]{}
  \renewcommand{\childdocby}[2][]{}
  \renewcommand{\childdocforward}[2][]{}
  \renewcommand{\childdocdisable}{}
}
%    \end{macrocode}

% \macro{\childdocmain}
% The macro |\childdocmain| is to be called at the top of the main file
% with nothing or the main filename (without extension) as argument.
% First, it breaks loops.
% If the argument is not empty and does not match |\childdocname|
% (which is set by the first inclusion of |childdoc.def|),
% |\ifchilddoc| is set to true, |\includeonly| is applied to the child file
% and |\jobname| is set to the main file
% (for proper handling of |.aux| files):
%    \begin{macrocode}
\newcommand{\childdocmain}[1]
{
  \childdocdisable\childdocmain{}
  \if?#1?\else
    \begingroup
      \def\childdoctmp{#1}
      \ifx\childdoctmp\childdocname
        \def\childdoctmp{}
      \else
        \def\childdoctmp
        {
          \childdoctrue
          \includeonly{\childdocname}
          \def\childdocjob{#1}
          \def\jobname{#1}
        }
      \fi
      \expandafter
    \endgroup
    \childdoctmp
  \fi
}
%    \end{macrocode}

% \macro{\childdocof}
% The command |\childdocof| redirects
% compilation to the main file |#1|.
%    \begin{macrocode}
\newcommand{\childdocof}[1]
{
  \childdocdisable
  \childdoctrue
  \includeonly{\childdocname}
  \def\jobname{#1}
  \def\childdocjob{#1}
  \input{#1}
}
%    \end{macrocode}

% \macro{\childdocby}
% The command |\childdocby| ....
%    \begin{macrocode}
\newcommand{\childdocby}[2][]
{
  \childdocdisable
  \childdoctrue
  \childdocmanualtrue
  \if?#1?\else
    \def\jobname{#2}
  \fi
  \def\childdocjob{#2}
  \input{#2}
  \endinput
}
%    \end{macrocode}

% \macro{\childdocforward}
% The command |\childdocforward| redirects
% compilation to the main file or
% (if the optional argument is given) a child file.
% Parameters are set as if the main file
% or a child file starting with |\childdocof| was compiled.
% Then compilation is handed over to the main file:
%    \begin{macrocode}
\newcommand{\childdocforward}[2][]
{
  \begingroup
    \if?#1?
      \def\childdoctmp
      {
        \def\childdocname{#2}
        \def\childdocjob{#2}
        \def\jobname{#2}
        \input{#2}
        \endinput
      }
    \else
      \def\childdoctmp
      {
        \childdocdisable
        \def\childdocname{#2}
        \childdoctrue
        \includeonly{#2}
        \def\childdocjob{#1}
        \def\jobname{#1}
        \input{#1}
        \endinput
      }
    \fi
    \expandafter
  \endgroup
  \childdoctmp
}
%    \end{macrocode}

% \macro{\childdocforwardprefix}
% The command |\childdocforwardprefix| redirects
% compilation to the main or a child file by means of a pattern.
% The prefix |#1| in the current filename is replaced by |#2|
% and the suffix of the current filename is kept
% (it is assumed that the filename does not contain the substring `|~~~|'
% which is used as a delimiter).
% Compilation is handed over to the new file by |\childdocforward|:
%    \begin{macrocode}
\newcommand{\childdocforwardprefix}[3][]
{
  \begingroup
    \def\childdocextract #2##1~~~{\def\childdoctmp{\childdocforward[#1]{#3##1}}}
    \expandafter\childdocextract\childdocname~~~
    \expandafter
  \endgroup
  \childdoctmp
}
%    \end{macrocode}

% \macro{\childdoc}
% The deprecated macro |\childdoc| is a legacy version of |\childdocmain|:
%    \begin{macrocode}
\newcommand{\childdoc}{\childdocmain}
%    \end{macrocode}

% \macro{\childdocredirect}
% The deprecated macro |\childdocredirect| is a legacy version
% of |\childdocforward| and |\childdocforwardprefix|:
%    \begin{macrocode}
\newcommand{\childdocredirect}[2][]
{
  \begingroup
    \if?#1?
      \def\childdoctmp{\childdocforward{#2}}
    \else
      \def\childdoctmp{\childdocforwardprefix{#1}{#2}}
    \fi
    \expandafter
  \endgroup
  \childdoctmp
}
%    \end{macrocode}

%\iffalse
%</package>
%\fi
%
\endinput
\childdocforward{cdocsch2}"|
% \end{tabular}
% \end{center}
% Note that the trailing backslash on each first line
% merely continues the input to the second line
% (for convenient cut ant paste).
% Furthermore, the command |latex| can be replaced by any
% of its alternative versions such as |pdflatex|.
%
% %%%%%%%%%%%%%%%%%%%%%%%%%%%%%%%%%%%%%%%%%%%%%%%%%%%%%%%%%%%%%%%%%%%%%%%%%%%%%%
% %%%%%%%%%%%%%%%%%%%%%%%%%%%%%%%%%%%%%%%%%%%%%%%%%%%%%%%%%%%%%%%%%%%%%%%%%%%%%%
% \section{Implementation}
%\iffalse
%<*package>
%\fi
%
% This section describes the definitions file |childdoc.def|.

% The definitions cannot be loaded using |\usepackage| or |\RequirePackage|
% which has a mechanism to prevent loading a style file more than once.
% When loading the definitions by means of |\input|
% multiple instances have to be prevented manually:
%\iffalse
%This code needs to be before the `\ProvidesFile' directive
%which is defined at the beginning of this file.
%Therefore it is also placed there and commented out here.
%</package>
%<*discard>
%\fi
%    \begin{macrocode}
\ifdefined\childdocmain\endinput\fi
%    \end{macrocode}
%\iffalse
%</discard>
%<*package>
%\fi
%
% \macro{\ifchilddoc}
% \macro{\ifchilddocmanual}
% The conditional |\ifchilddoc| tells whether a
% child (true) or main (false) document is being compiled.
% The conditional |\ifchilddocmanual| tells whether
% the |\includeonly| mechanism is used (false) or
% the selection of child files must be performed manually (true).
% The definitions initialise to false:
%    \begin{macrocode}
\newif\ifchilddoc
\newif\ifchilddocmanual
%    \end{macrocode}

% \macro{\childdocname}
% \macro{\childdocjob}
% The macro |\childdocname| stores the name of the main document
% to be compiled. The macro |\childdocjob| stores the name of
% the document on which the \LaTeX{} compiler was originally invoked.
% The content of |\jobname| cannot be compared
% to filenames specified in the source due to different catcodes.
% The following code rescans |\jobname|, stores the result
% in |\childdocname| and saves a copy in |\childdocjob|:
%    \begin{macrocode}
\edef\childdocname{\scantokens\expandafter{\jobname\noexpand}}
\let\childdocjob\childdocname
%    \end{macrocode}

% \macro{\childdocdisable}
% The macro |\childdocdisable| prevents the main file
% from being processed more than once.
% At this stage, the main document command |\childdocmain|
% is assumed to be called once again where it should do nothing.
% Any subsequent call to it should prevent
% a secondary processing of the main document
% It overwrites the forwarding commands
% |\childdocof| and |\childdocforward|
% with empty macros to prevent further inclusions of the main document:
%    \begin{macrocode}
\newcommand{\childdocdisable}
{
  \renewcommand{\childdocmain}[1]{\renewcommand{\childdocmain}[1]{\endinput}}
  \renewcommand{\childdocof}[1]{}
  \renewcommand{\childdocby}[2][]{}
  \renewcommand{\childdocforward}[2][]{}
  \renewcommand{\childdocdisable}{}
}
%    \end{macrocode}

% \macro{\childdocmain}
% The macro |\childdocmain| is to be called at the top of the main file
% with nothing or the main filename (without extension) as argument.
% First, it breaks loops.
% If the argument is not empty and does not match |\childdocname|
% (which is set by the first inclusion of |childdoc.def|),
% |\ifchilddoc| is set to true, |\includeonly| is applied to the child file
% and |\jobname| is set to the main file
% (for proper handling of |.aux| files):
%    \begin{macrocode}
\newcommand{\childdocmain}[1]
{
  \childdocdisable\childdocmain{}
  \if?#1?\else
    \begingroup
      \def\childdoctmp{#1}
      \ifx\childdoctmp\childdocname
        \def\childdoctmp{}
      \else
        \def\childdoctmp
        {
          \childdoctrue
          \includeonly{\childdocname}
          \def\childdocjob{#1}
          \def\jobname{#1}
        }
      \fi
      \expandafter
    \endgroup
    \childdoctmp
  \fi
}
%    \end{macrocode}

% \macro{\childdocof}
% The command |\childdocof| redirects
% compilation to the main file |#1|.
%    \begin{macrocode}
\newcommand{\childdocof}[1]
{
  \childdocdisable
  \childdoctrue
  \includeonly{\childdocname}
  \def\jobname{#1}
  \def\childdocjob{#1}
  \input{#1}
}
%    \end{macrocode}

% \macro{\childdocby}
% The command |\childdocby| ....
%    \begin{macrocode}
\newcommand{\childdocby}[2][]
{
  \childdocdisable
  \childdoctrue
  \childdocmanualtrue
  \if?#1?\else
    \def\jobname{#2}
  \fi
  \def\childdocjob{#2}
  \input{#2}
  \endinput
}
%    \end{macrocode}

% \macro{\childdocforward}
% The command |\childdocforward| redirects
% compilation to the main file or
% (if the optional argument is given) a child file.
% Parameters are set as if the main file
% or a child file starting with |\childdocof| was compiled.
% Then compilation is handed over to the main file:
%    \begin{macrocode}
\newcommand{\childdocforward}[2][]
{
  \begingroup
    \if?#1?
      \def\childdoctmp
      {
        \def\childdocname{#2}
        \def\childdocjob{#2}
        \def\jobname{#2}
        \input{#2}
        \endinput
      }
    \else
      \def\childdoctmp
      {
        \childdocdisable
        \def\childdocname{#2}
        \childdoctrue
        \includeonly{#2}
        \def\childdocjob{#1}
        \def\jobname{#1}
        \input{#1}
        \endinput
      }
    \fi
    \expandafter
  \endgroup
  \childdoctmp
}
%    \end{macrocode}

% \macro{\childdocforwardprefix}
% The command |\childdocforwardprefix| redirects
% compilation to the main or a child file by means of a pattern.
% The prefix |#1| in the current filename is replaced by |#2|
% and the suffix of the current filename is kept
% (it is assumed that the filename does not contain the substring `|~~~|'
% which is used as a delimiter).
% Compilation is handed over to the new file by |\childdocforward|:
%    \begin{macrocode}
\newcommand{\childdocforwardprefix}[3][]
{
  \begingroup
    \def\childdocextract #2##1~~~{\def\childdoctmp{\childdocforward[#1]{#3##1}}}
    \expandafter\childdocextract\childdocname~~~
    \expandafter
  \endgroup
  \childdoctmp
}
%    \end{macrocode}

% \macro{\childdoc}
% The deprecated macro |\childdoc| is a legacy version of |\childdocmain|:
%    \begin{macrocode}
\newcommand{\childdoc}{\childdocmain}
%    \end{macrocode}

% \macro{\childdocredirect}
% The deprecated macro |\childdocredirect| is a legacy version
% of |\childdocforward| and |\childdocforwardprefix|:
%    \begin{macrocode}
\newcommand{\childdocredirect}[2][]
{
  \begingroup
    \if?#1?
      \def\childdoctmp{\childdocforward{#2}}
    \else
      \def\childdoctmp{\childdocforwardprefix{#1}{#2}}
    \fi
    \expandafter
  \endgroup
  \childdoctmp
}
%    \end{macrocode}

%\iffalse
%</package>
%\fi
%
\endinput
\childdocforward{cdocsamp}"|\\
% |latex -jobname cdocscl1 \|\\
% |  "% \iffalse
%
% childdoc.dtx Copyright (C) 2017-2018 Niklas Beisert
%
% This work may be distributed and/or modified under the
% conditions of the LaTeX Project Public License, either version 1.3
% of this license or (at your option) any later version.
% The latest version of this license is in
%   http://www.latex-project.org/lppl.txt
% and version 1.3 or later is part of all distributions of LaTeX
% version 2005/12/01 or later.
%
% This work has the LPPL maintenance status `maintained'.
%
% The Current Maintainer of this work is Niklas Beisert.
%
% This work consists of the files childdoc.dtx and childdoc.ins
% and the derived files childdoc.def and cdocsamp.tex with
% cdocsch1.tex, cdocsch2.tex, cdocsdrf.tex, cdocsfn1.tex, cdocsfn2.tex.
%
%<package>\ifdefined\childdocmain\endinput\fi
%<package>\ProvidesFile{childdoc.def}[2018/12/30 v2.0 child document driver]
%<samplemain>\ProvidesFile{cdocsamp.tex}[2018/12/30 v2.0 sample for childdoc]
%<*driver>
%\ProvidesFile{childdoc.drv}[2018/12/30 v2.0 childdoc reference manual file]
\PassOptionsToClass{10pt,a4paper}{article}
\documentclass{ltxdoc}

\usepackage[margin=35mm]{geometry}
\usepackage{hyperref}
\usepackage{hyperxmp}
\usepackage[usenames]{color}

\hypersetup{colorlinks=true}
\hypersetup{pdfstartview=FitH}
\hypersetup{pdfpagemode=UseNone}
\hypersetup{pdfsource={}}
\hypersetup{pdflang={en-UK}}
\hypersetup{pdfcopyright={Copyright 2017-2018 Niklas Beisert.
  This work may be distributed and/or modified under the
  conditions of the LaTeX Project Public License, either version 1.3
  of this license or (at your option) any later version.}}
\hypersetup{pdflicenseurl={http://www.latex-project.org/lppl.txt}}
\hypersetup{pdfcontactaddress={ETH Zurich, ITP, HIT K,
  Wolfgang-Pauli-Strasse 27}}
\hypersetup{pdfcontactpostcode={8093}}
\hypersetup{pdfcontactcity={Zurich}}
\hypersetup{pdfcontactcountry={Switzerland}}
\hypersetup{pdfcontactemail={nbeisert@itp.phys.ethz.ch}}
\hypersetup{pdfcontacturl={http://people.phys.ethz.ch/\xmptilde nbeisert/}}

\newcommand{\secref}[1]{\hyperref[#1]{section \ref*{#1}}}

\parskip1ex
\parindent0pt
\let\olditemize\itemize
\def\itemize{\olditemize\parskip0pt}

\begin{document}

\title{The \textsf{childdoc} Package}
\hypersetup{pdftitle={The childdoc Package}}
\author{Niklas Beisert\\[2ex]
  Institut f\"ur Theoretische Physik\\
  Eidgen\"ossische Technische Hochschule Z\"urich\\
  Wolfgang-Pauli-Strasse 27, 8093 Z\"urich, Switzerland\\[1ex]
  \href{mailto:nbeisert@itp.phys.ethz.ch}
  {\texttt{nbeisert@itp.phys.ethz.ch}}}
\hypersetup{pdfauthor={Niklas Beisert}}
\hypersetup{pdfsubject={Manual for the LaTeX2e Package childdoc}}
\date{30 December 2018, \textsf{v2.0}}
\maketitle

\begin{abstract}\noindent
\textsf{childdoc} is a \LaTeXe{} package
that enables the direct compilation
of document sections included by |\include|
to individual files.
\end{abstract}

\begingroup
\parskip0ex
\tableofcontents
\endgroup

%%%%%%%%%%%%%%%%%%%%%%%%%%%%%%%%%%%%%%%%%%%%%%%%%%%%%%%%%%%%%%%%%%%%%%%%%%%%%%%%
%%%%%%%%%%%%%%%%%%%%%%%%%%%%%%%%%%%%%%%%%%%%%%%%%%%%%%%%%%%%%%%%%%%%%%%%%%%%%%%%
\section{Introduction}

\LaTeX{} provides a mechanism to structure a large document (such as a book)
into a main file and several child files (containing the chapters)
using the |\include| command.
This mechanism is beneficial for documents
which span hundreds of pages in order to
make the source file(s) more manageable.
Moreover, compilation can be restricted to
selected child files by means of the |\includeonly| command.
The latter feature can be used to reduce the compilation time while editing
(this was significantly more useful in the earlier days of \LaTeX{})
or to generate a smaller document which is easier to navigate.
Another application of |\includeonly| is to generate
documents consisting of selected parts of the complete document.

However, there are a few drawbacks of the plain |\include| mechanism:
\begin{itemize}
\item
The child files cannot be compiled on their own,
they can only be compiled via the main file.
A naive editing environment
(such as a text editor with an option
to have the current file processed by \LaTeX)
may require one to switch to the main file before compiling;
attempting to compile the child file produces errors.
\item
The main file must be modified (each time)
to adjust the |\includeonly| command
to the present needs. This easily leaves the main file in a messy state.
\item
The generated document will always carry the filename
of the main document. This is inconvenient if
several child files are to be compiled and
to be kept for distribution.
\end{itemize}

The present package provides a simple interface
to make child files individually compilable by \LaTeX{}.
Compiling a child file then has the same effect as compiling
the main file with an |\includeonly| command
to select the appropriate child.
Moreover the generated document will carry the name of the child
rather than the main file.
This resolves all three above issues.

This feature is meant to make the editing of books,
thesis documents and lecture notes somewhat more convenient.
However, the package can also be used efficiently for
composing a series of documents (such as exercise sheets)
which are typically distributed individually.
It then assists the author in generating the individual documents
(potentially in different versions)
as well as a document containing the collected series.
Another application is in developing style files
or other kinds of included material
where compilation of the style file could redirect
to a sample or test file.

%%%%%%%%%%%%%%%%%%%%%%%%%%%%%%%%%%%%%%%%%%%%%%%%%%%%%%%%%%%%%%%%%%%%%%%%%%%%%%%%
%%%%%%%%%%%%%%%%%%%%%%%%%%%%%%%%%%%%%%%%%%%%%%%%%%%%%%%%%%%%%%%%%%%%%%%%%%%%%%%%
\section{Usage}

First of all, the package \textsf{childdoc} is \emph{not} a standard
\LaTeXe{} |.sty| style file! Therefore it needs to be invoked in
a non-standard way.

%%%%%%%%%%%%%%%%%%%%%%%%%%%%%%%%%%%%%%%%%%%%%%%%%%%%%%%%%%%%%%%%%%%%%%%%%%%%%%%%
\subsection{Included Files}
\label{sec:include}

%%%%%%%%%%%%%%%%%%%%%%%%%%%%%%%%%%%%%%%%
\DescribeMacro{\childdocmain}
To use the package, add the commands
\begin{center}
\begin{tabular}{l}
|% \iffalse
%
% childdoc.dtx Copyright (C) 2017-2018 Niklas Beisert
%
% This work may be distributed and/or modified under the
% conditions of the LaTeX Project Public License, either version 1.3
% of this license or (at your option) any later version.
% The latest version of this license is in
%   http://www.latex-project.org/lppl.txt
% and version 1.3 or later is part of all distributions of LaTeX
% version 2005/12/01 or later.
%
% This work has the LPPL maintenance status `maintained'.
%
% The Current Maintainer of this work is Niklas Beisert.
%
% This work consists of the files childdoc.dtx and childdoc.ins
% and the derived files childdoc.def and cdocsamp.tex with
% cdocsch1.tex, cdocsch2.tex, cdocsdrf.tex, cdocsfn1.tex, cdocsfn2.tex.
%
%<package>\ifdefined\childdocmain\endinput\fi
%<package>\ProvidesFile{childdoc.def}[2018/12/30 v2.0 child document driver]
%<samplemain>\ProvidesFile{cdocsamp.tex}[2018/12/30 v2.0 sample for childdoc]
%<*driver>
%\ProvidesFile{childdoc.drv}[2018/12/30 v2.0 childdoc reference manual file]
\PassOptionsToClass{10pt,a4paper}{article}
\documentclass{ltxdoc}

\usepackage[margin=35mm]{geometry}
\usepackage{hyperref}
\usepackage{hyperxmp}
\usepackage[usenames]{color}

\hypersetup{colorlinks=true}
\hypersetup{pdfstartview=FitH}
\hypersetup{pdfpagemode=UseNone}
\hypersetup{pdfsource={}}
\hypersetup{pdflang={en-UK}}
\hypersetup{pdfcopyright={Copyright 2017-2018 Niklas Beisert.
  This work may be distributed and/or modified under the
  conditions of the LaTeX Project Public License, either version 1.3
  of this license or (at your option) any later version.}}
\hypersetup{pdflicenseurl={http://www.latex-project.org/lppl.txt}}
\hypersetup{pdfcontactaddress={ETH Zurich, ITP, HIT K,
  Wolfgang-Pauli-Strasse 27}}
\hypersetup{pdfcontactpostcode={8093}}
\hypersetup{pdfcontactcity={Zurich}}
\hypersetup{pdfcontactcountry={Switzerland}}
\hypersetup{pdfcontactemail={nbeisert@itp.phys.ethz.ch}}
\hypersetup{pdfcontacturl={http://people.phys.ethz.ch/\xmptilde nbeisert/}}

\newcommand{\secref}[1]{\hyperref[#1]{section \ref*{#1}}}

\parskip1ex
\parindent0pt
\let\olditemize\itemize
\def\itemize{\olditemize\parskip0pt}

\begin{document}

\title{The \textsf{childdoc} Package}
\hypersetup{pdftitle={The childdoc Package}}
\author{Niklas Beisert\\[2ex]
  Institut f\"ur Theoretische Physik\\
  Eidgen\"ossische Technische Hochschule Z\"urich\\
  Wolfgang-Pauli-Strasse 27, 8093 Z\"urich, Switzerland\\[1ex]
  \href{mailto:nbeisert@itp.phys.ethz.ch}
  {\texttt{nbeisert@itp.phys.ethz.ch}}}
\hypersetup{pdfauthor={Niklas Beisert}}
\hypersetup{pdfsubject={Manual for the LaTeX2e Package childdoc}}
\date{30 December 2018, \textsf{v2.0}}
\maketitle

\begin{abstract}\noindent
\textsf{childdoc} is a \LaTeXe{} package
that enables the direct compilation
of document sections included by |\include|
to individual files.
\end{abstract}

\begingroup
\parskip0ex
\tableofcontents
\endgroup

%%%%%%%%%%%%%%%%%%%%%%%%%%%%%%%%%%%%%%%%%%%%%%%%%%%%%%%%%%%%%%%%%%%%%%%%%%%%%%%%
%%%%%%%%%%%%%%%%%%%%%%%%%%%%%%%%%%%%%%%%%%%%%%%%%%%%%%%%%%%%%%%%%%%%%%%%%%%%%%%%
\section{Introduction}

\LaTeX{} provides a mechanism to structure a large document (such as a book)
into a main file and several child files (containing the chapters)
using the |\include| command.
This mechanism is beneficial for documents
which span hundreds of pages in order to
make the source file(s) more manageable.
Moreover, compilation can be restricted to
selected child files by means of the |\includeonly| command.
The latter feature can be used to reduce the compilation time while editing
(this was significantly more useful in the earlier days of \LaTeX{})
or to generate a smaller document which is easier to navigate.
Another application of |\includeonly| is to generate
documents consisting of selected parts of the complete document.

However, there are a few drawbacks of the plain |\include| mechanism:
\begin{itemize}
\item
The child files cannot be compiled on their own,
they can only be compiled via the main file.
A naive editing environment
(such as a text editor with an option
to have the current file processed by \LaTeX)
may require one to switch to the main file before compiling;
attempting to compile the child file produces errors.
\item
The main file must be modified (each time)
to adjust the |\includeonly| command
to the present needs. This easily leaves the main file in a messy state.
\item
The generated document will always carry the filename
of the main document. This is inconvenient if
several child files are to be compiled and
to be kept for distribution.
\end{itemize}

The present package provides a simple interface
to make child files individually compilable by \LaTeX{}.
Compiling a child file then has the same effect as compiling
the main file with an |\includeonly| command
to select the appropriate child.
Moreover the generated document will carry the name of the child
rather than the main file.
This resolves all three above issues.

This feature is meant to make the editing of books,
thesis documents and lecture notes somewhat more convenient.
However, the package can also be used efficiently for
composing a series of documents (such as exercise sheets)
which are typically distributed individually.
It then assists the author in generating the individual documents
(potentially in different versions)
as well as a document containing the collected series.
Another application is in developing style files
or other kinds of included material
where compilation of the style file could redirect
to a sample or test file.

%%%%%%%%%%%%%%%%%%%%%%%%%%%%%%%%%%%%%%%%%%%%%%%%%%%%%%%%%%%%%%%%%%%%%%%%%%%%%%%%
%%%%%%%%%%%%%%%%%%%%%%%%%%%%%%%%%%%%%%%%%%%%%%%%%%%%%%%%%%%%%%%%%%%%%%%%%%%%%%%%
\section{Usage}

First of all, the package \textsf{childdoc} is \emph{not} a standard
\LaTeXe{} |.sty| style file! Therefore it needs to be invoked in
a non-standard way.

%%%%%%%%%%%%%%%%%%%%%%%%%%%%%%%%%%%%%%%%%%%%%%%%%%%%%%%%%%%%%%%%%%%%%%%%%%%%%%%%
\subsection{Included Files}
\label{sec:include}

%%%%%%%%%%%%%%%%%%%%%%%%%%%%%%%%%%%%%%%%
\DescribeMacro{\childdocmain}
To use the package, add the commands
\begin{center}
\begin{tabular}{l}
|\input{childdoc.def}|\\
|\childdocmain{}|\\
\end{tabular}
\end{center}
at the very top of the main \LaTeX{} file,
in particular \emph{before} the |\documentclass| statement!
The argument of |\childdocmain| should be left empty
(but it must be present).

%%%%%%%%%%%%%%%%%%%%%%%%%%%%%%%%%%%%%%%%
\DescribeMacro{\childdocof}
Furthermore, add the commands
\begin{center}
\begin{tabular}{l}
|\input{childdoc.def}|\\
|\childdocof{|\textit{main}|}|\\
\end{tabular}
\end{center}
at the top of every child file \textit{child}
which is included by |\include{|\textit{child}|}|
from within the main file
(or at least for those files to be compiled individually).
The argument \textit{main} must be the filename of the main file.

There are a couple of
considerations in setting up the main and child documents:

%%%%%%%%%%%%%%%%%%%%%%%%%%%%%%%%%%%%%%%%
\paragraph{Restrictions.}

Please note the following restrictions:
\begin{itemize}
\item
|\childdocmain| must be called with one argument \textit{main}
to ensure compatibility with earlier version of the package.
It must either be empty (|\childdocmain{}|)
or precisely match the filename of the main file in which it is specified.
See \secref{sec:detection} for further information.
\item
The filename \textit{main} must be specified without the |.tex| extension.
\item
The filename \textit{main} is case sensitive
(even in case-insensitive file systems)
due to internal string comparison.
\item
The argument \textit{main} should be fully expanded, it cannot be a macro.
\item
Subdirectories and special characters should be avoided in filenames.
\item
The command |\childdocmain{|\textit{main}|}| must be followed by a whitespace.
It should not be followed immediately by another command
or by a comment mark `|%|'.
This is because the \TeX{} parser reads the token immediately following
the argument of |\childdocmain| and puts it
at the beginning of every child section;
however, a white\-space is ignored.
\end{itemize}

%%%%%%%%%%%%%%%%%%%%%%%%%%%%%%%%%%%%%%%%
\paragraph{Content of Main File.}

It is advisable to place all content in the child files included by |\include|.
Any output contained in the main file will appear in all child documents
unless suppressed manually;
it cannot be suppressed automatically by the |\includeonly| directive
and thus should normally be avoided.
A method to include some content in the main file
by means of conditional processing is described in \secref{sec:conditional}.

%%%%%%%%%%%%%%%%%%%%%%%%%%%%%%%%%%%%%%%%
\paragraph{Page Numbering.}

When only a part of the document is compiled,
the appropriate numbering of pages
(as well as other status parameters)
is determined from the |.aux| files.
The latter contain information from previous passes.
However this information needs to propagate through
all intermediate child documents.
Therefore the page numbering in child documents may well
be inconsistent until the complete document is compiled at least once.

A useful (if unconventional) way to always ensure a consistent
page numbering is to restart the numbering in each child document
and denote the pages by `\textit{child}|.|\textit{page}'
where \textit{child} represents the chapter/section number of the child file.
This can be achieved by the command
|\numberwithin{page}{|\textit{child}|}|
of the \textsf{amsmath} package
where \textit{child} can be |chapter| or |section|
depending on the chosen structuring.
Alternatively, one can modify the macro |\thepage| appropriately
and reset the counter |page| at the start of each child file.

%%%%%%%%%%%%%%%%%%%%%%%%%%%%%%%%%%%%%%%%%%%%%%%%%%%%%%%%%%%%%%%%%%%%%%%%%%%%%%%%
\subsection{Conditional Processing}
\label{sec:conditional}

The package provides a mechanism to compile different versions
of a document. To customise the versions further some conditional processing
can come in handy to distinguish which version is being compiled.
The package provides two macros to describe the compilation context:

%%%%%%%%%%%%%%%%%%%%%%%%%%%%%%%%%%%%%%%%
\DescribeMacro{\ifchilddoc}
The conditional |\ifchilddoc| distinguishes between the compilation of
child documents and the main document:
%
\begin{center}
|\ifchilddoc |\textit{child-code}| |[|\||else |\textit{main-code}]| \||fi|
\end{center}

%%%%%%%%%%%%%%%%%%%%%%%%%%%%%%%%%%%%%%%%
\DescribeMacro{\childdocname}
\DescribeMacro{\childdocjob}
The macro |\childdocname| contains the filename (without extension)
of the main or child file being processed.
Note that |\childdocjob| will always contain the name of the main file.

%%%%%%%%%%%%%%%%%%%%%%%%%%%%%%%%%%%%%%%%
\paragraph{Title Page.}

Conditional processing can be used to include a title or banner page
in the main document when proper precautions are taken.
Importantly, the code in the main file should ensure that the page counter
(as well as other status parameters which are stored in the |.aux| files)
takes the same value after the conditional processing.
Otherwise the page numbers may take divergent values
depending on which part is compiled.

For example, a title page could be declared by:
%
\begin{center}
\begin{tabular}{l}
|\ifchilddoc\||else|\\
|\addtocounter{page}{-1}|\\
\textit{code for title page}\\
|\newpage|\\
|\||fi|
\end{tabular}
\end{center}
%
A banner page for the child documents can be generated by:
%
\begin{center}
\begin{tabular}{l}
|\ifchilddoc|\\
|\addtocounter{page}{-1}|\\
\textit{code for banner page}\\
|\newpage|\\
|\||fi|
\end{tabular}
\end{center}
%
Here one could write a message such as:
\begin{center}
|This is the part \childdocname{} of \childdocjob{}.|
\end{center}

%%%%%%%%%%%%%%%%%%%%%%%%%%%%%%%%%%%%%%%%%%%%%%%%%%%%%%%%%%%%%%%%%%%%%%%%%%%%%%%%
\subsection{Flags}
\label{sec:flags}

The package makes it easy to generate different versions
of the main or child documents.
To this end compilation flags can be defined
and assigned different default values.
They will be particularly useful in conjunction
with the forwarding mechanism described in \secref{sec:forward}.

For example, it may be useful to have a flag |\version|
which can be set to |draft| or |final|.
The document source will contain some conditional code
depending on the value of |\version|.
Suppose further, the flag should default to |final| for the main file
and to |draft| for child files
which is a natural assignment for editing the document.
This is achieved by placing the following code
in the preamble of the main document
(below the |\childdocmain| directive):
%
\begin{center}
\begin{tabular}{l}
|\ifchilddoc|\\
|\providecommand{\version}{draft}|\\
|\||else|\\
|\providecommand{\version}{final}|\\
|\||fi|
\end{tabular}
\end{center}
%
The definition by |\providecommand| makes sure
that previous definitions are not overwritten.
Further statements |\providecommand{\version}{...}|
can thus be added before the above code to override it.

For the main file, one might add a line
(between |\childdocmain| and the above block)
%
\begin{center}
|%\ifchilddoc\||else\providecommand{\version}{draft}\||fi|
\end{center}
%
which can be uncommented to produce a draft version.
Likewise one can add a line to the very top of a child file
(above the |\childdocof{|\textit{main}|}| directive)
%
\begin{center}
|%\providecommand{\version}{final}|
\end{center}
%
which can be uncommented to produce the final version of this child document.

%%%%%%%%%%%%%%%%%%%%%%%%%%%%%%%%%%%%%%%%%%%%%%%%%%%%%%%%%%%%%%%%%%%%%%%%%%%%%%%%
\subsection{Forwarding}
\label{sec:forward}

Different versions of the main or child documents
using compilation flags as described in \secref{sec:flags}
can be (permanently) stored in different files
for convenient compilation, viewing and distribution.
To this end, the package defines a command
to pass on compilation to a different file:

%%%%%%%%%%%%%%%%%%%%%%%%%%%%%%%%%%%%%%%%
\DescribeMacro{\childdocforward}
The command |\childdocforward| redirects processing to
another source file:
%
\begin{center}
\begin{tabular}{l}
|\input{childdoc.def}|\\
|\childdocforward[|\textit{main}|]{|\textit{dest}|}|\\
\end{tabular}
\end{center}
%
The argument \textit{dest} is the destination file
(without extension).
It should be the main file or one of the child files.
Note that further \textsf{childdoc} directives
such as |\childdocof| and |\childdocforward|
in the indicated file will be processed in this form.
The optional argument \textit{main}
passes on directly to the main file \textit{main}
while pretending to compile the child \textit{dest}.
This form behaves as if \textit{dest}
issues |\childdocof{|\textit{main}|}| right away,
and no further \textsf{childdoc} directives will be processed.

%%%%%%%%%%%%%%%%%%%%%%%%%%%%%%%%%%%%%%%%
\DescribeMacro{\...prefix}
In the alternative form |\childdocforwardprefix|,
%
\begin{center}
\begin{tabular}{l}
|\input{childdoc.def}|\\
|\childdocforwardprefix[|\textit{main}|]{|\textit{prefix}|}{|\textit{dest}|}|
\end{tabular}
\end{center}
%
the destination file is determined by a pattern
depending on the current file:
To make this work, the current file must be called
`{\textit{prefix}\hspace{0.2em}\textit{suffix}}'
with \textit{prefix} matching precisely the argument.
Processing is then passed on to the file
`{\textit{dest}\hspace{0.2em}\textit{suffix}}'.
Surely, the same effect is achieved by
directly specifying the
argument `{\textit{dest}\hspace{0.2em}\textit{suffix}}'
in the first form.
However, that requires to set up a different file
for each child. With the alternative form of the command
all these files can have exactly the same content
which simplifies setting them up and maintaining them.

For example, the following file |draft.tex|
with a compilation flag |\version| as described in \secref{sec:flags}
compiles the main document as a draft:
%
\begin{center}
\begin{tabular}{l}
|\def\version{draft}|\\
|\input{childdoc.def}|\\
|\childdocforward{|\textit{main}|}|
\end{tabular}
\end{center}
%
Likewise, the following files |final|\textit{nn}|.tex|
compile the final version of the child document
|child|\textit{nn}|.tex|:
%
\begin{center}
\begin{tabular}{l}
|\def\version{final}|\\
|\input{childdoc.def}|\\
|\childdocforwardprefix{final}{child}|
\end{tabular}
\end{center}
%

Note that when several versions of a main file and/or of each child file
are to be generated, it may be convenient to set up a |Makefile| or
shell script to automatise the process.

%%%%%%%%%%%%%%%%%%%%%%%%%%%%%%%%%%%%%%%%%%%%%%%%%%%%%%%%%%%%%%%%%%%%%%%%%%%%%%%%
\subsection{Command Line Processing}
\label{sec:commandline}

The effect of redirection files can also be achieved by invoking
the \LaTeX{} compiler with a more elaborate command line.
Most conveniently this should be done as part
of a shell script or a |Makefile|.

When using \textsf{childdoc} in the main file, the following
command lines effectively perform a redirection
(note that depending on the shell being used,
backslashes may have to be doubled: `|\|' $\to$ `|\\|'):
%
\begin{center}
|... -jobname "|\textit{target}|" |\\|"|[\textit{flags}]%
|\input{childdoc.def}\childdocforward[|\textit{main}|]{|\textit{dest}|}"|
\end{center}
%
Here \textit{target} is the name of the output file,
\textit{main} is the name of the main file
and \textit{dest} is the name of the main or child file to be processed
(all filenames without extensions).
The optional argument \textit{main} can be omitted
if \textit{main} matches \textit{dest}.
Optionally, compilation \textit{flags} can be defined via |\def| commands.
This command line makes the \TeX{} engine believe
it is compiling the file \textit{target}
whose content is specified as the latter parameter.
The provided code then forwards the processing to
\textit{main} or \textit{dest} as described in \secref{sec:forward}.

%%%%%%%%%%%%%%%%%%%%%%%%%%%%%%%%%%%%%%%%%%%%%%%%%%%%%%%%%%%%%%%%%%%%%%%%%%%%%%%%
\subsection{Include by Input}
\label{sec:input}

Including child documents by |\include| has some restrictions by design.
Most notably, the content of a child document always occupies
its own set of pages; pages cannot be shared between child documents.
Usually, this behaviour makes perfect sense
because each child document contain an essential part of the document.
However, in some situations it may be desirable to compose
a document from a collection of parts
without having mandatory page breaks between then.
For this case, the package
provides a mechanism to include parts
by |\input| which can also be processed individually.
However, by construction this mechanism
requires manual handling of the content to be output.

%%%%%%%%%%%%%%%%%%%%%%%%%%%%%%%%%%%%%%%%
\DescribeMacro{\ifchilddocmanual}
The main file should be prepared as usual, see \secref{sec:include}.
However, the document body must make a distinction
between processing of an individual part and of the main document, e.g.:
%
\begin{center}
\begin{tabular}{l}
|\ifchilddocmanual|\\
|\input{\childdocname}|\\
|\||else|\\
\textit{document body with }|\input{|\textit{part}|}|\\
|\||fi|
\end{tabular}
\end{center}
%
The conditional |\ifchilddocmanual| is true whenever
a part to be included by |\input| is being compiled,
and the name of the part is stored in |\childdocname|.

%%%%%%%%%%%%%%%%%%%%%%%%%%%%%%%%%%%%%%%%
\DescribeMacro{\childdocby}
Each part to be included by |\input| should start with:
%
\begin{center}
\begin{tabular}{l}
|\input{childdoc.def}|\\
|\childdocby{|\textit{main}|}|\\
\end{tabular}
\end{center}
%
The directive |\childdocby| is similar to |\childdocof|
described in \secref{sec:include},
but the subsequent selection of content must be done manually.
To that end, both |\ifchilddoc| and |\ifchilddocmanual|
will be true upon processing of a part,
and the name of the part is stored in |\childdocname|.
Note that |\jobname| will be set to the filename of the current part
so that each part receives an individual |.aux| file
that does not interfere with the |.aux| file(s) of the main document.
This behaviour can be altered by the alternative form
|\childdocby[*]{|\textit{main}|}| (with a non-empty optional argument)
which uses the |.aux| file of the main document
by setting |\jobname| to \textit{main}.

%%%%%%%%%%%%%%%%%%%%%%%%%%%%%%%%%%%%%%%%%%%%%%%%%%%%%%%%%%%%%%%%%%%%%%%%%%%%%%%%
\subsection{Driver Development}
\label{sec:driver}

The \textsf{childdoc} mechanism can also be use for the development
of definition files such as \LaTeX{} styles or classes.
This case differs from the above setup with multiple parts
included by |\include| in that no |\includeonly| should be invoked.
This can be achieved by starting the include file
(before |\ProvidesPackage|) with:
%
\begin{center}
\begin{tabular}{l}
|\input{childdoc.def}|\\
|\childdocforward{|\textit{main}|}|\\
\end{tabular}
\end{center}
%
or alternatively with:
%
\begin{center}
\begin{tabular}{l}
|\input{childdoc.def}|\\
|\childdocby{|\textit{main}|}|\\
\end{tabular}
\end{center}
%
Both forms have slightly different effects as described above.
The main file is prepared as usual, see \secref{sec:include}.

%%%%%%%%%%%%%%%%%%%%%%%%%%%%%%%%%%%%%%%%%%%%%%%%%%%%%%%%%%%%%%%%%%%%%%%%%%%%%%%%
\subsection{Legacy Detection}
\label{sec:detection}

The directive |\childdocmain| in the main file can detect
whether the complete document or merely a child is to be compiled
even without using the directive |\childdocof|.
This method is deprecated because it is less robust
and there is no compelling reason to use it;
it is merely provided for backward compatibility
and it may be removed in future versions.

If the detection mechanism is to be used,
it is mandatory to correctly specify
the filename of the main file as the argument of |\childdocmain|:
%
\begin{center}
\begin{tabular}{l}
|\input{childdoc.def}|\\
|\childdocmain{|\textit{main}|}|\\
\end{tabular}
\end{center}
%
If |\jobname| does not match the argument \textit{main} of |\childdocmain|,
it is assumed that |\jobname| points to the child file to be compiled.
When using |\childdocmain| with the main file specified as argument,
it suffices to start a child file
with just |\input{|\textit{main}|}|
without loading of the package and using |\childdocof|.
If instead all processing is done
with the appropriate \textsf{childdoc} directives,
the argument of \textit{main} of |\childdocmain| can be empty.

An alternative version of the command line processing described
in \secref{sec:commandline} using the detection mechanism reads:
%
\begin{center}
|... -jobname "|\textit{target}|" "|[\textit{flags}]%
[|\def\jobname{|\textit{dest}|}|]|\input{|\textit{main}|}"|
\end{center}

%%%%%%%%%%%%%%%%%%%%%%%%%%%%%%%%%%%%%%%%%%%%%%%%%%%%%%%%%%%%%%%%%%%%%%%%%%%%%%%%
\subsection{Manual Code}
\label{sec:manual}

In case one cannot be certain whether the definitions file |childdoc.def|
is installed on the target \TeX{} distribution
and one prefers not to ship it,
it is conceivable to paste a few relevant commands into the sources.

To that end, drop all statements |\input{childdoc.def}|
and perform the replacements as outlined below.
Instead of |\childdocmain{|\textit{main}|}| add the following code
to the top of the main file:
%
\begin{center}
\begin{tabular}{l}
|\||ifdefined\childdocname\endinput\||fi\newif\ifchilddoc|\\
|\edef\childdocname{\scantokens\expandafter{\jobname\noexpand}}|\\
|\def\childdocmain{|\textit{main}|}\||ifx\childdocmain\childdocname\||else|\\
|\childdoctrue\includeonly{\childdocname}\let\jobname\childdocmain\||fi|\\
\end{tabular}
\end{center}
%
Instead of |\childdocof{|\textit{main}|}| just include the main file
at the top of each child file:
%
\begin{center}
|\input{|\textit{main}|}|
\end{center}
%
A simple redirection |\childdocforward{|\textit{dest}|}| is achieved by:
%
\begin{center}
|\def\jobname{|\textit{dest}|}\input{\jobname}|
\end{center}
%
The redirection with prefix
|\childdocforwardprefix[|\textit{prefix}|]{|\textit{dest}|}|
is accomplished by:
%
\begin{center}
\begin{tabular}{l}
|{\edef\jobname{\scantokens\expandafter{\jobname\noexpand}}|\\
|\def\redirectjob |\textit{prefix}|#1~~~{\gdef\jobname{|\textit{dest}|#1}}|\\
|\expandafter\redirectjob\jobname~~~}\input{\jobname}|
\end{tabular}
\end{center}

In an alternative approach,
child documents can be compiled by a specific command line
without additional code or specific definitions:
%
\begin{center}
|... -jobname "|\textit{target}|" "|[\textit{flags}]%
|\includeonly{|\textit{dest}|}\input{|\textit{main}|}"|
\end{center}
%

%%%%%%%%%%%%%%%%%%%%%%%%%%%%%%%%%%%%%%%%%%%%%%%%%%%%%%%%%%%%%%%%%%%%%%%%%%%%%%%%
%%%%%%%%%%%%%%%%%%%%%%%%%%%%%%%%%%%%%%%%%%%%%%%%%%%%%%%%%%%%%%%%%%%%%%%%%%%%%%%%
\section{Information}

%%%%%%%%%%%%%%%%%%%%%%%%%%%%%%%%%%%%%%%%%%%%%%%%%%%%%%%%%%%%%%%%%%%%%%%%%%%%%%%%
\subsection{Copyright}

Copyright \copyright{} 2017--2018 Niklas Beisert

This work may be distributed and/or modified under the
conditions of the \LaTeX{} Project Public License, either version 1.3
of this license or (at your option) any later version.
The latest version of this license is in
  \url{http://www.latex-project.org/lppl.txt}
and version 1.3 or later is part of all distributions of \LaTeX{}
version 2005/12/01 or later.

This work has the LPPL maintenance status `maintained'.

The Current Maintainer of this work is Niklas Beisert.

This work consists of the files |README.txt|, |childdoc.ins| and |childdoc.dtx|
as well as the derived files |childdoc.def|, |cdocsamp.tex|
with |cdocsch1.tex|, |cdocsch2.tex|, |cdocspt3.tex|, |cdocspt4.tex|,
|cdocsdrf.tex|, |cdocsfn1.tex|, |cdocsfn2.tex|
as well as |childdoc.pdf|.

%%%%%%%%%%%%%%%%%%%%%%%%%%%%%%%%%%%%%%%%%%%%%%%%%%%%%%%%%%%%%%%%%%%%%%%%%%%%%%%%
\subsection{Files and Installation}

The package consists of the files:
%
\begin{center}
\begin{tabular}{ll}
    |README.txt|   & readme file \\
    |childdoc.ins| & installation file \\
    |childdoc.dtx| & source file \\
    |childdoc.def| & definition file \\
    |cdocsamp.tex| & sample main file \\
    |cdocsch1.tex| & sample include file \\
    |cdocsch2.tex| & sample include file \\
    |cdocspt3.tex| & sample part file \\
    |cdocspt4.tex| & sample part file \\
    |cdocsdrf.tex| & sample redirection file \\
    |cdocsfn1.tex| & sample redirection file \\
    |cdocsfn2.tex| & sample redirection file \\
    |childdoc.pdf| & manual
\end{tabular}
\end{center}
%
The distribution consists of the files
|README.txt|, |childdoc.ins| and |childdoc.dtx|.
%
\begin{itemize}
\item
Run (pdf)\LaTeX{} on |childdoc.dtx|
to compile the manual |childdoc.pdf| (this file).
\item
Run \LaTeX{} on |childdoc.ins| to create the definitions file |childdoc.def|
and the sample |cdocsamp.tex| with include files
|cdocsch1.tex|, |cdocsch2.tex|, |cdocspt3.tex|, |cdocspt4.tex|,
|cdocsdrf.tex|, |cdocsfn1.tex|, |cdocsfn2.tex|.
Then copy the file |childdoc.def| to an appropriate directory of your \LaTeX{}
distribution, e.g.\ \textit{texmf-root}|/tex/latex/childdoc|.
\end{itemize}

%%%%%%%%%%%%%%%%%%%%%%%%%%%%%%%%%%%%%%%%%%%%%%%%%%%%%%%%%%%%%%%%%%%%%%%%%%%%%%%%
\subsection{Related CTAN Packages}

There are several other packages which offer a similar functionality:
%
\begin{itemize}
\item
The packages
\href{http://ctan.org/pkg/docmute}{\textsf{docmute}},
\href{http://ctan.org/pkg/includex}{\textsf{includex}} and
\href{http://ctan.org/pkg/standalone}{\textsf{standalone}}
provide commands to include only the document body of
a child file thus allowing both files to be compiled individually.
\item
The packages \href{http://ctan.org/pkg/subdocs}{\textsf{subdocs}}
and \href{http://ctan.org/pkg/subfiles}{\textsf{subfiles}}
provide structures in which the main and child documents can be
encapsulated and allowing them to be compiled individually.
The inclusion mechanism is different from the conventional |\include|.
\item
The package \href{http://ctan.org/pkg/combine}{\textsf{combine}}
is an elaborate solution to combine several documents into one.
\end{itemize}
%
See also the CTAN topic \href{http://ctan.org/topic/subdocs}{\textsf{subdocs}}
for further related packages.
The present package differs from the above solutions in that
a document structure constructed with the conventional |\include| mechanism
just needs two extra commands at the top of every file
such that all constituent files can be compiled individually.

%%%%%%%%%%%%%%%%%%%%%%%%%%%%%%%%%%%%%%%%%%%%%%%%%%%%%%%%%%%%%%%%%%%%%%%%%%%%%%%%
%\subsection{Feature Suggestions}
%
%The following is a list of features which may be useful for future
%versions of this package:
%%
%\begin{itemize}
%\item
%\ldots
%\end{itemize}

%%%%%%%%%%%%%%%%%%%%%%%%%%%%%%%%%%%%%%%%%%%%%%%%%%%%%%%%%%%%%%%%%%%%%%%%%%%%%%%%
\subsection{Revision History}

%%%%%%%%%%%%%%%%%%%%%%%%%%%%%%%%%%%%%%%%
\paragraph{v2.0:} 2018/12/30

\begin{itemize}
\item
immediate forward processing
\item
added |\childdocby| mechanism
\item
manual restructured
\end{itemize}

%%%%%%%%%%%%%%%%%%%%%%%%%%%%%%%%%%%%%%%%
\paragraph{v1.6:} 2018/01/17

\begin{itemize}
\item
application for development of include files
\item
corrections to manual
\end{itemize}

%%%%%%%%%%%%%%%%%%%%%%%%%%%%%%%%%%%%%%%%
\paragraph{v1.5:} 2017/05/21

\begin{itemize}
\item
more complete structuring introduced
\item
|\childdocof| introduced
\item
|\childdoc| renamed to |\childdocmain|
\item
|\childredirect| renamed to |\childdocforward| and |\childdocforwardprefix|
and functionality expanded
\end{itemize}

%%%%%%%%%%%%%%%%%%%%%%%%%%%%%%%%%%%%%%%%
\paragraph{v1.0:} 2017/04/27

\begin{itemize}
\item
manual and install package
\item
first version published on CTAN
\end{itemize}

%%%%%%%%%%%%%%%%%%%%%%%%%%%%%%%%%%%%%%%%
\paragraph{v0.6:} 2017/04/26

\begin{itemize}
\item
redirection mechanism added
\end{itemize}

%%%%%%%%%%%%%%%%%%%%%%%%%%%%%%%%%%%%%%%%
\paragraph{v0.5:} 2017/04/26

\begin{itemize}
\item
functionality in definition file
\end{itemize}


%%%%%%%%%%%%%%%%%%%%%%%%%%%%%%%%%%%%%%%%%%%%%%%%%%%%%%%%%%%%%%%%%%%%%%%%%%%%%%%%
%%%%%%%%%%%%%%%%%%%%%%%%%%%%%%%%%%%%%%%%%%%%%%%%%%%%%%%%%%%%%%%%%%%%%%%%%%%%%%%%
%%%%%%%%%%%%%%%%%%%%%%%%%%%%%%%%%%%%%%%%%%%%%%%%%%%%%%%%%%%%%%%%%%%%%%%%%%%%%%%%
\appendix

\settowidth\MacroIndent{\rmfamily\scriptsize 000\ }

 \DocInput{childdoc.dtx}

\end{document}
%</driver>
% \fi
%
% %%%%%%%%%%%%%%%%%%%%%%%%%%%%%%%%%%%%%%%%%%%%%%%%%%%%%%%%%%%%%%%%%%%%%%%%%%%%%%
% %%%%%%%%%%%%%%%%%%%%%%%%%%%%%%%%%%%%%%%%%%%%%%%%%%%%%%%%%%%%%%%%%%%%%%%%%%%%%%
% \section{Sample}
%\iffalse
%<*samplemain>
%\fi
%
% The following presents a sample document
% with two chapters, two parts, a title page,
% a compile flag as well as three forwarding files to set the flag.
% It consists of eight |.tex| files:
% \begin{center}
% \begin{tabular}{ll}
% |cdocsamp.tex|&main file\\
% |cdocsch1.tex|&include file for chapter 1\\
% |cdocsch2.tex|&include file for chapter 2\\
% |cdocspt3.tex|&include file for part 3\\
% |cdocspt4.tex|&include file for part 4\\
% |cdocsdrf.tex|&forwarding file for main file in draft mode\\
% |cdocsfi1.tex|&forwarding file for final version of chapter 1\\
% |cdocsfi2.tex|&forwarding file for final version of chapter 2\\
% \end{tabular}
% \end{center}
% Each of the eight files can be compiled directly by the \LaTeX{} compiler.
%
% %%%%%%%%%%%%%%%%%%%%%%%%%%%%%%%%%%%%%%
% \paragraph{Main File.}
%
% The main file is called |cdocsamp.tex|.
%
% Load the \textsf{childdoc} definitions and
% declare the filename for the main document:
%    \begin{macrocode}
\input{childdoc.def}
\childdocmain{}
%    \end{macrocode}

% Optional override for |\version| flag:
%    \begin{macrocode}
%%\ifchilddoc\else\providecommand{\version}{draft}\fi
%    \end{macrocode}

% Define the default values for the |\version| flag
% (|final| for the main file and |draft| for childs):
%    \begin{macrocode}
\ifchilddoc
\providecommand{\version}{draft}
\else
\providecommand{\version}{final}
\fi
%    \end{macrocode}

% Load the standard document class:
%    \begin{macrocode}
\documentclass[12pt]{article}
%    \end{macrocode}

% Start the document body:
%    \begin{macrocode}
\begin{document}
%    \end{macrocode}

% Declare a title page.
% Print title, part of document being processed and version flag:
%    \begin{macrocode}
\addtocounter{page}{-1}
\begin{center}
{\LARGE\bfseries{}childdoc example\par}
\vspace{1cm}
\ifchilddoc
\ifchilddocmanual part\else chapter\fi:
`\childdocname' of `\childdocjob'\par
\else
main document: `\childdocjob'\par
\fi
version: \version\par
\end{center}
\newpage
%    \end{macrocode}

% Manually include selected file,
% otherwise process as usual:
%    \begin{macrocode}
\ifchilddocmanual
\section*{part `\childdocname'}
\input{\childdocname}
\else
%    \end{macrocode}

% Include the two chapters:
%    \begin{macrocode}
\include{cdocsch1}
\include{cdocsch2}
%    \end{macrocode}

% Include the two parts unless only chapters should be displayed:
%    \begin{macrocode}
\ifchilddoc\else
\section{part three}
\input{cdocspt3}
\section{part four}
\input{cdocspt4}
\fi
%    \end{macrocode}

% Process as usual until here:
%    \begin{macrocode}
\fi
%    \end{macrocode}

% End of document body:
%    \begin{macrocode}
\end{document}
%    \end{macrocode}
%\iffalse
%</samplemain>
%\fi
%
% %%%%%%%%%%%%%%%%%%%%%%%%%%%%%%%%%%%%%%
% \paragraph{Chapter Include Files.}
%
% The include files are called |cdocsch1.tex| and |cdocsch2.tex|.
%
%\iffalse
%<*samplechap1|samplechap2>
%\fi

% Optional override for |\version| flag:
%    \begin{macrocode}
%%\providecommand{\version}{final}
%    \end{macrocode}

% Include the main document:
%    \begin{macrocode}
\input{childdoc.def}
\childdocof{cdocsamp}
%    \end{macrocode}

%\iffalse
%</samplechap1|samplechap2>
%\fi
%
%\iffalse
%<*samplechap1>
%\fi
% Some text for chapter 1:
%    \begin{macrocode}
\section{one}
some text in chapter one
%    \end{macrocode}

%\iffalse
%</samplechap1>
%\fi
% Some text for chapter 2:
%\iffalse
%<*samplechap2>
%\fi
%    \begin{macrocode}
\section{two}
more text in chapter two
%    \end{macrocode}

%\iffalse
%</samplechap2>
%\fi
%
% %%%%%%%%%%%%%%%%%%%%%%%%%%%%%%%%%%%%%%
% \paragraph{Part Include Files.}
%
% The include files are called |cdocspt3.tex| and |cdocspt4.tex|.
%
%\iffalse
%<*samplepart3|samplepart4>
%\fi

% Optional override for |\version| flag:
%    \begin{macrocode}
%%\providecommand{\version}{final}
%    \end{macrocode}

% Include the main document:
%    \begin{macrocode}
\input{childdoc.def}
\childdocby{cdocsamp}
%    \end{macrocode}

%\iffalse
%</samplepart3|samplepart4>
%\fi
%
%\iffalse
%<*samplepart3>
%\fi
% Some text for part 3:
%    \begin{macrocode}
some text in part three
%    \end{macrocode}

%\iffalse
%</samplepart3>
%\fi
% Some text for part 4:
%\iffalse
%<*samplepart4>
%\fi
%    \begin{macrocode}
more text in part four
%    \end{macrocode}

%\iffalse
%</samplepart4>
%\fi
%
% %%%%%%%%%%%%%%%%%%%%%%%%%%%%%%%%%%%%%%
% \paragraph{Forwarding for a Complete Draft.}
%
% The following forwarding file |cdocsdrf.tex|
% compiles the main document in draft mode:
%\iffalse
%<*sampledraft>
%\fi
%    \begin{macrocode}
\def\version{draft}
\input{childdoc.def}
\childdocforward{cdocsamp}
%    \end{macrocode}

%\iffalse
%</sampledraft>
%\fi
%
% %%%%%%%%%%%%%%%%%%%%%%%%%%%%%%%%%%%%%%
% \paragraph{Forwarding for Final Version of the Chapters.}
%
% The following forwarding files |cdocsfn1.tex| and |cdocsfn2.tex|
% (with identical content)
% compile the final versions of the child documents
% |cdocsch1.tex| and |cdocsch2.tex|, respectively:
%\iffalse
%<*samplefinal>
%\fi
%    \begin{macrocode}
\def\version{final}
\input{childdoc.def}
\childdocforwardprefix[cdocsamp]{cdocsfn}{cdocsch}
%    \end{macrocode}

%\iffalse
%</samplefinal>
%\fi
%
% %%%%%%%%%%%%%%%%%%%%%%%%%%%%%%%%%%%%%%
% \paragraph{Command Line Processing.}
%
% The following three command lines generate the output files
% |cdocscld|, |cdocscl1| and |cdocscl2|
% which should be identical to
% |cdocsdrf|, |cdocsch1| and |cdocsfn2|, respectively:
% \begin{center}
% \begin{tabular}{l}
% |latex -jobname cdocscld \|\\
% |  "\def\version{draft}\input{childdoc.def}\childdocforward{cdocsamp}"|\\
% |latex -jobname cdocscl1 \|\\
% |  "\input{childdoc.def}\childdocforward[cdocsamp]{cdocsch1}"|\\
% |latex -jobname cdocscl2 \|\\
% |  "\def\version{final}\input{childdoc.def}\childdocforward{cdocsch2}"|
% \end{tabular}
% \end{center}
% Note that the trailing backslash on each first line
% merely continues the input to the second line
% (for convenient cut ant paste).
% Furthermore, the command |latex| can be replaced by any
% of its alternative versions such as |pdflatex|.
%
% %%%%%%%%%%%%%%%%%%%%%%%%%%%%%%%%%%%%%%%%%%%%%%%%%%%%%%%%%%%%%%%%%%%%%%%%%%%%%%
% %%%%%%%%%%%%%%%%%%%%%%%%%%%%%%%%%%%%%%%%%%%%%%%%%%%%%%%%%%%%%%%%%%%%%%%%%%%%%%
% \section{Implementation}
%\iffalse
%<*package>
%\fi
%
% This section describes the definitions file |childdoc.def|.

% The definitions cannot be loaded using |\usepackage| or |\RequirePackage|
% which has a mechanism to prevent loading a style file more than once.
% When loading the definitions by means of |\input|
% multiple instances have to be prevented manually:
%\iffalse
%This code needs to be before the `\ProvidesFile' directive
%which is defined at the beginning of this file.
%Therefore it is also placed there and commented out here.
%</package>
%<*discard>
%\fi
%    \begin{macrocode}
\ifdefined\childdocmain\endinput\fi
%    \end{macrocode}
%\iffalse
%</discard>
%<*package>
%\fi
%
% \macro{\ifchilddoc}
% \macro{\ifchilddocmanual}
% The conditional |\ifchilddoc| tells whether a
% child (true) or main (false) document is being compiled.
% The conditional |\ifchilddocmanual| tells whether
% the |\includeonly| mechanism is used (false) or
% the selection of child files must be performed manually (true).
% The definitions initialise to false:
%    \begin{macrocode}
\newif\ifchilddoc
\newif\ifchilddocmanual
%    \end{macrocode}

% \macro{\childdocname}
% \macro{\childdocjob}
% The macro |\childdocname| stores the name of the main document
% to be compiled. The macro |\childdocjob| stores the name of
% the document on which the \LaTeX{} compiler was originally invoked.
% The content of |\jobname| cannot be compared
% to filenames specified in the source due to different catcodes.
% The following code rescans |\jobname|, stores the result
% in |\childdocname| and saves a copy in |\childdocjob|:
%    \begin{macrocode}
\edef\childdocname{\scantokens\expandafter{\jobname\noexpand}}
\let\childdocjob\childdocname
%    \end{macrocode}

% \macro{\childdocdisable}
% The macro |\childdocdisable| prevents the main file
% from being processed more than once.
% At this stage, the main document command |\childdocmain|
% is assumed to be called once again where it should do nothing.
% Any subsequent call to it should prevent
% a secondary processing of the main document
% It overwrites the forwarding commands
% |\childdocof| and |\childdocforward|
% with empty macros to prevent further inclusions of the main document:
%    \begin{macrocode}
\newcommand{\childdocdisable}
{
  \renewcommand{\childdocmain}[1]{\renewcommand{\childdocmain}[1]{\endinput}}
  \renewcommand{\childdocof}[1]{}
  \renewcommand{\childdocby}[2][]{}
  \renewcommand{\childdocforward}[2][]{}
  \renewcommand{\childdocdisable}{}
}
%    \end{macrocode}

% \macro{\childdocmain}
% The macro |\childdocmain| is to be called at the top of the main file
% with nothing or the main filename (without extension) as argument.
% First, it breaks loops.
% If the argument is not empty and does not match |\childdocname|
% (which is set by the first inclusion of |childdoc.def|),
% |\ifchilddoc| is set to true, |\includeonly| is applied to the child file
% and |\jobname| is set to the main file
% (for proper handling of |.aux| files):
%    \begin{macrocode}
\newcommand{\childdocmain}[1]
{
  \childdocdisable\childdocmain{}
  \if?#1?\else
    \begingroup
      \def\childdoctmp{#1}
      \ifx\childdoctmp\childdocname
        \def\childdoctmp{}
      \else
        \def\childdoctmp
        {
          \childdoctrue
          \includeonly{\childdocname}
          \def\childdocjob{#1}
          \def\jobname{#1}
        }
      \fi
      \expandafter
    \endgroup
    \childdoctmp
  \fi
}
%    \end{macrocode}

% \macro{\childdocof}
% The command |\childdocof| redirects
% compilation to the main file |#1|.
%    \begin{macrocode}
\newcommand{\childdocof}[1]
{
  \childdocdisable
  \childdoctrue
  \includeonly{\childdocname}
  \def\jobname{#1}
  \def\childdocjob{#1}
  \input{#1}
}
%    \end{macrocode}

% \macro{\childdocby}
% The command |\childdocby| ....
%    \begin{macrocode}
\newcommand{\childdocby}[2][]
{
  \childdocdisable
  \childdoctrue
  \childdocmanualtrue
  \if?#1?\else
    \def\jobname{#2}
  \fi
  \def\childdocjob{#2}
  \input{#2}
  \endinput
}
%    \end{macrocode}

% \macro{\childdocforward}
% The command |\childdocforward| redirects
% compilation to the main file or
% (if the optional argument is given) a child file.
% Parameters are set as if the main file
% or a child file starting with |\childdocof| was compiled.
% Then compilation is handed over to the main file:
%    \begin{macrocode}
\newcommand{\childdocforward}[2][]
{
  \begingroup
    \if?#1?
      \def\childdoctmp
      {
        \def\childdocname{#2}
        \def\childdocjob{#2}
        \def\jobname{#2}
        \input{#2}
        \endinput
      }
    \else
      \def\childdoctmp
      {
        \childdocdisable
        \def\childdocname{#2}
        \childdoctrue
        \includeonly{#2}
        \def\childdocjob{#1}
        \def\jobname{#1}
        \input{#1}
        \endinput
      }
    \fi
    \expandafter
  \endgroup
  \childdoctmp
}
%    \end{macrocode}

% \macro{\childdocforwardprefix}
% The command |\childdocforwardprefix| redirects
% compilation to the main or a child file by means of a pattern.
% The prefix |#1| in the current filename is replaced by |#2|
% and the suffix of the current filename is kept
% (it is assumed that the filename does not contain the substring `|~~~|'
% which is used as a delimiter).
% Compilation is handed over to the new file by |\childdocforward|:
%    \begin{macrocode}
\newcommand{\childdocforwardprefix}[3][]
{
  \begingroup
    \def\childdocextract #2##1~~~{\def\childdoctmp{\childdocforward[#1]{#3##1}}}
    \expandafter\childdocextract\childdocname~~~
    \expandafter
  \endgroup
  \childdoctmp
}
%    \end{macrocode}

% \macro{\childdoc}
% The deprecated macro |\childdoc| is a legacy version of |\childdocmain|:
%    \begin{macrocode}
\newcommand{\childdoc}{\childdocmain}
%    \end{macrocode}

% \macro{\childdocredirect}
% The deprecated macro |\childdocredirect| is a legacy version
% of |\childdocforward| and |\childdocforwardprefix|:
%    \begin{macrocode}
\newcommand{\childdocredirect}[2][]
{
  \begingroup
    \if?#1?
      \def\childdoctmp{\childdocforward{#2}}
    \else
      \def\childdoctmp{\childdocforwardprefix{#1}{#2}}
    \fi
    \expandafter
  \endgroup
  \childdoctmp
}
%    \end{macrocode}

%\iffalse
%</package>
%\fi
%
\endinput
|\\
|\childdocmain{}|\\
\end{tabular}
\end{center}
at the very top of the main \LaTeX{} file,
in particular \emph{before} the |\documentclass| statement!
The argument of |\childdocmain| should be left empty
(but it must be present).

%%%%%%%%%%%%%%%%%%%%%%%%%%%%%%%%%%%%%%%%
\DescribeMacro{\childdocof}
Furthermore, add the commands
\begin{center}
\begin{tabular}{l}
|% \iffalse
%
% childdoc.dtx Copyright (C) 2017-2018 Niklas Beisert
%
% This work may be distributed and/or modified under the
% conditions of the LaTeX Project Public License, either version 1.3
% of this license or (at your option) any later version.
% The latest version of this license is in
%   http://www.latex-project.org/lppl.txt
% and version 1.3 or later is part of all distributions of LaTeX
% version 2005/12/01 or later.
%
% This work has the LPPL maintenance status `maintained'.
%
% The Current Maintainer of this work is Niklas Beisert.
%
% This work consists of the files childdoc.dtx and childdoc.ins
% and the derived files childdoc.def and cdocsamp.tex with
% cdocsch1.tex, cdocsch2.tex, cdocsdrf.tex, cdocsfn1.tex, cdocsfn2.tex.
%
%<package>\ifdefined\childdocmain\endinput\fi
%<package>\ProvidesFile{childdoc.def}[2018/12/30 v2.0 child document driver]
%<samplemain>\ProvidesFile{cdocsamp.tex}[2018/12/30 v2.0 sample for childdoc]
%<*driver>
%\ProvidesFile{childdoc.drv}[2018/12/30 v2.0 childdoc reference manual file]
\PassOptionsToClass{10pt,a4paper}{article}
\documentclass{ltxdoc}

\usepackage[margin=35mm]{geometry}
\usepackage{hyperref}
\usepackage{hyperxmp}
\usepackage[usenames]{color}

\hypersetup{colorlinks=true}
\hypersetup{pdfstartview=FitH}
\hypersetup{pdfpagemode=UseNone}
\hypersetup{pdfsource={}}
\hypersetup{pdflang={en-UK}}
\hypersetup{pdfcopyright={Copyright 2017-2018 Niklas Beisert.
  This work may be distributed and/or modified under the
  conditions of the LaTeX Project Public License, either version 1.3
  of this license or (at your option) any later version.}}
\hypersetup{pdflicenseurl={http://www.latex-project.org/lppl.txt}}
\hypersetup{pdfcontactaddress={ETH Zurich, ITP, HIT K,
  Wolfgang-Pauli-Strasse 27}}
\hypersetup{pdfcontactpostcode={8093}}
\hypersetup{pdfcontactcity={Zurich}}
\hypersetup{pdfcontactcountry={Switzerland}}
\hypersetup{pdfcontactemail={nbeisert@itp.phys.ethz.ch}}
\hypersetup{pdfcontacturl={http://people.phys.ethz.ch/\xmptilde nbeisert/}}

\newcommand{\secref}[1]{\hyperref[#1]{section \ref*{#1}}}

\parskip1ex
\parindent0pt
\let\olditemize\itemize
\def\itemize{\olditemize\parskip0pt}

\begin{document}

\title{The \textsf{childdoc} Package}
\hypersetup{pdftitle={The childdoc Package}}
\author{Niklas Beisert\\[2ex]
  Institut f\"ur Theoretische Physik\\
  Eidgen\"ossische Technische Hochschule Z\"urich\\
  Wolfgang-Pauli-Strasse 27, 8093 Z\"urich, Switzerland\\[1ex]
  \href{mailto:nbeisert@itp.phys.ethz.ch}
  {\texttt{nbeisert@itp.phys.ethz.ch}}}
\hypersetup{pdfauthor={Niklas Beisert}}
\hypersetup{pdfsubject={Manual for the LaTeX2e Package childdoc}}
\date{30 December 2018, \textsf{v2.0}}
\maketitle

\begin{abstract}\noindent
\textsf{childdoc} is a \LaTeXe{} package
that enables the direct compilation
of document sections included by |\include|
to individual files.
\end{abstract}

\begingroup
\parskip0ex
\tableofcontents
\endgroup

%%%%%%%%%%%%%%%%%%%%%%%%%%%%%%%%%%%%%%%%%%%%%%%%%%%%%%%%%%%%%%%%%%%%%%%%%%%%%%%%
%%%%%%%%%%%%%%%%%%%%%%%%%%%%%%%%%%%%%%%%%%%%%%%%%%%%%%%%%%%%%%%%%%%%%%%%%%%%%%%%
\section{Introduction}

\LaTeX{} provides a mechanism to structure a large document (such as a book)
into a main file and several child files (containing the chapters)
using the |\include| command.
This mechanism is beneficial for documents
which span hundreds of pages in order to
make the source file(s) more manageable.
Moreover, compilation can be restricted to
selected child files by means of the |\includeonly| command.
The latter feature can be used to reduce the compilation time while editing
(this was significantly more useful in the earlier days of \LaTeX{})
or to generate a smaller document which is easier to navigate.
Another application of |\includeonly| is to generate
documents consisting of selected parts of the complete document.

However, there are a few drawbacks of the plain |\include| mechanism:
\begin{itemize}
\item
The child files cannot be compiled on their own,
they can only be compiled via the main file.
A naive editing environment
(such as a text editor with an option
to have the current file processed by \LaTeX)
may require one to switch to the main file before compiling;
attempting to compile the child file produces errors.
\item
The main file must be modified (each time)
to adjust the |\includeonly| command
to the present needs. This easily leaves the main file in a messy state.
\item
The generated document will always carry the filename
of the main document. This is inconvenient if
several child files are to be compiled and
to be kept for distribution.
\end{itemize}

The present package provides a simple interface
to make child files individually compilable by \LaTeX{}.
Compiling a child file then has the same effect as compiling
the main file with an |\includeonly| command
to select the appropriate child.
Moreover the generated document will carry the name of the child
rather than the main file.
This resolves all three above issues.

This feature is meant to make the editing of books,
thesis documents and lecture notes somewhat more convenient.
However, the package can also be used efficiently for
composing a series of documents (such as exercise sheets)
which are typically distributed individually.
It then assists the author in generating the individual documents
(potentially in different versions)
as well as a document containing the collected series.
Another application is in developing style files
or other kinds of included material
where compilation of the style file could redirect
to a sample or test file.

%%%%%%%%%%%%%%%%%%%%%%%%%%%%%%%%%%%%%%%%%%%%%%%%%%%%%%%%%%%%%%%%%%%%%%%%%%%%%%%%
%%%%%%%%%%%%%%%%%%%%%%%%%%%%%%%%%%%%%%%%%%%%%%%%%%%%%%%%%%%%%%%%%%%%%%%%%%%%%%%%
\section{Usage}

First of all, the package \textsf{childdoc} is \emph{not} a standard
\LaTeXe{} |.sty| style file! Therefore it needs to be invoked in
a non-standard way.

%%%%%%%%%%%%%%%%%%%%%%%%%%%%%%%%%%%%%%%%%%%%%%%%%%%%%%%%%%%%%%%%%%%%%%%%%%%%%%%%
\subsection{Included Files}
\label{sec:include}

%%%%%%%%%%%%%%%%%%%%%%%%%%%%%%%%%%%%%%%%
\DescribeMacro{\childdocmain}
To use the package, add the commands
\begin{center}
\begin{tabular}{l}
|\input{childdoc.def}|\\
|\childdocmain{}|\\
\end{tabular}
\end{center}
at the very top of the main \LaTeX{} file,
in particular \emph{before} the |\documentclass| statement!
The argument of |\childdocmain| should be left empty
(but it must be present).

%%%%%%%%%%%%%%%%%%%%%%%%%%%%%%%%%%%%%%%%
\DescribeMacro{\childdocof}
Furthermore, add the commands
\begin{center}
\begin{tabular}{l}
|\input{childdoc.def}|\\
|\childdocof{|\textit{main}|}|\\
\end{tabular}
\end{center}
at the top of every child file \textit{child}
which is included by |\include{|\textit{child}|}|
from within the main file
(or at least for those files to be compiled individually).
The argument \textit{main} must be the filename of the main file.

There are a couple of
considerations in setting up the main and child documents:

%%%%%%%%%%%%%%%%%%%%%%%%%%%%%%%%%%%%%%%%
\paragraph{Restrictions.}

Please note the following restrictions:
\begin{itemize}
\item
|\childdocmain| must be called with one argument \textit{main}
to ensure compatibility with earlier version of the package.
It must either be empty (|\childdocmain{}|)
or precisely match the filename of the main file in which it is specified.
See \secref{sec:detection} for further information.
\item
The filename \textit{main} must be specified without the |.tex| extension.
\item
The filename \textit{main} is case sensitive
(even in case-insensitive file systems)
due to internal string comparison.
\item
The argument \textit{main} should be fully expanded, it cannot be a macro.
\item
Subdirectories and special characters should be avoided in filenames.
\item
The command |\childdocmain{|\textit{main}|}| must be followed by a whitespace.
It should not be followed immediately by another command
or by a comment mark `|%|'.
This is because the \TeX{} parser reads the token immediately following
the argument of |\childdocmain| and puts it
at the beginning of every child section;
however, a white\-space is ignored.
\end{itemize}

%%%%%%%%%%%%%%%%%%%%%%%%%%%%%%%%%%%%%%%%
\paragraph{Content of Main File.}

It is advisable to place all content in the child files included by |\include|.
Any output contained in the main file will appear in all child documents
unless suppressed manually;
it cannot be suppressed automatically by the |\includeonly| directive
and thus should normally be avoided.
A method to include some content in the main file
by means of conditional processing is described in \secref{sec:conditional}.

%%%%%%%%%%%%%%%%%%%%%%%%%%%%%%%%%%%%%%%%
\paragraph{Page Numbering.}

When only a part of the document is compiled,
the appropriate numbering of pages
(as well as other status parameters)
is determined from the |.aux| files.
The latter contain information from previous passes.
However this information needs to propagate through
all intermediate child documents.
Therefore the page numbering in child documents may well
be inconsistent until the complete document is compiled at least once.

A useful (if unconventional) way to always ensure a consistent
page numbering is to restart the numbering in each child document
and denote the pages by `\textit{child}|.|\textit{page}'
where \textit{child} represents the chapter/section number of the child file.
This can be achieved by the command
|\numberwithin{page}{|\textit{child}|}|
of the \textsf{amsmath} package
where \textit{child} can be |chapter| or |section|
depending on the chosen structuring.
Alternatively, one can modify the macro |\thepage| appropriately
and reset the counter |page| at the start of each child file.

%%%%%%%%%%%%%%%%%%%%%%%%%%%%%%%%%%%%%%%%%%%%%%%%%%%%%%%%%%%%%%%%%%%%%%%%%%%%%%%%
\subsection{Conditional Processing}
\label{sec:conditional}

The package provides a mechanism to compile different versions
of a document. To customise the versions further some conditional processing
can come in handy to distinguish which version is being compiled.
The package provides two macros to describe the compilation context:

%%%%%%%%%%%%%%%%%%%%%%%%%%%%%%%%%%%%%%%%
\DescribeMacro{\ifchilddoc}
The conditional |\ifchilddoc| distinguishes between the compilation of
child documents and the main document:
%
\begin{center}
|\ifchilddoc |\textit{child-code}| |[|\||else |\textit{main-code}]| \||fi|
\end{center}

%%%%%%%%%%%%%%%%%%%%%%%%%%%%%%%%%%%%%%%%
\DescribeMacro{\childdocname}
\DescribeMacro{\childdocjob}
The macro |\childdocname| contains the filename (without extension)
of the main or child file being processed.
Note that |\childdocjob| will always contain the name of the main file.

%%%%%%%%%%%%%%%%%%%%%%%%%%%%%%%%%%%%%%%%
\paragraph{Title Page.}

Conditional processing can be used to include a title or banner page
in the main document when proper precautions are taken.
Importantly, the code in the main file should ensure that the page counter
(as well as other status parameters which are stored in the |.aux| files)
takes the same value after the conditional processing.
Otherwise the page numbers may take divergent values
depending on which part is compiled.

For example, a title page could be declared by:
%
\begin{center}
\begin{tabular}{l}
|\ifchilddoc\||else|\\
|\addtocounter{page}{-1}|\\
\textit{code for title page}\\
|\newpage|\\
|\||fi|
\end{tabular}
\end{center}
%
A banner page for the child documents can be generated by:
%
\begin{center}
\begin{tabular}{l}
|\ifchilddoc|\\
|\addtocounter{page}{-1}|\\
\textit{code for banner page}\\
|\newpage|\\
|\||fi|
\end{tabular}
\end{center}
%
Here one could write a message such as:
\begin{center}
|This is the part \childdocname{} of \childdocjob{}.|
\end{center}

%%%%%%%%%%%%%%%%%%%%%%%%%%%%%%%%%%%%%%%%%%%%%%%%%%%%%%%%%%%%%%%%%%%%%%%%%%%%%%%%
\subsection{Flags}
\label{sec:flags}

The package makes it easy to generate different versions
of the main or child documents.
To this end compilation flags can be defined
and assigned different default values.
They will be particularly useful in conjunction
with the forwarding mechanism described in \secref{sec:forward}.

For example, it may be useful to have a flag |\version|
which can be set to |draft| or |final|.
The document source will contain some conditional code
depending on the value of |\version|.
Suppose further, the flag should default to |final| for the main file
and to |draft| for child files
which is a natural assignment for editing the document.
This is achieved by placing the following code
in the preamble of the main document
(below the |\childdocmain| directive):
%
\begin{center}
\begin{tabular}{l}
|\ifchilddoc|\\
|\providecommand{\version}{draft}|\\
|\||else|\\
|\providecommand{\version}{final}|\\
|\||fi|
\end{tabular}
\end{center}
%
The definition by |\providecommand| makes sure
that previous definitions are not overwritten.
Further statements |\providecommand{\version}{...}|
can thus be added before the above code to override it.

For the main file, one might add a line
(between |\childdocmain| and the above block)
%
\begin{center}
|%\ifchilddoc\||else\providecommand{\version}{draft}\||fi|
\end{center}
%
which can be uncommented to produce a draft version.
Likewise one can add a line to the very top of a child file
(above the |\childdocof{|\textit{main}|}| directive)
%
\begin{center}
|%\providecommand{\version}{final}|
\end{center}
%
which can be uncommented to produce the final version of this child document.

%%%%%%%%%%%%%%%%%%%%%%%%%%%%%%%%%%%%%%%%%%%%%%%%%%%%%%%%%%%%%%%%%%%%%%%%%%%%%%%%
\subsection{Forwarding}
\label{sec:forward}

Different versions of the main or child documents
using compilation flags as described in \secref{sec:flags}
can be (permanently) stored in different files
for convenient compilation, viewing and distribution.
To this end, the package defines a command
to pass on compilation to a different file:

%%%%%%%%%%%%%%%%%%%%%%%%%%%%%%%%%%%%%%%%
\DescribeMacro{\childdocforward}
The command |\childdocforward| redirects processing to
another source file:
%
\begin{center}
\begin{tabular}{l}
|\input{childdoc.def}|\\
|\childdocforward[|\textit{main}|]{|\textit{dest}|}|\\
\end{tabular}
\end{center}
%
The argument \textit{dest} is the destination file
(without extension).
It should be the main file or one of the child files.
Note that further \textsf{childdoc} directives
such as |\childdocof| and |\childdocforward|
in the indicated file will be processed in this form.
The optional argument \textit{main}
passes on directly to the main file \textit{main}
while pretending to compile the child \textit{dest}.
This form behaves as if \textit{dest}
issues |\childdocof{|\textit{main}|}| right away,
and no further \textsf{childdoc} directives will be processed.

%%%%%%%%%%%%%%%%%%%%%%%%%%%%%%%%%%%%%%%%
\DescribeMacro{\...prefix}
In the alternative form |\childdocforwardprefix|,
%
\begin{center}
\begin{tabular}{l}
|\input{childdoc.def}|\\
|\childdocforwardprefix[|\textit{main}|]{|\textit{prefix}|}{|\textit{dest}|}|
\end{tabular}
\end{center}
%
the destination file is determined by a pattern
depending on the current file:
To make this work, the current file must be called
`{\textit{prefix}\hspace{0.2em}\textit{suffix}}'
with \textit{prefix} matching precisely the argument.
Processing is then passed on to the file
`{\textit{dest}\hspace{0.2em}\textit{suffix}}'.
Surely, the same effect is achieved by
directly specifying the
argument `{\textit{dest}\hspace{0.2em}\textit{suffix}}'
in the first form.
However, that requires to set up a different file
for each child. With the alternative form of the command
all these files can have exactly the same content
which simplifies setting them up and maintaining them.

For example, the following file |draft.tex|
with a compilation flag |\version| as described in \secref{sec:flags}
compiles the main document as a draft:
%
\begin{center}
\begin{tabular}{l}
|\def\version{draft}|\\
|\input{childdoc.def}|\\
|\childdocforward{|\textit{main}|}|
\end{tabular}
\end{center}
%
Likewise, the following files |final|\textit{nn}|.tex|
compile the final version of the child document
|child|\textit{nn}|.tex|:
%
\begin{center}
\begin{tabular}{l}
|\def\version{final}|\\
|\input{childdoc.def}|\\
|\childdocforwardprefix{final}{child}|
\end{tabular}
\end{center}
%

Note that when several versions of a main file and/or of each child file
are to be generated, it may be convenient to set up a |Makefile| or
shell script to automatise the process.

%%%%%%%%%%%%%%%%%%%%%%%%%%%%%%%%%%%%%%%%%%%%%%%%%%%%%%%%%%%%%%%%%%%%%%%%%%%%%%%%
\subsection{Command Line Processing}
\label{sec:commandline}

The effect of redirection files can also be achieved by invoking
the \LaTeX{} compiler with a more elaborate command line.
Most conveniently this should be done as part
of a shell script or a |Makefile|.

When using \textsf{childdoc} in the main file, the following
command lines effectively perform a redirection
(note that depending on the shell being used,
backslashes may have to be doubled: `|\|' $\to$ `|\\|'):
%
\begin{center}
|... -jobname "|\textit{target}|" |\\|"|[\textit{flags}]%
|\input{childdoc.def}\childdocforward[|\textit{main}|]{|\textit{dest}|}"|
\end{center}
%
Here \textit{target} is the name of the output file,
\textit{main} is the name of the main file
and \textit{dest} is the name of the main or child file to be processed
(all filenames without extensions).
The optional argument \textit{main} can be omitted
if \textit{main} matches \textit{dest}.
Optionally, compilation \textit{flags} can be defined via |\def| commands.
This command line makes the \TeX{} engine believe
it is compiling the file \textit{target}
whose content is specified as the latter parameter.
The provided code then forwards the processing to
\textit{main} or \textit{dest} as described in \secref{sec:forward}.

%%%%%%%%%%%%%%%%%%%%%%%%%%%%%%%%%%%%%%%%%%%%%%%%%%%%%%%%%%%%%%%%%%%%%%%%%%%%%%%%
\subsection{Include by Input}
\label{sec:input}

Including child documents by |\include| has some restrictions by design.
Most notably, the content of a child document always occupies
its own set of pages; pages cannot be shared between child documents.
Usually, this behaviour makes perfect sense
because each child document contain an essential part of the document.
However, in some situations it may be desirable to compose
a document from a collection of parts
without having mandatory page breaks between then.
For this case, the package
provides a mechanism to include parts
by |\input| which can also be processed individually.
However, by construction this mechanism
requires manual handling of the content to be output.

%%%%%%%%%%%%%%%%%%%%%%%%%%%%%%%%%%%%%%%%
\DescribeMacro{\ifchilddocmanual}
The main file should be prepared as usual, see \secref{sec:include}.
However, the document body must make a distinction
between processing of an individual part and of the main document, e.g.:
%
\begin{center}
\begin{tabular}{l}
|\ifchilddocmanual|\\
|\input{\childdocname}|\\
|\||else|\\
\textit{document body with }|\input{|\textit{part}|}|\\
|\||fi|
\end{tabular}
\end{center}
%
The conditional |\ifchilddocmanual| is true whenever
a part to be included by |\input| is being compiled,
and the name of the part is stored in |\childdocname|.

%%%%%%%%%%%%%%%%%%%%%%%%%%%%%%%%%%%%%%%%
\DescribeMacro{\childdocby}
Each part to be included by |\input| should start with:
%
\begin{center}
\begin{tabular}{l}
|\input{childdoc.def}|\\
|\childdocby{|\textit{main}|}|\\
\end{tabular}
\end{center}
%
The directive |\childdocby| is similar to |\childdocof|
described in \secref{sec:include},
but the subsequent selection of content must be done manually.
To that end, both |\ifchilddoc| and |\ifchilddocmanual|
will be true upon processing of a part,
and the name of the part is stored in |\childdocname|.
Note that |\jobname| will be set to the filename of the current part
so that each part receives an individual |.aux| file
that does not interfere with the |.aux| file(s) of the main document.
This behaviour can be altered by the alternative form
|\childdocby[*]{|\textit{main}|}| (with a non-empty optional argument)
which uses the |.aux| file of the main document
by setting |\jobname| to \textit{main}.

%%%%%%%%%%%%%%%%%%%%%%%%%%%%%%%%%%%%%%%%%%%%%%%%%%%%%%%%%%%%%%%%%%%%%%%%%%%%%%%%
\subsection{Driver Development}
\label{sec:driver}

The \textsf{childdoc} mechanism can also be use for the development
of definition files such as \LaTeX{} styles or classes.
This case differs from the above setup with multiple parts
included by |\include| in that no |\includeonly| should be invoked.
This can be achieved by starting the include file
(before |\ProvidesPackage|) with:
%
\begin{center}
\begin{tabular}{l}
|\input{childdoc.def}|\\
|\childdocforward{|\textit{main}|}|\\
\end{tabular}
\end{center}
%
or alternatively with:
%
\begin{center}
\begin{tabular}{l}
|\input{childdoc.def}|\\
|\childdocby{|\textit{main}|}|\\
\end{tabular}
\end{center}
%
Both forms have slightly different effects as described above.
The main file is prepared as usual, see \secref{sec:include}.

%%%%%%%%%%%%%%%%%%%%%%%%%%%%%%%%%%%%%%%%%%%%%%%%%%%%%%%%%%%%%%%%%%%%%%%%%%%%%%%%
\subsection{Legacy Detection}
\label{sec:detection}

The directive |\childdocmain| in the main file can detect
whether the complete document or merely a child is to be compiled
even without using the directive |\childdocof|.
This method is deprecated because it is less robust
and there is no compelling reason to use it;
it is merely provided for backward compatibility
and it may be removed in future versions.

If the detection mechanism is to be used,
it is mandatory to correctly specify
the filename of the main file as the argument of |\childdocmain|:
%
\begin{center}
\begin{tabular}{l}
|\input{childdoc.def}|\\
|\childdocmain{|\textit{main}|}|\\
\end{tabular}
\end{center}
%
If |\jobname| does not match the argument \textit{main} of |\childdocmain|,
it is assumed that |\jobname| points to the child file to be compiled.
When using |\childdocmain| with the main file specified as argument,
it suffices to start a child file
with just |\input{|\textit{main}|}|
without loading of the package and using |\childdocof|.
If instead all processing is done
with the appropriate \textsf{childdoc} directives,
the argument of \textit{main} of |\childdocmain| can be empty.

An alternative version of the command line processing described
in \secref{sec:commandline} using the detection mechanism reads:
%
\begin{center}
|... -jobname "|\textit{target}|" "|[\textit{flags}]%
[|\def\jobname{|\textit{dest}|}|]|\input{|\textit{main}|}"|
\end{center}

%%%%%%%%%%%%%%%%%%%%%%%%%%%%%%%%%%%%%%%%%%%%%%%%%%%%%%%%%%%%%%%%%%%%%%%%%%%%%%%%
\subsection{Manual Code}
\label{sec:manual}

In case one cannot be certain whether the definitions file |childdoc.def|
is installed on the target \TeX{} distribution
and one prefers not to ship it,
it is conceivable to paste a few relevant commands into the sources.

To that end, drop all statements |\input{childdoc.def}|
and perform the replacements as outlined below.
Instead of |\childdocmain{|\textit{main}|}| add the following code
to the top of the main file:
%
\begin{center}
\begin{tabular}{l}
|\||ifdefined\childdocname\endinput\||fi\newif\ifchilddoc|\\
|\edef\childdocname{\scantokens\expandafter{\jobname\noexpand}}|\\
|\def\childdocmain{|\textit{main}|}\||ifx\childdocmain\childdocname\||else|\\
|\childdoctrue\includeonly{\childdocname}\let\jobname\childdocmain\||fi|\\
\end{tabular}
\end{center}
%
Instead of |\childdocof{|\textit{main}|}| just include the main file
at the top of each child file:
%
\begin{center}
|\input{|\textit{main}|}|
\end{center}
%
A simple redirection |\childdocforward{|\textit{dest}|}| is achieved by:
%
\begin{center}
|\def\jobname{|\textit{dest}|}\input{\jobname}|
\end{center}
%
The redirection with prefix
|\childdocforwardprefix[|\textit{prefix}|]{|\textit{dest}|}|
is accomplished by:
%
\begin{center}
\begin{tabular}{l}
|{\edef\jobname{\scantokens\expandafter{\jobname\noexpand}}|\\
|\def\redirectjob |\textit{prefix}|#1~~~{\gdef\jobname{|\textit{dest}|#1}}|\\
|\expandafter\redirectjob\jobname~~~}\input{\jobname}|
\end{tabular}
\end{center}

In an alternative approach,
child documents can be compiled by a specific command line
without additional code or specific definitions:
%
\begin{center}
|... -jobname "|\textit{target}|" "|[\textit{flags}]%
|\includeonly{|\textit{dest}|}\input{|\textit{main}|}"|
\end{center}
%

%%%%%%%%%%%%%%%%%%%%%%%%%%%%%%%%%%%%%%%%%%%%%%%%%%%%%%%%%%%%%%%%%%%%%%%%%%%%%%%%
%%%%%%%%%%%%%%%%%%%%%%%%%%%%%%%%%%%%%%%%%%%%%%%%%%%%%%%%%%%%%%%%%%%%%%%%%%%%%%%%
\section{Information}

%%%%%%%%%%%%%%%%%%%%%%%%%%%%%%%%%%%%%%%%%%%%%%%%%%%%%%%%%%%%%%%%%%%%%%%%%%%%%%%%
\subsection{Copyright}

Copyright \copyright{} 2017--2018 Niklas Beisert

This work may be distributed and/or modified under the
conditions of the \LaTeX{} Project Public License, either version 1.3
of this license or (at your option) any later version.
The latest version of this license is in
  \url{http://www.latex-project.org/lppl.txt}
and version 1.3 or later is part of all distributions of \LaTeX{}
version 2005/12/01 or later.

This work has the LPPL maintenance status `maintained'.

The Current Maintainer of this work is Niklas Beisert.

This work consists of the files |README.txt|, |childdoc.ins| and |childdoc.dtx|
as well as the derived files |childdoc.def|, |cdocsamp.tex|
with |cdocsch1.tex|, |cdocsch2.tex|, |cdocspt3.tex|, |cdocspt4.tex|,
|cdocsdrf.tex|, |cdocsfn1.tex|, |cdocsfn2.tex|
as well as |childdoc.pdf|.

%%%%%%%%%%%%%%%%%%%%%%%%%%%%%%%%%%%%%%%%%%%%%%%%%%%%%%%%%%%%%%%%%%%%%%%%%%%%%%%%
\subsection{Files and Installation}

The package consists of the files:
%
\begin{center}
\begin{tabular}{ll}
    |README.txt|   & readme file \\
    |childdoc.ins| & installation file \\
    |childdoc.dtx| & source file \\
    |childdoc.def| & definition file \\
    |cdocsamp.tex| & sample main file \\
    |cdocsch1.tex| & sample include file \\
    |cdocsch2.tex| & sample include file \\
    |cdocspt3.tex| & sample part file \\
    |cdocspt4.tex| & sample part file \\
    |cdocsdrf.tex| & sample redirection file \\
    |cdocsfn1.tex| & sample redirection file \\
    |cdocsfn2.tex| & sample redirection file \\
    |childdoc.pdf| & manual
\end{tabular}
\end{center}
%
The distribution consists of the files
|README.txt|, |childdoc.ins| and |childdoc.dtx|.
%
\begin{itemize}
\item
Run (pdf)\LaTeX{} on |childdoc.dtx|
to compile the manual |childdoc.pdf| (this file).
\item
Run \LaTeX{} on |childdoc.ins| to create the definitions file |childdoc.def|
and the sample |cdocsamp.tex| with include files
|cdocsch1.tex|, |cdocsch2.tex|, |cdocspt3.tex|, |cdocspt4.tex|,
|cdocsdrf.tex|, |cdocsfn1.tex|, |cdocsfn2.tex|.
Then copy the file |childdoc.def| to an appropriate directory of your \LaTeX{}
distribution, e.g.\ \textit{texmf-root}|/tex/latex/childdoc|.
\end{itemize}

%%%%%%%%%%%%%%%%%%%%%%%%%%%%%%%%%%%%%%%%%%%%%%%%%%%%%%%%%%%%%%%%%%%%%%%%%%%%%%%%
\subsection{Related CTAN Packages}

There are several other packages which offer a similar functionality:
%
\begin{itemize}
\item
The packages
\href{http://ctan.org/pkg/docmute}{\textsf{docmute}},
\href{http://ctan.org/pkg/includex}{\textsf{includex}} and
\href{http://ctan.org/pkg/standalone}{\textsf{standalone}}
provide commands to include only the document body of
a child file thus allowing both files to be compiled individually.
\item
The packages \href{http://ctan.org/pkg/subdocs}{\textsf{subdocs}}
and \href{http://ctan.org/pkg/subfiles}{\textsf{subfiles}}
provide structures in which the main and child documents can be
encapsulated and allowing them to be compiled individually.
The inclusion mechanism is different from the conventional |\include|.
\item
The package \href{http://ctan.org/pkg/combine}{\textsf{combine}}
is an elaborate solution to combine several documents into one.
\end{itemize}
%
See also the CTAN topic \href{http://ctan.org/topic/subdocs}{\textsf{subdocs}}
for further related packages.
The present package differs from the above solutions in that
a document structure constructed with the conventional |\include| mechanism
just needs two extra commands at the top of every file
such that all constituent files can be compiled individually.

%%%%%%%%%%%%%%%%%%%%%%%%%%%%%%%%%%%%%%%%%%%%%%%%%%%%%%%%%%%%%%%%%%%%%%%%%%%%%%%%
%\subsection{Feature Suggestions}
%
%The following is a list of features which may be useful for future
%versions of this package:
%%
%\begin{itemize}
%\item
%\ldots
%\end{itemize}

%%%%%%%%%%%%%%%%%%%%%%%%%%%%%%%%%%%%%%%%%%%%%%%%%%%%%%%%%%%%%%%%%%%%%%%%%%%%%%%%
\subsection{Revision History}

%%%%%%%%%%%%%%%%%%%%%%%%%%%%%%%%%%%%%%%%
\paragraph{v2.0:} 2018/12/30

\begin{itemize}
\item
immediate forward processing
\item
added |\childdocby| mechanism
\item
manual restructured
\end{itemize}

%%%%%%%%%%%%%%%%%%%%%%%%%%%%%%%%%%%%%%%%
\paragraph{v1.6:} 2018/01/17

\begin{itemize}
\item
application for development of include files
\item
corrections to manual
\end{itemize}

%%%%%%%%%%%%%%%%%%%%%%%%%%%%%%%%%%%%%%%%
\paragraph{v1.5:} 2017/05/21

\begin{itemize}
\item
more complete structuring introduced
\item
|\childdocof| introduced
\item
|\childdoc| renamed to |\childdocmain|
\item
|\childredirect| renamed to |\childdocforward| and |\childdocforwardprefix|
and functionality expanded
\end{itemize}

%%%%%%%%%%%%%%%%%%%%%%%%%%%%%%%%%%%%%%%%
\paragraph{v1.0:} 2017/04/27

\begin{itemize}
\item
manual and install package
\item
first version published on CTAN
\end{itemize}

%%%%%%%%%%%%%%%%%%%%%%%%%%%%%%%%%%%%%%%%
\paragraph{v0.6:} 2017/04/26

\begin{itemize}
\item
redirection mechanism added
\end{itemize}

%%%%%%%%%%%%%%%%%%%%%%%%%%%%%%%%%%%%%%%%
\paragraph{v0.5:} 2017/04/26

\begin{itemize}
\item
functionality in definition file
\end{itemize}


%%%%%%%%%%%%%%%%%%%%%%%%%%%%%%%%%%%%%%%%%%%%%%%%%%%%%%%%%%%%%%%%%%%%%%%%%%%%%%%%
%%%%%%%%%%%%%%%%%%%%%%%%%%%%%%%%%%%%%%%%%%%%%%%%%%%%%%%%%%%%%%%%%%%%%%%%%%%%%%%%
%%%%%%%%%%%%%%%%%%%%%%%%%%%%%%%%%%%%%%%%%%%%%%%%%%%%%%%%%%%%%%%%%%%%%%%%%%%%%%%%
\appendix

\settowidth\MacroIndent{\rmfamily\scriptsize 000\ }

 \DocInput{childdoc.dtx}

\end{document}
%</driver>
% \fi
%
% %%%%%%%%%%%%%%%%%%%%%%%%%%%%%%%%%%%%%%%%%%%%%%%%%%%%%%%%%%%%%%%%%%%%%%%%%%%%%%
% %%%%%%%%%%%%%%%%%%%%%%%%%%%%%%%%%%%%%%%%%%%%%%%%%%%%%%%%%%%%%%%%%%%%%%%%%%%%%%
% \section{Sample}
%\iffalse
%<*samplemain>
%\fi
%
% The following presents a sample document
% with two chapters, two parts, a title page,
% a compile flag as well as three forwarding files to set the flag.
% It consists of eight |.tex| files:
% \begin{center}
% \begin{tabular}{ll}
% |cdocsamp.tex|&main file\\
% |cdocsch1.tex|&include file for chapter 1\\
% |cdocsch2.tex|&include file for chapter 2\\
% |cdocspt3.tex|&include file for part 3\\
% |cdocspt4.tex|&include file for part 4\\
% |cdocsdrf.tex|&forwarding file for main file in draft mode\\
% |cdocsfi1.tex|&forwarding file for final version of chapter 1\\
% |cdocsfi2.tex|&forwarding file for final version of chapter 2\\
% \end{tabular}
% \end{center}
% Each of the eight files can be compiled directly by the \LaTeX{} compiler.
%
% %%%%%%%%%%%%%%%%%%%%%%%%%%%%%%%%%%%%%%
% \paragraph{Main File.}
%
% The main file is called |cdocsamp.tex|.
%
% Load the \textsf{childdoc} definitions and
% declare the filename for the main document:
%    \begin{macrocode}
\input{childdoc.def}
\childdocmain{}
%    \end{macrocode}

% Optional override for |\version| flag:
%    \begin{macrocode}
%%\ifchilddoc\else\providecommand{\version}{draft}\fi
%    \end{macrocode}

% Define the default values for the |\version| flag
% (|final| for the main file and |draft| for childs):
%    \begin{macrocode}
\ifchilddoc
\providecommand{\version}{draft}
\else
\providecommand{\version}{final}
\fi
%    \end{macrocode}

% Load the standard document class:
%    \begin{macrocode}
\documentclass[12pt]{article}
%    \end{macrocode}

% Start the document body:
%    \begin{macrocode}
\begin{document}
%    \end{macrocode}

% Declare a title page.
% Print title, part of document being processed and version flag:
%    \begin{macrocode}
\addtocounter{page}{-1}
\begin{center}
{\LARGE\bfseries{}childdoc example\par}
\vspace{1cm}
\ifchilddoc
\ifchilddocmanual part\else chapter\fi:
`\childdocname' of `\childdocjob'\par
\else
main document: `\childdocjob'\par
\fi
version: \version\par
\end{center}
\newpage
%    \end{macrocode}

% Manually include selected file,
% otherwise process as usual:
%    \begin{macrocode}
\ifchilddocmanual
\section*{part `\childdocname'}
\input{\childdocname}
\else
%    \end{macrocode}

% Include the two chapters:
%    \begin{macrocode}
\include{cdocsch1}
\include{cdocsch2}
%    \end{macrocode}

% Include the two parts unless only chapters should be displayed:
%    \begin{macrocode}
\ifchilddoc\else
\section{part three}
\input{cdocspt3}
\section{part four}
\input{cdocspt4}
\fi
%    \end{macrocode}

% Process as usual until here:
%    \begin{macrocode}
\fi
%    \end{macrocode}

% End of document body:
%    \begin{macrocode}
\end{document}
%    \end{macrocode}
%\iffalse
%</samplemain>
%\fi
%
% %%%%%%%%%%%%%%%%%%%%%%%%%%%%%%%%%%%%%%
% \paragraph{Chapter Include Files.}
%
% The include files are called |cdocsch1.tex| and |cdocsch2.tex|.
%
%\iffalse
%<*samplechap1|samplechap2>
%\fi

% Optional override for |\version| flag:
%    \begin{macrocode}
%%\providecommand{\version}{final}
%    \end{macrocode}

% Include the main document:
%    \begin{macrocode}
\input{childdoc.def}
\childdocof{cdocsamp}
%    \end{macrocode}

%\iffalse
%</samplechap1|samplechap2>
%\fi
%
%\iffalse
%<*samplechap1>
%\fi
% Some text for chapter 1:
%    \begin{macrocode}
\section{one}
some text in chapter one
%    \end{macrocode}

%\iffalse
%</samplechap1>
%\fi
% Some text for chapter 2:
%\iffalse
%<*samplechap2>
%\fi
%    \begin{macrocode}
\section{two}
more text in chapter two
%    \end{macrocode}

%\iffalse
%</samplechap2>
%\fi
%
% %%%%%%%%%%%%%%%%%%%%%%%%%%%%%%%%%%%%%%
% \paragraph{Part Include Files.}
%
% The include files are called |cdocspt3.tex| and |cdocspt4.tex|.
%
%\iffalse
%<*samplepart3|samplepart4>
%\fi

% Optional override for |\version| flag:
%    \begin{macrocode}
%%\providecommand{\version}{final}
%    \end{macrocode}

% Include the main document:
%    \begin{macrocode}
\input{childdoc.def}
\childdocby{cdocsamp}
%    \end{macrocode}

%\iffalse
%</samplepart3|samplepart4>
%\fi
%
%\iffalse
%<*samplepart3>
%\fi
% Some text for part 3:
%    \begin{macrocode}
some text in part three
%    \end{macrocode}

%\iffalse
%</samplepart3>
%\fi
% Some text for part 4:
%\iffalse
%<*samplepart4>
%\fi
%    \begin{macrocode}
more text in part four
%    \end{macrocode}

%\iffalse
%</samplepart4>
%\fi
%
% %%%%%%%%%%%%%%%%%%%%%%%%%%%%%%%%%%%%%%
% \paragraph{Forwarding for a Complete Draft.}
%
% The following forwarding file |cdocsdrf.tex|
% compiles the main document in draft mode:
%\iffalse
%<*sampledraft>
%\fi
%    \begin{macrocode}
\def\version{draft}
\input{childdoc.def}
\childdocforward{cdocsamp}
%    \end{macrocode}

%\iffalse
%</sampledraft>
%\fi
%
% %%%%%%%%%%%%%%%%%%%%%%%%%%%%%%%%%%%%%%
% \paragraph{Forwarding for Final Version of the Chapters.}
%
% The following forwarding files |cdocsfn1.tex| and |cdocsfn2.tex|
% (with identical content)
% compile the final versions of the child documents
% |cdocsch1.tex| and |cdocsch2.tex|, respectively:
%\iffalse
%<*samplefinal>
%\fi
%    \begin{macrocode}
\def\version{final}
\input{childdoc.def}
\childdocforwardprefix[cdocsamp]{cdocsfn}{cdocsch}
%    \end{macrocode}

%\iffalse
%</samplefinal>
%\fi
%
% %%%%%%%%%%%%%%%%%%%%%%%%%%%%%%%%%%%%%%
% \paragraph{Command Line Processing.}
%
% The following three command lines generate the output files
% |cdocscld|, |cdocscl1| and |cdocscl2|
% which should be identical to
% |cdocsdrf|, |cdocsch1| and |cdocsfn2|, respectively:
% \begin{center}
% \begin{tabular}{l}
% |latex -jobname cdocscld \|\\
% |  "\def\version{draft}\input{childdoc.def}\childdocforward{cdocsamp}"|\\
% |latex -jobname cdocscl1 \|\\
% |  "\input{childdoc.def}\childdocforward[cdocsamp]{cdocsch1}"|\\
% |latex -jobname cdocscl2 \|\\
% |  "\def\version{final}\input{childdoc.def}\childdocforward{cdocsch2}"|
% \end{tabular}
% \end{center}
% Note that the trailing backslash on each first line
% merely continues the input to the second line
% (for convenient cut ant paste).
% Furthermore, the command |latex| can be replaced by any
% of its alternative versions such as |pdflatex|.
%
% %%%%%%%%%%%%%%%%%%%%%%%%%%%%%%%%%%%%%%%%%%%%%%%%%%%%%%%%%%%%%%%%%%%%%%%%%%%%%%
% %%%%%%%%%%%%%%%%%%%%%%%%%%%%%%%%%%%%%%%%%%%%%%%%%%%%%%%%%%%%%%%%%%%%%%%%%%%%%%
% \section{Implementation}
%\iffalse
%<*package>
%\fi
%
% This section describes the definitions file |childdoc.def|.

% The definitions cannot be loaded using |\usepackage| or |\RequirePackage|
% which has a mechanism to prevent loading a style file more than once.
% When loading the definitions by means of |\input|
% multiple instances have to be prevented manually:
%\iffalse
%This code needs to be before the `\ProvidesFile' directive
%which is defined at the beginning of this file.
%Therefore it is also placed there and commented out here.
%</package>
%<*discard>
%\fi
%    \begin{macrocode}
\ifdefined\childdocmain\endinput\fi
%    \end{macrocode}
%\iffalse
%</discard>
%<*package>
%\fi
%
% \macro{\ifchilddoc}
% \macro{\ifchilddocmanual}
% The conditional |\ifchilddoc| tells whether a
% child (true) or main (false) document is being compiled.
% The conditional |\ifchilddocmanual| tells whether
% the |\includeonly| mechanism is used (false) or
% the selection of child files must be performed manually (true).
% The definitions initialise to false:
%    \begin{macrocode}
\newif\ifchilddoc
\newif\ifchilddocmanual
%    \end{macrocode}

% \macro{\childdocname}
% \macro{\childdocjob}
% The macro |\childdocname| stores the name of the main document
% to be compiled. The macro |\childdocjob| stores the name of
% the document on which the \LaTeX{} compiler was originally invoked.
% The content of |\jobname| cannot be compared
% to filenames specified in the source due to different catcodes.
% The following code rescans |\jobname|, stores the result
% in |\childdocname| and saves a copy in |\childdocjob|:
%    \begin{macrocode}
\edef\childdocname{\scantokens\expandafter{\jobname\noexpand}}
\let\childdocjob\childdocname
%    \end{macrocode}

% \macro{\childdocdisable}
% The macro |\childdocdisable| prevents the main file
% from being processed more than once.
% At this stage, the main document command |\childdocmain|
% is assumed to be called once again where it should do nothing.
% Any subsequent call to it should prevent
% a secondary processing of the main document
% It overwrites the forwarding commands
% |\childdocof| and |\childdocforward|
% with empty macros to prevent further inclusions of the main document:
%    \begin{macrocode}
\newcommand{\childdocdisable}
{
  \renewcommand{\childdocmain}[1]{\renewcommand{\childdocmain}[1]{\endinput}}
  \renewcommand{\childdocof}[1]{}
  \renewcommand{\childdocby}[2][]{}
  \renewcommand{\childdocforward}[2][]{}
  \renewcommand{\childdocdisable}{}
}
%    \end{macrocode}

% \macro{\childdocmain}
% The macro |\childdocmain| is to be called at the top of the main file
% with nothing or the main filename (without extension) as argument.
% First, it breaks loops.
% If the argument is not empty and does not match |\childdocname|
% (which is set by the first inclusion of |childdoc.def|),
% |\ifchilddoc| is set to true, |\includeonly| is applied to the child file
% and |\jobname| is set to the main file
% (for proper handling of |.aux| files):
%    \begin{macrocode}
\newcommand{\childdocmain}[1]
{
  \childdocdisable\childdocmain{}
  \if?#1?\else
    \begingroup
      \def\childdoctmp{#1}
      \ifx\childdoctmp\childdocname
        \def\childdoctmp{}
      \else
        \def\childdoctmp
        {
          \childdoctrue
          \includeonly{\childdocname}
          \def\childdocjob{#1}
          \def\jobname{#1}
        }
      \fi
      \expandafter
    \endgroup
    \childdoctmp
  \fi
}
%    \end{macrocode}

% \macro{\childdocof}
% The command |\childdocof| redirects
% compilation to the main file |#1|.
%    \begin{macrocode}
\newcommand{\childdocof}[1]
{
  \childdocdisable
  \childdoctrue
  \includeonly{\childdocname}
  \def\jobname{#1}
  \def\childdocjob{#1}
  \input{#1}
}
%    \end{macrocode}

% \macro{\childdocby}
% The command |\childdocby| ....
%    \begin{macrocode}
\newcommand{\childdocby}[2][]
{
  \childdocdisable
  \childdoctrue
  \childdocmanualtrue
  \if?#1?\else
    \def\jobname{#2}
  \fi
  \def\childdocjob{#2}
  \input{#2}
  \endinput
}
%    \end{macrocode}

% \macro{\childdocforward}
% The command |\childdocforward| redirects
% compilation to the main file or
% (if the optional argument is given) a child file.
% Parameters are set as if the main file
% or a child file starting with |\childdocof| was compiled.
% Then compilation is handed over to the main file:
%    \begin{macrocode}
\newcommand{\childdocforward}[2][]
{
  \begingroup
    \if?#1?
      \def\childdoctmp
      {
        \def\childdocname{#2}
        \def\childdocjob{#2}
        \def\jobname{#2}
        \input{#2}
        \endinput
      }
    \else
      \def\childdoctmp
      {
        \childdocdisable
        \def\childdocname{#2}
        \childdoctrue
        \includeonly{#2}
        \def\childdocjob{#1}
        \def\jobname{#1}
        \input{#1}
        \endinput
      }
    \fi
    \expandafter
  \endgroup
  \childdoctmp
}
%    \end{macrocode}

% \macro{\childdocforwardprefix}
% The command |\childdocforwardprefix| redirects
% compilation to the main or a child file by means of a pattern.
% The prefix |#1| in the current filename is replaced by |#2|
% and the suffix of the current filename is kept
% (it is assumed that the filename does not contain the substring `|~~~|'
% which is used as a delimiter).
% Compilation is handed over to the new file by |\childdocforward|:
%    \begin{macrocode}
\newcommand{\childdocforwardprefix}[3][]
{
  \begingroup
    \def\childdocextract #2##1~~~{\def\childdoctmp{\childdocforward[#1]{#3##1}}}
    \expandafter\childdocextract\childdocname~~~
    \expandafter
  \endgroup
  \childdoctmp
}
%    \end{macrocode}

% \macro{\childdoc}
% The deprecated macro |\childdoc| is a legacy version of |\childdocmain|:
%    \begin{macrocode}
\newcommand{\childdoc}{\childdocmain}
%    \end{macrocode}

% \macro{\childdocredirect}
% The deprecated macro |\childdocredirect| is a legacy version
% of |\childdocforward| and |\childdocforwardprefix|:
%    \begin{macrocode}
\newcommand{\childdocredirect}[2][]
{
  \begingroup
    \if?#1?
      \def\childdoctmp{\childdocforward{#2}}
    \else
      \def\childdoctmp{\childdocforwardprefix{#1}{#2}}
    \fi
    \expandafter
  \endgroup
  \childdoctmp
}
%    \end{macrocode}

%\iffalse
%</package>
%\fi
%
\endinput
|\\
|\childdocof{|\textit{main}|}|\\
\end{tabular}
\end{center}
at the top of every child file \textit{child}
which is included by |\include{|\textit{child}|}|
from within the main file
(or at least for those files to be compiled individually).
The argument \textit{main} must be the filename of the main file.

There are a couple of
considerations in setting up the main and child documents:

%%%%%%%%%%%%%%%%%%%%%%%%%%%%%%%%%%%%%%%%
\paragraph{Restrictions.}

Please note the following restrictions:
\begin{itemize}
\item
|\childdocmain| must be called with one argument \textit{main}
to ensure compatibility with earlier version of the package.
It must either be empty (|\childdocmain{}|)
or precisely match the filename of the main file in which it is specified.
See \secref{sec:detection} for further information.
\item
The filename \textit{main} must be specified without the |.tex| extension.
\item
The filename \textit{main} is case sensitive
(even in case-insensitive file systems)
due to internal string comparison.
\item
The argument \textit{main} should be fully expanded, it cannot be a macro.
\item
Subdirectories and special characters should be avoided in filenames.
\item
The command |\childdocmain{|\textit{main}|}| must be followed by a whitespace.
It should not be followed immediately by another command
or by a comment mark `|%|'.
This is because the \TeX{} parser reads the token immediately following
the argument of |\childdocmain| and puts it
at the beginning of every child section;
however, a white\-space is ignored.
\end{itemize}

%%%%%%%%%%%%%%%%%%%%%%%%%%%%%%%%%%%%%%%%
\paragraph{Content of Main File.}

It is advisable to place all content in the child files included by |\include|.
Any output contained in the main file will appear in all child documents
unless suppressed manually;
it cannot be suppressed automatically by the |\includeonly| directive
and thus should normally be avoided.
A method to include some content in the main file
by means of conditional processing is described in \secref{sec:conditional}.

%%%%%%%%%%%%%%%%%%%%%%%%%%%%%%%%%%%%%%%%
\paragraph{Page Numbering.}

When only a part of the document is compiled,
the appropriate numbering of pages
(as well as other status parameters)
is determined from the |.aux| files.
The latter contain information from previous passes.
However this information needs to propagate through
all intermediate child documents.
Therefore the page numbering in child documents may well
be inconsistent until the complete document is compiled at least once.

A useful (if unconventional) way to always ensure a consistent
page numbering is to restart the numbering in each child document
and denote the pages by `\textit{child}|.|\textit{page}'
where \textit{child} represents the chapter/section number of the child file.
This can be achieved by the command
|\numberwithin{page}{|\textit{child}|}|
of the \textsf{amsmath} package
where \textit{child} can be |chapter| or |section|
depending on the chosen structuring.
Alternatively, one can modify the macro |\thepage| appropriately
and reset the counter |page| at the start of each child file.

%%%%%%%%%%%%%%%%%%%%%%%%%%%%%%%%%%%%%%%%%%%%%%%%%%%%%%%%%%%%%%%%%%%%%%%%%%%%%%%%
\subsection{Conditional Processing}
\label{sec:conditional}

The package provides a mechanism to compile different versions
of a document. To customise the versions further some conditional processing
can come in handy to distinguish which version is being compiled.
The package provides two macros to describe the compilation context:

%%%%%%%%%%%%%%%%%%%%%%%%%%%%%%%%%%%%%%%%
\DescribeMacro{\ifchilddoc}
The conditional |\ifchilddoc| distinguishes between the compilation of
child documents and the main document:
%
\begin{center}
|\ifchilddoc |\textit{child-code}| |[|\||else |\textit{main-code}]| \||fi|
\end{center}

%%%%%%%%%%%%%%%%%%%%%%%%%%%%%%%%%%%%%%%%
\DescribeMacro{\childdocname}
\DescribeMacro{\childdocjob}
The macro |\childdocname| contains the filename (without extension)
of the main or child file being processed.
Note that |\childdocjob| will always contain the name of the main file.

%%%%%%%%%%%%%%%%%%%%%%%%%%%%%%%%%%%%%%%%
\paragraph{Title Page.}

Conditional processing can be used to include a title or banner page
in the main document when proper precautions are taken.
Importantly, the code in the main file should ensure that the page counter
(as well as other status parameters which are stored in the |.aux| files)
takes the same value after the conditional processing.
Otherwise the page numbers may take divergent values
depending on which part is compiled.

For example, a title page could be declared by:
%
\begin{center}
\begin{tabular}{l}
|\ifchilddoc\||else|\\
|\addtocounter{page}{-1}|\\
\textit{code for title page}\\
|\newpage|\\
|\||fi|
\end{tabular}
\end{center}
%
A banner page for the child documents can be generated by:
%
\begin{center}
\begin{tabular}{l}
|\ifchilddoc|\\
|\addtocounter{page}{-1}|\\
\textit{code for banner page}\\
|\newpage|\\
|\||fi|
\end{tabular}
\end{center}
%
Here one could write a message such as:
\begin{center}
|This is the part \childdocname{} of \childdocjob{}.|
\end{center}

%%%%%%%%%%%%%%%%%%%%%%%%%%%%%%%%%%%%%%%%%%%%%%%%%%%%%%%%%%%%%%%%%%%%%%%%%%%%%%%%
\subsection{Flags}
\label{sec:flags}

The package makes it easy to generate different versions
of the main or child documents.
To this end compilation flags can be defined
and assigned different default values.
They will be particularly useful in conjunction
with the forwarding mechanism described in \secref{sec:forward}.

For example, it may be useful to have a flag |\version|
which can be set to |draft| or |final|.
The document source will contain some conditional code
depending on the value of |\version|.
Suppose further, the flag should default to |final| for the main file
and to |draft| for child files
which is a natural assignment for editing the document.
This is achieved by placing the following code
in the preamble of the main document
(below the |\childdocmain| directive):
%
\begin{center}
\begin{tabular}{l}
|\ifchilddoc|\\
|\providecommand{\version}{draft}|\\
|\||else|\\
|\providecommand{\version}{final}|\\
|\||fi|
\end{tabular}
\end{center}
%
The definition by |\providecommand| makes sure
that previous definitions are not overwritten.
Further statements |\providecommand{\version}{...}|
can thus be added before the above code to override it.

For the main file, one might add a line
(between |\childdocmain| and the above block)
%
\begin{center}
|%\ifchilddoc\||else\providecommand{\version}{draft}\||fi|
\end{center}
%
which can be uncommented to produce a draft version.
Likewise one can add a line to the very top of a child file
(above the |\childdocof{|\textit{main}|}| directive)
%
\begin{center}
|%\providecommand{\version}{final}|
\end{center}
%
which can be uncommented to produce the final version of this child document.

%%%%%%%%%%%%%%%%%%%%%%%%%%%%%%%%%%%%%%%%%%%%%%%%%%%%%%%%%%%%%%%%%%%%%%%%%%%%%%%%
\subsection{Forwarding}
\label{sec:forward}

Different versions of the main or child documents
using compilation flags as described in \secref{sec:flags}
can be (permanently) stored in different files
for convenient compilation, viewing and distribution.
To this end, the package defines a command
to pass on compilation to a different file:

%%%%%%%%%%%%%%%%%%%%%%%%%%%%%%%%%%%%%%%%
\DescribeMacro{\childdocforward}
The command |\childdocforward| redirects processing to
another source file:
%
\begin{center}
\begin{tabular}{l}
|% \iffalse
%
% childdoc.dtx Copyright (C) 2017-2018 Niklas Beisert
%
% This work may be distributed and/or modified under the
% conditions of the LaTeX Project Public License, either version 1.3
% of this license or (at your option) any later version.
% The latest version of this license is in
%   http://www.latex-project.org/lppl.txt
% and version 1.3 or later is part of all distributions of LaTeX
% version 2005/12/01 or later.
%
% This work has the LPPL maintenance status `maintained'.
%
% The Current Maintainer of this work is Niklas Beisert.
%
% This work consists of the files childdoc.dtx and childdoc.ins
% and the derived files childdoc.def and cdocsamp.tex with
% cdocsch1.tex, cdocsch2.tex, cdocsdrf.tex, cdocsfn1.tex, cdocsfn2.tex.
%
%<package>\ifdefined\childdocmain\endinput\fi
%<package>\ProvidesFile{childdoc.def}[2018/12/30 v2.0 child document driver]
%<samplemain>\ProvidesFile{cdocsamp.tex}[2018/12/30 v2.0 sample for childdoc]
%<*driver>
%\ProvidesFile{childdoc.drv}[2018/12/30 v2.0 childdoc reference manual file]
\PassOptionsToClass{10pt,a4paper}{article}
\documentclass{ltxdoc}

\usepackage[margin=35mm]{geometry}
\usepackage{hyperref}
\usepackage{hyperxmp}
\usepackage[usenames]{color}

\hypersetup{colorlinks=true}
\hypersetup{pdfstartview=FitH}
\hypersetup{pdfpagemode=UseNone}
\hypersetup{pdfsource={}}
\hypersetup{pdflang={en-UK}}
\hypersetup{pdfcopyright={Copyright 2017-2018 Niklas Beisert.
  This work may be distributed and/or modified under the
  conditions of the LaTeX Project Public License, either version 1.3
  of this license or (at your option) any later version.}}
\hypersetup{pdflicenseurl={http://www.latex-project.org/lppl.txt}}
\hypersetup{pdfcontactaddress={ETH Zurich, ITP, HIT K,
  Wolfgang-Pauli-Strasse 27}}
\hypersetup{pdfcontactpostcode={8093}}
\hypersetup{pdfcontactcity={Zurich}}
\hypersetup{pdfcontactcountry={Switzerland}}
\hypersetup{pdfcontactemail={nbeisert@itp.phys.ethz.ch}}
\hypersetup{pdfcontacturl={http://people.phys.ethz.ch/\xmptilde nbeisert/}}

\newcommand{\secref}[1]{\hyperref[#1]{section \ref*{#1}}}

\parskip1ex
\parindent0pt
\let\olditemize\itemize
\def\itemize{\olditemize\parskip0pt}

\begin{document}

\title{The \textsf{childdoc} Package}
\hypersetup{pdftitle={The childdoc Package}}
\author{Niklas Beisert\\[2ex]
  Institut f\"ur Theoretische Physik\\
  Eidgen\"ossische Technische Hochschule Z\"urich\\
  Wolfgang-Pauli-Strasse 27, 8093 Z\"urich, Switzerland\\[1ex]
  \href{mailto:nbeisert@itp.phys.ethz.ch}
  {\texttt{nbeisert@itp.phys.ethz.ch}}}
\hypersetup{pdfauthor={Niklas Beisert}}
\hypersetup{pdfsubject={Manual for the LaTeX2e Package childdoc}}
\date{30 December 2018, \textsf{v2.0}}
\maketitle

\begin{abstract}\noindent
\textsf{childdoc} is a \LaTeXe{} package
that enables the direct compilation
of document sections included by |\include|
to individual files.
\end{abstract}

\begingroup
\parskip0ex
\tableofcontents
\endgroup

%%%%%%%%%%%%%%%%%%%%%%%%%%%%%%%%%%%%%%%%%%%%%%%%%%%%%%%%%%%%%%%%%%%%%%%%%%%%%%%%
%%%%%%%%%%%%%%%%%%%%%%%%%%%%%%%%%%%%%%%%%%%%%%%%%%%%%%%%%%%%%%%%%%%%%%%%%%%%%%%%
\section{Introduction}

\LaTeX{} provides a mechanism to structure a large document (such as a book)
into a main file and several child files (containing the chapters)
using the |\include| command.
This mechanism is beneficial for documents
which span hundreds of pages in order to
make the source file(s) more manageable.
Moreover, compilation can be restricted to
selected child files by means of the |\includeonly| command.
The latter feature can be used to reduce the compilation time while editing
(this was significantly more useful in the earlier days of \LaTeX{})
or to generate a smaller document which is easier to navigate.
Another application of |\includeonly| is to generate
documents consisting of selected parts of the complete document.

However, there are a few drawbacks of the plain |\include| mechanism:
\begin{itemize}
\item
The child files cannot be compiled on their own,
they can only be compiled via the main file.
A naive editing environment
(such as a text editor with an option
to have the current file processed by \LaTeX)
may require one to switch to the main file before compiling;
attempting to compile the child file produces errors.
\item
The main file must be modified (each time)
to adjust the |\includeonly| command
to the present needs. This easily leaves the main file in a messy state.
\item
The generated document will always carry the filename
of the main document. This is inconvenient if
several child files are to be compiled and
to be kept for distribution.
\end{itemize}

The present package provides a simple interface
to make child files individually compilable by \LaTeX{}.
Compiling a child file then has the same effect as compiling
the main file with an |\includeonly| command
to select the appropriate child.
Moreover the generated document will carry the name of the child
rather than the main file.
This resolves all three above issues.

This feature is meant to make the editing of books,
thesis documents and lecture notes somewhat more convenient.
However, the package can also be used efficiently for
composing a series of documents (such as exercise sheets)
which are typically distributed individually.
It then assists the author in generating the individual documents
(potentially in different versions)
as well as a document containing the collected series.
Another application is in developing style files
or other kinds of included material
where compilation of the style file could redirect
to a sample or test file.

%%%%%%%%%%%%%%%%%%%%%%%%%%%%%%%%%%%%%%%%%%%%%%%%%%%%%%%%%%%%%%%%%%%%%%%%%%%%%%%%
%%%%%%%%%%%%%%%%%%%%%%%%%%%%%%%%%%%%%%%%%%%%%%%%%%%%%%%%%%%%%%%%%%%%%%%%%%%%%%%%
\section{Usage}

First of all, the package \textsf{childdoc} is \emph{not} a standard
\LaTeXe{} |.sty| style file! Therefore it needs to be invoked in
a non-standard way.

%%%%%%%%%%%%%%%%%%%%%%%%%%%%%%%%%%%%%%%%%%%%%%%%%%%%%%%%%%%%%%%%%%%%%%%%%%%%%%%%
\subsection{Included Files}
\label{sec:include}

%%%%%%%%%%%%%%%%%%%%%%%%%%%%%%%%%%%%%%%%
\DescribeMacro{\childdocmain}
To use the package, add the commands
\begin{center}
\begin{tabular}{l}
|\input{childdoc.def}|\\
|\childdocmain{}|\\
\end{tabular}
\end{center}
at the very top of the main \LaTeX{} file,
in particular \emph{before} the |\documentclass| statement!
The argument of |\childdocmain| should be left empty
(but it must be present).

%%%%%%%%%%%%%%%%%%%%%%%%%%%%%%%%%%%%%%%%
\DescribeMacro{\childdocof}
Furthermore, add the commands
\begin{center}
\begin{tabular}{l}
|\input{childdoc.def}|\\
|\childdocof{|\textit{main}|}|\\
\end{tabular}
\end{center}
at the top of every child file \textit{child}
which is included by |\include{|\textit{child}|}|
from within the main file
(or at least for those files to be compiled individually).
The argument \textit{main} must be the filename of the main file.

There are a couple of
considerations in setting up the main and child documents:

%%%%%%%%%%%%%%%%%%%%%%%%%%%%%%%%%%%%%%%%
\paragraph{Restrictions.}

Please note the following restrictions:
\begin{itemize}
\item
|\childdocmain| must be called with one argument \textit{main}
to ensure compatibility with earlier version of the package.
It must either be empty (|\childdocmain{}|)
or precisely match the filename of the main file in which it is specified.
See \secref{sec:detection} for further information.
\item
The filename \textit{main} must be specified without the |.tex| extension.
\item
The filename \textit{main} is case sensitive
(even in case-insensitive file systems)
due to internal string comparison.
\item
The argument \textit{main} should be fully expanded, it cannot be a macro.
\item
Subdirectories and special characters should be avoided in filenames.
\item
The command |\childdocmain{|\textit{main}|}| must be followed by a whitespace.
It should not be followed immediately by another command
or by a comment mark `|%|'.
This is because the \TeX{} parser reads the token immediately following
the argument of |\childdocmain| and puts it
at the beginning of every child section;
however, a white\-space is ignored.
\end{itemize}

%%%%%%%%%%%%%%%%%%%%%%%%%%%%%%%%%%%%%%%%
\paragraph{Content of Main File.}

It is advisable to place all content in the child files included by |\include|.
Any output contained in the main file will appear in all child documents
unless suppressed manually;
it cannot be suppressed automatically by the |\includeonly| directive
and thus should normally be avoided.
A method to include some content in the main file
by means of conditional processing is described in \secref{sec:conditional}.

%%%%%%%%%%%%%%%%%%%%%%%%%%%%%%%%%%%%%%%%
\paragraph{Page Numbering.}

When only a part of the document is compiled,
the appropriate numbering of pages
(as well as other status parameters)
is determined from the |.aux| files.
The latter contain information from previous passes.
However this information needs to propagate through
all intermediate child documents.
Therefore the page numbering in child documents may well
be inconsistent until the complete document is compiled at least once.

A useful (if unconventional) way to always ensure a consistent
page numbering is to restart the numbering in each child document
and denote the pages by `\textit{child}|.|\textit{page}'
where \textit{child} represents the chapter/section number of the child file.
This can be achieved by the command
|\numberwithin{page}{|\textit{child}|}|
of the \textsf{amsmath} package
where \textit{child} can be |chapter| or |section|
depending on the chosen structuring.
Alternatively, one can modify the macro |\thepage| appropriately
and reset the counter |page| at the start of each child file.

%%%%%%%%%%%%%%%%%%%%%%%%%%%%%%%%%%%%%%%%%%%%%%%%%%%%%%%%%%%%%%%%%%%%%%%%%%%%%%%%
\subsection{Conditional Processing}
\label{sec:conditional}

The package provides a mechanism to compile different versions
of a document. To customise the versions further some conditional processing
can come in handy to distinguish which version is being compiled.
The package provides two macros to describe the compilation context:

%%%%%%%%%%%%%%%%%%%%%%%%%%%%%%%%%%%%%%%%
\DescribeMacro{\ifchilddoc}
The conditional |\ifchilddoc| distinguishes between the compilation of
child documents and the main document:
%
\begin{center}
|\ifchilddoc |\textit{child-code}| |[|\||else |\textit{main-code}]| \||fi|
\end{center}

%%%%%%%%%%%%%%%%%%%%%%%%%%%%%%%%%%%%%%%%
\DescribeMacro{\childdocname}
\DescribeMacro{\childdocjob}
The macro |\childdocname| contains the filename (without extension)
of the main or child file being processed.
Note that |\childdocjob| will always contain the name of the main file.

%%%%%%%%%%%%%%%%%%%%%%%%%%%%%%%%%%%%%%%%
\paragraph{Title Page.}

Conditional processing can be used to include a title or banner page
in the main document when proper precautions are taken.
Importantly, the code in the main file should ensure that the page counter
(as well as other status parameters which are stored in the |.aux| files)
takes the same value after the conditional processing.
Otherwise the page numbers may take divergent values
depending on which part is compiled.

For example, a title page could be declared by:
%
\begin{center}
\begin{tabular}{l}
|\ifchilddoc\||else|\\
|\addtocounter{page}{-1}|\\
\textit{code for title page}\\
|\newpage|\\
|\||fi|
\end{tabular}
\end{center}
%
A banner page for the child documents can be generated by:
%
\begin{center}
\begin{tabular}{l}
|\ifchilddoc|\\
|\addtocounter{page}{-1}|\\
\textit{code for banner page}\\
|\newpage|\\
|\||fi|
\end{tabular}
\end{center}
%
Here one could write a message such as:
\begin{center}
|This is the part \childdocname{} of \childdocjob{}.|
\end{center}

%%%%%%%%%%%%%%%%%%%%%%%%%%%%%%%%%%%%%%%%%%%%%%%%%%%%%%%%%%%%%%%%%%%%%%%%%%%%%%%%
\subsection{Flags}
\label{sec:flags}

The package makes it easy to generate different versions
of the main or child documents.
To this end compilation flags can be defined
and assigned different default values.
They will be particularly useful in conjunction
with the forwarding mechanism described in \secref{sec:forward}.

For example, it may be useful to have a flag |\version|
which can be set to |draft| or |final|.
The document source will contain some conditional code
depending on the value of |\version|.
Suppose further, the flag should default to |final| for the main file
and to |draft| for child files
which is a natural assignment for editing the document.
This is achieved by placing the following code
in the preamble of the main document
(below the |\childdocmain| directive):
%
\begin{center}
\begin{tabular}{l}
|\ifchilddoc|\\
|\providecommand{\version}{draft}|\\
|\||else|\\
|\providecommand{\version}{final}|\\
|\||fi|
\end{tabular}
\end{center}
%
The definition by |\providecommand| makes sure
that previous definitions are not overwritten.
Further statements |\providecommand{\version}{...}|
can thus be added before the above code to override it.

For the main file, one might add a line
(between |\childdocmain| and the above block)
%
\begin{center}
|%\ifchilddoc\||else\providecommand{\version}{draft}\||fi|
\end{center}
%
which can be uncommented to produce a draft version.
Likewise one can add a line to the very top of a child file
(above the |\childdocof{|\textit{main}|}| directive)
%
\begin{center}
|%\providecommand{\version}{final}|
\end{center}
%
which can be uncommented to produce the final version of this child document.

%%%%%%%%%%%%%%%%%%%%%%%%%%%%%%%%%%%%%%%%%%%%%%%%%%%%%%%%%%%%%%%%%%%%%%%%%%%%%%%%
\subsection{Forwarding}
\label{sec:forward}

Different versions of the main or child documents
using compilation flags as described in \secref{sec:flags}
can be (permanently) stored in different files
for convenient compilation, viewing and distribution.
To this end, the package defines a command
to pass on compilation to a different file:

%%%%%%%%%%%%%%%%%%%%%%%%%%%%%%%%%%%%%%%%
\DescribeMacro{\childdocforward}
The command |\childdocforward| redirects processing to
another source file:
%
\begin{center}
\begin{tabular}{l}
|\input{childdoc.def}|\\
|\childdocforward[|\textit{main}|]{|\textit{dest}|}|\\
\end{tabular}
\end{center}
%
The argument \textit{dest} is the destination file
(without extension).
It should be the main file or one of the child files.
Note that further \textsf{childdoc} directives
such as |\childdocof| and |\childdocforward|
in the indicated file will be processed in this form.
The optional argument \textit{main}
passes on directly to the main file \textit{main}
while pretending to compile the child \textit{dest}.
This form behaves as if \textit{dest}
issues |\childdocof{|\textit{main}|}| right away,
and no further \textsf{childdoc} directives will be processed.

%%%%%%%%%%%%%%%%%%%%%%%%%%%%%%%%%%%%%%%%
\DescribeMacro{\...prefix}
In the alternative form |\childdocforwardprefix|,
%
\begin{center}
\begin{tabular}{l}
|\input{childdoc.def}|\\
|\childdocforwardprefix[|\textit{main}|]{|\textit{prefix}|}{|\textit{dest}|}|
\end{tabular}
\end{center}
%
the destination file is determined by a pattern
depending on the current file:
To make this work, the current file must be called
`{\textit{prefix}\hspace{0.2em}\textit{suffix}}'
with \textit{prefix} matching precisely the argument.
Processing is then passed on to the file
`{\textit{dest}\hspace{0.2em}\textit{suffix}}'.
Surely, the same effect is achieved by
directly specifying the
argument `{\textit{dest}\hspace{0.2em}\textit{suffix}}'
in the first form.
However, that requires to set up a different file
for each child. With the alternative form of the command
all these files can have exactly the same content
which simplifies setting them up and maintaining them.

For example, the following file |draft.tex|
with a compilation flag |\version| as described in \secref{sec:flags}
compiles the main document as a draft:
%
\begin{center}
\begin{tabular}{l}
|\def\version{draft}|\\
|\input{childdoc.def}|\\
|\childdocforward{|\textit{main}|}|
\end{tabular}
\end{center}
%
Likewise, the following files |final|\textit{nn}|.tex|
compile the final version of the child document
|child|\textit{nn}|.tex|:
%
\begin{center}
\begin{tabular}{l}
|\def\version{final}|\\
|\input{childdoc.def}|\\
|\childdocforwardprefix{final}{child}|
\end{tabular}
\end{center}
%

Note that when several versions of a main file and/or of each child file
are to be generated, it may be convenient to set up a |Makefile| or
shell script to automatise the process.

%%%%%%%%%%%%%%%%%%%%%%%%%%%%%%%%%%%%%%%%%%%%%%%%%%%%%%%%%%%%%%%%%%%%%%%%%%%%%%%%
\subsection{Command Line Processing}
\label{sec:commandline}

The effect of redirection files can also be achieved by invoking
the \LaTeX{} compiler with a more elaborate command line.
Most conveniently this should be done as part
of a shell script or a |Makefile|.

When using \textsf{childdoc} in the main file, the following
command lines effectively perform a redirection
(note that depending on the shell being used,
backslashes may have to be doubled: `|\|' $\to$ `|\\|'):
%
\begin{center}
|... -jobname "|\textit{target}|" |\\|"|[\textit{flags}]%
|\input{childdoc.def}\childdocforward[|\textit{main}|]{|\textit{dest}|}"|
\end{center}
%
Here \textit{target} is the name of the output file,
\textit{main} is the name of the main file
and \textit{dest} is the name of the main or child file to be processed
(all filenames without extensions).
The optional argument \textit{main} can be omitted
if \textit{main} matches \textit{dest}.
Optionally, compilation \textit{flags} can be defined via |\def| commands.
This command line makes the \TeX{} engine believe
it is compiling the file \textit{target}
whose content is specified as the latter parameter.
The provided code then forwards the processing to
\textit{main} or \textit{dest} as described in \secref{sec:forward}.

%%%%%%%%%%%%%%%%%%%%%%%%%%%%%%%%%%%%%%%%%%%%%%%%%%%%%%%%%%%%%%%%%%%%%%%%%%%%%%%%
\subsection{Include by Input}
\label{sec:input}

Including child documents by |\include| has some restrictions by design.
Most notably, the content of a child document always occupies
its own set of pages; pages cannot be shared between child documents.
Usually, this behaviour makes perfect sense
because each child document contain an essential part of the document.
However, in some situations it may be desirable to compose
a document from a collection of parts
without having mandatory page breaks between then.
For this case, the package
provides a mechanism to include parts
by |\input| which can also be processed individually.
However, by construction this mechanism
requires manual handling of the content to be output.

%%%%%%%%%%%%%%%%%%%%%%%%%%%%%%%%%%%%%%%%
\DescribeMacro{\ifchilddocmanual}
The main file should be prepared as usual, see \secref{sec:include}.
However, the document body must make a distinction
between processing of an individual part and of the main document, e.g.:
%
\begin{center}
\begin{tabular}{l}
|\ifchilddocmanual|\\
|\input{\childdocname}|\\
|\||else|\\
\textit{document body with }|\input{|\textit{part}|}|\\
|\||fi|
\end{tabular}
\end{center}
%
The conditional |\ifchilddocmanual| is true whenever
a part to be included by |\input| is being compiled,
and the name of the part is stored in |\childdocname|.

%%%%%%%%%%%%%%%%%%%%%%%%%%%%%%%%%%%%%%%%
\DescribeMacro{\childdocby}
Each part to be included by |\input| should start with:
%
\begin{center}
\begin{tabular}{l}
|\input{childdoc.def}|\\
|\childdocby{|\textit{main}|}|\\
\end{tabular}
\end{center}
%
The directive |\childdocby| is similar to |\childdocof|
described in \secref{sec:include},
but the subsequent selection of content must be done manually.
To that end, both |\ifchilddoc| and |\ifchilddocmanual|
will be true upon processing of a part,
and the name of the part is stored in |\childdocname|.
Note that |\jobname| will be set to the filename of the current part
so that each part receives an individual |.aux| file
that does not interfere with the |.aux| file(s) of the main document.
This behaviour can be altered by the alternative form
|\childdocby[*]{|\textit{main}|}| (with a non-empty optional argument)
which uses the |.aux| file of the main document
by setting |\jobname| to \textit{main}.

%%%%%%%%%%%%%%%%%%%%%%%%%%%%%%%%%%%%%%%%%%%%%%%%%%%%%%%%%%%%%%%%%%%%%%%%%%%%%%%%
\subsection{Driver Development}
\label{sec:driver}

The \textsf{childdoc} mechanism can also be use for the development
of definition files such as \LaTeX{} styles or classes.
This case differs from the above setup with multiple parts
included by |\include| in that no |\includeonly| should be invoked.
This can be achieved by starting the include file
(before |\ProvidesPackage|) with:
%
\begin{center}
\begin{tabular}{l}
|\input{childdoc.def}|\\
|\childdocforward{|\textit{main}|}|\\
\end{tabular}
\end{center}
%
or alternatively with:
%
\begin{center}
\begin{tabular}{l}
|\input{childdoc.def}|\\
|\childdocby{|\textit{main}|}|\\
\end{tabular}
\end{center}
%
Both forms have slightly different effects as described above.
The main file is prepared as usual, see \secref{sec:include}.

%%%%%%%%%%%%%%%%%%%%%%%%%%%%%%%%%%%%%%%%%%%%%%%%%%%%%%%%%%%%%%%%%%%%%%%%%%%%%%%%
\subsection{Legacy Detection}
\label{sec:detection}

The directive |\childdocmain| in the main file can detect
whether the complete document or merely a child is to be compiled
even without using the directive |\childdocof|.
This method is deprecated because it is less robust
and there is no compelling reason to use it;
it is merely provided for backward compatibility
and it may be removed in future versions.

If the detection mechanism is to be used,
it is mandatory to correctly specify
the filename of the main file as the argument of |\childdocmain|:
%
\begin{center}
\begin{tabular}{l}
|\input{childdoc.def}|\\
|\childdocmain{|\textit{main}|}|\\
\end{tabular}
\end{center}
%
If |\jobname| does not match the argument \textit{main} of |\childdocmain|,
it is assumed that |\jobname| points to the child file to be compiled.
When using |\childdocmain| with the main file specified as argument,
it suffices to start a child file
with just |\input{|\textit{main}|}|
without loading of the package and using |\childdocof|.
If instead all processing is done
with the appropriate \textsf{childdoc} directives,
the argument of \textit{main} of |\childdocmain| can be empty.

An alternative version of the command line processing described
in \secref{sec:commandline} using the detection mechanism reads:
%
\begin{center}
|... -jobname "|\textit{target}|" "|[\textit{flags}]%
[|\def\jobname{|\textit{dest}|}|]|\input{|\textit{main}|}"|
\end{center}

%%%%%%%%%%%%%%%%%%%%%%%%%%%%%%%%%%%%%%%%%%%%%%%%%%%%%%%%%%%%%%%%%%%%%%%%%%%%%%%%
\subsection{Manual Code}
\label{sec:manual}

In case one cannot be certain whether the definitions file |childdoc.def|
is installed on the target \TeX{} distribution
and one prefers not to ship it,
it is conceivable to paste a few relevant commands into the sources.

To that end, drop all statements |\input{childdoc.def}|
and perform the replacements as outlined below.
Instead of |\childdocmain{|\textit{main}|}| add the following code
to the top of the main file:
%
\begin{center}
\begin{tabular}{l}
|\||ifdefined\childdocname\endinput\||fi\newif\ifchilddoc|\\
|\edef\childdocname{\scantokens\expandafter{\jobname\noexpand}}|\\
|\def\childdocmain{|\textit{main}|}\||ifx\childdocmain\childdocname\||else|\\
|\childdoctrue\includeonly{\childdocname}\let\jobname\childdocmain\||fi|\\
\end{tabular}
\end{center}
%
Instead of |\childdocof{|\textit{main}|}| just include the main file
at the top of each child file:
%
\begin{center}
|\input{|\textit{main}|}|
\end{center}
%
A simple redirection |\childdocforward{|\textit{dest}|}| is achieved by:
%
\begin{center}
|\def\jobname{|\textit{dest}|}\input{\jobname}|
\end{center}
%
The redirection with prefix
|\childdocforwardprefix[|\textit{prefix}|]{|\textit{dest}|}|
is accomplished by:
%
\begin{center}
\begin{tabular}{l}
|{\edef\jobname{\scantokens\expandafter{\jobname\noexpand}}|\\
|\def\redirectjob |\textit{prefix}|#1~~~{\gdef\jobname{|\textit{dest}|#1}}|\\
|\expandafter\redirectjob\jobname~~~}\input{\jobname}|
\end{tabular}
\end{center}

In an alternative approach,
child documents can be compiled by a specific command line
without additional code or specific definitions:
%
\begin{center}
|... -jobname "|\textit{target}|" "|[\textit{flags}]%
|\includeonly{|\textit{dest}|}\input{|\textit{main}|}"|
\end{center}
%

%%%%%%%%%%%%%%%%%%%%%%%%%%%%%%%%%%%%%%%%%%%%%%%%%%%%%%%%%%%%%%%%%%%%%%%%%%%%%%%%
%%%%%%%%%%%%%%%%%%%%%%%%%%%%%%%%%%%%%%%%%%%%%%%%%%%%%%%%%%%%%%%%%%%%%%%%%%%%%%%%
\section{Information}

%%%%%%%%%%%%%%%%%%%%%%%%%%%%%%%%%%%%%%%%%%%%%%%%%%%%%%%%%%%%%%%%%%%%%%%%%%%%%%%%
\subsection{Copyright}

Copyright \copyright{} 2017--2018 Niklas Beisert

This work may be distributed and/or modified under the
conditions of the \LaTeX{} Project Public License, either version 1.3
of this license or (at your option) any later version.
The latest version of this license is in
  \url{http://www.latex-project.org/lppl.txt}
and version 1.3 or later is part of all distributions of \LaTeX{}
version 2005/12/01 or later.

This work has the LPPL maintenance status `maintained'.

The Current Maintainer of this work is Niklas Beisert.

This work consists of the files |README.txt|, |childdoc.ins| and |childdoc.dtx|
as well as the derived files |childdoc.def|, |cdocsamp.tex|
with |cdocsch1.tex|, |cdocsch2.tex|, |cdocspt3.tex|, |cdocspt4.tex|,
|cdocsdrf.tex|, |cdocsfn1.tex|, |cdocsfn2.tex|
as well as |childdoc.pdf|.

%%%%%%%%%%%%%%%%%%%%%%%%%%%%%%%%%%%%%%%%%%%%%%%%%%%%%%%%%%%%%%%%%%%%%%%%%%%%%%%%
\subsection{Files and Installation}

The package consists of the files:
%
\begin{center}
\begin{tabular}{ll}
    |README.txt|   & readme file \\
    |childdoc.ins| & installation file \\
    |childdoc.dtx| & source file \\
    |childdoc.def| & definition file \\
    |cdocsamp.tex| & sample main file \\
    |cdocsch1.tex| & sample include file \\
    |cdocsch2.tex| & sample include file \\
    |cdocspt3.tex| & sample part file \\
    |cdocspt4.tex| & sample part file \\
    |cdocsdrf.tex| & sample redirection file \\
    |cdocsfn1.tex| & sample redirection file \\
    |cdocsfn2.tex| & sample redirection file \\
    |childdoc.pdf| & manual
\end{tabular}
\end{center}
%
The distribution consists of the files
|README.txt|, |childdoc.ins| and |childdoc.dtx|.
%
\begin{itemize}
\item
Run (pdf)\LaTeX{} on |childdoc.dtx|
to compile the manual |childdoc.pdf| (this file).
\item
Run \LaTeX{} on |childdoc.ins| to create the definitions file |childdoc.def|
and the sample |cdocsamp.tex| with include files
|cdocsch1.tex|, |cdocsch2.tex|, |cdocspt3.tex|, |cdocspt4.tex|,
|cdocsdrf.tex|, |cdocsfn1.tex|, |cdocsfn2.tex|.
Then copy the file |childdoc.def| to an appropriate directory of your \LaTeX{}
distribution, e.g.\ \textit{texmf-root}|/tex/latex/childdoc|.
\end{itemize}

%%%%%%%%%%%%%%%%%%%%%%%%%%%%%%%%%%%%%%%%%%%%%%%%%%%%%%%%%%%%%%%%%%%%%%%%%%%%%%%%
\subsection{Related CTAN Packages}

There are several other packages which offer a similar functionality:
%
\begin{itemize}
\item
The packages
\href{http://ctan.org/pkg/docmute}{\textsf{docmute}},
\href{http://ctan.org/pkg/includex}{\textsf{includex}} and
\href{http://ctan.org/pkg/standalone}{\textsf{standalone}}
provide commands to include only the document body of
a child file thus allowing both files to be compiled individually.
\item
The packages \href{http://ctan.org/pkg/subdocs}{\textsf{subdocs}}
and \href{http://ctan.org/pkg/subfiles}{\textsf{subfiles}}
provide structures in which the main and child documents can be
encapsulated and allowing them to be compiled individually.
The inclusion mechanism is different from the conventional |\include|.
\item
The package \href{http://ctan.org/pkg/combine}{\textsf{combine}}
is an elaborate solution to combine several documents into one.
\end{itemize}
%
See also the CTAN topic \href{http://ctan.org/topic/subdocs}{\textsf{subdocs}}
for further related packages.
The present package differs from the above solutions in that
a document structure constructed with the conventional |\include| mechanism
just needs two extra commands at the top of every file
such that all constituent files can be compiled individually.

%%%%%%%%%%%%%%%%%%%%%%%%%%%%%%%%%%%%%%%%%%%%%%%%%%%%%%%%%%%%%%%%%%%%%%%%%%%%%%%%
%\subsection{Feature Suggestions}
%
%The following is a list of features which may be useful for future
%versions of this package:
%%
%\begin{itemize}
%\item
%\ldots
%\end{itemize}

%%%%%%%%%%%%%%%%%%%%%%%%%%%%%%%%%%%%%%%%%%%%%%%%%%%%%%%%%%%%%%%%%%%%%%%%%%%%%%%%
\subsection{Revision History}

%%%%%%%%%%%%%%%%%%%%%%%%%%%%%%%%%%%%%%%%
\paragraph{v2.0:} 2018/12/30

\begin{itemize}
\item
immediate forward processing
\item
added |\childdocby| mechanism
\item
manual restructured
\end{itemize}

%%%%%%%%%%%%%%%%%%%%%%%%%%%%%%%%%%%%%%%%
\paragraph{v1.6:} 2018/01/17

\begin{itemize}
\item
application for development of include files
\item
corrections to manual
\end{itemize}

%%%%%%%%%%%%%%%%%%%%%%%%%%%%%%%%%%%%%%%%
\paragraph{v1.5:} 2017/05/21

\begin{itemize}
\item
more complete structuring introduced
\item
|\childdocof| introduced
\item
|\childdoc| renamed to |\childdocmain|
\item
|\childredirect| renamed to |\childdocforward| and |\childdocforwardprefix|
and functionality expanded
\end{itemize}

%%%%%%%%%%%%%%%%%%%%%%%%%%%%%%%%%%%%%%%%
\paragraph{v1.0:} 2017/04/27

\begin{itemize}
\item
manual and install package
\item
first version published on CTAN
\end{itemize}

%%%%%%%%%%%%%%%%%%%%%%%%%%%%%%%%%%%%%%%%
\paragraph{v0.6:} 2017/04/26

\begin{itemize}
\item
redirection mechanism added
\end{itemize}

%%%%%%%%%%%%%%%%%%%%%%%%%%%%%%%%%%%%%%%%
\paragraph{v0.5:} 2017/04/26

\begin{itemize}
\item
functionality in definition file
\end{itemize}


%%%%%%%%%%%%%%%%%%%%%%%%%%%%%%%%%%%%%%%%%%%%%%%%%%%%%%%%%%%%%%%%%%%%%%%%%%%%%%%%
%%%%%%%%%%%%%%%%%%%%%%%%%%%%%%%%%%%%%%%%%%%%%%%%%%%%%%%%%%%%%%%%%%%%%%%%%%%%%%%%
%%%%%%%%%%%%%%%%%%%%%%%%%%%%%%%%%%%%%%%%%%%%%%%%%%%%%%%%%%%%%%%%%%%%%%%%%%%%%%%%
\appendix

\settowidth\MacroIndent{\rmfamily\scriptsize 000\ }

 \DocInput{childdoc.dtx}

\end{document}
%</driver>
% \fi
%
% %%%%%%%%%%%%%%%%%%%%%%%%%%%%%%%%%%%%%%%%%%%%%%%%%%%%%%%%%%%%%%%%%%%%%%%%%%%%%%
% %%%%%%%%%%%%%%%%%%%%%%%%%%%%%%%%%%%%%%%%%%%%%%%%%%%%%%%%%%%%%%%%%%%%%%%%%%%%%%
% \section{Sample}
%\iffalse
%<*samplemain>
%\fi
%
% The following presents a sample document
% with two chapters, two parts, a title page,
% a compile flag as well as three forwarding files to set the flag.
% It consists of eight |.tex| files:
% \begin{center}
% \begin{tabular}{ll}
% |cdocsamp.tex|&main file\\
% |cdocsch1.tex|&include file for chapter 1\\
% |cdocsch2.tex|&include file for chapter 2\\
% |cdocspt3.tex|&include file for part 3\\
% |cdocspt4.tex|&include file for part 4\\
% |cdocsdrf.tex|&forwarding file for main file in draft mode\\
% |cdocsfi1.tex|&forwarding file for final version of chapter 1\\
% |cdocsfi2.tex|&forwarding file for final version of chapter 2\\
% \end{tabular}
% \end{center}
% Each of the eight files can be compiled directly by the \LaTeX{} compiler.
%
% %%%%%%%%%%%%%%%%%%%%%%%%%%%%%%%%%%%%%%
% \paragraph{Main File.}
%
% The main file is called |cdocsamp.tex|.
%
% Load the \textsf{childdoc} definitions and
% declare the filename for the main document:
%    \begin{macrocode}
\input{childdoc.def}
\childdocmain{}
%    \end{macrocode}

% Optional override for |\version| flag:
%    \begin{macrocode}
%%\ifchilddoc\else\providecommand{\version}{draft}\fi
%    \end{macrocode}

% Define the default values for the |\version| flag
% (|final| for the main file and |draft| for childs):
%    \begin{macrocode}
\ifchilddoc
\providecommand{\version}{draft}
\else
\providecommand{\version}{final}
\fi
%    \end{macrocode}

% Load the standard document class:
%    \begin{macrocode}
\documentclass[12pt]{article}
%    \end{macrocode}

% Start the document body:
%    \begin{macrocode}
\begin{document}
%    \end{macrocode}

% Declare a title page.
% Print title, part of document being processed and version flag:
%    \begin{macrocode}
\addtocounter{page}{-1}
\begin{center}
{\LARGE\bfseries{}childdoc example\par}
\vspace{1cm}
\ifchilddoc
\ifchilddocmanual part\else chapter\fi:
`\childdocname' of `\childdocjob'\par
\else
main document: `\childdocjob'\par
\fi
version: \version\par
\end{center}
\newpage
%    \end{macrocode}

% Manually include selected file,
% otherwise process as usual:
%    \begin{macrocode}
\ifchilddocmanual
\section*{part `\childdocname'}
\input{\childdocname}
\else
%    \end{macrocode}

% Include the two chapters:
%    \begin{macrocode}
\include{cdocsch1}
\include{cdocsch2}
%    \end{macrocode}

% Include the two parts unless only chapters should be displayed:
%    \begin{macrocode}
\ifchilddoc\else
\section{part three}
\input{cdocspt3}
\section{part four}
\input{cdocspt4}
\fi
%    \end{macrocode}

% Process as usual until here:
%    \begin{macrocode}
\fi
%    \end{macrocode}

% End of document body:
%    \begin{macrocode}
\end{document}
%    \end{macrocode}
%\iffalse
%</samplemain>
%\fi
%
% %%%%%%%%%%%%%%%%%%%%%%%%%%%%%%%%%%%%%%
% \paragraph{Chapter Include Files.}
%
% The include files are called |cdocsch1.tex| and |cdocsch2.tex|.
%
%\iffalse
%<*samplechap1|samplechap2>
%\fi

% Optional override for |\version| flag:
%    \begin{macrocode}
%%\providecommand{\version}{final}
%    \end{macrocode}

% Include the main document:
%    \begin{macrocode}
\input{childdoc.def}
\childdocof{cdocsamp}
%    \end{macrocode}

%\iffalse
%</samplechap1|samplechap2>
%\fi
%
%\iffalse
%<*samplechap1>
%\fi
% Some text for chapter 1:
%    \begin{macrocode}
\section{one}
some text in chapter one
%    \end{macrocode}

%\iffalse
%</samplechap1>
%\fi
% Some text for chapter 2:
%\iffalse
%<*samplechap2>
%\fi
%    \begin{macrocode}
\section{two}
more text in chapter two
%    \end{macrocode}

%\iffalse
%</samplechap2>
%\fi
%
% %%%%%%%%%%%%%%%%%%%%%%%%%%%%%%%%%%%%%%
% \paragraph{Part Include Files.}
%
% The include files are called |cdocspt3.tex| and |cdocspt4.tex|.
%
%\iffalse
%<*samplepart3|samplepart4>
%\fi

% Optional override for |\version| flag:
%    \begin{macrocode}
%%\providecommand{\version}{final}
%    \end{macrocode}

% Include the main document:
%    \begin{macrocode}
\input{childdoc.def}
\childdocby{cdocsamp}
%    \end{macrocode}

%\iffalse
%</samplepart3|samplepart4>
%\fi
%
%\iffalse
%<*samplepart3>
%\fi
% Some text for part 3:
%    \begin{macrocode}
some text in part three
%    \end{macrocode}

%\iffalse
%</samplepart3>
%\fi
% Some text for part 4:
%\iffalse
%<*samplepart4>
%\fi
%    \begin{macrocode}
more text in part four
%    \end{macrocode}

%\iffalse
%</samplepart4>
%\fi
%
% %%%%%%%%%%%%%%%%%%%%%%%%%%%%%%%%%%%%%%
% \paragraph{Forwarding for a Complete Draft.}
%
% The following forwarding file |cdocsdrf.tex|
% compiles the main document in draft mode:
%\iffalse
%<*sampledraft>
%\fi
%    \begin{macrocode}
\def\version{draft}
\input{childdoc.def}
\childdocforward{cdocsamp}
%    \end{macrocode}

%\iffalse
%</sampledraft>
%\fi
%
% %%%%%%%%%%%%%%%%%%%%%%%%%%%%%%%%%%%%%%
% \paragraph{Forwarding for Final Version of the Chapters.}
%
% The following forwarding files |cdocsfn1.tex| and |cdocsfn2.tex|
% (with identical content)
% compile the final versions of the child documents
% |cdocsch1.tex| and |cdocsch2.tex|, respectively:
%\iffalse
%<*samplefinal>
%\fi
%    \begin{macrocode}
\def\version{final}
\input{childdoc.def}
\childdocforwardprefix[cdocsamp]{cdocsfn}{cdocsch}
%    \end{macrocode}

%\iffalse
%</samplefinal>
%\fi
%
% %%%%%%%%%%%%%%%%%%%%%%%%%%%%%%%%%%%%%%
% \paragraph{Command Line Processing.}
%
% The following three command lines generate the output files
% |cdocscld|, |cdocscl1| and |cdocscl2|
% which should be identical to
% |cdocsdrf|, |cdocsch1| and |cdocsfn2|, respectively:
% \begin{center}
% \begin{tabular}{l}
% |latex -jobname cdocscld \|\\
% |  "\def\version{draft}\input{childdoc.def}\childdocforward{cdocsamp}"|\\
% |latex -jobname cdocscl1 \|\\
% |  "\input{childdoc.def}\childdocforward[cdocsamp]{cdocsch1}"|\\
% |latex -jobname cdocscl2 \|\\
% |  "\def\version{final}\input{childdoc.def}\childdocforward{cdocsch2}"|
% \end{tabular}
% \end{center}
% Note that the trailing backslash on each first line
% merely continues the input to the second line
% (for convenient cut ant paste).
% Furthermore, the command |latex| can be replaced by any
% of its alternative versions such as |pdflatex|.
%
% %%%%%%%%%%%%%%%%%%%%%%%%%%%%%%%%%%%%%%%%%%%%%%%%%%%%%%%%%%%%%%%%%%%%%%%%%%%%%%
% %%%%%%%%%%%%%%%%%%%%%%%%%%%%%%%%%%%%%%%%%%%%%%%%%%%%%%%%%%%%%%%%%%%%%%%%%%%%%%
% \section{Implementation}
%\iffalse
%<*package>
%\fi
%
% This section describes the definitions file |childdoc.def|.

% The definitions cannot be loaded using |\usepackage| or |\RequirePackage|
% which has a mechanism to prevent loading a style file more than once.
% When loading the definitions by means of |\input|
% multiple instances have to be prevented manually:
%\iffalse
%This code needs to be before the `\ProvidesFile' directive
%which is defined at the beginning of this file.
%Therefore it is also placed there and commented out here.
%</package>
%<*discard>
%\fi
%    \begin{macrocode}
\ifdefined\childdocmain\endinput\fi
%    \end{macrocode}
%\iffalse
%</discard>
%<*package>
%\fi
%
% \macro{\ifchilddoc}
% \macro{\ifchilddocmanual}
% The conditional |\ifchilddoc| tells whether a
% child (true) or main (false) document is being compiled.
% The conditional |\ifchilddocmanual| tells whether
% the |\includeonly| mechanism is used (false) or
% the selection of child files must be performed manually (true).
% The definitions initialise to false:
%    \begin{macrocode}
\newif\ifchilddoc
\newif\ifchilddocmanual
%    \end{macrocode}

% \macro{\childdocname}
% \macro{\childdocjob}
% The macro |\childdocname| stores the name of the main document
% to be compiled. The macro |\childdocjob| stores the name of
% the document on which the \LaTeX{} compiler was originally invoked.
% The content of |\jobname| cannot be compared
% to filenames specified in the source due to different catcodes.
% The following code rescans |\jobname|, stores the result
% in |\childdocname| and saves a copy in |\childdocjob|:
%    \begin{macrocode}
\edef\childdocname{\scantokens\expandafter{\jobname\noexpand}}
\let\childdocjob\childdocname
%    \end{macrocode}

% \macro{\childdocdisable}
% The macro |\childdocdisable| prevents the main file
% from being processed more than once.
% At this stage, the main document command |\childdocmain|
% is assumed to be called once again where it should do nothing.
% Any subsequent call to it should prevent
% a secondary processing of the main document
% It overwrites the forwarding commands
% |\childdocof| and |\childdocforward|
% with empty macros to prevent further inclusions of the main document:
%    \begin{macrocode}
\newcommand{\childdocdisable}
{
  \renewcommand{\childdocmain}[1]{\renewcommand{\childdocmain}[1]{\endinput}}
  \renewcommand{\childdocof}[1]{}
  \renewcommand{\childdocby}[2][]{}
  \renewcommand{\childdocforward}[2][]{}
  \renewcommand{\childdocdisable}{}
}
%    \end{macrocode}

% \macro{\childdocmain}
% The macro |\childdocmain| is to be called at the top of the main file
% with nothing or the main filename (without extension) as argument.
% First, it breaks loops.
% If the argument is not empty and does not match |\childdocname|
% (which is set by the first inclusion of |childdoc.def|),
% |\ifchilddoc| is set to true, |\includeonly| is applied to the child file
% and |\jobname| is set to the main file
% (for proper handling of |.aux| files):
%    \begin{macrocode}
\newcommand{\childdocmain}[1]
{
  \childdocdisable\childdocmain{}
  \if?#1?\else
    \begingroup
      \def\childdoctmp{#1}
      \ifx\childdoctmp\childdocname
        \def\childdoctmp{}
      \else
        \def\childdoctmp
        {
          \childdoctrue
          \includeonly{\childdocname}
          \def\childdocjob{#1}
          \def\jobname{#1}
        }
      \fi
      \expandafter
    \endgroup
    \childdoctmp
  \fi
}
%    \end{macrocode}

% \macro{\childdocof}
% The command |\childdocof| redirects
% compilation to the main file |#1|.
%    \begin{macrocode}
\newcommand{\childdocof}[1]
{
  \childdocdisable
  \childdoctrue
  \includeonly{\childdocname}
  \def\jobname{#1}
  \def\childdocjob{#1}
  \input{#1}
}
%    \end{macrocode}

% \macro{\childdocby}
% The command |\childdocby| ....
%    \begin{macrocode}
\newcommand{\childdocby}[2][]
{
  \childdocdisable
  \childdoctrue
  \childdocmanualtrue
  \if?#1?\else
    \def\jobname{#2}
  \fi
  \def\childdocjob{#2}
  \input{#2}
  \endinput
}
%    \end{macrocode}

% \macro{\childdocforward}
% The command |\childdocforward| redirects
% compilation to the main file or
% (if the optional argument is given) a child file.
% Parameters are set as if the main file
% or a child file starting with |\childdocof| was compiled.
% Then compilation is handed over to the main file:
%    \begin{macrocode}
\newcommand{\childdocforward}[2][]
{
  \begingroup
    \if?#1?
      \def\childdoctmp
      {
        \def\childdocname{#2}
        \def\childdocjob{#2}
        \def\jobname{#2}
        \input{#2}
        \endinput
      }
    \else
      \def\childdoctmp
      {
        \childdocdisable
        \def\childdocname{#2}
        \childdoctrue
        \includeonly{#2}
        \def\childdocjob{#1}
        \def\jobname{#1}
        \input{#1}
        \endinput
      }
    \fi
    \expandafter
  \endgroup
  \childdoctmp
}
%    \end{macrocode}

% \macro{\childdocforwardprefix}
% The command |\childdocforwardprefix| redirects
% compilation to the main or a child file by means of a pattern.
% The prefix |#1| in the current filename is replaced by |#2|
% and the suffix of the current filename is kept
% (it is assumed that the filename does not contain the substring `|~~~|'
% which is used as a delimiter).
% Compilation is handed over to the new file by |\childdocforward|:
%    \begin{macrocode}
\newcommand{\childdocforwardprefix}[3][]
{
  \begingroup
    \def\childdocextract #2##1~~~{\def\childdoctmp{\childdocforward[#1]{#3##1}}}
    \expandafter\childdocextract\childdocname~~~
    \expandafter
  \endgroup
  \childdoctmp
}
%    \end{macrocode}

% \macro{\childdoc}
% The deprecated macro |\childdoc| is a legacy version of |\childdocmain|:
%    \begin{macrocode}
\newcommand{\childdoc}{\childdocmain}
%    \end{macrocode}

% \macro{\childdocredirect}
% The deprecated macro |\childdocredirect| is a legacy version
% of |\childdocforward| and |\childdocforwardprefix|:
%    \begin{macrocode}
\newcommand{\childdocredirect}[2][]
{
  \begingroup
    \if?#1?
      \def\childdoctmp{\childdocforward{#2}}
    \else
      \def\childdoctmp{\childdocforwardprefix{#1}{#2}}
    \fi
    \expandafter
  \endgroup
  \childdoctmp
}
%    \end{macrocode}

%\iffalse
%</package>
%\fi
%
\endinput
|\\
|\childdocforward[|\textit{main}|]{|\textit{dest}|}|\\
\end{tabular}
\end{center}
%
The argument \textit{dest} is the destination file
(without extension).
It should be the main file or one of the child files.
Note that further \textsf{childdoc} directives
such as |\childdocof| and |\childdocforward|
in the indicated file will be processed in this form.
The optional argument \textit{main}
passes on directly to the main file \textit{main}
while pretending to compile the child \textit{dest}.
This form behaves as if \textit{dest}
issues |\childdocof{|\textit{main}|}| right away,
and no further \textsf{childdoc} directives will be processed.

%%%%%%%%%%%%%%%%%%%%%%%%%%%%%%%%%%%%%%%%
\DescribeMacro{\...prefix}
In the alternative form |\childdocforwardprefix|,
%
\begin{center}
\begin{tabular}{l}
|% \iffalse
%
% childdoc.dtx Copyright (C) 2017-2018 Niklas Beisert
%
% This work may be distributed and/or modified under the
% conditions of the LaTeX Project Public License, either version 1.3
% of this license or (at your option) any later version.
% The latest version of this license is in
%   http://www.latex-project.org/lppl.txt
% and version 1.3 or later is part of all distributions of LaTeX
% version 2005/12/01 or later.
%
% This work has the LPPL maintenance status `maintained'.
%
% The Current Maintainer of this work is Niklas Beisert.
%
% This work consists of the files childdoc.dtx and childdoc.ins
% and the derived files childdoc.def and cdocsamp.tex with
% cdocsch1.tex, cdocsch2.tex, cdocsdrf.tex, cdocsfn1.tex, cdocsfn2.tex.
%
%<package>\ifdefined\childdocmain\endinput\fi
%<package>\ProvidesFile{childdoc.def}[2018/12/30 v2.0 child document driver]
%<samplemain>\ProvidesFile{cdocsamp.tex}[2018/12/30 v2.0 sample for childdoc]
%<*driver>
%\ProvidesFile{childdoc.drv}[2018/12/30 v2.0 childdoc reference manual file]
\PassOptionsToClass{10pt,a4paper}{article}
\documentclass{ltxdoc}

\usepackage[margin=35mm]{geometry}
\usepackage{hyperref}
\usepackage{hyperxmp}
\usepackage[usenames]{color}

\hypersetup{colorlinks=true}
\hypersetup{pdfstartview=FitH}
\hypersetup{pdfpagemode=UseNone}
\hypersetup{pdfsource={}}
\hypersetup{pdflang={en-UK}}
\hypersetup{pdfcopyright={Copyright 2017-2018 Niklas Beisert.
  This work may be distributed and/or modified under the
  conditions of the LaTeX Project Public License, either version 1.3
  of this license or (at your option) any later version.}}
\hypersetup{pdflicenseurl={http://www.latex-project.org/lppl.txt}}
\hypersetup{pdfcontactaddress={ETH Zurich, ITP, HIT K,
  Wolfgang-Pauli-Strasse 27}}
\hypersetup{pdfcontactpostcode={8093}}
\hypersetup{pdfcontactcity={Zurich}}
\hypersetup{pdfcontactcountry={Switzerland}}
\hypersetup{pdfcontactemail={nbeisert@itp.phys.ethz.ch}}
\hypersetup{pdfcontacturl={http://people.phys.ethz.ch/\xmptilde nbeisert/}}

\newcommand{\secref}[1]{\hyperref[#1]{section \ref*{#1}}}

\parskip1ex
\parindent0pt
\let\olditemize\itemize
\def\itemize{\olditemize\parskip0pt}

\begin{document}

\title{The \textsf{childdoc} Package}
\hypersetup{pdftitle={The childdoc Package}}
\author{Niklas Beisert\\[2ex]
  Institut f\"ur Theoretische Physik\\
  Eidgen\"ossische Technische Hochschule Z\"urich\\
  Wolfgang-Pauli-Strasse 27, 8093 Z\"urich, Switzerland\\[1ex]
  \href{mailto:nbeisert@itp.phys.ethz.ch}
  {\texttt{nbeisert@itp.phys.ethz.ch}}}
\hypersetup{pdfauthor={Niklas Beisert}}
\hypersetup{pdfsubject={Manual for the LaTeX2e Package childdoc}}
\date{30 December 2018, \textsf{v2.0}}
\maketitle

\begin{abstract}\noindent
\textsf{childdoc} is a \LaTeXe{} package
that enables the direct compilation
of document sections included by |\include|
to individual files.
\end{abstract}

\begingroup
\parskip0ex
\tableofcontents
\endgroup

%%%%%%%%%%%%%%%%%%%%%%%%%%%%%%%%%%%%%%%%%%%%%%%%%%%%%%%%%%%%%%%%%%%%%%%%%%%%%%%%
%%%%%%%%%%%%%%%%%%%%%%%%%%%%%%%%%%%%%%%%%%%%%%%%%%%%%%%%%%%%%%%%%%%%%%%%%%%%%%%%
\section{Introduction}

\LaTeX{} provides a mechanism to structure a large document (such as a book)
into a main file and several child files (containing the chapters)
using the |\include| command.
This mechanism is beneficial for documents
which span hundreds of pages in order to
make the source file(s) more manageable.
Moreover, compilation can be restricted to
selected child files by means of the |\includeonly| command.
The latter feature can be used to reduce the compilation time while editing
(this was significantly more useful in the earlier days of \LaTeX{})
or to generate a smaller document which is easier to navigate.
Another application of |\includeonly| is to generate
documents consisting of selected parts of the complete document.

However, there are a few drawbacks of the plain |\include| mechanism:
\begin{itemize}
\item
The child files cannot be compiled on their own,
they can only be compiled via the main file.
A naive editing environment
(such as a text editor with an option
to have the current file processed by \LaTeX)
may require one to switch to the main file before compiling;
attempting to compile the child file produces errors.
\item
The main file must be modified (each time)
to adjust the |\includeonly| command
to the present needs. This easily leaves the main file in a messy state.
\item
The generated document will always carry the filename
of the main document. This is inconvenient if
several child files are to be compiled and
to be kept for distribution.
\end{itemize}

The present package provides a simple interface
to make child files individually compilable by \LaTeX{}.
Compiling a child file then has the same effect as compiling
the main file with an |\includeonly| command
to select the appropriate child.
Moreover the generated document will carry the name of the child
rather than the main file.
This resolves all three above issues.

This feature is meant to make the editing of books,
thesis documents and lecture notes somewhat more convenient.
However, the package can also be used efficiently for
composing a series of documents (such as exercise sheets)
which are typically distributed individually.
It then assists the author in generating the individual documents
(potentially in different versions)
as well as a document containing the collected series.
Another application is in developing style files
or other kinds of included material
where compilation of the style file could redirect
to a sample or test file.

%%%%%%%%%%%%%%%%%%%%%%%%%%%%%%%%%%%%%%%%%%%%%%%%%%%%%%%%%%%%%%%%%%%%%%%%%%%%%%%%
%%%%%%%%%%%%%%%%%%%%%%%%%%%%%%%%%%%%%%%%%%%%%%%%%%%%%%%%%%%%%%%%%%%%%%%%%%%%%%%%
\section{Usage}

First of all, the package \textsf{childdoc} is \emph{not} a standard
\LaTeXe{} |.sty| style file! Therefore it needs to be invoked in
a non-standard way.

%%%%%%%%%%%%%%%%%%%%%%%%%%%%%%%%%%%%%%%%%%%%%%%%%%%%%%%%%%%%%%%%%%%%%%%%%%%%%%%%
\subsection{Included Files}
\label{sec:include}

%%%%%%%%%%%%%%%%%%%%%%%%%%%%%%%%%%%%%%%%
\DescribeMacro{\childdocmain}
To use the package, add the commands
\begin{center}
\begin{tabular}{l}
|\input{childdoc.def}|\\
|\childdocmain{}|\\
\end{tabular}
\end{center}
at the very top of the main \LaTeX{} file,
in particular \emph{before} the |\documentclass| statement!
The argument of |\childdocmain| should be left empty
(but it must be present).

%%%%%%%%%%%%%%%%%%%%%%%%%%%%%%%%%%%%%%%%
\DescribeMacro{\childdocof}
Furthermore, add the commands
\begin{center}
\begin{tabular}{l}
|\input{childdoc.def}|\\
|\childdocof{|\textit{main}|}|\\
\end{tabular}
\end{center}
at the top of every child file \textit{child}
which is included by |\include{|\textit{child}|}|
from within the main file
(or at least for those files to be compiled individually).
The argument \textit{main} must be the filename of the main file.

There are a couple of
considerations in setting up the main and child documents:

%%%%%%%%%%%%%%%%%%%%%%%%%%%%%%%%%%%%%%%%
\paragraph{Restrictions.}

Please note the following restrictions:
\begin{itemize}
\item
|\childdocmain| must be called with one argument \textit{main}
to ensure compatibility with earlier version of the package.
It must either be empty (|\childdocmain{}|)
or precisely match the filename of the main file in which it is specified.
See \secref{sec:detection} for further information.
\item
The filename \textit{main} must be specified without the |.tex| extension.
\item
The filename \textit{main} is case sensitive
(even in case-insensitive file systems)
due to internal string comparison.
\item
The argument \textit{main} should be fully expanded, it cannot be a macro.
\item
Subdirectories and special characters should be avoided in filenames.
\item
The command |\childdocmain{|\textit{main}|}| must be followed by a whitespace.
It should not be followed immediately by another command
or by a comment mark `|%|'.
This is because the \TeX{} parser reads the token immediately following
the argument of |\childdocmain| and puts it
at the beginning of every child section;
however, a white\-space is ignored.
\end{itemize}

%%%%%%%%%%%%%%%%%%%%%%%%%%%%%%%%%%%%%%%%
\paragraph{Content of Main File.}

It is advisable to place all content in the child files included by |\include|.
Any output contained in the main file will appear in all child documents
unless suppressed manually;
it cannot be suppressed automatically by the |\includeonly| directive
and thus should normally be avoided.
A method to include some content in the main file
by means of conditional processing is described in \secref{sec:conditional}.

%%%%%%%%%%%%%%%%%%%%%%%%%%%%%%%%%%%%%%%%
\paragraph{Page Numbering.}

When only a part of the document is compiled,
the appropriate numbering of pages
(as well as other status parameters)
is determined from the |.aux| files.
The latter contain information from previous passes.
However this information needs to propagate through
all intermediate child documents.
Therefore the page numbering in child documents may well
be inconsistent until the complete document is compiled at least once.

A useful (if unconventional) way to always ensure a consistent
page numbering is to restart the numbering in each child document
and denote the pages by `\textit{child}|.|\textit{page}'
where \textit{child} represents the chapter/section number of the child file.
This can be achieved by the command
|\numberwithin{page}{|\textit{child}|}|
of the \textsf{amsmath} package
where \textit{child} can be |chapter| or |section|
depending on the chosen structuring.
Alternatively, one can modify the macro |\thepage| appropriately
and reset the counter |page| at the start of each child file.

%%%%%%%%%%%%%%%%%%%%%%%%%%%%%%%%%%%%%%%%%%%%%%%%%%%%%%%%%%%%%%%%%%%%%%%%%%%%%%%%
\subsection{Conditional Processing}
\label{sec:conditional}

The package provides a mechanism to compile different versions
of a document. To customise the versions further some conditional processing
can come in handy to distinguish which version is being compiled.
The package provides two macros to describe the compilation context:

%%%%%%%%%%%%%%%%%%%%%%%%%%%%%%%%%%%%%%%%
\DescribeMacro{\ifchilddoc}
The conditional |\ifchilddoc| distinguishes between the compilation of
child documents and the main document:
%
\begin{center}
|\ifchilddoc |\textit{child-code}| |[|\||else |\textit{main-code}]| \||fi|
\end{center}

%%%%%%%%%%%%%%%%%%%%%%%%%%%%%%%%%%%%%%%%
\DescribeMacro{\childdocname}
\DescribeMacro{\childdocjob}
The macro |\childdocname| contains the filename (without extension)
of the main or child file being processed.
Note that |\childdocjob| will always contain the name of the main file.

%%%%%%%%%%%%%%%%%%%%%%%%%%%%%%%%%%%%%%%%
\paragraph{Title Page.}

Conditional processing can be used to include a title or banner page
in the main document when proper precautions are taken.
Importantly, the code in the main file should ensure that the page counter
(as well as other status parameters which are stored in the |.aux| files)
takes the same value after the conditional processing.
Otherwise the page numbers may take divergent values
depending on which part is compiled.

For example, a title page could be declared by:
%
\begin{center}
\begin{tabular}{l}
|\ifchilddoc\||else|\\
|\addtocounter{page}{-1}|\\
\textit{code for title page}\\
|\newpage|\\
|\||fi|
\end{tabular}
\end{center}
%
A banner page for the child documents can be generated by:
%
\begin{center}
\begin{tabular}{l}
|\ifchilddoc|\\
|\addtocounter{page}{-1}|\\
\textit{code for banner page}\\
|\newpage|\\
|\||fi|
\end{tabular}
\end{center}
%
Here one could write a message such as:
\begin{center}
|This is the part \childdocname{} of \childdocjob{}.|
\end{center}

%%%%%%%%%%%%%%%%%%%%%%%%%%%%%%%%%%%%%%%%%%%%%%%%%%%%%%%%%%%%%%%%%%%%%%%%%%%%%%%%
\subsection{Flags}
\label{sec:flags}

The package makes it easy to generate different versions
of the main or child documents.
To this end compilation flags can be defined
and assigned different default values.
They will be particularly useful in conjunction
with the forwarding mechanism described in \secref{sec:forward}.

For example, it may be useful to have a flag |\version|
which can be set to |draft| or |final|.
The document source will contain some conditional code
depending on the value of |\version|.
Suppose further, the flag should default to |final| for the main file
and to |draft| for child files
which is a natural assignment for editing the document.
This is achieved by placing the following code
in the preamble of the main document
(below the |\childdocmain| directive):
%
\begin{center}
\begin{tabular}{l}
|\ifchilddoc|\\
|\providecommand{\version}{draft}|\\
|\||else|\\
|\providecommand{\version}{final}|\\
|\||fi|
\end{tabular}
\end{center}
%
The definition by |\providecommand| makes sure
that previous definitions are not overwritten.
Further statements |\providecommand{\version}{...}|
can thus be added before the above code to override it.

For the main file, one might add a line
(between |\childdocmain| and the above block)
%
\begin{center}
|%\ifchilddoc\||else\providecommand{\version}{draft}\||fi|
\end{center}
%
which can be uncommented to produce a draft version.
Likewise one can add a line to the very top of a child file
(above the |\childdocof{|\textit{main}|}| directive)
%
\begin{center}
|%\providecommand{\version}{final}|
\end{center}
%
which can be uncommented to produce the final version of this child document.

%%%%%%%%%%%%%%%%%%%%%%%%%%%%%%%%%%%%%%%%%%%%%%%%%%%%%%%%%%%%%%%%%%%%%%%%%%%%%%%%
\subsection{Forwarding}
\label{sec:forward}

Different versions of the main or child documents
using compilation flags as described in \secref{sec:flags}
can be (permanently) stored in different files
for convenient compilation, viewing and distribution.
To this end, the package defines a command
to pass on compilation to a different file:

%%%%%%%%%%%%%%%%%%%%%%%%%%%%%%%%%%%%%%%%
\DescribeMacro{\childdocforward}
The command |\childdocforward| redirects processing to
another source file:
%
\begin{center}
\begin{tabular}{l}
|\input{childdoc.def}|\\
|\childdocforward[|\textit{main}|]{|\textit{dest}|}|\\
\end{tabular}
\end{center}
%
The argument \textit{dest} is the destination file
(without extension).
It should be the main file or one of the child files.
Note that further \textsf{childdoc} directives
such as |\childdocof| and |\childdocforward|
in the indicated file will be processed in this form.
The optional argument \textit{main}
passes on directly to the main file \textit{main}
while pretending to compile the child \textit{dest}.
This form behaves as if \textit{dest}
issues |\childdocof{|\textit{main}|}| right away,
and no further \textsf{childdoc} directives will be processed.

%%%%%%%%%%%%%%%%%%%%%%%%%%%%%%%%%%%%%%%%
\DescribeMacro{\...prefix}
In the alternative form |\childdocforwardprefix|,
%
\begin{center}
\begin{tabular}{l}
|\input{childdoc.def}|\\
|\childdocforwardprefix[|\textit{main}|]{|\textit{prefix}|}{|\textit{dest}|}|
\end{tabular}
\end{center}
%
the destination file is determined by a pattern
depending on the current file:
To make this work, the current file must be called
`{\textit{prefix}\hspace{0.2em}\textit{suffix}}'
with \textit{prefix} matching precisely the argument.
Processing is then passed on to the file
`{\textit{dest}\hspace{0.2em}\textit{suffix}}'.
Surely, the same effect is achieved by
directly specifying the
argument `{\textit{dest}\hspace{0.2em}\textit{suffix}}'
in the first form.
However, that requires to set up a different file
for each child. With the alternative form of the command
all these files can have exactly the same content
which simplifies setting them up and maintaining them.

For example, the following file |draft.tex|
with a compilation flag |\version| as described in \secref{sec:flags}
compiles the main document as a draft:
%
\begin{center}
\begin{tabular}{l}
|\def\version{draft}|\\
|\input{childdoc.def}|\\
|\childdocforward{|\textit{main}|}|
\end{tabular}
\end{center}
%
Likewise, the following files |final|\textit{nn}|.tex|
compile the final version of the child document
|child|\textit{nn}|.tex|:
%
\begin{center}
\begin{tabular}{l}
|\def\version{final}|\\
|\input{childdoc.def}|\\
|\childdocforwardprefix{final}{child}|
\end{tabular}
\end{center}
%

Note that when several versions of a main file and/or of each child file
are to be generated, it may be convenient to set up a |Makefile| or
shell script to automatise the process.

%%%%%%%%%%%%%%%%%%%%%%%%%%%%%%%%%%%%%%%%%%%%%%%%%%%%%%%%%%%%%%%%%%%%%%%%%%%%%%%%
\subsection{Command Line Processing}
\label{sec:commandline}

The effect of redirection files can also be achieved by invoking
the \LaTeX{} compiler with a more elaborate command line.
Most conveniently this should be done as part
of a shell script or a |Makefile|.

When using \textsf{childdoc} in the main file, the following
command lines effectively perform a redirection
(note that depending on the shell being used,
backslashes may have to be doubled: `|\|' $\to$ `|\\|'):
%
\begin{center}
|... -jobname "|\textit{target}|" |\\|"|[\textit{flags}]%
|\input{childdoc.def}\childdocforward[|\textit{main}|]{|\textit{dest}|}"|
\end{center}
%
Here \textit{target} is the name of the output file,
\textit{main} is the name of the main file
and \textit{dest} is the name of the main or child file to be processed
(all filenames without extensions).
The optional argument \textit{main} can be omitted
if \textit{main} matches \textit{dest}.
Optionally, compilation \textit{flags} can be defined via |\def| commands.
This command line makes the \TeX{} engine believe
it is compiling the file \textit{target}
whose content is specified as the latter parameter.
The provided code then forwards the processing to
\textit{main} or \textit{dest} as described in \secref{sec:forward}.

%%%%%%%%%%%%%%%%%%%%%%%%%%%%%%%%%%%%%%%%%%%%%%%%%%%%%%%%%%%%%%%%%%%%%%%%%%%%%%%%
\subsection{Include by Input}
\label{sec:input}

Including child documents by |\include| has some restrictions by design.
Most notably, the content of a child document always occupies
its own set of pages; pages cannot be shared between child documents.
Usually, this behaviour makes perfect sense
because each child document contain an essential part of the document.
However, in some situations it may be desirable to compose
a document from a collection of parts
without having mandatory page breaks between then.
For this case, the package
provides a mechanism to include parts
by |\input| which can also be processed individually.
However, by construction this mechanism
requires manual handling of the content to be output.

%%%%%%%%%%%%%%%%%%%%%%%%%%%%%%%%%%%%%%%%
\DescribeMacro{\ifchilddocmanual}
The main file should be prepared as usual, see \secref{sec:include}.
However, the document body must make a distinction
between processing of an individual part and of the main document, e.g.:
%
\begin{center}
\begin{tabular}{l}
|\ifchilddocmanual|\\
|\input{\childdocname}|\\
|\||else|\\
\textit{document body with }|\input{|\textit{part}|}|\\
|\||fi|
\end{tabular}
\end{center}
%
The conditional |\ifchilddocmanual| is true whenever
a part to be included by |\input| is being compiled,
and the name of the part is stored in |\childdocname|.

%%%%%%%%%%%%%%%%%%%%%%%%%%%%%%%%%%%%%%%%
\DescribeMacro{\childdocby}
Each part to be included by |\input| should start with:
%
\begin{center}
\begin{tabular}{l}
|\input{childdoc.def}|\\
|\childdocby{|\textit{main}|}|\\
\end{tabular}
\end{center}
%
The directive |\childdocby| is similar to |\childdocof|
described in \secref{sec:include},
but the subsequent selection of content must be done manually.
To that end, both |\ifchilddoc| and |\ifchilddocmanual|
will be true upon processing of a part,
and the name of the part is stored in |\childdocname|.
Note that |\jobname| will be set to the filename of the current part
so that each part receives an individual |.aux| file
that does not interfere with the |.aux| file(s) of the main document.
This behaviour can be altered by the alternative form
|\childdocby[*]{|\textit{main}|}| (with a non-empty optional argument)
which uses the |.aux| file of the main document
by setting |\jobname| to \textit{main}.

%%%%%%%%%%%%%%%%%%%%%%%%%%%%%%%%%%%%%%%%%%%%%%%%%%%%%%%%%%%%%%%%%%%%%%%%%%%%%%%%
\subsection{Driver Development}
\label{sec:driver}

The \textsf{childdoc} mechanism can also be use for the development
of definition files such as \LaTeX{} styles or classes.
This case differs from the above setup with multiple parts
included by |\include| in that no |\includeonly| should be invoked.
This can be achieved by starting the include file
(before |\ProvidesPackage|) with:
%
\begin{center}
\begin{tabular}{l}
|\input{childdoc.def}|\\
|\childdocforward{|\textit{main}|}|\\
\end{tabular}
\end{center}
%
or alternatively with:
%
\begin{center}
\begin{tabular}{l}
|\input{childdoc.def}|\\
|\childdocby{|\textit{main}|}|\\
\end{tabular}
\end{center}
%
Both forms have slightly different effects as described above.
The main file is prepared as usual, see \secref{sec:include}.

%%%%%%%%%%%%%%%%%%%%%%%%%%%%%%%%%%%%%%%%%%%%%%%%%%%%%%%%%%%%%%%%%%%%%%%%%%%%%%%%
\subsection{Legacy Detection}
\label{sec:detection}

The directive |\childdocmain| in the main file can detect
whether the complete document or merely a child is to be compiled
even without using the directive |\childdocof|.
This method is deprecated because it is less robust
and there is no compelling reason to use it;
it is merely provided for backward compatibility
and it may be removed in future versions.

If the detection mechanism is to be used,
it is mandatory to correctly specify
the filename of the main file as the argument of |\childdocmain|:
%
\begin{center}
\begin{tabular}{l}
|\input{childdoc.def}|\\
|\childdocmain{|\textit{main}|}|\\
\end{tabular}
\end{center}
%
If |\jobname| does not match the argument \textit{main} of |\childdocmain|,
it is assumed that |\jobname| points to the child file to be compiled.
When using |\childdocmain| with the main file specified as argument,
it suffices to start a child file
with just |\input{|\textit{main}|}|
without loading of the package and using |\childdocof|.
If instead all processing is done
with the appropriate \textsf{childdoc} directives,
the argument of \textit{main} of |\childdocmain| can be empty.

An alternative version of the command line processing described
in \secref{sec:commandline} using the detection mechanism reads:
%
\begin{center}
|... -jobname "|\textit{target}|" "|[\textit{flags}]%
[|\def\jobname{|\textit{dest}|}|]|\input{|\textit{main}|}"|
\end{center}

%%%%%%%%%%%%%%%%%%%%%%%%%%%%%%%%%%%%%%%%%%%%%%%%%%%%%%%%%%%%%%%%%%%%%%%%%%%%%%%%
\subsection{Manual Code}
\label{sec:manual}

In case one cannot be certain whether the definitions file |childdoc.def|
is installed on the target \TeX{} distribution
and one prefers not to ship it,
it is conceivable to paste a few relevant commands into the sources.

To that end, drop all statements |\input{childdoc.def}|
and perform the replacements as outlined below.
Instead of |\childdocmain{|\textit{main}|}| add the following code
to the top of the main file:
%
\begin{center}
\begin{tabular}{l}
|\||ifdefined\childdocname\endinput\||fi\newif\ifchilddoc|\\
|\edef\childdocname{\scantokens\expandafter{\jobname\noexpand}}|\\
|\def\childdocmain{|\textit{main}|}\||ifx\childdocmain\childdocname\||else|\\
|\childdoctrue\includeonly{\childdocname}\let\jobname\childdocmain\||fi|\\
\end{tabular}
\end{center}
%
Instead of |\childdocof{|\textit{main}|}| just include the main file
at the top of each child file:
%
\begin{center}
|\input{|\textit{main}|}|
\end{center}
%
A simple redirection |\childdocforward{|\textit{dest}|}| is achieved by:
%
\begin{center}
|\def\jobname{|\textit{dest}|}\input{\jobname}|
\end{center}
%
The redirection with prefix
|\childdocforwardprefix[|\textit{prefix}|]{|\textit{dest}|}|
is accomplished by:
%
\begin{center}
\begin{tabular}{l}
|{\edef\jobname{\scantokens\expandafter{\jobname\noexpand}}|\\
|\def\redirectjob |\textit{prefix}|#1~~~{\gdef\jobname{|\textit{dest}|#1}}|\\
|\expandafter\redirectjob\jobname~~~}\input{\jobname}|
\end{tabular}
\end{center}

In an alternative approach,
child documents can be compiled by a specific command line
without additional code or specific definitions:
%
\begin{center}
|... -jobname "|\textit{target}|" "|[\textit{flags}]%
|\includeonly{|\textit{dest}|}\input{|\textit{main}|}"|
\end{center}
%

%%%%%%%%%%%%%%%%%%%%%%%%%%%%%%%%%%%%%%%%%%%%%%%%%%%%%%%%%%%%%%%%%%%%%%%%%%%%%%%%
%%%%%%%%%%%%%%%%%%%%%%%%%%%%%%%%%%%%%%%%%%%%%%%%%%%%%%%%%%%%%%%%%%%%%%%%%%%%%%%%
\section{Information}

%%%%%%%%%%%%%%%%%%%%%%%%%%%%%%%%%%%%%%%%%%%%%%%%%%%%%%%%%%%%%%%%%%%%%%%%%%%%%%%%
\subsection{Copyright}

Copyright \copyright{} 2017--2018 Niklas Beisert

This work may be distributed and/or modified under the
conditions of the \LaTeX{} Project Public License, either version 1.3
of this license or (at your option) any later version.
The latest version of this license is in
  \url{http://www.latex-project.org/lppl.txt}
and version 1.3 or later is part of all distributions of \LaTeX{}
version 2005/12/01 or later.

This work has the LPPL maintenance status `maintained'.

The Current Maintainer of this work is Niklas Beisert.

This work consists of the files |README.txt|, |childdoc.ins| and |childdoc.dtx|
as well as the derived files |childdoc.def|, |cdocsamp.tex|
with |cdocsch1.tex|, |cdocsch2.tex|, |cdocspt3.tex|, |cdocspt4.tex|,
|cdocsdrf.tex|, |cdocsfn1.tex|, |cdocsfn2.tex|
as well as |childdoc.pdf|.

%%%%%%%%%%%%%%%%%%%%%%%%%%%%%%%%%%%%%%%%%%%%%%%%%%%%%%%%%%%%%%%%%%%%%%%%%%%%%%%%
\subsection{Files and Installation}

The package consists of the files:
%
\begin{center}
\begin{tabular}{ll}
    |README.txt|   & readme file \\
    |childdoc.ins| & installation file \\
    |childdoc.dtx| & source file \\
    |childdoc.def| & definition file \\
    |cdocsamp.tex| & sample main file \\
    |cdocsch1.tex| & sample include file \\
    |cdocsch2.tex| & sample include file \\
    |cdocspt3.tex| & sample part file \\
    |cdocspt4.tex| & sample part file \\
    |cdocsdrf.tex| & sample redirection file \\
    |cdocsfn1.tex| & sample redirection file \\
    |cdocsfn2.tex| & sample redirection file \\
    |childdoc.pdf| & manual
\end{tabular}
\end{center}
%
The distribution consists of the files
|README.txt|, |childdoc.ins| and |childdoc.dtx|.
%
\begin{itemize}
\item
Run (pdf)\LaTeX{} on |childdoc.dtx|
to compile the manual |childdoc.pdf| (this file).
\item
Run \LaTeX{} on |childdoc.ins| to create the definitions file |childdoc.def|
and the sample |cdocsamp.tex| with include files
|cdocsch1.tex|, |cdocsch2.tex|, |cdocspt3.tex|, |cdocspt4.tex|,
|cdocsdrf.tex|, |cdocsfn1.tex|, |cdocsfn2.tex|.
Then copy the file |childdoc.def| to an appropriate directory of your \LaTeX{}
distribution, e.g.\ \textit{texmf-root}|/tex/latex/childdoc|.
\end{itemize}

%%%%%%%%%%%%%%%%%%%%%%%%%%%%%%%%%%%%%%%%%%%%%%%%%%%%%%%%%%%%%%%%%%%%%%%%%%%%%%%%
\subsection{Related CTAN Packages}

There are several other packages which offer a similar functionality:
%
\begin{itemize}
\item
The packages
\href{http://ctan.org/pkg/docmute}{\textsf{docmute}},
\href{http://ctan.org/pkg/includex}{\textsf{includex}} and
\href{http://ctan.org/pkg/standalone}{\textsf{standalone}}
provide commands to include only the document body of
a child file thus allowing both files to be compiled individually.
\item
The packages \href{http://ctan.org/pkg/subdocs}{\textsf{subdocs}}
and \href{http://ctan.org/pkg/subfiles}{\textsf{subfiles}}
provide structures in which the main and child documents can be
encapsulated and allowing them to be compiled individually.
The inclusion mechanism is different from the conventional |\include|.
\item
The package \href{http://ctan.org/pkg/combine}{\textsf{combine}}
is an elaborate solution to combine several documents into one.
\end{itemize}
%
See also the CTAN topic \href{http://ctan.org/topic/subdocs}{\textsf{subdocs}}
for further related packages.
The present package differs from the above solutions in that
a document structure constructed with the conventional |\include| mechanism
just needs two extra commands at the top of every file
such that all constituent files can be compiled individually.

%%%%%%%%%%%%%%%%%%%%%%%%%%%%%%%%%%%%%%%%%%%%%%%%%%%%%%%%%%%%%%%%%%%%%%%%%%%%%%%%
%\subsection{Feature Suggestions}
%
%The following is a list of features which may be useful for future
%versions of this package:
%%
%\begin{itemize}
%\item
%\ldots
%\end{itemize}

%%%%%%%%%%%%%%%%%%%%%%%%%%%%%%%%%%%%%%%%%%%%%%%%%%%%%%%%%%%%%%%%%%%%%%%%%%%%%%%%
\subsection{Revision History}

%%%%%%%%%%%%%%%%%%%%%%%%%%%%%%%%%%%%%%%%
\paragraph{v2.0:} 2018/12/30

\begin{itemize}
\item
immediate forward processing
\item
added |\childdocby| mechanism
\item
manual restructured
\end{itemize}

%%%%%%%%%%%%%%%%%%%%%%%%%%%%%%%%%%%%%%%%
\paragraph{v1.6:} 2018/01/17

\begin{itemize}
\item
application for development of include files
\item
corrections to manual
\end{itemize}

%%%%%%%%%%%%%%%%%%%%%%%%%%%%%%%%%%%%%%%%
\paragraph{v1.5:} 2017/05/21

\begin{itemize}
\item
more complete structuring introduced
\item
|\childdocof| introduced
\item
|\childdoc| renamed to |\childdocmain|
\item
|\childredirect| renamed to |\childdocforward| and |\childdocforwardprefix|
and functionality expanded
\end{itemize}

%%%%%%%%%%%%%%%%%%%%%%%%%%%%%%%%%%%%%%%%
\paragraph{v1.0:} 2017/04/27

\begin{itemize}
\item
manual and install package
\item
first version published on CTAN
\end{itemize}

%%%%%%%%%%%%%%%%%%%%%%%%%%%%%%%%%%%%%%%%
\paragraph{v0.6:} 2017/04/26

\begin{itemize}
\item
redirection mechanism added
\end{itemize}

%%%%%%%%%%%%%%%%%%%%%%%%%%%%%%%%%%%%%%%%
\paragraph{v0.5:} 2017/04/26

\begin{itemize}
\item
functionality in definition file
\end{itemize}


%%%%%%%%%%%%%%%%%%%%%%%%%%%%%%%%%%%%%%%%%%%%%%%%%%%%%%%%%%%%%%%%%%%%%%%%%%%%%%%%
%%%%%%%%%%%%%%%%%%%%%%%%%%%%%%%%%%%%%%%%%%%%%%%%%%%%%%%%%%%%%%%%%%%%%%%%%%%%%%%%
%%%%%%%%%%%%%%%%%%%%%%%%%%%%%%%%%%%%%%%%%%%%%%%%%%%%%%%%%%%%%%%%%%%%%%%%%%%%%%%%
\appendix

\settowidth\MacroIndent{\rmfamily\scriptsize 000\ }

 \DocInput{childdoc.dtx}

\end{document}
%</driver>
% \fi
%
% %%%%%%%%%%%%%%%%%%%%%%%%%%%%%%%%%%%%%%%%%%%%%%%%%%%%%%%%%%%%%%%%%%%%%%%%%%%%%%
% %%%%%%%%%%%%%%%%%%%%%%%%%%%%%%%%%%%%%%%%%%%%%%%%%%%%%%%%%%%%%%%%%%%%%%%%%%%%%%
% \section{Sample}
%\iffalse
%<*samplemain>
%\fi
%
% The following presents a sample document
% with two chapters, two parts, a title page,
% a compile flag as well as three forwarding files to set the flag.
% It consists of eight |.tex| files:
% \begin{center}
% \begin{tabular}{ll}
% |cdocsamp.tex|&main file\\
% |cdocsch1.tex|&include file for chapter 1\\
% |cdocsch2.tex|&include file for chapter 2\\
% |cdocspt3.tex|&include file for part 3\\
% |cdocspt4.tex|&include file for part 4\\
% |cdocsdrf.tex|&forwarding file for main file in draft mode\\
% |cdocsfi1.tex|&forwarding file for final version of chapter 1\\
% |cdocsfi2.tex|&forwarding file for final version of chapter 2\\
% \end{tabular}
% \end{center}
% Each of the eight files can be compiled directly by the \LaTeX{} compiler.
%
% %%%%%%%%%%%%%%%%%%%%%%%%%%%%%%%%%%%%%%
% \paragraph{Main File.}
%
% The main file is called |cdocsamp.tex|.
%
% Load the \textsf{childdoc} definitions and
% declare the filename for the main document:
%    \begin{macrocode}
\input{childdoc.def}
\childdocmain{}
%    \end{macrocode}

% Optional override for |\version| flag:
%    \begin{macrocode}
%%\ifchilddoc\else\providecommand{\version}{draft}\fi
%    \end{macrocode}

% Define the default values for the |\version| flag
% (|final| for the main file and |draft| for childs):
%    \begin{macrocode}
\ifchilddoc
\providecommand{\version}{draft}
\else
\providecommand{\version}{final}
\fi
%    \end{macrocode}

% Load the standard document class:
%    \begin{macrocode}
\documentclass[12pt]{article}
%    \end{macrocode}

% Start the document body:
%    \begin{macrocode}
\begin{document}
%    \end{macrocode}

% Declare a title page.
% Print title, part of document being processed and version flag:
%    \begin{macrocode}
\addtocounter{page}{-1}
\begin{center}
{\LARGE\bfseries{}childdoc example\par}
\vspace{1cm}
\ifchilddoc
\ifchilddocmanual part\else chapter\fi:
`\childdocname' of `\childdocjob'\par
\else
main document: `\childdocjob'\par
\fi
version: \version\par
\end{center}
\newpage
%    \end{macrocode}

% Manually include selected file,
% otherwise process as usual:
%    \begin{macrocode}
\ifchilddocmanual
\section*{part `\childdocname'}
\input{\childdocname}
\else
%    \end{macrocode}

% Include the two chapters:
%    \begin{macrocode}
\include{cdocsch1}
\include{cdocsch2}
%    \end{macrocode}

% Include the two parts unless only chapters should be displayed:
%    \begin{macrocode}
\ifchilddoc\else
\section{part three}
\input{cdocspt3}
\section{part four}
\input{cdocspt4}
\fi
%    \end{macrocode}

% Process as usual until here:
%    \begin{macrocode}
\fi
%    \end{macrocode}

% End of document body:
%    \begin{macrocode}
\end{document}
%    \end{macrocode}
%\iffalse
%</samplemain>
%\fi
%
% %%%%%%%%%%%%%%%%%%%%%%%%%%%%%%%%%%%%%%
% \paragraph{Chapter Include Files.}
%
% The include files are called |cdocsch1.tex| and |cdocsch2.tex|.
%
%\iffalse
%<*samplechap1|samplechap2>
%\fi

% Optional override for |\version| flag:
%    \begin{macrocode}
%%\providecommand{\version}{final}
%    \end{macrocode}

% Include the main document:
%    \begin{macrocode}
\input{childdoc.def}
\childdocof{cdocsamp}
%    \end{macrocode}

%\iffalse
%</samplechap1|samplechap2>
%\fi
%
%\iffalse
%<*samplechap1>
%\fi
% Some text for chapter 1:
%    \begin{macrocode}
\section{one}
some text in chapter one
%    \end{macrocode}

%\iffalse
%</samplechap1>
%\fi
% Some text for chapter 2:
%\iffalse
%<*samplechap2>
%\fi
%    \begin{macrocode}
\section{two}
more text in chapter two
%    \end{macrocode}

%\iffalse
%</samplechap2>
%\fi
%
% %%%%%%%%%%%%%%%%%%%%%%%%%%%%%%%%%%%%%%
% \paragraph{Part Include Files.}
%
% The include files are called |cdocspt3.tex| and |cdocspt4.tex|.
%
%\iffalse
%<*samplepart3|samplepart4>
%\fi

% Optional override for |\version| flag:
%    \begin{macrocode}
%%\providecommand{\version}{final}
%    \end{macrocode}

% Include the main document:
%    \begin{macrocode}
\input{childdoc.def}
\childdocby{cdocsamp}
%    \end{macrocode}

%\iffalse
%</samplepart3|samplepart4>
%\fi
%
%\iffalse
%<*samplepart3>
%\fi
% Some text for part 3:
%    \begin{macrocode}
some text in part three
%    \end{macrocode}

%\iffalse
%</samplepart3>
%\fi
% Some text for part 4:
%\iffalse
%<*samplepart4>
%\fi
%    \begin{macrocode}
more text in part four
%    \end{macrocode}

%\iffalse
%</samplepart4>
%\fi
%
% %%%%%%%%%%%%%%%%%%%%%%%%%%%%%%%%%%%%%%
% \paragraph{Forwarding for a Complete Draft.}
%
% The following forwarding file |cdocsdrf.tex|
% compiles the main document in draft mode:
%\iffalse
%<*sampledraft>
%\fi
%    \begin{macrocode}
\def\version{draft}
\input{childdoc.def}
\childdocforward{cdocsamp}
%    \end{macrocode}

%\iffalse
%</sampledraft>
%\fi
%
% %%%%%%%%%%%%%%%%%%%%%%%%%%%%%%%%%%%%%%
% \paragraph{Forwarding for Final Version of the Chapters.}
%
% The following forwarding files |cdocsfn1.tex| and |cdocsfn2.tex|
% (with identical content)
% compile the final versions of the child documents
% |cdocsch1.tex| and |cdocsch2.tex|, respectively:
%\iffalse
%<*samplefinal>
%\fi
%    \begin{macrocode}
\def\version{final}
\input{childdoc.def}
\childdocforwardprefix[cdocsamp]{cdocsfn}{cdocsch}
%    \end{macrocode}

%\iffalse
%</samplefinal>
%\fi
%
% %%%%%%%%%%%%%%%%%%%%%%%%%%%%%%%%%%%%%%
% \paragraph{Command Line Processing.}
%
% The following three command lines generate the output files
% |cdocscld|, |cdocscl1| and |cdocscl2|
% which should be identical to
% |cdocsdrf|, |cdocsch1| and |cdocsfn2|, respectively:
% \begin{center}
% \begin{tabular}{l}
% |latex -jobname cdocscld \|\\
% |  "\def\version{draft}\input{childdoc.def}\childdocforward{cdocsamp}"|\\
% |latex -jobname cdocscl1 \|\\
% |  "\input{childdoc.def}\childdocforward[cdocsamp]{cdocsch1}"|\\
% |latex -jobname cdocscl2 \|\\
% |  "\def\version{final}\input{childdoc.def}\childdocforward{cdocsch2}"|
% \end{tabular}
% \end{center}
% Note that the trailing backslash on each first line
% merely continues the input to the second line
% (for convenient cut ant paste).
% Furthermore, the command |latex| can be replaced by any
% of its alternative versions such as |pdflatex|.
%
% %%%%%%%%%%%%%%%%%%%%%%%%%%%%%%%%%%%%%%%%%%%%%%%%%%%%%%%%%%%%%%%%%%%%%%%%%%%%%%
% %%%%%%%%%%%%%%%%%%%%%%%%%%%%%%%%%%%%%%%%%%%%%%%%%%%%%%%%%%%%%%%%%%%%%%%%%%%%%%
% \section{Implementation}
%\iffalse
%<*package>
%\fi
%
% This section describes the definitions file |childdoc.def|.

% The definitions cannot be loaded using |\usepackage| or |\RequirePackage|
% which has a mechanism to prevent loading a style file more than once.
% When loading the definitions by means of |\input|
% multiple instances have to be prevented manually:
%\iffalse
%This code needs to be before the `\ProvidesFile' directive
%which is defined at the beginning of this file.
%Therefore it is also placed there and commented out here.
%</package>
%<*discard>
%\fi
%    \begin{macrocode}
\ifdefined\childdocmain\endinput\fi
%    \end{macrocode}
%\iffalse
%</discard>
%<*package>
%\fi
%
% \macro{\ifchilddoc}
% \macro{\ifchilddocmanual}
% The conditional |\ifchilddoc| tells whether a
% child (true) or main (false) document is being compiled.
% The conditional |\ifchilddocmanual| tells whether
% the |\includeonly| mechanism is used (false) or
% the selection of child files must be performed manually (true).
% The definitions initialise to false:
%    \begin{macrocode}
\newif\ifchilddoc
\newif\ifchilddocmanual
%    \end{macrocode}

% \macro{\childdocname}
% \macro{\childdocjob}
% The macro |\childdocname| stores the name of the main document
% to be compiled. The macro |\childdocjob| stores the name of
% the document on which the \LaTeX{} compiler was originally invoked.
% The content of |\jobname| cannot be compared
% to filenames specified in the source due to different catcodes.
% The following code rescans |\jobname|, stores the result
% in |\childdocname| and saves a copy in |\childdocjob|:
%    \begin{macrocode}
\edef\childdocname{\scantokens\expandafter{\jobname\noexpand}}
\let\childdocjob\childdocname
%    \end{macrocode}

% \macro{\childdocdisable}
% The macro |\childdocdisable| prevents the main file
% from being processed more than once.
% At this stage, the main document command |\childdocmain|
% is assumed to be called once again where it should do nothing.
% Any subsequent call to it should prevent
% a secondary processing of the main document
% It overwrites the forwarding commands
% |\childdocof| and |\childdocforward|
% with empty macros to prevent further inclusions of the main document:
%    \begin{macrocode}
\newcommand{\childdocdisable}
{
  \renewcommand{\childdocmain}[1]{\renewcommand{\childdocmain}[1]{\endinput}}
  \renewcommand{\childdocof}[1]{}
  \renewcommand{\childdocby}[2][]{}
  \renewcommand{\childdocforward}[2][]{}
  \renewcommand{\childdocdisable}{}
}
%    \end{macrocode}

% \macro{\childdocmain}
% The macro |\childdocmain| is to be called at the top of the main file
% with nothing or the main filename (without extension) as argument.
% First, it breaks loops.
% If the argument is not empty and does not match |\childdocname|
% (which is set by the first inclusion of |childdoc.def|),
% |\ifchilddoc| is set to true, |\includeonly| is applied to the child file
% and |\jobname| is set to the main file
% (for proper handling of |.aux| files):
%    \begin{macrocode}
\newcommand{\childdocmain}[1]
{
  \childdocdisable\childdocmain{}
  \if?#1?\else
    \begingroup
      \def\childdoctmp{#1}
      \ifx\childdoctmp\childdocname
        \def\childdoctmp{}
      \else
        \def\childdoctmp
        {
          \childdoctrue
          \includeonly{\childdocname}
          \def\childdocjob{#1}
          \def\jobname{#1}
        }
      \fi
      \expandafter
    \endgroup
    \childdoctmp
  \fi
}
%    \end{macrocode}

% \macro{\childdocof}
% The command |\childdocof| redirects
% compilation to the main file |#1|.
%    \begin{macrocode}
\newcommand{\childdocof}[1]
{
  \childdocdisable
  \childdoctrue
  \includeonly{\childdocname}
  \def\jobname{#1}
  \def\childdocjob{#1}
  \input{#1}
}
%    \end{macrocode}

% \macro{\childdocby}
% The command |\childdocby| ....
%    \begin{macrocode}
\newcommand{\childdocby}[2][]
{
  \childdocdisable
  \childdoctrue
  \childdocmanualtrue
  \if?#1?\else
    \def\jobname{#2}
  \fi
  \def\childdocjob{#2}
  \input{#2}
  \endinput
}
%    \end{macrocode}

% \macro{\childdocforward}
% The command |\childdocforward| redirects
% compilation to the main file or
% (if the optional argument is given) a child file.
% Parameters are set as if the main file
% or a child file starting with |\childdocof| was compiled.
% Then compilation is handed over to the main file:
%    \begin{macrocode}
\newcommand{\childdocforward}[2][]
{
  \begingroup
    \if?#1?
      \def\childdoctmp
      {
        \def\childdocname{#2}
        \def\childdocjob{#2}
        \def\jobname{#2}
        \input{#2}
        \endinput
      }
    \else
      \def\childdoctmp
      {
        \childdocdisable
        \def\childdocname{#2}
        \childdoctrue
        \includeonly{#2}
        \def\childdocjob{#1}
        \def\jobname{#1}
        \input{#1}
        \endinput
      }
    \fi
    \expandafter
  \endgroup
  \childdoctmp
}
%    \end{macrocode}

% \macro{\childdocforwardprefix}
% The command |\childdocforwardprefix| redirects
% compilation to the main or a child file by means of a pattern.
% The prefix |#1| in the current filename is replaced by |#2|
% and the suffix of the current filename is kept
% (it is assumed that the filename does not contain the substring `|~~~|'
% which is used as a delimiter).
% Compilation is handed over to the new file by |\childdocforward|:
%    \begin{macrocode}
\newcommand{\childdocforwardprefix}[3][]
{
  \begingroup
    \def\childdocextract #2##1~~~{\def\childdoctmp{\childdocforward[#1]{#3##1}}}
    \expandafter\childdocextract\childdocname~~~
    \expandafter
  \endgroup
  \childdoctmp
}
%    \end{macrocode}

% \macro{\childdoc}
% The deprecated macro |\childdoc| is a legacy version of |\childdocmain|:
%    \begin{macrocode}
\newcommand{\childdoc}{\childdocmain}
%    \end{macrocode}

% \macro{\childdocredirect}
% The deprecated macro |\childdocredirect| is a legacy version
% of |\childdocforward| and |\childdocforwardprefix|:
%    \begin{macrocode}
\newcommand{\childdocredirect}[2][]
{
  \begingroup
    \if?#1?
      \def\childdoctmp{\childdocforward{#2}}
    \else
      \def\childdoctmp{\childdocforwardprefix{#1}{#2}}
    \fi
    \expandafter
  \endgroup
  \childdoctmp
}
%    \end{macrocode}

%\iffalse
%</package>
%\fi
%
\endinput
|\\
|\childdocforwardprefix[|\textit{main}|]{|\textit{prefix}|}{|\textit{dest}|}|
\end{tabular}
\end{center}
%
the destination file is determined by a pattern
depending on the current file:
To make this work, the current file must be called
`{\textit{prefix}\hspace{0.2em}\textit{suffix}}'
with \textit{prefix} matching precisely the argument.
Processing is then passed on to the file
`{\textit{dest}\hspace{0.2em}\textit{suffix}}'.
Surely, the same effect is achieved by
directly specifying the
argument `{\textit{dest}\hspace{0.2em}\textit{suffix}}'
in the first form.
However, that requires to set up a different file
for each child. With the alternative form of the command
all these files can have exactly the same content
which simplifies setting them up and maintaining them.

For example, the following file |draft.tex|
with a compilation flag |\version| as described in \secref{sec:flags}
compiles the main document as a draft:
%
\begin{center}
\begin{tabular}{l}
|\def\version{draft}|\\
|% \iffalse
%
% childdoc.dtx Copyright (C) 2017-2018 Niklas Beisert
%
% This work may be distributed and/or modified under the
% conditions of the LaTeX Project Public License, either version 1.3
% of this license or (at your option) any later version.
% The latest version of this license is in
%   http://www.latex-project.org/lppl.txt
% and version 1.3 or later is part of all distributions of LaTeX
% version 2005/12/01 or later.
%
% This work has the LPPL maintenance status `maintained'.
%
% The Current Maintainer of this work is Niklas Beisert.
%
% This work consists of the files childdoc.dtx and childdoc.ins
% and the derived files childdoc.def and cdocsamp.tex with
% cdocsch1.tex, cdocsch2.tex, cdocsdrf.tex, cdocsfn1.tex, cdocsfn2.tex.
%
%<package>\ifdefined\childdocmain\endinput\fi
%<package>\ProvidesFile{childdoc.def}[2018/12/30 v2.0 child document driver]
%<samplemain>\ProvidesFile{cdocsamp.tex}[2018/12/30 v2.0 sample for childdoc]
%<*driver>
%\ProvidesFile{childdoc.drv}[2018/12/30 v2.0 childdoc reference manual file]
\PassOptionsToClass{10pt,a4paper}{article}
\documentclass{ltxdoc}

\usepackage[margin=35mm]{geometry}
\usepackage{hyperref}
\usepackage{hyperxmp}
\usepackage[usenames]{color}

\hypersetup{colorlinks=true}
\hypersetup{pdfstartview=FitH}
\hypersetup{pdfpagemode=UseNone}
\hypersetup{pdfsource={}}
\hypersetup{pdflang={en-UK}}
\hypersetup{pdfcopyright={Copyright 2017-2018 Niklas Beisert.
  This work may be distributed and/or modified under the
  conditions of the LaTeX Project Public License, either version 1.3
  of this license or (at your option) any later version.}}
\hypersetup{pdflicenseurl={http://www.latex-project.org/lppl.txt}}
\hypersetup{pdfcontactaddress={ETH Zurich, ITP, HIT K,
  Wolfgang-Pauli-Strasse 27}}
\hypersetup{pdfcontactpostcode={8093}}
\hypersetup{pdfcontactcity={Zurich}}
\hypersetup{pdfcontactcountry={Switzerland}}
\hypersetup{pdfcontactemail={nbeisert@itp.phys.ethz.ch}}
\hypersetup{pdfcontacturl={http://people.phys.ethz.ch/\xmptilde nbeisert/}}

\newcommand{\secref}[1]{\hyperref[#1]{section \ref*{#1}}}

\parskip1ex
\parindent0pt
\let\olditemize\itemize
\def\itemize{\olditemize\parskip0pt}

\begin{document}

\title{The \textsf{childdoc} Package}
\hypersetup{pdftitle={The childdoc Package}}
\author{Niklas Beisert\\[2ex]
  Institut f\"ur Theoretische Physik\\
  Eidgen\"ossische Technische Hochschule Z\"urich\\
  Wolfgang-Pauli-Strasse 27, 8093 Z\"urich, Switzerland\\[1ex]
  \href{mailto:nbeisert@itp.phys.ethz.ch}
  {\texttt{nbeisert@itp.phys.ethz.ch}}}
\hypersetup{pdfauthor={Niklas Beisert}}
\hypersetup{pdfsubject={Manual for the LaTeX2e Package childdoc}}
\date{30 December 2018, \textsf{v2.0}}
\maketitle

\begin{abstract}\noindent
\textsf{childdoc} is a \LaTeXe{} package
that enables the direct compilation
of document sections included by |\include|
to individual files.
\end{abstract}

\begingroup
\parskip0ex
\tableofcontents
\endgroup

%%%%%%%%%%%%%%%%%%%%%%%%%%%%%%%%%%%%%%%%%%%%%%%%%%%%%%%%%%%%%%%%%%%%%%%%%%%%%%%%
%%%%%%%%%%%%%%%%%%%%%%%%%%%%%%%%%%%%%%%%%%%%%%%%%%%%%%%%%%%%%%%%%%%%%%%%%%%%%%%%
\section{Introduction}

\LaTeX{} provides a mechanism to structure a large document (such as a book)
into a main file and several child files (containing the chapters)
using the |\include| command.
This mechanism is beneficial for documents
which span hundreds of pages in order to
make the source file(s) more manageable.
Moreover, compilation can be restricted to
selected child files by means of the |\includeonly| command.
The latter feature can be used to reduce the compilation time while editing
(this was significantly more useful in the earlier days of \LaTeX{})
or to generate a smaller document which is easier to navigate.
Another application of |\includeonly| is to generate
documents consisting of selected parts of the complete document.

However, there are a few drawbacks of the plain |\include| mechanism:
\begin{itemize}
\item
The child files cannot be compiled on their own,
they can only be compiled via the main file.
A naive editing environment
(such as a text editor with an option
to have the current file processed by \LaTeX)
may require one to switch to the main file before compiling;
attempting to compile the child file produces errors.
\item
The main file must be modified (each time)
to adjust the |\includeonly| command
to the present needs. This easily leaves the main file in a messy state.
\item
The generated document will always carry the filename
of the main document. This is inconvenient if
several child files are to be compiled and
to be kept for distribution.
\end{itemize}

The present package provides a simple interface
to make child files individually compilable by \LaTeX{}.
Compiling a child file then has the same effect as compiling
the main file with an |\includeonly| command
to select the appropriate child.
Moreover the generated document will carry the name of the child
rather than the main file.
This resolves all three above issues.

This feature is meant to make the editing of books,
thesis documents and lecture notes somewhat more convenient.
However, the package can also be used efficiently for
composing a series of documents (such as exercise sheets)
which are typically distributed individually.
It then assists the author in generating the individual documents
(potentially in different versions)
as well as a document containing the collected series.
Another application is in developing style files
or other kinds of included material
where compilation of the style file could redirect
to a sample or test file.

%%%%%%%%%%%%%%%%%%%%%%%%%%%%%%%%%%%%%%%%%%%%%%%%%%%%%%%%%%%%%%%%%%%%%%%%%%%%%%%%
%%%%%%%%%%%%%%%%%%%%%%%%%%%%%%%%%%%%%%%%%%%%%%%%%%%%%%%%%%%%%%%%%%%%%%%%%%%%%%%%
\section{Usage}

First of all, the package \textsf{childdoc} is \emph{not} a standard
\LaTeXe{} |.sty| style file! Therefore it needs to be invoked in
a non-standard way.

%%%%%%%%%%%%%%%%%%%%%%%%%%%%%%%%%%%%%%%%%%%%%%%%%%%%%%%%%%%%%%%%%%%%%%%%%%%%%%%%
\subsection{Included Files}
\label{sec:include}

%%%%%%%%%%%%%%%%%%%%%%%%%%%%%%%%%%%%%%%%
\DescribeMacro{\childdocmain}
To use the package, add the commands
\begin{center}
\begin{tabular}{l}
|\input{childdoc.def}|\\
|\childdocmain{}|\\
\end{tabular}
\end{center}
at the very top of the main \LaTeX{} file,
in particular \emph{before} the |\documentclass| statement!
The argument of |\childdocmain| should be left empty
(but it must be present).

%%%%%%%%%%%%%%%%%%%%%%%%%%%%%%%%%%%%%%%%
\DescribeMacro{\childdocof}
Furthermore, add the commands
\begin{center}
\begin{tabular}{l}
|\input{childdoc.def}|\\
|\childdocof{|\textit{main}|}|\\
\end{tabular}
\end{center}
at the top of every child file \textit{child}
which is included by |\include{|\textit{child}|}|
from within the main file
(or at least for those files to be compiled individually).
The argument \textit{main} must be the filename of the main file.

There are a couple of
considerations in setting up the main and child documents:

%%%%%%%%%%%%%%%%%%%%%%%%%%%%%%%%%%%%%%%%
\paragraph{Restrictions.}

Please note the following restrictions:
\begin{itemize}
\item
|\childdocmain| must be called with one argument \textit{main}
to ensure compatibility with earlier version of the package.
It must either be empty (|\childdocmain{}|)
or precisely match the filename of the main file in which it is specified.
See \secref{sec:detection} for further information.
\item
The filename \textit{main} must be specified without the |.tex| extension.
\item
The filename \textit{main} is case sensitive
(even in case-insensitive file systems)
due to internal string comparison.
\item
The argument \textit{main} should be fully expanded, it cannot be a macro.
\item
Subdirectories and special characters should be avoided in filenames.
\item
The command |\childdocmain{|\textit{main}|}| must be followed by a whitespace.
It should not be followed immediately by another command
or by a comment mark `|%|'.
This is because the \TeX{} parser reads the token immediately following
the argument of |\childdocmain| and puts it
at the beginning of every child section;
however, a white\-space is ignored.
\end{itemize}

%%%%%%%%%%%%%%%%%%%%%%%%%%%%%%%%%%%%%%%%
\paragraph{Content of Main File.}

It is advisable to place all content in the child files included by |\include|.
Any output contained in the main file will appear in all child documents
unless suppressed manually;
it cannot be suppressed automatically by the |\includeonly| directive
and thus should normally be avoided.
A method to include some content in the main file
by means of conditional processing is described in \secref{sec:conditional}.

%%%%%%%%%%%%%%%%%%%%%%%%%%%%%%%%%%%%%%%%
\paragraph{Page Numbering.}

When only a part of the document is compiled,
the appropriate numbering of pages
(as well as other status parameters)
is determined from the |.aux| files.
The latter contain information from previous passes.
However this information needs to propagate through
all intermediate child documents.
Therefore the page numbering in child documents may well
be inconsistent until the complete document is compiled at least once.

A useful (if unconventional) way to always ensure a consistent
page numbering is to restart the numbering in each child document
and denote the pages by `\textit{child}|.|\textit{page}'
where \textit{child} represents the chapter/section number of the child file.
This can be achieved by the command
|\numberwithin{page}{|\textit{child}|}|
of the \textsf{amsmath} package
where \textit{child} can be |chapter| or |section|
depending on the chosen structuring.
Alternatively, one can modify the macro |\thepage| appropriately
and reset the counter |page| at the start of each child file.

%%%%%%%%%%%%%%%%%%%%%%%%%%%%%%%%%%%%%%%%%%%%%%%%%%%%%%%%%%%%%%%%%%%%%%%%%%%%%%%%
\subsection{Conditional Processing}
\label{sec:conditional}

The package provides a mechanism to compile different versions
of a document. To customise the versions further some conditional processing
can come in handy to distinguish which version is being compiled.
The package provides two macros to describe the compilation context:

%%%%%%%%%%%%%%%%%%%%%%%%%%%%%%%%%%%%%%%%
\DescribeMacro{\ifchilddoc}
The conditional |\ifchilddoc| distinguishes between the compilation of
child documents and the main document:
%
\begin{center}
|\ifchilddoc |\textit{child-code}| |[|\||else |\textit{main-code}]| \||fi|
\end{center}

%%%%%%%%%%%%%%%%%%%%%%%%%%%%%%%%%%%%%%%%
\DescribeMacro{\childdocname}
\DescribeMacro{\childdocjob}
The macro |\childdocname| contains the filename (without extension)
of the main or child file being processed.
Note that |\childdocjob| will always contain the name of the main file.

%%%%%%%%%%%%%%%%%%%%%%%%%%%%%%%%%%%%%%%%
\paragraph{Title Page.}

Conditional processing can be used to include a title or banner page
in the main document when proper precautions are taken.
Importantly, the code in the main file should ensure that the page counter
(as well as other status parameters which are stored in the |.aux| files)
takes the same value after the conditional processing.
Otherwise the page numbers may take divergent values
depending on which part is compiled.

For example, a title page could be declared by:
%
\begin{center}
\begin{tabular}{l}
|\ifchilddoc\||else|\\
|\addtocounter{page}{-1}|\\
\textit{code for title page}\\
|\newpage|\\
|\||fi|
\end{tabular}
\end{center}
%
A banner page for the child documents can be generated by:
%
\begin{center}
\begin{tabular}{l}
|\ifchilddoc|\\
|\addtocounter{page}{-1}|\\
\textit{code for banner page}\\
|\newpage|\\
|\||fi|
\end{tabular}
\end{center}
%
Here one could write a message such as:
\begin{center}
|This is the part \childdocname{} of \childdocjob{}.|
\end{center}

%%%%%%%%%%%%%%%%%%%%%%%%%%%%%%%%%%%%%%%%%%%%%%%%%%%%%%%%%%%%%%%%%%%%%%%%%%%%%%%%
\subsection{Flags}
\label{sec:flags}

The package makes it easy to generate different versions
of the main or child documents.
To this end compilation flags can be defined
and assigned different default values.
They will be particularly useful in conjunction
with the forwarding mechanism described in \secref{sec:forward}.

For example, it may be useful to have a flag |\version|
which can be set to |draft| or |final|.
The document source will contain some conditional code
depending on the value of |\version|.
Suppose further, the flag should default to |final| for the main file
and to |draft| for child files
which is a natural assignment for editing the document.
This is achieved by placing the following code
in the preamble of the main document
(below the |\childdocmain| directive):
%
\begin{center}
\begin{tabular}{l}
|\ifchilddoc|\\
|\providecommand{\version}{draft}|\\
|\||else|\\
|\providecommand{\version}{final}|\\
|\||fi|
\end{tabular}
\end{center}
%
The definition by |\providecommand| makes sure
that previous definitions are not overwritten.
Further statements |\providecommand{\version}{...}|
can thus be added before the above code to override it.

For the main file, one might add a line
(between |\childdocmain| and the above block)
%
\begin{center}
|%\ifchilddoc\||else\providecommand{\version}{draft}\||fi|
\end{center}
%
which can be uncommented to produce a draft version.
Likewise one can add a line to the very top of a child file
(above the |\childdocof{|\textit{main}|}| directive)
%
\begin{center}
|%\providecommand{\version}{final}|
\end{center}
%
which can be uncommented to produce the final version of this child document.

%%%%%%%%%%%%%%%%%%%%%%%%%%%%%%%%%%%%%%%%%%%%%%%%%%%%%%%%%%%%%%%%%%%%%%%%%%%%%%%%
\subsection{Forwarding}
\label{sec:forward}

Different versions of the main or child documents
using compilation flags as described in \secref{sec:flags}
can be (permanently) stored in different files
for convenient compilation, viewing and distribution.
To this end, the package defines a command
to pass on compilation to a different file:

%%%%%%%%%%%%%%%%%%%%%%%%%%%%%%%%%%%%%%%%
\DescribeMacro{\childdocforward}
The command |\childdocforward| redirects processing to
another source file:
%
\begin{center}
\begin{tabular}{l}
|\input{childdoc.def}|\\
|\childdocforward[|\textit{main}|]{|\textit{dest}|}|\\
\end{tabular}
\end{center}
%
The argument \textit{dest} is the destination file
(without extension).
It should be the main file or one of the child files.
Note that further \textsf{childdoc} directives
such as |\childdocof| and |\childdocforward|
in the indicated file will be processed in this form.
The optional argument \textit{main}
passes on directly to the main file \textit{main}
while pretending to compile the child \textit{dest}.
This form behaves as if \textit{dest}
issues |\childdocof{|\textit{main}|}| right away,
and no further \textsf{childdoc} directives will be processed.

%%%%%%%%%%%%%%%%%%%%%%%%%%%%%%%%%%%%%%%%
\DescribeMacro{\...prefix}
In the alternative form |\childdocforwardprefix|,
%
\begin{center}
\begin{tabular}{l}
|\input{childdoc.def}|\\
|\childdocforwardprefix[|\textit{main}|]{|\textit{prefix}|}{|\textit{dest}|}|
\end{tabular}
\end{center}
%
the destination file is determined by a pattern
depending on the current file:
To make this work, the current file must be called
`{\textit{prefix}\hspace{0.2em}\textit{suffix}}'
with \textit{prefix} matching precisely the argument.
Processing is then passed on to the file
`{\textit{dest}\hspace{0.2em}\textit{suffix}}'.
Surely, the same effect is achieved by
directly specifying the
argument `{\textit{dest}\hspace{0.2em}\textit{suffix}}'
in the first form.
However, that requires to set up a different file
for each child. With the alternative form of the command
all these files can have exactly the same content
which simplifies setting them up and maintaining them.

For example, the following file |draft.tex|
with a compilation flag |\version| as described in \secref{sec:flags}
compiles the main document as a draft:
%
\begin{center}
\begin{tabular}{l}
|\def\version{draft}|\\
|\input{childdoc.def}|\\
|\childdocforward{|\textit{main}|}|
\end{tabular}
\end{center}
%
Likewise, the following files |final|\textit{nn}|.tex|
compile the final version of the child document
|child|\textit{nn}|.tex|:
%
\begin{center}
\begin{tabular}{l}
|\def\version{final}|\\
|\input{childdoc.def}|\\
|\childdocforwardprefix{final}{child}|
\end{tabular}
\end{center}
%

Note that when several versions of a main file and/or of each child file
are to be generated, it may be convenient to set up a |Makefile| or
shell script to automatise the process.

%%%%%%%%%%%%%%%%%%%%%%%%%%%%%%%%%%%%%%%%%%%%%%%%%%%%%%%%%%%%%%%%%%%%%%%%%%%%%%%%
\subsection{Command Line Processing}
\label{sec:commandline}

The effect of redirection files can also be achieved by invoking
the \LaTeX{} compiler with a more elaborate command line.
Most conveniently this should be done as part
of a shell script or a |Makefile|.

When using \textsf{childdoc} in the main file, the following
command lines effectively perform a redirection
(note that depending on the shell being used,
backslashes may have to be doubled: `|\|' $\to$ `|\\|'):
%
\begin{center}
|... -jobname "|\textit{target}|" |\\|"|[\textit{flags}]%
|\input{childdoc.def}\childdocforward[|\textit{main}|]{|\textit{dest}|}"|
\end{center}
%
Here \textit{target} is the name of the output file,
\textit{main} is the name of the main file
and \textit{dest} is the name of the main or child file to be processed
(all filenames without extensions).
The optional argument \textit{main} can be omitted
if \textit{main} matches \textit{dest}.
Optionally, compilation \textit{flags} can be defined via |\def| commands.
This command line makes the \TeX{} engine believe
it is compiling the file \textit{target}
whose content is specified as the latter parameter.
The provided code then forwards the processing to
\textit{main} or \textit{dest} as described in \secref{sec:forward}.

%%%%%%%%%%%%%%%%%%%%%%%%%%%%%%%%%%%%%%%%%%%%%%%%%%%%%%%%%%%%%%%%%%%%%%%%%%%%%%%%
\subsection{Include by Input}
\label{sec:input}

Including child documents by |\include| has some restrictions by design.
Most notably, the content of a child document always occupies
its own set of pages; pages cannot be shared between child documents.
Usually, this behaviour makes perfect sense
because each child document contain an essential part of the document.
However, in some situations it may be desirable to compose
a document from a collection of parts
without having mandatory page breaks between then.
For this case, the package
provides a mechanism to include parts
by |\input| which can also be processed individually.
However, by construction this mechanism
requires manual handling of the content to be output.

%%%%%%%%%%%%%%%%%%%%%%%%%%%%%%%%%%%%%%%%
\DescribeMacro{\ifchilddocmanual}
The main file should be prepared as usual, see \secref{sec:include}.
However, the document body must make a distinction
between processing of an individual part and of the main document, e.g.:
%
\begin{center}
\begin{tabular}{l}
|\ifchilddocmanual|\\
|\input{\childdocname}|\\
|\||else|\\
\textit{document body with }|\input{|\textit{part}|}|\\
|\||fi|
\end{tabular}
\end{center}
%
The conditional |\ifchilddocmanual| is true whenever
a part to be included by |\input| is being compiled,
and the name of the part is stored in |\childdocname|.

%%%%%%%%%%%%%%%%%%%%%%%%%%%%%%%%%%%%%%%%
\DescribeMacro{\childdocby}
Each part to be included by |\input| should start with:
%
\begin{center}
\begin{tabular}{l}
|\input{childdoc.def}|\\
|\childdocby{|\textit{main}|}|\\
\end{tabular}
\end{center}
%
The directive |\childdocby| is similar to |\childdocof|
described in \secref{sec:include},
but the subsequent selection of content must be done manually.
To that end, both |\ifchilddoc| and |\ifchilddocmanual|
will be true upon processing of a part,
and the name of the part is stored in |\childdocname|.
Note that |\jobname| will be set to the filename of the current part
so that each part receives an individual |.aux| file
that does not interfere with the |.aux| file(s) of the main document.
This behaviour can be altered by the alternative form
|\childdocby[*]{|\textit{main}|}| (with a non-empty optional argument)
which uses the |.aux| file of the main document
by setting |\jobname| to \textit{main}.

%%%%%%%%%%%%%%%%%%%%%%%%%%%%%%%%%%%%%%%%%%%%%%%%%%%%%%%%%%%%%%%%%%%%%%%%%%%%%%%%
\subsection{Driver Development}
\label{sec:driver}

The \textsf{childdoc} mechanism can also be use for the development
of definition files such as \LaTeX{} styles or classes.
This case differs from the above setup with multiple parts
included by |\include| in that no |\includeonly| should be invoked.
This can be achieved by starting the include file
(before |\ProvidesPackage|) with:
%
\begin{center}
\begin{tabular}{l}
|\input{childdoc.def}|\\
|\childdocforward{|\textit{main}|}|\\
\end{tabular}
\end{center}
%
or alternatively with:
%
\begin{center}
\begin{tabular}{l}
|\input{childdoc.def}|\\
|\childdocby{|\textit{main}|}|\\
\end{tabular}
\end{center}
%
Both forms have slightly different effects as described above.
The main file is prepared as usual, see \secref{sec:include}.

%%%%%%%%%%%%%%%%%%%%%%%%%%%%%%%%%%%%%%%%%%%%%%%%%%%%%%%%%%%%%%%%%%%%%%%%%%%%%%%%
\subsection{Legacy Detection}
\label{sec:detection}

The directive |\childdocmain| in the main file can detect
whether the complete document or merely a child is to be compiled
even without using the directive |\childdocof|.
This method is deprecated because it is less robust
and there is no compelling reason to use it;
it is merely provided for backward compatibility
and it may be removed in future versions.

If the detection mechanism is to be used,
it is mandatory to correctly specify
the filename of the main file as the argument of |\childdocmain|:
%
\begin{center}
\begin{tabular}{l}
|\input{childdoc.def}|\\
|\childdocmain{|\textit{main}|}|\\
\end{tabular}
\end{center}
%
If |\jobname| does not match the argument \textit{main} of |\childdocmain|,
it is assumed that |\jobname| points to the child file to be compiled.
When using |\childdocmain| with the main file specified as argument,
it suffices to start a child file
with just |\input{|\textit{main}|}|
without loading of the package and using |\childdocof|.
If instead all processing is done
with the appropriate \textsf{childdoc} directives,
the argument of \textit{main} of |\childdocmain| can be empty.

An alternative version of the command line processing described
in \secref{sec:commandline} using the detection mechanism reads:
%
\begin{center}
|... -jobname "|\textit{target}|" "|[\textit{flags}]%
[|\def\jobname{|\textit{dest}|}|]|\input{|\textit{main}|}"|
\end{center}

%%%%%%%%%%%%%%%%%%%%%%%%%%%%%%%%%%%%%%%%%%%%%%%%%%%%%%%%%%%%%%%%%%%%%%%%%%%%%%%%
\subsection{Manual Code}
\label{sec:manual}

In case one cannot be certain whether the definitions file |childdoc.def|
is installed on the target \TeX{} distribution
and one prefers not to ship it,
it is conceivable to paste a few relevant commands into the sources.

To that end, drop all statements |\input{childdoc.def}|
and perform the replacements as outlined below.
Instead of |\childdocmain{|\textit{main}|}| add the following code
to the top of the main file:
%
\begin{center}
\begin{tabular}{l}
|\||ifdefined\childdocname\endinput\||fi\newif\ifchilddoc|\\
|\edef\childdocname{\scantokens\expandafter{\jobname\noexpand}}|\\
|\def\childdocmain{|\textit{main}|}\||ifx\childdocmain\childdocname\||else|\\
|\childdoctrue\includeonly{\childdocname}\let\jobname\childdocmain\||fi|\\
\end{tabular}
\end{center}
%
Instead of |\childdocof{|\textit{main}|}| just include the main file
at the top of each child file:
%
\begin{center}
|\input{|\textit{main}|}|
\end{center}
%
A simple redirection |\childdocforward{|\textit{dest}|}| is achieved by:
%
\begin{center}
|\def\jobname{|\textit{dest}|}\input{\jobname}|
\end{center}
%
The redirection with prefix
|\childdocforwardprefix[|\textit{prefix}|]{|\textit{dest}|}|
is accomplished by:
%
\begin{center}
\begin{tabular}{l}
|{\edef\jobname{\scantokens\expandafter{\jobname\noexpand}}|\\
|\def\redirectjob |\textit{prefix}|#1~~~{\gdef\jobname{|\textit{dest}|#1}}|\\
|\expandafter\redirectjob\jobname~~~}\input{\jobname}|
\end{tabular}
\end{center}

In an alternative approach,
child documents can be compiled by a specific command line
without additional code or specific definitions:
%
\begin{center}
|... -jobname "|\textit{target}|" "|[\textit{flags}]%
|\includeonly{|\textit{dest}|}\input{|\textit{main}|}"|
\end{center}
%

%%%%%%%%%%%%%%%%%%%%%%%%%%%%%%%%%%%%%%%%%%%%%%%%%%%%%%%%%%%%%%%%%%%%%%%%%%%%%%%%
%%%%%%%%%%%%%%%%%%%%%%%%%%%%%%%%%%%%%%%%%%%%%%%%%%%%%%%%%%%%%%%%%%%%%%%%%%%%%%%%
\section{Information}

%%%%%%%%%%%%%%%%%%%%%%%%%%%%%%%%%%%%%%%%%%%%%%%%%%%%%%%%%%%%%%%%%%%%%%%%%%%%%%%%
\subsection{Copyright}

Copyright \copyright{} 2017--2018 Niklas Beisert

This work may be distributed and/or modified under the
conditions of the \LaTeX{} Project Public License, either version 1.3
of this license or (at your option) any later version.
The latest version of this license is in
  \url{http://www.latex-project.org/lppl.txt}
and version 1.3 or later is part of all distributions of \LaTeX{}
version 2005/12/01 or later.

This work has the LPPL maintenance status `maintained'.

The Current Maintainer of this work is Niklas Beisert.

This work consists of the files |README.txt|, |childdoc.ins| and |childdoc.dtx|
as well as the derived files |childdoc.def|, |cdocsamp.tex|
with |cdocsch1.tex|, |cdocsch2.tex|, |cdocspt3.tex|, |cdocspt4.tex|,
|cdocsdrf.tex|, |cdocsfn1.tex|, |cdocsfn2.tex|
as well as |childdoc.pdf|.

%%%%%%%%%%%%%%%%%%%%%%%%%%%%%%%%%%%%%%%%%%%%%%%%%%%%%%%%%%%%%%%%%%%%%%%%%%%%%%%%
\subsection{Files and Installation}

The package consists of the files:
%
\begin{center}
\begin{tabular}{ll}
    |README.txt|   & readme file \\
    |childdoc.ins| & installation file \\
    |childdoc.dtx| & source file \\
    |childdoc.def| & definition file \\
    |cdocsamp.tex| & sample main file \\
    |cdocsch1.tex| & sample include file \\
    |cdocsch2.tex| & sample include file \\
    |cdocspt3.tex| & sample part file \\
    |cdocspt4.tex| & sample part file \\
    |cdocsdrf.tex| & sample redirection file \\
    |cdocsfn1.tex| & sample redirection file \\
    |cdocsfn2.tex| & sample redirection file \\
    |childdoc.pdf| & manual
\end{tabular}
\end{center}
%
The distribution consists of the files
|README.txt|, |childdoc.ins| and |childdoc.dtx|.
%
\begin{itemize}
\item
Run (pdf)\LaTeX{} on |childdoc.dtx|
to compile the manual |childdoc.pdf| (this file).
\item
Run \LaTeX{} on |childdoc.ins| to create the definitions file |childdoc.def|
and the sample |cdocsamp.tex| with include files
|cdocsch1.tex|, |cdocsch2.tex|, |cdocspt3.tex|, |cdocspt4.tex|,
|cdocsdrf.tex|, |cdocsfn1.tex|, |cdocsfn2.tex|.
Then copy the file |childdoc.def| to an appropriate directory of your \LaTeX{}
distribution, e.g.\ \textit{texmf-root}|/tex/latex/childdoc|.
\end{itemize}

%%%%%%%%%%%%%%%%%%%%%%%%%%%%%%%%%%%%%%%%%%%%%%%%%%%%%%%%%%%%%%%%%%%%%%%%%%%%%%%%
\subsection{Related CTAN Packages}

There are several other packages which offer a similar functionality:
%
\begin{itemize}
\item
The packages
\href{http://ctan.org/pkg/docmute}{\textsf{docmute}},
\href{http://ctan.org/pkg/includex}{\textsf{includex}} and
\href{http://ctan.org/pkg/standalone}{\textsf{standalone}}
provide commands to include only the document body of
a child file thus allowing both files to be compiled individually.
\item
The packages \href{http://ctan.org/pkg/subdocs}{\textsf{subdocs}}
and \href{http://ctan.org/pkg/subfiles}{\textsf{subfiles}}
provide structures in which the main and child documents can be
encapsulated and allowing them to be compiled individually.
The inclusion mechanism is different from the conventional |\include|.
\item
The package \href{http://ctan.org/pkg/combine}{\textsf{combine}}
is an elaborate solution to combine several documents into one.
\end{itemize}
%
See also the CTAN topic \href{http://ctan.org/topic/subdocs}{\textsf{subdocs}}
for further related packages.
The present package differs from the above solutions in that
a document structure constructed with the conventional |\include| mechanism
just needs two extra commands at the top of every file
such that all constituent files can be compiled individually.

%%%%%%%%%%%%%%%%%%%%%%%%%%%%%%%%%%%%%%%%%%%%%%%%%%%%%%%%%%%%%%%%%%%%%%%%%%%%%%%%
%\subsection{Feature Suggestions}
%
%The following is a list of features which may be useful for future
%versions of this package:
%%
%\begin{itemize}
%\item
%\ldots
%\end{itemize}

%%%%%%%%%%%%%%%%%%%%%%%%%%%%%%%%%%%%%%%%%%%%%%%%%%%%%%%%%%%%%%%%%%%%%%%%%%%%%%%%
\subsection{Revision History}

%%%%%%%%%%%%%%%%%%%%%%%%%%%%%%%%%%%%%%%%
\paragraph{v2.0:} 2018/12/30

\begin{itemize}
\item
immediate forward processing
\item
added |\childdocby| mechanism
\item
manual restructured
\end{itemize}

%%%%%%%%%%%%%%%%%%%%%%%%%%%%%%%%%%%%%%%%
\paragraph{v1.6:} 2018/01/17

\begin{itemize}
\item
application for development of include files
\item
corrections to manual
\end{itemize}

%%%%%%%%%%%%%%%%%%%%%%%%%%%%%%%%%%%%%%%%
\paragraph{v1.5:} 2017/05/21

\begin{itemize}
\item
more complete structuring introduced
\item
|\childdocof| introduced
\item
|\childdoc| renamed to |\childdocmain|
\item
|\childredirect| renamed to |\childdocforward| and |\childdocforwardprefix|
and functionality expanded
\end{itemize}

%%%%%%%%%%%%%%%%%%%%%%%%%%%%%%%%%%%%%%%%
\paragraph{v1.0:} 2017/04/27

\begin{itemize}
\item
manual and install package
\item
first version published on CTAN
\end{itemize}

%%%%%%%%%%%%%%%%%%%%%%%%%%%%%%%%%%%%%%%%
\paragraph{v0.6:} 2017/04/26

\begin{itemize}
\item
redirection mechanism added
\end{itemize}

%%%%%%%%%%%%%%%%%%%%%%%%%%%%%%%%%%%%%%%%
\paragraph{v0.5:} 2017/04/26

\begin{itemize}
\item
functionality in definition file
\end{itemize}


%%%%%%%%%%%%%%%%%%%%%%%%%%%%%%%%%%%%%%%%%%%%%%%%%%%%%%%%%%%%%%%%%%%%%%%%%%%%%%%%
%%%%%%%%%%%%%%%%%%%%%%%%%%%%%%%%%%%%%%%%%%%%%%%%%%%%%%%%%%%%%%%%%%%%%%%%%%%%%%%%
%%%%%%%%%%%%%%%%%%%%%%%%%%%%%%%%%%%%%%%%%%%%%%%%%%%%%%%%%%%%%%%%%%%%%%%%%%%%%%%%
\appendix

\settowidth\MacroIndent{\rmfamily\scriptsize 000\ }

 \DocInput{childdoc.dtx}

\end{document}
%</driver>
% \fi
%
% %%%%%%%%%%%%%%%%%%%%%%%%%%%%%%%%%%%%%%%%%%%%%%%%%%%%%%%%%%%%%%%%%%%%%%%%%%%%%%
% %%%%%%%%%%%%%%%%%%%%%%%%%%%%%%%%%%%%%%%%%%%%%%%%%%%%%%%%%%%%%%%%%%%%%%%%%%%%%%
% \section{Sample}
%\iffalse
%<*samplemain>
%\fi
%
% The following presents a sample document
% with two chapters, two parts, a title page,
% a compile flag as well as three forwarding files to set the flag.
% It consists of eight |.tex| files:
% \begin{center}
% \begin{tabular}{ll}
% |cdocsamp.tex|&main file\\
% |cdocsch1.tex|&include file for chapter 1\\
% |cdocsch2.tex|&include file for chapter 2\\
% |cdocspt3.tex|&include file for part 3\\
% |cdocspt4.tex|&include file for part 4\\
% |cdocsdrf.tex|&forwarding file for main file in draft mode\\
% |cdocsfi1.tex|&forwarding file for final version of chapter 1\\
% |cdocsfi2.tex|&forwarding file for final version of chapter 2\\
% \end{tabular}
% \end{center}
% Each of the eight files can be compiled directly by the \LaTeX{} compiler.
%
% %%%%%%%%%%%%%%%%%%%%%%%%%%%%%%%%%%%%%%
% \paragraph{Main File.}
%
% The main file is called |cdocsamp.tex|.
%
% Load the \textsf{childdoc} definitions and
% declare the filename for the main document:
%    \begin{macrocode}
\input{childdoc.def}
\childdocmain{}
%    \end{macrocode}

% Optional override for |\version| flag:
%    \begin{macrocode}
%%\ifchilddoc\else\providecommand{\version}{draft}\fi
%    \end{macrocode}

% Define the default values for the |\version| flag
% (|final| for the main file and |draft| for childs):
%    \begin{macrocode}
\ifchilddoc
\providecommand{\version}{draft}
\else
\providecommand{\version}{final}
\fi
%    \end{macrocode}

% Load the standard document class:
%    \begin{macrocode}
\documentclass[12pt]{article}
%    \end{macrocode}

% Start the document body:
%    \begin{macrocode}
\begin{document}
%    \end{macrocode}

% Declare a title page.
% Print title, part of document being processed and version flag:
%    \begin{macrocode}
\addtocounter{page}{-1}
\begin{center}
{\LARGE\bfseries{}childdoc example\par}
\vspace{1cm}
\ifchilddoc
\ifchilddocmanual part\else chapter\fi:
`\childdocname' of `\childdocjob'\par
\else
main document: `\childdocjob'\par
\fi
version: \version\par
\end{center}
\newpage
%    \end{macrocode}

% Manually include selected file,
% otherwise process as usual:
%    \begin{macrocode}
\ifchilddocmanual
\section*{part `\childdocname'}
\input{\childdocname}
\else
%    \end{macrocode}

% Include the two chapters:
%    \begin{macrocode}
\include{cdocsch1}
\include{cdocsch2}
%    \end{macrocode}

% Include the two parts unless only chapters should be displayed:
%    \begin{macrocode}
\ifchilddoc\else
\section{part three}
\input{cdocspt3}
\section{part four}
\input{cdocspt4}
\fi
%    \end{macrocode}

% Process as usual until here:
%    \begin{macrocode}
\fi
%    \end{macrocode}

% End of document body:
%    \begin{macrocode}
\end{document}
%    \end{macrocode}
%\iffalse
%</samplemain>
%\fi
%
% %%%%%%%%%%%%%%%%%%%%%%%%%%%%%%%%%%%%%%
% \paragraph{Chapter Include Files.}
%
% The include files are called |cdocsch1.tex| and |cdocsch2.tex|.
%
%\iffalse
%<*samplechap1|samplechap2>
%\fi

% Optional override for |\version| flag:
%    \begin{macrocode}
%%\providecommand{\version}{final}
%    \end{macrocode}

% Include the main document:
%    \begin{macrocode}
\input{childdoc.def}
\childdocof{cdocsamp}
%    \end{macrocode}

%\iffalse
%</samplechap1|samplechap2>
%\fi
%
%\iffalse
%<*samplechap1>
%\fi
% Some text for chapter 1:
%    \begin{macrocode}
\section{one}
some text in chapter one
%    \end{macrocode}

%\iffalse
%</samplechap1>
%\fi
% Some text for chapter 2:
%\iffalse
%<*samplechap2>
%\fi
%    \begin{macrocode}
\section{two}
more text in chapter two
%    \end{macrocode}

%\iffalse
%</samplechap2>
%\fi
%
% %%%%%%%%%%%%%%%%%%%%%%%%%%%%%%%%%%%%%%
% \paragraph{Part Include Files.}
%
% The include files are called |cdocspt3.tex| and |cdocspt4.tex|.
%
%\iffalse
%<*samplepart3|samplepart4>
%\fi

% Optional override for |\version| flag:
%    \begin{macrocode}
%%\providecommand{\version}{final}
%    \end{macrocode}

% Include the main document:
%    \begin{macrocode}
\input{childdoc.def}
\childdocby{cdocsamp}
%    \end{macrocode}

%\iffalse
%</samplepart3|samplepart4>
%\fi
%
%\iffalse
%<*samplepart3>
%\fi
% Some text for part 3:
%    \begin{macrocode}
some text in part three
%    \end{macrocode}

%\iffalse
%</samplepart3>
%\fi
% Some text for part 4:
%\iffalse
%<*samplepart4>
%\fi
%    \begin{macrocode}
more text in part four
%    \end{macrocode}

%\iffalse
%</samplepart4>
%\fi
%
% %%%%%%%%%%%%%%%%%%%%%%%%%%%%%%%%%%%%%%
% \paragraph{Forwarding for a Complete Draft.}
%
% The following forwarding file |cdocsdrf.tex|
% compiles the main document in draft mode:
%\iffalse
%<*sampledraft>
%\fi
%    \begin{macrocode}
\def\version{draft}
\input{childdoc.def}
\childdocforward{cdocsamp}
%    \end{macrocode}

%\iffalse
%</sampledraft>
%\fi
%
% %%%%%%%%%%%%%%%%%%%%%%%%%%%%%%%%%%%%%%
% \paragraph{Forwarding for Final Version of the Chapters.}
%
% The following forwarding files |cdocsfn1.tex| and |cdocsfn2.tex|
% (with identical content)
% compile the final versions of the child documents
% |cdocsch1.tex| and |cdocsch2.tex|, respectively:
%\iffalse
%<*samplefinal>
%\fi
%    \begin{macrocode}
\def\version{final}
\input{childdoc.def}
\childdocforwardprefix[cdocsamp]{cdocsfn}{cdocsch}
%    \end{macrocode}

%\iffalse
%</samplefinal>
%\fi
%
% %%%%%%%%%%%%%%%%%%%%%%%%%%%%%%%%%%%%%%
% \paragraph{Command Line Processing.}
%
% The following three command lines generate the output files
% |cdocscld|, |cdocscl1| and |cdocscl2|
% which should be identical to
% |cdocsdrf|, |cdocsch1| and |cdocsfn2|, respectively:
% \begin{center}
% \begin{tabular}{l}
% |latex -jobname cdocscld \|\\
% |  "\def\version{draft}\input{childdoc.def}\childdocforward{cdocsamp}"|\\
% |latex -jobname cdocscl1 \|\\
% |  "\input{childdoc.def}\childdocforward[cdocsamp]{cdocsch1}"|\\
% |latex -jobname cdocscl2 \|\\
% |  "\def\version{final}\input{childdoc.def}\childdocforward{cdocsch2}"|
% \end{tabular}
% \end{center}
% Note that the trailing backslash on each first line
% merely continues the input to the second line
% (for convenient cut ant paste).
% Furthermore, the command |latex| can be replaced by any
% of its alternative versions such as |pdflatex|.
%
% %%%%%%%%%%%%%%%%%%%%%%%%%%%%%%%%%%%%%%%%%%%%%%%%%%%%%%%%%%%%%%%%%%%%%%%%%%%%%%
% %%%%%%%%%%%%%%%%%%%%%%%%%%%%%%%%%%%%%%%%%%%%%%%%%%%%%%%%%%%%%%%%%%%%%%%%%%%%%%
% \section{Implementation}
%\iffalse
%<*package>
%\fi
%
% This section describes the definitions file |childdoc.def|.

% The definitions cannot be loaded using |\usepackage| or |\RequirePackage|
% which has a mechanism to prevent loading a style file more than once.
% When loading the definitions by means of |\input|
% multiple instances have to be prevented manually:
%\iffalse
%This code needs to be before the `\ProvidesFile' directive
%which is defined at the beginning of this file.
%Therefore it is also placed there and commented out here.
%</package>
%<*discard>
%\fi
%    \begin{macrocode}
\ifdefined\childdocmain\endinput\fi
%    \end{macrocode}
%\iffalse
%</discard>
%<*package>
%\fi
%
% \macro{\ifchilddoc}
% \macro{\ifchilddocmanual}
% The conditional |\ifchilddoc| tells whether a
% child (true) or main (false) document is being compiled.
% The conditional |\ifchilddocmanual| tells whether
% the |\includeonly| mechanism is used (false) or
% the selection of child files must be performed manually (true).
% The definitions initialise to false:
%    \begin{macrocode}
\newif\ifchilddoc
\newif\ifchilddocmanual
%    \end{macrocode}

% \macro{\childdocname}
% \macro{\childdocjob}
% The macro |\childdocname| stores the name of the main document
% to be compiled. The macro |\childdocjob| stores the name of
% the document on which the \LaTeX{} compiler was originally invoked.
% The content of |\jobname| cannot be compared
% to filenames specified in the source due to different catcodes.
% The following code rescans |\jobname|, stores the result
% in |\childdocname| and saves a copy in |\childdocjob|:
%    \begin{macrocode}
\edef\childdocname{\scantokens\expandafter{\jobname\noexpand}}
\let\childdocjob\childdocname
%    \end{macrocode}

% \macro{\childdocdisable}
% The macro |\childdocdisable| prevents the main file
% from being processed more than once.
% At this stage, the main document command |\childdocmain|
% is assumed to be called once again where it should do nothing.
% Any subsequent call to it should prevent
% a secondary processing of the main document
% It overwrites the forwarding commands
% |\childdocof| and |\childdocforward|
% with empty macros to prevent further inclusions of the main document:
%    \begin{macrocode}
\newcommand{\childdocdisable}
{
  \renewcommand{\childdocmain}[1]{\renewcommand{\childdocmain}[1]{\endinput}}
  \renewcommand{\childdocof}[1]{}
  \renewcommand{\childdocby}[2][]{}
  \renewcommand{\childdocforward}[2][]{}
  \renewcommand{\childdocdisable}{}
}
%    \end{macrocode}

% \macro{\childdocmain}
% The macro |\childdocmain| is to be called at the top of the main file
% with nothing or the main filename (without extension) as argument.
% First, it breaks loops.
% If the argument is not empty and does not match |\childdocname|
% (which is set by the first inclusion of |childdoc.def|),
% |\ifchilddoc| is set to true, |\includeonly| is applied to the child file
% and |\jobname| is set to the main file
% (for proper handling of |.aux| files):
%    \begin{macrocode}
\newcommand{\childdocmain}[1]
{
  \childdocdisable\childdocmain{}
  \if?#1?\else
    \begingroup
      \def\childdoctmp{#1}
      \ifx\childdoctmp\childdocname
        \def\childdoctmp{}
      \else
        \def\childdoctmp
        {
          \childdoctrue
          \includeonly{\childdocname}
          \def\childdocjob{#1}
          \def\jobname{#1}
        }
      \fi
      \expandafter
    \endgroup
    \childdoctmp
  \fi
}
%    \end{macrocode}

% \macro{\childdocof}
% The command |\childdocof| redirects
% compilation to the main file |#1|.
%    \begin{macrocode}
\newcommand{\childdocof}[1]
{
  \childdocdisable
  \childdoctrue
  \includeonly{\childdocname}
  \def\jobname{#1}
  \def\childdocjob{#1}
  \input{#1}
}
%    \end{macrocode}

% \macro{\childdocby}
% The command |\childdocby| ....
%    \begin{macrocode}
\newcommand{\childdocby}[2][]
{
  \childdocdisable
  \childdoctrue
  \childdocmanualtrue
  \if?#1?\else
    \def\jobname{#2}
  \fi
  \def\childdocjob{#2}
  \input{#2}
  \endinput
}
%    \end{macrocode}

% \macro{\childdocforward}
% The command |\childdocforward| redirects
% compilation to the main file or
% (if the optional argument is given) a child file.
% Parameters are set as if the main file
% or a child file starting with |\childdocof| was compiled.
% Then compilation is handed over to the main file:
%    \begin{macrocode}
\newcommand{\childdocforward}[2][]
{
  \begingroup
    \if?#1?
      \def\childdoctmp
      {
        \def\childdocname{#2}
        \def\childdocjob{#2}
        \def\jobname{#2}
        \input{#2}
        \endinput
      }
    \else
      \def\childdoctmp
      {
        \childdocdisable
        \def\childdocname{#2}
        \childdoctrue
        \includeonly{#2}
        \def\childdocjob{#1}
        \def\jobname{#1}
        \input{#1}
        \endinput
      }
    \fi
    \expandafter
  \endgroup
  \childdoctmp
}
%    \end{macrocode}

% \macro{\childdocforwardprefix}
% The command |\childdocforwardprefix| redirects
% compilation to the main or a child file by means of a pattern.
% The prefix |#1| in the current filename is replaced by |#2|
% and the suffix of the current filename is kept
% (it is assumed that the filename does not contain the substring `|~~~|'
% which is used as a delimiter).
% Compilation is handed over to the new file by |\childdocforward|:
%    \begin{macrocode}
\newcommand{\childdocforwardprefix}[3][]
{
  \begingroup
    \def\childdocextract #2##1~~~{\def\childdoctmp{\childdocforward[#1]{#3##1}}}
    \expandafter\childdocextract\childdocname~~~
    \expandafter
  \endgroup
  \childdoctmp
}
%    \end{macrocode}

% \macro{\childdoc}
% The deprecated macro |\childdoc| is a legacy version of |\childdocmain|:
%    \begin{macrocode}
\newcommand{\childdoc}{\childdocmain}
%    \end{macrocode}

% \macro{\childdocredirect}
% The deprecated macro |\childdocredirect| is a legacy version
% of |\childdocforward| and |\childdocforwardprefix|:
%    \begin{macrocode}
\newcommand{\childdocredirect}[2][]
{
  \begingroup
    \if?#1?
      \def\childdoctmp{\childdocforward{#2}}
    \else
      \def\childdoctmp{\childdocforwardprefix{#1}{#2}}
    \fi
    \expandafter
  \endgroup
  \childdoctmp
}
%    \end{macrocode}

%\iffalse
%</package>
%\fi
%
\endinput
|\\
|\childdocforward{|\textit{main}|}|
\end{tabular}
\end{center}
%
Likewise, the following files |final|\textit{nn}|.tex|
compile the final version of the child document
|child|\textit{nn}|.tex|:
%
\begin{center}
\begin{tabular}{l}
|\def\version{final}|\\
|% \iffalse
%
% childdoc.dtx Copyright (C) 2017-2018 Niklas Beisert
%
% This work may be distributed and/or modified under the
% conditions of the LaTeX Project Public License, either version 1.3
% of this license or (at your option) any later version.
% The latest version of this license is in
%   http://www.latex-project.org/lppl.txt
% and version 1.3 or later is part of all distributions of LaTeX
% version 2005/12/01 or later.
%
% This work has the LPPL maintenance status `maintained'.
%
% The Current Maintainer of this work is Niklas Beisert.
%
% This work consists of the files childdoc.dtx and childdoc.ins
% and the derived files childdoc.def and cdocsamp.tex with
% cdocsch1.tex, cdocsch2.tex, cdocsdrf.tex, cdocsfn1.tex, cdocsfn2.tex.
%
%<package>\ifdefined\childdocmain\endinput\fi
%<package>\ProvidesFile{childdoc.def}[2018/12/30 v2.0 child document driver]
%<samplemain>\ProvidesFile{cdocsamp.tex}[2018/12/30 v2.0 sample for childdoc]
%<*driver>
%\ProvidesFile{childdoc.drv}[2018/12/30 v2.0 childdoc reference manual file]
\PassOptionsToClass{10pt,a4paper}{article}
\documentclass{ltxdoc}

\usepackage[margin=35mm]{geometry}
\usepackage{hyperref}
\usepackage{hyperxmp}
\usepackage[usenames]{color}

\hypersetup{colorlinks=true}
\hypersetup{pdfstartview=FitH}
\hypersetup{pdfpagemode=UseNone}
\hypersetup{pdfsource={}}
\hypersetup{pdflang={en-UK}}
\hypersetup{pdfcopyright={Copyright 2017-2018 Niklas Beisert.
  This work may be distributed and/or modified under the
  conditions of the LaTeX Project Public License, either version 1.3
  of this license or (at your option) any later version.}}
\hypersetup{pdflicenseurl={http://www.latex-project.org/lppl.txt}}
\hypersetup{pdfcontactaddress={ETH Zurich, ITP, HIT K,
  Wolfgang-Pauli-Strasse 27}}
\hypersetup{pdfcontactpostcode={8093}}
\hypersetup{pdfcontactcity={Zurich}}
\hypersetup{pdfcontactcountry={Switzerland}}
\hypersetup{pdfcontactemail={nbeisert@itp.phys.ethz.ch}}
\hypersetup{pdfcontacturl={http://people.phys.ethz.ch/\xmptilde nbeisert/}}

\newcommand{\secref}[1]{\hyperref[#1]{section \ref*{#1}}}

\parskip1ex
\parindent0pt
\let\olditemize\itemize
\def\itemize{\olditemize\parskip0pt}

\begin{document}

\title{The \textsf{childdoc} Package}
\hypersetup{pdftitle={The childdoc Package}}
\author{Niklas Beisert\\[2ex]
  Institut f\"ur Theoretische Physik\\
  Eidgen\"ossische Technische Hochschule Z\"urich\\
  Wolfgang-Pauli-Strasse 27, 8093 Z\"urich, Switzerland\\[1ex]
  \href{mailto:nbeisert@itp.phys.ethz.ch}
  {\texttt{nbeisert@itp.phys.ethz.ch}}}
\hypersetup{pdfauthor={Niklas Beisert}}
\hypersetup{pdfsubject={Manual for the LaTeX2e Package childdoc}}
\date{30 December 2018, \textsf{v2.0}}
\maketitle

\begin{abstract}\noindent
\textsf{childdoc} is a \LaTeXe{} package
that enables the direct compilation
of document sections included by |\include|
to individual files.
\end{abstract}

\begingroup
\parskip0ex
\tableofcontents
\endgroup

%%%%%%%%%%%%%%%%%%%%%%%%%%%%%%%%%%%%%%%%%%%%%%%%%%%%%%%%%%%%%%%%%%%%%%%%%%%%%%%%
%%%%%%%%%%%%%%%%%%%%%%%%%%%%%%%%%%%%%%%%%%%%%%%%%%%%%%%%%%%%%%%%%%%%%%%%%%%%%%%%
\section{Introduction}

\LaTeX{} provides a mechanism to structure a large document (such as a book)
into a main file and several child files (containing the chapters)
using the |\include| command.
This mechanism is beneficial for documents
which span hundreds of pages in order to
make the source file(s) more manageable.
Moreover, compilation can be restricted to
selected child files by means of the |\includeonly| command.
The latter feature can be used to reduce the compilation time while editing
(this was significantly more useful in the earlier days of \LaTeX{})
or to generate a smaller document which is easier to navigate.
Another application of |\includeonly| is to generate
documents consisting of selected parts of the complete document.

However, there are a few drawbacks of the plain |\include| mechanism:
\begin{itemize}
\item
The child files cannot be compiled on their own,
they can only be compiled via the main file.
A naive editing environment
(such as a text editor with an option
to have the current file processed by \LaTeX)
may require one to switch to the main file before compiling;
attempting to compile the child file produces errors.
\item
The main file must be modified (each time)
to adjust the |\includeonly| command
to the present needs. This easily leaves the main file in a messy state.
\item
The generated document will always carry the filename
of the main document. This is inconvenient if
several child files are to be compiled and
to be kept for distribution.
\end{itemize}

The present package provides a simple interface
to make child files individually compilable by \LaTeX{}.
Compiling a child file then has the same effect as compiling
the main file with an |\includeonly| command
to select the appropriate child.
Moreover the generated document will carry the name of the child
rather than the main file.
This resolves all three above issues.

This feature is meant to make the editing of books,
thesis documents and lecture notes somewhat more convenient.
However, the package can also be used efficiently for
composing a series of documents (such as exercise sheets)
which are typically distributed individually.
It then assists the author in generating the individual documents
(potentially in different versions)
as well as a document containing the collected series.
Another application is in developing style files
or other kinds of included material
where compilation of the style file could redirect
to a sample or test file.

%%%%%%%%%%%%%%%%%%%%%%%%%%%%%%%%%%%%%%%%%%%%%%%%%%%%%%%%%%%%%%%%%%%%%%%%%%%%%%%%
%%%%%%%%%%%%%%%%%%%%%%%%%%%%%%%%%%%%%%%%%%%%%%%%%%%%%%%%%%%%%%%%%%%%%%%%%%%%%%%%
\section{Usage}

First of all, the package \textsf{childdoc} is \emph{not} a standard
\LaTeXe{} |.sty| style file! Therefore it needs to be invoked in
a non-standard way.

%%%%%%%%%%%%%%%%%%%%%%%%%%%%%%%%%%%%%%%%%%%%%%%%%%%%%%%%%%%%%%%%%%%%%%%%%%%%%%%%
\subsection{Included Files}
\label{sec:include}

%%%%%%%%%%%%%%%%%%%%%%%%%%%%%%%%%%%%%%%%
\DescribeMacro{\childdocmain}
To use the package, add the commands
\begin{center}
\begin{tabular}{l}
|\input{childdoc.def}|\\
|\childdocmain{}|\\
\end{tabular}
\end{center}
at the very top of the main \LaTeX{} file,
in particular \emph{before} the |\documentclass| statement!
The argument of |\childdocmain| should be left empty
(but it must be present).

%%%%%%%%%%%%%%%%%%%%%%%%%%%%%%%%%%%%%%%%
\DescribeMacro{\childdocof}
Furthermore, add the commands
\begin{center}
\begin{tabular}{l}
|\input{childdoc.def}|\\
|\childdocof{|\textit{main}|}|\\
\end{tabular}
\end{center}
at the top of every child file \textit{child}
which is included by |\include{|\textit{child}|}|
from within the main file
(or at least for those files to be compiled individually).
The argument \textit{main} must be the filename of the main file.

There are a couple of
considerations in setting up the main and child documents:

%%%%%%%%%%%%%%%%%%%%%%%%%%%%%%%%%%%%%%%%
\paragraph{Restrictions.}

Please note the following restrictions:
\begin{itemize}
\item
|\childdocmain| must be called with one argument \textit{main}
to ensure compatibility with earlier version of the package.
It must either be empty (|\childdocmain{}|)
or precisely match the filename of the main file in which it is specified.
See \secref{sec:detection} for further information.
\item
The filename \textit{main} must be specified without the |.tex| extension.
\item
The filename \textit{main} is case sensitive
(even in case-insensitive file systems)
due to internal string comparison.
\item
The argument \textit{main} should be fully expanded, it cannot be a macro.
\item
Subdirectories and special characters should be avoided in filenames.
\item
The command |\childdocmain{|\textit{main}|}| must be followed by a whitespace.
It should not be followed immediately by another command
or by a comment mark `|%|'.
This is because the \TeX{} parser reads the token immediately following
the argument of |\childdocmain| and puts it
at the beginning of every child section;
however, a white\-space is ignored.
\end{itemize}

%%%%%%%%%%%%%%%%%%%%%%%%%%%%%%%%%%%%%%%%
\paragraph{Content of Main File.}

It is advisable to place all content in the child files included by |\include|.
Any output contained in the main file will appear in all child documents
unless suppressed manually;
it cannot be suppressed automatically by the |\includeonly| directive
and thus should normally be avoided.
A method to include some content in the main file
by means of conditional processing is described in \secref{sec:conditional}.

%%%%%%%%%%%%%%%%%%%%%%%%%%%%%%%%%%%%%%%%
\paragraph{Page Numbering.}

When only a part of the document is compiled,
the appropriate numbering of pages
(as well as other status parameters)
is determined from the |.aux| files.
The latter contain information from previous passes.
However this information needs to propagate through
all intermediate child documents.
Therefore the page numbering in child documents may well
be inconsistent until the complete document is compiled at least once.

A useful (if unconventional) way to always ensure a consistent
page numbering is to restart the numbering in each child document
and denote the pages by `\textit{child}|.|\textit{page}'
where \textit{child} represents the chapter/section number of the child file.
This can be achieved by the command
|\numberwithin{page}{|\textit{child}|}|
of the \textsf{amsmath} package
where \textit{child} can be |chapter| or |section|
depending on the chosen structuring.
Alternatively, one can modify the macro |\thepage| appropriately
and reset the counter |page| at the start of each child file.

%%%%%%%%%%%%%%%%%%%%%%%%%%%%%%%%%%%%%%%%%%%%%%%%%%%%%%%%%%%%%%%%%%%%%%%%%%%%%%%%
\subsection{Conditional Processing}
\label{sec:conditional}

The package provides a mechanism to compile different versions
of a document. To customise the versions further some conditional processing
can come in handy to distinguish which version is being compiled.
The package provides two macros to describe the compilation context:

%%%%%%%%%%%%%%%%%%%%%%%%%%%%%%%%%%%%%%%%
\DescribeMacro{\ifchilddoc}
The conditional |\ifchilddoc| distinguishes between the compilation of
child documents and the main document:
%
\begin{center}
|\ifchilddoc |\textit{child-code}| |[|\||else |\textit{main-code}]| \||fi|
\end{center}

%%%%%%%%%%%%%%%%%%%%%%%%%%%%%%%%%%%%%%%%
\DescribeMacro{\childdocname}
\DescribeMacro{\childdocjob}
The macro |\childdocname| contains the filename (without extension)
of the main or child file being processed.
Note that |\childdocjob| will always contain the name of the main file.

%%%%%%%%%%%%%%%%%%%%%%%%%%%%%%%%%%%%%%%%
\paragraph{Title Page.}

Conditional processing can be used to include a title or banner page
in the main document when proper precautions are taken.
Importantly, the code in the main file should ensure that the page counter
(as well as other status parameters which are stored in the |.aux| files)
takes the same value after the conditional processing.
Otherwise the page numbers may take divergent values
depending on which part is compiled.

For example, a title page could be declared by:
%
\begin{center}
\begin{tabular}{l}
|\ifchilddoc\||else|\\
|\addtocounter{page}{-1}|\\
\textit{code for title page}\\
|\newpage|\\
|\||fi|
\end{tabular}
\end{center}
%
A banner page for the child documents can be generated by:
%
\begin{center}
\begin{tabular}{l}
|\ifchilddoc|\\
|\addtocounter{page}{-1}|\\
\textit{code for banner page}\\
|\newpage|\\
|\||fi|
\end{tabular}
\end{center}
%
Here one could write a message such as:
\begin{center}
|This is the part \childdocname{} of \childdocjob{}.|
\end{center}

%%%%%%%%%%%%%%%%%%%%%%%%%%%%%%%%%%%%%%%%%%%%%%%%%%%%%%%%%%%%%%%%%%%%%%%%%%%%%%%%
\subsection{Flags}
\label{sec:flags}

The package makes it easy to generate different versions
of the main or child documents.
To this end compilation flags can be defined
and assigned different default values.
They will be particularly useful in conjunction
with the forwarding mechanism described in \secref{sec:forward}.

For example, it may be useful to have a flag |\version|
which can be set to |draft| or |final|.
The document source will contain some conditional code
depending on the value of |\version|.
Suppose further, the flag should default to |final| for the main file
and to |draft| for child files
which is a natural assignment for editing the document.
This is achieved by placing the following code
in the preamble of the main document
(below the |\childdocmain| directive):
%
\begin{center}
\begin{tabular}{l}
|\ifchilddoc|\\
|\providecommand{\version}{draft}|\\
|\||else|\\
|\providecommand{\version}{final}|\\
|\||fi|
\end{tabular}
\end{center}
%
The definition by |\providecommand| makes sure
that previous definitions are not overwritten.
Further statements |\providecommand{\version}{...}|
can thus be added before the above code to override it.

For the main file, one might add a line
(between |\childdocmain| and the above block)
%
\begin{center}
|%\ifchilddoc\||else\providecommand{\version}{draft}\||fi|
\end{center}
%
which can be uncommented to produce a draft version.
Likewise one can add a line to the very top of a child file
(above the |\childdocof{|\textit{main}|}| directive)
%
\begin{center}
|%\providecommand{\version}{final}|
\end{center}
%
which can be uncommented to produce the final version of this child document.

%%%%%%%%%%%%%%%%%%%%%%%%%%%%%%%%%%%%%%%%%%%%%%%%%%%%%%%%%%%%%%%%%%%%%%%%%%%%%%%%
\subsection{Forwarding}
\label{sec:forward}

Different versions of the main or child documents
using compilation flags as described in \secref{sec:flags}
can be (permanently) stored in different files
for convenient compilation, viewing and distribution.
To this end, the package defines a command
to pass on compilation to a different file:

%%%%%%%%%%%%%%%%%%%%%%%%%%%%%%%%%%%%%%%%
\DescribeMacro{\childdocforward}
The command |\childdocforward| redirects processing to
another source file:
%
\begin{center}
\begin{tabular}{l}
|\input{childdoc.def}|\\
|\childdocforward[|\textit{main}|]{|\textit{dest}|}|\\
\end{tabular}
\end{center}
%
The argument \textit{dest} is the destination file
(without extension).
It should be the main file or one of the child files.
Note that further \textsf{childdoc} directives
such as |\childdocof| and |\childdocforward|
in the indicated file will be processed in this form.
The optional argument \textit{main}
passes on directly to the main file \textit{main}
while pretending to compile the child \textit{dest}.
This form behaves as if \textit{dest}
issues |\childdocof{|\textit{main}|}| right away,
and no further \textsf{childdoc} directives will be processed.

%%%%%%%%%%%%%%%%%%%%%%%%%%%%%%%%%%%%%%%%
\DescribeMacro{\...prefix}
In the alternative form |\childdocforwardprefix|,
%
\begin{center}
\begin{tabular}{l}
|\input{childdoc.def}|\\
|\childdocforwardprefix[|\textit{main}|]{|\textit{prefix}|}{|\textit{dest}|}|
\end{tabular}
\end{center}
%
the destination file is determined by a pattern
depending on the current file:
To make this work, the current file must be called
`{\textit{prefix}\hspace{0.2em}\textit{suffix}}'
with \textit{prefix} matching precisely the argument.
Processing is then passed on to the file
`{\textit{dest}\hspace{0.2em}\textit{suffix}}'.
Surely, the same effect is achieved by
directly specifying the
argument `{\textit{dest}\hspace{0.2em}\textit{suffix}}'
in the first form.
However, that requires to set up a different file
for each child. With the alternative form of the command
all these files can have exactly the same content
which simplifies setting them up and maintaining them.

For example, the following file |draft.tex|
with a compilation flag |\version| as described in \secref{sec:flags}
compiles the main document as a draft:
%
\begin{center}
\begin{tabular}{l}
|\def\version{draft}|\\
|\input{childdoc.def}|\\
|\childdocforward{|\textit{main}|}|
\end{tabular}
\end{center}
%
Likewise, the following files |final|\textit{nn}|.tex|
compile the final version of the child document
|child|\textit{nn}|.tex|:
%
\begin{center}
\begin{tabular}{l}
|\def\version{final}|\\
|\input{childdoc.def}|\\
|\childdocforwardprefix{final}{child}|
\end{tabular}
\end{center}
%

Note that when several versions of a main file and/or of each child file
are to be generated, it may be convenient to set up a |Makefile| or
shell script to automatise the process.

%%%%%%%%%%%%%%%%%%%%%%%%%%%%%%%%%%%%%%%%%%%%%%%%%%%%%%%%%%%%%%%%%%%%%%%%%%%%%%%%
\subsection{Command Line Processing}
\label{sec:commandline}

The effect of redirection files can also be achieved by invoking
the \LaTeX{} compiler with a more elaborate command line.
Most conveniently this should be done as part
of a shell script or a |Makefile|.

When using \textsf{childdoc} in the main file, the following
command lines effectively perform a redirection
(note that depending on the shell being used,
backslashes may have to be doubled: `|\|' $\to$ `|\\|'):
%
\begin{center}
|... -jobname "|\textit{target}|" |\\|"|[\textit{flags}]%
|\input{childdoc.def}\childdocforward[|\textit{main}|]{|\textit{dest}|}"|
\end{center}
%
Here \textit{target} is the name of the output file,
\textit{main} is the name of the main file
and \textit{dest} is the name of the main or child file to be processed
(all filenames without extensions).
The optional argument \textit{main} can be omitted
if \textit{main} matches \textit{dest}.
Optionally, compilation \textit{flags} can be defined via |\def| commands.
This command line makes the \TeX{} engine believe
it is compiling the file \textit{target}
whose content is specified as the latter parameter.
The provided code then forwards the processing to
\textit{main} or \textit{dest} as described in \secref{sec:forward}.

%%%%%%%%%%%%%%%%%%%%%%%%%%%%%%%%%%%%%%%%%%%%%%%%%%%%%%%%%%%%%%%%%%%%%%%%%%%%%%%%
\subsection{Include by Input}
\label{sec:input}

Including child documents by |\include| has some restrictions by design.
Most notably, the content of a child document always occupies
its own set of pages; pages cannot be shared between child documents.
Usually, this behaviour makes perfect sense
because each child document contain an essential part of the document.
However, in some situations it may be desirable to compose
a document from a collection of parts
without having mandatory page breaks between then.
For this case, the package
provides a mechanism to include parts
by |\input| which can also be processed individually.
However, by construction this mechanism
requires manual handling of the content to be output.

%%%%%%%%%%%%%%%%%%%%%%%%%%%%%%%%%%%%%%%%
\DescribeMacro{\ifchilddocmanual}
The main file should be prepared as usual, see \secref{sec:include}.
However, the document body must make a distinction
between processing of an individual part and of the main document, e.g.:
%
\begin{center}
\begin{tabular}{l}
|\ifchilddocmanual|\\
|\input{\childdocname}|\\
|\||else|\\
\textit{document body with }|\input{|\textit{part}|}|\\
|\||fi|
\end{tabular}
\end{center}
%
The conditional |\ifchilddocmanual| is true whenever
a part to be included by |\input| is being compiled,
and the name of the part is stored in |\childdocname|.

%%%%%%%%%%%%%%%%%%%%%%%%%%%%%%%%%%%%%%%%
\DescribeMacro{\childdocby}
Each part to be included by |\input| should start with:
%
\begin{center}
\begin{tabular}{l}
|\input{childdoc.def}|\\
|\childdocby{|\textit{main}|}|\\
\end{tabular}
\end{center}
%
The directive |\childdocby| is similar to |\childdocof|
described in \secref{sec:include},
but the subsequent selection of content must be done manually.
To that end, both |\ifchilddoc| and |\ifchilddocmanual|
will be true upon processing of a part,
and the name of the part is stored in |\childdocname|.
Note that |\jobname| will be set to the filename of the current part
so that each part receives an individual |.aux| file
that does not interfere with the |.aux| file(s) of the main document.
This behaviour can be altered by the alternative form
|\childdocby[*]{|\textit{main}|}| (with a non-empty optional argument)
which uses the |.aux| file of the main document
by setting |\jobname| to \textit{main}.

%%%%%%%%%%%%%%%%%%%%%%%%%%%%%%%%%%%%%%%%%%%%%%%%%%%%%%%%%%%%%%%%%%%%%%%%%%%%%%%%
\subsection{Driver Development}
\label{sec:driver}

The \textsf{childdoc} mechanism can also be use for the development
of definition files such as \LaTeX{} styles or classes.
This case differs from the above setup with multiple parts
included by |\include| in that no |\includeonly| should be invoked.
This can be achieved by starting the include file
(before |\ProvidesPackage|) with:
%
\begin{center}
\begin{tabular}{l}
|\input{childdoc.def}|\\
|\childdocforward{|\textit{main}|}|\\
\end{tabular}
\end{center}
%
or alternatively with:
%
\begin{center}
\begin{tabular}{l}
|\input{childdoc.def}|\\
|\childdocby{|\textit{main}|}|\\
\end{tabular}
\end{center}
%
Both forms have slightly different effects as described above.
The main file is prepared as usual, see \secref{sec:include}.

%%%%%%%%%%%%%%%%%%%%%%%%%%%%%%%%%%%%%%%%%%%%%%%%%%%%%%%%%%%%%%%%%%%%%%%%%%%%%%%%
\subsection{Legacy Detection}
\label{sec:detection}

The directive |\childdocmain| in the main file can detect
whether the complete document or merely a child is to be compiled
even without using the directive |\childdocof|.
This method is deprecated because it is less robust
and there is no compelling reason to use it;
it is merely provided for backward compatibility
and it may be removed in future versions.

If the detection mechanism is to be used,
it is mandatory to correctly specify
the filename of the main file as the argument of |\childdocmain|:
%
\begin{center}
\begin{tabular}{l}
|\input{childdoc.def}|\\
|\childdocmain{|\textit{main}|}|\\
\end{tabular}
\end{center}
%
If |\jobname| does not match the argument \textit{main} of |\childdocmain|,
it is assumed that |\jobname| points to the child file to be compiled.
When using |\childdocmain| with the main file specified as argument,
it suffices to start a child file
with just |\input{|\textit{main}|}|
without loading of the package and using |\childdocof|.
If instead all processing is done
with the appropriate \textsf{childdoc} directives,
the argument of \textit{main} of |\childdocmain| can be empty.

An alternative version of the command line processing described
in \secref{sec:commandline} using the detection mechanism reads:
%
\begin{center}
|... -jobname "|\textit{target}|" "|[\textit{flags}]%
[|\def\jobname{|\textit{dest}|}|]|\input{|\textit{main}|}"|
\end{center}

%%%%%%%%%%%%%%%%%%%%%%%%%%%%%%%%%%%%%%%%%%%%%%%%%%%%%%%%%%%%%%%%%%%%%%%%%%%%%%%%
\subsection{Manual Code}
\label{sec:manual}

In case one cannot be certain whether the definitions file |childdoc.def|
is installed on the target \TeX{} distribution
and one prefers not to ship it,
it is conceivable to paste a few relevant commands into the sources.

To that end, drop all statements |\input{childdoc.def}|
and perform the replacements as outlined below.
Instead of |\childdocmain{|\textit{main}|}| add the following code
to the top of the main file:
%
\begin{center}
\begin{tabular}{l}
|\||ifdefined\childdocname\endinput\||fi\newif\ifchilddoc|\\
|\edef\childdocname{\scantokens\expandafter{\jobname\noexpand}}|\\
|\def\childdocmain{|\textit{main}|}\||ifx\childdocmain\childdocname\||else|\\
|\childdoctrue\includeonly{\childdocname}\let\jobname\childdocmain\||fi|\\
\end{tabular}
\end{center}
%
Instead of |\childdocof{|\textit{main}|}| just include the main file
at the top of each child file:
%
\begin{center}
|\input{|\textit{main}|}|
\end{center}
%
A simple redirection |\childdocforward{|\textit{dest}|}| is achieved by:
%
\begin{center}
|\def\jobname{|\textit{dest}|}\input{\jobname}|
\end{center}
%
The redirection with prefix
|\childdocforwardprefix[|\textit{prefix}|]{|\textit{dest}|}|
is accomplished by:
%
\begin{center}
\begin{tabular}{l}
|{\edef\jobname{\scantokens\expandafter{\jobname\noexpand}}|\\
|\def\redirectjob |\textit{prefix}|#1~~~{\gdef\jobname{|\textit{dest}|#1}}|\\
|\expandafter\redirectjob\jobname~~~}\input{\jobname}|
\end{tabular}
\end{center}

In an alternative approach,
child documents can be compiled by a specific command line
without additional code or specific definitions:
%
\begin{center}
|... -jobname "|\textit{target}|" "|[\textit{flags}]%
|\includeonly{|\textit{dest}|}\input{|\textit{main}|}"|
\end{center}
%

%%%%%%%%%%%%%%%%%%%%%%%%%%%%%%%%%%%%%%%%%%%%%%%%%%%%%%%%%%%%%%%%%%%%%%%%%%%%%%%%
%%%%%%%%%%%%%%%%%%%%%%%%%%%%%%%%%%%%%%%%%%%%%%%%%%%%%%%%%%%%%%%%%%%%%%%%%%%%%%%%
\section{Information}

%%%%%%%%%%%%%%%%%%%%%%%%%%%%%%%%%%%%%%%%%%%%%%%%%%%%%%%%%%%%%%%%%%%%%%%%%%%%%%%%
\subsection{Copyright}

Copyright \copyright{} 2017--2018 Niklas Beisert

This work may be distributed and/or modified under the
conditions of the \LaTeX{} Project Public License, either version 1.3
of this license or (at your option) any later version.
The latest version of this license is in
  \url{http://www.latex-project.org/lppl.txt}
and version 1.3 or later is part of all distributions of \LaTeX{}
version 2005/12/01 or later.

This work has the LPPL maintenance status `maintained'.

The Current Maintainer of this work is Niklas Beisert.

This work consists of the files |README.txt|, |childdoc.ins| and |childdoc.dtx|
as well as the derived files |childdoc.def|, |cdocsamp.tex|
with |cdocsch1.tex|, |cdocsch2.tex|, |cdocspt3.tex|, |cdocspt4.tex|,
|cdocsdrf.tex|, |cdocsfn1.tex|, |cdocsfn2.tex|
as well as |childdoc.pdf|.

%%%%%%%%%%%%%%%%%%%%%%%%%%%%%%%%%%%%%%%%%%%%%%%%%%%%%%%%%%%%%%%%%%%%%%%%%%%%%%%%
\subsection{Files and Installation}

The package consists of the files:
%
\begin{center}
\begin{tabular}{ll}
    |README.txt|   & readme file \\
    |childdoc.ins| & installation file \\
    |childdoc.dtx| & source file \\
    |childdoc.def| & definition file \\
    |cdocsamp.tex| & sample main file \\
    |cdocsch1.tex| & sample include file \\
    |cdocsch2.tex| & sample include file \\
    |cdocspt3.tex| & sample part file \\
    |cdocspt4.tex| & sample part file \\
    |cdocsdrf.tex| & sample redirection file \\
    |cdocsfn1.tex| & sample redirection file \\
    |cdocsfn2.tex| & sample redirection file \\
    |childdoc.pdf| & manual
\end{tabular}
\end{center}
%
The distribution consists of the files
|README.txt|, |childdoc.ins| and |childdoc.dtx|.
%
\begin{itemize}
\item
Run (pdf)\LaTeX{} on |childdoc.dtx|
to compile the manual |childdoc.pdf| (this file).
\item
Run \LaTeX{} on |childdoc.ins| to create the definitions file |childdoc.def|
and the sample |cdocsamp.tex| with include files
|cdocsch1.tex|, |cdocsch2.tex|, |cdocspt3.tex|, |cdocspt4.tex|,
|cdocsdrf.tex|, |cdocsfn1.tex|, |cdocsfn2.tex|.
Then copy the file |childdoc.def| to an appropriate directory of your \LaTeX{}
distribution, e.g.\ \textit{texmf-root}|/tex/latex/childdoc|.
\end{itemize}

%%%%%%%%%%%%%%%%%%%%%%%%%%%%%%%%%%%%%%%%%%%%%%%%%%%%%%%%%%%%%%%%%%%%%%%%%%%%%%%%
\subsection{Related CTAN Packages}

There are several other packages which offer a similar functionality:
%
\begin{itemize}
\item
The packages
\href{http://ctan.org/pkg/docmute}{\textsf{docmute}},
\href{http://ctan.org/pkg/includex}{\textsf{includex}} and
\href{http://ctan.org/pkg/standalone}{\textsf{standalone}}
provide commands to include only the document body of
a child file thus allowing both files to be compiled individually.
\item
The packages \href{http://ctan.org/pkg/subdocs}{\textsf{subdocs}}
and \href{http://ctan.org/pkg/subfiles}{\textsf{subfiles}}
provide structures in which the main and child documents can be
encapsulated and allowing them to be compiled individually.
The inclusion mechanism is different from the conventional |\include|.
\item
The package \href{http://ctan.org/pkg/combine}{\textsf{combine}}
is an elaborate solution to combine several documents into one.
\end{itemize}
%
See also the CTAN topic \href{http://ctan.org/topic/subdocs}{\textsf{subdocs}}
for further related packages.
The present package differs from the above solutions in that
a document structure constructed with the conventional |\include| mechanism
just needs two extra commands at the top of every file
such that all constituent files can be compiled individually.

%%%%%%%%%%%%%%%%%%%%%%%%%%%%%%%%%%%%%%%%%%%%%%%%%%%%%%%%%%%%%%%%%%%%%%%%%%%%%%%%
%\subsection{Feature Suggestions}
%
%The following is a list of features which may be useful for future
%versions of this package:
%%
%\begin{itemize}
%\item
%\ldots
%\end{itemize}

%%%%%%%%%%%%%%%%%%%%%%%%%%%%%%%%%%%%%%%%%%%%%%%%%%%%%%%%%%%%%%%%%%%%%%%%%%%%%%%%
\subsection{Revision History}

%%%%%%%%%%%%%%%%%%%%%%%%%%%%%%%%%%%%%%%%
\paragraph{v2.0:} 2018/12/30

\begin{itemize}
\item
immediate forward processing
\item
added |\childdocby| mechanism
\item
manual restructured
\end{itemize}

%%%%%%%%%%%%%%%%%%%%%%%%%%%%%%%%%%%%%%%%
\paragraph{v1.6:} 2018/01/17

\begin{itemize}
\item
application for development of include files
\item
corrections to manual
\end{itemize}

%%%%%%%%%%%%%%%%%%%%%%%%%%%%%%%%%%%%%%%%
\paragraph{v1.5:} 2017/05/21

\begin{itemize}
\item
more complete structuring introduced
\item
|\childdocof| introduced
\item
|\childdoc| renamed to |\childdocmain|
\item
|\childredirect| renamed to |\childdocforward| and |\childdocforwardprefix|
and functionality expanded
\end{itemize}

%%%%%%%%%%%%%%%%%%%%%%%%%%%%%%%%%%%%%%%%
\paragraph{v1.0:} 2017/04/27

\begin{itemize}
\item
manual and install package
\item
first version published on CTAN
\end{itemize}

%%%%%%%%%%%%%%%%%%%%%%%%%%%%%%%%%%%%%%%%
\paragraph{v0.6:} 2017/04/26

\begin{itemize}
\item
redirection mechanism added
\end{itemize}

%%%%%%%%%%%%%%%%%%%%%%%%%%%%%%%%%%%%%%%%
\paragraph{v0.5:} 2017/04/26

\begin{itemize}
\item
functionality in definition file
\end{itemize}


%%%%%%%%%%%%%%%%%%%%%%%%%%%%%%%%%%%%%%%%%%%%%%%%%%%%%%%%%%%%%%%%%%%%%%%%%%%%%%%%
%%%%%%%%%%%%%%%%%%%%%%%%%%%%%%%%%%%%%%%%%%%%%%%%%%%%%%%%%%%%%%%%%%%%%%%%%%%%%%%%
%%%%%%%%%%%%%%%%%%%%%%%%%%%%%%%%%%%%%%%%%%%%%%%%%%%%%%%%%%%%%%%%%%%%%%%%%%%%%%%%
\appendix

\settowidth\MacroIndent{\rmfamily\scriptsize 000\ }

 \DocInput{childdoc.dtx}

\end{document}
%</driver>
% \fi
%
% %%%%%%%%%%%%%%%%%%%%%%%%%%%%%%%%%%%%%%%%%%%%%%%%%%%%%%%%%%%%%%%%%%%%%%%%%%%%%%
% %%%%%%%%%%%%%%%%%%%%%%%%%%%%%%%%%%%%%%%%%%%%%%%%%%%%%%%%%%%%%%%%%%%%%%%%%%%%%%
% \section{Sample}
%\iffalse
%<*samplemain>
%\fi
%
% The following presents a sample document
% with two chapters, two parts, a title page,
% a compile flag as well as three forwarding files to set the flag.
% It consists of eight |.tex| files:
% \begin{center}
% \begin{tabular}{ll}
% |cdocsamp.tex|&main file\\
% |cdocsch1.tex|&include file for chapter 1\\
% |cdocsch2.tex|&include file for chapter 2\\
% |cdocspt3.tex|&include file for part 3\\
% |cdocspt4.tex|&include file for part 4\\
% |cdocsdrf.tex|&forwarding file for main file in draft mode\\
% |cdocsfi1.tex|&forwarding file for final version of chapter 1\\
% |cdocsfi2.tex|&forwarding file for final version of chapter 2\\
% \end{tabular}
% \end{center}
% Each of the eight files can be compiled directly by the \LaTeX{} compiler.
%
% %%%%%%%%%%%%%%%%%%%%%%%%%%%%%%%%%%%%%%
% \paragraph{Main File.}
%
% The main file is called |cdocsamp.tex|.
%
% Load the \textsf{childdoc} definitions and
% declare the filename for the main document:
%    \begin{macrocode}
\input{childdoc.def}
\childdocmain{}
%    \end{macrocode}

% Optional override for |\version| flag:
%    \begin{macrocode}
%%\ifchilddoc\else\providecommand{\version}{draft}\fi
%    \end{macrocode}

% Define the default values for the |\version| flag
% (|final| for the main file and |draft| for childs):
%    \begin{macrocode}
\ifchilddoc
\providecommand{\version}{draft}
\else
\providecommand{\version}{final}
\fi
%    \end{macrocode}

% Load the standard document class:
%    \begin{macrocode}
\documentclass[12pt]{article}
%    \end{macrocode}

% Start the document body:
%    \begin{macrocode}
\begin{document}
%    \end{macrocode}

% Declare a title page.
% Print title, part of document being processed and version flag:
%    \begin{macrocode}
\addtocounter{page}{-1}
\begin{center}
{\LARGE\bfseries{}childdoc example\par}
\vspace{1cm}
\ifchilddoc
\ifchilddocmanual part\else chapter\fi:
`\childdocname' of `\childdocjob'\par
\else
main document: `\childdocjob'\par
\fi
version: \version\par
\end{center}
\newpage
%    \end{macrocode}

% Manually include selected file,
% otherwise process as usual:
%    \begin{macrocode}
\ifchilddocmanual
\section*{part `\childdocname'}
\input{\childdocname}
\else
%    \end{macrocode}

% Include the two chapters:
%    \begin{macrocode}
\include{cdocsch1}
\include{cdocsch2}
%    \end{macrocode}

% Include the two parts unless only chapters should be displayed:
%    \begin{macrocode}
\ifchilddoc\else
\section{part three}
\input{cdocspt3}
\section{part four}
\input{cdocspt4}
\fi
%    \end{macrocode}

% Process as usual until here:
%    \begin{macrocode}
\fi
%    \end{macrocode}

% End of document body:
%    \begin{macrocode}
\end{document}
%    \end{macrocode}
%\iffalse
%</samplemain>
%\fi
%
% %%%%%%%%%%%%%%%%%%%%%%%%%%%%%%%%%%%%%%
% \paragraph{Chapter Include Files.}
%
% The include files are called |cdocsch1.tex| and |cdocsch2.tex|.
%
%\iffalse
%<*samplechap1|samplechap2>
%\fi

% Optional override for |\version| flag:
%    \begin{macrocode}
%%\providecommand{\version}{final}
%    \end{macrocode}

% Include the main document:
%    \begin{macrocode}
\input{childdoc.def}
\childdocof{cdocsamp}
%    \end{macrocode}

%\iffalse
%</samplechap1|samplechap2>
%\fi
%
%\iffalse
%<*samplechap1>
%\fi
% Some text for chapter 1:
%    \begin{macrocode}
\section{one}
some text in chapter one
%    \end{macrocode}

%\iffalse
%</samplechap1>
%\fi
% Some text for chapter 2:
%\iffalse
%<*samplechap2>
%\fi
%    \begin{macrocode}
\section{two}
more text in chapter two
%    \end{macrocode}

%\iffalse
%</samplechap2>
%\fi
%
% %%%%%%%%%%%%%%%%%%%%%%%%%%%%%%%%%%%%%%
% \paragraph{Part Include Files.}
%
% The include files are called |cdocspt3.tex| and |cdocspt4.tex|.
%
%\iffalse
%<*samplepart3|samplepart4>
%\fi

% Optional override for |\version| flag:
%    \begin{macrocode}
%%\providecommand{\version}{final}
%    \end{macrocode}

% Include the main document:
%    \begin{macrocode}
\input{childdoc.def}
\childdocby{cdocsamp}
%    \end{macrocode}

%\iffalse
%</samplepart3|samplepart4>
%\fi
%
%\iffalse
%<*samplepart3>
%\fi
% Some text for part 3:
%    \begin{macrocode}
some text in part three
%    \end{macrocode}

%\iffalse
%</samplepart3>
%\fi
% Some text for part 4:
%\iffalse
%<*samplepart4>
%\fi
%    \begin{macrocode}
more text in part four
%    \end{macrocode}

%\iffalse
%</samplepart4>
%\fi
%
% %%%%%%%%%%%%%%%%%%%%%%%%%%%%%%%%%%%%%%
% \paragraph{Forwarding for a Complete Draft.}
%
% The following forwarding file |cdocsdrf.tex|
% compiles the main document in draft mode:
%\iffalse
%<*sampledraft>
%\fi
%    \begin{macrocode}
\def\version{draft}
\input{childdoc.def}
\childdocforward{cdocsamp}
%    \end{macrocode}

%\iffalse
%</sampledraft>
%\fi
%
% %%%%%%%%%%%%%%%%%%%%%%%%%%%%%%%%%%%%%%
% \paragraph{Forwarding for Final Version of the Chapters.}
%
% The following forwarding files |cdocsfn1.tex| and |cdocsfn2.tex|
% (with identical content)
% compile the final versions of the child documents
% |cdocsch1.tex| and |cdocsch2.tex|, respectively:
%\iffalse
%<*samplefinal>
%\fi
%    \begin{macrocode}
\def\version{final}
\input{childdoc.def}
\childdocforwardprefix[cdocsamp]{cdocsfn}{cdocsch}
%    \end{macrocode}

%\iffalse
%</samplefinal>
%\fi
%
% %%%%%%%%%%%%%%%%%%%%%%%%%%%%%%%%%%%%%%
% \paragraph{Command Line Processing.}
%
% The following three command lines generate the output files
% |cdocscld|, |cdocscl1| and |cdocscl2|
% which should be identical to
% |cdocsdrf|, |cdocsch1| and |cdocsfn2|, respectively:
% \begin{center}
% \begin{tabular}{l}
% |latex -jobname cdocscld \|\\
% |  "\def\version{draft}\input{childdoc.def}\childdocforward{cdocsamp}"|\\
% |latex -jobname cdocscl1 \|\\
% |  "\input{childdoc.def}\childdocforward[cdocsamp]{cdocsch1}"|\\
% |latex -jobname cdocscl2 \|\\
% |  "\def\version{final}\input{childdoc.def}\childdocforward{cdocsch2}"|
% \end{tabular}
% \end{center}
% Note that the trailing backslash on each first line
% merely continues the input to the second line
% (for convenient cut ant paste).
% Furthermore, the command |latex| can be replaced by any
% of its alternative versions such as |pdflatex|.
%
% %%%%%%%%%%%%%%%%%%%%%%%%%%%%%%%%%%%%%%%%%%%%%%%%%%%%%%%%%%%%%%%%%%%%%%%%%%%%%%
% %%%%%%%%%%%%%%%%%%%%%%%%%%%%%%%%%%%%%%%%%%%%%%%%%%%%%%%%%%%%%%%%%%%%%%%%%%%%%%
% \section{Implementation}
%\iffalse
%<*package>
%\fi
%
% This section describes the definitions file |childdoc.def|.

% The definitions cannot be loaded using |\usepackage| or |\RequirePackage|
% which has a mechanism to prevent loading a style file more than once.
% When loading the definitions by means of |\input|
% multiple instances have to be prevented manually:
%\iffalse
%This code needs to be before the `\ProvidesFile' directive
%which is defined at the beginning of this file.
%Therefore it is also placed there and commented out here.
%</package>
%<*discard>
%\fi
%    \begin{macrocode}
\ifdefined\childdocmain\endinput\fi
%    \end{macrocode}
%\iffalse
%</discard>
%<*package>
%\fi
%
% \macro{\ifchilddoc}
% \macro{\ifchilddocmanual}
% The conditional |\ifchilddoc| tells whether a
% child (true) or main (false) document is being compiled.
% The conditional |\ifchilddocmanual| tells whether
% the |\includeonly| mechanism is used (false) or
% the selection of child files must be performed manually (true).
% The definitions initialise to false:
%    \begin{macrocode}
\newif\ifchilddoc
\newif\ifchilddocmanual
%    \end{macrocode}

% \macro{\childdocname}
% \macro{\childdocjob}
% The macro |\childdocname| stores the name of the main document
% to be compiled. The macro |\childdocjob| stores the name of
% the document on which the \LaTeX{} compiler was originally invoked.
% The content of |\jobname| cannot be compared
% to filenames specified in the source due to different catcodes.
% The following code rescans |\jobname|, stores the result
% in |\childdocname| and saves a copy in |\childdocjob|:
%    \begin{macrocode}
\edef\childdocname{\scantokens\expandafter{\jobname\noexpand}}
\let\childdocjob\childdocname
%    \end{macrocode}

% \macro{\childdocdisable}
% The macro |\childdocdisable| prevents the main file
% from being processed more than once.
% At this stage, the main document command |\childdocmain|
% is assumed to be called once again where it should do nothing.
% Any subsequent call to it should prevent
% a secondary processing of the main document
% It overwrites the forwarding commands
% |\childdocof| and |\childdocforward|
% with empty macros to prevent further inclusions of the main document:
%    \begin{macrocode}
\newcommand{\childdocdisable}
{
  \renewcommand{\childdocmain}[1]{\renewcommand{\childdocmain}[1]{\endinput}}
  \renewcommand{\childdocof}[1]{}
  \renewcommand{\childdocby}[2][]{}
  \renewcommand{\childdocforward}[2][]{}
  \renewcommand{\childdocdisable}{}
}
%    \end{macrocode}

% \macro{\childdocmain}
% The macro |\childdocmain| is to be called at the top of the main file
% with nothing or the main filename (without extension) as argument.
% First, it breaks loops.
% If the argument is not empty and does not match |\childdocname|
% (which is set by the first inclusion of |childdoc.def|),
% |\ifchilddoc| is set to true, |\includeonly| is applied to the child file
% and |\jobname| is set to the main file
% (for proper handling of |.aux| files):
%    \begin{macrocode}
\newcommand{\childdocmain}[1]
{
  \childdocdisable\childdocmain{}
  \if?#1?\else
    \begingroup
      \def\childdoctmp{#1}
      \ifx\childdoctmp\childdocname
        \def\childdoctmp{}
      \else
        \def\childdoctmp
        {
          \childdoctrue
          \includeonly{\childdocname}
          \def\childdocjob{#1}
          \def\jobname{#1}
        }
      \fi
      \expandafter
    \endgroup
    \childdoctmp
  \fi
}
%    \end{macrocode}

% \macro{\childdocof}
% The command |\childdocof| redirects
% compilation to the main file |#1|.
%    \begin{macrocode}
\newcommand{\childdocof}[1]
{
  \childdocdisable
  \childdoctrue
  \includeonly{\childdocname}
  \def\jobname{#1}
  \def\childdocjob{#1}
  \input{#1}
}
%    \end{macrocode}

% \macro{\childdocby}
% The command |\childdocby| ....
%    \begin{macrocode}
\newcommand{\childdocby}[2][]
{
  \childdocdisable
  \childdoctrue
  \childdocmanualtrue
  \if?#1?\else
    \def\jobname{#2}
  \fi
  \def\childdocjob{#2}
  \input{#2}
  \endinput
}
%    \end{macrocode}

% \macro{\childdocforward}
% The command |\childdocforward| redirects
% compilation to the main file or
% (if the optional argument is given) a child file.
% Parameters are set as if the main file
% or a child file starting with |\childdocof| was compiled.
% Then compilation is handed over to the main file:
%    \begin{macrocode}
\newcommand{\childdocforward}[2][]
{
  \begingroup
    \if?#1?
      \def\childdoctmp
      {
        \def\childdocname{#2}
        \def\childdocjob{#2}
        \def\jobname{#2}
        \input{#2}
        \endinput
      }
    \else
      \def\childdoctmp
      {
        \childdocdisable
        \def\childdocname{#2}
        \childdoctrue
        \includeonly{#2}
        \def\childdocjob{#1}
        \def\jobname{#1}
        \input{#1}
        \endinput
      }
    \fi
    \expandafter
  \endgroup
  \childdoctmp
}
%    \end{macrocode}

% \macro{\childdocforwardprefix}
% The command |\childdocforwardprefix| redirects
% compilation to the main or a child file by means of a pattern.
% The prefix |#1| in the current filename is replaced by |#2|
% and the suffix of the current filename is kept
% (it is assumed that the filename does not contain the substring `|~~~|'
% which is used as a delimiter).
% Compilation is handed over to the new file by |\childdocforward|:
%    \begin{macrocode}
\newcommand{\childdocforwardprefix}[3][]
{
  \begingroup
    \def\childdocextract #2##1~~~{\def\childdoctmp{\childdocforward[#1]{#3##1}}}
    \expandafter\childdocextract\childdocname~~~
    \expandafter
  \endgroup
  \childdoctmp
}
%    \end{macrocode}

% \macro{\childdoc}
% The deprecated macro |\childdoc| is a legacy version of |\childdocmain|:
%    \begin{macrocode}
\newcommand{\childdoc}{\childdocmain}
%    \end{macrocode}

% \macro{\childdocredirect}
% The deprecated macro |\childdocredirect| is a legacy version
% of |\childdocforward| and |\childdocforwardprefix|:
%    \begin{macrocode}
\newcommand{\childdocredirect}[2][]
{
  \begingroup
    \if?#1?
      \def\childdoctmp{\childdocforward{#2}}
    \else
      \def\childdoctmp{\childdocforwardprefix{#1}{#2}}
    \fi
    \expandafter
  \endgroup
  \childdoctmp
}
%    \end{macrocode}

%\iffalse
%</package>
%\fi
%
\endinput
|\\
|\childdocforwardprefix{final}{child}|
\end{tabular}
\end{center}
%

Note that when several versions of a main file and/or of each child file
are to be generated, it may be convenient to set up a |Makefile| or
shell script to automatise the process.

%%%%%%%%%%%%%%%%%%%%%%%%%%%%%%%%%%%%%%%%%%%%%%%%%%%%%%%%%%%%%%%%%%%%%%%%%%%%%%%%
\subsection{Command Line Processing}
\label{sec:commandline}

The effect of redirection files can also be achieved by invoking
the \LaTeX{} compiler with a more elaborate command line.
Most conveniently this should be done as part
of a shell script or a |Makefile|.

When using \textsf{childdoc} in the main file, the following
command lines effectively perform a redirection
(note that depending on the shell being used,
backslashes may have to be doubled: `|\|' $\to$ `|\\|'):
%
\begin{center}
|... -jobname "|\textit{target}|" |\\|"|[\textit{flags}]%
|% \iffalse
%
% childdoc.dtx Copyright (C) 2017-2018 Niklas Beisert
%
% This work may be distributed and/or modified under the
% conditions of the LaTeX Project Public License, either version 1.3
% of this license or (at your option) any later version.
% The latest version of this license is in
%   http://www.latex-project.org/lppl.txt
% and version 1.3 or later is part of all distributions of LaTeX
% version 2005/12/01 or later.
%
% This work has the LPPL maintenance status `maintained'.
%
% The Current Maintainer of this work is Niklas Beisert.
%
% This work consists of the files childdoc.dtx and childdoc.ins
% and the derived files childdoc.def and cdocsamp.tex with
% cdocsch1.tex, cdocsch2.tex, cdocsdrf.tex, cdocsfn1.tex, cdocsfn2.tex.
%
%<package>\ifdefined\childdocmain\endinput\fi
%<package>\ProvidesFile{childdoc.def}[2018/12/30 v2.0 child document driver]
%<samplemain>\ProvidesFile{cdocsamp.tex}[2018/12/30 v2.0 sample for childdoc]
%<*driver>
%\ProvidesFile{childdoc.drv}[2018/12/30 v2.0 childdoc reference manual file]
\PassOptionsToClass{10pt,a4paper}{article}
\documentclass{ltxdoc}

\usepackage[margin=35mm]{geometry}
\usepackage{hyperref}
\usepackage{hyperxmp}
\usepackage[usenames]{color}

\hypersetup{colorlinks=true}
\hypersetup{pdfstartview=FitH}
\hypersetup{pdfpagemode=UseNone}
\hypersetup{pdfsource={}}
\hypersetup{pdflang={en-UK}}
\hypersetup{pdfcopyright={Copyright 2017-2018 Niklas Beisert.
  This work may be distributed and/or modified under the
  conditions of the LaTeX Project Public License, either version 1.3
  of this license or (at your option) any later version.}}
\hypersetup{pdflicenseurl={http://www.latex-project.org/lppl.txt}}
\hypersetup{pdfcontactaddress={ETH Zurich, ITP, HIT K,
  Wolfgang-Pauli-Strasse 27}}
\hypersetup{pdfcontactpostcode={8093}}
\hypersetup{pdfcontactcity={Zurich}}
\hypersetup{pdfcontactcountry={Switzerland}}
\hypersetup{pdfcontactemail={nbeisert@itp.phys.ethz.ch}}
\hypersetup{pdfcontacturl={http://people.phys.ethz.ch/\xmptilde nbeisert/}}

\newcommand{\secref}[1]{\hyperref[#1]{section \ref*{#1}}}

\parskip1ex
\parindent0pt
\let\olditemize\itemize
\def\itemize{\olditemize\parskip0pt}

\begin{document}

\title{The \textsf{childdoc} Package}
\hypersetup{pdftitle={The childdoc Package}}
\author{Niklas Beisert\\[2ex]
  Institut f\"ur Theoretische Physik\\
  Eidgen\"ossische Technische Hochschule Z\"urich\\
  Wolfgang-Pauli-Strasse 27, 8093 Z\"urich, Switzerland\\[1ex]
  \href{mailto:nbeisert@itp.phys.ethz.ch}
  {\texttt{nbeisert@itp.phys.ethz.ch}}}
\hypersetup{pdfauthor={Niklas Beisert}}
\hypersetup{pdfsubject={Manual for the LaTeX2e Package childdoc}}
\date{30 December 2018, \textsf{v2.0}}
\maketitle

\begin{abstract}\noindent
\textsf{childdoc} is a \LaTeXe{} package
that enables the direct compilation
of document sections included by |\include|
to individual files.
\end{abstract}

\begingroup
\parskip0ex
\tableofcontents
\endgroup

%%%%%%%%%%%%%%%%%%%%%%%%%%%%%%%%%%%%%%%%%%%%%%%%%%%%%%%%%%%%%%%%%%%%%%%%%%%%%%%%
%%%%%%%%%%%%%%%%%%%%%%%%%%%%%%%%%%%%%%%%%%%%%%%%%%%%%%%%%%%%%%%%%%%%%%%%%%%%%%%%
\section{Introduction}

\LaTeX{} provides a mechanism to structure a large document (such as a book)
into a main file and several child files (containing the chapters)
using the |\include| command.
This mechanism is beneficial for documents
which span hundreds of pages in order to
make the source file(s) more manageable.
Moreover, compilation can be restricted to
selected child files by means of the |\includeonly| command.
The latter feature can be used to reduce the compilation time while editing
(this was significantly more useful in the earlier days of \LaTeX{})
or to generate a smaller document which is easier to navigate.
Another application of |\includeonly| is to generate
documents consisting of selected parts of the complete document.

However, there are a few drawbacks of the plain |\include| mechanism:
\begin{itemize}
\item
The child files cannot be compiled on their own,
they can only be compiled via the main file.
A naive editing environment
(such as a text editor with an option
to have the current file processed by \LaTeX)
may require one to switch to the main file before compiling;
attempting to compile the child file produces errors.
\item
The main file must be modified (each time)
to adjust the |\includeonly| command
to the present needs. This easily leaves the main file in a messy state.
\item
The generated document will always carry the filename
of the main document. This is inconvenient if
several child files are to be compiled and
to be kept for distribution.
\end{itemize}

The present package provides a simple interface
to make child files individually compilable by \LaTeX{}.
Compiling a child file then has the same effect as compiling
the main file with an |\includeonly| command
to select the appropriate child.
Moreover the generated document will carry the name of the child
rather than the main file.
This resolves all three above issues.

This feature is meant to make the editing of books,
thesis documents and lecture notes somewhat more convenient.
However, the package can also be used efficiently for
composing a series of documents (such as exercise sheets)
which are typically distributed individually.
It then assists the author in generating the individual documents
(potentially in different versions)
as well as a document containing the collected series.
Another application is in developing style files
or other kinds of included material
where compilation of the style file could redirect
to a sample or test file.

%%%%%%%%%%%%%%%%%%%%%%%%%%%%%%%%%%%%%%%%%%%%%%%%%%%%%%%%%%%%%%%%%%%%%%%%%%%%%%%%
%%%%%%%%%%%%%%%%%%%%%%%%%%%%%%%%%%%%%%%%%%%%%%%%%%%%%%%%%%%%%%%%%%%%%%%%%%%%%%%%
\section{Usage}

First of all, the package \textsf{childdoc} is \emph{not} a standard
\LaTeXe{} |.sty| style file! Therefore it needs to be invoked in
a non-standard way.

%%%%%%%%%%%%%%%%%%%%%%%%%%%%%%%%%%%%%%%%%%%%%%%%%%%%%%%%%%%%%%%%%%%%%%%%%%%%%%%%
\subsection{Included Files}
\label{sec:include}

%%%%%%%%%%%%%%%%%%%%%%%%%%%%%%%%%%%%%%%%
\DescribeMacro{\childdocmain}
To use the package, add the commands
\begin{center}
\begin{tabular}{l}
|\input{childdoc.def}|\\
|\childdocmain{}|\\
\end{tabular}
\end{center}
at the very top of the main \LaTeX{} file,
in particular \emph{before} the |\documentclass| statement!
The argument of |\childdocmain| should be left empty
(but it must be present).

%%%%%%%%%%%%%%%%%%%%%%%%%%%%%%%%%%%%%%%%
\DescribeMacro{\childdocof}
Furthermore, add the commands
\begin{center}
\begin{tabular}{l}
|\input{childdoc.def}|\\
|\childdocof{|\textit{main}|}|\\
\end{tabular}
\end{center}
at the top of every child file \textit{child}
which is included by |\include{|\textit{child}|}|
from within the main file
(or at least for those files to be compiled individually).
The argument \textit{main} must be the filename of the main file.

There are a couple of
considerations in setting up the main and child documents:

%%%%%%%%%%%%%%%%%%%%%%%%%%%%%%%%%%%%%%%%
\paragraph{Restrictions.}

Please note the following restrictions:
\begin{itemize}
\item
|\childdocmain| must be called with one argument \textit{main}
to ensure compatibility with earlier version of the package.
It must either be empty (|\childdocmain{}|)
or precisely match the filename of the main file in which it is specified.
See \secref{sec:detection} for further information.
\item
The filename \textit{main} must be specified without the |.tex| extension.
\item
The filename \textit{main} is case sensitive
(even in case-insensitive file systems)
due to internal string comparison.
\item
The argument \textit{main} should be fully expanded, it cannot be a macro.
\item
Subdirectories and special characters should be avoided in filenames.
\item
The command |\childdocmain{|\textit{main}|}| must be followed by a whitespace.
It should not be followed immediately by another command
or by a comment mark `|%|'.
This is because the \TeX{} parser reads the token immediately following
the argument of |\childdocmain| and puts it
at the beginning of every child section;
however, a white\-space is ignored.
\end{itemize}

%%%%%%%%%%%%%%%%%%%%%%%%%%%%%%%%%%%%%%%%
\paragraph{Content of Main File.}

It is advisable to place all content in the child files included by |\include|.
Any output contained in the main file will appear in all child documents
unless suppressed manually;
it cannot be suppressed automatically by the |\includeonly| directive
and thus should normally be avoided.
A method to include some content in the main file
by means of conditional processing is described in \secref{sec:conditional}.

%%%%%%%%%%%%%%%%%%%%%%%%%%%%%%%%%%%%%%%%
\paragraph{Page Numbering.}

When only a part of the document is compiled,
the appropriate numbering of pages
(as well as other status parameters)
is determined from the |.aux| files.
The latter contain information from previous passes.
However this information needs to propagate through
all intermediate child documents.
Therefore the page numbering in child documents may well
be inconsistent until the complete document is compiled at least once.

A useful (if unconventional) way to always ensure a consistent
page numbering is to restart the numbering in each child document
and denote the pages by `\textit{child}|.|\textit{page}'
where \textit{child} represents the chapter/section number of the child file.
This can be achieved by the command
|\numberwithin{page}{|\textit{child}|}|
of the \textsf{amsmath} package
where \textit{child} can be |chapter| or |section|
depending on the chosen structuring.
Alternatively, one can modify the macro |\thepage| appropriately
and reset the counter |page| at the start of each child file.

%%%%%%%%%%%%%%%%%%%%%%%%%%%%%%%%%%%%%%%%%%%%%%%%%%%%%%%%%%%%%%%%%%%%%%%%%%%%%%%%
\subsection{Conditional Processing}
\label{sec:conditional}

The package provides a mechanism to compile different versions
of a document. To customise the versions further some conditional processing
can come in handy to distinguish which version is being compiled.
The package provides two macros to describe the compilation context:

%%%%%%%%%%%%%%%%%%%%%%%%%%%%%%%%%%%%%%%%
\DescribeMacro{\ifchilddoc}
The conditional |\ifchilddoc| distinguishes between the compilation of
child documents and the main document:
%
\begin{center}
|\ifchilddoc |\textit{child-code}| |[|\||else |\textit{main-code}]| \||fi|
\end{center}

%%%%%%%%%%%%%%%%%%%%%%%%%%%%%%%%%%%%%%%%
\DescribeMacro{\childdocname}
\DescribeMacro{\childdocjob}
The macro |\childdocname| contains the filename (without extension)
of the main or child file being processed.
Note that |\childdocjob| will always contain the name of the main file.

%%%%%%%%%%%%%%%%%%%%%%%%%%%%%%%%%%%%%%%%
\paragraph{Title Page.}

Conditional processing can be used to include a title or banner page
in the main document when proper precautions are taken.
Importantly, the code in the main file should ensure that the page counter
(as well as other status parameters which are stored in the |.aux| files)
takes the same value after the conditional processing.
Otherwise the page numbers may take divergent values
depending on which part is compiled.

For example, a title page could be declared by:
%
\begin{center}
\begin{tabular}{l}
|\ifchilddoc\||else|\\
|\addtocounter{page}{-1}|\\
\textit{code for title page}\\
|\newpage|\\
|\||fi|
\end{tabular}
\end{center}
%
A banner page for the child documents can be generated by:
%
\begin{center}
\begin{tabular}{l}
|\ifchilddoc|\\
|\addtocounter{page}{-1}|\\
\textit{code for banner page}\\
|\newpage|\\
|\||fi|
\end{tabular}
\end{center}
%
Here one could write a message such as:
\begin{center}
|This is the part \childdocname{} of \childdocjob{}.|
\end{center}

%%%%%%%%%%%%%%%%%%%%%%%%%%%%%%%%%%%%%%%%%%%%%%%%%%%%%%%%%%%%%%%%%%%%%%%%%%%%%%%%
\subsection{Flags}
\label{sec:flags}

The package makes it easy to generate different versions
of the main or child documents.
To this end compilation flags can be defined
and assigned different default values.
They will be particularly useful in conjunction
with the forwarding mechanism described in \secref{sec:forward}.

For example, it may be useful to have a flag |\version|
which can be set to |draft| or |final|.
The document source will contain some conditional code
depending on the value of |\version|.
Suppose further, the flag should default to |final| for the main file
and to |draft| for child files
which is a natural assignment for editing the document.
This is achieved by placing the following code
in the preamble of the main document
(below the |\childdocmain| directive):
%
\begin{center}
\begin{tabular}{l}
|\ifchilddoc|\\
|\providecommand{\version}{draft}|\\
|\||else|\\
|\providecommand{\version}{final}|\\
|\||fi|
\end{tabular}
\end{center}
%
The definition by |\providecommand| makes sure
that previous definitions are not overwritten.
Further statements |\providecommand{\version}{...}|
can thus be added before the above code to override it.

For the main file, one might add a line
(between |\childdocmain| and the above block)
%
\begin{center}
|%\ifchilddoc\||else\providecommand{\version}{draft}\||fi|
\end{center}
%
which can be uncommented to produce a draft version.
Likewise one can add a line to the very top of a child file
(above the |\childdocof{|\textit{main}|}| directive)
%
\begin{center}
|%\providecommand{\version}{final}|
\end{center}
%
which can be uncommented to produce the final version of this child document.

%%%%%%%%%%%%%%%%%%%%%%%%%%%%%%%%%%%%%%%%%%%%%%%%%%%%%%%%%%%%%%%%%%%%%%%%%%%%%%%%
\subsection{Forwarding}
\label{sec:forward}

Different versions of the main or child documents
using compilation flags as described in \secref{sec:flags}
can be (permanently) stored in different files
for convenient compilation, viewing and distribution.
To this end, the package defines a command
to pass on compilation to a different file:

%%%%%%%%%%%%%%%%%%%%%%%%%%%%%%%%%%%%%%%%
\DescribeMacro{\childdocforward}
The command |\childdocforward| redirects processing to
another source file:
%
\begin{center}
\begin{tabular}{l}
|\input{childdoc.def}|\\
|\childdocforward[|\textit{main}|]{|\textit{dest}|}|\\
\end{tabular}
\end{center}
%
The argument \textit{dest} is the destination file
(without extension).
It should be the main file or one of the child files.
Note that further \textsf{childdoc} directives
such as |\childdocof| and |\childdocforward|
in the indicated file will be processed in this form.
The optional argument \textit{main}
passes on directly to the main file \textit{main}
while pretending to compile the child \textit{dest}.
This form behaves as if \textit{dest}
issues |\childdocof{|\textit{main}|}| right away,
and no further \textsf{childdoc} directives will be processed.

%%%%%%%%%%%%%%%%%%%%%%%%%%%%%%%%%%%%%%%%
\DescribeMacro{\...prefix}
In the alternative form |\childdocforwardprefix|,
%
\begin{center}
\begin{tabular}{l}
|\input{childdoc.def}|\\
|\childdocforwardprefix[|\textit{main}|]{|\textit{prefix}|}{|\textit{dest}|}|
\end{tabular}
\end{center}
%
the destination file is determined by a pattern
depending on the current file:
To make this work, the current file must be called
`{\textit{prefix}\hspace{0.2em}\textit{suffix}}'
with \textit{prefix} matching precisely the argument.
Processing is then passed on to the file
`{\textit{dest}\hspace{0.2em}\textit{suffix}}'.
Surely, the same effect is achieved by
directly specifying the
argument `{\textit{dest}\hspace{0.2em}\textit{suffix}}'
in the first form.
However, that requires to set up a different file
for each child. With the alternative form of the command
all these files can have exactly the same content
which simplifies setting them up and maintaining them.

For example, the following file |draft.tex|
with a compilation flag |\version| as described in \secref{sec:flags}
compiles the main document as a draft:
%
\begin{center}
\begin{tabular}{l}
|\def\version{draft}|\\
|\input{childdoc.def}|\\
|\childdocforward{|\textit{main}|}|
\end{tabular}
\end{center}
%
Likewise, the following files |final|\textit{nn}|.tex|
compile the final version of the child document
|child|\textit{nn}|.tex|:
%
\begin{center}
\begin{tabular}{l}
|\def\version{final}|\\
|\input{childdoc.def}|\\
|\childdocforwardprefix{final}{child}|
\end{tabular}
\end{center}
%

Note that when several versions of a main file and/or of each child file
are to be generated, it may be convenient to set up a |Makefile| or
shell script to automatise the process.

%%%%%%%%%%%%%%%%%%%%%%%%%%%%%%%%%%%%%%%%%%%%%%%%%%%%%%%%%%%%%%%%%%%%%%%%%%%%%%%%
\subsection{Command Line Processing}
\label{sec:commandline}

The effect of redirection files can also be achieved by invoking
the \LaTeX{} compiler with a more elaborate command line.
Most conveniently this should be done as part
of a shell script or a |Makefile|.

When using \textsf{childdoc} in the main file, the following
command lines effectively perform a redirection
(note that depending on the shell being used,
backslashes may have to be doubled: `|\|' $\to$ `|\\|'):
%
\begin{center}
|... -jobname "|\textit{target}|" |\\|"|[\textit{flags}]%
|\input{childdoc.def}\childdocforward[|\textit{main}|]{|\textit{dest}|}"|
\end{center}
%
Here \textit{target} is the name of the output file,
\textit{main} is the name of the main file
and \textit{dest} is the name of the main or child file to be processed
(all filenames without extensions).
The optional argument \textit{main} can be omitted
if \textit{main} matches \textit{dest}.
Optionally, compilation \textit{flags} can be defined via |\def| commands.
This command line makes the \TeX{} engine believe
it is compiling the file \textit{target}
whose content is specified as the latter parameter.
The provided code then forwards the processing to
\textit{main} or \textit{dest} as described in \secref{sec:forward}.

%%%%%%%%%%%%%%%%%%%%%%%%%%%%%%%%%%%%%%%%%%%%%%%%%%%%%%%%%%%%%%%%%%%%%%%%%%%%%%%%
\subsection{Include by Input}
\label{sec:input}

Including child documents by |\include| has some restrictions by design.
Most notably, the content of a child document always occupies
its own set of pages; pages cannot be shared between child documents.
Usually, this behaviour makes perfect sense
because each child document contain an essential part of the document.
However, in some situations it may be desirable to compose
a document from a collection of parts
without having mandatory page breaks between then.
For this case, the package
provides a mechanism to include parts
by |\input| which can also be processed individually.
However, by construction this mechanism
requires manual handling of the content to be output.

%%%%%%%%%%%%%%%%%%%%%%%%%%%%%%%%%%%%%%%%
\DescribeMacro{\ifchilddocmanual}
The main file should be prepared as usual, see \secref{sec:include}.
However, the document body must make a distinction
between processing of an individual part and of the main document, e.g.:
%
\begin{center}
\begin{tabular}{l}
|\ifchilddocmanual|\\
|\input{\childdocname}|\\
|\||else|\\
\textit{document body with }|\input{|\textit{part}|}|\\
|\||fi|
\end{tabular}
\end{center}
%
The conditional |\ifchilddocmanual| is true whenever
a part to be included by |\input| is being compiled,
and the name of the part is stored in |\childdocname|.

%%%%%%%%%%%%%%%%%%%%%%%%%%%%%%%%%%%%%%%%
\DescribeMacro{\childdocby}
Each part to be included by |\input| should start with:
%
\begin{center}
\begin{tabular}{l}
|\input{childdoc.def}|\\
|\childdocby{|\textit{main}|}|\\
\end{tabular}
\end{center}
%
The directive |\childdocby| is similar to |\childdocof|
described in \secref{sec:include},
but the subsequent selection of content must be done manually.
To that end, both |\ifchilddoc| and |\ifchilddocmanual|
will be true upon processing of a part,
and the name of the part is stored in |\childdocname|.
Note that |\jobname| will be set to the filename of the current part
so that each part receives an individual |.aux| file
that does not interfere with the |.aux| file(s) of the main document.
This behaviour can be altered by the alternative form
|\childdocby[*]{|\textit{main}|}| (with a non-empty optional argument)
which uses the |.aux| file of the main document
by setting |\jobname| to \textit{main}.

%%%%%%%%%%%%%%%%%%%%%%%%%%%%%%%%%%%%%%%%%%%%%%%%%%%%%%%%%%%%%%%%%%%%%%%%%%%%%%%%
\subsection{Driver Development}
\label{sec:driver}

The \textsf{childdoc} mechanism can also be use for the development
of definition files such as \LaTeX{} styles or classes.
This case differs from the above setup with multiple parts
included by |\include| in that no |\includeonly| should be invoked.
This can be achieved by starting the include file
(before |\ProvidesPackage|) with:
%
\begin{center}
\begin{tabular}{l}
|\input{childdoc.def}|\\
|\childdocforward{|\textit{main}|}|\\
\end{tabular}
\end{center}
%
or alternatively with:
%
\begin{center}
\begin{tabular}{l}
|\input{childdoc.def}|\\
|\childdocby{|\textit{main}|}|\\
\end{tabular}
\end{center}
%
Both forms have slightly different effects as described above.
The main file is prepared as usual, see \secref{sec:include}.

%%%%%%%%%%%%%%%%%%%%%%%%%%%%%%%%%%%%%%%%%%%%%%%%%%%%%%%%%%%%%%%%%%%%%%%%%%%%%%%%
\subsection{Legacy Detection}
\label{sec:detection}

The directive |\childdocmain| in the main file can detect
whether the complete document or merely a child is to be compiled
even without using the directive |\childdocof|.
This method is deprecated because it is less robust
and there is no compelling reason to use it;
it is merely provided for backward compatibility
and it may be removed in future versions.

If the detection mechanism is to be used,
it is mandatory to correctly specify
the filename of the main file as the argument of |\childdocmain|:
%
\begin{center}
\begin{tabular}{l}
|\input{childdoc.def}|\\
|\childdocmain{|\textit{main}|}|\\
\end{tabular}
\end{center}
%
If |\jobname| does not match the argument \textit{main} of |\childdocmain|,
it is assumed that |\jobname| points to the child file to be compiled.
When using |\childdocmain| with the main file specified as argument,
it suffices to start a child file
with just |\input{|\textit{main}|}|
without loading of the package and using |\childdocof|.
If instead all processing is done
with the appropriate \textsf{childdoc} directives,
the argument of \textit{main} of |\childdocmain| can be empty.

An alternative version of the command line processing described
in \secref{sec:commandline} using the detection mechanism reads:
%
\begin{center}
|... -jobname "|\textit{target}|" "|[\textit{flags}]%
[|\def\jobname{|\textit{dest}|}|]|\input{|\textit{main}|}"|
\end{center}

%%%%%%%%%%%%%%%%%%%%%%%%%%%%%%%%%%%%%%%%%%%%%%%%%%%%%%%%%%%%%%%%%%%%%%%%%%%%%%%%
\subsection{Manual Code}
\label{sec:manual}

In case one cannot be certain whether the definitions file |childdoc.def|
is installed on the target \TeX{} distribution
and one prefers not to ship it,
it is conceivable to paste a few relevant commands into the sources.

To that end, drop all statements |\input{childdoc.def}|
and perform the replacements as outlined below.
Instead of |\childdocmain{|\textit{main}|}| add the following code
to the top of the main file:
%
\begin{center}
\begin{tabular}{l}
|\||ifdefined\childdocname\endinput\||fi\newif\ifchilddoc|\\
|\edef\childdocname{\scantokens\expandafter{\jobname\noexpand}}|\\
|\def\childdocmain{|\textit{main}|}\||ifx\childdocmain\childdocname\||else|\\
|\childdoctrue\includeonly{\childdocname}\let\jobname\childdocmain\||fi|\\
\end{tabular}
\end{center}
%
Instead of |\childdocof{|\textit{main}|}| just include the main file
at the top of each child file:
%
\begin{center}
|\input{|\textit{main}|}|
\end{center}
%
A simple redirection |\childdocforward{|\textit{dest}|}| is achieved by:
%
\begin{center}
|\def\jobname{|\textit{dest}|}\input{\jobname}|
\end{center}
%
The redirection with prefix
|\childdocforwardprefix[|\textit{prefix}|]{|\textit{dest}|}|
is accomplished by:
%
\begin{center}
\begin{tabular}{l}
|{\edef\jobname{\scantokens\expandafter{\jobname\noexpand}}|\\
|\def\redirectjob |\textit{prefix}|#1~~~{\gdef\jobname{|\textit{dest}|#1}}|\\
|\expandafter\redirectjob\jobname~~~}\input{\jobname}|
\end{tabular}
\end{center}

In an alternative approach,
child documents can be compiled by a specific command line
without additional code or specific definitions:
%
\begin{center}
|... -jobname "|\textit{target}|" "|[\textit{flags}]%
|\includeonly{|\textit{dest}|}\input{|\textit{main}|}"|
\end{center}
%

%%%%%%%%%%%%%%%%%%%%%%%%%%%%%%%%%%%%%%%%%%%%%%%%%%%%%%%%%%%%%%%%%%%%%%%%%%%%%%%%
%%%%%%%%%%%%%%%%%%%%%%%%%%%%%%%%%%%%%%%%%%%%%%%%%%%%%%%%%%%%%%%%%%%%%%%%%%%%%%%%
\section{Information}

%%%%%%%%%%%%%%%%%%%%%%%%%%%%%%%%%%%%%%%%%%%%%%%%%%%%%%%%%%%%%%%%%%%%%%%%%%%%%%%%
\subsection{Copyright}

Copyright \copyright{} 2017--2018 Niklas Beisert

This work may be distributed and/or modified under the
conditions of the \LaTeX{} Project Public License, either version 1.3
of this license or (at your option) any later version.
The latest version of this license is in
  \url{http://www.latex-project.org/lppl.txt}
and version 1.3 or later is part of all distributions of \LaTeX{}
version 2005/12/01 or later.

This work has the LPPL maintenance status `maintained'.

The Current Maintainer of this work is Niklas Beisert.

This work consists of the files |README.txt|, |childdoc.ins| and |childdoc.dtx|
as well as the derived files |childdoc.def|, |cdocsamp.tex|
with |cdocsch1.tex|, |cdocsch2.tex|, |cdocspt3.tex|, |cdocspt4.tex|,
|cdocsdrf.tex|, |cdocsfn1.tex|, |cdocsfn2.tex|
as well as |childdoc.pdf|.

%%%%%%%%%%%%%%%%%%%%%%%%%%%%%%%%%%%%%%%%%%%%%%%%%%%%%%%%%%%%%%%%%%%%%%%%%%%%%%%%
\subsection{Files and Installation}

The package consists of the files:
%
\begin{center}
\begin{tabular}{ll}
    |README.txt|   & readme file \\
    |childdoc.ins| & installation file \\
    |childdoc.dtx| & source file \\
    |childdoc.def| & definition file \\
    |cdocsamp.tex| & sample main file \\
    |cdocsch1.tex| & sample include file \\
    |cdocsch2.tex| & sample include file \\
    |cdocspt3.tex| & sample part file \\
    |cdocspt4.tex| & sample part file \\
    |cdocsdrf.tex| & sample redirection file \\
    |cdocsfn1.tex| & sample redirection file \\
    |cdocsfn2.tex| & sample redirection file \\
    |childdoc.pdf| & manual
\end{tabular}
\end{center}
%
The distribution consists of the files
|README.txt|, |childdoc.ins| and |childdoc.dtx|.
%
\begin{itemize}
\item
Run (pdf)\LaTeX{} on |childdoc.dtx|
to compile the manual |childdoc.pdf| (this file).
\item
Run \LaTeX{} on |childdoc.ins| to create the definitions file |childdoc.def|
and the sample |cdocsamp.tex| with include files
|cdocsch1.tex|, |cdocsch2.tex|, |cdocspt3.tex|, |cdocspt4.tex|,
|cdocsdrf.tex|, |cdocsfn1.tex|, |cdocsfn2.tex|.
Then copy the file |childdoc.def| to an appropriate directory of your \LaTeX{}
distribution, e.g.\ \textit{texmf-root}|/tex/latex/childdoc|.
\end{itemize}

%%%%%%%%%%%%%%%%%%%%%%%%%%%%%%%%%%%%%%%%%%%%%%%%%%%%%%%%%%%%%%%%%%%%%%%%%%%%%%%%
\subsection{Related CTAN Packages}

There are several other packages which offer a similar functionality:
%
\begin{itemize}
\item
The packages
\href{http://ctan.org/pkg/docmute}{\textsf{docmute}},
\href{http://ctan.org/pkg/includex}{\textsf{includex}} and
\href{http://ctan.org/pkg/standalone}{\textsf{standalone}}
provide commands to include only the document body of
a child file thus allowing both files to be compiled individually.
\item
The packages \href{http://ctan.org/pkg/subdocs}{\textsf{subdocs}}
and \href{http://ctan.org/pkg/subfiles}{\textsf{subfiles}}
provide structures in which the main and child documents can be
encapsulated and allowing them to be compiled individually.
The inclusion mechanism is different from the conventional |\include|.
\item
The package \href{http://ctan.org/pkg/combine}{\textsf{combine}}
is an elaborate solution to combine several documents into one.
\end{itemize}
%
See also the CTAN topic \href{http://ctan.org/topic/subdocs}{\textsf{subdocs}}
for further related packages.
The present package differs from the above solutions in that
a document structure constructed with the conventional |\include| mechanism
just needs two extra commands at the top of every file
such that all constituent files can be compiled individually.

%%%%%%%%%%%%%%%%%%%%%%%%%%%%%%%%%%%%%%%%%%%%%%%%%%%%%%%%%%%%%%%%%%%%%%%%%%%%%%%%
%\subsection{Feature Suggestions}
%
%The following is a list of features which may be useful for future
%versions of this package:
%%
%\begin{itemize}
%\item
%\ldots
%\end{itemize}

%%%%%%%%%%%%%%%%%%%%%%%%%%%%%%%%%%%%%%%%%%%%%%%%%%%%%%%%%%%%%%%%%%%%%%%%%%%%%%%%
\subsection{Revision History}

%%%%%%%%%%%%%%%%%%%%%%%%%%%%%%%%%%%%%%%%
\paragraph{v2.0:} 2018/12/30

\begin{itemize}
\item
immediate forward processing
\item
added |\childdocby| mechanism
\item
manual restructured
\end{itemize}

%%%%%%%%%%%%%%%%%%%%%%%%%%%%%%%%%%%%%%%%
\paragraph{v1.6:} 2018/01/17

\begin{itemize}
\item
application for development of include files
\item
corrections to manual
\end{itemize}

%%%%%%%%%%%%%%%%%%%%%%%%%%%%%%%%%%%%%%%%
\paragraph{v1.5:} 2017/05/21

\begin{itemize}
\item
more complete structuring introduced
\item
|\childdocof| introduced
\item
|\childdoc| renamed to |\childdocmain|
\item
|\childredirect| renamed to |\childdocforward| and |\childdocforwardprefix|
and functionality expanded
\end{itemize}

%%%%%%%%%%%%%%%%%%%%%%%%%%%%%%%%%%%%%%%%
\paragraph{v1.0:} 2017/04/27

\begin{itemize}
\item
manual and install package
\item
first version published on CTAN
\end{itemize}

%%%%%%%%%%%%%%%%%%%%%%%%%%%%%%%%%%%%%%%%
\paragraph{v0.6:} 2017/04/26

\begin{itemize}
\item
redirection mechanism added
\end{itemize}

%%%%%%%%%%%%%%%%%%%%%%%%%%%%%%%%%%%%%%%%
\paragraph{v0.5:} 2017/04/26

\begin{itemize}
\item
functionality in definition file
\end{itemize}


%%%%%%%%%%%%%%%%%%%%%%%%%%%%%%%%%%%%%%%%%%%%%%%%%%%%%%%%%%%%%%%%%%%%%%%%%%%%%%%%
%%%%%%%%%%%%%%%%%%%%%%%%%%%%%%%%%%%%%%%%%%%%%%%%%%%%%%%%%%%%%%%%%%%%%%%%%%%%%%%%
%%%%%%%%%%%%%%%%%%%%%%%%%%%%%%%%%%%%%%%%%%%%%%%%%%%%%%%%%%%%%%%%%%%%%%%%%%%%%%%%
\appendix

\settowidth\MacroIndent{\rmfamily\scriptsize 000\ }

 \DocInput{childdoc.dtx}

\end{document}
%</driver>
% \fi
%
% %%%%%%%%%%%%%%%%%%%%%%%%%%%%%%%%%%%%%%%%%%%%%%%%%%%%%%%%%%%%%%%%%%%%%%%%%%%%%%
% %%%%%%%%%%%%%%%%%%%%%%%%%%%%%%%%%%%%%%%%%%%%%%%%%%%%%%%%%%%%%%%%%%%%%%%%%%%%%%
% \section{Sample}
%\iffalse
%<*samplemain>
%\fi
%
% The following presents a sample document
% with two chapters, two parts, a title page,
% a compile flag as well as three forwarding files to set the flag.
% It consists of eight |.tex| files:
% \begin{center}
% \begin{tabular}{ll}
% |cdocsamp.tex|&main file\\
% |cdocsch1.tex|&include file for chapter 1\\
% |cdocsch2.tex|&include file for chapter 2\\
% |cdocspt3.tex|&include file for part 3\\
% |cdocspt4.tex|&include file for part 4\\
% |cdocsdrf.tex|&forwarding file for main file in draft mode\\
% |cdocsfi1.tex|&forwarding file for final version of chapter 1\\
% |cdocsfi2.tex|&forwarding file for final version of chapter 2\\
% \end{tabular}
% \end{center}
% Each of the eight files can be compiled directly by the \LaTeX{} compiler.
%
% %%%%%%%%%%%%%%%%%%%%%%%%%%%%%%%%%%%%%%
% \paragraph{Main File.}
%
% The main file is called |cdocsamp.tex|.
%
% Load the \textsf{childdoc} definitions and
% declare the filename for the main document:
%    \begin{macrocode}
\input{childdoc.def}
\childdocmain{}
%    \end{macrocode}

% Optional override for |\version| flag:
%    \begin{macrocode}
%%\ifchilddoc\else\providecommand{\version}{draft}\fi
%    \end{macrocode}

% Define the default values for the |\version| flag
% (|final| for the main file and |draft| for childs):
%    \begin{macrocode}
\ifchilddoc
\providecommand{\version}{draft}
\else
\providecommand{\version}{final}
\fi
%    \end{macrocode}

% Load the standard document class:
%    \begin{macrocode}
\documentclass[12pt]{article}
%    \end{macrocode}

% Start the document body:
%    \begin{macrocode}
\begin{document}
%    \end{macrocode}

% Declare a title page.
% Print title, part of document being processed and version flag:
%    \begin{macrocode}
\addtocounter{page}{-1}
\begin{center}
{\LARGE\bfseries{}childdoc example\par}
\vspace{1cm}
\ifchilddoc
\ifchilddocmanual part\else chapter\fi:
`\childdocname' of `\childdocjob'\par
\else
main document: `\childdocjob'\par
\fi
version: \version\par
\end{center}
\newpage
%    \end{macrocode}

% Manually include selected file,
% otherwise process as usual:
%    \begin{macrocode}
\ifchilddocmanual
\section*{part `\childdocname'}
\input{\childdocname}
\else
%    \end{macrocode}

% Include the two chapters:
%    \begin{macrocode}
\include{cdocsch1}
\include{cdocsch2}
%    \end{macrocode}

% Include the two parts unless only chapters should be displayed:
%    \begin{macrocode}
\ifchilddoc\else
\section{part three}
\input{cdocspt3}
\section{part four}
\input{cdocspt4}
\fi
%    \end{macrocode}

% Process as usual until here:
%    \begin{macrocode}
\fi
%    \end{macrocode}

% End of document body:
%    \begin{macrocode}
\end{document}
%    \end{macrocode}
%\iffalse
%</samplemain>
%\fi
%
% %%%%%%%%%%%%%%%%%%%%%%%%%%%%%%%%%%%%%%
% \paragraph{Chapter Include Files.}
%
% The include files are called |cdocsch1.tex| and |cdocsch2.tex|.
%
%\iffalse
%<*samplechap1|samplechap2>
%\fi

% Optional override for |\version| flag:
%    \begin{macrocode}
%%\providecommand{\version}{final}
%    \end{macrocode}

% Include the main document:
%    \begin{macrocode}
\input{childdoc.def}
\childdocof{cdocsamp}
%    \end{macrocode}

%\iffalse
%</samplechap1|samplechap2>
%\fi
%
%\iffalse
%<*samplechap1>
%\fi
% Some text for chapter 1:
%    \begin{macrocode}
\section{one}
some text in chapter one
%    \end{macrocode}

%\iffalse
%</samplechap1>
%\fi
% Some text for chapter 2:
%\iffalse
%<*samplechap2>
%\fi
%    \begin{macrocode}
\section{two}
more text in chapter two
%    \end{macrocode}

%\iffalse
%</samplechap2>
%\fi
%
% %%%%%%%%%%%%%%%%%%%%%%%%%%%%%%%%%%%%%%
% \paragraph{Part Include Files.}
%
% The include files are called |cdocspt3.tex| and |cdocspt4.tex|.
%
%\iffalse
%<*samplepart3|samplepart4>
%\fi

% Optional override for |\version| flag:
%    \begin{macrocode}
%%\providecommand{\version}{final}
%    \end{macrocode}

% Include the main document:
%    \begin{macrocode}
\input{childdoc.def}
\childdocby{cdocsamp}
%    \end{macrocode}

%\iffalse
%</samplepart3|samplepart4>
%\fi
%
%\iffalse
%<*samplepart3>
%\fi
% Some text for part 3:
%    \begin{macrocode}
some text in part three
%    \end{macrocode}

%\iffalse
%</samplepart3>
%\fi
% Some text for part 4:
%\iffalse
%<*samplepart4>
%\fi
%    \begin{macrocode}
more text in part four
%    \end{macrocode}

%\iffalse
%</samplepart4>
%\fi
%
% %%%%%%%%%%%%%%%%%%%%%%%%%%%%%%%%%%%%%%
% \paragraph{Forwarding for a Complete Draft.}
%
% The following forwarding file |cdocsdrf.tex|
% compiles the main document in draft mode:
%\iffalse
%<*sampledraft>
%\fi
%    \begin{macrocode}
\def\version{draft}
\input{childdoc.def}
\childdocforward{cdocsamp}
%    \end{macrocode}

%\iffalse
%</sampledraft>
%\fi
%
% %%%%%%%%%%%%%%%%%%%%%%%%%%%%%%%%%%%%%%
% \paragraph{Forwarding for Final Version of the Chapters.}
%
% The following forwarding files |cdocsfn1.tex| and |cdocsfn2.tex|
% (with identical content)
% compile the final versions of the child documents
% |cdocsch1.tex| and |cdocsch2.tex|, respectively:
%\iffalse
%<*samplefinal>
%\fi
%    \begin{macrocode}
\def\version{final}
\input{childdoc.def}
\childdocforwardprefix[cdocsamp]{cdocsfn}{cdocsch}
%    \end{macrocode}

%\iffalse
%</samplefinal>
%\fi
%
% %%%%%%%%%%%%%%%%%%%%%%%%%%%%%%%%%%%%%%
% \paragraph{Command Line Processing.}
%
% The following three command lines generate the output files
% |cdocscld|, |cdocscl1| and |cdocscl2|
% which should be identical to
% |cdocsdrf|, |cdocsch1| and |cdocsfn2|, respectively:
% \begin{center}
% \begin{tabular}{l}
% |latex -jobname cdocscld \|\\
% |  "\def\version{draft}\input{childdoc.def}\childdocforward{cdocsamp}"|\\
% |latex -jobname cdocscl1 \|\\
% |  "\input{childdoc.def}\childdocforward[cdocsamp]{cdocsch1}"|\\
% |latex -jobname cdocscl2 \|\\
% |  "\def\version{final}\input{childdoc.def}\childdocforward{cdocsch2}"|
% \end{tabular}
% \end{center}
% Note that the trailing backslash on each first line
% merely continues the input to the second line
% (for convenient cut ant paste).
% Furthermore, the command |latex| can be replaced by any
% of its alternative versions such as |pdflatex|.
%
% %%%%%%%%%%%%%%%%%%%%%%%%%%%%%%%%%%%%%%%%%%%%%%%%%%%%%%%%%%%%%%%%%%%%%%%%%%%%%%
% %%%%%%%%%%%%%%%%%%%%%%%%%%%%%%%%%%%%%%%%%%%%%%%%%%%%%%%%%%%%%%%%%%%%%%%%%%%%%%
% \section{Implementation}
%\iffalse
%<*package>
%\fi
%
% This section describes the definitions file |childdoc.def|.

% The definitions cannot be loaded using |\usepackage| or |\RequirePackage|
% which has a mechanism to prevent loading a style file more than once.
% When loading the definitions by means of |\input|
% multiple instances have to be prevented manually:
%\iffalse
%This code needs to be before the `\ProvidesFile' directive
%which is defined at the beginning of this file.
%Therefore it is also placed there and commented out here.
%</package>
%<*discard>
%\fi
%    \begin{macrocode}
\ifdefined\childdocmain\endinput\fi
%    \end{macrocode}
%\iffalse
%</discard>
%<*package>
%\fi
%
% \macro{\ifchilddoc}
% \macro{\ifchilddocmanual}
% The conditional |\ifchilddoc| tells whether a
% child (true) or main (false) document is being compiled.
% The conditional |\ifchilddocmanual| tells whether
% the |\includeonly| mechanism is used (false) or
% the selection of child files must be performed manually (true).
% The definitions initialise to false:
%    \begin{macrocode}
\newif\ifchilddoc
\newif\ifchilddocmanual
%    \end{macrocode}

% \macro{\childdocname}
% \macro{\childdocjob}
% The macro |\childdocname| stores the name of the main document
% to be compiled. The macro |\childdocjob| stores the name of
% the document on which the \LaTeX{} compiler was originally invoked.
% The content of |\jobname| cannot be compared
% to filenames specified in the source due to different catcodes.
% The following code rescans |\jobname|, stores the result
% in |\childdocname| and saves a copy in |\childdocjob|:
%    \begin{macrocode}
\edef\childdocname{\scantokens\expandafter{\jobname\noexpand}}
\let\childdocjob\childdocname
%    \end{macrocode}

% \macro{\childdocdisable}
% The macro |\childdocdisable| prevents the main file
% from being processed more than once.
% At this stage, the main document command |\childdocmain|
% is assumed to be called once again where it should do nothing.
% Any subsequent call to it should prevent
% a secondary processing of the main document
% It overwrites the forwarding commands
% |\childdocof| and |\childdocforward|
% with empty macros to prevent further inclusions of the main document:
%    \begin{macrocode}
\newcommand{\childdocdisable}
{
  \renewcommand{\childdocmain}[1]{\renewcommand{\childdocmain}[1]{\endinput}}
  \renewcommand{\childdocof}[1]{}
  \renewcommand{\childdocby}[2][]{}
  \renewcommand{\childdocforward}[2][]{}
  \renewcommand{\childdocdisable}{}
}
%    \end{macrocode}

% \macro{\childdocmain}
% The macro |\childdocmain| is to be called at the top of the main file
% with nothing or the main filename (without extension) as argument.
% First, it breaks loops.
% If the argument is not empty and does not match |\childdocname|
% (which is set by the first inclusion of |childdoc.def|),
% |\ifchilddoc| is set to true, |\includeonly| is applied to the child file
% and |\jobname| is set to the main file
% (for proper handling of |.aux| files):
%    \begin{macrocode}
\newcommand{\childdocmain}[1]
{
  \childdocdisable\childdocmain{}
  \if?#1?\else
    \begingroup
      \def\childdoctmp{#1}
      \ifx\childdoctmp\childdocname
        \def\childdoctmp{}
      \else
        \def\childdoctmp
        {
          \childdoctrue
          \includeonly{\childdocname}
          \def\childdocjob{#1}
          \def\jobname{#1}
        }
      \fi
      \expandafter
    \endgroup
    \childdoctmp
  \fi
}
%    \end{macrocode}

% \macro{\childdocof}
% The command |\childdocof| redirects
% compilation to the main file |#1|.
%    \begin{macrocode}
\newcommand{\childdocof}[1]
{
  \childdocdisable
  \childdoctrue
  \includeonly{\childdocname}
  \def\jobname{#1}
  \def\childdocjob{#1}
  \input{#1}
}
%    \end{macrocode}

% \macro{\childdocby}
% The command |\childdocby| ....
%    \begin{macrocode}
\newcommand{\childdocby}[2][]
{
  \childdocdisable
  \childdoctrue
  \childdocmanualtrue
  \if?#1?\else
    \def\jobname{#2}
  \fi
  \def\childdocjob{#2}
  \input{#2}
  \endinput
}
%    \end{macrocode}

% \macro{\childdocforward}
% The command |\childdocforward| redirects
% compilation to the main file or
% (if the optional argument is given) a child file.
% Parameters are set as if the main file
% or a child file starting with |\childdocof| was compiled.
% Then compilation is handed over to the main file:
%    \begin{macrocode}
\newcommand{\childdocforward}[2][]
{
  \begingroup
    \if?#1?
      \def\childdoctmp
      {
        \def\childdocname{#2}
        \def\childdocjob{#2}
        \def\jobname{#2}
        \input{#2}
        \endinput
      }
    \else
      \def\childdoctmp
      {
        \childdocdisable
        \def\childdocname{#2}
        \childdoctrue
        \includeonly{#2}
        \def\childdocjob{#1}
        \def\jobname{#1}
        \input{#1}
        \endinput
      }
    \fi
    \expandafter
  \endgroup
  \childdoctmp
}
%    \end{macrocode}

% \macro{\childdocforwardprefix}
% The command |\childdocforwardprefix| redirects
% compilation to the main or a child file by means of a pattern.
% The prefix |#1| in the current filename is replaced by |#2|
% and the suffix of the current filename is kept
% (it is assumed that the filename does not contain the substring `|~~~|'
% which is used as a delimiter).
% Compilation is handed over to the new file by |\childdocforward|:
%    \begin{macrocode}
\newcommand{\childdocforwardprefix}[3][]
{
  \begingroup
    \def\childdocextract #2##1~~~{\def\childdoctmp{\childdocforward[#1]{#3##1}}}
    \expandafter\childdocextract\childdocname~~~
    \expandafter
  \endgroup
  \childdoctmp
}
%    \end{macrocode}

% \macro{\childdoc}
% The deprecated macro |\childdoc| is a legacy version of |\childdocmain|:
%    \begin{macrocode}
\newcommand{\childdoc}{\childdocmain}
%    \end{macrocode}

% \macro{\childdocredirect}
% The deprecated macro |\childdocredirect| is a legacy version
% of |\childdocforward| and |\childdocforwardprefix|:
%    \begin{macrocode}
\newcommand{\childdocredirect}[2][]
{
  \begingroup
    \if?#1?
      \def\childdoctmp{\childdocforward{#2}}
    \else
      \def\childdoctmp{\childdocforwardprefix{#1}{#2}}
    \fi
    \expandafter
  \endgroup
  \childdoctmp
}
%    \end{macrocode}

%\iffalse
%</package>
%\fi
%
\endinput
\childdocforward[|\textit{main}|]{|\textit{dest}|}"|
\end{center}
%
Here \textit{target} is the name of the output file,
\textit{main} is the name of the main file
and \textit{dest} is the name of the main or child file to be processed
(all filenames without extensions).
The optional argument \textit{main} can be omitted
if \textit{main} matches \textit{dest}.
Optionally, compilation \textit{flags} can be defined via |\def| commands.
This command line makes the \TeX{} engine believe
it is compiling the file \textit{target}
whose content is specified as the latter parameter.
The provided code then forwards the processing to
\textit{main} or \textit{dest} as described in \secref{sec:forward}.

%%%%%%%%%%%%%%%%%%%%%%%%%%%%%%%%%%%%%%%%%%%%%%%%%%%%%%%%%%%%%%%%%%%%%%%%%%%%%%%%
\subsection{Include by Input}
\label{sec:input}

Including child documents by |\include| has some restrictions by design.
Most notably, the content of a child document always occupies
its own set of pages; pages cannot be shared between child documents.
Usually, this behaviour makes perfect sense
because each child document contain an essential part of the document.
However, in some situations it may be desirable to compose
a document from a collection of parts
without having mandatory page breaks between then.
For this case, the package
provides a mechanism to include parts
by |\input| which can also be processed individually.
However, by construction this mechanism
requires manual handling of the content to be output.

%%%%%%%%%%%%%%%%%%%%%%%%%%%%%%%%%%%%%%%%
\DescribeMacro{\ifchilddocmanual}
The main file should be prepared as usual, see \secref{sec:include}.
However, the document body must make a distinction
between processing of an individual part and of the main document, e.g.:
%
\begin{center}
\begin{tabular}{l}
|\ifchilddocmanual|\\
|\input{\childdocname}|\\
|\||else|\\
\textit{document body with }|\input{|\textit{part}|}|\\
|\||fi|
\end{tabular}
\end{center}
%
The conditional |\ifchilddocmanual| is true whenever
a part to be included by |\input| is being compiled,
and the name of the part is stored in |\childdocname|.

%%%%%%%%%%%%%%%%%%%%%%%%%%%%%%%%%%%%%%%%
\DescribeMacro{\childdocby}
Each part to be included by |\input| should start with:
%
\begin{center}
\begin{tabular}{l}
|% \iffalse
%
% childdoc.dtx Copyright (C) 2017-2018 Niklas Beisert
%
% This work may be distributed and/or modified under the
% conditions of the LaTeX Project Public License, either version 1.3
% of this license or (at your option) any later version.
% The latest version of this license is in
%   http://www.latex-project.org/lppl.txt
% and version 1.3 or later is part of all distributions of LaTeX
% version 2005/12/01 or later.
%
% This work has the LPPL maintenance status `maintained'.
%
% The Current Maintainer of this work is Niklas Beisert.
%
% This work consists of the files childdoc.dtx and childdoc.ins
% and the derived files childdoc.def and cdocsamp.tex with
% cdocsch1.tex, cdocsch2.tex, cdocsdrf.tex, cdocsfn1.tex, cdocsfn2.tex.
%
%<package>\ifdefined\childdocmain\endinput\fi
%<package>\ProvidesFile{childdoc.def}[2018/12/30 v2.0 child document driver]
%<samplemain>\ProvidesFile{cdocsamp.tex}[2018/12/30 v2.0 sample for childdoc]
%<*driver>
%\ProvidesFile{childdoc.drv}[2018/12/30 v2.0 childdoc reference manual file]
\PassOptionsToClass{10pt,a4paper}{article}
\documentclass{ltxdoc}

\usepackage[margin=35mm]{geometry}
\usepackage{hyperref}
\usepackage{hyperxmp}
\usepackage[usenames]{color}

\hypersetup{colorlinks=true}
\hypersetup{pdfstartview=FitH}
\hypersetup{pdfpagemode=UseNone}
\hypersetup{pdfsource={}}
\hypersetup{pdflang={en-UK}}
\hypersetup{pdfcopyright={Copyright 2017-2018 Niklas Beisert.
  This work may be distributed and/or modified under the
  conditions of the LaTeX Project Public License, either version 1.3
  of this license or (at your option) any later version.}}
\hypersetup{pdflicenseurl={http://www.latex-project.org/lppl.txt}}
\hypersetup{pdfcontactaddress={ETH Zurich, ITP, HIT K,
  Wolfgang-Pauli-Strasse 27}}
\hypersetup{pdfcontactpostcode={8093}}
\hypersetup{pdfcontactcity={Zurich}}
\hypersetup{pdfcontactcountry={Switzerland}}
\hypersetup{pdfcontactemail={nbeisert@itp.phys.ethz.ch}}
\hypersetup{pdfcontacturl={http://people.phys.ethz.ch/\xmptilde nbeisert/}}

\newcommand{\secref}[1]{\hyperref[#1]{section \ref*{#1}}}

\parskip1ex
\parindent0pt
\let\olditemize\itemize
\def\itemize{\olditemize\parskip0pt}

\begin{document}

\title{The \textsf{childdoc} Package}
\hypersetup{pdftitle={The childdoc Package}}
\author{Niklas Beisert\\[2ex]
  Institut f\"ur Theoretische Physik\\
  Eidgen\"ossische Technische Hochschule Z\"urich\\
  Wolfgang-Pauli-Strasse 27, 8093 Z\"urich, Switzerland\\[1ex]
  \href{mailto:nbeisert@itp.phys.ethz.ch}
  {\texttt{nbeisert@itp.phys.ethz.ch}}}
\hypersetup{pdfauthor={Niklas Beisert}}
\hypersetup{pdfsubject={Manual for the LaTeX2e Package childdoc}}
\date{30 December 2018, \textsf{v2.0}}
\maketitle

\begin{abstract}\noindent
\textsf{childdoc} is a \LaTeXe{} package
that enables the direct compilation
of document sections included by |\include|
to individual files.
\end{abstract}

\begingroup
\parskip0ex
\tableofcontents
\endgroup

%%%%%%%%%%%%%%%%%%%%%%%%%%%%%%%%%%%%%%%%%%%%%%%%%%%%%%%%%%%%%%%%%%%%%%%%%%%%%%%%
%%%%%%%%%%%%%%%%%%%%%%%%%%%%%%%%%%%%%%%%%%%%%%%%%%%%%%%%%%%%%%%%%%%%%%%%%%%%%%%%
\section{Introduction}

\LaTeX{} provides a mechanism to structure a large document (such as a book)
into a main file and several child files (containing the chapters)
using the |\include| command.
This mechanism is beneficial for documents
which span hundreds of pages in order to
make the source file(s) more manageable.
Moreover, compilation can be restricted to
selected child files by means of the |\includeonly| command.
The latter feature can be used to reduce the compilation time while editing
(this was significantly more useful in the earlier days of \LaTeX{})
or to generate a smaller document which is easier to navigate.
Another application of |\includeonly| is to generate
documents consisting of selected parts of the complete document.

However, there are a few drawbacks of the plain |\include| mechanism:
\begin{itemize}
\item
The child files cannot be compiled on their own,
they can only be compiled via the main file.
A naive editing environment
(such as a text editor with an option
to have the current file processed by \LaTeX)
may require one to switch to the main file before compiling;
attempting to compile the child file produces errors.
\item
The main file must be modified (each time)
to adjust the |\includeonly| command
to the present needs. This easily leaves the main file in a messy state.
\item
The generated document will always carry the filename
of the main document. This is inconvenient if
several child files are to be compiled and
to be kept for distribution.
\end{itemize}

The present package provides a simple interface
to make child files individually compilable by \LaTeX{}.
Compiling a child file then has the same effect as compiling
the main file with an |\includeonly| command
to select the appropriate child.
Moreover the generated document will carry the name of the child
rather than the main file.
This resolves all three above issues.

This feature is meant to make the editing of books,
thesis documents and lecture notes somewhat more convenient.
However, the package can also be used efficiently for
composing a series of documents (such as exercise sheets)
which are typically distributed individually.
It then assists the author in generating the individual documents
(potentially in different versions)
as well as a document containing the collected series.
Another application is in developing style files
or other kinds of included material
where compilation of the style file could redirect
to a sample or test file.

%%%%%%%%%%%%%%%%%%%%%%%%%%%%%%%%%%%%%%%%%%%%%%%%%%%%%%%%%%%%%%%%%%%%%%%%%%%%%%%%
%%%%%%%%%%%%%%%%%%%%%%%%%%%%%%%%%%%%%%%%%%%%%%%%%%%%%%%%%%%%%%%%%%%%%%%%%%%%%%%%
\section{Usage}

First of all, the package \textsf{childdoc} is \emph{not} a standard
\LaTeXe{} |.sty| style file! Therefore it needs to be invoked in
a non-standard way.

%%%%%%%%%%%%%%%%%%%%%%%%%%%%%%%%%%%%%%%%%%%%%%%%%%%%%%%%%%%%%%%%%%%%%%%%%%%%%%%%
\subsection{Included Files}
\label{sec:include}

%%%%%%%%%%%%%%%%%%%%%%%%%%%%%%%%%%%%%%%%
\DescribeMacro{\childdocmain}
To use the package, add the commands
\begin{center}
\begin{tabular}{l}
|\input{childdoc.def}|\\
|\childdocmain{}|\\
\end{tabular}
\end{center}
at the very top of the main \LaTeX{} file,
in particular \emph{before} the |\documentclass| statement!
The argument of |\childdocmain| should be left empty
(but it must be present).

%%%%%%%%%%%%%%%%%%%%%%%%%%%%%%%%%%%%%%%%
\DescribeMacro{\childdocof}
Furthermore, add the commands
\begin{center}
\begin{tabular}{l}
|\input{childdoc.def}|\\
|\childdocof{|\textit{main}|}|\\
\end{tabular}
\end{center}
at the top of every child file \textit{child}
which is included by |\include{|\textit{child}|}|
from within the main file
(or at least for those files to be compiled individually).
The argument \textit{main} must be the filename of the main file.

There are a couple of
considerations in setting up the main and child documents:

%%%%%%%%%%%%%%%%%%%%%%%%%%%%%%%%%%%%%%%%
\paragraph{Restrictions.}

Please note the following restrictions:
\begin{itemize}
\item
|\childdocmain| must be called with one argument \textit{main}
to ensure compatibility with earlier version of the package.
It must either be empty (|\childdocmain{}|)
or precisely match the filename of the main file in which it is specified.
See \secref{sec:detection} for further information.
\item
The filename \textit{main} must be specified without the |.tex| extension.
\item
The filename \textit{main} is case sensitive
(even in case-insensitive file systems)
due to internal string comparison.
\item
The argument \textit{main} should be fully expanded, it cannot be a macro.
\item
Subdirectories and special characters should be avoided in filenames.
\item
The command |\childdocmain{|\textit{main}|}| must be followed by a whitespace.
It should not be followed immediately by another command
or by a comment mark `|%|'.
This is because the \TeX{} parser reads the token immediately following
the argument of |\childdocmain| and puts it
at the beginning of every child section;
however, a white\-space is ignored.
\end{itemize}

%%%%%%%%%%%%%%%%%%%%%%%%%%%%%%%%%%%%%%%%
\paragraph{Content of Main File.}

It is advisable to place all content in the child files included by |\include|.
Any output contained in the main file will appear in all child documents
unless suppressed manually;
it cannot be suppressed automatically by the |\includeonly| directive
and thus should normally be avoided.
A method to include some content in the main file
by means of conditional processing is described in \secref{sec:conditional}.

%%%%%%%%%%%%%%%%%%%%%%%%%%%%%%%%%%%%%%%%
\paragraph{Page Numbering.}

When only a part of the document is compiled,
the appropriate numbering of pages
(as well as other status parameters)
is determined from the |.aux| files.
The latter contain information from previous passes.
However this information needs to propagate through
all intermediate child documents.
Therefore the page numbering in child documents may well
be inconsistent until the complete document is compiled at least once.

A useful (if unconventional) way to always ensure a consistent
page numbering is to restart the numbering in each child document
and denote the pages by `\textit{child}|.|\textit{page}'
where \textit{child} represents the chapter/section number of the child file.
This can be achieved by the command
|\numberwithin{page}{|\textit{child}|}|
of the \textsf{amsmath} package
where \textit{child} can be |chapter| or |section|
depending on the chosen structuring.
Alternatively, one can modify the macro |\thepage| appropriately
and reset the counter |page| at the start of each child file.

%%%%%%%%%%%%%%%%%%%%%%%%%%%%%%%%%%%%%%%%%%%%%%%%%%%%%%%%%%%%%%%%%%%%%%%%%%%%%%%%
\subsection{Conditional Processing}
\label{sec:conditional}

The package provides a mechanism to compile different versions
of a document. To customise the versions further some conditional processing
can come in handy to distinguish which version is being compiled.
The package provides two macros to describe the compilation context:

%%%%%%%%%%%%%%%%%%%%%%%%%%%%%%%%%%%%%%%%
\DescribeMacro{\ifchilddoc}
The conditional |\ifchilddoc| distinguishes between the compilation of
child documents and the main document:
%
\begin{center}
|\ifchilddoc |\textit{child-code}| |[|\||else |\textit{main-code}]| \||fi|
\end{center}

%%%%%%%%%%%%%%%%%%%%%%%%%%%%%%%%%%%%%%%%
\DescribeMacro{\childdocname}
\DescribeMacro{\childdocjob}
The macro |\childdocname| contains the filename (without extension)
of the main or child file being processed.
Note that |\childdocjob| will always contain the name of the main file.

%%%%%%%%%%%%%%%%%%%%%%%%%%%%%%%%%%%%%%%%
\paragraph{Title Page.}

Conditional processing can be used to include a title or banner page
in the main document when proper precautions are taken.
Importantly, the code in the main file should ensure that the page counter
(as well as other status parameters which are stored in the |.aux| files)
takes the same value after the conditional processing.
Otherwise the page numbers may take divergent values
depending on which part is compiled.

For example, a title page could be declared by:
%
\begin{center}
\begin{tabular}{l}
|\ifchilddoc\||else|\\
|\addtocounter{page}{-1}|\\
\textit{code for title page}\\
|\newpage|\\
|\||fi|
\end{tabular}
\end{center}
%
A banner page for the child documents can be generated by:
%
\begin{center}
\begin{tabular}{l}
|\ifchilddoc|\\
|\addtocounter{page}{-1}|\\
\textit{code for banner page}\\
|\newpage|\\
|\||fi|
\end{tabular}
\end{center}
%
Here one could write a message such as:
\begin{center}
|This is the part \childdocname{} of \childdocjob{}.|
\end{center}

%%%%%%%%%%%%%%%%%%%%%%%%%%%%%%%%%%%%%%%%%%%%%%%%%%%%%%%%%%%%%%%%%%%%%%%%%%%%%%%%
\subsection{Flags}
\label{sec:flags}

The package makes it easy to generate different versions
of the main or child documents.
To this end compilation flags can be defined
and assigned different default values.
They will be particularly useful in conjunction
with the forwarding mechanism described in \secref{sec:forward}.

For example, it may be useful to have a flag |\version|
which can be set to |draft| or |final|.
The document source will contain some conditional code
depending on the value of |\version|.
Suppose further, the flag should default to |final| for the main file
and to |draft| for child files
which is a natural assignment for editing the document.
This is achieved by placing the following code
in the preamble of the main document
(below the |\childdocmain| directive):
%
\begin{center}
\begin{tabular}{l}
|\ifchilddoc|\\
|\providecommand{\version}{draft}|\\
|\||else|\\
|\providecommand{\version}{final}|\\
|\||fi|
\end{tabular}
\end{center}
%
The definition by |\providecommand| makes sure
that previous definitions are not overwritten.
Further statements |\providecommand{\version}{...}|
can thus be added before the above code to override it.

For the main file, one might add a line
(between |\childdocmain| and the above block)
%
\begin{center}
|%\ifchilddoc\||else\providecommand{\version}{draft}\||fi|
\end{center}
%
which can be uncommented to produce a draft version.
Likewise one can add a line to the very top of a child file
(above the |\childdocof{|\textit{main}|}| directive)
%
\begin{center}
|%\providecommand{\version}{final}|
\end{center}
%
which can be uncommented to produce the final version of this child document.

%%%%%%%%%%%%%%%%%%%%%%%%%%%%%%%%%%%%%%%%%%%%%%%%%%%%%%%%%%%%%%%%%%%%%%%%%%%%%%%%
\subsection{Forwarding}
\label{sec:forward}

Different versions of the main or child documents
using compilation flags as described in \secref{sec:flags}
can be (permanently) stored in different files
for convenient compilation, viewing and distribution.
To this end, the package defines a command
to pass on compilation to a different file:

%%%%%%%%%%%%%%%%%%%%%%%%%%%%%%%%%%%%%%%%
\DescribeMacro{\childdocforward}
The command |\childdocforward| redirects processing to
another source file:
%
\begin{center}
\begin{tabular}{l}
|\input{childdoc.def}|\\
|\childdocforward[|\textit{main}|]{|\textit{dest}|}|\\
\end{tabular}
\end{center}
%
The argument \textit{dest} is the destination file
(without extension).
It should be the main file or one of the child files.
Note that further \textsf{childdoc} directives
such as |\childdocof| and |\childdocforward|
in the indicated file will be processed in this form.
The optional argument \textit{main}
passes on directly to the main file \textit{main}
while pretending to compile the child \textit{dest}.
This form behaves as if \textit{dest}
issues |\childdocof{|\textit{main}|}| right away,
and no further \textsf{childdoc} directives will be processed.

%%%%%%%%%%%%%%%%%%%%%%%%%%%%%%%%%%%%%%%%
\DescribeMacro{\...prefix}
In the alternative form |\childdocforwardprefix|,
%
\begin{center}
\begin{tabular}{l}
|\input{childdoc.def}|\\
|\childdocforwardprefix[|\textit{main}|]{|\textit{prefix}|}{|\textit{dest}|}|
\end{tabular}
\end{center}
%
the destination file is determined by a pattern
depending on the current file:
To make this work, the current file must be called
`{\textit{prefix}\hspace{0.2em}\textit{suffix}}'
with \textit{prefix} matching precisely the argument.
Processing is then passed on to the file
`{\textit{dest}\hspace{0.2em}\textit{suffix}}'.
Surely, the same effect is achieved by
directly specifying the
argument `{\textit{dest}\hspace{0.2em}\textit{suffix}}'
in the first form.
However, that requires to set up a different file
for each child. With the alternative form of the command
all these files can have exactly the same content
which simplifies setting them up and maintaining them.

For example, the following file |draft.tex|
with a compilation flag |\version| as described in \secref{sec:flags}
compiles the main document as a draft:
%
\begin{center}
\begin{tabular}{l}
|\def\version{draft}|\\
|\input{childdoc.def}|\\
|\childdocforward{|\textit{main}|}|
\end{tabular}
\end{center}
%
Likewise, the following files |final|\textit{nn}|.tex|
compile the final version of the child document
|child|\textit{nn}|.tex|:
%
\begin{center}
\begin{tabular}{l}
|\def\version{final}|\\
|\input{childdoc.def}|\\
|\childdocforwardprefix{final}{child}|
\end{tabular}
\end{center}
%

Note that when several versions of a main file and/or of each child file
are to be generated, it may be convenient to set up a |Makefile| or
shell script to automatise the process.

%%%%%%%%%%%%%%%%%%%%%%%%%%%%%%%%%%%%%%%%%%%%%%%%%%%%%%%%%%%%%%%%%%%%%%%%%%%%%%%%
\subsection{Command Line Processing}
\label{sec:commandline}

The effect of redirection files can also be achieved by invoking
the \LaTeX{} compiler with a more elaborate command line.
Most conveniently this should be done as part
of a shell script or a |Makefile|.

When using \textsf{childdoc} in the main file, the following
command lines effectively perform a redirection
(note that depending on the shell being used,
backslashes may have to be doubled: `|\|' $\to$ `|\\|'):
%
\begin{center}
|... -jobname "|\textit{target}|" |\\|"|[\textit{flags}]%
|\input{childdoc.def}\childdocforward[|\textit{main}|]{|\textit{dest}|}"|
\end{center}
%
Here \textit{target} is the name of the output file,
\textit{main} is the name of the main file
and \textit{dest} is the name of the main or child file to be processed
(all filenames without extensions).
The optional argument \textit{main} can be omitted
if \textit{main} matches \textit{dest}.
Optionally, compilation \textit{flags} can be defined via |\def| commands.
This command line makes the \TeX{} engine believe
it is compiling the file \textit{target}
whose content is specified as the latter parameter.
The provided code then forwards the processing to
\textit{main} or \textit{dest} as described in \secref{sec:forward}.

%%%%%%%%%%%%%%%%%%%%%%%%%%%%%%%%%%%%%%%%%%%%%%%%%%%%%%%%%%%%%%%%%%%%%%%%%%%%%%%%
\subsection{Include by Input}
\label{sec:input}

Including child documents by |\include| has some restrictions by design.
Most notably, the content of a child document always occupies
its own set of pages; pages cannot be shared between child documents.
Usually, this behaviour makes perfect sense
because each child document contain an essential part of the document.
However, in some situations it may be desirable to compose
a document from a collection of parts
without having mandatory page breaks between then.
For this case, the package
provides a mechanism to include parts
by |\input| which can also be processed individually.
However, by construction this mechanism
requires manual handling of the content to be output.

%%%%%%%%%%%%%%%%%%%%%%%%%%%%%%%%%%%%%%%%
\DescribeMacro{\ifchilddocmanual}
The main file should be prepared as usual, see \secref{sec:include}.
However, the document body must make a distinction
between processing of an individual part and of the main document, e.g.:
%
\begin{center}
\begin{tabular}{l}
|\ifchilddocmanual|\\
|\input{\childdocname}|\\
|\||else|\\
\textit{document body with }|\input{|\textit{part}|}|\\
|\||fi|
\end{tabular}
\end{center}
%
The conditional |\ifchilddocmanual| is true whenever
a part to be included by |\input| is being compiled,
and the name of the part is stored in |\childdocname|.

%%%%%%%%%%%%%%%%%%%%%%%%%%%%%%%%%%%%%%%%
\DescribeMacro{\childdocby}
Each part to be included by |\input| should start with:
%
\begin{center}
\begin{tabular}{l}
|\input{childdoc.def}|\\
|\childdocby{|\textit{main}|}|\\
\end{tabular}
\end{center}
%
The directive |\childdocby| is similar to |\childdocof|
described in \secref{sec:include},
but the subsequent selection of content must be done manually.
To that end, both |\ifchilddoc| and |\ifchilddocmanual|
will be true upon processing of a part,
and the name of the part is stored in |\childdocname|.
Note that |\jobname| will be set to the filename of the current part
so that each part receives an individual |.aux| file
that does not interfere with the |.aux| file(s) of the main document.
This behaviour can be altered by the alternative form
|\childdocby[*]{|\textit{main}|}| (with a non-empty optional argument)
which uses the |.aux| file of the main document
by setting |\jobname| to \textit{main}.

%%%%%%%%%%%%%%%%%%%%%%%%%%%%%%%%%%%%%%%%%%%%%%%%%%%%%%%%%%%%%%%%%%%%%%%%%%%%%%%%
\subsection{Driver Development}
\label{sec:driver}

The \textsf{childdoc} mechanism can also be use for the development
of definition files such as \LaTeX{} styles or classes.
This case differs from the above setup with multiple parts
included by |\include| in that no |\includeonly| should be invoked.
This can be achieved by starting the include file
(before |\ProvidesPackage|) with:
%
\begin{center}
\begin{tabular}{l}
|\input{childdoc.def}|\\
|\childdocforward{|\textit{main}|}|\\
\end{tabular}
\end{center}
%
or alternatively with:
%
\begin{center}
\begin{tabular}{l}
|\input{childdoc.def}|\\
|\childdocby{|\textit{main}|}|\\
\end{tabular}
\end{center}
%
Both forms have slightly different effects as described above.
The main file is prepared as usual, see \secref{sec:include}.

%%%%%%%%%%%%%%%%%%%%%%%%%%%%%%%%%%%%%%%%%%%%%%%%%%%%%%%%%%%%%%%%%%%%%%%%%%%%%%%%
\subsection{Legacy Detection}
\label{sec:detection}

The directive |\childdocmain| in the main file can detect
whether the complete document or merely a child is to be compiled
even without using the directive |\childdocof|.
This method is deprecated because it is less robust
and there is no compelling reason to use it;
it is merely provided for backward compatibility
and it may be removed in future versions.

If the detection mechanism is to be used,
it is mandatory to correctly specify
the filename of the main file as the argument of |\childdocmain|:
%
\begin{center}
\begin{tabular}{l}
|\input{childdoc.def}|\\
|\childdocmain{|\textit{main}|}|\\
\end{tabular}
\end{center}
%
If |\jobname| does not match the argument \textit{main} of |\childdocmain|,
it is assumed that |\jobname| points to the child file to be compiled.
When using |\childdocmain| with the main file specified as argument,
it suffices to start a child file
with just |\input{|\textit{main}|}|
without loading of the package and using |\childdocof|.
If instead all processing is done
with the appropriate \textsf{childdoc} directives,
the argument of \textit{main} of |\childdocmain| can be empty.

An alternative version of the command line processing described
in \secref{sec:commandline} using the detection mechanism reads:
%
\begin{center}
|... -jobname "|\textit{target}|" "|[\textit{flags}]%
[|\def\jobname{|\textit{dest}|}|]|\input{|\textit{main}|}"|
\end{center}

%%%%%%%%%%%%%%%%%%%%%%%%%%%%%%%%%%%%%%%%%%%%%%%%%%%%%%%%%%%%%%%%%%%%%%%%%%%%%%%%
\subsection{Manual Code}
\label{sec:manual}

In case one cannot be certain whether the definitions file |childdoc.def|
is installed on the target \TeX{} distribution
and one prefers not to ship it,
it is conceivable to paste a few relevant commands into the sources.

To that end, drop all statements |\input{childdoc.def}|
and perform the replacements as outlined below.
Instead of |\childdocmain{|\textit{main}|}| add the following code
to the top of the main file:
%
\begin{center}
\begin{tabular}{l}
|\||ifdefined\childdocname\endinput\||fi\newif\ifchilddoc|\\
|\edef\childdocname{\scantokens\expandafter{\jobname\noexpand}}|\\
|\def\childdocmain{|\textit{main}|}\||ifx\childdocmain\childdocname\||else|\\
|\childdoctrue\includeonly{\childdocname}\let\jobname\childdocmain\||fi|\\
\end{tabular}
\end{center}
%
Instead of |\childdocof{|\textit{main}|}| just include the main file
at the top of each child file:
%
\begin{center}
|\input{|\textit{main}|}|
\end{center}
%
A simple redirection |\childdocforward{|\textit{dest}|}| is achieved by:
%
\begin{center}
|\def\jobname{|\textit{dest}|}\input{\jobname}|
\end{center}
%
The redirection with prefix
|\childdocforwardprefix[|\textit{prefix}|]{|\textit{dest}|}|
is accomplished by:
%
\begin{center}
\begin{tabular}{l}
|{\edef\jobname{\scantokens\expandafter{\jobname\noexpand}}|\\
|\def\redirectjob |\textit{prefix}|#1~~~{\gdef\jobname{|\textit{dest}|#1}}|\\
|\expandafter\redirectjob\jobname~~~}\input{\jobname}|
\end{tabular}
\end{center}

In an alternative approach,
child documents can be compiled by a specific command line
without additional code or specific definitions:
%
\begin{center}
|... -jobname "|\textit{target}|" "|[\textit{flags}]%
|\includeonly{|\textit{dest}|}\input{|\textit{main}|}"|
\end{center}
%

%%%%%%%%%%%%%%%%%%%%%%%%%%%%%%%%%%%%%%%%%%%%%%%%%%%%%%%%%%%%%%%%%%%%%%%%%%%%%%%%
%%%%%%%%%%%%%%%%%%%%%%%%%%%%%%%%%%%%%%%%%%%%%%%%%%%%%%%%%%%%%%%%%%%%%%%%%%%%%%%%
\section{Information}

%%%%%%%%%%%%%%%%%%%%%%%%%%%%%%%%%%%%%%%%%%%%%%%%%%%%%%%%%%%%%%%%%%%%%%%%%%%%%%%%
\subsection{Copyright}

Copyright \copyright{} 2017--2018 Niklas Beisert

This work may be distributed and/or modified under the
conditions of the \LaTeX{} Project Public License, either version 1.3
of this license or (at your option) any later version.
The latest version of this license is in
  \url{http://www.latex-project.org/lppl.txt}
and version 1.3 or later is part of all distributions of \LaTeX{}
version 2005/12/01 or later.

This work has the LPPL maintenance status `maintained'.

The Current Maintainer of this work is Niklas Beisert.

This work consists of the files |README.txt|, |childdoc.ins| and |childdoc.dtx|
as well as the derived files |childdoc.def|, |cdocsamp.tex|
with |cdocsch1.tex|, |cdocsch2.tex|, |cdocspt3.tex|, |cdocspt4.tex|,
|cdocsdrf.tex|, |cdocsfn1.tex|, |cdocsfn2.tex|
as well as |childdoc.pdf|.

%%%%%%%%%%%%%%%%%%%%%%%%%%%%%%%%%%%%%%%%%%%%%%%%%%%%%%%%%%%%%%%%%%%%%%%%%%%%%%%%
\subsection{Files and Installation}

The package consists of the files:
%
\begin{center}
\begin{tabular}{ll}
    |README.txt|   & readme file \\
    |childdoc.ins| & installation file \\
    |childdoc.dtx| & source file \\
    |childdoc.def| & definition file \\
    |cdocsamp.tex| & sample main file \\
    |cdocsch1.tex| & sample include file \\
    |cdocsch2.tex| & sample include file \\
    |cdocspt3.tex| & sample part file \\
    |cdocspt4.tex| & sample part file \\
    |cdocsdrf.tex| & sample redirection file \\
    |cdocsfn1.tex| & sample redirection file \\
    |cdocsfn2.tex| & sample redirection file \\
    |childdoc.pdf| & manual
\end{tabular}
\end{center}
%
The distribution consists of the files
|README.txt|, |childdoc.ins| and |childdoc.dtx|.
%
\begin{itemize}
\item
Run (pdf)\LaTeX{} on |childdoc.dtx|
to compile the manual |childdoc.pdf| (this file).
\item
Run \LaTeX{} on |childdoc.ins| to create the definitions file |childdoc.def|
and the sample |cdocsamp.tex| with include files
|cdocsch1.tex|, |cdocsch2.tex|, |cdocspt3.tex|, |cdocspt4.tex|,
|cdocsdrf.tex|, |cdocsfn1.tex|, |cdocsfn2.tex|.
Then copy the file |childdoc.def| to an appropriate directory of your \LaTeX{}
distribution, e.g.\ \textit{texmf-root}|/tex/latex/childdoc|.
\end{itemize}

%%%%%%%%%%%%%%%%%%%%%%%%%%%%%%%%%%%%%%%%%%%%%%%%%%%%%%%%%%%%%%%%%%%%%%%%%%%%%%%%
\subsection{Related CTAN Packages}

There are several other packages which offer a similar functionality:
%
\begin{itemize}
\item
The packages
\href{http://ctan.org/pkg/docmute}{\textsf{docmute}},
\href{http://ctan.org/pkg/includex}{\textsf{includex}} and
\href{http://ctan.org/pkg/standalone}{\textsf{standalone}}
provide commands to include only the document body of
a child file thus allowing both files to be compiled individually.
\item
The packages \href{http://ctan.org/pkg/subdocs}{\textsf{subdocs}}
and \href{http://ctan.org/pkg/subfiles}{\textsf{subfiles}}
provide structures in which the main and child documents can be
encapsulated and allowing them to be compiled individually.
The inclusion mechanism is different from the conventional |\include|.
\item
The package \href{http://ctan.org/pkg/combine}{\textsf{combine}}
is an elaborate solution to combine several documents into one.
\end{itemize}
%
See also the CTAN topic \href{http://ctan.org/topic/subdocs}{\textsf{subdocs}}
for further related packages.
The present package differs from the above solutions in that
a document structure constructed with the conventional |\include| mechanism
just needs two extra commands at the top of every file
such that all constituent files can be compiled individually.

%%%%%%%%%%%%%%%%%%%%%%%%%%%%%%%%%%%%%%%%%%%%%%%%%%%%%%%%%%%%%%%%%%%%%%%%%%%%%%%%
%\subsection{Feature Suggestions}
%
%The following is a list of features which may be useful for future
%versions of this package:
%%
%\begin{itemize}
%\item
%\ldots
%\end{itemize}

%%%%%%%%%%%%%%%%%%%%%%%%%%%%%%%%%%%%%%%%%%%%%%%%%%%%%%%%%%%%%%%%%%%%%%%%%%%%%%%%
\subsection{Revision History}

%%%%%%%%%%%%%%%%%%%%%%%%%%%%%%%%%%%%%%%%
\paragraph{v2.0:} 2018/12/30

\begin{itemize}
\item
immediate forward processing
\item
added |\childdocby| mechanism
\item
manual restructured
\end{itemize}

%%%%%%%%%%%%%%%%%%%%%%%%%%%%%%%%%%%%%%%%
\paragraph{v1.6:} 2018/01/17

\begin{itemize}
\item
application for development of include files
\item
corrections to manual
\end{itemize}

%%%%%%%%%%%%%%%%%%%%%%%%%%%%%%%%%%%%%%%%
\paragraph{v1.5:} 2017/05/21

\begin{itemize}
\item
more complete structuring introduced
\item
|\childdocof| introduced
\item
|\childdoc| renamed to |\childdocmain|
\item
|\childredirect| renamed to |\childdocforward| and |\childdocforwardprefix|
and functionality expanded
\end{itemize}

%%%%%%%%%%%%%%%%%%%%%%%%%%%%%%%%%%%%%%%%
\paragraph{v1.0:} 2017/04/27

\begin{itemize}
\item
manual and install package
\item
first version published on CTAN
\end{itemize}

%%%%%%%%%%%%%%%%%%%%%%%%%%%%%%%%%%%%%%%%
\paragraph{v0.6:} 2017/04/26

\begin{itemize}
\item
redirection mechanism added
\end{itemize}

%%%%%%%%%%%%%%%%%%%%%%%%%%%%%%%%%%%%%%%%
\paragraph{v0.5:} 2017/04/26

\begin{itemize}
\item
functionality in definition file
\end{itemize}


%%%%%%%%%%%%%%%%%%%%%%%%%%%%%%%%%%%%%%%%%%%%%%%%%%%%%%%%%%%%%%%%%%%%%%%%%%%%%%%%
%%%%%%%%%%%%%%%%%%%%%%%%%%%%%%%%%%%%%%%%%%%%%%%%%%%%%%%%%%%%%%%%%%%%%%%%%%%%%%%%
%%%%%%%%%%%%%%%%%%%%%%%%%%%%%%%%%%%%%%%%%%%%%%%%%%%%%%%%%%%%%%%%%%%%%%%%%%%%%%%%
\appendix

\settowidth\MacroIndent{\rmfamily\scriptsize 000\ }

 \DocInput{childdoc.dtx}

\end{document}
%</driver>
% \fi
%
% %%%%%%%%%%%%%%%%%%%%%%%%%%%%%%%%%%%%%%%%%%%%%%%%%%%%%%%%%%%%%%%%%%%%%%%%%%%%%%
% %%%%%%%%%%%%%%%%%%%%%%%%%%%%%%%%%%%%%%%%%%%%%%%%%%%%%%%%%%%%%%%%%%%%%%%%%%%%%%
% \section{Sample}
%\iffalse
%<*samplemain>
%\fi
%
% The following presents a sample document
% with two chapters, two parts, a title page,
% a compile flag as well as three forwarding files to set the flag.
% It consists of eight |.tex| files:
% \begin{center}
% \begin{tabular}{ll}
% |cdocsamp.tex|&main file\\
% |cdocsch1.tex|&include file for chapter 1\\
% |cdocsch2.tex|&include file for chapter 2\\
% |cdocspt3.tex|&include file for part 3\\
% |cdocspt4.tex|&include file for part 4\\
% |cdocsdrf.tex|&forwarding file for main file in draft mode\\
% |cdocsfi1.tex|&forwarding file for final version of chapter 1\\
% |cdocsfi2.tex|&forwarding file for final version of chapter 2\\
% \end{tabular}
% \end{center}
% Each of the eight files can be compiled directly by the \LaTeX{} compiler.
%
% %%%%%%%%%%%%%%%%%%%%%%%%%%%%%%%%%%%%%%
% \paragraph{Main File.}
%
% The main file is called |cdocsamp.tex|.
%
% Load the \textsf{childdoc} definitions and
% declare the filename for the main document:
%    \begin{macrocode}
\input{childdoc.def}
\childdocmain{}
%    \end{macrocode}

% Optional override for |\version| flag:
%    \begin{macrocode}
%%\ifchilddoc\else\providecommand{\version}{draft}\fi
%    \end{macrocode}

% Define the default values for the |\version| flag
% (|final| for the main file and |draft| for childs):
%    \begin{macrocode}
\ifchilddoc
\providecommand{\version}{draft}
\else
\providecommand{\version}{final}
\fi
%    \end{macrocode}

% Load the standard document class:
%    \begin{macrocode}
\documentclass[12pt]{article}
%    \end{macrocode}

% Start the document body:
%    \begin{macrocode}
\begin{document}
%    \end{macrocode}

% Declare a title page.
% Print title, part of document being processed and version flag:
%    \begin{macrocode}
\addtocounter{page}{-1}
\begin{center}
{\LARGE\bfseries{}childdoc example\par}
\vspace{1cm}
\ifchilddoc
\ifchilddocmanual part\else chapter\fi:
`\childdocname' of `\childdocjob'\par
\else
main document: `\childdocjob'\par
\fi
version: \version\par
\end{center}
\newpage
%    \end{macrocode}

% Manually include selected file,
% otherwise process as usual:
%    \begin{macrocode}
\ifchilddocmanual
\section*{part `\childdocname'}
\input{\childdocname}
\else
%    \end{macrocode}

% Include the two chapters:
%    \begin{macrocode}
\include{cdocsch1}
\include{cdocsch2}
%    \end{macrocode}

% Include the two parts unless only chapters should be displayed:
%    \begin{macrocode}
\ifchilddoc\else
\section{part three}
\input{cdocspt3}
\section{part four}
\input{cdocspt4}
\fi
%    \end{macrocode}

% Process as usual until here:
%    \begin{macrocode}
\fi
%    \end{macrocode}

% End of document body:
%    \begin{macrocode}
\end{document}
%    \end{macrocode}
%\iffalse
%</samplemain>
%\fi
%
% %%%%%%%%%%%%%%%%%%%%%%%%%%%%%%%%%%%%%%
% \paragraph{Chapter Include Files.}
%
% The include files are called |cdocsch1.tex| and |cdocsch2.tex|.
%
%\iffalse
%<*samplechap1|samplechap2>
%\fi

% Optional override for |\version| flag:
%    \begin{macrocode}
%%\providecommand{\version}{final}
%    \end{macrocode}

% Include the main document:
%    \begin{macrocode}
\input{childdoc.def}
\childdocof{cdocsamp}
%    \end{macrocode}

%\iffalse
%</samplechap1|samplechap2>
%\fi
%
%\iffalse
%<*samplechap1>
%\fi
% Some text for chapter 1:
%    \begin{macrocode}
\section{one}
some text in chapter one
%    \end{macrocode}

%\iffalse
%</samplechap1>
%\fi
% Some text for chapter 2:
%\iffalse
%<*samplechap2>
%\fi
%    \begin{macrocode}
\section{two}
more text in chapter two
%    \end{macrocode}

%\iffalse
%</samplechap2>
%\fi
%
% %%%%%%%%%%%%%%%%%%%%%%%%%%%%%%%%%%%%%%
% \paragraph{Part Include Files.}
%
% The include files are called |cdocspt3.tex| and |cdocspt4.tex|.
%
%\iffalse
%<*samplepart3|samplepart4>
%\fi

% Optional override for |\version| flag:
%    \begin{macrocode}
%%\providecommand{\version}{final}
%    \end{macrocode}

% Include the main document:
%    \begin{macrocode}
\input{childdoc.def}
\childdocby{cdocsamp}
%    \end{macrocode}

%\iffalse
%</samplepart3|samplepart4>
%\fi
%
%\iffalse
%<*samplepart3>
%\fi
% Some text for part 3:
%    \begin{macrocode}
some text in part three
%    \end{macrocode}

%\iffalse
%</samplepart3>
%\fi
% Some text for part 4:
%\iffalse
%<*samplepart4>
%\fi
%    \begin{macrocode}
more text in part four
%    \end{macrocode}

%\iffalse
%</samplepart4>
%\fi
%
% %%%%%%%%%%%%%%%%%%%%%%%%%%%%%%%%%%%%%%
% \paragraph{Forwarding for a Complete Draft.}
%
% The following forwarding file |cdocsdrf.tex|
% compiles the main document in draft mode:
%\iffalse
%<*sampledraft>
%\fi
%    \begin{macrocode}
\def\version{draft}
\input{childdoc.def}
\childdocforward{cdocsamp}
%    \end{macrocode}

%\iffalse
%</sampledraft>
%\fi
%
% %%%%%%%%%%%%%%%%%%%%%%%%%%%%%%%%%%%%%%
% \paragraph{Forwarding for Final Version of the Chapters.}
%
% The following forwarding files |cdocsfn1.tex| and |cdocsfn2.tex|
% (with identical content)
% compile the final versions of the child documents
% |cdocsch1.tex| and |cdocsch2.tex|, respectively:
%\iffalse
%<*samplefinal>
%\fi
%    \begin{macrocode}
\def\version{final}
\input{childdoc.def}
\childdocforwardprefix[cdocsamp]{cdocsfn}{cdocsch}
%    \end{macrocode}

%\iffalse
%</samplefinal>
%\fi
%
% %%%%%%%%%%%%%%%%%%%%%%%%%%%%%%%%%%%%%%
% \paragraph{Command Line Processing.}
%
% The following three command lines generate the output files
% |cdocscld|, |cdocscl1| and |cdocscl2|
% which should be identical to
% |cdocsdrf|, |cdocsch1| and |cdocsfn2|, respectively:
% \begin{center}
% \begin{tabular}{l}
% |latex -jobname cdocscld \|\\
% |  "\def\version{draft}\input{childdoc.def}\childdocforward{cdocsamp}"|\\
% |latex -jobname cdocscl1 \|\\
% |  "\input{childdoc.def}\childdocforward[cdocsamp]{cdocsch1}"|\\
% |latex -jobname cdocscl2 \|\\
% |  "\def\version{final}\input{childdoc.def}\childdocforward{cdocsch2}"|
% \end{tabular}
% \end{center}
% Note that the trailing backslash on each first line
% merely continues the input to the second line
% (for convenient cut ant paste).
% Furthermore, the command |latex| can be replaced by any
% of its alternative versions such as |pdflatex|.
%
% %%%%%%%%%%%%%%%%%%%%%%%%%%%%%%%%%%%%%%%%%%%%%%%%%%%%%%%%%%%%%%%%%%%%%%%%%%%%%%
% %%%%%%%%%%%%%%%%%%%%%%%%%%%%%%%%%%%%%%%%%%%%%%%%%%%%%%%%%%%%%%%%%%%%%%%%%%%%%%
% \section{Implementation}
%\iffalse
%<*package>
%\fi
%
% This section describes the definitions file |childdoc.def|.

% The definitions cannot be loaded using |\usepackage| or |\RequirePackage|
% which has a mechanism to prevent loading a style file more than once.
% When loading the definitions by means of |\input|
% multiple instances have to be prevented manually:
%\iffalse
%This code needs to be before the `\ProvidesFile' directive
%which is defined at the beginning of this file.
%Therefore it is also placed there and commented out here.
%</package>
%<*discard>
%\fi
%    \begin{macrocode}
\ifdefined\childdocmain\endinput\fi
%    \end{macrocode}
%\iffalse
%</discard>
%<*package>
%\fi
%
% \macro{\ifchilddoc}
% \macro{\ifchilddocmanual}
% The conditional |\ifchilddoc| tells whether a
% child (true) or main (false) document is being compiled.
% The conditional |\ifchilddocmanual| tells whether
% the |\includeonly| mechanism is used (false) or
% the selection of child files must be performed manually (true).
% The definitions initialise to false:
%    \begin{macrocode}
\newif\ifchilddoc
\newif\ifchilddocmanual
%    \end{macrocode}

% \macro{\childdocname}
% \macro{\childdocjob}
% The macro |\childdocname| stores the name of the main document
% to be compiled. The macro |\childdocjob| stores the name of
% the document on which the \LaTeX{} compiler was originally invoked.
% The content of |\jobname| cannot be compared
% to filenames specified in the source due to different catcodes.
% The following code rescans |\jobname|, stores the result
% in |\childdocname| and saves a copy in |\childdocjob|:
%    \begin{macrocode}
\edef\childdocname{\scantokens\expandafter{\jobname\noexpand}}
\let\childdocjob\childdocname
%    \end{macrocode}

% \macro{\childdocdisable}
% The macro |\childdocdisable| prevents the main file
% from being processed more than once.
% At this stage, the main document command |\childdocmain|
% is assumed to be called once again where it should do nothing.
% Any subsequent call to it should prevent
% a secondary processing of the main document
% It overwrites the forwarding commands
% |\childdocof| and |\childdocforward|
% with empty macros to prevent further inclusions of the main document:
%    \begin{macrocode}
\newcommand{\childdocdisable}
{
  \renewcommand{\childdocmain}[1]{\renewcommand{\childdocmain}[1]{\endinput}}
  \renewcommand{\childdocof}[1]{}
  \renewcommand{\childdocby}[2][]{}
  \renewcommand{\childdocforward}[2][]{}
  \renewcommand{\childdocdisable}{}
}
%    \end{macrocode}

% \macro{\childdocmain}
% The macro |\childdocmain| is to be called at the top of the main file
% with nothing or the main filename (without extension) as argument.
% First, it breaks loops.
% If the argument is not empty and does not match |\childdocname|
% (which is set by the first inclusion of |childdoc.def|),
% |\ifchilddoc| is set to true, |\includeonly| is applied to the child file
% and |\jobname| is set to the main file
% (for proper handling of |.aux| files):
%    \begin{macrocode}
\newcommand{\childdocmain}[1]
{
  \childdocdisable\childdocmain{}
  \if?#1?\else
    \begingroup
      \def\childdoctmp{#1}
      \ifx\childdoctmp\childdocname
        \def\childdoctmp{}
      \else
        \def\childdoctmp
        {
          \childdoctrue
          \includeonly{\childdocname}
          \def\childdocjob{#1}
          \def\jobname{#1}
        }
      \fi
      \expandafter
    \endgroup
    \childdoctmp
  \fi
}
%    \end{macrocode}

% \macro{\childdocof}
% The command |\childdocof| redirects
% compilation to the main file |#1|.
%    \begin{macrocode}
\newcommand{\childdocof}[1]
{
  \childdocdisable
  \childdoctrue
  \includeonly{\childdocname}
  \def\jobname{#1}
  \def\childdocjob{#1}
  \input{#1}
}
%    \end{macrocode}

% \macro{\childdocby}
% The command |\childdocby| ....
%    \begin{macrocode}
\newcommand{\childdocby}[2][]
{
  \childdocdisable
  \childdoctrue
  \childdocmanualtrue
  \if?#1?\else
    \def\jobname{#2}
  \fi
  \def\childdocjob{#2}
  \input{#2}
  \endinput
}
%    \end{macrocode}

% \macro{\childdocforward}
% The command |\childdocforward| redirects
% compilation to the main file or
% (if the optional argument is given) a child file.
% Parameters are set as if the main file
% or a child file starting with |\childdocof| was compiled.
% Then compilation is handed over to the main file:
%    \begin{macrocode}
\newcommand{\childdocforward}[2][]
{
  \begingroup
    \if?#1?
      \def\childdoctmp
      {
        \def\childdocname{#2}
        \def\childdocjob{#2}
        \def\jobname{#2}
        \input{#2}
        \endinput
      }
    \else
      \def\childdoctmp
      {
        \childdocdisable
        \def\childdocname{#2}
        \childdoctrue
        \includeonly{#2}
        \def\childdocjob{#1}
        \def\jobname{#1}
        \input{#1}
        \endinput
      }
    \fi
    \expandafter
  \endgroup
  \childdoctmp
}
%    \end{macrocode}

% \macro{\childdocforwardprefix}
% The command |\childdocforwardprefix| redirects
% compilation to the main or a child file by means of a pattern.
% The prefix |#1| in the current filename is replaced by |#2|
% and the suffix of the current filename is kept
% (it is assumed that the filename does not contain the substring `|~~~|'
% which is used as a delimiter).
% Compilation is handed over to the new file by |\childdocforward|:
%    \begin{macrocode}
\newcommand{\childdocforwardprefix}[3][]
{
  \begingroup
    \def\childdocextract #2##1~~~{\def\childdoctmp{\childdocforward[#1]{#3##1}}}
    \expandafter\childdocextract\childdocname~~~
    \expandafter
  \endgroup
  \childdoctmp
}
%    \end{macrocode}

% \macro{\childdoc}
% The deprecated macro |\childdoc| is a legacy version of |\childdocmain|:
%    \begin{macrocode}
\newcommand{\childdoc}{\childdocmain}
%    \end{macrocode}

% \macro{\childdocredirect}
% The deprecated macro |\childdocredirect| is a legacy version
% of |\childdocforward| and |\childdocforwardprefix|:
%    \begin{macrocode}
\newcommand{\childdocredirect}[2][]
{
  \begingroup
    \if?#1?
      \def\childdoctmp{\childdocforward{#2}}
    \else
      \def\childdoctmp{\childdocforwardprefix{#1}{#2}}
    \fi
    \expandafter
  \endgroup
  \childdoctmp
}
%    \end{macrocode}

%\iffalse
%</package>
%\fi
%
\endinput
|\\
|\childdocby{|\textit{main}|}|\\
\end{tabular}
\end{center}
%
The directive |\childdocby| is similar to |\childdocof|
described in \secref{sec:include},
but the subsequent selection of content must be done manually.
To that end, both |\ifchilddoc| and |\ifchilddocmanual|
will be true upon processing of a part,
and the name of the part is stored in |\childdocname|.
Note that |\jobname| will be set to the filename of the current part
so that each part receives an individual |.aux| file
that does not interfere with the |.aux| file(s) of the main document.
This behaviour can be altered by the alternative form
|\childdocby[*]{|\textit{main}|}| (with a non-empty optional argument)
which uses the |.aux| file of the main document
by setting |\jobname| to \textit{main}.

%%%%%%%%%%%%%%%%%%%%%%%%%%%%%%%%%%%%%%%%%%%%%%%%%%%%%%%%%%%%%%%%%%%%%%%%%%%%%%%%
\subsection{Driver Development}
\label{sec:driver}

The \textsf{childdoc} mechanism can also be use for the development
of definition files such as \LaTeX{} styles or classes.
This case differs from the above setup with multiple parts
included by |\include| in that no |\includeonly| should be invoked.
This can be achieved by starting the include file
(before |\ProvidesPackage|) with:
%
\begin{center}
\begin{tabular}{l}
|% \iffalse
%
% childdoc.dtx Copyright (C) 2017-2018 Niklas Beisert
%
% This work may be distributed and/or modified under the
% conditions of the LaTeX Project Public License, either version 1.3
% of this license or (at your option) any later version.
% The latest version of this license is in
%   http://www.latex-project.org/lppl.txt
% and version 1.3 or later is part of all distributions of LaTeX
% version 2005/12/01 or later.
%
% This work has the LPPL maintenance status `maintained'.
%
% The Current Maintainer of this work is Niklas Beisert.
%
% This work consists of the files childdoc.dtx and childdoc.ins
% and the derived files childdoc.def and cdocsamp.tex with
% cdocsch1.tex, cdocsch2.tex, cdocsdrf.tex, cdocsfn1.tex, cdocsfn2.tex.
%
%<package>\ifdefined\childdocmain\endinput\fi
%<package>\ProvidesFile{childdoc.def}[2018/12/30 v2.0 child document driver]
%<samplemain>\ProvidesFile{cdocsamp.tex}[2018/12/30 v2.0 sample for childdoc]
%<*driver>
%\ProvidesFile{childdoc.drv}[2018/12/30 v2.0 childdoc reference manual file]
\PassOptionsToClass{10pt,a4paper}{article}
\documentclass{ltxdoc}

\usepackage[margin=35mm]{geometry}
\usepackage{hyperref}
\usepackage{hyperxmp}
\usepackage[usenames]{color}

\hypersetup{colorlinks=true}
\hypersetup{pdfstartview=FitH}
\hypersetup{pdfpagemode=UseNone}
\hypersetup{pdfsource={}}
\hypersetup{pdflang={en-UK}}
\hypersetup{pdfcopyright={Copyright 2017-2018 Niklas Beisert.
  This work may be distributed and/or modified under the
  conditions of the LaTeX Project Public License, either version 1.3
  of this license or (at your option) any later version.}}
\hypersetup{pdflicenseurl={http://www.latex-project.org/lppl.txt}}
\hypersetup{pdfcontactaddress={ETH Zurich, ITP, HIT K,
  Wolfgang-Pauli-Strasse 27}}
\hypersetup{pdfcontactpostcode={8093}}
\hypersetup{pdfcontactcity={Zurich}}
\hypersetup{pdfcontactcountry={Switzerland}}
\hypersetup{pdfcontactemail={nbeisert@itp.phys.ethz.ch}}
\hypersetup{pdfcontacturl={http://people.phys.ethz.ch/\xmptilde nbeisert/}}

\newcommand{\secref}[1]{\hyperref[#1]{section \ref*{#1}}}

\parskip1ex
\parindent0pt
\let\olditemize\itemize
\def\itemize{\olditemize\parskip0pt}

\begin{document}

\title{The \textsf{childdoc} Package}
\hypersetup{pdftitle={The childdoc Package}}
\author{Niklas Beisert\\[2ex]
  Institut f\"ur Theoretische Physik\\
  Eidgen\"ossische Technische Hochschule Z\"urich\\
  Wolfgang-Pauli-Strasse 27, 8093 Z\"urich, Switzerland\\[1ex]
  \href{mailto:nbeisert@itp.phys.ethz.ch}
  {\texttt{nbeisert@itp.phys.ethz.ch}}}
\hypersetup{pdfauthor={Niklas Beisert}}
\hypersetup{pdfsubject={Manual for the LaTeX2e Package childdoc}}
\date{30 December 2018, \textsf{v2.0}}
\maketitle

\begin{abstract}\noindent
\textsf{childdoc} is a \LaTeXe{} package
that enables the direct compilation
of document sections included by |\include|
to individual files.
\end{abstract}

\begingroup
\parskip0ex
\tableofcontents
\endgroup

%%%%%%%%%%%%%%%%%%%%%%%%%%%%%%%%%%%%%%%%%%%%%%%%%%%%%%%%%%%%%%%%%%%%%%%%%%%%%%%%
%%%%%%%%%%%%%%%%%%%%%%%%%%%%%%%%%%%%%%%%%%%%%%%%%%%%%%%%%%%%%%%%%%%%%%%%%%%%%%%%
\section{Introduction}

\LaTeX{} provides a mechanism to structure a large document (such as a book)
into a main file and several child files (containing the chapters)
using the |\include| command.
This mechanism is beneficial for documents
which span hundreds of pages in order to
make the source file(s) more manageable.
Moreover, compilation can be restricted to
selected child files by means of the |\includeonly| command.
The latter feature can be used to reduce the compilation time while editing
(this was significantly more useful in the earlier days of \LaTeX{})
or to generate a smaller document which is easier to navigate.
Another application of |\includeonly| is to generate
documents consisting of selected parts of the complete document.

However, there are a few drawbacks of the plain |\include| mechanism:
\begin{itemize}
\item
The child files cannot be compiled on their own,
they can only be compiled via the main file.
A naive editing environment
(such as a text editor with an option
to have the current file processed by \LaTeX)
may require one to switch to the main file before compiling;
attempting to compile the child file produces errors.
\item
The main file must be modified (each time)
to adjust the |\includeonly| command
to the present needs. This easily leaves the main file in a messy state.
\item
The generated document will always carry the filename
of the main document. This is inconvenient if
several child files are to be compiled and
to be kept for distribution.
\end{itemize}

The present package provides a simple interface
to make child files individually compilable by \LaTeX{}.
Compiling a child file then has the same effect as compiling
the main file with an |\includeonly| command
to select the appropriate child.
Moreover the generated document will carry the name of the child
rather than the main file.
This resolves all three above issues.

This feature is meant to make the editing of books,
thesis documents and lecture notes somewhat more convenient.
However, the package can also be used efficiently for
composing a series of documents (such as exercise sheets)
which are typically distributed individually.
It then assists the author in generating the individual documents
(potentially in different versions)
as well as a document containing the collected series.
Another application is in developing style files
or other kinds of included material
where compilation of the style file could redirect
to a sample or test file.

%%%%%%%%%%%%%%%%%%%%%%%%%%%%%%%%%%%%%%%%%%%%%%%%%%%%%%%%%%%%%%%%%%%%%%%%%%%%%%%%
%%%%%%%%%%%%%%%%%%%%%%%%%%%%%%%%%%%%%%%%%%%%%%%%%%%%%%%%%%%%%%%%%%%%%%%%%%%%%%%%
\section{Usage}

First of all, the package \textsf{childdoc} is \emph{not} a standard
\LaTeXe{} |.sty| style file! Therefore it needs to be invoked in
a non-standard way.

%%%%%%%%%%%%%%%%%%%%%%%%%%%%%%%%%%%%%%%%%%%%%%%%%%%%%%%%%%%%%%%%%%%%%%%%%%%%%%%%
\subsection{Included Files}
\label{sec:include}

%%%%%%%%%%%%%%%%%%%%%%%%%%%%%%%%%%%%%%%%
\DescribeMacro{\childdocmain}
To use the package, add the commands
\begin{center}
\begin{tabular}{l}
|\input{childdoc.def}|\\
|\childdocmain{}|\\
\end{tabular}
\end{center}
at the very top of the main \LaTeX{} file,
in particular \emph{before} the |\documentclass| statement!
The argument of |\childdocmain| should be left empty
(but it must be present).

%%%%%%%%%%%%%%%%%%%%%%%%%%%%%%%%%%%%%%%%
\DescribeMacro{\childdocof}
Furthermore, add the commands
\begin{center}
\begin{tabular}{l}
|\input{childdoc.def}|\\
|\childdocof{|\textit{main}|}|\\
\end{tabular}
\end{center}
at the top of every child file \textit{child}
which is included by |\include{|\textit{child}|}|
from within the main file
(or at least for those files to be compiled individually).
The argument \textit{main} must be the filename of the main file.

There are a couple of
considerations in setting up the main and child documents:

%%%%%%%%%%%%%%%%%%%%%%%%%%%%%%%%%%%%%%%%
\paragraph{Restrictions.}

Please note the following restrictions:
\begin{itemize}
\item
|\childdocmain| must be called with one argument \textit{main}
to ensure compatibility with earlier version of the package.
It must either be empty (|\childdocmain{}|)
or precisely match the filename of the main file in which it is specified.
See \secref{sec:detection} for further information.
\item
The filename \textit{main} must be specified without the |.tex| extension.
\item
The filename \textit{main} is case sensitive
(even in case-insensitive file systems)
due to internal string comparison.
\item
The argument \textit{main} should be fully expanded, it cannot be a macro.
\item
Subdirectories and special characters should be avoided in filenames.
\item
The command |\childdocmain{|\textit{main}|}| must be followed by a whitespace.
It should not be followed immediately by another command
or by a comment mark `|%|'.
This is because the \TeX{} parser reads the token immediately following
the argument of |\childdocmain| and puts it
at the beginning of every child section;
however, a white\-space is ignored.
\end{itemize}

%%%%%%%%%%%%%%%%%%%%%%%%%%%%%%%%%%%%%%%%
\paragraph{Content of Main File.}

It is advisable to place all content in the child files included by |\include|.
Any output contained in the main file will appear in all child documents
unless suppressed manually;
it cannot be suppressed automatically by the |\includeonly| directive
and thus should normally be avoided.
A method to include some content in the main file
by means of conditional processing is described in \secref{sec:conditional}.

%%%%%%%%%%%%%%%%%%%%%%%%%%%%%%%%%%%%%%%%
\paragraph{Page Numbering.}

When only a part of the document is compiled,
the appropriate numbering of pages
(as well as other status parameters)
is determined from the |.aux| files.
The latter contain information from previous passes.
However this information needs to propagate through
all intermediate child documents.
Therefore the page numbering in child documents may well
be inconsistent until the complete document is compiled at least once.

A useful (if unconventional) way to always ensure a consistent
page numbering is to restart the numbering in each child document
and denote the pages by `\textit{child}|.|\textit{page}'
where \textit{child} represents the chapter/section number of the child file.
This can be achieved by the command
|\numberwithin{page}{|\textit{child}|}|
of the \textsf{amsmath} package
where \textit{child} can be |chapter| or |section|
depending on the chosen structuring.
Alternatively, one can modify the macro |\thepage| appropriately
and reset the counter |page| at the start of each child file.

%%%%%%%%%%%%%%%%%%%%%%%%%%%%%%%%%%%%%%%%%%%%%%%%%%%%%%%%%%%%%%%%%%%%%%%%%%%%%%%%
\subsection{Conditional Processing}
\label{sec:conditional}

The package provides a mechanism to compile different versions
of a document. To customise the versions further some conditional processing
can come in handy to distinguish which version is being compiled.
The package provides two macros to describe the compilation context:

%%%%%%%%%%%%%%%%%%%%%%%%%%%%%%%%%%%%%%%%
\DescribeMacro{\ifchilddoc}
The conditional |\ifchilddoc| distinguishes between the compilation of
child documents and the main document:
%
\begin{center}
|\ifchilddoc |\textit{child-code}| |[|\||else |\textit{main-code}]| \||fi|
\end{center}

%%%%%%%%%%%%%%%%%%%%%%%%%%%%%%%%%%%%%%%%
\DescribeMacro{\childdocname}
\DescribeMacro{\childdocjob}
The macro |\childdocname| contains the filename (without extension)
of the main or child file being processed.
Note that |\childdocjob| will always contain the name of the main file.

%%%%%%%%%%%%%%%%%%%%%%%%%%%%%%%%%%%%%%%%
\paragraph{Title Page.}

Conditional processing can be used to include a title or banner page
in the main document when proper precautions are taken.
Importantly, the code in the main file should ensure that the page counter
(as well as other status parameters which are stored in the |.aux| files)
takes the same value after the conditional processing.
Otherwise the page numbers may take divergent values
depending on which part is compiled.

For example, a title page could be declared by:
%
\begin{center}
\begin{tabular}{l}
|\ifchilddoc\||else|\\
|\addtocounter{page}{-1}|\\
\textit{code for title page}\\
|\newpage|\\
|\||fi|
\end{tabular}
\end{center}
%
A banner page for the child documents can be generated by:
%
\begin{center}
\begin{tabular}{l}
|\ifchilddoc|\\
|\addtocounter{page}{-1}|\\
\textit{code for banner page}\\
|\newpage|\\
|\||fi|
\end{tabular}
\end{center}
%
Here one could write a message such as:
\begin{center}
|This is the part \childdocname{} of \childdocjob{}.|
\end{center}

%%%%%%%%%%%%%%%%%%%%%%%%%%%%%%%%%%%%%%%%%%%%%%%%%%%%%%%%%%%%%%%%%%%%%%%%%%%%%%%%
\subsection{Flags}
\label{sec:flags}

The package makes it easy to generate different versions
of the main or child documents.
To this end compilation flags can be defined
and assigned different default values.
They will be particularly useful in conjunction
with the forwarding mechanism described in \secref{sec:forward}.

For example, it may be useful to have a flag |\version|
which can be set to |draft| or |final|.
The document source will contain some conditional code
depending on the value of |\version|.
Suppose further, the flag should default to |final| for the main file
and to |draft| for child files
which is a natural assignment for editing the document.
This is achieved by placing the following code
in the preamble of the main document
(below the |\childdocmain| directive):
%
\begin{center}
\begin{tabular}{l}
|\ifchilddoc|\\
|\providecommand{\version}{draft}|\\
|\||else|\\
|\providecommand{\version}{final}|\\
|\||fi|
\end{tabular}
\end{center}
%
The definition by |\providecommand| makes sure
that previous definitions are not overwritten.
Further statements |\providecommand{\version}{...}|
can thus be added before the above code to override it.

For the main file, one might add a line
(between |\childdocmain| and the above block)
%
\begin{center}
|%\ifchilddoc\||else\providecommand{\version}{draft}\||fi|
\end{center}
%
which can be uncommented to produce a draft version.
Likewise one can add a line to the very top of a child file
(above the |\childdocof{|\textit{main}|}| directive)
%
\begin{center}
|%\providecommand{\version}{final}|
\end{center}
%
which can be uncommented to produce the final version of this child document.

%%%%%%%%%%%%%%%%%%%%%%%%%%%%%%%%%%%%%%%%%%%%%%%%%%%%%%%%%%%%%%%%%%%%%%%%%%%%%%%%
\subsection{Forwarding}
\label{sec:forward}

Different versions of the main or child documents
using compilation flags as described in \secref{sec:flags}
can be (permanently) stored in different files
for convenient compilation, viewing and distribution.
To this end, the package defines a command
to pass on compilation to a different file:

%%%%%%%%%%%%%%%%%%%%%%%%%%%%%%%%%%%%%%%%
\DescribeMacro{\childdocforward}
The command |\childdocforward| redirects processing to
another source file:
%
\begin{center}
\begin{tabular}{l}
|\input{childdoc.def}|\\
|\childdocforward[|\textit{main}|]{|\textit{dest}|}|\\
\end{tabular}
\end{center}
%
The argument \textit{dest} is the destination file
(without extension).
It should be the main file or one of the child files.
Note that further \textsf{childdoc} directives
such as |\childdocof| and |\childdocforward|
in the indicated file will be processed in this form.
The optional argument \textit{main}
passes on directly to the main file \textit{main}
while pretending to compile the child \textit{dest}.
This form behaves as if \textit{dest}
issues |\childdocof{|\textit{main}|}| right away,
and no further \textsf{childdoc} directives will be processed.

%%%%%%%%%%%%%%%%%%%%%%%%%%%%%%%%%%%%%%%%
\DescribeMacro{\...prefix}
In the alternative form |\childdocforwardprefix|,
%
\begin{center}
\begin{tabular}{l}
|\input{childdoc.def}|\\
|\childdocforwardprefix[|\textit{main}|]{|\textit{prefix}|}{|\textit{dest}|}|
\end{tabular}
\end{center}
%
the destination file is determined by a pattern
depending on the current file:
To make this work, the current file must be called
`{\textit{prefix}\hspace{0.2em}\textit{suffix}}'
with \textit{prefix} matching precisely the argument.
Processing is then passed on to the file
`{\textit{dest}\hspace{0.2em}\textit{suffix}}'.
Surely, the same effect is achieved by
directly specifying the
argument `{\textit{dest}\hspace{0.2em}\textit{suffix}}'
in the first form.
However, that requires to set up a different file
for each child. With the alternative form of the command
all these files can have exactly the same content
which simplifies setting them up and maintaining them.

For example, the following file |draft.tex|
with a compilation flag |\version| as described in \secref{sec:flags}
compiles the main document as a draft:
%
\begin{center}
\begin{tabular}{l}
|\def\version{draft}|\\
|\input{childdoc.def}|\\
|\childdocforward{|\textit{main}|}|
\end{tabular}
\end{center}
%
Likewise, the following files |final|\textit{nn}|.tex|
compile the final version of the child document
|child|\textit{nn}|.tex|:
%
\begin{center}
\begin{tabular}{l}
|\def\version{final}|\\
|\input{childdoc.def}|\\
|\childdocforwardprefix{final}{child}|
\end{tabular}
\end{center}
%

Note that when several versions of a main file and/or of each child file
are to be generated, it may be convenient to set up a |Makefile| or
shell script to automatise the process.

%%%%%%%%%%%%%%%%%%%%%%%%%%%%%%%%%%%%%%%%%%%%%%%%%%%%%%%%%%%%%%%%%%%%%%%%%%%%%%%%
\subsection{Command Line Processing}
\label{sec:commandline}

The effect of redirection files can also be achieved by invoking
the \LaTeX{} compiler with a more elaborate command line.
Most conveniently this should be done as part
of a shell script or a |Makefile|.

When using \textsf{childdoc} in the main file, the following
command lines effectively perform a redirection
(note that depending on the shell being used,
backslashes may have to be doubled: `|\|' $\to$ `|\\|'):
%
\begin{center}
|... -jobname "|\textit{target}|" |\\|"|[\textit{flags}]%
|\input{childdoc.def}\childdocforward[|\textit{main}|]{|\textit{dest}|}"|
\end{center}
%
Here \textit{target} is the name of the output file,
\textit{main} is the name of the main file
and \textit{dest} is the name of the main or child file to be processed
(all filenames without extensions).
The optional argument \textit{main} can be omitted
if \textit{main} matches \textit{dest}.
Optionally, compilation \textit{flags} can be defined via |\def| commands.
This command line makes the \TeX{} engine believe
it is compiling the file \textit{target}
whose content is specified as the latter parameter.
The provided code then forwards the processing to
\textit{main} or \textit{dest} as described in \secref{sec:forward}.

%%%%%%%%%%%%%%%%%%%%%%%%%%%%%%%%%%%%%%%%%%%%%%%%%%%%%%%%%%%%%%%%%%%%%%%%%%%%%%%%
\subsection{Include by Input}
\label{sec:input}

Including child documents by |\include| has some restrictions by design.
Most notably, the content of a child document always occupies
its own set of pages; pages cannot be shared between child documents.
Usually, this behaviour makes perfect sense
because each child document contain an essential part of the document.
However, in some situations it may be desirable to compose
a document from a collection of parts
without having mandatory page breaks between then.
For this case, the package
provides a mechanism to include parts
by |\input| which can also be processed individually.
However, by construction this mechanism
requires manual handling of the content to be output.

%%%%%%%%%%%%%%%%%%%%%%%%%%%%%%%%%%%%%%%%
\DescribeMacro{\ifchilddocmanual}
The main file should be prepared as usual, see \secref{sec:include}.
However, the document body must make a distinction
between processing of an individual part and of the main document, e.g.:
%
\begin{center}
\begin{tabular}{l}
|\ifchilddocmanual|\\
|\input{\childdocname}|\\
|\||else|\\
\textit{document body with }|\input{|\textit{part}|}|\\
|\||fi|
\end{tabular}
\end{center}
%
The conditional |\ifchilddocmanual| is true whenever
a part to be included by |\input| is being compiled,
and the name of the part is stored in |\childdocname|.

%%%%%%%%%%%%%%%%%%%%%%%%%%%%%%%%%%%%%%%%
\DescribeMacro{\childdocby}
Each part to be included by |\input| should start with:
%
\begin{center}
\begin{tabular}{l}
|\input{childdoc.def}|\\
|\childdocby{|\textit{main}|}|\\
\end{tabular}
\end{center}
%
The directive |\childdocby| is similar to |\childdocof|
described in \secref{sec:include},
but the subsequent selection of content must be done manually.
To that end, both |\ifchilddoc| and |\ifchilddocmanual|
will be true upon processing of a part,
and the name of the part is stored in |\childdocname|.
Note that |\jobname| will be set to the filename of the current part
so that each part receives an individual |.aux| file
that does not interfere with the |.aux| file(s) of the main document.
This behaviour can be altered by the alternative form
|\childdocby[*]{|\textit{main}|}| (with a non-empty optional argument)
which uses the |.aux| file of the main document
by setting |\jobname| to \textit{main}.

%%%%%%%%%%%%%%%%%%%%%%%%%%%%%%%%%%%%%%%%%%%%%%%%%%%%%%%%%%%%%%%%%%%%%%%%%%%%%%%%
\subsection{Driver Development}
\label{sec:driver}

The \textsf{childdoc} mechanism can also be use for the development
of definition files such as \LaTeX{} styles or classes.
This case differs from the above setup with multiple parts
included by |\include| in that no |\includeonly| should be invoked.
This can be achieved by starting the include file
(before |\ProvidesPackage|) with:
%
\begin{center}
\begin{tabular}{l}
|\input{childdoc.def}|\\
|\childdocforward{|\textit{main}|}|\\
\end{tabular}
\end{center}
%
or alternatively with:
%
\begin{center}
\begin{tabular}{l}
|\input{childdoc.def}|\\
|\childdocby{|\textit{main}|}|\\
\end{tabular}
\end{center}
%
Both forms have slightly different effects as described above.
The main file is prepared as usual, see \secref{sec:include}.

%%%%%%%%%%%%%%%%%%%%%%%%%%%%%%%%%%%%%%%%%%%%%%%%%%%%%%%%%%%%%%%%%%%%%%%%%%%%%%%%
\subsection{Legacy Detection}
\label{sec:detection}

The directive |\childdocmain| in the main file can detect
whether the complete document or merely a child is to be compiled
even without using the directive |\childdocof|.
This method is deprecated because it is less robust
and there is no compelling reason to use it;
it is merely provided for backward compatibility
and it may be removed in future versions.

If the detection mechanism is to be used,
it is mandatory to correctly specify
the filename of the main file as the argument of |\childdocmain|:
%
\begin{center}
\begin{tabular}{l}
|\input{childdoc.def}|\\
|\childdocmain{|\textit{main}|}|\\
\end{tabular}
\end{center}
%
If |\jobname| does not match the argument \textit{main} of |\childdocmain|,
it is assumed that |\jobname| points to the child file to be compiled.
When using |\childdocmain| with the main file specified as argument,
it suffices to start a child file
with just |\input{|\textit{main}|}|
without loading of the package and using |\childdocof|.
If instead all processing is done
with the appropriate \textsf{childdoc} directives,
the argument of \textit{main} of |\childdocmain| can be empty.

An alternative version of the command line processing described
in \secref{sec:commandline} using the detection mechanism reads:
%
\begin{center}
|... -jobname "|\textit{target}|" "|[\textit{flags}]%
[|\def\jobname{|\textit{dest}|}|]|\input{|\textit{main}|}"|
\end{center}

%%%%%%%%%%%%%%%%%%%%%%%%%%%%%%%%%%%%%%%%%%%%%%%%%%%%%%%%%%%%%%%%%%%%%%%%%%%%%%%%
\subsection{Manual Code}
\label{sec:manual}

In case one cannot be certain whether the definitions file |childdoc.def|
is installed on the target \TeX{} distribution
and one prefers not to ship it,
it is conceivable to paste a few relevant commands into the sources.

To that end, drop all statements |\input{childdoc.def}|
and perform the replacements as outlined below.
Instead of |\childdocmain{|\textit{main}|}| add the following code
to the top of the main file:
%
\begin{center}
\begin{tabular}{l}
|\||ifdefined\childdocname\endinput\||fi\newif\ifchilddoc|\\
|\edef\childdocname{\scantokens\expandafter{\jobname\noexpand}}|\\
|\def\childdocmain{|\textit{main}|}\||ifx\childdocmain\childdocname\||else|\\
|\childdoctrue\includeonly{\childdocname}\let\jobname\childdocmain\||fi|\\
\end{tabular}
\end{center}
%
Instead of |\childdocof{|\textit{main}|}| just include the main file
at the top of each child file:
%
\begin{center}
|\input{|\textit{main}|}|
\end{center}
%
A simple redirection |\childdocforward{|\textit{dest}|}| is achieved by:
%
\begin{center}
|\def\jobname{|\textit{dest}|}\input{\jobname}|
\end{center}
%
The redirection with prefix
|\childdocforwardprefix[|\textit{prefix}|]{|\textit{dest}|}|
is accomplished by:
%
\begin{center}
\begin{tabular}{l}
|{\edef\jobname{\scantokens\expandafter{\jobname\noexpand}}|\\
|\def\redirectjob |\textit{prefix}|#1~~~{\gdef\jobname{|\textit{dest}|#1}}|\\
|\expandafter\redirectjob\jobname~~~}\input{\jobname}|
\end{tabular}
\end{center}

In an alternative approach,
child documents can be compiled by a specific command line
without additional code or specific definitions:
%
\begin{center}
|... -jobname "|\textit{target}|" "|[\textit{flags}]%
|\includeonly{|\textit{dest}|}\input{|\textit{main}|}"|
\end{center}
%

%%%%%%%%%%%%%%%%%%%%%%%%%%%%%%%%%%%%%%%%%%%%%%%%%%%%%%%%%%%%%%%%%%%%%%%%%%%%%%%%
%%%%%%%%%%%%%%%%%%%%%%%%%%%%%%%%%%%%%%%%%%%%%%%%%%%%%%%%%%%%%%%%%%%%%%%%%%%%%%%%
\section{Information}

%%%%%%%%%%%%%%%%%%%%%%%%%%%%%%%%%%%%%%%%%%%%%%%%%%%%%%%%%%%%%%%%%%%%%%%%%%%%%%%%
\subsection{Copyright}

Copyright \copyright{} 2017--2018 Niklas Beisert

This work may be distributed and/or modified under the
conditions of the \LaTeX{} Project Public License, either version 1.3
of this license or (at your option) any later version.
The latest version of this license is in
  \url{http://www.latex-project.org/lppl.txt}
and version 1.3 or later is part of all distributions of \LaTeX{}
version 2005/12/01 or later.

This work has the LPPL maintenance status `maintained'.

The Current Maintainer of this work is Niklas Beisert.

This work consists of the files |README.txt|, |childdoc.ins| and |childdoc.dtx|
as well as the derived files |childdoc.def|, |cdocsamp.tex|
with |cdocsch1.tex|, |cdocsch2.tex|, |cdocspt3.tex|, |cdocspt4.tex|,
|cdocsdrf.tex|, |cdocsfn1.tex|, |cdocsfn2.tex|
as well as |childdoc.pdf|.

%%%%%%%%%%%%%%%%%%%%%%%%%%%%%%%%%%%%%%%%%%%%%%%%%%%%%%%%%%%%%%%%%%%%%%%%%%%%%%%%
\subsection{Files and Installation}

The package consists of the files:
%
\begin{center}
\begin{tabular}{ll}
    |README.txt|   & readme file \\
    |childdoc.ins| & installation file \\
    |childdoc.dtx| & source file \\
    |childdoc.def| & definition file \\
    |cdocsamp.tex| & sample main file \\
    |cdocsch1.tex| & sample include file \\
    |cdocsch2.tex| & sample include file \\
    |cdocspt3.tex| & sample part file \\
    |cdocspt4.tex| & sample part file \\
    |cdocsdrf.tex| & sample redirection file \\
    |cdocsfn1.tex| & sample redirection file \\
    |cdocsfn2.tex| & sample redirection file \\
    |childdoc.pdf| & manual
\end{tabular}
\end{center}
%
The distribution consists of the files
|README.txt|, |childdoc.ins| and |childdoc.dtx|.
%
\begin{itemize}
\item
Run (pdf)\LaTeX{} on |childdoc.dtx|
to compile the manual |childdoc.pdf| (this file).
\item
Run \LaTeX{} on |childdoc.ins| to create the definitions file |childdoc.def|
and the sample |cdocsamp.tex| with include files
|cdocsch1.tex|, |cdocsch2.tex|, |cdocspt3.tex|, |cdocspt4.tex|,
|cdocsdrf.tex|, |cdocsfn1.tex|, |cdocsfn2.tex|.
Then copy the file |childdoc.def| to an appropriate directory of your \LaTeX{}
distribution, e.g.\ \textit{texmf-root}|/tex/latex/childdoc|.
\end{itemize}

%%%%%%%%%%%%%%%%%%%%%%%%%%%%%%%%%%%%%%%%%%%%%%%%%%%%%%%%%%%%%%%%%%%%%%%%%%%%%%%%
\subsection{Related CTAN Packages}

There are several other packages which offer a similar functionality:
%
\begin{itemize}
\item
The packages
\href{http://ctan.org/pkg/docmute}{\textsf{docmute}},
\href{http://ctan.org/pkg/includex}{\textsf{includex}} and
\href{http://ctan.org/pkg/standalone}{\textsf{standalone}}
provide commands to include only the document body of
a child file thus allowing both files to be compiled individually.
\item
The packages \href{http://ctan.org/pkg/subdocs}{\textsf{subdocs}}
and \href{http://ctan.org/pkg/subfiles}{\textsf{subfiles}}
provide structures in which the main and child documents can be
encapsulated and allowing them to be compiled individually.
The inclusion mechanism is different from the conventional |\include|.
\item
The package \href{http://ctan.org/pkg/combine}{\textsf{combine}}
is an elaborate solution to combine several documents into one.
\end{itemize}
%
See also the CTAN topic \href{http://ctan.org/topic/subdocs}{\textsf{subdocs}}
for further related packages.
The present package differs from the above solutions in that
a document structure constructed with the conventional |\include| mechanism
just needs two extra commands at the top of every file
such that all constituent files can be compiled individually.

%%%%%%%%%%%%%%%%%%%%%%%%%%%%%%%%%%%%%%%%%%%%%%%%%%%%%%%%%%%%%%%%%%%%%%%%%%%%%%%%
%\subsection{Feature Suggestions}
%
%The following is a list of features which may be useful for future
%versions of this package:
%%
%\begin{itemize}
%\item
%\ldots
%\end{itemize}

%%%%%%%%%%%%%%%%%%%%%%%%%%%%%%%%%%%%%%%%%%%%%%%%%%%%%%%%%%%%%%%%%%%%%%%%%%%%%%%%
\subsection{Revision History}

%%%%%%%%%%%%%%%%%%%%%%%%%%%%%%%%%%%%%%%%
\paragraph{v2.0:} 2018/12/30

\begin{itemize}
\item
immediate forward processing
\item
added |\childdocby| mechanism
\item
manual restructured
\end{itemize}

%%%%%%%%%%%%%%%%%%%%%%%%%%%%%%%%%%%%%%%%
\paragraph{v1.6:} 2018/01/17

\begin{itemize}
\item
application for development of include files
\item
corrections to manual
\end{itemize}

%%%%%%%%%%%%%%%%%%%%%%%%%%%%%%%%%%%%%%%%
\paragraph{v1.5:} 2017/05/21

\begin{itemize}
\item
more complete structuring introduced
\item
|\childdocof| introduced
\item
|\childdoc| renamed to |\childdocmain|
\item
|\childredirect| renamed to |\childdocforward| and |\childdocforwardprefix|
and functionality expanded
\end{itemize}

%%%%%%%%%%%%%%%%%%%%%%%%%%%%%%%%%%%%%%%%
\paragraph{v1.0:} 2017/04/27

\begin{itemize}
\item
manual and install package
\item
first version published on CTAN
\end{itemize}

%%%%%%%%%%%%%%%%%%%%%%%%%%%%%%%%%%%%%%%%
\paragraph{v0.6:} 2017/04/26

\begin{itemize}
\item
redirection mechanism added
\end{itemize}

%%%%%%%%%%%%%%%%%%%%%%%%%%%%%%%%%%%%%%%%
\paragraph{v0.5:} 2017/04/26

\begin{itemize}
\item
functionality in definition file
\end{itemize}


%%%%%%%%%%%%%%%%%%%%%%%%%%%%%%%%%%%%%%%%%%%%%%%%%%%%%%%%%%%%%%%%%%%%%%%%%%%%%%%%
%%%%%%%%%%%%%%%%%%%%%%%%%%%%%%%%%%%%%%%%%%%%%%%%%%%%%%%%%%%%%%%%%%%%%%%%%%%%%%%%
%%%%%%%%%%%%%%%%%%%%%%%%%%%%%%%%%%%%%%%%%%%%%%%%%%%%%%%%%%%%%%%%%%%%%%%%%%%%%%%%
\appendix

\settowidth\MacroIndent{\rmfamily\scriptsize 000\ }

 \DocInput{childdoc.dtx}

\end{document}
%</driver>
% \fi
%
% %%%%%%%%%%%%%%%%%%%%%%%%%%%%%%%%%%%%%%%%%%%%%%%%%%%%%%%%%%%%%%%%%%%%%%%%%%%%%%
% %%%%%%%%%%%%%%%%%%%%%%%%%%%%%%%%%%%%%%%%%%%%%%%%%%%%%%%%%%%%%%%%%%%%%%%%%%%%%%
% \section{Sample}
%\iffalse
%<*samplemain>
%\fi
%
% The following presents a sample document
% with two chapters, two parts, a title page,
% a compile flag as well as three forwarding files to set the flag.
% It consists of eight |.tex| files:
% \begin{center}
% \begin{tabular}{ll}
% |cdocsamp.tex|&main file\\
% |cdocsch1.tex|&include file for chapter 1\\
% |cdocsch2.tex|&include file for chapter 2\\
% |cdocspt3.tex|&include file for part 3\\
% |cdocspt4.tex|&include file for part 4\\
% |cdocsdrf.tex|&forwarding file for main file in draft mode\\
% |cdocsfi1.tex|&forwarding file for final version of chapter 1\\
% |cdocsfi2.tex|&forwarding file for final version of chapter 2\\
% \end{tabular}
% \end{center}
% Each of the eight files can be compiled directly by the \LaTeX{} compiler.
%
% %%%%%%%%%%%%%%%%%%%%%%%%%%%%%%%%%%%%%%
% \paragraph{Main File.}
%
% The main file is called |cdocsamp.tex|.
%
% Load the \textsf{childdoc} definitions and
% declare the filename for the main document:
%    \begin{macrocode}
\input{childdoc.def}
\childdocmain{}
%    \end{macrocode}

% Optional override for |\version| flag:
%    \begin{macrocode}
%%\ifchilddoc\else\providecommand{\version}{draft}\fi
%    \end{macrocode}

% Define the default values for the |\version| flag
% (|final| for the main file and |draft| for childs):
%    \begin{macrocode}
\ifchilddoc
\providecommand{\version}{draft}
\else
\providecommand{\version}{final}
\fi
%    \end{macrocode}

% Load the standard document class:
%    \begin{macrocode}
\documentclass[12pt]{article}
%    \end{macrocode}

% Start the document body:
%    \begin{macrocode}
\begin{document}
%    \end{macrocode}

% Declare a title page.
% Print title, part of document being processed and version flag:
%    \begin{macrocode}
\addtocounter{page}{-1}
\begin{center}
{\LARGE\bfseries{}childdoc example\par}
\vspace{1cm}
\ifchilddoc
\ifchilddocmanual part\else chapter\fi:
`\childdocname' of `\childdocjob'\par
\else
main document: `\childdocjob'\par
\fi
version: \version\par
\end{center}
\newpage
%    \end{macrocode}

% Manually include selected file,
% otherwise process as usual:
%    \begin{macrocode}
\ifchilddocmanual
\section*{part `\childdocname'}
\input{\childdocname}
\else
%    \end{macrocode}

% Include the two chapters:
%    \begin{macrocode}
\include{cdocsch1}
\include{cdocsch2}
%    \end{macrocode}

% Include the two parts unless only chapters should be displayed:
%    \begin{macrocode}
\ifchilddoc\else
\section{part three}
\input{cdocspt3}
\section{part four}
\input{cdocspt4}
\fi
%    \end{macrocode}

% Process as usual until here:
%    \begin{macrocode}
\fi
%    \end{macrocode}

% End of document body:
%    \begin{macrocode}
\end{document}
%    \end{macrocode}
%\iffalse
%</samplemain>
%\fi
%
% %%%%%%%%%%%%%%%%%%%%%%%%%%%%%%%%%%%%%%
% \paragraph{Chapter Include Files.}
%
% The include files are called |cdocsch1.tex| and |cdocsch2.tex|.
%
%\iffalse
%<*samplechap1|samplechap2>
%\fi

% Optional override for |\version| flag:
%    \begin{macrocode}
%%\providecommand{\version}{final}
%    \end{macrocode}

% Include the main document:
%    \begin{macrocode}
\input{childdoc.def}
\childdocof{cdocsamp}
%    \end{macrocode}

%\iffalse
%</samplechap1|samplechap2>
%\fi
%
%\iffalse
%<*samplechap1>
%\fi
% Some text for chapter 1:
%    \begin{macrocode}
\section{one}
some text in chapter one
%    \end{macrocode}

%\iffalse
%</samplechap1>
%\fi
% Some text for chapter 2:
%\iffalse
%<*samplechap2>
%\fi
%    \begin{macrocode}
\section{two}
more text in chapter two
%    \end{macrocode}

%\iffalse
%</samplechap2>
%\fi
%
% %%%%%%%%%%%%%%%%%%%%%%%%%%%%%%%%%%%%%%
% \paragraph{Part Include Files.}
%
% The include files are called |cdocspt3.tex| and |cdocspt4.tex|.
%
%\iffalse
%<*samplepart3|samplepart4>
%\fi

% Optional override for |\version| flag:
%    \begin{macrocode}
%%\providecommand{\version}{final}
%    \end{macrocode}

% Include the main document:
%    \begin{macrocode}
\input{childdoc.def}
\childdocby{cdocsamp}
%    \end{macrocode}

%\iffalse
%</samplepart3|samplepart4>
%\fi
%
%\iffalse
%<*samplepart3>
%\fi
% Some text for part 3:
%    \begin{macrocode}
some text in part three
%    \end{macrocode}

%\iffalse
%</samplepart3>
%\fi
% Some text for part 4:
%\iffalse
%<*samplepart4>
%\fi
%    \begin{macrocode}
more text in part four
%    \end{macrocode}

%\iffalse
%</samplepart4>
%\fi
%
% %%%%%%%%%%%%%%%%%%%%%%%%%%%%%%%%%%%%%%
% \paragraph{Forwarding for a Complete Draft.}
%
% The following forwarding file |cdocsdrf.tex|
% compiles the main document in draft mode:
%\iffalse
%<*sampledraft>
%\fi
%    \begin{macrocode}
\def\version{draft}
\input{childdoc.def}
\childdocforward{cdocsamp}
%    \end{macrocode}

%\iffalse
%</sampledraft>
%\fi
%
% %%%%%%%%%%%%%%%%%%%%%%%%%%%%%%%%%%%%%%
% \paragraph{Forwarding for Final Version of the Chapters.}
%
% The following forwarding files |cdocsfn1.tex| and |cdocsfn2.tex|
% (with identical content)
% compile the final versions of the child documents
% |cdocsch1.tex| and |cdocsch2.tex|, respectively:
%\iffalse
%<*samplefinal>
%\fi
%    \begin{macrocode}
\def\version{final}
\input{childdoc.def}
\childdocforwardprefix[cdocsamp]{cdocsfn}{cdocsch}
%    \end{macrocode}

%\iffalse
%</samplefinal>
%\fi
%
% %%%%%%%%%%%%%%%%%%%%%%%%%%%%%%%%%%%%%%
% \paragraph{Command Line Processing.}
%
% The following three command lines generate the output files
% |cdocscld|, |cdocscl1| and |cdocscl2|
% which should be identical to
% |cdocsdrf|, |cdocsch1| and |cdocsfn2|, respectively:
% \begin{center}
% \begin{tabular}{l}
% |latex -jobname cdocscld \|\\
% |  "\def\version{draft}\input{childdoc.def}\childdocforward{cdocsamp}"|\\
% |latex -jobname cdocscl1 \|\\
% |  "\input{childdoc.def}\childdocforward[cdocsamp]{cdocsch1}"|\\
% |latex -jobname cdocscl2 \|\\
% |  "\def\version{final}\input{childdoc.def}\childdocforward{cdocsch2}"|
% \end{tabular}
% \end{center}
% Note that the trailing backslash on each first line
% merely continues the input to the second line
% (for convenient cut ant paste).
% Furthermore, the command |latex| can be replaced by any
% of its alternative versions such as |pdflatex|.
%
% %%%%%%%%%%%%%%%%%%%%%%%%%%%%%%%%%%%%%%%%%%%%%%%%%%%%%%%%%%%%%%%%%%%%%%%%%%%%%%
% %%%%%%%%%%%%%%%%%%%%%%%%%%%%%%%%%%%%%%%%%%%%%%%%%%%%%%%%%%%%%%%%%%%%%%%%%%%%%%
% \section{Implementation}
%\iffalse
%<*package>
%\fi
%
% This section describes the definitions file |childdoc.def|.

% The definitions cannot be loaded using |\usepackage| or |\RequirePackage|
% which has a mechanism to prevent loading a style file more than once.
% When loading the definitions by means of |\input|
% multiple instances have to be prevented manually:
%\iffalse
%This code needs to be before the `\ProvidesFile' directive
%which is defined at the beginning of this file.
%Therefore it is also placed there and commented out here.
%</package>
%<*discard>
%\fi
%    \begin{macrocode}
\ifdefined\childdocmain\endinput\fi
%    \end{macrocode}
%\iffalse
%</discard>
%<*package>
%\fi
%
% \macro{\ifchilddoc}
% \macro{\ifchilddocmanual}
% The conditional |\ifchilddoc| tells whether a
% child (true) or main (false) document is being compiled.
% The conditional |\ifchilddocmanual| tells whether
% the |\includeonly| mechanism is used (false) or
% the selection of child files must be performed manually (true).
% The definitions initialise to false:
%    \begin{macrocode}
\newif\ifchilddoc
\newif\ifchilddocmanual
%    \end{macrocode}

% \macro{\childdocname}
% \macro{\childdocjob}
% The macro |\childdocname| stores the name of the main document
% to be compiled. The macro |\childdocjob| stores the name of
% the document on which the \LaTeX{} compiler was originally invoked.
% The content of |\jobname| cannot be compared
% to filenames specified in the source due to different catcodes.
% The following code rescans |\jobname|, stores the result
% in |\childdocname| and saves a copy in |\childdocjob|:
%    \begin{macrocode}
\edef\childdocname{\scantokens\expandafter{\jobname\noexpand}}
\let\childdocjob\childdocname
%    \end{macrocode}

% \macro{\childdocdisable}
% The macro |\childdocdisable| prevents the main file
% from being processed more than once.
% At this stage, the main document command |\childdocmain|
% is assumed to be called once again where it should do nothing.
% Any subsequent call to it should prevent
% a secondary processing of the main document
% It overwrites the forwarding commands
% |\childdocof| and |\childdocforward|
% with empty macros to prevent further inclusions of the main document:
%    \begin{macrocode}
\newcommand{\childdocdisable}
{
  \renewcommand{\childdocmain}[1]{\renewcommand{\childdocmain}[1]{\endinput}}
  \renewcommand{\childdocof}[1]{}
  \renewcommand{\childdocby}[2][]{}
  \renewcommand{\childdocforward}[2][]{}
  \renewcommand{\childdocdisable}{}
}
%    \end{macrocode}

% \macro{\childdocmain}
% The macro |\childdocmain| is to be called at the top of the main file
% with nothing or the main filename (without extension) as argument.
% First, it breaks loops.
% If the argument is not empty and does not match |\childdocname|
% (which is set by the first inclusion of |childdoc.def|),
% |\ifchilddoc| is set to true, |\includeonly| is applied to the child file
% and |\jobname| is set to the main file
% (for proper handling of |.aux| files):
%    \begin{macrocode}
\newcommand{\childdocmain}[1]
{
  \childdocdisable\childdocmain{}
  \if?#1?\else
    \begingroup
      \def\childdoctmp{#1}
      \ifx\childdoctmp\childdocname
        \def\childdoctmp{}
      \else
        \def\childdoctmp
        {
          \childdoctrue
          \includeonly{\childdocname}
          \def\childdocjob{#1}
          \def\jobname{#1}
        }
      \fi
      \expandafter
    \endgroup
    \childdoctmp
  \fi
}
%    \end{macrocode}

% \macro{\childdocof}
% The command |\childdocof| redirects
% compilation to the main file |#1|.
%    \begin{macrocode}
\newcommand{\childdocof}[1]
{
  \childdocdisable
  \childdoctrue
  \includeonly{\childdocname}
  \def\jobname{#1}
  \def\childdocjob{#1}
  \input{#1}
}
%    \end{macrocode}

% \macro{\childdocby}
% The command |\childdocby| ....
%    \begin{macrocode}
\newcommand{\childdocby}[2][]
{
  \childdocdisable
  \childdoctrue
  \childdocmanualtrue
  \if?#1?\else
    \def\jobname{#2}
  \fi
  \def\childdocjob{#2}
  \input{#2}
  \endinput
}
%    \end{macrocode}

% \macro{\childdocforward}
% The command |\childdocforward| redirects
% compilation to the main file or
% (if the optional argument is given) a child file.
% Parameters are set as if the main file
% or a child file starting with |\childdocof| was compiled.
% Then compilation is handed over to the main file:
%    \begin{macrocode}
\newcommand{\childdocforward}[2][]
{
  \begingroup
    \if?#1?
      \def\childdoctmp
      {
        \def\childdocname{#2}
        \def\childdocjob{#2}
        \def\jobname{#2}
        \input{#2}
        \endinput
      }
    \else
      \def\childdoctmp
      {
        \childdocdisable
        \def\childdocname{#2}
        \childdoctrue
        \includeonly{#2}
        \def\childdocjob{#1}
        \def\jobname{#1}
        \input{#1}
        \endinput
      }
    \fi
    \expandafter
  \endgroup
  \childdoctmp
}
%    \end{macrocode}

% \macro{\childdocforwardprefix}
% The command |\childdocforwardprefix| redirects
% compilation to the main or a child file by means of a pattern.
% The prefix |#1| in the current filename is replaced by |#2|
% and the suffix of the current filename is kept
% (it is assumed that the filename does not contain the substring `|~~~|'
% which is used as a delimiter).
% Compilation is handed over to the new file by |\childdocforward|:
%    \begin{macrocode}
\newcommand{\childdocforwardprefix}[3][]
{
  \begingroup
    \def\childdocextract #2##1~~~{\def\childdoctmp{\childdocforward[#1]{#3##1}}}
    \expandafter\childdocextract\childdocname~~~
    \expandafter
  \endgroup
  \childdoctmp
}
%    \end{macrocode}

% \macro{\childdoc}
% The deprecated macro |\childdoc| is a legacy version of |\childdocmain|:
%    \begin{macrocode}
\newcommand{\childdoc}{\childdocmain}
%    \end{macrocode}

% \macro{\childdocredirect}
% The deprecated macro |\childdocredirect| is a legacy version
% of |\childdocforward| and |\childdocforwardprefix|:
%    \begin{macrocode}
\newcommand{\childdocredirect}[2][]
{
  \begingroup
    \if?#1?
      \def\childdoctmp{\childdocforward{#2}}
    \else
      \def\childdoctmp{\childdocforwardprefix{#1}{#2}}
    \fi
    \expandafter
  \endgroup
  \childdoctmp
}
%    \end{macrocode}

%\iffalse
%</package>
%\fi
%
\endinput
|\\
|\childdocforward{|\textit{main}|}|\\
\end{tabular}
\end{center}
%
or alternatively with:
%
\begin{center}
\begin{tabular}{l}
|% \iffalse
%
% childdoc.dtx Copyright (C) 2017-2018 Niklas Beisert
%
% This work may be distributed and/or modified under the
% conditions of the LaTeX Project Public License, either version 1.3
% of this license or (at your option) any later version.
% The latest version of this license is in
%   http://www.latex-project.org/lppl.txt
% and version 1.3 or later is part of all distributions of LaTeX
% version 2005/12/01 or later.
%
% This work has the LPPL maintenance status `maintained'.
%
% The Current Maintainer of this work is Niklas Beisert.
%
% This work consists of the files childdoc.dtx and childdoc.ins
% and the derived files childdoc.def and cdocsamp.tex with
% cdocsch1.tex, cdocsch2.tex, cdocsdrf.tex, cdocsfn1.tex, cdocsfn2.tex.
%
%<package>\ifdefined\childdocmain\endinput\fi
%<package>\ProvidesFile{childdoc.def}[2018/12/30 v2.0 child document driver]
%<samplemain>\ProvidesFile{cdocsamp.tex}[2018/12/30 v2.0 sample for childdoc]
%<*driver>
%\ProvidesFile{childdoc.drv}[2018/12/30 v2.0 childdoc reference manual file]
\PassOptionsToClass{10pt,a4paper}{article}
\documentclass{ltxdoc}

\usepackage[margin=35mm]{geometry}
\usepackage{hyperref}
\usepackage{hyperxmp}
\usepackage[usenames]{color}

\hypersetup{colorlinks=true}
\hypersetup{pdfstartview=FitH}
\hypersetup{pdfpagemode=UseNone}
\hypersetup{pdfsource={}}
\hypersetup{pdflang={en-UK}}
\hypersetup{pdfcopyright={Copyright 2017-2018 Niklas Beisert.
  This work may be distributed and/or modified under the
  conditions of the LaTeX Project Public License, either version 1.3
  of this license or (at your option) any later version.}}
\hypersetup{pdflicenseurl={http://www.latex-project.org/lppl.txt}}
\hypersetup{pdfcontactaddress={ETH Zurich, ITP, HIT K,
  Wolfgang-Pauli-Strasse 27}}
\hypersetup{pdfcontactpostcode={8093}}
\hypersetup{pdfcontactcity={Zurich}}
\hypersetup{pdfcontactcountry={Switzerland}}
\hypersetup{pdfcontactemail={nbeisert@itp.phys.ethz.ch}}
\hypersetup{pdfcontacturl={http://people.phys.ethz.ch/\xmptilde nbeisert/}}

\newcommand{\secref}[1]{\hyperref[#1]{section \ref*{#1}}}

\parskip1ex
\parindent0pt
\let\olditemize\itemize
\def\itemize{\olditemize\parskip0pt}

\begin{document}

\title{The \textsf{childdoc} Package}
\hypersetup{pdftitle={The childdoc Package}}
\author{Niklas Beisert\\[2ex]
  Institut f\"ur Theoretische Physik\\
  Eidgen\"ossische Technische Hochschule Z\"urich\\
  Wolfgang-Pauli-Strasse 27, 8093 Z\"urich, Switzerland\\[1ex]
  \href{mailto:nbeisert@itp.phys.ethz.ch}
  {\texttt{nbeisert@itp.phys.ethz.ch}}}
\hypersetup{pdfauthor={Niklas Beisert}}
\hypersetup{pdfsubject={Manual for the LaTeX2e Package childdoc}}
\date{30 December 2018, \textsf{v2.0}}
\maketitle

\begin{abstract}\noindent
\textsf{childdoc} is a \LaTeXe{} package
that enables the direct compilation
of document sections included by |\include|
to individual files.
\end{abstract}

\begingroup
\parskip0ex
\tableofcontents
\endgroup

%%%%%%%%%%%%%%%%%%%%%%%%%%%%%%%%%%%%%%%%%%%%%%%%%%%%%%%%%%%%%%%%%%%%%%%%%%%%%%%%
%%%%%%%%%%%%%%%%%%%%%%%%%%%%%%%%%%%%%%%%%%%%%%%%%%%%%%%%%%%%%%%%%%%%%%%%%%%%%%%%
\section{Introduction}

\LaTeX{} provides a mechanism to structure a large document (such as a book)
into a main file and several child files (containing the chapters)
using the |\include| command.
This mechanism is beneficial for documents
which span hundreds of pages in order to
make the source file(s) more manageable.
Moreover, compilation can be restricted to
selected child files by means of the |\includeonly| command.
The latter feature can be used to reduce the compilation time while editing
(this was significantly more useful in the earlier days of \LaTeX{})
or to generate a smaller document which is easier to navigate.
Another application of |\includeonly| is to generate
documents consisting of selected parts of the complete document.

However, there are a few drawbacks of the plain |\include| mechanism:
\begin{itemize}
\item
The child files cannot be compiled on their own,
they can only be compiled via the main file.
A naive editing environment
(such as a text editor with an option
to have the current file processed by \LaTeX)
may require one to switch to the main file before compiling;
attempting to compile the child file produces errors.
\item
The main file must be modified (each time)
to adjust the |\includeonly| command
to the present needs. This easily leaves the main file in a messy state.
\item
The generated document will always carry the filename
of the main document. This is inconvenient if
several child files are to be compiled and
to be kept for distribution.
\end{itemize}

The present package provides a simple interface
to make child files individually compilable by \LaTeX{}.
Compiling a child file then has the same effect as compiling
the main file with an |\includeonly| command
to select the appropriate child.
Moreover the generated document will carry the name of the child
rather than the main file.
This resolves all three above issues.

This feature is meant to make the editing of books,
thesis documents and lecture notes somewhat more convenient.
However, the package can also be used efficiently for
composing a series of documents (such as exercise sheets)
which are typically distributed individually.
It then assists the author in generating the individual documents
(potentially in different versions)
as well as a document containing the collected series.
Another application is in developing style files
or other kinds of included material
where compilation of the style file could redirect
to a sample or test file.

%%%%%%%%%%%%%%%%%%%%%%%%%%%%%%%%%%%%%%%%%%%%%%%%%%%%%%%%%%%%%%%%%%%%%%%%%%%%%%%%
%%%%%%%%%%%%%%%%%%%%%%%%%%%%%%%%%%%%%%%%%%%%%%%%%%%%%%%%%%%%%%%%%%%%%%%%%%%%%%%%
\section{Usage}

First of all, the package \textsf{childdoc} is \emph{not} a standard
\LaTeXe{} |.sty| style file! Therefore it needs to be invoked in
a non-standard way.

%%%%%%%%%%%%%%%%%%%%%%%%%%%%%%%%%%%%%%%%%%%%%%%%%%%%%%%%%%%%%%%%%%%%%%%%%%%%%%%%
\subsection{Included Files}
\label{sec:include}

%%%%%%%%%%%%%%%%%%%%%%%%%%%%%%%%%%%%%%%%
\DescribeMacro{\childdocmain}
To use the package, add the commands
\begin{center}
\begin{tabular}{l}
|\input{childdoc.def}|\\
|\childdocmain{}|\\
\end{tabular}
\end{center}
at the very top of the main \LaTeX{} file,
in particular \emph{before} the |\documentclass| statement!
The argument of |\childdocmain| should be left empty
(but it must be present).

%%%%%%%%%%%%%%%%%%%%%%%%%%%%%%%%%%%%%%%%
\DescribeMacro{\childdocof}
Furthermore, add the commands
\begin{center}
\begin{tabular}{l}
|\input{childdoc.def}|\\
|\childdocof{|\textit{main}|}|\\
\end{tabular}
\end{center}
at the top of every child file \textit{child}
which is included by |\include{|\textit{child}|}|
from within the main file
(or at least for those files to be compiled individually).
The argument \textit{main} must be the filename of the main file.

There are a couple of
considerations in setting up the main and child documents:

%%%%%%%%%%%%%%%%%%%%%%%%%%%%%%%%%%%%%%%%
\paragraph{Restrictions.}

Please note the following restrictions:
\begin{itemize}
\item
|\childdocmain| must be called with one argument \textit{main}
to ensure compatibility with earlier version of the package.
It must either be empty (|\childdocmain{}|)
or precisely match the filename of the main file in which it is specified.
See \secref{sec:detection} for further information.
\item
The filename \textit{main} must be specified without the |.tex| extension.
\item
The filename \textit{main} is case sensitive
(even in case-insensitive file systems)
due to internal string comparison.
\item
The argument \textit{main} should be fully expanded, it cannot be a macro.
\item
Subdirectories and special characters should be avoided in filenames.
\item
The command |\childdocmain{|\textit{main}|}| must be followed by a whitespace.
It should not be followed immediately by another command
or by a comment mark `|%|'.
This is because the \TeX{} parser reads the token immediately following
the argument of |\childdocmain| and puts it
at the beginning of every child section;
however, a white\-space is ignored.
\end{itemize}

%%%%%%%%%%%%%%%%%%%%%%%%%%%%%%%%%%%%%%%%
\paragraph{Content of Main File.}

It is advisable to place all content in the child files included by |\include|.
Any output contained in the main file will appear in all child documents
unless suppressed manually;
it cannot be suppressed automatically by the |\includeonly| directive
and thus should normally be avoided.
A method to include some content in the main file
by means of conditional processing is described in \secref{sec:conditional}.

%%%%%%%%%%%%%%%%%%%%%%%%%%%%%%%%%%%%%%%%
\paragraph{Page Numbering.}

When only a part of the document is compiled,
the appropriate numbering of pages
(as well as other status parameters)
is determined from the |.aux| files.
The latter contain information from previous passes.
However this information needs to propagate through
all intermediate child documents.
Therefore the page numbering in child documents may well
be inconsistent until the complete document is compiled at least once.

A useful (if unconventional) way to always ensure a consistent
page numbering is to restart the numbering in each child document
and denote the pages by `\textit{child}|.|\textit{page}'
where \textit{child} represents the chapter/section number of the child file.
This can be achieved by the command
|\numberwithin{page}{|\textit{child}|}|
of the \textsf{amsmath} package
where \textit{child} can be |chapter| or |section|
depending on the chosen structuring.
Alternatively, one can modify the macro |\thepage| appropriately
and reset the counter |page| at the start of each child file.

%%%%%%%%%%%%%%%%%%%%%%%%%%%%%%%%%%%%%%%%%%%%%%%%%%%%%%%%%%%%%%%%%%%%%%%%%%%%%%%%
\subsection{Conditional Processing}
\label{sec:conditional}

The package provides a mechanism to compile different versions
of a document. To customise the versions further some conditional processing
can come in handy to distinguish which version is being compiled.
The package provides two macros to describe the compilation context:

%%%%%%%%%%%%%%%%%%%%%%%%%%%%%%%%%%%%%%%%
\DescribeMacro{\ifchilddoc}
The conditional |\ifchilddoc| distinguishes between the compilation of
child documents and the main document:
%
\begin{center}
|\ifchilddoc |\textit{child-code}| |[|\||else |\textit{main-code}]| \||fi|
\end{center}

%%%%%%%%%%%%%%%%%%%%%%%%%%%%%%%%%%%%%%%%
\DescribeMacro{\childdocname}
\DescribeMacro{\childdocjob}
The macro |\childdocname| contains the filename (without extension)
of the main or child file being processed.
Note that |\childdocjob| will always contain the name of the main file.

%%%%%%%%%%%%%%%%%%%%%%%%%%%%%%%%%%%%%%%%
\paragraph{Title Page.}

Conditional processing can be used to include a title or banner page
in the main document when proper precautions are taken.
Importantly, the code in the main file should ensure that the page counter
(as well as other status parameters which are stored in the |.aux| files)
takes the same value after the conditional processing.
Otherwise the page numbers may take divergent values
depending on which part is compiled.

For example, a title page could be declared by:
%
\begin{center}
\begin{tabular}{l}
|\ifchilddoc\||else|\\
|\addtocounter{page}{-1}|\\
\textit{code for title page}\\
|\newpage|\\
|\||fi|
\end{tabular}
\end{center}
%
A banner page for the child documents can be generated by:
%
\begin{center}
\begin{tabular}{l}
|\ifchilddoc|\\
|\addtocounter{page}{-1}|\\
\textit{code for banner page}\\
|\newpage|\\
|\||fi|
\end{tabular}
\end{center}
%
Here one could write a message such as:
\begin{center}
|This is the part \childdocname{} of \childdocjob{}.|
\end{center}

%%%%%%%%%%%%%%%%%%%%%%%%%%%%%%%%%%%%%%%%%%%%%%%%%%%%%%%%%%%%%%%%%%%%%%%%%%%%%%%%
\subsection{Flags}
\label{sec:flags}

The package makes it easy to generate different versions
of the main or child documents.
To this end compilation flags can be defined
and assigned different default values.
They will be particularly useful in conjunction
with the forwarding mechanism described in \secref{sec:forward}.

For example, it may be useful to have a flag |\version|
which can be set to |draft| or |final|.
The document source will contain some conditional code
depending on the value of |\version|.
Suppose further, the flag should default to |final| for the main file
and to |draft| for child files
which is a natural assignment for editing the document.
This is achieved by placing the following code
in the preamble of the main document
(below the |\childdocmain| directive):
%
\begin{center}
\begin{tabular}{l}
|\ifchilddoc|\\
|\providecommand{\version}{draft}|\\
|\||else|\\
|\providecommand{\version}{final}|\\
|\||fi|
\end{tabular}
\end{center}
%
The definition by |\providecommand| makes sure
that previous definitions are not overwritten.
Further statements |\providecommand{\version}{...}|
can thus be added before the above code to override it.

For the main file, one might add a line
(between |\childdocmain| and the above block)
%
\begin{center}
|%\ifchilddoc\||else\providecommand{\version}{draft}\||fi|
\end{center}
%
which can be uncommented to produce a draft version.
Likewise one can add a line to the very top of a child file
(above the |\childdocof{|\textit{main}|}| directive)
%
\begin{center}
|%\providecommand{\version}{final}|
\end{center}
%
which can be uncommented to produce the final version of this child document.

%%%%%%%%%%%%%%%%%%%%%%%%%%%%%%%%%%%%%%%%%%%%%%%%%%%%%%%%%%%%%%%%%%%%%%%%%%%%%%%%
\subsection{Forwarding}
\label{sec:forward}

Different versions of the main or child documents
using compilation flags as described in \secref{sec:flags}
can be (permanently) stored in different files
for convenient compilation, viewing and distribution.
To this end, the package defines a command
to pass on compilation to a different file:

%%%%%%%%%%%%%%%%%%%%%%%%%%%%%%%%%%%%%%%%
\DescribeMacro{\childdocforward}
The command |\childdocforward| redirects processing to
another source file:
%
\begin{center}
\begin{tabular}{l}
|\input{childdoc.def}|\\
|\childdocforward[|\textit{main}|]{|\textit{dest}|}|\\
\end{tabular}
\end{center}
%
The argument \textit{dest} is the destination file
(without extension).
It should be the main file or one of the child files.
Note that further \textsf{childdoc} directives
such as |\childdocof| and |\childdocforward|
in the indicated file will be processed in this form.
The optional argument \textit{main}
passes on directly to the main file \textit{main}
while pretending to compile the child \textit{dest}.
This form behaves as if \textit{dest}
issues |\childdocof{|\textit{main}|}| right away,
and no further \textsf{childdoc} directives will be processed.

%%%%%%%%%%%%%%%%%%%%%%%%%%%%%%%%%%%%%%%%
\DescribeMacro{\...prefix}
In the alternative form |\childdocforwardprefix|,
%
\begin{center}
\begin{tabular}{l}
|\input{childdoc.def}|\\
|\childdocforwardprefix[|\textit{main}|]{|\textit{prefix}|}{|\textit{dest}|}|
\end{tabular}
\end{center}
%
the destination file is determined by a pattern
depending on the current file:
To make this work, the current file must be called
`{\textit{prefix}\hspace{0.2em}\textit{suffix}}'
with \textit{prefix} matching precisely the argument.
Processing is then passed on to the file
`{\textit{dest}\hspace{0.2em}\textit{suffix}}'.
Surely, the same effect is achieved by
directly specifying the
argument `{\textit{dest}\hspace{0.2em}\textit{suffix}}'
in the first form.
However, that requires to set up a different file
for each child. With the alternative form of the command
all these files can have exactly the same content
which simplifies setting them up and maintaining them.

For example, the following file |draft.tex|
with a compilation flag |\version| as described in \secref{sec:flags}
compiles the main document as a draft:
%
\begin{center}
\begin{tabular}{l}
|\def\version{draft}|\\
|\input{childdoc.def}|\\
|\childdocforward{|\textit{main}|}|
\end{tabular}
\end{center}
%
Likewise, the following files |final|\textit{nn}|.tex|
compile the final version of the child document
|child|\textit{nn}|.tex|:
%
\begin{center}
\begin{tabular}{l}
|\def\version{final}|\\
|\input{childdoc.def}|\\
|\childdocforwardprefix{final}{child}|
\end{tabular}
\end{center}
%

Note that when several versions of a main file and/or of each child file
are to be generated, it may be convenient to set up a |Makefile| or
shell script to automatise the process.

%%%%%%%%%%%%%%%%%%%%%%%%%%%%%%%%%%%%%%%%%%%%%%%%%%%%%%%%%%%%%%%%%%%%%%%%%%%%%%%%
\subsection{Command Line Processing}
\label{sec:commandline}

The effect of redirection files can also be achieved by invoking
the \LaTeX{} compiler with a more elaborate command line.
Most conveniently this should be done as part
of a shell script or a |Makefile|.

When using \textsf{childdoc} in the main file, the following
command lines effectively perform a redirection
(note that depending on the shell being used,
backslashes may have to be doubled: `|\|' $\to$ `|\\|'):
%
\begin{center}
|... -jobname "|\textit{target}|" |\\|"|[\textit{flags}]%
|\input{childdoc.def}\childdocforward[|\textit{main}|]{|\textit{dest}|}"|
\end{center}
%
Here \textit{target} is the name of the output file,
\textit{main} is the name of the main file
and \textit{dest} is the name of the main or child file to be processed
(all filenames without extensions).
The optional argument \textit{main} can be omitted
if \textit{main} matches \textit{dest}.
Optionally, compilation \textit{flags} can be defined via |\def| commands.
This command line makes the \TeX{} engine believe
it is compiling the file \textit{target}
whose content is specified as the latter parameter.
The provided code then forwards the processing to
\textit{main} or \textit{dest} as described in \secref{sec:forward}.

%%%%%%%%%%%%%%%%%%%%%%%%%%%%%%%%%%%%%%%%%%%%%%%%%%%%%%%%%%%%%%%%%%%%%%%%%%%%%%%%
\subsection{Include by Input}
\label{sec:input}

Including child documents by |\include| has some restrictions by design.
Most notably, the content of a child document always occupies
its own set of pages; pages cannot be shared between child documents.
Usually, this behaviour makes perfect sense
because each child document contain an essential part of the document.
However, in some situations it may be desirable to compose
a document from a collection of parts
without having mandatory page breaks between then.
For this case, the package
provides a mechanism to include parts
by |\input| which can also be processed individually.
However, by construction this mechanism
requires manual handling of the content to be output.

%%%%%%%%%%%%%%%%%%%%%%%%%%%%%%%%%%%%%%%%
\DescribeMacro{\ifchilddocmanual}
The main file should be prepared as usual, see \secref{sec:include}.
However, the document body must make a distinction
between processing of an individual part and of the main document, e.g.:
%
\begin{center}
\begin{tabular}{l}
|\ifchilddocmanual|\\
|\input{\childdocname}|\\
|\||else|\\
\textit{document body with }|\input{|\textit{part}|}|\\
|\||fi|
\end{tabular}
\end{center}
%
The conditional |\ifchilddocmanual| is true whenever
a part to be included by |\input| is being compiled,
and the name of the part is stored in |\childdocname|.

%%%%%%%%%%%%%%%%%%%%%%%%%%%%%%%%%%%%%%%%
\DescribeMacro{\childdocby}
Each part to be included by |\input| should start with:
%
\begin{center}
\begin{tabular}{l}
|\input{childdoc.def}|\\
|\childdocby{|\textit{main}|}|\\
\end{tabular}
\end{center}
%
The directive |\childdocby| is similar to |\childdocof|
described in \secref{sec:include},
but the subsequent selection of content must be done manually.
To that end, both |\ifchilddoc| and |\ifchilddocmanual|
will be true upon processing of a part,
and the name of the part is stored in |\childdocname|.
Note that |\jobname| will be set to the filename of the current part
so that each part receives an individual |.aux| file
that does not interfere with the |.aux| file(s) of the main document.
This behaviour can be altered by the alternative form
|\childdocby[*]{|\textit{main}|}| (with a non-empty optional argument)
which uses the |.aux| file of the main document
by setting |\jobname| to \textit{main}.

%%%%%%%%%%%%%%%%%%%%%%%%%%%%%%%%%%%%%%%%%%%%%%%%%%%%%%%%%%%%%%%%%%%%%%%%%%%%%%%%
\subsection{Driver Development}
\label{sec:driver}

The \textsf{childdoc} mechanism can also be use for the development
of definition files such as \LaTeX{} styles or classes.
This case differs from the above setup with multiple parts
included by |\include| in that no |\includeonly| should be invoked.
This can be achieved by starting the include file
(before |\ProvidesPackage|) with:
%
\begin{center}
\begin{tabular}{l}
|\input{childdoc.def}|\\
|\childdocforward{|\textit{main}|}|\\
\end{tabular}
\end{center}
%
or alternatively with:
%
\begin{center}
\begin{tabular}{l}
|\input{childdoc.def}|\\
|\childdocby{|\textit{main}|}|\\
\end{tabular}
\end{center}
%
Both forms have slightly different effects as described above.
The main file is prepared as usual, see \secref{sec:include}.

%%%%%%%%%%%%%%%%%%%%%%%%%%%%%%%%%%%%%%%%%%%%%%%%%%%%%%%%%%%%%%%%%%%%%%%%%%%%%%%%
\subsection{Legacy Detection}
\label{sec:detection}

The directive |\childdocmain| in the main file can detect
whether the complete document or merely a child is to be compiled
even without using the directive |\childdocof|.
This method is deprecated because it is less robust
and there is no compelling reason to use it;
it is merely provided for backward compatibility
and it may be removed in future versions.

If the detection mechanism is to be used,
it is mandatory to correctly specify
the filename of the main file as the argument of |\childdocmain|:
%
\begin{center}
\begin{tabular}{l}
|\input{childdoc.def}|\\
|\childdocmain{|\textit{main}|}|\\
\end{tabular}
\end{center}
%
If |\jobname| does not match the argument \textit{main} of |\childdocmain|,
it is assumed that |\jobname| points to the child file to be compiled.
When using |\childdocmain| with the main file specified as argument,
it suffices to start a child file
with just |\input{|\textit{main}|}|
without loading of the package and using |\childdocof|.
If instead all processing is done
with the appropriate \textsf{childdoc} directives,
the argument of \textit{main} of |\childdocmain| can be empty.

An alternative version of the command line processing described
in \secref{sec:commandline} using the detection mechanism reads:
%
\begin{center}
|... -jobname "|\textit{target}|" "|[\textit{flags}]%
[|\def\jobname{|\textit{dest}|}|]|\input{|\textit{main}|}"|
\end{center}

%%%%%%%%%%%%%%%%%%%%%%%%%%%%%%%%%%%%%%%%%%%%%%%%%%%%%%%%%%%%%%%%%%%%%%%%%%%%%%%%
\subsection{Manual Code}
\label{sec:manual}

In case one cannot be certain whether the definitions file |childdoc.def|
is installed on the target \TeX{} distribution
and one prefers not to ship it,
it is conceivable to paste a few relevant commands into the sources.

To that end, drop all statements |\input{childdoc.def}|
and perform the replacements as outlined below.
Instead of |\childdocmain{|\textit{main}|}| add the following code
to the top of the main file:
%
\begin{center}
\begin{tabular}{l}
|\||ifdefined\childdocname\endinput\||fi\newif\ifchilddoc|\\
|\edef\childdocname{\scantokens\expandafter{\jobname\noexpand}}|\\
|\def\childdocmain{|\textit{main}|}\||ifx\childdocmain\childdocname\||else|\\
|\childdoctrue\includeonly{\childdocname}\let\jobname\childdocmain\||fi|\\
\end{tabular}
\end{center}
%
Instead of |\childdocof{|\textit{main}|}| just include the main file
at the top of each child file:
%
\begin{center}
|\input{|\textit{main}|}|
\end{center}
%
A simple redirection |\childdocforward{|\textit{dest}|}| is achieved by:
%
\begin{center}
|\def\jobname{|\textit{dest}|}\input{\jobname}|
\end{center}
%
The redirection with prefix
|\childdocforwardprefix[|\textit{prefix}|]{|\textit{dest}|}|
is accomplished by:
%
\begin{center}
\begin{tabular}{l}
|{\edef\jobname{\scantokens\expandafter{\jobname\noexpand}}|\\
|\def\redirectjob |\textit{prefix}|#1~~~{\gdef\jobname{|\textit{dest}|#1}}|\\
|\expandafter\redirectjob\jobname~~~}\input{\jobname}|
\end{tabular}
\end{center}

In an alternative approach,
child documents can be compiled by a specific command line
without additional code or specific definitions:
%
\begin{center}
|... -jobname "|\textit{target}|" "|[\textit{flags}]%
|\includeonly{|\textit{dest}|}\input{|\textit{main}|}"|
\end{center}
%

%%%%%%%%%%%%%%%%%%%%%%%%%%%%%%%%%%%%%%%%%%%%%%%%%%%%%%%%%%%%%%%%%%%%%%%%%%%%%%%%
%%%%%%%%%%%%%%%%%%%%%%%%%%%%%%%%%%%%%%%%%%%%%%%%%%%%%%%%%%%%%%%%%%%%%%%%%%%%%%%%
\section{Information}

%%%%%%%%%%%%%%%%%%%%%%%%%%%%%%%%%%%%%%%%%%%%%%%%%%%%%%%%%%%%%%%%%%%%%%%%%%%%%%%%
\subsection{Copyright}

Copyright \copyright{} 2017--2018 Niklas Beisert

This work may be distributed and/or modified under the
conditions of the \LaTeX{} Project Public License, either version 1.3
of this license or (at your option) any later version.
The latest version of this license is in
  \url{http://www.latex-project.org/lppl.txt}
and version 1.3 or later is part of all distributions of \LaTeX{}
version 2005/12/01 or later.

This work has the LPPL maintenance status `maintained'.

The Current Maintainer of this work is Niklas Beisert.

This work consists of the files |README.txt|, |childdoc.ins| and |childdoc.dtx|
as well as the derived files |childdoc.def|, |cdocsamp.tex|
with |cdocsch1.tex|, |cdocsch2.tex|, |cdocspt3.tex|, |cdocspt4.tex|,
|cdocsdrf.tex|, |cdocsfn1.tex|, |cdocsfn2.tex|
as well as |childdoc.pdf|.

%%%%%%%%%%%%%%%%%%%%%%%%%%%%%%%%%%%%%%%%%%%%%%%%%%%%%%%%%%%%%%%%%%%%%%%%%%%%%%%%
\subsection{Files and Installation}

The package consists of the files:
%
\begin{center}
\begin{tabular}{ll}
    |README.txt|   & readme file \\
    |childdoc.ins| & installation file \\
    |childdoc.dtx| & source file \\
    |childdoc.def| & definition file \\
    |cdocsamp.tex| & sample main file \\
    |cdocsch1.tex| & sample include file \\
    |cdocsch2.tex| & sample include file \\
    |cdocspt3.tex| & sample part file \\
    |cdocspt4.tex| & sample part file \\
    |cdocsdrf.tex| & sample redirection file \\
    |cdocsfn1.tex| & sample redirection file \\
    |cdocsfn2.tex| & sample redirection file \\
    |childdoc.pdf| & manual
\end{tabular}
\end{center}
%
The distribution consists of the files
|README.txt|, |childdoc.ins| and |childdoc.dtx|.
%
\begin{itemize}
\item
Run (pdf)\LaTeX{} on |childdoc.dtx|
to compile the manual |childdoc.pdf| (this file).
\item
Run \LaTeX{} on |childdoc.ins| to create the definitions file |childdoc.def|
and the sample |cdocsamp.tex| with include files
|cdocsch1.tex|, |cdocsch2.tex|, |cdocspt3.tex|, |cdocspt4.tex|,
|cdocsdrf.tex|, |cdocsfn1.tex|, |cdocsfn2.tex|.
Then copy the file |childdoc.def| to an appropriate directory of your \LaTeX{}
distribution, e.g.\ \textit{texmf-root}|/tex/latex/childdoc|.
\end{itemize}

%%%%%%%%%%%%%%%%%%%%%%%%%%%%%%%%%%%%%%%%%%%%%%%%%%%%%%%%%%%%%%%%%%%%%%%%%%%%%%%%
\subsection{Related CTAN Packages}

There are several other packages which offer a similar functionality:
%
\begin{itemize}
\item
The packages
\href{http://ctan.org/pkg/docmute}{\textsf{docmute}},
\href{http://ctan.org/pkg/includex}{\textsf{includex}} and
\href{http://ctan.org/pkg/standalone}{\textsf{standalone}}
provide commands to include only the document body of
a child file thus allowing both files to be compiled individually.
\item
The packages \href{http://ctan.org/pkg/subdocs}{\textsf{subdocs}}
and \href{http://ctan.org/pkg/subfiles}{\textsf{subfiles}}
provide structures in which the main and child documents can be
encapsulated and allowing them to be compiled individually.
The inclusion mechanism is different from the conventional |\include|.
\item
The package \href{http://ctan.org/pkg/combine}{\textsf{combine}}
is an elaborate solution to combine several documents into one.
\end{itemize}
%
See also the CTAN topic \href{http://ctan.org/topic/subdocs}{\textsf{subdocs}}
for further related packages.
The present package differs from the above solutions in that
a document structure constructed with the conventional |\include| mechanism
just needs two extra commands at the top of every file
such that all constituent files can be compiled individually.

%%%%%%%%%%%%%%%%%%%%%%%%%%%%%%%%%%%%%%%%%%%%%%%%%%%%%%%%%%%%%%%%%%%%%%%%%%%%%%%%
%\subsection{Feature Suggestions}
%
%The following is a list of features which may be useful for future
%versions of this package:
%%
%\begin{itemize}
%\item
%\ldots
%\end{itemize}

%%%%%%%%%%%%%%%%%%%%%%%%%%%%%%%%%%%%%%%%%%%%%%%%%%%%%%%%%%%%%%%%%%%%%%%%%%%%%%%%
\subsection{Revision History}

%%%%%%%%%%%%%%%%%%%%%%%%%%%%%%%%%%%%%%%%
\paragraph{v2.0:} 2018/12/30

\begin{itemize}
\item
immediate forward processing
\item
added |\childdocby| mechanism
\item
manual restructured
\end{itemize}

%%%%%%%%%%%%%%%%%%%%%%%%%%%%%%%%%%%%%%%%
\paragraph{v1.6:} 2018/01/17

\begin{itemize}
\item
application for development of include files
\item
corrections to manual
\end{itemize}

%%%%%%%%%%%%%%%%%%%%%%%%%%%%%%%%%%%%%%%%
\paragraph{v1.5:} 2017/05/21

\begin{itemize}
\item
more complete structuring introduced
\item
|\childdocof| introduced
\item
|\childdoc| renamed to |\childdocmain|
\item
|\childredirect| renamed to |\childdocforward| and |\childdocforwardprefix|
and functionality expanded
\end{itemize}

%%%%%%%%%%%%%%%%%%%%%%%%%%%%%%%%%%%%%%%%
\paragraph{v1.0:} 2017/04/27

\begin{itemize}
\item
manual and install package
\item
first version published on CTAN
\end{itemize}

%%%%%%%%%%%%%%%%%%%%%%%%%%%%%%%%%%%%%%%%
\paragraph{v0.6:} 2017/04/26

\begin{itemize}
\item
redirection mechanism added
\end{itemize}

%%%%%%%%%%%%%%%%%%%%%%%%%%%%%%%%%%%%%%%%
\paragraph{v0.5:} 2017/04/26

\begin{itemize}
\item
functionality in definition file
\end{itemize}


%%%%%%%%%%%%%%%%%%%%%%%%%%%%%%%%%%%%%%%%%%%%%%%%%%%%%%%%%%%%%%%%%%%%%%%%%%%%%%%%
%%%%%%%%%%%%%%%%%%%%%%%%%%%%%%%%%%%%%%%%%%%%%%%%%%%%%%%%%%%%%%%%%%%%%%%%%%%%%%%%
%%%%%%%%%%%%%%%%%%%%%%%%%%%%%%%%%%%%%%%%%%%%%%%%%%%%%%%%%%%%%%%%%%%%%%%%%%%%%%%%
\appendix

\settowidth\MacroIndent{\rmfamily\scriptsize 000\ }

 \DocInput{childdoc.dtx}

\end{document}
%</driver>
% \fi
%
% %%%%%%%%%%%%%%%%%%%%%%%%%%%%%%%%%%%%%%%%%%%%%%%%%%%%%%%%%%%%%%%%%%%%%%%%%%%%%%
% %%%%%%%%%%%%%%%%%%%%%%%%%%%%%%%%%%%%%%%%%%%%%%%%%%%%%%%%%%%%%%%%%%%%%%%%%%%%%%
% \section{Sample}
%\iffalse
%<*samplemain>
%\fi
%
% The following presents a sample document
% with two chapters, two parts, a title page,
% a compile flag as well as three forwarding files to set the flag.
% It consists of eight |.tex| files:
% \begin{center}
% \begin{tabular}{ll}
% |cdocsamp.tex|&main file\\
% |cdocsch1.tex|&include file for chapter 1\\
% |cdocsch2.tex|&include file for chapter 2\\
% |cdocspt3.tex|&include file for part 3\\
% |cdocspt4.tex|&include file for part 4\\
% |cdocsdrf.tex|&forwarding file for main file in draft mode\\
% |cdocsfi1.tex|&forwarding file for final version of chapter 1\\
% |cdocsfi2.tex|&forwarding file for final version of chapter 2\\
% \end{tabular}
% \end{center}
% Each of the eight files can be compiled directly by the \LaTeX{} compiler.
%
% %%%%%%%%%%%%%%%%%%%%%%%%%%%%%%%%%%%%%%
% \paragraph{Main File.}
%
% The main file is called |cdocsamp.tex|.
%
% Load the \textsf{childdoc} definitions and
% declare the filename for the main document:
%    \begin{macrocode}
\input{childdoc.def}
\childdocmain{}
%    \end{macrocode}

% Optional override for |\version| flag:
%    \begin{macrocode}
%%\ifchilddoc\else\providecommand{\version}{draft}\fi
%    \end{macrocode}

% Define the default values for the |\version| flag
% (|final| for the main file and |draft| for childs):
%    \begin{macrocode}
\ifchilddoc
\providecommand{\version}{draft}
\else
\providecommand{\version}{final}
\fi
%    \end{macrocode}

% Load the standard document class:
%    \begin{macrocode}
\documentclass[12pt]{article}
%    \end{macrocode}

% Start the document body:
%    \begin{macrocode}
\begin{document}
%    \end{macrocode}

% Declare a title page.
% Print title, part of document being processed and version flag:
%    \begin{macrocode}
\addtocounter{page}{-1}
\begin{center}
{\LARGE\bfseries{}childdoc example\par}
\vspace{1cm}
\ifchilddoc
\ifchilddocmanual part\else chapter\fi:
`\childdocname' of `\childdocjob'\par
\else
main document: `\childdocjob'\par
\fi
version: \version\par
\end{center}
\newpage
%    \end{macrocode}

% Manually include selected file,
% otherwise process as usual:
%    \begin{macrocode}
\ifchilddocmanual
\section*{part `\childdocname'}
\input{\childdocname}
\else
%    \end{macrocode}

% Include the two chapters:
%    \begin{macrocode}
\include{cdocsch1}
\include{cdocsch2}
%    \end{macrocode}

% Include the two parts unless only chapters should be displayed:
%    \begin{macrocode}
\ifchilddoc\else
\section{part three}
\input{cdocspt3}
\section{part four}
\input{cdocspt4}
\fi
%    \end{macrocode}

% Process as usual until here:
%    \begin{macrocode}
\fi
%    \end{macrocode}

% End of document body:
%    \begin{macrocode}
\end{document}
%    \end{macrocode}
%\iffalse
%</samplemain>
%\fi
%
% %%%%%%%%%%%%%%%%%%%%%%%%%%%%%%%%%%%%%%
% \paragraph{Chapter Include Files.}
%
% The include files are called |cdocsch1.tex| and |cdocsch2.tex|.
%
%\iffalse
%<*samplechap1|samplechap2>
%\fi

% Optional override for |\version| flag:
%    \begin{macrocode}
%%\providecommand{\version}{final}
%    \end{macrocode}

% Include the main document:
%    \begin{macrocode}
\input{childdoc.def}
\childdocof{cdocsamp}
%    \end{macrocode}

%\iffalse
%</samplechap1|samplechap2>
%\fi
%
%\iffalse
%<*samplechap1>
%\fi
% Some text for chapter 1:
%    \begin{macrocode}
\section{one}
some text in chapter one
%    \end{macrocode}

%\iffalse
%</samplechap1>
%\fi
% Some text for chapter 2:
%\iffalse
%<*samplechap2>
%\fi
%    \begin{macrocode}
\section{two}
more text in chapter two
%    \end{macrocode}

%\iffalse
%</samplechap2>
%\fi
%
% %%%%%%%%%%%%%%%%%%%%%%%%%%%%%%%%%%%%%%
% \paragraph{Part Include Files.}
%
% The include files are called |cdocspt3.tex| and |cdocspt4.tex|.
%
%\iffalse
%<*samplepart3|samplepart4>
%\fi

% Optional override for |\version| flag:
%    \begin{macrocode}
%%\providecommand{\version}{final}
%    \end{macrocode}

% Include the main document:
%    \begin{macrocode}
\input{childdoc.def}
\childdocby{cdocsamp}
%    \end{macrocode}

%\iffalse
%</samplepart3|samplepart4>
%\fi
%
%\iffalse
%<*samplepart3>
%\fi
% Some text for part 3:
%    \begin{macrocode}
some text in part three
%    \end{macrocode}

%\iffalse
%</samplepart3>
%\fi
% Some text for part 4:
%\iffalse
%<*samplepart4>
%\fi
%    \begin{macrocode}
more text in part four
%    \end{macrocode}

%\iffalse
%</samplepart4>
%\fi
%
% %%%%%%%%%%%%%%%%%%%%%%%%%%%%%%%%%%%%%%
% \paragraph{Forwarding for a Complete Draft.}
%
% The following forwarding file |cdocsdrf.tex|
% compiles the main document in draft mode:
%\iffalse
%<*sampledraft>
%\fi
%    \begin{macrocode}
\def\version{draft}
\input{childdoc.def}
\childdocforward{cdocsamp}
%    \end{macrocode}

%\iffalse
%</sampledraft>
%\fi
%
% %%%%%%%%%%%%%%%%%%%%%%%%%%%%%%%%%%%%%%
% \paragraph{Forwarding for Final Version of the Chapters.}
%
% The following forwarding files |cdocsfn1.tex| and |cdocsfn2.tex|
% (with identical content)
% compile the final versions of the child documents
% |cdocsch1.tex| and |cdocsch2.tex|, respectively:
%\iffalse
%<*samplefinal>
%\fi
%    \begin{macrocode}
\def\version{final}
\input{childdoc.def}
\childdocforwardprefix[cdocsamp]{cdocsfn}{cdocsch}
%    \end{macrocode}

%\iffalse
%</samplefinal>
%\fi
%
% %%%%%%%%%%%%%%%%%%%%%%%%%%%%%%%%%%%%%%
% \paragraph{Command Line Processing.}
%
% The following three command lines generate the output files
% |cdocscld|, |cdocscl1| and |cdocscl2|
% which should be identical to
% |cdocsdrf|, |cdocsch1| and |cdocsfn2|, respectively:
% \begin{center}
% \begin{tabular}{l}
% |latex -jobname cdocscld \|\\
% |  "\def\version{draft}\input{childdoc.def}\childdocforward{cdocsamp}"|\\
% |latex -jobname cdocscl1 \|\\
% |  "\input{childdoc.def}\childdocforward[cdocsamp]{cdocsch1}"|\\
% |latex -jobname cdocscl2 \|\\
% |  "\def\version{final}\input{childdoc.def}\childdocforward{cdocsch2}"|
% \end{tabular}
% \end{center}
% Note that the trailing backslash on each first line
% merely continues the input to the second line
% (for convenient cut ant paste).
% Furthermore, the command |latex| can be replaced by any
% of its alternative versions such as |pdflatex|.
%
% %%%%%%%%%%%%%%%%%%%%%%%%%%%%%%%%%%%%%%%%%%%%%%%%%%%%%%%%%%%%%%%%%%%%%%%%%%%%%%
% %%%%%%%%%%%%%%%%%%%%%%%%%%%%%%%%%%%%%%%%%%%%%%%%%%%%%%%%%%%%%%%%%%%%%%%%%%%%%%
% \section{Implementation}
%\iffalse
%<*package>
%\fi
%
% This section describes the definitions file |childdoc.def|.

% The definitions cannot be loaded using |\usepackage| or |\RequirePackage|
% which has a mechanism to prevent loading a style file more than once.
% When loading the definitions by means of |\input|
% multiple instances have to be prevented manually:
%\iffalse
%This code needs to be before the `\ProvidesFile' directive
%which is defined at the beginning of this file.
%Therefore it is also placed there and commented out here.
%</package>
%<*discard>
%\fi
%    \begin{macrocode}
\ifdefined\childdocmain\endinput\fi
%    \end{macrocode}
%\iffalse
%</discard>
%<*package>
%\fi
%
% \macro{\ifchilddoc}
% \macro{\ifchilddocmanual}
% The conditional |\ifchilddoc| tells whether a
% child (true) or main (false) document is being compiled.
% The conditional |\ifchilddocmanual| tells whether
% the |\includeonly| mechanism is used (false) or
% the selection of child files must be performed manually (true).
% The definitions initialise to false:
%    \begin{macrocode}
\newif\ifchilddoc
\newif\ifchilddocmanual
%    \end{macrocode}

% \macro{\childdocname}
% \macro{\childdocjob}
% The macro |\childdocname| stores the name of the main document
% to be compiled. The macro |\childdocjob| stores the name of
% the document on which the \LaTeX{} compiler was originally invoked.
% The content of |\jobname| cannot be compared
% to filenames specified in the source due to different catcodes.
% The following code rescans |\jobname|, stores the result
% in |\childdocname| and saves a copy in |\childdocjob|:
%    \begin{macrocode}
\edef\childdocname{\scantokens\expandafter{\jobname\noexpand}}
\let\childdocjob\childdocname
%    \end{macrocode}

% \macro{\childdocdisable}
% The macro |\childdocdisable| prevents the main file
% from being processed more than once.
% At this stage, the main document command |\childdocmain|
% is assumed to be called once again where it should do nothing.
% Any subsequent call to it should prevent
% a secondary processing of the main document
% It overwrites the forwarding commands
% |\childdocof| and |\childdocforward|
% with empty macros to prevent further inclusions of the main document:
%    \begin{macrocode}
\newcommand{\childdocdisable}
{
  \renewcommand{\childdocmain}[1]{\renewcommand{\childdocmain}[1]{\endinput}}
  \renewcommand{\childdocof}[1]{}
  \renewcommand{\childdocby}[2][]{}
  \renewcommand{\childdocforward}[2][]{}
  \renewcommand{\childdocdisable}{}
}
%    \end{macrocode}

% \macro{\childdocmain}
% The macro |\childdocmain| is to be called at the top of the main file
% with nothing or the main filename (without extension) as argument.
% First, it breaks loops.
% If the argument is not empty and does not match |\childdocname|
% (which is set by the first inclusion of |childdoc.def|),
% |\ifchilddoc| is set to true, |\includeonly| is applied to the child file
% and |\jobname| is set to the main file
% (for proper handling of |.aux| files):
%    \begin{macrocode}
\newcommand{\childdocmain}[1]
{
  \childdocdisable\childdocmain{}
  \if?#1?\else
    \begingroup
      \def\childdoctmp{#1}
      \ifx\childdoctmp\childdocname
        \def\childdoctmp{}
      \else
        \def\childdoctmp
        {
          \childdoctrue
          \includeonly{\childdocname}
          \def\childdocjob{#1}
          \def\jobname{#1}
        }
      \fi
      \expandafter
    \endgroup
    \childdoctmp
  \fi
}
%    \end{macrocode}

% \macro{\childdocof}
% The command |\childdocof| redirects
% compilation to the main file |#1|.
%    \begin{macrocode}
\newcommand{\childdocof}[1]
{
  \childdocdisable
  \childdoctrue
  \includeonly{\childdocname}
  \def\jobname{#1}
  \def\childdocjob{#1}
  \input{#1}
}
%    \end{macrocode}

% \macro{\childdocby}
% The command |\childdocby| ....
%    \begin{macrocode}
\newcommand{\childdocby}[2][]
{
  \childdocdisable
  \childdoctrue
  \childdocmanualtrue
  \if?#1?\else
    \def\jobname{#2}
  \fi
  \def\childdocjob{#2}
  \input{#2}
  \endinput
}
%    \end{macrocode}

% \macro{\childdocforward}
% The command |\childdocforward| redirects
% compilation to the main file or
% (if the optional argument is given) a child file.
% Parameters are set as if the main file
% or a child file starting with |\childdocof| was compiled.
% Then compilation is handed over to the main file:
%    \begin{macrocode}
\newcommand{\childdocforward}[2][]
{
  \begingroup
    \if?#1?
      \def\childdoctmp
      {
        \def\childdocname{#2}
        \def\childdocjob{#2}
        \def\jobname{#2}
        \input{#2}
        \endinput
      }
    \else
      \def\childdoctmp
      {
        \childdocdisable
        \def\childdocname{#2}
        \childdoctrue
        \includeonly{#2}
        \def\childdocjob{#1}
        \def\jobname{#1}
        \input{#1}
        \endinput
      }
    \fi
    \expandafter
  \endgroup
  \childdoctmp
}
%    \end{macrocode}

% \macro{\childdocforwardprefix}
% The command |\childdocforwardprefix| redirects
% compilation to the main or a child file by means of a pattern.
% The prefix |#1| in the current filename is replaced by |#2|
% and the suffix of the current filename is kept
% (it is assumed that the filename does not contain the substring `|~~~|'
% which is used as a delimiter).
% Compilation is handed over to the new file by |\childdocforward|:
%    \begin{macrocode}
\newcommand{\childdocforwardprefix}[3][]
{
  \begingroup
    \def\childdocextract #2##1~~~{\def\childdoctmp{\childdocforward[#1]{#3##1}}}
    \expandafter\childdocextract\childdocname~~~
    \expandafter
  \endgroup
  \childdoctmp
}
%    \end{macrocode}

% \macro{\childdoc}
% The deprecated macro |\childdoc| is a legacy version of |\childdocmain|:
%    \begin{macrocode}
\newcommand{\childdoc}{\childdocmain}
%    \end{macrocode}

% \macro{\childdocredirect}
% The deprecated macro |\childdocredirect| is a legacy version
% of |\childdocforward| and |\childdocforwardprefix|:
%    \begin{macrocode}
\newcommand{\childdocredirect}[2][]
{
  \begingroup
    \if?#1?
      \def\childdoctmp{\childdocforward{#2}}
    \else
      \def\childdoctmp{\childdocforwardprefix{#1}{#2}}
    \fi
    \expandafter
  \endgroup
  \childdoctmp
}
%    \end{macrocode}

%\iffalse
%</package>
%\fi
%
\endinput
|\\
|\childdocby{|\textit{main}|}|\\
\end{tabular}
\end{center}
%
Both forms have slightly different effects as described above.
The main file is prepared as usual, see \secref{sec:include}.

%%%%%%%%%%%%%%%%%%%%%%%%%%%%%%%%%%%%%%%%%%%%%%%%%%%%%%%%%%%%%%%%%%%%%%%%%%%%%%%%
\subsection{Legacy Detection}
\label{sec:detection}

The directive |\childdocmain| in the main file can detect
whether the complete document or merely a child is to be compiled
even without using the directive |\childdocof|.
This method is deprecated because it is less robust
and there is no compelling reason to use it;
it is merely provided for backward compatibility
and it may be removed in future versions.

If the detection mechanism is to be used,
it is mandatory to correctly specify
the filename of the main file as the argument of |\childdocmain|:
%
\begin{center}
\begin{tabular}{l}
|% \iffalse
%
% childdoc.dtx Copyright (C) 2017-2018 Niklas Beisert
%
% This work may be distributed and/or modified under the
% conditions of the LaTeX Project Public License, either version 1.3
% of this license or (at your option) any later version.
% The latest version of this license is in
%   http://www.latex-project.org/lppl.txt
% and version 1.3 or later is part of all distributions of LaTeX
% version 2005/12/01 or later.
%
% This work has the LPPL maintenance status `maintained'.
%
% The Current Maintainer of this work is Niklas Beisert.
%
% This work consists of the files childdoc.dtx and childdoc.ins
% and the derived files childdoc.def and cdocsamp.tex with
% cdocsch1.tex, cdocsch2.tex, cdocsdrf.tex, cdocsfn1.tex, cdocsfn2.tex.
%
%<package>\ifdefined\childdocmain\endinput\fi
%<package>\ProvidesFile{childdoc.def}[2018/12/30 v2.0 child document driver]
%<samplemain>\ProvidesFile{cdocsamp.tex}[2018/12/30 v2.0 sample for childdoc]
%<*driver>
%\ProvidesFile{childdoc.drv}[2018/12/30 v2.0 childdoc reference manual file]
\PassOptionsToClass{10pt,a4paper}{article}
\documentclass{ltxdoc}

\usepackage[margin=35mm]{geometry}
\usepackage{hyperref}
\usepackage{hyperxmp}
\usepackage[usenames]{color}

\hypersetup{colorlinks=true}
\hypersetup{pdfstartview=FitH}
\hypersetup{pdfpagemode=UseNone}
\hypersetup{pdfsource={}}
\hypersetup{pdflang={en-UK}}
\hypersetup{pdfcopyright={Copyright 2017-2018 Niklas Beisert.
  This work may be distributed and/or modified under the
  conditions of the LaTeX Project Public License, either version 1.3
  of this license or (at your option) any later version.}}
\hypersetup{pdflicenseurl={http://www.latex-project.org/lppl.txt}}
\hypersetup{pdfcontactaddress={ETH Zurich, ITP, HIT K,
  Wolfgang-Pauli-Strasse 27}}
\hypersetup{pdfcontactpostcode={8093}}
\hypersetup{pdfcontactcity={Zurich}}
\hypersetup{pdfcontactcountry={Switzerland}}
\hypersetup{pdfcontactemail={nbeisert@itp.phys.ethz.ch}}
\hypersetup{pdfcontacturl={http://people.phys.ethz.ch/\xmptilde nbeisert/}}

\newcommand{\secref}[1]{\hyperref[#1]{section \ref*{#1}}}

\parskip1ex
\parindent0pt
\let\olditemize\itemize
\def\itemize{\olditemize\parskip0pt}

\begin{document}

\title{The \textsf{childdoc} Package}
\hypersetup{pdftitle={The childdoc Package}}
\author{Niklas Beisert\\[2ex]
  Institut f\"ur Theoretische Physik\\
  Eidgen\"ossische Technische Hochschule Z\"urich\\
  Wolfgang-Pauli-Strasse 27, 8093 Z\"urich, Switzerland\\[1ex]
  \href{mailto:nbeisert@itp.phys.ethz.ch}
  {\texttt{nbeisert@itp.phys.ethz.ch}}}
\hypersetup{pdfauthor={Niklas Beisert}}
\hypersetup{pdfsubject={Manual for the LaTeX2e Package childdoc}}
\date{30 December 2018, \textsf{v2.0}}
\maketitle

\begin{abstract}\noindent
\textsf{childdoc} is a \LaTeXe{} package
that enables the direct compilation
of document sections included by |\include|
to individual files.
\end{abstract}

\begingroup
\parskip0ex
\tableofcontents
\endgroup

%%%%%%%%%%%%%%%%%%%%%%%%%%%%%%%%%%%%%%%%%%%%%%%%%%%%%%%%%%%%%%%%%%%%%%%%%%%%%%%%
%%%%%%%%%%%%%%%%%%%%%%%%%%%%%%%%%%%%%%%%%%%%%%%%%%%%%%%%%%%%%%%%%%%%%%%%%%%%%%%%
\section{Introduction}

\LaTeX{} provides a mechanism to structure a large document (such as a book)
into a main file and several child files (containing the chapters)
using the |\include| command.
This mechanism is beneficial for documents
which span hundreds of pages in order to
make the source file(s) more manageable.
Moreover, compilation can be restricted to
selected child files by means of the |\includeonly| command.
The latter feature can be used to reduce the compilation time while editing
(this was significantly more useful in the earlier days of \LaTeX{})
or to generate a smaller document which is easier to navigate.
Another application of |\includeonly| is to generate
documents consisting of selected parts of the complete document.

However, there are a few drawbacks of the plain |\include| mechanism:
\begin{itemize}
\item
The child files cannot be compiled on their own,
they can only be compiled via the main file.
A naive editing environment
(such as a text editor with an option
to have the current file processed by \LaTeX)
may require one to switch to the main file before compiling;
attempting to compile the child file produces errors.
\item
The main file must be modified (each time)
to adjust the |\includeonly| command
to the present needs. This easily leaves the main file in a messy state.
\item
The generated document will always carry the filename
of the main document. This is inconvenient if
several child files are to be compiled and
to be kept for distribution.
\end{itemize}

The present package provides a simple interface
to make child files individually compilable by \LaTeX{}.
Compiling a child file then has the same effect as compiling
the main file with an |\includeonly| command
to select the appropriate child.
Moreover the generated document will carry the name of the child
rather than the main file.
This resolves all three above issues.

This feature is meant to make the editing of books,
thesis documents and lecture notes somewhat more convenient.
However, the package can also be used efficiently for
composing a series of documents (such as exercise sheets)
which are typically distributed individually.
It then assists the author in generating the individual documents
(potentially in different versions)
as well as a document containing the collected series.
Another application is in developing style files
or other kinds of included material
where compilation of the style file could redirect
to a sample or test file.

%%%%%%%%%%%%%%%%%%%%%%%%%%%%%%%%%%%%%%%%%%%%%%%%%%%%%%%%%%%%%%%%%%%%%%%%%%%%%%%%
%%%%%%%%%%%%%%%%%%%%%%%%%%%%%%%%%%%%%%%%%%%%%%%%%%%%%%%%%%%%%%%%%%%%%%%%%%%%%%%%
\section{Usage}

First of all, the package \textsf{childdoc} is \emph{not} a standard
\LaTeXe{} |.sty| style file! Therefore it needs to be invoked in
a non-standard way.

%%%%%%%%%%%%%%%%%%%%%%%%%%%%%%%%%%%%%%%%%%%%%%%%%%%%%%%%%%%%%%%%%%%%%%%%%%%%%%%%
\subsection{Included Files}
\label{sec:include}

%%%%%%%%%%%%%%%%%%%%%%%%%%%%%%%%%%%%%%%%
\DescribeMacro{\childdocmain}
To use the package, add the commands
\begin{center}
\begin{tabular}{l}
|\input{childdoc.def}|\\
|\childdocmain{}|\\
\end{tabular}
\end{center}
at the very top of the main \LaTeX{} file,
in particular \emph{before} the |\documentclass| statement!
The argument of |\childdocmain| should be left empty
(but it must be present).

%%%%%%%%%%%%%%%%%%%%%%%%%%%%%%%%%%%%%%%%
\DescribeMacro{\childdocof}
Furthermore, add the commands
\begin{center}
\begin{tabular}{l}
|\input{childdoc.def}|\\
|\childdocof{|\textit{main}|}|\\
\end{tabular}
\end{center}
at the top of every child file \textit{child}
which is included by |\include{|\textit{child}|}|
from within the main file
(or at least for those files to be compiled individually).
The argument \textit{main} must be the filename of the main file.

There are a couple of
considerations in setting up the main and child documents:

%%%%%%%%%%%%%%%%%%%%%%%%%%%%%%%%%%%%%%%%
\paragraph{Restrictions.}

Please note the following restrictions:
\begin{itemize}
\item
|\childdocmain| must be called with one argument \textit{main}
to ensure compatibility with earlier version of the package.
It must either be empty (|\childdocmain{}|)
or precisely match the filename of the main file in which it is specified.
See \secref{sec:detection} for further information.
\item
The filename \textit{main} must be specified without the |.tex| extension.
\item
The filename \textit{main} is case sensitive
(even in case-insensitive file systems)
due to internal string comparison.
\item
The argument \textit{main} should be fully expanded, it cannot be a macro.
\item
Subdirectories and special characters should be avoided in filenames.
\item
The command |\childdocmain{|\textit{main}|}| must be followed by a whitespace.
It should not be followed immediately by another command
or by a comment mark `|%|'.
This is because the \TeX{} parser reads the token immediately following
the argument of |\childdocmain| and puts it
at the beginning of every child section;
however, a white\-space is ignored.
\end{itemize}

%%%%%%%%%%%%%%%%%%%%%%%%%%%%%%%%%%%%%%%%
\paragraph{Content of Main File.}

It is advisable to place all content in the child files included by |\include|.
Any output contained in the main file will appear in all child documents
unless suppressed manually;
it cannot be suppressed automatically by the |\includeonly| directive
and thus should normally be avoided.
A method to include some content in the main file
by means of conditional processing is described in \secref{sec:conditional}.

%%%%%%%%%%%%%%%%%%%%%%%%%%%%%%%%%%%%%%%%
\paragraph{Page Numbering.}

When only a part of the document is compiled,
the appropriate numbering of pages
(as well as other status parameters)
is determined from the |.aux| files.
The latter contain information from previous passes.
However this information needs to propagate through
all intermediate child documents.
Therefore the page numbering in child documents may well
be inconsistent until the complete document is compiled at least once.

A useful (if unconventional) way to always ensure a consistent
page numbering is to restart the numbering in each child document
and denote the pages by `\textit{child}|.|\textit{page}'
where \textit{child} represents the chapter/section number of the child file.
This can be achieved by the command
|\numberwithin{page}{|\textit{child}|}|
of the \textsf{amsmath} package
where \textit{child} can be |chapter| or |section|
depending on the chosen structuring.
Alternatively, one can modify the macro |\thepage| appropriately
and reset the counter |page| at the start of each child file.

%%%%%%%%%%%%%%%%%%%%%%%%%%%%%%%%%%%%%%%%%%%%%%%%%%%%%%%%%%%%%%%%%%%%%%%%%%%%%%%%
\subsection{Conditional Processing}
\label{sec:conditional}

The package provides a mechanism to compile different versions
of a document. To customise the versions further some conditional processing
can come in handy to distinguish which version is being compiled.
The package provides two macros to describe the compilation context:

%%%%%%%%%%%%%%%%%%%%%%%%%%%%%%%%%%%%%%%%
\DescribeMacro{\ifchilddoc}
The conditional |\ifchilddoc| distinguishes between the compilation of
child documents and the main document:
%
\begin{center}
|\ifchilddoc |\textit{child-code}| |[|\||else |\textit{main-code}]| \||fi|
\end{center}

%%%%%%%%%%%%%%%%%%%%%%%%%%%%%%%%%%%%%%%%
\DescribeMacro{\childdocname}
\DescribeMacro{\childdocjob}
The macro |\childdocname| contains the filename (without extension)
of the main or child file being processed.
Note that |\childdocjob| will always contain the name of the main file.

%%%%%%%%%%%%%%%%%%%%%%%%%%%%%%%%%%%%%%%%
\paragraph{Title Page.}

Conditional processing can be used to include a title or banner page
in the main document when proper precautions are taken.
Importantly, the code in the main file should ensure that the page counter
(as well as other status parameters which are stored in the |.aux| files)
takes the same value after the conditional processing.
Otherwise the page numbers may take divergent values
depending on which part is compiled.

For example, a title page could be declared by:
%
\begin{center}
\begin{tabular}{l}
|\ifchilddoc\||else|\\
|\addtocounter{page}{-1}|\\
\textit{code for title page}\\
|\newpage|\\
|\||fi|
\end{tabular}
\end{center}
%
A banner page for the child documents can be generated by:
%
\begin{center}
\begin{tabular}{l}
|\ifchilddoc|\\
|\addtocounter{page}{-1}|\\
\textit{code for banner page}\\
|\newpage|\\
|\||fi|
\end{tabular}
\end{center}
%
Here one could write a message such as:
\begin{center}
|This is the part \childdocname{} of \childdocjob{}.|
\end{center}

%%%%%%%%%%%%%%%%%%%%%%%%%%%%%%%%%%%%%%%%%%%%%%%%%%%%%%%%%%%%%%%%%%%%%%%%%%%%%%%%
\subsection{Flags}
\label{sec:flags}

The package makes it easy to generate different versions
of the main or child documents.
To this end compilation flags can be defined
and assigned different default values.
They will be particularly useful in conjunction
with the forwarding mechanism described in \secref{sec:forward}.

For example, it may be useful to have a flag |\version|
which can be set to |draft| or |final|.
The document source will contain some conditional code
depending on the value of |\version|.
Suppose further, the flag should default to |final| for the main file
and to |draft| for child files
which is a natural assignment for editing the document.
This is achieved by placing the following code
in the preamble of the main document
(below the |\childdocmain| directive):
%
\begin{center}
\begin{tabular}{l}
|\ifchilddoc|\\
|\providecommand{\version}{draft}|\\
|\||else|\\
|\providecommand{\version}{final}|\\
|\||fi|
\end{tabular}
\end{center}
%
The definition by |\providecommand| makes sure
that previous definitions are not overwritten.
Further statements |\providecommand{\version}{...}|
can thus be added before the above code to override it.

For the main file, one might add a line
(between |\childdocmain| and the above block)
%
\begin{center}
|%\ifchilddoc\||else\providecommand{\version}{draft}\||fi|
\end{center}
%
which can be uncommented to produce a draft version.
Likewise one can add a line to the very top of a child file
(above the |\childdocof{|\textit{main}|}| directive)
%
\begin{center}
|%\providecommand{\version}{final}|
\end{center}
%
which can be uncommented to produce the final version of this child document.

%%%%%%%%%%%%%%%%%%%%%%%%%%%%%%%%%%%%%%%%%%%%%%%%%%%%%%%%%%%%%%%%%%%%%%%%%%%%%%%%
\subsection{Forwarding}
\label{sec:forward}

Different versions of the main or child documents
using compilation flags as described in \secref{sec:flags}
can be (permanently) stored in different files
for convenient compilation, viewing and distribution.
To this end, the package defines a command
to pass on compilation to a different file:

%%%%%%%%%%%%%%%%%%%%%%%%%%%%%%%%%%%%%%%%
\DescribeMacro{\childdocforward}
The command |\childdocforward| redirects processing to
another source file:
%
\begin{center}
\begin{tabular}{l}
|\input{childdoc.def}|\\
|\childdocforward[|\textit{main}|]{|\textit{dest}|}|\\
\end{tabular}
\end{center}
%
The argument \textit{dest} is the destination file
(without extension).
It should be the main file or one of the child files.
Note that further \textsf{childdoc} directives
such as |\childdocof| and |\childdocforward|
in the indicated file will be processed in this form.
The optional argument \textit{main}
passes on directly to the main file \textit{main}
while pretending to compile the child \textit{dest}.
This form behaves as if \textit{dest}
issues |\childdocof{|\textit{main}|}| right away,
and no further \textsf{childdoc} directives will be processed.

%%%%%%%%%%%%%%%%%%%%%%%%%%%%%%%%%%%%%%%%
\DescribeMacro{\...prefix}
In the alternative form |\childdocforwardprefix|,
%
\begin{center}
\begin{tabular}{l}
|\input{childdoc.def}|\\
|\childdocforwardprefix[|\textit{main}|]{|\textit{prefix}|}{|\textit{dest}|}|
\end{tabular}
\end{center}
%
the destination file is determined by a pattern
depending on the current file:
To make this work, the current file must be called
`{\textit{prefix}\hspace{0.2em}\textit{suffix}}'
with \textit{prefix} matching precisely the argument.
Processing is then passed on to the file
`{\textit{dest}\hspace{0.2em}\textit{suffix}}'.
Surely, the same effect is achieved by
directly specifying the
argument `{\textit{dest}\hspace{0.2em}\textit{suffix}}'
in the first form.
However, that requires to set up a different file
for each child. With the alternative form of the command
all these files can have exactly the same content
which simplifies setting them up and maintaining them.

For example, the following file |draft.tex|
with a compilation flag |\version| as described in \secref{sec:flags}
compiles the main document as a draft:
%
\begin{center}
\begin{tabular}{l}
|\def\version{draft}|\\
|\input{childdoc.def}|\\
|\childdocforward{|\textit{main}|}|
\end{tabular}
\end{center}
%
Likewise, the following files |final|\textit{nn}|.tex|
compile the final version of the child document
|child|\textit{nn}|.tex|:
%
\begin{center}
\begin{tabular}{l}
|\def\version{final}|\\
|\input{childdoc.def}|\\
|\childdocforwardprefix{final}{child}|
\end{tabular}
\end{center}
%

Note that when several versions of a main file and/or of each child file
are to be generated, it may be convenient to set up a |Makefile| or
shell script to automatise the process.

%%%%%%%%%%%%%%%%%%%%%%%%%%%%%%%%%%%%%%%%%%%%%%%%%%%%%%%%%%%%%%%%%%%%%%%%%%%%%%%%
\subsection{Command Line Processing}
\label{sec:commandline}

The effect of redirection files can also be achieved by invoking
the \LaTeX{} compiler with a more elaborate command line.
Most conveniently this should be done as part
of a shell script or a |Makefile|.

When using \textsf{childdoc} in the main file, the following
command lines effectively perform a redirection
(note that depending on the shell being used,
backslashes may have to be doubled: `|\|' $\to$ `|\\|'):
%
\begin{center}
|... -jobname "|\textit{target}|" |\\|"|[\textit{flags}]%
|\input{childdoc.def}\childdocforward[|\textit{main}|]{|\textit{dest}|}"|
\end{center}
%
Here \textit{target} is the name of the output file,
\textit{main} is the name of the main file
and \textit{dest} is the name of the main or child file to be processed
(all filenames without extensions).
The optional argument \textit{main} can be omitted
if \textit{main} matches \textit{dest}.
Optionally, compilation \textit{flags} can be defined via |\def| commands.
This command line makes the \TeX{} engine believe
it is compiling the file \textit{target}
whose content is specified as the latter parameter.
The provided code then forwards the processing to
\textit{main} or \textit{dest} as described in \secref{sec:forward}.

%%%%%%%%%%%%%%%%%%%%%%%%%%%%%%%%%%%%%%%%%%%%%%%%%%%%%%%%%%%%%%%%%%%%%%%%%%%%%%%%
\subsection{Include by Input}
\label{sec:input}

Including child documents by |\include| has some restrictions by design.
Most notably, the content of a child document always occupies
its own set of pages; pages cannot be shared between child documents.
Usually, this behaviour makes perfect sense
because each child document contain an essential part of the document.
However, in some situations it may be desirable to compose
a document from a collection of parts
without having mandatory page breaks between then.
For this case, the package
provides a mechanism to include parts
by |\input| which can also be processed individually.
However, by construction this mechanism
requires manual handling of the content to be output.

%%%%%%%%%%%%%%%%%%%%%%%%%%%%%%%%%%%%%%%%
\DescribeMacro{\ifchilddocmanual}
The main file should be prepared as usual, see \secref{sec:include}.
However, the document body must make a distinction
between processing of an individual part and of the main document, e.g.:
%
\begin{center}
\begin{tabular}{l}
|\ifchilddocmanual|\\
|\input{\childdocname}|\\
|\||else|\\
\textit{document body with }|\input{|\textit{part}|}|\\
|\||fi|
\end{tabular}
\end{center}
%
The conditional |\ifchilddocmanual| is true whenever
a part to be included by |\input| is being compiled,
and the name of the part is stored in |\childdocname|.

%%%%%%%%%%%%%%%%%%%%%%%%%%%%%%%%%%%%%%%%
\DescribeMacro{\childdocby}
Each part to be included by |\input| should start with:
%
\begin{center}
\begin{tabular}{l}
|\input{childdoc.def}|\\
|\childdocby{|\textit{main}|}|\\
\end{tabular}
\end{center}
%
The directive |\childdocby| is similar to |\childdocof|
described in \secref{sec:include},
but the subsequent selection of content must be done manually.
To that end, both |\ifchilddoc| and |\ifchilddocmanual|
will be true upon processing of a part,
and the name of the part is stored in |\childdocname|.
Note that |\jobname| will be set to the filename of the current part
so that each part receives an individual |.aux| file
that does not interfere with the |.aux| file(s) of the main document.
This behaviour can be altered by the alternative form
|\childdocby[*]{|\textit{main}|}| (with a non-empty optional argument)
which uses the |.aux| file of the main document
by setting |\jobname| to \textit{main}.

%%%%%%%%%%%%%%%%%%%%%%%%%%%%%%%%%%%%%%%%%%%%%%%%%%%%%%%%%%%%%%%%%%%%%%%%%%%%%%%%
\subsection{Driver Development}
\label{sec:driver}

The \textsf{childdoc} mechanism can also be use for the development
of definition files such as \LaTeX{} styles or classes.
This case differs from the above setup with multiple parts
included by |\include| in that no |\includeonly| should be invoked.
This can be achieved by starting the include file
(before |\ProvidesPackage|) with:
%
\begin{center}
\begin{tabular}{l}
|\input{childdoc.def}|\\
|\childdocforward{|\textit{main}|}|\\
\end{tabular}
\end{center}
%
or alternatively with:
%
\begin{center}
\begin{tabular}{l}
|\input{childdoc.def}|\\
|\childdocby{|\textit{main}|}|\\
\end{tabular}
\end{center}
%
Both forms have slightly different effects as described above.
The main file is prepared as usual, see \secref{sec:include}.

%%%%%%%%%%%%%%%%%%%%%%%%%%%%%%%%%%%%%%%%%%%%%%%%%%%%%%%%%%%%%%%%%%%%%%%%%%%%%%%%
\subsection{Legacy Detection}
\label{sec:detection}

The directive |\childdocmain| in the main file can detect
whether the complete document or merely a child is to be compiled
even without using the directive |\childdocof|.
This method is deprecated because it is less robust
and there is no compelling reason to use it;
it is merely provided for backward compatibility
and it may be removed in future versions.

If the detection mechanism is to be used,
it is mandatory to correctly specify
the filename of the main file as the argument of |\childdocmain|:
%
\begin{center}
\begin{tabular}{l}
|\input{childdoc.def}|\\
|\childdocmain{|\textit{main}|}|\\
\end{tabular}
\end{center}
%
If |\jobname| does not match the argument \textit{main} of |\childdocmain|,
it is assumed that |\jobname| points to the child file to be compiled.
When using |\childdocmain| with the main file specified as argument,
it suffices to start a child file
with just |\input{|\textit{main}|}|
without loading of the package and using |\childdocof|.
If instead all processing is done
with the appropriate \textsf{childdoc} directives,
the argument of \textit{main} of |\childdocmain| can be empty.

An alternative version of the command line processing described
in \secref{sec:commandline} using the detection mechanism reads:
%
\begin{center}
|... -jobname "|\textit{target}|" "|[\textit{flags}]%
[|\def\jobname{|\textit{dest}|}|]|\input{|\textit{main}|}"|
\end{center}

%%%%%%%%%%%%%%%%%%%%%%%%%%%%%%%%%%%%%%%%%%%%%%%%%%%%%%%%%%%%%%%%%%%%%%%%%%%%%%%%
\subsection{Manual Code}
\label{sec:manual}

In case one cannot be certain whether the definitions file |childdoc.def|
is installed on the target \TeX{} distribution
and one prefers not to ship it,
it is conceivable to paste a few relevant commands into the sources.

To that end, drop all statements |\input{childdoc.def}|
and perform the replacements as outlined below.
Instead of |\childdocmain{|\textit{main}|}| add the following code
to the top of the main file:
%
\begin{center}
\begin{tabular}{l}
|\||ifdefined\childdocname\endinput\||fi\newif\ifchilddoc|\\
|\edef\childdocname{\scantokens\expandafter{\jobname\noexpand}}|\\
|\def\childdocmain{|\textit{main}|}\||ifx\childdocmain\childdocname\||else|\\
|\childdoctrue\includeonly{\childdocname}\let\jobname\childdocmain\||fi|\\
\end{tabular}
\end{center}
%
Instead of |\childdocof{|\textit{main}|}| just include the main file
at the top of each child file:
%
\begin{center}
|\input{|\textit{main}|}|
\end{center}
%
A simple redirection |\childdocforward{|\textit{dest}|}| is achieved by:
%
\begin{center}
|\def\jobname{|\textit{dest}|}\input{\jobname}|
\end{center}
%
The redirection with prefix
|\childdocforwardprefix[|\textit{prefix}|]{|\textit{dest}|}|
is accomplished by:
%
\begin{center}
\begin{tabular}{l}
|{\edef\jobname{\scantokens\expandafter{\jobname\noexpand}}|\\
|\def\redirectjob |\textit{prefix}|#1~~~{\gdef\jobname{|\textit{dest}|#1}}|\\
|\expandafter\redirectjob\jobname~~~}\input{\jobname}|
\end{tabular}
\end{center}

In an alternative approach,
child documents can be compiled by a specific command line
without additional code or specific definitions:
%
\begin{center}
|... -jobname "|\textit{target}|" "|[\textit{flags}]%
|\includeonly{|\textit{dest}|}\input{|\textit{main}|}"|
\end{center}
%

%%%%%%%%%%%%%%%%%%%%%%%%%%%%%%%%%%%%%%%%%%%%%%%%%%%%%%%%%%%%%%%%%%%%%%%%%%%%%%%%
%%%%%%%%%%%%%%%%%%%%%%%%%%%%%%%%%%%%%%%%%%%%%%%%%%%%%%%%%%%%%%%%%%%%%%%%%%%%%%%%
\section{Information}

%%%%%%%%%%%%%%%%%%%%%%%%%%%%%%%%%%%%%%%%%%%%%%%%%%%%%%%%%%%%%%%%%%%%%%%%%%%%%%%%
\subsection{Copyright}

Copyright \copyright{} 2017--2018 Niklas Beisert

This work may be distributed and/or modified under the
conditions of the \LaTeX{} Project Public License, either version 1.3
of this license or (at your option) any later version.
The latest version of this license is in
  \url{http://www.latex-project.org/lppl.txt}
and version 1.3 or later is part of all distributions of \LaTeX{}
version 2005/12/01 or later.

This work has the LPPL maintenance status `maintained'.

The Current Maintainer of this work is Niklas Beisert.

This work consists of the files |README.txt|, |childdoc.ins| and |childdoc.dtx|
as well as the derived files |childdoc.def|, |cdocsamp.tex|
with |cdocsch1.tex|, |cdocsch2.tex|, |cdocspt3.tex|, |cdocspt4.tex|,
|cdocsdrf.tex|, |cdocsfn1.tex|, |cdocsfn2.tex|
as well as |childdoc.pdf|.

%%%%%%%%%%%%%%%%%%%%%%%%%%%%%%%%%%%%%%%%%%%%%%%%%%%%%%%%%%%%%%%%%%%%%%%%%%%%%%%%
\subsection{Files and Installation}

The package consists of the files:
%
\begin{center}
\begin{tabular}{ll}
    |README.txt|   & readme file \\
    |childdoc.ins| & installation file \\
    |childdoc.dtx| & source file \\
    |childdoc.def| & definition file \\
    |cdocsamp.tex| & sample main file \\
    |cdocsch1.tex| & sample include file \\
    |cdocsch2.tex| & sample include file \\
    |cdocspt3.tex| & sample part file \\
    |cdocspt4.tex| & sample part file \\
    |cdocsdrf.tex| & sample redirection file \\
    |cdocsfn1.tex| & sample redirection file \\
    |cdocsfn2.tex| & sample redirection file \\
    |childdoc.pdf| & manual
\end{tabular}
\end{center}
%
The distribution consists of the files
|README.txt|, |childdoc.ins| and |childdoc.dtx|.
%
\begin{itemize}
\item
Run (pdf)\LaTeX{} on |childdoc.dtx|
to compile the manual |childdoc.pdf| (this file).
\item
Run \LaTeX{} on |childdoc.ins| to create the definitions file |childdoc.def|
and the sample |cdocsamp.tex| with include files
|cdocsch1.tex|, |cdocsch2.tex|, |cdocspt3.tex|, |cdocspt4.tex|,
|cdocsdrf.tex|, |cdocsfn1.tex|, |cdocsfn2.tex|.
Then copy the file |childdoc.def| to an appropriate directory of your \LaTeX{}
distribution, e.g.\ \textit{texmf-root}|/tex/latex/childdoc|.
\end{itemize}

%%%%%%%%%%%%%%%%%%%%%%%%%%%%%%%%%%%%%%%%%%%%%%%%%%%%%%%%%%%%%%%%%%%%%%%%%%%%%%%%
\subsection{Related CTAN Packages}

There are several other packages which offer a similar functionality:
%
\begin{itemize}
\item
The packages
\href{http://ctan.org/pkg/docmute}{\textsf{docmute}},
\href{http://ctan.org/pkg/includex}{\textsf{includex}} and
\href{http://ctan.org/pkg/standalone}{\textsf{standalone}}
provide commands to include only the document body of
a child file thus allowing both files to be compiled individually.
\item
The packages \href{http://ctan.org/pkg/subdocs}{\textsf{subdocs}}
and \href{http://ctan.org/pkg/subfiles}{\textsf{subfiles}}
provide structures in which the main and child documents can be
encapsulated and allowing them to be compiled individually.
The inclusion mechanism is different from the conventional |\include|.
\item
The package \href{http://ctan.org/pkg/combine}{\textsf{combine}}
is an elaborate solution to combine several documents into one.
\end{itemize}
%
See also the CTAN topic \href{http://ctan.org/topic/subdocs}{\textsf{subdocs}}
for further related packages.
The present package differs from the above solutions in that
a document structure constructed with the conventional |\include| mechanism
just needs two extra commands at the top of every file
such that all constituent files can be compiled individually.

%%%%%%%%%%%%%%%%%%%%%%%%%%%%%%%%%%%%%%%%%%%%%%%%%%%%%%%%%%%%%%%%%%%%%%%%%%%%%%%%
%\subsection{Feature Suggestions}
%
%The following is a list of features which may be useful for future
%versions of this package:
%%
%\begin{itemize}
%\item
%\ldots
%\end{itemize}

%%%%%%%%%%%%%%%%%%%%%%%%%%%%%%%%%%%%%%%%%%%%%%%%%%%%%%%%%%%%%%%%%%%%%%%%%%%%%%%%
\subsection{Revision History}

%%%%%%%%%%%%%%%%%%%%%%%%%%%%%%%%%%%%%%%%
\paragraph{v2.0:} 2018/12/30

\begin{itemize}
\item
immediate forward processing
\item
added |\childdocby| mechanism
\item
manual restructured
\end{itemize}

%%%%%%%%%%%%%%%%%%%%%%%%%%%%%%%%%%%%%%%%
\paragraph{v1.6:} 2018/01/17

\begin{itemize}
\item
application for development of include files
\item
corrections to manual
\end{itemize}

%%%%%%%%%%%%%%%%%%%%%%%%%%%%%%%%%%%%%%%%
\paragraph{v1.5:} 2017/05/21

\begin{itemize}
\item
more complete structuring introduced
\item
|\childdocof| introduced
\item
|\childdoc| renamed to |\childdocmain|
\item
|\childredirect| renamed to |\childdocforward| and |\childdocforwardprefix|
and functionality expanded
\end{itemize}

%%%%%%%%%%%%%%%%%%%%%%%%%%%%%%%%%%%%%%%%
\paragraph{v1.0:} 2017/04/27

\begin{itemize}
\item
manual and install package
\item
first version published on CTAN
\end{itemize}

%%%%%%%%%%%%%%%%%%%%%%%%%%%%%%%%%%%%%%%%
\paragraph{v0.6:} 2017/04/26

\begin{itemize}
\item
redirection mechanism added
\end{itemize}

%%%%%%%%%%%%%%%%%%%%%%%%%%%%%%%%%%%%%%%%
\paragraph{v0.5:} 2017/04/26

\begin{itemize}
\item
functionality in definition file
\end{itemize}


%%%%%%%%%%%%%%%%%%%%%%%%%%%%%%%%%%%%%%%%%%%%%%%%%%%%%%%%%%%%%%%%%%%%%%%%%%%%%%%%
%%%%%%%%%%%%%%%%%%%%%%%%%%%%%%%%%%%%%%%%%%%%%%%%%%%%%%%%%%%%%%%%%%%%%%%%%%%%%%%%
%%%%%%%%%%%%%%%%%%%%%%%%%%%%%%%%%%%%%%%%%%%%%%%%%%%%%%%%%%%%%%%%%%%%%%%%%%%%%%%%
\appendix

\settowidth\MacroIndent{\rmfamily\scriptsize 000\ }

 \DocInput{childdoc.dtx}

\end{document}
%</driver>
% \fi
%
% %%%%%%%%%%%%%%%%%%%%%%%%%%%%%%%%%%%%%%%%%%%%%%%%%%%%%%%%%%%%%%%%%%%%%%%%%%%%%%
% %%%%%%%%%%%%%%%%%%%%%%%%%%%%%%%%%%%%%%%%%%%%%%%%%%%%%%%%%%%%%%%%%%%%%%%%%%%%%%
% \section{Sample}
%\iffalse
%<*samplemain>
%\fi
%
% The following presents a sample document
% with two chapters, two parts, a title page,
% a compile flag as well as three forwarding files to set the flag.
% It consists of eight |.tex| files:
% \begin{center}
% \begin{tabular}{ll}
% |cdocsamp.tex|&main file\\
% |cdocsch1.tex|&include file for chapter 1\\
% |cdocsch2.tex|&include file for chapter 2\\
% |cdocspt3.tex|&include file for part 3\\
% |cdocspt4.tex|&include file for part 4\\
% |cdocsdrf.tex|&forwarding file for main file in draft mode\\
% |cdocsfi1.tex|&forwarding file for final version of chapter 1\\
% |cdocsfi2.tex|&forwarding file for final version of chapter 2\\
% \end{tabular}
% \end{center}
% Each of the eight files can be compiled directly by the \LaTeX{} compiler.
%
% %%%%%%%%%%%%%%%%%%%%%%%%%%%%%%%%%%%%%%
% \paragraph{Main File.}
%
% The main file is called |cdocsamp.tex|.
%
% Load the \textsf{childdoc} definitions and
% declare the filename for the main document:
%    \begin{macrocode}
\input{childdoc.def}
\childdocmain{}
%    \end{macrocode}

% Optional override for |\version| flag:
%    \begin{macrocode}
%%\ifchilddoc\else\providecommand{\version}{draft}\fi
%    \end{macrocode}

% Define the default values for the |\version| flag
% (|final| for the main file and |draft| for childs):
%    \begin{macrocode}
\ifchilddoc
\providecommand{\version}{draft}
\else
\providecommand{\version}{final}
\fi
%    \end{macrocode}

% Load the standard document class:
%    \begin{macrocode}
\documentclass[12pt]{article}
%    \end{macrocode}

% Start the document body:
%    \begin{macrocode}
\begin{document}
%    \end{macrocode}

% Declare a title page.
% Print title, part of document being processed and version flag:
%    \begin{macrocode}
\addtocounter{page}{-1}
\begin{center}
{\LARGE\bfseries{}childdoc example\par}
\vspace{1cm}
\ifchilddoc
\ifchilddocmanual part\else chapter\fi:
`\childdocname' of `\childdocjob'\par
\else
main document: `\childdocjob'\par
\fi
version: \version\par
\end{center}
\newpage
%    \end{macrocode}

% Manually include selected file,
% otherwise process as usual:
%    \begin{macrocode}
\ifchilddocmanual
\section*{part `\childdocname'}
\input{\childdocname}
\else
%    \end{macrocode}

% Include the two chapters:
%    \begin{macrocode}
\include{cdocsch1}
\include{cdocsch2}
%    \end{macrocode}

% Include the two parts unless only chapters should be displayed:
%    \begin{macrocode}
\ifchilddoc\else
\section{part three}
\input{cdocspt3}
\section{part four}
\input{cdocspt4}
\fi
%    \end{macrocode}

% Process as usual until here:
%    \begin{macrocode}
\fi
%    \end{macrocode}

% End of document body:
%    \begin{macrocode}
\end{document}
%    \end{macrocode}
%\iffalse
%</samplemain>
%\fi
%
% %%%%%%%%%%%%%%%%%%%%%%%%%%%%%%%%%%%%%%
% \paragraph{Chapter Include Files.}
%
% The include files are called |cdocsch1.tex| and |cdocsch2.tex|.
%
%\iffalse
%<*samplechap1|samplechap2>
%\fi

% Optional override for |\version| flag:
%    \begin{macrocode}
%%\providecommand{\version}{final}
%    \end{macrocode}

% Include the main document:
%    \begin{macrocode}
\input{childdoc.def}
\childdocof{cdocsamp}
%    \end{macrocode}

%\iffalse
%</samplechap1|samplechap2>
%\fi
%
%\iffalse
%<*samplechap1>
%\fi
% Some text for chapter 1:
%    \begin{macrocode}
\section{one}
some text in chapter one
%    \end{macrocode}

%\iffalse
%</samplechap1>
%\fi
% Some text for chapter 2:
%\iffalse
%<*samplechap2>
%\fi
%    \begin{macrocode}
\section{two}
more text in chapter two
%    \end{macrocode}

%\iffalse
%</samplechap2>
%\fi
%
% %%%%%%%%%%%%%%%%%%%%%%%%%%%%%%%%%%%%%%
% \paragraph{Part Include Files.}
%
% The include files are called |cdocspt3.tex| and |cdocspt4.tex|.
%
%\iffalse
%<*samplepart3|samplepart4>
%\fi

% Optional override for |\version| flag:
%    \begin{macrocode}
%%\providecommand{\version}{final}
%    \end{macrocode}

% Include the main document:
%    \begin{macrocode}
\input{childdoc.def}
\childdocby{cdocsamp}
%    \end{macrocode}

%\iffalse
%</samplepart3|samplepart4>
%\fi
%
%\iffalse
%<*samplepart3>
%\fi
% Some text for part 3:
%    \begin{macrocode}
some text in part three
%    \end{macrocode}

%\iffalse
%</samplepart3>
%\fi
% Some text for part 4:
%\iffalse
%<*samplepart4>
%\fi
%    \begin{macrocode}
more text in part four
%    \end{macrocode}

%\iffalse
%</samplepart4>
%\fi
%
% %%%%%%%%%%%%%%%%%%%%%%%%%%%%%%%%%%%%%%
% \paragraph{Forwarding for a Complete Draft.}
%
% The following forwarding file |cdocsdrf.tex|
% compiles the main document in draft mode:
%\iffalse
%<*sampledraft>
%\fi
%    \begin{macrocode}
\def\version{draft}
\input{childdoc.def}
\childdocforward{cdocsamp}
%    \end{macrocode}

%\iffalse
%</sampledraft>
%\fi
%
% %%%%%%%%%%%%%%%%%%%%%%%%%%%%%%%%%%%%%%
% \paragraph{Forwarding for Final Version of the Chapters.}
%
% The following forwarding files |cdocsfn1.tex| and |cdocsfn2.tex|
% (with identical content)
% compile the final versions of the child documents
% |cdocsch1.tex| and |cdocsch2.tex|, respectively:
%\iffalse
%<*samplefinal>
%\fi
%    \begin{macrocode}
\def\version{final}
\input{childdoc.def}
\childdocforwardprefix[cdocsamp]{cdocsfn}{cdocsch}
%    \end{macrocode}

%\iffalse
%</samplefinal>
%\fi
%
% %%%%%%%%%%%%%%%%%%%%%%%%%%%%%%%%%%%%%%
% \paragraph{Command Line Processing.}
%
% The following three command lines generate the output files
% |cdocscld|, |cdocscl1| and |cdocscl2|
% which should be identical to
% |cdocsdrf|, |cdocsch1| and |cdocsfn2|, respectively:
% \begin{center}
% \begin{tabular}{l}
% |latex -jobname cdocscld \|\\
% |  "\def\version{draft}\input{childdoc.def}\childdocforward{cdocsamp}"|\\
% |latex -jobname cdocscl1 \|\\
% |  "\input{childdoc.def}\childdocforward[cdocsamp]{cdocsch1}"|\\
% |latex -jobname cdocscl2 \|\\
% |  "\def\version{final}\input{childdoc.def}\childdocforward{cdocsch2}"|
% \end{tabular}
% \end{center}
% Note that the trailing backslash on each first line
% merely continues the input to the second line
% (for convenient cut ant paste).
% Furthermore, the command |latex| can be replaced by any
% of its alternative versions such as |pdflatex|.
%
% %%%%%%%%%%%%%%%%%%%%%%%%%%%%%%%%%%%%%%%%%%%%%%%%%%%%%%%%%%%%%%%%%%%%%%%%%%%%%%
% %%%%%%%%%%%%%%%%%%%%%%%%%%%%%%%%%%%%%%%%%%%%%%%%%%%%%%%%%%%%%%%%%%%%%%%%%%%%%%
% \section{Implementation}
%\iffalse
%<*package>
%\fi
%
% This section describes the definitions file |childdoc.def|.

% The definitions cannot be loaded using |\usepackage| or |\RequirePackage|
% which has a mechanism to prevent loading a style file more than once.
% When loading the definitions by means of |\input|
% multiple instances have to be prevented manually:
%\iffalse
%This code needs to be before the `\ProvidesFile' directive
%which is defined at the beginning of this file.
%Therefore it is also placed there and commented out here.
%</package>
%<*discard>
%\fi
%    \begin{macrocode}
\ifdefined\childdocmain\endinput\fi
%    \end{macrocode}
%\iffalse
%</discard>
%<*package>
%\fi
%
% \macro{\ifchilddoc}
% \macro{\ifchilddocmanual}
% The conditional |\ifchilddoc| tells whether a
% child (true) or main (false) document is being compiled.
% The conditional |\ifchilddocmanual| tells whether
% the |\includeonly| mechanism is used (false) or
% the selection of child files must be performed manually (true).
% The definitions initialise to false:
%    \begin{macrocode}
\newif\ifchilddoc
\newif\ifchilddocmanual
%    \end{macrocode}

% \macro{\childdocname}
% \macro{\childdocjob}
% The macro |\childdocname| stores the name of the main document
% to be compiled. The macro |\childdocjob| stores the name of
% the document on which the \LaTeX{} compiler was originally invoked.
% The content of |\jobname| cannot be compared
% to filenames specified in the source due to different catcodes.
% The following code rescans |\jobname|, stores the result
% in |\childdocname| and saves a copy in |\childdocjob|:
%    \begin{macrocode}
\edef\childdocname{\scantokens\expandafter{\jobname\noexpand}}
\let\childdocjob\childdocname
%    \end{macrocode}

% \macro{\childdocdisable}
% The macro |\childdocdisable| prevents the main file
% from being processed more than once.
% At this stage, the main document command |\childdocmain|
% is assumed to be called once again where it should do nothing.
% Any subsequent call to it should prevent
% a secondary processing of the main document
% It overwrites the forwarding commands
% |\childdocof| and |\childdocforward|
% with empty macros to prevent further inclusions of the main document:
%    \begin{macrocode}
\newcommand{\childdocdisable}
{
  \renewcommand{\childdocmain}[1]{\renewcommand{\childdocmain}[1]{\endinput}}
  \renewcommand{\childdocof}[1]{}
  \renewcommand{\childdocby}[2][]{}
  \renewcommand{\childdocforward}[2][]{}
  \renewcommand{\childdocdisable}{}
}
%    \end{macrocode}

% \macro{\childdocmain}
% The macro |\childdocmain| is to be called at the top of the main file
% with nothing or the main filename (without extension) as argument.
% First, it breaks loops.
% If the argument is not empty and does not match |\childdocname|
% (which is set by the first inclusion of |childdoc.def|),
% |\ifchilddoc| is set to true, |\includeonly| is applied to the child file
% and |\jobname| is set to the main file
% (for proper handling of |.aux| files):
%    \begin{macrocode}
\newcommand{\childdocmain}[1]
{
  \childdocdisable\childdocmain{}
  \if?#1?\else
    \begingroup
      \def\childdoctmp{#1}
      \ifx\childdoctmp\childdocname
        \def\childdoctmp{}
      \else
        \def\childdoctmp
        {
          \childdoctrue
          \includeonly{\childdocname}
          \def\childdocjob{#1}
          \def\jobname{#1}
        }
      \fi
      \expandafter
    \endgroup
    \childdoctmp
  \fi
}
%    \end{macrocode}

% \macro{\childdocof}
% The command |\childdocof| redirects
% compilation to the main file |#1|.
%    \begin{macrocode}
\newcommand{\childdocof}[1]
{
  \childdocdisable
  \childdoctrue
  \includeonly{\childdocname}
  \def\jobname{#1}
  \def\childdocjob{#1}
  \input{#1}
}
%    \end{macrocode}

% \macro{\childdocby}
% The command |\childdocby| ....
%    \begin{macrocode}
\newcommand{\childdocby}[2][]
{
  \childdocdisable
  \childdoctrue
  \childdocmanualtrue
  \if?#1?\else
    \def\jobname{#2}
  \fi
  \def\childdocjob{#2}
  \input{#2}
  \endinput
}
%    \end{macrocode}

% \macro{\childdocforward}
% The command |\childdocforward| redirects
% compilation to the main file or
% (if the optional argument is given) a child file.
% Parameters are set as if the main file
% or a child file starting with |\childdocof| was compiled.
% Then compilation is handed over to the main file:
%    \begin{macrocode}
\newcommand{\childdocforward}[2][]
{
  \begingroup
    \if?#1?
      \def\childdoctmp
      {
        \def\childdocname{#2}
        \def\childdocjob{#2}
        \def\jobname{#2}
        \input{#2}
        \endinput
      }
    \else
      \def\childdoctmp
      {
        \childdocdisable
        \def\childdocname{#2}
        \childdoctrue
        \includeonly{#2}
        \def\childdocjob{#1}
        \def\jobname{#1}
        \input{#1}
        \endinput
      }
    \fi
    \expandafter
  \endgroup
  \childdoctmp
}
%    \end{macrocode}

% \macro{\childdocforwardprefix}
% The command |\childdocforwardprefix| redirects
% compilation to the main or a child file by means of a pattern.
% The prefix |#1| in the current filename is replaced by |#2|
% and the suffix of the current filename is kept
% (it is assumed that the filename does not contain the substring `|~~~|'
% which is used as a delimiter).
% Compilation is handed over to the new file by |\childdocforward|:
%    \begin{macrocode}
\newcommand{\childdocforwardprefix}[3][]
{
  \begingroup
    \def\childdocextract #2##1~~~{\def\childdoctmp{\childdocforward[#1]{#3##1}}}
    \expandafter\childdocextract\childdocname~~~
    \expandafter
  \endgroup
  \childdoctmp
}
%    \end{macrocode}

% \macro{\childdoc}
% The deprecated macro |\childdoc| is a legacy version of |\childdocmain|:
%    \begin{macrocode}
\newcommand{\childdoc}{\childdocmain}
%    \end{macrocode}

% \macro{\childdocredirect}
% The deprecated macro |\childdocredirect| is a legacy version
% of |\childdocforward| and |\childdocforwardprefix|:
%    \begin{macrocode}
\newcommand{\childdocredirect}[2][]
{
  \begingroup
    \if?#1?
      \def\childdoctmp{\childdocforward{#2}}
    \else
      \def\childdoctmp{\childdocforwardprefix{#1}{#2}}
    \fi
    \expandafter
  \endgroup
  \childdoctmp
}
%    \end{macrocode}

%\iffalse
%</package>
%\fi
%
\endinput
|\\
|\childdocmain{|\textit{main}|}|\\
\end{tabular}
\end{center}
%
If |\jobname| does not match the argument \textit{main} of |\childdocmain|,
it is assumed that |\jobname| points to the child file to be compiled.
When using |\childdocmain| with the main file specified as argument,
it suffices to start a child file
with just |\input{|\textit{main}|}|
without loading of the package and using |\childdocof|.
If instead all processing is done
with the appropriate \textsf{childdoc} directives,
the argument of \textit{main} of |\childdocmain| can be empty.

An alternative version of the command line processing described
in \secref{sec:commandline} using the detection mechanism reads:
%
\begin{center}
|... -jobname "|\textit{target}|" "|[\textit{flags}]%
[|\def\jobname{|\textit{dest}|}|]|\input{|\textit{main}|}"|
\end{center}

%%%%%%%%%%%%%%%%%%%%%%%%%%%%%%%%%%%%%%%%%%%%%%%%%%%%%%%%%%%%%%%%%%%%%%%%%%%%%%%%
\subsection{Manual Code}
\label{sec:manual}

In case one cannot be certain whether the definitions file |childdoc.def|
is installed on the target \TeX{} distribution
and one prefers not to ship it,
it is conceivable to paste a few relevant commands into the sources.

To that end, drop all statements |% \iffalse
%
% childdoc.dtx Copyright (C) 2017-2018 Niklas Beisert
%
% This work may be distributed and/or modified under the
% conditions of the LaTeX Project Public License, either version 1.3
% of this license or (at your option) any later version.
% The latest version of this license is in
%   http://www.latex-project.org/lppl.txt
% and version 1.3 or later is part of all distributions of LaTeX
% version 2005/12/01 or later.
%
% This work has the LPPL maintenance status `maintained'.
%
% The Current Maintainer of this work is Niklas Beisert.
%
% This work consists of the files childdoc.dtx and childdoc.ins
% and the derived files childdoc.def and cdocsamp.tex with
% cdocsch1.tex, cdocsch2.tex, cdocsdrf.tex, cdocsfn1.tex, cdocsfn2.tex.
%
%<package>\ifdefined\childdocmain\endinput\fi
%<package>\ProvidesFile{childdoc.def}[2018/12/30 v2.0 child document driver]
%<samplemain>\ProvidesFile{cdocsamp.tex}[2018/12/30 v2.0 sample for childdoc]
%<*driver>
%\ProvidesFile{childdoc.drv}[2018/12/30 v2.0 childdoc reference manual file]
\PassOptionsToClass{10pt,a4paper}{article}
\documentclass{ltxdoc}

\usepackage[margin=35mm]{geometry}
\usepackage{hyperref}
\usepackage{hyperxmp}
\usepackage[usenames]{color}

\hypersetup{colorlinks=true}
\hypersetup{pdfstartview=FitH}
\hypersetup{pdfpagemode=UseNone}
\hypersetup{pdfsource={}}
\hypersetup{pdflang={en-UK}}
\hypersetup{pdfcopyright={Copyright 2017-2018 Niklas Beisert.
  This work may be distributed and/or modified under the
  conditions of the LaTeX Project Public License, either version 1.3
  of this license or (at your option) any later version.}}
\hypersetup{pdflicenseurl={http://www.latex-project.org/lppl.txt}}
\hypersetup{pdfcontactaddress={ETH Zurich, ITP, HIT K,
  Wolfgang-Pauli-Strasse 27}}
\hypersetup{pdfcontactpostcode={8093}}
\hypersetup{pdfcontactcity={Zurich}}
\hypersetup{pdfcontactcountry={Switzerland}}
\hypersetup{pdfcontactemail={nbeisert@itp.phys.ethz.ch}}
\hypersetup{pdfcontacturl={http://people.phys.ethz.ch/\xmptilde nbeisert/}}

\newcommand{\secref}[1]{\hyperref[#1]{section \ref*{#1}}}

\parskip1ex
\parindent0pt
\let\olditemize\itemize
\def\itemize{\olditemize\parskip0pt}

\begin{document}

\title{The \textsf{childdoc} Package}
\hypersetup{pdftitle={The childdoc Package}}
\author{Niklas Beisert\\[2ex]
  Institut f\"ur Theoretische Physik\\
  Eidgen\"ossische Technische Hochschule Z\"urich\\
  Wolfgang-Pauli-Strasse 27, 8093 Z\"urich, Switzerland\\[1ex]
  \href{mailto:nbeisert@itp.phys.ethz.ch}
  {\texttt{nbeisert@itp.phys.ethz.ch}}}
\hypersetup{pdfauthor={Niklas Beisert}}
\hypersetup{pdfsubject={Manual for the LaTeX2e Package childdoc}}
\date{30 December 2018, \textsf{v2.0}}
\maketitle

\begin{abstract}\noindent
\textsf{childdoc} is a \LaTeXe{} package
that enables the direct compilation
of document sections included by |\include|
to individual files.
\end{abstract}

\begingroup
\parskip0ex
\tableofcontents
\endgroup

%%%%%%%%%%%%%%%%%%%%%%%%%%%%%%%%%%%%%%%%%%%%%%%%%%%%%%%%%%%%%%%%%%%%%%%%%%%%%%%%
%%%%%%%%%%%%%%%%%%%%%%%%%%%%%%%%%%%%%%%%%%%%%%%%%%%%%%%%%%%%%%%%%%%%%%%%%%%%%%%%
\section{Introduction}

\LaTeX{} provides a mechanism to structure a large document (such as a book)
into a main file and several child files (containing the chapters)
using the |\include| command.
This mechanism is beneficial for documents
which span hundreds of pages in order to
make the source file(s) more manageable.
Moreover, compilation can be restricted to
selected child files by means of the |\includeonly| command.
The latter feature can be used to reduce the compilation time while editing
(this was significantly more useful in the earlier days of \LaTeX{})
or to generate a smaller document which is easier to navigate.
Another application of |\includeonly| is to generate
documents consisting of selected parts of the complete document.

However, there are a few drawbacks of the plain |\include| mechanism:
\begin{itemize}
\item
The child files cannot be compiled on their own,
they can only be compiled via the main file.
A naive editing environment
(such as a text editor with an option
to have the current file processed by \LaTeX)
may require one to switch to the main file before compiling;
attempting to compile the child file produces errors.
\item
The main file must be modified (each time)
to adjust the |\includeonly| command
to the present needs. This easily leaves the main file in a messy state.
\item
The generated document will always carry the filename
of the main document. This is inconvenient if
several child files are to be compiled and
to be kept for distribution.
\end{itemize}

The present package provides a simple interface
to make child files individually compilable by \LaTeX{}.
Compiling a child file then has the same effect as compiling
the main file with an |\includeonly| command
to select the appropriate child.
Moreover the generated document will carry the name of the child
rather than the main file.
This resolves all three above issues.

This feature is meant to make the editing of books,
thesis documents and lecture notes somewhat more convenient.
However, the package can also be used efficiently for
composing a series of documents (such as exercise sheets)
which are typically distributed individually.
It then assists the author in generating the individual documents
(potentially in different versions)
as well as a document containing the collected series.
Another application is in developing style files
or other kinds of included material
where compilation of the style file could redirect
to a sample or test file.

%%%%%%%%%%%%%%%%%%%%%%%%%%%%%%%%%%%%%%%%%%%%%%%%%%%%%%%%%%%%%%%%%%%%%%%%%%%%%%%%
%%%%%%%%%%%%%%%%%%%%%%%%%%%%%%%%%%%%%%%%%%%%%%%%%%%%%%%%%%%%%%%%%%%%%%%%%%%%%%%%
\section{Usage}

First of all, the package \textsf{childdoc} is \emph{not} a standard
\LaTeXe{} |.sty| style file! Therefore it needs to be invoked in
a non-standard way.

%%%%%%%%%%%%%%%%%%%%%%%%%%%%%%%%%%%%%%%%%%%%%%%%%%%%%%%%%%%%%%%%%%%%%%%%%%%%%%%%
\subsection{Included Files}
\label{sec:include}

%%%%%%%%%%%%%%%%%%%%%%%%%%%%%%%%%%%%%%%%
\DescribeMacro{\childdocmain}
To use the package, add the commands
\begin{center}
\begin{tabular}{l}
|\input{childdoc.def}|\\
|\childdocmain{}|\\
\end{tabular}
\end{center}
at the very top of the main \LaTeX{} file,
in particular \emph{before} the |\documentclass| statement!
The argument of |\childdocmain| should be left empty
(but it must be present).

%%%%%%%%%%%%%%%%%%%%%%%%%%%%%%%%%%%%%%%%
\DescribeMacro{\childdocof}
Furthermore, add the commands
\begin{center}
\begin{tabular}{l}
|\input{childdoc.def}|\\
|\childdocof{|\textit{main}|}|\\
\end{tabular}
\end{center}
at the top of every child file \textit{child}
which is included by |\include{|\textit{child}|}|
from within the main file
(or at least for those files to be compiled individually).
The argument \textit{main} must be the filename of the main file.

There are a couple of
considerations in setting up the main and child documents:

%%%%%%%%%%%%%%%%%%%%%%%%%%%%%%%%%%%%%%%%
\paragraph{Restrictions.}

Please note the following restrictions:
\begin{itemize}
\item
|\childdocmain| must be called with one argument \textit{main}
to ensure compatibility with earlier version of the package.
It must either be empty (|\childdocmain{}|)
or precisely match the filename of the main file in which it is specified.
See \secref{sec:detection} for further information.
\item
The filename \textit{main} must be specified without the |.tex| extension.
\item
The filename \textit{main} is case sensitive
(even in case-insensitive file systems)
due to internal string comparison.
\item
The argument \textit{main} should be fully expanded, it cannot be a macro.
\item
Subdirectories and special characters should be avoided in filenames.
\item
The command |\childdocmain{|\textit{main}|}| must be followed by a whitespace.
It should not be followed immediately by another command
or by a comment mark `|%|'.
This is because the \TeX{} parser reads the token immediately following
the argument of |\childdocmain| and puts it
at the beginning of every child section;
however, a white\-space is ignored.
\end{itemize}

%%%%%%%%%%%%%%%%%%%%%%%%%%%%%%%%%%%%%%%%
\paragraph{Content of Main File.}

It is advisable to place all content in the child files included by |\include|.
Any output contained in the main file will appear in all child documents
unless suppressed manually;
it cannot be suppressed automatically by the |\includeonly| directive
and thus should normally be avoided.
A method to include some content in the main file
by means of conditional processing is described in \secref{sec:conditional}.

%%%%%%%%%%%%%%%%%%%%%%%%%%%%%%%%%%%%%%%%
\paragraph{Page Numbering.}

When only a part of the document is compiled,
the appropriate numbering of pages
(as well as other status parameters)
is determined from the |.aux| files.
The latter contain information from previous passes.
However this information needs to propagate through
all intermediate child documents.
Therefore the page numbering in child documents may well
be inconsistent until the complete document is compiled at least once.

A useful (if unconventional) way to always ensure a consistent
page numbering is to restart the numbering in each child document
and denote the pages by `\textit{child}|.|\textit{page}'
where \textit{child} represents the chapter/section number of the child file.
This can be achieved by the command
|\numberwithin{page}{|\textit{child}|}|
of the \textsf{amsmath} package
where \textit{child} can be |chapter| or |section|
depending on the chosen structuring.
Alternatively, one can modify the macro |\thepage| appropriately
and reset the counter |page| at the start of each child file.

%%%%%%%%%%%%%%%%%%%%%%%%%%%%%%%%%%%%%%%%%%%%%%%%%%%%%%%%%%%%%%%%%%%%%%%%%%%%%%%%
\subsection{Conditional Processing}
\label{sec:conditional}

The package provides a mechanism to compile different versions
of a document. To customise the versions further some conditional processing
can come in handy to distinguish which version is being compiled.
The package provides two macros to describe the compilation context:

%%%%%%%%%%%%%%%%%%%%%%%%%%%%%%%%%%%%%%%%
\DescribeMacro{\ifchilddoc}
The conditional |\ifchilddoc| distinguishes between the compilation of
child documents and the main document:
%
\begin{center}
|\ifchilddoc |\textit{child-code}| |[|\||else |\textit{main-code}]| \||fi|
\end{center}

%%%%%%%%%%%%%%%%%%%%%%%%%%%%%%%%%%%%%%%%
\DescribeMacro{\childdocname}
\DescribeMacro{\childdocjob}
The macro |\childdocname| contains the filename (without extension)
of the main or child file being processed.
Note that |\childdocjob| will always contain the name of the main file.

%%%%%%%%%%%%%%%%%%%%%%%%%%%%%%%%%%%%%%%%
\paragraph{Title Page.}

Conditional processing can be used to include a title or banner page
in the main document when proper precautions are taken.
Importantly, the code in the main file should ensure that the page counter
(as well as other status parameters which are stored in the |.aux| files)
takes the same value after the conditional processing.
Otherwise the page numbers may take divergent values
depending on which part is compiled.

For example, a title page could be declared by:
%
\begin{center}
\begin{tabular}{l}
|\ifchilddoc\||else|\\
|\addtocounter{page}{-1}|\\
\textit{code for title page}\\
|\newpage|\\
|\||fi|
\end{tabular}
\end{center}
%
A banner page for the child documents can be generated by:
%
\begin{center}
\begin{tabular}{l}
|\ifchilddoc|\\
|\addtocounter{page}{-1}|\\
\textit{code for banner page}\\
|\newpage|\\
|\||fi|
\end{tabular}
\end{center}
%
Here one could write a message such as:
\begin{center}
|This is the part \childdocname{} of \childdocjob{}.|
\end{center}

%%%%%%%%%%%%%%%%%%%%%%%%%%%%%%%%%%%%%%%%%%%%%%%%%%%%%%%%%%%%%%%%%%%%%%%%%%%%%%%%
\subsection{Flags}
\label{sec:flags}

The package makes it easy to generate different versions
of the main or child documents.
To this end compilation flags can be defined
and assigned different default values.
They will be particularly useful in conjunction
with the forwarding mechanism described in \secref{sec:forward}.

For example, it may be useful to have a flag |\version|
which can be set to |draft| or |final|.
The document source will contain some conditional code
depending on the value of |\version|.
Suppose further, the flag should default to |final| for the main file
and to |draft| for child files
which is a natural assignment for editing the document.
This is achieved by placing the following code
in the preamble of the main document
(below the |\childdocmain| directive):
%
\begin{center}
\begin{tabular}{l}
|\ifchilddoc|\\
|\providecommand{\version}{draft}|\\
|\||else|\\
|\providecommand{\version}{final}|\\
|\||fi|
\end{tabular}
\end{center}
%
The definition by |\providecommand| makes sure
that previous definitions are not overwritten.
Further statements |\providecommand{\version}{...}|
can thus be added before the above code to override it.

For the main file, one might add a line
(between |\childdocmain| and the above block)
%
\begin{center}
|%\ifchilddoc\||else\providecommand{\version}{draft}\||fi|
\end{center}
%
which can be uncommented to produce a draft version.
Likewise one can add a line to the very top of a child file
(above the |\childdocof{|\textit{main}|}| directive)
%
\begin{center}
|%\providecommand{\version}{final}|
\end{center}
%
which can be uncommented to produce the final version of this child document.

%%%%%%%%%%%%%%%%%%%%%%%%%%%%%%%%%%%%%%%%%%%%%%%%%%%%%%%%%%%%%%%%%%%%%%%%%%%%%%%%
\subsection{Forwarding}
\label{sec:forward}

Different versions of the main or child documents
using compilation flags as described in \secref{sec:flags}
can be (permanently) stored in different files
for convenient compilation, viewing and distribution.
To this end, the package defines a command
to pass on compilation to a different file:

%%%%%%%%%%%%%%%%%%%%%%%%%%%%%%%%%%%%%%%%
\DescribeMacro{\childdocforward}
The command |\childdocforward| redirects processing to
another source file:
%
\begin{center}
\begin{tabular}{l}
|\input{childdoc.def}|\\
|\childdocforward[|\textit{main}|]{|\textit{dest}|}|\\
\end{tabular}
\end{center}
%
The argument \textit{dest} is the destination file
(without extension).
It should be the main file or one of the child files.
Note that further \textsf{childdoc} directives
such as |\childdocof| and |\childdocforward|
in the indicated file will be processed in this form.
The optional argument \textit{main}
passes on directly to the main file \textit{main}
while pretending to compile the child \textit{dest}.
This form behaves as if \textit{dest}
issues |\childdocof{|\textit{main}|}| right away,
and no further \textsf{childdoc} directives will be processed.

%%%%%%%%%%%%%%%%%%%%%%%%%%%%%%%%%%%%%%%%
\DescribeMacro{\...prefix}
In the alternative form |\childdocforwardprefix|,
%
\begin{center}
\begin{tabular}{l}
|\input{childdoc.def}|\\
|\childdocforwardprefix[|\textit{main}|]{|\textit{prefix}|}{|\textit{dest}|}|
\end{tabular}
\end{center}
%
the destination file is determined by a pattern
depending on the current file:
To make this work, the current file must be called
`{\textit{prefix}\hspace{0.2em}\textit{suffix}}'
with \textit{prefix} matching precisely the argument.
Processing is then passed on to the file
`{\textit{dest}\hspace{0.2em}\textit{suffix}}'.
Surely, the same effect is achieved by
directly specifying the
argument `{\textit{dest}\hspace{0.2em}\textit{suffix}}'
in the first form.
However, that requires to set up a different file
for each child. With the alternative form of the command
all these files can have exactly the same content
which simplifies setting them up and maintaining them.

For example, the following file |draft.tex|
with a compilation flag |\version| as described in \secref{sec:flags}
compiles the main document as a draft:
%
\begin{center}
\begin{tabular}{l}
|\def\version{draft}|\\
|\input{childdoc.def}|\\
|\childdocforward{|\textit{main}|}|
\end{tabular}
\end{center}
%
Likewise, the following files |final|\textit{nn}|.tex|
compile the final version of the child document
|child|\textit{nn}|.tex|:
%
\begin{center}
\begin{tabular}{l}
|\def\version{final}|\\
|\input{childdoc.def}|\\
|\childdocforwardprefix{final}{child}|
\end{tabular}
\end{center}
%

Note that when several versions of a main file and/or of each child file
are to be generated, it may be convenient to set up a |Makefile| or
shell script to automatise the process.

%%%%%%%%%%%%%%%%%%%%%%%%%%%%%%%%%%%%%%%%%%%%%%%%%%%%%%%%%%%%%%%%%%%%%%%%%%%%%%%%
\subsection{Command Line Processing}
\label{sec:commandline}

The effect of redirection files can also be achieved by invoking
the \LaTeX{} compiler with a more elaborate command line.
Most conveniently this should be done as part
of a shell script or a |Makefile|.

When using \textsf{childdoc} in the main file, the following
command lines effectively perform a redirection
(note that depending on the shell being used,
backslashes may have to be doubled: `|\|' $\to$ `|\\|'):
%
\begin{center}
|... -jobname "|\textit{target}|" |\\|"|[\textit{flags}]%
|\input{childdoc.def}\childdocforward[|\textit{main}|]{|\textit{dest}|}"|
\end{center}
%
Here \textit{target} is the name of the output file,
\textit{main} is the name of the main file
and \textit{dest} is the name of the main or child file to be processed
(all filenames without extensions).
The optional argument \textit{main} can be omitted
if \textit{main} matches \textit{dest}.
Optionally, compilation \textit{flags} can be defined via |\def| commands.
This command line makes the \TeX{} engine believe
it is compiling the file \textit{target}
whose content is specified as the latter parameter.
The provided code then forwards the processing to
\textit{main} or \textit{dest} as described in \secref{sec:forward}.

%%%%%%%%%%%%%%%%%%%%%%%%%%%%%%%%%%%%%%%%%%%%%%%%%%%%%%%%%%%%%%%%%%%%%%%%%%%%%%%%
\subsection{Include by Input}
\label{sec:input}

Including child documents by |\include| has some restrictions by design.
Most notably, the content of a child document always occupies
its own set of pages; pages cannot be shared between child documents.
Usually, this behaviour makes perfect sense
because each child document contain an essential part of the document.
However, in some situations it may be desirable to compose
a document from a collection of parts
without having mandatory page breaks between then.
For this case, the package
provides a mechanism to include parts
by |\input| which can also be processed individually.
However, by construction this mechanism
requires manual handling of the content to be output.

%%%%%%%%%%%%%%%%%%%%%%%%%%%%%%%%%%%%%%%%
\DescribeMacro{\ifchilddocmanual}
The main file should be prepared as usual, see \secref{sec:include}.
However, the document body must make a distinction
between processing of an individual part and of the main document, e.g.:
%
\begin{center}
\begin{tabular}{l}
|\ifchilddocmanual|\\
|\input{\childdocname}|\\
|\||else|\\
\textit{document body with }|\input{|\textit{part}|}|\\
|\||fi|
\end{tabular}
\end{center}
%
The conditional |\ifchilddocmanual| is true whenever
a part to be included by |\input| is being compiled,
and the name of the part is stored in |\childdocname|.

%%%%%%%%%%%%%%%%%%%%%%%%%%%%%%%%%%%%%%%%
\DescribeMacro{\childdocby}
Each part to be included by |\input| should start with:
%
\begin{center}
\begin{tabular}{l}
|\input{childdoc.def}|\\
|\childdocby{|\textit{main}|}|\\
\end{tabular}
\end{center}
%
The directive |\childdocby| is similar to |\childdocof|
described in \secref{sec:include},
but the subsequent selection of content must be done manually.
To that end, both |\ifchilddoc| and |\ifchilddocmanual|
will be true upon processing of a part,
and the name of the part is stored in |\childdocname|.
Note that |\jobname| will be set to the filename of the current part
so that each part receives an individual |.aux| file
that does not interfere with the |.aux| file(s) of the main document.
This behaviour can be altered by the alternative form
|\childdocby[*]{|\textit{main}|}| (with a non-empty optional argument)
which uses the |.aux| file of the main document
by setting |\jobname| to \textit{main}.

%%%%%%%%%%%%%%%%%%%%%%%%%%%%%%%%%%%%%%%%%%%%%%%%%%%%%%%%%%%%%%%%%%%%%%%%%%%%%%%%
\subsection{Driver Development}
\label{sec:driver}

The \textsf{childdoc} mechanism can also be use for the development
of definition files such as \LaTeX{} styles or classes.
This case differs from the above setup with multiple parts
included by |\include| in that no |\includeonly| should be invoked.
This can be achieved by starting the include file
(before |\ProvidesPackage|) with:
%
\begin{center}
\begin{tabular}{l}
|\input{childdoc.def}|\\
|\childdocforward{|\textit{main}|}|\\
\end{tabular}
\end{center}
%
or alternatively with:
%
\begin{center}
\begin{tabular}{l}
|\input{childdoc.def}|\\
|\childdocby{|\textit{main}|}|\\
\end{tabular}
\end{center}
%
Both forms have slightly different effects as described above.
The main file is prepared as usual, see \secref{sec:include}.

%%%%%%%%%%%%%%%%%%%%%%%%%%%%%%%%%%%%%%%%%%%%%%%%%%%%%%%%%%%%%%%%%%%%%%%%%%%%%%%%
\subsection{Legacy Detection}
\label{sec:detection}

The directive |\childdocmain| in the main file can detect
whether the complete document or merely a child is to be compiled
even without using the directive |\childdocof|.
This method is deprecated because it is less robust
and there is no compelling reason to use it;
it is merely provided for backward compatibility
and it may be removed in future versions.

If the detection mechanism is to be used,
it is mandatory to correctly specify
the filename of the main file as the argument of |\childdocmain|:
%
\begin{center}
\begin{tabular}{l}
|\input{childdoc.def}|\\
|\childdocmain{|\textit{main}|}|\\
\end{tabular}
\end{center}
%
If |\jobname| does not match the argument \textit{main} of |\childdocmain|,
it is assumed that |\jobname| points to the child file to be compiled.
When using |\childdocmain| with the main file specified as argument,
it suffices to start a child file
with just |\input{|\textit{main}|}|
without loading of the package and using |\childdocof|.
If instead all processing is done
with the appropriate \textsf{childdoc} directives,
the argument of \textit{main} of |\childdocmain| can be empty.

An alternative version of the command line processing described
in \secref{sec:commandline} using the detection mechanism reads:
%
\begin{center}
|... -jobname "|\textit{target}|" "|[\textit{flags}]%
[|\def\jobname{|\textit{dest}|}|]|\input{|\textit{main}|}"|
\end{center}

%%%%%%%%%%%%%%%%%%%%%%%%%%%%%%%%%%%%%%%%%%%%%%%%%%%%%%%%%%%%%%%%%%%%%%%%%%%%%%%%
\subsection{Manual Code}
\label{sec:manual}

In case one cannot be certain whether the definitions file |childdoc.def|
is installed on the target \TeX{} distribution
and one prefers not to ship it,
it is conceivable to paste a few relevant commands into the sources.

To that end, drop all statements |\input{childdoc.def}|
and perform the replacements as outlined below.
Instead of |\childdocmain{|\textit{main}|}| add the following code
to the top of the main file:
%
\begin{center}
\begin{tabular}{l}
|\||ifdefined\childdocname\endinput\||fi\newif\ifchilddoc|\\
|\edef\childdocname{\scantokens\expandafter{\jobname\noexpand}}|\\
|\def\childdocmain{|\textit{main}|}\||ifx\childdocmain\childdocname\||else|\\
|\childdoctrue\includeonly{\childdocname}\let\jobname\childdocmain\||fi|\\
\end{tabular}
\end{center}
%
Instead of |\childdocof{|\textit{main}|}| just include the main file
at the top of each child file:
%
\begin{center}
|\input{|\textit{main}|}|
\end{center}
%
A simple redirection |\childdocforward{|\textit{dest}|}| is achieved by:
%
\begin{center}
|\def\jobname{|\textit{dest}|}\input{\jobname}|
\end{center}
%
The redirection with prefix
|\childdocforwardprefix[|\textit{prefix}|]{|\textit{dest}|}|
is accomplished by:
%
\begin{center}
\begin{tabular}{l}
|{\edef\jobname{\scantokens\expandafter{\jobname\noexpand}}|\\
|\def\redirectjob |\textit{prefix}|#1~~~{\gdef\jobname{|\textit{dest}|#1}}|\\
|\expandafter\redirectjob\jobname~~~}\input{\jobname}|
\end{tabular}
\end{center}

In an alternative approach,
child documents can be compiled by a specific command line
without additional code or specific definitions:
%
\begin{center}
|... -jobname "|\textit{target}|" "|[\textit{flags}]%
|\includeonly{|\textit{dest}|}\input{|\textit{main}|}"|
\end{center}
%

%%%%%%%%%%%%%%%%%%%%%%%%%%%%%%%%%%%%%%%%%%%%%%%%%%%%%%%%%%%%%%%%%%%%%%%%%%%%%%%%
%%%%%%%%%%%%%%%%%%%%%%%%%%%%%%%%%%%%%%%%%%%%%%%%%%%%%%%%%%%%%%%%%%%%%%%%%%%%%%%%
\section{Information}

%%%%%%%%%%%%%%%%%%%%%%%%%%%%%%%%%%%%%%%%%%%%%%%%%%%%%%%%%%%%%%%%%%%%%%%%%%%%%%%%
\subsection{Copyright}

Copyright \copyright{} 2017--2018 Niklas Beisert

This work may be distributed and/or modified under the
conditions of the \LaTeX{} Project Public License, either version 1.3
of this license or (at your option) any later version.
The latest version of this license is in
  \url{http://www.latex-project.org/lppl.txt}
and version 1.3 or later is part of all distributions of \LaTeX{}
version 2005/12/01 or later.

This work has the LPPL maintenance status `maintained'.

The Current Maintainer of this work is Niklas Beisert.

This work consists of the files |README.txt|, |childdoc.ins| and |childdoc.dtx|
as well as the derived files |childdoc.def|, |cdocsamp.tex|
with |cdocsch1.tex|, |cdocsch2.tex|, |cdocspt3.tex|, |cdocspt4.tex|,
|cdocsdrf.tex|, |cdocsfn1.tex|, |cdocsfn2.tex|
as well as |childdoc.pdf|.

%%%%%%%%%%%%%%%%%%%%%%%%%%%%%%%%%%%%%%%%%%%%%%%%%%%%%%%%%%%%%%%%%%%%%%%%%%%%%%%%
\subsection{Files and Installation}

The package consists of the files:
%
\begin{center}
\begin{tabular}{ll}
    |README.txt|   & readme file \\
    |childdoc.ins| & installation file \\
    |childdoc.dtx| & source file \\
    |childdoc.def| & definition file \\
    |cdocsamp.tex| & sample main file \\
    |cdocsch1.tex| & sample include file \\
    |cdocsch2.tex| & sample include file \\
    |cdocspt3.tex| & sample part file \\
    |cdocspt4.tex| & sample part file \\
    |cdocsdrf.tex| & sample redirection file \\
    |cdocsfn1.tex| & sample redirection file \\
    |cdocsfn2.tex| & sample redirection file \\
    |childdoc.pdf| & manual
\end{tabular}
\end{center}
%
The distribution consists of the files
|README.txt|, |childdoc.ins| and |childdoc.dtx|.
%
\begin{itemize}
\item
Run (pdf)\LaTeX{} on |childdoc.dtx|
to compile the manual |childdoc.pdf| (this file).
\item
Run \LaTeX{} on |childdoc.ins| to create the definitions file |childdoc.def|
and the sample |cdocsamp.tex| with include files
|cdocsch1.tex|, |cdocsch2.tex|, |cdocspt3.tex|, |cdocspt4.tex|,
|cdocsdrf.tex|, |cdocsfn1.tex|, |cdocsfn2.tex|.
Then copy the file |childdoc.def| to an appropriate directory of your \LaTeX{}
distribution, e.g.\ \textit{texmf-root}|/tex/latex/childdoc|.
\end{itemize}

%%%%%%%%%%%%%%%%%%%%%%%%%%%%%%%%%%%%%%%%%%%%%%%%%%%%%%%%%%%%%%%%%%%%%%%%%%%%%%%%
\subsection{Related CTAN Packages}

There are several other packages which offer a similar functionality:
%
\begin{itemize}
\item
The packages
\href{http://ctan.org/pkg/docmute}{\textsf{docmute}},
\href{http://ctan.org/pkg/includex}{\textsf{includex}} and
\href{http://ctan.org/pkg/standalone}{\textsf{standalone}}
provide commands to include only the document body of
a child file thus allowing both files to be compiled individually.
\item
The packages \href{http://ctan.org/pkg/subdocs}{\textsf{subdocs}}
and \href{http://ctan.org/pkg/subfiles}{\textsf{subfiles}}
provide structures in which the main and child documents can be
encapsulated and allowing them to be compiled individually.
The inclusion mechanism is different from the conventional |\include|.
\item
The package \href{http://ctan.org/pkg/combine}{\textsf{combine}}
is an elaborate solution to combine several documents into one.
\end{itemize}
%
See also the CTAN topic \href{http://ctan.org/topic/subdocs}{\textsf{subdocs}}
for further related packages.
The present package differs from the above solutions in that
a document structure constructed with the conventional |\include| mechanism
just needs two extra commands at the top of every file
such that all constituent files can be compiled individually.

%%%%%%%%%%%%%%%%%%%%%%%%%%%%%%%%%%%%%%%%%%%%%%%%%%%%%%%%%%%%%%%%%%%%%%%%%%%%%%%%
%\subsection{Feature Suggestions}
%
%The following is a list of features which may be useful for future
%versions of this package:
%%
%\begin{itemize}
%\item
%\ldots
%\end{itemize}

%%%%%%%%%%%%%%%%%%%%%%%%%%%%%%%%%%%%%%%%%%%%%%%%%%%%%%%%%%%%%%%%%%%%%%%%%%%%%%%%
\subsection{Revision History}

%%%%%%%%%%%%%%%%%%%%%%%%%%%%%%%%%%%%%%%%
\paragraph{v2.0:} 2018/12/30

\begin{itemize}
\item
immediate forward processing
\item
added |\childdocby| mechanism
\item
manual restructured
\end{itemize}

%%%%%%%%%%%%%%%%%%%%%%%%%%%%%%%%%%%%%%%%
\paragraph{v1.6:} 2018/01/17

\begin{itemize}
\item
application for development of include files
\item
corrections to manual
\end{itemize}

%%%%%%%%%%%%%%%%%%%%%%%%%%%%%%%%%%%%%%%%
\paragraph{v1.5:} 2017/05/21

\begin{itemize}
\item
more complete structuring introduced
\item
|\childdocof| introduced
\item
|\childdoc| renamed to |\childdocmain|
\item
|\childredirect| renamed to |\childdocforward| and |\childdocforwardprefix|
and functionality expanded
\end{itemize}

%%%%%%%%%%%%%%%%%%%%%%%%%%%%%%%%%%%%%%%%
\paragraph{v1.0:} 2017/04/27

\begin{itemize}
\item
manual and install package
\item
first version published on CTAN
\end{itemize}

%%%%%%%%%%%%%%%%%%%%%%%%%%%%%%%%%%%%%%%%
\paragraph{v0.6:} 2017/04/26

\begin{itemize}
\item
redirection mechanism added
\end{itemize}

%%%%%%%%%%%%%%%%%%%%%%%%%%%%%%%%%%%%%%%%
\paragraph{v0.5:} 2017/04/26

\begin{itemize}
\item
functionality in definition file
\end{itemize}


%%%%%%%%%%%%%%%%%%%%%%%%%%%%%%%%%%%%%%%%%%%%%%%%%%%%%%%%%%%%%%%%%%%%%%%%%%%%%%%%
%%%%%%%%%%%%%%%%%%%%%%%%%%%%%%%%%%%%%%%%%%%%%%%%%%%%%%%%%%%%%%%%%%%%%%%%%%%%%%%%
%%%%%%%%%%%%%%%%%%%%%%%%%%%%%%%%%%%%%%%%%%%%%%%%%%%%%%%%%%%%%%%%%%%%%%%%%%%%%%%%
\appendix

\settowidth\MacroIndent{\rmfamily\scriptsize 000\ }

 \DocInput{childdoc.dtx}

\end{document}
%</driver>
% \fi
%
% %%%%%%%%%%%%%%%%%%%%%%%%%%%%%%%%%%%%%%%%%%%%%%%%%%%%%%%%%%%%%%%%%%%%%%%%%%%%%%
% %%%%%%%%%%%%%%%%%%%%%%%%%%%%%%%%%%%%%%%%%%%%%%%%%%%%%%%%%%%%%%%%%%%%%%%%%%%%%%
% \section{Sample}
%\iffalse
%<*samplemain>
%\fi
%
% The following presents a sample document
% with two chapters, two parts, a title page,
% a compile flag as well as three forwarding files to set the flag.
% It consists of eight |.tex| files:
% \begin{center}
% \begin{tabular}{ll}
% |cdocsamp.tex|&main file\\
% |cdocsch1.tex|&include file for chapter 1\\
% |cdocsch2.tex|&include file for chapter 2\\
% |cdocspt3.tex|&include file for part 3\\
% |cdocspt4.tex|&include file for part 4\\
% |cdocsdrf.tex|&forwarding file for main file in draft mode\\
% |cdocsfi1.tex|&forwarding file for final version of chapter 1\\
% |cdocsfi2.tex|&forwarding file for final version of chapter 2\\
% \end{tabular}
% \end{center}
% Each of the eight files can be compiled directly by the \LaTeX{} compiler.
%
% %%%%%%%%%%%%%%%%%%%%%%%%%%%%%%%%%%%%%%
% \paragraph{Main File.}
%
% The main file is called |cdocsamp.tex|.
%
% Load the \textsf{childdoc} definitions and
% declare the filename for the main document:
%    \begin{macrocode}
\input{childdoc.def}
\childdocmain{}
%    \end{macrocode}

% Optional override for |\version| flag:
%    \begin{macrocode}
%%\ifchilddoc\else\providecommand{\version}{draft}\fi
%    \end{macrocode}

% Define the default values for the |\version| flag
% (|final| for the main file and |draft| for childs):
%    \begin{macrocode}
\ifchilddoc
\providecommand{\version}{draft}
\else
\providecommand{\version}{final}
\fi
%    \end{macrocode}

% Load the standard document class:
%    \begin{macrocode}
\documentclass[12pt]{article}
%    \end{macrocode}

% Start the document body:
%    \begin{macrocode}
\begin{document}
%    \end{macrocode}

% Declare a title page.
% Print title, part of document being processed and version flag:
%    \begin{macrocode}
\addtocounter{page}{-1}
\begin{center}
{\LARGE\bfseries{}childdoc example\par}
\vspace{1cm}
\ifchilddoc
\ifchilddocmanual part\else chapter\fi:
`\childdocname' of `\childdocjob'\par
\else
main document: `\childdocjob'\par
\fi
version: \version\par
\end{center}
\newpage
%    \end{macrocode}

% Manually include selected file,
% otherwise process as usual:
%    \begin{macrocode}
\ifchilddocmanual
\section*{part `\childdocname'}
\input{\childdocname}
\else
%    \end{macrocode}

% Include the two chapters:
%    \begin{macrocode}
\include{cdocsch1}
\include{cdocsch2}
%    \end{macrocode}

% Include the two parts unless only chapters should be displayed:
%    \begin{macrocode}
\ifchilddoc\else
\section{part three}
\input{cdocspt3}
\section{part four}
\input{cdocspt4}
\fi
%    \end{macrocode}

% Process as usual until here:
%    \begin{macrocode}
\fi
%    \end{macrocode}

% End of document body:
%    \begin{macrocode}
\end{document}
%    \end{macrocode}
%\iffalse
%</samplemain>
%\fi
%
% %%%%%%%%%%%%%%%%%%%%%%%%%%%%%%%%%%%%%%
% \paragraph{Chapter Include Files.}
%
% The include files are called |cdocsch1.tex| and |cdocsch2.tex|.
%
%\iffalse
%<*samplechap1|samplechap2>
%\fi

% Optional override for |\version| flag:
%    \begin{macrocode}
%%\providecommand{\version}{final}
%    \end{macrocode}

% Include the main document:
%    \begin{macrocode}
\input{childdoc.def}
\childdocof{cdocsamp}
%    \end{macrocode}

%\iffalse
%</samplechap1|samplechap2>
%\fi
%
%\iffalse
%<*samplechap1>
%\fi
% Some text for chapter 1:
%    \begin{macrocode}
\section{one}
some text in chapter one
%    \end{macrocode}

%\iffalse
%</samplechap1>
%\fi
% Some text for chapter 2:
%\iffalse
%<*samplechap2>
%\fi
%    \begin{macrocode}
\section{two}
more text in chapter two
%    \end{macrocode}

%\iffalse
%</samplechap2>
%\fi
%
% %%%%%%%%%%%%%%%%%%%%%%%%%%%%%%%%%%%%%%
% \paragraph{Part Include Files.}
%
% The include files are called |cdocspt3.tex| and |cdocspt4.tex|.
%
%\iffalse
%<*samplepart3|samplepart4>
%\fi

% Optional override for |\version| flag:
%    \begin{macrocode}
%%\providecommand{\version}{final}
%    \end{macrocode}

% Include the main document:
%    \begin{macrocode}
\input{childdoc.def}
\childdocby{cdocsamp}
%    \end{macrocode}

%\iffalse
%</samplepart3|samplepart4>
%\fi
%
%\iffalse
%<*samplepart3>
%\fi
% Some text for part 3:
%    \begin{macrocode}
some text in part three
%    \end{macrocode}

%\iffalse
%</samplepart3>
%\fi
% Some text for part 4:
%\iffalse
%<*samplepart4>
%\fi
%    \begin{macrocode}
more text in part four
%    \end{macrocode}

%\iffalse
%</samplepart4>
%\fi
%
% %%%%%%%%%%%%%%%%%%%%%%%%%%%%%%%%%%%%%%
% \paragraph{Forwarding for a Complete Draft.}
%
% The following forwarding file |cdocsdrf.tex|
% compiles the main document in draft mode:
%\iffalse
%<*sampledraft>
%\fi
%    \begin{macrocode}
\def\version{draft}
\input{childdoc.def}
\childdocforward{cdocsamp}
%    \end{macrocode}

%\iffalse
%</sampledraft>
%\fi
%
% %%%%%%%%%%%%%%%%%%%%%%%%%%%%%%%%%%%%%%
% \paragraph{Forwarding for Final Version of the Chapters.}
%
% The following forwarding files |cdocsfn1.tex| and |cdocsfn2.tex|
% (with identical content)
% compile the final versions of the child documents
% |cdocsch1.tex| and |cdocsch2.tex|, respectively:
%\iffalse
%<*samplefinal>
%\fi
%    \begin{macrocode}
\def\version{final}
\input{childdoc.def}
\childdocforwardprefix[cdocsamp]{cdocsfn}{cdocsch}
%    \end{macrocode}

%\iffalse
%</samplefinal>
%\fi
%
% %%%%%%%%%%%%%%%%%%%%%%%%%%%%%%%%%%%%%%
% \paragraph{Command Line Processing.}
%
% The following three command lines generate the output files
% |cdocscld|, |cdocscl1| and |cdocscl2|
% which should be identical to
% |cdocsdrf|, |cdocsch1| and |cdocsfn2|, respectively:
% \begin{center}
% \begin{tabular}{l}
% |latex -jobname cdocscld \|\\
% |  "\def\version{draft}\input{childdoc.def}\childdocforward{cdocsamp}"|\\
% |latex -jobname cdocscl1 \|\\
% |  "\input{childdoc.def}\childdocforward[cdocsamp]{cdocsch1}"|\\
% |latex -jobname cdocscl2 \|\\
% |  "\def\version{final}\input{childdoc.def}\childdocforward{cdocsch2}"|
% \end{tabular}
% \end{center}
% Note that the trailing backslash on each first line
% merely continues the input to the second line
% (for convenient cut ant paste).
% Furthermore, the command |latex| can be replaced by any
% of its alternative versions such as |pdflatex|.
%
% %%%%%%%%%%%%%%%%%%%%%%%%%%%%%%%%%%%%%%%%%%%%%%%%%%%%%%%%%%%%%%%%%%%%%%%%%%%%%%
% %%%%%%%%%%%%%%%%%%%%%%%%%%%%%%%%%%%%%%%%%%%%%%%%%%%%%%%%%%%%%%%%%%%%%%%%%%%%%%
% \section{Implementation}
%\iffalse
%<*package>
%\fi
%
% This section describes the definitions file |childdoc.def|.

% The definitions cannot be loaded using |\usepackage| or |\RequirePackage|
% which has a mechanism to prevent loading a style file more than once.
% When loading the definitions by means of |\input|
% multiple instances have to be prevented manually:
%\iffalse
%This code needs to be before the `\ProvidesFile' directive
%which is defined at the beginning of this file.
%Therefore it is also placed there and commented out here.
%</package>
%<*discard>
%\fi
%    \begin{macrocode}
\ifdefined\childdocmain\endinput\fi
%    \end{macrocode}
%\iffalse
%</discard>
%<*package>
%\fi
%
% \macro{\ifchilddoc}
% \macro{\ifchilddocmanual}
% The conditional |\ifchilddoc| tells whether a
% child (true) or main (false) document is being compiled.
% The conditional |\ifchilddocmanual| tells whether
% the |\includeonly| mechanism is used (false) or
% the selection of child files must be performed manually (true).
% The definitions initialise to false:
%    \begin{macrocode}
\newif\ifchilddoc
\newif\ifchilddocmanual
%    \end{macrocode}

% \macro{\childdocname}
% \macro{\childdocjob}
% The macro |\childdocname| stores the name of the main document
% to be compiled. The macro |\childdocjob| stores the name of
% the document on which the \LaTeX{} compiler was originally invoked.
% The content of |\jobname| cannot be compared
% to filenames specified in the source due to different catcodes.
% The following code rescans |\jobname|, stores the result
% in |\childdocname| and saves a copy in |\childdocjob|:
%    \begin{macrocode}
\edef\childdocname{\scantokens\expandafter{\jobname\noexpand}}
\let\childdocjob\childdocname
%    \end{macrocode}

% \macro{\childdocdisable}
% The macro |\childdocdisable| prevents the main file
% from being processed more than once.
% At this stage, the main document command |\childdocmain|
% is assumed to be called once again where it should do nothing.
% Any subsequent call to it should prevent
% a secondary processing of the main document
% It overwrites the forwarding commands
% |\childdocof| and |\childdocforward|
% with empty macros to prevent further inclusions of the main document:
%    \begin{macrocode}
\newcommand{\childdocdisable}
{
  \renewcommand{\childdocmain}[1]{\renewcommand{\childdocmain}[1]{\endinput}}
  \renewcommand{\childdocof}[1]{}
  \renewcommand{\childdocby}[2][]{}
  \renewcommand{\childdocforward}[2][]{}
  \renewcommand{\childdocdisable}{}
}
%    \end{macrocode}

% \macro{\childdocmain}
% The macro |\childdocmain| is to be called at the top of the main file
% with nothing or the main filename (without extension) as argument.
% First, it breaks loops.
% If the argument is not empty and does not match |\childdocname|
% (which is set by the first inclusion of |childdoc.def|),
% |\ifchilddoc| is set to true, |\includeonly| is applied to the child file
% and |\jobname| is set to the main file
% (for proper handling of |.aux| files):
%    \begin{macrocode}
\newcommand{\childdocmain}[1]
{
  \childdocdisable\childdocmain{}
  \if?#1?\else
    \begingroup
      \def\childdoctmp{#1}
      \ifx\childdoctmp\childdocname
        \def\childdoctmp{}
      \else
        \def\childdoctmp
        {
          \childdoctrue
          \includeonly{\childdocname}
          \def\childdocjob{#1}
          \def\jobname{#1}
        }
      \fi
      \expandafter
    \endgroup
    \childdoctmp
  \fi
}
%    \end{macrocode}

% \macro{\childdocof}
% The command |\childdocof| redirects
% compilation to the main file |#1|.
%    \begin{macrocode}
\newcommand{\childdocof}[1]
{
  \childdocdisable
  \childdoctrue
  \includeonly{\childdocname}
  \def\jobname{#1}
  \def\childdocjob{#1}
  \input{#1}
}
%    \end{macrocode}

% \macro{\childdocby}
% The command |\childdocby| ....
%    \begin{macrocode}
\newcommand{\childdocby}[2][]
{
  \childdocdisable
  \childdoctrue
  \childdocmanualtrue
  \if?#1?\else
    \def\jobname{#2}
  \fi
  \def\childdocjob{#2}
  \input{#2}
  \endinput
}
%    \end{macrocode}

% \macro{\childdocforward}
% The command |\childdocforward| redirects
% compilation to the main file or
% (if the optional argument is given) a child file.
% Parameters are set as if the main file
% or a child file starting with |\childdocof| was compiled.
% Then compilation is handed over to the main file:
%    \begin{macrocode}
\newcommand{\childdocforward}[2][]
{
  \begingroup
    \if?#1?
      \def\childdoctmp
      {
        \def\childdocname{#2}
        \def\childdocjob{#2}
        \def\jobname{#2}
        \input{#2}
        \endinput
      }
    \else
      \def\childdoctmp
      {
        \childdocdisable
        \def\childdocname{#2}
        \childdoctrue
        \includeonly{#2}
        \def\childdocjob{#1}
        \def\jobname{#1}
        \input{#1}
        \endinput
      }
    \fi
    \expandafter
  \endgroup
  \childdoctmp
}
%    \end{macrocode}

% \macro{\childdocforwardprefix}
% The command |\childdocforwardprefix| redirects
% compilation to the main or a child file by means of a pattern.
% The prefix |#1| in the current filename is replaced by |#2|
% and the suffix of the current filename is kept
% (it is assumed that the filename does not contain the substring `|~~~|'
% which is used as a delimiter).
% Compilation is handed over to the new file by |\childdocforward|:
%    \begin{macrocode}
\newcommand{\childdocforwardprefix}[3][]
{
  \begingroup
    \def\childdocextract #2##1~~~{\def\childdoctmp{\childdocforward[#1]{#3##1}}}
    \expandafter\childdocextract\childdocname~~~
    \expandafter
  \endgroup
  \childdoctmp
}
%    \end{macrocode}

% \macro{\childdoc}
% The deprecated macro |\childdoc| is a legacy version of |\childdocmain|:
%    \begin{macrocode}
\newcommand{\childdoc}{\childdocmain}
%    \end{macrocode}

% \macro{\childdocredirect}
% The deprecated macro |\childdocredirect| is a legacy version
% of |\childdocforward| and |\childdocforwardprefix|:
%    \begin{macrocode}
\newcommand{\childdocredirect}[2][]
{
  \begingroup
    \if?#1?
      \def\childdoctmp{\childdocforward{#2}}
    \else
      \def\childdoctmp{\childdocforwardprefix{#1}{#2}}
    \fi
    \expandafter
  \endgroup
  \childdoctmp
}
%    \end{macrocode}

%\iffalse
%</package>
%\fi
%
\endinput
|
and perform the replacements as outlined below.
Instead of |\childdocmain{|\textit{main}|}| add the following code
to the top of the main file:
%
\begin{center}
\begin{tabular}{l}
|\||ifdefined\childdocname\endinput\||fi\newif\ifchilddoc|\\
|\edef\childdocname{\scantokens\expandafter{\jobname\noexpand}}|\\
|\def\childdocmain{|\textit{main}|}\||ifx\childdocmain\childdocname\||else|\\
|\childdoctrue\includeonly{\childdocname}\let\jobname\childdocmain\||fi|\\
\end{tabular}
\end{center}
%
Instead of |\childdocof{|\textit{main}|}| just include the main file
at the top of each child file:
%
\begin{center}
|\input{|\textit{main}|}|
\end{center}
%
A simple redirection |\childdocforward{|\textit{dest}|}| is achieved by:
%
\begin{center}
|\def\jobname{|\textit{dest}|}\input{\jobname}|
\end{center}
%
The redirection with prefix
|\childdocforwardprefix[|\textit{prefix}|]{|\textit{dest}|}|
is accomplished by:
%
\begin{center}
\begin{tabular}{l}
|{\edef\jobname{\scantokens\expandafter{\jobname\noexpand}}|\\
|\def\redirectjob |\textit{prefix}|#1~~~{\gdef\jobname{|\textit{dest}|#1}}|\\
|\expandafter\redirectjob\jobname~~~}\input{\jobname}|
\end{tabular}
\end{center}

In an alternative approach,
child documents can be compiled by a specific command line
without additional code or specific definitions:
%
\begin{center}
|... -jobname "|\textit{target}|" "|[\textit{flags}]%
|\includeonly{|\textit{dest}|}\input{|\textit{main}|}"|
\end{center}
%

%%%%%%%%%%%%%%%%%%%%%%%%%%%%%%%%%%%%%%%%%%%%%%%%%%%%%%%%%%%%%%%%%%%%%%%%%%%%%%%%
%%%%%%%%%%%%%%%%%%%%%%%%%%%%%%%%%%%%%%%%%%%%%%%%%%%%%%%%%%%%%%%%%%%%%%%%%%%%%%%%
\section{Information}

%%%%%%%%%%%%%%%%%%%%%%%%%%%%%%%%%%%%%%%%%%%%%%%%%%%%%%%%%%%%%%%%%%%%%%%%%%%%%%%%
\subsection{Copyright}

Copyright \copyright{} 2017--2018 Niklas Beisert

This work may be distributed and/or modified under the
conditions of the \LaTeX{} Project Public License, either version 1.3
of this license or (at your option) any later version.
The latest version of this license is in
  \url{http://www.latex-project.org/lppl.txt}
and version 1.3 or later is part of all distributions of \LaTeX{}
version 2005/12/01 or later.

This work has the LPPL maintenance status `maintained'.

The Current Maintainer of this work is Niklas Beisert.

This work consists of the files |README.txt|, |childdoc.ins| and |childdoc.dtx|
as well as the derived files |childdoc.def|, |cdocsamp.tex|
with |cdocsch1.tex|, |cdocsch2.tex|, |cdocspt3.tex|, |cdocspt4.tex|,
|cdocsdrf.tex|, |cdocsfn1.tex|, |cdocsfn2.tex|
as well as |childdoc.pdf|.

%%%%%%%%%%%%%%%%%%%%%%%%%%%%%%%%%%%%%%%%%%%%%%%%%%%%%%%%%%%%%%%%%%%%%%%%%%%%%%%%
\subsection{Files and Installation}

The package consists of the files:
%
\begin{center}
\begin{tabular}{ll}
    |README.txt|   & readme file \\
    |childdoc.ins| & installation file \\
    |childdoc.dtx| & source file \\
    |childdoc.def| & definition file \\
    |cdocsamp.tex| & sample main file \\
    |cdocsch1.tex| & sample include file \\
    |cdocsch2.tex| & sample include file \\
    |cdocspt3.tex| & sample part file \\
    |cdocspt4.tex| & sample part file \\
    |cdocsdrf.tex| & sample redirection file \\
    |cdocsfn1.tex| & sample redirection file \\
    |cdocsfn2.tex| & sample redirection file \\
    |childdoc.pdf| & manual
\end{tabular}
\end{center}
%
The distribution consists of the files
|README.txt|, |childdoc.ins| and |childdoc.dtx|.
%
\begin{itemize}
\item
Run (pdf)\LaTeX{} on |childdoc.dtx|
to compile the manual |childdoc.pdf| (this file).
\item
Run \LaTeX{} on |childdoc.ins| to create the definitions file |childdoc.def|
and the sample |cdocsamp.tex| with include files
|cdocsch1.tex|, |cdocsch2.tex|, |cdocspt3.tex|, |cdocspt4.tex|,
|cdocsdrf.tex|, |cdocsfn1.tex|, |cdocsfn2.tex|.
Then copy the file |childdoc.def| to an appropriate directory of your \LaTeX{}
distribution, e.g.\ \textit{texmf-root}|/tex/latex/childdoc|.
\end{itemize}

%%%%%%%%%%%%%%%%%%%%%%%%%%%%%%%%%%%%%%%%%%%%%%%%%%%%%%%%%%%%%%%%%%%%%%%%%%%%%%%%
\subsection{Related CTAN Packages}

There are several other packages which offer a similar functionality:
%
\begin{itemize}
\item
The packages
\href{http://ctan.org/pkg/docmute}{\textsf{docmute}},
\href{http://ctan.org/pkg/includex}{\textsf{includex}} and
\href{http://ctan.org/pkg/standalone}{\textsf{standalone}}
provide commands to include only the document body of
a child file thus allowing both files to be compiled individually.
\item
The packages \href{http://ctan.org/pkg/subdocs}{\textsf{subdocs}}
and \href{http://ctan.org/pkg/subfiles}{\textsf{subfiles}}
provide structures in which the main and child documents can be
encapsulated and allowing them to be compiled individually.
The inclusion mechanism is different from the conventional |\include|.
\item
The package \href{http://ctan.org/pkg/combine}{\textsf{combine}}
is an elaborate solution to combine several documents into one.
\end{itemize}
%
See also the CTAN topic \href{http://ctan.org/topic/subdocs}{\textsf{subdocs}}
for further related packages.
The present package differs from the above solutions in that
a document structure constructed with the conventional |\include| mechanism
just needs two extra commands at the top of every file
such that all constituent files can be compiled individually.

%%%%%%%%%%%%%%%%%%%%%%%%%%%%%%%%%%%%%%%%%%%%%%%%%%%%%%%%%%%%%%%%%%%%%%%%%%%%%%%%
%\subsection{Feature Suggestions}
%
%The following is a list of features which may be useful for future
%versions of this package:
%%
%\begin{itemize}
%\item
%\ldots
%\end{itemize}

%%%%%%%%%%%%%%%%%%%%%%%%%%%%%%%%%%%%%%%%%%%%%%%%%%%%%%%%%%%%%%%%%%%%%%%%%%%%%%%%
\subsection{Revision History}

%%%%%%%%%%%%%%%%%%%%%%%%%%%%%%%%%%%%%%%%
\paragraph{v2.0:} 2018/12/30

\begin{itemize}
\item
immediate forward processing
\item
added |\childdocby| mechanism
\item
manual restructured
\end{itemize}

%%%%%%%%%%%%%%%%%%%%%%%%%%%%%%%%%%%%%%%%
\paragraph{v1.6:} 2018/01/17

\begin{itemize}
\item
application for development of include files
\item
corrections to manual
\end{itemize}

%%%%%%%%%%%%%%%%%%%%%%%%%%%%%%%%%%%%%%%%
\paragraph{v1.5:} 2017/05/21

\begin{itemize}
\item
more complete structuring introduced
\item
|\childdocof| introduced
\item
|\childdoc| renamed to |\childdocmain|
\item
|\childredirect| renamed to |\childdocforward| and |\childdocforwardprefix|
and functionality expanded
\end{itemize}

%%%%%%%%%%%%%%%%%%%%%%%%%%%%%%%%%%%%%%%%
\paragraph{v1.0:} 2017/04/27

\begin{itemize}
\item
manual and install package
\item
first version published on CTAN
\end{itemize}

%%%%%%%%%%%%%%%%%%%%%%%%%%%%%%%%%%%%%%%%
\paragraph{v0.6:} 2017/04/26

\begin{itemize}
\item
redirection mechanism added
\end{itemize}

%%%%%%%%%%%%%%%%%%%%%%%%%%%%%%%%%%%%%%%%
\paragraph{v0.5:} 2017/04/26

\begin{itemize}
\item
functionality in definition file
\end{itemize}


%%%%%%%%%%%%%%%%%%%%%%%%%%%%%%%%%%%%%%%%%%%%%%%%%%%%%%%%%%%%%%%%%%%%%%%%%%%%%%%%
%%%%%%%%%%%%%%%%%%%%%%%%%%%%%%%%%%%%%%%%%%%%%%%%%%%%%%%%%%%%%%%%%%%%%%%%%%%%%%%%
%%%%%%%%%%%%%%%%%%%%%%%%%%%%%%%%%%%%%%%%%%%%%%%%%%%%%%%%%%%%%%%%%%%%%%%%%%%%%%%%
\appendix

\settowidth\MacroIndent{\rmfamily\scriptsize 000\ }

 \DocInput{childdoc.dtx}

\end{document}
%</driver>
% \fi
%
% %%%%%%%%%%%%%%%%%%%%%%%%%%%%%%%%%%%%%%%%%%%%%%%%%%%%%%%%%%%%%%%%%%%%%%%%%%%%%%
% %%%%%%%%%%%%%%%%%%%%%%%%%%%%%%%%%%%%%%%%%%%%%%%%%%%%%%%%%%%%%%%%%%%%%%%%%%%%%%
% \section{Sample}
%\iffalse
%<*samplemain>
%\fi
%
% The following presents a sample document
% with two chapters, two parts, a title page,
% a compile flag as well as three forwarding files to set the flag.
% It consists of eight |.tex| files:
% \begin{center}
% \begin{tabular}{ll}
% |cdocsamp.tex|&main file\\
% |cdocsch1.tex|&include file for chapter 1\\
% |cdocsch2.tex|&include file for chapter 2\\
% |cdocspt3.tex|&include file for part 3\\
% |cdocspt4.tex|&include file for part 4\\
% |cdocsdrf.tex|&forwarding file for main file in draft mode\\
% |cdocsfi1.tex|&forwarding file for final version of chapter 1\\
% |cdocsfi2.tex|&forwarding file for final version of chapter 2\\
% \end{tabular}
% \end{center}
% Each of the eight files can be compiled directly by the \LaTeX{} compiler.
%
% %%%%%%%%%%%%%%%%%%%%%%%%%%%%%%%%%%%%%%
% \paragraph{Main File.}
%
% The main file is called |cdocsamp.tex|.
%
% Load the \textsf{childdoc} definitions and
% declare the filename for the main document:
%    \begin{macrocode}
% \iffalse
%
% childdoc.dtx Copyright (C) 2017-2018 Niklas Beisert
%
% This work may be distributed and/or modified under the
% conditions of the LaTeX Project Public License, either version 1.3
% of this license or (at your option) any later version.
% The latest version of this license is in
%   http://www.latex-project.org/lppl.txt
% and version 1.3 or later is part of all distributions of LaTeX
% version 2005/12/01 or later.
%
% This work has the LPPL maintenance status `maintained'.
%
% The Current Maintainer of this work is Niklas Beisert.
%
% This work consists of the files childdoc.dtx and childdoc.ins
% and the derived files childdoc.def and cdocsamp.tex with
% cdocsch1.tex, cdocsch2.tex, cdocsdrf.tex, cdocsfn1.tex, cdocsfn2.tex.
%
%<package>\ifdefined\childdocmain\endinput\fi
%<package>\ProvidesFile{childdoc.def}[2018/12/30 v2.0 child document driver]
%<samplemain>\ProvidesFile{cdocsamp.tex}[2018/12/30 v2.0 sample for childdoc]
%<*driver>
%\ProvidesFile{childdoc.drv}[2018/12/30 v2.0 childdoc reference manual file]
\PassOptionsToClass{10pt,a4paper}{article}
\documentclass{ltxdoc}

\usepackage[margin=35mm]{geometry}
\usepackage{hyperref}
\usepackage{hyperxmp}
\usepackage[usenames]{color}

\hypersetup{colorlinks=true}
\hypersetup{pdfstartview=FitH}
\hypersetup{pdfpagemode=UseNone}
\hypersetup{pdfsource={}}
\hypersetup{pdflang={en-UK}}
\hypersetup{pdfcopyright={Copyright 2017-2018 Niklas Beisert.
  This work may be distributed and/or modified under the
  conditions of the LaTeX Project Public License, either version 1.3
  of this license or (at your option) any later version.}}
\hypersetup{pdflicenseurl={http://www.latex-project.org/lppl.txt}}
\hypersetup{pdfcontactaddress={ETH Zurich, ITP, HIT K,
  Wolfgang-Pauli-Strasse 27}}
\hypersetup{pdfcontactpostcode={8093}}
\hypersetup{pdfcontactcity={Zurich}}
\hypersetup{pdfcontactcountry={Switzerland}}
\hypersetup{pdfcontactemail={nbeisert@itp.phys.ethz.ch}}
\hypersetup{pdfcontacturl={http://people.phys.ethz.ch/\xmptilde nbeisert/}}

\newcommand{\secref}[1]{\hyperref[#1]{section \ref*{#1}}}

\parskip1ex
\parindent0pt
\let\olditemize\itemize
\def\itemize{\olditemize\parskip0pt}

\begin{document}

\title{The \textsf{childdoc} Package}
\hypersetup{pdftitle={The childdoc Package}}
\author{Niklas Beisert\\[2ex]
  Institut f\"ur Theoretische Physik\\
  Eidgen\"ossische Technische Hochschule Z\"urich\\
  Wolfgang-Pauli-Strasse 27, 8093 Z\"urich, Switzerland\\[1ex]
  \href{mailto:nbeisert@itp.phys.ethz.ch}
  {\texttt{nbeisert@itp.phys.ethz.ch}}}
\hypersetup{pdfauthor={Niklas Beisert}}
\hypersetup{pdfsubject={Manual for the LaTeX2e Package childdoc}}
\date{30 December 2018, \textsf{v2.0}}
\maketitle

\begin{abstract}\noindent
\textsf{childdoc} is a \LaTeXe{} package
that enables the direct compilation
of document sections included by |\include|
to individual files.
\end{abstract}

\begingroup
\parskip0ex
\tableofcontents
\endgroup

%%%%%%%%%%%%%%%%%%%%%%%%%%%%%%%%%%%%%%%%%%%%%%%%%%%%%%%%%%%%%%%%%%%%%%%%%%%%%%%%
%%%%%%%%%%%%%%%%%%%%%%%%%%%%%%%%%%%%%%%%%%%%%%%%%%%%%%%%%%%%%%%%%%%%%%%%%%%%%%%%
\section{Introduction}

\LaTeX{} provides a mechanism to structure a large document (such as a book)
into a main file and several child files (containing the chapters)
using the |\include| command.
This mechanism is beneficial for documents
which span hundreds of pages in order to
make the source file(s) more manageable.
Moreover, compilation can be restricted to
selected child files by means of the |\includeonly| command.
The latter feature can be used to reduce the compilation time while editing
(this was significantly more useful in the earlier days of \LaTeX{})
or to generate a smaller document which is easier to navigate.
Another application of |\includeonly| is to generate
documents consisting of selected parts of the complete document.

However, there are a few drawbacks of the plain |\include| mechanism:
\begin{itemize}
\item
The child files cannot be compiled on their own,
they can only be compiled via the main file.
A naive editing environment
(such as a text editor with an option
to have the current file processed by \LaTeX)
may require one to switch to the main file before compiling;
attempting to compile the child file produces errors.
\item
The main file must be modified (each time)
to adjust the |\includeonly| command
to the present needs. This easily leaves the main file in a messy state.
\item
The generated document will always carry the filename
of the main document. This is inconvenient if
several child files are to be compiled and
to be kept for distribution.
\end{itemize}

The present package provides a simple interface
to make child files individually compilable by \LaTeX{}.
Compiling a child file then has the same effect as compiling
the main file with an |\includeonly| command
to select the appropriate child.
Moreover the generated document will carry the name of the child
rather than the main file.
This resolves all three above issues.

This feature is meant to make the editing of books,
thesis documents and lecture notes somewhat more convenient.
However, the package can also be used efficiently for
composing a series of documents (such as exercise sheets)
which are typically distributed individually.
It then assists the author in generating the individual documents
(potentially in different versions)
as well as a document containing the collected series.
Another application is in developing style files
or other kinds of included material
where compilation of the style file could redirect
to a sample or test file.

%%%%%%%%%%%%%%%%%%%%%%%%%%%%%%%%%%%%%%%%%%%%%%%%%%%%%%%%%%%%%%%%%%%%%%%%%%%%%%%%
%%%%%%%%%%%%%%%%%%%%%%%%%%%%%%%%%%%%%%%%%%%%%%%%%%%%%%%%%%%%%%%%%%%%%%%%%%%%%%%%
\section{Usage}

First of all, the package \textsf{childdoc} is \emph{not} a standard
\LaTeXe{} |.sty| style file! Therefore it needs to be invoked in
a non-standard way.

%%%%%%%%%%%%%%%%%%%%%%%%%%%%%%%%%%%%%%%%%%%%%%%%%%%%%%%%%%%%%%%%%%%%%%%%%%%%%%%%
\subsection{Included Files}
\label{sec:include}

%%%%%%%%%%%%%%%%%%%%%%%%%%%%%%%%%%%%%%%%
\DescribeMacro{\childdocmain}
To use the package, add the commands
\begin{center}
\begin{tabular}{l}
|\input{childdoc.def}|\\
|\childdocmain{}|\\
\end{tabular}
\end{center}
at the very top of the main \LaTeX{} file,
in particular \emph{before} the |\documentclass| statement!
The argument of |\childdocmain| should be left empty
(but it must be present).

%%%%%%%%%%%%%%%%%%%%%%%%%%%%%%%%%%%%%%%%
\DescribeMacro{\childdocof}
Furthermore, add the commands
\begin{center}
\begin{tabular}{l}
|\input{childdoc.def}|\\
|\childdocof{|\textit{main}|}|\\
\end{tabular}
\end{center}
at the top of every child file \textit{child}
which is included by |\include{|\textit{child}|}|
from within the main file
(or at least for those files to be compiled individually).
The argument \textit{main} must be the filename of the main file.

There are a couple of
considerations in setting up the main and child documents:

%%%%%%%%%%%%%%%%%%%%%%%%%%%%%%%%%%%%%%%%
\paragraph{Restrictions.}

Please note the following restrictions:
\begin{itemize}
\item
|\childdocmain| must be called with one argument \textit{main}
to ensure compatibility with earlier version of the package.
It must either be empty (|\childdocmain{}|)
or precisely match the filename of the main file in which it is specified.
See \secref{sec:detection} for further information.
\item
The filename \textit{main} must be specified without the |.tex| extension.
\item
The filename \textit{main} is case sensitive
(even in case-insensitive file systems)
due to internal string comparison.
\item
The argument \textit{main} should be fully expanded, it cannot be a macro.
\item
Subdirectories and special characters should be avoided in filenames.
\item
The command |\childdocmain{|\textit{main}|}| must be followed by a whitespace.
It should not be followed immediately by another command
or by a comment mark `|%|'.
This is because the \TeX{} parser reads the token immediately following
the argument of |\childdocmain| and puts it
at the beginning of every child section;
however, a white\-space is ignored.
\end{itemize}

%%%%%%%%%%%%%%%%%%%%%%%%%%%%%%%%%%%%%%%%
\paragraph{Content of Main File.}

It is advisable to place all content in the child files included by |\include|.
Any output contained in the main file will appear in all child documents
unless suppressed manually;
it cannot be suppressed automatically by the |\includeonly| directive
and thus should normally be avoided.
A method to include some content in the main file
by means of conditional processing is described in \secref{sec:conditional}.

%%%%%%%%%%%%%%%%%%%%%%%%%%%%%%%%%%%%%%%%
\paragraph{Page Numbering.}

When only a part of the document is compiled,
the appropriate numbering of pages
(as well as other status parameters)
is determined from the |.aux| files.
The latter contain information from previous passes.
However this information needs to propagate through
all intermediate child documents.
Therefore the page numbering in child documents may well
be inconsistent until the complete document is compiled at least once.

A useful (if unconventional) way to always ensure a consistent
page numbering is to restart the numbering in each child document
and denote the pages by `\textit{child}|.|\textit{page}'
where \textit{child} represents the chapter/section number of the child file.
This can be achieved by the command
|\numberwithin{page}{|\textit{child}|}|
of the \textsf{amsmath} package
where \textit{child} can be |chapter| or |section|
depending on the chosen structuring.
Alternatively, one can modify the macro |\thepage| appropriately
and reset the counter |page| at the start of each child file.

%%%%%%%%%%%%%%%%%%%%%%%%%%%%%%%%%%%%%%%%%%%%%%%%%%%%%%%%%%%%%%%%%%%%%%%%%%%%%%%%
\subsection{Conditional Processing}
\label{sec:conditional}

The package provides a mechanism to compile different versions
of a document. To customise the versions further some conditional processing
can come in handy to distinguish which version is being compiled.
The package provides two macros to describe the compilation context:

%%%%%%%%%%%%%%%%%%%%%%%%%%%%%%%%%%%%%%%%
\DescribeMacro{\ifchilddoc}
The conditional |\ifchilddoc| distinguishes between the compilation of
child documents and the main document:
%
\begin{center}
|\ifchilddoc |\textit{child-code}| |[|\||else |\textit{main-code}]| \||fi|
\end{center}

%%%%%%%%%%%%%%%%%%%%%%%%%%%%%%%%%%%%%%%%
\DescribeMacro{\childdocname}
\DescribeMacro{\childdocjob}
The macro |\childdocname| contains the filename (without extension)
of the main or child file being processed.
Note that |\childdocjob| will always contain the name of the main file.

%%%%%%%%%%%%%%%%%%%%%%%%%%%%%%%%%%%%%%%%
\paragraph{Title Page.}

Conditional processing can be used to include a title or banner page
in the main document when proper precautions are taken.
Importantly, the code in the main file should ensure that the page counter
(as well as other status parameters which are stored in the |.aux| files)
takes the same value after the conditional processing.
Otherwise the page numbers may take divergent values
depending on which part is compiled.

For example, a title page could be declared by:
%
\begin{center}
\begin{tabular}{l}
|\ifchilddoc\||else|\\
|\addtocounter{page}{-1}|\\
\textit{code for title page}\\
|\newpage|\\
|\||fi|
\end{tabular}
\end{center}
%
A banner page for the child documents can be generated by:
%
\begin{center}
\begin{tabular}{l}
|\ifchilddoc|\\
|\addtocounter{page}{-1}|\\
\textit{code for banner page}\\
|\newpage|\\
|\||fi|
\end{tabular}
\end{center}
%
Here one could write a message such as:
\begin{center}
|This is the part \childdocname{} of \childdocjob{}.|
\end{center}

%%%%%%%%%%%%%%%%%%%%%%%%%%%%%%%%%%%%%%%%%%%%%%%%%%%%%%%%%%%%%%%%%%%%%%%%%%%%%%%%
\subsection{Flags}
\label{sec:flags}

The package makes it easy to generate different versions
of the main or child documents.
To this end compilation flags can be defined
and assigned different default values.
They will be particularly useful in conjunction
with the forwarding mechanism described in \secref{sec:forward}.

For example, it may be useful to have a flag |\version|
which can be set to |draft| or |final|.
The document source will contain some conditional code
depending on the value of |\version|.
Suppose further, the flag should default to |final| for the main file
and to |draft| for child files
which is a natural assignment for editing the document.
This is achieved by placing the following code
in the preamble of the main document
(below the |\childdocmain| directive):
%
\begin{center}
\begin{tabular}{l}
|\ifchilddoc|\\
|\providecommand{\version}{draft}|\\
|\||else|\\
|\providecommand{\version}{final}|\\
|\||fi|
\end{tabular}
\end{center}
%
The definition by |\providecommand| makes sure
that previous definitions are not overwritten.
Further statements |\providecommand{\version}{...}|
can thus be added before the above code to override it.

For the main file, one might add a line
(between |\childdocmain| and the above block)
%
\begin{center}
|%\ifchilddoc\||else\providecommand{\version}{draft}\||fi|
\end{center}
%
which can be uncommented to produce a draft version.
Likewise one can add a line to the very top of a child file
(above the |\childdocof{|\textit{main}|}| directive)
%
\begin{center}
|%\providecommand{\version}{final}|
\end{center}
%
which can be uncommented to produce the final version of this child document.

%%%%%%%%%%%%%%%%%%%%%%%%%%%%%%%%%%%%%%%%%%%%%%%%%%%%%%%%%%%%%%%%%%%%%%%%%%%%%%%%
\subsection{Forwarding}
\label{sec:forward}

Different versions of the main or child documents
using compilation flags as described in \secref{sec:flags}
can be (permanently) stored in different files
for convenient compilation, viewing and distribution.
To this end, the package defines a command
to pass on compilation to a different file:

%%%%%%%%%%%%%%%%%%%%%%%%%%%%%%%%%%%%%%%%
\DescribeMacro{\childdocforward}
The command |\childdocforward| redirects processing to
another source file:
%
\begin{center}
\begin{tabular}{l}
|\input{childdoc.def}|\\
|\childdocforward[|\textit{main}|]{|\textit{dest}|}|\\
\end{tabular}
\end{center}
%
The argument \textit{dest} is the destination file
(without extension).
It should be the main file or one of the child files.
Note that further \textsf{childdoc} directives
such as |\childdocof| and |\childdocforward|
in the indicated file will be processed in this form.
The optional argument \textit{main}
passes on directly to the main file \textit{main}
while pretending to compile the child \textit{dest}.
This form behaves as if \textit{dest}
issues |\childdocof{|\textit{main}|}| right away,
and no further \textsf{childdoc} directives will be processed.

%%%%%%%%%%%%%%%%%%%%%%%%%%%%%%%%%%%%%%%%
\DescribeMacro{\...prefix}
In the alternative form |\childdocforwardprefix|,
%
\begin{center}
\begin{tabular}{l}
|\input{childdoc.def}|\\
|\childdocforwardprefix[|\textit{main}|]{|\textit{prefix}|}{|\textit{dest}|}|
\end{tabular}
\end{center}
%
the destination file is determined by a pattern
depending on the current file:
To make this work, the current file must be called
`{\textit{prefix}\hspace{0.2em}\textit{suffix}}'
with \textit{prefix} matching precisely the argument.
Processing is then passed on to the file
`{\textit{dest}\hspace{0.2em}\textit{suffix}}'.
Surely, the same effect is achieved by
directly specifying the
argument `{\textit{dest}\hspace{0.2em}\textit{suffix}}'
in the first form.
However, that requires to set up a different file
for each child. With the alternative form of the command
all these files can have exactly the same content
which simplifies setting them up and maintaining them.

For example, the following file |draft.tex|
with a compilation flag |\version| as described in \secref{sec:flags}
compiles the main document as a draft:
%
\begin{center}
\begin{tabular}{l}
|\def\version{draft}|\\
|\input{childdoc.def}|\\
|\childdocforward{|\textit{main}|}|
\end{tabular}
\end{center}
%
Likewise, the following files |final|\textit{nn}|.tex|
compile the final version of the child document
|child|\textit{nn}|.tex|:
%
\begin{center}
\begin{tabular}{l}
|\def\version{final}|\\
|\input{childdoc.def}|\\
|\childdocforwardprefix{final}{child}|
\end{tabular}
\end{center}
%

Note that when several versions of a main file and/or of each child file
are to be generated, it may be convenient to set up a |Makefile| or
shell script to automatise the process.

%%%%%%%%%%%%%%%%%%%%%%%%%%%%%%%%%%%%%%%%%%%%%%%%%%%%%%%%%%%%%%%%%%%%%%%%%%%%%%%%
\subsection{Command Line Processing}
\label{sec:commandline}

The effect of redirection files can also be achieved by invoking
the \LaTeX{} compiler with a more elaborate command line.
Most conveniently this should be done as part
of a shell script or a |Makefile|.

When using \textsf{childdoc} in the main file, the following
command lines effectively perform a redirection
(note that depending on the shell being used,
backslashes may have to be doubled: `|\|' $\to$ `|\\|'):
%
\begin{center}
|... -jobname "|\textit{target}|" |\\|"|[\textit{flags}]%
|\input{childdoc.def}\childdocforward[|\textit{main}|]{|\textit{dest}|}"|
\end{center}
%
Here \textit{target} is the name of the output file,
\textit{main} is the name of the main file
and \textit{dest} is the name of the main or child file to be processed
(all filenames without extensions).
The optional argument \textit{main} can be omitted
if \textit{main} matches \textit{dest}.
Optionally, compilation \textit{flags} can be defined via |\def| commands.
This command line makes the \TeX{} engine believe
it is compiling the file \textit{target}
whose content is specified as the latter parameter.
The provided code then forwards the processing to
\textit{main} or \textit{dest} as described in \secref{sec:forward}.

%%%%%%%%%%%%%%%%%%%%%%%%%%%%%%%%%%%%%%%%%%%%%%%%%%%%%%%%%%%%%%%%%%%%%%%%%%%%%%%%
\subsection{Include by Input}
\label{sec:input}

Including child documents by |\include| has some restrictions by design.
Most notably, the content of a child document always occupies
its own set of pages; pages cannot be shared between child documents.
Usually, this behaviour makes perfect sense
because each child document contain an essential part of the document.
However, in some situations it may be desirable to compose
a document from a collection of parts
without having mandatory page breaks between then.
For this case, the package
provides a mechanism to include parts
by |\input| which can also be processed individually.
However, by construction this mechanism
requires manual handling of the content to be output.

%%%%%%%%%%%%%%%%%%%%%%%%%%%%%%%%%%%%%%%%
\DescribeMacro{\ifchilddocmanual}
The main file should be prepared as usual, see \secref{sec:include}.
However, the document body must make a distinction
between processing of an individual part and of the main document, e.g.:
%
\begin{center}
\begin{tabular}{l}
|\ifchilddocmanual|\\
|\input{\childdocname}|\\
|\||else|\\
\textit{document body with }|\input{|\textit{part}|}|\\
|\||fi|
\end{tabular}
\end{center}
%
The conditional |\ifchilddocmanual| is true whenever
a part to be included by |\input| is being compiled,
and the name of the part is stored in |\childdocname|.

%%%%%%%%%%%%%%%%%%%%%%%%%%%%%%%%%%%%%%%%
\DescribeMacro{\childdocby}
Each part to be included by |\input| should start with:
%
\begin{center}
\begin{tabular}{l}
|\input{childdoc.def}|\\
|\childdocby{|\textit{main}|}|\\
\end{tabular}
\end{center}
%
The directive |\childdocby| is similar to |\childdocof|
described in \secref{sec:include},
but the subsequent selection of content must be done manually.
To that end, both |\ifchilddoc| and |\ifchilddocmanual|
will be true upon processing of a part,
and the name of the part is stored in |\childdocname|.
Note that |\jobname| will be set to the filename of the current part
so that each part receives an individual |.aux| file
that does not interfere with the |.aux| file(s) of the main document.
This behaviour can be altered by the alternative form
|\childdocby[*]{|\textit{main}|}| (with a non-empty optional argument)
which uses the |.aux| file of the main document
by setting |\jobname| to \textit{main}.

%%%%%%%%%%%%%%%%%%%%%%%%%%%%%%%%%%%%%%%%%%%%%%%%%%%%%%%%%%%%%%%%%%%%%%%%%%%%%%%%
\subsection{Driver Development}
\label{sec:driver}

The \textsf{childdoc} mechanism can also be use for the development
of definition files such as \LaTeX{} styles or classes.
This case differs from the above setup with multiple parts
included by |\include| in that no |\includeonly| should be invoked.
This can be achieved by starting the include file
(before |\ProvidesPackage|) with:
%
\begin{center}
\begin{tabular}{l}
|\input{childdoc.def}|\\
|\childdocforward{|\textit{main}|}|\\
\end{tabular}
\end{center}
%
or alternatively with:
%
\begin{center}
\begin{tabular}{l}
|\input{childdoc.def}|\\
|\childdocby{|\textit{main}|}|\\
\end{tabular}
\end{center}
%
Both forms have slightly different effects as described above.
The main file is prepared as usual, see \secref{sec:include}.

%%%%%%%%%%%%%%%%%%%%%%%%%%%%%%%%%%%%%%%%%%%%%%%%%%%%%%%%%%%%%%%%%%%%%%%%%%%%%%%%
\subsection{Legacy Detection}
\label{sec:detection}

The directive |\childdocmain| in the main file can detect
whether the complete document or merely a child is to be compiled
even without using the directive |\childdocof|.
This method is deprecated because it is less robust
and there is no compelling reason to use it;
it is merely provided for backward compatibility
and it may be removed in future versions.

If the detection mechanism is to be used,
it is mandatory to correctly specify
the filename of the main file as the argument of |\childdocmain|:
%
\begin{center}
\begin{tabular}{l}
|\input{childdoc.def}|\\
|\childdocmain{|\textit{main}|}|\\
\end{tabular}
\end{center}
%
If |\jobname| does not match the argument \textit{main} of |\childdocmain|,
it is assumed that |\jobname| points to the child file to be compiled.
When using |\childdocmain| with the main file specified as argument,
it suffices to start a child file
with just |\input{|\textit{main}|}|
without loading of the package and using |\childdocof|.
If instead all processing is done
with the appropriate \textsf{childdoc} directives,
the argument of \textit{main} of |\childdocmain| can be empty.

An alternative version of the command line processing described
in \secref{sec:commandline} using the detection mechanism reads:
%
\begin{center}
|... -jobname "|\textit{target}|" "|[\textit{flags}]%
[|\def\jobname{|\textit{dest}|}|]|\input{|\textit{main}|}"|
\end{center}

%%%%%%%%%%%%%%%%%%%%%%%%%%%%%%%%%%%%%%%%%%%%%%%%%%%%%%%%%%%%%%%%%%%%%%%%%%%%%%%%
\subsection{Manual Code}
\label{sec:manual}

In case one cannot be certain whether the definitions file |childdoc.def|
is installed on the target \TeX{} distribution
and one prefers not to ship it,
it is conceivable to paste a few relevant commands into the sources.

To that end, drop all statements |\input{childdoc.def}|
and perform the replacements as outlined below.
Instead of |\childdocmain{|\textit{main}|}| add the following code
to the top of the main file:
%
\begin{center}
\begin{tabular}{l}
|\||ifdefined\childdocname\endinput\||fi\newif\ifchilddoc|\\
|\edef\childdocname{\scantokens\expandafter{\jobname\noexpand}}|\\
|\def\childdocmain{|\textit{main}|}\||ifx\childdocmain\childdocname\||else|\\
|\childdoctrue\includeonly{\childdocname}\let\jobname\childdocmain\||fi|\\
\end{tabular}
\end{center}
%
Instead of |\childdocof{|\textit{main}|}| just include the main file
at the top of each child file:
%
\begin{center}
|\input{|\textit{main}|}|
\end{center}
%
A simple redirection |\childdocforward{|\textit{dest}|}| is achieved by:
%
\begin{center}
|\def\jobname{|\textit{dest}|}\input{\jobname}|
\end{center}
%
The redirection with prefix
|\childdocforwardprefix[|\textit{prefix}|]{|\textit{dest}|}|
is accomplished by:
%
\begin{center}
\begin{tabular}{l}
|{\edef\jobname{\scantokens\expandafter{\jobname\noexpand}}|\\
|\def\redirectjob |\textit{prefix}|#1~~~{\gdef\jobname{|\textit{dest}|#1}}|\\
|\expandafter\redirectjob\jobname~~~}\input{\jobname}|
\end{tabular}
\end{center}

In an alternative approach,
child documents can be compiled by a specific command line
without additional code or specific definitions:
%
\begin{center}
|... -jobname "|\textit{target}|" "|[\textit{flags}]%
|\includeonly{|\textit{dest}|}\input{|\textit{main}|}"|
\end{center}
%

%%%%%%%%%%%%%%%%%%%%%%%%%%%%%%%%%%%%%%%%%%%%%%%%%%%%%%%%%%%%%%%%%%%%%%%%%%%%%%%%
%%%%%%%%%%%%%%%%%%%%%%%%%%%%%%%%%%%%%%%%%%%%%%%%%%%%%%%%%%%%%%%%%%%%%%%%%%%%%%%%
\section{Information}

%%%%%%%%%%%%%%%%%%%%%%%%%%%%%%%%%%%%%%%%%%%%%%%%%%%%%%%%%%%%%%%%%%%%%%%%%%%%%%%%
\subsection{Copyright}

Copyright \copyright{} 2017--2018 Niklas Beisert

This work may be distributed and/or modified under the
conditions of the \LaTeX{} Project Public License, either version 1.3
of this license or (at your option) any later version.
The latest version of this license is in
  \url{http://www.latex-project.org/lppl.txt}
and version 1.3 or later is part of all distributions of \LaTeX{}
version 2005/12/01 or later.

This work has the LPPL maintenance status `maintained'.

The Current Maintainer of this work is Niklas Beisert.

This work consists of the files |README.txt|, |childdoc.ins| and |childdoc.dtx|
as well as the derived files |childdoc.def|, |cdocsamp.tex|
with |cdocsch1.tex|, |cdocsch2.tex|, |cdocspt3.tex|, |cdocspt4.tex|,
|cdocsdrf.tex|, |cdocsfn1.tex|, |cdocsfn2.tex|
as well as |childdoc.pdf|.

%%%%%%%%%%%%%%%%%%%%%%%%%%%%%%%%%%%%%%%%%%%%%%%%%%%%%%%%%%%%%%%%%%%%%%%%%%%%%%%%
\subsection{Files and Installation}

The package consists of the files:
%
\begin{center}
\begin{tabular}{ll}
    |README.txt|   & readme file \\
    |childdoc.ins| & installation file \\
    |childdoc.dtx| & source file \\
    |childdoc.def| & definition file \\
    |cdocsamp.tex| & sample main file \\
    |cdocsch1.tex| & sample include file \\
    |cdocsch2.tex| & sample include file \\
    |cdocspt3.tex| & sample part file \\
    |cdocspt4.tex| & sample part file \\
    |cdocsdrf.tex| & sample redirection file \\
    |cdocsfn1.tex| & sample redirection file \\
    |cdocsfn2.tex| & sample redirection file \\
    |childdoc.pdf| & manual
\end{tabular}
\end{center}
%
The distribution consists of the files
|README.txt|, |childdoc.ins| and |childdoc.dtx|.
%
\begin{itemize}
\item
Run (pdf)\LaTeX{} on |childdoc.dtx|
to compile the manual |childdoc.pdf| (this file).
\item
Run \LaTeX{} on |childdoc.ins| to create the definitions file |childdoc.def|
and the sample |cdocsamp.tex| with include files
|cdocsch1.tex|, |cdocsch2.tex|, |cdocspt3.tex|, |cdocspt4.tex|,
|cdocsdrf.tex|, |cdocsfn1.tex|, |cdocsfn2.tex|.
Then copy the file |childdoc.def| to an appropriate directory of your \LaTeX{}
distribution, e.g.\ \textit{texmf-root}|/tex/latex/childdoc|.
\end{itemize}

%%%%%%%%%%%%%%%%%%%%%%%%%%%%%%%%%%%%%%%%%%%%%%%%%%%%%%%%%%%%%%%%%%%%%%%%%%%%%%%%
\subsection{Related CTAN Packages}

There are several other packages which offer a similar functionality:
%
\begin{itemize}
\item
The packages
\href{http://ctan.org/pkg/docmute}{\textsf{docmute}},
\href{http://ctan.org/pkg/includex}{\textsf{includex}} and
\href{http://ctan.org/pkg/standalone}{\textsf{standalone}}
provide commands to include only the document body of
a child file thus allowing both files to be compiled individually.
\item
The packages \href{http://ctan.org/pkg/subdocs}{\textsf{subdocs}}
and \href{http://ctan.org/pkg/subfiles}{\textsf{subfiles}}
provide structures in which the main and child documents can be
encapsulated and allowing them to be compiled individually.
The inclusion mechanism is different from the conventional |\include|.
\item
The package \href{http://ctan.org/pkg/combine}{\textsf{combine}}
is an elaborate solution to combine several documents into one.
\end{itemize}
%
See also the CTAN topic \href{http://ctan.org/topic/subdocs}{\textsf{subdocs}}
for further related packages.
The present package differs from the above solutions in that
a document structure constructed with the conventional |\include| mechanism
just needs two extra commands at the top of every file
such that all constituent files can be compiled individually.

%%%%%%%%%%%%%%%%%%%%%%%%%%%%%%%%%%%%%%%%%%%%%%%%%%%%%%%%%%%%%%%%%%%%%%%%%%%%%%%%
%\subsection{Feature Suggestions}
%
%The following is a list of features which may be useful for future
%versions of this package:
%%
%\begin{itemize}
%\item
%\ldots
%\end{itemize}

%%%%%%%%%%%%%%%%%%%%%%%%%%%%%%%%%%%%%%%%%%%%%%%%%%%%%%%%%%%%%%%%%%%%%%%%%%%%%%%%
\subsection{Revision History}

%%%%%%%%%%%%%%%%%%%%%%%%%%%%%%%%%%%%%%%%
\paragraph{v2.0:} 2018/12/30

\begin{itemize}
\item
immediate forward processing
\item
added |\childdocby| mechanism
\item
manual restructured
\end{itemize}

%%%%%%%%%%%%%%%%%%%%%%%%%%%%%%%%%%%%%%%%
\paragraph{v1.6:} 2018/01/17

\begin{itemize}
\item
application for development of include files
\item
corrections to manual
\end{itemize}

%%%%%%%%%%%%%%%%%%%%%%%%%%%%%%%%%%%%%%%%
\paragraph{v1.5:} 2017/05/21

\begin{itemize}
\item
more complete structuring introduced
\item
|\childdocof| introduced
\item
|\childdoc| renamed to |\childdocmain|
\item
|\childredirect| renamed to |\childdocforward| and |\childdocforwardprefix|
and functionality expanded
\end{itemize}

%%%%%%%%%%%%%%%%%%%%%%%%%%%%%%%%%%%%%%%%
\paragraph{v1.0:} 2017/04/27

\begin{itemize}
\item
manual and install package
\item
first version published on CTAN
\end{itemize}

%%%%%%%%%%%%%%%%%%%%%%%%%%%%%%%%%%%%%%%%
\paragraph{v0.6:} 2017/04/26

\begin{itemize}
\item
redirection mechanism added
\end{itemize}

%%%%%%%%%%%%%%%%%%%%%%%%%%%%%%%%%%%%%%%%
\paragraph{v0.5:} 2017/04/26

\begin{itemize}
\item
functionality in definition file
\end{itemize}


%%%%%%%%%%%%%%%%%%%%%%%%%%%%%%%%%%%%%%%%%%%%%%%%%%%%%%%%%%%%%%%%%%%%%%%%%%%%%%%%
%%%%%%%%%%%%%%%%%%%%%%%%%%%%%%%%%%%%%%%%%%%%%%%%%%%%%%%%%%%%%%%%%%%%%%%%%%%%%%%%
%%%%%%%%%%%%%%%%%%%%%%%%%%%%%%%%%%%%%%%%%%%%%%%%%%%%%%%%%%%%%%%%%%%%%%%%%%%%%%%%
\appendix

\settowidth\MacroIndent{\rmfamily\scriptsize 000\ }

 \DocInput{childdoc.dtx}

\end{document}
%</driver>
% \fi
%
% %%%%%%%%%%%%%%%%%%%%%%%%%%%%%%%%%%%%%%%%%%%%%%%%%%%%%%%%%%%%%%%%%%%%%%%%%%%%%%
% %%%%%%%%%%%%%%%%%%%%%%%%%%%%%%%%%%%%%%%%%%%%%%%%%%%%%%%%%%%%%%%%%%%%%%%%%%%%%%
% \section{Sample}
%\iffalse
%<*samplemain>
%\fi
%
% The following presents a sample document
% with two chapters, two parts, a title page,
% a compile flag as well as three forwarding files to set the flag.
% It consists of eight |.tex| files:
% \begin{center}
% \begin{tabular}{ll}
% |cdocsamp.tex|&main file\\
% |cdocsch1.tex|&include file for chapter 1\\
% |cdocsch2.tex|&include file for chapter 2\\
% |cdocspt3.tex|&include file for part 3\\
% |cdocspt4.tex|&include file for part 4\\
% |cdocsdrf.tex|&forwarding file for main file in draft mode\\
% |cdocsfi1.tex|&forwarding file for final version of chapter 1\\
% |cdocsfi2.tex|&forwarding file for final version of chapter 2\\
% \end{tabular}
% \end{center}
% Each of the eight files can be compiled directly by the \LaTeX{} compiler.
%
% %%%%%%%%%%%%%%%%%%%%%%%%%%%%%%%%%%%%%%
% \paragraph{Main File.}
%
% The main file is called |cdocsamp.tex|.
%
% Load the \textsf{childdoc} definitions and
% declare the filename for the main document:
%    \begin{macrocode}
\input{childdoc.def}
\childdocmain{}
%    \end{macrocode}

% Optional override for |\version| flag:
%    \begin{macrocode}
%%\ifchilddoc\else\providecommand{\version}{draft}\fi
%    \end{macrocode}

% Define the default values for the |\version| flag
% (|final| for the main file and |draft| for childs):
%    \begin{macrocode}
\ifchilddoc
\providecommand{\version}{draft}
\else
\providecommand{\version}{final}
\fi
%    \end{macrocode}

% Load the standard document class:
%    \begin{macrocode}
\documentclass[12pt]{article}
%    \end{macrocode}

% Start the document body:
%    \begin{macrocode}
\begin{document}
%    \end{macrocode}

% Declare a title page.
% Print title, part of document being processed and version flag:
%    \begin{macrocode}
\addtocounter{page}{-1}
\begin{center}
{\LARGE\bfseries{}childdoc example\par}
\vspace{1cm}
\ifchilddoc
\ifchilddocmanual part\else chapter\fi:
`\childdocname' of `\childdocjob'\par
\else
main document: `\childdocjob'\par
\fi
version: \version\par
\end{center}
\newpage
%    \end{macrocode}

% Manually include selected file,
% otherwise process as usual:
%    \begin{macrocode}
\ifchilddocmanual
\section*{part `\childdocname'}
\input{\childdocname}
\else
%    \end{macrocode}

% Include the two chapters:
%    \begin{macrocode}
\include{cdocsch1}
\include{cdocsch2}
%    \end{macrocode}

% Include the two parts unless only chapters should be displayed:
%    \begin{macrocode}
\ifchilddoc\else
\section{part three}
\input{cdocspt3}
\section{part four}
\input{cdocspt4}
\fi
%    \end{macrocode}

% Process as usual until here:
%    \begin{macrocode}
\fi
%    \end{macrocode}

% End of document body:
%    \begin{macrocode}
\end{document}
%    \end{macrocode}
%\iffalse
%</samplemain>
%\fi
%
% %%%%%%%%%%%%%%%%%%%%%%%%%%%%%%%%%%%%%%
% \paragraph{Chapter Include Files.}
%
% The include files are called |cdocsch1.tex| and |cdocsch2.tex|.
%
%\iffalse
%<*samplechap1|samplechap2>
%\fi

% Optional override for |\version| flag:
%    \begin{macrocode}
%%\providecommand{\version}{final}
%    \end{macrocode}

% Include the main document:
%    \begin{macrocode}
\input{childdoc.def}
\childdocof{cdocsamp}
%    \end{macrocode}

%\iffalse
%</samplechap1|samplechap2>
%\fi
%
%\iffalse
%<*samplechap1>
%\fi
% Some text for chapter 1:
%    \begin{macrocode}
\section{one}
some text in chapter one
%    \end{macrocode}

%\iffalse
%</samplechap1>
%\fi
% Some text for chapter 2:
%\iffalse
%<*samplechap2>
%\fi
%    \begin{macrocode}
\section{two}
more text in chapter two
%    \end{macrocode}

%\iffalse
%</samplechap2>
%\fi
%
% %%%%%%%%%%%%%%%%%%%%%%%%%%%%%%%%%%%%%%
% \paragraph{Part Include Files.}
%
% The include files are called |cdocspt3.tex| and |cdocspt4.tex|.
%
%\iffalse
%<*samplepart3|samplepart4>
%\fi

% Optional override for |\version| flag:
%    \begin{macrocode}
%%\providecommand{\version}{final}
%    \end{macrocode}

% Include the main document:
%    \begin{macrocode}
\input{childdoc.def}
\childdocby{cdocsamp}
%    \end{macrocode}

%\iffalse
%</samplepart3|samplepart4>
%\fi
%
%\iffalse
%<*samplepart3>
%\fi
% Some text for part 3:
%    \begin{macrocode}
some text in part three
%    \end{macrocode}

%\iffalse
%</samplepart3>
%\fi
% Some text for part 4:
%\iffalse
%<*samplepart4>
%\fi
%    \begin{macrocode}
more text in part four
%    \end{macrocode}

%\iffalse
%</samplepart4>
%\fi
%
% %%%%%%%%%%%%%%%%%%%%%%%%%%%%%%%%%%%%%%
% \paragraph{Forwarding for a Complete Draft.}
%
% The following forwarding file |cdocsdrf.tex|
% compiles the main document in draft mode:
%\iffalse
%<*sampledraft>
%\fi
%    \begin{macrocode}
\def\version{draft}
\input{childdoc.def}
\childdocforward{cdocsamp}
%    \end{macrocode}

%\iffalse
%</sampledraft>
%\fi
%
% %%%%%%%%%%%%%%%%%%%%%%%%%%%%%%%%%%%%%%
% \paragraph{Forwarding for Final Version of the Chapters.}
%
% The following forwarding files |cdocsfn1.tex| and |cdocsfn2.tex|
% (with identical content)
% compile the final versions of the child documents
% |cdocsch1.tex| and |cdocsch2.tex|, respectively:
%\iffalse
%<*samplefinal>
%\fi
%    \begin{macrocode}
\def\version{final}
\input{childdoc.def}
\childdocforwardprefix[cdocsamp]{cdocsfn}{cdocsch}
%    \end{macrocode}

%\iffalse
%</samplefinal>
%\fi
%
% %%%%%%%%%%%%%%%%%%%%%%%%%%%%%%%%%%%%%%
% \paragraph{Command Line Processing.}
%
% The following three command lines generate the output files
% |cdocscld|, |cdocscl1| and |cdocscl2|
% which should be identical to
% |cdocsdrf|, |cdocsch1| and |cdocsfn2|, respectively:
% \begin{center}
% \begin{tabular}{l}
% |latex -jobname cdocscld \|\\
% |  "\def\version{draft}\input{childdoc.def}\childdocforward{cdocsamp}"|\\
% |latex -jobname cdocscl1 \|\\
% |  "\input{childdoc.def}\childdocforward[cdocsamp]{cdocsch1}"|\\
% |latex -jobname cdocscl2 \|\\
% |  "\def\version{final}\input{childdoc.def}\childdocforward{cdocsch2}"|
% \end{tabular}
% \end{center}
% Note that the trailing backslash on each first line
% merely continues the input to the second line
% (for convenient cut ant paste).
% Furthermore, the command |latex| can be replaced by any
% of its alternative versions such as |pdflatex|.
%
% %%%%%%%%%%%%%%%%%%%%%%%%%%%%%%%%%%%%%%%%%%%%%%%%%%%%%%%%%%%%%%%%%%%%%%%%%%%%%%
% %%%%%%%%%%%%%%%%%%%%%%%%%%%%%%%%%%%%%%%%%%%%%%%%%%%%%%%%%%%%%%%%%%%%%%%%%%%%%%
% \section{Implementation}
%\iffalse
%<*package>
%\fi
%
% This section describes the definitions file |childdoc.def|.

% The definitions cannot be loaded using |\usepackage| or |\RequirePackage|
% which has a mechanism to prevent loading a style file more than once.
% When loading the definitions by means of |\input|
% multiple instances have to be prevented manually:
%\iffalse
%This code needs to be before the `\ProvidesFile' directive
%which is defined at the beginning of this file.
%Therefore it is also placed there and commented out here.
%</package>
%<*discard>
%\fi
%    \begin{macrocode}
\ifdefined\childdocmain\endinput\fi
%    \end{macrocode}
%\iffalse
%</discard>
%<*package>
%\fi
%
% \macro{\ifchilddoc}
% \macro{\ifchilddocmanual}
% The conditional |\ifchilddoc| tells whether a
% child (true) or main (false) document is being compiled.
% The conditional |\ifchilddocmanual| tells whether
% the |\includeonly| mechanism is used (false) or
% the selection of child files must be performed manually (true).
% The definitions initialise to false:
%    \begin{macrocode}
\newif\ifchilddoc
\newif\ifchilddocmanual
%    \end{macrocode}

% \macro{\childdocname}
% \macro{\childdocjob}
% The macro |\childdocname| stores the name of the main document
% to be compiled. The macro |\childdocjob| stores the name of
% the document on which the \LaTeX{} compiler was originally invoked.
% The content of |\jobname| cannot be compared
% to filenames specified in the source due to different catcodes.
% The following code rescans |\jobname|, stores the result
% in |\childdocname| and saves a copy in |\childdocjob|:
%    \begin{macrocode}
\edef\childdocname{\scantokens\expandafter{\jobname\noexpand}}
\let\childdocjob\childdocname
%    \end{macrocode}

% \macro{\childdocdisable}
% The macro |\childdocdisable| prevents the main file
% from being processed more than once.
% At this stage, the main document command |\childdocmain|
% is assumed to be called once again where it should do nothing.
% Any subsequent call to it should prevent
% a secondary processing of the main document
% It overwrites the forwarding commands
% |\childdocof| and |\childdocforward|
% with empty macros to prevent further inclusions of the main document:
%    \begin{macrocode}
\newcommand{\childdocdisable}
{
  \renewcommand{\childdocmain}[1]{\renewcommand{\childdocmain}[1]{\endinput}}
  \renewcommand{\childdocof}[1]{}
  \renewcommand{\childdocby}[2][]{}
  \renewcommand{\childdocforward}[2][]{}
  \renewcommand{\childdocdisable}{}
}
%    \end{macrocode}

% \macro{\childdocmain}
% The macro |\childdocmain| is to be called at the top of the main file
% with nothing or the main filename (without extension) as argument.
% First, it breaks loops.
% If the argument is not empty and does not match |\childdocname|
% (which is set by the first inclusion of |childdoc.def|),
% |\ifchilddoc| is set to true, |\includeonly| is applied to the child file
% and |\jobname| is set to the main file
% (for proper handling of |.aux| files):
%    \begin{macrocode}
\newcommand{\childdocmain}[1]
{
  \childdocdisable\childdocmain{}
  \if?#1?\else
    \begingroup
      \def\childdoctmp{#1}
      \ifx\childdoctmp\childdocname
        \def\childdoctmp{}
      \else
        \def\childdoctmp
        {
          \childdoctrue
          \includeonly{\childdocname}
          \def\childdocjob{#1}
          \def\jobname{#1}
        }
      \fi
      \expandafter
    \endgroup
    \childdoctmp
  \fi
}
%    \end{macrocode}

% \macro{\childdocof}
% The command |\childdocof| redirects
% compilation to the main file |#1|.
%    \begin{macrocode}
\newcommand{\childdocof}[1]
{
  \childdocdisable
  \childdoctrue
  \includeonly{\childdocname}
  \def\jobname{#1}
  \def\childdocjob{#1}
  \input{#1}
}
%    \end{macrocode}

% \macro{\childdocby}
% The command |\childdocby| ....
%    \begin{macrocode}
\newcommand{\childdocby}[2][]
{
  \childdocdisable
  \childdoctrue
  \childdocmanualtrue
  \if?#1?\else
    \def\jobname{#2}
  \fi
  \def\childdocjob{#2}
  \input{#2}
  \endinput
}
%    \end{macrocode}

% \macro{\childdocforward}
% The command |\childdocforward| redirects
% compilation to the main file or
% (if the optional argument is given) a child file.
% Parameters are set as if the main file
% or a child file starting with |\childdocof| was compiled.
% Then compilation is handed over to the main file:
%    \begin{macrocode}
\newcommand{\childdocforward}[2][]
{
  \begingroup
    \if?#1?
      \def\childdoctmp
      {
        \def\childdocname{#2}
        \def\childdocjob{#2}
        \def\jobname{#2}
        \input{#2}
        \endinput
      }
    \else
      \def\childdoctmp
      {
        \childdocdisable
        \def\childdocname{#2}
        \childdoctrue
        \includeonly{#2}
        \def\childdocjob{#1}
        \def\jobname{#1}
        \input{#1}
        \endinput
      }
    \fi
    \expandafter
  \endgroup
  \childdoctmp
}
%    \end{macrocode}

% \macro{\childdocforwardprefix}
% The command |\childdocforwardprefix| redirects
% compilation to the main or a child file by means of a pattern.
% The prefix |#1| in the current filename is replaced by |#2|
% and the suffix of the current filename is kept
% (it is assumed that the filename does not contain the substring `|~~~|'
% which is used as a delimiter).
% Compilation is handed over to the new file by |\childdocforward|:
%    \begin{macrocode}
\newcommand{\childdocforwardprefix}[3][]
{
  \begingroup
    \def\childdocextract #2##1~~~{\def\childdoctmp{\childdocforward[#1]{#3##1}}}
    \expandafter\childdocextract\childdocname~~~
    \expandafter
  \endgroup
  \childdoctmp
}
%    \end{macrocode}

% \macro{\childdoc}
% The deprecated macro |\childdoc| is a legacy version of |\childdocmain|:
%    \begin{macrocode}
\newcommand{\childdoc}{\childdocmain}
%    \end{macrocode}

% \macro{\childdocredirect}
% The deprecated macro |\childdocredirect| is a legacy version
% of |\childdocforward| and |\childdocforwardprefix|:
%    \begin{macrocode}
\newcommand{\childdocredirect}[2][]
{
  \begingroup
    \if?#1?
      \def\childdoctmp{\childdocforward{#2}}
    \else
      \def\childdoctmp{\childdocforwardprefix{#1}{#2}}
    \fi
    \expandafter
  \endgroup
  \childdoctmp
}
%    \end{macrocode}

%\iffalse
%</package>
%\fi
%
\endinput

\childdocmain{}
%    \end{macrocode}

% Optional override for |\version| flag:
%    \begin{macrocode}
%%\ifchilddoc\else\providecommand{\version}{draft}\fi
%    \end{macrocode}

% Define the default values for the |\version| flag
% (|final| for the main file and |draft| for childs):
%    \begin{macrocode}
\ifchilddoc
\providecommand{\version}{draft}
\else
\providecommand{\version}{final}
\fi
%    \end{macrocode}

% Load the standard document class:
%    \begin{macrocode}
\documentclass[12pt]{article}
%    \end{macrocode}

% Start the document body:
%    \begin{macrocode}
\begin{document}
%    \end{macrocode}

% Declare a title page.
% Print title, part of document being processed and version flag:
%    \begin{macrocode}
\addtocounter{page}{-1}
\begin{center}
{\LARGE\bfseries{}childdoc example\par}
\vspace{1cm}
\ifchilddoc
\ifchilddocmanual part\else chapter\fi:
`\childdocname' of `\childdocjob'\par
\else
main document: `\childdocjob'\par
\fi
version: \version\par
\end{center}
\newpage
%    \end{macrocode}

% Manually include selected file,
% otherwise process as usual:
%    \begin{macrocode}
\ifchilddocmanual
\section*{part `\childdocname'}
\input{\childdocname}
\else
%    \end{macrocode}

% Include the two chapters:
%    \begin{macrocode}
\include{cdocsch1}
\include{cdocsch2}
%    \end{macrocode}

% Include the two parts unless only chapters should be displayed:
%    \begin{macrocode}
\ifchilddoc\else
\section{part three}
\input{cdocspt3}
\section{part four}
\input{cdocspt4}
\fi
%    \end{macrocode}

% Process as usual until here:
%    \begin{macrocode}
\fi
%    \end{macrocode}

% End of document body:
%    \begin{macrocode}
\end{document}
%    \end{macrocode}
%\iffalse
%</samplemain>
%\fi
%
% %%%%%%%%%%%%%%%%%%%%%%%%%%%%%%%%%%%%%%
% \paragraph{Chapter Include Files.}
%
% The include files are called |cdocsch1.tex| and |cdocsch2.tex|.
%
%\iffalse
%<*samplechap1|samplechap2>
%\fi

% Optional override for |\version| flag:
%    \begin{macrocode}
%%\providecommand{\version}{final}
%    \end{macrocode}

% Include the main document:
%    \begin{macrocode}
% \iffalse
%
% childdoc.dtx Copyright (C) 2017-2018 Niklas Beisert
%
% This work may be distributed and/or modified under the
% conditions of the LaTeX Project Public License, either version 1.3
% of this license or (at your option) any later version.
% The latest version of this license is in
%   http://www.latex-project.org/lppl.txt
% and version 1.3 or later is part of all distributions of LaTeX
% version 2005/12/01 or later.
%
% This work has the LPPL maintenance status `maintained'.
%
% The Current Maintainer of this work is Niklas Beisert.
%
% This work consists of the files childdoc.dtx and childdoc.ins
% and the derived files childdoc.def and cdocsamp.tex with
% cdocsch1.tex, cdocsch2.tex, cdocsdrf.tex, cdocsfn1.tex, cdocsfn2.tex.
%
%<package>\ifdefined\childdocmain\endinput\fi
%<package>\ProvidesFile{childdoc.def}[2018/12/30 v2.0 child document driver]
%<samplemain>\ProvidesFile{cdocsamp.tex}[2018/12/30 v2.0 sample for childdoc]
%<*driver>
%\ProvidesFile{childdoc.drv}[2018/12/30 v2.0 childdoc reference manual file]
\PassOptionsToClass{10pt,a4paper}{article}
\documentclass{ltxdoc}

\usepackage[margin=35mm]{geometry}
\usepackage{hyperref}
\usepackage{hyperxmp}
\usepackage[usenames]{color}

\hypersetup{colorlinks=true}
\hypersetup{pdfstartview=FitH}
\hypersetup{pdfpagemode=UseNone}
\hypersetup{pdfsource={}}
\hypersetup{pdflang={en-UK}}
\hypersetup{pdfcopyright={Copyright 2017-2018 Niklas Beisert.
  This work may be distributed and/or modified under the
  conditions of the LaTeX Project Public License, either version 1.3
  of this license or (at your option) any later version.}}
\hypersetup{pdflicenseurl={http://www.latex-project.org/lppl.txt}}
\hypersetup{pdfcontactaddress={ETH Zurich, ITP, HIT K,
  Wolfgang-Pauli-Strasse 27}}
\hypersetup{pdfcontactpostcode={8093}}
\hypersetup{pdfcontactcity={Zurich}}
\hypersetup{pdfcontactcountry={Switzerland}}
\hypersetup{pdfcontactemail={nbeisert@itp.phys.ethz.ch}}
\hypersetup{pdfcontacturl={http://people.phys.ethz.ch/\xmptilde nbeisert/}}

\newcommand{\secref}[1]{\hyperref[#1]{section \ref*{#1}}}

\parskip1ex
\parindent0pt
\let\olditemize\itemize
\def\itemize{\olditemize\parskip0pt}

\begin{document}

\title{The \textsf{childdoc} Package}
\hypersetup{pdftitle={The childdoc Package}}
\author{Niklas Beisert\\[2ex]
  Institut f\"ur Theoretische Physik\\
  Eidgen\"ossische Technische Hochschule Z\"urich\\
  Wolfgang-Pauli-Strasse 27, 8093 Z\"urich, Switzerland\\[1ex]
  \href{mailto:nbeisert@itp.phys.ethz.ch}
  {\texttt{nbeisert@itp.phys.ethz.ch}}}
\hypersetup{pdfauthor={Niklas Beisert}}
\hypersetup{pdfsubject={Manual for the LaTeX2e Package childdoc}}
\date{30 December 2018, \textsf{v2.0}}
\maketitle

\begin{abstract}\noindent
\textsf{childdoc} is a \LaTeXe{} package
that enables the direct compilation
of document sections included by |\include|
to individual files.
\end{abstract}

\begingroup
\parskip0ex
\tableofcontents
\endgroup

%%%%%%%%%%%%%%%%%%%%%%%%%%%%%%%%%%%%%%%%%%%%%%%%%%%%%%%%%%%%%%%%%%%%%%%%%%%%%%%%
%%%%%%%%%%%%%%%%%%%%%%%%%%%%%%%%%%%%%%%%%%%%%%%%%%%%%%%%%%%%%%%%%%%%%%%%%%%%%%%%
\section{Introduction}

\LaTeX{} provides a mechanism to structure a large document (such as a book)
into a main file and several child files (containing the chapters)
using the |\include| command.
This mechanism is beneficial for documents
which span hundreds of pages in order to
make the source file(s) more manageable.
Moreover, compilation can be restricted to
selected child files by means of the |\includeonly| command.
The latter feature can be used to reduce the compilation time while editing
(this was significantly more useful in the earlier days of \LaTeX{})
or to generate a smaller document which is easier to navigate.
Another application of |\includeonly| is to generate
documents consisting of selected parts of the complete document.

However, there are a few drawbacks of the plain |\include| mechanism:
\begin{itemize}
\item
The child files cannot be compiled on their own,
they can only be compiled via the main file.
A naive editing environment
(such as a text editor with an option
to have the current file processed by \LaTeX)
may require one to switch to the main file before compiling;
attempting to compile the child file produces errors.
\item
The main file must be modified (each time)
to adjust the |\includeonly| command
to the present needs. This easily leaves the main file in a messy state.
\item
The generated document will always carry the filename
of the main document. This is inconvenient if
several child files are to be compiled and
to be kept for distribution.
\end{itemize}

The present package provides a simple interface
to make child files individually compilable by \LaTeX{}.
Compiling a child file then has the same effect as compiling
the main file with an |\includeonly| command
to select the appropriate child.
Moreover the generated document will carry the name of the child
rather than the main file.
This resolves all three above issues.

This feature is meant to make the editing of books,
thesis documents and lecture notes somewhat more convenient.
However, the package can also be used efficiently for
composing a series of documents (such as exercise sheets)
which are typically distributed individually.
It then assists the author in generating the individual documents
(potentially in different versions)
as well as a document containing the collected series.
Another application is in developing style files
or other kinds of included material
where compilation of the style file could redirect
to a sample or test file.

%%%%%%%%%%%%%%%%%%%%%%%%%%%%%%%%%%%%%%%%%%%%%%%%%%%%%%%%%%%%%%%%%%%%%%%%%%%%%%%%
%%%%%%%%%%%%%%%%%%%%%%%%%%%%%%%%%%%%%%%%%%%%%%%%%%%%%%%%%%%%%%%%%%%%%%%%%%%%%%%%
\section{Usage}

First of all, the package \textsf{childdoc} is \emph{not} a standard
\LaTeXe{} |.sty| style file! Therefore it needs to be invoked in
a non-standard way.

%%%%%%%%%%%%%%%%%%%%%%%%%%%%%%%%%%%%%%%%%%%%%%%%%%%%%%%%%%%%%%%%%%%%%%%%%%%%%%%%
\subsection{Included Files}
\label{sec:include}

%%%%%%%%%%%%%%%%%%%%%%%%%%%%%%%%%%%%%%%%
\DescribeMacro{\childdocmain}
To use the package, add the commands
\begin{center}
\begin{tabular}{l}
|\input{childdoc.def}|\\
|\childdocmain{}|\\
\end{tabular}
\end{center}
at the very top of the main \LaTeX{} file,
in particular \emph{before} the |\documentclass| statement!
The argument of |\childdocmain| should be left empty
(but it must be present).

%%%%%%%%%%%%%%%%%%%%%%%%%%%%%%%%%%%%%%%%
\DescribeMacro{\childdocof}
Furthermore, add the commands
\begin{center}
\begin{tabular}{l}
|\input{childdoc.def}|\\
|\childdocof{|\textit{main}|}|\\
\end{tabular}
\end{center}
at the top of every child file \textit{child}
which is included by |\include{|\textit{child}|}|
from within the main file
(or at least for those files to be compiled individually).
The argument \textit{main} must be the filename of the main file.

There are a couple of
considerations in setting up the main and child documents:

%%%%%%%%%%%%%%%%%%%%%%%%%%%%%%%%%%%%%%%%
\paragraph{Restrictions.}

Please note the following restrictions:
\begin{itemize}
\item
|\childdocmain| must be called with one argument \textit{main}
to ensure compatibility with earlier version of the package.
It must either be empty (|\childdocmain{}|)
or precisely match the filename of the main file in which it is specified.
See \secref{sec:detection} for further information.
\item
The filename \textit{main} must be specified without the |.tex| extension.
\item
The filename \textit{main} is case sensitive
(even in case-insensitive file systems)
due to internal string comparison.
\item
The argument \textit{main} should be fully expanded, it cannot be a macro.
\item
Subdirectories and special characters should be avoided in filenames.
\item
The command |\childdocmain{|\textit{main}|}| must be followed by a whitespace.
It should not be followed immediately by another command
or by a comment mark `|%|'.
This is because the \TeX{} parser reads the token immediately following
the argument of |\childdocmain| and puts it
at the beginning of every child section;
however, a white\-space is ignored.
\end{itemize}

%%%%%%%%%%%%%%%%%%%%%%%%%%%%%%%%%%%%%%%%
\paragraph{Content of Main File.}

It is advisable to place all content in the child files included by |\include|.
Any output contained in the main file will appear in all child documents
unless suppressed manually;
it cannot be suppressed automatically by the |\includeonly| directive
and thus should normally be avoided.
A method to include some content in the main file
by means of conditional processing is described in \secref{sec:conditional}.

%%%%%%%%%%%%%%%%%%%%%%%%%%%%%%%%%%%%%%%%
\paragraph{Page Numbering.}

When only a part of the document is compiled,
the appropriate numbering of pages
(as well as other status parameters)
is determined from the |.aux| files.
The latter contain information from previous passes.
However this information needs to propagate through
all intermediate child documents.
Therefore the page numbering in child documents may well
be inconsistent until the complete document is compiled at least once.

A useful (if unconventional) way to always ensure a consistent
page numbering is to restart the numbering in each child document
and denote the pages by `\textit{child}|.|\textit{page}'
where \textit{child} represents the chapter/section number of the child file.
This can be achieved by the command
|\numberwithin{page}{|\textit{child}|}|
of the \textsf{amsmath} package
where \textit{child} can be |chapter| or |section|
depending on the chosen structuring.
Alternatively, one can modify the macro |\thepage| appropriately
and reset the counter |page| at the start of each child file.

%%%%%%%%%%%%%%%%%%%%%%%%%%%%%%%%%%%%%%%%%%%%%%%%%%%%%%%%%%%%%%%%%%%%%%%%%%%%%%%%
\subsection{Conditional Processing}
\label{sec:conditional}

The package provides a mechanism to compile different versions
of a document. To customise the versions further some conditional processing
can come in handy to distinguish which version is being compiled.
The package provides two macros to describe the compilation context:

%%%%%%%%%%%%%%%%%%%%%%%%%%%%%%%%%%%%%%%%
\DescribeMacro{\ifchilddoc}
The conditional |\ifchilddoc| distinguishes between the compilation of
child documents and the main document:
%
\begin{center}
|\ifchilddoc |\textit{child-code}| |[|\||else |\textit{main-code}]| \||fi|
\end{center}

%%%%%%%%%%%%%%%%%%%%%%%%%%%%%%%%%%%%%%%%
\DescribeMacro{\childdocname}
\DescribeMacro{\childdocjob}
The macro |\childdocname| contains the filename (without extension)
of the main or child file being processed.
Note that |\childdocjob| will always contain the name of the main file.

%%%%%%%%%%%%%%%%%%%%%%%%%%%%%%%%%%%%%%%%
\paragraph{Title Page.}

Conditional processing can be used to include a title or banner page
in the main document when proper precautions are taken.
Importantly, the code in the main file should ensure that the page counter
(as well as other status parameters which are stored in the |.aux| files)
takes the same value after the conditional processing.
Otherwise the page numbers may take divergent values
depending on which part is compiled.

For example, a title page could be declared by:
%
\begin{center}
\begin{tabular}{l}
|\ifchilddoc\||else|\\
|\addtocounter{page}{-1}|\\
\textit{code for title page}\\
|\newpage|\\
|\||fi|
\end{tabular}
\end{center}
%
A banner page for the child documents can be generated by:
%
\begin{center}
\begin{tabular}{l}
|\ifchilddoc|\\
|\addtocounter{page}{-1}|\\
\textit{code for banner page}\\
|\newpage|\\
|\||fi|
\end{tabular}
\end{center}
%
Here one could write a message such as:
\begin{center}
|This is the part \childdocname{} of \childdocjob{}.|
\end{center}

%%%%%%%%%%%%%%%%%%%%%%%%%%%%%%%%%%%%%%%%%%%%%%%%%%%%%%%%%%%%%%%%%%%%%%%%%%%%%%%%
\subsection{Flags}
\label{sec:flags}

The package makes it easy to generate different versions
of the main or child documents.
To this end compilation flags can be defined
and assigned different default values.
They will be particularly useful in conjunction
with the forwarding mechanism described in \secref{sec:forward}.

For example, it may be useful to have a flag |\version|
which can be set to |draft| or |final|.
The document source will contain some conditional code
depending on the value of |\version|.
Suppose further, the flag should default to |final| for the main file
and to |draft| for child files
which is a natural assignment for editing the document.
This is achieved by placing the following code
in the preamble of the main document
(below the |\childdocmain| directive):
%
\begin{center}
\begin{tabular}{l}
|\ifchilddoc|\\
|\providecommand{\version}{draft}|\\
|\||else|\\
|\providecommand{\version}{final}|\\
|\||fi|
\end{tabular}
\end{center}
%
The definition by |\providecommand| makes sure
that previous definitions are not overwritten.
Further statements |\providecommand{\version}{...}|
can thus be added before the above code to override it.

For the main file, one might add a line
(between |\childdocmain| and the above block)
%
\begin{center}
|%\ifchilddoc\||else\providecommand{\version}{draft}\||fi|
\end{center}
%
which can be uncommented to produce a draft version.
Likewise one can add a line to the very top of a child file
(above the |\childdocof{|\textit{main}|}| directive)
%
\begin{center}
|%\providecommand{\version}{final}|
\end{center}
%
which can be uncommented to produce the final version of this child document.

%%%%%%%%%%%%%%%%%%%%%%%%%%%%%%%%%%%%%%%%%%%%%%%%%%%%%%%%%%%%%%%%%%%%%%%%%%%%%%%%
\subsection{Forwarding}
\label{sec:forward}

Different versions of the main or child documents
using compilation flags as described in \secref{sec:flags}
can be (permanently) stored in different files
for convenient compilation, viewing and distribution.
To this end, the package defines a command
to pass on compilation to a different file:

%%%%%%%%%%%%%%%%%%%%%%%%%%%%%%%%%%%%%%%%
\DescribeMacro{\childdocforward}
The command |\childdocforward| redirects processing to
another source file:
%
\begin{center}
\begin{tabular}{l}
|\input{childdoc.def}|\\
|\childdocforward[|\textit{main}|]{|\textit{dest}|}|\\
\end{tabular}
\end{center}
%
The argument \textit{dest} is the destination file
(without extension).
It should be the main file or one of the child files.
Note that further \textsf{childdoc} directives
such as |\childdocof| and |\childdocforward|
in the indicated file will be processed in this form.
The optional argument \textit{main}
passes on directly to the main file \textit{main}
while pretending to compile the child \textit{dest}.
This form behaves as if \textit{dest}
issues |\childdocof{|\textit{main}|}| right away,
and no further \textsf{childdoc} directives will be processed.

%%%%%%%%%%%%%%%%%%%%%%%%%%%%%%%%%%%%%%%%
\DescribeMacro{\...prefix}
In the alternative form |\childdocforwardprefix|,
%
\begin{center}
\begin{tabular}{l}
|\input{childdoc.def}|\\
|\childdocforwardprefix[|\textit{main}|]{|\textit{prefix}|}{|\textit{dest}|}|
\end{tabular}
\end{center}
%
the destination file is determined by a pattern
depending on the current file:
To make this work, the current file must be called
`{\textit{prefix}\hspace{0.2em}\textit{suffix}}'
with \textit{prefix} matching precisely the argument.
Processing is then passed on to the file
`{\textit{dest}\hspace{0.2em}\textit{suffix}}'.
Surely, the same effect is achieved by
directly specifying the
argument `{\textit{dest}\hspace{0.2em}\textit{suffix}}'
in the first form.
However, that requires to set up a different file
for each child. With the alternative form of the command
all these files can have exactly the same content
which simplifies setting them up and maintaining them.

For example, the following file |draft.tex|
with a compilation flag |\version| as described in \secref{sec:flags}
compiles the main document as a draft:
%
\begin{center}
\begin{tabular}{l}
|\def\version{draft}|\\
|\input{childdoc.def}|\\
|\childdocforward{|\textit{main}|}|
\end{tabular}
\end{center}
%
Likewise, the following files |final|\textit{nn}|.tex|
compile the final version of the child document
|child|\textit{nn}|.tex|:
%
\begin{center}
\begin{tabular}{l}
|\def\version{final}|\\
|\input{childdoc.def}|\\
|\childdocforwardprefix{final}{child}|
\end{tabular}
\end{center}
%

Note that when several versions of a main file and/or of each child file
are to be generated, it may be convenient to set up a |Makefile| or
shell script to automatise the process.

%%%%%%%%%%%%%%%%%%%%%%%%%%%%%%%%%%%%%%%%%%%%%%%%%%%%%%%%%%%%%%%%%%%%%%%%%%%%%%%%
\subsection{Command Line Processing}
\label{sec:commandline}

The effect of redirection files can also be achieved by invoking
the \LaTeX{} compiler with a more elaborate command line.
Most conveniently this should be done as part
of a shell script or a |Makefile|.

When using \textsf{childdoc} in the main file, the following
command lines effectively perform a redirection
(note that depending on the shell being used,
backslashes may have to be doubled: `|\|' $\to$ `|\\|'):
%
\begin{center}
|... -jobname "|\textit{target}|" |\\|"|[\textit{flags}]%
|\input{childdoc.def}\childdocforward[|\textit{main}|]{|\textit{dest}|}"|
\end{center}
%
Here \textit{target} is the name of the output file,
\textit{main} is the name of the main file
and \textit{dest} is the name of the main or child file to be processed
(all filenames without extensions).
The optional argument \textit{main} can be omitted
if \textit{main} matches \textit{dest}.
Optionally, compilation \textit{flags} can be defined via |\def| commands.
This command line makes the \TeX{} engine believe
it is compiling the file \textit{target}
whose content is specified as the latter parameter.
The provided code then forwards the processing to
\textit{main} or \textit{dest} as described in \secref{sec:forward}.

%%%%%%%%%%%%%%%%%%%%%%%%%%%%%%%%%%%%%%%%%%%%%%%%%%%%%%%%%%%%%%%%%%%%%%%%%%%%%%%%
\subsection{Include by Input}
\label{sec:input}

Including child documents by |\include| has some restrictions by design.
Most notably, the content of a child document always occupies
its own set of pages; pages cannot be shared between child documents.
Usually, this behaviour makes perfect sense
because each child document contain an essential part of the document.
However, in some situations it may be desirable to compose
a document from a collection of parts
without having mandatory page breaks between then.
For this case, the package
provides a mechanism to include parts
by |\input| which can also be processed individually.
However, by construction this mechanism
requires manual handling of the content to be output.

%%%%%%%%%%%%%%%%%%%%%%%%%%%%%%%%%%%%%%%%
\DescribeMacro{\ifchilddocmanual}
The main file should be prepared as usual, see \secref{sec:include}.
However, the document body must make a distinction
between processing of an individual part and of the main document, e.g.:
%
\begin{center}
\begin{tabular}{l}
|\ifchilddocmanual|\\
|\input{\childdocname}|\\
|\||else|\\
\textit{document body with }|\input{|\textit{part}|}|\\
|\||fi|
\end{tabular}
\end{center}
%
The conditional |\ifchilddocmanual| is true whenever
a part to be included by |\input| is being compiled,
and the name of the part is stored in |\childdocname|.

%%%%%%%%%%%%%%%%%%%%%%%%%%%%%%%%%%%%%%%%
\DescribeMacro{\childdocby}
Each part to be included by |\input| should start with:
%
\begin{center}
\begin{tabular}{l}
|\input{childdoc.def}|\\
|\childdocby{|\textit{main}|}|\\
\end{tabular}
\end{center}
%
The directive |\childdocby| is similar to |\childdocof|
described in \secref{sec:include},
but the subsequent selection of content must be done manually.
To that end, both |\ifchilddoc| and |\ifchilddocmanual|
will be true upon processing of a part,
and the name of the part is stored in |\childdocname|.
Note that |\jobname| will be set to the filename of the current part
so that each part receives an individual |.aux| file
that does not interfere with the |.aux| file(s) of the main document.
This behaviour can be altered by the alternative form
|\childdocby[*]{|\textit{main}|}| (with a non-empty optional argument)
which uses the |.aux| file of the main document
by setting |\jobname| to \textit{main}.

%%%%%%%%%%%%%%%%%%%%%%%%%%%%%%%%%%%%%%%%%%%%%%%%%%%%%%%%%%%%%%%%%%%%%%%%%%%%%%%%
\subsection{Driver Development}
\label{sec:driver}

The \textsf{childdoc} mechanism can also be use for the development
of definition files such as \LaTeX{} styles or classes.
This case differs from the above setup with multiple parts
included by |\include| in that no |\includeonly| should be invoked.
This can be achieved by starting the include file
(before |\ProvidesPackage|) with:
%
\begin{center}
\begin{tabular}{l}
|\input{childdoc.def}|\\
|\childdocforward{|\textit{main}|}|\\
\end{tabular}
\end{center}
%
or alternatively with:
%
\begin{center}
\begin{tabular}{l}
|\input{childdoc.def}|\\
|\childdocby{|\textit{main}|}|\\
\end{tabular}
\end{center}
%
Both forms have slightly different effects as described above.
The main file is prepared as usual, see \secref{sec:include}.

%%%%%%%%%%%%%%%%%%%%%%%%%%%%%%%%%%%%%%%%%%%%%%%%%%%%%%%%%%%%%%%%%%%%%%%%%%%%%%%%
\subsection{Legacy Detection}
\label{sec:detection}

The directive |\childdocmain| in the main file can detect
whether the complete document or merely a child is to be compiled
even without using the directive |\childdocof|.
This method is deprecated because it is less robust
and there is no compelling reason to use it;
it is merely provided for backward compatibility
and it may be removed in future versions.

If the detection mechanism is to be used,
it is mandatory to correctly specify
the filename of the main file as the argument of |\childdocmain|:
%
\begin{center}
\begin{tabular}{l}
|\input{childdoc.def}|\\
|\childdocmain{|\textit{main}|}|\\
\end{tabular}
\end{center}
%
If |\jobname| does not match the argument \textit{main} of |\childdocmain|,
it is assumed that |\jobname| points to the child file to be compiled.
When using |\childdocmain| with the main file specified as argument,
it suffices to start a child file
with just |\input{|\textit{main}|}|
without loading of the package and using |\childdocof|.
If instead all processing is done
with the appropriate \textsf{childdoc} directives,
the argument of \textit{main} of |\childdocmain| can be empty.

An alternative version of the command line processing described
in \secref{sec:commandline} using the detection mechanism reads:
%
\begin{center}
|... -jobname "|\textit{target}|" "|[\textit{flags}]%
[|\def\jobname{|\textit{dest}|}|]|\input{|\textit{main}|}"|
\end{center}

%%%%%%%%%%%%%%%%%%%%%%%%%%%%%%%%%%%%%%%%%%%%%%%%%%%%%%%%%%%%%%%%%%%%%%%%%%%%%%%%
\subsection{Manual Code}
\label{sec:manual}

In case one cannot be certain whether the definitions file |childdoc.def|
is installed on the target \TeX{} distribution
and one prefers not to ship it,
it is conceivable to paste a few relevant commands into the sources.

To that end, drop all statements |\input{childdoc.def}|
and perform the replacements as outlined below.
Instead of |\childdocmain{|\textit{main}|}| add the following code
to the top of the main file:
%
\begin{center}
\begin{tabular}{l}
|\||ifdefined\childdocname\endinput\||fi\newif\ifchilddoc|\\
|\edef\childdocname{\scantokens\expandafter{\jobname\noexpand}}|\\
|\def\childdocmain{|\textit{main}|}\||ifx\childdocmain\childdocname\||else|\\
|\childdoctrue\includeonly{\childdocname}\let\jobname\childdocmain\||fi|\\
\end{tabular}
\end{center}
%
Instead of |\childdocof{|\textit{main}|}| just include the main file
at the top of each child file:
%
\begin{center}
|\input{|\textit{main}|}|
\end{center}
%
A simple redirection |\childdocforward{|\textit{dest}|}| is achieved by:
%
\begin{center}
|\def\jobname{|\textit{dest}|}\input{\jobname}|
\end{center}
%
The redirection with prefix
|\childdocforwardprefix[|\textit{prefix}|]{|\textit{dest}|}|
is accomplished by:
%
\begin{center}
\begin{tabular}{l}
|{\edef\jobname{\scantokens\expandafter{\jobname\noexpand}}|\\
|\def\redirectjob |\textit{prefix}|#1~~~{\gdef\jobname{|\textit{dest}|#1}}|\\
|\expandafter\redirectjob\jobname~~~}\input{\jobname}|
\end{tabular}
\end{center}

In an alternative approach,
child documents can be compiled by a specific command line
without additional code or specific definitions:
%
\begin{center}
|... -jobname "|\textit{target}|" "|[\textit{flags}]%
|\includeonly{|\textit{dest}|}\input{|\textit{main}|}"|
\end{center}
%

%%%%%%%%%%%%%%%%%%%%%%%%%%%%%%%%%%%%%%%%%%%%%%%%%%%%%%%%%%%%%%%%%%%%%%%%%%%%%%%%
%%%%%%%%%%%%%%%%%%%%%%%%%%%%%%%%%%%%%%%%%%%%%%%%%%%%%%%%%%%%%%%%%%%%%%%%%%%%%%%%
\section{Information}

%%%%%%%%%%%%%%%%%%%%%%%%%%%%%%%%%%%%%%%%%%%%%%%%%%%%%%%%%%%%%%%%%%%%%%%%%%%%%%%%
\subsection{Copyright}

Copyright \copyright{} 2017--2018 Niklas Beisert

This work may be distributed and/or modified under the
conditions of the \LaTeX{} Project Public License, either version 1.3
of this license or (at your option) any later version.
The latest version of this license is in
  \url{http://www.latex-project.org/lppl.txt}
and version 1.3 or later is part of all distributions of \LaTeX{}
version 2005/12/01 or later.

This work has the LPPL maintenance status `maintained'.

The Current Maintainer of this work is Niklas Beisert.

This work consists of the files |README.txt|, |childdoc.ins| and |childdoc.dtx|
as well as the derived files |childdoc.def|, |cdocsamp.tex|
with |cdocsch1.tex|, |cdocsch2.tex|, |cdocspt3.tex|, |cdocspt4.tex|,
|cdocsdrf.tex|, |cdocsfn1.tex|, |cdocsfn2.tex|
as well as |childdoc.pdf|.

%%%%%%%%%%%%%%%%%%%%%%%%%%%%%%%%%%%%%%%%%%%%%%%%%%%%%%%%%%%%%%%%%%%%%%%%%%%%%%%%
\subsection{Files and Installation}

The package consists of the files:
%
\begin{center}
\begin{tabular}{ll}
    |README.txt|   & readme file \\
    |childdoc.ins| & installation file \\
    |childdoc.dtx| & source file \\
    |childdoc.def| & definition file \\
    |cdocsamp.tex| & sample main file \\
    |cdocsch1.tex| & sample include file \\
    |cdocsch2.tex| & sample include file \\
    |cdocspt3.tex| & sample part file \\
    |cdocspt4.tex| & sample part file \\
    |cdocsdrf.tex| & sample redirection file \\
    |cdocsfn1.tex| & sample redirection file \\
    |cdocsfn2.tex| & sample redirection file \\
    |childdoc.pdf| & manual
\end{tabular}
\end{center}
%
The distribution consists of the files
|README.txt|, |childdoc.ins| and |childdoc.dtx|.
%
\begin{itemize}
\item
Run (pdf)\LaTeX{} on |childdoc.dtx|
to compile the manual |childdoc.pdf| (this file).
\item
Run \LaTeX{} on |childdoc.ins| to create the definitions file |childdoc.def|
and the sample |cdocsamp.tex| with include files
|cdocsch1.tex|, |cdocsch2.tex|, |cdocspt3.tex|, |cdocspt4.tex|,
|cdocsdrf.tex|, |cdocsfn1.tex|, |cdocsfn2.tex|.
Then copy the file |childdoc.def| to an appropriate directory of your \LaTeX{}
distribution, e.g.\ \textit{texmf-root}|/tex/latex/childdoc|.
\end{itemize}

%%%%%%%%%%%%%%%%%%%%%%%%%%%%%%%%%%%%%%%%%%%%%%%%%%%%%%%%%%%%%%%%%%%%%%%%%%%%%%%%
\subsection{Related CTAN Packages}

There are several other packages which offer a similar functionality:
%
\begin{itemize}
\item
The packages
\href{http://ctan.org/pkg/docmute}{\textsf{docmute}},
\href{http://ctan.org/pkg/includex}{\textsf{includex}} and
\href{http://ctan.org/pkg/standalone}{\textsf{standalone}}
provide commands to include only the document body of
a child file thus allowing both files to be compiled individually.
\item
The packages \href{http://ctan.org/pkg/subdocs}{\textsf{subdocs}}
and \href{http://ctan.org/pkg/subfiles}{\textsf{subfiles}}
provide structures in which the main and child documents can be
encapsulated and allowing them to be compiled individually.
The inclusion mechanism is different from the conventional |\include|.
\item
The package \href{http://ctan.org/pkg/combine}{\textsf{combine}}
is an elaborate solution to combine several documents into one.
\end{itemize}
%
See also the CTAN topic \href{http://ctan.org/topic/subdocs}{\textsf{subdocs}}
for further related packages.
The present package differs from the above solutions in that
a document structure constructed with the conventional |\include| mechanism
just needs two extra commands at the top of every file
such that all constituent files can be compiled individually.

%%%%%%%%%%%%%%%%%%%%%%%%%%%%%%%%%%%%%%%%%%%%%%%%%%%%%%%%%%%%%%%%%%%%%%%%%%%%%%%%
%\subsection{Feature Suggestions}
%
%The following is a list of features which may be useful for future
%versions of this package:
%%
%\begin{itemize}
%\item
%\ldots
%\end{itemize}

%%%%%%%%%%%%%%%%%%%%%%%%%%%%%%%%%%%%%%%%%%%%%%%%%%%%%%%%%%%%%%%%%%%%%%%%%%%%%%%%
\subsection{Revision History}

%%%%%%%%%%%%%%%%%%%%%%%%%%%%%%%%%%%%%%%%
\paragraph{v2.0:} 2018/12/30

\begin{itemize}
\item
immediate forward processing
\item
added |\childdocby| mechanism
\item
manual restructured
\end{itemize}

%%%%%%%%%%%%%%%%%%%%%%%%%%%%%%%%%%%%%%%%
\paragraph{v1.6:} 2018/01/17

\begin{itemize}
\item
application for development of include files
\item
corrections to manual
\end{itemize}

%%%%%%%%%%%%%%%%%%%%%%%%%%%%%%%%%%%%%%%%
\paragraph{v1.5:} 2017/05/21

\begin{itemize}
\item
more complete structuring introduced
\item
|\childdocof| introduced
\item
|\childdoc| renamed to |\childdocmain|
\item
|\childredirect| renamed to |\childdocforward| and |\childdocforwardprefix|
and functionality expanded
\end{itemize}

%%%%%%%%%%%%%%%%%%%%%%%%%%%%%%%%%%%%%%%%
\paragraph{v1.0:} 2017/04/27

\begin{itemize}
\item
manual and install package
\item
first version published on CTAN
\end{itemize}

%%%%%%%%%%%%%%%%%%%%%%%%%%%%%%%%%%%%%%%%
\paragraph{v0.6:} 2017/04/26

\begin{itemize}
\item
redirection mechanism added
\end{itemize}

%%%%%%%%%%%%%%%%%%%%%%%%%%%%%%%%%%%%%%%%
\paragraph{v0.5:} 2017/04/26

\begin{itemize}
\item
functionality in definition file
\end{itemize}


%%%%%%%%%%%%%%%%%%%%%%%%%%%%%%%%%%%%%%%%%%%%%%%%%%%%%%%%%%%%%%%%%%%%%%%%%%%%%%%%
%%%%%%%%%%%%%%%%%%%%%%%%%%%%%%%%%%%%%%%%%%%%%%%%%%%%%%%%%%%%%%%%%%%%%%%%%%%%%%%%
%%%%%%%%%%%%%%%%%%%%%%%%%%%%%%%%%%%%%%%%%%%%%%%%%%%%%%%%%%%%%%%%%%%%%%%%%%%%%%%%
\appendix

\settowidth\MacroIndent{\rmfamily\scriptsize 000\ }

 \DocInput{childdoc.dtx}

\end{document}
%</driver>
% \fi
%
% %%%%%%%%%%%%%%%%%%%%%%%%%%%%%%%%%%%%%%%%%%%%%%%%%%%%%%%%%%%%%%%%%%%%%%%%%%%%%%
% %%%%%%%%%%%%%%%%%%%%%%%%%%%%%%%%%%%%%%%%%%%%%%%%%%%%%%%%%%%%%%%%%%%%%%%%%%%%%%
% \section{Sample}
%\iffalse
%<*samplemain>
%\fi
%
% The following presents a sample document
% with two chapters, two parts, a title page,
% a compile flag as well as three forwarding files to set the flag.
% It consists of eight |.tex| files:
% \begin{center}
% \begin{tabular}{ll}
% |cdocsamp.tex|&main file\\
% |cdocsch1.tex|&include file for chapter 1\\
% |cdocsch2.tex|&include file for chapter 2\\
% |cdocspt3.tex|&include file for part 3\\
% |cdocspt4.tex|&include file for part 4\\
% |cdocsdrf.tex|&forwarding file for main file in draft mode\\
% |cdocsfi1.tex|&forwarding file for final version of chapter 1\\
% |cdocsfi2.tex|&forwarding file for final version of chapter 2\\
% \end{tabular}
% \end{center}
% Each of the eight files can be compiled directly by the \LaTeX{} compiler.
%
% %%%%%%%%%%%%%%%%%%%%%%%%%%%%%%%%%%%%%%
% \paragraph{Main File.}
%
% The main file is called |cdocsamp.tex|.
%
% Load the \textsf{childdoc} definitions and
% declare the filename for the main document:
%    \begin{macrocode}
\input{childdoc.def}
\childdocmain{}
%    \end{macrocode}

% Optional override for |\version| flag:
%    \begin{macrocode}
%%\ifchilddoc\else\providecommand{\version}{draft}\fi
%    \end{macrocode}

% Define the default values for the |\version| flag
% (|final| for the main file and |draft| for childs):
%    \begin{macrocode}
\ifchilddoc
\providecommand{\version}{draft}
\else
\providecommand{\version}{final}
\fi
%    \end{macrocode}

% Load the standard document class:
%    \begin{macrocode}
\documentclass[12pt]{article}
%    \end{macrocode}

% Start the document body:
%    \begin{macrocode}
\begin{document}
%    \end{macrocode}

% Declare a title page.
% Print title, part of document being processed and version flag:
%    \begin{macrocode}
\addtocounter{page}{-1}
\begin{center}
{\LARGE\bfseries{}childdoc example\par}
\vspace{1cm}
\ifchilddoc
\ifchilddocmanual part\else chapter\fi:
`\childdocname' of `\childdocjob'\par
\else
main document: `\childdocjob'\par
\fi
version: \version\par
\end{center}
\newpage
%    \end{macrocode}

% Manually include selected file,
% otherwise process as usual:
%    \begin{macrocode}
\ifchilddocmanual
\section*{part `\childdocname'}
\input{\childdocname}
\else
%    \end{macrocode}

% Include the two chapters:
%    \begin{macrocode}
\include{cdocsch1}
\include{cdocsch2}
%    \end{macrocode}

% Include the two parts unless only chapters should be displayed:
%    \begin{macrocode}
\ifchilddoc\else
\section{part three}
\input{cdocspt3}
\section{part four}
\input{cdocspt4}
\fi
%    \end{macrocode}

% Process as usual until here:
%    \begin{macrocode}
\fi
%    \end{macrocode}

% End of document body:
%    \begin{macrocode}
\end{document}
%    \end{macrocode}
%\iffalse
%</samplemain>
%\fi
%
% %%%%%%%%%%%%%%%%%%%%%%%%%%%%%%%%%%%%%%
% \paragraph{Chapter Include Files.}
%
% The include files are called |cdocsch1.tex| and |cdocsch2.tex|.
%
%\iffalse
%<*samplechap1|samplechap2>
%\fi

% Optional override for |\version| flag:
%    \begin{macrocode}
%%\providecommand{\version}{final}
%    \end{macrocode}

% Include the main document:
%    \begin{macrocode}
\input{childdoc.def}
\childdocof{cdocsamp}
%    \end{macrocode}

%\iffalse
%</samplechap1|samplechap2>
%\fi
%
%\iffalse
%<*samplechap1>
%\fi
% Some text for chapter 1:
%    \begin{macrocode}
\section{one}
some text in chapter one
%    \end{macrocode}

%\iffalse
%</samplechap1>
%\fi
% Some text for chapter 2:
%\iffalse
%<*samplechap2>
%\fi
%    \begin{macrocode}
\section{two}
more text in chapter two
%    \end{macrocode}

%\iffalse
%</samplechap2>
%\fi
%
% %%%%%%%%%%%%%%%%%%%%%%%%%%%%%%%%%%%%%%
% \paragraph{Part Include Files.}
%
% The include files are called |cdocspt3.tex| and |cdocspt4.tex|.
%
%\iffalse
%<*samplepart3|samplepart4>
%\fi

% Optional override for |\version| flag:
%    \begin{macrocode}
%%\providecommand{\version}{final}
%    \end{macrocode}

% Include the main document:
%    \begin{macrocode}
\input{childdoc.def}
\childdocby{cdocsamp}
%    \end{macrocode}

%\iffalse
%</samplepart3|samplepart4>
%\fi
%
%\iffalse
%<*samplepart3>
%\fi
% Some text for part 3:
%    \begin{macrocode}
some text in part three
%    \end{macrocode}

%\iffalse
%</samplepart3>
%\fi
% Some text for part 4:
%\iffalse
%<*samplepart4>
%\fi
%    \begin{macrocode}
more text in part four
%    \end{macrocode}

%\iffalse
%</samplepart4>
%\fi
%
% %%%%%%%%%%%%%%%%%%%%%%%%%%%%%%%%%%%%%%
% \paragraph{Forwarding for a Complete Draft.}
%
% The following forwarding file |cdocsdrf.tex|
% compiles the main document in draft mode:
%\iffalse
%<*sampledraft>
%\fi
%    \begin{macrocode}
\def\version{draft}
\input{childdoc.def}
\childdocforward{cdocsamp}
%    \end{macrocode}

%\iffalse
%</sampledraft>
%\fi
%
% %%%%%%%%%%%%%%%%%%%%%%%%%%%%%%%%%%%%%%
% \paragraph{Forwarding for Final Version of the Chapters.}
%
% The following forwarding files |cdocsfn1.tex| and |cdocsfn2.tex|
% (with identical content)
% compile the final versions of the child documents
% |cdocsch1.tex| and |cdocsch2.tex|, respectively:
%\iffalse
%<*samplefinal>
%\fi
%    \begin{macrocode}
\def\version{final}
\input{childdoc.def}
\childdocforwardprefix[cdocsamp]{cdocsfn}{cdocsch}
%    \end{macrocode}

%\iffalse
%</samplefinal>
%\fi
%
% %%%%%%%%%%%%%%%%%%%%%%%%%%%%%%%%%%%%%%
% \paragraph{Command Line Processing.}
%
% The following three command lines generate the output files
% |cdocscld|, |cdocscl1| and |cdocscl2|
% which should be identical to
% |cdocsdrf|, |cdocsch1| and |cdocsfn2|, respectively:
% \begin{center}
% \begin{tabular}{l}
% |latex -jobname cdocscld \|\\
% |  "\def\version{draft}\input{childdoc.def}\childdocforward{cdocsamp}"|\\
% |latex -jobname cdocscl1 \|\\
% |  "\input{childdoc.def}\childdocforward[cdocsamp]{cdocsch1}"|\\
% |latex -jobname cdocscl2 \|\\
% |  "\def\version{final}\input{childdoc.def}\childdocforward{cdocsch2}"|
% \end{tabular}
% \end{center}
% Note that the trailing backslash on each first line
% merely continues the input to the second line
% (for convenient cut ant paste).
% Furthermore, the command |latex| can be replaced by any
% of its alternative versions such as |pdflatex|.
%
% %%%%%%%%%%%%%%%%%%%%%%%%%%%%%%%%%%%%%%%%%%%%%%%%%%%%%%%%%%%%%%%%%%%%%%%%%%%%%%
% %%%%%%%%%%%%%%%%%%%%%%%%%%%%%%%%%%%%%%%%%%%%%%%%%%%%%%%%%%%%%%%%%%%%%%%%%%%%%%
% \section{Implementation}
%\iffalse
%<*package>
%\fi
%
% This section describes the definitions file |childdoc.def|.

% The definitions cannot be loaded using |\usepackage| or |\RequirePackage|
% which has a mechanism to prevent loading a style file more than once.
% When loading the definitions by means of |\input|
% multiple instances have to be prevented manually:
%\iffalse
%This code needs to be before the `\ProvidesFile' directive
%which is defined at the beginning of this file.
%Therefore it is also placed there and commented out here.
%</package>
%<*discard>
%\fi
%    \begin{macrocode}
\ifdefined\childdocmain\endinput\fi
%    \end{macrocode}
%\iffalse
%</discard>
%<*package>
%\fi
%
% \macro{\ifchilddoc}
% \macro{\ifchilddocmanual}
% The conditional |\ifchilddoc| tells whether a
% child (true) or main (false) document is being compiled.
% The conditional |\ifchilddocmanual| tells whether
% the |\includeonly| mechanism is used (false) or
% the selection of child files must be performed manually (true).
% The definitions initialise to false:
%    \begin{macrocode}
\newif\ifchilddoc
\newif\ifchilddocmanual
%    \end{macrocode}

% \macro{\childdocname}
% \macro{\childdocjob}
% The macro |\childdocname| stores the name of the main document
% to be compiled. The macro |\childdocjob| stores the name of
% the document on which the \LaTeX{} compiler was originally invoked.
% The content of |\jobname| cannot be compared
% to filenames specified in the source due to different catcodes.
% The following code rescans |\jobname|, stores the result
% in |\childdocname| and saves a copy in |\childdocjob|:
%    \begin{macrocode}
\edef\childdocname{\scantokens\expandafter{\jobname\noexpand}}
\let\childdocjob\childdocname
%    \end{macrocode}

% \macro{\childdocdisable}
% The macro |\childdocdisable| prevents the main file
% from being processed more than once.
% At this stage, the main document command |\childdocmain|
% is assumed to be called once again where it should do nothing.
% Any subsequent call to it should prevent
% a secondary processing of the main document
% It overwrites the forwarding commands
% |\childdocof| and |\childdocforward|
% with empty macros to prevent further inclusions of the main document:
%    \begin{macrocode}
\newcommand{\childdocdisable}
{
  \renewcommand{\childdocmain}[1]{\renewcommand{\childdocmain}[1]{\endinput}}
  \renewcommand{\childdocof}[1]{}
  \renewcommand{\childdocby}[2][]{}
  \renewcommand{\childdocforward}[2][]{}
  \renewcommand{\childdocdisable}{}
}
%    \end{macrocode}

% \macro{\childdocmain}
% The macro |\childdocmain| is to be called at the top of the main file
% with nothing or the main filename (without extension) as argument.
% First, it breaks loops.
% If the argument is not empty and does not match |\childdocname|
% (which is set by the first inclusion of |childdoc.def|),
% |\ifchilddoc| is set to true, |\includeonly| is applied to the child file
% and |\jobname| is set to the main file
% (for proper handling of |.aux| files):
%    \begin{macrocode}
\newcommand{\childdocmain}[1]
{
  \childdocdisable\childdocmain{}
  \if?#1?\else
    \begingroup
      \def\childdoctmp{#1}
      \ifx\childdoctmp\childdocname
        \def\childdoctmp{}
      \else
        \def\childdoctmp
        {
          \childdoctrue
          \includeonly{\childdocname}
          \def\childdocjob{#1}
          \def\jobname{#1}
        }
      \fi
      \expandafter
    \endgroup
    \childdoctmp
  \fi
}
%    \end{macrocode}

% \macro{\childdocof}
% The command |\childdocof| redirects
% compilation to the main file |#1|.
%    \begin{macrocode}
\newcommand{\childdocof}[1]
{
  \childdocdisable
  \childdoctrue
  \includeonly{\childdocname}
  \def\jobname{#1}
  \def\childdocjob{#1}
  \input{#1}
}
%    \end{macrocode}

% \macro{\childdocby}
% The command |\childdocby| ....
%    \begin{macrocode}
\newcommand{\childdocby}[2][]
{
  \childdocdisable
  \childdoctrue
  \childdocmanualtrue
  \if?#1?\else
    \def\jobname{#2}
  \fi
  \def\childdocjob{#2}
  \input{#2}
  \endinput
}
%    \end{macrocode}

% \macro{\childdocforward}
% The command |\childdocforward| redirects
% compilation to the main file or
% (if the optional argument is given) a child file.
% Parameters are set as if the main file
% or a child file starting with |\childdocof| was compiled.
% Then compilation is handed over to the main file:
%    \begin{macrocode}
\newcommand{\childdocforward}[2][]
{
  \begingroup
    \if?#1?
      \def\childdoctmp
      {
        \def\childdocname{#2}
        \def\childdocjob{#2}
        \def\jobname{#2}
        \input{#2}
        \endinput
      }
    \else
      \def\childdoctmp
      {
        \childdocdisable
        \def\childdocname{#2}
        \childdoctrue
        \includeonly{#2}
        \def\childdocjob{#1}
        \def\jobname{#1}
        \input{#1}
        \endinput
      }
    \fi
    \expandafter
  \endgroup
  \childdoctmp
}
%    \end{macrocode}

% \macro{\childdocforwardprefix}
% The command |\childdocforwardprefix| redirects
% compilation to the main or a child file by means of a pattern.
% The prefix |#1| in the current filename is replaced by |#2|
% and the suffix of the current filename is kept
% (it is assumed that the filename does not contain the substring `|~~~|'
% which is used as a delimiter).
% Compilation is handed over to the new file by |\childdocforward|:
%    \begin{macrocode}
\newcommand{\childdocforwardprefix}[3][]
{
  \begingroup
    \def\childdocextract #2##1~~~{\def\childdoctmp{\childdocforward[#1]{#3##1}}}
    \expandafter\childdocextract\childdocname~~~
    \expandafter
  \endgroup
  \childdoctmp
}
%    \end{macrocode}

% \macro{\childdoc}
% The deprecated macro |\childdoc| is a legacy version of |\childdocmain|:
%    \begin{macrocode}
\newcommand{\childdoc}{\childdocmain}
%    \end{macrocode}

% \macro{\childdocredirect}
% The deprecated macro |\childdocredirect| is a legacy version
% of |\childdocforward| and |\childdocforwardprefix|:
%    \begin{macrocode}
\newcommand{\childdocredirect}[2][]
{
  \begingroup
    \if?#1?
      \def\childdoctmp{\childdocforward{#2}}
    \else
      \def\childdoctmp{\childdocforwardprefix{#1}{#2}}
    \fi
    \expandafter
  \endgroup
  \childdoctmp
}
%    \end{macrocode}

%\iffalse
%</package>
%\fi
%
\endinput

\childdocof{cdocsamp}
%    \end{macrocode}

%\iffalse
%</samplechap1|samplechap2>
%\fi
%
%\iffalse
%<*samplechap1>
%\fi
% Some text for chapter 1:
%    \begin{macrocode}
\section{one}
some text in chapter one
%    \end{macrocode}

%\iffalse
%</samplechap1>
%\fi
% Some text for chapter 2:
%\iffalse
%<*samplechap2>
%\fi
%    \begin{macrocode}
\section{two}
more text in chapter two
%    \end{macrocode}

%\iffalse
%</samplechap2>
%\fi
%
% %%%%%%%%%%%%%%%%%%%%%%%%%%%%%%%%%%%%%%
% \paragraph{Part Include Files.}
%
% The include files are called |cdocspt3.tex| and |cdocspt4.tex|.
%
%\iffalse
%<*samplepart3|samplepart4>
%\fi

% Optional override for |\version| flag:
%    \begin{macrocode}
%%\providecommand{\version}{final}
%    \end{macrocode}

% Include the main document:
%    \begin{macrocode}
% \iffalse
%
% childdoc.dtx Copyright (C) 2017-2018 Niklas Beisert
%
% This work may be distributed and/or modified under the
% conditions of the LaTeX Project Public License, either version 1.3
% of this license or (at your option) any later version.
% The latest version of this license is in
%   http://www.latex-project.org/lppl.txt
% and version 1.3 or later is part of all distributions of LaTeX
% version 2005/12/01 or later.
%
% This work has the LPPL maintenance status `maintained'.
%
% The Current Maintainer of this work is Niklas Beisert.
%
% This work consists of the files childdoc.dtx and childdoc.ins
% and the derived files childdoc.def and cdocsamp.tex with
% cdocsch1.tex, cdocsch2.tex, cdocsdrf.tex, cdocsfn1.tex, cdocsfn2.tex.
%
%<package>\ifdefined\childdocmain\endinput\fi
%<package>\ProvidesFile{childdoc.def}[2018/12/30 v2.0 child document driver]
%<samplemain>\ProvidesFile{cdocsamp.tex}[2018/12/30 v2.0 sample for childdoc]
%<*driver>
%\ProvidesFile{childdoc.drv}[2018/12/30 v2.0 childdoc reference manual file]
\PassOptionsToClass{10pt,a4paper}{article}
\documentclass{ltxdoc}

\usepackage[margin=35mm]{geometry}
\usepackage{hyperref}
\usepackage{hyperxmp}
\usepackage[usenames]{color}

\hypersetup{colorlinks=true}
\hypersetup{pdfstartview=FitH}
\hypersetup{pdfpagemode=UseNone}
\hypersetup{pdfsource={}}
\hypersetup{pdflang={en-UK}}
\hypersetup{pdfcopyright={Copyright 2017-2018 Niklas Beisert.
  This work may be distributed and/or modified under the
  conditions of the LaTeX Project Public License, either version 1.3
  of this license or (at your option) any later version.}}
\hypersetup{pdflicenseurl={http://www.latex-project.org/lppl.txt}}
\hypersetup{pdfcontactaddress={ETH Zurich, ITP, HIT K,
  Wolfgang-Pauli-Strasse 27}}
\hypersetup{pdfcontactpostcode={8093}}
\hypersetup{pdfcontactcity={Zurich}}
\hypersetup{pdfcontactcountry={Switzerland}}
\hypersetup{pdfcontactemail={nbeisert@itp.phys.ethz.ch}}
\hypersetup{pdfcontacturl={http://people.phys.ethz.ch/\xmptilde nbeisert/}}

\newcommand{\secref}[1]{\hyperref[#1]{section \ref*{#1}}}

\parskip1ex
\parindent0pt
\let\olditemize\itemize
\def\itemize{\olditemize\parskip0pt}

\begin{document}

\title{The \textsf{childdoc} Package}
\hypersetup{pdftitle={The childdoc Package}}
\author{Niklas Beisert\\[2ex]
  Institut f\"ur Theoretische Physik\\
  Eidgen\"ossische Technische Hochschule Z\"urich\\
  Wolfgang-Pauli-Strasse 27, 8093 Z\"urich, Switzerland\\[1ex]
  \href{mailto:nbeisert@itp.phys.ethz.ch}
  {\texttt{nbeisert@itp.phys.ethz.ch}}}
\hypersetup{pdfauthor={Niklas Beisert}}
\hypersetup{pdfsubject={Manual for the LaTeX2e Package childdoc}}
\date{30 December 2018, \textsf{v2.0}}
\maketitle

\begin{abstract}\noindent
\textsf{childdoc} is a \LaTeXe{} package
that enables the direct compilation
of document sections included by |\include|
to individual files.
\end{abstract}

\begingroup
\parskip0ex
\tableofcontents
\endgroup

%%%%%%%%%%%%%%%%%%%%%%%%%%%%%%%%%%%%%%%%%%%%%%%%%%%%%%%%%%%%%%%%%%%%%%%%%%%%%%%%
%%%%%%%%%%%%%%%%%%%%%%%%%%%%%%%%%%%%%%%%%%%%%%%%%%%%%%%%%%%%%%%%%%%%%%%%%%%%%%%%
\section{Introduction}

\LaTeX{} provides a mechanism to structure a large document (such as a book)
into a main file and several child files (containing the chapters)
using the |\include| command.
This mechanism is beneficial for documents
which span hundreds of pages in order to
make the source file(s) more manageable.
Moreover, compilation can be restricted to
selected child files by means of the |\includeonly| command.
The latter feature can be used to reduce the compilation time while editing
(this was significantly more useful in the earlier days of \LaTeX{})
or to generate a smaller document which is easier to navigate.
Another application of |\includeonly| is to generate
documents consisting of selected parts of the complete document.

However, there are a few drawbacks of the plain |\include| mechanism:
\begin{itemize}
\item
The child files cannot be compiled on their own,
they can only be compiled via the main file.
A naive editing environment
(such as a text editor with an option
to have the current file processed by \LaTeX)
may require one to switch to the main file before compiling;
attempting to compile the child file produces errors.
\item
The main file must be modified (each time)
to adjust the |\includeonly| command
to the present needs. This easily leaves the main file in a messy state.
\item
The generated document will always carry the filename
of the main document. This is inconvenient if
several child files are to be compiled and
to be kept for distribution.
\end{itemize}

The present package provides a simple interface
to make child files individually compilable by \LaTeX{}.
Compiling a child file then has the same effect as compiling
the main file with an |\includeonly| command
to select the appropriate child.
Moreover the generated document will carry the name of the child
rather than the main file.
This resolves all three above issues.

This feature is meant to make the editing of books,
thesis documents and lecture notes somewhat more convenient.
However, the package can also be used efficiently for
composing a series of documents (such as exercise sheets)
which are typically distributed individually.
It then assists the author in generating the individual documents
(potentially in different versions)
as well as a document containing the collected series.
Another application is in developing style files
or other kinds of included material
where compilation of the style file could redirect
to a sample or test file.

%%%%%%%%%%%%%%%%%%%%%%%%%%%%%%%%%%%%%%%%%%%%%%%%%%%%%%%%%%%%%%%%%%%%%%%%%%%%%%%%
%%%%%%%%%%%%%%%%%%%%%%%%%%%%%%%%%%%%%%%%%%%%%%%%%%%%%%%%%%%%%%%%%%%%%%%%%%%%%%%%
\section{Usage}

First of all, the package \textsf{childdoc} is \emph{not} a standard
\LaTeXe{} |.sty| style file! Therefore it needs to be invoked in
a non-standard way.

%%%%%%%%%%%%%%%%%%%%%%%%%%%%%%%%%%%%%%%%%%%%%%%%%%%%%%%%%%%%%%%%%%%%%%%%%%%%%%%%
\subsection{Included Files}
\label{sec:include}

%%%%%%%%%%%%%%%%%%%%%%%%%%%%%%%%%%%%%%%%
\DescribeMacro{\childdocmain}
To use the package, add the commands
\begin{center}
\begin{tabular}{l}
|\input{childdoc.def}|\\
|\childdocmain{}|\\
\end{tabular}
\end{center}
at the very top of the main \LaTeX{} file,
in particular \emph{before} the |\documentclass| statement!
The argument of |\childdocmain| should be left empty
(but it must be present).

%%%%%%%%%%%%%%%%%%%%%%%%%%%%%%%%%%%%%%%%
\DescribeMacro{\childdocof}
Furthermore, add the commands
\begin{center}
\begin{tabular}{l}
|\input{childdoc.def}|\\
|\childdocof{|\textit{main}|}|\\
\end{tabular}
\end{center}
at the top of every child file \textit{child}
which is included by |\include{|\textit{child}|}|
from within the main file
(or at least for those files to be compiled individually).
The argument \textit{main} must be the filename of the main file.

There are a couple of
considerations in setting up the main and child documents:

%%%%%%%%%%%%%%%%%%%%%%%%%%%%%%%%%%%%%%%%
\paragraph{Restrictions.}

Please note the following restrictions:
\begin{itemize}
\item
|\childdocmain| must be called with one argument \textit{main}
to ensure compatibility with earlier version of the package.
It must either be empty (|\childdocmain{}|)
or precisely match the filename of the main file in which it is specified.
See \secref{sec:detection} for further information.
\item
The filename \textit{main} must be specified without the |.tex| extension.
\item
The filename \textit{main} is case sensitive
(even in case-insensitive file systems)
due to internal string comparison.
\item
The argument \textit{main} should be fully expanded, it cannot be a macro.
\item
Subdirectories and special characters should be avoided in filenames.
\item
The command |\childdocmain{|\textit{main}|}| must be followed by a whitespace.
It should not be followed immediately by another command
or by a comment mark `|%|'.
This is because the \TeX{} parser reads the token immediately following
the argument of |\childdocmain| and puts it
at the beginning of every child section;
however, a white\-space is ignored.
\end{itemize}

%%%%%%%%%%%%%%%%%%%%%%%%%%%%%%%%%%%%%%%%
\paragraph{Content of Main File.}

It is advisable to place all content in the child files included by |\include|.
Any output contained in the main file will appear in all child documents
unless suppressed manually;
it cannot be suppressed automatically by the |\includeonly| directive
and thus should normally be avoided.
A method to include some content in the main file
by means of conditional processing is described in \secref{sec:conditional}.

%%%%%%%%%%%%%%%%%%%%%%%%%%%%%%%%%%%%%%%%
\paragraph{Page Numbering.}

When only a part of the document is compiled,
the appropriate numbering of pages
(as well as other status parameters)
is determined from the |.aux| files.
The latter contain information from previous passes.
However this information needs to propagate through
all intermediate child documents.
Therefore the page numbering in child documents may well
be inconsistent until the complete document is compiled at least once.

A useful (if unconventional) way to always ensure a consistent
page numbering is to restart the numbering in each child document
and denote the pages by `\textit{child}|.|\textit{page}'
where \textit{child} represents the chapter/section number of the child file.
This can be achieved by the command
|\numberwithin{page}{|\textit{child}|}|
of the \textsf{amsmath} package
where \textit{child} can be |chapter| or |section|
depending on the chosen structuring.
Alternatively, one can modify the macro |\thepage| appropriately
and reset the counter |page| at the start of each child file.

%%%%%%%%%%%%%%%%%%%%%%%%%%%%%%%%%%%%%%%%%%%%%%%%%%%%%%%%%%%%%%%%%%%%%%%%%%%%%%%%
\subsection{Conditional Processing}
\label{sec:conditional}

The package provides a mechanism to compile different versions
of a document. To customise the versions further some conditional processing
can come in handy to distinguish which version is being compiled.
The package provides two macros to describe the compilation context:

%%%%%%%%%%%%%%%%%%%%%%%%%%%%%%%%%%%%%%%%
\DescribeMacro{\ifchilddoc}
The conditional |\ifchilddoc| distinguishes between the compilation of
child documents and the main document:
%
\begin{center}
|\ifchilddoc |\textit{child-code}| |[|\||else |\textit{main-code}]| \||fi|
\end{center}

%%%%%%%%%%%%%%%%%%%%%%%%%%%%%%%%%%%%%%%%
\DescribeMacro{\childdocname}
\DescribeMacro{\childdocjob}
The macro |\childdocname| contains the filename (without extension)
of the main or child file being processed.
Note that |\childdocjob| will always contain the name of the main file.

%%%%%%%%%%%%%%%%%%%%%%%%%%%%%%%%%%%%%%%%
\paragraph{Title Page.}

Conditional processing can be used to include a title or banner page
in the main document when proper precautions are taken.
Importantly, the code in the main file should ensure that the page counter
(as well as other status parameters which are stored in the |.aux| files)
takes the same value after the conditional processing.
Otherwise the page numbers may take divergent values
depending on which part is compiled.

For example, a title page could be declared by:
%
\begin{center}
\begin{tabular}{l}
|\ifchilddoc\||else|\\
|\addtocounter{page}{-1}|\\
\textit{code for title page}\\
|\newpage|\\
|\||fi|
\end{tabular}
\end{center}
%
A banner page for the child documents can be generated by:
%
\begin{center}
\begin{tabular}{l}
|\ifchilddoc|\\
|\addtocounter{page}{-1}|\\
\textit{code for banner page}\\
|\newpage|\\
|\||fi|
\end{tabular}
\end{center}
%
Here one could write a message such as:
\begin{center}
|This is the part \childdocname{} of \childdocjob{}.|
\end{center}

%%%%%%%%%%%%%%%%%%%%%%%%%%%%%%%%%%%%%%%%%%%%%%%%%%%%%%%%%%%%%%%%%%%%%%%%%%%%%%%%
\subsection{Flags}
\label{sec:flags}

The package makes it easy to generate different versions
of the main or child documents.
To this end compilation flags can be defined
and assigned different default values.
They will be particularly useful in conjunction
with the forwarding mechanism described in \secref{sec:forward}.

For example, it may be useful to have a flag |\version|
which can be set to |draft| or |final|.
The document source will contain some conditional code
depending on the value of |\version|.
Suppose further, the flag should default to |final| for the main file
and to |draft| for child files
which is a natural assignment for editing the document.
This is achieved by placing the following code
in the preamble of the main document
(below the |\childdocmain| directive):
%
\begin{center}
\begin{tabular}{l}
|\ifchilddoc|\\
|\providecommand{\version}{draft}|\\
|\||else|\\
|\providecommand{\version}{final}|\\
|\||fi|
\end{tabular}
\end{center}
%
The definition by |\providecommand| makes sure
that previous definitions are not overwritten.
Further statements |\providecommand{\version}{...}|
can thus be added before the above code to override it.

For the main file, one might add a line
(between |\childdocmain| and the above block)
%
\begin{center}
|%\ifchilddoc\||else\providecommand{\version}{draft}\||fi|
\end{center}
%
which can be uncommented to produce a draft version.
Likewise one can add a line to the very top of a child file
(above the |\childdocof{|\textit{main}|}| directive)
%
\begin{center}
|%\providecommand{\version}{final}|
\end{center}
%
which can be uncommented to produce the final version of this child document.

%%%%%%%%%%%%%%%%%%%%%%%%%%%%%%%%%%%%%%%%%%%%%%%%%%%%%%%%%%%%%%%%%%%%%%%%%%%%%%%%
\subsection{Forwarding}
\label{sec:forward}

Different versions of the main or child documents
using compilation flags as described in \secref{sec:flags}
can be (permanently) stored in different files
for convenient compilation, viewing and distribution.
To this end, the package defines a command
to pass on compilation to a different file:

%%%%%%%%%%%%%%%%%%%%%%%%%%%%%%%%%%%%%%%%
\DescribeMacro{\childdocforward}
The command |\childdocforward| redirects processing to
another source file:
%
\begin{center}
\begin{tabular}{l}
|\input{childdoc.def}|\\
|\childdocforward[|\textit{main}|]{|\textit{dest}|}|\\
\end{tabular}
\end{center}
%
The argument \textit{dest} is the destination file
(without extension).
It should be the main file or one of the child files.
Note that further \textsf{childdoc} directives
such as |\childdocof| and |\childdocforward|
in the indicated file will be processed in this form.
The optional argument \textit{main}
passes on directly to the main file \textit{main}
while pretending to compile the child \textit{dest}.
This form behaves as if \textit{dest}
issues |\childdocof{|\textit{main}|}| right away,
and no further \textsf{childdoc} directives will be processed.

%%%%%%%%%%%%%%%%%%%%%%%%%%%%%%%%%%%%%%%%
\DescribeMacro{\...prefix}
In the alternative form |\childdocforwardprefix|,
%
\begin{center}
\begin{tabular}{l}
|\input{childdoc.def}|\\
|\childdocforwardprefix[|\textit{main}|]{|\textit{prefix}|}{|\textit{dest}|}|
\end{tabular}
\end{center}
%
the destination file is determined by a pattern
depending on the current file:
To make this work, the current file must be called
`{\textit{prefix}\hspace{0.2em}\textit{suffix}}'
with \textit{prefix} matching precisely the argument.
Processing is then passed on to the file
`{\textit{dest}\hspace{0.2em}\textit{suffix}}'.
Surely, the same effect is achieved by
directly specifying the
argument `{\textit{dest}\hspace{0.2em}\textit{suffix}}'
in the first form.
However, that requires to set up a different file
for each child. With the alternative form of the command
all these files can have exactly the same content
which simplifies setting them up and maintaining them.

For example, the following file |draft.tex|
with a compilation flag |\version| as described in \secref{sec:flags}
compiles the main document as a draft:
%
\begin{center}
\begin{tabular}{l}
|\def\version{draft}|\\
|\input{childdoc.def}|\\
|\childdocforward{|\textit{main}|}|
\end{tabular}
\end{center}
%
Likewise, the following files |final|\textit{nn}|.tex|
compile the final version of the child document
|child|\textit{nn}|.tex|:
%
\begin{center}
\begin{tabular}{l}
|\def\version{final}|\\
|\input{childdoc.def}|\\
|\childdocforwardprefix{final}{child}|
\end{tabular}
\end{center}
%

Note that when several versions of a main file and/or of each child file
are to be generated, it may be convenient to set up a |Makefile| or
shell script to automatise the process.

%%%%%%%%%%%%%%%%%%%%%%%%%%%%%%%%%%%%%%%%%%%%%%%%%%%%%%%%%%%%%%%%%%%%%%%%%%%%%%%%
\subsection{Command Line Processing}
\label{sec:commandline}

The effect of redirection files can also be achieved by invoking
the \LaTeX{} compiler with a more elaborate command line.
Most conveniently this should be done as part
of a shell script or a |Makefile|.

When using \textsf{childdoc} in the main file, the following
command lines effectively perform a redirection
(note that depending on the shell being used,
backslashes may have to be doubled: `|\|' $\to$ `|\\|'):
%
\begin{center}
|... -jobname "|\textit{target}|" |\\|"|[\textit{flags}]%
|\input{childdoc.def}\childdocforward[|\textit{main}|]{|\textit{dest}|}"|
\end{center}
%
Here \textit{target} is the name of the output file,
\textit{main} is the name of the main file
and \textit{dest} is the name of the main or child file to be processed
(all filenames without extensions).
The optional argument \textit{main} can be omitted
if \textit{main} matches \textit{dest}.
Optionally, compilation \textit{flags} can be defined via |\def| commands.
This command line makes the \TeX{} engine believe
it is compiling the file \textit{target}
whose content is specified as the latter parameter.
The provided code then forwards the processing to
\textit{main} or \textit{dest} as described in \secref{sec:forward}.

%%%%%%%%%%%%%%%%%%%%%%%%%%%%%%%%%%%%%%%%%%%%%%%%%%%%%%%%%%%%%%%%%%%%%%%%%%%%%%%%
\subsection{Include by Input}
\label{sec:input}

Including child documents by |\include| has some restrictions by design.
Most notably, the content of a child document always occupies
its own set of pages; pages cannot be shared between child documents.
Usually, this behaviour makes perfect sense
because each child document contain an essential part of the document.
However, in some situations it may be desirable to compose
a document from a collection of parts
without having mandatory page breaks between then.
For this case, the package
provides a mechanism to include parts
by |\input| which can also be processed individually.
However, by construction this mechanism
requires manual handling of the content to be output.

%%%%%%%%%%%%%%%%%%%%%%%%%%%%%%%%%%%%%%%%
\DescribeMacro{\ifchilddocmanual}
The main file should be prepared as usual, see \secref{sec:include}.
However, the document body must make a distinction
between processing of an individual part and of the main document, e.g.:
%
\begin{center}
\begin{tabular}{l}
|\ifchilddocmanual|\\
|\input{\childdocname}|\\
|\||else|\\
\textit{document body with }|\input{|\textit{part}|}|\\
|\||fi|
\end{tabular}
\end{center}
%
The conditional |\ifchilddocmanual| is true whenever
a part to be included by |\input| is being compiled,
and the name of the part is stored in |\childdocname|.

%%%%%%%%%%%%%%%%%%%%%%%%%%%%%%%%%%%%%%%%
\DescribeMacro{\childdocby}
Each part to be included by |\input| should start with:
%
\begin{center}
\begin{tabular}{l}
|\input{childdoc.def}|\\
|\childdocby{|\textit{main}|}|\\
\end{tabular}
\end{center}
%
The directive |\childdocby| is similar to |\childdocof|
described in \secref{sec:include},
but the subsequent selection of content must be done manually.
To that end, both |\ifchilddoc| and |\ifchilddocmanual|
will be true upon processing of a part,
and the name of the part is stored in |\childdocname|.
Note that |\jobname| will be set to the filename of the current part
so that each part receives an individual |.aux| file
that does not interfere with the |.aux| file(s) of the main document.
This behaviour can be altered by the alternative form
|\childdocby[*]{|\textit{main}|}| (with a non-empty optional argument)
which uses the |.aux| file of the main document
by setting |\jobname| to \textit{main}.

%%%%%%%%%%%%%%%%%%%%%%%%%%%%%%%%%%%%%%%%%%%%%%%%%%%%%%%%%%%%%%%%%%%%%%%%%%%%%%%%
\subsection{Driver Development}
\label{sec:driver}

The \textsf{childdoc} mechanism can also be use for the development
of definition files such as \LaTeX{} styles or classes.
This case differs from the above setup with multiple parts
included by |\include| in that no |\includeonly| should be invoked.
This can be achieved by starting the include file
(before |\ProvidesPackage|) with:
%
\begin{center}
\begin{tabular}{l}
|\input{childdoc.def}|\\
|\childdocforward{|\textit{main}|}|\\
\end{tabular}
\end{center}
%
or alternatively with:
%
\begin{center}
\begin{tabular}{l}
|\input{childdoc.def}|\\
|\childdocby{|\textit{main}|}|\\
\end{tabular}
\end{center}
%
Both forms have slightly different effects as described above.
The main file is prepared as usual, see \secref{sec:include}.

%%%%%%%%%%%%%%%%%%%%%%%%%%%%%%%%%%%%%%%%%%%%%%%%%%%%%%%%%%%%%%%%%%%%%%%%%%%%%%%%
\subsection{Legacy Detection}
\label{sec:detection}

The directive |\childdocmain| in the main file can detect
whether the complete document or merely a child is to be compiled
even without using the directive |\childdocof|.
This method is deprecated because it is less robust
and there is no compelling reason to use it;
it is merely provided for backward compatibility
and it may be removed in future versions.

If the detection mechanism is to be used,
it is mandatory to correctly specify
the filename of the main file as the argument of |\childdocmain|:
%
\begin{center}
\begin{tabular}{l}
|\input{childdoc.def}|\\
|\childdocmain{|\textit{main}|}|\\
\end{tabular}
\end{center}
%
If |\jobname| does not match the argument \textit{main} of |\childdocmain|,
it is assumed that |\jobname| points to the child file to be compiled.
When using |\childdocmain| with the main file specified as argument,
it suffices to start a child file
with just |\input{|\textit{main}|}|
without loading of the package and using |\childdocof|.
If instead all processing is done
with the appropriate \textsf{childdoc} directives,
the argument of \textit{main} of |\childdocmain| can be empty.

An alternative version of the command line processing described
in \secref{sec:commandline} using the detection mechanism reads:
%
\begin{center}
|... -jobname "|\textit{target}|" "|[\textit{flags}]%
[|\def\jobname{|\textit{dest}|}|]|\input{|\textit{main}|}"|
\end{center}

%%%%%%%%%%%%%%%%%%%%%%%%%%%%%%%%%%%%%%%%%%%%%%%%%%%%%%%%%%%%%%%%%%%%%%%%%%%%%%%%
\subsection{Manual Code}
\label{sec:manual}

In case one cannot be certain whether the definitions file |childdoc.def|
is installed on the target \TeX{} distribution
and one prefers not to ship it,
it is conceivable to paste a few relevant commands into the sources.

To that end, drop all statements |\input{childdoc.def}|
and perform the replacements as outlined below.
Instead of |\childdocmain{|\textit{main}|}| add the following code
to the top of the main file:
%
\begin{center}
\begin{tabular}{l}
|\||ifdefined\childdocname\endinput\||fi\newif\ifchilddoc|\\
|\edef\childdocname{\scantokens\expandafter{\jobname\noexpand}}|\\
|\def\childdocmain{|\textit{main}|}\||ifx\childdocmain\childdocname\||else|\\
|\childdoctrue\includeonly{\childdocname}\let\jobname\childdocmain\||fi|\\
\end{tabular}
\end{center}
%
Instead of |\childdocof{|\textit{main}|}| just include the main file
at the top of each child file:
%
\begin{center}
|\input{|\textit{main}|}|
\end{center}
%
A simple redirection |\childdocforward{|\textit{dest}|}| is achieved by:
%
\begin{center}
|\def\jobname{|\textit{dest}|}\input{\jobname}|
\end{center}
%
The redirection with prefix
|\childdocforwardprefix[|\textit{prefix}|]{|\textit{dest}|}|
is accomplished by:
%
\begin{center}
\begin{tabular}{l}
|{\edef\jobname{\scantokens\expandafter{\jobname\noexpand}}|\\
|\def\redirectjob |\textit{prefix}|#1~~~{\gdef\jobname{|\textit{dest}|#1}}|\\
|\expandafter\redirectjob\jobname~~~}\input{\jobname}|
\end{tabular}
\end{center}

In an alternative approach,
child documents can be compiled by a specific command line
without additional code or specific definitions:
%
\begin{center}
|... -jobname "|\textit{target}|" "|[\textit{flags}]%
|\includeonly{|\textit{dest}|}\input{|\textit{main}|}"|
\end{center}
%

%%%%%%%%%%%%%%%%%%%%%%%%%%%%%%%%%%%%%%%%%%%%%%%%%%%%%%%%%%%%%%%%%%%%%%%%%%%%%%%%
%%%%%%%%%%%%%%%%%%%%%%%%%%%%%%%%%%%%%%%%%%%%%%%%%%%%%%%%%%%%%%%%%%%%%%%%%%%%%%%%
\section{Information}

%%%%%%%%%%%%%%%%%%%%%%%%%%%%%%%%%%%%%%%%%%%%%%%%%%%%%%%%%%%%%%%%%%%%%%%%%%%%%%%%
\subsection{Copyright}

Copyright \copyright{} 2017--2018 Niklas Beisert

This work may be distributed and/or modified under the
conditions of the \LaTeX{} Project Public License, either version 1.3
of this license or (at your option) any later version.
The latest version of this license is in
  \url{http://www.latex-project.org/lppl.txt}
and version 1.3 or later is part of all distributions of \LaTeX{}
version 2005/12/01 or later.

This work has the LPPL maintenance status `maintained'.

The Current Maintainer of this work is Niklas Beisert.

This work consists of the files |README.txt|, |childdoc.ins| and |childdoc.dtx|
as well as the derived files |childdoc.def|, |cdocsamp.tex|
with |cdocsch1.tex|, |cdocsch2.tex|, |cdocspt3.tex|, |cdocspt4.tex|,
|cdocsdrf.tex|, |cdocsfn1.tex|, |cdocsfn2.tex|
as well as |childdoc.pdf|.

%%%%%%%%%%%%%%%%%%%%%%%%%%%%%%%%%%%%%%%%%%%%%%%%%%%%%%%%%%%%%%%%%%%%%%%%%%%%%%%%
\subsection{Files and Installation}

The package consists of the files:
%
\begin{center}
\begin{tabular}{ll}
    |README.txt|   & readme file \\
    |childdoc.ins| & installation file \\
    |childdoc.dtx| & source file \\
    |childdoc.def| & definition file \\
    |cdocsamp.tex| & sample main file \\
    |cdocsch1.tex| & sample include file \\
    |cdocsch2.tex| & sample include file \\
    |cdocspt3.tex| & sample part file \\
    |cdocspt4.tex| & sample part file \\
    |cdocsdrf.tex| & sample redirection file \\
    |cdocsfn1.tex| & sample redirection file \\
    |cdocsfn2.tex| & sample redirection file \\
    |childdoc.pdf| & manual
\end{tabular}
\end{center}
%
The distribution consists of the files
|README.txt|, |childdoc.ins| and |childdoc.dtx|.
%
\begin{itemize}
\item
Run (pdf)\LaTeX{} on |childdoc.dtx|
to compile the manual |childdoc.pdf| (this file).
\item
Run \LaTeX{} on |childdoc.ins| to create the definitions file |childdoc.def|
and the sample |cdocsamp.tex| with include files
|cdocsch1.tex|, |cdocsch2.tex|, |cdocspt3.tex|, |cdocspt4.tex|,
|cdocsdrf.tex|, |cdocsfn1.tex|, |cdocsfn2.tex|.
Then copy the file |childdoc.def| to an appropriate directory of your \LaTeX{}
distribution, e.g.\ \textit{texmf-root}|/tex/latex/childdoc|.
\end{itemize}

%%%%%%%%%%%%%%%%%%%%%%%%%%%%%%%%%%%%%%%%%%%%%%%%%%%%%%%%%%%%%%%%%%%%%%%%%%%%%%%%
\subsection{Related CTAN Packages}

There are several other packages which offer a similar functionality:
%
\begin{itemize}
\item
The packages
\href{http://ctan.org/pkg/docmute}{\textsf{docmute}},
\href{http://ctan.org/pkg/includex}{\textsf{includex}} and
\href{http://ctan.org/pkg/standalone}{\textsf{standalone}}
provide commands to include only the document body of
a child file thus allowing both files to be compiled individually.
\item
The packages \href{http://ctan.org/pkg/subdocs}{\textsf{subdocs}}
and \href{http://ctan.org/pkg/subfiles}{\textsf{subfiles}}
provide structures in which the main and child documents can be
encapsulated and allowing them to be compiled individually.
The inclusion mechanism is different from the conventional |\include|.
\item
The package \href{http://ctan.org/pkg/combine}{\textsf{combine}}
is an elaborate solution to combine several documents into one.
\end{itemize}
%
See also the CTAN topic \href{http://ctan.org/topic/subdocs}{\textsf{subdocs}}
for further related packages.
The present package differs from the above solutions in that
a document structure constructed with the conventional |\include| mechanism
just needs two extra commands at the top of every file
such that all constituent files can be compiled individually.

%%%%%%%%%%%%%%%%%%%%%%%%%%%%%%%%%%%%%%%%%%%%%%%%%%%%%%%%%%%%%%%%%%%%%%%%%%%%%%%%
%\subsection{Feature Suggestions}
%
%The following is a list of features which may be useful for future
%versions of this package:
%%
%\begin{itemize}
%\item
%\ldots
%\end{itemize}

%%%%%%%%%%%%%%%%%%%%%%%%%%%%%%%%%%%%%%%%%%%%%%%%%%%%%%%%%%%%%%%%%%%%%%%%%%%%%%%%
\subsection{Revision History}

%%%%%%%%%%%%%%%%%%%%%%%%%%%%%%%%%%%%%%%%
\paragraph{v2.0:} 2018/12/30

\begin{itemize}
\item
immediate forward processing
\item
added |\childdocby| mechanism
\item
manual restructured
\end{itemize}

%%%%%%%%%%%%%%%%%%%%%%%%%%%%%%%%%%%%%%%%
\paragraph{v1.6:} 2018/01/17

\begin{itemize}
\item
application for development of include files
\item
corrections to manual
\end{itemize}

%%%%%%%%%%%%%%%%%%%%%%%%%%%%%%%%%%%%%%%%
\paragraph{v1.5:} 2017/05/21

\begin{itemize}
\item
more complete structuring introduced
\item
|\childdocof| introduced
\item
|\childdoc| renamed to |\childdocmain|
\item
|\childredirect| renamed to |\childdocforward| and |\childdocforwardprefix|
and functionality expanded
\end{itemize}

%%%%%%%%%%%%%%%%%%%%%%%%%%%%%%%%%%%%%%%%
\paragraph{v1.0:} 2017/04/27

\begin{itemize}
\item
manual and install package
\item
first version published on CTAN
\end{itemize}

%%%%%%%%%%%%%%%%%%%%%%%%%%%%%%%%%%%%%%%%
\paragraph{v0.6:} 2017/04/26

\begin{itemize}
\item
redirection mechanism added
\end{itemize}

%%%%%%%%%%%%%%%%%%%%%%%%%%%%%%%%%%%%%%%%
\paragraph{v0.5:} 2017/04/26

\begin{itemize}
\item
functionality in definition file
\end{itemize}


%%%%%%%%%%%%%%%%%%%%%%%%%%%%%%%%%%%%%%%%%%%%%%%%%%%%%%%%%%%%%%%%%%%%%%%%%%%%%%%%
%%%%%%%%%%%%%%%%%%%%%%%%%%%%%%%%%%%%%%%%%%%%%%%%%%%%%%%%%%%%%%%%%%%%%%%%%%%%%%%%
%%%%%%%%%%%%%%%%%%%%%%%%%%%%%%%%%%%%%%%%%%%%%%%%%%%%%%%%%%%%%%%%%%%%%%%%%%%%%%%%
\appendix

\settowidth\MacroIndent{\rmfamily\scriptsize 000\ }

 \DocInput{childdoc.dtx}

\end{document}
%</driver>
% \fi
%
% %%%%%%%%%%%%%%%%%%%%%%%%%%%%%%%%%%%%%%%%%%%%%%%%%%%%%%%%%%%%%%%%%%%%%%%%%%%%%%
% %%%%%%%%%%%%%%%%%%%%%%%%%%%%%%%%%%%%%%%%%%%%%%%%%%%%%%%%%%%%%%%%%%%%%%%%%%%%%%
% \section{Sample}
%\iffalse
%<*samplemain>
%\fi
%
% The following presents a sample document
% with two chapters, two parts, a title page,
% a compile flag as well as three forwarding files to set the flag.
% It consists of eight |.tex| files:
% \begin{center}
% \begin{tabular}{ll}
% |cdocsamp.tex|&main file\\
% |cdocsch1.tex|&include file for chapter 1\\
% |cdocsch2.tex|&include file for chapter 2\\
% |cdocspt3.tex|&include file for part 3\\
% |cdocspt4.tex|&include file for part 4\\
% |cdocsdrf.tex|&forwarding file for main file in draft mode\\
% |cdocsfi1.tex|&forwarding file for final version of chapter 1\\
% |cdocsfi2.tex|&forwarding file for final version of chapter 2\\
% \end{tabular}
% \end{center}
% Each of the eight files can be compiled directly by the \LaTeX{} compiler.
%
% %%%%%%%%%%%%%%%%%%%%%%%%%%%%%%%%%%%%%%
% \paragraph{Main File.}
%
% The main file is called |cdocsamp.tex|.
%
% Load the \textsf{childdoc} definitions and
% declare the filename for the main document:
%    \begin{macrocode}
\input{childdoc.def}
\childdocmain{}
%    \end{macrocode}

% Optional override for |\version| flag:
%    \begin{macrocode}
%%\ifchilddoc\else\providecommand{\version}{draft}\fi
%    \end{macrocode}

% Define the default values for the |\version| flag
% (|final| for the main file and |draft| for childs):
%    \begin{macrocode}
\ifchilddoc
\providecommand{\version}{draft}
\else
\providecommand{\version}{final}
\fi
%    \end{macrocode}

% Load the standard document class:
%    \begin{macrocode}
\documentclass[12pt]{article}
%    \end{macrocode}

% Start the document body:
%    \begin{macrocode}
\begin{document}
%    \end{macrocode}

% Declare a title page.
% Print title, part of document being processed and version flag:
%    \begin{macrocode}
\addtocounter{page}{-1}
\begin{center}
{\LARGE\bfseries{}childdoc example\par}
\vspace{1cm}
\ifchilddoc
\ifchilddocmanual part\else chapter\fi:
`\childdocname' of `\childdocjob'\par
\else
main document: `\childdocjob'\par
\fi
version: \version\par
\end{center}
\newpage
%    \end{macrocode}

% Manually include selected file,
% otherwise process as usual:
%    \begin{macrocode}
\ifchilddocmanual
\section*{part `\childdocname'}
\input{\childdocname}
\else
%    \end{macrocode}

% Include the two chapters:
%    \begin{macrocode}
\include{cdocsch1}
\include{cdocsch2}
%    \end{macrocode}

% Include the two parts unless only chapters should be displayed:
%    \begin{macrocode}
\ifchilddoc\else
\section{part three}
\input{cdocspt3}
\section{part four}
\input{cdocspt4}
\fi
%    \end{macrocode}

% Process as usual until here:
%    \begin{macrocode}
\fi
%    \end{macrocode}

% End of document body:
%    \begin{macrocode}
\end{document}
%    \end{macrocode}
%\iffalse
%</samplemain>
%\fi
%
% %%%%%%%%%%%%%%%%%%%%%%%%%%%%%%%%%%%%%%
% \paragraph{Chapter Include Files.}
%
% The include files are called |cdocsch1.tex| and |cdocsch2.tex|.
%
%\iffalse
%<*samplechap1|samplechap2>
%\fi

% Optional override for |\version| flag:
%    \begin{macrocode}
%%\providecommand{\version}{final}
%    \end{macrocode}

% Include the main document:
%    \begin{macrocode}
\input{childdoc.def}
\childdocof{cdocsamp}
%    \end{macrocode}

%\iffalse
%</samplechap1|samplechap2>
%\fi
%
%\iffalse
%<*samplechap1>
%\fi
% Some text for chapter 1:
%    \begin{macrocode}
\section{one}
some text in chapter one
%    \end{macrocode}

%\iffalse
%</samplechap1>
%\fi
% Some text for chapter 2:
%\iffalse
%<*samplechap2>
%\fi
%    \begin{macrocode}
\section{two}
more text in chapter two
%    \end{macrocode}

%\iffalse
%</samplechap2>
%\fi
%
% %%%%%%%%%%%%%%%%%%%%%%%%%%%%%%%%%%%%%%
% \paragraph{Part Include Files.}
%
% The include files are called |cdocspt3.tex| and |cdocspt4.tex|.
%
%\iffalse
%<*samplepart3|samplepart4>
%\fi

% Optional override for |\version| flag:
%    \begin{macrocode}
%%\providecommand{\version}{final}
%    \end{macrocode}

% Include the main document:
%    \begin{macrocode}
\input{childdoc.def}
\childdocby{cdocsamp}
%    \end{macrocode}

%\iffalse
%</samplepart3|samplepart4>
%\fi
%
%\iffalse
%<*samplepart3>
%\fi
% Some text for part 3:
%    \begin{macrocode}
some text in part three
%    \end{macrocode}

%\iffalse
%</samplepart3>
%\fi
% Some text for part 4:
%\iffalse
%<*samplepart4>
%\fi
%    \begin{macrocode}
more text in part four
%    \end{macrocode}

%\iffalse
%</samplepart4>
%\fi
%
% %%%%%%%%%%%%%%%%%%%%%%%%%%%%%%%%%%%%%%
% \paragraph{Forwarding for a Complete Draft.}
%
% The following forwarding file |cdocsdrf.tex|
% compiles the main document in draft mode:
%\iffalse
%<*sampledraft>
%\fi
%    \begin{macrocode}
\def\version{draft}
\input{childdoc.def}
\childdocforward{cdocsamp}
%    \end{macrocode}

%\iffalse
%</sampledraft>
%\fi
%
% %%%%%%%%%%%%%%%%%%%%%%%%%%%%%%%%%%%%%%
% \paragraph{Forwarding for Final Version of the Chapters.}
%
% The following forwarding files |cdocsfn1.tex| and |cdocsfn2.tex|
% (with identical content)
% compile the final versions of the child documents
% |cdocsch1.tex| and |cdocsch2.tex|, respectively:
%\iffalse
%<*samplefinal>
%\fi
%    \begin{macrocode}
\def\version{final}
\input{childdoc.def}
\childdocforwardprefix[cdocsamp]{cdocsfn}{cdocsch}
%    \end{macrocode}

%\iffalse
%</samplefinal>
%\fi
%
% %%%%%%%%%%%%%%%%%%%%%%%%%%%%%%%%%%%%%%
% \paragraph{Command Line Processing.}
%
% The following three command lines generate the output files
% |cdocscld|, |cdocscl1| and |cdocscl2|
% which should be identical to
% |cdocsdrf|, |cdocsch1| and |cdocsfn2|, respectively:
% \begin{center}
% \begin{tabular}{l}
% |latex -jobname cdocscld \|\\
% |  "\def\version{draft}\input{childdoc.def}\childdocforward{cdocsamp}"|\\
% |latex -jobname cdocscl1 \|\\
% |  "\input{childdoc.def}\childdocforward[cdocsamp]{cdocsch1}"|\\
% |latex -jobname cdocscl2 \|\\
% |  "\def\version{final}\input{childdoc.def}\childdocforward{cdocsch2}"|
% \end{tabular}
% \end{center}
% Note that the trailing backslash on each first line
% merely continues the input to the second line
% (for convenient cut ant paste).
% Furthermore, the command |latex| can be replaced by any
% of its alternative versions such as |pdflatex|.
%
% %%%%%%%%%%%%%%%%%%%%%%%%%%%%%%%%%%%%%%%%%%%%%%%%%%%%%%%%%%%%%%%%%%%%%%%%%%%%%%
% %%%%%%%%%%%%%%%%%%%%%%%%%%%%%%%%%%%%%%%%%%%%%%%%%%%%%%%%%%%%%%%%%%%%%%%%%%%%%%
% \section{Implementation}
%\iffalse
%<*package>
%\fi
%
% This section describes the definitions file |childdoc.def|.

% The definitions cannot be loaded using |\usepackage| or |\RequirePackage|
% which has a mechanism to prevent loading a style file more than once.
% When loading the definitions by means of |\input|
% multiple instances have to be prevented manually:
%\iffalse
%This code needs to be before the `\ProvidesFile' directive
%which is defined at the beginning of this file.
%Therefore it is also placed there and commented out here.
%</package>
%<*discard>
%\fi
%    \begin{macrocode}
\ifdefined\childdocmain\endinput\fi
%    \end{macrocode}
%\iffalse
%</discard>
%<*package>
%\fi
%
% \macro{\ifchilddoc}
% \macro{\ifchilddocmanual}
% The conditional |\ifchilddoc| tells whether a
% child (true) or main (false) document is being compiled.
% The conditional |\ifchilddocmanual| tells whether
% the |\includeonly| mechanism is used (false) or
% the selection of child files must be performed manually (true).
% The definitions initialise to false:
%    \begin{macrocode}
\newif\ifchilddoc
\newif\ifchilddocmanual
%    \end{macrocode}

% \macro{\childdocname}
% \macro{\childdocjob}
% The macro |\childdocname| stores the name of the main document
% to be compiled. The macro |\childdocjob| stores the name of
% the document on which the \LaTeX{} compiler was originally invoked.
% The content of |\jobname| cannot be compared
% to filenames specified in the source due to different catcodes.
% The following code rescans |\jobname|, stores the result
% in |\childdocname| and saves a copy in |\childdocjob|:
%    \begin{macrocode}
\edef\childdocname{\scantokens\expandafter{\jobname\noexpand}}
\let\childdocjob\childdocname
%    \end{macrocode}

% \macro{\childdocdisable}
% The macro |\childdocdisable| prevents the main file
% from being processed more than once.
% At this stage, the main document command |\childdocmain|
% is assumed to be called once again where it should do nothing.
% Any subsequent call to it should prevent
% a secondary processing of the main document
% It overwrites the forwarding commands
% |\childdocof| and |\childdocforward|
% with empty macros to prevent further inclusions of the main document:
%    \begin{macrocode}
\newcommand{\childdocdisable}
{
  \renewcommand{\childdocmain}[1]{\renewcommand{\childdocmain}[1]{\endinput}}
  \renewcommand{\childdocof}[1]{}
  \renewcommand{\childdocby}[2][]{}
  \renewcommand{\childdocforward}[2][]{}
  \renewcommand{\childdocdisable}{}
}
%    \end{macrocode}

% \macro{\childdocmain}
% The macro |\childdocmain| is to be called at the top of the main file
% with nothing or the main filename (without extension) as argument.
% First, it breaks loops.
% If the argument is not empty and does not match |\childdocname|
% (which is set by the first inclusion of |childdoc.def|),
% |\ifchilddoc| is set to true, |\includeonly| is applied to the child file
% and |\jobname| is set to the main file
% (for proper handling of |.aux| files):
%    \begin{macrocode}
\newcommand{\childdocmain}[1]
{
  \childdocdisable\childdocmain{}
  \if?#1?\else
    \begingroup
      \def\childdoctmp{#1}
      \ifx\childdoctmp\childdocname
        \def\childdoctmp{}
      \else
        \def\childdoctmp
        {
          \childdoctrue
          \includeonly{\childdocname}
          \def\childdocjob{#1}
          \def\jobname{#1}
        }
      \fi
      \expandafter
    \endgroup
    \childdoctmp
  \fi
}
%    \end{macrocode}

% \macro{\childdocof}
% The command |\childdocof| redirects
% compilation to the main file |#1|.
%    \begin{macrocode}
\newcommand{\childdocof}[1]
{
  \childdocdisable
  \childdoctrue
  \includeonly{\childdocname}
  \def\jobname{#1}
  \def\childdocjob{#1}
  \input{#1}
}
%    \end{macrocode}

% \macro{\childdocby}
% The command |\childdocby| ....
%    \begin{macrocode}
\newcommand{\childdocby}[2][]
{
  \childdocdisable
  \childdoctrue
  \childdocmanualtrue
  \if?#1?\else
    \def\jobname{#2}
  \fi
  \def\childdocjob{#2}
  \input{#2}
  \endinput
}
%    \end{macrocode}

% \macro{\childdocforward}
% The command |\childdocforward| redirects
% compilation to the main file or
% (if the optional argument is given) a child file.
% Parameters are set as if the main file
% or a child file starting with |\childdocof| was compiled.
% Then compilation is handed over to the main file:
%    \begin{macrocode}
\newcommand{\childdocforward}[2][]
{
  \begingroup
    \if?#1?
      \def\childdoctmp
      {
        \def\childdocname{#2}
        \def\childdocjob{#2}
        \def\jobname{#2}
        \input{#2}
        \endinput
      }
    \else
      \def\childdoctmp
      {
        \childdocdisable
        \def\childdocname{#2}
        \childdoctrue
        \includeonly{#2}
        \def\childdocjob{#1}
        \def\jobname{#1}
        \input{#1}
        \endinput
      }
    \fi
    \expandafter
  \endgroup
  \childdoctmp
}
%    \end{macrocode}

% \macro{\childdocforwardprefix}
% The command |\childdocforwardprefix| redirects
% compilation to the main or a child file by means of a pattern.
% The prefix |#1| in the current filename is replaced by |#2|
% and the suffix of the current filename is kept
% (it is assumed that the filename does not contain the substring `|~~~|'
% which is used as a delimiter).
% Compilation is handed over to the new file by |\childdocforward|:
%    \begin{macrocode}
\newcommand{\childdocforwardprefix}[3][]
{
  \begingroup
    \def\childdocextract #2##1~~~{\def\childdoctmp{\childdocforward[#1]{#3##1}}}
    \expandafter\childdocextract\childdocname~~~
    \expandafter
  \endgroup
  \childdoctmp
}
%    \end{macrocode}

% \macro{\childdoc}
% The deprecated macro |\childdoc| is a legacy version of |\childdocmain|:
%    \begin{macrocode}
\newcommand{\childdoc}{\childdocmain}
%    \end{macrocode}

% \macro{\childdocredirect}
% The deprecated macro |\childdocredirect| is a legacy version
% of |\childdocforward| and |\childdocforwardprefix|:
%    \begin{macrocode}
\newcommand{\childdocredirect}[2][]
{
  \begingroup
    \if?#1?
      \def\childdoctmp{\childdocforward{#2}}
    \else
      \def\childdoctmp{\childdocforwardprefix{#1}{#2}}
    \fi
    \expandafter
  \endgroup
  \childdoctmp
}
%    \end{macrocode}

%\iffalse
%</package>
%\fi
%
\endinput

\childdocby{cdocsamp}
%    \end{macrocode}

%\iffalse
%</samplepart3|samplepart4>
%\fi
%
%\iffalse
%<*samplepart3>
%\fi
% Some text for part 3:
%    \begin{macrocode}
some text in part three
%    \end{macrocode}

%\iffalse
%</samplepart3>
%\fi
% Some text for part 4:
%\iffalse
%<*samplepart4>
%\fi
%    \begin{macrocode}
more text in part four
%    \end{macrocode}

%\iffalse
%</samplepart4>
%\fi
%
% %%%%%%%%%%%%%%%%%%%%%%%%%%%%%%%%%%%%%%
% \paragraph{Forwarding for a Complete Draft.}
%
% The following forwarding file |cdocsdrf.tex|
% compiles the main document in draft mode:
%\iffalse
%<*sampledraft>
%\fi
%    \begin{macrocode}
\def\version{draft}
% \iffalse
%
% childdoc.dtx Copyright (C) 2017-2018 Niklas Beisert
%
% This work may be distributed and/or modified under the
% conditions of the LaTeX Project Public License, either version 1.3
% of this license or (at your option) any later version.
% The latest version of this license is in
%   http://www.latex-project.org/lppl.txt
% and version 1.3 or later is part of all distributions of LaTeX
% version 2005/12/01 or later.
%
% This work has the LPPL maintenance status `maintained'.
%
% The Current Maintainer of this work is Niklas Beisert.
%
% This work consists of the files childdoc.dtx and childdoc.ins
% and the derived files childdoc.def and cdocsamp.tex with
% cdocsch1.tex, cdocsch2.tex, cdocsdrf.tex, cdocsfn1.tex, cdocsfn2.tex.
%
%<package>\ifdefined\childdocmain\endinput\fi
%<package>\ProvidesFile{childdoc.def}[2018/12/30 v2.0 child document driver]
%<samplemain>\ProvidesFile{cdocsamp.tex}[2018/12/30 v2.0 sample for childdoc]
%<*driver>
%\ProvidesFile{childdoc.drv}[2018/12/30 v2.0 childdoc reference manual file]
\PassOptionsToClass{10pt,a4paper}{article}
\documentclass{ltxdoc}

\usepackage[margin=35mm]{geometry}
\usepackage{hyperref}
\usepackage{hyperxmp}
\usepackage[usenames]{color}

\hypersetup{colorlinks=true}
\hypersetup{pdfstartview=FitH}
\hypersetup{pdfpagemode=UseNone}
\hypersetup{pdfsource={}}
\hypersetup{pdflang={en-UK}}
\hypersetup{pdfcopyright={Copyright 2017-2018 Niklas Beisert.
  This work may be distributed and/or modified under the
  conditions of the LaTeX Project Public License, either version 1.3
  of this license or (at your option) any later version.}}
\hypersetup{pdflicenseurl={http://www.latex-project.org/lppl.txt}}
\hypersetup{pdfcontactaddress={ETH Zurich, ITP, HIT K,
  Wolfgang-Pauli-Strasse 27}}
\hypersetup{pdfcontactpostcode={8093}}
\hypersetup{pdfcontactcity={Zurich}}
\hypersetup{pdfcontactcountry={Switzerland}}
\hypersetup{pdfcontactemail={nbeisert@itp.phys.ethz.ch}}
\hypersetup{pdfcontacturl={http://people.phys.ethz.ch/\xmptilde nbeisert/}}

\newcommand{\secref}[1]{\hyperref[#1]{section \ref*{#1}}}

\parskip1ex
\parindent0pt
\let\olditemize\itemize
\def\itemize{\olditemize\parskip0pt}

\begin{document}

\title{The \textsf{childdoc} Package}
\hypersetup{pdftitle={The childdoc Package}}
\author{Niklas Beisert\\[2ex]
  Institut f\"ur Theoretische Physik\\
  Eidgen\"ossische Technische Hochschule Z\"urich\\
  Wolfgang-Pauli-Strasse 27, 8093 Z\"urich, Switzerland\\[1ex]
  \href{mailto:nbeisert@itp.phys.ethz.ch}
  {\texttt{nbeisert@itp.phys.ethz.ch}}}
\hypersetup{pdfauthor={Niklas Beisert}}
\hypersetup{pdfsubject={Manual for the LaTeX2e Package childdoc}}
\date{30 December 2018, \textsf{v2.0}}
\maketitle

\begin{abstract}\noindent
\textsf{childdoc} is a \LaTeXe{} package
that enables the direct compilation
of document sections included by |\include|
to individual files.
\end{abstract}

\begingroup
\parskip0ex
\tableofcontents
\endgroup

%%%%%%%%%%%%%%%%%%%%%%%%%%%%%%%%%%%%%%%%%%%%%%%%%%%%%%%%%%%%%%%%%%%%%%%%%%%%%%%%
%%%%%%%%%%%%%%%%%%%%%%%%%%%%%%%%%%%%%%%%%%%%%%%%%%%%%%%%%%%%%%%%%%%%%%%%%%%%%%%%
\section{Introduction}

\LaTeX{} provides a mechanism to structure a large document (such as a book)
into a main file and several child files (containing the chapters)
using the |\include| command.
This mechanism is beneficial for documents
which span hundreds of pages in order to
make the source file(s) more manageable.
Moreover, compilation can be restricted to
selected child files by means of the |\includeonly| command.
The latter feature can be used to reduce the compilation time while editing
(this was significantly more useful in the earlier days of \LaTeX{})
or to generate a smaller document which is easier to navigate.
Another application of |\includeonly| is to generate
documents consisting of selected parts of the complete document.

However, there are a few drawbacks of the plain |\include| mechanism:
\begin{itemize}
\item
The child files cannot be compiled on their own,
they can only be compiled via the main file.
A naive editing environment
(such as a text editor with an option
to have the current file processed by \LaTeX)
may require one to switch to the main file before compiling;
attempting to compile the child file produces errors.
\item
The main file must be modified (each time)
to adjust the |\includeonly| command
to the present needs. This easily leaves the main file in a messy state.
\item
The generated document will always carry the filename
of the main document. This is inconvenient if
several child files are to be compiled and
to be kept for distribution.
\end{itemize}

The present package provides a simple interface
to make child files individually compilable by \LaTeX{}.
Compiling a child file then has the same effect as compiling
the main file with an |\includeonly| command
to select the appropriate child.
Moreover the generated document will carry the name of the child
rather than the main file.
This resolves all three above issues.

This feature is meant to make the editing of books,
thesis documents and lecture notes somewhat more convenient.
However, the package can also be used efficiently for
composing a series of documents (such as exercise sheets)
which are typically distributed individually.
It then assists the author in generating the individual documents
(potentially in different versions)
as well as a document containing the collected series.
Another application is in developing style files
or other kinds of included material
where compilation of the style file could redirect
to a sample or test file.

%%%%%%%%%%%%%%%%%%%%%%%%%%%%%%%%%%%%%%%%%%%%%%%%%%%%%%%%%%%%%%%%%%%%%%%%%%%%%%%%
%%%%%%%%%%%%%%%%%%%%%%%%%%%%%%%%%%%%%%%%%%%%%%%%%%%%%%%%%%%%%%%%%%%%%%%%%%%%%%%%
\section{Usage}

First of all, the package \textsf{childdoc} is \emph{not} a standard
\LaTeXe{} |.sty| style file! Therefore it needs to be invoked in
a non-standard way.

%%%%%%%%%%%%%%%%%%%%%%%%%%%%%%%%%%%%%%%%%%%%%%%%%%%%%%%%%%%%%%%%%%%%%%%%%%%%%%%%
\subsection{Included Files}
\label{sec:include}

%%%%%%%%%%%%%%%%%%%%%%%%%%%%%%%%%%%%%%%%
\DescribeMacro{\childdocmain}
To use the package, add the commands
\begin{center}
\begin{tabular}{l}
|\input{childdoc.def}|\\
|\childdocmain{}|\\
\end{tabular}
\end{center}
at the very top of the main \LaTeX{} file,
in particular \emph{before} the |\documentclass| statement!
The argument of |\childdocmain| should be left empty
(but it must be present).

%%%%%%%%%%%%%%%%%%%%%%%%%%%%%%%%%%%%%%%%
\DescribeMacro{\childdocof}
Furthermore, add the commands
\begin{center}
\begin{tabular}{l}
|\input{childdoc.def}|\\
|\childdocof{|\textit{main}|}|\\
\end{tabular}
\end{center}
at the top of every child file \textit{child}
which is included by |\include{|\textit{child}|}|
from within the main file
(or at least for those files to be compiled individually).
The argument \textit{main} must be the filename of the main file.

There are a couple of
considerations in setting up the main and child documents:

%%%%%%%%%%%%%%%%%%%%%%%%%%%%%%%%%%%%%%%%
\paragraph{Restrictions.}

Please note the following restrictions:
\begin{itemize}
\item
|\childdocmain| must be called with one argument \textit{main}
to ensure compatibility with earlier version of the package.
It must either be empty (|\childdocmain{}|)
or precisely match the filename of the main file in which it is specified.
See \secref{sec:detection} for further information.
\item
The filename \textit{main} must be specified without the |.tex| extension.
\item
The filename \textit{main} is case sensitive
(even in case-insensitive file systems)
due to internal string comparison.
\item
The argument \textit{main} should be fully expanded, it cannot be a macro.
\item
Subdirectories and special characters should be avoided in filenames.
\item
The command |\childdocmain{|\textit{main}|}| must be followed by a whitespace.
It should not be followed immediately by another command
or by a comment mark `|%|'.
This is because the \TeX{} parser reads the token immediately following
the argument of |\childdocmain| and puts it
at the beginning of every child section;
however, a white\-space is ignored.
\end{itemize}

%%%%%%%%%%%%%%%%%%%%%%%%%%%%%%%%%%%%%%%%
\paragraph{Content of Main File.}

It is advisable to place all content in the child files included by |\include|.
Any output contained in the main file will appear in all child documents
unless suppressed manually;
it cannot be suppressed automatically by the |\includeonly| directive
and thus should normally be avoided.
A method to include some content in the main file
by means of conditional processing is described in \secref{sec:conditional}.

%%%%%%%%%%%%%%%%%%%%%%%%%%%%%%%%%%%%%%%%
\paragraph{Page Numbering.}

When only a part of the document is compiled,
the appropriate numbering of pages
(as well as other status parameters)
is determined from the |.aux| files.
The latter contain information from previous passes.
However this information needs to propagate through
all intermediate child documents.
Therefore the page numbering in child documents may well
be inconsistent until the complete document is compiled at least once.

A useful (if unconventional) way to always ensure a consistent
page numbering is to restart the numbering in each child document
and denote the pages by `\textit{child}|.|\textit{page}'
where \textit{child} represents the chapter/section number of the child file.
This can be achieved by the command
|\numberwithin{page}{|\textit{child}|}|
of the \textsf{amsmath} package
where \textit{child} can be |chapter| or |section|
depending on the chosen structuring.
Alternatively, one can modify the macro |\thepage| appropriately
and reset the counter |page| at the start of each child file.

%%%%%%%%%%%%%%%%%%%%%%%%%%%%%%%%%%%%%%%%%%%%%%%%%%%%%%%%%%%%%%%%%%%%%%%%%%%%%%%%
\subsection{Conditional Processing}
\label{sec:conditional}

The package provides a mechanism to compile different versions
of a document. To customise the versions further some conditional processing
can come in handy to distinguish which version is being compiled.
The package provides two macros to describe the compilation context:

%%%%%%%%%%%%%%%%%%%%%%%%%%%%%%%%%%%%%%%%
\DescribeMacro{\ifchilddoc}
The conditional |\ifchilddoc| distinguishes between the compilation of
child documents and the main document:
%
\begin{center}
|\ifchilddoc |\textit{child-code}| |[|\||else |\textit{main-code}]| \||fi|
\end{center}

%%%%%%%%%%%%%%%%%%%%%%%%%%%%%%%%%%%%%%%%
\DescribeMacro{\childdocname}
\DescribeMacro{\childdocjob}
The macro |\childdocname| contains the filename (without extension)
of the main or child file being processed.
Note that |\childdocjob| will always contain the name of the main file.

%%%%%%%%%%%%%%%%%%%%%%%%%%%%%%%%%%%%%%%%
\paragraph{Title Page.}

Conditional processing can be used to include a title or banner page
in the main document when proper precautions are taken.
Importantly, the code in the main file should ensure that the page counter
(as well as other status parameters which are stored in the |.aux| files)
takes the same value after the conditional processing.
Otherwise the page numbers may take divergent values
depending on which part is compiled.

For example, a title page could be declared by:
%
\begin{center}
\begin{tabular}{l}
|\ifchilddoc\||else|\\
|\addtocounter{page}{-1}|\\
\textit{code for title page}\\
|\newpage|\\
|\||fi|
\end{tabular}
\end{center}
%
A banner page for the child documents can be generated by:
%
\begin{center}
\begin{tabular}{l}
|\ifchilddoc|\\
|\addtocounter{page}{-1}|\\
\textit{code for banner page}\\
|\newpage|\\
|\||fi|
\end{tabular}
\end{center}
%
Here one could write a message such as:
\begin{center}
|This is the part \childdocname{} of \childdocjob{}.|
\end{center}

%%%%%%%%%%%%%%%%%%%%%%%%%%%%%%%%%%%%%%%%%%%%%%%%%%%%%%%%%%%%%%%%%%%%%%%%%%%%%%%%
\subsection{Flags}
\label{sec:flags}

The package makes it easy to generate different versions
of the main or child documents.
To this end compilation flags can be defined
and assigned different default values.
They will be particularly useful in conjunction
with the forwarding mechanism described in \secref{sec:forward}.

For example, it may be useful to have a flag |\version|
which can be set to |draft| or |final|.
The document source will contain some conditional code
depending on the value of |\version|.
Suppose further, the flag should default to |final| for the main file
and to |draft| for child files
which is a natural assignment for editing the document.
This is achieved by placing the following code
in the preamble of the main document
(below the |\childdocmain| directive):
%
\begin{center}
\begin{tabular}{l}
|\ifchilddoc|\\
|\providecommand{\version}{draft}|\\
|\||else|\\
|\providecommand{\version}{final}|\\
|\||fi|
\end{tabular}
\end{center}
%
The definition by |\providecommand| makes sure
that previous definitions are not overwritten.
Further statements |\providecommand{\version}{...}|
can thus be added before the above code to override it.

For the main file, one might add a line
(between |\childdocmain| and the above block)
%
\begin{center}
|%\ifchilddoc\||else\providecommand{\version}{draft}\||fi|
\end{center}
%
which can be uncommented to produce a draft version.
Likewise one can add a line to the very top of a child file
(above the |\childdocof{|\textit{main}|}| directive)
%
\begin{center}
|%\providecommand{\version}{final}|
\end{center}
%
which can be uncommented to produce the final version of this child document.

%%%%%%%%%%%%%%%%%%%%%%%%%%%%%%%%%%%%%%%%%%%%%%%%%%%%%%%%%%%%%%%%%%%%%%%%%%%%%%%%
\subsection{Forwarding}
\label{sec:forward}

Different versions of the main or child documents
using compilation flags as described in \secref{sec:flags}
can be (permanently) stored in different files
for convenient compilation, viewing and distribution.
To this end, the package defines a command
to pass on compilation to a different file:

%%%%%%%%%%%%%%%%%%%%%%%%%%%%%%%%%%%%%%%%
\DescribeMacro{\childdocforward}
The command |\childdocforward| redirects processing to
another source file:
%
\begin{center}
\begin{tabular}{l}
|\input{childdoc.def}|\\
|\childdocforward[|\textit{main}|]{|\textit{dest}|}|\\
\end{tabular}
\end{center}
%
The argument \textit{dest} is the destination file
(without extension).
It should be the main file or one of the child files.
Note that further \textsf{childdoc} directives
such as |\childdocof| and |\childdocforward|
in the indicated file will be processed in this form.
The optional argument \textit{main}
passes on directly to the main file \textit{main}
while pretending to compile the child \textit{dest}.
This form behaves as if \textit{dest}
issues |\childdocof{|\textit{main}|}| right away,
and no further \textsf{childdoc} directives will be processed.

%%%%%%%%%%%%%%%%%%%%%%%%%%%%%%%%%%%%%%%%
\DescribeMacro{\...prefix}
In the alternative form |\childdocforwardprefix|,
%
\begin{center}
\begin{tabular}{l}
|\input{childdoc.def}|\\
|\childdocforwardprefix[|\textit{main}|]{|\textit{prefix}|}{|\textit{dest}|}|
\end{tabular}
\end{center}
%
the destination file is determined by a pattern
depending on the current file:
To make this work, the current file must be called
`{\textit{prefix}\hspace{0.2em}\textit{suffix}}'
with \textit{prefix} matching precisely the argument.
Processing is then passed on to the file
`{\textit{dest}\hspace{0.2em}\textit{suffix}}'.
Surely, the same effect is achieved by
directly specifying the
argument `{\textit{dest}\hspace{0.2em}\textit{suffix}}'
in the first form.
However, that requires to set up a different file
for each child. With the alternative form of the command
all these files can have exactly the same content
which simplifies setting them up and maintaining them.

For example, the following file |draft.tex|
with a compilation flag |\version| as described in \secref{sec:flags}
compiles the main document as a draft:
%
\begin{center}
\begin{tabular}{l}
|\def\version{draft}|\\
|\input{childdoc.def}|\\
|\childdocforward{|\textit{main}|}|
\end{tabular}
\end{center}
%
Likewise, the following files |final|\textit{nn}|.tex|
compile the final version of the child document
|child|\textit{nn}|.tex|:
%
\begin{center}
\begin{tabular}{l}
|\def\version{final}|\\
|\input{childdoc.def}|\\
|\childdocforwardprefix{final}{child}|
\end{tabular}
\end{center}
%

Note that when several versions of a main file and/or of each child file
are to be generated, it may be convenient to set up a |Makefile| or
shell script to automatise the process.

%%%%%%%%%%%%%%%%%%%%%%%%%%%%%%%%%%%%%%%%%%%%%%%%%%%%%%%%%%%%%%%%%%%%%%%%%%%%%%%%
\subsection{Command Line Processing}
\label{sec:commandline}

The effect of redirection files can also be achieved by invoking
the \LaTeX{} compiler with a more elaborate command line.
Most conveniently this should be done as part
of a shell script or a |Makefile|.

When using \textsf{childdoc} in the main file, the following
command lines effectively perform a redirection
(note that depending on the shell being used,
backslashes may have to be doubled: `|\|' $\to$ `|\\|'):
%
\begin{center}
|... -jobname "|\textit{target}|" |\\|"|[\textit{flags}]%
|\input{childdoc.def}\childdocforward[|\textit{main}|]{|\textit{dest}|}"|
\end{center}
%
Here \textit{target} is the name of the output file,
\textit{main} is the name of the main file
and \textit{dest} is the name of the main or child file to be processed
(all filenames without extensions).
The optional argument \textit{main} can be omitted
if \textit{main} matches \textit{dest}.
Optionally, compilation \textit{flags} can be defined via |\def| commands.
This command line makes the \TeX{} engine believe
it is compiling the file \textit{target}
whose content is specified as the latter parameter.
The provided code then forwards the processing to
\textit{main} or \textit{dest} as described in \secref{sec:forward}.

%%%%%%%%%%%%%%%%%%%%%%%%%%%%%%%%%%%%%%%%%%%%%%%%%%%%%%%%%%%%%%%%%%%%%%%%%%%%%%%%
\subsection{Include by Input}
\label{sec:input}

Including child documents by |\include| has some restrictions by design.
Most notably, the content of a child document always occupies
its own set of pages; pages cannot be shared between child documents.
Usually, this behaviour makes perfect sense
because each child document contain an essential part of the document.
However, in some situations it may be desirable to compose
a document from a collection of parts
without having mandatory page breaks between then.
For this case, the package
provides a mechanism to include parts
by |\input| which can also be processed individually.
However, by construction this mechanism
requires manual handling of the content to be output.

%%%%%%%%%%%%%%%%%%%%%%%%%%%%%%%%%%%%%%%%
\DescribeMacro{\ifchilddocmanual}
The main file should be prepared as usual, see \secref{sec:include}.
However, the document body must make a distinction
between processing of an individual part and of the main document, e.g.:
%
\begin{center}
\begin{tabular}{l}
|\ifchilddocmanual|\\
|\input{\childdocname}|\\
|\||else|\\
\textit{document body with }|\input{|\textit{part}|}|\\
|\||fi|
\end{tabular}
\end{center}
%
The conditional |\ifchilddocmanual| is true whenever
a part to be included by |\input| is being compiled,
and the name of the part is stored in |\childdocname|.

%%%%%%%%%%%%%%%%%%%%%%%%%%%%%%%%%%%%%%%%
\DescribeMacro{\childdocby}
Each part to be included by |\input| should start with:
%
\begin{center}
\begin{tabular}{l}
|\input{childdoc.def}|\\
|\childdocby{|\textit{main}|}|\\
\end{tabular}
\end{center}
%
The directive |\childdocby| is similar to |\childdocof|
described in \secref{sec:include},
but the subsequent selection of content must be done manually.
To that end, both |\ifchilddoc| and |\ifchilddocmanual|
will be true upon processing of a part,
and the name of the part is stored in |\childdocname|.
Note that |\jobname| will be set to the filename of the current part
so that each part receives an individual |.aux| file
that does not interfere with the |.aux| file(s) of the main document.
This behaviour can be altered by the alternative form
|\childdocby[*]{|\textit{main}|}| (with a non-empty optional argument)
which uses the |.aux| file of the main document
by setting |\jobname| to \textit{main}.

%%%%%%%%%%%%%%%%%%%%%%%%%%%%%%%%%%%%%%%%%%%%%%%%%%%%%%%%%%%%%%%%%%%%%%%%%%%%%%%%
\subsection{Driver Development}
\label{sec:driver}

The \textsf{childdoc} mechanism can also be use for the development
of definition files such as \LaTeX{} styles or classes.
This case differs from the above setup with multiple parts
included by |\include| in that no |\includeonly| should be invoked.
This can be achieved by starting the include file
(before |\ProvidesPackage|) with:
%
\begin{center}
\begin{tabular}{l}
|\input{childdoc.def}|\\
|\childdocforward{|\textit{main}|}|\\
\end{tabular}
\end{center}
%
or alternatively with:
%
\begin{center}
\begin{tabular}{l}
|\input{childdoc.def}|\\
|\childdocby{|\textit{main}|}|\\
\end{tabular}
\end{center}
%
Both forms have slightly different effects as described above.
The main file is prepared as usual, see \secref{sec:include}.

%%%%%%%%%%%%%%%%%%%%%%%%%%%%%%%%%%%%%%%%%%%%%%%%%%%%%%%%%%%%%%%%%%%%%%%%%%%%%%%%
\subsection{Legacy Detection}
\label{sec:detection}

The directive |\childdocmain| in the main file can detect
whether the complete document or merely a child is to be compiled
even without using the directive |\childdocof|.
This method is deprecated because it is less robust
and there is no compelling reason to use it;
it is merely provided for backward compatibility
and it may be removed in future versions.

If the detection mechanism is to be used,
it is mandatory to correctly specify
the filename of the main file as the argument of |\childdocmain|:
%
\begin{center}
\begin{tabular}{l}
|\input{childdoc.def}|\\
|\childdocmain{|\textit{main}|}|\\
\end{tabular}
\end{center}
%
If |\jobname| does not match the argument \textit{main} of |\childdocmain|,
it is assumed that |\jobname| points to the child file to be compiled.
When using |\childdocmain| with the main file specified as argument,
it suffices to start a child file
with just |\input{|\textit{main}|}|
without loading of the package and using |\childdocof|.
If instead all processing is done
with the appropriate \textsf{childdoc} directives,
the argument of \textit{main} of |\childdocmain| can be empty.

An alternative version of the command line processing described
in \secref{sec:commandline} using the detection mechanism reads:
%
\begin{center}
|... -jobname "|\textit{target}|" "|[\textit{flags}]%
[|\def\jobname{|\textit{dest}|}|]|\input{|\textit{main}|}"|
\end{center}

%%%%%%%%%%%%%%%%%%%%%%%%%%%%%%%%%%%%%%%%%%%%%%%%%%%%%%%%%%%%%%%%%%%%%%%%%%%%%%%%
\subsection{Manual Code}
\label{sec:manual}

In case one cannot be certain whether the definitions file |childdoc.def|
is installed on the target \TeX{} distribution
and one prefers not to ship it,
it is conceivable to paste a few relevant commands into the sources.

To that end, drop all statements |\input{childdoc.def}|
and perform the replacements as outlined below.
Instead of |\childdocmain{|\textit{main}|}| add the following code
to the top of the main file:
%
\begin{center}
\begin{tabular}{l}
|\||ifdefined\childdocname\endinput\||fi\newif\ifchilddoc|\\
|\edef\childdocname{\scantokens\expandafter{\jobname\noexpand}}|\\
|\def\childdocmain{|\textit{main}|}\||ifx\childdocmain\childdocname\||else|\\
|\childdoctrue\includeonly{\childdocname}\let\jobname\childdocmain\||fi|\\
\end{tabular}
\end{center}
%
Instead of |\childdocof{|\textit{main}|}| just include the main file
at the top of each child file:
%
\begin{center}
|\input{|\textit{main}|}|
\end{center}
%
A simple redirection |\childdocforward{|\textit{dest}|}| is achieved by:
%
\begin{center}
|\def\jobname{|\textit{dest}|}\input{\jobname}|
\end{center}
%
The redirection with prefix
|\childdocforwardprefix[|\textit{prefix}|]{|\textit{dest}|}|
is accomplished by:
%
\begin{center}
\begin{tabular}{l}
|{\edef\jobname{\scantokens\expandafter{\jobname\noexpand}}|\\
|\def\redirectjob |\textit{prefix}|#1~~~{\gdef\jobname{|\textit{dest}|#1}}|\\
|\expandafter\redirectjob\jobname~~~}\input{\jobname}|
\end{tabular}
\end{center}

In an alternative approach,
child documents can be compiled by a specific command line
without additional code or specific definitions:
%
\begin{center}
|... -jobname "|\textit{target}|" "|[\textit{flags}]%
|\includeonly{|\textit{dest}|}\input{|\textit{main}|}"|
\end{center}
%

%%%%%%%%%%%%%%%%%%%%%%%%%%%%%%%%%%%%%%%%%%%%%%%%%%%%%%%%%%%%%%%%%%%%%%%%%%%%%%%%
%%%%%%%%%%%%%%%%%%%%%%%%%%%%%%%%%%%%%%%%%%%%%%%%%%%%%%%%%%%%%%%%%%%%%%%%%%%%%%%%
\section{Information}

%%%%%%%%%%%%%%%%%%%%%%%%%%%%%%%%%%%%%%%%%%%%%%%%%%%%%%%%%%%%%%%%%%%%%%%%%%%%%%%%
\subsection{Copyright}

Copyright \copyright{} 2017--2018 Niklas Beisert

This work may be distributed and/or modified under the
conditions of the \LaTeX{} Project Public License, either version 1.3
of this license or (at your option) any later version.
The latest version of this license is in
  \url{http://www.latex-project.org/lppl.txt}
and version 1.3 or later is part of all distributions of \LaTeX{}
version 2005/12/01 or later.

This work has the LPPL maintenance status `maintained'.

The Current Maintainer of this work is Niklas Beisert.

This work consists of the files |README.txt|, |childdoc.ins| and |childdoc.dtx|
as well as the derived files |childdoc.def|, |cdocsamp.tex|
with |cdocsch1.tex|, |cdocsch2.tex|, |cdocspt3.tex|, |cdocspt4.tex|,
|cdocsdrf.tex|, |cdocsfn1.tex|, |cdocsfn2.tex|
as well as |childdoc.pdf|.

%%%%%%%%%%%%%%%%%%%%%%%%%%%%%%%%%%%%%%%%%%%%%%%%%%%%%%%%%%%%%%%%%%%%%%%%%%%%%%%%
\subsection{Files and Installation}

The package consists of the files:
%
\begin{center}
\begin{tabular}{ll}
    |README.txt|   & readme file \\
    |childdoc.ins| & installation file \\
    |childdoc.dtx| & source file \\
    |childdoc.def| & definition file \\
    |cdocsamp.tex| & sample main file \\
    |cdocsch1.tex| & sample include file \\
    |cdocsch2.tex| & sample include file \\
    |cdocspt3.tex| & sample part file \\
    |cdocspt4.tex| & sample part file \\
    |cdocsdrf.tex| & sample redirection file \\
    |cdocsfn1.tex| & sample redirection file \\
    |cdocsfn2.tex| & sample redirection file \\
    |childdoc.pdf| & manual
\end{tabular}
\end{center}
%
The distribution consists of the files
|README.txt|, |childdoc.ins| and |childdoc.dtx|.
%
\begin{itemize}
\item
Run (pdf)\LaTeX{} on |childdoc.dtx|
to compile the manual |childdoc.pdf| (this file).
\item
Run \LaTeX{} on |childdoc.ins| to create the definitions file |childdoc.def|
and the sample |cdocsamp.tex| with include files
|cdocsch1.tex|, |cdocsch2.tex|, |cdocspt3.tex|, |cdocspt4.tex|,
|cdocsdrf.tex|, |cdocsfn1.tex|, |cdocsfn2.tex|.
Then copy the file |childdoc.def| to an appropriate directory of your \LaTeX{}
distribution, e.g.\ \textit{texmf-root}|/tex/latex/childdoc|.
\end{itemize}

%%%%%%%%%%%%%%%%%%%%%%%%%%%%%%%%%%%%%%%%%%%%%%%%%%%%%%%%%%%%%%%%%%%%%%%%%%%%%%%%
\subsection{Related CTAN Packages}

There are several other packages which offer a similar functionality:
%
\begin{itemize}
\item
The packages
\href{http://ctan.org/pkg/docmute}{\textsf{docmute}},
\href{http://ctan.org/pkg/includex}{\textsf{includex}} and
\href{http://ctan.org/pkg/standalone}{\textsf{standalone}}
provide commands to include only the document body of
a child file thus allowing both files to be compiled individually.
\item
The packages \href{http://ctan.org/pkg/subdocs}{\textsf{subdocs}}
and \href{http://ctan.org/pkg/subfiles}{\textsf{subfiles}}
provide structures in which the main and child documents can be
encapsulated and allowing them to be compiled individually.
The inclusion mechanism is different from the conventional |\include|.
\item
The package \href{http://ctan.org/pkg/combine}{\textsf{combine}}
is an elaborate solution to combine several documents into one.
\end{itemize}
%
See also the CTAN topic \href{http://ctan.org/topic/subdocs}{\textsf{subdocs}}
for further related packages.
The present package differs from the above solutions in that
a document structure constructed with the conventional |\include| mechanism
just needs two extra commands at the top of every file
such that all constituent files can be compiled individually.

%%%%%%%%%%%%%%%%%%%%%%%%%%%%%%%%%%%%%%%%%%%%%%%%%%%%%%%%%%%%%%%%%%%%%%%%%%%%%%%%
%\subsection{Feature Suggestions}
%
%The following is a list of features which may be useful for future
%versions of this package:
%%
%\begin{itemize}
%\item
%\ldots
%\end{itemize}

%%%%%%%%%%%%%%%%%%%%%%%%%%%%%%%%%%%%%%%%%%%%%%%%%%%%%%%%%%%%%%%%%%%%%%%%%%%%%%%%
\subsection{Revision History}

%%%%%%%%%%%%%%%%%%%%%%%%%%%%%%%%%%%%%%%%
\paragraph{v2.0:} 2018/12/30

\begin{itemize}
\item
immediate forward processing
\item
added |\childdocby| mechanism
\item
manual restructured
\end{itemize}

%%%%%%%%%%%%%%%%%%%%%%%%%%%%%%%%%%%%%%%%
\paragraph{v1.6:} 2018/01/17

\begin{itemize}
\item
application for development of include files
\item
corrections to manual
\end{itemize}

%%%%%%%%%%%%%%%%%%%%%%%%%%%%%%%%%%%%%%%%
\paragraph{v1.5:} 2017/05/21

\begin{itemize}
\item
more complete structuring introduced
\item
|\childdocof| introduced
\item
|\childdoc| renamed to |\childdocmain|
\item
|\childredirect| renamed to |\childdocforward| and |\childdocforwardprefix|
and functionality expanded
\end{itemize}

%%%%%%%%%%%%%%%%%%%%%%%%%%%%%%%%%%%%%%%%
\paragraph{v1.0:} 2017/04/27

\begin{itemize}
\item
manual and install package
\item
first version published on CTAN
\end{itemize}

%%%%%%%%%%%%%%%%%%%%%%%%%%%%%%%%%%%%%%%%
\paragraph{v0.6:} 2017/04/26

\begin{itemize}
\item
redirection mechanism added
\end{itemize}

%%%%%%%%%%%%%%%%%%%%%%%%%%%%%%%%%%%%%%%%
\paragraph{v0.5:} 2017/04/26

\begin{itemize}
\item
functionality in definition file
\end{itemize}


%%%%%%%%%%%%%%%%%%%%%%%%%%%%%%%%%%%%%%%%%%%%%%%%%%%%%%%%%%%%%%%%%%%%%%%%%%%%%%%%
%%%%%%%%%%%%%%%%%%%%%%%%%%%%%%%%%%%%%%%%%%%%%%%%%%%%%%%%%%%%%%%%%%%%%%%%%%%%%%%%
%%%%%%%%%%%%%%%%%%%%%%%%%%%%%%%%%%%%%%%%%%%%%%%%%%%%%%%%%%%%%%%%%%%%%%%%%%%%%%%%
\appendix

\settowidth\MacroIndent{\rmfamily\scriptsize 000\ }

 \DocInput{childdoc.dtx}

\end{document}
%</driver>
% \fi
%
% %%%%%%%%%%%%%%%%%%%%%%%%%%%%%%%%%%%%%%%%%%%%%%%%%%%%%%%%%%%%%%%%%%%%%%%%%%%%%%
% %%%%%%%%%%%%%%%%%%%%%%%%%%%%%%%%%%%%%%%%%%%%%%%%%%%%%%%%%%%%%%%%%%%%%%%%%%%%%%
% \section{Sample}
%\iffalse
%<*samplemain>
%\fi
%
% The following presents a sample document
% with two chapters, two parts, a title page,
% a compile flag as well as three forwarding files to set the flag.
% It consists of eight |.tex| files:
% \begin{center}
% \begin{tabular}{ll}
% |cdocsamp.tex|&main file\\
% |cdocsch1.tex|&include file for chapter 1\\
% |cdocsch2.tex|&include file for chapter 2\\
% |cdocspt3.tex|&include file for part 3\\
% |cdocspt4.tex|&include file for part 4\\
% |cdocsdrf.tex|&forwarding file for main file in draft mode\\
% |cdocsfi1.tex|&forwarding file for final version of chapter 1\\
% |cdocsfi2.tex|&forwarding file for final version of chapter 2\\
% \end{tabular}
% \end{center}
% Each of the eight files can be compiled directly by the \LaTeX{} compiler.
%
% %%%%%%%%%%%%%%%%%%%%%%%%%%%%%%%%%%%%%%
% \paragraph{Main File.}
%
% The main file is called |cdocsamp.tex|.
%
% Load the \textsf{childdoc} definitions and
% declare the filename for the main document:
%    \begin{macrocode}
\input{childdoc.def}
\childdocmain{}
%    \end{macrocode}

% Optional override for |\version| flag:
%    \begin{macrocode}
%%\ifchilddoc\else\providecommand{\version}{draft}\fi
%    \end{macrocode}

% Define the default values for the |\version| flag
% (|final| for the main file and |draft| for childs):
%    \begin{macrocode}
\ifchilddoc
\providecommand{\version}{draft}
\else
\providecommand{\version}{final}
\fi
%    \end{macrocode}

% Load the standard document class:
%    \begin{macrocode}
\documentclass[12pt]{article}
%    \end{macrocode}

% Start the document body:
%    \begin{macrocode}
\begin{document}
%    \end{macrocode}

% Declare a title page.
% Print title, part of document being processed and version flag:
%    \begin{macrocode}
\addtocounter{page}{-1}
\begin{center}
{\LARGE\bfseries{}childdoc example\par}
\vspace{1cm}
\ifchilddoc
\ifchilddocmanual part\else chapter\fi:
`\childdocname' of `\childdocjob'\par
\else
main document: `\childdocjob'\par
\fi
version: \version\par
\end{center}
\newpage
%    \end{macrocode}

% Manually include selected file,
% otherwise process as usual:
%    \begin{macrocode}
\ifchilddocmanual
\section*{part `\childdocname'}
\input{\childdocname}
\else
%    \end{macrocode}

% Include the two chapters:
%    \begin{macrocode}
\include{cdocsch1}
\include{cdocsch2}
%    \end{macrocode}

% Include the two parts unless only chapters should be displayed:
%    \begin{macrocode}
\ifchilddoc\else
\section{part three}
\input{cdocspt3}
\section{part four}
\input{cdocspt4}
\fi
%    \end{macrocode}

% Process as usual until here:
%    \begin{macrocode}
\fi
%    \end{macrocode}

% End of document body:
%    \begin{macrocode}
\end{document}
%    \end{macrocode}
%\iffalse
%</samplemain>
%\fi
%
% %%%%%%%%%%%%%%%%%%%%%%%%%%%%%%%%%%%%%%
% \paragraph{Chapter Include Files.}
%
% The include files are called |cdocsch1.tex| and |cdocsch2.tex|.
%
%\iffalse
%<*samplechap1|samplechap2>
%\fi

% Optional override for |\version| flag:
%    \begin{macrocode}
%%\providecommand{\version}{final}
%    \end{macrocode}

% Include the main document:
%    \begin{macrocode}
\input{childdoc.def}
\childdocof{cdocsamp}
%    \end{macrocode}

%\iffalse
%</samplechap1|samplechap2>
%\fi
%
%\iffalse
%<*samplechap1>
%\fi
% Some text for chapter 1:
%    \begin{macrocode}
\section{one}
some text in chapter one
%    \end{macrocode}

%\iffalse
%</samplechap1>
%\fi
% Some text for chapter 2:
%\iffalse
%<*samplechap2>
%\fi
%    \begin{macrocode}
\section{two}
more text in chapter two
%    \end{macrocode}

%\iffalse
%</samplechap2>
%\fi
%
% %%%%%%%%%%%%%%%%%%%%%%%%%%%%%%%%%%%%%%
% \paragraph{Part Include Files.}
%
% The include files are called |cdocspt3.tex| and |cdocspt4.tex|.
%
%\iffalse
%<*samplepart3|samplepart4>
%\fi

% Optional override for |\version| flag:
%    \begin{macrocode}
%%\providecommand{\version}{final}
%    \end{macrocode}

% Include the main document:
%    \begin{macrocode}
\input{childdoc.def}
\childdocby{cdocsamp}
%    \end{macrocode}

%\iffalse
%</samplepart3|samplepart4>
%\fi
%
%\iffalse
%<*samplepart3>
%\fi
% Some text for part 3:
%    \begin{macrocode}
some text in part three
%    \end{macrocode}

%\iffalse
%</samplepart3>
%\fi
% Some text for part 4:
%\iffalse
%<*samplepart4>
%\fi
%    \begin{macrocode}
more text in part four
%    \end{macrocode}

%\iffalse
%</samplepart4>
%\fi
%
% %%%%%%%%%%%%%%%%%%%%%%%%%%%%%%%%%%%%%%
% \paragraph{Forwarding for a Complete Draft.}
%
% The following forwarding file |cdocsdrf.tex|
% compiles the main document in draft mode:
%\iffalse
%<*sampledraft>
%\fi
%    \begin{macrocode}
\def\version{draft}
\input{childdoc.def}
\childdocforward{cdocsamp}
%    \end{macrocode}

%\iffalse
%</sampledraft>
%\fi
%
% %%%%%%%%%%%%%%%%%%%%%%%%%%%%%%%%%%%%%%
% \paragraph{Forwarding for Final Version of the Chapters.}
%
% The following forwarding files |cdocsfn1.tex| and |cdocsfn2.tex|
% (with identical content)
% compile the final versions of the child documents
% |cdocsch1.tex| and |cdocsch2.tex|, respectively:
%\iffalse
%<*samplefinal>
%\fi
%    \begin{macrocode}
\def\version{final}
\input{childdoc.def}
\childdocforwardprefix[cdocsamp]{cdocsfn}{cdocsch}
%    \end{macrocode}

%\iffalse
%</samplefinal>
%\fi
%
% %%%%%%%%%%%%%%%%%%%%%%%%%%%%%%%%%%%%%%
% \paragraph{Command Line Processing.}
%
% The following three command lines generate the output files
% |cdocscld|, |cdocscl1| and |cdocscl2|
% which should be identical to
% |cdocsdrf|, |cdocsch1| and |cdocsfn2|, respectively:
% \begin{center}
% \begin{tabular}{l}
% |latex -jobname cdocscld \|\\
% |  "\def\version{draft}\input{childdoc.def}\childdocforward{cdocsamp}"|\\
% |latex -jobname cdocscl1 \|\\
% |  "\input{childdoc.def}\childdocforward[cdocsamp]{cdocsch1}"|\\
% |latex -jobname cdocscl2 \|\\
% |  "\def\version{final}\input{childdoc.def}\childdocforward{cdocsch2}"|
% \end{tabular}
% \end{center}
% Note that the trailing backslash on each first line
% merely continues the input to the second line
% (for convenient cut ant paste).
% Furthermore, the command |latex| can be replaced by any
% of its alternative versions such as |pdflatex|.
%
% %%%%%%%%%%%%%%%%%%%%%%%%%%%%%%%%%%%%%%%%%%%%%%%%%%%%%%%%%%%%%%%%%%%%%%%%%%%%%%
% %%%%%%%%%%%%%%%%%%%%%%%%%%%%%%%%%%%%%%%%%%%%%%%%%%%%%%%%%%%%%%%%%%%%%%%%%%%%%%
% \section{Implementation}
%\iffalse
%<*package>
%\fi
%
% This section describes the definitions file |childdoc.def|.

% The definitions cannot be loaded using |\usepackage| or |\RequirePackage|
% which has a mechanism to prevent loading a style file more than once.
% When loading the definitions by means of |\input|
% multiple instances have to be prevented manually:
%\iffalse
%This code needs to be before the `\ProvidesFile' directive
%which is defined at the beginning of this file.
%Therefore it is also placed there and commented out here.
%</package>
%<*discard>
%\fi
%    \begin{macrocode}
\ifdefined\childdocmain\endinput\fi
%    \end{macrocode}
%\iffalse
%</discard>
%<*package>
%\fi
%
% \macro{\ifchilddoc}
% \macro{\ifchilddocmanual}
% The conditional |\ifchilddoc| tells whether a
% child (true) or main (false) document is being compiled.
% The conditional |\ifchilddocmanual| tells whether
% the |\includeonly| mechanism is used (false) or
% the selection of child files must be performed manually (true).
% The definitions initialise to false:
%    \begin{macrocode}
\newif\ifchilddoc
\newif\ifchilddocmanual
%    \end{macrocode}

% \macro{\childdocname}
% \macro{\childdocjob}
% The macro |\childdocname| stores the name of the main document
% to be compiled. The macro |\childdocjob| stores the name of
% the document on which the \LaTeX{} compiler was originally invoked.
% The content of |\jobname| cannot be compared
% to filenames specified in the source due to different catcodes.
% The following code rescans |\jobname|, stores the result
% in |\childdocname| and saves a copy in |\childdocjob|:
%    \begin{macrocode}
\edef\childdocname{\scantokens\expandafter{\jobname\noexpand}}
\let\childdocjob\childdocname
%    \end{macrocode}

% \macro{\childdocdisable}
% The macro |\childdocdisable| prevents the main file
% from being processed more than once.
% At this stage, the main document command |\childdocmain|
% is assumed to be called once again where it should do nothing.
% Any subsequent call to it should prevent
% a secondary processing of the main document
% It overwrites the forwarding commands
% |\childdocof| and |\childdocforward|
% with empty macros to prevent further inclusions of the main document:
%    \begin{macrocode}
\newcommand{\childdocdisable}
{
  \renewcommand{\childdocmain}[1]{\renewcommand{\childdocmain}[1]{\endinput}}
  \renewcommand{\childdocof}[1]{}
  \renewcommand{\childdocby}[2][]{}
  \renewcommand{\childdocforward}[2][]{}
  \renewcommand{\childdocdisable}{}
}
%    \end{macrocode}

% \macro{\childdocmain}
% The macro |\childdocmain| is to be called at the top of the main file
% with nothing or the main filename (without extension) as argument.
% First, it breaks loops.
% If the argument is not empty and does not match |\childdocname|
% (which is set by the first inclusion of |childdoc.def|),
% |\ifchilddoc| is set to true, |\includeonly| is applied to the child file
% and |\jobname| is set to the main file
% (for proper handling of |.aux| files):
%    \begin{macrocode}
\newcommand{\childdocmain}[1]
{
  \childdocdisable\childdocmain{}
  \if?#1?\else
    \begingroup
      \def\childdoctmp{#1}
      \ifx\childdoctmp\childdocname
        \def\childdoctmp{}
      \else
        \def\childdoctmp
        {
          \childdoctrue
          \includeonly{\childdocname}
          \def\childdocjob{#1}
          \def\jobname{#1}
        }
      \fi
      \expandafter
    \endgroup
    \childdoctmp
  \fi
}
%    \end{macrocode}

% \macro{\childdocof}
% The command |\childdocof| redirects
% compilation to the main file |#1|.
%    \begin{macrocode}
\newcommand{\childdocof}[1]
{
  \childdocdisable
  \childdoctrue
  \includeonly{\childdocname}
  \def\jobname{#1}
  \def\childdocjob{#1}
  \input{#1}
}
%    \end{macrocode}

% \macro{\childdocby}
% The command |\childdocby| ....
%    \begin{macrocode}
\newcommand{\childdocby}[2][]
{
  \childdocdisable
  \childdoctrue
  \childdocmanualtrue
  \if?#1?\else
    \def\jobname{#2}
  \fi
  \def\childdocjob{#2}
  \input{#2}
  \endinput
}
%    \end{macrocode}

% \macro{\childdocforward}
% The command |\childdocforward| redirects
% compilation to the main file or
% (if the optional argument is given) a child file.
% Parameters are set as if the main file
% or a child file starting with |\childdocof| was compiled.
% Then compilation is handed over to the main file:
%    \begin{macrocode}
\newcommand{\childdocforward}[2][]
{
  \begingroup
    \if?#1?
      \def\childdoctmp
      {
        \def\childdocname{#2}
        \def\childdocjob{#2}
        \def\jobname{#2}
        \input{#2}
        \endinput
      }
    \else
      \def\childdoctmp
      {
        \childdocdisable
        \def\childdocname{#2}
        \childdoctrue
        \includeonly{#2}
        \def\childdocjob{#1}
        \def\jobname{#1}
        \input{#1}
        \endinput
      }
    \fi
    \expandafter
  \endgroup
  \childdoctmp
}
%    \end{macrocode}

% \macro{\childdocforwardprefix}
% The command |\childdocforwardprefix| redirects
% compilation to the main or a child file by means of a pattern.
% The prefix |#1| in the current filename is replaced by |#2|
% and the suffix of the current filename is kept
% (it is assumed that the filename does not contain the substring `|~~~|'
% which is used as a delimiter).
% Compilation is handed over to the new file by |\childdocforward|:
%    \begin{macrocode}
\newcommand{\childdocforwardprefix}[3][]
{
  \begingroup
    \def\childdocextract #2##1~~~{\def\childdoctmp{\childdocforward[#1]{#3##1}}}
    \expandafter\childdocextract\childdocname~~~
    \expandafter
  \endgroup
  \childdoctmp
}
%    \end{macrocode}

% \macro{\childdoc}
% The deprecated macro |\childdoc| is a legacy version of |\childdocmain|:
%    \begin{macrocode}
\newcommand{\childdoc}{\childdocmain}
%    \end{macrocode}

% \macro{\childdocredirect}
% The deprecated macro |\childdocredirect| is a legacy version
% of |\childdocforward| and |\childdocforwardprefix|:
%    \begin{macrocode}
\newcommand{\childdocredirect}[2][]
{
  \begingroup
    \if?#1?
      \def\childdoctmp{\childdocforward{#2}}
    \else
      \def\childdoctmp{\childdocforwardprefix{#1}{#2}}
    \fi
    \expandafter
  \endgroup
  \childdoctmp
}
%    \end{macrocode}

%\iffalse
%</package>
%\fi
%
\endinput

\childdocforward{cdocsamp}
%    \end{macrocode}

%\iffalse
%</sampledraft>
%\fi
%
% %%%%%%%%%%%%%%%%%%%%%%%%%%%%%%%%%%%%%%
% \paragraph{Forwarding for Final Version of the Chapters.}
%
% The following forwarding files |cdocsfn1.tex| and |cdocsfn2.tex|
% (with identical content)
% compile the final versions of the child documents
% |cdocsch1.tex| and |cdocsch2.tex|, respectively:
%\iffalse
%<*samplefinal>
%\fi
%    \begin{macrocode}
\def\version{final}
% \iffalse
%
% childdoc.dtx Copyright (C) 2017-2018 Niklas Beisert
%
% This work may be distributed and/or modified under the
% conditions of the LaTeX Project Public License, either version 1.3
% of this license or (at your option) any later version.
% The latest version of this license is in
%   http://www.latex-project.org/lppl.txt
% and version 1.3 or later is part of all distributions of LaTeX
% version 2005/12/01 or later.
%
% This work has the LPPL maintenance status `maintained'.
%
% The Current Maintainer of this work is Niklas Beisert.
%
% This work consists of the files childdoc.dtx and childdoc.ins
% and the derived files childdoc.def and cdocsamp.tex with
% cdocsch1.tex, cdocsch2.tex, cdocsdrf.tex, cdocsfn1.tex, cdocsfn2.tex.
%
%<package>\ifdefined\childdocmain\endinput\fi
%<package>\ProvidesFile{childdoc.def}[2018/12/30 v2.0 child document driver]
%<samplemain>\ProvidesFile{cdocsamp.tex}[2018/12/30 v2.0 sample for childdoc]
%<*driver>
%\ProvidesFile{childdoc.drv}[2018/12/30 v2.0 childdoc reference manual file]
\PassOptionsToClass{10pt,a4paper}{article}
\documentclass{ltxdoc}

\usepackage[margin=35mm]{geometry}
\usepackage{hyperref}
\usepackage{hyperxmp}
\usepackage[usenames]{color}

\hypersetup{colorlinks=true}
\hypersetup{pdfstartview=FitH}
\hypersetup{pdfpagemode=UseNone}
\hypersetup{pdfsource={}}
\hypersetup{pdflang={en-UK}}
\hypersetup{pdfcopyright={Copyright 2017-2018 Niklas Beisert.
  This work may be distributed and/or modified under the
  conditions of the LaTeX Project Public License, either version 1.3
  of this license or (at your option) any later version.}}
\hypersetup{pdflicenseurl={http://www.latex-project.org/lppl.txt}}
\hypersetup{pdfcontactaddress={ETH Zurich, ITP, HIT K,
  Wolfgang-Pauli-Strasse 27}}
\hypersetup{pdfcontactpostcode={8093}}
\hypersetup{pdfcontactcity={Zurich}}
\hypersetup{pdfcontactcountry={Switzerland}}
\hypersetup{pdfcontactemail={nbeisert@itp.phys.ethz.ch}}
\hypersetup{pdfcontacturl={http://people.phys.ethz.ch/\xmptilde nbeisert/}}

\newcommand{\secref}[1]{\hyperref[#1]{section \ref*{#1}}}

\parskip1ex
\parindent0pt
\let\olditemize\itemize
\def\itemize{\olditemize\parskip0pt}

\begin{document}

\title{The \textsf{childdoc} Package}
\hypersetup{pdftitle={The childdoc Package}}
\author{Niklas Beisert\\[2ex]
  Institut f\"ur Theoretische Physik\\
  Eidgen\"ossische Technische Hochschule Z\"urich\\
  Wolfgang-Pauli-Strasse 27, 8093 Z\"urich, Switzerland\\[1ex]
  \href{mailto:nbeisert@itp.phys.ethz.ch}
  {\texttt{nbeisert@itp.phys.ethz.ch}}}
\hypersetup{pdfauthor={Niklas Beisert}}
\hypersetup{pdfsubject={Manual for the LaTeX2e Package childdoc}}
\date{30 December 2018, \textsf{v2.0}}
\maketitle

\begin{abstract}\noindent
\textsf{childdoc} is a \LaTeXe{} package
that enables the direct compilation
of document sections included by |\include|
to individual files.
\end{abstract}

\begingroup
\parskip0ex
\tableofcontents
\endgroup

%%%%%%%%%%%%%%%%%%%%%%%%%%%%%%%%%%%%%%%%%%%%%%%%%%%%%%%%%%%%%%%%%%%%%%%%%%%%%%%%
%%%%%%%%%%%%%%%%%%%%%%%%%%%%%%%%%%%%%%%%%%%%%%%%%%%%%%%%%%%%%%%%%%%%%%%%%%%%%%%%
\section{Introduction}

\LaTeX{} provides a mechanism to structure a large document (such as a book)
into a main file and several child files (containing the chapters)
using the |\include| command.
This mechanism is beneficial for documents
which span hundreds of pages in order to
make the source file(s) more manageable.
Moreover, compilation can be restricted to
selected child files by means of the |\includeonly| command.
The latter feature can be used to reduce the compilation time while editing
(this was significantly more useful in the earlier days of \LaTeX{})
or to generate a smaller document which is easier to navigate.
Another application of |\includeonly| is to generate
documents consisting of selected parts of the complete document.

However, there are a few drawbacks of the plain |\include| mechanism:
\begin{itemize}
\item
The child files cannot be compiled on their own,
they can only be compiled via the main file.
A naive editing environment
(such as a text editor with an option
to have the current file processed by \LaTeX)
may require one to switch to the main file before compiling;
attempting to compile the child file produces errors.
\item
The main file must be modified (each time)
to adjust the |\includeonly| command
to the present needs. This easily leaves the main file in a messy state.
\item
The generated document will always carry the filename
of the main document. This is inconvenient if
several child files are to be compiled and
to be kept for distribution.
\end{itemize}

The present package provides a simple interface
to make child files individually compilable by \LaTeX{}.
Compiling a child file then has the same effect as compiling
the main file with an |\includeonly| command
to select the appropriate child.
Moreover the generated document will carry the name of the child
rather than the main file.
This resolves all three above issues.

This feature is meant to make the editing of books,
thesis documents and lecture notes somewhat more convenient.
However, the package can also be used efficiently for
composing a series of documents (such as exercise sheets)
which are typically distributed individually.
It then assists the author in generating the individual documents
(potentially in different versions)
as well as a document containing the collected series.
Another application is in developing style files
or other kinds of included material
where compilation of the style file could redirect
to a sample or test file.

%%%%%%%%%%%%%%%%%%%%%%%%%%%%%%%%%%%%%%%%%%%%%%%%%%%%%%%%%%%%%%%%%%%%%%%%%%%%%%%%
%%%%%%%%%%%%%%%%%%%%%%%%%%%%%%%%%%%%%%%%%%%%%%%%%%%%%%%%%%%%%%%%%%%%%%%%%%%%%%%%
\section{Usage}

First of all, the package \textsf{childdoc} is \emph{not} a standard
\LaTeXe{} |.sty| style file! Therefore it needs to be invoked in
a non-standard way.

%%%%%%%%%%%%%%%%%%%%%%%%%%%%%%%%%%%%%%%%%%%%%%%%%%%%%%%%%%%%%%%%%%%%%%%%%%%%%%%%
\subsection{Included Files}
\label{sec:include}

%%%%%%%%%%%%%%%%%%%%%%%%%%%%%%%%%%%%%%%%
\DescribeMacro{\childdocmain}
To use the package, add the commands
\begin{center}
\begin{tabular}{l}
|\input{childdoc.def}|\\
|\childdocmain{}|\\
\end{tabular}
\end{center}
at the very top of the main \LaTeX{} file,
in particular \emph{before} the |\documentclass| statement!
The argument of |\childdocmain| should be left empty
(but it must be present).

%%%%%%%%%%%%%%%%%%%%%%%%%%%%%%%%%%%%%%%%
\DescribeMacro{\childdocof}
Furthermore, add the commands
\begin{center}
\begin{tabular}{l}
|\input{childdoc.def}|\\
|\childdocof{|\textit{main}|}|\\
\end{tabular}
\end{center}
at the top of every child file \textit{child}
which is included by |\include{|\textit{child}|}|
from within the main file
(or at least for those files to be compiled individually).
The argument \textit{main} must be the filename of the main file.

There are a couple of
considerations in setting up the main and child documents:

%%%%%%%%%%%%%%%%%%%%%%%%%%%%%%%%%%%%%%%%
\paragraph{Restrictions.}

Please note the following restrictions:
\begin{itemize}
\item
|\childdocmain| must be called with one argument \textit{main}
to ensure compatibility with earlier version of the package.
It must either be empty (|\childdocmain{}|)
or precisely match the filename of the main file in which it is specified.
See \secref{sec:detection} for further information.
\item
The filename \textit{main} must be specified without the |.tex| extension.
\item
The filename \textit{main} is case sensitive
(even in case-insensitive file systems)
due to internal string comparison.
\item
The argument \textit{main} should be fully expanded, it cannot be a macro.
\item
Subdirectories and special characters should be avoided in filenames.
\item
The command |\childdocmain{|\textit{main}|}| must be followed by a whitespace.
It should not be followed immediately by another command
or by a comment mark `|%|'.
This is because the \TeX{} parser reads the token immediately following
the argument of |\childdocmain| and puts it
at the beginning of every child section;
however, a white\-space is ignored.
\end{itemize}

%%%%%%%%%%%%%%%%%%%%%%%%%%%%%%%%%%%%%%%%
\paragraph{Content of Main File.}

It is advisable to place all content in the child files included by |\include|.
Any output contained in the main file will appear in all child documents
unless suppressed manually;
it cannot be suppressed automatically by the |\includeonly| directive
and thus should normally be avoided.
A method to include some content in the main file
by means of conditional processing is described in \secref{sec:conditional}.

%%%%%%%%%%%%%%%%%%%%%%%%%%%%%%%%%%%%%%%%
\paragraph{Page Numbering.}

When only a part of the document is compiled,
the appropriate numbering of pages
(as well as other status parameters)
is determined from the |.aux| files.
The latter contain information from previous passes.
However this information needs to propagate through
all intermediate child documents.
Therefore the page numbering in child documents may well
be inconsistent until the complete document is compiled at least once.

A useful (if unconventional) way to always ensure a consistent
page numbering is to restart the numbering in each child document
and denote the pages by `\textit{child}|.|\textit{page}'
where \textit{child} represents the chapter/section number of the child file.
This can be achieved by the command
|\numberwithin{page}{|\textit{child}|}|
of the \textsf{amsmath} package
where \textit{child} can be |chapter| or |section|
depending on the chosen structuring.
Alternatively, one can modify the macro |\thepage| appropriately
and reset the counter |page| at the start of each child file.

%%%%%%%%%%%%%%%%%%%%%%%%%%%%%%%%%%%%%%%%%%%%%%%%%%%%%%%%%%%%%%%%%%%%%%%%%%%%%%%%
\subsection{Conditional Processing}
\label{sec:conditional}

The package provides a mechanism to compile different versions
of a document. To customise the versions further some conditional processing
can come in handy to distinguish which version is being compiled.
The package provides two macros to describe the compilation context:

%%%%%%%%%%%%%%%%%%%%%%%%%%%%%%%%%%%%%%%%
\DescribeMacro{\ifchilddoc}
The conditional |\ifchilddoc| distinguishes between the compilation of
child documents and the main document:
%
\begin{center}
|\ifchilddoc |\textit{child-code}| |[|\||else |\textit{main-code}]| \||fi|
\end{center}

%%%%%%%%%%%%%%%%%%%%%%%%%%%%%%%%%%%%%%%%
\DescribeMacro{\childdocname}
\DescribeMacro{\childdocjob}
The macro |\childdocname| contains the filename (without extension)
of the main or child file being processed.
Note that |\childdocjob| will always contain the name of the main file.

%%%%%%%%%%%%%%%%%%%%%%%%%%%%%%%%%%%%%%%%
\paragraph{Title Page.}

Conditional processing can be used to include a title or banner page
in the main document when proper precautions are taken.
Importantly, the code in the main file should ensure that the page counter
(as well as other status parameters which are stored in the |.aux| files)
takes the same value after the conditional processing.
Otherwise the page numbers may take divergent values
depending on which part is compiled.

For example, a title page could be declared by:
%
\begin{center}
\begin{tabular}{l}
|\ifchilddoc\||else|\\
|\addtocounter{page}{-1}|\\
\textit{code for title page}\\
|\newpage|\\
|\||fi|
\end{tabular}
\end{center}
%
A banner page for the child documents can be generated by:
%
\begin{center}
\begin{tabular}{l}
|\ifchilddoc|\\
|\addtocounter{page}{-1}|\\
\textit{code for banner page}\\
|\newpage|\\
|\||fi|
\end{tabular}
\end{center}
%
Here one could write a message such as:
\begin{center}
|This is the part \childdocname{} of \childdocjob{}.|
\end{center}

%%%%%%%%%%%%%%%%%%%%%%%%%%%%%%%%%%%%%%%%%%%%%%%%%%%%%%%%%%%%%%%%%%%%%%%%%%%%%%%%
\subsection{Flags}
\label{sec:flags}

The package makes it easy to generate different versions
of the main or child documents.
To this end compilation flags can be defined
and assigned different default values.
They will be particularly useful in conjunction
with the forwarding mechanism described in \secref{sec:forward}.

For example, it may be useful to have a flag |\version|
which can be set to |draft| or |final|.
The document source will contain some conditional code
depending on the value of |\version|.
Suppose further, the flag should default to |final| for the main file
and to |draft| for child files
which is a natural assignment for editing the document.
This is achieved by placing the following code
in the preamble of the main document
(below the |\childdocmain| directive):
%
\begin{center}
\begin{tabular}{l}
|\ifchilddoc|\\
|\providecommand{\version}{draft}|\\
|\||else|\\
|\providecommand{\version}{final}|\\
|\||fi|
\end{tabular}
\end{center}
%
The definition by |\providecommand| makes sure
that previous definitions are not overwritten.
Further statements |\providecommand{\version}{...}|
can thus be added before the above code to override it.

For the main file, one might add a line
(between |\childdocmain| and the above block)
%
\begin{center}
|%\ifchilddoc\||else\providecommand{\version}{draft}\||fi|
\end{center}
%
which can be uncommented to produce a draft version.
Likewise one can add a line to the very top of a child file
(above the |\childdocof{|\textit{main}|}| directive)
%
\begin{center}
|%\providecommand{\version}{final}|
\end{center}
%
which can be uncommented to produce the final version of this child document.

%%%%%%%%%%%%%%%%%%%%%%%%%%%%%%%%%%%%%%%%%%%%%%%%%%%%%%%%%%%%%%%%%%%%%%%%%%%%%%%%
\subsection{Forwarding}
\label{sec:forward}

Different versions of the main or child documents
using compilation flags as described in \secref{sec:flags}
can be (permanently) stored in different files
for convenient compilation, viewing and distribution.
To this end, the package defines a command
to pass on compilation to a different file:

%%%%%%%%%%%%%%%%%%%%%%%%%%%%%%%%%%%%%%%%
\DescribeMacro{\childdocforward}
The command |\childdocforward| redirects processing to
another source file:
%
\begin{center}
\begin{tabular}{l}
|\input{childdoc.def}|\\
|\childdocforward[|\textit{main}|]{|\textit{dest}|}|\\
\end{tabular}
\end{center}
%
The argument \textit{dest} is the destination file
(without extension).
It should be the main file or one of the child files.
Note that further \textsf{childdoc} directives
such as |\childdocof| and |\childdocforward|
in the indicated file will be processed in this form.
The optional argument \textit{main}
passes on directly to the main file \textit{main}
while pretending to compile the child \textit{dest}.
This form behaves as if \textit{dest}
issues |\childdocof{|\textit{main}|}| right away,
and no further \textsf{childdoc} directives will be processed.

%%%%%%%%%%%%%%%%%%%%%%%%%%%%%%%%%%%%%%%%
\DescribeMacro{\...prefix}
In the alternative form |\childdocforwardprefix|,
%
\begin{center}
\begin{tabular}{l}
|\input{childdoc.def}|\\
|\childdocforwardprefix[|\textit{main}|]{|\textit{prefix}|}{|\textit{dest}|}|
\end{tabular}
\end{center}
%
the destination file is determined by a pattern
depending on the current file:
To make this work, the current file must be called
`{\textit{prefix}\hspace{0.2em}\textit{suffix}}'
with \textit{prefix} matching precisely the argument.
Processing is then passed on to the file
`{\textit{dest}\hspace{0.2em}\textit{suffix}}'.
Surely, the same effect is achieved by
directly specifying the
argument `{\textit{dest}\hspace{0.2em}\textit{suffix}}'
in the first form.
However, that requires to set up a different file
for each child. With the alternative form of the command
all these files can have exactly the same content
which simplifies setting them up and maintaining them.

For example, the following file |draft.tex|
with a compilation flag |\version| as described in \secref{sec:flags}
compiles the main document as a draft:
%
\begin{center}
\begin{tabular}{l}
|\def\version{draft}|\\
|\input{childdoc.def}|\\
|\childdocforward{|\textit{main}|}|
\end{tabular}
\end{center}
%
Likewise, the following files |final|\textit{nn}|.tex|
compile the final version of the child document
|child|\textit{nn}|.tex|:
%
\begin{center}
\begin{tabular}{l}
|\def\version{final}|\\
|\input{childdoc.def}|\\
|\childdocforwardprefix{final}{child}|
\end{tabular}
\end{center}
%

Note that when several versions of a main file and/or of each child file
are to be generated, it may be convenient to set up a |Makefile| or
shell script to automatise the process.

%%%%%%%%%%%%%%%%%%%%%%%%%%%%%%%%%%%%%%%%%%%%%%%%%%%%%%%%%%%%%%%%%%%%%%%%%%%%%%%%
\subsection{Command Line Processing}
\label{sec:commandline}

The effect of redirection files can also be achieved by invoking
the \LaTeX{} compiler with a more elaborate command line.
Most conveniently this should be done as part
of a shell script or a |Makefile|.

When using \textsf{childdoc} in the main file, the following
command lines effectively perform a redirection
(note that depending on the shell being used,
backslashes may have to be doubled: `|\|' $\to$ `|\\|'):
%
\begin{center}
|... -jobname "|\textit{target}|" |\\|"|[\textit{flags}]%
|\input{childdoc.def}\childdocforward[|\textit{main}|]{|\textit{dest}|}"|
\end{center}
%
Here \textit{target} is the name of the output file,
\textit{main} is the name of the main file
and \textit{dest} is the name of the main or child file to be processed
(all filenames without extensions).
The optional argument \textit{main} can be omitted
if \textit{main} matches \textit{dest}.
Optionally, compilation \textit{flags} can be defined via |\def| commands.
This command line makes the \TeX{} engine believe
it is compiling the file \textit{target}
whose content is specified as the latter parameter.
The provided code then forwards the processing to
\textit{main} or \textit{dest} as described in \secref{sec:forward}.

%%%%%%%%%%%%%%%%%%%%%%%%%%%%%%%%%%%%%%%%%%%%%%%%%%%%%%%%%%%%%%%%%%%%%%%%%%%%%%%%
\subsection{Include by Input}
\label{sec:input}

Including child documents by |\include| has some restrictions by design.
Most notably, the content of a child document always occupies
its own set of pages; pages cannot be shared between child documents.
Usually, this behaviour makes perfect sense
because each child document contain an essential part of the document.
However, in some situations it may be desirable to compose
a document from a collection of parts
without having mandatory page breaks between then.
For this case, the package
provides a mechanism to include parts
by |\input| which can also be processed individually.
However, by construction this mechanism
requires manual handling of the content to be output.

%%%%%%%%%%%%%%%%%%%%%%%%%%%%%%%%%%%%%%%%
\DescribeMacro{\ifchilddocmanual}
The main file should be prepared as usual, see \secref{sec:include}.
However, the document body must make a distinction
between processing of an individual part and of the main document, e.g.:
%
\begin{center}
\begin{tabular}{l}
|\ifchilddocmanual|\\
|\input{\childdocname}|\\
|\||else|\\
\textit{document body with }|\input{|\textit{part}|}|\\
|\||fi|
\end{tabular}
\end{center}
%
The conditional |\ifchilddocmanual| is true whenever
a part to be included by |\input| is being compiled,
and the name of the part is stored in |\childdocname|.

%%%%%%%%%%%%%%%%%%%%%%%%%%%%%%%%%%%%%%%%
\DescribeMacro{\childdocby}
Each part to be included by |\input| should start with:
%
\begin{center}
\begin{tabular}{l}
|\input{childdoc.def}|\\
|\childdocby{|\textit{main}|}|\\
\end{tabular}
\end{center}
%
The directive |\childdocby| is similar to |\childdocof|
described in \secref{sec:include},
but the subsequent selection of content must be done manually.
To that end, both |\ifchilddoc| and |\ifchilddocmanual|
will be true upon processing of a part,
and the name of the part is stored in |\childdocname|.
Note that |\jobname| will be set to the filename of the current part
so that each part receives an individual |.aux| file
that does not interfere with the |.aux| file(s) of the main document.
This behaviour can be altered by the alternative form
|\childdocby[*]{|\textit{main}|}| (with a non-empty optional argument)
which uses the |.aux| file of the main document
by setting |\jobname| to \textit{main}.

%%%%%%%%%%%%%%%%%%%%%%%%%%%%%%%%%%%%%%%%%%%%%%%%%%%%%%%%%%%%%%%%%%%%%%%%%%%%%%%%
\subsection{Driver Development}
\label{sec:driver}

The \textsf{childdoc} mechanism can also be use for the development
of definition files such as \LaTeX{} styles or classes.
This case differs from the above setup with multiple parts
included by |\include| in that no |\includeonly| should be invoked.
This can be achieved by starting the include file
(before |\ProvidesPackage|) with:
%
\begin{center}
\begin{tabular}{l}
|\input{childdoc.def}|\\
|\childdocforward{|\textit{main}|}|\\
\end{tabular}
\end{center}
%
or alternatively with:
%
\begin{center}
\begin{tabular}{l}
|\input{childdoc.def}|\\
|\childdocby{|\textit{main}|}|\\
\end{tabular}
\end{center}
%
Both forms have slightly different effects as described above.
The main file is prepared as usual, see \secref{sec:include}.

%%%%%%%%%%%%%%%%%%%%%%%%%%%%%%%%%%%%%%%%%%%%%%%%%%%%%%%%%%%%%%%%%%%%%%%%%%%%%%%%
\subsection{Legacy Detection}
\label{sec:detection}

The directive |\childdocmain| in the main file can detect
whether the complete document or merely a child is to be compiled
even without using the directive |\childdocof|.
This method is deprecated because it is less robust
and there is no compelling reason to use it;
it is merely provided for backward compatibility
and it may be removed in future versions.

If the detection mechanism is to be used,
it is mandatory to correctly specify
the filename of the main file as the argument of |\childdocmain|:
%
\begin{center}
\begin{tabular}{l}
|\input{childdoc.def}|\\
|\childdocmain{|\textit{main}|}|\\
\end{tabular}
\end{center}
%
If |\jobname| does not match the argument \textit{main} of |\childdocmain|,
it is assumed that |\jobname| points to the child file to be compiled.
When using |\childdocmain| with the main file specified as argument,
it suffices to start a child file
with just |\input{|\textit{main}|}|
without loading of the package and using |\childdocof|.
If instead all processing is done
with the appropriate \textsf{childdoc} directives,
the argument of \textit{main} of |\childdocmain| can be empty.

An alternative version of the command line processing described
in \secref{sec:commandline} using the detection mechanism reads:
%
\begin{center}
|... -jobname "|\textit{target}|" "|[\textit{flags}]%
[|\def\jobname{|\textit{dest}|}|]|\input{|\textit{main}|}"|
\end{center}

%%%%%%%%%%%%%%%%%%%%%%%%%%%%%%%%%%%%%%%%%%%%%%%%%%%%%%%%%%%%%%%%%%%%%%%%%%%%%%%%
\subsection{Manual Code}
\label{sec:manual}

In case one cannot be certain whether the definitions file |childdoc.def|
is installed on the target \TeX{} distribution
and one prefers not to ship it,
it is conceivable to paste a few relevant commands into the sources.

To that end, drop all statements |\input{childdoc.def}|
and perform the replacements as outlined below.
Instead of |\childdocmain{|\textit{main}|}| add the following code
to the top of the main file:
%
\begin{center}
\begin{tabular}{l}
|\||ifdefined\childdocname\endinput\||fi\newif\ifchilddoc|\\
|\edef\childdocname{\scantokens\expandafter{\jobname\noexpand}}|\\
|\def\childdocmain{|\textit{main}|}\||ifx\childdocmain\childdocname\||else|\\
|\childdoctrue\includeonly{\childdocname}\let\jobname\childdocmain\||fi|\\
\end{tabular}
\end{center}
%
Instead of |\childdocof{|\textit{main}|}| just include the main file
at the top of each child file:
%
\begin{center}
|\input{|\textit{main}|}|
\end{center}
%
A simple redirection |\childdocforward{|\textit{dest}|}| is achieved by:
%
\begin{center}
|\def\jobname{|\textit{dest}|}\input{\jobname}|
\end{center}
%
The redirection with prefix
|\childdocforwardprefix[|\textit{prefix}|]{|\textit{dest}|}|
is accomplished by:
%
\begin{center}
\begin{tabular}{l}
|{\edef\jobname{\scantokens\expandafter{\jobname\noexpand}}|\\
|\def\redirectjob |\textit{prefix}|#1~~~{\gdef\jobname{|\textit{dest}|#1}}|\\
|\expandafter\redirectjob\jobname~~~}\input{\jobname}|
\end{tabular}
\end{center}

In an alternative approach,
child documents can be compiled by a specific command line
without additional code or specific definitions:
%
\begin{center}
|... -jobname "|\textit{target}|" "|[\textit{flags}]%
|\includeonly{|\textit{dest}|}\input{|\textit{main}|}"|
\end{center}
%

%%%%%%%%%%%%%%%%%%%%%%%%%%%%%%%%%%%%%%%%%%%%%%%%%%%%%%%%%%%%%%%%%%%%%%%%%%%%%%%%
%%%%%%%%%%%%%%%%%%%%%%%%%%%%%%%%%%%%%%%%%%%%%%%%%%%%%%%%%%%%%%%%%%%%%%%%%%%%%%%%
\section{Information}

%%%%%%%%%%%%%%%%%%%%%%%%%%%%%%%%%%%%%%%%%%%%%%%%%%%%%%%%%%%%%%%%%%%%%%%%%%%%%%%%
\subsection{Copyright}

Copyright \copyright{} 2017--2018 Niklas Beisert

This work may be distributed and/or modified under the
conditions of the \LaTeX{} Project Public License, either version 1.3
of this license or (at your option) any later version.
The latest version of this license is in
  \url{http://www.latex-project.org/lppl.txt}
and version 1.3 or later is part of all distributions of \LaTeX{}
version 2005/12/01 or later.

This work has the LPPL maintenance status `maintained'.

The Current Maintainer of this work is Niklas Beisert.

This work consists of the files |README.txt|, |childdoc.ins| and |childdoc.dtx|
as well as the derived files |childdoc.def|, |cdocsamp.tex|
with |cdocsch1.tex|, |cdocsch2.tex|, |cdocspt3.tex|, |cdocspt4.tex|,
|cdocsdrf.tex|, |cdocsfn1.tex|, |cdocsfn2.tex|
as well as |childdoc.pdf|.

%%%%%%%%%%%%%%%%%%%%%%%%%%%%%%%%%%%%%%%%%%%%%%%%%%%%%%%%%%%%%%%%%%%%%%%%%%%%%%%%
\subsection{Files and Installation}

The package consists of the files:
%
\begin{center}
\begin{tabular}{ll}
    |README.txt|   & readme file \\
    |childdoc.ins| & installation file \\
    |childdoc.dtx| & source file \\
    |childdoc.def| & definition file \\
    |cdocsamp.tex| & sample main file \\
    |cdocsch1.tex| & sample include file \\
    |cdocsch2.tex| & sample include file \\
    |cdocspt3.tex| & sample part file \\
    |cdocspt4.tex| & sample part file \\
    |cdocsdrf.tex| & sample redirection file \\
    |cdocsfn1.tex| & sample redirection file \\
    |cdocsfn2.tex| & sample redirection file \\
    |childdoc.pdf| & manual
\end{tabular}
\end{center}
%
The distribution consists of the files
|README.txt|, |childdoc.ins| and |childdoc.dtx|.
%
\begin{itemize}
\item
Run (pdf)\LaTeX{} on |childdoc.dtx|
to compile the manual |childdoc.pdf| (this file).
\item
Run \LaTeX{} on |childdoc.ins| to create the definitions file |childdoc.def|
and the sample |cdocsamp.tex| with include files
|cdocsch1.tex|, |cdocsch2.tex|, |cdocspt3.tex|, |cdocspt4.tex|,
|cdocsdrf.tex|, |cdocsfn1.tex|, |cdocsfn2.tex|.
Then copy the file |childdoc.def| to an appropriate directory of your \LaTeX{}
distribution, e.g.\ \textit{texmf-root}|/tex/latex/childdoc|.
\end{itemize}

%%%%%%%%%%%%%%%%%%%%%%%%%%%%%%%%%%%%%%%%%%%%%%%%%%%%%%%%%%%%%%%%%%%%%%%%%%%%%%%%
\subsection{Related CTAN Packages}

There are several other packages which offer a similar functionality:
%
\begin{itemize}
\item
The packages
\href{http://ctan.org/pkg/docmute}{\textsf{docmute}},
\href{http://ctan.org/pkg/includex}{\textsf{includex}} and
\href{http://ctan.org/pkg/standalone}{\textsf{standalone}}
provide commands to include only the document body of
a child file thus allowing both files to be compiled individually.
\item
The packages \href{http://ctan.org/pkg/subdocs}{\textsf{subdocs}}
and \href{http://ctan.org/pkg/subfiles}{\textsf{subfiles}}
provide structures in which the main and child documents can be
encapsulated and allowing them to be compiled individually.
The inclusion mechanism is different from the conventional |\include|.
\item
The package \href{http://ctan.org/pkg/combine}{\textsf{combine}}
is an elaborate solution to combine several documents into one.
\end{itemize}
%
See also the CTAN topic \href{http://ctan.org/topic/subdocs}{\textsf{subdocs}}
for further related packages.
The present package differs from the above solutions in that
a document structure constructed with the conventional |\include| mechanism
just needs two extra commands at the top of every file
such that all constituent files can be compiled individually.

%%%%%%%%%%%%%%%%%%%%%%%%%%%%%%%%%%%%%%%%%%%%%%%%%%%%%%%%%%%%%%%%%%%%%%%%%%%%%%%%
%\subsection{Feature Suggestions}
%
%The following is a list of features which may be useful for future
%versions of this package:
%%
%\begin{itemize}
%\item
%\ldots
%\end{itemize}

%%%%%%%%%%%%%%%%%%%%%%%%%%%%%%%%%%%%%%%%%%%%%%%%%%%%%%%%%%%%%%%%%%%%%%%%%%%%%%%%
\subsection{Revision History}

%%%%%%%%%%%%%%%%%%%%%%%%%%%%%%%%%%%%%%%%
\paragraph{v2.0:} 2018/12/30

\begin{itemize}
\item
immediate forward processing
\item
added |\childdocby| mechanism
\item
manual restructured
\end{itemize}

%%%%%%%%%%%%%%%%%%%%%%%%%%%%%%%%%%%%%%%%
\paragraph{v1.6:} 2018/01/17

\begin{itemize}
\item
application for development of include files
\item
corrections to manual
\end{itemize}

%%%%%%%%%%%%%%%%%%%%%%%%%%%%%%%%%%%%%%%%
\paragraph{v1.5:} 2017/05/21

\begin{itemize}
\item
more complete structuring introduced
\item
|\childdocof| introduced
\item
|\childdoc| renamed to |\childdocmain|
\item
|\childredirect| renamed to |\childdocforward| and |\childdocforwardprefix|
and functionality expanded
\end{itemize}

%%%%%%%%%%%%%%%%%%%%%%%%%%%%%%%%%%%%%%%%
\paragraph{v1.0:} 2017/04/27

\begin{itemize}
\item
manual and install package
\item
first version published on CTAN
\end{itemize}

%%%%%%%%%%%%%%%%%%%%%%%%%%%%%%%%%%%%%%%%
\paragraph{v0.6:} 2017/04/26

\begin{itemize}
\item
redirection mechanism added
\end{itemize}

%%%%%%%%%%%%%%%%%%%%%%%%%%%%%%%%%%%%%%%%
\paragraph{v0.5:} 2017/04/26

\begin{itemize}
\item
functionality in definition file
\end{itemize}


%%%%%%%%%%%%%%%%%%%%%%%%%%%%%%%%%%%%%%%%%%%%%%%%%%%%%%%%%%%%%%%%%%%%%%%%%%%%%%%%
%%%%%%%%%%%%%%%%%%%%%%%%%%%%%%%%%%%%%%%%%%%%%%%%%%%%%%%%%%%%%%%%%%%%%%%%%%%%%%%%
%%%%%%%%%%%%%%%%%%%%%%%%%%%%%%%%%%%%%%%%%%%%%%%%%%%%%%%%%%%%%%%%%%%%%%%%%%%%%%%%
\appendix

\settowidth\MacroIndent{\rmfamily\scriptsize 000\ }

 \DocInput{childdoc.dtx}

\end{document}
%</driver>
% \fi
%
% %%%%%%%%%%%%%%%%%%%%%%%%%%%%%%%%%%%%%%%%%%%%%%%%%%%%%%%%%%%%%%%%%%%%%%%%%%%%%%
% %%%%%%%%%%%%%%%%%%%%%%%%%%%%%%%%%%%%%%%%%%%%%%%%%%%%%%%%%%%%%%%%%%%%%%%%%%%%%%
% \section{Sample}
%\iffalse
%<*samplemain>
%\fi
%
% The following presents a sample document
% with two chapters, two parts, a title page,
% a compile flag as well as three forwarding files to set the flag.
% It consists of eight |.tex| files:
% \begin{center}
% \begin{tabular}{ll}
% |cdocsamp.tex|&main file\\
% |cdocsch1.tex|&include file for chapter 1\\
% |cdocsch2.tex|&include file for chapter 2\\
% |cdocspt3.tex|&include file for part 3\\
% |cdocspt4.tex|&include file for part 4\\
% |cdocsdrf.tex|&forwarding file for main file in draft mode\\
% |cdocsfi1.tex|&forwarding file for final version of chapter 1\\
% |cdocsfi2.tex|&forwarding file for final version of chapter 2\\
% \end{tabular}
% \end{center}
% Each of the eight files can be compiled directly by the \LaTeX{} compiler.
%
% %%%%%%%%%%%%%%%%%%%%%%%%%%%%%%%%%%%%%%
% \paragraph{Main File.}
%
% The main file is called |cdocsamp.tex|.
%
% Load the \textsf{childdoc} definitions and
% declare the filename for the main document:
%    \begin{macrocode}
\input{childdoc.def}
\childdocmain{}
%    \end{macrocode}

% Optional override for |\version| flag:
%    \begin{macrocode}
%%\ifchilddoc\else\providecommand{\version}{draft}\fi
%    \end{macrocode}

% Define the default values for the |\version| flag
% (|final| for the main file and |draft| for childs):
%    \begin{macrocode}
\ifchilddoc
\providecommand{\version}{draft}
\else
\providecommand{\version}{final}
\fi
%    \end{macrocode}

% Load the standard document class:
%    \begin{macrocode}
\documentclass[12pt]{article}
%    \end{macrocode}

% Start the document body:
%    \begin{macrocode}
\begin{document}
%    \end{macrocode}

% Declare a title page.
% Print title, part of document being processed and version flag:
%    \begin{macrocode}
\addtocounter{page}{-1}
\begin{center}
{\LARGE\bfseries{}childdoc example\par}
\vspace{1cm}
\ifchilddoc
\ifchilddocmanual part\else chapter\fi:
`\childdocname' of `\childdocjob'\par
\else
main document: `\childdocjob'\par
\fi
version: \version\par
\end{center}
\newpage
%    \end{macrocode}

% Manually include selected file,
% otherwise process as usual:
%    \begin{macrocode}
\ifchilddocmanual
\section*{part `\childdocname'}
\input{\childdocname}
\else
%    \end{macrocode}

% Include the two chapters:
%    \begin{macrocode}
\include{cdocsch1}
\include{cdocsch2}
%    \end{macrocode}

% Include the two parts unless only chapters should be displayed:
%    \begin{macrocode}
\ifchilddoc\else
\section{part three}
\input{cdocspt3}
\section{part four}
\input{cdocspt4}
\fi
%    \end{macrocode}

% Process as usual until here:
%    \begin{macrocode}
\fi
%    \end{macrocode}

% End of document body:
%    \begin{macrocode}
\end{document}
%    \end{macrocode}
%\iffalse
%</samplemain>
%\fi
%
% %%%%%%%%%%%%%%%%%%%%%%%%%%%%%%%%%%%%%%
% \paragraph{Chapter Include Files.}
%
% The include files are called |cdocsch1.tex| and |cdocsch2.tex|.
%
%\iffalse
%<*samplechap1|samplechap2>
%\fi

% Optional override for |\version| flag:
%    \begin{macrocode}
%%\providecommand{\version}{final}
%    \end{macrocode}

% Include the main document:
%    \begin{macrocode}
\input{childdoc.def}
\childdocof{cdocsamp}
%    \end{macrocode}

%\iffalse
%</samplechap1|samplechap2>
%\fi
%
%\iffalse
%<*samplechap1>
%\fi
% Some text for chapter 1:
%    \begin{macrocode}
\section{one}
some text in chapter one
%    \end{macrocode}

%\iffalse
%</samplechap1>
%\fi
% Some text for chapter 2:
%\iffalse
%<*samplechap2>
%\fi
%    \begin{macrocode}
\section{two}
more text in chapter two
%    \end{macrocode}

%\iffalse
%</samplechap2>
%\fi
%
% %%%%%%%%%%%%%%%%%%%%%%%%%%%%%%%%%%%%%%
% \paragraph{Part Include Files.}
%
% The include files are called |cdocspt3.tex| and |cdocspt4.tex|.
%
%\iffalse
%<*samplepart3|samplepart4>
%\fi

% Optional override for |\version| flag:
%    \begin{macrocode}
%%\providecommand{\version}{final}
%    \end{macrocode}

% Include the main document:
%    \begin{macrocode}
\input{childdoc.def}
\childdocby{cdocsamp}
%    \end{macrocode}

%\iffalse
%</samplepart3|samplepart4>
%\fi
%
%\iffalse
%<*samplepart3>
%\fi
% Some text for part 3:
%    \begin{macrocode}
some text in part three
%    \end{macrocode}

%\iffalse
%</samplepart3>
%\fi
% Some text for part 4:
%\iffalse
%<*samplepart4>
%\fi
%    \begin{macrocode}
more text in part four
%    \end{macrocode}

%\iffalse
%</samplepart4>
%\fi
%
% %%%%%%%%%%%%%%%%%%%%%%%%%%%%%%%%%%%%%%
% \paragraph{Forwarding for a Complete Draft.}
%
% The following forwarding file |cdocsdrf.tex|
% compiles the main document in draft mode:
%\iffalse
%<*sampledraft>
%\fi
%    \begin{macrocode}
\def\version{draft}
\input{childdoc.def}
\childdocforward{cdocsamp}
%    \end{macrocode}

%\iffalse
%</sampledraft>
%\fi
%
% %%%%%%%%%%%%%%%%%%%%%%%%%%%%%%%%%%%%%%
% \paragraph{Forwarding for Final Version of the Chapters.}
%
% The following forwarding files |cdocsfn1.tex| and |cdocsfn2.tex|
% (with identical content)
% compile the final versions of the child documents
% |cdocsch1.tex| and |cdocsch2.tex|, respectively:
%\iffalse
%<*samplefinal>
%\fi
%    \begin{macrocode}
\def\version{final}
\input{childdoc.def}
\childdocforwardprefix[cdocsamp]{cdocsfn}{cdocsch}
%    \end{macrocode}

%\iffalse
%</samplefinal>
%\fi
%
% %%%%%%%%%%%%%%%%%%%%%%%%%%%%%%%%%%%%%%
% \paragraph{Command Line Processing.}
%
% The following three command lines generate the output files
% |cdocscld|, |cdocscl1| and |cdocscl2|
% which should be identical to
% |cdocsdrf|, |cdocsch1| and |cdocsfn2|, respectively:
% \begin{center}
% \begin{tabular}{l}
% |latex -jobname cdocscld \|\\
% |  "\def\version{draft}\input{childdoc.def}\childdocforward{cdocsamp}"|\\
% |latex -jobname cdocscl1 \|\\
% |  "\input{childdoc.def}\childdocforward[cdocsamp]{cdocsch1}"|\\
% |latex -jobname cdocscl2 \|\\
% |  "\def\version{final}\input{childdoc.def}\childdocforward{cdocsch2}"|
% \end{tabular}
% \end{center}
% Note that the trailing backslash on each first line
% merely continues the input to the second line
% (for convenient cut ant paste).
% Furthermore, the command |latex| can be replaced by any
% of its alternative versions such as |pdflatex|.
%
% %%%%%%%%%%%%%%%%%%%%%%%%%%%%%%%%%%%%%%%%%%%%%%%%%%%%%%%%%%%%%%%%%%%%%%%%%%%%%%
% %%%%%%%%%%%%%%%%%%%%%%%%%%%%%%%%%%%%%%%%%%%%%%%%%%%%%%%%%%%%%%%%%%%%%%%%%%%%%%
% \section{Implementation}
%\iffalse
%<*package>
%\fi
%
% This section describes the definitions file |childdoc.def|.

% The definitions cannot be loaded using |\usepackage| or |\RequirePackage|
% which has a mechanism to prevent loading a style file more than once.
% When loading the definitions by means of |\input|
% multiple instances have to be prevented manually:
%\iffalse
%This code needs to be before the `\ProvidesFile' directive
%which is defined at the beginning of this file.
%Therefore it is also placed there and commented out here.
%</package>
%<*discard>
%\fi
%    \begin{macrocode}
\ifdefined\childdocmain\endinput\fi
%    \end{macrocode}
%\iffalse
%</discard>
%<*package>
%\fi
%
% \macro{\ifchilddoc}
% \macro{\ifchilddocmanual}
% The conditional |\ifchilddoc| tells whether a
% child (true) or main (false) document is being compiled.
% The conditional |\ifchilddocmanual| tells whether
% the |\includeonly| mechanism is used (false) or
% the selection of child files must be performed manually (true).
% The definitions initialise to false:
%    \begin{macrocode}
\newif\ifchilddoc
\newif\ifchilddocmanual
%    \end{macrocode}

% \macro{\childdocname}
% \macro{\childdocjob}
% The macro |\childdocname| stores the name of the main document
% to be compiled. The macro |\childdocjob| stores the name of
% the document on which the \LaTeX{} compiler was originally invoked.
% The content of |\jobname| cannot be compared
% to filenames specified in the source due to different catcodes.
% The following code rescans |\jobname|, stores the result
% in |\childdocname| and saves a copy in |\childdocjob|:
%    \begin{macrocode}
\edef\childdocname{\scantokens\expandafter{\jobname\noexpand}}
\let\childdocjob\childdocname
%    \end{macrocode}

% \macro{\childdocdisable}
% The macro |\childdocdisable| prevents the main file
% from being processed more than once.
% At this stage, the main document command |\childdocmain|
% is assumed to be called once again where it should do nothing.
% Any subsequent call to it should prevent
% a secondary processing of the main document
% It overwrites the forwarding commands
% |\childdocof| and |\childdocforward|
% with empty macros to prevent further inclusions of the main document:
%    \begin{macrocode}
\newcommand{\childdocdisable}
{
  \renewcommand{\childdocmain}[1]{\renewcommand{\childdocmain}[1]{\endinput}}
  \renewcommand{\childdocof}[1]{}
  \renewcommand{\childdocby}[2][]{}
  \renewcommand{\childdocforward}[2][]{}
  \renewcommand{\childdocdisable}{}
}
%    \end{macrocode}

% \macro{\childdocmain}
% The macro |\childdocmain| is to be called at the top of the main file
% with nothing or the main filename (without extension) as argument.
% First, it breaks loops.
% If the argument is not empty and does not match |\childdocname|
% (which is set by the first inclusion of |childdoc.def|),
% |\ifchilddoc| is set to true, |\includeonly| is applied to the child file
% and |\jobname| is set to the main file
% (for proper handling of |.aux| files):
%    \begin{macrocode}
\newcommand{\childdocmain}[1]
{
  \childdocdisable\childdocmain{}
  \if?#1?\else
    \begingroup
      \def\childdoctmp{#1}
      \ifx\childdoctmp\childdocname
        \def\childdoctmp{}
      \else
        \def\childdoctmp
        {
          \childdoctrue
          \includeonly{\childdocname}
          \def\childdocjob{#1}
          \def\jobname{#1}
        }
      \fi
      \expandafter
    \endgroup
    \childdoctmp
  \fi
}
%    \end{macrocode}

% \macro{\childdocof}
% The command |\childdocof| redirects
% compilation to the main file |#1|.
%    \begin{macrocode}
\newcommand{\childdocof}[1]
{
  \childdocdisable
  \childdoctrue
  \includeonly{\childdocname}
  \def\jobname{#1}
  \def\childdocjob{#1}
  \input{#1}
}
%    \end{macrocode}

% \macro{\childdocby}
% The command |\childdocby| ....
%    \begin{macrocode}
\newcommand{\childdocby}[2][]
{
  \childdocdisable
  \childdoctrue
  \childdocmanualtrue
  \if?#1?\else
    \def\jobname{#2}
  \fi
  \def\childdocjob{#2}
  \input{#2}
  \endinput
}
%    \end{macrocode}

% \macro{\childdocforward}
% The command |\childdocforward| redirects
% compilation to the main file or
% (if the optional argument is given) a child file.
% Parameters are set as if the main file
% or a child file starting with |\childdocof| was compiled.
% Then compilation is handed over to the main file:
%    \begin{macrocode}
\newcommand{\childdocforward}[2][]
{
  \begingroup
    \if?#1?
      \def\childdoctmp
      {
        \def\childdocname{#2}
        \def\childdocjob{#2}
        \def\jobname{#2}
        \input{#2}
        \endinput
      }
    \else
      \def\childdoctmp
      {
        \childdocdisable
        \def\childdocname{#2}
        \childdoctrue
        \includeonly{#2}
        \def\childdocjob{#1}
        \def\jobname{#1}
        \input{#1}
        \endinput
      }
    \fi
    \expandafter
  \endgroup
  \childdoctmp
}
%    \end{macrocode}

% \macro{\childdocforwardprefix}
% The command |\childdocforwardprefix| redirects
% compilation to the main or a child file by means of a pattern.
% The prefix |#1| in the current filename is replaced by |#2|
% and the suffix of the current filename is kept
% (it is assumed that the filename does not contain the substring `|~~~|'
% which is used as a delimiter).
% Compilation is handed over to the new file by |\childdocforward|:
%    \begin{macrocode}
\newcommand{\childdocforwardprefix}[3][]
{
  \begingroup
    \def\childdocextract #2##1~~~{\def\childdoctmp{\childdocforward[#1]{#3##1}}}
    \expandafter\childdocextract\childdocname~~~
    \expandafter
  \endgroup
  \childdoctmp
}
%    \end{macrocode}

% \macro{\childdoc}
% The deprecated macro |\childdoc| is a legacy version of |\childdocmain|:
%    \begin{macrocode}
\newcommand{\childdoc}{\childdocmain}
%    \end{macrocode}

% \macro{\childdocredirect}
% The deprecated macro |\childdocredirect| is a legacy version
% of |\childdocforward| and |\childdocforwardprefix|:
%    \begin{macrocode}
\newcommand{\childdocredirect}[2][]
{
  \begingroup
    \if?#1?
      \def\childdoctmp{\childdocforward{#2}}
    \else
      \def\childdoctmp{\childdocforwardprefix{#1}{#2}}
    \fi
    \expandafter
  \endgroup
  \childdoctmp
}
%    \end{macrocode}

%\iffalse
%</package>
%\fi
%
\endinput

\childdocforwardprefix[cdocsamp]{cdocsfn}{cdocsch}
%    \end{macrocode}

%\iffalse
%</samplefinal>
%\fi
%
% %%%%%%%%%%%%%%%%%%%%%%%%%%%%%%%%%%%%%%
% \paragraph{Command Line Processing.}
%
% The following three command lines generate the output files
% |cdocscld|, |cdocscl1| and |cdocscl2|
% which should be identical to
% |cdocsdrf|, |cdocsch1| and |cdocsfn2|, respectively:
% \begin{center}
% \begin{tabular}{l}
% |latex -jobname cdocscld \|\\
% |  "\def\version{draft}% \iffalse
%
% childdoc.dtx Copyright (C) 2017-2018 Niklas Beisert
%
% This work may be distributed and/or modified under the
% conditions of the LaTeX Project Public License, either version 1.3
% of this license or (at your option) any later version.
% The latest version of this license is in
%   http://www.latex-project.org/lppl.txt
% and version 1.3 or later is part of all distributions of LaTeX
% version 2005/12/01 or later.
%
% This work has the LPPL maintenance status `maintained'.
%
% The Current Maintainer of this work is Niklas Beisert.
%
% This work consists of the files childdoc.dtx and childdoc.ins
% and the derived files childdoc.def and cdocsamp.tex with
% cdocsch1.tex, cdocsch2.tex, cdocsdrf.tex, cdocsfn1.tex, cdocsfn2.tex.
%
%<package>\ifdefined\childdocmain\endinput\fi
%<package>\ProvidesFile{childdoc.def}[2018/12/30 v2.0 child document driver]
%<samplemain>\ProvidesFile{cdocsamp.tex}[2018/12/30 v2.0 sample for childdoc]
%<*driver>
%\ProvidesFile{childdoc.drv}[2018/12/30 v2.0 childdoc reference manual file]
\PassOptionsToClass{10pt,a4paper}{article}
\documentclass{ltxdoc}

\usepackage[margin=35mm]{geometry}
\usepackage{hyperref}
\usepackage{hyperxmp}
\usepackage[usenames]{color}

\hypersetup{colorlinks=true}
\hypersetup{pdfstartview=FitH}
\hypersetup{pdfpagemode=UseNone}
\hypersetup{pdfsource={}}
\hypersetup{pdflang={en-UK}}
\hypersetup{pdfcopyright={Copyright 2017-2018 Niklas Beisert.
  This work may be distributed and/or modified under the
  conditions of the LaTeX Project Public License, either version 1.3
  of this license or (at your option) any later version.}}
\hypersetup{pdflicenseurl={http://www.latex-project.org/lppl.txt}}
\hypersetup{pdfcontactaddress={ETH Zurich, ITP, HIT K,
  Wolfgang-Pauli-Strasse 27}}
\hypersetup{pdfcontactpostcode={8093}}
\hypersetup{pdfcontactcity={Zurich}}
\hypersetup{pdfcontactcountry={Switzerland}}
\hypersetup{pdfcontactemail={nbeisert@itp.phys.ethz.ch}}
\hypersetup{pdfcontacturl={http://people.phys.ethz.ch/\xmptilde nbeisert/}}

\newcommand{\secref}[1]{\hyperref[#1]{section \ref*{#1}}}

\parskip1ex
\parindent0pt
\let\olditemize\itemize
\def\itemize{\olditemize\parskip0pt}

\begin{document}

\title{The \textsf{childdoc} Package}
\hypersetup{pdftitle={The childdoc Package}}
\author{Niklas Beisert\\[2ex]
  Institut f\"ur Theoretische Physik\\
  Eidgen\"ossische Technische Hochschule Z\"urich\\
  Wolfgang-Pauli-Strasse 27, 8093 Z\"urich, Switzerland\\[1ex]
  \href{mailto:nbeisert@itp.phys.ethz.ch}
  {\texttt{nbeisert@itp.phys.ethz.ch}}}
\hypersetup{pdfauthor={Niklas Beisert}}
\hypersetup{pdfsubject={Manual for the LaTeX2e Package childdoc}}
\date{30 December 2018, \textsf{v2.0}}
\maketitle

\begin{abstract}\noindent
\textsf{childdoc} is a \LaTeXe{} package
that enables the direct compilation
of document sections included by |\include|
to individual files.
\end{abstract}

\begingroup
\parskip0ex
\tableofcontents
\endgroup

%%%%%%%%%%%%%%%%%%%%%%%%%%%%%%%%%%%%%%%%%%%%%%%%%%%%%%%%%%%%%%%%%%%%%%%%%%%%%%%%
%%%%%%%%%%%%%%%%%%%%%%%%%%%%%%%%%%%%%%%%%%%%%%%%%%%%%%%%%%%%%%%%%%%%%%%%%%%%%%%%
\section{Introduction}

\LaTeX{} provides a mechanism to structure a large document (such as a book)
into a main file and several child files (containing the chapters)
using the |\include| command.
This mechanism is beneficial for documents
which span hundreds of pages in order to
make the source file(s) more manageable.
Moreover, compilation can be restricted to
selected child files by means of the |\includeonly| command.
The latter feature can be used to reduce the compilation time while editing
(this was significantly more useful in the earlier days of \LaTeX{})
or to generate a smaller document which is easier to navigate.
Another application of |\includeonly| is to generate
documents consisting of selected parts of the complete document.

However, there are a few drawbacks of the plain |\include| mechanism:
\begin{itemize}
\item
The child files cannot be compiled on their own,
they can only be compiled via the main file.
A naive editing environment
(such as a text editor with an option
to have the current file processed by \LaTeX)
may require one to switch to the main file before compiling;
attempting to compile the child file produces errors.
\item
The main file must be modified (each time)
to adjust the |\includeonly| command
to the present needs. This easily leaves the main file in a messy state.
\item
The generated document will always carry the filename
of the main document. This is inconvenient if
several child files are to be compiled and
to be kept for distribution.
\end{itemize}

The present package provides a simple interface
to make child files individually compilable by \LaTeX{}.
Compiling a child file then has the same effect as compiling
the main file with an |\includeonly| command
to select the appropriate child.
Moreover the generated document will carry the name of the child
rather than the main file.
This resolves all three above issues.

This feature is meant to make the editing of books,
thesis documents and lecture notes somewhat more convenient.
However, the package can also be used efficiently for
composing a series of documents (such as exercise sheets)
which are typically distributed individually.
It then assists the author in generating the individual documents
(potentially in different versions)
as well as a document containing the collected series.
Another application is in developing style files
or other kinds of included material
where compilation of the style file could redirect
to a sample or test file.

%%%%%%%%%%%%%%%%%%%%%%%%%%%%%%%%%%%%%%%%%%%%%%%%%%%%%%%%%%%%%%%%%%%%%%%%%%%%%%%%
%%%%%%%%%%%%%%%%%%%%%%%%%%%%%%%%%%%%%%%%%%%%%%%%%%%%%%%%%%%%%%%%%%%%%%%%%%%%%%%%
\section{Usage}

First of all, the package \textsf{childdoc} is \emph{not} a standard
\LaTeXe{} |.sty| style file! Therefore it needs to be invoked in
a non-standard way.

%%%%%%%%%%%%%%%%%%%%%%%%%%%%%%%%%%%%%%%%%%%%%%%%%%%%%%%%%%%%%%%%%%%%%%%%%%%%%%%%
\subsection{Included Files}
\label{sec:include}

%%%%%%%%%%%%%%%%%%%%%%%%%%%%%%%%%%%%%%%%
\DescribeMacro{\childdocmain}
To use the package, add the commands
\begin{center}
\begin{tabular}{l}
|\input{childdoc.def}|\\
|\childdocmain{}|\\
\end{tabular}
\end{center}
at the very top of the main \LaTeX{} file,
in particular \emph{before} the |\documentclass| statement!
The argument of |\childdocmain| should be left empty
(but it must be present).

%%%%%%%%%%%%%%%%%%%%%%%%%%%%%%%%%%%%%%%%
\DescribeMacro{\childdocof}
Furthermore, add the commands
\begin{center}
\begin{tabular}{l}
|\input{childdoc.def}|\\
|\childdocof{|\textit{main}|}|\\
\end{tabular}
\end{center}
at the top of every child file \textit{child}
which is included by |\include{|\textit{child}|}|
from within the main file
(or at least for those files to be compiled individually).
The argument \textit{main} must be the filename of the main file.

There are a couple of
considerations in setting up the main and child documents:

%%%%%%%%%%%%%%%%%%%%%%%%%%%%%%%%%%%%%%%%
\paragraph{Restrictions.}

Please note the following restrictions:
\begin{itemize}
\item
|\childdocmain| must be called with one argument \textit{main}
to ensure compatibility with earlier version of the package.
It must either be empty (|\childdocmain{}|)
or precisely match the filename of the main file in which it is specified.
See \secref{sec:detection} for further information.
\item
The filename \textit{main} must be specified without the |.tex| extension.
\item
The filename \textit{main} is case sensitive
(even in case-insensitive file systems)
due to internal string comparison.
\item
The argument \textit{main} should be fully expanded, it cannot be a macro.
\item
Subdirectories and special characters should be avoided in filenames.
\item
The command |\childdocmain{|\textit{main}|}| must be followed by a whitespace.
It should not be followed immediately by another command
or by a comment mark `|%|'.
This is because the \TeX{} parser reads the token immediately following
the argument of |\childdocmain| and puts it
at the beginning of every child section;
however, a white\-space is ignored.
\end{itemize}

%%%%%%%%%%%%%%%%%%%%%%%%%%%%%%%%%%%%%%%%
\paragraph{Content of Main File.}

It is advisable to place all content in the child files included by |\include|.
Any output contained in the main file will appear in all child documents
unless suppressed manually;
it cannot be suppressed automatically by the |\includeonly| directive
and thus should normally be avoided.
A method to include some content in the main file
by means of conditional processing is described in \secref{sec:conditional}.

%%%%%%%%%%%%%%%%%%%%%%%%%%%%%%%%%%%%%%%%
\paragraph{Page Numbering.}

When only a part of the document is compiled,
the appropriate numbering of pages
(as well as other status parameters)
is determined from the |.aux| files.
The latter contain information from previous passes.
However this information needs to propagate through
all intermediate child documents.
Therefore the page numbering in child documents may well
be inconsistent until the complete document is compiled at least once.

A useful (if unconventional) way to always ensure a consistent
page numbering is to restart the numbering in each child document
and denote the pages by `\textit{child}|.|\textit{page}'
where \textit{child} represents the chapter/section number of the child file.
This can be achieved by the command
|\numberwithin{page}{|\textit{child}|}|
of the \textsf{amsmath} package
where \textit{child} can be |chapter| or |section|
depending on the chosen structuring.
Alternatively, one can modify the macro |\thepage| appropriately
and reset the counter |page| at the start of each child file.

%%%%%%%%%%%%%%%%%%%%%%%%%%%%%%%%%%%%%%%%%%%%%%%%%%%%%%%%%%%%%%%%%%%%%%%%%%%%%%%%
\subsection{Conditional Processing}
\label{sec:conditional}

The package provides a mechanism to compile different versions
of a document. To customise the versions further some conditional processing
can come in handy to distinguish which version is being compiled.
The package provides two macros to describe the compilation context:

%%%%%%%%%%%%%%%%%%%%%%%%%%%%%%%%%%%%%%%%
\DescribeMacro{\ifchilddoc}
The conditional |\ifchilddoc| distinguishes between the compilation of
child documents and the main document:
%
\begin{center}
|\ifchilddoc |\textit{child-code}| |[|\||else |\textit{main-code}]| \||fi|
\end{center}

%%%%%%%%%%%%%%%%%%%%%%%%%%%%%%%%%%%%%%%%
\DescribeMacro{\childdocname}
\DescribeMacro{\childdocjob}
The macro |\childdocname| contains the filename (without extension)
of the main or child file being processed.
Note that |\childdocjob| will always contain the name of the main file.

%%%%%%%%%%%%%%%%%%%%%%%%%%%%%%%%%%%%%%%%
\paragraph{Title Page.}

Conditional processing can be used to include a title or banner page
in the main document when proper precautions are taken.
Importantly, the code in the main file should ensure that the page counter
(as well as other status parameters which are stored in the |.aux| files)
takes the same value after the conditional processing.
Otherwise the page numbers may take divergent values
depending on which part is compiled.

For example, a title page could be declared by:
%
\begin{center}
\begin{tabular}{l}
|\ifchilddoc\||else|\\
|\addtocounter{page}{-1}|\\
\textit{code for title page}\\
|\newpage|\\
|\||fi|
\end{tabular}
\end{center}
%
A banner page for the child documents can be generated by:
%
\begin{center}
\begin{tabular}{l}
|\ifchilddoc|\\
|\addtocounter{page}{-1}|\\
\textit{code for banner page}\\
|\newpage|\\
|\||fi|
\end{tabular}
\end{center}
%
Here one could write a message such as:
\begin{center}
|This is the part \childdocname{} of \childdocjob{}.|
\end{center}

%%%%%%%%%%%%%%%%%%%%%%%%%%%%%%%%%%%%%%%%%%%%%%%%%%%%%%%%%%%%%%%%%%%%%%%%%%%%%%%%
\subsection{Flags}
\label{sec:flags}

The package makes it easy to generate different versions
of the main or child documents.
To this end compilation flags can be defined
and assigned different default values.
They will be particularly useful in conjunction
with the forwarding mechanism described in \secref{sec:forward}.

For example, it may be useful to have a flag |\version|
which can be set to |draft| or |final|.
The document source will contain some conditional code
depending on the value of |\version|.
Suppose further, the flag should default to |final| for the main file
and to |draft| for child files
which is a natural assignment for editing the document.
This is achieved by placing the following code
in the preamble of the main document
(below the |\childdocmain| directive):
%
\begin{center}
\begin{tabular}{l}
|\ifchilddoc|\\
|\providecommand{\version}{draft}|\\
|\||else|\\
|\providecommand{\version}{final}|\\
|\||fi|
\end{tabular}
\end{center}
%
The definition by |\providecommand| makes sure
that previous definitions are not overwritten.
Further statements |\providecommand{\version}{...}|
can thus be added before the above code to override it.

For the main file, one might add a line
(between |\childdocmain| and the above block)
%
\begin{center}
|%\ifchilddoc\||else\providecommand{\version}{draft}\||fi|
\end{center}
%
which can be uncommented to produce a draft version.
Likewise one can add a line to the very top of a child file
(above the |\childdocof{|\textit{main}|}| directive)
%
\begin{center}
|%\providecommand{\version}{final}|
\end{center}
%
which can be uncommented to produce the final version of this child document.

%%%%%%%%%%%%%%%%%%%%%%%%%%%%%%%%%%%%%%%%%%%%%%%%%%%%%%%%%%%%%%%%%%%%%%%%%%%%%%%%
\subsection{Forwarding}
\label{sec:forward}

Different versions of the main or child documents
using compilation flags as described in \secref{sec:flags}
can be (permanently) stored in different files
for convenient compilation, viewing and distribution.
To this end, the package defines a command
to pass on compilation to a different file:

%%%%%%%%%%%%%%%%%%%%%%%%%%%%%%%%%%%%%%%%
\DescribeMacro{\childdocforward}
The command |\childdocforward| redirects processing to
another source file:
%
\begin{center}
\begin{tabular}{l}
|\input{childdoc.def}|\\
|\childdocforward[|\textit{main}|]{|\textit{dest}|}|\\
\end{tabular}
\end{center}
%
The argument \textit{dest} is the destination file
(without extension).
It should be the main file or one of the child files.
Note that further \textsf{childdoc} directives
such as |\childdocof| and |\childdocforward|
in the indicated file will be processed in this form.
The optional argument \textit{main}
passes on directly to the main file \textit{main}
while pretending to compile the child \textit{dest}.
This form behaves as if \textit{dest}
issues |\childdocof{|\textit{main}|}| right away,
and no further \textsf{childdoc} directives will be processed.

%%%%%%%%%%%%%%%%%%%%%%%%%%%%%%%%%%%%%%%%
\DescribeMacro{\...prefix}
In the alternative form |\childdocforwardprefix|,
%
\begin{center}
\begin{tabular}{l}
|\input{childdoc.def}|\\
|\childdocforwardprefix[|\textit{main}|]{|\textit{prefix}|}{|\textit{dest}|}|
\end{tabular}
\end{center}
%
the destination file is determined by a pattern
depending on the current file:
To make this work, the current file must be called
`{\textit{prefix}\hspace{0.2em}\textit{suffix}}'
with \textit{prefix} matching precisely the argument.
Processing is then passed on to the file
`{\textit{dest}\hspace{0.2em}\textit{suffix}}'.
Surely, the same effect is achieved by
directly specifying the
argument `{\textit{dest}\hspace{0.2em}\textit{suffix}}'
in the first form.
However, that requires to set up a different file
for each child. With the alternative form of the command
all these files can have exactly the same content
which simplifies setting them up and maintaining them.

For example, the following file |draft.tex|
with a compilation flag |\version| as described in \secref{sec:flags}
compiles the main document as a draft:
%
\begin{center}
\begin{tabular}{l}
|\def\version{draft}|\\
|\input{childdoc.def}|\\
|\childdocforward{|\textit{main}|}|
\end{tabular}
\end{center}
%
Likewise, the following files |final|\textit{nn}|.tex|
compile the final version of the child document
|child|\textit{nn}|.tex|:
%
\begin{center}
\begin{tabular}{l}
|\def\version{final}|\\
|\input{childdoc.def}|\\
|\childdocforwardprefix{final}{child}|
\end{tabular}
\end{center}
%

Note that when several versions of a main file and/or of each child file
are to be generated, it may be convenient to set up a |Makefile| or
shell script to automatise the process.

%%%%%%%%%%%%%%%%%%%%%%%%%%%%%%%%%%%%%%%%%%%%%%%%%%%%%%%%%%%%%%%%%%%%%%%%%%%%%%%%
\subsection{Command Line Processing}
\label{sec:commandline}

The effect of redirection files can also be achieved by invoking
the \LaTeX{} compiler with a more elaborate command line.
Most conveniently this should be done as part
of a shell script or a |Makefile|.

When using \textsf{childdoc} in the main file, the following
command lines effectively perform a redirection
(note that depending on the shell being used,
backslashes may have to be doubled: `|\|' $\to$ `|\\|'):
%
\begin{center}
|... -jobname "|\textit{target}|" |\\|"|[\textit{flags}]%
|\input{childdoc.def}\childdocforward[|\textit{main}|]{|\textit{dest}|}"|
\end{center}
%
Here \textit{target} is the name of the output file,
\textit{main} is the name of the main file
and \textit{dest} is the name of the main or child file to be processed
(all filenames without extensions).
The optional argument \textit{main} can be omitted
if \textit{main} matches \textit{dest}.
Optionally, compilation \textit{flags} can be defined via |\def| commands.
This command line makes the \TeX{} engine believe
it is compiling the file \textit{target}
whose content is specified as the latter parameter.
The provided code then forwards the processing to
\textit{main} or \textit{dest} as described in \secref{sec:forward}.

%%%%%%%%%%%%%%%%%%%%%%%%%%%%%%%%%%%%%%%%%%%%%%%%%%%%%%%%%%%%%%%%%%%%%%%%%%%%%%%%
\subsection{Include by Input}
\label{sec:input}

Including child documents by |\include| has some restrictions by design.
Most notably, the content of a child document always occupies
its own set of pages; pages cannot be shared between child documents.
Usually, this behaviour makes perfect sense
because each child document contain an essential part of the document.
However, in some situations it may be desirable to compose
a document from a collection of parts
without having mandatory page breaks between then.
For this case, the package
provides a mechanism to include parts
by |\input| which can also be processed individually.
However, by construction this mechanism
requires manual handling of the content to be output.

%%%%%%%%%%%%%%%%%%%%%%%%%%%%%%%%%%%%%%%%
\DescribeMacro{\ifchilddocmanual}
The main file should be prepared as usual, see \secref{sec:include}.
However, the document body must make a distinction
between processing of an individual part and of the main document, e.g.:
%
\begin{center}
\begin{tabular}{l}
|\ifchilddocmanual|\\
|\input{\childdocname}|\\
|\||else|\\
\textit{document body with }|\input{|\textit{part}|}|\\
|\||fi|
\end{tabular}
\end{center}
%
The conditional |\ifchilddocmanual| is true whenever
a part to be included by |\input| is being compiled,
and the name of the part is stored in |\childdocname|.

%%%%%%%%%%%%%%%%%%%%%%%%%%%%%%%%%%%%%%%%
\DescribeMacro{\childdocby}
Each part to be included by |\input| should start with:
%
\begin{center}
\begin{tabular}{l}
|\input{childdoc.def}|\\
|\childdocby{|\textit{main}|}|\\
\end{tabular}
\end{center}
%
The directive |\childdocby| is similar to |\childdocof|
described in \secref{sec:include},
but the subsequent selection of content must be done manually.
To that end, both |\ifchilddoc| and |\ifchilddocmanual|
will be true upon processing of a part,
and the name of the part is stored in |\childdocname|.
Note that |\jobname| will be set to the filename of the current part
so that each part receives an individual |.aux| file
that does not interfere with the |.aux| file(s) of the main document.
This behaviour can be altered by the alternative form
|\childdocby[*]{|\textit{main}|}| (with a non-empty optional argument)
which uses the |.aux| file of the main document
by setting |\jobname| to \textit{main}.

%%%%%%%%%%%%%%%%%%%%%%%%%%%%%%%%%%%%%%%%%%%%%%%%%%%%%%%%%%%%%%%%%%%%%%%%%%%%%%%%
\subsection{Driver Development}
\label{sec:driver}

The \textsf{childdoc} mechanism can also be use for the development
of definition files such as \LaTeX{} styles or classes.
This case differs from the above setup with multiple parts
included by |\include| in that no |\includeonly| should be invoked.
This can be achieved by starting the include file
(before |\ProvidesPackage|) with:
%
\begin{center}
\begin{tabular}{l}
|\input{childdoc.def}|\\
|\childdocforward{|\textit{main}|}|\\
\end{tabular}
\end{center}
%
or alternatively with:
%
\begin{center}
\begin{tabular}{l}
|\input{childdoc.def}|\\
|\childdocby{|\textit{main}|}|\\
\end{tabular}
\end{center}
%
Both forms have slightly different effects as described above.
The main file is prepared as usual, see \secref{sec:include}.

%%%%%%%%%%%%%%%%%%%%%%%%%%%%%%%%%%%%%%%%%%%%%%%%%%%%%%%%%%%%%%%%%%%%%%%%%%%%%%%%
\subsection{Legacy Detection}
\label{sec:detection}

The directive |\childdocmain| in the main file can detect
whether the complete document or merely a child is to be compiled
even without using the directive |\childdocof|.
This method is deprecated because it is less robust
and there is no compelling reason to use it;
it is merely provided for backward compatibility
and it may be removed in future versions.

If the detection mechanism is to be used,
it is mandatory to correctly specify
the filename of the main file as the argument of |\childdocmain|:
%
\begin{center}
\begin{tabular}{l}
|\input{childdoc.def}|\\
|\childdocmain{|\textit{main}|}|\\
\end{tabular}
\end{center}
%
If |\jobname| does not match the argument \textit{main} of |\childdocmain|,
it is assumed that |\jobname| points to the child file to be compiled.
When using |\childdocmain| with the main file specified as argument,
it suffices to start a child file
with just |\input{|\textit{main}|}|
without loading of the package and using |\childdocof|.
If instead all processing is done
with the appropriate \textsf{childdoc} directives,
the argument of \textit{main} of |\childdocmain| can be empty.

An alternative version of the command line processing described
in \secref{sec:commandline} using the detection mechanism reads:
%
\begin{center}
|... -jobname "|\textit{target}|" "|[\textit{flags}]%
[|\def\jobname{|\textit{dest}|}|]|\input{|\textit{main}|}"|
\end{center}

%%%%%%%%%%%%%%%%%%%%%%%%%%%%%%%%%%%%%%%%%%%%%%%%%%%%%%%%%%%%%%%%%%%%%%%%%%%%%%%%
\subsection{Manual Code}
\label{sec:manual}

In case one cannot be certain whether the definitions file |childdoc.def|
is installed on the target \TeX{} distribution
and one prefers not to ship it,
it is conceivable to paste a few relevant commands into the sources.

To that end, drop all statements |\input{childdoc.def}|
and perform the replacements as outlined below.
Instead of |\childdocmain{|\textit{main}|}| add the following code
to the top of the main file:
%
\begin{center}
\begin{tabular}{l}
|\||ifdefined\childdocname\endinput\||fi\newif\ifchilddoc|\\
|\edef\childdocname{\scantokens\expandafter{\jobname\noexpand}}|\\
|\def\childdocmain{|\textit{main}|}\||ifx\childdocmain\childdocname\||else|\\
|\childdoctrue\includeonly{\childdocname}\let\jobname\childdocmain\||fi|\\
\end{tabular}
\end{center}
%
Instead of |\childdocof{|\textit{main}|}| just include the main file
at the top of each child file:
%
\begin{center}
|\input{|\textit{main}|}|
\end{center}
%
A simple redirection |\childdocforward{|\textit{dest}|}| is achieved by:
%
\begin{center}
|\def\jobname{|\textit{dest}|}\input{\jobname}|
\end{center}
%
The redirection with prefix
|\childdocforwardprefix[|\textit{prefix}|]{|\textit{dest}|}|
is accomplished by:
%
\begin{center}
\begin{tabular}{l}
|{\edef\jobname{\scantokens\expandafter{\jobname\noexpand}}|\\
|\def\redirectjob |\textit{prefix}|#1~~~{\gdef\jobname{|\textit{dest}|#1}}|\\
|\expandafter\redirectjob\jobname~~~}\input{\jobname}|
\end{tabular}
\end{center}

In an alternative approach,
child documents can be compiled by a specific command line
without additional code or specific definitions:
%
\begin{center}
|... -jobname "|\textit{target}|" "|[\textit{flags}]%
|\includeonly{|\textit{dest}|}\input{|\textit{main}|}"|
\end{center}
%

%%%%%%%%%%%%%%%%%%%%%%%%%%%%%%%%%%%%%%%%%%%%%%%%%%%%%%%%%%%%%%%%%%%%%%%%%%%%%%%%
%%%%%%%%%%%%%%%%%%%%%%%%%%%%%%%%%%%%%%%%%%%%%%%%%%%%%%%%%%%%%%%%%%%%%%%%%%%%%%%%
\section{Information}

%%%%%%%%%%%%%%%%%%%%%%%%%%%%%%%%%%%%%%%%%%%%%%%%%%%%%%%%%%%%%%%%%%%%%%%%%%%%%%%%
\subsection{Copyright}

Copyright \copyright{} 2017--2018 Niklas Beisert

This work may be distributed and/or modified under the
conditions of the \LaTeX{} Project Public License, either version 1.3
of this license or (at your option) any later version.
The latest version of this license is in
  \url{http://www.latex-project.org/lppl.txt}
and version 1.3 or later is part of all distributions of \LaTeX{}
version 2005/12/01 or later.

This work has the LPPL maintenance status `maintained'.

The Current Maintainer of this work is Niklas Beisert.

This work consists of the files |README.txt|, |childdoc.ins| and |childdoc.dtx|
as well as the derived files |childdoc.def|, |cdocsamp.tex|
with |cdocsch1.tex|, |cdocsch2.tex|, |cdocspt3.tex|, |cdocspt4.tex|,
|cdocsdrf.tex|, |cdocsfn1.tex|, |cdocsfn2.tex|
as well as |childdoc.pdf|.

%%%%%%%%%%%%%%%%%%%%%%%%%%%%%%%%%%%%%%%%%%%%%%%%%%%%%%%%%%%%%%%%%%%%%%%%%%%%%%%%
\subsection{Files and Installation}

The package consists of the files:
%
\begin{center}
\begin{tabular}{ll}
    |README.txt|   & readme file \\
    |childdoc.ins| & installation file \\
    |childdoc.dtx| & source file \\
    |childdoc.def| & definition file \\
    |cdocsamp.tex| & sample main file \\
    |cdocsch1.tex| & sample include file \\
    |cdocsch2.tex| & sample include file \\
    |cdocspt3.tex| & sample part file \\
    |cdocspt4.tex| & sample part file \\
    |cdocsdrf.tex| & sample redirection file \\
    |cdocsfn1.tex| & sample redirection file \\
    |cdocsfn2.tex| & sample redirection file \\
    |childdoc.pdf| & manual
\end{tabular}
\end{center}
%
The distribution consists of the files
|README.txt|, |childdoc.ins| and |childdoc.dtx|.
%
\begin{itemize}
\item
Run (pdf)\LaTeX{} on |childdoc.dtx|
to compile the manual |childdoc.pdf| (this file).
\item
Run \LaTeX{} on |childdoc.ins| to create the definitions file |childdoc.def|
and the sample |cdocsamp.tex| with include files
|cdocsch1.tex|, |cdocsch2.tex|, |cdocspt3.tex|, |cdocspt4.tex|,
|cdocsdrf.tex|, |cdocsfn1.tex|, |cdocsfn2.tex|.
Then copy the file |childdoc.def| to an appropriate directory of your \LaTeX{}
distribution, e.g.\ \textit{texmf-root}|/tex/latex/childdoc|.
\end{itemize}

%%%%%%%%%%%%%%%%%%%%%%%%%%%%%%%%%%%%%%%%%%%%%%%%%%%%%%%%%%%%%%%%%%%%%%%%%%%%%%%%
\subsection{Related CTAN Packages}

There are several other packages which offer a similar functionality:
%
\begin{itemize}
\item
The packages
\href{http://ctan.org/pkg/docmute}{\textsf{docmute}},
\href{http://ctan.org/pkg/includex}{\textsf{includex}} and
\href{http://ctan.org/pkg/standalone}{\textsf{standalone}}
provide commands to include only the document body of
a child file thus allowing both files to be compiled individually.
\item
The packages \href{http://ctan.org/pkg/subdocs}{\textsf{subdocs}}
and \href{http://ctan.org/pkg/subfiles}{\textsf{subfiles}}
provide structures in which the main and child documents can be
encapsulated and allowing them to be compiled individually.
The inclusion mechanism is different from the conventional |\include|.
\item
The package \href{http://ctan.org/pkg/combine}{\textsf{combine}}
is an elaborate solution to combine several documents into one.
\end{itemize}
%
See also the CTAN topic \href{http://ctan.org/topic/subdocs}{\textsf{subdocs}}
for further related packages.
The present package differs from the above solutions in that
a document structure constructed with the conventional |\include| mechanism
just needs two extra commands at the top of every file
such that all constituent files can be compiled individually.

%%%%%%%%%%%%%%%%%%%%%%%%%%%%%%%%%%%%%%%%%%%%%%%%%%%%%%%%%%%%%%%%%%%%%%%%%%%%%%%%
%\subsection{Feature Suggestions}
%
%The following is a list of features which may be useful for future
%versions of this package:
%%
%\begin{itemize}
%\item
%\ldots
%\end{itemize}

%%%%%%%%%%%%%%%%%%%%%%%%%%%%%%%%%%%%%%%%%%%%%%%%%%%%%%%%%%%%%%%%%%%%%%%%%%%%%%%%
\subsection{Revision History}

%%%%%%%%%%%%%%%%%%%%%%%%%%%%%%%%%%%%%%%%
\paragraph{v2.0:} 2018/12/30

\begin{itemize}
\item
immediate forward processing
\item
added |\childdocby| mechanism
\item
manual restructured
\end{itemize}

%%%%%%%%%%%%%%%%%%%%%%%%%%%%%%%%%%%%%%%%
\paragraph{v1.6:} 2018/01/17

\begin{itemize}
\item
application for development of include files
\item
corrections to manual
\end{itemize}

%%%%%%%%%%%%%%%%%%%%%%%%%%%%%%%%%%%%%%%%
\paragraph{v1.5:} 2017/05/21

\begin{itemize}
\item
more complete structuring introduced
\item
|\childdocof| introduced
\item
|\childdoc| renamed to |\childdocmain|
\item
|\childredirect| renamed to |\childdocforward| and |\childdocforwardprefix|
and functionality expanded
\end{itemize}

%%%%%%%%%%%%%%%%%%%%%%%%%%%%%%%%%%%%%%%%
\paragraph{v1.0:} 2017/04/27

\begin{itemize}
\item
manual and install package
\item
first version published on CTAN
\end{itemize}

%%%%%%%%%%%%%%%%%%%%%%%%%%%%%%%%%%%%%%%%
\paragraph{v0.6:} 2017/04/26

\begin{itemize}
\item
redirection mechanism added
\end{itemize}

%%%%%%%%%%%%%%%%%%%%%%%%%%%%%%%%%%%%%%%%
\paragraph{v0.5:} 2017/04/26

\begin{itemize}
\item
functionality in definition file
\end{itemize}


%%%%%%%%%%%%%%%%%%%%%%%%%%%%%%%%%%%%%%%%%%%%%%%%%%%%%%%%%%%%%%%%%%%%%%%%%%%%%%%%
%%%%%%%%%%%%%%%%%%%%%%%%%%%%%%%%%%%%%%%%%%%%%%%%%%%%%%%%%%%%%%%%%%%%%%%%%%%%%%%%
%%%%%%%%%%%%%%%%%%%%%%%%%%%%%%%%%%%%%%%%%%%%%%%%%%%%%%%%%%%%%%%%%%%%%%%%%%%%%%%%
\appendix

\settowidth\MacroIndent{\rmfamily\scriptsize 000\ }

 \DocInput{childdoc.dtx}

\end{document}
%</driver>
% \fi
%
% %%%%%%%%%%%%%%%%%%%%%%%%%%%%%%%%%%%%%%%%%%%%%%%%%%%%%%%%%%%%%%%%%%%%%%%%%%%%%%
% %%%%%%%%%%%%%%%%%%%%%%%%%%%%%%%%%%%%%%%%%%%%%%%%%%%%%%%%%%%%%%%%%%%%%%%%%%%%%%
% \section{Sample}
%\iffalse
%<*samplemain>
%\fi
%
% The following presents a sample document
% with two chapters, two parts, a title page,
% a compile flag as well as three forwarding files to set the flag.
% It consists of eight |.tex| files:
% \begin{center}
% \begin{tabular}{ll}
% |cdocsamp.tex|&main file\\
% |cdocsch1.tex|&include file for chapter 1\\
% |cdocsch2.tex|&include file for chapter 2\\
% |cdocspt3.tex|&include file for part 3\\
% |cdocspt4.tex|&include file for part 4\\
% |cdocsdrf.tex|&forwarding file for main file in draft mode\\
% |cdocsfi1.tex|&forwarding file for final version of chapter 1\\
% |cdocsfi2.tex|&forwarding file for final version of chapter 2\\
% \end{tabular}
% \end{center}
% Each of the eight files can be compiled directly by the \LaTeX{} compiler.
%
% %%%%%%%%%%%%%%%%%%%%%%%%%%%%%%%%%%%%%%
% \paragraph{Main File.}
%
% The main file is called |cdocsamp.tex|.
%
% Load the \textsf{childdoc} definitions and
% declare the filename for the main document:
%    \begin{macrocode}
\input{childdoc.def}
\childdocmain{}
%    \end{macrocode}

% Optional override for |\version| flag:
%    \begin{macrocode}
%%\ifchilddoc\else\providecommand{\version}{draft}\fi
%    \end{macrocode}

% Define the default values for the |\version| flag
% (|final| for the main file and |draft| for childs):
%    \begin{macrocode}
\ifchilddoc
\providecommand{\version}{draft}
\else
\providecommand{\version}{final}
\fi
%    \end{macrocode}

% Load the standard document class:
%    \begin{macrocode}
\documentclass[12pt]{article}
%    \end{macrocode}

% Start the document body:
%    \begin{macrocode}
\begin{document}
%    \end{macrocode}

% Declare a title page.
% Print title, part of document being processed and version flag:
%    \begin{macrocode}
\addtocounter{page}{-1}
\begin{center}
{\LARGE\bfseries{}childdoc example\par}
\vspace{1cm}
\ifchilddoc
\ifchilddocmanual part\else chapter\fi:
`\childdocname' of `\childdocjob'\par
\else
main document: `\childdocjob'\par
\fi
version: \version\par
\end{center}
\newpage
%    \end{macrocode}

% Manually include selected file,
% otherwise process as usual:
%    \begin{macrocode}
\ifchilddocmanual
\section*{part `\childdocname'}
\input{\childdocname}
\else
%    \end{macrocode}

% Include the two chapters:
%    \begin{macrocode}
\include{cdocsch1}
\include{cdocsch2}
%    \end{macrocode}

% Include the two parts unless only chapters should be displayed:
%    \begin{macrocode}
\ifchilddoc\else
\section{part three}
\input{cdocspt3}
\section{part four}
\input{cdocspt4}
\fi
%    \end{macrocode}

% Process as usual until here:
%    \begin{macrocode}
\fi
%    \end{macrocode}

% End of document body:
%    \begin{macrocode}
\end{document}
%    \end{macrocode}
%\iffalse
%</samplemain>
%\fi
%
% %%%%%%%%%%%%%%%%%%%%%%%%%%%%%%%%%%%%%%
% \paragraph{Chapter Include Files.}
%
% The include files are called |cdocsch1.tex| and |cdocsch2.tex|.
%
%\iffalse
%<*samplechap1|samplechap2>
%\fi

% Optional override for |\version| flag:
%    \begin{macrocode}
%%\providecommand{\version}{final}
%    \end{macrocode}

% Include the main document:
%    \begin{macrocode}
\input{childdoc.def}
\childdocof{cdocsamp}
%    \end{macrocode}

%\iffalse
%</samplechap1|samplechap2>
%\fi
%
%\iffalse
%<*samplechap1>
%\fi
% Some text for chapter 1:
%    \begin{macrocode}
\section{one}
some text in chapter one
%    \end{macrocode}

%\iffalse
%</samplechap1>
%\fi
% Some text for chapter 2:
%\iffalse
%<*samplechap2>
%\fi
%    \begin{macrocode}
\section{two}
more text in chapter two
%    \end{macrocode}

%\iffalse
%</samplechap2>
%\fi
%
% %%%%%%%%%%%%%%%%%%%%%%%%%%%%%%%%%%%%%%
% \paragraph{Part Include Files.}
%
% The include files are called |cdocspt3.tex| and |cdocspt4.tex|.
%
%\iffalse
%<*samplepart3|samplepart4>
%\fi

% Optional override for |\version| flag:
%    \begin{macrocode}
%%\providecommand{\version}{final}
%    \end{macrocode}

% Include the main document:
%    \begin{macrocode}
\input{childdoc.def}
\childdocby{cdocsamp}
%    \end{macrocode}

%\iffalse
%</samplepart3|samplepart4>
%\fi
%
%\iffalse
%<*samplepart3>
%\fi
% Some text for part 3:
%    \begin{macrocode}
some text in part three
%    \end{macrocode}

%\iffalse
%</samplepart3>
%\fi
% Some text for part 4:
%\iffalse
%<*samplepart4>
%\fi
%    \begin{macrocode}
more text in part four
%    \end{macrocode}

%\iffalse
%</samplepart4>
%\fi
%
% %%%%%%%%%%%%%%%%%%%%%%%%%%%%%%%%%%%%%%
% \paragraph{Forwarding for a Complete Draft.}
%
% The following forwarding file |cdocsdrf.tex|
% compiles the main document in draft mode:
%\iffalse
%<*sampledraft>
%\fi
%    \begin{macrocode}
\def\version{draft}
\input{childdoc.def}
\childdocforward{cdocsamp}
%    \end{macrocode}

%\iffalse
%</sampledraft>
%\fi
%
% %%%%%%%%%%%%%%%%%%%%%%%%%%%%%%%%%%%%%%
% \paragraph{Forwarding for Final Version of the Chapters.}
%
% The following forwarding files |cdocsfn1.tex| and |cdocsfn2.tex|
% (with identical content)
% compile the final versions of the child documents
% |cdocsch1.tex| and |cdocsch2.tex|, respectively:
%\iffalse
%<*samplefinal>
%\fi
%    \begin{macrocode}
\def\version{final}
\input{childdoc.def}
\childdocforwardprefix[cdocsamp]{cdocsfn}{cdocsch}
%    \end{macrocode}

%\iffalse
%</samplefinal>
%\fi
%
% %%%%%%%%%%%%%%%%%%%%%%%%%%%%%%%%%%%%%%
% \paragraph{Command Line Processing.}
%
% The following three command lines generate the output files
% |cdocscld|, |cdocscl1| and |cdocscl2|
% which should be identical to
% |cdocsdrf|, |cdocsch1| and |cdocsfn2|, respectively:
% \begin{center}
% \begin{tabular}{l}
% |latex -jobname cdocscld \|\\
% |  "\def\version{draft}\input{childdoc.def}\childdocforward{cdocsamp}"|\\
% |latex -jobname cdocscl1 \|\\
% |  "\input{childdoc.def}\childdocforward[cdocsamp]{cdocsch1}"|\\
% |latex -jobname cdocscl2 \|\\
% |  "\def\version{final}\input{childdoc.def}\childdocforward{cdocsch2}"|
% \end{tabular}
% \end{center}
% Note that the trailing backslash on each first line
% merely continues the input to the second line
% (for convenient cut ant paste).
% Furthermore, the command |latex| can be replaced by any
% of its alternative versions such as |pdflatex|.
%
% %%%%%%%%%%%%%%%%%%%%%%%%%%%%%%%%%%%%%%%%%%%%%%%%%%%%%%%%%%%%%%%%%%%%%%%%%%%%%%
% %%%%%%%%%%%%%%%%%%%%%%%%%%%%%%%%%%%%%%%%%%%%%%%%%%%%%%%%%%%%%%%%%%%%%%%%%%%%%%
% \section{Implementation}
%\iffalse
%<*package>
%\fi
%
% This section describes the definitions file |childdoc.def|.

% The definitions cannot be loaded using |\usepackage| or |\RequirePackage|
% which has a mechanism to prevent loading a style file more than once.
% When loading the definitions by means of |\input|
% multiple instances have to be prevented manually:
%\iffalse
%This code needs to be before the `\ProvidesFile' directive
%which is defined at the beginning of this file.
%Therefore it is also placed there and commented out here.
%</package>
%<*discard>
%\fi
%    \begin{macrocode}
\ifdefined\childdocmain\endinput\fi
%    \end{macrocode}
%\iffalse
%</discard>
%<*package>
%\fi
%
% \macro{\ifchilddoc}
% \macro{\ifchilddocmanual}
% The conditional |\ifchilddoc| tells whether a
% child (true) or main (false) document is being compiled.
% The conditional |\ifchilddocmanual| tells whether
% the |\includeonly| mechanism is used (false) or
% the selection of child files must be performed manually (true).
% The definitions initialise to false:
%    \begin{macrocode}
\newif\ifchilddoc
\newif\ifchilddocmanual
%    \end{macrocode}

% \macro{\childdocname}
% \macro{\childdocjob}
% The macro |\childdocname| stores the name of the main document
% to be compiled. The macro |\childdocjob| stores the name of
% the document on which the \LaTeX{} compiler was originally invoked.
% The content of |\jobname| cannot be compared
% to filenames specified in the source due to different catcodes.
% The following code rescans |\jobname|, stores the result
% in |\childdocname| and saves a copy in |\childdocjob|:
%    \begin{macrocode}
\edef\childdocname{\scantokens\expandafter{\jobname\noexpand}}
\let\childdocjob\childdocname
%    \end{macrocode}

% \macro{\childdocdisable}
% The macro |\childdocdisable| prevents the main file
% from being processed more than once.
% At this stage, the main document command |\childdocmain|
% is assumed to be called once again where it should do nothing.
% Any subsequent call to it should prevent
% a secondary processing of the main document
% It overwrites the forwarding commands
% |\childdocof| and |\childdocforward|
% with empty macros to prevent further inclusions of the main document:
%    \begin{macrocode}
\newcommand{\childdocdisable}
{
  \renewcommand{\childdocmain}[1]{\renewcommand{\childdocmain}[1]{\endinput}}
  \renewcommand{\childdocof}[1]{}
  \renewcommand{\childdocby}[2][]{}
  \renewcommand{\childdocforward}[2][]{}
  \renewcommand{\childdocdisable}{}
}
%    \end{macrocode}

% \macro{\childdocmain}
% The macro |\childdocmain| is to be called at the top of the main file
% with nothing or the main filename (without extension) as argument.
% First, it breaks loops.
% If the argument is not empty and does not match |\childdocname|
% (which is set by the first inclusion of |childdoc.def|),
% |\ifchilddoc| is set to true, |\includeonly| is applied to the child file
% and |\jobname| is set to the main file
% (for proper handling of |.aux| files):
%    \begin{macrocode}
\newcommand{\childdocmain}[1]
{
  \childdocdisable\childdocmain{}
  \if?#1?\else
    \begingroup
      \def\childdoctmp{#1}
      \ifx\childdoctmp\childdocname
        \def\childdoctmp{}
      \else
        \def\childdoctmp
        {
          \childdoctrue
          \includeonly{\childdocname}
          \def\childdocjob{#1}
          \def\jobname{#1}
        }
      \fi
      \expandafter
    \endgroup
    \childdoctmp
  \fi
}
%    \end{macrocode}

% \macro{\childdocof}
% The command |\childdocof| redirects
% compilation to the main file |#1|.
%    \begin{macrocode}
\newcommand{\childdocof}[1]
{
  \childdocdisable
  \childdoctrue
  \includeonly{\childdocname}
  \def\jobname{#1}
  \def\childdocjob{#1}
  \input{#1}
}
%    \end{macrocode}

% \macro{\childdocby}
% The command |\childdocby| ....
%    \begin{macrocode}
\newcommand{\childdocby}[2][]
{
  \childdocdisable
  \childdoctrue
  \childdocmanualtrue
  \if?#1?\else
    \def\jobname{#2}
  \fi
  \def\childdocjob{#2}
  \input{#2}
  \endinput
}
%    \end{macrocode}

% \macro{\childdocforward}
% The command |\childdocforward| redirects
% compilation to the main file or
% (if the optional argument is given) a child file.
% Parameters are set as if the main file
% or a child file starting with |\childdocof| was compiled.
% Then compilation is handed over to the main file:
%    \begin{macrocode}
\newcommand{\childdocforward}[2][]
{
  \begingroup
    \if?#1?
      \def\childdoctmp
      {
        \def\childdocname{#2}
        \def\childdocjob{#2}
        \def\jobname{#2}
        \input{#2}
        \endinput
      }
    \else
      \def\childdoctmp
      {
        \childdocdisable
        \def\childdocname{#2}
        \childdoctrue
        \includeonly{#2}
        \def\childdocjob{#1}
        \def\jobname{#1}
        \input{#1}
        \endinput
      }
    \fi
    \expandafter
  \endgroup
  \childdoctmp
}
%    \end{macrocode}

% \macro{\childdocforwardprefix}
% The command |\childdocforwardprefix| redirects
% compilation to the main or a child file by means of a pattern.
% The prefix |#1| in the current filename is replaced by |#2|
% and the suffix of the current filename is kept
% (it is assumed that the filename does not contain the substring `|~~~|'
% which is used as a delimiter).
% Compilation is handed over to the new file by |\childdocforward|:
%    \begin{macrocode}
\newcommand{\childdocforwardprefix}[3][]
{
  \begingroup
    \def\childdocextract #2##1~~~{\def\childdoctmp{\childdocforward[#1]{#3##1}}}
    \expandafter\childdocextract\childdocname~~~
    \expandafter
  \endgroup
  \childdoctmp
}
%    \end{macrocode}

% \macro{\childdoc}
% The deprecated macro |\childdoc| is a legacy version of |\childdocmain|:
%    \begin{macrocode}
\newcommand{\childdoc}{\childdocmain}
%    \end{macrocode}

% \macro{\childdocredirect}
% The deprecated macro |\childdocredirect| is a legacy version
% of |\childdocforward| and |\childdocforwardprefix|:
%    \begin{macrocode}
\newcommand{\childdocredirect}[2][]
{
  \begingroup
    \if?#1?
      \def\childdoctmp{\childdocforward{#2}}
    \else
      \def\childdoctmp{\childdocforwardprefix{#1}{#2}}
    \fi
    \expandafter
  \endgroup
  \childdoctmp
}
%    \end{macrocode}

%\iffalse
%</package>
%\fi
%
\endinput
\childdocforward{cdocsamp}"|\\
% |latex -jobname cdocscl1 \|\\
% |  "% \iffalse
%
% childdoc.dtx Copyright (C) 2017-2018 Niklas Beisert
%
% This work may be distributed and/or modified under the
% conditions of the LaTeX Project Public License, either version 1.3
% of this license or (at your option) any later version.
% The latest version of this license is in
%   http://www.latex-project.org/lppl.txt
% and version 1.3 or later is part of all distributions of LaTeX
% version 2005/12/01 or later.
%
% This work has the LPPL maintenance status `maintained'.
%
% The Current Maintainer of this work is Niklas Beisert.
%
% This work consists of the files childdoc.dtx and childdoc.ins
% and the derived files childdoc.def and cdocsamp.tex with
% cdocsch1.tex, cdocsch2.tex, cdocsdrf.tex, cdocsfn1.tex, cdocsfn2.tex.
%
%<package>\ifdefined\childdocmain\endinput\fi
%<package>\ProvidesFile{childdoc.def}[2018/12/30 v2.0 child document driver]
%<samplemain>\ProvidesFile{cdocsamp.tex}[2018/12/30 v2.0 sample for childdoc]
%<*driver>
%\ProvidesFile{childdoc.drv}[2018/12/30 v2.0 childdoc reference manual file]
\PassOptionsToClass{10pt,a4paper}{article}
\documentclass{ltxdoc}

\usepackage[margin=35mm]{geometry}
\usepackage{hyperref}
\usepackage{hyperxmp}
\usepackage[usenames]{color}

\hypersetup{colorlinks=true}
\hypersetup{pdfstartview=FitH}
\hypersetup{pdfpagemode=UseNone}
\hypersetup{pdfsource={}}
\hypersetup{pdflang={en-UK}}
\hypersetup{pdfcopyright={Copyright 2017-2018 Niklas Beisert.
  This work may be distributed and/or modified under the
  conditions of the LaTeX Project Public License, either version 1.3
  of this license or (at your option) any later version.}}
\hypersetup{pdflicenseurl={http://www.latex-project.org/lppl.txt}}
\hypersetup{pdfcontactaddress={ETH Zurich, ITP, HIT K,
  Wolfgang-Pauli-Strasse 27}}
\hypersetup{pdfcontactpostcode={8093}}
\hypersetup{pdfcontactcity={Zurich}}
\hypersetup{pdfcontactcountry={Switzerland}}
\hypersetup{pdfcontactemail={nbeisert@itp.phys.ethz.ch}}
\hypersetup{pdfcontacturl={http://people.phys.ethz.ch/\xmptilde nbeisert/}}

\newcommand{\secref}[1]{\hyperref[#1]{section \ref*{#1}}}

\parskip1ex
\parindent0pt
\let\olditemize\itemize
\def\itemize{\olditemize\parskip0pt}

\begin{document}

\title{The \textsf{childdoc} Package}
\hypersetup{pdftitle={The childdoc Package}}
\author{Niklas Beisert\\[2ex]
  Institut f\"ur Theoretische Physik\\
  Eidgen\"ossische Technische Hochschule Z\"urich\\
  Wolfgang-Pauli-Strasse 27, 8093 Z\"urich, Switzerland\\[1ex]
  \href{mailto:nbeisert@itp.phys.ethz.ch}
  {\texttt{nbeisert@itp.phys.ethz.ch}}}
\hypersetup{pdfauthor={Niklas Beisert}}
\hypersetup{pdfsubject={Manual for the LaTeX2e Package childdoc}}
\date{30 December 2018, \textsf{v2.0}}
\maketitle

\begin{abstract}\noindent
\textsf{childdoc} is a \LaTeXe{} package
that enables the direct compilation
of document sections included by |\include|
to individual files.
\end{abstract}

\begingroup
\parskip0ex
\tableofcontents
\endgroup

%%%%%%%%%%%%%%%%%%%%%%%%%%%%%%%%%%%%%%%%%%%%%%%%%%%%%%%%%%%%%%%%%%%%%%%%%%%%%%%%
%%%%%%%%%%%%%%%%%%%%%%%%%%%%%%%%%%%%%%%%%%%%%%%%%%%%%%%%%%%%%%%%%%%%%%%%%%%%%%%%
\section{Introduction}

\LaTeX{} provides a mechanism to structure a large document (such as a book)
into a main file and several child files (containing the chapters)
using the |\include| command.
This mechanism is beneficial for documents
which span hundreds of pages in order to
make the source file(s) more manageable.
Moreover, compilation can be restricted to
selected child files by means of the |\includeonly| command.
The latter feature can be used to reduce the compilation time while editing
(this was significantly more useful in the earlier days of \LaTeX{})
or to generate a smaller document which is easier to navigate.
Another application of |\includeonly| is to generate
documents consisting of selected parts of the complete document.

However, there are a few drawbacks of the plain |\include| mechanism:
\begin{itemize}
\item
The child files cannot be compiled on their own,
they can only be compiled via the main file.
A naive editing environment
(such as a text editor with an option
to have the current file processed by \LaTeX)
may require one to switch to the main file before compiling;
attempting to compile the child file produces errors.
\item
The main file must be modified (each time)
to adjust the |\includeonly| command
to the present needs. This easily leaves the main file in a messy state.
\item
The generated document will always carry the filename
of the main document. This is inconvenient if
several child files are to be compiled and
to be kept for distribution.
\end{itemize}

The present package provides a simple interface
to make child files individually compilable by \LaTeX{}.
Compiling a child file then has the same effect as compiling
the main file with an |\includeonly| command
to select the appropriate child.
Moreover the generated document will carry the name of the child
rather than the main file.
This resolves all three above issues.

This feature is meant to make the editing of books,
thesis documents and lecture notes somewhat more convenient.
However, the package can also be used efficiently for
composing a series of documents (such as exercise sheets)
which are typically distributed individually.
It then assists the author in generating the individual documents
(potentially in different versions)
as well as a document containing the collected series.
Another application is in developing style files
or other kinds of included material
where compilation of the style file could redirect
to a sample or test file.

%%%%%%%%%%%%%%%%%%%%%%%%%%%%%%%%%%%%%%%%%%%%%%%%%%%%%%%%%%%%%%%%%%%%%%%%%%%%%%%%
%%%%%%%%%%%%%%%%%%%%%%%%%%%%%%%%%%%%%%%%%%%%%%%%%%%%%%%%%%%%%%%%%%%%%%%%%%%%%%%%
\section{Usage}

First of all, the package \textsf{childdoc} is \emph{not} a standard
\LaTeXe{} |.sty| style file! Therefore it needs to be invoked in
a non-standard way.

%%%%%%%%%%%%%%%%%%%%%%%%%%%%%%%%%%%%%%%%%%%%%%%%%%%%%%%%%%%%%%%%%%%%%%%%%%%%%%%%
\subsection{Included Files}
\label{sec:include}

%%%%%%%%%%%%%%%%%%%%%%%%%%%%%%%%%%%%%%%%
\DescribeMacro{\childdocmain}
To use the package, add the commands
\begin{center}
\begin{tabular}{l}
|\input{childdoc.def}|\\
|\childdocmain{}|\\
\end{tabular}
\end{center}
at the very top of the main \LaTeX{} file,
in particular \emph{before} the |\documentclass| statement!
The argument of |\childdocmain| should be left empty
(but it must be present).

%%%%%%%%%%%%%%%%%%%%%%%%%%%%%%%%%%%%%%%%
\DescribeMacro{\childdocof}
Furthermore, add the commands
\begin{center}
\begin{tabular}{l}
|\input{childdoc.def}|\\
|\childdocof{|\textit{main}|}|\\
\end{tabular}
\end{center}
at the top of every child file \textit{child}
which is included by |\include{|\textit{child}|}|
from within the main file
(or at least for those files to be compiled individually).
The argument \textit{main} must be the filename of the main file.

There are a couple of
considerations in setting up the main and child documents:

%%%%%%%%%%%%%%%%%%%%%%%%%%%%%%%%%%%%%%%%
\paragraph{Restrictions.}

Please note the following restrictions:
\begin{itemize}
\item
|\childdocmain| must be called with one argument \textit{main}
to ensure compatibility with earlier version of the package.
It must either be empty (|\childdocmain{}|)
or precisely match the filename of the main file in which it is specified.
See \secref{sec:detection} for further information.
\item
The filename \textit{main} must be specified without the |.tex| extension.
\item
The filename \textit{main} is case sensitive
(even in case-insensitive file systems)
due to internal string comparison.
\item
The argument \textit{main} should be fully expanded, it cannot be a macro.
\item
Subdirectories and special characters should be avoided in filenames.
\item
The command |\childdocmain{|\textit{main}|}| must be followed by a whitespace.
It should not be followed immediately by another command
or by a comment mark `|%|'.
This is because the \TeX{} parser reads the token immediately following
the argument of |\childdocmain| and puts it
at the beginning of every child section;
however, a white\-space is ignored.
\end{itemize}

%%%%%%%%%%%%%%%%%%%%%%%%%%%%%%%%%%%%%%%%
\paragraph{Content of Main File.}

It is advisable to place all content in the child files included by |\include|.
Any output contained in the main file will appear in all child documents
unless suppressed manually;
it cannot be suppressed automatically by the |\includeonly| directive
and thus should normally be avoided.
A method to include some content in the main file
by means of conditional processing is described in \secref{sec:conditional}.

%%%%%%%%%%%%%%%%%%%%%%%%%%%%%%%%%%%%%%%%
\paragraph{Page Numbering.}

When only a part of the document is compiled,
the appropriate numbering of pages
(as well as other status parameters)
is determined from the |.aux| files.
The latter contain information from previous passes.
However this information needs to propagate through
all intermediate child documents.
Therefore the page numbering in child documents may well
be inconsistent until the complete document is compiled at least once.

A useful (if unconventional) way to always ensure a consistent
page numbering is to restart the numbering in each child document
and denote the pages by `\textit{child}|.|\textit{page}'
where \textit{child} represents the chapter/section number of the child file.
This can be achieved by the command
|\numberwithin{page}{|\textit{child}|}|
of the \textsf{amsmath} package
where \textit{child} can be |chapter| or |section|
depending on the chosen structuring.
Alternatively, one can modify the macro |\thepage| appropriately
and reset the counter |page| at the start of each child file.

%%%%%%%%%%%%%%%%%%%%%%%%%%%%%%%%%%%%%%%%%%%%%%%%%%%%%%%%%%%%%%%%%%%%%%%%%%%%%%%%
\subsection{Conditional Processing}
\label{sec:conditional}

The package provides a mechanism to compile different versions
of a document. To customise the versions further some conditional processing
can come in handy to distinguish which version is being compiled.
The package provides two macros to describe the compilation context:

%%%%%%%%%%%%%%%%%%%%%%%%%%%%%%%%%%%%%%%%
\DescribeMacro{\ifchilddoc}
The conditional |\ifchilddoc| distinguishes between the compilation of
child documents and the main document:
%
\begin{center}
|\ifchilddoc |\textit{child-code}| |[|\||else |\textit{main-code}]| \||fi|
\end{center}

%%%%%%%%%%%%%%%%%%%%%%%%%%%%%%%%%%%%%%%%
\DescribeMacro{\childdocname}
\DescribeMacro{\childdocjob}
The macro |\childdocname| contains the filename (without extension)
of the main or child file being processed.
Note that |\childdocjob| will always contain the name of the main file.

%%%%%%%%%%%%%%%%%%%%%%%%%%%%%%%%%%%%%%%%
\paragraph{Title Page.}

Conditional processing can be used to include a title or banner page
in the main document when proper precautions are taken.
Importantly, the code in the main file should ensure that the page counter
(as well as other status parameters which are stored in the |.aux| files)
takes the same value after the conditional processing.
Otherwise the page numbers may take divergent values
depending on which part is compiled.

For example, a title page could be declared by:
%
\begin{center}
\begin{tabular}{l}
|\ifchilddoc\||else|\\
|\addtocounter{page}{-1}|\\
\textit{code for title page}\\
|\newpage|\\
|\||fi|
\end{tabular}
\end{center}
%
A banner page for the child documents can be generated by:
%
\begin{center}
\begin{tabular}{l}
|\ifchilddoc|\\
|\addtocounter{page}{-1}|\\
\textit{code for banner page}\\
|\newpage|\\
|\||fi|
\end{tabular}
\end{center}
%
Here one could write a message such as:
\begin{center}
|This is the part \childdocname{} of \childdocjob{}.|
\end{center}

%%%%%%%%%%%%%%%%%%%%%%%%%%%%%%%%%%%%%%%%%%%%%%%%%%%%%%%%%%%%%%%%%%%%%%%%%%%%%%%%
\subsection{Flags}
\label{sec:flags}

The package makes it easy to generate different versions
of the main or child documents.
To this end compilation flags can be defined
and assigned different default values.
They will be particularly useful in conjunction
with the forwarding mechanism described in \secref{sec:forward}.

For example, it may be useful to have a flag |\version|
which can be set to |draft| or |final|.
The document source will contain some conditional code
depending on the value of |\version|.
Suppose further, the flag should default to |final| for the main file
and to |draft| for child files
which is a natural assignment for editing the document.
This is achieved by placing the following code
in the preamble of the main document
(below the |\childdocmain| directive):
%
\begin{center}
\begin{tabular}{l}
|\ifchilddoc|\\
|\providecommand{\version}{draft}|\\
|\||else|\\
|\providecommand{\version}{final}|\\
|\||fi|
\end{tabular}
\end{center}
%
The definition by |\providecommand| makes sure
that previous definitions are not overwritten.
Further statements |\providecommand{\version}{...}|
can thus be added before the above code to override it.

For the main file, one might add a line
(between |\childdocmain| and the above block)
%
\begin{center}
|%\ifchilddoc\||else\providecommand{\version}{draft}\||fi|
\end{center}
%
which can be uncommented to produce a draft version.
Likewise one can add a line to the very top of a child file
(above the |\childdocof{|\textit{main}|}| directive)
%
\begin{center}
|%\providecommand{\version}{final}|
\end{center}
%
which can be uncommented to produce the final version of this child document.

%%%%%%%%%%%%%%%%%%%%%%%%%%%%%%%%%%%%%%%%%%%%%%%%%%%%%%%%%%%%%%%%%%%%%%%%%%%%%%%%
\subsection{Forwarding}
\label{sec:forward}

Different versions of the main or child documents
using compilation flags as described in \secref{sec:flags}
can be (permanently) stored in different files
for convenient compilation, viewing and distribution.
To this end, the package defines a command
to pass on compilation to a different file:

%%%%%%%%%%%%%%%%%%%%%%%%%%%%%%%%%%%%%%%%
\DescribeMacro{\childdocforward}
The command |\childdocforward| redirects processing to
another source file:
%
\begin{center}
\begin{tabular}{l}
|\input{childdoc.def}|\\
|\childdocforward[|\textit{main}|]{|\textit{dest}|}|\\
\end{tabular}
\end{center}
%
The argument \textit{dest} is the destination file
(without extension).
It should be the main file or one of the child files.
Note that further \textsf{childdoc} directives
such as |\childdocof| and |\childdocforward|
in the indicated file will be processed in this form.
The optional argument \textit{main}
passes on directly to the main file \textit{main}
while pretending to compile the child \textit{dest}.
This form behaves as if \textit{dest}
issues |\childdocof{|\textit{main}|}| right away,
and no further \textsf{childdoc} directives will be processed.

%%%%%%%%%%%%%%%%%%%%%%%%%%%%%%%%%%%%%%%%
\DescribeMacro{\...prefix}
In the alternative form |\childdocforwardprefix|,
%
\begin{center}
\begin{tabular}{l}
|\input{childdoc.def}|\\
|\childdocforwardprefix[|\textit{main}|]{|\textit{prefix}|}{|\textit{dest}|}|
\end{tabular}
\end{center}
%
the destination file is determined by a pattern
depending on the current file:
To make this work, the current file must be called
`{\textit{prefix}\hspace{0.2em}\textit{suffix}}'
with \textit{prefix} matching precisely the argument.
Processing is then passed on to the file
`{\textit{dest}\hspace{0.2em}\textit{suffix}}'.
Surely, the same effect is achieved by
directly specifying the
argument `{\textit{dest}\hspace{0.2em}\textit{suffix}}'
in the first form.
However, that requires to set up a different file
for each child. With the alternative form of the command
all these files can have exactly the same content
which simplifies setting them up and maintaining them.

For example, the following file |draft.tex|
with a compilation flag |\version| as described in \secref{sec:flags}
compiles the main document as a draft:
%
\begin{center}
\begin{tabular}{l}
|\def\version{draft}|\\
|\input{childdoc.def}|\\
|\childdocforward{|\textit{main}|}|
\end{tabular}
\end{center}
%
Likewise, the following files |final|\textit{nn}|.tex|
compile the final version of the child document
|child|\textit{nn}|.tex|:
%
\begin{center}
\begin{tabular}{l}
|\def\version{final}|\\
|\input{childdoc.def}|\\
|\childdocforwardprefix{final}{child}|
\end{tabular}
\end{center}
%

Note that when several versions of a main file and/or of each child file
are to be generated, it may be convenient to set up a |Makefile| or
shell script to automatise the process.

%%%%%%%%%%%%%%%%%%%%%%%%%%%%%%%%%%%%%%%%%%%%%%%%%%%%%%%%%%%%%%%%%%%%%%%%%%%%%%%%
\subsection{Command Line Processing}
\label{sec:commandline}

The effect of redirection files can also be achieved by invoking
the \LaTeX{} compiler with a more elaborate command line.
Most conveniently this should be done as part
of a shell script or a |Makefile|.

When using \textsf{childdoc} in the main file, the following
command lines effectively perform a redirection
(note that depending on the shell being used,
backslashes may have to be doubled: `|\|' $\to$ `|\\|'):
%
\begin{center}
|... -jobname "|\textit{target}|" |\\|"|[\textit{flags}]%
|\input{childdoc.def}\childdocforward[|\textit{main}|]{|\textit{dest}|}"|
\end{center}
%
Here \textit{target} is the name of the output file,
\textit{main} is the name of the main file
and \textit{dest} is the name of the main or child file to be processed
(all filenames without extensions).
The optional argument \textit{main} can be omitted
if \textit{main} matches \textit{dest}.
Optionally, compilation \textit{flags} can be defined via |\def| commands.
This command line makes the \TeX{} engine believe
it is compiling the file \textit{target}
whose content is specified as the latter parameter.
The provided code then forwards the processing to
\textit{main} or \textit{dest} as described in \secref{sec:forward}.

%%%%%%%%%%%%%%%%%%%%%%%%%%%%%%%%%%%%%%%%%%%%%%%%%%%%%%%%%%%%%%%%%%%%%%%%%%%%%%%%
\subsection{Include by Input}
\label{sec:input}

Including child documents by |\include| has some restrictions by design.
Most notably, the content of a child document always occupies
its own set of pages; pages cannot be shared between child documents.
Usually, this behaviour makes perfect sense
because each child document contain an essential part of the document.
However, in some situations it may be desirable to compose
a document from a collection of parts
without having mandatory page breaks between then.
For this case, the package
provides a mechanism to include parts
by |\input| which can also be processed individually.
However, by construction this mechanism
requires manual handling of the content to be output.

%%%%%%%%%%%%%%%%%%%%%%%%%%%%%%%%%%%%%%%%
\DescribeMacro{\ifchilddocmanual}
The main file should be prepared as usual, see \secref{sec:include}.
However, the document body must make a distinction
between processing of an individual part and of the main document, e.g.:
%
\begin{center}
\begin{tabular}{l}
|\ifchilddocmanual|\\
|\input{\childdocname}|\\
|\||else|\\
\textit{document body with }|\input{|\textit{part}|}|\\
|\||fi|
\end{tabular}
\end{center}
%
The conditional |\ifchilddocmanual| is true whenever
a part to be included by |\input| is being compiled,
and the name of the part is stored in |\childdocname|.

%%%%%%%%%%%%%%%%%%%%%%%%%%%%%%%%%%%%%%%%
\DescribeMacro{\childdocby}
Each part to be included by |\input| should start with:
%
\begin{center}
\begin{tabular}{l}
|\input{childdoc.def}|\\
|\childdocby{|\textit{main}|}|\\
\end{tabular}
\end{center}
%
The directive |\childdocby| is similar to |\childdocof|
described in \secref{sec:include},
but the subsequent selection of content must be done manually.
To that end, both |\ifchilddoc| and |\ifchilddocmanual|
will be true upon processing of a part,
and the name of the part is stored in |\childdocname|.
Note that |\jobname| will be set to the filename of the current part
so that each part receives an individual |.aux| file
that does not interfere with the |.aux| file(s) of the main document.
This behaviour can be altered by the alternative form
|\childdocby[*]{|\textit{main}|}| (with a non-empty optional argument)
which uses the |.aux| file of the main document
by setting |\jobname| to \textit{main}.

%%%%%%%%%%%%%%%%%%%%%%%%%%%%%%%%%%%%%%%%%%%%%%%%%%%%%%%%%%%%%%%%%%%%%%%%%%%%%%%%
\subsection{Driver Development}
\label{sec:driver}

The \textsf{childdoc} mechanism can also be use for the development
of definition files such as \LaTeX{} styles or classes.
This case differs from the above setup with multiple parts
included by |\include| in that no |\includeonly| should be invoked.
This can be achieved by starting the include file
(before |\ProvidesPackage|) with:
%
\begin{center}
\begin{tabular}{l}
|\input{childdoc.def}|\\
|\childdocforward{|\textit{main}|}|\\
\end{tabular}
\end{center}
%
or alternatively with:
%
\begin{center}
\begin{tabular}{l}
|\input{childdoc.def}|\\
|\childdocby{|\textit{main}|}|\\
\end{tabular}
\end{center}
%
Both forms have slightly different effects as described above.
The main file is prepared as usual, see \secref{sec:include}.

%%%%%%%%%%%%%%%%%%%%%%%%%%%%%%%%%%%%%%%%%%%%%%%%%%%%%%%%%%%%%%%%%%%%%%%%%%%%%%%%
\subsection{Legacy Detection}
\label{sec:detection}

The directive |\childdocmain| in the main file can detect
whether the complete document or merely a child is to be compiled
even without using the directive |\childdocof|.
This method is deprecated because it is less robust
and there is no compelling reason to use it;
it is merely provided for backward compatibility
and it may be removed in future versions.

If the detection mechanism is to be used,
it is mandatory to correctly specify
the filename of the main file as the argument of |\childdocmain|:
%
\begin{center}
\begin{tabular}{l}
|\input{childdoc.def}|\\
|\childdocmain{|\textit{main}|}|\\
\end{tabular}
\end{center}
%
If |\jobname| does not match the argument \textit{main} of |\childdocmain|,
it is assumed that |\jobname| points to the child file to be compiled.
When using |\childdocmain| with the main file specified as argument,
it suffices to start a child file
with just |\input{|\textit{main}|}|
without loading of the package and using |\childdocof|.
If instead all processing is done
with the appropriate \textsf{childdoc} directives,
the argument of \textit{main} of |\childdocmain| can be empty.

An alternative version of the command line processing described
in \secref{sec:commandline} using the detection mechanism reads:
%
\begin{center}
|... -jobname "|\textit{target}|" "|[\textit{flags}]%
[|\def\jobname{|\textit{dest}|}|]|\input{|\textit{main}|}"|
\end{center}

%%%%%%%%%%%%%%%%%%%%%%%%%%%%%%%%%%%%%%%%%%%%%%%%%%%%%%%%%%%%%%%%%%%%%%%%%%%%%%%%
\subsection{Manual Code}
\label{sec:manual}

In case one cannot be certain whether the definitions file |childdoc.def|
is installed on the target \TeX{} distribution
and one prefers not to ship it,
it is conceivable to paste a few relevant commands into the sources.

To that end, drop all statements |\input{childdoc.def}|
and perform the replacements as outlined below.
Instead of |\childdocmain{|\textit{main}|}| add the following code
to the top of the main file:
%
\begin{center}
\begin{tabular}{l}
|\||ifdefined\childdocname\endinput\||fi\newif\ifchilddoc|\\
|\edef\childdocname{\scantokens\expandafter{\jobname\noexpand}}|\\
|\def\childdocmain{|\textit{main}|}\||ifx\childdocmain\childdocname\||else|\\
|\childdoctrue\includeonly{\childdocname}\let\jobname\childdocmain\||fi|\\
\end{tabular}
\end{center}
%
Instead of |\childdocof{|\textit{main}|}| just include the main file
at the top of each child file:
%
\begin{center}
|\input{|\textit{main}|}|
\end{center}
%
A simple redirection |\childdocforward{|\textit{dest}|}| is achieved by:
%
\begin{center}
|\def\jobname{|\textit{dest}|}\input{\jobname}|
\end{center}
%
The redirection with prefix
|\childdocforwardprefix[|\textit{prefix}|]{|\textit{dest}|}|
is accomplished by:
%
\begin{center}
\begin{tabular}{l}
|{\edef\jobname{\scantokens\expandafter{\jobname\noexpand}}|\\
|\def\redirectjob |\textit{prefix}|#1~~~{\gdef\jobname{|\textit{dest}|#1}}|\\
|\expandafter\redirectjob\jobname~~~}\input{\jobname}|
\end{tabular}
\end{center}

In an alternative approach,
child documents can be compiled by a specific command line
without additional code or specific definitions:
%
\begin{center}
|... -jobname "|\textit{target}|" "|[\textit{flags}]%
|\includeonly{|\textit{dest}|}\input{|\textit{main}|}"|
\end{center}
%

%%%%%%%%%%%%%%%%%%%%%%%%%%%%%%%%%%%%%%%%%%%%%%%%%%%%%%%%%%%%%%%%%%%%%%%%%%%%%%%%
%%%%%%%%%%%%%%%%%%%%%%%%%%%%%%%%%%%%%%%%%%%%%%%%%%%%%%%%%%%%%%%%%%%%%%%%%%%%%%%%
\section{Information}

%%%%%%%%%%%%%%%%%%%%%%%%%%%%%%%%%%%%%%%%%%%%%%%%%%%%%%%%%%%%%%%%%%%%%%%%%%%%%%%%
\subsection{Copyright}

Copyright \copyright{} 2017--2018 Niklas Beisert

This work may be distributed and/or modified under the
conditions of the \LaTeX{} Project Public License, either version 1.3
of this license or (at your option) any later version.
The latest version of this license is in
  \url{http://www.latex-project.org/lppl.txt}
and version 1.3 or later is part of all distributions of \LaTeX{}
version 2005/12/01 or later.

This work has the LPPL maintenance status `maintained'.

The Current Maintainer of this work is Niklas Beisert.

This work consists of the files |README.txt|, |childdoc.ins| and |childdoc.dtx|
as well as the derived files |childdoc.def|, |cdocsamp.tex|
with |cdocsch1.tex|, |cdocsch2.tex|, |cdocspt3.tex|, |cdocspt4.tex|,
|cdocsdrf.tex|, |cdocsfn1.tex|, |cdocsfn2.tex|
as well as |childdoc.pdf|.

%%%%%%%%%%%%%%%%%%%%%%%%%%%%%%%%%%%%%%%%%%%%%%%%%%%%%%%%%%%%%%%%%%%%%%%%%%%%%%%%
\subsection{Files and Installation}

The package consists of the files:
%
\begin{center}
\begin{tabular}{ll}
    |README.txt|   & readme file \\
    |childdoc.ins| & installation file \\
    |childdoc.dtx| & source file \\
    |childdoc.def| & definition file \\
    |cdocsamp.tex| & sample main file \\
    |cdocsch1.tex| & sample include file \\
    |cdocsch2.tex| & sample include file \\
    |cdocspt3.tex| & sample part file \\
    |cdocspt4.tex| & sample part file \\
    |cdocsdrf.tex| & sample redirection file \\
    |cdocsfn1.tex| & sample redirection file \\
    |cdocsfn2.tex| & sample redirection file \\
    |childdoc.pdf| & manual
\end{tabular}
\end{center}
%
The distribution consists of the files
|README.txt|, |childdoc.ins| and |childdoc.dtx|.
%
\begin{itemize}
\item
Run (pdf)\LaTeX{} on |childdoc.dtx|
to compile the manual |childdoc.pdf| (this file).
\item
Run \LaTeX{} on |childdoc.ins| to create the definitions file |childdoc.def|
and the sample |cdocsamp.tex| with include files
|cdocsch1.tex|, |cdocsch2.tex|, |cdocspt3.tex|, |cdocspt4.tex|,
|cdocsdrf.tex|, |cdocsfn1.tex|, |cdocsfn2.tex|.
Then copy the file |childdoc.def| to an appropriate directory of your \LaTeX{}
distribution, e.g.\ \textit{texmf-root}|/tex/latex/childdoc|.
\end{itemize}

%%%%%%%%%%%%%%%%%%%%%%%%%%%%%%%%%%%%%%%%%%%%%%%%%%%%%%%%%%%%%%%%%%%%%%%%%%%%%%%%
\subsection{Related CTAN Packages}

There are several other packages which offer a similar functionality:
%
\begin{itemize}
\item
The packages
\href{http://ctan.org/pkg/docmute}{\textsf{docmute}},
\href{http://ctan.org/pkg/includex}{\textsf{includex}} and
\href{http://ctan.org/pkg/standalone}{\textsf{standalone}}
provide commands to include only the document body of
a child file thus allowing both files to be compiled individually.
\item
The packages \href{http://ctan.org/pkg/subdocs}{\textsf{subdocs}}
and \href{http://ctan.org/pkg/subfiles}{\textsf{subfiles}}
provide structures in which the main and child documents can be
encapsulated and allowing them to be compiled individually.
The inclusion mechanism is different from the conventional |\include|.
\item
The package \href{http://ctan.org/pkg/combine}{\textsf{combine}}
is an elaborate solution to combine several documents into one.
\end{itemize}
%
See also the CTAN topic \href{http://ctan.org/topic/subdocs}{\textsf{subdocs}}
for further related packages.
The present package differs from the above solutions in that
a document structure constructed with the conventional |\include| mechanism
just needs two extra commands at the top of every file
such that all constituent files can be compiled individually.

%%%%%%%%%%%%%%%%%%%%%%%%%%%%%%%%%%%%%%%%%%%%%%%%%%%%%%%%%%%%%%%%%%%%%%%%%%%%%%%%
%\subsection{Feature Suggestions}
%
%The following is a list of features which may be useful for future
%versions of this package:
%%
%\begin{itemize}
%\item
%\ldots
%\end{itemize}

%%%%%%%%%%%%%%%%%%%%%%%%%%%%%%%%%%%%%%%%%%%%%%%%%%%%%%%%%%%%%%%%%%%%%%%%%%%%%%%%
\subsection{Revision History}

%%%%%%%%%%%%%%%%%%%%%%%%%%%%%%%%%%%%%%%%
\paragraph{v2.0:} 2018/12/30

\begin{itemize}
\item
immediate forward processing
\item
added |\childdocby| mechanism
\item
manual restructured
\end{itemize}

%%%%%%%%%%%%%%%%%%%%%%%%%%%%%%%%%%%%%%%%
\paragraph{v1.6:} 2018/01/17

\begin{itemize}
\item
application for development of include files
\item
corrections to manual
\end{itemize}

%%%%%%%%%%%%%%%%%%%%%%%%%%%%%%%%%%%%%%%%
\paragraph{v1.5:} 2017/05/21

\begin{itemize}
\item
more complete structuring introduced
\item
|\childdocof| introduced
\item
|\childdoc| renamed to |\childdocmain|
\item
|\childredirect| renamed to |\childdocforward| and |\childdocforwardprefix|
and functionality expanded
\end{itemize}

%%%%%%%%%%%%%%%%%%%%%%%%%%%%%%%%%%%%%%%%
\paragraph{v1.0:} 2017/04/27

\begin{itemize}
\item
manual and install package
\item
first version published on CTAN
\end{itemize}

%%%%%%%%%%%%%%%%%%%%%%%%%%%%%%%%%%%%%%%%
\paragraph{v0.6:} 2017/04/26

\begin{itemize}
\item
redirection mechanism added
\end{itemize}

%%%%%%%%%%%%%%%%%%%%%%%%%%%%%%%%%%%%%%%%
\paragraph{v0.5:} 2017/04/26

\begin{itemize}
\item
functionality in definition file
\end{itemize}


%%%%%%%%%%%%%%%%%%%%%%%%%%%%%%%%%%%%%%%%%%%%%%%%%%%%%%%%%%%%%%%%%%%%%%%%%%%%%%%%
%%%%%%%%%%%%%%%%%%%%%%%%%%%%%%%%%%%%%%%%%%%%%%%%%%%%%%%%%%%%%%%%%%%%%%%%%%%%%%%%
%%%%%%%%%%%%%%%%%%%%%%%%%%%%%%%%%%%%%%%%%%%%%%%%%%%%%%%%%%%%%%%%%%%%%%%%%%%%%%%%
\appendix

\settowidth\MacroIndent{\rmfamily\scriptsize 000\ }

 \DocInput{childdoc.dtx}

\end{document}
%</driver>
% \fi
%
% %%%%%%%%%%%%%%%%%%%%%%%%%%%%%%%%%%%%%%%%%%%%%%%%%%%%%%%%%%%%%%%%%%%%%%%%%%%%%%
% %%%%%%%%%%%%%%%%%%%%%%%%%%%%%%%%%%%%%%%%%%%%%%%%%%%%%%%%%%%%%%%%%%%%%%%%%%%%%%
% \section{Sample}
%\iffalse
%<*samplemain>
%\fi
%
% The following presents a sample document
% with two chapters, two parts, a title page,
% a compile flag as well as three forwarding files to set the flag.
% It consists of eight |.tex| files:
% \begin{center}
% \begin{tabular}{ll}
% |cdocsamp.tex|&main file\\
% |cdocsch1.tex|&include file for chapter 1\\
% |cdocsch2.tex|&include file for chapter 2\\
% |cdocspt3.tex|&include file for part 3\\
% |cdocspt4.tex|&include file for part 4\\
% |cdocsdrf.tex|&forwarding file for main file in draft mode\\
% |cdocsfi1.tex|&forwarding file for final version of chapter 1\\
% |cdocsfi2.tex|&forwarding file for final version of chapter 2\\
% \end{tabular}
% \end{center}
% Each of the eight files can be compiled directly by the \LaTeX{} compiler.
%
% %%%%%%%%%%%%%%%%%%%%%%%%%%%%%%%%%%%%%%
% \paragraph{Main File.}
%
% The main file is called |cdocsamp.tex|.
%
% Load the \textsf{childdoc} definitions and
% declare the filename for the main document:
%    \begin{macrocode}
\input{childdoc.def}
\childdocmain{}
%    \end{macrocode}

% Optional override for |\version| flag:
%    \begin{macrocode}
%%\ifchilddoc\else\providecommand{\version}{draft}\fi
%    \end{macrocode}

% Define the default values for the |\version| flag
% (|final| for the main file and |draft| for childs):
%    \begin{macrocode}
\ifchilddoc
\providecommand{\version}{draft}
\else
\providecommand{\version}{final}
\fi
%    \end{macrocode}

% Load the standard document class:
%    \begin{macrocode}
\documentclass[12pt]{article}
%    \end{macrocode}

% Start the document body:
%    \begin{macrocode}
\begin{document}
%    \end{macrocode}

% Declare a title page.
% Print title, part of document being processed and version flag:
%    \begin{macrocode}
\addtocounter{page}{-1}
\begin{center}
{\LARGE\bfseries{}childdoc example\par}
\vspace{1cm}
\ifchilddoc
\ifchilddocmanual part\else chapter\fi:
`\childdocname' of `\childdocjob'\par
\else
main document: `\childdocjob'\par
\fi
version: \version\par
\end{center}
\newpage
%    \end{macrocode}

% Manually include selected file,
% otherwise process as usual:
%    \begin{macrocode}
\ifchilddocmanual
\section*{part `\childdocname'}
\input{\childdocname}
\else
%    \end{macrocode}

% Include the two chapters:
%    \begin{macrocode}
\include{cdocsch1}
\include{cdocsch2}
%    \end{macrocode}

% Include the two parts unless only chapters should be displayed:
%    \begin{macrocode}
\ifchilddoc\else
\section{part three}
\input{cdocspt3}
\section{part four}
\input{cdocspt4}
\fi
%    \end{macrocode}

% Process as usual until here:
%    \begin{macrocode}
\fi
%    \end{macrocode}

% End of document body:
%    \begin{macrocode}
\end{document}
%    \end{macrocode}
%\iffalse
%</samplemain>
%\fi
%
% %%%%%%%%%%%%%%%%%%%%%%%%%%%%%%%%%%%%%%
% \paragraph{Chapter Include Files.}
%
% The include files are called |cdocsch1.tex| and |cdocsch2.tex|.
%
%\iffalse
%<*samplechap1|samplechap2>
%\fi

% Optional override for |\version| flag:
%    \begin{macrocode}
%%\providecommand{\version}{final}
%    \end{macrocode}

% Include the main document:
%    \begin{macrocode}
\input{childdoc.def}
\childdocof{cdocsamp}
%    \end{macrocode}

%\iffalse
%</samplechap1|samplechap2>
%\fi
%
%\iffalse
%<*samplechap1>
%\fi
% Some text for chapter 1:
%    \begin{macrocode}
\section{one}
some text in chapter one
%    \end{macrocode}

%\iffalse
%</samplechap1>
%\fi
% Some text for chapter 2:
%\iffalse
%<*samplechap2>
%\fi
%    \begin{macrocode}
\section{two}
more text in chapter two
%    \end{macrocode}

%\iffalse
%</samplechap2>
%\fi
%
% %%%%%%%%%%%%%%%%%%%%%%%%%%%%%%%%%%%%%%
% \paragraph{Part Include Files.}
%
% The include files are called |cdocspt3.tex| and |cdocspt4.tex|.
%
%\iffalse
%<*samplepart3|samplepart4>
%\fi

% Optional override for |\version| flag:
%    \begin{macrocode}
%%\providecommand{\version}{final}
%    \end{macrocode}

% Include the main document:
%    \begin{macrocode}
\input{childdoc.def}
\childdocby{cdocsamp}
%    \end{macrocode}

%\iffalse
%</samplepart3|samplepart4>
%\fi
%
%\iffalse
%<*samplepart3>
%\fi
% Some text for part 3:
%    \begin{macrocode}
some text in part three
%    \end{macrocode}

%\iffalse
%</samplepart3>
%\fi
% Some text for part 4:
%\iffalse
%<*samplepart4>
%\fi
%    \begin{macrocode}
more text in part four
%    \end{macrocode}

%\iffalse
%</samplepart4>
%\fi
%
% %%%%%%%%%%%%%%%%%%%%%%%%%%%%%%%%%%%%%%
% \paragraph{Forwarding for a Complete Draft.}
%
% The following forwarding file |cdocsdrf.tex|
% compiles the main document in draft mode:
%\iffalse
%<*sampledraft>
%\fi
%    \begin{macrocode}
\def\version{draft}
\input{childdoc.def}
\childdocforward{cdocsamp}
%    \end{macrocode}

%\iffalse
%</sampledraft>
%\fi
%
% %%%%%%%%%%%%%%%%%%%%%%%%%%%%%%%%%%%%%%
% \paragraph{Forwarding for Final Version of the Chapters.}
%
% The following forwarding files |cdocsfn1.tex| and |cdocsfn2.tex|
% (with identical content)
% compile the final versions of the child documents
% |cdocsch1.tex| and |cdocsch2.tex|, respectively:
%\iffalse
%<*samplefinal>
%\fi
%    \begin{macrocode}
\def\version{final}
\input{childdoc.def}
\childdocforwardprefix[cdocsamp]{cdocsfn}{cdocsch}
%    \end{macrocode}

%\iffalse
%</samplefinal>
%\fi
%
% %%%%%%%%%%%%%%%%%%%%%%%%%%%%%%%%%%%%%%
% \paragraph{Command Line Processing.}
%
% The following three command lines generate the output files
% |cdocscld|, |cdocscl1| and |cdocscl2|
% which should be identical to
% |cdocsdrf|, |cdocsch1| and |cdocsfn2|, respectively:
% \begin{center}
% \begin{tabular}{l}
% |latex -jobname cdocscld \|\\
% |  "\def\version{draft}\input{childdoc.def}\childdocforward{cdocsamp}"|\\
% |latex -jobname cdocscl1 \|\\
% |  "\input{childdoc.def}\childdocforward[cdocsamp]{cdocsch1}"|\\
% |latex -jobname cdocscl2 \|\\
% |  "\def\version{final}\input{childdoc.def}\childdocforward{cdocsch2}"|
% \end{tabular}
% \end{center}
% Note that the trailing backslash on each first line
% merely continues the input to the second line
% (for convenient cut ant paste).
% Furthermore, the command |latex| can be replaced by any
% of its alternative versions such as |pdflatex|.
%
% %%%%%%%%%%%%%%%%%%%%%%%%%%%%%%%%%%%%%%%%%%%%%%%%%%%%%%%%%%%%%%%%%%%%%%%%%%%%%%
% %%%%%%%%%%%%%%%%%%%%%%%%%%%%%%%%%%%%%%%%%%%%%%%%%%%%%%%%%%%%%%%%%%%%%%%%%%%%%%
% \section{Implementation}
%\iffalse
%<*package>
%\fi
%
% This section describes the definitions file |childdoc.def|.

% The definitions cannot be loaded using |\usepackage| or |\RequirePackage|
% which has a mechanism to prevent loading a style file more than once.
% When loading the definitions by means of |\input|
% multiple instances have to be prevented manually:
%\iffalse
%This code needs to be before the `\ProvidesFile' directive
%which is defined at the beginning of this file.
%Therefore it is also placed there and commented out here.
%</package>
%<*discard>
%\fi
%    \begin{macrocode}
\ifdefined\childdocmain\endinput\fi
%    \end{macrocode}
%\iffalse
%</discard>
%<*package>
%\fi
%
% \macro{\ifchilddoc}
% \macro{\ifchilddocmanual}
% The conditional |\ifchilddoc| tells whether a
% child (true) or main (false) document is being compiled.
% The conditional |\ifchilddocmanual| tells whether
% the |\includeonly| mechanism is used (false) or
% the selection of child files must be performed manually (true).
% The definitions initialise to false:
%    \begin{macrocode}
\newif\ifchilddoc
\newif\ifchilddocmanual
%    \end{macrocode}

% \macro{\childdocname}
% \macro{\childdocjob}
% The macro |\childdocname| stores the name of the main document
% to be compiled. The macro |\childdocjob| stores the name of
% the document on which the \LaTeX{} compiler was originally invoked.
% The content of |\jobname| cannot be compared
% to filenames specified in the source due to different catcodes.
% The following code rescans |\jobname|, stores the result
% in |\childdocname| and saves a copy in |\childdocjob|:
%    \begin{macrocode}
\edef\childdocname{\scantokens\expandafter{\jobname\noexpand}}
\let\childdocjob\childdocname
%    \end{macrocode}

% \macro{\childdocdisable}
% The macro |\childdocdisable| prevents the main file
% from being processed more than once.
% At this stage, the main document command |\childdocmain|
% is assumed to be called once again where it should do nothing.
% Any subsequent call to it should prevent
% a secondary processing of the main document
% It overwrites the forwarding commands
% |\childdocof| and |\childdocforward|
% with empty macros to prevent further inclusions of the main document:
%    \begin{macrocode}
\newcommand{\childdocdisable}
{
  \renewcommand{\childdocmain}[1]{\renewcommand{\childdocmain}[1]{\endinput}}
  \renewcommand{\childdocof}[1]{}
  \renewcommand{\childdocby}[2][]{}
  \renewcommand{\childdocforward}[2][]{}
  \renewcommand{\childdocdisable}{}
}
%    \end{macrocode}

% \macro{\childdocmain}
% The macro |\childdocmain| is to be called at the top of the main file
% with nothing or the main filename (without extension) as argument.
% First, it breaks loops.
% If the argument is not empty and does not match |\childdocname|
% (which is set by the first inclusion of |childdoc.def|),
% |\ifchilddoc| is set to true, |\includeonly| is applied to the child file
% and |\jobname| is set to the main file
% (for proper handling of |.aux| files):
%    \begin{macrocode}
\newcommand{\childdocmain}[1]
{
  \childdocdisable\childdocmain{}
  \if?#1?\else
    \begingroup
      \def\childdoctmp{#1}
      \ifx\childdoctmp\childdocname
        \def\childdoctmp{}
      \else
        \def\childdoctmp
        {
          \childdoctrue
          \includeonly{\childdocname}
          \def\childdocjob{#1}
          \def\jobname{#1}
        }
      \fi
      \expandafter
    \endgroup
    \childdoctmp
  \fi
}
%    \end{macrocode}

% \macro{\childdocof}
% The command |\childdocof| redirects
% compilation to the main file |#1|.
%    \begin{macrocode}
\newcommand{\childdocof}[1]
{
  \childdocdisable
  \childdoctrue
  \includeonly{\childdocname}
  \def\jobname{#1}
  \def\childdocjob{#1}
  \input{#1}
}
%    \end{macrocode}

% \macro{\childdocby}
% The command |\childdocby| ....
%    \begin{macrocode}
\newcommand{\childdocby}[2][]
{
  \childdocdisable
  \childdoctrue
  \childdocmanualtrue
  \if?#1?\else
    \def\jobname{#2}
  \fi
  \def\childdocjob{#2}
  \input{#2}
  \endinput
}
%    \end{macrocode}

% \macro{\childdocforward}
% The command |\childdocforward| redirects
% compilation to the main file or
% (if the optional argument is given) a child file.
% Parameters are set as if the main file
% or a child file starting with |\childdocof| was compiled.
% Then compilation is handed over to the main file:
%    \begin{macrocode}
\newcommand{\childdocforward}[2][]
{
  \begingroup
    \if?#1?
      \def\childdoctmp
      {
        \def\childdocname{#2}
        \def\childdocjob{#2}
        \def\jobname{#2}
        \input{#2}
        \endinput
      }
    \else
      \def\childdoctmp
      {
        \childdocdisable
        \def\childdocname{#2}
        \childdoctrue
        \includeonly{#2}
        \def\childdocjob{#1}
        \def\jobname{#1}
        \input{#1}
        \endinput
      }
    \fi
    \expandafter
  \endgroup
  \childdoctmp
}
%    \end{macrocode}

% \macro{\childdocforwardprefix}
% The command |\childdocforwardprefix| redirects
% compilation to the main or a child file by means of a pattern.
% The prefix |#1| in the current filename is replaced by |#2|
% and the suffix of the current filename is kept
% (it is assumed that the filename does not contain the substring `|~~~|'
% which is used as a delimiter).
% Compilation is handed over to the new file by |\childdocforward|:
%    \begin{macrocode}
\newcommand{\childdocforwardprefix}[3][]
{
  \begingroup
    \def\childdocextract #2##1~~~{\def\childdoctmp{\childdocforward[#1]{#3##1}}}
    \expandafter\childdocextract\childdocname~~~
    \expandafter
  \endgroup
  \childdoctmp
}
%    \end{macrocode}

% \macro{\childdoc}
% The deprecated macro |\childdoc| is a legacy version of |\childdocmain|:
%    \begin{macrocode}
\newcommand{\childdoc}{\childdocmain}
%    \end{macrocode}

% \macro{\childdocredirect}
% The deprecated macro |\childdocredirect| is a legacy version
% of |\childdocforward| and |\childdocforwardprefix|:
%    \begin{macrocode}
\newcommand{\childdocredirect}[2][]
{
  \begingroup
    \if?#1?
      \def\childdoctmp{\childdocforward{#2}}
    \else
      \def\childdoctmp{\childdocforwardprefix{#1}{#2}}
    \fi
    \expandafter
  \endgroup
  \childdoctmp
}
%    \end{macrocode}

%\iffalse
%</package>
%\fi
%
\endinput
\childdocforward[cdocsamp]{cdocsch1}"|\\
% |latex -jobname cdocscl2 \|\\
% |  "\def\version{final}% \iffalse
%
% childdoc.dtx Copyright (C) 2017-2018 Niklas Beisert
%
% This work may be distributed and/or modified under the
% conditions of the LaTeX Project Public License, either version 1.3
% of this license or (at your option) any later version.
% The latest version of this license is in
%   http://www.latex-project.org/lppl.txt
% and version 1.3 or later is part of all distributions of LaTeX
% version 2005/12/01 or later.
%
% This work has the LPPL maintenance status `maintained'.
%
% The Current Maintainer of this work is Niklas Beisert.
%
% This work consists of the files childdoc.dtx and childdoc.ins
% and the derived files childdoc.def and cdocsamp.tex with
% cdocsch1.tex, cdocsch2.tex, cdocsdrf.tex, cdocsfn1.tex, cdocsfn2.tex.
%
%<package>\ifdefined\childdocmain\endinput\fi
%<package>\ProvidesFile{childdoc.def}[2018/12/30 v2.0 child document driver]
%<samplemain>\ProvidesFile{cdocsamp.tex}[2018/12/30 v2.0 sample for childdoc]
%<*driver>
%\ProvidesFile{childdoc.drv}[2018/12/30 v2.0 childdoc reference manual file]
\PassOptionsToClass{10pt,a4paper}{article}
\documentclass{ltxdoc}

\usepackage[margin=35mm]{geometry}
\usepackage{hyperref}
\usepackage{hyperxmp}
\usepackage[usenames]{color}

\hypersetup{colorlinks=true}
\hypersetup{pdfstartview=FitH}
\hypersetup{pdfpagemode=UseNone}
\hypersetup{pdfsource={}}
\hypersetup{pdflang={en-UK}}
\hypersetup{pdfcopyright={Copyright 2017-2018 Niklas Beisert.
  This work may be distributed and/or modified under the
  conditions of the LaTeX Project Public License, either version 1.3
  of this license or (at your option) any later version.}}
\hypersetup{pdflicenseurl={http://www.latex-project.org/lppl.txt}}
\hypersetup{pdfcontactaddress={ETH Zurich, ITP, HIT K,
  Wolfgang-Pauli-Strasse 27}}
\hypersetup{pdfcontactpostcode={8093}}
\hypersetup{pdfcontactcity={Zurich}}
\hypersetup{pdfcontactcountry={Switzerland}}
\hypersetup{pdfcontactemail={nbeisert@itp.phys.ethz.ch}}
\hypersetup{pdfcontacturl={http://people.phys.ethz.ch/\xmptilde nbeisert/}}

\newcommand{\secref}[1]{\hyperref[#1]{section \ref*{#1}}}

\parskip1ex
\parindent0pt
\let\olditemize\itemize
\def\itemize{\olditemize\parskip0pt}

\begin{document}

\title{The \textsf{childdoc} Package}
\hypersetup{pdftitle={The childdoc Package}}
\author{Niklas Beisert\\[2ex]
  Institut f\"ur Theoretische Physik\\
  Eidgen\"ossische Technische Hochschule Z\"urich\\
  Wolfgang-Pauli-Strasse 27, 8093 Z\"urich, Switzerland\\[1ex]
  \href{mailto:nbeisert@itp.phys.ethz.ch}
  {\texttt{nbeisert@itp.phys.ethz.ch}}}
\hypersetup{pdfauthor={Niklas Beisert}}
\hypersetup{pdfsubject={Manual for the LaTeX2e Package childdoc}}
\date{30 December 2018, \textsf{v2.0}}
\maketitle

\begin{abstract}\noindent
\textsf{childdoc} is a \LaTeXe{} package
that enables the direct compilation
of document sections included by |\include|
to individual files.
\end{abstract}

\begingroup
\parskip0ex
\tableofcontents
\endgroup

%%%%%%%%%%%%%%%%%%%%%%%%%%%%%%%%%%%%%%%%%%%%%%%%%%%%%%%%%%%%%%%%%%%%%%%%%%%%%%%%
%%%%%%%%%%%%%%%%%%%%%%%%%%%%%%%%%%%%%%%%%%%%%%%%%%%%%%%%%%%%%%%%%%%%%%%%%%%%%%%%
\section{Introduction}

\LaTeX{} provides a mechanism to structure a large document (such as a book)
into a main file and several child files (containing the chapters)
using the |\include| command.
This mechanism is beneficial for documents
which span hundreds of pages in order to
make the source file(s) more manageable.
Moreover, compilation can be restricted to
selected child files by means of the |\includeonly| command.
The latter feature can be used to reduce the compilation time while editing
(this was significantly more useful in the earlier days of \LaTeX{})
or to generate a smaller document which is easier to navigate.
Another application of |\includeonly| is to generate
documents consisting of selected parts of the complete document.

However, there are a few drawbacks of the plain |\include| mechanism:
\begin{itemize}
\item
The child files cannot be compiled on their own,
they can only be compiled via the main file.
A naive editing environment
(such as a text editor with an option
to have the current file processed by \LaTeX)
may require one to switch to the main file before compiling;
attempting to compile the child file produces errors.
\item
The main file must be modified (each time)
to adjust the |\includeonly| command
to the present needs. This easily leaves the main file in a messy state.
\item
The generated document will always carry the filename
of the main document. This is inconvenient if
several child files are to be compiled and
to be kept for distribution.
\end{itemize}

The present package provides a simple interface
to make child files individually compilable by \LaTeX{}.
Compiling a child file then has the same effect as compiling
the main file with an |\includeonly| command
to select the appropriate child.
Moreover the generated document will carry the name of the child
rather than the main file.
This resolves all three above issues.

This feature is meant to make the editing of books,
thesis documents and lecture notes somewhat more convenient.
However, the package can also be used efficiently for
composing a series of documents (such as exercise sheets)
which are typically distributed individually.
It then assists the author in generating the individual documents
(potentially in different versions)
as well as a document containing the collected series.
Another application is in developing style files
or other kinds of included material
where compilation of the style file could redirect
to a sample or test file.

%%%%%%%%%%%%%%%%%%%%%%%%%%%%%%%%%%%%%%%%%%%%%%%%%%%%%%%%%%%%%%%%%%%%%%%%%%%%%%%%
%%%%%%%%%%%%%%%%%%%%%%%%%%%%%%%%%%%%%%%%%%%%%%%%%%%%%%%%%%%%%%%%%%%%%%%%%%%%%%%%
\section{Usage}

First of all, the package \textsf{childdoc} is \emph{not} a standard
\LaTeXe{} |.sty| style file! Therefore it needs to be invoked in
a non-standard way.

%%%%%%%%%%%%%%%%%%%%%%%%%%%%%%%%%%%%%%%%%%%%%%%%%%%%%%%%%%%%%%%%%%%%%%%%%%%%%%%%
\subsection{Included Files}
\label{sec:include}

%%%%%%%%%%%%%%%%%%%%%%%%%%%%%%%%%%%%%%%%
\DescribeMacro{\childdocmain}
To use the package, add the commands
\begin{center}
\begin{tabular}{l}
|\input{childdoc.def}|\\
|\childdocmain{}|\\
\end{tabular}
\end{center}
at the very top of the main \LaTeX{} file,
in particular \emph{before} the |\documentclass| statement!
The argument of |\childdocmain| should be left empty
(but it must be present).

%%%%%%%%%%%%%%%%%%%%%%%%%%%%%%%%%%%%%%%%
\DescribeMacro{\childdocof}
Furthermore, add the commands
\begin{center}
\begin{tabular}{l}
|\input{childdoc.def}|\\
|\childdocof{|\textit{main}|}|\\
\end{tabular}
\end{center}
at the top of every child file \textit{child}
which is included by |\include{|\textit{child}|}|
from within the main file
(or at least for those files to be compiled individually).
The argument \textit{main} must be the filename of the main file.

There are a couple of
considerations in setting up the main and child documents:

%%%%%%%%%%%%%%%%%%%%%%%%%%%%%%%%%%%%%%%%
\paragraph{Restrictions.}

Please note the following restrictions:
\begin{itemize}
\item
|\childdocmain| must be called with one argument \textit{main}
to ensure compatibility with earlier version of the package.
It must either be empty (|\childdocmain{}|)
or precisely match the filename of the main file in which it is specified.
See \secref{sec:detection} for further information.
\item
The filename \textit{main} must be specified without the |.tex| extension.
\item
The filename \textit{main} is case sensitive
(even in case-insensitive file systems)
due to internal string comparison.
\item
The argument \textit{main} should be fully expanded, it cannot be a macro.
\item
Subdirectories and special characters should be avoided in filenames.
\item
The command |\childdocmain{|\textit{main}|}| must be followed by a whitespace.
It should not be followed immediately by another command
or by a comment mark `|%|'.
This is because the \TeX{} parser reads the token immediately following
the argument of |\childdocmain| and puts it
at the beginning of every child section;
however, a white\-space is ignored.
\end{itemize}

%%%%%%%%%%%%%%%%%%%%%%%%%%%%%%%%%%%%%%%%
\paragraph{Content of Main File.}

It is advisable to place all content in the child files included by |\include|.
Any output contained in the main file will appear in all child documents
unless suppressed manually;
it cannot be suppressed automatically by the |\includeonly| directive
and thus should normally be avoided.
A method to include some content in the main file
by means of conditional processing is described in \secref{sec:conditional}.

%%%%%%%%%%%%%%%%%%%%%%%%%%%%%%%%%%%%%%%%
\paragraph{Page Numbering.}

When only a part of the document is compiled,
the appropriate numbering of pages
(as well as other status parameters)
is determined from the |.aux| files.
The latter contain information from previous passes.
However this information needs to propagate through
all intermediate child documents.
Therefore the page numbering in child documents may well
be inconsistent until the complete document is compiled at least once.

A useful (if unconventional) way to always ensure a consistent
page numbering is to restart the numbering in each child document
and denote the pages by `\textit{child}|.|\textit{page}'
where \textit{child} represents the chapter/section number of the child file.
This can be achieved by the command
|\numberwithin{page}{|\textit{child}|}|
of the \textsf{amsmath} package
where \textit{child} can be |chapter| or |section|
depending on the chosen structuring.
Alternatively, one can modify the macro |\thepage| appropriately
and reset the counter |page| at the start of each child file.

%%%%%%%%%%%%%%%%%%%%%%%%%%%%%%%%%%%%%%%%%%%%%%%%%%%%%%%%%%%%%%%%%%%%%%%%%%%%%%%%
\subsection{Conditional Processing}
\label{sec:conditional}

The package provides a mechanism to compile different versions
of a document. To customise the versions further some conditional processing
can come in handy to distinguish which version is being compiled.
The package provides two macros to describe the compilation context:

%%%%%%%%%%%%%%%%%%%%%%%%%%%%%%%%%%%%%%%%
\DescribeMacro{\ifchilddoc}
The conditional |\ifchilddoc| distinguishes between the compilation of
child documents and the main document:
%
\begin{center}
|\ifchilddoc |\textit{child-code}| |[|\||else |\textit{main-code}]| \||fi|
\end{center}

%%%%%%%%%%%%%%%%%%%%%%%%%%%%%%%%%%%%%%%%
\DescribeMacro{\childdocname}
\DescribeMacro{\childdocjob}
The macro |\childdocname| contains the filename (without extension)
of the main or child file being processed.
Note that |\childdocjob| will always contain the name of the main file.

%%%%%%%%%%%%%%%%%%%%%%%%%%%%%%%%%%%%%%%%
\paragraph{Title Page.}

Conditional processing can be used to include a title or banner page
in the main document when proper precautions are taken.
Importantly, the code in the main file should ensure that the page counter
(as well as other status parameters which are stored in the |.aux| files)
takes the same value after the conditional processing.
Otherwise the page numbers may take divergent values
depending on which part is compiled.

For example, a title page could be declared by:
%
\begin{center}
\begin{tabular}{l}
|\ifchilddoc\||else|\\
|\addtocounter{page}{-1}|\\
\textit{code for title page}\\
|\newpage|\\
|\||fi|
\end{tabular}
\end{center}
%
A banner page for the child documents can be generated by:
%
\begin{center}
\begin{tabular}{l}
|\ifchilddoc|\\
|\addtocounter{page}{-1}|\\
\textit{code for banner page}\\
|\newpage|\\
|\||fi|
\end{tabular}
\end{center}
%
Here one could write a message such as:
\begin{center}
|This is the part \childdocname{} of \childdocjob{}.|
\end{center}

%%%%%%%%%%%%%%%%%%%%%%%%%%%%%%%%%%%%%%%%%%%%%%%%%%%%%%%%%%%%%%%%%%%%%%%%%%%%%%%%
\subsection{Flags}
\label{sec:flags}

The package makes it easy to generate different versions
of the main or child documents.
To this end compilation flags can be defined
and assigned different default values.
They will be particularly useful in conjunction
with the forwarding mechanism described in \secref{sec:forward}.

For example, it may be useful to have a flag |\version|
which can be set to |draft| or |final|.
The document source will contain some conditional code
depending on the value of |\version|.
Suppose further, the flag should default to |final| for the main file
and to |draft| for child files
which is a natural assignment for editing the document.
This is achieved by placing the following code
in the preamble of the main document
(below the |\childdocmain| directive):
%
\begin{center}
\begin{tabular}{l}
|\ifchilddoc|\\
|\providecommand{\version}{draft}|\\
|\||else|\\
|\providecommand{\version}{final}|\\
|\||fi|
\end{tabular}
\end{center}
%
The definition by |\providecommand| makes sure
that previous definitions are not overwritten.
Further statements |\providecommand{\version}{...}|
can thus be added before the above code to override it.

For the main file, one might add a line
(between |\childdocmain| and the above block)
%
\begin{center}
|%\ifchilddoc\||else\providecommand{\version}{draft}\||fi|
\end{center}
%
which can be uncommented to produce a draft version.
Likewise one can add a line to the very top of a child file
(above the |\childdocof{|\textit{main}|}| directive)
%
\begin{center}
|%\providecommand{\version}{final}|
\end{center}
%
which can be uncommented to produce the final version of this child document.

%%%%%%%%%%%%%%%%%%%%%%%%%%%%%%%%%%%%%%%%%%%%%%%%%%%%%%%%%%%%%%%%%%%%%%%%%%%%%%%%
\subsection{Forwarding}
\label{sec:forward}

Different versions of the main or child documents
using compilation flags as described in \secref{sec:flags}
can be (permanently) stored in different files
for convenient compilation, viewing and distribution.
To this end, the package defines a command
to pass on compilation to a different file:

%%%%%%%%%%%%%%%%%%%%%%%%%%%%%%%%%%%%%%%%
\DescribeMacro{\childdocforward}
The command |\childdocforward| redirects processing to
another source file:
%
\begin{center}
\begin{tabular}{l}
|\input{childdoc.def}|\\
|\childdocforward[|\textit{main}|]{|\textit{dest}|}|\\
\end{tabular}
\end{center}
%
The argument \textit{dest} is the destination file
(without extension).
It should be the main file or one of the child files.
Note that further \textsf{childdoc} directives
such as |\childdocof| and |\childdocforward|
in the indicated file will be processed in this form.
The optional argument \textit{main}
passes on directly to the main file \textit{main}
while pretending to compile the child \textit{dest}.
This form behaves as if \textit{dest}
issues |\childdocof{|\textit{main}|}| right away,
and no further \textsf{childdoc} directives will be processed.

%%%%%%%%%%%%%%%%%%%%%%%%%%%%%%%%%%%%%%%%
\DescribeMacro{\...prefix}
In the alternative form |\childdocforwardprefix|,
%
\begin{center}
\begin{tabular}{l}
|\input{childdoc.def}|\\
|\childdocforwardprefix[|\textit{main}|]{|\textit{prefix}|}{|\textit{dest}|}|
\end{tabular}
\end{center}
%
the destination file is determined by a pattern
depending on the current file:
To make this work, the current file must be called
`{\textit{prefix}\hspace{0.2em}\textit{suffix}}'
with \textit{prefix} matching precisely the argument.
Processing is then passed on to the file
`{\textit{dest}\hspace{0.2em}\textit{suffix}}'.
Surely, the same effect is achieved by
directly specifying the
argument `{\textit{dest}\hspace{0.2em}\textit{suffix}}'
in the first form.
However, that requires to set up a different file
for each child. With the alternative form of the command
all these files can have exactly the same content
which simplifies setting them up and maintaining them.

For example, the following file |draft.tex|
with a compilation flag |\version| as described in \secref{sec:flags}
compiles the main document as a draft:
%
\begin{center}
\begin{tabular}{l}
|\def\version{draft}|\\
|\input{childdoc.def}|\\
|\childdocforward{|\textit{main}|}|
\end{tabular}
\end{center}
%
Likewise, the following files |final|\textit{nn}|.tex|
compile the final version of the child document
|child|\textit{nn}|.tex|:
%
\begin{center}
\begin{tabular}{l}
|\def\version{final}|\\
|\input{childdoc.def}|\\
|\childdocforwardprefix{final}{child}|
\end{tabular}
\end{center}
%

Note that when several versions of a main file and/or of each child file
are to be generated, it may be convenient to set up a |Makefile| or
shell script to automatise the process.

%%%%%%%%%%%%%%%%%%%%%%%%%%%%%%%%%%%%%%%%%%%%%%%%%%%%%%%%%%%%%%%%%%%%%%%%%%%%%%%%
\subsection{Command Line Processing}
\label{sec:commandline}

The effect of redirection files can also be achieved by invoking
the \LaTeX{} compiler with a more elaborate command line.
Most conveniently this should be done as part
of a shell script or a |Makefile|.

When using \textsf{childdoc} in the main file, the following
command lines effectively perform a redirection
(note that depending on the shell being used,
backslashes may have to be doubled: `|\|' $\to$ `|\\|'):
%
\begin{center}
|... -jobname "|\textit{target}|" |\\|"|[\textit{flags}]%
|\input{childdoc.def}\childdocforward[|\textit{main}|]{|\textit{dest}|}"|
\end{center}
%
Here \textit{target} is the name of the output file,
\textit{main} is the name of the main file
and \textit{dest} is the name of the main or child file to be processed
(all filenames without extensions).
The optional argument \textit{main} can be omitted
if \textit{main} matches \textit{dest}.
Optionally, compilation \textit{flags} can be defined via |\def| commands.
This command line makes the \TeX{} engine believe
it is compiling the file \textit{target}
whose content is specified as the latter parameter.
The provided code then forwards the processing to
\textit{main} or \textit{dest} as described in \secref{sec:forward}.

%%%%%%%%%%%%%%%%%%%%%%%%%%%%%%%%%%%%%%%%%%%%%%%%%%%%%%%%%%%%%%%%%%%%%%%%%%%%%%%%
\subsection{Include by Input}
\label{sec:input}

Including child documents by |\include| has some restrictions by design.
Most notably, the content of a child document always occupies
its own set of pages; pages cannot be shared between child documents.
Usually, this behaviour makes perfect sense
because each child document contain an essential part of the document.
However, in some situations it may be desirable to compose
a document from a collection of parts
without having mandatory page breaks between then.
For this case, the package
provides a mechanism to include parts
by |\input| which can also be processed individually.
However, by construction this mechanism
requires manual handling of the content to be output.

%%%%%%%%%%%%%%%%%%%%%%%%%%%%%%%%%%%%%%%%
\DescribeMacro{\ifchilddocmanual}
The main file should be prepared as usual, see \secref{sec:include}.
However, the document body must make a distinction
between processing of an individual part and of the main document, e.g.:
%
\begin{center}
\begin{tabular}{l}
|\ifchilddocmanual|\\
|\input{\childdocname}|\\
|\||else|\\
\textit{document body with }|\input{|\textit{part}|}|\\
|\||fi|
\end{tabular}
\end{center}
%
The conditional |\ifchilddocmanual| is true whenever
a part to be included by |\input| is being compiled,
and the name of the part is stored in |\childdocname|.

%%%%%%%%%%%%%%%%%%%%%%%%%%%%%%%%%%%%%%%%
\DescribeMacro{\childdocby}
Each part to be included by |\input| should start with:
%
\begin{center}
\begin{tabular}{l}
|\input{childdoc.def}|\\
|\childdocby{|\textit{main}|}|\\
\end{tabular}
\end{center}
%
The directive |\childdocby| is similar to |\childdocof|
described in \secref{sec:include},
but the subsequent selection of content must be done manually.
To that end, both |\ifchilddoc| and |\ifchilddocmanual|
will be true upon processing of a part,
and the name of the part is stored in |\childdocname|.
Note that |\jobname| will be set to the filename of the current part
so that each part receives an individual |.aux| file
that does not interfere with the |.aux| file(s) of the main document.
This behaviour can be altered by the alternative form
|\childdocby[*]{|\textit{main}|}| (with a non-empty optional argument)
which uses the |.aux| file of the main document
by setting |\jobname| to \textit{main}.

%%%%%%%%%%%%%%%%%%%%%%%%%%%%%%%%%%%%%%%%%%%%%%%%%%%%%%%%%%%%%%%%%%%%%%%%%%%%%%%%
\subsection{Driver Development}
\label{sec:driver}

The \textsf{childdoc} mechanism can also be use for the development
of definition files such as \LaTeX{} styles or classes.
This case differs from the above setup with multiple parts
included by |\include| in that no |\includeonly| should be invoked.
This can be achieved by starting the include file
(before |\ProvidesPackage|) with:
%
\begin{center}
\begin{tabular}{l}
|\input{childdoc.def}|\\
|\childdocforward{|\textit{main}|}|\\
\end{tabular}
\end{center}
%
or alternatively with:
%
\begin{center}
\begin{tabular}{l}
|\input{childdoc.def}|\\
|\childdocby{|\textit{main}|}|\\
\end{tabular}
\end{center}
%
Both forms have slightly different effects as described above.
The main file is prepared as usual, see \secref{sec:include}.

%%%%%%%%%%%%%%%%%%%%%%%%%%%%%%%%%%%%%%%%%%%%%%%%%%%%%%%%%%%%%%%%%%%%%%%%%%%%%%%%
\subsection{Legacy Detection}
\label{sec:detection}

The directive |\childdocmain| in the main file can detect
whether the complete document or merely a child is to be compiled
even without using the directive |\childdocof|.
This method is deprecated because it is less robust
and there is no compelling reason to use it;
it is merely provided for backward compatibility
and it may be removed in future versions.

If the detection mechanism is to be used,
it is mandatory to correctly specify
the filename of the main file as the argument of |\childdocmain|:
%
\begin{center}
\begin{tabular}{l}
|\input{childdoc.def}|\\
|\childdocmain{|\textit{main}|}|\\
\end{tabular}
\end{center}
%
If |\jobname| does not match the argument \textit{main} of |\childdocmain|,
it is assumed that |\jobname| points to the child file to be compiled.
When using |\childdocmain| with the main file specified as argument,
it suffices to start a child file
with just |\input{|\textit{main}|}|
without loading of the package and using |\childdocof|.
If instead all processing is done
with the appropriate \textsf{childdoc} directives,
the argument of \textit{main} of |\childdocmain| can be empty.

An alternative version of the command line processing described
in \secref{sec:commandline} using the detection mechanism reads:
%
\begin{center}
|... -jobname "|\textit{target}|" "|[\textit{flags}]%
[|\def\jobname{|\textit{dest}|}|]|\input{|\textit{main}|}"|
\end{center}

%%%%%%%%%%%%%%%%%%%%%%%%%%%%%%%%%%%%%%%%%%%%%%%%%%%%%%%%%%%%%%%%%%%%%%%%%%%%%%%%
\subsection{Manual Code}
\label{sec:manual}

In case one cannot be certain whether the definitions file |childdoc.def|
is installed on the target \TeX{} distribution
and one prefers not to ship it,
it is conceivable to paste a few relevant commands into the sources.

To that end, drop all statements |\input{childdoc.def}|
and perform the replacements as outlined below.
Instead of |\childdocmain{|\textit{main}|}| add the following code
to the top of the main file:
%
\begin{center}
\begin{tabular}{l}
|\||ifdefined\childdocname\endinput\||fi\newif\ifchilddoc|\\
|\edef\childdocname{\scantokens\expandafter{\jobname\noexpand}}|\\
|\def\childdocmain{|\textit{main}|}\||ifx\childdocmain\childdocname\||else|\\
|\childdoctrue\includeonly{\childdocname}\let\jobname\childdocmain\||fi|\\
\end{tabular}
\end{center}
%
Instead of |\childdocof{|\textit{main}|}| just include the main file
at the top of each child file:
%
\begin{center}
|\input{|\textit{main}|}|
\end{center}
%
A simple redirection |\childdocforward{|\textit{dest}|}| is achieved by:
%
\begin{center}
|\def\jobname{|\textit{dest}|}\input{\jobname}|
\end{center}
%
The redirection with prefix
|\childdocforwardprefix[|\textit{prefix}|]{|\textit{dest}|}|
is accomplished by:
%
\begin{center}
\begin{tabular}{l}
|{\edef\jobname{\scantokens\expandafter{\jobname\noexpand}}|\\
|\def\redirectjob |\textit{prefix}|#1~~~{\gdef\jobname{|\textit{dest}|#1}}|\\
|\expandafter\redirectjob\jobname~~~}\input{\jobname}|
\end{tabular}
\end{center}

In an alternative approach,
child documents can be compiled by a specific command line
without additional code or specific definitions:
%
\begin{center}
|... -jobname "|\textit{target}|" "|[\textit{flags}]%
|\includeonly{|\textit{dest}|}\input{|\textit{main}|}"|
\end{center}
%

%%%%%%%%%%%%%%%%%%%%%%%%%%%%%%%%%%%%%%%%%%%%%%%%%%%%%%%%%%%%%%%%%%%%%%%%%%%%%%%%
%%%%%%%%%%%%%%%%%%%%%%%%%%%%%%%%%%%%%%%%%%%%%%%%%%%%%%%%%%%%%%%%%%%%%%%%%%%%%%%%
\section{Information}

%%%%%%%%%%%%%%%%%%%%%%%%%%%%%%%%%%%%%%%%%%%%%%%%%%%%%%%%%%%%%%%%%%%%%%%%%%%%%%%%
\subsection{Copyright}

Copyright \copyright{} 2017--2018 Niklas Beisert

This work may be distributed and/or modified under the
conditions of the \LaTeX{} Project Public License, either version 1.3
of this license or (at your option) any later version.
The latest version of this license is in
  \url{http://www.latex-project.org/lppl.txt}
and version 1.3 or later is part of all distributions of \LaTeX{}
version 2005/12/01 or later.

This work has the LPPL maintenance status `maintained'.

The Current Maintainer of this work is Niklas Beisert.

This work consists of the files |README.txt|, |childdoc.ins| and |childdoc.dtx|
as well as the derived files |childdoc.def|, |cdocsamp.tex|
with |cdocsch1.tex|, |cdocsch2.tex|, |cdocspt3.tex|, |cdocspt4.tex|,
|cdocsdrf.tex|, |cdocsfn1.tex|, |cdocsfn2.tex|
as well as |childdoc.pdf|.

%%%%%%%%%%%%%%%%%%%%%%%%%%%%%%%%%%%%%%%%%%%%%%%%%%%%%%%%%%%%%%%%%%%%%%%%%%%%%%%%
\subsection{Files and Installation}

The package consists of the files:
%
\begin{center}
\begin{tabular}{ll}
    |README.txt|   & readme file \\
    |childdoc.ins| & installation file \\
    |childdoc.dtx| & source file \\
    |childdoc.def| & definition file \\
    |cdocsamp.tex| & sample main file \\
    |cdocsch1.tex| & sample include file \\
    |cdocsch2.tex| & sample include file \\
    |cdocspt3.tex| & sample part file \\
    |cdocspt4.tex| & sample part file \\
    |cdocsdrf.tex| & sample redirection file \\
    |cdocsfn1.tex| & sample redirection file \\
    |cdocsfn2.tex| & sample redirection file \\
    |childdoc.pdf| & manual
\end{tabular}
\end{center}
%
The distribution consists of the files
|README.txt|, |childdoc.ins| and |childdoc.dtx|.
%
\begin{itemize}
\item
Run (pdf)\LaTeX{} on |childdoc.dtx|
to compile the manual |childdoc.pdf| (this file).
\item
Run \LaTeX{} on |childdoc.ins| to create the definitions file |childdoc.def|
and the sample |cdocsamp.tex| with include files
|cdocsch1.tex|, |cdocsch2.tex|, |cdocspt3.tex|, |cdocspt4.tex|,
|cdocsdrf.tex|, |cdocsfn1.tex|, |cdocsfn2.tex|.
Then copy the file |childdoc.def| to an appropriate directory of your \LaTeX{}
distribution, e.g.\ \textit{texmf-root}|/tex/latex/childdoc|.
\end{itemize}

%%%%%%%%%%%%%%%%%%%%%%%%%%%%%%%%%%%%%%%%%%%%%%%%%%%%%%%%%%%%%%%%%%%%%%%%%%%%%%%%
\subsection{Related CTAN Packages}

There are several other packages which offer a similar functionality:
%
\begin{itemize}
\item
The packages
\href{http://ctan.org/pkg/docmute}{\textsf{docmute}},
\href{http://ctan.org/pkg/includex}{\textsf{includex}} and
\href{http://ctan.org/pkg/standalone}{\textsf{standalone}}
provide commands to include only the document body of
a child file thus allowing both files to be compiled individually.
\item
The packages \href{http://ctan.org/pkg/subdocs}{\textsf{subdocs}}
and \href{http://ctan.org/pkg/subfiles}{\textsf{subfiles}}
provide structures in which the main and child documents can be
encapsulated and allowing them to be compiled individually.
The inclusion mechanism is different from the conventional |\include|.
\item
The package \href{http://ctan.org/pkg/combine}{\textsf{combine}}
is an elaborate solution to combine several documents into one.
\end{itemize}
%
See also the CTAN topic \href{http://ctan.org/topic/subdocs}{\textsf{subdocs}}
for further related packages.
The present package differs from the above solutions in that
a document structure constructed with the conventional |\include| mechanism
just needs two extra commands at the top of every file
such that all constituent files can be compiled individually.

%%%%%%%%%%%%%%%%%%%%%%%%%%%%%%%%%%%%%%%%%%%%%%%%%%%%%%%%%%%%%%%%%%%%%%%%%%%%%%%%
%\subsection{Feature Suggestions}
%
%The following is a list of features which may be useful for future
%versions of this package:
%%
%\begin{itemize}
%\item
%\ldots
%\end{itemize}

%%%%%%%%%%%%%%%%%%%%%%%%%%%%%%%%%%%%%%%%%%%%%%%%%%%%%%%%%%%%%%%%%%%%%%%%%%%%%%%%
\subsection{Revision History}

%%%%%%%%%%%%%%%%%%%%%%%%%%%%%%%%%%%%%%%%
\paragraph{v2.0:} 2018/12/30

\begin{itemize}
\item
immediate forward processing
\item
added |\childdocby| mechanism
\item
manual restructured
\end{itemize}

%%%%%%%%%%%%%%%%%%%%%%%%%%%%%%%%%%%%%%%%
\paragraph{v1.6:} 2018/01/17

\begin{itemize}
\item
application for development of include files
\item
corrections to manual
\end{itemize}

%%%%%%%%%%%%%%%%%%%%%%%%%%%%%%%%%%%%%%%%
\paragraph{v1.5:} 2017/05/21

\begin{itemize}
\item
more complete structuring introduced
\item
|\childdocof| introduced
\item
|\childdoc| renamed to |\childdocmain|
\item
|\childredirect| renamed to |\childdocforward| and |\childdocforwardprefix|
and functionality expanded
\end{itemize}

%%%%%%%%%%%%%%%%%%%%%%%%%%%%%%%%%%%%%%%%
\paragraph{v1.0:} 2017/04/27

\begin{itemize}
\item
manual and install package
\item
first version published on CTAN
\end{itemize}

%%%%%%%%%%%%%%%%%%%%%%%%%%%%%%%%%%%%%%%%
\paragraph{v0.6:} 2017/04/26

\begin{itemize}
\item
redirection mechanism added
\end{itemize}

%%%%%%%%%%%%%%%%%%%%%%%%%%%%%%%%%%%%%%%%
\paragraph{v0.5:} 2017/04/26

\begin{itemize}
\item
functionality in definition file
\end{itemize}


%%%%%%%%%%%%%%%%%%%%%%%%%%%%%%%%%%%%%%%%%%%%%%%%%%%%%%%%%%%%%%%%%%%%%%%%%%%%%%%%
%%%%%%%%%%%%%%%%%%%%%%%%%%%%%%%%%%%%%%%%%%%%%%%%%%%%%%%%%%%%%%%%%%%%%%%%%%%%%%%%
%%%%%%%%%%%%%%%%%%%%%%%%%%%%%%%%%%%%%%%%%%%%%%%%%%%%%%%%%%%%%%%%%%%%%%%%%%%%%%%%
\appendix

\settowidth\MacroIndent{\rmfamily\scriptsize 000\ }

 \DocInput{childdoc.dtx}

\end{document}
%</driver>
% \fi
%
% %%%%%%%%%%%%%%%%%%%%%%%%%%%%%%%%%%%%%%%%%%%%%%%%%%%%%%%%%%%%%%%%%%%%%%%%%%%%%%
% %%%%%%%%%%%%%%%%%%%%%%%%%%%%%%%%%%%%%%%%%%%%%%%%%%%%%%%%%%%%%%%%%%%%%%%%%%%%%%
% \section{Sample}
%\iffalse
%<*samplemain>
%\fi
%
% The following presents a sample document
% with two chapters, two parts, a title page,
% a compile flag as well as three forwarding files to set the flag.
% It consists of eight |.tex| files:
% \begin{center}
% \begin{tabular}{ll}
% |cdocsamp.tex|&main file\\
% |cdocsch1.tex|&include file for chapter 1\\
% |cdocsch2.tex|&include file for chapter 2\\
% |cdocspt3.tex|&include file for part 3\\
% |cdocspt4.tex|&include file for part 4\\
% |cdocsdrf.tex|&forwarding file for main file in draft mode\\
% |cdocsfi1.tex|&forwarding file for final version of chapter 1\\
% |cdocsfi2.tex|&forwarding file for final version of chapter 2\\
% \end{tabular}
% \end{center}
% Each of the eight files can be compiled directly by the \LaTeX{} compiler.
%
% %%%%%%%%%%%%%%%%%%%%%%%%%%%%%%%%%%%%%%
% \paragraph{Main File.}
%
% The main file is called |cdocsamp.tex|.
%
% Load the \textsf{childdoc} definitions and
% declare the filename for the main document:
%    \begin{macrocode}
\input{childdoc.def}
\childdocmain{}
%    \end{macrocode}

% Optional override for |\version| flag:
%    \begin{macrocode}
%%\ifchilddoc\else\providecommand{\version}{draft}\fi
%    \end{macrocode}

% Define the default values for the |\version| flag
% (|final| for the main file and |draft| for childs):
%    \begin{macrocode}
\ifchilddoc
\providecommand{\version}{draft}
\else
\providecommand{\version}{final}
\fi
%    \end{macrocode}

% Load the standard document class:
%    \begin{macrocode}
\documentclass[12pt]{article}
%    \end{macrocode}

% Start the document body:
%    \begin{macrocode}
\begin{document}
%    \end{macrocode}

% Declare a title page.
% Print title, part of document being processed and version flag:
%    \begin{macrocode}
\addtocounter{page}{-1}
\begin{center}
{\LARGE\bfseries{}childdoc example\par}
\vspace{1cm}
\ifchilddoc
\ifchilddocmanual part\else chapter\fi:
`\childdocname' of `\childdocjob'\par
\else
main document: `\childdocjob'\par
\fi
version: \version\par
\end{center}
\newpage
%    \end{macrocode}

% Manually include selected file,
% otherwise process as usual:
%    \begin{macrocode}
\ifchilddocmanual
\section*{part `\childdocname'}
\input{\childdocname}
\else
%    \end{macrocode}

% Include the two chapters:
%    \begin{macrocode}
\include{cdocsch1}
\include{cdocsch2}
%    \end{macrocode}

% Include the two parts unless only chapters should be displayed:
%    \begin{macrocode}
\ifchilddoc\else
\section{part three}
\input{cdocspt3}
\section{part four}
\input{cdocspt4}
\fi
%    \end{macrocode}

% Process as usual until here:
%    \begin{macrocode}
\fi
%    \end{macrocode}

% End of document body:
%    \begin{macrocode}
\end{document}
%    \end{macrocode}
%\iffalse
%</samplemain>
%\fi
%
% %%%%%%%%%%%%%%%%%%%%%%%%%%%%%%%%%%%%%%
% \paragraph{Chapter Include Files.}
%
% The include files are called |cdocsch1.tex| and |cdocsch2.tex|.
%
%\iffalse
%<*samplechap1|samplechap2>
%\fi

% Optional override for |\version| flag:
%    \begin{macrocode}
%%\providecommand{\version}{final}
%    \end{macrocode}

% Include the main document:
%    \begin{macrocode}
\input{childdoc.def}
\childdocof{cdocsamp}
%    \end{macrocode}

%\iffalse
%</samplechap1|samplechap2>
%\fi
%
%\iffalse
%<*samplechap1>
%\fi
% Some text for chapter 1:
%    \begin{macrocode}
\section{one}
some text in chapter one
%    \end{macrocode}

%\iffalse
%</samplechap1>
%\fi
% Some text for chapter 2:
%\iffalse
%<*samplechap2>
%\fi
%    \begin{macrocode}
\section{two}
more text in chapter two
%    \end{macrocode}

%\iffalse
%</samplechap2>
%\fi
%
% %%%%%%%%%%%%%%%%%%%%%%%%%%%%%%%%%%%%%%
% \paragraph{Part Include Files.}
%
% The include files are called |cdocspt3.tex| and |cdocspt4.tex|.
%
%\iffalse
%<*samplepart3|samplepart4>
%\fi

% Optional override for |\version| flag:
%    \begin{macrocode}
%%\providecommand{\version}{final}
%    \end{macrocode}

% Include the main document:
%    \begin{macrocode}
\input{childdoc.def}
\childdocby{cdocsamp}
%    \end{macrocode}

%\iffalse
%</samplepart3|samplepart4>
%\fi
%
%\iffalse
%<*samplepart3>
%\fi
% Some text for part 3:
%    \begin{macrocode}
some text in part three
%    \end{macrocode}

%\iffalse
%</samplepart3>
%\fi
% Some text for part 4:
%\iffalse
%<*samplepart4>
%\fi
%    \begin{macrocode}
more text in part four
%    \end{macrocode}

%\iffalse
%</samplepart4>
%\fi
%
% %%%%%%%%%%%%%%%%%%%%%%%%%%%%%%%%%%%%%%
% \paragraph{Forwarding for a Complete Draft.}
%
% The following forwarding file |cdocsdrf.tex|
% compiles the main document in draft mode:
%\iffalse
%<*sampledraft>
%\fi
%    \begin{macrocode}
\def\version{draft}
\input{childdoc.def}
\childdocforward{cdocsamp}
%    \end{macrocode}

%\iffalse
%</sampledraft>
%\fi
%
% %%%%%%%%%%%%%%%%%%%%%%%%%%%%%%%%%%%%%%
% \paragraph{Forwarding for Final Version of the Chapters.}
%
% The following forwarding files |cdocsfn1.tex| and |cdocsfn2.tex|
% (with identical content)
% compile the final versions of the child documents
% |cdocsch1.tex| and |cdocsch2.tex|, respectively:
%\iffalse
%<*samplefinal>
%\fi
%    \begin{macrocode}
\def\version{final}
\input{childdoc.def}
\childdocforwardprefix[cdocsamp]{cdocsfn}{cdocsch}
%    \end{macrocode}

%\iffalse
%</samplefinal>
%\fi
%
% %%%%%%%%%%%%%%%%%%%%%%%%%%%%%%%%%%%%%%
% \paragraph{Command Line Processing.}
%
% The following three command lines generate the output files
% |cdocscld|, |cdocscl1| and |cdocscl2|
% which should be identical to
% |cdocsdrf|, |cdocsch1| and |cdocsfn2|, respectively:
% \begin{center}
% \begin{tabular}{l}
% |latex -jobname cdocscld \|\\
% |  "\def\version{draft}\input{childdoc.def}\childdocforward{cdocsamp}"|\\
% |latex -jobname cdocscl1 \|\\
% |  "\input{childdoc.def}\childdocforward[cdocsamp]{cdocsch1}"|\\
% |latex -jobname cdocscl2 \|\\
% |  "\def\version{final}\input{childdoc.def}\childdocforward{cdocsch2}"|
% \end{tabular}
% \end{center}
% Note that the trailing backslash on each first line
% merely continues the input to the second line
% (for convenient cut ant paste).
% Furthermore, the command |latex| can be replaced by any
% of its alternative versions such as |pdflatex|.
%
% %%%%%%%%%%%%%%%%%%%%%%%%%%%%%%%%%%%%%%%%%%%%%%%%%%%%%%%%%%%%%%%%%%%%%%%%%%%%%%
% %%%%%%%%%%%%%%%%%%%%%%%%%%%%%%%%%%%%%%%%%%%%%%%%%%%%%%%%%%%%%%%%%%%%%%%%%%%%%%
% \section{Implementation}
%\iffalse
%<*package>
%\fi
%
% This section describes the definitions file |childdoc.def|.

% The definitions cannot be loaded using |\usepackage| or |\RequirePackage|
% which has a mechanism to prevent loading a style file more than once.
% When loading the definitions by means of |\input|
% multiple instances have to be prevented manually:
%\iffalse
%This code needs to be before the `\ProvidesFile' directive
%which is defined at the beginning of this file.
%Therefore it is also placed there and commented out here.
%</package>
%<*discard>
%\fi
%    \begin{macrocode}
\ifdefined\childdocmain\endinput\fi
%    \end{macrocode}
%\iffalse
%</discard>
%<*package>
%\fi
%
% \macro{\ifchilddoc}
% \macro{\ifchilddocmanual}
% The conditional |\ifchilddoc| tells whether a
% child (true) or main (false) document is being compiled.
% The conditional |\ifchilddocmanual| tells whether
% the |\includeonly| mechanism is used (false) or
% the selection of child files must be performed manually (true).
% The definitions initialise to false:
%    \begin{macrocode}
\newif\ifchilddoc
\newif\ifchilddocmanual
%    \end{macrocode}

% \macro{\childdocname}
% \macro{\childdocjob}
% The macro |\childdocname| stores the name of the main document
% to be compiled. The macro |\childdocjob| stores the name of
% the document on which the \LaTeX{} compiler was originally invoked.
% The content of |\jobname| cannot be compared
% to filenames specified in the source due to different catcodes.
% The following code rescans |\jobname|, stores the result
% in |\childdocname| and saves a copy in |\childdocjob|:
%    \begin{macrocode}
\edef\childdocname{\scantokens\expandafter{\jobname\noexpand}}
\let\childdocjob\childdocname
%    \end{macrocode}

% \macro{\childdocdisable}
% The macro |\childdocdisable| prevents the main file
% from being processed more than once.
% At this stage, the main document command |\childdocmain|
% is assumed to be called once again where it should do nothing.
% Any subsequent call to it should prevent
% a secondary processing of the main document
% It overwrites the forwarding commands
% |\childdocof| and |\childdocforward|
% with empty macros to prevent further inclusions of the main document:
%    \begin{macrocode}
\newcommand{\childdocdisable}
{
  \renewcommand{\childdocmain}[1]{\renewcommand{\childdocmain}[1]{\endinput}}
  \renewcommand{\childdocof}[1]{}
  \renewcommand{\childdocby}[2][]{}
  \renewcommand{\childdocforward}[2][]{}
  \renewcommand{\childdocdisable}{}
}
%    \end{macrocode}

% \macro{\childdocmain}
% The macro |\childdocmain| is to be called at the top of the main file
% with nothing or the main filename (without extension) as argument.
% First, it breaks loops.
% If the argument is not empty and does not match |\childdocname|
% (which is set by the first inclusion of |childdoc.def|),
% |\ifchilddoc| is set to true, |\includeonly| is applied to the child file
% and |\jobname| is set to the main file
% (for proper handling of |.aux| files):
%    \begin{macrocode}
\newcommand{\childdocmain}[1]
{
  \childdocdisable\childdocmain{}
  \if?#1?\else
    \begingroup
      \def\childdoctmp{#1}
      \ifx\childdoctmp\childdocname
        \def\childdoctmp{}
      \else
        \def\childdoctmp
        {
          \childdoctrue
          \includeonly{\childdocname}
          \def\childdocjob{#1}
          \def\jobname{#1}
        }
      \fi
      \expandafter
    \endgroup
    \childdoctmp
  \fi
}
%    \end{macrocode}

% \macro{\childdocof}
% The command |\childdocof| redirects
% compilation to the main file |#1|.
%    \begin{macrocode}
\newcommand{\childdocof}[1]
{
  \childdocdisable
  \childdoctrue
  \includeonly{\childdocname}
  \def\jobname{#1}
  \def\childdocjob{#1}
  \input{#1}
}
%    \end{macrocode}

% \macro{\childdocby}
% The command |\childdocby| ....
%    \begin{macrocode}
\newcommand{\childdocby}[2][]
{
  \childdocdisable
  \childdoctrue
  \childdocmanualtrue
  \if?#1?\else
    \def\jobname{#2}
  \fi
  \def\childdocjob{#2}
  \input{#2}
  \endinput
}
%    \end{macrocode}

% \macro{\childdocforward}
% The command |\childdocforward| redirects
% compilation to the main file or
% (if the optional argument is given) a child file.
% Parameters are set as if the main file
% or a child file starting with |\childdocof| was compiled.
% Then compilation is handed over to the main file:
%    \begin{macrocode}
\newcommand{\childdocforward}[2][]
{
  \begingroup
    \if?#1?
      \def\childdoctmp
      {
        \def\childdocname{#2}
        \def\childdocjob{#2}
        \def\jobname{#2}
        \input{#2}
        \endinput
      }
    \else
      \def\childdoctmp
      {
        \childdocdisable
        \def\childdocname{#2}
        \childdoctrue
        \includeonly{#2}
        \def\childdocjob{#1}
        \def\jobname{#1}
        \input{#1}
        \endinput
      }
    \fi
    \expandafter
  \endgroup
  \childdoctmp
}
%    \end{macrocode}

% \macro{\childdocforwardprefix}
% The command |\childdocforwardprefix| redirects
% compilation to the main or a child file by means of a pattern.
% The prefix |#1| in the current filename is replaced by |#2|
% and the suffix of the current filename is kept
% (it is assumed that the filename does not contain the substring `|~~~|'
% which is used as a delimiter).
% Compilation is handed over to the new file by |\childdocforward|:
%    \begin{macrocode}
\newcommand{\childdocforwardprefix}[3][]
{
  \begingroup
    \def\childdocextract #2##1~~~{\def\childdoctmp{\childdocforward[#1]{#3##1}}}
    \expandafter\childdocextract\childdocname~~~
    \expandafter
  \endgroup
  \childdoctmp
}
%    \end{macrocode}

% \macro{\childdoc}
% The deprecated macro |\childdoc| is a legacy version of |\childdocmain|:
%    \begin{macrocode}
\newcommand{\childdoc}{\childdocmain}
%    \end{macrocode}

% \macro{\childdocredirect}
% The deprecated macro |\childdocredirect| is a legacy version
% of |\childdocforward| and |\childdocforwardprefix|:
%    \begin{macrocode}
\newcommand{\childdocredirect}[2][]
{
  \begingroup
    \if?#1?
      \def\childdoctmp{\childdocforward{#2}}
    \else
      \def\childdoctmp{\childdocforwardprefix{#1}{#2}}
    \fi
    \expandafter
  \endgroup
  \childdoctmp
}
%    \end{macrocode}

%\iffalse
%</package>
%\fi
%
\endinput
\childdocforward{cdocsch2}"|
% \end{tabular}
% \end{center}
% Note that the trailing backslash on each first line
% merely continues the input to the second line
% (for convenient cut ant paste).
% Furthermore, the command |latex| can be replaced by any
% of its alternative versions such as |pdflatex|.
%
% %%%%%%%%%%%%%%%%%%%%%%%%%%%%%%%%%%%%%%%%%%%%%%%%%%%%%%%%%%%%%%%%%%%%%%%%%%%%%%
% %%%%%%%%%%%%%%%%%%%%%%%%%%%%%%%%%%%%%%%%%%%%%%%%%%%%%%%%%%%%%%%%%%%%%%%%%%%%%%
% \section{Implementation}
%\iffalse
%<*package>
%\fi
%
% This section describes the definitions file |childdoc.def|.

% The definitions cannot be loaded using |\usepackage| or |\RequirePackage|
% which has a mechanism to prevent loading a style file more than once.
% When loading the definitions by means of |\input|
% multiple instances have to be prevented manually:
%\iffalse
%This code needs to be before the `\ProvidesFile' directive
%which is defined at the beginning of this file.
%Therefore it is also placed there and commented out here.
%</package>
%<*discard>
%\fi
%    \begin{macrocode}
\ifdefined\childdocmain\endinput\fi
%    \end{macrocode}
%\iffalse
%</discard>
%<*package>
%\fi
%
% \macro{\ifchilddoc}
% \macro{\ifchilddocmanual}
% The conditional |\ifchilddoc| tells whether a
% child (true) or main (false) document is being compiled.
% The conditional |\ifchilddocmanual| tells whether
% the |\includeonly| mechanism is used (false) or
% the selection of child files must be performed manually (true).
% The definitions initialise to false:
%    \begin{macrocode}
\newif\ifchilddoc
\newif\ifchilddocmanual
%    \end{macrocode}

% \macro{\childdocname}
% \macro{\childdocjob}
% The macro |\childdocname| stores the name of the main document
% to be compiled. The macro |\childdocjob| stores the name of
% the document on which the \LaTeX{} compiler was originally invoked.
% The content of |\jobname| cannot be compared
% to filenames specified in the source due to different catcodes.
% The following code rescans |\jobname|, stores the result
% in |\childdocname| and saves a copy in |\childdocjob|:
%    \begin{macrocode}
\edef\childdocname{\scantokens\expandafter{\jobname\noexpand}}
\let\childdocjob\childdocname
%    \end{macrocode}

% \macro{\childdocdisable}
% The macro |\childdocdisable| prevents the main file
% from being processed more than once.
% At this stage, the main document command |\childdocmain|
% is assumed to be called once again where it should do nothing.
% Any subsequent call to it should prevent
% a secondary processing of the main document
% It overwrites the forwarding commands
% |\childdocof| and |\childdocforward|
% with empty macros to prevent further inclusions of the main document:
%    \begin{macrocode}
\newcommand{\childdocdisable}
{
  \renewcommand{\childdocmain}[1]{\renewcommand{\childdocmain}[1]{\endinput}}
  \renewcommand{\childdocof}[1]{}
  \renewcommand{\childdocby}[2][]{}
  \renewcommand{\childdocforward}[2][]{}
  \renewcommand{\childdocdisable}{}
}
%    \end{macrocode}

% \macro{\childdocmain}
% The macro |\childdocmain| is to be called at the top of the main file
% with nothing or the main filename (without extension) as argument.
% First, it breaks loops.
% If the argument is not empty and does not match |\childdocname|
% (which is set by the first inclusion of |childdoc.def|),
% |\ifchilddoc| is set to true, |\includeonly| is applied to the child file
% and |\jobname| is set to the main file
% (for proper handling of |.aux| files):
%    \begin{macrocode}
\newcommand{\childdocmain}[1]
{
  \childdocdisable\childdocmain{}
  \if?#1?\else
    \begingroup
      \def\childdoctmp{#1}
      \ifx\childdoctmp\childdocname
        \def\childdoctmp{}
      \else
        \def\childdoctmp
        {
          \childdoctrue
          \includeonly{\childdocname}
          \def\childdocjob{#1}
          \def\jobname{#1}
        }
      \fi
      \expandafter
    \endgroup
    \childdoctmp
  \fi
}
%    \end{macrocode}

% \macro{\childdocof}
% The command |\childdocof| redirects
% compilation to the main file |#1|.
%    \begin{macrocode}
\newcommand{\childdocof}[1]
{
  \childdocdisable
  \childdoctrue
  \includeonly{\childdocname}
  \def\jobname{#1}
  \def\childdocjob{#1}
  \input{#1}
}
%    \end{macrocode}

% \macro{\childdocby}
% The command |\childdocby| ....
%    \begin{macrocode}
\newcommand{\childdocby}[2][]
{
  \childdocdisable
  \childdoctrue
  \childdocmanualtrue
  \if?#1?\else
    \def\jobname{#2}
  \fi
  \def\childdocjob{#2}
  \input{#2}
  \endinput
}
%    \end{macrocode}

% \macro{\childdocforward}
% The command |\childdocforward| redirects
% compilation to the main file or
% (if the optional argument is given) a child file.
% Parameters are set as if the main file
% or a child file starting with |\childdocof| was compiled.
% Then compilation is handed over to the main file:
%    \begin{macrocode}
\newcommand{\childdocforward}[2][]
{
  \begingroup
    \if?#1?
      \def\childdoctmp
      {
        \def\childdocname{#2}
        \def\childdocjob{#2}
        \def\jobname{#2}
        \input{#2}
        \endinput
      }
    \else
      \def\childdoctmp
      {
        \childdocdisable
        \def\childdocname{#2}
        \childdoctrue
        \includeonly{#2}
        \def\childdocjob{#1}
        \def\jobname{#1}
        \input{#1}
        \endinput
      }
    \fi
    \expandafter
  \endgroup
  \childdoctmp
}
%    \end{macrocode}

% \macro{\childdocforwardprefix}
% The command |\childdocforwardprefix| redirects
% compilation to the main or a child file by means of a pattern.
% The prefix |#1| in the current filename is replaced by |#2|
% and the suffix of the current filename is kept
% (it is assumed that the filename does not contain the substring `|~~~|'
% which is used as a delimiter).
% Compilation is handed over to the new file by |\childdocforward|:
%    \begin{macrocode}
\newcommand{\childdocforwardprefix}[3][]
{
  \begingroup
    \def\childdocextract #2##1~~~{\def\childdoctmp{\childdocforward[#1]{#3##1}}}
    \expandafter\childdocextract\childdocname~~~
    \expandafter
  \endgroup
  \childdoctmp
}
%    \end{macrocode}

% \macro{\childdoc}
% The deprecated macro |\childdoc| is a legacy version of |\childdocmain|:
%    \begin{macrocode}
\newcommand{\childdoc}{\childdocmain}
%    \end{macrocode}

% \macro{\childdocredirect}
% The deprecated macro |\childdocredirect| is a legacy version
% of |\childdocforward| and |\childdocforwardprefix|:
%    \begin{macrocode}
\newcommand{\childdocredirect}[2][]
{
  \begingroup
    \if?#1?
      \def\childdoctmp{\childdocforward{#2}}
    \else
      \def\childdoctmp{\childdocforwardprefix{#1}{#2}}
    \fi
    \expandafter
  \endgroup
  \childdoctmp
}
%    \end{macrocode}

%\iffalse
%</package>
%\fi
%
\endinput
\childdocforward[cdocsamp]{cdocsch1}"|\\
% |latex -jobname cdocscl2 \|\\
% |  "\def\version{final}% \iffalse
%
% childdoc.dtx Copyright (C) 2017-2018 Niklas Beisert
%
% This work may be distributed and/or modified under the
% conditions of the LaTeX Project Public License, either version 1.3
% of this license or (at your option) any later version.
% The latest version of this license is in
%   http://www.latex-project.org/lppl.txt
% and version 1.3 or later is part of all distributions of LaTeX
% version 2005/12/01 or later.
%
% This work has the LPPL maintenance status `maintained'.
%
% The Current Maintainer of this work is Niklas Beisert.
%
% This work consists of the files childdoc.dtx and childdoc.ins
% and the derived files childdoc.def and cdocsamp.tex with
% cdocsch1.tex, cdocsch2.tex, cdocsdrf.tex, cdocsfn1.tex, cdocsfn2.tex.
%
%<package>\ifdefined\childdocmain\endinput\fi
%<package>\ProvidesFile{childdoc.def}[2018/12/30 v2.0 child document driver]
%<samplemain>\ProvidesFile{cdocsamp.tex}[2018/12/30 v2.0 sample for childdoc]
%<*driver>
%\ProvidesFile{childdoc.drv}[2018/12/30 v2.0 childdoc reference manual file]
\PassOptionsToClass{10pt,a4paper}{article}
\documentclass{ltxdoc}

\usepackage[margin=35mm]{geometry}
\usepackage{hyperref}
\usepackage{hyperxmp}
\usepackage[usenames]{color}

\hypersetup{colorlinks=true}
\hypersetup{pdfstartview=FitH}
\hypersetup{pdfpagemode=UseNone}
\hypersetup{pdfsource={}}
\hypersetup{pdflang={en-UK}}
\hypersetup{pdfcopyright={Copyright 2017-2018 Niklas Beisert.
  This work may be distributed and/or modified under the
  conditions of the LaTeX Project Public License, either version 1.3
  of this license or (at your option) any later version.}}
\hypersetup{pdflicenseurl={http://www.latex-project.org/lppl.txt}}
\hypersetup{pdfcontactaddress={ETH Zurich, ITP, HIT K,
  Wolfgang-Pauli-Strasse 27}}
\hypersetup{pdfcontactpostcode={8093}}
\hypersetup{pdfcontactcity={Zurich}}
\hypersetup{pdfcontactcountry={Switzerland}}
\hypersetup{pdfcontactemail={nbeisert@itp.phys.ethz.ch}}
\hypersetup{pdfcontacturl={http://people.phys.ethz.ch/\xmptilde nbeisert/}}

\newcommand{\secref}[1]{\hyperref[#1]{section \ref*{#1}}}

\parskip1ex
\parindent0pt
\let\olditemize\itemize
\def\itemize{\olditemize\parskip0pt}

\begin{document}

\title{The \textsf{childdoc} Package}
\hypersetup{pdftitle={The childdoc Package}}
\author{Niklas Beisert\\[2ex]
  Institut f\"ur Theoretische Physik\\
  Eidgen\"ossische Technische Hochschule Z\"urich\\
  Wolfgang-Pauli-Strasse 27, 8093 Z\"urich, Switzerland\\[1ex]
  \href{mailto:nbeisert@itp.phys.ethz.ch}
  {\texttt{nbeisert@itp.phys.ethz.ch}}}
\hypersetup{pdfauthor={Niklas Beisert}}
\hypersetup{pdfsubject={Manual for the LaTeX2e Package childdoc}}
\date{30 December 2018, \textsf{v2.0}}
\maketitle

\begin{abstract}\noindent
\textsf{childdoc} is a \LaTeXe{} package
that enables the direct compilation
of document sections included by |\include|
to individual files.
\end{abstract}

\begingroup
\parskip0ex
\tableofcontents
\endgroup

%%%%%%%%%%%%%%%%%%%%%%%%%%%%%%%%%%%%%%%%%%%%%%%%%%%%%%%%%%%%%%%%%%%%%%%%%%%%%%%%
%%%%%%%%%%%%%%%%%%%%%%%%%%%%%%%%%%%%%%%%%%%%%%%%%%%%%%%%%%%%%%%%%%%%%%%%%%%%%%%%
\section{Introduction}

\LaTeX{} provides a mechanism to structure a large document (such as a book)
into a main file and several child files (containing the chapters)
using the |\include| command.
This mechanism is beneficial for documents
which span hundreds of pages in order to
make the source file(s) more manageable.
Moreover, compilation can be restricted to
selected child files by means of the |\includeonly| command.
The latter feature can be used to reduce the compilation time while editing
(this was significantly more useful in the earlier days of \LaTeX{})
or to generate a smaller document which is easier to navigate.
Another application of |\includeonly| is to generate
documents consisting of selected parts of the complete document.

However, there are a few drawbacks of the plain |\include| mechanism:
\begin{itemize}
\item
The child files cannot be compiled on their own,
they can only be compiled via the main file.
A naive editing environment
(such as a text editor with an option
to have the current file processed by \LaTeX)
may require one to switch to the main file before compiling;
attempting to compile the child file produces errors.
\item
The main file must be modified (each time)
to adjust the |\includeonly| command
to the present needs. This easily leaves the main file in a messy state.
\item
The generated document will always carry the filename
of the main document. This is inconvenient if
several child files are to be compiled and
to be kept for distribution.
\end{itemize}

The present package provides a simple interface
to make child files individually compilable by \LaTeX{}.
Compiling a child file then has the same effect as compiling
the main file with an |\includeonly| command
to select the appropriate child.
Moreover the generated document will carry the name of the child
rather than the main file.
This resolves all three above issues.

This feature is meant to make the editing of books,
thesis documents and lecture notes somewhat more convenient.
However, the package can also be used efficiently for
composing a series of documents (such as exercise sheets)
which are typically distributed individually.
It then assists the author in generating the individual documents
(potentially in different versions)
as well as a document containing the collected series.
Another application is in developing style files
or other kinds of included material
where compilation of the style file could redirect
to a sample or test file.

%%%%%%%%%%%%%%%%%%%%%%%%%%%%%%%%%%%%%%%%%%%%%%%%%%%%%%%%%%%%%%%%%%%%%%%%%%%%%%%%
%%%%%%%%%%%%%%%%%%%%%%%%%%%%%%%%%%%%%%%%%%%%%%%%%%%%%%%%%%%%%%%%%%%%%%%%%%%%%%%%
\section{Usage}

First of all, the package \textsf{childdoc} is \emph{not} a standard
\LaTeXe{} |.sty| style file! Therefore it needs to be invoked in
a non-standard way.

%%%%%%%%%%%%%%%%%%%%%%%%%%%%%%%%%%%%%%%%%%%%%%%%%%%%%%%%%%%%%%%%%%%%%%%%%%%%%%%%
\subsection{Included Files}
\label{sec:include}

%%%%%%%%%%%%%%%%%%%%%%%%%%%%%%%%%%%%%%%%
\DescribeMacro{\childdocmain}
To use the package, add the commands
\begin{center}
\begin{tabular}{l}
|% \iffalse
%
% childdoc.dtx Copyright (C) 2017-2018 Niklas Beisert
%
% This work may be distributed and/or modified under the
% conditions of the LaTeX Project Public License, either version 1.3
% of this license or (at your option) any later version.
% The latest version of this license is in
%   http://www.latex-project.org/lppl.txt
% and version 1.3 or later is part of all distributions of LaTeX
% version 2005/12/01 or later.
%
% This work has the LPPL maintenance status `maintained'.
%
% The Current Maintainer of this work is Niklas Beisert.
%
% This work consists of the files childdoc.dtx and childdoc.ins
% and the derived files childdoc.def and cdocsamp.tex with
% cdocsch1.tex, cdocsch2.tex, cdocsdrf.tex, cdocsfn1.tex, cdocsfn2.tex.
%
%<package>\ifdefined\childdocmain\endinput\fi
%<package>\ProvidesFile{childdoc.def}[2018/12/30 v2.0 child document driver]
%<samplemain>\ProvidesFile{cdocsamp.tex}[2018/12/30 v2.0 sample for childdoc]
%<*driver>
%\ProvidesFile{childdoc.drv}[2018/12/30 v2.0 childdoc reference manual file]
\PassOptionsToClass{10pt,a4paper}{article}
\documentclass{ltxdoc}

\usepackage[margin=35mm]{geometry}
\usepackage{hyperref}
\usepackage{hyperxmp}
\usepackage[usenames]{color}

\hypersetup{colorlinks=true}
\hypersetup{pdfstartview=FitH}
\hypersetup{pdfpagemode=UseNone}
\hypersetup{pdfsource={}}
\hypersetup{pdflang={en-UK}}
\hypersetup{pdfcopyright={Copyright 2017-2018 Niklas Beisert.
  This work may be distributed and/or modified under the
  conditions of the LaTeX Project Public License, either version 1.3
  of this license or (at your option) any later version.}}
\hypersetup{pdflicenseurl={http://www.latex-project.org/lppl.txt}}
\hypersetup{pdfcontactaddress={ETH Zurich, ITP, HIT K,
  Wolfgang-Pauli-Strasse 27}}
\hypersetup{pdfcontactpostcode={8093}}
\hypersetup{pdfcontactcity={Zurich}}
\hypersetup{pdfcontactcountry={Switzerland}}
\hypersetup{pdfcontactemail={nbeisert@itp.phys.ethz.ch}}
\hypersetup{pdfcontacturl={http://people.phys.ethz.ch/\xmptilde nbeisert/}}

\newcommand{\secref}[1]{\hyperref[#1]{section \ref*{#1}}}

\parskip1ex
\parindent0pt
\let\olditemize\itemize
\def\itemize{\olditemize\parskip0pt}

\begin{document}

\title{The \textsf{childdoc} Package}
\hypersetup{pdftitle={The childdoc Package}}
\author{Niklas Beisert\\[2ex]
  Institut f\"ur Theoretische Physik\\
  Eidgen\"ossische Technische Hochschule Z\"urich\\
  Wolfgang-Pauli-Strasse 27, 8093 Z\"urich, Switzerland\\[1ex]
  \href{mailto:nbeisert@itp.phys.ethz.ch}
  {\texttt{nbeisert@itp.phys.ethz.ch}}}
\hypersetup{pdfauthor={Niklas Beisert}}
\hypersetup{pdfsubject={Manual for the LaTeX2e Package childdoc}}
\date{30 December 2018, \textsf{v2.0}}
\maketitle

\begin{abstract}\noindent
\textsf{childdoc} is a \LaTeXe{} package
that enables the direct compilation
of document sections included by |\include|
to individual files.
\end{abstract}

\begingroup
\parskip0ex
\tableofcontents
\endgroup

%%%%%%%%%%%%%%%%%%%%%%%%%%%%%%%%%%%%%%%%%%%%%%%%%%%%%%%%%%%%%%%%%%%%%%%%%%%%%%%%
%%%%%%%%%%%%%%%%%%%%%%%%%%%%%%%%%%%%%%%%%%%%%%%%%%%%%%%%%%%%%%%%%%%%%%%%%%%%%%%%
\section{Introduction}

\LaTeX{} provides a mechanism to structure a large document (such as a book)
into a main file and several child files (containing the chapters)
using the |\include| command.
This mechanism is beneficial for documents
which span hundreds of pages in order to
make the source file(s) more manageable.
Moreover, compilation can be restricted to
selected child files by means of the |\includeonly| command.
The latter feature can be used to reduce the compilation time while editing
(this was significantly more useful in the earlier days of \LaTeX{})
or to generate a smaller document which is easier to navigate.
Another application of |\includeonly| is to generate
documents consisting of selected parts of the complete document.

However, there are a few drawbacks of the plain |\include| mechanism:
\begin{itemize}
\item
The child files cannot be compiled on their own,
they can only be compiled via the main file.
A naive editing environment
(such as a text editor with an option
to have the current file processed by \LaTeX)
may require one to switch to the main file before compiling;
attempting to compile the child file produces errors.
\item
The main file must be modified (each time)
to adjust the |\includeonly| command
to the present needs. This easily leaves the main file in a messy state.
\item
The generated document will always carry the filename
of the main document. This is inconvenient if
several child files are to be compiled and
to be kept for distribution.
\end{itemize}

The present package provides a simple interface
to make child files individually compilable by \LaTeX{}.
Compiling a child file then has the same effect as compiling
the main file with an |\includeonly| command
to select the appropriate child.
Moreover the generated document will carry the name of the child
rather than the main file.
This resolves all three above issues.

This feature is meant to make the editing of books,
thesis documents and lecture notes somewhat more convenient.
However, the package can also be used efficiently for
composing a series of documents (such as exercise sheets)
which are typically distributed individually.
It then assists the author in generating the individual documents
(potentially in different versions)
as well as a document containing the collected series.
Another application is in developing style files
or other kinds of included material
where compilation of the style file could redirect
to a sample or test file.

%%%%%%%%%%%%%%%%%%%%%%%%%%%%%%%%%%%%%%%%%%%%%%%%%%%%%%%%%%%%%%%%%%%%%%%%%%%%%%%%
%%%%%%%%%%%%%%%%%%%%%%%%%%%%%%%%%%%%%%%%%%%%%%%%%%%%%%%%%%%%%%%%%%%%%%%%%%%%%%%%
\section{Usage}

First of all, the package \textsf{childdoc} is \emph{not} a standard
\LaTeXe{} |.sty| style file! Therefore it needs to be invoked in
a non-standard way.

%%%%%%%%%%%%%%%%%%%%%%%%%%%%%%%%%%%%%%%%%%%%%%%%%%%%%%%%%%%%%%%%%%%%%%%%%%%%%%%%
\subsection{Included Files}
\label{sec:include}

%%%%%%%%%%%%%%%%%%%%%%%%%%%%%%%%%%%%%%%%
\DescribeMacro{\childdocmain}
To use the package, add the commands
\begin{center}
\begin{tabular}{l}
|\input{childdoc.def}|\\
|\childdocmain{}|\\
\end{tabular}
\end{center}
at the very top of the main \LaTeX{} file,
in particular \emph{before} the |\documentclass| statement!
The argument of |\childdocmain| should be left empty
(but it must be present).

%%%%%%%%%%%%%%%%%%%%%%%%%%%%%%%%%%%%%%%%
\DescribeMacro{\childdocof}
Furthermore, add the commands
\begin{center}
\begin{tabular}{l}
|\input{childdoc.def}|\\
|\childdocof{|\textit{main}|}|\\
\end{tabular}
\end{center}
at the top of every child file \textit{child}
which is included by |\include{|\textit{child}|}|
from within the main file
(or at least for those files to be compiled individually).
The argument \textit{main} must be the filename of the main file.

There are a couple of
considerations in setting up the main and child documents:

%%%%%%%%%%%%%%%%%%%%%%%%%%%%%%%%%%%%%%%%
\paragraph{Restrictions.}

Please note the following restrictions:
\begin{itemize}
\item
|\childdocmain| must be called with one argument \textit{main}
to ensure compatibility with earlier version of the package.
It must either be empty (|\childdocmain{}|)
or precisely match the filename of the main file in which it is specified.
See \secref{sec:detection} for further information.
\item
The filename \textit{main} must be specified without the |.tex| extension.
\item
The filename \textit{main} is case sensitive
(even in case-insensitive file systems)
due to internal string comparison.
\item
The argument \textit{main} should be fully expanded, it cannot be a macro.
\item
Subdirectories and special characters should be avoided in filenames.
\item
The command |\childdocmain{|\textit{main}|}| must be followed by a whitespace.
It should not be followed immediately by another command
or by a comment mark `|%|'.
This is because the \TeX{} parser reads the token immediately following
the argument of |\childdocmain| and puts it
at the beginning of every child section;
however, a white\-space is ignored.
\end{itemize}

%%%%%%%%%%%%%%%%%%%%%%%%%%%%%%%%%%%%%%%%
\paragraph{Content of Main File.}

It is advisable to place all content in the child files included by |\include|.
Any output contained in the main file will appear in all child documents
unless suppressed manually;
it cannot be suppressed automatically by the |\includeonly| directive
and thus should normally be avoided.
A method to include some content in the main file
by means of conditional processing is described in \secref{sec:conditional}.

%%%%%%%%%%%%%%%%%%%%%%%%%%%%%%%%%%%%%%%%
\paragraph{Page Numbering.}

When only a part of the document is compiled,
the appropriate numbering of pages
(as well as other status parameters)
is determined from the |.aux| files.
The latter contain information from previous passes.
However this information needs to propagate through
all intermediate child documents.
Therefore the page numbering in child documents may well
be inconsistent until the complete document is compiled at least once.

A useful (if unconventional) way to always ensure a consistent
page numbering is to restart the numbering in each child document
and denote the pages by `\textit{child}|.|\textit{page}'
where \textit{child} represents the chapter/section number of the child file.
This can be achieved by the command
|\numberwithin{page}{|\textit{child}|}|
of the \textsf{amsmath} package
where \textit{child} can be |chapter| or |section|
depending on the chosen structuring.
Alternatively, one can modify the macro |\thepage| appropriately
and reset the counter |page| at the start of each child file.

%%%%%%%%%%%%%%%%%%%%%%%%%%%%%%%%%%%%%%%%%%%%%%%%%%%%%%%%%%%%%%%%%%%%%%%%%%%%%%%%
\subsection{Conditional Processing}
\label{sec:conditional}

The package provides a mechanism to compile different versions
of a document. To customise the versions further some conditional processing
can come in handy to distinguish which version is being compiled.
The package provides two macros to describe the compilation context:

%%%%%%%%%%%%%%%%%%%%%%%%%%%%%%%%%%%%%%%%
\DescribeMacro{\ifchilddoc}
The conditional |\ifchilddoc| distinguishes between the compilation of
child documents and the main document:
%
\begin{center}
|\ifchilddoc |\textit{child-code}| |[|\||else |\textit{main-code}]| \||fi|
\end{center}

%%%%%%%%%%%%%%%%%%%%%%%%%%%%%%%%%%%%%%%%
\DescribeMacro{\childdocname}
\DescribeMacro{\childdocjob}
The macro |\childdocname| contains the filename (without extension)
of the main or child file being processed.
Note that |\childdocjob| will always contain the name of the main file.

%%%%%%%%%%%%%%%%%%%%%%%%%%%%%%%%%%%%%%%%
\paragraph{Title Page.}

Conditional processing can be used to include a title or banner page
in the main document when proper precautions are taken.
Importantly, the code in the main file should ensure that the page counter
(as well as other status parameters which are stored in the |.aux| files)
takes the same value after the conditional processing.
Otherwise the page numbers may take divergent values
depending on which part is compiled.

For example, a title page could be declared by:
%
\begin{center}
\begin{tabular}{l}
|\ifchilddoc\||else|\\
|\addtocounter{page}{-1}|\\
\textit{code for title page}\\
|\newpage|\\
|\||fi|
\end{tabular}
\end{center}
%
A banner page for the child documents can be generated by:
%
\begin{center}
\begin{tabular}{l}
|\ifchilddoc|\\
|\addtocounter{page}{-1}|\\
\textit{code for banner page}\\
|\newpage|\\
|\||fi|
\end{tabular}
\end{center}
%
Here one could write a message such as:
\begin{center}
|This is the part \childdocname{} of \childdocjob{}.|
\end{center}

%%%%%%%%%%%%%%%%%%%%%%%%%%%%%%%%%%%%%%%%%%%%%%%%%%%%%%%%%%%%%%%%%%%%%%%%%%%%%%%%
\subsection{Flags}
\label{sec:flags}

The package makes it easy to generate different versions
of the main or child documents.
To this end compilation flags can be defined
and assigned different default values.
They will be particularly useful in conjunction
with the forwarding mechanism described in \secref{sec:forward}.

For example, it may be useful to have a flag |\version|
which can be set to |draft| or |final|.
The document source will contain some conditional code
depending on the value of |\version|.
Suppose further, the flag should default to |final| for the main file
and to |draft| for child files
which is a natural assignment for editing the document.
This is achieved by placing the following code
in the preamble of the main document
(below the |\childdocmain| directive):
%
\begin{center}
\begin{tabular}{l}
|\ifchilddoc|\\
|\providecommand{\version}{draft}|\\
|\||else|\\
|\providecommand{\version}{final}|\\
|\||fi|
\end{tabular}
\end{center}
%
The definition by |\providecommand| makes sure
that previous definitions are not overwritten.
Further statements |\providecommand{\version}{...}|
can thus be added before the above code to override it.

For the main file, one might add a line
(between |\childdocmain| and the above block)
%
\begin{center}
|%\ifchilddoc\||else\providecommand{\version}{draft}\||fi|
\end{center}
%
which can be uncommented to produce a draft version.
Likewise one can add a line to the very top of a child file
(above the |\childdocof{|\textit{main}|}| directive)
%
\begin{center}
|%\providecommand{\version}{final}|
\end{center}
%
which can be uncommented to produce the final version of this child document.

%%%%%%%%%%%%%%%%%%%%%%%%%%%%%%%%%%%%%%%%%%%%%%%%%%%%%%%%%%%%%%%%%%%%%%%%%%%%%%%%
\subsection{Forwarding}
\label{sec:forward}

Different versions of the main or child documents
using compilation flags as described in \secref{sec:flags}
can be (permanently) stored in different files
for convenient compilation, viewing and distribution.
To this end, the package defines a command
to pass on compilation to a different file:

%%%%%%%%%%%%%%%%%%%%%%%%%%%%%%%%%%%%%%%%
\DescribeMacro{\childdocforward}
The command |\childdocforward| redirects processing to
another source file:
%
\begin{center}
\begin{tabular}{l}
|\input{childdoc.def}|\\
|\childdocforward[|\textit{main}|]{|\textit{dest}|}|\\
\end{tabular}
\end{center}
%
The argument \textit{dest} is the destination file
(without extension).
It should be the main file or one of the child files.
Note that further \textsf{childdoc} directives
such as |\childdocof| and |\childdocforward|
in the indicated file will be processed in this form.
The optional argument \textit{main}
passes on directly to the main file \textit{main}
while pretending to compile the child \textit{dest}.
This form behaves as if \textit{dest}
issues |\childdocof{|\textit{main}|}| right away,
and no further \textsf{childdoc} directives will be processed.

%%%%%%%%%%%%%%%%%%%%%%%%%%%%%%%%%%%%%%%%
\DescribeMacro{\...prefix}
In the alternative form |\childdocforwardprefix|,
%
\begin{center}
\begin{tabular}{l}
|\input{childdoc.def}|\\
|\childdocforwardprefix[|\textit{main}|]{|\textit{prefix}|}{|\textit{dest}|}|
\end{tabular}
\end{center}
%
the destination file is determined by a pattern
depending on the current file:
To make this work, the current file must be called
`{\textit{prefix}\hspace{0.2em}\textit{suffix}}'
with \textit{prefix} matching precisely the argument.
Processing is then passed on to the file
`{\textit{dest}\hspace{0.2em}\textit{suffix}}'.
Surely, the same effect is achieved by
directly specifying the
argument `{\textit{dest}\hspace{0.2em}\textit{suffix}}'
in the first form.
However, that requires to set up a different file
for each child. With the alternative form of the command
all these files can have exactly the same content
which simplifies setting them up and maintaining them.

For example, the following file |draft.tex|
with a compilation flag |\version| as described in \secref{sec:flags}
compiles the main document as a draft:
%
\begin{center}
\begin{tabular}{l}
|\def\version{draft}|\\
|\input{childdoc.def}|\\
|\childdocforward{|\textit{main}|}|
\end{tabular}
\end{center}
%
Likewise, the following files |final|\textit{nn}|.tex|
compile the final version of the child document
|child|\textit{nn}|.tex|:
%
\begin{center}
\begin{tabular}{l}
|\def\version{final}|\\
|\input{childdoc.def}|\\
|\childdocforwardprefix{final}{child}|
\end{tabular}
\end{center}
%

Note that when several versions of a main file and/or of each child file
are to be generated, it may be convenient to set up a |Makefile| or
shell script to automatise the process.

%%%%%%%%%%%%%%%%%%%%%%%%%%%%%%%%%%%%%%%%%%%%%%%%%%%%%%%%%%%%%%%%%%%%%%%%%%%%%%%%
\subsection{Command Line Processing}
\label{sec:commandline}

The effect of redirection files can also be achieved by invoking
the \LaTeX{} compiler with a more elaborate command line.
Most conveniently this should be done as part
of a shell script or a |Makefile|.

When using \textsf{childdoc} in the main file, the following
command lines effectively perform a redirection
(note that depending on the shell being used,
backslashes may have to be doubled: `|\|' $\to$ `|\\|'):
%
\begin{center}
|... -jobname "|\textit{target}|" |\\|"|[\textit{flags}]%
|\input{childdoc.def}\childdocforward[|\textit{main}|]{|\textit{dest}|}"|
\end{center}
%
Here \textit{target} is the name of the output file,
\textit{main} is the name of the main file
and \textit{dest} is the name of the main or child file to be processed
(all filenames without extensions).
The optional argument \textit{main} can be omitted
if \textit{main} matches \textit{dest}.
Optionally, compilation \textit{flags} can be defined via |\def| commands.
This command line makes the \TeX{} engine believe
it is compiling the file \textit{target}
whose content is specified as the latter parameter.
The provided code then forwards the processing to
\textit{main} or \textit{dest} as described in \secref{sec:forward}.

%%%%%%%%%%%%%%%%%%%%%%%%%%%%%%%%%%%%%%%%%%%%%%%%%%%%%%%%%%%%%%%%%%%%%%%%%%%%%%%%
\subsection{Include by Input}
\label{sec:input}

Including child documents by |\include| has some restrictions by design.
Most notably, the content of a child document always occupies
its own set of pages; pages cannot be shared between child documents.
Usually, this behaviour makes perfect sense
because each child document contain an essential part of the document.
However, in some situations it may be desirable to compose
a document from a collection of parts
without having mandatory page breaks between then.
For this case, the package
provides a mechanism to include parts
by |\input| which can also be processed individually.
However, by construction this mechanism
requires manual handling of the content to be output.

%%%%%%%%%%%%%%%%%%%%%%%%%%%%%%%%%%%%%%%%
\DescribeMacro{\ifchilddocmanual}
The main file should be prepared as usual, see \secref{sec:include}.
However, the document body must make a distinction
between processing of an individual part and of the main document, e.g.:
%
\begin{center}
\begin{tabular}{l}
|\ifchilddocmanual|\\
|\input{\childdocname}|\\
|\||else|\\
\textit{document body with }|\input{|\textit{part}|}|\\
|\||fi|
\end{tabular}
\end{center}
%
The conditional |\ifchilddocmanual| is true whenever
a part to be included by |\input| is being compiled,
and the name of the part is stored in |\childdocname|.

%%%%%%%%%%%%%%%%%%%%%%%%%%%%%%%%%%%%%%%%
\DescribeMacro{\childdocby}
Each part to be included by |\input| should start with:
%
\begin{center}
\begin{tabular}{l}
|\input{childdoc.def}|\\
|\childdocby{|\textit{main}|}|\\
\end{tabular}
\end{center}
%
The directive |\childdocby| is similar to |\childdocof|
described in \secref{sec:include},
but the subsequent selection of content must be done manually.
To that end, both |\ifchilddoc| and |\ifchilddocmanual|
will be true upon processing of a part,
and the name of the part is stored in |\childdocname|.
Note that |\jobname| will be set to the filename of the current part
so that each part receives an individual |.aux| file
that does not interfere with the |.aux| file(s) of the main document.
This behaviour can be altered by the alternative form
|\childdocby[*]{|\textit{main}|}| (with a non-empty optional argument)
which uses the |.aux| file of the main document
by setting |\jobname| to \textit{main}.

%%%%%%%%%%%%%%%%%%%%%%%%%%%%%%%%%%%%%%%%%%%%%%%%%%%%%%%%%%%%%%%%%%%%%%%%%%%%%%%%
\subsection{Driver Development}
\label{sec:driver}

The \textsf{childdoc} mechanism can also be use for the development
of definition files such as \LaTeX{} styles or classes.
This case differs from the above setup with multiple parts
included by |\include| in that no |\includeonly| should be invoked.
This can be achieved by starting the include file
(before |\ProvidesPackage|) with:
%
\begin{center}
\begin{tabular}{l}
|\input{childdoc.def}|\\
|\childdocforward{|\textit{main}|}|\\
\end{tabular}
\end{center}
%
or alternatively with:
%
\begin{center}
\begin{tabular}{l}
|\input{childdoc.def}|\\
|\childdocby{|\textit{main}|}|\\
\end{tabular}
\end{center}
%
Both forms have slightly different effects as described above.
The main file is prepared as usual, see \secref{sec:include}.

%%%%%%%%%%%%%%%%%%%%%%%%%%%%%%%%%%%%%%%%%%%%%%%%%%%%%%%%%%%%%%%%%%%%%%%%%%%%%%%%
\subsection{Legacy Detection}
\label{sec:detection}

The directive |\childdocmain| in the main file can detect
whether the complete document or merely a child is to be compiled
even without using the directive |\childdocof|.
This method is deprecated because it is less robust
and there is no compelling reason to use it;
it is merely provided for backward compatibility
and it may be removed in future versions.

If the detection mechanism is to be used,
it is mandatory to correctly specify
the filename of the main file as the argument of |\childdocmain|:
%
\begin{center}
\begin{tabular}{l}
|\input{childdoc.def}|\\
|\childdocmain{|\textit{main}|}|\\
\end{tabular}
\end{center}
%
If |\jobname| does not match the argument \textit{main} of |\childdocmain|,
it is assumed that |\jobname| points to the child file to be compiled.
When using |\childdocmain| with the main file specified as argument,
it suffices to start a child file
with just |\input{|\textit{main}|}|
without loading of the package and using |\childdocof|.
If instead all processing is done
with the appropriate \textsf{childdoc} directives,
the argument of \textit{main} of |\childdocmain| can be empty.

An alternative version of the command line processing described
in \secref{sec:commandline} using the detection mechanism reads:
%
\begin{center}
|... -jobname "|\textit{target}|" "|[\textit{flags}]%
[|\def\jobname{|\textit{dest}|}|]|\input{|\textit{main}|}"|
\end{center}

%%%%%%%%%%%%%%%%%%%%%%%%%%%%%%%%%%%%%%%%%%%%%%%%%%%%%%%%%%%%%%%%%%%%%%%%%%%%%%%%
\subsection{Manual Code}
\label{sec:manual}

In case one cannot be certain whether the definitions file |childdoc.def|
is installed on the target \TeX{} distribution
and one prefers not to ship it,
it is conceivable to paste a few relevant commands into the sources.

To that end, drop all statements |\input{childdoc.def}|
and perform the replacements as outlined below.
Instead of |\childdocmain{|\textit{main}|}| add the following code
to the top of the main file:
%
\begin{center}
\begin{tabular}{l}
|\||ifdefined\childdocname\endinput\||fi\newif\ifchilddoc|\\
|\edef\childdocname{\scantokens\expandafter{\jobname\noexpand}}|\\
|\def\childdocmain{|\textit{main}|}\||ifx\childdocmain\childdocname\||else|\\
|\childdoctrue\includeonly{\childdocname}\let\jobname\childdocmain\||fi|\\
\end{tabular}
\end{center}
%
Instead of |\childdocof{|\textit{main}|}| just include the main file
at the top of each child file:
%
\begin{center}
|\input{|\textit{main}|}|
\end{center}
%
A simple redirection |\childdocforward{|\textit{dest}|}| is achieved by:
%
\begin{center}
|\def\jobname{|\textit{dest}|}\input{\jobname}|
\end{center}
%
The redirection with prefix
|\childdocforwardprefix[|\textit{prefix}|]{|\textit{dest}|}|
is accomplished by:
%
\begin{center}
\begin{tabular}{l}
|{\edef\jobname{\scantokens\expandafter{\jobname\noexpand}}|\\
|\def\redirectjob |\textit{prefix}|#1~~~{\gdef\jobname{|\textit{dest}|#1}}|\\
|\expandafter\redirectjob\jobname~~~}\input{\jobname}|
\end{tabular}
\end{center}

In an alternative approach,
child documents can be compiled by a specific command line
without additional code or specific definitions:
%
\begin{center}
|... -jobname "|\textit{target}|" "|[\textit{flags}]%
|\includeonly{|\textit{dest}|}\input{|\textit{main}|}"|
\end{center}
%

%%%%%%%%%%%%%%%%%%%%%%%%%%%%%%%%%%%%%%%%%%%%%%%%%%%%%%%%%%%%%%%%%%%%%%%%%%%%%%%%
%%%%%%%%%%%%%%%%%%%%%%%%%%%%%%%%%%%%%%%%%%%%%%%%%%%%%%%%%%%%%%%%%%%%%%%%%%%%%%%%
\section{Information}

%%%%%%%%%%%%%%%%%%%%%%%%%%%%%%%%%%%%%%%%%%%%%%%%%%%%%%%%%%%%%%%%%%%%%%%%%%%%%%%%
\subsection{Copyright}

Copyright \copyright{} 2017--2018 Niklas Beisert

This work may be distributed and/or modified under the
conditions of the \LaTeX{} Project Public License, either version 1.3
of this license or (at your option) any later version.
The latest version of this license is in
  \url{http://www.latex-project.org/lppl.txt}
and version 1.3 or later is part of all distributions of \LaTeX{}
version 2005/12/01 or later.

This work has the LPPL maintenance status `maintained'.

The Current Maintainer of this work is Niklas Beisert.

This work consists of the files |README.txt|, |childdoc.ins| and |childdoc.dtx|
as well as the derived files |childdoc.def|, |cdocsamp.tex|
with |cdocsch1.tex|, |cdocsch2.tex|, |cdocspt3.tex|, |cdocspt4.tex|,
|cdocsdrf.tex|, |cdocsfn1.tex|, |cdocsfn2.tex|
as well as |childdoc.pdf|.

%%%%%%%%%%%%%%%%%%%%%%%%%%%%%%%%%%%%%%%%%%%%%%%%%%%%%%%%%%%%%%%%%%%%%%%%%%%%%%%%
\subsection{Files and Installation}

The package consists of the files:
%
\begin{center}
\begin{tabular}{ll}
    |README.txt|   & readme file \\
    |childdoc.ins| & installation file \\
    |childdoc.dtx| & source file \\
    |childdoc.def| & definition file \\
    |cdocsamp.tex| & sample main file \\
    |cdocsch1.tex| & sample include file \\
    |cdocsch2.tex| & sample include file \\
    |cdocspt3.tex| & sample part file \\
    |cdocspt4.tex| & sample part file \\
    |cdocsdrf.tex| & sample redirection file \\
    |cdocsfn1.tex| & sample redirection file \\
    |cdocsfn2.tex| & sample redirection file \\
    |childdoc.pdf| & manual
\end{tabular}
\end{center}
%
The distribution consists of the files
|README.txt|, |childdoc.ins| and |childdoc.dtx|.
%
\begin{itemize}
\item
Run (pdf)\LaTeX{} on |childdoc.dtx|
to compile the manual |childdoc.pdf| (this file).
\item
Run \LaTeX{} on |childdoc.ins| to create the definitions file |childdoc.def|
and the sample |cdocsamp.tex| with include files
|cdocsch1.tex|, |cdocsch2.tex|, |cdocspt3.tex|, |cdocspt4.tex|,
|cdocsdrf.tex|, |cdocsfn1.tex|, |cdocsfn2.tex|.
Then copy the file |childdoc.def| to an appropriate directory of your \LaTeX{}
distribution, e.g.\ \textit{texmf-root}|/tex/latex/childdoc|.
\end{itemize}

%%%%%%%%%%%%%%%%%%%%%%%%%%%%%%%%%%%%%%%%%%%%%%%%%%%%%%%%%%%%%%%%%%%%%%%%%%%%%%%%
\subsection{Related CTAN Packages}

There are several other packages which offer a similar functionality:
%
\begin{itemize}
\item
The packages
\href{http://ctan.org/pkg/docmute}{\textsf{docmute}},
\href{http://ctan.org/pkg/includex}{\textsf{includex}} and
\href{http://ctan.org/pkg/standalone}{\textsf{standalone}}
provide commands to include only the document body of
a child file thus allowing both files to be compiled individually.
\item
The packages \href{http://ctan.org/pkg/subdocs}{\textsf{subdocs}}
and \href{http://ctan.org/pkg/subfiles}{\textsf{subfiles}}
provide structures in which the main and child documents can be
encapsulated and allowing them to be compiled individually.
The inclusion mechanism is different from the conventional |\include|.
\item
The package \href{http://ctan.org/pkg/combine}{\textsf{combine}}
is an elaborate solution to combine several documents into one.
\end{itemize}
%
See also the CTAN topic \href{http://ctan.org/topic/subdocs}{\textsf{subdocs}}
for further related packages.
The present package differs from the above solutions in that
a document structure constructed with the conventional |\include| mechanism
just needs two extra commands at the top of every file
such that all constituent files can be compiled individually.

%%%%%%%%%%%%%%%%%%%%%%%%%%%%%%%%%%%%%%%%%%%%%%%%%%%%%%%%%%%%%%%%%%%%%%%%%%%%%%%%
%\subsection{Feature Suggestions}
%
%The following is a list of features which may be useful for future
%versions of this package:
%%
%\begin{itemize}
%\item
%\ldots
%\end{itemize}

%%%%%%%%%%%%%%%%%%%%%%%%%%%%%%%%%%%%%%%%%%%%%%%%%%%%%%%%%%%%%%%%%%%%%%%%%%%%%%%%
\subsection{Revision History}

%%%%%%%%%%%%%%%%%%%%%%%%%%%%%%%%%%%%%%%%
\paragraph{v2.0:} 2018/12/30

\begin{itemize}
\item
immediate forward processing
\item
added |\childdocby| mechanism
\item
manual restructured
\end{itemize}

%%%%%%%%%%%%%%%%%%%%%%%%%%%%%%%%%%%%%%%%
\paragraph{v1.6:} 2018/01/17

\begin{itemize}
\item
application for development of include files
\item
corrections to manual
\end{itemize}

%%%%%%%%%%%%%%%%%%%%%%%%%%%%%%%%%%%%%%%%
\paragraph{v1.5:} 2017/05/21

\begin{itemize}
\item
more complete structuring introduced
\item
|\childdocof| introduced
\item
|\childdoc| renamed to |\childdocmain|
\item
|\childredirect| renamed to |\childdocforward| and |\childdocforwardprefix|
and functionality expanded
\end{itemize}

%%%%%%%%%%%%%%%%%%%%%%%%%%%%%%%%%%%%%%%%
\paragraph{v1.0:} 2017/04/27

\begin{itemize}
\item
manual and install package
\item
first version published on CTAN
\end{itemize}

%%%%%%%%%%%%%%%%%%%%%%%%%%%%%%%%%%%%%%%%
\paragraph{v0.6:} 2017/04/26

\begin{itemize}
\item
redirection mechanism added
\end{itemize}

%%%%%%%%%%%%%%%%%%%%%%%%%%%%%%%%%%%%%%%%
\paragraph{v0.5:} 2017/04/26

\begin{itemize}
\item
functionality in definition file
\end{itemize}


%%%%%%%%%%%%%%%%%%%%%%%%%%%%%%%%%%%%%%%%%%%%%%%%%%%%%%%%%%%%%%%%%%%%%%%%%%%%%%%%
%%%%%%%%%%%%%%%%%%%%%%%%%%%%%%%%%%%%%%%%%%%%%%%%%%%%%%%%%%%%%%%%%%%%%%%%%%%%%%%%
%%%%%%%%%%%%%%%%%%%%%%%%%%%%%%%%%%%%%%%%%%%%%%%%%%%%%%%%%%%%%%%%%%%%%%%%%%%%%%%%
\appendix

\settowidth\MacroIndent{\rmfamily\scriptsize 000\ }

 \DocInput{childdoc.dtx}

\end{document}
%</driver>
% \fi
%
% %%%%%%%%%%%%%%%%%%%%%%%%%%%%%%%%%%%%%%%%%%%%%%%%%%%%%%%%%%%%%%%%%%%%%%%%%%%%%%
% %%%%%%%%%%%%%%%%%%%%%%%%%%%%%%%%%%%%%%%%%%%%%%%%%%%%%%%%%%%%%%%%%%%%%%%%%%%%%%
% \section{Sample}
%\iffalse
%<*samplemain>
%\fi
%
% The following presents a sample document
% with two chapters, two parts, a title page,
% a compile flag as well as three forwarding files to set the flag.
% It consists of eight |.tex| files:
% \begin{center}
% \begin{tabular}{ll}
% |cdocsamp.tex|&main file\\
% |cdocsch1.tex|&include file for chapter 1\\
% |cdocsch2.tex|&include file for chapter 2\\
% |cdocspt3.tex|&include file for part 3\\
% |cdocspt4.tex|&include file for part 4\\
% |cdocsdrf.tex|&forwarding file for main file in draft mode\\
% |cdocsfi1.tex|&forwarding file for final version of chapter 1\\
% |cdocsfi2.tex|&forwarding file for final version of chapter 2\\
% \end{tabular}
% \end{center}
% Each of the eight files can be compiled directly by the \LaTeX{} compiler.
%
% %%%%%%%%%%%%%%%%%%%%%%%%%%%%%%%%%%%%%%
% \paragraph{Main File.}
%
% The main file is called |cdocsamp.tex|.
%
% Load the \textsf{childdoc} definitions and
% declare the filename for the main document:
%    \begin{macrocode}
\input{childdoc.def}
\childdocmain{}
%    \end{macrocode}

% Optional override for |\version| flag:
%    \begin{macrocode}
%%\ifchilddoc\else\providecommand{\version}{draft}\fi
%    \end{macrocode}

% Define the default values for the |\version| flag
% (|final| for the main file and |draft| for childs):
%    \begin{macrocode}
\ifchilddoc
\providecommand{\version}{draft}
\else
\providecommand{\version}{final}
\fi
%    \end{macrocode}

% Load the standard document class:
%    \begin{macrocode}
\documentclass[12pt]{article}
%    \end{macrocode}

% Start the document body:
%    \begin{macrocode}
\begin{document}
%    \end{macrocode}

% Declare a title page.
% Print title, part of document being processed and version flag:
%    \begin{macrocode}
\addtocounter{page}{-1}
\begin{center}
{\LARGE\bfseries{}childdoc example\par}
\vspace{1cm}
\ifchilddoc
\ifchilddocmanual part\else chapter\fi:
`\childdocname' of `\childdocjob'\par
\else
main document: `\childdocjob'\par
\fi
version: \version\par
\end{center}
\newpage
%    \end{macrocode}

% Manually include selected file,
% otherwise process as usual:
%    \begin{macrocode}
\ifchilddocmanual
\section*{part `\childdocname'}
\input{\childdocname}
\else
%    \end{macrocode}

% Include the two chapters:
%    \begin{macrocode}
\include{cdocsch1}
\include{cdocsch2}
%    \end{macrocode}

% Include the two parts unless only chapters should be displayed:
%    \begin{macrocode}
\ifchilddoc\else
\section{part three}
\input{cdocspt3}
\section{part four}
\input{cdocspt4}
\fi
%    \end{macrocode}

% Process as usual until here:
%    \begin{macrocode}
\fi
%    \end{macrocode}

% End of document body:
%    \begin{macrocode}
\end{document}
%    \end{macrocode}
%\iffalse
%</samplemain>
%\fi
%
% %%%%%%%%%%%%%%%%%%%%%%%%%%%%%%%%%%%%%%
% \paragraph{Chapter Include Files.}
%
% The include files are called |cdocsch1.tex| and |cdocsch2.tex|.
%
%\iffalse
%<*samplechap1|samplechap2>
%\fi

% Optional override for |\version| flag:
%    \begin{macrocode}
%%\providecommand{\version}{final}
%    \end{macrocode}

% Include the main document:
%    \begin{macrocode}
\input{childdoc.def}
\childdocof{cdocsamp}
%    \end{macrocode}

%\iffalse
%</samplechap1|samplechap2>
%\fi
%
%\iffalse
%<*samplechap1>
%\fi
% Some text for chapter 1:
%    \begin{macrocode}
\section{one}
some text in chapter one
%    \end{macrocode}

%\iffalse
%</samplechap1>
%\fi
% Some text for chapter 2:
%\iffalse
%<*samplechap2>
%\fi
%    \begin{macrocode}
\section{two}
more text in chapter two
%    \end{macrocode}

%\iffalse
%</samplechap2>
%\fi
%
% %%%%%%%%%%%%%%%%%%%%%%%%%%%%%%%%%%%%%%
% \paragraph{Part Include Files.}
%
% The include files are called |cdocspt3.tex| and |cdocspt4.tex|.
%
%\iffalse
%<*samplepart3|samplepart4>
%\fi

% Optional override for |\version| flag:
%    \begin{macrocode}
%%\providecommand{\version}{final}
%    \end{macrocode}

% Include the main document:
%    \begin{macrocode}
\input{childdoc.def}
\childdocby{cdocsamp}
%    \end{macrocode}

%\iffalse
%</samplepart3|samplepart4>
%\fi
%
%\iffalse
%<*samplepart3>
%\fi
% Some text for part 3:
%    \begin{macrocode}
some text in part three
%    \end{macrocode}

%\iffalse
%</samplepart3>
%\fi
% Some text for part 4:
%\iffalse
%<*samplepart4>
%\fi
%    \begin{macrocode}
more text in part four
%    \end{macrocode}

%\iffalse
%</samplepart4>
%\fi
%
% %%%%%%%%%%%%%%%%%%%%%%%%%%%%%%%%%%%%%%
% \paragraph{Forwarding for a Complete Draft.}
%
% The following forwarding file |cdocsdrf.tex|
% compiles the main document in draft mode:
%\iffalse
%<*sampledraft>
%\fi
%    \begin{macrocode}
\def\version{draft}
\input{childdoc.def}
\childdocforward{cdocsamp}
%    \end{macrocode}

%\iffalse
%</sampledraft>
%\fi
%
% %%%%%%%%%%%%%%%%%%%%%%%%%%%%%%%%%%%%%%
% \paragraph{Forwarding for Final Version of the Chapters.}
%
% The following forwarding files |cdocsfn1.tex| and |cdocsfn2.tex|
% (with identical content)
% compile the final versions of the child documents
% |cdocsch1.tex| and |cdocsch2.tex|, respectively:
%\iffalse
%<*samplefinal>
%\fi
%    \begin{macrocode}
\def\version{final}
\input{childdoc.def}
\childdocforwardprefix[cdocsamp]{cdocsfn}{cdocsch}
%    \end{macrocode}

%\iffalse
%</samplefinal>
%\fi
%
% %%%%%%%%%%%%%%%%%%%%%%%%%%%%%%%%%%%%%%
% \paragraph{Command Line Processing.}
%
% The following three command lines generate the output files
% |cdocscld|, |cdocscl1| and |cdocscl2|
% which should be identical to
% |cdocsdrf|, |cdocsch1| and |cdocsfn2|, respectively:
% \begin{center}
% \begin{tabular}{l}
% |latex -jobname cdocscld \|\\
% |  "\def\version{draft}\input{childdoc.def}\childdocforward{cdocsamp}"|\\
% |latex -jobname cdocscl1 \|\\
% |  "\input{childdoc.def}\childdocforward[cdocsamp]{cdocsch1}"|\\
% |latex -jobname cdocscl2 \|\\
% |  "\def\version{final}\input{childdoc.def}\childdocforward{cdocsch2}"|
% \end{tabular}
% \end{center}
% Note that the trailing backslash on each first line
% merely continues the input to the second line
% (for convenient cut ant paste).
% Furthermore, the command |latex| can be replaced by any
% of its alternative versions such as |pdflatex|.
%
% %%%%%%%%%%%%%%%%%%%%%%%%%%%%%%%%%%%%%%%%%%%%%%%%%%%%%%%%%%%%%%%%%%%%%%%%%%%%%%
% %%%%%%%%%%%%%%%%%%%%%%%%%%%%%%%%%%%%%%%%%%%%%%%%%%%%%%%%%%%%%%%%%%%%%%%%%%%%%%
% \section{Implementation}
%\iffalse
%<*package>
%\fi
%
% This section describes the definitions file |childdoc.def|.

% The definitions cannot be loaded using |\usepackage| or |\RequirePackage|
% which has a mechanism to prevent loading a style file more than once.
% When loading the definitions by means of |\input|
% multiple instances have to be prevented manually:
%\iffalse
%This code needs to be before the `\ProvidesFile' directive
%which is defined at the beginning of this file.
%Therefore it is also placed there and commented out here.
%</package>
%<*discard>
%\fi
%    \begin{macrocode}
\ifdefined\childdocmain\endinput\fi
%    \end{macrocode}
%\iffalse
%</discard>
%<*package>
%\fi
%
% \macro{\ifchilddoc}
% \macro{\ifchilddocmanual}
% The conditional |\ifchilddoc| tells whether a
% child (true) or main (false) document is being compiled.
% The conditional |\ifchilddocmanual| tells whether
% the |\includeonly| mechanism is used (false) or
% the selection of child files must be performed manually (true).
% The definitions initialise to false:
%    \begin{macrocode}
\newif\ifchilddoc
\newif\ifchilddocmanual
%    \end{macrocode}

% \macro{\childdocname}
% \macro{\childdocjob}
% The macro |\childdocname| stores the name of the main document
% to be compiled. The macro |\childdocjob| stores the name of
% the document on which the \LaTeX{} compiler was originally invoked.
% The content of |\jobname| cannot be compared
% to filenames specified in the source due to different catcodes.
% The following code rescans |\jobname|, stores the result
% in |\childdocname| and saves a copy in |\childdocjob|:
%    \begin{macrocode}
\edef\childdocname{\scantokens\expandafter{\jobname\noexpand}}
\let\childdocjob\childdocname
%    \end{macrocode}

% \macro{\childdocdisable}
% The macro |\childdocdisable| prevents the main file
% from being processed more than once.
% At this stage, the main document command |\childdocmain|
% is assumed to be called once again where it should do nothing.
% Any subsequent call to it should prevent
% a secondary processing of the main document
% It overwrites the forwarding commands
% |\childdocof| and |\childdocforward|
% with empty macros to prevent further inclusions of the main document:
%    \begin{macrocode}
\newcommand{\childdocdisable}
{
  \renewcommand{\childdocmain}[1]{\renewcommand{\childdocmain}[1]{\endinput}}
  \renewcommand{\childdocof}[1]{}
  \renewcommand{\childdocby}[2][]{}
  \renewcommand{\childdocforward}[2][]{}
  \renewcommand{\childdocdisable}{}
}
%    \end{macrocode}

% \macro{\childdocmain}
% The macro |\childdocmain| is to be called at the top of the main file
% with nothing or the main filename (without extension) as argument.
% First, it breaks loops.
% If the argument is not empty and does not match |\childdocname|
% (which is set by the first inclusion of |childdoc.def|),
% |\ifchilddoc| is set to true, |\includeonly| is applied to the child file
% and |\jobname| is set to the main file
% (for proper handling of |.aux| files):
%    \begin{macrocode}
\newcommand{\childdocmain}[1]
{
  \childdocdisable\childdocmain{}
  \if?#1?\else
    \begingroup
      \def\childdoctmp{#1}
      \ifx\childdoctmp\childdocname
        \def\childdoctmp{}
      \else
        \def\childdoctmp
        {
          \childdoctrue
          \includeonly{\childdocname}
          \def\childdocjob{#1}
          \def\jobname{#1}
        }
      \fi
      \expandafter
    \endgroup
    \childdoctmp
  \fi
}
%    \end{macrocode}

% \macro{\childdocof}
% The command |\childdocof| redirects
% compilation to the main file |#1|.
%    \begin{macrocode}
\newcommand{\childdocof}[1]
{
  \childdocdisable
  \childdoctrue
  \includeonly{\childdocname}
  \def\jobname{#1}
  \def\childdocjob{#1}
  \input{#1}
}
%    \end{macrocode}

% \macro{\childdocby}
% The command |\childdocby| ....
%    \begin{macrocode}
\newcommand{\childdocby}[2][]
{
  \childdocdisable
  \childdoctrue
  \childdocmanualtrue
  \if?#1?\else
    \def\jobname{#2}
  \fi
  \def\childdocjob{#2}
  \input{#2}
  \endinput
}
%    \end{macrocode}

% \macro{\childdocforward}
% The command |\childdocforward| redirects
% compilation to the main file or
% (if the optional argument is given) a child file.
% Parameters are set as if the main file
% or a child file starting with |\childdocof| was compiled.
% Then compilation is handed over to the main file:
%    \begin{macrocode}
\newcommand{\childdocforward}[2][]
{
  \begingroup
    \if?#1?
      \def\childdoctmp
      {
        \def\childdocname{#2}
        \def\childdocjob{#2}
        \def\jobname{#2}
        \input{#2}
        \endinput
      }
    \else
      \def\childdoctmp
      {
        \childdocdisable
        \def\childdocname{#2}
        \childdoctrue
        \includeonly{#2}
        \def\childdocjob{#1}
        \def\jobname{#1}
        \input{#1}
        \endinput
      }
    \fi
    \expandafter
  \endgroup
  \childdoctmp
}
%    \end{macrocode}

% \macro{\childdocforwardprefix}
% The command |\childdocforwardprefix| redirects
% compilation to the main or a child file by means of a pattern.
% The prefix |#1| in the current filename is replaced by |#2|
% and the suffix of the current filename is kept
% (it is assumed that the filename does not contain the substring `|~~~|'
% which is used as a delimiter).
% Compilation is handed over to the new file by |\childdocforward|:
%    \begin{macrocode}
\newcommand{\childdocforwardprefix}[3][]
{
  \begingroup
    \def\childdocextract #2##1~~~{\def\childdoctmp{\childdocforward[#1]{#3##1}}}
    \expandafter\childdocextract\childdocname~~~
    \expandafter
  \endgroup
  \childdoctmp
}
%    \end{macrocode}

% \macro{\childdoc}
% The deprecated macro |\childdoc| is a legacy version of |\childdocmain|:
%    \begin{macrocode}
\newcommand{\childdoc}{\childdocmain}
%    \end{macrocode}

% \macro{\childdocredirect}
% The deprecated macro |\childdocredirect| is a legacy version
% of |\childdocforward| and |\childdocforwardprefix|:
%    \begin{macrocode}
\newcommand{\childdocredirect}[2][]
{
  \begingroup
    \if?#1?
      \def\childdoctmp{\childdocforward{#2}}
    \else
      \def\childdoctmp{\childdocforwardprefix{#1}{#2}}
    \fi
    \expandafter
  \endgroup
  \childdoctmp
}
%    \end{macrocode}

%\iffalse
%</package>
%\fi
%
\endinput
|\\
|\childdocmain{}|\\
\end{tabular}
\end{center}
at the very top of the main \LaTeX{} file,
in particular \emph{before} the |\documentclass| statement!
The argument of |\childdocmain| should be left empty
(but it must be present).

%%%%%%%%%%%%%%%%%%%%%%%%%%%%%%%%%%%%%%%%
\DescribeMacro{\childdocof}
Furthermore, add the commands
\begin{center}
\begin{tabular}{l}
|% \iffalse
%
% childdoc.dtx Copyright (C) 2017-2018 Niklas Beisert
%
% This work may be distributed and/or modified under the
% conditions of the LaTeX Project Public License, either version 1.3
% of this license or (at your option) any later version.
% The latest version of this license is in
%   http://www.latex-project.org/lppl.txt
% and version 1.3 or later is part of all distributions of LaTeX
% version 2005/12/01 or later.
%
% This work has the LPPL maintenance status `maintained'.
%
% The Current Maintainer of this work is Niklas Beisert.
%
% This work consists of the files childdoc.dtx and childdoc.ins
% and the derived files childdoc.def and cdocsamp.tex with
% cdocsch1.tex, cdocsch2.tex, cdocsdrf.tex, cdocsfn1.tex, cdocsfn2.tex.
%
%<package>\ifdefined\childdocmain\endinput\fi
%<package>\ProvidesFile{childdoc.def}[2018/12/30 v2.0 child document driver]
%<samplemain>\ProvidesFile{cdocsamp.tex}[2018/12/30 v2.0 sample for childdoc]
%<*driver>
%\ProvidesFile{childdoc.drv}[2018/12/30 v2.0 childdoc reference manual file]
\PassOptionsToClass{10pt,a4paper}{article}
\documentclass{ltxdoc}

\usepackage[margin=35mm]{geometry}
\usepackage{hyperref}
\usepackage{hyperxmp}
\usepackage[usenames]{color}

\hypersetup{colorlinks=true}
\hypersetup{pdfstartview=FitH}
\hypersetup{pdfpagemode=UseNone}
\hypersetup{pdfsource={}}
\hypersetup{pdflang={en-UK}}
\hypersetup{pdfcopyright={Copyright 2017-2018 Niklas Beisert.
  This work may be distributed and/or modified under the
  conditions of the LaTeX Project Public License, either version 1.3
  of this license or (at your option) any later version.}}
\hypersetup{pdflicenseurl={http://www.latex-project.org/lppl.txt}}
\hypersetup{pdfcontactaddress={ETH Zurich, ITP, HIT K,
  Wolfgang-Pauli-Strasse 27}}
\hypersetup{pdfcontactpostcode={8093}}
\hypersetup{pdfcontactcity={Zurich}}
\hypersetup{pdfcontactcountry={Switzerland}}
\hypersetup{pdfcontactemail={nbeisert@itp.phys.ethz.ch}}
\hypersetup{pdfcontacturl={http://people.phys.ethz.ch/\xmptilde nbeisert/}}

\newcommand{\secref}[1]{\hyperref[#1]{section \ref*{#1}}}

\parskip1ex
\parindent0pt
\let\olditemize\itemize
\def\itemize{\olditemize\parskip0pt}

\begin{document}

\title{The \textsf{childdoc} Package}
\hypersetup{pdftitle={The childdoc Package}}
\author{Niklas Beisert\\[2ex]
  Institut f\"ur Theoretische Physik\\
  Eidgen\"ossische Technische Hochschule Z\"urich\\
  Wolfgang-Pauli-Strasse 27, 8093 Z\"urich, Switzerland\\[1ex]
  \href{mailto:nbeisert@itp.phys.ethz.ch}
  {\texttt{nbeisert@itp.phys.ethz.ch}}}
\hypersetup{pdfauthor={Niklas Beisert}}
\hypersetup{pdfsubject={Manual for the LaTeX2e Package childdoc}}
\date{30 December 2018, \textsf{v2.0}}
\maketitle

\begin{abstract}\noindent
\textsf{childdoc} is a \LaTeXe{} package
that enables the direct compilation
of document sections included by |\include|
to individual files.
\end{abstract}

\begingroup
\parskip0ex
\tableofcontents
\endgroup

%%%%%%%%%%%%%%%%%%%%%%%%%%%%%%%%%%%%%%%%%%%%%%%%%%%%%%%%%%%%%%%%%%%%%%%%%%%%%%%%
%%%%%%%%%%%%%%%%%%%%%%%%%%%%%%%%%%%%%%%%%%%%%%%%%%%%%%%%%%%%%%%%%%%%%%%%%%%%%%%%
\section{Introduction}

\LaTeX{} provides a mechanism to structure a large document (such as a book)
into a main file and several child files (containing the chapters)
using the |\include| command.
This mechanism is beneficial for documents
which span hundreds of pages in order to
make the source file(s) more manageable.
Moreover, compilation can be restricted to
selected child files by means of the |\includeonly| command.
The latter feature can be used to reduce the compilation time while editing
(this was significantly more useful in the earlier days of \LaTeX{})
or to generate a smaller document which is easier to navigate.
Another application of |\includeonly| is to generate
documents consisting of selected parts of the complete document.

However, there are a few drawbacks of the plain |\include| mechanism:
\begin{itemize}
\item
The child files cannot be compiled on their own,
they can only be compiled via the main file.
A naive editing environment
(such as a text editor with an option
to have the current file processed by \LaTeX)
may require one to switch to the main file before compiling;
attempting to compile the child file produces errors.
\item
The main file must be modified (each time)
to adjust the |\includeonly| command
to the present needs. This easily leaves the main file in a messy state.
\item
The generated document will always carry the filename
of the main document. This is inconvenient if
several child files are to be compiled and
to be kept for distribution.
\end{itemize}

The present package provides a simple interface
to make child files individually compilable by \LaTeX{}.
Compiling a child file then has the same effect as compiling
the main file with an |\includeonly| command
to select the appropriate child.
Moreover the generated document will carry the name of the child
rather than the main file.
This resolves all three above issues.

This feature is meant to make the editing of books,
thesis documents and lecture notes somewhat more convenient.
However, the package can also be used efficiently for
composing a series of documents (such as exercise sheets)
which are typically distributed individually.
It then assists the author in generating the individual documents
(potentially in different versions)
as well as a document containing the collected series.
Another application is in developing style files
or other kinds of included material
where compilation of the style file could redirect
to a sample or test file.

%%%%%%%%%%%%%%%%%%%%%%%%%%%%%%%%%%%%%%%%%%%%%%%%%%%%%%%%%%%%%%%%%%%%%%%%%%%%%%%%
%%%%%%%%%%%%%%%%%%%%%%%%%%%%%%%%%%%%%%%%%%%%%%%%%%%%%%%%%%%%%%%%%%%%%%%%%%%%%%%%
\section{Usage}

First of all, the package \textsf{childdoc} is \emph{not} a standard
\LaTeXe{} |.sty| style file! Therefore it needs to be invoked in
a non-standard way.

%%%%%%%%%%%%%%%%%%%%%%%%%%%%%%%%%%%%%%%%%%%%%%%%%%%%%%%%%%%%%%%%%%%%%%%%%%%%%%%%
\subsection{Included Files}
\label{sec:include}

%%%%%%%%%%%%%%%%%%%%%%%%%%%%%%%%%%%%%%%%
\DescribeMacro{\childdocmain}
To use the package, add the commands
\begin{center}
\begin{tabular}{l}
|\input{childdoc.def}|\\
|\childdocmain{}|\\
\end{tabular}
\end{center}
at the very top of the main \LaTeX{} file,
in particular \emph{before} the |\documentclass| statement!
The argument of |\childdocmain| should be left empty
(but it must be present).

%%%%%%%%%%%%%%%%%%%%%%%%%%%%%%%%%%%%%%%%
\DescribeMacro{\childdocof}
Furthermore, add the commands
\begin{center}
\begin{tabular}{l}
|\input{childdoc.def}|\\
|\childdocof{|\textit{main}|}|\\
\end{tabular}
\end{center}
at the top of every child file \textit{child}
which is included by |\include{|\textit{child}|}|
from within the main file
(or at least for those files to be compiled individually).
The argument \textit{main} must be the filename of the main file.

There are a couple of
considerations in setting up the main and child documents:

%%%%%%%%%%%%%%%%%%%%%%%%%%%%%%%%%%%%%%%%
\paragraph{Restrictions.}

Please note the following restrictions:
\begin{itemize}
\item
|\childdocmain| must be called with one argument \textit{main}
to ensure compatibility with earlier version of the package.
It must either be empty (|\childdocmain{}|)
or precisely match the filename of the main file in which it is specified.
See \secref{sec:detection} for further information.
\item
The filename \textit{main} must be specified without the |.tex| extension.
\item
The filename \textit{main} is case sensitive
(even in case-insensitive file systems)
due to internal string comparison.
\item
The argument \textit{main} should be fully expanded, it cannot be a macro.
\item
Subdirectories and special characters should be avoided in filenames.
\item
The command |\childdocmain{|\textit{main}|}| must be followed by a whitespace.
It should not be followed immediately by another command
or by a comment mark `|%|'.
This is because the \TeX{} parser reads the token immediately following
the argument of |\childdocmain| and puts it
at the beginning of every child section;
however, a white\-space is ignored.
\end{itemize}

%%%%%%%%%%%%%%%%%%%%%%%%%%%%%%%%%%%%%%%%
\paragraph{Content of Main File.}

It is advisable to place all content in the child files included by |\include|.
Any output contained in the main file will appear in all child documents
unless suppressed manually;
it cannot be suppressed automatically by the |\includeonly| directive
and thus should normally be avoided.
A method to include some content in the main file
by means of conditional processing is described in \secref{sec:conditional}.

%%%%%%%%%%%%%%%%%%%%%%%%%%%%%%%%%%%%%%%%
\paragraph{Page Numbering.}

When only a part of the document is compiled,
the appropriate numbering of pages
(as well as other status parameters)
is determined from the |.aux| files.
The latter contain information from previous passes.
However this information needs to propagate through
all intermediate child documents.
Therefore the page numbering in child documents may well
be inconsistent until the complete document is compiled at least once.

A useful (if unconventional) way to always ensure a consistent
page numbering is to restart the numbering in each child document
and denote the pages by `\textit{child}|.|\textit{page}'
where \textit{child} represents the chapter/section number of the child file.
This can be achieved by the command
|\numberwithin{page}{|\textit{child}|}|
of the \textsf{amsmath} package
where \textit{child} can be |chapter| or |section|
depending on the chosen structuring.
Alternatively, one can modify the macro |\thepage| appropriately
and reset the counter |page| at the start of each child file.

%%%%%%%%%%%%%%%%%%%%%%%%%%%%%%%%%%%%%%%%%%%%%%%%%%%%%%%%%%%%%%%%%%%%%%%%%%%%%%%%
\subsection{Conditional Processing}
\label{sec:conditional}

The package provides a mechanism to compile different versions
of a document. To customise the versions further some conditional processing
can come in handy to distinguish which version is being compiled.
The package provides two macros to describe the compilation context:

%%%%%%%%%%%%%%%%%%%%%%%%%%%%%%%%%%%%%%%%
\DescribeMacro{\ifchilddoc}
The conditional |\ifchilddoc| distinguishes between the compilation of
child documents and the main document:
%
\begin{center}
|\ifchilddoc |\textit{child-code}| |[|\||else |\textit{main-code}]| \||fi|
\end{center}

%%%%%%%%%%%%%%%%%%%%%%%%%%%%%%%%%%%%%%%%
\DescribeMacro{\childdocname}
\DescribeMacro{\childdocjob}
The macro |\childdocname| contains the filename (without extension)
of the main or child file being processed.
Note that |\childdocjob| will always contain the name of the main file.

%%%%%%%%%%%%%%%%%%%%%%%%%%%%%%%%%%%%%%%%
\paragraph{Title Page.}

Conditional processing can be used to include a title or banner page
in the main document when proper precautions are taken.
Importantly, the code in the main file should ensure that the page counter
(as well as other status parameters which are stored in the |.aux| files)
takes the same value after the conditional processing.
Otherwise the page numbers may take divergent values
depending on which part is compiled.

For example, a title page could be declared by:
%
\begin{center}
\begin{tabular}{l}
|\ifchilddoc\||else|\\
|\addtocounter{page}{-1}|\\
\textit{code for title page}\\
|\newpage|\\
|\||fi|
\end{tabular}
\end{center}
%
A banner page for the child documents can be generated by:
%
\begin{center}
\begin{tabular}{l}
|\ifchilddoc|\\
|\addtocounter{page}{-1}|\\
\textit{code for banner page}\\
|\newpage|\\
|\||fi|
\end{tabular}
\end{center}
%
Here one could write a message such as:
\begin{center}
|This is the part \childdocname{} of \childdocjob{}.|
\end{center}

%%%%%%%%%%%%%%%%%%%%%%%%%%%%%%%%%%%%%%%%%%%%%%%%%%%%%%%%%%%%%%%%%%%%%%%%%%%%%%%%
\subsection{Flags}
\label{sec:flags}

The package makes it easy to generate different versions
of the main or child documents.
To this end compilation flags can be defined
and assigned different default values.
They will be particularly useful in conjunction
with the forwarding mechanism described in \secref{sec:forward}.

For example, it may be useful to have a flag |\version|
which can be set to |draft| or |final|.
The document source will contain some conditional code
depending on the value of |\version|.
Suppose further, the flag should default to |final| for the main file
and to |draft| for child files
which is a natural assignment for editing the document.
This is achieved by placing the following code
in the preamble of the main document
(below the |\childdocmain| directive):
%
\begin{center}
\begin{tabular}{l}
|\ifchilddoc|\\
|\providecommand{\version}{draft}|\\
|\||else|\\
|\providecommand{\version}{final}|\\
|\||fi|
\end{tabular}
\end{center}
%
The definition by |\providecommand| makes sure
that previous definitions are not overwritten.
Further statements |\providecommand{\version}{...}|
can thus be added before the above code to override it.

For the main file, one might add a line
(between |\childdocmain| and the above block)
%
\begin{center}
|%\ifchilddoc\||else\providecommand{\version}{draft}\||fi|
\end{center}
%
which can be uncommented to produce a draft version.
Likewise one can add a line to the very top of a child file
(above the |\childdocof{|\textit{main}|}| directive)
%
\begin{center}
|%\providecommand{\version}{final}|
\end{center}
%
which can be uncommented to produce the final version of this child document.

%%%%%%%%%%%%%%%%%%%%%%%%%%%%%%%%%%%%%%%%%%%%%%%%%%%%%%%%%%%%%%%%%%%%%%%%%%%%%%%%
\subsection{Forwarding}
\label{sec:forward}

Different versions of the main or child documents
using compilation flags as described in \secref{sec:flags}
can be (permanently) stored in different files
for convenient compilation, viewing and distribution.
To this end, the package defines a command
to pass on compilation to a different file:

%%%%%%%%%%%%%%%%%%%%%%%%%%%%%%%%%%%%%%%%
\DescribeMacro{\childdocforward}
The command |\childdocforward| redirects processing to
another source file:
%
\begin{center}
\begin{tabular}{l}
|\input{childdoc.def}|\\
|\childdocforward[|\textit{main}|]{|\textit{dest}|}|\\
\end{tabular}
\end{center}
%
The argument \textit{dest} is the destination file
(without extension).
It should be the main file or one of the child files.
Note that further \textsf{childdoc} directives
such as |\childdocof| and |\childdocforward|
in the indicated file will be processed in this form.
The optional argument \textit{main}
passes on directly to the main file \textit{main}
while pretending to compile the child \textit{dest}.
This form behaves as if \textit{dest}
issues |\childdocof{|\textit{main}|}| right away,
and no further \textsf{childdoc} directives will be processed.

%%%%%%%%%%%%%%%%%%%%%%%%%%%%%%%%%%%%%%%%
\DescribeMacro{\...prefix}
In the alternative form |\childdocforwardprefix|,
%
\begin{center}
\begin{tabular}{l}
|\input{childdoc.def}|\\
|\childdocforwardprefix[|\textit{main}|]{|\textit{prefix}|}{|\textit{dest}|}|
\end{tabular}
\end{center}
%
the destination file is determined by a pattern
depending on the current file:
To make this work, the current file must be called
`{\textit{prefix}\hspace{0.2em}\textit{suffix}}'
with \textit{prefix} matching precisely the argument.
Processing is then passed on to the file
`{\textit{dest}\hspace{0.2em}\textit{suffix}}'.
Surely, the same effect is achieved by
directly specifying the
argument `{\textit{dest}\hspace{0.2em}\textit{suffix}}'
in the first form.
However, that requires to set up a different file
for each child. With the alternative form of the command
all these files can have exactly the same content
which simplifies setting them up and maintaining them.

For example, the following file |draft.tex|
with a compilation flag |\version| as described in \secref{sec:flags}
compiles the main document as a draft:
%
\begin{center}
\begin{tabular}{l}
|\def\version{draft}|\\
|\input{childdoc.def}|\\
|\childdocforward{|\textit{main}|}|
\end{tabular}
\end{center}
%
Likewise, the following files |final|\textit{nn}|.tex|
compile the final version of the child document
|child|\textit{nn}|.tex|:
%
\begin{center}
\begin{tabular}{l}
|\def\version{final}|\\
|\input{childdoc.def}|\\
|\childdocforwardprefix{final}{child}|
\end{tabular}
\end{center}
%

Note that when several versions of a main file and/or of each child file
are to be generated, it may be convenient to set up a |Makefile| or
shell script to automatise the process.

%%%%%%%%%%%%%%%%%%%%%%%%%%%%%%%%%%%%%%%%%%%%%%%%%%%%%%%%%%%%%%%%%%%%%%%%%%%%%%%%
\subsection{Command Line Processing}
\label{sec:commandline}

The effect of redirection files can also be achieved by invoking
the \LaTeX{} compiler with a more elaborate command line.
Most conveniently this should be done as part
of a shell script or a |Makefile|.

When using \textsf{childdoc} in the main file, the following
command lines effectively perform a redirection
(note that depending on the shell being used,
backslashes may have to be doubled: `|\|' $\to$ `|\\|'):
%
\begin{center}
|... -jobname "|\textit{target}|" |\\|"|[\textit{flags}]%
|\input{childdoc.def}\childdocforward[|\textit{main}|]{|\textit{dest}|}"|
\end{center}
%
Here \textit{target} is the name of the output file,
\textit{main} is the name of the main file
and \textit{dest} is the name of the main or child file to be processed
(all filenames without extensions).
The optional argument \textit{main} can be omitted
if \textit{main} matches \textit{dest}.
Optionally, compilation \textit{flags} can be defined via |\def| commands.
This command line makes the \TeX{} engine believe
it is compiling the file \textit{target}
whose content is specified as the latter parameter.
The provided code then forwards the processing to
\textit{main} or \textit{dest} as described in \secref{sec:forward}.

%%%%%%%%%%%%%%%%%%%%%%%%%%%%%%%%%%%%%%%%%%%%%%%%%%%%%%%%%%%%%%%%%%%%%%%%%%%%%%%%
\subsection{Include by Input}
\label{sec:input}

Including child documents by |\include| has some restrictions by design.
Most notably, the content of a child document always occupies
its own set of pages; pages cannot be shared between child documents.
Usually, this behaviour makes perfect sense
because each child document contain an essential part of the document.
However, in some situations it may be desirable to compose
a document from a collection of parts
without having mandatory page breaks between then.
For this case, the package
provides a mechanism to include parts
by |\input| which can also be processed individually.
However, by construction this mechanism
requires manual handling of the content to be output.

%%%%%%%%%%%%%%%%%%%%%%%%%%%%%%%%%%%%%%%%
\DescribeMacro{\ifchilddocmanual}
The main file should be prepared as usual, see \secref{sec:include}.
However, the document body must make a distinction
between processing of an individual part and of the main document, e.g.:
%
\begin{center}
\begin{tabular}{l}
|\ifchilddocmanual|\\
|\input{\childdocname}|\\
|\||else|\\
\textit{document body with }|\input{|\textit{part}|}|\\
|\||fi|
\end{tabular}
\end{center}
%
The conditional |\ifchilddocmanual| is true whenever
a part to be included by |\input| is being compiled,
and the name of the part is stored in |\childdocname|.

%%%%%%%%%%%%%%%%%%%%%%%%%%%%%%%%%%%%%%%%
\DescribeMacro{\childdocby}
Each part to be included by |\input| should start with:
%
\begin{center}
\begin{tabular}{l}
|\input{childdoc.def}|\\
|\childdocby{|\textit{main}|}|\\
\end{tabular}
\end{center}
%
The directive |\childdocby| is similar to |\childdocof|
described in \secref{sec:include},
but the subsequent selection of content must be done manually.
To that end, both |\ifchilddoc| and |\ifchilddocmanual|
will be true upon processing of a part,
and the name of the part is stored in |\childdocname|.
Note that |\jobname| will be set to the filename of the current part
so that each part receives an individual |.aux| file
that does not interfere with the |.aux| file(s) of the main document.
This behaviour can be altered by the alternative form
|\childdocby[*]{|\textit{main}|}| (with a non-empty optional argument)
which uses the |.aux| file of the main document
by setting |\jobname| to \textit{main}.

%%%%%%%%%%%%%%%%%%%%%%%%%%%%%%%%%%%%%%%%%%%%%%%%%%%%%%%%%%%%%%%%%%%%%%%%%%%%%%%%
\subsection{Driver Development}
\label{sec:driver}

The \textsf{childdoc} mechanism can also be use for the development
of definition files such as \LaTeX{} styles or classes.
This case differs from the above setup with multiple parts
included by |\include| in that no |\includeonly| should be invoked.
This can be achieved by starting the include file
(before |\ProvidesPackage|) with:
%
\begin{center}
\begin{tabular}{l}
|\input{childdoc.def}|\\
|\childdocforward{|\textit{main}|}|\\
\end{tabular}
\end{center}
%
or alternatively with:
%
\begin{center}
\begin{tabular}{l}
|\input{childdoc.def}|\\
|\childdocby{|\textit{main}|}|\\
\end{tabular}
\end{center}
%
Both forms have slightly different effects as described above.
The main file is prepared as usual, see \secref{sec:include}.

%%%%%%%%%%%%%%%%%%%%%%%%%%%%%%%%%%%%%%%%%%%%%%%%%%%%%%%%%%%%%%%%%%%%%%%%%%%%%%%%
\subsection{Legacy Detection}
\label{sec:detection}

The directive |\childdocmain| in the main file can detect
whether the complete document or merely a child is to be compiled
even without using the directive |\childdocof|.
This method is deprecated because it is less robust
and there is no compelling reason to use it;
it is merely provided for backward compatibility
and it may be removed in future versions.

If the detection mechanism is to be used,
it is mandatory to correctly specify
the filename of the main file as the argument of |\childdocmain|:
%
\begin{center}
\begin{tabular}{l}
|\input{childdoc.def}|\\
|\childdocmain{|\textit{main}|}|\\
\end{tabular}
\end{center}
%
If |\jobname| does not match the argument \textit{main} of |\childdocmain|,
it is assumed that |\jobname| points to the child file to be compiled.
When using |\childdocmain| with the main file specified as argument,
it suffices to start a child file
with just |\input{|\textit{main}|}|
without loading of the package and using |\childdocof|.
If instead all processing is done
with the appropriate \textsf{childdoc} directives,
the argument of \textit{main} of |\childdocmain| can be empty.

An alternative version of the command line processing described
in \secref{sec:commandline} using the detection mechanism reads:
%
\begin{center}
|... -jobname "|\textit{target}|" "|[\textit{flags}]%
[|\def\jobname{|\textit{dest}|}|]|\input{|\textit{main}|}"|
\end{center}

%%%%%%%%%%%%%%%%%%%%%%%%%%%%%%%%%%%%%%%%%%%%%%%%%%%%%%%%%%%%%%%%%%%%%%%%%%%%%%%%
\subsection{Manual Code}
\label{sec:manual}

In case one cannot be certain whether the definitions file |childdoc.def|
is installed on the target \TeX{} distribution
and one prefers not to ship it,
it is conceivable to paste a few relevant commands into the sources.

To that end, drop all statements |\input{childdoc.def}|
and perform the replacements as outlined below.
Instead of |\childdocmain{|\textit{main}|}| add the following code
to the top of the main file:
%
\begin{center}
\begin{tabular}{l}
|\||ifdefined\childdocname\endinput\||fi\newif\ifchilddoc|\\
|\edef\childdocname{\scantokens\expandafter{\jobname\noexpand}}|\\
|\def\childdocmain{|\textit{main}|}\||ifx\childdocmain\childdocname\||else|\\
|\childdoctrue\includeonly{\childdocname}\let\jobname\childdocmain\||fi|\\
\end{tabular}
\end{center}
%
Instead of |\childdocof{|\textit{main}|}| just include the main file
at the top of each child file:
%
\begin{center}
|\input{|\textit{main}|}|
\end{center}
%
A simple redirection |\childdocforward{|\textit{dest}|}| is achieved by:
%
\begin{center}
|\def\jobname{|\textit{dest}|}\input{\jobname}|
\end{center}
%
The redirection with prefix
|\childdocforwardprefix[|\textit{prefix}|]{|\textit{dest}|}|
is accomplished by:
%
\begin{center}
\begin{tabular}{l}
|{\edef\jobname{\scantokens\expandafter{\jobname\noexpand}}|\\
|\def\redirectjob |\textit{prefix}|#1~~~{\gdef\jobname{|\textit{dest}|#1}}|\\
|\expandafter\redirectjob\jobname~~~}\input{\jobname}|
\end{tabular}
\end{center}

In an alternative approach,
child documents can be compiled by a specific command line
without additional code or specific definitions:
%
\begin{center}
|... -jobname "|\textit{target}|" "|[\textit{flags}]%
|\includeonly{|\textit{dest}|}\input{|\textit{main}|}"|
\end{center}
%

%%%%%%%%%%%%%%%%%%%%%%%%%%%%%%%%%%%%%%%%%%%%%%%%%%%%%%%%%%%%%%%%%%%%%%%%%%%%%%%%
%%%%%%%%%%%%%%%%%%%%%%%%%%%%%%%%%%%%%%%%%%%%%%%%%%%%%%%%%%%%%%%%%%%%%%%%%%%%%%%%
\section{Information}

%%%%%%%%%%%%%%%%%%%%%%%%%%%%%%%%%%%%%%%%%%%%%%%%%%%%%%%%%%%%%%%%%%%%%%%%%%%%%%%%
\subsection{Copyright}

Copyright \copyright{} 2017--2018 Niklas Beisert

This work may be distributed and/or modified under the
conditions of the \LaTeX{} Project Public License, either version 1.3
of this license or (at your option) any later version.
The latest version of this license is in
  \url{http://www.latex-project.org/lppl.txt}
and version 1.3 or later is part of all distributions of \LaTeX{}
version 2005/12/01 or later.

This work has the LPPL maintenance status `maintained'.

The Current Maintainer of this work is Niklas Beisert.

This work consists of the files |README.txt|, |childdoc.ins| and |childdoc.dtx|
as well as the derived files |childdoc.def|, |cdocsamp.tex|
with |cdocsch1.tex|, |cdocsch2.tex|, |cdocspt3.tex|, |cdocspt4.tex|,
|cdocsdrf.tex|, |cdocsfn1.tex|, |cdocsfn2.tex|
as well as |childdoc.pdf|.

%%%%%%%%%%%%%%%%%%%%%%%%%%%%%%%%%%%%%%%%%%%%%%%%%%%%%%%%%%%%%%%%%%%%%%%%%%%%%%%%
\subsection{Files and Installation}

The package consists of the files:
%
\begin{center}
\begin{tabular}{ll}
    |README.txt|   & readme file \\
    |childdoc.ins| & installation file \\
    |childdoc.dtx| & source file \\
    |childdoc.def| & definition file \\
    |cdocsamp.tex| & sample main file \\
    |cdocsch1.tex| & sample include file \\
    |cdocsch2.tex| & sample include file \\
    |cdocspt3.tex| & sample part file \\
    |cdocspt4.tex| & sample part file \\
    |cdocsdrf.tex| & sample redirection file \\
    |cdocsfn1.tex| & sample redirection file \\
    |cdocsfn2.tex| & sample redirection file \\
    |childdoc.pdf| & manual
\end{tabular}
\end{center}
%
The distribution consists of the files
|README.txt|, |childdoc.ins| and |childdoc.dtx|.
%
\begin{itemize}
\item
Run (pdf)\LaTeX{} on |childdoc.dtx|
to compile the manual |childdoc.pdf| (this file).
\item
Run \LaTeX{} on |childdoc.ins| to create the definitions file |childdoc.def|
and the sample |cdocsamp.tex| with include files
|cdocsch1.tex|, |cdocsch2.tex|, |cdocspt3.tex|, |cdocspt4.tex|,
|cdocsdrf.tex|, |cdocsfn1.tex|, |cdocsfn2.tex|.
Then copy the file |childdoc.def| to an appropriate directory of your \LaTeX{}
distribution, e.g.\ \textit{texmf-root}|/tex/latex/childdoc|.
\end{itemize}

%%%%%%%%%%%%%%%%%%%%%%%%%%%%%%%%%%%%%%%%%%%%%%%%%%%%%%%%%%%%%%%%%%%%%%%%%%%%%%%%
\subsection{Related CTAN Packages}

There are several other packages which offer a similar functionality:
%
\begin{itemize}
\item
The packages
\href{http://ctan.org/pkg/docmute}{\textsf{docmute}},
\href{http://ctan.org/pkg/includex}{\textsf{includex}} and
\href{http://ctan.org/pkg/standalone}{\textsf{standalone}}
provide commands to include only the document body of
a child file thus allowing both files to be compiled individually.
\item
The packages \href{http://ctan.org/pkg/subdocs}{\textsf{subdocs}}
and \href{http://ctan.org/pkg/subfiles}{\textsf{subfiles}}
provide structures in which the main and child documents can be
encapsulated and allowing them to be compiled individually.
The inclusion mechanism is different from the conventional |\include|.
\item
The package \href{http://ctan.org/pkg/combine}{\textsf{combine}}
is an elaborate solution to combine several documents into one.
\end{itemize}
%
See also the CTAN topic \href{http://ctan.org/topic/subdocs}{\textsf{subdocs}}
for further related packages.
The present package differs from the above solutions in that
a document structure constructed with the conventional |\include| mechanism
just needs two extra commands at the top of every file
such that all constituent files can be compiled individually.

%%%%%%%%%%%%%%%%%%%%%%%%%%%%%%%%%%%%%%%%%%%%%%%%%%%%%%%%%%%%%%%%%%%%%%%%%%%%%%%%
%\subsection{Feature Suggestions}
%
%The following is a list of features which may be useful for future
%versions of this package:
%%
%\begin{itemize}
%\item
%\ldots
%\end{itemize}

%%%%%%%%%%%%%%%%%%%%%%%%%%%%%%%%%%%%%%%%%%%%%%%%%%%%%%%%%%%%%%%%%%%%%%%%%%%%%%%%
\subsection{Revision History}

%%%%%%%%%%%%%%%%%%%%%%%%%%%%%%%%%%%%%%%%
\paragraph{v2.0:} 2018/12/30

\begin{itemize}
\item
immediate forward processing
\item
added |\childdocby| mechanism
\item
manual restructured
\end{itemize}

%%%%%%%%%%%%%%%%%%%%%%%%%%%%%%%%%%%%%%%%
\paragraph{v1.6:} 2018/01/17

\begin{itemize}
\item
application for development of include files
\item
corrections to manual
\end{itemize}

%%%%%%%%%%%%%%%%%%%%%%%%%%%%%%%%%%%%%%%%
\paragraph{v1.5:} 2017/05/21

\begin{itemize}
\item
more complete structuring introduced
\item
|\childdocof| introduced
\item
|\childdoc| renamed to |\childdocmain|
\item
|\childredirect| renamed to |\childdocforward| and |\childdocforwardprefix|
and functionality expanded
\end{itemize}

%%%%%%%%%%%%%%%%%%%%%%%%%%%%%%%%%%%%%%%%
\paragraph{v1.0:} 2017/04/27

\begin{itemize}
\item
manual and install package
\item
first version published on CTAN
\end{itemize}

%%%%%%%%%%%%%%%%%%%%%%%%%%%%%%%%%%%%%%%%
\paragraph{v0.6:} 2017/04/26

\begin{itemize}
\item
redirection mechanism added
\end{itemize}

%%%%%%%%%%%%%%%%%%%%%%%%%%%%%%%%%%%%%%%%
\paragraph{v0.5:} 2017/04/26

\begin{itemize}
\item
functionality in definition file
\end{itemize}


%%%%%%%%%%%%%%%%%%%%%%%%%%%%%%%%%%%%%%%%%%%%%%%%%%%%%%%%%%%%%%%%%%%%%%%%%%%%%%%%
%%%%%%%%%%%%%%%%%%%%%%%%%%%%%%%%%%%%%%%%%%%%%%%%%%%%%%%%%%%%%%%%%%%%%%%%%%%%%%%%
%%%%%%%%%%%%%%%%%%%%%%%%%%%%%%%%%%%%%%%%%%%%%%%%%%%%%%%%%%%%%%%%%%%%%%%%%%%%%%%%
\appendix

\settowidth\MacroIndent{\rmfamily\scriptsize 000\ }

 \DocInput{childdoc.dtx}

\end{document}
%</driver>
% \fi
%
% %%%%%%%%%%%%%%%%%%%%%%%%%%%%%%%%%%%%%%%%%%%%%%%%%%%%%%%%%%%%%%%%%%%%%%%%%%%%%%
% %%%%%%%%%%%%%%%%%%%%%%%%%%%%%%%%%%%%%%%%%%%%%%%%%%%%%%%%%%%%%%%%%%%%%%%%%%%%%%
% \section{Sample}
%\iffalse
%<*samplemain>
%\fi
%
% The following presents a sample document
% with two chapters, two parts, a title page,
% a compile flag as well as three forwarding files to set the flag.
% It consists of eight |.tex| files:
% \begin{center}
% \begin{tabular}{ll}
% |cdocsamp.tex|&main file\\
% |cdocsch1.tex|&include file for chapter 1\\
% |cdocsch2.tex|&include file for chapter 2\\
% |cdocspt3.tex|&include file for part 3\\
% |cdocspt4.tex|&include file for part 4\\
% |cdocsdrf.tex|&forwarding file for main file in draft mode\\
% |cdocsfi1.tex|&forwarding file for final version of chapter 1\\
% |cdocsfi2.tex|&forwarding file for final version of chapter 2\\
% \end{tabular}
% \end{center}
% Each of the eight files can be compiled directly by the \LaTeX{} compiler.
%
% %%%%%%%%%%%%%%%%%%%%%%%%%%%%%%%%%%%%%%
% \paragraph{Main File.}
%
% The main file is called |cdocsamp.tex|.
%
% Load the \textsf{childdoc} definitions and
% declare the filename for the main document:
%    \begin{macrocode}
\input{childdoc.def}
\childdocmain{}
%    \end{macrocode}

% Optional override for |\version| flag:
%    \begin{macrocode}
%%\ifchilddoc\else\providecommand{\version}{draft}\fi
%    \end{macrocode}

% Define the default values for the |\version| flag
% (|final| for the main file and |draft| for childs):
%    \begin{macrocode}
\ifchilddoc
\providecommand{\version}{draft}
\else
\providecommand{\version}{final}
\fi
%    \end{macrocode}

% Load the standard document class:
%    \begin{macrocode}
\documentclass[12pt]{article}
%    \end{macrocode}

% Start the document body:
%    \begin{macrocode}
\begin{document}
%    \end{macrocode}

% Declare a title page.
% Print title, part of document being processed and version flag:
%    \begin{macrocode}
\addtocounter{page}{-1}
\begin{center}
{\LARGE\bfseries{}childdoc example\par}
\vspace{1cm}
\ifchilddoc
\ifchilddocmanual part\else chapter\fi:
`\childdocname' of `\childdocjob'\par
\else
main document: `\childdocjob'\par
\fi
version: \version\par
\end{center}
\newpage
%    \end{macrocode}

% Manually include selected file,
% otherwise process as usual:
%    \begin{macrocode}
\ifchilddocmanual
\section*{part `\childdocname'}
\input{\childdocname}
\else
%    \end{macrocode}

% Include the two chapters:
%    \begin{macrocode}
\include{cdocsch1}
\include{cdocsch2}
%    \end{macrocode}

% Include the two parts unless only chapters should be displayed:
%    \begin{macrocode}
\ifchilddoc\else
\section{part three}
\input{cdocspt3}
\section{part four}
\input{cdocspt4}
\fi
%    \end{macrocode}

% Process as usual until here:
%    \begin{macrocode}
\fi
%    \end{macrocode}

% End of document body:
%    \begin{macrocode}
\end{document}
%    \end{macrocode}
%\iffalse
%</samplemain>
%\fi
%
% %%%%%%%%%%%%%%%%%%%%%%%%%%%%%%%%%%%%%%
% \paragraph{Chapter Include Files.}
%
% The include files are called |cdocsch1.tex| and |cdocsch2.tex|.
%
%\iffalse
%<*samplechap1|samplechap2>
%\fi

% Optional override for |\version| flag:
%    \begin{macrocode}
%%\providecommand{\version}{final}
%    \end{macrocode}

% Include the main document:
%    \begin{macrocode}
\input{childdoc.def}
\childdocof{cdocsamp}
%    \end{macrocode}

%\iffalse
%</samplechap1|samplechap2>
%\fi
%
%\iffalse
%<*samplechap1>
%\fi
% Some text for chapter 1:
%    \begin{macrocode}
\section{one}
some text in chapter one
%    \end{macrocode}

%\iffalse
%</samplechap1>
%\fi
% Some text for chapter 2:
%\iffalse
%<*samplechap2>
%\fi
%    \begin{macrocode}
\section{two}
more text in chapter two
%    \end{macrocode}

%\iffalse
%</samplechap2>
%\fi
%
% %%%%%%%%%%%%%%%%%%%%%%%%%%%%%%%%%%%%%%
% \paragraph{Part Include Files.}
%
% The include files are called |cdocspt3.tex| and |cdocspt4.tex|.
%
%\iffalse
%<*samplepart3|samplepart4>
%\fi

% Optional override for |\version| flag:
%    \begin{macrocode}
%%\providecommand{\version}{final}
%    \end{macrocode}

% Include the main document:
%    \begin{macrocode}
\input{childdoc.def}
\childdocby{cdocsamp}
%    \end{macrocode}

%\iffalse
%</samplepart3|samplepart4>
%\fi
%
%\iffalse
%<*samplepart3>
%\fi
% Some text for part 3:
%    \begin{macrocode}
some text in part three
%    \end{macrocode}

%\iffalse
%</samplepart3>
%\fi
% Some text for part 4:
%\iffalse
%<*samplepart4>
%\fi
%    \begin{macrocode}
more text in part four
%    \end{macrocode}

%\iffalse
%</samplepart4>
%\fi
%
% %%%%%%%%%%%%%%%%%%%%%%%%%%%%%%%%%%%%%%
% \paragraph{Forwarding for a Complete Draft.}
%
% The following forwarding file |cdocsdrf.tex|
% compiles the main document in draft mode:
%\iffalse
%<*sampledraft>
%\fi
%    \begin{macrocode}
\def\version{draft}
\input{childdoc.def}
\childdocforward{cdocsamp}
%    \end{macrocode}

%\iffalse
%</sampledraft>
%\fi
%
% %%%%%%%%%%%%%%%%%%%%%%%%%%%%%%%%%%%%%%
% \paragraph{Forwarding for Final Version of the Chapters.}
%
% The following forwarding files |cdocsfn1.tex| and |cdocsfn2.tex|
% (with identical content)
% compile the final versions of the child documents
% |cdocsch1.tex| and |cdocsch2.tex|, respectively:
%\iffalse
%<*samplefinal>
%\fi
%    \begin{macrocode}
\def\version{final}
\input{childdoc.def}
\childdocforwardprefix[cdocsamp]{cdocsfn}{cdocsch}
%    \end{macrocode}

%\iffalse
%</samplefinal>
%\fi
%
% %%%%%%%%%%%%%%%%%%%%%%%%%%%%%%%%%%%%%%
% \paragraph{Command Line Processing.}
%
% The following three command lines generate the output files
% |cdocscld|, |cdocscl1| and |cdocscl2|
% which should be identical to
% |cdocsdrf|, |cdocsch1| and |cdocsfn2|, respectively:
% \begin{center}
% \begin{tabular}{l}
% |latex -jobname cdocscld \|\\
% |  "\def\version{draft}\input{childdoc.def}\childdocforward{cdocsamp}"|\\
% |latex -jobname cdocscl1 \|\\
% |  "\input{childdoc.def}\childdocforward[cdocsamp]{cdocsch1}"|\\
% |latex -jobname cdocscl2 \|\\
% |  "\def\version{final}\input{childdoc.def}\childdocforward{cdocsch2}"|
% \end{tabular}
% \end{center}
% Note that the trailing backslash on each first line
% merely continues the input to the second line
% (for convenient cut ant paste).
% Furthermore, the command |latex| can be replaced by any
% of its alternative versions such as |pdflatex|.
%
% %%%%%%%%%%%%%%%%%%%%%%%%%%%%%%%%%%%%%%%%%%%%%%%%%%%%%%%%%%%%%%%%%%%%%%%%%%%%%%
% %%%%%%%%%%%%%%%%%%%%%%%%%%%%%%%%%%%%%%%%%%%%%%%%%%%%%%%%%%%%%%%%%%%%%%%%%%%%%%
% \section{Implementation}
%\iffalse
%<*package>
%\fi
%
% This section describes the definitions file |childdoc.def|.

% The definitions cannot be loaded using |\usepackage| or |\RequirePackage|
% which has a mechanism to prevent loading a style file more than once.
% When loading the definitions by means of |\input|
% multiple instances have to be prevented manually:
%\iffalse
%This code needs to be before the `\ProvidesFile' directive
%which is defined at the beginning of this file.
%Therefore it is also placed there and commented out here.
%</package>
%<*discard>
%\fi
%    \begin{macrocode}
\ifdefined\childdocmain\endinput\fi
%    \end{macrocode}
%\iffalse
%</discard>
%<*package>
%\fi
%
% \macro{\ifchilddoc}
% \macro{\ifchilddocmanual}
% The conditional |\ifchilddoc| tells whether a
% child (true) or main (false) document is being compiled.
% The conditional |\ifchilddocmanual| tells whether
% the |\includeonly| mechanism is used (false) or
% the selection of child files must be performed manually (true).
% The definitions initialise to false:
%    \begin{macrocode}
\newif\ifchilddoc
\newif\ifchilddocmanual
%    \end{macrocode}

% \macro{\childdocname}
% \macro{\childdocjob}
% The macro |\childdocname| stores the name of the main document
% to be compiled. The macro |\childdocjob| stores the name of
% the document on which the \LaTeX{} compiler was originally invoked.
% The content of |\jobname| cannot be compared
% to filenames specified in the source due to different catcodes.
% The following code rescans |\jobname|, stores the result
% in |\childdocname| and saves a copy in |\childdocjob|:
%    \begin{macrocode}
\edef\childdocname{\scantokens\expandafter{\jobname\noexpand}}
\let\childdocjob\childdocname
%    \end{macrocode}

% \macro{\childdocdisable}
% The macro |\childdocdisable| prevents the main file
% from being processed more than once.
% At this stage, the main document command |\childdocmain|
% is assumed to be called once again where it should do nothing.
% Any subsequent call to it should prevent
% a secondary processing of the main document
% It overwrites the forwarding commands
% |\childdocof| and |\childdocforward|
% with empty macros to prevent further inclusions of the main document:
%    \begin{macrocode}
\newcommand{\childdocdisable}
{
  \renewcommand{\childdocmain}[1]{\renewcommand{\childdocmain}[1]{\endinput}}
  \renewcommand{\childdocof}[1]{}
  \renewcommand{\childdocby}[2][]{}
  \renewcommand{\childdocforward}[2][]{}
  \renewcommand{\childdocdisable}{}
}
%    \end{macrocode}

% \macro{\childdocmain}
% The macro |\childdocmain| is to be called at the top of the main file
% with nothing or the main filename (without extension) as argument.
% First, it breaks loops.
% If the argument is not empty and does not match |\childdocname|
% (which is set by the first inclusion of |childdoc.def|),
% |\ifchilddoc| is set to true, |\includeonly| is applied to the child file
% and |\jobname| is set to the main file
% (for proper handling of |.aux| files):
%    \begin{macrocode}
\newcommand{\childdocmain}[1]
{
  \childdocdisable\childdocmain{}
  \if?#1?\else
    \begingroup
      \def\childdoctmp{#1}
      \ifx\childdoctmp\childdocname
        \def\childdoctmp{}
      \else
        \def\childdoctmp
        {
          \childdoctrue
          \includeonly{\childdocname}
          \def\childdocjob{#1}
          \def\jobname{#1}
        }
      \fi
      \expandafter
    \endgroup
    \childdoctmp
  \fi
}
%    \end{macrocode}

% \macro{\childdocof}
% The command |\childdocof| redirects
% compilation to the main file |#1|.
%    \begin{macrocode}
\newcommand{\childdocof}[1]
{
  \childdocdisable
  \childdoctrue
  \includeonly{\childdocname}
  \def\jobname{#1}
  \def\childdocjob{#1}
  \input{#1}
}
%    \end{macrocode}

% \macro{\childdocby}
% The command |\childdocby| ....
%    \begin{macrocode}
\newcommand{\childdocby}[2][]
{
  \childdocdisable
  \childdoctrue
  \childdocmanualtrue
  \if?#1?\else
    \def\jobname{#2}
  \fi
  \def\childdocjob{#2}
  \input{#2}
  \endinput
}
%    \end{macrocode}

% \macro{\childdocforward}
% The command |\childdocforward| redirects
% compilation to the main file or
% (if the optional argument is given) a child file.
% Parameters are set as if the main file
% or a child file starting with |\childdocof| was compiled.
% Then compilation is handed over to the main file:
%    \begin{macrocode}
\newcommand{\childdocforward}[2][]
{
  \begingroup
    \if?#1?
      \def\childdoctmp
      {
        \def\childdocname{#2}
        \def\childdocjob{#2}
        \def\jobname{#2}
        \input{#2}
        \endinput
      }
    \else
      \def\childdoctmp
      {
        \childdocdisable
        \def\childdocname{#2}
        \childdoctrue
        \includeonly{#2}
        \def\childdocjob{#1}
        \def\jobname{#1}
        \input{#1}
        \endinput
      }
    \fi
    \expandafter
  \endgroup
  \childdoctmp
}
%    \end{macrocode}

% \macro{\childdocforwardprefix}
% The command |\childdocforwardprefix| redirects
% compilation to the main or a child file by means of a pattern.
% The prefix |#1| in the current filename is replaced by |#2|
% and the suffix of the current filename is kept
% (it is assumed that the filename does not contain the substring `|~~~|'
% which is used as a delimiter).
% Compilation is handed over to the new file by |\childdocforward|:
%    \begin{macrocode}
\newcommand{\childdocforwardprefix}[3][]
{
  \begingroup
    \def\childdocextract #2##1~~~{\def\childdoctmp{\childdocforward[#1]{#3##1}}}
    \expandafter\childdocextract\childdocname~~~
    \expandafter
  \endgroup
  \childdoctmp
}
%    \end{macrocode}

% \macro{\childdoc}
% The deprecated macro |\childdoc| is a legacy version of |\childdocmain|:
%    \begin{macrocode}
\newcommand{\childdoc}{\childdocmain}
%    \end{macrocode}

% \macro{\childdocredirect}
% The deprecated macro |\childdocredirect| is a legacy version
% of |\childdocforward| and |\childdocforwardprefix|:
%    \begin{macrocode}
\newcommand{\childdocredirect}[2][]
{
  \begingroup
    \if?#1?
      \def\childdoctmp{\childdocforward{#2}}
    \else
      \def\childdoctmp{\childdocforwardprefix{#1}{#2}}
    \fi
    \expandafter
  \endgroup
  \childdoctmp
}
%    \end{macrocode}

%\iffalse
%</package>
%\fi
%
\endinput
|\\
|\childdocof{|\textit{main}|}|\\
\end{tabular}
\end{center}
at the top of every child file \textit{child}
which is included by |\include{|\textit{child}|}|
from within the main file
(or at least for those files to be compiled individually).
The argument \textit{main} must be the filename of the main file.

There are a couple of
considerations in setting up the main and child documents:

%%%%%%%%%%%%%%%%%%%%%%%%%%%%%%%%%%%%%%%%
\paragraph{Restrictions.}

Please note the following restrictions:
\begin{itemize}
\item
|\childdocmain| must be called with one argument \textit{main}
to ensure compatibility with earlier version of the package.
It must either be empty (|\childdocmain{}|)
or precisely match the filename of the main file in which it is specified.
See \secref{sec:detection} for further information.
\item
The filename \textit{main} must be specified without the |.tex| extension.
\item
The filename \textit{main} is case sensitive
(even in case-insensitive file systems)
due to internal string comparison.
\item
The argument \textit{main} should be fully expanded, it cannot be a macro.
\item
Subdirectories and special characters should be avoided in filenames.
\item
The command |\childdocmain{|\textit{main}|}| must be followed by a whitespace.
It should not be followed immediately by another command
or by a comment mark `|%|'.
This is because the \TeX{} parser reads the token immediately following
the argument of |\childdocmain| and puts it
at the beginning of every child section;
however, a white\-space is ignored.
\end{itemize}

%%%%%%%%%%%%%%%%%%%%%%%%%%%%%%%%%%%%%%%%
\paragraph{Content of Main File.}

It is advisable to place all content in the child files included by |\include|.
Any output contained in the main file will appear in all child documents
unless suppressed manually;
it cannot be suppressed automatically by the |\includeonly| directive
and thus should normally be avoided.
A method to include some content in the main file
by means of conditional processing is described in \secref{sec:conditional}.

%%%%%%%%%%%%%%%%%%%%%%%%%%%%%%%%%%%%%%%%
\paragraph{Page Numbering.}

When only a part of the document is compiled,
the appropriate numbering of pages
(as well as other status parameters)
is determined from the |.aux| files.
The latter contain information from previous passes.
However this information needs to propagate through
all intermediate child documents.
Therefore the page numbering in child documents may well
be inconsistent until the complete document is compiled at least once.

A useful (if unconventional) way to always ensure a consistent
page numbering is to restart the numbering in each child document
and denote the pages by `\textit{child}|.|\textit{page}'
where \textit{child} represents the chapter/section number of the child file.
This can be achieved by the command
|\numberwithin{page}{|\textit{child}|}|
of the \textsf{amsmath} package
where \textit{child} can be |chapter| or |section|
depending on the chosen structuring.
Alternatively, one can modify the macro |\thepage| appropriately
and reset the counter |page| at the start of each child file.

%%%%%%%%%%%%%%%%%%%%%%%%%%%%%%%%%%%%%%%%%%%%%%%%%%%%%%%%%%%%%%%%%%%%%%%%%%%%%%%%
\subsection{Conditional Processing}
\label{sec:conditional}

The package provides a mechanism to compile different versions
of a document. To customise the versions further some conditional processing
can come in handy to distinguish which version is being compiled.
The package provides two macros to describe the compilation context:

%%%%%%%%%%%%%%%%%%%%%%%%%%%%%%%%%%%%%%%%
\DescribeMacro{\ifchilddoc}
The conditional |\ifchilddoc| distinguishes between the compilation of
child documents and the main document:
%
\begin{center}
|\ifchilddoc |\textit{child-code}| |[|\||else |\textit{main-code}]| \||fi|
\end{center}

%%%%%%%%%%%%%%%%%%%%%%%%%%%%%%%%%%%%%%%%
\DescribeMacro{\childdocname}
\DescribeMacro{\childdocjob}
The macro |\childdocname| contains the filename (without extension)
of the main or child file being processed.
Note that |\childdocjob| will always contain the name of the main file.

%%%%%%%%%%%%%%%%%%%%%%%%%%%%%%%%%%%%%%%%
\paragraph{Title Page.}

Conditional processing can be used to include a title or banner page
in the main document when proper precautions are taken.
Importantly, the code in the main file should ensure that the page counter
(as well as other status parameters which are stored in the |.aux| files)
takes the same value after the conditional processing.
Otherwise the page numbers may take divergent values
depending on which part is compiled.

For example, a title page could be declared by:
%
\begin{center}
\begin{tabular}{l}
|\ifchilddoc\||else|\\
|\addtocounter{page}{-1}|\\
\textit{code for title page}\\
|\newpage|\\
|\||fi|
\end{tabular}
\end{center}
%
A banner page for the child documents can be generated by:
%
\begin{center}
\begin{tabular}{l}
|\ifchilddoc|\\
|\addtocounter{page}{-1}|\\
\textit{code for banner page}\\
|\newpage|\\
|\||fi|
\end{tabular}
\end{center}
%
Here one could write a message such as:
\begin{center}
|This is the part \childdocname{} of \childdocjob{}.|
\end{center}

%%%%%%%%%%%%%%%%%%%%%%%%%%%%%%%%%%%%%%%%%%%%%%%%%%%%%%%%%%%%%%%%%%%%%%%%%%%%%%%%
\subsection{Flags}
\label{sec:flags}

The package makes it easy to generate different versions
of the main or child documents.
To this end compilation flags can be defined
and assigned different default values.
They will be particularly useful in conjunction
with the forwarding mechanism described in \secref{sec:forward}.

For example, it may be useful to have a flag |\version|
which can be set to |draft| or |final|.
The document source will contain some conditional code
depending on the value of |\version|.
Suppose further, the flag should default to |final| for the main file
and to |draft| for child files
which is a natural assignment for editing the document.
This is achieved by placing the following code
in the preamble of the main document
(below the |\childdocmain| directive):
%
\begin{center}
\begin{tabular}{l}
|\ifchilddoc|\\
|\providecommand{\version}{draft}|\\
|\||else|\\
|\providecommand{\version}{final}|\\
|\||fi|
\end{tabular}
\end{center}
%
The definition by |\providecommand| makes sure
that previous definitions are not overwritten.
Further statements |\providecommand{\version}{...}|
can thus be added before the above code to override it.

For the main file, one might add a line
(between |\childdocmain| and the above block)
%
\begin{center}
|%\ifchilddoc\||else\providecommand{\version}{draft}\||fi|
\end{center}
%
which can be uncommented to produce a draft version.
Likewise one can add a line to the very top of a child file
(above the |\childdocof{|\textit{main}|}| directive)
%
\begin{center}
|%\providecommand{\version}{final}|
\end{center}
%
which can be uncommented to produce the final version of this child document.

%%%%%%%%%%%%%%%%%%%%%%%%%%%%%%%%%%%%%%%%%%%%%%%%%%%%%%%%%%%%%%%%%%%%%%%%%%%%%%%%
\subsection{Forwarding}
\label{sec:forward}

Different versions of the main or child documents
using compilation flags as described in \secref{sec:flags}
can be (permanently) stored in different files
for convenient compilation, viewing and distribution.
To this end, the package defines a command
to pass on compilation to a different file:

%%%%%%%%%%%%%%%%%%%%%%%%%%%%%%%%%%%%%%%%
\DescribeMacro{\childdocforward}
The command |\childdocforward| redirects processing to
another source file:
%
\begin{center}
\begin{tabular}{l}
|% \iffalse
%
% childdoc.dtx Copyright (C) 2017-2018 Niklas Beisert
%
% This work may be distributed and/or modified under the
% conditions of the LaTeX Project Public License, either version 1.3
% of this license or (at your option) any later version.
% The latest version of this license is in
%   http://www.latex-project.org/lppl.txt
% and version 1.3 or later is part of all distributions of LaTeX
% version 2005/12/01 or later.
%
% This work has the LPPL maintenance status `maintained'.
%
% The Current Maintainer of this work is Niklas Beisert.
%
% This work consists of the files childdoc.dtx and childdoc.ins
% and the derived files childdoc.def and cdocsamp.tex with
% cdocsch1.tex, cdocsch2.tex, cdocsdrf.tex, cdocsfn1.tex, cdocsfn2.tex.
%
%<package>\ifdefined\childdocmain\endinput\fi
%<package>\ProvidesFile{childdoc.def}[2018/12/30 v2.0 child document driver]
%<samplemain>\ProvidesFile{cdocsamp.tex}[2018/12/30 v2.0 sample for childdoc]
%<*driver>
%\ProvidesFile{childdoc.drv}[2018/12/30 v2.0 childdoc reference manual file]
\PassOptionsToClass{10pt,a4paper}{article}
\documentclass{ltxdoc}

\usepackage[margin=35mm]{geometry}
\usepackage{hyperref}
\usepackage{hyperxmp}
\usepackage[usenames]{color}

\hypersetup{colorlinks=true}
\hypersetup{pdfstartview=FitH}
\hypersetup{pdfpagemode=UseNone}
\hypersetup{pdfsource={}}
\hypersetup{pdflang={en-UK}}
\hypersetup{pdfcopyright={Copyright 2017-2018 Niklas Beisert.
  This work may be distributed and/or modified under the
  conditions of the LaTeX Project Public License, either version 1.3
  of this license or (at your option) any later version.}}
\hypersetup{pdflicenseurl={http://www.latex-project.org/lppl.txt}}
\hypersetup{pdfcontactaddress={ETH Zurich, ITP, HIT K,
  Wolfgang-Pauli-Strasse 27}}
\hypersetup{pdfcontactpostcode={8093}}
\hypersetup{pdfcontactcity={Zurich}}
\hypersetup{pdfcontactcountry={Switzerland}}
\hypersetup{pdfcontactemail={nbeisert@itp.phys.ethz.ch}}
\hypersetup{pdfcontacturl={http://people.phys.ethz.ch/\xmptilde nbeisert/}}

\newcommand{\secref}[1]{\hyperref[#1]{section \ref*{#1}}}

\parskip1ex
\parindent0pt
\let\olditemize\itemize
\def\itemize{\olditemize\parskip0pt}

\begin{document}

\title{The \textsf{childdoc} Package}
\hypersetup{pdftitle={The childdoc Package}}
\author{Niklas Beisert\\[2ex]
  Institut f\"ur Theoretische Physik\\
  Eidgen\"ossische Technische Hochschule Z\"urich\\
  Wolfgang-Pauli-Strasse 27, 8093 Z\"urich, Switzerland\\[1ex]
  \href{mailto:nbeisert@itp.phys.ethz.ch}
  {\texttt{nbeisert@itp.phys.ethz.ch}}}
\hypersetup{pdfauthor={Niklas Beisert}}
\hypersetup{pdfsubject={Manual for the LaTeX2e Package childdoc}}
\date{30 December 2018, \textsf{v2.0}}
\maketitle

\begin{abstract}\noindent
\textsf{childdoc} is a \LaTeXe{} package
that enables the direct compilation
of document sections included by |\include|
to individual files.
\end{abstract}

\begingroup
\parskip0ex
\tableofcontents
\endgroup

%%%%%%%%%%%%%%%%%%%%%%%%%%%%%%%%%%%%%%%%%%%%%%%%%%%%%%%%%%%%%%%%%%%%%%%%%%%%%%%%
%%%%%%%%%%%%%%%%%%%%%%%%%%%%%%%%%%%%%%%%%%%%%%%%%%%%%%%%%%%%%%%%%%%%%%%%%%%%%%%%
\section{Introduction}

\LaTeX{} provides a mechanism to structure a large document (such as a book)
into a main file and several child files (containing the chapters)
using the |\include| command.
This mechanism is beneficial for documents
which span hundreds of pages in order to
make the source file(s) more manageable.
Moreover, compilation can be restricted to
selected child files by means of the |\includeonly| command.
The latter feature can be used to reduce the compilation time while editing
(this was significantly more useful in the earlier days of \LaTeX{})
or to generate a smaller document which is easier to navigate.
Another application of |\includeonly| is to generate
documents consisting of selected parts of the complete document.

However, there are a few drawbacks of the plain |\include| mechanism:
\begin{itemize}
\item
The child files cannot be compiled on their own,
they can only be compiled via the main file.
A naive editing environment
(such as a text editor with an option
to have the current file processed by \LaTeX)
may require one to switch to the main file before compiling;
attempting to compile the child file produces errors.
\item
The main file must be modified (each time)
to adjust the |\includeonly| command
to the present needs. This easily leaves the main file in a messy state.
\item
The generated document will always carry the filename
of the main document. This is inconvenient if
several child files are to be compiled and
to be kept for distribution.
\end{itemize}

The present package provides a simple interface
to make child files individually compilable by \LaTeX{}.
Compiling a child file then has the same effect as compiling
the main file with an |\includeonly| command
to select the appropriate child.
Moreover the generated document will carry the name of the child
rather than the main file.
This resolves all three above issues.

This feature is meant to make the editing of books,
thesis documents and lecture notes somewhat more convenient.
However, the package can also be used efficiently for
composing a series of documents (such as exercise sheets)
which are typically distributed individually.
It then assists the author in generating the individual documents
(potentially in different versions)
as well as a document containing the collected series.
Another application is in developing style files
or other kinds of included material
where compilation of the style file could redirect
to a sample or test file.

%%%%%%%%%%%%%%%%%%%%%%%%%%%%%%%%%%%%%%%%%%%%%%%%%%%%%%%%%%%%%%%%%%%%%%%%%%%%%%%%
%%%%%%%%%%%%%%%%%%%%%%%%%%%%%%%%%%%%%%%%%%%%%%%%%%%%%%%%%%%%%%%%%%%%%%%%%%%%%%%%
\section{Usage}

First of all, the package \textsf{childdoc} is \emph{not} a standard
\LaTeXe{} |.sty| style file! Therefore it needs to be invoked in
a non-standard way.

%%%%%%%%%%%%%%%%%%%%%%%%%%%%%%%%%%%%%%%%%%%%%%%%%%%%%%%%%%%%%%%%%%%%%%%%%%%%%%%%
\subsection{Included Files}
\label{sec:include}

%%%%%%%%%%%%%%%%%%%%%%%%%%%%%%%%%%%%%%%%
\DescribeMacro{\childdocmain}
To use the package, add the commands
\begin{center}
\begin{tabular}{l}
|\input{childdoc.def}|\\
|\childdocmain{}|\\
\end{tabular}
\end{center}
at the very top of the main \LaTeX{} file,
in particular \emph{before} the |\documentclass| statement!
The argument of |\childdocmain| should be left empty
(but it must be present).

%%%%%%%%%%%%%%%%%%%%%%%%%%%%%%%%%%%%%%%%
\DescribeMacro{\childdocof}
Furthermore, add the commands
\begin{center}
\begin{tabular}{l}
|\input{childdoc.def}|\\
|\childdocof{|\textit{main}|}|\\
\end{tabular}
\end{center}
at the top of every child file \textit{child}
which is included by |\include{|\textit{child}|}|
from within the main file
(or at least for those files to be compiled individually).
The argument \textit{main} must be the filename of the main file.

There are a couple of
considerations in setting up the main and child documents:

%%%%%%%%%%%%%%%%%%%%%%%%%%%%%%%%%%%%%%%%
\paragraph{Restrictions.}

Please note the following restrictions:
\begin{itemize}
\item
|\childdocmain| must be called with one argument \textit{main}
to ensure compatibility with earlier version of the package.
It must either be empty (|\childdocmain{}|)
or precisely match the filename of the main file in which it is specified.
See \secref{sec:detection} for further information.
\item
The filename \textit{main} must be specified without the |.tex| extension.
\item
The filename \textit{main} is case sensitive
(even in case-insensitive file systems)
due to internal string comparison.
\item
The argument \textit{main} should be fully expanded, it cannot be a macro.
\item
Subdirectories and special characters should be avoided in filenames.
\item
The command |\childdocmain{|\textit{main}|}| must be followed by a whitespace.
It should not be followed immediately by another command
or by a comment mark `|%|'.
This is because the \TeX{} parser reads the token immediately following
the argument of |\childdocmain| and puts it
at the beginning of every child section;
however, a white\-space is ignored.
\end{itemize}

%%%%%%%%%%%%%%%%%%%%%%%%%%%%%%%%%%%%%%%%
\paragraph{Content of Main File.}

It is advisable to place all content in the child files included by |\include|.
Any output contained in the main file will appear in all child documents
unless suppressed manually;
it cannot be suppressed automatically by the |\includeonly| directive
and thus should normally be avoided.
A method to include some content in the main file
by means of conditional processing is described in \secref{sec:conditional}.

%%%%%%%%%%%%%%%%%%%%%%%%%%%%%%%%%%%%%%%%
\paragraph{Page Numbering.}

When only a part of the document is compiled,
the appropriate numbering of pages
(as well as other status parameters)
is determined from the |.aux| files.
The latter contain information from previous passes.
However this information needs to propagate through
all intermediate child documents.
Therefore the page numbering in child documents may well
be inconsistent until the complete document is compiled at least once.

A useful (if unconventional) way to always ensure a consistent
page numbering is to restart the numbering in each child document
and denote the pages by `\textit{child}|.|\textit{page}'
where \textit{child} represents the chapter/section number of the child file.
This can be achieved by the command
|\numberwithin{page}{|\textit{child}|}|
of the \textsf{amsmath} package
where \textit{child} can be |chapter| or |section|
depending on the chosen structuring.
Alternatively, one can modify the macro |\thepage| appropriately
and reset the counter |page| at the start of each child file.

%%%%%%%%%%%%%%%%%%%%%%%%%%%%%%%%%%%%%%%%%%%%%%%%%%%%%%%%%%%%%%%%%%%%%%%%%%%%%%%%
\subsection{Conditional Processing}
\label{sec:conditional}

The package provides a mechanism to compile different versions
of a document. To customise the versions further some conditional processing
can come in handy to distinguish which version is being compiled.
The package provides two macros to describe the compilation context:

%%%%%%%%%%%%%%%%%%%%%%%%%%%%%%%%%%%%%%%%
\DescribeMacro{\ifchilddoc}
The conditional |\ifchilddoc| distinguishes between the compilation of
child documents and the main document:
%
\begin{center}
|\ifchilddoc |\textit{child-code}| |[|\||else |\textit{main-code}]| \||fi|
\end{center}

%%%%%%%%%%%%%%%%%%%%%%%%%%%%%%%%%%%%%%%%
\DescribeMacro{\childdocname}
\DescribeMacro{\childdocjob}
The macro |\childdocname| contains the filename (without extension)
of the main or child file being processed.
Note that |\childdocjob| will always contain the name of the main file.

%%%%%%%%%%%%%%%%%%%%%%%%%%%%%%%%%%%%%%%%
\paragraph{Title Page.}

Conditional processing can be used to include a title or banner page
in the main document when proper precautions are taken.
Importantly, the code in the main file should ensure that the page counter
(as well as other status parameters which are stored in the |.aux| files)
takes the same value after the conditional processing.
Otherwise the page numbers may take divergent values
depending on which part is compiled.

For example, a title page could be declared by:
%
\begin{center}
\begin{tabular}{l}
|\ifchilddoc\||else|\\
|\addtocounter{page}{-1}|\\
\textit{code for title page}\\
|\newpage|\\
|\||fi|
\end{tabular}
\end{center}
%
A banner page for the child documents can be generated by:
%
\begin{center}
\begin{tabular}{l}
|\ifchilddoc|\\
|\addtocounter{page}{-1}|\\
\textit{code for banner page}\\
|\newpage|\\
|\||fi|
\end{tabular}
\end{center}
%
Here one could write a message such as:
\begin{center}
|This is the part \childdocname{} of \childdocjob{}.|
\end{center}

%%%%%%%%%%%%%%%%%%%%%%%%%%%%%%%%%%%%%%%%%%%%%%%%%%%%%%%%%%%%%%%%%%%%%%%%%%%%%%%%
\subsection{Flags}
\label{sec:flags}

The package makes it easy to generate different versions
of the main or child documents.
To this end compilation flags can be defined
and assigned different default values.
They will be particularly useful in conjunction
with the forwarding mechanism described in \secref{sec:forward}.

For example, it may be useful to have a flag |\version|
which can be set to |draft| or |final|.
The document source will contain some conditional code
depending on the value of |\version|.
Suppose further, the flag should default to |final| for the main file
and to |draft| for child files
which is a natural assignment for editing the document.
This is achieved by placing the following code
in the preamble of the main document
(below the |\childdocmain| directive):
%
\begin{center}
\begin{tabular}{l}
|\ifchilddoc|\\
|\providecommand{\version}{draft}|\\
|\||else|\\
|\providecommand{\version}{final}|\\
|\||fi|
\end{tabular}
\end{center}
%
The definition by |\providecommand| makes sure
that previous definitions are not overwritten.
Further statements |\providecommand{\version}{...}|
can thus be added before the above code to override it.

For the main file, one might add a line
(between |\childdocmain| and the above block)
%
\begin{center}
|%\ifchilddoc\||else\providecommand{\version}{draft}\||fi|
\end{center}
%
which can be uncommented to produce a draft version.
Likewise one can add a line to the very top of a child file
(above the |\childdocof{|\textit{main}|}| directive)
%
\begin{center}
|%\providecommand{\version}{final}|
\end{center}
%
which can be uncommented to produce the final version of this child document.

%%%%%%%%%%%%%%%%%%%%%%%%%%%%%%%%%%%%%%%%%%%%%%%%%%%%%%%%%%%%%%%%%%%%%%%%%%%%%%%%
\subsection{Forwarding}
\label{sec:forward}

Different versions of the main or child documents
using compilation flags as described in \secref{sec:flags}
can be (permanently) stored in different files
for convenient compilation, viewing and distribution.
To this end, the package defines a command
to pass on compilation to a different file:

%%%%%%%%%%%%%%%%%%%%%%%%%%%%%%%%%%%%%%%%
\DescribeMacro{\childdocforward}
The command |\childdocforward| redirects processing to
another source file:
%
\begin{center}
\begin{tabular}{l}
|\input{childdoc.def}|\\
|\childdocforward[|\textit{main}|]{|\textit{dest}|}|\\
\end{tabular}
\end{center}
%
The argument \textit{dest} is the destination file
(without extension).
It should be the main file or one of the child files.
Note that further \textsf{childdoc} directives
such as |\childdocof| and |\childdocforward|
in the indicated file will be processed in this form.
The optional argument \textit{main}
passes on directly to the main file \textit{main}
while pretending to compile the child \textit{dest}.
This form behaves as if \textit{dest}
issues |\childdocof{|\textit{main}|}| right away,
and no further \textsf{childdoc} directives will be processed.

%%%%%%%%%%%%%%%%%%%%%%%%%%%%%%%%%%%%%%%%
\DescribeMacro{\...prefix}
In the alternative form |\childdocforwardprefix|,
%
\begin{center}
\begin{tabular}{l}
|\input{childdoc.def}|\\
|\childdocforwardprefix[|\textit{main}|]{|\textit{prefix}|}{|\textit{dest}|}|
\end{tabular}
\end{center}
%
the destination file is determined by a pattern
depending on the current file:
To make this work, the current file must be called
`{\textit{prefix}\hspace{0.2em}\textit{suffix}}'
with \textit{prefix} matching precisely the argument.
Processing is then passed on to the file
`{\textit{dest}\hspace{0.2em}\textit{suffix}}'.
Surely, the same effect is achieved by
directly specifying the
argument `{\textit{dest}\hspace{0.2em}\textit{suffix}}'
in the first form.
However, that requires to set up a different file
for each child. With the alternative form of the command
all these files can have exactly the same content
which simplifies setting them up and maintaining them.

For example, the following file |draft.tex|
with a compilation flag |\version| as described in \secref{sec:flags}
compiles the main document as a draft:
%
\begin{center}
\begin{tabular}{l}
|\def\version{draft}|\\
|\input{childdoc.def}|\\
|\childdocforward{|\textit{main}|}|
\end{tabular}
\end{center}
%
Likewise, the following files |final|\textit{nn}|.tex|
compile the final version of the child document
|child|\textit{nn}|.tex|:
%
\begin{center}
\begin{tabular}{l}
|\def\version{final}|\\
|\input{childdoc.def}|\\
|\childdocforwardprefix{final}{child}|
\end{tabular}
\end{center}
%

Note that when several versions of a main file and/or of each child file
are to be generated, it may be convenient to set up a |Makefile| or
shell script to automatise the process.

%%%%%%%%%%%%%%%%%%%%%%%%%%%%%%%%%%%%%%%%%%%%%%%%%%%%%%%%%%%%%%%%%%%%%%%%%%%%%%%%
\subsection{Command Line Processing}
\label{sec:commandline}

The effect of redirection files can also be achieved by invoking
the \LaTeX{} compiler with a more elaborate command line.
Most conveniently this should be done as part
of a shell script or a |Makefile|.

When using \textsf{childdoc} in the main file, the following
command lines effectively perform a redirection
(note that depending on the shell being used,
backslashes may have to be doubled: `|\|' $\to$ `|\\|'):
%
\begin{center}
|... -jobname "|\textit{target}|" |\\|"|[\textit{flags}]%
|\input{childdoc.def}\childdocforward[|\textit{main}|]{|\textit{dest}|}"|
\end{center}
%
Here \textit{target} is the name of the output file,
\textit{main} is the name of the main file
and \textit{dest} is the name of the main or child file to be processed
(all filenames without extensions).
The optional argument \textit{main} can be omitted
if \textit{main} matches \textit{dest}.
Optionally, compilation \textit{flags} can be defined via |\def| commands.
This command line makes the \TeX{} engine believe
it is compiling the file \textit{target}
whose content is specified as the latter parameter.
The provided code then forwards the processing to
\textit{main} or \textit{dest} as described in \secref{sec:forward}.

%%%%%%%%%%%%%%%%%%%%%%%%%%%%%%%%%%%%%%%%%%%%%%%%%%%%%%%%%%%%%%%%%%%%%%%%%%%%%%%%
\subsection{Include by Input}
\label{sec:input}

Including child documents by |\include| has some restrictions by design.
Most notably, the content of a child document always occupies
its own set of pages; pages cannot be shared between child documents.
Usually, this behaviour makes perfect sense
because each child document contain an essential part of the document.
However, in some situations it may be desirable to compose
a document from a collection of parts
without having mandatory page breaks between then.
For this case, the package
provides a mechanism to include parts
by |\input| which can also be processed individually.
However, by construction this mechanism
requires manual handling of the content to be output.

%%%%%%%%%%%%%%%%%%%%%%%%%%%%%%%%%%%%%%%%
\DescribeMacro{\ifchilddocmanual}
The main file should be prepared as usual, see \secref{sec:include}.
However, the document body must make a distinction
between processing of an individual part and of the main document, e.g.:
%
\begin{center}
\begin{tabular}{l}
|\ifchilddocmanual|\\
|\input{\childdocname}|\\
|\||else|\\
\textit{document body with }|\input{|\textit{part}|}|\\
|\||fi|
\end{tabular}
\end{center}
%
The conditional |\ifchilddocmanual| is true whenever
a part to be included by |\input| is being compiled,
and the name of the part is stored in |\childdocname|.

%%%%%%%%%%%%%%%%%%%%%%%%%%%%%%%%%%%%%%%%
\DescribeMacro{\childdocby}
Each part to be included by |\input| should start with:
%
\begin{center}
\begin{tabular}{l}
|\input{childdoc.def}|\\
|\childdocby{|\textit{main}|}|\\
\end{tabular}
\end{center}
%
The directive |\childdocby| is similar to |\childdocof|
described in \secref{sec:include},
but the subsequent selection of content must be done manually.
To that end, both |\ifchilddoc| and |\ifchilddocmanual|
will be true upon processing of a part,
and the name of the part is stored in |\childdocname|.
Note that |\jobname| will be set to the filename of the current part
so that each part receives an individual |.aux| file
that does not interfere with the |.aux| file(s) of the main document.
This behaviour can be altered by the alternative form
|\childdocby[*]{|\textit{main}|}| (with a non-empty optional argument)
which uses the |.aux| file of the main document
by setting |\jobname| to \textit{main}.

%%%%%%%%%%%%%%%%%%%%%%%%%%%%%%%%%%%%%%%%%%%%%%%%%%%%%%%%%%%%%%%%%%%%%%%%%%%%%%%%
\subsection{Driver Development}
\label{sec:driver}

The \textsf{childdoc} mechanism can also be use for the development
of definition files such as \LaTeX{} styles or classes.
This case differs from the above setup with multiple parts
included by |\include| in that no |\includeonly| should be invoked.
This can be achieved by starting the include file
(before |\ProvidesPackage|) with:
%
\begin{center}
\begin{tabular}{l}
|\input{childdoc.def}|\\
|\childdocforward{|\textit{main}|}|\\
\end{tabular}
\end{center}
%
or alternatively with:
%
\begin{center}
\begin{tabular}{l}
|\input{childdoc.def}|\\
|\childdocby{|\textit{main}|}|\\
\end{tabular}
\end{center}
%
Both forms have slightly different effects as described above.
The main file is prepared as usual, see \secref{sec:include}.

%%%%%%%%%%%%%%%%%%%%%%%%%%%%%%%%%%%%%%%%%%%%%%%%%%%%%%%%%%%%%%%%%%%%%%%%%%%%%%%%
\subsection{Legacy Detection}
\label{sec:detection}

The directive |\childdocmain| in the main file can detect
whether the complete document or merely a child is to be compiled
even without using the directive |\childdocof|.
This method is deprecated because it is less robust
and there is no compelling reason to use it;
it is merely provided for backward compatibility
and it may be removed in future versions.

If the detection mechanism is to be used,
it is mandatory to correctly specify
the filename of the main file as the argument of |\childdocmain|:
%
\begin{center}
\begin{tabular}{l}
|\input{childdoc.def}|\\
|\childdocmain{|\textit{main}|}|\\
\end{tabular}
\end{center}
%
If |\jobname| does not match the argument \textit{main} of |\childdocmain|,
it is assumed that |\jobname| points to the child file to be compiled.
When using |\childdocmain| with the main file specified as argument,
it suffices to start a child file
with just |\input{|\textit{main}|}|
without loading of the package and using |\childdocof|.
If instead all processing is done
with the appropriate \textsf{childdoc} directives,
the argument of \textit{main} of |\childdocmain| can be empty.

An alternative version of the command line processing described
in \secref{sec:commandline} using the detection mechanism reads:
%
\begin{center}
|... -jobname "|\textit{target}|" "|[\textit{flags}]%
[|\def\jobname{|\textit{dest}|}|]|\input{|\textit{main}|}"|
\end{center}

%%%%%%%%%%%%%%%%%%%%%%%%%%%%%%%%%%%%%%%%%%%%%%%%%%%%%%%%%%%%%%%%%%%%%%%%%%%%%%%%
\subsection{Manual Code}
\label{sec:manual}

In case one cannot be certain whether the definitions file |childdoc.def|
is installed on the target \TeX{} distribution
and one prefers not to ship it,
it is conceivable to paste a few relevant commands into the sources.

To that end, drop all statements |\input{childdoc.def}|
and perform the replacements as outlined below.
Instead of |\childdocmain{|\textit{main}|}| add the following code
to the top of the main file:
%
\begin{center}
\begin{tabular}{l}
|\||ifdefined\childdocname\endinput\||fi\newif\ifchilddoc|\\
|\edef\childdocname{\scantokens\expandafter{\jobname\noexpand}}|\\
|\def\childdocmain{|\textit{main}|}\||ifx\childdocmain\childdocname\||else|\\
|\childdoctrue\includeonly{\childdocname}\let\jobname\childdocmain\||fi|\\
\end{tabular}
\end{center}
%
Instead of |\childdocof{|\textit{main}|}| just include the main file
at the top of each child file:
%
\begin{center}
|\input{|\textit{main}|}|
\end{center}
%
A simple redirection |\childdocforward{|\textit{dest}|}| is achieved by:
%
\begin{center}
|\def\jobname{|\textit{dest}|}\input{\jobname}|
\end{center}
%
The redirection with prefix
|\childdocforwardprefix[|\textit{prefix}|]{|\textit{dest}|}|
is accomplished by:
%
\begin{center}
\begin{tabular}{l}
|{\edef\jobname{\scantokens\expandafter{\jobname\noexpand}}|\\
|\def\redirectjob |\textit{prefix}|#1~~~{\gdef\jobname{|\textit{dest}|#1}}|\\
|\expandafter\redirectjob\jobname~~~}\input{\jobname}|
\end{tabular}
\end{center}

In an alternative approach,
child documents can be compiled by a specific command line
without additional code or specific definitions:
%
\begin{center}
|... -jobname "|\textit{target}|" "|[\textit{flags}]%
|\includeonly{|\textit{dest}|}\input{|\textit{main}|}"|
\end{center}
%

%%%%%%%%%%%%%%%%%%%%%%%%%%%%%%%%%%%%%%%%%%%%%%%%%%%%%%%%%%%%%%%%%%%%%%%%%%%%%%%%
%%%%%%%%%%%%%%%%%%%%%%%%%%%%%%%%%%%%%%%%%%%%%%%%%%%%%%%%%%%%%%%%%%%%%%%%%%%%%%%%
\section{Information}

%%%%%%%%%%%%%%%%%%%%%%%%%%%%%%%%%%%%%%%%%%%%%%%%%%%%%%%%%%%%%%%%%%%%%%%%%%%%%%%%
\subsection{Copyright}

Copyright \copyright{} 2017--2018 Niklas Beisert

This work may be distributed and/or modified under the
conditions of the \LaTeX{} Project Public License, either version 1.3
of this license or (at your option) any later version.
The latest version of this license is in
  \url{http://www.latex-project.org/lppl.txt}
and version 1.3 or later is part of all distributions of \LaTeX{}
version 2005/12/01 or later.

This work has the LPPL maintenance status `maintained'.

The Current Maintainer of this work is Niklas Beisert.

This work consists of the files |README.txt|, |childdoc.ins| and |childdoc.dtx|
as well as the derived files |childdoc.def|, |cdocsamp.tex|
with |cdocsch1.tex|, |cdocsch2.tex|, |cdocspt3.tex|, |cdocspt4.tex|,
|cdocsdrf.tex|, |cdocsfn1.tex|, |cdocsfn2.tex|
as well as |childdoc.pdf|.

%%%%%%%%%%%%%%%%%%%%%%%%%%%%%%%%%%%%%%%%%%%%%%%%%%%%%%%%%%%%%%%%%%%%%%%%%%%%%%%%
\subsection{Files and Installation}

The package consists of the files:
%
\begin{center}
\begin{tabular}{ll}
    |README.txt|   & readme file \\
    |childdoc.ins| & installation file \\
    |childdoc.dtx| & source file \\
    |childdoc.def| & definition file \\
    |cdocsamp.tex| & sample main file \\
    |cdocsch1.tex| & sample include file \\
    |cdocsch2.tex| & sample include file \\
    |cdocspt3.tex| & sample part file \\
    |cdocspt4.tex| & sample part file \\
    |cdocsdrf.tex| & sample redirection file \\
    |cdocsfn1.tex| & sample redirection file \\
    |cdocsfn2.tex| & sample redirection file \\
    |childdoc.pdf| & manual
\end{tabular}
\end{center}
%
The distribution consists of the files
|README.txt|, |childdoc.ins| and |childdoc.dtx|.
%
\begin{itemize}
\item
Run (pdf)\LaTeX{} on |childdoc.dtx|
to compile the manual |childdoc.pdf| (this file).
\item
Run \LaTeX{} on |childdoc.ins| to create the definitions file |childdoc.def|
and the sample |cdocsamp.tex| with include files
|cdocsch1.tex|, |cdocsch2.tex|, |cdocspt3.tex|, |cdocspt4.tex|,
|cdocsdrf.tex|, |cdocsfn1.tex|, |cdocsfn2.tex|.
Then copy the file |childdoc.def| to an appropriate directory of your \LaTeX{}
distribution, e.g.\ \textit{texmf-root}|/tex/latex/childdoc|.
\end{itemize}

%%%%%%%%%%%%%%%%%%%%%%%%%%%%%%%%%%%%%%%%%%%%%%%%%%%%%%%%%%%%%%%%%%%%%%%%%%%%%%%%
\subsection{Related CTAN Packages}

There are several other packages which offer a similar functionality:
%
\begin{itemize}
\item
The packages
\href{http://ctan.org/pkg/docmute}{\textsf{docmute}},
\href{http://ctan.org/pkg/includex}{\textsf{includex}} and
\href{http://ctan.org/pkg/standalone}{\textsf{standalone}}
provide commands to include only the document body of
a child file thus allowing both files to be compiled individually.
\item
The packages \href{http://ctan.org/pkg/subdocs}{\textsf{subdocs}}
and \href{http://ctan.org/pkg/subfiles}{\textsf{subfiles}}
provide structures in which the main and child documents can be
encapsulated and allowing them to be compiled individually.
The inclusion mechanism is different from the conventional |\include|.
\item
The package \href{http://ctan.org/pkg/combine}{\textsf{combine}}
is an elaborate solution to combine several documents into one.
\end{itemize}
%
See also the CTAN topic \href{http://ctan.org/topic/subdocs}{\textsf{subdocs}}
for further related packages.
The present package differs from the above solutions in that
a document structure constructed with the conventional |\include| mechanism
just needs two extra commands at the top of every file
such that all constituent files can be compiled individually.

%%%%%%%%%%%%%%%%%%%%%%%%%%%%%%%%%%%%%%%%%%%%%%%%%%%%%%%%%%%%%%%%%%%%%%%%%%%%%%%%
%\subsection{Feature Suggestions}
%
%The following is a list of features which may be useful for future
%versions of this package:
%%
%\begin{itemize}
%\item
%\ldots
%\end{itemize}

%%%%%%%%%%%%%%%%%%%%%%%%%%%%%%%%%%%%%%%%%%%%%%%%%%%%%%%%%%%%%%%%%%%%%%%%%%%%%%%%
\subsection{Revision History}

%%%%%%%%%%%%%%%%%%%%%%%%%%%%%%%%%%%%%%%%
\paragraph{v2.0:} 2018/12/30

\begin{itemize}
\item
immediate forward processing
\item
added |\childdocby| mechanism
\item
manual restructured
\end{itemize}

%%%%%%%%%%%%%%%%%%%%%%%%%%%%%%%%%%%%%%%%
\paragraph{v1.6:} 2018/01/17

\begin{itemize}
\item
application for development of include files
\item
corrections to manual
\end{itemize}

%%%%%%%%%%%%%%%%%%%%%%%%%%%%%%%%%%%%%%%%
\paragraph{v1.5:} 2017/05/21

\begin{itemize}
\item
more complete structuring introduced
\item
|\childdocof| introduced
\item
|\childdoc| renamed to |\childdocmain|
\item
|\childredirect| renamed to |\childdocforward| and |\childdocforwardprefix|
and functionality expanded
\end{itemize}

%%%%%%%%%%%%%%%%%%%%%%%%%%%%%%%%%%%%%%%%
\paragraph{v1.0:} 2017/04/27

\begin{itemize}
\item
manual and install package
\item
first version published on CTAN
\end{itemize}

%%%%%%%%%%%%%%%%%%%%%%%%%%%%%%%%%%%%%%%%
\paragraph{v0.6:} 2017/04/26

\begin{itemize}
\item
redirection mechanism added
\end{itemize}

%%%%%%%%%%%%%%%%%%%%%%%%%%%%%%%%%%%%%%%%
\paragraph{v0.5:} 2017/04/26

\begin{itemize}
\item
functionality in definition file
\end{itemize}


%%%%%%%%%%%%%%%%%%%%%%%%%%%%%%%%%%%%%%%%%%%%%%%%%%%%%%%%%%%%%%%%%%%%%%%%%%%%%%%%
%%%%%%%%%%%%%%%%%%%%%%%%%%%%%%%%%%%%%%%%%%%%%%%%%%%%%%%%%%%%%%%%%%%%%%%%%%%%%%%%
%%%%%%%%%%%%%%%%%%%%%%%%%%%%%%%%%%%%%%%%%%%%%%%%%%%%%%%%%%%%%%%%%%%%%%%%%%%%%%%%
\appendix

\settowidth\MacroIndent{\rmfamily\scriptsize 000\ }

 \DocInput{childdoc.dtx}

\end{document}
%</driver>
% \fi
%
% %%%%%%%%%%%%%%%%%%%%%%%%%%%%%%%%%%%%%%%%%%%%%%%%%%%%%%%%%%%%%%%%%%%%%%%%%%%%%%
% %%%%%%%%%%%%%%%%%%%%%%%%%%%%%%%%%%%%%%%%%%%%%%%%%%%%%%%%%%%%%%%%%%%%%%%%%%%%%%
% \section{Sample}
%\iffalse
%<*samplemain>
%\fi
%
% The following presents a sample document
% with two chapters, two parts, a title page,
% a compile flag as well as three forwarding files to set the flag.
% It consists of eight |.tex| files:
% \begin{center}
% \begin{tabular}{ll}
% |cdocsamp.tex|&main file\\
% |cdocsch1.tex|&include file for chapter 1\\
% |cdocsch2.tex|&include file for chapter 2\\
% |cdocspt3.tex|&include file for part 3\\
% |cdocspt4.tex|&include file for part 4\\
% |cdocsdrf.tex|&forwarding file for main file in draft mode\\
% |cdocsfi1.tex|&forwarding file for final version of chapter 1\\
% |cdocsfi2.tex|&forwarding file for final version of chapter 2\\
% \end{tabular}
% \end{center}
% Each of the eight files can be compiled directly by the \LaTeX{} compiler.
%
% %%%%%%%%%%%%%%%%%%%%%%%%%%%%%%%%%%%%%%
% \paragraph{Main File.}
%
% The main file is called |cdocsamp.tex|.
%
% Load the \textsf{childdoc} definitions and
% declare the filename for the main document:
%    \begin{macrocode}
\input{childdoc.def}
\childdocmain{}
%    \end{macrocode}

% Optional override for |\version| flag:
%    \begin{macrocode}
%%\ifchilddoc\else\providecommand{\version}{draft}\fi
%    \end{macrocode}

% Define the default values for the |\version| flag
% (|final| for the main file and |draft| for childs):
%    \begin{macrocode}
\ifchilddoc
\providecommand{\version}{draft}
\else
\providecommand{\version}{final}
\fi
%    \end{macrocode}

% Load the standard document class:
%    \begin{macrocode}
\documentclass[12pt]{article}
%    \end{macrocode}

% Start the document body:
%    \begin{macrocode}
\begin{document}
%    \end{macrocode}

% Declare a title page.
% Print title, part of document being processed and version flag:
%    \begin{macrocode}
\addtocounter{page}{-1}
\begin{center}
{\LARGE\bfseries{}childdoc example\par}
\vspace{1cm}
\ifchilddoc
\ifchilddocmanual part\else chapter\fi:
`\childdocname' of `\childdocjob'\par
\else
main document: `\childdocjob'\par
\fi
version: \version\par
\end{center}
\newpage
%    \end{macrocode}

% Manually include selected file,
% otherwise process as usual:
%    \begin{macrocode}
\ifchilddocmanual
\section*{part `\childdocname'}
\input{\childdocname}
\else
%    \end{macrocode}

% Include the two chapters:
%    \begin{macrocode}
\include{cdocsch1}
\include{cdocsch2}
%    \end{macrocode}

% Include the two parts unless only chapters should be displayed:
%    \begin{macrocode}
\ifchilddoc\else
\section{part three}
\input{cdocspt3}
\section{part four}
\input{cdocspt4}
\fi
%    \end{macrocode}

% Process as usual until here:
%    \begin{macrocode}
\fi
%    \end{macrocode}

% End of document body:
%    \begin{macrocode}
\end{document}
%    \end{macrocode}
%\iffalse
%</samplemain>
%\fi
%
% %%%%%%%%%%%%%%%%%%%%%%%%%%%%%%%%%%%%%%
% \paragraph{Chapter Include Files.}
%
% The include files are called |cdocsch1.tex| and |cdocsch2.tex|.
%
%\iffalse
%<*samplechap1|samplechap2>
%\fi

% Optional override for |\version| flag:
%    \begin{macrocode}
%%\providecommand{\version}{final}
%    \end{macrocode}

% Include the main document:
%    \begin{macrocode}
\input{childdoc.def}
\childdocof{cdocsamp}
%    \end{macrocode}

%\iffalse
%</samplechap1|samplechap2>
%\fi
%
%\iffalse
%<*samplechap1>
%\fi
% Some text for chapter 1:
%    \begin{macrocode}
\section{one}
some text in chapter one
%    \end{macrocode}

%\iffalse
%</samplechap1>
%\fi
% Some text for chapter 2:
%\iffalse
%<*samplechap2>
%\fi
%    \begin{macrocode}
\section{two}
more text in chapter two
%    \end{macrocode}

%\iffalse
%</samplechap2>
%\fi
%
% %%%%%%%%%%%%%%%%%%%%%%%%%%%%%%%%%%%%%%
% \paragraph{Part Include Files.}
%
% The include files are called |cdocspt3.tex| and |cdocspt4.tex|.
%
%\iffalse
%<*samplepart3|samplepart4>
%\fi

% Optional override for |\version| flag:
%    \begin{macrocode}
%%\providecommand{\version}{final}
%    \end{macrocode}

% Include the main document:
%    \begin{macrocode}
\input{childdoc.def}
\childdocby{cdocsamp}
%    \end{macrocode}

%\iffalse
%</samplepart3|samplepart4>
%\fi
%
%\iffalse
%<*samplepart3>
%\fi
% Some text for part 3:
%    \begin{macrocode}
some text in part three
%    \end{macrocode}

%\iffalse
%</samplepart3>
%\fi
% Some text for part 4:
%\iffalse
%<*samplepart4>
%\fi
%    \begin{macrocode}
more text in part four
%    \end{macrocode}

%\iffalse
%</samplepart4>
%\fi
%
% %%%%%%%%%%%%%%%%%%%%%%%%%%%%%%%%%%%%%%
% \paragraph{Forwarding for a Complete Draft.}
%
% The following forwarding file |cdocsdrf.tex|
% compiles the main document in draft mode:
%\iffalse
%<*sampledraft>
%\fi
%    \begin{macrocode}
\def\version{draft}
\input{childdoc.def}
\childdocforward{cdocsamp}
%    \end{macrocode}

%\iffalse
%</sampledraft>
%\fi
%
% %%%%%%%%%%%%%%%%%%%%%%%%%%%%%%%%%%%%%%
% \paragraph{Forwarding for Final Version of the Chapters.}
%
% The following forwarding files |cdocsfn1.tex| and |cdocsfn2.tex|
% (with identical content)
% compile the final versions of the child documents
% |cdocsch1.tex| and |cdocsch2.tex|, respectively:
%\iffalse
%<*samplefinal>
%\fi
%    \begin{macrocode}
\def\version{final}
\input{childdoc.def}
\childdocforwardprefix[cdocsamp]{cdocsfn}{cdocsch}
%    \end{macrocode}

%\iffalse
%</samplefinal>
%\fi
%
% %%%%%%%%%%%%%%%%%%%%%%%%%%%%%%%%%%%%%%
% \paragraph{Command Line Processing.}
%
% The following three command lines generate the output files
% |cdocscld|, |cdocscl1| and |cdocscl2|
% which should be identical to
% |cdocsdrf|, |cdocsch1| and |cdocsfn2|, respectively:
% \begin{center}
% \begin{tabular}{l}
% |latex -jobname cdocscld \|\\
% |  "\def\version{draft}\input{childdoc.def}\childdocforward{cdocsamp}"|\\
% |latex -jobname cdocscl1 \|\\
% |  "\input{childdoc.def}\childdocforward[cdocsamp]{cdocsch1}"|\\
% |latex -jobname cdocscl2 \|\\
% |  "\def\version{final}\input{childdoc.def}\childdocforward{cdocsch2}"|
% \end{tabular}
% \end{center}
% Note that the trailing backslash on each first line
% merely continues the input to the second line
% (for convenient cut ant paste).
% Furthermore, the command |latex| can be replaced by any
% of its alternative versions such as |pdflatex|.
%
% %%%%%%%%%%%%%%%%%%%%%%%%%%%%%%%%%%%%%%%%%%%%%%%%%%%%%%%%%%%%%%%%%%%%%%%%%%%%%%
% %%%%%%%%%%%%%%%%%%%%%%%%%%%%%%%%%%%%%%%%%%%%%%%%%%%%%%%%%%%%%%%%%%%%%%%%%%%%%%
% \section{Implementation}
%\iffalse
%<*package>
%\fi
%
% This section describes the definitions file |childdoc.def|.

% The definitions cannot be loaded using |\usepackage| or |\RequirePackage|
% which has a mechanism to prevent loading a style file more than once.
% When loading the definitions by means of |\input|
% multiple instances have to be prevented manually:
%\iffalse
%This code needs to be before the `\ProvidesFile' directive
%which is defined at the beginning of this file.
%Therefore it is also placed there and commented out here.
%</package>
%<*discard>
%\fi
%    \begin{macrocode}
\ifdefined\childdocmain\endinput\fi
%    \end{macrocode}
%\iffalse
%</discard>
%<*package>
%\fi
%
% \macro{\ifchilddoc}
% \macro{\ifchilddocmanual}
% The conditional |\ifchilddoc| tells whether a
% child (true) or main (false) document is being compiled.
% The conditional |\ifchilddocmanual| tells whether
% the |\includeonly| mechanism is used (false) or
% the selection of child files must be performed manually (true).
% The definitions initialise to false:
%    \begin{macrocode}
\newif\ifchilddoc
\newif\ifchilddocmanual
%    \end{macrocode}

% \macro{\childdocname}
% \macro{\childdocjob}
% The macro |\childdocname| stores the name of the main document
% to be compiled. The macro |\childdocjob| stores the name of
% the document on which the \LaTeX{} compiler was originally invoked.
% The content of |\jobname| cannot be compared
% to filenames specified in the source due to different catcodes.
% The following code rescans |\jobname|, stores the result
% in |\childdocname| and saves a copy in |\childdocjob|:
%    \begin{macrocode}
\edef\childdocname{\scantokens\expandafter{\jobname\noexpand}}
\let\childdocjob\childdocname
%    \end{macrocode}

% \macro{\childdocdisable}
% The macro |\childdocdisable| prevents the main file
% from being processed more than once.
% At this stage, the main document command |\childdocmain|
% is assumed to be called once again where it should do nothing.
% Any subsequent call to it should prevent
% a secondary processing of the main document
% It overwrites the forwarding commands
% |\childdocof| and |\childdocforward|
% with empty macros to prevent further inclusions of the main document:
%    \begin{macrocode}
\newcommand{\childdocdisable}
{
  \renewcommand{\childdocmain}[1]{\renewcommand{\childdocmain}[1]{\endinput}}
  \renewcommand{\childdocof}[1]{}
  \renewcommand{\childdocby}[2][]{}
  \renewcommand{\childdocforward}[2][]{}
  \renewcommand{\childdocdisable}{}
}
%    \end{macrocode}

% \macro{\childdocmain}
% The macro |\childdocmain| is to be called at the top of the main file
% with nothing or the main filename (without extension) as argument.
% First, it breaks loops.
% If the argument is not empty and does not match |\childdocname|
% (which is set by the first inclusion of |childdoc.def|),
% |\ifchilddoc| is set to true, |\includeonly| is applied to the child file
% and |\jobname| is set to the main file
% (for proper handling of |.aux| files):
%    \begin{macrocode}
\newcommand{\childdocmain}[1]
{
  \childdocdisable\childdocmain{}
  \if?#1?\else
    \begingroup
      \def\childdoctmp{#1}
      \ifx\childdoctmp\childdocname
        \def\childdoctmp{}
      \else
        \def\childdoctmp
        {
          \childdoctrue
          \includeonly{\childdocname}
          \def\childdocjob{#1}
          \def\jobname{#1}
        }
      \fi
      \expandafter
    \endgroup
    \childdoctmp
  \fi
}
%    \end{macrocode}

% \macro{\childdocof}
% The command |\childdocof| redirects
% compilation to the main file |#1|.
%    \begin{macrocode}
\newcommand{\childdocof}[1]
{
  \childdocdisable
  \childdoctrue
  \includeonly{\childdocname}
  \def\jobname{#1}
  \def\childdocjob{#1}
  \input{#1}
}
%    \end{macrocode}

% \macro{\childdocby}
% The command |\childdocby| ....
%    \begin{macrocode}
\newcommand{\childdocby}[2][]
{
  \childdocdisable
  \childdoctrue
  \childdocmanualtrue
  \if?#1?\else
    \def\jobname{#2}
  \fi
  \def\childdocjob{#2}
  \input{#2}
  \endinput
}
%    \end{macrocode}

% \macro{\childdocforward}
% The command |\childdocforward| redirects
% compilation to the main file or
% (if the optional argument is given) a child file.
% Parameters are set as if the main file
% or a child file starting with |\childdocof| was compiled.
% Then compilation is handed over to the main file:
%    \begin{macrocode}
\newcommand{\childdocforward}[2][]
{
  \begingroup
    \if?#1?
      \def\childdoctmp
      {
        \def\childdocname{#2}
        \def\childdocjob{#2}
        \def\jobname{#2}
        \input{#2}
        \endinput
      }
    \else
      \def\childdoctmp
      {
        \childdocdisable
        \def\childdocname{#2}
        \childdoctrue
        \includeonly{#2}
        \def\childdocjob{#1}
        \def\jobname{#1}
        \input{#1}
        \endinput
      }
    \fi
    \expandafter
  \endgroup
  \childdoctmp
}
%    \end{macrocode}

% \macro{\childdocforwardprefix}
% The command |\childdocforwardprefix| redirects
% compilation to the main or a child file by means of a pattern.
% The prefix |#1| in the current filename is replaced by |#2|
% and the suffix of the current filename is kept
% (it is assumed that the filename does not contain the substring `|~~~|'
% which is used as a delimiter).
% Compilation is handed over to the new file by |\childdocforward|:
%    \begin{macrocode}
\newcommand{\childdocforwardprefix}[3][]
{
  \begingroup
    \def\childdocextract #2##1~~~{\def\childdoctmp{\childdocforward[#1]{#3##1}}}
    \expandafter\childdocextract\childdocname~~~
    \expandafter
  \endgroup
  \childdoctmp
}
%    \end{macrocode}

% \macro{\childdoc}
% The deprecated macro |\childdoc| is a legacy version of |\childdocmain|:
%    \begin{macrocode}
\newcommand{\childdoc}{\childdocmain}
%    \end{macrocode}

% \macro{\childdocredirect}
% The deprecated macro |\childdocredirect| is a legacy version
% of |\childdocforward| and |\childdocforwardprefix|:
%    \begin{macrocode}
\newcommand{\childdocredirect}[2][]
{
  \begingroup
    \if?#1?
      \def\childdoctmp{\childdocforward{#2}}
    \else
      \def\childdoctmp{\childdocforwardprefix{#1}{#2}}
    \fi
    \expandafter
  \endgroup
  \childdoctmp
}
%    \end{macrocode}

%\iffalse
%</package>
%\fi
%
\endinput
|\\
|\childdocforward[|\textit{main}|]{|\textit{dest}|}|\\
\end{tabular}
\end{center}
%
The argument \textit{dest} is the destination file
(without extension).
It should be the main file or one of the child files.
Note that further \textsf{childdoc} directives
such as |\childdocof| and |\childdocforward|
in the indicated file will be processed in this form.
The optional argument \textit{main}
passes on directly to the main file \textit{main}
while pretending to compile the child \textit{dest}.
This form behaves as if \textit{dest}
issues |\childdocof{|\textit{main}|}| right away,
and no further \textsf{childdoc} directives will be processed.

%%%%%%%%%%%%%%%%%%%%%%%%%%%%%%%%%%%%%%%%
\DescribeMacro{\...prefix}
In the alternative form |\childdocforwardprefix|,
%
\begin{center}
\begin{tabular}{l}
|% \iffalse
%
% childdoc.dtx Copyright (C) 2017-2018 Niklas Beisert
%
% This work may be distributed and/or modified under the
% conditions of the LaTeX Project Public License, either version 1.3
% of this license or (at your option) any later version.
% The latest version of this license is in
%   http://www.latex-project.org/lppl.txt
% and version 1.3 or later is part of all distributions of LaTeX
% version 2005/12/01 or later.
%
% This work has the LPPL maintenance status `maintained'.
%
% The Current Maintainer of this work is Niklas Beisert.
%
% This work consists of the files childdoc.dtx and childdoc.ins
% and the derived files childdoc.def and cdocsamp.tex with
% cdocsch1.tex, cdocsch2.tex, cdocsdrf.tex, cdocsfn1.tex, cdocsfn2.tex.
%
%<package>\ifdefined\childdocmain\endinput\fi
%<package>\ProvidesFile{childdoc.def}[2018/12/30 v2.0 child document driver]
%<samplemain>\ProvidesFile{cdocsamp.tex}[2018/12/30 v2.0 sample for childdoc]
%<*driver>
%\ProvidesFile{childdoc.drv}[2018/12/30 v2.0 childdoc reference manual file]
\PassOptionsToClass{10pt,a4paper}{article}
\documentclass{ltxdoc}

\usepackage[margin=35mm]{geometry}
\usepackage{hyperref}
\usepackage{hyperxmp}
\usepackage[usenames]{color}

\hypersetup{colorlinks=true}
\hypersetup{pdfstartview=FitH}
\hypersetup{pdfpagemode=UseNone}
\hypersetup{pdfsource={}}
\hypersetup{pdflang={en-UK}}
\hypersetup{pdfcopyright={Copyright 2017-2018 Niklas Beisert.
  This work may be distributed and/or modified under the
  conditions of the LaTeX Project Public License, either version 1.3
  of this license or (at your option) any later version.}}
\hypersetup{pdflicenseurl={http://www.latex-project.org/lppl.txt}}
\hypersetup{pdfcontactaddress={ETH Zurich, ITP, HIT K,
  Wolfgang-Pauli-Strasse 27}}
\hypersetup{pdfcontactpostcode={8093}}
\hypersetup{pdfcontactcity={Zurich}}
\hypersetup{pdfcontactcountry={Switzerland}}
\hypersetup{pdfcontactemail={nbeisert@itp.phys.ethz.ch}}
\hypersetup{pdfcontacturl={http://people.phys.ethz.ch/\xmptilde nbeisert/}}

\newcommand{\secref}[1]{\hyperref[#1]{section \ref*{#1}}}

\parskip1ex
\parindent0pt
\let\olditemize\itemize
\def\itemize{\olditemize\parskip0pt}

\begin{document}

\title{The \textsf{childdoc} Package}
\hypersetup{pdftitle={The childdoc Package}}
\author{Niklas Beisert\\[2ex]
  Institut f\"ur Theoretische Physik\\
  Eidgen\"ossische Technische Hochschule Z\"urich\\
  Wolfgang-Pauli-Strasse 27, 8093 Z\"urich, Switzerland\\[1ex]
  \href{mailto:nbeisert@itp.phys.ethz.ch}
  {\texttt{nbeisert@itp.phys.ethz.ch}}}
\hypersetup{pdfauthor={Niklas Beisert}}
\hypersetup{pdfsubject={Manual for the LaTeX2e Package childdoc}}
\date{30 December 2018, \textsf{v2.0}}
\maketitle

\begin{abstract}\noindent
\textsf{childdoc} is a \LaTeXe{} package
that enables the direct compilation
of document sections included by |\include|
to individual files.
\end{abstract}

\begingroup
\parskip0ex
\tableofcontents
\endgroup

%%%%%%%%%%%%%%%%%%%%%%%%%%%%%%%%%%%%%%%%%%%%%%%%%%%%%%%%%%%%%%%%%%%%%%%%%%%%%%%%
%%%%%%%%%%%%%%%%%%%%%%%%%%%%%%%%%%%%%%%%%%%%%%%%%%%%%%%%%%%%%%%%%%%%%%%%%%%%%%%%
\section{Introduction}

\LaTeX{} provides a mechanism to structure a large document (such as a book)
into a main file and several child files (containing the chapters)
using the |\include| command.
This mechanism is beneficial for documents
which span hundreds of pages in order to
make the source file(s) more manageable.
Moreover, compilation can be restricted to
selected child files by means of the |\includeonly| command.
The latter feature can be used to reduce the compilation time while editing
(this was significantly more useful in the earlier days of \LaTeX{})
or to generate a smaller document which is easier to navigate.
Another application of |\includeonly| is to generate
documents consisting of selected parts of the complete document.

However, there are a few drawbacks of the plain |\include| mechanism:
\begin{itemize}
\item
The child files cannot be compiled on their own,
they can only be compiled via the main file.
A naive editing environment
(such as a text editor with an option
to have the current file processed by \LaTeX)
may require one to switch to the main file before compiling;
attempting to compile the child file produces errors.
\item
The main file must be modified (each time)
to adjust the |\includeonly| command
to the present needs. This easily leaves the main file in a messy state.
\item
The generated document will always carry the filename
of the main document. This is inconvenient if
several child files are to be compiled and
to be kept for distribution.
\end{itemize}

The present package provides a simple interface
to make child files individually compilable by \LaTeX{}.
Compiling a child file then has the same effect as compiling
the main file with an |\includeonly| command
to select the appropriate child.
Moreover the generated document will carry the name of the child
rather than the main file.
This resolves all three above issues.

This feature is meant to make the editing of books,
thesis documents and lecture notes somewhat more convenient.
However, the package can also be used efficiently for
composing a series of documents (such as exercise sheets)
which are typically distributed individually.
It then assists the author in generating the individual documents
(potentially in different versions)
as well as a document containing the collected series.
Another application is in developing style files
or other kinds of included material
where compilation of the style file could redirect
to a sample or test file.

%%%%%%%%%%%%%%%%%%%%%%%%%%%%%%%%%%%%%%%%%%%%%%%%%%%%%%%%%%%%%%%%%%%%%%%%%%%%%%%%
%%%%%%%%%%%%%%%%%%%%%%%%%%%%%%%%%%%%%%%%%%%%%%%%%%%%%%%%%%%%%%%%%%%%%%%%%%%%%%%%
\section{Usage}

First of all, the package \textsf{childdoc} is \emph{not} a standard
\LaTeXe{} |.sty| style file! Therefore it needs to be invoked in
a non-standard way.

%%%%%%%%%%%%%%%%%%%%%%%%%%%%%%%%%%%%%%%%%%%%%%%%%%%%%%%%%%%%%%%%%%%%%%%%%%%%%%%%
\subsection{Included Files}
\label{sec:include}

%%%%%%%%%%%%%%%%%%%%%%%%%%%%%%%%%%%%%%%%
\DescribeMacro{\childdocmain}
To use the package, add the commands
\begin{center}
\begin{tabular}{l}
|\input{childdoc.def}|\\
|\childdocmain{}|\\
\end{tabular}
\end{center}
at the very top of the main \LaTeX{} file,
in particular \emph{before} the |\documentclass| statement!
The argument of |\childdocmain| should be left empty
(but it must be present).

%%%%%%%%%%%%%%%%%%%%%%%%%%%%%%%%%%%%%%%%
\DescribeMacro{\childdocof}
Furthermore, add the commands
\begin{center}
\begin{tabular}{l}
|\input{childdoc.def}|\\
|\childdocof{|\textit{main}|}|\\
\end{tabular}
\end{center}
at the top of every child file \textit{child}
which is included by |\include{|\textit{child}|}|
from within the main file
(or at least for those files to be compiled individually).
The argument \textit{main} must be the filename of the main file.

There are a couple of
considerations in setting up the main and child documents:

%%%%%%%%%%%%%%%%%%%%%%%%%%%%%%%%%%%%%%%%
\paragraph{Restrictions.}

Please note the following restrictions:
\begin{itemize}
\item
|\childdocmain| must be called with one argument \textit{main}
to ensure compatibility with earlier version of the package.
It must either be empty (|\childdocmain{}|)
or precisely match the filename of the main file in which it is specified.
See \secref{sec:detection} for further information.
\item
The filename \textit{main} must be specified without the |.tex| extension.
\item
The filename \textit{main} is case sensitive
(even in case-insensitive file systems)
due to internal string comparison.
\item
The argument \textit{main} should be fully expanded, it cannot be a macro.
\item
Subdirectories and special characters should be avoided in filenames.
\item
The command |\childdocmain{|\textit{main}|}| must be followed by a whitespace.
It should not be followed immediately by another command
or by a comment mark `|%|'.
This is because the \TeX{} parser reads the token immediately following
the argument of |\childdocmain| and puts it
at the beginning of every child section;
however, a white\-space is ignored.
\end{itemize}

%%%%%%%%%%%%%%%%%%%%%%%%%%%%%%%%%%%%%%%%
\paragraph{Content of Main File.}

It is advisable to place all content in the child files included by |\include|.
Any output contained in the main file will appear in all child documents
unless suppressed manually;
it cannot be suppressed automatically by the |\includeonly| directive
and thus should normally be avoided.
A method to include some content in the main file
by means of conditional processing is described in \secref{sec:conditional}.

%%%%%%%%%%%%%%%%%%%%%%%%%%%%%%%%%%%%%%%%
\paragraph{Page Numbering.}

When only a part of the document is compiled,
the appropriate numbering of pages
(as well as other status parameters)
is determined from the |.aux| files.
The latter contain information from previous passes.
However this information needs to propagate through
all intermediate child documents.
Therefore the page numbering in child documents may well
be inconsistent until the complete document is compiled at least once.

A useful (if unconventional) way to always ensure a consistent
page numbering is to restart the numbering in each child document
and denote the pages by `\textit{child}|.|\textit{page}'
where \textit{child} represents the chapter/section number of the child file.
This can be achieved by the command
|\numberwithin{page}{|\textit{child}|}|
of the \textsf{amsmath} package
where \textit{child} can be |chapter| or |section|
depending on the chosen structuring.
Alternatively, one can modify the macro |\thepage| appropriately
and reset the counter |page| at the start of each child file.

%%%%%%%%%%%%%%%%%%%%%%%%%%%%%%%%%%%%%%%%%%%%%%%%%%%%%%%%%%%%%%%%%%%%%%%%%%%%%%%%
\subsection{Conditional Processing}
\label{sec:conditional}

The package provides a mechanism to compile different versions
of a document. To customise the versions further some conditional processing
can come in handy to distinguish which version is being compiled.
The package provides two macros to describe the compilation context:

%%%%%%%%%%%%%%%%%%%%%%%%%%%%%%%%%%%%%%%%
\DescribeMacro{\ifchilddoc}
The conditional |\ifchilddoc| distinguishes between the compilation of
child documents and the main document:
%
\begin{center}
|\ifchilddoc |\textit{child-code}| |[|\||else |\textit{main-code}]| \||fi|
\end{center}

%%%%%%%%%%%%%%%%%%%%%%%%%%%%%%%%%%%%%%%%
\DescribeMacro{\childdocname}
\DescribeMacro{\childdocjob}
The macro |\childdocname| contains the filename (without extension)
of the main or child file being processed.
Note that |\childdocjob| will always contain the name of the main file.

%%%%%%%%%%%%%%%%%%%%%%%%%%%%%%%%%%%%%%%%
\paragraph{Title Page.}

Conditional processing can be used to include a title or banner page
in the main document when proper precautions are taken.
Importantly, the code in the main file should ensure that the page counter
(as well as other status parameters which are stored in the |.aux| files)
takes the same value after the conditional processing.
Otherwise the page numbers may take divergent values
depending on which part is compiled.

For example, a title page could be declared by:
%
\begin{center}
\begin{tabular}{l}
|\ifchilddoc\||else|\\
|\addtocounter{page}{-1}|\\
\textit{code for title page}\\
|\newpage|\\
|\||fi|
\end{tabular}
\end{center}
%
A banner page for the child documents can be generated by:
%
\begin{center}
\begin{tabular}{l}
|\ifchilddoc|\\
|\addtocounter{page}{-1}|\\
\textit{code for banner page}\\
|\newpage|\\
|\||fi|
\end{tabular}
\end{center}
%
Here one could write a message such as:
\begin{center}
|This is the part \childdocname{} of \childdocjob{}.|
\end{center}

%%%%%%%%%%%%%%%%%%%%%%%%%%%%%%%%%%%%%%%%%%%%%%%%%%%%%%%%%%%%%%%%%%%%%%%%%%%%%%%%
\subsection{Flags}
\label{sec:flags}

The package makes it easy to generate different versions
of the main or child documents.
To this end compilation flags can be defined
and assigned different default values.
They will be particularly useful in conjunction
with the forwarding mechanism described in \secref{sec:forward}.

For example, it may be useful to have a flag |\version|
which can be set to |draft| or |final|.
The document source will contain some conditional code
depending on the value of |\version|.
Suppose further, the flag should default to |final| for the main file
and to |draft| for child files
which is a natural assignment for editing the document.
This is achieved by placing the following code
in the preamble of the main document
(below the |\childdocmain| directive):
%
\begin{center}
\begin{tabular}{l}
|\ifchilddoc|\\
|\providecommand{\version}{draft}|\\
|\||else|\\
|\providecommand{\version}{final}|\\
|\||fi|
\end{tabular}
\end{center}
%
The definition by |\providecommand| makes sure
that previous definitions are not overwritten.
Further statements |\providecommand{\version}{...}|
can thus be added before the above code to override it.

For the main file, one might add a line
(between |\childdocmain| and the above block)
%
\begin{center}
|%\ifchilddoc\||else\providecommand{\version}{draft}\||fi|
\end{center}
%
which can be uncommented to produce a draft version.
Likewise one can add a line to the very top of a child file
(above the |\childdocof{|\textit{main}|}| directive)
%
\begin{center}
|%\providecommand{\version}{final}|
\end{center}
%
which can be uncommented to produce the final version of this child document.

%%%%%%%%%%%%%%%%%%%%%%%%%%%%%%%%%%%%%%%%%%%%%%%%%%%%%%%%%%%%%%%%%%%%%%%%%%%%%%%%
\subsection{Forwarding}
\label{sec:forward}

Different versions of the main or child documents
using compilation flags as described in \secref{sec:flags}
can be (permanently) stored in different files
for convenient compilation, viewing and distribution.
To this end, the package defines a command
to pass on compilation to a different file:

%%%%%%%%%%%%%%%%%%%%%%%%%%%%%%%%%%%%%%%%
\DescribeMacro{\childdocforward}
The command |\childdocforward| redirects processing to
another source file:
%
\begin{center}
\begin{tabular}{l}
|\input{childdoc.def}|\\
|\childdocforward[|\textit{main}|]{|\textit{dest}|}|\\
\end{tabular}
\end{center}
%
The argument \textit{dest} is the destination file
(without extension).
It should be the main file or one of the child files.
Note that further \textsf{childdoc} directives
such as |\childdocof| and |\childdocforward|
in the indicated file will be processed in this form.
The optional argument \textit{main}
passes on directly to the main file \textit{main}
while pretending to compile the child \textit{dest}.
This form behaves as if \textit{dest}
issues |\childdocof{|\textit{main}|}| right away,
and no further \textsf{childdoc} directives will be processed.

%%%%%%%%%%%%%%%%%%%%%%%%%%%%%%%%%%%%%%%%
\DescribeMacro{\...prefix}
In the alternative form |\childdocforwardprefix|,
%
\begin{center}
\begin{tabular}{l}
|\input{childdoc.def}|\\
|\childdocforwardprefix[|\textit{main}|]{|\textit{prefix}|}{|\textit{dest}|}|
\end{tabular}
\end{center}
%
the destination file is determined by a pattern
depending on the current file:
To make this work, the current file must be called
`{\textit{prefix}\hspace{0.2em}\textit{suffix}}'
with \textit{prefix} matching precisely the argument.
Processing is then passed on to the file
`{\textit{dest}\hspace{0.2em}\textit{suffix}}'.
Surely, the same effect is achieved by
directly specifying the
argument `{\textit{dest}\hspace{0.2em}\textit{suffix}}'
in the first form.
However, that requires to set up a different file
for each child. With the alternative form of the command
all these files can have exactly the same content
which simplifies setting them up and maintaining them.

For example, the following file |draft.tex|
with a compilation flag |\version| as described in \secref{sec:flags}
compiles the main document as a draft:
%
\begin{center}
\begin{tabular}{l}
|\def\version{draft}|\\
|\input{childdoc.def}|\\
|\childdocforward{|\textit{main}|}|
\end{tabular}
\end{center}
%
Likewise, the following files |final|\textit{nn}|.tex|
compile the final version of the child document
|child|\textit{nn}|.tex|:
%
\begin{center}
\begin{tabular}{l}
|\def\version{final}|\\
|\input{childdoc.def}|\\
|\childdocforwardprefix{final}{child}|
\end{tabular}
\end{center}
%

Note that when several versions of a main file and/or of each child file
are to be generated, it may be convenient to set up a |Makefile| or
shell script to automatise the process.

%%%%%%%%%%%%%%%%%%%%%%%%%%%%%%%%%%%%%%%%%%%%%%%%%%%%%%%%%%%%%%%%%%%%%%%%%%%%%%%%
\subsection{Command Line Processing}
\label{sec:commandline}

The effect of redirection files can also be achieved by invoking
the \LaTeX{} compiler with a more elaborate command line.
Most conveniently this should be done as part
of a shell script or a |Makefile|.

When using \textsf{childdoc} in the main file, the following
command lines effectively perform a redirection
(note that depending on the shell being used,
backslashes may have to be doubled: `|\|' $\to$ `|\\|'):
%
\begin{center}
|... -jobname "|\textit{target}|" |\\|"|[\textit{flags}]%
|\input{childdoc.def}\childdocforward[|\textit{main}|]{|\textit{dest}|}"|
\end{center}
%
Here \textit{target} is the name of the output file,
\textit{main} is the name of the main file
and \textit{dest} is the name of the main or child file to be processed
(all filenames without extensions).
The optional argument \textit{main} can be omitted
if \textit{main} matches \textit{dest}.
Optionally, compilation \textit{flags} can be defined via |\def| commands.
This command line makes the \TeX{} engine believe
it is compiling the file \textit{target}
whose content is specified as the latter parameter.
The provided code then forwards the processing to
\textit{main} or \textit{dest} as described in \secref{sec:forward}.

%%%%%%%%%%%%%%%%%%%%%%%%%%%%%%%%%%%%%%%%%%%%%%%%%%%%%%%%%%%%%%%%%%%%%%%%%%%%%%%%
\subsection{Include by Input}
\label{sec:input}

Including child documents by |\include| has some restrictions by design.
Most notably, the content of a child document always occupies
its own set of pages; pages cannot be shared between child documents.
Usually, this behaviour makes perfect sense
because each child document contain an essential part of the document.
However, in some situations it may be desirable to compose
a document from a collection of parts
without having mandatory page breaks between then.
For this case, the package
provides a mechanism to include parts
by |\input| which can also be processed individually.
However, by construction this mechanism
requires manual handling of the content to be output.

%%%%%%%%%%%%%%%%%%%%%%%%%%%%%%%%%%%%%%%%
\DescribeMacro{\ifchilddocmanual}
The main file should be prepared as usual, see \secref{sec:include}.
However, the document body must make a distinction
between processing of an individual part and of the main document, e.g.:
%
\begin{center}
\begin{tabular}{l}
|\ifchilddocmanual|\\
|\input{\childdocname}|\\
|\||else|\\
\textit{document body with }|\input{|\textit{part}|}|\\
|\||fi|
\end{tabular}
\end{center}
%
The conditional |\ifchilddocmanual| is true whenever
a part to be included by |\input| is being compiled,
and the name of the part is stored in |\childdocname|.

%%%%%%%%%%%%%%%%%%%%%%%%%%%%%%%%%%%%%%%%
\DescribeMacro{\childdocby}
Each part to be included by |\input| should start with:
%
\begin{center}
\begin{tabular}{l}
|\input{childdoc.def}|\\
|\childdocby{|\textit{main}|}|\\
\end{tabular}
\end{center}
%
The directive |\childdocby| is similar to |\childdocof|
described in \secref{sec:include},
but the subsequent selection of content must be done manually.
To that end, both |\ifchilddoc| and |\ifchilddocmanual|
will be true upon processing of a part,
and the name of the part is stored in |\childdocname|.
Note that |\jobname| will be set to the filename of the current part
so that each part receives an individual |.aux| file
that does not interfere with the |.aux| file(s) of the main document.
This behaviour can be altered by the alternative form
|\childdocby[*]{|\textit{main}|}| (with a non-empty optional argument)
which uses the |.aux| file of the main document
by setting |\jobname| to \textit{main}.

%%%%%%%%%%%%%%%%%%%%%%%%%%%%%%%%%%%%%%%%%%%%%%%%%%%%%%%%%%%%%%%%%%%%%%%%%%%%%%%%
\subsection{Driver Development}
\label{sec:driver}

The \textsf{childdoc} mechanism can also be use for the development
of definition files such as \LaTeX{} styles or classes.
This case differs from the above setup with multiple parts
included by |\include| in that no |\includeonly| should be invoked.
This can be achieved by starting the include file
(before |\ProvidesPackage|) with:
%
\begin{center}
\begin{tabular}{l}
|\input{childdoc.def}|\\
|\childdocforward{|\textit{main}|}|\\
\end{tabular}
\end{center}
%
or alternatively with:
%
\begin{center}
\begin{tabular}{l}
|\input{childdoc.def}|\\
|\childdocby{|\textit{main}|}|\\
\end{tabular}
\end{center}
%
Both forms have slightly different effects as described above.
The main file is prepared as usual, see \secref{sec:include}.

%%%%%%%%%%%%%%%%%%%%%%%%%%%%%%%%%%%%%%%%%%%%%%%%%%%%%%%%%%%%%%%%%%%%%%%%%%%%%%%%
\subsection{Legacy Detection}
\label{sec:detection}

The directive |\childdocmain| in the main file can detect
whether the complete document or merely a child is to be compiled
even without using the directive |\childdocof|.
This method is deprecated because it is less robust
and there is no compelling reason to use it;
it is merely provided for backward compatibility
and it may be removed in future versions.

If the detection mechanism is to be used,
it is mandatory to correctly specify
the filename of the main file as the argument of |\childdocmain|:
%
\begin{center}
\begin{tabular}{l}
|\input{childdoc.def}|\\
|\childdocmain{|\textit{main}|}|\\
\end{tabular}
\end{center}
%
If |\jobname| does not match the argument \textit{main} of |\childdocmain|,
it is assumed that |\jobname| points to the child file to be compiled.
When using |\childdocmain| with the main file specified as argument,
it suffices to start a child file
with just |\input{|\textit{main}|}|
without loading of the package and using |\childdocof|.
If instead all processing is done
with the appropriate \textsf{childdoc} directives,
the argument of \textit{main} of |\childdocmain| can be empty.

An alternative version of the command line processing described
in \secref{sec:commandline} using the detection mechanism reads:
%
\begin{center}
|... -jobname "|\textit{target}|" "|[\textit{flags}]%
[|\def\jobname{|\textit{dest}|}|]|\input{|\textit{main}|}"|
\end{center}

%%%%%%%%%%%%%%%%%%%%%%%%%%%%%%%%%%%%%%%%%%%%%%%%%%%%%%%%%%%%%%%%%%%%%%%%%%%%%%%%
\subsection{Manual Code}
\label{sec:manual}

In case one cannot be certain whether the definitions file |childdoc.def|
is installed on the target \TeX{} distribution
and one prefers not to ship it,
it is conceivable to paste a few relevant commands into the sources.

To that end, drop all statements |\input{childdoc.def}|
and perform the replacements as outlined below.
Instead of |\childdocmain{|\textit{main}|}| add the following code
to the top of the main file:
%
\begin{center}
\begin{tabular}{l}
|\||ifdefined\childdocname\endinput\||fi\newif\ifchilddoc|\\
|\edef\childdocname{\scantokens\expandafter{\jobname\noexpand}}|\\
|\def\childdocmain{|\textit{main}|}\||ifx\childdocmain\childdocname\||else|\\
|\childdoctrue\includeonly{\childdocname}\let\jobname\childdocmain\||fi|\\
\end{tabular}
\end{center}
%
Instead of |\childdocof{|\textit{main}|}| just include the main file
at the top of each child file:
%
\begin{center}
|\input{|\textit{main}|}|
\end{center}
%
A simple redirection |\childdocforward{|\textit{dest}|}| is achieved by:
%
\begin{center}
|\def\jobname{|\textit{dest}|}\input{\jobname}|
\end{center}
%
The redirection with prefix
|\childdocforwardprefix[|\textit{prefix}|]{|\textit{dest}|}|
is accomplished by:
%
\begin{center}
\begin{tabular}{l}
|{\edef\jobname{\scantokens\expandafter{\jobname\noexpand}}|\\
|\def\redirectjob |\textit{prefix}|#1~~~{\gdef\jobname{|\textit{dest}|#1}}|\\
|\expandafter\redirectjob\jobname~~~}\input{\jobname}|
\end{tabular}
\end{center}

In an alternative approach,
child documents can be compiled by a specific command line
without additional code or specific definitions:
%
\begin{center}
|... -jobname "|\textit{target}|" "|[\textit{flags}]%
|\includeonly{|\textit{dest}|}\input{|\textit{main}|}"|
\end{center}
%

%%%%%%%%%%%%%%%%%%%%%%%%%%%%%%%%%%%%%%%%%%%%%%%%%%%%%%%%%%%%%%%%%%%%%%%%%%%%%%%%
%%%%%%%%%%%%%%%%%%%%%%%%%%%%%%%%%%%%%%%%%%%%%%%%%%%%%%%%%%%%%%%%%%%%%%%%%%%%%%%%
\section{Information}

%%%%%%%%%%%%%%%%%%%%%%%%%%%%%%%%%%%%%%%%%%%%%%%%%%%%%%%%%%%%%%%%%%%%%%%%%%%%%%%%
\subsection{Copyright}

Copyright \copyright{} 2017--2018 Niklas Beisert

This work may be distributed and/or modified under the
conditions of the \LaTeX{} Project Public License, either version 1.3
of this license or (at your option) any later version.
The latest version of this license is in
  \url{http://www.latex-project.org/lppl.txt}
and version 1.3 or later is part of all distributions of \LaTeX{}
version 2005/12/01 or later.

This work has the LPPL maintenance status `maintained'.

The Current Maintainer of this work is Niklas Beisert.

This work consists of the files |README.txt|, |childdoc.ins| and |childdoc.dtx|
as well as the derived files |childdoc.def|, |cdocsamp.tex|
with |cdocsch1.tex|, |cdocsch2.tex|, |cdocspt3.tex|, |cdocspt4.tex|,
|cdocsdrf.tex|, |cdocsfn1.tex|, |cdocsfn2.tex|
as well as |childdoc.pdf|.

%%%%%%%%%%%%%%%%%%%%%%%%%%%%%%%%%%%%%%%%%%%%%%%%%%%%%%%%%%%%%%%%%%%%%%%%%%%%%%%%
\subsection{Files and Installation}

The package consists of the files:
%
\begin{center}
\begin{tabular}{ll}
    |README.txt|   & readme file \\
    |childdoc.ins| & installation file \\
    |childdoc.dtx| & source file \\
    |childdoc.def| & definition file \\
    |cdocsamp.tex| & sample main file \\
    |cdocsch1.tex| & sample include file \\
    |cdocsch2.tex| & sample include file \\
    |cdocspt3.tex| & sample part file \\
    |cdocspt4.tex| & sample part file \\
    |cdocsdrf.tex| & sample redirection file \\
    |cdocsfn1.tex| & sample redirection file \\
    |cdocsfn2.tex| & sample redirection file \\
    |childdoc.pdf| & manual
\end{tabular}
\end{center}
%
The distribution consists of the files
|README.txt|, |childdoc.ins| and |childdoc.dtx|.
%
\begin{itemize}
\item
Run (pdf)\LaTeX{} on |childdoc.dtx|
to compile the manual |childdoc.pdf| (this file).
\item
Run \LaTeX{} on |childdoc.ins| to create the definitions file |childdoc.def|
and the sample |cdocsamp.tex| with include files
|cdocsch1.tex|, |cdocsch2.tex|, |cdocspt3.tex|, |cdocspt4.tex|,
|cdocsdrf.tex|, |cdocsfn1.tex|, |cdocsfn2.tex|.
Then copy the file |childdoc.def| to an appropriate directory of your \LaTeX{}
distribution, e.g.\ \textit{texmf-root}|/tex/latex/childdoc|.
\end{itemize}

%%%%%%%%%%%%%%%%%%%%%%%%%%%%%%%%%%%%%%%%%%%%%%%%%%%%%%%%%%%%%%%%%%%%%%%%%%%%%%%%
\subsection{Related CTAN Packages}

There are several other packages which offer a similar functionality:
%
\begin{itemize}
\item
The packages
\href{http://ctan.org/pkg/docmute}{\textsf{docmute}},
\href{http://ctan.org/pkg/includex}{\textsf{includex}} and
\href{http://ctan.org/pkg/standalone}{\textsf{standalone}}
provide commands to include only the document body of
a child file thus allowing both files to be compiled individually.
\item
The packages \href{http://ctan.org/pkg/subdocs}{\textsf{subdocs}}
and \href{http://ctan.org/pkg/subfiles}{\textsf{subfiles}}
provide structures in which the main and child documents can be
encapsulated and allowing them to be compiled individually.
The inclusion mechanism is different from the conventional |\include|.
\item
The package \href{http://ctan.org/pkg/combine}{\textsf{combine}}
is an elaborate solution to combine several documents into one.
\end{itemize}
%
See also the CTAN topic \href{http://ctan.org/topic/subdocs}{\textsf{subdocs}}
for further related packages.
The present package differs from the above solutions in that
a document structure constructed with the conventional |\include| mechanism
just needs two extra commands at the top of every file
such that all constituent files can be compiled individually.

%%%%%%%%%%%%%%%%%%%%%%%%%%%%%%%%%%%%%%%%%%%%%%%%%%%%%%%%%%%%%%%%%%%%%%%%%%%%%%%%
%\subsection{Feature Suggestions}
%
%The following is a list of features which may be useful for future
%versions of this package:
%%
%\begin{itemize}
%\item
%\ldots
%\end{itemize}

%%%%%%%%%%%%%%%%%%%%%%%%%%%%%%%%%%%%%%%%%%%%%%%%%%%%%%%%%%%%%%%%%%%%%%%%%%%%%%%%
\subsection{Revision History}

%%%%%%%%%%%%%%%%%%%%%%%%%%%%%%%%%%%%%%%%
\paragraph{v2.0:} 2018/12/30

\begin{itemize}
\item
immediate forward processing
\item
added |\childdocby| mechanism
\item
manual restructured
\end{itemize}

%%%%%%%%%%%%%%%%%%%%%%%%%%%%%%%%%%%%%%%%
\paragraph{v1.6:} 2018/01/17

\begin{itemize}
\item
application for development of include files
\item
corrections to manual
\end{itemize}

%%%%%%%%%%%%%%%%%%%%%%%%%%%%%%%%%%%%%%%%
\paragraph{v1.5:} 2017/05/21

\begin{itemize}
\item
more complete structuring introduced
\item
|\childdocof| introduced
\item
|\childdoc| renamed to |\childdocmain|
\item
|\childredirect| renamed to |\childdocforward| and |\childdocforwardprefix|
and functionality expanded
\end{itemize}

%%%%%%%%%%%%%%%%%%%%%%%%%%%%%%%%%%%%%%%%
\paragraph{v1.0:} 2017/04/27

\begin{itemize}
\item
manual and install package
\item
first version published on CTAN
\end{itemize}

%%%%%%%%%%%%%%%%%%%%%%%%%%%%%%%%%%%%%%%%
\paragraph{v0.6:} 2017/04/26

\begin{itemize}
\item
redirection mechanism added
\end{itemize}

%%%%%%%%%%%%%%%%%%%%%%%%%%%%%%%%%%%%%%%%
\paragraph{v0.5:} 2017/04/26

\begin{itemize}
\item
functionality in definition file
\end{itemize}


%%%%%%%%%%%%%%%%%%%%%%%%%%%%%%%%%%%%%%%%%%%%%%%%%%%%%%%%%%%%%%%%%%%%%%%%%%%%%%%%
%%%%%%%%%%%%%%%%%%%%%%%%%%%%%%%%%%%%%%%%%%%%%%%%%%%%%%%%%%%%%%%%%%%%%%%%%%%%%%%%
%%%%%%%%%%%%%%%%%%%%%%%%%%%%%%%%%%%%%%%%%%%%%%%%%%%%%%%%%%%%%%%%%%%%%%%%%%%%%%%%
\appendix

\settowidth\MacroIndent{\rmfamily\scriptsize 000\ }

 \DocInput{childdoc.dtx}

\end{document}
%</driver>
% \fi
%
% %%%%%%%%%%%%%%%%%%%%%%%%%%%%%%%%%%%%%%%%%%%%%%%%%%%%%%%%%%%%%%%%%%%%%%%%%%%%%%
% %%%%%%%%%%%%%%%%%%%%%%%%%%%%%%%%%%%%%%%%%%%%%%%%%%%%%%%%%%%%%%%%%%%%%%%%%%%%%%
% \section{Sample}
%\iffalse
%<*samplemain>
%\fi
%
% The following presents a sample document
% with two chapters, two parts, a title page,
% a compile flag as well as three forwarding files to set the flag.
% It consists of eight |.tex| files:
% \begin{center}
% \begin{tabular}{ll}
% |cdocsamp.tex|&main file\\
% |cdocsch1.tex|&include file for chapter 1\\
% |cdocsch2.tex|&include file for chapter 2\\
% |cdocspt3.tex|&include file for part 3\\
% |cdocspt4.tex|&include file for part 4\\
% |cdocsdrf.tex|&forwarding file for main file in draft mode\\
% |cdocsfi1.tex|&forwarding file for final version of chapter 1\\
% |cdocsfi2.tex|&forwarding file for final version of chapter 2\\
% \end{tabular}
% \end{center}
% Each of the eight files can be compiled directly by the \LaTeX{} compiler.
%
% %%%%%%%%%%%%%%%%%%%%%%%%%%%%%%%%%%%%%%
% \paragraph{Main File.}
%
% The main file is called |cdocsamp.tex|.
%
% Load the \textsf{childdoc} definitions and
% declare the filename for the main document:
%    \begin{macrocode}
\input{childdoc.def}
\childdocmain{}
%    \end{macrocode}

% Optional override for |\version| flag:
%    \begin{macrocode}
%%\ifchilddoc\else\providecommand{\version}{draft}\fi
%    \end{macrocode}

% Define the default values for the |\version| flag
% (|final| for the main file and |draft| for childs):
%    \begin{macrocode}
\ifchilddoc
\providecommand{\version}{draft}
\else
\providecommand{\version}{final}
\fi
%    \end{macrocode}

% Load the standard document class:
%    \begin{macrocode}
\documentclass[12pt]{article}
%    \end{macrocode}

% Start the document body:
%    \begin{macrocode}
\begin{document}
%    \end{macrocode}

% Declare a title page.
% Print title, part of document being processed and version flag:
%    \begin{macrocode}
\addtocounter{page}{-1}
\begin{center}
{\LARGE\bfseries{}childdoc example\par}
\vspace{1cm}
\ifchilddoc
\ifchilddocmanual part\else chapter\fi:
`\childdocname' of `\childdocjob'\par
\else
main document: `\childdocjob'\par
\fi
version: \version\par
\end{center}
\newpage
%    \end{macrocode}

% Manually include selected file,
% otherwise process as usual:
%    \begin{macrocode}
\ifchilddocmanual
\section*{part `\childdocname'}
\input{\childdocname}
\else
%    \end{macrocode}

% Include the two chapters:
%    \begin{macrocode}
\include{cdocsch1}
\include{cdocsch2}
%    \end{macrocode}

% Include the two parts unless only chapters should be displayed:
%    \begin{macrocode}
\ifchilddoc\else
\section{part three}
\input{cdocspt3}
\section{part four}
\input{cdocspt4}
\fi
%    \end{macrocode}

% Process as usual until here:
%    \begin{macrocode}
\fi
%    \end{macrocode}

% End of document body:
%    \begin{macrocode}
\end{document}
%    \end{macrocode}
%\iffalse
%</samplemain>
%\fi
%
% %%%%%%%%%%%%%%%%%%%%%%%%%%%%%%%%%%%%%%
% \paragraph{Chapter Include Files.}
%
% The include files are called |cdocsch1.tex| and |cdocsch2.tex|.
%
%\iffalse
%<*samplechap1|samplechap2>
%\fi

% Optional override for |\version| flag:
%    \begin{macrocode}
%%\providecommand{\version}{final}
%    \end{macrocode}

% Include the main document:
%    \begin{macrocode}
\input{childdoc.def}
\childdocof{cdocsamp}
%    \end{macrocode}

%\iffalse
%</samplechap1|samplechap2>
%\fi
%
%\iffalse
%<*samplechap1>
%\fi
% Some text for chapter 1:
%    \begin{macrocode}
\section{one}
some text in chapter one
%    \end{macrocode}

%\iffalse
%</samplechap1>
%\fi
% Some text for chapter 2:
%\iffalse
%<*samplechap2>
%\fi
%    \begin{macrocode}
\section{two}
more text in chapter two
%    \end{macrocode}

%\iffalse
%</samplechap2>
%\fi
%
% %%%%%%%%%%%%%%%%%%%%%%%%%%%%%%%%%%%%%%
% \paragraph{Part Include Files.}
%
% The include files are called |cdocspt3.tex| and |cdocspt4.tex|.
%
%\iffalse
%<*samplepart3|samplepart4>
%\fi

% Optional override for |\version| flag:
%    \begin{macrocode}
%%\providecommand{\version}{final}
%    \end{macrocode}

% Include the main document:
%    \begin{macrocode}
\input{childdoc.def}
\childdocby{cdocsamp}
%    \end{macrocode}

%\iffalse
%</samplepart3|samplepart4>
%\fi
%
%\iffalse
%<*samplepart3>
%\fi
% Some text for part 3:
%    \begin{macrocode}
some text in part three
%    \end{macrocode}

%\iffalse
%</samplepart3>
%\fi
% Some text for part 4:
%\iffalse
%<*samplepart4>
%\fi
%    \begin{macrocode}
more text in part four
%    \end{macrocode}

%\iffalse
%</samplepart4>
%\fi
%
% %%%%%%%%%%%%%%%%%%%%%%%%%%%%%%%%%%%%%%
% \paragraph{Forwarding for a Complete Draft.}
%
% The following forwarding file |cdocsdrf.tex|
% compiles the main document in draft mode:
%\iffalse
%<*sampledraft>
%\fi
%    \begin{macrocode}
\def\version{draft}
\input{childdoc.def}
\childdocforward{cdocsamp}
%    \end{macrocode}

%\iffalse
%</sampledraft>
%\fi
%
% %%%%%%%%%%%%%%%%%%%%%%%%%%%%%%%%%%%%%%
% \paragraph{Forwarding for Final Version of the Chapters.}
%
% The following forwarding files |cdocsfn1.tex| and |cdocsfn2.tex|
% (with identical content)
% compile the final versions of the child documents
% |cdocsch1.tex| and |cdocsch2.tex|, respectively:
%\iffalse
%<*samplefinal>
%\fi
%    \begin{macrocode}
\def\version{final}
\input{childdoc.def}
\childdocforwardprefix[cdocsamp]{cdocsfn}{cdocsch}
%    \end{macrocode}

%\iffalse
%</samplefinal>
%\fi
%
% %%%%%%%%%%%%%%%%%%%%%%%%%%%%%%%%%%%%%%
% \paragraph{Command Line Processing.}
%
% The following three command lines generate the output files
% |cdocscld|, |cdocscl1| and |cdocscl2|
% which should be identical to
% |cdocsdrf|, |cdocsch1| and |cdocsfn2|, respectively:
% \begin{center}
% \begin{tabular}{l}
% |latex -jobname cdocscld \|\\
% |  "\def\version{draft}\input{childdoc.def}\childdocforward{cdocsamp}"|\\
% |latex -jobname cdocscl1 \|\\
% |  "\input{childdoc.def}\childdocforward[cdocsamp]{cdocsch1}"|\\
% |latex -jobname cdocscl2 \|\\
% |  "\def\version{final}\input{childdoc.def}\childdocforward{cdocsch2}"|
% \end{tabular}
% \end{center}
% Note that the trailing backslash on each first line
% merely continues the input to the second line
% (for convenient cut ant paste).
% Furthermore, the command |latex| can be replaced by any
% of its alternative versions such as |pdflatex|.
%
% %%%%%%%%%%%%%%%%%%%%%%%%%%%%%%%%%%%%%%%%%%%%%%%%%%%%%%%%%%%%%%%%%%%%%%%%%%%%%%
% %%%%%%%%%%%%%%%%%%%%%%%%%%%%%%%%%%%%%%%%%%%%%%%%%%%%%%%%%%%%%%%%%%%%%%%%%%%%%%
% \section{Implementation}
%\iffalse
%<*package>
%\fi
%
% This section describes the definitions file |childdoc.def|.

% The definitions cannot be loaded using |\usepackage| or |\RequirePackage|
% which has a mechanism to prevent loading a style file more than once.
% When loading the definitions by means of |\input|
% multiple instances have to be prevented manually:
%\iffalse
%This code needs to be before the `\ProvidesFile' directive
%which is defined at the beginning of this file.
%Therefore it is also placed there and commented out here.
%</package>
%<*discard>
%\fi
%    \begin{macrocode}
\ifdefined\childdocmain\endinput\fi
%    \end{macrocode}
%\iffalse
%</discard>
%<*package>
%\fi
%
% \macro{\ifchilddoc}
% \macro{\ifchilddocmanual}
% The conditional |\ifchilddoc| tells whether a
% child (true) or main (false) document is being compiled.
% The conditional |\ifchilddocmanual| tells whether
% the |\includeonly| mechanism is used (false) or
% the selection of child files must be performed manually (true).
% The definitions initialise to false:
%    \begin{macrocode}
\newif\ifchilddoc
\newif\ifchilddocmanual
%    \end{macrocode}

% \macro{\childdocname}
% \macro{\childdocjob}
% The macro |\childdocname| stores the name of the main document
% to be compiled. The macro |\childdocjob| stores the name of
% the document on which the \LaTeX{} compiler was originally invoked.
% The content of |\jobname| cannot be compared
% to filenames specified in the source due to different catcodes.
% The following code rescans |\jobname|, stores the result
% in |\childdocname| and saves a copy in |\childdocjob|:
%    \begin{macrocode}
\edef\childdocname{\scantokens\expandafter{\jobname\noexpand}}
\let\childdocjob\childdocname
%    \end{macrocode}

% \macro{\childdocdisable}
% The macro |\childdocdisable| prevents the main file
% from being processed more than once.
% At this stage, the main document command |\childdocmain|
% is assumed to be called once again where it should do nothing.
% Any subsequent call to it should prevent
% a secondary processing of the main document
% It overwrites the forwarding commands
% |\childdocof| and |\childdocforward|
% with empty macros to prevent further inclusions of the main document:
%    \begin{macrocode}
\newcommand{\childdocdisable}
{
  \renewcommand{\childdocmain}[1]{\renewcommand{\childdocmain}[1]{\endinput}}
  \renewcommand{\childdocof}[1]{}
  \renewcommand{\childdocby}[2][]{}
  \renewcommand{\childdocforward}[2][]{}
  \renewcommand{\childdocdisable}{}
}
%    \end{macrocode}

% \macro{\childdocmain}
% The macro |\childdocmain| is to be called at the top of the main file
% with nothing or the main filename (without extension) as argument.
% First, it breaks loops.
% If the argument is not empty and does not match |\childdocname|
% (which is set by the first inclusion of |childdoc.def|),
% |\ifchilddoc| is set to true, |\includeonly| is applied to the child file
% and |\jobname| is set to the main file
% (for proper handling of |.aux| files):
%    \begin{macrocode}
\newcommand{\childdocmain}[1]
{
  \childdocdisable\childdocmain{}
  \if?#1?\else
    \begingroup
      \def\childdoctmp{#1}
      \ifx\childdoctmp\childdocname
        \def\childdoctmp{}
      \else
        \def\childdoctmp
        {
          \childdoctrue
          \includeonly{\childdocname}
          \def\childdocjob{#1}
          \def\jobname{#1}
        }
      \fi
      \expandafter
    \endgroup
    \childdoctmp
  \fi
}
%    \end{macrocode}

% \macro{\childdocof}
% The command |\childdocof| redirects
% compilation to the main file |#1|.
%    \begin{macrocode}
\newcommand{\childdocof}[1]
{
  \childdocdisable
  \childdoctrue
  \includeonly{\childdocname}
  \def\jobname{#1}
  \def\childdocjob{#1}
  \input{#1}
}
%    \end{macrocode}

% \macro{\childdocby}
% The command |\childdocby| ....
%    \begin{macrocode}
\newcommand{\childdocby}[2][]
{
  \childdocdisable
  \childdoctrue
  \childdocmanualtrue
  \if?#1?\else
    \def\jobname{#2}
  \fi
  \def\childdocjob{#2}
  \input{#2}
  \endinput
}
%    \end{macrocode}

% \macro{\childdocforward}
% The command |\childdocforward| redirects
% compilation to the main file or
% (if the optional argument is given) a child file.
% Parameters are set as if the main file
% or a child file starting with |\childdocof| was compiled.
% Then compilation is handed over to the main file:
%    \begin{macrocode}
\newcommand{\childdocforward}[2][]
{
  \begingroup
    \if?#1?
      \def\childdoctmp
      {
        \def\childdocname{#2}
        \def\childdocjob{#2}
        \def\jobname{#2}
        \input{#2}
        \endinput
      }
    \else
      \def\childdoctmp
      {
        \childdocdisable
        \def\childdocname{#2}
        \childdoctrue
        \includeonly{#2}
        \def\childdocjob{#1}
        \def\jobname{#1}
        \input{#1}
        \endinput
      }
    \fi
    \expandafter
  \endgroup
  \childdoctmp
}
%    \end{macrocode}

% \macro{\childdocforwardprefix}
% The command |\childdocforwardprefix| redirects
% compilation to the main or a child file by means of a pattern.
% The prefix |#1| in the current filename is replaced by |#2|
% and the suffix of the current filename is kept
% (it is assumed that the filename does not contain the substring `|~~~|'
% which is used as a delimiter).
% Compilation is handed over to the new file by |\childdocforward|:
%    \begin{macrocode}
\newcommand{\childdocforwardprefix}[3][]
{
  \begingroup
    \def\childdocextract #2##1~~~{\def\childdoctmp{\childdocforward[#1]{#3##1}}}
    \expandafter\childdocextract\childdocname~~~
    \expandafter
  \endgroup
  \childdoctmp
}
%    \end{macrocode}

% \macro{\childdoc}
% The deprecated macro |\childdoc| is a legacy version of |\childdocmain|:
%    \begin{macrocode}
\newcommand{\childdoc}{\childdocmain}
%    \end{macrocode}

% \macro{\childdocredirect}
% The deprecated macro |\childdocredirect| is a legacy version
% of |\childdocforward| and |\childdocforwardprefix|:
%    \begin{macrocode}
\newcommand{\childdocredirect}[2][]
{
  \begingroup
    \if?#1?
      \def\childdoctmp{\childdocforward{#2}}
    \else
      \def\childdoctmp{\childdocforwardprefix{#1}{#2}}
    \fi
    \expandafter
  \endgroup
  \childdoctmp
}
%    \end{macrocode}

%\iffalse
%</package>
%\fi
%
\endinput
|\\
|\childdocforwardprefix[|\textit{main}|]{|\textit{prefix}|}{|\textit{dest}|}|
\end{tabular}
\end{center}
%
the destination file is determined by a pattern
depending on the current file:
To make this work, the current file must be called
`{\textit{prefix}\hspace{0.2em}\textit{suffix}}'
with \textit{prefix} matching precisely the argument.
Processing is then passed on to the file
`{\textit{dest}\hspace{0.2em}\textit{suffix}}'.
Surely, the same effect is achieved by
directly specifying the
argument `{\textit{dest}\hspace{0.2em}\textit{suffix}}'
in the first form.
However, that requires to set up a different file
for each child. With the alternative form of the command
all these files can have exactly the same content
which simplifies setting them up and maintaining them.

For example, the following file |draft.tex|
with a compilation flag |\version| as described in \secref{sec:flags}
compiles the main document as a draft:
%
\begin{center}
\begin{tabular}{l}
|\def\version{draft}|\\
|% \iffalse
%
% childdoc.dtx Copyright (C) 2017-2018 Niklas Beisert
%
% This work may be distributed and/or modified under the
% conditions of the LaTeX Project Public License, either version 1.3
% of this license or (at your option) any later version.
% The latest version of this license is in
%   http://www.latex-project.org/lppl.txt
% and version 1.3 or later is part of all distributions of LaTeX
% version 2005/12/01 or later.
%
% This work has the LPPL maintenance status `maintained'.
%
% The Current Maintainer of this work is Niklas Beisert.
%
% This work consists of the files childdoc.dtx and childdoc.ins
% and the derived files childdoc.def and cdocsamp.tex with
% cdocsch1.tex, cdocsch2.tex, cdocsdrf.tex, cdocsfn1.tex, cdocsfn2.tex.
%
%<package>\ifdefined\childdocmain\endinput\fi
%<package>\ProvidesFile{childdoc.def}[2018/12/30 v2.0 child document driver]
%<samplemain>\ProvidesFile{cdocsamp.tex}[2018/12/30 v2.0 sample for childdoc]
%<*driver>
%\ProvidesFile{childdoc.drv}[2018/12/30 v2.0 childdoc reference manual file]
\PassOptionsToClass{10pt,a4paper}{article}
\documentclass{ltxdoc}

\usepackage[margin=35mm]{geometry}
\usepackage{hyperref}
\usepackage{hyperxmp}
\usepackage[usenames]{color}

\hypersetup{colorlinks=true}
\hypersetup{pdfstartview=FitH}
\hypersetup{pdfpagemode=UseNone}
\hypersetup{pdfsource={}}
\hypersetup{pdflang={en-UK}}
\hypersetup{pdfcopyright={Copyright 2017-2018 Niklas Beisert.
  This work may be distributed and/or modified under the
  conditions of the LaTeX Project Public License, either version 1.3
  of this license or (at your option) any later version.}}
\hypersetup{pdflicenseurl={http://www.latex-project.org/lppl.txt}}
\hypersetup{pdfcontactaddress={ETH Zurich, ITP, HIT K,
  Wolfgang-Pauli-Strasse 27}}
\hypersetup{pdfcontactpostcode={8093}}
\hypersetup{pdfcontactcity={Zurich}}
\hypersetup{pdfcontactcountry={Switzerland}}
\hypersetup{pdfcontactemail={nbeisert@itp.phys.ethz.ch}}
\hypersetup{pdfcontacturl={http://people.phys.ethz.ch/\xmptilde nbeisert/}}

\newcommand{\secref}[1]{\hyperref[#1]{section \ref*{#1}}}

\parskip1ex
\parindent0pt
\let\olditemize\itemize
\def\itemize{\olditemize\parskip0pt}

\begin{document}

\title{The \textsf{childdoc} Package}
\hypersetup{pdftitle={The childdoc Package}}
\author{Niklas Beisert\\[2ex]
  Institut f\"ur Theoretische Physik\\
  Eidgen\"ossische Technische Hochschule Z\"urich\\
  Wolfgang-Pauli-Strasse 27, 8093 Z\"urich, Switzerland\\[1ex]
  \href{mailto:nbeisert@itp.phys.ethz.ch}
  {\texttt{nbeisert@itp.phys.ethz.ch}}}
\hypersetup{pdfauthor={Niklas Beisert}}
\hypersetup{pdfsubject={Manual for the LaTeX2e Package childdoc}}
\date{30 December 2018, \textsf{v2.0}}
\maketitle

\begin{abstract}\noindent
\textsf{childdoc} is a \LaTeXe{} package
that enables the direct compilation
of document sections included by |\include|
to individual files.
\end{abstract}

\begingroup
\parskip0ex
\tableofcontents
\endgroup

%%%%%%%%%%%%%%%%%%%%%%%%%%%%%%%%%%%%%%%%%%%%%%%%%%%%%%%%%%%%%%%%%%%%%%%%%%%%%%%%
%%%%%%%%%%%%%%%%%%%%%%%%%%%%%%%%%%%%%%%%%%%%%%%%%%%%%%%%%%%%%%%%%%%%%%%%%%%%%%%%
\section{Introduction}

\LaTeX{} provides a mechanism to structure a large document (such as a book)
into a main file and several child files (containing the chapters)
using the |\include| command.
This mechanism is beneficial for documents
which span hundreds of pages in order to
make the source file(s) more manageable.
Moreover, compilation can be restricted to
selected child files by means of the |\includeonly| command.
The latter feature can be used to reduce the compilation time while editing
(this was significantly more useful in the earlier days of \LaTeX{})
or to generate a smaller document which is easier to navigate.
Another application of |\includeonly| is to generate
documents consisting of selected parts of the complete document.

However, there are a few drawbacks of the plain |\include| mechanism:
\begin{itemize}
\item
The child files cannot be compiled on their own,
they can only be compiled via the main file.
A naive editing environment
(such as a text editor with an option
to have the current file processed by \LaTeX)
may require one to switch to the main file before compiling;
attempting to compile the child file produces errors.
\item
The main file must be modified (each time)
to adjust the |\includeonly| command
to the present needs. This easily leaves the main file in a messy state.
\item
The generated document will always carry the filename
of the main document. This is inconvenient if
several child files are to be compiled and
to be kept for distribution.
\end{itemize}

The present package provides a simple interface
to make child files individually compilable by \LaTeX{}.
Compiling a child file then has the same effect as compiling
the main file with an |\includeonly| command
to select the appropriate child.
Moreover the generated document will carry the name of the child
rather than the main file.
This resolves all three above issues.

This feature is meant to make the editing of books,
thesis documents and lecture notes somewhat more convenient.
However, the package can also be used efficiently for
composing a series of documents (such as exercise sheets)
which are typically distributed individually.
It then assists the author in generating the individual documents
(potentially in different versions)
as well as a document containing the collected series.
Another application is in developing style files
or other kinds of included material
where compilation of the style file could redirect
to a sample or test file.

%%%%%%%%%%%%%%%%%%%%%%%%%%%%%%%%%%%%%%%%%%%%%%%%%%%%%%%%%%%%%%%%%%%%%%%%%%%%%%%%
%%%%%%%%%%%%%%%%%%%%%%%%%%%%%%%%%%%%%%%%%%%%%%%%%%%%%%%%%%%%%%%%%%%%%%%%%%%%%%%%
\section{Usage}

First of all, the package \textsf{childdoc} is \emph{not} a standard
\LaTeXe{} |.sty| style file! Therefore it needs to be invoked in
a non-standard way.

%%%%%%%%%%%%%%%%%%%%%%%%%%%%%%%%%%%%%%%%%%%%%%%%%%%%%%%%%%%%%%%%%%%%%%%%%%%%%%%%
\subsection{Included Files}
\label{sec:include}

%%%%%%%%%%%%%%%%%%%%%%%%%%%%%%%%%%%%%%%%
\DescribeMacro{\childdocmain}
To use the package, add the commands
\begin{center}
\begin{tabular}{l}
|\input{childdoc.def}|\\
|\childdocmain{}|\\
\end{tabular}
\end{center}
at the very top of the main \LaTeX{} file,
in particular \emph{before} the |\documentclass| statement!
The argument of |\childdocmain| should be left empty
(but it must be present).

%%%%%%%%%%%%%%%%%%%%%%%%%%%%%%%%%%%%%%%%
\DescribeMacro{\childdocof}
Furthermore, add the commands
\begin{center}
\begin{tabular}{l}
|\input{childdoc.def}|\\
|\childdocof{|\textit{main}|}|\\
\end{tabular}
\end{center}
at the top of every child file \textit{child}
which is included by |\include{|\textit{child}|}|
from within the main file
(or at least for those files to be compiled individually).
The argument \textit{main} must be the filename of the main file.

There are a couple of
considerations in setting up the main and child documents:

%%%%%%%%%%%%%%%%%%%%%%%%%%%%%%%%%%%%%%%%
\paragraph{Restrictions.}

Please note the following restrictions:
\begin{itemize}
\item
|\childdocmain| must be called with one argument \textit{main}
to ensure compatibility with earlier version of the package.
It must either be empty (|\childdocmain{}|)
or precisely match the filename of the main file in which it is specified.
See \secref{sec:detection} for further information.
\item
The filename \textit{main} must be specified without the |.tex| extension.
\item
The filename \textit{main} is case sensitive
(even in case-insensitive file systems)
due to internal string comparison.
\item
The argument \textit{main} should be fully expanded, it cannot be a macro.
\item
Subdirectories and special characters should be avoided in filenames.
\item
The command |\childdocmain{|\textit{main}|}| must be followed by a whitespace.
It should not be followed immediately by another command
or by a comment mark `|%|'.
This is because the \TeX{} parser reads the token immediately following
the argument of |\childdocmain| and puts it
at the beginning of every child section;
however, a white\-space is ignored.
\end{itemize}

%%%%%%%%%%%%%%%%%%%%%%%%%%%%%%%%%%%%%%%%
\paragraph{Content of Main File.}

It is advisable to place all content in the child files included by |\include|.
Any output contained in the main file will appear in all child documents
unless suppressed manually;
it cannot be suppressed automatically by the |\includeonly| directive
and thus should normally be avoided.
A method to include some content in the main file
by means of conditional processing is described in \secref{sec:conditional}.

%%%%%%%%%%%%%%%%%%%%%%%%%%%%%%%%%%%%%%%%
\paragraph{Page Numbering.}

When only a part of the document is compiled,
the appropriate numbering of pages
(as well as other status parameters)
is determined from the |.aux| files.
The latter contain information from previous passes.
However this information needs to propagate through
all intermediate child documents.
Therefore the page numbering in child documents may well
be inconsistent until the complete document is compiled at least once.

A useful (if unconventional) way to always ensure a consistent
page numbering is to restart the numbering in each child document
and denote the pages by `\textit{child}|.|\textit{page}'
where \textit{child} represents the chapter/section number of the child file.
This can be achieved by the command
|\numberwithin{page}{|\textit{child}|}|
of the \textsf{amsmath} package
where \textit{child} can be |chapter| or |section|
depending on the chosen structuring.
Alternatively, one can modify the macro |\thepage| appropriately
and reset the counter |page| at the start of each child file.

%%%%%%%%%%%%%%%%%%%%%%%%%%%%%%%%%%%%%%%%%%%%%%%%%%%%%%%%%%%%%%%%%%%%%%%%%%%%%%%%
\subsection{Conditional Processing}
\label{sec:conditional}

The package provides a mechanism to compile different versions
of a document. To customise the versions further some conditional processing
can come in handy to distinguish which version is being compiled.
The package provides two macros to describe the compilation context:

%%%%%%%%%%%%%%%%%%%%%%%%%%%%%%%%%%%%%%%%
\DescribeMacro{\ifchilddoc}
The conditional |\ifchilddoc| distinguishes between the compilation of
child documents and the main document:
%
\begin{center}
|\ifchilddoc |\textit{child-code}| |[|\||else |\textit{main-code}]| \||fi|
\end{center}

%%%%%%%%%%%%%%%%%%%%%%%%%%%%%%%%%%%%%%%%
\DescribeMacro{\childdocname}
\DescribeMacro{\childdocjob}
The macro |\childdocname| contains the filename (without extension)
of the main or child file being processed.
Note that |\childdocjob| will always contain the name of the main file.

%%%%%%%%%%%%%%%%%%%%%%%%%%%%%%%%%%%%%%%%
\paragraph{Title Page.}

Conditional processing can be used to include a title or banner page
in the main document when proper precautions are taken.
Importantly, the code in the main file should ensure that the page counter
(as well as other status parameters which are stored in the |.aux| files)
takes the same value after the conditional processing.
Otherwise the page numbers may take divergent values
depending on which part is compiled.

For example, a title page could be declared by:
%
\begin{center}
\begin{tabular}{l}
|\ifchilddoc\||else|\\
|\addtocounter{page}{-1}|\\
\textit{code for title page}\\
|\newpage|\\
|\||fi|
\end{tabular}
\end{center}
%
A banner page for the child documents can be generated by:
%
\begin{center}
\begin{tabular}{l}
|\ifchilddoc|\\
|\addtocounter{page}{-1}|\\
\textit{code for banner page}\\
|\newpage|\\
|\||fi|
\end{tabular}
\end{center}
%
Here one could write a message such as:
\begin{center}
|This is the part \childdocname{} of \childdocjob{}.|
\end{center}

%%%%%%%%%%%%%%%%%%%%%%%%%%%%%%%%%%%%%%%%%%%%%%%%%%%%%%%%%%%%%%%%%%%%%%%%%%%%%%%%
\subsection{Flags}
\label{sec:flags}

The package makes it easy to generate different versions
of the main or child documents.
To this end compilation flags can be defined
and assigned different default values.
They will be particularly useful in conjunction
with the forwarding mechanism described in \secref{sec:forward}.

For example, it may be useful to have a flag |\version|
which can be set to |draft| or |final|.
The document source will contain some conditional code
depending on the value of |\version|.
Suppose further, the flag should default to |final| for the main file
and to |draft| for child files
which is a natural assignment for editing the document.
This is achieved by placing the following code
in the preamble of the main document
(below the |\childdocmain| directive):
%
\begin{center}
\begin{tabular}{l}
|\ifchilddoc|\\
|\providecommand{\version}{draft}|\\
|\||else|\\
|\providecommand{\version}{final}|\\
|\||fi|
\end{tabular}
\end{center}
%
The definition by |\providecommand| makes sure
that previous definitions are not overwritten.
Further statements |\providecommand{\version}{...}|
can thus be added before the above code to override it.

For the main file, one might add a line
(between |\childdocmain| and the above block)
%
\begin{center}
|%\ifchilddoc\||else\providecommand{\version}{draft}\||fi|
\end{center}
%
which can be uncommented to produce a draft version.
Likewise one can add a line to the very top of a child file
(above the |\childdocof{|\textit{main}|}| directive)
%
\begin{center}
|%\providecommand{\version}{final}|
\end{center}
%
which can be uncommented to produce the final version of this child document.

%%%%%%%%%%%%%%%%%%%%%%%%%%%%%%%%%%%%%%%%%%%%%%%%%%%%%%%%%%%%%%%%%%%%%%%%%%%%%%%%
\subsection{Forwarding}
\label{sec:forward}

Different versions of the main or child documents
using compilation flags as described in \secref{sec:flags}
can be (permanently) stored in different files
for convenient compilation, viewing and distribution.
To this end, the package defines a command
to pass on compilation to a different file:

%%%%%%%%%%%%%%%%%%%%%%%%%%%%%%%%%%%%%%%%
\DescribeMacro{\childdocforward}
The command |\childdocforward| redirects processing to
another source file:
%
\begin{center}
\begin{tabular}{l}
|\input{childdoc.def}|\\
|\childdocforward[|\textit{main}|]{|\textit{dest}|}|\\
\end{tabular}
\end{center}
%
The argument \textit{dest} is the destination file
(without extension).
It should be the main file or one of the child files.
Note that further \textsf{childdoc} directives
such as |\childdocof| and |\childdocforward|
in the indicated file will be processed in this form.
The optional argument \textit{main}
passes on directly to the main file \textit{main}
while pretending to compile the child \textit{dest}.
This form behaves as if \textit{dest}
issues |\childdocof{|\textit{main}|}| right away,
and no further \textsf{childdoc} directives will be processed.

%%%%%%%%%%%%%%%%%%%%%%%%%%%%%%%%%%%%%%%%
\DescribeMacro{\...prefix}
In the alternative form |\childdocforwardprefix|,
%
\begin{center}
\begin{tabular}{l}
|\input{childdoc.def}|\\
|\childdocforwardprefix[|\textit{main}|]{|\textit{prefix}|}{|\textit{dest}|}|
\end{tabular}
\end{center}
%
the destination file is determined by a pattern
depending on the current file:
To make this work, the current file must be called
`{\textit{prefix}\hspace{0.2em}\textit{suffix}}'
with \textit{prefix} matching precisely the argument.
Processing is then passed on to the file
`{\textit{dest}\hspace{0.2em}\textit{suffix}}'.
Surely, the same effect is achieved by
directly specifying the
argument `{\textit{dest}\hspace{0.2em}\textit{suffix}}'
in the first form.
However, that requires to set up a different file
for each child. With the alternative form of the command
all these files can have exactly the same content
which simplifies setting them up and maintaining them.

For example, the following file |draft.tex|
with a compilation flag |\version| as described in \secref{sec:flags}
compiles the main document as a draft:
%
\begin{center}
\begin{tabular}{l}
|\def\version{draft}|\\
|\input{childdoc.def}|\\
|\childdocforward{|\textit{main}|}|
\end{tabular}
\end{center}
%
Likewise, the following files |final|\textit{nn}|.tex|
compile the final version of the child document
|child|\textit{nn}|.tex|:
%
\begin{center}
\begin{tabular}{l}
|\def\version{final}|\\
|\input{childdoc.def}|\\
|\childdocforwardprefix{final}{child}|
\end{tabular}
\end{center}
%

Note that when several versions of a main file and/or of each child file
are to be generated, it may be convenient to set up a |Makefile| or
shell script to automatise the process.

%%%%%%%%%%%%%%%%%%%%%%%%%%%%%%%%%%%%%%%%%%%%%%%%%%%%%%%%%%%%%%%%%%%%%%%%%%%%%%%%
\subsection{Command Line Processing}
\label{sec:commandline}

The effect of redirection files can also be achieved by invoking
the \LaTeX{} compiler with a more elaborate command line.
Most conveniently this should be done as part
of a shell script or a |Makefile|.

When using \textsf{childdoc} in the main file, the following
command lines effectively perform a redirection
(note that depending on the shell being used,
backslashes may have to be doubled: `|\|' $\to$ `|\\|'):
%
\begin{center}
|... -jobname "|\textit{target}|" |\\|"|[\textit{flags}]%
|\input{childdoc.def}\childdocforward[|\textit{main}|]{|\textit{dest}|}"|
\end{center}
%
Here \textit{target} is the name of the output file,
\textit{main} is the name of the main file
and \textit{dest} is the name of the main or child file to be processed
(all filenames without extensions).
The optional argument \textit{main} can be omitted
if \textit{main} matches \textit{dest}.
Optionally, compilation \textit{flags} can be defined via |\def| commands.
This command line makes the \TeX{} engine believe
it is compiling the file \textit{target}
whose content is specified as the latter parameter.
The provided code then forwards the processing to
\textit{main} or \textit{dest} as described in \secref{sec:forward}.

%%%%%%%%%%%%%%%%%%%%%%%%%%%%%%%%%%%%%%%%%%%%%%%%%%%%%%%%%%%%%%%%%%%%%%%%%%%%%%%%
\subsection{Include by Input}
\label{sec:input}

Including child documents by |\include| has some restrictions by design.
Most notably, the content of a child document always occupies
its own set of pages; pages cannot be shared between child documents.
Usually, this behaviour makes perfect sense
because each child document contain an essential part of the document.
However, in some situations it may be desirable to compose
a document from a collection of parts
without having mandatory page breaks between then.
For this case, the package
provides a mechanism to include parts
by |\input| which can also be processed individually.
However, by construction this mechanism
requires manual handling of the content to be output.

%%%%%%%%%%%%%%%%%%%%%%%%%%%%%%%%%%%%%%%%
\DescribeMacro{\ifchilddocmanual}
The main file should be prepared as usual, see \secref{sec:include}.
However, the document body must make a distinction
between processing of an individual part and of the main document, e.g.:
%
\begin{center}
\begin{tabular}{l}
|\ifchilddocmanual|\\
|\input{\childdocname}|\\
|\||else|\\
\textit{document body with }|\input{|\textit{part}|}|\\
|\||fi|
\end{tabular}
\end{center}
%
The conditional |\ifchilddocmanual| is true whenever
a part to be included by |\input| is being compiled,
and the name of the part is stored in |\childdocname|.

%%%%%%%%%%%%%%%%%%%%%%%%%%%%%%%%%%%%%%%%
\DescribeMacro{\childdocby}
Each part to be included by |\input| should start with:
%
\begin{center}
\begin{tabular}{l}
|\input{childdoc.def}|\\
|\childdocby{|\textit{main}|}|\\
\end{tabular}
\end{center}
%
The directive |\childdocby| is similar to |\childdocof|
described in \secref{sec:include},
but the subsequent selection of content must be done manually.
To that end, both |\ifchilddoc| and |\ifchilddocmanual|
will be true upon processing of a part,
and the name of the part is stored in |\childdocname|.
Note that |\jobname| will be set to the filename of the current part
so that each part receives an individual |.aux| file
that does not interfere with the |.aux| file(s) of the main document.
This behaviour can be altered by the alternative form
|\childdocby[*]{|\textit{main}|}| (with a non-empty optional argument)
which uses the |.aux| file of the main document
by setting |\jobname| to \textit{main}.

%%%%%%%%%%%%%%%%%%%%%%%%%%%%%%%%%%%%%%%%%%%%%%%%%%%%%%%%%%%%%%%%%%%%%%%%%%%%%%%%
\subsection{Driver Development}
\label{sec:driver}

The \textsf{childdoc} mechanism can also be use for the development
of definition files such as \LaTeX{} styles or classes.
This case differs from the above setup with multiple parts
included by |\include| in that no |\includeonly| should be invoked.
This can be achieved by starting the include file
(before |\ProvidesPackage|) with:
%
\begin{center}
\begin{tabular}{l}
|\input{childdoc.def}|\\
|\childdocforward{|\textit{main}|}|\\
\end{tabular}
\end{center}
%
or alternatively with:
%
\begin{center}
\begin{tabular}{l}
|\input{childdoc.def}|\\
|\childdocby{|\textit{main}|}|\\
\end{tabular}
\end{center}
%
Both forms have slightly different effects as described above.
The main file is prepared as usual, see \secref{sec:include}.

%%%%%%%%%%%%%%%%%%%%%%%%%%%%%%%%%%%%%%%%%%%%%%%%%%%%%%%%%%%%%%%%%%%%%%%%%%%%%%%%
\subsection{Legacy Detection}
\label{sec:detection}

The directive |\childdocmain| in the main file can detect
whether the complete document or merely a child is to be compiled
even without using the directive |\childdocof|.
This method is deprecated because it is less robust
and there is no compelling reason to use it;
it is merely provided for backward compatibility
and it may be removed in future versions.

If the detection mechanism is to be used,
it is mandatory to correctly specify
the filename of the main file as the argument of |\childdocmain|:
%
\begin{center}
\begin{tabular}{l}
|\input{childdoc.def}|\\
|\childdocmain{|\textit{main}|}|\\
\end{tabular}
\end{center}
%
If |\jobname| does not match the argument \textit{main} of |\childdocmain|,
it is assumed that |\jobname| points to the child file to be compiled.
When using |\childdocmain| with the main file specified as argument,
it suffices to start a child file
with just |\input{|\textit{main}|}|
without loading of the package and using |\childdocof|.
If instead all processing is done
with the appropriate \textsf{childdoc} directives,
the argument of \textit{main} of |\childdocmain| can be empty.

An alternative version of the command line processing described
in \secref{sec:commandline} using the detection mechanism reads:
%
\begin{center}
|... -jobname "|\textit{target}|" "|[\textit{flags}]%
[|\def\jobname{|\textit{dest}|}|]|\input{|\textit{main}|}"|
\end{center}

%%%%%%%%%%%%%%%%%%%%%%%%%%%%%%%%%%%%%%%%%%%%%%%%%%%%%%%%%%%%%%%%%%%%%%%%%%%%%%%%
\subsection{Manual Code}
\label{sec:manual}

In case one cannot be certain whether the definitions file |childdoc.def|
is installed on the target \TeX{} distribution
and one prefers not to ship it,
it is conceivable to paste a few relevant commands into the sources.

To that end, drop all statements |\input{childdoc.def}|
and perform the replacements as outlined below.
Instead of |\childdocmain{|\textit{main}|}| add the following code
to the top of the main file:
%
\begin{center}
\begin{tabular}{l}
|\||ifdefined\childdocname\endinput\||fi\newif\ifchilddoc|\\
|\edef\childdocname{\scantokens\expandafter{\jobname\noexpand}}|\\
|\def\childdocmain{|\textit{main}|}\||ifx\childdocmain\childdocname\||else|\\
|\childdoctrue\includeonly{\childdocname}\let\jobname\childdocmain\||fi|\\
\end{tabular}
\end{center}
%
Instead of |\childdocof{|\textit{main}|}| just include the main file
at the top of each child file:
%
\begin{center}
|\input{|\textit{main}|}|
\end{center}
%
A simple redirection |\childdocforward{|\textit{dest}|}| is achieved by:
%
\begin{center}
|\def\jobname{|\textit{dest}|}\input{\jobname}|
\end{center}
%
The redirection with prefix
|\childdocforwardprefix[|\textit{prefix}|]{|\textit{dest}|}|
is accomplished by:
%
\begin{center}
\begin{tabular}{l}
|{\edef\jobname{\scantokens\expandafter{\jobname\noexpand}}|\\
|\def\redirectjob |\textit{prefix}|#1~~~{\gdef\jobname{|\textit{dest}|#1}}|\\
|\expandafter\redirectjob\jobname~~~}\input{\jobname}|
\end{tabular}
\end{center}

In an alternative approach,
child documents can be compiled by a specific command line
without additional code or specific definitions:
%
\begin{center}
|... -jobname "|\textit{target}|" "|[\textit{flags}]%
|\includeonly{|\textit{dest}|}\input{|\textit{main}|}"|
\end{center}
%

%%%%%%%%%%%%%%%%%%%%%%%%%%%%%%%%%%%%%%%%%%%%%%%%%%%%%%%%%%%%%%%%%%%%%%%%%%%%%%%%
%%%%%%%%%%%%%%%%%%%%%%%%%%%%%%%%%%%%%%%%%%%%%%%%%%%%%%%%%%%%%%%%%%%%%%%%%%%%%%%%
\section{Information}

%%%%%%%%%%%%%%%%%%%%%%%%%%%%%%%%%%%%%%%%%%%%%%%%%%%%%%%%%%%%%%%%%%%%%%%%%%%%%%%%
\subsection{Copyright}

Copyright \copyright{} 2017--2018 Niklas Beisert

This work may be distributed and/or modified under the
conditions of the \LaTeX{} Project Public License, either version 1.3
of this license or (at your option) any later version.
The latest version of this license is in
  \url{http://www.latex-project.org/lppl.txt}
and version 1.3 or later is part of all distributions of \LaTeX{}
version 2005/12/01 or later.

This work has the LPPL maintenance status `maintained'.

The Current Maintainer of this work is Niklas Beisert.

This work consists of the files |README.txt|, |childdoc.ins| and |childdoc.dtx|
as well as the derived files |childdoc.def|, |cdocsamp.tex|
with |cdocsch1.tex|, |cdocsch2.tex|, |cdocspt3.tex|, |cdocspt4.tex|,
|cdocsdrf.tex|, |cdocsfn1.tex|, |cdocsfn2.tex|
as well as |childdoc.pdf|.

%%%%%%%%%%%%%%%%%%%%%%%%%%%%%%%%%%%%%%%%%%%%%%%%%%%%%%%%%%%%%%%%%%%%%%%%%%%%%%%%
\subsection{Files and Installation}

The package consists of the files:
%
\begin{center}
\begin{tabular}{ll}
    |README.txt|   & readme file \\
    |childdoc.ins| & installation file \\
    |childdoc.dtx| & source file \\
    |childdoc.def| & definition file \\
    |cdocsamp.tex| & sample main file \\
    |cdocsch1.tex| & sample include file \\
    |cdocsch2.tex| & sample include file \\
    |cdocspt3.tex| & sample part file \\
    |cdocspt4.tex| & sample part file \\
    |cdocsdrf.tex| & sample redirection file \\
    |cdocsfn1.tex| & sample redirection file \\
    |cdocsfn2.tex| & sample redirection file \\
    |childdoc.pdf| & manual
\end{tabular}
\end{center}
%
The distribution consists of the files
|README.txt|, |childdoc.ins| and |childdoc.dtx|.
%
\begin{itemize}
\item
Run (pdf)\LaTeX{} on |childdoc.dtx|
to compile the manual |childdoc.pdf| (this file).
\item
Run \LaTeX{} on |childdoc.ins| to create the definitions file |childdoc.def|
and the sample |cdocsamp.tex| with include files
|cdocsch1.tex|, |cdocsch2.tex|, |cdocspt3.tex|, |cdocspt4.tex|,
|cdocsdrf.tex|, |cdocsfn1.tex|, |cdocsfn2.tex|.
Then copy the file |childdoc.def| to an appropriate directory of your \LaTeX{}
distribution, e.g.\ \textit{texmf-root}|/tex/latex/childdoc|.
\end{itemize}

%%%%%%%%%%%%%%%%%%%%%%%%%%%%%%%%%%%%%%%%%%%%%%%%%%%%%%%%%%%%%%%%%%%%%%%%%%%%%%%%
\subsection{Related CTAN Packages}

There are several other packages which offer a similar functionality:
%
\begin{itemize}
\item
The packages
\href{http://ctan.org/pkg/docmute}{\textsf{docmute}},
\href{http://ctan.org/pkg/includex}{\textsf{includex}} and
\href{http://ctan.org/pkg/standalone}{\textsf{standalone}}
provide commands to include only the document body of
a child file thus allowing both files to be compiled individually.
\item
The packages \href{http://ctan.org/pkg/subdocs}{\textsf{subdocs}}
and \href{http://ctan.org/pkg/subfiles}{\textsf{subfiles}}
provide structures in which the main and child documents can be
encapsulated and allowing them to be compiled individually.
The inclusion mechanism is different from the conventional |\include|.
\item
The package \href{http://ctan.org/pkg/combine}{\textsf{combine}}
is an elaborate solution to combine several documents into one.
\end{itemize}
%
See also the CTAN topic \href{http://ctan.org/topic/subdocs}{\textsf{subdocs}}
for further related packages.
The present package differs from the above solutions in that
a document structure constructed with the conventional |\include| mechanism
just needs two extra commands at the top of every file
such that all constituent files can be compiled individually.

%%%%%%%%%%%%%%%%%%%%%%%%%%%%%%%%%%%%%%%%%%%%%%%%%%%%%%%%%%%%%%%%%%%%%%%%%%%%%%%%
%\subsection{Feature Suggestions}
%
%The following is a list of features which may be useful for future
%versions of this package:
%%
%\begin{itemize}
%\item
%\ldots
%\end{itemize}

%%%%%%%%%%%%%%%%%%%%%%%%%%%%%%%%%%%%%%%%%%%%%%%%%%%%%%%%%%%%%%%%%%%%%%%%%%%%%%%%
\subsection{Revision History}

%%%%%%%%%%%%%%%%%%%%%%%%%%%%%%%%%%%%%%%%
\paragraph{v2.0:} 2018/12/30

\begin{itemize}
\item
immediate forward processing
\item
added |\childdocby| mechanism
\item
manual restructured
\end{itemize}

%%%%%%%%%%%%%%%%%%%%%%%%%%%%%%%%%%%%%%%%
\paragraph{v1.6:} 2018/01/17

\begin{itemize}
\item
application for development of include files
\item
corrections to manual
\end{itemize}

%%%%%%%%%%%%%%%%%%%%%%%%%%%%%%%%%%%%%%%%
\paragraph{v1.5:} 2017/05/21

\begin{itemize}
\item
more complete structuring introduced
\item
|\childdocof| introduced
\item
|\childdoc| renamed to |\childdocmain|
\item
|\childredirect| renamed to |\childdocforward| and |\childdocforwardprefix|
and functionality expanded
\end{itemize}

%%%%%%%%%%%%%%%%%%%%%%%%%%%%%%%%%%%%%%%%
\paragraph{v1.0:} 2017/04/27

\begin{itemize}
\item
manual and install package
\item
first version published on CTAN
\end{itemize}

%%%%%%%%%%%%%%%%%%%%%%%%%%%%%%%%%%%%%%%%
\paragraph{v0.6:} 2017/04/26

\begin{itemize}
\item
redirection mechanism added
\end{itemize}

%%%%%%%%%%%%%%%%%%%%%%%%%%%%%%%%%%%%%%%%
\paragraph{v0.5:} 2017/04/26

\begin{itemize}
\item
functionality in definition file
\end{itemize}


%%%%%%%%%%%%%%%%%%%%%%%%%%%%%%%%%%%%%%%%%%%%%%%%%%%%%%%%%%%%%%%%%%%%%%%%%%%%%%%%
%%%%%%%%%%%%%%%%%%%%%%%%%%%%%%%%%%%%%%%%%%%%%%%%%%%%%%%%%%%%%%%%%%%%%%%%%%%%%%%%
%%%%%%%%%%%%%%%%%%%%%%%%%%%%%%%%%%%%%%%%%%%%%%%%%%%%%%%%%%%%%%%%%%%%%%%%%%%%%%%%
\appendix

\settowidth\MacroIndent{\rmfamily\scriptsize 000\ }

 \DocInput{childdoc.dtx}

\end{document}
%</driver>
% \fi
%
% %%%%%%%%%%%%%%%%%%%%%%%%%%%%%%%%%%%%%%%%%%%%%%%%%%%%%%%%%%%%%%%%%%%%%%%%%%%%%%
% %%%%%%%%%%%%%%%%%%%%%%%%%%%%%%%%%%%%%%%%%%%%%%%%%%%%%%%%%%%%%%%%%%%%%%%%%%%%%%
% \section{Sample}
%\iffalse
%<*samplemain>
%\fi
%
% The following presents a sample document
% with two chapters, two parts, a title page,
% a compile flag as well as three forwarding files to set the flag.
% It consists of eight |.tex| files:
% \begin{center}
% \begin{tabular}{ll}
% |cdocsamp.tex|&main file\\
% |cdocsch1.tex|&include file for chapter 1\\
% |cdocsch2.tex|&include file for chapter 2\\
% |cdocspt3.tex|&include file for part 3\\
% |cdocspt4.tex|&include file for part 4\\
% |cdocsdrf.tex|&forwarding file for main file in draft mode\\
% |cdocsfi1.tex|&forwarding file for final version of chapter 1\\
% |cdocsfi2.tex|&forwarding file for final version of chapter 2\\
% \end{tabular}
% \end{center}
% Each of the eight files can be compiled directly by the \LaTeX{} compiler.
%
% %%%%%%%%%%%%%%%%%%%%%%%%%%%%%%%%%%%%%%
% \paragraph{Main File.}
%
% The main file is called |cdocsamp.tex|.
%
% Load the \textsf{childdoc} definitions and
% declare the filename for the main document:
%    \begin{macrocode}
\input{childdoc.def}
\childdocmain{}
%    \end{macrocode}

% Optional override for |\version| flag:
%    \begin{macrocode}
%%\ifchilddoc\else\providecommand{\version}{draft}\fi
%    \end{macrocode}

% Define the default values for the |\version| flag
% (|final| for the main file and |draft| for childs):
%    \begin{macrocode}
\ifchilddoc
\providecommand{\version}{draft}
\else
\providecommand{\version}{final}
\fi
%    \end{macrocode}

% Load the standard document class:
%    \begin{macrocode}
\documentclass[12pt]{article}
%    \end{macrocode}

% Start the document body:
%    \begin{macrocode}
\begin{document}
%    \end{macrocode}

% Declare a title page.
% Print title, part of document being processed and version flag:
%    \begin{macrocode}
\addtocounter{page}{-1}
\begin{center}
{\LARGE\bfseries{}childdoc example\par}
\vspace{1cm}
\ifchilddoc
\ifchilddocmanual part\else chapter\fi:
`\childdocname' of `\childdocjob'\par
\else
main document: `\childdocjob'\par
\fi
version: \version\par
\end{center}
\newpage
%    \end{macrocode}

% Manually include selected file,
% otherwise process as usual:
%    \begin{macrocode}
\ifchilddocmanual
\section*{part `\childdocname'}
\input{\childdocname}
\else
%    \end{macrocode}

% Include the two chapters:
%    \begin{macrocode}
\include{cdocsch1}
\include{cdocsch2}
%    \end{macrocode}

% Include the two parts unless only chapters should be displayed:
%    \begin{macrocode}
\ifchilddoc\else
\section{part three}
\input{cdocspt3}
\section{part four}
\input{cdocspt4}
\fi
%    \end{macrocode}

% Process as usual until here:
%    \begin{macrocode}
\fi
%    \end{macrocode}

% End of document body:
%    \begin{macrocode}
\end{document}
%    \end{macrocode}
%\iffalse
%</samplemain>
%\fi
%
% %%%%%%%%%%%%%%%%%%%%%%%%%%%%%%%%%%%%%%
% \paragraph{Chapter Include Files.}
%
% The include files are called |cdocsch1.tex| and |cdocsch2.tex|.
%
%\iffalse
%<*samplechap1|samplechap2>
%\fi

% Optional override for |\version| flag:
%    \begin{macrocode}
%%\providecommand{\version}{final}
%    \end{macrocode}

% Include the main document:
%    \begin{macrocode}
\input{childdoc.def}
\childdocof{cdocsamp}
%    \end{macrocode}

%\iffalse
%</samplechap1|samplechap2>
%\fi
%
%\iffalse
%<*samplechap1>
%\fi
% Some text for chapter 1:
%    \begin{macrocode}
\section{one}
some text in chapter one
%    \end{macrocode}

%\iffalse
%</samplechap1>
%\fi
% Some text for chapter 2:
%\iffalse
%<*samplechap2>
%\fi
%    \begin{macrocode}
\section{two}
more text in chapter two
%    \end{macrocode}

%\iffalse
%</samplechap2>
%\fi
%
% %%%%%%%%%%%%%%%%%%%%%%%%%%%%%%%%%%%%%%
% \paragraph{Part Include Files.}
%
% The include files are called |cdocspt3.tex| and |cdocspt4.tex|.
%
%\iffalse
%<*samplepart3|samplepart4>
%\fi

% Optional override for |\version| flag:
%    \begin{macrocode}
%%\providecommand{\version}{final}
%    \end{macrocode}

% Include the main document:
%    \begin{macrocode}
\input{childdoc.def}
\childdocby{cdocsamp}
%    \end{macrocode}

%\iffalse
%</samplepart3|samplepart4>
%\fi
%
%\iffalse
%<*samplepart3>
%\fi
% Some text for part 3:
%    \begin{macrocode}
some text in part three
%    \end{macrocode}

%\iffalse
%</samplepart3>
%\fi
% Some text for part 4:
%\iffalse
%<*samplepart4>
%\fi
%    \begin{macrocode}
more text in part four
%    \end{macrocode}

%\iffalse
%</samplepart4>
%\fi
%
% %%%%%%%%%%%%%%%%%%%%%%%%%%%%%%%%%%%%%%
% \paragraph{Forwarding for a Complete Draft.}
%
% The following forwarding file |cdocsdrf.tex|
% compiles the main document in draft mode:
%\iffalse
%<*sampledraft>
%\fi
%    \begin{macrocode}
\def\version{draft}
\input{childdoc.def}
\childdocforward{cdocsamp}
%    \end{macrocode}

%\iffalse
%</sampledraft>
%\fi
%
% %%%%%%%%%%%%%%%%%%%%%%%%%%%%%%%%%%%%%%
% \paragraph{Forwarding for Final Version of the Chapters.}
%
% The following forwarding files |cdocsfn1.tex| and |cdocsfn2.tex|
% (with identical content)
% compile the final versions of the child documents
% |cdocsch1.tex| and |cdocsch2.tex|, respectively:
%\iffalse
%<*samplefinal>
%\fi
%    \begin{macrocode}
\def\version{final}
\input{childdoc.def}
\childdocforwardprefix[cdocsamp]{cdocsfn}{cdocsch}
%    \end{macrocode}

%\iffalse
%</samplefinal>
%\fi
%
% %%%%%%%%%%%%%%%%%%%%%%%%%%%%%%%%%%%%%%
% \paragraph{Command Line Processing.}
%
% The following three command lines generate the output files
% |cdocscld|, |cdocscl1| and |cdocscl2|
% which should be identical to
% |cdocsdrf|, |cdocsch1| and |cdocsfn2|, respectively:
% \begin{center}
% \begin{tabular}{l}
% |latex -jobname cdocscld \|\\
% |  "\def\version{draft}\input{childdoc.def}\childdocforward{cdocsamp}"|\\
% |latex -jobname cdocscl1 \|\\
% |  "\input{childdoc.def}\childdocforward[cdocsamp]{cdocsch1}"|\\
% |latex -jobname cdocscl2 \|\\
% |  "\def\version{final}\input{childdoc.def}\childdocforward{cdocsch2}"|
% \end{tabular}
% \end{center}
% Note that the trailing backslash on each first line
% merely continues the input to the second line
% (for convenient cut ant paste).
% Furthermore, the command |latex| can be replaced by any
% of its alternative versions such as |pdflatex|.
%
% %%%%%%%%%%%%%%%%%%%%%%%%%%%%%%%%%%%%%%%%%%%%%%%%%%%%%%%%%%%%%%%%%%%%%%%%%%%%%%
% %%%%%%%%%%%%%%%%%%%%%%%%%%%%%%%%%%%%%%%%%%%%%%%%%%%%%%%%%%%%%%%%%%%%%%%%%%%%%%
% \section{Implementation}
%\iffalse
%<*package>
%\fi
%
% This section describes the definitions file |childdoc.def|.

% The definitions cannot be loaded using |\usepackage| or |\RequirePackage|
% which has a mechanism to prevent loading a style file more than once.
% When loading the definitions by means of |\input|
% multiple instances have to be prevented manually:
%\iffalse
%This code needs to be before the `\ProvidesFile' directive
%which is defined at the beginning of this file.
%Therefore it is also placed there and commented out here.
%</package>
%<*discard>
%\fi
%    \begin{macrocode}
\ifdefined\childdocmain\endinput\fi
%    \end{macrocode}
%\iffalse
%</discard>
%<*package>
%\fi
%
% \macro{\ifchilddoc}
% \macro{\ifchilddocmanual}
% The conditional |\ifchilddoc| tells whether a
% child (true) or main (false) document is being compiled.
% The conditional |\ifchilddocmanual| tells whether
% the |\includeonly| mechanism is used (false) or
% the selection of child files must be performed manually (true).
% The definitions initialise to false:
%    \begin{macrocode}
\newif\ifchilddoc
\newif\ifchilddocmanual
%    \end{macrocode}

% \macro{\childdocname}
% \macro{\childdocjob}
% The macro |\childdocname| stores the name of the main document
% to be compiled. The macro |\childdocjob| stores the name of
% the document on which the \LaTeX{} compiler was originally invoked.
% The content of |\jobname| cannot be compared
% to filenames specified in the source due to different catcodes.
% The following code rescans |\jobname|, stores the result
% in |\childdocname| and saves a copy in |\childdocjob|:
%    \begin{macrocode}
\edef\childdocname{\scantokens\expandafter{\jobname\noexpand}}
\let\childdocjob\childdocname
%    \end{macrocode}

% \macro{\childdocdisable}
% The macro |\childdocdisable| prevents the main file
% from being processed more than once.
% At this stage, the main document command |\childdocmain|
% is assumed to be called once again where it should do nothing.
% Any subsequent call to it should prevent
% a secondary processing of the main document
% It overwrites the forwarding commands
% |\childdocof| and |\childdocforward|
% with empty macros to prevent further inclusions of the main document:
%    \begin{macrocode}
\newcommand{\childdocdisable}
{
  \renewcommand{\childdocmain}[1]{\renewcommand{\childdocmain}[1]{\endinput}}
  \renewcommand{\childdocof}[1]{}
  \renewcommand{\childdocby}[2][]{}
  \renewcommand{\childdocforward}[2][]{}
  \renewcommand{\childdocdisable}{}
}
%    \end{macrocode}

% \macro{\childdocmain}
% The macro |\childdocmain| is to be called at the top of the main file
% with nothing or the main filename (without extension) as argument.
% First, it breaks loops.
% If the argument is not empty and does not match |\childdocname|
% (which is set by the first inclusion of |childdoc.def|),
% |\ifchilddoc| is set to true, |\includeonly| is applied to the child file
% and |\jobname| is set to the main file
% (for proper handling of |.aux| files):
%    \begin{macrocode}
\newcommand{\childdocmain}[1]
{
  \childdocdisable\childdocmain{}
  \if?#1?\else
    \begingroup
      \def\childdoctmp{#1}
      \ifx\childdoctmp\childdocname
        \def\childdoctmp{}
      \else
        \def\childdoctmp
        {
          \childdoctrue
          \includeonly{\childdocname}
          \def\childdocjob{#1}
          \def\jobname{#1}
        }
      \fi
      \expandafter
    \endgroup
    \childdoctmp
  \fi
}
%    \end{macrocode}

% \macro{\childdocof}
% The command |\childdocof| redirects
% compilation to the main file |#1|.
%    \begin{macrocode}
\newcommand{\childdocof}[1]
{
  \childdocdisable
  \childdoctrue
  \includeonly{\childdocname}
  \def\jobname{#1}
  \def\childdocjob{#1}
  \input{#1}
}
%    \end{macrocode}

% \macro{\childdocby}
% The command |\childdocby| ....
%    \begin{macrocode}
\newcommand{\childdocby}[2][]
{
  \childdocdisable
  \childdoctrue
  \childdocmanualtrue
  \if?#1?\else
    \def\jobname{#2}
  \fi
  \def\childdocjob{#2}
  \input{#2}
  \endinput
}
%    \end{macrocode}

% \macro{\childdocforward}
% The command |\childdocforward| redirects
% compilation to the main file or
% (if the optional argument is given) a child file.
% Parameters are set as if the main file
% or a child file starting with |\childdocof| was compiled.
% Then compilation is handed over to the main file:
%    \begin{macrocode}
\newcommand{\childdocforward}[2][]
{
  \begingroup
    \if?#1?
      \def\childdoctmp
      {
        \def\childdocname{#2}
        \def\childdocjob{#2}
        \def\jobname{#2}
        \input{#2}
        \endinput
      }
    \else
      \def\childdoctmp
      {
        \childdocdisable
        \def\childdocname{#2}
        \childdoctrue
        \includeonly{#2}
        \def\childdocjob{#1}
        \def\jobname{#1}
        \input{#1}
        \endinput
      }
    \fi
    \expandafter
  \endgroup
  \childdoctmp
}
%    \end{macrocode}

% \macro{\childdocforwardprefix}
% The command |\childdocforwardprefix| redirects
% compilation to the main or a child file by means of a pattern.
% The prefix |#1| in the current filename is replaced by |#2|
% and the suffix of the current filename is kept
% (it is assumed that the filename does not contain the substring `|~~~|'
% which is used as a delimiter).
% Compilation is handed over to the new file by |\childdocforward|:
%    \begin{macrocode}
\newcommand{\childdocforwardprefix}[3][]
{
  \begingroup
    \def\childdocextract #2##1~~~{\def\childdoctmp{\childdocforward[#1]{#3##1}}}
    \expandafter\childdocextract\childdocname~~~
    \expandafter
  \endgroup
  \childdoctmp
}
%    \end{macrocode}

% \macro{\childdoc}
% The deprecated macro |\childdoc| is a legacy version of |\childdocmain|:
%    \begin{macrocode}
\newcommand{\childdoc}{\childdocmain}
%    \end{macrocode}

% \macro{\childdocredirect}
% The deprecated macro |\childdocredirect| is a legacy version
% of |\childdocforward| and |\childdocforwardprefix|:
%    \begin{macrocode}
\newcommand{\childdocredirect}[2][]
{
  \begingroup
    \if?#1?
      \def\childdoctmp{\childdocforward{#2}}
    \else
      \def\childdoctmp{\childdocforwardprefix{#1}{#2}}
    \fi
    \expandafter
  \endgroup
  \childdoctmp
}
%    \end{macrocode}

%\iffalse
%</package>
%\fi
%
\endinput
|\\
|\childdocforward{|\textit{main}|}|
\end{tabular}
\end{center}
%
Likewise, the following files |final|\textit{nn}|.tex|
compile the final version of the child document
|child|\textit{nn}|.tex|:
%
\begin{center}
\begin{tabular}{l}
|\def\version{final}|\\
|% \iffalse
%
% childdoc.dtx Copyright (C) 2017-2018 Niklas Beisert
%
% This work may be distributed and/or modified under the
% conditions of the LaTeX Project Public License, either version 1.3
% of this license or (at your option) any later version.
% The latest version of this license is in
%   http://www.latex-project.org/lppl.txt
% and version 1.3 or later is part of all distributions of LaTeX
% version 2005/12/01 or later.
%
% This work has the LPPL maintenance status `maintained'.
%
% The Current Maintainer of this work is Niklas Beisert.
%
% This work consists of the files childdoc.dtx and childdoc.ins
% and the derived files childdoc.def and cdocsamp.tex with
% cdocsch1.tex, cdocsch2.tex, cdocsdrf.tex, cdocsfn1.tex, cdocsfn2.tex.
%
%<package>\ifdefined\childdocmain\endinput\fi
%<package>\ProvidesFile{childdoc.def}[2018/12/30 v2.0 child document driver]
%<samplemain>\ProvidesFile{cdocsamp.tex}[2018/12/30 v2.0 sample for childdoc]
%<*driver>
%\ProvidesFile{childdoc.drv}[2018/12/30 v2.0 childdoc reference manual file]
\PassOptionsToClass{10pt,a4paper}{article}
\documentclass{ltxdoc}

\usepackage[margin=35mm]{geometry}
\usepackage{hyperref}
\usepackage{hyperxmp}
\usepackage[usenames]{color}

\hypersetup{colorlinks=true}
\hypersetup{pdfstartview=FitH}
\hypersetup{pdfpagemode=UseNone}
\hypersetup{pdfsource={}}
\hypersetup{pdflang={en-UK}}
\hypersetup{pdfcopyright={Copyright 2017-2018 Niklas Beisert.
  This work may be distributed and/or modified under the
  conditions of the LaTeX Project Public License, either version 1.3
  of this license or (at your option) any later version.}}
\hypersetup{pdflicenseurl={http://www.latex-project.org/lppl.txt}}
\hypersetup{pdfcontactaddress={ETH Zurich, ITP, HIT K,
  Wolfgang-Pauli-Strasse 27}}
\hypersetup{pdfcontactpostcode={8093}}
\hypersetup{pdfcontactcity={Zurich}}
\hypersetup{pdfcontactcountry={Switzerland}}
\hypersetup{pdfcontactemail={nbeisert@itp.phys.ethz.ch}}
\hypersetup{pdfcontacturl={http://people.phys.ethz.ch/\xmptilde nbeisert/}}

\newcommand{\secref}[1]{\hyperref[#1]{section \ref*{#1}}}

\parskip1ex
\parindent0pt
\let\olditemize\itemize
\def\itemize{\olditemize\parskip0pt}

\begin{document}

\title{The \textsf{childdoc} Package}
\hypersetup{pdftitle={The childdoc Package}}
\author{Niklas Beisert\\[2ex]
  Institut f\"ur Theoretische Physik\\
  Eidgen\"ossische Technische Hochschule Z\"urich\\
  Wolfgang-Pauli-Strasse 27, 8093 Z\"urich, Switzerland\\[1ex]
  \href{mailto:nbeisert@itp.phys.ethz.ch}
  {\texttt{nbeisert@itp.phys.ethz.ch}}}
\hypersetup{pdfauthor={Niklas Beisert}}
\hypersetup{pdfsubject={Manual for the LaTeX2e Package childdoc}}
\date{30 December 2018, \textsf{v2.0}}
\maketitle

\begin{abstract}\noindent
\textsf{childdoc} is a \LaTeXe{} package
that enables the direct compilation
of document sections included by |\include|
to individual files.
\end{abstract}

\begingroup
\parskip0ex
\tableofcontents
\endgroup

%%%%%%%%%%%%%%%%%%%%%%%%%%%%%%%%%%%%%%%%%%%%%%%%%%%%%%%%%%%%%%%%%%%%%%%%%%%%%%%%
%%%%%%%%%%%%%%%%%%%%%%%%%%%%%%%%%%%%%%%%%%%%%%%%%%%%%%%%%%%%%%%%%%%%%%%%%%%%%%%%
\section{Introduction}

\LaTeX{} provides a mechanism to structure a large document (such as a book)
into a main file and several child files (containing the chapters)
using the |\include| command.
This mechanism is beneficial for documents
which span hundreds of pages in order to
make the source file(s) more manageable.
Moreover, compilation can be restricted to
selected child files by means of the |\includeonly| command.
The latter feature can be used to reduce the compilation time while editing
(this was significantly more useful in the earlier days of \LaTeX{})
or to generate a smaller document which is easier to navigate.
Another application of |\includeonly| is to generate
documents consisting of selected parts of the complete document.

However, there are a few drawbacks of the plain |\include| mechanism:
\begin{itemize}
\item
The child files cannot be compiled on their own,
they can only be compiled via the main file.
A naive editing environment
(such as a text editor with an option
to have the current file processed by \LaTeX)
may require one to switch to the main file before compiling;
attempting to compile the child file produces errors.
\item
The main file must be modified (each time)
to adjust the |\includeonly| command
to the present needs. This easily leaves the main file in a messy state.
\item
The generated document will always carry the filename
of the main document. This is inconvenient if
several child files are to be compiled and
to be kept for distribution.
\end{itemize}

The present package provides a simple interface
to make child files individually compilable by \LaTeX{}.
Compiling a child file then has the same effect as compiling
the main file with an |\includeonly| command
to select the appropriate child.
Moreover the generated document will carry the name of the child
rather than the main file.
This resolves all three above issues.

This feature is meant to make the editing of books,
thesis documents and lecture notes somewhat more convenient.
However, the package can also be used efficiently for
composing a series of documents (such as exercise sheets)
which are typically distributed individually.
It then assists the author in generating the individual documents
(potentially in different versions)
as well as a document containing the collected series.
Another application is in developing style files
or other kinds of included material
where compilation of the style file could redirect
to a sample or test file.

%%%%%%%%%%%%%%%%%%%%%%%%%%%%%%%%%%%%%%%%%%%%%%%%%%%%%%%%%%%%%%%%%%%%%%%%%%%%%%%%
%%%%%%%%%%%%%%%%%%%%%%%%%%%%%%%%%%%%%%%%%%%%%%%%%%%%%%%%%%%%%%%%%%%%%%%%%%%%%%%%
\section{Usage}

First of all, the package \textsf{childdoc} is \emph{not} a standard
\LaTeXe{} |.sty| style file! Therefore it needs to be invoked in
a non-standard way.

%%%%%%%%%%%%%%%%%%%%%%%%%%%%%%%%%%%%%%%%%%%%%%%%%%%%%%%%%%%%%%%%%%%%%%%%%%%%%%%%
\subsection{Included Files}
\label{sec:include}

%%%%%%%%%%%%%%%%%%%%%%%%%%%%%%%%%%%%%%%%
\DescribeMacro{\childdocmain}
To use the package, add the commands
\begin{center}
\begin{tabular}{l}
|\input{childdoc.def}|\\
|\childdocmain{}|\\
\end{tabular}
\end{center}
at the very top of the main \LaTeX{} file,
in particular \emph{before} the |\documentclass| statement!
The argument of |\childdocmain| should be left empty
(but it must be present).

%%%%%%%%%%%%%%%%%%%%%%%%%%%%%%%%%%%%%%%%
\DescribeMacro{\childdocof}
Furthermore, add the commands
\begin{center}
\begin{tabular}{l}
|\input{childdoc.def}|\\
|\childdocof{|\textit{main}|}|\\
\end{tabular}
\end{center}
at the top of every child file \textit{child}
which is included by |\include{|\textit{child}|}|
from within the main file
(or at least for those files to be compiled individually).
The argument \textit{main} must be the filename of the main file.

There are a couple of
considerations in setting up the main and child documents:

%%%%%%%%%%%%%%%%%%%%%%%%%%%%%%%%%%%%%%%%
\paragraph{Restrictions.}

Please note the following restrictions:
\begin{itemize}
\item
|\childdocmain| must be called with one argument \textit{main}
to ensure compatibility with earlier version of the package.
It must either be empty (|\childdocmain{}|)
or precisely match the filename of the main file in which it is specified.
See \secref{sec:detection} for further information.
\item
The filename \textit{main} must be specified without the |.tex| extension.
\item
The filename \textit{main} is case sensitive
(even in case-insensitive file systems)
due to internal string comparison.
\item
The argument \textit{main} should be fully expanded, it cannot be a macro.
\item
Subdirectories and special characters should be avoided in filenames.
\item
The command |\childdocmain{|\textit{main}|}| must be followed by a whitespace.
It should not be followed immediately by another command
or by a comment mark `|%|'.
This is because the \TeX{} parser reads the token immediately following
the argument of |\childdocmain| and puts it
at the beginning of every child section;
however, a white\-space is ignored.
\end{itemize}

%%%%%%%%%%%%%%%%%%%%%%%%%%%%%%%%%%%%%%%%
\paragraph{Content of Main File.}

It is advisable to place all content in the child files included by |\include|.
Any output contained in the main file will appear in all child documents
unless suppressed manually;
it cannot be suppressed automatically by the |\includeonly| directive
and thus should normally be avoided.
A method to include some content in the main file
by means of conditional processing is described in \secref{sec:conditional}.

%%%%%%%%%%%%%%%%%%%%%%%%%%%%%%%%%%%%%%%%
\paragraph{Page Numbering.}

When only a part of the document is compiled,
the appropriate numbering of pages
(as well as other status parameters)
is determined from the |.aux| files.
The latter contain information from previous passes.
However this information needs to propagate through
all intermediate child documents.
Therefore the page numbering in child documents may well
be inconsistent until the complete document is compiled at least once.

A useful (if unconventional) way to always ensure a consistent
page numbering is to restart the numbering in each child document
and denote the pages by `\textit{child}|.|\textit{page}'
where \textit{child} represents the chapter/section number of the child file.
This can be achieved by the command
|\numberwithin{page}{|\textit{child}|}|
of the \textsf{amsmath} package
where \textit{child} can be |chapter| or |section|
depending on the chosen structuring.
Alternatively, one can modify the macro |\thepage| appropriately
and reset the counter |page| at the start of each child file.

%%%%%%%%%%%%%%%%%%%%%%%%%%%%%%%%%%%%%%%%%%%%%%%%%%%%%%%%%%%%%%%%%%%%%%%%%%%%%%%%
\subsection{Conditional Processing}
\label{sec:conditional}

The package provides a mechanism to compile different versions
of a document. To customise the versions further some conditional processing
can come in handy to distinguish which version is being compiled.
The package provides two macros to describe the compilation context:

%%%%%%%%%%%%%%%%%%%%%%%%%%%%%%%%%%%%%%%%
\DescribeMacro{\ifchilddoc}
The conditional |\ifchilddoc| distinguishes between the compilation of
child documents and the main document:
%
\begin{center}
|\ifchilddoc |\textit{child-code}| |[|\||else |\textit{main-code}]| \||fi|
\end{center}

%%%%%%%%%%%%%%%%%%%%%%%%%%%%%%%%%%%%%%%%
\DescribeMacro{\childdocname}
\DescribeMacro{\childdocjob}
The macro |\childdocname| contains the filename (without extension)
of the main or child file being processed.
Note that |\childdocjob| will always contain the name of the main file.

%%%%%%%%%%%%%%%%%%%%%%%%%%%%%%%%%%%%%%%%
\paragraph{Title Page.}

Conditional processing can be used to include a title or banner page
in the main document when proper precautions are taken.
Importantly, the code in the main file should ensure that the page counter
(as well as other status parameters which are stored in the |.aux| files)
takes the same value after the conditional processing.
Otherwise the page numbers may take divergent values
depending on which part is compiled.

For example, a title page could be declared by:
%
\begin{center}
\begin{tabular}{l}
|\ifchilddoc\||else|\\
|\addtocounter{page}{-1}|\\
\textit{code for title page}\\
|\newpage|\\
|\||fi|
\end{tabular}
\end{center}
%
A banner page for the child documents can be generated by:
%
\begin{center}
\begin{tabular}{l}
|\ifchilddoc|\\
|\addtocounter{page}{-1}|\\
\textit{code for banner page}\\
|\newpage|\\
|\||fi|
\end{tabular}
\end{center}
%
Here one could write a message such as:
\begin{center}
|This is the part \childdocname{} of \childdocjob{}.|
\end{center}

%%%%%%%%%%%%%%%%%%%%%%%%%%%%%%%%%%%%%%%%%%%%%%%%%%%%%%%%%%%%%%%%%%%%%%%%%%%%%%%%
\subsection{Flags}
\label{sec:flags}

The package makes it easy to generate different versions
of the main or child documents.
To this end compilation flags can be defined
and assigned different default values.
They will be particularly useful in conjunction
with the forwarding mechanism described in \secref{sec:forward}.

For example, it may be useful to have a flag |\version|
which can be set to |draft| or |final|.
The document source will contain some conditional code
depending on the value of |\version|.
Suppose further, the flag should default to |final| for the main file
and to |draft| for child files
which is a natural assignment for editing the document.
This is achieved by placing the following code
in the preamble of the main document
(below the |\childdocmain| directive):
%
\begin{center}
\begin{tabular}{l}
|\ifchilddoc|\\
|\providecommand{\version}{draft}|\\
|\||else|\\
|\providecommand{\version}{final}|\\
|\||fi|
\end{tabular}
\end{center}
%
The definition by |\providecommand| makes sure
that previous definitions are not overwritten.
Further statements |\providecommand{\version}{...}|
can thus be added before the above code to override it.

For the main file, one might add a line
(between |\childdocmain| and the above block)
%
\begin{center}
|%\ifchilddoc\||else\providecommand{\version}{draft}\||fi|
\end{center}
%
which can be uncommented to produce a draft version.
Likewise one can add a line to the very top of a child file
(above the |\childdocof{|\textit{main}|}| directive)
%
\begin{center}
|%\providecommand{\version}{final}|
\end{center}
%
which can be uncommented to produce the final version of this child document.

%%%%%%%%%%%%%%%%%%%%%%%%%%%%%%%%%%%%%%%%%%%%%%%%%%%%%%%%%%%%%%%%%%%%%%%%%%%%%%%%
\subsection{Forwarding}
\label{sec:forward}

Different versions of the main or child documents
using compilation flags as described in \secref{sec:flags}
can be (permanently) stored in different files
for convenient compilation, viewing and distribution.
To this end, the package defines a command
to pass on compilation to a different file:

%%%%%%%%%%%%%%%%%%%%%%%%%%%%%%%%%%%%%%%%
\DescribeMacro{\childdocforward}
The command |\childdocforward| redirects processing to
another source file:
%
\begin{center}
\begin{tabular}{l}
|\input{childdoc.def}|\\
|\childdocforward[|\textit{main}|]{|\textit{dest}|}|\\
\end{tabular}
\end{center}
%
The argument \textit{dest} is the destination file
(without extension).
It should be the main file or one of the child files.
Note that further \textsf{childdoc} directives
such as |\childdocof| and |\childdocforward|
in the indicated file will be processed in this form.
The optional argument \textit{main}
passes on directly to the main file \textit{main}
while pretending to compile the child \textit{dest}.
This form behaves as if \textit{dest}
issues |\childdocof{|\textit{main}|}| right away,
and no further \textsf{childdoc} directives will be processed.

%%%%%%%%%%%%%%%%%%%%%%%%%%%%%%%%%%%%%%%%
\DescribeMacro{\...prefix}
In the alternative form |\childdocforwardprefix|,
%
\begin{center}
\begin{tabular}{l}
|\input{childdoc.def}|\\
|\childdocforwardprefix[|\textit{main}|]{|\textit{prefix}|}{|\textit{dest}|}|
\end{tabular}
\end{center}
%
the destination file is determined by a pattern
depending on the current file:
To make this work, the current file must be called
`{\textit{prefix}\hspace{0.2em}\textit{suffix}}'
with \textit{prefix} matching precisely the argument.
Processing is then passed on to the file
`{\textit{dest}\hspace{0.2em}\textit{suffix}}'.
Surely, the same effect is achieved by
directly specifying the
argument `{\textit{dest}\hspace{0.2em}\textit{suffix}}'
in the first form.
However, that requires to set up a different file
for each child. With the alternative form of the command
all these files can have exactly the same content
which simplifies setting them up and maintaining them.

For example, the following file |draft.tex|
with a compilation flag |\version| as described in \secref{sec:flags}
compiles the main document as a draft:
%
\begin{center}
\begin{tabular}{l}
|\def\version{draft}|\\
|\input{childdoc.def}|\\
|\childdocforward{|\textit{main}|}|
\end{tabular}
\end{center}
%
Likewise, the following files |final|\textit{nn}|.tex|
compile the final version of the child document
|child|\textit{nn}|.tex|:
%
\begin{center}
\begin{tabular}{l}
|\def\version{final}|\\
|\input{childdoc.def}|\\
|\childdocforwardprefix{final}{child}|
\end{tabular}
\end{center}
%

Note that when several versions of a main file and/or of each child file
are to be generated, it may be convenient to set up a |Makefile| or
shell script to automatise the process.

%%%%%%%%%%%%%%%%%%%%%%%%%%%%%%%%%%%%%%%%%%%%%%%%%%%%%%%%%%%%%%%%%%%%%%%%%%%%%%%%
\subsection{Command Line Processing}
\label{sec:commandline}

The effect of redirection files can also be achieved by invoking
the \LaTeX{} compiler with a more elaborate command line.
Most conveniently this should be done as part
of a shell script or a |Makefile|.

When using \textsf{childdoc} in the main file, the following
command lines effectively perform a redirection
(note that depending on the shell being used,
backslashes may have to be doubled: `|\|' $\to$ `|\\|'):
%
\begin{center}
|... -jobname "|\textit{target}|" |\\|"|[\textit{flags}]%
|\input{childdoc.def}\childdocforward[|\textit{main}|]{|\textit{dest}|}"|
\end{center}
%
Here \textit{target} is the name of the output file,
\textit{main} is the name of the main file
and \textit{dest} is the name of the main or child file to be processed
(all filenames without extensions).
The optional argument \textit{main} can be omitted
if \textit{main} matches \textit{dest}.
Optionally, compilation \textit{flags} can be defined via |\def| commands.
This command line makes the \TeX{} engine believe
it is compiling the file \textit{target}
whose content is specified as the latter parameter.
The provided code then forwards the processing to
\textit{main} or \textit{dest} as described in \secref{sec:forward}.

%%%%%%%%%%%%%%%%%%%%%%%%%%%%%%%%%%%%%%%%%%%%%%%%%%%%%%%%%%%%%%%%%%%%%%%%%%%%%%%%
\subsection{Include by Input}
\label{sec:input}

Including child documents by |\include| has some restrictions by design.
Most notably, the content of a child document always occupies
its own set of pages; pages cannot be shared between child documents.
Usually, this behaviour makes perfect sense
because each child document contain an essential part of the document.
However, in some situations it may be desirable to compose
a document from a collection of parts
without having mandatory page breaks between then.
For this case, the package
provides a mechanism to include parts
by |\input| which can also be processed individually.
However, by construction this mechanism
requires manual handling of the content to be output.

%%%%%%%%%%%%%%%%%%%%%%%%%%%%%%%%%%%%%%%%
\DescribeMacro{\ifchilddocmanual}
The main file should be prepared as usual, see \secref{sec:include}.
However, the document body must make a distinction
between processing of an individual part and of the main document, e.g.:
%
\begin{center}
\begin{tabular}{l}
|\ifchilddocmanual|\\
|\input{\childdocname}|\\
|\||else|\\
\textit{document body with }|\input{|\textit{part}|}|\\
|\||fi|
\end{tabular}
\end{center}
%
The conditional |\ifchilddocmanual| is true whenever
a part to be included by |\input| is being compiled,
and the name of the part is stored in |\childdocname|.

%%%%%%%%%%%%%%%%%%%%%%%%%%%%%%%%%%%%%%%%
\DescribeMacro{\childdocby}
Each part to be included by |\input| should start with:
%
\begin{center}
\begin{tabular}{l}
|\input{childdoc.def}|\\
|\childdocby{|\textit{main}|}|\\
\end{tabular}
\end{center}
%
The directive |\childdocby| is similar to |\childdocof|
described in \secref{sec:include},
but the subsequent selection of content must be done manually.
To that end, both |\ifchilddoc| and |\ifchilddocmanual|
will be true upon processing of a part,
and the name of the part is stored in |\childdocname|.
Note that |\jobname| will be set to the filename of the current part
so that each part receives an individual |.aux| file
that does not interfere with the |.aux| file(s) of the main document.
This behaviour can be altered by the alternative form
|\childdocby[*]{|\textit{main}|}| (with a non-empty optional argument)
which uses the |.aux| file of the main document
by setting |\jobname| to \textit{main}.

%%%%%%%%%%%%%%%%%%%%%%%%%%%%%%%%%%%%%%%%%%%%%%%%%%%%%%%%%%%%%%%%%%%%%%%%%%%%%%%%
\subsection{Driver Development}
\label{sec:driver}

The \textsf{childdoc} mechanism can also be use for the development
of definition files such as \LaTeX{} styles or classes.
This case differs from the above setup with multiple parts
included by |\include| in that no |\includeonly| should be invoked.
This can be achieved by starting the include file
(before |\ProvidesPackage|) with:
%
\begin{center}
\begin{tabular}{l}
|\input{childdoc.def}|\\
|\childdocforward{|\textit{main}|}|\\
\end{tabular}
\end{center}
%
or alternatively with:
%
\begin{center}
\begin{tabular}{l}
|\input{childdoc.def}|\\
|\childdocby{|\textit{main}|}|\\
\end{tabular}
\end{center}
%
Both forms have slightly different effects as described above.
The main file is prepared as usual, see \secref{sec:include}.

%%%%%%%%%%%%%%%%%%%%%%%%%%%%%%%%%%%%%%%%%%%%%%%%%%%%%%%%%%%%%%%%%%%%%%%%%%%%%%%%
\subsection{Legacy Detection}
\label{sec:detection}

The directive |\childdocmain| in the main file can detect
whether the complete document or merely a child is to be compiled
even without using the directive |\childdocof|.
This method is deprecated because it is less robust
and there is no compelling reason to use it;
it is merely provided for backward compatibility
and it may be removed in future versions.

If the detection mechanism is to be used,
it is mandatory to correctly specify
the filename of the main file as the argument of |\childdocmain|:
%
\begin{center}
\begin{tabular}{l}
|\input{childdoc.def}|\\
|\childdocmain{|\textit{main}|}|\\
\end{tabular}
\end{center}
%
If |\jobname| does not match the argument \textit{main} of |\childdocmain|,
it is assumed that |\jobname| points to the child file to be compiled.
When using |\childdocmain| with the main file specified as argument,
it suffices to start a child file
with just |\input{|\textit{main}|}|
without loading of the package and using |\childdocof|.
If instead all processing is done
with the appropriate \textsf{childdoc} directives,
the argument of \textit{main} of |\childdocmain| can be empty.

An alternative version of the command line processing described
in \secref{sec:commandline} using the detection mechanism reads:
%
\begin{center}
|... -jobname "|\textit{target}|" "|[\textit{flags}]%
[|\def\jobname{|\textit{dest}|}|]|\input{|\textit{main}|}"|
\end{center}

%%%%%%%%%%%%%%%%%%%%%%%%%%%%%%%%%%%%%%%%%%%%%%%%%%%%%%%%%%%%%%%%%%%%%%%%%%%%%%%%
\subsection{Manual Code}
\label{sec:manual}

In case one cannot be certain whether the definitions file |childdoc.def|
is installed on the target \TeX{} distribution
and one prefers not to ship it,
it is conceivable to paste a few relevant commands into the sources.

To that end, drop all statements |\input{childdoc.def}|
and perform the replacements as outlined below.
Instead of |\childdocmain{|\textit{main}|}| add the following code
to the top of the main file:
%
\begin{center}
\begin{tabular}{l}
|\||ifdefined\childdocname\endinput\||fi\newif\ifchilddoc|\\
|\edef\childdocname{\scantokens\expandafter{\jobname\noexpand}}|\\
|\def\childdocmain{|\textit{main}|}\||ifx\childdocmain\childdocname\||else|\\
|\childdoctrue\includeonly{\childdocname}\let\jobname\childdocmain\||fi|\\
\end{tabular}
\end{center}
%
Instead of |\childdocof{|\textit{main}|}| just include the main file
at the top of each child file:
%
\begin{center}
|\input{|\textit{main}|}|
\end{center}
%
A simple redirection |\childdocforward{|\textit{dest}|}| is achieved by:
%
\begin{center}
|\def\jobname{|\textit{dest}|}\input{\jobname}|
\end{center}
%
The redirection with prefix
|\childdocforwardprefix[|\textit{prefix}|]{|\textit{dest}|}|
is accomplished by:
%
\begin{center}
\begin{tabular}{l}
|{\edef\jobname{\scantokens\expandafter{\jobname\noexpand}}|\\
|\def\redirectjob |\textit{prefix}|#1~~~{\gdef\jobname{|\textit{dest}|#1}}|\\
|\expandafter\redirectjob\jobname~~~}\input{\jobname}|
\end{tabular}
\end{center}

In an alternative approach,
child documents can be compiled by a specific command line
without additional code or specific definitions:
%
\begin{center}
|... -jobname "|\textit{target}|" "|[\textit{flags}]%
|\includeonly{|\textit{dest}|}\input{|\textit{main}|}"|
\end{center}
%

%%%%%%%%%%%%%%%%%%%%%%%%%%%%%%%%%%%%%%%%%%%%%%%%%%%%%%%%%%%%%%%%%%%%%%%%%%%%%%%%
%%%%%%%%%%%%%%%%%%%%%%%%%%%%%%%%%%%%%%%%%%%%%%%%%%%%%%%%%%%%%%%%%%%%%%%%%%%%%%%%
\section{Information}

%%%%%%%%%%%%%%%%%%%%%%%%%%%%%%%%%%%%%%%%%%%%%%%%%%%%%%%%%%%%%%%%%%%%%%%%%%%%%%%%
\subsection{Copyright}

Copyright \copyright{} 2017--2018 Niklas Beisert

This work may be distributed and/or modified under the
conditions of the \LaTeX{} Project Public License, either version 1.3
of this license or (at your option) any later version.
The latest version of this license is in
  \url{http://www.latex-project.org/lppl.txt}
and version 1.3 or later is part of all distributions of \LaTeX{}
version 2005/12/01 or later.

This work has the LPPL maintenance status `maintained'.

The Current Maintainer of this work is Niklas Beisert.

This work consists of the files |README.txt|, |childdoc.ins| and |childdoc.dtx|
as well as the derived files |childdoc.def|, |cdocsamp.tex|
with |cdocsch1.tex|, |cdocsch2.tex|, |cdocspt3.tex|, |cdocspt4.tex|,
|cdocsdrf.tex|, |cdocsfn1.tex|, |cdocsfn2.tex|
as well as |childdoc.pdf|.

%%%%%%%%%%%%%%%%%%%%%%%%%%%%%%%%%%%%%%%%%%%%%%%%%%%%%%%%%%%%%%%%%%%%%%%%%%%%%%%%
\subsection{Files and Installation}

The package consists of the files:
%
\begin{center}
\begin{tabular}{ll}
    |README.txt|   & readme file \\
    |childdoc.ins| & installation file \\
    |childdoc.dtx| & source file \\
    |childdoc.def| & definition file \\
    |cdocsamp.tex| & sample main file \\
    |cdocsch1.tex| & sample include file \\
    |cdocsch2.tex| & sample include file \\
    |cdocspt3.tex| & sample part file \\
    |cdocspt4.tex| & sample part file \\
    |cdocsdrf.tex| & sample redirection file \\
    |cdocsfn1.tex| & sample redirection file \\
    |cdocsfn2.tex| & sample redirection file \\
    |childdoc.pdf| & manual
\end{tabular}
\end{center}
%
The distribution consists of the files
|README.txt|, |childdoc.ins| and |childdoc.dtx|.
%
\begin{itemize}
\item
Run (pdf)\LaTeX{} on |childdoc.dtx|
to compile the manual |childdoc.pdf| (this file).
\item
Run \LaTeX{} on |childdoc.ins| to create the definitions file |childdoc.def|
and the sample |cdocsamp.tex| with include files
|cdocsch1.tex|, |cdocsch2.tex|, |cdocspt3.tex|, |cdocspt4.tex|,
|cdocsdrf.tex|, |cdocsfn1.tex|, |cdocsfn2.tex|.
Then copy the file |childdoc.def| to an appropriate directory of your \LaTeX{}
distribution, e.g.\ \textit{texmf-root}|/tex/latex/childdoc|.
\end{itemize}

%%%%%%%%%%%%%%%%%%%%%%%%%%%%%%%%%%%%%%%%%%%%%%%%%%%%%%%%%%%%%%%%%%%%%%%%%%%%%%%%
\subsection{Related CTAN Packages}

There are several other packages which offer a similar functionality:
%
\begin{itemize}
\item
The packages
\href{http://ctan.org/pkg/docmute}{\textsf{docmute}},
\href{http://ctan.org/pkg/includex}{\textsf{includex}} and
\href{http://ctan.org/pkg/standalone}{\textsf{standalone}}
provide commands to include only the document body of
a child file thus allowing both files to be compiled individually.
\item
The packages \href{http://ctan.org/pkg/subdocs}{\textsf{subdocs}}
and \href{http://ctan.org/pkg/subfiles}{\textsf{subfiles}}
provide structures in which the main and child documents can be
encapsulated and allowing them to be compiled individually.
The inclusion mechanism is different from the conventional |\include|.
\item
The package \href{http://ctan.org/pkg/combine}{\textsf{combine}}
is an elaborate solution to combine several documents into one.
\end{itemize}
%
See also the CTAN topic \href{http://ctan.org/topic/subdocs}{\textsf{subdocs}}
for further related packages.
The present package differs from the above solutions in that
a document structure constructed with the conventional |\include| mechanism
just needs two extra commands at the top of every file
such that all constituent files can be compiled individually.

%%%%%%%%%%%%%%%%%%%%%%%%%%%%%%%%%%%%%%%%%%%%%%%%%%%%%%%%%%%%%%%%%%%%%%%%%%%%%%%%
%\subsection{Feature Suggestions}
%
%The following is a list of features which may be useful for future
%versions of this package:
%%
%\begin{itemize}
%\item
%\ldots
%\end{itemize}

%%%%%%%%%%%%%%%%%%%%%%%%%%%%%%%%%%%%%%%%%%%%%%%%%%%%%%%%%%%%%%%%%%%%%%%%%%%%%%%%
\subsection{Revision History}

%%%%%%%%%%%%%%%%%%%%%%%%%%%%%%%%%%%%%%%%
\paragraph{v2.0:} 2018/12/30

\begin{itemize}
\item
immediate forward processing
\item
added |\childdocby| mechanism
\item
manual restructured
\end{itemize}

%%%%%%%%%%%%%%%%%%%%%%%%%%%%%%%%%%%%%%%%
\paragraph{v1.6:} 2018/01/17

\begin{itemize}
\item
application for development of include files
\item
corrections to manual
\end{itemize}

%%%%%%%%%%%%%%%%%%%%%%%%%%%%%%%%%%%%%%%%
\paragraph{v1.5:} 2017/05/21

\begin{itemize}
\item
more complete structuring introduced
\item
|\childdocof| introduced
\item
|\childdoc| renamed to |\childdocmain|
\item
|\childredirect| renamed to |\childdocforward| and |\childdocforwardprefix|
and functionality expanded
\end{itemize}

%%%%%%%%%%%%%%%%%%%%%%%%%%%%%%%%%%%%%%%%
\paragraph{v1.0:} 2017/04/27

\begin{itemize}
\item
manual and install package
\item
first version published on CTAN
\end{itemize}

%%%%%%%%%%%%%%%%%%%%%%%%%%%%%%%%%%%%%%%%
\paragraph{v0.6:} 2017/04/26

\begin{itemize}
\item
redirection mechanism added
\end{itemize}

%%%%%%%%%%%%%%%%%%%%%%%%%%%%%%%%%%%%%%%%
\paragraph{v0.5:} 2017/04/26

\begin{itemize}
\item
functionality in definition file
\end{itemize}


%%%%%%%%%%%%%%%%%%%%%%%%%%%%%%%%%%%%%%%%%%%%%%%%%%%%%%%%%%%%%%%%%%%%%%%%%%%%%%%%
%%%%%%%%%%%%%%%%%%%%%%%%%%%%%%%%%%%%%%%%%%%%%%%%%%%%%%%%%%%%%%%%%%%%%%%%%%%%%%%%
%%%%%%%%%%%%%%%%%%%%%%%%%%%%%%%%%%%%%%%%%%%%%%%%%%%%%%%%%%%%%%%%%%%%%%%%%%%%%%%%
\appendix

\settowidth\MacroIndent{\rmfamily\scriptsize 000\ }

 \DocInput{childdoc.dtx}

\end{document}
%</driver>
% \fi
%
% %%%%%%%%%%%%%%%%%%%%%%%%%%%%%%%%%%%%%%%%%%%%%%%%%%%%%%%%%%%%%%%%%%%%%%%%%%%%%%
% %%%%%%%%%%%%%%%%%%%%%%%%%%%%%%%%%%%%%%%%%%%%%%%%%%%%%%%%%%%%%%%%%%%%%%%%%%%%%%
% \section{Sample}
%\iffalse
%<*samplemain>
%\fi
%
% The following presents a sample document
% with two chapters, two parts, a title page,
% a compile flag as well as three forwarding files to set the flag.
% It consists of eight |.tex| files:
% \begin{center}
% \begin{tabular}{ll}
% |cdocsamp.tex|&main file\\
% |cdocsch1.tex|&include file for chapter 1\\
% |cdocsch2.tex|&include file for chapter 2\\
% |cdocspt3.tex|&include file for part 3\\
% |cdocspt4.tex|&include file for part 4\\
% |cdocsdrf.tex|&forwarding file for main file in draft mode\\
% |cdocsfi1.tex|&forwarding file for final version of chapter 1\\
% |cdocsfi2.tex|&forwarding file for final version of chapter 2\\
% \end{tabular}
% \end{center}
% Each of the eight files can be compiled directly by the \LaTeX{} compiler.
%
% %%%%%%%%%%%%%%%%%%%%%%%%%%%%%%%%%%%%%%
% \paragraph{Main File.}
%
% The main file is called |cdocsamp.tex|.
%
% Load the \textsf{childdoc} definitions and
% declare the filename for the main document:
%    \begin{macrocode}
\input{childdoc.def}
\childdocmain{}
%    \end{macrocode}

% Optional override for |\version| flag:
%    \begin{macrocode}
%%\ifchilddoc\else\providecommand{\version}{draft}\fi
%    \end{macrocode}

% Define the default values for the |\version| flag
% (|final| for the main file and |draft| for childs):
%    \begin{macrocode}
\ifchilddoc
\providecommand{\version}{draft}
\else
\providecommand{\version}{final}
\fi
%    \end{macrocode}

% Load the standard document class:
%    \begin{macrocode}
\documentclass[12pt]{article}
%    \end{macrocode}

% Start the document body:
%    \begin{macrocode}
\begin{document}
%    \end{macrocode}

% Declare a title page.
% Print title, part of document being processed and version flag:
%    \begin{macrocode}
\addtocounter{page}{-1}
\begin{center}
{\LARGE\bfseries{}childdoc example\par}
\vspace{1cm}
\ifchilddoc
\ifchilddocmanual part\else chapter\fi:
`\childdocname' of `\childdocjob'\par
\else
main document: `\childdocjob'\par
\fi
version: \version\par
\end{center}
\newpage
%    \end{macrocode}

% Manually include selected file,
% otherwise process as usual:
%    \begin{macrocode}
\ifchilddocmanual
\section*{part `\childdocname'}
\input{\childdocname}
\else
%    \end{macrocode}

% Include the two chapters:
%    \begin{macrocode}
\include{cdocsch1}
\include{cdocsch2}
%    \end{macrocode}

% Include the two parts unless only chapters should be displayed:
%    \begin{macrocode}
\ifchilddoc\else
\section{part three}
\input{cdocspt3}
\section{part four}
\input{cdocspt4}
\fi
%    \end{macrocode}

% Process as usual until here:
%    \begin{macrocode}
\fi
%    \end{macrocode}

% End of document body:
%    \begin{macrocode}
\end{document}
%    \end{macrocode}
%\iffalse
%</samplemain>
%\fi
%
% %%%%%%%%%%%%%%%%%%%%%%%%%%%%%%%%%%%%%%
% \paragraph{Chapter Include Files.}
%
% The include files are called |cdocsch1.tex| and |cdocsch2.tex|.
%
%\iffalse
%<*samplechap1|samplechap2>
%\fi

% Optional override for |\version| flag:
%    \begin{macrocode}
%%\providecommand{\version}{final}
%    \end{macrocode}

% Include the main document:
%    \begin{macrocode}
\input{childdoc.def}
\childdocof{cdocsamp}
%    \end{macrocode}

%\iffalse
%</samplechap1|samplechap2>
%\fi
%
%\iffalse
%<*samplechap1>
%\fi
% Some text for chapter 1:
%    \begin{macrocode}
\section{one}
some text in chapter one
%    \end{macrocode}

%\iffalse
%</samplechap1>
%\fi
% Some text for chapter 2:
%\iffalse
%<*samplechap2>
%\fi
%    \begin{macrocode}
\section{two}
more text in chapter two
%    \end{macrocode}

%\iffalse
%</samplechap2>
%\fi
%
% %%%%%%%%%%%%%%%%%%%%%%%%%%%%%%%%%%%%%%
% \paragraph{Part Include Files.}
%
% The include files are called |cdocspt3.tex| and |cdocspt4.tex|.
%
%\iffalse
%<*samplepart3|samplepart4>
%\fi

% Optional override for |\version| flag:
%    \begin{macrocode}
%%\providecommand{\version}{final}
%    \end{macrocode}

% Include the main document:
%    \begin{macrocode}
\input{childdoc.def}
\childdocby{cdocsamp}
%    \end{macrocode}

%\iffalse
%</samplepart3|samplepart4>
%\fi
%
%\iffalse
%<*samplepart3>
%\fi
% Some text for part 3:
%    \begin{macrocode}
some text in part three
%    \end{macrocode}

%\iffalse
%</samplepart3>
%\fi
% Some text for part 4:
%\iffalse
%<*samplepart4>
%\fi
%    \begin{macrocode}
more text in part four
%    \end{macrocode}

%\iffalse
%</samplepart4>
%\fi
%
% %%%%%%%%%%%%%%%%%%%%%%%%%%%%%%%%%%%%%%
% \paragraph{Forwarding for a Complete Draft.}
%
% The following forwarding file |cdocsdrf.tex|
% compiles the main document in draft mode:
%\iffalse
%<*sampledraft>
%\fi
%    \begin{macrocode}
\def\version{draft}
\input{childdoc.def}
\childdocforward{cdocsamp}
%    \end{macrocode}

%\iffalse
%</sampledraft>
%\fi
%
% %%%%%%%%%%%%%%%%%%%%%%%%%%%%%%%%%%%%%%
% \paragraph{Forwarding for Final Version of the Chapters.}
%
% The following forwarding files |cdocsfn1.tex| and |cdocsfn2.tex|
% (with identical content)
% compile the final versions of the child documents
% |cdocsch1.tex| and |cdocsch2.tex|, respectively:
%\iffalse
%<*samplefinal>
%\fi
%    \begin{macrocode}
\def\version{final}
\input{childdoc.def}
\childdocforwardprefix[cdocsamp]{cdocsfn}{cdocsch}
%    \end{macrocode}

%\iffalse
%</samplefinal>
%\fi
%
% %%%%%%%%%%%%%%%%%%%%%%%%%%%%%%%%%%%%%%
% \paragraph{Command Line Processing.}
%
% The following three command lines generate the output files
% |cdocscld|, |cdocscl1| and |cdocscl2|
% which should be identical to
% |cdocsdrf|, |cdocsch1| and |cdocsfn2|, respectively:
% \begin{center}
% \begin{tabular}{l}
% |latex -jobname cdocscld \|\\
% |  "\def\version{draft}\input{childdoc.def}\childdocforward{cdocsamp}"|\\
% |latex -jobname cdocscl1 \|\\
% |  "\input{childdoc.def}\childdocforward[cdocsamp]{cdocsch1}"|\\
% |latex -jobname cdocscl2 \|\\
% |  "\def\version{final}\input{childdoc.def}\childdocforward{cdocsch2}"|
% \end{tabular}
% \end{center}
% Note that the trailing backslash on each first line
% merely continues the input to the second line
% (for convenient cut ant paste).
% Furthermore, the command |latex| can be replaced by any
% of its alternative versions such as |pdflatex|.
%
% %%%%%%%%%%%%%%%%%%%%%%%%%%%%%%%%%%%%%%%%%%%%%%%%%%%%%%%%%%%%%%%%%%%%%%%%%%%%%%
% %%%%%%%%%%%%%%%%%%%%%%%%%%%%%%%%%%%%%%%%%%%%%%%%%%%%%%%%%%%%%%%%%%%%%%%%%%%%%%
% \section{Implementation}
%\iffalse
%<*package>
%\fi
%
% This section describes the definitions file |childdoc.def|.

% The definitions cannot be loaded using |\usepackage| or |\RequirePackage|
% which has a mechanism to prevent loading a style file more than once.
% When loading the definitions by means of |\input|
% multiple instances have to be prevented manually:
%\iffalse
%This code needs to be before the `\ProvidesFile' directive
%which is defined at the beginning of this file.
%Therefore it is also placed there and commented out here.
%</package>
%<*discard>
%\fi
%    \begin{macrocode}
\ifdefined\childdocmain\endinput\fi
%    \end{macrocode}
%\iffalse
%</discard>
%<*package>
%\fi
%
% \macro{\ifchilddoc}
% \macro{\ifchilddocmanual}
% The conditional |\ifchilddoc| tells whether a
% child (true) or main (false) document is being compiled.
% The conditional |\ifchilddocmanual| tells whether
% the |\includeonly| mechanism is used (false) or
% the selection of child files must be performed manually (true).
% The definitions initialise to false:
%    \begin{macrocode}
\newif\ifchilddoc
\newif\ifchilddocmanual
%    \end{macrocode}

% \macro{\childdocname}
% \macro{\childdocjob}
% The macro |\childdocname| stores the name of the main document
% to be compiled. The macro |\childdocjob| stores the name of
% the document on which the \LaTeX{} compiler was originally invoked.
% The content of |\jobname| cannot be compared
% to filenames specified in the source due to different catcodes.
% The following code rescans |\jobname|, stores the result
% in |\childdocname| and saves a copy in |\childdocjob|:
%    \begin{macrocode}
\edef\childdocname{\scantokens\expandafter{\jobname\noexpand}}
\let\childdocjob\childdocname
%    \end{macrocode}

% \macro{\childdocdisable}
% The macro |\childdocdisable| prevents the main file
% from being processed more than once.
% At this stage, the main document command |\childdocmain|
% is assumed to be called once again where it should do nothing.
% Any subsequent call to it should prevent
% a secondary processing of the main document
% It overwrites the forwarding commands
% |\childdocof| and |\childdocforward|
% with empty macros to prevent further inclusions of the main document:
%    \begin{macrocode}
\newcommand{\childdocdisable}
{
  \renewcommand{\childdocmain}[1]{\renewcommand{\childdocmain}[1]{\endinput}}
  \renewcommand{\childdocof}[1]{}
  \renewcommand{\childdocby}[2][]{}
  \renewcommand{\childdocforward}[2][]{}
  \renewcommand{\childdocdisable}{}
}
%    \end{macrocode}

% \macro{\childdocmain}
% The macro |\childdocmain| is to be called at the top of the main file
% with nothing or the main filename (without extension) as argument.
% First, it breaks loops.
% If the argument is not empty and does not match |\childdocname|
% (which is set by the first inclusion of |childdoc.def|),
% |\ifchilddoc| is set to true, |\includeonly| is applied to the child file
% and |\jobname| is set to the main file
% (for proper handling of |.aux| files):
%    \begin{macrocode}
\newcommand{\childdocmain}[1]
{
  \childdocdisable\childdocmain{}
  \if?#1?\else
    \begingroup
      \def\childdoctmp{#1}
      \ifx\childdoctmp\childdocname
        \def\childdoctmp{}
      \else
        \def\childdoctmp
        {
          \childdoctrue
          \includeonly{\childdocname}
          \def\childdocjob{#1}
          \def\jobname{#1}
        }
      \fi
      \expandafter
    \endgroup
    \childdoctmp
  \fi
}
%    \end{macrocode}

% \macro{\childdocof}
% The command |\childdocof| redirects
% compilation to the main file |#1|.
%    \begin{macrocode}
\newcommand{\childdocof}[1]
{
  \childdocdisable
  \childdoctrue
  \includeonly{\childdocname}
  \def\jobname{#1}
  \def\childdocjob{#1}
  \input{#1}
}
%    \end{macrocode}

% \macro{\childdocby}
% The command |\childdocby| ....
%    \begin{macrocode}
\newcommand{\childdocby}[2][]
{
  \childdocdisable
  \childdoctrue
  \childdocmanualtrue
  \if?#1?\else
    \def\jobname{#2}
  \fi
  \def\childdocjob{#2}
  \input{#2}
  \endinput
}
%    \end{macrocode}

% \macro{\childdocforward}
% The command |\childdocforward| redirects
% compilation to the main file or
% (if the optional argument is given) a child file.
% Parameters are set as if the main file
% or a child file starting with |\childdocof| was compiled.
% Then compilation is handed over to the main file:
%    \begin{macrocode}
\newcommand{\childdocforward}[2][]
{
  \begingroup
    \if?#1?
      \def\childdoctmp
      {
        \def\childdocname{#2}
        \def\childdocjob{#2}
        \def\jobname{#2}
        \input{#2}
        \endinput
      }
    \else
      \def\childdoctmp
      {
        \childdocdisable
        \def\childdocname{#2}
        \childdoctrue
        \includeonly{#2}
        \def\childdocjob{#1}
        \def\jobname{#1}
        \input{#1}
        \endinput
      }
    \fi
    \expandafter
  \endgroup
  \childdoctmp
}
%    \end{macrocode}

% \macro{\childdocforwardprefix}
% The command |\childdocforwardprefix| redirects
% compilation to the main or a child file by means of a pattern.
% The prefix |#1| in the current filename is replaced by |#2|
% and the suffix of the current filename is kept
% (it is assumed that the filename does not contain the substring `|~~~|'
% which is used as a delimiter).
% Compilation is handed over to the new file by |\childdocforward|:
%    \begin{macrocode}
\newcommand{\childdocforwardprefix}[3][]
{
  \begingroup
    \def\childdocextract #2##1~~~{\def\childdoctmp{\childdocforward[#1]{#3##1}}}
    \expandafter\childdocextract\childdocname~~~
    \expandafter
  \endgroup
  \childdoctmp
}
%    \end{macrocode}

% \macro{\childdoc}
% The deprecated macro |\childdoc| is a legacy version of |\childdocmain|:
%    \begin{macrocode}
\newcommand{\childdoc}{\childdocmain}
%    \end{macrocode}

% \macro{\childdocredirect}
% The deprecated macro |\childdocredirect| is a legacy version
% of |\childdocforward| and |\childdocforwardprefix|:
%    \begin{macrocode}
\newcommand{\childdocredirect}[2][]
{
  \begingroup
    \if?#1?
      \def\childdoctmp{\childdocforward{#2}}
    \else
      \def\childdoctmp{\childdocforwardprefix{#1}{#2}}
    \fi
    \expandafter
  \endgroup
  \childdoctmp
}
%    \end{macrocode}

%\iffalse
%</package>
%\fi
%
\endinput
|\\
|\childdocforwardprefix{final}{child}|
\end{tabular}
\end{center}
%

Note that when several versions of a main file and/or of each child file
are to be generated, it may be convenient to set up a |Makefile| or
shell script to automatise the process.

%%%%%%%%%%%%%%%%%%%%%%%%%%%%%%%%%%%%%%%%%%%%%%%%%%%%%%%%%%%%%%%%%%%%%%%%%%%%%%%%
\subsection{Command Line Processing}
\label{sec:commandline}

The effect of redirection files can also be achieved by invoking
the \LaTeX{} compiler with a more elaborate command line.
Most conveniently this should be done as part
of a shell script or a |Makefile|.

When using \textsf{childdoc} in the main file, the following
command lines effectively perform a redirection
(note that depending on the shell being used,
backslashes may have to be doubled: `|\|' $\to$ `|\\|'):
%
\begin{center}
|... -jobname "|\textit{target}|" |\\|"|[\textit{flags}]%
|% \iffalse
%
% childdoc.dtx Copyright (C) 2017-2018 Niklas Beisert
%
% This work may be distributed and/or modified under the
% conditions of the LaTeX Project Public License, either version 1.3
% of this license or (at your option) any later version.
% The latest version of this license is in
%   http://www.latex-project.org/lppl.txt
% and version 1.3 or later is part of all distributions of LaTeX
% version 2005/12/01 or later.
%
% This work has the LPPL maintenance status `maintained'.
%
% The Current Maintainer of this work is Niklas Beisert.
%
% This work consists of the files childdoc.dtx and childdoc.ins
% and the derived files childdoc.def and cdocsamp.tex with
% cdocsch1.tex, cdocsch2.tex, cdocsdrf.tex, cdocsfn1.tex, cdocsfn2.tex.
%
%<package>\ifdefined\childdocmain\endinput\fi
%<package>\ProvidesFile{childdoc.def}[2018/12/30 v2.0 child document driver]
%<samplemain>\ProvidesFile{cdocsamp.tex}[2018/12/30 v2.0 sample for childdoc]
%<*driver>
%\ProvidesFile{childdoc.drv}[2018/12/30 v2.0 childdoc reference manual file]
\PassOptionsToClass{10pt,a4paper}{article}
\documentclass{ltxdoc}

\usepackage[margin=35mm]{geometry}
\usepackage{hyperref}
\usepackage{hyperxmp}
\usepackage[usenames]{color}

\hypersetup{colorlinks=true}
\hypersetup{pdfstartview=FitH}
\hypersetup{pdfpagemode=UseNone}
\hypersetup{pdfsource={}}
\hypersetup{pdflang={en-UK}}
\hypersetup{pdfcopyright={Copyright 2017-2018 Niklas Beisert.
  This work may be distributed and/or modified under the
  conditions of the LaTeX Project Public License, either version 1.3
  of this license or (at your option) any later version.}}
\hypersetup{pdflicenseurl={http://www.latex-project.org/lppl.txt}}
\hypersetup{pdfcontactaddress={ETH Zurich, ITP, HIT K,
  Wolfgang-Pauli-Strasse 27}}
\hypersetup{pdfcontactpostcode={8093}}
\hypersetup{pdfcontactcity={Zurich}}
\hypersetup{pdfcontactcountry={Switzerland}}
\hypersetup{pdfcontactemail={nbeisert@itp.phys.ethz.ch}}
\hypersetup{pdfcontacturl={http://people.phys.ethz.ch/\xmptilde nbeisert/}}

\newcommand{\secref}[1]{\hyperref[#1]{section \ref*{#1}}}

\parskip1ex
\parindent0pt
\let\olditemize\itemize
\def\itemize{\olditemize\parskip0pt}

\begin{document}

\title{The \textsf{childdoc} Package}
\hypersetup{pdftitle={The childdoc Package}}
\author{Niklas Beisert\\[2ex]
  Institut f\"ur Theoretische Physik\\
  Eidgen\"ossische Technische Hochschule Z\"urich\\
  Wolfgang-Pauli-Strasse 27, 8093 Z\"urich, Switzerland\\[1ex]
  \href{mailto:nbeisert@itp.phys.ethz.ch}
  {\texttt{nbeisert@itp.phys.ethz.ch}}}
\hypersetup{pdfauthor={Niklas Beisert}}
\hypersetup{pdfsubject={Manual for the LaTeX2e Package childdoc}}
\date{30 December 2018, \textsf{v2.0}}
\maketitle

\begin{abstract}\noindent
\textsf{childdoc} is a \LaTeXe{} package
that enables the direct compilation
of document sections included by |\include|
to individual files.
\end{abstract}

\begingroup
\parskip0ex
\tableofcontents
\endgroup

%%%%%%%%%%%%%%%%%%%%%%%%%%%%%%%%%%%%%%%%%%%%%%%%%%%%%%%%%%%%%%%%%%%%%%%%%%%%%%%%
%%%%%%%%%%%%%%%%%%%%%%%%%%%%%%%%%%%%%%%%%%%%%%%%%%%%%%%%%%%%%%%%%%%%%%%%%%%%%%%%
\section{Introduction}

\LaTeX{} provides a mechanism to structure a large document (such as a book)
into a main file and several child files (containing the chapters)
using the |\include| command.
This mechanism is beneficial for documents
which span hundreds of pages in order to
make the source file(s) more manageable.
Moreover, compilation can be restricted to
selected child files by means of the |\includeonly| command.
The latter feature can be used to reduce the compilation time while editing
(this was significantly more useful in the earlier days of \LaTeX{})
or to generate a smaller document which is easier to navigate.
Another application of |\includeonly| is to generate
documents consisting of selected parts of the complete document.

However, there are a few drawbacks of the plain |\include| mechanism:
\begin{itemize}
\item
The child files cannot be compiled on their own,
they can only be compiled via the main file.
A naive editing environment
(such as a text editor with an option
to have the current file processed by \LaTeX)
may require one to switch to the main file before compiling;
attempting to compile the child file produces errors.
\item
The main file must be modified (each time)
to adjust the |\includeonly| command
to the present needs. This easily leaves the main file in a messy state.
\item
The generated document will always carry the filename
of the main document. This is inconvenient if
several child files are to be compiled and
to be kept for distribution.
\end{itemize}

The present package provides a simple interface
to make child files individually compilable by \LaTeX{}.
Compiling a child file then has the same effect as compiling
the main file with an |\includeonly| command
to select the appropriate child.
Moreover the generated document will carry the name of the child
rather than the main file.
This resolves all three above issues.

This feature is meant to make the editing of books,
thesis documents and lecture notes somewhat more convenient.
However, the package can also be used efficiently for
composing a series of documents (such as exercise sheets)
which are typically distributed individually.
It then assists the author in generating the individual documents
(potentially in different versions)
as well as a document containing the collected series.
Another application is in developing style files
or other kinds of included material
where compilation of the style file could redirect
to a sample or test file.

%%%%%%%%%%%%%%%%%%%%%%%%%%%%%%%%%%%%%%%%%%%%%%%%%%%%%%%%%%%%%%%%%%%%%%%%%%%%%%%%
%%%%%%%%%%%%%%%%%%%%%%%%%%%%%%%%%%%%%%%%%%%%%%%%%%%%%%%%%%%%%%%%%%%%%%%%%%%%%%%%
\section{Usage}

First of all, the package \textsf{childdoc} is \emph{not} a standard
\LaTeXe{} |.sty| style file! Therefore it needs to be invoked in
a non-standard way.

%%%%%%%%%%%%%%%%%%%%%%%%%%%%%%%%%%%%%%%%%%%%%%%%%%%%%%%%%%%%%%%%%%%%%%%%%%%%%%%%
\subsection{Included Files}
\label{sec:include}

%%%%%%%%%%%%%%%%%%%%%%%%%%%%%%%%%%%%%%%%
\DescribeMacro{\childdocmain}
To use the package, add the commands
\begin{center}
\begin{tabular}{l}
|\input{childdoc.def}|\\
|\childdocmain{}|\\
\end{tabular}
\end{center}
at the very top of the main \LaTeX{} file,
in particular \emph{before} the |\documentclass| statement!
The argument of |\childdocmain| should be left empty
(but it must be present).

%%%%%%%%%%%%%%%%%%%%%%%%%%%%%%%%%%%%%%%%
\DescribeMacro{\childdocof}
Furthermore, add the commands
\begin{center}
\begin{tabular}{l}
|\input{childdoc.def}|\\
|\childdocof{|\textit{main}|}|\\
\end{tabular}
\end{center}
at the top of every child file \textit{child}
which is included by |\include{|\textit{child}|}|
from within the main file
(or at least for those files to be compiled individually).
The argument \textit{main} must be the filename of the main file.

There are a couple of
considerations in setting up the main and child documents:

%%%%%%%%%%%%%%%%%%%%%%%%%%%%%%%%%%%%%%%%
\paragraph{Restrictions.}

Please note the following restrictions:
\begin{itemize}
\item
|\childdocmain| must be called with one argument \textit{main}
to ensure compatibility with earlier version of the package.
It must either be empty (|\childdocmain{}|)
or precisely match the filename of the main file in which it is specified.
See \secref{sec:detection} for further information.
\item
The filename \textit{main} must be specified without the |.tex| extension.
\item
The filename \textit{main} is case sensitive
(even in case-insensitive file systems)
due to internal string comparison.
\item
The argument \textit{main} should be fully expanded, it cannot be a macro.
\item
Subdirectories and special characters should be avoided in filenames.
\item
The command |\childdocmain{|\textit{main}|}| must be followed by a whitespace.
It should not be followed immediately by another command
or by a comment mark `|%|'.
This is because the \TeX{} parser reads the token immediately following
the argument of |\childdocmain| and puts it
at the beginning of every child section;
however, a white\-space is ignored.
\end{itemize}

%%%%%%%%%%%%%%%%%%%%%%%%%%%%%%%%%%%%%%%%
\paragraph{Content of Main File.}

It is advisable to place all content in the child files included by |\include|.
Any output contained in the main file will appear in all child documents
unless suppressed manually;
it cannot be suppressed automatically by the |\includeonly| directive
and thus should normally be avoided.
A method to include some content in the main file
by means of conditional processing is described in \secref{sec:conditional}.

%%%%%%%%%%%%%%%%%%%%%%%%%%%%%%%%%%%%%%%%
\paragraph{Page Numbering.}

When only a part of the document is compiled,
the appropriate numbering of pages
(as well as other status parameters)
is determined from the |.aux| files.
The latter contain information from previous passes.
However this information needs to propagate through
all intermediate child documents.
Therefore the page numbering in child documents may well
be inconsistent until the complete document is compiled at least once.

A useful (if unconventional) way to always ensure a consistent
page numbering is to restart the numbering in each child document
and denote the pages by `\textit{child}|.|\textit{page}'
where \textit{child} represents the chapter/section number of the child file.
This can be achieved by the command
|\numberwithin{page}{|\textit{child}|}|
of the \textsf{amsmath} package
where \textit{child} can be |chapter| or |section|
depending on the chosen structuring.
Alternatively, one can modify the macro |\thepage| appropriately
and reset the counter |page| at the start of each child file.

%%%%%%%%%%%%%%%%%%%%%%%%%%%%%%%%%%%%%%%%%%%%%%%%%%%%%%%%%%%%%%%%%%%%%%%%%%%%%%%%
\subsection{Conditional Processing}
\label{sec:conditional}

The package provides a mechanism to compile different versions
of a document. To customise the versions further some conditional processing
can come in handy to distinguish which version is being compiled.
The package provides two macros to describe the compilation context:

%%%%%%%%%%%%%%%%%%%%%%%%%%%%%%%%%%%%%%%%
\DescribeMacro{\ifchilddoc}
The conditional |\ifchilddoc| distinguishes between the compilation of
child documents and the main document:
%
\begin{center}
|\ifchilddoc |\textit{child-code}| |[|\||else |\textit{main-code}]| \||fi|
\end{center}

%%%%%%%%%%%%%%%%%%%%%%%%%%%%%%%%%%%%%%%%
\DescribeMacro{\childdocname}
\DescribeMacro{\childdocjob}
The macro |\childdocname| contains the filename (without extension)
of the main or child file being processed.
Note that |\childdocjob| will always contain the name of the main file.

%%%%%%%%%%%%%%%%%%%%%%%%%%%%%%%%%%%%%%%%
\paragraph{Title Page.}

Conditional processing can be used to include a title or banner page
in the main document when proper precautions are taken.
Importantly, the code in the main file should ensure that the page counter
(as well as other status parameters which are stored in the |.aux| files)
takes the same value after the conditional processing.
Otherwise the page numbers may take divergent values
depending on which part is compiled.

For example, a title page could be declared by:
%
\begin{center}
\begin{tabular}{l}
|\ifchilddoc\||else|\\
|\addtocounter{page}{-1}|\\
\textit{code for title page}\\
|\newpage|\\
|\||fi|
\end{tabular}
\end{center}
%
A banner page for the child documents can be generated by:
%
\begin{center}
\begin{tabular}{l}
|\ifchilddoc|\\
|\addtocounter{page}{-1}|\\
\textit{code for banner page}\\
|\newpage|\\
|\||fi|
\end{tabular}
\end{center}
%
Here one could write a message such as:
\begin{center}
|This is the part \childdocname{} of \childdocjob{}.|
\end{center}

%%%%%%%%%%%%%%%%%%%%%%%%%%%%%%%%%%%%%%%%%%%%%%%%%%%%%%%%%%%%%%%%%%%%%%%%%%%%%%%%
\subsection{Flags}
\label{sec:flags}

The package makes it easy to generate different versions
of the main or child documents.
To this end compilation flags can be defined
and assigned different default values.
They will be particularly useful in conjunction
with the forwarding mechanism described in \secref{sec:forward}.

For example, it may be useful to have a flag |\version|
which can be set to |draft| or |final|.
The document source will contain some conditional code
depending on the value of |\version|.
Suppose further, the flag should default to |final| for the main file
and to |draft| for child files
which is a natural assignment for editing the document.
This is achieved by placing the following code
in the preamble of the main document
(below the |\childdocmain| directive):
%
\begin{center}
\begin{tabular}{l}
|\ifchilddoc|\\
|\providecommand{\version}{draft}|\\
|\||else|\\
|\providecommand{\version}{final}|\\
|\||fi|
\end{tabular}
\end{center}
%
The definition by |\providecommand| makes sure
that previous definitions are not overwritten.
Further statements |\providecommand{\version}{...}|
can thus be added before the above code to override it.

For the main file, one might add a line
(between |\childdocmain| and the above block)
%
\begin{center}
|%\ifchilddoc\||else\providecommand{\version}{draft}\||fi|
\end{center}
%
which can be uncommented to produce a draft version.
Likewise one can add a line to the very top of a child file
(above the |\childdocof{|\textit{main}|}| directive)
%
\begin{center}
|%\providecommand{\version}{final}|
\end{center}
%
which can be uncommented to produce the final version of this child document.

%%%%%%%%%%%%%%%%%%%%%%%%%%%%%%%%%%%%%%%%%%%%%%%%%%%%%%%%%%%%%%%%%%%%%%%%%%%%%%%%
\subsection{Forwarding}
\label{sec:forward}

Different versions of the main or child documents
using compilation flags as described in \secref{sec:flags}
can be (permanently) stored in different files
for convenient compilation, viewing and distribution.
To this end, the package defines a command
to pass on compilation to a different file:

%%%%%%%%%%%%%%%%%%%%%%%%%%%%%%%%%%%%%%%%
\DescribeMacro{\childdocforward}
The command |\childdocforward| redirects processing to
another source file:
%
\begin{center}
\begin{tabular}{l}
|\input{childdoc.def}|\\
|\childdocforward[|\textit{main}|]{|\textit{dest}|}|\\
\end{tabular}
\end{center}
%
The argument \textit{dest} is the destination file
(without extension).
It should be the main file or one of the child files.
Note that further \textsf{childdoc} directives
such as |\childdocof| and |\childdocforward|
in the indicated file will be processed in this form.
The optional argument \textit{main}
passes on directly to the main file \textit{main}
while pretending to compile the child \textit{dest}.
This form behaves as if \textit{dest}
issues |\childdocof{|\textit{main}|}| right away,
and no further \textsf{childdoc} directives will be processed.

%%%%%%%%%%%%%%%%%%%%%%%%%%%%%%%%%%%%%%%%
\DescribeMacro{\...prefix}
In the alternative form |\childdocforwardprefix|,
%
\begin{center}
\begin{tabular}{l}
|\input{childdoc.def}|\\
|\childdocforwardprefix[|\textit{main}|]{|\textit{prefix}|}{|\textit{dest}|}|
\end{tabular}
\end{center}
%
the destination file is determined by a pattern
depending on the current file:
To make this work, the current file must be called
`{\textit{prefix}\hspace{0.2em}\textit{suffix}}'
with \textit{prefix} matching precisely the argument.
Processing is then passed on to the file
`{\textit{dest}\hspace{0.2em}\textit{suffix}}'.
Surely, the same effect is achieved by
directly specifying the
argument `{\textit{dest}\hspace{0.2em}\textit{suffix}}'
in the first form.
However, that requires to set up a different file
for each child. With the alternative form of the command
all these files can have exactly the same content
which simplifies setting them up and maintaining them.

For example, the following file |draft.tex|
with a compilation flag |\version| as described in \secref{sec:flags}
compiles the main document as a draft:
%
\begin{center}
\begin{tabular}{l}
|\def\version{draft}|\\
|\input{childdoc.def}|\\
|\childdocforward{|\textit{main}|}|
\end{tabular}
\end{center}
%
Likewise, the following files |final|\textit{nn}|.tex|
compile the final version of the child document
|child|\textit{nn}|.tex|:
%
\begin{center}
\begin{tabular}{l}
|\def\version{final}|\\
|\input{childdoc.def}|\\
|\childdocforwardprefix{final}{child}|
\end{tabular}
\end{center}
%

Note that when several versions of a main file and/or of each child file
are to be generated, it may be convenient to set up a |Makefile| or
shell script to automatise the process.

%%%%%%%%%%%%%%%%%%%%%%%%%%%%%%%%%%%%%%%%%%%%%%%%%%%%%%%%%%%%%%%%%%%%%%%%%%%%%%%%
\subsection{Command Line Processing}
\label{sec:commandline}

The effect of redirection files can also be achieved by invoking
the \LaTeX{} compiler with a more elaborate command line.
Most conveniently this should be done as part
of a shell script or a |Makefile|.

When using \textsf{childdoc} in the main file, the following
command lines effectively perform a redirection
(note that depending on the shell being used,
backslashes may have to be doubled: `|\|' $\to$ `|\\|'):
%
\begin{center}
|... -jobname "|\textit{target}|" |\\|"|[\textit{flags}]%
|\input{childdoc.def}\childdocforward[|\textit{main}|]{|\textit{dest}|}"|
\end{center}
%
Here \textit{target} is the name of the output file,
\textit{main} is the name of the main file
and \textit{dest} is the name of the main or child file to be processed
(all filenames without extensions).
The optional argument \textit{main} can be omitted
if \textit{main} matches \textit{dest}.
Optionally, compilation \textit{flags} can be defined via |\def| commands.
This command line makes the \TeX{} engine believe
it is compiling the file \textit{target}
whose content is specified as the latter parameter.
The provided code then forwards the processing to
\textit{main} or \textit{dest} as described in \secref{sec:forward}.

%%%%%%%%%%%%%%%%%%%%%%%%%%%%%%%%%%%%%%%%%%%%%%%%%%%%%%%%%%%%%%%%%%%%%%%%%%%%%%%%
\subsection{Include by Input}
\label{sec:input}

Including child documents by |\include| has some restrictions by design.
Most notably, the content of a child document always occupies
its own set of pages; pages cannot be shared between child documents.
Usually, this behaviour makes perfect sense
because each child document contain an essential part of the document.
However, in some situations it may be desirable to compose
a document from a collection of parts
without having mandatory page breaks between then.
For this case, the package
provides a mechanism to include parts
by |\input| which can also be processed individually.
However, by construction this mechanism
requires manual handling of the content to be output.

%%%%%%%%%%%%%%%%%%%%%%%%%%%%%%%%%%%%%%%%
\DescribeMacro{\ifchilddocmanual}
The main file should be prepared as usual, see \secref{sec:include}.
However, the document body must make a distinction
between processing of an individual part and of the main document, e.g.:
%
\begin{center}
\begin{tabular}{l}
|\ifchilddocmanual|\\
|\input{\childdocname}|\\
|\||else|\\
\textit{document body with }|\input{|\textit{part}|}|\\
|\||fi|
\end{tabular}
\end{center}
%
The conditional |\ifchilddocmanual| is true whenever
a part to be included by |\input| is being compiled,
and the name of the part is stored in |\childdocname|.

%%%%%%%%%%%%%%%%%%%%%%%%%%%%%%%%%%%%%%%%
\DescribeMacro{\childdocby}
Each part to be included by |\input| should start with:
%
\begin{center}
\begin{tabular}{l}
|\input{childdoc.def}|\\
|\childdocby{|\textit{main}|}|\\
\end{tabular}
\end{center}
%
The directive |\childdocby| is similar to |\childdocof|
described in \secref{sec:include},
but the subsequent selection of content must be done manually.
To that end, both |\ifchilddoc| and |\ifchilddocmanual|
will be true upon processing of a part,
and the name of the part is stored in |\childdocname|.
Note that |\jobname| will be set to the filename of the current part
so that each part receives an individual |.aux| file
that does not interfere with the |.aux| file(s) of the main document.
This behaviour can be altered by the alternative form
|\childdocby[*]{|\textit{main}|}| (with a non-empty optional argument)
which uses the |.aux| file of the main document
by setting |\jobname| to \textit{main}.

%%%%%%%%%%%%%%%%%%%%%%%%%%%%%%%%%%%%%%%%%%%%%%%%%%%%%%%%%%%%%%%%%%%%%%%%%%%%%%%%
\subsection{Driver Development}
\label{sec:driver}

The \textsf{childdoc} mechanism can also be use for the development
of definition files such as \LaTeX{} styles or classes.
This case differs from the above setup with multiple parts
included by |\include| in that no |\includeonly| should be invoked.
This can be achieved by starting the include file
(before |\ProvidesPackage|) with:
%
\begin{center}
\begin{tabular}{l}
|\input{childdoc.def}|\\
|\childdocforward{|\textit{main}|}|\\
\end{tabular}
\end{center}
%
or alternatively with:
%
\begin{center}
\begin{tabular}{l}
|\input{childdoc.def}|\\
|\childdocby{|\textit{main}|}|\\
\end{tabular}
\end{center}
%
Both forms have slightly different effects as described above.
The main file is prepared as usual, see \secref{sec:include}.

%%%%%%%%%%%%%%%%%%%%%%%%%%%%%%%%%%%%%%%%%%%%%%%%%%%%%%%%%%%%%%%%%%%%%%%%%%%%%%%%
\subsection{Legacy Detection}
\label{sec:detection}

The directive |\childdocmain| in the main file can detect
whether the complete document or merely a child is to be compiled
even without using the directive |\childdocof|.
This method is deprecated because it is less robust
and there is no compelling reason to use it;
it is merely provided for backward compatibility
and it may be removed in future versions.

If the detection mechanism is to be used,
it is mandatory to correctly specify
the filename of the main file as the argument of |\childdocmain|:
%
\begin{center}
\begin{tabular}{l}
|\input{childdoc.def}|\\
|\childdocmain{|\textit{main}|}|\\
\end{tabular}
\end{center}
%
If |\jobname| does not match the argument \textit{main} of |\childdocmain|,
it is assumed that |\jobname| points to the child file to be compiled.
When using |\childdocmain| with the main file specified as argument,
it suffices to start a child file
with just |\input{|\textit{main}|}|
without loading of the package and using |\childdocof|.
If instead all processing is done
with the appropriate \textsf{childdoc} directives,
the argument of \textit{main} of |\childdocmain| can be empty.

An alternative version of the command line processing described
in \secref{sec:commandline} using the detection mechanism reads:
%
\begin{center}
|... -jobname "|\textit{target}|" "|[\textit{flags}]%
[|\def\jobname{|\textit{dest}|}|]|\input{|\textit{main}|}"|
\end{center}

%%%%%%%%%%%%%%%%%%%%%%%%%%%%%%%%%%%%%%%%%%%%%%%%%%%%%%%%%%%%%%%%%%%%%%%%%%%%%%%%
\subsection{Manual Code}
\label{sec:manual}

In case one cannot be certain whether the definitions file |childdoc.def|
is installed on the target \TeX{} distribution
and one prefers not to ship it,
it is conceivable to paste a few relevant commands into the sources.

To that end, drop all statements |\input{childdoc.def}|
and perform the replacements as outlined below.
Instead of |\childdocmain{|\textit{main}|}| add the following code
to the top of the main file:
%
\begin{center}
\begin{tabular}{l}
|\||ifdefined\childdocname\endinput\||fi\newif\ifchilddoc|\\
|\edef\childdocname{\scantokens\expandafter{\jobname\noexpand}}|\\
|\def\childdocmain{|\textit{main}|}\||ifx\childdocmain\childdocname\||else|\\
|\childdoctrue\includeonly{\childdocname}\let\jobname\childdocmain\||fi|\\
\end{tabular}
\end{center}
%
Instead of |\childdocof{|\textit{main}|}| just include the main file
at the top of each child file:
%
\begin{center}
|\input{|\textit{main}|}|
\end{center}
%
A simple redirection |\childdocforward{|\textit{dest}|}| is achieved by:
%
\begin{center}
|\def\jobname{|\textit{dest}|}\input{\jobname}|
\end{center}
%
The redirection with prefix
|\childdocforwardprefix[|\textit{prefix}|]{|\textit{dest}|}|
is accomplished by:
%
\begin{center}
\begin{tabular}{l}
|{\edef\jobname{\scantokens\expandafter{\jobname\noexpand}}|\\
|\def\redirectjob |\textit{prefix}|#1~~~{\gdef\jobname{|\textit{dest}|#1}}|\\
|\expandafter\redirectjob\jobname~~~}\input{\jobname}|
\end{tabular}
\end{center}

In an alternative approach,
child documents can be compiled by a specific command line
without additional code or specific definitions:
%
\begin{center}
|... -jobname "|\textit{target}|" "|[\textit{flags}]%
|\includeonly{|\textit{dest}|}\input{|\textit{main}|}"|
\end{center}
%

%%%%%%%%%%%%%%%%%%%%%%%%%%%%%%%%%%%%%%%%%%%%%%%%%%%%%%%%%%%%%%%%%%%%%%%%%%%%%%%%
%%%%%%%%%%%%%%%%%%%%%%%%%%%%%%%%%%%%%%%%%%%%%%%%%%%%%%%%%%%%%%%%%%%%%%%%%%%%%%%%
\section{Information}

%%%%%%%%%%%%%%%%%%%%%%%%%%%%%%%%%%%%%%%%%%%%%%%%%%%%%%%%%%%%%%%%%%%%%%%%%%%%%%%%
\subsection{Copyright}

Copyright \copyright{} 2017--2018 Niklas Beisert

This work may be distributed and/or modified under the
conditions of the \LaTeX{} Project Public License, either version 1.3
of this license or (at your option) any later version.
The latest version of this license is in
  \url{http://www.latex-project.org/lppl.txt}
and version 1.3 or later is part of all distributions of \LaTeX{}
version 2005/12/01 or later.

This work has the LPPL maintenance status `maintained'.

The Current Maintainer of this work is Niklas Beisert.

This work consists of the files |README.txt|, |childdoc.ins| and |childdoc.dtx|
as well as the derived files |childdoc.def|, |cdocsamp.tex|
with |cdocsch1.tex|, |cdocsch2.tex|, |cdocspt3.tex|, |cdocspt4.tex|,
|cdocsdrf.tex|, |cdocsfn1.tex|, |cdocsfn2.tex|
as well as |childdoc.pdf|.

%%%%%%%%%%%%%%%%%%%%%%%%%%%%%%%%%%%%%%%%%%%%%%%%%%%%%%%%%%%%%%%%%%%%%%%%%%%%%%%%
\subsection{Files and Installation}

The package consists of the files:
%
\begin{center}
\begin{tabular}{ll}
    |README.txt|   & readme file \\
    |childdoc.ins| & installation file \\
    |childdoc.dtx| & source file \\
    |childdoc.def| & definition file \\
    |cdocsamp.tex| & sample main file \\
    |cdocsch1.tex| & sample include file \\
    |cdocsch2.tex| & sample include file \\
    |cdocspt3.tex| & sample part file \\
    |cdocspt4.tex| & sample part file \\
    |cdocsdrf.tex| & sample redirection file \\
    |cdocsfn1.tex| & sample redirection file \\
    |cdocsfn2.tex| & sample redirection file \\
    |childdoc.pdf| & manual
\end{tabular}
\end{center}
%
The distribution consists of the files
|README.txt|, |childdoc.ins| and |childdoc.dtx|.
%
\begin{itemize}
\item
Run (pdf)\LaTeX{} on |childdoc.dtx|
to compile the manual |childdoc.pdf| (this file).
\item
Run \LaTeX{} on |childdoc.ins| to create the definitions file |childdoc.def|
and the sample |cdocsamp.tex| with include files
|cdocsch1.tex|, |cdocsch2.tex|, |cdocspt3.tex|, |cdocspt4.tex|,
|cdocsdrf.tex|, |cdocsfn1.tex|, |cdocsfn2.tex|.
Then copy the file |childdoc.def| to an appropriate directory of your \LaTeX{}
distribution, e.g.\ \textit{texmf-root}|/tex/latex/childdoc|.
\end{itemize}

%%%%%%%%%%%%%%%%%%%%%%%%%%%%%%%%%%%%%%%%%%%%%%%%%%%%%%%%%%%%%%%%%%%%%%%%%%%%%%%%
\subsection{Related CTAN Packages}

There are several other packages which offer a similar functionality:
%
\begin{itemize}
\item
The packages
\href{http://ctan.org/pkg/docmute}{\textsf{docmute}},
\href{http://ctan.org/pkg/includex}{\textsf{includex}} and
\href{http://ctan.org/pkg/standalone}{\textsf{standalone}}
provide commands to include only the document body of
a child file thus allowing both files to be compiled individually.
\item
The packages \href{http://ctan.org/pkg/subdocs}{\textsf{subdocs}}
and \href{http://ctan.org/pkg/subfiles}{\textsf{subfiles}}
provide structures in which the main and child documents can be
encapsulated and allowing them to be compiled individually.
The inclusion mechanism is different from the conventional |\include|.
\item
The package \href{http://ctan.org/pkg/combine}{\textsf{combine}}
is an elaborate solution to combine several documents into one.
\end{itemize}
%
See also the CTAN topic \href{http://ctan.org/topic/subdocs}{\textsf{subdocs}}
for further related packages.
The present package differs from the above solutions in that
a document structure constructed with the conventional |\include| mechanism
just needs two extra commands at the top of every file
such that all constituent files can be compiled individually.

%%%%%%%%%%%%%%%%%%%%%%%%%%%%%%%%%%%%%%%%%%%%%%%%%%%%%%%%%%%%%%%%%%%%%%%%%%%%%%%%
%\subsection{Feature Suggestions}
%
%The following is a list of features which may be useful for future
%versions of this package:
%%
%\begin{itemize}
%\item
%\ldots
%\end{itemize}

%%%%%%%%%%%%%%%%%%%%%%%%%%%%%%%%%%%%%%%%%%%%%%%%%%%%%%%%%%%%%%%%%%%%%%%%%%%%%%%%
\subsection{Revision History}

%%%%%%%%%%%%%%%%%%%%%%%%%%%%%%%%%%%%%%%%
\paragraph{v2.0:} 2018/12/30

\begin{itemize}
\item
immediate forward processing
\item
added |\childdocby| mechanism
\item
manual restructured
\end{itemize}

%%%%%%%%%%%%%%%%%%%%%%%%%%%%%%%%%%%%%%%%
\paragraph{v1.6:} 2018/01/17

\begin{itemize}
\item
application for development of include files
\item
corrections to manual
\end{itemize}

%%%%%%%%%%%%%%%%%%%%%%%%%%%%%%%%%%%%%%%%
\paragraph{v1.5:} 2017/05/21

\begin{itemize}
\item
more complete structuring introduced
\item
|\childdocof| introduced
\item
|\childdoc| renamed to |\childdocmain|
\item
|\childredirect| renamed to |\childdocforward| and |\childdocforwardprefix|
and functionality expanded
\end{itemize}

%%%%%%%%%%%%%%%%%%%%%%%%%%%%%%%%%%%%%%%%
\paragraph{v1.0:} 2017/04/27

\begin{itemize}
\item
manual and install package
\item
first version published on CTAN
\end{itemize}

%%%%%%%%%%%%%%%%%%%%%%%%%%%%%%%%%%%%%%%%
\paragraph{v0.6:} 2017/04/26

\begin{itemize}
\item
redirection mechanism added
\end{itemize}

%%%%%%%%%%%%%%%%%%%%%%%%%%%%%%%%%%%%%%%%
\paragraph{v0.5:} 2017/04/26

\begin{itemize}
\item
functionality in definition file
\end{itemize}


%%%%%%%%%%%%%%%%%%%%%%%%%%%%%%%%%%%%%%%%%%%%%%%%%%%%%%%%%%%%%%%%%%%%%%%%%%%%%%%%
%%%%%%%%%%%%%%%%%%%%%%%%%%%%%%%%%%%%%%%%%%%%%%%%%%%%%%%%%%%%%%%%%%%%%%%%%%%%%%%%
%%%%%%%%%%%%%%%%%%%%%%%%%%%%%%%%%%%%%%%%%%%%%%%%%%%%%%%%%%%%%%%%%%%%%%%%%%%%%%%%
\appendix

\settowidth\MacroIndent{\rmfamily\scriptsize 000\ }

 \DocInput{childdoc.dtx}

\end{document}
%</driver>
% \fi
%
% %%%%%%%%%%%%%%%%%%%%%%%%%%%%%%%%%%%%%%%%%%%%%%%%%%%%%%%%%%%%%%%%%%%%%%%%%%%%%%
% %%%%%%%%%%%%%%%%%%%%%%%%%%%%%%%%%%%%%%%%%%%%%%%%%%%%%%%%%%%%%%%%%%%%%%%%%%%%%%
% \section{Sample}
%\iffalse
%<*samplemain>
%\fi
%
% The following presents a sample document
% with two chapters, two parts, a title page,
% a compile flag as well as three forwarding files to set the flag.
% It consists of eight |.tex| files:
% \begin{center}
% \begin{tabular}{ll}
% |cdocsamp.tex|&main file\\
% |cdocsch1.tex|&include file for chapter 1\\
% |cdocsch2.tex|&include file for chapter 2\\
% |cdocspt3.tex|&include file for part 3\\
% |cdocspt4.tex|&include file for part 4\\
% |cdocsdrf.tex|&forwarding file for main file in draft mode\\
% |cdocsfi1.tex|&forwarding file for final version of chapter 1\\
% |cdocsfi2.tex|&forwarding file for final version of chapter 2\\
% \end{tabular}
% \end{center}
% Each of the eight files can be compiled directly by the \LaTeX{} compiler.
%
% %%%%%%%%%%%%%%%%%%%%%%%%%%%%%%%%%%%%%%
% \paragraph{Main File.}
%
% The main file is called |cdocsamp.tex|.
%
% Load the \textsf{childdoc} definitions and
% declare the filename for the main document:
%    \begin{macrocode}
\input{childdoc.def}
\childdocmain{}
%    \end{macrocode}

% Optional override for |\version| flag:
%    \begin{macrocode}
%%\ifchilddoc\else\providecommand{\version}{draft}\fi
%    \end{macrocode}

% Define the default values for the |\version| flag
% (|final| for the main file and |draft| for childs):
%    \begin{macrocode}
\ifchilddoc
\providecommand{\version}{draft}
\else
\providecommand{\version}{final}
\fi
%    \end{macrocode}

% Load the standard document class:
%    \begin{macrocode}
\documentclass[12pt]{article}
%    \end{macrocode}

% Start the document body:
%    \begin{macrocode}
\begin{document}
%    \end{macrocode}

% Declare a title page.
% Print title, part of document being processed and version flag:
%    \begin{macrocode}
\addtocounter{page}{-1}
\begin{center}
{\LARGE\bfseries{}childdoc example\par}
\vspace{1cm}
\ifchilddoc
\ifchilddocmanual part\else chapter\fi:
`\childdocname' of `\childdocjob'\par
\else
main document: `\childdocjob'\par
\fi
version: \version\par
\end{center}
\newpage
%    \end{macrocode}

% Manually include selected file,
% otherwise process as usual:
%    \begin{macrocode}
\ifchilddocmanual
\section*{part `\childdocname'}
\input{\childdocname}
\else
%    \end{macrocode}

% Include the two chapters:
%    \begin{macrocode}
\include{cdocsch1}
\include{cdocsch2}
%    \end{macrocode}

% Include the two parts unless only chapters should be displayed:
%    \begin{macrocode}
\ifchilddoc\else
\section{part three}
\input{cdocspt3}
\section{part four}
\input{cdocspt4}
\fi
%    \end{macrocode}

% Process as usual until here:
%    \begin{macrocode}
\fi
%    \end{macrocode}

% End of document body:
%    \begin{macrocode}
\end{document}
%    \end{macrocode}
%\iffalse
%</samplemain>
%\fi
%
% %%%%%%%%%%%%%%%%%%%%%%%%%%%%%%%%%%%%%%
% \paragraph{Chapter Include Files.}
%
% The include files are called |cdocsch1.tex| and |cdocsch2.tex|.
%
%\iffalse
%<*samplechap1|samplechap2>
%\fi

% Optional override for |\version| flag:
%    \begin{macrocode}
%%\providecommand{\version}{final}
%    \end{macrocode}

% Include the main document:
%    \begin{macrocode}
\input{childdoc.def}
\childdocof{cdocsamp}
%    \end{macrocode}

%\iffalse
%</samplechap1|samplechap2>
%\fi
%
%\iffalse
%<*samplechap1>
%\fi
% Some text for chapter 1:
%    \begin{macrocode}
\section{one}
some text in chapter one
%    \end{macrocode}

%\iffalse
%</samplechap1>
%\fi
% Some text for chapter 2:
%\iffalse
%<*samplechap2>
%\fi
%    \begin{macrocode}
\section{two}
more text in chapter two
%    \end{macrocode}

%\iffalse
%</samplechap2>
%\fi
%
% %%%%%%%%%%%%%%%%%%%%%%%%%%%%%%%%%%%%%%
% \paragraph{Part Include Files.}
%
% The include files are called |cdocspt3.tex| and |cdocspt4.tex|.
%
%\iffalse
%<*samplepart3|samplepart4>
%\fi

% Optional override for |\version| flag:
%    \begin{macrocode}
%%\providecommand{\version}{final}
%    \end{macrocode}

% Include the main document:
%    \begin{macrocode}
\input{childdoc.def}
\childdocby{cdocsamp}
%    \end{macrocode}

%\iffalse
%</samplepart3|samplepart4>
%\fi
%
%\iffalse
%<*samplepart3>
%\fi
% Some text for part 3:
%    \begin{macrocode}
some text in part three
%    \end{macrocode}

%\iffalse
%</samplepart3>
%\fi
% Some text for part 4:
%\iffalse
%<*samplepart4>
%\fi
%    \begin{macrocode}
more text in part four
%    \end{macrocode}

%\iffalse
%</samplepart4>
%\fi
%
% %%%%%%%%%%%%%%%%%%%%%%%%%%%%%%%%%%%%%%
% \paragraph{Forwarding for a Complete Draft.}
%
% The following forwarding file |cdocsdrf.tex|
% compiles the main document in draft mode:
%\iffalse
%<*sampledraft>
%\fi
%    \begin{macrocode}
\def\version{draft}
\input{childdoc.def}
\childdocforward{cdocsamp}
%    \end{macrocode}

%\iffalse
%</sampledraft>
%\fi
%
% %%%%%%%%%%%%%%%%%%%%%%%%%%%%%%%%%%%%%%
% \paragraph{Forwarding for Final Version of the Chapters.}
%
% The following forwarding files |cdocsfn1.tex| and |cdocsfn2.tex|
% (with identical content)
% compile the final versions of the child documents
% |cdocsch1.tex| and |cdocsch2.tex|, respectively:
%\iffalse
%<*samplefinal>
%\fi
%    \begin{macrocode}
\def\version{final}
\input{childdoc.def}
\childdocforwardprefix[cdocsamp]{cdocsfn}{cdocsch}
%    \end{macrocode}

%\iffalse
%</samplefinal>
%\fi
%
% %%%%%%%%%%%%%%%%%%%%%%%%%%%%%%%%%%%%%%
% \paragraph{Command Line Processing.}
%
% The following three command lines generate the output files
% |cdocscld|, |cdocscl1| and |cdocscl2|
% which should be identical to
% |cdocsdrf|, |cdocsch1| and |cdocsfn2|, respectively:
% \begin{center}
% \begin{tabular}{l}
% |latex -jobname cdocscld \|\\
% |  "\def\version{draft}\input{childdoc.def}\childdocforward{cdocsamp}"|\\
% |latex -jobname cdocscl1 \|\\
% |  "\input{childdoc.def}\childdocforward[cdocsamp]{cdocsch1}"|\\
% |latex -jobname cdocscl2 \|\\
% |  "\def\version{final}\input{childdoc.def}\childdocforward{cdocsch2}"|
% \end{tabular}
% \end{center}
% Note that the trailing backslash on each first line
% merely continues the input to the second line
% (for convenient cut ant paste).
% Furthermore, the command |latex| can be replaced by any
% of its alternative versions such as |pdflatex|.
%
% %%%%%%%%%%%%%%%%%%%%%%%%%%%%%%%%%%%%%%%%%%%%%%%%%%%%%%%%%%%%%%%%%%%%%%%%%%%%%%
% %%%%%%%%%%%%%%%%%%%%%%%%%%%%%%%%%%%%%%%%%%%%%%%%%%%%%%%%%%%%%%%%%%%%%%%%%%%%%%
% \section{Implementation}
%\iffalse
%<*package>
%\fi
%
% This section describes the definitions file |childdoc.def|.

% The definitions cannot be loaded using |\usepackage| or |\RequirePackage|
% which has a mechanism to prevent loading a style file more than once.
% When loading the definitions by means of |\input|
% multiple instances have to be prevented manually:
%\iffalse
%This code needs to be before the `\ProvidesFile' directive
%which is defined at the beginning of this file.
%Therefore it is also placed there and commented out here.
%</package>
%<*discard>
%\fi
%    \begin{macrocode}
\ifdefined\childdocmain\endinput\fi
%    \end{macrocode}
%\iffalse
%</discard>
%<*package>
%\fi
%
% \macro{\ifchilddoc}
% \macro{\ifchilddocmanual}
% The conditional |\ifchilddoc| tells whether a
% child (true) or main (false) document is being compiled.
% The conditional |\ifchilddocmanual| tells whether
% the |\includeonly| mechanism is used (false) or
% the selection of child files must be performed manually (true).
% The definitions initialise to false:
%    \begin{macrocode}
\newif\ifchilddoc
\newif\ifchilddocmanual
%    \end{macrocode}

% \macro{\childdocname}
% \macro{\childdocjob}
% The macro |\childdocname| stores the name of the main document
% to be compiled. The macro |\childdocjob| stores the name of
% the document on which the \LaTeX{} compiler was originally invoked.
% The content of |\jobname| cannot be compared
% to filenames specified in the source due to different catcodes.
% The following code rescans |\jobname|, stores the result
% in |\childdocname| and saves a copy in |\childdocjob|:
%    \begin{macrocode}
\edef\childdocname{\scantokens\expandafter{\jobname\noexpand}}
\let\childdocjob\childdocname
%    \end{macrocode}

% \macro{\childdocdisable}
% The macro |\childdocdisable| prevents the main file
% from being processed more than once.
% At this stage, the main document command |\childdocmain|
% is assumed to be called once again where it should do nothing.
% Any subsequent call to it should prevent
% a secondary processing of the main document
% It overwrites the forwarding commands
% |\childdocof| and |\childdocforward|
% with empty macros to prevent further inclusions of the main document:
%    \begin{macrocode}
\newcommand{\childdocdisable}
{
  \renewcommand{\childdocmain}[1]{\renewcommand{\childdocmain}[1]{\endinput}}
  \renewcommand{\childdocof}[1]{}
  \renewcommand{\childdocby}[2][]{}
  \renewcommand{\childdocforward}[2][]{}
  \renewcommand{\childdocdisable}{}
}
%    \end{macrocode}

% \macro{\childdocmain}
% The macro |\childdocmain| is to be called at the top of the main file
% with nothing or the main filename (without extension) as argument.
% First, it breaks loops.
% If the argument is not empty and does not match |\childdocname|
% (which is set by the first inclusion of |childdoc.def|),
% |\ifchilddoc| is set to true, |\includeonly| is applied to the child file
% and |\jobname| is set to the main file
% (for proper handling of |.aux| files):
%    \begin{macrocode}
\newcommand{\childdocmain}[1]
{
  \childdocdisable\childdocmain{}
  \if?#1?\else
    \begingroup
      \def\childdoctmp{#1}
      \ifx\childdoctmp\childdocname
        \def\childdoctmp{}
      \else
        \def\childdoctmp
        {
          \childdoctrue
          \includeonly{\childdocname}
          \def\childdocjob{#1}
          \def\jobname{#1}
        }
      \fi
      \expandafter
    \endgroup
    \childdoctmp
  \fi
}
%    \end{macrocode}

% \macro{\childdocof}
% The command |\childdocof| redirects
% compilation to the main file |#1|.
%    \begin{macrocode}
\newcommand{\childdocof}[1]
{
  \childdocdisable
  \childdoctrue
  \includeonly{\childdocname}
  \def\jobname{#1}
  \def\childdocjob{#1}
  \input{#1}
}
%    \end{macrocode}

% \macro{\childdocby}
% The command |\childdocby| ....
%    \begin{macrocode}
\newcommand{\childdocby}[2][]
{
  \childdocdisable
  \childdoctrue
  \childdocmanualtrue
  \if?#1?\else
    \def\jobname{#2}
  \fi
  \def\childdocjob{#2}
  \input{#2}
  \endinput
}
%    \end{macrocode}

% \macro{\childdocforward}
% The command |\childdocforward| redirects
% compilation to the main file or
% (if the optional argument is given) a child file.
% Parameters are set as if the main file
% or a child file starting with |\childdocof| was compiled.
% Then compilation is handed over to the main file:
%    \begin{macrocode}
\newcommand{\childdocforward}[2][]
{
  \begingroup
    \if?#1?
      \def\childdoctmp
      {
        \def\childdocname{#2}
        \def\childdocjob{#2}
        \def\jobname{#2}
        \input{#2}
        \endinput
      }
    \else
      \def\childdoctmp
      {
        \childdocdisable
        \def\childdocname{#2}
        \childdoctrue
        \includeonly{#2}
        \def\childdocjob{#1}
        \def\jobname{#1}
        \input{#1}
        \endinput
      }
    \fi
    \expandafter
  \endgroup
  \childdoctmp
}
%    \end{macrocode}

% \macro{\childdocforwardprefix}
% The command |\childdocforwardprefix| redirects
% compilation to the main or a child file by means of a pattern.
% The prefix |#1| in the current filename is replaced by |#2|
% and the suffix of the current filename is kept
% (it is assumed that the filename does not contain the substring `|~~~|'
% which is used as a delimiter).
% Compilation is handed over to the new file by |\childdocforward|:
%    \begin{macrocode}
\newcommand{\childdocforwardprefix}[3][]
{
  \begingroup
    \def\childdocextract #2##1~~~{\def\childdoctmp{\childdocforward[#1]{#3##1}}}
    \expandafter\childdocextract\childdocname~~~
    \expandafter
  \endgroup
  \childdoctmp
}
%    \end{macrocode}

% \macro{\childdoc}
% The deprecated macro |\childdoc| is a legacy version of |\childdocmain|:
%    \begin{macrocode}
\newcommand{\childdoc}{\childdocmain}
%    \end{macrocode}

% \macro{\childdocredirect}
% The deprecated macro |\childdocredirect| is a legacy version
% of |\childdocforward| and |\childdocforwardprefix|:
%    \begin{macrocode}
\newcommand{\childdocredirect}[2][]
{
  \begingroup
    \if?#1?
      \def\childdoctmp{\childdocforward{#2}}
    \else
      \def\childdoctmp{\childdocforwardprefix{#1}{#2}}
    \fi
    \expandafter
  \endgroup
  \childdoctmp
}
%    \end{macrocode}

%\iffalse
%</package>
%\fi
%
\endinput
\childdocforward[|\textit{main}|]{|\textit{dest}|}"|
\end{center}
%
Here \textit{target} is the name of the output file,
\textit{main} is the name of the main file
and \textit{dest} is the name of the main or child file to be processed
(all filenames without extensions).
The optional argument \textit{main} can be omitted
if \textit{main} matches \textit{dest}.
Optionally, compilation \textit{flags} can be defined via |\def| commands.
This command line makes the \TeX{} engine believe
it is compiling the file \textit{target}
whose content is specified as the latter parameter.
The provided code then forwards the processing to
\textit{main} or \textit{dest} as described in \secref{sec:forward}.

%%%%%%%%%%%%%%%%%%%%%%%%%%%%%%%%%%%%%%%%%%%%%%%%%%%%%%%%%%%%%%%%%%%%%%%%%%%%%%%%
\subsection{Include by Input}
\label{sec:input}

Including child documents by |\include| has some restrictions by design.
Most notably, the content of a child document always occupies
its own set of pages; pages cannot be shared between child documents.
Usually, this behaviour makes perfect sense
because each child document contain an essential part of the document.
However, in some situations it may be desirable to compose
a document from a collection of parts
without having mandatory page breaks between then.
For this case, the package
provides a mechanism to include parts
by |\input| which can also be processed individually.
However, by construction this mechanism
requires manual handling of the content to be output.

%%%%%%%%%%%%%%%%%%%%%%%%%%%%%%%%%%%%%%%%
\DescribeMacro{\ifchilddocmanual}
The main file should be prepared as usual, see \secref{sec:include}.
However, the document body must make a distinction
between processing of an individual part and of the main document, e.g.:
%
\begin{center}
\begin{tabular}{l}
|\ifchilddocmanual|\\
|\input{\childdocname}|\\
|\||else|\\
\textit{document body with }|\input{|\textit{part}|}|\\
|\||fi|
\end{tabular}
\end{center}
%
The conditional |\ifchilddocmanual| is true whenever
a part to be included by |\input| is being compiled,
and the name of the part is stored in |\childdocname|.

%%%%%%%%%%%%%%%%%%%%%%%%%%%%%%%%%%%%%%%%
\DescribeMacro{\childdocby}
Each part to be included by |\input| should start with:
%
\begin{center}
\begin{tabular}{l}
|% \iffalse
%
% childdoc.dtx Copyright (C) 2017-2018 Niklas Beisert
%
% This work may be distributed and/or modified under the
% conditions of the LaTeX Project Public License, either version 1.3
% of this license or (at your option) any later version.
% The latest version of this license is in
%   http://www.latex-project.org/lppl.txt
% and version 1.3 or later is part of all distributions of LaTeX
% version 2005/12/01 or later.
%
% This work has the LPPL maintenance status `maintained'.
%
% The Current Maintainer of this work is Niklas Beisert.
%
% This work consists of the files childdoc.dtx and childdoc.ins
% and the derived files childdoc.def and cdocsamp.tex with
% cdocsch1.tex, cdocsch2.tex, cdocsdrf.tex, cdocsfn1.tex, cdocsfn2.tex.
%
%<package>\ifdefined\childdocmain\endinput\fi
%<package>\ProvidesFile{childdoc.def}[2018/12/30 v2.0 child document driver]
%<samplemain>\ProvidesFile{cdocsamp.tex}[2018/12/30 v2.0 sample for childdoc]
%<*driver>
%\ProvidesFile{childdoc.drv}[2018/12/30 v2.0 childdoc reference manual file]
\PassOptionsToClass{10pt,a4paper}{article}
\documentclass{ltxdoc}

\usepackage[margin=35mm]{geometry}
\usepackage{hyperref}
\usepackage{hyperxmp}
\usepackage[usenames]{color}

\hypersetup{colorlinks=true}
\hypersetup{pdfstartview=FitH}
\hypersetup{pdfpagemode=UseNone}
\hypersetup{pdfsource={}}
\hypersetup{pdflang={en-UK}}
\hypersetup{pdfcopyright={Copyright 2017-2018 Niklas Beisert.
  This work may be distributed and/or modified under the
  conditions of the LaTeX Project Public License, either version 1.3
  of this license or (at your option) any later version.}}
\hypersetup{pdflicenseurl={http://www.latex-project.org/lppl.txt}}
\hypersetup{pdfcontactaddress={ETH Zurich, ITP, HIT K,
  Wolfgang-Pauli-Strasse 27}}
\hypersetup{pdfcontactpostcode={8093}}
\hypersetup{pdfcontactcity={Zurich}}
\hypersetup{pdfcontactcountry={Switzerland}}
\hypersetup{pdfcontactemail={nbeisert@itp.phys.ethz.ch}}
\hypersetup{pdfcontacturl={http://people.phys.ethz.ch/\xmptilde nbeisert/}}

\newcommand{\secref}[1]{\hyperref[#1]{section \ref*{#1}}}

\parskip1ex
\parindent0pt
\let\olditemize\itemize
\def\itemize{\olditemize\parskip0pt}

\begin{document}

\title{The \textsf{childdoc} Package}
\hypersetup{pdftitle={The childdoc Package}}
\author{Niklas Beisert\\[2ex]
  Institut f\"ur Theoretische Physik\\
  Eidgen\"ossische Technische Hochschule Z\"urich\\
  Wolfgang-Pauli-Strasse 27, 8093 Z\"urich, Switzerland\\[1ex]
  \href{mailto:nbeisert@itp.phys.ethz.ch}
  {\texttt{nbeisert@itp.phys.ethz.ch}}}
\hypersetup{pdfauthor={Niklas Beisert}}
\hypersetup{pdfsubject={Manual for the LaTeX2e Package childdoc}}
\date{30 December 2018, \textsf{v2.0}}
\maketitle

\begin{abstract}\noindent
\textsf{childdoc} is a \LaTeXe{} package
that enables the direct compilation
of document sections included by |\include|
to individual files.
\end{abstract}

\begingroup
\parskip0ex
\tableofcontents
\endgroup

%%%%%%%%%%%%%%%%%%%%%%%%%%%%%%%%%%%%%%%%%%%%%%%%%%%%%%%%%%%%%%%%%%%%%%%%%%%%%%%%
%%%%%%%%%%%%%%%%%%%%%%%%%%%%%%%%%%%%%%%%%%%%%%%%%%%%%%%%%%%%%%%%%%%%%%%%%%%%%%%%
\section{Introduction}

\LaTeX{} provides a mechanism to structure a large document (such as a book)
into a main file and several child files (containing the chapters)
using the |\include| command.
This mechanism is beneficial for documents
which span hundreds of pages in order to
make the source file(s) more manageable.
Moreover, compilation can be restricted to
selected child files by means of the |\includeonly| command.
The latter feature can be used to reduce the compilation time while editing
(this was significantly more useful in the earlier days of \LaTeX{})
or to generate a smaller document which is easier to navigate.
Another application of |\includeonly| is to generate
documents consisting of selected parts of the complete document.

However, there are a few drawbacks of the plain |\include| mechanism:
\begin{itemize}
\item
The child files cannot be compiled on their own,
they can only be compiled via the main file.
A naive editing environment
(such as a text editor with an option
to have the current file processed by \LaTeX)
may require one to switch to the main file before compiling;
attempting to compile the child file produces errors.
\item
The main file must be modified (each time)
to adjust the |\includeonly| command
to the present needs. This easily leaves the main file in a messy state.
\item
The generated document will always carry the filename
of the main document. This is inconvenient if
several child files are to be compiled and
to be kept for distribution.
\end{itemize}

The present package provides a simple interface
to make child files individually compilable by \LaTeX{}.
Compiling a child file then has the same effect as compiling
the main file with an |\includeonly| command
to select the appropriate child.
Moreover the generated document will carry the name of the child
rather than the main file.
This resolves all three above issues.

This feature is meant to make the editing of books,
thesis documents and lecture notes somewhat more convenient.
However, the package can also be used efficiently for
composing a series of documents (such as exercise sheets)
which are typically distributed individually.
It then assists the author in generating the individual documents
(potentially in different versions)
as well as a document containing the collected series.
Another application is in developing style files
or other kinds of included material
where compilation of the style file could redirect
to a sample or test file.

%%%%%%%%%%%%%%%%%%%%%%%%%%%%%%%%%%%%%%%%%%%%%%%%%%%%%%%%%%%%%%%%%%%%%%%%%%%%%%%%
%%%%%%%%%%%%%%%%%%%%%%%%%%%%%%%%%%%%%%%%%%%%%%%%%%%%%%%%%%%%%%%%%%%%%%%%%%%%%%%%
\section{Usage}

First of all, the package \textsf{childdoc} is \emph{not} a standard
\LaTeXe{} |.sty| style file! Therefore it needs to be invoked in
a non-standard way.

%%%%%%%%%%%%%%%%%%%%%%%%%%%%%%%%%%%%%%%%%%%%%%%%%%%%%%%%%%%%%%%%%%%%%%%%%%%%%%%%
\subsection{Included Files}
\label{sec:include}

%%%%%%%%%%%%%%%%%%%%%%%%%%%%%%%%%%%%%%%%
\DescribeMacro{\childdocmain}
To use the package, add the commands
\begin{center}
\begin{tabular}{l}
|\input{childdoc.def}|\\
|\childdocmain{}|\\
\end{tabular}
\end{center}
at the very top of the main \LaTeX{} file,
in particular \emph{before} the |\documentclass| statement!
The argument of |\childdocmain| should be left empty
(but it must be present).

%%%%%%%%%%%%%%%%%%%%%%%%%%%%%%%%%%%%%%%%
\DescribeMacro{\childdocof}
Furthermore, add the commands
\begin{center}
\begin{tabular}{l}
|\input{childdoc.def}|\\
|\childdocof{|\textit{main}|}|\\
\end{tabular}
\end{center}
at the top of every child file \textit{child}
which is included by |\include{|\textit{child}|}|
from within the main file
(or at least for those files to be compiled individually).
The argument \textit{main} must be the filename of the main file.

There are a couple of
considerations in setting up the main and child documents:

%%%%%%%%%%%%%%%%%%%%%%%%%%%%%%%%%%%%%%%%
\paragraph{Restrictions.}

Please note the following restrictions:
\begin{itemize}
\item
|\childdocmain| must be called with one argument \textit{main}
to ensure compatibility with earlier version of the package.
It must either be empty (|\childdocmain{}|)
or precisely match the filename of the main file in which it is specified.
See \secref{sec:detection} for further information.
\item
The filename \textit{main} must be specified without the |.tex| extension.
\item
The filename \textit{main} is case sensitive
(even in case-insensitive file systems)
due to internal string comparison.
\item
The argument \textit{main} should be fully expanded, it cannot be a macro.
\item
Subdirectories and special characters should be avoided in filenames.
\item
The command |\childdocmain{|\textit{main}|}| must be followed by a whitespace.
It should not be followed immediately by another command
or by a comment mark `|%|'.
This is because the \TeX{} parser reads the token immediately following
the argument of |\childdocmain| and puts it
at the beginning of every child section;
however, a white\-space is ignored.
\end{itemize}

%%%%%%%%%%%%%%%%%%%%%%%%%%%%%%%%%%%%%%%%
\paragraph{Content of Main File.}

It is advisable to place all content in the child files included by |\include|.
Any output contained in the main file will appear in all child documents
unless suppressed manually;
it cannot be suppressed automatically by the |\includeonly| directive
and thus should normally be avoided.
A method to include some content in the main file
by means of conditional processing is described in \secref{sec:conditional}.

%%%%%%%%%%%%%%%%%%%%%%%%%%%%%%%%%%%%%%%%
\paragraph{Page Numbering.}

When only a part of the document is compiled,
the appropriate numbering of pages
(as well as other status parameters)
is determined from the |.aux| files.
The latter contain information from previous passes.
However this information needs to propagate through
all intermediate child documents.
Therefore the page numbering in child documents may well
be inconsistent until the complete document is compiled at least once.

A useful (if unconventional) way to always ensure a consistent
page numbering is to restart the numbering in each child document
and denote the pages by `\textit{child}|.|\textit{page}'
where \textit{child} represents the chapter/section number of the child file.
This can be achieved by the command
|\numberwithin{page}{|\textit{child}|}|
of the \textsf{amsmath} package
where \textit{child} can be |chapter| or |section|
depending on the chosen structuring.
Alternatively, one can modify the macro |\thepage| appropriately
and reset the counter |page| at the start of each child file.

%%%%%%%%%%%%%%%%%%%%%%%%%%%%%%%%%%%%%%%%%%%%%%%%%%%%%%%%%%%%%%%%%%%%%%%%%%%%%%%%
\subsection{Conditional Processing}
\label{sec:conditional}

The package provides a mechanism to compile different versions
of a document. To customise the versions further some conditional processing
can come in handy to distinguish which version is being compiled.
The package provides two macros to describe the compilation context:

%%%%%%%%%%%%%%%%%%%%%%%%%%%%%%%%%%%%%%%%
\DescribeMacro{\ifchilddoc}
The conditional |\ifchilddoc| distinguishes between the compilation of
child documents and the main document:
%
\begin{center}
|\ifchilddoc |\textit{child-code}| |[|\||else |\textit{main-code}]| \||fi|
\end{center}

%%%%%%%%%%%%%%%%%%%%%%%%%%%%%%%%%%%%%%%%
\DescribeMacro{\childdocname}
\DescribeMacro{\childdocjob}
The macro |\childdocname| contains the filename (without extension)
of the main or child file being processed.
Note that |\childdocjob| will always contain the name of the main file.

%%%%%%%%%%%%%%%%%%%%%%%%%%%%%%%%%%%%%%%%
\paragraph{Title Page.}

Conditional processing can be used to include a title or banner page
in the main document when proper precautions are taken.
Importantly, the code in the main file should ensure that the page counter
(as well as other status parameters which are stored in the |.aux| files)
takes the same value after the conditional processing.
Otherwise the page numbers may take divergent values
depending on which part is compiled.

For example, a title page could be declared by:
%
\begin{center}
\begin{tabular}{l}
|\ifchilddoc\||else|\\
|\addtocounter{page}{-1}|\\
\textit{code for title page}\\
|\newpage|\\
|\||fi|
\end{tabular}
\end{center}
%
A banner page for the child documents can be generated by:
%
\begin{center}
\begin{tabular}{l}
|\ifchilddoc|\\
|\addtocounter{page}{-1}|\\
\textit{code for banner page}\\
|\newpage|\\
|\||fi|
\end{tabular}
\end{center}
%
Here one could write a message such as:
\begin{center}
|This is the part \childdocname{} of \childdocjob{}.|
\end{center}

%%%%%%%%%%%%%%%%%%%%%%%%%%%%%%%%%%%%%%%%%%%%%%%%%%%%%%%%%%%%%%%%%%%%%%%%%%%%%%%%
\subsection{Flags}
\label{sec:flags}

The package makes it easy to generate different versions
of the main or child documents.
To this end compilation flags can be defined
and assigned different default values.
They will be particularly useful in conjunction
with the forwarding mechanism described in \secref{sec:forward}.

For example, it may be useful to have a flag |\version|
which can be set to |draft| or |final|.
The document source will contain some conditional code
depending on the value of |\version|.
Suppose further, the flag should default to |final| for the main file
and to |draft| for child files
which is a natural assignment for editing the document.
This is achieved by placing the following code
in the preamble of the main document
(below the |\childdocmain| directive):
%
\begin{center}
\begin{tabular}{l}
|\ifchilddoc|\\
|\providecommand{\version}{draft}|\\
|\||else|\\
|\providecommand{\version}{final}|\\
|\||fi|
\end{tabular}
\end{center}
%
The definition by |\providecommand| makes sure
that previous definitions are not overwritten.
Further statements |\providecommand{\version}{...}|
can thus be added before the above code to override it.

For the main file, one might add a line
(between |\childdocmain| and the above block)
%
\begin{center}
|%\ifchilddoc\||else\providecommand{\version}{draft}\||fi|
\end{center}
%
which can be uncommented to produce a draft version.
Likewise one can add a line to the very top of a child file
(above the |\childdocof{|\textit{main}|}| directive)
%
\begin{center}
|%\providecommand{\version}{final}|
\end{center}
%
which can be uncommented to produce the final version of this child document.

%%%%%%%%%%%%%%%%%%%%%%%%%%%%%%%%%%%%%%%%%%%%%%%%%%%%%%%%%%%%%%%%%%%%%%%%%%%%%%%%
\subsection{Forwarding}
\label{sec:forward}

Different versions of the main or child documents
using compilation flags as described in \secref{sec:flags}
can be (permanently) stored in different files
for convenient compilation, viewing and distribution.
To this end, the package defines a command
to pass on compilation to a different file:

%%%%%%%%%%%%%%%%%%%%%%%%%%%%%%%%%%%%%%%%
\DescribeMacro{\childdocforward}
The command |\childdocforward| redirects processing to
another source file:
%
\begin{center}
\begin{tabular}{l}
|\input{childdoc.def}|\\
|\childdocforward[|\textit{main}|]{|\textit{dest}|}|\\
\end{tabular}
\end{center}
%
The argument \textit{dest} is the destination file
(without extension).
It should be the main file or one of the child files.
Note that further \textsf{childdoc} directives
such as |\childdocof| and |\childdocforward|
in the indicated file will be processed in this form.
The optional argument \textit{main}
passes on directly to the main file \textit{main}
while pretending to compile the child \textit{dest}.
This form behaves as if \textit{dest}
issues |\childdocof{|\textit{main}|}| right away,
and no further \textsf{childdoc} directives will be processed.

%%%%%%%%%%%%%%%%%%%%%%%%%%%%%%%%%%%%%%%%
\DescribeMacro{\...prefix}
In the alternative form |\childdocforwardprefix|,
%
\begin{center}
\begin{tabular}{l}
|\input{childdoc.def}|\\
|\childdocforwardprefix[|\textit{main}|]{|\textit{prefix}|}{|\textit{dest}|}|
\end{tabular}
\end{center}
%
the destination file is determined by a pattern
depending on the current file:
To make this work, the current file must be called
`{\textit{prefix}\hspace{0.2em}\textit{suffix}}'
with \textit{prefix} matching precisely the argument.
Processing is then passed on to the file
`{\textit{dest}\hspace{0.2em}\textit{suffix}}'.
Surely, the same effect is achieved by
directly specifying the
argument `{\textit{dest}\hspace{0.2em}\textit{suffix}}'
in the first form.
However, that requires to set up a different file
for each child. With the alternative form of the command
all these files can have exactly the same content
which simplifies setting them up and maintaining them.

For example, the following file |draft.tex|
with a compilation flag |\version| as described in \secref{sec:flags}
compiles the main document as a draft:
%
\begin{center}
\begin{tabular}{l}
|\def\version{draft}|\\
|\input{childdoc.def}|\\
|\childdocforward{|\textit{main}|}|
\end{tabular}
\end{center}
%
Likewise, the following files |final|\textit{nn}|.tex|
compile the final version of the child document
|child|\textit{nn}|.tex|:
%
\begin{center}
\begin{tabular}{l}
|\def\version{final}|\\
|\input{childdoc.def}|\\
|\childdocforwardprefix{final}{child}|
\end{tabular}
\end{center}
%

Note that when several versions of a main file and/or of each child file
are to be generated, it may be convenient to set up a |Makefile| or
shell script to automatise the process.

%%%%%%%%%%%%%%%%%%%%%%%%%%%%%%%%%%%%%%%%%%%%%%%%%%%%%%%%%%%%%%%%%%%%%%%%%%%%%%%%
\subsection{Command Line Processing}
\label{sec:commandline}

The effect of redirection files can also be achieved by invoking
the \LaTeX{} compiler with a more elaborate command line.
Most conveniently this should be done as part
of a shell script or a |Makefile|.

When using \textsf{childdoc} in the main file, the following
command lines effectively perform a redirection
(note that depending on the shell being used,
backslashes may have to be doubled: `|\|' $\to$ `|\\|'):
%
\begin{center}
|... -jobname "|\textit{target}|" |\\|"|[\textit{flags}]%
|\input{childdoc.def}\childdocforward[|\textit{main}|]{|\textit{dest}|}"|
\end{center}
%
Here \textit{target} is the name of the output file,
\textit{main} is the name of the main file
and \textit{dest} is the name of the main or child file to be processed
(all filenames without extensions).
The optional argument \textit{main} can be omitted
if \textit{main} matches \textit{dest}.
Optionally, compilation \textit{flags} can be defined via |\def| commands.
This command line makes the \TeX{} engine believe
it is compiling the file \textit{target}
whose content is specified as the latter parameter.
The provided code then forwards the processing to
\textit{main} or \textit{dest} as described in \secref{sec:forward}.

%%%%%%%%%%%%%%%%%%%%%%%%%%%%%%%%%%%%%%%%%%%%%%%%%%%%%%%%%%%%%%%%%%%%%%%%%%%%%%%%
\subsection{Include by Input}
\label{sec:input}

Including child documents by |\include| has some restrictions by design.
Most notably, the content of a child document always occupies
its own set of pages; pages cannot be shared between child documents.
Usually, this behaviour makes perfect sense
because each child document contain an essential part of the document.
However, in some situations it may be desirable to compose
a document from a collection of parts
without having mandatory page breaks between then.
For this case, the package
provides a mechanism to include parts
by |\input| which can also be processed individually.
However, by construction this mechanism
requires manual handling of the content to be output.

%%%%%%%%%%%%%%%%%%%%%%%%%%%%%%%%%%%%%%%%
\DescribeMacro{\ifchilddocmanual}
The main file should be prepared as usual, see \secref{sec:include}.
However, the document body must make a distinction
between processing of an individual part and of the main document, e.g.:
%
\begin{center}
\begin{tabular}{l}
|\ifchilddocmanual|\\
|\input{\childdocname}|\\
|\||else|\\
\textit{document body with }|\input{|\textit{part}|}|\\
|\||fi|
\end{tabular}
\end{center}
%
The conditional |\ifchilddocmanual| is true whenever
a part to be included by |\input| is being compiled,
and the name of the part is stored in |\childdocname|.

%%%%%%%%%%%%%%%%%%%%%%%%%%%%%%%%%%%%%%%%
\DescribeMacro{\childdocby}
Each part to be included by |\input| should start with:
%
\begin{center}
\begin{tabular}{l}
|\input{childdoc.def}|\\
|\childdocby{|\textit{main}|}|\\
\end{tabular}
\end{center}
%
The directive |\childdocby| is similar to |\childdocof|
described in \secref{sec:include},
but the subsequent selection of content must be done manually.
To that end, both |\ifchilddoc| and |\ifchilddocmanual|
will be true upon processing of a part,
and the name of the part is stored in |\childdocname|.
Note that |\jobname| will be set to the filename of the current part
so that each part receives an individual |.aux| file
that does not interfere with the |.aux| file(s) of the main document.
This behaviour can be altered by the alternative form
|\childdocby[*]{|\textit{main}|}| (with a non-empty optional argument)
which uses the |.aux| file of the main document
by setting |\jobname| to \textit{main}.

%%%%%%%%%%%%%%%%%%%%%%%%%%%%%%%%%%%%%%%%%%%%%%%%%%%%%%%%%%%%%%%%%%%%%%%%%%%%%%%%
\subsection{Driver Development}
\label{sec:driver}

The \textsf{childdoc} mechanism can also be use for the development
of definition files such as \LaTeX{} styles or classes.
This case differs from the above setup with multiple parts
included by |\include| in that no |\includeonly| should be invoked.
This can be achieved by starting the include file
(before |\ProvidesPackage|) with:
%
\begin{center}
\begin{tabular}{l}
|\input{childdoc.def}|\\
|\childdocforward{|\textit{main}|}|\\
\end{tabular}
\end{center}
%
or alternatively with:
%
\begin{center}
\begin{tabular}{l}
|\input{childdoc.def}|\\
|\childdocby{|\textit{main}|}|\\
\end{tabular}
\end{center}
%
Both forms have slightly different effects as described above.
The main file is prepared as usual, see \secref{sec:include}.

%%%%%%%%%%%%%%%%%%%%%%%%%%%%%%%%%%%%%%%%%%%%%%%%%%%%%%%%%%%%%%%%%%%%%%%%%%%%%%%%
\subsection{Legacy Detection}
\label{sec:detection}

The directive |\childdocmain| in the main file can detect
whether the complete document or merely a child is to be compiled
even without using the directive |\childdocof|.
This method is deprecated because it is less robust
and there is no compelling reason to use it;
it is merely provided for backward compatibility
and it may be removed in future versions.

If the detection mechanism is to be used,
it is mandatory to correctly specify
the filename of the main file as the argument of |\childdocmain|:
%
\begin{center}
\begin{tabular}{l}
|\input{childdoc.def}|\\
|\childdocmain{|\textit{main}|}|\\
\end{tabular}
\end{center}
%
If |\jobname| does not match the argument \textit{main} of |\childdocmain|,
it is assumed that |\jobname| points to the child file to be compiled.
When using |\childdocmain| with the main file specified as argument,
it suffices to start a child file
with just |\input{|\textit{main}|}|
without loading of the package and using |\childdocof|.
If instead all processing is done
with the appropriate \textsf{childdoc} directives,
the argument of \textit{main} of |\childdocmain| can be empty.

An alternative version of the command line processing described
in \secref{sec:commandline} using the detection mechanism reads:
%
\begin{center}
|... -jobname "|\textit{target}|" "|[\textit{flags}]%
[|\def\jobname{|\textit{dest}|}|]|\input{|\textit{main}|}"|
\end{center}

%%%%%%%%%%%%%%%%%%%%%%%%%%%%%%%%%%%%%%%%%%%%%%%%%%%%%%%%%%%%%%%%%%%%%%%%%%%%%%%%
\subsection{Manual Code}
\label{sec:manual}

In case one cannot be certain whether the definitions file |childdoc.def|
is installed on the target \TeX{} distribution
and one prefers not to ship it,
it is conceivable to paste a few relevant commands into the sources.

To that end, drop all statements |\input{childdoc.def}|
and perform the replacements as outlined below.
Instead of |\childdocmain{|\textit{main}|}| add the following code
to the top of the main file:
%
\begin{center}
\begin{tabular}{l}
|\||ifdefined\childdocname\endinput\||fi\newif\ifchilddoc|\\
|\edef\childdocname{\scantokens\expandafter{\jobname\noexpand}}|\\
|\def\childdocmain{|\textit{main}|}\||ifx\childdocmain\childdocname\||else|\\
|\childdoctrue\includeonly{\childdocname}\let\jobname\childdocmain\||fi|\\
\end{tabular}
\end{center}
%
Instead of |\childdocof{|\textit{main}|}| just include the main file
at the top of each child file:
%
\begin{center}
|\input{|\textit{main}|}|
\end{center}
%
A simple redirection |\childdocforward{|\textit{dest}|}| is achieved by:
%
\begin{center}
|\def\jobname{|\textit{dest}|}\input{\jobname}|
\end{center}
%
The redirection with prefix
|\childdocforwardprefix[|\textit{prefix}|]{|\textit{dest}|}|
is accomplished by:
%
\begin{center}
\begin{tabular}{l}
|{\edef\jobname{\scantokens\expandafter{\jobname\noexpand}}|\\
|\def\redirectjob |\textit{prefix}|#1~~~{\gdef\jobname{|\textit{dest}|#1}}|\\
|\expandafter\redirectjob\jobname~~~}\input{\jobname}|
\end{tabular}
\end{center}

In an alternative approach,
child documents can be compiled by a specific command line
without additional code or specific definitions:
%
\begin{center}
|... -jobname "|\textit{target}|" "|[\textit{flags}]%
|\includeonly{|\textit{dest}|}\input{|\textit{main}|}"|
\end{center}
%

%%%%%%%%%%%%%%%%%%%%%%%%%%%%%%%%%%%%%%%%%%%%%%%%%%%%%%%%%%%%%%%%%%%%%%%%%%%%%%%%
%%%%%%%%%%%%%%%%%%%%%%%%%%%%%%%%%%%%%%%%%%%%%%%%%%%%%%%%%%%%%%%%%%%%%%%%%%%%%%%%
\section{Information}

%%%%%%%%%%%%%%%%%%%%%%%%%%%%%%%%%%%%%%%%%%%%%%%%%%%%%%%%%%%%%%%%%%%%%%%%%%%%%%%%
\subsection{Copyright}

Copyright \copyright{} 2017--2018 Niklas Beisert

This work may be distributed and/or modified under the
conditions of the \LaTeX{} Project Public License, either version 1.3
of this license or (at your option) any later version.
The latest version of this license is in
  \url{http://www.latex-project.org/lppl.txt}
and version 1.3 or later is part of all distributions of \LaTeX{}
version 2005/12/01 or later.

This work has the LPPL maintenance status `maintained'.

The Current Maintainer of this work is Niklas Beisert.

This work consists of the files |README.txt|, |childdoc.ins| and |childdoc.dtx|
as well as the derived files |childdoc.def|, |cdocsamp.tex|
with |cdocsch1.tex|, |cdocsch2.tex|, |cdocspt3.tex|, |cdocspt4.tex|,
|cdocsdrf.tex|, |cdocsfn1.tex|, |cdocsfn2.tex|
as well as |childdoc.pdf|.

%%%%%%%%%%%%%%%%%%%%%%%%%%%%%%%%%%%%%%%%%%%%%%%%%%%%%%%%%%%%%%%%%%%%%%%%%%%%%%%%
\subsection{Files and Installation}

The package consists of the files:
%
\begin{center}
\begin{tabular}{ll}
    |README.txt|   & readme file \\
    |childdoc.ins| & installation file \\
    |childdoc.dtx| & source file \\
    |childdoc.def| & definition file \\
    |cdocsamp.tex| & sample main file \\
    |cdocsch1.tex| & sample include file \\
    |cdocsch2.tex| & sample include file \\
    |cdocspt3.tex| & sample part file \\
    |cdocspt4.tex| & sample part file \\
    |cdocsdrf.tex| & sample redirection file \\
    |cdocsfn1.tex| & sample redirection file \\
    |cdocsfn2.tex| & sample redirection file \\
    |childdoc.pdf| & manual
\end{tabular}
\end{center}
%
The distribution consists of the files
|README.txt|, |childdoc.ins| and |childdoc.dtx|.
%
\begin{itemize}
\item
Run (pdf)\LaTeX{} on |childdoc.dtx|
to compile the manual |childdoc.pdf| (this file).
\item
Run \LaTeX{} on |childdoc.ins| to create the definitions file |childdoc.def|
and the sample |cdocsamp.tex| with include files
|cdocsch1.tex|, |cdocsch2.tex|, |cdocspt3.tex|, |cdocspt4.tex|,
|cdocsdrf.tex|, |cdocsfn1.tex|, |cdocsfn2.tex|.
Then copy the file |childdoc.def| to an appropriate directory of your \LaTeX{}
distribution, e.g.\ \textit{texmf-root}|/tex/latex/childdoc|.
\end{itemize}

%%%%%%%%%%%%%%%%%%%%%%%%%%%%%%%%%%%%%%%%%%%%%%%%%%%%%%%%%%%%%%%%%%%%%%%%%%%%%%%%
\subsection{Related CTAN Packages}

There are several other packages which offer a similar functionality:
%
\begin{itemize}
\item
The packages
\href{http://ctan.org/pkg/docmute}{\textsf{docmute}},
\href{http://ctan.org/pkg/includex}{\textsf{includex}} and
\href{http://ctan.org/pkg/standalone}{\textsf{standalone}}
provide commands to include only the document body of
a child file thus allowing both files to be compiled individually.
\item
The packages \href{http://ctan.org/pkg/subdocs}{\textsf{subdocs}}
and \href{http://ctan.org/pkg/subfiles}{\textsf{subfiles}}
provide structures in which the main and child documents can be
encapsulated and allowing them to be compiled individually.
The inclusion mechanism is different from the conventional |\include|.
\item
The package \href{http://ctan.org/pkg/combine}{\textsf{combine}}
is an elaborate solution to combine several documents into one.
\end{itemize}
%
See also the CTAN topic \href{http://ctan.org/topic/subdocs}{\textsf{subdocs}}
for further related packages.
The present package differs from the above solutions in that
a document structure constructed with the conventional |\include| mechanism
just needs two extra commands at the top of every file
such that all constituent files can be compiled individually.

%%%%%%%%%%%%%%%%%%%%%%%%%%%%%%%%%%%%%%%%%%%%%%%%%%%%%%%%%%%%%%%%%%%%%%%%%%%%%%%%
%\subsection{Feature Suggestions}
%
%The following is a list of features which may be useful for future
%versions of this package:
%%
%\begin{itemize}
%\item
%\ldots
%\end{itemize}

%%%%%%%%%%%%%%%%%%%%%%%%%%%%%%%%%%%%%%%%%%%%%%%%%%%%%%%%%%%%%%%%%%%%%%%%%%%%%%%%
\subsection{Revision History}

%%%%%%%%%%%%%%%%%%%%%%%%%%%%%%%%%%%%%%%%
\paragraph{v2.0:} 2018/12/30

\begin{itemize}
\item
immediate forward processing
\item
added |\childdocby| mechanism
\item
manual restructured
\end{itemize}

%%%%%%%%%%%%%%%%%%%%%%%%%%%%%%%%%%%%%%%%
\paragraph{v1.6:} 2018/01/17

\begin{itemize}
\item
application for development of include files
\item
corrections to manual
\end{itemize}

%%%%%%%%%%%%%%%%%%%%%%%%%%%%%%%%%%%%%%%%
\paragraph{v1.5:} 2017/05/21

\begin{itemize}
\item
more complete structuring introduced
\item
|\childdocof| introduced
\item
|\childdoc| renamed to |\childdocmain|
\item
|\childredirect| renamed to |\childdocforward| and |\childdocforwardprefix|
and functionality expanded
\end{itemize}

%%%%%%%%%%%%%%%%%%%%%%%%%%%%%%%%%%%%%%%%
\paragraph{v1.0:} 2017/04/27

\begin{itemize}
\item
manual and install package
\item
first version published on CTAN
\end{itemize}

%%%%%%%%%%%%%%%%%%%%%%%%%%%%%%%%%%%%%%%%
\paragraph{v0.6:} 2017/04/26

\begin{itemize}
\item
redirection mechanism added
\end{itemize}

%%%%%%%%%%%%%%%%%%%%%%%%%%%%%%%%%%%%%%%%
\paragraph{v0.5:} 2017/04/26

\begin{itemize}
\item
functionality in definition file
\end{itemize}


%%%%%%%%%%%%%%%%%%%%%%%%%%%%%%%%%%%%%%%%%%%%%%%%%%%%%%%%%%%%%%%%%%%%%%%%%%%%%%%%
%%%%%%%%%%%%%%%%%%%%%%%%%%%%%%%%%%%%%%%%%%%%%%%%%%%%%%%%%%%%%%%%%%%%%%%%%%%%%%%%
%%%%%%%%%%%%%%%%%%%%%%%%%%%%%%%%%%%%%%%%%%%%%%%%%%%%%%%%%%%%%%%%%%%%%%%%%%%%%%%%
\appendix

\settowidth\MacroIndent{\rmfamily\scriptsize 000\ }

 \DocInput{childdoc.dtx}

\end{document}
%</driver>
% \fi
%
% %%%%%%%%%%%%%%%%%%%%%%%%%%%%%%%%%%%%%%%%%%%%%%%%%%%%%%%%%%%%%%%%%%%%%%%%%%%%%%
% %%%%%%%%%%%%%%%%%%%%%%%%%%%%%%%%%%%%%%%%%%%%%%%%%%%%%%%%%%%%%%%%%%%%%%%%%%%%%%
% \section{Sample}
%\iffalse
%<*samplemain>
%\fi
%
% The following presents a sample document
% with two chapters, two parts, a title page,
% a compile flag as well as three forwarding files to set the flag.
% It consists of eight |.tex| files:
% \begin{center}
% \begin{tabular}{ll}
% |cdocsamp.tex|&main file\\
% |cdocsch1.tex|&include file for chapter 1\\
% |cdocsch2.tex|&include file for chapter 2\\
% |cdocspt3.tex|&include file for part 3\\
% |cdocspt4.tex|&include file for part 4\\
% |cdocsdrf.tex|&forwarding file for main file in draft mode\\
% |cdocsfi1.tex|&forwarding file for final version of chapter 1\\
% |cdocsfi2.tex|&forwarding file for final version of chapter 2\\
% \end{tabular}
% \end{center}
% Each of the eight files can be compiled directly by the \LaTeX{} compiler.
%
% %%%%%%%%%%%%%%%%%%%%%%%%%%%%%%%%%%%%%%
% \paragraph{Main File.}
%
% The main file is called |cdocsamp.tex|.
%
% Load the \textsf{childdoc} definitions and
% declare the filename for the main document:
%    \begin{macrocode}
\input{childdoc.def}
\childdocmain{}
%    \end{macrocode}

% Optional override for |\version| flag:
%    \begin{macrocode}
%%\ifchilddoc\else\providecommand{\version}{draft}\fi
%    \end{macrocode}

% Define the default values for the |\version| flag
% (|final| for the main file and |draft| for childs):
%    \begin{macrocode}
\ifchilddoc
\providecommand{\version}{draft}
\else
\providecommand{\version}{final}
\fi
%    \end{macrocode}

% Load the standard document class:
%    \begin{macrocode}
\documentclass[12pt]{article}
%    \end{macrocode}

% Start the document body:
%    \begin{macrocode}
\begin{document}
%    \end{macrocode}

% Declare a title page.
% Print title, part of document being processed and version flag:
%    \begin{macrocode}
\addtocounter{page}{-1}
\begin{center}
{\LARGE\bfseries{}childdoc example\par}
\vspace{1cm}
\ifchilddoc
\ifchilddocmanual part\else chapter\fi:
`\childdocname' of `\childdocjob'\par
\else
main document: `\childdocjob'\par
\fi
version: \version\par
\end{center}
\newpage
%    \end{macrocode}

% Manually include selected file,
% otherwise process as usual:
%    \begin{macrocode}
\ifchilddocmanual
\section*{part `\childdocname'}
\input{\childdocname}
\else
%    \end{macrocode}

% Include the two chapters:
%    \begin{macrocode}
\include{cdocsch1}
\include{cdocsch2}
%    \end{macrocode}

% Include the two parts unless only chapters should be displayed:
%    \begin{macrocode}
\ifchilddoc\else
\section{part three}
\input{cdocspt3}
\section{part four}
\input{cdocspt4}
\fi
%    \end{macrocode}

% Process as usual until here:
%    \begin{macrocode}
\fi
%    \end{macrocode}

% End of document body:
%    \begin{macrocode}
\end{document}
%    \end{macrocode}
%\iffalse
%</samplemain>
%\fi
%
% %%%%%%%%%%%%%%%%%%%%%%%%%%%%%%%%%%%%%%
% \paragraph{Chapter Include Files.}
%
% The include files are called |cdocsch1.tex| and |cdocsch2.tex|.
%
%\iffalse
%<*samplechap1|samplechap2>
%\fi

% Optional override for |\version| flag:
%    \begin{macrocode}
%%\providecommand{\version}{final}
%    \end{macrocode}

% Include the main document:
%    \begin{macrocode}
\input{childdoc.def}
\childdocof{cdocsamp}
%    \end{macrocode}

%\iffalse
%</samplechap1|samplechap2>
%\fi
%
%\iffalse
%<*samplechap1>
%\fi
% Some text for chapter 1:
%    \begin{macrocode}
\section{one}
some text in chapter one
%    \end{macrocode}

%\iffalse
%</samplechap1>
%\fi
% Some text for chapter 2:
%\iffalse
%<*samplechap2>
%\fi
%    \begin{macrocode}
\section{two}
more text in chapter two
%    \end{macrocode}

%\iffalse
%</samplechap2>
%\fi
%
% %%%%%%%%%%%%%%%%%%%%%%%%%%%%%%%%%%%%%%
% \paragraph{Part Include Files.}
%
% The include files are called |cdocspt3.tex| and |cdocspt4.tex|.
%
%\iffalse
%<*samplepart3|samplepart4>
%\fi

% Optional override for |\version| flag:
%    \begin{macrocode}
%%\providecommand{\version}{final}
%    \end{macrocode}

% Include the main document:
%    \begin{macrocode}
\input{childdoc.def}
\childdocby{cdocsamp}
%    \end{macrocode}

%\iffalse
%</samplepart3|samplepart4>
%\fi
%
%\iffalse
%<*samplepart3>
%\fi
% Some text for part 3:
%    \begin{macrocode}
some text in part three
%    \end{macrocode}

%\iffalse
%</samplepart3>
%\fi
% Some text for part 4:
%\iffalse
%<*samplepart4>
%\fi
%    \begin{macrocode}
more text in part four
%    \end{macrocode}

%\iffalse
%</samplepart4>
%\fi
%
% %%%%%%%%%%%%%%%%%%%%%%%%%%%%%%%%%%%%%%
% \paragraph{Forwarding for a Complete Draft.}
%
% The following forwarding file |cdocsdrf.tex|
% compiles the main document in draft mode:
%\iffalse
%<*sampledraft>
%\fi
%    \begin{macrocode}
\def\version{draft}
\input{childdoc.def}
\childdocforward{cdocsamp}
%    \end{macrocode}

%\iffalse
%</sampledraft>
%\fi
%
% %%%%%%%%%%%%%%%%%%%%%%%%%%%%%%%%%%%%%%
% \paragraph{Forwarding for Final Version of the Chapters.}
%
% The following forwarding files |cdocsfn1.tex| and |cdocsfn2.tex|
% (with identical content)
% compile the final versions of the child documents
% |cdocsch1.tex| and |cdocsch2.tex|, respectively:
%\iffalse
%<*samplefinal>
%\fi
%    \begin{macrocode}
\def\version{final}
\input{childdoc.def}
\childdocforwardprefix[cdocsamp]{cdocsfn}{cdocsch}
%    \end{macrocode}

%\iffalse
%</samplefinal>
%\fi
%
% %%%%%%%%%%%%%%%%%%%%%%%%%%%%%%%%%%%%%%
% \paragraph{Command Line Processing.}
%
% The following three command lines generate the output files
% |cdocscld|, |cdocscl1| and |cdocscl2|
% which should be identical to
% |cdocsdrf|, |cdocsch1| and |cdocsfn2|, respectively:
% \begin{center}
% \begin{tabular}{l}
% |latex -jobname cdocscld \|\\
% |  "\def\version{draft}\input{childdoc.def}\childdocforward{cdocsamp}"|\\
% |latex -jobname cdocscl1 \|\\
% |  "\input{childdoc.def}\childdocforward[cdocsamp]{cdocsch1}"|\\
% |latex -jobname cdocscl2 \|\\
% |  "\def\version{final}\input{childdoc.def}\childdocforward{cdocsch2}"|
% \end{tabular}
% \end{center}
% Note that the trailing backslash on each first line
% merely continues the input to the second line
% (for convenient cut ant paste).
% Furthermore, the command |latex| can be replaced by any
% of its alternative versions such as |pdflatex|.
%
% %%%%%%%%%%%%%%%%%%%%%%%%%%%%%%%%%%%%%%%%%%%%%%%%%%%%%%%%%%%%%%%%%%%%%%%%%%%%%%
% %%%%%%%%%%%%%%%%%%%%%%%%%%%%%%%%%%%%%%%%%%%%%%%%%%%%%%%%%%%%%%%%%%%%%%%%%%%%%%
% \section{Implementation}
%\iffalse
%<*package>
%\fi
%
% This section describes the definitions file |childdoc.def|.

% The definitions cannot be loaded using |\usepackage| or |\RequirePackage|
% which has a mechanism to prevent loading a style file more than once.
% When loading the definitions by means of |\input|
% multiple instances have to be prevented manually:
%\iffalse
%This code needs to be before the `\ProvidesFile' directive
%which is defined at the beginning of this file.
%Therefore it is also placed there and commented out here.
%</package>
%<*discard>
%\fi
%    \begin{macrocode}
\ifdefined\childdocmain\endinput\fi
%    \end{macrocode}
%\iffalse
%</discard>
%<*package>
%\fi
%
% \macro{\ifchilddoc}
% \macro{\ifchilddocmanual}
% The conditional |\ifchilddoc| tells whether a
% child (true) or main (false) document is being compiled.
% The conditional |\ifchilddocmanual| tells whether
% the |\includeonly| mechanism is used (false) or
% the selection of child files must be performed manually (true).
% The definitions initialise to false:
%    \begin{macrocode}
\newif\ifchilddoc
\newif\ifchilddocmanual
%    \end{macrocode}

% \macro{\childdocname}
% \macro{\childdocjob}
% The macro |\childdocname| stores the name of the main document
% to be compiled. The macro |\childdocjob| stores the name of
% the document on which the \LaTeX{} compiler was originally invoked.
% The content of |\jobname| cannot be compared
% to filenames specified in the source due to different catcodes.
% The following code rescans |\jobname|, stores the result
% in |\childdocname| and saves a copy in |\childdocjob|:
%    \begin{macrocode}
\edef\childdocname{\scantokens\expandafter{\jobname\noexpand}}
\let\childdocjob\childdocname
%    \end{macrocode}

% \macro{\childdocdisable}
% The macro |\childdocdisable| prevents the main file
% from being processed more than once.
% At this stage, the main document command |\childdocmain|
% is assumed to be called once again where it should do nothing.
% Any subsequent call to it should prevent
% a secondary processing of the main document
% It overwrites the forwarding commands
% |\childdocof| and |\childdocforward|
% with empty macros to prevent further inclusions of the main document:
%    \begin{macrocode}
\newcommand{\childdocdisable}
{
  \renewcommand{\childdocmain}[1]{\renewcommand{\childdocmain}[1]{\endinput}}
  \renewcommand{\childdocof}[1]{}
  \renewcommand{\childdocby}[2][]{}
  \renewcommand{\childdocforward}[2][]{}
  \renewcommand{\childdocdisable}{}
}
%    \end{macrocode}

% \macro{\childdocmain}
% The macro |\childdocmain| is to be called at the top of the main file
% with nothing or the main filename (without extension) as argument.
% First, it breaks loops.
% If the argument is not empty and does not match |\childdocname|
% (which is set by the first inclusion of |childdoc.def|),
% |\ifchilddoc| is set to true, |\includeonly| is applied to the child file
% and |\jobname| is set to the main file
% (for proper handling of |.aux| files):
%    \begin{macrocode}
\newcommand{\childdocmain}[1]
{
  \childdocdisable\childdocmain{}
  \if?#1?\else
    \begingroup
      \def\childdoctmp{#1}
      \ifx\childdoctmp\childdocname
        \def\childdoctmp{}
      \else
        \def\childdoctmp
        {
          \childdoctrue
          \includeonly{\childdocname}
          \def\childdocjob{#1}
          \def\jobname{#1}
        }
      \fi
      \expandafter
    \endgroup
    \childdoctmp
  \fi
}
%    \end{macrocode}

% \macro{\childdocof}
% The command |\childdocof| redirects
% compilation to the main file |#1|.
%    \begin{macrocode}
\newcommand{\childdocof}[1]
{
  \childdocdisable
  \childdoctrue
  \includeonly{\childdocname}
  \def\jobname{#1}
  \def\childdocjob{#1}
  \input{#1}
}
%    \end{macrocode}

% \macro{\childdocby}
% The command |\childdocby| ....
%    \begin{macrocode}
\newcommand{\childdocby}[2][]
{
  \childdocdisable
  \childdoctrue
  \childdocmanualtrue
  \if?#1?\else
    \def\jobname{#2}
  \fi
  \def\childdocjob{#2}
  \input{#2}
  \endinput
}
%    \end{macrocode}

% \macro{\childdocforward}
% The command |\childdocforward| redirects
% compilation to the main file or
% (if the optional argument is given) a child file.
% Parameters are set as if the main file
% or a child file starting with |\childdocof| was compiled.
% Then compilation is handed over to the main file:
%    \begin{macrocode}
\newcommand{\childdocforward}[2][]
{
  \begingroup
    \if?#1?
      \def\childdoctmp
      {
        \def\childdocname{#2}
        \def\childdocjob{#2}
        \def\jobname{#2}
        \input{#2}
        \endinput
      }
    \else
      \def\childdoctmp
      {
        \childdocdisable
        \def\childdocname{#2}
        \childdoctrue
        \includeonly{#2}
        \def\childdocjob{#1}
        \def\jobname{#1}
        \input{#1}
        \endinput
      }
    \fi
    \expandafter
  \endgroup
  \childdoctmp
}
%    \end{macrocode}

% \macro{\childdocforwardprefix}
% The command |\childdocforwardprefix| redirects
% compilation to the main or a child file by means of a pattern.
% The prefix |#1| in the current filename is replaced by |#2|
% and the suffix of the current filename is kept
% (it is assumed that the filename does not contain the substring `|~~~|'
% which is used as a delimiter).
% Compilation is handed over to the new file by |\childdocforward|:
%    \begin{macrocode}
\newcommand{\childdocforwardprefix}[3][]
{
  \begingroup
    \def\childdocextract #2##1~~~{\def\childdoctmp{\childdocforward[#1]{#3##1}}}
    \expandafter\childdocextract\childdocname~~~
    \expandafter
  \endgroup
  \childdoctmp
}
%    \end{macrocode}

% \macro{\childdoc}
% The deprecated macro |\childdoc| is a legacy version of |\childdocmain|:
%    \begin{macrocode}
\newcommand{\childdoc}{\childdocmain}
%    \end{macrocode}

% \macro{\childdocredirect}
% The deprecated macro |\childdocredirect| is a legacy version
% of |\childdocforward| and |\childdocforwardprefix|:
%    \begin{macrocode}
\newcommand{\childdocredirect}[2][]
{
  \begingroup
    \if?#1?
      \def\childdoctmp{\childdocforward{#2}}
    \else
      \def\childdoctmp{\childdocforwardprefix{#1}{#2}}
    \fi
    \expandafter
  \endgroup
  \childdoctmp
}
%    \end{macrocode}

%\iffalse
%</package>
%\fi
%
\endinput
|\\
|\childdocby{|\textit{main}|}|\\
\end{tabular}
\end{center}
%
The directive |\childdocby| is similar to |\childdocof|
described in \secref{sec:include},
but the subsequent selection of content must be done manually.
To that end, both |\ifchilddoc| and |\ifchilddocmanual|
will be true upon processing of a part,
and the name of the part is stored in |\childdocname|.
Note that |\jobname| will be set to the filename of the current part
so that each part receives an individual |.aux| file
that does not interfere with the |.aux| file(s) of the main document.
This behaviour can be altered by the alternative form
|\childdocby[*]{|\textit{main}|}| (with a non-empty optional argument)
which uses the |.aux| file of the main document
by setting |\jobname| to \textit{main}.

%%%%%%%%%%%%%%%%%%%%%%%%%%%%%%%%%%%%%%%%%%%%%%%%%%%%%%%%%%%%%%%%%%%%%%%%%%%%%%%%
\subsection{Driver Development}
\label{sec:driver}

The \textsf{childdoc} mechanism can also be use for the development
of definition files such as \LaTeX{} styles or classes.
This case differs from the above setup with multiple parts
included by |\include| in that no |\includeonly| should be invoked.
This can be achieved by starting the include file
(before |\ProvidesPackage|) with:
%
\begin{center}
\begin{tabular}{l}
|% \iffalse
%
% childdoc.dtx Copyright (C) 2017-2018 Niklas Beisert
%
% This work may be distributed and/or modified under the
% conditions of the LaTeX Project Public License, either version 1.3
% of this license or (at your option) any later version.
% The latest version of this license is in
%   http://www.latex-project.org/lppl.txt
% and version 1.3 or later is part of all distributions of LaTeX
% version 2005/12/01 or later.
%
% This work has the LPPL maintenance status `maintained'.
%
% The Current Maintainer of this work is Niklas Beisert.
%
% This work consists of the files childdoc.dtx and childdoc.ins
% and the derived files childdoc.def and cdocsamp.tex with
% cdocsch1.tex, cdocsch2.tex, cdocsdrf.tex, cdocsfn1.tex, cdocsfn2.tex.
%
%<package>\ifdefined\childdocmain\endinput\fi
%<package>\ProvidesFile{childdoc.def}[2018/12/30 v2.0 child document driver]
%<samplemain>\ProvidesFile{cdocsamp.tex}[2018/12/30 v2.0 sample for childdoc]
%<*driver>
%\ProvidesFile{childdoc.drv}[2018/12/30 v2.0 childdoc reference manual file]
\PassOptionsToClass{10pt,a4paper}{article}
\documentclass{ltxdoc}

\usepackage[margin=35mm]{geometry}
\usepackage{hyperref}
\usepackage{hyperxmp}
\usepackage[usenames]{color}

\hypersetup{colorlinks=true}
\hypersetup{pdfstartview=FitH}
\hypersetup{pdfpagemode=UseNone}
\hypersetup{pdfsource={}}
\hypersetup{pdflang={en-UK}}
\hypersetup{pdfcopyright={Copyright 2017-2018 Niklas Beisert.
  This work may be distributed and/or modified under the
  conditions of the LaTeX Project Public License, either version 1.3
  of this license or (at your option) any later version.}}
\hypersetup{pdflicenseurl={http://www.latex-project.org/lppl.txt}}
\hypersetup{pdfcontactaddress={ETH Zurich, ITP, HIT K,
  Wolfgang-Pauli-Strasse 27}}
\hypersetup{pdfcontactpostcode={8093}}
\hypersetup{pdfcontactcity={Zurich}}
\hypersetup{pdfcontactcountry={Switzerland}}
\hypersetup{pdfcontactemail={nbeisert@itp.phys.ethz.ch}}
\hypersetup{pdfcontacturl={http://people.phys.ethz.ch/\xmptilde nbeisert/}}

\newcommand{\secref}[1]{\hyperref[#1]{section \ref*{#1}}}

\parskip1ex
\parindent0pt
\let\olditemize\itemize
\def\itemize{\olditemize\parskip0pt}

\begin{document}

\title{The \textsf{childdoc} Package}
\hypersetup{pdftitle={The childdoc Package}}
\author{Niklas Beisert\\[2ex]
  Institut f\"ur Theoretische Physik\\
  Eidgen\"ossische Technische Hochschule Z\"urich\\
  Wolfgang-Pauli-Strasse 27, 8093 Z\"urich, Switzerland\\[1ex]
  \href{mailto:nbeisert@itp.phys.ethz.ch}
  {\texttt{nbeisert@itp.phys.ethz.ch}}}
\hypersetup{pdfauthor={Niklas Beisert}}
\hypersetup{pdfsubject={Manual for the LaTeX2e Package childdoc}}
\date{30 December 2018, \textsf{v2.0}}
\maketitle

\begin{abstract}\noindent
\textsf{childdoc} is a \LaTeXe{} package
that enables the direct compilation
of document sections included by |\include|
to individual files.
\end{abstract}

\begingroup
\parskip0ex
\tableofcontents
\endgroup

%%%%%%%%%%%%%%%%%%%%%%%%%%%%%%%%%%%%%%%%%%%%%%%%%%%%%%%%%%%%%%%%%%%%%%%%%%%%%%%%
%%%%%%%%%%%%%%%%%%%%%%%%%%%%%%%%%%%%%%%%%%%%%%%%%%%%%%%%%%%%%%%%%%%%%%%%%%%%%%%%
\section{Introduction}

\LaTeX{} provides a mechanism to structure a large document (such as a book)
into a main file and several child files (containing the chapters)
using the |\include| command.
This mechanism is beneficial for documents
which span hundreds of pages in order to
make the source file(s) more manageable.
Moreover, compilation can be restricted to
selected child files by means of the |\includeonly| command.
The latter feature can be used to reduce the compilation time while editing
(this was significantly more useful in the earlier days of \LaTeX{})
or to generate a smaller document which is easier to navigate.
Another application of |\includeonly| is to generate
documents consisting of selected parts of the complete document.

However, there are a few drawbacks of the plain |\include| mechanism:
\begin{itemize}
\item
The child files cannot be compiled on their own,
they can only be compiled via the main file.
A naive editing environment
(such as a text editor with an option
to have the current file processed by \LaTeX)
may require one to switch to the main file before compiling;
attempting to compile the child file produces errors.
\item
The main file must be modified (each time)
to adjust the |\includeonly| command
to the present needs. This easily leaves the main file in a messy state.
\item
The generated document will always carry the filename
of the main document. This is inconvenient if
several child files are to be compiled and
to be kept for distribution.
\end{itemize}

The present package provides a simple interface
to make child files individually compilable by \LaTeX{}.
Compiling a child file then has the same effect as compiling
the main file with an |\includeonly| command
to select the appropriate child.
Moreover the generated document will carry the name of the child
rather than the main file.
This resolves all three above issues.

This feature is meant to make the editing of books,
thesis documents and lecture notes somewhat more convenient.
However, the package can also be used efficiently for
composing a series of documents (such as exercise sheets)
which are typically distributed individually.
It then assists the author in generating the individual documents
(potentially in different versions)
as well as a document containing the collected series.
Another application is in developing style files
or other kinds of included material
where compilation of the style file could redirect
to a sample or test file.

%%%%%%%%%%%%%%%%%%%%%%%%%%%%%%%%%%%%%%%%%%%%%%%%%%%%%%%%%%%%%%%%%%%%%%%%%%%%%%%%
%%%%%%%%%%%%%%%%%%%%%%%%%%%%%%%%%%%%%%%%%%%%%%%%%%%%%%%%%%%%%%%%%%%%%%%%%%%%%%%%
\section{Usage}

First of all, the package \textsf{childdoc} is \emph{not} a standard
\LaTeXe{} |.sty| style file! Therefore it needs to be invoked in
a non-standard way.

%%%%%%%%%%%%%%%%%%%%%%%%%%%%%%%%%%%%%%%%%%%%%%%%%%%%%%%%%%%%%%%%%%%%%%%%%%%%%%%%
\subsection{Included Files}
\label{sec:include}

%%%%%%%%%%%%%%%%%%%%%%%%%%%%%%%%%%%%%%%%
\DescribeMacro{\childdocmain}
To use the package, add the commands
\begin{center}
\begin{tabular}{l}
|\input{childdoc.def}|\\
|\childdocmain{}|\\
\end{tabular}
\end{center}
at the very top of the main \LaTeX{} file,
in particular \emph{before} the |\documentclass| statement!
The argument of |\childdocmain| should be left empty
(but it must be present).

%%%%%%%%%%%%%%%%%%%%%%%%%%%%%%%%%%%%%%%%
\DescribeMacro{\childdocof}
Furthermore, add the commands
\begin{center}
\begin{tabular}{l}
|\input{childdoc.def}|\\
|\childdocof{|\textit{main}|}|\\
\end{tabular}
\end{center}
at the top of every child file \textit{child}
which is included by |\include{|\textit{child}|}|
from within the main file
(or at least for those files to be compiled individually).
The argument \textit{main} must be the filename of the main file.

There are a couple of
considerations in setting up the main and child documents:

%%%%%%%%%%%%%%%%%%%%%%%%%%%%%%%%%%%%%%%%
\paragraph{Restrictions.}

Please note the following restrictions:
\begin{itemize}
\item
|\childdocmain| must be called with one argument \textit{main}
to ensure compatibility with earlier version of the package.
It must either be empty (|\childdocmain{}|)
or precisely match the filename of the main file in which it is specified.
See \secref{sec:detection} for further information.
\item
The filename \textit{main} must be specified without the |.tex| extension.
\item
The filename \textit{main} is case sensitive
(even in case-insensitive file systems)
due to internal string comparison.
\item
The argument \textit{main} should be fully expanded, it cannot be a macro.
\item
Subdirectories and special characters should be avoided in filenames.
\item
The command |\childdocmain{|\textit{main}|}| must be followed by a whitespace.
It should not be followed immediately by another command
or by a comment mark `|%|'.
This is because the \TeX{} parser reads the token immediately following
the argument of |\childdocmain| and puts it
at the beginning of every child section;
however, a white\-space is ignored.
\end{itemize}

%%%%%%%%%%%%%%%%%%%%%%%%%%%%%%%%%%%%%%%%
\paragraph{Content of Main File.}

It is advisable to place all content in the child files included by |\include|.
Any output contained in the main file will appear in all child documents
unless suppressed manually;
it cannot be suppressed automatically by the |\includeonly| directive
and thus should normally be avoided.
A method to include some content in the main file
by means of conditional processing is described in \secref{sec:conditional}.

%%%%%%%%%%%%%%%%%%%%%%%%%%%%%%%%%%%%%%%%
\paragraph{Page Numbering.}

When only a part of the document is compiled,
the appropriate numbering of pages
(as well as other status parameters)
is determined from the |.aux| files.
The latter contain information from previous passes.
However this information needs to propagate through
all intermediate child documents.
Therefore the page numbering in child documents may well
be inconsistent until the complete document is compiled at least once.

A useful (if unconventional) way to always ensure a consistent
page numbering is to restart the numbering in each child document
and denote the pages by `\textit{child}|.|\textit{page}'
where \textit{child} represents the chapter/section number of the child file.
This can be achieved by the command
|\numberwithin{page}{|\textit{child}|}|
of the \textsf{amsmath} package
where \textit{child} can be |chapter| or |section|
depending on the chosen structuring.
Alternatively, one can modify the macro |\thepage| appropriately
and reset the counter |page| at the start of each child file.

%%%%%%%%%%%%%%%%%%%%%%%%%%%%%%%%%%%%%%%%%%%%%%%%%%%%%%%%%%%%%%%%%%%%%%%%%%%%%%%%
\subsection{Conditional Processing}
\label{sec:conditional}

The package provides a mechanism to compile different versions
of a document. To customise the versions further some conditional processing
can come in handy to distinguish which version is being compiled.
The package provides two macros to describe the compilation context:

%%%%%%%%%%%%%%%%%%%%%%%%%%%%%%%%%%%%%%%%
\DescribeMacro{\ifchilddoc}
The conditional |\ifchilddoc| distinguishes between the compilation of
child documents and the main document:
%
\begin{center}
|\ifchilddoc |\textit{child-code}| |[|\||else |\textit{main-code}]| \||fi|
\end{center}

%%%%%%%%%%%%%%%%%%%%%%%%%%%%%%%%%%%%%%%%
\DescribeMacro{\childdocname}
\DescribeMacro{\childdocjob}
The macro |\childdocname| contains the filename (without extension)
of the main or child file being processed.
Note that |\childdocjob| will always contain the name of the main file.

%%%%%%%%%%%%%%%%%%%%%%%%%%%%%%%%%%%%%%%%
\paragraph{Title Page.}

Conditional processing can be used to include a title or banner page
in the main document when proper precautions are taken.
Importantly, the code in the main file should ensure that the page counter
(as well as other status parameters which are stored in the |.aux| files)
takes the same value after the conditional processing.
Otherwise the page numbers may take divergent values
depending on which part is compiled.

For example, a title page could be declared by:
%
\begin{center}
\begin{tabular}{l}
|\ifchilddoc\||else|\\
|\addtocounter{page}{-1}|\\
\textit{code for title page}\\
|\newpage|\\
|\||fi|
\end{tabular}
\end{center}
%
A banner page for the child documents can be generated by:
%
\begin{center}
\begin{tabular}{l}
|\ifchilddoc|\\
|\addtocounter{page}{-1}|\\
\textit{code for banner page}\\
|\newpage|\\
|\||fi|
\end{tabular}
\end{center}
%
Here one could write a message such as:
\begin{center}
|This is the part \childdocname{} of \childdocjob{}.|
\end{center}

%%%%%%%%%%%%%%%%%%%%%%%%%%%%%%%%%%%%%%%%%%%%%%%%%%%%%%%%%%%%%%%%%%%%%%%%%%%%%%%%
\subsection{Flags}
\label{sec:flags}

The package makes it easy to generate different versions
of the main or child documents.
To this end compilation flags can be defined
and assigned different default values.
They will be particularly useful in conjunction
with the forwarding mechanism described in \secref{sec:forward}.

For example, it may be useful to have a flag |\version|
which can be set to |draft| or |final|.
The document source will contain some conditional code
depending on the value of |\version|.
Suppose further, the flag should default to |final| for the main file
and to |draft| for child files
which is a natural assignment for editing the document.
This is achieved by placing the following code
in the preamble of the main document
(below the |\childdocmain| directive):
%
\begin{center}
\begin{tabular}{l}
|\ifchilddoc|\\
|\providecommand{\version}{draft}|\\
|\||else|\\
|\providecommand{\version}{final}|\\
|\||fi|
\end{tabular}
\end{center}
%
The definition by |\providecommand| makes sure
that previous definitions are not overwritten.
Further statements |\providecommand{\version}{...}|
can thus be added before the above code to override it.

For the main file, one might add a line
(between |\childdocmain| and the above block)
%
\begin{center}
|%\ifchilddoc\||else\providecommand{\version}{draft}\||fi|
\end{center}
%
which can be uncommented to produce a draft version.
Likewise one can add a line to the very top of a child file
(above the |\childdocof{|\textit{main}|}| directive)
%
\begin{center}
|%\providecommand{\version}{final}|
\end{center}
%
which can be uncommented to produce the final version of this child document.

%%%%%%%%%%%%%%%%%%%%%%%%%%%%%%%%%%%%%%%%%%%%%%%%%%%%%%%%%%%%%%%%%%%%%%%%%%%%%%%%
\subsection{Forwarding}
\label{sec:forward}

Different versions of the main or child documents
using compilation flags as described in \secref{sec:flags}
can be (permanently) stored in different files
for convenient compilation, viewing and distribution.
To this end, the package defines a command
to pass on compilation to a different file:

%%%%%%%%%%%%%%%%%%%%%%%%%%%%%%%%%%%%%%%%
\DescribeMacro{\childdocforward}
The command |\childdocforward| redirects processing to
another source file:
%
\begin{center}
\begin{tabular}{l}
|\input{childdoc.def}|\\
|\childdocforward[|\textit{main}|]{|\textit{dest}|}|\\
\end{tabular}
\end{center}
%
The argument \textit{dest} is the destination file
(without extension).
It should be the main file or one of the child files.
Note that further \textsf{childdoc} directives
such as |\childdocof| and |\childdocforward|
in the indicated file will be processed in this form.
The optional argument \textit{main}
passes on directly to the main file \textit{main}
while pretending to compile the child \textit{dest}.
This form behaves as if \textit{dest}
issues |\childdocof{|\textit{main}|}| right away,
and no further \textsf{childdoc} directives will be processed.

%%%%%%%%%%%%%%%%%%%%%%%%%%%%%%%%%%%%%%%%
\DescribeMacro{\...prefix}
In the alternative form |\childdocforwardprefix|,
%
\begin{center}
\begin{tabular}{l}
|\input{childdoc.def}|\\
|\childdocforwardprefix[|\textit{main}|]{|\textit{prefix}|}{|\textit{dest}|}|
\end{tabular}
\end{center}
%
the destination file is determined by a pattern
depending on the current file:
To make this work, the current file must be called
`{\textit{prefix}\hspace{0.2em}\textit{suffix}}'
with \textit{prefix} matching precisely the argument.
Processing is then passed on to the file
`{\textit{dest}\hspace{0.2em}\textit{suffix}}'.
Surely, the same effect is achieved by
directly specifying the
argument `{\textit{dest}\hspace{0.2em}\textit{suffix}}'
in the first form.
However, that requires to set up a different file
for each child. With the alternative form of the command
all these files can have exactly the same content
which simplifies setting them up and maintaining them.

For example, the following file |draft.tex|
with a compilation flag |\version| as described in \secref{sec:flags}
compiles the main document as a draft:
%
\begin{center}
\begin{tabular}{l}
|\def\version{draft}|\\
|\input{childdoc.def}|\\
|\childdocforward{|\textit{main}|}|
\end{tabular}
\end{center}
%
Likewise, the following files |final|\textit{nn}|.tex|
compile the final version of the child document
|child|\textit{nn}|.tex|:
%
\begin{center}
\begin{tabular}{l}
|\def\version{final}|\\
|\input{childdoc.def}|\\
|\childdocforwardprefix{final}{child}|
\end{tabular}
\end{center}
%

Note that when several versions of a main file and/or of each child file
are to be generated, it may be convenient to set up a |Makefile| or
shell script to automatise the process.

%%%%%%%%%%%%%%%%%%%%%%%%%%%%%%%%%%%%%%%%%%%%%%%%%%%%%%%%%%%%%%%%%%%%%%%%%%%%%%%%
\subsection{Command Line Processing}
\label{sec:commandline}

The effect of redirection files can also be achieved by invoking
the \LaTeX{} compiler with a more elaborate command line.
Most conveniently this should be done as part
of a shell script or a |Makefile|.

When using \textsf{childdoc} in the main file, the following
command lines effectively perform a redirection
(note that depending on the shell being used,
backslashes may have to be doubled: `|\|' $\to$ `|\\|'):
%
\begin{center}
|... -jobname "|\textit{target}|" |\\|"|[\textit{flags}]%
|\input{childdoc.def}\childdocforward[|\textit{main}|]{|\textit{dest}|}"|
\end{center}
%
Here \textit{target} is the name of the output file,
\textit{main} is the name of the main file
and \textit{dest} is the name of the main or child file to be processed
(all filenames without extensions).
The optional argument \textit{main} can be omitted
if \textit{main} matches \textit{dest}.
Optionally, compilation \textit{flags} can be defined via |\def| commands.
This command line makes the \TeX{} engine believe
it is compiling the file \textit{target}
whose content is specified as the latter parameter.
The provided code then forwards the processing to
\textit{main} or \textit{dest} as described in \secref{sec:forward}.

%%%%%%%%%%%%%%%%%%%%%%%%%%%%%%%%%%%%%%%%%%%%%%%%%%%%%%%%%%%%%%%%%%%%%%%%%%%%%%%%
\subsection{Include by Input}
\label{sec:input}

Including child documents by |\include| has some restrictions by design.
Most notably, the content of a child document always occupies
its own set of pages; pages cannot be shared between child documents.
Usually, this behaviour makes perfect sense
because each child document contain an essential part of the document.
However, in some situations it may be desirable to compose
a document from a collection of parts
without having mandatory page breaks between then.
For this case, the package
provides a mechanism to include parts
by |\input| which can also be processed individually.
However, by construction this mechanism
requires manual handling of the content to be output.

%%%%%%%%%%%%%%%%%%%%%%%%%%%%%%%%%%%%%%%%
\DescribeMacro{\ifchilddocmanual}
The main file should be prepared as usual, see \secref{sec:include}.
However, the document body must make a distinction
between processing of an individual part and of the main document, e.g.:
%
\begin{center}
\begin{tabular}{l}
|\ifchilddocmanual|\\
|\input{\childdocname}|\\
|\||else|\\
\textit{document body with }|\input{|\textit{part}|}|\\
|\||fi|
\end{tabular}
\end{center}
%
The conditional |\ifchilddocmanual| is true whenever
a part to be included by |\input| is being compiled,
and the name of the part is stored in |\childdocname|.

%%%%%%%%%%%%%%%%%%%%%%%%%%%%%%%%%%%%%%%%
\DescribeMacro{\childdocby}
Each part to be included by |\input| should start with:
%
\begin{center}
\begin{tabular}{l}
|\input{childdoc.def}|\\
|\childdocby{|\textit{main}|}|\\
\end{tabular}
\end{center}
%
The directive |\childdocby| is similar to |\childdocof|
described in \secref{sec:include},
but the subsequent selection of content must be done manually.
To that end, both |\ifchilddoc| and |\ifchilddocmanual|
will be true upon processing of a part,
and the name of the part is stored in |\childdocname|.
Note that |\jobname| will be set to the filename of the current part
so that each part receives an individual |.aux| file
that does not interfere with the |.aux| file(s) of the main document.
This behaviour can be altered by the alternative form
|\childdocby[*]{|\textit{main}|}| (with a non-empty optional argument)
which uses the |.aux| file of the main document
by setting |\jobname| to \textit{main}.

%%%%%%%%%%%%%%%%%%%%%%%%%%%%%%%%%%%%%%%%%%%%%%%%%%%%%%%%%%%%%%%%%%%%%%%%%%%%%%%%
\subsection{Driver Development}
\label{sec:driver}

The \textsf{childdoc} mechanism can also be use for the development
of definition files such as \LaTeX{} styles or classes.
This case differs from the above setup with multiple parts
included by |\include| in that no |\includeonly| should be invoked.
This can be achieved by starting the include file
(before |\ProvidesPackage|) with:
%
\begin{center}
\begin{tabular}{l}
|\input{childdoc.def}|\\
|\childdocforward{|\textit{main}|}|\\
\end{tabular}
\end{center}
%
or alternatively with:
%
\begin{center}
\begin{tabular}{l}
|\input{childdoc.def}|\\
|\childdocby{|\textit{main}|}|\\
\end{tabular}
\end{center}
%
Both forms have slightly different effects as described above.
The main file is prepared as usual, see \secref{sec:include}.

%%%%%%%%%%%%%%%%%%%%%%%%%%%%%%%%%%%%%%%%%%%%%%%%%%%%%%%%%%%%%%%%%%%%%%%%%%%%%%%%
\subsection{Legacy Detection}
\label{sec:detection}

The directive |\childdocmain| in the main file can detect
whether the complete document or merely a child is to be compiled
even without using the directive |\childdocof|.
This method is deprecated because it is less robust
and there is no compelling reason to use it;
it is merely provided for backward compatibility
and it may be removed in future versions.

If the detection mechanism is to be used,
it is mandatory to correctly specify
the filename of the main file as the argument of |\childdocmain|:
%
\begin{center}
\begin{tabular}{l}
|\input{childdoc.def}|\\
|\childdocmain{|\textit{main}|}|\\
\end{tabular}
\end{center}
%
If |\jobname| does not match the argument \textit{main} of |\childdocmain|,
it is assumed that |\jobname| points to the child file to be compiled.
When using |\childdocmain| with the main file specified as argument,
it suffices to start a child file
with just |\input{|\textit{main}|}|
without loading of the package and using |\childdocof|.
If instead all processing is done
with the appropriate \textsf{childdoc} directives,
the argument of \textit{main} of |\childdocmain| can be empty.

An alternative version of the command line processing described
in \secref{sec:commandline} using the detection mechanism reads:
%
\begin{center}
|... -jobname "|\textit{target}|" "|[\textit{flags}]%
[|\def\jobname{|\textit{dest}|}|]|\input{|\textit{main}|}"|
\end{center}

%%%%%%%%%%%%%%%%%%%%%%%%%%%%%%%%%%%%%%%%%%%%%%%%%%%%%%%%%%%%%%%%%%%%%%%%%%%%%%%%
\subsection{Manual Code}
\label{sec:manual}

In case one cannot be certain whether the definitions file |childdoc.def|
is installed on the target \TeX{} distribution
and one prefers not to ship it,
it is conceivable to paste a few relevant commands into the sources.

To that end, drop all statements |\input{childdoc.def}|
and perform the replacements as outlined below.
Instead of |\childdocmain{|\textit{main}|}| add the following code
to the top of the main file:
%
\begin{center}
\begin{tabular}{l}
|\||ifdefined\childdocname\endinput\||fi\newif\ifchilddoc|\\
|\edef\childdocname{\scantokens\expandafter{\jobname\noexpand}}|\\
|\def\childdocmain{|\textit{main}|}\||ifx\childdocmain\childdocname\||else|\\
|\childdoctrue\includeonly{\childdocname}\let\jobname\childdocmain\||fi|\\
\end{tabular}
\end{center}
%
Instead of |\childdocof{|\textit{main}|}| just include the main file
at the top of each child file:
%
\begin{center}
|\input{|\textit{main}|}|
\end{center}
%
A simple redirection |\childdocforward{|\textit{dest}|}| is achieved by:
%
\begin{center}
|\def\jobname{|\textit{dest}|}\input{\jobname}|
\end{center}
%
The redirection with prefix
|\childdocforwardprefix[|\textit{prefix}|]{|\textit{dest}|}|
is accomplished by:
%
\begin{center}
\begin{tabular}{l}
|{\edef\jobname{\scantokens\expandafter{\jobname\noexpand}}|\\
|\def\redirectjob |\textit{prefix}|#1~~~{\gdef\jobname{|\textit{dest}|#1}}|\\
|\expandafter\redirectjob\jobname~~~}\input{\jobname}|
\end{tabular}
\end{center}

In an alternative approach,
child documents can be compiled by a specific command line
without additional code or specific definitions:
%
\begin{center}
|... -jobname "|\textit{target}|" "|[\textit{flags}]%
|\includeonly{|\textit{dest}|}\input{|\textit{main}|}"|
\end{center}
%

%%%%%%%%%%%%%%%%%%%%%%%%%%%%%%%%%%%%%%%%%%%%%%%%%%%%%%%%%%%%%%%%%%%%%%%%%%%%%%%%
%%%%%%%%%%%%%%%%%%%%%%%%%%%%%%%%%%%%%%%%%%%%%%%%%%%%%%%%%%%%%%%%%%%%%%%%%%%%%%%%
\section{Information}

%%%%%%%%%%%%%%%%%%%%%%%%%%%%%%%%%%%%%%%%%%%%%%%%%%%%%%%%%%%%%%%%%%%%%%%%%%%%%%%%
\subsection{Copyright}

Copyright \copyright{} 2017--2018 Niklas Beisert

This work may be distributed and/or modified under the
conditions of the \LaTeX{} Project Public License, either version 1.3
of this license or (at your option) any later version.
The latest version of this license is in
  \url{http://www.latex-project.org/lppl.txt}
and version 1.3 or later is part of all distributions of \LaTeX{}
version 2005/12/01 or later.

This work has the LPPL maintenance status `maintained'.

The Current Maintainer of this work is Niklas Beisert.

This work consists of the files |README.txt|, |childdoc.ins| and |childdoc.dtx|
as well as the derived files |childdoc.def|, |cdocsamp.tex|
with |cdocsch1.tex|, |cdocsch2.tex|, |cdocspt3.tex|, |cdocspt4.tex|,
|cdocsdrf.tex|, |cdocsfn1.tex|, |cdocsfn2.tex|
as well as |childdoc.pdf|.

%%%%%%%%%%%%%%%%%%%%%%%%%%%%%%%%%%%%%%%%%%%%%%%%%%%%%%%%%%%%%%%%%%%%%%%%%%%%%%%%
\subsection{Files and Installation}

The package consists of the files:
%
\begin{center}
\begin{tabular}{ll}
    |README.txt|   & readme file \\
    |childdoc.ins| & installation file \\
    |childdoc.dtx| & source file \\
    |childdoc.def| & definition file \\
    |cdocsamp.tex| & sample main file \\
    |cdocsch1.tex| & sample include file \\
    |cdocsch2.tex| & sample include file \\
    |cdocspt3.tex| & sample part file \\
    |cdocspt4.tex| & sample part file \\
    |cdocsdrf.tex| & sample redirection file \\
    |cdocsfn1.tex| & sample redirection file \\
    |cdocsfn2.tex| & sample redirection file \\
    |childdoc.pdf| & manual
\end{tabular}
\end{center}
%
The distribution consists of the files
|README.txt|, |childdoc.ins| and |childdoc.dtx|.
%
\begin{itemize}
\item
Run (pdf)\LaTeX{} on |childdoc.dtx|
to compile the manual |childdoc.pdf| (this file).
\item
Run \LaTeX{} on |childdoc.ins| to create the definitions file |childdoc.def|
and the sample |cdocsamp.tex| with include files
|cdocsch1.tex|, |cdocsch2.tex|, |cdocspt3.tex|, |cdocspt4.tex|,
|cdocsdrf.tex|, |cdocsfn1.tex|, |cdocsfn2.tex|.
Then copy the file |childdoc.def| to an appropriate directory of your \LaTeX{}
distribution, e.g.\ \textit{texmf-root}|/tex/latex/childdoc|.
\end{itemize}

%%%%%%%%%%%%%%%%%%%%%%%%%%%%%%%%%%%%%%%%%%%%%%%%%%%%%%%%%%%%%%%%%%%%%%%%%%%%%%%%
\subsection{Related CTAN Packages}

There are several other packages which offer a similar functionality:
%
\begin{itemize}
\item
The packages
\href{http://ctan.org/pkg/docmute}{\textsf{docmute}},
\href{http://ctan.org/pkg/includex}{\textsf{includex}} and
\href{http://ctan.org/pkg/standalone}{\textsf{standalone}}
provide commands to include only the document body of
a child file thus allowing both files to be compiled individually.
\item
The packages \href{http://ctan.org/pkg/subdocs}{\textsf{subdocs}}
and \href{http://ctan.org/pkg/subfiles}{\textsf{subfiles}}
provide structures in which the main and child documents can be
encapsulated and allowing them to be compiled individually.
The inclusion mechanism is different from the conventional |\include|.
\item
The package \href{http://ctan.org/pkg/combine}{\textsf{combine}}
is an elaborate solution to combine several documents into one.
\end{itemize}
%
See also the CTAN topic \href{http://ctan.org/topic/subdocs}{\textsf{subdocs}}
for further related packages.
The present package differs from the above solutions in that
a document structure constructed with the conventional |\include| mechanism
just needs two extra commands at the top of every file
such that all constituent files can be compiled individually.

%%%%%%%%%%%%%%%%%%%%%%%%%%%%%%%%%%%%%%%%%%%%%%%%%%%%%%%%%%%%%%%%%%%%%%%%%%%%%%%%
%\subsection{Feature Suggestions}
%
%The following is a list of features which may be useful for future
%versions of this package:
%%
%\begin{itemize}
%\item
%\ldots
%\end{itemize}

%%%%%%%%%%%%%%%%%%%%%%%%%%%%%%%%%%%%%%%%%%%%%%%%%%%%%%%%%%%%%%%%%%%%%%%%%%%%%%%%
\subsection{Revision History}

%%%%%%%%%%%%%%%%%%%%%%%%%%%%%%%%%%%%%%%%
\paragraph{v2.0:} 2018/12/30

\begin{itemize}
\item
immediate forward processing
\item
added |\childdocby| mechanism
\item
manual restructured
\end{itemize}

%%%%%%%%%%%%%%%%%%%%%%%%%%%%%%%%%%%%%%%%
\paragraph{v1.6:} 2018/01/17

\begin{itemize}
\item
application for development of include files
\item
corrections to manual
\end{itemize}

%%%%%%%%%%%%%%%%%%%%%%%%%%%%%%%%%%%%%%%%
\paragraph{v1.5:} 2017/05/21

\begin{itemize}
\item
more complete structuring introduced
\item
|\childdocof| introduced
\item
|\childdoc| renamed to |\childdocmain|
\item
|\childredirect| renamed to |\childdocforward| and |\childdocforwardprefix|
and functionality expanded
\end{itemize}

%%%%%%%%%%%%%%%%%%%%%%%%%%%%%%%%%%%%%%%%
\paragraph{v1.0:} 2017/04/27

\begin{itemize}
\item
manual and install package
\item
first version published on CTAN
\end{itemize}

%%%%%%%%%%%%%%%%%%%%%%%%%%%%%%%%%%%%%%%%
\paragraph{v0.6:} 2017/04/26

\begin{itemize}
\item
redirection mechanism added
\end{itemize}

%%%%%%%%%%%%%%%%%%%%%%%%%%%%%%%%%%%%%%%%
\paragraph{v0.5:} 2017/04/26

\begin{itemize}
\item
functionality in definition file
\end{itemize}


%%%%%%%%%%%%%%%%%%%%%%%%%%%%%%%%%%%%%%%%%%%%%%%%%%%%%%%%%%%%%%%%%%%%%%%%%%%%%%%%
%%%%%%%%%%%%%%%%%%%%%%%%%%%%%%%%%%%%%%%%%%%%%%%%%%%%%%%%%%%%%%%%%%%%%%%%%%%%%%%%
%%%%%%%%%%%%%%%%%%%%%%%%%%%%%%%%%%%%%%%%%%%%%%%%%%%%%%%%%%%%%%%%%%%%%%%%%%%%%%%%
\appendix

\settowidth\MacroIndent{\rmfamily\scriptsize 000\ }

 \DocInput{childdoc.dtx}

\end{document}
%</driver>
% \fi
%
% %%%%%%%%%%%%%%%%%%%%%%%%%%%%%%%%%%%%%%%%%%%%%%%%%%%%%%%%%%%%%%%%%%%%%%%%%%%%%%
% %%%%%%%%%%%%%%%%%%%%%%%%%%%%%%%%%%%%%%%%%%%%%%%%%%%%%%%%%%%%%%%%%%%%%%%%%%%%%%
% \section{Sample}
%\iffalse
%<*samplemain>
%\fi
%
% The following presents a sample document
% with two chapters, two parts, a title page,
% a compile flag as well as three forwarding files to set the flag.
% It consists of eight |.tex| files:
% \begin{center}
% \begin{tabular}{ll}
% |cdocsamp.tex|&main file\\
% |cdocsch1.tex|&include file for chapter 1\\
% |cdocsch2.tex|&include file for chapter 2\\
% |cdocspt3.tex|&include file for part 3\\
% |cdocspt4.tex|&include file for part 4\\
% |cdocsdrf.tex|&forwarding file for main file in draft mode\\
% |cdocsfi1.tex|&forwarding file for final version of chapter 1\\
% |cdocsfi2.tex|&forwarding file for final version of chapter 2\\
% \end{tabular}
% \end{center}
% Each of the eight files can be compiled directly by the \LaTeX{} compiler.
%
% %%%%%%%%%%%%%%%%%%%%%%%%%%%%%%%%%%%%%%
% \paragraph{Main File.}
%
% The main file is called |cdocsamp.tex|.
%
% Load the \textsf{childdoc} definitions and
% declare the filename for the main document:
%    \begin{macrocode}
\input{childdoc.def}
\childdocmain{}
%    \end{macrocode}

% Optional override for |\version| flag:
%    \begin{macrocode}
%%\ifchilddoc\else\providecommand{\version}{draft}\fi
%    \end{macrocode}

% Define the default values for the |\version| flag
% (|final| for the main file and |draft| for childs):
%    \begin{macrocode}
\ifchilddoc
\providecommand{\version}{draft}
\else
\providecommand{\version}{final}
\fi
%    \end{macrocode}

% Load the standard document class:
%    \begin{macrocode}
\documentclass[12pt]{article}
%    \end{macrocode}

% Start the document body:
%    \begin{macrocode}
\begin{document}
%    \end{macrocode}

% Declare a title page.
% Print title, part of document being processed and version flag:
%    \begin{macrocode}
\addtocounter{page}{-1}
\begin{center}
{\LARGE\bfseries{}childdoc example\par}
\vspace{1cm}
\ifchilddoc
\ifchilddocmanual part\else chapter\fi:
`\childdocname' of `\childdocjob'\par
\else
main document: `\childdocjob'\par
\fi
version: \version\par
\end{center}
\newpage
%    \end{macrocode}

% Manually include selected file,
% otherwise process as usual:
%    \begin{macrocode}
\ifchilddocmanual
\section*{part `\childdocname'}
\input{\childdocname}
\else
%    \end{macrocode}

% Include the two chapters:
%    \begin{macrocode}
\include{cdocsch1}
\include{cdocsch2}
%    \end{macrocode}

% Include the two parts unless only chapters should be displayed:
%    \begin{macrocode}
\ifchilddoc\else
\section{part three}
\input{cdocspt3}
\section{part four}
\input{cdocspt4}
\fi
%    \end{macrocode}

% Process as usual until here:
%    \begin{macrocode}
\fi
%    \end{macrocode}

% End of document body:
%    \begin{macrocode}
\end{document}
%    \end{macrocode}
%\iffalse
%</samplemain>
%\fi
%
% %%%%%%%%%%%%%%%%%%%%%%%%%%%%%%%%%%%%%%
% \paragraph{Chapter Include Files.}
%
% The include files are called |cdocsch1.tex| and |cdocsch2.tex|.
%
%\iffalse
%<*samplechap1|samplechap2>
%\fi

% Optional override for |\version| flag:
%    \begin{macrocode}
%%\providecommand{\version}{final}
%    \end{macrocode}

% Include the main document:
%    \begin{macrocode}
\input{childdoc.def}
\childdocof{cdocsamp}
%    \end{macrocode}

%\iffalse
%</samplechap1|samplechap2>
%\fi
%
%\iffalse
%<*samplechap1>
%\fi
% Some text for chapter 1:
%    \begin{macrocode}
\section{one}
some text in chapter one
%    \end{macrocode}

%\iffalse
%</samplechap1>
%\fi
% Some text for chapter 2:
%\iffalse
%<*samplechap2>
%\fi
%    \begin{macrocode}
\section{two}
more text in chapter two
%    \end{macrocode}

%\iffalse
%</samplechap2>
%\fi
%
% %%%%%%%%%%%%%%%%%%%%%%%%%%%%%%%%%%%%%%
% \paragraph{Part Include Files.}
%
% The include files are called |cdocspt3.tex| and |cdocspt4.tex|.
%
%\iffalse
%<*samplepart3|samplepart4>
%\fi

% Optional override for |\version| flag:
%    \begin{macrocode}
%%\providecommand{\version}{final}
%    \end{macrocode}

% Include the main document:
%    \begin{macrocode}
\input{childdoc.def}
\childdocby{cdocsamp}
%    \end{macrocode}

%\iffalse
%</samplepart3|samplepart4>
%\fi
%
%\iffalse
%<*samplepart3>
%\fi
% Some text for part 3:
%    \begin{macrocode}
some text in part three
%    \end{macrocode}

%\iffalse
%</samplepart3>
%\fi
% Some text for part 4:
%\iffalse
%<*samplepart4>
%\fi
%    \begin{macrocode}
more text in part four
%    \end{macrocode}

%\iffalse
%</samplepart4>
%\fi
%
% %%%%%%%%%%%%%%%%%%%%%%%%%%%%%%%%%%%%%%
% \paragraph{Forwarding for a Complete Draft.}
%
% The following forwarding file |cdocsdrf.tex|
% compiles the main document in draft mode:
%\iffalse
%<*sampledraft>
%\fi
%    \begin{macrocode}
\def\version{draft}
\input{childdoc.def}
\childdocforward{cdocsamp}
%    \end{macrocode}

%\iffalse
%</sampledraft>
%\fi
%
% %%%%%%%%%%%%%%%%%%%%%%%%%%%%%%%%%%%%%%
% \paragraph{Forwarding for Final Version of the Chapters.}
%
% The following forwarding files |cdocsfn1.tex| and |cdocsfn2.tex|
% (with identical content)
% compile the final versions of the child documents
% |cdocsch1.tex| and |cdocsch2.tex|, respectively:
%\iffalse
%<*samplefinal>
%\fi
%    \begin{macrocode}
\def\version{final}
\input{childdoc.def}
\childdocforwardprefix[cdocsamp]{cdocsfn}{cdocsch}
%    \end{macrocode}

%\iffalse
%</samplefinal>
%\fi
%
% %%%%%%%%%%%%%%%%%%%%%%%%%%%%%%%%%%%%%%
% \paragraph{Command Line Processing.}
%
% The following three command lines generate the output files
% |cdocscld|, |cdocscl1| and |cdocscl2|
% which should be identical to
% |cdocsdrf|, |cdocsch1| and |cdocsfn2|, respectively:
% \begin{center}
% \begin{tabular}{l}
% |latex -jobname cdocscld \|\\
% |  "\def\version{draft}\input{childdoc.def}\childdocforward{cdocsamp}"|\\
% |latex -jobname cdocscl1 \|\\
% |  "\input{childdoc.def}\childdocforward[cdocsamp]{cdocsch1}"|\\
% |latex -jobname cdocscl2 \|\\
% |  "\def\version{final}\input{childdoc.def}\childdocforward{cdocsch2}"|
% \end{tabular}
% \end{center}
% Note that the trailing backslash on each first line
% merely continues the input to the second line
% (for convenient cut ant paste).
% Furthermore, the command |latex| can be replaced by any
% of its alternative versions such as |pdflatex|.
%
% %%%%%%%%%%%%%%%%%%%%%%%%%%%%%%%%%%%%%%%%%%%%%%%%%%%%%%%%%%%%%%%%%%%%%%%%%%%%%%
% %%%%%%%%%%%%%%%%%%%%%%%%%%%%%%%%%%%%%%%%%%%%%%%%%%%%%%%%%%%%%%%%%%%%%%%%%%%%%%
% \section{Implementation}
%\iffalse
%<*package>
%\fi
%
% This section describes the definitions file |childdoc.def|.

% The definitions cannot be loaded using |\usepackage| or |\RequirePackage|
% which has a mechanism to prevent loading a style file more than once.
% When loading the definitions by means of |\input|
% multiple instances have to be prevented manually:
%\iffalse
%This code needs to be before the `\ProvidesFile' directive
%which is defined at the beginning of this file.
%Therefore it is also placed there and commented out here.
%</package>
%<*discard>
%\fi
%    \begin{macrocode}
\ifdefined\childdocmain\endinput\fi
%    \end{macrocode}
%\iffalse
%</discard>
%<*package>
%\fi
%
% \macro{\ifchilddoc}
% \macro{\ifchilddocmanual}
% The conditional |\ifchilddoc| tells whether a
% child (true) or main (false) document is being compiled.
% The conditional |\ifchilddocmanual| tells whether
% the |\includeonly| mechanism is used (false) or
% the selection of child files must be performed manually (true).
% The definitions initialise to false:
%    \begin{macrocode}
\newif\ifchilddoc
\newif\ifchilddocmanual
%    \end{macrocode}

% \macro{\childdocname}
% \macro{\childdocjob}
% The macro |\childdocname| stores the name of the main document
% to be compiled. The macro |\childdocjob| stores the name of
% the document on which the \LaTeX{} compiler was originally invoked.
% The content of |\jobname| cannot be compared
% to filenames specified in the source due to different catcodes.
% The following code rescans |\jobname|, stores the result
% in |\childdocname| and saves a copy in |\childdocjob|:
%    \begin{macrocode}
\edef\childdocname{\scantokens\expandafter{\jobname\noexpand}}
\let\childdocjob\childdocname
%    \end{macrocode}

% \macro{\childdocdisable}
% The macro |\childdocdisable| prevents the main file
% from being processed more than once.
% At this stage, the main document command |\childdocmain|
% is assumed to be called once again where it should do nothing.
% Any subsequent call to it should prevent
% a secondary processing of the main document
% It overwrites the forwarding commands
% |\childdocof| and |\childdocforward|
% with empty macros to prevent further inclusions of the main document:
%    \begin{macrocode}
\newcommand{\childdocdisable}
{
  \renewcommand{\childdocmain}[1]{\renewcommand{\childdocmain}[1]{\endinput}}
  \renewcommand{\childdocof}[1]{}
  \renewcommand{\childdocby}[2][]{}
  \renewcommand{\childdocforward}[2][]{}
  \renewcommand{\childdocdisable}{}
}
%    \end{macrocode}

% \macro{\childdocmain}
% The macro |\childdocmain| is to be called at the top of the main file
% with nothing or the main filename (without extension) as argument.
% First, it breaks loops.
% If the argument is not empty and does not match |\childdocname|
% (which is set by the first inclusion of |childdoc.def|),
% |\ifchilddoc| is set to true, |\includeonly| is applied to the child file
% and |\jobname| is set to the main file
% (for proper handling of |.aux| files):
%    \begin{macrocode}
\newcommand{\childdocmain}[1]
{
  \childdocdisable\childdocmain{}
  \if?#1?\else
    \begingroup
      \def\childdoctmp{#1}
      \ifx\childdoctmp\childdocname
        \def\childdoctmp{}
      \else
        \def\childdoctmp
        {
          \childdoctrue
          \includeonly{\childdocname}
          \def\childdocjob{#1}
          \def\jobname{#1}
        }
      \fi
      \expandafter
    \endgroup
    \childdoctmp
  \fi
}
%    \end{macrocode}

% \macro{\childdocof}
% The command |\childdocof| redirects
% compilation to the main file |#1|.
%    \begin{macrocode}
\newcommand{\childdocof}[1]
{
  \childdocdisable
  \childdoctrue
  \includeonly{\childdocname}
  \def\jobname{#1}
  \def\childdocjob{#1}
  \input{#1}
}
%    \end{macrocode}

% \macro{\childdocby}
% The command |\childdocby| ....
%    \begin{macrocode}
\newcommand{\childdocby}[2][]
{
  \childdocdisable
  \childdoctrue
  \childdocmanualtrue
  \if?#1?\else
    \def\jobname{#2}
  \fi
  \def\childdocjob{#2}
  \input{#2}
  \endinput
}
%    \end{macrocode}

% \macro{\childdocforward}
% The command |\childdocforward| redirects
% compilation to the main file or
% (if the optional argument is given) a child file.
% Parameters are set as if the main file
% or a child file starting with |\childdocof| was compiled.
% Then compilation is handed over to the main file:
%    \begin{macrocode}
\newcommand{\childdocforward}[2][]
{
  \begingroup
    \if?#1?
      \def\childdoctmp
      {
        \def\childdocname{#2}
        \def\childdocjob{#2}
        \def\jobname{#2}
        \input{#2}
        \endinput
      }
    \else
      \def\childdoctmp
      {
        \childdocdisable
        \def\childdocname{#2}
        \childdoctrue
        \includeonly{#2}
        \def\childdocjob{#1}
        \def\jobname{#1}
        \input{#1}
        \endinput
      }
    \fi
    \expandafter
  \endgroup
  \childdoctmp
}
%    \end{macrocode}

% \macro{\childdocforwardprefix}
% The command |\childdocforwardprefix| redirects
% compilation to the main or a child file by means of a pattern.
% The prefix |#1| in the current filename is replaced by |#2|
% and the suffix of the current filename is kept
% (it is assumed that the filename does not contain the substring `|~~~|'
% which is used as a delimiter).
% Compilation is handed over to the new file by |\childdocforward|:
%    \begin{macrocode}
\newcommand{\childdocforwardprefix}[3][]
{
  \begingroup
    \def\childdocextract #2##1~~~{\def\childdoctmp{\childdocforward[#1]{#3##1}}}
    \expandafter\childdocextract\childdocname~~~
    \expandafter
  \endgroup
  \childdoctmp
}
%    \end{macrocode}

% \macro{\childdoc}
% The deprecated macro |\childdoc| is a legacy version of |\childdocmain|:
%    \begin{macrocode}
\newcommand{\childdoc}{\childdocmain}
%    \end{macrocode}

% \macro{\childdocredirect}
% The deprecated macro |\childdocredirect| is a legacy version
% of |\childdocforward| and |\childdocforwardprefix|:
%    \begin{macrocode}
\newcommand{\childdocredirect}[2][]
{
  \begingroup
    \if?#1?
      \def\childdoctmp{\childdocforward{#2}}
    \else
      \def\childdoctmp{\childdocforwardprefix{#1}{#2}}
    \fi
    \expandafter
  \endgroup
  \childdoctmp
}
%    \end{macrocode}

%\iffalse
%</package>
%\fi
%
\endinput
|\\
|\childdocforward{|\textit{main}|}|\\
\end{tabular}
\end{center}
%
or alternatively with:
%
\begin{center}
\begin{tabular}{l}
|% \iffalse
%
% childdoc.dtx Copyright (C) 2017-2018 Niklas Beisert
%
% This work may be distributed and/or modified under the
% conditions of the LaTeX Project Public License, either version 1.3
% of this license or (at your option) any later version.
% The latest version of this license is in
%   http://www.latex-project.org/lppl.txt
% and version 1.3 or later is part of all distributions of LaTeX
% version 2005/12/01 or later.
%
% This work has the LPPL maintenance status `maintained'.
%
% The Current Maintainer of this work is Niklas Beisert.
%
% This work consists of the files childdoc.dtx and childdoc.ins
% and the derived files childdoc.def and cdocsamp.tex with
% cdocsch1.tex, cdocsch2.tex, cdocsdrf.tex, cdocsfn1.tex, cdocsfn2.tex.
%
%<package>\ifdefined\childdocmain\endinput\fi
%<package>\ProvidesFile{childdoc.def}[2018/12/30 v2.0 child document driver]
%<samplemain>\ProvidesFile{cdocsamp.tex}[2018/12/30 v2.0 sample for childdoc]
%<*driver>
%\ProvidesFile{childdoc.drv}[2018/12/30 v2.0 childdoc reference manual file]
\PassOptionsToClass{10pt,a4paper}{article}
\documentclass{ltxdoc}

\usepackage[margin=35mm]{geometry}
\usepackage{hyperref}
\usepackage{hyperxmp}
\usepackage[usenames]{color}

\hypersetup{colorlinks=true}
\hypersetup{pdfstartview=FitH}
\hypersetup{pdfpagemode=UseNone}
\hypersetup{pdfsource={}}
\hypersetup{pdflang={en-UK}}
\hypersetup{pdfcopyright={Copyright 2017-2018 Niklas Beisert.
  This work may be distributed and/or modified under the
  conditions of the LaTeX Project Public License, either version 1.3
  of this license or (at your option) any later version.}}
\hypersetup{pdflicenseurl={http://www.latex-project.org/lppl.txt}}
\hypersetup{pdfcontactaddress={ETH Zurich, ITP, HIT K,
  Wolfgang-Pauli-Strasse 27}}
\hypersetup{pdfcontactpostcode={8093}}
\hypersetup{pdfcontactcity={Zurich}}
\hypersetup{pdfcontactcountry={Switzerland}}
\hypersetup{pdfcontactemail={nbeisert@itp.phys.ethz.ch}}
\hypersetup{pdfcontacturl={http://people.phys.ethz.ch/\xmptilde nbeisert/}}

\newcommand{\secref}[1]{\hyperref[#1]{section \ref*{#1}}}

\parskip1ex
\parindent0pt
\let\olditemize\itemize
\def\itemize{\olditemize\parskip0pt}

\begin{document}

\title{The \textsf{childdoc} Package}
\hypersetup{pdftitle={The childdoc Package}}
\author{Niklas Beisert\\[2ex]
  Institut f\"ur Theoretische Physik\\
  Eidgen\"ossische Technische Hochschule Z\"urich\\
  Wolfgang-Pauli-Strasse 27, 8093 Z\"urich, Switzerland\\[1ex]
  \href{mailto:nbeisert@itp.phys.ethz.ch}
  {\texttt{nbeisert@itp.phys.ethz.ch}}}
\hypersetup{pdfauthor={Niklas Beisert}}
\hypersetup{pdfsubject={Manual for the LaTeX2e Package childdoc}}
\date{30 December 2018, \textsf{v2.0}}
\maketitle

\begin{abstract}\noindent
\textsf{childdoc} is a \LaTeXe{} package
that enables the direct compilation
of document sections included by |\include|
to individual files.
\end{abstract}

\begingroup
\parskip0ex
\tableofcontents
\endgroup

%%%%%%%%%%%%%%%%%%%%%%%%%%%%%%%%%%%%%%%%%%%%%%%%%%%%%%%%%%%%%%%%%%%%%%%%%%%%%%%%
%%%%%%%%%%%%%%%%%%%%%%%%%%%%%%%%%%%%%%%%%%%%%%%%%%%%%%%%%%%%%%%%%%%%%%%%%%%%%%%%
\section{Introduction}

\LaTeX{} provides a mechanism to structure a large document (such as a book)
into a main file and several child files (containing the chapters)
using the |\include| command.
This mechanism is beneficial for documents
which span hundreds of pages in order to
make the source file(s) more manageable.
Moreover, compilation can be restricted to
selected child files by means of the |\includeonly| command.
The latter feature can be used to reduce the compilation time while editing
(this was significantly more useful in the earlier days of \LaTeX{})
or to generate a smaller document which is easier to navigate.
Another application of |\includeonly| is to generate
documents consisting of selected parts of the complete document.

However, there are a few drawbacks of the plain |\include| mechanism:
\begin{itemize}
\item
The child files cannot be compiled on their own,
they can only be compiled via the main file.
A naive editing environment
(such as a text editor with an option
to have the current file processed by \LaTeX)
may require one to switch to the main file before compiling;
attempting to compile the child file produces errors.
\item
The main file must be modified (each time)
to adjust the |\includeonly| command
to the present needs. This easily leaves the main file in a messy state.
\item
The generated document will always carry the filename
of the main document. This is inconvenient if
several child files are to be compiled and
to be kept for distribution.
\end{itemize}

The present package provides a simple interface
to make child files individually compilable by \LaTeX{}.
Compiling a child file then has the same effect as compiling
the main file with an |\includeonly| command
to select the appropriate child.
Moreover the generated document will carry the name of the child
rather than the main file.
This resolves all three above issues.

This feature is meant to make the editing of books,
thesis documents and lecture notes somewhat more convenient.
However, the package can also be used efficiently for
composing a series of documents (such as exercise sheets)
which are typically distributed individually.
It then assists the author in generating the individual documents
(potentially in different versions)
as well as a document containing the collected series.
Another application is in developing style files
or other kinds of included material
where compilation of the style file could redirect
to a sample or test file.

%%%%%%%%%%%%%%%%%%%%%%%%%%%%%%%%%%%%%%%%%%%%%%%%%%%%%%%%%%%%%%%%%%%%%%%%%%%%%%%%
%%%%%%%%%%%%%%%%%%%%%%%%%%%%%%%%%%%%%%%%%%%%%%%%%%%%%%%%%%%%%%%%%%%%%%%%%%%%%%%%
\section{Usage}

First of all, the package \textsf{childdoc} is \emph{not} a standard
\LaTeXe{} |.sty| style file! Therefore it needs to be invoked in
a non-standard way.

%%%%%%%%%%%%%%%%%%%%%%%%%%%%%%%%%%%%%%%%%%%%%%%%%%%%%%%%%%%%%%%%%%%%%%%%%%%%%%%%
\subsection{Included Files}
\label{sec:include}

%%%%%%%%%%%%%%%%%%%%%%%%%%%%%%%%%%%%%%%%
\DescribeMacro{\childdocmain}
To use the package, add the commands
\begin{center}
\begin{tabular}{l}
|\input{childdoc.def}|\\
|\childdocmain{}|\\
\end{tabular}
\end{center}
at the very top of the main \LaTeX{} file,
in particular \emph{before} the |\documentclass| statement!
The argument of |\childdocmain| should be left empty
(but it must be present).

%%%%%%%%%%%%%%%%%%%%%%%%%%%%%%%%%%%%%%%%
\DescribeMacro{\childdocof}
Furthermore, add the commands
\begin{center}
\begin{tabular}{l}
|\input{childdoc.def}|\\
|\childdocof{|\textit{main}|}|\\
\end{tabular}
\end{center}
at the top of every child file \textit{child}
which is included by |\include{|\textit{child}|}|
from within the main file
(or at least for those files to be compiled individually).
The argument \textit{main} must be the filename of the main file.

There are a couple of
considerations in setting up the main and child documents:

%%%%%%%%%%%%%%%%%%%%%%%%%%%%%%%%%%%%%%%%
\paragraph{Restrictions.}

Please note the following restrictions:
\begin{itemize}
\item
|\childdocmain| must be called with one argument \textit{main}
to ensure compatibility with earlier version of the package.
It must either be empty (|\childdocmain{}|)
or precisely match the filename of the main file in which it is specified.
See \secref{sec:detection} for further information.
\item
The filename \textit{main} must be specified without the |.tex| extension.
\item
The filename \textit{main} is case sensitive
(even in case-insensitive file systems)
due to internal string comparison.
\item
The argument \textit{main} should be fully expanded, it cannot be a macro.
\item
Subdirectories and special characters should be avoided in filenames.
\item
The command |\childdocmain{|\textit{main}|}| must be followed by a whitespace.
It should not be followed immediately by another command
or by a comment mark `|%|'.
This is because the \TeX{} parser reads the token immediately following
the argument of |\childdocmain| and puts it
at the beginning of every child section;
however, a white\-space is ignored.
\end{itemize}

%%%%%%%%%%%%%%%%%%%%%%%%%%%%%%%%%%%%%%%%
\paragraph{Content of Main File.}

It is advisable to place all content in the child files included by |\include|.
Any output contained in the main file will appear in all child documents
unless suppressed manually;
it cannot be suppressed automatically by the |\includeonly| directive
and thus should normally be avoided.
A method to include some content in the main file
by means of conditional processing is described in \secref{sec:conditional}.

%%%%%%%%%%%%%%%%%%%%%%%%%%%%%%%%%%%%%%%%
\paragraph{Page Numbering.}

When only a part of the document is compiled,
the appropriate numbering of pages
(as well as other status parameters)
is determined from the |.aux| files.
The latter contain information from previous passes.
However this information needs to propagate through
all intermediate child documents.
Therefore the page numbering in child documents may well
be inconsistent until the complete document is compiled at least once.

A useful (if unconventional) way to always ensure a consistent
page numbering is to restart the numbering in each child document
and denote the pages by `\textit{child}|.|\textit{page}'
where \textit{child} represents the chapter/section number of the child file.
This can be achieved by the command
|\numberwithin{page}{|\textit{child}|}|
of the \textsf{amsmath} package
where \textit{child} can be |chapter| or |section|
depending on the chosen structuring.
Alternatively, one can modify the macro |\thepage| appropriately
and reset the counter |page| at the start of each child file.

%%%%%%%%%%%%%%%%%%%%%%%%%%%%%%%%%%%%%%%%%%%%%%%%%%%%%%%%%%%%%%%%%%%%%%%%%%%%%%%%
\subsection{Conditional Processing}
\label{sec:conditional}

The package provides a mechanism to compile different versions
of a document. To customise the versions further some conditional processing
can come in handy to distinguish which version is being compiled.
The package provides two macros to describe the compilation context:

%%%%%%%%%%%%%%%%%%%%%%%%%%%%%%%%%%%%%%%%
\DescribeMacro{\ifchilddoc}
The conditional |\ifchilddoc| distinguishes between the compilation of
child documents and the main document:
%
\begin{center}
|\ifchilddoc |\textit{child-code}| |[|\||else |\textit{main-code}]| \||fi|
\end{center}

%%%%%%%%%%%%%%%%%%%%%%%%%%%%%%%%%%%%%%%%
\DescribeMacro{\childdocname}
\DescribeMacro{\childdocjob}
The macro |\childdocname| contains the filename (without extension)
of the main or child file being processed.
Note that |\childdocjob| will always contain the name of the main file.

%%%%%%%%%%%%%%%%%%%%%%%%%%%%%%%%%%%%%%%%
\paragraph{Title Page.}

Conditional processing can be used to include a title or banner page
in the main document when proper precautions are taken.
Importantly, the code in the main file should ensure that the page counter
(as well as other status parameters which are stored in the |.aux| files)
takes the same value after the conditional processing.
Otherwise the page numbers may take divergent values
depending on which part is compiled.

For example, a title page could be declared by:
%
\begin{center}
\begin{tabular}{l}
|\ifchilddoc\||else|\\
|\addtocounter{page}{-1}|\\
\textit{code for title page}\\
|\newpage|\\
|\||fi|
\end{tabular}
\end{center}
%
A banner page for the child documents can be generated by:
%
\begin{center}
\begin{tabular}{l}
|\ifchilddoc|\\
|\addtocounter{page}{-1}|\\
\textit{code for banner page}\\
|\newpage|\\
|\||fi|
\end{tabular}
\end{center}
%
Here one could write a message such as:
\begin{center}
|This is the part \childdocname{} of \childdocjob{}.|
\end{center}

%%%%%%%%%%%%%%%%%%%%%%%%%%%%%%%%%%%%%%%%%%%%%%%%%%%%%%%%%%%%%%%%%%%%%%%%%%%%%%%%
\subsection{Flags}
\label{sec:flags}

The package makes it easy to generate different versions
of the main or child documents.
To this end compilation flags can be defined
and assigned different default values.
They will be particularly useful in conjunction
with the forwarding mechanism described in \secref{sec:forward}.

For example, it may be useful to have a flag |\version|
which can be set to |draft| or |final|.
The document source will contain some conditional code
depending on the value of |\version|.
Suppose further, the flag should default to |final| for the main file
and to |draft| for child files
which is a natural assignment for editing the document.
This is achieved by placing the following code
in the preamble of the main document
(below the |\childdocmain| directive):
%
\begin{center}
\begin{tabular}{l}
|\ifchilddoc|\\
|\providecommand{\version}{draft}|\\
|\||else|\\
|\providecommand{\version}{final}|\\
|\||fi|
\end{tabular}
\end{center}
%
The definition by |\providecommand| makes sure
that previous definitions are not overwritten.
Further statements |\providecommand{\version}{...}|
can thus be added before the above code to override it.

For the main file, one might add a line
(between |\childdocmain| and the above block)
%
\begin{center}
|%\ifchilddoc\||else\providecommand{\version}{draft}\||fi|
\end{center}
%
which can be uncommented to produce a draft version.
Likewise one can add a line to the very top of a child file
(above the |\childdocof{|\textit{main}|}| directive)
%
\begin{center}
|%\providecommand{\version}{final}|
\end{center}
%
which can be uncommented to produce the final version of this child document.

%%%%%%%%%%%%%%%%%%%%%%%%%%%%%%%%%%%%%%%%%%%%%%%%%%%%%%%%%%%%%%%%%%%%%%%%%%%%%%%%
\subsection{Forwarding}
\label{sec:forward}

Different versions of the main or child documents
using compilation flags as described in \secref{sec:flags}
can be (permanently) stored in different files
for convenient compilation, viewing and distribution.
To this end, the package defines a command
to pass on compilation to a different file:

%%%%%%%%%%%%%%%%%%%%%%%%%%%%%%%%%%%%%%%%
\DescribeMacro{\childdocforward}
The command |\childdocforward| redirects processing to
another source file:
%
\begin{center}
\begin{tabular}{l}
|\input{childdoc.def}|\\
|\childdocforward[|\textit{main}|]{|\textit{dest}|}|\\
\end{tabular}
\end{center}
%
The argument \textit{dest} is the destination file
(without extension).
It should be the main file or one of the child files.
Note that further \textsf{childdoc} directives
such as |\childdocof| and |\childdocforward|
in the indicated file will be processed in this form.
The optional argument \textit{main}
passes on directly to the main file \textit{main}
while pretending to compile the child \textit{dest}.
This form behaves as if \textit{dest}
issues |\childdocof{|\textit{main}|}| right away,
and no further \textsf{childdoc} directives will be processed.

%%%%%%%%%%%%%%%%%%%%%%%%%%%%%%%%%%%%%%%%
\DescribeMacro{\...prefix}
In the alternative form |\childdocforwardprefix|,
%
\begin{center}
\begin{tabular}{l}
|\input{childdoc.def}|\\
|\childdocforwardprefix[|\textit{main}|]{|\textit{prefix}|}{|\textit{dest}|}|
\end{tabular}
\end{center}
%
the destination file is determined by a pattern
depending on the current file:
To make this work, the current file must be called
`{\textit{prefix}\hspace{0.2em}\textit{suffix}}'
with \textit{prefix} matching precisely the argument.
Processing is then passed on to the file
`{\textit{dest}\hspace{0.2em}\textit{suffix}}'.
Surely, the same effect is achieved by
directly specifying the
argument `{\textit{dest}\hspace{0.2em}\textit{suffix}}'
in the first form.
However, that requires to set up a different file
for each child. With the alternative form of the command
all these files can have exactly the same content
which simplifies setting them up and maintaining them.

For example, the following file |draft.tex|
with a compilation flag |\version| as described in \secref{sec:flags}
compiles the main document as a draft:
%
\begin{center}
\begin{tabular}{l}
|\def\version{draft}|\\
|\input{childdoc.def}|\\
|\childdocforward{|\textit{main}|}|
\end{tabular}
\end{center}
%
Likewise, the following files |final|\textit{nn}|.tex|
compile the final version of the child document
|child|\textit{nn}|.tex|:
%
\begin{center}
\begin{tabular}{l}
|\def\version{final}|\\
|\input{childdoc.def}|\\
|\childdocforwardprefix{final}{child}|
\end{tabular}
\end{center}
%

Note that when several versions of a main file and/or of each child file
are to be generated, it may be convenient to set up a |Makefile| or
shell script to automatise the process.

%%%%%%%%%%%%%%%%%%%%%%%%%%%%%%%%%%%%%%%%%%%%%%%%%%%%%%%%%%%%%%%%%%%%%%%%%%%%%%%%
\subsection{Command Line Processing}
\label{sec:commandline}

The effect of redirection files can also be achieved by invoking
the \LaTeX{} compiler with a more elaborate command line.
Most conveniently this should be done as part
of a shell script or a |Makefile|.

When using \textsf{childdoc} in the main file, the following
command lines effectively perform a redirection
(note that depending on the shell being used,
backslashes may have to be doubled: `|\|' $\to$ `|\\|'):
%
\begin{center}
|... -jobname "|\textit{target}|" |\\|"|[\textit{flags}]%
|\input{childdoc.def}\childdocforward[|\textit{main}|]{|\textit{dest}|}"|
\end{center}
%
Here \textit{target} is the name of the output file,
\textit{main} is the name of the main file
and \textit{dest} is the name of the main or child file to be processed
(all filenames without extensions).
The optional argument \textit{main} can be omitted
if \textit{main} matches \textit{dest}.
Optionally, compilation \textit{flags} can be defined via |\def| commands.
This command line makes the \TeX{} engine believe
it is compiling the file \textit{target}
whose content is specified as the latter parameter.
The provided code then forwards the processing to
\textit{main} or \textit{dest} as described in \secref{sec:forward}.

%%%%%%%%%%%%%%%%%%%%%%%%%%%%%%%%%%%%%%%%%%%%%%%%%%%%%%%%%%%%%%%%%%%%%%%%%%%%%%%%
\subsection{Include by Input}
\label{sec:input}

Including child documents by |\include| has some restrictions by design.
Most notably, the content of a child document always occupies
its own set of pages; pages cannot be shared between child documents.
Usually, this behaviour makes perfect sense
because each child document contain an essential part of the document.
However, in some situations it may be desirable to compose
a document from a collection of parts
without having mandatory page breaks between then.
For this case, the package
provides a mechanism to include parts
by |\input| which can also be processed individually.
However, by construction this mechanism
requires manual handling of the content to be output.

%%%%%%%%%%%%%%%%%%%%%%%%%%%%%%%%%%%%%%%%
\DescribeMacro{\ifchilddocmanual}
The main file should be prepared as usual, see \secref{sec:include}.
However, the document body must make a distinction
between processing of an individual part and of the main document, e.g.:
%
\begin{center}
\begin{tabular}{l}
|\ifchilddocmanual|\\
|\input{\childdocname}|\\
|\||else|\\
\textit{document body with }|\input{|\textit{part}|}|\\
|\||fi|
\end{tabular}
\end{center}
%
The conditional |\ifchilddocmanual| is true whenever
a part to be included by |\input| is being compiled,
and the name of the part is stored in |\childdocname|.

%%%%%%%%%%%%%%%%%%%%%%%%%%%%%%%%%%%%%%%%
\DescribeMacro{\childdocby}
Each part to be included by |\input| should start with:
%
\begin{center}
\begin{tabular}{l}
|\input{childdoc.def}|\\
|\childdocby{|\textit{main}|}|\\
\end{tabular}
\end{center}
%
The directive |\childdocby| is similar to |\childdocof|
described in \secref{sec:include},
but the subsequent selection of content must be done manually.
To that end, both |\ifchilddoc| and |\ifchilddocmanual|
will be true upon processing of a part,
and the name of the part is stored in |\childdocname|.
Note that |\jobname| will be set to the filename of the current part
so that each part receives an individual |.aux| file
that does not interfere with the |.aux| file(s) of the main document.
This behaviour can be altered by the alternative form
|\childdocby[*]{|\textit{main}|}| (with a non-empty optional argument)
which uses the |.aux| file of the main document
by setting |\jobname| to \textit{main}.

%%%%%%%%%%%%%%%%%%%%%%%%%%%%%%%%%%%%%%%%%%%%%%%%%%%%%%%%%%%%%%%%%%%%%%%%%%%%%%%%
\subsection{Driver Development}
\label{sec:driver}

The \textsf{childdoc} mechanism can also be use for the development
of definition files such as \LaTeX{} styles or classes.
This case differs from the above setup with multiple parts
included by |\include| in that no |\includeonly| should be invoked.
This can be achieved by starting the include file
(before |\ProvidesPackage|) with:
%
\begin{center}
\begin{tabular}{l}
|\input{childdoc.def}|\\
|\childdocforward{|\textit{main}|}|\\
\end{tabular}
\end{center}
%
or alternatively with:
%
\begin{center}
\begin{tabular}{l}
|\input{childdoc.def}|\\
|\childdocby{|\textit{main}|}|\\
\end{tabular}
\end{center}
%
Both forms have slightly different effects as described above.
The main file is prepared as usual, see \secref{sec:include}.

%%%%%%%%%%%%%%%%%%%%%%%%%%%%%%%%%%%%%%%%%%%%%%%%%%%%%%%%%%%%%%%%%%%%%%%%%%%%%%%%
\subsection{Legacy Detection}
\label{sec:detection}

The directive |\childdocmain| in the main file can detect
whether the complete document or merely a child is to be compiled
even without using the directive |\childdocof|.
This method is deprecated because it is less robust
and there is no compelling reason to use it;
it is merely provided for backward compatibility
and it may be removed in future versions.

If the detection mechanism is to be used,
it is mandatory to correctly specify
the filename of the main file as the argument of |\childdocmain|:
%
\begin{center}
\begin{tabular}{l}
|\input{childdoc.def}|\\
|\childdocmain{|\textit{main}|}|\\
\end{tabular}
\end{center}
%
If |\jobname| does not match the argument \textit{main} of |\childdocmain|,
it is assumed that |\jobname| points to the child file to be compiled.
When using |\childdocmain| with the main file specified as argument,
it suffices to start a child file
with just |\input{|\textit{main}|}|
without loading of the package and using |\childdocof|.
If instead all processing is done
with the appropriate \textsf{childdoc} directives,
the argument of \textit{main} of |\childdocmain| can be empty.

An alternative version of the command line processing described
in \secref{sec:commandline} using the detection mechanism reads:
%
\begin{center}
|... -jobname "|\textit{target}|" "|[\textit{flags}]%
[|\def\jobname{|\textit{dest}|}|]|\input{|\textit{main}|}"|
\end{center}

%%%%%%%%%%%%%%%%%%%%%%%%%%%%%%%%%%%%%%%%%%%%%%%%%%%%%%%%%%%%%%%%%%%%%%%%%%%%%%%%
\subsection{Manual Code}
\label{sec:manual}

In case one cannot be certain whether the definitions file |childdoc.def|
is installed on the target \TeX{} distribution
and one prefers not to ship it,
it is conceivable to paste a few relevant commands into the sources.

To that end, drop all statements |\input{childdoc.def}|
and perform the replacements as outlined below.
Instead of |\childdocmain{|\textit{main}|}| add the following code
to the top of the main file:
%
\begin{center}
\begin{tabular}{l}
|\||ifdefined\childdocname\endinput\||fi\newif\ifchilddoc|\\
|\edef\childdocname{\scantokens\expandafter{\jobname\noexpand}}|\\
|\def\childdocmain{|\textit{main}|}\||ifx\childdocmain\childdocname\||else|\\
|\childdoctrue\includeonly{\childdocname}\let\jobname\childdocmain\||fi|\\
\end{tabular}
\end{center}
%
Instead of |\childdocof{|\textit{main}|}| just include the main file
at the top of each child file:
%
\begin{center}
|\input{|\textit{main}|}|
\end{center}
%
A simple redirection |\childdocforward{|\textit{dest}|}| is achieved by:
%
\begin{center}
|\def\jobname{|\textit{dest}|}\input{\jobname}|
\end{center}
%
The redirection with prefix
|\childdocforwardprefix[|\textit{prefix}|]{|\textit{dest}|}|
is accomplished by:
%
\begin{center}
\begin{tabular}{l}
|{\edef\jobname{\scantokens\expandafter{\jobname\noexpand}}|\\
|\def\redirectjob |\textit{prefix}|#1~~~{\gdef\jobname{|\textit{dest}|#1}}|\\
|\expandafter\redirectjob\jobname~~~}\input{\jobname}|
\end{tabular}
\end{center}

In an alternative approach,
child documents can be compiled by a specific command line
without additional code or specific definitions:
%
\begin{center}
|... -jobname "|\textit{target}|" "|[\textit{flags}]%
|\includeonly{|\textit{dest}|}\input{|\textit{main}|}"|
\end{center}
%

%%%%%%%%%%%%%%%%%%%%%%%%%%%%%%%%%%%%%%%%%%%%%%%%%%%%%%%%%%%%%%%%%%%%%%%%%%%%%%%%
%%%%%%%%%%%%%%%%%%%%%%%%%%%%%%%%%%%%%%%%%%%%%%%%%%%%%%%%%%%%%%%%%%%%%%%%%%%%%%%%
\section{Information}

%%%%%%%%%%%%%%%%%%%%%%%%%%%%%%%%%%%%%%%%%%%%%%%%%%%%%%%%%%%%%%%%%%%%%%%%%%%%%%%%
\subsection{Copyright}

Copyright \copyright{} 2017--2018 Niklas Beisert

This work may be distributed and/or modified under the
conditions of the \LaTeX{} Project Public License, either version 1.3
of this license or (at your option) any later version.
The latest version of this license is in
  \url{http://www.latex-project.org/lppl.txt}
and version 1.3 or later is part of all distributions of \LaTeX{}
version 2005/12/01 or later.

This work has the LPPL maintenance status `maintained'.

The Current Maintainer of this work is Niklas Beisert.

This work consists of the files |README.txt|, |childdoc.ins| and |childdoc.dtx|
as well as the derived files |childdoc.def|, |cdocsamp.tex|
with |cdocsch1.tex|, |cdocsch2.tex|, |cdocspt3.tex|, |cdocspt4.tex|,
|cdocsdrf.tex|, |cdocsfn1.tex|, |cdocsfn2.tex|
as well as |childdoc.pdf|.

%%%%%%%%%%%%%%%%%%%%%%%%%%%%%%%%%%%%%%%%%%%%%%%%%%%%%%%%%%%%%%%%%%%%%%%%%%%%%%%%
\subsection{Files and Installation}

The package consists of the files:
%
\begin{center}
\begin{tabular}{ll}
    |README.txt|   & readme file \\
    |childdoc.ins| & installation file \\
    |childdoc.dtx| & source file \\
    |childdoc.def| & definition file \\
    |cdocsamp.tex| & sample main file \\
    |cdocsch1.tex| & sample include file \\
    |cdocsch2.tex| & sample include file \\
    |cdocspt3.tex| & sample part file \\
    |cdocspt4.tex| & sample part file \\
    |cdocsdrf.tex| & sample redirection file \\
    |cdocsfn1.tex| & sample redirection file \\
    |cdocsfn2.tex| & sample redirection file \\
    |childdoc.pdf| & manual
\end{tabular}
\end{center}
%
The distribution consists of the files
|README.txt|, |childdoc.ins| and |childdoc.dtx|.
%
\begin{itemize}
\item
Run (pdf)\LaTeX{} on |childdoc.dtx|
to compile the manual |childdoc.pdf| (this file).
\item
Run \LaTeX{} on |childdoc.ins| to create the definitions file |childdoc.def|
and the sample |cdocsamp.tex| with include files
|cdocsch1.tex|, |cdocsch2.tex|, |cdocspt3.tex|, |cdocspt4.tex|,
|cdocsdrf.tex|, |cdocsfn1.tex|, |cdocsfn2.tex|.
Then copy the file |childdoc.def| to an appropriate directory of your \LaTeX{}
distribution, e.g.\ \textit{texmf-root}|/tex/latex/childdoc|.
\end{itemize}

%%%%%%%%%%%%%%%%%%%%%%%%%%%%%%%%%%%%%%%%%%%%%%%%%%%%%%%%%%%%%%%%%%%%%%%%%%%%%%%%
\subsection{Related CTAN Packages}

There are several other packages which offer a similar functionality:
%
\begin{itemize}
\item
The packages
\href{http://ctan.org/pkg/docmute}{\textsf{docmute}},
\href{http://ctan.org/pkg/includex}{\textsf{includex}} and
\href{http://ctan.org/pkg/standalone}{\textsf{standalone}}
provide commands to include only the document body of
a child file thus allowing both files to be compiled individually.
\item
The packages \href{http://ctan.org/pkg/subdocs}{\textsf{subdocs}}
and \href{http://ctan.org/pkg/subfiles}{\textsf{subfiles}}
provide structures in which the main and child documents can be
encapsulated and allowing them to be compiled individually.
The inclusion mechanism is different from the conventional |\include|.
\item
The package \href{http://ctan.org/pkg/combine}{\textsf{combine}}
is an elaborate solution to combine several documents into one.
\end{itemize}
%
See also the CTAN topic \href{http://ctan.org/topic/subdocs}{\textsf{subdocs}}
for further related packages.
The present package differs from the above solutions in that
a document structure constructed with the conventional |\include| mechanism
just needs two extra commands at the top of every file
such that all constituent files can be compiled individually.

%%%%%%%%%%%%%%%%%%%%%%%%%%%%%%%%%%%%%%%%%%%%%%%%%%%%%%%%%%%%%%%%%%%%%%%%%%%%%%%%
%\subsection{Feature Suggestions}
%
%The following is a list of features which may be useful for future
%versions of this package:
%%
%\begin{itemize}
%\item
%\ldots
%\end{itemize}

%%%%%%%%%%%%%%%%%%%%%%%%%%%%%%%%%%%%%%%%%%%%%%%%%%%%%%%%%%%%%%%%%%%%%%%%%%%%%%%%
\subsection{Revision History}

%%%%%%%%%%%%%%%%%%%%%%%%%%%%%%%%%%%%%%%%
\paragraph{v2.0:} 2018/12/30

\begin{itemize}
\item
immediate forward processing
\item
added |\childdocby| mechanism
\item
manual restructured
\end{itemize}

%%%%%%%%%%%%%%%%%%%%%%%%%%%%%%%%%%%%%%%%
\paragraph{v1.6:} 2018/01/17

\begin{itemize}
\item
application for development of include files
\item
corrections to manual
\end{itemize}

%%%%%%%%%%%%%%%%%%%%%%%%%%%%%%%%%%%%%%%%
\paragraph{v1.5:} 2017/05/21

\begin{itemize}
\item
more complete structuring introduced
\item
|\childdocof| introduced
\item
|\childdoc| renamed to |\childdocmain|
\item
|\childredirect| renamed to |\childdocforward| and |\childdocforwardprefix|
and functionality expanded
\end{itemize}

%%%%%%%%%%%%%%%%%%%%%%%%%%%%%%%%%%%%%%%%
\paragraph{v1.0:} 2017/04/27

\begin{itemize}
\item
manual and install package
\item
first version published on CTAN
\end{itemize}

%%%%%%%%%%%%%%%%%%%%%%%%%%%%%%%%%%%%%%%%
\paragraph{v0.6:} 2017/04/26

\begin{itemize}
\item
redirection mechanism added
\end{itemize}

%%%%%%%%%%%%%%%%%%%%%%%%%%%%%%%%%%%%%%%%
\paragraph{v0.5:} 2017/04/26

\begin{itemize}
\item
functionality in definition file
\end{itemize}


%%%%%%%%%%%%%%%%%%%%%%%%%%%%%%%%%%%%%%%%%%%%%%%%%%%%%%%%%%%%%%%%%%%%%%%%%%%%%%%%
%%%%%%%%%%%%%%%%%%%%%%%%%%%%%%%%%%%%%%%%%%%%%%%%%%%%%%%%%%%%%%%%%%%%%%%%%%%%%%%%
%%%%%%%%%%%%%%%%%%%%%%%%%%%%%%%%%%%%%%%%%%%%%%%%%%%%%%%%%%%%%%%%%%%%%%%%%%%%%%%%
\appendix

\settowidth\MacroIndent{\rmfamily\scriptsize 000\ }

 \DocInput{childdoc.dtx}

\end{document}
%</driver>
% \fi
%
% %%%%%%%%%%%%%%%%%%%%%%%%%%%%%%%%%%%%%%%%%%%%%%%%%%%%%%%%%%%%%%%%%%%%%%%%%%%%%%
% %%%%%%%%%%%%%%%%%%%%%%%%%%%%%%%%%%%%%%%%%%%%%%%%%%%%%%%%%%%%%%%%%%%%%%%%%%%%%%
% \section{Sample}
%\iffalse
%<*samplemain>
%\fi
%
% The following presents a sample document
% with two chapters, two parts, a title page,
% a compile flag as well as three forwarding files to set the flag.
% It consists of eight |.tex| files:
% \begin{center}
% \begin{tabular}{ll}
% |cdocsamp.tex|&main file\\
% |cdocsch1.tex|&include file for chapter 1\\
% |cdocsch2.tex|&include file for chapter 2\\
% |cdocspt3.tex|&include file for part 3\\
% |cdocspt4.tex|&include file for part 4\\
% |cdocsdrf.tex|&forwarding file for main file in draft mode\\
% |cdocsfi1.tex|&forwarding file for final version of chapter 1\\
% |cdocsfi2.tex|&forwarding file for final version of chapter 2\\
% \end{tabular}
% \end{center}
% Each of the eight files can be compiled directly by the \LaTeX{} compiler.
%
% %%%%%%%%%%%%%%%%%%%%%%%%%%%%%%%%%%%%%%
% \paragraph{Main File.}
%
% The main file is called |cdocsamp.tex|.
%
% Load the \textsf{childdoc} definitions and
% declare the filename for the main document:
%    \begin{macrocode}
\input{childdoc.def}
\childdocmain{}
%    \end{macrocode}

% Optional override for |\version| flag:
%    \begin{macrocode}
%%\ifchilddoc\else\providecommand{\version}{draft}\fi
%    \end{macrocode}

% Define the default values for the |\version| flag
% (|final| for the main file and |draft| for childs):
%    \begin{macrocode}
\ifchilddoc
\providecommand{\version}{draft}
\else
\providecommand{\version}{final}
\fi
%    \end{macrocode}

% Load the standard document class:
%    \begin{macrocode}
\documentclass[12pt]{article}
%    \end{macrocode}

% Start the document body:
%    \begin{macrocode}
\begin{document}
%    \end{macrocode}

% Declare a title page.
% Print title, part of document being processed and version flag:
%    \begin{macrocode}
\addtocounter{page}{-1}
\begin{center}
{\LARGE\bfseries{}childdoc example\par}
\vspace{1cm}
\ifchilddoc
\ifchilddocmanual part\else chapter\fi:
`\childdocname' of `\childdocjob'\par
\else
main document: `\childdocjob'\par
\fi
version: \version\par
\end{center}
\newpage
%    \end{macrocode}

% Manually include selected file,
% otherwise process as usual:
%    \begin{macrocode}
\ifchilddocmanual
\section*{part `\childdocname'}
\input{\childdocname}
\else
%    \end{macrocode}

% Include the two chapters:
%    \begin{macrocode}
\include{cdocsch1}
\include{cdocsch2}
%    \end{macrocode}

% Include the two parts unless only chapters should be displayed:
%    \begin{macrocode}
\ifchilddoc\else
\section{part three}
\input{cdocspt3}
\section{part four}
\input{cdocspt4}
\fi
%    \end{macrocode}

% Process as usual until here:
%    \begin{macrocode}
\fi
%    \end{macrocode}

% End of document body:
%    \begin{macrocode}
\end{document}
%    \end{macrocode}
%\iffalse
%</samplemain>
%\fi
%
% %%%%%%%%%%%%%%%%%%%%%%%%%%%%%%%%%%%%%%
% \paragraph{Chapter Include Files.}
%
% The include files are called |cdocsch1.tex| and |cdocsch2.tex|.
%
%\iffalse
%<*samplechap1|samplechap2>
%\fi

% Optional override for |\version| flag:
%    \begin{macrocode}
%%\providecommand{\version}{final}
%    \end{macrocode}

% Include the main document:
%    \begin{macrocode}
\input{childdoc.def}
\childdocof{cdocsamp}
%    \end{macrocode}

%\iffalse
%</samplechap1|samplechap2>
%\fi
%
%\iffalse
%<*samplechap1>
%\fi
% Some text for chapter 1:
%    \begin{macrocode}
\section{one}
some text in chapter one
%    \end{macrocode}

%\iffalse
%</samplechap1>
%\fi
% Some text for chapter 2:
%\iffalse
%<*samplechap2>
%\fi
%    \begin{macrocode}
\section{two}
more text in chapter two
%    \end{macrocode}

%\iffalse
%</samplechap2>
%\fi
%
% %%%%%%%%%%%%%%%%%%%%%%%%%%%%%%%%%%%%%%
% \paragraph{Part Include Files.}
%
% The include files are called |cdocspt3.tex| and |cdocspt4.tex|.
%
%\iffalse
%<*samplepart3|samplepart4>
%\fi

% Optional override for |\version| flag:
%    \begin{macrocode}
%%\providecommand{\version}{final}
%    \end{macrocode}

% Include the main document:
%    \begin{macrocode}
\input{childdoc.def}
\childdocby{cdocsamp}
%    \end{macrocode}

%\iffalse
%</samplepart3|samplepart4>
%\fi
%
%\iffalse
%<*samplepart3>
%\fi
% Some text for part 3:
%    \begin{macrocode}
some text in part three
%    \end{macrocode}

%\iffalse
%</samplepart3>
%\fi
% Some text for part 4:
%\iffalse
%<*samplepart4>
%\fi
%    \begin{macrocode}
more text in part four
%    \end{macrocode}

%\iffalse
%</samplepart4>
%\fi
%
% %%%%%%%%%%%%%%%%%%%%%%%%%%%%%%%%%%%%%%
% \paragraph{Forwarding for a Complete Draft.}
%
% The following forwarding file |cdocsdrf.tex|
% compiles the main document in draft mode:
%\iffalse
%<*sampledraft>
%\fi
%    \begin{macrocode}
\def\version{draft}
\input{childdoc.def}
\childdocforward{cdocsamp}
%    \end{macrocode}

%\iffalse
%</sampledraft>
%\fi
%
% %%%%%%%%%%%%%%%%%%%%%%%%%%%%%%%%%%%%%%
% \paragraph{Forwarding for Final Version of the Chapters.}
%
% The following forwarding files |cdocsfn1.tex| and |cdocsfn2.tex|
% (with identical content)
% compile the final versions of the child documents
% |cdocsch1.tex| and |cdocsch2.tex|, respectively:
%\iffalse
%<*samplefinal>
%\fi
%    \begin{macrocode}
\def\version{final}
\input{childdoc.def}
\childdocforwardprefix[cdocsamp]{cdocsfn}{cdocsch}
%    \end{macrocode}

%\iffalse
%</samplefinal>
%\fi
%
% %%%%%%%%%%%%%%%%%%%%%%%%%%%%%%%%%%%%%%
% \paragraph{Command Line Processing.}
%
% The following three command lines generate the output files
% |cdocscld|, |cdocscl1| and |cdocscl2|
% which should be identical to
% |cdocsdrf|, |cdocsch1| and |cdocsfn2|, respectively:
% \begin{center}
% \begin{tabular}{l}
% |latex -jobname cdocscld \|\\
% |  "\def\version{draft}\input{childdoc.def}\childdocforward{cdocsamp}"|\\
% |latex -jobname cdocscl1 \|\\
% |  "\input{childdoc.def}\childdocforward[cdocsamp]{cdocsch1}"|\\
% |latex -jobname cdocscl2 \|\\
% |  "\def\version{final}\input{childdoc.def}\childdocforward{cdocsch2}"|
% \end{tabular}
% \end{center}
% Note that the trailing backslash on each first line
% merely continues the input to the second line
% (for convenient cut ant paste).
% Furthermore, the command |latex| can be replaced by any
% of its alternative versions such as |pdflatex|.
%
% %%%%%%%%%%%%%%%%%%%%%%%%%%%%%%%%%%%%%%%%%%%%%%%%%%%%%%%%%%%%%%%%%%%%%%%%%%%%%%
% %%%%%%%%%%%%%%%%%%%%%%%%%%%%%%%%%%%%%%%%%%%%%%%%%%%%%%%%%%%%%%%%%%%%%%%%%%%%%%
% \section{Implementation}
%\iffalse
%<*package>
%\fi
%
% This section describes the definitions file |childdoc.def|.

% The definitions cannot be loaded using |\usepackage| or |\RequirePackage|
% which has a mechanism to prevent loading a style file more than once.
% When loading the definitions by means of |\input|
% multiple instances have to be prevented manually:
%\iffalse
%This code needs to be before the `\ProvidesFile' directive
%which is defined at the beginning of this file.
%Therefore it is also placed there and commented out here.
%</package>
%<*discard>
%\fi
%    \begin{macrocode}
\ifdefined\childdocmain\endinput\fi
%    \end{macrocode}
%\iffalse
%</discard>
%<*package>
%\fi
%
% \macro{\ifchilddoc}
% \macro{\ifchilddocmanual}
% The conditional |\ifchilddoc| tells whether a
% child (true) or main (false) document is being compiled.
% The conditional |\ifchilddocmanual| tells whether
% the |\includeonly| mechanism is used (false) or
% the selection of child files must be performed manually (true).
% The definitions initialise to false:
%    \begin{macrocode}
\newif\ifchilddoc
\newif\ifchilddocmanual
%    \end{macrocode}

% \macro{\childdocname}
% \macro{\childdocjob}
% The macro |\childdocname| stores the name of the main document
% to be compiled. The macro |\childdocjob| stores the name of
% the document on which the \LaTeX{} compiler was originally invoked.
% The content of |\jobname| cannot be compared
% to filenames specified in the source due to different catcodes.
% The following code rescans |\jobname|, stores the result
% in |\childdocname| and saves a copy in |\childdocjob|:
%    \begin{macrocode}
\edef\childdocname{\scantokens\expandafter{\jobname\noexpand}}
\let\childdocjob\childdocname
%    \end{macrocode}

% \macro{\childdocdisable}
% The macro |\childdocdisable| prevents the main file
% from being processed more than once.
% At this stage, the main document command |\childdocmain|
% is assumed to be called once again where it should do nothing.
% Any subsequent call to it should prevent
% a secondary processing of the main document
% It overwrites the forwarding commands
% |\childdocof| and |\childdocforward|
% with empty macros to prevent further inclusions of the main document:
%    \begin{macrocode}
\newcommand{\childdocdisable}
{
  \renewcommand{\childdocmain}[1]{\renewcommand{\childdocmain}[1]{\endinput}}
  \renewcommand{\childdocof}[1]{}
  \renewcommand{\childdocby}[2][]{}
  \renewcommand{\childdocforward}[2][]{}
  \renewcommand{\childdocdisable}{}
}
%    \end{macrocode}

% \macro{\childdocmain}
% The macro |\childdocmain| is to be called at the top of the main file
% with nothing or the main filename (without extension) as argument.
% First, it breaks loops.
% If the argument is not empty and does not match |\childdocname|
% (which is set by the first inclusion of |childdoc.def|),
% |\ifchilddoc| is set to true, |\includeonly| is applied to the child file
% and |\jobname| is set to the main file
% (for proper handling of |.aux| files):
%    \begin{macrocode}
\newcommand{\childdocmain}[1]
{
  \childdocdisable\childdocmain{}
  \if?#1?\else
    \begingroup
      \def\childdoctmp{#1}
      \ifx\childdoctmp\childdocname
        \def\childdoctmp{}
      \else
        \def\childdoctmp
        {
          \childdoctrue
          \includeonly{\childdocname}
          \def\childdocjob{#1}
          \def\jobname{#1}
        }
      \fi
      \expandafter
    \endgroup
    \childdoctmp
  \fi
}
%    \end{macrocode}

% \macro{\childdocof}
% The command |\childdocof| redirects
% compilation to the main file |#1|.
%    \begin{macrocode}
\newcommand{\childdocof}[1]
{
  \childdocdisable
  \childdoctrue
  \includeonly{\childdocname}
  \def\jobname{#1}
  \def\childdocjob{#1}
  \input{#1}
}
%    \end{macrocode}

% \macro{\childdocby}
% The command |\childdocby| ....
%    \begin{macrocode}
\newcommand{\childdocby}[2][]
{
  \childdocdisable
  \childdoctrue
  \childdocmanualtrue
  \if?#1?\else
    \def\jobname{#2}
  \fi
  \def\childdocjob{#2}
  \input{#2}
  \endinput
}
%    \end{macrocode}

% \macro{\childdocforward}
% The command |\childdocforward| redirects
% compilation to the main file or
% (if the optional argument is given) a child file.
% Parameters are set as if the main file
% or a child file starting with |\childdocof| was compiled.
% Then compilation is handed over to the main file:
%    \begin{macrocode}
\newcommand{\childdocforward}[2][]
{
  \begingroup
    \if?#1?
      \def\childdoctmp
      {
        \def\childdocname{#2}
        \def\childdocjob{#2}
        \def\jobname{#2}
        \input{#2}
        \endinput
      }
    \else
      \def\childdoctmp
      {
        \childdocdisable
        \def\childdocname{#2}
        \childdoctrue
        \includeonly{#2}
        \def\childdocjob{#1}
        \def\jobname{#1}
        \input{#1}
        \endinput
      }
    \fi
    \expandafter
  \endgroup
  \childdoctmp
}
%    \end{macrocode}

% \macro{\childdocforwardprefix}
% The command |\childdocforwardprefix| redirects
% compilation to the main or a child file by means of a pattern.
% The prefix |#1| in the current filename is replaced by |#2|
% and the suffix of the current filename is kept
% (it is assumed that the filename does not contain the substring `|~~~|'
% which is used as a delimiter).
% Compilation is handed over to the new file by |\childdocforward|:
%    \begin{macrocode}
\newcommand{\childdocforwardprefix}[3][]
{
  \begingroup
    \def\childdocextract #2##1~~~{\def\childdoctmp{\childdocforward[#1]{#3##1}}}
    \expandafter\childdocextract\childdocname~~~
    \expandafter
  \endgroup
  \childdoctmp
}
%    \end{macrocode}

% \macro{\childdoc}
% The deprecated macro |\childdoc| is a legacy version of |\childdocmain|:
%    \begin{macrocode}
\newcommand{\childdoc}{\childdocmain}
%    \end{macrocode}

% \macro{\childdocredirect}
% The deprecated macro |\childdocredirect| is a legacy version
% of |\childdocforward| and |\childdocforwardprefix|:
%    \begin{macrocode}
\newcommand{\childdocredirect}[2][]
{
  \begingroup
    \if?#1?
      \def\childdoctmp{\childdocforward{#2}}
    \else
      \def\childdoctmp{\childdocforwardprefix{#1}{#2}}
    \fi
    \expandafter
  \endgroup
  \childdoctmp
}
%    \end{macrocode}

%\iffalse
%</package>
%\fi
%
\endinput
|\\
|\childdocby{|\textit{main}|}|\\
\end{tabular}
\end{center}
%
Both forms have slightly different effects as described above.
The main file is prepared as usual, see \secref{sec:include}.

%%%%%%%%%%%%%%%%%%%%%%%%%%%%%%%%%%%%%%%%%%%%%%%%%%%%%%%%%%%%%%%%%%%%%%%%%%%%%%%%
\subsection{Legacy Detection}
\label{sec:detection}

The directive |\childdocmain| in the main file can detect
whether the complete document or merely a child is to be compiled
even without using the directive |\childdocof|.
This method is deprecated because it is less robust
and there is no compelling reason to use it;
it is merely provided for backward compatibility
and it may be removed in future versions.

If the detection mechanism is to be used,
it is mandatory to correctly specify
the filename of the main file as the argument of |\childdocmain|:
%
\begin{center}
\begin{tabular}{l}
|% \iffalse
%
% childdoc.dtx Copyright (C) 2017-2018 Niklas Beisert
%
% This work may be distributed and/or modified under the
% conditions of the LaTeX Project Public License, either version 1.3
% of this license or (at your option) any later version.
% The latest version of this license is in
%   http://www.latex-project.org/lppl.txt
% and version 1.3 or later is part of all distributions of LaTeX
% version 2005/12/01 or later.
%
% This work has the LPPL maintenance status `maintained'.
%
% The Current Maintainer of this work is Niklas Beisert.
%
% This work consists of the files childdoc.dtx and childdoc.ins
% and the derived files childdoc.def and cdocsamp.tex with
% cdocsch1.tex, cdocsch2.tex, cdocsdrf.tex, cdocsfn1.tex, cdocsfn2.tex.
%
%<package>\ifdefined\childdocmain\endinput\fi
%<package>\ProvidesFile{childdoc.def}[2018/12/30 v2.0 child document driver]
%<samplemain>\ProvidesFile{cdocsamp.tex}[2018/12/30 v2.0 sample for childdoc]
%<*driver>
%\ProvidesFile{childdoc.drv}[2018/12/30 v2.0 childdoc reference manual file]
\PassOptionsToClass{10pt,a4paper}{article}
\documentclass{ltxdoc}

\usepackage[margin=35mm]{geometry}
\usepackage{hyperref}
\usepackage{hyperxmp}
\usepackage[usenames]{color}

\hypersetup{colorlinks=true}
\hypersetup{pdfstartview=FitH}
\hypersetup{pdfpagemode=UseNone}
\hypersetup{pdfsource={}}
\hypersetup{pdflang={en-UK}}
\hypersetup{pdfcopyright={Copyright 2017-2018 Niklas Beisert.
  This work may be distributed and/or modified under the
  conditions of the LaTeX Project Public License, either version 1.3
  of this license or (at your option) any later version.}}
\hypersetup{pdflicenseurl={http://www.latex-project.org/lppl.txt}}
\hypersetup{pdfcontactaddress={ETH Zurich, ITP, HIT K,
  Wolfgang-Pauli-Strasse 27}}
\hypersetup{pdfcontactpostcode={8093}}
\hypersetup{pdfcontactcity={Zurich}}
\hypersetup{pdfcontactcountry={Switzerland}}
\hypersetup{pdfcontactemail={nbeisert@itp.phys.ethz.ch}}
\hypersetup{pdfcontacturl={http://people.phys.ethz.ch/\xmptilde nbeisert/}}

\newcommand{\secref}[1]{\hyperref[#1]{section \ref*{#1}}}

\parskip1ex
\parindent0pt
\let\olditemize\itemize
\def\itemize{\olditemize\parskip0pt}

\begin{document}

\title{The \textsf{childdoc} Package}
\hypersetup{pdftitle={The childdoc Package}}
\author{Niklas Beisert\\[2ex]
  Institut f\"ur Theoretische Physik\\
  Eidgen\"ossische Technische Hochschule Z\"urich\\
  Wolfgang-Pauli-Strasse 27, 8093 Z\"urich, Switzerland\\[1ex]
  \href{mailto:nbeisert@itp.phys.ethz.ch}
  {\texttt{nbeisert@itp.phys.ethz.ch}}}
\hypersetup{pdfauthor={Niklas Beisert}}
\hypersetup{pdfsubject={Manual for the LaTeX2e Package childdoc}}
\date{30 December 2018, \textsf{v2.0}}
\maketitle

\begin{abstract}\noindent
\textsf{childdoc} is a \LaTeXe{} package
that enables the direct compilation
of document sections included by |\include|
to individual files.
\end{abstract}

\begingroup
\parskip0ex
\tableofcontents
\endgroup

%%%%%%%%%%%%%%%%%%%%%%%%%%%%%%%%%%%%%%%%%%%%%%%%%%%%%%%%%%%%%%%%%%%%%%%%%%%%%%%%
%%%%%%%%%%%%%%%%%%%%%%%%%%%%%%%%%%%%%%%%%%%%%%%%%%%%%%%%%%%%%%%%%%%%%%%%%%%%%%%%
\section{Introduction}

\LaTeX{} provides a mechanism to structure a large document (such as a book)
into a main file and several child files (containing the chapters)
using the |\include| command.
This mechanism is beneficial for documents
which span hundreds of pages in order to
make the source file(s) more manageable.
Moreover, compilation can be restricted to
selected child files by means of the |\includeonly| command.
The latter feature can be used to reduce the compilation time while editing
(this was significantly more useful in the earlier days of \LaTeX{})
or to generate a smaller document which is easier to navigate.
Another application of |\includeonly| is to generate
documents consisting of selected parts of the complete document.

However, there are a few drawbacks of the plain |\include| mechanism:
\begin{itemize}
\item
The child files cannot be compiled on their own,
they can only be compiled via the main file.
A naive editing environment
(such as a text editor with an option
to have the current file processed by \LaTeX)
may require one to switch to the main file before compiling;
attempting to compile the child file produces errors.
\item
The main file must be modified (each time)
to adjust the |\includeonly| command
to the present needs. This easily leaves the main file in a messy state.
\item
The generated document will always carry the filename
of the main document. This is inconvenient if
several child files are to be compiled and
to be kept for distribution.
\end{itemize}

The present package provides a simple interface
to make child files individually compilable by \LaTeX{}.
Compiling a child file then has the same effect as compiling
the main file with an |\includeonly| command
to select the appropriate child.
Moreover the generated document will carry the name of the child
rather than the main file.
This resolves all three above issues.

This feature is meant to make the editing of books,
thesis documents and lecture notes somewhat more convenient.
However, the package can also be used efficiently for
composing a series of documents (such as exercise sheets)
which are typically distributed individually.
It then assists the author in generating the individual documents
(potentially in different versions)
as well as a document containing the collected series.
Another application is in developing style files
or other kinds of included material
where compilation of the style file could redirect
to a sample or test file.

%%%%%%%%%%%%%%%%%%%%%%%%%%%%%%%%%%%%%%%%%%%%%%%%%%%%%%%%%%%%%%%%%%%%%%%%%%%%%%%%
%%%%%%%%%%%%%%%%%%%%%%%%%%%%%%%%%%%%%%%%%%%%%%%%%%%%%%%%%%%%%%%%%%%%%%%%%%%%%%%%
\section{Usage}

First of all, the package \textsf{childdoc} is \emph{not} a standard
\LaTeXe{} |.sty| style file! Therefore it needs to be invoked in
a non-standard way.

%%%%%%%%%%%%%%%%%%%%%%%%%%%%%%%%%%%%%%%%%%%%%%%%%%%%%%%%%%%%%%%%%%%%%%%%%%%%%%%%
\subsection{Included Files}
\label{sec:include}

%%%%%%%%%%%%%%%%%%%%%%%%%%%%%%%%%%%%%%%%
\DescribeMacro{\childdocmain}
To use the package, add the commands
\begin{center}
\begin{tabular}{l}
|\input{childdoc.def}|\\
|\childdocmain{}|\\
\end{tabular}
\end{center}
at the very top of the main \LaTeX{} file,
in particular \emph{before} the |\documentclass| statement!
The argument of |\childdocmain| should be left empty
(but it must be present).

%%%%%%%%%%%%%%%%%%%%%%%%%%%%%%%%%%%%%%%%
\DescribeMacro{\childdocof}
Furthermore, add the commands
\begin{center}
\begin{tabular}{l}
|\input{childdoc.def}|\\
|\childdocof{|\textit{main}|}|\\
\end{tabular}
\end{center}
at the top of every child file \textit{child}
which is included by |\include{|\textit{child}|}|
from within the main file
(or at least for those files to be compiled individually).
The argument \textit{main} must be the filename of the main file.

There are a couple of
considerations in setting up the main and child documents:

%%%%%%%%%%%%%%%%%%%%%%%%%%%%%%%%%%%%%%%%
\paragraph{Restrictions.}

Please note the following restrictions:
\begin{itemize}
\item
|\childdocmain| must be called with one argument \textit{main}
to ensure compatibility with earlier version of the package.
It must either be empty (|\childdocmain{}|)
or precisely match the filename of the main file in which it is specified.
See \secref{sec:detection} for further information.
\item
The filename \textit{main} must be specified without the |.tex| extension.
\item
The filename \textit{main} is case sensitive
(even in case-insensitive file systems)
due to internal string comparison.
\item
The argument \textit{main} should be fully expanded, it cannot be a macro.
\item
Subdirectories and special characters should be avoided in filenames.
\item
The command |\childdocmain{|\textit{main}|}| must be followed by a whitespace.
It should not be followed immediately by another command
or by a comment mark `|%|'.
This is because the \TeX{} parser reads the token immediately following
the argument of |\childdocmain| and puts it
at the beginning of every child section;
however, a white\-space is ignored.
\end{itemize}

%%%%%%%%%%%%%%%%%%%%%%%%%%%%%%%%%%%%%%%%
\paragraph{Content of Main File.}

It is advisable to place all content in the child files included by |\include|.
Any output contained in the main file will appear in all child documents
unless suppressed manually;
it cannot be suppressed automatically by the |\includeonly| directive
and thus should normally be avoided.
A method to include some content in the main file
by means of conditional processing is described in \secref{sec:conditional}.

%%%%%%%%%%%%%%%%%%%%%%%%%%%%%%%%%%%%%%%%
\paragraph{Page Numbering.}

When only a part of the document is compiled,
the appropriate numbering of pages
(as well as other status parameters)
is determined from the |.aux| files.
The latter contain information from previous passes.
However this information needs to propagate through
all intermediate child documents.
Therefore the page numbering in child documents may well
be inconsistent until the complete document is compiled at least once.

A useful (if unconventional) way to always ensure a consistent
page numbering is to restart the numbering in each child document
and denote the pages by `\textit{child}|.|\textit{page}'
where \textit{child} represents the chapter/section number of the child file.
This can be achieved by the command
|\numberwithin{page}{|\textit{child}|}|
of the \textsf{amsmath} package
where \textit{child} can be |chapter| or |section|
depending on the chosen structuring.
Alternatively, one can modify the macro |\thepage| appropriately
and reset the counter |page| at the start of each child file.

%%%%%%%%%%%%%%%%%%%%%%%%%%%%%%%%%%%%%%%%%%%%%%%%%%%%%%%%%%%%%%%%%%%%%%%%%%%%%%%%
\subsection{Conditional Processing}
\label{sec:conditional}

The package provides a mechanism to compile different versions
of a document. To customise the versions further some conditional processing
can come in handy to distinguish which version is being compiled.
The package provides two macros to describe the compilation context:

%%%%%%%%%%%%%%%%%%%%%%%%%%%%%%%%%%%%%%%%
\DescribeMacro{\ifchilddoc}
The conditional |\ifchilddoc| distinguishes between the compilation of
child documents and the main document:
%
\begin{center}
|\ifchilddoc |\textit{child-code}| |[|\||else |\textit{main-code}]| \||fi|
\end{center}

%%%%%%%%%%%%%%%%%%%%%%%%%%%%%%%%%%%%%%%%
\DescribeMacro{\childdocname}
\DescribeMacro{\childdocjob}
The macro |\childdocname| contains the filename (without extension)
of the main or child file being processed.
Note that |\childdocjob| will always contain the name of the main file.

%%%%%%%%%%%%%%%%%%%%%%%%%%%%%%%%%%%%%%%%
\paragraph{Title Page.}

Conditional processing can be used to include a title or banner page
in the main document when proper precautions are taken.
Importantly, the code in the main file should ensure that the page counter
(as well as other status parameters which are stored in the |.aux| files)
takes the same value after the conditional processing.
Otherwise the page numbers may take divergent values
depending on which part is compiled.

For example, a title page could be declared by:
%
\begin{center}
\begin{tabular}{l}
|\ifchilddoc\||else|\\
|\addtocounter{page}{-1}|\\
\textit{code for title page}\\
|\newpage|\\
|\||fi|
\end{tabular}
\end{center}
%
A banner page for the child documents can be generated by:
%
\begin{center}
\begin{tabular}{l}
|\ifchilddoc|\\
|\addtocounter{page}{-1}|\\
\textit{code for banner page}\\
|\newpage|\\
|\||fi|
\end{tabular}
\end{center}
%
Here one could write a message such as:
\begin{center}
|This is the part \childdocname{} of \childdocjob{}.|
\end{center}

%%%%%%%%%%%%%%%%%%%%%%%%%%%%%%%%%%%%%%%%%%%%%%%%%%%%%%%%%%%%%%%%%%%%%%%%%%%%%%%%
\subsection{Flags}
\label{sec:flags}

The package makes it easy to generate different versions
of the main or child documents.
To this end compilation flags can be defined
and assigned different default values.
They will be particularly useful in conjunction
with the forwarding mechanism described in \secref{sec:forward}.

For example, it may be useful to have a flag |\version|
which can be set to |draft| or |final|.
The document source will contain some conditional code
depending on the value of |\version|.
Suppose further, the flag should default to |final| for the main file
and to |draft| for child files
which is a natural assignment for editing the document.
This is achieved by placing the following code
in the preamble of the main document
(below the |\childdocmain| directive):
%
\begin{center}
\begin{tabular}{l}
|\ifchilddoc|\\
|\providecommand{\version}{draft}|\\
|\||else|\\
|\providecommand{\version}{final}|\\
|\||fi|
\end{tabular}
\end{center}
%
The definition by |\providecommand| makes sure
that previous definitions are not overwritten.
Further statements |\providecommand{\version}{...}|
can thus be added before the above code to override it.

For the main file, one might add a line
(between |\childdocmain| and the above block)
%
\begin{center}
|%\ifchilddoc\||else\providecommand{\version}{draft}\||fi|
\end{center}
%
which can be uncommented to produce a draft version.
Likewise one can add a line to the very top of a child file
(above the |\childdocof{|\textit{main}|}| directive)
%
\begin{center}
|%\providecommand{\version}{final}|
\end{center}
%
which can be uncommented to produce the final version of this child document.

%%%%%%%%%%%%%%%%%%%%%%%%%%%%%%%%%%%%%%%%%%%%%%%%%%%%%%%%%%%%%%%%%%%%%%%%%%%%%%%%
\subsection{Forwarding}
\label{sec:forward}

Different versions of the main or child documents
using compilation flags as described in \secref{sec:flags}
can be (permanently) stored in different files
for convenient compilation, viewing and distribution.
To this end, the package defines a command
to pass on compilation to a different file:

%%%%%%%%%%%%%%%%%%%%%%%%%%%%%%%%%%%%%%%%
\DescribeMacro{\childdocforward}
The command |\childdocforward| redirects processing to
another source file:
%
\begin{center}
\begin{tabular}{l}
|\input{childdoc.def}|\\
|\childdocforward[|\textit{main}|]{|\textit{dest}|}|\\
\end{tabular}
\end{center}
%
The argument \textit{dest} is the destination file
(without extension).
It should be the main file or one of the child files.
Note that further \textsf{childdoc} directives
such as |\childdocof| and |\childdocforward|
in the indicated file will be processed in this form.
The optional argument \textit{main}
passes on directly to the main file \textit{main}
while pretending to compile the child \textit{dest}.
This form behaves as if \textit{dest}
issues |\childdocof{|\textit{main}|}| right away,
and no further \textsf{childdoc} directives will be processed.

%%%%%%%%%%%%%%%%%%%%%%%%%%%%%%%%%%%%%%%%
\DescribeMacro{\...prefix}
In the alternative form |\childdocforwardprefix|,
%
\begin{center}
\begin{tabular}{l}
|\input{childdoc.def}|\\
|\childdocforwardprefix[|\textit{main}|]{|\textit{prefix}|}{|\textit{dest}|}|
\end{tabular}
\end{center}
%
the destination file is determined by a pattern
depending on the current file:
To make this work, the current file must be called
`{\textit{prefix}\hspace{0.2em}\textit{suffix}}'
with \textit{prefix} matching precisely the argument.
Processing is then passed on to the file
`{\textit{dest}\hspace{0.2em}\textit{suffix}}'.
Surely, the same effect is achieved by
directly specifying the
argument `{\textit{dest}\hspace{0.2em}\textit{suffix}}'
in the first form.
However, that requires to set up a different file
for each child. With the alternative form of the command
all these files can have exactly the same content
which simplifies setting them up and maintaining them.

For example, the following file |draft.tex|
with a compilation flag |\version| as described in \secref{sec:flags}
compiles the main document as a draft:
%
\begin{center}
\begin{tabular}{l}
|\def\version{draft}|\\
|\input{childdoc.def}|\\
|\childdocforward{|\textit{main}|}|
\end{tabular}
\end{center}
%
Likewise, the following files |final|\textit{nn}|.tex|
compile the final version of the child document
|child|\textit{nn}|.tex|:
%
\begin{center}
\begin{tabular}{l}
|\def\version{final}|\\
|\input{childdoc.def}|\\
|\childdocforwardprefix{final}{child}|
\end{tabular}
\end{center}
%

Note that when several versions of a main file and/or of each child file
are to be generated, it may be convenient to set up a |Makefile| or
shell script to automatise the process.

%%%%%%%%%%%%%%%%%%%%%%%%%%%%%%%%%%%%%%%%%%%%%%%%%%%%%%%%%%%%%%%%%%%%%%%%%%%%%%%%
\subsection{Command Line Processing}
\label{sec:commandline}

The effect of redirection files can also be achieved by invoking
the \LaTeX{} compiler with a more elaborate command line.
Most conveniently this should be done as part
of a shell script or a |Makefile|.

When using \textsf{childdoc} in the main file, the following
command lines effectively perform a redirection
(note that depending on the shell being used,
backslashes may have to be doubled: `|\|' $\to$ `|\\|'):
%
\begin{center}
|... -jobname "|\textit{target}|" |\\|"|[\textit{flags}]%
|\input{childdoc.def}\childdocforward[|\textit{main}|]{|\textit{dest}|}"|
\end{center}
%
Here \textit{target} is the name of the output file,
\textit{main} is the name of the main file
and \textit{dest} is the name of the main or child file to be processed
(all filenames without extensions).
The optional argument \textit{main} can be omitted
if \textit{main} matches \textit{dest}.
Optionally, compilation \textit{flags} can be defined via |\def| commands.
This command line makes the \TeX{} engine believe
it is compiling the file \textit{target}
whose content is specified as the latter parameter.
The provided code then forwards the processing to
\textit{main} or \textit{dest} as described in \secref{sec:forward}.

%%%%%%%%%%%%%%%%%%%%%%%%%%%%%%%%%%%%%%%%%%%%%%%%%%%%%%%%%%%%%%%%%%%%%%%%%%%%%%%%
\subsection{Include by Input}
\label{sec:input}

Including child documents by |\include| has some restrictions by design.
Most notably, the content of a child document always occupies
its own set of pages; pages cannot be shared between child documents.
Usually, this behaviour makes perfect sense
because each child document contain an essential part of the document.
However, in some situations it may be desirable to compose
a document from a collection of parts
without having mandatory page breaks between then.
For this case, the package
provides a mechanism to include parts
by |\input| which can also be processed individually.
However, by construction this mechanism
requires manual handling of the content to be output.

%%%%%%%%%%%%%%%%%%%%%%%%%%%%%%%%%%%%%%%%
\DescribeMacro{\ifchilddocmanual}
The main file should be prepared as usual, see \secref{sec:include}.
However, the document body must make a distinction
between processing of an individual part and of the main document, e.g.:
%
\begin{center}
\begin{tabular}{l}
|\ifchilddocmanual|\\
|\input{\childdocname}|\\
|\||else|\\
\textit{document body with }|\input{|\textit{part}|}|\\
|\||fi|
\end{tabular}
\end{center}
%
The conditional |\ifchilddocmanual| is true whenever
a part to be included by |\input| is being compiled,
and the name of the part is stored in |\childdocname|.

%%%%%%%%%%%%%%%%%%%%%%%%%%%%%%%%%%%%%%%%
\DescribeMacro{\childdocby}
Each part to be included by |\input| should start with:
%
\begin{center}
\begin{tabular}{l}
|\input{childdoc.def}|\\
|\childdocby{|\textit{main}|}|\\
\end{tabular}
\end{center}
%
The directive |\childdocby| is similar to |\childdocof|
described in \secref{sec:include},
but the subsequent selection of content must be done manually.
To that end, both |\ifchilddoc| and |\ifchilddocmanual|
will be true upon processing of a part,
and the name of the part is stored in |\childdocname|.
Note that |\jobname| will be set to the filename of the current part
so that each part receives an individual |.aux| file
that does not interfere with the |.aux| file(s) of the main document.
This behaviour can be altered by the alternative form
|\childdocby[*]{|\textit{main}|}| (with a non-empty optional argument)
which uses the |.aux| file of the main document
by setting |\jobname| to \textit{main}.

%%%%%%%%%%%%%%%%%%%%%%%%%%%%%%%%%%%%%%%%%%%%%%%%%%%%%%%%%%%%%%%%%%%%%%%%%%%%%%%%
\subsection{Driver Development}
\label{sec:driver}

The \textsf{childdoc} mechanism can also be use for the development
of definition files such as \LaTeX{} styles or classes.
This case differs from the above setup with multiple parts
included by |\include| in that no |\includeonly| should be invoked.
This can be achieved by starting the include file
(before |\ProvidesPackage|) with:
%
\begin{center}
\begin{tabular}{l}
|\input{childdoc.def}|\\
|\childdocforward{|\textit{main}|}|\\
\end{tabular}
\end{center}
%
or alternatively with:
%
\begin{center}
\begin{tabular}{l}
|\input{childdoc.def}|\\
|\childdocby{|\textit{main}|}|\\
\end{tabular}
\end{center}
%
Both forms have slightly different effects as described above.
The main file is prepared as usual, see \secref{sec:include}.

%%%%%%%%%%%%%%%%%%%%%%%%%%%%%%%%%%%%%%%%%%%%%%%%%%%%%%%%%%%%%%%%%%%%%%%%%%%%%%%%
\subsection{Legacy Detection}
\label{sec:detection}

The directive |\childdocmain| in the main file can detect
whether the complete document or merely a child is to be compiled
even without using the directive |\childdocof|.
This method is deprecated because it is less robust
and there is no compelling reason to use it;
it is merely provided for backward compatibility
and it may be removed in future versions.

If the detection mechanism is to be used,
it is mandatory to correctly specify
the filename of the main file as the argument of |\childdocmain|:
%
\begin{center}
\begin{tabular}{l}
|\input{childdoc.def}|\\
|\childdocmain{|\textit{main}|}|\\
\end{tabular}
\end{center}
%
If |\jobname| does not match the argument \textit{main} of |\childdocmain|,
it is assumed that |\jobname| points to the child file to be compiled.
When using |\childdocmain| with the main file specified as argument,
it suffices to start a child file
with just |\input{|\textit{main}|}|
without loading of the package and using |\childdocof|.
If instead all processing is done
with the appropriate \textsf{childdoc} directives,
the argument of \textit{main} of |\childdocmain| can be empty.

An alternative version of the command line processing described
in \secref{sec:commandline} using the detection mechanism reads:
%
\begin{center}
|... -jobname "|\textit{target}|" "|[\textit{flags}]%
[|\def\jobname{|\textit{dest}|}|]|\input{|\textit{main}|}"|
\end{center}

%%%%%%%%%%%%%%%%%%%%%%%%%%%%%%%%%%%%%%%%%%%%%%%%%%%%%%%%%%%%%%%%%%%%%%%%%%%%%%%%
\subsection{Manual Code}
\label{sec:manual}

In case one cannot be certain whether the definitions file |childdoc.def|
is installed on the target \TeX{} distribution
and one prefers not to ship it,
it is conceivable to paste a few relevant commands into the sources.

To that end, drop all statements |\input{childdoc.def}|
and perform the replacements as outlined below.
Instead of |\childdocmain{|\textit{main}|}| add the following code
to the top of the main file:
%
\begin{center}
\begin{tabular}{l}
|\||ifdefined\childdocname\endinput\||fi\newif\ifchilddoc|\\
|\edef\childdocname{\scantokens\expandafter{\jobname\noexpand}}|\\
|\def\childdocmain{|\textit{main}|}\||ifx\childdocmain\childdocname\||else|\\
|\childdoctrue\includeonly{\childdocname}\let\jobname\childdocmain\||fi|\\
\end{tabular}
\end{center}
%
Instead of |\childdocof{|\textit{main}|}| just include the main file
at the top of each child file:
%
\begin{center}
|\input{|\textit{main}|}|
\end{center}
%
A simple redirection |\childdocforward{|\textit{dest}|}| is achieved by:
%
\begin{center}
|\def\jobname{|\textit{dest}|}\input{\jobname}|
\end{center}
%
The redirection with prefix
|\childdocforwardprefix[|\textit{prefix}|]{|\textit{dest}|}|
is accomplished by:
%
\begin{center}
\begin{tabular}{l}
|{\edef\jobname{\scantokens\expandafter{\jobname\noexpand}}|\\
|\def\redirectjob |\textit{prefix}|#1~~~{\gdef\jobname{|\textit{dest}|#1}}|\\
|\expandafter\redirectjob\jobname~~~}\input{\jobname}|
\end{tabular}
\end{center}

In an alternative approach,
child documents can be compiled by a specific command line
without additional code or specific definitions:
%
\begin{center}
|... -jobname "|\textit{target}|" "|[\textit{flags}]%
|\includeonly{|\textit{dest}|}\input{|\textit{main}|}"|
\end{center}
%

%%%%%%%%%%%%%%%%%%%%%%%%%%%%%%%%%%%%%%%%%%%%%%%%%%%%%%%%%%%%%%%%%%%%%%%%%%%%%%%%
%%%%%%%%%%%%%%%%%%%%%%%%%%%%%%%%%%%%%%%%%%%%%%%%%%%%%%%%%%%%%%%%%%%%%%%%%%%%%%%%
\section{Information}

%%%%%%%%%%%%%%%%%%%%%%%%%%%%%%%%%%%%%%%%%%%%%%%%%%%%%%%%%%%%%%%%%%%%%%%%%%%%%%%%
\subsection{Copyright}

Copyright \copyright{} 2017--2018 Niklas Beisert

This work may be distributed and/or modified under the
conditions of the \LaTeX{} Project Public License, either version 1.3
of this license or (at your option) any later version.
The latest version of this license is in
  \url{http://www.latex-project.org/lppl.txt}
and version 1.3 or later is part of all distributions of \LaTeX{}
version 2005/12/01 or later.

This work has the LPPL maintenance status `maintained'.

The Current Maintainer of this work is Niklas Beisert.

This work consists of the files |README.txt|, |childdoc.ins| and |childdoc.dtx|
as well as the derived files |childdoc.def|, |cdocsamp.tex|
with |cdocsch1.tex|, |cdocsch2.tex|, |cdocspt3.tex|, |cdocspt4.tex|,
|cdocsdrf.tex|, |cdocsfn1.tex|, |cdocsfn2.tex|
as well as |childdoc.pdf|.

%%%%%%%%%%%%%%%%%%%%%%%%%%%%%%%%%%%%%%%%%%%%%%%%%%%%%%%%%%%%%%%%%%%%%%%%%%%%%%%%
\subsection{Files and Installation}

The package consists of the files:
%
\begin{center}
\begin{tabular}{ll}
    |README.txt|   & readme file \\
    |childdoc.ins| & installation file \\
    |childdoc.dtx| & source file \\
    |childdoc.def| & definition file \\
    |cdocsamp.tex| & sample main file \\
    |cdocsch1.tex| & sample include file \\
    |cdocsch2.tex| & sample include file \\
    |cdocspt3.tex| & sample part file \\
    |cdocspt4.tex| & sample part file \\
    |cdocsdrf.tex| & sample redirection file \\
    |cdocsfn1.tex| & sample redirection file \\
    |cdocsfn2.tex| & sample redirection file \\
    |childdoc.pdf| & manual
\end{tabular}
\end{center}
%
The distribution consists of the files
|README.txt|, |childdoc.ins| and |childdoc.dtx|.
%
\begin{itemize}
\item
Run (pdf)\LaTeX{} on |childdoc.dtx|
to compile the manual |childdoc.pdf| (this file).
\item
Run \LaTeX{} on |childdoc.ins| to create the definitions file |childdoc.def|
and the sample |cdocsamp.tex| with include files
|cdocsch1.tex|, |cdocsch2.tex|, |cdocspt3.tex|, |cdocspt4.tex|,
|cdocsdrf.tex|, |cdocsfn1.tex|, |cdocsfn2.tex|.
Then copy the file |childdoc.def| to an appropriate directory of your \LaTeX{}
distribution, e.g.\ \textit{texmf-root}|/tex/latex/childdoc|.
\end{itemize}

%%%%%%%%%%%%%%%%%%%%%%%%%%%%%%%%%%%%%%%%%%%%%%%%%%%%%%%%%%%%%%%%%%%%%%%%%%%%%%%%
\subsection{Related CTAN Packages}

There are several other packages which offer a similar functionality:
%
\begin{itemize}
\item
The packages
\href{http://ctan.org/pkg/docmute}{\textsf{docmute}},
\href{http://ctan.org/pkg/includex}{\textsf{includex}} and
\href{http://ctan.org/pkg/standalone}{\textsf{standalone}}
provide commands to include only the document body of
a child file thus allowing both files to be compiled individually.
\item
The packages \href{http://ctan.org/pkg/subdocs}{\textsf{subdocs}}
and \href{http://ctan.org/pkg/subfiles}{\textsf{subfiles}}
provide structures in which the main and child documents can be
encapsulated and allowing them to be compiled individually.
The inclusion mechanism is different from the conventional |\include|.
\item
The package \href{http://ctan.org/pkg/combine}{\textsf{combine}}
is an elaborate solution to combine several documents into one.
\end{itemize}
%
See also the CTAN topic \href{http://ctan.org/topic/subdocs}{\textsf{subdocs}}
for further related packages.
The present package differs from the above solutions in that
a document structure constructed with the conventional |\include| mechanism
just needs two extra commands at the top of every file
such that all constituent files can be compiled individually.

%%%%%%%%%%%%%%%%%%%%%%%%%%%%%%%%%%%%%%%%%%%%%%%%%%%%%%%%%%%%%%%%%%%%%%%%%%%%%%%%
%\subsection{Feature Suggestions}
%
%The following is a list of features which may be useful for future
%versions of this package:
%%
%\begin{itemize}
%\item
%\ldots
%\end{itemize}

%%%%%%%%%%%%%%%%%%%%%%%%%%%%%%%%%%%%%%%%%%%%%%%%%%%%%%%%%%%%%%%%%%%%%%%%%%%%%%%%
\subsection{Revision History}

%%%%%%%%%%%%%%%%%%%%%%%%%%%%%%%%%%%%%%%%
\paragraph{v2.0:} 2018/12/30

\begin{itemize}
\item
immediate forward processing
\item
added |\childdocby| mechanism
\item
manual restructured
\end{itemize}

%%%%%%%%%%%%%%%%%%%%%%%%%%%%%%%%%%%%%%%%
\paragraph{v1.6:} 2018/01/17

\begin{itemize}
\item
application for development of include files
\item
corrections to manual
\end{itemize}

%%%%%%%%%%%%%%%%%%%%%%%%%%%%%%%%%%%%%%%%
\paragraph{v1.5:} 2017/05/21

\begin{itemize}
\item
more complete structuring introduced
\item
|\childdocof| introduced
\item
|\childdoc| renamed to |\childdocmain|
\item
|\childredirect| renamed to |\childdocforward| and |\childdocforwardprefix|
and functionality expanded
\end{itemize}

%%%%%%%%%%%%%%%%%%%%%%%%%%%%%%%%%%%%%%%%
\paragraph{v1.0:} 2017/04/27

\begin{itemize}
\item
manual and install package
\item
first version published on CTAN
\end{itemize}

%%%%%%%%%%%%%%%%%%%%%%%%%%%%%%%%%%%%%%%%
\paragraph{v0.6:} 2017/04/26

\begin{itemize}
\item
redirection mechanism added
\end{itemize}

%%%%%%%%%%%%%%%%%%%%%%%%%%%%%%%%%%%%%%%%
\paragraph{v0.5:} 2017/04/26

\begin{itemize}
\item
functionality in definition file
\end{itemize}


%%%%%%%%%%%%%%%%%%%%%%%%%%%%%%%%%%%%%%%%%%%%%%%%%%%%%%%%%%%%%%%%%%%%%%%%%%%%%%%%
%%%%%%%%%%%%%%%%%%%%%%%%%%%%%%%%%%%%%%%%%%%%%%%%%%%%%%%%%%%%%%%%%%%%%%%%%%%%%%%%
%%%%%%%%%%%%%%%%%%%%%%%%%%%%%%%%%%%%%%%%%%%%%%%%%%%%%%%%%%%%%%%%%%%%%%%%%%%%%%%%
\appendix

\settowidth\MacroIndent{\rmfamily\scriptsize 000\ }

 \DocInput{childdoc.dtx}

\end{document}
%</driver>
% \fi
%
% %%%%%%%%%%%%%%%%%%%%%%%%%%%%%%%%%%%%%%%%%%%%%%%%%%%%%%%%%%%%%%%%%%%%%%%%%%%%%%
% %%%%%%%%%%%%%%%%%%%%%%%%%%%%%%%%%%%%%%%%%%%%%%%%%%%%%%%%%%%%%%%%%%%%%%%%%%%%%%
% \section{Sample}
%\iffalse
%<*samplemain>
%\fi
%
% The following presents a sample document
% with two chapters, two parts, a title page,
% a compile flag as well as three forwarding files to set the flag.
% It consists of eight |.tex| files:
% \begin{center}
% \begin{tabular}{ll}
% |cdocsamp.tex|&main file\\
% |cdocsch1.tex|&include file for chapter 1\\
% |cdocsch2.tex|&include file for chapter 2\\
% |cdocspt3.tex|&include file for part 3\\
% |cdocspt4.tex|&include file for part 4\\
% |cdocsdrf.tex|&forwarding file for main file in draft mode\\
% |cdocsfi1.tex|&forwarding file for final version of chapter 1\\
% |cdocsfi2.tex|&forwarding file for final version of chapter 2\\
% \end{tabular}
% \end{center}
% Each of the eight files can be compiled directly by the \LaTeX{} compiler.
%
% %%%%%%%%%%%%%%%%%%%%%%%%%%%%%%%%%%%%%%
% \paragraph{Main File.}
%
% The main file is called |cdocsamp.tex|.
%
% Load the \textsf{childdoc} definitions and
% declare the filename for the main document:
%    \begin{macrocode}
\input{childdoc.def}
\childdocmain{}
%    \end{macrocode}

% Optional override for |\version| flag:
%    \begin{macrocode}
%%\ifchilddoc\else\providecommand{\version}{draft}\fi
%    \end{macrocode}

% Define the default values for the |\version| flag
% (|final| for the main file and |draft| for childs):
%    \begin{macrocode}
\ifchilddoc
\providecommand{\version}{draft}
\else
\providecommand{\version}{final}
\fi
%    \end{macrocode}

% Load the standard document class:
%    \begin{macrocode}
\documentclass[12pt]{article}
%    \end{macrocode}

% Start the document body:
%    \begin{macrocode}
\begin{document}
%    \end{macrocode}

% Declare a title page.
% Print title, part of document being processed and version flag:
%    \begin{macrocode}
\addtocounter{page}{-1}
\begin{center}
{\LARGE\bfseries{}childdoc example\par}
\vspace{1cm}
\ifchilddoc
\ifchilddocmanual part\else chapter\fi:
`\childdocname' of `\childdocjob'\par
\else
main document: `\childdocjob'\par
\fi
version: \version\par
\end{center}
\newpage
%    \end{macrocode}

% Manually include selected file,
% otherwise process as usual:
%    \begin{macrocode}
\ifchilddocmanual
\section*{part `\childdocname'}
\input{\childdocname}
\else
%    \end{macrocode}

% Include the two chapters:
%    \begin{macrocode}
\include{cdocsch1}
\include{cdocsch2}
%    \end{macrocode}

% Include the two parts unless only chapters should be displayed:
%    \begin{macrocode}
\ifchilddoc\else
\section{part three}
\input{cdocspt3}
\section{part four}
\input{cdocspt4}
\fi
%    \end{macrocode}

% Process as usual until here:
%    \begin{macrocode}
\fi
%    \end{macrocode}

% End of document body:
%    \begin{macrocode}
\end{document}
%    \end{macrocode}
%\iffalse
%</samplemain>
%\fi
%
% %%%%%%%%%%%%%%%%%%%%%%%%%%%%%%%%%%%%%%
% \paragraph{Chapter Include Files.}
%
% The include files are called |cdocsch1.tex| and |cdocsch2.tex|.
%
%\iffalse
%<*samplechap1|samplechap2>
%\fi

% Optional override for |\version| flag:
%    \begin{macrocode}
%%\providecommand{\version}{final}
%    \end{macrocode}

% Include the main document:
%    \begin{macrocode}
\input{childdoc.def}
\childdocof{cdocsamp}
%    \end{macrocode}

%\iffalse
%</samplechap1|samplechap2>
%\fi
%
%\iffalse
%<*samplechap1>
%\fi
% Some text for chapter 1:
%    \begin{macrocode}
\section{one}
some text in chapter one
%    \end{macrocode}

%\iffalse
%</samplechap1>
%\fi
% Some text for chapter 2:
%\iffalse
%<*samplechap2>
%\fi
%    \begin{macrocode}
\section{two}
more text in chapter two
%    \end{macrocode}

%\iffalse
%</samplechap2>
%\fi
%
% %%%%%%%%%%%%%%%%%%%%%%%%%%%%%%%%%%%%%%
% \paragraph{Part Include Files.}
%
% The include files are called |cdocspt3.tex| and |cdocspt4.tex|.
%
%\iffalse
%<*samplepart3|samplepart4>
%\fi

% Optional override for |\version| flag:
%    \begin{macrocode}
%%\providecommand{\version}{final}
%    \end{macrocode}

% Include the main document:
%    \begin{macrocode}
\input{childdoc.def}
\childdocby{cdocsamp}
%    \end{macrocode}

%\iffalse
%</samplepart3|samplepart4>
%\fi
%
%\iffalse
%<*samplepart3>
%\fi
% Some text for part 3:
%    \begin{macrocode}
some text in part three
%    \end{macrocode}

%\iffalse
%</samplepart3>
%\fi
% Some text for part 4:
%\iffalse
%<*samplepart4>
%\fi
%    \begin{macrocode}
more text in part four
%    \end{macrocode}

%\iffalse
%</samplepart4>
%\fi
%
% %%%%%%%%%%%%%%%%%%%%%%%%%%%%%%%%%%%%%%
% \paragraph{Forwarding for a Complete Draft.}
%
% The following forwarding file |cdocsdrf.tex|
% compiles the main document in draft mode:
%\iffalse
%<*sampledraft>
%\fi
%    \begin{macrocode}
\def\version{draft}
\input{childdoc.def}
\childdocforward{cdocsamp}
%    \end{macrocode}

%\iffalse
%</sampledraft>
%\fi
%
% %%%%%%%%%%%%%%%%%%%%%%%%%%%%%%%%%%%%%%
% \paragraph{Forwarding for Final Version of the Chapters.}
%
% The following forwarding files |cdocsfn1.tex| and |cdocsfn2.tex|
% (with identical content)
% compile the final versions of the child documents
% |cdocsch1.tex| and |cdocsch2.tex|, respectively:
%\iffalse
%<*samplefinal>
%\fi
%    \begin{macrocode}
\def\version{final}
\input{childdoc.def}
\childdocforwardprefix[cdocsamp]{cdocsfn}{cdocsch}
%    \end{macrocode}

%\iffalse
%</samplefinal>
%\fi
%
% %%%%%%%%%%%%%%%%%%%%%%%%%%%%%%%%%%%%%%
% \paragraph{Command Line Processing.}
%
% The following three command lines generate the output files
% |cdocscld|, |cdocscl1| and |cdocscl2|
% which should be identical to
% |cdocsdrf|, |cdocsch1| and |cdocsfn2|, respectively:
% \begin{center}
% \begin{tabular}{l}
% |latex -jobname cdocscld \|\\
% |  "\def\version{draft}\input{childdoc.def}\childdocforward{cdocsamp}"|\\
% |latex -jobname cdocscl1 \|\\
% |  "\input{childdoc.def}\childdocforward[cdocsamp]{cdocsch1}"|\\
% |latex -jobname cdocscl2 \|\\
% |  "\def\version{final}\input{childdoc.def}\childdocforward{cdocsch2}"|
% \end{tabular}
% \end{center}
% Note that the trailing backslash on each first line
% merely continues the input to the second line
% (for convenient cut ant paste).
% Furthermore, the command |latex| can be replaced by any
% of its alternative versions such as |pdflatex|.
%
% %%%%%%%%%%%%%%%%%%%%%%%%%%%%%%%%%%%%%%%%%%%%%%%%%%%%%%%%%%%%%%%%%%%%%%%%%%%%%%
% %%%%%%%%%%%%%%%%%%%%%%%%%%%%%%%%%%%%%%%%%%%%%%%%%%%%%%%%%%%%%%%%%%%%%%%%%%%%%%
% \section{Implementation}
%\iffalse
%<*package>
%\fi
%
% This section describes the definitions file |childdoc.def|.

% The definitions cannot be loaded using |\usepackage| or |\RequirePackage|
% which has a mechanism to prevent loading a style file more than once.
% When loading the definitions by means of |\input|
% multiple instances have to be prevented manually:
%\iffalse
%This code needs to be before the `\ProvidesFile' directive
%which is defined at the beginning of this file.
%Therefore it is also placed there and commented out here.
%</package>
%<*discard>
%\fi
%    \begin{macrocode}
\ifdefined\childdocmain\endinput\fi
%    \end{macrocode}
%\iffalse
%</discard>
%<*package>
%\fi
%
% \macro{\ifchilddoc}
% \macro{\ifchilddocmanual}
% The conditional |\ifchilddoc| tells whether a
% child (true) or main (false) document is being compiled.
% The conditional |\ifchilddocmanual| tells whether
% the |\includeonly| mechanism is used (false) or
% the selection of child files must be performed manually (true).
% The definitions initialise to false:
%    \begin{macrocode}
\newif\ifchilddoc
\newif\ifchilddocmanual
%    \end{macrocode}

% \macro{\childdocname}
% \macro{\childdocjob}
% The macro |\childdocname| stores the name of the main document
% to be compiled. The macro |\childdocjob| stores the name of
% the document on which the \LaTeX{} compiler was originally invoked.
% The content of |\jobname| cannot be compared
% to filenames specified in the source due to different catcodes.
% The following code rescans |\jobname|, stores the result
% in |\childdocname| and saves a copy in |\childdocjob|:
%    \begin{macrocode}
\edef\childdocname{\scantokens\expandafter{\jobname\noexpand}}
\let\childdocjob\childdocname
%    \end{macrocode}

% \macro{\childdocdisable}
% The macro |\childdocdisable| prevents the main file
% from being processed more than once.
% At this stage, the main document command |\childdocmain|
% is assumed to be called once again where it should do nothing.
% Any subsequent call to it should prevent
% a secondary processing of the main document
% It overwrites the forwarding commands
% |\childdocof| and |\childdocforward|
% with empty macros to prevent further inclusions of the main document:
%    \begin{macrocode}
\newcommand{\childdocdisable}
{
  \renewcommand{\childdocmain}[1]{\renewcommand{\childdocmain}[1]{\endinput}}
  \renewcommand{\childdocof}[1]{}
  \renewcommand{\childdocby}[2][]{}
  \renewcommand{\childdocforward}[2][]{}
  \renewcommand{\childdocdisable}{}
}
%    \end{macrocode}

% \macro{\childdocmain}
% The macro |\childdocmain| is to be called at the top of the main file
% with nothing or the main filename (without extension) as argument.
% First, it breaks loops.
% If the argument is not empty and does not match |\childdocname|
% (which is set by the first inclusion of |childdoc.def|),
% |\ifchilddoc| is set to true, |\includeonly| is applied to the child file
% and |\jobname| is set to the main file
% (for proper handling of |.aux| files):
%    \begin{macrocode}
\newcommand{\childdocmain}[1]
{
  \childdocdisable\childdocmain{}
  \if?#1?\else
    \begingroup
      \def\childdoctmp{#1}
      \ifx\childdoctmp\childdocname
        \def\childdoctmp{}
      \else
        \def\childdoctmp
        {
          \childdoctrue
          \includeonly{\childdocname}
          \def\childdocjob{#1}
          \def\jobname{#1}
        }
      \fi
      \expandafter
    \endgroup
    \childdoctmp
  \fi
}
%    \end{macrocode}

% \macro{\childdocof}
% The command |\childdocof| redirects
% compilation to the main file |#1|.
%    \begin{macrocode}
\newcommand{\childdocof}[1]
{
  \childdocdisable
  \childdoctrue
  \includeonly{\childdocname}
  \def\jobname{#1}
  \def\childdocjob{#1}
  \input{#1}
}
%    \end{macrocode}

% \macro{\childdocby}
% The command |\childdocby| ....
%    \begin{macrocode}
\newcommand{\childdocby}[2][]
{
  \childdocdisable
  \childdoctrue
  \childdocmanualtrue
  \if?#1?\else
    \def\jobname{#2}
  \fi
  \def\childdocjob{#2}
  \input{#2}
  \endinput
}
%    \end{macrocode}

% \macro{\childdocforward}
% The command |\childdocforward| redirects
% compilation to the main file or
% (if the optional argument is given) a child file.
% Parameters are set as if the main file
% or a child file starting with |\childdocof| was compiled.
% Then compilation is handed over to the main file:
%    \begin{macrocode}
\newcommand{\childdocforward}[2][]
{
  \begingroup
    \if?#1?
      \def\childdoctmp
      {
        \def\childdocname{#2}
        \def\childdocjob{#2}
        \def\jobname{#2}
        \input{#2}
        \endinput
      }
    \else
      \def\childdoctmp
      {
        \childdocdisable
        \def\childdocname{#2}
        \childdoctrue
        \includeonly{#2}
        \def\childdocjob{#1}
        \def\jobname{#1}
        \input{#1}
        \endinput
      }
    \fi
    \expandafter
  \endgroup
  \childdoctmp
}
%    \end{macrocode}

% \macro{\childdocforwardprefix}
% The command |\childdocforwardprefix| redirects
% compilation to the main or a child file by means of a pattern.
% The prefix |#1| in the current filename is replaced by |#2|
% and the suffix of the current filename is kept
% (it is assumed that the filename does not contain the substring `|~~~|'
% which is used as a delimiter).
% Compilation is handed over to the new file by |\childdocforward|:
%    \begin{macrocode}
\newcommand{\childdocforwardprefix}[3][]
{
  \begingroup
    \def\childdocextract #2##1~~~{\def\childdoctmp{\childdocforward[#1]{#3##1}}}
    \expandafter\childdocextract\childdocname~~~
    \expandafter
  \endgroup
  \childdoctmp
}
%    \end{macrocode}

% \macro{\childdoc}
% The deprecated macro |\childdoc| is a legacy version of |\childdocmain|:
%    \begin{macrocode}
\newcommand{\childdoc}{\childdocmain}
%    \end{macrocode}

% \macro{\childdocredirect}
% The deprecated macro |\childdocredirect| is a legacy version
% of |\childdocforward| and |\childdocforwardprefix|:
%    \begin{macrocode}
\newcommand{\childdocredirect}[2][]
{
  \begingroup
    \if?#1?
      \def\childdoctmp{\childdocforward{#2}}
    \else
      \def\childdoctmp{\childdocforwardprefix{#1}{#2}}
    \fi
    \expandafter
  \endgroup
  \childdoctmp
}
%    \end{macrocode}

%\iffalse
%</package>
%\fi
%
\endinput
|\\
|\childdocmain{|\textit{main}|}|\\
\end{tabular}
\end{center}
%
If |\jobname| does not match the argument \textit{main} of |\childdocmain|,
it is assumed that |\jobname| points to the child file to be compiled.
When using |\childdocmain| with the main file specified as argument,
it suffices to start a child file
with just |\input{|\textit{main}|}|
without loading of the package and using |\childdocof|.
If instead all processing is done
with the appropriate \textsf{childdoc} directives,
the argument of \textit{main} of |\childdocmain| can be empty.

An alternative version of the command line processing described
in \secref{sec:commandline} using the detection mechanism reads:
%
\begin{center}
|... -jobname "|\textit{target}|" "|[\textit{flags}]%
[|\def\jobname{|\textit{dest}|}|]|\input{|\textit{main}|}"|
\end{center}

%%%%%%%%%%%%%%%%%%%%%%%%%%%%%%%%%%%%%%%%%%%%%%%%%%%%%%%%%%%%%%%%%%%%%%%%%%%%%%%%
\subsection{Manual Code}
\label{sec:manual}

In case one cannot be certain whether the definitions file |childdoc.def|
is installed on the target \TeX{} distribution
and one prefers not to ship it,
it is conceivable to paste a few relevant commands into the sources.

To that end, drop all statements |% \iffalse
%
% childdoc.dtx Copyright (C) 2017-2018 Niklas Beisert
%
% This work may be distributed and/or modified under the
% conditions of the LaTeX Project Public License, either version 1.3
% of this license or (at your option) any later version.
% The latest version of this license is in
%   http://www.latex-project.org/lppl.txt
% and version 1.3 or later is part of all distributions of LaTeX
% version 2005/12/01 or later.
%
% This work has the LPPL maintenance status `maintained'.
%
% The Current Maintainer of this work is Niklas Beisert.
%
% This work consists of the files childdoc.dtx and childdoc.ins
% and the derived files childdoc.def and cdocsamp.tex with
% cdocsch1.tex, cdocsch2.tex, cdocsdrf.tex, cdocsfn1.tex, cdocsfn2.tex.
%
%<package>\ifdefined\childdocmain\endinput\fi
%<package>\ProvidesFile{childdoc.def}[2018/12/30 v2.0 child document driver]
%<samplemain>\ProvidesFile{cdocsamp.tex}[2018/12/30 v2.0 sample for childdoc]
%<*driver>
%\ProvidesFile{childdoc.drv}[2018/12/30 v2.0 childdoc reference manual file]
\PassOptionsToClass{10pt,a4paper}{article}
\documentclass{ltxdoc}

\usepackage[margin=35mm]{geometry}
\usepackage{hyperref}
\usepackage{hyperxmp}
\usepackage[usenames]{color}

\hypersetup{colorlinks=true}
\hypersetup{pdfstartview=FitH}
\hypersetup{pdfpagemode=UseNone}
\hypersetup{pdfsource={}}
\hypersetup{pdflang={en-UK}}
\hypersetup{pdfcopyright={Copyright 2017-2018 Niklas Beisert.
  This work may be distributed and/or modified under the
  conditions of the LaTeX Project Public License, either version 1.3
  of this license or (at your option) any later version.}}
\hypersetup{pdflicenseurl={http://www.latex-project.org/lppl.txt}}
\hypersetup{pdfcontactaddress={ETH Zurich, ITP, HIT K,
  Wolfgang-Pauli-Strasse 27}}
\hypersetup{pdfcontactpostcode={8093}}
\hypersetup{pdfcontactcity={Zurich}}
\hypersetup{pdfcontactcountry={Switzerland}}
\hypersetup{pdfcontactemail={nbeisert@itp.phys.ethz.ch}}
\hypersetup{pdfcontacturl={http://people.phys.ethz.ch/\xmptilde nbeisert/}}

\newcommand{\secref}[1]{\hyperref[#1]{section \ref*{#1}}}

\parskip1ex
\parindent0pt
\let\olditemize\itemize
\def\itemize{\olditemize\parskip0pt}

\begin{document}

\title{The \textsf{childdoc} Package}
\hypersetup{pdftitle={The childdoc Package}}
\author{Niklas Beisert\\[2ex]
  Institut f\"ur Theoretische Physik\\
  Eidgen\"ossische Technische Hochschule Z\"urich\\
  Wolfgang-Pauli-Strasse 27, 8093 Z\"urich, Switzerland\\[1ex]
  \href{mailto:nbeisert@itp.phys.ethz.ch}
  {\texttt{nbeisert@itp.phys.ethz.ch}}}
\hypersetup{pdfauthor={Niklas Beisert}}
\hypersetup{pdfsubject={Manual for the LaTeX2e Package childdoc}}
\date{30 December 2018, \textsf{v2.0}}
\maketitle

\begin{abstract}\noindent
\textsf{childdoc} is a \LaTeXe{} package
that enables the direct compilation
of document sections included by |\include|
to individual files.
\end{abstract}

\begingroup
\parskip0ex
\tableofcontents
\endgroup

%%%%%%%%%%%%%%%%%%%%%%%%%%%%%%%%%%%%%%%%%%%%%%%%%%%%%%%%%%%%%%%%%%%%%%%%%%%%%%%%
%%%%%%%%%%%%%%%%%%%%%%%%%%%%%%%%%%%%%%%%%%%%%%%%%%%%%%%%%%%%%%%%%%%%%%%%%%%%%%%%
\section{Introduction}

\LaTeX{} provides a mechanism to structure a large document (such as a book)
into a main file and several child files (containing the chapters)
using the |\include| command.
This mechanism is beneficial for documents
which span hundreds of pages in order to
make the source file(s) more manageable.
Moreover, compilation can be restricted to
selected child files by means of the |\includeonly| command.
The latter feature can be used to reduce the compilation time while editing
(this was significantly more useful in the earlier days of \LaTeX{})
or to generate a smaller document which is easier to navigate.
Another application of |\includeonly| is to generate
documents consisting of selected parts of the complete document.

However, there are a few drawbacks of the plain |\include| mechanism:
\begin{itemize}
\item
The child files cannot be compiled on their own,
they can only be compiled via the main file.
A naive editing environment
(such as a text editor with an option
to have the current file processed by \LaTeX)
may require one to switch to the main file before compiling;
attempting to compile the child file produces errors.
\item
The main file must be modified (each time)
to adjust the |\includeonly| command
to the present needs. This easily leaves the main file in a messy state.
\item
The generated document will always carry the filename
of the main document. This is inconvenient if
several child files are to be compiled and
to be kept for distribution.
\end{itemize}

The present package provides a simple interface
to make child files individually compilable by \LaTeX{}.
Compiling a child file then has the same effect as compiling
the main file with an |\includeonly| command
to select the appropriate child.
Moreover the generated document will carry the name of the child
rather than the main file.
This resolves all three above issues.

This feature is meant to make the editing of books,
thesis documents and lecture notes somewhat more convenient.
However, the package can also be used efficiently for
composing a series of documents (such as exercise sheets)
which are typically distributed individually.
It then assists the author in generating the individual documents
(potentially in different versions)
as well as a document containing the collected series.
Another application is in developing style files
or other kinds of included material
where compilation of the style file could redirect
to a sample or test file.

%%%%%%%%%%%%%%%%%%%%%%%%%%%%%%%%%%%%%%%%%%%%%%%%%%%%%%%%%%%%%%%%%%%%%%%%%%%%%%%%
%%%%%%%%%%%%%%%%%%%%%%%%%%%%%%%%%%%%%%%%%%%%%%%%%%%%%%%%%%%%%%%%%%%%%%%%%%%%%%%%
\section{Usage}

First of all, the package \textsf{childdoc} is \emph{not} a standard
\LaTeXe{} |.sty| style file! Therefore it needs to be invoked in
a non-standard way.

%%%%%%%%%%%%%%%%%%%%%%%%%%%%%%%%%%%%%%%%%%%%%%%%%%%%%%%%%%%%%%%%%%%%%%%%%%%%%%%%
\subsection{Included Files}
\label{sec:include}

%%%%%%%%%%%%%%%%%%%%%%%%%%%%%%%%%%%%%%%%
\DescribeMacro{\childdocmain}
To use the package, add the commands
\begin{center}
\begin{tabular}{l}
|\input{childdoc.def}|\\
|\childdocmain{}|\\
\end{tabular}
\end{center}
at the very top of the main \LaTeX{} file,
in particular \emph{before} the |\documentclass| statement!
The argument of |\childdocmain| should be left empty
(but it must be present).

%%%%%%%%%%%%%%%%%%%%%%%%%%%%%%%%%%%%%%%%
\DescribeMacro{\childdocof}
Furthermore, add the commands
\begin{center}
\begin{tabular}{l}
|\input{childdoc.def}|\\
|\childdocof{|\textit{main}|}|\\
\end{tabular}
\end{center}
at the top of every child file \textit{child}
which is included by |\include{|\textit{child}|}|
from within the main file
(or at least for those files to be compiled individually).
The argument \textit{main} must be the filename of the main file.

There are a couple of
considerations in setting up the main and child documents:

%%%%%%%%%%%%%%%%%%%%%%%%%%%%%%%%%%%%%%%%
\paragraph{Restrictions.}

Please note the following restrictions:
\begin{itemize}
\item
|\childdocmain| must be called with one argument \textit{main}
to ensure compatibility with earlier version of the package.
It must either be empty (|\childdocmain{}|)
or precisely match the filename of the main file in which it is specified.
See \secref{sec:detection} for further information.
\item
The filename \textit{main} must be specified without the |.tex| extension.
\item
The filename \textit{main} is case sensitive
(even in case-insensitive file systems)
due to internal string comparison.
\item
The argument \textit{main} should be fully expanded, it cannot be a macro.
\item
Subdirectories and special characters should be avoided in filenames.
\item
The command |\childdocmain{|\textit{main}|}| must be followed by a whitespace.
It should not be followed immediately by another command
or by a comment mark `|%|'.
This is because the \TeX{} parser reads the token immediately following
the argument of |\childdocmain| and puts it
at the beginning of every child section;
however, a white\-space is ignored.
\end{itemize}

%%%%%%%%%%%%%%%%%%%%%%%%%%%%%%%%%%%%%%%%
\paragraph{Content of Main File.}

It is advisable to place all content in the child files included by |\include|.
Any output contained in the main file will appear in all child documents
unless suppressed manually;
it cannot be suppressed automatically by the |\includeonly| directive
and thus should normally be avoided.
A method to include some content in the main file
by means of conditional processing is described in \secref{sec:conditional}.

%%%%%%%%%%%%%%%%%%%%%%%%%%%%%%%%%%%%%%%%
\paragraph{Page Numbering.}

When only a part of the document is compiled,
the appropriate numbering of pages
(as well as other status parameters)
is determined from the |.aux| files.
The latter contain information from previous passes.
However this information needs to propagate through
all intermediate child documents.
Therefore the page numbering in child documents may well
be inconsistent until the complete document is compiled at least once.

A useful (if unconventional) way to always ensure a consistent
page numbering is to restart the numbering in each child document
and denote the pages by `\textit{child}|.|\textit{page}'
where \textit{child} represents the chapter/section number of the child file.
This can be achieved by the command
|\numberwithin{page}{|\textit{child}|}|
of the \textsf{amsmath} package
where \textit{child} can be |chapter| or |section|
depending on the chosen structuring.
Alternatively, one can modify the macro |\thepage| appropriately
and reset the counter |page| at the start of each child file.

%%%%%%%%%%%%%%%%%%%%%%%%%%%%%%%%%%%%%%%%%%%%%%%%%%%%%%%%%%%%%%%%%%%%%%%%%%%%%%%%
\subsection{Conditional Processing}
\label{sec:conditional}

The package provides a mechanism to compile different versions
of a document. To customise the versions further some conditional processing
can come in handy to distinguish which version is being compiled.
The package provides two macros to describe the compilation context:

%%%%%%%%%%%%%%%%%%%%%%%%%%%%%%%%%%%%%%%%
\DescribeMacro{\ifchilddoc}
The conditional |\ifchilddoc| distinguishes between the compilation of
child documents and the main document:
%
\begin{center}
|\ifchilddoc |\textit{child-code}| |[|\||else |\textit{main-code}]| \||fi|
\end{center}

%%%%%%%%%%%%%%%%%%%%%%%%%%%%%%%%%%%%%%%%
\DescribeMacro{\childdocname}
\DescribeMacro{\childdocjob}
The macro |\childdocname| contains the filename (without extension)
of the main or child file being processed.
Note that |\childdocjob| will always contain the name of the main file.

%%%%%%%%%%%%%%%%%%%%%%%%%%%%%%%%%%%%%%%%
\paragraph{Title Page.}

Conditional processing can be used to include a title or banner page
in the main document when proper precautions are taken.
Importantly, the code in the main file should ensure that the page counter
(as well as other status parameters which are stored in the |.aux| files)
takes the same value after the conditional processing.
Otherwise the page numbers may take divergent values
depending on which part is compiled.

For example, a title page could be declared by:
%
\begin{center}
\begin{tabular}{l}
|\ifchilddoc\||else|\\
|\addtocounter{page}{-1}|\\
\textit{code for title page}\\
|\newpage|\\
|\||fi|
\end{tabular}
\end{center}
%
A banner page for the child documents can be generated by:
%
\begin{center}
\begin{tabular}{l}
|\ifchilddoc|\\
|\addtocounter{page}{-1}|\\
\textit{code for banner page}\\
|\newpage|\\
|\||fi|
\end{tabular}
\end{center}
%
Here one could write a message such as:
\begin{center}
|This is the part \childdocname{} of \childdocjob{}.|
\end{center}

%%%%%%%%%%%%%%%%%%%%%%%%%%%%%%%%%%%%%%%%%%%%%%%%%%%%%%%%%%%%%%%%%%%%%%%%%%%%%%%%
\subsection{Flags}
\label{sec:flags}

The package makes it easy to generate different versions
of the main or child documents.
To this end compilation flags can be defined
and assigned different default values.
They will be particularly useful in conjunction
with the forwarding mechanism described in \secref{sec:forward}.

For example, it may be useful to have a flag |\version|
which can be set to |draft| or |final|.
The document source will contain some conditional code
depending on the value of |\version|.
Suppose further, the flag should default to |final| for the main file
and to |draft| for child files
which is a natural assignment for editing the document.
This is achieved by placing the following code
in the preamble of the main document
(below the |\childdocmain| directive):
%
\begin{center}
\begin{tabular}{l}
|\ifchilddoc|\\
|\providecommand{\version}{draft}|\\
|\||else|\\
|\providecommand{\version}{final}|\\
|\||fi|
\end{tabular}
\end{center}
%
The definition by |\providecommand| makes sure
that previous definitions are not overwritten.
Further statements |\providecommand{\version}{...}|
can thus be added before the above code to override it.

For the main file, one might add a line
(between |\childdocmain| and the above block)
%
\begin{center}
|%\ifchilddoc\||else\providecommand{\version}{draft}\||fi|
\end{center}
%
which can be uncommented to produce a draft version.
Likewise one can add a line to the very top of a child file
(above the |\childdocof{|\textit{main}|}| directive)
%
\begin{center}
|%\providecommand{\version}{final}|
\end{center}
%
which can be uncommented to produce the final version of this child document.

%%%%%%%%%%%%%%%%%%%%%%%%%%%%%%%%%%%%%%%%%%%%%%%%%%%%%%%%%%%%%%%%%%%%%%%%%%%%%%%%
\subsection{Forwarding}
\label{sec:forward}

Different versions of the main or child documents
using compilation flags as described in \secref{sec:flags}
can be (permanently) stored in different files
for convenient compilation, viewing and distribution.
To this end, the package defines a command
to pass on compilation to a different file:

%%%%%%%%%%%%%%%%%%%%%%%%%%%%%%%%%%%%%%%%
\DescribeMacro{\childdocforward}
The command |\childdocforward| redirects processing to
another source file:
%
\begin{center}
\begin{tabular}{l}
|\input{childdoc.def}|\\
|\childdocforward[|\textit{main}|]{|\textit{dest}|}|\\
\end{tabular}
\end{center}
%
The argument \textit{dest} is the destination file
(without extension).
It should be the main file or one of the child files.
Note that further \textsf{childdoc} directives
such as |\childdocof| and |\childdocforward|
in the indicated file will be processed in this form.
The optional argument \textit{main}
passes on directly to the main file \textit{main}
while pretending to compile the child \textit{dest}.
This form behaves as if \textit{dest}
issues |\childdocof{|\textit{main}|}| right away,
and no further \textsf{childdoc} directives will be processed.

%%%%%%%%%%%%%%%%%%%%%%%%%%%%%%%%%%%%%%%%
\DescribeMacro{\...prefix}
In the alternative form |\childdocforwardprefix|,
%
\begin{center}
\begin{tabular}{l}
|\input{childdoc.def}|\\
|\childdocforwardprefix[|\textit{main}|]{|\textit{prefix}|}{|\textit{dest}|}|
\end{tabular}
\end{center}
%
the destination file is determined by a pattern
depending on the current file:
To make this work, the current file must be called
`{\textit{prefix}\hspace{0.2em}\textit{suffix}}'
with \textit{prefix} matching precisely the argument.
Processing is then passed on to the file
`{\textit{dest}\hspace{0.2em}\textit{suffix}}'.
Surely, the same effect is achieved by
directly specifying the
argument `{\textit{dest}\hspace{0.2em}\textit{suffix}}'
in the first form.
However, that requires to set up a different file
for each child. With the alternative form of the command
all these files can have exactly the same content
which simplifies setting them up and maintaining them.

For example, the following file |draft.tex|
with a compilation flag |\version| as described in \secref{sec:flags}
compiles the main document as a draft:
%
\begin{center}
\begin{tabular}{l}
|\def\version{draft}|\\
|\input{childdoc.def}|\\
|\childdocforward{|\textit{main}|}|
\end{tabular}
\end{center}
%
Likewise, the following files |final|\textit{nn}|.tex|
compile the final version of the child document
|child|\textit{nn}|.tex|:
%
\begin{center}
\begin{tabular}{l}
|\def\version{final}|\\
|\input{childdoc.def}|\\
|\childdocforwardprefix{final}{child}|
\end{tabular}
\end{center}
%

Note that when several versions of a main file and/or of each child file
are to be generated, it may be convenient to set up a |Makefile| or
shell script to automatise the process.

%%%%%%%%%%%%%%%%%%%%%%%%%%%%%%%%%%%%%%%%%%%%%%%%%%%%%%%%%%%%%%%%%%%%%%%%%%%%%%%%
\subsection{Command Line Processing}
\label{sec:commandline}

The effect of redirection files can also be achieved by invoking
the \LaTeX{} compiler with a more elaborate command line.
Most conveniently this should be done as part
of a shell script or a |Makefile|.

When using \textsf{childdoc} in the main file, the following
command lines effectively perform a redirection
(note that depending on the shell being used,
backslashes may have to be doubled: `|\|' $\to$ `|\\|'):
%
\begin{center}
|... -jobname "|\textit{target}|" |\\|"|[\textit{flags}]%
|\input{childdoc.def}\childdocforward[|\textit{main}|]{|\textit{dest}|}"|
\end{center}
%
Here \textit{target} is the name of the output file,
\textit{main} is the name of the main file
and \textit{dest} is the name of the main or child file to be processed
(all filenames without extensions).
The optional argument \textit{main} can be omitted
if \textit{main} matches \textit{dest}.
Optionally, compilation \textit{flags} can be defined via |\def| commands.
This command line makes the \TeX{} engine believe
it is compiling the file \textit{target}
whose content is specified as the latter parameter.
The provided code then forwards the processing to
\textit{main} or \textit{dest} as described in \secref{sec:forward}.

%%%%%%%%%%%%%%%%%%%%%%%%%%%%%%%%%%%%%%%%%%%%%%%%%%%%%%%%%%%%%%%%%%%%%%%%%%%%%%%%
\subsection{Include by Input}
\label{sec:input}

Including child documents by |\include| has some restrictions by design.
Most notably, the content of a child document always occupies
its own set of pages; pages cannot be shared between child documents.
Usually, this behaviour makes perfect sense
because each child document contain an essential part of the document.
However, in some situations it may be desirable to compose
a document from a collection of parts
without having mandatory page breaks between then.
For this case, the package
provides a mechanism to include parts
by |\input| which can also be processed individually.
However, by construction this mechanism
requires manual handling of the content to be output.

%%%%%%%%%%%%%%%%%%%%%%%%%%%%%%%%%%%%%%%%
\DescribeMacro{\ifchilddocmanual}
The main file should be prepared as usual, see \secref{sec:include}.
However, the document body must make a distinction
between processing of an individual part and of the main document, e.g.:
%
\begin{center}
\begin{tabular}{l}
|\ifchilddocmanual|\\
|\input{\childdocname}|\\
|\||else|\\
\textit{document body with }|\input{|\textit{part}|}|\\
|\||fi|
\end{tabular}
\end{center}
%
The conditional |\ifchilddocmanual| is true whenever
a part to be included by |\input| is being compiled,
and the name of the part is stored in |\childdocname|.

%%%%%%%%%%%%%%%%%%%%%%%%%%%%%%%%%%%%%%%%
\DescribeMacro{\childdocby}
Each part to be included by |\input| should start with:
%
\begin{center}
\begin{tabular}{l}
|\input{childdoc.def}|\\
|\childdocby{|\textit{main}|}|\\
\end{tabular}
\end{center}
%
The directive |\childdocby| is similar to |\childdocof|
described in \secref{sec:include},
but the subsequent selection of content must be done manually.
To that end, both |\ifchilddoc| and |\ifchilddocmanual|
will be true upon processing of a part,
and the name of the part is stored in |\childdocname|.
Note that |\jobname| will be set to the filename of the current part
so that each part receives an individual |.aux| file
that does not interfere with the |.aux| file(s) of the main document.
This behaviour can be altered by the alternative form
|\childdocby[*]{|\textit{main}|}| (with a non-empty optional argument)
which uses the |.aux| file of the main document
by setting |\jobname| to \textit{main}.

%%%%%%%%%%%%%%%%%%%%%%%%%%%%%%%%%%%%%%%%%%%%%%%%%%%%%%%%%%%%%%%%%%%%%%%%%%%%%%%%
\subsection{Driver Development}
\label{sec:driver}

The \textsf{childdoc} mechanism can also be use for the development
of definition files such as \LaTeX{} styles or classes.
This case differs from the above setup with multiple parts
included by |\include| in that no |\includeonly| should be invoked.
This can be achieved by starting the include file
(before |\ProvidesPackage|) with:
%
\begin{center}
\begin{tabular}{l}
|\input{childdoc.def}|\\
|\childdocforward{|\textit{main}|}|\\
\end{tabular}
\end{center}
%
or alternatively with:
%
\begin{center}
\begin{tabular}{l}
|\input{childdoc.def}|\\
|\childdocby{|\textit{main}|}|\\
\end{tabular}
\end{center}
%
Both forms have slightly different effects as described above.
The main file is prepared as usual, see \secref{sec:include}.

%%%%%%%%%%%%%%%%%%%%%%%%%%%%%%%%%%%%%%%%%%%%%%%%%%%%%%%%%%%%%%%%%%%%%%%%%%%%%%%%
\subsection{Legacy Detection}
\label{sec:detection}

The directive |\childdocmain| in the main file can detect
whether the complete document or merely a child is to be compiled
even without using the directive |\childdocof|.
This method is deprecated because it is less robust
and there is no compelling reason to use it;
it is merely provided for backward compatibility
and it may be removed in future versions.

If the detection mechanism is to be used,
it is mandatory to correctly specify
the filename of the main file as the argument of |\childdocmain|:
%
\begin{center}
\begin{tabular}{l}
|\input{childdoc.def}|\\
|\childdocmain{|\textit{main}|}|\\
\end{tabular}
\end{center}
%
If |\jobname| does not match the argument \textit{main} of |\childdocmain|,
it is assumed that |\jobname| points to the child file to be compiled.
When using |\childdocmain| with the main file specified as argument,
it suffices to start a child file
with just |\input{|\textit{main}|}|
without loading of the package and using |\childdocof|.
If instead all processing is done
with the appropriate \textsf{childdoc} directives,
the argument of \textit{main} of |\childdocmain| can be empty.

An alternative version of the command line processing described
in \secref{sec:commandline} using the detection mechanism reads:
%
\begin{center}
|... -jobname "|\textit{target}|" "|[\textit{flags}]%
[|\def\jobname{|\textit{dest}|}|]|\input{|\textit{main}|}"|
\end{center}

%%%%%%%%%%%%%%%%%%%%%%%%%%%%%%%%%%%%%%%%%%%%%%%%%%%%%%%%%%%%%%%%%%%%%%%%%%%%%%%%
\subsection{Manual Code}
\label{sec:manual}

In case one cannot be certain whether the definitions file |childdoc.def|
is installed on the target \TeX{} distribution
and one prefers not to ship it,
it is conceivable to paste a few relevant commands into the sources.

To that end, drop all statements |\input{childdoc.def}|
and perform the replacements as outlined below.
Instead of |\childdocmain{|\textit{main}|}| add the following code
to the top of the main file:
%
\begin{center}
\begin{tabular}{l}
|\||ifdefined\childdocname\endinput\||fi\newif\ifchilddoc|\\
|\edef\childdocname{\scantokens\expandafter{\jobname\noexpand}}|\\
|\def\childdocmain{|\textit{main}|}\||ifx\childdocmain\childdocname\||else|\\
|\childdoctrue\includeonly{\childdocname}\let\jobname\childdocmain\||fi|\\
\end{tabular}
\end{center}
%
Instead of |\childdocof{|\textit{main}|}| just include the main file
at the top of each child file:
%
\begin{center}
|\input{|\textit{main}|}|
\end{center}
%
A simple redirection |\childdocforward{|\textit{dest}|}| is achieved by:
%
\begin{center}
|\def\jobname{|\textit{dest}|}\input{\jobname}|
\end{center}
%
The redirection with prefix
|\childdocforwardprefix[|\textit{prefix}|]{|\textit{dest}|}|
is accomplished by:
%
\begin{center}
\begin{tabular}{l}
|{\edef\jobname{\scantokens\expandafter{\jobname\noexpand}}|\\
|\def\redirectjob |\textit{prefix}|#1~~~{\gdef\jobname{|\textit{dest}|#1}}|\\
|\expandafter\redirectjob\jobname~~~}\input{\jobname}|
\end{tabular}
\end{center}

In an alternative approach,
child documents can be compiled by a specific command line
without additional code or specific definitions:
%
\begin{center}
|... -jobname "|\textit{target}|" "|[\textit{flags}]%
|\includeonly{|\textit{dest}|}\input{|\textit{main}|}"|
\end{center}
%

%%%%%%%%%%%%%%%%%%%%%%%%%%%%%%%%%%%%%%%%%%%%%%%%%%%%%%%%%%%%%%%%%%%%%%%%%%%%%%%%
%%%%%%%%%%%%%%%%%%%%%%%%%%%%%%%%%%%%%%%%%%%%%%%%%%%%%%%%%%%%%%%%%%%%%%%%%%%%%%%%
\section{Information}

%%%%%%%%%%%%%%%%%%%%%%%%%%%%%%%%%%%%%%%%%%%%%%%%%%%%%%%%%%%%%%%%%%%%%%%%%%%%%%%%
\subsection{Copyright}

Copyright \copyright{} 2017--2018 Niklas Beisert

This work may be distributed and/or modified under the
conditions of the \LaTeX{} Project Public License, either version 1.3
of this license or (at your option) any later version.
The latest version of this license is in
  \url{http://www.latex-project.org/lppl.txt}
and version 1.3 or later is part of all distributions of \LaTeX{}
version 2005/12/01 or later.

This work has the LPPL maintenance status `maintained'.

The Current Maintainer of this work is Niklas Beisert.

This work consists of the files |README.txt|, |childdoc.ins| and |childdoc.dtx|
as well as the derived files |childdoc.def|, |cdocsamp.tex|
with |cdocsch1.tex|, |cdocsch2.tex|, |cdocspt3.tex|, |cdocspt4.tex|,
|cdocsdrf.tex|, |cdocsfn1.tex|, |cdocsfn2.tex|
as well as |childdoc.pdf|.

%%%%%%%%%%%%%%%%%%%%%%%%%%%%%%%%%%%%%%%%%%%%%%%%%%%%%%%%%%%%%%%%%%%%%%%%%%%%%%%%
\subsection{Files and Installation}

The package consists of the files:
%
\begin{center}
\begin{tabular}{ll}
    |README.txt|   & readme file \\
    |childdoc.ins| & installation file \\
    |childdoc.dtx| & source file \\
    |childdoc.def| & definition file \\
    |cdocsamp.tex| & sample main file \\
    |cdocsch1.tex| & sample include file \\
    |cdocsch2.tex| & sample include file \\
    |cdocspt3.tex| & sample part file \\
    |cdocspt4.tex| & sample part file \\
    |cdocsdrf.tex| & sample redirection file \\
    |cdocsfn1.tex| & sample redirection file \\
    |cdocsfn2.tex| & sample redirection file \\
    |childdoc.pdf| & manual
\end{tabular}
\end{center}
%
The distribution consists of the files
|README.txt|, |childdoc.ins| and |childdoc.dtx|.
%
\begin{itemize}
\item
Run (pdf)\LaTeX{} on |childdoc.dtx|
to compile the manual |childdoc.pdf| (this file).
\item
Run \LaTeX{} on |childdoc.ins| to create the definitions file |childdoc.def|
and the sample |cdocsamp.tex| with include files
|cdocsch1.tex|, |cdocsch2.tex|, |cdocspt3.tex|, |cdocspt4.tex|,
|cdocsdrf.tex|, |cdocsfn1.tex|, |cdocsfn2.tex|.
Then copy the file |childdoc.def| to an appropriate directory of your \LaTeX{}
distribution, e.g.\ \textit{texmf-root}|/tex/latex/childdoc|.
\end{itemize}

%%%%%%%%%%%%%%%%%%%%%%%%%%%%%%%%%%%%%%%%%%%%%%%%%%%%%%%%%%%%%%%%%%%%%%%%%%%%%%%%
\subsection{Related CTAN Packages}

There are several other packages which offer a similar functionality:
%
\begin{itemize}
\item
The packages
\href{http://ctan.org/pkg/docmute}{\textsf{docmute}},
\href{http://ctan.org/pkg/includex}{\textsf{includex}} and
\href{http://ctan.org/pkg/standalone}{\textsf{standalone}}
provide commands to include only the document body of
a child file thus allowing both files to be compiled individually.
\item
The packages \href{http://ctan.org/pkg/subdocs}{\textsf{subdocs}}
and \href{http://ctan.org/pkg/subfiles}{\textsf{subfiles}}
provide structures in which the main and child documents can be
encapsulated and allowing them to be compiled individually.
The inclusion mechanism is different from the conventional |\include|.
\item
The package \href{http://ctan.org/pkg/combine}{\textsf{combine}}
is an elaborate solution to combine several documents into one.
\end{itemize}
%
See also the CTAN topic \href{http://ctan.org/topic/subdocs}{\textsf{subdocs}}
for further related packages.
The present package differs from the above solutions in that
a document structure constructed with the conventional |\include| mechanism
just needs two extra commands at the top of every file
such that all constituent files can be compiled individually.

%%%%%%%%%%%%%%%%%%%%%%%%%%%%%%%%%%%%%%%%%%%%%%%%%%%%%%%%%%%%%%%%%%%%%%%%%%%%%%%%
%\subsection{Feature Suggestions}
%
%The following is a list of features which may be useful for future
%versions of this package:
%%
%\begin{itemize}
%\item
%\ldots
%\end{itemize}

%%%%%%%%%%%%%%%%%%%%%%%%%%%%%%%%%%%%%%%%%%%%%%%%%%%%%%%%%%%%%%%%%%%%%%%%%%%%%%%%
\subsection{Revision History}

%%%%%%%%%%%%%%%%%%%%%%%%%%%%%%%%%%%%%%%%
\paragraph{v2.0:} 2018/12/30

\begin{itemize}
\item
immediate forward processing
\item
added |\childdocby| mechanism
\item
manual restructured
\end{itemize}

%%%%%%%%%%%%%%%%%%%%%%%%%%%%%%%%%%%%%%%%
\paragraph{v1.6:} 2018/01/17

\begin{itemize}
\item
application for development of include files
\item
corrections to manual
\end{itemize}

%%%%%%%%%%%%%%%%%%%%%%%%%%%%%%%%%%%%%%%%
\paragraph{v1.5:} 2017/05/21

\begin{itemize}
\item
more complete structuring introduced
\item
|\childdocof| introduced
\item
|\childdoc| renamed to |\childdocmain|
\item
|\childredirect| renamed to |\childdocforward| and |\childdocforwardprefix|
and functionality expanded
\end{itemize}

%%%%%%%%%%%%%%%%%%%%%%%%%%%%%%%%%%%%%%%%
\paragraph{v1.0:} 2017/04/27

\begin{itemize}
\item
manual and install package
\item
first version published on CTAN
\end{itemize}

%%%%%%%%%%%%%%%%%%%%%%%%%%%%%%%%%%%%%%%%
\paragraph{v0.6:} 2017/04/26

\begin{itemize}
\item
redirection mechanism added
\end{itemize}

%%%%%%%%%%%%%%%%%%%%%%%%%%%%%%%%%%%%%%%%
\paragraph{v0.5:} 2017/04/26

\begin{itemize}
\item
functionality in definition file
\end{itemize}


%%%%%%%%%%%%%%%%%%%%%%%%%%%%%%%%%%%%%%%%%%%%%%%%%%%%%%%%%%%%%%%%%%%%%%%%%%%%%%%%
%%%%%%%%%%%%%%%%%%%%%%%%%%%%%%%%%%%%%%%%%%%%%%%%%%%%%%%%%%%%%%%%%%%%%%%%%%%%%%%%
%%%%%%%%%%%%%%%%%%%%%%%%%%%%%%%%%%%%%%%%%%%%%%%%%%%%%%%%%%%%%%%%%%%%%%%%%%%%%%%%
\appendix

\settowidth\MacroIndent{\rmfamily\scriptsize 000\ }

 \DocInput{childdoc.dtx}

\end{document}
%</driver>
% \fi
%
% %%%%%%%%%%%%%%%%%%%%%%%%%%%%%%%%%%%%%%%%%%%%%%%%%%%%%%%%%%%%%%%%%%%%%%%%%%%%%%
% %%%%%%%%%%%%%%%%%%%%%%%%%%%%%%%%%%%%%%%%%%%%%%%%%%%%%%%%%%%%%%%%%%%%%%%%%%%%%%
% \section{Sample}
%\iffalse
%<*samplemain>
%\fi
%
% The following presents a sample document
% with two chapters, two parts, a title page,
% a compile flag as well as three forwarding files to set the flag.
% It consists of eight |.tex| files:
% \begin{center}
% \begin{tabular}{ll}
% |cdocsamp.tex|&main file\\
% |cdocsch1.tex|&include file for chapter 1\\
% |cdocsch2.tex|&include file for chapter 2\\
% |cdocspt3.tex|&include file for part 3\\
% |cdocspt4.tex|&include file for part 4\\
% |cdocsdrf.tex|&forwarding file for main file in draft mode\\
% |cdocsfi1.tex|&forwarding file for final version of chapter 1\\
% |cdocsfi2.tex|&forwarding file for final version of chapter 2\\
% \end{tabular}
% \end{center}
% Each of the eight files can be compiled directly by the \LaTeX{} compiler.
%
% %%%%%%%%%%%%%%%%%%%%%%%%%%%%%%%%%%%%%%
% \paragraph{Main File.}
%
% The main file is called |cdocsamp.tex|.
%
% Load the \textsf{childdoc} definitions and
% declare the filename for the main document:
%    \begin{macrocode}
\input{childdoc.def}
\childdocmain{}
%    \end{macrocode}

% Optional override for |\version| flag:
%    \begin{macrocode}
%%\ifchilddoc\else\providecommand{\version}{draft}\fi
%    \end{macrocode}

% Define the default values for the |\version| flag
% (|final| for the main file and |draft| for childs):
%    \begin{macrocode}
\ifchilddoc
\providecommand{\version}{draft}
\else
\providecommand{\version}{final}
\fi
%    \end{macrocode}

% Load the standard document class:
%    \begin{macrocode}
\documentclass[12pt]{article}
%    \end{macrocode}

% Start the document body:
%    \begin{macrocode}
\begin{document}
%    \end{macrocode}

% Declare a title page.
% Print title, part of document being processed and version flag:
%    \begin{macrocode}
\addtocounter{page}{-1}
\begin{center}
{\LARGE\bfseries{}childdoc example\par}
\vspace{1cm}
\ifchilddoc
\ifchilddocmanual part\else chapter\fi:
`\childdocname' of `\childdocjob'\par
\else
main document: `\childdocjob'\par
\fi
version: \version\par
\end{center}
\newpage
%    \end{macrocode}

% Manually include selected file,
% otherwise process as usual:
%    \begin{macrocode}
\ifchilddocmanual
\section*{part `\childdocname'}
\input{\childdocname}
\else
%    \end{macrocode}

% Include the two chapters:
%    \begin{macrocode}
\include{cdocsch1}
\include{cdocsch2}
%    \end{macrocode}

% Include the two parts unless only chapters should be displayed:
%    \begin{macrocode}
\ifchilddoc\else
\section{part three}
\input{cdocspt3}
\section{part four}
\input{cdocspt4}
\fi
%    \end{macrocode}

% Process as usual until here:
%    \begin{macrocode}
\fi
%    \end{macrocode}

% End of document body:
%    \begin{macrocode}
\end{document}
%    \end{macrocode}
%\iffalse
%</samplemain>
%\fi
%
% %%%%%%%%%%%%%%%%%%%%%%%%%%%%%%%%%%%%%%
% \paragraph{Chapter Include Files.}
%
% The include files are called |cdocsch1.tex| and |cdocsch2.tex|.
%
%\iffalse
%<*samplechap1|samplechap2>
%\fi

% Optional override for |\version| flag:
%    \begin{macrocode}
%%\providecommand{\version}{final}
%    \end{macrocode}

% Include the main document:
%    \begin{macrocode}
\input{childdoc.def}
\childdocof{cdocsamp}
%    \end{macrocode}

%\iffalse
%</samplechap1|samplechap2>
%\fi
%
%\iffalse
%<*samplechap1>
%\fi
% Some text for chapter 1:
%    \begin{macrocode}
\section{one}
some text in chapter one
%    \end{macrocode}

%\iffalse
%</samplechap1>
%\fi
% Some text for chapter 2:
%\iffalse
%<*samplechap2>
%\fi
%    \begin{macrocode}
\section{two}
more text in chapter two
%    \end{macrocode}

%\iffalse
%</samplechap2>
%\fi
%
% %%%%%%%%%%%%%%%%%%%%%%%%%%%%%%%%%%%%%%
% \paragraph{Part Include Files.}
%
% The include files are called |cdocspt3.tex| and |cdocspt4.tex|.
%
%\iffalse
%<*samplepart3|samplepart4>
%\fi

% Optional override for |\version| flag:
%    \begin{macrocode}
%%\providecommand{\version}{final}
%    \end{macrocode}

% Include the main document:
%    \begin{macrocode}
\input{childdoc.def}
\childdocby{cdocsamp}
%    \end{macrocode}

%\iffalse
%</samplepart3|samplepart4>
%\fi
%
%\iffalse
%<*samplepart3>
%\fi
% Some text for part 3:
%    \begin{macrocode}
some text in part three
%    \end{macrocode}

%\iffalse
%</samplepart3>
%\fi
% Some text for part 4:
%\iffalse
%<*samplepart4>
%\fi
%    \begin{macrocode}
more text in part four
%    \end{macrocode}

%\iffalse
%</samplepart4>
%\fi
%
% %%%%%%%%%%%%%%%%%%%%%%%%%%%%%%%%%%%%%%
% \paragraph{Forwarding for a Complete Draft.}
%
% The following forwarding file |cdocsdrf.tex|
% compiles the main document in draft mode:
%\iffalse
%<*sampledraft>
%\fi
%    \begin{macrocode}
\def\version{draft}
\input{childdoc.def}
\childdocforward{cdocsamp}
%    \end{macrocode}

%\iffalse
%</sampledraft>
%\fi
%
% %%%%%%%%%%%%%%%%%%%%%%%%%%%%%%%%%%%%%%
% \paragraph{Forwarding for Final Version of the Chapters.}
%
% The following forwarding files |cdocsfn1.tex| and |cdocsfn2.tex|
% (with identical content)
% compile the final versions of the child documents
% |cdocsch1.tex| and |cdocsch2.tex|, respectively:
%\iffalse
%<*samplefinal>
%\fi
%    \begin{macrocode}
\def\version{final}
\input{childdoc.def}
\childdocforwardprefix[cdocsamp]{cdocsfn}{cdocsch}
%    \end{macrocode}

%\iffalse
%</samplefinal>
%\fi
%
% %%%%%%%%%%%%%%%%%%%%%%%%%%%%%%%%%%%%%%
% \paragraph{Command Line Processing.}
%
% The following three command lines generate the output files
% |cdocscld|, |cdocscl1| and |cdocscl2|
% which should be identical to
% |cdocsdrf|, |cdocsch1| and |cdocsfn2|, respectively:
% \begin{center}
% \begin{tabular}{l}
% |latex -jobname cdocscld \|\\
% |  "\def\version{draft}\input{childdoc.def}\childdocforward{cdocsamp}"|\\
% |latex -jobname cdocscl1 \|\\
% |  "\input{childdoc.def}\childdocforward[cdocsamp]{cdocsch1}"|\\
% |latex -jobname cdocscl2 \|\\
% |  "\def\version{final}\input{childdoc.def}\childdocforward{cdocsch2}"|
% \end{tabular}
% \end{center}
% Note that the trailing backslash on each first line
% merely continues the input to the second line
% (for convenient cut ant paste).
% Furthermore, the command |latex| can be replaced by any
% of its alternative versions such as |pdflatex|.
%
% %%%%%%%%%%%%%%%%%%%%%%%%%%%%%%%%%%%%%%%%%%%%%%%%%%%%%%%%%%%%%%%%%%%%%%%%%%%%%%
% %%%%%%%%%%%%%%%%%%%%%%%%%%%%%%%%%%%%%%%%%%%%%%%%%%%%%%%%%%%%%%%%%%%%%%%%%%%%%%
% \section{Implementation}
%\iffalse
%<*package>
%\fi
%
% This section describes the definitions file |childdoc.def|.

% The definitions cannot be loaded using |\usepackage| or |\RequirePackage|
% which has a mechanism to prevent loading a style file more than once.
% When loading the definitions by means of |\input|
% multiple instances have to be prevented manually:
%\iffalse
%This code needs to be before the `\ProvidesFile' directive
%which is defined at the beginning of this file.
%Therefore it is also placed there and commented out here.
%</package>
%<*discard>
%\fi
%    \begin{macrocode}
\ifdefined\childdocmain\endinput\fi
%    \end{macrocode}
%\iffalse
%</discard>
%<*package>
%\fi
%
% \macro{\ifchilddoc}
% \macro{\ifchilddocmanual}
% The conditional |\ifchilddoc| tells whether a
% child (true) or main (false) document is being compiled.
% The conditional |\ifchilddocmanual| tells whether
% the |\includeonly| mechanism is used (false) or
% the selection of child files must be performed manually (true).
% The definitions initialise to false:
%    \begin{macrocode}
\newif\ifchilddoc
\newif\ifchilddocmanual
%    \end{macrocode}

% \macro{\childdocname}
% \macro{\childdocjob}
% The macro |\childdocname| stores the name of the main document
% to be compiled. The macro |\childdocjob| stores the name of
% the document on which the \LaTeX{} compiler was originally invoked.
% The content of |\jobname| cannot be compared
% to filenames specified in the source due to different catcodes.
% The following code rescans |\jobname|, stores the result
% in |\childdocname| and saves a copy in |\childdocjob|:
%    \begin{macrocode}
\edef\childdocname{\scantokens\expandafter{\jobname\noexpand}}
\let\childdocjob\childdocname
%    \end{macrocode}

% \macro{\childdocdisable}
% The macro |\childdocdisable| prevents the main file
% from being processed more than once.
% At this stage, the main document command |\childdocmain|
% is assumed to be called once again where it should do nothing.
% Any subsequent call to it should prevent
% a secondary processing of the main document
% It overwrites the forwarding commands
% |\childdocof| and |\childdocforward|
% with empty macros to prevent further inclusions of the main document:
%    \begin{macrocode}
\newcommand{\childdocdisable}
{
  \renewcommand{\childdocmain}[1]{\renewcommand{\childdocmain}[1]{\endinput}}
  \renewcommand{\childdocof}[1]{}
  \renewcommand{\childdocby}[2][]{}
  \renewcommand{\childdocforward}[2][]{}
  \renewcommand{\childdocdisable}{}
}
%    \end{macrocode}

% \macro{\childdocmain}
% The macro |\childdocmain| is to be called at the top of the main file
% with nothing or the main filename (without extension) as argument.
% First, it breaks loops.
% If the argument is not empty and does not match |\childdocname|
% (which is set by the first inclusion of |childdoc.def|),
% |\ifchilddoc| is set to true, |\includeonly| is applied to the child file
% and |\jobname| is set to the main file
% (for proper handling of |.aux| files):
%    \begin{macrocode}
\newcommand{\childdocmain}[1]
{
  \childdocdisable\childdocmain{}
  \if?#1?\else
    \begingroup
      \def\childdoctmp{#1}
      \ifx\childdoctmp\childdocname
        \def\childdoctmp{}
      \else
        \def\childdoctmp
        {
          \childdoctrue
          \includeonly{\childdocname}
          \def\childdocjob{#1}
          \def\jobname{#1}
        }
      \fi
      \expandafter
    \endgroup
    \childdoctmp
  \fi
}
%    \end{macrocode}

% \macro{\childdocof}
% The command |\childdocof| redirects
% compilation to the main file |#1|.
%    \begin{macrocode}
\newcommand{\childdocof}[1]
{
  \childdocdisable
  \childdoctrue
  \includeonly{\childdocname}
  \def\jobname{#1}
  \def\childdocjob{#1}
  \input{#1}
}
%    \end{macrocode}

% \macro{\childdocby}
% The command |\childdocby| ....
%    \begin{macrocode}
\newcommand{\childdocby}[2][]
{
  \childdocdisable
  \childdoctrue
  \childdocmanualtrue
  \if?#1?\else
    \def\jobname{#2}
  \fi
  \def\childdocjob{#2}
  \input{#2}
  \endinput
}
%    \end{macrocode}

% \macro{\childdocforward}
% The command |\childdocforward| redirects
% compilation to the main file or
% (if the optional argument is given) a child file.
% Parameters are set as if the main file
% or a child file starting with |\childdocof| was compiled.
% Then compilation is handed over to the main file:
%    \begin{macrocode}
\newcommand{\childdocforward}[2][]
{
  \begingroup
    \if?#1?
      \def\childdoctmp
      {
        \def\childdocname{#2}
        \def\childdocjob{#2}
        \def\jobname{#2}
        \input{#2}
        \endinput
      }
    \else
      \def\childdoctmp
      {
        \childdocdisable
        \def\childdocname{#2}
        \childdoctrue
        \includeonly{#2}
        \def\childdocjob{#1}
        \def\jobname{#1}
        \input{#1}
        \endinput
      }
    \fi
    \expandafter
  \endgroup
  \childdoctmp
}
%    \end{macrocode}

% \macro{\childdocforwardprefix}
% The command |\childdocforwardprefix| redirects
% compilation to the main or a child file by means of a pattern.
% The prefix |#1| in the current filename is replaced by |#2|
% and the suffix of the current filename is kept
% (it is assumed that the filename does not contain the substring `|~~~|'
% which is used as a delimiter).
% Compilation is handed over to the new file by |\childdocforward|:
%    \begin{macrocode}
\newcommand{\childdocforwardprefix}[3][]
{
  \begingroup
    \def\childdocextract #2##1~~~{\def\childdoctmp{\childdocforward[#1]{#3##1}}}
    \expandafter\childdocextract\childdocname~~~
    \expandafter
  \endgroup
  \childdoctmp
}
%    \end{macrocode}

% \macro{\childdoc}
% The deprecated macro |\childdoc| is a legacy version of |\childdocmain|:
%    \begin{macrocode}
\newcommand{\childdoc}{\childdocmain}
%    \end{macrocode}

% \macro{\childdocredirect}
% The deprecated macro |\childdocredirect| is a legacy version
% of |\childdocforward| and |\childdocforwardprefix|:
%    \begin{macrocode}
\newcommand{\childdocredirect}[2][]
{
  \begingroup
    \if?#1?
      \def\childdoctmp{\childdocforward{#2}}
    \else
      \def\childdoctmp{\childdocforwardprefix{#1}{#2}}
    \fi
    \expandafter
  \endgroup
  \childdoctmp
}
%    \end{macrocode}

%\iffalse
%</package>
%\fi
%
\endinput
|
and perform the replacements as outlined below.
Instead of |\childdocmain{|\textit{main}|}| add the following code
to the top of the main file:
%
\begin{center}
\begin{tabular}{l}
|\||ifdefined\childdocname\endinput\||fi\newif\ifchilddoc|\\
|\edef\childdocname{\scantokens\expandafter{\jobname\noexpand}}|\\
|\def\childdocmain{|\textit{main}|}\||ifx\childdocmain\childdocname\||else|\\
|\childdoctrue\includeonly{\childdocname}\let\jobname\childdocmain\||fi|\\
\end{tabular}
\end{center}
%
Instead of |\childdocof{|\textit{main}|}| just include the main file
at the top of each child file:
%
\begin{center}
|\input{|\textit{main}|}|
\end{center}
%
A simple redirection |\childdocforward{|\textit{dest}|}| is achieved by:
%
\begin{center}
|\def\jobname{|\textit{dest}|}\input{\jobname}|
\end{center}
%
The redirection with prefix
|\childdocforwardprefix[|\textit{prefix}|]{|\textit{dest}|}|
is accomplished by:
%
\begin{center}
\begin{tabular}{l}
|{\edef\jobname{\scantokens\expandafter{\jobname\noexpand}}|\\
|\def\redirectjob |\textit{prefix}|#1~~~{\gdef\jobname{|\textit{dest}|#1}}|\\
|\expandafter\redirectjob\jobname~~~}\input{\jobname}|
\end{tabular}
\end{center}

In an alternative approach,
child documents can be compiled by a specific command line
without additional code or specific definitions:
%
\begin{center}
|... -jobname "|\textit{target}|" "|[\textit{flags}]%
|\includeonly{|\textit{dest}|}\input{|\textit{main}|}"|
\end{center}
%

%%%%%%%%%%%%%%%%%%%%%%%%%%%%%%%%%%%%%%%%%%%%%%%%%%%%%%%%%%%%%%%%%%%%%%%%%%%%%%%%
%%%%%%%%%%%%%%%%%%%%%%%%%%%%%%%%%%%%%%%%%%%%%%%%%%%%%%%%%%%%%%%%%%%%%%%%%%%%%%%%
\section{Information}

%%%%%%%%%%%%%%%%%%%%%%%%%%%%%%%%%%%%%%%%%%%%%%%%%%%%%%%%%%%%%%%%%%%%%%%%%%%%%%%%
\subsection{Copyright}

Copyright \copyright{} 2017--2018 Niklas Beisert

This work may be distributed and/or modified under the
conditions of the \LaTeX{} Project Public License, either version 1.3
of this license or (at your option) any later version.
The latest version of this license is in
  \url{http://www.latex-project.org/lppl.txt}
and version 1.3 or later is part of all distributions of \LaTeX{}
version 2005/12/01 or later.

This work has the LPPL maintenance status `maintained'.

The Current Maintainer of this work is Niklas Beisert.

This work consists of the files |README.txt|, |childdoc.ins| and |childdoc.dtx|
as well as the derived files |childdoc.def|, |cdocsamp.tex|
with |cdocsch1.tex|, |cdocsch2.tex|, |cdocspt3.tex|, |cdocspt4.tex|,
|cdocsdrf.tex|, |cdocsfn1.tex|, |cdocsfn2.tex|
as well as |childdoc.pdf|.

%%%%%%%%%%%%%%%%%%%%%%%%%%%%%%%%%%%%%%%%%%%%%%%%%%%%%%%%%%%%%%%%%%%%%%%%%%%%%%%%
\subsection{Files and Installation}

The package consists of the files:
%
\begin{center}
\begin{tabular}{ll}
    |README.txt|   & readme file \\
    |childdoc.ins| & installation file \\
    |childdoc.dtx| & source file \\
    |childdoc.def| & definition file \\
    |cdocsamp.tex| & sample main file \\
    |cdocsch1.tex| & sample include file \\
    |cdocsch2.tex| & sample include file \\
    |cdocspt3.tex| & sample part file \\
    |cdocspt4.tex| & sample part file \\
    |cdocsdrf.tex| & sample redirection file \\
    |cdocsfn1.tex| & sample redirection file \\
    |cdocsfn2.tex| & sample redirection file \\
    |childdoc.pdf| & manual
\end{tabular}
\end{center}
%
The distribution consists of the files
|README.txt|, |childdoc.ins| and |childdoc.dtx|.
%
\begin{itemize}
\item
Run (pdf)\LaTeX{} on |childdoc.dtx|
to compile the manual |childdoc.pdf| (this file).
\item
Run \LaTeX{} on |childdoc.ins| to create the definitions file |childdoc.def|
and the sample |cdocsamp.tex| with include files
|cdocsch1.tex|, |cdocsch2.tex|, |cdocspt3.tex|, |cdocspt4.tex|,
|cdocsdrf.tex|, |cdocsfn1.tex|, |cdocsfn2.tex|.
Then copy the file |childdoc.def| to an appropriate directory of your \LaTeX{}
distribution, e.g.\ \textit{texmf-root}|/tex/latex/childdoc|.
\end{itemize}

%%%%%%%%%%%%%%%%%%%%%%%%%%%%%%%%%%%%%%%%%%%%%%%%%%%%%%%%%%%%%%%%%%%%%%%%%%%%%%%%
\subsection{Related CTAN Packages}

There are several other packages which offer a similar functionality:
%
\begin{itemize}
\item
The packages
\href{http://ctan.org/pkg/docmute}{\textsf{docmute}},
\href{http://ctan.org/pkg/includex}{\textsf{includex}} and
\href{http://ctan.org/pkg/standalone}{\textsf{standalone}}
provide commands to include only the document body of
a child file thus allowing both files to be compiled individually.
\item
The packages \href{http://ctan.org/pkg/subdocs}{\textsf{subdocs}}
and \href{http://ctan.org/pkg/subfiles}{\textsf{subfiles}}
provide structures in which the main and child documents can be
encapsulated and allowing them to be compiled individually.
The inclusion mechanism is different from the conventional |\include|.
\item
The package \href{http://ctan.org/pkg/combine}{\textsf{combine}}
is an elaborate solution to combine several documents into one.
\end{itemize}
%
See also the CTAN topic \href{http://ctan.org/topic/subdocs}{\textsf{subdocs}}
for further related packages.
The present package differs from the above solutions in that
a document structure constructed with the conventional |\include| mechanism
just needs two extra commands at the top of every file
such that all constituent files can be compiled individually.

%%%%%%%%%%%%%%%%%%%%%%%%%%%%%%%%%%%%%%%%%%%%%%%%%%%%%%%%%%%%%%%%%%%%%%%%%%%%%%%%
%\subsection{Feature Suggestions}
%
%The following is a list of features which may be useful for future
%versions of this package:
%%
%\begin{itemize}
%\item
%\ldots
%\end{itemize}

%%%%%%%%%%%%%%%%%%%%%%%%%%%%%%%%%%%%%%%%%%%%%%%%%%%%%%%%%%%%%%%%%%%%%%%%%%%%%%%%
\subsection{Revision History}

%%%%%%%%%%%%%%%%%%%%%%%%%%%%%%%%%%%%%%%%
\paragraph{v2.0:} 2018/12/30

\begin{itemize}
\item
immediate forward processing
\item
added |\childdocby| mechanism
\item
manual restructured
\end{itemize}

%%%%%%%%%%%%%%%%%%%%%%%%%%%%%%%%%%%%%%%%
\paragraph{v1.6:} 2018/01/17

\begin{itemize}
\item
application for development of include files
\item
corrections to manual
\end{itemize}

%%%%%%%%%%%%%%%%%%%%%%%%%%%%%%%%%%%%%%%%
\paragraph{v1.5:} 2017/05/21

\begin{itemize}
\item
more complete structuring introduced
\item
|\childdocof| introduced
\item
|\childdoc| renamed to |\childdocmain|
\item
|\childredirect| renamed to |\childdocforward| and |\childdocforwardprefix|
and functionality expanded
\end{itemize}

%%%%%%%%%%%%%%%%%%%%%%%%%%%%%%%%%%%%%%%%
\paragraph{v1.0:} 2017/04/27

\begin{itemize}
\item
manual and install package
\item
first version published on CTAN
\end{itemize}

%%%%%%%%%%%%%%%%%%%%%%%%%%%%%%%%%%%%%%%%
\paragraph{v0.6:} 2017/04/26

\begin{itemize}
\item
redirection mechanism added
\end{itemize}

%%%%%%%%%%%%%%%%%%%%%%%%%%%%%%%%%%%%%%%%
\paragraph{v0.5:} 2017/04/26

\begin{itemize}
\item
functionality in definition file
\end{itemize}


%%%%%%%%%%%%%%%%%%%%%%%%%%%%%%%%%%%%%%%%%%%%%%%%%%%%%%%%%%%%%%%%%%%%%%%%%%%%%%%%
%%%%%%%%%%%%%%%%%%%%%%%%%%%%%%%%%%%%%%%%%%%%%%%%%%%%%%%%%%%%%%%%%%%%%%%%%%%%%%%%
%%%%%%%%%%%%%%%%%%%%%%%%%%%%%%%%%%%%%%%%%%%%%%%%%%%%%%%%%%%%%%%%%%%%%%%%%%%%%%%%
\appendix

\settowidth\MacroIndent{\rmfamily\scriptsize 000\ }

 \DocInput{childdoc.dtx}

\end{document}
%</driver>
% \fi
%
% %%%%%%%%%%%%%%%%%%%%%%%%%%%%%%%%%%%%%%%%%%%%%%%%%%%%%%%%%%%%%%%%%%%%%%%%%%%%%%
% %%%%%%%%%%%%%%%%%%%%%%%%%%%%%%%%%%%%%%%%%%%%%%%%%%%%%%%%%%%%%%%%%%%%%%%%%%%%%%
% \section{Sample}
%\iffalse
%<*samplemain>
%\fi
%
% The following presents a sample document
% with two chapters, two parts, a title page,
% a compile flag as well as three forwarding files to set the flag.
% It consists of eight |.tex| files:
% \begin{center}
% \begin{tabular}{ll}
% |cdocsamp.tex|&main file\\
% |cdocsch1.tex|&include file for chapter 1\\
% |cdocsch2.tex|&include file for chapter 2\\
% |cdocspt3.tex|&include file for part 3\\
% |cdocspt4.tex|&include file for part 4\\
% |cdocsdrf.tex|&forwarding file for main file in draft mode\\
% |cdocsfi1.tex|&forwarding file for final version of chapter 1\\
% |cdocsfi2.tex|&forwarding file for final version of chapter 2\\
% \end{tabular}
% \end{center}
% Each of the eight files can be compiled directly by the \LaTeX{} compiler.
%
% %%%%%%%%%%%%%%%%%%%%%%%%%%%%%%%%%%%%%%
% \paragraph{Main File.}
%
% The main file is called |cdocsamp.tex|.
%
% Load the \textsf{childdoc} definitions and
% declare the filename for the main document:
%    \begin{macrocode}
% \iffalse
%
% childdoc.dtx Copyright (C) 2017-2018 Niklas Beisert
%
% This work may be distributed and/or modified under the
% conditions of the LaTeX Project Public License, either version 1.3
% of this license or (at your option) any later version.
% The latest version of this license is in
%   http://www.latex-project.org/lppl.txt
% and version 1.3 or later is part of all distributions of LaTeX
% version 2005/12/01 or later.
%
% This work has the LPPL maintenance status `maintained'.
%
% The Current Maintainer of this work is Niklas Beisert.
%
% This work consists of the files childdoc.dtx and childdoc.ins
% and the derived files childdoc.def and cdocsamp.tex with
% cdocsch1.tex, cdocsch2.tex, cdocsdrf.tex, cdocsfn1.tex, cdocsfn2.tex.
%
%<package>\ifdefined\childdocmain\endinput\fi
%<package>\ProvidesFile{childdoc.def}[2018/12/30 v2.0 child document driver]
%<samplemain>\ProvidesFile{cdocsamp.tex}[2018/12/30 v2.0 sample for childdoc]
%<*driver>
%\ProvidesFile{childdoc.drv}[2018/12/30 v2.0 childdoc reference manual file]
\PassOptionsToClass{10pt,a4paper}{article}
\documentclass{ltxdoc}

\usepackage[margin=35mm]{geometry}
\usepackage{hyperref}
\usepackage{hyperxmp}
\usepackage[usenames]{color}

\hypersetup{colorlinks=true}
\hypersetup{pdfstartview=FitH}
\hypersetup{pdfpagemode=UseNone}
\hypersetup{pdfsource={}}
\hypersetup{pdflang={en-UK}}
\hypersetup{pdfcopyright={Copyright 2017-2018 Niklas Beisert.
  This work may be distributed and/or modified under the
  conditions of the LaTeX Project Public License, either version 1.3
  of this license or (at your option) any later version.}}
\hypersetup{pdflicenseurl={http://www.latex-project.org/lppl.txt}}
\hypersetup{pdfcontactaddress={ETH Zurich, ITP, HIT K,
  Wolfgang-Pauli-Strasse 27}}
\hypersetup{pdfcontactpostcode={8093}}
\hypersetup{pdfcontactcity={Zurich}}
\hypersetup{pdfcontactcountry={Switzerland}}
\hypersetup{pdfcontactemail={nbeisert@itp.phys.ethz.ch}}
\hypersetup{pdfcontacturl={http://people.phys.ethz.ch/\xmptilde nbeisert/}}

\newcommand{\secref}[1]{\hyperref[#1]{section \ref*{#1}}}

\parskip1ex
\parindent0pt
\let\olditemize\itemize
\def\itemize{\olditemize\parskip0pt}

\begin{document}

\title{The \textsf{childdoc} Package}
\hypersetup{pdftitle={The childdoc Package}}
\author{Niklas Beisert\\[2ex]
  Institut f\"ur Theoretische Physik\\
  Eidgen\"ossische Technische Hochschule Z\"urich\\
  Wolfgang-Pauli-Strasse 27, 8093 Z\"urich, Switzerland\\[1ex]
  \href{mailto:nbeisert@itp.phys.ethz.ch}
  {\texttt{nbeisert@itp.phys.ethz.ch}}}
\hypersetup{pdfauthor={Niklas Beisert}}
\hypersetup{pdfsubject={Manual for the LaTeX2e Package childdoc}}
\date{30 December 2018, \textsf{v2.0}}
\maketitle

\begin{abstract}\noindent
\textsf{childdoc} is a \LaTeXe{} package
that enables the direct compilation
of document sections included by |\include|
to individual files.
\end{abstract}

\begingroup
\parskip0ex
\tableofcontents
\endgroup

%%%%%%%%%%%%%%%%%%%%%%%%%%%%%%%%%%%%%%%%%%%%%%%%%%%%%%%%%%%%%%%%%%%%%%%%%%%%%%%%
%%%%%%%%%%%%%%%%%%%%%%%%%%%%%%%%%%%%%%%%%%%%%%%%%%%%%%%%%%%%%%%%%%%%%%%%%%%%%%%%
\section{Introduction}

\LaTeX{} provides a mechanism to structure a large document (such as a book)
into a main file and several child files (containing the chapters)
using the |\include| command.
This mechanism is beneficial for documents
which span hundreds of pages in order to
make the source file(s) more manageable.
Moreover, compilation can be restricted to
selected child files by means of the |\includeonly| command.
The latter feature can be used to reduce the compilation time while editing
(this was significantly more useful in the earlier days of \LaTeX{})
or to generate a smaller document which is easier to navigate.
Another application of |\includeonly| is to generate
documents consisting of selected parts of the complete document.

However, there are a few drawbacks of the plain |\include| mechanism:
\begin{itemize}
\item
The child files cannot be compiled on their own,
they can only be compiled via the main file.
A naive editing environment
(such as a text editor with an option
to have the current file processed by \LaTeX)
may require one to switch to the main file before compiling;
attempting to compile the child file produces errors.
\item
The main file must be modified (each time)
to adjust the |\includeonly| command
to the present needs. This easily leaves the main file in a messy state.
\item
The generated document will always carry the filename
of the main document. This is inconvenient if
several child files are to be compiled and
to be kept for distribution.
\end{itemize}

The present package provides a simple interface
to make child files individually compilable by \LaTeX{}.
Compiling a child file then has the same effect as compiling
the main file with an |\includeonly| command
to select the appropriate child.
Moreover the generated document will carry the name of the child
rather than the main file.
This resolves all three above issues.

This feature is meant to make the editing of books,
thesis documents and lecture notes somewhat more convenient.
However, the package can also be used efficiently for
composing a series of documents (such as exercise sheets)
which are typically distributed individually.
It then assists the author in generating the individual documents
(potentially in different versions)
as well as a document containing the collected series.
Another application is in developing style files
or other kinds of included material
where compilation of the style file could redirect
to a sample or test file.

%%%%%%%%%%%%%%%%%%%%%%%%%%%%%%%%%%%%%%%%%%%%%%%%%%%%%%%%%%%%%%%%%%%%%%%%%%%%%%%%
%%%%%%%%%%%%%%%%%%%%%%%%%%%%%%%%%%%%%%%%%%%%%%%%%%%%%%%%%%%%%%%%%%%%%%%%%%%%%%%%
\section{Usage}

First of all, the package \textsf{childdoc} is \emph{not} a standard
\LaTeXe{} |.sty| style file! Therefore it needs to be invoked in
a non-standard way.

%%%%%%%%%%%%%%%%%%%%%%%%%%%%%%%%%%%%%%%%%%%%%%%%%%%%%%%%%%%%%%%%%%%%%%%%%%%%%%%%
\subsection{Included Files}
\label{sec:include}

%%%%%%%%%%%%%%%%%%%%%%%%%%%%%%%%%%%%%%%%
\DescribeMacro{\childdocmain}
To use the package, add the commands
\begin{center}
\begin{tabular}{l}
|\input{childdoc.def}|\\
|\childdocmain{}|\\
\end{tabular}
\end{center}
at the very top of the main \LaTeX{} file,
in particular \emph{before} the |\documentclass| statement!
The argument of |\childdocmain| should be left empty
(but it must be present).

%%%%%%%%%%%%%%%%%%%%%%%%%%%%%%%%%%%%%%%%
\DescribeMacro{\childdocof}
Furthermore, add the commands
\begin{center}
\begin{tabular}{l}
|\input{childdoc.def}|\\
|\childdocof{|\textit{main}|}|\\
\end{tabular}
\end{center}
at the top of every child file \textit{child}
which is included by |\include{|\textit{child}|}|
from within the main file
(or at least for those files to be compiled individually).
The argument \textit{main} must be the filename of the main file.

There are a couple of
considerations in setting up the main and child documents:

%%%%%%%%%%%%%%%%%%%%%%%%%%%%%%%%%%%%%%%%
\paragraph{Restrictions.}

Please note the following restrictions:
\begin{itemize}
\item
|\childdocmain| must be called with one argument \textit{main}
to ensure compatibility with earlier version of the package.
It must either be empty (|\childdocmain{}|)
or precisely match the filename of the main file in which it is specified.
See \secref{sec:detection} for further information.
\item
The filename \textit{main} must be specified without the |.tex| extension.
\item
The filename \textit{main} is case sensitive
(even in case-insensitive file systems)
due to internal string comparison.
\item
The argument \textit{main} should be fully expanded, it cannot be a macro.
\item
Subdirectories and special characters should be avoided in filenames.
\item
The command |\childdocmain{|\textit{main}|}| must be followed by a whitespace.
It should not be followed immediately by another command
or by a comment mark `|%|'.
This is because the \TeX{} parser reads the token immediately following
the argument of |\childdocmain| and puts it
at the beginning of every child section;
however, a white\-space is ignored.
\end{itemize}

%%%%%%%%%%%%%%%%%%%%%%%%%%%%%%%%%%%%%%%%
\paragraph{Content of Main File.}

It is advisable to place all content in the child files included by |\include|.
Any output contained in the main file will appear in all child documents
unless suppressed manually;
it cannot be suppressed automatically by the |\includeonly| directive
and thus should normally be avoided.
A method to include some content in the main file
by means of conditional processing is described in \secref{sec:conditional}.

%%%%%%%%%%%%%%%%%%%%%%%%%%%%%%%%%%%%%%%%
\paragraph{Page Numbering.}

When only a part of the document is compiled,
the appropriate numbering of pages
(as well as other status parameters)
is determined from the |.aux| files.
The latter contain information from previous passes.
However this information needs to propagate through
all intermediate child documents.
Therefore the page numbering in child documents may well
be inconsistent until the complete document is compiled at least once.

A useful (if unconventional) way to always ensure a consistent
page numbering is to restart the numbering in each child document
and denote the pages by `\textit{child}|.|\textit{page}'
where \textit{child} represents the chapter/section number of the child file.
This can be achieved by the command
|\numberwithin{page}{|\textit{child}|}|
of the \textsf{amsmath} package
where \textit{child} can be |chapter| or |section|
depending on the chosen structuring.
Alternatively, one can modify the macro |\thepage| appropriately
and reset the counter |page| at the start of each child file.

%%%%%%%%%%%%%%%%%%%%%%%%%%%%%%%%%%%%%%%%%%%%%%%%%%%%%%%%%%%%%%%%%%%%%%%%%%%%%%%%
\subsection{Conditional Processing}
\label{sec:conditional}

The package provides a mechanism to compile different versions
of a document. To customise the versions further some conditional processing
can come in handy to distinguish which version is being compiled.
The package provides two macros to describe the compilation context:

%%%%%%%%%%%%%%%%%%%%%%%%%%%%%%%%%%%%%%%%
\DescribeMacro{\ifchilddoc}
The conditional |\ifchilddoc| distinguishes between the compilation of
child documents and the main document:
%
\begin{center}
|\ifchilddoc |\textit{child-code}| |[|\||else |\textit{main-code}]| \||fi|
\end{center}

%%%%%%%%%%%%%%%%%%%%%%%%%%%%%%%%%%%%%%%%
\DescribeMacro{\childdocname}
\DescribeMacro{\childdocjob}
The macro |\childdocname| contains the filename (without extension)
of the main or child file being processed.
Note that |\childdocjob| will always contain the name of the main file.

%%%%%%%%%%%%%%%%%%%%%%%%%%%%%%%%%%%%%%%%
\paragraph{Title Page.}

Conditional processing can be used to include a title or banner page
in the main document when proper precautions are taken.
Importantly, the code in the main file should ensure that the page counter
(as well as other status parameters which are stored in the |.aux| files)
takes the same value after the conditional processing.
Otherwise the page numbers may take divergent values
depending on which part is compiled.

For example, a title page could be declared by:
%
\begin{center}
\begin{tabular}{l}
|\ifchilddoc\||else|\\
|\addtocounter{page}{-1}|\\
\textit{code for title page}\\
|\newpage|\\
|\||fi|
\end{tabular}
\end{center}
%
A banner page for the child documents can be generated by:
%
\begin{center}
\begin{tabular}{l}
|\ifchilddoc|\\
|\addtocounter{page}{-1}|\\
\textit{code for banner page}\\
|\newpage|\\
|\||fi|
\end{tabular}
\end{center}
%
Here one could write a message such as:
\begin{center}
|This is the part \childdocname{} of \childdocjob{}.|
\end{center}

%%%%%%%%%%%%%%%%%%%%%%%%%%%%%%%%%%%%%%%%%%%%%%%%%%%%%%%%%%%%%%%%%%%%%%%%%%%%%%%%
\subsection{Flags}
\label{sec:flags}

The package makes it easy to generate different versions
of the main or child documents.
To this end compilation flags can be defined
and assigned different default values.
They will be particularly useful in conjunction
with the forwarding mechanism described in \secref{sec:forward}.

For example, it may be useful to have a flag |\version|
which can be set to |draft| or |final|.
The document source will contain some conditional code
depending on the value of |\version|.
Suppose further, the flag should default to |final| for the main file
and to |draft| for child files
which is a natural assignment for editing the document.
This is achieved by placing the following code
in the preamble of the main document
(below the |\childdocmain| directive):
%
\begin{center}
\begin{tabular}{l}
|\ifchilddoc|\\
|\providecommand{\version}{draft}|\\
|\||else|\\
|\providecommand{\version}{final}|\\
|\||fi|
\end{tabular}
\end{center}
%
The definition by |\providecommand| makes sure
that previous definitions are not overwritten.
Further statements |\providecommand{\version}{...}|
can thus be added before the above code to override it.

For the main file, one might add a line
(between |\childdocmain| and the above block)
%
\begin{center}
|%\ifchilddoc\||else\providecommand{\version}{draft}\||fi|
\end{center}
%
which can be uncommented to produce a draft version.
Likewise one can add a line to the very top of a child file
(above the |\childdocof{|\textit{main}|}| directive)
%
\begin{center}
|%\providecommand{\version}{final}|
\end{center}
%
which can be uncommented to produce the final version of this child document.

%%%%%%%%%%%%%%%%%%%%%%%%%%%%%%%%%%%%%%%%%%%%%%%%%%%%%%%%%%%%%%%%%%%%%%%%%%%%%%%%
\subsection{Forwarding}
\label{sec:forward}

Different versions of the main or child documents
using compilation flags as described in \secref{sec:flags}
can be (permanently) stored in different files
for convenient compilation, viewing and distribution.
To this end, the package defines a command
to pass on compilation to a different file:

%%%%%%%%%%%%%%%%%%%%%%%%%%%%%%%%%%%%%%%%
\DescribeMacro{\childdocforward}
The command |\childdocforward| redirects processing to
another source file:
%
\begin{center}
\begin{tabular}{l}
|\input{childdoc.def}|\\
|\childdocforward[|\textit{main}|]{|\textit{dest}|}|\\
\end{tabular}
\end{center}
%
The argument \textit{dest} is the destination file
(without extension).
It should be the main file or one of the child files.
Note that further \textsf{childdoc} directives
such as |\childdocof| and |\childdocforward|
in the indicated file will be processed in this form.
The optional argument \textit{main}
passes on directly to the main file \textit{main}
while pretending to compile the child \textit{dest}.
This form behaves as if \textit{dest}
issues |\childdocof{|\textit{main}|}| right away,
and no further \textsf{childdoc} directives will be processed.

%%%%%%%%%%%%%%%%%%%%%%%%%%%%%%%%%%%%%%%%
\DescribeMacro{\...prefix}
In the alternative form |\childdocforwardprefix|,
%
\begin{center}
\begin{tabular}{l}
|\input{childdoc.def}|\\
|\childdocforwardprefix[|\textit{main}|]{|\textit{prefix}|}{|\textit{dest}|}|
\end{tabular}
\end{center}
%
the destination file is determined by a pattern
depending on the current file:
To make this work, the current file must be called
`{\textit{prefix}\hspace{0.2em}\textit{suffix}}'
with \textit{prefix} matching precisely the argument.
Processing is then passed on to the file
`{\textit{dest}\hspace{0.2em}\textit{suffix}}'.
Surely, the same effect is achieved by
directly specifying the
argument `{\textit{dest}\hspace{0.2em}\textit{suffix}}'
in the first form.
However, that requires to set up a different file
for each child. With the alternative form of the command
all these files can have exactly the same content
which simplifies setting them up and maintaining them.

For example, the following file |draft.tex|
with a compilation flag |\version| as described in \secref{sec:flags}
compiles the main document as a draft:
%
\begin{center}
\begin{tabular}{l}
|\def\version{draft}|\\
|\input{childdoc.def}|\\
|\childdocforward{|\textit{main}|}|
\end{tabular}
\end{center}
%
Likewise, the following files |final|\textit{nn}|.tex|
compile the final version of the child document
|child|\textit{nn}|.tex|:
%
\begin{center}
\begin{tabular}{l}
|\def\version{final}|\\
|\input{childdoc.def}|\\
|\childdocforwardprefix{final}{child}|
\end{tabular}
\end{center}
%

Note that when several versions of a main file and/or of each child file
are to be generated, it may be convenient to set up a |Makefile| or
shell script to automatise the process.

%%%%%%%%%%%%%%%%%%%%%%%%%%%%%%%%%%%%%%%%%%%%%%%%%%%%%%%%%%%%%%%%%%%%%%%%%%%%%%%%
\subsection{Command Line Processing}
\label{sec:commandline}

The effect of redirection files can also be achieved by invoking
the \LaTeX{} compiler with a more elaborate command line.
Most conveniently this should be done as part
of a shell script or a |Makefile|.

When using \textsf{childdoc} in the main file, the following
command lines effectively perform a redirection
(note that depending on the shell being used,
backslashes may have to be doubled: `|\|' $\to$ `|\\|'):
%
\begin{center}
|... -jobname "|\textit{target}|" |\\|"|[\textit{flags}]%
|\input{childdoc.def}\childdocforward[|\textit{main}|]{|\textit{dest}|}"|
\end{center}
%
Here \textit{target} is the name of the output file,
\textit{main} is the name of the main file
and \textit{dest} is the name of the main or child file to be processed
(all filenames without extensions).
The optional argument \textit{main} can be omitted
if \textit{main} matches \textit{dest}.
Optionally, compilation \textit{flags} can be defined via |\def| commands.
This command line makes the \TeX{} engine believe
it is compiling the file \textit{target}
whose content is specified as the latter parameter.
The provided code then forwards the processing to
\textit{main} or \textit{dest} as described in \secref{sec:forward}.

%%%%%%%%%%%%%%%%%%%%%%%%%%%%%%%%%%%%%%%%%%%%%%%%%%%%%%%%%%%%%%%%%%%%%%%%%%%%%%%%
\subsection{Include by Input}
\label{sec:input}

Including child documents by |\include| has some restrictions by design.
Most notably, the content of a child document always occupies
its own set of pages; pages cannot be shared between child documents.
Usually, this behaviour makes perfect sense
because each child document contain an essential part of the document.
However, in some situations it may be desirable to compose
a document from a collection of parts
without having mandatory page breaks between then.
For this case, the package
provides a mechanism to include parts
by |\input| which can also be processed individually.
However, by construction this mechanism
requires manual handling of the content to be output.

%%%%%%%%%%%%%%%%%%%%%%%%%%%%%%%%%%%%%%%%
\DescribeMacro{\ifchilddocmanual}
The main file should be prepared as usual, see \secref{sec:include}.
However, the document body must make a distinction
between processing of an individual part and of the main document, e.g.:
%
\begin{center}
\begin{tabular}{l}
|\ifchilddocmanual|\\
|\input{\childdocname}|\\
|\||else|\\
\textit{document body with }|\input{|\textit{part}|}|\\
|\||fi|
\end{tabular}
\end{center}
%
The conditional |\ifchilddocmanual| is true whenever
a part to be included by |\input| is being compiled,
and the name of the part is stored in |\childdocname|.

%%%%%%%%%%%%%%%%%%%%%%%%%%%%%%%%%%%%%%%%
\DescribeMacro{\childdocby}
Each part to be included by |\input| should start with:
%
\begin{center}
\begin{tabular}{l}
|\input{childdoc.def}|\\
|\childdocby{|\textit{main}|}|\\
\end{tabular}
\end{center}
%
The directive |\childdocby| is similar to |\childdocof|
described in \secref{sec:include},
but the subsequent selection of content must be done manually.
To that end, both |\ifchilddoc| and |\ifchilddocmanual|
will be true upon processing of a part,
and the name of the part is stored in |\childdocname|.
Note that |\jobname| will be set to the filename of the current part
so that each part receives an individual |.aux| file
that does not interfere with the |.aux| file(s) of the main document.
This behaviour can be altered by the alternative form
|\childdocby[*]{|\textit{main}|}| (with a non-empty optional argument)
which uses the |.aux| file of the main document
by setting |\jobname| to \textit{main}.

%%%%%%%%%%%%%%%%%%%%%%%%%%%%%%%%%%%%%%%%%%%%%%%%%%%%%%%%%%%%%%%%%%%%%%%%%%%%%%%%
\subsection{Driver Development}
\label{sec:driver}

The \textsf{childdoc} mechanism can also be use for the development
of definition files such as \LaTeX{} styles or classes.
This case differs from the above setup with multiple parts
included by |\include| in that no |\includeonly| should be invoked.
This can be achieved by starting the include file
(before |\ProvidesPackage|) with:
%
\begin{center}
\begin{tabular}{l}
|\input{childdoc.def}|\\
|\childdocforward{|\textit{main}|}|\\
\end{tabular}
\end{center}
%
or alternatively with:
%
\begin{center}
\begin{tabular}{l}
|\input{childdoc.def}|\\
|\childdocby{|\textit{main}|}|\\
\end{tabular}
\end{center}
%
Both forms have slightly different effects as described above.
The main file is prepared as usual, see \secref{sec:include}.

%%%%%%%%%%%%%%%%%%%%%%%%%%%%%%%%%%%%%%%%%%%%%%%%%%%%%%%%%%%%%%%%%%%%%%%%%%%%%%%%
\subsection{Legacy Detection}
\label{sec:detection}

The directive |\childdocmain| in the main file can detect
whether the complete document or merely a child is to be compiled
even without using the directive |\childdocof|.
This method is deprecated because it is less robust
and there is no compelling reason to use it;
it is merely provided for backward compatibility
and it may be removed in future versions.

If the detection mechanism is to be used,
it is mandatory to correctly specify
the filename of the main file as the argument of |\childdocmain|:
%
\begin{center}
\begin{tabular}{l}
|\input{childdoc.def}|\\
|\childdocmain{|\textit{main}|}|\\
\end{tabular}
\end{center}
%
If |\jobname| does not match the argument \textit{main} of |\childdocmain|,
it is assumed that |\jobname| points to the child file to be compiled.
When using |\childdocmain| with the main file specified as argument,
it suffices to start a child file
with just |\input{|\textit{main}|}|
without loading of the package and using |\childdocof|.
If instead all processing is done
with the appropriate \textsf{childdoc} directives,
the argument of \textit{main} of |\childdocmain| can be empty.

An alternative version of the command line processing described
in \secref{sec:commandline} using the detection mechanism reads:
%
\begin{center}
|... -jobname "|\textit{target}|" "|[\textit{flags}]%
[|\def\jobname{|\textit{dest}|}|]|\input{|\textit{main}|}"|
\end{center}

%%%%%%%%%%%%%%%%%%%%%%%%%%%%%%%%%%%%%%%%%%%%%%%%%%%%%%%%%%%%%%%%%%%%%%%%%%%%%%%%
\subsection{Manual Code}
\label{sec:manual}

In case one cannot be certain whether the definitions file |childdoc.def|
is installed on the target \TeX{} distribution
and one prefers not to ship it,
it is conceivable to paste a few relevant commands into the sources.

To that end, drop all statements |\input{childdoc.def}|
and perform the replacements as outlined below.
Instead of |\childdocmain{|\textit{main}|}| add the following code
to the top of the main file:
%
\begin{center}
\begin{tabular}{l}
|\||ifdefined\childdocname\endinput\||fi\newif\ifchilddoc|\\
|\edef\childdocname{\scantokens\expandafter{\jobname\noexpand}}|\\
|\def\childdocmain{|\textit{main}|}\||ifx\childdocmain\childdocname\||else|\\
|\childdoctrue\includeonly{\childdocname}\let\jobname\childdocmain\||fi|\\
\end{tabular}
\end{center}
%
Instead of |\childdocof{|\textit{main}|}| just include the main file
at the top of each child file:
%
\begin{center}
|\input{|\textit{main}|}|
\end{center}
%
A simple redirection |\childdocforward{|\textit{dest}|}| is achieved by:
%
\begin{center}
|\def\jobname{|\textit{dest}|}\input{\jobname}|
\end{center}
%
The redirection with prefix
|\childdocforwardprefix[|\textit{prefix}|]{|\textit{dest}|}|
is accomplished by:
%
\begin{center}
\begin{tabular}{l}
|{\edef\jobname{\scantokens\expandafter{\jobname\noexpand}}|\\
|\def\redirectjob |\textit{prefix}|#1~~~{\gdef\jobname{|\textit{dest}|#1}}|\\
|\expandafter\redirectjob\jobname~~~}\input{\jobname}|
\end{tabular}
\end{center}

In an alternative approach,
child documents can be compiled by a specific command line
without additional code or specific definitions:
%
\begin{center}
|... -jobname "|\textit{target}|" "|[\textit{flags}]%
|\includeonly{|\textit{dest}|}\input{|\textit{main}|}"|
\end{center}
%

%%%%%%%%%%%%%%%%%%%%%%%%%%%%%%%%%%%%%%%%%%%%%%%%%%%%%%%%%%%%%%%%%%%%%%%%%%%%%%%%
%%%%%%%%%%%%%%%%%%%%%%%%%%%%%%%%%%%%%%%%%%%%%%%%%%%%%%%%%%%%%%%%%%%%%%%%%%%%%%%%
\section{Information}

%%%%%%%%%%%%%%%%%%%%%%%%%%%%%%%%%%%%%%%%%%%%%%%%%%%%%%%%%%%%%%%%%%%%%%%%%%%%%%%%
\subsection{Copyright}

Copyright \copyright{} 2017--2018 Niklas Beisert

This work may be distributed and/or modified under the
conditions of the \LaTeX{} Project Public License, either version 1.3
of this license or (at your option) any later version.
The latest version of this license is in
  \url{http://www.latex-project.org/lppl.txt}
and version 1.3 or later is part of all distributions of \LaTeX{}
version 2005/12/01 or later.

This work has the LPPL maintenance status `maintained'.

The Current Maintainer of this work is Niklas Beisert.

This work consists of the files |README.txt|, |childdoc.ins| and |childdoc.dtx|
as well as the derived files |childdoc.def|, |cdocsamp.tex|
with |cdocsch1.tex|, |cdocsch2.tex|, |cdocspt3.tex|, |cdocspt4.tex|,
|cdocsdrf.tex|, |cdocsfn1.tex|, |cdocsfn2.tex|
as well as |childdoc.pdf|.

%%%%%%%%%%%%%%%%%%%%%%%%%%%%%%%%%%%%%%%%%%%%%%%%%%%%%%%%%%%%%%%%%%%%%%%%%%%%%%%%
\subsection{Files and Installation}

The package consists of the files:
%
\begin{center}
\begin{tabular}{ll}
    |README.txt|   & readme file \\
    |childdoc.ins| & installation file \\
    |childdoc.dtx| & source file \\
    |childdoc.def| & definition file \\
    |cdocsamp.tex| & sample main file \\
    |cdocsch1.tex| & sample include file \\
    |cdocsch2.tex| & sample include file \\
    |cdocspt3.tex| & sample part file \\
    |cdocspt4.tex| & sample part file \\
    |cdocsdrf.tex| & sample redirection file \\
    |cdocsfn1.tex| & sample redirection file \\
    |cdocsfn2.tex| & sample redirection file \\
    |childdoc.pdf| & manual
\end{tabular}
\end{center}
%
The distribution consists of the files
|README.txt|, |childdoc.ins| and |childdoc.dtx|.
%
\begin{itemize}
\item
Run (pdf)\LaTeX{} on |childdoc.dtx|
to compile the manual |childdoc.pdf| (this file).
\item
Run \LaTeX{} on |childdoc.ins| to create the definitions file |childdoc.def|
and the sample |cdocsamp.tex| with include files
|cdocsch1.tex|, |cdocsch2.tex|, |cdocspt3.tex|, |cdocspt4.tex|,
|cdocsdrf.tex|, |cdocsfn1.tex|, |cdocsfn2.tex|.
Then copy the file |childdoc.def| to an appropriate directory of your \LaTeX{}
distribution, e.g.\ \textit{texmf-root}|/tex/latex/childdoc|.
\end{itemize}

%%%%%%%%%%%%%%%%%%%%%%%%%%%%%%%%%%%%%%%%%%%%%%%%%%%%%%%%%%%%%%%%%%%%%%%%%%%%%%%%
\subsection{Related CTAN Packages}

There are several other packages which offer a similar functionality:
%
\begin{itemize}
\item
The packages
\href{http://ctan.org/pkg/docmute}{\textsf{docmute}},
\href{http://ctan.org/pkg/includex}{\textsf{includex}} and
\href{http://ctan.org/pkg/standalone}{\textsf{standalone}}
provide commands to include only the document body of
a child file thus allowing both files to be compiled individually.
\item
The packages \href{http://ctan.org/pkg/subdocs}{\textsf{subdocs}}
and \href{http://ctan.org/pkg/subfiles}{\textsf{subfiles}}
provide structures in which the main and child documents can be
encapsulated and allowing them to be compiled individually.
The inclusion mechanism is different from the conventional |\include|.
\item
The package \href{http://ctan.org/pkg/combine}{\textsf{combine}}
is an elaborate solution to combine several documents into one.
\end{itemize}
%
See also the CTAN topic \href{http://ctan.org/topic/subdocs}{\textsf{subdocs}}
for further related packages.
The present package differs from the above solutions in that
a document structure constructed with the conventional |\include| mechanism
just needs two extra commands at the top of every file
such that all constituent files can be compiled individually.

%%%%%%%%%%%%%%%%%%%%%%%%%%%%%%%%%%%%%%%%%%%%%%%%%%%%%%%%%%%%%%%%%%%%%%%%%%%%%%%%
%\subsection{Feature Suggestions}
%
%The following is a list of features which may be useful for future
%versions of this package:
%%
%\begin{itemize}
%\item
%\ldots
%\end{itemize}

%%%%%%%%%%%%%%%%%%%%%%%%%%%%%%%%%%%%%%%%%%%%%%%%%%%%%%%%%%%%%%%%%%%%%%%%%%%%%%%%
\subsection{Revision History}

%%%%%%%%%%%%%%%%%%%%%%%%%%%%%%%%%%%%%%%%
\paragraph{v2.0:} 2018/12/30

\begin{itemize}
\item
immediate forward processing
\item
added |\childdocby| mechanism
\item
manual restructured
\end{itemize}

%%%%%%%%%%%%%%%%%%%%%%%%%%%%%%%%%%%%%%%%
\paragraph{v1.6:} 2018/01/17

\begin{itemize}
\item
application for development of include files
\item
corrections to manual
\end{itemize}

%%%%%%%%%%%%%%%%%%%%%%%%%%%%%%%%%%%%%%%%
\paragraph{v1.5:} 2017/05/21

\begin{itemize}
\item
more complete structuring introduced
\item
|\childdocof| introduced
\item
|\childdoc| renamed to |\childdocmain|
\item
|\childredirect| renamed to |\childdocforward| and |\childdocforwardprefix|
and functionality expanded
\end{itemize}

%%%%%%%%%%%%%%%%%%%%%%%%%%%%%%%%%%%%%%%%
\paragraph{v1.0:} 2017/04/27

\begin{itemize}
\item
manual and install package
\item
first version published on CTAN
\end{itemize}

%%%%%%%%%%%%%%%%%%%%%%%%%%%%%%%%%%%%%%%%
\paragraph{v0.6:} 2017/04/26

\begin{itemize}
\item
redirection mechanism added
\end{itemize}

%%%%%%%%%%%%%%%%%%%%%%%%%%%%%%%%%%%%%%%%
\paragraph{v0.5:} 2017/04/26

\begin{itemize}
\item
functionality in definition file
\end{itemize}


%%%%%%%%%%%%%%%%%%%%%%%%%%%%%%%%%%%%%%%%%%%%%%%%%%%%%%%%%%%%%%%%%%%%%%%%%%%%%%%%
%%%%%%%%%%%%%%%%%%%%%%%%%%%%%%%%%%%%%%%%%%%%%%%%%%%%%%%%%%%%%%%%%%%%%%%%%%%%%%%%
%%%%%%%%%%%%%%%%%%%%%%%%%%%%%%%%%%%%%%%%%%%%%%%%%%%%%%%%%%%%%%%%%%%%%%%%%%%%%%%%
\appendix

\settowidth\MacroIndent{\rmfamily\scriptsize 000\ }

 \DocInput{childdoc.dtx}

\end{document}
%</driver>
% \fi
%
% %%%%%%%%%%%%%%%%%%%%%%%%%%%%%%%%%%%%%%%%%%%%%%%%%%%%%%%%%%%%%%%%%%%%%%%%%%%%%%
% %%%%%%%%%%%%%%%%%%%%%%%%%%%%%%%%%%%%%%%%%%%%%%%%%%%%%%%%%%%%%%%%%%%%%%%%%%%%%%
% \section{Sample}
%\iffalse
%<*samplemain>
%\fi
%
% The following presents a sample document
% with two chapters, two parts, a title page,
% a compile flag as well as three forwarding files to set the flag.
% It consists of eight |.tex| files:
% \begin{center}
% \begin{tabular}{ll}
% |cdocsamp.tex|&main file\\
% |cdocsch1.tex|&include file for chapter 1\\
% |cdocsch2.tex|&include file for chapter 2\\
% |cdocspt3.tex|&include file for part 3\\
% |cdocspt4.tex|&include file for part 4\\
% |cdocsdrf.tex|&forwarding file for main file in draft mode\\
% |cdocsfi1.tex|&forwarding file for final version of chapter 1\\
% |cdocsfi2.tex|&forwarding file for final version of chapter 2\\
% \end{tabular}
% \end{center}
% Each of the eight files can be compiled directly by the \LaTeX{} compiler.
%
% %%%%%%%%%%%%%%%%%%%%%%%%%%%%%%%%%%%%%%
% \paragraph{Main File.}
%
% The main file is called |cdocsamp.tex|.
%
% Load the \textsf{childdoc} definitions and
% declare the filename for the main document:
%    \begin{macrocode}
\input{childdoc.def}
\childdocmain{}
%    \end{macrocode}

% Optional override for |\version| flag:
%    \begin{macrocode}
%%\ifchilddoc\else\providecommand{\version}{draft}\fi
%    \end{macrocode}

% Define the default values for the |\version| flag
% (|final| for the main file and |draft| for childs):
%    \begin{macrocode}
\ifchilddoc
\providecommand{\version}{draft}
\else
\providecommand{\version}{final}
\fi
%    \end{macrocode}

% Load the standard document class:
%    \begin{macrocode}
\documentclass[12pt]{article}
%    \end{macrocode}

% Start the document body:
%    \begin{macrocode}
\begin{document}
%    \end{macrocode}

% Declare a title page.
% Print title, part of document being processed and version flag:
%    \begin{macrocode}
\addtocounter{page}{-1}
\begin{center}
{\LARGE\bfseries{}childdoc example\par}
\vspace{1cm}
\ifchilddoc
\ifchilddocmanual part\else chapter\fi:
`\childdocname' of `\childdocjob'\par
\else
main document: `\childdocjob'\par
\fi
version: \version\par
\end{center}
\newpage
%    \end{macrocode}

% Manually include selected file,
% otherwise process as usual:
%    \begin{macrocode}
\ifchilddocmanual
\section*{part `\childdocname'}
\input{\childdocname}
\else
%    \end{macrocode}

% Include the two chapters:
%    \begin{macrocode}
\include{cdocsch1}
\include{cdocsch2}
%    \end{macrocode}

% Include the two parts unless only chapters should be displayed:
%    \begin{macrocode}
\ifchilddoc\else
\section{part three}
\input{cdocspt3}
\section{part four}
\input{cdocspt4}
\fi
%    \end{macrocode}

% Process as usual until here:
%    \begin{macrocode}
\fi
%    \end{macrocode}

% End of document body:
%    \begin{macrocode}
\end{document}
%    \end{macrocode}
%\iffalse
%</samplemain>
%\fi
%
% %%%%%%%%%%%%%%%%%%%%%%%%%%%%%%%%%%%%%%
% \paragraph{Chapter Include Files.}
%
% The include files are called |cdocsch1.tex| and |cdocsch2.tex|.
%
%\iffalse
%<*samplechap1|samplechap2>
%\fi

% Optional override for |\version| flag:
%    \begin{macrocode}
%%\providecommand{\version}{final}
%    \end{macrocode}

% Include the main document:
%    \begin{macrocode}
\input{childdoc.def}
\childdocof{cdocsamp}
%    \end{macrocode}

%\iffalse
%</samplechap1|samplechap2>
%\fi
%
%\iffalse
%<*samplechap1>
%\fi
% Some text for chapter 1:
%    \begin{macrocode}
\section{one}
some text in chapter one
%    \end{macrocode}

%\iffalse
%</samplechap1>
%\fi
% Some text for chapter 2:
%\iffalse
%<*samplechap2>
%\fi
%    \begin{macrocode}
\section{two}
more text in chapter two
%    \end{macrocode}

%\iffalse
%</samplechap2>
%\fi
%
% %%%%%%%%%%%%%%%%%%%%%%%%%%%%%%%%%%%%%%
% \paragraph{Part Include Files.}
%
% The include files are called |cdocspt3.tex| and |cdocspt4.tex|.
%
%\iffalse
%<*samplepart3|samplepart4>
%\fi

% Optional override for |\version| flag:
%    \begin{macrocode}
%%\providecommand{\version}{final}
%    \end{macrocode}

% Include the main document:
%    \begin{macrocode}
\input{childdoc.def}
\childdocby{cdocsamp}
%    \end{macrocode}

%\iffalse
%</samplepart3|samplepart4>
%\fi
%
%\iffalse
%<*samplepart3>
%\fi
% Some text for part 3:
%    \begin{macrocode}
some text in part three
%    \end{macrocode}

%\iffalse
%</samplepart3>
%\fi
% Some text for part 4:
%\iffalse
%<*samplepart4>
%\fi
%    \begin{macrocode}
more text in part four
%    \end{macrocode}

%\iffalse
%</samplepart4>
%\fi
%
% %%%%%%%%%%%%%%%%%%%%%%%%%%%%%%%%%%%%%%
% \paragraph{Forwarding for a Complete Draft.}
%
% The following forwarding file |cdocsdrf.tex|
% compiles the main document in draft mode:
%\iffalse
%<*sampledraft>
%\fi
%    \begin{macrocode}
\def\version{draft}
\input{childdoc.def}
\childdocforward{cdocsamp}
%    \end{macrocode}

%\iffalse
%</sampledraft>
%\fi
%
% %%%%%%%%%%%%%%%%%%%%%%%%%%%%%%%%%%%%%%
% \paragraph{Forwarding for Final Version of the Chapters.}
%
% The following forwarding files |cdocsfn1.tex| and |cdocsfn2.tex|
% (with identical content)
% compile the final versions of the child documents
% |cdocsch1.tex| and |cdocsch2.tex|, respectively:
%\iffalse
%<*samplefinal>
%\fi
%    \begin{macrocode}
\def\version{final}
\input{childdoc.def}
\childdocforwardprefix[cdocsamp]{cdocsfn}{cdocsch}
%    \end{macrocode}

%\iffalse
%</samplefinal>
%\fi
%
% %%%%%%%%%%%%%%%%%%%%%%%%%%%%%%%%%%%%%%
% \paragraph{Command Line Processing.}
%
% The following three command lines generate the output files
% |cdocscld|, |cdocscl1| and |cdocscl2|
% which should be identical to
% |cdocsdrf|, |cdocsch1| and |cdocsfn2|, respectively:
% \begin{center}
% \begin{tabular}{l}
% |latex -jobname cdocscld \|\\
% |  "\def\version{draft}\input{childdoc.def}\childdocforward{cdocsamp}"|\\
% |latex -jobname cdocscl1 \|\\
% |  "\input{childdoc.def}\childdocforward[cdocsamp]{cdocsch1}"|\\
% |latex -jobname cdocscl2 \|\\
% |  "\def\version{final}\input{childdoc.def}\childdocforward{cdocsch2}"|
% \end{tabular}
% \end{center}
% Note that the trailing backslash on each first line
% merely continues the input to the second line
% (for convenient cut ant paste).
% Furthermore, the command |latex| can be replaced by any
% of its alternative versions such as |pdflatex|.
%
% %%%%%%%%%%%%%%%%%%%%%%%%%%%%%%%%%%%%%%%%%%%%%%%%%%%%%%%%%%%%%%%%%%%%%%%%%%%%%%
% %%%%%%%%%%%%%%%%%%%%%%%%%%%%%%%%%%%%%%%%%%%%%%%%%%%%%%%%%%%%%%%%%%%%%%%%%%%%%%
% \section{Implementation}
%\iffalse
%<*package>
%\fi
%
% This section describes the definitions file |childdoc.def|.

% The definitions cannot be loaded using |\usepackage| or |\RequirePackage|
% which has a mechanism to prevent loading a style file more than once.
% When loading the definitions by means of |\input|
% multiple instances have to be prevented manually:
%\iffalse
%This code needs to be before the `\ProvidesFile' directive
%which is defined at the beginning of this file.
%Therefore it is also placed there and commented out here.
%</package>
%<*discard>
%\fi
%    \begin{macrocode}
\ifdefined\childdocmain\endinput\fi
%    \end{macrocode}
%\iffalse
%</discard>
%<*package>
%\fi
%
% \macro{\ifchilddoc}
% \macro{\ifchilddocmanual}
% The conditional |\ifchilddoc| tells whether a
% child (true) or main (false) document is being compiled.
% The conditional |\ifchilddocmanual| tells whether
% the |\includeonly| mechanism is used (false) or
% the selection of child files must be performed manually (true).
% The definitions initialise to false:
%    \begin{macrocode}
\newif\ifchilddoc
\newif\ifchilddocmanual
%    \end{macrocode}

% \macro{\childdocname}
% \macro{\childdocjob}
% The macro |\childdocname| stores the name of the main document
% to be compiled. The macro |\childdocjob| stores the name of
% the document on which the \LaTeX{} compiler was originally invoked.
% The content of |\jobname| cannot be compared
% to filenames specified in the source due to different catcodes.
% The following code rescans |\jobname|, stores the result
% in |\childdocname| and saves a copy in |\childdocjob|:
%    \begin{macrocode}
\edef\childdocname{\scantokens\expandafter{\jobname\noexpand}}
\let\childdocjob\childdocname
%    \end{macrocode}

% \macro{\childdocdisable}
% The macro |\childdocdisable| prevents the main file
% from being processed more than once.
% At this stage, the main document command |\childdocmain|
% is assumed to be called once again where it should do nothing.
% Any subsequent call to it should prevent
% a secondary processing of the main document
% It overwrites the forwarding commands
% |\childdocof| and |\childdocforward|
% with empty macros to prevent further inclusions of the main document:
%    \begin{macrocode}
\newcommand{\childdocdisable}
{
  \renewcommand{\childdocmain}[1]{\renewcommand{\childdocmain}[1]{\endinput}}
  \renewcommand{\childdocof}[1]{}
  \renewcommand{\childdocby}[2][]{}
  \renewcommand{\childdocforward}[2][]{}
  \renewcommand{\childdocdisable}{}
}
%    \end{macrocode}

% \macro{\childdocmain}
% The macro |\childdocmain| is to be called at the top of the main file
% with nothing or the main filename (without extension) as argument.
% First, it breaks loops.
% If the argument is not empty and does not match |\childdocname|
% (which is set by the first inclusion of |childdoc.def|),
% |\ifchilddoc| is set to true, |\includeonly| is applied to the child file
% and |\jobname| is set to the main file
% (for proper handling of |.aux| files):
%    \begin{macrocode}
\newcommand{\childdocmain}[1]
{
  \childdocdisable\childdocmain{}
  \if?#1?\else
    \begingroup
      \def\childdoctmp{#1}
      \ifx\childdoctmp\childdocname
        \def\childdoctmp{}
      \else
        \def\childdoctmp
        {
          \childdoctrue
          \includeonly{\childdocname}
          \def\childdocjob{#1}
          \def\jobname{#1}
        }
      \fi
      \expandafter
    \endgroup
    \childdoctmp
  \fi
}
%    \end{macrocode}

% \macro{\childdocof}
% The command |\childdocof| redirects
% compilation to the main file |#1|.
%    \begin{macrocode}
\newcommand{\childdocof}[1]
{
  \childdocdisable
  \childdoctrue
  \includeonly{\childdocname}
  \def\jobname{#1}
  \def\childdocjob{#1}
  \input{#1}
}
%    \end{macrocode}

% \macro{\childdocby}
% The command |\childdocby| ....
%    \begin{macrocode}
\newcommand{\childdocby}[2][]
{
  \childdocdisable
  \childdoctrue
  \childdocmanualtrue
  \if?#1?\else
    \def\jobname{#2}
  \fi
  \def\childdocjob{#2}
  \input{#2}
  \endinput
}
%    \end{macrocode}

% \macro{\childdocforward}
% The command |\childdocforward| redirects
% compilation to the main file or
% (if the optional argument is given) a child file.
% Parameters are set as if the main file
% or a child file starting with |\childdocof| was compiled.
% Then compilation is handed over to the main file:
%    \begin{macrocode}
\newcommand{\childdocforward}[2][]
{
  \begingroup
    \if?#1?
      \def\childdoctmp
      {
        \def\childdocname{#2}
        \def\childdocjob{#2}
        \def\jobname{#2}
        \input{#2}
        \endinput
      }
    \else
      \def\childdoctmp
      {
        \childdocdisable
        \def\childdocname{#2}
        \childdoctrue
        \includeonly{#2}
        \def\childdocjob{#1}
        \def\jobname{#1}
        \input{#1}
        \endinput
      }
    \fi
    \expandafter
  \endgroup
  \childdoctmp
}
%    \end{macrocode}

% \macro{\childdocforwardprefix}
% The command |\childdocforwardprefix| redirects
% compilation to the main or a child file by means of a pattern.
% The prefix |#1| in the current filename is replaced by |#2|
% and the suffix of the current filename is kept
% (it is assumed that the filename does not contain the substring `|~~~|'
% which is used as a delimiter).
% Compilation is handed over to the new file by |\childdocforward|:
%    \begin{macrocode}
\newcommand{\childdocforwardprefix}[3][]
{
  \begingroup
    \def\childdocextract #2##1~~~{\def\childdoctmp{\childdocforward[#1]{#3##1}}}
    \expandafter\childdocextract\childdocname~~~
    \expandafter
  \endgroup
  \childdoctmp
}
%    \end{macrocode}

% \macro{\childdoc}
% The deprecated macro |\childdoc| is a legacy version of |\childdocmain|:
%    \begin{macrocode}
\newcommand{\childdoc}{\childdocmain}
%    \end{macrocode}

% \macro{\childdocredirect}
% The deprecated macro |\childdocredirect| is a legacy version
% of |\childdocforward| and |\childdocforwardprefix|:
%    \begin{macrocode}
\newcommand{\childdocredirect}[2][]
{
  \begingroup
    \if?#1?
      \def\childdoctmp{\childdocforward{#2}}
    \else
      \def\childdoctmp{\childdocforwardprefix{#1}{#2}}
    \fi
    \expandafter
  \endgroup
  \childdoctmp
}
%    \end{macrocode}

%\iffalse
%</package>
%\fi
%
\endinput

\childdocmain{}
%    \end{macrocode}

% Optional override for |\version| flag:
%    \begin{macrocode}
%%\ifchilddoc\else\providecommand{\version}{draft}\fi
%    \end{macrocode}

% Define the default values for the |\version| flag
% (|final| for the main file and |draft| for childs):
%    \begin{macrocode}
\ifchilddoc
\providecommand{\version}{draft}
\else
\providecommand{\version}{final}
\fi
%    \end{macrocode}

% Load the standard document class:
%    \begin{macrocode}
\documentclass[12pt]{article}
%    \end{macrocode}

% Start the document body:
%    \begin{macrocode}
\begin{document}
%    \end{macrocode}

% Declare a title page.
% Print title, part of document being processed and version flag:
%    \begin{macrocode}
\addtocounter{page}{-1}
\begin{center}
{\LARGE\bfseries{}childdoc example\par}
\vspace{1cm}
\ifchilddoc
\ifchilddocmanual part\else chapter\fi:
`\childdocname' of `\childdocjob'\par
\else
main document: `\childdocjob'\par
\fi
version: \version\par
\end{center}
\newpage
%    \end{macrocode}

% Manually include selected file,
% otherwise process as usual:
%    \begin{macrocode}
\ifchilddocmanual
\section*{part `\childdocname'}
\input{\childdocname}
\else
%    \end{macrocode}

% Include the two chapters:
%    \begin{macrocode}
\include{cdocsch1}
\include{cdocsch2}
%    \end{macrocode}

% Include the two parts unless only chapters should be displayed:
%    \begin{macrocode}
\ifchilddoc\else
\section{part three}
\input{cdocspt3}
\section{part four}
\input{cdocspt4}
\fi
%    \end{macrocode}

% Process as usual until here:
%    \begin{macrocode}
\fi
%    \end{macrocode}

% End of document body:
%    \begin{macrocode}
\end{document}
%    \end{macrocode}
%\iffalse
%</samplemain>
%\fi
%
% %%%%%%%%%%%%%%%%%%%%%%%%%%%%%%%%%%%%%%
% \paragraph{Chapter Include Files.}
%
% The include files are called |cdocsch1.tex| and |cdocsch2.tex|.
%
%\iffalse
%<*samplechap1|samplechap2>
%\fi

% Optional override for |\version| flag:
%    \begin{macrocode}
%%\providecommand{\version}{final}
%    \end{macrocode}

% Include the main document:
%    \begin{macrocode}
% \iffalse
%
% childdoc.dtx Copyright (C) 2017-2018 Niklas Beisert
%
% This work may be distributed and/or modified under the
% conditions of the LaTeX Project Public License, either version 1.3
% of this license or (at your option) any later version.
% The latest version of this license is in
%   http://www.latex-project.org/lppl.txt
% and version 1.3 or later is part of all distributions of LaTeX
% version 2005/12/01 or later.
%
% This work has the LPPL maintenance status `maintained'.
%
% The Current Maintainer of this work is Niklas Beisert.
%
% This work consists of the files childdoc.dtx and childdoc.ins
% and the derived files childdoc.def and cdocsamp.tex with
% cdocsch1.tex, cdocsch2.tex, cdocsdrf.tex, cdocsfn1.tex, cdocsfn2.tex.
%
%<package>\ifdefined\childdocmain\endinput\fi
%<package>\ProvidesFile{childdoc.def}[2018/12/30 v2.0 child document driver]
%<samplemain>\ProvidesFile{cdocsamp.tex}[2018/12/30 v2.0 sample for childdoc]
%<*driver>
%\ProvidesFile{childdoc.drv}[2018/12/30 v2.0 childdoc reference manual file]
\PassOptionsToClass{10pt,a4paper}{article}
\documentclass{ltxdoc}

\usepackage[margin=35mm]{geometry}
\usepackage{hyperref}
\usepackage{hyperxmp}
\usepackage[usenames]{color}

\hypersetup{colorlinks=true}
\hypersetup{pdfstartview=FitH}
\hypersetup{pdfpagemode=UseNone}
\hypersetup{pdfsource={}}
\hypersetup{pdflang={en-UK}}
\hypersetup{pdfcopyright={Copyright 2017-2018 Niklas Beisert.
  This work may be distributed and/or modified under the
  conditions of the LaTeX Project Public License, either version 1.3
  of this license or (at your option) any later version.}}
\hypersetup{pdflicenseurl={http://www.latex-project.org/lppl.txt}}
\hypersetup{pdfcontactaddress={ETH Zurich, ITP, HIT K,
  Wolfgang-Pauli-Strasse 27}}
\hypersetup{pdfcontactpostcode={8093}}
\hypersetup{pdfcontactcity={Zurich}}
\hypersetup{pdfcontactcountry={Switzerland}}
\hypersetup{pdfcontactemail={nbeisert@itp.phys.ethz.ch}}
\hypersetup{pdfcontacturl={http://people.phys.ethz.ch/\xmptilde nbeisert/}}

\newcommand{\secref}[1]{\hyperref[#1]{section \ref*{#1}}}

\parskip1ex
\parindent0pt
\let\olditemize\itemize
\def\itemize{\olditemize\parskip0pt}

\begin{document}

\title{The \textsf{childdoc} Package}
\hypersetup{pdftitle={The childdoc Package}}
\author{Niklas Beisert\\[2ex]
  Institut f\"ur Theoretische Physik\\
  Eidgen\"ossische Technische Hochschule Z\"urich\\
  Wolfgang-Pauli-Strasse 27, 8093 Z\"urich, Switzerland\\[1ex]
  \href{mailto:nbeisert@itp.phys.ethz.ch}
  {\texttt{nbeisert@itp.phys.ethz.ch}}}
\hypersetup{pdfauthor={Niklas Beisert}}
\hypersetup{pdfsubject={Manual for the LaTeX2e Package childdoc}}
\date{30 December 2018, \textsf{v2.0}}
\maketitle

\begin{abstract}\noindent
\textsf{childdoc} is a \LaTeXe{} package
that enables the direct compilation
of document sections included by |\include|
to individual files.
\end{abstract}

\begingroup
\parskip0ex
\tableofcontents
\endgroup

%%%%%%%%%%%%%%%%%%%%%%%%%%%%%%%%%%%%%%%%%%%%%%%%%%%%%%%%%%%%%%%%%%%%%%%%%%%%%%%%
%%%%%%%%%%%%%%%%%%%%%%%%%%%%%%%%%%%%%%%%%%%%%%%%%%%%%%%%%%%%%%%%%%%%%%%%%%%%%%%%
\section{Introduction}

\LaTeX{} provides a mechanism to structure a large document (such as a book)
into a main file and several child files (containing the chapters)
using the |\include| command.
This mechanism is beneficial for documents
which span hundreds of pages in order to
make the source file(s) more manageable.
Moreover, compilation can be restricted to
selected child files by means of the |\includeonly| command.
The latter feature can be used to reduce the compilation time while editing
(this was significantly more useful in the earlier days of \LaTeX{})
or to generate a smaller document which is easier to navigate.
Another application of |\includeonly| is to generate
documents consisting of selected parts of the complete document.

However, there are a few drawbacks of the plain |\include| mechanism:
\begin{itemize}
\item
The child files cannot be compiled on their own,
they can only be compiled via the main file.
A naive editing environment
(such as a text editor with an option
to have the current file processed by \LaTeX)
may require one to switch to the main file before compiling;
attempting to compile the child file produces errors.
\item
The main file must be modified (each time)
to adjust the |\includeonly| command
to the present needs. This easily leaves the main file in a messy state.
\item
The generated document will always carry the filename
of the main document. This is inconvenient if
several child files are to be compiled and
to be kept for distribution.
\end{itemize}

The present package provides a simple interface
to make child files individually compilable by \LaTeX{}.
Compiling a child file then has the same effect as compiling
the main file with an |\includeonly| command
to select the appropriate child.
Moreover the generated document will carry the name of the child
rather than the main file.
This resolves all three above issues.

This feature is meant to make the editing of books,
thesis documents and lecture notes somewhat more convenient.
However, the package can also be used efficiently for
composing a series of documents (such as exercise sheets)
which are typically distributed individually.
It then assists the author in generating the individual documents
(potentially in different versions)
as well as a document containing the collected series.
Another application is in developing style files
or other kinds of included material
where compilation of the style file could redirect
to a sample or test file.

%%%%%%%%%%%%%%%%%%%%%%%%%%%%%%%%%%%%%%%%%%%%%%%%%%%%%%%%%%%%%%%%%%%%%%%%%%%%%%%%
%%%%%%%%%%%%%%%%%%%%%%%%%%%%%%%%%%%%%%%%%%%%%%%%%%%%%%%%%%%%%%%%%%%%%%%%%%%%%%%%
\section{Usage}

First of all, the package \textsf{childdoc} is \emph{not} a standard
\LaTeXe{} |.sty| style file! Therefore it needs to be invoked in
a non-standard way.

%%%%%%%%%%%%%%%%%%%%%%%%%%%%%%%%%%%%%%%%%%%%%%%%%%%%%%%%%%%%%%%%%%%%%%%%%%%%%%%%
\subsection{Included Files}
\label{sec:include}

%%%%%%%%%%%%%%%%%%%%%%%%%%%%%%%%%%%%%%%%
\DescribeMacro{\childdocmain}
To use the package, add the commands
\begin{center}
\begin{tabular}{l}
|\input{childdoc.def}|\\
|\childdocmain{}|\\
\end{tabular}
\end{center}
at the very top of the main \LaTeX{} file,
in particular \emph{before} the |\documentclass| statement!
The argument of |\childdocmain| should be left empty
(but it must be present).

%%%%%%%%%%%%%%%%%%%%%%%%%%%%%%%%%%%%%%%%
\DescribeMacro{\childdocof}
Furthermore, add the commands
\begin{center}
\begin{tabular}{l}
|\input{childdoc.def}|\\
|\childdocof{|\textit{main}|}|\\
\end{tabular}
\end{center}
at the top of every child file \textit{child}
which is included by |\include{|\textit{child}|}|
from within the main file
(or at least for those files to be compiled individually).
The argument \textit{main} must be the filename of the main file.

There are a couple of
considerations in setting up the main and child documents:

%%%%%%%%%%%%%%%%%%%%%%%%%%%%%%%%%%%%%%%%
\paragraph{Restrictions.}

Please note the following restrictions:
\begin{itemize}
\item
|\childdocmain| must be called with one argument \textit{main}
to ensure compatibility with earlier version of the package.
It must either be empty (|\childdocmain{}|)
or precisely match the filename of the main file in which it is specified.
See \secref{sec:detection} for further information.
\item
The filename \textit{main} must be specified without the |.tex| extension.
\item
The filename \textit{main} is case sensitive
(even in case-insensitive file systems)
due to internal string comparison.
\item
The argument \textit{main} should be fully expanded, it cannot be a macro.
\item
Subdirectories and special characters should be avoided in filenames.
\item
The command |\childdocmain{|\textit{main}|}| must be followed by a whitespace.
It should not be followed immediately by another command
or by a comment mark `|%|'.
This is because the \TeX{} parser reads the token immediately following
the argument of |\childdocmain| and puts it
at the beginning of every child section;
however, a white\-space is ignored.
\end{itemize}

%%%%%%%%%%%%%%%%%%%%%%%%%%%%%%%%%%%%%%%%
\paragraph{Content of Main File.}

It is advisable to place all content in the child files included by |\include|.
Any output contained in the main file will appear in all child documents
unless suppressed manually;
it cannot be suppressed automatically by the |\includeonly| directive
and thus should normally be avoided.
A method to include some content in the main file
by means of conditional processing is described in \secref{sec:conditional}.

%%%%%%%%%%%%%%%%%%%%%%%%%%%%%%%%%%%%%%%%
\paragraph{Page Numbering.}

When only a part of the document is compiled,
the appropriate numbering of pages
(as well as other status parameters)
is determined from the |.aux| files.
The latter contain information from previous passes.
However this information needs to propagate through
all intermediate child documents.
Therefore the page numbering in child documents may well
be inconsistent until the complete document is compiled at least once.

A useful (if unconventional) way to always ensure a consistent
page numbering is to restart the numbering in each child document
and denote the pages by `\textit{child}|.|\textit{page}'
where \textit{child} represents the chapter/section number of the child file.
This can be achieved by the command
|\numberwithin{page}{|\textit{child}|}|
of the \textsf{amsmath} package
where \textit{child} can be |chapter| or |section|
depending on the chosen structuring.
Alternatively, one can modify the macro |\thepage| appropriately
and reset the counter |page| at the start of each child file.

%%%%%%%%%%%%%%%%%%%%%%%%%%%%%%%%%%%%%%%%%%%%%%%%%%%%%%%%%%%%%%%%%%%%%%%%%%%%%%%%
\subsection{Conditional Processing}
\label{sec:conditional}

The package provides a mechanism to compile different versions
of a document. To customise the versions further some conditional processing
can come in handy to distinguish which version is being compiled.
The package provides two macros to describe the compilation context:

%%%%%%%%%%%%%%%%%%%%%%%%%%%%%%%%%%%%%%%%
\DescribeMacro{\ifchilddoc}
The conditional |\ifchilddoc| distinguishes between the compilation of
child documents and the main document:
%
\begin{center}
|\ifchilddoc |\textit{child-code}| |[|\||else |\textit{main-code}]| \||fi|
\end{center}

%%%%%%%%%%%%%%%%%%%%%%%%%%%%%%%%%%%%%%%%
\DescribeMacro{\childdocname}
\DescribeMacro{\childdocjob}
The macro |\childdocname| contains the filename (without extension)
of the main or child file being processed.
Note that |\childdocjob| will always contain the name of the main file.

%%%%%%%%%%%%%%%%%%%%%%%%%%%%%%%%%%%%%%%%
\paragraph{Title Page.}

Conditional processing can be used to include a title or banner page
in the main document when proper precautions are taken.
Importantly, the code in the main file should ensure that the page counter
(as well as other status parameters which are stored in the |.aux| files)
takes the same value after the conditional processing.
Otherwise the page numbers may take divergent values
depending on which part is compiled.

For example, a title page could be declared by:
%
\begin{center}
\begin{tabular}{l}
|\ifchilddoc\||else|\\
|\addtocounter{page}{-1}|\\
\textit{code for title page}\\
|\newpage|\\
|\||fi|
\end{tabular}
\end{center}
%
A banner page for the child documents can be generated by:
%
\begin{center}
\begin{tabular}{l}
|\ifchilddoc|\\
|\addtocounter{page}{-1}|\\
\textit{code for banner page}\\
|\newpage|\\
|\||fi|
\end{tabular}
\end{center}
%
Here one could write a message such as:
\begin{center}
|This is the part \childdocname{} of \childdocjob{}.|
\end{center}

%%%%%%%%%%%%%%%%%%%%%%%%%%%%%%%%%%%%%%%%%%%%%%%%%%%%%%%%%%%%%%%%%%%%%%%%%%%%%%%%
\subsection{Flags}
\label{sec:flags}

The package makes it easy to generate different versions
of the main or child documents.
To this end compilation flags can be defined
and assigned different default values.
They will be particularly useful in conjunction
with the forwarding mechanism described in \secref{sec:forward}.

For example, it may be useful to have a flag |\version|
which can be set to |draft| or |final|.
The document source will contain some conditional code
depending on the value of |\version|.
Suppose further, the flag should default to |final| for the main file
and to |draft| for child files
which is a natural assignment for editing the document.
This is achieved by placing the following code
in the preamble of the main document
(below the |\childdocmain| directive):
%
\begin{center}
\begin{tabular}{l}
|\ifchilddoc|\\
|\providecommand{\version}{draft}|\\
|\||else|\\
|\providecommand{\version}{final}|\\
|\||fi|
\end{tabular}
\end{center}
%
The definition by |\providecommand| makes sure
that previous definitions are not overwritten.
Further statements |\providecommand{\version}{...}|
can thus be added before the above code to override it.

For the main file, one might add a line
(between |\childdocmain| and the above block)
%
\begin{center}
|%\ifchilddoc\||else\providecommand{\version}{draft}\||fi|
\end{center}
%
which can be uncommented to produce a draft version.
Likewise one can add a line to the very top of a child file
(above the |\childdocof{|\textit{main}|}| directive)
%
\begin{center}
|%\providecommand{\version}{final}|
\end{center}
%
which can be uncommented to produce the final version of this child document.

%%%%%%%%%%%%%%%%%%%%%%%%%%%%%%%%%%%%%%%%%%%%%%%%%%%%%%%%%%%%%%%%%%%%%%%%%%%%%%%%
\subsection{Forwarding}
\label{sec:forward}

Different versions of the main or child documents
using compilation flags as described in \secref{sec:flags}
can be (permanently) stored in different files
for convenient compilation, viewing and distribution.
To this end, the package defines a command
to pass on compilation to a different file:

%%%%%%%%%%%%%%%%%%%%%%%%%%%%%%%%%%%%%%%%
\DescribeMacro{\childdocforward}
The command |\childdocforward| redirects processing to
another source file:
%
\begin{center}
\begin{tabular}{l}
|\input{childdoc.def}|\\
|\childdocforward[|\textit{main}|]{|\textit{dest}|}|\\
\end{tabular}
\end{center}
%
The argument \textit{dest} is the destination file
(without extension).
It should be the main file or one of the child files.
Note that further \textsf{childdoc} directives
such as |\childdocof| and |\childdocforward|
in the indicated file will be processed in this form.
The optional argument \textit{main}
passes on directly to the main file \textit{main}
while pretending to compile the child \textit{dest}.
This form behaves as if \textit{dest}
issues |\childdocof{|\textit{main}|}| right away,
and no further \textsf{childdoc} directives will be processed.

%%%%%%%%%%%%%%%%%%%%%%%%%%%%%%%%%%%%%%%%
\DescribeMacro{\...prefix}
In the alternative form |\childdocforwardprefix|,
%
\begin{center}
\begin{tabular}{l}
|\input{childdoc.def}|\\
|\childdocforwardprefix[|\textit{main}|]{|\textit{prefix}|}{|\textit{dest}|}|
\end{tabular}
\end{center}
%
the destination file is determined by a pattern
depending on the current file:
To make this work, the current file must be called
`{\textit{prefix}\hspace{0.2em}\textit{suffix}}'
with \textit{prefix} matching precisely the argument.
Processing is then passed on to the file
`{\textit{dest}\hspace{0.2em}\textit{suffix}}'.
Surely, the same effect is achieved by
directly specifying the
argument `{\textit{dest}\hspace{0.2em}\textit{suffix}}'
in the first form.
However, that requires to set up a different file
for each child. With the alternative form of the command
all these files can have exactly the same content
which simplifies setting them up and maintaining them.

For example, the following file |draft.tex|
with a compilation flag |\version| as described in \secref{sec:flags}
compiles the main document as a draft:
%
\begin{center}
\begin{tabular}{l}
|\def\version{draft}|\\
|\input{childdoc.def}|\\
|\childdocforward{|\textit{main}|}|
\end{tabular}
\end{center}
%
Likewise, the following files |final|\textit{nn}|.tex|
compile the final version of the child document
|child|\textit{nn}|.tex|:
%
\begin{center}
\begin{tabular}{l}
|\def\version{final}|\\
|\input{childdoc.def}|\\
|\childdocforwardprefix{final}{child}|
\end{tabular}
\end{center}
%

Note that when several versions of a main file and/or of each child file
are to be generated, it may be convenient to set up a |Makefile| or
shell script to automatise the process.

%%%%%%%%%%%%%%%%%%%%%%%%%%%%%%%%%%%%%%%%%%%%%%%%%%%%%%%%%%%%%%%%%%%%%%%%%%%%%%%%
\subsection{Command Line Processing}
\label{sec:commandline}

The effect of redirection files can also be achieved by invoking
the \LaTeX{} compiler with a more elaborate command line.
Most conveniently this should be done as part
of a shell script or a |Makefile|.

When using \textsf{childdoc} in the main file, the following
command lines effectively perform a redirection
(note that depending on the shell being used,
backslashes may have to be doubled: `|\|' $\to$ `|\\|'):
%
\begin{center}
|... -jobname "|\textit{target}|" |\\|"|[\textit{flags}]%
|\input{childdoc.def}\childdocforward[|\textit{main}|]{|\textit{dest}|}"|
\end{center}
%
Here \textit{target} is the name of the output file,
\textit{main} is the name of the main file
and \textit{dest} is the name of the main or child file to be processed
(all filenames without extensions).
The optional argument \textit{main} can be omitted
if \textit{main} matches \textit{dest}.
Optionally, compilation \textit{flags} can be defined via |\def| commands.
This command line makes the \TeX{} engine believe
it is compiling the file \textit{target}
whose content is specified as the latter parameter.
The provided code then forwards the processing to
\textit{main} or \textit{dest} as described in \secref{sec:forward}.

%%%%%%%%%%%%%%%%%%%%%%%%%%%%%%%%%%%%%%%%%%%%%%%%%%%%%%%%%%%%%%%%%%%%%%%%%%%%%%%%
\subsection{Include by Input}
\label{sec:input}

Including child documents by |\include| has some restrictions by design.
Most notably, the content of a child document always occupies
its own set of pages; pages cannot be shared between child documents.
Usually, this behaviour makes perfect sense
because each child document contain an essential part of the document.
However, in some situations it may be desirable to compose
a document from a collection of parts
without having mandatory page breaks between then.
For this case, the package
provides a mechanism to include parts
by |\input| which can also be processed individually.
However, by construction this mechanism
requires manual handling of the content to be output.

%%%%%%%%%%%%%%%%%%%%%%%%%%%%%%%%%%%%%%%%
\DescribeMacro{\ifchilddocmanual}
The main file should be prepared as usual, see \secref{sec:include}.
However, the document body must make a distinction
between processing of an individual part and of the main document, e.g.:
%
\begin{center}
\begin{tabular}{l}
|\ifchilddocmanual|\\
|\input{\childdocname}|\\
|\||else|\\
\textit{document body with }|\input{|\textit{part}|}|\\
|\||fi|
\end{tabular}
\end{center}
%
The conditional |\ifchilddocmanual| is true whenever
a part to be included by |\input| is being compiled,
and the name of the part is stored in |\childdocname|.

%%%%%%%%%%%%%%%%%%%%%%%%%%%%%%%%%%%%%%%%
\DescribeMacro{\childdocby}
Each part to be included by |\input| should start with:
%
\begin{center}
\begin{tabular}{l}
|\input{childdoc.def}|\\
|\childdocby{|\textit{main}|}|\\
\end{tabular}
\end{center}
%
The directive |\childdocby| is similar to |\childdocof|
described in \secref{sec:include},
but the subsequent selection of content must be done manually.
To that end, both |\ifchilddoc| and |\ifchilddocmanual|
will be true upon processing of a part,
and the name of the part is stored in |\childdocname|.
Note that |\jobname| will be set to the filename of the current part
so that each part receives an individual |.aux| file
that does not interfere with the |.aux| file(s) of the main document.
This behaviour can be altered by the alternative form
|\childdocby[*]{|\textit{main}|}| (with a non-empty optional argument)
which uses the |.aux| file of the main document
by setting |\jobname| to \textit{main}.

%%%%%%%%%%%%%%%%%%%%%%%%%%%%%%%%%%%%%%%%%%%%%%%%%%%%%%%%%%%%%%%%%%%%%%%%%%%%%%%%
\subsection{Driver Development}
\label{sec:driver}

The \textsf{childdoc} mechanism can also be use for the development
of definition files such as \LaTeX{} styles or classes.
This case differs from the above setup with multiple parts
included by |\include| in that no |\includeonly| should be invoked.
This can be achieved by starting the include file
(before |\ProvidesPackage|) with:
%
\begin{center}
\begin{tabular}{l}
|\input{childdoc.def}|\\
|\childdocforward{|\textit{main}|}|\\
\end{tabular}
\end{center}
%
or alternatively with:
%
\begin{center}
\begin{tabular}{l}
|\input{childdoc.def}|\\
|\childdocby{|\textit{main}|}|\\
\end{tabular}
\end{center}
%
Both forms have slightly different effects as described above.
The main file is prepared as usual, see \secref{sec:include}.

%%%%%%%%%%%%%%%%%%%%%%%%%%%%%%%%%%%%%%%%%%%%%%%%%%%%%%%%%%%%%%%%%%%%%%%%%%%%%%%%
\subsection{Legacy Detection}
\label{sec:detection}

The directive |\childdocmain| in the main file can detect
whether the complete document or merely a child is to be compiled
even without using the directive |\childdocof|.
This method is deprecated because it is less robust
and there is no compelling reason to use it;
it is merely provided for backward compatibility
and it may be removed in future versions.

If the detection mechanism is to be used,
it is mandatory to correctly specify
the filename of the main file as the argument of |\childdocmain|:
%
\begin{center}
\begin{tabular}{l}
|\input{childdoc.def}|\\
|\childdocmain{|\textit{main}|}|\\
\end{tabular}
\end{center}
%
If |\jobname| does not match the argument \textit{main} of |\childdocmain|,
it is assumed that |\jobname| points to the child file to be compiled.
When using |\childdocmain| with the main file specified as argument,
it suffices to start a child file
with just |\input{|\textit{main}|}|
without loading of the package and using |\childdocof|.
If instead all processing is done
with the appropriate \textsf{childdoc} directives,
the argument of \textit{main} of |\childdocmain| can be empty.

An alternative version of the command line processing described
in \secref{sec:commandline} using the detection mechanism reads:
%
\begin{center}
|... -jobname "|\textit{target}|" "|[\textit{flags}]%
[|\def\jobname{|\textit{dest}|}|]|\input{|\textit{main}|}"|
\end{center}

%%%%%%%%%%%%%%%%%%%%%%%%%%%%%%%%%%%%%%%%%%%%%%%%%%%%%%%%%%%%%%%%%%%%%%%%%%%%%%%%
\subsection{Manual Code}
\label{sec:manual}

In case one cannot be certain whether the definitions file |childdoc.def|
is installed on the target \TeX{} distribution
and one prefers not to ship it,
it is conceivable to paste a few relevant commands into the sources.

To that end, drop all statements |\input{childdoc.def}|
and perform the replacements as outlined below.
Instead of |\childdocmain{|\textit{main}|}| add the following code
to the top of the main file:
%
\begin{center}
\begin{tabular}{l}
|\||ifdefined\childdocname\endinput\||fi\newif\ifchilddoc|\\
|\edef\childdocname{\scantokens\expandafter{\jobname\noexpand}}|\\
|\def\childdocmain{|\textit{main}|}\||ifx\childdocmain\childdocname\||else|\\
|\childdoctrue\includeonly{\childdocname}\let\jobname\childdocmain\||fi|\\
\end{tabular}
\end{center}
%
Instead of |\childdocof{|\textit{main}|}| just include the main file
at the top of each child file:
%
\begin{center}
|\input{|\textit{main}|}|
\end{center}
%
A simple redirection |\childdocforward{|\textit{dest}|}| is achieved by:
%
\begin{center}
|\def\jobname{|\textit{dest}|}\input{\jobname}|
\end{center}
%
The redirection with prefix
|\childdocforwardprefix[|\textit{prefix}|]{|\textit{dest}|}|
is accomplished by:
%
\begin{center}
\begin{tabular}{l}
|{\edef\jobname{\scantokens\expandafter{\jobname\noexpand}}|\\
|\def\redirectjob |\textit{prefix}|#1~~~{\gdef\jobname{|\textit{dest}|#1}}|\\
|\expandafter\redirectjob\jobname~~~}\input{\jobname}|
\end{tabular}
\end{center}

In an alternative approach,
child documents can be compiled by a specific command line
without additional code or specific definitions:
%
\begin{center}
|... -jobname "|\textit{target}|" "|[\textit{flags}]%
|\includeonly{|\textit{dest}|}\input{|\textit{main}|}"|
\end{center}
%

%%%%%%%%%%%%%%%%%%%%%%%%%%%%%%%%%%%%%%%%%%%%%%%%%%%%%%%%%%%%%%%%%%%%%%%%%%%%%%%%
%%%%%%%%%%%%%%%%%%%%%%%%%%%%%%%%%%%%%%%%%%%%%%%%%%%%%%%%%%%%%%%%%%%%%%%%%%%%%%%%
\section{Information}

%%%%%%%%%%%%%%%%%%%%%%%%%%%%%%%%%%%%%%%%%%%%%%%%%%%%%%%%%%%%%%%%%%%%%%%%%%%%%%%%
\subsection{Copyright}

Copyright \copyright{} 2017--2018 Niklas Beisert

This work may be distributed and/or modified under the
conditions of the \LaTeX{} Project Public License, either version 1.3
of this license or (at your option) any later version.
The latest version of this license is in
  \url{http://www.latex-project.org/lppl.txt}
and version 1.3 or later is part of all distributions of \LaTeX{}
version 2005/12/01 or later.

This work has the LPPL maintenance status `maintained'.

The Current Maintainer of this work is Niklas Beisert.

This work consists of the files |README.txt|, |childdoc.ins| and |childdoc.dtx|
as well as the derived files |childdoc.def|, |cdocsamp.tex|
with |cdocsch1.tex|, |cdocsch2.tex|, |cdocspt3.tex|, |cdocspt4.tex|,
|cdocsdrf.tex|, |cdocsfn1.tex|, |cdocsfn2.tex|
as well as |childdoc.pdf|.

%%%%%%%%%%%%%%%%%%%%%%%%%%%%%%%%%%%%%%%%%%%%%%%%%%%%%%%%%%%%%%%%%%%%%%%%%%%%%%%%
\subsection{Files and Installation}

The package consists of the files:
%
\begin{center}
\begin{tabular}{ll}
    |README.txt|   & readme file \\
    |childdoc.ins| & installation file \\
    |childdoc.dtx| & source file \\
    |childdoc.def| & definition file \\
    |cdocsamp.tex| & sample main file \\
    |cdocsch1.tex| & sample include file \\
    |cdocsch2.tex| & sample include file \\
    |cdocspt3.tex| & sample part file \\
    |cdocspt4.tex| & sample part file \\
    |cdocsdrf.tex| & sample redirection file \\
    |cdocsfn1.tex| & sample redirection file \\
    |cdocsfn2.tex| & sample redirection file \\
    |childdoc.pdf| & manual
\end{tabular}
\end{center}
%
The distribution consists of the files
|README.txt|, |childdoc.ins| and |childdoc.dtx|.
%
\begin{itemize}
\item
Run (pdf)\LaTeX{} on |childdoc.dtx|
to compile the manual |childdoc.pdf| (this file).
\item
Run \LaTeX{} on |childdoc.ins| to create the definitions file |childdoc.def|
and the sample |cdocsamp.tex| with include files
|cdocsch1.tex|, |cdocsch2.tex|, |cdocspt3.tex|, |cdocspt4.tex|,
|cdocsdrf.tex|, |cdocsfn1.tex|, |cdocsfn2.tex|.
Then copy the file |childdoc.def| to an appropriate directory of your \LaTeX{}
distribution, e.g.\ \textit{texmf-root}|/tex/latex/childdoc|.
\end{itemize}

%%%%%%%%%%%%%%%%%%%%%%%%%%%%%%%%%%%%%%%%%%%%%%%%%%%%%%%%%%%%%%%%%%%%%%%%%%%%%%%%
\subsection{Related CTAN Packages}

There are several other packages which offer a similar functionality:
%
\begin{itemize}
\item
The packages
\href{http://ctan.org/pkg/docmute}{\textsf{docmute}},
\href{http://ctan.org/pkg/includex}{\textsf{includex}} and
\href{http://ctan.org/pkg/standalone}{\textsf{standalone}}
provide commands to include only the document body of
a child file thus allowing both files to be compiled individually.
\item
The packages \href{http://ctan.org/pkg/subdocs}{\textsf{subdocs}}
and \href{http://ctan.org/pkg/subfiles}{\textsf{subfiles}}
provide structures in which the main and child documents can be
encapsulated and allowing them to be compiled individually.
The inclusion mechanism is different from the conventional |\include|.
\item
The package \href{http://ctan.org/pkg/combine}{\textsf{combine}}
is an elaborate solution to combine several documents into one.
\end{itemize}
%
See also the CTAN topic \href{http://ctan.org/topic/subdocs}{\textsf{subdocs}}
for further related packages.
The present package differs from the above solutions in that
a document structure constructed with the conventional |\include| mechanism
just needs two extra commands at the top of every file
such that all constituent files can be compiled individually.

%%%%%%%%%%%%%%%%%%%%%%%%%%%%%%%%%%%%%%%%%%%%%%%%%%%%%%%%%%%%%%%%%%%%%%%%%%%%%%%%
%\subsection{Feature Suggestions}
%
%The following is a list of features which may be useful for future
%versions of this package:
%%
%\begin{itemize}
%\item
%\ldots
%\end{itemize}

%%%%%%%%%%%%%%%%%%%%%%%%%%%%%%%%%%%%%%%%%%%%%%%%%%%%%%%%%%%%%%%%%%%%%%%%%%%%%%%%
\subsection{Revision History}

%%%%%%%%%%%%%%%%%%%%%%%%%%%%%%%%%%%%%%%%
\paragraph{v2.0:} 2018/12/30

\begin{itemize}
\item
immediate forward processing
\item
added |\childdocby| mechanism
\item
manual restructured
\end{itemize}

%%%%%%%%%%%%%%%%%%%%%%%%%%%%%%%%%%%%%%%%
\paragraph{v1.6:} 2018/01/17

\begin{itemize}
\item
application for development of include files
\item
corrections to manual
\end{itemize}

%%%%%%%%%%%%%%%%%%%%%%%%%%%%%%%%%%%%%%%%
\paragraph{v1.5:} 2017/05/21

\begin{itemize}
\item
more complete structuring introduced
\item
|\childdocof| introduced
\item
|\childdoc| renamed to |\childdocmain|
\item
|\childredirect| renamed to |\childdocforward| and |\childdocforwardprefix|
and functionality expanded
\end{itemize}

%%%%%%%%%%%%%%%%%%%%%%%%%%%%%%%%%%%%%%%%
\paragraph{v1.0:} 2017/04/27

\begin{itemize}
\item
manual and install package
\item
first version published on CTAN
\end{itemize}

%%%%%%%%%%%%%%%%%%%%%%%%%%%%%%%%%%%%%%%%
\paragraph{v0.6:} 2017/04/26

\begin{itemize}
\item
redirection mechanism added
\end{itemize}

%%%%%%%%%%%%%%%%%%%%%%%%%%%%%%%%%%%%%%%%
\paragraph{v0.5:} 2017/04/26

\begin{itemize}
\item
functionality in definition file
\end{itemize}


%%%%%%%%%%%%%%%%%%%%%%%%%%%%%%%%%%%%%%%%%%%%%%%%%%%%%%%%%%%%%%%%%%%%%%%%%%%%%%%%
%%%%%%%%%%%%%%%%%%%%%%%%%%%%%%%%%%%%%%%%%%%%%%%%%%%%%%%%%%%%%%%%%%%%%%%%%%%%%%%%
%%%%%%%%%%%%%%%%%%%%%%%%%%%%%%%%%%%%%%%%%%%%%%%%%%%%%%%%%%%%%%%%%%%%%%%%%%%%%%%%
\appendix

\settowidth\MacroIndent{\rmfamily\scriptsize 000\ }

 \DocInput{childdoc.dtx}

\end{document}
%</driver>
% \fi
%
% %%%%%%%%%%%%%%%%%%%%%%%%%%%%%%%%%%%%%%%%%%%%%%%%%%%%%%%%%%%%%%%%%%%%%%%%%%%%%%
% %%%%%%%%%%%%%%%%%%%%%%%%%%%%%%%%%%%%%%%%%%%%%%%%%%%%%%%%%%%%%%%%%%%%%%%%%%%%%%
% \section{Sample}
%\iffalse
%<*samplemain>
%\fi
%
% The following presents a sample document
% with two chapters, two parts, a title page,
% a compile flag as well as three forwarding files to set the flag.
% It consists of eight |.tex| files:
% \begin{center}
% \begin{tabular}{ll}
% |cdocsamp.tex|&main file\\
% |cdocsch1.tex|&include file for chapter 1\\
% |cdocsch2.tex|&include file for chapter 2\\
% |cdocspt3.tex|&include file for part 3\\
% |cdocspt4.tex|&include file for part 4\\
% |cdocsdrf.tex|&forwarding file for main file in draft mode\\
% |cdocsfi1.tex|&forwarding file for final version of chapter 1\\
% |cdocsfi2.tex|&forwarding file for final version of chapter 2\\
% \end{tabular}
% \end{center}
% Each of the eight files can be compiled directly by the \LaTeX{} compiler.
%
% %%%%%%%%%%%%%%%%%%%%%%%%%%%%%%%%%%%%%%
% \paragraph{Main File.}
%
% The main file is called |cdocsamp.tex|.
%
% Load the \textsf{childdoc} definitions and
% declare the filename for the main document:
%    \begin{macrocode}
\input{childdoc.def}
\childdocmain{}
%    \end{macrocode}

% Optional override for |\version| flag:
%    \begin{macrocode}
%%\ifchilddoc\else\providecommand{\version}{draft}\fi
%    \end{macrocode}

% Define the default values for the |\version| flag
% (|final| for the main file and |draft| for childs):
%    \begin{macrocode}
\ifchilddoc
\providecommand{\version}{draft}
\else
\providecommand{\version}{final}
\fi
%    \end{macrocode}

% Load the standard document class:
%    \begin{macrocode}
\documentclass[12pt]{article}
%    \end{macrocode}

% Start the document body:
%    \begin{macrocode}
\begin{document}
%    \end{macrocode}

% Declare a title page.
% Print title, part of document being processed and version flag:
%    \begin{macrocode}
\addtocounter{page}{-1}
\begin{center}
{\LARGE\bfseries{}childdoc example\par}
\vspace{1cm}
\ifchilddoc
\ifchilddocmanual part\else chapter\fi:
`\childdocname' of `\childdocjob'\par
\else
main document: `\childdocjob'\par
\fi
version: \version\par
\end{center}
\newpage
%    \end{macrocode}

% Manually include selected file,
% otherwise process as usual:
%    \begin{macrocode}
\ifchilddocmanual
\section*{part `\childdocname'}
\input{\childdocname}
\else
%    \end{macrocode}

% Include the two chapters:
%    \begin{macrocode}
\include{cdocsch1}
\include{cdocsch2}
%    \end{macrocode}

% Include the two parts unless only chapters should be displayed:
%    \begin{macrocode}
\ifchilddoc\else
\section{part three}
\input{cdocspt3}
\section{part four}
\input{cdocspt4}
\fi
%    \end{macrocode}

% Process as usual until here:
%    \begin{macrocode}
\fi
%    \end{macrocode}

% End of document body:
%    \begin{macrocode}
\end{document}
%    \end{macrocode}
%\iffalse
%</samplemain>
%\fi
%
% %%%%%%%%%%%%%%%%%%%%%%%%%%%%%%%%%%%%%%
% \paragraph{Chapter Include Files.}
%
% The include files are called |cdocsch1.tex| and |cdocsch2.tex|.
%
%\iffalse
%<*samplechap1|samplechap2>
%\fi

% Optional override for |\version| flag:
%    \begin{macrocode}
%%\providecommand{\version}{final}
%    \end{macrocode}

% Include the main document:
%    \begin{macrocode}
\input{childdoc.def}
\childdocof{cdocsamp}
%    \end{macrocode}

%\iffalse
%</samplechap1|samplechap2>
%\fi
%
%\iffalse
%<*samplechap1>
%\fi
% Some text for chapter 1:
%    \begin{macrocode}
\section{one}
some text in chapter one
%    \end{macrocode}

%\iffalse
%</samplechap1>
%\fi
% Some text for chapter 2:
%\iffalse
%<*samplechap2>
%\fi
%    \begin{macrocode}
\section{two}
more text in chapter two
%    \end{macrocode}

%\iffalse
%</samplechap2>
%\fi
%
% %%%%%%%%%%%%%%%%%%%%%%%%%%%%%%%%%%%%%%
% \paragraph{Part Include Files.}
%
% The include files are called |cdocspt3.tex| and |cdocspt4.tex|.
%
%\iffalse
%<*samplepart3|samplepart4>
%\fi

% Optional override for |\version| flag:
%    \begin{macrocode}
%%\providecommand{\version}{final}
%    \end{macrocode}

% Include the main document:
%    \begin{macrocode}
\input{childdoc.def}
\childdocby{cdocsamp}
%    \end{macrocode}

%\iffalse
%</samplepart3|samplepart4>
%\fi
%
%\iffalse
%<*samplepart3>
%\fi
% Some text for part 3:
%    \begin{macrocode}
some text in part three
%    \end{macrocode}

%\iffalse
%</samplepart3>
%\fi
% Some text for part 4:
%\iffalse
%<*samplepart4>
%\fi
%    \begin{macrocode}
more text in part four
%    \end{macrocode}

%\iffalse
%</samplepart4>
%\fi
%
% %%%%%%%%%%%%%%%%%%%%%%%%%%%%%%%%%%%%%%
% \paragraph{Forwarding for a Complete Draft.}
%
% The following forwarding file |cdocsdrf.tex|
% compiles the main document in draft mode:
%\iffalse
%<*sampledraft>
%\fi
%    \begin{macrocode}
\def\version{draft}
\input{childdoc.def}
\childdocforward{cdocsamp}
%    \end{macrocode}

%\iffalse
%</sampledraft>
%\fi
%
% %%%%%%%%%%%%%%%%%%%%%%%%%%%%%%%%%%%%%%
% \paragraph{Forwarding for Final Version of the Chapters.}
%
% The following forwarding files |cdocsfn1.tex| and |cdocsfn2.tex|
% (with identical content)
% compile the final versions of the child documents
% |cdocsch1.tex| and |cdocsch2.tex|, respectively:
%\iffalse
%<*samplefinal>
%\fi
%    \begin{macrocode}
\def\version{final}
\input{childdoc.def}
\childdocforwardprefix[cdocsamp]{cdocsfn}{cdocsch}
%    \end{macrocode}

%\iffalse
%</samplefinal>
%\fi
%
% %%%%%%%%%%%%%%%%%%%%%%%%%%%%%%%%%%%%%%
% \paragraph{Command Line Processing.}
%
% The following three command lines generate the output files
% |cdocscld|, |cdocscl1| and |cdocscl2|
% which should be identical to
% |cdocsdrf|, |cdocsch1| and |cdocsfn2|, respectively:
% \begin{center}
% \begin{tabular}{l}
% |latex -jobname cdocscld \|\\
% |  "\def\version{draft}\input{childdoc.def}\childdocforward{cdocsamp}"|\\
% |latex -jobname cdocscl1 \|\\
% |  "\input{childdoc.def}\childdocforward[cdocsamp]{cdocsch1}"|\\
% |latex -jobname cdocscl2 \|\\
% |  "\def\version{final}\input{childdoc.def}\childdocforward{cdocsch2}"|
% \end{tabular}
% \end{center}
% Note that the trailing backslash on each first line
% merely continues the input to the second line
% (for convenient cut ant paste).
% Furthermore, the command |latex| can be replaced by any
% of its alternative versions such as |pdflatex|.
%
% %%%%%%%%%%%%%%%%%%%%%%%%%%%%%%%%%%%%%%%%%%%%%%%%%%%%%%%%%%%%%%%%%%%%%%%%%%%%%%
% %%%%%%%%%%%%%%%%%%%%%%%%%%%%%%%%%%%%%%%%%%%%%%%%%%%%%%%%%%%%%%%%%%%%%%%%%%%%%%
% \section{Implementation}
%\iffalse
%<*package>
%\fi
%
% This section describes the definitions file |childdoc.def|.

% The definitions cannot be loaded using |\usepackage| or |\RequirePackage|
% which has a mechanism to prevent loading a style file more than once.
% When loading the definitions by means of |\input|
% multiple instances have to be prevented manually:
%\iffalse
%This code needs to be before the `\ProvidesFile' directive
%which is defined at the beginning of this file.
%Therefore it is also placed there and commented out here.
%</package>
%<*discard>
%\fi
%    \begin{macrocode}
\ifdefined\childdocmain\endinput\fi
%    \end{macrocode}
%\iffalse
%</discard>
%<*package>
%\fi
%
% \macro{\ifchilddoc}
% \macro{\ifchilddocmanual}
% The conditional |\ifchilddoc| tells whether a
% child (true) or main (false) document is being compiled.
% The conditional |\ifchilddocmanual| tells whether
% the |\includeonly| mechanism is used (false) or
% the selection of child files must be performed manually (true).
% The definitions initialise to false:
%    \begin{macrocode}
\newif\ifchilddoc
\newif\ifchilddocmanual
%    \end{macrocode}

% \macro{\childdocname}
% \macro{\childdocjob}
% The macro |\childdocname| stores the name of the main document
% to be compiled. The macro |\childdocjob| stores the name of
% the document on which the \LaTeX{} compiler was originally invoked.
% The content of |\jobname| cannot be compared
% to filenames specified in the source due to different catcodes.
% The following code rescans |\jobname|, stores the result
% in |\childdocname| and saves a copy in |\childdocjob|:
%    \begin{macrocode}
\edef\childdocname{\scantokens\expandafter{\jobname\noexpand}}
\let\childdocjob\childdocname
%    \end{macrocode}

% \macro{\childdocdisable}
% The macro |\childdocdisable| prevents the main file
% from being processed more than once.
% At this stage, the main document command |\childdocmain|
% is assumed to be called once again where it should do nothing.
% Any subsequent call to it should prevent
% a secondary processing of the main document
% It overwrites the forwarding commands
% |\childdocof| and |\childdocforward|
% with empty macros to prevent further inclusions of the main document:
%    \begin{macrocode}
\newcommand{\childdocdisable}
{
  \renewcommand{\childdocmain}[1]{\renewcommand{\childdocmain}[1]{\endinput}}
  \renewcommand{\childdocof}[1]{}
  \renewcommand{\childdocby}[2][]{}
  \renewcommand{\childdocforward}[2][]{}
  \renewcommand{\childdocdisable}{}
}
%    \end{macrocode}

% \macro{\childdocmain}
% The macro |\childdocmain| is to be called at the top of the main file
% with nothing or the main filename (without extension) as argument.
% First, it breaks loops.
% If the argument is not empty and does not match |\childdocname|
% (which is set by the first inclusion of |childdoc.def|),
% |\ifchilddoc| is set to true, |\includeonly| is applied to the child file
% and |\jobname| is set to the main file
% (for proper handling of |.aux| files):
%    \begin{macrocode}
\newcommand{\childdocmain}[1]
{
  \childdocdisable\childdocmain{}
  \if?#1?\else
    \begingroup
      \def\childdoctmp{#1}
      \ifx\childdoctmp\childdocname
        \def\childdoctmp{}
      \else
        \def\childdoctmp
        {
          \childdoctrue
          \includeonly{\childdocname}
          \def\childdocjob{#1}
          \def\jobname{#1}
        }
      \fi
      \expandafter
    \endgroup
    \childdoctmp
  \fi
}
%    \end{macrocode}

% \macro{\childdocof}
% The command |\childdocof| redirects
% compilation to the main file |#1|.
%    \begin{macrocode}
\newcommand{\childdocof}[1]
{
  \childdocdisable
  \childdoctrue
  \includeonly{\childdocname}
  \def\jobname{#1}
  \def\childdocjob{#1}
  \input{#1}
}
%    \end{macrocode}

% \macro{\childdocby}
% The command |\childdocby| ....
%    \begin{macrocode}
\newcommand{\childdocby}[2][]
{
  \childdocdisable
  \childdoctrue
  \childdocmanualtrue
  \if?#1?\else
    \def\jobname{#2}
  \fi
  \def\childdocjob{#2}
  \input{#2}
  \endinput
}
%    \end{macrocode}

% \macro{\childdocforward}
% The command |\childdocforward| redirects
% compilation to the main file or
% (if the optional argument is given) a child file.
% Parameters are set as if the main file
% or a child file starting with |\childdocof| was compiled.
% Then compilation is handed over to the main file:
%    \begin{macrocode}
\newcommand{\childdocforward}[2][]
{
  \begingroup
    \if?#1?
      \def\childdoctmp
      {
        \def\childdocname{#2}
        \def\childdocjob{#2}
        \def\jobname{#2}
        \input{#2}
        \endinput
      }
    \else
      \def\childdoctmp
      {
        \childdocdisable
        \def\childdocname{#2}
        \childdoctrue
        \includeonly{#2}
        \def\childdocjob{#1}
        \def\jobname{#1}
        \input{#1}
        \endinput
      }
    \fi
    \expandafter
  \endgroup
  \childdoctmp
}
%    \end{macrocode}

% \macro{\childdocforwardprefix}
% The command |\childdocforwardprefix| redirects
% compilation to the main or a child file by means of a pattern.
% The prefix |#1| in the current filename is replaced by |#2|
% and the suffix of the current filename is kept
% (it is assumed that the filename does not contain the substring `|~~~|'
% which is used as a delimiter).
% Compilation is handed over to the new file by |\childdocforward|:
%    \begin{macrocode}
\newcommand{\childdocforwardprefix}[3][]
{
  \begingroup
    \def\childdocextract #2##1~~~{\def\childdoctmp{\childdocforward[#1]{#3##1}}}
    \expandafter\childdocextract\childdocname~~~
    \expandafter
  \endgroup
  \childdoctmp
}
%    \end{macrocode}

% \macro{\childdoc}
% The deprecated macro |\childdoc| is a legacy version of |\childdocmain|:
%    \begin{macrocode}
\newcommand{\childdoc}{\childdocmain}
%    \end{macrocode}

% \macro{\childdocredirect}
% The deprecated macro |\childdocredirect| is a legacy version
% of |\childdocforward| and |\childdocforwardprefix|:
%    \begin{macrocode}
\newcommand{\childdocredirect}[2][]
{
  \begingroup
    \if?#1?
      \def\childdoctmp{\childdocforward{#2}}
    \else
      \def\childdoctmp{\childdocforwardprefix{#1}{#2}}
    \fi
    \expandafter
  \endgroup
  \childdoctmp
}
%    \end{macrocode}

%\iffalse
%</package>
%\fi
%
\endinput

\childdocof{cdocsamp}
%    \end{macrocode}

%\iffalse
%</samplechap1|samplechap2>
%\fi
%
%\iffalse
%<*samplechap1>
%\fi
% Some text for chapter 1:
%    \begin{macrocode}
\section{one}
some text in chapter one
%    \end{macrocode}

%\iffalse
%</samplechap1>
%\fi
% Some text for chapter 2:
%\iffalse
%<*samplechap2>
%\fi
%    \begin{macrocode}
\section{two}
more text in chapter two
%    \end{macrocode}

%\iffalse
%</samplechap2>
%\fi
%
% %%%%%%%%%%%%%%%%%%%%%%%%%%%%%%%%%%%%%%
% \paragraph{Part Include Files.}
%
% The include files are called |cdocspt3.tex| and |cdocspt4.tex|.
%
%\iffalse
%<*samplepart3|samplepart4>
%\fi

% Optional override for |\version| flag:
%    \begin{macrocode}
%%\providecommand{\version}{final}
%    \end{macrocode}

% Include the main document:
%    \begin{macrocode}
% \iffalse
%
% childdoc.dtx Copyright (C) 2017-2018 Niklas Beisert
%
% This work may be distributed and/or modified under the
% conditions of the LaTeX Project Public License, either version 1.3
% of this license or (at your option) any later version.
% The latest version of this license is in
%   http://www.latex-project.org/lppl.txt
% and version 1.3 or later is part of all distributions of LaTeX
% version 2005/12/01 or later.
%
% This work has the LPPL maintenance status `maintained'.
%
% The Current Maintainer of this work is Niklas Beisert.
%
% This work consists of the files childdoc.dtx and childdoc.ins
% and the derived files childdoc.def and cdocsamp.tex with
% cdocsch1.tex, cdocsch2.tex, cdocsdrf.tex, cdocsfn1.tex, cdocsfn2.tex.
%
%<package>\ifdefined\childdocmain\endinput\fi
%<package>\ProvidesFile{childdoc.def}[2018/12/30 v2.0 child document driver]
%<samplemain>\ProvidesFile{cdocsamp.tex}[2018/12/30 v2.0 sample for childdoc]
%<*driver>
%\ProvidesFile{childdoc.drv}[2018/12/30 v2.0 childdoc reference manual file]
\PassOptionsToClass{10pt,a4paper}{article}
\documentclass{ltxdoc}

\usepackage[margin=35mm]{geometry}
\usepackage{hyperref}
\usepackage{hyperxmp}
\usepackage[usenames]{color}

\hypersetup{colorlinks=true}
\hypersetup{pdfstartview=FitH}
\hypersetup{pdfpagemode=UseNone}
\hypersetup{pdfsource={}}
\hypersetup{pdflang={en-UK}}
\hypersetup{pdfcopyright={Copyright 2017-2018 Niklas Beisert.
  This work may be distributed and/or modified under the
  conditions of the LaTeX Project Public License, either version 1.3
  of this license or (at your option) any later version.}}
\hypersetup{pdflicenseurl={http://www.latex-project.org/lppl.txt}}
\hypersetup{pdfcontactaddress={ETH Zurich, ITP, HIT K,
  Wolfgang-Pauli-Strasse 27}}
\hypersetup{pdfcontactpostcode={8093}}
\hypersetup{pdfcontactcity={Zurich}}
\hypersetup{pdfcontactcountry={Switzerland}}
\hypersetup{pdfcontactemail={nbeisert@itp.phys.ethz.ch}}
\hypersetup{pdfcontacturl={http://people.phys.ethz.ch/\xmptilde nbeisert/}}

\newcommand{\secref}[1]{\hyperref[#1]{section \ref*{#1}}}

\parskip1ex
\parindent0pt
\let\olditemize\itemize
\def\itemize{\olditemize\parskip0pt}

\begin{document}

\title{The \textsf{childdoc} Package}
\hypersetup{pdftitle={The childdoc Package}}
\author{Niklas Beisert\\[2ex]
  Institut f\"ur Theoretische Physik\\
  Eidgen\"ossische Technische Hochschule Z\"urich\\
  Wolfgang-Pauli-Strasse 27, 8093 Z\"urich, Switzerland\\[1ex]
  \href{mailto:nbeisert@itp.phys.ethz.ch}
  {\texttt{nbeisert@itp.phys.ethz.ch}}}
\hypersetup{pdfauthor={Niklas Beisert}}
\hypersetup{pdfsubject={Manual for the LaTeX2e Package childdoc}}
\date{30 December 2018, \textsf{v2.0}}
\maketitle

\begin{abstract}\noindent
\textsf{childdoc} is a \LaTeXe{} package
that enables the direct compilation
of document sections included by |\include|
to individual files.
\end{abstract}

\begingroup
\parskip0ex
\tableofcontents
\endgroup

%%%%%%%%%%%%%%%%%%%%%%%%%%%%%%%%%%%%%%%%%%%%%%%%%%%%%%%%%%%%%%%%%%%%%%%%%%%%%%%%
%%%%%%%%%%%%%%%%%%%%%%%%%%%%%%%%%%%%%%%%%%%%%%%%%%%%%%%%%%%%%%%%%%%%%%%%%%%%%%%%
\section{Introduction}

\LaTeX{} provides a mechanism to structure a large document (such as a book)
into a main file and several child files (containing the chapters)
using the |\include| command.
This mechanism is beneficial for documents
which span hundreds of pages in order to
make the source file(s) more manageable.
Moreover, compilation can be restricted to
selected child files by means of the |\includeonly| command.
The latter feature can be used to reduce the compilation time while editing
(this was significantly more useful in the earlier days of \LaTeX{})
or to generate a smaller document which is easier to navigate.
Another application of |\includeonly| is to generate
documents consisting of selected parts of the complete document.

However, there are a few drawbacks of the plain |\include| mechanism:
\begin{itemize}
\item
The child files cannot be compiled on their own,
they can only be compiled via the main file.
A naive editing environment
(such as a text editor with an option
to have the current file processed by \LaTeX)
may require one to switch to the main file before compiling;
attempting to compile the child file produces errors.
\item
The main file must be modified (each time)
to adjust the |\includeonly| command
to the present needs. This easily leaves the main file in a messy state.
\item
The generated document will always carry the filename
of the main document. This is inconvenient if
several child files are to be compiled and
to be kept for distribution.
\end{itemize}

The present package provides a simple interface
to make child files individually compilable by \LaTeX{}.
Compiling a child file then has the same effect as compiling
the main file with an |\includeonly| command
to select the appropriate child.
Moreover the generated document will carry the name of the child
rather than the main file.
This resolves all three above issues.

This feature is meant to make the editing of books,
thesis documents and lecture notes somewhat more convenient.
However, the package can also be used efficiently for
composing a series of documents (such as exercise sheets)
which are typically distributed individually.
It then assists the author in generating the individual documents
(potentially in different versions)
as well as a document containing the collected series.
Another application is in developing style files
or other kinds of included material
where compilation of the style file could redirect
to a sample or test file.

%%%%%%%%%%%%%%%%%%%%%%%%%%%%%%%%%%%%%%%%%%%%%%%%%%%%%%%%%%%%%%%%%%%%%%%%%%%%%%%%
%%%%%%%%%%%%%%%%%%%%%%%%%%%%%%%%%%%%%%%%%%%%%%%%%%%%%%%%%%%%%%%%%%%%%%%%%%%%%%%%
\section{Usage}

First of all, the package \textsf{childdoc} is \emph{not} a standard
\LaTeXe{} |.sty| style file! Therefore it needs to be invoked in
a non-standard way.

%%%%%%%%%%%%%%%%%%%%%%%%%%%%%%%%%%%%%%%%%%%%%%%%%%%%%%%%%%%%%%%%%%%%%%%%%%%%%%%%
\subsection{Included Files}
\label{sec:include}

%%%%%%%%%%%%%%%%%%%%%%%%%%%%%%%%%%%%%%%%
\DescribeMacro{\childdocmain}
To use the package, add the commands
\begin{center}
\begin{tabular}{l}
|\input{childdoc.def}|\\
|\childdocmain{}|\\
\end{tabular}
\end{center}
at the very top of the main \LaTeX{} file,
in particular \emph{before} the |\documentclass| statement!
The argument of |\childdocmain| should be left empty
(but it must be present).

%%%%%%%%%%%%%%%%%%%%%%%%%%%%%%%%%%%%%%%%
\DescribeMacro{\childdocof}
Furthermore, add the commands
\begin{center}
\begin{tabular}{l}
|\input{childdoc.def}|\\
|\childdocof{|\textit{main}|}|\\
\end{tabular}
\end{center}
at the top of every child file \textit{child}
which is included by |\include{|\textit{child}|}|
from within the main file
(or at least for those files to be compiled individually).
The argument \textit{main} must be the filename of the main file.

There are a couple of
considerations in setting up the main and child documents:

%%%%%%%%%%%%%%%%%%%%%%%%%%%%%%%%%%%%%%%%
\paragraph{Restrictions.}

Please note the following restrictions:
\begin{itemize}
\item
|\childdocmain| must be called with one argument \textit{main}
to ensure compatibility with earlier version of the package.
It must either be empty (|\childdocmain{}|)
or precisely match the filename of the main file in which it is specified.
See \secref{sec:detection} for further information.
\item
The filename \textit{main} must be specified without the |.tex| extension.
\item
The filename \textit{main} is case sensitive
(even in case-insensitive file systems)
due to internal string comparison.
\item
The argument \textit{main} should be fully expanded, it cannot be a macro.
\item
Subdirectories and special characters should be avoided in filenames.
\item
The command |\childdocmain{|\textit{main}|}| must be followed by a whitespace.
It should not be followed immediately by another command
or by a comment mark `|%|'.
This is because the \TeX{} parser reads the token immediately following
the argument of |\childdocmain| and puts it
at the beginning of every child section;
however, a white\-space is ignored.
\end{itemize}

%%%%%%%%%%%%%%%%%%%%%%%%%%%%%%%%%%%%%%%%
\paragraph{Content of Main File.}

It is advisable to place all content in the child files included by |\include|.
Any output contained in the main file will appear in all child documents
unless suppressed manually;
it cannot be suppressed automatically by the |\includeonly| directive
and thus should normally be avoided.
A method to include some content in the main file
by means of conditional processing is described in \secref{sec:conditional}.

%%%%%%%%%%%%%%%%%%%%%%%%%%%%%%%%%%%%%%%%
\paragraph{Page Numbering.}

When only a part of the document is compiled,
the appropriate numbering of pages
(as well as other status parameters)
is determined from the |.aux| files.
The latter contain information from previous passes.
However this information needs to propagate through
all intermediate child documents.
Therefore the page numbering in child documents may well
be inconsistent until the complete document is compiled at least once.

A useful (if unconventional) way to always ensure a consistent
page numbering is to restart the numbering in each child document
and denote the pages by `\textit{child}|.|\textit{page}'
where \textit{child} represents the chapter/section number of the child file.
This can be achieved by the command
|\numberwithin{page}{|\textit{child}|}|
of the \textsf{amsmath} package
where \textit{child} can be |chapter| or |section|
depending on the chosen structuring.
Alternatively, one can modify the macro |\thepage| appropriately
and reset the counter |page| at the start of each child file.

%%%%%%%%%%%%%%%%%%%%%%%%%%%%%%%%%%%%%%%%%%%%%%%%%%%%%%%%%%%%%%%%%%%%%%%%%%%%%%%%
\subsection{Conditional Processing}
\label{sec:conditional}

The package provides a mechanism to compile different versions
of a document. To customise the versions further some conditional processing
can come in handy to distinguish which version is being compiled.
The package provides two macros to describe the compilation context:

%%%%%%%%%%%%%%%%%%%%%%%%%%%%%%%%%%%%%%%%
\DescribeMacro{\ifchilddoc}
The conditional |\ifchilddoc| distinguishes between the compilation of
child documents and the main document:
%
\begin{center}
|\ifchilddoc |\textit{child-code}| |[|\||else |\textit{main-code}]| \||fi|
\end{center}

%%%%%%%%%%%%%%%%%%%%%%%%%%%%%%%%%%%%%%%%
\DescribeMacro{\childdocname}
\DescribeMacro{\childdocjob}
The macro |\childdocname| contains the filename (without extension)
of the main or child file being processed.
Note that |\childdocjob| will always contain the name of the main file.

%%%%%%%%%%%%%%%%%%%%%%%%%%%%%%%%%%%%%%%%
\paragraph{Title Page.}

Conditional processing can be used to include a title or banner page
in the main document when proper precautions are taken.
Importantly, the code in the main file should ensure that the page counter
(as well as other status parameters which are stored in the |.aux| files)
takes the same value after the conditional processing.
Otherwise the page numbers may take divergent values
depending on which part is compiled.

For example, a title page could be declared by:
%
\begin{center}
\begin{tabular}{l}
|\ifchilddoc\||else|\\
|\addtocounter{page}{-1}|\\
\textit{code for title page}\\
|\newpage|\\
|\||fi|
\end{tabular}
\end{center}
%
A banner page for the child documents can be generated by:
%
\begin{center}
\begin{tabular}{l}
|\ifchilddoc|\\
|\addtocounter{page}{-1}|\\
\textit{code for banner page}\\
|\newpage|\\
|\||fi|
\end{tabular}
\end{center}
%
Here one could write a message such as:
\begin{center}
|This is the part \childdocname{} of \childdocjob{}.|
\end{center}

%%%%%%%%%%%%%%%%%%%%%%%%%%%%%%%%%%%%%%%%%%%%%%%%%%%%%%%%%%%%%%%%%%%%%%%%%%%%%%%%
\subsection{Flags}
\label{sec:flags}

The package makes it easy to generate different versions
of the main or child documents.
To this end compilation flags can be defined
and assigned different default values.
They will be particularly useful in conjunction
with the forwarding mechanism described in \secref{sec:forward}.

For example, it may be useful to have a flag |\version|
which can be set to |draft| or |final|.
The document source will contain some conditional code
depending on the value of |\version|.
Suppose further, the flag should default to |final| for the main file
and to |draft| for child files
which is a natural assignment for editing the document.
This is achieved by placing the following code
in the preamble of the main document
(below the |\childdocmain| directive):
%
\begin{center}
\begin{tabular}{l}
|\ifchilddoc|\\
|\providecommand{\version}{draft}|\\
|\||else|\\
|\providecommand{\version}{final}|\\
|\||fi|
\end{tabular}
\end{center}
%
The definition by |\providecommand| makes sure
that previous definitions are not overwritten.
Further statements |\providecommand{\version}{...}|
can thus be added before the above code to override it.

For the main file, one might add a line
(between |\childdocmain| and the above block)
%
\begin{center}
|%\ifchilddoc\||else\providecommand{\version}{draft}\||fi|
\end{center}
%
which can be uncommented to produce a draft version.
Likewise one can add a line to the very top of a child file
(above the |\childdocof{|\textit{main}|}| directive)
%
\begin{center}
|%\providecommand{\version}{final}|
\end{center}
%
which can be uncommented to produce the final version of this child document.

%%%%%%%%%%%%%%%%%%%%%%%%%%%%%%%%%%%%%%%%%%%%%%%%%%%%%%%%%%%%%%%%%%%%%%%%%%%%%%%%
\subsection{Forwarding}
\label{sec:forward}

Different versions of the main or child documents
using compilation flags as described in \secref{sec:flags}
can be (permanently) stored in different files
for convenient compilation, viewing and distribution.
To this end, the package defines a command
to pass on compilation to a different file:

%%%%%%%%%%%%%%%%%%%%%%%%%%%%%%%%%%%%%%%%
\DescribeMacro{\childdocforward}
The command |\childdocforward| redirects processing to
another source file:
%
\begin{center}
\begin{tabular}{l}
|\input{childdoc.def}|\\
|\childdocforward[|\textit{main}|]{|\textit{dest}|}|\\
\end{tabular}
\end{center}
%
The argument \textit{dest} is the destination file
(without extension).
It should be the main file or one of the child files.
Note that further \textsf{childdoc} directives
such as |\childdocof| and |\childdocforward|
in the indicated file will be processed in this form.
The optional argument \textit{main}
passes on directly to the main file \textit{main}
while pretending to compile the child \textit{dest}.
This form behaves as if \textit{dest}
issues |\childdocof{|\textit{main}|}| right away,
and no further \textsf{childdoc} directives will be processed.

%%%%%%%%%%%%%%%%%%%%%%%%%%%%%%%%%%%%%%%%
\DescribeMacro{\...prefix}
In the alternative form |\childdocforwardprefix|,
%
\begin{center}
\begin{tabular}{l}
|\input{childdoc.def}|\\
|\childdocforwardprefix[|\textit{main}|]{|\textit{prefix}|}{|\textit{dest}|}|
\end{tabular}
\end{center}
%
the destination file is determined by a pattern
depending on the current file:
To make this work, the current file must be called
`{\textit{prefix}\hspace{0.2em}\textit{suffix}}'
with \textit{prefix} matching precisely the argument.
Processing is then passed on to the file
`{\textit{dest}\hspace{0.2em}\textit{suffix}}'.
Surely, the same effect is achieved by
directly specifying the
argument `{\textit{dest}\hspace{0.2em}\textit{suffix}}'
in the first form.
However, that requires to set up a different file
for each child. With the alternative form of the command
all these files can have exactly the same content
which simplifies setting them up and maintaining them.

For example, the following file |draft.tex|
with a compilation flag |\version| as described in \secref{sec:flags}
compiles the main document as a draft:
%
\begin{center}
\begin{tabular}{l}
|\def\version{draft}|\\
|\input{childdoc.def}|\\
|\childdocforward{|\textit{main}|}|
\end{tabular}
\end{center}
%
Likewise, the following files |final|\textit{nn}|.tex|
compile the final version of the child document
|child|\textit{nn}|.tex|:
%
\begin{center}
\begin{tabular}{l}
|\def\version{final}|\\
|\input{childdoc.def}|\\
|\childdocforwardprefix{final}{child}|
\end{tabular}
\end{center}
%

Note that when several versions of a main file and/or of each child file
are to be generated, it may be convenient to set up a |Makefile| or
shell script to automatise the process.

%%%%%%%%%%%%%%%%%%%%%%%%%%%%%%%%%%%%%%%%%%%%%%%%%%%%%%%%%%%%%%%%%%%%%%%%%%%%%%%%
\subsection{Command Line Processing}
\label{sec:commandline}

The effect of redirection files can also be achieved by invoking
the \LaTeX{} compiler with a more elaborate command line.
Most conveniently this should be done as part
of a shell script or a |Makefile|.

When using \textsf{childdoc} in the main file, the following
command lines effectively perform a redirection
(note that depending on the shell being used,
backslashes may have to be doubled: `|\|' $\to$ `|\\|'):
%
\begin{center}
|... -jobname "|\textit{target}|" |\\|"|[\textit{flags}]%
|\input{childdoc.def}\childdocforward[|\textit{main}|]{|\textit{dest}|}"|
\end{center}
%
Here \textit{target} is the name of the output file,
\textit{main} is the name of the main file
and \textit{dest} is the name of the main or child file to be processed
(all filenames without extensions).
The optional argument \textit{main} can be omitted
if \textit{main} matches \textit{dest}.
Optionally, compilation \textit{flags} can be defined via |\def| commands.
This command line makes the \TeX{} engine believe
it is compiling the file \textit{target}
whose content is specified as the latter parameter.
The provided code then forwards the processing to
\textit{main} or \textit{dest} as described in \secref{sec:forward}.

%%%%%%%%%%%%%%%%%%%%%%%%%%%%%%%%%%%%%%%%%%%%%%%%%%%%%%%%%%%%%%%%%%%%%%%%%%%%%%%%
\subsection{Include by Input}
\label{sec:input}

Including child documents by |\include| has some restrictions by design.
Most notably, the content of a child document always occupies
its own set of pages; pages cannot be shared between child documents.
Usually, this behaviour makes perfect sense
because each child document contain an essential part of the document.
However, in some situations it may be desirable to compose
a document from a collection of parts
without having mandatory page breaks between then.
For this case, the package
provides a mechanism to include parts
by |\input| which can also be processed individually.
However, by construction this mechanism
requires manual handling of the content to be output.

%%%%%%%%%%%%%%%%%%%%%%%%%%%%%%%%%%%%%%%%
\DescribeMacro{\ifchilddocmanual}
The main file should be prepared as usual, see \secref{sec:include}.
However, the document body must make a distinction
between processing of an individual part and of the main document, e.g.:
%
\begin{center}
\begin{tabular}{l}
|\ifchilddocmanual|\\
|\input{\childdocname}|\\
|\||else|\\
\textit{document body with }|\input{|\textit{part}|}|\\
|\||fi|
\end{tabular}
\end{center}
%
The conditional |\ifchilddocmanual| is true whenever
a part to be included by |\input| is being compiled,
and the name of the part is stored in |\childdocname|.

%%%%%%%%%%%%%%%%%%%%%%%%%%%%%%%%%%%%%%%%
\DescribeMacro{\childdocby}
Each part to be included by |\input| should start with:
%
\begin{center}
\begin{tabular}{l}
|\input{childdoc.def}|\\
|\childdocby{|\textit{main}|}|\\
\end{tabular}
\end{center}
%
The directive |\childdocby| is similar to |\childdocof|
described in \secref{sec:include},
but the subsequent selection of content must be done manually.
To that end, both |\ifchilddoc| and |\ifchilddocmanual|
will be true upon processing of a part,
and the name of the part is stored in |\childdocname|.
Note that |\jobname| will be set to the filename of the current part
so that each part receives an individual |.aux| file
that does not interfere with the |.aux| file(s) of the main document.
This behaviour can be altered by the alternative form
|\childdocby[*]{|\textit{main}|}| (with a non-empty optional argument)
which uses the |.aux| file of the main document
by setting |\jobname| to \textit{main}.

%%%%%%%%%%%%%%%%%%%%%%%%%%%%%%%%%%%%%%%%%%%%%%%%%%%%%%%%%%%%%%%%%%%%%%%%%%%%%%%%
\subsection{Driver Development}
\label{sec:driver}

The \textsf{childdoc} mechanism can also be use for the development
of definition files such as \LaTeX{} styles or classes.
This case differs from the above setup with multiple parts
included by |\include| in that no |\includeonly| should be invoked.
This can be achieved by starting the include file
(before |\ProvidesPackage|) with:
%
\begin{center}
\begin{tabular}{l}
|\input{childdoc.def}|\\
|\childdocforward{|\textit{main}|}|\\
\end{tabular}
\end{center}
%
or alternatively with:
%
\begin{center}
\begin{tabular}{l}
|\input{childdoc.def}|\\
|\childdocby{|\textit{main}|}|\\
\end{tabular}
\end{center}
%
Both forms have slightly different effects as described above.
The main file is prepared as usual, see \secref{sec:include}.

%%%%%%%%%%%%%%%%%%%%%%%%%%%%%%%%%%%%%%%%%%%%%%%%%%%%%%%%%%%%%%%%%%%%%%%%%%%%%%%%
\subsection{Legacy Detection}
\label{sec:detection}

The directive |\childdocmain| in the main file can detect
whether the complete document or merely a child is to be compiled
even without using the directive |\childdocof|.
This method is deprecated because it is less robust
and there is no compelling reason to use it;
it is merely provided for backward compatibility
and it may be removed in future versions.

If the detection mechanism is to be used,
it is mandatory to correctly specify
the filename of the main file as the argument of |\childdocmain|:
%
\begin{center}
\begin{tabular}{l}
|\input{childdoc.def}|\\
|\childdocmain{|\textit{main}|}|\\
\end{tabular}
\end{center}
%
If |\jobname| does not match the argument \textit{main} of |\childdocmain|,
it is assumed that |\jobname| points to the child file to be compiled.
When using |\childdocmain| with the main file specified as argument,
it suffices to start a child file
with just |\input{|\textit{main}|}|
without loading of the package and using |\childdocof|.
If instead all processing is done
with the appropriate \textsf{childdoc} directives,
the argument of \textit{main} of |\childdocmain| can be empty.

An alternative version of the command line processing described
in \secref{sec:commandline} using the detection mechanism reads:
%
\begin{center}
|... -jobname "|\textit{target}|" "|[\textit{flags}]%
[|\def\jobname{|\textit{dest}|}|]|\input{|\textit{main}|}"|
\end{center}

%%%%%%%%%%%%%%%%%%%%%%%%%%%%%%%%%%%%%%%%%%%%%%%%%%%%%%%%%%%%%%%%%%%%%%%%%%%%%%%%
\subsection{Manual Code}
\label{sec:manual}

In case one cannot be certain whether the definitions file |childdoc.def|
is installed on the target \TeX{} distribution
and one prefers not to ship it,
it is conceivable to paste a few relevant commands into the sources.

To that end, drop all statements |\input{childdoc.def}|
and perform the replacements as outlined below.
Instead of |\childdocmain{|\textit{main}|}| add the following code
to the top of the main file:
%
\begin{center}
\begin{tabular}{l}
|\||ifdefined\childdocname\endinput\||fi\newif\ifchilddoc|\\
|\edef\childdocname{\scantokens\expandafter{\jobname\noexpand}}|\\
|\def\childdocmain{|\textit{main}|}\||ifx\childdocmain\childdocname\||else|\\
|\childdoctrue\includeonly{\childdocname}\let\jobname\childdocmain\||fi|\\
\end{tabular}
\end{center}
%
Instead of |\childdocof{|\textit{main}|}| just include the main file
at the top of each child file:
%
\begin{center}
|\input{|\textit{main}|}|
\end{center}
%
A simple redirection |\childdocforward{|\textit{dest}|}| is achieved by:
%
\begin{center}
|\def\jobname{|\textit{dest}|}\input{\jobname}|
\end{center}
%
The redirection with prefix
|\childdocforwardprefix[|\textit{prefix}|]{|\textit{dest}|}|
is accomplished by:
%
\begin{center}
\begin{tabular}{l}
|{\edef\jobname{\scantokens\expandafter{\jobname\noexpand}}|\\
|\def\redirectjob |\textit{prefix}|#1~~~{\gdef\jobname{|\textit{dest}|#1}}|\\
|\expandafter\redirectjob\jobname~~~}\input{\jobname}|
\end{tabular}
\end{center}

In an alternative approach,
child documents can be compiled by a specific command line
without additional code or specific definitions:
%
\begin{center}
|... -jobname "|\textit{target}|" "|[\textit{flags}]%
|\includeonly{|\textit{dest}|}\input{|\textit{main}|}"|
\end{center}
%

%%%%%%%%%%%%%%%%%%%%%%%%%%%%%%%%%%%%%%%%%%%%%%%%%%%%%%%%%%%%%%%%%%%%%%%%%%%%%%%%
%%%%%%%%%%%%%%%%%%%%%%%%%%%%%%%%%%%%%%%%%%%%%%%%%%%%%%%%%%%%%%%%%%%%%%%%%%%%%%%%
\section{Information}

%%%%%%%%%%%%%%%%%%%%%%%%%%%%%%%%%%%%%%%%%%%%%%%%%%%%%%%%%%%%%%%%%%%%%%%%%%%%%%%%
\subsection{Copyright}

Copyright \copyright{} 2017--2018 Niklas Beisert

This work may be distributed and/or modified under the
conditions of the \LaTeX{} Project Public License, either version 1.3
of this license or (at your option) any later version.
The latest version of this license is in
  \url{http://www.latex-project.org/lppl.txt}
and version 1.3 or later is part of all distributions of \LaTeX{}
version 2005/12/01 or later.

This work has the LPPL maintenance status `maintained'.

The Current Maintainer of this work is Niklas Beisert.

This work consists of the files |README.txt|, |childdoc.ins| and |childdoc.dtx|
as well as the derived files |childdoc.def|, |cdocsamp.tex|
with |cdocsch1.tex|, |cdocsch2.tex|, |cdocspt3.tex|, |cdocspt4.tex|,
|cdocsdrf.tex|, |cdocsfn1.tex|, |cdocsfn2.tex|
as well as |childdoc.pdf|.

%%%%%%%%%%%%%%%%%%%%%%%%%%%%%%%%%%%%%%%%%%%%%%%%%%%%%%%%%%%%%%%%%%%%%%%%%%%%%%%%
\subsection{Files and Installation}

The package consists of the files:
%
\begin{center}
\begin{tabular}{ll}
    |README.txt|   & readme file \\
    |childdoc.ins| & installation file \\
    |childdoc.dtx| & source file \\
    |childdoc.def| & definition file \\
    |cdocsamp.tex| & sample main file \\
    |cdocsch1.tex| & sample include file \\
    |cdocsch2.tex| & sample include file \\
    |cdocspt3.tex| & sample part file \\
    |cdocspt4.tex| & sample part file \\
    |cdocsdrf.tex| & sample redirection file \\
    |cdocsfn1.tex| & sample redirection file \\
    |cdocsfn2.tex| & sample redirection file \\
    |childdoc.pdf| & manual
\end{tabular}
\end{center}
%
The distribution consists of the files
|README.txt|, |childdoc.ins| and |childdoc.dtx|.
%
\begin{itemize}
\item
Run (pdf)\LaTeX{} on |childdoc.dtx|
to compile the manual |childdoc.pdf| (this file).
\item
Run \LaTeX{} on |childdoc.ins| to create the definitions file |childdoc.def|
and the sample |cdocsamp.tex| with include files
|cdocsch1.tex|, |cdocsch2.tex|, |cdocspt3.tex|, |cdocspt4.tex|,
|cdocsdrf.tex|, |cdocsfn1.tex|, |cdocsfn2.tex|.
Then copy the file |childdoc.def| to an appropriate directory of your \LaTeX{}
distribution, e.g.\ \textit{texmf-root}|/tex/latex/childdoc|.
\end{itemize}

%%%%%%%%%%%%%%%%%%%%%%%%%%%%%%%%%%%%%%%%%%%%%%%%%%%%%%%%%%%%%%%%%%%%%%%%%%%%%%%%
\subsection{Related CTAN Packages}

There are several other packages which offer a similar functionality:
%
\begin{itemize}
\item
The packages
\href{http://ctan.org/pkg/docmute}{\textsf{docmute}},
\href{http://ctan.org/pkg/includex}{\textsf{includex}} and
\href{http://ctan.org/pkg/standalone}{\textsf{standalone}}
provide commands to include only the document body of
a child file thus allowing both files to be compiled individually.
\item
The packages \href{http://ctan.org/pkg/subdocs}{\textsf{subdocs}}
and \href{http://ctan.org/pkg/subfiles}{\textsf{subfiles}}
provide structures in which the main and child documents can be
encapsulated and allowing them to be compiled individually.
The inclusion mechanism is different from the conventional |\include|.
\item
The package \href{http://ctan.org/pkg/combine}{\textsf{combine}}
is an elaborate solution to combine several documents into one.
\end{itemize}
%
See also the CTAN topic \href{http://ctan.org/topic/subdocs}{\textsf{subdocs}}
for further related packages.
The present package differs from the above solutions in that
a document structure constructed with the conventional |\include| mechanism
just needs two extra commands at the top of every file
such that all constituent files can be compiled individually.

%%%%%%%%%%%%%%%%%%%%%%%%%%%%%%%%%%%%%%%%%%%%%%%%%%%%%%%%%%%%%%%%%%%%%%%%%%%%%%%%
%\subsection{Feature Suggestions}
%
%The following is a list of features which may be useful for future
%versions of this package:
%%
%\begin{itemize}
%\item
%\ldots
%\end{itemize}

%%%%%%%%%%%%%%%%%%%%%%%%%%%%%%%%%%%%%%%%%%%%%%%%%%%%%%%%%%%%%%%%%%%%%%%%%%%%%%%%
\subsection{Revision History}

%%%%%%%%%%%%%%%%%%%%%%%%%%%%%%%%%%%%%%%%
\paragraph{v2.0:} 2018/12/30

\begin{itemize}
\item
immediate forward processing
\item
added |\childdocby| mechanism
\item
manual restructured
\end{itemize}

%%%%%%%%%%%%%%%%%%%%%%%%%%%%%%%%%%%%%%%%
\paragraph{v1.6:} 2018/01/17

\begin{itemize}
\item
application for development of include files
\item
corrections to manual
\end{itemize}

%%%%%%%%%%%%%%%%%%%%%%%%%%%%%%%%%%%%%%%%
\paragraph{v1.5:} 2017/05/21

\begin{itemize}
\item
more complete structuring introduced
\item
|\childdocof| introduced
\item
|\childdoc| renamed to |\childdocmain|
\item
|\childredirect| renamed to |\childdocforward| and |\childdocforwardprefix|
and functionality expanded
\end{itemize}

%%%%%%%%%%%%%%%%%%%%%%%%%%%%%%%%%%%%%%%%
\paragraph{v1.0:} 2017/04/27

\begin{itemize}
\item
manual and install package
\item
first version published on CTAN
\end{itemize}

%%%%%%%%%%%%%%%%%%%%%%%%%%%%%%%%%%%%%%%%
\paragraph{v0.6:} 2017/04/26

\begin{itemize}
\item
redirection mechanism added
\end{itemize}

%%%%%%%%%%%%%%%%%%%%%%%%%%%%%%%%%%%%%%%%
\paragraph{v0.5:} 2017/04/26

\begin{itemize}
\item
functionality in definition file
\end{itemize}


%%%%%%%%%%%%%%%%%%%%%%%%%%%%%%%%%%%%%%%%%%%%%%%%%%%%%%%%%%%%%%%%%%%%%%%%%%%%%%%%
%%%%%%%%%%%%%%%%%%%%%%%%%%%%%%%%%%%%%%%%%%%%%%%%%%%%%%%%%%%%%%%%%%%%%%%%%%%%%%%%
%%%%%%%%%%%%%%%%%%%%%%%%%%%%%%%%%%%%%%%%%%%%%%%%%%%%%%%%%%%%%%%%%%%%%%%%%%%%%%%%
\appendix

\settowidth\MacroIndent{\rmfamily\scriptsize 000\ }

 \DocInput{childdoc.dtx}

\end{document}
%</driver>
% \fi
%
% %%%%%%%%%%%%%%%%%%%%%%%%%%%%%%%%%%%%%%%%%%%%%%%%%%%%%%%%%%%%%%%%%%%%%%%%%%%%%%
% %%%%%%%%%%%%%%%%%%%%%%%%%%%%%%%%%%%%%%%%%%%%%%%%%%%%%%%%%%%%%%%%%%%%%%%%%%%%%%
% \section{Sample}
%\iffalse
%<*samplemain>
%\fi
%
% The following presents a sample document
% with two chapters, two parts, a title page,
% a compile flag as well as three forwarding files to set the flag.
% It consists of eight |.tex| files:
% \begin{center}
% \begin{tabular}{ll}
% |cdocsamp.tex|&main file\\
% |cdocsch1.tex|&include file for chapter 1\\
% |cdocsch2.tex|&include file for chapter 2\\
% |cdocspt3.tex|&include file for part 3\\
% |cdocspt4.tex|&include file for part 4\\
% |cdocsdrf.tex|&forwarding file for main file in draft mode\\
% |cdocsfi1.tex|&forwarding file for final version of chapter 1\\
% |cdocsfi2.tex|&forwarding file for final version of chapter 2\\
% \end{tabular}
% \end{center}
% Each of the eight files can be compiled directly by the \LaTeX{} compiler.
%
% %%%%%%%%%%%%%%%%%%%%%%%%%%%%%%%%%%%%%%
% \paragraph{Main File.}
%
% The main file is called |cdocsamp.tex|.
%
% Load the \textsf{childdoc} definitions and
% declare the filename for the main document:
%    \begin{macrocode}
\input{childdoc.def}
\childdocmain{}
%    \end{macrocode}

% Optional override for |\version| flag:
%    \begin{macrocode}
%%\ifchilddoc\else\providecommand{\version}{draft}\fi
%    \end{macrocode}

% Define the default values for the |\version| flag
% (|final| for the main file and |draft| for childs):
%    \begin{macrocode}
\ifchilddoc
\providecommand{\version}{draft}
\else
\providecommand{\version}{final}
\fi
%    \end{macrocode}

% Load the standard document class:
%    \begin{macrocode}
\documentclass[12pt]{article}
%    \end{macrocode}

% Start the document body:
%    \begin{macrocode}
\begin{document}
%    \end{macrocode}

% Declare a title page.
% Print title, part of document being processed and version flag:
%    \begin{macrocode}
\addtocounter{page}{-1}
\begin{center}
{\LARGE\bfseries{}childdoc example\par}
\vspace{1cm}
\ifchilddoc
\ifchilddocmanual part\else chapter\fi:
`\childdocname' of `\childdocjob'\par
\else
main document: `\childdocjob'\par
\fi
version: \version\par
\end{center}
\newpage
%    \end{macrocode}

% Manually include selected file,
% otherwise process as usual:
%    \begin{macrocode}
\ifchilddocmanual
\section*{part `\childdocname'}
\input{\childdocname}
\else
%    \end{macrocode}

% Include the two chapters:
%    \begin{macrocode}
\include{cdocsch1}
\include{cdocsch2}
%    \end{macrocode}

% Include the two parts unless only chapters should be displayed:
%    \begin{macrocode}
\ifchilddoc\else
\section{part three}
\input{cdocspt3}
\section{part four}
\input{cdocspt4}
\fi
%    \end{macrocode}

% Process as usual until here:
%    \begin{macrocode}
\fi
%    \end{macrocode}

% End of document body:
%    \begin{macrocode}
\end{document}
%    \end{macrocode}
%\iffalse
%</samplemain>
%\fi
%
% %%%%%%%%%%%%%%%%%%%%%%%%%%%%%%%%%%%%%%
% \paragraph{Chapter Include Files.}
%
% The include files are called |cdocsch1.tex| and |cdocsch2.tex|.
%
%\iffalse
%<*samplechap1|samplechap2>
%\fi

% Optional override for |\version| flag:
%    \begin{macrocode}
%%\providecommand{\version}{final}
%    \end{macrocode}

% Include the main document:
%    \begin{macrocode}
\input{childdoc.def}
\childdocof{cdocsamp}
%    \end{macrocode}

%\iffalse
%</samplechap1|samplechap2>
%\fi
%
%\iffalse
%<*samplechap1>
%\fi
% Some text for chapter 1:
%    \begin{macrocode}
\section{one}
some text in chapter one
%    \end{macrocode}

%\iffalse
%</samplechap1>
%\fi
% Some text for chapter 2:
%\iffalse
%<*samplechap2>
%\fi
%    \begin{macrocode}
\section{two}
more text in chapter two
%    \end{macrocode}

%\iffalse
%</samplechap2>
%\fi
%
% %%%%%%%%%%%%%%%%%%%%%%%%%%%%%%%%%%%%%%
% \paragraph{Part Include Files.}
%
% The include files are called |cdocspt3.tex| and |cdocspt4.tex|.
%
%\iffalse
%<*samplepart3|samplepart4>
%\fi

% Optional override for |\version| flag:
%    \begin{macrocode}
%%\providecommand{\version}{final}
%    \end{macrocode}

% Include the main document:
%    \begin{macrocode}
\input{childdoc.def}
\childdocby{cdocsamp}
%    \end{macrocode}

%\iffalse
%</samplepart3|samplepart4>
%\fi
%
%\iffalse
%<*samplepart3>
%\fi
% Some text for part 3:
%    \begin{macrocode}
some text in part three
%    \end{macrocode}

%\iffalse
%</samplepart3>
%\fi
% Some text for part 4:
%\iffalse
%<*samplepart4>
%\fi
%    \begin{macrocode}
more text in part four
%    \end{macrocode}

%\iffalse
%</samplepart4>
%\fi
%
% %%%%%%%%%%%%%%%%%%%%%%%%%%%%%%%%%%%%%%
% \paragraph{Forwarding for a Complete Draft.}
%
% The following forwarding file |cdocsdrf.tex|
% compiles the main document in draft mode:
%\iffalse
%<*sampledraft>
%\fi
%    \begin{macrocode}
\def\version{draft}
\input{childdoc.def}
\childdocforward{cdocsamp}
%    \end{macrocode}

%\iffalse
%</sampledraft>
%\fi
%
% %%%%%%%%%%%%%%%%%%%%%%%%%%%%%%%%%%%%%%
% \paragraph{Forwarding for Final Version of the Chapters.}
%
% The following forwarding files |cdocsfn1.tex| and |cdocsfn2.tex|
% (with identical content)
% compile the final versions of the child documents
% |cdocsch1.tex| and |cdocsch2.tex|, respectively:
%\iffalse
%<*samplefinal>
%\fi
%    \begin{macrocode}
\def\version{final}
\input{childdoc.def}
\childdocforwardprefix[cdocsamp]{cdocsfn}{cdocsch}
%    \end{macrocode}

%\iffalse
%</samplefinal>
%\fi
%
% %%%%%%%%%%%%%%%%%%%%%%%%%%%%%%%%%%%%%%
% \paragraph{Command Line Processing.}
%
% The following three command lines generate the output files
% |cdocscld|, |cdocscl1| and |cdocscl2|
% which should be identical to
% |cdocsdrf|, |cdocsch1| and |cdocsfn2|, respectively:
% \begin{center}
% \begin{tabular}{l}
% |latex -jobname cdocscld \|\\
% |  "\def\version{draft}\input{childdoc.def}\childdocforward{cdocsamp}"|\\
% |latex -jobname cdocscl1 \|\\
% |  "\input{childdoc.def}\childdocforward[cdocsamp]{cdocsch1}"|\\
% |latex -jobname cdocscl2 \|\\
% |  "\def\version{final}\input{childdoc.def}\childdocforward{cdocsch2}"|
% \end{tabular}
% \end{center}
% Note that the trailing backslash on each first line
% merely continues the input to the second line
% (for convenient cut ant paste).
% Furthermore, the command |latex| can be replaced by any
% of its alternative versions such as |pdflatex|.
%
% %%%%%%%%%%%%%%%%%%%%%%%%%%%%%%%%%%%%%%%%%%%%%%%%%%%%%%%%%%%%%%%%%%%%%%%%%%%%%%
% %%%%%%%%%%%%%%%%%%%%%%%%%%%%%%%%%%%%%%%%%%%%%%%%%%%%%%%%%%%%%%%%%%%%%%%%%%%%%%
% \section{Implementation}
%\iffalse
%<*package>
%\fi
%
% This section describes the definitions file |childdoc.def|.

% The definitions cannot be loaded using |\usepackage| or |\RequirePackage|
% which has a mechanism to prevent loading a style file more than once.
% When loading the definitions by means of |\input|
% multiple instances have to be prevented manually:
%\iffalse
%This code needs to be before the `\ProvidesFile' directive
%which is defined at the beginning of this file.
%Therefore it is also placed there and commented out here.
%</package>
%<*discard>
%\fi
%    \begin{macrocode}
\ifdefined\childdocmain\endinput\fi
%    \end{macrocode}
%\iffalse
%</discard>
%<*package>
%\fi
%
% \macro{\ifchilddoc}
% \macro{\ifchilddocmanual}
% The conditional |\ifchilddoc| tells whether a
% child (true) or main (false) document is being compiled.
% The conditional |\ifchilddocmanual| tells whether
% the |\includeonly| mechanism is used (false) or
% the selection of child files must be performed manually (true).
% The definitions initialise to false:
%    \begin{macrocode}
\newif\ifchilddoc
\newif\ifchilddocmanual
%    \end{macrocode}

% \macro{\childdocname}
% \macro{\childdocjob}
% The macro |\childdocname| stores the name of the main document
% to be compiled. The macro |\childdocjob| stores the name of
% the document on which the \LaTeX{} compiler was originally invoked.
% The content of |\jobname| cannot be compared
% to filenames specified in the source due to different catcodes.
% The following code rescans |\jobname|, stores the result
% in |\childdocname| and saves a copy in |\childdocjob|:
%    \begin{macrocode}
\edef\childdocname{\scantokens\expandafter{\jobname\noexpand}}
\let\childdocjob\childdocname
%    \end{macrocode}

% \macro{\childdocdisable}
% The macro |\childdocdisable| prevents the main file
% from being processed more than once.
% At this stage, the main document command |\childdocmain|
% is assumed to be called once again where it should do nothing.
% Any subsequent call to it should prevent
% a secondary processing of the main document
% It overwrites the forwarding commands
% |\childdocof| and |\childdocforward|
% with empty macros to prevent further inclusions of the main document:
%    \begin{macrocode}
\newcommand{\childdocdisable}
{
  \renewcommand{\childdocmain}[1]{\renewcommand{\childdocmain}[1]{\endinput}}
  \renewcommand{\childdocof}[1]{}
  \renewcommand{\childdocby}[2][]{}
  \renewcommand{\childdocforward}[2][]{}
  \renewcommand{\childdocdisable}{}
}
%    \end{macrocode}

% \macro{\childdocmain}
% The macro |\childdocmain| is to be called at the top of the main file
% with nothing or the main filename (without extension) as argument.
% First, it breaks loops.
% If the argument is not empty and does not match |\childdocname|
% (which is set by the first inclusion of |childdoc.def|),
% |\ifchilddoc| is set to true, |\includeonly| is applied to the child file
% and |\jobname| is set to the main file
% (for proper handling of |.aux| files):
%    \begin{macrocode}
\newcommand{\childdocmain}[1]
{
  \childdocdisable\childdocmain{}
  \if?#1?\else
    \begingroup
      \def\childdoctmp{#1}
      \ifx\childdoctmp\childdocname
        \def\childdoctmp{}
      \else
        \def\childdoctmp
        {
          \childdoctrue
          \includeonly{\childdocname}
          \def\childdocjob{#1}
          \def\jobname{#1}
        }
      \fi
      \expandafter
    \endgroup
    \childdoctmp
  \fi
}
%    \end{macrocode}

% \macro{\childdocof}
% The command |\childdocof| redirects
% compilation to the main file |#1|.
%    \begin{macrocode}
\newcommand{\childdocof}[1]
{
  \childdocdisable
  \childdoctrue
  \includeonly{\childdocname}
  \def\jobname{#1}
  \def\childdocjob{#1}
  \input{#1}
}
%    \end{macrocode}

% \macro{\childdocby}
% The command |\childdocby| ....
%    \begin{macrocode}
\newcommand{\childdocby}[2][]
{
  \childdocdisable
  \childdoctrue
  \childdocmanualtrue
  \if?#1?\else
    \def\jobname{#2}
  \fi
  \def\childdocjob{#2}
  \input{#2}
  \endinput
}
%    \end{macrocode}

% \macro{\childdocforward}
% The command |\childdocforward| redirects
% compilation to the main file or
% (if the optional argument is given) a child file.
% Parameters are set as if the main file
% or a child file starting with |\childdocof| was compiled.
% Then compilation is handed over to the main file:
%    \begin{macrocode}
\newcommand{\childdocforward}[2][]
{
  \begingroup
    \if?#1?
      \def\childdoctmp
      {
        \def\childdocname{#2}
        \def\childdocjob{#2}
        \def\jobname{#2}
        \input{#2}
        \endinput
      }
    \else
      \def\childdoctmp
      {
        \childdocdisable
        \def\childdocname{#2}
        \childdoctrue
        \includeonly{#2}
        \def\childdocjob{#1}
        \def\jobname{#1}
        \input{#1}
        \endinput
      }
    \fi
    \expandafter
  \endgroup
  \childdoctmp
}
%    \end{macrocode}

% \macro{\childdocforwardprefix}
% The command |\childdocforwardprefix| redirects
% compilation to the main or a child file by means of a pattern.
% The prefix |#1| in the current filename is replaced by |#2|
% and the suffix of the current filename is kept
% (it is assumed that the filename does not contain the substring `|~~~|'
% which is used as a delimiter).
% Compilation is handed over to the new file by |\childdocforward|:
%    \begin{macrocode}
\newcommand{\childdocforwardprefix}[3][]
{
  \begingroup
    \def\childdocextract #2##1~~~{\def\childdoctmp{\childdocforward[#1]{#3##1}}}
    \expandafter\childdocextract\childdocname~~~
    \expandafter
  \endgroup
  \childdoctmp
}
%    \end{macrocode}

% \macro{\childdoc}
% The deprecated macro |\childdoc| is a legacy version of |\childdocmain|:
%    \begin{macrocode}
\newcommand{\childdoc}{\childdocmain}
%    \end{macrocode}

% \macro{\childdocredirect}
% The deprecated macro |\childdocredirect| is a legacy version
% of |\childdocforward| and |\childdocforwardprefix|:
%    \begin{macrocode}
\newcommand{\childdocredirect}[2][]
{
  \begingroup
    \if?#1?
      \def\childdoctmp{\childdocforward{#2}}
    \else
      \def\childdoctmp{\childdocforwardprefix{#1}{#2}}
    \fi
    \expandafter
  \endgroup
  \childdoctmp
}
%    \end{macrocode}

%\iffalse
%</package>
%\fi
%
\endinput

\childdocby{cdocsamp}
%    \end{macrocode}

%\iffalse
%</samplepart3|samplepart4>
%\fi
%
%\iffalse
%<*samplepart3>
%\fi
% Some text for part 3:
%    \begin{macrocode}
some text in part three
%    \end{macrocode}

%\iffalse
%</samplepart3>
%\fi
% Some text for part 4:
%\iffalse
%<*samplepart4>
%\fi
%    \begin{macrocode}
more text in part four
%    \end{macrocode}

%\iffalse
%</samplepart4>
%\fi
%
% %%%%%%%%%%%%%%%%%%%%%%%%%%%%%%%%%%%%%%
% \paragraph{Forwarding for a Complete Draft.}
%
% The following forwarding file |cdocsdrf.tex|
% compiles the main document in draft mode:
%\iffalse
%<*sampledraft>
%\fi
%    \begin{macrocode}
\def\version{draft}
% \iffalse
%
% childdoc.dtx Copyright (C) 2017-2018 Niklas Beisert
%
% This work may be distributed and/or modified under the
% conditions of the LaTeX Project Public License, either version 1.3
% of this license or (at your option) any later version.
% The latest version of this license is in
%   http://www.latex-project.org/lppl.txt
% and version 1.3 or later is part of all distributions of LaTeX
% version 2005/12/01 or later.
%
% This work has the LPPL maintenance status `maintained'.
%
% The Current Maintainer of this work is Niklas Beisert.
%
% This work consists of the files childdoc.dtx and childdoc.ins
% and the derived files childdoc.def and cdocsamp.tex with
% cdocsch1.tex, cdocsch2.tex, cdocsdrf.tex, cdocsfn1.tex, cdocsfn2.tex.
%
%<package>\ifdefined\childdocmain\endinput\fi
%<package>\ProvidesFile{childdoc.def}[2018/12/30 v2.0 child document driver]
%<samplemain>\ProvidesFile{cdocsamp.tex}[2018/12/30 v2.0 sample for childdoc]
%<*driver>
%\ProvidesFile{childdoc.drv}[2018/12/30 v2.0 childdoc reference manual file]
\PassOptionsToClass{10pt,a4paper}{article}
\documentclass{ltxdoc}

\usepackage[margin=35mm]{geometry}
\usepackage{hyperref}
\usepackage{hyperxmp}
\usepackage[usenames]{color}

\hypersetup{colorlinks=true}
\hypersetup{pdfstartview=FitH}
\hypersetup{pdfpagemode=UseNone}
\hypersetup{pdfsource={}}
\hypersetup{pdflang={en-UK}}
\hypersetup{pdfcopyright={Copyright 2017-2018 Niklas Beisert.
  This work may be distributed and/or modified under the
  conditions of the LaTeX Project Public License, either version 1.3
  of this license or (at your option) any later version.}}
\hypersetup{pdflicenseurl={http://www.latex-project.org/lppl.txt}}
\hypersetup{pdfcontactaddress={ETH Zurich, ITP, HIT K,
  Wolfgang-Pauli-Strasse 27}}
\hypersetup{pdfcontactpostcode={8093}}
\hypersetup{pdfcontactcity={Zurich}}
\hypersetup{pdfcontactcountry={Switzerland}}
\hypersetup{pdfcontactemail={nbeisert@itp.phys.ethz.ch}}
\hypersetup{pdfcontacturl={http://people.phys.ethz.ch/\xmptilde nbeisert/}}

\newcommand{\secref}[1]{\hyperref[#1]{section \ref*{#1}}}

\parskip1ex
\parindent0pt
\let\olditemize\itemize
\def\itemize{\olditemize\parskip0pt}

\begin{document}

\title{The \textsf{childdoc} Package}
\hypersetup{pdftitle={The childdoc Package}}
\author{Niklas Beisert\\[2ex]
  Institut f\"ur Theoretische Physik\\
  Eidgen\"ossische Technische Hochschule Z\"urich\\
  Wolfgang-Pauli-Strasse 27, 8093 Z\"urich, Switzerland\\[1ex]
  \href{mailto:nbeisert@itp.phys.ethz.ch}
  {\texttt{nbeisert@itp.phys.ethz.ch}}}
\hypersetup{pdfauthor={Niklas Beisert}}
\hypersetup{pdfsubject={Manual for the LaTeX2e Package childdoc}}
\date{30 December 2018, \textsf{v2.0}}
\maketitle

\begin{abstract}\noindent
\textsf{childdoc} is a \LaTeXe{} package
that enables the direct compilation
of document sections included by |\include|
to individual files.
\end{abstract}

\begingroup
\parskip0ex
\tableofcontents
\endgroup

%%%%%%%%%%%%%%%%%%%%%%%%%%%%%%%%%%%%%%%%%%%%%%%%%%%%%%%%%%%%%%%%%%%%%%%%%%%%%%%%
%%%%%%%%%%%%%%%%%%%%%%%%%%%%%%%%%%%%%%%%%%%%%%%%%%%%%%%%%%%%%%%%%%%%%%%%%%%%%%%%
\section{Introduction}

\LaTeX{} provides a mechanism to structure a large document (such as a book)
into a main file and several child files (containing the chapters)
using the |\include| command.
This mechanism is beneficial for documents
which span hundreds of pages in order to
make the source file(s) more manageable.
Moreover, compilation can be restricted to
selected child files by means of the |\includeonly| command.
The latter feature can be used to reduce the compilation time while editing
(this was significantly more useful in the earlier days of \LaTeX{})
or to generate a smaller document which is easier to navigate.
Another application of |\includeonly| is to generate
documents consisting of selected parts of the complete document.

However, there are a few drawbacks of the plain |\include| mechanism:
\begin{itemize}
\item
The child files cannot be compiled on their own,
they can only be compiled via the main file.
A naive editing environment
(such as a text editor with an option
to have the current file processed by \LaTeX)
may require one to switch to the main file before compiling;
attempting to compile the child file produces errors.
\item
The main file must be modified (each time)
to adjust the |\includeonly| command
to the present needs. This easily leaves the main file in a messy state.
\item
The generated document will always carry the filename
of the main document. This is inconvenient if
several child files are to be compiled and
to be kept for distribution.
\end{itemize}

The present package provides a simple interface
to make child files individually compilable by \LaTeX{}.
Compiling a child file then has the same effect as compiling
the main file with an |\includeonly| command
to select the appropriate child.
Moreover the generated document will carry the name of the child
rather than the main file.
This resolves all three above issues.

This feature is meant to make the editing of books,
thesis documents and lecture notes somewhat more convenient.
However, the package can also be used efficiently for
composing a series of documents (such as exercise sheets)
which are typically distributed individually.
It then assists the author in generating the individual documents
(potentially in different versions)
as well as a document containing the collected series.
Another application is in developing style files
or other kinds of included material
where compilation of the style file could redirect
to a sample or test file.

%%%%%%%%%%%%%%%%%%%%%%%%%%%%%%%%%%%%%%%%%%%%%%%%%%%%%%%%%%%%%%%%%%%%%%%%%%%%%%%%
%%%%%%%%%%%%%%%%%%%%%%%%%%%%%%%%%%%%%%%%%%%%%%%%%%%%%%%%%%%%%%%%%%%%%%%%%%%%%%%%
\section{Usage}

First of all, the package \textsf{childdoc} is \emph{not} a standard
\LaTeXe{} |.sty| style file! Therefore it needs to be invoked in
a non-standard way.

%%%%%%%%%%%%%%%%%%%%%%%%%%%%%%%%%%%%%%%%%%%%%%%%%%%%%%%%%%%%%%%%%%%%%%%%%%%%%%%%
\subsection{Included Files}
\label{sec:include}

%%%%%%%%%%%%%%%%%%%%%%%%%%%%%%%%%%%%%%%%
\DescribeMacro{\childdocmain}
To use the package, add the commands
\begin{center}
\begin{tabular}{l}
|\input{childdoc.def}|\\
|\childdocmain{}|\\
\end{tabular}
\end{center}
at the very top of the main \LaTeX{} file,
in particular \emph{before} the |\documentclass| statement!
The argument of |\childdocmain| should be left empty
(but it must be present).

%%%%%%%%%%%%%%%%%%%%%%%%%%%%%%%%%%%%%%%%
\DescribeMacro{\childdocof}
Furthermore, add the commands
\begin{center}
\begin{tabular}{l}
|\input{childdoc.def}|\\
|\childdocof{|\textit{main}|}|\\
\end{tabular}
\end{center}
at the top of every child file \textit{child}
which is included by |\include{|\textit{child}|}|
from within the main file
(or at least for those files to be compiled individually).
The argument \textit{main} must be the filename of the main file.

There are a couple of
considerations in setting up the main and child documents:

%%%%%%%%%%%%%%%%%%%%%%%%%%%%%%%%%%%%%%%%
\paragraph{Restrictions.}

Please note the following restrictions:
\begin{itemize}
\item
|\childdocmain| must be called with one argument \textit{main}
to ensure compatibility with earlier version of the package.
It must either be empty (|\childdocmain{}|)
or precisely match the filename of the main file in which it is specified.
See \secref{sec:detection} for further information.
\item
The filename \textit{main} must be specified without the |.tex| extension.
\item
The filename \textit{main} is case sensitive
(even in case-insensitive file systems)
due to internal string comparison.
\item
The argument \textit{main} should be fully expanded, it cannot be a macro.
\item
Subdirectories and special characters should be avoided in filenames.
\item
The command |\childdocmain{|\textit{main}|}| must be followed by a whitespace.
It should not be followed immediately by another command
or by a comment mark `|%|'.
This is because the \TeX{} parser reads the token immediately following
the argument of |\childdocmain| and puts it
at the beginning of every child section;
however, a white\-space is ignored.
\end{itemize}

%%%%%%%%%%%%%%%%%%%%%%%%%%%%%%%%%%%%%%%%
\paragraph{Content of Main File.}

It is advisable to place all content in the child files included by |\include|.
Any output contained in the main file will appear in all child documents
unless suppressed manually;
it cannot be suppressed automatically by the |\includeonly| directive
and thus should normally be avoided.
A method to include some content in the main file
by means of conditional processing is described in \secref{sec:conditional}.

%%%%%%%%%%%%%%%%%%%%%%%%%%%%%%%%%%%%%%%%
\paragraph{Page Numbering.}

When only a part of the document is compiled,
the appropriate numbering of pages
(as well as other status parameters)
is determined from the |.aux| files.
The latter contain information from previous passes.
However this information needs to propagate through
all intermediate child documents.
Therefore the page numbering in child documents may well
be inconsistent until the complete document is compiled at least once.

A useful (if unconventional) way to always ensure a consistent
page numbering is to restart the numbering in each child document
and denote the pages by `\textit{child}|.|\textit{page}'
where \textit{child} represents the chapter/section number of the child file.
This can be achieved by the command
|\numberwithin{page}{|\textit{child}|}|
of the \textsf{amsmath} package
where \textit{child} can be |chapter| or |section|
depending on the chosen structuring.
Alternatively, one can modify the macro |\thepage| appropriately
and reset the counter |page| at the start of each child file.

%%%%%%%%%%%%%%%%%%%%%%%%%%%%%%%%%%%%%%%%%%%%%%%%%%%%%%%%%%%%%%%%%%%%%%%%%%%%%%%%
\subsection{Conditional Processing}
\label{sec:conditional}

The package provides a mechanism to compile different versions
of a document. To customise the versions further some conditional processing
can come in handy to distinguish which version is being compiled.
The package provides two macros to describe the compilation context:

%%%%%%%%%%%%%%%%%%%%%%%%%%%%%%%%%%%%%%%%
\DescribeMacro{\ifchilddoc}
The conditional |\ifchilddoc| distinguishes between the compilation of
child documents and the main document:
%
\begin{center}
|\ifchilddoc |\textit{child-code}| |[|\||else |\textit{main-code}]| \||fi|
\end{center}

%%%%%%%%%%%%%%%%%%%%%%%%%%%%%%%%%%%%%%%%
\DescribeMacro{\childdocname}
\DescribeMacro{\childdocjob}
The macro |\childdocname| contains the filename (without extension)
of the main or child file being processed.
Note that |\childdocjob| will always contain the name of the main file.

%%%%%%%%%%%%%%%%%%%%%%%%%%%%%%%%%%%%%%%%
\paragraph{Title Page.}

Conditional processing can be used to include a title or banner page
in the main document when proper precautions are taken.
Importantly, the code in the main file should ensure that the page counter
(as well as other status parameters which are stored in the |.aux| files)
takes the same value after the conditional processing.
Otherwise the page numbers may take divergent values
depending on which part is compiled.

For example, a title page could be declared by:
%
\begin{center}
\begin{tabular}{l}
|\ifchilddoc\||else|\\
|\addtocounter{page}{-1}|\\
\textit{code for title page}\\
|\newpage|\\
|\||fi|
\end{tabular}
\end{center}
%
A banner page for the child documents can be generated by:
%
\begin{center}
\begin{tabular}{l}
|\ifchilddoc|\\
|\addtocounter{page}{-1}|\\
\textit{code for banner page}\\
|\newpage|\\
|\||fi|
\end{tabular}
\end{center}
%
Here one could write a message such as:
\begin{center}
|This is the part \childdocname{} of \childdocjob{}.|
\end{center}

%%%%%%%%%%%%%%%%%%%%%%%%%%%%%%%%%%%%%%%%%%%%%%%%%%%%%%%%%%%%%%%%%%%%%%%%%%%%%%%%
\subsection{Flags}
\label{sec:flags}

The package makes it easy to generate different versions
of the main or child documents.
To this end compilation flags can be defined
and assigned different default values.
They will be particularly useful in conjunction
with the forwarding mechanism described in \secref{sec:forward}.

For example, it may be useful to have a flag |\version|
which can be set to |draft| or |final|.
The document source will contain some conditional code
depending on the value of |\version|.
Suppose further, the flag should default to |final| for the main file
and to |draft| for child files
which is a natural assignment for editing the document.
This is achieved by placing the following code
in the preamble of the main document
(below the |\childdocmain| directive):
%
\begin{center}
\begin{tabular}{l}
|\ifchilddoc|\\
|\providecommand{\version}{draft}|\\
|\||else|\\
|\providecommand{\version}{final}|\\
|\||fi|
\end{tabular}
\end{center}
%
The definition by |\providecommand| makes sure
that previous definitions are not overwritten.
Further statements |\providecommand{\version}{...}|
can thus be added before the above code to override it.

For the main file, one might add a line
(between |\childdocmain| and the above block)
%
\begin{center}
|%\ifchilddoc\||else\providecommand{\version}{draft}\||fi|
\end{center}
%
which can be uncommented to produce a draft version.
Likewise one can add a line to the very top of a child file
(above the |\childdocof{|\textit{main}|}| directive)
%
\begin{center}
|%\providecommand{\version}{final}|
\end{center}
%
which can be uncommented to produce the final version of this child document.

%%%%%%%%%%%%%%%%%%%%%%%%%%%%%%%%%%%%%%%%%%%%%%%%%%%%%%%%%%%%%%%%%%%%%%%%%%%%%%%%
\subsection{Forwarding}
\label{sec:forward}

Different versions of the main or child documents
using compilation flags as described in \secref{sec:flags}
can be (permanently) stored in different files
for convenient compilation, viewing and distribution.
To this end, the package defines a command
to pass on compilation to a different file:

%%%%%%%%%%%%%%%%%%%%%%%%%%%%%%%%%%%%%%%%
\DescribeMacro{\childdocforward}
The command |\childdocforward| redirects processing to
another source file:
%
\begin{center}
\begin{tabular}{l}
|\input{childdoc.def}|\\
|\childdocforward[|\textit{main}|]{|\textit{dest}|}|\\
\end{tabular}
\end{center}
%
The argument \textit{dest} is the destination file
(without extension).
It should be the main file or one of the child files.
Note that further \textsf{childdoc} directives
such as |\childdocof| and |\childdocforward|
in the indicated file will be processed in this form.
The optional argument \textit{main}
passes on directly to the main file \textit{main}
while pretending to compile the child \textit{dest}.
This form behaves as if \textit{dest}
issues |\childdocof{|\textit{main}|}| right away,
and no further \textsf{childdoc} directives will be processed.

%%%%%%%%%%%%%%%%%%%%%%%%%%%%%%%%%%%%%%%%
\DescribeMacro{\...prefix}
In the alternative form |\childdocforwardprefix|,
%
\begin{center}
\begin{tabular}{l}
|\input{childdoc.def}|\\
|\childdocforwardprefix[|\textit{main}|]{|\textit{prefix}|}{|\textit{dest}|}|
\end{tabular}
\end{center}
%
the destination file is determined by a pattern
depending on the current file:
To make this work, the current file must be called
`{\textit{prefix}\hspace{0.2em}\textit{suffix}}'
with \textit{prefix} matching precisely the argument.
Processing is then passed on to the file
`{\textit{dest}\hspace{0.2em}\textit{suffix}}'.
Surely, the same effect is achieved by
directly specifying the
argument `{\textit{dest}\hspace{0.2em}\textit{suffix}}'
in the first form.
However, that requires to set up a different file
for each child. With the alternative form of the command
all these files can have exactly the same content
which simplifies setting them up and maintaining them.

For example, the following file |draft.tex|
with a compilation flag |\version| as described in \secref{sec:flags}
compiles the main document as a draft:
%
\begin{center}
\begin{tabular}{l}
|\def\version{draft}|\\
|\input{childdoc.def}|\\
|\childdocforward{|\textit{main}|}|
\end{tabular}
\end{center}
%
Likewise, the following files |final|\textit{nn}|.tex|
compile the final version of the child document
|child|\textit{nn}|.tex|:
%
\begin{center}
\begin{tabular}{l}
|\def\version{final}|\\
|\input{childdoc.def}|\\
|\childdocforwardprefix{final}{child}|
\end{tabular}
\end{center}
%

Note that when several versions of a main file and/or of each child file
are to be generated, it may be convenient to set up a |Makefile| or
shell script to automatise the process.

%%%%%%%%%%%%%%%%%%%%%%%%%%%%%%%%%%%%%%%%%%%%%%%%%%%%%%%%%%%%%%%%%%%%%%%%%%%%%%%%
\subsection{Command Line Processing}
\label{sec:commandline}

The effect of redirection files can also be achieved by invoking
the \LaTeX{} compiler with a more elaborate command line.
Most conveniently this should be done as part
of a shell script or a |Makefile|.

When using \textsf{childdoc} in the main file, the following
command lines effectively perform a redirection
(note that depending on the shell being used,
backslashes may have to be doubled: `|\|' $\to$ `|\\|'):
%
\begin{center}
|... -jobname "|\textit{target}|" |\\|"|[\textit{flags}]%
|\input{childdoc.def}\childdocforward[|\textit{main}|]{|\textit{dest}|}"|
\end{center}
%
Here \textit{target} is the name of the output file,
\textit{main} is the name of the main file
and \textit{dest} is the name of the main or child file to be processed
(all filenames without extensions).
The optional argument \textit{main} can be omitted
if \textit{main} matches \textit{dest}.
Optionally, compilation \textit{flags} can be defined via |\def| commands.
This command line makes the \TeX{} engine believe
it is compiling the file \textit{target}
whose content is specified as the latter parameter.
The provided code then forwards the processing to
\textit{main} or \textit{dest} as described in \secref{sec:forward}.

%%%%%%%%%%%%%%%%%%%%%%%%%%%%%%%%%%%%%%%%%%%%%%%%%%%%%%%%%%%%%%%%%%%%%%%%%%%%%%%%
\subsection{Include by Input}
\label{sec:input}

Including child documents by |\include| has some restrictions by design.
Most notably, the content of a child document always occupies
its own set of pages; pages cannot be shared between child documents.
Usually, this behaviour makes perfect sense
because each child document contain an essential part of the document.
However, in some situations it may be desirable to compose
a document from a collection of parts
without having mandatory page breaks between then.
For this case, the package
provides a mechanism to include parts
by |\input| which can also be processed individually.
However, by construction this mechanism
requires manual handling of the content to be output.

%%%%%%%%%%%%%%%%%%%%%%%%%%%%%%%%%%%%%%%%
\DescribeMacro{\ifchilddocmanual}
The main file should be prepared as usual, see \secref{sec:include}.
However, the document body must make a distinction
between processing of an individual part and of the main document, e.g.:
%
\begin{center}
\begin{tabular}{l}
|\ifchilddocmanual|\\
|\input{\childdocname}|\\
|\||else|\\
\textit{document body with }|\input{|\textit{part}|}|\\
|\||fi|
\end{tabular}
\end{center}
%
The conditional |\ifchilddocmanual| is true whenever
a part to be included by |\input| is being compiled,
and the name of the part is stored in |\childdocname|.

%%%%%%%%%%%%%%%%%%%%%%%%%%%%%%%%%%%%%%%%
\DescribeMacro{\childdocby}
Each part to be included by |\input| should start with:
%
\begin{center}
\begin{tabular}{l}
|\input{childdoc.def}|\\
|\childdocby{|\textit{main}|}|\\
\end{tabular}
\end{center}
%
The directive |\childdocby| is similar to |\childdocof|
described in \secref{sec:include},
but the subsequent selection of content must be done manually.
To that end, both |\ifchilddoc| and |\ifchilddocmanual|
will be true upon processing of a part,
and the name of the part is stored in |\childdocname|.
Note that |\jobname| will be set to the filename of the current part
so that each part receives an individual |.aux| file
that does not interfere with the |.aux| file(s) of the main document.
This behaviour can be altered by the alternative form
|\childdocby[*]{|\textit{main}|}| (with a non-empty optional argument)
which uses the |.aux| file of the main document
by setting |\jobname| to \textit{main}.

%%%%%%%%%%%%%%%%%%%%%%%%%%%%%%%%%%%%%%%%%%%%%%%%%%%%%%%%%%%%%%%%%%%%%%%%%%%%%%%%
\subsection{Driver Development}
\label{sec:driver}

The \textsf{childdoc} mechanism can also be use for the development
of definition files such as \LaTeX{} styles or classes.
This case differs from the above setup with multiple parts
included by |\include| in that no |\includeonly| should be invoked.
This can be achieved by starting the include file
(before |\ProvidesPackage|) with:
%
\begin{center}
\begin{tabular}{l}
|\input{childdoc.def}|\\
|\childdocforward{|\textit{main}|}|\\
\end{tabular}
\end{center}
%
or alternatively with:
%
\begin{center}
\begin{tabular}{l}
|\input{childdoc.def}|\\
|\childdocby{|\textit{main}|}|\\
\end{tabular}
\end{center}
%
Both forms have slightly different effects as described above.
The main file is prepared as usual, see \secref{sec:include}.

%%%%%%%%%%%%%%%%%%%%%%%%%%%%%%%%%%%%%%%%%%%%%%%%%%%%%%%%%%%%%%%%%%%%%%%%%%%%%%%%
\subsection{Legacy Detection}
\label{sec:detection}

The directive |\childdocmain| in the main file can detect
whether the complete document or merely a child is to be compiled
even without using the directive |\childdocof|.
This method is deprecated because it is less robust
and there is no compelling reason to use it;
it is merely provided for backward compatibility
and it may be removed in future versions.

If the detection mechanism is to be used,
it is mandatory to correctly specify
the filename of the main file as the argument of |\childdocmain|:
%
\begin{center}
\begin{tabular}{l}
|\input{childdoc.def}|\\
|\childdocmain{|\textit{main}|}|\\
\end{tabular}
\end{center}
%
If |\jobname| does not match the argument \textit{main} of |\childdocmain|,
it is assumed that |\jobname| points to the child file to be compiled.
When using |\childdocmain| with the main file specified as argument,
it suffices to start a child file
with just |\input{|\textit{main}|}|
without loading of the package and using |\childdocof|.
If instead all processing is done
with the appropriate \textsf{childdoc} directives,
the argument of \textit{main} of |\childdocmain| can be empty.

An alternative version of the command line processing described
in \secref{sec:commandline} using the detection mechanism reads:
%
\begin{center}
|... -jobname "|\textit{target}|" "|[\textit{flags}]%
[|\def\jobname{|\textit{dest}|}|]|\input{|\textit{main}|}"|
\end{center}

%%%%%%%%%%%%%%%%%%%%%%%%%%%%%%%%%%%%%%%%%%%%%%%%%%%%%%%%%%%%%%%%%%%%%%%%%%%%%%%%
\subsection{Manual Code}
\label{sec:manual}

In case one cannot be certain whether the definitions file |childdoc.def|
is installed on the target \TeX{} distribution
and one prefers not to ship it,
it is conceivable to paste a few relevant commands into the sources.

To that end, drop all statements |\input{childdoc.def}|
and perform the replacements as outlined below.
Instead of |\childdocmain{|\textit{main}|}| add the following code
to the top of the main file:
%
\begin{center}
\begin{tabular}{l}
|\||ifdefined\childdocname\endinput\||fi\newif\ifchilddoc|\\
|\edef\childdocname{\scantokens\expandafter{\jobname\noexpand}}|\\
|\def\childdocmain{|\textit{main}|}\||ifx\childdocmain\childdocname\||else|\\
|\childdoctrue\includeonly{\childdocname}\let\jobname\childdocmain\||fi|\\
\end{tabular}
\end{center}
%
Instead of |\childdocof{|\textit{main}|}| just include the main file
at the top of each child file:
%
\begin{center}
|\input{|\textit{main}|}|
\end{center}
%
A simple redirection |\childdocforward{|\textit{dest}|}| is achieved by:
%
\begin{center}
|\def\jobname{|\textit{dest}|}\input{\jobname}|
\end{center}
%
The redirection with prefix
|\childdocforwardprefix[|\textit{prefix}|]{|\textit{dest}|}|
is accomplished by:
%
\begin{center}
\begin{tabular}{l}
|{\edef\jobname{\scantokens\expandafter{\jobname\noexpand}}|\\
|\def\redirectjob |\textit{prefix}|#1~~~{\gdef\jobname{|\textit{dest}|#1}}|\\
|\expandafter\redirectjob\jobname~~~}\input{\jobname}|
\end{tabular}
\end{center}

In an alternative approach,
child documents can be compiled by a specific command line
without additional code or specific definitions:
%
\begin{center}
|... -jobname "|\textit{target}|" "|[\textit{flags}]%
|\includeonly{|\textit{dest}|}\input{|\textit{main}|}"|
\end{center}
%

%%%%%%%%%%%%%%%%%%%%%%%%%%%%%%%%%%%%%%%%%%%%%%%%%%%%%%%%%%%%%%%%%%%%%%%%%%%%%%%%
%%%%%%%%%%%%%%%%%%%%%%%%%%%%%%%%%%%%%%%%%%%%%%%%%%%%%%%%%%%%%%%%%%%%%%%%%%%%%%%%
\section{Information}

%%%%%%%%%%%%%%%%%%%%%%%%%%%%%%%%%%%%%%%%%%%%%%%%%%%%%%%%%%%%%%%%%%%%%%%%%%%%%%%%
\subsection{Copyright}

Copyright \copyright{} 2017--2018 Niklas Beisert

This work may be distributed and/or modified under the
conditions of the \LaTeX{} Project Public License, either version 1.3
of this license or (at your option) any later version.
The latest version of this license is in
  \url{http://www.latex-project.org/lppl.txt}
and version 1.3 or later is part of all distributions of \LaTeX{}
version 2005/12/01 or later.

This work has the LPPL maintenance status `maintained'.

The Current Maintainer of this work is Niklas Beisert.

This work consists of the files |README.txt|, |childdoc.ins| and |childdoc.dtx|
as well as the derived files |childdoc.def|, |cdocsamp.tex|
with |cdocsch1.tex|, |cdocsch2.tex|, |cdocspt3.tex|, |cdocspt4.tex|,
|cdocsdrf.tex|, |cdocsfn1.tex|, |cdocsfn2.tex|
as well as |childdoc.pdf|.

%%%%%%%%%%%%%%%%%%%%%%%%%%%%%%%%%%%%%%%%%%%%%%%%%%%%%%%%%%%%%%%%%%%%%%%%%%%%%%%%
\subsection{Files and Installation}

The package consists of the files:
%
\begin{center}
\begin{tabular}{ll}
    |README.txt|   & readme file \\
    |childdoc.ins| & installation file \\
    |childdoc.dtx| & source file \\
    |childdoc.def| & definition file \\
    |cdocsamp.tex| & sample main file \\
    |cdocsch1.tex| & sample include file \\
    |cdocsch2.tex| & sample include file \\
    |cdocspt3.tex| & sample part file \\
    |cdocspt4.tex| & sample part file \\
    |cdocsdrf.tex| & sample redirection file \\
    |cdocsfn1.tex| & sample redirection file \\
    |cdocsfn2.tex| & sample redirection file \\
    |childdoc.pdf| & manual
\end{tabular}
\end{center}
%
The distribution consists of the files
|README.txt|, |childdoc.ins| and |childdoc.dtx|.
%
\begin{itemize}
\item
Run (pdf)\LaTeX{} on |childdoc.dtx|
to compile the manual |childdoc.pdf| (this file).
\item
Run \LaTeX{} on |childdoc.ins| to create the definitions file |childdoc.def|
and the sample |cdocsamp.tex| with include files
|cdocsch1.tex|, |cdocsch2.tex|, |cdocspt3.tex|, |cdocspt4.tex|,
|cdocsdrf.tex|, |cdocsfn1.tex|, |cdocsfn2.tex|.
Then copy the file |childdoc.def| to an appropriate directory of your \LaTeX{}
distribution, e.g.\ \textit{texmf-root}|/tex/latex/childdoc|.
\end{itemize}

%%%%%%%%%%%%%%%%%%%%%%%%%%%%%%%%%%%%%%%%%%%%%%%%%%%%%%%%%%%%%%%%%%%%%%%%%%%%%%%%
\subsection{Related CTAN Packages}

There are several other packages which offer a similar functionality:
%
\begin{itemize}
\item
The packages
\href{http://ctan.org/pkg/docmute}{\textsf{docmute}},
\href{http://ctan.org/pkg/includex}{\textsf{includex}} and
\href{http://ctan.org/pkg/standalone}{\textsf{standalone}}
provide commands to include only the document body of
a child file thus allowing both files to be compiled individually.
\item
The packages \href{http://ctan.org/pkg/subdocs}{\textsf{subdocs}}
and \href{http://ctan.org/pkg/subfiles}{\textsf{subfiles}}
provide structures in which the main and child documents can be
encapsulated and allowing them to be compiled individually.
The inclusion mechanism is different from the conventional |\include|.
\item
The package \href{http://ctan.org/pkg/combine}{\textsf{combine}}
is an elaborate solution to combine several documents into one.
\end{itemize}
%
See also the CTAN topic \href{http://ctan.org/topic/subdocs}{\textsf{subdocs}}
for further related packages.
The present package differs from the above solutions in that
a document structure constructed with the conventional |\include| mechanism
just needs two extra commands at the top of every file
such that all constituent files can be compiled individually.

%%%%%%%%%%%%%%%%%%%%%%%%%%%%%%%%%%%%%%%%%%%%%%%%%%%%%%%%%%%%%%%%%%%%%%%%%%%%%%%%
%\subsection{Feature Suggestions}
%
%The following is a list of features which may be useful for future
%versions of this package:
%%
%\begin{itemize}
%\item
%\ldots
%\end{itemize}

%%%%%%%%%%%%%%%%%%%%%%%%%%%%%%%%%%%%%%%%%%%%%%%%%%%%%%%%%%%%%%%%%%%%%%%%%%%%%%%%
\subsection{Revision History}

%%%%%%%%%%%%%%%%%%%%%%%%%%%%%%%%%%%%%%%%
\paragraph{v2.0:} 2018/12/30

\begin{itemize}
\item
immediate forward processing
\item
added |\childdocby| mechanism
\item
manual restructured
\end{itemize}

%%%%%%%%%%%%%%%%%%%%%%%%%%%%%%%%%%%%%%%%
\paragraph{v1.6:} 2018/01/17

\begin{itemize}
\item
application for development of include files
\item
corrections to manual
\end{itemize}

%%%%%%%%%%%%%%%%%%%%%%%%%%%%%%%%%%%%%%%%
\paragraph{v1.5:} 2017/05/21

\begin{itemize}
\item
more complete structuring introduced
\item
|\childdocof| introduced
\item
|\childdoc| renamed to |\childdocmain|
\item
|\childredirect| renamed to |\childdocforward| and |\childdocforwardprefix|
and functionality expanded
\end{itemize}

%%%%%%%%%%%%%%%%%%%%%%%%%%%%%%%%%%%%%%%%
\paragraph{v1.0:} 2017/04/27

\begin{itemize}
\item
manual and install package
\item
first version published on CTAN
\end{itemize}

%%%%%%%%%%%%%%%%%%%%%%%%%%%%%%%%%%%%%%%%
\paragraph{v0.6:} 2017/04/26

\begin{itemize}
\item
redirection mechanism added
\end{itemize}

%%%%%%%%%%%%%%%%%%%%%%%%%%%%%%%%%%%%%%%%
\paragraph{v0.5:} 2017/04/26

\begin{itemize}
\item
functionality in definition file
\end{itemize}


%%%%%%%%%%%%%%%%%%%%%%%%%%%%%%%%%%%%%%%%%%%%%%%%%%%%%%%%%%%%%%%%%%%%%%%%%%%%%%%%
%%%%%%%%%%%%%%%%%%%%%%%%%%%%%%%%%%%%%%%%%%%%%%%%%%%%%%%%%%%%%%%%%%%%%%%%%%%%%%%%
%%%%%%%%%%%%%%%%%%%%%%%%%%%%%%%%%%%%%%%%%%%%%%%%%%%%%%%%%%%%%%%%%%%%%%%%%%%%%%%%
\appendix

\settowidth\MacroIndent{\rmfamily\scriptsize 000\ }

 \DocInput{childdoc.dtx}

\end{document}
%</driver>
% \fi
%
% %%%%%%%%%%%%%%%%%%%%%%%%%%%%%%%%%%%%%%%%%%%%%%%%%%%%%%%%%%%%%%%%%%%%%%%%%%%%%%
% %%%%%%%%%%%%%%%%%%%%%%%%%%%%%%%%%%%%%%%%%%%%%%%%%%%%%%%%%%%%%%%%%%%%%%%%%%%%%%
% \section{Sample}
%\iffalse
%<*samplemain>
%\fi
%
% The following presents a sample document
% with two chapters, two parts, a title page,
% a compile flag as well as three forwarding files to set the flag.
% It consists of eight |.tex| files:
% \begin{center}
% \begin{tabular}{ll}
% |cdocsamp.tex|&main file\\
% |cdocsch1.tex|&include file for chapter 1\\
% |cdocsch2.tex|&include file for chapter 2\\
% |cdocspt3.tex|&include file for part 3\\
% |cdocspt4.tex|&include file for part 4\\
% |cdocsdrf.tex|&forwarding file for main file in draft mode\\
% |cdocsfi1.tex|&forwarding file for final version of chapter 1\\
% |cdocsfi2.tex|&forwarding file for final version of chapter 2\\
% \end{tabular}
% \end{center}
% Each of the eight files can be compiled directly by the \LaTeX{} compiler.
%
% %%%%%%%%%%%%%%%%%%%%%%%%%%%%%%%%%%%%%%
% \paragraph{Main File.}
%
% The main file is called |cdocsamp.tex|.
%
% Load the \textsf{childdoc} definitions and
% declare the filename for the main document:
%    \begin{macrocode}
\input{childdoc.def}
\childdocmain{}
%    \end{macrocode}

% Optional override for |\version| flag:
%    \begin{macrocode}
%%\ifchilddoc\else\providecommand{\version}{draft}\fi
%    \end{macrocode}

% Define the default values for the |\version| flag
% (|final| for the main file and |draft| for childs):
%    \begin{macrocode}
\ifchilddoc
\providecommand{\version}{draft}
\else
\providecommand{\version}{final}
\fi
%    \end{macrocode}

% Load the standard document class:
%    \begin{macrocode}
\documentclass[12pt]{article}
%    \end{macrocode}

% Start the document body:
%    \begin{macrocode}
\begin{document}
%    \end{macrocode}

% Declare a title page.
% Print title, part of document being processed and version flag:
%    \begin{macrocode}
\addtocounter{page}{-1}
\begin{center}
{\LARGE\bfseries{}childdoc example\par}
\vspace{1cm}
\ifchilddoc
\ifchilddocmanual part\else chapter\fi:
`\childdocname' of `\childdocjob'\par
\else
main document: `\childdocjob'\par
\fi
version: \version\par
\end{center}
\newpage
%    \end{macrocode}

% Manually include selected file,
% otherwise process as usual:
%    \begin{macrocode}
\ifchilddocmanual
\section*{part `\childdocname'}
\input{\childdocname}
\else
%    \end{macrocode}

% Include the two chapters:
%    \begin{macrocode}
\include{cdocsch1}
\include{cdocsch2}
%    \end{macrocode}

% Include the two parts unless only chapters should be displayed:
%    \begin{macrocode}
\ifchilddoc\else
\section{part three}
\input{cdocspt3}
\section{part four}
\input{cdocspt4}
\fi
%    \end{macrocode}

% Process as usual until here:
%    \begin{macrocode}
\fi
%    \end{macrocode}

% End of document body:
%    \begin{macrocode}
\end{document}
%    \end{macrocode}
%\iffalse
%</samplemain>
%\fi
%
% %%%%%%%%%%%%%%%%%%%%%%%%%%%%%%%%%%%%%%
% \paragraph{Chapter Include Files.}
%
% The include files are called |cdocsch1.tex| and |cdocsch2.tex|.
%
%\iffalse
%<*samplechap1|samplechap2>
%\fi

% Optional override for |\version| flag:
%    \begin{macrocode}
%%\providecommand{\version}{final}
%    \end{macrocode}

% Include the main document:
%    \begin{macrocode}
\input{childdoc.def}
\childdocof{cdocsamp}
%    \end{macrocode}

%\iffalse
%</samplechap1|samplechap2>
%\fi
%
%\iffalse
%<*samplechap1>
%\fi
% Some text for chapter 1:
%    \begin{macrocode}
\section{one}
some text in chapter one
%    \end{macrocode}

%\iffalse
%</samplechap1>
%\fi
% Some text for chapter 2:
%\iffalse
%<*samplechap2>
%\fi
%    \begin{macrocode}
\section{two}
more text in chapter two
%    \end{macrocode}

%\iffalse
%</samplechap2>
%\fi
%
% %%%%%%%%%%%%%%%%%%%%%%%%%%%%%%%%%%%%%%
% \paragraph{Part Include Files.}
%
% The include files are called |cdocspt3.tex| and |cdocspt4.tex|.
%
%\iffalse
%<*samplepart3|samplepart4>
%\fi

% Optional override for |\version| flag:
%    \begin{macrocode}
%%\providecommand{\version}{final}
%    \end{macrocode}

% Include the main document:
%    \begin{macrocode}
\input{childdoc.def}
\childdocby{cdocsamp}
%    \end{macrocode}

%\iffalse
%</samplepart3|samplepart4>
%\fi
%
%\iffalse
%<*samplepart3>
%\fi
% Some text for part 3:
%    \begin{macrocode}
some text in part three
%    \end{macrocode}

%\iffalse
%</samplepart3>
%\fi
% Some text for part 4:
%\iffalse
%<*samplepart4>
%\fi
%    \begin{macrocode}
more text in part four
%    \end{macrocode}

%\iffalse
%</samplepart4>
%\fi
%
% %%%%%%%%%%%%%%%%%%%%%%%%%%%%%%%%%%%%%%
% \paragraph{Forwarding for a Complete Draft.}
%
% The following forwarding file |cdocsdrf.tex|
% compiles the main document in draft mode:
%\iffalse
%<*sampledraft>
%\fi
%    \begin{macrocode}
\def\version{draft}
\input{childdoc.def}
\childdocforward{cdocsamp}
%    \end{macrocode}

%\iffalse
%</sampledraft>
%\fi
%
% %%%%%%%%%%%%%%%%%%%%%%%%%%%%%%%%%%%%%%
% \paragraph{Forwarding for Final Version of the Chapters.}
%
% The following forwarding files |cdocsfn1.tex| and |cdocsfn2.tex|
% (with identical content)
% compile the final versions of the child documents
% |cdocsch1.tex| and |cdocsch2.tex|, respectively:
%\iffalse
%<*samplefinal>
%\fi
%    \begin{macrocode}
\def\version{final}
\input{childdoc.def}
\childdocforwardprefix[cdocsamp]{cdocsfn}{cdocsch}
%    \end{macrocode}

%\iffalse
%</samplefinal>
%\fi
%
% %%%%%%%%%%%%%%%%%%%%%%%%%%%%%%%%%%%%%%
% \paragraph{Command Line Processing.}
%
% The following three command lines generate the output files
% |cdocscld|, |cdocscl1| and |cdocscl2|
% which should be identical to
% |cdocsdrf|, |cdocsch1| and |cdocsfn2|, respectively:
% \begin{center}
% \begin{tabular}{l}
% |latex -jobname cdocscld \|\\
% |  "\def\version{draft}\input{childdoc.def}\childdocforward{cdocsamp}"|\\
% |latex -jobname cdocscl1 \|\\
% |  "\input{childdoc.def}\childdocforward[cdocsamp]{cdocsch1}"|\\
% |latex -jobname cdocscl2 \|\\
% |  "\def\version{final}\input{childdoc.def}\childdocforward{cdocsch2}"|
% \end{tabular}
% \end{center}
% Note that the trailing backslash on each first line
% merely continues the input to the second line
% (for convenient cut ant paste).
% Furthermore, the command |latex| can be replaced by any
% of its alternative versions such as |pdflatex|.
%
% %%%%%%%%%%%%%%%%%%%%%%%%%%%%%%%%%%%%%%%%%%%%%%%%%%%%%%%%%%%%%%%%%%%%%%%%%%%%%%
% %%%%%%%%%%%%%%%%%%%%%%%%%%%%%%%%%%%%%%%%%%%%%%%%%%%%%%%%%%%%%%%%%%%%%%%%%%%%%%
% \section{Implementation}
%\iffalse
%<*package>
%\fi
%
% This section describes the definitions file |childdoc.def|.

% The definitions cannot be loaded using |\usepackage| or |\RequirePackage|
% which has a mechanism to prevent loading a style file more than once.
% When loading the definitions by means of |\input|
% multiple instances have to be prevented manually:
%\iffalse
%This code needs to be before the `\ProvidesFile' directive
%which is defined at the beginning of this file.
%Therefore it is also placed there and commented out here.
%</package>
%<*discard>
%\fi
%    \begin{macrocode}
\ifdefined\childdocmain\endinput\fi
%    \end{macrocode}
%\iffalse
%</discard>
%<*package>
%\fi
%
% \macro{\ifchilddoc}
% \macro{\ifchilddocmanual}
% The conditional |\ifchilddoc| tells whether a
% child (true) or main (false) document is being compiled.
% The conditional |\ifchilddocmanual| tells whether
% the |\includeonly| mechanism is used (false) or
% the selection of child files must be performed manually (true).
% The definitions initialise to false:
%    \begin{macrocode}
\newif\ifchilddoc
\newif\ifchilddocmanual
%    \end{macrocode}

% \macro{\childdocname}
% \macro{\childdocjob}
% The macro |\childdocname| stores the name of the main document
% to be compiled. The macro |\childdocjob| stores the name of
% the document on which the \LaTeX{} compiler was originally invoked.
% The content of |\jobname| cannot be compared
% to filenames specified in the source due to different catcodes.
% The following code rescans |\jobname|, stores the result
% in |\childdocname| and saves a copy in |\childdocjob|:
%    \begin{macrocode}
\edef\childdocname{\scantokens\expandafter{\jobname\noexpand}}
\let\childdocjob\childdocname
%    \end{macrocode}

% \macro{\childdocdisable}
% The macro |\childdocdisable| prevents the main file
% from being processed more than once.
% At this stage, the main document command |\childdocmain|
% is assumed to be called once again where it should do nothing.
% Any subsequent call to it should prevent
% a secondary processing of the main document
% It overwrites the forwarding commands
% |\childdocof| and |\childdocforward|
% with empty macros to prevent further inclusions of the main document:
%    \begin{macrocode}
\newcommand{\childdocdisable}
{
  \renewcommand{\childdocmain}[1]{\renewcommand{\childdocmain}[1]{\endinput}}
  \renewcommand{\childdocof}[1]{}
  \renewcommand{\childdocby}[2][]{}
  \renewcommand{\childdocforward}[2][]{}
  \renewcommand{\childdocdisable}{}
}
%    \end{macrocode}

% \macro{\childdocmain}
% The macro |\childdocmain| is to be called at the top of the main file
% with nothing or the main filename (without extension) as argument.
% First, it breaks loops.
% If the argument is not empty and does not match |\childdocname|
% (which is set by the first inclusion of |childdoc.def|),
% |\ifchilddoc| is set to true, |\includeonly| is applied to the child file
% and |\jobname| is set to the main file
% (for proper handling of |.aux| files):
%    \begin{macrocode}
\newcommand{\childdocmain}[1]
{
  \childdocdisable\childdocmain{}
  \if?#1?\else
    \begingroup
      \def\childdoctmp{#1}
      \ifx\childdoctmp\childdocname
        \def\childdoctmp{}
      \else
        \def\childdoctmp
        {
          \childdoctrue
          \includeonly{\childdocname}
          \def\childdocjob{#1}
          \def\jobname{#1}
        }
      \fi
      \expandafter
    \endgroup
    \childdoctmp
  \fi
}
%    \end{macrocode}

% \macro{\childdocof}
% The command |\childdocof| redirects
% compilation to the main file |#1|.
%    \begin{macrocode}
\newcommand{\childdocof}[1]
{
  \childdocdisable
  \childdoctrue
  \includeonly{\childdocname}
  \def\jobname{#1}
  \def\childdocjob{#1}
  \input{#1}
}
%    \end{macrocode}

% \macro{\childdocby}
% The command |\childdocby| ....
%    \begin{macrocode}
\newcommand{\childdocby}[2][]
{
  \childdocdisable
  \childdoctrue
  \childdocmanualtrue
  \if?#1?\else
    \def\jobname{#2}
  \fi
  \def\childdocjob{#2}
  \input{#2}
  \endinput
}
%    \end{macrocode}

% \macro{\childdocforward}
% The command |\childdocforward| redirects
% compilation to the main file or
% (if the optional argument is given) a child file.
% Parameters are set as if the main file
% or a child file starting with |\childdocof| was compiled.
% Then compilation is handed over to the main file:
%    \begin{macrocode}
\newcommand{\childdocforward}[2][]
{
  \begingroup
    \if?#1?
      \def\childdoctmp
      {
        \def\childdocname{#2}
        \def\childdocjob{#2}
        \def\jobname{#2}
        \input{#2}
        \endinput
      }
    \else
      \def\childdoctmp
      {
        \childdocdisable
        \def\childdocname{#2}
        \childdoctrue
        \includeonly{#2}
        \def\childdocjob{#1}
        \def\jobname{#1}
        \input{#1}
        \endinput
      }
    \fi
    \expandafter
  \endgroup
  \childdoctmp
}
%    \end{macrocode}

% \macro{\childdocforwardprefix}
% The command |\childdocforwardprefix| redirects
% compilation to the main or a child file by means of a pattern.
% The prefix |#1| in the current filename is replaced by |#2|
% and the suffix of the current filename is kept
% (it is assumed that the filename does not contain the substring `|~~~|'
% which is used as a delimiter).
% Compilation is handed over to the new file by |\childdocforward|:
%    \begin{macrocode}
\newcommand{\childdocforwardprefix}[3][]
{
  \begingroup
    \def\childdocextract #2##1~~~{\def\childdoctmp{\childdocforward[#1]{#3##1}}}
    \expandafter\childdocextract\childdocname~~~
    \expandafter
  \endgroup
  \childdoctmp
}
%    \end{macrocode}

% \macro{\childdoc}
% The deprecated macro |\childdoc| is a legacy version of |\childdocmain|:
%    \begin{macrocode}
\newcommand{\childdoc}{\childdocmain}
%    \end{macrocode}

% \macro{\childdocredirect}
% The deprecated macro |\childdocredirect| is a legacy version
% of |\childdocforward| and |\childdocforwardprefix|:
%    \begin{macrocode}
\newcommand{\childdocredirect}[2][]
{
  \begingroup
    \if?#1?
      \def\childdoctmp{\childdocforward{#2}}
    \else
      \def\childdoctmp{\childdocforwardprefix{#1}{#2}}
    \fi
    \expandafter
  \endgroup
  \childdoctmp
}
%    \end{macrocode}

%\iffalse
%</package>
%\fi
%
\endinput

\childdocforward{cdocsamp}
%    \end{macrocode}

%\iffalse
%</sampledraft>
%\fi
%
% %%%%%%%%%%%%%%%%%%%%%%%%%%%%%%%%%%%%%%
% \paragraph{Forwarding for Final Version of the Chapters.}
%
% The following forwarding files |cdocsfn1.tex| and |cdocsfn2.tex|
% (with identical content)
% compile the final versions of the child documents
% |cdocsch1.tex| and |cdocsch2.tex|, respectively:
%\iffalse
%<*samplefinal>
%\fi
%    \begin{macrocode}
\def\version{final}
% \iffalse
%
% childdoc.dtx Copyright (C) 2017-2018 Niklas Beisert
%
% This work may be distributed and/or modified under the
% conditions of the LaTeX Project Public License, either version 1.3
% of this license or (at your option) any later version.
% The latest version of this license is in
%   http://www.latex-project.org/lppl.txt
% and version 1.3 or later is part of all distributions of LaTeX
% version 2005/12/01 or later.
%
% This work has the LPPL maintenance status `maintained'.
%
% The Current Maintainer of this work is Niklas Beisert.
%
% This work consists of the files childdoc.dtx and childdoc.ins
% and the derived files childdoc.def and cdocsamp.tex with
% cdocsch1.tex, cdocsch2.tex, cdocsdrf.tex, cdocsfn1.tex, cdocsfn2.tex.
%
%<package>\ifdefined\childdocmain\endinput\fi
%<package>\ProvidesFile{childdoc.def}[2018/12/30 v2.0 child document driver]
%<samplemain>\ProvidesFile{cdocsamp.tex}[2018/12/30 v2.0 sample for childdoc]
%<*driver>
%\ProvidesFile{childdoc.drv}[2018/12/30 v2.0 childdoc reference manual file]
\PassOptionsToClass{10pt,a4paper}{article}
\documentclass{ltxdoc}

\usepackage[margin=35mm]{geometry}
\usepackage{hyperref}
\usepackage{hyperxmp}
\usepackage[usenames]{color}

\hypersetup{colorlinks=true}
\hypersetup{pdfstartview=FitH}
\hypersetup{pdfpagemode=UseNone}
\hypersetup{pdfsource={}}
\hypersetup{pdflang={en-UK}}
\hypersetup{pdfcopyright={Copyright 2017-2018 Niklas Beisert.
  This work may be distributed and/or modified under the
  conditions of the LaTeX Project Public License, either version 1.3
  of this license or (at your option) any later version.}}
\hypersetup{pdflicenseurl={http://www.latex-project.org/lppl.txt}}
\hypersetup{pdfcontactaddress={ETH Zurich, ITP, HIT K,
  Wolfgang-Pauli-Strasse 27}}
\hypersetup{pdfcontactpostcode={8093}}
\hypersetup{pdfcontactcity={Zurich}}
\hypersetup{pdfcontactcountry={Switzerland}}
\hypersetup{pdfcontactemail={nbeisert@itp.phys.ethz.ch}}
\hypersetup{pdfcontacturl={http://people.phys.ethz.ch/\xmptilde nbeisert/}}

\newcommand{\secref}[1]{\hyperref[#1]{section \ref*{#1}}}

\parskip1ex
\parindent0pt
\let\olditemize\itemize
\def\itemize{\olditemize\parskip0pt}

\begin{document}

\title{The \textsf{childdoc} Package}
\hypersetup{pdftitle={The childdoc Package}}
\author{Niklas Beisert\\[2ex]
  Institut f\"ur Theoretische Physik\\
  Eidgen\"ossische Technische Hochschule Z\"urich\\
  Wolfgang-Pauli-Strasse 27, 8093 Z\"urich, Switzerland\\[1ex]
  \href{mailto:nbeisert@itp.phys.ethz.ch}
  {\texttt{nbeisert@itp.phys.ethz.ch}}}
\hypersetup{pdfauthor={Niklas Beisert}}
\hypersetup{pdfsubject={Manual for the LaTeX2e Package childdoc}}
\date{30 December 2018, \textsf{v2.0}}
\maketitle

\begin{abstract}\noindent
\textsf{childdoc} is a \LaTeXe{} package
that enables the direct compilation
of document sections included by |\include|
to individual files.
\end{abstract}

\begingroup
\parskip0ex
\tableofcontents
\endgroup

%%%%%%%%%%%%%%%%%%%%%%%%%%%%%%%%%%%%%%%%%%%%%%%%%%%%%%%%%%%%%%%%%%%%%%%%%%%%%%%%
%%%%%%%%%%%%%%%%%%%%%%%%%%%%%%%%%%%%%%%%%%%%%%%%%%%%%%%%%%%%%%%%%%%%%%%%%%%%%%%%
\section{Introduction}

\LaTeX{} provides a mechanism to structure a large document (such as a book)
into a main file and several child files (containing the chapters)
using the |\include| command.
This mechanism is beneficial for documents
which span hundreds of pages in order to
make the source file(s) more manageable.
Moreover, compilation can be restricted to
selected child files by means of the |\includeonly| command.
The latter feature can be used to reduce the compilation time while editing
(this was significantly more useful in the earlier days of \LaTeX{})
or to generate a smaller document which is easier to navigate.
Another application of |\includeonly| is to generate
documents consisting of selected parts of the complete document.

However, there are a few drawbacks of the plain |\include| mechanism:
\begin{itemize}
\item
The child files cannot be compiled on their own,
they can only be compiled via the main file.
A naive editing environment
(such as a text editor with an option
to have the current file processed by \LaTeX)
may require one to switch to the main file before compiling;
attempting to compile the child file produces errors.
\item
The main file must be modified (each time)
to adjust the |\includeonly| command
to the present needs. This easily leaves the main file in a messy state.
\item
The generated document will always carry the filename
of the main document. This is inconvenient if
several child files are to be compiled and
to be kept for distribution.
\end{itemize}

The present package provides a simple interface
to make child files individually compilable by \LaTeX{}.
Compiling a child file then has the same effect as compiling
the main file with an |\includeonly| command
to select the appropriate child.
Moreover the generated document will carry the name of the child
rather than the main file.
This resolves all three above issues.

This feature is meant to make the editing of books,
thesis documents and lecture notes somewhat more convenient.
However, the package can also be used efficiently for
composing a series of documents (such as exercise sheets)
which are typically distributed individually.
It then assists the author in generating the individual documents
(potentially in different versions)
as well as a document containing the collected series.
Another application is in developing style files
or other kinds of included material
where compilation of the style file could redirect
to a sample or test file.

%%%%%%%%%%%%%%%%%%%%%%%%%%%%%%%%%%%%%%%%%%%%%%%%%%%%%%%%%%%%%%%%%%%%%%%%%%%%%%%%
%%%%%%%%%%%%%%%%%%%%%%%%%%%%%%%%%%%%%%%%%%%%%%%%%%%%%%%%%%%%%%%%%%%%%%%%%%%%%%%%
\section{Usage}

First of all, the package \textsf{childdoc} is \emph{not} a standard
\LaTeXe{} |.sty| style file! Therefore it needs to be invoked in
a non-standard way.

%%%%%%%%%%%%%%%%%%%%%%%%%%%%%%%%%%%%%%%%%%%%%%%%%%%%%%%%%%%%%%%%%%%%%%%%%%%%%%%%
\subsection{Included Files}
\label{sec:include}

%%%%%%%%%%%%%%%%%%%%%%%%%%%%%%%%%%%%%%%%
\DescribeMacro{\childdocmain}
To use the package, add the commands
\begin{center}
\begin{tabular}{l}
|\input{childdoc.def}|\\
|\childdocmain{}|\\
\end{tabular}
\end{center}
at the very top of the main \LaTeX{} file,
in particular \emph{before} the |\documentclass| statement!
The argument of |\childdocmain| should be left empty
(but it must be present).

%%%%%%%%%%%%%%%%%%%%%%%%%%%%%%%%%%%%%%%%
\DescribeMacro{\childdocof}
Furthermore, add the commands
\begin{center}
\begin{tabular}{l}
|\input{childdoc.def}|\\
|\childdocof{|\textit{main}|}|\\
\end{tabular}
\end{center}
at the top of every child file \textit{child}
which is included by |\include{|\textit{child}|}|
from within the main file
(or at least for those files to be compiled individually).
The argument \textit{main} must be the filename of the main file.

There are a couple of
considerations in setting up the main and child documents:

%%%%%%%%%%%%%%%%%%%%%%%%%%%%%%%%%%%%%%%%
\paragraph{Restrictions.}

Please note the following restrictions:
\begin{itemize}
\item
|\childdocmain| must be called with one argument \textit{main}
to ensure compatibility with earlier version of the package.
It must either be empty (|\childdocmain{}|)
or precisely match the filename of the main file in which it is specified.
See \secref{sec:detection} for further information.
\item
The filename \textit{main} must be specified without the |.tex| extension.
\item
The filename \textit{main} is case sensitive
(even in case-insensitive file systems)
due to internal string comparison.
\item
The argument \textit{main} should be fully expanded, it cannot be a macro.
\item
Subdirectories and special characters should be avoided in filenames.
\item
The command |\childdocmain{|\textit{main}|}| must be followed by a whitespace.
It should not be followed immediately by another command
or by a comment mark `|%|'.
This is because the \TeX{} parser reads the token immediately following
the argument of |\childdocmain| and puts it
at the beginning of every child section;
however, a white\-space is ignored.
\end{itemize}

%%%%%%%%%%%%%%%%%%%%%%%%%%%%%%%%%%%%%%%%
\paragraph{Content of Main File.}

It is advisable to place all content in the child files included by |\include|.
Any output contained in the main file will appear in all child documents
unless suppressed manually;
it cannot be suppressed automatically by the |\includeonly| directive
and thus should normally be avoided.
A method to include some content in the main file
by means of conditional processing is described in \secref{sec:conditional}.

%%%%%%%%%%%%%%%%%%%%%%%%%%%%%%%%%%%%%%%%
\paragraph{Page Numbering.}

When only a part of the document is compiled,
the appropriate numbering of pages
(as well as other status parameters)
is determined from the |.aux| files.
The latter contain information from previous passes.
However this information needs to propagate through
all intermediate child documents.
Therefore the page numbering in child documents may well
be inconsistent until the complete document is compiled at least once.

A useful (if unconventional) way to always ensure a consistent
page numbering is to restart the numbering in each child document
and denote the pages by `\textit{child}|.|\textit{page}'
where \textit{child} represents the chapter/section number of the child file.
This can be achieved by the command
|\numberwithin{page}{|\textit{child}|}|
of the \textsf{amsmath} package
where \textit{child} can be |chapter| or |section|
depending on the chosen structuring.
Alternatively, one can modify the macro |\thepage| appropriately
and reset the counter |page| at the start of each child file.

%%%%%%%%%%%%%%%%%%%%%%%%%%%%%%%%%%%%%%%%%%%%%%%%%%%%%%%%%%%%%%%%%%%%%%%%%%%%%%%%
\subsection{Conditional Processing}
\label{sec:conditional}

The package provides a mechanism to compile different versions
of a document. To customise the versions further some conditional processing
can come in handy to distinguish which version is being compiled.
The package provides two macros to describe the compilation context:

%%%%%%%%%%%%%%%%%%%%%%%%%%%%%%%%%%%%%%%%
\DescribeMacro{\ifchilddoc}
The conditional |\ifchilddoc| distinguishes between the compilation of
child documents and the main document:
%
\begin{center}
|\ifchilddoc |\textit{child-code}| |[|\||else |\textit{main-code}]| \||fi|
\end{center}

%%%%%%%%%%%%%%%%%%%%%%%%%%%%%%%%%%%%%%%%
\DescribeMacro{\childdocname}
\DescribeMacro{\childdocjob}
The macro |\childdocname| contains the filename (without extension)
of the main or child file being processed.
Note that |\childdocjob| will always contain the name of the main file.

%%%%%%%%%%%%%%%%%%%%%%%%%%%%%%%%%%%%%%%%
\paragraph{Title Page.}

Conditional processing can be used to include a title or banner page
in the main document when proper precautions are taken.
Importantly, the code in the main file should ensure that the page counter
(as well as other status parameters which are stored in the |.aux| files)
takes the same value after the conditional processing.
Otherwise the page numbers may take divergent values
depending on which part is compiled.

For example, a title page could be declared by:
%
\begin{center}
\begin{tabular}{l}
|\ifchilddoc\||else|\\
|\addtocounter{page}{-1}|\\
\textit{code for title page}\\
|\newpage|\\
|\||fi|
\end{tabular}
\end{center}
%
A banner page for the child documents can be generated by:
%
\begin{center}
\begin{tabular}{l}
|\ifchilddoc|\\
|\addtocounter{page}{-1}|\\
\textit{code for banner page}\\
|\newpage|\\
|\||fi|
\end{tabular}
\end{center}
%
Here one could write a message such as:
\begin{center}
|This is the part \childdocname{} of \childdocjob{}.|
\end{center}

%%%%%%%%%%%%%%%%%%%%%%%%%%%%%%%%%%%%%%%%%%%%%%%%%%%%%%%%%%%%%%%%%%%%%%%%%%%%%%%%
\subsection{Flags}
\label{sec:flags}

The package makes it easy to generate different versions
of the main or child documents.
To this end compilation flags can be defined
and assigned different default values.
They will be particularly useful in conjunction
with the forwarding mechanism described in \secref{sec:forward}.

For example, it may be useful to have a flag |\version|
which can be set to |draft| or |final|.
The document source will contain some conditional code
depending on the value of |\version|.
Suppose further, the flag should default to |final| for the main file
and to |draft| for child files
which is a natural assignment for editing the document.
This is achieved by placing the following code
in the preamble of the main document
(below the |\childdocmain| directive):
%
\begin{center}
\begin{tabular}{l}
|\ifchilddoc|\\
|\providecommand{\version}{draft}|\\
|\||else|\\
|\providecommand{\version}{final}|\\
|\||fi|
\end{tabular}
\end{center}
%
The definition by |\providecommand| makes sure
that previous definitions are not overwritten.
Further statements |\providecommand{\version}{...}|
can thus be added before the above code to override it.

For the main file, one might add a line
(between |\childdocmain| and the above block)
%
\begin{center}
|%\ifchilddoc\||else\providecommand{\version}{draft}\||fi|
\end{center}
%
which can be uncommented to produce a draft version.
Likewise one can add a line to the very top of a child file
(above the |\childdocof{|\textit{main}|}| directive)
%
\begin{center}
|%\providecommand{\version}{final}|
\end{center}
%
which can be uncommented to produce the final version of this child document.

%%%%%%%%%%%%%%%%%%%%%%%%%%%%%%%%%%%%%%%%%%%%%%%%%%%%%%%%%%%%%%%%%%%%%%%%%%%%%%%%
\subsection{Forwarding}
\label{sec:forward}

Different versions of the main or child documents
using compilation flags as described in \secref{sec:flags}
can be (permanently) stored in different files
for convenient compilation, viewing and distribution.
To this end, the package defines a command
to pass on compilation to a different file:

%%%%%%%%%%%%%%%%%%%%%%%%%%%%%%%%%%%%%%%%
\DescribeMacro{\childdocforward}
The command |\childdocforward| redirects processing to
another source file:
%
\begin{center}
\begin{tabular}{l}
|\input{childdoc.def}|\\
|\childdocforward[|\textit{main}|]{|\textit{dest}|}|\\
\end{tabular}
\end{center}
%
The argument \textit{dest} is the destination file
(without extension).
It should be the main file or one of the child files.
Note that further \textsf{childdoc} directives
such as |\childdocof| and |\childdocforward|
in the indicated file will be processed in this form.
The optional argument \textit{main}
passes on directly to the main file \textit{main}
while pretending to compile the child \textit{dest}.
This form behaves as if \textit{dest}
issues |\childdocof{|\textit{main}|}| right away,
and no further \textsf{childdoc} directives will be processed.

%%%%%%%%%%%%%%%%%%%%%%%%%%%%%%%%%%%%%%%%
\DescribeMacro{\...prefix}
In the alternative form |\childdocforwardprefix|,
%
\begin{center}
\begin{tabular}{l}
|\input{childdoc.def}|\\
|\childdocforwardprefix[|\textit{main}|]{|\textit{prefix}|}{|\textit{dest}|}|
\end{tabular}
\end{center}
%
the destination file is determined by a pattern
depending on the current file:
To make this work, the current file must be called
`{\textit{prefix}\hspace{0.2em}\textit{suffix}}'
with \textit{prefix} matching precisely the argument.
Processing is then passed on to the file
`{\textit{dest}\hspace{0.2em}\textit{suffix}}'.
Surely, the same effect is achieved by
directly specifying the
argument `{\textit{dest}\hspace{0.2em}\textit{suffix}}'
in the first form.
However, that requires to set up a different file
for each child. With the alternative form of the command
all these files can have exactly the same content
which simplifies setting them up and maintaining them.

For example, the following file |draft.tex|
with a compilation flag |\version| as described in \secref{sec:flags}
compiles the main document as a draft:
%
\begin{center}
\begin{tabular}{l}
|\def\version{draft}|\\
|\input{childdoc.def}|\\
|\childdocforward{|\textit{main}|}|
\end{tabular}
\end{center}
%
Likewise, the following files |final|\textit{nn}|.tex|
compile the final version of the child document
|child|\textit{nn}|.tex|:
%
\begin{center}
\begin{tabular}{l}
|\def\version{final}|\\
|\input{childdoc.def}|\\
|\childdocforwardprefix{final}{child}|
\end{tabular}
\end{center}
%

Note that when several versions of a main file and/or of each child file
are to be generated, it may be convenient to set up a |Makefile| or
shell script to automatise the process.

%%%%%%%%%%%%%%%%%%%%%%%%%%%%%%%%%%%%%%%%%%%%%%%%%%%%%%%%%%%%%%%%%%%%%%%%%%%%%%%%
\subsection{Command Line Processing}
\label{sec:commandline}

The effect of redirection files can also be achieved by invoking
the \LaTeX{} compiler with a more elaborate command line.
Most conveniently this should be done as part
of a shell script or a |Makefile|.

When using \textsf{childdoc} in the main file, the following
command lines effectively perform a redirection
(note that depending on the shell being used,
backslashes may have to be doubled: `|\|' $\to$ `|\\|'):
%
\begin{center}
|... -jobname "|\textit{target}|" |\\|"|[\textit{flags}]%
|\input{childdoc.def}\childdocforward[|\textit{main}|]{|\textit{dest}|}"|
\end{center}
%
Here \textit{target} is the name of the output file,
\textit{main} is the name of the main file
and \textit{dest} is the name of the main or child file to be processed
(all filenames without extensions).
The optional argument \textit{main} can be omitted
if \textit{main} matches \textit{dest}.
Optionally, compilation \textit{flags} can be defined via |\def| commands.
This command line makes the \TeX{} engine believe
it is compiling the file \textit{target}
whose content is specified as the latter parameter.
The provided code then forwards the processing to
\textit{main} or \textit{dest} as described in \secref{sec:forward}.

%%%%%%%%%%%%%%%%%%%%%%%%%%%%%%%%%%%%%%%%%%%%%%%%%%%%%%%%%%%%%%%%%%%%%%%%%%%%%%%%
\subsection{Include by Input}
\label{sec:input}

Including child documents by |\include| has some restrictions by design.
Most notably, the content of a child document always occupies
its own set of pages; pages cannot be shared between child documents.
Usually, this behaviour makes perfect sense
because each child document contain an essential part of the document.
However, in some situations it may be desirable to compose
a document from a collection of parts
without having mandatory page breaks between then.
For this case, the package
provides a mechanism to include parts
by |\input| which can also be processed individually.
However, by construction this mechanism
requires manual handling of the content to be output.

%%%%%%%%%%%%%%%%%%%%%%%%%%%%%%%%%%%%%%%%
\DescribeMacro{\ifchilddocmanual}
The main file should be prepared as usual, see \secref{sec:include}.
However, the document body must make a distinction
between processing of an individual part and of the main document, e.g.:
%
\begin{center}
\begin{tabular}{l}
|\ifchilddocmanual|\\
|\input{\childdocname}|\\
|\||else|\\
\textit{document body with }|\input{|\textit{part}|}|\\
|\||fi|
\end{tabular}
\end{center}
%
The conditional |\ifchilddocmanual| is true whenever
a part to be included by |\input| is being compiled,
and the name of the part is stored in |\childdocname|.

%%%%%%%%%%%%%%%%%%%%%%%%%%%%%%%%%%%%%%%%
\DescribeMacro{\childdocby}
Each part to be included by |\input| should start with:
%
\begin{center}
\begin{tabular}{l}
|\input{childdoc.def}|\\
|\childdocby{|\textit{main}|}|\\
\end{tabular}
\end{center}
%
The directive |\childdocby| is similar to |\childdocof|
described in \secref{sec:include},
but the subsequent selection of content must be done manually.
To that end, both |\ifchilddoc| and |\ifchilddocmanual|
will be true upon processing of a part,
and the name of the part is stored in |\childdocname|.
Note that |\jobname| will be set to the filename of the current part
so that each part receives an individual |.aux| file
that does not interfere with the |.aux| file(s) of the main document.
This behaviour can be altered by the alternative form
|\childdocby[*]{|\textit{main}|}| (with a non-empty optional argument)
which uses the |.aux| file of the main document
by setting |\jobname| to \textit{main}.

%%%%%%%%%%%%%%%%%%%%%%%%%%%%%%%%%%%%%%%%%%%%%%%%%%%%%%%%%%%%%%%%%%%%%%%%%%%%%%%%
\subsection{Driver Development}
\label{sec:driver}

The \textsf{childdoc} mechanism can also be use for the development
of definition files such as \LaTeX{} styles or classes.
This case differs from the above setup with multiple parts
included by |\include| in that no |\includeonly| should be invoked.
This can be achieved by starting the include file
(before |\ProvidesPackage|) with:
%
\begin{center}
\begin{tabular}{l}
|\input{childdoc.def}|\\
|\childdocforward{|\textit{main}|}|\\
\end{tabular}
\end{center}
%
or alternatively with:
%
\begin{center}
\begin{tabular}{l}
|\input{childdoc.def}|\\
|\childdocby{|\textit{main}|}|\\
\end{tabular}
\end{center}
%
Both forms have slightly different effects as described above.
The main file is prepared as usual, see \secref{sec:include}.

%%%%%%%%%%%%%%%%%%%%%%%%%%%%%%%%%%%%%%%%%%%%%%%%%%%%%%%%%%%%%%%%%%%%%%%%%%%%%%%%
\subsection{Legacy Detection}
\label{sec:detection}

The directive |\childdocmain| in the main file can detect
whether the complete document or merely a child is to be compiled
even without using the directive |\childdocof|.
This method is deprecated because it is less robust
and there is no compelling reason to use it;
it is merely provided for backward compatibility
and it may be removed in future versions.

If the detection mechanism is to be used,
it is mandatory to correctly specify
the filename of the main file as the argument of |\childdocmain|:
%
\begin{center}
\begin{tabular}{l}
|\input{childdoc.def}|\\
|\childdocmain{|\textit{main}|}|\\
\end{tabular}
\end{center}
%
If |\jobname| does not match the argument \textit{main} of |\childdocmain|,
it is assumed that |\jobname| points to the child file to be compiled.
When using |\childdocmain| with the main file specified as argument,
it suffices to start a child file
with just |\input{|\textit{main}|}|
without loading of the package and using |\childdocof|.
If instead all processing is done
with the appropriate \textsf{childdoc} directives,
the argument of \textit{main} of |\childdocmain| can be empty.

An alternative version of the command line processing described
in \secref{sec:commandline} using the detection mechanism reads:
%
\begin{center}
|... -jobname "|\textit{target}|" "|[\textit{flags}]%
[|\def\jobname{|\textit{dest}|}|]|\input{|\textit{main}|}"|
\end{center}

%%%%%%%%%%%%%%%%%%%%%%%%%%%%%%%%%%%%%%%%%%%%%%%%%%%%%%%%%%%%%%%%%%%%%%%%%%%%%%%%
\subsection{Manual Code}
\label{sec:manual}

In case one cannot be certain whether the definitions file |childdoc.def|
is installed on the target \TeX{} distribution
and one prefers not to ship it,
it is conceivable to paste a few relevant commands into the sources.

To that end, drop all statements |\input{childdoc.def}|
and perform the replacements as outlined below.
Instead of |\childdocmain{|\textit{main}|}| add the following code
to the top of the main file:
%
\begin{center}
\begin{tabular}{l}
|\||ifdefined\childdocname\endinput\||fi\newif\ifchilddoc|\\
|\edef\childdocname{\scantokens\expandafter{\jobname\noexpand}}|\\
|\def\childdocmain{|\textit{main}|}\||ifx\childdocmain\childdocname\||else|\\
|\childdoctrue\includeonly{\childdocname}\let\jobname\childdocmain\||fi|\\
\end{tabular}
\end{center}
%
Instead of |\childdocof{|\textit{main}|}| just include the main file
at the top of each child file:
%
\begin{center}
|\input{|\textit{main}|}|
\end{center}
%
A simple redirection |\childdocforward{|\textit{dest}|}| is achieved by:
%
\begin{center}
|\def\jobname{|\textit{dest}|}\input{\jobname}|
\end{center}
%
The redirection with prefix
|\childdocforwardprefix[|\textit{prefix}|]{|\textit{dest}|}|
is accomplished by:
%
\begin{center}
\begin{tabular}{l}
|{\edef\jobname{\scantokens\expandafter{\jobname\noexpand}}|\\
|\def\redirectjob |\textit{prefix}|#1~~~{\gdef\jobname{|\textit{dest}|#1}}|\\
|\expandafter\redirectjob\jobname~~~}\input{\jobname}|
\end{tabular}
\end{center}

In an alternative approach,
child documents can be compiled by a specific command line
without additional code or specific definitions:
%
\begin{center}
|... -jobname "|\textit{target}|" "|[\textit{flags}]%
|\includeonly{|\textit{dest}|}\input{|\textit{main}|}"|
\end{center}
%

%%%%%%%%%%%%%%%%%%%%%%%%%%%%%%%%%%%%%%%%%%%%%%%%%%%%%%%%%%%%%%%%%%%%%%%%%%%%%%%%
%%%%%%%%%%%%%%%%%%%%%%%%%%%%%%%%%%%%%%%%%%%%%%%%%%%%%%%%%%%%%%%%%%%%%%%%%%%%%%%%
\section{Information}

%%%%%%%%%%%%%%%%%%%%%%%%%%%%%%%%%%%%%%%%%%%%%%%%%%%%%%%%%%%%%%%%%%%%%%%%%%%%%%%%
\subsection{Copyright}

Copyright \copyright{} 2017--2018 Niklas Beisert

This work may be distributed and/or modified under the
conditions of the \LaTeX{} Project Public License, either version 1.3
of this license or (at your option) any later version.
The latest version of this license is in
  \url{http://www.latex-project.org/lppl.txt}
and version 1.3 or later is part of all distributions of \LaTeX{}
version 2005/12/01 or later.

This work has the LPPL maintenance status `maintained'.

The Current Maintainer of this work is Niklas Beisert.

This work consists of the files |README.txt|, |childdoc.ins| and |childdoc.dtx|
as well as the derived files |childdoc.def|, |cdocsamp.tex|
with |cdocsch1.tex|, |cdocsch2.tex|, |cdocspt3.tex|, |cdocspt4.tex|,
|cdocsdrf.tex|, |cdocsfn1.tex|, |cdocsfn2.tex|
as well as |childdoc.pdf|.

%%%%%%%%%%%%%%%%%%%%%%%%%%%%%%%%%%%%%%%%%%%%%%%%%%%%%%%%%%%%%%%%%%%%%%%%%%%%%%%%
\subsection{Files and Installation}

The package consists of the files:
%
\begin{center}
\begin{tabular}{ll}
    |README.txt|   & readme file \\
    |childdoc.ins| & installation file \\
    |childdoc.dtx| & source file \\
    |childdoc.def| & definition file \\
    |cdocsamp.tex| & sample main file \\
    |cdocsch1.tex| & sample include file \\
    |cdocsch2.tex| & sample include file \\
    |cdocspt3.tex| & sample part file \\
    |cdocspt4.tex| & sample part file \\
    |cdocsdrf.tex| & sample redirection file \\
    |cdocsfn1.tex| & sample redirection file \\
    |cdocsfn2.tex| & sample redirection file \\
    |childdoc.pdf| & manual
\end{tabular}
\end{center}
%
The distribution consists of the files
|README.txt|, |childdoc.ins| and |childdoc.dtx|.
%
\begin{itemize}
\item
Run (pdf)\LaTeX{} on |childdoc.dtx|
to compile the manual |childdoc.pdf| (this file).
\item
Run \LaTeX{} on |childdoc.ins| to create the definitions file |childdoc.def|
and the sample |cdocsamp.tex| with include files
|cdocsch1.tex|, |cdocsch2.tex|, |cdocspt3.tex|, |cdocspt4.tex|,
|cdocsdrf.tex|, |cdocsfn1.tex|, |cdocsfn2.tex|.
Then copy the file |childdoc.def| to an appropriate directory of your \LaTeX{}
distribution, e.g.\ \textit{texmf-root}|/tex/latex/childdoc|.
\end{itemize}

%%%%%%%%%%%%%%%%%%%%%%%%%%%%%%%%%%%%%%%%%%%%%%%%%%%%%%%%%%%%%%%%%%%%%%%%%%%%%%%%
\subsection{Related CTAN Packages}

There are several other packages which offer a similar functionality:
%
\begin{itemize}
\item
The packages
\href{http://ctan.org/pkg/docmute}{\textsf{docmute}},
\href{http://ctan.org/pkg/includex}{\textsf{includex}} and
\href{http://ctan.org/pkg/standalone}{\textsf{standalone}}
provide commands to include only the document body of
a child file thus allowing both files to be compiled individually.
\item
The packages \href{http://ctan.org/pkg/subdocs}{\textsf{subdocs}}
and \href{http://ctan.org/pkg/subfiles}{\textsf{subfiles}}
provide structures in which the main and child documents can be
encapsulated and allowing them to be compiled individually.
The inclusion mechanism is different from the conventional |\include|.
\item
The package \href{http://ctan.org/pkg/combine}{\textsf{combine}}
is an elaborate solution to combine several documents into one.
\end{itemize}
%
See also the CTAN topic \href{http://ctan.org/topic/subdocs}{\textsf{subdocs}}
for further related packages.
The present package differs from the above solutions in that
a document structure constructed with the conventional |\include| mechanism
just needs two extra commands at the top of every file
such that all constituent files can be compiled individually.

%%%%%%%%%%%%%%%%%%%%%%%%%%%%%%%%%%%%%%%%%%%%%%%%%%%%%%%%%%%%%%%%%%%%%%%%%%%%%%%%
%\subsection{Feature Suggestions}
%
%The following is a list of features which may be useful for future
%versions of this package:
%%
%\begin{itemize}
%\item
%\ldots
%\end{itemize}

%%%%%%%%%%%%%%%%%%%%%%%%%%%%%%%%%%%%%%%%%%%%%%%%%%%%%%%%%%%%%%%%%%%%%%%%%%%%%%%%
\subsection{Revision History}

%%%%%%%%%%%%%%%%%%%%%%%%%%%%%%%%%%%%%%%%
\paragraph{v2.0:} 2018/12/30

\begin{itemize}
\item
immediate forward processing
\item
added |\childdocby| mechanism
\item
manual restructured
\end{itemize}

%%%%%%%%%%%%%%%%%%%%%%%%%%%%%%%%%%%%%%%%
\paragraph{v1.6:} 2018/01/17

\begin{itemize}
\item
application for development of include files
\item
corrections to manual
\end{itemize}

%%%%%%%%%%%%%%%%%%%%%%%%%%%%%%%%%%%%%%%%
\paragraph{v1.5:} 2017/05/21

\begin{itemize}
\item
more complete structuring introduced
\item
|\childdocof| introduced
\item
|\childdoc| renamed to |\childdocmain|
\item
|\childredirect| renamed to |\childdocforward| and |\childdocforwardprefix|
and functionality expanded
\end{itemize}

%%%%%%%%%%%%%%%%%%%%%%%%%%%%%%%%%%%%%%%%
\paragraph{v1.0:} 2017/04/27

\begin{itemize}
\item
manual and install package
\item
first version published on CTAN
\end{itemize}

%%%%%%%%%%%%%%%%%%%%%%%%%%%%%%%%%%%%%%%%
\paragraph{v0.6:} 2017/04/26

\begin{itemize}
\item
redirection mechanism added
\end{itemize}

%%%%%%%%%%%%%%%%%%%%%%%%%%%%%%%%%%%%%%%%
\paragraph{v0.5:} 2017/04/26

\begin{itemize}
\item
functionality in definition file
\end{itemize}


%%%%%%%%%%%%%%%%%%%%%%%%%%%%%%%%%%%%%%%%%%%%%%%%%%%%%%%%%%%%%%%%%%%%%%%%%%%%%%%%
%%%%%%%%%%%%%%%%%%%%%%%%%%%%%%%%%%%%%%%%%%%%%%%%%%%%%%%%%%%%%%%%%%%%%%%%%%%%%%%%
%%%%%%%%%%%%%%%%%%%%%%%%%%%%%%%%%%%%%%%%%%%%%%%%%%%%%%%%%%%%%%%%%%%%%%%%%%%%%%%%
\appendix

\settowidth\MacroIndent{\rmfamily\scriptsize 000\ }

 \DocInput{childdoc.dtx}

\end{document}
%</driver>
% \fi
%
% %%%%%%%%%%%%%%%%%%%%%%%%%%%%%%%%%%%%%%%%%%%%%%%%%%%%%%%%%%%%%%%%%%%%%%%%%%%%%%
% %%%%%%%%%%%%%%%%%%%%%%%%%%%%%%%%%%%%%%%%%%%%%%%%%%%%%%%%%%%%%%%%%%%%%%%%%%%%%%
% \section{Sample}
%\iffalse
%<*samplemain>
%\fi
%
% The following presents a sample document
% with two chapters, two parts, a title page,
% a compile flag as well as three forwarding files to set the flag.
% It consists of eight |.tex| files:
% \begin{center}
% \begin{tabular}{ll}
% |cdocsamp.tex|&main file\\
% |cdocsch1.tex|&include file for chapter 1\\
% |cdocsch2.tex|&include file for chapter 2\\
% |cdocspt3.tex|&include file for part 3\\
% |cdocspt4.tex|&include file for part 4\\
% |cdocsdrf.tex|&forwarding file for main file in draft mode\\
% |cdocsfi1.tex|&forwarding file for final version of chapter 1\\
% |cdocsfi2.tex|&forwarding file for final version of chapter 2\\
% \end{tabular}
% \end{center}
% Each of the eight files can be compiled directly by the \LaTeX{} compiler.
%
% %%%%%%%%%%%%%%%%%%%%%%%%%%%%%%%%%%%%%%
% \paragraph{Main File.}
%
% The main file is called |cdocsamp.tex|.
%
% Load the \textsf{childdoc} definitions and
% declare the filename for the main document:
%    \begin{macrocode}
\input{childdoc.def}
\childdocmain{}
%    \end{macrocode}

% Optional override for |\version| flag:
%    \begin{macrocode}
%%\ifchilddoc\else\providecommand{\version}{draft}\fi
%    \end{macrocode}

% Define the default values for the |\version| flag
% (|final| for the main file and |draft| for childs):
%    \begin{macrocode}
\ifchilddoc
\providecommand{\version}{draft}
\else
\providecommand{\version}{final}
\fi
%    \end{macrocode}

% Load the standard document class:
%    \begin{macrocode}
\documentclass[12pt]{article}
%    \end{macrocode}

% Start the document body:
%    \begin{macrocode}
\begin{document}
%    \end{macrocode}

% Declare a title page.
% Print title, part of document being processed and version flag:
%    \begin{macrocode}
\addtocounter{page}{-1}
\begin{center}
{\LARGE\bfseries{}childdoc example\par}
\vspace{1cm}
\ifchilddoc
\ifchilddocmanual part\else chapter\fi:
`\childdocname' of `\childdocjob'\par
\else
main document: `\childdocjob'\par
\fi
version: \version\par
\end{center}
\newpage
%    \end{macrocode}

% Manually include selected file,
% otherwise process as usual:
%    \begin{macrocode}
\ifchilddocmanual
\section*{part `\childdocname'}
\input{\childdocname}
\else
%    \end{macrocode}

% Include the two chapters:
%    \begin{macrocode}
\include{cdocsch1}
\include{cdocsch2}
%    \end{macrocode}

% Include the two parts unless only chapters should be displayed:
%    \begin{macrocode}
\ifchilddoc\else
\section{part three}
\input{cdocspt3}
\section{part four}
\input{cdocspt4}
\fi
%    \end{macrocode}

% Process as usual until here:
%    \begin{macrocode}
\fi
%    \end{macrocode}

% End of document body:
%    \begin{macrocode}
\end{document}
%    \end{macrocode}
%\iffalse
%</samplemain>
%\fi
%
% %%%%%%%%%%%%%%%%%%%%%%%%%%%%%%%%%%%%%%
% \paragraph{Chapter Include Files.}
%
% The include files are called |cdocsch1.tex| and |cdocsch2.tex|.
%
%\iffalse
%<*samplechap1|samplechap2>
%\fi

% Optional override for |\version| flag:
%    \begin{macrocode}
%%\providecommand{\version}{final}
%    \end{macrocode}

% Include the main document:
%    \begin{macrocode}
\input{childdoc.def}
\childdocof{cdocsamp}
%    \end{macrocode}

%\iffalse
%</samplechap1|samplechap2>
%\fi
%
%\iffalse
%<*samplechap1>
%\fi
% Some text for chapter 1:
%    \begin{macrocode}
\section{one}
some text in chapter one
%    \end{macrocode}

%\iffalse
%</samplechap1>
%\fi
% Some text for chapter 2:
%\iffalse
%<*samplechap2>
%\fi
%    \begin{macrocode}
\section{two}
more text in chapter two
%    \end{macrocode}

%\iffalse
%</samplechap2>
%\fi
%
% %%%%%%%%%%%%%%%%%%%%%%%%%%%%%%%%%%%%%%
% \paragraph{Part Include Files.}
%
% The include files are called |cdocspt3.tex| and |cdocspt4.tex|.
%
%\iffalse
%<*samplepart3|samplepart4>
%\fi

% Optional override for |\version| flag:
%    \begin{macrocode}
%%\providecommand{\version}{final}
%    \end{macrocode}

% Include the main document:
%    \begin{macrocode}
\input{childdoc.def}
\childdocby{cdocsamp}
%    \end{macrocode}

%\iffalse
%</samplepart3|samplepart4>
%\fi
%
%\iffalse
%<*samplepart3>
%\fi
% Some text for part 3:
%    \begin{macrocode}
some text in part three
%    \end{macrocode}

%\iffalse
%</samplepart3>
%\fi
% Some text for part 4:
%\iffalse
%<*samplepart4>
%\fi
%    \begin{macrocode}
more text in part four
%    \end{macrocode}

%\iffalse
%</samplepart4>
%\fi
%
% %%%%%%%%%%%%%%%%%%%%%%%%%%%%%%%%%%%%%%
% \paragraph{Forwarding for a Complete Draft.}
%
% The following forwarding file |cdocsdrf.tex|
% compiles the main document in draft mode:
%\iffalse
%<*sampledraft>
%\fi
%    \begin{macrocode}
\def\version{draft}
\input{childdoc.def}
\childdocforward{cdocsamp}
%    \end{macrocode}

%\iffalse
%</sampledraft>
%\fi
%
% %%%%%%%%%%%%%%%%%%%%%%%%%%%%%%%%%%%%%%
% \paragraph{Forwarding for Final Version of the Chapters.}
%
% The following forwarding files |cdocsfn1.tex| and |cdocsfn2.tex|
% (with identical content)
% compile the final versions of the child documents
% |cdocsch1.tex| and |cdocsch2.tex|, respectively:
%\iffalse
%<*samplefinal>
%\fi
%    \begin{macrocode}
\def\version{final}
\input{childdoc.def}
\childdocforwardprefix[cdocsamp]{cdocsfn}{cdocsch}
%    \end{macrocode}

%\iffalse
%</samplefinal>
%\fi
%
% %%%%%%%%%%%%%%%%%%%%%%%%%%%%%%%%%%%%%%
% \paragraph{Command Line Processing.}
%
% The following three command lines generate the output files
% |cdocscld|, |cdocscl1| and |cdocscl2|
% which should be identical to
% |cdocsdrf|, |cdocsch1| and |cdocsfn2|, respectively:
% \begin{center}
% \begin{tabular}{l}
% |latex -jobname cdocscld \|\\
% |  "\def\version{draft}\input{childdoc.def}\childdocforward{cdocsamp}"|\\
% |latex -jobname cdocscl1 \|\\
% |  "\input{childdoc.def}\childdocforward[cdocsamp]{cdocsch1}"|\\
% |latex -jobname cdocscl2 \|\\
% |  "\def\version{final}\input{childdoc.def}\childdocforward{cdocsch2}"|
% \end{tabular}
% \end{center}
% Note that the trailing backslash on each first line
% merely continues the input to the second line
% (for convenient cut ant paste).
% Furthermore, the command |latex| can be replaced by any
% of its alternative versions such as |pdflatex|.
%
% %%%%%%%%%%%%%%%%%%%%%%%%%%%%%%%%%%%%%%%%%%%%%%%%%%%%%%%%%%%%%%%%%%%%%%%%%%%%%%
% %%%%%%%%%%%%%%%%%%%%%%%%%%%%%%%%%%%%%%%%%%%%%%%%%%%%%%%%%%%%%%%%%%%%%%%%%%%%%%
% \section{Implementation}
%\iffalse
%<*package>
%\fi
%
% This section describes the definitions file |childdoc.def|.

% The definitions cannot be loaded using |\usepackage| or |\RequirePackage|
% which has a mechanism to prevent loading a style file more than once.
% When loading the definitions by means of |\input|
% multiple instances have to be prevented manually:
%\iffalse
%This code needs to be before the `\ProvidesFile' directive
%which is defined at the beginning of this file.
%Therefore it is also placed there and commented out here.
%</package>
%<*discard>
%\fi
%    \begin{macrocode}
\ifdefined\childdocmain\endinput\fi
%    \end{macrocode}
%\iffalse
%</discard>
%<*package>
%\fi
%
% \macro{\ifchilddoc}
% \macro{\ifchilddocmanual}
% The conditional |\ifchilddoc| tells whether a
% child (true) or main (false) document is being compiled.
% The conditional |\ifchilddocmanual| tells whether
% the |\includeonly| mechanism is used (false) or
% the selection of child files must be performed manually (true).
% The definitions initialise to false:
%    \begin{macrocode}
\newif\ifchilddoc
\newif\ifchilddocmanual
%    \end{macrocode}

% \macro{\childdocname}
% \macro{\childdocjob}
% The macro |\childdocname| stores the name of the main document
% to be compiled. The macro |\childdocjob| stores the name of
% the document on which the \LaTeX{} compiler was originally invoked.
% The content of |\jobname| cannot be compared
% to filenames specified in the source due to different catcodes.
% The following code rescans |\jobname|, stores the result
% in |\childdocname| and saves a copy in |\childdocjob|:
%    \begin{macrocode}
\edef\childdocname{\scantokens\expandafter{\jobname\noexpand}}
\let\childdocjob\childdocname
%    \end{macrocode}

% \macro{\childdocdisable}
% The macro |\childdocdisable| prevents the main file
% from being processed more than once.
% At this stage, the main document command |\childdocmain|
% is assumed to be called once again where it should do nothing.
% Any subsequent call to it should prevent
% a secondary processing of the main document
% It overwrites the forwarding commands
% |\childdocof| and |\childdocforward|
% with empty macros to prevent further inclusions of the main document:
%    \begin{macrocode}
\newcommand{\childdocdisable}
{
  \renewcommand{\childdocmain}[1]{\renewcommand{\childdocmain}[1]{\endinput}}
  \renewcommand{\childdocof}[1]{}
  \renewcommand{\childdocby}[2][]{}
  \renewcommand{\childdocforward}[2][]{}
  \renewcommand{\childdocdisable}{}
}
%    \end{macrocode}

% \macro{\childdocmain}
% The macro |\childdocmain| is to be called at the top of the main file
% with nothing or the main filename (without extension) as argument.
% First, it breaks loops.
% If the argument is not empty and does not match |\childdocname|
% (which is set by the first inclusion of |childdoc.def|),
% |\ifchilddoc| is set to true, |\includeonly| is applied to the child file
% and |\jobname| is set to the main file
% (for proper handling of |.aux| files):
%    \begin{macrocode}
\newcommand{\childdocmain}[1]
{
  \childdocdisable\childdocmain{}
  \if?#1?\else
    \begingroup
      \def\childdoctmp{#1}
      \ifx\childdoctmp\childdocname
        \def\childdoctmp{}
      \else
        \def\childdoctmp
        {
          \childdoctrue
          \includeonly{\childdocname}
          \def\childdocjob{#1}
          \def\jobname{#1}
        }
      \fi
      \expandafter
    \endgroup
    \childdoctmp
  \fi
}
%    \end{macrocode}

% \macro{\childdocof}
% The command |\childdocof| redirects
% compilation to the main file |#1|.
%    \begin{macrocode}
\newcommand{\childdocof}[1]
{
  \childdocdisable
  \childdoctrue
  \includeonly{\childdocname}
  \def\jobname{#1}
  \def\childdocjob{#1}
  \input{#1}
}
%    \end{macrocode}

% \macro{\childdocby}
% The command |\childdocby| ....
%    \begin{macrocode}
\newcommand{\childdocby}[2][]
{
  \childdocdisable
  \childdoctrue
  \childdocmanualtrue
  \if?#1?\else
    \def\jobname{#2}
  \fi
  \def\childdocjob{#2}
  \input{#2}
  \endinput
}
%    \end{macrocode}

% \macro{\childdocforward}
% The command |\childdocforward| redirects
% compilation to the main file or
% (if the optional argument is given) a child file.
% Parameters are set as if the main file
% or a child file starting with |\childdocof| was compiled.
% Then compilation is handed over to the main file:
%    \begin{macrocode}
\newcommand{\childdocforward}[2][]
{
  \begingroup
    \if?#1?
      \def\childdoctmp
      {
        \def\childdocname{#2}
        \def\childdocjob{#2}
        \def\jobname{#2}
        \input{#2}
        \endinput
      }
    \else
      \def\childdoctmp
      {
        \childdocdisable
        \def\childdocname{#2}
        \childdoctrue
        \includeonly{#2}
        \def\childdocjob{#1}
        \def\jobname{#1}
        \input{#1}
        \endinput
      }
    \fi
    \expandafter
  \endgroup
  \childdoctmp
}
%    \end{macrocode}

% \macro{\childdocforwardprefix}
% The command |\childdocforwardprefix| redirects
% compilation to the main or a child file by means of a pattern.
% The prefix |#1| in the current filename is replaced by |#2|
% and the suffix of the current filename is kept
% (it is assumed that the filename does not contain the substring `|~~~|'
% which is used as a delimiter).
% Compilation is handed over to the new file by |\childdocforward|:
%    \begin{macrocode}
\newcommand{\childdocforwardprefix}[3][]
{
  \begingroup
    \def\childdocextract #2##1~~~{\def\childdoctmp{\childdocforward[#1]{#3##1}}}
    \expandafter\childdocextract\childdocname~~~
    \expandafter
  \endgroup
  \childdoctmp
}
%    \end{macrocode}

% \macro{\childdoc}
% The deprecated macro |\childdoc| is a legacy version of |\childdocmain|:
%    \begin{macrocode}
\newcommand{\childdoc}{\childdocmain}
%    \end{macrocode}

% \macro{\childdocredirect}
% The deprecated macro |\childdocredirect| is a legacy version
% of |\childdocforward| and |\childdocforwardprefix|:
%    \begin{macrocode}
\newcommand{\childdocredirect}[2][]
{
  \begingroup
    \if?#1?
      \def\childdoctmp{\childdocforward{#2}}
    \else
      \def\childdoctmp{\childdocforwardprefix{#1}{#2}}
    \fi
    \expandafter
  \endgroup
  \childdoctmp
}
%    \end{macrocode}

%\iffalse
%</package>
%\fi
%
\endinput

\childdocforwardprefix[cdocsamp]{cdocsfn}{cdocsch}
%    \end{macrocode}

%\iffalse
%</samplefinal>
%\fi
%
% %%%%%%%%%%%%%%%%%%%%%%%%%%%%%%%%%%%%%%
% \paragraph{Command Line Processing.}
%
% The following three command lines generate the output files
% |cdocscld|, |cdocscl1| and |cdocscl2|
% which should be identical to
% |cdocsdrf|, |cdocsch1| and |cdocsfn2|, respectively:
% \begin{center}
% \begin{tabular}{l}
% |latex -jobname cdocscld \|\\
% |  "\def\version{draft}% \iffalse
%
% childdoc.dtx Copyright (C) 2017-2018 Niklas Beisert
%
% This work may be distributed and/or modified under the
% conditions of the LaTeX Project Public License, either version 1.3
% of this license or (at your option) any later version.
% The latest version of this license is in
%   http://www.latex-project.org/lppl.txt
% and version 1.3 or later is part of all distributions of LaTeX
% version 2005/12/01 or later.
%
% This work has the LPPL maintenance status `maintained'.
%
% The Current Maintainer of this work is Niklas Beisert.
%
% This work consists of the files childdoc.dtx and childdoc.ins
% and the derived files childdoc.def and cdocsamp.tex with
% cdocsch1.tex, cdocsch2.tex, cdocsdrf.tex, cdocsfn1.tex, cdocsfn2.tex.
%
%<package>\ifdefined\childdocmain\endinput\fi
%<package>\ProvidesFile{childdoc.def}[2018/12/30 v2.0 child document driver]
%<samplemain>\ProvidesFile{cdocsamp.tex}[2018/12/30 v2.0 sample for childdoc]
%<*driver>
%\ProvidesFile{childdoc.drv}[2018/12/30 v2.0 childdoc reference manual file]
\PassOptionsToClass{10pt,a4paper}{article}
\documentclass{ltxdoc}

\usepackage[margin=35mm]{geometry}
\usepackage{hyperref}
\usepackage{hyperxmp}
\usepackage[usenames]{color}

\hypersetup{colorlinks=true}
\hypersetup{pdfstartview=FitH}
\hypersetup{pdfpagemode=UseNone}
\hypersetup{pdfsource={}}
\hypersetup{pdflang={en-UK}}
\hypersetup{pdfcopyright={Copyright 2017-2018 Niklas Beisert.
  This work may be distributed and/or modified under the
  conditions of the LaTeX Project Public License, either version 1.3
  of this license or (at your option) any later version.}}
\hypersetup{pdflicenseurl={http://www.latex-project.org/lppl.txt}}
\hypersetup{pdfcontactaddress={ETH Zurich, ITP, HIT K,
  Wolfgang-Pauli-Strasse 27}}
\hypersetup{pdfcontactpostcode={8093}}
\hypersetup{pdfcontactcity={Zurich}}
\hypersetup{pdfcontactcountry={Switzerland}}
\hypersetup{pdfcontactemail={nbeisert@itp.phys.ethz.ch}}
\hypersetup{pdfcontacturl={http://people.phys.ethz.ch/\xmptilde nbeisert/}}

\newcommand{\secref}[1]{\hyperref[#1]{section \ref*{#1}}}

\parskip1ex
\parindent0pt
\let\olditemize\itemize
\def\itemize{\olditemize\parskip0pt}

\begin{document}

\title{The \textsf{childdoc} Package}
\hypersetup{pdftitle={The childdoc Package}}
\author{Niklas Beisert\\[2ex]
  Institut f\"ur Theoretische Physik\\
  Eidgen\"ossische Technische Hochschule Z\"urich\\
  Wolfgang-Pauli-Strasse 27, 8093 Z\"urich, Switzerland\\[1ex]
  \href{mailto:nbeisert@itp.phys.ethz.ch}
  {\texttt{nbeisert@itp.phys.ethz.ch}}}
\hypersetup{pdfauthor={Niklas Beisert}}
\hypersetup{pdfsubject={Manual for the LaTeX2e Package childdoc}}
\date{30 December 2018, \textsf{v2.0}}
\maketitle

\begin{abstract}\noindent
\textsf{childdoc} is a \LaTeXe{} package
that enables the direct compilation
of document sections included by |\include|
to individual files.
\end{abstract}

\begingroup
\parskip0ex
\tableofcontents
\endgroup

%%%%%%%%%%%%%%%%%%%%%%%%%%%%%%%%%%%%%%%%%%%%%%%%%%%%%%%%%%%%%%%%%%%%%%%%%%%%%%%%
%%%%%%%%%%%%%%%%%%%%%%%%%%%%%%%%%%%%%%%%%%%%%%%%%%%%%%%%%%%%%%%%%%%%%%%%%%%%%%%%
\section{Introduction}

\LaTeX{} provides a mechanism to structure a large document (such as a book)
into a main file and several child files (containing the chapters)
using the |\include| command.
This mechanism is beneficial for documents
which span hundreds of pages in order to
make the source file(s) more manageable.
Moreover, compilation can be restricted to
selected child files by means of the |\includeonly| command.
The latter feature can be used to reduce the compilation time while editing
(this was significantly more useful in the earlier days of \LaTeX{})
or to generate a smaller document which is easier to navigate.
Another application of |\includeonly| is to generate
documents consisting of selected parts of the complete document.

However, there are a few drawbacks of the plain |\include| mechanism:
\begin{itemize}
\item
The child files cannot be compiled on their own,
they can only be compiled via the main file.
A naive editing environment
(such as a text editor with an option
to have the current file processed by \LaTeX)
may require one to switch to the main file before compiling;
attempting to compile the child file produces errors.
\item
The main file must be modified (each time)
to adjust the |\includeonly| command
to the present needs. This easily leaves the main file in a messy state.
\item
The generated document will always carry the filename
of the main document. This is inconvenient if
several child files are to be compiled and
to be kept for distribution.
\end{itemize}

The present package provides a simple interface
to make child files individually compilable by \LaTeX{}.
Compiling a child file then has the same effect as compiling
the main file with an |\includeonly| command
to select the appropriate child.
Moreover the generated document will carry the name of the child
rather than the main file.
This resolves all three above issues.

This feature is meant to make the editing of books,
thesis documents and lecture notes somewhat more convenient.
However, the package can also be used efficiently for
composing a series of documents (such as exercise sheets)
which are typically distributed individually.
It then assists the author in generating the individual documents
(potentially in different versions)
as well as a document containing the collected series.
Another application is in developing style files
or other kinds of included material
where compilation of the style file could redirect
to a sample or test file.

%%%%%%%%%%%%%%%%%%%%%%%%%%%%%%%%%%%%%%%%%%%%%%%%%%%%%%%%%%%%%%%%%%%%%%%%%%%%%%%%
%%%%%%%%%%%%%%%%%%%%%%%%%%%%%%%%%%%%%%%%%%%%%%%%%%%%%%%%%%%%%%%%%%%%%%%%%%%%%%%%
\section{Usage}

First of all, the package \textsf{childdoc} is \emph{not} a standard
\LaTeXe{} |.sty| style file! Therefore it needs to be invoked in
a non-standard way.

%%%%%%%%%%%%%%%%%%%%%%%%%%%%%%%%%%%%%%%%%%%%%%%%%%%%%%%%%%%%%%%%%%%%%%%%%%%%%%%%
\subsection{Included Files}
\label{sec:include}

%%%%%%%%%%%%%%%%%%%%%%%%%%%%%%%%%%%%%%%%
\DescribeMacro{\childdocmain}
To use the package, add the commands
\begin{center}
\begin{tabular}{l}
|\input{childdoc.def}|\\
|\childdocmain{}|\\
\end{tabular}
\end{center}
at the very top of the main \LaTeX{} file,
in particular \emph{before} the |\documentclass| statement!
The argument of |\childdocmain| should be left empty
(but it must be present).

%%%%%%%%%%%%%%%%%%%%%%%%%%%%%%%%%%%%%%%%
\DescribeMacro{\childdocof}
Furthermore, add the commands
\begin{center}
\begin{tabular}{l}
|\input{childdoc.def}|\\
|\childdocof{|\textit{main}|}|\\
\end{tabular}
\end{center}
at the top of every child file \textit{child}
which is included by |\include{|\textit{child}|}|
from within the main file
(or at least for those files to be compiled individually).
The argument \textit{main} must be the filename of the main file.

There are a couple of
considerations in setting up the main and child documents:

%%%%%%%%%%%%%%%%%%%%%%%%%%%%%%%%%%%%%%%%
\paragraph{Restrictions.}

Please note the following restrictions:
\begin{itemize}
\item
|\childdocmain| must be called with one argument \textit{main}
to ensure compatibility with earlier version of the package.
It must either be empty (|\childdocmain{}|)
or precisely match the filename of the main file in which it is specified.
See \secref{sec:detection} for further information.
\item
The filename \textit{main} must be specified without the |.tex| extension.
\item
The filename \textit{main} is case sensitive
(even in case-insensitive file systems)
due to internal string comparison.
\item
The argument \textit{main} should be fully expanded, it cannot be a macro.
\item
Subdirectories and special characters should be avoided in filenames.
\item
The command |\childdocmain{|\textit{main}|}| must be followed by a whitespace.
It should not be followed immediately by another command
or by a comment mark `|%|'.
This is because the \TeX{} parser reads the token immediately following
the argument of |\childdocmain| and puts it
at the beginning of every child section;
however, a white\-space is ignored.
\end{itemize}

%%%%%%%%%%%%%%%%%%%%%%%%%%%%%%%%%%%%%%%%
\paragraph{Content of Main File.}

It is advisable to place all content in the child files included by |\include|.
Any output contained in the main file will appear in all child documents
unless suppressed manually;
it cannot be suppressed automatically by the |\includeonly| directive
and thus should normally be avoided.
A method to include some content in the main file
by means of conditional processing is described in \secref{sec:conditional}.

%%%%%%%%%%%%%%%%%%%%%%%%%%%%%%%%%%%%%%%%
\paragraph{Page Numbering.}

When only a part of the document is compiled,
the appropriate numbering of pages
(as well as other status parameters)
is determined from the |.aux| files.
The latter contain information from previous passes.
However this information needs to propagate through
all intermediate child documents.
Therefore the page numbering in child documents may well
be inconsistent until the complete document is compiled at least once.

A useful (if unconventional) way to always ensure a consistent
page numbering is to restart the numbering in each child document
and denote the pages by `\textit{child}|.|\textit{page}'
where \textit{child} represents the chapter/section number of the child file.
This can be achieved by the command
|\numberwithin{page}{|\textit{child}|}|
of the \textsf{amsmath} package
where \textit{child} can be |chapter| or |section|
depending on the chosen structuring.
Alternatively, one can modify the macro |\thepage| appropriately
and reset the counter |page| at the start of each child file.

%%%%%%%%%%%%%%%%%%%%%%%%%%%%%%%%%%%%%%%%%%%%%%%%%%%%%%%%%%%%%%%%%%%%%%%%%%%%%%%%
\subsection{Conditional Processing}
\label{sec:conditional}

The package provides a mechanism to compile different versions
of a document. To customise the versions further some conditional processing
can come in handy to distinguish which version is being compiled.
The package provides two macros to describe the compilation context:

%%%%%%%%%%%%%%%%%%%%%%%%%%%%%%%%%%%%%%%%
\DescribeMacro{\ifchilddoc}
The conditional |\ifchilddoc| distinguishes between the compilation of
child documents and the main document:
%
\begin{center}
|\ifchilddoc |\textit{child-code}| |[|\||else |\textit{main-code}]| \||fi|
\end{center}

%%%%%%%%%%%%%%%%%%%%%%%%%%%%%%%%%%%%%%%%
\DescribeMacro{\childdocname}
\DescribeMacro{\childdocjob}
The macro |\childdocname| contains the filename (without extension)
of the main or child file being processed.
Note that |\childdocjob| will always contain the name of the main file.

%%%%%%%%%%%%%%%%%%%%%%%%%%%%%%%%%%%%%%%%
\paragraph{Title Page.}

Conditional processing can be used to include a title or banner page
in the main document when proper precautions are taken.
Importantly, the code in the main file should ensure that the page counter
(as well as other status parameters which are stored in the |.aux| files)
takes the same value after the conditional processing.
Otherwise the page numbers may take divergent values
depending on which part is compiled.

For example, a title page could be declared by:
%
\begin{center}
\begin{tabular}{l}
|\ifchilddoc\||else|\\
|\addtocounter{page}{-1}|\\
\textit{code for title page}\\
|\newpage|\\
|\||fi|
\end{tabular}
\end{center}
%
A banner page for the child documents can be generated by:
%
\begin{center}
\begin{tabular}{l}
|\ifchilddoc|\\
|\addtocounter{page}{-1}|\\
\textit{code for banner page}\\
|\newpage|\\
|\||fi|
\end{tabular}
\end{center}
%
Here one could write a message such as:
\begin{center}
|This is the part \childdocname{} of \childdocjob{}.|
\end{center}

%%%%%%%%%%%%%%%%%%%%%%%%%%%%%%%%%%%%%%%%%%%%%%%%%%%%%%%%%%%%%%%%%%%%%%%%%%%%%%%%
\subsection{Flags}
\label{sec:flags}

The package makes it easy to generate different versions
of the main or child documents.
To this end compilation flags can be defined
and assigned different default values.
They will be particularly useful in conjunction
with the forwarding mechanism described in \secref{sec:forward}.

For example, it may be useful to have a flag |\version|
which can be set to |draft| or |final|.
The document source will contain some conditional code
depending on the value of |\version|.
Suppose further, the flag should default to |final| for the main file
and to |draft| for child files
which is a natural assignment for editing the document.
This is achieved by placing the following code
in the preamble of the main document
(below the |\childdocmain| directive):
%
\begin{center}
\begin{tabular}{l}
|\ifchilddoc|\\
|\providecommand{\version}{draft}|\\
|\||else|\\
|\providecommand{\version}{final}|\\
|\||fi|
\end{tabular}
\end{center}
%
The definition by |\providecommand| makes sure
that previous definitions are not overwritten.
Further statements |\providecommand{\version}{...}|
can thus be added before the above code to override it.

For the main file, one might add a line
(between |\childdocmain| and the above block)
%
\begin{center}
|%\ifchilddoc\||else\providecommand{\version}{draft}\||fi|
\end{center}
%
which can be uncommented to produce a draft version.
Likewise one can add a line to the very top of a child file
(above the |\childdocof{|\textit{main}|}| directive)
%
\begin{center}
|%\providecommand{\version}{final}|
\end{center}
%
which can be uncommented to produce the final version of this child document.

%%%%%%%%%%%%%%%%%%%%%%%%%%%%%%%%%%%%%%%%%%%%%%%%%%%%%%%%%%%%%%%%%%%%%%%%%%%%%%%%
\subsection{Forwarding}
\label{sec:forward}

Different versions of the main or child documents
using compilation flags as described in \secref{sec:flags}
can be (permanently) stored in different files
for convenient compilation, viewing and distribution.
To this end, the package defines a command
to pass on compilation to a different file:

%%%%%%%%%%%%%%%%%%%%%%%%%%%%%%%%%%%%%%%%
\DescribeMacro{\childdocforward}
The command |\childdocforward| redirects processing to
another source file:
%
\begin{center}
\begin{tabular}{l}
|\input{childdoc.def}|\\
|\childdocforward[|\textit{main}|]{|\textit{dest}|}|\\
\end{tabular}
\end{center}
%
The argument \textit{dest} is the destination file
(without extension).
It should be the main file or one of the child files.
Note that further \textsf{childdoc} directives
such as |\childdocof| and |\childdocforward|
in the indicated file will be processed in this form.
The optional argument \textit{main}
passes on directly to the main file \textit{main}
while pretending to compile the child \textit{dest}.
This form behaves as if \textit{dest}
issues |\childdocof{|\textit{main}|}| right away,
and no further \textsf{childdoc} directives will be processed.

%%%%%%%%%%%%%%%%%%%%%%%%%%%%%%%%%%%%%%%%
\DescribeMacro{\...prefix}
In the alternative form |\childdocforwardprefix|,
%
\begin{center}
\begin{tabular}{l}
|\input{childdoc.def}|\\
|\childdocforwardprefix[|\textit{main}|]{|\textit{prefix}|}{|\textit{dest}|}|
\end{tabular}
\end{center}
%
the destination file is determined by a pattern
depending on the current file:
To make this work, the current file must be called
`{\textit{prefix}\hspace{0.2em}\textit{suffix}}'
with \textit{prefix} matching precisely the argument.
Processing is then passed on to the file
`{\textit{dest}\hspace{0.2em}\textit{suffix}}'.
Surely, the same effect is achieved by
directly specifying the
argument `{\textit{dest}\hspace{0.2em}\textit{suffix}}'
in the first form.
However, that requires to set up a different file
for each child. With the alternative form of the command
all these files can have exactly the same content
which simplifies setting them up and maintaining them.

For example, the following file |draft.tex|
with a compilation flag |\version| as described in \secref{sec:flags}
compiles the main document as a draft:
%
\begin{center}
\begin{tabular}{l}
|\def\version{draft}|\\
|\input{childdoc.def}|\\
|\childdocforward{|\textit{main}|}|
\end{tabular}
\end{center}
%
Likewise, the following files |final|\textit{nn}|.tex|
compile the final version of the child document
|child|\textit{nn}|.tex|:
%
\begin{center}
\begin{tabular}{l}
|\def\version{final}|\\
|\input{childdoc.def}|\\
|\childdocforwardprefix{final}{child}|
\end{tabular}
\end{center}
%

Note that when several versions of a main file and/or of each child file
are to be generated, it may be convenient to set up a |Makefile| or
shell script to automatise the process.

%%%%%%%%%%%%%%%%%%%%%%%%%%%%%%%%%%%%%%%%%%%%%%%%%%%%%%%%%%%%%%%%%%%%%%%%%%%%%%%%
\subsection{Command Line Processing}
\label{sec:commandline}

The effect of redirection files can also be achieved by invoking
the \LaTeX{} compiler with a more elaborate command line.
Most conveniently this should be done as part
of a shell script or a |Makefile|.

When using \textsf{childdoc} in the main file, the following
command lines effectively perform a redirection
(note that depending on the shell being used,
backslashes may have to be doubled: `|\|' $\to$ `|\\|'):
%
\begin{center}
|... -jobname "|\textit{target}|" |\\|"|[\textit{flags}]%
|\input{childdoc.def}\childdocforward[|\textit{main}|]{|\textit{dest}|}"|
\end{center}
%
Here \textit{target} is the name of the output file,
\textit{main} is the name of the main file
and \textit{dest} is the name of the main or child file to be processed
(all filenames without extensions).
The optional argument \textit{main} can be omitted
if \textit{main} matches \textit{dest}.
Optionally, compilation \textit{flags} can be defined via |\def| commands.
This command line makes the \TeX{} engine believe
it is compiling the file \textit{target}
whose content is specified as the latter parameter.
The provided code then forwards the processing to
\textit{main} or \textit{dest} as described in \secref{sec:forward}.

%%%%%%%%%%%%%%%%%%%%%%%%%%%%%%%%%%%%%%%%%%%%%%%%%%%%%%%%%%%%%%%%%%%%%%%%%%%%%%%%
\subsection{Include by Input}
\label{sec:input}

Including child documents by |\include| has some restrictions by design.
Most notably, the content of a child document always occupies
its own set of pages; pages cannot be shared between child documents.
Usually, this behaviour makes perfect sense
because each child document contain an essential part of the document.
However, in some situations it may be desirable to compose
a document from a collection of parts
without having mandatory page breaks between then.
For this case, the package
provides a mechanism to include parts
by |\input| which can also be processed individually.
However, by construction this mechanism
requires manual handling of the content to be output.

%%%%%%%%%%%%%%%%%%%%%%%%%%%%%%%%%%%%%%%%
\DescribeMacro{\ifchilddocmanual}
The main file should be prepared as usual, see \secref{sec:include}.
However, the document body must make a distinction
between processing of an individual part and of the main document, e.g.:
%
\begin{center}
\begin{tabular}{l}
|\ifchilddocmanual|\\
|\input{\childdocname}|\\
|\||else|\\
\textit{document body with }|\input{|\textit{part}|}|\\
|\||fi|
\end{tabular}
\end{center}
%
The conditional |\ifchilddocmanual| is true whenever
a part to be included by |\input| is being compiled,
and the name of the part is stored in |\childdocname|.

%%%%%%%%%%%%%%%%%%%%%%%%%%%%%%%%%%%%%%%%
\DescribeMacro{\childdocby}
Each part to be included by |\input| should start with:
%
\begin{center}
\begin{tabular}{l}
|\input{childdoc.def}|\\
|\childdocby{|\textit{main}|}|\\
\end{tabular}
\end{center}
%
The directive |\childdocby| is similar to |\childdocof|
described in \secref{sec:include},
but the subsequent selection of content must be done manually.
To that end, both |\ifchilddoc| and |\ifchilddocmanual|
will be true upon processing of a part,
and the name of the part is stored in |\childdocname|.
Note that |\jobname| will be set to the filename of the current part
so that each part receives an individual |.aux| file
that does not interfere with the |.aux| file(s) of the main document.
This behaviour can be altered by the alternative form
|\childdocby[*]{|\textit{main}|}| (with a non-empty optional argument)
which uses the |.aux| file of the main document
by setting |\jobname| to \textit{main}.

%%%%%%%%%%%%%%%%%%%%%%%%%%%%%%%%%%%%%%%%%%%%%%%%%%%%%%%%%%%%%%%%%%%%%%%%%%%%%%%%
\subsection{Driver Development}
\label{sec:driver}

The \textsf{childdoc} mechanism can also be use for the development
of definition files such as \LaTeX{} styles or classes.
This case differs from the above setup with multiple parts
included by |\include| in that no |\includeonly| should be invoked.
This can be achieved by starting the include file
(before |\ProvidesPackage|) with:
%
\begin{center}
\begin{tabular}{l}
|\input{childdoc.def}|\\
|\childdocforward{|\textit{main}|}|\\
\end{tabular}
\end{center}
%
or alternatively with:
%
\begin{center}
\begin{tabular}{l}
|\input{childdoc.def}|\\
|\childdocby{|\textit{main}|}|\\
\end{tabular}
\end{center}
%
Both forms have slightly different effects as described above.
The main file is prepared as usual, see \secref{sec:include}.

%%%%%%%%%%%%%%%%%%%%%%%%%%%%%%%%%%%%%%%%%%%%%%%%%%%%%%%%%%%%%%%%%%%%%%%%%%%%%%%%
\subsection{Legacy Detection}
\label{sec:detection}

The directive |\childdocmain| in the main file can detect
whether the complete document or merely a child is to be compiled
even without using the directive |\childdocof|.
This method is deprecated because it is less robust
and there is no compelling reason to use it;
it is merely provided for backward compatibility
and it may be removed in future versions.

If the detection mechanism is to be used,
it is mandatory to correctly specify
the filename of the main file as the argument of |\childdocmain|:
%
\begin{center}
\begin{tabular}{l}
|\input{childdoc.def}|\\
|\childdocmain{|\textit{main}|}|\\
\end{tabular}
\end{center}
%
If |\jobname| does not match the argument \textit{main} of |\childdocmain|,
it is assumed that |\jobname| points to the child file to be compiled.
When using |\childdocmain| with the main file specified as argument,
it suffices to start a child file
with just |\input{|\textit{main}|}|
without loading of the package and using |\childdocof|.
If instead all processing is done
with the appropriate \textsf{childdoc} directives,
the argument of \textit{main} of |\childdocmain| can be empty.

An alternative version of the command line processing described
in \secref{sec:commandline} using the detection mechanism reads:
%
\begin{center}
|... -jobname "|\textit{target}|" "|[\textit{flags}]%
[|\def\jobname{|\textit{dest}|}|]|\input{|\textit{main}|}"|
\end{center}

%%%%%%%%%%%%%%%%%%%%%%%%%%%%%%%%%%%%%%%%%%%%%%%%%%%%%%%%%%%%%%%%%%%%%%%%%%%%%%%%
\subsection{Manual Code}
\label{sec:manual}

In case one cannot be certain whether the definitions file |childdoc.def|
is installed on the target \TeX{} distribution
and one prefers not to ship it,
it is conceivable to paste a few relevant commands into the sources.

To that end, drop all statements |\input{childdoc.def}|
and perform the replacements as outlined below.
Instead of |\childdocmain{|\textit{main}|}| add the following code
to the top of the main file:
%
\begin{center}
\begin{tabular}{l}
|\||ifdefined\childdocname\endinput\||fi\newif\ifchilddoc|\\
|\edef\childdocname{\scantokens\expandafter{\jobname\noexpand}}|\\
|\def\childdocmain{|\textit{main}|}\||ifx\childdocmain\childdocname\||else|\\
|\childdoctrue\includeonly{\childdocname}\let\jobname\childdocmain\||fi|\\
\end{tabular}
\end{center}
%
Instead of |\childdocof{|\textit{main}|}| just include the main file
at the top of each child file:
%
\begin{center}
|\input{|\textit{main}|}|
\end{center}
%
A simple redirection |\childdocforward{|\textit{dest}|}| is achieved by:
%
\begin{center}
|\def\jobname{|\textit{dest}|}\input{\jobname}|
\end{center}
%
The redirection with prefix
|\childdocforwardprefix[|\textit{prefix}|]{|\textit{dest}|}|
is accomplished by:
%
\begin{center}
\begin{tabular}{l}
|{\edef\jobname{\scantokens\expandafter{\jobname\noexpand}}|\\
|\def\redirectjob |\textit{prefix}|#1~~~{\gdef\jobname{|\textit{dest}|#1}}|\\
|\expandafter\redirectjob\jobname~~~}\input{\jobname}|
\end{tabular}
\end{center}

In an alternative approach,
child documents can be compiled by a specific command line
without additional code or specific definitions:
%
\begin{center}
|... -jobname "|\textit{target}|" "|[\textit{flags}]%
|\includeonly{|\textit{dest}|}\input{|\textit{main}|}"|
\end{center}
%

%%%%%%%%%%%%%%%%%%%%%%%%%%%%%%%%%%%%%%%%%%%%%%%%%%%%%%%%%%%%%%%%%%%%%%%%%%%%%%%%
%%%%%%%%%%%%%%%%%%%%%%%%%%%%%%%%%%%%%%%%%%%%%%%%%%%%%%%%%%%%%%%%%%%%%%%%%%%%%%%%
\section{Information}

%%%%%%%%%%%%%%%%%%%%%%%%%%%%%%%%%%%%%%%%%%%%%%%%%%%%%%%%%%%%%%%%%%%%%%%%%%%%%%%%
\subsection{Copyright}

Copyright \copyright{} 2017--2018 Niklas Beisert

This work may be distributed and/or modified under the
conditions of the \LaTeX{} Project Public License, either version 1.3
of this license or (at your option) any later version.
The latest version of this license is in
  \url{http://www.latex-project.org/lppl.txt}
and version 1.3 or later is part of all distributions of \LaTeX{}
version 2005/12/01 or later.

This work has the LPPL maintenance status `maintained'.

The Current Maintainer of this work is Niklas Beisert.

This work consists of the files |README.txt|, |childdoc.ins| and |childdoc.dtx|
as well as the derived files |childdoc.def|, |cdocsamp.tex|
with |cdocsch1.tex|, |cdocsch2.tex|, |cdocspt3.tex|, |cdocspt4.tex|,
|cdocsdrf.tex|, |cdocsfn1.tex|, |cdocsfn2.tex|
as well as |childdoc.pdf|.

%%%%%%%%%%%%%%%%%%%%%%%%%%%%%%%%%%%%%%%%%%%%%%%%%%%%%%%%%%%%%%%%%%%%%%%%%%%%%%%%
\subsection{Files and Installation}

The package consists of the files:
%
\begin{center}
\begin{tabular}{ll}
    |README.txt|   & readme file \\
    |childdoc.ins| & installation file \\
    |childdoc.dtx| & source file \\
    |childdoc.def| & definition file \\
    |cdocsamp.tex| & sample main file \\
    |cdocsch1.tex| & sample include file \\
    |cdocsch2.tex| & sample include file \\
    |cdocspt3.tex| & sample part file \\
    |cdocspt4.tex| & sample part file \\
    |cdocsdrf.tex| & sample redirection file \\
    |cdocsfn1.tex| & sample redirection file \\
    |cdocsfn2.tex| & sample redirection file \\
    |childdoc.pdf| & manual
\end{tabular}
\end{center}
%
The distribution consists of the files
|README.txt|, |childdoc.ins| and |childdoc.dtx|.
%
\begin{itemize}
\item
Run (pdf)\LaTeX{} on |childdoc.dtx|
to compile the manual |childdoc.pdf| (this file).
\item
Run \LaTeX{} on |childdoc.ins| to create the definitions file |childdoc.def|
and the sample |cdocsamp.tex| with include files
|cdocsch1.tex|, |cdocsch2.tex|, |cdocspt3.tex|, |cdocspt4.tex|,
|cdocsdrf.tex|, |cdocsfn1.tex|, |cdocsfn2.tex|.
Then copy the file |childdoc.def| to an appropriate directory of your \LaTeX{}
distribution, e.g.\ \textit{texmf-root}|/tex/latex/childdoc|.
\end{itemize}

%%%%%%%%%%%%%%%%%%%%%%%%%%%%%%%%%%%%%%%%%%%%%%%%%%%%%%%%%%%%%%%%%%%%%%%%%%%%%%%%
\subsection{Related CTAN Packages}

There are several other packages which offer a similar functionality:
%
\begin{itemize}
\item
The packages
\href{http://ctan.org/pkg/docmute}{\textsf{docmute}},
\href{http://ctan.org/pkg/includex}{\textsf{includex}} and
\href{http://ctan.org/pkg/standalone}{\textsf{standalone}}
provide commands to include only the document body of
a child file thus allowing both files to be compiled individually.
\item
The packages \href{http://ctan.org/pkg/subdocs}{\textsf{subdocs}}
and \href{http://ctan.org/pkg/subfiles}{\textsf{subfiles}}
provide structures in which the main and child documents can be
encapsulated and allowing them to be compiled individually.
The inclusion mechanism is different from the conventional |\include|.
\item
The package \href{http://ctan.org/pkg/combine}{\textsf{combine}}
is an elaborate solution to combine several documents into one.
\end{itemize}
%
See also the CTAN topic \href{http://ctan.org/topic/subdocs}{\textsf{subdocs}}
for further related packages.
The present package differs from the above solutions in that
a document structure constructed with the conventional |\include| mechanism
just needs two extra commands at the top of every file
such that all constituent files can be compiled individually.

%%%%%%%%%%%%%%%%%%%%%%%%%%%%%%%%%%%%%%%%%%%%%%%%%%%%%%%%%%%%%%%%%%%%%%%%%%%%%%%%
%\subsection{Feature Suggestions}
%
%The following is a list of features which may be useful for future
%versions of this package:
%%
%\begin{itemize}
%\item
%\ldots
%\end{itemize}

%%%%%%%%%%%%%%%%%%%%%%%%%%%%%%%%%%%%%%%%%%%%%%%%%%%%%%%%%%%%%%%%%%%%%%%%%%%%%%%%
\subsection{Revision History}

%%%%%%%%%%%%%%%%%%%%%%%%%%%%%%%%%%%%%%%%
\paragraph{v2.0:} 2018/12/30

\begin{itemize}
\item
immediate forward processing
\item
added |\childdocby| mechanism
\item
manual restructured
\end{itemize}

%%%%%%%%%%%%%%%%%%%%%%%%%%%%%%%%%%%%%%%%
\paragraph{v1.6:} 2018/01/17

\begin{itemize}
\item
application for development of include files
\item
corrections to manual
\end{itemize}

%%%%%%%%%%%%%%%%%%%%%%%%%%%%%%%%%%%%%%%%
\paragraph{v1.5:} 2017/05/21

\begin{itemize}
\item
more complete structuring introduced
\item
|\childdocof| introduced
\item
|\childdoc| renamed to |\childdocmain|
\item
|\childredirect| renamed to |\childdocforward| and |\childdocforwardprefix|
and functionality expanded
\end{itemize}

%%%%%%%%%%%%%%%%%%%%%%%%%%%%%%%%%%%%%%%%
\paragraph{v1.0:} 2017/04/27

\begin{itemize}
\item
manual and install package
\item
first version published on CTAN
\end{itemize}

%%%%%%%%%%%%%%%%%%%%%%%%%%%%%%%%%%%%%%%%
\paragraph{v0.6:} 2017/04/26

\begin{itemize}
\item
redirection mechanism added
\end{itemize}

%%%%%%%%%%%%%%%%%%%%%%%%%%%%%%%%%%%%%%%%
\paragraph{v0.5:} 2017/04/26

\begin{itemize}
\item
functionality in definition file
\end{itemize}


%%%%%%%%%%%%%%%%%%%%%%%%%%%%%%%%%%%%%%%%%%%%%%%%%%%%%%%%%%%%%%%%%%%%%%%%%%%%%%%%
%%%%%%%%%%%%%%%%%%%%%%%%%%%%%%%%%%%%%%%%%%%%%%%%%%%%%%%%%%%%%%%%%%%%%%%%%%%%%%%%
%%%%%%%%%%%%%%%%%%%%%%%%%%%%%%%%%%%%%%%%%%%%%%%%%%%%%%%%%%%%%%%%%%%%%%%%%%%%%%%%
\appendix

\settowidth\MacroIndent{\rmfamily\scriptsize 000\ }

 \DocInput{childdoc.dtx}

\end{document}
%</driver>
% \fi
%
% %%%%%%%%%%%%%%%%%%%%%%%%%%%%%%%%%%%%%%%%%%%%%%%%%%%%%%%%%%%%%%%%%%%%%%%%%%%%%%
% %%%%%%%%%%%%%%%%%%%%%%%%%%%%%%%%%%%%%%%%%%%%%%%%%%%%%%%%%%%%%%%%%%%%%%%%%%%%%%
% \section{Sample}
%\iffalse
%<*samplemain>
%\fi
%
% The following presents a sample document
% with two chapters, two parts, a title page,
% a compile flag as well as three forwarding files to set the flag.
% It consists of eight |.tex| files:
% \begin{center}
% \begin{tabular}{ll}
% |cdocsamp.tex|&main file\\
% |cdocsch1.tex|&include file for chapter 1\\
% |cdocsch2.tex|&include file for chapter 2\\
% |cdocspt3.tex|&include file for part 3\\
% |cdocspt4.tex|&include file for part 4\\
% |cdocsdrf.tex|&forwarding file for main file in draft mode\\
% |cdocsfi1.tex|&forwarding file for final version of chapter 1\\
% |cdocsfi2.tex|&forwarding file for final version of chapter 2\\
% \end{tabular}
% \end{center}
% Each of the eight files can be compiled directly by the \LaTeX{} compiler.
%
% %%%%%%%%%%%%%%%%%%%%%%%%%%%%%%%%%%%%%%
% \paragraph{Main File.}
%
% The main file is called |cdocsamp.tex|.
%
% Load the \textsf{childdoc} definitions and
% declare the filename for the main document:
%    \begin{macrocode}
\input{childdoc.def}
\childdocmain{}
%    \end{macrocode}

% Optional override for |\version| flag:
%    \begin{macrocode}
%%\ifchilddoc\else\providecommand{\version}{draft}\fi
%    \end{macrocode}

% Define the default values for the |\version| flag
% (|final| for the main file and |draft| for childs):
%    \begin{macrocode}
\ifchilddoc
\providecommand{\version}{draft}
\else
\providecommand{\version}{final}
\fi
%    \end{macrocode}

% Load the standard document class:
%    \begin{macrocode}
\documentclass[12pt]{article}
%    \end{macrocode}

% Start the document body:
%    \begin{macrocode}
\begin{document}
%    \end{macrocode}

% Declare a title page.
% Print title, part of document being processed and version flag:
%    \begin{macrocode}
\addtocounter{page}{-1}
\begin{center}
{\LARGE\bfseries{}childdoc example\par}
\vspace{1cm}
\ifchilddoc
\ifchilddocmanual part\else chapter\fi:
`\childdocname' of `\childdocjob'\par
\else
main document: `\childdocjob'\par
\fi
version: \version\par
\end{center}
\newpage
%    \end{macrocode}

% Manually include selected file,
% otherwise process as usual:
%    \begin{macrocode}
\ifchilddocmanual
\section*{part `\childdocname'}
\input{\childdocname}
\else
%    \end{macrocode}

% Include the two chapters:
%    \begin{macrocode}
\include{cdocsch1}
\include{cdocsch2}
%    \end{macrocode}

% Include the two parts unless only chapters should be displayed:
%    \begin{macrocode}
\ifchilddoc\else
\section{part three}
\input{cdocspt3}
\section{part four}
\input{cdocspt4}
\fi
%    \end{macrocode}

% Process as usual until here:
%    \begin{macrocode}
\fi
%    \end{macrocode}

% End of document body:
%    \begin{macrocode}
\end{document}
%    \end{macrocode}
%\iffalse
%</samplemain>
%\fi
%
% %%%%%%%%%%%%%%%%%%%%%%%%%%%%%%%%%%%%%%
% \paragraph{Chapter Include Files.}
%
% The include files are called |cdocsch1.tex| and |cdocsch2.tex|.
%
%\iffalse
%<*samplechap1|samplechap2>
%\fi

% Optional override for |\version| flag:
%    \begin{macrocode}
%%\providecommand{\version}{final}
%    \end{macrocode}

% Include the main document:
%    \begin{macrocode}
\input{childdoc.def}
\childdocof{cdocsamp}
%    \end{macrocode}

%\iffalse
%</samplechap1|samplechap2>
%\fi
%
%\iffalse
%<*samplechap1>
%\fi
% Some text for chapter 1:
%    \begin{macrocode}
\section{one}
some text in chapter one
%    \end{macrocode}

%\iffalse
%</samplechap1>
%\fi
% Some text for chapter 2:
%\iffalse
%<*samplechap2>
%\fi
%    \begin{macrocode}
\section{two}
more text in chapter two
%    \end{macrocode}

%\iffalse
%</samplechap2>
%\fi
%
% %%%%%%%%%%%%%%%%%%%%%%%%%%%%%%%%%%%%%%
% \paragraph{Part Include Files.}
%
% The include files are called |cdocspt3.tex| and |cdocspt4.tex|.
%
%\iffalse
%<*samplepart3|samplepart4>
%\fi

% Optional override for |\version| flag:
%    \begin{macrocode}
%%\providecommand{\version}{final}
%    \end{macrocode}

% Include the main document:
%    \begin{macrocode}
\input{childdoc.def}
\childdocby{cdocsamp}
%    \end{macrocode}

%\iffalse
%</samplepart3|samplepart4>
%\fi
%
%\iffalse
%<*samplepart3>
%\fi
% Some text for part 3:
%    \begin{macrocode}
some text in part three
%    \end{macrocode}

%\iffalse
%</samplepart3>
%\fi
% Some text for part 4:
%\iffalse
%<*samplepart4>
%\fi
%    \begin{macrocode}
more text in part four
%    \end{macrocode}

%\iffalse
%</samplepart4>
%\fi
%
% %%%%%%%%%%%%%%%%%%%%%%%%%%%%%%%%%%%%%%
% \paragraph{Forwarding for a Complete Draft.}
%
% The following forwarding file |cdocsdrf.tex|
% compiles the main document in draft mode:
%\iffalse
%<*sampledraft>
%\fi
%    \begin{macrocode}
\def\version{draft}
\input{childdoc.def}
\childdocforward{cdocsamp}
%    \end{macrocode}

%\iffalse
%</sampledraft>
%\fi
%
% %%%%%%%%%%%%%%%%%%%%%%%%%%%%%%%%%%%%%%
% \paragraph{Forwarding for Final Version of the Chapters.}
%
% The following forwarding files |cdocsfn1.tex| and |cdocsfn2.tex|
% (with identical content)
% compile the final versions of the child documents
% |cdocsch1.tex| and |cdocsch2.tex|, respectively:
%\iffalse
%<*samplefinal>
%\fi
%    \begin{macrocode}
\def\version{final}
\input{childdoc.def}
\childdocforwardprefix[cdocsamp]{cdocsfn}{cdocsch}
%    \end{macrocode}

%\iffalse
%</samplefinal>
%\fi
%
% %%%%%%%%%%%%%%%%%%%%%%%%%%%%%%%%%%%%%%
% \paragraph{Command Line Processing.}
%
% The following three command lines generate the output files
% |cdocscld|, |cdocscl1| and |cdocscl2|
% which should be identical to
% |cdocsdrf|, |cdocsch1| and |cdocsfn2|, respectively:
% \begin{center}
% \begin{tabular}{l}
% |latex -jobname cdocscld \|\\
% |  "\def\version{draft}\input{childdoc.def}\childdocforward{cdocsamp}"|\\
% |latex -jobname cdocscl1 \|\\
% |  "\input{childdoc.def}\childdocforward[cdocsamp]{cdocsch1}"|\\
% |latex -jobname cdocscl2 \|\\
% |  "\def\version{final}\input{childdoc.def}\childdocforward{cdocsch2}"|
% \end{tabular}
% \end{center}
% Note that the trailing backslash on each first line
% merely continues the input to the second line
% (for convenient cut ant paste).
% Furthermore, the command |latex| can be replaced by any
% of its alternative versions such as |pdflatex|.
%
% %%%%%%%%%%%%%%%%%%%%%%%%%%%%%%%%%%%%%%%%%%%%%%%%%%%%%%%%%%%%%%%%%%%%%%%%%%%%%%
% %%%%%%%%%%%%%%%%%%%%%%%%%%%%%%%%%%%%%%%%%%%%%%%%%%%%%%%%%%%%%%%%%%%%%%%%%%%%%%
% \section{Implementation}
%\iffalse
%<*package>
%\fi
%
% This section describes the definitions file |childdoc.def|.

% The definitions cannot be loaded using |\usepackage| or |\RequirePackage|
% which has a mechanism to prevent loading a style file more than once.
% When loading the definitions by means of |\input|
% multiple instances have to be prevented manually:
%\iffalse
%This code needs to be before the `\ProvidesFile' directive
%which is defined at the beginning of this file.
%Therefore it is also placed there and commented out here.
%</package>
%<*discard>
%\fi
%    \begin{macrocode}
\ifdefined\childdocmain\endinput\fi
%    \end{macrocode}
%\iffalse
%</discard>
%<*package>
%\fi
%
% \macro{\ifchilddoc}
% \macro{\ifchilddocmanual}
% The conditional |\ifchilddoc| tells whether a
% child (true) or main (false) document is being compiled.
% The conditional |\ifchilddocmanual| tells whether
% the |\includeonly| mechanism is used (false) or
% the selection of child files must be performed manually (true).
% The definitions initialise to false:
%    \begin{macrocode}
\newif\ifchilddoc
\newif\ifchilddocmanual
%    \end{macrocode}

% \macro{\childdocname}
% \macro{\childdocjob}
% The macro |\childdocname| stores the name of the main document
% to be compiled. The macro |\childdocjob| stores the name of
% the document on which the \LaTeX{} compiler was originally invoked.
% The content of |\jobname| cannot be compared
% to filenames specified in the source due to different catcodes.
% The following code rescans |\jobname|, stores the result
% in |\childdocname| and saves a copy in |\childdocjob|:
%    \begin{macrocode}
\edef\childdocname{\scantokens\expandafter{\jobname\noexpand}}
\let\childdocjob\childdocname
%    \end{macrocode}

% \macro{\childdocdisable}
% The macro |\childdocdisable| prevents the main file
% from being processed more than once.
% At this stage, the main document command |\childdocmain|
% is assumed to be called once again where it should do nothing.
% Any subsequent call to it should prevent
% a secondary processing of the main document
% It overwrites the forwarding commands
% |\childdocof| and |\childdocforward|
% with empty macros to prevent further inclusions of the main document:
%    \begin{macrocode}
\newcommand{\childdocdisable}
{
  \renewcommand{\childdocmain}[1]{\renewcommand{\childdocmain}[1]{\endinput}}
  \renewcommand{\childdocof}[1]{}
  \renewcommand{\childdocby}[2][]{}
  \renewcommand{\childdocforward}[2][]{}
  \renewcommand{\childdocdisable}{}
}
%    \end{macrocode}

% \macro{\childdocmain}
% The macro |\childdocmain| is to be called at the top of the main file
% with nothing or the main filename (without extension) as argument.
% First, it breaks loops.
% If the argument is not empty and does not match |\childdocname|
% (which is set by the first inclusion of |childdoc.def|),
% |\ifchilddoc| is set to true, |\includeonly| is applied to the child file
% and |\jobname| is set to the main file
% (for proper handling of |.aux| files):
%    \begin{macrocode}
\newcommand{\childdocmain}[1]
{
  \childdocdisable\childdocmain{}
  \if?#1?\else
    \begingroup
      \def\childdoctmp{#1}
      \ifx\childdoctmp\childdocname
        \def\childdoctmp{}
      \else
        \def\childdoctmp
        {
          \childdoctrue
          \includeonly{\childdocname}
          \def\childdocjob{#1}
          \def\jobname{#1}
        }
      \fi
      \expandafter
    \endgroup
    \childdoctmp
  \fi
}
%    \end{macrocode}

% \macro{\childdocof}
% The command |\childdocof| redirects
% compilation to the main file |#1|.
%    \begin{macrocode}
\newcommand{\childdocof}[1]
{
  \childdocdisable
  \childdoctrue
  \includeonly{\childdocname}
  \def\jobname{#1}
  \def\childdocjob{#1}
  \input{#1}
}
%    \end{macrocode}

% \macro{\childdocby}
% The command |\childdocby| ....
%    \begin{macrocode}
\newcommand{\childdocby}[2][]
{
  \childdocdisable
  \childdoctrue
  \childdocmanualtrue
  \if?#1?\else
    \def\jobname{#2}
  \fi
  \def\childdocjob{#2}
  \input{#2}
  \endinput
}
%    \end{macrocode}

% \macro{\childdocforward}
% The command |\childdocforward| redirects
% compilation to the main file or
% (if the optional argument is given) a child file.
% Parameters are set as if the main file
% or a child file starting with |\childdocof| was compiled.
% Then compilation is handed over to the main file:
%    \begin{macrocode}
\newcommand{\childdocforward}[2][]
{
  \begingroup
    \if?#1?
      \def\childdoctmp
      {
        \def\childdocname{#2}
        \def\childdocjob{#2}
        \def\jobname{#2}
        \input{#2}
        \endinput
      }
    \else
      \def\childdoctmp
      {
        \childdocdisable
        \def\childdocname{#2}
        \childdoctrue
        \includeonly{#2}
        \def\childdocjob{#1}
        \def\jobname{#1}
        \input{#1}
        \endinput
      }
    \fi
    \expandafter
  \endgroup
  \childdoctmp
}
%    \end{macrocode}

% \macro{\childdocforwardprefix}
% The command |\childdocforwardprefix| redirects
% compilation to the main or a child file by means of a pattern.
% The prefix |#1| in the current filename is replaced by |#2|
% and the suffix of the current filename is kept
% (it is assumed that the filename does not contain the substring `|~~~|'
% which is used as a delimiter).
% Compilation is handed over to the new file by |\childdocforward|:
%    \begin{macrocode}
\newcommand{\childdocforwardprefix}[3][]
{
  \begingroup
    \def\childdocextract #2##1~~~{\def\childdoctmp{\childdocforward[#1]{#3##1}}}
    \expandafter\childdocextract\childdocname~~~
    \expandafter
  \endgroup
  \childdoctmp
}
%    \end{macrocode}

% \macro{\childdoc}
% The deprecated macro |\childdoc| is a legacy version of |\childdocmain|:
%    \begin{macrocode}
\newcommand{\childdoc}{\childdocmain}
%    \end{macrocode}

% \macro{\childdocredirect}
% The deprecated macro |\childdocredirect| is a legacy version
% of |\childdocforward| and |\childdocforwardprefix|:
%    \begin{macrocode}
\newcommand{\childdocredirect}[2][]
{
  \begingroup
    \if?#1?
      \def\childdoctmp{\childdocforward{#2}}
    \else
      \def\childdoctmp{\childdocforwardprefix{#1}{#2}}
    \fi
    \expandafter
  \endgroup
  \childdoctmp
}
%    \end{macrocode}

%\iffalse
%</package>
%\fi
%
\endinput
\childdocforward{cdocsamp}"|\\
% |latex -jobname cdocscl1 \|\\
% |  "% \iffalse
%
% childdoc.dtx Copyright (C) 2017-2018 Niklas Beisert
%
% This work may be distributed and/or modified under the
% conditions of the LaTeX Project Public License, either version 1.3
% of this license or (at your option) any later version.
% The latest version of this license is in
%   http://www.latex-project.org/lppl.txt
% and version 1.3 or later is part of all distributions of LaTeX
% version 2005/12/01 or later.
%
% This work has the LPPL maintenance status `maintained'.
%
% The Current Maintainer of this work is Niklas Beisert.
%
% This work consists of the files childdoc.dtx and childdoc.ins
% and the derived files childdoc.def and cdocsamp.tex with
% cdocsch1.tex, cdocsch2.tex, cdocsdrf.tex, cdocsfn1.tex, cdocsfn2.tex.
%
%<package>\ifdefined\childdocmain\endinput\fi
%<package>\ProvidesFile{childdoc.def}[2018/12/30 v2.0 child document driver]
%<samplemain>\ProvidesFile{cdocsamp.tex}[2018/12/30 v2.0 sample for childdoc]
%<*driver>
%\ProvidesFile{childdoc.drv}[2018/12/30 v2.0 childdoc reference manual file]
\PassOptionsToClass{10pt,a4paper}{article}
\documentclass{ltxdoc}

\usepackage[margin=35mm]{geometry}
\usepackage{hyperref}
\usepackage{hyperxmp}
\usepackage[usenames]{color}

\hypersetup{colorlinks=true}
\hypersetup{pdfstartview=FitH}
\hypersetup{pdfpagemode=UseNone}
\hypersetup{pdfsource={}}
\hypersetup{pdflang={en-UK}}
\hypersetup{pdfcopyright={Copyright 2017-2018 Niklas Beisert.
  This work may be distributed and/or modified under the
  conditions of the LaTeX Project Public License, either version 1.3
  of this license or (at your option) any later version.}}
\hypersetup{pdflicenseurl={http://www.latex-project.org/lppl.txt}}
\hypersetup{pdfcontactaddress={ETH Zurich, ITP, HIT K,
  Wolfgang-Pauli-Strasse 27}}
\hypersetup{pdfcontactpostcode={8093}}
\hypersetup{pdfcontactcity={Zurich}}
\hypersetup{pdfcontactcountry={Switzerland}}
\hypersetup{pdfcontactemail={nbeisert@itp.phys.ethz.ch}}
\hypersetup{pdfcontacturl={http://people.phys.ethz.ch/\xmptilde nbeisert/}}

\newcommand{\secref}[1]{\hyperref[#1]{section \ref*{#1}}}

\parskip1ex
\parindent0pt
\let\olditemize\itemize
\def\itemize{\olditemize\parskip0pt}

\begin{document}

\title{The \textsf{childdoc} Package}
\hypersetup{pdftitle={The childdoc Package}}
\author{Niklas Beisert\\[2ex]
  Institut f\"ur Theoretische Physik\\
  Eidgen\"ossische Technische Hochschule Z\"urich\\
  Wolfgang-Pauli-Strasse 27, 8093 Z\"urich, Switzerland\\[1ex]
  \href{mailto:nbeisert@itp.phys.ethz.ch}
  {\texttt{nbeisert@itp.phys.ethz.ch}}}
\hypersetup{pdfauthor={Niklas Beisert}}
\hypersetup{pdfsubject={Manual for the LaTeX2e Package childdoc}}
\date{30 December 2018, \textsf{v2.0}}
\maketitle

\begin{abstract}\noindent
\textsf{childdoc} is a \LaTeXe{} package
that enables the direct compilation
of document sections included by |\include|
to individual files.
\end{abstract}

\begingroup
\parskip0ex
\tableofcontents
\endgroup

%%%%%%%%%%%%%%%%%%%%%%%%%%%%%%%%%%%%%%%%%%%%%%%%%%%%%%%%%%%%%%%%%%%%%%%%%%%%%%%%
%%%%%%%%%%%%%%%%%%%%%%%%%%%%%%%%%%%%%%%%%%%%%%%%%%%%%%%%%%%%%%%%%%%%%%%%%%%%%%%%
\section{Introduction}

\LaTeX{} provides a mechanism to structure a large document (such as a book)
into a main file and several child files (containing the chapters)
using the |\include| command.
This mechanism is beneficial for documents
which span hundreds of pages in order to
make the source file(s) more manageable.
Moreover, compilation can be restricted to
selected child files by means of the |\includeonly| command.
The latter feature can be used to reduce the compilation time while editing
(this was significantly more useful in the earlier days of \LaTeX{})
or to generate a smaller document which is easier to navigate.
Another application of |\includeonly| is to generate
documents consisting of selected parts of the complete document.

However, there are a few drawbacks of the plain |\include| mechanism:
\begin{itemize}
\item
The child files cannot be compiled on their own,
they can only be compiled via the main file.
A naive editing environment
(such as a text editor with an option
to have the current file processed by \LaTeX)
may require one to switch to the main file before compiling;
attempting to compile the child file produces errors.
\item
The main file must be modified (each time)
to adjust the |\includeonly| command
to the present needs. This easily leaves the main file in a messy state.
\item
The generated document will always carry the filename
of the main document. This is inconvenient if
several child files are to be compiled and
to be kept for distribution.
\end{itemize}

The present package provides a simple interface
to make child files individually compilable by \LaTeX{}.
Compiling a child file then has the same effect as compiling
the main file with an |\includeonly| command
to select the appropriate child.
Moreover the generated document will carry the name of the child
rather than the main file.
This resolves all three above issues.

This feature is meant to make the editing of books,
thesis documents and lecture notes somewhat more convenient.
However, the package can also be used efficiently for
composing a series of documents (such as exercise sheets)
which are typically distributed individually.
It then assists the author in generating the individual documents
(potentially in different versions)
as well as a document containing the collected series.
Another application is in developing style files
or other kinds of included material
where compilation of the style file could redirect
to a sample or test file.

%%%%%%%%%%%%%%%%%%%%%%%%%%%%%%%%%%%%%%%%%%%%%%%%%%%%%%%%%%%%%%%%%%%%%%%%%%%%%%%%
%%%%%%%%%%%%%%%%%%%%%%%%%%%%%%%%%%%%%%%%%%%%%%%%%%%%%%%%%%%%%%%%%%%%%%%%%%%%%%%%
\section{Usage}

First of all, the package \textsf{childdoc} is \emph{not} a standard
\LaTeXe{} |.sty| style file! Therefore it needs to be invoked in
a non-standard way.

%%%%%%%%%%%%%%%%%%%%%%%%%%%%%%%%%%%%%%%%%%%%%%%%%%%%%%%%%%%%%%%%%%%%%%%%%%%%%%%%
\subsection{Included Files}
\label{sec:include}

%%%%%%%%%%%%%%%%%%%%%%%%%%%%%%%%%%%%%%%%
\DescribeMacro{\childdocmain}
To use the package, add the commands
\begin{center}
\begin{tabular}{l}
|\input{childdoc.def}|\\
|\childdocmain{}|\\
\end{tabular}
\end{center}
at the very top of the main \LaTeX{} file,
in particular \emph{before} the |\documentclass| statement!
The argument of |\childdocmain| should be left empty
(but it must be present).

%%%%%%%%%%%%%%%%%%%%%%%%%%%%%%%%%%%%%%%%
\DescribeMacro{\childdocof}
Furthermore, add the commands
\begin{center}
\begin{tabular}{l}
|\input{childdoc.def}|\\
|\childdocof{|\textit{main}|}|\\
\end{tabular}
\end{center}
at the top of every child file \textit{child}
which is included by |\include{|\textit{child}|}|
from within the main file
(or at least for those files to be compiled individually).
The argument \textit{main} must be the filename of the main file.

There are a couple of
considerations in setting up the main and child documents:

%%%%%%%%%%%%%%%%%%%%%%%%%%%%%%%%%%%%%%%%
\paragraph{Restrictions.}

Please note the following restrictions:
\begin{itemize}
\item
|\childdocmain| must be called with one argument \textit{main}
to ensure compatibility with earlier version of the package.
It must either be empty (|\childdocmain{}|)
or precisely match the filename of the main file in which it is specified.
See \secref{sec:detection} for further information.
\item
The filename \textit{main} must be specified without the |.tex| extension.
\item
The filename \textit{main} is case sensitive
(even in case-insensitive file systems)
due to internal string comparison.
\item
The argument \textit{main} should be fully expanded, it cannot be a macro.
\item
Subdirectories and special characters should be avoided in filenames.
\item
The command |\childdocmain{|\textit{main}|}| must be followed by a whitespace.
It should not be followed immediately by another command
or by a comment mark `|%|'.
This is because the \TeX{} parser reads the token immediately following
the argument of |\childdocmain| and puts it
at the beginning of every child section;
however, a white\-space is ignored.
\end{itemize}

%%%%%%%%%%%%%%%%%%%%%%%%%%%%%%%%%%%%%%%%
\paragraph{Content of Main File.}

It is advisable to place all content in the child files included by |\include|.
Any output contained in the main file will appear in all child documents
unless suppressed manually;
it cannot be suppressed automatically by the |\includeonly| directive
and thus should normally be avoided.
A method to include some content in the main file
by means of conditional processing is described in \secref{sec:conditional}.

%%%%%%%%%%%%%%%%%%%%%%%%%%%%%%%%%%%%%%%%
\paragraph{Page Numbering.}

When only a part of the document is compiled,
the appropriate numbering of pages
(as well as other status parameters)
is determined from the |.aux| files.
The latter contain information from previous passes.
However this information needs to propagate through
all intermediate child documents.
Therefore the page numbering in child documents may well
be inconsistent until the complete document is compiled at least once.

A useful (if unconventional) way to always ensure a consistent
page numbering is to restart the numbering in each child document
and denote the pages by `\textit{child}|.|\textit{page}'
where \textit{child} represents the chapter/section number of the child file.
This can be achieved by the command
|\numberwithin{page}{|\textit{child}|}|
of the \textsf{amsmath} package
where \textit{child} can be |chapter| or |section|
depending on the chosen structuring.
Alternatively, one can modify the macro |\thepage| appropriately
and reset the counter |page| at the start of each child file.

%%%%%%%%%%%%%%%%%%%%%%%%%%%%%%%%%%%%%%%%%%%%%%%%%%%%%%%%%%%%%%%%%%%%%%%%%%%%%%%%
\subsection{Conditional Processing}
\label{sec:conditional}

The package provides a mechanism to compile different versions
of a document. To customise the versions further some conditional processing
can come in handy to distinguish which version is being compiled.
The package provides two macros to describe the compilation context:

%%%%%%%%%%%%%%%%%%%%%%%%%%%%%%%%%%%%%%%%
\DescribeMacro{\ifchilddoc}
The conditional |\ifchilddoc| distinguishes between the compilation of
child documents and the main document:
%
\begin{center}
|\ifchilddoc |\textit{child-code}| |[|\||else |\textit{main-code}]| \||fi|
\end{center}

%%%%%%%%%%%%%%%%%%%%%%%%%%%%%%%%%%%%%%%%
\DescribeMacro{\childdocname}
\DescribeMacro{\childdocjob}
The macro |\childdocname| contains the filename (without extension)
of the main or child file being processed.
Note that |\childdocjob| will always contain the name of the main file.

%%%%%%%%%%%%%%%%%%%%%%%%%%%%%%%%%%%%%%%%
\paragraph{Title Page.}

Conditional processing can be used to include a title or banner page
in the main document when proper precautions are taken.
Importantly, the code in the main file should ensure that the page counter
(as well as other status parameters which are stored in the |.aux| files)
takes the same value after the conditional processing.
Otherwise the page numbers may take divergent values
depending on which part is compiled.

For example, a title page could be declared by:
%
\begin{center}
\begin{tabular}{l}
|\ifchilddoc\||else|\\
|\addtocounter{page}{-1}|\\
\textit{code for title page}\\
|\newpage|\\
|\||fi|
\end{tabular}
\end{center}
%
A banner page for the child documents can be generated by:
%
\begin{center}
\begin{tabular}{l}
|\ifchilddoc|\\
|\addtocounter{page}{-1}|\\
\textit{code for banner page}\\
|\newpage|\\
|\||fi|
\end{tabular}
\end{center}
%
Here one could write a message such as:
\begin{center}
|This is the part \childdocname{} of \childdocjob{}.|
\end{center}

%%%%%%%%%%%%%%%%%%%%%%%%%%%%%%%%%%%%%%%%%%%%%%%%%%%%%%%%%%%%%%%%%%%%%%%%%%%%%%%%
\subsection{Flags}
\label{sec:flags}

The package makes it easy to generate different versions
of the main or child documents.
To this end compilation flags can be defined
and assigned different default values.
They will be particularly useful in conjunction
with the forwarding mechanism described in \secref{sec:forward}.

For example, it may be useful to have a flag |\version|
which can be set to |draft| or |final|.
The document source will contain some conditional code
depending on the value of |\version|.
Suppose further, the flag should default to |final| for the main file
and to |draft| for child files
which is a natural assignment for editing the document.
This is achieved by placing the following code
in the preamble of the main document
(below the |\childdocmain| directive):
%
\begin{center}
\begin{tabular}{l}
|\ifchilddoc|\\
|\providecommand{\version}{draft}|\\
|\||else|\\
|\providecommand{\version}{final}|\\
|\||fi|
\end{tabular}
\end{center}
%
The definition by |\providecommand| makes sure
that previous definitions are not overwritten.
Further statements |\providecommand{\version}{...}|
can thus be added before the above code to override it.

For the main file, one might add a line
(between |\childdocmain| and the above block)
%
\begin{center}
|%\ifchilddoc\||else\providecommand{\version}{draft}\||fi|
\end{center}
%
which can be uncommented to produce a draft version.
Likewise one can add a line to the very top of a child file
(above the |\childdocof{|\textit{main}|}| directive)
%
\begin{center}
|%\providecommand{\version}{final}|
\end{center}
%
which can be uncommented to produce the final version of this child document.

%%%%%%%%%%%%%%%%%%%%%%%%%%%%%%%%%%%%%%%%%%%%%%%%%%%%%%%%%%%%%%%%%%%%%%%%%%%%%%%%
\subsection{Forwarding}
\label{sec:forward}

Different versions of the main or child documents
using compilation flags as described in \secref{sec:flags}
can be (permanently) stored in different files
for convenient compilation, viewing and distribution.
To this end, the package defines a command
to pass on compilation to a different file:

%%%%%%%%%%%%%%%%%%%%%%%%%%%%%%%%%%%%%%%%
\DescribeMacro{\childdocforward}
The command |\childdocforward| redirects processing to
another source file:
%
\begin{center}
\begin{tabular}{l}
|\input{childdoc.def}|\\
|\childdocforward[|\textit{main}|]{|\textit{dest}|}|\\
\end{tabular}
\end{center}
%
The argument \textit{dest} is the destination file
(without extension).
It should be the main file or one of the child files.
Note that further \textsf{childdoc} directives
such as |\childdocof| and |\childdocforward|
in the indicated file will be processed in this form.
The optional argument \textit{main}
passes on directly to the main file \textit{main}
while pretending to compile the child \textit{dest}.
This form behaves as if \textit{dest}
issues |\childdocof{|\textit{main}|}| right away,
and no further \textsf{childdoc} directives will be processed.

%%%%%%%%%%%%%%%%%%%%%%%%%%%%%%%%%%%%%%%%
\DescribeMacro{\...prefix}
In the alternative form |\childdocforwardprefix|,
%
\begin{center}
\begin{tabular}{l}
|\input{childdoc.def}|\\
|\childdocforwardprefix[|\textit{main}|]{|\textit{prefix}|}{|\textit{dest}|}|
\end{tabular}
\end{center}
%
the destination file is determined by a pattern
depending on the current file:
To make this work, the current file must be called
`{\textit{prefix}\hspace{0.2em}\textit{suffix}}'
with \textit{prefix} matching precisely the argument.
Processing is then passed on to the file
`{\textit{dest}\hspace{0.2em}\textit{suffix}}'.
Surely, the same effect is achieved by
directly specifying the
argument `{\textit{dest}\hspace{0.2em}\textit{suffix}}'
in the first form.
However, that requires to set up a different file
for each child. With the alternative form of the command
all these files can have exactly the same content
which simplifies setting them up and maintaining them.

For example, the following file |draft.tex|
with a compilation flag |\version| as described in \secref{sec:flags}
compiles the main document as a draft:
%
\begin{center}
\begin{tabular}{l}
|\def\version{draft}|\\
|\input{childdoc.def}|\\
|\childdocforward{|\textit{main}|}|
\end{tabular}
\end{center}
%
Likewise, the following files |final|\textit{nn}|.tex|
compile the final version of the child document
|child|\textit{nn}|.tex|:
%
\begin{center}
\begin{tabular}{l}
|\def\version{final}|\\
|\input{childdoc.def}|\\
|\childdocforwardprefix{final}{child}|
\end{tabular}
\end{center}
%

Note that when several versions of a main file and/or of each child file
are to be generated, it may be convenient to set up a |Makefile| or
shell script to automatise the process.

%%%%%%%%%%%%%%%%%%%%%%%%%%%%%%%%%%%%%%%%%%%%%%%%%%%%%%%%%%%%%%%%%%%%%%%%%%%%%%%%
\subsection{Command Line Processing}
\label{sec:commandline}

The effect of redirection files can also be achieved by invoking
the \LaTeX{} compiler with a more elaborate command line.
Most conveniently this should be done as part
of a shell script or a |Makefile|.

When using \textsf{childdoc} in the main file, the following
command lines effectively perform a redirection
(note that depending on the shell being used,
backslashes may have to be doubled: `|\|' $\to$ `|\\|'):
%
\begin{center}
|... -jobname "|\textit{target}|" |\\|"|[\textit{flags}]%
|\input{childdoc.def}\childdocforward[|\textit{main}|]{|\textit{dest}|}"|
\end{center}
%
Here \textit{target} is the name of the output file,
\textit{main} is the name of the main file
and \textit{dest} is the name of the main or child file to be processed
(all filenames without extensions).
The optional argument \textit{main} can be omitted
if \textit{main} matches \textit{dest}.
Optionally, compilation \textit{flags} can be defined via |\def| commands.
This command line makes the \TeX{} engine believe
it is compiling the file \textit{target}
whose content is specified as the latter parameter.
The provided code then forwards the processing to
\textit{main} or \textit{dest} as described in \secref{sec:forward}.

%%%%%%%%%%%%%%%%%%%%%%%%%%%%%%%%%%%%%%%%%%%%%%%%%%%%%%%%%%%%%%%%%%%%%%%%%%%%%%%%
\subsection{Include by Input}
\label{sec:input}

Including child documents by |\include| has some restrictions by design.
Most notably, the content of a child document always occupies
its own set of pages; pages cannot be shared between child documents.
Usually, this behaviour makes perfect sense
because each child document contain an essential part of the document.
However, in some situations it may be desirable to compose
a document from a collection of parts
without having mandatory page breaks between then.
For this case, the package
provides a mechanism to include parts
by |\input| which can also be processed individually.
However, by construction this mechanism
requires manual handling of the content to be output.

%%%%%%%%%%%%%%%%%%%%%%%%%%%%%%%%%%%%%%%%
\DescribeMacro{\ifchilddocmanual}
The main file should be prepared as usual, see \secref{sec:include}.
However, the document body must make a distinction
between processing of an individual part and of the main document, e.g.:
%
\begin{center}
\begin{tabular}{l}
|\ifchilddocmanual|\\
|\input{\childdocname}|\\
|\||else|\\
\textit{document body with }|\input{|\textit{part}|}|\\
|\||fi|
\end{tabular}
\end{center}
%
The conditional |\ifchilddocmanual| is true whenever
a part to be included by |\input| is being compiled,
and the name of the part is stored in |\childdocname|.

%%%%%%%%%%%%%%%%%%%%%%%%%%%%%%%%%%%%%%%%
\DescribeMacro{\childdocby}
Each part to be included by |\input| should start with:
%
\begin{center}
\begin{tabular}{l}
|\input{childdoc.def}|\\
|\childdocby{|\textit{main}|}|\\
\end{tabular}
\end{center}
%
The directive |\childdocby| is similar to |\childdocof|
described in \secref{sec:include},
but the subsequent selection of content must be done manually.
To that end, both |\ifchilddoc| and |\ifchilddocmanual|
will be true upon processing of a part,
and the name of the part is stored in |\childdocname|.
Note that |\jobname| will be set to the filename of the current part
so that each part receives an individual |.aux| file
that does not interfere with the |.aux| file(s) of the main document.
This behaviour can be altered by the alternative form
|\childdocby[*]{|\textit{main}|}| (with a non-empty optional argument)
which uses the |.aux| file of the main document
by setting |\jobname| to \textit{main}.

%%%%%%%%%%%%%%%%%%%%%%%%%%%%%%%%%%%%%%%%%%%%%%%%%%%%%%%%%%%%%%%%%%%%%%%%%%%%%%%%
\subsection{Driver Development}
\label{sec:driver}

The \textsf{childdoc} mechanism can also be use for the development
of definition files such as \LaTeX{} styles or classes.
This case differs from the above setup with multiple parts
included by |\include| in that no |\includeonly| should be invoked.
This can be achieved by starting the include file
(before |\ProvidesPackage|) with:
%
\begin{center}
\begin{tabular}{l}
|\input{childdoc.def}|\\
|\childdocforward{|\textit{main}|}|\\
\end{tabular}
\end{center}
%
or alternatively with:
%
\begin{center}
\begin{tabular}{l}
|\input{childdoc.def}|\\
|\childdocby{|\textit{main}|}|\\
\end{tabular}
\end{center}
%
Both forms have slightly different effects as described above.
The main file is prepared as usual, see \secref{sec:include}.

%%%%%%%%%%%%%%%%%%%%%%%%%%%%%%%%%%%%%%%%%%%%%%%%%%%%%%%%%%%%%%%%%%%%%%%%%%%%%%%%
\subsection{Legacy Detection}
\label{sec:detection}

The directive |\childdocmain| in the main file can detect
whether the complete document or merely a child is to be compiled
even without using the directive |\childdocof|.
This method is deprecated because it is less robust
and there is no compelling reason to use it;
it is merely provided for backward compatibility
and it may be removed in future versions.

If the detection mechanism is to be used,
it is mandatory to correctly specify
the filename of the main file as the argument of |\childdocmain|:
%
\begin{center}
\begin{tabular}{l}
|\input{childdoc.def}|\\
|\childdocmain{|\textit{main}|}|\\
\end{tabular}
\end{center}
%
If |\jobname| does not match the argument \textit{main} of |\childdocmain|,
it is assumed that |\jobname| points to the child file to be compiled.
When using |\childdocmain| with the main file specified as argument,
it suffices to start a child file
with just |\input{|\textit{main}|}|
without loading of the package and using |\childdocof|.
If instead all processing is done
with the appropriate \textsf{childdoc} directives,
the argument of \textit{main} of |\childdocmain| can be empty.

An alternative version of the command line processing described
in \secref{sec:commandline} using the detection mechanism reads:
%
\begin{center}
|... -jobname "|\textit{target}|" "|[\textit{flags}]%
[|\def\jobname{|\textit{dest}|}|]|\input{|\textit{main}|}"|
\end{center}

%%%%%%%%%%%%%%%%%%%%%%%%%%%%%%%%%%%%%%%%%%%%%%%%%%%%%%%%%%%%%%%%%%%%%%%%%%%%%%%%
\subsection{Manual Code}
\label{sec:manual}

In case one cannot be certain whether the definitions file |childdoc.def|
is installed on the target \TeX{} distribution
and one prefers not to ship it,
it is conceivable to paste a few relevant commands into the sources.

To that end, drop all statements |\input{childdoc.def}|
and perform the replacements as outlined below.
Instead of |\childdocmain{|\textit{main}|}| add the following code
to the top of the main file:
%
\begin{center}
\begin{tabular}{l}
|\||ifdefined\childdocname\endinput\||fi\newif\ifchilddoc|\\
|\edef\childdocname{\scantokens\expandafter{\jobname\noexpand}}|\\
|\def\childdocmain{|\textit{main}|}\||ifx\childdocmain\childdocname\||else|\\
|\childdoctrue\includeonly{\childdocname}\let\jobname\childdocmain\||fi|\\
\end{tabular}
\end{center}
%
Instead of |\childdocof{|\textit{main}|}| just include the main file
at the top of each child file:
%
\begin{center}
|\input{|\textit{main}|}|
\end{center}
%
A simple redirection |\childdocforward{|\textit{dest}|}| is achieved by:
%
\begin{center}
|\def\jobname{|\textit{dest}|}\input{\jobname}|
\end{center}
%
The redirection with prefix
|\childdocforwardprefix[|\textit{prefix}|]{|\textit{dest}|}|
is accomplished by:
%
\begin{center}
\begin{tabular}{l}
|{\edef\jobname{\scantokens\expandafter{\jobname\noexpand}}|\\
|\def\redirectjob |\textit{prefix}|#1~~~{\gdef\jobname{|\textit{dest}|#1}}|\\
|\expandafter\redirectjob\jobname~~~}\input{\jobname}|
\end{tabular}
\end{center}

In an alternative approach,
child documents can be compiled by a specific command line
without additional code or specific definitions:
%
\begin{center}
|... -jobname "|\textit{target}|" "|[\textit{flags}]%
|\includeonly{|\textit{dest}|}\input{|\textit{main}|}"|
\end{center}
%

%%%%%%%%%%%%%%%%%%%%%%%%%%%%%%%%%%%%%%%%%%%%%%%%%%%%%%%%%%%%%%%%%%%%%%%%%%%%%%%%
%%%%%%%%%%%%%%%%%%%%%%%%%%%%%%%%%%%%%%%%%%%%%%%%%%%%%%%%%%%%%%%%%%%%%%%%%%%%%%%%
\section{Information}

%%%%%%%%%%%%%%%%%%%%%%%%%%%%%%%%%%%%%%%%%%%%%%%%%%%%%%%%%%%%%%%%%%%%%%%%%%%%%%%%
\subsection{Copyright}

Copyright \copyright{} 2017--2018 Niklas Beisert

This work may be distributed and/or modified under the
conditions of the \LaTeX{} Project Public License, either version 1.3
of this license or (at your option) any later version.
The latest version of this license is in
  \url{http://www.latex-project.org/lppl.txt}
and version 1.3 or later is part of all distributions of \LaTeX{}
version 2005/12/01 or later.

This work has the LPPL maintenance status `maintained'.

The Current Maintainer of this work is Niklas Beisert.

This work consists of the files |README.txt|, |childdoc.ins| and |childdoc.dtx|
as well as the derived files |childdoc.def|, |cdocsamp.tex|
with |cdocsch1.tex|, |cdocsch2.tex|, |cdocspt3.tex|, |cdocspt4.tex|,
|cdocsdrf.tex|, |cdocsfn1.tex|, |cdocsfn2.tex|
as well as |childdoc.pdf|.

%%%%%%%%%%%%%%%%%%%%%%%%%%%%%%%%%%%%%%%%%%%%%%%%%%%%%%%%%%%%%%%%%%%%%%%%%%%%%%%%
\subsection{Files and Installation}

The package consists of the files:
%
\begin{center}
\begin{tabular}{ll}
    |README.txt|   & readme file \\
    |childdoc.ins| & installation file \\
    |childdoc.dtx| & source file \\
    |childdoc.def| & definition file \\
    |cdocsamp.tex| & sample main file \\
    |cdocsch1.tex| & sample include file \\
    |cdocsch2.tex| & sample include file \\
    |cdocspt3.tex| & sample part file \\
    |cdocspt4.tex| & sample part file \\
    |cdocsdrf.tex| & sample redirection file \\
    |cdocsfn1.tex| & sample redirection file \\
    |cdocsfn2.tex| & sample redirection file \\
    |childdoc.pdf| & manual
\end{tabular}
\end{center}
%
The distribution consists of the files
|README.txt|, |childdoc.ins| and |childdoc.dtx|.
%
\begin{itemize}
\item
Run (pdf)\LaTeX{} on |childdoc.dtx|
to compile the manual |childdoc.pdf| (this file).
\item
Run \LaTeX{} on |childdoc.ins| to create the definitions file |childdoc.def|
and the sample |cdocsamp.tex| with include files
|cdocsch1.tex|, |cdocsch2.tex|, |cdocspt3.tex|, |cdocspt4.tex|,
|cdocsdrf.tex|, |cdocsfn1.tex|, |cdocsfn2.tex|.
Then copy the file |childdoc.def| to an appropriate directory of your \LaTeX{}
distribution, e.g.\ \textit{texmf-root}|/tex/latex/childdoc|.
\end{itemize}

%%%%%%%%%%%%%%%%%%%%%%%%%%%%%%%%%%%%%%%%%%%%%%%%%%%%%%%%%%%%%%%%%%%%%%%%%%%%%%%%
\subsection{Related CTAN Packages}

There are several other packages which offer a similar functionality:
%
\begin{itemize}
\item
The packages
\href{http://ctan.org/pkg/docmute}{\textsf{docmute}},
\href{http://ctan.org/pkg/includex}{\textsf{includex}} and
\href{http://ctan.org/pkg/standalone}{\textsf{standalone}}
provide commands to include only the document body of
a child file thus allowing both files to be compiled individually.
\item
The packages \href{http://ctan.org/pkg/subdocs}{\textsf{subdocs}}
and \href{http://ctan.org/pkg/subfiles}{\textsf{subfiles}}
provide structures in which the main and child documents can be
encapsulated and allowing them to be compiled individually.
The inclusion mechanism is different from the conventional |\include|.
\item
The package \href{http://ctan.org/pkg/combine}{\textsf{combine}}
is an elaborate solution to combine several documents into one.
\end{itemize}
%
See also the CTAN topic \href{http://ctan.org/topic/subdocs}{\textsf{subdocs}}
for further related packages.
The present package differs from the above solutions in that
a document structure constructed with the conventional |\include| mechanism
just needs two extra commands at the top of every file
such that all constituent files can be compiled individually.

%%%%%%%%%%%%%%%%%%%%%%%%%%%%%%%%%%%%%%%%%%%%%%%%%%%%%%%%%%%%%%%%%%%%%%%%%%%%%%%%
%\subsection{Feature Suggestions}
%
%The following is a list of features which may be useful for future
%versions of this package:
%%
%\begin{itemize}
%\item
%\ldots
%\end{itemize}

%%%%%%%%%%%%%%%%%%%%%%%%%%%%%%%%%%%%%%%%%%%%%%%%%%%%%%%%%%%%%%%%%%%%%%%%%%%%%%%%
\subsection{Revision History}

%%%%%%%%%%%%%%%%%%%%%%%%%%%%%%%%%%%%%%%%
\paragraph{v2.0:} 2018/12/30

\begin{itemize}
\item
immediate forward processing
\item
added |\childdocby| mechanism
\item
manual restructured
\end{itemize}

%%%%%%%%%%%%%%%%%%%%%%%%%%%%%%%%%%%%%%%%
\paragraph{v1.6:} 2018/01/17

\begin{itemize}
\item
application for development of include files
\item
corrections to manual
\end{itemize}

%%%%%%%%%%%%%%%%%%%%%%%%%%%%%%%%%%%%%%%%
\paragraph{v1.5:} 2017/05/21

\begin{itemize}
\item
more complete structuring introduced
\item
|\childdocof| introduced
\item
|\childdoc| renamed to |\childdocmain|
\item
|\childredirect| renamed to |\childdocforward| and |\childdocforwardprefix|
and functionality expanded
\end{itemize}

%%%%%%%%%%%%%%%%%%%%%%%%%%%%%%%%%%%%%%%%
\paragraph{v1.0:} 2017/04/27

\begin{itemize}
\item
manual and install package
\item
first version published on CTAN
\end{itemize}

%%%%%%%%%%%%%%%%%%%%%%%%%%%%%%%%%%%%%%%%
\paragraph{v0.6:} 2017/04/26

\begin{itemize}
\item
redirection mechanism added
\end{itemize}

%%%%%%%%%%%%%%%%%%%%%%%%%%%%%%%%%%%%%%%%
\paragraph{v0.5:} 2017/04/26

\begin{itemize}
\item
functionality in definition file
\end{itemize}


%%%%%%%%%%%%%%%%%%%%%%%%%%%%%%%%%%%%%%%%%%%%%%%%%%%%%%%%%%%%%%%%%%%%%%%%%%%%%%%%
%%%%%%%%%%%%%%%%%%%%%%%%%%%%%%%%%%%%%%%%%%%%%%%%%%%%%%%%%%%%%%%%%%%%%%%%%%%%%%%%
%%%%%%%%%%%%%%%%%%%%%%%%%%%%%%%%%%%%%%%%%%%%%%%%%%%%%%%%%%%%%%%%%%%%%%%%%%%%%%%%
\appendix

\settowidth\MacroIndent{\rmfamily\scriptsize 000\ }

 \DocInput{childdoc.dtx}

\end{document}
%</driver>
% \fi
%
% %%%%%%%%%%%%%%%%%%%%%%%%%%%%%%%%%%%%%%%%%%%%%%%%%%%%%%%%%%%%%%%%%%%%%%%%%%%%%%
% %%%%%%%%%%%%%%%%%%%%%%%%%%%%%%%%%%%%%%%%%%%%%%%%%%%%%%%%%%%%%%%%%%%%%%%%%%%%%%
% \section{Sample}
%\iffalse
%<*samplemain>
%\fi
%
% The following presents a sample document
% with two chapters, two parts, a title page,
% a compile flag as well as three forwarding files to set the flag.
% It consists of eight |.tex| files:
% \begin{center}
% \begin{tabular}{ll}
% |cdocsamp.tex|&main file\\
% |cdocsch1.tex|&include file for chapter 1\\
% |cdocsch2.tex|&include file for chapter 2\\
% |cdocspt3.tex|&include file for part 3\\
% |cdocspt4.tex|&include file for part 4\\
% |cdocsdrf.tex|&forwarding file for main file in draft mode\\
% |cdocsfi1.tex|&forwarding file for final version of chapter 1\\
% |cdocsfi2.tex|&forwarding file for final version of chapter 2\\
% \end{tabular}
% \end{center}
% Each of the eight files can be compiled directly by the \LaTeX{} compiler.
%
% %%%%%%%%%%%%%%%%%%%%%%%%%%%%%%%%%%%%%%
% \paragraph{Main File.}
%
% The main file is called |cdocsamp.tex|.
%
% Load the \textsf{childdoc} definitions and
% declare the filename for the main document:
%    \begin{macrocode}
\input{childdoc.def}
\childdocmain{}
%    \end{macrocode}

% Optional override for |\version| flag:
%    \begin{macrocode}
%%\ifchilddoc\else\providecommand{\version}{draft}\fi
%    \end{macrocode}

% Define the default values for the |\version| flag
% (|final| for the main file and |draft| for childs):
%    \begin{macrocode}
\ifchilddoc
\providecommand{\version}{draft}
\else
\providecommand{\version}{final}
\fi
%    \end{macrocode}

% Load the standard document class:
%    \begin{macrocode}
\documentclass[12pt]{article}
%    \end{macrocode}

% Start the document body:
%    \begin{macrocode}
\begin{document}
%    \end{macrocode}

% Declare a title page.
% Print title, part of document being processed and version flag:
%    \begin{macrocode}
\addtocounter{page}{-1}
\begin{center}
{\LARGE\bfseries{}childdoc example\par}
\vspace{1cm}
\ifchilddoc
\ifchilddocmanual part\else chapter\fi:
`\childdocname' of `\childdocjob'\par
\else
main document: `\childdocjob'\par
\fi
version: \version\par
\end{center}
\newpage
%    \end{macrocode}

% Manually include selected file,
% otherwise process as usual:
%    \begin{macrocode}
\ifchilddocmanual
\section*{part `\childdocname'}
\input{\childdocname}
\else
%    \end{macrocode}

% Include the two chapters:
%    \begin{macrocode}
\include{cdocsch1}
\include{cdocsch2}
%    \end{macrocode}

% Include the two parts unless only chapters should be displayed:
%    \begin{macrocode}
\ifchilddoc\else
\section{part three}
\input{cdocspt3}
\section{part four}
\input{cdocspt4}
\fi
%    \end{macrocode}

% Process as usual until here:
%    \begin{macrocode}
\fi
%    \end{macrocode}

% End of document body:
%    \begin{macrocode}
\end{document}
%    \end{macrocode}
%\iffalse
%</samplemain>
%\fi
%
% %%%%%%%%%%%%%%%%%%%%%%%%%%%%%%%%%%%%%%
% \paragraph{Chapter Include Files.}
%
% The include files are called |cdocsch1.tex| and |cdocsch2.tex|.
%
%\iffalse
%<*samplechap1|samplechap2>
%\fi

% Optional override for |\version| flag:
%    \begin{macrocode}
%%\providecommand{\version}{final}
%    \end{macrocode}

% Include the main document:
%    \begin{macrocode}
\input{childdoc.def}
\childdocof{cdocsamp}
%    \end{macrocode}

%\iffalse
%</samplechap1|samplechap2>
%\fi
%
%\iffalse
%<*samplechap1>
%\fi
% Some text for chapter 1:
%    \begin{macrocode}
\section{one}
some text in chapter one
%    \end{macrocode}

%\iffalse
%</samplechap1>
%\fi
% Some text for chapter 2:
%\iffalse
%<*samplechap2>
%\fi
%    \begin{macrocode}
\section{two}
more text in chapter two
%    \end{macrocode}

%\iffalse
%</samplechap2>
%\fi
%
% %%%%%%%%%%%%%%%%%%%%%%%%%%%%%%%%%%%%%%
% \paragraph{Part Include Files.}
%
% The include files are called |cdocspt3.tex| and |cdocspt4.tex|.
%
%\iffalse
%<*samplepart3|samplepart4>
%\fi

% Optional override for |\version| flag:
%    \begin{macrocode}
%%\providecommand{\version}{final}
%    \end{macrocode}

% Include the main document:
%    \begin{macrocode}
\input{childdoc.def}
\childdocby{cdocsamp}
%    \end{macrocode}

%\iffalse
%</samplepart3|samplepart4>
%\fi
%
%\iffalse
%<*samplepart3>
%\fi
% Some text for part 3:
%    \begin{macrocode}
some text in part three
%    \end{macrocode}

%\iffalse
%</samplepart3>
%\fi
% Some text for part 4:
%\iffalse
%<*samplepart4>
%\fi
%    \begin{macrocode}
more text in part four
%    \end{macrocode}

%\iffalse
%</samplepart4>
%\fi
%
% %%%%%%%%%%%%%%%%%%%%%%%%%%%%%%%%%%%%%%
% \paragraph{Forwarding for a Complete Draft.}
%
% The following forwarding file |cdocsdrf.tex|
% compiles the main document in draft mode:
%\iffalse
%<*sampledraft>
%\fi
%    \begin{macrocode}
\def\version{draft}
\input{childdoc.def}
\childdocforward{cdocsamp}
%    \end{macrocode}

%\iffalse
%</sampledraft>
%\fi
%
% %%%%%%%%%%%%%%%%%%%%%%%%%%%%%%%%%%%%%%
% \paragraph{Forwarding for Final Version of the Chapters.}
%
% The following forwarding files |cdocsfn1.tex| and |cdocsfn2.tex|
% (with identical content)
% compile the final versions of the child documents
% |cdocsch1.tex| and |cdocsch2.tex|, respectively:
%\iffalse
%<*samplefinal>
%\fi
%    \begin{macrocode}
\def\version{final}
\input{childdoc.def}
\childdocforwardprefix[cdocsamp]{cdocsfn}{cdocsch}
%    \end{macrocode}

%\iffalse
%</samplefinal>
%\fi
%
% %%%%%%%%%%%%%%%%%%%%%%%%%%%%%%%%%%%%%%
% \paragraph{Command Line Processing.}
%
% The following three command lines generate the output files
% |cdocscld|, |cdocscl1| and |cdocscl2|
% which should be identical to
% |cdocsdrf|, |cdocsch1| and |cdocsfn2|, respectively:
% \begin{center}
% \begin{tabular}{l}
% |latex -jobname cdocscld \|\\
% |  "\def\version{draft}\input{childdoc.def}\childdocforward{cdocsamp}"|\\
% |latex -jobname cdocscl1 \|\\
% |  "\input{childdoc.def}\childdocforward[cdocsamp]{cdocsch1}"|\\
% |latex -jobname cdocscl2 \|\\
% |  "\def\version{final}\input{childdoc.def}\childdocforward{cdocsch2}"|
% \end{tabular}
% \end{center}
% Note that the trailing backslash on each first line
% merely continues the input to the second line
% (for convenient cut ant paste).
% Furthermore, the command |latex| can be replaced by any
% of its alternative versions such as |pdflatex|.
%
% %%%%%%%%%%%%%%%%%%%%%%%%%%%%%%%%%%%%%%%%%%%%%%%%%%%%%%%%%%%%%%%%%%%%%%%%%%%%%%
% %%%%%%%%%%%%%%%%%%%%%%%%%%%%%%%%%%%%%%%%%%%%%%%%%%%%%%%%%%%%%%%%%%%%%%%%%%%%%%
% \section{Implementation}
%\iffalse
%<*package>
%\fi
%
% This section describes the definitions file |childdoc.def|.

% The definitions cannot be loaded using |\usepackage| or |\RequirePackage|
% which has a mechanism to prevent loading a style file more than once.
% When loading the definitions by means of |\input|
% multiple instances have to be prevented manually:
%\iffalse
%This code needs to be before the `\ProvidesFile' directive
%which is defined at the beginning of this file.
%Therefore it is also placed there and commented out here.
%</package>
%<*discard>
%\fi
%    \begin{macrocode}
\ifdefined\childdocmain\endinput\fi
%    \end{macrocode}
%\iffalse
%</discard>
%<*package>
%\fi
%
% \macro{\ifchilddoc}
% \macro{\ifchilddocmanual}
% The conditional |\ifchilddoc| tells whether a
% child (true) or main (false) document is being compiled.
% The conditional |\ifchilddocmanual| tells whether
% the |\includeonly| mechanism is used (false) or
% the selection of child files must be performed manually (true).
% The definitions initialise to false:
%    \begin{macrocode}
\newif\ifchilddoc
\newif\ifchilddocmanual
%    \end{macrocode}

% \macro{\childdocname}
% \macro{\childdocjob}
% The macro |\childdocname| stores the name of the main document
% to be compiled. The macro |\childdocjob| stores the name of
% the document on which the \LaTeX{} compiler was originally invoked.
% The content of |\jobname| cannot be compared
% to filenames specified in the source due to different catcodes.
% The following code rescans |\jobname|, stores the result
% in |\childdocname| and saves a copy in |\childdocjob|:
%    \begin{macrocode}
\edef\childdocname{\scantokens\expandafter{\jobname\noexpand}}
\let\childdocjob\childdocname
%    \end{macrocode}

% \macro{\childdocdisable}
% The macro |\childdocdisable| prevents the main file
% from being processed more than once.
% At this stage, the main document command |\childdocmain|
% is assumed to be called once again where it should do nothing.
% Any subsequent call to it should prevent
% a secondary processing of the main document
% It overwrites the forwarding commands
% |\childdocof| and |\childdocforward|
% with empty macros to prevent further inclusions of the main document:
%    \begin{macrocode}
\newcommand{\childdocdisable}
{
  \renewcommand{\childdocmain}[1]{\renewcommand{\childdocmain}[1]{\endinput}}
  \renewcommand{\childdocof}[1]{}
  \renewcommand{\childdocby}[2][]{}
  \renewcommand{\childdocforward}[2][]{}
  \renewcommand{\childdocdisable}{}
}
%    \end{macrocode}

% \macro{\childdocmain}
% The macro |\childdocmain| is to be called at the top of the main file
% with nothing or the main filename (without extension) as argument.
% First, it breaks loops.
% If the argument is not empty and does not match |\childdocname|
% (which is set by the first inclusion of |childdoc.def|),
% |\ifchilddoc| is set to true, |\includeonly| is applied to the child file
% and |\jobname| is set to the main file
% (for proper handling of |.aux| files):
%    \begin{macrocode}
\newcommand{\childdocmain}[1]
{
  \childdocdisable\childdocmain{}
  \if?#1?\else
    \begingroup
      \def\childdoctmp{#1}
      \ifx\childdoctmp\childdocname
        \def\childdoctmp{}
      \else
        \def\childdoctmp
        {
          \childdoctrue
          \includeonly{\childdocname}
          \def\childdocjob{#1}
          \def\jobname{#1}
        }
      \fi
      \expandafter
    \endgroup
    \childdoctmp
  \fi
}
%    \end{macrocode}

% \macro{\childdocof}
% The command |\childdocof| redirects
% compilation to the main file |#1|.
%    \begin{macrocode}
\newcommand{\childdocof}[1]
{
  \childdocdisable
  \childdoctrue
  \includeonly{\childdocname}
  \def\jobname{#1}
  \def\childdocjob{#1}
  \input{#1}
}
%    \end{macrocode}

% \macro{\childdocby}
% The command |\childdocby| ....
%    \begin{macrocode}
\newcommand{\childdocby}[2][]
{
  \childdocdisable
  \childdoctrue
  \childdocmanualtrue
  \if?#1?\else
    \def\jobname{#2}
  \fi
  \def\childdocjob{#2}
  \input{#2}
  \endinput
}
%    \end{macrocode}

% \macro{\childdocforward}
% The command |\childdocforward| redirects
% compilation to the main file or
% (if the optional argument is given) a child file.
% Parameters are set as if the main file
% or a child file starting with |\childdocof| was compiled.
% Then compilation is handed over to the main file:
%    \begin{macrocode}
\newcommand{\childdocforward}[2][]
{
  \begingroup
    \if?#1?
      \def\childdoctmp
      {
        \def\childdocname{#2}
        \def\childdocjob{#2}
        \def\jobname{#2}
        \input{#2}
        \endinput
      }
    \else
      \def\childdoctmp
      {
        \childdocdisable
        \def\childdocname{#2}
        \childdoctrue
        \includeonly{#2}
        \def\childdocjob{#1}
        \def\jobname{#1}
        \input{#1}
        \endinput
      }
    \fi
    \expandafter
  \endgroup
  \childdoctmp
}
%    \end{macrocode}

% \macro{\childdocforwardprefix}
% The command |\childdocforwardprefix| redirects
% compilation to the main or a child file by means of a pattern.
% The prefix |#1| in the current filename is replaced by |#2|
% and the suffix of the current filename is kept
% (it is assumed that the filename does not contain the substring `|~~~|'
% which is used as a delimiter).
% Compilation is handed over to the new file by |\childdocforward|:
%    \begin{macrocode}
\newcommand{\childdocforwardprefix}[3][]
{
  \begingroup
    \def\childdocextract #2##1~~~{\def\childdoctmp{\childdocforward[#1]{#3##1}}}
    \expandafter\childdocextract\childdocname~~~
    \expandafter
  \endgroup
  \childdoctmp
}
%    \end{macrocode}

% \macro{\childdoc}
% The deprecated macro |\childdoc| is a legacy version of |\childdocmain|:
%    \begin{macrocode}
\newcommand{\childdoc}{\childdocmain}
%    \end{macrocode}

% \macro{\childdocredirect}
% The deprecated macro |\childdocredirect| is a legacy version
% of |\childdocforward| and |\childdocforwardprefix|:
%    \begin{macrocode}
\newcommand{\childdocredirect}[2][]
{
  \begingroup
    \if?#1?
      \def\childdoctmp{\childdocforward{#2}}
    \else
      \def\childdoctmp{\childdocforwardprefix{#1}{#2}}
    \fi
    \expandafter
  \endgroup
  \childdoctmp
}
%    \end{macrocode}

%\iffalse
%</package>
%\fi
%
\endinput
\childdocforward[cdocsamp]{cdocsch1}"|\\
% |latex -jobname cdocscl2 \|\\
% |  "\def\version{final}% \iffalse
%
% childdoc.dtx Copyright (C) 2017-2018 Niklas Beisert
%
% This work may be distributed and/or modified under the
% conditions of the LaTeX Project Public License, either version 1.3
% of this license or (at your option) any later version.
% The latest version of this license is in
%   http://www.latex-project.org/lppl.txt
% and version 1.3 or later is part of all distributions of LaTeX
% version 2005/12/01 or later.
%
% This work has the LPPL maintenance status `maintained'.
%
% The Current Maintainer of this work is Niklas Beisert.
%
% This work consists of the files childdoc.dtx and childdoc.ins
% and the derived files childdoc.def and cdocsamp.tex with
% cdocsch1.tex, cdocsch2.tex, cdocsdrf.tex, cdocsfn1.tex, cdocsfn2.tex.
%
%<package>\ifdefined\childdocmain\endinput\fi
%<package>\ProvidesFile{childdoc.def}[2018/12/30 v2.0 child document driver]
%<samplemain>\ProvidesFile{cdocsamp.tex}[2018/12/30 v2.0 sample for childdoc]
%<*driver>
%\ProvidesFile{childdoc.drv}[2018/12/30 v2.0 childdoc reference manual file]
\PassOptionsToClass{10pt,a4paper}{article}
\documentclass{ltxdoc}

\usepackage[margin=35mm]{geometry}
\usepackage{hyperref}
\usepackage{hyperxmp}
\usepackage[usenames]{color}

\hypersetup{colorlinks=true}
\hypersetup{pdfstartview=FitH}
\hypersetup{pdfpagemode=UseNone}
\hypersetup{pdfsource={}}
\hypersetup{pdflang={en-UK}}
\hypersetup{pdfcopyright={Copyright 2017-2018 Niklas Beisert.
  This work may be distributed and/or modified under the
  conditions of the LaTeX Project Public License, either version 1.3
  of this license or (at your option) any later version.}}
\hypersetup{pdflicenseurl={http://www.latex-project.org/lppl.txt}}
\hypersetup{pdfcontactaddress={ETH Zurich, ITP, HIT K,
  Wolfgang-Pauli-Strasse 27}}
\hypersetup{pdfcontactpostcode={8093}}
\hypersetup{pdfcontactcity={Zurich}}
\hypersetup{pdfcontactcountry={Switzerland}}
\hypersetup{pdfcontactemail={nbeisert@itp.phys.ethz.ch}}
\hypersetup{pdfcontacturl={http://people.phys.ethz.ch/\xmptilde nbeisert/}}

\newcommand{\secref}[1]{\hyperref[#1]{section \ref*{#1}}}

\parskip1ex
\parindent0pt
\let\olditemize\itemize
\def\itemize{\olditemize\parskip0pt}

\begin{document}

\title{The \textsf{childdoc} Package}
\hypersetup{pdftitle={The childdoc Package}}
\author{Niklas Beisert\\[2ex]
  Institut f\"ur Theoretische Physik\\
  Eidgen\"ossische Technische Hochschule Z\"urich\\
  Wolfgang-Pauli-Strasse 27, 8093 Z\"urich, Switzerland\\[1ex]
  \href{mailto:nbeisert@itp.phys.ethz.ch}
  {\texttt{nbeisert@itp.phys.ethz.ch}}}
\hypersetup{pdfauthor={Niklas Beisert}}
\hypersetup{pdfsubject={Manual for the LaTeX2e Package childdoc}}
\date{30 December 2018, \textsf{v2.0}}
\maketitle

\begin{abstract}\noindent
\textsf{childdoc} is a \LaTeXe{} package
that enables the direct compilation
of document sections included by |\include|
to individual files.
\end{abstract}

\begingroup
\parskip0ex
\tableofcontents
\endgroup

%%%%%%%%%%%%%%%%%%%%%%%%%%%%%%%%%%%%%%%%%%%%%%%%%%%%%%%%%%%%%%%%%%%%%%%%%%%%%%%%
%%%%%%%%%%%%%%%%%%%%%%%%%%%%%%%%%%%%%%%%%%%%%%%%%%%%%%%%%%%%%%%%%%%%%%%%%%%%%%%%
\section{Introduction}

\LaTeX{} provides a mechanism to structure a large document (such as a book)
into a main file and several child files (containing the chapters)
using the |\include| command.
This mechanism is beneficial for documents
which span hundreds of pages in order to
make the source file(s) more manageable.
Moreover, compilation can be restricted to
selected child files by means of the |\includeonly| command.
The latter feature can be used to reduce the compilation time while editing
(this was significantly more useful in the earlier days of \LaTeX{})
or to generate a smaller document which is easier to navigate.
Another application of |\includeonly| is to generate
documents consisting of selected parts of the complete document.

However, there are a few drawbacks of the plain |\include| mechanism:
\begin{itemize}
\item
The child files cannot be compiled on their own,
they can only be compiled via the main file.
A naive editing environment
(such as a text editor with an option
to have the current file processed by \LaTeX)
may require one to switch to the main file before compiling;
attempting to compile the child file produces errors.
\item
The main file must be modified (each time)
to adjust the |\includeonly| command
to the present needs. This easily leaves the main file in a messy state.
\item
The generated document will always carry the filename
of the main document. This is inconvenient if
several child files are to be compiled and
to be kept for distribution.
\end{itemize}

The present package provides a simple interface
to make child files individually compilable by \LaTeX{}.
Compiling a child file then has the same effect as compiling
the main file with an |\includeonly| command
to select the appropriate child.
Moreover the generated document will carry the name of the child
rather than the main file.
This resolves all three above issues.

This feature is meant to make the editing of books,
thesis documents and lecture notes somewhat more convenient.
However, the package can also be used efficiently for
composing a series of documents (such as exercise sheets)
which are typically distributed individually.
It then assists the author in generating the individual documents
(potentially in different versions)
as well as a document containing the collected series.
Another application is in developing style files
or other kinds of included material
where compilation of the style file could redirect
to a sample or test file.

%%%%%%%%%%%%%%%%%%%%%%%%%%%%%%%%%%%%%%%%%%%%%%%%%%%%%%%%%%%%%%%%%%%%%%%%%%%%%%%%
%%%%%%%%%%%%%%%%%%%%%%%%%%%%%%%%%%%%%%%%%%%%%%%%%%%%%%%%%%%%%%%%%%%%%%%%%%%%%%%%
\section{Usage}

First of all, the package \textsf{childdoc} is \emph{not} a standard
\LaTeXe{} |.sty| style file! Therefore it needs to be invoked in
a non-standard way.

%%%%%%%%%%%%%%%%%%%%%%%%%%%%%%%%%%%%%%%%%%%%%%%%%%%%%%%%%%%%%%%%%%%%%%%%%%%%%%%%
\subsection{Included Files}
\label{sec:include}

%%%%%%%%%%%%%%%%%%%%%%%%%%%%%%%%%%%%%%%%
\DescribeMacro{\childdocmain}
To use the package, add the commands
\begin{center}
\begin{tabular}{l}
|\input{childdoc.def}|\\
|\childdocmain{}|\\
\end{tabular}
\end{center}
at the very top of the main \LaTeX{} file,
in particular \emph{before} the |\documentclass| statement!
The argument of |\childdocmain| should be left empty
(but it must be present).

%%%%%%%%%%%%%%%%%%%%%%%%%%%%%%%%%%%%%%%%
\DescribeMacro{\childdocof}
Furthermore, add the commands
\begin{center}
\begin{tabular}{l}
|\input{childdoc.def}|\\
|\childdocof{|\textit{main}|}|\\
\end{tabular}
\end{center}
at the top of every child file \textit{child}
which is included by |\include{|\textit{child}|}|
from within the main file
(or at least for those files to be compiled individually).
The argument \textit{main} must be the filename of the main file.

There are a couple of
considerations in setting up the main and child documents:

%%%%%%%%%%%%%%%%%%%%%%%%%%%%%%%%%%%%%%%%
\paragraph{Restrictions.}

Please note the following restrictions:
\begin{itemize}
\item
|\childdocmain| must be called with one argument \textit{main}
to ensure compatibility with earlier version of the package.
It must either be empty (|\childdocmain{}|)
or precisely match the filename of the main file in which it is specified.
See \secref{sec:detection} for further information.
\item
The filename \textit{main} must be specified without the |.tex| extension.
\item
The filename \textit{main} is case sensitive
(even in case-insensitive file systems)
due to internal string comparison.
\item
The argument \textit{main} should be fully expanded, it cannot be a macro.
\item
Subdirectories and special characters should be avoided in filenames.
\item
The command |\childdocmain{|\textit{main}|}| must be followed by a whitespace.
It should not be followed immediately by another command
or by a comment mark `|%|'.
This is because the \TeX{} parser reads the token immediately following
the argument of |\childdocmain| and puts it
at the beginning of every child section;
however, a white\-space is ignored.
\end{itemize}

%%%%%%%%%%%%%%%%%%%%%%%%%%%%%%%%%%%%%%%%
\paragraph{Content of Main File.}

It is advisable to place all content in the child files included by |\include|.
Any output contained in the main file will appear in all child documents
unless suppressed manually;
it cannot be suppressed automatically by the |\includeonly| directive
and thus should normally be avoided.
A method to include some content in the main file
by means of conditional processing is described in \secref{sec:conditional}.

%%%%%%%%%%%%%%%%%%%%%%%%%%%%%%%%%%%%%%%%
\paragraph{Page Numbering.}

When only a part of the document is compiled,
the appropriate numbering of pages
(as well as other status parameters)
is determined from the |.aux| files.
The latter contain information from previous passes.
However this information needs to propagate through
all intermediate child documents.
Therefore the page numbering in child documents may well
be inconsistent until the complete document is compiled at least once.

A useful (if unconventional) way to always ensure a consistent
page numbering is to restart the numbering in each child document
and denote the pages by `\textit{child}|.|\textit{page}'
where \textit{child} represents the chapter/section number of the child file.
This can be achieved by the command
|\numberwithin{page}{|\textit{child}|}|
of the \textsf{amsmath} package
where \textit{child} can be |chapter| or |section|
depending on the chosen structuring.
Alternatively, one can modify the macro |\thepage| appropriately
and reset the counter |page| at the start of each child file.

%%%%%%%%%%%%%%%%%%%%%%%%%%%%%%%%%%%%%%%%%%%%%%%%%%%%%%%%%%%%%%%%%%%%%%%%%%%%%%%%
\subsection{Conditional Processing}
\label{sec:conditional}

The package provides a mechanism to compile different versions
of a document. To customise the versions further some conditional processing
can come in handy to distinguish which version is being compiled.
The package provides two macros to describe the compilation context:

%%%%%%%%%%%%%%%%%%%%%%%%%%%%%%%%%%%%%%%%
\DescribeMacro{\ifchilddoc}
The conditional |\ifchilddoc| distinguishes between the compilation of
child documents and the main document:
%
\begin{center}
|\ifchilddoc |\textit{child-code}| |[|\||else |\textit{main-code}]| \||fi|
\end{center}

%%%%%%%%%%%%%%%%%%%%%%%%%%%%%%%%%%%%%%%%
\DescribeMacro{\childdocname}
\DescribeMacro{\childdocjob}
The macro |\childdocname| contains the filename (without extension)
of the main or child file being processed.
Note that |\childdocjob| will always contain the name of the main file.

%%%%%%%%%%%%%%%%%%%%%%%%%%%%%%%%%%%%%%%%
\paragraph{Title Page.}

Conditional processing can be used to include a title or banner page
in the main document when proper precautions are taken.
Importantly, the code in the main file should ensure that the page counter
(as well as other status parameters which are stored in the |.aux| files)
takes the same value after the conditional processing.
Otherwise the page numbers may take divergent values
depending on which part is compiled.

For example, a title page could be declared by:
%
\begin{center}
\begin{tabular}{l}
|\ifchilddoc\||else|\\
|\addtocounter{page}{-1}|\\
\textit{code for title page}\\
|\newpage|\\
|\||fi|
\end{tabular}
\end{center}
%
A banner page for the child documents can be generated by:
%
\begin{center}
\begin{tabular}{l}
|\ifchilddoc|\\
|\addtocounter{page}{-1}|\\
\textit{code for banner page}\\
|\newpage|\\
|\||fi|
\end{tabular}
\end{center}
%
Here one could write a message such as:
\begin{center}
|This is the part \childdocname{} of \childdocjob{}.|
\end{center}

%%%%%%%%%%%%%%%%%%%%%%%%%%%%%%%%%%%%%%%%%%%%%%%%%%%%%%%%%%%%%%%%%%%%%%%%%%%%%%%%
\subsection{Flags}
\label{sec:flags}

The package makes it easy to generate different versions
of the main or child documents.
To this end compilation flags can be defined
and assigned different default values.
They will be particularly useful in conjunction
with the forwarding mechanism described in \secref{sec:forward}.

For example, it may be useful to have a flag |\version|
which can be set to |draft| or |final|.
The document source will contain some conditional code
depending on the value of |\version|.
Suppose further, the flag should default to |final| for the main file
and to |draft| for child files
which is a natural assignment for editing the document.
This is achieved by placing the following code
in the preamble of the main document
(below the |\childdocmain| directive):
%
\begin{center}
\begin{tabular}{l}
|\ifchilddoc|\\
|\providecommand{\version}{draft}|\\
|\||else|\\
|\providecommand{\version}{final}|\\
|\||fi|
\end{tabular}
\end{center}
%
The definition by |\providecommand| makes sure
that previous definitions are not overwritten.
Further statements |\providecommand{\version}{...}|
can thus be added before the above code to override it.

For the main file, one might add a line
(between |\childdocmain| and the above block)
%
\begin{center}
|%\ifchilddoc\||else\providecommand{\version}{draft}\||fi|
\end{center}
%
which can be uncommented to produce a draft version.
Likewise one can add a line to the very top of a child file
(above the |\childdocof{|\textit{main}|}| directive)
%
\begin{center}
|%\providecommand{\version}{final}|
\end{center}
%
which can be uncommented to produce the final version of this child document.

%%%%%%%%%%%%%%%%%%%%%%%%%%%%%%%%%%%%%%%%%%%%%%%%%%%%%%%%%%%%%%%%%%%%%%%%%%%%%%%%
\subsection{Forwarding}
\label{sec:forward}

Different versions of the main or child documents
using compilation flags as described in \secref{sec:flags}
can be (permanently) stored in different files
for convenient compilation, viewing and distribution.
To this end, the package defines a command
to pass on compilation to a different file:

%%%%%%%%%%%%%%%%%%%%%%%%%%%%%%%%%%%%%%%%
\DescribeMacro{\childdocforward}
The command |\childdocforward| redirects processing to
another source file:
%
\begin{center}
\begin{tabular}{l}
|\input{childdoc.def}|\\
|\childdocforward[|\textit{main}|]{|\textit{dest}|}|\\
\end{tabular}
\end{center}
%
The argument \textit{dest} is the destination file
(without extension).
It should be the main file or one of the child files.
Note that further \textsf{childdoc} directives
such as |\childdocof| and |\childdocforward|
in the indicated file will be processed in this form.
The optional argument \textit{main}
passes on directly to the main file \textit{main}
while pretending to compile the child \textit{dest}.
This form behaves as if \textit{dest}
issues |\childdocof{|\textit{main}|}| right away,
and no further \textsf{childdoc} directives will be processed.

%%%%%%%%%%%%%%%%%%%%%%%%%%%%%%%%%%%%%%%%
\DescribeMacro{\...prefix}
In the alternative form |\childdocforwardprefix|,
%
\begin{center}
\begin{tabular}{l}
|\input{childdoc.def}|\\
|\childdocforwardprefix[|\textit{main}|]{|\textit{prefix}|}{|\textit{dest}|}|
\end{tabular}
\end{center}
%
the destination file is determined by a pattern
depending on the current file:
To make this work, the current file must be called
`{\textit{prefix}\hspace{0.2em}\textit{suffix}}'
with \textit{prefix} matching precisely the argument.
Processing is then passed on to the file
`{\textit{dest}\hspace{0.2em}\textit{suffix}}'.
Surely, the same effect is achieved by
directly specifying the
argument `{\textit{dest}\hspace{0.2em}\textit{suffix}}'
in the first form.
However, that requires to set up a different file
for each child. With the alternative form of the command
all these files can have exactly the same content
which simplifies setting them up and maintaining them.

For example, the following file |draft.tex|
with a compilation flag |\version| as described in \secref{sec:flags}
compiles the main document as a draft:
%
\begin{center}
\begin{tabular}{l}
|\def\version{draft}|\\
|\input{childdoc.def}|\\
|\childdocforward{|\textit{main}|}|
\end{tabular}
\end{center}
%
Likewise, the following files |final|\textit{nn}|.tex|
compile the final version of the child document
|child|\textit{nn}|.tex|:
%
\begin{center}
\begin{tabular}{l}
|\def\version{final}|\\
|\input{childdoc.def}|\\
|\childdocforwardprefix{final}{child}|
\end{tabular}
\end{center}
%

Note that when several versions of a main file and/or of each child file
are to be generated, it may be convenient to set up a |Makefile| or
shell script to automatise the process.

%%%%%%%%%%%%%%%%%%%%%%%%%%%%%%%%%%%%%%%%%%%%%%%%%%%%%%%%%%%%%%%%%%%%%%%%%%%%%%%%
\subsection{Command Line Processing}
\label{sec:commandline}

The effect of redirection files can also be achieved by invoking
the \LaTeX{} compiler with a more elaborate command line.
Most conveniently this should be done as part
of a shell script or a |Makefile|.

When using \textsf{childdoc} in the main file, the following
command lines effectively perform a redirection
(note that depending on the shell being used,
backslashes may have to be doubled: `|\|' $\to$ `|\\|'):
%
\begin{center}
|... -jobname "|\textit{target}|" |\\|"|[\textit{flags}]%
|\input{childdoc.def}\childdocforward[|\textit{main}|]{|\textit{dest}|}"|
\end{center}
%
Here \textit{target} is the name of the output file,
\textit{main} is the name of the main file
and \textit{dest} is the name of the main or child file to be processed
(all filenames without extensions).
The optional argument \textit{main} can be omitted
if \textit{main} matches \textit{dest}.
Optionally, compilation \textit{flags} can be defined via |\def| commands.
This command line makes the \TeX{} engine believe
it is compiling the file \textit{target}
whose content is specified as the latter parameter.
The provided code then forwards the processing to
\textit{main} or \textit{dest} as described in \secref{sec:forward}.

%%%%%%%%%%%%%%%%%%%%%%%%%%%%%%%%%%%%%%%%%%%%%%%%%%%%%%%%%%%%%%%%%%%%%%%%%%%%%%%%
\subsection{Include by Input}
\label{sec:input}

Including child documents by |\include| has some restrictions by design.
Most notably, the content of a child document always occupies
its own set of pages; pages cannot be shared between child documents.
Usually, this behaviour makes perfect sense
because each child document contain an essential part of the document.
However, in some situations it may be desirable to compose
a document from a collection of parts
without having mandatory page breaks between then.
For this case, the package
provides a mechanism to include parts
by |\input| which can also be processed individually.
However, by construction this mechanism
requires manual handling of the content to be output.

%%%%%%%%%%%%%%%%%%%%%%%%%%%%%%%%%%%%%%%%
\DescribeMacro{\ifchilddocmanual}
The main file should be prepared as usual, see \secref{sec:include}.
However, the document body must make a distinction
between processing of an individual part and of the main document, e.g.:
%
\begin{center}
\begin{tabular}{l}
|\ifchilddocmanual|\\
|\input{\childdocname}|\\
|\||else|\\
\textit{document body with }|\input{|\textit{part}|}|\\
|\||fi|
\end{tabular}
\end{center}
%
The conditional |\ifchilddocmanual| is true whenever
a part to be included by |\input| is being compiled,
and the name of the part is stored in |\childdocname|.

%%%%%%%%%%%%%%%%%%%%%%%%%%%%%%%%%%%%%%%%
\DescribeMacro{\childdocby}
Each part to be included by |\input| should start with:
%
\begin{center}
\begin{tabular}{l}
|\input{childdoc.def}|\\
|\childdocby{|\textit{main}|}|\\
\end{tabular}
\end{center}
%
The directive |\childdocby| is similar to |\childdocof|
described in \secref{sec:include},
but the subsequent selection of content must be done manually.
To that end, both |\ifchilddoc| and |\ifchilddocmanual|
will be true upon processing of a part,
and the name of the part is stored in |\childdocname|.
Note that |\jobname| will be set to the filename of the current part
so that each part receives an individual |.aux| file
that does not interfere with the |.aux| file(s) of the main document.
This behaviour can be altered by the alternative form
|\childdocby[*]{|\textit{main}|}| (with a non-empty optional argument)
which uses the |.aux| file of the main document
by setting |\jobname| to \textit{main}.

%%%%%%%%%%%%%%%%%%%%%%%%%%%%%%%%%%%%%%%%%%%%%%%%%%%%%%%%%%%%%%%%%%%%%%%%%%%%%%%%
\subsection{Driver Development}
\label{sec:driver}

The \textsf{childdoc} mechanism can also be use for the development
of definition files such as \LaTeX{} styles or classes.
This case differs from the above setup with multiple parts
included by |\include| in that no |\includeonly| should be invoked.
This can be achieved by starting the include file
(before |\ProvidesPackage|) with:
%
\begin{center}
\begin{tabular}{l}
|\input{childdoc.def}|\\
|\childdocforward{|\textit{main}|}|\\
\end{tabular}
\end{center}
%
or alternatively with:
%
\begin{center}
\begin{tabular}{l}
|\input{childdoc.def}|\\
|\childdocby{|\textit{main}|}|\\
\end{tabular}
\end{center}
%
Both forms have slightly different effects as described above.
The main file is prepared as usual, see \secref{sec:include}.

%%%%%%%%%%%%%%%%%%%%%%%%%%%%%%%%%%%%%%%%%%%%%%%%%%%%%%%%%%%%%%%%%%%%%%%%%%%%%%%%
\subsection{Legacy Detection}
\label{sec:detection}

The directive |\childdocmain| in the main file can detect
whether the complete document or merely a child is to be compiled
even without using the directive |\childdocof|.
This method is deprecated because it is less robust
and there is no compelling reason to use it;
it is merely provided for backward compatibility
and it may be removed in future versions.

If the detection mechanism is to be used,
it is mandatory to correctly specify
the filename of the main file as the argument of |\childdocmain|:
%
\begin{center}
\begin{tabular}{l}
|\input{childdoc.def}|\\
|\childdocmain{|\textit{main}|}|\\
\end{tabular}
\end{center}
%
If |\jobname| does not match the argument \textit{main} of |\childdocmain|,
it is assumed that |\jobname| points to the child file to be compiled.
When using |\childdocmain| with the main file specified as argument,
it suffices to start a child file
with just |\input{|\textit{main}|}|
without loading of the package and using |\childdocof|.
If instead all processing is done
with the appropriate \textsf{childdoc} directives,
the argument of \textit{main} of |\childdocmain| can be empty.

An alternative version of the command line processing described
in \secref{sec:commandline} using the detection mechanism reads:
%
\begin{center}
|... -jobname "|\textit{target}|" "|[\textit{flags}]%
[|\def\jobname{|\textit{dest}|}|]|\input{|\textit{main}|}"|
\end{center}

%%%%%%%%%%%%%%%%%%%%%%%%%%%%%%%%%%%%%%%%%%%%%%%%%%%%%%%%%%%%%%%%%%%%%%%%%%%%%%%%
\subsection{Manual Code}
\label{sec:manual}

In case one cannot be certain whether the definitions file |childdoc.def|
is installed on the target \TeX{} distribution
and one prefers not to ship it,
it is conceivable to paste a few relevant commands into the sources.

To that end, drop all statements |\input{childdoc.def}|
and perform the replacements as outlined below.
Instead of |\childdocmain{|\textit{main}|}| add the following code
to the top of the main file:
%
\begin{center}
\begin{tabular}{l}
|\||ifdefined\childdocname\endinput\||fi\newif\ifchilddoc|\\
|\edef\childdocname{\scantokens\expandafter{\jobname\noexpand}}|\\
|\def\childdocmain{|\textit{main}|}\||ifx\childdocmain\childdocname\||else|\\
|\childdoctrue\includeonly{\childdocname}\let\jobname\childdocmain\||fi|\\
\end{tabular}
\end{center}
%
Instead of |\childdocof{|\textit{main}|}| just include the main file
at the top of each child file:
%
\begin{center}
|\input{|\textit{main}|}|
\end{center}
%
A simple redirection |\childdocforward{|\textit{dest}|}| is achieved by:
%
\begin{center}
|\def\jobname{|\textit{dest}|}\input{\jobname}|
\end{center}
%
The redirection with prefix
|\childdocforwardprefix[|\textit{prefix}|]{|\textit{dest}|}|
is accomplished by:
%
\begin{center}
\begin{tabular}{l}
|{\edef\jobname{\scantokens\expandafter{\jobname\noexpand}}|\\
|\def\redirectjob |\textit{prefix}|#1~~~{\gdef\jobname{|\textit{dest}|#1}}|\\
|\expandafter\redirectjob\jobname~~~}\input{\jobname}|
\end{tabular}
\end{center}

In an alternative approach,
child documents can be compiled by a specific command line
without additional code or specific definitions:
%
\begin{center}
|... -jobname "|\textit{target}|" "|[\textit{flags}]%
|\includeonly{|\textit{dest}|}\input{|\textit{main}|}"|
\end{center}
%

%%%%%%%%%%%%%%%%%%%%%%%%%%%%%%%%%%%%%%%%%%%%%%%%%%%%%%%%%%%%%%%%%%%%%%%%%%%%%%%%
%%%%%%%%%%%%%%%%%%%%%%%%%%%%%%%%%%%%%%%%%%%%%%%%%%%%%%%%%%%%%%%%%%%%%%%%%%%%%%%%
\section{Information}

%%%%%%%%%%%%%%%%%%%%%%%%%%%%%%%%%%%%%%%%%%%%%%%%%%%%%%%%%%%%%%%%%%%%%%%%%%%%%%%%
\subsection{Copyright}

Copyright \copyright{} 2017--2018 Niklas Beisert

This work may be distributed and/or modified under the
conditions of the \LaTeX{} Project Public License, either version 1.3
of this license or (at your option) any later version.
The latest version of this license is in
  \url{http://www.latex-project.org/lppl.txt}
and version 1.3 or later is part of all distributions of \LaTeX{}
version 2005/12/01 or later.

This work has the LPPL maintenance status `maintained'.

The Current Maintainer of this work is Niklas Beisert.

This work consists of the files |README.txt|, |childdoc.ins| and |childdoc.dtx|
as well as the derived files |childdoc.def|, |cdocsamp.tex|
with |cdocsch1.tex|, |cdocsch2.tex|, |cdocspt3.tex|, |cdocspt4.tex|,
|cdocsdrf.tex|, |cdocsfn1.tex|, |cdocsfn2.tex|
as well as |childdoc.pdf|.

%%%%%%%%%%%%%%%%%%%%%%%%%%%%%%%%%%%%%%%%%%%%%%%%%%%%%%%%%%%%%%%%%%%%%%%%%%%%%%%%
\subsection{Files and Installation}

The package consists of the files:
%
\begin{center}
\begin{tabular}{ll}
    |README.txt|   & readme file \\
    |childdoc.ins| & installation file \\
    |childdoc.dtx| & source file \\
    |childdoc.def| & definition file \\
    |cdocsamp.tex| & sample main file \\
    |cdocsch1.tex| & sample include file \\
    |cdocsch2.tex| & sample include file \\
    |cdocspt3.tex| & sample part file \\
    |cdocspt4.tex| & sample part file \\
    |cdocsdrf.tex| & sample redirection file \\
    |cdocsfn1.tex| & sample redirection file \\
    |cdocsfn2.tex| & sample redirection file \\
    |childdoc.pdf| & manual
\end{tabular}
\end{center}
%
The distribution consists of the files
|README.txt|, |childdoc.ins| and |childdoc.dtx|.
%
\begin{itemize}
\item
Run (pdf)\LaTeX{} on |childdoc.dtx|
to compile the manual |childdoc.pdf| (this file).
\item
Run \LaTeX{} on |childdoc.ins| to create the definitions file |childdoc.def|
and the sample |cdocsamp.tex| with include files
|cdocsch1.tex|, |cdocsch2.tex|, |cdocspt3.tex|, |cdocspt4.tex|,
|cdocsdrf.tex|, |cdocsfn1.tex|, |cdocsfn2.tex|.
Then copy the file |childdoc.def| to an appropriate directory of your \LaTeX{}
distribution, e.g.\ \textit{texmf-root}|/tex/latex/childdoc|.
\end{itemize}

%%%%%%%%%%%%%%%%%%%%%%%%%%%%%%%%%%%%%%%%%%%%%%%%%%%%%%%%%%%%%%%%%%%%%%%%%%%%%%%%
\subsection{Related CTAN Packages}

There are several other packages which offer a similar functionality:
%
\begin{itemize}
\item
The packages
\href{http://ctan.org/pkg/docmute}{\textsf{docmute}},
\href{http://ctan.org/pkg/includex}{\textsf{includex}} and
\href{http://ctan.org/pkg/standalone}{\textsf{standalone}}
provide commands to include only the document body of
a child file thus allowing both files to be compiled individually.
\item
The packages \href{http://ctan.org/pkg/subdocs}{\textsf{subdocs}}
and \href{http://ctan.org/pkg/subfiles}{\textsf{subfiles}}
provide structures in which the main and child documents can be
encapsulated and allowing them to be compiled individually.
The inclusion mechanism is different from the conventional |\include|.
\item
The package \href{http://ctan.org/pkg/combine}{\textsf{combine}}
is an elaborate solution to combine several documents into one.
\end{itemize}
%
See also the CTAN topic \href{http://ctan.org/topic/subdocs}{\textsf{subdocs}}
for further related packages.
The present package differs from the above solutions in that
a document structure constructed with the conventional |\include| mechanism
just needs two extra commands at the top of every file
such that all constituent files can be compiled individually.

%%%%%%%%%%%%%%%%%%%%%%%%%%%%%%%%%%%%%%%%%%%%%%%%%%%%%%%%%%%%%%%%%%%%%%%%%%%%%%%%
%\subsection{Feature Suggestions}
%
%The following is a list of features which may be useful for future
%versions of this package:
%%
%\begin{itemize}
%\item
%\ldots
%\end{itemize}

%%%%%%%%%%%%%%%%%%%%%%%%%%%%%%%%%%%%%%%%%%%%%%%%%%%%%%%%%%%%%%%%%%%%%%%%%%%%%%%%
\subsection{Revision History}

%%%%%%%%%%%%%%%%%%%%%%%%%%%%%%%%%%%%%%%%
\paragraph{v2.0:} 2018/12/30

\begin{itemize}
\item
immediate forward processing
\item
added |\childdocby| mechanism
\item
manual restructured
\end{itemize}

%%%%%%%%%%%%%%%%%%%%%%%%%%%%%%%%%%%%%%%%
\paragraph{v1.6:} 2018/01/17

\begin{itemize}
\item
application for development of include files
\item
corrections to manual
\end{itemize}

%%%%%%%%%%%%%%%%%%%%%%%%%%%%%%%%%%%%%%%%
\paragraph{v1.5:} 2017/05/21

\begin{itemize}
\item
more complete structuring introduced
\item
|\childdocof| introduced
\item
|\childdoc| renamed to |\childdocmain|
\item
|\childredirect| renamed to |\childdocforward| and |\childdocforwardprefix|
and functionality expanded
\end{itemize}

%%%%%%%%%%%%%%%%%%%%%%%%%%%%%%%%%%%%%%%%
\paragraph{v1.0:} 2017/04/27

\begin{itemize}
\item
manual and install package
\item
first version published on CTAN
\end{itemize}

%%%%%%%%%%%%%%%%%%%%%%%%%%%%%%%%%%%%%%%%
\paragraph{v0.6:} 2017/04/26

\begin{itemize}
\item
redirection mechanism added
\end{itemize}

%%%%%%%%%%%%%%%%%%%%%%%%%%%%%%%%%%%%%%%%
\paragraph{v0.5:} 2017/04/26

\begin{itemize}
\item
functionality in definition file
\end{itemize}


%%%%%%%%%%%%%%%%%%%%%%%%%%%%%%%%%%%%%%%%%%%%%%%%%%%%%%%%%%%%%%%%%%%%%%%%%%%%%%%%
%%%%%%%%%%%%%%%%%%%%%%%%%%%%%%%%%%%%%%%%%%%%%%%%%%%%%%%%%%%%%%%%%%%%%%%%%%%%%%%%
%%%%%%%%%%%%%%%%%%%%%%%%%%%%%%%%%%%%%%%%%%%%%%%%%%%%%%%%%%%%%%%%%%%%%%%%%%%%%%%%
\appendix

\settowidth\MacroIndent{\rmfamily\scriptsize 000\ }

 \DocInput{childdoc.dtx}

\end{document}
%</driver>
% \fi
%
% %%%%%%%%%%%%%%%%%%%%%%%%%%%%%%%%%%%%%%%%%%%%%%%%%%%%%%%%%%%%%%%%%%%%%%%%%%%%%%
% %%%%%%%%%%%%%%%%%%%%%%%%%%%%%%%%%%%%%%%%%%%%%%%%%%%%%%%%%%%%%%%%%%%%%%%%%%%%%%
% \section{Sample}
%\iffalse
%<*samplemain>
%\fi
%
% The following presents a sample document
% with two chapters, two parts, a title page,
% a compile flag as well as three forwarding files to set the flag.
% It consists of eight |.tex| files:
% \begin{center}
% \begin{tabular}{ll}
% |cdocsamp.tex|&main file\\
% |cdocsch1.tex|&include file for chapter 1\\
% |cdocsch2.tex|&include file for chapter 2\\
% |cdocspt3.tex|&include file for part 3\\
% |cdocspt4.tex|&include file for part 4\\
% |cdocsdrf.tex|&forwarding file for main file in draft mode\\
% |cdocsfi1.tex|&forwarding file for final version of chapter 1\\
% |cdocsfi2.tex|&forwarding file for final version of chapter 2\\
% \end{tabular}
% \end{center}
% Each of the eight files can be compiled directly by the \LaTeX{} compiler.
%
% %%%%%%%%%%%%%%%%%%%%%%%%%%%%%%%%%%%%%%
% \paragraph{Main File.}
%
% The main file is called |cdocsamp.tex|.
%
% Load the \textsf{childdoc} definitions and
% declare the filename for the main document:
%    \begin{macrocode}
\input{childdoc.def}
\childdocmain{}
%    \end{macrocode}

% Optional override for |\version| flag:
%    \begin{macrocode}
%%\ifchilddoc\else\providecommand{\version}{draft}\fi
%    \end{macrocode}

% Define the default values for the |\version| flag
% (|final| for the main file and |draft| for childs):
%    \begin{macrocode}
\ifchilddoc
\providecommand{\version}{draft}
\else
\providecommand{\version}{final}
\fi
%    \end{macrocode}

% Load the standard document class:
%    \begin{macrocode}
\documentclass[12pt]{article}
%    \end{macrocode}

% Start the document body:
%    \begin{macrocode}
\begin{document}
%    \end{macrocode}

% Declare a title page.
% Print title, part of document being processed and version flag:
%    \begin{macrocode}
\addtocounter{page}{-1}
\begin{center}
{\LARGE\bfseries{}childdoc example\par}
\vspace{1cm}
\ifchilddoc
\ifchilddocmanual part\else chapter\fi:
`\childdocname' of `\childdocjob'\par
\else
main document: `\childdocjob'\par
\fi
version: \version\par
\end{center}
\newpage
%    \end{macrocode}

% Manually include selected file,
% otherwise process as usual:
%    \begin{macrocode}
\ifchilddocmanual
\section*{part `\childdocname'}
\input{\childdocname}
\else
%    \end{macrocode}

% Include the two chapters:
%    \begin{macrocode}
\include{cdocsch1}
\include{cdocsch2}
%    \end{macrocode}

% Include the two parts unless only chapters should be displayed:
%    \begin{macrocode}
\ifchilddoc\else
\section{part three}
\input{cdocspt3}
\section{part four}
\input{cdocspt4}
\fi
%    \end{macrocode}

% Process as usual until here:
%    \begin{macrocode}
\fi
%    \end{macrocode}

% End of document body:
%    \begin{macrocode}
\end{document}
%    \end{macrocode}
%\iffalse
%</samplemain>
%\fi
%
% %%%%%%%%%%%%%%%%%%%%%%%%%%%%%%%%%%%%%%
% \paragraph{Chapter Include Files.}
%
% The include files are called |cdocsch1.tex| and |cdocsch2.tex|.
%
%\iffalse
%<*samplechap1|samplechap2>
%\fi

% Optional override for |\version| flag:
%    \begin{macrocode}
%%\providecommand{\version}{final}
%    \end{macrocode}

% Include the main document:
%    \begin{macrocode}
\input{childdoc.def}
\childdocof{cdocsamp}
%    \end{macrocode}

%\iffalse
%</samplechap1|samplechap2>
%\fi
%
%\iffalse
%<*samplechap1>
%\fi
% Some text for chapter 1:
%    \begin{macrocode}
\section{one}
some text in chapter one
%    \end{macrocode}

%\iffalse
%</samplechap1>
%\fi
% Some text for chapter 2:
%\iffalse
%<*samplechap2>
%\fi
%    \begin{macrocode}
\section{two}
more text in chapter two
%    \end{macrocode}

%\iffalse
%</samplechap2>
%\fi
%
% %%%%%%%%%%%%%%%%%%%%%%%%%%%%%%%%%%%%%%
% \paragraph{Part Include Files.}
%
% The include files are called |cdocspt3.tex| and |cdocspt4.tex|.
%
%\iffalse
%<*samplepart3|samplepart4>
%\fi

% Optional override for |\version| flag:
%    \begin{macrocode}
%%\providecommand{\version}{final}
%    \end{macrocode}

% Include the main document:
%    \begin{macrocode}
\input{childdoc.def}
\childdocby{cdocsamp}
%    \end{macrocode}

%\iffalse
%</samplepart3|samplepart4>
%\fi
%
%\iffalse
%<*samplepart3>
%\fi
% Some text for part 3:
%    \begin{macrocode}
some text in part three
%    \end{macrocode}

%\iffalse
%</samplepart3>
%\fi
% Some text for part 4:
%\iffalse
%<*samplepart4>
%\fi
%    \begin{macrocode}
more text in part four
%    \end{macrocode}

%\iffalse
%</samplepart4>
%\fi
%
% %%%%%%%%%%%%%%%%%%%%%%%%%%%%%%%%%%%%%%
% \paragraph{Forwarding for a Complete Draft.}
%
% The following forwarding file |cdocsdrf.tex|
% compiles the main document in draft mode:
%\iffalse
%<*sampledraft>
%\fi
%    \begin{macrocode}
\def\version{draft}
\input{childdoc.def}
\childdocforward{cdocsamp}
%    \end{macrocode}

%\iffalse
%</sampledraft>
%\fi
%
% %%%%%%%%%%%%%%%%%%%%%%%%%%%%%%%%%%%%%%
% \paragraph{Forwarding for Final Version of the Chapters.}
%
% The following forwarding files |cdocsfn1.tex| and |cdocsfn2.tex|
% (with identical content)
% compile the final versions of the child documents
% |cdocsch1.tex| and |cdocsch2.tex|, respectively:
%\iffalse
%<*samplefinal>
%\fi
%    \begin{macrocode}
\def\version{final}
\input{childdoc.def}
\childdocforwardprefix[cdocsamp]{cdocsfn}{cdocsch}
%    \end{macrocode}

%\iffalse
%</samplefinal>
%\fi
%
% %%%%%%%%%%%%%%%%%%%%%%%%%%%%%%%%%%%%%%
% \paragraph{Command Line Processing.}
%
% The following three command lines generate the output files
% |cdocscld|, |cdocscl1| and |cdocscl2|
% which should be identical to
% |cdocsdrf|, |cdocsch1| and |cdocsfn2|, respectively:
% \begin{center}
% \begin{tabular}{l}
% |latex -jobname cdocscld \|\\
% |  "\def\version{draft}\input{childdoc.def}\childdocforward{cdocsamp}"|\\
% |latex -jobname cdocscl1 \|\\
% |  "\input{childdoc.def}\childdocforward[cdocsamp]{cdocsch1}"|\\
% |latex -jobname cdocscl2 \|\\
% |  "\def\version{final}\input{childdoc.def}\childdocforward{cdocsch2}"|
% \end{tabular}
% \end{center}
% Note that the trailing backslash on each first line
% merely continues the input to the second line
% (for convenient cut ant paste).
% Furthermore, the command |latex| can be replaced by any
% of its alternative versions such as |pdflatex|.
%
% %%%%%%%%%%%%%%%%%%%%%%%%%%%%%%%%%%%%%%%%%%%%%%%%%%%%%%%%%%%%%%%%%%%%%%%%%%%%%%
% %%%%%%%%%%%%%%%%%%%%%%%%%%%%%%%%%%%%%%%%%%%%%%%%%%%%%%%%%%%%%%%%%%%%%%%%%%%%%%
% \section{Implementation}
%\iffalse
%<*package>
%\fi
%
% This section describes the definitions file |childdoc.def|.

% The definitions cannot be loaded using |\usepackage| or |\RequirePackage|
% which has a mechanism to prevent loading a style file more than once.
% When loading the definitions by means of |\input|
% multiple instances have to be prevented manually:
%\iffalse
%This code needs to be before the `\ProvidesFile' directive
%which is defined at the beginning of this file.
%Therefore it is also placed there and commented out here.
%</package>
%<*discard>
%\fi
%    \begin{macrocode}
\ifdefined\childdocmain\endinput\fi
%    \end{macrocode}
%\iffalse
%</discard>
%<*package>
%\fi
%
% \macro{\ifchilddoc}
% \macro{\ifchilddocmanual}
% The conditional |\ifchilddoc| tells whether a
% child (true) or main (false) document is being compiled.
% The conditional |\ifchilddocmanual| tells whether
% the |\includeonly| mechanism is used (false) or
% the selection of child files must be performed manually (true).
% The definitions initialise to false:
%    \begin{macrocode}
\newif\ifchilddoc
\newif\ifchilddocmanual
%    \end{macrocode}

% \macro{\childdocname}
% \macro{\childdocjob}
% The macro |\childdocname| stores the name of the main document
% to be compiled. The macro |\childdocjob| stores the name of
% the document on which the \LaTeX{} compiler was originally invoked.
% The content of |\jobname| cannot be compared
% to filenames specified in the source due to different catcodes.
% The following code rescans |\jobname|, stores the result
% in |\childdocname| and saves a copy in |\childdocjob|:
%    \begin{macrocode}
\edef\childdocname{\scantokens\expandafter{\jobname\noexpand}}
\let\childdocjob\childdocname
%    \end{macrocode}

% \macro{\childdocdisable}
% The macro |\childdocdisable| prevents the main file
% from being processed more than once.
% At this stage, the main document command |\childdocmain|
% is assumed to be called once again where it should do nothing.
% Any subsequent call to it should prevent
% a secondary processing of the main document
% It overwrites the forwarding commands
% |\childdocof| and |\childdocforward|
% with empty macros to prevent further inclusions of the main document:
%    \begin{macrocode}
\newcommand{\childdocdisable}
{
  \renewcommand{\childdocmain}[1]{\renewcommand{\childdocmain}[1]{\endinput}}
  \renewcommand{\childdocof}[1]{}
  \renewcommand{\childdocby}[2][]{}
  \renewcommand{\childdocforward}[2][]{}
  \renewcommand{\childdocdisable}{}
}
%    \end{macrocode}

% \macro{\childdocmain}
% The macro |\childdocmain| is to be called at the top of the main file
% with nothing or the main filename (without extension) as argument.
% First, it breaks loops.
% If the argument is not empty and does not match |\childdocname|
% (which is set by the first inclusion of |childdoc.def|),
% |\ifchilddoc| is set to true, |\includeonly| is applied to the child file
% and |\jobname| is set to the main file
% (for proper handling of |.aux| files):
%    \begin{macrocode}
\newcommand{\childdocmain}[1]
{
  \childdocdisable\childdocmain{}
  \if?#1?\else
    \begingroup
      \def\childdoctmp{#1}
      \ifx\childdoctmp\childdocname
        \def\childdoctmp{}
      \else
        \def\childdoctmp
        {
          \childdoctrue
          \includeonly{\childdocname}
          \def\childdocjob{#1}
          \def\jobname{#1}
        }
      \fi
      \expandafter
    \endgroup
    \childdoctmp
  \fi
}
%    \end{macrocode}

% \macro{\childdocof}
% The command |\childdocof| redirects
% compilation to the main file |#1|.
%    \begin{macrocode}
\newcommand{\childdocof}[1]
{
  \childdocdisable
  \childdoctrue
  \includeonly{\childdocname}
  \def\jobname{#1}
  \def\childdocjob{#1}
  \input{#1}
}
%    \end{macrocode}

% \macro{\childdocby}
% The command |\childdocby| ....
%    \begin{macrocode}
\newcommand{\childdocby}[2][]
{
  \childdocdisable
  \childdoctrue
  \childdocmanualtrue
  \if?#1?\else
    \def\jobname{#2}
  \fi
  \def\childdocjob{#2}
  \input{#2}
  \endinput
}
%    \end{macrocode}

% \macro{\childdocforward}
% The command |\childdocforward| redirects
% compilation to the main file or
% (if the optional argument is given) a child file.
% Parameters are set as if the main file
% or a child file starting with |\childdocof| was compiled.
% Then compilation is handed over to the main file:
%    \begin{macrocode}
\newcommand{\childdocforward}[2][]
{
  \begingroup
    \if?#1?
      \def\childdoctmp
      {
        \def\childdocname{#2}
        \def\childdocjob{#2}
        \def\jobname{#2}
        \input{#2}
        \endinput
      }
    \else
      \def\childdoctmp
      {
        \childdocdisable
        \def\childdocname{#2}
        \childdoctrue
        \includeonly{#2}
        \def\childdocjob{#1}
        \def\jobname{#1}
        \input{#1}
        \endinput
      }
    \fi
    \expandafter
  \endgroup
  \childdoctmp
}
%    \end{macrocode}

% \macro{\childdocforwardprefix}
% The command |\childdocforwardprefix| redirects
% compilation to the main or a child file by means of a pattern.
% The prefix |#1| in the current filename is replaced by |#2|
% and the suffix of the current filename is kept
% (it is assumed that the filename does not contain the substring `|~~~|'
% which is used as a delimiter).
% Compilation is handed over to the new file by |\childdocforward|:
%    \begin{macrocode}
\newcommand{\childdocforwardprefix}[3][]
{
  \begingroup
    \def\childdocextract #2##1~~~{\def\childdoctmp{\childdocforward[#1]{#3##1}}}
    \expandafter\childdocextract\childdocname~~~
    \expandafter
  \endgroup
  \childdoctmp
}
%    \end{macrocode}

% \macro{\childdoc}
% The deprecated macro |\childdoc| is a legacy version of |\childdocmain|:
%    \begin{macrocode}
\newcommand{\childdoc}{\childdocmain}
%    \end{macrocode}

% \macro{\childdocredirect}
% The deprecated macro |\childdocredirect| is a legacy version
% of |\childdocforward| and |\childdocforwardprefix|:
%    \begin{macrocode}
\newcommand{\childdocredirect}[2][]
{
  \begingroup
    \if?#1?
      \def\childdoctmp{\childdocforward{#2}}
    \else
      \def\childdoctmp{\childdocforwardprefix{#1}{#2}}
    \fi
    \expandafter
  \endgroup
  \childdoctmp
}
%    \end{macrocode}

%\iffalse
%</package>
%\fi
%
\endinput
\childdocforward{cdocsch2}"|
% \end{tabular}
% \end{center}
% Note that the trailing backslash on each first line
% merely continues the input to the second line
% (for convenient cut ant paste).
% Furthermore, the command |latex| can be replaced by any
% of its alternative versions such as |pdflatex|.
%
% %%%%%%%%%%%%%%%%%%%%%%%%%%%%%%%%%%%%%%%%%%%%%%%%%%%%%%%%%%%%%%%%%%%%%%%%%%%%%%
% %%%%%%%%%%%%%%%%%%%%%%%%%%%%%%%%%%%%%%%%%%%%%%%%%%%%%%%%%%%%%%%%%%%%%%%%%%%%%%
% \section{Implementation}
%\iffalse
%<*package>
%\fi
%
% This section describes the definitions file |childdoc.def|.

% The definitions cannot be loaded using |\usepackage| or |\RequirePackage|
% which has a mechanism to prevent loading a style file more than once.
% When loading the definitions by means of |\input|
% multiple instances have to be prevented manually:
%\iffalse
%This code needs to be before the `\ProvidesFile' directive
%which is defined at the beginning of this file.
%Therefore it is also placed there and commented out here.
%</package>
%<*discard>
%\fi
%    \begin{macrocode}
\ifdefined\childdocmain\endinput\fi
%    \end{macrocode}
%\iffalse
%</discard>
%<*package>
%\fi
%
% \macro{\ifchilddoc}
% \macro{\ifchilddocmanual}
% The conditional |\ifchilddoc| tells whether a
% child (true) or main (false) document is being compiled.
% The conditional |\ifchilddocmanual| tells whether
% the |\includeonly| mechanism is used (false) or
% the selection of child files must be performed manually (true).
% The definitions initialise to false:
%    \begin{macrocode}
\newif\ifchilddoc
\newif\ifchilddocmanual
%    \end{macrocode}

% \macro{\childdocname}
% \macro{\childdocjob}
% The macro |\childdocname| stores the name of the main document
% to be compiled. The macro |\childdocjob| stores the name of
% the document on which the \LaTeX{} compiler was originally invoked.
% The content of |\jobname| cannot be compared
% to filenames specified in the source due to different catcodes.
% The following code rescans |\jobname|, stores the result
% in |\childdocname| and saves a copy in |\childdocjob|:
%    \begin{macrocode}
\edef\childdocname{\scantokens\expandafter{\jobname\noexpand}}
\let\childdocjob\childdocname
%    \end{macrocode}

% \macro{\childdocdisable}
% The macro |\childdocdisable| prevents the main file
% from being processed more than once.
% At this stage, the main document command |\childdocmain|
% is assumed to be called once again where it should do nothing.
% Any subsequent call to it should prevent
% a secondary processing of the main document
% It overwrites the forwarding commands
% |\childdocof| and |\childdocforward|
% with empty macros to prevent further inclusions of the main document:
%    \begin{macrocode}
\newcommand{\childdocdisable}
{
  \renewcommand{\childdocmain}[1]{\renewcommand{\childdocmain}[1]{\endinput}}
  \renewcommand{\childdocof}[1]{}
  \renewcommand{\childdocby}[2][]{}
  \renewcommand{\childdocforward}[2][]{}
  \renewcommand{\childdocdisable}{}
}
%    \end{macrocode}

% \macro{\childdocmain}
% The macro |\childdocmain| is to be called at the top of the main file
% with nothing or the main filename (without extension) as argument.
% First, it breaks loops.
% If the argument is not empty and does not match |\childdocname|
% (which is set by the first inclusion of |childdoc.def|),
% |\ifchilddoc| is set to true, |\includeonly| is applied to the child file
% and |\jobname| is set to the main file
% (for proper handling of |.aux| files):
%    \begin{macrocode}
\newcommand{\childdocmain}[1]
{
  \childdocdisable\childdocmain{}
  \if?#1?\else
    \begingroup
      \def\childdoctmp{#1}
      \ifx\childdoctmp\childdocname
        \def\childdoctmp{}
      \else
        \def\childdoctmp
        {
          \childdoctrue
          \includeonly{\childdocname}
          \def\childdocjob{#1}
          \def\jobname{#1}
        }
      \fi
      \expandafter
    \endgroup
    \childdoctmp
  \fi
}
%    \end{macrocode}

% \macro{\childdocof}
% The command |\childdocof| redirects
% compilation to the main file |#1|.
%    \begin{macrocode}
\newcommand{\childdocof}[1]
{
  \childdocdisable
  \childdoctrue
  \includeonly{\childdocname}
  \def\jobname{#1}
  \def\childdocjob{#1}
  \input{#1}
}
%    \end{macrocode}

% \macro{\childdocby}
% The command |\childdocby| ....
%    \begin{macrocode}
\newcommand{\childdocby}[2][]
{
  \childdocdisable
  \childdoctrue
  \childdocmanualtrue
  \if?#1?\else
    \def\jobname{#2}
  \fi
  \def\childdocjob{#2}
  \input{#2}
  \endinput
}
%    \end{macrocode}

% \macro{\childdocforward}
% The command |\childdocforward| redirects
% compilation to the main file or
% (if the optional argument is given) a child file.
% Parameters are set as if the main file
% or a child file starting with |\childdocof| was compiled.
% Then compilation is handed over to the main file:
%    \begin{macrocode}
\newcommand{\childdocforward}[2][]
{
  \begingroup
    \if?#1?
      \def\childdoctmp
      {
        \def\childdocname{#2}
        \def\childdocjob{#2}
        \def\jobname{#2}
        \input{#2}
        \endinput
      }
    \else
      \def\childdoctmp
      {
        \childdocdisable
        \def\childdocname{#2}
        \childdoctrue
        \includeonly{#2}
        \def\childdocjob{#1}
        \def\jobname{#1}
        \input{#1}
        \endinput
      }
    \fi
    \expandafter
  \endgroup
  \childdoctmp
}
%    \end{macrocode}

% \macro{\childdocforwardprefix}
% The command |\childdocforwardprefix| redirects
% compilation to the main or a child file by means of a pattern.
% The prefix |#1| in the current filename is replaced by |#2|
% and the suffix of the current filename is kept
% (it is assumed that the filename does not contain the substring `|~~~|'
% which is used as a delimiter).
% Compilation is handed over to the new file by |\childdocforward|:
%    \begin{macrocode}
\newcommand{\childdocforwardprefix}[3][]
{
  \begingroup
    \def\childdocextract #2##1~~~{\def\childdoctmp{\childdocforward[#1]{#3##1}}}
    \expandafter\childdocextract\childdocname~~~
    \expandafter
  \endgroup
  \childdoctmp
}
%    \end{macrocode}

% \macro{\childdoc}
% The deprecated macro |\childdoc| is a legacy version of |\childdocmain|:
%    \begin{macrocode}
\newcommand{\childdoc}{\childdocmain}
%    \end{macrocode}

% \macro{\childdocredirect}
% The deprecated macro |\childdocredirect| is a legacy version
% of |\childdocforward| and |\childdocforwardprefix|:
%    \begin{macrocode}
\newcommand{\childdocredirect}[2][]
{
  \begingroup
    \if?#1?
      \def\childdoctmp{\childdocforward{#2}}
    \else
      \def\childdoctmp{\childdocforwardprefix{#1}{#2}}
    \fi
    \expandafter
  \endgroup
  \childdoctmp
}
%    \end{macrocode}

%\iffalse
%</package>
%\fi
%
\endinput
\childdocforward{cdocsch2}"|
% \end{tabular}
% \end{center}
% Note that the trailing backslash on each first line
% merely continues the input to the second line
% (for convenient cut ant paste).
% Furthermore, the command |latex| can be replaced by any
% of its alternative versions such as |pdflatex|.
%
% %%%%%%%%%%%%%%%%%%%%%%%%%%%%%%%%%%%%%%%%%%%%%%%%%%%%%%%%%%%%%%%%%%%%%%%%%%%%%%
% %%%%%%%%%%%%%%%%%%%%%%%%%%%%%%%%%%%%%%%%%%%%%%%%%%%%%%%%%%%%%%%%%%%%%%%%%%%%%%
% \section{Implementation}
%\iffalse
%<*package>
%\fi
%
% This section describes the definitions file |childdoc.def|.

% The definitions cannot be loaded using |\usepackage| or |\RequirePackage|
% which has a mechanism to prevent loading a style file more than once.
% When loading the definitions by means of |\input|
% multiple instances have to be prevented manually:
%\iffalse
%This code needs to be before the `\ProvidesFile' directive
%which is defined at the beginning of this file.
%Therefore it is also placed there and commented out here.
%</package>
%<*discard>
%\fi
%    \begin{macrocode}
\ifdefined\childdocmain\endinput\fi
%    \end{macrocode}
%\iffalse
%</discard>
%<*package>
%\fi
%
% \macro{\ifchilddoc}
% \macro{\ifchilddocmanual}
% The conditional |\ifchilddoc| tells whether a
% child (true) or main (false) document is being compiled.
% The conditional |\ifchilddocmanual| tells whether
% the |\includeonly| mechanism is used (false) or
% the selection of child files must be performed manually (true).
% The definitions initialise to false:
%    \begin{macrocode}
\newif\ifchilddoc
\newif\ifchilddocmanual
%    \end{macrocode}

% \macro{\childdocname}
% \macro{\childdocjob}
% The macro |\childdocname| stores the name of the main document
% to be compiled. The macro |\childdocjob| stores the name of
% the document on which the \LaTeX{} compiler was originally invoked.
% The content of |\jobname| cannot be compared
% to filenames specified in the source due to different catcodes.
% The following code rescans |\jobname|, stores the result
% in |\childdocname| and saves a copy in |\childdocjob|:
%    \begin{macrocode}
\edef\childdocname{\scantokens\expandafter{\jobname\noexpand}}
\let\childdocjob\childdocname
%    \end{macrocode}

% \macro{\childdocdisable}
% The macro |\childdocdisable| prevents the main file
% from being processed more than once.
% At this stage, the main document command |\childdocmain|
% is assumed to be called once again where it should do nothing.
% Any subsequent call to it should prevent
% a secondary processing of the main document
% It overwrites the forwarding commands
% |\childdocof| and |\childdocforward|
% with empty macros to prevent further inclusions of the main document:
%    \begin{macrocode}
\newcommand{\childdocdisable}
{
  \renewcommand{\childdocmain}[1]{\renewcommand{\childdocmain}[1]{\endinput}}
  \renewcommand{\childdocof}[1]{}
  \renewcommand{\childdocby}[2][]{}
  \renewcommand{\childdocforward}[2][]{}
  \renewcommand{\childdocdisable}{}
}
%    \end{macrocode}

% \macro{\childdocmain}
% The macro |\childdocmain| is to be called at the top of the main file
% with nothing or the main filename (without extension) as argument.
% First, it breaks loops.
% If the argument is not empty and does not match |\childdocname|
% (which is set by the first inclusion of |childdoc.def|),
% |\ifchilddoc| is set to true, |\includeonly| is applied to the child file
% and |\jobname| is set to the main file
% (for proper handling of |.aux| files):
%    \begin{macrocode}
\newcommand{\childdocmain}[1]
{
  \childdocdisable\childdocmain{}
  \if?#1?\else
    \begingroup
      \def\childdoctmp{#1}
      \ifx\childdoctmp\childdocname
        \def\childdoctmp{}
      \else
        \def\childdoctmp
        {
          \childdoctrue
          \includeonly{\childdocname}
          \def\childdocjob{#1}
          \def\jobname{#1}
        }
      \fi
      \expandafter
    \endgroup
    \childdoctmp
  \fi
}
%    \end{macrocode}

% \macro{\childdocof}
% The command |\childdocof| redirects
% compilation to the main file |#1|.
%    \begin{macrocode}
\newcommand{\childdocof}[1]
{
  \childdocdisable
  \childdoctrue
  \includeonly{\childdocname}
  \def\jobname{#1}
  \def\childdocjob{#1}
  \input{#1}
}
%    \end{macrocode}

% \macro{\childdocby}
% The command |\childdocby| ....
%    \begin{macrocode}
\newcommand{\childdocby}[2][]
{
  \childdocdisable
  \childdoctrue
  \childdocmanualtrue
  \if?#1?\else
    \def\jobname{#2}
  \fi
  \def\childdocjob{#2}
  \input{#2}
  \endinput
}
%    \end{macrocode}

% \macro{\childdocforward}
% The command |\childdocforward| redirects
% compilation to the main file or
% (if the optional argument is given) a child file.
% Parameters are set as if the main file
% or a child file starting with |\childdocof| was compiled.
% Then compilation is handed over to the main file:
%    \begin{macrocode}
\newcommand{\childdocforward}[2][]
{
  \begingroup
    \if?#1?
      \def\childdoctmp
      {
        \def\childdocname{#2}
        \def\childdocjob{#2}
        \def\jobname{#2}
        \input{#2}
        \endinput
      }
    \else
      \def\childdoctmp
      {
        \childdocdisable
        \def\childdocname{#2}
        \childdoctrue
        \includeonly{#2}
        \def\childdocjob{#1}
        \def\jobname{#1}
        \input{#1}
        \endinput
      }
    \fi
    \expandafter
  \endgroup
  \childdoctmp
}
%    \end{macrocode}

% \macro{\childdocforwardprefix}
% The command |\childdocforwardprefix| redirects
% compilation to the main or a child file by means of a pattern.
% The prefix |#1| in the current filename is replaced by |#2|
% and the suffix of the current filename is kept
% (it is assumed that the filename does not contain the substring `|~~~|'
% which is used as a delimiter).
% Compilation is handed over to the new file by |\childdocforward|:
%    \begin{macrocode}
\newcommand{\childdocforwardprefix}[3][]
{
  \begingroup
    \def\childdocextract #2##1~~~{\def\childdoctmp{\childdocforward[#1]{#3##1}}}
    \expandafter\childdocextract\childdocname~~~
    \expandafter
  \endgroup
  \childdoctmp
}
%    \end{macrocode}

% \macro{\childdoc}
% The deprecated macro |\childdoc| is a legacy version of |\childdocmain|:
%    \begin{macrocode}
\newcommand{\childdoc}{\childdocmain}
%    \end{macrocode}

% \macro{\childdocredirect}
% The deprecated macro |\childdocredirect| is a legacy version
% of |\childdocforward| and |\childdocforwardprefix|:
%    \begin{macrocode}
\newcommand{\childdocredirect}[2][]
{
  \begingroup
    \if?#1?
      \def\childdoctmp{\childdocforward{#2}}
    \else
      \def\childdoctmp{\childdocforwardprefix{#1}{#2}}
    \fi
    \expandafter
  \endgroup
  \childdoctmp
}
%    \end{macrocode}

%\iffalse
%</package>
%\fi
%
\endinput

\childdocby{exfserm}
%    \end{macrocode}
% and end with |\endinput|.
% Then compose the problem sheet
% by including the appropriate set of problem files
% via |\input{exfserp|\textit{nn}|}|.
%
% In the example, a sheet file |exfser03.tex|
% includes two problem files |exfserpe.tex| and |exfserpf.tex|.
%
%\iffalse
%</discard>
%\fi
%
% %%%%%%%%%%%%%%%%%%%%%%%%%%%%%%%%%%%%%%%%%%%%%%%%%%%%%%%%%%%%%%%%%%%%%%%%%%%%%%
% %%%%%%%%%%%%%%%%%%%%%%%%%%%%%%%%%%%%%%%%%%%%%%%%%%%%%%%%%%%%%%%%%%%%%%%%%%%%%%
% \subsection{Make Scripts}
% \label{sec:samplemultimake}
%\iffalse
%<*samplemultiscript>
%\fi
%
% The setup allows the compilation in various modes for editing purposes.
% In order to generate and update a complete set of documents for distribution
% (all individual sheets and a collection of sheets;
% with and without solutions),
% it makes sense to use the software development utility |make|.
% The compilation of individual components is simplified
% by a |bash| shell script.
%
% %%%%%%%%%%%%%%%%%%%%%%%%%%%%%%%%%%%%%%
% \paragraph{Compile Script.}
%
% The |bash| shell script |exfsermk.sh| compiles one part of
% the collection of problem sheets in a given mode.
%
% Shebang for |bash| script:
%    \begin{macrocode}
#!/bin/bash
%    \end{macrocode}

% Configure and declare variables with default values.
% |srcmain| defines the name of the main source file;
% |srcsec|\textit{nn} defines the name of the sheet source file;
% |trglist| defines the target file names;
% |trgsol| defines the target modes;
% |sheets| is a list of allowable sheet identifiers \textit{nn}:
%    \begin{macrocode}
srcmain="exfserm"
srcsec="exfser"
trglist=(Problems Solutions)
trgsol=(n y)
secnum="01 02 03 aa"
%    \end{macrocode}

% Display usage:
%    \begin{macrocode}
if [ -z $1 ]
then
  echo "Usage:
  $0 number [version]
    number: number of sheet, 0 for combined document
    version: 0 for problems, 1 for solutions
  $0 filename
    filename: target file to be compiled"
  exit 1
fi
%    \end{macrocode}
%\iffalse$\fi

% Configure and declare variables with default values.
% |num| takes the sheet number;
% |ver| takes the compile mode;
%    \begin{macrocode}
num="$1"
ver="$2"
nl=$'\n'
secokay=""
make=".pdf"
%    \end{macrocode}
%\iffalse$\fi

% Check if the parameter matches any of the acceptable output file names:
%    \begin{macrocode}
for v in "${trglist[@]}"
do
  if [[ $num =~ ^$v ]]
  then
    ver=$v
    num=${num#$v}
    if [[ $num =~ ^.*\.tex$ ]]; then make=".tex"; fi
    num=${num%%.*}
  fi
done
%    \end{macrocode}
%\iffalse$\fi

% Ensure that |num| is a two-digit number, prepend `0' otherwise:
%    \begin{macrocode}
if [[ $num =~ ^[0-9]$ ]]; then num="0$num"; fi
if [[ $num == "00" ]]; then num=""; fi
%    \end{macrocode}

% Check whether |num| is acceptable:
%    \begin{macrocode}
if [[ -z $num ]]; then secokay="okay"; fi
for v in $secnum
do
  if [[ "$num" == "$v" ]]; then secokay="okay"; fi
done
%    \end{macrocode}

% Otherwise display error message and exit:
%    \begin{macrocode}
if [[ -z $secokay ]]
then
  echo "error: unknown sheet"
  exit 1
fi
%    \end{macrocode}
%\iffalse$\fi

% Disable newline character for command line tex code:
%    \begin{macrocode}
if [[ "$make" == ".pdf" ]]; then nl=""; fi
%    \end{macrocode}
%\iffalse$\fi

% Function to compile a component.
% Set up \textsf{childdoc} mechanism according to desired component.
% Compile two passes, first in |-draftmode|.
% Suppress messages by |mpost|.
% Display warning messages in log file:
%    \begin{macrocode}
function docompile
{
  if [[ -z $num ]]
  then
    job="$srcmain"
    fwd="\\childdocforward{$srcmain}"
  else
    job="$srcsec$num"
    fwd="\\childdocforward[$srcmain]{$srcsec$num}"
  fi
  body="\\def\\jobname{$job}$optdef\% \iffalse
%
% childdoc.dtx Copyright (C) 2017-2018 Niklas Beisert
%
% This work may be distributed and/or modified under the
% conditions of the LaTeX Project Public License, either version 1.3
% of this license or (at your option) any later version.
% The latest version of this license is in
%   http://www.latex-project.org/lppl.txt
% and version 1.3 or later is part of all distributions of LaTeX
% version 2005/12/01 or later.
%
% This work has the LPPL maintenance status `maintained'.
%
% The Current Maintainer of this work is Niklas Beisert.
%
% This work consists of the files childdoc.dtx and childdoc.ins
% and the derived files childdoc.def and cdocsamp.tex with
% cdocsch1.tex, cdocsch2.tex, cdocsdrf.tex, cdocsfn1.tex, cdocsfn2.tex.
%
%<package>\ifdefined\childdocmain\endinput\fi
%<package>\ProvidesFile{childdoc.def}[2018/12/30 v2.0 child document driver]
%<samplemain>\ProvidesFile{cdocsamp.tex}[2018/12/30 v2.0 sample for childdoc]
%<*driver>
%\ProvidesFile{childdoc.drv}[2018/12/30 v2.0 childdoc reference manual file]
\PassOptionsToClass{10pt,a4paper}{article}
\documentclass{ltxdoc}

\usepackage[margin=35mm]{geometry}
\usepackage{hyperref}
\usepackage{hyperxmp}
\usepackage[usenames]{color}

\hypersetup{colorlinks=true}
\hypersetup{pdfstartview=FitH}
\hypersetup{pdfpagemode=UseNone}
\hypersetup{pdfsource={}}
\hypersetup{pdflang={en-UK}}
\hypersetup{pdfcopyright={Copyright 2017-2018 Niklas Beisert.
  This work may be distributed and/or modified under the
  conditions of the LaTeX Project Public License, either version 1.3
  of this license or (at your option) any later version.}}
\hypersetup{pdflicenseurl={http://www.latex-project.org/lppl.txt}}
\hypersetup{pdfcontactaddress={ETH Zurich, ITP, HIT K,
  Wolfgang-Pauli-Strasse 27}}
\hypersetup{pdfcontactpostcode={8093}}
\hypersetup{pdfcontactcity={Zurich}}
\hypersetup{pdfcontactcountry={Switzerland}}
\hypersetup{pdfcontactemail={nbeisert@itp.phys.ethz.ch}}
\hypersetup{pdfcontacturl={http://people.phys.ethz.ch/\xmptilde nbeisert/}}

\newcommand{\secref}[1]{\hyperref[#1]{section \ref*{#1}}}

\parskip1ex
\parindent0pt
\let\olditemize\itemize
\def\itemize{\olditemize\parskip0pt}

\begin{document}

\title{The \textsf{childdoc} Package}
\hypersetup{pdftitle={The childdoc Package}}
\author{Niklas Beisert\\[2ex]
  Institut f\"ur Theoretische Physik\\
  Eidgen\"ossische Technische Hochschule Z\"urich\\
  Wolfgang-Pauli-Strasse 27, 8093 Z\"urich, Switzerland\\[1ex]
  \href{mailto:nbeisert@itp.phys.ethz.ch}
  {\texttt{nbeisert@itp.phys.ethz.ch}}}
\hypersetup{pdfauthor={Niklas Beisert}}
\hypersetup{pdfsubject={Manual for the LaTeX2e Package childdoc}}
\date{30 December 2018, \textsf{v2.0}}
\maketitle

\begin{abstract}\noindent
\textsf{childdoc} is a \LaTeXe{} package
that enables the direct compilation
of document sections included by |\include|
to individual files.
\end{abstract}

\begingroup
\parskip0ex
\tableofcontents
\endgroup

%%%%%%%%%%%%%%%%%%%%%%%%%%%%%%%%%%%%%%%%%%%%%%%%%%%%%%%%%%%%%%%%%%%%%%%%%%%%%%%%
%%%%%%%%%%%%%%%%%%%%%%%%%%%%%%%%%%%%%%%%%%%%%%%%%%%%%%%%%%%%%%%%%%%%%%%%%%%%%%%%
\section{Introduction}

\LaTeX{} provides a mechanism to structure a large document (such as a book)
into a main file and several child files (containing the chapters)
using the |\include| command.
This mechanism is beneficial for documents
which span hundreds of pages in order to
make the source file(s) more manageable.
Moreover, compilation can be restricted to
selected child files by means of the |\includeonly| command.
The latter feature can be used to reduce the compilation time while editing
(this was significantly more useful in the earlier days of \LaTeX{})
or to generate a smaller document which is easier to navigate.
Another application of |\includeonly| is to generate
documents consisting of selected parts of the complete document.

However, there are a few drawbacks of the plain |\include| mechanism:
\begin{itemize}
\item
The child files cannot be compiled on their own,
they can only be compiled via the main file.
A naive editing environment
(such as a text editor with an option
to have the current file processed by \LaTeX)
may require one to switch to the main file before compiling;
attempting to compile the child file produces errors.
\item
The main file must be modified (each time)
to adjust the |\includeonly| command
to the present needs. This easily leaves the main file in a messy state.
\item
The generated document will always carry the filename
of the main document. This is inconvenient if
several child files are to be compiled and
to be kept for distribution.
\end{itemize}

The present package provides a simple interface
to make child files individually compilable by \LaTeX{}.
Compiling a child file then has the same effect as compiling
the main file with an |\includeonly| command
to select the appropriate child.
Moreover the generated document will carry the name of the child
rather than the main file.
This resolves all three above issues.

This feature is meant to make the editing of books,
thesis documents and lecture notes somewhat more convenient.
However, the package can also be used efficiently for
composing a series of documents (such as exercise sheets)
which are typically distributed individually.
It then assists the author in generating the individual documents
(potentially in different versions)
as well as a document containing the collected series.
Another application is in developing style files
or other kinds of included material
where compilation of the style file could redirect
to a sample or test file.

%%%%%%%%%%%%%%%%%%%%%%%%%%%%%%%%%%%%%%%%%%%%%%%%%%%%%%%%%%%%%%%%%%%%%%%%%%%%%%%%
%%%%%%%%%%%%%%%%%%%%%%%%%%%%%%%%%%%%%%%%%%%%%%%%%%%%%%%%%%%%%%%%%%%%%%%%%%%%%%%%
\section{Usage}

First of all, the package \textsf{childdoc} is \emph{not} a standard
\LaTeXe{} |.sty| style file! Therefore it needs to be invoked in
a non-standard way.

%%%%%%%%%%%%%%%%%%%%%%%%%%%%%%%%%%%%%%%%%%%%%%%%%%%%%%%%%%%%%%%%%%%%%%%%%%%%%%%%
\subsection{Included Files}
\label{sec:include}

%%%%%%%%%%%%%%%%%%%%%%%%%%%%%%%%%%%%%%%%
\DescribeMacro{\childdocmain}
To use the package, add the commands
\begin{center}
\begin{tabular}{l}
|% \iffalse
%
% childdoc.dtx Copyright (C) 2017-2018 Niklas Beisert
%
% This work may be distributed and/or modified under the
% conditions of the LaTeX Project Public License, either version 1.3
% of this license or (at your option) any later version.
% The latest version of this license is in
%   http://www.latex-project.org/lppl.txt
% and version 1.3 or later is part of all distributions of LaTeX
% version 2005/12/01 or later.
%
% This work has the LPPL maintenance status `maintained'.
%
% The Current Maintainer of this work is Niklas Beisert.
%
% This work consists of the files childdoc.dtx and childdoc.ins
% and the derived files childdoc.def and cdocsamp.tex with
% cdocsch1.tex, cdocsch2.tex, cdocsdrf.tex, cdocsfn1.tex, cdocsfn2.tex.
%
%<package>\ifdefined\childdocmain\endinput\fi
%<package>\ProvidesFile{childdoc.def}[2018/12/30 v2.0 child document driver]
%<samplemain>\ProvidesFile{cdocsamp.tex}[2018/12/30 v2.0 sample for childdoc]
%<*driver>
%\ProvidesFile{childdoc.drv}[2018/12/30 v2.0 childdoc reference manual file]
\PassOptionsToClass{10pt,a4paper}{article}
\documentclass{ltxdoc}

\usepackage[margin=35mm]{geometry}
\usepackage{hyperref}
\usepackage{hyperxmp}
\usepackage[usenames]{color}

\hypersetup{colorlinks=true}
\hypersetup{pdfstartview=FitH}
\hypersetup{pdfpagemode=UseNone}
\hypersetup{pdfsource={}}
\hypersetup{pdflang={en-UK}}
\hypersetup{pdfcopyright={Copyright 2017-2018 Niklas Beisert.
  This work may be distributed and/or modified under the
  conditions of the LaTeX Project Public License, either version 1.3
  of this license or (at your option) any later version.}}
\hypersetup{pdflicenseurl={http://www.latex-project.org/lppl.txt}}
\hypersetup{pdfcontactaddress={ETH Zurich, ITP, HIT K,
  Wolfgang-Pauli-Strasse 27}}
\hypersetup{pdfcontactpostcode={8093}}
\hypersetup{pdfcontactcity={Zurich}}
\hypersetup{pdfcontactcountry={Switzerland}}
\hypersetup{pdfcontactemail={nbeisert@itp.phys.ethz.ch}}
\hypersetup{pdfcontacturl={http://people.phys.ethz.ch/\xmptilde nbeisert/}}

\newcommand{\secref}[1]{\hyperref[#1]{section \ref*{#1}}}

\parskip1ex
\parindent0pt
\let\olditemize\itemize
\def\itemize{\olditemize\parskip0pt}

\begin{document}

\title{The \textsf{childdoc} Package}
\hypersetup{pdftitle={The childdoc Package}}
\author{Niklas Beisert\\[2ex]
  Institut f\"ur Theoretische Physik\\
  Eidgen\"ossische Technische Hochschule Z\"urich\\
  Wolfgang-Pauli-Strasse 27, 8093 Z\"urich, Switzerland\\[1ex]
  \href{mailto:nbeisert@itp.phys.ethz.ch}
  {\texttt{nbeisert@itp.phys.ethz.ch}}}
\hypersetup{pdfauthor={Niklas Beisert}}
\hypersetup{pdfsubject={Manual for the LaTeX2e Package childdoc}}
\date{30 December 2018, \textsf{v2.0}}
\maketitle

\begin{abstract}\noindent
\textsf{childdoc} is a \LaTeXe{} package
that enables the direct compilation
of document sections included by |\include|
to individual files.
\end{abstract}

\begingroup
\parskip0ex
\tableofcontents
\endgroup

%%%%%%%%%%%%%%%%%%%%%%%%%%%%%%%%%%%%%%%%%%%%%%%%%%%%%%%%%%%%%%%%%%%%%%%%%%%%%%%%
%%%%%%%%%%%%%%%%%%%%%%%%%%%%%%%%%%%%%%%%%%%%%%%%%%%%%%%%%%%%%%%%%%%%%%%%%%%%%%%%
\section{Introduction}

\LaTeX{} provides a mechanism to structure a large document (such as a book)
into a main file and several child files (containing the chapters)
using the |\include| command.
This mechanism is beneficial for documents
which span hundreds of pages in order to
make the source file(s) more manageable.
Moreover, compilation can be restricted to
selected child files by means of the |\includeonly| command.
The latter feature can be used to reduce the compilation time while editing
(this was significantly more useful in the earlier days of \LaTeX{})
or to generate a smaller document which is easier to navigate.
Another application of |\includeonly| is to generate
documents consisting of selected parts of the complete document.

However, there are a few drawbacks of the plain |\include| mechanism:
\begin{itemize}
\item
The child files cannot be compiled on their own,
they can only be compiled via the main file.
A naive editing environment
(such as a text editor with an option
to have the current file processed by \LaTeX)
may require one to switch to the main file before compiling;
attempting to compile the child file produces errors.
\item
The main file must be modified (each time)
to adjust the |\includeonly| command
to the present needs. This easily leaves the main file in a messy state.
\item
The generated document will always carry the filename
of the main document. This is inconvenient if
several child files are to be compiled and
to be kept for distribution.
\end{itemize}

The present package provides a simple interface
to make child files individually compilable by \LaTeX{}.
Compiling a child file then has the same effect as compiling
the main file with an |\includeonly| command
to select the appropriate child.
Moreover the generated document will carry the name of the child
rather than the main file.
This resolves all three above issues.

This feature is meant to make the editing of books,
thesis documents and lecture notes somewhat more convenient.
However, the package can also be used efficiently for
composing a series of documents (such as exercise sheets)
which are typically distributed individually.
It then assists the author in generating the individual documents
(potentially in different versions)
as well as a document containing the collected series.
Another application is in developing style files
or other kinds of included material
where compilation of the style file could redirect
to a sample or test file.

%%%%%%%%%%%%%%%%%%%%%%%%%%%%%%%%%%%%%%%%%%%%%%%%%%%%%%%%%%%%%%%%%%%%%%%%%%%%%%%%
%%%%%%%%%%%%%%%%%%%%%%%%%%%%%%%%%%%%%%%%%%%%%%%%%%%%%%%%%%%%%%%%%%%%%%%%%%%%%%%%
\section{Usage}

First of all, the package \textsf{childdoc} is \emph{not} a standard
\LaTeXe{} |.sty| style file! Therefore it needs to be invoked in
a non-standard way.

%%%%%%%%%%%%%%%%%%%%%%%%%%%%%%%%%%%%%%%%%%%%%%%%%%%%%%%%%%%%%%%%%%%%%%%%%%%%%%%%
\subsection{Included Files}
\label{sec:include}

%%%%%%%%%%%%%%%%%%%%%%%%%%%%%%%%%%%%%%%%
\DescribeMacro{\childdocmain}
To use the package, add the commands
\begin{center}
\begin{tabular}{l}
|% \iffalse
%
% childdoc.dtx Copyright (C) 2017-2018 Niklas Beisert
%
% This work may be distributed and/or modified under the
% conditions of the LaTeX Project Public License, either version 1.3
% of this license or (at your option) any later version.
% The latest version of this license is in
%   http://www.latex-project.org/lppl.txt
% and version 1.3 or later is part of all distributions of LaTeX
% version 2005/12/01 or later.
%
% This work has the LPPL maintenance status `maintained'.
%
% The Current Maintainer of this work is Niklas Beisert.
%
% This work consists of the files childdoc.dtx and childdoc.ins
% and the derived files childdoc.def and cdocsamp.tex with
% cdocsch1.tex, cdocsch2.tex, cdocsdrf.tex, cdocsfn1.tex, cdocsfn2.tex.
%
%<package>\ifdefined\childdocmain\endinput\fi
%<package>\ProvidesFile{childdoc.def}[2018/12/30 v2.0 child document driver]
%<samplemain>\ProvidesFile{cdocsamp.tex}[2018/12/30 v2.0 sample for childdoc]
%<*driver>
%\ProvidesFile{childdoc.drv}[2018/12/30 v2.0 childdoc reference manual file]
\PassOptionsToClass{10pt,a4paper}{article}
\documentclass{ltxdoc}

\usepackage[margin=35mm]{geometry}
\usepackage{hyperref}
\usepackage{hyperxmp}
\usepackage[usenames]{color}

\hypersetup{colorlinks=true}
\hypersetup{pdfstartview=FitH}
\hypersetup{pdfpagemode=UseNone}
\hypersetup{pdfsource={}}
\hypersetup{pdflang={en-UK}}
\hypersetup{pdfcopyright={Copyright 2017-2018 Niklas Beisert.
  This work may be distributed and/or modified under the
  conditions of the LaTeX Project Public License, either version 1.3
  of this license or (at your option) any later version.}}
\hypersetup{pdflicenseurl={http://www.latex-project.org/lppl.txt}}
\hypersetup{pdfcontactaddress={ETH Zurich, ITP, HIT K,
  Wolfgang-Pauli-Strasse 27}}
\hypersetup{pdfcontactpostcode={8093}}
\hypersetup{pdfcontactcity={Zurich}}
\hypersetup{pdfcontactcountry={Switzerland}}
\hypersetup{pdfcontactemail={nbeisert@itp.phys.ethz.ch}}
\hypersetup{pdfcontacturl={http://people.phys.ethz.ch/\xmptilde nbeisert/}}

\newcommand{\secref}[1]{\hyperref[#1]{section \ref*{#1}}}

\parskip1ex
\parindent0pt
\let\olditemize\itemize
\def\itemize{\olditemize\parskip0pt}

\begin{document}

\title{The \textsf{childdoc} Package}
\hypersetup{pdftitle={The childdoc Package}}
\author{Niklas Beisert\\[2ex]
  Institut f\"ur Theoretische Physik\\
  Eidgen\"ossische Technische Hochschule Z\"urich\\
  Wolfgang-Pauli-Strasse 27, 8093 Z\"urich, Switzerland\\[1ex]
  \href{mailto:nbeisert@itp.phys.ethz.ch}
  {\texttt{nbeisert@itp.phys.ethz.ch}}}
\hypersetup{pdfauthor={Niklas Beisert}}
\hypersetup{pdfsubject={Manual for the LaTeX2e Package childdoc}}
\date{30 December 2018, \textsf{v2.0}}
\maketitle

\begin{abstract}\noindent
\textsf{childdoc} is a \LaTeXe{} package
that enables the direct compilation
of document sections included by |\include|
to individual files.
\end{abstract}

\begingroup
\parskip0ex
\tableofcontents
\endgroup

%%%%%%%%%%%%%%%%%%%%%%%%%%%%%%%%%%%%%%%%%%%%%%%%%%%%%%%%%%%%%%%%%%%%%%%%%%%%%%%%
%%%%%%%%%%%%%%%%%%%%%%%%%%%%%%%%%%%%%%%%%%%%%%%%%%%%%%%%%%%%%%%%%%%%%%%%%%%%%%%%
\section{Introduction}

\LaTeX{} provides a mechanism to structure a large document (such as a book)
into a main file and several child files (containing the chapters)
using the |\include| command.
This mechanism is beneficial for documents
which span hundreds of pages in order to
make the source file(s) more manageable.
Moreover, compilation can be restricted to
selected child files by means of the |\includeonly| command.
The latter feature can be used to reduce the compilation time while editing
(this was significantly more useful in the earlier days of \LaTeX{})
or to generate a smaller document which is easier to navigate.
Another application of |\includeonly| is to generate
documents consisting of selected parts of the complete document.

However, there are a few drawbacks of the plain |\include| mechanism:
\begin{itemize}
\item
The child files cannot be compiled on their own,
they can only be compiled via the main file.
A naive editing environment
(such as a text editor with an option
to have the current file processed by \LaTeX)
may require one to switch to the main file before compiling;
attempting to compile the child file produces errors.
\item
The main file must be modified (each time)
to adjust the |\includeonly| command
to the present needs. This easily leaves the main file in a messy state.
\item
The generated document will always carry the filename
of the main document. This is inconvenient if
several child files are to be compiled and
to be kept for distribution.
\end{itemize}

The present package provides a simple interface
to make child files individually compilable by \LaTeX{}.
Compiling a child file then has the same effect as compiling
the main file with an |\includeonly| command
to select the appropriate child.
Moreover the generated document will carry the name of the child
rather than the main file.
This resolves all three above issues.

This feature is meant to make the editing of books,
thesis documents and lecture notes somewhat more convenient.
However, the package can also be used efficiently for
composing a series of documents (such as exercise sheets)
which are typically distributed individually.
It then assists the author in generating the individual documents
(potentially in different versions)
as well as a document containing the collected series.
Another application is in developing style files
or other kinds of included material
where compilation of the style file could redirect
to a sample or test file.

%%%%%%%%%%%%%%%%%%%%%%%%%%%%%%%%%%%%%%%%%%%%%%%%%%%%%%%%%%%%%%%%%%%%%%%%%%%%%%%%
%%%%%%%%%%%%%%%%%%%%%%%%%%%%%%%%%%%%%%%%%%%%%%%%%%%%%%%%%%%%%%%%%%%%%%%%%%%%%%%%
\section{Usage}

First of all, the package \textsf{childdoc} is \emph{not} a standard
\LaTeXe{} |.sty| style file! Therefore it needs to be invoked in
a non-standard way.

%%%%%%%%%%%%%%%%%%%%%%%%%%%%%%%%%%%%%%%%%%%%%%%%%%%%%%%%%%%%%%%%%%%%%%%%%%%%%%%%
\subsection{Included Files}
\label{sec:include}

%%%%%%%%%%%%%%%%%%%%%%%%%%%%%%%%%%%%%%%%
\DescribeMacro{\childdocmain}
To use the package, add the commands
\begin{center}
\begin{tabular}{l}
|\input{childdoc.def}|\\
|\childdocmain{}|\\
\end{tabular}
\end{center}
at the very top of the main \LaTeX{} file,
in particular \emph{before} the |\documentclass| statement!
The argument of |\childdocmain| should be left empty
(but it must be present).

%%%%%%%%%%%%%%%%%%%%%%%%%%%%%%%%%%%%%%%%
\DescribeMacro{\childdocof}
Furthermore, add the commands
\begin{center}
\begin{tabular}{l}
|\input{childdoc.def}|\\
|\childdocof{|\textit{main}|}|\\
\end{tabular}
\end{center}
at the top of every child file \textit{child}
which is included by |\include{|\textit{child}|}|
from within the main file
(or at least for those files to be compiled individually).
The argument \textit{main} must be the filename of the main file.

There are a couple of
considerations in setting up the main and child documents:

%%%%%%%%%%%%%%%%%%%%%%%%%%%%%%%%%%%%%%%%
\paragraph{Restrictions.}

Please note the following restrictions:
\begin{itemize}
\item
|\childdocmain| must be called with one argument \textit{main}
to ensure compatibility with earlier version of the package.
It must either be empty (|\childdocmain{}|)
or precisely match the filename of the main file in which it is specified.
See \secref{sec:detection} for further information.
\item
The filename \textit{main} must be specified without the |.tex| extension.
\item
The filename \textit{main} is case sensitive
(even in case-insensitive file systems)
due to internal string comparison.
\item
The argument \textit{main} should be fully expanded, it cannot be a macro.
\item
Subdirectories and special characters should be avoided in filenames.
\item
The command |\childdocmain{|\textit{main}|}| must be followed by a whitespace.
It should not be followed immediately by another command
or by a comment mark `|%|'.
This is because the \TeX{} parser reads the token immediately following
the argument of |\childdocmain| and puts it
at the beginning of every child section;
however, a white\-space is ignored.
\end{itemize}

%%%%%%%%%%%%%%%%%%%%%%%%%%%%%%%%%%%%%%%%
\paragraph{Content of Main File.}

It is advisable to place all content in the child files included by |\include|.
Any output contained in the main file will appear in all child documents
unless suppressed manually;
it cannot be suppressed automatically by the |\includeonly| directive
and thus should normally be avoided.
A method to include some content in the main file
by means of conditional processing is described in \secref{sec:conditional}.

%%%%%%%%%%%%%%%%%%%%%%%%%%%%%%%%%%%%%%%%
\paragraph{Page Numbering.}

When only a part of the document is compiled,
the appropriate numbering of pages
(as well as other status parameters)
is determined from the |.aux| files.
The latter contain information from previous passes.
However this information needs to propagate through
all intermediate child documents.
Therefore the page numbering in child documents may well
be inconsistent until the complete document is compiled at least once.

A useful (if unconventional) way to always ensure a consistent
page numbering is to restart the numbering in each child document
and denote the pages by `\textit{child}|.|\textit{page}'
where \textit{child} represents the chapter/section number of the child file.
This can be achieved by the command
|\numberwithin{page}{|\textit{child}|}|
of the \textsf{amsmath} package
where \textit{child} can be |chapter| or |section|
depending on the chosen structuring.
Alternatively, one can modify the macro |\thepage| appropriately
and reset the counter |page| at the start of each child file.

%%%%%%%%%%%%%%%%%%%%%%%%%%%%%%%%%%%%%%%%%%%%%%%%%%%%%%%%%%%%%%%%%%%%%%%%%%%%%%%%
\subsection{Conditional Processing}
\label{sec:conditional}

The package provides a mechanism to compile different versions
of a document. To customise the versions further some conditional processing
can come in handy to distinguish which version is being compiled.
The package provides two macros to describe the compilation context:

%%%%%%%%%%%%%%%%%%%%%%%%%%%%%%%%%%%%%%%%
\DescribeMacro{\ifchilddoc}
The conditional |\ifchilddoc| distinguishes between the compilation of
child documents and the main document:
%
\begin{center}
|\ifchilddoc |\textit{child-code}| |[|\||else |\textit{main-code}]| \||fi|
\end{center}

%%%%%%%%%%%%%%%%%%%%%%%%%%%%%%%%%%%%%%%%
\DescribeMacro{\childdocname}
\DescribeMacro{\childdocjob}
The macro |\childdocname| contains the filename (without extension)
of the main or child file being processed.
Note that |\childdocjob| will always contain the name of the main file.

%%%%%%%%%%%%%%%%%%%%%%%%%%%%%%%%%%%%%%%%
\paragraph{Title Page.}

Conditional processing can be used to include a title or banner page
in the main document when proper precautions are taken.
Importantly, the code in the main file should ensure that the page counter
(as well as other status parameters which are stored in the |.aux| files)
takes the same value after the conditional processing.
Otherwise the page numbers may take divergent values
depending on which part is compiled.

For example, a title page could be declared by:
%
\begin{center}
\begin{tabular}{l}
|\ifchilddoc\||else|\\
|\addtocounter{page}{-1}|\\
\textit{code for title page}\\
|\newpage|\\
|\||fi|
\end{tabular}
\end{center}
%
A banner page for the child documents can be generated by:
%
\begin{center}
\begin{tabular}{l}
|\ifchilddoc|\\
|\addtocounter{page}{-1}|\\
\textit{code for banner page}\\
|\newpage|\\
|\||fi|
\end{tabular}
\end{center}
%
Here one could write a message such as:
\begin{center}
|This is the part \childdocname{} of \childdocjob{}.|
\end{center}

%%%%%%%%%%%%%%%%%%%%%%%%%%%%%%%%%%%%%%%%%%%%%%%%%%%%%%%%%%%%%%%%%%%%%%%%%%%%%%%%
\subsection{Flags}
\label{sec:flags}

The package makes it easy to generate different versions
of the main or child documents.
To this end compilation flags can be defined
and assigned different default values.
They will be particularly useful in conjunction
with the forwarding mechanism described in \secref{sec:forward}.

For example, it may be useful to have a flag |\version|
which can be set to |draft| or |final|.
The document source will contain some conditional code
depending on the value of |\version|.
Suppose further, the flag should default to |final| for the main file
and to |draft| for child files
which is a natural assignment for editing the document.
This is achieved by placing the following code
in the preamble of the main document
(below the |\childdocmain| directive):
%
\begin{center}
\begin{tabular}{l}
|\ifchilddoc|\\
|\providecommand{\version}{draft}|\\
|\||else|\\
|\providecommand{\version}{final}|\\
|\||fi|
\end{tabular}
\end{center}
%
The definition by |\providecommand| makes sure
that previous definitions are not overwritten.
Further statements |\providecommand{\version}{...}|
can thus be added before the above code to override it.

For the main file, one might add a line
(between |\childdocmain| and the above block)
%
\begin{center}
|%\ifchilddoc\||else\providecommand{\version}{draft}\||fi|
\end{center}
%
which can be uncommented to produce a draft version.
Likewise one can add a line to the very top of a child file
(above the |\childdocof{|\textit{main}|}| directive)
%
\begin{center}
|%\providecommand{\version}{final}|
\end{center}
%
which can be uncommented to produce the final version of this child document.

%%%%%%%%%%%%%%%%%%%%%%%%%%%%%%%%%%%%%%%%%%%%%%%%%%%%%%%%%%%%%%%%%%%%%%%%%%%%%%%%
\subsection{Forwarding}
\label{sec:forward}

Different versions of the main or child documents
using compilation flags as described in \secref{sec:flags}
can be (permanently) stored in different files
for convenient compilation, viewing and distribution.
To this end, the package defines a command
to pass on compilation to a different file:

%%%%%%%%%%%%%%%%%%%%%%%%%%%%%%%%%%%%%%%%
\DescribeMacro{\childdocforward}
The command |\childdocforward| redirects processing to
another source file:
%
\begin{center}
\begin{tabular}{l}
|\input{childdoc.def}|\\
|\childdocforward[|\textit{main}|]{|\textit{dest}|}|\\
\end{tabular}
\end{center}
%
The argument \textit{dest} is the destination file
(without extension).
It should be the main file or one of the child files.
Note that further \textsf{childdoc} directives
such as |\childdocof| and |\childdocforward|
in the indicated file will be processed in this form.
The optional argument \textit{main}
passes on directly to the main file \textit{main}
while pretending to compile the child \textit{dest}.
This form behaves as if \textit{dest}
issues |\childdocof{|\textit{main}|}| right away,
and no further \textsf{childdoc} directives will be processed.

%%%%%%%%%%%%%%%%%%%%%%%%%%%%%%%%%%%%%%%%
\DescribeMacro{\...prefix}
In the alternative form |\childdocforwardprefix|,
%
\begin{center}
\begin{tabular}{l}
|\input{childdoc.def}|\\
|\childdocforwardprefix[|\textit{main}|]{|\textit{prefix}|}{|\textit{dest}|}|
\end{tabular}
\end{center}
%
the destination file is determined by a pattern
depending on the current file:
To make this work, the current file must be called
`{\textit{prefix}\hspace{0.2em}\textit{suffix}}'
with \textit{prefix} matching precisely the argument.
Processing is then passed on to the file
`{\textit{dest}\hspace{0.2em}\textit{suffix}}'.
Surely, the same effect is achieved by
directly specifying the
argument `{\textit{dest}\hspace{0.2em}\textit{suffix}}'
in the first form.
However, that requires to set up a different file
for each child. With the alternative form of the command
all these files can have exactly the same content
which simplifies setting them up and maintaining them.

For example, the following file |draft.tex|
with a compilation flag |\version| as described in \secref{sec:flags}
compiles the main document as a draft:
%
\begin{center}
\begin{tabular}{l}
|\def\version{draft}|\\
|\input{childdoc.def}|\\
|\childdocforward{|\textit{main}|}|
\end{tabular}
\end{center}
%
Likewise, the following files |final|\textit{nn}|.tex|
compile the final version of the child document
|child|\textit{nn}|.tex|:
%
\begin{center}
\begin{tabular}{l}
|\def\version{final}|\\
|\input{childdoc.def}|\\
|\childdocforwardprefix{final}{child}|
\end{tabular}
\end{center}
%

Note that when several versions of a main file and/or of each child file
are to be generated, it may be convenient to set up a |Makefile| or
shell script to automatise the process.

%%%%%%%%%%%%%%%%%%%%%%%%%%%%%%%%%%%%%%%%%%%%%%%%%%%%%%%%%%%%%%%%%%%%%%%%%%%%%%%%
\subsection{Command Line Processing}
\label{sec:commandline}

The effect of redirection files can also be achieved by invoking
the \LaTeX{} compiler with a more elaborate command line.
Most conveniently this should be done as part
of a shell script or a |Makefile|.

When using \textsf{childdoc} in the main file, the following
command lines effectively perform a redirection
(note that depending on the shell being used,
backslashes may have to be doubled: `|\|' $\to$ `|\\|'):
%
\begin{center}
|... -jobname "|\textit{target}|" |\\|"|[\textit{flags}]%
|\input{childdoc.def}\childdocforward[|\textit{main}|]{|\textit{dest}|}"|
\end{center}
%
Here \textit{target} is the name of the output file,
\textit{main} is the name of the main file
and \textit{dest} is the name of the main or child file to be processed
(all filenames without extensions).
The optional argument \textit{main} can be omitted
if \textit{main} matches \textit{dest}.
Optionally, compilation \textit{flags} can be defined via |\def| commands.
This command line makes the \TeX{} engine believe
it is compiling the file \textit{target}
whose content is specified as the latter parameter.
The provided code then forwards the processing to
\textit{main} or \textit{dest} as described in \secref{sec:forward}.

%%%%%%%%%%%%%%%%%%%%%%%%%%%%%%%%%%%%%%%%%%%%%%%%%%%%%%%%%%%%%%%%%%%%%%%%%%%%%%%%
\subsection{Include by Input}
\label{sec:input}

Including child documents by |\include| has some restrictions by design.
Most notably, the content of a child document always occupies
its own set of pages; pages cannot be shared between child documents.
Usually, this behaviour makes perfect sense
because each child document contain an essential part of the document.
However, in some situations it may be desirable to compose
a document from a collection of parts
without having mandatory page breaks between then.
For this case, the package
provides a mechanism to include parts
by |\input| which can also be processed individually.
However, by construction this mechanism
requires manual handling of the content to be output.

%%%%%%%%%%%%%%%%%%%%%%%%%%%%%%%%%%%%%%%%
\DescribeMacro{\ifchilddocmanual}
The main file should be prepared as usual, see \secref{sec:include}.
However, the document body must make a distinction
between processing of an individual part and of the main document, e.g.:
%
\begin{center}
\begin{tabular}{l}
|\ifchilddocmanual|\\
|\input{\childdocname}|\\
|\||else|\\
\textit{document body with }|\input{|\textit{part}|}|\\
|\||fi|
\end{tabular}
\end{center}
%
The conditional |\ifchilddocmanual| is true whenever
a part to be included by |\input| is being compiled,
and the name of the part is stored in |\childdocname|.

%%%%%%%%%%%%%%%%%%%%%%%%%%%%%%%%%%%%%%%%
\DescribeMacro{\childdocby}
Each part to be included by |\input| should start with:
%
\begin{center}
\begin{tabular}{l}
|\input{childdoc.def}|\\
|\childdocby{|\textit{main}|}|\\
\end{tabular}
\end{center}
%
The directive |\childdocby| is similar to |\childdocof|
described in \secref{sec:include},
but the subsequent selection of content must be done manually.
To that end, both |\ifchilddoc| and |\ifchilddocmanual|
will be true upon processing of a part,
and the name of the part is stored in |\childdocname|.
Note that |\jobname| will be set to the filename of the current part
so that each part receives an individual |.aux| file
that does not interfere with the |.aux| file(s) of the main document.
This behaviour can be altered by the alternative form
|\childdocby[*]{|\textit{main}|}| (with a non-empty optional argument)
which uses the |.aux| file of the main document
by setting |\jobname| to \textit{main}.

%%%%%%%%%%%%%%%%%%%%%%%%%%%%%%%%%%%%%%%%%%%%%%%%%%%%%%%%%%%%%%%%%%%%%%%%%%%%%%%%
\subsection{Driver Development}
\label{sec:driver}

The \textsf{childdoc} mechanism can also be use for the development
of definition files such as \LaTeX{} styles or classes.
This case differs from the above setup with multiple parts
included by |\include| in that no |\includeonly| should be invoked.
This can be achieved by starting the include file
(before |\ProvidesPackage|) with:
%
\begin{center}
\begin{tabular}{l}
|\input{childdoc.def}|\\
|\childdocforward{|\textit{main}|}|\\
\end{tabular}
\end{center}
%
or alternatively with:
%
\begin{center}
\begin{tabular}{l}
|\input{childdoc.def}|\\
|\childdocby{|\textit{main}|}|\\
\end{tabular}
\end{center}
%
Both forms have slightly different effects as described above.
The main file is prepared as usual, see \secref{sec:include}.

%%%%%%%%%%%%%%%%%%%%%%%%%%%%%%%%%%%%%%%%%%%%%%%%%%%%%%%%%%%%%%%%%%%%%%%%%%%%%%%%
\subsection{Legacy Detection}
\label{sec:detection}

The directive |\childdocmain| in the main file can detect
whether the complete document or merely a child is to be compiled
even without using the directive |\childdocof|.
This method is deprecated because it is less robust
and there is no compelling reason to use it;
it is merely provided for backward compatibility
and it may be removed in future versions.

If the detection mechanism is to be used,
it is mandatory to correctly specify
the filename of the main file as the argument of |\childdocmain|:
%
\begin{center}
\begin{tabular}{l}
|\input{childdoc.def}|\\
|\childdocmain{|\textit{main}|}|\\
\end{tabular}
\end{center}
%
If |\jobname| does not match the argument \textit{main} of |\childdocmain|,
it is assumed that |\jobname| points to the child file to be compiled.
When using |\childdocmain| with the main file specified as argument,
it suffices to start a child file
with just |\input{|\textit{main}|}|
without loading of the package and using |\childdocof|.
If instead all processing is done
with the appropriate \textsf{childdoc} directives,
the argument of \textit{main} of |\childdocmain| can be empty.

An alternative version of the command line processing described
in \secref{sec:commandline} using the detection mechanism reads:
%
\begin{center}
|... -jobname "|\textit{target}|" "|[\textit{flags}]%
[|\def\jobname{|\textit{dest}|}|]|\input{|\textit{main}|}"|
\end{center}

%%%%%%%%%%%%%%%%%%%%%%%%%%%%%%%%%%%%%%%%%%%%%%%%%%%%%%%%%%%%%%%%%%%%%%%%%%%%%%%%
\subsection{Manual Code}
\label{sec:manual}

In case one cannot be certain whether the definitions file |childdoc.def|
is installed on the target \TeX{} distribution
and one prefers not to ship it,
it is conceivable to paste a few relevant commands into the sources.

To that end, drop all statements |\input{childdoc.def}|
and perform the replacements as outlined below.
Instead of |\childdocmain{|\textit{main}|}| add the following code
to the top of the main file:
%
\begin{center}
\begin{tabular}{l}
|\||ifdefined\childdocname\endinput\||fi\newif\ifchilddoc|\\
|\edef\childdocname{\scantokens\expandafter{\jobname\noexpand}}|\\
|\def\childdocmain{|\textit{main}|}\||ifx\childdocmain\childdocname\||else|\\
|\childdoctrue\includeonly{\childdocname}\let\jobname\childdocmain\||fi|\\
\end{tabular}
\end{center}
%
Instead of |\childdocof{|\textit{main}|}| just include the main file
at the top of each child file:
%
\begin{center}
|\input{|\textit{main}|}|
\end{center}
%
A simple redirection |\childdocforward{|\textit{dest}|}| is achieved by:
%
\begin{center}
|\def\jobname{|\textit{dest}|}\input{\jobname}|
\end{center}
%
The redirection with prefix
|\childdocforwardprefix[|\textit{prefix}|]{|\textit{dest}|}|
is accomplished by:
%
\begin{center}
\begin{tabular}{l}
|{\edef\jobname{\scantokens\expandafter{\jobname\noexpand}}|\\
|\def\redirectjob |\textit{prefix}|#1~~~{\gdef\jobname{|\textit{dest}|#1}}|\\
|\expandafter\redirectjob\jobname~~~}\input{\jobname}|
\end{tabular}
\end{center}

In an alternative approach,
child documents can be compiled by a specific command line
without additional code or specific definitions:
%
\begin{center}
|... -jobname "|\textit{target}|" "|[\textit{flags}]%
|\includeonly{|\textit{dest}|}\input{|\textit{main}|}"|
\end{center}
%

%%%%%%%%%%%%%%%%%%%%%%%%%%%%%%%%%%%%%%%%%%%%%%%%%%%%%%%%%%%%%%%%%%%%%%%%%%%%%%%%
%%%%%%%%%%%%%%%%%%%%%%%%%%%%%%%%%%%%%%%%%%%%%%%%%%%%%%%%%%%%%%%%%%%%%%%%%%%%%%%%
\section{Information}

%%%%%%%%%%%%%%%%%%%%%%%%%%%%%%%%%%%%%%%%%%%%%%%%%%%%%%%%%%%%%%%%%%%%%%%%%%%%%%%%
\subsection{Copyright}

Copyright \copyright{} 2017--2018 Niklas Beisert

This work may be distributed and/or modified under the
conditions of the \LaTeX{} Project Public License, either version 1.3
of this license or (at your option) any later version.
The latest version of this license is in
  \url{http://www.latex-project.org/lppl.txt}
and version 1.3 or later is part of all distributions of \LaTeX{}
version 2005/12/01 or later.

This work has the LPPL maintenance status `maintained'.

The Current Maintainer of this work is Niklas Beisert.

This work consists of the files |README.txt|, |childdoc.ins| and |childdoc.dtx|
as well as the derived files |childdoc.def|, |cdocsamp.tex|
with |cdocsch1.tex|, |cdocsch2.tex|, |cdocspt3.tex|, |cdocspt4.tex|,
|cdocsdrf.tex|, |cdocsfn1.tex|, |cdocsfn2.tex|
as well as |childdoc.pdf|.

%%%%%%%%%%%%%%%%%%%%%%%%%%%%%%%%%%%%%%%%%%%%%%%%%%%%%%%%%%%%%%%%%%%%%%%%%%%%%%%%
\subsection{Files and Installation}

The package consists of the files:
%
\begin{center}
\begin{tabular}{ll}
    |README.txt|   & readme file \\
    |childdoc.ins| & installation file \\
    |childdoc.dtx| & source file \\
    |childdoc.def| & definition file \\
    |cdocsamp.tex| & sample main file \\
    |cdocsch1.tex| & sample include file \\
    |cdocsch2.tex| & sample include file \\
    |cdocspt3.tex| & sample part file \\
    |cdocspt4.tex| & sample part file \\
    |cdocsdrf.tex| & sample redirection file \\
    |cdocsfn1.tex| & sample redirection file \\
    |cdocsfn2.tex| & sample redirection file \\
    |childdoc.pdf| & manual
\end{tabular}
\end{center}
%
The distribution consists of the files
|README.txt|, |childdoc.ins| and |childdoc.dtx|.
%
\begin{itemize}
\item
Run (pdf)\LaTeX{} on |childdoc.dtx|
to compile the manual |childdoc.pdf| (this file).
\item
Run \LaTeX{} on |childdoc.ins| to create the definitions file |childdoc.def|
and the sample |cdocsamp.tex| with include files
|cdocsch1.tex|, |cdocsch2.tex|, |cdocspt3.tex|, |cdocspt4.tex|,
|cdocsdrf.tex|, |cdocsfn1.tex|, |cdocsfn2.tex|.
Then copy the file |childdoc.def| to an appropriate directory of your \LaTeX{}
distribution, e.g.\ \textit{texmf-root}|/tex/latex/childdoc|.
\end{itemize}

%%%%%%%%%%%%%%%%%%%%%%%%%%%%%%%%%%%%%%%%%%%%%%%%%%%%%%%%%%%%%%%%%%%%%%%%%%%%%%%%
\subsection{Related CTAN Packages}

There are several other packages which offer a similar functionality:
%
\begin{itemize}
\item
The packages
\href{http://ctan.org/pkg/docmute}{\textsf{docmute}},
\href{http://ctan.org/pkg/includex}{\textsf{includex}} and
\href{http://ctan.org/pkg/standalone}{\textsf{standalone}}
provide commands to include only the document body of
a child file thus allowing both files to be compiled individually.
\item
The packages \href{http://ctan.org/pkg/subdocs}{\textsf{subdocs}}
and \href{http://ctan.org/pkg/subfiles}{\textsf{subfiles}}
provide structures in which the main and child documents can be
encapsulated and allowing them to be compiled individually.
The inclusion mechanism is different from the conventional |\include|.
\item
The package \href{http://ctan.org/pkg/combine}{\textsf{combine}}
is an elaborate solution to combine several documents into one.
\end{itemize}
%
See also the CTAN topic \href{http://ctan.org/topic/subdocs}{\textsf{subdocs}}
for further related packages.
The present package differs from the above solutions in that
a document structure constructed with the conventional |\include| mechanism
just needs two extra commands at the top of every file
such that all constituent files can be compiled individually.

%%%%%%%%%%%%%%%%%%%%%%%%%%%%%%%%%%%%%%%%%%%%%%%%%%%%%%%%%%%%%%%%%%%%%%%%%%%%%%%%
%\subsection{Feature Suggestions}
%
%The following is a list of features which may be useful for future
%versions of this package:
%%
%\begin{itemize}
%\item
%\ldots
%\end{itemize}

%%%%%%%%%%%%%%%%%%%%%%%%%%%%%%%%%%%%%%%%%%%%%%%%%%%%%%%%%%%%%%%%%%%%%%%%%%%%%%%%
\subsection{Revision History}

%%%%%%%%%%%%%%%%%%%%%%%%%%%%%%%%%%%%%%%%
\paragraph{v2.0:} 2018/12/30

\begin{itemize}
\item
immediate forward processing
\item
added |\childdocby| mechanism
\item
manual restructured
\end{itemize}

%%%%%%%%%%%%%%%%%%%%%%%%%%%%%%%%%%%%%%%%
\paragraph{v1.6:} 2018/01/17

\begin{itemize}
\item
application for development of include files
\item
corrections to manual
\end{itemize}

%%%%%%%%%%%%%%%%%%%%%%%%%%%%%%%%%%%%%%%%
\paragraph{v1.5:} 2017/05/21

\begin{itemize}
\item
more complete structuring introduced
\item
|\childdocof| introduced
\item
|\childdoc| renamed to |\childdocmain|
\item
|\childredirect| renamed to |\childdocforward| and |\childdocforwardprefix|
and functionality expanded
\end{itemize}

%%%%%%%%%%%%%%%%%%%%%%%%%%%%%%%%%%%%%%%%
\paragraph{v1.0:} 2017/04/27

\begin{itemize}
\item
manual and install package
\item
first version published on CTAN
\end{itemize}

%%%%%%%%%%%%%%%%%%%%%%%%%%%%%%%%%%%%%%%%
\paragraph{v0.6:} 2017/04/26

\begin{itemize}
\item
redirection mechanism added
\end{itemize}

%%%%%%%%%%%%%%%%%%%%%%%%%%%%%%%%%%%%%%%%
\paragraph{v0.5:} 2017/04/26

\begin{itemize}
\item
functionality in definition file
\end{itemize}


%%%%%%%%%%%%%%%%%%%%%%%%%%%%%%%%%%%%%%%%%%%%%%%%%%%%%%%%%%%%%%%%%%%%%%%%%%%%%%%%
%%%%%%%%%%%%%%%%%%%%%%%%%%%%%%%%%%%%%%%%%%%%%%%%%%%%%%%%%%%%%%%%%%%%%%%%%%%%%%%%
%%%%%%%%%%%%%%%%%%%%%%%%%%%%%%%%%%%%%%%%%%%%%%%%%%%%%%%%%%%%%%%%%%%%%%%%%%%%%%%%
\appendix

\settowidth\MacroIndent{\rmfamily\scriptsize 000\ }

 \DocInput{childdoc.dtx}

\end{document}
%</driver>
% \fi
%
% %%%%%%%%%%%%%%%%%%%%%%%%%%%%%%%%%%%%%%%%%%%%%%%%%%%%%%%%%%%%%%%%%%%%%%%%%%%%%%
% %%%%%%%%%%%%%%%%%%%%%%%%%%%%%%%%%%%%%%%%%%%%%%%%%%%%%%%%%%%%%%%%%%%%%%%%%%%%%%
% \section{Sample}
%\iffalse
%<*samplemain>
%\fi
%
% The following presents a sample document
% with two chapters, two parts, a title page,
% a compile flag as well as three forwarding files to set the flag.
% It consists of eight |.tex| files:
% \begin{center}
% \begin{tabular}{ll}
% |cdocsamp.tex|&main file\\
% |cdocsch1.tex|&include file for chapter 1\\
% |cdocsch2.tex|&include file for chapter 2\\
% |cdocspt3.tex|&include file for part 3\\
% |cdocspt4.tex|&include file for part 4\\
% |cdocsdrf.tex|&forwarding file for main file in draft mode\\
% |cdocsfi1.tex|&forwarding file for final version of chapter 1\\
% |cdocsfi2.tex|&forwarding file for final version of chapter 2\\
% \end{tabular}
% \end{center}
% Each of the eight files can be compiled directly by the \LaTeX{} compiler.
%
% %%%%%%%%%%%%%%%%%%%%%%%%%%%%%%%%%%%%%%
% \paragraph{Main File.}
%
% The main file is called |cdocsamp.tex|.
%
% Load the \textsf{childdoc} definitions and
% declare the filename for the main document:
%    \begin{macrocode}
\input{childdoc.def}
\childdocmain{}
%    \end{macrocode}

% Optional override for |\version| flag:
%    \begin{macrocode}
%%\ifchilddoc\else\providecommand{\version}{draft}\fi
%    \end{macrocode}

% Define the default values for the |\version| flag
% (|final| for the main file and |draft| for childs):
%    \begin{macrocode}
\ifchilddoc
\providecommand{\version}{draft}
\else
\providecommand{\version}{final}
\fi
%    \end{macrocode}

% Load the standard document class:
%    \begin{macrocode}
\documentclass[12pt]{article}
%    \end{macrocode}

% Start the document body:
%    \begin{macrocode}
\begin{document}
%    \end{macrocode}

% Declare a title page.
% Print title, part of document being processed and version flag:
%    \begin{macrocode}
\addtocounter{page}{-1}
\begin{center}
{\LARGE\bfseries{}childdoc example\par}
\vspace{1cm}
\ifchilddoc
\ifchilddocmanual part\else chapter\fi:
`\childdocname' of `\childdocjob'\par
\else
main document: `\childdocjob'\par
\fi
version: \version\par
\end{center}
\newpage
%    \end{macrocode}

% Manually include selected file,
% otherwise process as usual:
%    \begin{macrocode}
\ifchilddocmanual
\section*{part `\childdocname'}
\input{\childdocname}
\else
%    \end{macrocode}

% Include the two chapters:
%    \begin{macrocode}
\include{cdocsch1}
\include{cdocsch2}
%    \end{macrocode}

% Include the two parts unless only chapters should be displayed:
%    \begin{macrocode}
\ifchilddoc\else
\section{part three}
\input{cdocspt3}
\section{part four}
\input{cdocspt4}
\fi
%    \end{macrocode}

% Process as usual until here:
%    \begin{macrocode}
\fi
%    \end{macrocode}

% End of document body:
%    \begin{macrocode}
\end{document}
%    \end{macrocode}
%\iffalse
%</samplemain>
%\fi
%
% %%%%%%%%%%%%%%%%%%%%%%%%%%%%%%%%%%%%%%
% \paragraph{Chapter Include Files.}
%
% The include files are called |cdocsch1.tex| and |cdocsch2.tex|.
%
%\iffalse
%<*samplechap1|samplechap2>
%\fi

% Optional override for |\version| flag:
%    \begin{macrocode}
%%\providecommand{\version}{final}
%    \end{macrocode}

% Include the main document:
%    \begin{macrocode}
\input{childdoc.def}
\childdocof{cdocsamp}
%    \end{macrocode}

%\iffalse
%</samplechap1|samplechap2>
%\fi
%
%\iffalse
%<*samplechap1>
%\fi
% Some text for chapter 1:
%    \begin{macrocode}
\section{one}
some text in chapter one
%    \end{macrocode}

%\iffalse
%</samplechap1>
%\fi
% Some text for chapter 2:
%\iffalse
%<*samplechap2>
%\fi
%    \begin{macrocode}
\section{two}
more text in chapter two
%    \end{macrocode}

%\iffalse
%</samplechap2>
%\fi
%
% %%%%%%%%%%%%%%%%%%%%%%%%%%%%%%%%%%%%%%
% \paragraph{Part Include Files.}
%
% The include files are called |cdocspt3.tex| and |cdocspt4.tex|.
%
%\iffalse
%<*samplepart3|samplepart4>
%\fi

% Optional override for |\version| flag:
%    \begin{macrocode}
%%\providecommand{\version}{final}
%    \end{macrocode}

% Include the main document:
%    \begin{macrocode}
\input{childdoc.def}
\childdocby{cdocsamp}
%    \end{macrocode}

%\iffalse
%</samplepart3|samplepart4>
%\fi
%
%\iffalse
%<*samplepart3>
%\fi
% Some text for part 3:
%    \begin{macrocode}
some text in part three
%    \end{macrocode}

%\iffalse
%</samplepart3>
%\fi
% Some text for part 4:
%\iffalse
%<*samplepart4>
%\fi
%    \begin{macrocode}
more text in part four
%    \end{macrocode}

%\iffalse
%</samplepart4>
%\fi
%
% %%%%%%%%%%%%%%%%%%%%%%%%%%%%%%%%%%%%%%
% \paragraph{Forwarding for a Complete Draft.}
%
% The following forwarding file |cdocsdrf.tex|
% compiles the main document in draft mode:
%\iffalse
%<*sampledraft>
%\fi
%    \begin{macrocode}
\def\version{draft}
\input{childdoc.def}
\childdocforward{cdocsamp}
%    \end{macrocode}

%\iffalse
%</sampledraft>
%\fi
%
% %%%%%%%%%%%%%%%%%%%%%%%%%%%%%%%%%%%%%%
% \paragraph{Forwarding for Final Version of the Chapters.}
%
% The following forwarding files |cdocsfn1.tex| and |cdocsfn2.tex|
% (with identical content)
% compile the final versions of the child documents
% |cdocsch1.tex| and |cdocsch2.tex|, respectively:
%\iffalse
%<*samplefinal>
%\fi
%    \begin{macrocode}
\def\version{final}
\input{childdoc.def}
\childdocforwardprefix[cdocsamp]{cdocsfn}{cdocsch}
%    \end{macrocode}

%\iffalse
%</samplefinal>
%\fi
%
% %%%%%%%%%%%%%%%%%%%%%%%%%%%%%%%%%%%%%%
% \paragraph{Command Line Processing.}
%
% The following three command lines generate the output files
% |cdocscld|, |cdocscl1| and |cdocscl2|
% which should be identical to
% |cdocsdrf|, |cdocsch1| and |cdocsfn2|, respectively:
% \begin{center}
% \begin{tabular}{l}
% |latex -jobname cdocscld \|\\
% |  "\def\version{draft}\input{childdoc.def}\childdocforward{cdocsamp}"|\\
% |latex -jobname cdocscl1 \|\\
% |  "\input{childdoc.def}\childdocforward[cdocsamp]{cdocsch1}"|\\
% |latex -jobname cdocscl2 \|\\
% |  "\def\version{final}\input{childdoc.def}\childdocforward{cdocsch2}"|
% \end{tabular}
% \end{center}
% Note that the trailing backslash on each first line
% merely continues the input to the second line
% (for convenient cut ant paste).
% Furthermore, the command |latex| can be replaced by any
% of its alternative versions such as |pdflatex|.
%
% %%%%%%%%%%%%%%%%%%%%%%%%%%%%%%%%%%%%%%%%%%%%%%%%%%%%%%%%%%%%%%%%%%%%%%%%%%%%%%
% %%%%%%%%%%%%%%%%%%%%%%%%%%%%%%%%%%%%%%%%%%%%%%%%%%%%%%%%%%%%%%%%%%%%%%%%%%%%%%
% \section{Implementation}
%\iffalse
%<*package>
%\fi
%
% This section describes the definitions file |childdoc.def|.

% The definitions cannot be loaded using |\usepackage| or |\RequirePackage|
% which has a mechanism to prevent loading a style file more than once.
% When loading the definitions by means of |\input|
% multiple instances have to be prevented manually:
%\iffalse
%This code needs to be before the `\ProvidesFile' directive
%which is defined at the beginning of this file.
%Therefore it is also placed there and commented out here.
%</package>
%<*discard>
%\fi
%    \begin{macrocode}
\ifdefined\childdocmain\endinput\fi
%    \end{macrocode}
%\iffalse
%</discard>
%<*package>
%\fi
%
% \macro{\ifchilddoc}
% \macro{\ifchilddocmanual}
% The conditional |\ifchilddoc| tells whether a
% child (true) or main (false) document is being compiled.
% The conditional |\ifchilddocmanual| tells whether
% the |\includeonly| mechanism is used (false) or
% the selection of child files must be performed manually (true).
% The definitions initialise to false:
%    \begin{macrocode}
\newif\ifchilddoc
\newif\ifchilddocmanual
%    \end{macrocode}

% \macro{\childdocname}
% \macro{\childdocjob}
% The macro |\childdocname| stores the name of the main document
% to be compiled. The macro |\childdocjob| stores the name of
% the document on which the \LaTeX{} compiler was originally invoked.
% The content of |\jobname| cannot be compared
% to filenames specified in the source due to different catcodes.
% The following code rescans |\jobname|, stores the result
% in |\childdocname| and saves a copy in |\childdocjob|:
%    \begin{macrocode}
\edef\childdocname{\scantokens\expandafter{\jobname\noexpand}}
\let\childdocjob\childdocname
%    \end{macrocode}

% \macro{\childdocdisable}
% The macro |\childdocdisable| prevents the main file
% from being processed more than once.
% At this stage, the main document command |\childdocmain|
% is assumed to be called once again where it should do nothing.
% Any subsequent call to it should prevent
% a secondary processing of the main document
% It overwrites the forwarding commands
% |\childdocof| and |\childdocforward|
% with empty macros to prevent further inclusions of the main document:
%    \begin{macrocode}
\newcommand{\childdocdisable}
{
  \renewcommand{\childdocmain}[1]{\renewcommand{\childdocmain}[1]{\endinput}}
  \renewcommand{\childdocof}[1]{}
  \renewcommand{\childdocby}[2][]{}
  \renewcommand{\childdocforward}[2][]{}
  \renewcommand{\childdocdisable}{}
}
%    \end{macrocode}

% \macro{\childdocmain}
% The macro |\childdocmain| is to be called at the top of the main file
% with nothing or the main filename (without extension) as argument.
% First, it breaks loops.
% If the argument is not empty and does not match |\childdocname|
% (which is set by the first inclusion of |childdoc.def|),
% |\ifchilddoc| is set to true, |\includeonly| is applied to the child file
% and |\jobname| is set to the main file
% (for proper handling of |.aux| files):
%    \begin{macrocode}
\newcommand{\childdocmain}[1]
{
  \childdocdisable\childdocmain{}
  \if?#1?\else
    \begingroup
      \def\childdoctmp{#1}
      \ifx\childdoctmp\childdocname
        \def\childdoctmp{}
      \else
        \def\childdoctmp
        {
          \childdoctrue
          \includeonly{\childdocname}
          \def\childdocjob{#1}
          \def\jobname{#1}
        }
      \fi
      \expandafter
    \endgroup
    \childdoctmp
  \fi
}
%    \end{macrocode}

% \macro{\childdocof}
% The command |\childdocof| redirects
% compilation to the main file |#1|.
%    \begin{macrocode}
\newcommand{\childdocof}[1]
{
  \childdocdisable
  \childdoctrue
  \includeonly{\childdocname}
  \def\jobname{#1}
  \def\childdocjob{#1}
  \input{#1}
}
%    \end{macrocode}

% \macro{\childdocby}
% The command |\childdocby| ....
%    \begin{macrocode}
\newcommand{\childdocby}[2][]
{
  \childdocdisable
  \childdoctrue
  \childdocmanualtrue
  \if?#1?\else
    \def\jobname{#2}
  \fi
  \def\childdocjob{#2}
  \input{#2}
  \endinput
}
%    \end{macrocode}

% \macro{\childdocforward}
% The command |\childdocforward| redirects
% compilation to the main file or
% (if the optional argument is given) a child file.
% Parameters are set as if the main file
% or a child file starting with |\childdocof| was compiled.
% Then compilation is handed over to the main file:
%    \begin{macrocode}
\newcommand{\childdocforward}[2][]
{
  \begingroup
    \if?#1?
      \def\childdoctmp
      {
        \def\childdocname{#2}
        \def\childdocjob{#2}
        \def\jobname{#2}
        \input{#2}
        \endinput
      }
    \else
      \def\childdoctmp
      {
        \childdocdisable
        \def\childdocname{#2}
        \childdoctrue
        \includeonly{#2}
        \def\childdocjob{#1}
        \def\jobname{#1}
        \input{#1}
        \endinput
      }
    \fi
    \expandafter
  \endgroup
  \childdoctmp
}
%    \end{macrocode}

% \macro{\childdocforwardprefix}
% The command |\childdocforwardprefix| redirects
% compilation to the main or a child file by means of a pattern.
% The prefix |#1| in the current filename is replaced by |#2|
% and the suffix of the current filename is kept
% (it is assumed that the filename does not contain the substring `|~~~|'
% which is used as a delimiter).
% Compilation is handed over to the new file by |\childdocforward|:
%    \begin{macrocode}
\newcommand{\childdocforwardprefix}[3][]
{
  \begingroup
    \def\childdocextract #2##1~~~{\def\childdoctmp{\childdocforward[#1]{#3##1}}}
    \expandafter\childdocextract\childdocname~~~
    \expandafter
  \endgroup
  \childdoctmp
}
%    \end{macrocode}

% \macro{\childdoc}
% The deprecated macro |\childdoc| is a legacy version of |\childdocmain|:
%    \begin{macrocode}
\newcommand{\childdoc}{\childdocmain}
%    \end{macrocode}

% \macro{\childdocredirect}
% The deprecated macro |\childdocredirect| is a legacy version
% of |\childdocforward| and |\childdocforwardprefix|:
%    \begin{macrocode}
\newcommand{\childdocredirect}[2][]
{
  \begingroup
    \if?#1?
      \def\childdoctmp{\childdocforward{#2}}
    \else
      \def\childdoctmp{\childdocforwardprefix{#1}{#2}}
    \fi
    \expandafter
  \endgroup
  \childdoctmp
}
%    \end{macrocode}

%\iffalse
%</package>
%\fi
%
\endinput
|\\
|\childdocmain{}|\\
\end{tabular}
\end{center}
at the very top of the main \LaTeX{} file,
in particular \emph{before} the |\documentclass| statement!
The argument of |\childdocmain| should be left empty
(but it must be present).

%%%%%%%%%%%%%%%%%%%%%%%%%%%%%%%%%%%%%%%%
\DescribeMacro{\childdocof}
Furthermore, add the commands
\begin{center}
\begin{tabular}{l}
|% \iffalse
%
% childdoc.dtx Copyright (C) 2017-2018 Niklas Beisert
%
% This work may be distributed and/or modified under the
% conditions of the LaTeX Project Public License, either version 1.3
% of this license or (at your option) any later version.
% The latest version of this license is in
%   http://www.latex-project.org/lppl.txt
% and version 1.3 or later is part of all distributions of LaTeX
% version 2005/12/01 or later.
%
% This work has the LPPL maintenance status `maintained'.
%
% The Current Maintainer of this work is Niklas Beisert.
%
% This work consists of the files childdoc.dtx and childdoc.ins
% and the derived files childdoc.def and cdocsamp.tex with
% cdocsch1.tex, cdocsch2.tex, cdocsdrf.tex, cdocsfn1.tex, cdocsfn2.tex.
%
%<package>\ifdefined\childdocmain\endinput\fi
%<package>\ProvidesFile{childdoc.def}[2018/12/30 v2.0 child document driver]
%<samplemain>\ProvidesFile{cdocsamp.tex}[2018/12/30 v2.0 sample for childdoc]
%<*driver>
%\ProvidesFile{childdoc.drv}[2018/12/30 v2.0 childdoc reference manual file]
\PassOptionsToClass{10pt,a4paper}{article}
\documentclass{ltxdoc}

\usepackage[margin=35mm]{geometry}
\usepackage{hyperref}
\usepackage{hyperxmp}
\usepackage[usenames]{color}

\hypersetup{colorlinks=true}
\hypersetup{pdfstartview=FitH}
\hypersetup{pdfpagemode=UseNone}
\hypersetup{pdfsource={}}
\hypersetup{pdflang={en-UK}}
\hypersetup{pdfcopyright={Copyright 2017-2018 Niklas Beisert.
  This work may be distributed and/or modified under the
  conditions of the LaTeX Project Public License, either version 1.3
  of this license or (at your option) any later version.}}
\hypersetup{pdflicenseurl={http://www.latex-project.org/lppl.txt}}
\hypersetup{pdfcontactaddress={ETH Zurich, ITP, HIT K,
  Wolfgang-Pauli-Strasse 27}}
\hypersetup{pdfcontactpostcode={8093}}
\hypersetup{pdfcontactcity={Zurich}}
\hypersetup{pdfcontactcountry={Switzerland}}
\hypersetup{pdfcontactemail={nbeisert@itp.phys.ethz.ch}}
\hypersetup{pdfcontacturl={http://people.phys.ethz.ch/\xmptilde nbeisert/}}

\newcommand{\secref}[1]{\hyperref[#1]{section \ref*{#1}}}

\parskip1ex
\parindent0pt
\let\olditemize\itemize
\def\itemize{\olditemize\parskip0pt}

\begin{document}

\title{The \textsf{childdoc} Package}
\hypersetup{pdftitle={The childdoc Package}}
\author{Niklas Beisert\\[2ex]
  Institut f\"ur Theoretische Physik\\
  Eidgen\"ossische Technische Hochschule Z\"urich\\
  Wolfgang-Pauli-Strasse 27, 8093 Z\"urich, Switzerland\\[1ex]
  \href{mailto:nbeisert@itp.phys.ethz.ch}
  {\texttt{nbeisert@itp.phys.ethz.ch}}}
\hypersetup{pdfauthor={Niklas Beisert}}
\hypersetup{pdfsubject={Manual for the LaTeX2e Package childdoc}}
\date{30 December 2018, \textsf{v2.0}}
\maketitle

\begin{abstract}\noindent
\textsf{childdoc} is a \LaTeXe{} package
that enables the direct compilation
of document sections included by |\include|
to individual files.
\end{abstract}

\begingroup
\parskip0ex
\tableofcontents
\endgroup

%%%%%%%%%%%%%%%%%%%%%%%%%%%%%%%%%%%%%%%%%%%%%%%%%%%%%%%%%%%%%%%%%%%%%%%%%%%%%%%%
%%%%%%%%%%%%%%%%%%%%%%%%%%%%%%%%%%%%%%%%%%%%%%%%%%%%%%%%%%%%%%%%%%%%%%%%%%%%%%%%
\section{Introduction}

\LaTeX{} provides a mechanism to structure a large document (such as a book)
into a main file and several child files (containing the chapters)
using the |\include| command.
This mechanism is beneficial for documents
which span hundreds of pages in order to
make the source file(s) more manageable.
Moreover, compilation can be restricted to
selected child files by means of the |\includeonly| command.
The latter feature can be used to reduce the compilation time while editing
(this was significantly more useful in the earlier days of \LaTeX{})
or to generate a smaller document which is easier to navigate.
Another application of |\includeonly| is to generate
documents consisting of selected parts of the complete document.

However, there are a few drawbacks of the plain |\include| mechanism:
\begin{itemize}
\item
The child files cannot be compiled on their own,
they can only be compiled via the main file.
A naive editing environment
(such as a text editor with an option
to have the current file processed by \LaTeX)
may require one to switch to the main file before compiling;
attempting to compile the child file produces errors.
\item
The main file must be modified (each time)
to adjust the |\includeonly| command
to the present needs. This easily leaves the main file in a messy state.
\item
The generated document will always carry the filename
of the main document. This is inconvenient if
several child files are to be compiled and
to be kept for distribution.
\end{itemize}

The present package provides a simple interface
to make child files individually compilable by \LaTeX{}.
Compiling a child file then has the same effect as compiling
the main file with an |\includeonly| command
to select the appropriate child.
Moreover the generated document will carry the name of the child
rather than the main file.
This resolves all three above issues.

This feature is meant to make the editing of books,
thesis documents and lecture notes somewhat more convenient.
However, the package can also be used efficiently for
composing a series of documents (such as exercise sheets)
which are typically distributed individually.
It then assists the author in generating the individual documents
(potentially in different versions)
as well as a document containing the collected series.
Another application is in developing style files
or other kinds of included material
where compilation of the style file could redirect
to a sample or test file.

%%%%%%%%%%%%%%%%%%%%%%%%%%%%%%%%%%%%%%%%%%%%%%%%%%%%%%%%%%%%%%%%%%%%%%%%%%%%%%%%
%%%%%%%%%%%%%%%%%%%%%%%%%%%%%%%%%%%%%%%%%%%%%%%%%%%%%%%%%%%%%%%%%%%%%%%%%%%%%%%%
\section{Usage}

First of all, the package \textsf{childdoc} is \emph{not} a standard
\LaTeXe{} |.sty| style file! Therefore it needs to be invoked in
a non-standard way.

%%%%%%%%%%%%%%%%%%%%%%%%%%%%%%%%%%%%%%%%%%%%%%%%%%%%%%%%%%%%%%%%%%%%%%%%%%%%%%%%
\subsection{Included Files}
\label{sec:include}

%%%%%%%%%%%%%%%%%%%%%%%%%%%%%%%%%%%%%%%%
\DescribeMacro{\childdocmain}
To use the package, add the commands
\begin{center}
\begin{tabular}{l}
|\input{childdoc.def}|\\
|\childdocmain{}|\\
\end{tabular}
\end{center}
at the very top of the main \LaTeX{} file,
in particular \emph{before} the |\documentclass| statement!
The argument of |\childdocmain| should be left empty
(but it must be present).

%%%%%%%%%%%%%%%%%%%%%%%%%%%%%%%%%%%%%%%%
\DescribeMacro{\childdocof}
Furthermore, add the commands
\begin{center}
\begin{tabular}{l}
|\input{childdoc.def}|\\
|\childdocof{|\textit{main}|}|\\
\end{tabular}
\end{center}
at the top of every child file \textit{child}
which is included by |\include{|\textit{child}|}|
from within the main file
(or at least for those files to be compiled individually).
The argument \textit{main} must be the filename of the main file.

There are a couple of
considerations in setting up the main and child documents:

%%%%%%%%%%%%%%%%%%%%%%%%%%%%%%%%%%%%%%%%
\paragraph{Restrictions.}

Please note the following restrictions:
\begin{itemize}
\item
|\childdocmain| must be called with one argument \textit{main}
to ensure compatibility with earlier version of the package.
It must either be empty (|\childdocmain{}|)
or precisely match the filename of the main file in which it is specified.
See \secref{sec:detection} for further information.
\item
The filename \textit{main} must be specified without the |.tex| extension.
\item
The filename \textit{main} is case sensitive
(even in case-insensitive file systems)
due to internal string comparison.
\item
The argument \textit{main} should be fully expanded, it cannot be a macro.
\item
Subdirectories and special characters should be avoided in filenames.
\item
The command |\childdocmain{|\textit{main}|}| must be followed by a whitespace.
It should not be followed immediately by another command
or by a comment mark `|%|'.
This is because the \TeX{} parser reads the token immediately following
the argument of |\childdocmain| and puts it
at the beginning of every child section;
however, a white\-space is ignored.
\end{itemize}

%%%%%%%%%%%%%%%%%%%%%%%%%%%%%%%%%%%%%%%%
\paragraph{Content of Main File.}

It is advisable to place all content in the child files included by |\include|.
Any output contained in the main file will appear in all child documents
unless suppressed manually;
it cannot be suppressed automatically by the |\includeonly| directive
and thus should normally be avoided.
A method to include some content in the main file
by means of conditional processing is described in \secref{sec:conditional}.

%%%%%%%%%%%%%%%%%%%%%%%%%%%%%%%%%%%%%%%%
\paragraph{Page Numbering.}

When only a part of the document is compiled,
the appropriate numbering of pages
(as well as other status parameters)
is determined from the |.aux| files.
The latter contain information from previous passes.
However this information needs to propagate through
all intermediate child documents.
Therefore the page numbering in child documents may well
be inconsistent until the complete document is compiled at least once.

A useful (if unconventional) way to always ensure a consistent
page numbering is to restart the numbering in each child document
and denote the pages by `\textit{child}|.|\textit{page}'
where \textit{child} represents the chapter/section number of the child file.
This can be achieved by the command
|\numberwithin{page}{|\textit{child}|}|
of the \textsf{amsmath} package
where \textit{child} can be |chapter| or |section|
depending on the chosen structuring.
Alternatively, one can modify the macro |\thepage| appropriately
and reset the counter |page| at the start of each child file.

%%%%%%%%%%%%%%%%%%%%%%%%%%%%%%%%%%%%%%%%%%%%%%%%%%%%%%%%%%%%%%%%%%%%%%%%%%%%%%%%
\subsection{Conditional Processing}
\label{sec:conditional}

The package provides a mechanism to compile different versions
of a document. To customise the versions further some conditional processing
can come in handy to distinguish which version is being compiled.
The package provides two macros to describe the compilation context:

%%%%%%%%%%%%%%%%%%%%%%%%%%%%%%%%%%%%%%%%
\DescribeMacro{\ifchilddoc}
The conditional |\ifchilddoc| distinguishes between the compilation of
child documents and the main document:
%
\begin{center}
|\ifchilddoc |\textit{child-code}| |[|\||else |\textit{main-code}]| \||fi|
\end{center}

%%%%%%%%%%%%%%%%%%%%%%%%%%%%%%%%%%%%%%%%
\DescribeMacro{\childdocname}
\DescribeMacro{\childdocjob}
The macro |\childdocname| contains the filename (without extension)
of the main or child file being processed.
Note that |\childdocjob| will always contain the name of the main file.

%%%%%%%%%%%%%%%%%%%%%%%%%%%%%%%%%%%%%%%%
\paragraph{Title Page.}

Conditional processing can be used to include a title or banner page
in the main document when proper precautions are taken.
Importantly, the code in the main file should ensure that the page counter
(as well as other status parameters which are stored in the |.aux| files)
takes the same value after the conditional processing.
Otherwise the page numbers may take divergent values
depending on which part is compiled.

For example, a title page could be declared by:
%
\begin{center}
\begin{tabular}{l}
|\ifchilddoc\||else|\\
|\addtocounter{page}{-1}|\\
\textit{code for title page}\\
|\newpage|\\
|\||fi|
\end{tabular}
\end{center}
%
A banner page for the child documents can be generated by:
%
\begin{center}
\begin{tabular}{l}
|\ifchilddoc|\\
|\addtocounter{page}{-1}|\\
\textit{code for banner page}\\
|\newpage|\\
|\||fi|
\end{tabular}
\end{center}
%
Here one could write a message such as:
\begin{center}
|This is the part \childdocname{} of \childdocjob{}.|
\end{center}

%%%%%%%%%%%%%%%%%%%%%%%%%%%%%%%%%%%%%%%%%%%%%%%%%%%%%%%%%%%%%%%%%%%%%%%%%%%%%%%%
\subsection{Flags}
\label{sec:flags}

The package makes it easy to generate different versions
of the main or child documents.
To this end compilation flags can be defined
and assigned different default values.
They will be particularly useful in conjunction
with the forwarding mechanism described in \secref{sec:forward}.

For example, it may be useful to have a flag |\version|
which can be set to |draft| or |final|.
The document source will contain some conditional code
depending on the value of |\version|.
Suppose further, the flag should default to |final| for the main file
and to |draft| for child files
which is a natural assignment for editing the document.
This is achieved by placing the following code
in the preamble of the main document
(below the |\childdocmain| directive):
%
\begin{center}
\begin{tabular}{l}
|\ifchilddoc|\\
|\providecommand{\version}{draft}|\\
|\||else|\\
|\providecommand{\version}{final}|\\
|\||fi|
\end{tabular}
\end{center}
%
The definition by |\providecommand| makes sure
that previous definitions are not overwritten.
Further statements |\providecommand{\version}{...}|
can thus be added before the above code to override it.

For the main file, one might add a line
(between |\childdocmain| and the above block)
%
\begin{center}
|%\ifchilddoc\||else\providecommand{\version}{draft}\||fi|
\end{center}
%
which can be uncommented to produce a draft version.
Likewise one can add a line to the very top of a child file
(above the |\childdocof{|\textit{main}|}| directive)
%
\begin{center}
|%\providecommand{\version}{final}|
\end{center}
%
which can be uncommented to produce the final version of this child document.

%%%%%%%%%%%%%%%%%%%%%%%%%%%%%%%%%%%%%%%%%%%%%%%%%%%%%%%%%%%%%%%%%%%%%%%%%%%%%%%%
\subsection{Forwarding}
\label{sec:forward}

Different versions of the main or child documents
using compilation flags as described in \secref{sec:flags}
can be (permanently) stored in different files
for convenient compilation, viewing and distribution.
To this end, the package defines a command
to pass on compilation to a different file:

%%%%%%%%%%%%%%%%%%%%%%%%%%%%%%%%%%%%%%%%
\DescribeMacro{\childdocforward}
The command |\childdocforward| redirects processing to
another source file:
%
\begin{center}
\begin{tabular}{l}
|\input{childdoc.def}|\\
|\childdocforward[|\textit{main}|]{|\textit{dest}|}|\\
\end{tabular}
\end{center}
%
The argument \textit{dest} is the destination file
(without extension).
It should be the main file or one of the child files.
Note that further \textsf{childdoc} directives
such as |\childdocof| and |\childdocforward|
in the indicated file will be processed in this form.
The optional argument \textit{main}
passes on directly to the main file \textit{main}
while pretending to compile the child \textit{dest}.
This form behaves as if \textit{dest}
issues |\childdocof{|\textit{main}|}| right away,
and no further \textsf{childdoc} directives will be processed.

%%%%%%%%%%%%%%%%%%%%%%%%%%%%%%%%%%%%%%%%
\DescribeMacro{\...prefix}
In the alternative form |\childdocforwardprefix|,
%
\begin{center}
\begin{tabular}{l}
|\input{childdoc.def}|\\
|\childdocforwardprefix[|\textit{main}|]{|\textit{prefix}|}{|\textit{dest}|}|
\end{tabular}
\end{center}
%
the destination file is determined by a pattern
depending on the current file:
To make this work, the current file must be called
`{\textit{prefix}\hspace{0.2em}\textit{suffix}}'
with \textit{prefix} matching precisely the argument.
Processing is then passed on to the file
`{\textit{dest}\hspace{0.2em}\textit{suffix}}'.
Surely, the same effect is achieved by
directly specifying the
argument `{\textit{dest}\hspace{0.2em}\textit{suffix}}'
in the first form.
However, that requires to set up a different file
for each child. With the alternative form of the command
all these files can have exactly the same content
which simplifies setting them up and maintaining them.

For example, the following file |draft.tex|
with a compilation flag |\version| as described in \secref{sec:flags}
compiles the main document as a draft:
%
\begin{center}
\begin{tabular}{l}
|\def\version{draft}|\\
|\input{childdoc.def}|\\
|\childdocforward{|\textit{main}|}|
\end{tabular}
\end{center}
%
Likewise, the following files |final|\textit{nn}|.tex|
compile the final version of the child document
|child|\textit{nn}|.tex|:
%
\begin{center}
\begin{tabular}{l}
|\def\version{final}|\\
|\input{childdoc.def}|\\
|\childdocforwardprefix{final}{child}|
\end{tabular}
\end{center}
%

Note that when several versions of a main file and/or of each child file
are to be generated, it may be convenient to set up a |Makefile| or
shell script to automatise the process.

%%%%%%%%%%%%%%%%%%%%%%%%%%%%%%%%%%%%%%%%%%%%%%%%%%%%%%%%%%%%%%%%%%%%%%%%%%%%%%%%
\subsection{Command Line Processing}
\label{sec:commandline}

The effect of redirection files can also be achieved by invoking
the \LaTeX{} compiler with a more elaborate command line.
Most conveniently this should be done as part
of a shell script or a |Makefile|.

When using \textsf{childdoc} in the main file, the following
command lines effectively perform a redirection
(note that depending on the shell being used,
backslashes may have to be doubled: `|\|' $\to$ `|\\|'):
%
\begin{center}
|... -jobname "|\textit{target}|" |\\|"|[\textit{flags}]%
|\input{childdoc.def}\childdocforward[|\textit{main}|]{|\textit{dest}|}"|
\end{center}
%
Here \textit{target} is the name of the output file,
\textit{main} is the name of the main file
and \textit{dest} is the name of the main or child file to be processed
(all filenames without extensions).
The optional argument \textit{main} can be omitted
if \textit{main} matches \textit{dest}.
Optionally, compilation \textit{flags} can be defined via |\def| commands.
This command line makes the \TeX{} engine believe
it is compiling the file \textit{target}
whose content is specified as the latter parameter.
The provided code then forwards the processing to
\textit{main} or \textit{dest} as described in \secref{sec:forward}.

%%%%%%%%%%%%%%%%%%%%%%%%%%%%%%%%%%%%%%%%%%%%%%%%%%%%%%%%%%%%%%%%%%%%%%%%%%%%%%%%
\subsection{Include by Input}
\label{sec:input}

Including child documents by |\include| has some restrictions by design.
Most notably, the content of a child document always occupies
its own set of pages; pages cannot be shared between child documents.
Usually, this behaviour makes perfect sense
because each child document contain an essential part of the document.
However, in some situations it may be desirable to compose
a document from a collection of parts
without having mandatory page breaks between then.
For this case, the package
provides a mechanism to include parts
by |\input| which can also be processed individually.
However, by construction this mechanism
requires manual handling of the content to be output.

%%%%%%%%%%%%%%%%%%%%%%%%%%%%%%%%%%%%%%%%
\DescribeMacro{\ifchilddocmanual}
The main file should be prepared as usual, see \secref{sec:include}.
However, the document body must make a distinction
between processing of an individual part and of the main document, e.g.:
%
\begin{center}
\begin{tabular}{l}
|\ifchilddocmanual|\\
|\input{\childdocname}|\\
|\||else|\\
\textit{document body with }|\input{|\textit{part}|}|\\
|\||fi|
\end{tabular}
\end{center}
%
The conditional |\ifchilddocmanual| is true whenever
a part to be included by |\input| is being compiled,
and the name of the part is stored in |\childdocname|.

%%%%%%%%%%%%%%%%%%%%%%%%%%%%%%%%%%%%%%%%
\DescribeMacro{\childdocby}
Each part to be included by |\input| should start with:
%
\begin{center}
\begin{tabular}{l}
|\input{childdoc.def}|\\
|\childdocby{|\textit{main}|}|\\
\end{tabular}
\end{center}
%
The directive |\childdocby| is similar to |\childdocof|
described in \secref{sec:include},
but the subsequent selection of content must be done manually.
To that end, both |\ifchilddoc| and |\ifchilddocmanual|
will be true upon processing of a part,
and the name of the part is stored in |\childdocname|.
Note that |\jobname| will be set to the filename of the current part
so that each part receives an individual |.aux| file
that does not interfere with the |.aux| file(s) of the main document.
This behaviour can be altered by the alternative form
|\childdocby[*]{|\textit{main}|}| (with a non-empty optional argument)
which uses the |.aux| file of the main document
by setting |\jobname| to \textit{main}.

%%%%%%%%%%%%%%%%%%%%%%%%%%%%%%%%%%%%%%%%%%%%%%%%%%%%%%%%%%%%%%%%%%%%%%%%%%%%%%%%
\subsection{Driver Development}
\label{sec:driver}

The \textsf{childdoc} mechanism can also be use for the development
of definition files such as \LaTeX{} styles or classes.
This case differs from the above setup with multiple parts
included by |\include| in that no |\includeonly| should be invoked.
This can be achieved by starting the include file
(before |\ProvidesPackage|) with:
%
\begin{center}
\begin{tabular}{l}
|\input{childdoc.def}|\\
|\childdocforward{|\textit{main}|}|\\
\end{tabular}
\end{center}
%
or alternatively with:
%
\begin{center}
\begin{tabular}{l}
|\input{childdoc.def}|\\
|\childdocby{|\textit{main}|}|\\
\end{tabular}
\end{center}
%
Both forms have slightly different effects as described above.
The main file is prepared as usual, see \secref{sec:include}.

%%%%%%%%%%%%%%%%%%%%%%%%%%%%%%%%%%%%%%%%%%%%%%%%%%%%%%%%%%%%%%%%%%%%%%%%%%%%%%%%
\subsection{Legacy Detection}
\label{sec:detection}

The directive |\childdocmain| in the main file can detect
whether the complete document or merely a child is to be compiled
even without using the directive |\childdocof|.
This method is deprecated because it is less robust
and there is no compelling reason to use it;
it is merely provided for backward compatibility
and it may be removed in future versions.

If the detection mechanism is to be used,
it is mandatory to correctly specify
the filename of the main file as the argument of |\childdocmain|:
%
\begin{center}
\begin{tabular}{l}
|\input{childdoc.def}|\\
|\childdocmain{|\textit{main}|}|\\
\end{tabular}
\end{center}
%
If |\jobname| does not match the argument \textit{main} of |\childdocmain|,
it is assumed that |\jobname| points to the child file to be compiled.
When using |\childdocmain| with the main file specified as argument,
it suffices to start a child file
with just |\input{|\textit{main}|}|
without loading of the package and using |\childdocof|.
If instead all processing is done
with the appropriate \textsf{childdoc} directives,
the argument of \textit{main} of |\childdocmain| can be empty.

An alternative version of the command line processing described
in \secref{sec:commandline} using the detection mechanism reads:
%
\begin{center}
|... -jobname "|\textit{target}|" "|[\textit{flags}]%
[|\def\jobname{|\textit{dest}|}|]|\input{|\textit{main}|}"|
\end{center}

%%%%%%%%%%%%%%%%%%%%%%%%%%%%%%%%%%%%%%%%%%%%%%%%%%%%%%%%%%%%%%%%%%%%%%%%%%%%%%%%
\subsection{Manual Code}
\label{sec:manual}

In case one cannot be certain whether the definitions file |childdoc.def|
is installed on the target \TeX{} distribution
and one prefers not to ship it,
it is conceivable to paste a few relevant commands into the sources.

To that end, drop all statements |\input{childdoc.def}|
and perform the replacements as outlined below.
Instead of |\childdocmain{|\textit{main}|}| add the following code
to the top of the main file:
%
\begin{center}
\begin{tabular}{l}
|\||ifdefined\childdocname\endinput\||fi\newif\ifchilddoc|\\
|\edef\childdocname{\scantokens\expandafter{\jobname\noexpand}}|\\
|\def\childdocmain{|\textit{main}|}\||ifx\childdocmain\childdocname\||else|\\
|\childdoctrue\includeonly{\childdocname}\let\jobname\childdocmain\||fi|\\
\end{tabular}
\end{center}
%
Instead of |\childdocof{|\textit{main}|}| just include the main file
at the top of each child file:
%
\begin{center}
|\input{|\textit{main}|}|
\end{center}
%
A simple redirection |\childdocforward{|\textit{dest}|}| is achieved by:
%
\begin{center}
|\def\jobname{|\textit{dest}|}\input{\jobname}|
\end{center}
%
The redirection with prefix
|\childdocforwardprefix[|\textit{prefix}|]{|\textit{dest}|}|
is accomplished by:
%
\begin{center}
\begin{tabular}{l}
|{\edef\jobname{\scantokens\expandafter{\jobname\noexpand}}|\\
|\def\redirectjob |\textit{prefix}|#1~~~{\gdef\jobname{|\textit{dest}|#1}}|\\
|\expandafter\redirectjob\jobname~~~}\input{\jobname}|
\end{tabular}
\end{center}

In an alternative approach,
child documents can be compiled by a specific command line
without additional code or specific definitions:
%
\begin{center}
|... -jobname "|\textit{target}|" "|[\textit{flags}]%
|\includeonly{|\textit{dest}|}\input{|\textit{main}|}"|
\end{center}
%

%%%%%%%%%%%%%%%%%%%%%%%%%%%%%%%%%%%%%%%%%%%%%%%%%%%%%%%%%%%%%%%%%%%%%%%%%%%%%%%%
%%%%%%%%%%%%%%%%%%%%%%%%%%%%%%%%%%%%%%%%%%%%%%%%%%%%%%%%%%%%%%%%%%%%%%%%%%%%%%%%
\section{Information}

%%%%%%%%%%%%%%%%%%%%%%%%%%%%%%%%%%%%%%%%%%%%%%%%%%%%%%%%%%%%%%%%%%%%%%%%%%%%%%%%
\subsection{Copyright}

Copyright \copyright{} 2017--2018 Niklas Beisert

This work may be distributed and/or modified under the
conditions of the \LaTeX{} Project Public License, either version 1.3
of this license or (at your option) any later version.
The latest version of this license is in
  \url{http://www.latex-project.org/lppl.txt}
and version 1.3 or later is part of all distributions of \LaTeX{}
version 2005/12/01 or later.

This work has the LPPL maintenance status `maintained'.

The Current Maintainer of this work is Niklas Beisert.

This work consists of the files |README.txt|, |childdoc.ins| and |childdoc.dtx|
as well as the derived files |childdoc.def|, |cdocsamp.tex|
with |cdocsch1.tex|, |cdocsch2.tex|, |cdocspt3.tex|, |cdocspt4.tex|,
|cdocsdrf.tex|, |cdocsfn1.tex|, |cdocsfn2.tex|
as well as |childdoc.pdf|.

%%%%%%%%%%%%%%%%%%%%%%%%%%%%%%%%%%%%%%%%%%%%%%%%%%%%%%%%%%%%%%%%%%%%%%%%%%%%%%%%
\subsection{Files and Installation}

The package consists of the files:
%
\begin{center}
\begin{tabular}{ll}
    |README.txt|   & readme file \\
    |childdoc.ins| & installation file \\
    |childdoc.dtx| & source file \\
    |childdoc.def| & definition file \\
    |cdocsamp.tex| & sample main file \\
    |cdocsch1.tex| & sample include file \\
    |cdocsch2.tex| & sample include file \\
    |cdocspt3.tex| & sample part file \\
    |cdocspt4.tex| & sample part file \\
    |cdocsdrf.tex| & sample redirection file \\
    |cdocsfn1.tex| & sample redirection file \\
    |cdocsfn2.tex| & sample redirection file \\
    |childdoc.pdf| & manual
\end{tabular}
\end{center}
%
The distribution consists of the files
|README.txt|, |childdoc.ins| and |childdoc.dtx|.
%
\begin{itemize}
\item
Run (pdf)\LaTeX{} on |childdoc.dtx|
to compile the manual |childdoc.pdf| (this file).
\item
Run \LaTeX{} on |childdoc.ins| to create the definitions file |childdoc.def|
and the sample |cdocsamp.tex| with include files
|cdocsch1.tex|, |cdocsch2.tex|, |cdocspt3.tex|, |cdocspt4.tex|,
|cdocsdrf.tex|, |cdocsfn1.tex|, |cdocsfn2.tex|.
Then copy the file |childdoc.def| to an appropriate directory of your \LaTeX{}
distribution, e.g.\ \textit{texmf-root}|/tex/latex/childdoc|.
\end{itemize}

%%%%%%%%%%%%%%%%%%%%%%%%%%%%%%%%%%%%%%%%%%%%%%%%%%%%%%%%%%%%%%%%%%%%%%%%%%%%%%%%
\subsection{Related CTAN Packages}

There are several other packages which offer a similar functionality:
%
\begin{itemize}
\item
The packages
\href{http://ctan.org/pkg/docmute}{\textsf{docmute}},
\href{http://ctan.org/pkg/includex}{\textsf{includex}} and
\href{http://ctan.org/pkg/standalone}{\textsf{standalone}}
provide commands to include only the document body of
a child file thus allowing both files to be compiled individually.
\item
The packages \href{http://ctan.org/pkg/subdocs}{\textsf{subdocs}}
and \href{http://ctan.org/pkg/subfiles}{\textsf{subfiles}}
provide structures in which the main and child documents can be
encapsulated and allowing them to be compiled individually.
The inclusion mechanism is different from the conventional |\include|.
\item
The package \href{http://ctan.org/pkg/combine}{\textsf{combine}}
is an elaborate solution to combine several documents into one.
\end{itemize}
%
See also the CTAN topic \href{http://ctan.org/topic/subdocs}{\textsf{subdocs}}
for further related packages.
The present package differs from the above solutions in that
a document structure constructed with the conventional |\include| mechanism
just needs two extra commands at the top of every file
such that all constituent files can be compiled individually.

%%%%%%%%%%%%%%%%%%%%%%%%%%%%%%%%%%%%%%%%%%%%%%%%%%%%%%%%%%%%%%%%%%%%%%%%%%%%%%%%
%\subsection{Feature Suggestions}
%
%The following is a list of features which may be useful for future
%versions of this package:
%%
%\begin{itemize}
%\item
%\ldots
%\end{itemize}

%%%%%%%%%%%%%%%%%%%%%%%%%%%%%%%%%%%%%%%%%%%%%%%%%%%%%%%%%%%%%%%%%%%%%%%%%%%%%%%%
\subsection{Revision History}

%%%%%%%%%%%%%%%%%%%%%%%%%%%%%%%%%%%%%%%%
\paragraph{v2.0:} 2018/12/30

\begin{itemize}
\item
immediate forward processing
\item
added |\childdocby| mechanism
\item
manual restructured
\end{itemize}

%%%%%%%%%%%%%%%%%%%%%%%%%%%%%%%%%%%%%%%%
\paragraph{v1.6:} 2018/01/17

\begin{itemize}
\item
application for development of include files
\item
corrections to manual
\end{itemize}

%%%%%%%%%%%%%%%%%%%%%%%%%%%%%%%%%%%%%%%%
\paragraph{v1.5:} 2017/05/21

\begin{itemize}
\item
more complete structuring introduced
\item
|\childdocof| introduced
\item
|\childdoc| renamed to |\childdocmain|
\item
|\childredirect| renamed to |\childdocforward| and |\childdocforwardprefix|
and functionality expanded
\end{itemize}

%%%%%%%%%%%%%%%%%%%%%%%%%%%%%%%%%%%%%%%%
\paragraph{v1.0:} 2017/04/27

\begin{itemize}
\item
manual and install package
\item
first version published on CTAN
\end{itemize}

%%%%%%%%%%%%%%%%%%%%%%%%%%%%%%%%%%%%%%%%
\paragraph{v0.6:} 2017/04/26

\begin{itemize}
\item
redirection mechanism added
\end{itemize}

%%%%%%%%%%%%%%%%%%%%%%%%%%%%%%%%%%%%%%%%
\paragraph{v0.5:} 2017/04/26

\begin{itemize}
\item
functionality in definition file
\end{itemize}


%%%%%%%%%%%%%%%%%%%%%%%%%%%%%%%%%%%%%%%%%%%%%%%%%%%%%%%%%%%%%%%%%%%%%%%%%%%%%%%%
%%%%%%%%%%%%%%%%%%%%%%%%%%%%%%%%%%%%%%%%%%%%%%%%%%%%%%%%%%%%%%%%%%%%%%%%%%%%%%%%
%%%%%%%%%%%%%%%%%%%%%%%%%%%%%%%%%%%%%%%%%%%%%%%%%%%%%%%%%%%%%%%%%%%%%%%%%%%%%%%%
\appendix

\settowidth\MacroIndent{\rmfamily\scriptsize 000\ }

 \DocInput{childdoc.dtx}

\end{document}
%</driver>
% \fi
%
% %%%%%%%%%%%%%%%%%%%%%%%%%%%%%%%%%%%%%%%%%%%%%%%%%%%%%%%%%%%%%%%%%%%%%%%%%%%%%%
% %%%%%%%%%%%%%%%%%%%%%%%%%%%%%%%%%%%%%%%%%%%%%%%%%%%%%%%%%%%%%%%%%%%%%%%%%%%%%%
% \section{Sample}
%\iffalse
%<*samplemain>
%\fi
%
% The following presents a sample document
% with two chapters, two parts, a title page,
% a compile flag as well as three forwarding files to set the flag.
% It consists of eight |.tex| files:
% \begin{center}
% \begin{tabular}{ll}
% |cdocsamp.tex|&main file\\
% |cdocsch1.tex|&include file for chapter 1\\
% |cdocsch2.tex|&include file for chapter 2\\
% |cdocspt3.tex|&include file for part 3\\
% |cdocspt4.tex|&include file for part 4\\
% |cdocsdrf.tex|&forwarding file for main file in draft mode\\
% |cdocsfi1.tex|&forwarding file for final version of chapter 1\\
% |cdocsfi2.tex|&forwarding file for final version of chapter 2\\
% \end{tabular}
% \end{center}
% Each of the eight files can be compiled directly by the \LaTeX{} compiler.
%
% %%%%%%%%%%%%%%%%%%%%%%%%%%%%%%%%%%%%%%
% \paragraph{Main File.}
%
% The main file is called |cdocsamp.tex|.
%
% Load the \textsf{childdoc} definitions and
% declare the filename for the main document:
%    \begin{macrocode}
\input{childdoc.def}
\childdocmain{}
%    \end{macrocode}

% Optional override for |\version| flag:
%    \begin{macrocode}
%%\ifchilddoc\else\providecommand{\version}{draft}\fi
%    \end{macrocode}

% Define the default values for the |\version| flag
% (|final| for the main file and |draft| for childs):
%    \begin{macrocode}
\ifchilddoc
\providecommand{\version}{draft}
\else
\providecommand{\version}{final}
\fi
%    \end{macrocode}

% Load the standard document class:
%    \begin{macrocode}
\documentclass[12pt]{article}
%    \end{macrocode}

% Start the document body:
%    \begin{macrocode}
\begin{document}
%    \end{macrocode}

% Declare a title page.
% Print title, part of document being processed and version flag:
%    \begin{macrocode}
\addtocounter{page}{-1}
\begin{center}
{\LARGE\bfseries{}childdoc example\par}
\vspace{1cm}
\ifchilddoc
\ifchilddocmanual part\else chapter\fi:
`\childdocname' of `\childdocjob'\par
\else
main document: `\childdocjob'\par
\fi
version: \version\par
\end{center}
\newpage
%    \end{macrocode}

% Manually include selected file,
% otherwise process as usual:
%    \begin{macrocode}
\ifchilddocmanual
\section*{part `\childdocname'}
\input{\childdocname}
\else
%    \end{macrocode}

% Include the two chapters:
%    \begin{macrocode}
\include{cdocsch1}
\include{cdocsch2}
%    \end{macrocode}

% Include the two parts unless only chapters should be displayed:
%    \begin{macrocode}
\ifchilddoc\else
\section{part three}
\input{cdocspt3}
\section{part four}
\input{cdocspt4}
\fi
%    \end{macrocode}

% Process as usual until here:
%    \begin{macrocode}
\fi
%    \end{macrocode}

% End of document body:
%    \begin{macrocode}
\end{document}
%    \end{macrocode}
%\iffalse
%</samplemain>
%\fi
%
% %%%%%%%%%%%%%%%%%%%%%%%%%%%%%%%%%%%%%%
% \paragraph{Chapter Include Files.}
%
% The include files are called |cdocsch1.tex| and |cdocsch2.tex|.
%
%\iffalse
%<*samplechap1|samplechap2>
%\fi

% Optional override for |\version| flag:
%    \begin{macrocode}
%%\providecommand{\version}{final}
%    \end{macrocode}

% Include the main document:
%    \begin{macrocode}
\input{childdoc.def}
\childdocof{cdocsamp}
%    \end{macrocode}

%\iffalse
%</samplechap1|samplechap2>
%\fi
%
%\iffalse
%<*samplechap1>
%\fi
% Some text for chapter 1:
%    \begin{macrocode}
\section{one}
some text in chapter one
%    \end{macrocode}

%\iffalse
%</samplechap1>
%\fi
% Some text for chapter 2:
%\iffalse
%<*samplechap2>
%\fi
%    \begin{macrocode}
\section{two}
more text in chapter two
%    \end{macrocode}

%\iffalse
%</samplechap2>
%\fi
%
% %%%%%%%%%%%%%%%%%%%%%%%%%%%%%%%%%%%%%%
% \paragraph{Part Include Files.}
%
% The include files are called |cdocspt3.tex| and |cdocspt4.tex|.
%
%\iffalse
%<*samplepart3|samplepart4>
%\fi

% Optional override for |\version| flag:
%    \begin{macrocode}
%%\providecommand{\version}{final}
%    \end{macrocode}

% Include the main document:
%    \begin{macrocode}
\input{childdoc.def}
\childdocby{cdocsamp}
%    \end{macrocode}

%\iffalse
%</samplepart3|samplepart4>
%\fi
%
%\iffalse
%<*samplepart3>
%\fi
% Some text for part 3:
%    \begin{macrocode}
some text in part three
%    \end{macrocode}

%\iffalse
%</samplepart3>
%\fi
% Some text for part 4:
%\iffalse
%<*samplepart4>
%\fi
%    \begin{macrocode}
more text in part four
%    \end{macrocode}

%\iffalse
%</samplepart4>
%\fi
%
% %%%%%%%%%%%%%%%%%%%%%%%%%%%%%%%%%%%%%%
% \paragraph{Forwarding for a Complete Draft.}
%
% The following forwarding file |cdocsdrf.tex|
% compiles the main document in draft mode:
%\iffalse
%<*sampledraft>
%\fi
%    \begin{macrocode}
\def\version{draft}
\input{childdoc.def}
\childdocforward{cdocsamp}
%    \end{macrocode}

%\iffalse
%</sampledraft>
%\fi
%
% %%%%%%%%%%%%%%%%%%%%%%%%%%%%%%%%%%%%%%
% \paragraph{Forwarding for Final Version of the Chapters.}
%
% The following forwarding files |cdocsfn1.tex| and |cdocsfn2.tex|
% (with identical content)
% compile the final versions of the child documents
% |cdocsch1.tex| and |cdocsch2.tex|, respectively:
%\iffalse
%<*samplefinal>
%\fi
%    \begin{macrocode}
\def\version{final}
\input{childdoc.def}
\childdocforwardprefix[cdocsamp]{cdocsfn}{cdocsch}
%    \end{macrocode}

%\iffalse
%</samplefinal>
%\fi
%
% %%%%%%%%%%%%%%%%%%%%%%%%%%%%%%%%%%%%%%
% \paragraph{Command Line Processing.}
%
% The following three command lines generate the output files
% |cdocscld|, |cdocscl1| and |cdocscl2|
% which should be identical to
% |cdocsdrf|, |cdocsch1| and |cdocsfn2|, respectively:
% \begin{center}
% \begin{tabular}{l}
% |latex -jobname cdocscld \|\\
% |  "\def\version{draft}\input{childdoc.def}\childdocforward{cdocsamp}"|\\
% |latex -jobname cdocscl1 \|\\
% |  "\input{childdoc.def}\childdocforward[cdocsamp]{cdocsch1}"|\\
% |latex -jobname cdocscl2 \|\\
% |  "\def\version{final}\input{childdoc.def}\childdocforward{cdocsch2}"|
% \end{tabular}
% \end{center}
% Note that the trailing backslash on each first line
% merely continues the input to the second line
% (for convenient cut ant paste).
% Furthermore, the command |latex| can be replaced by any
% of its alternative versions such as |pdflatex|.
%
% %%%%%%%%%%%%%%%%%%%%%%%%%%%%%%%%%%%%%%%%%%%%%%%%%%%%%%%%%%%%%%%%%%%%%%%%%%%%%%
% %%%%%%%%%%%%%%%%%%%%%%%%%%%%%%%%%%%%%%%%%%%%%%%%%%%%%%%%%%%%%%%%%%%%%%%%%%%%%%
% \section{Implementation}
%\iffalse
%<*package>
%\fi
%
% This section describes the definitions file |childdoc.def|.

% The definitions cannot be loaded using |\usepackage| or |\RequirePackage|
% which has a mechanism to prevent loading a style file more than once.
% When loading the definitions by means of |\input|
% multiple instances have to be prevented manually:
%\iffalse
%This code needs to be before the `\ProvidesFile' directive
%which is defined at the beginning of this file.
%Therefore it is also placed there and commented out here.
%</package>
%<*discard>
%\fi
%    \begin{macrocode}
\ifdefined\childdocmain\endinput\fi
%    \end{macrocode}
%\iffalse
%</discard>
%<*package>
%\fi
%
% \macro{\ifchilddoc}
% \macro{\ifchilddocmanual}
% The conditional |\ifchilddoc| tells whether a
% child (true) or main (false) document is being compiled.
% The conditional |\ifchilddocmanual| tells whether
% the |\includeonly| mechanism is used (false) or
% the selection of child files must be performed manually (true).
% The definitions initialise to false:
%    \begin{macrocode}
\newif\ifchilddoc
\newif\ifchilddocmanual
%    \end{macrocode}

% \macro{\childdocname}
% \macro{\childdocjob}
% The macro |\childdocname| stores the name of the main document
% to be compiled. The macro |\childdocjob| stores the name of
% the document on which the \LaTeX{} compiler was originally invoked.
% The content of |\jobname| cannot be compared
% to filenames specified in the source due to different catcodes.
% The following code rescans |\jobname|, stores the result
% in |\childdocname| and saves a copy in |\childdocjob|:
%    \begin{macrocode}
\edef\childdocname{\scantokens\expandafter{\jobname\noexpand}}
\let\childdocjob\childdocname
%    \end{macrocode}

% \macro{\childdocdisable}
% The macro |\childdocdisable| prevents the main file
% from being processed more than once.
% At this stage, the main document command |\childdocmain|
% is assumed to be called once again where it should do nothing.
% Any subsequent call to it should prevent
% a secondary processing of the main document
% It overwrites the forwarding commands
% |\childdocof| and |\childdocforward|
% with empty macros to prevent further inclusions of the main document:
%    \begin{macrocode}
\newcommand{\childdocdisable}
{
  \renewcommand{\childdocmain}[1]{\renewcommand{\childdocmain}[1]{\endinput}}
  \renewcommand{\childdocof}[1]{}
  \renewcommand{\childdocby}[2][]{}
  \renewcommand{\childdocforward}[2][]{}
  \renewcommand{\childdocdisable}{}
}
%    \end{macrocode}

% \macro{\childdocmain}
% The macro |\childdocmain| is to be called at the top of the main file
% with nothing or the main filename (without extension) as argument.
% First, it breaks loops.
% If the argument is not empty and does not match |\childdocname|
% (which is set by the first inclusion of |childdoc.def|),
% |\ifchilddoc| is set to true, |\includeonly| is applied to the child file
% and |\jobname| is set to the main file
% (for proper handling of |.aux| files):
%    \begin{macrocode}
\newcommand{\childdocmain}[1]
{
  \childdocdisable\childdocmain{}
  \if?#1?\else
    \begingroup
      \def\childdoctmp{#1}
      \ifx\childdoctmp\childdocname
        \def\childdoctmp{}
      \else
        \def\childdoctmp
        {
          \childdoctrue
          \includeonly{\childdocname}
          \def\childdocjob{#1}
          \def\jobname{#1}
        }
      \fi
      \expandafter
    \endgroup
    \childdoctmp
  \fi
}
%    \end{macrocode}

% \macro{\childdocof}
% The command |\childdocof| redirects
% compilation to the main file |#1|.
%    \begin{macrocode}
\newcommand{\childdocof}[1]
{
  \childdocdisable
  \childdoctrue
  \includeonly{\childdocname}
  \def\jobname{#1}
  \def\childdocjob{#1}
  \input{#1}
}
%    \end{macrocode}

% \macro{\childdocby}
% The command |\childdocby| ....
%    \begin{macrocode}
\newcommand{\childdocby}[2][]
{
  \childdocdisable
  \childdoctrue
  \childdocmanualtrue
  \if?#1?\else
    \def\jobname{#2}
  \fi
  \def\childdocjob{#2}
  \input{#2}
  \endinput
}
%    \end{macrocode}

% \macro{\childdocforward}
% The command |\childdocforward| redirects
% compilation to the main file or
% (if the optional argument is given) a child file.
% Parameters are set as if the main file
% or a child file starting with |\childdocof| was compiled.
% Then compilation is handed over to the main file:
%    \begin{macrocode}
\newcommand{\childdocforward}[2][]
{
  \begingroup
    \if?#1?
      \def\childdoctmp
      {
        \def\childdocname{#2}
        \def\childdocjob{#2}
        \def\jobname{#2}
        \input{#2}
        \endinput
      }
    \else
      \def\childdoctmp
      {
        \childdocdisable
        \def\childdocname{#2}
        \childdoctrue
        \includeonly{#2}
        \def\childdocjob{#1}
        \def\jobname{#1}
        \input{#1}
        \endinput
      }
    \fi
    \expandafter
  \endgroup
  \childdoctmp
}
%    \end{macrocode}

% \macro{\childdocforwardprefix}
% The command |\childdocforwardprefix| redirects
% compilation to the main or a child file by means of a pattern.
% The prefix |#1| in the current filename is replaced by |#2|
% and the suffix of the current filename is kept
% (it is assumed that the filename does not contain the substring `|~~~|'
% which is used as a delimiter).
% Compilation is handed over to the new file by |\childdocforward|:
%    \begin{macrocode}
\newcommand{\childdocforwardprefix}[3][]
{
  \begingroup
    \def\childdocextract #2##1~~~{\def\childdoctmp{\childdocforward[#1]{#3##1}}}
    \expandafter\childdocextract\childdocname~~~
    \expandafter
  \endgroup
  \childdoctmp
}
%    \end{macrocode}

% \macro{\childdoc}
% The deprecated macro |\childdoc| is a legacy version of |\childdocmain|:
%    \begin{macrocode}
\newcommand{\childdoc}{\childdocmain}
%    \end{macrocode}

% \macro{\childdocredirect}
% The deprecated macro |\childdocredirect| is a legacy version
% of |\childdocforward| and |\childdocforwardprefix|:
%    \begin{macrocode}
\newcommand{\childdocredirect}[2][]
{
  \begingroup
    \if?#1?
      \def\childdoctmp{\childdocforward{#2}}
    \else
      \def\childdoctmp{\childdocforwardprefix{#1}{#2}}
    \fi
    \expandafter
  \endgroup
  \childdoctmp
}
%    \end{macrocode}

%\iffalse
%</package>
%\fi
%
\endinput
|\\
|\childdocof{|\textit{main}|}|\\
\end{tabular}
\end{center}
at the top of every child file \textit{child}
which is included by |\include{|\textit{child}|}|
from within the main file
(or at least for those files to be compiled individually).
The argument \textit{main} must be the filename of the main file.

There are a couple of
considerations in setting up the main and child documents:

%%%%%%%%%%%%%%%%%%%%%%%%%%%%%%%%%%%%%%%%
\paragraph{Restrictions.}

Please note the following restrictions:
\begin{itemize}
\item
|\childdocmain| must be called with one argument \textit{main}
to ensure compatibility with earlier version of the package.
It must either be empty (|\childdocmain{}|)
or precisely match the filename of the main file in which it is specified.
See \secref{sec:detection} for further information.
\item
The filename \textit{main} must be specified without the |.tex| extension.
\item
The filename \textit{main} is case sensitive
(even in case-insensitive file systems)
due to internal string comparison.
\item
The argument \textit{main} should be fully expanded, it cannot be a macro.
\item
Subdirectories and special characters should be avoided in filenames.
\item
The command |\childdocmain{|\textit{main}|}| must be followed by a whitespace.
It should not be followed immediately by another command
or by a comment mark `|%|'.
This is because the \TeX{} parser reads the token immediately following
the argument of |\childdocmain| and puts it
at the beginning of every child section;
however, a white\-space is ignored.
\end{itemize}

%%%%%%%%%%%%%%%%%%%%%%%%%%%%%%%%%%%%%%%%
\paragraph{Content of Main File.}

It is advisable to place all content in the child files included by |\include|.
Any output contained in the main file will appear in all child documents
unless suppressed manually;
it cannot be suppressed automatically by the |\includeonly| directive
and thus should normally be avoided.
A method to include some content in the main file
by means of conditional processing is described in \secref{sec:conditional}.

%%%%%%%%%%%%%%%%%%%%%%%%%%%%%%%%%%%%%%%%
\paragraph{Page Numbering.}

When only a part of the document is compiled,
the appropriate numbering of pages
(as well as other status parameters)
is determined from the |.aux| files.
The latter contain information from previous passes.
However this information needs to propagate through
all intermediate child documents.
Therefore the page numbering in child documents may well
be inconsistent until the complete document is compiled at least once.

A useful (if unconventional) way to always ensure a consistent
page numbering is to restart the numbering in each child document
and denote the pages by `\textit{child}|.|\textit{page}'
where \textit{child} represents the chapter/section number of the child file.
This can be achieved by the command
|\numberwithin{page}{|\textit{child}|}|
of the \textsf{amsmath} package
where \textit{child} can be |chapter| or |section|
depending on the chosen structuring.
Alternatively, one can modify the macro |\thepage| appropriately
and reset the counter |page| at the start of each child file.

%%%%%%%%%%%%%%%%%%%%%%%%%%%%%%%%%%%%%%%%%%%%%%%%%%%%%%%%%%%%%%%%%%%%%%%%%%%%%%%%
\subsection{Conditional Processing}
\label{sec:conditional}

The package provides a mechanism to compile different versions
of a document. To customise the versions further some conditional processing
can come in handy to distinguish which version is being compiled.
The package provides two macros to describe the compilation context:

%%%%%%%%%%%%%%%%%%%%%%%%%%%%%%%%%%%%%%%%
\DescribeMacro{\ifchilddoc}
The conditional |\ifchilddoc| distinguishes between the compilation of
child documents and the main document:
%
\begin{center}
|\ifchilddoc |\textit{child-code}| |[|\||else |\textit{main-code}]| \||fi|
\end{center}

%%%%%%%%%%%%%%%%%%%%%%%%%%%%%%%%%%%%%%%%
\DescribeMacro{\childdocname}
\DescribeMacro{\childdocjob}
The macro |\childdocname| contains the filename (without extension)
of the main or child file being processed.
Note that |\childdocjob| will always contain the name of the main file.

%%%%%%%%%%%%%%%%%%%%%%%%%%%%%%%%%%%%%%%%
\paragraph{Title Page.}

Conditional processing can be used to include a title or banner page
in the main document when proper precautions are taken.
Importantly, the code in the main file should ensure that the page counter
(as well as other status parameters which are stored in the |.aux| files)
takes the same value after the conditional processing.
Otherwise the page numbers may take divergent values
depending on which part is compiled.

For example, a title page could be declared by:
%
\begin{center}
\begin{tabular}{l}
|\ifchilddoc\||else|\\
|\addtocounter{page}{-1}|\\
\textit{code for title page}\\
|\newpage|\\
|\||fi|
\end{tabular}
\end{center}
%
A banner page for the child documents can be generated by:
%
\begin{center}
\begin{tabular}{l}
|\ifchilddoc|\\
|\addtocounter{page}{-1}|\\
\textit{code for banner page}\\
|\newpage|\\
|\||fi|
\end{tabular}
\end{center}
%
Here one could write a message such as:
\begin{center}
|This is the part \childdocname{} of \childdocjob{}.|
\end{center}

%%%%%%%%%%%%%%%%%%%%%%%%%%%%%%%%%%%%%%%%%%%%%%%%%%%%%%%%%%%%%%%%%%%%%%%%%%%%%%%%
\subsection{Flags}
\label{sec:flags}

The package makes it easy to generate different versions
of the main or child documents.
To this end compilation flags can be defined
and assigned different default values.
They will be particularly useful in conjunction
with the forwarding mechanism described in \secref{sec:forward}.

For example, it may be useful to have a flag |\version|
which can be set to |draft| or |final|.
The document source will contain some conditional code
depending on the value of |\version|.
Suppose further, the flag should default to |final| for the main file
and to |draft| for child files
which is a natural assignment for editing the document.
This is achieved by placing the following code
in the preamble of the main document
(below the |\childdocmain| directive):
%
\begin{center}
\begin{tabular}{l}
|\ifchilddoc|\\
|\providecommand{\version}{draft}|\\
|\||else|\\
|\providecommand{\version}{final}|\\
|\||fi|
\end{tabular}
\end{center}
%
The definition by |\providecommand| makes sure
that previous definitions are not overwritten.
Further statements |\providecommand{\version}{...}|
can thus be added before the above code to override it.

For the main file, one might add a line
(between |\childdocmain| and the above block)
%
\begin{center}
|%\ifchilddoc\||else\providecommand{\version}{draft}\||fi|
\end{center}
%
which can be uncommented to produce a draft version.
Likewise one can add a line to the very top of a child file
(above the |\childdocof{|\textit{main}|}| directive)
%
\begin{center}
|%\providecommand{\version}{final}|
\end{center}
%
which can be uncommented to produce the final version of this child document.

%%%%%%%%%%%%%%%%%%%%%%%%%%%%%%%%%%%%%%%%%%%%%%%%%%%%%%%%%%%%%%%%%%%%%%%%%%%%%%%%
\subsection{Forwarding}
\label{sec:forward}

Different versions of the main or child documents
using compilation flags as described in \secref{sec:flags}
can be (permanently) stored in different files
for convenient compilation, viewing and distribution.
To this end, the package defines a command
to pass on compilation to a different file:

%%%%%%%%%%%%%%%%%%%%%%%%%%%%%%%%%%%%%%%%
\DescribeMacro{\childdocforward}
The command |\childdocforward| redirects processing to
another source file:
%
\begin{center}
\begin{tabular}{l}
|% \iffalse
%
% childdoc.dtx Copyright (C) 2017-2018 Niklas Beisert
%
% This work may be distributed and/or modified under the
% conditions of the LaTeX Project Public License, either version 1.3
% of this license or (at your option) any later version.
% The latest version of this license is in
%   http://www.latex-project.org/lppl.txt
% and version 1.3 or later is part of all distributions of LaTeX
% version 2005/12/01 or later.
%
% This work has the LPPL maintenance status `maintained'.
%
% The Current Maintainer of this work is Niklas Beisert.
%
% This work consists of the files childdoc.dtx and childdoc.ins
% and the derived files childdoc.def and cdocsamp.tex with
% cdocsch1.tex, cdocsch2.tex, cdocsdrf.tex, cdocsfn1.tex, cdocsfn2.tex.
%
%<package>\ifdefined\childdocmain\endinput\fi
%<package>\ProvidesFile{childdoc.def}[2018/12/30 v2.0 child document driver]
%<samplemain>\ProvidesFile{cdocsamp.tex}[2018/12/30 v2.0 sample for childdoc]
%<*driver>
%\ProvidesFile{childdoc.drv}[2018/12/30 v2.0 childdoc reference manual file]
\PassOptionsToClass{10pt,a4paper}{article}
\documentclass{ltxdoc}

\usepackage[margin=35mm]{geometry}
\usepackage{hyperref}
\usepackage{hyperxmp}
\usepackage[usenames]{color}

\hypersetup{colorlinks=true}
\hypersetup{pdfstartview=FitH}
\hypersetup{pdfpagemode=UseNone}
\hypersetup{pdfsource={}}
\hypersetup{pdflang={en-UK}}
\hypersetup{pdfcopyright={Copyright 2017-2018 Niklas Beisert.
  This work may be distributed and/or modified under the
  conditions of the LaTeX Project Public License, either version 1.3
  of this license or (at your option) any later version.}}
\hypersetup{pdflicenseurl={http://www.latex-project.org/lppl.txt}}
\hypersetup{pdfcontactaddress={ETH Zurich, ITP, HIT K,
  Wolfgang-Pauli-Strasse 27}}
\hypersetup{pdfcontactpostcode={8093}}
\hypersetup{pdfcontactcity={Zurich}}
\hypersetup{pdfcontactcountry={Switzerland}}
\hypersetup{pdfcontactemail={nbeisert@itp.phys.ethz.ch}}
\hypersetup{pdfcontacturl={http://people.phys.ethz.ch/\xmptilde nbeisert/}}

\newcommand{\secref}[1]{\hyperref[#1]{section \ref*{#1}}}

\parskip1ex
\parindent0pt
\let\olditemize\itemize
\def\itemize{\olditemize\parskip0pt}

\begin{document}

\title{The \textsf{childdoc} Package}
\hypersetup{pdftitle={The childdoc Package}}
\author{Niklas Beisert\\[2ex]
  Institut f\"ur Theoretische Physik\\
  Eidgen\"ossische Technische Hochschule Z\"urich\\
  Wolfgang-Pauli-Strasse 27, 8093 Z\"urich, Switzerland\\[1ex]
  \href{mailto:nbeisert@itp.phys.ethz.ch}
  {\texttt{nbeisert@itp.phys.ethz.ch}}}
\hypersetup{pdfauthor={Niklas Beisert}}
\hypersetup{pdfsubject={Manual for the LaTeX2e Package childdoc}}
\date{30 December 2018, \textsf{v2.0}}
\maketitle

\begin{abstract}\noindent
\textsf{childdoc} is a \LaTeXe{} package
that enables the direct compilation
of document sections included by |\include|
to individual files.
\end{abstract}

\begingroup
\parskip0ex
\tableofcontents
\endgroup

%%%%%%%%%%%%%%%%%%%%%%%%%%%%%%%%%%%%%%%%%%%%%%%%%%%%%%%%%%%%%%%%%%%%%%%%%%%%%%%%
%%%%%%%%%%%%%%%%%%%%%%%%%%%%%%%%%%%%%%%%%%%%%%%%%%%%%%%%%%%%%%%%%%%%%%%%%%%%%%%%
\section{Introduction}

\LaTeX{} provides a mechanism to structure a large document (such as a book)
into a main file and several child files (containing the chapters)
using the |\include| command.
This mechanism is beneficial for documents
which span hundreds of pages in order to
make the source file(s) more manageable.
Moreover, compilation can be restricted to
selected child files by means of the |\includeonly| command.
The latter feature can be used to reduce the compilation time while editing
(this was significantly more useful in the earlier days of \LaTeX{})
or to generate a smaller document which is easier to navigate.
Another application of |\includeonly| is to generate
documents consisting of selected parts of the complete document.

However, there are a few drawbacks of the plain |\include| mechanism:
\begin{itemize}
\item
The child files cannot be compiled on their own,
they can only be compiled via the main file.
A naive editing environment
(such as a text editor with an option
to have the current file processed by \LaTeX)
may require one to switch to the main file before compiling;
attempting to compile the child file produces errors.
\item
The main file must be modified (each time)
to adjust the |\includeonly| command
to the present needs. This easily leaves the main file in a messy state.
\item
The generated document will always carry the filename
of the main document. This is inconvenient if
several child files are to be compiled and
to be kept for distribution.
\end{itemize}

The present package provides a simple interface
to make child files individually compilable by \LaTeX{}.
Compiling a child file then has the same effect as compiling
the main file with an |\includeonly| command
to select the appropriate child.
Moreover the generated document will carry the name of the child
rather than the main file.
This resolves all three above issues.

This feature is meant to make the editing of books,
thesis documents and lecture notes somewhat more convenient.
However, the package can also be used efficiently for
composing a series of documents (such as exercise sheets)
which are typically distributed individually.
It then assists the author in generating the individual documents
(potentially in different versions)
as well as a document containing the collected series.
Another application is in developing style files
or other kinds of included material
where compilation of the style file could redirect
to a sample or test file.

%%%%%%%%%%%%%%%%%%%%%%%%%%%%%%%%%%%%%%%%%%%%%%%%%%%%%%%%%%%%%%%%%%%%%%%%%%%%%%%%
%%%%%%%%%%%%%%%%%%%%%%%%%%%%%%%%%%%%%%%%%%%%%%%%%%%%%%%%%%%%%%%%%%%%%%%%%%%%%%%%
\section{Usage}

First of all, the package \textsf{childdoc} is \emph{not} a standard
\LaTeXe{} |.sty| style file! Therefore it needs to be invoked in
a non-standard way.

%%%%%%%%%%%%%%%%%%%%%%%%%%%%%%%%%%%%%%%%%%%%%%%%%%%%%%%%%%%%%%%%%%%%%%%%%%%%%%%%
\subsection{Included Files}
\label{sec:include}

%%%%%%%%%%%%%%%%%%%%%%%%%%%%%%%%%%%%%%%%
\DescribeMacro{\childdocmain}
To use the package, add the commands
\begin{center}
\begin{tabular}{l}
|\input{childdoc.def}|\\
|\childdocmain{}|\\
\end{tabular}
\end{center}
at the very top of the main \LaTeX{} file,
in particular \emph{before} the |\documentclass| statement!
The argument of |\childdocmain| should be left empty
(but it must be present).

%%%%%%%%%%%%%%%%%%%%%%%%%%%%%%%%%%%%%%%%
\DescribeMacro{\childdocof}
Furthermore, add the commands
\begin{center}
\begin{tabular}{l}
|\input{childdoc.def}|\\
|\childdocof{|\textit{main}|}|\\
\end{tabular}
\end{center}
at the top of every child file \textit{child}
which is included by |\include{|\textit{child}|}|
from within the main file
(or at least for those files to be compiled individually).
The argument \textit{main} must be the filename of the main file.

There are a couple of
considerations in setting up the main and child documents:

%%%%%%%%%%%%%%%%%%%%%%%%%%%%%%%%%%%%%%%%
\paragraph{Restrictions.}

Please note the following restrictions:
\begin{itemize}
\item
|\childdocmain| must be called with one argument \textit{main}
to ensure compatibility with earlier version of the package.
It must either be empty (|\childdocmain{}|)
or precisely match the filename of the main file in which it is specified.
See \secref{sec:detection} for further information.
\item
The filename \textit{main} must be specified without the |.tex| extension.
\item
The filename \textit{main} is case sensitive
(even in case-insensitive file systems)
due to internal string comparison.
\item
The argument \textit{main} should be fully expanded, it cannot be a macro.
\item
Subdirectories and special characters should be avoided in filenames.
\item
The command |\childdocmain{|\textit{main}|}| must be followed by a whitespace.
It should not be followed immediately by another command
or by a comment mark `|%|'.
This is because the \TeX{} parser reads the token immediately following
the argument of |\childdocmain| and puts it
at the beginning of every child section;
however, a white\-space is ignored.
\end{itemize}

%%%%%%%%%%%%%%%%%%%%%%%%%%%%%%%%%%%%%%%%
\paragraph{Content of Main File.}

It is advisable to place all content in the child files included by |\include|.
Any output contained in the main file will appear in all child documents
unless suppressed manually;
it cannot be suppressed automatically by the |\includeonly| directive
and thus should normally be avoided.
A method to include some content in the main file
by means of conditional processing is described in \secref{sec:conditional}.

%%%%%%%%%%%%%%%%%%%%%%%%%%%%%%%%%%%%%%%%
\paragraph{Page Numbering.}

When only a part of the document is compiled,
the appropriate numbering of pages
(as well as other status parameters)
is determined from the |.aux| files.
The latter contain information from previous passes.
However this information needs to propagate through
all intermediate child documents.
Therefore the page numbering in child documents may well
be inconsistent until the complete document is compiled at least once.

A useful (if unconventional) way to always ensure a consistent
page numbering is to restart the numbering in each child document
and denote the pages by `\textit{child}|.|\textit{page}'
where \textit{child} represents the chapter/section number of the child file.
This can be achieved by the command
|\numberwithin{page}{|\textit{child}|}|
of the \textsf{amsmath} package
where \textit{child} can be |chapter| or |section|
depending on the chosen structuring.
Alternatively, one can modify the macro |\thepage| appropriately
and reset the counter |page| at the start of each child file.

%%%%%%%%%%%%%%%%%%%%%%%%%%%%%%%%%%%%%%%%%%%%%%%%%%%%%%%%%%%%%%%%%%%%%%%%%%%%%%%%
\subsection{Conditional Processing}
\label{sec:conditional}

The package provides a mechanism to compile different versions
of a document. To customise the versions further some conditional processing
can come in handy to distinguish which version is being compiled.
The package provides two macros to describe the compilation context:

%%%%%%%%%%%%%%%%%%%%%%%%%%%%%%%%%%%%%%%%
\DescribeMacro{\ifchilddoc}
The conditional |\ifchilddoc| distinguishes between the compilation of
child documents and the main document:
%
\begin{center}
|\ifchilddoc |\textit{child-code}| |[|\||else |\textit{main-code}]| \||fi|
\end{center}

%%%%%%%%%%%%%%%%%%%%%%%%%%%%%%%%%%%%%%%%
\DescribeMacro{\childdocname}
\DescribeMacro{\childdocjob}
The macro |\childdocname| contains the filename (without extension)
of the main or child file being processed.
Note that |\childdocjob| will always contain the name of the main file.

%%%%%%%%%%%%%%%%%%%%%%%%%%%%%%%%%%%%%%%%
\paragraph{Title Page.}

Conditional processing can be used to include a title or banner page
in the main document when proper precautions are taken.
Importantly, the code in the main file should ensure that the page counter
(as well as other status parameters which are stored in the |.aux| files)
takes the same value after the conditional processing.
Otherwise the page numbers may take divergent values
depending on which part is compiled.

For example, a title page could be declared by:
%
\begin{center}
\begin{tabular}{l}
|\ifchilddoc\||else|\\
|\addtocounter{page}{-1}|\\
\textit{code for title page}\\
|\newpage|\\
|\||fi|
\end{tabular}
\end{center}
%
A banner page for the child documents can be generated by:
%
\begin{center}
\begin{tabular}{l}
|\ifchilddoc|\\
|\addtocounter{page}{-1}|\\
\textit{code for banner page}\\
|\newpage|\\
|\||fi|
\end{tabular}
\end{center}
%
Here one could write a message such as:
\begin{center}
|This is the part \childdocname{} of \childdocjob{}.|
\end{center}

%%%%%%%%%%%%%%%%%%%%%%%%%%%%%%%%%%%%%%%%%%%%%%%%%%%%%%%%%%%%%%%%%%%%%%%%%%%%%%%%
\subsection{Flags}
\label{sec:flags}

The package makes it easy to generate different versions
of the main or child documents.
To this end compilation flags can be defined
and assigned different default values.
They will be particularly useful in conjunction
with the forwarding mechanism described in \secref{sec:forward}.

For example, it may be useful to have a flag |\version|
which can be set to |draft| or |final|.
The document source will contain some conditional code
depending on the value of |\version|.
Suppose further, the flag should default to |final| for the main file
and to |draft| for child files
which is a natural assignment for editing the document.
This is achieved by placing the following code
in the preamble of the main document
(below the |\childdocmain| directive):
%
\begin{center}
\begin{tabular}{l}
|\ifchilddoc|\\
|\providecommand{\version}{draft}|\\
|\||else|\\
|\providecommand{\version}{final}|\\
|\||fi|
\end{tabular}
\end{center}
%
The definition by |\providecommand| makes sure
that previous definitions are not overwritten.
Further statements |\providecommand{\version}{...}|
can thus be added before the above code to override it.

For the main file, one might add a line
(between |\childdocmain| and the above block)
%
\begin{center}
|%\ifchilddoc\||else\providecommand{\version}{draft}\||fi|
\end{center}
%
which can be uncommented to produce a draft version.
Likewise one can add a line to the very top of a child file
(above the |\childdocof{|\textit{main}|}| directive)
%
\begin{center}
|%\providecommand{\version}{final}|
\end{center}
%
which can be uncommented to produce the final version of this child document.

%%%%%%%%%%%%%%%%%%%%%%%%%%%%%%%%%%%%%%%%%%%%%%%%%%%%%%%%%%%%%%%%%%%%%%%%%%%%%%%%
\subsection{Forwarding}
\label{sec:forward}

Different versions of the main or child documents
using compilation flags as described in \secref{sec:flags}
can be (permanently) stored in different files
for convenient compilation, viewing and distribution.
To this end, the package defines a command
to pass on compilation to a different file:

%%%%%%%%%%%%%%%%%%%%%%%%%%%%%%%%%%%%%%%%
\DescribeMacro{\childdocforward}
The command |\childdocforward| redirects processing to
another source file:
%
\begin{center}
\begin{tabular}{l}
|\input{childdoc.def}|\\
|\childdocforward[|\textit{main}|]{|\textit{dest}|}|\\
\end{tabular}
\end{center}
%
The argument \textit{dest} is the destination file
(without extension).
It should be the main file or one of the child files.
Note that further \textsf{childdoc} directives
such as |\childdocof| and |\childdocforward|
in the indicated file will be processed in this form.
The optional argument \textit{main}
passes on directly to the main file \textit{main}
while pretending to compile the child \textit{dest}.
This form behaves as if \textit{dest}
issues |\childdocof{|\textit{main}|}| right away,
and no further \textsf{childdoc} directives will be processed.

%%%%%%%%%%%%%%%%%%%%%%%%%%%%%%%%%%%%%%%%
\DescribeMacro{\...prefix}
In the alternative form |\childdocforwardprefix|,
%
\begin{center}
\begin{tabular}{l}
|\input{childdoc.def}|\\
|\childdocforwardprefix[|\textit{main}|]{|\textit{prefix}|}{|\textit{dest}|}|
\end{tabular}
\end{center}
%
the destination file is determined by a pattern
depending on the current file:
To make this work, the current file must be called
`{\textit{prefix}\hspace{0.2em}\textit{suffix}}'
with \textit{prefix} matching precisely the argument.
Processing is then passed on to the file
`{\textit{dest}\hspace{0.2em}\textit{suffix}}'.
Surely, the same effect is achieved by
directly specifying the
argument `{\textit{dest}\hspace{0.2em}\textit{suffix}}'
in the first form.
However, that requires to set up a different file
for each child. With the alternative form of the command
all these files can have exactly the same content
which simplifies setting them up and maintaining them.

For example, the following file |draft.tex|
with a compilation flag |\version| as described in \secref{sec:flags}
compiles the main document as a draft:
%
\begin{center}
\begin{tabular}{l}
|\def\version{draft}|\\
|\input{childdoc.def}|\\
|\childdocforward{|\textit{main}|}|
\end{tabular}
\end{center}
%
Likewise, the following files |final|\textit{nn}|.tex|
compile the final version of the child document
|child|\textit{nn}|.tex|:
%
\begin{center}
\begin{tabular}{l}
|\def\version{final}|\\
|\input{childdoc.def}|\\
|\childdocforwardprefix{final}{child}|
\end{tabular}
\end{center}
%

Note that when several versions of a main file and/or of each child file
are to be generated, it may be convenient to set up a |Makefile| or
shell script to automatise the process.

%%%%%%%%%%%%%%%%%%%%%%%%%%%%%%%%%%%%%%%%%%%%%%%%%%%%%%%%%%%%%%%%%%%%%%%%%%%%%%%%
\subsection{Command Line Processing}
\label{sec:commandline}

The effect of redirection files can also be achieved by invoking
the \LaTeX{} compiler with a more elaborate command line.
Most conveniently this should be done as part
of a shell script or a |Makefile|.

When using \textsf{childdoc} in the main file, the following
command lines effectively perform a redirection
(note that depending on the shell being used,
backslashes may have to be doubled: `|\|' $\to$ `|\\|'):
%
\begin{center}
|... -jobname "|\textit{target}|" |\\|"|[\textit{flags}]%
|\input{childdoc.def}\childdocforward[|\textit{main}|]{|\textit{dest}|}"|
\end{center}
%
Here \textit{target} is the name of the output file,
\textit{main} is the name of the main file
and \textit{dest} is the name of the main or child file to be processed
(all filenames without extensions).
The optional argument \textit{main} can be omitted
if \textit{main} matches \textit{dest}.
Optionally, compilation \textit{flags} can be defined via |\def| commands.
This command line makes the \TeX{} engine believe
it is compiling the file \textit{target}
whose content is specified as the latter parameter.
The provided code then forwards the processing to
\textit{main} or \textit{dest} as described in \secref{sec:forward}.

%%%%%%%%%%%%%%%%%%%%%%%%%%%%%%%%%%%%%%%%%%%%%%%%%%%%%%%%%%%%%%%%%%%%%%%%%%%%%%%%
\subsection{Include by Input}
\label{sec:input}

Including child documents by |\include| has some restrictions by design.
Most notably, the content of a child document always occupies
its own set of pages; pages cannot be shared between child documents.
Usually, this behaviour makes perfect sense
because each child document contain an essential part of the document.
However, in some situations it may be desirable to compose
a document from a collection of parts
without having mandatory page breaks between then.
For this case, the package
provides a mechanism to include parts
by |\input| which can also be processed individually.
However, by construction this mechanism
requires manual handling of the content to be output.

%%%%%%%%%%%%%%%%%%%%%%%%%%%%%%%%%%%%%%%%
\DescribeMacro{\ifchilddocmanual}
The main file should be prepared as usual, see \secref{sec:include}.
However, the document body must make a distinction
between processing of an individual part and of the main document, e.g.:
%
\begin{center}
\begin{tabular}{l}
|\ifchilddocmanual|\\
|\input{\childdocname}|\\
|\||else|\\
\textit{document body with }|\input{|\textit{part}|}|\\
|\||fi|
\end{tabular}
\end{center}
%
The conditional |\ifchilddocmanual| is true whenever
a part to be included by |\input| is being compiled,
and the name of the part is stored in |\childdocname|.

%%%%%%%%%%%%%%%%%%%%%%%%%%%%%%%%%%%%%%%%
\DescribeMacro{\childdocby}
Each part to be included by |\input| should start with:
%
\begin{center}
\begin{tabular}{l}
|\input{childdoc.def}|\\
|\childdocby{|\textit{main}|}|\\
\end{tabular}
\end{center}
%
The directive |\childdocby| is similar to |\childdocof|
described in \secref{sec:include},
but the subsequent selection of content must be done manually.
To that end, both |\ifchilddoc| and |\ifchilddocmanual|
will be true upon processing of a part,
and the name of the part is stored in |\childdocname|.
Note that |\jobname| will be set to the filename of the current part
so that each part receives an individual |.aux| file
that does not interfere with the |.aux| file(s) of the main document.
This behaviour can be altered by the alternative form
|\childdocby[*]{|\textit{main}|}| (with a non-empty optional argument)
which uses the |.aux| file of the main document
by setting |\jobname| to \textit{main}.

%%%%%%%%%%%%%%%%%%%%%%%%%%%%%%%%%%%%%%%%%%%%%%%%%%%%%%%%%%%%%%%%%%%%%%%%%%%%%%%%
\subsection{Driver Development}
\label{sec:driver}

The \textsf{childdoc} mechanism can also be use for the development
of definition files such as \LaTeX{} styles or classes.
This case differs from the above setup with multiple parts
included by |\include| in that no |\includeonly| should be invoked.
This can be achieved by starting the include file
(before |\ProvidesPackage|) with:
%
\begin{center}
\begin{tabular}{l}
|\input{childdoc.def}|\\
|\childdocforward{|\textit{main}|}|\\
\end{tabular}
\end{center}
%
or alternatively with:
%
\begin{center}
\begin{tabular}{l}
|\input{childdoc.def}|\\
|\childdocby{|\textit{main}|}|\\
\end{tabular}
\end{center}
%
Both forms have slightly different effects as described above.
The main file is prepared as usual, see \secref{sec:include}.

%%%%%%%%%%%%%%%%%%%%%%%%%%%%%%%%%%%%%%%%%%%%%%%%%%%%%%%%%%%%%%%%%%%%%%%%%%%%%%%%
\subsection{Legacy Detection}
\label{sec:detection}

The directive |\childdocmain| in the main file can detect
whether the complete document or merely a child is to be compiled
even without using the directive |\childdocof|.
This method is deprecated because it is less robust
and there is no compelling reason to use it;
it is merely provided for backward compatibility
and it may be removed in future versions.

If the detection mechanism is to be used,
it is mandatory to correctly specify
the filename of the main file as the argument of |\childdocmain|:
%
\begin{center}
\begin{tabular}{l}
|\input{childdoc.def}|\\
|\childdocmain{|\textit{main}|}|\\
\end{tabular}
\end{center}
%
If |\jobname| does not match the argument \textit{main} of |\childdocmain|,
it is assumed that |\jobname| points to the child file to be compiled.
When using |\childdocmain| with the main file specified as argument,
it suffices to start a child file
with just |\input{|\textit{main}|}|
without loading of the package and using |\childdocof|.
If instead all processing is done
with the appropriate \textsf{childdoc} directives,
the argument of \textit{main} of |\childdocmain| can be empty.

An alternative version of the command line processing described
in \secref{sec:commandline} using the detection mechanism reads:
%
\begin{center}
|... -jobname "|\textit{target}|" "|[\textit{flags}]%
[|\def\jobname{|\textit{dest}|}|]|\input{|\textit{main}|}"|
\end{center}

%%%%%%%%%%%%%%%%%%%%%%%%%%%%%%%%%%%%%%%%%%%%%%%%%%%%%%%%%%%%%%%%%%%%%%%%%%%%%%%%
\subsection{Manual Code}
\label{sec:manual}

In case one cannot be certain whether the definitions file |childdoc.def|
is installed on the target \TeX{} distribution
and one prefers not to ship it,
it is conceivable to paste a few relevant commands into the sources.

To that end, drop all statements |\input{childdoc.def}|
and perform the replacements as outlined below.
Instead of |\childdocmain{|\textit{main}|}| add the following code
to the top of the main file:
%
\begin{center}
\begin{tabular}{l}
|\||ifdefined\childdocname\endinput\||fi\newif\ifchilddoc|\\
|\edef\childdocname{\scantokens\expandafter{\jobname\noexpand}}|\\
|\def\childdocmain{|\textit{main}|}\||ifx\childdocmain\childdocname\||else|\\
|\childdoctrue\includeonly{\childdocname}\let\jobname\childdocmain\||fi|\\
\end{tabular}
\end{center}
%
Instead of |\childdocof{|\textit{main}|}| just include the main file
at the top of each child file:
%
\begin{center}
|\input{|\textit{main}|}|
\end{center}
%
A simple redirection |\childdocforward{|\textit{dest}|}| is achieved by:
%
\begin{center}
|\def\jobname{|\textit{dest}|}\input{\jobname}|
\end{center}
%
The redirection with prefix
|\childdocforwardprefix[|\textit{prefix}|]{|\textit{dest}|}|
is accomplished by:
%
\begin{center}
\begin{tabular}{l}
|{\edef\jobname{\scantokens\expandafter{\jobname\noexpand}}|\\
|\def\redirectjob |\textit{prefix}|#1~~~{\gdef\jobname{|\textit{dest}|#1}}|\\
|\expandafter\redirectjob\jobname~~~}\input{\jobname}|
\end{tabular}
\end{center}

In an alternative approach,
child documents can be compiled by a specific command line
without additional code or specific definitions:
%
\begin{center}
|... -jobname "|\textit{target}|" "|[\textit{flags}]%
|\includeonly{|\textit{dest}|}\input{|\textit{main}|}"|
\end{center}
%

%%%%%%%%%%%%%%%%%%%%%%%%%%%%%%%%%%%%%%%%%%%%%%%%%%%%%%%%%%%%%%%%%%%%%%%%%%%%%%%%
%%%%%%%%%%%%%%%%%%%%%%%%%%%%%%%%%%%%%%%%%%%%%%%%%%%%%%%%%%%%%%%%%%%%%%%%%%%%%%%%
\section{Information}

%%%%%%%%%%%%%%%%%%%%%%%%%%%%%%%%%%%%%%%%%%%%%%%%%%%%%%%%%%%%%%%%%%%%%%%%%%%%%%%%
\subsection{Copyright}

Copyright \copyright{} 2017--2018 Niklas Beisert

This work may be distributed and/or modified under the
conditions of the \LaTeX{} Project Public License, either version 1.3
of this license or (at your option) any later version.
The latest version of this license is in
  \url{http://www.latex-project.org/lppl.txt}
and version 1.3 or later is part of all distributions of \LaTeX{}
version 2005/12/01 or later.

This work has the LPPL maintenance status `maintained'.

The Current Maintainer of this work is Niklas Beisert.

This work consists of the files |README.txt|, |childdoc.ins| and |childdoc.dtx|
as well as the derived files |childdoc.def|, |cdocsamp.tex|
with |cdocsch1.tex|, |cdocsch2.tex|, |cdocspt3.tex|, |cdocspt4.tex|,
|cdocsdrf.tex|, |cdocsfn1.tex|, |cdocsfn2.tex|
as well as |childdoc.pdf|.

%%%%%%%%%%%%%%%%%%%%%%%%%%%%%%%%%%%%%%%%%%%%%%%%%%%%%%%%%%%%%%%%%%%%%%%%%%%%%%%%
\subsection{Files and Installation}

The package consists of the files:
%
\begin{center}
\begin{tabular}{ll}
    |README.txt|   & readme file \\
    |childdoc.ins| & installation file \\
    |childdoc.dtx| & source file \\
    |childdoc.def| & definition file \\
    |cdocsamp.tex| & sample main file \\
    |cdocsch1.tex| & sample include file \\
    |cdocsch2.tex| & sample include file \\
    |cdocspt3.tex| & sample part file \\
    |cdocspt4.tex| & sample part file \\
    |cdocsdrf.tex| & sample redirection file \\
    |cdocsfn1.tex| & sample redirection file \\
    |cdocsfn2.tex| & sample redirection file \\
    |childdoc.pdf| & manual
\end{tabular}
\end{center}
%
The distribution consists of the files
|README.txt|, |childdoc.ins| and |childdoc.dtx|.
%
\begin{itemize}
\item
Run (pdf)\LaTeX{} on |childdoc.dtx|
to compile the manual |childdoc.pdf| (this file).
\item
Run \LaTeX{} on |childdoc.ins| to create the definitions file |childdoc.def|
and the sample |cdocsamp.tex| with include files
|cdocsch1.tex|, |cdocsch2.tex|, |cdocspt3.tex|, |cdocspt4.tex|,
|cdocsdrf.tex|, |cdocsfn1.tex|, |cdocsfn2.tex|.
Then copy the file |childdoc.def| to an appropriate directory of your \LaTeX{}
distribution, e.g.\ \textit{texmf-root}|/tex/latex/childdoc|.
\end{itemize}

%%%%%%%%%%%%%%%%%%%%%%%%%%%%%%%%%%%%%%%%%%%%%%%%%%%%%%%%%%%%%%%%%%%%%%%%%%%%%%%%
\subsection{Related CTAN Packages}

There are several other packages which offer a similar functionality:
%
\begin{itemize}
\item
The packages
\href{http://ctan.org/pkg/docmute}{\textsf{docmute}},
\href{http://ctan.org/pkg/includex}{\textsf{includex}} and
\href{http://ctan.org/pkg/standalone}{\textsf{standalone}}
provide commands to include only the document body of
a child file thus allowing both files to be compiled individually.
\item
The packages \href{http://ctan.org/pkg/subdocs}{\textsf{subdocs}}
and \href{http://ctan.org/pkg/subfiles}{\textsf{subfiles}}
provide structures in which the main and child documents can be
encapsulated and allowing them to be compiled individually.
The inclusion mechanism is different from the conventional |\include|.
\item
The package \href{http://ctan.org/pkg/combine}{\textsf{combine}}
is an elaborate solution to combine several documents into one.
\end{itemize}
%
See also the CTAN topic \href{http://ctan.org/topic/subdocs}{\textsf{subdocs}}
for further related packages.
The present package differs from the above solutions in that
a document structure constructed with the conventional |\include| mechanism
just needs two extra commands at the top of every file
such that all constituent files can be compiled individually.

%%%%%%%%%%%%%%%%%%%%%%%%%%%%%%%%%%%%%%%%%%%%%%%%%%%%%%%%%%%%%%%%%%%%%%%%%%%%%%%%
%\subsection{Feature Suggestions}
%
%The following is a list of features which may be useful for future
%versions of this package:
%%
%\begin{itemize}
%\item
%\ldots
%\end{itemize}

%%%%%%%%%%%%%%%%%%%%%%%%%%%%%%%%%%%%%%%%%%%%%%%%%%%%%%%%%%%%%%%%%%%%%%%%%%%%%%%%
\subsection{Revision History}

%%%%%%%%%%%%%%%%%%%%%%%%%%%%%%%%%%%%%%%%
\paragraph{v2.0:} 2018/12/30

\begin{itemize}
\item
immediate forward processing
\item
added |\childdocby| mechanism
\item
manual restructured
\end{itemize}

%%%%%%%%%%%%%%%%%%%%%%%%%%%%%%%%%%%%%%%%
\paragraph{v1.6:} 2018/01/17

\begin{itemize}
\item
application for development of include files
\item
corrections to manual
\end{itemize}

%%%%%%%%%%%%%%%%%%%%%%%%%%%%%%%%%%%%%%%%
\paragraph{v1.5:} 2017/05/21

\begin{itemize}
\item
more complete structuring introduced
\item
|\childdocof| introduced
\item
|\childdoc| renamed to |\childdocmain|
\item
|\childredirect| renamed to |\childdocforward| and |\childdocforwardprefix|
and functionality expanded
\end{itemize}

%%%%%%%%%%%%%%%%%%%%%%%%%%%%%%%%%%%%%%%%
\paragraph{v1.0:} 2017/04/27

\begin{itemize}
\item
manual and install package
\item
first version published on CTAN
\end{itemize}

%%%%%%%%%%%%%%%%%%%%%%%%%%%%%%%%%%%%%%%%
\paragraph{v0.6:} 2017/04/26

\begin{itemize}
\item
redirection mechanism added
\end{itemize}

%%%%%%%%%%%%%%%%%%%%%%%%%%%%%%%%%%%%%%%%
\paragraph{v0.5:} 2017/04/26

\begin{itemize}
\item
functionality in definition file
\end{itemize}


%%%%%%%%%%%%%%%%%%%%%%%%%%%%%%%%%%%%%%%%%%%%%%%%%%%%%%%%%%%%%%%%%%%%%%%%%%%%%%%%
%%%%%%%%%%%%%%%%%%%%%%%%%%%%%%%%%%%%%%%%%%%%%%%%%%%%%%%%%%%%%%%%%%%%%%%%%%%%%%%%
%%%%%%%%%%%%%%%%%%%%%%%%%%%%%%%%%%%%%%%%%%%%%%%%%%%%%%%%%%%%%%%%%%%%%%%%%%%%%%%%
\appendix

\settowidth\MacroIndent{\rmfamily\scriptsize 000\ }

 \DocInput{childdoc.dtx}

\end{document}
%</driver>
% \fi
%
% %%%%%%%%%%%%%%%%%%%%%%%%%%%%%%%%%%%%%%%%%%%%%%%%%%%%%%%%%%%%%%%%%%%%%%%%%%%%%%
% %%%%%%%%%%%%%%%%%%%%%%%%%%%%%%%%%%%%%%%%%%%%%%%%%%%%%%%%%%%%%%%%%%%%%%%%%%%%%%
% \section{Sample}
%\iffalse
%<*samplemain>
%\fi
%
% The following presents a sample document
% with two chapters, two parts, a title page,
% a compile flag as well as three forwarding files to set the flag.
% It consists of eight |.tex| files:
% \begin{center}
% \begin{tabular}{ll}
% |cdocsamp.tex|&main file\\
% |cdocsch1.tex|&include file for chapter 1\\
% |cdocsch2.tex|&include file for chapter 2\\
% |cdocspt3.tex|&include file for part 3\\
% |cdocspt4.tex|&include file for part 4\\
% |cdocsdrf.tex|&forwarding file for main file in draft mode\\
% |cdocsfi1.tex|&forwarding file for final version of chapter 1\\
% |cdocsfi2.tex|&forwarding file for final version of chapter 2\\
% \end{tabular}
% \end{center}
% Each of the eight files can be compiled directly by the \LaTeX{} compiler.
%
% %%%%%%%%%%%%%%%%%%%%%%%%%%%%%%%%%%%%%%
% \paragraph{Main File.}
%
% The main file is called |cdocsamp.tex|.
%
% Load the \textsf{childdoc} definitions and
% declare the filename for the main document:
%    \begin{macrocode}
\input{childdoc.def}
\childdocmain{}
%    \end{macrocode}

% Optional override for |\version| flag:
%    \begin{macrocode}
%%\ifchilddoc\else\providecommand{\version}{draft}\fi
%    \end{macrocode}

% Define the default values for the |\version| flag
% (|final| for the main file and |draft| for childs):
%    \begin{macrocode}
\ifchilddoc
\providecommand{\version}{draft}
\else
\providecommand{\version}{final}
\fi
%    \end{macrocode}

% Load the standard document class:
%    \begin{macrocode}
\documentclass[12pt]{article}
%    \end{macrocode}

% Start the document body:
%    \begin{macrocode}
\begin{document}
%    \end{macrocode}

% Declare a title page.
% Print title, part of document being processed and version flag:
%    \begin{macrocode}
\addtocounter{page}{-1}
\begin{center}
{\LARGE\bfseries{}childdoc example\par}
\vspace{1cm}
\ifchilddoc
\ifchilddocmanual part\else chapter\fi:
`\childdocname' of `\childdocjob'\par
\else
main document: `\childdocjob'\par
\fi
version: \version\par
\end{center}
\newpage
%    \end{macrocode}

% Manually include selected file,
% otherwise process as usual:
%    \begin{macrocode}
\ifchilddocmanual
\section*{part `\childdocname'}
\input{\childdocname}
\else
%    \end{macrocode}

% Include the two chapters:
%    \begin{macrocode}
\include{cdocsch1}
\include{cdocsch2}
%    \end{macrocode}

% Include the two parts unless only chapters should be displayed:
%    \begin{macrocode}
\ifchilddoc\else
\section{part three}
\input{cdocspt3}
\section{part four}
\input{cdocspt4}
\fi
%    \end{macrocode}

% Process as usual until here:
%    \begin{macrocode}
\fi
%    \end{macrocode}

% End of document body:
%    \begin{macrocode}
\end{document}
%    \end{macrocode}
%\iffalse
%</samplemain>
%\fi
%
% %%%%%%%%%%%%%%%%%%%%%%%%%%%%%%%%%%%%%%
% \paragraph{Chapter Include Files.}
%
% The include files are called |cdocsch1.tex| and |cdocsch2.tex|.
%
%\iffalse
%<*samplechap1|samplechap2>
%\fi

% Optional override for |\version| flag:
%    \begin{macrocode}
%%\providecommand{\version}{final}
%    \end{macrocode}

% Include the main document:
%    \begin{macrocode}
\input{childdoc.def}
\childdocof{cdocsamp}
%    \end{macrocode}

%\iffalse
%</samplechap1|samplechap2>
%\fi
%
%\iffalse
%<*samplechap1>
%\fi
% Some text for chapter 1:
%    \begin{macrocode}
\section{one}
some text in chapter one
%    \end{macrocode}

%\iffalse
%</samplechap1>
%\fi
% Some text for chapter 2:
%\iffalse
%<*samplechap2>
%\fi
%    \begin{macrocode}
\section{two}
more text in chapter two
%    \end{macrocode}

%\iffalse
%</samplechap2>
%\fi
%
% %%%%%%%%%%%%%%%%%%%%%%%%%%%%%%%%%%%%%%
% \paragraph{Part Include Files.}
%
% The include files are called |cdocspt3.tex| and |cdocspt4.tex|.
%
%\iffalse
%<*samplepart3|samplepart4>
%\fi

% Optional override for |\version| flag:
%    \begin{macrocode}
%%\providecommand{\version}{final}
%    \end{macrocode}

% Include the main document:
%    \begin{macrocode}
\input{childdoc.def}
\childdocby{cdocsamp}
%    \end{macrocode}

%\iffalse
%</samplepart3|samplepart4>
%\fi
%
%\iffalse
%<*samplepart3>
%\fi
% Some text for part 3:
%    \begin{macrocode}
some text in part three
%    \end{macrocode}

%\iffalse
%</samplepart3>
%\fi
% Some text for part 4:
%\iffalse
%<*samplepart4>
%\fi
%    \begin{macrocode}
more text in part four
%    \end{macrocode}

%\iffalse
%</samplepart4>
%\fi
%
% %%%%%%%%%%%%%%%%%%%%%%%%%%%%%%%%%%%%%%
% \paragraph{Forwarding for a Complete Draft.}
%
% The following forwarding file |cdocsdrf.tex|
% compiles the main document in draft mode:
%\iffalse
%<*sampledraft>
%\fi
%    \begin{macrocode}
\def\version{draft}
\input{childdoc.def}
\childdocforward{cdocsamp}
%    \end{macrocode}

%\iffalse
%</sampledraft>
%\fi
%
% %%%%%%%%%%%%%%%%%%%%%%%%%%%%%%%%%%%%%%
% \paragraph{Forwarding for Final Version of the Chapters.}
%
% The following forwarding files |cdocsfn1.tex| and |cdocsfn2.tex|
% (with identical content)
% compile the final versions of the child documents
% |cdocsch1.tex| and |cdocsch2.tex|, respectively:
%\iffalse
%<*samplefinal>
%\fi
%    \begin{macrocode}
\def\version{final}
\input{childdoc.def}
\childdocforwardprefix[cdocsamp]{cdocsfn}{cdocsch}
%    \end{macrocode}

%\iffalse
%</samplefinal>
%\fi
%
% %%%%%%%%%%%%%%%%%%%%%%%%%%%%%%%%%%%%%%
% \paragraph{Command Line Processing.}
%
% The following three command lines generate the output files
% |cdocscld|, |cdocscl1| and |cdocscl2|
% which should be identical to
% |cdocsdrf|, |cdocsch1| and |cdocsfn2|, respectively:
% \begin{center}
% \begin{tabular}{l}
% |latex -jobname cdocscld \|\\
% |  "\def\version{draft}\input{childdoc.def}\childdocforward{cdocsamp}"|\\
% |latex -jobname cdocscl1 \|\\
% |  "\input{childdoc.def}\childdocforward[cdocsamp]{cdocsch1}"|\\
% |latex -jobname cdocscl2 \|\\
% |  "\def\version{final}\input{childdoc.def}\childdocforward{cdocsch2}"|
% \end{tabular}
% \end{center}
% Note that the trailing backslash on each first line
% merely continues the input to the second line
% (for convenient cut ant paste).
% Furthermore, the command |latex| can be replaced by any
% of its alternative versions such as |pdflatex|.
%
% %%%%%%%%%%%%%%%%%%%%%%%%%%%%%%%%%%%%%%%%%%%%%%%%%%%%%%%%%%%%%%%%%%%%%%%%%%%%%%
% %%%%%%%%%%%%%%%%%%%%%%%%%%%%%%%%%%%%%%%%%%%%%%%%%%%%%%%%%%%%%%%%%%%%%%%%%%%%%%
% \section{Implementation}
%\iffalse
%<*package>
%\fi
%
% This section describes the definitions file |childdoc.def|.

% The definitions cannot be loaded using |\usepackage| or |\RequirePackage|
% which has a mechanism to prevent loading a style file more than once.
% When loading the definitions by means of |\input|
% multiple instances have to be prevented manually:
%\iffalse
%This code needs to be before the `\ProvidesFile' directive
%which is defined at the beginning of this file.
%Therefore it is also placed there and commented out here.
%</package>
%<*discard>
%\fi
%    \begin{macrocode}
\ifdefined\childdocmain\endinput\fi
%    \end{macrocode}
%\iffalse
%</discard>
%<*package>
%\fi
%
% \macro{\ifchilddoc}
% \macro{\ifchilddocmanual}
% The conditional |\ifchilddoc| tells whether a
% child (true) or main (false) document is being compiled.
% The conditional |\ifchilddocmanual| tells whether
% the |\includeonly| mechanism is used (false) or
% the selection of child files must be performed manually (true).
% The definitions initialise to false:
%    \begin{macrocode}
\newif\ifchilddoc
\newif\ifchilddocmanual
%    \end{macrocode}

% \macro{\childdocname}
% \macro{\childdocjob}
% The macro |\childdocname| stores the name of the main document
% to be compiled. The macro |\childdocjob| stores the name of
% the document on which the \LaTeX{} compiler was originally invoked.
% The content of |\jobname| cannot be compared
% to filenames specified in the source due to different catcodes.
% The following code rescans |\jobname|, stores the result
% in |\childdocname| and saves a copy in |\childdocjob|:
%    \begin{macrocode}
\edef\childdocname{\scantokens\expandafter{\jobname\noexpand}}
\let\childdocjob\childdocname
%    \end{macrocode}

% \macro{\childdocdisable}
% The macro |\childdocdisable| prevents the main file
% from being processed more than once.
% At this stage, the main document command |\childdocmain|
% is assumed to be called once again where it should do nothing.
% Any subsequent call to it should prevent
% a secondary processing of the main document
% It overwrites the forwarding commands
% |\childdocof| and |\childdocforward|
% with empty macros to prevent further inclusions of the main document:
%    \begin{macrocode}
\newcommand{\childdocdisable}
{
  \renewcommand{\childdocmain}[1]{\renewcommand{\childdocmain}[1]{\endinput}}
  \renewcommand{\childdocof}[1]{}
  \renewcommand{\childdocby}[2][]{}
  \renewcommand{\childdocforward}[2][]{}
  \renewcommand{\childdocdisable}{}
}
%    \end{macrocode}

% \macro{\childdocmain}
% The macro |\childdocmain| is to be called at the top of the main file
% with nothing or the main filename (without extension) as argument.
% First, it breaks loops.
% If the argument is not empty and does not match |\childdocname|
% (which is set by the first inclusion of |childdoc.def|),
% |\ifchilddoc| is set to true, |\includeonly| is applied to the child file
% and |\jobname| is set to the main file
% (for proper handling of |.aux| files):
%    \begin{macrocode}
\newcommand{\childdocmain}[1]
{
  \childdocdisable\childdocmain{}
  \if?#1?\else
    \begingroup
      \def\childdoctmp{#1}
      \ifx\childdoctmp\childdocname
        \def\childdoctmp{}
      \else
        \def\childdoctmp
        {
          \childdoctrue
          \includeonly{\childdocname}
          \def\childdocjob{#1}
          \def\jobname{#1}
        }
      \fi
      \expandafter
    \endgroup
    \childdoctmp
  \fi
}
%    \end{macrocode}

% \macro{\childdocof}
% The command |\childdocof| redirects
% compilation to the main file |#1|.
%    \begin{macrocode}
\newcommand{\childdocof}[1]
{
  \childdocdisable
  \childdoctrue
  \includeonly{\childdocname}
  \def\jobname{#1}
  \def\childdocjob{#1}
  \input{#1}
}
%    \end{macrocode}

% \macro{\childdocby}
% The command |\childdocby| ....
%    \begin{macrocode}
\newcommand{\childdocby}[2][]
{
  \childdocdisable
  \childdoctrue
  \childdocmanualtrue
  \if?#1?\else
    \def\jobname{#2}
  \fi
  \def\childdocjob{#2}
  \input{#2}
  \endinput
}
%    \end{macrocode}

% \macro{\childdocforward}
% The command |\childdocforward| redirects
% compilation to the main file or
% (if the optional argument is given) a child file.
% Parameters are set as if the main file
% or a child file starting with |\childdocof| was compiled.
% Then compilation is handed over to the main file:
%    \begin{macrocode}
\newcommand{\childdocforward}[2][]
{
  \begingroup
    \if?#1?
      \def\childdoctmp
      {
        \def\childdocname{#2}
        \def\childdocjob{#2}
        \def\jobname{#2}
        \input{#2}
        \endinput
      }
    \else
      \def\childdoctmp
      {
        \childdocdisable
        \def\childdocname{#2}
        \childdoctrue
        \includeonly{#2}
        \def\childdocjob{#1}
        \def\jobname{#1}
        \input{#1}
        \endinput
      }
    \fi
    \expandafter
  \endgroup
  \childdoctmp
}
%    \end{macrocode}

% \macro{\childdocforwardprefix}
% The command |\childdocforwardprefix| redirects
% compilation to the main or a child file by means of a pattern.
% The prefix |#1| in the current filename is replaced by |#2|
% and the suffix of the current filename is kept
% (it is assumed that the filename does not contain the substring `|~~~|'
% which is used as a delimiter).
% Compilation is handed over to the new file by |\childdocforward|:
%    \begin{macrocode}
\newcommand{\childdocforwardprefix}[3][]
{
  \begingroup
    \def\childdocextract #2##1~~~{\def\childdoctmp{\childdocforward[#1]{#3##1}}}
    \expandafter\childdocextract\childdocname~~~
    \expandafter
  \endgroup
  \childdoctmp
}
%    \end{macrocode}

% \macro{\childdoc}
% The deprecated macro |\childdoc| is a legacy version of |\childdocmain|:
%    \begin{macrocode}
\newcommand{\childdoc}{\childdocmain}
%    \end{macrocode}

% \macro{\childdocredirect}
% The deprecated macro |\childdocredirect| is a legacy version
% of |\childdocforward| and |\childdocforwardprefix|:
%    \begin{macrocode}
\newcommand{\childdocredirect}[2][]
{
  \begingroup
    \if?#1?
      \def\childdoctmp{\childdocforward{#2}}
    \else
      \def\childdoctmp{\childdocforwardprefix{#1}{#2}}
    \fi
    \expandafter
  \endgroup
  \childdoctmp
}
%    \end{macrocode}

%\iffalse
%</package>
%\fi
%
\endinput
|\\
|\childdocforward[|\textit{main}|]{|\textit{dest}|}|\\
\end{tabular}
\end{center}
%
The argument \textit{dest} is the destination file
(without extension).
It should be the main file or one of the child files.
Note that further \textsf{childdoc} directives
such as |\childdocof| and |\childdocforward|
in the indicated file will be processed in this form.
The optional argument \textit{main}
passes on directly to the main file \textit{main}
while pretending to compile the child \textit{dest}.
This form behaves as if \textit{dest}
issues |\childdocof{|\textit{main}|}| right away,
and no further \textsf{childdoc} directives will be processed.

%%%%%%%%%%%%%%%%%%%%%%%%%%%%%%%%%%%%%%%%
\DescribeMacro{\...prefix}
In the alternative form |\childdocforwardprefix|,
%
\begin{center}
\begin{tabular}{l}
|% \iffalse
%
% childdoc.dtx Copyright (C) 2017-2018 Niklas Beisert
%
% This work may be distributed and/or modified under the
% conditions of the LaTeX Project Public License, either version 1.3
% of this license or (at your option) any later version.
% The latest version of this license is in
%   http://www.latex-project.org/lppl.txt
% and version 1.3 or later is part of all distributions of LaTeX
% version 2005/12/01 or later.
%
% This work has the LPPL maintenance status `maintained'.
%
% The Current Maintainer of this work is Niklas Beisert.
%
% This work consists of the files childdoc.dtx and childdoc.ins
% and the derived files childdoc.def and cdocsamp.tex with
% cdocsch1.tex, cdocsch2.tex, cdocsdrf.tex, cdocsfn1.tex, cdocsfn2.tex.
%
%<package>\ifdefined\childdocmain\endinput\fi
%<package>\ProvidesFile{childdoc.def}[2018/12/30 v2.0 child document driver]
%<samplemain>\ProvidesFile{cdocsamp.tex}[2018/12/30 v2.0 sample for childdoc]
%<*driver>
%\ProvidesFile{childdoc.drv}[2018/12/30 v2.0 childdoc reference manual file]
\PassOptionsToClass{10pt,a4paper}{article}
\documentclass{ltxdoc}

\usepackage[margin=35mm]{geometry}
\usepackage{hyperref}
\usepackage{hyperxmp}
\usepackage[usenames]{color}

\hypersetup{colorlinks=true}
\hypersetup{pdfstartview=FitH}
\hypersetup{pdfpagemode=UseNone}
\hypersetup{pdfsource={}}
\hypersetup{pdflang={en-UK}}
\hypersetup{pdfcopyright={Copyright 2017-2018 Niklas Beisert.
  This work may be distributed and/or modified under the
  conditions of the LaTeX Project Public License, either version 1.3
  of this license or (at your option) any later version.}}
\hypersetup{pdflicenseurl={http://www.latex-project.org/lppl.txt}}
\hypersetup{pdfcontactaddress={ETH Zurich, ITP, HIT K,
  Wolfgang-Pauli-Strasse 27}}
\hypersetup{pdfcontactpostcode={8093}}
\hypersetup{pdfcontactcity={Zurich}}
\hypersetup{pdfcontactcountry={Switzerland}}
\hypersetup{pdfcontactemail={nbeisert@itp.phys.ethz.ch}}
\hypersetup{pdfcontacturl={http://people.phys.ethz.ch/\xmptilde nbeisert/}}

\newcommand{\secref}[1]{\hyperref[#1]{section \ref*{#1}}}

\parskip1ex
\parindent0pt
\let\olditemize\itemize
\def\itemize{\olditemize\parskip0pt}

\begin{document}

\title{The \textsf{childdoc} Package}
\hypersetup{pdftitle={The childdoc Package}}
\author{Niklas Beisert\\[2ex]
  Institut f\"ur Theoretische Physik\\
  Eidgen\"ossische Technische Hochschule Z\"urich\\
  Wolfgang-Pauli-Strasse 27, 8093 Z\"urich, Switzerland\\[1ex]
  \href{mailto:nbeisert@itp.phys.ethz.ch}
  {\texttt{nbeisert@itp.phys.ethz.ch}}}
\hypersetup{pdfauthor={Niklas Beisert}}
\hypersetup{pdfsubject={Manual for the LaTeX2e Package childdoc}}
\date{30 December 2018, \textsf{v2.0}}
\maketitle

\begin{abstract}\noindent
\textsf{childdoc} is a \LaTeXe{} package
that enables the direct compilation
of document sections included by |\include|
to individual files.
\end{abstract}

\begingroup
\parskip0ex
\tableofcontents
\endgroup

%%%%%%%%%%%%%%%%%%%%%%%%%%%%%%%%%%%%%%%%%%%%%%%%%%%%%%%%%%%%%%%%%%%%%%%%%%%%%%%%
%%%%%%%%%%%%%%%%%%%%%%%%%%%%%%%%%%%%%%%%%%%%%%%%%%%%%%%%%%%%%%%%%%%%%%%%%%%%%%%%
\section{Introduction}

\LaTeX{} provides a mechanism to structure a large document (such as a book)
into a main file and several child files (containing the chapters)
using the |\include| command.
This mechanism is beneficial for documents
which span hundreds of pages in order to
make the source file(s) more manageable.
Moreover, compilation can be restricted to
selected child files by means of the |\includeonly| command.
The latter feature can be used to reduce the compilation time while editing
(this was significantly more useful in the earlier days of \LaTeX{})
or to generate a smaller document which is easier to navigate.
Another application of |\includeonly| is to generate
documents consisting of selected parts of the complete document.

However, there are a few drawbacks of the plain |\include| mechanism:
\begin{itemize}
\item
The child files cannot be compiled on their own,
they can only be compiled via the main file.
A naive editing environment
(such as a text editor with an option
to have the current file processed by \LaTeX)
may require one to switch to the main file before compiling;
attempting to compile the child file produces errors.
\item
The main file must be modified (each time)
to adjust the |\includeonly| command
to the present needs. This easily leaves the main file in a messy state.
\item
The generated document will always carry the filename
of the main document. This is inconvenient if
several child files are to be compiled and
to be kept for distribution.
\end{itemize}

The present package provides a simple interface
to make child files individually compilable by \LaTeX{}.
Compiling a child file then has the same effect as compiling
the main file with an |\includeonly| command
to select the appropriate child.
Moreover the generated document will carry the name of the child
rather than the main file.
This resolves all three above issues.

This feature is meant to make the editing of books,
thesis documents and lecture notes somewhat more convenient.
However, the package can also be used efficiently for
composing a series of documents (such as exercise sheets)
which are typically distributed individually.
It then assists the author in generating the individual documents
(potentially in different versions)
as well as a document containing the collected series.
Another application is in developing style files
or other kinds of included material
where compilation of the style file could redirect
to a sample or test file.

%%%%%%%%%%%%%%%%%%%%%%%%%%%%%%%%%%%%%%%%%%%%%%%%%%%%%%%%%%%%%%%%%%%%%%%%%%%%%%%%
%%%%%%%%%%%%%%%%%%%%%%%%%%%%%%%%%%%%%%%%%%%%%%%%%%%%%%%%%%%%%%%%%%%%%%%%%%%%%%%%
\section{Usage}

First of all, the package \textsf{childdoc} is \emph{not} a standard
\LaTeXe{} |.sty| style file! Therefore it needs to be invoked in
a non-standard way.

%%%%%%%%%%%%%%%%%%%%%%%%%%%%%%%%%%%%%%%%%%%%%%%%%%%%%%%%%%%%%%%%%%%%%%%%%%%%%%%%
\subsection{Included Files}
\label{sec:include}

%%%%%%%%%%%%%%%%%%%%%%%%%%%%%%%%%%%%%%%%
\DescribeMacro{\childdocmain}
To use the package, add the commands
\begin{center}
\begin{tabular}{l}
|\input{childdoc.def}|\\
|\childdocmain{}|\\
\end{tabular}
\end{center}
at the very top of the main \LaTeX{} file,
in particular \emph{before} the |\documentclass| statement!
The argument of |\childdocmain| should be left empty
(but it must be present).

%%%%%%%%%%%%%%%%%%%%%%%%%%%%%%%%%%%%%%%%
\DescribeMacro{\childdocof}
Furthermore, add the commands
\begin{center}
\begin{tabular}{l}
|\input{childdoc.def}|\\
|\childdocof{|\textit{main}|}|\\
\end{tabular}
\end{center}
at the top of every child file \textit{child}
which is included by |\include{|\textit{child}|}|
from within the main file
(or at least for those files to be compiled individually).
The argument \textit{main} must be the filename of the main file.

There are a couple of
considerations in setting up the main and child documents:

%%%%%%%%%%%%%%%%%%%%%%%%%%%%%%%%%%%%%%%%
\paragraph{Restrictions.}

Please note the following restrictions:
\begin{itemize}
\item
|\childdocmain| must be called with one argument \textit{main}
to ensure compatibility with earlier version of the package.
It must either be empty (|\childdocmain{}|)
or precisely match the filename of the main file in which it is specified.
See \secref{sec:detection} for further information.
\item
The filename \textit{main} must be specified without the |.tex| extension.
\item
The filename \textit{main} is case sensitive
(even in case-insensitive file systems)
due to internal string comparison.
\item
The argument \textit{main} should be fully expanded, it cannot be a macro.
\item
Subdirectories and special characters should be avoided in filenames.
\item
The command |\childdocmain{|\textit{main}|}| must be followed by a whitespace.
It should not be followed immediately by another command
or by a comment mark `|%|'.
This is because the \TeX{} parser reads the token immediately following
the argument of |\childdocmain| and puts it
at the beginning of every child section;
however, a white\-space is ignored.
\end{itemize}

%%%%%%%%%%%%%%%%%%%%%%%%%%%%%%%%%%%%%%%%
\paragraph{Content of Main File.}

It is advisable to place all content in the child files included by |\include|.
Any output contained in the main file will appear in all child documents
unless suppressed manually;
it cannot be suppressed automatically by the |\includeonly| directive
and thus should normally be avoided.
A method to include some content in the main file
by means of conditional processing is described in \secref{sec:conditional}.

%%%%%%%%%%%%%%%%%%%%%%%%%%%%%%%%%%%%%%%%
\paragraph{Page Numbering.}

When only a part of the document is compiled,
the appropriate numbering of pages
(as well as other status parameters)
is determined from the |.aux| files.
The latter contain information from previous passes.
However this information needs to propagate through
all intermediate child documents.
Therefore the page numbering in child documents may well
be inconsistent until the complete document is compiled at least once.

A useful (if unconventional) way to always ensure a consistent
page numbering is to restart the numbering in each child document
and denote the pages by `\textit{child}|.|\textit{page}'
where \textit{child} represents the chapter/section number of the child file.
This can be achieved by the command
|\numberwithin{page}{|\textit{child}|}|
of the \textsf{amsmath} package
where \textit{child} can be |chapter| or |section|
depending on the chosen structuring.
Alternatively, one can modify the macro |\thepage| appropriately
and reset the counter |page| at the start of each child file.

%%%%%%%%%%%%%%%%%%%%%%%%%%%%%%%%%%%%%%%%%%%%%%%%%%%%%%%%%%%%%%%%%%%%%%%%%%%%%%%%
\subsection{Conditional Processing}
\label{sec:conditional}

The package provides a mechanism to compile different versions
of a document. To customise the versions further some conditional processing
can come in handy to distinguish which version is being compiled.
The package provides two macros to describe the compilation context:

%%%%%%%%%%%%%%%%%%%%%%%%%%%%%%%%%%%%%%%%
\DescribeMacro{\ifchilddoc}
The conditional |\ifchilddoc| distinguishes between the compilation of
child documents and the main document:
%
\begin{center}
|\ifchilddoc |\textit{child-code}| |[|\||else |\textit{main-code}]| \||fi|
\end{center}

%%%%%%%%%%%%%%%%%%%%%%%%%%%%%%%%%%%%%%%%
\DescribeMacro{\childdocname}
\DescribeMacro{\childdocjob}
The macro |\childdocname| contains the filename (without extension)
of the main or child file being processed.
Note that |\childdocjob| will always contain the name of the main file.

%%%%%%%%%%%%%%%%%%%%%%%%%%%%%%%%%%%%%%%%
\paragraph{Title Page.}

Conditional processing can be used to include a title or banner page
in the main document when proper precautions are taken.
Importantly, the code in the main file should ensure that the page counter
(as well as other status parameters which are stored in the |.aux| files)
takes the same value after the conditional processing.
Otherwise the page numbers may take divergent values
depending on which part is compiled.

For example, a title page could be declared by:
%
\begin{center}
\begin{tabular}{l}
|\ifchilddoc\||else|\\
|\addtocounter{page}{-1}|\\
\textit{code for title page}\\
|\newpage|\\
|\||fi|
\end{tabular}
\end{center}
%
A banner page for the child documents can be generated by:
%
\begin{center}
\begin{tabular}{l}
|\ifchilddoc|\\
|\addtocounter{page}{-1}|\\
\textit{code for banner page}\\
|\newpage|\\
|\||fi|
\end{tabular}
\end{center}
%
Here one could write a message such as:
\begin{center}
|This is the part \childdocname{} of \childdocjob{}.|
\end{center}

%%%%%%%%%%%%%%%%%%%%%%%%%%%%%%%%%%%%%%%%%%%%%%%%%%%%%%%%%%%%%%%%%%%%%%%%%%%%%%%%
\subsection{Flags}
\label{sec:flags}

The package makes it easy to generate different versions
of the main or child documents.
To this end compilation flags can be defined
and assigned different default values.
They will be particularly useful in conjunction
with the forwarding mechanism described in \secref{sec:forward}.

For example, it may be useful to have a flag |\version|
which can be set to |draft| or |final|.
The document source will contain some conditional code
depending on the value of |\version|.
Suppose further, the flag should default to |final| for the main file
and to |draft| for child files
which is a natural assignment for editing the document.
This is achieved by placing the following code
in the preamble of the main document
(below the |\childdocmain| directive):
%
\begin{center}
\begin{tabular}{l}
|\ifchilddoc|\\
|\providecommand{\version}{draft}|\\
|\||else|\\
|\providecommand{\version}{final}|\\
|\||fi|
\end{tabular}
\end{center}
%
The definition by |\providecommand| makes sure
that previous definitions are not overwritten.
Further statements |\providecommand{\version}{...}|
can thus be added before the above code to override it.

For the main file, one might add a line
(between |\childdocmain| and the above block)
%
\begin{center}
|%\ifchilddoc\||else\providecommand{\version}{draft}\||fi|
\end{center}
%
which can be uncommented to produce a draft version.
Likewise one can add a line to the very top of a child file
(above the |\childdocof{|\textit{main}|}| directive)
%
\begin{center}
|%\providecommand{\version}{final}|
\end{center}
%
which can be uncommented to produce the final version of this child document.

%%%%%%%%%%%%%%%%%%%%%%%%%%%%%%%%%%%%%%%%%%%%%%%%%%%%%%%%%%%%%%%%%%%%%%%%%%%%%%%%
\subsection{Forwarding}
\label{sec:forward}

Different versions of the main or child documents
using compilation flags as described in \secref{sec:flags}
can be (permanently) stored in different files
for convenient compilation, viewing and distribution.
To this end, the package defines a command
to pass on compilation to a different file:

%%%%%%%%%%%%%%%%%%%%%%%%%%%%%%%%%%%%%%%%
\DescribeMacro{\childdocforward}
The command |\childdocforward| redirects processing to
another source file:
%
\begin{center}
\begin{tabular}{l}
|\input{childdoc.def}|\\
|\childdocforward[|\textit{main}|]{|\textit{dest}|}|\\
\end{tabular}
\end{center}
%
The argument \textit{dest} is the destination file
(without extension).
It should be the main file or one of the child files.
Note that further \textsf{childdoc} directives
such as |\childdocof| and |\childdocforward|
in the indicated file will be processed in this form.
The optional argument \textit{main}
passes on directly to the main file \textit{main}
while pretending to compile the child \textit{dest}.
This form behaves as if \textit{dest}
issues |\childdocof{|\textit{main}|}| right away,
and no further \textsf{childdoc} directives will be processed.

%%%%%%%%%%%%%%%%%%%%%%%%%%%%%%%%%%%%%%%%
\DescribeMacro{\...prefix}
In the alternative form |\childdocforwardprefix|,
%
\begin{center}
\begin{tabular}{l}
|\input{childdoc.def}|\\
|\childdocforwardprefix[|\textit{main}|]{|\textit{prefix}|}{|\textit{dest}|}|
\end{tabular}
\end{center}
%
the destination file is determined by a pattern
depending on the current file:
To make this work, the current file must be called
`{\textit{prefix}\hspace{0.2em}\textit{suffix}}'
with \textit{prefix} matching precisely the argument.
Processing is then passed on to the file
`{\textit{dest}\hspace{0.2em}\textit{suffix}}'.
Surely, the same effect is achieved by
directly specifying the
argument `{\textit{dest}\hspace{0.2em}\textit{suffix}}'
in the first form.
However, that requires to set up a different file
for each child. With the alternative form of the command
all these files can have exactly the same content
which simplifies setting them up and maintaining them.

For example, the following file |draft.tex|
with a compilation flag |\version| as described in \secref{sec:flags}
compiles the main document as a draft:
%
\begin{center}
\begin{tabular}{l}
|\def\version{draft}|\\
|\input{childdoc.def}|\\
|\childdocforward{|\textit{main}|}|
\end{tabular}
\end{center}
%
Likewise, the following files |final|\textit{nn}|.tex|
compile the final version of the child document
|child|\textit{nn}|.tex|:
%
\begin{center}
\begin{tabular}{l}
|\def\version{final}|\\
|\input{childdoc.def}|\\
|\childdocforwardprefix{final}{child}|
\end{tabular}
\end{center}
%

Note that when several versions of a main file and/or of each child file
are to be generated, it may be convenient to set up a |Makefile| or
shell script to automatise the process.

%%%%%%%%%%%%%%%%%%%%%%%%%%%%%%%%%%%%%%%%%%%%%%%%%%%%%%%%%%%%%%%%%%%%%%%%%%%%%%%%
\subsection{Command Line Processing}
\label{sec:commandline}

The effect of redirection files can also be achieved by invoking
the \LaTeX{} compiler with a more elaborate command line.
Most conveniently this should be done as part
of a shell script or a |Makefile|.

When using \textsf{childdoc} in the main file, the following
command lines effectively perform a redirection
(note that depending on the shell being used,
backslashes may have to be doubled: `|\|' $\to$ `|\\|'):
%
\begin{center}
|... -jobname "|\textit{target}|" |\\|"|[\textit{flags}]%
|\input{childdoc.def}\childdocforward[|\textit{main}|]{|\textit{dest}|}"|
\end{center}
%
Here \textit{target} is the name of the output file,
\textit{main} is the name of the main file
and \textit{dest} is the name of the main or child file to be processed
(all filenames without extensions).
The optional argument \textit{main} can be omitted
if \textit{main} matches \textit{dest}.
Optionally, compilation \textit{flags} can be defined via |\def| commands.
This command line makes the \TeX{} engine believe
it is compiling the file \textit{target}
whose content is specified as the latter parameter.
The provided code then forwards the processing to
\textit{main} or \textit{dest} as described in \secref{sec:forward}.

%%%%%%%%%%%%%%%%%%%%%%%%%%%%%%%%%%%%%%%%%%%%%%%%%%%%%%%%%%%%%%%%%%%%%%%%%%%%%%%%
\subsection{Include by Input}
\label{sec:input}

Including child documents by |\include| has some restrictions by design.
Most notably, the content of a child document always occupies
its own set of pages; pages cannot be shared between child documents.
Usually, this behaviour makes perfect sense
because each child document contain an essential part of the document.
However, in some situations it may be desirable to compose
a document from a collection of parts
without having mandatory page breaks between then.
For this case, the package
provides a mechanism to include parts
by |\input| which can also be processed individually.
However, by construction this mechanism
requires manual handling of the content to be output.

%%%%%%%%%%%%%%%%%%%%%%%%%%%%%%%%%%%%%%%%
\DescribeMacro{\ifchilddocmanual}
The main file should be prepared as usual, see \secref{sec:include}.
However, the document body must make a distinction
between processing of an individual part and of the main document, e.g.:
%
\begin{center}
\begin{tabular}{l}
|\ifchilddocmanual|\\
|\input{\childdocname}|\\
|\||else|\\
\textit{document body with }|\input{|\textit{part}|}|\\
|\||fi|
\end{tabular}
\end{center}
%
The conditional |\ifchilddocmanual| is true whenever
a part to be included by |\input| is being compiled,
and the name of the part is stored in |\childdocname|.

%%%%%%%%%%%%%%%%%%%%%%%%%%%%%%%%%%%%%%%%
\DescribeMacro{\childdocby}
Each part to be included by |\input| should start with:
%
\begin{center}
\begin{tabular}{l}
|\input{childdoc.def}|\\
|\childdocby{|\textit{main}|}|\\
\end{tabular}
\end{center}
%
The directive |\childdocby| is similar to |\childdocof|
described in \secref{sec:include},
but the subsequent selection of content must be done manually.
To that end, both |\ifchilddoc| and |\ifchilddocmanual|
will be true upon processing of a part,
and the name of the part is stored in |\childdocname|.
Note that |\jobname| will be set to the filename of the current part
so that each part receives an individual |.aux| file
that does not interfere with the |.aux| file(s) of the main document.
This behaviour can be altered by the alternative form
|\childdocby[*]{|\textit{main}|}| (with a non-empty optional argument)
which uses the |.aux| file of the main document
by setting |\jobname| to \textit{main}.

%%%%%%%%%%%%%%%%%%%%%%%%%%%%%%%%%%%%%%%%%%%%%%%%%%%%%%%%%%%%%%%%%%%%%%%%%%%%%%%%
\subsection{Driver Development}
\label{sec:driver}

The \textsf{childdoc} mechanism can also be use for the development
of definition files such as \LaTeX{} styles or classes.
This case differs from the above setup with multiple parts
included by |\include| in that no |\includeonly| should be invoked.
This can be achieved by starting the include file
(before |\ProvidesPackage|) with:
%
\begin{center}
\begin{tabular}{l}
|\input{childdoc.def}|\\
|\childdocforward{|\textit{main}|}|\\
\end{tabular}
\end{center}
%
or alternatively with:
%
\begin{center}
\begin{tabular}{l}
|\input{childdoc.def}|\\
|\childdocby{|\textit{main}|}|\\
\end{tabular}
\end{center}
%
Both forms have slightly different effects as described above.
The main file is prepared as usual, see \secref{sec:include}.

%%%%%%%%%%%%%%%%%%%%%%%%%%%%%%%%%%%%%%%%%%%%%%%%%%%%%%%%%%%%%%%%%%%%%%%%%%%%%%%%
\subsection{Legacy Detection}
\label{sec:detection}

The directive |\childdocmain| in the main file can detect
whether the complete document or merely a child is to be compiled
even without using the directive |\childdocof|.
This method is deprecated because it is less robust
and there is no compelling reason to use it;
it is merely provided for backward compatibility
and it may be removed in future versions.

If the detection mechanism is to be used,
it is mandatory to correctly specify
the filename of the main file as the argument of |\childdocmain|:
%
\begin{center}
\begin{tabular}{l}
|\input{childdoc.def}|\\
|\childdocmain{|\textit{main}|}|\\
\end{tabular}
\end{center}
%
If |\jobname| does not match the argument \textit{main} of |\childdocmain|,
it is assumed that |\jobname| points to the child file to be compiled.
When using |\childdocmain| with the main file specified as argument,
it suffices to start a child file
with just |\input{|\textit{main}|}|
without loading of the package and using |\childdocof|.
If instead all processing is done
with the appropriate \textsf{childdoc} directives,
the argument of \textit{main} of |\childdocmain| can be empty.

An alternative version of the command line processing described
in \secref{sec:commandline} using the detection mechanism reads:
%
\begin{center}
|... -jobname "|\textit{target}|" "|[\textit{flags}]%
[|\def\jobname{|\textit{dest}|}|]|\input{|\textit{main}|}"|
\end{center}

%%%%%%%%%%%%%%%%%%%%%%%%%%%%%%%%%%%%%%%%%%%%%%%%%%%%%%%%%%%%%%%%%%%%%%%%%%%%%%%%
\subsection{Manual Code}
\label{sec:manual}

In case one cannot be certain whether the definitions file |childdoc.def|
is installed on the target \TeX{} distribution
and one prefers not to ship it,
it is conceivable to paste a few relevant commands into the sources.

To that end, drop all statements |\input{childdoc.def}|
and perform the replacements as outlined below.
Instead of |\childdocmain{|\textit{main}|}| add the following code
to the top of the main file:
%
\begin{center}
\begin{tabular}{l}
|\||ifdefined\childdocname\endinput\||fi\newif\ifchilddoc|\\
|\edef\childdocname{\scantokens\expandafter{\jobname\noexpand}}|\\
|\def\childdocmain{|\textit{main}|}\||ifx\childdocmain\childdocname\||else|\\
|\childdoctrue\includeonly{\childdocname}\let\jobname\childdocmain\||fi|\\
\end{tabular}
\end{center}
%
Instead of |\childdocof{|\textit{main}|}| just include the main file
at the top of each child file:
%
\begin{center}
|\input{|\textit{main}|}|
\end{center}
%
A simple redirection |\childdocforward{|\textit{dest}|}| is achieved by:
%
\begin{center}
|\def\jobname{|\textit{dest}|}\input{\jobname}|
\end{center}
%
The redirection with prefix
|\childdocforwardprefix[|\textit{prefix}|]{|\textit{dest}|}|
is accomplished by:
%
\begin{center}
\begin{tabular}{l}
|{\edef\jobname{\scantokens\expandafter{\jobname\noexpand}}|\\
|\def\redirectjob |\textit{prefix}|#1~~~{\gdef\jobname{|\textit{dest}|#1}}|\\
|\expandafter\redirectjob\jobname~~~}\input{\jobname}|
\end{tabular}
\end{center}

In an alternative approach,
child documents can be compiled by a specific command line
without additional code or specific definitions:
%
\begin{center}
|... -jobname "|\textit{target}|" "|[\textit{flags}]%
|\includeonly{|\textit{dest}|}\input{|\textit{main}|}"|
\end{center}
%

%%%%%%%%%%%%%%%%%%%%%%%%%%%%%%%%%%%%%%%%%%%%%%%%%%%%%%%%%%%%%%%%%%%%%%%%%%%%%%%%
%%%%%%%%%%%%%%%%%%%%%%%%%%%%%%%%%%%%%%%%%%%%%%%%%%%%%%%%%%%%%%%%%%%%%%%%%%%%%%%%
\section{Information}

%%%%%%%%%%%%%%%%%%%%%%%%%%%%%%%%%%%%%%%%%%%%%%%%%%%%%%%%%%%%%%%%%%%%%%%%%%%%%%%%
\subsection{Copyright}

Copyright \copyright{} 2017--2018 Niklas Beisert

This work may be distributed and/or modified under the
conditions of the \LaTeX{} Project Public License, either version 1.3
of this license or (at your option) any later version.
The latest version of this license is in
  \url{http://www.latex-project.org/lppl.txt}
and version 1.3 or later is part of all distributions of \LaTeX{}
version 2005/12/01 or later.

This work has the LPPL maintenance status `maintained'.

The Current Maintainer of this work is Niklas Beisert.

This work consists of the files |README.txt|, |childdoc.ins| and |childdoc.dtx|
as well as the derived files |childdoc.def|, |cdocsamp.tex|
with |cdocsch1.tex|, |cdocsch2.tex|, |cdocspt3.tex|, |cdocspt4.tex|,
|cdocsdrf.tex|, |cdocsfn1.tex|, |cdocsfn2.tex|
as well as |childdoc.pdf|.

%%%%%%%%%%%%%%%%%%%%%%%%%%%%%%%%%%%%%%%%%%%%%%%%%%%%%%%%%%%%%%%%%%%%%%%%%%%%%%%%
\subsection{Files and Installation}

The package consists of the files:
%
\begin{center}
\begin{tabular}{ll}
    |README.txt|   & readme file \\
    |childdoc.ins| & installation file \\
    |childdoc.dtx| & source file \\
    |childdoc.def| & definition file \\
    |cdocsamp.tex| & sample main file \\
    |cdocsch1.tex| & sample include file \\
    |cdocsch2.tex| & sample include file \\
    |cdocspt3.tex| & sample part file \\
    |cdocspt4.tex| & sample part file \\
    |cdocsdrf.tex| & sample redirection file \\
    |cdocsfn1.tex| & sample redirection file \\
    |cdocsfn2.tex| & sample redirection file \\
    |childdoc.pdf| & manual
\end{tabular}
\end{center}
%
The distribution consists of the files
|README.txt|, |childdoc.ins| and |childdoc.dtx|.
%
\begin{itemize}
\item
Run (pdf)\LaTeX{} on |childdoc.dtx|
to compile the manual |childdoc.pdf| (this file).
\item
Run \LaTeX{} on |childdoc.ins| to create the definitions file |childdoc.def|
and the sample |cdocsamp.tex| with include files
|cdocsch1.tex|, |cdocsch2.tex|, |cdocspt3.tex|, |cdocspt4.tex|,
|cdocsdrf.tex|, |cdocsfn1.tex|, |cdocsfn2.tex|.
Then copy the file |childdoc.def| to an appropriate directory of your \LaTeX{}
distribution, e.g.\ \textit{texmf-root}|/tex/latex/childdoc|.
\end{itemize}

%%%%%%%%%%%%%%%%%%%%%%%%%%%%%%%%%%%%%%%%%%%%%%%%%%%%%%%%%%%%%%%%%%%%%%%%%%%%%%%%
\subsection{Related CTAN Packages}

There are several other packages which offer a similar functionality:
%
\begin{itemize}
\item
The packages
\href{http://ctan.org/pkg/docmute}{\textsf{docmute}},
\href{http://ctan.org/pkg/includex}{\textsf{includex}} and
\href{http://ctan.org/pkg/standalone}{\textsf{standalone}}
provide commands to include only the document body of
a child file thus allowing both files to be compiled individually.
\item
The packages \href{http://ctan.org/pkg/subdocs}{\textsf{subdocs}}
and \href{http://ctan.org/pkg/subfiles}{\textsf{subfiles}}
provide structures in which the main and child documents can be
encapsulated and allowing them to be compiled individually.
The inclusion mechanism is different from the conventional |\include|.
\item
The package \href{http://ctan.org/pkg/combine}{\textsf{combine}}
is an elaborate solution to combine several documents into one.
\end{itemize}
%
See also the CTAN topic \href{http://ctan.org/topic/subdocs}{\textsf{subdocs}}
for further related packages.
The present package differs from the above solutions in that
a document structure constructed with the conventional |\include| mechanism
just needs two extra commands at the top of every file
such that all constituent files can be compiled individually.

%%%%%%%%%%%%%%%%%%%%%%%%%%%%%%%%%%%%%%%%%%%%%%%%%%%%%%%%%%%%%%%%%%%%%%%%%%%%%%%%
%\subsection{Feature Suggestions}
%
%The following is a list of features which may be useful for future
%versions of this package:
%%
%\begin{itemize}
%\item
%\ldots
%\end{itemize}

%%%%%%%%%%%%%%%%%%%%%%%%%%%%%%%%%%%%%%%%%%%%%%%%%%%%%%%%%%%%%%%%%%%%%%%%%%%%%%%%
\subsection{Revision History}

%%%%%%%%%%%%%%%%%%%%%%%%%%%%%%%%%%%%%%%%
\paragraph{v2.0:} 2018/12/30

\begin{itemize}
\item
immediate forward processing
\item
added |\childdocby| mechanism
\item
manual restructured
\end{itemize}

%%%%%%%%%%%%%%%%%%%%%%%%%%%%%%%%%%%%%%%%
\paragraph{v1.6:} 2018/01/17

\begin{itemize}
\item
application for development of include files
\item
corrections to manual
\end{itemize}

%%%%%%%%%%%%%%%%%%%%%%%%%%%%%%%%%%%%%%%%
\paragraph{v1.5:} 2017/05/21

\begin{itemize}
\item
more complete structuring introduced
\item
|\childdocof| introduced
\item
|\childdoc| renamed to |\childdocmain|
\item
|\childredirect| renamed to |\childdocforward| and |\childdocforwardprefix|
and functionality expanded
\end{itemize}

%%%%%%%%%%%%%%%%%%%%%%%%%%%%%%%%%%%%%%%%
\paragraph{v1.0:} 2017/04/27

\begin{itemize}
\item
manual and install package
\item
first version published on CTAN
\end{itemize}

%%%%%%%%%%%%%%%%%%%%%%%%%%%%%%%%%%%%%%%%
\paragraph{v0.6:} 2017/04/26

\begin{itemize}
\item
redirection mechanism added
\end{itemize}

%%%%%%%%%%%%%%%%%%%%%%%%%%%%%%%%%%%%%%%%
\paragraph{v0.5:} 2017/04/26

\begin{itemize}
\item
functionality in definition file
\end{itemize}


%%%%%%%%%%%%%%%%%%%%%%%%%%%%%%%%%%%%%%%%%%%%%%%%%%%%%%%%%%%%%%%%%%%%%%%%%%%%%%%%
%%%%%%%%%%%%%%%%%%%%%%%%%%%%%%%%%%%%%%%%%%%%%%%%%%%%%%%%%%%%%%%%%%%%%%%%%%%%%%%%
%%%%%%%%%%%%%%%%%%%%%%%%%%%%%%%%%%%%%%%%%%%%%%%%%%%%%%%%%%%%%%%%%%%%%%%%%%%%%%%%
\appendix

\settowidth\MacroIndent{\rmfamily\scriptsize 000\ }

 \DocInput{childdoc.dtx}

\end{document}
%</driver>
% \fi
%
% %%%%%%%%%%%%%%%%%%%%%%%%%%%%%%%%%%%%%%%%%%%%%%%%%%%%%%%%%%%%%%%%%%%%%%%%%%%%%%
% %%%%%%%%%%%%%%%%%%%%%%%%%%%%%%%%%%%%%%%%%%%%%%%%%%%%%%%%%%%%%%%%%%%%%%%%%%%%%%
% \section{Sample}
%\iffalse
%<*samplemain>
%\fi
%
% The following presents a sample document
% with two chapters, two parts, a title page,
% a compile flag as well as three forwarding files to set the flag.
% It consists of eight |.tex| files:
% \begin{center}
% \begin{tabular}{ll}
% |cdocsamp.tex|&main file\\
% |cdocsch1.tex|&include file for chapter 1\\
% |cdocsch2.tex|&include file for chapter 2\\
% |cdocspt3.tex|&include file for part 3\\
% |cdocspt4.tex|&include file for part 4\\
% |cdocsdrf.tex|&forwarding file for main file in draft mode\\
% |cdocsfi1.tex|&forwarding file for final version of chapter 1\\
% |cdocsfi2.tex|&forwarding file for final version of chapter 2\\
% \end{tabular}
% \end{center}
% Each of the eight files can be compiled directly by the \LaTeX{} compiler.
%
% %%%%%%%%%%%%%%%%%%%%%%%%%%%%%%%%%%%%%%
% \paragraph{Main File.}
%
% The main file is called |cdocsamp.tex|.
%
% Load the \textsf{childdoc} definitions and
% declare the filename for the main document:
%    \begin{macrocode}
\input{childdoc.def}
\childdocmain{}
%    \end{macrocode}

% Optional override for |\version| flag:
%    \begin{macrocode}
%%\ifchilddoc\else\providecommand{\version}{draft}\fi
%    \end{macrocode}

% Define the default values for the |\version| flag
% (|final| for the main file and |draft| for childs):
%    \begin{macrocode}
\ifchilddoc
\providecommand{\version}{draft}
\else
\providecommand{\version}{final}
\fi
%    \end{macrocode}

% Load the standard document class:
%    \begin{macrocode}
\documentclass[12pt]{article}
%    \end{macrocode}

% Start the document body:
%    \begin{macrocode}
\begin{document}
%    \end{macrocode}

% Declare a title page.
% Print title, part of document being processed and version flag:
%    \begin{macrocode}
\addtocounter{page}{-1}
\begin{center}
{\LARGE\bfseries{}childdoc example\par}
\vspace{1cm}
\ifchilddoc
\ifchilddocmanual part\else chapter\fi:
`\childdocname' of `\childdocjob'\par
\else
main document: `\childdocjob'\par
\fi
version: \version\par
\end{center}
\newpage
%    \end{macrocode}

% Manually include selected file,
% otherwise process as usual:
%    \begin{macrocode}
\ifchilddocmanual
\section*{part `\childdocname'}
\input{\childdocname}
\else
%    \end{macrocode}

% Include the two chapters:
%    \begin{macrocode}
\include{cdocsch1}
\include{cdocsch2}
%    \end{macrocode}

% Include the two parts unless only chapters should be displayed:
%    \begin{macrocode}
\ifchilddoc\else
\section{part three}
\input{cdocspt3}
\section{part four}
\input{cdocspt4}
\fi
%    \end{macrocode}

% Process as usual until here:
%    \begin{macrocode}
\fi
%    \end{macrocode}

% End of document body:
%    \begin{macrocode}
\end{document}
%    \end{macrocode}
%\iffalse
%</samplemain>
%\fi
%
% %%%%%%%%%%%%%%%%%%%%%%%%%%%%%%%%%%%%%%
% \paragraph{Chapter Include Files.}
%
% The include files are called |cdocsch1.tex| and |cdocsch2.tex|.
%
%\iffalse
%<*samplechap1|samplechap2>
%\fi

% Optional override for |\version| flag:
%    \begin{macrocode}
%%\providecommand{\version}{final}
%    \end{macrocode}

% Include the main document:
%    \begin{macrocode}
\input{childdoc.def}
\childdocof{cdocsamp}
%    \end{macrocode}

%\iffalse
%</samplechap1|samplechap2>
%\fi
%
%\iffalse
%<*samplechap1>
%\fi
% Some text for chapter 1:
%    \begin{macrocode}
\section{one}
some text in chapter one
%    \end{macrocode}

%\iffalse
%</samplechap1>
%\fi
% Some text for chapter 2:
%\iffalse
%<*samplechap2>
%\fi
%    \begin{macrocode}
\section{two}
more text in chapter two
%    \end{macrocode}

%\iffalse
%</samplechap2>
%\fi
%
% %%%%%%%%%%%%%%%%%%%%%%%%%%%%%%%%%%%%%%
% \paragraph{Part Include Files.}
%
% The include files are called |cdocspt3.tex| and |cdocspt4.tex|.
%
%\iffalse
%<*samplepart3|samplepart4>
%\fi

% Optional override for |\version| flag:
%    \begin{macrocode}
%%\providecommand{\version}{final}
%    \end{macrocode}

% Include the main document:
%    \begin{macrocode}
\input{childdoc.def}
\childdocby{cdocsamp}
%    \end{macrocode}

%\iffalse
%</samplepart3|samplepart4>
%\fi
%
%\iffalse
%<*samplepart3>
%\fi
% Some text for part 3:
%    \begin{macrocode}
some text in part three
%    \end{macrocode}

%\iffalse
%</samplepart3>
%\fi
% Some text for part 4:
%\iffalse
%<*samplepart4>
%\fi
%    \begin{macrocode}
more text in part four
%    \end{macrocode}

%\iffalse
%</samplepart4>
%\fi
%
% %%%%%%%%%%%%%%%%%%%%%%%%%%%%%%%%%%%%%%
% \paragraph{Forwarding for a Complete Draft.}
%
% The following forwarding file |cdocsdrf.tex|
% compiles the main document in draft mode:
%\iffalse
%<*sampledraft>
%\fi
%    \begin{macrocode}
\def\version{draft}
\input{childdoc.def}
\childdocforward{cdocsamp}
%    \end{macrocode}

%\iffalse
%</sampledraft>
%\fi
%
% %%%%%%%%%%%%%%%%%%%%%%%%%%%%%%%%%%%%%%
% \paragraph{Forwarding for Final Version of the Chapters.}
%
% The following forwarding files |cdocsfn1.tex| and |cdocsfn2.tex|
% (with identical content)
% compile the final versions of the child documents
% |cdocsch1.tex| and |cdocsch2.tex|, respectively:
%\iffalse
%<*samplefinal>
%\fi
%    \begin{macrocode}
\def\version{final}
\input{childdoc.def}
\childdocforwardprefix[cdocsamp]{cdocsfn}{cdocsch}
%    \end{macrocode}

%\iffalse
%</samplefinal>
%\fi
%
% %%%%%%%%%%%%%%%%%%%%%%%%%%%%%%%%%%%%%%
% \paragraph{Command Line Processing.}
%
% The following three command lines generate the output files
% |cdocscld|, |cdocscl1| and |cdocscl2|
% which should be identical to
% |cdocsdrf|, |cdocsch1| and |cdocsfn2|, respectively:
% \begin{center}
% \begin{tabular}{l}
% |latex -jobname cdocscld \|\\
% |  "\def\version{draft}\input{childdoc.def}\childdocforward{cdocsamp}"|\\
% |latex -jobname cdocscl1 \|\\
% |  "\input{childdoc.def}\childdocforward[cdocsamp]{cdocsch1}"|\\
% |latex -jobname cdocscl2 \|\\
% |  "\def\version{final}\input{childdoc.def}\childdocforward{cdocsch2}"|
% \end{tabular}
% \end{center}
% Note that the trailing backslash on each first line
% merely continues the input to the second line
% (for convenient cut ant paste).
% Furthermore, the command |latex| can be replaced by any
% of its alternative versions such as |pdflatex|.
%
% %%%%%%%%%%%%%%%%%%%%%%%%%%%%%%%%%%%%%%%%%%%%%%%%%%%%%%%%%%%%%%%%%%%%%%%%%%%%%%
% %%%%%%%%%%%%%%%%%%%%%%%%%%%%%%%%%%%%%%%%%%%%%%%%%%%%%%%%%%%%%%%%%%%%%%%%%%%%%%
% \section{Implementation}
%\iffalse
%<*package>
%\fi
%
% This section describes the definitions file |childdoc.def|.

% The definitions cannot be loaded using |\usepackage| or |\RequirePackage|
% which has a mechanism to prevent loading a style file more than once.
% When loading the definitions by means of |\input|
% multiple instances have to be prevented manually:
%\iffalse
%This code needs to be before the `\ProvidesFile' directive
%which is defined at the beginning of this file.
%Therefore it is also placed there and commented out here.
%</package>
%<*discard>
%\fi
%    \begin{macrocode}
\ifdefined\childdocmain\endinput\fi
%    \end{macrocode}
%\iffalse
%</discard>
%<*package>
%\fi
%
% \macro{\ifchilddoc}
% \macro{\ifchilddocmanual}
% The conditional |\ifchilddoc| tells whether a
% child (true) or main (false) document is being compiled.
% The conditional |\ifchilddocmanual| tells whether
% the |\includeonly| mechanism is used (false) or
% the selection of child files must be performed manually (true).
% The definitions initialise to false:
%    \begin{macrocode}
\newif\ifchilddoc
\newif\ifchilddocmanual
%    \end{macrocode}

% \macro{\childdocname}
% \macro{\childdocjob}
% The macro |\childdocname| stores the name of the main document
% to be compiled. The macro |\childdocjob| stores the name of
% the document on which the \LaTeX{} compiler was originally invoked.
% The content of |\jobname| cannot be compared
% to filenames specified in the source due to different catcodes.
% The following code rescans |\jobname|, stores the result
% in |\childdocname| and saves a copy in |\childdocjob|:
%    \begin{macrocode}
\edef\childdocname{\scantokens\expandafter{\jobname\noexpand}}
\let\childdocjob\childdocname
%    \end{macrocode}

% \macro{\childdocdisable}
% The macro |\childdocdisable| prevents the main file
% from being processed more than once.
% At this stage, the main document command |\childdocmain|
% is assumed to be called once again where it should do nothing.
% Any subsequent call to it should prevent
% a secondary processing of the main document
% It overwrites the forwarding commands
% |\childdocof| and |\childdocforward|
% with empty macros to prevent further inclusions of the main document:
%    \begin{macrocode}
\newcommand{\childdocdisable}
{
  \renewcommand{\childdocmain}[1]{\renewcommand{\childdocmain}[1]{\endinput}}
  \renewcommand{\childdocof}[1]{}
  \renewcommand{\childdocby}[2][]{}
  \renewcommand{\childdocforward}[2][]{}
  \renewcommand{\childdocdisable}{}
}
%    \end{macrocode}

% \macro{\childdocmain}
% The macro |\childdocmain| is to be called at the top of the main file
% with nothing or the main filename (without extension) as argument.
% First, it breaks loops.
% If the argument is not empty and does not match |\childdocname|
% (which is set by the first inclusion of |childdoc.def|),
% |\ifchilddoc| is set to true, |\includeonly| is applied to the child file
% and |\jobname| is set to the main file
% (for proper handling of |.aux| files):
%    \begin{macrocode}
\newcommand{\childdocmain}[1]
{
  \childdocdisable\childdocmain{}
  \if?#1?\else
    \begingroup
      \def\childdoctmp{#1}
      \ifx\childdoctmp\childdocname
        \def\childdoctmp{}
      \else
        \def\childdoctmp
        {
          \childdoctrue
          \includeonly{\childdocname}
          \def\childdocjob{#1}
          \def\jobname{#1}
        }
      \fi
      \expandafter
    \endgroup
    \childdoctmp
  \fi
}
%    \end{macrocode}

% \macro{\childdocof}
% The command |\childdocof| redirects
% compilation to the main file |#1|.
%    \begin{macrocode}
\newcommand{\childdocof}[1]
{
  \childdocdisable
  \childdoctrue
  \includeonly{\childdocname}
  \def\jobname{#1}
  \def\childdocjob{#1}
  \input{#1}
}
%    \end{macrocode}

% \macro{\childdocby}
% The command |\childdocby| ....
%    \begin{macrocode}
\newcommand{\childdocby}[2][]
{
  \childdocdisable
  \childdoctrue
  \childdocmanualtrue
  \if?#1?\else
    \def\jobname{#2}
  \fi
  \def\childdocjob{#2}
  \input{#2}
  \endinput
}
%    \end{macrocode}

% \macro{\childdocforward}
% The command |\childdocforward| redirects
% compilation to the main file or
% (if the optional argument is given) a child file.
% Parameters are set as if the main file
% or a child file starting with |\childdocof| was compiled.
% Then compilation is handed over to the main file:
%    \begin{macrocode}
\newcommand{\childdocforward}[2][]
{
  \begingroup
    \if?#1?
      \def\childdoctmp
      {
        \def\childdocname{#2}
        \def\childdocjob{#2}
        \def\jobname{#2}
        \input{#2}
        \endinput
      }
    \else
      \def\childdoctmp
      {
        \childdocdisable
        \def\childdocname{#2}
        \childdoctrue
        \includeonly{#2}
        \def\childdocjob{#1}
        \def\jobname{#1}
        \input{#1}
        \endinput
      }
    \fi
    \expandafter
  \endgroup
  \childdoctmp
}
%    \end{macrocode}

% \macro{\childdocforwardprefix}
% The command |\childdocforwardprefix| redirects
% compilation to the main or a child file by means of a pattern.
% The prefix |#1| in the current filename is replaced by |#2|
% and the suffix of the current filename is kept
% (it is assumed that the filename does not contain the substring `|~~~|'
% which is used as a delimiter).
% Compilation is handed over to the new file by |\childdocforward|:
%    \begin{macrocode}
\newcommand{\childdocforwardprefix}[3][]
{
  \begingroup
    \def\childdocextract #2##1~~~{\def\childdoctmp{\childdocforward[#1]{#3##1}}}
    \expandafter\childdocextract\childdocname~~~
    \expandafter
  \endgroup
  \childdoctmp
}
%    \end{macrocode}

% \macro{\childdoc}
% The deprecated macro |\childdoc| is a legacy version of |\childdocmain|:
%    \begin{macrocode}
\newcommand{\childdoc}{\childdocmain}
%    \end{macrocode}

% \macro{\childdocredirect}
% The deprecated macro |\childdocredirect| is a legacy version
% of |\childdocforward| and |\childdocforwardprefix|:
%    \begin{macrocode}
\newcommand{\childdocredirect}[2][]
{
  \begingroup
    \if?#1?
      \def\childdoctmp{\childdocforward{#2}}
    \else
      \def\childdoctmp{\childdocforwardprefix{#1}{#2}}
    \fi
    \expandafter
  \endgroup
  \childdoctmp
}
%    \end{macrocode}

%\iffalse
%</package>
%\fi
%
\endinput
|\\
|\childdocforwardprefix[|\textit{main}|]{|\textit{prefix}|}{|\textit{dest}|}|
\end{tabular}
\end{center}
%
the destination file is determined by a pattern
depending on the current file:
To make this work, the current file must be called
`{\textit{prefix}\hspace{0.2em}\textit{suffix}}'
with \textit{prefix} matching precisely the argument.
Processing is then passed on to the file
`{\textit{dest}\hspace{0.2em}\textit{suffix}}'.
Surely, the same effect is achieved by
directly specifying the
argument `{\textit{dest}\hspace{0.2em}\textit{suffix}}'
in the first form.
However, that requires to set up a different file
for each child. With the alternative form of the command
all these files can have exactly the same content
which simplifies setting them up and maintaining them.

For example, the following file |draft.tex|
with a compilation flag |\version| as described in \secref{sec:flags}
compiles the main document as a draft:
%
\begin{center}
\begin{tabular}{l}
|\def\version{draft}|\\
|% \iffalse
%
% childdoc.dtx Copyright (C) 2017-2018 Niklas Beisert
%
% This work may be distributed and/or modified under the
% conditions of the LaTeX Project Public License, either version 1.3
% of this license or (at your option) any later version.
% The latest version of this license is in
%   http://www.latex-project.org/lppl.txt
% and version 1.3 or later is part of all distributions of LaTeX
% version 2005/12/01 or later.
%
% This work has the LPPL maintenance status `maintained'.
%
% The Current Maintainer of this work is Niklas Beisert.
%
% This work consists of the files childdoc.dtx and childdoc.ins
% and the derived files childdoc.def and cdocsamp.tex with
% cdocsch1.tex, cdocsch2.tex, cdocsdrf.tex, cdocsfn1.tex, cdocsfn2.tex.
%
%<package>\ifdefined\childdocmain\endinput\fi
%<package>\ProvidesFile{childdoc.def}[2018/12/30 v2.0 child document driver]
%<samplemain>\ProvidesFile{cdocsamp.tex}[2018/12/30 v2.0 sample for childdoc]
%<*driver>
%\ProvidesFile{childdoc.drv}[2018/12/30 v2.0 childdoc reference manual file]
\PassOptionsToClass{10pt,a4paper}{article}
\documentclass{ltxdoc}

\usepackage[margin=35mm]{geometry}
\usepackage{hyperref}
\usepackage{hyperxmp}
\usepackage[usenames]{color}

\hypersetup{colorlinks=true}
\hypersetup{pdfstartview=FitH}
\hypersetup{pdfpagemode=UseNone}
\hypersetup{pdfsource={}}
\hypersetup{pdflang={en-UK}}
\hypersetup{pdfcopyright={Copyright 2017-2018 Niklas Beisert.
  This work may be distributed and/or modified under the
  conditions of the LaTeX Project Public License, either version 1.3
  of this license or (at your option) any later version.}}
\hypersetup{pdflicenseurl={http://www.latex-project.org/lppl.txt}}
\hypersetup{pdfcontactaddress={ETH Zurich, ITP, HIT K,
  Wolfgang-Pauli-Strasse 27}}
\hypersetup{pdfcontactpostcode={8093}}
\hypersetup{pdfcontactcity={Zurich}}
\hypersetup{pdfcontactcountry={Switzerland}}
\hypersetup{pdfcontactemail={nbeisert@itp.phys.ethz.ch}}
\hypersetup{pdfcontacturl={http://people.phys.ethz.ch/\xmptilde nbeisert/}}

\newcommand{\secref}[1]{\hyperref[#1]{section \ref*{#1}}}

\parskip1ex
\parindent0pt
\let\olditemize\itemize
\def\itemize{\olditemize\parskip0pt}

\begin{document}

\title{The \textsf{childdoc} Package}
\hypersetup{pdftitle={The childdoc Package}}
\author{Niklas Beisert\\[2ex]
  Institut f\"ur Theoretische Physik\\
  Eidgen\"ossische Technische Hochschule Z\"urich\\
  Wolfgang-Pauli-Strasse 27, 8093 Z\"urich, Switzerland\\[1ex]
  \href{mailto:nbeisert@itp.phys.ethz.ch}
  {\texttt{nbeisert@itp.phys.ethz.ch}}}
\hypersetup{pdfauthor={Niklas Beisert}}
\hypersetup{pdfsubject={Manual for the LaTeX2e Package childdoc}}
\date{30 December 2018, \textsf{v2.0}}
\maketitle

\begin{abstract}\noindent
\textsf{childdoc} is a \LaTeXe{} package
that enables the direct compilation
of document sections included by |\include|
to individual files.
\end{abstract}

\begingroup
\parskip0ex
\tableofcontents
\endgroup

%%%%%%%%%%%%%%%%%%%%%%%%%%%%%%%%%%%%%%%%%%%%%%%%%%%%%%%%%%%%%%%%%%%%%%%%%%%%%%%%
%%%%%%%%%%%%%%%%%%%%%%%%%%%%%%%%%%%%%%%%%%%%%%%%%%%%%%%%%%%%%%%%%%%%%%%%%%%%%%%%
\section{Introduction}

\LaTeX{} provides a mechanism to structure a large document (such as a book)
into a main file and several child files (containing the chapters)
using the |\include| command.
This mechanism is beneficial for documents
which span hundreds of pages in order to
make the source file(s) more manageable.
Moreover, compilation can be restricted to
selected child files by means of the |\includeonly| command.
The latter feature can be used to reduce the compilation time while editing
(this was significantly more useful in the earlier days of \LaTeX{})
or to generate a smaller document which is easier to navigate.
Another application of |\includeonly| is to generate
documents consisting of selected parts of the complete document.

However, there are a few drawbacks of the plain |\include| mechanism:
\begin{itemize}
\item
The child files cannot be compiled on their own,
they can only be compiled via the main file.
A naive editing environment
(such as a text editor with an option
to have the current file processed by \LaTeX)
may require one to switch to the main file before compiling;
attempting to compile the child file produces errors.
\item
The main file must be modified (each time)
to adjust the |\includeonly| command
to the present needs. This easily leaves the main file in a messy state.
\item
The generated document will always carry the filename
of the main document. This is inconvenient if
several child files are to be compiled and
to be kept for distribution.
\end{itemize}

The present package provides a simple interface
to make child files individually compilable by \LaTeX{}.
Compiling a child file then has the same effect as compiling
the main file with an |\includeonly| command
to select the appropriate child.
Moreover the generated document will carry the name of the child
rather than the main file.
This resolves all three above issues.

This feature is meant to make the editing of books,
thesis documents and lecture notes somewhat more convenient.
However, the package can also be used efficiently for
composing a series of documents (such as exercise sheets)
which are typically distributed individually.
It then assists the author in generating the individual documents
(potentially in different versions)
as well as a document containing the collected series.
Another application is in developing style files
or other kinds of included material
where compilation of the style file could redirect
to a sample or test file.

%%%%%%%%%%%%%%%%%%%%%%%%%%%%%%%%%%%%%%%%%%%%%%%%%%%%%%%%%%%%%%%%%%%%%%%%%%%%%%%%
%%%%%%%%%%%%%%%%%%%%%%%%%%%%%%%%%%%%%%%%%%%%%%%%%%%%%%%%%%%%%%%%%%%%%%%%%%%%%%%%
\section{Usage}

First of all, the package \textsf{childdoc} is \emph{not} a standard
\LaTeXe{} |.sty| style file! Therefore it needs to be invoked in
a non-standard way.

%%%%%%%%%%%%%%%%%%%%%%%%%%%%%%%%%%%%%%%%%%%%%%%%%%%%%%%%%%%%%%%%%%%%%%%%%%%%%%%%
\subsection{Included Files}
\label{sec:include}

%%%%%%%%%%%%%%%%%%%%%%%%%%%%%%%%%%%%%%%%
\DescribeMacro{\childdocmain}
To use the package, add the commands
\begin{center}
\begin{tabular}{l}
|\input{childdoc.def}|\\
|\childdocmain{}|\\
\end{tabular}
\end{center}
at the very top of the main \LaTeX{} file,
in particular \emph{before} the |\documentclass| statement!
The argument of |\childdocmain| should be left empty
(but it must be present).

%%%%%%%%%%%%%%%%%%%%%%%%%%%%%%%%%%%%%%%%
\DescribeMacro{\childdocof}
Furthermore, add the commands
\begin{center}
\begin{tabular}{l}
|\input{childdoc.def}|\\
|\childdocof{|\textit{main}|}|\\
\end{tabular}
\end{center}
at the top of every child file \textit{child}
which is included by |\include{|\textit{child}|}|
from within the main file
(or at least for those files to be compiled individually).
The argument \textit{main} must be the filename of the main file.

There are a couple of
considerations in setting up the main and child documents:

%%%%%%%%%%%%%%%%%%%%%%%%%%%%%%%%%%%%%%%%
\paragraph{Restrictions.}

Please note the following restrictions:
\begin{itemize}
\item
|\childdocmain| must be called with one argument \textit{main}
to ensure compatibility with earlier version of the package.
It must either be empty (|\childdocmain{}|)
or precisely match the filename of the main file in which it is specified.
See \secref{sec:detection} for further information.
\item
The filename \textit{main} must be specified without the |.tex| extension.
\item
The filename \textit{main} is case sensitive
(even in case-insensitive file systems)
due to internal string comparison.
\item
The argument \textit{main} should be fully expanded, it cannot be a macro.
\item
Subdirectories and special characters should be avoided in filenames.
\item
The command |\childdocmain{|\textit{main}|}| must be followed by a whitespace.
It should not be followed immediately by another command
or by a comment mark `|%|'.
This is because the \TeX{} parser reads the token immediately following
the argument of |\childdocmain| and puts it
at the beginning of every child section;
however, a white\-space is ignored.
\end{itemize}

%%%%%%%%%%%%%%%%%%%%%%%%%%%%%%%%%%%%%%%%
\paragraph{Content of Main File.}

It is advisable to place all content in the child files included by |\include|.
Any output contained in the main file will appear in all child documents
unless suppressed manually;
it cannot be suppressed automatically by the |\includeonly| directive
and thus should normally be avoided.
A method to include some content in the main file
by means of conditional processing is described in \secref{sec:conditional}.

%%%%%%%%%%%%%%%%%%%%%%%%%%%%%%%%%%%%%%%%
\paragraph{Page Numbering.}

When only a part of the document is compiled,
the appropriate numbering of pages
(as well as other status parameters)
is determined from the |.aux| files.
The latter contain information from previous passes.
However this information needs to propagate through
all intermediate child documents.
Therefore the page numbering in child documents may well
be inconsistent until the complete document is compiled at least once.

A useful (if unconventional) way to always ensure a consistent
page numbering is to restart the numbering in each child document
and denote the pages by `\textit{child}|.|\textit{page}'
where \textit{child} represents the chapter/section number of the child file.
This can be achieved by the command
|\numberwithin{page}{|\textit{child}|}|
of the \textsf{amsmath} package
where \textit{child} can be |chapter| or |section|
depending on the chosen structuring.
Alternatively, one can modify the macro |\thepage| appropriately
and reset the counter |page| at the start of each child file.

%%%%%%%%%%%%%%%%%%%%%%%%%%%%%%%%%%%%%%%%%%%%%%%%%%%%%%%%%%%%%%%%%%%%%%%%%%%%%%%%
\subsection{Conditional Processing}
\label{sec:conditional}

The package provides a mechanism to compile different versions
of a document. To customise the versions further some conditional processing
can come in handy to distinguish which version is being compiled.
The package provides two macros to describe the compilation context:

%%%%%%%%%%%%%%%%%%%%%%%%%%%%%%%%%%%%%%%%
\DescribeMacro{\ifchilddoc}
The conditional |\ifchilddoc| distinguishes between the compilation of
child documents and the main document:
%
\begin{center}
|\ifchilddoc |\textit{child-code}| |[|\||else |\textit{main-code}]| \||fi|
\end{center}

%%%%%%%%%%%%%%%%%%%%%%%%%%%%%%%%%%%%%%%%
\DescribeMacro{\childdocname}
\DescribeMacro{\childdocjob}
The macro |\childdocname| contains the filename (without extension)
of the main or child file being processed.
Note that |\childdocjob| will always contain the name of the main file.

%%%%%%%%%%%%%%%%%%%%%%%%%%%%%%%%%%%%%%%%
\paragraph{Title Page.}

Conditional processing can be used to include a title or banner page
in the main document when proper precautions are taken.
Importantly, the code in the main file should ensure that the page counter
(as well as other status parameters which are stored in the |.aux| files)
takes the same value after the conditional processing.
Otherwise the page numbers may take divergent values
depending on which part is compiled.

For example, a title page could be declared by:
%
\begin{center}
\begin{tabular}{l}
|\ifchilddoc\||else|\\
|\addtocounter{page}{-1}|\\
\textit{code for title page}\\
|\newpage|\\
|\||fi|
\end{tabular}
\end{center}
%
A banner page for the child documents can be generated by:
%
\begin{center}
\begin{tabular}{l}
|\ifchilddoc|\\
|\addtocounter{page}{-1}|\\
\textit{code for banner page}\\
|\newpage|\\
|\||fi|
\end{tabular}
\end{center}
%
Here one could write a message such as:
\begin{center}
|This is the part \childdocname{} of \childdocjob{}.|
\end{center}

%%%%%%%%%%%%%%%%%%%%%%%%%%%%%%%%%%%%%%%%%%%%%%%%%%%%%%%%%%%%%%%%%%%%%%%%%%%%%%%%
\subsection{Flags}
\label{sec:flags}

The package makes it easy to generate different versions
of the main or child documents.
To this end compilation flags can be defined
and assigned different default values.
They will be particularly useful in conjunction
with the forwarding mechanism described in \secref{sec:forward}.

For example, it may be useful to have a flag |\version|
which can be set to |draft| or |final|.
The document source will contain some conditional code
depending on the value of |\version|.
Suppose further, the flag should default to |final| for the main file
and to |draft| for child files
which is a natural assignment for editing the document.
This is achieved by placing the following code
in the preamble of the main document
(below the |\childdocmain| directive):
%
\begin{center}
\begin{tabular}{l}
|\ifchilddoc|\\
|\providecommand{\version}{draft}|\\
|\||else|\\
|\providecommand{\version}{final}|\\
|\||fi|
\end{tabular}
\end{center}
%
The definition by |\providecommand| makes sure
that previous definitions are not overwritten.
Further statements |\providecommand{\version}{...}|
can thus be added before the above code to override it.

For the main file, one might add a line
(between |\childdocmain| and the above block)
%
\begin{center}
|%\ifchilddoc\||else\providecommand{\version}{draft}\||fi|
\end{center}
%
which can be uncommented to produce a draft version.
Likewise one can add a line to the very top of a child file
(above the |\childdocof{|\textit{main}|}| directive)
%
\begin{center}
|%\providecommand{\version}{final}|
\end{center}
%
which can be uncommented to produce the final version of this child document.

%%%%%%%%%%%%%%%%%%%%%%%%%%%%%%%%%%%%%%%%%%%%%%%%%%%%%%%%%%%%%%%%%%%%%%%%%%%%%%%%
\subsection{Forwarding}
\label{sec:forward}

Different versions of the main or child documents
using compilation flags as described in \secref{sec:flags}
can be (permanently) stored in different files
for convenient compilation, viewing and distribution.
To this end, the package defines a command
to pass on compilation to a different file:

%%%%%%%%%%%%%%%%%%%%%%%%%%%%%%%%%%%%%%%%
\DescribeMacro{\childdocforward}
The command |\childdocforward| redirects processing to
another source file:
%
\begin{center}
\begin{tabular}{l}
|\input{childdoc.def}|\\
|\childdocforward[|\textit{main}|]{|\textit{dest}|}|\\
\end{tabular}
\end{center}
%
The argument \textit{dest} is the destination file
(without extension).
It should be the main file or one of the child files.
Note that further \textsf{childdoc} directives
such as |\childdocof| and |\childdocforward|
in the indicated file will be processed in this form.
The optional argument \textit{main}
passes on directly to the main file \textit{main}
while pretending to compile the child \textit{dest}.
This form behaves as if \textit{dest}
issues |\childdocof{|\textit{main}|}| right away,
and no further \textsf{childdoc} directives will be processed.

%%%%%%%%%%%%%%%%%%%%%%%%%%%%%%%%%%%%%%%%
\DescribeMacro{\...prefix}
In the alternative form |\childdocforwardprefix|,
%
\begin{center}
\begin{tabular}{l}
|\input{childdoc.def}|\\
|\childdocforwardprefix[|\textit{main}|]{|\textit{prefix}|}{|\textit{dest}|}|
\end{tabular}
\end{center}
%
the destination file is determined by a pattern
depending on the current file:
To make this work, the current file must be called
`{\textit{prefix}\hspace{0.2em}\textit{suffix}}'
with \textit{prefix} matching precisely the argument.
Processing is then passed on to the file
`{\textit{dest}\hspace{0.2em}\textit{suffix}}'.
Surely, the same effect is achieved by
directly specifying the
argument `{\textit{dest}\hspace{0.2em}\textit{suffix}}'
in the first form.
However, that requires to set up a different file
for each child. With the alternative form of the command
all these files can have exactly the same content
which simplifies setting them up and maintaining them.

For example, the following file |draft.tex|
with a compilation flag |\version| as described in \secref{sec:flags}
compiles the main document as a draft:
%
\begin{center}
\begin{tabular}{l}
|\def\version{draft}|\\
|\input{childdoc.def}|\\
|\childdocforward{|\textit{main}|}|
\end{tabular}
\end{center}
%
Likewise, the following files |final|\textit{nn}|.tex|
compile the final version of the child document
|child|\textit{nn}|.tex|:
%
\begin{center}
\begin{tabular}{l}
|\def\version{final}|\\
|\input{childdoc.def}|\\
|\childdocforwardprefix{final}{child}|
\end{tabular}
\end{center}
%

Note that when several versions of a main file and/or of each child file
are to be generated, it may be convenient to set up a |Makefile| or
shell script to automatise the process.

%%%%%%%%%%%%%%%%%%%%%%%%%%%%%%%%%%%%%%%%%%%%%%%%%%%%%%%%%%%%%%%%%%%%%%%%%%%%%%%%
\subsection{Command Line Processing}
\label{sec:commandline}

The effect of redirection files can also be achieved by invoking
the \LaTeX{} compiler with a more elaborate command line.
Most conveniently this should be done as part
of a shell script or a |Makefile|.

When using \textsf{childdoc} in the main file, the following
command lines effectively perform a redirection
(note that depending on the shell being used,
backslashes may have to be doubled: `|\|' $\to$ `|\\|'):
%
\begin{center}
|... -jobname "|\textit{target}|" |\\|"|[\textit{flags}]%
|\input{childdoc.def}\childdocforward[|\textit{main}|]{|\textit{dest}|}"|
\end{center}
%
Here \textit{target} is the name of the output file,
\textit{main} is the name of the main file
and \textit{dest} is the name of the main or child file to be processed
(all filenames without extensions).
The optional argument \textit{main} can be omitted
if \textit{main} matches \textit{dest}.
Optionally, compilation \textit{flags} can be defined via |\def| commands.
This command line makes the \TeX{} engine believe
it is compiling the file \textit{target}
whose content is specified as the latter parameter.
The provided code then forwards the processing to
\textit{main} or \textit{dest} as described in \secref{sec:forward}.

%%%%%%%%%%%%%%%%%%%%%%%%%%%%%%%%%%%%%%%%%%%%%%%%%%%%%%%%%%%%%%%%%%%%%%%%%%%%%%%%
\subsection{Include by Input}
\label{sec:input}

Including child documents by |\include| has some restrictions by design.
Most notably, the content of a child document always occupies
its own set of pages; pages cannot be shared between child documents.
Usually, this behaviour makes perfect sense
because each child document contain an essential part of the document.
However, in some situations it may be desirable to compose
a document from a collection of parts
without having mandatory page breaks between then.
For this case, the package
provides a mechanism to include parts
by |\input| which can also be processed individually.
However, by construction this mechanism
requires manual handling of the content to be output.

%%%%%%%%%%%%%%%%%%%%%%%%%%%%%%%%%%%%%%%%
\DescribeMacro{\ifchilddocmanual}
The main file should be prepared as usual, see \secref{sec:include}.
However, the document body must make a distinction
between processing of an individual part and of the main document, e.g.:
%
\begin{center}
\begin{tabular}{l}
|\ifchilddocmanual|\\
|\input{\childdocname}|\\
|\||else|\\
\textit{document body with }|\input{|\textit{part}|}|\\
|\||fi|
\end{tabular}
\end{center}
%
The conditional |\ifchilddocmanual| is true whenever
a part to be included by |\input| is being compiled,
and the name of the part is stored in |\childdocname|.

%%%%%%%%%%%%%%%%%%%%%%%%%%%%%%%%%%%%%%%%
\DescribeMacro{\childdocby}
Each part to be included by |\input| should start with:
%
\begin{center}
\begin{tabular}{l}
|\input{childdoc.def}|\\
|\childdocby{|\textit{main}|}|\\
\end{tabular}
\end{center}
%
The directive |\childdocby| is similar to |\childdocof|
described in \secref{sec:include},
but the subsequent selection of content must be done manually.
To that end, both |\ifchilddoc| and |\ifchilddocmanual|
will be true upon processing of a part,
and the name of the part is stored in |\childdocname|.
Note that |\jobname| will be set to the filename of the current part
so that each part receives an individual |.aux| file
that does not interfere with the |.aux| file(s) of the main document.
This behaviour can be altered by the alternative form
|\childdocby[*]{|\textit{main}|}| (with a non-empty optional argument)
which uses the |.aux| file of the main document
by setting |\jobname| to \textit{main}.

%%%%%%%%%%%%%%%%%%%%%%%%%%%%%%%%%%%%%%%%%%%%%%%%%%%%%%%%%%%%%%%%%%%%%%%%%%%%%%%%
\subsection{Driver Development}
\label{sec:driver}

The \textsf{childdoc} mechanism can also be use for the development
of definition files such as \LaTeX{} styles or classes.
This case differs from the above setup with multiple parts
included by |\include| in that no |\includeonly| should be invoked.
This can be achieved by starting the include file
(before |\ProvidesPackage|) with:
%
\begin{center}
\begin{tabular}{l}
|\input{childdoc.def}|\\
|\childdocforward{|\textit{main}|}|\\
\end{tabular}
\end{center}
%
or alternatively with:
%
\begin{center}
\begin{tabular}{l}
|\input{childdoc.def}|\\
|\childdocby{|\textit{main}|}|\\
\end{tabular}
\end{center}
%
Both forms have slightly different effects as described above.
The main file is prepared as usual, see \secref{sec:include}.

%%%%%%%%%%%%%%%%%%%%%%%%%%%%%%%%%%%%%%%%%%%%%%%%%%%%%%%%%%%%%%%%%%%%%%%%%%%%%%%%
\subsection{Legacy Detection}
\label{sec:detection}

The directive |\childdocmain| in the main file can detect
whether the complete document or merely a child is to be compiled
even without using the directive |\childdocof|.
This method is deprecated because it is less robust
and there is no compelling reason to use it;
it is merely provided for backward compatibility
and it may be removed in future versions.

If the detection mechanism is to be used,
it is mandatory to correctly specify
the filename of the main file as the argument of |\childdocmain|:
%
\begin{center}
\begin{tabular}{l}
|\input{childdoc.def}|\\
|\childdocmain{|\textit{main}|}|\\
\end{tabular}
\end{center}
%
If |\jobname| does not match the argument \textit{main} of |\childdocmain|,
it is assumed that |\jobname| points to the child file to be compiled.
When using |\childdocmain| with the main file specified as argument,
it suffices to start a child file
with just |\input{|\textit{main}|}|
without loading of the package and using |\childdocof|.
If instead all processing is done
with the appropriate \textsf{childdoc} directives,
the argument of \textit{main} of |\childdocmain| can be empty.

An alternative version of the command line processing described
in \secref{sec:commandline} using the detection mechanism reads:
%
\begin{center}
|... -jobname "|\textit{target}|" "|[\textit{flags}]%
[|\def\jobname{|\textit{dest}|}|]|\input{|\textit{main}|}"|
\end{center}

%%%%%%%%%%%%%%%%%%%%%%%%%%%%%%%%%%%%%%%%%%%%%%%%%%%%%%%%%%%%%%%%%%%%%%%%%%%%%%%%
\subsection{Manual Code}
\label{sec:manual}

In case one cannot be certain whether the definitions file |childdoc.def|
is installed on the target \TeX{} distribution
and one prefers not to ship it,
it is conceivable to paste a few relevant commands into the sources.

To that end, drop all statements |\input{childdoc.def}|
and perform the replacements as outlined below.
Instead of |\childdocmain{|\textit{main}|}| add the following code
to the top of the main file:
%
\begin{center}
\begin{tabular}{l}
|\||ifdefined\childdocname\endinput\||fi\newif\ifchilddoc|\\
|\edef\childdocname{\scantokens\expandafter{\jobname\noexpand}}|\\
|\def\childdocmain{|\textit{main}|}\||ifx\childdocmain\childdocname\||else|\\
|\childdoctrue\includeonly{\childdocname}\let\jobname\childdocmain\||fi|\\
\end{tabular}
\end{center}
%
Instead of |\childdocof{|\textit{main}|}| just include the main file
at the top of each child file:
%
\begin{center}
|\input{|\textit{main}|}|
\end{center}
%
A simple redirection |\childdocforward{|\textit{dest}|}| is achieved by:
%
\begin{center}
|\def\jobname{|\textit{dest}|}\input{\jobname}|
\end{center}
%
The redirection with prefix
|\childdocforwardprefix[|\textit{prefix}|]{|\textit{dest}|}|
is accomplished by:
%
\begin{center}
\begin{tabular}{l}
|{\edef\jobname{\scantokens\expandafter{\jobname\noexpand}}|\\
|\def\redirectjob |\textit{prefix}|#1~~~{\gdef\jobname{|\textit{dest}|#1}}|\\
|\expandafter\redirectjob\jobname~~~}\input{\jobname}|
\end{tabular}
\end{center}

In an alternative approach,
child documents can be compiled by a specific command line
without additional code or specific definitions:
%
\begin{center}
|... -jobname "|\textit{target}|" "|[\textit{flags}]%
|\includeonly{|\textit{dest}|}\input{|\textit{main}|}"|
\end{center}
%

%%%%%%%%%%%%%%%%%%%%%%%%%%%%%%%%%%%%%%%%%%%%%%%%%%%%%%%%%%%%%%%%%%%%%%%%%%%%%%%%
%%%%%%%%%%%%%%%%%%%%%%%%%%%%%%%%%%%%%%%%%%%%%%%%%%%%%%%%%%%%%%%%%%%%%%%%%%%%%%%%
\section{Information}

%%%%%%%%%%%%%%%%%%%%%%%%%%%%%%%%%%%%%%%%%%%%%%%%%%%%%%%%%%%%%%%%%%%%%%%%%%%%%%%%
\subsection{Copyright}

Copyright \copyright{} 2017--2018 Niklas Beisert

This work may be distributed and/or modified under the
conditions of the \LaTeX{} Project Public License, either version 1.3
of this license or (at your option) any later version.
The latest version of this license is in
  \url{http://www.latex-project.org/lppl.txt}
and version 1.3 or later is part of all distributions of \LaTeX{}
version 2005/12/01 or later.

This work has the LPPL maintenance status `maintained'.

The Current Maintainer of this work is Niklas Beisert.

This work consists of the files |README.txt|, |childdoc.ins| and |childdoc.dtx|
as well as the derived files |childdoc.def|, |cdocsamp.tex|
with |cdocsch1.tex|, |cdocsch2.tex|, |cdocspt3.tex|, |cdocspt4.tex|,
|cdocsdrf.tex|, |cdocsfn1.tex|, |cdocsfn2.tex|
as well as |childdoc.pdf|.

%%%%%%%%%%%%%%%%%%%%%%%%%%%%%%%%%%%%%%%%%%%%%%%%%%%%%%%%%%%%%%%%%%%%%%%%%%%%%%%%
\subsection{Files and Installation}

The package consists of the files:
%
\begin{center}
\begin{tabular}{ll}
    |README.txt|   & readme file \\
    |childdoc.ins| & installation file \\
    |childdoc.dtx| & source file \\
    |childdoc.def| & definition file \\
    |cdocsamp.tex| & sample main file \\
    |cdocsch1.tex| & sample include file \\
    |cdocsch2.tex| & sample include file \\
    |cdocspt3.tex| & sample part file \\
    |cdocspt4.tex| & sample part file \\
    |cdocsdrf.tex| & sample redirection file \\
    |cdocsfn1.tex| & sample redirection file \\
    |cdocsfn2.tex| & sample redirection file \\
    |childdoc.pdf| & manual
\end{tabular}
\end{center}
%
The distribution consists of the files
|README.txt|, |childdoc.ins| and |childdoc.dtx|.
%
\begin{itemize}
\item
Run (pdf)\LaTeX{} on |childdoc.dtx|
to compile the manual |childdoc.pdf| (this file).
\item
Run \LaTeX{} on |childdoc.ins| to create the definitions file |childdoc.def|
and the sample |cdocsamp.tex| with include files
|cdocsch1.tex|, |cdocsch2.tex|, |cdocspt3.tex|, |cdocspt4.tex|,
|cdocsdrf.tex|, |cdocsfn1.tex|, |cdocsfn2.tex|.
Then copy the file |childdoc.def| to an appropriate directory of your \LaTeX{}
distribution, e.g.\ \textit{texmf-root}|/tex/latex/childdoc|.
\end{itemize}

%%%%%%%%%%%%%%%%%%%%%%%%%%%%%%%%%%%%%%%%%%%%%%%%%%%%%%%%%%%%%%%%%%%%%%%%%%%%%%%%
\subsection{Related CTAN Packages}

There are several other packages which offer a similar functionality:
%
\begin{itemize}
\item
The packages
\href{http://ctan.org/pkg/docmute}{\textsf{docmute}},
\href{http://ctan.org/pkg/includex}{\textsf{includex}} and
\href{http://ctan.org/pkg/standalone}{\textsf{standalone}}
provide commands to include only the document body of
a child file thus allowing both files to be compiled individually.
\item
The packages \href{http://ctan.org/pkg/subdocs}{\textsf{subdocs}}
and \href{http://ctan.org/pkg/subfiles}{\textsf{subfiles}}
provide structures in which the main and child documents can be
encapsulated and allowing them to be compiled individually.
The inclusion mechanism is different from the conventional |\include|.
\item
The package \href{http://ctan.org/pkg/combine}{\textsf{combine}}
is an elaborate solution to combine several documents into one.
\end{itemize}
%
See also the CTAN topic \href{http://ctan.org/topic/subdocs}{\textsf{subdocs}}
for further related packages.
The present package differs from the above solutions in that
a document structure constructed with the conventional |\include| mechanism
just needs two extra commands at the top of every file
such that all constituent files can be compiled individually.

%%%%%%%%%%%%%%%%%%%%%%%%%%%%%%%%%%%%%%%%%%%%%%%%%%%%%%%%%%%%%%%%%%%%%%%%%%%%%%%%
%\subsection{Feature Suggestions}
%
%The following is a list of features which may be useful for future
%versions of this package:
%%
%\begin{itemize}
%\item
%\ldots
%\end{itemize}

%%%%%%%%%%%%%%%%%%%%%%%%%%%%%%%%%%%%%%%%%%%%%%%%%%%%%%%%%%%%%%%%%%%%%%%%%%%%%%%%
\subsection{Revision History}

%%%%%%%%%%%%%%%%%%%%%%%%%%%%%%%%%%%%%%%%
\paragraph{v2.0:} 2018/12/30

\begin{itemize}
\item
immediate forward processing
\item
added |\childdocby| mechanism
\item
manual restructured
\end{itemize}

%%%%%%%%%%%%%%%%%%%%%%%%%%%%%%%%%%%%%%%%
\paragraph{v1.6:} 2018/01/17

\begin{itemize}
\item
application for development of include files
\item
corrections to manual
\end{itemize}

%%%%%%%%%%%%%%%%%%%%%%%%%%%%%%%%%%%%%%%%
\paragraph{v1.5:} 2017/05/21

\begin{itemize}
\item
more complete structuring introduced
\item
|\childdocof| introduced
\item
|\childdoc| renamed to |\childdocmain|
\item
|\childredirect| renamed to |\childdocforward| and |\childdocforwardprefix|
and functionality expanded
\end{itemize}

%%%%%%%%%%%%%%%%%%%%%%%%%%%%%%%%%%%%%%%%
\paragraph{v1.0:} 2017/04/27

\begin{itemize}
\item
manual and install package
\item
first version published on CTAN
\end{itemize}

%%%%%%%%%%%%%%%%%%%%%%%%%%%%%%%%%%%%%%%%
\paragraph{v0.6:} 2017/04/26

\begin{itemize}
\item
redirection mechanism added
\end{itemize}

%%%%%%%%%%%%%%%%%%%%%%%%%%%%%%%%%%%%%%%%
\paragraph{v0.5:} 2017/04/26

\begin{itemize}
\item
functionality in definition file
\end{itemize}


%%%%%%%%%%%%%%%%%%%%%%%%%%%%%%%%%%%%%%%%%%%%%%%%%%%%%%%%%%%%%%%%%%%%%%%%%%%%%%%%
%%%%%%%%%%%%%%%%%%%%%%%%%%%%%%%%%%%%%%%%%%%%%%%%%%%%%%%%%%%%%%%%%%%%%%%%%%%%%%%%
%%%%%%%%%%%%%%%%%%%%%%%%%%%%%%%%%%%%%%%%%%%%%%%%%%%%%%%%%%%%%%%%%%%%%%%%%%%%%%%%
\appendix

\settowidth\MacroIndent{\rmfamily\scriptsize 000\ }

 \DocInput{childdoc.dtx}

\end{document}
%</driver>
% \fi
%
% %%%%%%%%%%%%%%%%%%%%%%%%%%%%%%%%%%%%%%%%%%%%%%%%%%%%%%%%%%%%%%%%%%%%%%%%%%%%%%
% %%%%%%%%%%%%%%%%%%%%%%%%%%%%%%%%%%%%%%%%%%%%%%%%%%%%%%%%%%%%%%%%%%%%%%%%%%%%%%
% \section{Sample}
%\iffalse
%<*samplemain>
%\fi
%
% The following presents a sample document
% with two chapters, two parts, a title page,
% a compile flag as well as three forwarding files to set the flag.
% It consists of eight |.tex| files:
% \begin{center}
% \begin{tabular}{ll}
% |cdocsamp.tex|&main file\\
% |cdocsch1.tex|&include file for chapter 1\\
% |cdocsch2.tex|&include file for chapter 2\\
% |cdocspt3.tex|&include file for part 3\\
% |cdocspt4.tex|&include file for part 4\\
% |cdocsdrf.tex|&forwarding file for main file in draft mode\\
% |cdocsfi1.tex|&forwarding file for final version of chapter 1\\
% |cdocsfi2.tex|&forwarding file for final version of chapter 2\\
% \end{tabular}
% \end{center}
% Each of the eight files can be compiled directly by the \LaTeX{} compiler.
%
% %%%%%%%%%%%%%%%%%%%%%%%%%%%%%%%%%%%%%%
% \paragraph{Main File.}
%
% The main file is called |cdocsamp.tex|.
%
% Load the \textsf{childdoc} definitions and
% declare the filename for the main document:
%    \begin{macrocode}
\input{childdoc.def}
\childdocmain{}
%    \end{macrocode}

% Optional override for |\version| flag:
%    \begin{macrocode}
%%\ifchilddoc\else\providecommand{\version}{draft}\fi
%    \end{macrocode}

% Define the default values for the |\version| flag
% (|final| for the main file and |draft| for childs):
%    \begin{macrocode}
\ifchilddoc
\providecommand{\version}{draft}
\else
\providecommand{\version}{final}
\fi
%    \end{macrocode}

% Load the standard document class:
%    \begin{macrocode}
\documentclass[12pt]{article}
%    \end{macrocode}

% Start the document body:
%    \begin{macrocode}
\begin{document}
%    \end{macrocode}

% Declare a title page.
% Print title, part of document being processed and version flag:
%    \begin{macrocode}
\addtocounter{page}{-1}
\begin{center}
{\LARGE\bfseries{}childdoc example\par}
\vspace{1cm}
\ifchilddoc
\ifchilddocmanual part\else chapter\fi:
`\childdocname' of `\childdocjob'\par
\else
main document: `\childdocjob'\par
\fi
version: \version\par
\end{center}
\newpage
%    \end{macrocode}

% Manually include selected file,
% otherwise process as usual:
%    \begin{macrocode}
\ifchilddocmanual
\section*{part `\childdocname'}
\input{\childdocname}
\else
%    \end{macrocode}

% Include the two chapters:
%    \begin{macrocode}
\include{cdocsch1}
\include{cdocsch2}
%    \end{macrocode}

% Include the two parts unless only chapters should be displayed:
%    \begin{macrocode}
\ifchilddoc\else
\section{part three}
\input{cdocspt3}
\section{part four}
\input{cdocspt4}
\fi
%    \end{macrocode}

% Process as usual until here:
%    \begin{macrocode}
\fi
%    \end{macrocode}

% End of document body:
%    \begin{macrocode}
\end{document}
%    \end{macrocode}
%\iffalse
%</samplemain>
%\fi
%
% %%%%%%%%%%%%%%%%%%%%%%%%%%%%%%%%%%%%%%
% \paragraph{Chapter Include Files.}
%
% The include files are called |cdocsch1.tex| and |cdocsch2.tex|.
%
%\iffalse
%<*samplechap1|samplechap2>
%\fi

% Optional override for |\version| flag:
%    \begin{macrocode}
%%\providecommand{\version}{final}
%    \end{macrocode}

% Include the main document:
%    \begin{macrocode}
\input{childdoc.def}
\childdocof{cdocsamp}
%    \end{macrocode}

%\iffalse
%</samplechap1|samplechap2>
%\fi
%
%\iffalse
%<*samplechap1>
%\fi
% Some text for chapter 1:
%    \begin{macrocode}
\section{one}
some text in chapter one
%    \end{macrocode}

%\iffalse
%</samplechap1>
%\fi
% Some text for chapter 2:
%\iffalse
%<*samplechap2>
%\fi
%    \begin{macrocode}
\section{two}
more text in chapter two
%    \end{macrocode}

%\iffalse
%</samplechap2>
%\fi
%
% %%%%%%%%%%%%%%%%%%%%%%%%%%%%%%%%%%%%%%
% \paragraph{Part Include Files.}
%
% The include files are called |cdocspt3.tex| and |cdocspt4.tex|.
%
%\iffalse
%<*samplepart3|samplepart4>
%\fi

% Optional override for |\version| flag:
%    \begin{macrocode}
%%\providecommand{\version}{final}
%    \end{macrocode}

% Include the main document:
%    \begin{macrocode}
\input{childdoc.def}
\childdocby{cdocsamp}
%    \end{macrocode}

%\iffalse
%</samplepart3|samplepart4>
%\fi
%
%\iffalse
%<*samplepart3>
%\fi
% Some text for part 3:
%    \begin{macrocode}
some text in part three
%    \end{macrocode}

%\iffalse
%</samplepart3>
%\fi
% Some text for part 4:
%\iffalse
%<*samplepart4>
%\fi
%    \begin{macrocode}
more text in part four
%    \end{macrocode}

%\iffalse
%</samplepart4>
%\fi
%
% %%%%%%%%%%%%%%%%%%%%%%%%%%%%%%%%%%%%%%
% \paragraph{Forwarding for a Complete Draft.}
%
% The following forwarding file |cdocsdrf.tex|
% compiles the main document in draft mode:
%\iffalse
%<*sampledraft>
%\fi
%    \begin{macrocode}
\def\version{draft}
\input{childdoc.def}
\childdocforward{cdocsamp}
%    \end{macrocode}

%\iffalse
%</sampledraft>
%\fi
%
% %%%%%%%%%%%%%%%%%%%%%%%%%%%%%%%%%%%%%%
% \paragraph{Forwarding for Final Version of the Chapters.}
%
% The following forwarding files |cdocsfn1.tex| and |cdocsfn2.tex|
% (with identical content)
% compile the final versions of the child documents
% |cdocsch1.tex| and |cdocsch2.tex|, respectively:
%\iffalse
%<*samplefinal>
%\fi
%    \begin{macrocode}
\def\version{final}
\input{childdoc.def}
\childdocforwardprefix[cdocsamp]{cdocsfn}{cdocsch}
%    \end{macrocode}

%\iffalse
%</samplefinal>
%\fi
%
% %%%%%%%%%%%%%%%%%%%%%%%%%%%%%%%%%%%%%%
% \paragraph{Command Line Processing.}
%
% The following three command lines generate the output files
% |cdocscld|, |cdocscl1| and |cdocscl2|
% which should be identical to
% |cdocsdrf|, |cdocsch1| and |cdocsfn2|, respectively:
% \begin{center}
% \begin{tabular}{l}
% |latex -jobname cdocscld \|\\
% |  "\def\version{draft}\input{childdoc.def}\childdocforward{cdocsamp}"|\\
% |latex -jobname cdocscl1 \|\\
% |  "\input{childdoc.def}\childdocforward[cdocsamp]{cdocsch1}"|\\
% |latex -jobname cdocscl2 \|\\
% |  "\def\version{final}\input{childdoc.def}\childdocforward{cdocsch2}"|
% \end{tabular}
% \end{center}
% Note that the trailing backslash on each first line
% merely continues the input to the second line
% (for convenient cut ant paste).
% Furthermore, the command |latex| can be replaced by any
% of its alternative versions such as |pdflatex|.
%
% %%%%%%%%%%%%%%%%%%%%%%%%%%%%%%%%%%%%%%%%%%%%%%%%%%%%%%%%%%%%%%%%%%%%%%%%%%%%%%
% %%%%%%%%%%%%%%%%%%%%%%%%%%%%%%%%%%%%%%%%%%%%%%%%%%%%%%%%%%%%%%%%%%%%%%%%%%%%%%
% \section{Implementation}
%\iffalse
%<*package>
%\fi
%
% This section describes the definitions file |childdoc.def|.

% The definitions cannot be loaded using |\usepackage| or |\RequirePackage|
% which has a mechanism to prevent loading a style file more than once.
% When loading the definitions by means of |\input|
% multiple instances have to be prevented manually:
%\iffalse
%This code needs to be before the `\ProvidesFile' directive
%which is defined at the beginning of this file.
%Therefore it is also placed there and commented out here.
%</package>
%<*discard>
%\fi
%    \begin{macrocode}
\ifdefined\childdocmain\endinput\fi
%    \end{macrocode}
%\iffalse
%</discard>
%<*package>
%\fi
%
% \macro{\ifchilddoc}
% \macro{\ifchilddocmanual}
% The conditional |\ifchilddoc| tells whether a
% child (true) or main (false) document is being compiled.
% The conditional |\ifchilddocmanual| tells whether
% the |\includeonly| mechanism is used (false) or
% the selection of child files must be performed manually (true).
% The definitions initialise to false:
%    \begin{macrocode}
\newif\ifchilddoc
\newif\ifchilddocmanual
%    \end{macrocode}

% \macro{\childdocname}
% \macro{\childdocjob}
% The macro |\childdocname| stores the name of the main document
% to be compiled. The macro |\childdocjob| stores the name of
% the document on which the \LaTeX{} compiler was originally invoked.
% The content of |\jobname| cannot be compared
% to filenames specified in the source due to different catcodes.
% The following code rescans |\jobname|, stores the result
% in |\childdocname| and saves a copy in |\childdocjob|:
%    \begin{macrocode}
\edef\childdocname{\scantokens\expandafter{\jobname\noexpand}}
\let\childdocjob\childdocname
%    \end{macrocode}

% \macro{\childdocdisable}
% The macro |\childdocdisable| prevents the main file
% from being processed more than once.
% At this stage, the main document command |\childdocmain|
% is assumed to be called once again where it should do nothing.
% Any subsequent call to it should prevent
% a secondary processing of the main document
% It overwrites the forwarding commands
% |\childdocof| and |\childdocforward|
% with empty macros to prevent further inclusions of the main document:
%    \begin{macrocode}
\newcommand{\childdocdisable}
{
  \renewcommand{\childdocmain}[1]{\renewcommand{\childdocmain}[1]{\endinput}}
  \renewcommand{\childdocof}[1]{}
  \renewcommand{\childdocby}[2][]{}
  \renewcommand{\childdocforward}[2][]{}
  \renewcommand{\childdocdisable}{}
}
%    \end{macrocode}

% \macro{\childdocmain}
% The macro |\childdocmain| is to be called at the top of the main file
% with nothing or the main filename (without extension) as argument.
% First, it breaks loops.
% If the argument is not empty and does not match |\childdocname|
% (which is set by the first inclusion of |childdoc.def|),
% |\ifchilddoc| is set to true, |\includeonly| is applied to the child file
% and |\jobname| is set to the main file
% (for proper handling of |.aux| files):
%    \begin{macrocode}
\newcommand{\childdocmain}[1]
{
  \childdocdisable\childdocmain{}
  \if?#1?\else
    \begingroup
      \def\childdoctmp{#1}
      \ifx\childdoctmp\childdocname
        \def\childdoctmp{}
      \else
        \def\childdoctmp
        {
          \childdoctrue
          \includeonly{\childdocname}
          \def\childdocjob{#1}
          \def\jobname{#1}
        }
      \fi
      \expandafter
    \endgroup
    \childdoctmp
  \fi
}
%    \end{macrocode}

% \macro{\childdocof}
% The command |\childdocof| redirects
% compilation to the main file |#1|.
%    \begin{macrocode}
\newcommand{\childdocof}[1]
{
  \childdocdisable
  \childdoctrue
  \includeonly{\childdocname}
  \def\jobname{#1}
  \def\childdocjob{#1}
  \input{#1}
}
%    \end{macrocode}

% \macro{\childdocby}
% The command |\childdocby| ....
%    \begin{macrocode}
\newcommand{\childdocby}[2][]
{
  \childdocdisable
  \childdoctrue
  \childdocmanualtrue
  \if?#1?\else
    \def\jobname{#2}
  \fi
  \def\childdocjob{#2}
  \input{#2}
  \endinput
}
%    \end{macrocode}

% \macro{\childdocforward}
% The command |\childdocforward| redirects
% compilation to the main file or
% (if the optional argument is given) a child file.
% Parameters are set as if the main file
% or a child file starting with |\childdocof| was compiled.
% Then compilation is handed over to the main file:
%    \begin{macrocode}
\newcommand{\childdocforward}[2][]
{
  \begingroup
    \if?#1?
      \def\childdoctmp
      {
        \def\childdocname{#2}
        \def\childdocjob{#2}
        \def\jobname{#2}
        \input{#2}
        \endinput
      }
    \else
      \def\childdoctmp
      {
        \childdocdisable
        \def\childdocname{#2}
        \childdoctrue
        \includeonly{#2}
        \def\childdocjob{#1}
        \def\jobname{#1}
        \input{#1}
        \endinput
      }
    \fi
    \expandafter
  \endgroup
  \childdoctmp
}
%    \end{macrocode}

% \macro{\childdocforwardprefix}
% The command |\childdocforwardprefix| redirects
% compilation to the main or a child file by means of a pattern.
% The prefix |#1| in the current filename is replaced by |#2|
% and the suffix of the current filename is kept
% (it is assumed that the filename does not contain the substring `|~~~|'
% which is used as a delimiter).
% Compilation is handed over to the new file by |\childdocforward|:
%    \begin{macrocode}
\newcommand{\childdocforwardprefix}[3][]
{
  \begingroup
    \def\childdocextract #2##1~~~{\def\childdoctmp{\childdocforward[#1]{#3##1}}}
    \expandafter\childdocextract\childdocname~~~
    \expandafter
  \endgroup
  \childdoctmp
}
%    \end{macrocode}

% \macro{\childdoc}
% The deprecated macro |\childdoc| is a legacy version of |\childdocmain|:
%    \begin{macrocode}
\newcommand{\childdoc}{\childdocmain}
%    \end{macrocode}

% \macro{\childdocredirect}
% The deprecated macro |\childdocredirect| is a legacy version
% of |\childdocforward| and |\childdocforwardprefix|:
%    \begin{macrocode}
\newcommand{\childdocredirect}[2][]
{
  \begingroup
    \if?#1?
      \def\childdoctmp{\childdocforward{#2}}
    \else
      \def\childdoctmp{\childdocforwardprefix{#1}{#2}}
    \fi
    \expandafter
  \endgroup
  \childdoctmp
}
%    \end{macrocode}

%\iffalse
%</package>
%\fi
%
\endinput
|\\
|\childdocforward{|\textit{main}|}|
\end{tabular}
\end{center}
%
Likewise, the following files |final|\textit{nn}|.tex|
compile the final version of the child document
|child|\textit{nn}|.tex|:
%
\begin{center}
\begin{tabular}{l}
|\def\version{final}|\\
|% \iffalse
%
% childdoc.dtx Copyright (C) 2017-2018 Niklas Beisert
%
% This work may be distributed and/or modified under the
% conditions of the LaTeX Project Public License, either version 1.3
% of this license or (at your option) any later version.
% The latest version of this license is in
%   http://www.latex-project.org/lppl.txt
% and version 1.3 or later is part of all distributions of LaTeX
% version 2005/12/01 or later.
%
% This work has the LPPL maintenance status `maintained'.
%
% The Current Maintainer of this work is Niklas Beisert.
%
% This work consists of the files childdoc.dtx and childdoc.ins
% and the derived files childdoc.def and cdocsamp.tex with
% cdocsch1.tex, cdocsch2.tex, cdocsdrf.tex, cdocsfn1.tex, cdocsfn2.tex.
%
%<package>\ifdefined\childdocmain\endinput\fi
%<package>\ProvidesFile{childdoc.def}[2018/12/30 v2.0 child document driver]
%<samplemain>\ProvidesFile{cdocsamp.tex}[2018/12/30 v2.0 sample for childdoc]
%<*driver>
%\ProvidesFile{childdoc.drv}[2018/12/30 v2.0 childdoc reference manual file]
\PassOptionsToClass{10pt,a4paper}{article}
\documentclass{ltxdoc}

\usepackage[margin=35mm]{geometry}
\usepackage{hyperref}
\usepackage{hyperxmp}
\usepackage[usenames]{color}

\hypersetup{colorlinks=true}
\hypersetup{pdfstartview=FitH}
\hypersetup{pdfpagemode=UseNone}
\hypersetup{pdfsource={}}
\hypersetup{pdflang={en-UK}}
\hypersetup{pdfcopyright={Copyright 2017-2018 Niklas Beisert.
  This work may be distributed and/or modified under the
  conditions of the LaTeX Project Public License, either version 1.3
  of this license or (at your option) any later version.}}
\hypersetup{pdflicenseurl={http://www.latex-project.org/lppl.txt}}
\hypersetup{pdfcontactaddress={ETH Zurich, ITP, HIT K,
  Wolfgang-Pauli-Strasse 27}}
\hypersetup{pdfcontactpostcode={8093}}
\hypersetup{pdfcontactcity={Zurich}}
\hypersetup{pdfcontactcountry={Switzerland}}
\hypersetup{pdfcontactemail={nbeisert@itp.phys.ethz.ch}}
\hypersetup{pdfcontacturl={http://people.phys.ethz.ch/\xmptilde nbeisert/}}

\newcommand{\secref}[1]{\hyperref[#1]{section \ref*{#1}}}

\parskip1ex
\parindent0pt
\let\olditemize\itemize
\def\itemize{\olditemize\parskip0pt}

\begin{document}

\title{The \textsf{childdoc} Package}
\hypersetup{pdftitle={The childdoc Package}}
\author{Niklas Beisert\\[2ex]
  Institut f\"ur Theoretische Physik\\
  Eidgen\"ossische Technische Hochschule Z\"urich\\
  Wolfgang-Pauli-Strasse 27, 8093 Z\"urich, Switzerland\\[1ex]
  \href{mailto:nbeisert@itp.phys.ethz.ch}
  {\texttt{nbeisert@itp.phys.ethz.ch}}}
\hypersetup{pdfauthor={Niklas Beisert}}
\hypersetup{pdfsubject={Manual for the LaTeX2e Package childdoc}}
\date{30 December 2018, \textsf{v2.0}}
\maketitle

\begin{abstract}\noindent
\textsf{childdoc} is a \LaTeXe{} package
that enables the direct compilation
of document sections included by |\include|
to individual files.
\end{abstract}

\begingroup
\parskip0ex
\tableofcontents
\endgroup

%%%%%%%%%%%%%%%%%%%%%%%%%%%%%%%%%%%%%%%%%%%%%%%%%%%%%%%%%%%%%%%%%%%%%%%%%%%%%%%%
%%%%%%%%%%%%%%%%%%%%%%%%%%%%%%%%%%%%%%%%%%%%%%%%%%%%%%%%%%%%%%%%%%%%%%%%%%%%%%%%
\section{Introduction}

\LaTeX{} provides a mechanism to structure a large document (such as a book)
into a main file and several child files (containing the chapters)
using the |\include| command.
This mechanism is beneficial for documents
which span hundreds of pages in order to
make the source file(s) more manageable.
Moreover, compilation can be restricted to
selected child files by means of the |\includeonly| command.
The latter feature can be used to reduce the compilation time while editing
(this was significantly more useful in the earlier days of \LaTeX{})
or to generate a smaller document which is easier to navigate.
Another application of |\includeonly| is to generate
documents consisting of selected parts of the complete document.

However, there are a few drawbacks of the plain |\include| mechanism:
\begin{itemize}
\item
The child files cannot be compiled on their own,
they can only be compiled via the main file.
A naive editing environment
(such as a text editor with an option
to have the current file processed by \LaTeX)
may require one to switch to the main file before compiling;
attempting to compile the child file produces errors.
\item
The main file must be modified (each time)
to adjust the |\includeonly| command
to the present needs. This easily leaves the main file in a messy state.
\item
The generated document will always carry the filename
of the main document. This is inconvenient if
several child files are to be compiled and
to be kept for distribution.
\end{itemize}

The present package provides a simple interface
to make child files individually compilable by \LaTeX{}.
Compiling a child file then has the same effect as compiling
the main file with an |\includeonly| command
to select the appropriate child.
Moreover the generated document will carry the name of the child
rather than the main file.
This resolves all three above issues.

This feature is meant to make the editing of books,
thesis documents and lecture notes somewhat more convenient.
However, the package can also be used efficiently for
composing a series of documents (such as exercise sheets)
which are typically distributed individually.
It then assists the author in generating the individual documents
(potentially in different versions)
as well as a document containing the collected series.
Another application is in developing style files
or other kinds of included material
where compilation of the style file could redirect
to a sample or test file.

%%%%%%%%%%%%%%%%%%%%%%%%%%%%%%%%%%%%%%%%%%%%%%%%%%%%%%%%%%%%%%%%%%%%%%%%%%%%%%%%
%%%%%%%%%%%%%%%%%%%%%%%%%%%%%%%%%%%%%%%%%%%%%%%%%%%%%%%%%%%%%%%%%%%%%%%%%%%%%%%%
\section{Usage}

First of all, the package \textsf{childdoc} is \emph{not} a standard
\LaTeXe{} |.sty| style file! Therefore it needs to be invoked in
a non-standard way.

%%%%%%%%%%%%%%%%%%%%%%%%%%%%%%%%%%%%%%%%%%%%%%%%%%%%%%%%%%%%%%%%%%%%%%%%%%%%%%%%
\subsection{Included Files}
\label{sec:include}

%%%%%%%%%%%%%%%%%%%%%%%%%%%%%%%%%%%%%%%%
\DescribeMacro{\childdocmain}
To use the package, add the commands
\begin{center}
\begin{tabular}{l}
|\input{childdoc.def}|\\
|\childdocmain{}|\\
\end{tabular}
\end{center}
at the very top of the main \LaTeX{} file,
in particular \emph{before} the |\documentclass| statement!
The argument of |\childdocmain| should be left empty
(but it must be present).

%%%%%%%%%%%%%%%%%%%%%%%%%%%%%%%%%%%%%%%%
\DescribeMacro{\childdocof}
Furthermore, add the commands
\begin{center}
\begin{tabular}{l}
|\input{childdoc.def}|\\
|\childdocof{|\textit{main}|}|\\
\end{tabular}
\end{center}
at the top of every child file \textit{child}
which is included by |\include{|\textit{child}|}|
from within the main file
(or at least for those files to be compiled individually).
The argument \textit{main} must be the filename of the main file.

There are a couple of
considerations in setting up the main and child documents:

%%%%%%%%%%%%%%%%%%%%%%%%%%%%%%%%%%%%%%%%
\paragraph{Restrictions.}

Please note the following restrictions:
\begin{itemize}
\item
|\childdocmain| must be called with one argument \textit{main}
to ensure compatibility with earlier version of the package.
It must either be empty (|\childdocmain{}|)
or precisely match the filename of the main file in which it is specified.
See \secref{sec:detection} for further information.
\item
The filename \textit{main} must be specified without the |.tex| extension.
\item
The filename \textit{main} is case sensitive
(even in case-insensitive file systems)
due to internal string comparison.
\item
The argument \textit{main} should be fully expanded, it cannot be a macro.
\item
Subdirectories and special characters should be avoided in filenames.
\item
The command |\childdocmain{|\textit{main}|}| must be followed by a whitespace.
It should not be followed immediately by another command
or by a comment mark `|%|'.
This is because the \TeX{} parser reads the token immediately following
the argument of |\childdocmain| and puts it
at the beginning of every child section;
however, a white\-space is ignored.
\end{itemize}

%%%%%%%%%%%%%%%%%%%%%%%%%%%%%%%%%%%%%%%%
\paragraph{Content of Main File.}

It is advisable to place all content in the child files included by |\include|.
Any output contained in the main file will appear in all child documents
unless suppressed manually;
it cannot be suppressed automatically by the |\includeonly| directive
and thus should normally be avoided.
A method to include some content in the main file
by means of conditional processing is described in \secref{sec:conditional}.

%%%%%%%%%%%%%%%%%%%%%%%%%%%%%%%%%%%%%%%%
\paragraph{Page Numbering.}

When only a part of the document is compiled,
the appropriate numbering of pages
(as well as other status parameters)
is determined from the |.aux| files.
The latter contain information from previous passes.
However this information needs to propagate through
all intermediate child documents.
Therefore the page numbering in child documents may well
be inconsistent until the complete document is compiled at least once.

A useful (if unconventional) way to always ensure a consistent
page numbering is to restart the numbering in each child document
and denote the pages by `\textit{child}|.|\textit{page}'
where \textit{child} represents the chapter/section number of the child file.
This can be achieved by the command
|\numberwithin{page}{|\textit{child}|}|
of the \textsf{amsmath} package
where \textit{child} can be |chapter| or |section|
depending on the chosen structuring.
Alternatively, one can modify the macro |\thepage| appropriately
and reset the counter |page| at the start of each child file.

%%%%%%%%%%%%%%%%%%%%%%%%%%%%%%%%%%%%%%%%%%%%%%%%%%%%%%%%%%%%%%%%%%%%%%%%%%%%%%%%
\subsection{Conditional Processing}
\label{sec:conditional}

The package provides a mechanism to compile different versions
of a document. To customise the versions further some conditional processing
can come in handy to distinguish which version is being compiled.
The package provides two macros to describe the compilation context:

%%%%%%%%%%%%%%%%%%%%%%%%%%%%%%%%%%%%%%%%
\DescribeMacro{\ifchilddoc}
The conditional |\ifchilddoc| distinguishes between the compilation of
child documents and the main document:
%
\begin{center}
|\ifchilddoc |\textit{child-code}| |[|\||else |\textit{main-code}]| \||fi|
\end{center}

%%%%%%%%%%%%%%%%%%%%%%%%%%%%%%%%%%%%%%%%
\DescribeMacro{\childdocname}
\DescribeMacro{\childdocjob}
The macro |\childdocname| contains the filename (without extension)
of the main or child file being processed.
Note that |\childdocjob| will always contain the name of the main file.

%%%%%%%%%%%%%%%%%%%%%%%%%%%%%%%%%%%%%%%%
\paragraph{Title Page.}

Conditional processing can be used to include a title or banner page
in the main document when proper precautions are taken.
Importantly, the code in the main file should ensure that the page counter
(as well as other status parameters which are stored in the |.aux| files)
takes the same value after the conditional processing.
Otherwise the page numbers may take divergent values
depending on which part is compiled.

For example, a title page could be declared by:
%
\begin{center}
\begin{tabular}{l}
|\ifchilddoc\||else|\\
|\addtocounter{page}{-1}|\\
\textit{code for title page}\\
|\newpage|\\
|\||fi|
\end{tabular}
\end{center}
%
A banner page for the child documents can be generated by:
%
\begin{center}
\begin{tabular}{l}
|\ifchilddoc|\\
|\addtocounter{page}{-1}|\\
\textit{code for banner page}\\
|\newpage|\\
|\||fi|
\end{tabular}
\end{center}
%
Here one could write a message such as:
\begin{center}
|This is the part \childdocname{} of \childdocjob{}.|
\end{center}

%%%%%%%%%%%%%%%%%%%%%%%%%%%%%%%%%%%%%%%%%%%%%%%%%%%%%%%%%%%%%%%%%%%%%%%%%%%%%%%%
\subsection{Flags}
\label{sec:flags}

The package makes it easy to generate different versions
of the main or child documents.
To this end compilation flags can be defined
and assigned different default values.
They will be particularly useful in conjunction
with the forwarding mechanism described in \secref{sec:forward}.

For example, it may be useful to have a flag |\version|
which can be set to |draft| or |final|.
The document source will contain some conditional code
depending on the value of |\version|.
Suppose further, the flag should default to |final| for the main file
and to |draft| for child files
which is a natural assignment for editing the document.
This is achieved by placing the following code
in the preamble of the main document
(below the |\childdocmain| directive):
%
\begin{center}
\begin{tabular}{l}
|\ifchilddoc|\\
|\providecommand{\version}{draft}|\\
|\||else|\\
|\providecommand{\version}{final}|\\
|\||fi|
\end{tabular}
\end{center}
%
The definition by |\providecommand| makes sure
that previous definitions are not overwritten.
Further statements |\providecommand{\version}{...}|
can thus be added before the above code to override it.

For the main file, one might add a line
(between |\childdocmain| and the above block)
%
\begin{center}
|%\ifchilddoc\||else\providecommand{\version}{draft}\||fi|
\end{center}
%
which can be uncommented to produce a draft version.
Likewise one can add a line to the very top of a child file
(above the |\childdocof{|\textit{main}|}| directive)
%
\begin{center}
|%\providecommand{\version}{final}|
\end{center}
%
which can be uncommented to produce the final version of this child document.

%%%%%%%%%%%%%%%%%%%%%%%%%%%%%%%%%%%%%%%%%%%%%%%%%%%%%%%%%%%%%%%%%%%%%%%%%%%%%%%%
\subsection{Forwarding}
\label{sec:forward}

Different versions of the main or child documents
using compilation flags as described in \secref{sec:flags}
can be (permanently) stored in different files
for convenient compilation, viewing and distribution.
To this end, the package defines a command
to pass on compilation to a different file:

%%%%%%%%%%%%%%%%%%%%%%%%%%%%%%%%%%%%%%%%
\DescribeMacro{\childdocforward}
The command |\childdocforward| redirects processing to
another source file:
%
\begin{center}
\begin{tabular}{l}
|\input{childdoc.def}|\\
|\childdocforward[|\textit{main}|]{|\textit{dest}|}|\\
\end{tabular}
\end{center}
%
The argument \textit{dest} is the destination file
(without extension).
It should be the main file or one of the child files.
Note that further \textsf{childdoc} directives
such as |\childdocof| and |\childdocforward|
in the indicated file will be processed in this form.
The optional argument \textit{main}
passes on directly to the main file \textit{main}
while pretending to compile the child \textit{dest}.
This form behaves as if \textit{dest}
issues |\childdocof{|\textit{main}|}| right away,
and no further \textsf{childdoc} directives will be processed.

%%%%%%%%%%%%%%%%%%%%%%%%%%%%%%%%%%%%%%%%
\DescribeMacro{\...prefix}
In the alternative form |\childdocforwardprefix|,
%
\begin{center}
\begin{tabular}{l}
|\input{childdoc.def}|\\
|\childdocforwardprefix[|\textit{main}|]{|\textit{prefix}|}{|\textit{dest}|}|
\end{tabular}
\end{center}
%
the destination file is determined by a pattern
depending on the current file:
To make this work, the current file must be called
`{\textit{prefix}\hspace{0.2em}\textit{suffix}}'
with \textit{prefix} matching precisely the argument.
Processing is then passed on to the file
`{\textit{dest}\hspace{0.2em}\textit{suffix}}'.
Surely, the same effect is achieved by
directly specifying the
argument `{\textit{dest}\hspace{0.2em}\textit{suffix}}'
in the first form.
However, that requires to set up a different file
for each child. With the alternative form of the command
all these files can have exactly the same content
which simplifies setting them up and maintaining them.

For example, the following file |draft.tex|
with a compilation flag |\version| as described in \secref{sec:flags}
compiles the main document as a draft:
%
\begin{center}
\begin{tabular}{l}
|\def\version{draft}|\\
|\input{childdoc.def}|\\
|\childdocforward{|\textit{main}|}|
\end{tabular}
\end{center}
%
Likewise, the following files |final|\textit{nn}|.tex|
compile the final version of the child document
|child|\textit{nn}|.tex|:
%
\begin{center}
\begin{tabular}{l}
|\def\version{final}|\\
|\input{childdoc.def}|\\
|\childdocforwardprefix{final}{child}|
\end{tabular}
\end{center}
%

Note that when several versions of a main file and/or of each child file
are to be generated, it may be convenient to set up a |Makefile| or
shell script to automatise the process.

%%%%%%%%%%%%%%%%%%%%%%%%%%%%%%%%%%%%%%%%%%%%%%%%%%%%%%%%%%%%%%%%%%%%%%%%%%%%%%%%
\subsection{Command Line Processing}
\label{sec:commandline}

The effect of redirection files can also be achieved by invoking
the \LaTeX{} compiler with a more elaborate command line.
Most conveniently this should be done as part
of a shell script or a |Makefile|.

When using \textsf{childdoc} in the main file, the following
command lines effectively perform a redirection
(note that depending on the shell being used,
backslashes may have to be doubled: `|\|' $\to$ `|\\|'):
%
\begin{center}
|... -jobname "|\textit{target}|" |\\|"|[\textit{flags}]%
|\input{childdoc.def}\childdocforward[|\textit{main}|]{|\textit{dest}|}"|
\end{center}
%
Here \textit{target} is the name of the output file,
\textit{main} is the name of the main file
and \textit{dest} is the name of the main or child file to be processed
(all filenames without extensions).
The optional argument \textit{main} can be omitted
if \textit{main} matches \textit{dest}.
Optionally, compilation \textit{flags} can be defined via |\def| commands.
This command line makes the \TeX{} engine believe
it is compiling the file \textit{target}
whose content is specified as the latter parameter.
The provided code then forwards the processing to
\textit{main} or \textit{dest} as described in \secref{sec:forward}.

%%%%%%%%%%%%%%%%%%%%%%%%%%%%%%%%%%%%%%%%%%%%%%%%%%%%%%%%%%%%%%%%%%%%%%%%%%%%%%%%
\subsection{Include by Input}
\label{sec:input}

Including child documents by |\include| has some restrictions by design.
Most notably, the content of a child document always occupies
its own set of pages; pages cannot be shared between child documents.
Usually, this behaviour makes perfect sense
because each child document contain an essential part of the document.
However, in some situations it may be desirable to compose
a document from a collection of parts
without having mandatory page breaks between then.
For this case, the package
provides a mechanism to include parts
by |\input| which can also be processed individually.
However, by construction this mechanism
requires manual handling of the content to be output.

%%%%%%%%%%%%%%%%%%%%%%%%%%%%%%%%%%%%%%%%
\DescribeMacro{\ifchilddocmanual}
The main file should be prepared as usual, see \secref{sec:include}.
However, the document body must make a distinction
between processing of an individual part and of the main document, e.g.:
%
\begin{center}
\begin{tabular}{l}
|\ifchilddocmanual|\\
|\input{\childdocname}|\\
|\||else|\\
\textit{document body with }|\input{|\textit{part}|}|\\
|\||fi|
\end{tabular}
\end{center}
%
The conditional |\ifchilddocmanual| is true whenever
a part to be included by |\input| is being compiled,
and the name of the part is stored in |\childdocname|.

%%%%%%%%%%%%%%%%%%%%%%%%%%%%%%%%%%%%%%%%
\DescribeMacro{\childdocby}
Each part to be included by |\input| should start with:
%
\begin{center}
\begin{tabular}{l}
|\input{childdoc.def}|\\
|\childdocby{|\textit{main}|}|\\
\end{tabular}
\end{center}
%
The directive |\childdocby| is similar to |\childdocof|
described in \secref{sec:include},
but the subsequent selection of content must be done manually.
To that end, both |\ifchilddoc| and |\ifchilddocmanual|
will be true upon processing of a part,
and the name of the part is stored in |\childdocname|.
Note that |\jobname| will be set to the filename of the current part
so that each part receives an individual |.aux| file
that does not interfere with the |.aux| file(s) of the main document.
This behaviour can be altered by the alternative form
|\childdocby[*]{|\textit{main}|}| (with a non-empty optional argument)
which uses the |.aux| file of the main document
by setting |\jobname| to \textit{main}.

%%%%%%%%%%%%%%%%%%%%%%%%%%%%%%%%%%%%%%%%%%%%%%%%%%%%%%%%%%%%%%%%%%%%%%%%%%%%%%%%
\subsection{Driver Development}
\label{sec:driver}

The \textsf{childdoc} mechanism can also be use for the development
of definition files such as \LaTeX{} styles or classes.
This case differs from the above setup with multiple parts
included by |\include| in that no |\includeonly| should be invoked.
This can be achieved by starting the include file
(before |\ProvidesPackage|) with:
%
\begin{center}
\begin{tabular}{l}
|\input{childdoc.def}|\\
|\childdocforward{|\textit{main}|}|\\
\end{tabular}
\end{center}
%
or alternatively with:
%
\begin{center}
\begin{tabular}{l}
|\input{childdoc.def}|\\
|\childdocby{|\textit{main}|}|\\
\end{tabular}
\end{center}
%
Both forms have slightly different effects as described above.
The main file is prepared as usual, see \secref{sec:include}.

%%%%%%%%%%%%%%%%%%%%%%%%%%%%%%%%%%%%%%%%%%%%%%%%%%%%%%%%%%%%%%%%%%%%%%%%%%%%%%%%
\subsection{Legacy Detection}
\label{sec:detection}

The directive |\childdocmain| in the main file can detect
whether the complete document or merely a child is to be compiled
even without using the directive |\childdocof|.
This method is deprecated because it is less robust
and there is no compelling reason to use it;
it is merely provided for backward compatibility
and it may be removed in future versions.

If the detection mechanism is to be used,
it is mandatory to correctly specify
the filename of the main file as the argument of |\childdocmain|:
%
\begin{center}
\begin{tabular}{l}
|\input{childdoc.def}|\\
|\childdocmain{|\textit{main}|}|\\
\end{tabular}
\end{center}
%
If |\jobname| does not match the argument \textit{main} of |\childdocmain|,
it is assumed that |\jobname| points to the child file to be compiled.
When using |\childdocmain| with the main file specified as argument,
it suffices to start a child file
with just |\input{|\textit{main}|}|
without loading of the package and using |\childdocof|.
If instead all processing is done
with the appropriate \textsf{childdoc} directives,
the argument of \textit{main} of |\childdocmain| can be empty.

An alternative version of the command line processing described
in \secref{sec:commandline} using the detection mechanism reads:
%
\begin{center}
|... -jobname "|\textit{target}|" "|[\textit{flags}]%
[|\def\jobname{|\textit{dest}|}|]|\input{|\textit{main}|}"|
\end{center}

%%%%%%%%%%%%%%%%%%%%%%%%%%%%%%%%%%%%%%%%%%%%%%%%%%%%%%%%%%%%%%%%%%%%%%%%%%%%%%%%
\subsection{Manual Code}
\label{sec:manual}

In case one cannot be certain whether the definitions file |childdoc.def|
is installed on the target \TeX{} distribution
and one prefers not to ship it,
it is conceivable to paste a few relevant commands into the sources.

To that end, drop all statements |\input{childdoc.def}|
and perform the replacements as outlined below.
Instead of |\childdocmain{|\textit{main}|}| add the following code
to the top of the main file:
%
\begin{center}
\begin{tabular}{l}
|\||ifdefined\childdocname\endinput\||fi\newif\ifchilddoc|\\
|\edef\childdocname{\scantokens\expandafter{\jobname\noexpand}}|\\
|\def\childdocmain{|\textit{main}|}\||ifx\childdocmain\childdocname\||else|\\
|\childdoctrue\includeonly{\childdocname}\let\jobname\childdocmain\||fi|\\
\end{tabular}
\end{center}
%
Instead of |\childdocof{|\textit{main}|}| just include the main file
at the top of each child file:
%
\begin{center}
|\input{|\textit{main}|}|
\end{center}
%
A simple redirection |\childdocforward{|\textit{dest}|}| is achieved by:
%
\begin{center}
|\def\jobname{|\textit{dest}|}\input{\jobname}|
\end{center}
%
The redirection with prefix
|\childdocforwardprefix[|\textit{prefix}|]{|\textit{dest}|}|
is accomplished by:
%
\begin{center}
\begin{tabular}{l}
|{\edef\jobname{\scantokens\expandafter{\jobname\noexpand}}|\\
|\def\redirectjob |\textit{prefix}|#1~~~{\gdef\jobname{|\textit{dest}|#1}}|\\
|\expandafter\redirectjob\jobname~~~}\input{\jobname}|
\end{tabular}
\end{center}

In an alternative approach,
child documents can be compiled by a specific command line
without additional code or specific definitions:
%
\begin{center}
|... -jobname "|\textit{target}|" "|[\textit{flags}]%
|\includeonly{|\textit{dest}|}\input{|\textit{main}|}"|
\end{center}
%

%%%%%%%%%%%%%%%%%%%%%%%%%%%%%%%%%%%%%%%%%%%%%%%%%%%%%%%%%%%%%%%%%%%%%%%%%%%%%%%%
%%%%%%%%%%%%%%%%%%%%%%%%%%%%%%%%%%%%%%%%%%%%%%%%%%%%%%%%%%%%%%%%%%%%%%%%%%%%%%%%
\section{Information}

%%%%%%%%%%%%%%%%%%%%%%%%%%%%%%%%%%%%%%%%%%%%%%%%%%%%%%%%%%%%%%%%%%%%%%%%%%%%%%%%
\subsection{Copyright}

Copyright \copyright{} 2017--2018 Niklas Beisert

This work may be distributed and/or modified under the
conditions of the \LaTeX{} Project Public License, either version 1.3
of this license or (at your option) any later version.
The latest version of this license is in
  \url{http://www.latex-project.org/lppl.txt}
and version 1.3 or later is part of all distributions of \LaTeX{}
version 2005/12/01 or later.

This work has the LPPL maintenance status `maintained'.

The Current Maintainer of this work is Niklas Beisert.

This work consists of the files |README.txt|, |childdoc.ins| and |childdoc.dtx|
as well as the derived files |childdoc.def|, |cdocsamp.tex|
with |cdocsch1.tex|, |cdocsch2.tex|, |cdocspt3.tex|, |cdocspt4.tex|,
|cdocsdrf.tex|, |cdocsfn1.tex|, |cdocsfn2.tex|
as well as |childdoc.pdf|.

%%%%%%%%%%%%%%%%%%%%%%%%%%%%%%%%%%%%%%%%%%%%%%%%%%%%%%%%%%%%%%%%%%%%%%%%%%%%%%%%
\subsection{Files and Installation}

The package consists of the files:
%
\begin{center}
\begin{tabular}{ll}
    |README.txt|   & readme file \\
    |childdoc.ins| & installation file \\
    |childdoc.dtx| & source file \\
    |childdoc.def| & definition file \\
    |cdocsamp.tex| & sample main file \\
    |cdocsch1.tex| & sample include file \\
    |cdocsch2.tex| & sample include file \\
    |cdocspt3.tex| & sample part file \\
    |cdocspt4.tex| & sample part file \\
    |cdocsdrf.tex| & sample redirection file \\
    |cdocsfn1.tex| & sample redirection file \\
    |cdocsfn2.tex| & sample redirection file \\
    |childdoc.pdf| & manual
\end{tabular}
\end{center}
%
The distribution consists of the files
|README.txt|, |childdoc.ins| and |childdoc.dtx|.
%
\begin{itemize}
\item
Run (pdf)\LaTeX{} on |childdoc.dtx|
to compile the manual |childdoc.pdf| (this file).
\item
Run \LaTeX{} on |childdoc.ins| to create the definitions file |childdoc.def|
and the sample |cdocsamp.tex| with include files
|cdocsch1.tex|, |cdocsch2.tex|, |cdocspt3.tex|, |cdocspt4.tex|,
|cdocsdrf.tex|, |cdocsfn1.tex|, |cdocsfn2.tex|.
Then copy the file |childdoc.def| to an appropriate directory of your \LaTeX{}
distribution, e.g.\ \textit{texmf-root}|/tex/latex/childdoc|.
\end{itemize}

%%%%%%%%%%%%%%%%%%%%%%%%%%%%%%%%%%%%%%%%%%%%%%%%%%%%%%%%%%%%%%%%%%%%%%%%%%%%%%%%
\subsection{Related CTAN Packages}

There are several other packages which offer a similar functionality:
%
\begin{itemize}
\item
The packages
\href{http://ctan.org/pkg/docmute}{\textsf{docmute}},
\href{http://ctan.org/pkg/includex}{\textsf{includex}} and
\href{http://ctan.org/pkg/standalone}{\textsf{standalone}}
provide commands to include only the document body of
a child file thus allowing both files to be compiled individually.
\item
The packages \href{http://ctan.org/pkg/subdocs}{\textsf{subdocs}}
and \href{http://ctan.org/pkg/subfiles}{\textsf{subfiles}}
provide structures in which the main and child documents can be
encapsulated and allowing them to be compiled individually.
The inclusion mechanism is different from the conventional |\include|.
\item
The package \href{http://ctan.org/pkg/combine}{\textsf{combine}}
is an elaborate solution to combine several documents into one.
\end{itemize}
%
See also the CTAN topic \href{http://ctan.org/topic/subdocs}{\textsf{subdocs}}
for further related packages.
The present package differs from the above solutions in that
a document structure constructed with the conventional |\include| mechanism
just needs two extra commands at the top of every file
such that all constituent files can be compiled individually.

%%%%%%%%%%%%%%%%%%%%%%%%%%%%%%%%%%%%%%%%%%%%%%%%%%%%%%%%%%%%%%%%%%%%%%%%%%%%%%%%
%\subsection{Feature Suggestions}
%
%The following is a list of features which may be useful for future
%versions of this package:
%%
%\begin{itemize}
%\item
%\ldots
%\end{itemize}

%%%%%%%%%%%%%%%%%%%%%%%%%%%%%%%%%%%%%%%%%%%%%%%%%%%%%%%%%%%%%%%%%%%%%%%%%%%%%%%%
\subsection{Revision History}

%%%%%%%%%%%%%%%%%%%%%%%%%%%%%%%%%%%%%%%%
\paragraph{v2.0:} 2018/12/30

\begin{itemize}
\item
immediate forward processing
\item
added |\childdocby| mechanism
\item
manual restructured
\end{itemize}

%%%%%%%%%%%%%%%%%%%%%%%%%%%%%%%%%%%%%%%%
\paragraph{v1.6:} 2018/01/17

\begin{itemize}
\item
application for development of include files
\item
corrections to manual
\end{itemize}

%%%%%%%%%%%%%%%%%%%%%%%%%%%%%%%%%%%%%%%%
\paragraph{v1.5:} 2017/05/21

\begin{itemize}
\item
more complete structuring introduced
\item
|\childdocof| introduced
\item
|\childdoc| renamed to |\childdocmain|
\item
|\childredirect| renamed to |\childdocforward| and |\childdocforwardprefix|
and functionality expanded
\end{itemize}

%%%%%%%%%%%%%%%%%%%%%%%%%%%%%%%%%%%%%%%%
\paragraph{v1.0:} 2017/04/27

\begin{itemize}
\item
manual and install package
\item
first version published on CTAN
\end{itemize}

%%%%%%%%%%%%%%%%%%%%%%%%%%%%%%%%%%%%%%%%
\paragraph{v0.6:} 2017/04/26

\begin{itemize}
\item
redirection mechanism added
\end{itemize}

%%%%%%%%%%%%%%%%%%%%%%%%%%%%%%%%%%%%%%%%
\paragraph{v0.5:} 2017/04/26

\begin{itemize}
\item
functionality in definition file
\end{itemize}


%%%%%%%%%%%%%%%%%%%%%%%%%%%%%%%%%%%%%%%%%%%%%%%%%%%%%%%%%%%%%%%%%%%%%%%%%%%%%%%%
%%%%%%%%%%%%%%%%%%%%%%%%%%%%%%%%%%%%%%%%%%%%%%%%%%%%%%%%%%%%%%%%%%%%%%%%%%%%%%%%
%%%%%%%%%%%%%%%%%%%%%%%%%%%%%%%%%%%%%%%%%%%%%%%%%%%%%%%%%%%%%%%%%%%%%%%%%%%%%%%%
\appendix

\settowidth\MacroIndent{\rmfamily\scriptsize 000\ }

 \DocInput{childdoc.dtx}

\end{document}
%</driver>
% \fi
%
% %%%%%%%%%%%%%%%%%%%%%%%%%%%%%%%%%%%%%%%%%%%%%%%%%%%%%%%%%%%%%%%%%%%%%%%%%%%%%%
% %%%%%%%%%%%%%%%%%%%%%%%%%%%%%%%%%%%%%%%%%%%%%%%%%%%%%%%%%%%%%%%%%%%%%%%%%%%%%%
% \section{Sample}
%\iffalse
%<*samplemain>
%\fi
%
% The following presents a sample document
% with two chapters, two parts, a title page,
% a compile flag as well as three forwarding files to set the flag.
% It consists of eight |.tex| files:
% \begin{center}
% \begin{tabular}{ll}
% |cdocsamp.tex|&main file\\
% |cdocsch1.tex|&include file for chapter 1\\
% |cdocsch2.tex|&include file for chapter 2\\
% |cdocspt3.tex|&include file for part 3\\
% |cdocspt4.tex|&include file for part 4\\
% |cdocsdrf.tex|&forwarding file for main file in draft mode\\
% |cdocsfi1.tex|&forwarding file for final version of chapter 1\\
% |cdocsfi2.tex|&forwarding file for final version of chapter 2\\
% \end{tabular}
% \end{center}
% Each of the eight files can be compiled directly by the \LaTeX{} compiler.
%
% %%%%%%%%%%%%%%%%%%%%%%%%%%%%%%%%%%%%%%
% \paragraph{Main File.}
%
% The main file is called |cdocsamp.tex|.
%
% Load the \textsf{childdoc} definitions and
% declare the filename for the main document:
%    \begin{macrocode}
\input{childdoc.def}
\childdocmain{}
%    \end{macrocode}

% Optional override for |\version| flag:
%    \begin{macrocode}
%%\ifchilddoc\else\providecommand{\version}{draft}\fi
%    \end{macrocode}

% Define the default values for the |\version| flag
% (|final| for the main file and |draft| for childs):
%    \begin{macrocode}
\ifchilddoc
\providecommand{\version}{draft}
\else
\providecommand{\version}{final}
\fi
%    \end{macrocode}

% Load the standard document class:
%    \begin{macrocode}
\documentclass[12pt]{article}
%    \end{macrocode}

% Start the document body:
%    \begin{macrocode}
\begin{document}
%    \end{macrocode}

% Declare a title page.
% Print title, part of document being processed and version flag:
%    \begin{macrocode}
\addtocounter{page}{-1}
\begin{center}
{\LARGE\bfseries{}childdoc example\par}
\vspace{1cm}
\ifchilddoc
\ifchilddocmanual part\else chapter\fi:
`\childdocname' of `\childdocjob'\par
\else
main document: `\childdocjob'\par
\fi
version: \version\par
\end{center}
\newpage
%    \end{macrocode}

% Manually include selected file,
% otherwise process as usual:
%    \begin{macrocode}
\ifchilddocmanual
\section*{part `\childdocname'}
\input{\childdocname}
\else
%    \end{macrocode}

% Include the two chapters:
%    \begin{macrocode}
\include{cdocsch1}
\include{cdocsch2}
%    \end{macrocode}

% Include the two parts unless only chapters should be displayed:
%    \begin{macrocode}
\ifchilddoc\else
\section{part three}
\input{cdocspt3}
\section{part four}
\input{cdocspt4}
\fi
%    \end{macrocode}

% Process as usual until here:
%    \begin{macrocode}
\fi
%    \end{macrocode}

% End of document body:
%    \begin{macrocode}
\end{document}
%    \end{macrocode}
%\iffalse
%</samplemain>
%\fi
%
% %%%%%%%%%%%%%%%%%%%%%%%%%%%%%%%%%%%%%%
% \paragraph{Chapter Include Files.}
%
% The include files are called |cdocsch1.tex| and |cdocsch2.tex|.
%
%\iffalse
%<*samplechap1|samplechap2>
%\fi

% Optional override for |\version| flag:
%    \begin{macrocode}
%%\providecommand{\version}{final}
%    \end{macrocode}

% Include the main document:
%    \begin{macrocode}
\input{childdoc.def}
\childdocof{cdocsamp}
%    \end{macrocode}

%\iffalse
%</samplechap1|samplechap2>
%\fi
%
%\iffalse
%<*samplechap1>
%\fi
% Some text for chapter 1:
%    \begin{macrocode}
\section{one}
some text in chapter one
%    \end{macrocode}

%\iffalse
%</samplechap1>
%\fi
% Some text for chapter 2:
%\iffalse
%<*samplechap2>
%\fi
%    \begin{macrocode}
\section{two}
more text in chapter two
%    \end{macrocode}

%\iffalse
%</samplechap2>
%\fi
%
% %%%%%%%%%%%%%%%%%%%%%%%%%%%%%%%%%%%%%%
% \paragraph{Part Include Files.}
%
% The include files are called |cdocspt3.tex| and |cdocspt4.tex|.
%
%\iffalse
%<*samplepart3|samplepart4>
%\fi

% Optional override for |\version| flag:
%    \begin{macrocode}
%%\providecommand{\version}{final}
%    \end{macrocode}

% Include the main document:
%    \begin{macrocode}
\input{childdoc.def}
\childdocby{cdocsamp}
%    \end{macrocode}

%\iffalse
%</samplepart3|samplepart4>
%\fi
%
%\iffalse
%<*samplepart3>
%\fi
% Some text for part 3:
%    \begin{macrocode}
some text in part three
%    \end{macrocode}

%\iffalse
%</samplepart3>
%\fi
% Some text for part 4:
%\iffalse
%<*samplepart4>
%\fi
%    \begin{macrocode}
more text in part four
%    \end{macrocode}

%\iffalse
%</samplepart4>
%\fi
%
% %%%%%%%%%%%%%%%%%%%%%%%%%%%%%%%%%%%%%%
% \paragraph{Forwarding for a Complete Draft.}
%
% The following forwarding file |cdocsdrf.tex|
% compiles the main document in draft mode:
%\iffalse
%<*sampledraft>
%\fi
%    \begin{macrocode}
\def\version{draft}
\input{childdoc.def}
\childdocforward{cdocsamp}
%    \end{macrocode}

%\iffalse
%</sampledraft>
%\fi
%
% %%%%%%%%%%%%%%%%%%%%%%%%%%%%%%%%%%%%%%
% \paragraph{Forwarding for Final Version of the Chapters.}
%
% The following forwarding files |cdocsfn1.tex| and |cdocsfn2.tex|
% (with identical content)
% compile the final versions of the child documents
% |cdocsch1.tex| and |cdocsch2.tex|, respectively:
%\iffalse
%<*samplefinal>
%\fi
%    \begin{macrocode}
\def\version{final}
\input{childdoc.def}
\childdocforwardprefix[cdocsamp]{cdocsfn}{cdocsch}
%    \end{macrocode}

%\iffalse
%</samplefinal>
%\fi
%
% %%%%%%%%%%%%%%%%%%%%%%%%%%%%%%%%%%%%%%
% \paragraph{Command Line Processing.}
%
% The following three command lines generate the output files
% |cdocscld|, |cdocscl1| and |cdocscl2|
% which should be identical to
% |cdocsdrf|, |cdocsch1| and |cdocsfn2|, respectively:
% \begin{center}
% \begin{tabular}{l}
% |latex -jobname cdocscld \|\\
% |  "\def\version{draft}\input{childdoc.def}\childdocforward{cdocsamp}"|\\
% |latex -jobname cdocscl1 \|\\
% |  "\input{childdoc.def}\childdocforward[cdocsamp]{cdocsch1}"|\\
% |latex -jobname cdocscl2 \|\\
% |  "\def\version{final}\input{childdoc.def}\childdocforward{cdocsch2}"|
% \end{tabular}
% \end{center}
% Note that the trailing backslash on each first line
% merely continues the input to the second line
% (for convenient cut ant paste).
% Furthermore, the command |latex| can be replaced by any
% of its alternative versions such as |pdflatex|.
%
% %%%%%%%%%%%%%%%%%%%%%%%%%%%%%%%%%%%%%%%%%%%%%%%%%%%%%%%%%%%%%%%%%%%%%%%%%%%%%%
% %%%%%%%%%%%%%%%%%%%%%%%%%%%%%%%%%%%%%%%%%%%%%%%%%%%%%%%%%%%%%%%%%%%%%%%%%%%%%%
% \section{Implementation}
%\iffalse
%<*package>
%\fi
%
% This section describes the definitions file |childdoc.def|.

% The definitions cannot be loaded using |\usepackage| or |\RequirePackage|
% which has a mechanism to prevent loading a style file more than once.
% When loading the definitions by means of |\input|
% multiple instances have to be prevented manually:
%\iffalse
%This code needs to be before the `\ProvidesFile' directive
%which is defined at the beginning of this file.
%Therefore it is also placed there and commented out here.
%</package>
%<*discard>
%\fi
%    \begin{macrocode}
\ifdefined\childdocmain\endinput\fi
%    \end{macrocode}
%\iffalse
%</discard>
%<*package>
%\fi
%
% \macro{\ifchilddoc}
% \macro{\ifchilddocmanual}
% The conditional |\ifchilddoc| tells whether a
% child (true) or main (false) document is being compiled.
% The conditional |\ifchilddocmanual| tells whether
% the |\includeonly| mechanism is used (false) or
% the selection of child files must be performed manually (true).
% The definitions initialise to false:
%    \begin{macrocode}
\newif\ifchilddoc
\newif\ifchilddocmanual
%    \end{macrocode}

% \macro{\childdocname}
% \macro{\childdocjob}
% The macro |\childdocname| stores the name of the main document
% to be compiled. The macro |\childdocjob| stores the name of
% the document on which the \LaTeX{} compiler was originally invoked.
% The content of |\jobname| cannot be compared
% to filenames specified in the source due to different catcodes.
% The following code rescans |\jobname|, stores the result
% in |\childdocname| and saves a copy in |\childdocjob|:
%    \begin{macrocode}
\edef\childdocname{\scantokens\expandafter{\jobname\noexpand}}
\let\childdocjob\childdocname
%    \end{macrocode}

% \macro{\childdocdisable}
% The macro |\childdocdisable| prevents the main file
% from being processed more than once.
% At this stage, the main document command |\childdocmain|
% is assumed to be called once again where it should do nothing.
% Any subsequent call to it should prevent
% a secondary processing of the main document
% It overwrites the forwarding commands
% |\childdocof| and |\childdocforward|
% with empty macros to prevent further inclusions of the main document:
%    \begin{macrocode}
\newcommand{\childdocdisable}
{
  \renewcommand{\childdocmain}[1]{\renewcommand{\childdocmain}[1]{\endinput}}
  \renewcommand{\childdocof}[1]{}
  \renewcommand{\childdocby}[2][]{}
  \renewcommand{\childdocforward}[2][]{}
  \renewcommand{\childdocdisable}{}
}
%    \end{macrocode}

% \macro{\childdocmain}
% The macro |\childdocmain| is to be called at the top of the main file
% with nothing or the main filename (without extension) as argument.
% First, it breaks loops.
% If the argument is not empty and does not match |\childdocname|
% (which is set by the first inclusion of |childdoc.def|),
% |\ifchilddoc| is set to true, |\includeonly| is applied to the child file
% and |\jobname| is set to the main file
% (for proper handling of |.aux| files):
%    \begin{macrocode}
\newcommand{\childdocmain}[1]
{
  \childdocdisable\childdocmain{}
  \if?#1?\else
    \begingroup
      \def\childdoctmp{#1}
      \ifx\childdoctmp\childdocname
        \def\childdoctmp{}
      \else
        \def\childdoctmp
        {
          \childdoctrue
          \includeonly{\childdocname}
          \def\childdocjob{#1}
          \def\jobname{#1}
        }
      \fi
      \expandafter
    \endgroup
    \childdoctmp
  \fi
}
%    \end{macrocode}

% \macro{\childdocof}
% The command |\childdocof| redirects
% compilation to the main file |#1|.
%    \begin{macrocode}
\newcommand{\childdocof}[1]
{
  \childdocdisable
  \childdoctrue
  \includeonly{\childdocname}
  \def\jobname{#1}
  \def\childdocjob{#1}
  \input{#1}
}
%    \end{macrocode}

% \macro{\childdocby}
% The command |\childdocby| ....
%    \begin{macrocode}
\newcommand{\childdocby}[2][]
{
  \childdocdisable
  \childdoctrue
  \childdocmanualtrue
  \if?#1?\else
    \def\jobname{#2}
  \fi
  \def\childdocjob{#2}
  \input{#2}
  \endinput
}
%    \end{macrocode}

% \macro{\childdocforward}
% The command |\childdocforward| redirects
% compilation to the main file or
% (if the optional argument is given) a child file.
% Parameters are set as if the main file
% or a child file starting with |\childdocof| was compiled.
% Then compilation is handed over to the main file:
%    \begin{macrocode}
\newcommand{\childdocforward}[2][]
{
  \begingroup
    \if?#1?
      \def\childdoctmp
      {
        \def\childdocname{#2}
        \def\childdocjob{#2}
        \def\jobname{#2}
        \input{#2}
        \endinput
      }
    \else
      \def\childdoctmp
      {
        \childdocdisable
        \def\childdocname{#2}
        \childdoctrue
        \includeonly{#2}
        \def\childdocjob{#1}
        \def\jobname{#1}
        \input{#1}
        \endinput
      }
    \fi
    \expandafter
  \endgroup
  \childdoctmp
}
%    \end{macrocode}

% \macro{\childdocforwardprefix}
% The command |\childdocforwardprefix| redirects
% compilation to the main or a child file by means of a pattern.
% The prefix |#1| in the current filename is replaced by |#2|
% and the suffix of the current filename is kept
% (it is assumed that the filename does not contain the substring `|~~~|'
% which is used as a delimiter).
% Compilation is handed over to the new file by |\childdocforward|:
%    \begin{macrocode}
\newcommand{\childdocforwardprefix}[3][]
{
  \begingroup
    \def\childdocextract #2##1~~~{\def\childdoctmp{\childdocforward[#1]{#3##1}}}
    \expandafter\childdocextract\childdocname~~~
    \expandafter
  \endgroup
  \childdoctmp
}
%    \end{macrocode}

% \macro{\childdoc}
% The deprecated macro |\childdoc| is a legacy version of |\childdocmain|:
%    \begin{macrocode}
\newcommand{\childdoc}{\childdocmain}
%    \end{macrocode}

% \macro{\childdocredirect}
% The deprecated macro |\childdocredirect| is a legacy version
% of |\childdocforward| and |\childdocforwardprefix|:
%    \begin{macrocode}
\newcommand{\childdocredirect}[2][]
{
  \begingroup
    \if?#1?
      \def\childdoctmp{\childdocforward{#2}}
    \else
      \def\childdoctmp{\childdocforwardprefix{#1}{#2}}
    \fi
    \expandafter
  \endgroup
  \childdoctmp
}
%    \end{macrocode}

%\iffalse
%</package>
%\fi
%
\endinput
|\\
|\childdocforwardprefix{final}{child}|
\end{tabular}
\end{center}
%

Note that when several versions of a main file and/or of each child file
are to be generated, it may be convenient to set up a |Makefile| or
shell script to automatise the process.

%%%%%%%%%%%%%%%%%%%%%%%%%%%%%%%%%%%%%%%%%%%%%%%%%%%%%%%%%%%%%%%%%%%%%%%%%%%%%%%%
\subsection{Command Line Processing}
\label{sec:commandline}

The effect of redirection files can also be achieved by invoking
the \LaTeX{} compiler with a more elaborate command line.
Most conveniently this should be done as part
of a shell script or a |Makefile|.

When using \textsf{childdoc} in the main file, the following
command lines effectively perform a redirection
(note that depending on the shell being used,
backslashes may have to be doubled: `|\|' $\to$ `|\\|'):
%
\begin{center}
|... -jobname "|\textit{target}|" |\\|"|[\textit{flags}]%
|% \iffalse
%
% childdoc.dtx Copyright (C) 2017-2018 Niklas Beisert
%
% This work may be distributed and/or modified under the
% conditions of the LaTeX Project Public License, either version 1.3
% of this license or (at your option) any later version.
% The latest version of this license is in
%   http://www.latex-project.org/lppl.txt
% and version 1.3 or later is part of all distributions of LaTeX
% version 2005/12/01 or later.
%
% This work has the LPPL maintenance status `maintained'.
%
% The Current Maintainer of this work is Niklas Beisert.
%
% This work consists of the files childdoc.dtx and childdoc.ins
% and the derived files childdoc.def and cdocsamp.tex with
% cdocsch1.tex, cdocsch2.tex, cdocsdrf.tex, cdocsfn1.tex, cdocsfn2.tex.
%
%<package>\ifdefined\childdocmain\endinput\fi
%<package>\ProvidesFile{childdoc.def}[2018/12/30 v2.0 child document driver]
%<samplemain>\ProvidesFile{cdocsamp.tex}[2018/12/30 v2.0 sample for childdoc]
%<*driver>
%\ProvidesFile{childdoc.drv}[2018/12/30 v2.0 childdoc reference manual file]
\PassOptionsToClass{10pt,a4paper}{article}
\documentclass{ltxdoc}

\usepackage[margin=35mm]{geometry}
\usepackage{hyperref}
\usepackage{hyperxmp}
\usepackage[usenames]{color}

\hypersetup{colorlinks=true}
\hypersetup{pdfstartview=FitH}
\hypersetup{pdfpagemode=UseNone}
\hypersetup{pdfsource={}}
\hypersetup{pdflang={en-UK}}
\hypersetup{pdfcopyright={Copyright 2017-2018 Niklas Beisert.
  This work may be distributed and/or modified under the
  conditions of the LaTeX Project Public License, either version 1.3
  of this license or (at your option) any later version.}}
\hypersetup{pdflicenseurl={http://www.latex-project.org/lppl.txt}}
\hypersetup{pdfcontactaddress={ETH Zurich, ITP, HIT K,
  Wolfgang-Pauli-Strasse 27}}
\hypersetup{pdfcontactpostcode={8093}}
\hypersetup{pdfcontactcity={Zurich}}
\hypersetup{pdfcontactcountry={Switzerland}}
\hypersetup{pdfcontactemail={nbeisert@itp.phys.ethz.ch}}
\hypersetup{pdfcontacturl={http://people.phys.ethz.ch/\xmptilde nbeisert/}}

\newcommand{\secref}[1]{\hyperref[#1]{section \ref*{#1}}}

\parskip1ex
\parindent0pt
\let\olditemize\itemize
\def\itemize{\olditemize\parskip0pt}

\begin{document}

\title{The \textsf{childdoc} Package}
\hypersetup{pdftitle={The childdoc Package}}
\author{Niklas Beisert\\[2ex]
  Institut f\"ur Theoretische Physik\\
  Eidgen\"ossische Technische Hochschule Z\"urich\\
  Wolfgang-Pauli-Strasse 27, 8093 Z\"urich, Switzerland\\[1ex]
  \href{mailto:nbeisert@itp.phys.ethz.ch}
  {\texttt{nbeisert@itp.phys.ethz.ch}}}
\hypersetup{pdfauthor={Niklas Beisert}}
\hypersetup{pdfsubject={Manual for the LaTeX2e Package childdoc}}
\date{30 December 2018, \textsf{v2.0}}
\maketitle

\begin{abstract}\noindent
\textsf{childdoc} is a \LaTeXe{} package
that enables the direct compilation
of document sections included by |\include|
to individual files.
\end{abstract}

\begingroup
\parskip0ex
\tableofcontents
\endgroup

%%%%%%%%%%%%%%%%%%%%%%%%%%%%%%%%%%%%%%%%%%%%%%%%%%%%%%%%%%%%%%%%%%%%%%%%%%%%%%%%
%%%%%%%%%%%%%%%%%%%%%%%%%%%%%%%%%%%%%%%%%%%%%%%%%%%%%%%%%%%%%%%%%%%%%%%%%%%%%%%%
\section{Introduction}

\LaTeX{} provides a mechanism to structure a large document (such as a book)
into a main file and several child files (containing the chapters)
using the |\include| command.
This mechanism is beneficial for documents
which span hundreds of pages in order to
make the source file(s) more manageable.
Moreover, compilation can be restricted to
selected child files by means of the |\includeonly| command.
The latter feature can be used to reduce the compilation time while editing
(this was significantly more useful in the earlier days of \LaTeX{})
or to generate a smaller document which is easier to navigate.
Another application of |\includeonly| is to generate
documents consisting of selected parts of the complete document.

However, there are a few drawbacks of the plain |\include| mechanism:
\begin{itemize}
\item
The child files cannot be compiled on their own,
they can only be compiled via the main file.
A naive editing environment
(such as a text editor with an option
to have the current file processed by \LaTeX)
may require one to switch to the main file before compiling;
attempting to compile the child file produces errors.
\item
The main file must be modified (each time)
to adjust the |\includeonly| command
to the present needs. This easily leaves the main file in a messy state.
\item
The generated document will always carry the filename
of the main document. This is inconvenient if
several child files are to be compiled and
to be kept for distribution.
\end{itemize}

The present package provides a simple interface
to make child files individually compilable by \LaTeX{}.
Compiling a child file then has the same effect as compiling
the main file with an |\includeonly| command
to select the appropriate child.
Moreover the generated document will carry the name of the child
rather than the main file.
This resolves all three above issues.

This feature is meant to make the editing of books,
thesis documents and lecture notes somewhat more convenient.
However, the package can also be used efficiently for
composing a series of documents (such as exercise sheets)
which are typically distributed individually.
It then assists the author in generating the individual documents
(potentially in different versions)
as well as a document containing the collected series.
Another application is in developing style files
or other kinds of included material
where compilation of the style file could redirect
to a sample or test file.

%%%%%%%%%%%%%%%%%%%%%%%%%%%%%%%%%%%%%%%%%%%%%%%%%%%%%%%%%%%%%%%%%%%%%%%%%%%%%%%%
%%%%%%%%%%%%%%%%%%%%%%%%%%%%%%%%%%%%%%%%%%%%%%%%%%%%%%%%%%%%%%%%%%%%%%%%%%%%%%%%
\section{Usage}

First of all, the package \textsf{childdoc} is \emph{not} a standard
\LaTeXe{} |.sty| style file! Therefore it needs to be invoked in
a non-standard way.

%%%%%%%%%%%%%%%%%%%%%%%%%%%%%%%%%%%%%%%%%%%%%%%%%%%%%%%%%%%%%%%%%%%%%%%%%%%%%%%%
\subsection{Included Files}
\label{sec:include}

%%%%%%%%%%%%%%%%%%%%%%%%%%%%%%%%%%%%%%%%
\DescribeMacro{\childdocmain}
To use the package, add the commands
\begin{center}
\begin{tabular}{l}
|\input{childdoc.def}|\\
|\childdocmain{}|\\
\end{tabular}
\end{center}
at the very top of the main \LaTeX{} file,
in particular \emph{before} the |\documentclass| statement!
The argument of |\childdocmain| should be left empty
(but it must be present).

%%%%%%%%%%%%%%%%%%%%%%%%%%%%%%%%%%%%%%%%
\DescribeMacro{\childdocof}
Furthermore, add the commands
\begin{center}
\begin{tabular}{l}
|\input{childdoc.def}|\\
|\childdocof{|\textit{main}|}|\\
\end{tabular}
\end{center}
at the top of every child file \textit{child}
which is included by |\include{|\textit{child}|}|
from within the main file
(or at least for those files to be compiled individually).
The argument \textit{main} must be the filename of the main file.

There are a couple of
considerations in setting up the main and child documents:

%%%%%%%%%%%%%%%%%%%%%%%%%%%%%%%%%%%%%%%%
\paragraph{Restrictions.}

Please note the following restrictions:
\begin{itemize}
\item
|\childdocmain| must be called with one argument \textit{main}
to ensure compatibility with earlier version of the package.
It must either be empty (|\childdocmain{}|)
or precisely match the filename of the main file in which it is specified.
See \secref{sec:detection} for further information.
\item
The filename \textit{main} must be specified without the |.tex| extension.
\item
The filename \textit{main} is case sensitive
(even in case-insensitive file systems)
due to internal string comparison.
\item
The argument \textit{main} should be fully expanded, it cannot be a macro.
\item
Subdirectories and special characters should be avoided in filenames.
\item
The command |\childdocmain{|\textit{main}|}| must be followed by a whitespace.
It should not be followed immediately by another command
or by a comment mark `|%|'.
This is because the \TeX{} parser reads the token immediately following
the argument of |\childdocmain| and puts it
at the beginning of every child section;
however, a white\-space is ignored.
\end{itemize}

%%%%%%%%%%%%%%%%%%%%%%%%%%%%%%%%%%%%%%%%
\paragraph{Content of Main File.}

It is advisable to place all content in the child files included by |\include|.
Any output contained in the main file will appear in all child documents
unless suppressed manually;
it cannot be suppressed automatically by the |\includeonly| directive
and thus should normally be avoided.
A method to include some content in the main file
by means of conditional processing is described in \secref{sec:conditional}.

%%%%%%%%%%%%%%%%%%%%%%%%%%%%%%%%%%%%%%%%
\paragraph{Page Numbering.}

When only a part of the document is compiled,
the appropriate numbering of pages
(as well as other status parameters)
is determined from the |.aux| files.
The latter contain information from previous passes.
However this information needs to propagate through
all intermediate child documents.
Therefore the page numbering in child documents may well
be inconsistent until the complete document is compiled at least once.

A useful (if unconventional) way to always ensure a consistent
page numbering is to restart the numbering in each child document
and denote the pages by `\textit{child}|.|\textit{page}'
where \textit{child} represents the chapter/section number of the child file.
This can be achieved by the command
|\numberwithin{page}{|\textit{child}|}|
of the \textsf{amsmath} package
where \textit{child} can be |chapter| or |section|
depending on the chosen structuring.
Alternatively, one can modify the macro |\thepage| appropriately
and reset the counter |page| at the start of each child file.

%%%%%%%%%%%%%%%%%%%%%%%%%%%%%%%%%%%%%%%%%%%%%%%%%%%%%%%%%%%%%%%%%%%%%%%%%%%%%%%%
\subsection{Conditional Processing}
\label{sec:conditional}

The package provides a mechanism to compile different versions
of a document. To customise the versions further some conditional processing
can come in handy to distinguish which version is being compiled.
The package provides two macros to describe the compilation context:

%%%%%%%%%%%%%%%%%%%%%%%%%%%%%%%%%%%%%%%%
\DescribeMacro{\ifchilddoc}
The conditional |\ifchilddoc| distinguishes between the compilation of
child documents and the main document:
%
\begin{center}
|\ifchilddoc |\textit{child-code}| |[|\||else |\textit{main-code}]| \||fi|
\end{center}

%%%%%%%%%%%%%%%%%%%%%%%%%%%%%%%%%%%%%%%%
\DescribeMacro{\childdocname}
\DescribeMacro{\childdocjob}
The macro |\childdocname| contains the filename (without extension)
of the main or child file being processed.
Note that |\childdocjob| will always contain the name of the main file.

%%%%%%%%%%%%%%%%%%%%%%%%%%%%%%%%%%%%%%%%
\paragraph{Title Page.}

Conditional processing can be used to include a title or banner page
in the main document when proper precautions are taken.
Importantly, the code in the main file should ensure that the page counter
(as well as other status parameters which are stored in the |.aux| files)
takes the same value after the conditional processing.
Otherwise the page numbers may take divergent values
depending on which part is compiled.

For example, a title page could be declared by:
%
\begin{center}
\begin{tabular}{l}
|\ifchilddoc\||else|\\
|\addtocounter{page}{-1}|\\
\textit{code for title page}\\
|\newpage|\\
|\||fi|
\end{tabular}
\end{center}
%
A banner page for the child documents can be generated by:
%
\begin{center}
\begin{tabular}{l}
|\ifchilddoc|\\
|\addtocounter{page}{-1}|\\
\textit{code for banner page}\\
|\newpage|\\
|\||fi|
\end{tabular}
\end{center}
%
Here one could write a message such as:
\begin{center}
|This is the part \childdocname{} of \childdocjob{}.|
\end{center}

%%%%%%%%%%%%%%%%%%%%%%%%%%%%%%%%%%%%%%%%%%%%%%%%%%%%%%%%%%%%%%%%%%%%%%%%%%%%%%%%
\subsection{Flags}
\label{sec:flags}

The package makes it easy to generate different versions
of the main or child documents.
To this end compilation flags can be defined
and assigned different default values.
They will be particularly useful in conjunction
with the forwarding mechanism described in \secref{sec:forward}.

For example, it may be useful to have a flag |\version|
which can be set to |draft| or |final|.
The document source will contain some conditional code
depending on the value of |\version|.
Suppose further, the flag should default to |final| for the main file
and to |draft| for child files
which is a natural assignment for editing the document.
This is achieved by placing the following code
in the preamble of the main document
(below the |\childdocmain| directive):
%
\begin{center}
\begin{tabular}{l}
|\ifchilddoc|\\
|\providecommand{\version}{draft}|\\
|\||else|\\
|\providecommand{\version}{final}|\\
|\||fi|
\end{tabular}
\end{center}
%
The definition by |\providecommand| makes sure
that previous definitions are not overwritten.
Further statements |\providecommand{\version}{...}|
can thus be added before the above code to override it.

For the main file, one might add a line
(between |\childdocmain| and the above block)
%
\begin{center}
|%\ifchilddoc\||else\providecommand{\version}{draft}\||fi|
\end{center}
%
which can be uncommented to produce a draft version.
Likewise one can add a line to the very top of a child file
(above the |\childdocof{|\textit{main}|}| directive)
%
\begin{center}
|%\providecommand{\version}{final}|
\end{center}
%
which can be uncommented to produce the final version of this child document.

%%%%%%%%%%%%%%%%%%%%%%%%%%%%%%%%%%%%%%%%%%%%%%%%%%%%%%%%%%%%%%%%%%%%%%%%%%%%%%%%
\subsection{Forwarding}
\label{sec:forward}

Different versions of the main or child documents
using compilation flags as described in \secref{sec:flags}
can be (permanently) stored in different files
for convenient compilation, viewing and distribution.
To this end, the package defines a command
to pass on compilation to a different file:

%%%%%%%%%%%%%%%%%%%%%%%%%%%%%%%%%%%%%%%%
\DescribeMacro{\childdocforward}
The command |\childdocforward| redirects processing to
another source file:
%
\begin{center}
\begin{tabular}{l}
|\input{childdoc.def}|\\
|\childdocforward[|\textit{main}|]{|\textit{dest}|}|\\
\end{tabular}
\end{center}
%
The argument \textit{dest} is the destination file
(without extension).
It should be the main file or one of the child files.
Note that further \textsf{childdoc} directives
such as |\childdocof| and |\childdocforward|
in the indicated file will be processed in this form.
The optional argument \textit{main}
passes on directly to the main file \textit{main}
while pretending to compile the child \textit{dest}.
This form behaves as if \textit{dest}
issues |\childdocof{|\textit{main}|}| right away,
and no further \textsf{childdoc} directives will be processed.

%%%%%%%%%%%%%%%%%%%%%%%%%%%%%%%%%%%%%%%%
\DescribeMacro{\...prefix}
In the alternative form |\childdocforwardprefix|,
%
\begin{center}
\begin{tabular}{l}
|\input{childdoc.def}|\\
|\childdocforwardprefix[|\textit{main}|]{|\textit{prefix}|}{|\textit{dest}|}|
\end{tabular}
\end{center}
%
the destination file is determined by a pattern
depending on the current file:
To make this work, the current file must be called
`{\textit{prefix}\hspace{0.2em}\textit{suffix}}'
with \textit{prefix} matching precisely the argument.
Processing is then passed on to the file
`{\textit{dest}\hspace{0.2em}\textit{suffix}}'.
Surely, the same effect is achieved by
directly specifying the
argument `{\textit{dest}\hspace{0.2em}\textit{suffix}}'
in the first form.
However, that requires to set up a different file
for each child. With the alternative form of the command
all these files can have exactly the same content
which simplifies setting them up and maintaining them.

For example, the following file |draft.tex|
with a compilation flag |\version| as described in \secref{sec:flags}
compiles the main document as a draft:
%
\begin{center}
\begin{tabular}{l}
|\def\version{draft}|\\
|\input{childdoc.def}|\\
|\childdocforward{|\textit{main}|}|
\end{tabular}
\end{center}
%
Likewise, the following files |final|\textit{nn}|.tex|
compile the final version of the child document
|child|\textit{nn}|.tex|:
%
\begin{center}
\begin{tabular}{l}
|\def\version{final}|\\
|\input{childdoc.def}|\\
|\childdocforwardprefix{final}{child}|
\end{tabular}
\end{center}
%

Note that when several versions of a main file and/or of each child file
are to be generated, it may be convenient to set up a |Makefile| or
shell script to automatise the process.

%%%%%%%%%%%%%%%%%%%%%%%%%%%%%%%%%%%%%%%%%%%%%%%%%%%%%%%%%%%%%%%%%%%%%%%%%%%%%%%%
\subsection{Command Line Processing}
\label{sec:commandline}

The effect of redirection files can also be achieved by invoking
the \LaTeX{} compiler with a more elaborate command line.
Most conveniently this should be done as part
of a shell script or a |Makefile|.

When using \textsf{childdoc} in the main file, the following
command lines effectively perform a redirection
(note that depending on the shell being used,
backslashes may have to be doubled: `|\|' $\to$ `|\\|'):
%
\begin{center}
|... -jobname "|\textit{target}|" |\\|"|[\textit{flags}]%
|\input{childdoc.def}\childdocforward[|\textit{main}|]{|\textit{dest}|}"|
\end{center}
%
Here \textit{target} is the name of the output file,
\textit{main} is the name of the main file
and \textit{dest} is the name of the main or child file to be processed
(all filenames without extensions).
The optional argument \textit{main} can be omitted
if \textit{main} matches \textit{dest}.
Optionally, compilation \textit{flags} can be defined via |\def| commands.
This command line makes the \TeX{} engine believe
it is compiling the file \textit{target}
whose content is specified as the latter parameter.
The provided code then forwards the processing to
\textit{main} or \textit{dest} as described in \secref{sec:forward}.

%%%%%%%%%%%%%%%%%%%%%%%%%%%%%%%%%%%%%%%%%%%%%%%%%%%%%%%%%%%%%%%%%%%%%%%%%%%%%%%%
\subsection{Include by Input}
\label{sec:input}

Including child documents by |\include| has some restrictions by design.
Most notably, the content of a child document always occupies
its own set of pages; pages cannot be shared between child documents.
Usually, this behaviour makes perfect sense
because each child document contain an essential part of the document.
However, in some situations it may be desirable to compose
a document from a collection of parts
without having mandatory page breaks between then.
For this case, the package
provides a mechanism to include parts
by |\input| which can also be processed individually.
However, by construction this mechanism
requires manual handling of the content to be output.

%%%%%%%%%%%%%%%%%%%%%%%%%%%%%%%%%%%%%%%%
\DescribeMacro{\ifchilddocmanual}
The main file should be prepared as usual, see \secref{sec:include}.
However, the document body must make a distinction
between processing of an individual part and of the main document, e.g.:
%
\begin{center}
\begin{tabular}{l}
|\ifchilddocmanual|\\
|\input{\childdocname}|\\
|\||else|\\
\textit{document body with }|\input{|\textit{part}|}|\\
|\||fi|
\end{tabular}
\end{center}
%
The conditional |\ifchilddocmanual| is true whenever
a part to be included by |\input| is being compiled,
and the name of the part is stored in |\childdocname|.

%%%%%%%%%%%%%%%%%%%%%%%%%%%%%%%%%%%%%%%%
\DescribeMacro{\childdocby}
Each part to be included by |\input| should start with:
%
\begin{center}
\begin{tabular}{l}
|\input{childdoc.def}|\\
|\childdocby{|\textit{main}|}|\\
\end{tabular}
\end{center}
%
The directive |\childdocby| is similar to |\childdocof|
described in \secref{sec:include},
but the subsequent selection of content must be done manually.
To that end, both |\ifchilddoc| and |\ifchilddocmanual|
will be true upon processing of a part,
and the name of the part is stored in |\childdocname|.
Note that |\jobname| will be set to the filename of the current part
so that each part receives an individual |.aux| file
that does not interfere with the |.aux| file(s) of the main document.
This behaviour can be altered by the alternative form
|\childdocby[*]{|\textit{main}|}| (with a non-empty optional argument)
which uses the |.aux| file of the main document
by setting |\jobname| to \textit{main}.

%%%%%%%%%%%%%%%%%%%%%%%%%%%%%%%%%%%%%%%%%%%%%%%%%%%%%%%%%%%%%%%%%%%%%%%%%%%%%%%%
\subsection{Driver Development}
\label{sec:driver}

The \textsf{childdoc} mechanism can also be use for the development
of definition files such as \LaTeX{} styles or classes.
This case differs from the above setup with multiple parts
included by |\include| in that no |\includeonly| should be invoked.
This can be achieved by starting the include file
(before |\ProvidesPackage|) with:
%
\begin{center}
\begin{tabular}{l}
|\input{childdoc.def}|\\
|\childdocforward{|\textit{main}|}|\\
\end{tabular}
\end{center}
%
or alternatively with:
%
\begin{center}
\begin{tabular}{l}
|\input{childdoc.def}|\\
|\childdocby{|\textit{main}|}|\\
\end{tabular}
\end{center}
%
Both forms have slightly different effects as described above.
The main file is prepared as usual, see \secref{sec:include}.

%%%%%%%%%%%%%%%%%%%%%%%%%%%%%%%%%%%%%%%%%%%%%%%%%%%%%%%%%%%%%%%%%%%%%%%%%%%%%%%%
\subsection{Legacy Detection}
\label{sec:detection}

The directive |\childdocmain| in the main file can detect
whether the complete document or merely a child is to be compiled
even without using the directive |\childdocof|.
This method is deprecated because it is less robust
and there is no compelling reason to use it;
it is merely provided for backward compatibility
and it may be removed in future versions.

If the detection mechanism is to be used,
it is mandatory to correctly specify
the filename of the main file as the argument of |\childdocmain|:
%
\begin{center}
\begin{tabular}{l}
|\input{childdoc.def}|\\
|\childdocmain{|\textit{main}|}|\\
\end{tabular}
\end{center}
%
If |\jobname| does not match the argument \textit{main} of |\childdocmain|,
it is assumed that |\jobname| points to the child file to be compiled.
When using |\childdocmain| with the main file specified as argument,
it suffices to start a child file
with just |\input{|\textit{main}|}|
without loading of the package and using |\childdocof|.
If instead all processing is done
with the appropriate \textsf{childdoc} directives,
the argument of \textit{main} of |\childdocmain| can be empty.

An alternative version of the command line processing described
in \secref{sec:commandline} using the detection mechanism reads:
%
\begin{center}
|... -jobname "|\textit{target}|" "|[\textit{flags}]%
[|\def\jobname{|\textit{dest}|}|]|\input{|\textit{main}|}"|
\end{center}

%%%%%%%%%%%%%%%%%%%%%%%%%%%%%%%%%%%%%%%%%%%%%%%%%%%%%%%%%%%%%%%%%%%%%%%%%%%%%%%%
\subsection{Manual Code}
\label{sec:manual}

In case one cannot be certain whether the definitions file |childdoc.def|
is installed on the target \TeX{} distribution
and one prefers not to ship it,
it is conceivable to paste a few relevant commands into the sources.

To that end, drop all statements |\input{childdoc.def}|
and perform the replacements as outlined below.
Instead of |\childdocmain{|\textit{main}|}| add the following code
to the top of the main file:
%
\begin{center}
\begin{tabular}{l}
|\||ifdefined\childdocname\endinput\||fi\newif\ifchilddoc|\\
|\edef\childdocname{\scantokens\expandafter{\jobname\noexpand}}|\\
|\def\childdocmain{|\textit{main}|}\||ifx\childdocmain\childdocname\||else|\\
|\childdoctrue\includeonly{\childdocname}\let\jobname\childdocmain\||fi|\\
\end{tabular}
\end{center}
%
Instead of |\childdocof{|\textit{main}|}| just include the main file
at the top of each child file:
%
\begin{center}
|\input{|\textit{main}|}|
\end{center}
%
A simple redirection |\childdocforward{|\textit{dest}|}| is achieved by:
%
\begin{center}
|\def\jobname{|\textit{dest}|}\input{\jobname}|
\end{center}
%
The redirection with prefix
|\childdocforwardprefix[|\textit{prefix}|]{|\textit{dest}|}|
is accomplished by:
%
\begin{center}
\begin{tabular}{l}
|{\edef\jobname{\scantokens\expandafter{\jobname\noexpand}}|\\
|\def\redirectjob |\textit{prefix}|#1~~~{\gdef\jobname{|\textit{dest}|#1}}|\\
|\expandafter\redirectjob\jobname~~~}\input{\jobname}|
\end{tabular}
\end{center}

In an alternative approach,
child documents can be compiled by a specific command line
without additional code or specific definitions:
%
\begin{center}
|... -jobname "|\textit{target}|" "|[\textit{flags}]%
|\includeonly{|\textit{dest}|}\input{|\textit{main}|}"|
\end{center}
%

%%%%%%%%%%%%%%%%%%%%%%%%%%%%%%%%%%%%%%%%%%%%%%%%%%%%%%%%%%%%%%%%%%%%%%%%%%%%%%%%
%%%%%%%%%%%%%%%%%%%%%%%%%%%%%%%%%%%%%%%%%%%%%%%%%%%%%%%%%%%%%%%%%%%%%%%%%%%%%%%%
\section{Information}

%%%%%%%%%%%%%%%%%%%%%%%%%%%%%%%%%%%%%%%%%%%%%%%%%%%%%%%%%%%%%%%%%%%%%%%%%%%%%%%%
\subsection{Copyright}

Copyright \copyright{} 2017--2018 Niklas Beisert

This work may be distributed and/or modified under the
conditions of the \LaTeX{} Project Public License, either version 1.3
of this license or (at your option) any later version.
The latest version of this license is in
  \url{http://www.latex-project.org/lppl.txt}
and version 1.3 or later is part of all distributions of \LaTeX{}
version 2005/12/01 or later.

This work has the LPPL maintenance status `maintained'.

The Current Maintainer of this work is Niklas Beisert.

This work consists of the files |README.txt|, |childdoc.ins| and |childdoc.dtx|
as well as the derived files |childdoc.def|, |cdocsamp.tex|
with |cdocsch1.tex|, |cdocsch2.tex|, |cdocspt3.tex|, |cdocspt4.tex|,
|cdocsdrf.tex|, |cdocsfn1.tex|, |cdocsfn2.tex|
as well as |childdoc.pdf|.

%%%%%%%%%%%%%%%%%%%%%%%%%%%%%%%%%%%%%%%%%%%%%%%%%%%%%%%%%%%%%%%%%%%%%%%%%%%%%%%%
\subsection{Files and Installation}

The package consists of the files:
%
\begin{center}
\begin{tabular}{ll}
    |README.txt|   & readme file \\
    |childdoc.ins| & installation file \\
    |childdoc.dtx| & source file \\
    |childdoc.def| & definition file \\
    |cdocsamp.tex| & sample main file \\
    |cdocsch1.tex| & sample include file \\
    |cdocsch2.tex| & sample include file \\
    |cdocspt3.tex| & sample part file \\
    |cdocspt4.tex| & sample part file \\
    |cdocsdrf.tex| & sample redirection file \\
    |cdocsfn1.tex| & sample redirection file \\
    |cdocsfn2.tex| & sample redirection file \\
    |childdoc.pdf| & manual
\end{tabular}
\end{center}
%
The distribution consists of the files
|README.txt|, |childdoc.ins| and |childdoc.dtx|.
%
\begin{itemize}
\item
Run (pdf)\LaTeX{} on |childdoc.dtx|
to compile the manual |childdoc.pdf| (this file).
\item
Run \LaTeX{} on |childdoc.ins| to create the definitions file |childdoc.def|
and the sample |cdocsamp.tex| with include files
|cdocsch1.tex|, |cdocsch2.tex|, |cdocspt3.tex|, |cdocspt4.tex|,
|cdocsdrf.tex|, |cdocsfn1.tex|, |cdocsfn2.tex|.
Then copy the file |childdoc.def| to an appropriate directory of your \LaTeX{}
distribution, e.g.\ \textit{texmf-root}|/tex/latex/childdoc|.
\end{itemize}

%%%%%%%%%%%%%%%%%%%%%%%%%%%%%%%%%%%%%%%%%%%%%%%%%%%%%%%%%%%%%%%%%%%%%%%%%%%%%%%%
\subsection{Related CTAN Packages}

There are several other packages which offer a similar functionality:
%
\begin{itemize}
\item
The packages
\href{http://ctan.org/pkg/docmute}{\textsf{docmute}},
\href{http://ctan.org/pkg/includex}{\textsf{includex}} and
\href{http://ctan.org/pkg/standalone}{\textsf{standalone}}
provide commands to include only the document body of
a child file thus allowing both files to be compiled individually.
\item
The packages \href{http://ctan.org/pkg/subdocs}{\textsf{subdocs}}
and \href{http://ctan.org/pkg/subfiles}{\textsf{subfiles}}
provide structures in which the main and child documents can be
encapsulated and allowing them to be compiled individually.
The inclusion mechanism is different from the conventional |\include|.
\item
The package \href{http://ctan.org/pkg/combine}{\textsf{combine}}
is an elaborate solution to combine several documents into one.
\end{itemize}
%
See also the CTAN topic \href{http://ctan.org/topic/subdocs}{\textsf{subdocs}}
for further related packages.
The present package differs from the above solutions in that
a document structure constructed with the conventional |\include| mechanism
just needs two extra commands at the top of every file
such that all constituent files can be compiled individually.

%%%%%%%%%%%%%%%%%%%%%%%%%%%%%%%%%%%%%%%%%%%%%%%%%%%%%%%%%%%%%%%%%%%%%%%%%%%%%%%%
%\subsection{Feature Suggestions}
%
%The following is a list of features which may be useful for future
%versions of this package:
%%
%\begin{itemize}
%\item
%\ldots
%\end{itemize}

%%%%%%%%%%%%%%%%%%%%%%%%%%%%%%%%%%%%%%%%%%%%%%%%%%%%%%%%%%%%%%%%%%%%%%%%%%%%%%%%
\subsection{Revision History}

%%%%%%%%%%%%%%%%%%%%%%%%%%%%%%%%%%%%%%%%
\paragraph{v2.0:} 2018/12/30

\begin{itemize}
\item
immediate forward processing
\item
added |\childdocby| mechanism
\item
manual restructured
\end{itemize}

%%%%%%%%%%%%%%%%%%%%%%%%%%%%%%%%%%%%%%%%
\paragraph{v1.6:} 2018/01/17

\begin{itemize}
\item
application for development of include files
\item
corrections to manual
\end{itemize}

%%%%%%%%%%%%%%%%%%%%%%%%%%%%%%%%%%%%%%%%
\paragraph{v1.5:} 2017/05/21

\begin{itemize}
\item
more complete structuring introduced
\item
|\childdocof| introduced
\item
|\childdoc| renamed to |\childdocmain|
\item
|\childredirect| renamed to |\childdocforward| and |\childdocforwardprefix|
and functionality expanded
\end{itemize}

%%%%%%%%%%%%%%%%%%%%%%%%%%%%%%%%%%%%%%%%
\paragraph{v1.0:} 2017/04/27

\begin{itemize}
\item
manual and install package
\item
first version published on CTAN
\end{itemize}

%%%%%%%%%%%%%%%%%%%%%%%%%%%%%%%%%%%%%%%%
\paragraph{v0.6:} 2017/04/26

\begin{itemize}
\item
redirection mechanism added
\end{itemize}

%%%%%%%%%%%%%%%%%%%%%%%%%%%%%%%%%%%%%%%%
\paragraph{v0.5:} 2017/04/26

\begin{itemize}
\item
functionality in definition file
\end{itemize}


%%%%%%%%%%%%%%%%%%%%%%%%%%%%%%%%%%%%%%%%%%%%%%%%%%%%%%%%%%%%%%%%%%%%%%%%%%%%%%%%
%%%%%%%%%%%%%%%%%%%%%%%%%%%%%%%%%%%%%%%%%%%%%%%%%%%%%%%%%%%%%%%%%%%%%%%%%%%%%%%%
%%%%%%%%%%%%%%%%%%%%%%%%%%%%%%%%%%%%%%%%%%%%%%%%%%%%%%%%%%%%%%%%%%%%%%%%%%%%%%%%
\appendix

\settowidth\MacroIndent{\rmfamily\scriptsize 000\ }

 \DocInput{childdoc.dtx}

\end{document}
%</driver>
% \fi
%
% %%%%%%%%%%%%%%%%%%%%%%%%%%%%%%%%%%%%%%%%%%%%%%%%%%%%%%%%%%%%%%%%%%%%%%%%%%%%%%
% %%%%%%%%%%%%%%%%%%%%%%%%%%%%%%%%%%%%%%%%%%%%%%%%%%%%%%%%%%%%%%%%%%%%%%%%%%%%%%
% \section{Sample}
%\iffalse
%<*samplemain>
%\fi
%
% The following presents a sample document
% with two chapters, two parts, a title page,
% a compile flag as well as three forwarding files to set the flag.
% It consists of eight |.tex| files:
% \begin{center}
% \begin{tabular}{ll}
% |cdocsamp.tex|&main file\\
% |cdocsch1.tex|&include file for chapter 1\\
% |cdocsch2.tex|&include file for chapter 2\\
% |cdocspt3.tex|&include file for part 3\\
% |cdocspt4.tex|&include file for part 4\\
% |cdocsdrf.tex|&forwarding file for main file in draft mode\\
% |cdocsfi1.tex|&forwarding file for final version of chapter 1\\
% |cdocsfi2.tex|&forwarding file for final version of chapter 2\\
% \end{tabular}
% \end{center}
% Each of the eight files can be compiled directly by the \LaTeX{} compiler.
%
% %%%%%%%%%%%%%%%%%%%%%%%%%%%%%%%%%%%%%%
% \paragraph{Main File.}
%
% The main file is called |cdocsamp.tex|.
%
% Load the \textsf{childdoc} definitions and
% declare the filename for the main document:
%    \begin{macrocode}
\input{childdoc.def}
\childdocmain{}
%    \end{macrocode}

% Optional override for |\version| flag:
%    \begin{macrocode}
%%\ifchilddoc\else\providecommand{\version}{draft}\fi
%    \end{macrocode}

% Define the default values for the |\version| flag
% (|final| for the main file and |draft| for childs):
%    \begin{macrocode}
\ifchilddoc
\providecommand{\version}{draft}
\else
\providecommand{\version}{final}
\fi
%    \end{macrocode}

% Load the standard document class:
%    \begin{macrocode}
\documentclass[12pt]{article}
%    \end{macrocode}

% Start the document body:
%    \begin{macrocode}
\begin{document}
%    \end{macrocode}

% Declare a title page.
% Print title, part of document being processed and version flag:
%    \begin{macrocode}
\addtocounter{page}{-1}
\begin{center}
{\LARGE\bfseries{}childdoc example\par}
\vspace{1cm}
\ifchilddoc
\ifchilddocmanual part\else chapter\fi:
`\childdocname' of `\childdocjob'\par
\else
main document: `\childdocjob'\par
\fi
version: \version\par
\end{center}
\newpage
%    \end{macrocode}

% Manually include selected file,
% otherwise process as usual:
%    \begin{macrocode}
\ifchilddocmanual
\section*{part `\childdocname'}
\input{\childdocname}
\else
%    \end{macrocode}

% Include the two chapters:
%    \begin{macrocode}
\include{cdocsch1}
\include{cdocsch2}
%    \end{macrocode}

% Include the two parts unless only chapters should be displayed:
%    \begin{macrocode}
\ifchilddoc\else
\section{part three}
\input{cdocspt3}
\section{part four}
\input{cdocspt4}
\fi
%    \end{macrocode}

% Process as usual until here:
%    \begin{macrocode}
\fi
%    \end{macrocode}

% End of document body:
%    \begin{macrocode}
\end{document}
%    \end{macrocode}
%\iffalse
%</samplemain>
%\fi
%
% %%%%%%%%%%%%%%%%%%%%%%%%%%%%%%%%%%%%%%
% \paragraph{Chapter Include Files.}
%
% The include files are called |cdocsch1.tex| and |cdocsch2.tex|.
%
%\iffalse
%<*samplechap1|samplechap2>
%\fi

% Optional override for |\version| flag:
%    \begin{macrocode}
%%\providecommand{\version}{final}
%    \end{macrocode}

% Include the main document:
%    \begin{macrocode}
\input{childdoc.def}
\childdocof{cdocsamp}
%    \end{macrocode}

%\iffalse
%</samplechap1|samplechap2>
%\fi
%
%\iffalse
%<*samplechap1>
%\fi
% Some text for chapter 1:
%    \begin{macrocode}
\section{one}
some text in chapter one
%    \end{macrocode}

%\iffalse
%</samplechap1>
%\fi
% Some text for chapter 2:
%\iffalse
%<*samplechap2>
%\fi
%    \begin{macrocode}
\section{two}
more text in chapter two
%    \end{macrocode}

%\iffalse
%</samplechap2>
%\fi
%
% %%%%%%%%%%%%%%%%%%%%%%%%%%%%%%%%%%%%%%
% \paragraph{Part Include Files.}
%
% The include files are called |cdocspt3.tex| and |cdocspt4.tex|.
%
%\iffalse
%<*samplepart3|samplepart4>
%\fi

% Optional override for |\version| flag:
%    \begin{macrocode}
%%\providecommand{\version}{final}
%    \end{macrocode}

% Include the main document:
%    \begin{macrocode}
\input{childdoc.def}
\childdocby{cdocsamp}
%    \end{macrocode}

%\iffalse
%</samplepart3|samplepart4>
%\fi
%
%\iffalse
%<*samplepart3>
%\fi
% Some text for part 3:
%    \begin{macrocode}
some text in part three
%    \end{macrocode}

%\iffalse
%</samplepart3>
%\fi
% Some text for part 4:
%\iffalse
%<*samplepart4>
%\fi
%    \begin{macrocode}
more text in part four
%    \end{macrocode}

%\iffalse
%</samplepart4>
%\fi
%
% %%%%%%%%%%%%%%%%%%%%%%%%%%%%%%%%%%%%%%
% \paragraph{Forwarding for a Complete Draft.}
%
% The following forwarding file |cdocsdrf.tex|
% compiles the main document in draft mode:
%\iffalse
%<*sampledraft>
%\fi
%    \begin{macrocode}
\def\version{draft}
\input{childdoc.def}
\childdocforward{cdocsamp}
%    \end{macrocode}

%\iffalse
%</sampledraft>
%\fi
%
% %%%%%%%%%%%%%%%%%%%%%%%%%%%%%%%%%%%%%%
% \paragraph{Forwarding for Final Version of the Chapters.}
%
% The following forwarding files |cdocsfn1.tex| and |cdocsfn2.tex|
% (with identical content)
% compile the final versions of the child documents
% |cdocsch1.tex| and |cdocsch2.tex|, respectively:
%\iffalse
%<*samplefinal>
%\fi
%    \begin{macrocode}
\def\version{final}
\input{childdoc.def}
\childdocforwardprefix[cdocsamp]{cdocsfn}{cdocsch}
%    \end{macrocode}

%\iffalse
%</samplefinal>
%\fi
%
% %%%%%%%%%%%%%%%%%%%%%%%%%%%%%%%%%%%%%%
% \paragraph{Command Line Processing.}
%
% The following three command lines generate the output files
% |cdocscld|, |cdocscl1| and |cdocscl2|
% which should be identical to
% |cdocsdrf|, |cdocsch1| and |cdocsfn2|, respectively:
% \begin{center}
% \begin{tabular}{l}
% |latex -jobname cdocscld \|\\
% |  "\def\version{draft}\input{childdoc.def}\childdocforward{cdocsamp}"|\\
% |latex -jobname cdocscl1 \|\\
% |  "\input{childdoc.def}\childdocforward[cdocsamp]{cdocsch1}"|\\
% |latex -jobname cdocscl2 \|\\
% |  "\def\version{final}\input{childdoc.def}\childdocforward{cdocsch2}"|
% \end{tabular}
% \end{center}
% Note that the trailing backslash on each first line
% merely continues the input to the second line
% (for convenient cut ant paste).
% Furthermore, the command |latex| can be replaced by any
% of its alternative versions such as |pdflatex|.
%
% %%%%%%%%%%%%%%%%%%%%%%%%%%%%%%%%%%%%%%%%%%%%%%%%%%%%%%%%%%%%%%%%%%%%%%%%%%%%%%
% %%%%%%%%%%%%%%%%%%%%%%%%%%%%%%%%%%%%%%%%%%%%%%%%%%%%%%%%%%%%%%%%%%%%%%%%%%%%%%
% \section{Implementation}
%\iffalse
%<*package>
%\fi
%
% This section describes the definitions file |childdoc.def|.

% The definitions cannot be loaded using |\usepackage| or |\RequirePackage|
% which has a mechanism to prevent loading a style file more than once.
% When loading the definitions by means of |\input|
% multiple instances have to be prevented manually:
%\iffalse
%This code needs to be before the `\ProvidesFile' directive
%which is defined at the beginning of this file.
%Therefore it is also placed there and commented out here.
%</package>
%<*discard>
%\fi
%    \begin{macrocode}
\ifdefined\childdocmain\endinput\fi
%    \end{macrocode}
%\iffalse
%</discard>
%<*package>
%\fi
%
% \macro{\ifchilddoc}
% \macro{\ifchilddocmanual}
% The conditional |\ifchilddoc| tells whether a
% child (true) or main (false) document is being compiled.
% The conditional |\ifchilddocmanual| tells whether
% the |\includeonly| mechanism is used (false) or
% the selection of child files must be performed manually (true).
% The definitions initialise to false:
%    \begin{macrocode}
\newif\ifchilddoc
\newif\ifchilddocmanual
%    \end{macrocode}

% \macro{\childdocname}
% \macro{\childdocjob}
% The macro |\childdocname| stores the name of the main document
% to be compiled. The macro |\childdocjob| stores the name of
% the document on which the \LaTeX{} compiler was originally invoked.
% The content of |\jobname| cannot be compared
% to filenames specified in the source due to different catcodes.
% The following code rescans |\jobname|, stores the result
% in |\childdocname| and saves a copy in |\childdocjob|:
%    \begin{macrocode}
\edef\childdocname{\scantokens\expandafter{\jobname\noexpand}}
\let\childdocjob\childdocname
%    \end{macrocode}

% \macro{\childdocdisable}
% The macro |\childdocdisable| prevents the main file
% from being processed more than once.
% At this stage, the main document command |\childdocmain|
% is assumed to be called once again where it should do nothing.
% Any subsequent call to it should prevent
% a secondary processing of the main document
% It overwrites the forwarding commands
% |\childdocof| and |\childdocforward|
% with empty macros to prevent further inclusions of the main document:
%    \begin{macrocode}
\newcommand{\childdocdisable}
{
  \renewcommand{\childdocmain}[1]{\renewcommand{\childdocmain}[1]{\endinput}}
  \renewcommand{\childdocof}[1]{}
  \renewcommand{\childdocby}[2][]{}
  \renewcommand{\childdocforward}[2][]{}
  \renewcommand{\childdocdisable}{}
}
%    \end{macrocode}

% \macro{\childdocmain}
% The macro |\childdocmain| is to be called at the top of the main file
% with nothing or the main filename (without extension) as argument.
% First, it breaks loops.
% If the argument is not empty and does not match |\childdocname|
% (which is set by the first inclusion of |childdoc.def|),
% |\ifchilddoc| is set to true, |\includeonly| is applied to the child file
% and |\jobname| is set to the main file
% (for proper handling of |.aux| files):
%    \begin{macrocode}
\newcommand{\childdocmain}[1]
{
  \childdocdisable\childdocmain{}
  \if?#1?\else
    \begingroup
      \def\childdoctmp{#1}
      \ifx\childdoctmp\childdocname
        \def\childdoctmp{}
      \else
        \def\childdoctmp
        {
          \childdoctrue
          \includeonly{\childdocname}
          \def\childdocjob{#1}
          \def\jobname{#1}
        }
      \fi
      \expandafter
    \endgroup
    \childdoctmp
  \fi
}
%    \end{macrocode}

% \macro{\childdocof}
% The command |\childdocof| redirects
% compilation to the main file |#1|.
%    \begin{macrocode}
\newcommand{\childdocof}[1]
{
  \childdocdisable
  \childdoctrue
  \includeonly{\childdocname}
  \def\jobname{#1}
  \def\childdocjob{#1}
  \input{#1}
}
%    \end{macrocode}

% \macro{\childdocby}
% The command |\childdocby| ....
%    \begin{macrocode}
\newcommand{\childdocby}[2][]
{
  \childdocdisable
  \childdoctrue
  \childdocmanualtrue
  \if?#1?\else
    \def\jobname{#2}
  \fi
  \def\childdocjob{#2}
  \input{#2}
  \endinput
}
%    \end{macrocode}

% \macro{\childdocforward}
% The command |\childdocforward| redirects
% compilation to the main file or
% (if the optional argument is given) a child file.
% Parameters are set as if the main file
% or a child file starting with |\childdocof| was compiled.
% Then compilation is handed over to the main file:
%    \begin{macrocode}
\newcommand{\childdocforward}[2][]
{
  \begingroup
    \if?#1?
      \def\childdoctmp
      {
        \def\childdocname{#2}
        \def\childdocjob{#2}
        \def\jobname{#2}
        \input{#2}
        \endinput
      }
    \else
      \def\childdoctmp
      {
        \childdocdisable
        \def\childdocname{#2}
        \childdoctrue
        \includeonly{#2}
        \def\childdocjob{#1}
        \def\jobname{#1}
        \input{#1}
        \endinput
      }
    \fi
    \expandafter
  \endgroup
  \childdoctmp
}
%    \end{macrocode}

% \macro{\childdocforwardprefix}
% The command |\childdocforwardprefix| redirects
% compilation to the main or a child file by means of a pattern.
% The prefix |#1| in the current filename is replaced by |#2|
% and the suffix of the current filename is kept
% (it is assumed that the filename does not contain the substring `|~~~|'
% which is used as a delimiter).
% Compilation is handed over to the new file by |\childdocforward|:
%    \begin{macrocode}
\newcommand{\childdocforwardprefix}[3][]
{
  \begingroup
    \def\childdocextract #2##1~~~{\def\childdoctmp{\childdocforward[#1]{#3##1}}}
    \expandafter\childdocextract\childdocname~~~
    \expandafter
  \endgroup
  \childdoctmp
}
%    \end{macrocode}

% \macro{\childdoc}
% The deprecated macro |\childdoc| is a legacy version of |\childdocmain|:
%    \begin{macrocode}
\newcommand{\childdoc}{\childdocmain}
%    \end{macrocode}

% \macro{\childdocredirect}
% The deprecated macro |\childdocredirect| is a legacy version
% of |\childdocforward| and |\childdocforwardprefix|:
%    \begin{macrocode}
\newcommand{\childdocredirect}[2][]
{
  \begingroup
    \if?#1?
      \def\childdoctmp{\childdocforward{#2}}
    \else
      \def\childdoctmp{\childdocforwardprefix{#1}{#2}}
    \fi
    \expandafter
  \endgroup
  \childdoctmp
}
%    \end{macrocode}

%\iffalse
%</package>
%\fi
%
\endinput
\childdocforward[|\textit{main}|]{|\textit{dest}|}"|
\end{center}
%
Here \textit{target} is the name of the output file,
\textit{main} is the name of the main file
and \textit{dest} is the name of the main or child file to be processed
(all filenames without extensions).
The optional argument \textit{main} can be omitted
if \textit{main} matches \textit{dest}.
Optionally, compilation \textit{flags} can be defined via |\def| commands.
This command line makes the \TeX{} engine believe
it is compiling the file \textit{target}
whose content is specified as the latter parameter.
The provided code then forwards the processing to
\textit{main} or \textit{dest} as described in \secref{sec:forward}.

%%%%%%%%%%%%%%%%%%%%%%%%%%%%%%%%%%%%%%%%%%%%%%%%%%%%%%%%%%%%%%%%%%%%%%%%%%%%%%%%
\subsection{Include by Input}
\label{sec:input}

Including child documents by |\include| has some restrictions by design.
Most notably, the content of a child document always occupies
its own set of pages; pages cannot be shared between child documents.
Usually, this behaviour makes perfect sense
because each child document contain an essential part of the document.
However, in some situations it may be desirable to compose
a document from a collection of parts
without having mandatory page breaks between then.
For this case, the package
provides a mechanism to include parts
by |\input| which can also be processed individually.
However, by construction this mechanism
requires manual handling of the content to be output.

%%%%%%%%%%%%%%%%%%%%%%%%%%%%%%%%%%%%%%%%
\DescribeMacro{\ifchilddocmanual}
The main file should be prepared as usual, see \secref{sec:include}.
However, the document body must make a distinction
between processing of an individual part and of the main document, e.g.:
%
\begin{center}
\begin{tabular}{l}
|\ifchilddocmanual|\\
|\input{\childdocname}|\\
|\||else|\\
\textit{document body with }|\input{|\textit{part}|}|\\
|\||fi|
\end{tabular}
\end{center}
%
The conditional |\ifchilddocmanual| is true whenever
a part to be included by |\input| is being compiled,
and the name of the part is stored in |\childdocname|.

%%%%%%%%%%%%%%%%%%%%%%%%%%%%%%%%%%%%%%%%
\DescribeMacro{\childdocby}
Each part to be included by |\input| should start with:
%
\begin{center}
\begin{tabular}{l}
|% \iffalse
%
% childdoc.dtx Copyright (C) 2017-2018 Niklas Beisert
%
% This work may be distributed and/or modified under the
% conditions of the LaTeX Project Public License, either version 1.3
% of this license or (at your option) any later version.
% The latest version of this license is in
%   http://www.latex-project.org/lppl.txt
% and version 1.3 or later is part of all distributions of LaTeX
% version 2005/12/01 or later.
%
% This work has the LPPL maintenance status `maintained'.
%
% The Current Maintainer of this work is Niklas Beisert.
%
% This work consists of the files childdoc.dtx and childdoc.ins
% and the derived files childdoc.def and cdocsamp.tex with
% cdocsch1.tex, cdocsch2.tex, cdocsdrf.tex, cdocsfn1.tex, cdocsfn2.tex.
%
%<package>\ifdefined\childdocmain\endinput\fi
%<package>\ProvidesFile{childdoc.def}[2018/12/30 v2.0 child document driver]
%<samplemain>\ProvidesFile{cdocsamp.tex}[2018/12/30 v2.0 sample for childdoc]
%<*driver>
%\ProvidesFile{childdoc.drv}[2018/12/30 v2.0 childdoc reference manual file]
\PassOptionsToClass{10pt,a4paper}{article}
\documentclass{ltxdoc}

\usepackage[margin=35mm]{geometry}
\usepackage{hyperref}
\usepackage{hyperxmp}
\usepackage[usenames]{color}

\hypersetup{colorlinks=true}
\hypersetup{pdfstartview=FitH}
\hypersetup{pdfpagemode=UseNone}
\hypersetup{pdfsource={}}
\hypersetup{pdflang={en-UK}}
\hypersetup{pdfcopyright={Copyright 2017-2018 Niklas Beisert.
  This work may be distributed and/or modified under the
  conditions of the LaTeX Project Public License, either version 1.3
  of this license or (at your option) any later version.}}
\hypersetup{pdflicenseurl={http://www.latex-project.org/lppl.txt}}
\hypersetup{pdfcontactaddress={ETH Zurich, ITP, HIT K,
  Wolfgang-Pauli-Strasse 27}}
\hypersetup{pdfcontactpostcode={8093}}
\hypersetup{pdfcontactcity={Zurich}}
\hypersetup{pdfcontactcountry={Switzerland}}
\hypersetup{pdfcontactemail={nbeisert@itp.phys.ethz.ch}}
\hypersetup{pdfcontacturl={http://people.phys.ethz.ch/\xmptilde nbeisert/}}

\newcommand{\secref}[1]{\hyperref[#1]{section \ref*{#1}}}

\parskip1ex
\parindent0pt
\let\olditemize\itemize
\def\itemize{\olditemize\parskip0pt}

\begin{document}

\title{The \textsf{childdoc} Package}
\hypersetup{pdftitle={The childdoc Package}}
\author{Niklas Beisert\\[2ex]
  Institut f\"ur Theoretische Physik\\
  Eidgen\"ossische Technische Hochschule Z\"urich\\
  Wolfgang-Pauli-Strasse 27, 8093 Z\"urich, Switzerland\\[1ex]
  \href{mailto:nbeisert@itp.phys.ethz.ch}
  {\texttt{nbeisert@itp.phys.ethz.ch}}}
\hypersetup{pdfauthor={Niklas Beisert}}
\hypersetup{pdfsubject={Manual for the LaTeX2e Package childdoc}}
\date{30 December 2018, \textsf{v2.0}}
\maketitle

\begin{abstract}\noindent
\textsf{childdoc} is a \LaTeXe{} package
that enables the direct compilation
of document sections included by |\include|
to individual files.
\end{abstract}

\begingroup
\parskip0ex
\tableofcontents
\endgroup

%%%%%%%%%%%%%%%%%%%%%%%%%%%%%%%%%%%%%%%%%%%%%%%%%%%%%%%%%%%%%%%%%%%%%%%%%%%%%%%%
%%%%%%%%%%%%%%%%%%%%%%%%%%%%%%%%%%%%%%%%%%%%%%%%%%%%%%%%%%%%%%%%%%%%%%%%%%%%%%%%
\section{Introduction}

\LaTeX{} provides a mechanism to structure a large document (such as a book)
into a main file and several child files (containing the chapters)
using the |\include| command.
This mechanism is beneficial for documents
which span hundreds of pages in order to
make the source file(s) more manageable.
Moreover, compilation can be restricted to
selected child files by means of the |\includeonly| command.
The latter feature can be used to reduce the compilation time while editing
(this was significantly more useful in the earlier days of \LaTeX{})
or to generate a smaller document which is easier to navigate.
Another application of |\includeonly| is to generate
documents consisting of selected parts of the complete document.

However, there are a few drawbacks of the plain |\include| mechanism:
\begin{itemize}
\item
The child files cannot be compiled on their own,
they can only be compiled via the main file.
A naive editing environment
(such as a text editor with an option
to have the current file processed by \LaTeX)
may require one to switch to the main file before compiling;
attempting to compile the child file produces errors.
\item
The main file must be modified (each time)
to adjust the |\includeonly| command
to the present needs. This easily leaves the main file in a messy state.
\item
The generated document will always carry the filename
of the main document. This is inconvenient if
several child files are to be compiled and
to be kept for distribution.
\end{itemize}

The present package provides a simple interface
to make child files individually compilable by \LaTeX{}.
Compiling a child file then has the same effect as compiling
the main file with an |\includeonly| command
to select the appropriate child.
Moreover the generated document will carry the name of the child
rather than the main file.
This resolves all three above issues.

This feature is meant to make the editing of books,
thesis documents and lecture notes somewhat more convenient.
However, the package can also be used efficiently for
composing a series of documents (such as exercise sheets)
which are typically distributed individually.
It then assists the author in generating the individual documents
(potentially in different versions)
as well as a document containing the collected series.
Another application is in developing style files
or other kinds of included material
where compilation of the style file could redirect
to a sample or test file.

%%%%%%%%%%%%%%%%%%%%%%%%%%%%%%%%%%%%%%%%%%%%%%%%%%%%%%%%%%%%%%%%%%%%%%%%%%%%%%%%
%%%%%%%%%%%%%%%%%%%%%%%%%%%%%%%%%%%%%%%%%%%%%%%%%%%%%%%%%%%%%%%%%%%%%%%%%%%%%%%%
\section{Usage}

First of all, the package \textsf{childdoc} is \emph{not} a standard
\LaTeXe{} |.sty| style file! Therefore it needs to be invoked in
a non-standard way.

%%%%%%%%%%%%%%%%%%%%%%%%%%%%%%%%%%%%%%%%%%%%%%%%%%%%%%%%%%%%%%%%%%%%%%%%%%%%%%%%
\subsection{Included Files}
\label{sec:include}

%%%%%%%%%%%%%%%%%%%%%%%%%%%%%%%%%%%%%%%%
\DescribeMacro{\childdocmain}
To use the package, add the commands
\begin{center}
\begin{tabular}{l}
|\input{childdoc.def}|\\
|\childdocmain{}|\\
\end{tabular}
\end{center}
at the very top of the main \LaTeX{} file,
in particular \emph{before} the |\documentclass| statement!
The argument of |\childdocmain| should be left empty
(but it must be present).

%%%%%%%%%%%%%%%%%%%%%%%%%%%%%%%%%%%%%%%%
\DescribeMacro{\childdocof}
Furthermore, add the commands
\begin{center}
\begin{tabular}{l}
|\input{childdoc.def}|\\
|\childdocof{|\textit{main}|}|\\
\end{tabular}
\end{center}
at the top of every child file \textit{child}
which is included by |\include{|\textit{child}|}|
from within the main file
(or at least for those files to be compiled individually).
The argument \textit{main} must be the filename of the main file.

There are a couple of
considerations in setting up the main and child documents:

%%%%%%%%%%%%%%%%%%%%%%%%%%%%%%%%%%%%%%%%
\paragraph{Restrictions.}

Please note the following restrictions:
\begin{itemize}
\item
|\childdocmain| must be called with one argument \textit{main}
to ensure compatibility with earlier version of the package.
It must either be empty (|\childdocmain{}|)
or precisely match the filename of the main file in which it is specified.
See \secref{sec:detection} for further information.
\item
The filename \textit{main} must be specified without the |.tex| extension.
\item
The filename \textit{main} is case sensitive
(even in case-insensitive file systems)
due to internal string comparison.
\item
The argument \textit{main} should be fully expanded, it cannot be a macro.
\item
Subdirectories and special characters should be avoided in filenames.
\item
The command |\childdocmain{|\textit{main}|}| must be followed by a whitespace.
It should not be followed immediately by another command
or by a comment mark `|%|'.
This is because the \TeX{} parser reads the token immediately following
the argument of |\childdocmain| and puts it
at the beginning of every child section;
however, a white\-space is ignored.
\end{itemize}

%%%%%%%%%%%%%%%%%%%%%%%%%%%%%%%%%%%%%%%%
\paragraph{Content of Main File.}

It is advisable to place all content in the child files included by |\include|.
Any output contained in the main file will appear in all child documents
unless suppressed manually;
it cannot be suppressed automatically by the |\includeonly| directive
and thus should normally be avoided.
A method to include some content in the main file
by means of conditional processing is described in \secref{sec:conditional}.

%%%%%%%%%%%%%%%%%%%%%%%%%%%%%%%%%%%%%%%%
\paragraph{Page Numbering.}

When only a part of the document is compiled,
the appropriate numbering of pages
(as well as other status parameters)
is determined from the |.aux| files.
The latter contain information from previous passes.
However this information needs to propagate through
all intermediate child documents.
Therefore the page numbering in child documents may well
be inconsistent until the complete document is compiled at least once.

A useful (if unconventional) way to always ensure a consistent
page numbering is to restart the numbering in each child document
and denote the pages by `\textit{child}|.|\textit{page}'
where \textit{child} represents the chapter/section number of the child file.
This can be achieved by the command
|\numberwithin{page}{|\textit{child}|}|
of the \textsf{amsmath} package
where \textit{child} can be |chapter| or |section|
depending on the chosen structuring.
Alternatively, one can modify the macro |\thepage| appropriately
and reset the counter |page| at the start of each child file.

%%%%%%%%%%%%%%%%%%%%%%%%%%%%%%%%%%%%%%%%%%%%%%%%%%%%%%%%%%%%%%%%%%%%%%%%%%%%%%%%
\subsection{Conditional Processing}
\label{sec:conditional}

The package provides a mechanism to compile different versions
of a document. To customise the versions further some conditional processing
can come in handy to distinguish which version is being compiled.
The package provides two macros to describe the compilation context:

%%%%%%%%%%%%%%%%%%%%%%%%%%%%%%%%%%%%%%%%
\DescribeMacro{\ifchilddoc}
The conditional |\ifchilddoc| distinguishes between the compilation of
child documents and the main document:
%
\begin{center}
|\ifchilddoc |\textit{child-code}| |[|\||else |\textit{main-code}]| \||fi|
\end{center}

%%%%%%%%%%%%%%%%%%%%%%%%%%%%%%%%%%%%%%%%
\DescribeMacro{\childdocname}
\DescribeMacro{\childdocjob}
The macro |\childdocname| contains the filename (without extension)
of the main or child file being processed.
Note that |\childdocjob| will always contain the name of the main file.

%%%%%%%%%%%%%%%%%%%%%%%%%%%%%%%%%%%%%%%%
\paragraph{Title Page.}

Conditional processing can be used to include a title or banner page
in the main document when proper precautions are taken.
Importantly, the code in the main file should ensure that the page counter
(as well as other status parameters which are stored in the |.aux| files)
takes the same value after the conditional processing.
Otherwise the page numbers may take divergent values
depending on which part is compiled.

For example, a title page could be declared by:
%
\begin{center}
\begin{tabular}{l}
|\ifchilddoc\||else|\\
|\addtocounter{page}{-1}|\\
\textit{code for title page}\\
|\newpage|\\
|\||fi|
\end{tabular}
\end{center}
%
A banner page for the child documents can be generated by:
%
\begin{center}
\begin{tabular}{l}
|\ifchilddoc|\\
|\addtocounter{page}{-1}|\\
\textit{code for banner page}\\
|\newpage|\\
|\||fi|
\end{tabular}
\end{center}
%
Here one could write a message such as:
\begin{center}
|This is the part \childdocname{} of \childdocjob{}.|
\end{center}

%%%%%%%%%%%%%%%%%%%%%%%%%%%%%%%%%%%%%%%%%%%%%%%%%%%%%%%%%%%%%%%%%%%%%%%%%%%%%%%%
\subsection{Flags}
\label{sec:flags}

The package makes it easy to generate different versions
of the main or child documents.
To this end compilation flags can be defined
and assigned different default values.
They will be particularly useful in conjunction
with the forwarding mechanism described in \secref{sec:forward}.

For example, it may be useful to have a flag |\version|
which can be set to |draft| or |final|.
The document source will contain some conditional code
depending on the value of |\version|.
Suppose further, the flag should default to |final| for the main file
and to |draft| for child files
which is a natural assignment for editing the document.
This is achieved by placing the following code
in the preamble of the main document
(below the |\childdocmain| directive):
%
\begin{center}
\begin{tabular}{l}
|\ifchilddoc|\\
|\providecommand{\version}{draft}|\\
|\||else|\\
|\providecommand{\version}{final}|\\
|\||fi|
\end{tabular}
\end{center}
%
The definition by |\providecommand| makes sure
that previous definitions are not overwritten.
Further statements |\providecommand{\version}{...}|
can thus be added before the above code to override it.

For the main file, one might add a line
(between |\childdocmain| and the above block)
%
\begin{center}
|%\ifchilddoc\||else\providecommand{\version}{draft}\||fi|
\end{center}
%
which can be uncommented to produce a draft version.
Likewise one can add a line to the very top of a child file
(above the |\childdocof{|\textit{main}|}| directive)
%
\begin{center}
|%\providecommand{\version}{final}|
\end{center}
%
which can be uncommented to produce the final version of this child document.

%%%%%%%%%%%%%%%%%%%%%%%%%%%%%%%%%%%%%%%%%%%%%%%%%%%%%%%%%%%%%%%%%%%%%%%%%%%%%%%%
\subsection{Forwarding}
\label{sec:forward}

Different versions of the main or child documents
using compilation flags as described in \secref{sec:flags}
can be (permanently) stored in different files
for convenient compilation, viewing and distribution.
To this end, the package defines a command
to pass on compilation to a different file:

%%%%%%%%%%%%%%%%%%%%%%%%%%%%%%%%%%%%%%%%
\DescribeMacro{\childdocforward}
The command |\childdocforward| redirects processing to
another source file:
%
\begin{center}
\begin{tabular}{l}
|\input{childdoc.def}|\\
|\childdocforward[|\textit{main}|]{|\textit{dest}|}|\\
\end{tabular}
\end{center}
%
The argument \textit{dest} is the destination file
(without extension).
It should be the main file or one of the child files.
Note that further \textsf{childdoc} directives
such as |\childdocof| and |\childdocforward|
in the indicated file will be processed in this form.
The optional argument \textit{main}
passes on directly to the main file \textit{main}
while pretending to compile the child \textit{dest}.
This form behaves as if \textit{dest}
issues |\childdocof{|\textit{main}|}| right away,
and no further \textsf{childdoc} directives will be processed.

%%%%%%%%%%%%%%%%%%%%%%%%%%%%%%%%%%%%%%%%
\DescribeMacro{\...prefix}
In the alternative form |\childdocforwardprefix|,
%
\begin{center}
\begin{tabular}{l}
|\input{childdoc.def}|\\
|\childdocforwardprefix[|\textit{main}|]{|\textit{prefix}|}{|\textit{dest}|}|
\end{tabular}
\end{center}
%
the destination file is determined by a pattern
depending on the current file:
To make this work, the current file must be called
`{\textit{prefix}\hspace{0.2em}\textit{suffix}}'
with \textit{prefix} matching precisely the argument.
Processing is then passed on to the file
`{\textit{dest}\hspace{0.2em}\textit{suffix}}'.
Surely, the same effect is achieved by
directly specifying the
argument `{\textit{dest}\hspace{0.2em}\textit{suffix}}'
in the first form.
However, that requires to set up a different file
for each child. With the alternative form of the command
all these files can have exactly the same content
which simplifies setting them up and maintaining them.

For example, the following file |draft.tex|
with a compilation flag |\version| as described in \secref{sec:flags}
compiles the main document as a draft:
%
\begin{center}
\begin{tabular}{l}
|\def\version{draft}|\\
|\input{childdoc.def}|\\
|\childdocforward{|\textit{main}|}|
\end{tabular}
\end{center}
%
Likewise, the following files |final|\textit{nn}|.tex|
compile the final version of the child document
|child|\textit{nn}|.tex|:
%
\begin{center}
\begin{tabular}{l}
|\def\version{final}|\\
|\input{childdoc.def}|\\
|\childdocforwardprefix{final}{child}|
\end{tabular}
\end{center}
%

Note that when several versions of a main file and/or of each child file
are to be generated, it may be convenient to set up a |Makefile| or
shell script to automatise the process.

%%%%%%%%%%%%%%%%%%%%%%%%%%%%%%%%%%%%%%%%%%%%%%%%%%%%%%%%%%%%%%%%%%%%%%%%%%%%%%%%
\subsection{Command Line Processing}
\label{sec:commandline}

The effect of redirection files can also be achieved by invoking
the \LaTeX{} compiler with a more elaborate command line.
Most conveniently this should be done as part
of a shell script or a |Makefile|.

When using \textsf{childdoc} in the main file, the following
command lines effectively perform a redirection
(note that depending on the shell being used,
backslashes may have to be doubled: `|\|' $\to$ `|\\|'):
%
\begin{center}
|... -jobname "|\textit{target}|" |\\|"|[\textit{flags}]%
|\input{childdoc.def}\childdocforward[|\textit{main}|]{|\textit{dest}|}"|
\end{center}
%
Here \textit{target} is the name of the output file,
\textit{main} is the name of the main file
and \textit{dest} is the name of the main or child file to be processed
(all filenames without extensions).
The optional argument \textit{main} can be omitted
if \textit{main} matches \textit{dest}.
Optionally, compilation \textit{flags} can be defined via |\def| commands.
This command line makes the \TeX{} engine believe
it is compiling the file \textit{target}
whose content is specified as the latter parameter.
The provided code then forwards the processing to
\textit{main} or \textit{dest} as described in \secref{sec:forward}.

%%%%%%%%%%%%%%%%%%%%%%%%%%%%%%%%%%%%%%%%%%%%%%%%%%%%%%%%%%%%%%%%%%%%%%%%%%%%%%%%
\subsection{Include by Input}
\label{sec:input}

Including child documents by |\include| has some restrictions by design.
Most notably, the content of a child document always occupies
its own set of pages; pages cannot be shared between child documents.
Usually, this behaviour makes perfect sense
because each child document contain an essential part of the document.
However, in some situations it may be desirable to compose
a document from a collection of parts
without having mandatory page breaks between then.
For this case, the package
provides a mechanism to include parts
by |\input| which can also be processed individually.
However, by construction this mechanism
requires manual handling of the content to be output.

%%%%%%%%%%%%%%%%%%%%%%%%%%%%%%%%%%%%%%%%
\DescribeMacro{\ifchilddocmanual}
The main file should be prepared as usual, see \secref{sec:include}.
However, the document body must make a distinction
between processing of an individual part and of the main document, e.g.:
%
\begin{center}
\begin{tabular}{l}
|\ifchilddocmanual|\\
|\input{\childdocname}|\\
|\||else|\\
\textit{document body with }|\input{|\textit{part}|}|\\
|\||fi|
\end{tabular}
\end{center}
%
The conditional |\ifchilddocmanual| is true whenever
a part to be included by |\input| is being compiled,
and the name of the part is stored in |\childdocname|.

%%%%%%%%%%%%%%%%%%%%%%%%%%%%%%%%%%%%%%%%
\DescribeMacro{\childdocby}
Each part to be included by |\input| should start with:
%
\begin{center}
\begin{tabular}{l}
|\input{childdoc.def}|\\
|\childdocby{|\textit{main}|}|\\
\end{tabular}
\end{center}
%
The directive |\childdocby| is similar to |\childdocof|
described in \secref{sec:include},
but the subsequent selection of content must be done manually.
To that end, both |\ifchilddoc| and |\ifchilddocmanual|
will be true upon processing of a part,
and the name of the part is stored in |\childdocname|.
Note that |\jobname| will be set to the filename of the current part
so that each part receives an individual |.aux| file
that does not interfere with the |.aux| file(s) of the main document.
This behaviour can be altered by the alternative form
|\childdocby[*]{|\textit{main}|}| (with a non-empty optional argument)
which uses the |.aux| file of the main document
by setting |\jobname| to \textit{main}.

%%%%%%%%%%%%%%%%%%%%%%%%%%%%%%%%%%%%%%%%%%%%%%%%%%%%%%%%%%%%%%%%%%%%%%%%%%%%%%%%
\subsection{Driver Development}
\label{sec:driver}

The \textsf{childdoc} mechanism can also be use for the development
of definition files such as \LaTeX{} styles or classes.
This case differs from the above setup with multiple parts
included by |\include| in that no |\includeonly| should be invoked.
This can be achieved by starting the include file
(before |\ProvidesPackage|) with:
%
\begin{center}
\begin{tabular}{l}
|\input{childdoc.def}|\\
|\childdocforward{|\textit{main}|}|\\
\end{tabular}
\end{center}
%
or alternatively with:
%
\begin{center}
\begin{tabular}{l}
|\input{childdoc.def}|\\
|\childdocby{|\textit{main}|}|\\
\end{tabular}
\end{center}
%
Both forms have slightly different effects as described above.
The main file is prepared as usual, see \secref{sec:include}.

%%%%%%%%%%%%%%%%%%%%%%%%%%%%%%%%%%%%%%%%%%%%%%%%%%%%%%%%%%%%%%%%%%%%%%%%%%%%%%%%
\subsection{Legacy Detection}
\label{sec:detection}

The directive |\childdocmain| in the main file can detect
whether the complete document or merely a child is to be compiled
even without using the directive |\childdocof|.
This method is deprecated because it is less robust
and there is no compelling reason to use it;
it is merely provided for backward compatibility
and it may be removed in future versions.

If the detection mechanism is to be used,
it is mandatory to correctly specify
the filename of the main file as the argument of |\childdocmain|:
%
\begin{center}
\begin{tabular}{l}
|\input{childdoc.def}|\\
|\childdocmain{|\textit{main}|}|\\
\end{tabular}
\end{center}
%
If |\jobname| does not match the argument \textit{main} of |\childdocmain|,
it is assumed that |\jobname| points to the child file to be compiled.
When using |\childdocmain| with the main file specified as argument,
it suffices to start a child file
with just |\input{|\textit{main}|}|
without loading of the package and using |\childdocof|.
If instead all processing is done
with the appropriate \textsf{childdoc} directives,
the argument of \textit{main} of |\childdocmain| can be empty.

An alternative version of the command line processing described
in \secref{sec:commandline} using the detection mechanism reads:
%
\begin{center}
|... -jobname "|\textit{target}|" "|[\textit{flags}]%
[|\def\jobname{|\textit{dest}|}|]|\input{|\textit{main}|}"|
\end{center}

%%%%%%%%%%%%%%%%%%%%%%%%%%%%%%%%%%%%%%%%%%%%%%%%%%%%%%%%%%%%%%%%%%%%%%%%%%%%%%%%
\subsection{Manual Code}
\label{sec:manual}

In case one cannot be certain whether the definitions file |childdoc.def|
is installed on the target \TeX{} distribution
and one prefers not to ship it,
it is conceivable to paste a few relevant commands into the sources.

To that end, drop all statements |\input{childdoc.def}|
and perform the replacements as outlined below.
Instead of |\childdocmain{|\textit{main}|}| add the following code
to the top of the main file:
%
\begin{center}
\begin{tabular}{l}
|\||ifdefined\childdocname\endinput\||fi\newif\ifchilddoc|\\
|\edef\childdocname{\scantokens\expandafter{\jobname\noexpand}}|\\
|\def\childdocmain{|\textit{main}|}\||ifx\childdocmain\childdocname\||else|\\
|\childdoctrue\includeonly{\childdocname}\let\jobname\childdocmain\||fi|\\
\end{tabular}
\end{center}
%
Instead of |\childdocof{|\textit{main}|}| just include the main file
at the top of each child file:
%
\begin{center}
|\input{|\textit{main}|}|
\end{center}
%
A simple redirection |\childdocforward{|\textit{dest}|}| is achieved by:
%
\begin{center}
|\def\jobname{|\textit{dest}|}\input{\jobname}|
\end{center}
%
The redirection with prefix
|\childdocforwardprefix[|\textit{prefix}|]{|\textit{dest}|}|
is accomplished by:
%
\begin{center}
\begin{tabular}{l}
|{\edef\jobname{\scantokens\expandafter{\jobname\noexpand}}|\\
|\def\redirectjob |\textit{prefix}|#1~~~{\gdef\jobname{|\textit{dest}|#1}}|\\
|\expandafter\redirectjob\jobname~~~}\input{\jobname}|
\end{tabular}
\end{center}

In an alternative approach,
child documents can be compiled by a specific command line
without additional code or specific definitions:
%
\begin{center}
|... -jobname "|\textit{target}|" "|[\textit{flags}]%
|\includeonly{|\textit{dest}|}\input{|\textit{main}|}"|
\end{center}
%

%%%%%%%%%%%%%%%%%%%%%%%%%%%%%%%%%%%%%%%%%%%%%%%%%%%%%%%%%%%%%%%%%%%%%%%%%%%%%%%%
%%%%%%%%%%%%%%%%%%%%%%%%%%%%%%%%%%%%%%%%%%%%%%%%%%%%%%%%%%%%%%%%%%%%%%%%%%%%%%%%
\section{Information}

%%%%%%%%%%%%%%%%%%%%%%%%%%%%%%%%%%%%%%%%%%%%%%%%%%%%%%%%%%%%%%%%%%%%%%%%%%%%%%%%
\subsection{Copyright}

Copyright \copyright{} 2017--2018 Niklas Beisert

This work may be distributed and/or modified under the
conditions of the \LaTeX{} Project Public License, either version 1.3
of this license or (at your option) any later version.
The latest version of this license is in
  \url{http://www.latex-project.org/lppl.txt}
and version 1.3 or later is part of all distributions of \LaTeX{}
version 2005/12/01 or later.

This work has the LPPL maintenance status `maintained'.

The Current Maintainer of this work is Niklas Beisert.

This work consists of the files |README.txt|, |childdoc.ins| and |childdoc.dtx|
as well as the derived files |childdoc.def|, |cdocsamp.tex|
with |cdocsch1.tex|, |cdocsch2.tex|, |cdocspt3.tex|, |cdocspt4.tex|,
|cdocsdrf.tex|, |cdocsfn1.tex|, |cdocsfn2.tex|
as well as |childdoc.pdf|.

%%%%%%%%%%%%%%%%%%%%%%%%%%%%%%%%%%%%%%%%%%%%%%%%%%%%%%%%%%%%%%%%%%%%%%%%%%%%%%%%
\subsection{Files and Installation}

The package consists of the files:
%
\begin{center}
\begin{tabular}{ll}
    |README.txt|   & readme file \\
    |childdoc.ins| & installation file \\
    |childdoc.dtx| & source file \\
    |childdoc.def| & definition file \\
    |cdocsamp.tex| & sample main file \\
    |cdocsch1.tex| & sample include file \\
    |cdocsch2.tex| & sample include file \\
    |cdocspt3.tex| & sample part file \\
    |cdocspt4.tex| & sample part file \\
    |cdocsdrf.tex| & sample redirection file \\
    |cdocsfn1.tex| & sample redirection file \\
    |cdocsfn2.tex| & sample redirection file \\
    |childdoc.pdf| & manual
\end{tabular}
\end{center}
%
The distribution consists of the files
|README.txt|, |childdoc.ins| and |childdoc.dtx|.
%
\begin{itemize}
\item
Run (pdf)\LaTeX{} on |childdoc.dtx|
to compile the manual |childdoc.pdf| (this file).
\item
Run \LaTeX{} on |childdoc.ins| to create the definitions file |childdoc.def|
and the sample |cdocsamp.tex| with include files
|cdocsch1.tex|, |cdocsch2.tex|, |cdocspt3.tex|, |cdocspt4.tex|,
|cdocsdrf.tex|, |cdocsfn1.tex|, |cdocsfn2.tex|.
Then copy the file |childdoc.def| to an appropriate directory of your \LaTeX{}
distribution, e.g.\ \textit{texmf-root}|/tex/latex/childdoc|.
\end{itemize}

%%%%%%%%%%%%%%%%%%%%%%%%%%%%%%%%%%%%%%%%%%%%%%%%%%%%%%%%%%%%%%%%%%%%%%%%%%%%%%%%
\subsection{Related CTAN Packages}

There are several other packages which offer a similar functionality:
%
\begin{itemize}
\item
The packages
\href{http://ctan.org/pkg/docmute}{\textsf{docmute}},
\href{http://ctan.org/pkg/includex}{\textsf{includex}} and
\href{http://ctan.org/pkg/standalone}{\textsf{standalone}}
provide commands to include only the document body of
a child file thus allowing both files to be compiled individually.
\item
The packages \href{http://ctan.org/pkg/subdocs}{\textsf{subdocs}}
and \href{http://ctan.org/pkg/subfiles}{\textsf{subfiles}}
provide structures in which the main and child documents can be
encapsulated and allowing them to be compiled individually.
The inclusion mechanism is different from the conventional |\include|.
\item
The package \href{http://ctan.org/pkg/combine}{\textsf{combine}}
is an elaborate solution to combine several documents into one.
\end{itemize}
%
See also the CTAN topic \href{http://ctan.org/topic/subdocs}{\textsf{subdocs}}
for further related packages.
The present package differs from the above solutions in that
a document structure constructed with the conventional |\include| mechanism
just needs two extra commands at the top of every file
such that all constituent files can be compiled individually.

%%%%%%%%%%%%%%%%%%%%%%%%%%%%%%%%%%%%%%%%%%%%%%%%%%%%%%%%%%%%%%%%%%%%%%%%%%%%%%%%
%\subsection{Feature Suggestions}
%
%The following is a list of features which may be useful for future
%versions of this package:
%%
%\begin{itemize}
%\item
%\ldots
%\end{itemize}

%%%%%%%%%%%%%%%%%%%%%%%%%%%%%%%%%%%%%%%%%%%%%%%%%%%%%%%%%%%%%%%%%%%%%%%%%%%%%%%%
\subsection{Revision History}

%%%%%%%%%%%%%%%%%%%%%%%%%%%%%%%%%%%%%%%%
\paragraph{v2.0:} 2018/12/30

\begin{itemize}
\item
immediate forward processing
\item
added |\childdocby| mechanism
\item
manual restructured
\end{itemize}

%%%%%%%%%%%%%%%%%%%%%%%%%%%%%%%%%%%%%%%%
\paragraph{v1.6:} 2018/01/17

\begin{itemize}
\item
application for development of include files
\item
corrections to manual
\end{itemize}

%%%%%%%%%%%%%%%%%%%%%%%%%%%%%%%%%%%%%%%%
\paragraph{v1.5:} 2017/05/21

\begin{itemize}
\item
more complete structuring introduced
\item
|\childdocof| introduced
\item
|\childdoc| renamed to |\childdocmain|
\item
|\childredirect| renamed to |\childdocforward| and |\childdocforwardprefix|
and functionality expanded
\end{itemize}

%%%%%%%%%%%%%%%%%%%%%%%%%%%%%%%%%%%%%%%%
\paragraph{v1.0:} 2017/04/27

\begin{itemize}
\item
manual and install package
\item
first version published on CTAN
\end{itemize}

%%%%%%%%%%%%%%%%%%%%%%%%%%%%%%%%%%%%%%%%
\paragraph{v0.6:} 2017/04/26

\begin{itemize}
\item
redirection mechanism added
\end{itemize}

%%%%%%%%%%%%%%%%%%%%%%%%%%%%%%%%%%%%%%%%
\paragraph{v0.5:} 2017/04/26

\begin{itemize}
\item
functionality in definition file
\end{itemize}


%%%%%%%%%%%%%%%%%%%%%%%%%%%%%%%%%%%%%%%%%%%%%%%%%%%%%%%%%%%%%%%%%%%%%%%%%%%%%%%%
%%%%%%%%%%%%%%%%%%%%%%%%%%%%%%%%%%%%%%%%%%%%%%%%%%%%%%%%%%%%%%%%%%%%%%%%%%%%%%%%
%%%%%%%%%%%%%%%%%%%%%%%%%%%%%%%%%%%%%%%%%%%%%%%%%%%%%%%%%%%%%%%%%%%%%%%%%%%%%%%%
\appendix

\settowidth\MacroIndent{\rmfamily\scriptsize 000\ }

 \DocInput{childdoc.dtx}

\end{document}
%</driver>
% \fi
%
% %%%%%%%%%%%%%%%%%%%%%%%%%%%%%%%%%%%%%%%%%%%%%%%%%%%%%%%%%%%%%%%%%%%%%%%%%%%%%%
% %%%%%%%%%%%%%%%%%%%%%%%%%%%%%%%%%%%%%%%%%%%%%%%%%%%%%%%%%%%%%%%%%%%%%%%%%%%%%%
% \section{Sample}
%\iffalse
%<*samplemain>
%\fi
%
% The following presents a sample document
% with two chapters, two parts, a title page,
% a compile flag as well as three forwarding files to set the flag.
% It consists of eight |.tex| files:
% \begin{center}
% \begin{tabular}{ll}
% |cdocsamp.tex|&main file\\
% |cdocsch1.tex|&include file for chapter 1\\
% |cdocsch2.tex|&include file for chapter 2\\
% |cdocspt3.tex|&include file for part 3\\
% |cdocspt4.tex|&include file for part 4\\
% |cdocsdrf.tex|&forwarding file for main file in draft mode\\
% |cdocsfi1.tex|&forwarding file for final version of chapter 1\\
% |cdocsfi2.tex|&forwarding file for final version of chapter 2\\
% \end{tabular}
% \end{center}
% Each of the eight files can be compiled directly by the \LaTeX{} compiler.
%
% %%%%%%%%%%%%%%%%%%%%%%%%%%%%%%%%%%%%%%
% \paragraph{Main File.}
%
% The main file is called |cdocsamp.tex|.
%
% Load the \textsf{childdoc} definitions and
% declare the filename for the main document:
%    \begin{macrocode}
\input{childdoc.def}
\childdocmain{}
%    \end{macrocode}

% Optional override for |\version| flag:
%    \begin{macrocode}
%%\ifchilddoc\else\providecommand{\version}{draft}\fi
%    \end{macrocode}

% Define the default values for the |\version| flag
% (|final| for the main file and |draft| for childs):
%    \begin{macrocode}
\ifchilddoc
\providecommand{\version}{draft}
\else
\providecommand{\version}{final}
\fi
%    \end{macrocode}

% Load the standard document class:
%    \begin{macrocode}
\documentclass[12pt]{article}
%    \end{macrocode}

% Start the document body:
%    \begin{macrocode}
\begin{document}
%    \end{macrocode}

% Declare a title page.
% Print title, part of document being processed and version flag:
%    \begin{macrocode}
\addtocounter{page}{-1}
\begin{center}
{\LARGE\bfseries{}childdoc example\par}
\vspace{1cm}
\ifchilddoc
\ifchilddocmanual part\else chapter\fi:
`\childdocname' of `\childdocjob'\par
\else
main document: `\childdocjob'\par
\fi
version: \version\par
\end{center}
\newpage
%    \end{macrocode}

% Manually include selected file,
% otherwise process as usual:
%    \begin{macrocode}
\ifchilddocmanual
\section*{part `\childdocname'}
\input{\childdocname}
\else
%    \end{macrocode}

% Include the two chapters:
%    \begin{macrocode}
\include{cdocsch1}
\include{cdocsch2}
%    \end{macrocode}

% Include the two parts unless only chapters should be displayed:
%    \begin{macrocode}
\ifchilddoc\else
\section{part three}
\input{cdocspt3}
\section{part four}
\input{cdocspt4}
\fi
%    \end{macrocode}

% Process as usual until here:
%    \begin{macrocode}
\fi
%    \end{macrocode}

% End of document body:
%    \begin{macrocode}
\end{document}
%    \end{macrocode}
%\iffalse
%</samplemain>
%\fi
%
% %%%%%%%%%%%%%%%%%%%%%%%%%%%%%%%%%%%%%%
% \paragraph{Chapter Include Files.}
%
% The include files are called |cdocsch1.tex| and |cdocsch2.tex|.
%
%\iffalse
%<*samplechap1|samplechap2>
%\fi

% Optional override for |\version| flag:
%    \begin{macrocode}
%%\providecommand{\version}{final}
%    \end{macrocode}

% Include the main document:
%    \begin{macrocode}
\input{childdoc.def}
\childdocof{cdocsamp}
%    \end{macrocode}

%\iffalse
%</samplechap1|samplechap2>
%\fi
%
%\iffalse
%<*samplechap1>
%\fi
% Some text for chapter 1:
%    \begin{macrocode}
\section{one}
some text in chapter one
%    \end{macrocode}

%\iffalse
%</samplechap1>
%\fi
% Some text for chapter 2:
%\iffalse
%<*samplechap2>
%\fi
%    \begin{macrocode}
\section{two}
more text in chapter two
%    \end{macrocode}

%\iffalse
%</samplechap2>
%\fi
%
% %%%%%%%%%%%%%%%%%%%%%%%%%%%%%%%%%%%%%%
% \paragraph{Part Include Files.}
%
% The include files are called |cdocspt3.tex| and |cdocspt4.tex|.
%
%\iffalse
%<*samplepart3|samplepart4>
%\fi

% Optional override for |\version| flag:
%    \begin{macrocode}
%%\providecommand{\version}{final}
%    \end{macrocode}

% Include the main document:
%    \begin{macrocode}
\input{childdoc.def}
\childdocby{cdocsamp}
%    \end{macrocode}

%\iffalse
%</samplepart3|samplepart4>
%\fi
%
%\iffalse
%<*samplepart3>
%\fi
% Some text for part 3:
%    \begin{macrocode}
some text in part three
%    \end{macrocode}

%\iffalse
%</samplepart3>
%\fi
% Some text for part 4:
%\iffalse
%<*samplepart4>
%\fi
%    \begin{macrocode}
more text in part four
%    \end{macrocode}

%\iffalse
%</samplepart4>
%\fi
%
% %%%%%%%%%%%%%%%%%%%%%%%%%%%%%%%%%%%%%%
% \paragraph{Forwarding for a Complete Draft.}
%
% The following forwarding file |cdocsdrf.tex|
% compiles the main document in draft mode:
%\iffalse
%<*sampledraft>
%\fi
%    \begin{macrocode}
\def\version{draft}
\input{childdoc.def}
\childdocforward{cdocsamp}
%    \end{macrocode}

%\iffalse
%</sampledraft>
%\fi
%
% %%%%%%%%%%%%%%%%%%%%%%%%%%%%%%%%%%%%%%
% \paragraph{Forwarding for Final Version of the Chapters.}
%
% The following forwarding files |cdocsfn1.tex| and |cdocsfn2.tex|
% (with identical content)
% compile the final versions of the child documents
% |cdocsch1.tex| and |cdocsch2.tex|, respectively:
%\iffalse
%<*samplefinal>
%\fi
%    \begin{macrocode}
\def\version{final}
\input{childdoc.def}
\childdocforwardprefix[cdocsamp]{cdocsfn}{cdocsch}
%    \end{macrocode}

%\iffalse
%</samplefinal>
%\fi
%
% %%%%%%%%%%%%%%%%%%%%%%%%%%%%%%%%%%%%%%
% \paragraph{Command Line Processing.}
%
% The following three command lines generate the output files
% |cdocscld|, |cdocscl1| and |cdocscl2|
% which should be identical to
% |cdocsdrf|, |cdocsch1| and |cdocsfn2|, respectively:
% \begin{center}
% \begin{tabular}{l}
% |latex -jobname cdocscld \|\\
% |  "\def\version{draft}\input{childdoc.def}\childdocforward{cdocsamp}"|\\
% |latex -jobname cdocscl1 \|\\
% |  "\input{childdoc.def}\childdocforward[cdocsamp]{cdocsch1}"|\\
% |latex -jobname cdocscl2 \|\\
% |  "\def\version{final}\input{childdoc.def}\childdocforward{cdocsch2}"|
% \end{tabular}
% \end{center}
% Note that the trailing backslash on each first line
% merely continues the input to the second line
% (for convenient cut ant paste).
% Furthermore, the command |latex| can be replaced by any
% of its alternative versions such as |pdflatex|.
%
% %%%%%%%%%%%%%%%%%%%%%%%%%%%%%%%%%%%%%%%%%%%%%%%%%%%%%%%%%%%%%%%%%%%%%%%%%%%%%%
% %%%%%%%%%%%%%%%%%%%%%%%%%%%%%%%%%%%%%%%%%%%%%%%%%%%%%%%%%%%%%%%%%%%%%%%%%%%%%%
% \section{Implementation}
%\iffalse
%<*package>
%\fi
%
% This section describes the definitions file |childdoc.def|.

% The definitions cannot be loaded using |\usepackage| or |\RequirePackage|
% which has a mechanism to prevent loading a style file more than once.
% When loading the definitions by means of |\input|
% multiple instances have to be prevented manually:
%\iffalse
%This code needs to be before the `\ProvidesFile' directive
%which is defined at the beginning of this file.
%Therefore it is also placed there and commented out here.
%</package>
%<*discard>
%\fi
%    \begin{macrocode}
\ifdefined\childdocmain\endinput\fi
%    \end{macrocode}
%\iffalse
%</discard>
%<*package>
%\fi
%
% \macro{\ifchilddoc}
% \macro{\ifchilddocmanual}
% The conditional |\ifchilddoc| tells whether a
% child (true) or main (false) document is being compiled.
% The conditional |\ifchilddocmanual| tells whether
% the |\includeonly| mechanism is used (false) or
% the selection of child files must be performed manually (true).
% The definitions initialise to false:
%    \begin{macrocode}
\newif\ifchilddoc
\newif\ifchilddocmanual
%    \end{macrocode}

% \macro{\childdocname}
% \macro{\childdocjob}
% The macro |\childdocname| stores the name of the main document
% to be compiled. The macro |\childdocjob| stores the name of
% the document on which the \LaTeX{} compiler was originally invoked.
% The content of |\jobname| cannot be compared
% to filenames specified in the source due to different catcodes.
% The following code rescans |\jobname|, stores the result
% in |\childdocname| and saves a copy in |\childdocjob|:
%    \begin{macrocode}
\edef\childdocname{\scantokens\expandafter{\jobname\noexpand}}
\let\childdocjob\childdocname
%    \end{macrocode}

% \macro{\childdocdisable}
% The macro |\childdocdisable| prevents the main file
% from being processed more than once.
% At this stage, the main document command |\childdocmain|
% is assumed to be called once again where it should do nothing.
% Any subsequent call to it should prevent
% a secondary processing of the main document
% It overwrites the forwarding commands
% |\childdocof| and |\childdocforward|
% with empty macros to prevent further inclusions of the main document:
%    \begin{macrocode}
\newcommand{\childdocdisable}
{
  \renewcommand{\childdocmain}[1]{\renewcommand{\childdocmain}[1]{\endinput}}
  \renewcommand{\childdocof}[1]{}
  \renewcommand{\childdocby}[2][]{}
  \renewcommand{\childdocforward}[2][]{}
  \renewcommand{\childdocdisable}{}
}
%    \end{macrocode}

% \macro{\childdocmain}
% The macro |\childdocmain| is to be called at the top of the main file
% with nothing or the main filename (without extension) as argument.
% First, it breaks loops.
% If the argument is not empty and does not match |\childdocname|
% (which is set by the first inclusion of |childdoc.def|),
% |\ifchilddoc| is set to true, |\includeonly| is applied to the child file
% and |\jobname| is set to the main file
% (for proper handling of |.aux| files):
%    \begin{macrocode}
\newcommand{\childdocmain}[1]
{
  \childdocdisable\childdocmain{}
  \if?#1?\else
    \begingroup
      \def\childdoctmp{#1}
      \ifx\childdoctmp\childdocname
        \def\childdoctmp{}
      \else
        \def\childdoctmp
        {
          \childdoctrue
          \includeonly{\childdocname}
          \def\childdocjob{#1}
          \def\jobname{#1}
        }
      \fi
      \expandafter
    \endgroup
    \childdoctmp
  \fi
}
%    \end{macrocode}

% \macro{\childdocof}
% The command |\childdocof| redirects
% compilation to the main file |#1|.
%    \begin{macrocode}
\newcommand{\childdocof}[1]
{
  \childdocdisable
  \childdoctrue
  \includeonly{\childdocname}
  \def\jobname{#1}
  \def\childdocjob{#1}
  \input{#1}
}
%    \end{macrocode}

% \macro{\childdocby}
% The command |\childdocby| ....
%    \begin{macrocode}
\newcommand{\childdocby}[2][]
{
  \childdocdisable
  \childdoctrue
  \childdocmanualtrue
  \if?#1?\else
    \def\jobname{#2}
  \fi
  \def\childdocjob{#2}
  \input{#2}
  \endinput
}
%    \end{macrocode}

% \macro{\childdocforward}
% The command |\childdocforward| redirects
% compilation to the main file or
% (if the optional argument is given) a child file.
% Parameters are set as if the main file
% or a child file starting with |\childdocof| was compiled.
% Then compilation is handed over to the main file:
%    \begin{macrocode}
\newcommand{\childdocforward}[2][]
{
  \begingroup
    \if?#1?
      \def\childdoctmp
      {
        \def\childdocname{#2}
        \def\childdocjob{#2}
        \def\jobname{#2}
        \input{#2}
        \endinput
      }
    \else
      \def\childdoctmp
      {
        \childdocdisable
        \def\childdocname{#2}
        \childdoctrue
        \includeonly{#2}
        \def\childdocjob{#1}
        \def\jobname{#1}
        \input{#1}
        \endinput
      }
    \fi
    \expandafter
  \endgroup
  \childdoctmp
}
%    \end{macrocode}

% \macro{\childdocforwardprefix}
% The command |\childdocforwardprefix| redirects
% compilation to the main or a child file by means of a pattern.
% The prefix |#1| in the current filename is replaced by |#2|
% and the suffix of the current filename is kept
% (it is assumed that the filename does not contain the substring `|~~~|'
% which is used as a delimiter).
% Compilation is handed over to the new file by |\childdocforward|:
%    \begin{macrocode}
\newcommand{\childdocforwardprefix}[3][]
{
  \begingroup
    \def\childdocextract #2##1~~~{\def\childdoctmp{\childdocforward[#1]{#3##1}}}
    \expandafter\childdocextract\childdocname~~~
    \expandafter
  \endgroup
  \childdoctmp
}
%    \end{macrocode}

% \macro{\childdoc}
% The deprecated macro |\childdoc| is a legacy version of |\childdocmain|:
%    \begin{macrocode}
\newcommand{\childdoc}{\childdocmain}
%    \end{macrocode}

% \macro{\childdocredirect}
% The deprecated macro |\childdocredirect| is a legacy version
% of |\childdocforward| and |\childdocforwardprefix|:
%    \begin{macrocode}
\newcommand{\childdocredirect}[2][]
{
  \begingroup
    \if?#1?
      \def\childdoctmp{\childdocforward{#2}}
    \else
      \def\childdoctmp{\childdocforwardprefix{#1}{#2}}
    \fi
    \expandafter
  \endgroup
  \childdoctmp
}
%    \end{macrocode}

%\iffalse
%</package>
%\fi
%
\endinput
|\\
|\childdocby{|\textit{main}|}|\\
\end{tabular}
\end{center}
%
The directive |\childdocby| is similar to |\childdocof|
described in \secref{sec:include},
but the subsequent selection of content must be done manually.
To that end, both |\ifchilddoc| and |\ifchilddocmanual|
will be true upon processing of a part,
and the name of the part is stored in |\childdocname|.
Note that |\jobname| will be set to the filename of the current part
so that each part receives an individual |.aux| file
that does not interfere with the |.aux| file(s) of the main document.
This behaviour can be altered by the alternative form
|\childdocby[*]{|\textit{main}|}| (with a non-empty optional argument)
which uses the |.aux| file of the main document
by setting |\jobname| to \textit{main}.

%%%%%%%%%%%%%%%%%%%%%%%%%%%%%%%%%%%%%%%%%%%%%%%%%%%%%%%%%%%%%%%%%%%%%%%%%%%%%%%%
\subsection{Driver Development}
\label{sec:driver}

The \textsf{childdoc} mechanism can also be use for the development
of definition files such as \LaTeX{} styles or classes.
This case differs from the above setup with multiple parts
included by |\include| in that no |\includeonly| should be invoked.
This can be achieved by starting the include file
(before |\ProvidesPackage|) with:
%
\begin{center}
\begin{tabular}{l}
|% \iffalse
%
% childdoc.dtx Copyright (C) 2017-2018 Niklas Beisert
%
% This work may be distributed and/or modified under the
% conditions of the LaTeX Project Public License, either version 1.3
% of this license or (at your option) any later version.
% The latest version of this license is in
%   http://www.latex-project.org/lppl.txt
% and version 1.3 or later is part of all distributions of LaTeX
% version 2005/12/01 or later.
%
% This work has the LPPL maintenance status `maintained'.
%
% The Current Maintainer of this work is Niklas Beisert.
%
% This work consists of the files childdoc.dtx and childdoc.ins
% and the derived files childdoc.def and cdocsamp.tex with
% cdocsch1.tex, cdocsch2.tex, cdocsdrf.tex, cdocsfn1.tex, cdocsfn2.tex.
%
%<package>\ifdefined\childdocmain\endinput\fi
%<package>\ProvidesFile{childdoc.def}[2018/12/30 v2.0 child document driver]
%<samplemain>\ProvidesFile{cdocsamp.tex}[2018/12/30 v2.0 sample for childdoc]
%<*driver>
%\ProvidesFile{childdoc.drv}[2018/12/30 v2.0 childdoc reference manual file]
\PassOptionsToClass{10pt,a4paper}{article}
\documentclass{ltxdoc}

\usepackage[margin=35mm]{geometry}
\usepackage{hyperref}
\usepackage{hyperxmp}
\usepackage[usenames]{color}

\hypersetup{colorlinks=true}
\hypersetup{pdfstartview=FitH}
\hypersetup{pdfpagemode=UseNone}
\hypersetup{pdfsource={}}
\hypersetup{pdflang={en-UK}}
\hypersetup{pdfcopyright={Copyright 2017-2018 Niklas Beisert.
  This work may be distributed and/or modified under the
  conditions of the LaTeX Project Public License, either version 1.3
  of this license or (at your option) any later version.}}
\hypersetup{pdflicenseurl={http://www.latex-project.org/lppl.txt}}
\hypersetup{pdfcontactaddress={ETH Zurich, ITP, HIT K,
  Wolfgang-Pauli-Strasse 27}}
\hypersetup{pdfcontactpostcode={8093}}
\hypersetup{pdfcontactcity={Zurich}}
\hypersetup{pdfcontactcountry={Switzerland}}
\hypersetup{pdfcontactemail={nbeisert@itp.phys.ethz.ch}}
\hypersetup{pdfcontacturl={http://people.phys.ethz.ch/\xmptilde nbeisert/}}

\newcommand{\secref}[1]{\hyperref[#1]{section \ref*{#1}}}

\parskip1ex
\parindent0pt
\let\olditemize\itemize
\def\itemize{\olditemize\parskip0pt}

\begin{document}

\title{The \textsf{childdoc} Package}
\hypersetup{pdftitle={The childdoc Package}}
\author{Niklas Beisert\\[2ex]
  Institut f\"ur Theoretische Physik\\
  Eidgen\"ossische Technische Hochschule Z\"urich\\
  Wolfgang-Pauli-Strasse 27, 8093 Z\"urich, Switzerland\\[1ex]
  \href{mailto:nbeisert@itp.phys.ethz.ch}
  {\texttt{nbeisert@itp.phys.ethz.ch}}}
\hypersetup{pdfauthor={Niklas Beisert}}
\hypersetup{pdfsubject={Manual for the LaTeX2e Package childdoc}}
\date{30 December 2018, \textsf{v2.0}}
\maketitle

\begin{abstract}\noindent
\textsf{childdoc} is a \LaTeXe{} package
that enables the direct compilation
of document sections included by |\include|
to individual files.
\end{abstract}

\begingroup
\parskip0ex
\tableofcontents
\endgroup

%%%%%%%%%%%%%%%%%%%%%%%%%%%%%%%%%%%%%%%%%%%%%%%%%%%%%%%%%%%%%%%%%%%%%%%%%%%%%%%%
%%%%%%%%%%%%%%%%%%%%%%%%%%%%%%%%%%%%%%%%%%%%%%%%%%%%%%%%%%%%%%%%%%%%%%%%%%%%%%%%
\section{Introduction}

\LaTeX{} provides a mechanism to structure a large document (such as a book)
into a main file and several child files (containing the chapters)
using the |\include| command.
This mechanism is beneficial for documents
which span hundreds of pages in order to
make the source file(s) more manageable.
Moreover, compilation can be restricted to
selected child files by means of the |\includeonly| command.
The latter feature can be used to reduce the compilation time while editing
(this was significantly more useful in the earlier days of \LaTeX{})
or to generate a smaller document which is easier to navigate.
Another application of |\includeonly| is to generate
documents consisting of selected parts of the complete document.

However, there are a few drawbacks of the plain |\include| mechanism:
\begin{itemize}
\item
The child files cannot be compiled on their own,
they can only be compiled via the main file.
A naive editing environment
(such as a text editor with an option
to have the current file processed by \LaTeX)
may require one to switch to the main file before compiling;
attempting to compile the child file produces errors.
\item
The main file must be modified (each time)
to adjust the |\includeonly| command
to the present needs. This easily leaves the main file in a messy state.
\item
The generated document will always carry the filename
of the main document. This is inconvenient if
several child files are to be compiled and
to be kept for distribution.
\end{itemize}

The present package provides a simple interface
to make child files individually compilable by \LaTeX{}.
Compiling a child file then has the same effect as compiling
the main file with an |\includeonly| command
to select the appropriate child.
Moreover the generated document will carry the name of the child
rather than the main file.
This resolves all three above issues.

This feature is meant to make the editing of books,
thesis documents and lecture notes somewhat more convenient.
However, the package can also be used efficiently for
composing a series of documents (such as exercise sheets)
which are typically distributed individually.
It then assists the author in generating the individual documents
(potentially in different versions)
as well as a document containing the collected series.
Another application is in developing style files
or other kinds of included material
where compilation of the style file could redirect
to a sample or test file.

%%%%%%%%%%%%%%%%%%%%%%%%%%%%%%%%%%%%%%%%%%%%%%%%%%%%%%%%%%%%%%%%%%%%%%%%%%%%%%%%
%%%%%%%%%%%%%%%%%%%%%%%%%%%%%%%%%%%%%%%%%%%%%%%%%%%%%%%%%%%%%%%%%%%%%%%%%%%%%%%%
\section{Usage}

First of all, the package \textsf{childdoc} is \emph{not} a standard
\LaTeXe{} |.sty| style file! Therefore it needs to be invoked in
a non-standard way.

%%%%%%%%%%%%%%%%%%%%%%%%%%%%%%%%%%%%%%%%%%%%%%%%%%%%%%%%%%%%%%%%%%%%%%%%%%%%%%%%
\subsection{Included Files}
\label{sec:include}

%%%%%%%%%%%%%%%%%%%%%%%%%%%%%%%%%%%%%%%%
\DescribeMacro{\childdocmain}
To use the package, add the commands
\begin{center}
\begin{tabular}{l}
|\input{childdoc.def}|\\
|\childdocmain{}|\\
\end{tabular}
\end{center}
at the very top of the main \LaTeX{} file,
in particular \emph{before} the |\documentclass| statement!
The argument of |\childdocmain| should be left empty
(but it must be present).

%%%%%%%%%%%%%%%%%%%%%%%%%%%%%%%%%%%%%%%%
\DescribeMacro{\childdocof}
Furthermore, add the commands
\begin{center}
\begin{tabular}{l}
|\input{childdoc.def}|\\
|\childdocof{|\textit{main}|}|\\
\end{tabular}
\end{center}
at the top of every child file \textit{child}
which is included by |\include{|\textit{child}|}|
from within the main file
(or at least for those files to be compiled individually).
The argument \textit{main} must be the filename of the main file.

There are a couple of
considerations in setting up the main and child documents:

%%%%%%%%%%%%%%%%%%%%%%%%%%%%%%%%%%%%%%%%
\paragraph{Restrictions.}

Please note the following restrictions:
\begin{itemize}
\item
|\childdocmain| must be called with one argument \textit{main}
to ensure compatibility with earlier version of the package.
It must either be empty (|\childdocmain{}|)
or precisely match the filename of the main file in which it is specified.
See \secref{sec:detection} for further information.
\item
The filename \textit{main} must be specified without the |.tex| extension.
\item
The filename \textit{main} is case sensitive
(even in case-insensitive file systems)
due to internal string comparison.
\item
The argument \textit{main} should be fully expanded, it cannot be a macro.
\item
Subdirectories and special characters should be avoided in filenames.
\item
The command |\childdocmain{|\textit{main}|}| must be followed by a whitespace.
It should not be followed immediately by another command
or by a comment mark `|%|'.
This is because the \TeX{} parser reads the token immediately following
the argument of |\childdocmain| and puts it
at the beginning of every child section;
however, a white\-space is ignored.
\end{itemize}

%%%%%%%%%%%%%%%%%%%%%%%%%%%%%%%%%%%%%%%%
\paragraph{Content of Main File.}

It is advisable to place all content in the child files included by |\include|.
Any output contained in the main file will appear in all child documents
unless suppressed manually;
it cannot be suppressed automatically by the |\includeonly| directive
and thus should normally be avoided.
A method to include some content in the main file
by means of conditional processing is described in \secref{sec:conditional}.

%%%%%%%%%%%%%%%%%%%%%%%%%%%%%%%%%%%%%%%%
\paragraph{Page Numbering.}

When only a part of the document is compiled,
the appropriate numbering of pages
(as well as other status parameters)
is determined from the |.aux| files.
The latter contain information from previous passes.
However this information needs to propagate through
all intermediate child documents.
Therefore the page numbering in child documents may well
be inconsistent until the complete document is compiled at least once.

A useful (if unconventional) way to always ensure a consistent
page numbering is to restart the numbering in each child document
and denote the pages by `\textit{child}|.|\textit{page}'
where \textit{child} represents the chapter/section number of the child file.
This can be achieved by the command
|\numberwithin{page}{|\textit{child}|}|
of the \textsf{amsmath} package
where \textit{child} can be |chapter| or |section|
depending on the chosen structuring.
Alternatively, one can modify the macro |\thepage| appropriately
and reset the counter |page| at the start of each child file.

%%%%%%%%%%%%%%%%%%%%%%%%%%%%%%%%%%%%%%%%%%%%%%%%%%%%%%%%%%%%%%%%%%%%%%%%%%%%%%%%
\subsection{Conditional Processing}
\label{sec:conditional}

The package provides a mechanism to compile different versions
of a document. To customise the versions further some conditional processing
can come in handy to distinguish which version is being compiled.
The package provides two macros to describe the compilation context:

%%%%%%%%%%%%%%%%%%%%%%%%%%%%%%%%%%%%%%%%
\DescribeMacro{\ifchilddoc}
The conditional |\ifchilddoc| distinguishes between the compilation of
child documents and the main document:
%
\begin{center}
|\ifchilddoc |\textit{child-code}| |[|\||else |\textit{main-code}]| \||fi|
\end{center}

%%%%%%%%%%%%%%%%%%%%%%%%%%%%%%%%%%%%%%%%
\DescribeMacro{\childdocname}
\DescribeMacro{\childdocjob}
The macro |\childdocname| contains the filename (without extension)
of the main or child file being processed.
Note that |\childdocjob| will always contain the name of the main file.

%%%%%%%%%%%%%%%%%%%%%%%%%%%%%%%%%%%%%%%%
\paragraph{Title Page.}

Conditional processing can be used to include a title or banner page
in the main document when proper precautions are taken.
Importantly, the code in the main file should ensure that the page counter
(as well as other status parameters which are stored in the |.aux| files)
takes the same value after the conditional processing.
Otherwise the page numbers may take divergent values
depending on which part is compiled.

For example, a title page could be declared by:
%
\begin{center}
\begin{tabular}{l}
|\ifchilddoc\||else|\\
|\addtocounter{page}{-1}|\\
\textit{code for title page}\\
|\newpage|\\
|\||fi|
\end{tabular}
\end{center}
%
A banner page for the child documents can be generated by:
%
\begin{center}
\begin{tabular}{l}
|\ifchilddoc|\\
|\addtocounter{page}{-1}|\\
\textit{code for banner page}\\
|\newpage|\\
|\||fi|
\end{tabular}
\end{center}
%
Here one could write a message such as:
\begin{center}
|This is the part \childdocname{} of \childdocjob{}.|
\end{center}

%%%%%%%%%%%%%%%%%%%%%%%%%%%%%%%%%%%%%%%%%%%%%%%%%%%%%%%%%%%%%%%%%%%%%%%%%%%%%%%%
\subsection{Flags}
\label{sec:flags}

The package makes it easy to generate different versions
of the main or child documents.
To this end compilation flags can be defined
and assigned different default values.
They will be particularly useful in conjunction
with the forwarding mechanism described in \secref{sec:forward}.

For example, it may be useful to have a flag |\version|
which can be set to |draft| or |final|.
The document source will contain some conditional code
depending on the value of |\version|.
Suppose further, the flag should default to |final| for the main file
and to |draft| for child files
which is a natural assignment for editing the document.
This is achieved by placing the following code
in the preamble of the main document
(below the |\childdocmain| directive):
%
\begin{center}
\begin{tabular}{l}
|\ifchilddoc|\\
|\providecommand{\version}{draft}|\\
|\||else|\\
|\providecommand{\version}{final}|\\
|\||fi|
\end{tabular}
\end{center}
%
The definition by |\providecommand| makes sure
that previous definitions are not overwritten.
Further statements |\providecommand{\version}{...}|
can thus be added before the above code to override it.

For the main file, one might add a line
(between |\childdocmain| and the above block)
%
\begin{center}
|%\ifchilddoc\||else\providecommand{\version}{draft}\||fi|
\end{center}
%
which can be uncommented to produce a draft version.
Likewise one can add a line to the very top of a child file
(above the |\childdocof{|\textit{main}|}| directive)
%
\begin{center}
|%\providecommand{\version}{final}|
\end{center}
%
which can be uncommented to produce the final version of this child document.

%%%%%%%%%%%%%%%%%%%%%%%%%%%%%%%%%%%%%%%%%%%%%%%%%%%%%%%%%%%%%%%%%%%%%%%%%%%%%%%%
\subsection{Forwarding}
\label{sec:forward}

Different versions of the main or child documents
using compilation flags as described in \secref{sec:flags}
can be (permanently) stored in different files
for convenient compilation, viewing and distribution.
To this end, the package defines a command
to pass on compilation to a different file:

%%%%%%%%%%%%%%%%%%%%%%%%%%%%%%%%%%%%%%%%
\DescribeMacro{\childdocforward}
The command |\childdocforward| redirects processing to
another source file:
%
\begin{center}
\begin{tabular}{l}
|\input{childdoc.def}|\\
|\childdocforward[|\textit{main}|]{|\textit{dest}|}|\\
\end{tabular}
\end{center}
%
The argument \textit{dest} is the destination file
(without extension).
It should be the main file or one of the child files.
Note that further \textsf{childdoc} directives
such as |\childdocof| and |\childdocforward|
in the indicated file will be processed in this form.
The optional argument \textit{main}
passes on directly to the main file \textit{main}
while pretending to compile the child \textit{dest}.
This form behaves as if \textit{dest}
issues |\childdocof{|\textit{main}|}| right away,
and no further \textsf{childdoc} directives will be processed.

%%%%%%%%%%%%%%%%%%%%%%%%%%%%%%%%%%%%%%%%
\DescribeMacro{\...prefix}
In the alternative form |\childdocforwardprefix|,
%
\begin{center}
\begin{tabular}{l}
|\input{childdoc.def}|\\
|\childdocforwardprefix[|\textit{main}|]{|\textit{prefix}|}{|\textit{dest}|}|
\end{tabular}
\end{center}
%
the destination file is determined by a pattern
depending on the current file:
To make this work, the current file must be called
`{\textit{prefix}\hspace{0.2em}\textit{suffix}}'
with \textit{prefix} matching precisely the argument.
Processing is then passed on to the file
`{\textit{dest}\hspace{0.2em}\textit{suffix}}'.
Surely, the same effect is achieved by
directly specifying the
argument `{\textit{dest}\hspace{0.2em}\textit{suffix}}'
in the first form.
However, that requires to set up a different file
for each child. With the alternative form of the command
all these files can have exactly the same content
which simplifies setting them up and maintaining them.

For example, the following file |draft.tex|
with a compilation flag |\version| as described in \secref{sec:flags}
compiles the main document as a draft:
%
\begin{center}
\begin{tabular}{l}
|\def\version{draft}|\\
|\input{childdoc.def}|\\
|\childdocforward{|\textit{main}|}|
\end{tabular}
\end{center}
%
Likewise, the following files |final|\textit{nn}|.tex|
compile the final version of the child document
|child|\textit{nn}|.tex|:
%
\begin{center}
\begin{tabular}{l}
|\def\version{final}|\\
|\input{childdoc.def}|\\
|\childdocforwardprefix{final}{child}|
\end{tabular}
\end{center}
%

Note that when several versions of a main file and/or of each child file
are to be generated, it may be convenient to set up a |Makefile| or
shell script to automatise the process.

%%%%%%%%%%%%%%%%%%%%%%%%%%%%%%%%%%%%%%%%%%%%%%%%%%%%%%%%%%%%%%%%%%%%%%%%%%%%%%%%
\subsection{Command Line Processing}
\label{sec:commandline}

The effect of redirection files can also be achieved by invoking
the \LaTeX{} compiler with a more elaborate command line.
Most conveniently this should be done as part
of a shell script or a |Makefile|.

When using \textsf{childdoc} in the main file, the following
command lines effectively perform a redirection
(note that depending on the shell being used,
backslashes may have to be doubled: `|\|' $\to$ `|\\|'):
%
\begin{center}
|... -jobname "|\textit{target}|" |\\|"|[\textit{flags}]%
|\input{childdoc.def}\childdocforward[|\textit{main}|]{|\textit{dest}|}"|
\end{center}
%
Here \textit{target} is the name of the output file,
\textit{main} is the name of the main file
and \textit{dest} is the name of the main or child file to be processed
(all filenames without extensions).
The optional argument \textit{main} can be omitted
if \textit{main} matches \textit{dest}.
Optionally, compilation \textit{flags} can be defined via |\def| commands.
This command line makes the \TeX{} engine believe
it is compiling the file \textit{target}
whose content is specified as the latter parameter.
The provided code then forwards the processing to
\textit{main} or \textit{dest} as described in \secref{sec:forward}.

%%%%%%%%%%%%%%%%%%%%%%%%%%%%%%%%%%%%%%%%%%%%%%%%%%%%%%%%%%%%%%%%%%%%%%%%%%%%%%%%
\subsection{Include by Input}
\label{sec:input}

Including child documents by |\include| has some restrictions by design.
Most notably, the content of a child document always occupies
its own set of pages; pages cannot be shared between child documents.
Usually, this behaviour makes perfect sense
because each child document contain an essential part of the document.
However, in some situations it may be desirable to compose
a document from a collection of parts
without having mandatory page breaks between then.
For this case, the package
provides a mechanism to include parts
by |\input| which can also be processed individually.
However, by construction this mechanism
requires manual handling of the content to be output.

%%%%%%%%%%%%%%%%%%%%%%%%%%%%%%%%%%%%%%%%
\DescribeMacro{\ifchilddocmanual}
The main file should be prepared as usual, see \secref{sec:include}.
However, the document body must make a distinction
between processing of an individual part and of the main document, e.g.:
%
\begin{center}
\begin{tabular}{l}
|\ifchilddocmanual|\\
|\input{\childdocname}|\\
|\||else|\\
\textit{document body with }|\input{|\textit{part}|}|\\
|\||fi|
\end{tabular}
\end{center}
%
The conditional |\ifchilddocmanual| is true whenever
a part to be included by |\input| is being compiled,
and the name of the part is stored in |\childdocname|.

%%%%%%%%%%%%%%%%%%%%%%%%%%%%%%%%%%%%%%%%
\DescribeMacro{\childdocby}
Each part to be included by |\input| should start with:
%
\begin{center}
\begin{tabular}{l}
|\input{childdoc.def}|\\
|\childdocby{|\textit{main}|}|\\
\end{tabular}
\end{center}
%
The directive |\childdocby| is similar to |\childdocof|
described in \secref{sec:include},
but the subsequent selection of content must be done manually.
To that end, both |\ifchilddoc| and |\ifchilddocmanual|
will be true upon processing of a part,
and the name of the part is stored in |\childdocname|.
Note that |\jobname| will be set to the filename of the current part
so that each part receives an individual |.aux| file
that does not interfere with the |.aux| file(s) of the main document.
This behaviour can be altered by the alternative form
|\childdocby[*]{|\textit{main}|}| (with a non-empty optional argument)
which uses the |.aux| file of the main document
by setting |\jobname| to \textit{main}.

%%%%%%%%%%%%%%%%%%%%%%%%%%%%%%%%%%%%%%%%%%%%%%%%%%%%%%%%%%%%%%%%%%%%%%%%%%%%%%%%
\subsection{Driver Development}
\label{sec:driver}

The \textsf{childdoc} mechanism can also be use for the development
of definition files such as \LaTeX{} styles or classes.
This case differs from the above setup with multiple parts
included by |\include| in that no |\includeonly| should be invoked.
This can be achieved by starting the include file
(before |\ProvidesPackage|) with:
%
\begin{center}
\begin{tabular}{l}
|\input{childdoc.def}|\\
|\childdocforward{|\textit{main}|}|\\
\end{tabular}
\end{center}
%
or alternatively with:
%
\begin{center}
\begin{tabular}{l}
|\input{childdoc.def}|\\
|\childdocby{|\textit{main}|}|\\
\end{tabular}
\end{center}
%
Both forms have slightly different effects as described above.
The main file is prepared as usual, see \secref{sec:include}.

%%%%%%%%%%%%%%%%%%%%%%%%%%%%%%%%%%%%%%%%%%%%%%%%%%%%%%%%%%%%%%%%%%%%%%%%%%%%%%%%
\subsection{Legacy Detection}
\label{sec:detection}

The directive |\childdocmain| in the main file can detect
whether the complete document or merely a child is to be compiled
even without using the directive |\childdocof|.
This method is deprecated because it is less robust
and there is no compelling reason to use it;
it is merely provided for backward compatibility
and it may be removed in future versions.

If the detection mechanism is to be used,
it is mandatory to correctly specify
the filename of the main file as the argument of |\childdocmain|:
%
\begin{center}
\begin{tabular}{l}
|\input{childdoc.def}|\\
|\childdocmain{|\textit{main}|}|\\
\end{tabular}
\end{center}
%
If |\jobname| does not match the argument \textit{main} of |\childdocmain|,
it is assumed that |\jobname| points to the child file to be compiled.
When using |\childdocmain| with the main file specified as argument,
it suffices to start a child file
with just |\input{|\textit{main}|}|
without loading of the package and using |\childdocof|.
If instead all processing is done
with the appropriate \textsf{childdoc} directives,
the argument of \textit{main} of |\childdocmain| can be empty.

An alternative version of the command line processing described
in \secref{sec:commandline} using the detection mechanism reads:
%
\begin{center}
|... -jobname "|\textit{target}|" "|[\textit{flags}]%
[|\def\jobname{|\textit{dest}|}|]|\input{|\textit{main}|}"|
\end{center}

%%%%%%%%%%%%%%%%%%%%%%%%%%%%%%%%%%%%%%%%%%%%%%%%%%%%%%%%%%%%%%%%%%%%%%%%%%%%%%%%
\subsection{Manual Code}
\label{sec:manual}

In case one cannot be certain whether the definitions file |childdoc.def|
is installed on the target \TeX{} distribution
and one prefers not to ship it,
it is conceivable to paste a few relevant commands into the sources.

To that end, drop all statements |\input{childdoc.def}|
and perform the replacements as outlined below.
Instead of |\childdocmain{|\textit{main}|}| add the following code
to the top of the main file:
%
\begin{center}
\begin{tabular}{l}
|\||ifdefined\childdocname\endinput\||fi\newif\ifchilddoc|\\
|\edef\childdocname{\scantokens\expandafter{\jobname\noexpand}}|\\
|\def\childdocmain{|\textit{main}|}\||ifx\childdocmain\childdocname\||else|\\
|\childdoctrue\includeonly{\childdocname}\let\jobname\childdocmain\||fi|\\
\end{tabular}
\end{center}
%
Instead of |\childdocof{|\textit{main}|}| just include the main file
at the top of each child file:
%
\begin{center}
|\input{|\textit{main}|}|
\end{center}
%
A simple redirection |\childdocforward{|\textit{dest}|}| is achieved by:
%
\begin{center}
|\def\jobname{|\textit{dest}|}\input{\jobname}|
\end{center}
%
The redirection with prefix
|\childdocforwardprefix[|\textit{prefix}|]{|\textit{dest}|}|
is accomplished by:
%
\begin{center}
\begin{tabular}{l}
|{\edef\jobname{\scantokens\expandafter{\jobname\noexpand}}|\\
|\def\redirectjob |\textit{prefix}|#1~~~{\gdef\jobname{|\textit{dest}|#1}}|\\
|\expandafter\redirectjob\jobname~~~}\input{\jobname}|
\end{tabular}
\end{center}

In an alternative approach,
child documents can be compiled by a specific command line
without additional code or specific definitions:
%
\begin{center}
|... -jobname "|\textit{target}|" "|[\textit{flags}]%
|\includeonly{|\textit{dest}|}\input{|\textit{main}|}"|
\end{center}
%

%%%%%%%%%%%%%%%%%%%%%%%%%%%%%%%%%%%%%%%%%%%%%%%%%%%%%%%%%%%%%%%%%%%%%%%%%%%%%%%%
%%%%%%%%%%%%%%%%%%%%%%%%%%%%%%%%%%%%%%%%%%%%%%%%%%%%%%%%%%%%%%%%%%%%%%%%%%%%%%%%
\section{Information}

%%%%%%%%%%%%%%%%%%%%%%%%%%%%%%%%%%%%%%%%%%%%%%%%%%%%%%%%%%%%%%%%%%%%%%%%%%%%%%%%
\subsection{Copyright}

Copyright \copyright{} 2017--2018 Niklas Beisert

This work may be distributed and/or modified under the
conditions of the \LaTeX{} Project Public License, either version 1.3
of this license or (at your option) any later version.
The latest version of this license is in
  \url{http://www.latex-project.org/lppl.txt}
and version 1.3 or later is part of all distributions of \LaTeX{}
version 2005/12/01 or later.

This work has the LPPL maintenance status `maintained'.

The Current Maintainer of this work is Niklas Beisert.

This work consists of the files |README.txt|, |childdoc.ins| and |childdoc.dtx|
as well as the derived files |childdoc.def|, |cdocsamp.tex|
with |cdocsch1.tex|, |cdocsch2.tex|, |cdocspt3.tex|, |cdocspt4.tex|,
|cdocsdrf.tex|, |cdocsfn1.tex|, |cdocsfn2.tex|
as well as |childdoc.pdf|.

%%%%%%%%%%%%%%%%%%%%%%%%%%%%%%%%%%%%%%%%%%%%%%%%%%%%%%%%%%%%%%%%%%%%%%%%%%%%%%%%
\subsection{Files and Installation}

The package consists of the files:
%
\begin{center}
\begin{tabular}{ll}
    |README.txt|   & readme file \\
    |childdoc.ins| & installation file \\
    |childdoc.dtx| & source file \\
    |childdoc.def| & definition file \\
    |cdocsamp.tex| & sample main file \\
    |cdocsch1.tex| & sample include file \\
    |cdocsch2.tex| & sample include file \\
    |cdocspt3.tex| & sample part file \\
    |cdocspt4.tex| & sample part file \\
    |cdocsdrf.tex| & sample redirection file \\
    |cdocsfn1.tex| & sample redirection file \\
    |cdocsfn2.tex| & sample redirection file \\
    |childdoc.pdf| & manual
\end{tabular}
\end{center}
%
The distribution consists of the files
|README.txt|, |childdoc.ins| and |childdoc.dtx|.
%
\begin{itemize}
\item
Run (pdf)\LaTeX{} on |childdoc.dtx|
to compile the manual |childdoc.pdf| (this file).
\item
Run \LaTeX{} on |childdoc.ins| to create the definitions file |childdoc.def|
and the sample |cdocsamp.tex| with include files
|cdocsch1.tex|, |cdocsch2.tex|, |cdocspt3.tex|, |cdocspt4.tex|,
|cdocsdrf.tex|, |cdocsfn1.tex|, |cdocsfn2.tex|.
Then copy the file |childdoc.def| to an appropriate directory of your \LaTeX{}
distribution, e.g.\ \textit{texmf-root}|/tex/latex/childdoc|.
\end{itemize}

%%%%%%%%%%%%%%%%%%%%%%%%%%%%%%%%%%%%%%%%%%%%%%%%%%%%%%%%%%%%%%%%%%%%%%%%%%%%%%%%
\subsection{Related CTAN Packages}

There are several other packages which offer a similar functionality:
%
\begin{itemize}
\item
The packages
\href{http://ctan.org/pkg/docmute}{\textsf{docmute}},
\href{http://ctan.org/pkg/includex}{\textsf{includex}} and
\href{http://ctan.org/pkg/standalone}{\textsf{standalone}}
provide commands to include only the document body of
a child file thus allowing both files to be compiled individually.
\item
The packages \href{http://ctan.org/pkg/subdocs}{\textsf{subdocs}}
and \href{http://ctan.org/pkg/subfiles}{\textsf{subfiles}}
provide structures in which the main and child documents can be
encapsulated and allowing them to be compiled individually.
The inclusion mechanism is different from the conventional |\include|.
\item
The package \href{http://ctan.org/pkg/combine}{\textsf{combine}}
is an elaborate solution to combine several documents into one.
\end{itemize}
%
See also the CTAN topic \href{http://ctan.org/topic/subdocs}{\textsf{subdocs}}
for further related packages.
The present package differs from the above solutions in that
a document structure constructed with the conventional |\include| mechanism
just needs two extra commands at the top of every file
such that all constituent files can be compiled individually.

%%%%%%%%%%%%%%%%%%%%%%%%%%%%%%%%%%%%%%%%%%%%%%%%%%%%%%%%%%%%%%%%%%%%%%%%%%%%%%%%
%\subsection{Feature Suggestions}
%
%The following is a list of features which may be useful for future
%versions of this package:
%%
%\begin{itemize}
%\item
%\ldots
%\end{itemize}

%%%%%%%%%%%%%%%%%%%%%%%%%%%%%%%%%%%%%%%%%%%%%%%%%%%%%%%%%%%%%%%%%%%%%%%%%%%%%%%%
\subsection{Revision History}

%%%%%%%%%%%%%%%%%%%%%%%%%%%%%%%%%%%%%%%%
\paragraph{v2.0:} 2018/12/30

\begin{itemize}
\item
immediate forward processing
\item
added |\childdocby| mechanism
\item
manual restructured
\end{itemize}

%%%%%%%%%%%%%%%%%%%%%%%%%%%%%%%%%%%%%%%%
\paragraph{v1.6:} 2018/01/17

\begin{itemize}
\item
application for development of include files
\item
corrections to manual
\end{itemize}

%%%%%%%%%%%%%%%%%%%%%%%%%%%%%%%%%%%%%%%%
\paragraph{v1.5:} 2017/05/21

\begin{itemize}
\item
more complete structuring introduced
\item
|\childdocof| introduced
\item
|\childdoc| renamed to |\childdocmain|
\item
|\childredirect| renamed to |\childdocforward| and |\childdocforwardprefix|
and functionality expanded
\end{itemize}

%%%%%%%%%%%%%%%%%%%%%%%%%%%%%%%%%%%%%%%%
\paragraph{v1.0:} 2017/04/27

\begin{itemize}
\item
manual and install package
\item
first version published on CTAN
\end{itemize}

%%%%%%%%%%%%%%%%%%%%%%%%%%%%%%%%%%%%%%%%
\paragraph{v0.6:} 2017/04/26

\begin{itemize}
\item
redirection mechanism added
\end{itemize}

%%%%%%%%%%%%%%%%%%%%%%%%%%%%%%%%%%%%%%%%
\paragraph{v0.5:} 2017/04/26

\begin{itemize}
\item
functionality in definition file
\end{itemize}


%%%%%%%%%%%%%%%%%%%%%%%%%%%%%%%%%%%%%%%%%%%%%%%%%%%%%%%%%%%%%%%%%%%%%%%%%%%%%%%%
%%%%%%%%%%%%%%%%%%%%%%%%%%%%%%%%%%%%%%%%%%%%%%%%%%%%%%%%%%%%%%%%%%%%%%%%%%%%%%%%
%%%%%%%%%%%%%%%%%%%%%%%%%%%%%%%%%%%%%%%%%%%%%%%%%%%%%%%%%%%%%%%%%%%%%%%%%%%%%%%%
\appendix

\settowidth\MacroIndent{\rmfamily\scriptsize 000\ }

 \DocInput{childdoc.dtx}

\end{document}
%</driver>
% \fi
%
% %%%%%%%%%%%%%%%%%%%%%%%%%%%%%%%%%%%%%%%%%%%%%%%%%%%%%%%%%%%%%%%%%%%%%%%%%%%%%%
% %%%%%%%%%%%%%%%%%%%%%%%%%%%%%%%%%%%%%%%%%%%%%%%%%%%%%%%%%%%%%%%%%%%%%%%%%%%%%%
% \section{Sample}
%\iffalse
%<*samplemain>
%\fi
%
% The following presents a sample document
% with two chapters, two parts, a title page,
% a compile flag as well as three forwarding files to set the flag.
% It consists of eight |.tex| files:
% \begin{center}
% \begin{tabular}{ll}
% |cdocsamp.tex|&main file\\
% |cdocsch1.tex|&include file for chapter 1\\
% |cdocsch2.tex|&include file for chapter 2\\
% |cdocspt3.tex|&include file for part 3\\
% |cdocspt4.tex|&include file for part 4\\
% |cdocsdrf.tex|&forwarding file for main file in draft mode\\
% |cdocsfi1.tex|&forwarding file for final version of chapter 1\\
% |cdocsfi2.tex|&forwarding file for final version of chapter 2\\
% \end{tabular}
% \end{center}
% Each of the eight files can be compiled directly by the \LaTeX{} compiler.
%
% %%%%%%%%%%%%%%%%%%%%%%%%%%%%%%%%%%%%%%
% \paragraph{Main File.}
%
% The main file is called |cdocsamp.tex|.
%
% Load the \textsf{childdoc} definitions and
% declare the filename for the main document:
%    \begin{macrocode}
\input{childdoc.def}
\childdocmain{}
%    \end{macrocode}

% Optional override for |\version| flag:
%    \begin{macrocode}
%%\ifchilddoc\else\providecommand{\version}{draft}\fi
%    \end{macrocode}

% Define the default values for the |\version| flag
% (|final| for the main file and |draft| for childs):
%    \begin{macrocode}
\ifchilddoc
\providecommand{\version}{draft}
\else
\providecommand{\version}{final}
\fi
%    \end{macrocode}

% Load the standard document class:
%    \begin{macrocode}
\documentclass[12pt]{article}
%    \end{macrocode}

% Start the document body:
%    \begin{macrocode}
\begin{document}
%    \end{macrocode}

% Declare a title page.
% Print title, part of document being processed and version flag:
%    \begin{macrocode}
\addtocounter{page}{-1}
\begin{center}
{\LARGE\bfseries{}childdoc example\par}
\vspace{1cm}
\ifchilddoc
\ifchilddocmanual part\else chapter\fi:
`\childdocname' of `\childdocjob'\par
\else
main document: `\childdocjob'\par
\fi
version: \version\par
\end{center}
\newpage
%    \end{macrocode}

% Manually include selected file,
% otherwise process as usual:
%    \begin{macrocode}
\ifchilddocmanual
\section*{part `\childdocname'}
\input{\childdocname}
\else
%    \end{macrocode}

% Include the two chapters:
%    \begin{macrocode}
\include{cdocsch1}
\include{cdocsch2}
%    \end{macrocode}

% Include the two parts unless only chapters should be displayed:
%    \begin{macrocode}
\ifchilddoc\else
\section{part three}
\input{cdocspt3}
\section{part four}
\input{cdocspt4}
\fi
%    \end{macrocode}

% Process as usual until here:
%    \begin{macrocode}
\fi
%    \end{macrocode}

% End of document body:
%    \begin{macrocode}
\end{document}
%    \end{macrocode}
%\iffalse
%</samplemain>
%\fi
%
% %%%%%%%%%%%%%%%%%%%%%%%%%%%%%%%%%%%%%%
% \paragraph{Chapter Include Files.}
%
% The include files are called |cdocsch1.tex| and |cdocsch2.tex|.
%
%\iffalse
%<*samplechap1|samplechap2>
%\fi

% Optional override for |\version| flag:
%    \begin{macrocode}
%%\providecommand{\version}{final}
%    \end{macrocode}

% Include the main document:
%    \begin{macrocode}
\input{childdoc.def}
\childdocof{cdocsamp}
%    \end{macrocode}

%\iffalse
%</samplechap1|samplechap2>
%\fi
%
%\iffalse
%<*samplechap1>
%\fi
% Some text for chapter 1:
%    \begin{macrocode}
\section{one}
some text in chapter one
%    \end{macrocode}

%\iffalse
%</samplechap1>
%\fi
% Some text for chapter 2:
%\iffalse
%<*samplechap2>
%\fi
%    \begin{macrocode}
\section{two}
more text in chapter two
%    \end{macrocode}

%\iffalse
%</samplechap2>
%\fi
%
% %%%%%%%%%%%%%%%%%%%%%%%%%%%%%%%%%%%%%%
% \paragraph{Part Include Files.}
%
% The include files are called |cdocspt3.tex| and |cdocspt4.tex|.
%
%\iffalse
%<*samplepart3|samplepart4>
%\fi

% Optional override for |\version| flag:
%    \begin{macrocode}
%%\providecommand{\version}{final}
%    \end{macrocode}

% Include the main document:
%    \begin{macrocode}
\input{childdoc.def}
\childdocby{cdocsamp}
%    \end{macrocode}

%\iffalse
%</samplepart3|samplepart4>
%\fi
%
%\iffalse
%<*samplepart3>
%\fi
% Some text for part 3:
%    \begin{macrocode}
some text in part three
%    \end{macrocode}

%\iffalse
%</samplepart3>
%\fi
% Some text for part 4:
%\iffalse
%<*samplepart4>
%\fi
%    \begin{macrocode}
more text in part four
%    \end{macrocode}

%\iffalse
%</samplepart4>
%\fi
%
% %%%%%%%%%%%%%%%%%%%%%%%%%%%%%%%%%%%%%%
% \paragraph{Forwarding for a Complete Draft.}
%
% The following forwarding file |cdocsdrf.tex|
% compiles the main document in draft mode:
%\iffalse
%<*sampledraft>
%\fi
%    \begin{macrocode}
\def\version{draft}
\input{childdoc.def}
\childdocforward{cdocsamp}
%    \end{macrocode}

%\iffalse
%</sampledraft>
%\fi
%
% %%%%%%%%%%%%%%%%%%%%%%%%%%%%%%%%%%%%%%
% \paragraph{Forwarding for Final Version of the Chapters.}
%
% The following forwarding files |cdocsfn1.tex| and |cdocsfn2.tex|
% (with identical content)
% compile the final versions of the child documents
% |cdocsch1.tex| and |cdocsch2.tex|, respectively:
%\iffalse
%<*samplefinal>
%\fi
%    \begin{macrocode}
\def\version{final}
\input{childdoc.def}
\childdocforwardprefix[cdocsamp]{cdocsfn}{cdocsch}
%    \end{macrocode}

%\iffalse
%</samplefinal>
%\fi
%
% %%%%%%%%%%%%%%%%%%%%%%%%%%%%%%%%%%%%%%
% \paragraph{Command Line Processing.}
%
% The following three command lines generate the output files
% |cdocscld|, |cdocscl1| and |cdocscl2|
% which should be identical to
% |cdocsdrf|, |cdocsch1| and |cdocsfn2|, respectively:
% \begin{center}
% \begin{tabular}{l}
% |latex -jobname cdocscld \|\\
% |  "\def\version{draft}\input{childdoc.def}\childdocforward{cdocsamp}"|\\
% |latex -jobname cdocscl1 \|\\
% |  "\input{childdoc.def}\childdocforward[cdocsamp]{cdocsch1}"|\\
% |latex -jobname cdocscl2 \|\\
% |  "\def\version{final}\input{childdoc.def}\childdocforward{cdocsch2}"|
% \end{tabular}
% \end{center}
% Note that the trailing backslash on each first line
% merely continues the input to the second line
% (for convenient cut ant paste).
% Furthermore, the command |latex| can be replaced by any
% of its alternative versions such as |pdflatex|.
%
% %%%%%%%%%%%%%%%%%%%%%%%%%%%%%%%%%%%%%%%%%%%%%%%%%%%%%%%%%%%%%%%%%%%%%%%%%%%%%%
% %%%%%%%%%%%%%%%%%%%%%%%%%%%%%%%%%%%%%%%%%%%%%%%%%%%%%%%%%%%%%%%%%%%%%%%%%%%%%%
% \section{Implementation}
%\iffalse
%<*package>
%\fi
%
% This section describes the definitions file |childdoc.def|.

% The definitions cannot be loaded using |\usepackage| or |\RequirePackage|
% which has a mechanism to prevent loading a style file more than once.
% When loading the definitions by means of |\input|
% multiple instances have to be prevented manually:
%\iffalse
%This code needs to be before the `\ProvidesFile' directive
%which is defined at the beginning of this file.
%Therefore it is also placed there and commented out here.
%</package>
%<*discard>
%\fi
%    \begin{macrocode}
\ifdefined\childdocmain\endinput\fi
%    \end{macrocode}
%\iffalse
%</discard>
%<*package>
%\fi
%
% \macro{\ifchilddoc}
% \macro{\ifchilddocmanual}
% The conditional |\ifchilddoc| tells whether a
% child (true) or main (false) document is being compiled.
% The conditional |\ifchilddocmanual| tells whether
% the |\includeonly| mechanism is used (false) or
% the selection of child files must be performed manually (true).
% The definitions initialise to false:
%    \begin{macrocode}
\newif\ifchilddoc
\newif\ifchilddocmanual
%    \end{macrocode}

% \macro{\childdocname}
% \macro{\childdocjob}
% The macro |\childdocname| stores the name of the main document
% to be compiled. The macro |\childdocjob| stores the name of
% the document on which the \LaTeX{} compiler was originally invoked.
% The content of |\jobname| cannot be compared
% to filenames specified in the source due to different catcodes.
% The following code rescans |\jobname|, stores the result
% in |\childdocname| and saves a copy in |\childdocjob|:
%    \begin{macrocode}
\edef\childdocname{\scantokens\expandafter{\jobname\noexpand}}
\let\childdocjob\childdocname
%    \end{macrocode}

% \macro{\childdocdisable}
% The macro |\childdocdisable| prevents the main file
% from being processed more than once.
% At this stage, the main document command |\childdocmain|
% is assumed to be called once again where it should do nothing.
% Any subsequent call to it should prevent
% a secondary processing of the main document
% It overwrites the forwarding commands
% |\childdocof| and |\childdocforward|
% with empty macros to prevent further inclusions of the main document:
%    \begin{macrocode}
\newcommand{\childdocdisable}
{
  \renewcommand{\childdocmain}[1]{\renewcommand{\childdocmain}[1]{\endinput}}
  \renewcommand{\childdocof}[1]{}
  \renewcommand{\childdocby}[2][]{}
  \renewcommand{\childdocforward}[2][]{}
  \renewcommand{\childdocdisable}{}
}
%    \end{macrocode}

% \macro{\childdocmain}
% The macro |\childdocmain| is to be called at the top of the main file
% with nothing or the main filename (without extension) as argument.
% First, it breaks loops.
% If the argument is not empty and does not match |\childdocname|
% (which is set by the first inclusion of |childdoc.def|),
% |\ifchilddoc| is set to true, |\includeonly| is applied to the child file
% and |\jobname| is set to the main file
% (for proper handling of |.aux| files):
%    \begin{macrocode}
\newcommand{\childdocmain}[1]
{
  \childdocdisable\childdocmain{}
  \if?#1?\else
    \begingroup
      \def\childdoctmp{#1}
      \ifx\childdoctmp\childdocname
        \def\childdoctmp{}
      \else
        \def\childdoctmp
        {
          \childdoctrue
          \includeonly{\childdocname}
          \def\childdocjob{#1}
          \def\jobname{#1}
        }
      \fi
      \expandafter
    \endgroup
    \childdoctmp
  \fi
}
%    \end{macrocode}

% \macro{\childdocof}
% The command |\childdocof| redirects
% compilation to the main file |#1|.
%    \begin{macrocode}
\newcommand{\childdocof}[1]
{
  \childdocdisable
  \childdoctrue
  \includeonly{\childdocname}
  \def\jobname{#1}
  \def\childdocjob{#1}
  \input{#1}
}
%    \end{macrocode}

% \macro{\childdocby}
% The command |\childdocby| ....
%    \begin{macrocode}
\newcommand{\childdocby}[2][]
{
  \childdocdisable
  \childdoctrue
  \childdocmanualtrue
  \if?#1?\else
    \def\jobname{#2}
  \fi
  \def\childdocjob{#2}
  \input{#2}
  \endinput
}
%    \end{macrocode}

% \macro{\childdocforward}
% The command |\childdocforward| redirects
% compilation to the main file or
% (if the optional argument is given) a child file.
% Parameters are set as if the main file
% or a child file starting with |\childdocof| was compiled.
% Then compilation is handed over to the main file:
%    \begin{macrocode}
\newcommand{\childdocforward}[2][]
{
  \begingroup
    \if?#1?
      \def\childdoctmp
      {
        \def\childdocname{#2}
        \def\childdocjob{#2}
        \def\jobname{#2}
        \input{#2}
        \endinput
      }
    \else
      \def\childdoctmp
      {
        \childdocdisable
        \def\childdocname{#2}
        \childdoctrue
        \includeonly{#2}
        \def\childdocjob{#1}
        \def\jobname{#1}
        \input{#1}
        \endinput
      }
    \fi
    \expandafter
  \endgroup
  \childdoctmp
}
%    \end{macrocode}

% \macro{\childdocforwardprefix}
% The command |\childdocforwardprefix| redirects
% compilation to the main or a child file by means of a pattern.
% The prefix |#1| in the current filename is replaced by |#2|
% and the suffix of the current filename is kept
% (it is assumed that the filename does not contain the substring `|~~~|'
% which is used as a delimiter).
% Compilation is handed over to the new file by |\childdocforward|:
%    \begin{macrocode}
\newcommand{\childdocforwardprefix}[3][]
{
  \begingroup
    \def\childdocextract #2##1~~~{\def\childdoctmp{\childdocforward[#1]{#3##1}}}
    \expandafter\childdocextract\childdocname~~~
    \expandafter
  \endgroup
  \childdoctmp
}
%    \end{macrocode}

% \macro{\childdoc}
% The deprecated macro |\childdoc| is a legacy version of |\childdocmain|:
%    \begin{macrocode}
\newcommand{\childdoc}{\childdocmain}
%    \end{macrocode}

% \macro{\childdocredirect}
% The deprecated macro |\childdocredirect| is a legacy version
% of |\childdocforward| and |\childdocforwardprefix|:
%    \begin{macrocode}
\newcommand{\childdocredirect}[2][]
{
  \begingroup
    \if?#1?
      \def\childdoctmp{\childdocforward{#2}}
    \else
      \def\childdoctmp{\childdocforwardprefix{#1}{#2}}
    \fi
    \expandafter
  \endgroup
  \childdoctmp
}
%    \end{macrocode}

%\iffalse
%</package>
%\fi
%
\endinput
|\\
|\childdocforward{|\textit{main}|}|\\
\end{tabular}
\end{center}
%
or alternatively with:
%
\begin{center}
\begin{tabular}{l}
|% \iffalse
%
% childdoc.dtx Copyright (C) 2017-2018 Niklas Beisert
%
% This work may be distributed and/or modified under the
% conditions of the LaTeX Project Public License, either version 1.3
% of this license or (at your option) any later version.
% The latest version of this license is in
%   http://www.latex-project.org/lppl.txt
% and version 1.3 or later is part of all distributions of LaTeX
% version 2005/12/01 or later.
%
% This work has the LPPL maintenance status `maintained'.
%
% The Current Maintainer of this work is Niklas Beisert.
%
% This work consists of the files childdoc.dtx and childdoc.ins
% and the derived files childdoc.def and cdocsamp.tex with
% cdocsch1.tex, cdocsch2.tex, cdocsdrf.tex, cdocsfn1.tex, cdocsfn2.tex.
%
%<package>\ifdefined\childdocmain\endinput\fi
%<package>\ProvidesFile{childdoc.def}[2018/12/30 v2.0 child document driver]
%<samplemain>\ProvidesFile{cdocsamp.tex}[2018/12/30 v2.0 sample for childdoc]
%<*driver>
%\ProvidesFile{childdoc.drv}[2018/12/30 v2.0 childdoc reference manual file]
\PassOptionsToClass{10pt,a4paper}{article}
\documentclass{ltxdoc}

\usepackage[margin=35mm]{geometry}
\usepackage{hyperref}
\usepackage{hyperxmp}
\usepackage[usenames]{color}

\hypersetup{colorlinks=true}
\hypersetup{pdfstartview=FitH}
\hypersetup{pdfpagemode=UseNone}
\hypersetup{pdfsource={}}
\hypersetup{pdflang={en-UK}}
\hypersetup{pdfcopyright={Copyright 2017-2018 Niklas Beisert.
  This work may be distributed and/or modified under the
  conditions of the LaTeX Project Public License, either version 1.3
  of this license or (at your option) any later version.}}
\hypersetup{pdflicenseurl={http://www.latex-project.org/lppl.txt}}
\hypersetup{pdfcontactaddress={ETH Zurich, ITP, HIT K,
  Wolfgang-Pauli-Strasse 27}}
\hypersetup{pdfcontactpostcode={8093}}
\hypersetup{pdfcontactcity={Zurich}}
\hypersetup{pdfcontactcountry={Switzerland}}
\hypersetup{pdfcontactemail={nbeisert@itp.phys.ethz.ch}}
\hypersetup{pdfcontacturl={http://people.phys.ethz.ch/\xmptilde nbeisert/}}

\newcommand{\secref}[1]{\hyperref[#1]{section \ref*{#1}}}

\parskip1ex
\parindent0pt
\let\olditemize\itemize
\def\itemize{\olditemize\parskip0pt}

\begin{document}

\title{The \textsf{childdoc} Package}
\hypersetup{pdftitle={The childdoc Package}}
\author{Niklas Beisert\\[2ex]
  Institut f\"ur Theoretische Physik\\
  Eidgen\"ossische Technische Hochschule Z\"urich\\
  Wolfgang-Pauli-Strasse 27, 8093 Z\"urich, Switzerland\\[1ex]
  \href{mailto:nbeisert@itp.phys.ethz.ch}
  {\texttt{nbeisert@itp.phys.ethz.ch}}}
\hypersetup{pdfauthor={Niklas Beisert}}
\hypersetup{pdfsubject={Manual for the LaTeX2e Package childdoc}}
\date{30 December 2018, \textsf{v2.0}}
\maketitle

\begin{abstract}\noindent
\textsf{childdoc} is a \LaTeXe{} package
that enables the direct compilation
of document sections included by |\include|
to individual files.
\end{abstract}

\begingroup
\parskip0ex
\tableofcontents
\endgroup

%%%%%%%%%%%%%%%%%%%%%%%%%%%%%%%%%%%%%%%%%%%%%%%%%%%%%%%%%%%%%%%%%%%%%%%%%%%%%%%%
%%%%%%%%%%%%%%%%%%%%%%%%%%%%%%%%%%%%%%%%%%%%%%%%%%%%%%%%%%%%%%%%%%%%%%%%%%%%%%%%
\section{Introduction}

\LaTeX{} provides a mechanism to structure a large document (such as a book)
into a main file and several child files (containing the chapters)
using the |\include| command.
This mechanism is beneficial for documents
which span hundreds of pages in order to
make the source file(s) more manageable.
Moreover, compilation can be restricted to
selected child files by means of the |\includeonly| command.
The latter feature can be used to reduce the compilation time while editing
(this was significantly more useful in the earlier days of \LaTeX{})
or to generate a smaller document which is easier to navigate.
Another application of |\includeonly| is to generate
documents consisting of selected parts of the complete document.

However, there are a few drawbacks of the plain |\include| mechanism:
\begin{itemize}
\item
The child files cannot be compiled on their own,
they can only be compiled via the main file.
A naive editing environment
(such as a text editor with an option
to have the current file processed by \LaTeX)
may require one to switch to the main file before compiling;
attempting to compile the child file produces errors.
\item
The main file must be modified (each time)
to adjust the |\includeonly| command
to the present needs. This easily leaves the main file in a messy state.
\item
The generated document will always carry the filename
of the main document. This is inconvenient if
several child files are to be compiled and
to be kept for distribution.
\end{itemize}

The present package provides a simple interface
to make child files individually compilable by \LaTeX{}.
Compiling a child file then has the same effect as compiling
the main file with an |\includeonly| command
to select the appropriate child.
Moreover the generated document will carry the name of the child
rather than the main file.
This resolves all three above issues.

This feature is meant to make the editing of books,
thesis documents and lecture notes somewhat more convenient.
However, the package can also be used efficiently for
composing a series of documents (such as exercise sheets)
which are typically distributed individually.
It then assists the author in generating the individual documents
(potentially in different versions)
as well as a document containing the collected series.
Another application is in developing style files
or other kinds of included material
where compilation of the style file could redirect
to a sample or test file.

%%%%%%%%%%%%%%%%%%%%%%%%%%%%%%%%%%%%%%%%%%%%%%%%%%%%%%%%%%%%%%%%%%%%%%%%%%%%%%%%
%%%%%%%%%%%%%%%%%%%%%%%%%%%%%%%%%%%%%%%%%%%%%%%%%%%%%%%%%%%%%%%%%%%%%%%%%%%%%%%%
\section{Usage}

First of all, the package \textsf{childdoc} is \emph{not} a standard
\LaTeXe{} |.sty| style file! Therefore it needs to be invoked in
a non-standard way.

%%%%%%%%%%%%%%%%%%%%%%%%%%%%%%%%%%%%%%%%%%%%%%%%%%%%%%%%%%%%%%%%%%%%%%%%%%%%%%%%
\subsection{Included Files}
\label{sec:include}

%%%%%%%%%%%%%%%%%%%%%%%%%%%%%%%%%%%%%%%%
\DescribeMacro{\childdocmain}
To use the package, add the commands
\begin{center}
\begin{tabular}{l}
|\input{childdoc.def}|\\
|\childdocmain{}|\\
\end{tabular}
\end{center}
at the very top of the main \LaTeX{} file,
in particular \emph{before} the |\documentclass| statement!
The argument of |\childdocmain| should be left empty
(but it must be present).

%%%%%%%%%%%%%%%%%%%%%%%%%%%%%%%%%%%%%%%%
\DescribeMacro{\childdocof}
Furthermore, add the commands
\begin{center}
\begin{tabular}{l}
|\input{childdoc.def}|\\
|\childdocof{|\textit{main}|}|\\
\end{tabular}
\end{center}
at the top of every child file \textit{child}
which is included by |\include{|\textit{child}|}|
from within the main file
(or at least for those files to be compiled individually).
The argument \textit{main} must be the filename of the main file.

There are a couple of
considerations in setting up the main and child documents:

%%%%%%%%%%%%%%%%%%%%%%%%%%%%%%%%%%%%%%%%
\paragraph{Restrictions.}

Please note the following restrictions:
\begin{itemize}
\item
|\childdocmain| must be called with one argument \textit{main}
to ensure compatibility with earlier version of the package.
It must either be empty (|\childdocmain{}|)
or precisely match the filename of the main file in which it is specified.
See \secref{sec:detection} for further information.
\item
The filename \textit{main} must be specified without the |.tex| extension.
\item
The filename \textit{main} is case sensitive
(even in case-insensitive file systems)
due to internal string comparison.
\item
The argument \textit{main} should be fully expanded, it cannot be a macro.
\item
Subdirectories and special characters should be avoided in filenames.
\item
The command |\childdocmain{|\textit{main}|}| must be followed by a whitespace.
It should not be followed immediately by another command
or by a comment mark `|%|'.
This is because the \TeX{} parser reads the token immediately following
the argument of |\childdocmain| and puts it
at the beginning of every child section;
however, a white\-space is ignored.
\end{itemize}

%%%%%%%%%%%%%%%%%%%%%%%%%%%%%%%%%%%%%%%%
\paragraph{Content of Main File.}

It is advisable to place all content in the child files included by |\include|.
Any output contained in the main file will appear in all child documents
unless suppressed manually;
it cannot be suppressed automatically by the |\includeonly| directive
and thus should normally be avoided.
A method to include some content in the main file
by means of conditional processing is described in \secref{sec:conditional}.

%%%%%%%%%%%%%%%%%%%%%%%%%%%%%%%%%%%%%%%%
\paragraph{Page Numbering.}

When only a part of the document is compiled,
the appropriate numbering of pages
(as well as other status parameters)
is determined from the |.aux| files.
The latter contain information from previous passes.
However this information needs to propagate through
all intermediate child documents.
Therefore the page numbering in child documents may well
be inconsistent until the complete document is compiled at least once.

A useful (if unconventional) way to always ensure a consistent
page numbering is to restart the numbering in each child document
and denote the pages by `\textit{child}|.|\textit{page}'
where \textit{child} represents the chapter/section number of the child file.
This can be achieved by the command
|\numberwithin{page}{|\textit{child}|}|
of the \textsf{amsmath} package
where \textit{child} can be |chapter| or |section|
depending on the chosen structuring.
Alternatively, one can modify the macro |\thepage| appropriately
and reset the counter |page| at the start of each child file.

%%%%%%%%%%%%%%%%%%%%%%%%%%%%%%%%%%%%%%%%%%%%%%%%%%%%%%%%%%%%%%%%%%%%%%%%%%%%%%%%
\subsection{Conditional Processing}
\label{sec:conditional}

The package provides a mechanism to compile different versions
of a document. To customise the versions further some conditional processing
can come in handy to distinguish which version is being compiled.
The package provides two macros to describe the compilation context:

%%%%%%%%%%%%%%%%%%%%%%%%%%%%%%%%%%%%%%%%
\DescribeMacro{\ifchilddoc}
The conditional |\ifchilddoc| distinguishes between the compilation of
child documents and the main document:
%
\begin{center}
|\ifchilddoc |\textit{child-code}| |[|\||else |\textit{main-code}]| \||fi|
\end{center}

%%%%%%%%%%%%%%%%%%%%%%%%%%%%%%%%%%%%%%%%
\DescribeMacro{\childdocname}
\DescribeMacro{\childdocjob}
The macro |\childdocname| contains the filename (without extension)
of the main or child file being processed.
Note that |\childdocjob| will always contain the name of the main file.

%%%%%%%%%%%%%%%%%%%%%%%%%%%%%%%%%%%%%%%%
\paragraph{Title Page.}

Conditional processing can be used to include a title or banner page
in the main document when proper precautions are taken.
Importantly, the code in the main file should ensure that the page counter
(as well as other status parameters which are stored in the |.aux| files)
takes the same value after the conditional processing.
Otherwise the page numbers may take divergent values
depending on which part is compiled.

For example, a title page could be declared by:
%
\begin{center}
\begin{tabular}{l}
|\ifchilddoc\||else|\\
|\addtocounter{page}{-1}|\\
\textit{code for title page}\\
|\newpage|\\
|\||fi|
\end{tabular}
\end{center}
%
A banner page for the child documents can be generated by:
%
\begin{center}
\begin{tabular}{l}
|\ifchilddoc|\\
|\addtocounter{page}{-1}|\\
\textit{code for banner page}\\
|\newpage|\\
|\||fi|
\end{tabular}
\end{center}
%
Here one could write a message such as:
\begin{center}
|This is the part \childdocname{} of \childdocjob{}.|
\end{center}

%%%%%%%%%%%%%%%%%%%%%%%%%%%%%%%%%%%%%%%%%%%%%%%%%%%%%%%%%%%%%%%%%%%%%%%%%%%%%%%%
\subsection{Flags}
\label{sec:flags}

The package makes it easy to generate different versions
of the main or child documents.
To this end compilation flags can be defined
and assigned different default values.
They will be particularly useful in conjunction
with the forwarding mechanism described in \secref{sec:forward}.

For example, it may be useful to have a flag |\version|
which can be set to |draft| or |final|.
The document source will contain some conditional code
depending on the value of |\version|.
Suppose further, the flag should default to |final| for the main file
and to |draft| for child files
which is a natural assignment for editing the document.
This is achieved by placing the following code
in the preamble of the main document
(below the |\childdocmain| directive):
%
\begin{center}
\begin{tabular}{l}
|\ifchilddoc|\\
|\providecommand{\version}{draft}|\\
|\||else|\\
|\providecommand{\version}{final}|\\
|\||fi|
\end{tabular}
\end{center}
%
The definition by |\providecommand| makes sure
that previous definitions are not overwritten.
Further statements |\providecommand{\version}{...}|
can thus be added before the above code to override it.

For the main file, one might add a line
(between |\childdocmain| and the above block)
%
\begin{center}
|%\ifchilddoc\||else\providecommand{\version}{draft}\||fi|
\end{center}
%
which can be uncommented to produce a draft version.
Likewise one can add a line to the very top of a child file
(above the |\childdocof{|\textit{main}|}| directive)
%
\begin{center}
|%\providecommand{\version}{final}|
\end{center}
%
which can be uncommented to produce the final version of this child document.

%%%%%%%%%%%%%%%%%%%%%%%%%%%%%%%%%%%%%%%%%%%%%%%%%%%%%%%%%%%%%%%%%%%%%%%%%%%%%%%%
\subsection{Forwarding}
\label{sec:forward}

Different versions of the main or child documents
using compilation flags as described in \secref{sec:flags}
can be (permanently) stored in different files
for convenient compilation, viewing and distribution.
To this end, the package defines a command
to pass on compilation to a different file:

%%%%%%%%%%%%%%%%%%%%%%%%%%%%%%%%%%%%%%%%
\DescribeMacro{\childdocforward}
The command |\childdocforward| redirects processing to
another source file:
%
\begin{center}
\begin{tabular}{l}
|\input{childdoc.def}|\\
|\childdocforward[|\textit{main}|]{|\textit{dest}|}|\\
\end{tabular}
\end{center}
%
The argument \textit{dest} is the destination file
(without extension).
It should be the main file or one of the child files.
Note that further \textsf{childdoc} directives
such as |\childdocof| and |\childdocforward|
in the indicated file will be processed in this form.
The optional argument \textit{main}
passes on directly to the main file \textit{main}
while pretending to compile the child \textit{dest}.
This form behaves as if \textit{dest}
issues |\childdocof{|\textit{main}|}| right away,
and no further \textsf{childdoc} directives will be processed.

%%%%%%%%%%%%%%%%%%%%%%%%%%%%%%%%%%%%%%%%
\DescribeMacro{\...prefix}
In the alternative form |\childdocforwardprefix|,
%
\begin{center}
\begin{tabular}{l}
|\input{childdoc.def}|\\
|\childdocforwardprefix[|\textit{main}|]{|\textit{prefix}|}{|\textit{dest}|}|
\end{tabular}
\end{center}
%
the destination file is determined by a pattern
depending on the current file:
To make this work, the current file must be called
`{\textit{prefix}\hspace{0.2em}\textit{suffix}}'
with \textit{prefix} matching precisely the argument.
Processing is then passed on to the file
`{\textit{dest}\hspace{0.2em}\textit{suffix}}'.
Surely, the same effect is achieved by
directly specifying the
argument `{\textit{dest}\hspace{0.2em}\textit{suffix}}'
in the first form.
However, that requires to set up a different file
for each child. With the alternative form of the command
all these files can have exactly the same content
which simplifies setting them up and maintaining them.

For example, the following file |draft.tex|
with a compilation flag |\version| as described in \secref{sec:flags}
compiles the main document as a draft:
%
\begin{center}
\begin{tabular}{l}
|\def\version{draft}|\\
|\input{childdoc.def}|\\
|\childdocforward{|\textit{main}|}|
\end{tabular}
\end{center}
%
Likewise, the following files |final|\textit{nn}|.tex|
compile the final version of the child document
|child|\textit{nn}|.tex|:
%
\begin{center}
\begin{tabular}{l}
|\def\version{final}|\\
|\input{childdoc.def}|\\
|\childdocforwardprefix{final}{child}|
\end{tabular}
\end{center}
%

Note that when several versions of a main file and/or of each child file
are to be generated, it may be convenient to set up a |Makefile| or
shell script to automatise the process.

%%%%%%%%%%%%%%%%%%%%%%%%%%%%%%%%%%%%%%%%%%%%%%%%%%%%%%%%%%%%%%%%%%%%%%%%%%%%%%%%
\subsection{Command Line Processing}
\label{sec:commandline}

The effect of redirection files can also be achieved by invoking
the \LaTeX{} compiler with a more elaborate command line.
Most conveniently this should be done as part
of a shell script or a |Makefile|.

When using \textsf{childdoc} in the main file, the following
command lines effectively perform a redirection
(note that depending on the shell being used,
backslashes may have to be doubled: `|\|' $\to$ `|\\|'):
%
\begin{center}
|... -jobname "|\textit{target}|" |\\|"|[\textit{flags}]%
|\input{childdoc.def}\childdocforward[|\textit{main}|]{|\textit{dest}|}"|
\end{center}
%
Here \textit{target} is the name of the output file,
\textit{main} is the name of the main file
and \textit{dest} is the name of the main or child file to be processed
(all filenames without extensions).
The optional argument \textit{main} can be omitted
if \textit{main} matches \textit{dest}.
Optionally, compilation \textit{flags} can be defined via |\def| commands.
This command line makes the \TeX{} engine believe
it is compiling the file \textit{target}
whose content is specified as the latter parameter.
The provided code then forwards the processing to
\textit{main} or \textit{dest} as described in \secref{sec:forward}.

%%%%%%%%%%%%%%%%%%%%%%%%%%%%%%%%%%%%%%%%%%%%%%%%%%%%%%%%%%%%%%%%%%%%%%%%%%%%%%%%
\subsection{Include by Input}
\label{sec:input}

Including child documents by |\include| has some restrictions by design.
Most notably, the content of a child document always occupies
its own set of pages; pages cannot be shared between child documents.
Usually, this behaviour makes perfect sense
because each child document contain an essential part of the document.
However, in some situations it may be desirable to compose
a document from a collection of parts
without having mandatory page breaks between then.
For this case, the package
provides a mechanism to include parts
by |\input| which can also be processed individually.
However, by construction this mechanism
requires manual handling of the content to be output.

%%%%%%%%%%%%%%%%%%%%%%%%%%%%%%%%%%%%%%%%
\DescribeMacro{\ifchilddocmanual}
The main file should be prepared as usual, see \secref{sec:include}.
However, the document body must make a distinction
between processing of an individual part and of the main document, e.g.:
%
\begin{center}
\begin{tabular}{l}
|\ifchilddocmanual|\\
|\input{\childdocname}|\\
|\||else|\\
\textit{document body with }|\input{|\textit{part}|}|\\
|\||fi|
\end{tabular}
\end{center}
%
The conditional |\ifchilddocmanual| is true whenever
a part to be included by |\input| is being compiled,
and the name of the part is stored in |\childdocname|.

%%%%%%%%%%%%%%%%%%%%%%%%%%%%%%%%%%%%%%%%
\DescribeMacro{\childdocby}
Each part to be included by |\input| should start with:
%
\begin{center}
\begin{tabular}{l}
|\input{childdoc.def}|\\
|\childdocby{|\textit{main}|}|\\
\end{tabular}
\end{center}
%
The directive |\childdocby| is similar to |\childdocof|
described in \secref{sec:include},
but the subsequent selection of content must be done manually.
To that end, both |\ifchilddoc| and |\ifchilddocmanual|
will be true upon processing of a part,
and the name of the part is stored in |\childdocname|.
Note that |\jobname| will be set to the filename of the current part
so that each part receives an individual |.aux| file
that does not interfere with the |.aux| file(s) of the main document.
This behaviour can be altered by the alternative form
|\childdocby[*]{|\textit{main}|}| (with a non-empty optional argument)
which uses the |.aux| file of the main document
by setting |\jobname| to \textit{main}.

%%%%%%%%%%%%%%%%%%%%%%%%%%%%%%%%%%%%%%%%%%%%%%%%%%%%%%%%%%%%%%%%%%%%%%%%%%%%%%%%
\subsection{Driver Development}
\label{sec:driver}

The \textsf{childdoc} mechanism can also be use for the development
of definition files such as \LaTeX{} styles or classes.
This case differs from the above setup with multiple parts
included by |\include| in that no |\includeonly| should be invoked.
This can be achieved by starting the include file
(before |\ProvidesPackage|) with:
%
\begin{center}
\begin{tabular}{l}
|\input{childdoc.def}|\\
|\childdocforward{|\textit{main}|}|\\
\end{tabular}
\end{center}
%
or alternatively with:
%
\begin{center}
\begin{tabular}{l}
|\input{childdoc.def}|\\
|\childdocby{|\textit{main}|}|\\
\end{tabular}
\end{center}
%
Both forms have slightly different effects as described above.
The main file is prepared as usual, see \secref{sec:include}.

%%%%%%%%%%%%%%%%%%%%%%%%%%%%%%%%%%%%%%%%%%%%%%%%%%%%%%%%%%%%%%%%%%%%%%%%%%%%%%%%
\subsection{Legacy Detection}
\label{sec:detection}

The directive |\childdocmain| in the main file can detect
whether the complete document or merely a child is to be compiled
even without using the directive |\childdocof|.
This method is deprecated because it is less robust
and there is no compelling reason to use it;
it is merely provided for backward compatibility
and it may be removed in future versions.

If the detection mechanism is to be used,
it is mandatory to correctly specify
the filename of the main file as the argument of |\childdocmain|:
%
\begin{center}
\begin{tabular}{l}
|\input{childdoc.def}|\\
|\childdocmain{|\textit{main}|}|\\
\end{tabular}
\end{center}
%
If |\jobname| does not match the argument \textit{main} of |\childdocmain|,
it is assumed that |\jobname| points to the child file to be compiled.
When using |\childdocmain| with the main file specified as argument,
it suffices to start a child file
with just |\input{|\textit{main}|}|
without loading of the package and using |\childdocof|.
If instead all processing is done
with the appropriate \textsf{childdoc} directives,
the argument of \textit{main} of |\childdocmain| can be empty.

An alternative version of the command line processing described
in \secref{sec:commandline} using the detection mechanism reads:
%
\begin{center}
|... -jobname "|\textit{target}|" "|[\textit{flags}]%
[|\def\jobname{|\textit{dest}|}|]|\input{|\textit{main}|}"|
\end{center}

%%%%%%%%%%%%%%%%%%%%%%%%%%%%%%%%%%%%%%%%%%%%%%%%%%%%%%%%%%%%%%%%%%%%%%%%%%%%%%%%
\subsection{Manual Code}
\label{sec:manual}

In case one cannot be certain whether the definitions file |childdoc.def|
is installed on the target \TeX{} distribution
and one prefers not to ship it,
it is conceivable to paste a few relevant commands into the sources.

To that end, drop all statements |\input{childdoc.def}|
and perform the replacements as outlined below.
Instead of |\childdocmain{|\textit{main}|}| add the following code
to the top of the main file:
%
\begin{center}
\begin{tabular}{l}
|\||ifdefined\childdocname\endinput\||fi\newif\ifchilddoc|\\
|\edef\childdocname{\scantokens\expandafter{\jobname\noexpand}}|\\
|\def\childdocmain{|\textit{main}|}\||ifx\childdocmain\childdocname\||else|\\
|\childdoctrue\includeonly{\childdocname}\let\jobname\childdocmain\||fi|\\
\end{tabular}
\end{center}
%
Instead of |\childdocof{|\textit{main}|}| just include the main file
at the top of each child file:
%
\begin{center}
|\input{|\textit{main}|}|
\end{center}
%
A simple redirection |\childdocforward{|\textit{dest}|}| is achieved by:
%
\begin{center}
|\def\jobname{|\textit{dest}|}\input{\jobname}|
\end{center}
%
The redirection with prefix
|\childdocforwardprefix[|\textit{prefix}|]{|\textit{dest}|}|
is accomplished by:
%
\begin{center}
\begin{tabular}{l}
|{\edef\jobname{\scantokens\expandafter{\jobname\noexpand}}|\\
|\def\redirectjob |\textit{prefix}|#1~~~{\gdef\jobname{|\textit{dest}|#1}}|\\
|\expandafter\redirectjob\jobname~~~}\input{\jobname}|
\end{tabular}
\end{center}

In an alternative approach,
child documents can be compiled by a specific command line
without additional code or specific definitions:
%
\begin{center}
|... -jobname "|\textit{target}|" "|[\textit{flags}]%
|\includeonly{|\textit{dest}|}\input{|\textit{main}|}"|
\end{center}
%

%%%%%%%%%%%%%%%%%%%%%%%%%%%%%%%%%%%%%%%%%%%%%%%%%%%%%%%%%%%%%%%%%%%%%%%%%%%%%%%%
%%%%%%%%%%%%%%%%%%%%%%%%%%%%%%%%%%%%%%%%%%%%%%%%%%%%%%%%%%%%%%%%%%%%%%%%%%%%%%%%
\section{Information}

%%%%%%%%%%%%%%%%%%%%%%%%%%%%%%%%%%%%%%%%%%%%%%%%%%%%%%%%%%%%%%%%%%%%%%%%%%%%%%%%
\subsection{Copyright}

Copyright \copyright{} 2017--2018 Niklas Beisert

This work may be distributed and/or modified under the
conditions of the \LaTeX{} Project Public License, either version 1.3
of this license or (at your option) any later version.
The latest version of this license is in
  \url{http://www.latex-project.org/lppl.txt}
and version 1.3 or later is part of all distributions of \LaTeX{}
version 2005/12/01 or later.

This work has the LPPL maintenance status `maintained'.

The Current Maintainer of this work is Niklas Beisert.

This work consists of the files |README.txt|, |childdoc.ins| and |childdoc.dtx|
as well as the derived files |childdoc.def|, |cdocsamp.tex|
with |cdocsch1.tex|, |cdocsch2.tex|, |cdocspt3.tex|, |cdocspt4.tex|,
|cdocsdrf.tex|, |cdocsfn1.tex|, |cdocsfn2.tex|
as well as |childdoc.pdf|.

%%%%%%%%%%%%%%%%%%%%%%%%%%%%%%%%%%%%%%%%%%%%%%%%%%%%%%%%%%%%%%%%%%%%%%%%%%%%%%%%
\subsection{Files and Installation}

The package consists of the files:
%
\begin{center}
\begin{tabular}{ll}
    |README.txt|   & readme file \\
    |childdoc.ins| & installation file \\
    |childdoc.dtx| & source file \\
    |childdoc.def| & definition file \\
    |cdocsamp.tex| & sample main file \\
    |cdocsch1.tex| & sample include file \\
    |cdocsch2.tex| & sample include file \\
    |cdocspt3.tex| & sample part file \\
    |cdocspt4.tex| & sample part file \\
    |cdocsdrf.tex| & sample redirection file \\
    |cdocsfn1.tex| & sample redirection file \\
    |cdocsfn2.tex| & sample redirection file \\
    |childdoc.pdf| & manual
\end{tabular}
\end{center}
%
The distribution consists of the files
|README.txt|, |childdoc.ins| and |childdoc.dtx|.
%
\begin{itemize}
\item
Run (pdf)\LaTeX{} on |childdoc.dtx|
to compile the manual |childdoc.pdf| (this file).
\item
Run \LaTeX{} on |childdoc.ins| to create the definitions file |childdoc.def|
and the sample |cdocsamp.tex| with include files
|cdocsch1.tex|, |cdocsch2.tex|, |cdocspt3.tex|, |cdocspt4.tex|,
|cdocsdrf.tex|, |cdocsfn1.tex|, |cdocsfn2.tex|.
Then copy the file |childdoc.def| to an appropriate directory of your \LaTeX{}
distribution, e.g.\ \textit{texmf-root}|/tex/latex/childdoc|.
\end{itemize}

%%%%%%%%%%%%%%%%%%%%%%%%%%%%%%%%%%%%%%%%%%%%%%%%%%%%%%%%%%%%%%%%%%%%%%%%%%%%%%%%
\subsection{Related CTAN Packages}

There are several other packages which offer a similar functionality:
%
\begin{itemize}
\item
The packages
\href{http://ctan.org/pkg/docmute}{\textsf{docmute}},
\href{http://ctan.org/pkg/includex}{\textsf{includex}} and
\href{http://ctan.org/pkg/standalone}{\textsf{standalone}}
provide commands to include only the document body of
a child file thus allowing both files to be compiled individually.
\item
The packages \href{http://ctan.org/pkg/subdocs}{\textsf{subdocs}}
and \href{http://ctan.org/pkg/subfiles}{\textsf{subfiles}}
provide structures in which the main and child documents can be
encapsulated and allowing them to be compiled individually.
The inclusion mechanism is different from the conventional |\include|.
\item
The package \href{http://ctan.org/pkg/combine}{\textsf{combine}}
is an elaborate solution to combine several documents into one.
\end{itemize}
%
See also the CTAN topic \href{http://ctan.org/topic/subdocs}{\textsf{subdocs}}
for further related packages.
The present package differs from the above solutions in that
a document structure constructed with the conventional |\include| mechanism
just needs two extra commands at the top of every file
such that all constituent files can be compiled individually.

%%%%%%%%%%%%%%%%%%%%%%%%%%%%%%%%%%%%%%%%%%%%%%%%%%%%%%%%%%%%%%%%%%%%%%%%%%%%%%%%
%\subsection{Feature Suggestions}
%
%The following is a list of features which may be useful for future
%versions of this package:
%%
%\begin{itemize}
%\item
%\ldots
%\end{itemize}

%%%%%%%%%%%%%%%%%%%%%%%%%%%%%%%%%%%%%%%%%%%%%%%%%%%%%%%%%%%%%%%%%%%%%%%%%%%%%%%%
\subsection{Revision History}

%%%%%%%%%%%%%%%%%%%%%%%%%%%%%%%%%%%%%%%%
\paragraph{v2.0:} 2018/12/30

\begin{itemize}
\item
immediate forward processing
\item
added |\childdocby| mechanism
\item
manual restructured
\end{itemize}

%%%%%%%%%%%%%%%%%%%%%%%%%%%%%%%%%%%%%%%%
\paragraph{v1.6:} 2018/01/17

\begin{itemize}
\item
application for development of include files
\item
corrections to manual
\end{itemize}

%%%%%%%%%%%%%%%%%%%%%%%%%%%%%%%%%%%%%%%%
\paragraph{v1.5:} 2017/05/21

\begin{itemize}
\item
more complete structuring introduced
\item
|\childdocof| introduced
\item
|\childdoc| renamed to |\childdocmain|
\item
|\childredirect| renamed to |\childdocforward| and |\childdocforwardprefix|
and functionality expanded
\end{itemize}

%%%%%%%%%%%%%%%%%%%%%%%%%%%%%%%%%%%%%%%%
\paragraph{v1.0:} 2017/04/27

\begin{itemize}
\item
manual and install package
\item
first version published on CTAN
\end{itemize}

%%%%%%%%%%%%%%%%%%%%%%%%%%%%%%%%%%%%%%%%
\paragraph{v0.6:} 2017/04/26

\begin{itemize}
\item
redirection mechanism added
\end{itemize}

%%%%%%%%%%%%%%%%%%%%%%%%%%%%%%%%%%%%%%%%
\paragraph{v0.5:} 2017/04/26

\begin{itemize}
\item
functionality in definition file
\end{itemize}


%%%%%%%%%%%%%%%%%%%%%%%%%%%%%%%%%%%%%%%%%%%%%%%%%%%%%%%%%%%%%%%%%%%%%%%%%%%%%%%%
%%%%%%%%%%%%%%%%%%%%%%%%%%%%%%%%%%%%%%%%%%%%%%%%%%%%%%%%%%%%%%%%%%%%%%%%%%%%%%%%
%%%%%%%%%%%%%%%%%%%%%%%%%%%%%%%%%%%%%%%%%%%%%%%%%%%%%%%%%%%%%%%%%%%%%%%%%%%%%%%%
\appendix

\settowidth\MacroIndent{\rmfamily\scriptsize 000\ }

 \DocInput{childdoc.dtx}

\end{document}
%</driver>
% \fi
%
% %%%%%%%%%%%%%%%%%%%%%%%%%%%%%%%%%%%%%%%%%%%%%%%%%%%%%%%%%%%%%%%%%%%%%%%%%%%%%%
% %%%%%%%%%%%%%%%%%%%%%%%%%%%%%%%%%%%%%%%%%%%%%%%%%%%%%%%%%%%%%%%%%%%%%%%%%%%%%%
% \section{Sample}
%\iffalse
%<*samplemain>
%\fi
%
% The following presents a sample document
% with two chapters, two parts, a title page,
% a compile flag as well as three forwarding files to set the flag.
% It consists of eight |.tex| files:
% \begin{center}
% \begin{tabular}{ll}
% |cdocsamp.tex|&main file\\
% |cdocsch1.tex|&include file for chapter 1\\
% |cdocsch2.tex|&include file for chapter 2\\
% |cdocspt3.tex|&include file for part 3\\
% |cdocspt4.tex|&include file for part 4\\
% |cdocsdrf.tex|&forwarding file for main file in draft mode\\
% |cdocsfi1.tex|&forwarding file for final version of chapter 1\\
% |cdocsfi2.tex|&forwarding file for final version of chapter 2\\
% \end{tabular}
% \end{center}
% Each of the eight files can be compiled directly by the \LaTeX{} compiler.
%
% %%%%%%%%%%%%%%%%%%%%%%%%%%%%%%%%%%%%%%
% \paragraph{Main File.}
%
% The main file is called |cdocsamp.tex|.
%
% Load the \textsf{childdoc} definitions and
% declare the filename for the main document:
%    \begin{macrocode}
\input{childdoc.def}
\childdocmain{}
%    \end{macrocode}

% Optional override for |\version| flag:
%    \begin{macrocode}
%%\ifchilddoc\else\providecommand{\version}{draft}\fi
%    \end{macrocode}

% Define the default values for the |\version| flag
% (|final| for the main file and |draft| for childs):
%    \begin{macrocode}
\ifchilddoc
\providecommand{\version}{draft}
\else
\providecommand{\version}{final}
\fi
%    \end{macrocode}

% Load the standard document class:
%    \begin{macrocode}
\documentclass[12pt]{article}
%    \end{macrocode}

% Start the document body:
%    \begin{macrocode}
\begin{document}
%    \end{macrocode}

% Declare a title page.
% Print title, part of document being processed and version flag:
%    \begin{macrocode}
\addtocounter{page}{-1}
\begin{center}
{\LARGE\bfseries{}childdoc example\par}
\vspace{1cm}
\ifchilddoc
\ifchilddocmanual part\else chapter\fi:
`\childdocname' of `\childdocjob'\par
\else
main document: `\childdocjob'\par
\fi
version: \version\par
\end{center}
\newpage
%    \end{macrocode}

% Manually include selected file,
% otherwise process as usual:
%    \begin{macrocode}
\ifchilddocmanual
\section*{part `\childdocname'}
\input{\childdocname}
\else
%    \end{macrocode}

% Include the two chapters:
%    \begin{macrocode}
\include{cdocsch1}
\include{cdocsch2}
%    \end{macrocode}

% Include the two parts unless only chapters should be displayed:
%    \begin{macrocode}
\ifchilddoc\else
\section{part three}
\input{cdocspt3}
\section{part four}
\input{cdocspt4}
\fi
%    \end{macrocode}

% Process as usual until here:
%    \begin{macrocode}
\fi
%    \end{macrocode}

% End of document body:
%    \begin{macrocode}
\end{document}
%    \end{macrocode}
%\iffalse
%</samplemain>
%\fi
%
% %%%%%%%%%%%%%%%%%%%%%%%%%%%%%%%%%%%%%%
% \paragraph{Chapter Include Files.}
%
% The include files are called |cdocsch1.tex| and |cdocsch2.tex|.
%
%\iffalse
%<*samplechap1|samplechap2>
%\fi

% Optional override for |\version| flag:
%    \begin{macrocode}
%%\providecommand{\version}{final}
%    \end{macrocode}

% Include the main document:
%    \begin{macrocode}
\input{childdoc.def}
\childdocof{cdocsamp}
%    \end{macrocode}

%\iffalse
%</samplechap1|samplechap2>
%\fi
%
%\iffalse
%<*samplechap1>
%\fi
% Some text for chapter 1:
%    \begin{macrocode}
\section{one}
some text in chapter one
%    \end{macrocode}

%\iffalse
%</samplechap1>
%\fi
% Some text for chapter 2:
%\iffalse
%<*samplechap2>
%\fi
%    \begin{macrocode}
\section{two}
more text in chapter two
%    \end{macrocode}

%\iffalse
%</samplechap2>
%\fi
%
% %%%%%%%%%%%%%%%%%%%%%%%%%%%%%%%%%%%%%%
% \paragraph{Part Include Files.}
%
% The include files are called |cdocspt3.tex| and |cdocspt4.tex|.
%
%\iffalse
%<*samplepart3|samplepart4>
%\fi

% Optional override for |\version| flag:
%    \begin{macrocode}
%%\providecommand{\version}{final}
%    \end{macrocode}

% Include the main document:
%    \begin{macrocode}
\input{childdoc.def}
\childdocby{cdocsamp}
%    \end{macrocode}

%\iffalse
%</samplepart3|samplepart4>
%\fi
%
%\iffalse
%<*samplepart3>
%\fi
% Some text for part 3:
%    \begin{macrocode}
some text in part three
%    \end{macrocode}

%\iffalse
%</samplepart3>
%\fi
% Some text for part 4:
%\iffalse
%<*samplepart4>
%\fi
%    \begin{macrocode}
more text in part four
%    \end{macrocode}

%\iffalse
%</samplepart4>
%\fi
%
% %%%%%%%%%%%%%%%%%%%%%%%%%%%%%%%%%%%%%%
% \paragraph{Forwarding for a Complete Draft.}
%
% The following forwarding file |cdocsdrf.tex|
% compiles the main document in draft mode:
%\iffalse
%<*sampledraft>
%\fi
%    \begin{macrocode}
\def\version{draft}
\input{childdoc.def}
\childdocforward{cdocsamp}
%    \end{macrocode}

%\iffalse
%</sampledraft>
%\fi
%
% %%%%%%%%%%%%%%%%%%%%%%%%%%%%%%%%%%%%%%
% \paragraph{Forwarding for Final Version of the Chapters.}
%
% The following forwarding files |cdocsfn1.tex| and |cdocsfn2.tex|
% (with identical content)
% compile the final versions of the child documents
% |cdocsch1.tex| and |cdocsch2.tex|, respectively:
%\iffalse
%<*samplefinal>
%\fi
%    \begin{macrocode}
\def\version{final}
\input{childdoc.def}
\childdocforwardprefix[cdocsamp]{cdocsfn}{cdocsch}
%    \end{macrocode}

%\iffalse
%</samplefinal>
%\fi
%
% %%%%%%%%%%%%%%%%%%%%%%%%%%%%%%%%%%%%%%
% \paragraph{Command Line Processing.}
%
% The following three command lines generate the output files
% |cdocscld|, |cdocscl1| and |cdocscl2|
% which should be identical to
% |cdocsdrf|, |cdocsch1| and |cdocsfn2|, respectively:
% \begin{center}
% \begin{tabular}{l}
% |latex -jobname cdocscld \|\\
% |  "\def\version{draft}\input{childdoc.def}\childdocforward{cdocsamp}"|\\
% |latex -jobname cdocscl1 \|\\
% |  "\input{childdoc.def}\childdocforward[cdocsamp]{cdocsch1}"|\\
% |latex -jobname cdocscl2 \|\\
% |  "\def\version{final}\input{childdoc.def}\childdocforward{cdocsch2}"|
% \end{tabular}
% \end{center}
% Note that the trailing backslash on each first line
% merely continues the input to the second line
% (for convenient cut ant paste).
% Furthermore, the command |latex| can be replaced by any
% of its alternative versions such as |pdflatex|.
%
% %%%%%%%%%%%%%%%%%%%%%%%%%%%%%%%%%%%%%%%%%%%%%%%%%%%%%%%%%%%%%%%%%%%%%%%%%%%%%%
% %%%%%%%%%%%%%%%%%%%%%%%%%%%%%%%%%%%%%%%%%%%%%%%%%%%%%%%%%%%%%%%%%%%%%%%%%%%%%%
% \section{Implementation}
%\iffalse
%<*package>
%\fi
%
% This section describes the definitions file |childdoc.def|.

% The definitions cannot be loaded using |\usepackage| or |\RequirePackage|
% which has a mechanism to prevent loading a style file more than once.
% When loading the definitions by means of |\input|
% multiple instances have to be prevented manually:
%\iffalse
%This code needs to be before the `\ProvidesFile' directive
%which is defined at the beginning of this file.
%Therefore it is also placed there and commented out here.
%</package>
%<*discard>
%\fi
%    \begin{macrocode}
\ifdefined\childdocmain\endinput\fi
%    \end{macrocode}
%\iffalse
%</discard>
%<*package>
%\fi
%
% \macro{\ifchilddoc}
% \macro{\ifchilddocmanual}
% The conditional |\ifchilddoc| tells whether a
% child (true) or main (false) document is being compiled.
% The conditional |\ifchilddocmanual| tells whether
% the |\includeonly| mechanism is used (false) or
% the selection of child files must be performed manually (true).
% The definitions initialise to false:
%    \begin{macrocode}
\newif\ifchilddoc
\newif\ifchilddocmanual
%    \end{macrocode}

% \macro{\childdocname}
% \macro{\childdocjob}
% The macro |\childdocname| stores the name of the main document
% to be compiled. The macro |\childdocjob| stores the name of
% the document on which the \LaTeX{} compiler was originally invoked.
% The content of |\jobname| cannot be compared
% to filenames specified in the source due to different catcodes.
% The following code rescans |\jobname|, stores the result
% in |\childdocname| and saves a copy in |\childdocjob|:
%    \begin{macrocode}
\edef\childdocname{\scantokens\expandafter{\jobname\noexpand}}
\let\childdocjob\childdocname
%    \end{macrocode}

% \macro{\childdocdisable}
% The macro |\childdocdisable| prevents the main file
% from being processed more than once.
% At this stage, the main document command |\childdocmain|
% is assumed to be called once again where it should do nothing.
% Any subsequent call to it should prevent
% a secondary processing of the main document
% It overwrites the forwarding commands
% |\childdocof| and |\childdocforward|
% with empty macros to prevent further inclusions of the main document:
%    \begin{macrocode}
\newcommand{\childdocdisable}
{
  \renewcommand{\childdocmain}[1]{\renewcommand{\childdocmain}[1]{\endinput}}
  \renewcommand{\childdocof}[1]{}
  \renewcommand{\childdocby}[2][]{}
  \renewcommand{\childdocforward}[2][]{}
  \renewcommand{\childdocdisable}{}
}
%    \end{macrocode}

% \macro{\childdocmain}
% The macro |\childdocmain| is to be called at the top of the main file
% with nothing or the main filename (without extension) as argument.
% First, it breaks loops.
% If the argument is not empty and does not match |\childdocname|
% (which is set by the first inclusion of |childdoc.def|),
% |\ifchilddoc| is set to true, |\includeonly| is applied to the child file
% and |\jobname| is set to the main file
% (for proper handling of |.aux| files):
%    \begin{macrocode}
\newcommand{\childdocmain}[1]
{
  \childdocdisable\childdocmain{}
  \if?#1?\else
    \begingroup
      \def\childdoctmp{#1}
      \ifx\childdoctmp\childdocname
        \def\childdoctmp{}
      \else
        \def\childdoctmp
        {
          \childdoctrue
          \includeonly{\childdocname}
          \def\childdocjob{#1}
          \def\jobname{#1}
        }
      \fi
      \expandafter
    \endgroup
    \childdoctmp
  \fi
}
%    \end{macrocode}

% \macro{\childdocof}
% The command |\childdocof| redirects
% compilation to the main file |#1|.
%    \begin{macrocode}
\newcommand{\childdocof}[1]
{
  \childdocdisable
  \childdoctrue
  \includeonly{\childdocname}
  \def\jobname{#1}
  \def\childdocjob{#1}
  \input{#1}
}
%    \end{macrocode}

% \macro{\childdocby}
% The command |\childdocby| ....
%    \begin{macrocode}
\newcommand{\childdocby}[2][]
{
  \childdocdisable
  \childdoctrue
  \childdocmanualtrue
  \if?#1?\else
    \def\jobname{#2}
  \fi
  \def\childdocjob{#2}
  \input{#2}
  \endinput
}
%    \end{macrocode}

% \macro{\childdocforward}
% The command |\childdocforward| redirects
% compilation to the main file or
% (if the optional argument is given) a child file.
% Parameters are set as if the main file
% or a child file starting with |\childdocof| was compiled.
% Then compilation is handed over to the main file:
%    \begin{macrocode}
\newcommand{\childdocforward}[2][]
{
  \begingroup
    \if?#1?
      \def\childdoctmp
      {
        \def\childdocname{#2}
        \def\childdocjob{#2}
        \def\jobname{#2}
        \input{#2}
        \endinput
      }
    \else
      \def\childdoctmp
      {
        \childdocdisable
        \def\childdocname{#2}
        \childdoctrue
        \includeonly{#2}
        \def\childdocjob{#1}
        \def\jobname{#1}
        \input{#1}
        \endinput
      }
    \fi
    \expandafter
  \endgroup
  \childdoctmp
}
%    \end{macrocode}

% \macro{\childdocforwardprefix}
% The command |\childdocforwardprefix| redirects
% compilation to the main or a child file by means of a pattern.
% The prefix |#1| in the current filename is replaced by |#2|
% and the suffix of the current filename is kept
% (it is assumed that the filename does not contain the substring `|~~~|'
% which is used as a delimiter).
% Compilation is handed over to the new file by |\childdocforward|:
%    \begin{macrocode}
\newcommand{\childdocforwardprefix}[3][]
{
  \begingroup
    \def\childdocextract #2##1~~~{\def\childdoctmp{\childdocforward[#1]{#3##1}}}
    \expandafter\childdocextract\childdocname~~~
    \expandafter
  \endgroup
  \childdoctmp
}
%    \end{macrocode}

% \macro{\childdoc}
% The deprecated macro |\childdoc| is a legacy version of |\childdocmain|:
%    \begin{macrocode}
\newcommand{\childdoc}{\childdocmain}
%    \end{macrocode}

% \macro{\childdocredirect}
% The deprecated macro |\childdocredirect| is a legacy version
% of |\childdocforward| and |\childdocforwardprefix|:
%    \begin{macrocode}
\newcommand{\childdocredirect}[2][]
{
  \begingroup
    \if?#1?
      \def\childdoctmp{\childdocforward{#2}}
    \else
      \def\childdoctmp{\childdocforwardprefix{#1}{#2}}
    \fi
    \expandafter
  \endgroup
  \childdoctmp
}
%    \end{macrocode}

%\iffalse
%</package>
%\fi
%
\endinput
|\\
|\childdocby{|\textit{main}|}|\\
\end{tabular}
\end{center}
%
Both forms have slightly different effects as described above.
The main file is prepared as usual, see \secref{sec:include}.

%%%%%%%%%%%%%%%%%%%%%%%%%%%%%%%%%%%%%%%%%%%%%%%%%%%%%%%%%%%%%%%%%%%%%%%%%%%%%%%%
\subsection{Legacy Detection}
\label{sec:detection}

The directive |\childdocmain| in the main file can detect
whether the complete document or merely a child is to be compiled
even without using the directive |\childdocof|.
This method is deprecated because it is less robust
and there is no compelling reason to use it;
it is merely provided for backward compatibility
and it may be removed in future versions.

If the detection mechanism is to be used,
it is mandatory to correctly specify
the filename of the main file as the argument of |\childdocmain|:
%
\begin{center}
\begin{tabular}{l}
|% \iffalse
%
% childdoc.dtx Copyright (C) 2017-2018 Niklas Beisert
%
% This work may be distributed and/or modified under the
% conditions of the LaTeX Project Public License, either version 1.3
% of this license or (at your option) any later version.
% The latest version of this license is in
%   http://www.latex-project.org/lppl.txt
% and version 1.3 or later is part of all distributions of LaTeX
% version 2005/12/01 or later.
%
% This work has the LPPL maintenance status `maintained'.
%
% The Current Maintainer of this work is Niklas Beisert.
%
% This work consists of the files childdoc.dtx and childdoc.ins
% and the derived files childdoc.def and cdocsamp.tex with
% cdocsch1.tex, cdocsch2.tex, cdocsdrf.tex, cdocsfn1.tex, cdocsfn2.tex.
%
%<package>\ifdefined\childdocmain\endinput\fi
%<package>\ProvidesFile{childdoc.def}[2018/12/30 v2.0 child document driver]
%<samplemain>\ProvidesFile{cdocsamp.tex}[2018/12/30 v2.0 sample for childdoc]
%<*driver>
%\ProvidesFile{childdoc.drv}[2018/12/30 v2.0 childdoc reference manual file]
\PassOptionsToClass{10pt,a4paper}{article}
\documentclass{ltxdoc}

\usepackage[margin=35mm]{geometry}
\usepackage{hyperref}
\usepackage{hyperxmp}
\usepackage[usenames]{color}

\hypersetup{colorlinks=true}
\hypersetup{pdfstartview=FitH}
\hypersetup{pdfpagemode=UseNone}
\hypersetup{pdfsource={}}
\hypersetup{pdflang={en-UK}}
\hypersetup{pdfcopyright={Copyright 2017-2018 Niklas Beisert.
  This work may be distributed and/or modified under the
  conditions of the LaTeX Project Public License, either version 1.3
  of this license or (at your option) any later version.}}
\hypersetup{pdflicenseurl={http://www.latex-project.org/lppl.txt}}
\hypersetup{pdfcontactaddress={ETH Zurich, ITP, HIT K,
  Wolfgang-Pauli-Strasse 27}}
\hypersetup{pdfcontactpostcode={8093}}
\hypersetup{pdfcontactcity={Zurich}}
\hypersetup{pdfcontactcountry={Switzerland}}
\hypersetup{pdfcontactemail={nbeisert@itp.phys.ethz.ch}}
\hypersetup{pdfcontacturl={http://people.phys.ethz.ch/\xmptilde nbeisert/}}

\newcommand{\secref}[1]{\hyperref[#1]{section \ref*{#1}}}

\parskip1ex
\parindent0pt
\let\olditemize\itemize
\def\itemize{\olditemize\parskip0pt}

\begin{document}

\title{The \textsf{childdoc} Package}
\hypersetup{pdftitle={The childdoc Package}}
\author{Niklas Beisert\\[2ex]
  Institut f\"ur Theoretische Physik\\
  Eidgen\"ossische Technische Hochschule Z\"urich\\
  Wolfgang-Pauli-Strasse 27, 8093 Z\"urich, Switzerland\\[1ex]
  \href{mailto:nbeisert@itp.phys.ethz.ch}
  {\texttt{nbeisert@itp.phys.ethz.ch}}}
\hypersetup{pdfauthor={Niklas Beisert}}
\hypersetup{pdfsubject={Manual for the LaTeX2e Package childdoc}}
\date{30 December 2018, \textsf{v2.0}}
\maketitle

\begin{abstract}\noindent
\textsf{childdoc} is a \LaTeXe{} package
that enables the direct compilation
of document sections included by |\include|
to individual files.
\end{abstract}

\begingroup
\parskip0ex
\tableofcontents
\endgroup

%%%%%%%%%%%%%%%%%%%%%%%%%%%%%%%%%%%%%%%%%%%%%%%%%%%%%%%%%%%%%%%%%%%%%%%%%%%%%%%%
%%%%%%%%%%%%%%%%%%%%%%%%%%%%%%%%%%%%%%%%%%%%%%%%%%%%%%%%%%%%%%%%%%%%%%%%%%%%%%%%
\section{Introduction}

\LaTeX{} provides a mechanism to structure a large document (such as a book)
into a main file and several child files (containing the chapters)
using the |\include| command.
This mechanism is beneficial for documents
which span hundreds of pages in order to
make the source file(s) more manageable.
Moreover, compilation can be restricted to
selected child files by means of the |\includeonly| command.
The latter feature can be used to reduce the compilation time while editing
(this was significantly more useful in the earlier days of \LaTeX{})
or to generate a smaller document which is easier to navigate.
Another application of |\includeonly| is to generate
documents consisting of selected parts of the complete document.

However, there are a few drawbacks of the plain |\include| mechanism:
\begin{itemize}
\item
The child files cannot be compiled on their own,
they can only be compiled via the main file.
A naive editing environment
(such as a text editor with an option
to have the current file processed by \LaTeX)
may require one to switch to the main file before compiling;
attempting to compile the child file produces errors.
\item
The main file must be modified (each time)
to adjust the |\includeonly| command
to the present needs. This easily leaves the main file in a messy state.
\item
The generated document will always carry the filename
of the main document. This is inconvenient if
several child files are to be compiled and
to be kept for distribution.
\end{itemize}

The present package provides a simple interface
to make child files individually compilable by \LaTeX{}.
Compiling a child file then has the same effect as compiling
the main file with an |\includeonly| command
to select the appropriate child.
Moreover the generated document will carry the name of the child
rather than the main file.
This resolves all three above issues.

This feature is meant to make the editing of books,
thesis documents and lecture notes somewhat more convenient.
However, the package can also be used efficiently for
composing a series of documents (such as exercise sheets)
which are typically distributed individually.
It then assists the author in generating the individual documents
(potentially in different versions)
as well as a document containing the collected series.
Another application is in developing style files
or other kinds of included material
where compilation of the style file could redirect
to a sample or test file.

%%%%%%%%%%%%%%%%%%%%%%%%%%%%%%%%%%%%%%%%%%%%%%%%%%%%%%%%%%%%%%%%%%%%%%%%%%%%%%%%
%%%%%%%%%%%%%%%%%%%%%%%%%%%%%%%%%%%%%%%%%%%%%%%%%%%%%%%%%%%%%%%%%%%%%%%%%%%%%%%%
\section{Usage}

First of all, the package \textsf{childdoc} is \emph{not} a standard
\LaTeXe{} |.sty| style file! Therefore it needs to be invoked in
a non-standard way.

%%%%%%%%%%%%%%%%%%%%%%%%%%%%%%%%%%%%%%%%%%%%%%%%%%%%%%%%%%%%%%%%%%%%%%%%%%%%%%%%
\subsection{Included Files}
\label{sec:include}

%%%%%%%%%%%%%%%%%%%%%%%%%%%%%%%%%%%%%%%%
\DescribeMacro{\childdocmain}
To use the package, add the commands
\begin{center}
\begin{tabular}{l}
|\input{childdoc.def}|\\
|\childdocmain{}|\\
\end{tabular}
\end{center}
at the very top of the main \LaTeX{} file,
in particular \emph{before} the |\documentclass| statement!
The argument of |\childdocmain| should be left empty
(but it must be present).

%%%%%%%%%%%%%%%%%%%%%%%%%%%%%%%%%%%%%%%%
\DescribeMacro{\childdocof}
Furthermore, add the commands
\begin{center}
\begin{tabular}{l}
|\input{childdoc.def}|\\
|\childdocof{|\textit{main}|}|\\
\end{tabular}
\end{center}
at the top of every child file \textit{child}
which is included by |\include{|\textit{child}|}|
from within the main file
(or at least for those files to be compiled individually).
The argument \textit{main} must be the filename of the main file.

There are a couple of
considerations in setting up the main and child documents:

%%%%%%%%%%%%%%%%%%%%%%%%%%%%%%%%%%%%%%%%
\paragraph{Restrictions.}

Please note the following restrictions:
\begin{itemize}
\item
|\childdocmain| must be called with one argument \textit{main}
to ensure compatibility with earlier version of the package.
It must either be empty (|\childdocmain{}|)
or precisely match the filename of the main file in which it is specified.
See \secref{sec:detection} for further information.
\item
The filename \textit{main} must be specified without the |.tex| extension.
\item
The filename \textit{main} is case sensitive
(even in case-insensitive file systems)
due to internal string comparison.
\item
The argument \textit{main} should be fully expanded, it cannot be a macro.
\item
Subdirectories and special characters should be avoided in filenames.
\item
The command |\childdocmain{|\textit{main}|}| must be followed by a whitespace.
It should not be followed immediately by another command
or by a comment mark `|%|'.
This is because the \TeX{} parser reads the token immediately following
the argument of |\childdocmain| and puts it
at the beginning of every child section;
however, a white\-space is ignored.
\end{itemize}

%%%%%%%%%%%%%%%%%%%%%%%%%%%%%%%%%%%%%%%%
\paragraph{Content of Main File.}

It is advisable to place all content in the child files included by |\include|.
Any output contained in the main file will appear in all child documents
unless suppressed manually;
it cannot be suppressed automatically by the |\includeonly| directive
and thus should normally be avoided.
A method to include some content in the main file
by means of conditional processing is described in \secref{sec:conditional}.

%%%%%%%%%%%%%%%%%%%%%%%%%%%%%%%%%%%%%%%%
\paragraph{Page Numbering.}

When only a part of the document is compiled,
the appropriate numbering of pages
(as well as other status parameters)
is determined from the |.aux| files.
The latter contain information from previous passes.
However this information needs to propagate through
all intermediate child documents.
Therefore the page numbering in child documents may well
be inconsistent until the complete document is compiled at least once.

A useful (if unconventional) way to always ensure a consistent
page numbering is to restart the numbering in each child document
and denote the pages by `\textit{child}|.|\textit{page}'
where \textit{child} represents the chapter/section number of the child file.
This can be achieved by the command
|\numberwithin{page}{|\textit{child}|}|
of the \textsf{amsmath} package
where \textit{child} can be |chapter| or |section|
depending on the chosen structuring.
Alternatively, one can modify the macro |\thepage| appropriately
and reset the counter |page| at the start of each child file.

%%%%%%%%%%%%%%%%%%%%%%%%%%%%%%%%%%%%%%%%%%%%%%%%%%%%%%%%%%%%%%%%%%%%%%%%%%%%%%%%
\subsection{Conditional Processing}
\label{sec:conditional}

The package provides a mechanism to compile different versions
of a document. To customise the versions further some conditional processing
can come in handy to distinguish which version is being compiled.
The package provides two macros to describe the compilation context:

%%%%%%%%%%%%%%%%%%%%%%%%%%%%%%%%%%%%%%%%
\DescribeMacro{\ifchilddoc}
The conditional |\ifchilddoc| distinguishes between the compilation of
child documents and the main document:
%
\begin{center}
|\ifchilddoc |\textit{child-code}| |[|\||else |\textit{main-code}]| \||fi|
\end{center}

%%%%%%%%%%%%%%%%%%%%%%%%%%%%%%%%%%%%%%%%
\DescribeMacro{\childdocname}
\DescribeMacro{\childdocjob}
The macro |\childdocname| contains the filename (without extension)
of the main or child file being processed.
Note that |\childdocjob| will always contain the name of the main file.

%%%%%%%%%%%%%%%%%%%%%%%%%%%%%%%%%%%%%%%%
\paragraph{Title Page.}

Conditional processing can be used to include a title or banner page
in the main document when proper precautions are taken.
Importantly, the code in the main file should ensure that the page counter
(as well as other status parameters which are stored in the |.aux| files)
takes the same value after the conditional processing.
Otherwise the page numbers may take divergent values
depending on which part is compiled.

For example, a title page could be declared by:
%
\begin{center}
\begin{tabular}{l}
|\ifchilddoc\||else|\\
|\addtocounter{page}{-1}|\\
\textit{code for title page}\\
|\newpage|\\
|\||fi|
\end{tabular}
\end{center}
%
A banner page for the child documents can be generated by:
%
\begin{center}
\begin{tabular}{l}
|\ifchilddoc|\\
|\addtocounter{page}{-1}|\\
\textit{code for banner page}\\
|\newpage|\\
|\||fi|
\end{tabular}
\end{center}
%
Here one could write a message such as:
\begin{center}
|This is the part \childdocname{} of \childdocjob{}.|
\end{center}

%%%%%%%%%%%%%%%%%%%%%%%%%%%%%%%%%%%%%%%%%%%%%%%%%%%%%%%%%%%%%%%%%%%%%%%%%%%%%%%%
\subsection{Flags}
\label{sec:flags}

The package makes it easy to generate different versions
of the main or child documents.
To this end compilation flags can be defined
and assigned different default values.
They will be particularly useful in conjunction
with the forwarding mechanism described in \secref{sec:forward}.

For example, it may be useful to have a flag |\version|
which can be set to |draft| or |final|.
The document source will contain some conditional code
depending on the value of |\version|.
Suppose further, the flag should default to |final| for the main file
and to |draft| for child files
which is a natural assignment for editing the document.
This is achieved by placing the following code
in the preamble of the main document
(below the |\childdocmain| directive):
%
\begin{center}
\begin{tabular}{l}
|\ifchilddoc|\\
|\providecommand{\version}{draft}|\\
|\||else|\\
|\providecommand{\version}{final}|\\
|\||fi|
\end{tabular}
\end{center}
%
The definition by |\providecommand| makes sure
that previous definitions are not overwritten.
Further statements |\providecommand{\version}{...}|
can thus be added before the above code to override it.

For the main file, one might add a line
(between |\childdocmain| and the above block)
%
\begin{center}
|%\ifchilddoc\||else\providecommand{\version}{draft}\||fi|
\end{center}
%
which can be uncommented to produce a draft version.
Likewise one can add a line to the very top of a child file
(above the |\childdocof{|\textit{main}|}| directive)
%
\begin{center}
|%\providecommand{\version}{final}|
\end{center}
%
which can be uncommented to produce the final version of this child document.

%%%%%%%%%%%%%%%%%%%%%%%%%%%%%%%%%%%%%%%%%%%%%%%%%%%%%%%%%%%%%%%%%%%%%%%%%%%%%%%%
\subsection{Forwarding}
\label{sec:forward}

Different versions of the main or child documents
using compilation flags as described in \secref{sec:flags}
can be (permanently) stored in different files
for convenient compilation, viewing and distribution.
To this end, the package defines a command
to pass on compilation to a different file:

%%%%%%%%%%%%%%%%%%%%%%%%%%%%%%%%%%%%%%%%
\DescribeMacro{\childdocforward}
The command |\childdocforward| redirects processing to
another source file:
%
\begin{center}
\begin{tabular}{l}
|\input{childdoc.def}|\\
|\childdocforward[|\textit{main}|]{|\textit{dest}|}|\\
\end{tabular}
\end{center}
%
The argument \textit{dest} is the destination file
(without extension).
It should be the main file or one of the child files.
Note that further \textsf{childdoc} directives
such as |\childdocof| and |\childdocforward|
in the indicated file will be processed in this form.
The optional argument \textit{main}
passes on directly to the main file \textit{main}
while pretending to compile the child \textit{dest}.
This form behaves as if \textit{dest}
issues |\childdocof{|\textit{main}|}| right away,
and no further \textsf{childdoc} directives will be processed.

%%%%%%%%%%%%%%%%%%%%%%%%%%%%%%%%%%%%%%%%
\DescribeMacro{\...prefix}
In the alternative form |\childdocforwardprefix|,
%
\begin{center}
\begin{tabular}{l}
|\input{childdoc.def}|\\
|\childdocforwardprefix[|\textit{main}|]{|\textit{prefix}|}{|\textit{dest}|}|
\end{tabular}
\end{center}
%
the destination file is determined by a pattern
depending on the current file:
To make this work, the current file must be called
`{\textit{prefix}\hspace{0.2em}\textit{suffix}}'
with \textit{prefix} matching precisely the argument.
Processing is then passed on to the file
`{\textit{dest}\hspace{0.2em}\textit{suffix}}'.
Surely, the same effect is achieved by
directly specifying the
argument `{\textit{dest}\hspace{0.2em}\textit{suffix}}'
in the first form.
However, that requires to set up a different file
for each child. With the alternative form of the command
all these files can have exactly the same content
which simplifies setting them up and maintaining them.

For example, the following file |draft.tex|
with a compilation flag |\version| as described in \secref{sec:flags}
compiles the main document as a draft:
%
\begin{center}
\begin{tabular}{l}
|\def\version{draft}|\\
|\input{childdoc.def}|\\
|\childdocforward{|\textit{main}|}|
\end{tabular}
\end{center}
%
Likewise, the following files |final|\textit{nn}|.tex|
compile the final version of the child document
|child|\textit{nn}|.tex|:
%
\begin{center}
\begin{tabular}{l}
|\def\version{final}|\\
|\input{childdoc.def}|\\
|\childdocforwardprefix{final}{child}|
\end{tabular}
\end{center}
%

Note that when several versions of a main file and/or of each child file
are to be generated, it may be convenient to set up a |Makefile| or
shell script to automatise the process.

%%%%%%%%%%%%%%%%%%%%%%%%%%%%%%%%%%%%%%%%%%%%%%%%%%%%%%%%%%%%%%%%%%%%%%%%%%%%%%%%
\subsection{Command Line Processing}
\label{sec:commandline}

The effect of redirection files can also be achieved by invoking
the \LaTeX{} compiler with a more elaborate command line.
Most conveniently this should be done as part
of a shell script or a |Makefile|.

When using \textsf{childdoc} in the main file, the following
command lines effectively perform a redirection
(note that depending on the shell being used,
backslashes may have to be doubled: `|\|' $\to$ `|\\|'):
%
\begin{center}
|... -jobname "|\textit{target}|" |\\|"|[\textit{flags}]%
|\input{childdoc.def}\childdocforward[|\textit{main}|]{|\textit{dest}|}"|
\end{center}
%
Here \textit{target} is the name of the output file,
\textit{main} is the name of the main file
and \textit{dest} is the name of the main or child file to be processed
(all filenames without extensions).
The optional argument \textit{main} can be omitted
if \textit{main} matches \textit{dest}.
Optionally, compilation \textit{flags} can be defined via |\def| commands.
This command line makes the \TeX{} engine believe
it is compiling the file \textit{target}
whose content is specified as the latter parameter.
The provided code then forwards the processing to
\textit{main} or \textit{dest} as described in \secref{sec:forward}.

%%%%%%%%%%%%%%%%%%%%%%%%%%%%%%%%%%%%%%%%%%%%%%%%%%%%%%%%%%%%%%%%%%%%%%%%%%%%%%%%
\subsection{Include by Input}
\label{sec:input}

Including child documents by |\include| has some restrictions by design.
Most notably, the content of a child document always occupies
its own set of pages; pages cannot be shared between child documents.
Usually, this behaviour makes perfect sense
because each child document contain an essential part of the document.
However, in some situations it may be desirable to compose
a document from a collection of parts
without having mandatory page breaks between then.
For this case, the package
provides a mechanism to include parts
by |\input| which can also be processed individually.
However, by construction this mechanism
requires manual handling of the content to be output.

%%%%%%%%%%%%%%%%%%%%%%%%%%%%%%%%%%%%%%%%
\DescribeMacro{\ifchilddocmanual}
The main file should be prepared as usual, see \secref{sec:include}.
However, the document body must make a distinction
between processing of an individual part and of the main document, e.g.:
%
\begin{center}
\begin{tabular}{l}
|\ifchilddocmanual|\\
|\input{\childdocname}|\\
|\||else|\\
\textit{document body with }|\input{|\textit{part}|}|\\
|\||fi|
\end{tabular}
\end{center}
%
The conditional |\ifchilddocmanual| is true whenever
a part to be included by |\input| is being compiled,
and the name of the part is stored in |\childdocname|.

%%%%%%%%%%%%%%%%%%%%%%%%%%%%%%%%%%%%%%%%
\DescribeMacro{\childdocby}
Each part to be included by |\input| should start with:
%
\begin{center}
\begin{tabular}{l}
|\input{childdoc.def}|\\
|\childdocby{|\textit{main}|}|\\
\end{tabular}
\end{center}
%
The directive |\childdocby| is similar to |\childdocof|
described in \secref{sec:include},
but the subsequent selection of content must be done manually.
To that end, both |\ifchilddoc| and |\ifchilddocmanual|
will be true upon processing of a part,
and the name of the part is stored in |\childdocname|.
Note that |\jobname| will be set to the filename of the current part
so that each part receives an individual |.aux| file
that does not interfere with the |.aux| file(s) of the main document.
This behaviour can be altered by the alternative form
|\childdocby[*]{|\textit{main}|}| (with a non-empty optional argument)
which uses the |.aux| file of the main document
by setting |\jobname| to \textit{main}.

%%%%%%%%%%%%%%%%%%%%%%%%%%%%%%%%%%%%%%%%%%%%%%%%%%%%%%%%%%%%%%%%%%%%%%%%%%%%%%%%
\subsection{Driver Development}
\label{sec:driver}

The \textsf{childdoc} mechanism can also be use for the development
of definition files such as \LaTeX{} styles or classes.
This case differs from the above setup with multiple parts
included by |\include| in that no |\includeonly| should be invoked.
This can be achieved by starting the include file
(before |\ProvidesPackage|) with:
%
\begin{center}
\begin{tabular}{l}
|\input{childdoc.def}|\\
|\childdocforward{|\textit{main}|}|\\
\end{tabular}
\end{center}
%
or alternatively with:
%
\begin{center}
\begin{tabular}{l}
|\input{childdoc.def}|\\
|\childdocby{|\textit{main}|}|\\
\end{tabular}
\end{center}
%
Both forms have slightly different effects as described above.
The main file is prepared as usual, see \secref{sec:include}.

%%%%%%%%%%%%%%%%%%%%%%%%%%%%%%%%%%%%%%%%%%%%%%%%%%%%%%%%%%%%%%%%%%%%%%%%%%%%%%%%
\subsection{Legacy Detection}
\label{sec:detection}

The directive |\childdocmain| in the main file can detect
whether the complete document or merely a child is to be compiled
even without using the directive |\childdocof|.
This method is deprecated because it is less robust
and there is no compelling reason to use it;
it is merely provided for backward compatibility
and it may be removed in future versions.

If the detection mechanism is to be used,
it is mandatory to correctly specify
the filename of the main file as the argument of |\childdocmain|:
%
\begin{center}
\begin{tabular}{l}
|\input{childdoc.def}|\\
|\childdocmain{|\textit{main}|}|\\
\end{tabular}
\end{center}
%
If |\jobname| does not match the argument \textit{main} of |\childdocmain|,
it is assumed that |\jobname| points to the child file to be compiled.
When using |\childdocmain| with the main file specified as argument,
it suffices to start a child file
with just |\input{|\textit{main}|}|
without loading of the package and using |\childdocof|.
If instead all processing is done
with the appropriate \textsf{childdoc} directives,
the argument of \textit{main} of |\childdocmain| can be empty.

An alternative version of the command line processing described
in \secref{sec:commandline} using the detection mechanism reads:
%
\begin{center}
|... -jobname "|\textit{target}|" "|[\textit{flags}]%
[|\def\jobname{|\textit{dest}|}|]|\input{|\textit{main}|}"|
\end{center}

%%%%%%%%%%%%%%%%%%%%%%%%%%%%%%%%%%%%%%%%%%%%%%%%%%%%%%%%%%%%%%%%%%%%%%%%%%%%%%%%
\subsection{Manual Code}
\label{sec:manual}

In case one cannot be certain whether the definitions file |childdoc.def|
is installed on the target \TeX{} distribution
and one prefers not to ship it,
it is conceivable to paste a few relevant commands into the sources.

To that end, drop all statements |\input{childdoc.def}|
and perform the replacements as outlined below.
Instead of |\childdocmain{|\textit{main}|}| add the following code
to the top of the main file:
%
\begin{center}
\begin{tabular}{l}
|\||ifdefined\childdocname\endinput\||fi\newif\ifchilddoc|\\
|\edef\childdocname{\scantokens\expandafter{\jobname\noexpand}}|\\
|\def\childdocmain{|\textit{main}|}\||ifx\childdocmain\childdocname\||else|\\
|\childdoctrue\includeonly{\childdocname}\let\jobname\childdocmain\||fi|\\
\end{tabular}
\end{center}
%
Instead of |\childdocof{|\textit{main}|}| just include the main file
at the top of each child file:
%
\begin{center}
|\input{|\textit{main}|}|
\end{center}
%
A simple redirection |\childdocforward{|\textit{dest}|}| is achieved by:
%
\begin{center}
|\def\jobname{|\textit{dest}|}\input{\jobname}|
\end{center}
%
The redirection with prefix
|\childdocforwardprefix[|\textit{prefix}|]{|\textit{dest}|}|
is accomplished by:
%
\begin{center}
\begin{tabular}{l}
|{\edef\jobname{\scantokens\expandafter{\jobname\noexpand}}|\\
|\def\redirectjob |\textit{prefix}|#1~~~{\gdef\jobname{|\textit{dest}|#1}}|\\
|\expandafter\redirectjob\jobname~~~}\input{\jobname}|
\end{tabular}
\end{center}

In an alternative approach,
child documents can be compiled by a specific command line
without additional code or specific definitions:
%
\begin{center}
|... -jobname "|\textit{target}|" "|[\textit{flags}]%
|\includeonly{|\textit{dest}|}\input{|\textit{main}|}"|
\end{center}
%

%%%%%%%%%%%%%%%%%%%%%%%%%%%%%%%%%%%%%%%%%%%%%%%%%%%%%%%%%%%%%%%%%%%%%%%%%%%%%%%%
%%%%%%%%%%%%%%%%%%%%%%%%%%%%%%%%%%%%%%%%%%%%%%%%%%%%%%%%%%%%%%%%%%%%%%%%%%%%%%%%
\section{Information}

%%%%%%%%%%%%%%%%%%%%%%%%%%%%%%%%%%%%%%%%%%%%%%%%%%%%%%%%%%%%%%%%%%%%%%%%%%%%%%%%
\subsection{Copyright}

Copyright \copyright{} 2017--2018 Niklas Beisert

This work may be distributed and/or modified under the
conditions of the \LaTeX{} Project Public License, either version 1.3
of this license or (at your option) any later version.
The latest version of this license is in
  \url{http://www.latex-project.org/lppl.txt}
and version 1.3 or later is part of all distributions of \LaTeX{}
version 2005/12/01 or later.

This work has the LPPL maintenance status `maintained'.

The Current Maintainer of this work is Niklas Beisert.

This work consists of the files |README.txt|, |childdoc.ins| and |childdoc.dtx|
as well as the derived files |childdoc.def|, |cdocsamp.tex|
with |cdocsch1.tex|, |cdocsch2.tex|, |cdocspt3.tex|, |cdocspt4.tex|,
|cdocsdrf.tex|, |cdocsfn1.tex|, |cdocsfn2.tex|
as well as |childdoc.pdf|.

%%%%%%%%%%%%%%%%%%%%%%%%%%%%%%%%%%%%%%%%%%%%%%%%%%%%%%%%%%%%%%%%%%%%%%%%%%%%%%%%
\subsection{Files and Installation}

The package consists of the files:
%
\begin{center}
\begin{tabular}{ll}
    |README.txt|   & readme file \\
    |childdoc.ins| & installation file \\
    |childdoc.dtx| & source file \\
    |childdoc.def| & definition file \\
    |cdocsamp.tex| & sample main file \\
    |cdocsch1.tex| & sample include file \\
    |cdocsch2.tex| & sample include file \\
    |cdocspt3.tex| & sample part file \\
    |cdocspt4.tex| & sample part file \\
    |cdocsdrf.tex| & sample redirection file \\
    |cdocsfn1.tex| & sample redirection file \\
    |cdocsfn2.tex| & sample redirection file \\
    |childdoc.pdf| & manual
\end{tabular}
\end{center}
%
The distribution consists of the files
|README.txt|, |childdoc.ins| and |childdoc.dtx|.
%
\begin{itemize}
\item
Run (pdf)\LaTeX{} on |childdoc.dtx|
to compile the manual |childdoc.pdf| (this file).
\item
Run \LaTeX{} on |childdoc.ins| to create the definitions file |childdoc.def|
and the sample |cdocsamp.tex| with include files
|cdocsch1.tex|, |cdocsch2.tex|, |cdocspt3.tex|, |cdocspt4.tex|,
|cdocsdrf.tex|, |cdocsfn1.tex|, |cdocsfn2.tex|.
Then copy the file |childdoc.def| to an appropriate directory of your \LaTeX{}
distribution, e.g.\ \textit{texmf-root}|/tex/latex/childdoc|.
\end{itemize}

%%%%%%%%%%%%%%%%%%%%%%%%%%%%%%%%%%%%%%%%%%%%%%%%%%%%%%%%%%%%%%%%%%%%%%%%%%%%%%%%
\subsection{Related CTAN Packages}

There are several other packages which offer a similar functionality:
%
\begin{itemize}
\item
The packages
\href{http://ctan.org/pkg/docmute}{\textsf{docmute}},
\href{http://ctan.org/pkg/includex}{\textsf{includex}} and
\href{http://ctan.org/pkg/standalone}{\textsf{standalone}}
provide commands to include only the document body of
a child file thus allowing both files to be compiled individually.
\item
The packages \href{http://ctan.org/pkg/subdocs}{\textsf{subdocs}}
and \href{http://ctan.org/pkg/subfiles}{\textsf{subfiles}}
provide structures in which the main and child documents can be
encapsulated and allowing them to be compiled individually.
The inclusion mechanism is different from the conventional |\include|.
\item
The package \href{http://ctan.org/pkg/combine}{\textsf{combine}}
is an elaborate solution to combine several documents into one.
\end{itemize}
%
See also the CTAN topic \href{http://ctan.org/topic/subdocs}{\textsf{subdocs}}
for further related packages.
The present package differs from the above solutions in that
a document structure constructed with the conventional |\include| mechanism
just needs two extra commands at the top of every file
such that all constituent files can be compiled individually.

%%%%%%%%%%%%%%%%%%%%%%%%%%%%%%%%%%%%%%%%%%%%%%%%%%%%%%%%%%%%%%%%%%%%%%%%%%%%%%%%
%\subsection{Feature Suggestions}
%
%The following is a list of features which may be useful for future
%versions of this package:
%%
%\begin{itemize}
%\item
%\ldots
%\end{itemize}

%%%%%%%%%%%%%%%%%%%%%%%%%%%%%%%%%%%%%%%%%%%%%%%%%%%%%%%%%%%%%%%%%%%%%%%%%%%%%%%%
\subsection{Revision History}

%%%%%%%%%%%%%%%%%%%%%%%%%%%%%%%%%%%%%%%%
\paragraph{v2.0:} 2018/12/30

\begin{itemize}
\item
immediate forward processing
\item
added |\childdocby| mechanism
\item
manual restructured
\end{itemize}

%%%%%%%%%%%%%%%%%%%%%%%%%%%%%%%%%%%%%%%%
\paragraph{v1.6:} 2018/01/17

\begin{itemize}
\item
application for development of include files
\item
corrections to manual
\end{itemize}

%%%%%%%%%%%%%%%%%%%%%%%%%%%%%%%%%%%%%%%%
\paragraph{v1.5:} 2017/05/21

\begin{itemize}
\item
more complete structuring introduced
\item
|\childdocof| introduced
\item
|\childdoc| renamed to |\childdocmain|
\item
|\childredirect| renamed to |\childdocforward| and |\childdocforwardprefix|
and functionality expanded
\end{itemize}

%%%%%%%%%%%%%%%%%%%%%%%%%%%%%%%%%%%%%%%%
\paragraph{v1.0:} 2017/04/27

\begin{itemize}
\item
manual and install package
\item
first version published on CTAN
\end{itemize}

%%%%%%%%%%%%%%%%%%%%%%%%%%%%%%%%%%%%%%%%
\paragraph{v0.6:} 2017/04/26

\begin{itemize}
\item
redirection mechanism added
\end{itemize}

%%%%%%%%%%%%%%%%%%%%%%%%%%%%%%%%%%%%%%%%
\paragraph{v0.5:} 2017/04/26

\begin{itemize}
\item
functionality in definition file
\end{itemize}


%%%%%%%%%%%%%%%%%%%%%%%%%%%%%%%%%%%%%%%%%%%%%%%%%%%%%%%%%%%%%%%%%%%%%%%%%%%%%%%%
%%%%%%%%%%%%%%%%%%%%%%%%%%%%%%%%%%%%%%%%%%%%%%%%%%%%%%%%%%%%%%%%%%%%%%%%%%%%%%%%
%%%%%%%%%%%%%%%%%%%%%%%%%%%%%%%%%%%%%%%%%%%%%%%%%%%%%%%%%%%%%%%%%%%%%%%%%%%%%%%%
\appendix

\settowidth\MacroIndent{\rmfamily\scriptsize 000\ }

 \DocInput{childdoc.dtx}

\end{document}
%</driver>
% \fi
%
% %%%%%%%%%%%%%%%%%%%%%%%%%%%%%%%%%%%%%%%%%%%%%%%%%%%%%%%%%%%%%%%%%%%%%%%%%%%%%%
% %%%%%%%%%%%%%%%%%%%%%%%%%%%%%%%%%%%%%%%%%%%%%%%%%%%%%%%%%%%%%%%%%%%%%%%%%%%%%%
% \section{Sample}
%\iffalse
%<*samplemain>
%\fi
%
% The following presents a sample document
% with two chapters, two parts, a title page,
% a compile flag as well as three forwarding files to set the flag.
% It consists of eight |.tex| files:
% \begin{center}
% \begin{tabular}{ll}
% |cdocsamp.tex|&main file\\
% |cdocsch1.tex|&include file for chapter 1\\
% |cdocsch2.tex|&include file for chapter 2\\
% |cdocspt3.tex|&include file for part 3\\
% |cdocspt4.tex|&include file for part 4\\
% |cdocsdrf.tex|&forwarding file for main file in draft mode\\
% |cdocsfi1.tex|&forwarding file for final version of chapter 1\\
% |cdocsfi2.tex|&forwarding file for final version of chapter 2\\
% \end{tabular}
% \end{center}
% Each of the eight files can be compiled directly by the \LaTeX{} compiler.
%
% %%%%%%%%%%%%%%%%%%%%%%%%%%%%%%%%%%%%%%
% \paragraph{Main File.}
%
% The main file is called |cdocsamp.tex|.
%
% Load the \textsf{childdoc} definitions and
% declare the filename for the main document:
%    \begin{macrocode}
\input{childdoc.def}
\childdocmain{}
%    \end{macrocode}

% Optional override for |\version| flag:
%    \begin{macrocode}
%%\ifchilddoc\else\providecommand{\version}{draft}\fi
%    \end{macrocode}

% Define the default values for the |\version| flag
% (|final| for the main file and |draft| for childs):
%    \begin{macrocode}
\ifchilddoc
\providecommand{\version}{draft}
\else
\providecommand{\version}{final}
\fi
%    \end{macrocode}

% Load the standard document class:
%    \begin{macrocode}
\documentclass[12pt]{article}
%    \end{macrocode}

% Start the document body:
%    \begin{macrocode}
\begin{document}
%    \end{macrocode}

% Declare a title page.
% Print title, part of document being processed and version flag:
%    \begin{macrocode}
\addtocounter{page}{-1}
\begin{center}
{\LARGE\bfseries{}childdoc example\par}
\vspace{1cm}
\ifchilddoc
\ifchilddocmanual part\else chapter\fi:
`\childdocname' of `\childdocjob'\par
\else
main document: `\childdocjob'\par
\fi
version: \version\par
\end{center}
\newpage
%    \end{macrocode}

% Manually include selected file,
% otherwise process as usual:
%    \begin{macrocode}
\ifchilddocmanual
\section*{part `\childdocname'}
\input{\childdocname}
\else
%    \end{macrocode}

% Include the two chapters:
%    \begin{macrocode}
\include{cdocsch1}
\include{cdocsch2}
%    \end{macrocode}

% Include the two parts unless only chapters should be displayed:
%    \begin{macrocode}
\ifchilddoc\else
\section{part three}
\input{cdocspt3}
\section{part four}
\input{cdocspt4}
\fi
%    \end{macrocode}

% Process as usual until here:
%    \begin{macrocode}
\fi
%    \end{macrocode}

% End of document body:
%    \begin{macrocode}
\end{document}
%    \end{macrocode}
%\iffalse
%</samplemain>
%\fi
%
% %%%%%%%%%%%%%%%%%%%%%%%%%%%%%%%%%%%%%%
% \paragraph{Chapter Include Files.}
%
% The include files are called |cdocsch1.tex| and |cdocsch2.tex|.
%
%\iffalse
%<*samplechap1|samplechap2>
%\fi

% Optional override for |\version| flag:
%    \begin{macrocode}
%%\providecommand{\version}{final}
%    \end{macrocode}

% Include the main document:
%    \begin{macrocode}
\input{childdoc.def}
\childdocof{cdocsamp}
%    \end{macrocode}

%\iffalse
%</samplechap1|samplechap2>
%\fi
%
%\iffalse
%<*samplechap1>
%\fi
% Some text for chapter 1:
%    \begin{macrocode}
\section{one}
some text in chapter one
%    \end{macrocode}

%\iffalse
%</samplechap1>
%\fi
% Some text for chapter 2:
%\iffalse
%<*samplechap2>
%\fi
%    \begin{macrocode}
\section{two}
more text in chapter two
%    \end{macrocode}

%\iffalse
%</samplechap2>
%\fi
%
% %%%%%%%%%%%%%%%%%%%%%%%%%%%%%%%%%%%%%%
% \paragraph{Part Include Files.}
%
% The include files are called |cdocspt3.tex| and |cdocspt4.tex|.
%
%\iffalse
%<*samplepart3|samplepart4>
%\fi

% Optional override for |\version| flag:
%    \begin{macrocode}
%%\providecommand{\version}{final}
%    \end{macrocode}

% Include the main document:
%    \begin{macrocode}
\input{childdoc.def}
\childdocby{cdocsamp}
%    \end{macrocode}

%\iffalse
%</samplepart3|samplepart4>
%\fi
%
%\iffalse
%<*samplepart3>
%\fi
% Some text for part 3:
%    \begin{macrocode}
some text in part three
%    \end{macrocode}

%\iffalse
%</samplepart3>
%\fi
% Some text for part 4:
%\iffalse
%<*samplepart4>
%\fi
%    \begin{macrocode}
more text in part four
%    \end{macrocode}

%\iffalse
%</samplepart4>
%\fi
%
% %%%%%%%%%%%%%%%%%%%%%%%%%%%%%%%%%%%%%%
% \paragraph{Forwarding for a Complete Draft.}
%
% The following forwarding file |cdocsdrf.tex|
% compiles the main document in draft mode:
%\iffalse
%<*sampledraft>
%\fi
%    \begin{macrocode}
\def\version{draft}
\input{childdoc.def}
\childdocforward{cdocsamp}
%    \end{macrocode}

%\iffalse
%</sampledraft>
%\fi
%
% %%%%%%%%%%%%%%%%%%%%%%%%%%%%%%%%%%%%%%
% \paragraph{Forwarding for Final Version of the Chapters.}
%
% The following forwarding files |cdocsfn1.tex| and |cdocsfn2.tex|
% (with identical content)
% compile the final versions of the child documents
% |cdocsch1.tex| and |cdocsch2.tex|, respectively:
%\iffalse
%<*samplefinal>
%\fi
%    \begin{macrocode}
\def\version{final}
\input{childdoc.def}
\childdocforwardprefix[cdocsamp]{cdocsfn}{cdocsch}
%    \end{macrocode}

%\iffalse
%</samplefinal>
%\fi
%
% %%%%%%%%%%%%%%%%%%%%%%%%%%%%%%%%%%%%%%
% \paragraph{Command Line Processing.}
%
% The following three command lines generate the output files
% |cdocscld|, |cdocscl1| and |cdocscl2|
% which should be identical to
% |cdocsdrf|, |cdocsch1| and |cdocsfn2|, respectively:
% \begin{center}
% \begin{tabular}{l}
% |latex -jobname cdocscld \|\\
% |  "\def\version{draft}\input{childdoc.def}\childdocforward{cdocsamp}"|\\
% |latex -jobname cdocscl1 \|\\
% |  "\input{childdoc.def}\childdocforward[cdocsamp]{cdocsch1}"|\\
% |latex -jobname cdocscl2 \|\\
% |  "\def\version{final}\input{childdoc.def}\childdocforward{cdocsch2}"|
% \end{tabular}
% \end{center}
% Note that the trailing backslash on each first line
% merely continues the input to the second line
% (for convenient cut ant paste).
% Furthermore, the command |latex| can be replaced by any
% of its alternative versions such as |pdflatex|.
%
% %%%%%%%%%%%%%%%%%%%%%%%%%%%%%%%%%%%%%%%%%%%%%%%%%%%%%%%%%%%%%%%%%%%%%%%%%%%%%%
% %%%%%%%%%%%%%%%%%%%%%%%%%%%%%%%%%%%%%%%%%%%%%%%%%%%%%%%%%%%%%%%%%%%%%%%%%%%%%%
% \section{Implementation}
%\iffalse
%<*package>
%\fi
%
% This section describes the definitions file |childdoc.def|.

% The definitions cannot be loaded using |\usepackage| or |\RequirePackage|
% which has a mechanism to prevent loading a style file more than once.
% When loading the definitions by means of |\input|
% multiple instances have to be prevented manually:
%\iffalse
%This code needs to be before the `\ProvidesFile' directive
%which is defined at the beginning of this file.
%Therefore it is also placed there and commented out here.
%</package>
%<*discard>
%\fi
%    \begin{macrocode}
\ifdefined\childdocmain\endinput\fi
%    \end{macrocode}
%\iffalse
%</discard>
%<*package>
%\fi
%
% \macro{\ifchilddoc}
% \macro{\ifchilddocmanual}
% The conditional |\ifchilddoc| tells whether a
% child (true) or main (false) document is being compiled.
% The conditional |\ifchilddocmanual| tells whether
% the |\includeonly| mechanism is used (false) or
% the selection of child files must be performed manually (true).
% The definitions initialise to false:
%    \begin{macrocode}
\newif\ifchilddoc
\newif\ifchilddocmanual
%    \end{macrocode}

% \macro{\childdocname}
% \macro{\childdocjob}
% The macro |\childdocname| stores the name of the main document
% to be compiled. The macro |\childdocjob| stores the name of
% the document on which the \LaTeX{} compiler was originally invoked.
% The content of |\jobname| cannot be compared
% to filenames specified in the source due to different catcodes.
% The following code rescans |\jobname|, stores the result
% in |\childdocname| and saves a copy in |\childdocjob|:
%    \begin{macrocode}
\edef\childdocname{\scantokens\expandafter{\jobname\noexpand}}
\let\childdocjob\childdocname
%    \end{macrocode}

% \macro{\childdocdisable}
% The macro |\childdocdisable| prevents the main file
% from being processed more than once.
% At this stage, the main document command |\childdocmain|
% is assumed to be called once again where it should do nothing.
% Any subsequent call to it should prevent
% a secondary processing of the main document
% It overwrites the forwarding commands
% |\childdocof| and |\childdocforward|
% with empty macros to prevent further inclusions of the main document:
%    \begin{macrocode}
\newcommand{\childdocdisable}
{
  \renewcommand{\childdocmain}[1]{\renewcommand{\childdocmain}[1]{\endinput}}
  \renewcommand{\childdocof}[1]{}
  \renewcommand{\childdocby}[2][]{}
  \renewcommand{\childdocforward}[2][]{}
  \renewcommand{\childdocdisable}{}
}
%    \end{macrocode}

% \macro{\childdocmain}
% The macro |\childdocmain| is to be called at the top of the main file
% with nothing or the main filename (without extension) as argument.
% First, it breaks loops.
% If the argument is not empty and does not match |\childdocname|
% (which is set by the first inclusion of |childdoc.def|),
% |\ifchilddoc| is set to true, |\includeonly| is applied to the child file
% and |\jobname| is set to the main file
% (for proper handling of |.aux| files):
%    \begin{macrocode}
\newcommand{\childdocmain}[1]
{
  \childdocdisable\childdocmain{}
  \if?#1?\else
    \begingroup
      \def\childdoctmp{#1}
      \ifx\childdoctmp\childdocname
        \def\childdoctmp{}
      \else
        \def\childdoctmp
        {
          \childdoctrue
          \includeonly{\childdocname}
          \def\childdocjob{#1}
          \def\jobname{#1}
        }
      \fi
      \expandafter
    \endgroup
    \childdoctmp
  \fi
}
%    \end{macrocode}

% \macro{\childdocof}
% The command |\childdocof| redirects
% compilation to the main file |#1|.
%    \begin{macrocode}
\newcommand{\childdocof}[1]
{
  \childdocdisable
  \childdoctrue
  \includeonly{\childdocname}
  \def\jobname{#1}
  \def\childdocjob{#1}
  \input{#1}
}
%    \end{macrocode}

% \macro{\childdocby}
% The command |\childdocby| ....
%    \begin{macrocode}
\newcommand{\childdocby}[2][]
{
  \childdocdisable
  \childdoctrue
  \childdocmanualtrue
  \if?#1?\else
    \def\jobname{#2}
  \fi
  \def\childdocjob{#2}
  \input{#2}
  \endinput
}
%    \end{macrocode}

% \macro{\childdocforward}
% The command |\childdocforward| redirects
% compilation to the main file or
% (if the optional argument is given) a child file.
% Parameters are set as if the main file
% or a child file starting with |\childdocof| was compiled.
% Then compilation is handed over to the main file:
%    \begin{macrocode}
\newcommand{\childdocforward}[2][]
{
  \begingroup
    \if?#1?
      \def\childdoctmp
      {
        \def\childdocname{#2}
        \def\childdocjob{#2}
        \def\jobname{#2}
        \input{#2}
        \endinput
      }
    \else
      \def\childdoctmp
      {
        \childdocdisable
        \def\childdocname{#2}
        \childdoctrue
        \includeonly{#2}
        \def\childdocjob{#1}
        \def\jobname{#1}
        \input{#1}
        \endinput
      }
    \fi
    \expandafter
  \endgroup
  \childdoctmp
}
%    \end{macrocode}

% \macro{\childdocforwardprefix}
% The command |\childdocforwardprefix| redirects
% compilation to the main or a child file by means of a pattern.
% The prefix |#1| in the current filename is replaced by |#2|
% and the suffix of the current filename is kept
% (it is assumed that the filename does not contain the substring `|~~~|'
% which is used as a delimiter).
% Compilation is handed over to the new file by |\childdocforward|:
%    \begin{macrocode}
\newcommand{\childdocforwardprefix}[3][]
{
  \begingroup
    \def\childdocextract #2##1~~~{\def\childdoctmp{\childdocforward[#1]{#3##1}}}
    \expandafter\childdocextract\childdocname~~~
    \expandafter
  \endgroup
  \childdoctmp
}
%    \end{macrocode}

% \macro{\childdoc}
% The deprecated macro |\childdoc| is a legacy version of |\childdocmain|:
%    \begin{macrocode}
\newcommand{\childdoc}{\childdocmain}
%    \end{macrocode}

% \macro{\childdocredirect}
% The deprecated macro |\childdocredirect| is a legacy version
% of |\childdocforward| and |\childdocforwardprefix|:
%    \begin{macrocode}
\newcommand{\childdocredirect}[2][]
{
  \begingroup
    \if?#1?
      \def\childdoctmp{\childdocforward{#2}}
    \else
      \def\childdoctmp{\childdocforwardprefix{#1}{#2}}
    \fi
    \expandafter
  \endgroup
  \childdoctmp
}
%    \end{macrocode}

%\iffalse
%</package>
%\fi
%
\endinput
|\\
|\childdocmain{|\textit{main}|}|\\
\end{tabular}
\end{center}
%
If |\jobname| does not match the argument \textit{main} of |\childdocmain|,
it is assumed that |\jobname| points to the child file to be compiled.
When using |\childdocmain| with the main file specified as argument,
it suffices to start a child file
with just |\input{|\textit{main}|}|
without loading of the package and using |\childdocof|.
If instead all processing is done
with the appropriate \textsf{childdoc} directives,
the argument of \textit{main} of |\childdocmain| can be empty.

An alternative version of the command line processing described
in \secref{sec:commandline} using the detection mechanism reads:
%
\begin{center}
|... -jobname "|\textit{target}|" "|[\textit{flags}]%
[|\def\jobname{|\textit{dest}|}|]|\input{|\textit{main}|}"|
\end{center}

%%%%%%%%%%%%%%%%%%%%%%%%%%%%%%%%%%%%%%%%%%%%%%%%%%%%%%%%%%%%%%%%%%%%%%%%%%%%%%%%
\subsection{Manual Code}
\label{sec:manual}

In case one cannot be certain whether the definitions file |childdoc.def|
is installed on the target \TeX{} distribution
and one prefers not to ship it,
it is conceivable to paste a few relevant commands into the sources.

To that end, drop all statements |% \iffalse
%
% childdoc.dtx Copyright (C) 2017-2018 Niklas Beisert
%
% This work may be distributed and/or modified under the
% conditions of the LaTeX Project Public License, either version 1.3
% of this license or (at your option) any later version.
% The latest version of this license is in
%   http://www.latex-project.org/lppl.txt
% and version 1.3 or later is part of all distributions of LaTeX
% version 2005/12/01 or later.
%
% This work has the LPPL maintenance status `maintained'.
%
% The Current Maintainer of this work is Niklas Beisert.
%
% This work consists of the files childdoc.dtx and childdoc.ins
% and the derived files childdoc.def and cdocsamp.tex with
% cdocsch1.tex, cdocsch2.tex, cdocsdrf.tex, cdocsfn1.tex, cdocsfn2.tex.
%
%<package>\ifdefined\childdocmain\endinput\fi
%<package>\ProvidesFile{childdoc.def}[2018/12/30 v2.0 child document driver]
%<samplemain>\ProvidesFile{cdocsamp.tex}[2018/12/30 v2.0 sample for childdoc]
%<*driver>
%\ProvidesFile{childdoc.drv}[2018/12/30 v2.0 childdoc reference manual file]
\PassOptionsToClass{10pt,a4paper}{article}
\documentclass{ltxdoc}

\usepackage[margin=35mm]{geometry}
\usepackage{hyperref}
\usepackage{hyperxmp}
\usepackage[usenames]{color}

\hypersetup{colorlinks=true}
\hypersetup{pdfstartview=FitH}
\hypersetup{pdfpagemode=UseNone}
\hypersetup{pdfsource={}}
\hypersetup{pdflang={en-UK}}
\hypersetup{pdfcopyright={Copyright 2017-2018 Niklas Beisert.
  This work may be distributed and/or modified under the
  conditions of the LaTeX Project Public License, either version 1.3
  of this license or (at your option) any later version.}}
\hypersetup{pdflicenseurl={http://www.latex-project.org/lppl.txt}}
\hypersetup{pdfcontactaddress={ETH Zurich, ITP, HIT K,
  Wolfgang-Pauli-Strasse 27}}
\hypersetup{pdfcontactpostcode={8093}}
\hypersetup{pdfcontactcity={Zurich}}
\hypersetup{pdfcontactcountry={Switzerland}}
\hypersetup{pdfcontactemail={nbeisert@itp.phys.ethz.ch}}
\hypersetup{pdfcontacturl={http://people.phys.ethz.ch/\xmptilde nbeisert/}}

\newcommand{\secref}[1]{\hyperref[#1]{section \ref*{#1}}}

\parskip1ex
\parindent0pt
\let\olditemize\itemize
\def\itemize{\olditemize\parskip0pt}

\begin{document}

\title{The \textsf{childdoc} Package}
\hypersetup{pdftitle={The childdoc Package}}
\author{Niklas Beisert\\[2ex]
  Institut f\"ur Theoretische Physik\\
  Eidgen\"ossische Technische Hochschule Z\"urich\\
  Wolfgang-Pauli-Strasse 27, 8093 Z\"urich, Switzerland\\[1ex]
  \href{mailto:nbeisert@itp.phys.ethz.ch}
  {\texttt{nbeisert@itp.phys.ethz.ch}}}
\hypersetup{pdfauthor={Niklas Beisert}}
\hypersetup{pdfsubject={Manual for the LaTeX2e Package childdoc}}
\date{30 December 2018, \textsf{v2.0}}
\maketitle

\begin{abstract}\noindent
\textsf{childdoc} is a \LaTeXe{} package
that enables the direct compilation
of document sections included by |\include|
to individual files.
\end{abstract}

\begingroup
\parskip0ex
\tableofcontents
\endgroup

%%%%%%%%%%%%%%%%%%%%%%%%%%%%%%%%%%%%%%%%%%%%%%%%%%%%%%%%%%%%%%%%%%%%%%%%%%%%%%%%
%%%%%%%%%%%%%%%%%%%%%%%%%%%%%%%%%%%%%%%%%%%%%%%%%%%%%%%%%%%%%%%%%%%%%%%%%%%%%%%%
\section{Introduction}

\LaTeX{} provides a mechanism to structure a large document (such as a book)
into a main file and several child files (containing the chapters)
using the |\include| command.
This mechanism is beneficial for documents
which span hundreds of pages in order to
make the source file(s) more manageable.
Moreover, compilation can be restricted to
selected child files by means of the |\includeonly| command.
The latter feature can be used to reduce the compilation time while editing
(this was significantly more useful in the earlier days of \LaTeX{})
or to generate a smaller document which is easier to navigate.
Another application of |\includeonly| is to generate
documents consisting of selected parts of the complete document.

However, there are a few drawbacks of the plain |\include| mechanism:
\begin{itemize}
\item
The child files cannot be compiled on their own,
they can only be compiled via the main file.
A naive editing environment
(such as a text editor with an option
to have the current file processed by \LaTeX)
may require one to switch to the main file before compiling;
attempting to compile the child file produces errors.
\item
The main file must be modified (each time)
to adjust the |\includeonly| command
to the present needs. This easily leaves the main file in a messy state.
\item
The generated document will always carry the filename
of the main document. This is inconvenient if
several child files are to be compiled and
to be kept for distribution.
\end{itemize}

The present package provides a simple interface
to make child files individually compilable by \LaTeX{}.
Compiling a child file then has the same effect as compiling
the main file with an |\includeonly| command
to select the appropriate child.
Moreover the generated document will carry the name of the child
rather than the main file.
This resolves all three above issues.

This feature is meant to make the editing of books,
thesis documents and lecture notes somewhat more convenient.
However, the package can also be used efficiently for
composing a series of documents (such as exercise sheets)
which are typically distributed individually.
It then assists the author in generating the individual documents
(potentially in different versions)
as well as a document containing the collected series.
Another application is in developing style files
or other kinds of included material
where compilation of the style file could redirect
to a sample or test file.

%%%%%%%%%%%%%%%%%%%%%%%%%%%%%%%%%%%%%%%%%%%%%%%%%%%%%%%%%%%%%%%%%%%%%%%%%%%%%%%%
%%%%%%%%%%%%%%%%%%%%%%%%%%%%%%%%%%%%%%%%%%%%%%%%%%%%%%%%%%%%%%%%%%%%%%%%%%%%%%%%
\section{Usage}

First of all, the package \textsf{childdoc} is \emph{not} a standard
\LaTeXe{} |.sty| style file! Therefore it needs to be invoked in
a non-standard way.

%%%%%%%%%%%%%%%%%%%%%%%%%%%%%%%%%%%%%%%%%%%%%%%%%%%%%%%%%%%%%%%%%%%%%%%%%%%%%%%%
\subsection{Included Files}
\label{sec:include}

%%%%%%%%%%%%%%%%%%%%%%%%%%%%%%%%%%%%%%%%
\DescribeMacro{\childdocmain}
To use the package, add the commands
\begin{center}
\begin{tabular}{l}
|\input{childdoc.def}|\\
|\childdocmain{}|\\
\end{tabular}
\end{center}
at the very top of the main \LaTeX{} file,
in particular \emph{before} the |\documentclass| statement!
The argument of |\childdocmain| should be left empty
(but it must be present).

%%%%%%%%%%%%%%%%%%%%%%%%%%%%%%%%%%%%%%%%
\DescribeMacro{\childdocof}
Furthermore, add the commands
\begin{center}
\begin{tabular}{l}
|\input{childdoc.def}|\\
|\childdocof{|\textit{main}|}|\\
\end{tabular}
\end{center}
at the top of every child file \textit{child}
which is included by |\include{|\textit{child}|}|
from within the main file
(or at least for those files to be compiled individually).
The argument \textit{main} must be the filename of the main file.

There are a couple of
considerations in setting up the main and child documents:

%%%%%%%%%%%%%%%%%%%%%%%%%%%%%%%%%%%%%%%%
\paragraph{Restrictions.}

Please note the following restrictions:
\begin{itemize}
\item
|\childdocmain| must be called with one argument \textit{main}
to ensure compatibility with earlier version of the package.
It must either be empty (|\childdocmain{}|)
or precisely match the filename of the main file in which it is specified.
See \secref{sec:detection} for further information.
\item
The filename \textit{main} must be specified without the |.tex| extension.
\item
The filename \textit{main} is case sensitive
(even in case-insensitive file systems)
due to internal string comparison.
\item
The argument \textit{main} should be fully expanded, it cannot be a macro.
\item
Subdirectories and special characters should be avoided in filenames.
\item
The command |\childdocmain{|\textit{main}|}| must be followed by a whitespace.
It should not be followed immediately by another command
or by a comment mark `|%|'.
This is because the \TeX{} parser reads the token immediately following
the argument of |\childdocmain| and puts it
at the beginning of every child section;
however, a white\-space is ignored.
\end{itemize}

%%%%%%%%%%%%%%%%%%%%%%%%%%%%%%%%%%%%%%%%
\paragraph{Content of Main File.}

It is advisable to place all content in the child files included by |\include|.
Any output contained in the main file will appear in all child documents
unless suppressed manually;
it cannot be suppressed automatically by the |\includeonly| directive
and thus should normally be avoided.
A method to include some content in the main file
by means of conditional processing is described in \secref{sec:conditional}.

%%%%%%%%%%%%%%%%%%%%%%%%%%%%%%%%%%%%%%%%
\paragraph{Page Numbering.}

When only a part of the document is compiled,
the appropriate numbering of pages
(as well as other status parameters)
is determined from the |.aux| files.
The latter contain information from previous passes.
However this information needs to propagate through
all intermediate child documents.
Therefore the page numbering in child documents may well
be inconsistent until the complete document is compiled at least once.

A useful (if unconventional) way to always ensure a consistent
page numbering is to restart the numbering in each child document
and denote the pages by `\textit{child}|.|\textit{page}'
where \textit{child} represents the chapter/section number of the child file.
This can be achieved by the command
|\numberwithin{page}{|\textit{child}|}|
of the \textsf{amsmath} package
where \textit{child} can be |chapter| or |section|
depending on the chosen structuring.
Alternatively, one can modify the macro |\thepage| appropriately
and reset the counter |page| at the start of each child file.

%%%%%%%%%%%%%%%%%%%%%%%%%%%%%%%%%%%%%%%%%%%%%%%%%%%%%%%%%%%%%%%%%%%%%%%%%%%%%%%%
\subsection{Conditional Processing}
\label{sec:conditional}

The package provides a mechanism to compile different versions
of a document. To customise the versions further some conditional processing
can come in handy to distinguish which version is being compiled.
The package provides two macros to describe the compilation context:

%%%%%%%%%%%%%%%%%%%%%%%%%%%%%%%%%%%%%%%%
\DescribeMacro{\ifchilddoc}
The conditional |\ifchilddoc| distinguishes between the compilation of
child documents and the main document:
%
\begin{center}
|\ifchilddoc |\textit{child-code}| |[|\||else |\textit{main-code}]| \||fi|
\end{center}

%%%%%%%%%%%%%%%%%%%%%%%%%%%%%%%%%%%%%%%%
\DescribeMacro{\childdocname}
\DescribeMacro{\childdocjob}
The macro |\childdocname| contains the filename (without extension)
of the main or child file being processed.
Note that |\childdocjob| will always contain the name of the main file.

%%%%%%%%%%%%%%%%%%%%%%%%%%%%%%%%%%%%%%%%
\paragraph{Title Page.}

Conditional processing can be used to include a title or banner page
in the main document when proper precautions are taken.
Importantly, the code in the main file should ensure that the page counter
(as well as other status parameters which are stored in the |.aux| files)
takes the same value after the conditional processing.
Otherwise the page numbers may take divergent values
depending on which part is compiled.

For example, a title page could be declared by:
%
\begin{center}
\begin{tabular}{l}
|\ifchilddoc\||else|\\
|\addtocounter{page}{-1}|\\
\textit{code for title page}\\
|\newpage|\\
|\||fi|
\end{tabular}
\end{center}
%
A banner page for the child documents can be generated by:
%
\begin{center}
\begin{tabular}{l}
|\ifchilddoc|\\
|\addtocounter{page}{-1}|\\
\textit{code for banner page}\\
|\newpage|\\
|\||fi|
\end{tabular}
\end{center}
%
Here one could write a message such as:
\begin{center}
|This is the part \childdocname{} of \childdocjob{}.|
\end{center}

%%%%%%%%%%%%%%%%%%%%%%%%%%%%%%%%%%%%%%%%%%%%%%%%%%%%%%%%%%%%%%%%%%%%%%%%%%%%%%%%
\subsection{Flags}
\label{sec:flags}

The package makes it easy to generate different versions
of the main or child documents.
To this end compilation flags can be defined
and assigned different default values.
They will be particularly useful in conjunction
with the forwarding mechanism described in \secref{sec:forward}.

For example, it may be useful to have a flag |\version|
which can be set to |draft| or |final|.
The document source will contain some conditional code
depending on the value of |\version|.
Suppose further, the flag should default to |final| for the main file
and to |draft| for child files
which is a natural assignment for editing the document.
This is achieved by placing the following code
in the preamble of the main document
(below the |\childdocmain| directive):
%
\begin{center}
\begin{tabular}{l}
|\ifchilddoc|\\
|\providecommand{\version}{draft}|\\
|\||else|\\
|\providecommand{\version}{final}|\\
|\||fi|
\end{tabular}
\end{center}
%
The definition by |\providecommand| makes sure
that previous definitions are not overwritten.
Further statements |\providecommand{\version}{...}|
can thus be added before the above code to override it.

For the main file, one might add a line
(between |\childdocmain| and the above block)
%
\begin{center}
|%\ifchilddoc\||else\providecommand{\version}{draft}\||fi|
\end{center}
%
which can be uncommented to produce a draft version.
Likewise one can add a line to the very top of a child file
(above the |\childdocof{|\textit{main}|}| directive)
%
\begin{center}
|%\providecommand{\version}{final}|
\end{center}
%
which can be uncommented to produce the final version of this child document.

%%%%%%%%%%%%%%%%%%%%%%%%%%%%%%%%%%%%%%%%%%%%%%%%%%%%%%%%%%%%%%%%%%%%%%%%%%%%%%%%
\subsection{Forwarding}
\label{sec:forward}

Different versions of the main or child documents
using compilation flags as described in \secref{sec:flags}
can be (permanently) stored in different files
for convenient compilation, viewing and distribution.
To this end, the package defines a command
to pass on compilation to a different file:

%%%%%%%%%%%%%%%%%%%%%%%%%%%%%%%%%%%%%%%%
\DescribeMacro{\childdocforward}
The command |\childdocforward| redirects processing to
another source file:
%
\begin{center}
\begin{tabular}{l}
|\input{childdoc.def}|\\
|\childdocforward[|\textit{main}|]{|\textit{dest}|}|\\
\end{tabular}
\end{center}
%
The argument \textit{dest} is the destination file
(without extension).
It should be the main file or one of the child files.
Note that further \textsf{childdoc} directives
such as |\childdocof| and |\childdocforward|
in the indicated file will be processed in this form.
The optional argument \textit{main}
passes on directly to the main file \textit{main}
while pretending to compile the child \textit{dest}.
This form behaves as if \textit{dest}
issues |\childdocof{|\textit{main}|}| right away,
and no further \textsf{childdoc} directives will be processed.

%%%%%%%%%%%%%%%%%%%%%%%%%%%%%%%%%%%%%%%%
\DescribeMacro{\...prefix}
In the alternative form |\childdocforwardprefix|,
%
\begin{center}
\begin{tabular}{l}
|\input{childdoc.def}|\\
|\childdocforwardprefix[|\textit{main}|]{|\textit{prefix}|}{|\textit{dest}|}|
\end{tabular}
\end{center}
%
the destination file is determined by a pattern
depending on the current file:
To make this work, the current file must be called
`{\textit{prefix}\hspace{0.2em}\textit{suffix}}'
with \textit{prefix} matching precisely the argument.
Processing is then passed on to the file
`{\textit{dest}\hspace{0.2em}\textit{suffix}}'.
Surely, the same effect is achieved by
directly specifying the
argument `{\textit{dest}\hspace{0.2em}\textit{suffix}}'
in the first form.
However, that requires to set up a different file
for each child. With the alternative form of the command
all these files can have exactly the same content
which simplifies setting them up and maintaining them.

For example, the following file |draft.tex|
with a compilation flag |\version| as described in \secref{sec:flags}
compiles the main document as a draft:
%
\begin{center}
\begin{tabular}{l}
|\def\version{draft}|\\
|\input{childdoc.def}|\\
|\childdocforward{|\textit{main}|}|
\end{tabular}
\end{center}
%
Likewise, the following files |final|\textit{nn}|.tex|
compile the final version of the child document
|child|\textit{nn}|.tex|:
%
\begin{center}
\begin{tabular}{l}
|\def\version{final}|\\
|\input{childdoc.def}|\\
|\childdocforwardprefix{final}{child}|
\end{tabular}
\end{center}
%

Note that when several versions of a main file and/or of each child file
are to be generated, it may be convenient to set up a |Makefile| or
shell script to automatise the process.

%%%%%%%%%%%%%%%%%%%%%%%%%%%%%%%%%%%%%%%%%%%%%%%%%%%%%%%%%%%%%%%%%%%%%%%%%%%%%%%%
\subsection{Command Line Processing}
\label{sec:commandline}

The effect of redirection files can also be achieved by invoking
the \LaTeX{} compiler with a more elaborate command line.
Most conveniently this should be done as part
of a shell script or a |Makefile|.

When using \textsf{childdoc} in the main file, the following
command lines effectively perform a redirection
(note that depending on the shell being used,
backslashes may have to be doubled: `|\|' $\to$ `|\\|'):
%
\begin{center}
|... -jobname "|\textit{target}|" |\\|"|[\textit{flags}]%
|\input{childdoc.def}\childdocforward[|\textit{main}|]{|\textit{dest}|}"|
\end{center}
%
Here \textit{target} is the name of the output file,
\textit{main} is the name of the main file
and \textit{dest} is the name of the main or child file to be processed
(all filenames without extensions).
The optional argument \textit{main} can be omitted
if \textit{main} matches \textit{dest}.
Optionally, compilation \textit{flags} can be defined via |\def| commands.
This command line makes the \TeX{} engine believe
it is compiling the file \textit{target}
whose content is specified as the latter parameter.
The provided code then forwards the processing to
\textit{main} or \textit{dest} as described in \secref{sec:forward}.

%%%%%%%%%%%%%%%%%%%%%%%%%%%%%%%%%%%%%%%%%%%%%%%%%%%%%%%%%%%%%%%%%%%%%%%%%%%%%%%%
\subsection{Include by Input}
\label{sec:input}

Including child documents by |\include| has some restrictions by design.
Most notably, the content of a child document always occupies
its own set of pages; pages cannot be shared between child documents.
Usually, this behaviour makes perfect sense
because each child document contain an essential part of the document.
However, in some situations it may be desirable to compose
a document from a collection of parts
without having mandatory page breaks between then.
For this case, the package
provides a mechanism to include parts
by |\input| which can also be processed individually.
However, by construction this mechanism
requires manual handling of the content to be output.

%%%%%%%%%%%%%%%%%%%%%%%%%%%%%%%%%%%%%%%%
\DescribeMacro{\ifchilddocmanual}
The main file should be prepared as usual, see \secref{sec:include}.
However, the document body must make a distinction
between processing of an individual part and of the main document, e.g.:
%
\begin{center}
\begin{tabular}{l}
|\ifchilddocmanual|\\
|\input{\childdocname}|\\
|\||else|\\
\textit{document body with }|\input{|\textit{part}|}|\\
|\||fi|
\end{tabular}
\end{center}
%
The conditional |\ifchilddocmanual| is true whenever
a part to be included by |\input| is being compiled,
and the name of the part is stored in |\childdocname|.

%%%%%%%%%%%%%%%%%%%%%%%%%%%%%%%%%%%%%%%%
\DescribeMacro{\childdocby}
Each part to be included by |\input| should start with:
%
\begin{center}
\begin{tabular}{l}
|\input{childdoc.def}|\\
|\childdocby{|\textit{main}|}|\\
\end{tabular}
\end{center}
%
The directive |\childdocby| is similar to |\childdocof|
described in \secref{sec:include},
but the subsequent selection of content must be done manually.
To that end, both |\ifchilddoc| and |\ifchilddocmanual|
will be true upon processing of a part,
and the name of the part is stored in |\childdocname|.
Note that |\jobname| will be set to the filename of the current part
so that each part receives an individual |.aux| file
that does not interfere with the |.aux| file(s) of the main document.
This behaviour can be altered by the alternative form
|\childdocby[*]{|\textit{main}|}| (with a non-empty optional argument)
which uses the |.aux| file of the main document
by setting |\jobname| to \textit{main}.

%%%%%%%%%%%%%%%%%%%%%%%%%%%%%%%%%%%%%%%%%%%%%%%%%%%%%%%%%%%%%%%%%%%%%%%%%%%%%%%%
\subsection{Driver Development}
\label{sec:driver}

The \textsf{childdoc} mechanism can also be use for the development
of definition files such as \LaTeX{} styles or classes.
This case differs from the above setup with multiple parts
included by |\include| in that no |\includeonly| should be invoked.
This can be achieved by starting the include file
(before |\ProvidesPackage|) with:
%
\begin{center}
\begin{tabular}{l}
|\input{childdoc.def}|\\
|\childdocforward{|\textit{main}|}|\\
\end{tabular}
\end{center}
%
or alternatively with:
%
\begin{center}
\begin{tabular}{l}
|\input{childdoc.def}|\\
|\childdocby{|\textit{main}|}|\\
\end{tabular}
\end{center}
%
Both forms have slightly different effects as described above.
The main file is prepared as usual, see \secref{sec:include}.

%%%%%%%%%%%%%%%%%%%%%%%%%%%%%%%%%%%%%%%%%%%%%%%%%%%%%%%%%%%%%%%%%%%%%%%%%%%%%%%%
\subsection{Legacy Detection}
\label{sec:detection}

The directive |\childdocmain| in the main file can detect
whether the complete document or merely a child is to be compiled
even without using the directive |\childdocof|.
This method is deprecated because it is less robust
and there is no compelling reason to use it;
it is merely provided for backward compatibility
and it may be removed in future versions.

If the detection mechanism is to be used,
it is mandatory to correctly specify
the filename of the main file as the argument of |\childdocmain|:
%
\begin{center}
\begin{tabular}{l}
|\input{childdoc.def}|\\
|\childdocmain{|\textit{main}|}|\\
\end{tabular}
\end{center}
%
If |\jobname| does not match the argument \textit{main} of |\childdocmain|,
it is assumed that |\jobname| points to the child file to be compiled.
When using |\childdocmain| with the main file specified as argument,
it suffices to start a child file
with just |\input{|\textit{main}|}|
without loading of the package and using |\childdocof|.
If instead all processing is done
with the appropriate \textsf{childdoc} directives,
the argument of \textit{main} of |\childdocmain| can be empty.

An alternative version of the command line processing described
in \secref{sec:commandline} using the detection mechanism reads:
%
\begin{center}
|... -jobname "|\textit{target}|" "|[\textit{flags}]%
[|\def\jobname{|\textit{dest}|}|]|\input{|\textit{main}|}"|
\end{center}

%%%%%%%%%%%%%%%%%%%%%%%%%%%%%%%%%%%%%%%%%%%%%%%%%%%%%%%%%%%%%%%%%%%%%%%%%%%%%%%%
\subsection{Manual Code}
\label{sec:manual}

In case one cannot be certain whether the definitions file |childdoc.def|
is installed on the target \TeX{} distribution
and one prefers not to ship it,
it is conceivable to paste a few relevant commands into the sources.

To that end, drop all statements |\input{childdoc.def}|
and perform the replacements as outlined below.
Instead of |\childdocmain{|\textit{main}|}| add the following code
to the top of the main file:
%
\begin{center}
\begin{tabular}{l}
|\||ifdefined\childdocname\endinput\||fi\newif\ifchilddoc|\\
|\edef\childdocname{\scantokens\expandafter{\jobname\noexpand}}|\\
|\def\childdocmain{|\textit{main}|}\||ifx\childdocmain\childdocname\||else|\\
|\childdoctrue\includeonly{\childdocname}\let\jobname\childdocmain\||fi|\\
\end{tabular}
\end{center}
%
Instead of |\childdocof{|\textit{main}|}| just include the main file
at the top of each child file:
%
\begin{center}
|\input{|\textit{main}|}|
\end{center}
%
A simple redirection |\childdocforward{|\textit{dest}|}| is achieved by:
%
\begin{center}
|\def\jobname{|\textit{dest}|}\input{\jobname}|
\end{center}
%
The redirection with prefix
|\childdocforwardprefix[|\textit{prefix}|]{|\textit{dest}|}|
is accomplished by:
%
\begin{center}
\begin{tabular}{l}
|{\edef\jobname{\scantokens\expandafter{\jobname\noexpand}}|\\
|\def\redirectjob |\textit{prefix}|#1~~~{\gdef\jobname{|\textit{dest}|#1}}|\\
|\expandafter\redirectjob\jobname~~~}\input{\jobname}|
\end{tabular}
\end{center}

In an alternative approach,
child documents can be compiled by a specific command line
without additional code or specific definitions:
%
\begin{center}
|... -jobname "|\textit{target}|" "|[\textit{flags}]%
|\includeonly{|\textit{dest}|}\input{|\textit{main}|}"|
\end{center}
%

%%%%%%%%%%%%%%%%%%%%%%%%%%%%%%%%%%%%%%%%%%%%%%%%%%%%%%%%%%%%%%%%%%%%%%%%%%%%%%%%
%%%%%%%%%%%%%%%%%%%%%%%%%%%%%%%%%%%%%%%%%%%%%%%%%%%%%%%%%%%%%%%%%%%%%%%%%%%%%%%%
\section{Information}

%%%%%%%%%%%%%%%%%%%%%%%%%%%%%%%%%%%%%%%%%%%%%%%%%%%%%%%%%%%%%%%%%%%%%%%%%%%%%%%%
\subsection{Copyright}

Copyright \copyright{} 2017--2018 Niklas Beisert

This work may be distributed and/or modified under the
conditions of the \LaTeX{} Project Public License, either version 1.3
of this license or (at your option) any later version.
The latest version of this license is in
  \url{http://www.latex-project.org/lppl.txt}
and version 1.3 or later is part of all distributions of \LaTeX{}
version 2005/12/01 or later.

This work has the LPPL maintenance status `maintained'.

The Current Maintainer of this work is Niklas Beisert.

This work consists of the files |README.txt|, |childdoc.ins| and |childdoc.dtx|
as well as the derived files |childdoc.def|, |cdocsamp.tex|
with |cdocsch1.tex|, |cdocsch2.tex|, |cdocspt3.tex|, |cdocspt4.tex|,
|cdocsdrf.tex|, |cdocsfn1.tex|, |cdocsfn2.tex|
as well as |childdoc.pdf|.

%%%%%%%%%%%%%%%%%%%%%%%%%%%%%%%%%%%%%%%%%%%%%%%%%%%%%%%%%%%%%%%%%%%%%%%%%%%%%%%%
\subsection{Files and Installation}

The package consists of the files:
%
\begin{center}
\begin{tabular}{ll}
    |README.txt|   & readme file \\
    |childdoc.ins| & installation file \\
    |childdoc.dtx| & source file \\
    |childdoc.def| & definition file \\
    |cdocsamp.tex| & sample main file \\
    |cdocsch1.tex| & sample include file \\
    |cdocsch2.tex| & sample include file \\
    |cdocspt3.tex| & sample part file \\
    |cdocspt4.tex| & sample part file \\
    |cdocsdrf.tex| & sample redirection file \\
    |cdocsfn1.tex| & sample redirection file \\
    |cdocsfn2.tex| & sample redirection file \\
    |childdoc.pdf| & manual
\end{tabular}
\end{center}
%
The distribution consists of the files
|README.txt|, |childdoc.ins| and |childdoc.dtx|.
%
\begin{itemize}
\item
Run (pdf)\LaTeX{} on |childdoc.dtx|
to compile the manual |childdoc.pdf| (this file).
\item
Run \LaTeX{} on |childdoc.ins| to create the definitions file |childdoc.def|
and the sample |cdocsamp.tex| with include files
|cdocsch1.tex|, |cdocsch2.tex|, |cdocspt3.tex|, |cdocspt4.tex|,
|cdocsdrf.tex|, |cdocsfn1.tex|, |cdocsfn2.tex|.
Then copy the file |childdoc.def| to an appropriate directory of your \LaTeX{}
distribution, e.g.\ \textit{texmf-root}|/tex/latex/childdoc|.
\end{itemize}

%%%%%%%%%%%%%%%%%%%%%%%%%%%%%%%%%%%%%%%%%%%%%%%%%%%%%%%%%%%%%%%%%%%%%%%%%%%%%%%%
\subsection{Related CTAN Packages}

There are several other packages which offer a similar functionality:
%
\begin{itemize}
\item
The packages
\href{http://ctan.org/pkg/docmute}{\textsf{docmute}},
\href{http://ctan.org/pkg/includex}{\textsf{includex}} and
\href{http://ctan.org/pkg/standalone}{\textsf{standalone}}
provide commands to include only the document body of
a child file thus allowing both files to be compiled individually.
\item
The packages \href{http://ctan.org/pkg/subdocs}{\textsf{subdocs}}
and \href{http://ctan.org/pkg/subfiles}{\textsf{subfiles}}
provide structures in which the main and child documents can be
encapsulated and allowing them to be compiled individually.
The inclusion mechanism is different from the conventional |\include|.
\item
The package \href{http://ctan.org/pkg/combine}{\textsf{combine}}
is an elaborate solution to combine several documents into one.
\end{itemize}
%
See also the CTAN topic \href{http://ctan.org/topic/subdocs}{\textsf{subdocs}}
for further related packages.
The present package differs from the above solutions in that
a document structure constructed with the conventional |\include| mechanism
just needs two extra commands at the top of every file
such that all constituent files can be compiled individually.

%%%%%%%%%%%%%%%%%%%%%%%%%%%%%%%%%%%%%%%%%%%%%%%%%%%%%%%%%%%%%%%%%%%%%%%%%%%%%%%%
%\subsection{Feature Suggestions}
%
%The following is a list of features which may be useful for future
%versions of this package:
%%
%\begin{itemize}
%\item
%\ldots
%\end{itemize}

%%%%%%%%%%%%%%%%%%%%%%%%%%%%%%%%%%%%%%%%%%%%%%%%%%%%%%%%%%%%%%%%%%%%%%%%%%%%%%%%
\subsection{Revision History}

%%%%%%%%%%%%%%%%%%%%%%%%%%%%%%%%%%%%%%%%
\paragraph{v2.0:} 2018/12/30

\begin{itemize}
\item
immediate forward processing
\item
added |\childdocby| mechanism
\item
manual restructured
\end{itemize}

%%%%%%%%%%%%%%%%%%%%%%%%%%%%%%%%%%%%%%%%
\paragraph{v1.6:} 2018/01/17

\begin{itemize}
\item
application for development of include files
\item
corrections to manual
\end{itemize}

%%%%%%%%%%%%%%%%%%%%%%%%%%%%%%%%%%%%%%%%
\paragraph{v1.5:} 2017/05/21

\begin{itemize}
\item
more complete structuring introduced
\item
|\childdocof| introduced
\item
|\childdoc| renamed to |\childdocmain|
\item
|\childredirect| renamed to |\childdocforward| and |\childdocforwardprefix|
and functionality expanded
\end{itemize}

%%%%%%%%%%%%%%%%%%%%%%%%%%%%%%%%%%%%%%%%
\paragraph{v1.0:} 2017/04/27

\begin{itemize}
\item
manual and install package
\item
first version published on CTAN
\end{itemize}

%%%%%%%%%%%%%%%%%%%%%%%%%%%%%%%%%%%%%%%%
\paragraph{v0.6:} 2017/04/26

\begin{itemize}
\item
redirection mechanism added
\end{itemize}

%%%%%%%%%%%%%%%%%%%%%%%%%%%%%%%%%%%%%%%%
\paragraph{v0.5:} 2017/04/26

\begin{itemize}
\item
functionality in definition file
\end{itemize}


%%%%%%%%%%%%%%%%%%%%%%%%%%%%%%%%%%%%%%%%%%%%%%%%%%%%%%%%%%%%%%%%%%%%%%%%%%%%%%%%
%%%%%%%%%%%%%%%%%%%%%%%%%%%%%%%%%%%%%%%%%%%%%%%%%%%%%%%%%%%%%%%%%%%%%%%%%%%%%%%%
%%%%%%%%%%%%%%%%%%%%%%%%%%%%%%%%%%%%%%%%%%%%%%%%%%%%%%%%%%%%%%%%%%%%%%%%%%%%%%%%
\appendix

\settowidth\MacroIndent{\rmfamily\scriptsize 000\ }

 \DocInput{childdoc.dtx}

\end{document}
%</driver>
% \fi
%
% %%%%%%%%%%%%%%%%%%%%%%%%%%%%%%%%%%%%%%%%%%%%%%%%%%%%%%%%%%%%%%%%%%%%%%%%%%%%%%
% %%%%%%%%%%%%%%%%%%%%%%%%%%%%%%%%%%%%%%%%%%%%%%%%%%%%%%%%%%%%%%%%%%%%%%%%%%%%%%
% \section{Sample}
%\iffalse
%<*samplemain>
%\fi
%
% The following presents a sample document
% with two chapters, two parts, a title page,
% a compile flag as well as three forwarding files to set the flag.
% It consists of eight |.tex| files:
% \begin{center}
% \begin{tabular}{ll}
% |cdocsamp.tex|&main file\\
% |cdocsch1.tex|&include file for chapter 1\\
% |cdocsch2.tex|&include file for chapter 2\\
% |cdocspt3.tex|&include file for part 3\\
% |cdocspt4.tex|&include file for part 4\\
% |cdocsdrf.tex|&forwarding file for main file in draft mode\\
% |cdocsfi1.tex|&forwarding file for final version of chapter 1\\
% |cdocsfi2.tex|&forwarding file for final version of chapter 2\\
% \end{tabular}
% \end{center}
% Each of the eight files can be compiled directly by the \LaTeX{} compiler.
%
% %%%%%%%%%%%%%%%%%%%%%%%%%%%%%%%%%%%%%%
% \paragraph{Main File.}
%
% The main file is called |cdocsamp.tex|.
%
% Load the \textsf{childdoc} definitions and
% declare the filename for the main document:
%    \begin{macrocode}
\input{childdoc.def}
\childdocmain{}
%    \end{macrocode}

% Optional override for |\version| flag:
%    \begin{macrocode}
%%\ifchilddoc\else\providecommand{\version}{draft}\fi
%    \end{macrocode}

% Define the default values for the |\version| flag
% (|final| for the main file and |draft| for childs):
%    \begin{macrocode}
\ifchilddoc
\providecommand{\version}{draft}
\else
\providecommand{\version}{final}
\fi
%    \end{macrocode}

% Load the standard document class:
%    \begin{macrocode}
\documentclass[12pt]{article}
%    \end{macrocode}

% Start the document body:
%    \begin{macrocode}
\begin{document}
%    \end{macrocode}

% Declare a title page.
% Print title, part of document being processed and version flag:
%    \begin{macrocode}
\addtocounter{page}{-1}
\begin{center}
{\LARGE\bfseries{}childdoc example\par}
\vspace{1cm}
\ifchilddoc
\ifchilddocmanual part\else chapter\fi:
`\childdocname' of `\childdocjob'\par
\else
main document: `\childdocjob'\par
\fi
version: \version\par
\end{center}
\newpage
%    \end{macrocode}

% Manually include selected file,
% otherwise process as usual:
%    \begin{macrocode}
\ifchilddocmanual
\section*{part `\childdocname'}
\input{\childdocname}
\else
%    \end{macrocode}

% Include the two chapters:
%    \begin{macrocode}
\include{cdocsch1}
\include{cdocsch2}
%    \end{macrocode}

% Include the two parts unless only chapters should be displayed:
%    \begin{macrocode}
\ifchilddoc\else
\section{part three}
\input{cdocspt3}
\section{part four}
\input{cdocspt4}
\fi
%    \end{macrocode}

% Process as usual until here:
%    \begin{macrocode}
\fi
%    \end{macrocode}

% End of document body:
%    \begin{macrocode}
\end{document}
%    \end{macrocode}
%\iffalse
%</samplemain>
%\fi
%
% %%%%%%%%%%%%%%%%%%%%%%%%%%%%%%%%%%%%%%
% \paragraph{Chapter Include Files.}
%
% The include files are called |cdocsch1.tex| and |cdocsch2.tex|.
%
%\iffalse
%<*samplechap1|samplechap2>
%\fi

% Optional override for |\version| flag:
%    \begin{macrocode}
%%\providecommand{\version}{final}
%    \end{macrocode}

% Include the main document:
%    \begin{macrocode}
\input{childdoc.def}
\childdocof{cdocsamp}
%    \end{macrocode}

%\iffalse
%</samplechap1|samplechap2>
%\fi
%
%\iffalse
%<*samplechap1>
%\fi
% Some text for chapter 1:
%    \begin{macrocode}
\section{one}
some text in chapter one
%    \end{macrocode}

%\iffalse
%</samplechap1>
%\fi
% Some text for chapter 2:
%\iffalse
%<*samplechap2>
%\fi
%    \begin{macrocode}
\section{two}
more text in chapter two
%    \end{macrocode}

%\iffalse
%</samplechap2>
%\fi
%
% %%%%%%%%%%%%%%%%%%%%%%%%%%%%%%%%%%%%%%
% \paragraph{Part Include Files.}
%
% The include files are called |cdocspt3.tex| and |cdocspt4.tex|.
%
%\iffalse
%<*samplepart3|samplepart4>
%\fi

% Optional override for |\version| flag:
%    \begin{macrocode}
%%\providecommand{\version}{final}
%    \end{macrocode}

% Include the main document:
%    \begin{macrocode}
\input{childdoc.def}
\childdocby{cdocsamp}
%    \end{macrocode}

%\iffalse
%</samplepart3|samplepart4>
%\fi
%
%\iffalse
%<*samplepart3>
%\fi
% Some text for part 3:
%    \begin{macrocode}
some text in part three
%    \end{macrocode}

%\iffalse
%</samplepart3>
%\fi
% Some text for part 4:
%\iffalse
%<*samplepart4>
%\fi
%    \begin{macrocode}
more text in part four
%    \end{macrocode}

%\iffalse
%</samplepart4>
%\fi
%
% %%%%%%%%%%%%%%%%%%%%%%%%%%%%%%%%%%%%%%
% \paragraph{Forwarding for a Complete Draft.}
%
% The following forwarding file |cdocsdrf.tex|
% compiles the main document in draft mode:
%\iffalse
%<*sampledraft>
%\fi
%    \begin{macrocode}
\def\version{draft}
\input{childdoc.def}
\childdocforward{cdocsamp}
%    \end{macrocode}

%\iffalse
%</sampledraft>
%\fi
%
% %%%%%%%%%%%%%%%%%%%%%%%%%%%%%%%%%%%%%%
% \paragraph{Forwarding for Final Version of the Chapters.}
%
% The following forwarding files |cdocsfn1.tex| and |cdocsfn2.tex|
% (with identical content)
% compile the final versions of the child documents
% |cdocsch1.tex| and |cdocsch2.tex|, respectively:
%\iffalse
%<*samplefinal>
%\fi
%    \begin{macrocode}
\def\version{final}
\input{childdoc.def}
\childdocforwardprefix[cdocsamp]{cdocsfn}{cdocsch}
%    \end{macrocode}

%\iffalse
%</samplefinal>
%\fi
%
% %%%%%%%%%%%%%%%%%%%%%%%%%%%%%%%%%%%%%%
% \paragraph{Command Line Processing.}
%
% The following three command lines generate the output files
% |cdocscld|, |cdocscl1| and |cdocscl2|
% which should be identical to
% |cdocsdrf|, |cdocsch1| and |cdocsfn2|, respectively:
% \begin{center}
% \begin{tabular}{l}
% |latex -jobname cdocscld \|\\
% |  "\def\version{draft}\input{childdoc.def}\childdocforward{cdocsamp}"|\\
% |latex -jobname cdocscl1 \|\\
% |  "\input{childdoc.def}\childdocforward[cdocsamp]{cdocsch1}"|\\
% |latex -jobname cdocscl2 \|\\
% |  "\def\version{final}\input{childdoc.def}\childdocforward{cdocsch2}"|
% \end{tabular}
% \end{center}
% Note that the trailing backslash on each first line
% merely continues the input to the second line
% (for convenient cut ant paste).
% Furthermore, the command |latex| can be replaced by any
% of its alternative versions such as |pdflatex|.
%
% %%%%%%%%%%%%%%%%%%%%%%%%%%%%%%%%%%%%%%%%%%%%%%%%%%%%%%%%%%%%%%%%%%%%%%%%%%%%%%
% %%%%%%%%%%%%%%%%%%%%%%%%%%%%%%%%%%%%%%%%%%%%%%%%%%%%%%%%%%%%%%%%%%%%%%%%%%%%%%
% \section{Implementation}
%\iffalse
%<*package>
%\fi
%
% This section describes the definitions file |childdoc.def|.

% The definitions cannot be loaded using |\usepackage| or |\RequirePackage|
% which has a mechanism to prevent loading a style file more than once.
% When loading the definitions by means of |\input|
% multiple instances have to be prevented manually:
%\iffalse
%This code needs to be before the `\ProvidesFile' directive
%which is defined at the beginning of this file.
%Therefore it is also placed there and commented out here.
%</package>
%<*discard>
%\fi
%    \begin{macrocode}
\ifdefined\childdocmain\endinput\fi
%    \end{macrocode}
%\iffalse
%</discard>
%<*package>
%\fi
%
% \macro{\ifchilddoc}
% \macro{\ifchilddocmanual}
% The conditional |\ifchilddoc| tells whether a
% child (true) or main (false) document is being compiled.
% The conditional |\ifchilddocmanual| tells whether
% the |\includeonly| mechanism is used (false) or
% the selection of child files must be performed manually (true).
% The definitions initialise to false:
%    \begin{macrocode}
\newif\ifchilddoc
\newif\ifchilddocmanual
%    \end{macrocode}

% \macro{\childdocname}
% \macro{\childdocjob}
% The macro |\childdocname| stores the name of the main document
% to be compiled. The macro |\childdocjob| stores the name of
% the document on which the \LaTeX{} compiler was originally invoked.
% The content of |\jobname| cannot be compared
% to filenames specified in the source due to different catcodes.
% The following code rescans |\jobname|, stores the result
% in |\childdocname| and saves a copy in |\childdocjob|:
%    \begin{macrocode}
\edef\childdocname{\scantokens\expandafter{\jobname\noexpand}}
\let\childdocjob\childdocname
%    \end{macrocode}

% \macro{\childdocdisable}
% The macro |\childdocdisable| prevents the main file
% from being processed more than once.
% At this stage, the main document command |\childdocmain|
% is assumed to be called once again where it should do nothing.
% Any subsequent call to it should prevent
% a secondary processing of the main document
% It overwrites the forwarding commands
% |\childdocof| and |\childdocforward|
% with empty macros to prevent further inclusions of the main document:
%    \begin{macrocode}
\newcommand{\childdocdisable}
{
  \renewcommand{\childdocmain}[1]{\renewcommand{\childdocmain}[1]{\endinput}}
  \renewcommand{\childdocof}[1]{}
  \renewcommand{\childdocby}[2][]{}
  \renewcommand{\childdocforward}[2][]{}
  \renewcommand{\childdocdisable}{}
}
%    \end{macrocode}

% \macro{\childdocmain}
% The macro |\childdocmain| is to be called at the top of the main file
% with nothing or the main filename (without extension) as argument.
% First, it breaks loops.
% If the argument is not empty and does not match |\childdocname|
% (which is set by the first inclusion of |childdoc.def|),
% |\ifchilddoc| is set to true, |\includeonly| is applied to the child file
% and |\jobname| is set to the main file
% (for proper handling of |.aux| files):
%    \begin{macrocode}
\newcommand{\childdocmain}[1]
{
  \childdocdisable\childdocmain{}
  \if?#1?\else
    \begingroup
      \def\childdoctmp{#1}
      \ifx\childdoctmp\childdocname
        \def\childdoctmp{}
      \else
        \def\childdoctmp
        {
          \childdoctrue
          \includeonly{\childdocname}
          \def\childdocjob{#1}
          \def\jobname{#1}
        }
      \fi
      \expandafter
    \endgroup
    \childdoctmp
  \fi
}
%    \end{macrocode}

% \macro{\childdocof}
% The command |\childdocof| redirects
% compilation to the main file |#1|.
%    \begin{macrocode}
\newcommand{\childdocof}[1]
{
  \childdocdisable
  \childdoctrue
  \includeonly{\childdocname}
  \def\jobname{#1}
  \def\childdocjob{#1}
  \input{#1}
}
%    \end{macrocode}

% \macro{\childdocby}
% The command |\childdocby| ....
%    \begin{macrocode}
\newcommand{\childdocby}[2][]
{
  \childdocdisable
  \childdoctrue
  \childdocmanualtrue
  \if?#1?\else
    \def\jobname{#2}
  \fi
  \def\childdocjob{#2}
  \input{#2}
  \endinput
}
%    \end{macrocode}

% \macro{\childdocforward}
% The command |\childdocforward| redirects
% compilation to the main file or
% (if the optional argument is given) a child file.
% Parameters are set as if the main file
% or a child file starting with |\childdocof| was compiled.
% Then compilation is handed over to the main file:
%    \begin{macrocode}
\newcommand{\childdocforward}[2][]
{
  \begingroup
    \if?#1?
      \def\childdoctmp
      {
        \def\childdocname{#2}
        \def\childdocjob{#2}
        \def\jobname{#2}
        \input{#2}
        \endinput
      }
    \else
      \def\childdoctmp
      {
        \childdocdisable
        \def\childdocname{#2}
        \childdoctrue
        \includeonly{#2}
        \def\childdocjob{#1}
        \def\jobname{#1}
        \input{#1}
        \endinput
      }
    \fi
    \expandafter
  \endgroup
  \childdoctmp
}
%    \end{macrocode}

% \macro{\childdocforwardprefix}
% The command |\childdocforwardprefix| redirects
% compilation to the main or a child file by means of a pattern.
% The prefix |#1| in the current filename is replaced by |#2|
% and the suffix of the current filename is kept
% (it is assumed that the filename does not contain the substring `|~~~|'
% which is used as a delimiter).
% Compilation is handed over to the new file by |\childdocforward|:
%    \begin{macrocode}
\newcommand{\childdocforwardprefix}[3][]
{
  \begingroup
    \def\childdocextract #2##1~~~{\def\childdoctmp{\childdocforward[#1]{#3##1}}}
    \expandafter\childdocextract\childdocname~~~
    \expandafter
  \endgroup
  \childdoctmp
}
%    \end{macrocode}

% \macro{\childdoc}
% The deprecated macro |\childdoc| is a legacy version of |\childdocmain|:
%    \begin{macrocode}
\newcommand{\childdoc}{\childdocmain}
%    \end{macrocode}

% \macro{\childdocredirect}
% The deprecated macro |\childdocredirect| is a legacy version
% of |\childdocforward| and |\childdocforwardprefix|:
%    \begin{macrocode}
\newcommand{\childdocredirect}[2][]
{
  \begingroup
    \if?#1?
      \def\childdoctmp{\childdocforward{#2}}
    \else
      \def\childdoctmp{\childdocforwardprefix{#1}{#2}}
    \fi
    \expandafter
  \endgroup
  \childdoctmp
}
%    \end{macrocode}

%\iffalse
%</package>
%\fi
%
\endinput
|
and perform the replacements as outlined below.
Instead of |\childdocmain{|\textit{main}|}| add the following code
to the top of the main file:
%
\begin{center}
\begin{tabular}{l}
|\||ifdefined\childdocname\endinput\||fi\newif\ifchilddoc|\\
|\edef\childdocname{\scantokens\expandafter{\jobname\noexpand}}|\\
|\def\childdocmain{|\textit{main}|}\||ifx\childdocmain\childdocname\||else|\\
|\childdoctrue\includeonly{\childdocname}\let\jobname\childdocmain\||fi|\\
\end{tabular}
\end{center}
%
Instead of |\childdocof{|\textit{main}|}| just include the main file
at the top of each child file:
%
\begin{center}
|\input{|\textit{main}|}|
\end{center}
%
A simple redirection |\childdocforward{|\textit{dest}|}| is achieved by:
%
\begin{center}
|\def\jobname{|\textit{dest}|}\input{\jobname}|
\end{center}
%
The redirection with prefix
|\childdocforwardprefix[|\textit{prefix}|]{|\textit{dest}|}|
is accomplished by:
%
\begin{center}
\begin{tabular}{l}
|{\edef\jobname{\scantokens\expandafter{\jobname\noexpand}}|\\
|\def\redirectjob |\textit{prefix}|#1~~~{\gdef\jobname{|\textit{dest}|#1}}|\\
|\expandafter\redirectjob\jobname~~~}\input{\jobname}|
\end{tabular}
\end{center}

In an alternative approach,
child documents can be compiled by a specific command line
without additional code or specific definitions:
%
\begin{center}
|... -jobname "|\textit{target}|" "|[\textit{flags}]%
|\includeonly{|\textit{dest}|}\input{|\textit{main}|}"|
\end{center}
%

%%%%%%%%%%%%%%%%%%%%%%%%%%%%%%%%%%%%%%%%%%%%%%%%%%%%%%%%%%%%%%%%%%%%%%%%%%%%%%%%
%%%%%%%%%%%%%%%%%%%%%%%%%%%%%%%%%%%%%%%%%%%%%%%%%%%%%%%%%%%%%%%%%%%%%%%%%%%%%%%%
\section{Information}

%%%%%%%%%%%%%%%%%%%%%%%%%%%%%%%%%%%%%%%%%%%%%%%%%%%%%%%%%%%%%%%%%%%%%%%%%%%%%%%%
\subsection{Copyright}

Copyright \copyright{} 2017--2018 Niklas Beisert

This work may be distributed and/or modified under the
conditions of the \LaTeX{} Project Public License, either version 1.3
of this license or (at your option) any later version.
The latest version of this license is in
  \url{http://www.latex-project.org/lppl.txt}
and version 1.3 or later is part of all distributions of \LaTeX{}
version 2005/12/01 or later.

This work has the LPPL maintenance status `maintained'.

The Current Maintainer of this work is Niklas Beisert.

This work consists of the files |README.txt|, |childdoc.ins| and |childdoc.dtx|
as well as the derived files |childdoc.def|, |cdocsamp.tex|
with |cdocsch1.tex|, |cdocsch2.tex|, |cdocspt3.tex|, |cdocspt4.tex|,
|cdocsdrf.tex|, |cdocsfn1.tex|, |cdocsfn2.tex|
as well as |childdoc.pdf|.

%%%%%%%%%%%%%%%%%%%%%%%%%%%%%%%%%%%%%%%%%%%%%%%%%%%%%%%%%%%%%%%%%%%%%%%%%%%%%%%%
\subsection{Files and Installation}

The package consists of the files:
%
\begin{center}
\begin{tabular}{ll}
    |README.txt|   & readme file \\
    |childdoc.ins| & installation file \\
    |childdoc.dtx| & source file \\
    |childdoc.def| & definition file \\
    |cdocsamp.tex| & sample main file \\
    |cdocsch1.tex| & sample include file \\
    |cdocsch2.tex| & sample include file \\
    |cdocspt3.tex| & sample part file \\
    |cdocspt4.tex| & sample part file \\
    |cdocsdrf.tex| & sample redirection file \\
    |cdocsfn1.tex| & sample redirection file \\
    |cdocsfn2.tex| & sample redirection file \\
    |childdoc.pdf| & manual
\end{tabular}
\end{center}
%
The distribution consists of the files
|README.txt|, |childdoc.ins| and |childdoc.dtx|.
%
\begin{itemize}
\item
Run (pdf)\LaTeX{} on |childdoc.dtx|
to compile the manual |childdoc.pdf| (this file).
\item
Run \LaTeX{} on |childdoc.ins| to create the definitions file |childdoc.def|
and the sample |cdocsamp.tex| with include files
|cdocsch1.tex|, |cdocsch2.tex|, |cdocspt3.tex|, |cdocspt4.tex|,
|cdocsdrf.tex|, |cdocsfn1.tex|, |cdocsfn2.tex|.
Then copy the file |childdoc.def| to an appropriate directory of your \LaTeX{}
distribution, e.g.\ \textit{texmf-root}|/tex/latex/childdoc|.
\end{itemize}

%%%%%%%%%%%%%%%%%%%%%%%%%%%%%%%%%%%%%%%%%%%%%%%%%%%%%%%%%%%%%%%%%%%%%%%%%%%%%%%%
\subsection{Related CTAN Packages}

There are several other packages which offer a similar functionality:
%
\begin{itemize}
\item
The packages
\href{http://ctan.org/pkg/docmute}{\textsf{docmute}},
\href{http://ctan.org/pkg/includex}{\textsf{includex}} and
\href{http://ctan.org/pkg/standalone}{\textsf{standalone}}
provide commands to include only the document body of
a child file thus allowing both files to be compiled individually.
\item
The packages \href{http://ctan.org/pkg/subdocs}{\textsf{subdocs}}
and \href{http://ctan.org/pkg/subfiles}{\textsf{subfiles}}
provide structures in which the main and child documents can be
encapsulated and allowing them to be compiled individually.
The inclusion mechanism is different from the conventional |\include|.
\item
The package \href{http://ctan.org/pkg/combine}{\textsf{combine}}
is an elaborate solution to combine several documents into one.
\end{itemize}
%
See also the CTAN topic \href{http://ctan.org/topic/subdocs}{\textsf{subdocs}}
for further related packages.
The present package differs from the above solutions in that
a document structure constructed with the conventional |\include| mechanism
just needs two extra commands at the top of every file
such that all constituent files can be compiled individually.

%%%%%%%%%%%%%%%%%%%%%%%%%%%%%%%%%%%%%%%%%%%%%%%%%%%%%%%%%%%%%%%%%%%%%%%%%%%%%%%%
%\subsection{Feature Suggestions}
%
%The following is a list of features which may be useful for future
%versions of this package:
%%
%\begin{itemize}
%\item
%\ldots
%\end{itemize}

%%%%%%%%%%%%%%%%%%%%%%%%%%%%%%%%%%%%%%%%%%%%%%%%%%%%%%%%%%%%%%%%%%%%%%%%%%%%%%%%
\subsection{Revision History}

%%%%%%%%%%%%%%%%%%%%%%%%%%%%%%%%%%%%%%%%
\paragraph{v2.0:} 2018/12/30

\begin{itemize}
\item
immediate forward processing
\item
added |\childdocby| mechanism
\item
manual restructured
\end{itemize}

%%%%%%%%%%%%%%%%%%%%%%%%%%%%%%%%%%%%%%%%
\paragraph{v1.6:} 2018/01/17

\begin{itemize}
\item
application for development of include files
\item
corrections to manual
\end{itemize}

%%%%%%%%%%%%%%%%%%%%%%%%%%%%%%%%%%%%%%%%
\paragraph{v1.5:} 2017/05/21

\begin{itemize}
\item
more complete structuring introduced
\item
|\childdocof| introduced
\item
|\childdoc| renamed to |\childdocmain|
\item
|\childredirect| renamed to |\childdocforward| and |\childdocforwardprefix|
and functionality expanded
\end{itemize}

%%%%%%%%%%%%%%%%%%%%%%%%%%%%%%%%%%%%%%%%
\paragraph{v1.0:} 2017/04/27

\begin{itemize}
\item
manual and install package
\item
first version published on CTAN
\end{itemize}

%%%%%%%%%%%%%%%%%%%%%%%%%%%%%%%%%%%%%%%%
\paragraph{v0.6:} 2017/04/26

\begin{itemize}
\item
redirection mechanism added
\end{itemize}

%%%%%%%%%%%%%%%%%%%%%%%%%%%%%%%%%%%%%%%%
\paragraph{v0.5:} 2017/04/26

\begin{itemize}
\item
functionality in definition file
\end{itemize}


%%%%%%%%%%%%%%%%%%%%%%%%%%%%%%%%%%%%%%%%%%%%%%%%%%%%%%%%%%%%%%%%%%%%%%%%%%%%%%%%
%%%%%%%%%%%%%%%%%%%%%%%%%%%%%%%%%%%%%%%%%%%%%%%%%%%%%%%%%%%%%%%%%%%%%%%%%%%%%%%%
%%%%%%%%%%%%%%%%%%%%%%%%%%%%%%%%%%%%%%%%%%%%%%%%%%%%%%%%%%%%%%%%%%%%%%%%%%%%%%%%
\appendix

\settowidth\MacroIndent{\rmfamily\scriptsize 000\ }

 \DocInput{childdoc.dtx}

\end{document}
%</driver>
% \fi
%
% %%%%%%%%%%%%%%%%%%%%%%%%%%%%%%%%%%%%%%%%%%%%%%%%%%%%%%%%%%%%%%%%%%%%%%%%%%%%%%
% %%%%%%%%%%%%%%%%%%%%%%%%%%%%%%%%%%%%%%%%%%%%%%%%%%%%%%%%%%%%%%%%%%%%%%%%%%%%%%
% \section{Sample}
%\iffalse
%<*samplemain>
%\fi
%
% The following presents a sample document
% with two chapters, two parts, a title page,
% a compile flag as well as three forwarding files to set the flag.
% It consists of eight |.tex| files:
% \begin{center}
% \begin{tabular}{ll}
% |cdocsamp.tex|&main file\\
% |cdocsch1.tex|&include file for chapter 1\\
% |cdocsch2.tex|&include file for chapter 2\\
% |cdocspt3.tex|&include file for part 3\\
% |cdocspt4.tex|&include file for part 4\\
% |cdocsdrf.tex|&forwarding file for main file in draft mode\\
% |cdocsfi1.tex|&forwarding file for final version of chapter 1\\
% |cdocsfi2.tex|&forwarding file for final version of chapter 2\\
% \end{tabular}
% \end{center}
% Each of the eight files can be compiled directly by the \LaTeX{} compiler.
%
% %%%%%%%%%%%%%%%%%%%%%%%%%%%%%%%%%%%%%%
% \paragraph{Main File.}
%
% The main file is called |cdocsamp.tex|.
%
% Load the \textsf{childdoc} definitions and
% declare the filename for the main document:
%    \begin{macrocode}
% \iffalse
%
% childdoc.dtx Copyright (C) 2017-2018 Niklas Beisert
%
% This work may be distributed and/or modified under the
% conditions of the LaTeX Project Public License, either version 1.3
% of this license or (at your option) any later version.
% The latest version of this license is in
%   http://www.latex-project.org/lppl.txt
% and version 1.3 or later is part of all distributions of LaTeX
% version 2005/12/01 or later.
%
% This work has the LPPL maintenance status `maintained'.
%
% The Current Maintainer of this work is Niklas Beisert.
%
% This work consists of the files childdoc.dtx and childdoc.ins
% and the derived files childdoc.def and cdocsamp.tex with
% cdocsch1.tex, cdocsch2.tex, cdocsdrf.tex, cdocsfn1.tex, cdocsfn2.tex.
%
%<package>\ifdefined\childdocmain\endinput\fi
%<package>\ProvidesFile{childdoc.def}[2018/12/30 v2.0 child document driver]
%<samplemain>\ProvidesFile{cdocsamp.tex}[2018/12/30 v2.0 sample for childdoc]
%<*driver>
%\ProvidesFile{childdoc.drv}[2018/12/30 v2.0 childdoc reference manual file]
\PassOptionsToClass{10pt,a4paper}{article}
\documentclass{ltxdoc}

\usepackage[margin=35mm]{geometry}
\usepackage{hyperref}
\usepackage{hyperxmp}
\usepackage[usenames]{color}

\hypersetup{colorlinks=true}
\hypersetup{pdfstartview=FitH}
\hypersetup{pdfpagemode=UseNone}
\hypersetup{pdfsource={}}
\hypersetup{pdflang={en-UK}}
\hypersetup{pdfcopyright={Copyright 2017-2018 Niklas Beisert.
  This work may be distributed and/or modified under the
  conditions of the LaTeX Project Public License, either version 1.3
  of this license or (at your option) any later version.}}
\hypersetup{pdflicenseurl={http://www.latex-project.org/lppl.txt}}
\hypersetup{pdfcontactaddress={ETH Zurich, ITP, HIT K,
  Wolfgang-Pauli-Strasse 27}}
\hypersetup{pdfcontactpostcode={8093}}
\hypersetup{pdfcontactcity={Zurich}}
\hypersetup{pdfcontactcountry={Switzerland}}
\hypersetup{pdfcontactemail={nbeisert@itp.phys.ethz.ch}}
\hypersetup{pdfcontacturl={http://people.phys.ethz.ch/\xmptilde nbeisert/}}

\newcommand{\secref}[1]{\hyperref[#1]{section \ref*{#1}}}

\parskip1ex
\parindent0pt
\let\olditemize\itemize
\def\itemize{\olditemize\parskip0pt}

\begin{document}

\title{The \textsf{childdoc} Package}
\hypersetup{pdftitle={The childdoc Package}}
\author{Niklas Beisert\\[2ex]
  Institut f\"ur Theoretische Physik\\
  Eidgen\"ossische Technische Hochschule Z\"urich\\
  Wolfgang-Pauli-Strasse 27, 8093 Z\"urich, Switzerland\\[1ex]
  \href{mailto:nbeisert@itp.phys.ethz.ch}
  {\texttt{nbeisert@itp.phys.ethz.ch}}}
\hypersetup{pdfauthor={Niklas Beisert}}
\hypersetup{pdfsubject={Manual for the LaTeX2e Package childdoc}}
\date{30 December 2018, \textsf{v2.0}}
\maketitle

\begin{abstract}\noindent
\textsf{childdoc} is a \LaTeXe{} package
that enables the direct compilation
of document sections included by |\include|
to individual files.
\end{abstract}

\begingroup
\parskip0ex
\tableofcontents
\endgroup

%%%%%%%%%%%%%%%%%%%%%%%%%%%%%%%%%%%%%%%%%%%%%%%%%%%%%%%%%%%%%%%%%%%%%%%%%%%%%%%%
%%%%%%%%%%%%%%%%%%%%%%%%%%%%%%%%%%%%%%%%%%%%%%%%%%%%%%%%%%%%%%%%%%%%%%%%%%%%%%%%
\section{Introduction}

\LaTeX{} provides a mechanism to structure a large document (such as a book)
into a main file and several child files (containing the chapters)
using the |\include| command.
This mechanism is beneficial for documents
which span hundreds of pages in order to
make the source file(s) more manageable.
Moreover, compilation can be restricted to
selected child files by means of the |\includeonly| command.
The latter feature can be used to reduce the compilation time while editing
(this was significantly more useful in the earlier days of \LaTeX{})
or to generate a smaller document which is easier to navigate.
Another application of |\includeonly| is to generate
documents consisting of selected parts of the complete document.

However, there are a few drawbacks of the plain |\include| mechanism:
\begin{itemize}
\item
The child files cannot be compiled on their own,
they can only be compiled via the main file.
A naive editing environment
(such as a text editor with an option
to have the current file processed by \LaTeX)
may require one to switch to the main file before compiling;
attempting to compile the child file produces errors.
\item
The main file must be modified (each time)
to adjust the |\includeonly| command
to the present needs. This easily leaves the main file in a messy state.
\item
The generated document will always carry the filename
of the main document. This is inconvenient if
several child files are to be compiled and
to be kept for distribution.
\end{itemize}

The present package provides a simple interface
to make child files individually compilable by \LaTeX{}.
Compiling a child file then has the same effect as compiling
the main file with an |\includeonly| command
to select the appropriate child.
Moreover the generated document will carry the name of the child
rather than the main file.
This resolves all three above issues.

This feature is meant to make the editing of books,
thesis documents and lecture notes somewhat more convenient.
However, the package can also be used efficiently for
composing a series of documents (such as exercise sheets)
which are typically distributed individually.
It then assists the author in generating the individual documents
(potentially in different versions)
as well as a document containing the collected series.
Another application is in developing style files
or other kinds of included material
where compilation of the style file could redirect
to a sample or test file.

%%%%%%%%%%%%%%%%%%%%%%%%%%%%%%%%%%%%%%%%%%%%%%%%%%%%%%%%%%%%%%%%%%%%%%%%%%%%%%%%
%%%%%%%%%%%%%%%%%%%%%%%%%%%%%%%%%%%%%%%%%%%%%%%%%%%%%%%%%%%%%%%%%%%%%%%%%%%%%%%%
\section{Usage}

First of all, the package \textsf{childdoc} is \emph{not} a standard
\LaTeXe{} |.sty| style file! Therefore it needs to be invoked in
a non-standard way.

%%%%%%%%%%%%%%%%%%%%%%%%%%%%%%%%%%%%%%%%%%%%%%%%%%%%%%%%%%%%%%%%%%%%%%%%%%%%%%%%
\subsection{Included Files}
\label{sec:include}

%%%%%%%%%%%%%%%%%%%%%%%%%%%%%%%%%%%%%%%%
\DescribeMacro{\childdocmain}
To use the package, add the commands
\begin{center}
\begin{tabular}{l}
|\input{childdoc.def}|\\
|\childdocmain{}|\\
\end{tabular}
\end{center}
at the very top of the main \LaTeX{} file,
in particular \emph{before} the |\documentclass| statement!
The argument of |\childdocmain| should be left empty
(but it must be present).

%%%%%%%%%%%%%%%%%%%%%%%%%%%%%%%%%%%%%%%%
\DescribeMacro{\childdocof}
Furthermore, add the commands
\begin{center}
\begin{tabular}{l}
|\input{childdoc.def}|\\
|\childdocof{|\textit{main}|}|\\
\end{tabular}
\end{center}
at the top of every child file \textit{child}
which is included by |\include{|\textit{child}|}|
from within the main file
(or at least for those files to be compiled individually).
The argument \textit{main} must be the filename of the main file.

There are a couple of
considerations in setting up the main and child documents:

%%%%%%%%%%%%%%%%%%%%%%%%%%%%%%%%%%%%%%%%
\paragraph{Restrictions.}

Please note the following restrictions:
\begin{itemize}
\item
|\childdocmain| must be called with one argument \textit{main}
to ensure compatibility with earlier version of the package.
It must either be empty (|\childdocmain{}|)
or precisely match the filename of the main file in which it is specified.
See \secref{sec:detection} for further information.
\item
The filename \textit{main} must be specified without the |.tex| extension.
\item
The filename \textit{main} is case sensitive
(even in case-insensitive file systems)
due to internal string comparison.
\item
The argument \textit{main} should be fully expanded, it cannot be a macro.
\item
Subdirectories and special characters should be avoided in filenames.
\item
The command |\childdocmain{|\textit{main}|}| must be followed by a whitespace.
It should not be followed immediately by another command
or by a comment mark `|%|'.
This is because the \TeX{} parser reads the token immediately following
the argument of |\childdocmain| and puts it
at the beginning of every child section;
however, a white\-space is ignored.
\end{itemize}

%%%%%%%%%%%%%%%%%%%%%%%%%%%%%%%%%%%%%%%%
\paragraph{Content of Main File.}

It is advisable to place all content in the child files included by |\include|.
Any output contained in the main file will appear in all child documents
unless suppressed manually;
it cannot be suppressed automatically by the |\includeonly| directive
and thus should normally be avoided.
A method to include some content in the main file
by means of conditional processing is described in \secref{sec:conditional}.

%%%%%%%%%%%%%%%%%%%%%%%%%%%%%%%%%%%%%%%%
\paragraph{Page Numbering.}

When only a part of the document is compiled,
the appropriate numbering of pages
(as well as other status parameters)
is determined from the |.aux| files.
The latter contain information from previous passes.
However this information needs to propagate through
all intermediate child documents.
Therefore the page numbering in child documents may well
be inconsistent until the complete document is compiled at least once.

A useful (if unconventional) way to always ensure a consistent
page numbering is to restart the numbering in each child document
and denote the pages by `\textit{child}|.|\textit{page}'
where \textit{child} represents the chapter/section number of the child file.
This can be achieved by the command
|\numberwithin{page}{|\textit{child}|}|
of the \textsf{amsmath} package
where \textit{child} can be |chapter| or |section|
depending on the chosen structuring.
Alternatively, one can modify the macro |\thepage| appropriately
and reset the counter |page| at the start of each child file.

%%%%%%%%%%%%%%%%%%%%%%%%%%%%%%%%%%%%%%%%%%%%%%%%%%%%%%%%%%%%%%%%%%%%%%%%%%%%%%%%
\subsection{Conditional Processing}
\label{sec:conditional}

The package provides a mechanism to compile different versions
of a document. To customise the versions further some conditional processing
can come in handy to distinguish which version is being compiled.
The package provides two macros to describe the compilation context:

%%%%%%%%%%%%%%%%%%%%%%%%%%%%%%%%%%%%%%%%
\DescribeMacro{\ifchilddoc}
The conditional |\ifchilddoc| distinguishes between the compilation of
child documents and the main document:
%
\begin{center}
|\ifchilddoc |\textit{child-code}| |[|\||else |\textit{main-code}]| \||fi|
\end{center}

%%%%%%%%%%%%%%%%%%%%%%%%%%%%%%%%%%%%%%%%
\DescribeMacro{\childdocname}
\DescribeMacro{\childdocjob}
The macro |\childdocname| contains the filename (without extension)
of the main or child file being processed.
Note that |\childdocjob| will always contain the name of the main file.

%%%%%%%%%%%%%%%%%%%%%%%%%%%%%%%%%%%%%%%%
\paragraph{Title Page.}

Conditional processing can be used to include a title or banner page
in the main document when proper precautions are taken.
Importantly, the code in the main file should ensure that the page counter
(as well as other status parameters which are stored in the |.aux| files)
takes the same value after the conditional processing.
Otherwise the page numbers may take divergent values
depending on which part is compiled.

For example, a title page could be declared by:
%
\begin{center}
\begin{tabular}{l}
|\ifchilddoc\||else|\\
|\addtocounter{page}{-1}|\\
\textit{code for title page}\\
|\newpage|\\
|\||fi|
\end{tabular}
\end{center}
%
A banner page for the child documents can be generated by:
%
\begin{center}
\begin{tabular}{l}
|\ifchilddoc|\\
|\addtocounter{page}{-1}|\\
\textit{code for banner page}\\
|\newpage|\\
|\||fi|
\end{tabular}
\end{center}
%
Here one could write a message such as:
\begin{center}
|This is the part \childdocname{} of \childdocjob{}.|
\end{center}

%%%%%%%%%%%%%%%%%%%%%%%%%%%%%%%%%%%%%%%%%%%%%%%%%%%%%%%%%%%%%%%%%%%%%%%%%%%%%%%%
\subsection{Flags}
\label{sec:flags}

The package makes it easy to generate different versions
of the main or child documents.
To this end compilation flags can be defined
and assigned different default values.
They will be particularly useful in conjunction
with the forwarding mechanism described in \secref{sec:forward}.

For example, it may be useful to have a flag |\version|
which can be set to |draft| or |final|.
The document source will contain some conditional code
depending on the value of |\version|.
Suppose further, the flag should default to |final| for the main file
and to |draft| for child files
which is a natural assignment for editing the document.
This is achieved by placing the following code
in the preamble of the main document
(below the |\childdocmain| directive):
%
\begin{center}
\begin{tabular}{l}
|\ifchilddoc|\\
|\providecommand{\version}{draft}|\\
|\||else|\\
|\providecommand{\version}{final}|\\
|\||fi|
\end{tabular}
\end{center}
%
The definition by |\providecommand| makes sure
that previous definitions are not overwritten.
Further statements |\providecommand{\version}{...}|
can thus be added before the above code to override it.

For the main file, one might add a line
(between |\childdocmain| and the above block)
%
\begin{center}
|%\ifchilddoc\||else\providecommand{\version}{draft}\||fi|
\end{center}
%
which can be uncommented to produce a draft version.
Likewise one can add a line to the very top of a child file
(above the |\childdocof{|\textit{main}|}| directive)
%
\begin{center}
|%\providecommand{\version}{final}|
\end{center}
%
which can be uncommented to produce the final version of this child document.

%%%%%%%%%%%%%%%%%%%%%%%%%%%%%%%%%%%%%%%%%%%%%%%%%%%%%%%%%%%%%%%%%%%%%%%%%%%%%%%%
\subsection{Forwarding}
\label{sec:forward}

Different versions of the main or child documents
using compilation flags as described in \secref{sec:flags}
can be (permanently) stored in different files
for convenient compilation, viewing and distribution.
To this end, the package defines a command
to pass on compilation to a different file:

%%%%%%%%%%%%%%%%%%%%%%%%%%%%%%%%%%%%%%%%
\DescribeMacro{\childdocforward}
The command |\childdocforward| redirects processing to
another source file:
%
\begin{center}
\begin{tabular}{l}
|\input{childdoc.def}|\\
|\childdocforward[|\textit{main}|]{|\textit{dest}|}|\\
\end{tabular}
\end{center}
%
The argument \textit{dest} is the destination file
(without extension).
It should be the main file or one of the child files.
Note that further \textsf{childdoc} directives
such as |\childdocof| and |\childdocforward|
in the indicated file will be processed in this form.
The optional argument \textit{main}
passes on directly to the main file \textit{main}
while pretending to compile the child \textit{dest}.
This form behaves as if \textit{dest}
issues |\childdocof{|\textit{main}|}| right away,
and no further \textsf{childdoc} directives will be processed.

%%%%%%%%%%%%%%%%%%%%%%%%%%%%%%%%%%%%%%%%
\DescribeMacro{\...prefix}
In the alternative form |\childdocforwardprefix|,
%
\begin{center}
\begin{tabular}{l}
|\input{childdoc.def}|\\
|\childdocforwardprefix[|\textit{main}|]{|\textit{prefix}|}{|\textit{dest}|}|
\end{tabular}
\end{center}
%
the destination file is determined by a pattern
depending on the current file:
To make this work, the current file must be called
`{\textit{prefix}\hspace{0.2em}\textit{suffix}}'
with \textit{prefix} matching precisely the argument.
Processing is then passed on to the file
`{\textit{dest}\hspace{0.2em}\textit{suffix}}'.
Surely, the same effect is achieved by
directly specifying the
argument `{\textit{dest}\hspace{0.2em}\textit{suffix}}'
in the first form.
However, that requires to set up a different file
for each child. With the alternative form of the command
all these files can have exactly the same content
which simplifies setting them up and maintaining them.

For example, the following file |draft.tex|
with a compilation flag |\version| as described in \secref{sec:flags}
compiles the main document as a draft:
%
\begin{center}
\begin{tabular}{l}
|\def\version{draft}|\\
|\input{childdoc.def}|\\
|\childdocforward{|\textit{main}|}|
\end{tabular}
\end{center}
%
Likewise, the following files |final|\textit{nn}|.tex|
compile the final version of the child document
|child|\textit{nn}|.tex|:
%
\begin{center}
\begin{tabular}{l}
|\def\version{final}|\\
|\input{childdoc.def}|\\
|\childdocforwardprefix{final}{child}|
\end{tabular}
\end{center}
%

Note that when several versions of a main file and/or of each child file
are to be generated, it may be convenient to set up a |Makefile| or
shell script to automatise the process.

%%%%%%%%%%%%%%%%%%%%%%%%%%%%%%%%%%%%%%%%%%%%%%%%%%%%%%%%%%%%%%%%%%%%%%%%%%%%%%%%
\subsection{Command Line Processing}
\label{sec:commandline}

The effect of redirection files can also be achieved by invoking
the \LaTeX{} compiler with a more elaborate command line.
Most conveniently this should be done as part
of a shell script or a |Makefile|.

When using \textsf{childdoc} in the main file, the following
command lines effectively perform a redirection
(note that depending on the shell being used,
backslashes may have to be doubled: `|\|' $\to$ `|\\|'):
%
\begin{center}
|... -jobname "|\textit{target}|" |\\|"|[\textit{flags}]%
|\input{childdoc.def}\childdocforward[|\textit{main}|]{|\textit{dest}|}"|
\end{center}
%
Here \textit{target} is the name of the output file,
\textit{main} is the name of the main file
and \textit{dest} is the name of the main or child file to be processed
(all filenames without extensions).
The optional argument \textit{main} can be omitted
if \textit{main} matches \textit{dest}.
Optionally, compilation \textit{flags} can be defined via |\def| commands.
This command line makes the \TeX{} engine believe
it is compiling the file \textit{target}
whose content is specified as the latter parameter.
The provided code then forwards the processing to
\textit{main} or \textit{dest} as described in \secref{sec:forward}.

%%%%%%%%%%%%%%%%%%%%%%%%%%%%%%%%%%%%%%%%%%%%%%%%%%%%%%%%%%%%%%%%%%%%%%%%%%%%%%%%
\subsection{Include by Input}
\label{sec:input}

Including child documents by |\include| has some restrictions by design.
Most notably, the content of a child document always occupies
its own set of pages; pages cannot be shared between child documents.
Usually, this behaviour makes perfect sense
because each child document contain an essential part of the document.
However, in some situations it may be desirable to compose
a document from a collection of parts
without having mandatory page breaks between then.
For this case, the package
provides a mechanism to include parts
by |\input| which can also be processed individually.
However, by construction this mechanism
requires manual handling of the content to be output.

%%%%%%%%%%%%%%%%%%%%%%%%%%%%%%%%%%%%%%%%
\DescribeMacro{\ifchilddocmanual}
The main file should be prepared as usual, see \secref{sec:include}.
However, the document body must make a distinction
between processing of an individual part and of the main document, e.g.:
%
\begin{center}
\begin{tabular}{l}
|\ifchilddocmanual|\\
|\input{\childdocname}|\\
|\||else|\\
\textit{document body with }|\input{|\textit{part}|}|\\
|\||fi|
\end{tabular}
\end{center}
%
The conditional |\ifchilddocmanual| is true whenever
a part to be included by |\input| is being compiled,
and the name of the part is stored in |\childdocname|.

%%%%%%%%%%%%%%%%%%%%%%%%%%%%%%%%%%%%%%%%
\DescribeMacro{\childdocby}
Each part to be included by |\input| should start with:
%
\begin{center}
\begin{tabular}{l}
|\input{childdoc.def}|\\
|\childdocby{|\textit{main}|}|\\
\end{tabular}
\end{center}
%
The directive |\childdocby| is similar to |\childdocof|
described in \secref{sec:include},
but the subsequent selection of content must be done manually.
To that end, both |\ifchilddoc| and |\ifchilddocmanual|
will be true upon processing of a part,
and the name of the part is stored in |\childdocname|.
Note that |\jobname| will be set to the filename of the current part
so that each part receives an individual |.aux| file
that does not interfere with the |.aux| file(s) of the main document.
This behaviour can be altered by the alternative form
|\childdocby[*]{|\textit{main}|}| (with a non-empty optional argument)
which uses the |.aux| file of the main document
by setting |\jobname| to \textit{main}.

%%%%%%%%%%%%%%%%%%%%%%%%%%%%%%%%%%%%%%%%%%%%%%%%%%%%%%%%%%%%%%%%%%%%%%%%%%%%%%%%
\subsection{Driver Development}
\label{sec:driver}

The \textsf{childdoc} mechanism can also be use for the development
of definition files such as \LaTeX{} styles or classes.
This case differs from the above setup with multiple parts
included by |\include| in that no |\includeonly| should be invoked.
This can be achieved by starting the include file
(before |\ProvidesPackage|) with:
%
\begin{center}
\begin{tabular}{l}
|\input{childdoc.def}|\\
|\childdocforward{|\textit{main}|}|\\
\end{tabular}
\end{center}
%
or alternatively with:
%
\begin{center}
\begin{tabular}{l}
|\input{childdoc.def}|\\
|\childdocby{|\textit{main}|}|\\
\end{tabular}
\end{center}
%
Both forms have slightly different effects as described above.
The main file is prepared as usual, see \secref{sec:include}.

%%%%%%%%%%%%%%%%%%%%%%%%%%%%%%%%%%%%%%%%%%%%%%%%%%%%%%%%%%%%%%%%%%%%%%%%%%%%%%%%
\subsection{Legacy Detection}
\label{sec:detection}

The directive |\childdocmain| in the main file can detect
whether the complete document or merely a child is to be compiled
even without using the directive |\childdocof|.
This method is deprecated because it is less robust
and there is no compelling reason to use it;
it is merely provided for backward compatibility
and it may be removed in future versions.

If the detection mechanism is to be used,
it is mandatory to correctly specify
the filename of the main file as the argument of |\childdocmain|:
%
\begin{center}
\begin{tabular}{l}
|\input{childdoc.def}|\\
|\childdocmain{|\textit{main}|}|\\
\end{tabular}
\end{center}
%
If |\jobname| does not match the argument \textit{main} of |\childdocmain|,
it is assumed that |\jobname| points to the child file to be compiled.
When using |\childdocmain| with the main file specified as argument,
it suffices to start a child file
with just |\input{|\textit{main}|}|
without loading of the package and using |\childdocof|.
If instead all processing is done
with the appropriate \textsf{childdoc} directives,
the argument of \textit{main} of |\childdocmain| can be empty.

An alternative version of the command line processing described
in \secref{sec:commandline} using the detection mechanism reads:
%
\begin{center}
|... -jobname "|\textit{target}|" "|[\textit{flags}]%
[|\def\jobname{|\textit{dest}|}|]|\input{|\textit{main}|}"|
\end{center}

%%%%%%%%%%%%%%%%%%%%%%%%%%%%%%%%%%%%%%%%%%%%%%%%%%%%%%%%%%%%%%%%%%%%%%%%%%%%%%%%
\subsection{Manual Code}
\label{sec:manual}

In case one cannot be certain whether the definitions file |childdoc.def|
is installed on the target \TeX{} distribution
and one prefers not to ship it,
it is conceivable to paste a few relevant commands into the sources.

To that end, drop all statements |\input{childdoc.def}|
and perform the replacements as outlined below.
Instead of |\childdocmain{|\textit{main}|}| add the following code
to the top of the main file:
%
\begin{center}
\begin{tabular}{l}
|\||ifdefined\childdocname\endinput\||fi\newif\ifchilddoc|\\
|\edef\childdocname{\scantokens\expandafter{\jobname\noexpand}}|\\
|\def\childdocmain{|\textit{main}|}\||ifx\childdocmain\childdocname\||else|\\
|\childdoctrue\includeonly{\childdocname}\let\jobname\childdocmain\||fi|\\
\end{tabular}
\end{center}
%
Instead of |\childdocof{|\textit{main}|}| just include the main file
at the top of each child file:
%
\begin{center}
|\input{|\textit{main}|}|
\end{center}
%
A simple redirection |\childdocforward{|\textit{dest}|}| is achieved by:
%
\begin{center}
|\def\jobname{|\textit{dest}|}\input{\jobname}|
\end{center}
%
The redirection with prefix
|\childdocforwardprefix[|\textit{prefix}|]{|\textit{dest}|}|
is accomplished by:
%
\begin{center}
\begin{tabular}{l}
|{\edef\jobname{\scantokens\expandafter{\jobname\noexpand}}|\\
|\def\redirectjob |\textit{prefix}|#1~~~{\gdef\jobname{|\textit{dest}|#1}}|\\
|\expandafter\redirectjob\jobname~~~}\input{\jobname}|
\end{tabular}
\end{center}

In an alternative approach,
child documents can be compiled by a specific command line
without additional code or specific definitions:
%
\begin{center}
|... -jobname "|\textit{target}|" "|[\textit{flags}]%
|\includeonly{|\textit{dest}|}\input{|\textit{main}|}"|
\end{center}
%

%%%%%%%%%%%%%%%%%%%%%%%%%%%%%%%%%%%%%%%%%%%%%%%%%%%%%%%%%%%%%%%%%%%%%%%%%%%%%%%%
%%%%%%%%%%%%%%%%%%%%%%%%%%%%%%%%%%%%%%%%%%%%%%%%%%%%%%%%%%%%%%%%%%%%%%%%%%%%%%%%
\section{Information}

%%%%%%%%%%%%%%%%%%%%%%%%%%%%%%%%%%%%%%%%%%%%%%%%%%%%%%%%%%%%%%%%%%%%%%%%%%%%%%%%
\subsection{Copyright}

Copyright \copyright{} 2017--2018 Niklas Beisert

This work may be distributed and/or modified under the
conditions of the \LaTeX{} Project Public License, either version 1.3
of this license or (at your option) any later version.
The latest version of this license is in
  \url{http://www.latex-project.org/lppl.txt}
and version 1.3 or later is part of all distributions of \LaTeX{}
version 2005/12/01 or later.

This work has the LPPL maintenance status `maintained'.

The Current Maintainer of this work is Niklas Beisert.

This work consists of the files |README.txt|, |childdoc.ins| and |childdoc.dtx|
as well as the derived files |childdoc.def|, |cdocsamp.tex|
with |cdocsch1.tex|, |cdocsch2.tex|, |cdocspt3.tex|, |cdocspt4.tex|,
|cdocsdrf.tex|, |cdocsfn1.tex|, |cdocsfn2.tex|
as well as |childdoc.pdf|.

%%%%%%%%%%%%%%%%%%%%%%%%%%%%%%%%%%%%%%%%%%%%%%%%%%%%%%%%%%%%%%%%%%%%%%%%%%%%%%%%
\subsection{Files and Installation}

The package consists of the files:
%
\begin{center}
\begin{tabular}{ll}
    |README.txt|   & readme file \\
    |childdoc.ins| & installation file \\
    |childdoc.dtx| & source file \\
    |childdoc.def| & definition file \\
    |cdocsamp.tex| & sample main file \\
    |cdocsch1.tex| & sample include file \\
    |cdocsch2.tex| & sample include file \\
    |cdocspt3.tex| & sample part file \\
    |cdocspt4.tex| & sample part file \\
    |cdocsdrf.tex| & sample redirection file \\
    |cdocsfn1.tex| & sample redirection file \\
    |cdocsfn2.tex| & sample redirection file \\
    |childdoc.pdf| & manual
\end{tabular}
\end{center}
%
The distribution consists of the files
|README.txt|, |childdoc.ins| and |childdoc.dtx|.
%
\begin{itemize}
\item
Run (pdf)\LaTeX{} on |childdoc.dtx|
to compile the manual |childdoc.pdf| (this file).
\item
Run \LaTeX{} on |childdoc.ins| to create the definitions file |childdoc.def|
and the sample |cdocsamp.tex| with include files
|cdocsch1.tex|, |cdocsch2.tex|, |cdocspt3.tex|, |cdocspt4.tex|,
|cdocsdrf.tex|, |cdocsfn1.tex|, |cdocsfn2.tex|.
Then copy the file |childdoc.def| to an appropriate directory of your \LaTeX{}
distribution, e.g.\ \textit{texmf-root}|/tex/latex/childdoc|.
\end{itemize}

%%%%%%%%%%%%%%%%%%%%%%%%%%%%%%%%%%%%%%%%%%%%%%%%%%%%%%%%%%%%%%%%%%%%%%%%%%%%%%%%
\subsection{Related CTAN Packages}

There are several other packages which offer a similar functionality:
%
\begin{itemize}
\item
The packages
\href{http://ctan.org/pkg/docmute}{\textsf{docmute}},
\href{http://ctan.org/pkg/includex}{\textsf{includex}} and
\href{http://ctan.org/pkg/standalone}{\textsf{standalone}}
provide commands to include only the document body of
a child file thus allowing both files to be compiled individually.
\item
The packages \href{http://ctan.org/pkg/subdocs}{\textsf{subdocs}}
and \href{http://ctan.org/pkg/subfiles}{\textsf{subfiles}}
provide structures in which the main and child documents can be
encapsulated and allowing them to be compiled individually.
The inclusion mechanism is different from the conventional |\include|.
\item
The package \href{http://ctan.org/pkg/combine}{\textsf{combine}}
is an elaborate solution to combine several documents into one.
\end{itemize}
%
See also the CTAN topic \href{http://ctan.org/topic/subdocs}{\textsf{subdocs}}
for further related packages.
The present package differs from the above solutions in that
a document structure constructed with the conventional |\include| mechanism
just needs two extra commands at the top of every file
such that all constituent files can be compiled individually.

%%%%%%%%%%%%%%%%%%%%%%%%%%%%%%%%%%%%%%%%%%%%%%%%%%%%%%%%%%%%%%%%%%%%%%%%%%%%%%%%
%\subsection{Feature Suggestions}
%
%The following is a list of features which may be useful for future
%versions of this package:
%%
%\begin{itemize}
%\item
%\ldots
%\end{itemize}

%%%%%%%%%%%%%%%%%%%%%%%%%%%%%%%%%%%%%%%%%%%%%%%%%%%%%%%%%%%%%%%%%%%%%%%%%%%%%%%%
\subsection{Revision History}

%%%%%%%%%%%%%%%%%%%%%%%%%%%%%%%%%%%%%%%%
\paragraph{v2.0:} 2018/12/30

\begin{itemize}
\item
immediate forward processing
\item
added |\childdocby| mechanism
\item
manual restructured
\end{itemize}

%%%%%%%%%%%%%%%%%%%%%%%%%%%%%%%%%%%%%%%%
\paragraph{v1.6:} 2018/01/17

\begin{itemize}
\item
application for development of include files
\item
corrections to manual
\end{itemize}

%%%%%%%%%%%%%%%%%%%%%%%%%%%%%%%%%%%%%%%%
\paragraph{v1.5:} 2017/05/21

\begin{itemize}
\item
more complete structuring introduced
\item
|\childdocof| introduced
\item
|\childdoc| renamed to |\childdocmain|
\item
|\childredirect| renamed to |\childdocforward| and |\childdocforwardprefix|
and functionality expanded
\end{itemize}

%%%%%%%%%%%%%%%%%%%%%%%%%%%%%%%%%%%%%%%%
\paragraph{v1.0:} 2017/04/27

\begin{itemize}
\item
manual and install package
\item
first version published on CTAN
\end{itemize}

%%%%%%%%%%%%%%%%%%%%%%%%%%%%%%%%%%%%%%%%
\paragraph{v0.6:} 2017/04/26

\begin{itemize}
\item
redirection mechanism added
\end{itemize}

%%%%%%%%%%%%%%%%%%%%%%%%%%%%%%%%%%%%%%%%
\paragraph{v0.5:} 2017/04/26

\begin{itemize}
\item
functionality in definition file
\end{itemize}


%%%%%%%%%%%%%%%%%%%%%%%%%%%%%%%%%%%%%%%%%%%%%%%%%%%%%%%%%%%%%%%%%%%%%%%%%%%%%%%%
%%%%%%%%%%%%%%%%%%%%%%%%%%%%%%%%%%%%%%%%%%%%%%%%%%%%%%%%%%%%%%%%%%%%%%%%%%%%%%%%
%%%%%%%%%%%%%%%%%%%%%%%%%%%%%%%%%%%%%%%%%%%%%%%%%%%%%%%%%%%%%%%%%%%%%%%%%%%%%%%%
\appendix

\settowidth\MacroIndent{\rmfamily\scriptsize 000\ }

 \DocInput{childdoc.dtx}

\end{document}
%</driver>
% \fi
%
% %%%%%%%%%%%%%%%%%%%%%%%%%%%%%%%%%%%%%%%%%%%%%%%%%%%%%%%%%%%%%%%%%%%%%%%%%%%%%%
% %%%%%%%%%%%%%%%%%%%%%%%%%%%%%%%%%%%%%%%%%%%%%%%%%%%%%%%%%%%%%%%%%%%%%%%%%%%%%%
% \section{Sample}
%\iffalse
%<*samplemain>
%\fi
%
% The following presents a sample document
% with two chapters, two parts, a title page,
% a compile flag as well as three forwarding files to set the flag.
% It consists of eight |.tex| files:
% \begin{center}
% \begin{tabular}{ll}
% |cdocsamp.tex|&main file\\
% |cdocsch1.tex|&include file for chapter 1\\
% |cdocsch2.tex|&include file for chapter 2\\
% |cdocspt3.tex|&include file for part 3\\
% |cdocspt4.tex|&include file for part 4\\
% |cdocsdrf.tex|&forwarding file for main file in draft mode\\
% |cdocsfi1.tex|&forwarding file for final version of chapter 1\\
% |cdocsfi2.tex|&forwarding file for final version of chapter 2\\
% \end{tabular}
% \end{center}
% Each of the eight files can be compiled directly by the \LaTeX{} compiler.
%
% %%%%%%%%%%%%%%%%%%%%%%%%%%%%%%%%%%%%%%
% \paragraph{Main File.}
%
% The main file is called |cdocsamp.tex|.
%
% Load the \textsf{childdoc} definitions and
% declare the filename for the main document:
%    \begin{macrocode}
\input{childdoc.def}
\childdocmain{}
%    \end{macrocode}

% Optional override for |\version| flag:
%    \begin{macrocode}
%%\ifchilddoc\else\providecommand{\version}{draft}\fi
%    \end{macrocode}

% Define the default values for the |\version| flag
% (|final| for the main file and |draft| for childs):
%    \begin{macrocode}
\ifchilddoc
\providecommand{\version}{draft}
\else
\providecommand{\version}{final}
\fi
%    \end{macrocode}

% Load the standard document class:
%    \begin{macrocode}
\documentclass[12pt]{article}
%    \end{macrocode}

% Start the document body:
%    \begin{macrocode}
\begin{document}
%    \end{macrocode}

% Declare a title page.
% Print title, part of document being processed and version flag:
%    \begin{macrocode}
\addtocounter{page}{-1}
\begin{center}
{\LARGE\bfseries{}childdoc example\par}
\vspace{1cm}
\ifchilddoc
\ifchilddocmanual part\else chapter\fi:
`\childdocname' of `\childdocjob'\par
\else
main document: `\childdocjob'\par
\fi
version: \version\par
\end{center}
\newpage
%    \end{macrocode}

% Manually include selected file,
% otherwise process as usual:
%    \begin{macrocode}
\ifchilddocmanual
\section*{part `\childdocname'}
\input{\childdocname}
\else
%    \end{macrocode}

% Include the two chapters:
%    \begin{macrocode}
\include{cdocsch1}
\include{cdocsch2}
%    \end{macrocode}

% Include the two parts unless only chapters should be displayed:
%    \begin{macrocode}
\ifchilddoc\else
\section{part three}
\input{cdocspt3}
\section{part four}
\input{cdocspt4}
\fi
%    \end{macrocode}

% Process as usual until here:
%    \begin{macrocode}
\fi
%    \end{macrocode}

% End of document body:
%    \begin{macrocode}
\end{document}
%    \end{macrocode}
%\iffalse
%</samplemain>
%\fi
%
% %%%%%%%%%%%%%%%%%%%%%%%%%%%%%%%%%%%%%%
% \paragraph{Chapter Include Files.}
%
% The include files are called |cdocsch1.tex| and |cdocsch2.tex|.
%
%\iffalse
%<*samplechap1|samplechap2>
%\fi

% Optional override for |\version| flag:
%    \begin{macrocode}
%%\providecommand{\version}{final}
%    \end{macrocode}

% Include the main document:
%    \begin{macrocode}
\input{childdoc.def}
\childdocof{cdocsamp}
%    \end{macrocode}

%\iffalse
%</samplechap1|samplechap2>
%\fi
%
%\iffalse
%<*samplechap1>
%\fi
% Some text for chapter 1:
%    \begin{macrocode}
\section{one}
some text in chapter one
%    \end{macrocode}

%\iffalse
%</samplechap1>
%\fi
% Some text for chapter 2:
%\iffalse
%<*samplechap2>
%\fi
%    \begin{macrocode}
\section{two}
more text in chapter two
%    \end{macrocode}

%\iffalse
%</samplechap2>
%\fi
%
% %%%%%%%%%%%%%%%%%%%%%%%%%%%%%%%%%%%%%%
% \paragraph{Part Include Files.}
%
% The include files are called |cdocspt3.tex| and |cdocspt4.tex|.
%
%\iffalse
%<*samplepart3|samplepart4>
%\fi

% Optional override for |\version| flag:
%    \begin{macrocode}
%%\providecommand{\version}{final}
%    \end{macrocode}

% Include the main document:
%    \begin{macrocode}
\input{childdoc.def}
\childdocby{cdocsamp}
%    \end{macrocode}

%\iffalse
%</samplepart3|samplepart4>
%\fi
%
%\iffalse
%<*samplepart3>
%\fi
% Some text for part 3:
%    \begin{macrocode}
some text in part three
%    \end{macrocode}

%\iffalse
%</samplepart3>
%\fi
% Some text for part 4:
%\iffalse
%<*samplepart4>
%\fi
%    \begin{macrocode}
more text in part four
%    \end{macrocode}

%\iffalse
%</samplepart4>
%\fi
%
% %%%%%%%%%%%%%%%%%%%%%%%%%%%%%%%%%%%%%%
% \paragraph{Forwarding for a Complete Draft.}
%
% The following forwarding file |cdocsdrf.tex|
% compiles the main document in draft mode:
%\iffalse
%<*sampledraft>
%\fi
%    \begin{macrocode}
\def\version{draft}
\input{childdoc.def}
\childdocforward{cdocsamp}
%    \end{macrocode}

%\iffalse
%</sampledraft>
%\fi
%
% %%%%%%%%%%%%%%%%%%%%%%%%%%%%%%%%%%%%%%
% \paragraph{Forwarding for Final Version of the Chapters.}
%
% The following forwarding files |cdocsfn1.tex| and |cdocsfn2.tex|
% (with identical content)
% compile the final versions of the child documents
% |cdocsch1.tex| and |cdocsch2.tex|, respectively:
%\iffalse
%<*samplefinal>
%\fi
%    \begin{macrocode}
\def\version{final}
\input{childdoc.def}
\childdocforwardprefix[cdocsamp]{cdocsfn}{cdocsch}
%    \end{macrocode}

%\iffalse
%</samplefinal>
%\fi
%
% %%%%%%%%%%%%%%%%%%%%%%%%%%%%%%%%%%%%%%
% \paragraph{Command Line Processing.}
%
% The following three command lines generate the output files
% |cdocscld|, |cdocscl1| and |cdocscl2|
% which should be identical to
% |cdocsdrf|, |cdocsch1| and |cdocsfn2|, respectively:
% \begin{center}
% \begin{tabular}{l}
% |latex -jobname cdocscld \|\\
% |  "\def\version{draft}\input{childdoc.def}\childdocforward{cdocsamp}"|\\
% |latex -jobname cdocscl1 \|\\
% |  "\input{childdoc.def}\childdocforward[cdocsamp]{cdocsch1}"|\\
% |latex -jobname cdocscl2 \|\\
% |  "\def\version{final}\input{childdoc.def}\childdocforward{cdocsch2}"|
% \end{tabular}
% \end{center}
% Note that the trailing backslash on each first line
% merely continues the input to the second line
% (for convenient cut ant paste).
% Furthermore, the command |latex| can be replaced by any
% of its alternative versions such as |pdflatex|.
%
% %%%%%%%%%%%%%%%%%%%%%%%%%%%%%%%%%%%%%%%%%%%%%%%%%%%%%%%%%%%%%%%%%%%%%%%%%%%%%%
% %%%%%%%%%%%%%%%%%%%%%%%%%%%%%%%%%%%%%%%%%%%%%%%%%%%%%%%%%%%%%%%%%%%%%%%%%%%%%%
% \section{Implementation}
%\iffalse
%<*package>
%\fi
%
% This section describes the definitions file |childdoc.def|.

% The definitions cannot be loaded using |\usepackage| or |\RequirePackage|
% which has a mechanism to prevent loading a style file more than once.
% When loading the definitions by means of |\input|
% multiple instances have to be prevented manually:
%\iffalse
%This code needs to be before the `\ProvidesFile' directive
%which is defined at the beginning of this file.
%Therefore it is also placed there and commented out here.
%</package>
%<*discard>
%\fi
%    \begin{macrocode}
\ifdefined\childdocmain\endinput\fi
%    \end{macrocode}
%\iffalse
%</discard>
%<*package>
%\fi
%
% \macro{\ifchilddoc}
% \macro{\ifchilddocmanual}
% The conditional |\ifchilddoc| tells whether a
% child (true) or main (false) document is being compiled.
% The conditional |\ifchilddocmanual| tells whether
% the |\includeonly| mechanism is used (false) or
% the selection of child files must be performed manually (true).
% The definitions initialise to false:
%    \begin{macrocode}
\newif\ifchilddoc
\newif\ifchilddocmanual
%    \end{macrocode}

% \macro{\childdocname}
% \macro{\childdocjob}
% The macro |\childdocname| stores the name of the main document
% to be compiled. The macro |\childdocjob| stores the name of
% the document on which the \LaTeX{} compiler was originally invoked.
% The content of |\jobname| cannot be compared
% to filenames specified in the source due to different catcodes.
% The following code rescans |\jobname|, stores the result
% in |\childdocname| and saves a copy in |\childdocjob|:
%    \begin{macrocode}
\edef\childdocname{\scantokens\expandafter{\jobname\noexpand}}
\let\childdocjob\childdocname
%    \end{macrocode}

% \macro{\childdocdisable}
% The macro |\childdocdisable| prevents the main file
% from being processed more than once.
% At this stage, the main document command |\childdocmain|
% is assumed to be called once again where it should do nothing.
% Any subsequent call to it should prevent
% a secondary processing of the main document
% It overwrites the forwarding commands
% |\childdocof| and |\childdocforward|
% with empty macros to prevent further inclusions of the main document:
%    \begin{macrocode}
\newcommand{\childdocdisable}
{
  \renewcommand{\childdocmain}[1]{\renewcommand{\childdocmain}[1]{\endinput}}
  \renewcommand{\childdocof}[1]{}
  \renewcommand{\childdocby}[2][]{}
  \renewcommand{\childdocforward}[2][]{}
  \renewcommand{\childdocdisable}{}
}
%    \end{macrocode}

% \macro{\childdocmain}
% The macro |\childdocmain| is to be called at the top of the main file
% with nothing or the main filename (without extension) as argument.
% First, it breaks loops.
% If the argument is not empty and does not match |\childdocname|
% (which is set by the first inclusion of |childdoc.def|),
% |\ifchilddoc| is set to true, |\includeonly| is applied to the child file
% and |\jobname| is set to the main file
% (for proper handling of |.aux| files):
%    \begin{macrocode}
\newcommand{\childdocmain}[1]
{
  \childdocdisable\childdocmain{}
  \if?#1?\else
    \begingroup
      \def\childdoctmp{#1}
      \ifx\childdoctmp\childdocname
        \def\childdoctmp{}
      \else
        \def\childdoctmp
        {
          \childdoctrue
          \includeonly{\childdocname}
          \def\childdocjob{#1}
          \def\jobname{#1}
        }
      \fi
      \expandafter
    \endgroup
    \childdoctmp
  \fi
}
%    \end{macrocode}

% \macro{\childdocof}
% The command |\childdocof| redirects
% compilation to the main file |#1|.
%    \begin{macrocode}
\newcommand{\childdocof}[1]
{
  \childdocdisable
  \childdoctrue
  \includeonly{\childdocname}
  \def\jobname{#1}
  \def\childdocjob{#1}
  \input{#1}
}
%    \end{macrocode}

% \macro{\childdocby}
% The command |\childdocby| ....
%    \begin{macrocode}
\newcommand{\childdocby}[2][]
{
  \childdocdisable
  \childdoctrue
  \childdocmanualtrue
  \if?#1?\else
    \def\jobname{#2}
  \fi
  \def\childdocjob{#2}
  \input{#2}
  \endinput
}
%    \end{macrocode}

% \macro{\childdocforward}
% The command |\childdocforward| redirects
% compilation to the main file or
% (if the optional argument is given) a child file.
% Parameters are set as if the main file
% or a child file starting with |\childdocof| was compiled.
% Then compilation is handed over to the main file:
%    \begin{macrocode}
\newcommand{\childdocforward}[2][]
{
  \begingroup
    \if?#1?
      \def\childdoctmp
      {
        \def\childdocname{#2}
        \def\childdocjob{#2}
        \def\jobname{#2}
        \input{#2}
        \endinput
      }
    \else
      \def\childdoctmp
      {
        \childdocdisable
        \def\childdocname{#2}
        \childdoctrue
        \includeonly{#2}
        \def\childdocjob{#1}
        \def\jobname{#1}
        \input{#1}
        \endinput
      }
    \fi
    \expandafter
  \endgroup
  \childdoctmp
}
%    \end{macrocode}

% \macro{\childdocforwardprefix}
% The command |\childdocforwardprefix| redirects
% compilation to the main or a child file by means of a pattern.
% The prefix |#1| in the current filename is replaced by |#2|
% and the suffix of the current filename is kept
% (it is assumed that the filename does not contain the substring `|~~~|'
% which is used as a delimiter).
% Compilation is handed over to the new file by |\childdocforward|:
%    \begin{macrocode}
\newcommand{\childdocforwardprefix}[3][]
{
  \begingroup
    \def\childdocextract #2##1~~~{\def\childdoctmp{\childdocforward[#1]{#3##1}}}
    \expandafter\childdocextract\childdocname~~~
    \expandafter
  \endgroup
  \childdoctmp
}
%    \end{macrocode}

% \macro{\childdoc}
% The deprecated macro |\childdoc| is a legacy version of |\childdocmain|:
%    \begin{macrocode}
\newcommand{\childdoc}{\childdocmain}
%    \end{macrocode}

% \macro{\childdocredirect}
% The deprecated macro |\childdocredirect| is a legacy version
% of |\childdocforward| and |\childdocforwardprefix|:
%    \begin{macrocode}
\newcommand{\childdocredirect}[2][]
{
  \begingroup
    \if?#1?
      \def\childdoctmp{\childdocforward{#2}}
    \else
      \def\childdoctmp{\childdocforwardprefix{#1}{#2}}
    \fi
    \expandafter
  \endgroup
  \childdoctmp
}
%    \end{macrocode}

%\iffalse
%</package>
%\fi
%
\endinput

\childdocmain{}
%    \end{macrocode}

% Optional override for |\version| flag:
%    \begin{macrocode}
%%\ifchilddoc\else\providecommand{\version}{draft}\fi
%    \end{macrocode}

% Define the default values for the |\version| flag
% (|final| for the main file and |draft| for childs):
%    \begin{macrocode}
\ifchilddoc
\providecommand{\version}{draft}
\else
\providecommand{\version}{final}
\fi
%    \end{macrocode}

% Load the standard document class:
%    \begin{macrocode}
\documentclass[12pt]{article}
%    \end{macrocode}

% Start the document body:
%    \begin{macrocode}
\begin{document}
%    \end{macrocode}

% Declare a title page.
% Print title, part of document being processed and version flag:
%    \begin{macrocode}
\addtocounter{page}{-1}
\begin{center}
{\LARGE\bfseries{}childdoc example\par}
\vspace{1cm}
\ifchilddoc
\ifchilddocmanual part\else chapter\fi:
`\childdocname' of `\childdocjob'\par
\else
main document: `\childdocjob'\par
\fi
version: \version\par
\end{center}
\newpage
%    \end{macrocode}

% Manually include selected file,
% otherwise process as usual:
%    \begin{macrocode}
\ifchilddocmanual
\section*{part `\childdocname'}
\input{\childdocname}
\else
%    \end{macrocode}

% Include the two chapters:
%    \begin{macrocode}
\include{cdocsch1}
\include{cdocsch2}
%    \end{macrocode}

% Include the two parts unless only chapters should be displayed:
%    \begin{macrocode}
\ifchilddoc\else
\section{part three}
\input{cdocspt3}
\section{part four}
\input{cdocspt4}
\fi
%    \end{macrocode}

% Process as usual until here:
%    \begin{macrocode}
\fi
%    \end{macrocode}

% End of document body:
%    \begin{macrocode}
\end{document}
%    \end{macrocode}
%\iffalse
%</samplemain>
%\fi
%
% %%%%%%%%%%%%%%%%%%%%%%%%%%%%%%%%%%%%%%
% \paragraph{Chapter Include Files.}
%
% The include files are called |cdocsch1.tex| and |cdocsch2.tex|.
%
%\iffalse
%<*samplechap1|samplechap2>
%\fi

% Optional override for |\version| flag:
%    \begin{macrocode}
%%\providecommand{\version}{final}
%    \end{macrocode}

% Include the main document:
%    \begin{macrocode}
% \iffalse
%
% childdoc.dtx Copyright (C) 2017-2018 Niklas Beisert
%
% This work may be distributed and/or modified under the
% conditions of the LaTeX Project Public License, either version 1.3
% of this license or (at your option) any later version.
% The latest version of this license is in
%   http://www.latex-project.org/lppl.txt
% and version 1.3 or later is part of all distributions of LaTeX
% version 2005/12/01 or later.
%
% This work has the LPPL maintenance status `maintained'.
%
% The Current Maintainer of this work is Niklas Beisert.
%
% This work consists of the files childdoc.dtx and childdoc.ins
% and the derived files childdoc.def and cdocsamp.tex with
% cdocsch1.tex, cdocsch2.tex, cdocsdrf.tex, cdocsfn1.tex, cdocsfn2.tex.
%
%<package>\ifdefined\childdocmain\endinput\fi
%<package>\ProvidesFile{childdoc.def}[2018/12/30 v2.0 child document driver]
%<samplemain>\ProvidesFile{cdocsamp.tex}[2018/12/30 v2.0 sample for childdoc]
%<*driver>
%\ProvidesFile{childdoc.drv}[2018/12/30 v2.0 childdoc reference manual file]
\PassOptionsToClass{10pt,a4paper}{article}
\documentclass{ltxdoc}

\usepackage[margin=35mm]{geometry}
\usepackage{hyperref}
\usepackage{hyperxmp}
\usepackage[usenames]{color}

\hypersetup{colorlinks=true}
\hypersetup{pdfstartview=FitH}
\hypersetup{pdfpagemode=UseNone}
\hypersetup{pdfsource={}}
\hypersetup{pdflang={en-UK}}
\hypersetup{pdfcopyright={Copyright 2017-2018 Niklas Beisert.
  This work may be distributed and/or modified under the
  conditions of the LaTeX Project Public License, either version 1.3
  of this license or (at your option) any later version.}}
\hypersetup{pdflicenseurl={http://www.latex-project.org/lppl.txt}}
\hypersetup{pdfcontactaddress={ETH Zurich, ITP, HIT K,
  Wolfgang-Pauli-Strasse 27}}
\hypersetup{pdfcontactpostcode={8093}}
\hypersetup{pdfcontactcity={Zurich}}
\hypersetup{pdfcontactcountry={Switzerland}}
\hypersetup{pdfcontactemail={nbeisert@itp.phys.ethz.ch}}
\hypersetup{pdfcontacturl={http://people.phys.ethz.ch/\xmptilde nbeisert/}}

\newcommand{\secref}[1]{\hyperref[#1]{section \ref*{#1}}}

\parskip1ex
\parindent0pt
\let\olditemize\itemize
\def\itemize{\olditemize\parskip0pt}

\begin{document}

\title{The \textsf{childdoc} Package}
\hypersetup{pdftitle={The childdoc Package}}
\author{Niklas Beisert\\[2ex]
  Institut f\"ur Theoretische Physik\\
  Eidgen\"ossische Technische Hochschule Z\"urich\\
  Wolfgang-Pauli-Strasse 27, 8093 Z\"urich, Switzerland\\[1ex]
  \href{mailto:nbeisert@itp.phys.ethz.ch}
  {\texttt{nbeisert@itp.phys.ethz.ch}}}
\hypersetup{pdfauthor={Niklas Beisert}}
\hypersetup{pdfsubject={Manual for the LaTeX2e Package childdoc}}
\date{30 December 2018, \textsf{v2.0}}
\maketitle

\begin{abstract}\noindent
\textsf{childdoc} is a \LaTeXe{} package
that enables the direct compilation
of document sections included by |\include|
to individual files.
\end{abstract}

\begingroup
\parskip0ex
\tableofcontents
\endgroup

%%%%%%%%%%%%%%%%%%%%%%%%%%%%%%%%%%%%%%%%%%%%%%%%%%%%%%%%%%%%%%%%%%%%%%%%%%%%%%%%
%%%%%%%%%%%%%%%%%%%%%%%%%%%%%%%%%%%%%%%%%%%%%%%%%%%%%%%%%%%%%%%%%%%%%%%%%%%%%%%%
\section{Introduction}

\LaTeX{} provides a mechanism to structure a large document (such as a book)
into a main file and several child files (containing the chapters)
using the |\include| command.
This mechanism is beneficial for documents
which span hundreds of pages in order to
make the source file(s) more manageable.
Moreover, compilation can be restricted to
selected child files by means of the |\includeonly| command.
The latter feature can be used to reduce the compilation time while editing
(this was significantly more useful in the earlier days of \LaTeX{})
or to generate a smaller document which is easier to navigate.
Another application of |\includeonly| is to generate
documents consisting of selected parts of the complete document.

However, there are a few drawbacks of the plain |\include| mechanism:
\begin{itemize}
\item
The child files cannot be compiled on their own,
they can only be compiled via the main file.
A naive editing environment
(such as a text editor with an option
to have the current file processed by \LaTeX)
may require one to switch to the main file before compiling;
attempting to compile the child file produces errors.
\item
The main file must be modified (each time)
to adjust the |\includeonly| command
to the present needs. This easily leaves the main file in a messy state.
\item
The generated document will always carry the filename
of the main document. This is inconvenient if
several child files are to be compiled and
to be kept for distribution.
\end{itemize}

The present package provides a simple interface
to make child files individually compilable by \LaTeX{}.
Compiling a child file then has the same effect as compiling
the main file with an |\includeonly| command
to select the appropriate child.
Moreover the generated document will carry the name of the child
rather than the main file.
This resolves all three above issues.

This feature is meant to make the editing of books,
thesis documents and lecture notes somewhat more convenient.
However, the package can also be used efficiently for
composing a series of documents (such as exercise sheets)
which are typically distributed individually.
It then assists the author in generating the individual documents
(potentially in different versions)
as well as a document containing the collected series.
Another application is in developing style files
or other kinds of included material
where compilation of the style file could redirect
to a sample or test file.

%%%%%%%%%%%%%%%%%%%%%%%%%%%%%%%%%%%%%%%%%%%%%%%%%%%%%%%%%%%%%%%%%%%%%%%%%%%%%%%%
%%%%%%%%%%%%%%%%%%%%%%%%%%%%%%%%%%%%%%%%%%%%%%%%%%%%%%%%%%%%%%%%%%%%%%%%%%%%%%%%
\section{Usage}

First of all, the package \textsf{childdoc} is \emph{not} a standard
\LaTeXe{} |.sty| style file! Therefore it needs to be invoked in
a non-standard way.

%%%%%%%%%%%%%%%%%%%%%%%%%%%%%%%%%%%%%%%%%%%%%%%%%%%%%%%%%%%%%%%%%%%%%%%%%%%%%%%%
\subsection{Included Files}
\label{sec:include}

%%%%%%%%%%%%%%%%%%%%%%%%%%%%%%%%%%%%%%%%
\DescribeMacro{\childdocmain}
To use the package, add the commands
\begin{center}
\begin{tabular}{l}
|\input{childdoc.def}|\\
|\childdocmain{}|\\
\end{tabular}
\end{center}
at the very top of the main \LaTeX{} file,
in particular \emph{before} the |\documentclass| statement!
The argument of |\childdocmain| should be left empty
(but it must be present).

%%%%%%%%%%%%%%%%%%%%%%%%%%%%%%%%%%%%%%%%
\DescribeMacro{\childdocof}
Furthermore, add the commands
\begin{center}
\begin{tabular}{l}
|\input{childdoc.def}|\\
|\childdocof{|\textit{main}|}|\\
\end{tabular}
\end{center}
at the top of every child file \textit{child}
which is included by |\include{|\textit{child}|}|
from within the main file
(or at least for those files to be compiled individually).
The argument \textit{main} must be the filename of the main file.

There are a couple of
considerations in setting up the main and child documents:

%%%%%%%%%%%%%%%%%%%%%%%%%%%%%%%%%%%%%%%%
\paragraph{Restrictions.}

Please note the following restrictions:
\begin{itemize}
\item
|\childdocmain| must be called with one argument \textit{main}
to ensure compatibility with earlier version of the package.
It must either be empty (|\childdocmain{}|)
or precisely match the filename of the main file in which it is specified.
See \secref{sec:detection} for further information.
\item
The filename \textit{main} must be specified without the |.tex| extension.
\item
The filename \textit{main} is case sensitive
(even in case-insensitive file systems)
due to internal string comparison.
\item
The argument \textit{main} should be fully expanded, it cannot be a macro.
\item
Subdirectories and special characters should be avoided in filenames.
\item
The command |\childdocmain{|\textit{main}|}| must be followed by a whitespace.
It should not be followed immediately by another command
or by a comment mark `|%|'.
This is because the \TeX{} parser reads the token immediately following
the argument of |\childdocmain| and puts it
at the beginning of every child section;
however, a white\-space is ignored.
\end{itemize}

%%%%%%%%%%%%%%%%%%%%%%%%%%%%%%%%%%%%%%%%
\paragraph{Content of Main File.}

It is advisable to place all content in the child files included by |\include|.
Any output contained in the main file will appear in all child documents
unless suppressed manually;
it cannot be suppressed automatically by the |\includeonly| directive
and thus should normally be avoided.
A method to include some content in the main file
by means of conditional processing is described in \secref{sec:conditional}.

%%%%%%%%%%%%%%%%%%%%%%%%%%%%%%%%%%%%%%%%
\paragraph{Page Numbering.}

When only a part of the document is compiled,
the appropriate numbering of pages
(as well as other status parameters)
is determined from the |.aux| files.
The latter contain information from previous passes.
However this information needs to propagate through
all intermediate child documents.
Therefore the page numbering in child documents may well
be inconsistent until the complete document is compiled at least once.

A useful (if unconventional) way to always ensure a consistent
page numbering is to restart the numbering in each child document
and denote the pages by `\textit{child}|.|\textit{page}'
where \textit{child} represents the chapter/section number of the child file.
This can be achieved by the command
|\numberwithin{page}{|\textit{child}|}|
of the \textsf{amsmath} package
where \textit{child} can be |chapter| or |section|
depending on the chosen structuring.
Alternatively, one can modify the macro |\thepage| appropriately
and reset the counter |page| at the start of each child file.

%%%%%%%%%%%%%%%%%%%%%%%%%%%%%%%%%%%%%%%%%%%%%%%%%%%%%%%%%%%%%%%%%%%%%%%%%%%%%%%%
\subsection{Conditional Processing}
\label{sec:conditional}

The package provides a mechanism to compile different versions
of a document. To customise the versions further some conditional processing
can come in handy to distinguish which version is being compiled.
The package provides two macros to describe the compilation context:

%%%%%%%%%%%%%%%%%%%%%%%%%%%%%%%%%%%%%%%%
\DescribeMacro{\ifchilddoc}
The conditional |\ifchilddoc| distinguishes between the compilation of
child documents and the main document:
%
\begin{center}
|\ifchilddoc |\textit{child-code}| |[|\||else |\textit{main-code}]| \||fi|
\end{center}

%%%%%%%%%%%%%%%%%%%%%%%%%%%%%%%%%%%%%%%%
\DescribeMacro{\childdocname}
\DescribeMacro{\childdocjob}
The macro |\childdocname| contains the filename (without extension)
of the main or child file being processed.
Note that |\childdocjob| will always contain the name of the main file.

%%%%%%%%%%%%%%%%%%%%%%%%%%%%%%%%%%%%%%%%
\paragraph{Title Page.}

Conditional processing can be used to include a title or banner page
in the main document when proper precautions are taken.
Importantly, the code in the main file should ensure that the page counter
(as well as other status parameters which are stored in the |.aux| files)
takes the same value after the conditional processing.
Otherwise the page numbers may take divergent values
depending on which part is compiled.

For example, a title page could be declared by:
%
\begin{center}
\begin{tabular}{l}
|\ifchilddoc\||else|\\
|\addtocounter{page}{-1}|\\
\textit{code for title page}\\
|\newpage|\\
|\||fi|
\end{tabular}
\end{center}
%
A banner page for the child documents can be generated by:
%
\begin{center}
\begin{tabular}{l}
|\ifchilddoc|\\
|\addtocounter{page}{-1}|\\
\textit{code for banner page}\\
|\newpage|\\
|\||fi|
\end{tabular}
\end{center}
%
Here one could write a message such as:
\begin{center}
|This is the part \childdocname{} of \childdocjob{}.|
\end{center}

%%%%%%%%%%%%%%%%%%%%%%%%%%%%%%%%%%%%%%%%%%%%%%%%%%%%%%%%%%%%%%%%%%%%%%%%%%%%%%%%
\subsection{Flags}
\label{sec:flags}

The package makes it easy to generate different versions
of the main or child documents.
To this end compilation flags can be defined
and assigned different default values.
They will be particularly useful in conjunction
with the forwarding mechanism described in \secref{sec:forward}.

For example, it may be useful to have a flag |\version|
which can be set to |draft| or |final|.
The document source will contain some conditional code
depending on the value of |\version|.
Suppose further, the flag should default to |final| for the main file
and to |draft| for child files
which is a natural assignment for editing the document.
This is achieved by placing the following code
in the preamble of the main document
(below the |\childdocmain| directive):
%
\begin{center}
\begin{tabular}{l}
|\ifchilddoc|\\
|\providecommand{\version}{draft}|\\
|\||else|\\
|\providecommand{\version}{final}|\\
|\||fi|
\end{tabular}
\end{center}
%
The definition by |\providecommand| makes sure
that previous definitions are not overwritten.
Further statements |\providecommand{\version}{...}|
can thus be added before the above code to override it.

For the main file, one might add a line
(between |\childdocmain| and the above block)
%
\begin{center}
|%\ifchilddoc\||else\providecommand{\version}{draft}\||fi|
\end{center}
%
which can be uncommented to produce a draft version.
Likewise one can add a line to the very top of a child file
(above the |\childdocof{|\textit{main}|}| directive)
%
\begin{center}
|%\providecommand{\version}{final}|
\end{center}
%
which can be uncommented to produce the final version of this child document.

%%%%%%%%%%%%%%%%%%%%%%%%%%%%%%%%%%%%%%%%%%%%%%%%%%%%%%%%%%%%%%%%%%%%%%%%%%%%%%%%
\subsection{Forwarding}
\label{sec:forward}

Different versions of the main or child documents
using compilation flags as described in \secref{sec:flags}
can be (permanently) stored in different files
for convenient compilation, viewing and distribution.
To this end, the package defines a command
to pass on compilation to a different file:

%%%%%%%%%%%%%%%%%%%%%%%%%%%%%%%%%%%%%%%%
\DescribeMacro{\childdocforward}
The command |\childdocforward| redirects processing to
another source file:
%
\begin{center}
\begin{tabular}{l}
|\input{childdoc.def}|\\
|\childdocforward[|\textit{main}|]{|\textit{dest}|}|\\
\end{tabular}
\end{center}
%
The argument \textit{dest} is the destination file
(without extension).
It should be the main file or one of the child files.
Note that further \textsf{childdoc} directives
such as |\childdocof| and |\childdocforward|
in the indicated file will be processed in this form.
The optional argument \textit{main}
passes on directly to the main file \textit{main}
while pretending to compile the child \textit{dest}.
This form behaves as if \textit{dest}
issues |\childdocof{|\textit{main}|}| right away,
and no further \textsf{childdoc} directives will be processed.

%%%%%%%%%%%%%%%%%%%%%%%%%%%%%%%%%%%%%%%%
\DescribeMacro{\...prefix}
In the alternative form |\childdocforwardprefix|,
%
\begin{center}
\begin{tabular}{l}
|\input{childdoc.def}|\\
|\childdocforwardprefix[|\textit{main}|]{|\textit{prefix}|}{|\textit{dest}|}|
\end{tabular}
\end{center}
%
the destination file is determined by a pattern
depending on the current file:
To make this work, the current file must be called
`{\textit{prefix}\hspace{0.2em}\textit{suffix}}'
with \textit{prefix} matching precisely the argument.
Processing is then passed on to the file
`{\textit{dest}\hspace{0.2em}\textit{suffix}}'.
Surely, the same effect is achieved by
directly specifying the
argument `{\textit{dest}\hspace{0.2em}\textit{suffix}}'
in the first form.
However, that requires to set up a different file
for each child. With the alternative form of the command
all these files can have exactly the same content
which simplifies setting them up and maintaining them.

For example, the following file |draft.tex|
with a compilation flag |\version| as described in \secref{sec:flags}
compiles the main document as a draft:
%
\begin{center}
\begin{tabular}{l}
|\def\version{draft}|\\
|\input{childdoc.def}|\\
|\childdocforward{|\textit{main}|}|
\end{tabular}
\end{center}
%
Likewise, the following files |final|\textit{nn}|.tex|
compile the final version of the child document
|child|\textit{nn}|.tex|:
%
\begin{center}
\begin{tabular}{l}
|\def\version{final}|\\
|\input{childdoc.def}|\\
|\childdocforwardprefix{final}{child}|
\end{tabular}
\end{center}
%

Note that when several versions of a main file and/or of each child file
are to be generated, it may be convenient to set up a |Makefile| or
shell script to automatise the process.

%%%%%%%%%%%%%%%%%%%%%%%%%%%%%%%%%%%%%%%%%%%%%%%%%%%%%%%%%%%%%%%%%%%%%%%%%%%%%%%%
\subsection{Command Line Processing}
\label{sec:commandline}

The effect of redirection files can also be achieved by invoking
the \LaTeX{} compiler with a more elaborate command line.
Most conveniently this should be done as part
of a shell script or a |Makefile|.

When using \textsf{childdoc} in the main file, the following
command lines effectively perform a redirection
(note that depending on the shell being used,
backslashes may have to be doubled: `|\|' $\to$ `|\\|'):
%
\begin{center}
|... -jobname "|\textit{target}|" |\\|"|[\textit{flags}]%
|\input{childdoc.def}\childdocforward[|\textit{main}|]{|\textit{dest}|}"|
\end{center}
%
Here \textit{target} is the name of the output file,
\textit{main} is the name of the main file
and \textit{dest} is the name of the main or child file to be processed
(all filenames without extensions).
The optional argument \textit{main} can be omitted
if \textit{main} matches \textit{dest}.
Optionally, compilation \textit{flags} can be defined via |\def| commands.
This command line makes the \TeX{} engine believe
it is compiling the file \textit{target}
whose content is specified as the latter parameter.
The provided code then forwards the processing to
\textit{main} or \textit{dest} as described in \secref{sec:forward}.

%%%%%%%%%%%%%%%%%%%%%%%%%%%%%%%%%%%%%%%%%%%%%%%%%%%%%%%%%%%%%%%%%%%%%%%%%%%%%%%%
\subsection{Include by Input}
\label{sec:input}

Including child documents by |\include| has some restrictions by design.
Most notably, the content of a child document always occupies
its own set of pages; pages cannot be shared between child documents.
Usually, this behaviour makes perfect sense
because each child document contain an essential part of the document.
However, in some situations it may be desirable to compose
a document from a collection of parts
without having mandatory page breaks between then.
For this case, the package
provides a mechanism to include parts
by |\input| which can also be processed individually.
However, by construction this mechanism
requires manual handling of the content to be output.

%%%%%%%%%%%%%%%%%%%%%%%%%%%%%%%%%%%%%%%%
\DescribeMacro{\ifchilddocmanual}
The main file should be prepared as usual, see \secref{sec:include}.
However, the document body must make a distinction
between processing of an individual part and of the main document, e.g.:
%
\begin{center}
\begin{tabular}{l}
|\ifchilddocmanual|\\
|\input{\childdocname}|\\
|\||else|\\
\textit{document body with }|\input{|\textit{part}|}|\\
|\||fi|
\end{tabular}
\end{center}
%
The conditional |\ifchilddocmanual| is true whenever
a part to be included by |\input| is being compiled,
and the name of the part is stored in |\childdocname|.

%%%%%%%%%%%%%%%%%%%%%%%%%%%%%%%%%%%%%%%%
\DescribeMacro{\childdocby}
Each part to be included by |\input| should start with:
%
\begin{center}
\begin{tabular}{l}
|\input{childdoc.def}|\\
|\childdocby{|\textit{main}|}|\\
\end{tabular}
\end{center}
%
The directive |\childdocby| is similar to |\childdocof|
described in \secref{sec:include},
but the subsequent selection of content must be done manually.
To that end, both |\ifchilddoc| and |\ifchilddocmanual|
will be true upon processing of a part,
and the name of the part is stored in |\childdocname|.
Note that |\jobname| will be set to the filename of the current part
so that each part receives an individual |.aux| file
that does not interfere with the |.aux| file(s) of the main document.
This behaviour can be altered by the alternative form
|\childdocby[*]{|\textit{main}|}| (with a non-empty optional argument)
which uses the |.aux| file of the main document
by setting |\jobname| to \textit{main}.

%%%%%%%%%%%%%%%%%%%%%%%%%%%%%%%%%%%%%%%%%%%%%%%%%%%%%%%%%%%%%%%%%%%%%%%%%%%%%%%%
\subsection{Driver Development}
\label{sec:driver}

The \textsf{childdoc} mechanism can also be use for the development
of definition files such as \LaTeX{} styles or classes.
This case differs from the above setup with multiple parts
included by |\include| in that no |\includeonly| should be invoked.
This can be achieved by starting the include file
(before |\ProvidesPackage|) with:
%
\begin{center}
\begin{tabular}{l}
|\input{childdoc.def}|\\
|\childdocforward{|\textit{main}|}|\\
\end{tabular}
\end{center}
%
or alternatively with:
%
\begin{center}
\begin{tabular}{l}
|\input{childdoc.def}|\\
|\childdocby{|\textit{main}|}|\\
\end{tabular}
\end{center}
%
Both forms have slightly different effects as described above.
The main file is prepared as usual, see \secref{sec:include}.

%%%%%%%%%%%%%%%%%%%%%%%%%%%%%%%%%%%%%%%%%%%%%%%%%%%%%%%%%%%%%%%%%%%%%%%%%%%%%%%%
\subsection{Legacy Detection}
\label{sec:detection}

The directive |\childdocmain| in the main file can detect
whether the complete document or merely a child is to be compiled
even without using the directive |\childdocof|.
This method is deprecated because it is less robust
and there is no compelling reason to use it;
it is merely provided for backward compatibility
and it may be removed in future versions.

If the detection mechanism is to be used,
it is mandatory to correctly specify
the filename of the main file as the argument of |\childdocmain|:
%
\begin{center}
\begin{tabular}{l}
|\input{childdoc.def}|\\
|\childdocmain{|\textit{main}|}|\\
\end{tabular}
\end{center}
%
If |\jobname| does not match the argument \textit{main} of |\childdocmain|,
it is assumed that |\jobname| points to the child file to be compiled.
When using |\childdocmain| with the main file specified as argument,
it suffices to start a child file
with just |\input{|\textit{main}|}|
without loading of the package and using |\childdocof|.
If instead all processing is done
with the appropriate \textsf{childdoc} directives,
the argument of \textit{main} of |\childdocmain| can be empty.

An alternative version of the command line processing described
in \secref{sec:commandline} using the detection mechanism reads:
%
\begin{center}
|... -jobname "|\textit{target}|" "|[\textit{flags}]%
[|\def\jobname{|\textit{dest}|}|]|\input{|\textit{main}|}"|
\end{center}

%%%%%%%%%%%%%%%%%%%%%%%%%%%%%%%%%%%%%%%%%%%%%%%%%%%%%%%%%%%%%%%%%%%%%%%%%%%%%%%%
\subsection{Manual Code}
\label{sec:manual}

In case one cannot be certain whether the definitions file |childdoc.def|
is installed on the target \TeX{} distribution
and one prefers not to ship it,
it is conceivable to paste a few relevant commands into the sources.

To that end, drop all statements |\input{childdoc.def}|
and perform the replacements as outlined below.
Instead of |\childdocmain{|\textit{main}|}| add the following code
to the top of the main file:
%
\begin{center}
\begin{tabular}{l}
|\||ifdefined\childdocname\endinput\||fi\newif\ifchilddoc|\\
|\edef\childdocname{\scantokens\expandafter{\jobname\noexpand}}|\\
|\def\childdocmain{|\textit{main}|}\||ifx\childdocmain\childdocname\||else|\\
|\childdoctrue\includeonly{\childdocname}\let\jobname\childdocmain\||fi|\\
\end{tabular}
\end{center}
%
Instead of |\childdocof{|\textit{main}|}| just include the main file
at the top of each child file:
%
\begin{center}
|\input{|\textit{main}|}|
\end{center}
%
A simple redirection |\childdocforward{|\textit{dest}|}| is achieved by:
%
\begin{center}
|\def\jobname{|\textit{dest}|}\input{\jobname}|
\end{center}
%
The redirection with prefix
|\childdocforwardprefix[|\textit{prefix}|]{|\textit{dest}|}|
is accomplished by:
%
\begin{center}
\begin{tabular}{l}
|{\edef\jobname{\scantokens\expandafter{\jobname\noexpand}}|\\
|\def\redirectjob |\textit{prefix}|#1~~~{\gdef\jobname{|\textit{dest}|#1}}|\\
|\expandafter\redirectjob\jobname~~~}\input{\jobname}|
\end{tabular}
\end{center}

In an alternative approach,
child documents can be compiled by a specific command line
without additional code or specific definitions:
%
\begin{center}
|... -jobname "|\textit{target}|" "|[\textit{flags}]%
|\includeonly{|\textit{dest}|}\input{|\textit{main}|}"|
\end{center}
%

%%%%%%%%%%%%%%%%%%%%%%%%%%%%%%%%%%%%%%%%%%%%%%%%%%%%%%%%%%%%%%%%%%%%%%%%%%%%%%%%
%%%%%%%%%%%%%%%%%%%%%%%%%%%%%%%%%%%%%%%%%%%%%%%%%%%%%%%%%%%%%%%%%%%%%%%%%%%%%%%%
\section{Information}

%%%%%%%%%%%%%%%%%%%%%%%%%%%%%%%%%%%%%%%%%%%%%%%%%%%%%%%%%%%%%%%%%%%%%%%%%%%%%%%%
\subsection{Copyright}

Copyright \copyright{} 2017--2018 Niklas Beisert

This work may be distributed and/or modified under the
conditions of the \LaTeX{} Project Public License, either version 1.3
of this license or (at your option) any later version.
The latest version of this license is in
  \url{http://www.latex-project.org/lppl.txt}
and version 1.3 or later is part of all distributions of \LaTeX{}
version 2005/12/01 or later.

This work has the LPPL maintenance status `maintained'.

The Current Maintainer of this work is Niklas Beisert.

This work consists of the files |README.txt|, |childdoc.ins| and |childdoc.dtx|
as well as the derived files |childdoc.def|, |cdocsamp.tex|
with |cdocsch1.tex|, |cdocsch2.tex|, |cdocspt3.tex|, |cdocspt4.tex|,
|cdocsdrf.tex|, |cdocsfn1.tex|, |cdocsfn2.tex|
as well as |childdoc.pdf|.

%%%%%%%%%%%%%%%%%%%%%%%%%%%%%%%%%%%%%%%%%%%%%%%%%%%%%%%%%%%%%%%%%%%%%%%%%%%%%%%%
\subsection{Files and Installation}

The package consists of the files:
%
\begin{center}
\begin{tabular}{ll}
    |README.txt|   & readme file \\
    |childdoc.ins| & installation file \\
    |childdoc.dtx| & source file \\
    |childdoc.def| & definition file \\
    |cdocsamp.tex| & sample main file \\
    |cdocsch1.tex| & sample include file \\
    |cdocsch2.tex| & sample include file \\
    |cdocspt3.tex| & sample part file \\
    |cdocspt4.tex| & sample part file \\
    |cdocsdrf.tex| & sample redirection file \\
    |cdocsfn1.tex| & sample redirection file \\
    |cdocsfn2.tex| & sample redirection file \\
    |childdoc.pdf| & manual
\end{tabular}
\end{center}
%
The distribution consists of the files
|README.txt|, |childdoc.ins| and |childdoc.dtx|.
%
\begin{itemize}
\item
Run (pdf)\LaTeX{} on |childdoc.dtx|
to compile the manual |childdoc.pdf| (this file).
\item
Run \LaTeX{} on |childdoc.ins| to create the definitions file |childdoc.def|
and the sample |cdocsamp.tex| with include files
|cdocsch1.tex|, |cdocsch2.tex|, |cdocspt3.tex|, |cdocspt4.tex|,
|cdocsdrf.tex|, |cdocsfn1.tex|, |cdocsfn2.tex|.
Then copy the file |childdoc.def| to an appropriate directory of your \LaTeX{}
distribution, e.g.\ \textit{texmf-root}|/tex/latex/childdoc|.
\end{itemize}

%%%%%%%%%%%%%%%%%%%%%%%%%%%%%%%%%%%%%%%%%%%%%%%%%%%%%%%%%%%%%%%%%%%%%%%%%%%%%%%%
\subsection{Related CTAN Packages}

There are several other packages which offer a similar functionality:
%
\begin{itemize}
\item
The packages
\href{http://ctan.org/pkg/docmute}{\textsf{docmute}},
\href{http://ctan.org/pkg/includex}{\textsf{includex}} and
\href{http://ctan.org/pkg/standalone}{\textsf{standalone}}
provide commands to include only the document body of
a child file thus allowing both files to be compiled individually.
\item
The packages \href{http://ctan.org/pkg/subdocs}{\textsf{subdocs}}
and \href{http://ctan.org/pkg/subfiles}{\textsf{subfiles}}
provide structures in which the main and child documents can be
encapsulated and allowing them to be compiled individually.
The inclusion mechanism is different from the conventional |\include|.
\item
The package \href{http://ctan.org/pkg/combine}{\textsf{combine}}
is an elaborate solution to combine several documents into one.
\end{itemize}
%
See also the CTAN topic \href{http://ctan.org/topic/subdocs}{\textsf{subdocs}}
for further related packages.
The present package differs from the above solutions in that
a document structure constructed with the conventional |\include| mechanism
just needs two extra commands at the top of every file
such that all constituent files can be compiled individually.

%%%%%%%%%%%%%%%%%%%%%%%%%%%%%%%%%%%%%%%%%%%%%%%%%%%%%%%%%%%%%%%%%%%%%%%%%%%%%%%%
%\subsection{Feature Suggestions}
%
%The following is a list of features which may be useful for future
%versions of this package:
%%
%\begin{itemize}
%\item
%\ldots
%\end{itemize}

%%%%%%%%%%%%%%%%%%%%%%%%%%%%%%%%%%%%%%%%%%%%%%%%%%%%%%%%%%%%%%%%%%%%%%%%%%%%%%%%
\subsection{Revision History}

%%%%%%%%%%%%%%%%%%%%%%%%%%%%%%%%%%%%%%%%
\paragraph{v2.0:} 2018/12/30

\begin{itemize}
\item
immediate forward processing
\item
added |\childdocby| mechanism
\item
manual restructured
\end{itemize}

%%%%%%%%%%%%%%%%%%%%%%%%%%%%%%%%%%%%%%%%
\paragraph{v1.6:} 2018/01/17

\begin{itemize}
\item
application for development of include files
\item
corrections to manual
\end{itemize}

%%%%%%%%%%%%%%%%%%%%%%%%%%%%%%%%%%%%%%%%
\paragraph{v1.5:} 2017/05/21

\begin{itemize}
\item
more complete structuring introduced
\item
|\childdocof| introduced
\item
|\childdoc| renamed to |\childdocmain|
\item
|\childredirect| renamed to |\childdocforward| and |\childdocforwardprefix|
and functionality expanded
\end{itemize}

%%%%%%%%%%%%%%%%%%%%%%%%%%%%%%%%%%%%%%%%
\paragraph{v1.0:} 2017/04/27

\begin{itemize}
\item
manual and install package
\item
first version published on CTAN
\end{itemize}

%%%%%%%%%%%%%%%%%%%%%%%%%%%%%%%%%%%%%%%%
\paragraph{v0.6:} 2017/04/26

\begin{itemize}
\item
redirection mechanism added
\end{itemize}

%%%%%%%%%%%%%%%%%%%%%%%%%%%%%%%%%%%%%%%%
\paragraph{v0.5:} 2017/04/26

\begin{itemize}
\item
functionality in definition file
\end{itemize}


%%%%%%%%%%%%%%%%%%%%%%%%%%%%%%%%%%%%%%%%%%%%%%%%%%%%%%%%%%%%%%%%%%%%%%%%%%%%%%%%
%%%%%%%%%%%%%%%%%%%%%%%%%%%%%%%%%%%%%%%%%%%%%%%%%%%%%%%%%%%%%%%%%%%%%%%%%%%%%%%%
%%%%%%%%%%%%%%%%%%%%%%%%%%%%%%%%%%%%%%%%%%%%%%%%%%%%%%%%%%%%%%%%%%%%%%%%%%%%%%%%
\appendix

\settowidth\MacroIndent{\rmfamily\scriptsize 000\ }

 \DocInput{childdoc.dtx}

\end{document}
%</driver>
% \fi
%
% %%%%%%%%%%%%%%%%%%%%%%%%%%%%%%%%%%%%%%%%%%%%%%%%%%%%%%%%%%%%%%%%%%%%%%%%%%%%%%
% %%%%%%%%%%%%%%%%%%%%%%%%%%%%%%%%%%%%%%%%%%%%%%%%%%%%%%%%%%%%%%%%%%%%%%%%%%%%%%
% \section{Sample}
%\iffalse
%<*samplemain>
%\fi
%
% The following presents a sample document
% with two chapters, two parts, a title page,
% a compile flag as well as three forwarding files to set the flag.
% It consists of eight |.tex| files:
% \begin{center}
% \begin{tabular}{ll}
% |cdocsamp.tex|&main file\\
% |cdocsch1.tex|&include file for chapter 1\\
% |cdocsch2.tex|&include file for chapter 2\\
% |cdocspt3.tex|&include file for part 3\\
% |cdocspt4.tex|&include file for part 4\\
% |cdocsdrf.tex|&forwarding file for main file in draft mode\\
% |cdocsfi1.tex|&forwarding file for final version of chapter 1\\
% |cdocsfi2.tex|&forwarding file for final version of chapter 2\\
% \end{tabular}
% \end{center}
% Each of the eight files can be compiled directly by the \LaTeX{} compiler.
%
% %%%%%%%%%%%%%%%%%%%%%%%%%%%%%%%%%%%%%%
% \paragraph{Main File.}
%
% The main file is called |cdocsamp.tex|.
%
% Load the \textsf{childdoc} definitions and
% declare the filename for the main document:
%    \begin{macrocode}
\input{childdoc.def}
\childdocmain{}
%    \end{macrocode}

% Optional override for |\version| flag:
%    \begin{macrocode}
%%\ifchilddoc\else\providecommand{\version}{draft}\fi
%    \end{macrocode}

% Define the default values for the |\version| flag
% (|final| for the main file and |draft| for childs):
%    \begin{macrocode}
\ifchilddoc
\providecommand{\version}{draft}
\else
\providecommand{\version}{final}
\fi
%    \end{macrocode}

% Load the standard document class:
%    \begin{macrocode}
\documentclass[12pt]{article}
%    \end{macrocode}

% Start the document body:
%    \begin{macrocode}
\begin{document}
%    \end{macrocode}

% Declare a title page.
% Print title, part of document being processed and version flag:
%    \begin{macrocode}
\addtocounter{page}{-1}
\begin{center}
{\LARGE\bfseries{}childdoc example\par}
\vspace{1cm}
\ifchilddoc
\ifchilddocmanual part\else chapter\fi:
`\childdocname' of `\childdocjob'\par
\else
main document: `\childdocjob'\par
\fi
version: \version\par
\end{center}
\newpage
%    \end{macrocode}

% Manually include selected file,
% otherwise process as usual:
%    \begin{macrocode}
\ifchilddocmanual
\section*{part `\childdocname'}
\input{\childdocname}
\else
%    \end{macrocode}

% Include the two chapters:
%    \begin{macrocode}
\include{cdocsch1}
\include{cdocsch2}
%    \end{macrocode}

% Include the two parts unless only chapters should be displayed:
%    \begin{macrocode}
\ifchilddoc\else
\section{part three}
\input{cdocspt3}
\section{part four}
\input{cdocspt4}
\fi
%    \end{macrocode}

% Process as usual until here:
%    \begin{macrocode}
\fi
%    \end{macrocode}

% End of document body:
%    \begin{macrocode}
\end{document}
%    \end{macrocode}
%\iffalse
%</samplemain>
%\fi
%
% %%%%%%%%%%%%%%%%%%%%%%%%%%%%%%%%%%%%%%
% \paragraph{Chapter Include Files.}
%
% The include files are called |cdocsch1.tex| and |cdocsch2.tex|.
%
%\iffalse
%<*samplechap1|samplechap2>
%\fi

% Optional override for |\version| flag:
%    \begin{macrocode}
%%\providecommand{\version}{final}
%    \end{macrocode}

% Include the main document:
%    \begin{macrocode}
\input{childdoc.def}
\childdocof{cdocsamp}
%    \end{macrocode}

%\iffalse
%</samplechap1|samplechap2>
%\fi
%
%\iffalse
%<*samplechap1>
%\fi
% Some text for chapter 1:
%    \begin{macrocode}
\section{one}
some text in chapter one
%    \end{macrocode}

%\iffalse
%</samplechap1>
%\fi
% Some text for chapter 2:
%\iffalse
%<*samplechap2>
%\fi
%    \begin{macrocode}
\section{two}
more text in chapter two
%    \end{macrocode}

%\iffalse
%</samplechap2>
%\fi
%
% %%%%%%%%%%%%%%%%%%%%%%%%%%%%%%%%%%%%%%
% \paragraph{Part Include Files.}
%
% The include files are called |cdocspt3.tex| and |cdocspt4.tex|.
%
%\iffalse
%<*samplepart3|samplepart4>
%\fi

% Optional override for |\version| flag:
%    \begin{macrocode}
%%\providecommand{\version}{final}
%    \end{macrocode}

% Include the main document:
%    \begin{macrocode}
\input{childdoc.def}
\childdocby{cdocsamp}
%    \end{macrocode}

%\iffalse
%</samplepart3|samplepart4>
%\fi
%
%\iffalse
%<*samplepart3>
%\fi
% Some text for part 3:
%    \begin{macrocode}
some text in part three
%    \end{macrocode}

%\iffalse
%</samplepart3>
%\fi
% Some text for part 4:
%\iffalse
%<*samplepart4>
%\fi
%    \begin{macrocode}
more text in part four
%    \end{macrocode}

%\iffalse
%</samplepart4>
%\fi
%
% %%%%%%%%%%%%%%%%%%%%%%%%%%%%%%%%%%%%%%
% \paragraph{Forwarding for a Complete Draft.}
%
% The following forwarding file |cdocsdrf.tex|
% compiles the main document in draft mode:
%\iffalse
%<*sampledraft>
%\fi
%    \begin{macrocode}
\def\version{draft}
\input{childdoc.def}
\childdocforward{cdocsamp}
%    \end{macrocode}

%\iffalse
%</sampledraft>
%\fi
%
% %%%%%%%%%%%%%%%%%%%%%%%%%%%%%%%%%%%%%%
% \paragraph{Forwarding for Final Version of the Chapters.}
%
% The following forwarding files |cdocsfn1.tex| and |cdocsfn2.tex|
% (with identical content)
% compile the final versions of the child documents
% |cdocsch1.tex| and |cdocsch2.tex|, respectively:
%\iffalse
%<*samplefinal>
%\fi
%    \begin{macrocode}
\def\version{final}
\input{childdoc.def}
\childdocforwardprefix[cdocsamp]{cdocsfn}{cdocsch}
%    \end{macrocode}

%\iffalse
%</samplefinal>
%\fi
%
% %%%%%%%%%%%%%%%%%%%%%%%%%%%%%%%%%%%%%%
% \paragraph{Command Line Processing.}
%
% The following three command lines generate the output files
% |cdocscld|, |cdocscl1| and |cdocscl2|
% which should be identical to
% |cdocsdrf|, |cdocsch1| and |cdocsfn2|, respectively:
% \begin{center}
% \begin{tabular}{l}
% |latex -jobname cdocscld \|\\
% |  "\def\version{draft}\input{childdoc.def}\childdocforward{cdocsamp}"|\\
% |latex -jobname cdocscl1 \|\\
% |  "\input{childdoc.def}\childdocforward[cdocsamp]{cdocsch1}"|\\
% |latex -jobname cdocscl2 \|\\
% |  "\def\version{final}\input{childdoc.def}\childdocforward{cdocsch2}"|
% \end{tabular}
% \end{center}
% Note that the trailing backslash on each first line
% merely continues the input to the second line
% (for convenient cut ant paste).
% Furthermore, the command |latex| can be replaced by any
% of its alternative versions such as |pdflatex|.
%
% %%%%%%%%%%%%%%%%%%%%%%%%%%%%%%%%%%%%%%%%%%%%%%%%%%%%%%%%%%%%%%%%%%%%%%%%%%%%%%
% %%%%%%%%%%%%%%%%%%%%%%%%%%%%%%%%%%%%%%%%%%%%%%%%%%%%%%%%%%%%%%%%%%%%%%%%%%%%%%
% \section{Implementation}
%\iffalse
%<*package>
%\fi
%
% This section describes the definitions file |childdoc.def|.

% The definitions cannot be loaded using |\usepackage| or |\RequirePackage|
% which has a mechanism to prevent loading a style file more than once.
% When loading the definitions by means of |\input|
% multiple instances have to be prevented manually:
%\iffalse
%This code needs to be before the `\ProvidesFile' directive
%which is defined at the beginning of this file.
%Therefore it is also placed there and commented out here.
%</package>
%<*discard>
%\fi
%    \begin{macrocode}
\ifdefined\childdocmain\endinput\fi
%    \end{macrocode}
%\iffalse
%</discard>
%<*package>
%\fi
%
% \macro{\ifchilddoc}
% \macro{\ifchilddocmanual}
% The conditional |\ifchilddoc| tells whether a
% child (true) or main (false) document is being compiled.
% The conditional |\ifchilddocmanual| tells whether
% the |\includeonly| mechanism is used (false) or
% the selection of child files must be performed manually (true).
% The definitions initialise to false:
%    \begin{macrocode}
\newif\ifchilddoc
\newif\ifchilddocmanual
%    \end{macrocode}

% \macro{\childdocname}
% \macro{\childdocjob}
% The macro |\childdocname| stores the name of the main document
% to be compiled. The macro |\childdocjob| stores the name of
% the document on which the \LaTeX{} compiler was originally invoked.
% The content of |\jobname| cannot be compared
% to filenames specified in the source due to different catcodes.
% The following code rescans |\jobname|, stores the result
% in |\childdocname| and saves a copy in |\childdocjob|:
%    \begin{macrocode}
\edef\childdocname{\scantokens\expandafter{\jobname\noexpand}}
\let\childdocjob\childdocname
%    \end{macrocode}

% \macro{\childdocdisable}
% The macro |\childdocdisable| prevents the main file
% from being processed more than once.
% At this stage, the main document command |\childdocmain|
% is assumed to be called once again where it should do nothing.
% Any subsequent call to it should prevent
% a secondary processing of the main document
% It overwrites the forwarding commands
% |\childdocof| and |\childdocforward|
% with empty macros to prevent further inclusions of the main document:
%    \begin{macrocode}
\newcommand{\childdocdisable}
{
  \renewcommand{\childdocmain}[1]{\renewcommand{\childdocmain}[1]{\endinput}}
  \renewcommand{\childdocof}[1]{}
  \renewcommand{\childdocby}[2][]{}
  \renewcommand{\childdocforward}[2][]{}
  \renewcommand{\childdocdisable}{}
}
%    \end{macrocode}

% \macro{\childdocmain}
% The macro |\childdocmain| is to be called at the top of the main file
% with nothing or the main filename (without extension) as argument.
% First, it breaks loops.
% If the argument is not empty and does not match |\childdocname|
% (which is set by the first inclusion of |childdoc.def|),
% |\ifchilddoc| is set to true, |\includeonly| is applied to the child file
% and |\jobname| is set to the main file
% (for proper handling of |.aux| files):
%    \begin{macrocode}
\newcommand{\childdocmain}[1]
{
  \childdocdisable\childdocmain{}
  \if?#1?\else
    \begingroup
      \def\childdoctmp{#1}
      \ifx\childdoctmp\childdocname
        \def\childdoctmp{}
      \else
        \def\childdoctmp
        {
          \childdoctrue
          \includeonly{\childdocname}
          \def\childdocjob{#1}
          \def\jobname{#1}
        }
      \fi
      \expandafter
    \endgroup
    \childdoctmp
  \fi
}
%    \end{macrocode}

% \macro{\childdocof}
% The command |\childdocof| redirects
% compilation to the main file |#1|.
%    \begin{macrocode}
\newcommand{\childdocof}[1]
{
  \childdocdisable
  \childdoctrue
  \includeonly{\childdocname}
  \def\jobname{#1}
  \def\childdocjob{#1}
  \input{#1}
}
%    \end{macrocode}

% \macro{\childdocby}
% The command |\childdocby| ....
%    \begin{macrocode}
\newcommand{\childdocby}[2][]
{
  \childdocdisable
  \childdoctrue
  \childdocmanualtrue
  \if?#1?\else
    \def\jobname{#2}
  \fi
  \def\childdocjob{#2}
  \input{#2}
  \endinput
}
%    \end{macrocode}

% \macro{\childdocforward}
% The command |\childdocforward| redirects
% compilation to the main file or
% (if the optional argument is given) a child file.
% Parameters are set as if the main file
% or a child file starting with |\childdocof| was compiled.
% Then compilation is handed over to the main file:
%    \begin{macrocode}
\newcommand{\childdocforward}[2][]
{
  \begingroup
    \if?#1?
      \def\childdoctmp
      {
        \def\childdocname{#2}
        \def\childdocjob{#2}
        \def\jobname{#2}
        \input{#2}
        \endinput
      }
    \else
      \def\childdoctmp
      {
        \childdocdisable
        \def\childdocname{#2}
        \childdoctrue
        \includeonly{#2}
        \def\childdocjob{#1}
        \def\jobname{#1}
        \input{#1}
        \endinput
      }
    \fi
    \expandafter
  \endgroup
  \childdoctmp
}
%    \end{macrocode}

% \macro{\childdocforwardprefix}
% The command |\childdocforwardprefix| redirects
% compilation to the main or a child file by means of a pattern.
% The prefix |#1| in the current filename is replaced by |#2|
% and the suffix of the current filename is kept
% (it is assumed that the filename does not contain the substring `|~~~|'
% which is used as a delimiter).
% Compilation is handed over to the new file by |\childdocforward|:
%    \begin{macrocode}
\newcommand{\childdocforwardprefix}[3][]
{
  \begingroup
    \def\childdocextract #2##1~~~{\def\childdoctmp{\childdocforward[#1]{#3##1}}}
    \expandafter\childdocextract\childdocname~~~
    \expandafter
  \endgroup
  \childdoctmp
}
%    \end{macrocode}

% \macro{\childdoc}
% The deprecated macro |\childdoc| is a legacy version of |\childdocmain|:
%    \begin{macrocode}
\newcommand{\childdoc}{\childdocmain}
%    \end{macrocode}

% \macro{\childdocredirect}
% The deprecated macro |\childdocredirect| is a legacy version
% of |\childdocforward| and |\childdocforwardprefix|:
%    \begin{macrocode}
\newcommand{\childdocredirect}[2][]
{
  \begingroup
    \if?#1?
      \def\childdoctmp{\childdocforward{#2}}
    \else
      \def\childdoctmp{\childdocforwardprefix{#1}{#2}}
    \fi
    \expandafter
  \endgroup
  \childdoctmp
}
%    \end{macrocode}

%\iffalse
%</package>
%\fi
%
\endinput

\childdocof{cdocsamp}
%    \end{macrocode}

%\iffalse
%</samplechap1|samplechap2>
%\fi
%
%\iffalse
%<*samplechap1>
%\fi
% Some text for chapter 1:
%    \begin{macrocode}
\section{one}
some text in chapter one
%    \end{macrocode}

%\iffalse
%</samplechap1>
%\fi
% Some text for chapter 2:
%\iffalse
%<*samplechap2>
%\fi
%    \begin{macrocode}
\section{two}
more text in chapter two
%    \end{macrocode}

%\iffalse
%</samplechap2>
%\fi
%
% %%%%%%%%%%%%%%%%%%%%%%%%%%%%%%%%%%%%%%
% \paragraph{Part Include Files.}
%
% The include files are called |cdocspt3.tex| and |cdocspt4.tex|.
%
%\iffalse
%<*samplepart3|samplepart4>
%\fi

% Optional override for |\version| flag:
%    \begin{macrocode}
%%\providecommand{\version}{final}
%    \end{macrocode}

% Include the main document:
%    \begin{macrocode}
% \iffalse
%
% childdoc.dtx Copyright (C) 2017-2018 Niklas Beisert
%
% This work may be distributed and/or modified under the
% conditions of the LaTeX Project Public License, either version 1.3
% of this license or (at your option) any later version.
% The latest version of this license is in
%   http://www.latex-project.org/lppl.txt
% and version 1.3 or later is part of all distributions of LaTeX
% version 2005/12/01 or later.
%
% This work has the LPPL maintenance status `maintained'.
%
% The Current Maintainer of this work is Niklas Beisert.
%
% This work consists of the files childdoc.dtx and childdoc.ins
% and the derived files childdoc.def and cdocsamp.tex with
% cdocsch1.tex, cdocsch2.tex, cdocsdrf.tex, cdocsfn1.tex, cdocsfn2.tex.
%
%<package>\ifdefined\childdocmain\endinput\fi
%<package>\ProvidesFile{childdoc.def}[2018/12/30 v2.0 child document driver]
%<samplemain>\ProvidesFile{cdocsamp.tex}[2018/12/30 v2.0 sample for childdoc]
%<*driver>
%\ProvidesFile{childdoc.drv}[2018/12/30 v2.0 childdoc reference manual file]
\PassOptionsToClass{10pt,a4paper}{article}
\documentclass{ltxdoc}

\usepackage[margin=35mm]{geometry}
\usepackage{hyperref}
\usepackage{hyperxmp}
\usepackage[usenames]{color}

\hypersetup{colorlinks=true}
\hypersetup{pdfstartview=FitH}
\hypersetup{pdfpagemode=UseNone}
\hypersetup{pdfsource={}}
\hypersetup{pdflang={en-UK}}
\hypersetup{pdfcopyright={Copyright 2017-2018 Niklas Beisert.
  This work may be distributed and/or modified under the
  conditions of the LaTeX Project Public License, either version 1.3
  of this license or (at your option) any later version.}}
\hypersetup{pdflicenseurl={http://www.latex-project.org/lppl.txt}}
\hypersetup{pdfcontactaddress={ETH Zurich, ITP, HIT K,
  Wolfgang-Pauli-Strasse 27}}
\hypersetup{pdfcontactpostcode={8093}}
\hypersetup{pdfcontactcity={Zurich}}
\hypersetup{pdfcontactcountry={Switzerland}}
\hypersetup{pdfcontactemail={nbeisert@itp.phys.ethz.ch}}
\hypersetup{pdfcontacturl={http://people.phys.ethz.ch/\xmptilde nbeisert/}}

\newcommand{\secref}[1]{\hyperref[#1]{section \ref*{#1}}}

\parskip1ex
\parindent0pt
\let\olditemize\itemize
\def\itemize{\olditemize\parskip0pt}

\begin{document}

\title{The \textsf{childdoc} Package}
\hypersetup{pdftitle={The childdoc Package}}
\author{Niklas Beisert\\[2ex]
  Institut f\"ur Theoretische Physik\\
  Eidgen\"ossische Technische Hochschule Z\"urich\\
  Wolfgang-Pauli-Strasse 27, 8093 Z\"urich, Switzerland\\[1ex]
  \href{mailto:nbeisert@itp.phys.ethz.ch}
  {\texttt{nbeisert@itp.phys.ethz.ch}}}
\hypersetup{pdfauthor={Niklas Beisert}}
\hypersetup{pdfsubject={Manual for the LaTeX2e Package childdoc}}
\date{30 December 2018, \textsf{v2.0}}
\maketitle

\begin{abstract}\noindent
\textsf{childdoc} is a \LaTeXe{} package
that enables the direct compilation
of document sections included by |\include|
to individual files.
\end{abstract}

\begingroup
\parskip0ex
\tableofcontents
\endgroup

%%%%%%%%%%%%%%%%%%%%%%%%%%%%%%%%%%%%%%%%%%%%%%%%%%%%%%%%%%%%%%%%%%%%%%%%%%%%%%%%
%%%%%%%%%%%%%%%%%%%%%%%%%%%%%%%%%%%%%%%%%%%%%%%%%%%%%%%%%%%%%%%%%%%%%%%%%%%%%%%%
\section{Introduction}

\LaTeX{} provides a mechanism to structure a large document (such as a book)
into a main file and several child files (containing the chapters)
using the |\include| command.
This mechanism is beneficial for documents
which span hundreds of pages in order to
make the source file(s) more manageable.
Moreover, compilation can be restricted to
selected child files by means of the |\includeonly| command.
The latter feature can be used to reduce the compilation time while editing
(this was significantly more useful in the earlier days of \LaTeX{})
or to generate a smaller document which is easier to navigate.
Another application of |\includeonly| is to generate
documents consisting of selected parts of the complete document.

However, there are a few drawbacks of the plain |\include| mechanism:
\begin{itemize}
\item
The child files cannot be compiled on their own,
they can only be compiled via the main file.
A naive editing environment
(such as a text editor with an option
to have the current file processed by \LaTeX)
may require one to switch to the main file before compiling;
attempting to compile the child file produces errors.
\item
The main file must be modified (each time)
to adjust the |\includeonly| command
to the present needs. This easily leaves the main file in a messy state.
\item
The generated document will always carry the filename
of the main document. This is inconvenient if
several child files are to be compiled and
to be kept for distribution.
\end{itemize}

The present package provides a simple interface
to make child files individually compilable by \LaTeX{}.
Compiling a child file then has the same effect as compiling
the main file with an |\includeonly| command
to select the appropriate child.
Moreover the generated document will carry the name of the child
rather than the main file.
This resolves all three above issues.

This feature is meant to make the editing of books,
thesis documents and lecture notes somewhat more convenient.
However, the package can also be used efficiently for
composing a series of documents (such as exercise sheets)
which are typically distributed individually.
It then assists the author in generating the individual documents
(potentially in different versions)
as well as a document containing the collected series.
Another application is in developing style files
or other kinds of included material
where compilation of the style file could redirect
to a sample or test file.

%%%%%%%%%%%%%%%%%%%%%%%%%%%%%%%%%%%%%%%%%%%%%%%%%%%%%%%%%%%%%%%%%%%%%%%%%%%%%%%%
%%%%%%%%%%%%%%%%%%%%%%%%%%%%%%%%%%%%%%%%%%%%%%%%%%%%%%%%%%%%%%%%%%%%%%%%%%%%%%%%
\section{Usage}

First of all, the package \textsf{childdoc} is \emph{not} a standard
\LaTeXe{} |.sty| style file! Therefore it needs to be invoked in
a non-standard way.

%%%%%%%%%%%%%%%%%%%%%%%%%%%%%%%%%%%%%%%%%%%%%%%%%%%%%%%%%%%%%%%%%%%%%%%%%%%%%%%%
\subsection{Included Files}
\label{sec:include}

%%%%%%%%%%%%%%%%%%%%%%%%%%%%%%%%%%%%%%%%
\DescribeMacro{\childdocmain}
To use the package, add the commands
\begin{center}
\begin{tabular}{l}
|\input{childdoc.def}|\\
|\childdocmain{}|\\
\end{tabular}
\end{center}
at the very top of the main \LaTeX{} file,
in particular \emph{before} the |\documentclass| statement!
The argument of |\childdocmain| should be left empty
(but it must be present).

%%%%%%%%%%%%%%%%%%%%%%%%%%%%%%%%%%%%%%%%
\DescribeMacro{\childdocof}
Furthermore, add the commands
\begin{center}
\begin{tabular}{l}
|\input{childdoc.def}|\\
|\childdocof{|\textit{main}|}|\\
\end{tabular}
\end{center}
at the top of every child file \textit{child}
which is included by |\include{|\textit{child}|}|
from within the main file
(or at least for those files to be compiled individually).
The argument \textit{main} must be the filename of the main file.

There are a couple of
considerations in setting up the main and child documents:

%%%%%%%%%%%%%%%%%%%%%%%%%%%%%%%%%%%%%%%%
\paragraph{Restrictions.}

Please note the following restrictions:
\begin{itemize}
\item
|\childdocmain| must be called with one argument \textit{main}
to ensure compatibility with earlier version of the package.
It must either be empty (|\childdocmain{}|)
or precisely match the filename of the main file in which it is specified.
See \secref{sec:detection} for further information.
\item
The filename \textit{main} must be specified without the |.tex| extension.
\item
The filename \textit{main} is case sensitive
(even in case-insensitive file systems)
due to internal string comparison.
\item
The argument \textit{main} should be fully expanded, it cannot be a macro.
\item
Subdirectories and special characters should be avoided in filenames.
\item
The command |\childdocmain{|\textit{main}|}| must be followed by a whitespace.
It should not be followed immediately by another command
or by a comment mark `|%|'.
This is because the \TeX{} parser reads the token immediately following
the argument of |\childdocmain| and puts it
at the beginning of every child section;
however, a white\-space is ignored.
\end{itemize}

%%%%%%%%%%%%%%%%%%%%%%%%%%%%%%%%%%%%%%%%
\paragraph{Content of Main File.}

It is advisable to place all content in the child files included by |\include|.
Any output contained in the main file will appear in all child documents
unless suppressed manually;
it cannot be suppressed automatically by the |\includeonly| directive
and thus should normally be avoided.
A method to include some content in the main file
by means of conditional processing is described in \secref{sec:conditional}.

%%%%%%%%%%%%%%%%%%%%%%%%%%%%%%%%%%%%%%%%
\paragraph{Page Numbering.}

When only a part of the document is compiled,
the appropriate numbering of pages
(as well as other status parameters)
is determined from the |.aux| files.
The latter contain information from previous passes.
However this information needs to propagate through
all intermediate child documents.
Therefore the page numbering in child documents may well
be inconsistent until the complete document is compiled at least once.

A useful (if unconventional) way to always ensure a consistent
page numbering is to restart the numbering in each child document
and denote the pages by `\textit{child}|.|\textit{page}'
where \textit{child} represents the chapter/section number of the child file.
This can be achieved by the command
|\numberwithin{page}{|\textit{child}|}|
of the \textsf{amsmath} package
where \textit{child} can be |chapter| or |section|
depending on the chosen structuring.
Alternatively, one can modify the macro |\thepage| appropriately
and reset the counter |page| at the start of each child file.

%%%%%%%%%%%%%%%%%%%%%%%%%%%%%%%%%%%%%%%%%%%%%%%%%%%%%%%%%%%%%%%%%%%%%%%%%%%%%%%%
\subsection{Conditional Processing}
\label{sec:conditional}

The package provides a mechanism to compile different versions
of a document. To customise the versions further some conditional processing
can come in handy to distinguish which version is being compiled.
The package provides two macros to describe the compilation context:

%%%%%%%%%%%%%%%%%%%%%%%%%%%%%%%%%%%%%%%%
\DescribeMacro{\ifchilddoc}
The conditional |\ifchilddoc| distinguishes between the compilation of
child documents and the main document:
%
\begin{center}
|\ifchilddoc |\textit{child-code}| |[|\||else |\textit{main-code}]| \||fi|
\end{center}

%%%%%%%%%%%%%%%%%%%%%%%%%%%%%%%%%%%%%%%%
\DescribeMacro{\childdocname}
\DescribeMacro{\childdocjob}
The macro |\childdocname| contains the filename (without extension)
of the main or child file being processed.
Note that |\childdocjob| will always contain the name of the main file.

%%%%%%%%%%%%%%%%%%%%%%%%%%%%%%%%%%%%%%%%
\paragraph{Title Page.}

Conditional processing can be used to include a title or banner page
in the main document when proper precautions are taken.
Importantly, the code in the main file should ensure that the page counter
(as well as other status parameters which are stored in the |.aux| files)
takes the same value after the conditional processing.
Otherwise the page numbers may take divergent values
depending on which part is compiled.

For example, a title page could be declared by:
%
\begin{center}
\begin{tabular}{l}
|\ifchilddoc\||else|\\
|\addtocounter{page}{-1}|\\
\textit{code for title page}\\
|\newpage|\\
|\||fi|
\end{tabular}
\end{center}
%
A banner page for the child documents can be generated by:
%
\begin{center}
\begin{tabular}{l}
|\ifchilddoc|\\
|\addtocounter{page}{-1}|\\
\textit{code for banner page}\\
|\newpage|\\
|\||fi|
\end{tabular}
\end{center}
%
Here one could write a message such as:
\begin{center}
|This is the part \childdocname{} of \childdocjob{}.|
\end{center}

%%%%%%%%%%%%%%%%%%%%%%%%%%%%%%%%%%%%%%%%%%%%%%%%%%%%%%%%%%%%%%%%%%%%%%%%%%%%%%%%
\subsection{Flags}
\label{sec:flags}

The package makes it easy to generate different versions
of the main or child documents.
To this end compilation flags can be defined
and assigned different default values.
They will be particularly useful in conjunction
with the forwarding mechanism described in \secref{sec:forward}.

For example, it may be useful to have a flag |\version|
which can be set to |draft| or |final|.
The document source will contain some conditional code
depending on the value of |\version|.
Suppose further, the flag should default to |final| for the main file
and to |draft| for child files
which is a natural assignment for editing the document.
This is achieved by placing the following code
in the preamble of the main document
(below the |\childdocmain| directive):
%
\begin{center}
\begin{tabular}{l}
|\ifchilddoc|\\
|\providecommand{\version}{draft}|\\
|\||else|\\
|\providecommand{\version}{final}|\\
|\||fi|
\end{tabular}
\end{center}
%
The definition by |\providecommand| makes sure
that previous definitions are not overwritten.
Further statements |\providecommand{\version}{...}|
can thus be added before the above code to override it.

For the main file, one might add a line
(between |\childdocmain| and the above block)
%
\begin{center}
|%\ifchilddoc\||else\providecommand{\version}{draft}\||fi|
\end{center}
%
which can be uncommented to produce a draft version.
Likewise one can add a line to the very top of a child file
(above the |\childdocof{|\textit{main}|}| directive)
%
\begin{center}
|%\providecommand{\version}{final}|
\end{center}
%
which can be uncommented to produce the final version of this child document.

%%%%%%%%%%%%%%%%%%%%%%%%%%%%%%%%%%%%%%%%%%%%%%%%%%%%%%%%%%%%%%%%%%%%%%%%%%%%%%%%
\subsection{Forwarding}
\label{sec:forward}

Different versions of the main or child documents
using compilation flags as described in \secref{sec:flags}
can be (permanently) stored in different files
for convenient compilation, viewing and distribution.
To this end, the package defines a command
to pass on compilation to a different file:

%%%%%%%%%%%%%%%%%%%%%%%%%%%%%%%%%%%%%%%%
\DescribeMacro{\childdocforward}
The command |\childdocforward| redirects processing to
another source file:
%
\begin{center}
\begin{tabular}{l}
|\input{childdoc.def}|\\
|\childdocforward[|\textit{main}|]{|\textit{dest}|}|\\
\end{tabular}
\end{center}
%
The argument \textit{dest} is the destination file
(without extension).
It should be the main file or one of the child files.
Note that further \textsf{childdoc} directives
such as |\childdocof| and |\childdocforward|
in the indicated file will be processed in this form.
The optional argument \textit{main}
passes on directly to the main file \textit{main}
while pretending to compile the child \textit{dest}.
This form behaves as if \textit{dest}
issues |\childdocof{|\textit{main}|}| right away,
and no further \textsf{childdoc} directives will be processed.

%%%%%%%%%%%%%%%%%%%%%%%%%%%%%%%%%%%%%%%%
\DescribeMacro{\...prefix}
In the alternative form |\childdocforwardprefix|,
%
\begin{center}
\begin{tabular}{l}
|\input{childdoc.def}|\\
|\childdocforwardprefix[|\textit{main}|]{|\textit{prefix}|}{|\textit{dest}|}|
\end{tabular}
\end{center}
%
the destination file is determined by a pattern
depending on the current file:
To make this work, the current file must be called
`{\textit{prefix}\hspace{0.2em}\textit{suffix}}'
with \textit{prefix} matching precisely the argument.
Processing is then passed on to the file
`{\textit{dest}\hspace{0.2em}\textit{suffix}}'.
Surely, the same effect is achieved by
directly specifying the
argument `{\textit{dest}\hspace{0.2em}\textit{suffix}}'
in the first form.
However, that requires to set up a different file
for each child. With the alternative form of the command
all these files can have exactly the same content
which simplifies setting them up and maintaining them.

For example, the following file |draft.tex|
with a compilation flag |\version| as described in \secref{sec:flags}
compiles the main document as a draft:
%
\begin{center}
\begin{tabular}{l}
|\def\version{draft}|\\
|\input{childdoc.def}|\\
|\childdocforward{|\textit{main}|}|
\end{tabular}
\end{center}
%
Likewise, the following files |final|\textit{nn}|.tex|
compile the final version of the child document
|child|\textit{nn}|.tex|:
%
\begin{center}
\begin{tabular}{l}
|\def\version{final}|\\
|\input{childdoc.def}|\\
|\childdocforwardprefix{final}{child}|
\end{tabular}
\end{center}
%

Note that when several versions of a main file and/or of each child file
are to be generated, it may be convenient to set up a |Makefile| or
shell script to automatise the process.

%%%%%%%%%%%%%%%%%%%%%%%%%%%%%%%%%%%%%%%%%%%%%%%%%%%%%%%%%%%%%%%%%%%%%%%%%%%%%%%%
\subsection{Command Line Processing}
\label{sec:commandline}

The effect of redirection files can also be achieved by invoking
the \LaTeX{} compiler with a more elaborate command line.
Most conveniently this should be done as part
of a shell script or a |Makefile|.

When using \textsf{childdoc} in the main file, the following
command lines effectively perform a redirection
(note that depending on the shell being used,
backslashes may have to be doubled: `|\|' $\to$ `|\\|'):
%
\begin{center}
|... -jobname "|\textit{target}|" |\\|"|[\textit{flags}]%
|\input{childdoc.def}\childdocforward[|\textit{main}|]{|\textit{dest}|}"|
\end{center}
%
Here \textit{target} is the name of the output file,
\textit{main} is the name of the main file
and \textit{dest} is the name of the main or child file to be processed
(all filenames without extensions).
The optional argument \textit{main} can be omitted
if \textit{main} matches \textit{dest}.
Optionally, compilation \textit{flags} can be defined via |\def| commands.
This command line makes the \TeX{} engine believe
it is compiling the file \textit{target}
whose content is specified as the latter parameter.
The provided code then forwards the processing to
\textit{main} or \textit{dest} as described in \secref{sec:forward}.

%%%%%%%%%%%%%%%%%%%%%%%%%%%%%%%%%%%%%%%%%%%%%%%%%%%%%%%%%%%%%%%%%%%%%%%%%%%%%%%%
\subsection{Include by Input}
\label{sec:input}

Including child documents by |\include| has some restrictions by design.
Most notably, the content of a child document always occupies
its own set of pages; pages cannot be shared between child documents.
Usually, this behaviour makes perfect sense
because each child document contain an essential part of the document.
However, in some situations it may be desirable to compose
a document from a collection of parts
without having mandatory page breaks between then.
For this case, the package
provides a mechanism to include parts
by |\input| which can also be processed individually.
However, by construction this mechanism
requires manual handling of the content to be output.

%%%%%%%%%%%%%%%%%%%%%%%%%%%%%%%%%%%%%%%%
\DescribeMacro{\ifchilddocmanual}
The main file should be prepared as usual, see \secref{sec:include}.
However, the document body must make a distinction
between processing of an individual part and of the main document, e.g.:
%
\begin{center}
\begin{tabular}{l}
|\ifchilddocmanual|\\
|\input{\childdocname}|\\
|\||else|\\
\textit{document body with }|\input{|\textit{part}|}|\\
|\||fi|
\end{tabular}
\end{center}
%
The conditional |\ifchilddocmanual| is true whenever
a part to be included by |\input| is being compiled,
and the name of the part is stored in |\childdocname|.

%%%%%%%%%%%%%%%%%%%%%%%%%%%%%%%%%%%%%%%%
\DescribeMacro{\childdocby}
Each part to be included by |\input| should start with:
%
\begin{center}
\begin{tabular}{l}
|\input{childdoc.def}|\\
|\childdocby{|\textit{main}|}|\\
\end{tabular}
\end{center}
%
The directive |\childdocby| is similar to |\childdocof|
described in \secref{sec:include},
but the subsequent selection of content must be done manually.
To that end, both |\ifchilddoc| and |\ifchilddocmanual|
will be true upon processing of a part,
and the name of the part is stored in |\childdocname|.
Note that |\jobname| will be set to the filename of the current part
so that each part receives an individual |.aux| file
that does not interfere with the |.aux| file(s) of the main document.
This behaviour can be altered by the alternative form
|\childdocby[*]{|\textit{main}|}| (with a non-empty optional argument)
which uses the |.aux| file of the main document
by setting |\jobname| to \textit{main}.

%%%%%%%%%%%%%%%%%%%%%%%%%%%%%%%%%%%%%%%%%%%%%%%%%%%%%%%%%%%%%%%%%%%%%%%%%%%%%%%%
\subsection{Driver Development}
\label{sec:driver}

The \textsf{childdoc} mechanism can also be use for the development
of definition files such as \LaTeX{} styles or classes.
This case differs from the above setup with multiple parts
included by |\include| in that no |\includeonly| should be invoked.
This can be achieved by starting the include file
(before |\ProvidesPackage|) with:
%
\begin{center}
\begin{tabular}{l}
|\input{childdoc.def}|\\
|\childdocforward{|\textit{main}|}|\\
\end{tabular}
\end{center}
%
or alternatively with:
%
\begin{center}
\begin{tabular}{l}
|\input{childdoc.def}|\\
|\childdocby{|\textit{main}|}|\\
\end{tabular}
\end{center}
%
Both forms have slightly different effects as described above.
The main file is prepared as usual, see \secref{sec:include}.

%%%%%%%%%%%%%%%%%%%%%%%%%%%%%%%%%%%%%%%%%%%%%%%%%%%%%%%%%%%%%%%%%%%%%%%%%%%%%%%%
\subsection{Legacy Detection}
\label{sec:detection}

The directive |\childdocmain| in the main file can detect
whether the complete document or merely a child is to be compiled
even without using the directive |\childdocof|.
This method is deprecated because it is less robust
and there is no compelling reason to use it;
it is merely provided for backward compatibility
and it may be removed in future versions.

If the detection mechanism is to be used,
it is mandatory to correctly specify
the filename of the main file as the argument of |\childdocmain|:
%
\begin{center}
\begin{tabular}{l}
|\input{childdoc.def}|\\
|\childdocmain{|\textit{main}|}|\\
\end{tabular}
\end{center}
%
If |\jobname| does not match the argument \textit{main} of |\childdocmain|,
it is assumed that |\jobname| points to the child file to be compiled.
When using |\childdocmain| with the main file specified as argument,
it suffices to start a child file
with just |\input{|\textit{main}|}|
without loading of the package and using |\childdocof|.
If instead all processing is done
with the appropriate \textsf{childdoc} directives,
the argument of \textit{main} of |\childdocmain| can be empty.

An alternative version of the command line processing described
in \secref{sec:commandline} using the detection mechanism reads:
%
\begin{center}
|... -jobname "|\textit{target}|" "|[\textit{flags}]%
[|\def\jobname{|\textit{dest}|}|]|\input{|\textit{main}|}"|
\end{center}

%%%%%%%%%%%%%%%%%%%%%%%%%%%%%%%%%%%%%%%%%%%%%%%%%%%%%%%%%%%%%%%%%%%%%%%%%%%%%%%%
\subsection{Manual Code}
\label{sec:manual}

In case one cannot be certain whether the definitions file |childdoc.def|
is installed on the target \TeX{} distribution
and one prefers not to ship it,
it is conceivable to paste a few relevant commands into the sources.

To that end, drop all statements |\input{childdoc.def}|
and perform the replacements as outlined below.
Instead of |\childdocmain{|\textit{main}|}| add the following code
to the top of the main file:
%
\begin{center}
\begin{tabular}{l}
|\||ifdefined\childdocname\endinput\||fi\newif\ifchilddoc|\\
|\edef\childdocname{\scantokens\expandafter{\jobname\noexpand}}|\\
|\def\childdocmain{|\textit{main}|}\||ifx\childdocmain\childdocname\||else|\\
|\childdoctrue\includeonly{\childdocname}\let\jobname\childdocmain\||fi|\\
\end{tabular}
\end{center}
%
Instead of |\childdocof{|\textit{main}|}| just include the main file
at the top of each child file:
%
\begin{center}
|\input{|\textit{main}|}|
\end{center}
%
A simple redirection |\childdocforward{|\textit{dest}|}| is achieved by:
%
\begin{center}
|\def\jobname{|\textit{dest}|}\input{\jobname}|
\end{center}
%
The redirection with prefix
|\childdocforwardprefix[|\textit{prefix}|]{|\textit{dest}|}|
is accomplished by:
%
\begin{center}
\begin{tabular}{l}
|{\edef\jobname{\scantokens\expandafter{\jobname\noexpand}}|\\
|\def\redirectjob |\textit{prefix}|#1~~~{\gdef\jobname{|\textit{dest}|#1}}|\\
|\expandafter\redirectjob\jobname~~~}\input{\jobname}|
\end{tabular}
\end{center}

In an alternative approach,
child documents can be compiled by a specific command line
without additional code or specific definitions:
%
\begin{center}
|... -jobname "|\textit{target}|" "|[\textit{flags}]%
|\includeonly{|\textit{dest}|}\input{|\textit{main}|}"|
\end{center}
%

%%%%%%%%%%%%%%%%%%%%%%%%%%%%%%%%%%%%%%%%%%%%%%%%%%%%%%%%%%%%%%%%%%%%%%%%%%%%%%%%
%%%%%%%%%%%%%%%%%%%%%%%%%%%%%%%%%%%%%%%%%%%%%%%%%%%%%%%%%%%%%%%%%%%%%%%%%%%%%%%%
\section{Information}

%%%%%%%%%%%%%%%%%%%%%%%%%%%%%%%%%%%%%%%%%%%%%%%%%%%%%%%%%%%%%%%%%%%%%%%%%%%%%%%%
\subsection{Copyright}

Copyright \copyright{} 2017--2018 Niklas Beisert

This work may be distributed and/or modified under the
conditions of the \LaTeX{} Project Public License, either version 1.3
of this license or (at your option) any later version.
The latest version of this license is in
  \url{http://www.latex-project.org/lppl.txt}
and version 1.3 or later is part of all distributions of \LaTeX{}
version 2005/12/01 or later.

This work has the LPPL maintenance status `maintained'.

The Current Maintainer of this work is Niklas Beisert.

This work consists of the files |README.txt|, |childdoc.ins| and |childdoc.dtx|
as well as the derived files |childdoc.def|, |cdocsamp.tex|
with |cdocsch1.tex|, |cdocsch2.tex|, |cdocspt3.tex|, |cdocspt4.tex|,
|cdocsdrf.tex|, |cdocsfn1.tex|, |cdocsfn2.tex|
as well as |childdoc.pdf|.

%%%%%%%%%%%%%%%%%%%%%%%%%%%%%%%%%%%%%%%%%%%%%%%%%%%%%%%%%%%%%%%%%%%%%%%%%%%%%%%%
\subsection{Files and Installation}

The package consists of the files:
%
\begin{center}
\begin{tabular}{ll}
    |README.txt|   & readme file \\
    |childdoc.ins| & installation file \\
    |childdoc.dtx| & source file \\
    |childdoc.def| & definition file \\
    |cdocsamp.tex| & sample main file \\
    |cdocsch1.tex| & sample include file \\
    |cdocsch2.tex| & sample include file \\
    |cdocspt3.tex| & sample part file \\
    |cdocspt4.tex| & sample part file \\
    |cdocsdrf.tex| & sample redirection file \\
    |cdocsfn1.tex| & sample redirection file \\
    |cdocsfn2.tex| & sample redirection file \\
    |childdoc.pdf| & manual
\end{tabular}
\end{center}
%
The distribution consists of the files
|README.txt|, |childdoc.ins| and |childdoc.dtx|.
%
\begin{itemize}
\item
Run (pdf)\LaTeX{} on |childdoc.dtx|
to compile the manual |childdoc.pdf| (this file).
\item
Run \LaTeX{} on |childdoc.ins| to create the definitions file |childdoc.def|
and the sample |cdocsamp.tex| with include files
|cdocsch1.tex|, |cdocsch2.tex|, |cdocspt3.tex|, |cdocspt4.tex|,
|cdocsdrf.tex|, |cdocsfn1.tex|, |cdocsfn2.tex|.
Then copy the file |childdoc.def| to an appropriate directory of your \LaTeX{}
distribution, e.g.\ \textit{texmf-root}|/tex/latex/childdoc|.
\end{itemize}

%%%%%%%%%%%%%%%%%%%%%%%%%%%%%%%%%%%%%%%%%%%%%%%%%%%%%%%%%%%%%%%%%%%%%%%%%%%%%%%%
\subsection{Related CTAN Packages}

There are several other packages which offer a similar functionality:
%
\begin{itemize}
\item
The packages
\href{http://ctan.org/pkg/docmute}{\textsf{docmute}},
\href{http://ctan.org/pkg/includex}{\textsf{includex}} and
\href{http://ctan.org/pkg/standalone}{\textsf{standalone}}
provide commands to include only the document body of
a child file thus allowing both files to be compiled individually.
\item
The packages \href{http://ctan.org/pkg/subdocs}{\textsf{subdocs}}
and \href{http://ctan.org/pkg/subfiles}{\textsf{subfiles}}
provide structures in which the main and child documents can be
encapsulated and allowing them to be compiled individually.
The inclusion mechanism is different from the conventional |\include|.
\item
The package \href{http://ctan.org/pkg/combine}{\textsf{combine}}
is an elaborate solution to combine several documents into one.
\end{itemize}
%
See also the CTAN topic \href{http://ctan.org/topic/subdocs}{\textsf{subdocs}}
for further related packages.
The present package differs from the above solutions in that
a document structure constructed with the conventional |\include| mechanism
just needs two extra commands at the top of every file
such that all constituent files can be compiled individually.

%%%%%%%%%%%%%%%%%%%%%%%%%%%%%%%%%%%%%%%%%%%%%%%%%%%%%%%%%%%%%%%%%%%%%%%%%%%%%%%%
%\subsection{Feature Suggestions}
%
%The following is a list of features which may be useful for future
%versions of this package:
%%
%\begin{itemize}
%\item
%\ldots
%\end{itemize}

%%%%%%%%%%%%%%%%%%%%%%%%%%%%%%%%%%%%%%%%%%%%%%%%%%%%%%%%%%%%%%%%%%%%%%%%%%%%%%%%
\subsection{Revision History}

%%%%%%%%%%%%%%%%%%%%%%%%%%%%%%%%%%%%%%%%
\paragraph{v2.0:} 2018/12/30

\begin{itemize}
\item
immediate forward processing
\item
added |\childdocby| mechanism
\item
manual restructured
\end{itemize}

%%%%%%%%%%%%%%%%%%%%%%%%%%%%%%%%%%%%%%%%
\paragraph{v1.6:} 2018/01/17

\begin{itemize}
\item
application for development of include files
\item
corrections to manual
\end{itemize}

%%%%%%%%%%%%%%%%%%%%%%%%%%%%%%%%%%%%%%%%
\paragraph{v1.5:} 2017/05/21

\begin{itemize}
\item
more complete structuring introduced
\item
|\childdocof| introduced
\item
|\childdoc| renamed to |\childdocmain|
\item
|\childredirect| renamed to |\childdocforward| and |\childdocforwardprefix|
and functionality expanded
\end{itemize}

%%%%%%%%%%%%%%%%%%%%%%%%%%%%%%%%%%%%%%%%
\paragraph{v1.0:} 2017/04/27

\begin{itemize}
\item
manual and install package
\item
first version published on CTAN
\end{itemize}

%%%%%%%%%%%%%%%%%%%%%%%%%%%%%%%%%%%%%%%%
\paragraph{v0.6:} 2017/04/26

\begin{itemize}
\item
redirection mechanism added
\end{itemize}

%%%%%%%%%%%%%%%%%%%%%%%%%%%%%%%%%%%%%%%%
\paragraph{v0.5:} 2017/04/26

\begin{itemize}
\item
functionality in definition file
\end{itemize}


%%%%%%%%%%%%%%%%%%%%%%%%%%%%%%%%%%%%%%%%%%%%%%%%%%%%%%%%%%%%%%%%%%%%%%%%%%%%%%%%
%%%%%%%%%%%%%%%%%%%%%%%%%%%%%%%%%%%%%%%%%%%%%%%%%%%%%%%%%%%%%%%%%%%%%%%%%%%%%%%%
%%%%%%%%%%%%%%%%%%%%%%%%%%%%%%%%%%%%%%%%%%%%%%%%%%%%%%%%%%%%%%%%%%%%%%%%%%%%%%%%
\appendix

\settowidth\MacroIndent{\rmfamily\scriptsize 000\ }

 \DocInput{childdoc.dtx}

\end{document}
%</driver>
% \fi
%
% %%%%%%%%%%%%%%%%%%%%%%%%%%%%%%%%%%%%%%%%%%%%%%%%%%%%%%%%%%%%%%%%%%%%%%%%%%%%%%
% %%%%%%%%%%%%%%%%%%%%%%%%%%%%%%%%%%%%%%%%%%%%%%%%%%%%%%%%%%%%%%%%%%%%%%%%%%%%%%
% \section{Sample}
%\iffalse
%<*samplemain>
%\fi
%
% The following presents a sample document
% with two chapters, two parts, a title page,
% a compile flag as well as three forwarding files to set the flag.
% It consists of eight |.tex| files:
% \begin{center}
% \begin{tabular}{ll}
% |cdocsamp.tex|&main file\\
% |cdocsch1.tex|&include file for chapter 1\\
% |cdocsch2.tex|&include file for chapter 2\\
% |cdocspt3.tex|&include file for part 3\\
% |cdocspt4.tex|&include file for part 4\\
% |cdocsdrf.tex|&forwarding file for main file in draft mode\\
% |cdocsfi1.tex|&forwarding file for final version of chapter 1\\
% |cdocsfi2.tex|&forwarding file for final version of chapter 2\\
% \end{tabular}
% \end{center}
% Each of the eight files can be compiled directly by the \LaTeX{} compiler.
%
% %%%%%%%%%%%%%%%%%%%%%%%%%%%%%%%%%%%%%%
% \paragraph{Main File.}
%
% The main file is called |cdocsamp.tex|.
%
% Load the \textsf{childdoc} definitions and
% declare the filename for the main document:
%    \begin{macrocode}
\input{childdoc.def}
\childdocmain{}
%    \end{macrocode}

% Optional override for |\version| flag:
%    \begin{macrocode}
%%\ifchilddoc\else\providecommand{\version}{draft}\fi
%    \end{macrocode}

% Define the default values for the |\version| flag
% (|final| for the main file and |draft| for childs):
%    \begin{macrocode}
\ifchilddoc
\providecommand{\version}{draft}
\else
\providecommand{\version}{final}
\fi
%    \end{macrocode}

% Load the standard document class:
%    \begin{macrocode}
\documentclass[12pt]{article}
%    \end{macrocode}

% Start the document body:
%    \begin{macrocode}
\begin{document}
%    \end{macrocode}

% Declare a title page.
% Print title, part of document being processed and version flag:
%    \begin{macrocode}
\addtocounter{page}{-1}
\begin{center}
{\LARGE\bfseries{}childdoc example\par}
\vspace{1cm}
\ifchilddoc
\ifchilddocmanual part\else chapter\fi:
`\childdocname' of `\childdocjob'\par
\else
main document: `\childdocjob'\par
\fi
version: \version\par
\end{center}
\newpage
%    \end{macrocode}

% Manually include selected file,
% otherwise process as usual:
%    \begin{macrocode}
\ifchilddocmanual
\section*{part `\childdocname'}
\input{\childdocname}
\else
%    \end{macrocode}

% Include the two chapters:
%    \begin{macrocode}
\include{cdocsch1}
\include{cdocsch2}
%    \end{macrocode}

% Include the two parts unless only chapters should be displayed:
%    \begin{macrocode}
\ifchilddoc\else
\section{part three}
\input{cdocspt3}
\section{part four}
\input{cdocspt4}
\fi
%    \end{macrocode}

% Process as usual until here:
%    \begin{macrocode}
\fi
%    \end{macrocode}

% End of document body:
%    \begin{macrocode}
\end{document}
%    \end{macrocode}
%\iffalse
%</samplemain>
%\fi
%
% %%%%%%%%%%%%%%%%%%%%%%%%%%%%%%%%%%%%%%
% \paragraph{Chapter Include Files.}
%
% The include files are called |cdocsch1.tex| and |cdocsch2.tex|.
%
%\iffalse
%<*samplechap1|samplechap2>
%\fi

% Optional override for |\version| flag:
%    \begin{macrocode}
%%\providecommand{\version}{final}
%    \end{macrocode}

% Include the main document:
%    \begin{macrocode}
\input{childdoc.def}
\childdocof{cdocsamp}
%    \end{macrocode}

%\iffalse
%</samplechap1|samplechap2>
%\fi
%
%\iffalse
%<*samplechap1>
%\fi
% Some text for chapter 1:
%    \begin{macrocode}
\section{one}
some text in chapter one
%    \end{macrocode}

%\iffalse
%</samplechap1>
%\fi
% Some text for chapter 2:
%\iffalse
%<*samplechap2>
%\fi
%    \begin{macrocode}
\section{two}
more text in chapter two
%    \end{macrocode}

%\iffalse
%</samplechap2>
%\fi
%
% %%%%%%%%%%%%%%%%%%%%%%%%%%%%%%%%%%%%%%
% \paragraph{Part Include Files.}
%
% The include files are called |cdocspt3.tex| and |cdocspt4.tex|.
%
%\iffalse
%<*samplepart3|samplepart4>
%\fi

% Optional override for |\version| flag:
%    \begin{macrocode}
%%\providecommand{\version}{final}
%    \end{macrocode}

% Include the main document:
%    \begin{macrocode}
\input{childdoc.def}
\childdocby{cdocsamp}
%    \end{macrocode}

%\iffalse
%</samplepart3|samplepart4>
%\fi
%
%\iffalse
%<*samplepart3>
%\fi
% Some text for part 3:
%    \begin{macrocode}
some text in part three
%    \end{macrocode}

%\iffalse
%</samplepart3>
%\fi
% Some text for part 4:
%\iffalse
%<*samplepart4>
%\fi
%    \begin{macrocode}
more text in part four
%    \end{macrocode}

%\iffalse
%</samplepart4>
%\fi
%
% %%%%%%%%%%%%%%%%%%%%%%%%%%%%%%%%%%%%%%
% \paragraph{Forwarding for a Complete Draft.}
%
% The following forwarding file |cdocsdrf.tex|
% compiles the main document in draft mode:
%\iffalse
%<*sampledraft>
%\fi
%    \begin{macrocode}
\def\version{draft}
\input{childdoc.def}
\childdocforward{cdocsamp}
%    \end{macrocode}

%\iffalse
%</sampledraft>
%\fi
%
% %%%%%%%%%%%%%%%%%%%%%%%%%%%%%%%%%%%%%%
% \paragraph{Forwarding for Final Version of the Chapters.}
%
% The following forwarding files |cdocsfn1.tex| and |cdocsfn2.tex|
% (with identical content)
% compile the final versions of the child documents
% |cdocsch1.tex| and |cdocsch2.tex|, respectively:
%\iffalse
%<*samplefinal>
%\fi
%    \begin{macrocode}
\def\version{final}
\input{childdoc.def}
\childdocforwardprefix[cdocsamp]{cdocsfn}{cdocsch}
%    \end{macrocode}

%\iffalse
%</samplefinal>
%\fi
%
% %%%%%%%%%%%%%%%%%%%%%%%%%%%%%%%%%%%%%%
% \paragraph{Command Line Processing.}
%
% The following three command lines generate the output files
% |cdocscld|, |cdocscl1| and |cdocscl2|
% which should be identical to
% |cdocsdrf|, |cdocsch1| and |cdocsfn2|, respectively:
% \begin{center}
% \begin{tabular}{l}
% |latex -jobname cdocscld \|\\
% |  "\def\version{draft}\input{childdoc.def}\childdocforward{cdocsamp}"|\\
% |latex -jobname cdocscl1 \|\\
% |  "\input{childdoc.def}\childdocforward[cdocsamp]{cdocsch1}"|\\
% |latex -jobname cdocscl2 \|\\
% |  "\def\version{final}\input{childdoc.def}\childdocforward{cdocsch2}"|
% \end{tabular}
% \end{center}
% Note that the trailing backslash on each first line
% merely continues the input to the second line
% (for convenient cut ant paste).
% Furthermore, the command |latex| can be replaced by any
% of its alternative versions such as |pdflatex|.
%
% %%%%%%%%%%%%%%%%%%%%%%%%%%%%%%%%%%%%%%%%%%%%%%%%%%%%%%%%%%%%%%%%%%%%%%%%%%%%%%
% %%%%%%%%%%%%%%%%%%%%%%%%%%%%%%%%%%%%%%%%%%%%%%%%%%%%%%%%%%%%%%%%%%%%%%%%%%%%%%
% \section{Implementation}
%\iffalse
%<*package>
%\fi
%
% This section describes the definitions file |childdoc.def|.

% The definitions cannot be loaded using |\usepackage| or |\RequirePackage|
% which has a mechanism to prevent loading a style file more than once.
% When loading the definitions by means of |\input|
% multiple instances have to be prevented manually:
%\iffalse
%This code needs to be before the `\ProvidesFile' directive
%which is defined at the beginning of this file.
%Therefore it is also placed there and commented out here.
%</package>
%<*discard>
%\fi
%    \begin{macrocode}
\ifdefined\childdocmain\endinput\fi
%    \end{macrocode}
%\iffalse
%</discard>
%<*package>
%\fi
%
% \macro{\ifchilddoc}
% \macro{\ifchilddocmanual}
% The conditional |\ifchilddoc| tells whether a
% child (true) or main (false) document is being compiled.
% The conditional |\ifchilddocmanual| tells whether
% the |\includeonly| mechanism is used (false) or
% the selection of child files must be performed manually (true).
% The definitions initialise to false:
%    \begin{macrocode}
\newif\ifchilddoc
\newif\ifchilddocmanual
%    \end{macrocode}

% \macro{\childdocname}
% \macro{\childdocjob}
% The macro |\childdocname| stores the name of the main document
% to be compiled. The macro |\childdocjob| stores the name of
% the document on which the \LaTeX{} compiler was originally invoked.
% The content of |\jobname| cannot be compared
% to filenames specified in the source due to different catcodes.
% The following code rescans |\jobname|, stores the result
% in |\childdocname| and saves a copy in |\childdocjob|:
%    \begin{macrocode}
\edef\childdocname{\scantokens\expandafter{\jobname\noexpand}}
\let\childdocjob\childdocname
%    \end{macrocode}

% \macro{\childdocdisable}
% The macro |\childdocdisable| prevents the main file
% from being processed more than once.
% At this stage, the main document command |\childdocmain|
% is assumed to be called once again where it should do nothing.
% Any subsequent call to it should prevent
% a secondary processing of the main document
% It overwrites the forwarding commands
% |\childdocof| and |\childdocforward|
% with empty macros to prevent further inclusions of the main document:
%    \begin{macrocode}
\newcommand{\childdocdisable}
{
  \renewcommand{\childdocmain}[1]{\renewcommand{\childdocmain}[1]{\endinput}}
  \renewcommand{\childdocof}[1]{}
  \renewcommand{\childdocby}[2][]{}
  \renewcommand{\childdocforward}[2][]{}
  \renewcommand{\childdocdisable}{}
}
%    \end{macrocode}

% \macro{\childdocmain}
% The macro |\childdocmain| is to be called at the top of the main file
% with nothing or the main filename (without extension) as argument.
% First, it breaks loops.
% If the argument is not empty and does not match |\childdocname|
% (which is set by the first inclusion of |childdoc.def|),
% |\ifchilddoc| is set to true, |\includeonly| is applied to the child file
% and |\jobname| is set to the main file
% (for proper handling of |.aux| files):
%    \begin{macrocode}
\newcommand{\childdocmain}[1]
{
  \childdocdisable\childdocmain{}
  \if?#1?\else
    \begingroup
      \def\childdoctmp{#1}
      \ifx\childdoctmp\childdocname
        \def\childdoctmp{}
      \else
        \def\childdoctmp
        {
          \childdoctrue
          \includeonly{\childdocname}
          \def\childdocjob{#1}
          \def\jobname{#1}
        }
      \fi
      \expandafter
    \endgroup
    \childdoctmp
  \fi
}
%    \end{macrocode}

% \macro{\childdocof}
% The command |\childdocof| redirects
% compilation to the main file |#1|.
%    \begin{macrocode}
\newcommand{\childdocof}[1]
{
  \childdocdisable
  \childdoctrue
  \includeonly{\childdocname}
  \def\jobname{#1}
  \def\childdocjob{#1}
  \input{#1}
}
%    \end{macrocode}

% \macro{\childdocby}
% The command |\childdocby| ....
%    \begin{macrocode}
\newcommand{\childdocby}[2][]
{
  \childdocdisable
  \childdoctrue
  \childdocmanualtrue
  \if?#1?\else
    \def\jobname{#2}
  \fi
  \def\childdocjob{#2}
  \input{#2}
  \endinput
}
%    \end{macrocode}

% \macro{\childdocforward}
% The command |\childdocforward| redirects
% compilation to the main file or
% (if the optional argument is given) a child file.
% Parameters are set as if the main file
% or a child file starting with |\childdocof| was compiled.
% Then compilation is handed over to the main file:
%    \begin{macrocode}
\newcommand{\childdocforward}[2][]
{
  \begingroup
    \if?#1?
      \def\childdoctmp
      {
        \def\childdocname{#2}
        \def\childdocjob{#2}
        \def\jobname{#2}
        \input{#2}
        \endinput
      }
    \else
      \def\childdoctmp
      {
        \childdocdisable
        \def\childdocname{#2}
        \childdoctrue
        \includeonly{#2}
        \def\childdocjob{#1}
        \def\jobname{#1}
        \input{#1}
        \endinput
      }
    \fi
    \expandafter
  \endgroup
  \childdoctmp
}
%    \end{macrocode}

% \macro{\childdocforwardprefix}
% The command |\childdocforwardprefix| redirects
% compilation to the main or a child file by means of a pattern.
% The prefix |#1| in the current filename is replaced by |#2|
% and the suffix of the current filename is kept
% (it is assumed that the filename does not contain the substring `|~~~|'
% which is used as a delimiter).
% Compilation is handed over to the new file by |\childdocforward|:
%    \begin{macrocode}
\newcommand{\childdocforwardprefix}[3][]
{
  \begingroup
    \def\childdocextract #2##1~~~{\def\childdoctmp{\childdocforward[#1]{#3##1}}}
    \expandafter\childdocextract\childdocname~~~
    \expandafter
  \endgroup
  \childdoctmp
}
%    \end{macrocode}

% \macro{\childdoc}
% The deprecated macro |\childdoc| is a legacy version of |\childdocmain|:
%    \begin{macrocode}
\newcommand{\childdoc}{\childdocmain}
%    \end{macrocode}

% \macro{\childdocredirect}
% The deprecated macro |\childdocredirect| is a legacy version
% of |\childdocforward| and |\childdocforwardprefix|:
%    \begin{macrocode}
\newcommand{\childdocredirect}[2][]
{
  \begingroup
    \if?#1?
      \def\childdoctmp{\childdocforward{#2}}
    \else
      \def\childdoctmp{\childdocforwardprefix{#1}{#2}}
    \fi
    \expandafter
  \endgroup
  \childdoctmp
}
%    \end{macrocode}

%\iffalse
%</package>
%\fi
%
\endinput

\childdocby{cdocsamp}
%    \end{macrocode}

%\iffalse
%</samplepart3|samplepart4>
%\fi
%
%\iffalse
%<*samplepart3>
%\fi
% Some text for part 3:
%    \begin{macrocode}
some text in part three
%    \end{macrocode}

%\iffalse
%</samplepart3>
%\fi
% Some text for part 4:
%\iffalse
%<*samplepart4>
%\fi
%    \begin{macrocode}
more text in part four
%    \end{macrocode}

%\iffalse
%</samplepart4>
%\fi
%
% %%%%%%%%%%%%%%%%%%%%%%%%%%%%%%%%%%%%%%
% \paragraph{Forwarding for a Complete Draft.}
%
% The following forwarding file |cdocsdrf.tex|
% compiles the main document in draft mode:
%\iffalse
%<*sampledraft>
%\fi
%    \begin{macrocode}
\def\version{draft}
% \iffalse
%
% childdoc.dtx Copyright (C) 2017-2018 Niklas Beisert
%
% This work may be distributed and/or modified under the
% conditions of the LaTeX Project Public License, either version 1.3
% of this license or (at your option) any later version.
% The latest version of this license is in
%   http://www.latex-project.org/lppl.txt
% and version 1.3 or later is part of all distributions of LaTeX
% version 2005/12/01 or later.
%
% This work has the LPPL maintenance status `maintained'.
%
% The Current Maintainer of this work is Niklas Beisert.
%
% This work consists of the files childdoc.dtx and childdoc.ins
% and the derived files childdoc.def and cdocsamp.tex with
% cdocsch1.tex, cdocsch2.tex, cdocsdrf.tex, cdocsfn1.tex, cdocsfn2.tex.
%
%<package>\ifdefined\childdocmain\endinput\fi
%<package>\ProvidesFile{childdoc.def}[2018/12/30 v2.0 child document driver]
%<samplemain>\ProvidesFile{cdocsamp.tex}[2018/12/30 v2.0 sample for childdoc]
%<*driver>
%\ProvidesFile{childdoc.drv}[2018/12/30 v2.0 childdoc reference manual file]
\PassOptionsToClass{10pt,a4paper}{article}
\documentclass{ltxdoc}

\usepackage[margin=35mm]{geometry}
\usepackage{hyperref}
\usepackage{hyperxmp}
\usepackage[usenames]{color}

\hypersetup{colorlinks=true}
\hypersetup{pdfstartview=FitH}
\hypersetup{pdfpagemode=UseNone}
\hypersetup{pdfsource={}}
\hypersetup{pdflang={en-UK}}
\hypersetup{pdfcopyright={Copyright 2017-2018 Niklas Beisert.
  This work may be distributed and/or modified under the
  conditions of the LaTeX Project Public License, either version 1.3
  of this license or (at your option) any later version.}}
\hypersetup{pdflicenseurl={http://www.latex-project.org/lppl.txt}}
\hypersetup{pdfcontactaddress={ETH Zurich, ITP, HIT K,
  Wolfgang-Pauli-Strasse 27}}
\hypersetup{pdfcontactpostcode={8093}}
\hypersetup{pdfcontactcity={Zurich}}
\hypersetup{pdfcontactcountry={Switzerland}}
\hypersetup{pdfcontactemail={nbeisert@itp.phys.ethz.ch}}
\hypersetup{pdfcontacturl={http://people.phys.ethz.ch/\xmptilde nbeisert/}}

\newcommand{\secref}[1]{\hyperref[#1]{section \ref*{#1}}}

\parskip1ex
\parindent0pt
\let\olditemize\itemize
\def\itemize{\olditemize\parskip0pt}

\begin{document}

\title{The \textsf{childdoc} Package}
\hypersetup{pdftitle={The childdoc Package}}
\author{Niklas Beisert\\[2ex]
  Institut f\"ur Theoretische Physik\\
  Eidgen\"ossische Technische Hochschule Z\"urich\\
  Wolfgang-Pauli-Strasse 27, 8093 Z\"urich, Switzerland\\[1ex]
  \href{mailto:nbeisert@itp.phys.ethz.ch}
  {\texttt{nbeisert@itp.phys.ethz.ch}}}
\hypersetup{pdfauthor={Niklas Beisert}}
\hypersetup{pdfsubject={Manual for the LaTeX2e Package childdoc}}
\date{30 December 2018, \textsf{v2.0}}
\maketitle

\begin{abstract}\noindent
\textsf{childdoc} is a \LaTeXe{} package
that enables the direct compilation
of document sections included by |\include|
to individual files.
\end{abstract}

\begingroup
\parskip0ex
\tableofcontents
\endgroup

%%%%%%%%%%%%%%%%%%%%%%%%%%%%%%%%%%%%%%%%%%%%%%%%%%%%%%%%%%%%%%%%%%%%%%%%%%%%%%%%
%%%%%%%%%%%%%%%%%%%%%%%%%%%%%%%%%%%%%%%%%%%%%%%%%%%%%%%%%%%%%%%%%%%%%%%%%%%%%%%%
\section{Introduction}

\LaTeX{} provides a mechanism to structure a large document (such as a book)
into a main file and several child files (containing the chapters)
using the |\include| command.
This mechanism is beneficial for documents
which span hundreds of pages in order to
make the source file(s) more manageable.
Moreover, compilation can be restricted to
selected child files by means of the |\includeonly| command.
The latter feature can be used to reduce the compilation time while editing
(this was significantly more useful in the earlier days of \LaTeX{})
or to generate a smaller document which is easier to navigate.
Another application of |\includeonly| is to generate
documents consisting of selected parts of the complete document.

However, there are a few drawbacks of the plain |\include| mechanism:
\begin{itemize}
\item
The child files cannot be compiled on their own,
they can only be compiled via the main file.
A naive editing environment
(such as a text editor with an option
to have the current file processed by \LaTeX)
may require one to switch to the main file before compiling;
attempting to compile the child file produces errors.
\item
The main file must be modified (each time)
to adjust the |\includeonly| command
to the present needs. This easily leaves the main file in a messy state.
\item
The generated document will always carry the filename
of the main document. This is inconvenient if
several child files are to be compiled and
to be kept for distribution.
\end{itemize}

The present package provides a simple interface
to make child files individually compilable by \LaTeX{}.
Compiling a child file then has the same effect as compiling
the main file with an |\includeonly| command
to select the appropriate child.
Moreover the generated document will carry the name of the child
rather than the main file.
This resolves all three above issues.

This feature is meant to make the editing of books,
thesis documents and lecture notes somewhat more convenient.
However, the package can also be used efficiently for
composing a series of documents (such as exercise sheets)
which are typically distributed individually.
It then assists the author in generating the individual documents
(potentially in different versions)
as well as a document containing the collected series.
Another application is in developing style files
or other kinds of included material
where compilation of the style file could redirect
to a sample or test file.

%%%%%%%%%%%%%%%%%%%%%%%%%%%%%%%%%%%%%%%%%%%%%%%%%%%%%%%%%%%%%%%%%%%%%%%%%%%%%%%%
%%%%%%%%%%%%%%%%%%%%%%%%%%%%%%%%%%%%%%%%%%%%%%%%%%%%%%%%%%%%%%%%%%%%%%%%%%%%%%%%
\section{Usage}

First of all, the package \textsf{childdoc} is \emph{not} a standard
\LaTeXe{} |.sty| style file! Therefore it needs to be invoked in
a non-standard way.

%%%%%%%%%%%%%%%%%%%%%%%%%%%%%%%%%%%%%%%%%%%%%%%%%%%%%%%%%%%%%%%%%%%%%%%%%%%%%%%%
\subsection{Included Files}
\label{sec:include}

%%%%%%%%%%%%%%%%%%%%%%%%%%%%%%%%%%%%%%%%
\DescribeMacro{\childdocmain}
To use the package, add the commands
\begin{center}
\begin{tabular}{l}
|\input{childdoc.def}|\\
|\childdocmain{}|\\
\end{tabular}
\end{center}
at the very top of the main \LaTeX{} file,
in particular \emph{before} the |\documentclass| statement!
The argument of |\childdocmain| should be left empty
(but it must be present).

%%%%%%%%%%%%%%%%%%%%%%%%%%%%%%%%%%%%%%%%
\DescribeMacro{\childdocof}
Furthermore, add the commands
\begin{center}
\begin{tabular}{l}
|\input{childdoc.def}|\\
|\childdocof{|\textit{main}|}|\\
\end{tabular}
\end{center}
at the top of every child file \textit{child}
which is included by |\include{|\textit{child}|}|
from within the main file
(or at least for those files to be compiled individually).
The argument \textit{main} must be the filename of the main file.

There are a couple of
considerations in setting up the main and child documents:

%%%%%%%%%%%%%%%%%%%%%%%%%%%%%%%%%%%%%%%%
\paragraph{Restrictions.}

Please note the following restrictions:
\begin{itemize}
\item
|\childdocmain| must be called with one argument \textit{main}
to ensure compatibility with earlier version of the package.
It must either be empty (|\childdocmain{}|)
or precisely match the filename of the main file in which it is specified.
See \secref{sec:detection} for further information.
\item
The filename \textit{main} must be specified without the |.tex| extension.
\item
The filename \textit{main} is case sensitive
(even in case-insensitive file systems)
due to internal string comparison.
\item
The argument \textit{main} should be fully expanded, it cannot be a macro.
\item
Subdirectories and special characters should be avoided in filenames.
\item
The command |\childdocmain{|\textit{main}|}| must be followed by a whitespace.
It should not be followed immediately by another command
or by a comment mark `|%|'.
This is because the \TeX{} parser reads the token immediately following
the argument of |\childdocmain| and puts it
at the beginning of every child section;
however, a white\-space is ignored.
\end{itemize}

%%%%%%%%%%%%%%%%%%%%%%%%%%%%%%%%%%%%%%%%
\paragraph{Content of Main File.}

It is advisable to place all content in the child files included by |\include|.
Any output contained in the main file will appear in all child documents
unless suppressed manually;
it cannot be suppressed automatically by the |\includeonly| directive
and thus should normally be avoided.
A method to include some content in the main file
by means of conditional processing is described in \secref{sec:conditional}.

%%%%%%%%%%%%%%%%%%%%%%%%%%%%%%%%%%%%%%%%
\paragraph{Page Numbering.}

When only a part of the document is compiled,
the appropriate numbering of pages
(as well as other status parameters)
is determined from the |.aux| files.
The latter contain information from previous passes.
However this information needs to propagate through
all intermediate child documents.
Therefore the page numbering in child documents may well
be inconsistent until the complete document is compiled at least once.

A useful (if unconventional) way to always ensure a consistent
page numbering is to restart the numbering in each child document
and denote the pages by `\textit{child}|.|\textit{page}'
where \textit{child} represents the chapter/section number of the child file.
This can be achieved by the command
|\numberwithin{page}{|\textit{child}|}|
of the \textsf{amsmath} package
where \textit{child} can be |chapter| or |section|
depending on the chosen structuring.
Alternatively, one can modify the macro |\thepage| appropriately
and reset the counter |page| at the start of each child file.

%%%%%%%%%%%%%%%%%%%%%%%%%%%%%%%%%%%%%%%%%%%%%%%%%%%%%%%%%%%%%%%%%%%%%%%%%%%%%%%%
\subsection{Conditional Processing}
\label{sec:conditional}

The package provides a mechanism to compile different versions
of a document. To customise the versions further some conditional processing
can come in handy to distinguish which version is being compiled.
The package provides two macros to describe the compilation context:

%%%%%%%%%%%%%%%%%%%%%%%%%%%%%%%%%%%%%%%%
\DescribeMacro{\ifchilddoc}
The conditional |\ifchilddoc| distinguishes between the compilation of
child documents and the main document:
%
\begin{center}
|\ifchilddoc |\textit{child-code}| |[|\||else |\textit{main-code}]| \||fi|
\end{center}

%%%%%%%%%%%%%%%%%%%%%%%%%%%%%%%%%%%%%%%%
\DescribeMacro{\childdocname}
\DescribeMacro{\childdocjob}
The macro |\childdocname| contains the filename (without extension)
of the main or child file being processed.
Note that |\childdocjob| will always contain the name of the main file.

%%%%%%%%%%%%%%%%%%%%%%%%%%%%%%%%%%%%%%%%
\paragraph{Title Page.}

Conditional processing can be used to include a title or banner page
in the main document when proper precautions are taken.
Importantly, the code in the main file should ensure that the page counter
(as well as other status parameters which are stored in the |.aux| files)
takes the same value after the conditional processing.
Otherwise the page numbers may take divergent values
depending on which part is compiled.

For example, a title page could be declared by:
%
\begin{center}
\begin{tabular}{l}
|\ifchilddoc\||else|\\
|\addtocounter{page}{-1}|\\
\textit{code for title page}\\
|\newpage|\\
|\||fi|
\end{tabular}
\end{center}
%
A banner page for the child documents can be generated by:
%
\begin{center}
\begin{tabular}{l}
|\ifchilddoc|\\
|\addtocounter{page}{-1}|\\
\textit{code for banner page}\\
|\newpage|\\
|\||fi|
\end{tabular}
\end{center}
%
Here one could write a message such as:
\begin{center}
|This is the part \childdocname{} of \childdocjob{}.|
\end{center}

%%%%%%%%%%%%%%%%%%%%%%%%%%%%%%%%%%%%%%%%%%%%%%%%%%%%%%%%%%%%%%%%%%%%%%%%%%%%%%%%
\subsection{Flags}
\label{sec:flags}

The package makes it easy to generate different versions
of the main or child documents.
To this end compilation flags can be defined
and assigned different default values.
They will be particularly useful in conjunction
with the forwarding mechanism described in \secref{sec:forward}.

For example, it may be useful to have a flag |\version|
which can be set to |draft| or |final|.
The document source will contain some conditional code
depending on the value of |\version|.
Suppose further, the flag should default to |final| for the main file
and to |draft| for child files
which is a natural assignment for editing the document.
This is achieved by placing the following code
in the preamble of the main document
(below the |\childdocmain| directive):
%
\begin{center}
\begin{tabular}{l}
|\ifchilddoc|\\
|\providecommand{\version}{draft}|\\
|\||else|\\
|\providecommand{\version}{final}|\\
|\||fi|
\end{tabular}
\end{center}
%
The definition by |\providecommand| makes sure
that previous definitions are not overwritten.
Further statements |\providecommand{\version}{...}|
can thus be added before the above code to override it.

For the main file, one might add a line
(between |\childdocmain| and the above block)
%
\begin{center}
|%\ifchilddoc\||else\providecommand{\version}{draft}\||fi|
\end{center}
%
which can be uncommented to produce a draft version.
Likewise one can add a line to the very top of a child file
(above the |\childdocof{|\textit{main}|}| directive)
%
\begin{center}
|%\providecommand{\version}{final}|
\end{center}
%
which can be uncommented to produce the final version of this child document.

%%%%%%%%%%%%%%%%%%%%%%%%%%%%%%%%%%%%%%%%%%%%%%%%%%%%%%%%%%%%%%%%%%%%%%%%%%%%%%%%
\subsection{Forwarding}
\label{sec:forward}

Different versions of the main or child documents
using compilation flags as described in \secref{sec:flags}
can be (permanently) stored in different files
for convenient compilation, viewing and distribution.
To this end, the package defines a command
to pass on compilation to a different file:

%%%%%%%%%%%%%%%%%%%%%%%%%%%%%%%%%%%%%%%%
\DescribeMacro{\childdocforward}
The command |\childdocforward| redirects processing to
another source file:
%
\begin{center}
\begin{tabular}{l}
|\input{childdoc.def}|\\
|\childdocforward[|\textit{main}|]{|\textit{dest}|}|\\
\end{tabular}
\end{center}
%
The argument \textit{dest} is the destination file
(without extension).
It should be the main file or one of the child files.
Note that further \textsf{childdoc} directives
such as |\childdocof| and |\childdocforward|
in the indicated file will be processed in this form.
The optional argument \textit{main}
passes on directly to the main file \textit{main}
while pretending to compile the child \textit{dest}.
This form behaves as if \textit{dest}
issues |\childdocof{|\textit{main}|}| right away,
and no further \textsf{childdoc} directives will be processed.

%%%%%%%%%%%%%%%%%%%%%%%%%%%%%%%%%%%%%%%%
\DescribeMacro{\...prefix}
In the alternative form |\childdocforwardprefix|,
%
\begin{center}
\begin{tabular}{l}
|\input{childdoc.def}|\\
|\childdocforwardprefix[|\textit{main}|]{|\textit{prefix}|}{|\textit{dest}|}|
\end{tabular}
\end{center}
%
the destination file is determined by a pattern
depending on the current file:
To make this work, the current file must be called
`{\textit{prefix}\hspace{0.2em}\textit{suffix}}'
with \textit{prefix} matching precisely the argument.
Processing is then passed on to the file
`{\textit{dest}\hspace{0.2em}\textit{suffix}}'.
Surely, the same effect is achieved by
directly specifying the
argument `{\textit{dest}\hspace{0.2em}\textit{suffix}}'
in the first form.
However, that requires to set up a different file
for each child. With the alternative form of the command
all these files can have exactly the same content
which simplifies setting them up and maintaining them.

For example, the following file |draft.tex|
with a compilation flag |\version| as described in \secref{sec:flags}
compiles the main document as a draft:
%
\begin{center}
\begin{tabular}{l}
|\def\version{draft}|\\
|\input{childdoc.def}|\\
|\childdocforward{|\textit{main}|}|
\end{tabular}
\end{center}
%
Likewise, the following files |final|\textit{nn}|.tex|
compile the final version of the child document
|child|\textit{nn}|.tex|:
%
\begin{center}
\begin{tabular}{l}
|\def\version{final}|\\
|\input{childdoc.def}|\\
|\childdocforwardprefix{final}{child}|
\end{tabular}
\end{center}
%

Note that when several versions of a main file and/or of each child file
are to be generated, it may be convenient to set up a |Makefile| or
shell script to automatise the process.

%%%%%%%%%%%%%%%%%%%%%%%%%%%%%%%%%%%%%%%%%%%%%%%%%%%%%%%%%%%%%%%%%%%%%%%%%%%%%%%%
\subsection{Command Line Processing}
\label{sec:commandline}

The effect of redirection files can also be achieved by invoking
the \LaTeX{} compiler with a more elaborate command line.
Most conveniently this should be done as part
of a shell script or a |Makefile|.

When using \textsf{childdoc} in the main file, the following
command lines effectively perform a redirection
(note that depending on the shell being used,
backslashes may have to be doubled: `|\|' $\to$ `|\\|'):
%
\begin{center}
|... -jobname "|\textit{target}|" |\\|"|[\textit{flags}]%
|\input{childdoc.def}\childdocforward[|\textit{main}|]{|\textit{dest}|}"|
\end{center}
%
Here \textit{target} is the name of the output file,
\textit{main} is the name of the main file
and \textit{dest} is the name of the main or child file to be processed
(all filenames without extensions).
The optional argument \textit{main} can be omitted
if \textit{main} matches \textit{dest}.
Optionally, compilation \textit{flags} can be defined via |\def| commands.
This command line makes the \TeX{} engine believe
it is compiling the file \textit{target}
whose content is specified as the latter parameter.
The provided code then forwards the processing to
\textit{main} or \textit{dest} as described in \secref{sec:forward}.

%%%%%%%%%%%%%%%%%%%%%%%%%%%%%%%%%%%%%%%%%%%%%%%%%%%%%%%%%%%%%%%%%%%%%%%%%%%%%%%%
\subsection{Include by Input}
\label{sec:input}

Including child documents by |\include| has some restrictions by design.
Most notably, the content of a child document always occupies
its own set of pages; pages cannot be shared between child documents.
Usually, this behaviour makes perfect sense
because each child document contain an essential part of the document.
However, in some situations it may be desirable to compose
a document from a collection of parts
without having mandatory page breaks between then.
For this case, the package
provides a mechanism to include parts
by |\input| which can also be processed individually.
However, by construction this mechanism
requires manual handling of the content to be output.

%%%%%%%%%%%%%%%%%%%%%%%%%%%%%%%%%%%%%%%%
\DescribeMacro{\ifchilddocmanual}
The main file should be prepared as usual, see \secref{sec:include}.
However, the document body must make a distinction
between processing of an individual part and of the main document, e.g.:
%
\begin{center}
\begin{tabular}{l}
|\ifchilddocmanual|\\
|\input{\childdocname}|\\
|\||else|\\
\textit{document body with }|\input{|\textit{part}|}|\\
|\||fi|
\end{tabular}
\end{center}
%
The conditional |\ifchilddocmanual| is true whenever
a part to be included by |\input| is being compiled,
and the name of the part is stored in |\childdocname|.

%%%%%%%%%%%%%%%%%%%%%%%%%%%%%%%%%%%%%%%%
\DescribeMacro{\childdocby}
Each part to be included by |\input| should start with:
%
\begin{center}
\begin{tabular}{l}
|\input{childdoc.def}|\\
|\childdocby{|\textit{main}|}|\\
\end{tabular}
\end{center}
%
The directive |\childdocby| is similar to |\childdocof|
described in \secref{sec:include},
but the subsequent selection of content must be done manually.
To that end, both |\ifchilddoc| and |\ifchilddocmanual|
will be true upon processing of a part,
and the name of the part is stored in |\childdocname|.
Note that |\jobname| will be set to the filename of the current part
so that each part receives an individual |.aux| file
that does not interfere with the |.aux| file(s) of the main document.
This behaviour can be altered by the alternative form
|\childdocby[*]{|\textit{main}|}| (with a non-empty optional argument)
which uses the |.aux| file of the main document
by setting |\jobname| to \textit{main}.

%%%%%%%%%%%%%%%%%%%%%%%%%%%%%%%%%%%%%%%%%%%%%%%%%%%%%%%%%%%%%%%%%%%%%%%%%%%%%%%%
\subsection{Driver Development}
\label{sec:driver}

The \textsf{childdoc} mechanism can also be use for the development
of definition files such as \LaTeX{} styles or classes.
This case differs from the above setup with multiple parts
included by |\include| in that no |\includeonly| should be invoked.
This can be achieved by starting the include file
(before |\ProvidesPackage|) with:
%
\begin{center}
\begin{tabular}{l}
|\input{childdoc.def}|\\
|\childdocforward{|\textit{main}|}|\\
\end{tabular}
\end{center}
%
or alternatively with:
%
\begin{center}
\begin{tabular}{l}
|\input{childdoc.def}|\\
|\childdocby{|\textit{main}|}|\\
\end{tabular}
\end{center}
%
Both forms have slightly different effects as described above.
The main file is prepared as usual, see \secref{sec:include}.

%%%%%%%%%%%%%%%%%%%%%%%%%%%%%%%%%%%%%%%%%%%%%%%%%%%%%%%%%%%%%%%%%%%%%%%%%%%%%%%%
\subsection{Legacy Detection}
\label{sec:detection}

The directive |\childdocmain| in the main file can detect
whether the complete document or merely a child is to be compiled
even without using the directive |\childdocof|.
This method is deprecated because it is less robust
and there is no compelling reason to use it;
it is merely provided for backward compatibility
and it may be removed in future versions.

If the detection mechanism is to be used,
it is mandatory to correctly specify
the filename of the main file as the argument of |\childdocmain|:
%
\begin{center}
\begin{tabular}{l}
|\input{childdoc.def}|\\
|\childdocmain{|\textit{main}|}|\\
\end{tabular}
\end{center}
%
If |\jobname| does not match the argument \textit{main} of |\childdocmain|,
it is assumed that |\jobname| points to the child file to be compiled.
When using |\childdocmain| with the main file specified as argument,
it suffices to start a child file
with just |\input{|\textit{main}|}|
without loading of the package and using |\childdocof|.
If instead all processing is done
with the appropriate \textsf{childdoc} directives,
the argument of \textit{main} of |\childdocmain| can be empty.

An alternative version of the command line processing described
in \secref{sec:commandline} using the detection mechanism reads:
%
\begin{center}
|... -jobname "|\textit{target}|" "|[\textit{flags}]%
[|\def\jobname{|\textit{dest}|}|]|\input{|\textit{main}|}"|
\end{center}

%%%%%%%%%%%%%%%%%%%%%%%%%%%%%%%%%%%%%%%%%%%%%%%%%%%%%%%%%%%%%%%%%%%%%%%%%%%%%%%%
\subsection{Manual Code}
\label{sec:manual}

In case one cannot be certain whether the definitions file |childdoc.def|
is installed on the target \TeX{} distribution
and one prefers not to ship it,
it is conceivable to paste a few relevant commands into the sources.

To that end, drop all statements |\input{childdoc.def}|
and perform the replacements as outlined below.
Instead of |\childdocmain{|\textit{main}|}| add the following code
to the top of the main file:
%
\begin{center}
\begin{tabular}{l}
|\||ifdefined\childdocname\endinput\||fi\newif\ifchilddoc|\\
|\edef\childdocname{\scantokens\expandafter{\jobname\noexpand}}|\\
|\def\childdocmain{|\textit{main}|}\||ifx\childdocmain\childdocname\||else|\\
|\childdoctrue\includeonly{\childdocname}\let\jobname\childdocmain\||fi|\\
\end{tabular}
\end{center}
%
Instead of |\childdocof{|\textit{main}|}| just include the main file
at the top of each child file:
%
\begin{center}
|\input{|\textit{main}|}|
\end{center}
%
A simple redirection |\childdocforward{|\textit{dest}|}| is achieved by:
%
\begin{center}
|\def\jobname{|\textit{dest}|}\input{\jobname}|
\end{center}
%
The redirection with prefix
|\childdocforwardprefix[|\textit{prefix}|]{|\textit{dest}|}|
is accomplished by:
%
\begin{center}
\begin{tabular}{l}
|{\edef\jobname{\scantokens\expandafter{\jobname\noexpand}}|\\
|\def\redirectjob |\textit{prefix}|#1~~~{\gdef\jobname{|\textit{dest}|#1}}|\\
|\expandafter\redirectjob\jobname~~~}\input{\jobname}|
\end{tabular}
\end{center}

In an alternative approach,
child documents can be compiled by a specific command line
without additional code or specific definitions:
%
\begin{center}
|... -jobname "|\textit{target}|" "|[\textit{flags}]%
|\includeonly{|\textit{dest}|}\input{|\textit{main}|}"|
\end{center}
%

%%%%%%%%%%%%%%%%%%%%%%%%%%%%%%%%%%%%%%%%%%%%%%%%%%%%%%%%%%%%%%%%%%%%%%%%%%%%%%%%
%%%%%%%%%%%%%%%%%%%%%%%%%%%%%%%%%%%%%%%%%%%%%%%%%%%%%%%%%%%%%%%%%%%%%%%%%%%%%%%%
\section{Information}

%%%%%%%%%%%%%%%%%%%%%%%%%%%%%%%%%%%%%%%%%%%%%%%%%%%%%%%%%%%%%%%%%%%%%%%%%%%%%%%%
\subsection{Copyright}

Copyright \copyright{} 2017--2018 Niklas Beisert

This work may be distributed and/or modified under the
conditions of the \LaTeX{} Project Public License, either version 1.3
of this license or (at your option) any later version.
The latest version of this license is in
  \url{http://www.latex-project.org/lppl.txt}
and version 1.3 or later is part of all distributions of \LaTeX{}
version 2005/12/01 or later.

This work has the LPPL maintenance status `maintained'.

The Current Maintainer of this work is Niklas Beisert.

This work consists of the files |README.txt|, |childdoc.ins| and |childdoc.dtx|
as well as the derived files |childdoc.def|, |cdocsamp.tex|
with |cdocsch1.tex|, |cdocsch2.tex|, |cdocspt3.tex|, |cdocspt4.tex|,
|cdocsdrf.tex|, |cdocsfn1.tex|, |cdocsfn2.tex|
as well as |childdoc.pdf|.

%%%%%%%%%%%%%%%%%%%%%%%%%%%%%%%%%%%%%%%%%%%%%%%%%%%%%%%%%%%%%%%%%%%%%%%%%%%%%%%%
\subsection{Files and Installation}

The package consists of the files:
%
\begin{center}
\begin{tabular}{ll}
    |README.txt|   & readme file \\
    |childdoc.ins| & installation file \\
    |childdoc.dtx| & source file \\
    |childdoc.def| & definition file \\
    |cdocsamp.tex| & sample main file \\
    |cdocsch1.tex| & sample include file \\
    |cdocsch2.tex| & sample include file \\
    |cdocspt3.tex| & sample part file \\
    |cdocspt4.tex| & sample part file \\
    |cdocsdrf.tex| & sample redirection file \\
    |cdocsfn1.tex| & sample redirection file \\
    |cdocsfn2.tex| & sample redirection file \\
    |childdoc.pdf| & manual
\end{tabular}
\end{center}
%
The distribution consists of the files
|README.txt|, |childdoc.ins| and |childdoc.dtx|.
%
\begin{itemize}
\item
Run (pdf)\LaTeX{} on |childdoc.dtx|
to compile the manual |childdoc.pdf| (this file).
\item
Run \LaTeX{} on |childdoc.ins| to create the definitions file |childdoc.def|
and the sample |cdocsamp.tex| with include files
|cdocsch1.tex|, |cdocsch2.tex|, |cdocspt3.tex|, |cdocspt4.tex|,
|cdocsdrf.tex|, |cdocsfn1.tex|, |cdocsfn2.tex|.
Then copy the file |childdoc.def| to an appropriate directory of your \LaTeX{}
distribution, e.g.\ \textit{texmf-root}|/tex/latex/childdoc|.
\end{itemize}

%%%%%%%%%%%%%%%%%%%%%%%%%%%%%%%%%%%%%%%%%%%%%%%%%%%%%%%%%%%%%%%%%%%%%%%%%%%%%%%%
\subsection{Related CTAN Packages}

There are several other packages which offer a similar functionality:
%
\begin{itemize}
\item
The packages
\href{http://ctan.org/pkg/docmute}{\textsf{docmute}},
\href{http://ctan.org/pkg/includex}{\textsf{includex}} and
\href{http://ctan.org/pkg/standalone}{\textsf{standalone}}
provide commands to include only the document body of
a child file thus allowing both files to be compiled individually.
\item
The packages \href{http://ctan.org/pkg/subdocs}{\textsf{subdocs}}
and \href{http://ctan.org/pkg/subfiles}{\textsf{subfiles}}
provide structures in which the main and child documents can be
encapsulated and allowing them to be compiled individually.
The inclusion mechanism is different from the conventional |\include|.
\item
The package \href{http://ctan.org/pkg/combine}{\textsf{combine}}
is an elaborate solution to combine several documents into one.
\end{itemize}
%
See also the CTAN topic \href{http://ctan.org/topic/subdocs}{\textsf{subdocs}}
for further related packages.
The present package differs from the above solutions in that
a document structure constructed with the conventional |\include| mechanism
just needs two extra commands at the top of every file
such that all constituent files can be compiled individually.

%%%%%%%%%%%%%%%%%%%%%%%%%%%%%%%%%%%%%%%%%%%%%%%%%%%%%%%%%%%%%%%%%%%%%%%%%%%%%%%%
%\subsection{Feature Suggestions}
%
%The following is a list of features which may be useful for future
%versions of this package:
%%
%\begin{itemize}
%\item
%\ldots
%\end{itemize}

%%%%%%%%%%%%%%%%%%%%%%%%%%%%%%%%%%%%%%%%%%%%%%%%%%%%%%%%%%%%%%%%%%%%%%%%%%%%%%%%
\subsection{Revision History}

%%%%%%%%%%%%%%%%%%%%%%%%%%%%%%%%%%%%%%%%
\paragraph{v2.0:} 2018/12/30

\begin{itemize}
\item
immediate forward processing
\item
added |\childdocby| mechanism
\item
manual restructured
\end{itemize}

%%%%%%%%%%%%%%%%%%%%%%%%%%%%%%%%%%%%%%%%
\paragraph{v1.6:} 2018/01/17

\begin{itemize}
\item
application for development of include files
\item
corrections to manual
\end{itemize}

%%%%%%%%%%%%%%%%%%%%%%%%%%%%%%%%%%%%%%%%
\paragraph{v1.5:} 2017/05/21

\begin{itemize}
\item
more complete structuring introduced
\item
|\childdocof| introduced
\item
|\childdoc| renamed to |\childdocmain|
\item
|\childredirect| renamed to |\childdocforward| and |\childdocforwardprefix|
and functionality expanded
\end{itemize}

%%%%%%%%%%%%%%%%%%%%%%%%%%%%%%%%%%%%%%%%
\paragraph{v1.0:} 2017/04/27

\begin{itemize}
\item
manual and install package
\item
first version published on CTAN
\end{itemize}

%%%%%%%%%%%%%%%%%%%%%%%%%%%%%%%%%%%%%%%%
\paragraph{v0.6:} 2017/04/26

\begin{itemize}
\item
redirection mechanism added
\end{itemize}

%%%%%%%%%%%%%%%%%%%%%%%%%%%%%%%%%%%%%%%%
\paragraph{v0.5:} 2017/04/26

\begin{itemize}
\item
functionality in definition file
\end{itemize}


%%%%%%%%%%%%%%%%%%%%%%%%%%%%%%%%%%%%%%%%%%%%%%%%%%%%%%%%%%%%%%%%%%%%%%%%%%%%%%%%
%%%%%%%%%%%%%%%%%%%%%%%%%%%%%%%%%%%%%%%%%%%%%%%%%%%%%%%%%%%%%%%%%%%%%%%%%%%%%%%%
%%%%%%%%%%%%%%%%%%%%%%%%%%%%%%%%%%%%%%%%%%%%%%%%%%%%%%%%%%%%%%%%%%%%%%%%%%%%%%%%
\appendix

\settowidth\MacroIndent{\rmfamily\scriptsize 000\ }

 \DocInput{childdoc.dtx}

\end{document}
%</driver>
% \fi
%
% %%%%%%%%%%%%%%%%%%%%%%%%%%%%%%%%%%%%%%%%%%%%%%%%%%%%%%%%%%%%%%%%%%%%%%%%%%%%%%
% %%%%%%%%%%%%%%%%%%%%%%%%%%%%%%%%%%%%%%%%%%%%%%%%%%%%%%%%%%%%%%%%%%%%%%%%%%%%%%
% \section{Sample}
%\iffalse
%<*samplemain>
%\fi
%
% The following presents a sample document
% with two chapters, two parts, a title page,
% a compile flag as well as three forwarding files to set the flag.
% It consists of eight |.tex| files:
% \begin{center}
% \begin{tabular}{ll}
% |cdocsamp.tex|&main file\\
% |cdocsch1.tex|&include file for chapter 1\\
% |cdocsch2.tex|&include file for chapter 2\\
% |cdocspt3.tex|&include file for part 3\\
% |cdocspt4.tex|&include file for part 4\\
% |cdocsdrf.tex|&forwarding file for main file in draft mode\\
% |cdocsfi1.tex|&forwarding file for final version of chapter 1\\
% |cdocsfi2.tex|&forwarding file for final version of chapter 2\\
% \end{tabular}
% \end{center}
% Each of the eight files can be compiled directly by the \LaTeX{} compiler.
%
% %%%%%%%%%%%%%%%%%%%%%%%%%%%%%%%%%%%%%%
% \paragraph{Main File.}
%
% The main file is called |cdocsamp.tex|.
%
% Load the \textsf{childdoc} definitions and
% declare the filename for the main document:
%    \begin{macrocode}
\input{childdoc.def}
\childdocmain{}
%    \end{macrocode}

% Optional override for |\version| flag:
%    \begin{macrocode}
%%\ifchilddoc\else\providecommand{\version}{draft}\fi
%    \end{macrocode}

% Define the default values for the |\version| flag
% (|final| for the main file and |draft| for childs):
%    \begin{macrocode}
\ifchilddoc
\providecommand{\version}{draft}
\else
\providecommand{\version}{final}
\fi
%    \end{macrocode}

% Load the standard document class:
%    \begin{macrocode}
\documentclass[12pt]{article}
%    \end{macrocode}

% Start the document body:
%    \begin{macrocode}
\begin{document}
%    \end{macrocode}

% Declare a title page.
% Print title, part of document being processed and version flag:
%    \begin{macrocode}
\addtocounter{page}{-1}
\begin{center}
{\LARGE\bfseries{}childdoc example\par}
\vspace{1cm}
\ifchilddoc
\ifchilddocmanual part\else chapter\fi:
`\childdocname' of `\childdocjob'\par
\else
main document: `\childdocjob'\par
\fi
version: \version\par
\end{center}
\newpage
%    \end{macrocode}

% Manually include selected file,
% otherwise process as usual:
%    \begin{macrocode}
\ifchilddocmanual
\section*{part `\childdocname'}
\input{\childdocname}
\else
%    \end{macrocode}

% Include the two chapters:
%    \begin{macrocode}
\include{cdocsch1}
\include{cdocsch2}
%    \end{macrocode}

% Include the two parts unless only chapters should be displayed:
%    \begin{macrocode}
\ifchilddoc\else
\section{part three}
\input{cdocspt3}
\section{part four}
\input{cdocspt4}
\fi
%    \end{macrocode}

% Process as usual until here:
%    \begin{macrocode}
\fi
%    \end{macrocode}

% End of document body:
%    \begin{macrocode}
\end{document}
%    \end{macrocode}
%\iffalse
%</samplemain>
%\fi
%
% %%%%%%%%%%%%%%%%%%%%%%%%%%%%%%%%%%%%%%
% \paragraph{Chapter Include Files.}
%
% The include files are called |cdocsch1.tex| and |cdocsch2.tex|.
%
%\iffalse
%<*samplechap1|samplechap2>
%\fi

% Optional override for |\version| flag:
%    \begin{macrocode}
%%\providecommand{\version}{final}
%    \end{macrocode}

% Include the main document:
%    \begin{macrocode}
\input{childdoc.def}
\childdocof{cdocsamp}
%    \end{macrocode}

%\iffalse
%</samplechap1|samplechap2>
%\fi
%
%\iffalse
%<*samplechap1>
%\fi
% Some text for chapter 1:
%    \begin{macrocode}
\section{one}
some text in chapter one
%    \end{macrocode}

%\iffalse
%</samplechap1>
%\fi
% Some text for chapter 2:
%\iffalse
%<*samplechap2>
%\fi
%    \begin{macrocode}
\section{two}
more text in chapter two
%    \end{macrocode}

%\iffalse
%</samplechap2>
%\fi
%
% %%%%%%%%%%%%%%%%%%%%%%%%%%%%%%%%%%%%%%
% \paragraph{Part Include Files.}
%
% The include files are called |cdocspt3.tex| and |cdocspt4.tex|.
%
%\iffalse
%<*samplepart3|samplepart4>
%\fi

% Optional override for |\version| flag:
%    \begin{macrocode}
%%\providecommand{\version}{final}
%    \end{macrocode}

% Include the main document:
%    \begin{macrocode}
\input{childdoc.def}
\childdocby{cdocsamp}
%    \end{macrocode}

%\iffalse
%</samplepart3|samplepart4>
%\fi
%
%\iffalse
%<*samplepart3>
%\fi
% Some text for part 3:
%    \begin{macrocode}
some text in part three
%    \end{macrocode}

%\iffalse
%</samplepart3>
%\fi
% Some text for part 4:
%\iffalse
%<*samplepart4>
%\fi
%    \begin{macrocode}
more text in part four
%    \end{macrocode}

%\iffalse
%</samplepart4>
%\fi
%
% %%%%%%%%%%%%%%%%%%%%%%%%%%%%%%%%%%%%%%
% \paragraph{Forwarding for a Complete Draft.}
%
% The following forwarding file |cdocsdrf.tex|
% compiles the main document in draft mode:
%\iffalse
%<*sampledraft>
%\fi
%    \begin{macrocode}
\def\version{draft}
\input{childdoc.def}
\childdocforward{cdocsamp}
%    \end{macrocode}

%\iffalse
%</sampledraft>
%\fi
%
% %%%%%%%%%%%%%%%%%%%%%%%%%%%%%%%%%%%%%%
% \paragraph{Forwarding for Final Version of the Chapters.}
%
% The following forwarding files |cdocsfn1.tex| and |cdocsfn2.tex|
% (with identical content)
% compile the final versions of the child documents
% |cdocsch1.tex| and |cdocsch2.tex|, respectively:
%\iffalse
%<*samplefinal>
%\fi
%    \begin{macrocode}
\def\version{final}
\input{childdoc.def}
\childdocforwardprefix[cdocsamp]{cdocsfn}{cdocsch}
%    \end{macrocode}

%\iffalse
%</samplefinal>
%\fi
%
% %%%%%%%%%%%%%%%%%%%%%%%%%%%%%%%%%%%%%%
% \paragraph{Command Line Processing.}
%
% The following three command lines generate the output files
% |cdocscld|, |cdocscl1| and |cdocscl2|
% which should be identical to
% |cdocsdrf|, |cdocsch1| and |cdocsfn2|, respectively:
% \begin{center}
% \begin{tabular}{l}
% |latex -jobname cdocscld \|\\
% |  "\def\version{draft}\input{childdoc.def}\childdocforward{cdocsamp}"|\\
% |latex -jobname cdocscl1 \|\\
% |  "\input{childdoc.def}\childdocforward[cdocsamp]{cdocsch1}"|\\
% |latex -jobname cdocscl2 \|\\
% |  "\def\version{final}\input{childdoc.def}\childdocforward{cdocsch2}"|
% \end{tabular}
% \end{center}
% Note that the trailing backslash on each first line
% merely continues the input to the second line
% (for convenient cut ant paste).
% Furthermore, the command |latex| can be replaced by any
% of its alternative versions such as |pdflatex|.
%
% %%%%%%%%%%%%%%%%%%%%%%%%%%%%%%%%%%%%%%%%%%%%%%%%%%%%%%%%%%%%%%%%%%%%%%%%%%%%%%
% %%%%%%%%%%%%%%%%%%%%%%%%%%%%%%%%%%%%%%%%%%%%%%%%%%%%%%%%%%%%%%%%%%%%%%%%%%%%%%
% \section{Implementation}
%\iffalse
%<*package>
%\fi
%
% This section describes the definitions file |childdoc.def|.

% The definitions cannot be loaded using |\usepackage| or |\RequirePackage|
% which has a mechanism to prevent loading a style file more than once.
% When loading the definitions by means of |\input|
% multiple instances have to be prevented manually:
%\iffalse
%This code needs to be before the `\ProvidesFile' directive
%which is defined at the beginning of this file.
%Therefore it is also placed there and commented out here.
%</package>
%<*discard>
%\fi
%    \begin{macrocode}
\ifdefined\childdocmain\endinput\fi
%    \end{macrocode}
%\iffalse
%</discard>
%<*package>
%\fi
%
% \macro{\ifchilddoc}
% \macro{\ifchilddocmanual}
% The conditional |\ifchilddoc| tells whether a
% child (true) or main (false) document is being compiled.
% The conditional |\ifchilddocmanual| tells whether
% the |\includeonly| mechanism is used (false) or
% the selection of child files must be performed manually (true).
% The definitions initialise to false:
%    \begin{macrocode}
\newif\ifchilddoc
\newif\ifchilddocmanual
%    \end{macrocode}

% \macro{\childdocname}
% \macro{\childdocjob}
% The macro |\childdocname| stores the name of the main document
% to be compiled. The macro |\childdocjob| stores the name of
% the document on which the \LaTeX{} compiler was originally invoked.
% The content of |\jobname| cannot be compared
% to filenames specified in the source due to different catcodes.
% The following code rescans |\jobname|, stores the result
% in |\childdocname| and saves a copy in |\childdocjob|:
%    \begin{macrocode}
\edef\childdocname{\scantokens\expandafter{\jobname\noexpand}}
\let\childdocjob\childdocname
%    \end{macrocode}

% \macro{\childdocdisable}
% The macro |\childdocdisable| prevents the main file
% from being processed more than once.
% At this stage, the main document command |\childdocmain|
% is assumed to be called once again where it should do nothing.
% Any subsequent call to it should prevent
% a secondary processing of the main document
% It overwrites the forwarding commands
% |\childdocof| and |\childdocforward|
% with empty macros to prevent further inclusions of the main document:
%    \begin{macrocode}
\newcommand{\childdocdisable}
{
  \renewcommand{\childdocmain}[1]{\renewcommand{\childdocmain}[1]{\endinput}}
  \renewcommand{\childdocof}[1]{}
  \renewcommand{\childdocby}[2][]{}
  \renewcommand{\childdocforward}[2][]{}
  \renewcommand{\childdocdisable}{}
}
%    \end{macrocode}

% \macro{\childdocmain}
% The macro |\childdocmain| is to be called at the top of the main file
% with nothing or the main filename (without extension) as argument.
% First, it breaks loops.
% If the argument is not empty and does not match |\childdocname|
% (which is set by the first inclusion of |childdoc.def|),
% |\ifchilddoc| is set to true, |\includeonly| is applied to the child file
% and |\jobname| is set to the main file
% (for proper handling of |.aux| files):
%    \begin{macrocode}
\newcommand{\childdocmain}[1]
{
  \childdocdisable\childdocmain{}
  \if?#1?\else
    \begingroup
      \def\childdoctmp{#1}
      \ifx\childdoctmp\childdocname
        \def\childdoctmp{}
      \else
        \def\childdoctmp
        {
          \childdoctrue
          \includeonly{\childdocname}
          \def\childdocjob{#1}
          \def\jobname{#1}
        }
      \fi
      \expandafter
    \endgroup
    \childdoctmp
  \fi
}
%    \end{macrocode}

% \macro{\childdocof}
% The command |\childdocof| redirects
% compilation to the main file |#1|.
%    \begin{macrocode}
\newcommand{\childdocof}[1]
{
  \childdocdisable
  \childdoctrue
  \includeonly{\childdocname}
  \def\jobname{#1}
  \def\childdocjob{#1}
  \input{#1}
}
%    \end{macrocode}

% \macro{\childdocby}
% The command |\childdocby| ....
%    \begin{macrocode}
\newcommand{\childdocby}[2][]
{
  \childdocdisable
  \childdoctrue
  \childdocmanualtrue
  \if?#1?\else
    \def\jobname{#2}
  \fi
  \def\childdocjob{#2}
  \input{#2}
  \endinput
}
%    \end{macrocode}

% \macro{\childdocforward}
% The command |\childdocforward| redirects
% compilation to the main file or
% (if the optional argument is given) a child file.
% Parameters are set as if the main file
% or a child file starting with |\childdocof| was compiled.
% Then compilation is handed over to the main file:
%    \begin{macrocode}
\newcommand{\childdocforward}[2][]
{
  \begingroup
    \if?#1?
      \def\childdoctmp
      {
        \def\childdocname{#2}
        \def\childdocjob{#2}
        \def\jobname{#2}
        \input{#2}
        \endinput
      }
    \else
      \def\childdoctmp
      {
        \childdocdisable
        \def\childdocname{#2}
        \childdoctrue
        \includeonly{#2}
        \def\childdocjob{#1}
        \def\jobname{#1}
        \input{#1}
        \endinput
      }
    \fi
    \expandafter
  \endgroup
  \childdoctmp
}
%    \end{macrocode}

% \macro{\childdocforwardprefix}
% The command |\childdocforwardprefix| redirects
% compilation to the main or a child file by means of a pattern.
% The prefix |#1| in the current filename is replaced by |#2|
% and the suffix of the current filename is kept
% (it is assumed that the filename does not contain the substring `|~~~|'
% which is used as a delimiter).
% Compilation is handed over to the new file by |\childdocforward|:
%    \begin{macrocode}
\newcommand{\childdocforwardprefix}[3][]
{
  \begingroup
    \def\childdocextract #2##1~~~{\def\childdoctmp{\childdocforward[#1]{#3##1}}}
    \expandafter\childdocextract\childdocname~~~
    \expandafter
  \endgroup
  \childdoctmp
}
%    \end{macrocode}

% \macro{\childdoc}
% The deprecated macro |\childdoc| is a legacy version of |\childdocmain|:
%    \begin{macrocode}
\newcommand{\childdoc}{\childdocmain}
%    \end{macrocode}

% \macro{\childdocredirect}
% The deprecated macro |\childdocredirect| is a legacy version
% of |\childdocforward| and |\childdocforwardprefix|:
%    \begin{macrocode}
\newcommand{\childdocredirect}[2][]
{
  \begingroup
    \if?#1?
      \def\childdoctmp{\childdocforward{#2}}
    \else
      \def\childdoctmp{\childdocforwardprefix{#1}{#2}}
    \fi
    \expandafter
  \endgroup
  \childdoctmp
}
%    \end{macrocode}

%\iffalse
%</package>
%\fi
%
\endinput

\childdocforward{cdocsamp}
%    \end{macrocode}

%\iffalse
%</sampledraft>
%\fi
%
% %%%%%%%%%%%%%%%%%%%%%%%%%%%%%%%%%%%%%%
% \paragraph{Forwarding for Final Version of the Chapters.}
%
% The following forwarding files |cdocsfn1.tex| and |cdocsfn2.tex|
% (with identical content)
% compile the final versions of the child documents
% |cdocsch1.tex| and |cdocsch2.tex|, respectively:
%\iffalse
%<*samplefinal>
%\fi
%    \begin{macrocode}
\def\version{final}
% \iffalse
%
% childdoc.dtx Copyright (C) 2017-2018 Niklas Beisert
%
% This work may be distributed and/or modified under the
% conditions of the LaTeX Project Public License, either version 1.3
% of this license or (at your option) any later version.
% The latest version of this license is in
%   http://www.latex-project.org/lppl.txt
% and version 1.3 or later is part of all distributions of LaTeX
% version 2005/12/01 or later.
%
% This work has the LPPL maintenance status `maintained'.
%
% The Current Maintainer of this work is Niklas Beisert.
%
% This work consists of the files childdoc.dtx and childdoc.ins
% and the derived files childdoc.def and cdocsamp.tex with
% cdocsch1.tex, cdocsch2.tex, cdocsdrf.tex, cdocsfn1.tex, cdocsfn2.tex.
%
%<package>\ifdefined\childdocmain\endinput\fi
%<package>\ProvidesFile{childdoc.def}[2018/12/30 v2.0 child document driver]
%<samplemain>\ProvidesFile{cdocsamp.tex}[2018/12/30 v2.0 sample for childdoc]
%<*driver>
%\ProvidesFile{childdoc.drv}[2018/12/30 v2.0 childdoc reference manual file]
\PassOptionsToClass{10pt,a4paper}{article}
\documentclass{ltxdoc}

\usepackage[margin=35mm]{geometry}
\usepackage{hyperref}
\usepackage{hyperxmp}
\usepackage[usenames]{color}

\hypersetup{colorlinks=true}
\hypersetup{pdfstartview=FitH}
\hypersetup{pdfpagemode=UseNone}
\hypersetup{pdfsource={}}
\hypersetup{pdflang={en-UK}}
\hypersetup{pdfcopyright={Copyright 2017-2018 Niklas Beisert.
  This work may be distributed and/or modified under the
  conditions of the LaTeX Project Public License, either version 1.3
  of this license or (at your option) any later version.}}
\hypersetup{pdflicenseurl={http://www.latex-project.org/lppl.txt}}
\hypersetup{pdfcontactaddress={ETH Zurich, ITP, HIT K,
  Wolfgang-Pauli-Strasse 27}}
\hypersetup{pdfcontactpostcode={8093}}
\hypersetup{pdfcontactcity={Zurich}}
\hypersetup{pdfcontactcountry={Switzerland}}
\hypersetup{pdfcontactemail={nbeisert@itp.phys.ethz.ch}}
\hypersetup{pdfcontacturl={http://people.phys.ethz.ch/\xmptilde nbeisert/}}

\newcommand{\secref}[1]{\hyperref[#1]{section \ref*{#1}}}

\parskip1ex
\parindent0pt
\let\olditemize\itemize
\def\itemize{\olditemize\parskip0pt}

\begin{document}

\title{The \textsf{childdoc} Package}
\hypersetup{pdftitle={The childdoc Package}}
\author{Niklas Beisert\\[2ex]
  Institut f\"ur Theoretische Physik\\
  Eidgen\"ossische Technische Hochschule Z\"urich\\
  Wolfgang-Pauli-Strasse 27, 8093 Z\"urich, Switzerland\\[1ex]
  \href{mailto:nbeisert@itp.phys.ethz.ch}
  {\texttt{nbeisert@itp.phys.ethz.ch}}}
\hypersetup{pdfauthor={Niklas Beisert}}
\hypersetup{pdfsubject={Manual for the LaTeX2e Package childdoc}}
\date{30 December 2018, \textsf{v2.0}}
\maketitle

\begin{abstract}\noindent
\textsf{childdoc} is a \LaTeXe{} package
that enables the direct compilation
of document sections included by |\include|
to individual files.
\end{abstract}

\begingroup
\parskip0ex
\tableofcontents
\endgroup

%%%%%%%%%%%%%%%%%%%%%%%%%%%%%%%%%%%%%%%%%%%%%%%%%%%%%%%%%%%%%%%%%%%%%%%%%%%%%%%%
%%%%%%%%%%%%%%%%%%%%%%%%%%%%%%%%%%%%%%%%%%%%%%%%%%%%%%%%%%%%%%%%%%%%%%%%%%%%%%%%
\section{Introduction}

\LaTeX{} provides a mechanism to structure a large document (such as a book)
into a main file and several child files (containing the chapters)
using the |\include| command.
This mechanism is beneficial for documents
which span hundreds of pages in order to
make the source file(s) more manageable.
Moreover, compilation can be restricted to
selected child files by means of the |\includeonly| command.
The latter feature can be used to reduce the compilation time while editing
(this was significantly more useful in the earlier days of \LaTeX{})
or to generate a smaller document which is easier to navigate.
Another application of |\includeonly| is to generate
documents consisting of selected parts of the complete document.

However, there are a few drawbacks of the plain |\include| mechanism:
\begin{itemize}
\item
The child files cannot be compiled on their own,
they can only be compiled via the main file.
A naive editing environment
(such as a text editor with an option
to have the current file processed by \LaTeX)
may require one to switch to the main file before compiling;
attempting to compile the child file produces errors.
\item
The main file must be modified (each time)
to adjust the |\includeonly| command
to the present needs. This easily leaves the main file in a messy state.
\item
The generated document will always carry the filename
of the main document. This is inconvenient if
several child files are to be compiled and
to be kept for distribution.
\end{itemize}

The present package provides a simple interface
to make child files individually compilable by \LaTeX{}.
Compiling a child file then has the same effect as compiling
the main file with an |\includeonly| command
to select the appropriate child.
Moreover the generated document will carry the name of the child
rather than the main file.
This resolves all three above issues.

This feature is meant to make the editing of books,
thesis documents and lecture notes somewhat more convenient.
However, the package can also be used efficiently for
composing a series of documents (such as exercise sheets)
which are typically distributed individually.
It then assists the author in generating the individual documents
(potentially in different versions)
as well as a document containing the collected series.
Another application is in developing style files
or other kinds of included material
where compilation of the style file could redirect
to a sample or test file.

%%%%%%%%%%%%%%%%%%%%%%%%%%%%%%%%%%%%%%%%%%%%%%%%%%%%%%%%%%%%%%%%%%%%%%%%%%%%%%%%
%%%%%%%%%%%%%%%%%%%%%%%%%%%%%%%%%%%%%%%%%%%%%%%%%%%%%%%%%%%%%%%%%%%%%%%%%%%%%%%%
\section{Usage}

First of all, the package \textsf{childdoc} is \emph{not} a standard
\LaTeXe{} |.sty| style file! Therefore it needs to be invoked in
a non-standard way.

%%%%%%%%%%%%%%%%%%%%%%%%%%%%%%%%%%%%%%%%%%%%%%%%%%%%%%%%%%%%%%%%%%%%%%%%%%%%%%%%
\subsection{Included Files}
\label{sec:include}

%%%%%%%%%%%%%%%%%%%%%%%%%%%%%%%%%%%%%%%%
\DescribeMacro{\childdocmain}
To use the package, add the commands
\begin{center}
\begin{tabular}{l}
|\input{childdoc.def}|\\
|\childdocmain{}|\\
\end{tabular}
\end{center}
at the very top of the main \LaTeX{} file,
in particular \emph{before} the |\documentclass| statement!
The argument of |\childdocmain| should be left empty
(but it must be present).

%%%%%%%%%%%%%%%%%%%%%%%%%%%%%%%%%%%%%%%%
\DescribeMacro{\childdocof}
Furthermore, add the commands
\begin{center}
\begin{tabular}{l}
|\input{childdoc.def}|\\
|\childdocof{|\textit{main}|}|\\
\end{tabular}
\end{center}
at the top of every child file \textit{child}
which is included by |\include{|\textit{child}|}|
from within the main file
(or at least for those files to be compiled individually).
The argument \textit{main} must be the filename of the main file.

There are a couple of
considerations in setting up the main and child documents:

%%%%%%%%%%%%%%%%%%%%%%%%%%%%%%%%%%%%%%%%
\paragraph{Restrictions.}

Please note the following restrictions:
\begin{itemize}
\item
|\childdocmain| must be called with one argument \textit{main}
to ensure compatibility with earlier version of the package.
It must either be empty (|\childdocmain{}|)
or precisely match the filename of the main file in which it is specified.
See \secref{sec:detection} for further information.
\item
The filename \textit{main} must be specified without the |.tex| extension.
\item
The filename \textit{main} is case sensitive
(even in case-insensitive file systems)
due to internal string comparison.
\item
The argument \textit{main} should be fully expanded, it cannot be a macro.
\item
Subdirectories and special characters should be avoided in filenames.
\item
The command |\childdocmain{|\textit{main}|}| must be followed by a whitespace.
It should not be followed immediately by another command
or by a comment mark `|%|'.
This is because the \TeX{} parser reads the token immediately following
the argument of |\childdocmain| and puts it
at the beginning of every child section;
however, a white\-space is ignored.
\end{itemize}

%%%%%%%%%%%%%%%%%%%%%%%%%%%%%%%%%%%%%%%%
\paragraph{Content of Main File.}

It is advisable to place all content in the child files included by |\include|.
Any output contained in the main file will appear in all child documents
unless suppressed manually;
it cannot be suppressed automatically by the |\includeonly| directive
and thus should normally be avoided.
A method to include some content in the main file
by means of conditional processing is described in \secref{sec:conditional}.

%%%%%%%%%%%%%%%%%%%%%%%%%%%%%%%%%%%%%%%%
\paragraph{Page Numbering.}

When only a part of the document is compiled,
the appropriate numbering of pages
(as well as other status parameters)
is determined from the |.aux| files.
The latter contain information from previous passes.
However this information needs to propagate through
all intermediate child documents.
Therefore the page numbering in child documents may well
be inconsistent until the complete document is compiled at least once.

A useful (if unconventional) way to always ensure a consistent
page numbering is to restart the numbering in each child document
and denote the pages by `\textit{child}|.|\textit{page}'
where \textit{child} represents the chapter/section number of the child file.
This can be achieved by the command
|\numberwithin{page}{|\textit{child}|}|
of the \textsf{amsmath} package
where \textit{child} can be |chapter| or |section|
depending on the chosen structuring.
Alternatively, one can modify the macro |\thepage| appropriately
and reset the counter |page| at the start of each child file.

%%%%%%%%%%%%%%%%%%%%%%%%%%%%%%%%%%%%%%%%%%%%%%%%%%%%%%%%%%%%%%%%%%%%%%%%%%%%%%%%
\subsection{Conditional Processing}
\label{sec:conditional}

The package provides a mechanism to compile different versions
of a document. To customise the versions further some conditional processing
can come in handy to distinguish which version is being compiled.
The package provides two macros to describe the compilation context:

%%%%%%%%%%%%%%%%%%%%%%%%%%%%%%%%%%%%%%%%
\DescribeMacro{\ifchilddoc}
The conditional |\ifchilddoc| distinguishes between the compilation of
child documents and the main document:
%
\begin{center}
|\ifchilddoc |\textit{child-code}| |[|\||else |\textit{main-code}]| \||fi|
\end{center}

%%%%%%%%%%%%%%%%%%%%%%%%%%%%%%%%%%%%%%%%
\DescribeMacro{\childdocname}
\DescribeMacro{\childdocjob}
The macro |\childdocname| contains the filename (without extension)
of the main or child file being processed.
Note that |\childdocjob| will always contain the name of the main file.

%%%%%%%%%%%%%%%%%%%%%%%%%%%%%%%%%%%%%%%%
\paragraph{Title Page.}

Conditional processing can be used to include a title or banner page
in the main document when proper precautions are taken.
Importantly, the code in the main file should ensure that the page counter
(as well as other status parameters which are stored in the |.aux| files)
takes the same value after the conditional processing.
Otherwise the page numbers may take divergent values
depending on which part is compiled.

For example, a title page could be declared by:
%
\begin{center}
\begin{tabular}{l}
|\ifchilddoc\||else|\\
|\addtocounter{page}{-1}|\\
\textit{code for title page}\\
|\newpage|\\
|\||fi|
\end{tabular}
\end{center}
%
A banner page for the child documents can be generated by:
%
\begin{center}
\begin{tabular}{l}
|\ifchilddoc|\\
|\addtocounter{page}{-1}|\\
\textit{code for banner page}\\
|\newpage|\\
|\||fi|
\end{tabular}
\end{center}
%
Here one could write a message such as:
\begin{center}
|This is the part \childdocname{} of \childdocjob{}.|
\end{center}

%%%%%%%%%%%%%%%%%%%%%%%%%%%%%%%%%%%%%%%%%%%%%%%%%%%%%%%%%%%%%%%%%%%%%%%%%%%%%%%%
\subsection{Flags}
\label{sec:flags}

The package makes it easy to generate different versions
of the main or child documents.
To this end compilation flags can be defined
and assigned different default values.
They will be particularly useful in conjunction
with the forwarding mechanism described in \secref{sec:forward}.

For example, it may be useful to have a flag |\version|
which can be set to |draft| or |final|.
The document source will contain some conditional code
depending on the value of |\version|.
Suppose further, the flag should default to |final| for the main file
and to |draft| for child files
which is a natural assignment for editing the document.
This is achieved by placing the following code
in the preamble of the main document
(below the |\childdocmain| directive):
%
\begin{center}
\begin{tabular}{l}
|\ifchilddoc|\\
|\providecommand{\version}{draft}|\\
|\||else|\\
|\providecommand{\version}{final}|\\
|\||fi|
\end{tabular}
\end{center}
%
The definition by |\providecommand| makes sure
that previous definitions are not overwritten.
Further statements |\providecommand{\version}{...}|
can thus be added before the above code to override it.

For the main file, one might add a line
(between |\childdocmain| and the above block)
%
\begin{center}
|%\ifchilddoc\||else\providecommand{\version}{draft}\||fi|
\end{center}
%
which can be uncommented to produce a draft version.
Likewise one can add a line to the very top of a child file
(above the |\childdocof{|\textit{main}|}| directive)
%
\begin{center}
|%\providecommand{\version}{final}|
\end{center}
%
which can be uncommented to produce the final version of this child document.

%%%%%%%%%%%%%%%%%%%%%%%%%%%%%%%%%%%%%%%%%%%%%%%%%%%%%%%%%%%%%%%%%%%%%%%%%%%%%%%%
\subsection{Forwarding}
\label{sec:forward}

Different versions of the main or child documents
using compilation flags as described in \secref{sec:flags}
can be (permanently) stored in different files
for convenient compilation, viewing and distribution.
To this end, the package defines a command
to pass on compilation to a different file:

%%%%%%%%%%%%%%%%%%%%%%%%%%%%%%%%%%%%%%%%
\DescribeMacro{\childdocforward}
The command |\childdocforward| redirects processing to
another source file:
%
\begin{center}
\begin{tabular}{l}
|\input{childdoc.def}|\\
|\childdocforward[|\textit{main}|]{|\textit{dest}|}|\\
\end{tabular}
\end{center}
%
The argument \textit{dest} is the destination file
(without extension).
It should be the main file or one of the child files.
Note that further \textsf{childdoc} directives
such as |\childdocof| and |\childdocforward|
in the indicated file will be processed in this form.
The optional argument \textit{main}
passes on directly to the main file \textit{main}
while pretending to compile the child \textit{dest}.
This form behaves as if \textit{dest}
issues |\childdocof{|\textit{main}|}| right away,
and no further \textsf{childdoc} directives will be processed.

%%%%%%%%%%%%%%%%%%%%%%%%%%%%%%%%%%%%%%%%
\DescribeMacro{\...prefix}
In the alternative form |\childdocforwardprefix|,
%
\begin{center}
\begin{tabular}{l}
|\input{childdoc.def}|\\
|\childdocforwardprefix[|\textit{main}|]{|\textit{prefix}|}{|\textit{dest}|}|
\end{tabular}
\end{center}
%
the destination file is determined by a pattern
depending on the current file:
To make this work, the current file must be called
`{\textit{prefix}\hspace{0.2em}\textit{suffix}}'
with \textit{prefix} matching precisely the argument.
Processing is then passed on to the file
`{\textit{dest}\hspace{0.2em}\textit{suffix}}'.
Surely, the same effect is achieved by
directly specifying the
argument `{\textit{dest}\hspace{0.2em}\textit{suffix}}'
in the first form.
However, that requires to set up a different file
for each child. With the alternative form of the command
all these files can have exactly the same content
which simplifies setting them up and maintaining them.

For example, the following file |draft.tex|
with a compilation flag |\version| as described in \secref{sec:flags}
compiles the main document as a draft:
%
\begin{center}
\begin{tabular}{l}
|\def\version{draft}|\\
|\input{childdoc.def}|\\
|\childdocforward{|\textit{main}|}|
\end{tabular}
\end{center}
%
Likewise, the following files |final|\textit{nn}|.tex|
compile the final version of the child document
|child|\textit{nn}|.tex|:
%
\begin{center}
\begin{tabular}{l}
|\def\version{final}|\\
|\input{childdoc.def}|\\
|\childdocforwardprefix{final}{child}|
\end{tabular}
\end{center}
%

Note that when several versions of a main file and/or of each child file
are to be generated, it may be convenient to set up a |Makefile| or
shell script to automatise the process.

%%%%%%%%%%%%%%%%%%%%%%%%%%%%%%%%%%%%%%%%%%%%%%%%%%%%%%%%%%%%%%%%%%%%%%%%%%%%%%%%
\subsection{Command Line Processing}
\label{sec:commandline}

The effect of redirection files can also be achieved by invoking
the \LaTeX{} compiler with a more elaborate command line.
Most conveniently this should be done as part
of a shell script or a |Makefile|.

When using \textsf{childdoc} in the main file, the following
command lines effectively perform a redirection
(note that depending on the shell being used,
backslashes may have to be doubled: `|\|' $\to$ `|\\|'):
%
\begin{center}
|... -jobname "|\textit{target}|" |\\|"|[\textit{flags}]%
|\input{childdoc.def}\childdocforward[|\textit{main}|]{|\textit{dest}|}"|
\end{center}
%
Here \textit{target} is the name of the output file,
\textit{main} is the name of the main file
and \textit{dest} is the name of the main or child file to be processed
(all filenames without extensions).
The optional argument \textit{main} can be omitted
if \textit{main} matches \textit{dest}.
Optionally, compilation \textit{flags} can be defined via |\def| commands.
This command line makes the \TeX{} engine believe
it is compiling the file \textit{target}
whose content is specified as the latter parameter.
The provided code then forwards the processing to
\textit{main} or \textit{dest} as described in \secref{sec:forward}.

%%%%%%%%%%%%%%%%%%%%%%%%%%%%%%%%%%%%%%%%%%%%%%%%%%%%%%%%%%%%%%%%%%%%%%%%%%%%%%%%
\subsection{Include by Input}
\label{sec:input}

Including child documents by |\include| has some restrictions by design.
Most notably, the content of a child document always occupies
its own set of pages; pages cannot be shared between child documents.
Usually, this behaviour makes perfect sense
because each child document contain an essential part of the document.
However, in some situations it may be desirable to compose
a document from a collection of parts
without having mandatory page breaks between then.
For this case, the package
provides a mechanism to include parts
by |\input| which can also be processed individually.
However, by construction this mechanism
requires manual handling of the content to be output.

%%%%%%%%%%%%%%%%%%%%%%%%%%%%%%%%%%%%%%%%
\DescribeMacro{\ifchilddocmanual}
The main file should be prepared as usual, see \secref{sec:include}.
However, the document body must make a distinction
between processing of an individual part and of the main document, e.g.:
%
\begin{center}
\begin{tabular}{l}
|\ifchilddocmanual|\\
|\input{\childdocname}|\\
|\||else|\\
\textit{document body with }|\input{|\textit{part}|}|\\
|\||fi|
\end{tabular}
\end{center}
%
The conditional |\ifchilddocmanual| is true whenever
a part to be included by |\input| is being compiled,
and the name of the part is stored in |\childdocname|.

%%%%%%%%%%%%%%%%%%%%%%%%%%%%%%%%%%%%%%%%
\DescribeMacro{\childdocby}
Each part to be included by |\input| should start with:
%
\begin{center}
\begin{tabular}{l}
|\input{childdoc.def}|\\
|\childdocby{|\textit{main}|}|\\
\end{tabular}
\end{center}
%
The directive |\childdocby| is similar to |\childdocof|
described in \secref{sec:include},
but the subsequent selection of content must be done manually.
To that end, both |\ifchilddoc| and |\ifchilddocmanual|
will be true upon processing of a part,
and the name of the part is stored in |\childdocname|.
Note that |\jobname| will be set to the filename of the current part
so that each part receives an individual |.aux| file
that does not interfere with the |.aux| file(s) of the main document.
This behaviour can be altered by the alternative form
|\childdocby[*]{|\textit{main}|}| (with a non-empty optional argument)
which uses the |.aux| file of the main document
by setting |\jobname| to \textit{main}.

%%%%%%%%%%%%%%%%%%%%%%%%%%%%%%%%%%%%%%%%%%%%%%%%%%%%%%%%%%%%%%%%%%%%%%%%%%%%%%%%
\subsection{Driver Development}
\label{sec:driver}

The \textsf{childdoc} mechanism can also be use for the development
of definition files such as \LaTeX{} styles or classes.
This case differs from the above setup with multiple parts
included by |\include| in that no |\includeonly| should be invoked.
This can be achieved by starting the include file
(before |\ProvidesPackage|) with:
%
\begin{center}
\begin{tabular}{l}
|\input{childdoc.def}|\\
|\childdocforward{|\textit{main}|}|\\
\end{tabular}
\end{center}
%
or alternatively with:
%
\begin{center}
\begin{tabular}{l}
|\input{childdoc.def}|\\
|\childdocby{|\textit{main}|}|\\
\end{tabular}
\end{center}
%
Both forms have slightly different effects as described above.
The main file is prepared as usual, see \secref{sec:include}.

%%%%%%%%%%%%%%%%%%%%%%%%%%%%%%%%%%%%%%%%%%%%%%%%%%%%%%%%%%%%%%%%%%%%%%%%%%%%%%%%
\subsection{Legacy Detection}
\label{sec:detection}

The directive |\childdocmain| in the main file can detect
whether the complete document or merely a child is to be compiled
even without using the directive |\childdocof|.
This method is deprecated because it is less robust
and there is no compelling reason to use it;
it is merely provided for backward compatibility
and it may be removed in future versions.

If the detection mechanism is to be used,
it is mandatory to correctly specify
the filename of the main file as the argument of |\childdocmain|:
%
\begin{center}
\begin{tabular}{l}
|\input{childdoc.def}|\\
|\childdocmain{|\textit{main}|}|\\
\end{tabular}
\end{center}
%
If |\jobname| does not match the argument \textit{main} of |\childdocmain|,
it is assumed that |\jobname| points to the child file to be compiled.
When using |\childdocmain| with the main file specified as argument,
it suffices to start a child file
with just |\input{|\textit{main}|}|
without loading of the package and using |\childdocof|.
If instead all processing is done
with the appropriate \textsf{childdoc} directives,
the argument of \textit{main} of |\childdocmain| can be empty.

An alternative version of the command line processing described
in \secref{sec:commandline} using the detection mechanism reads:
%
\begin{center}
|... -jobname "|\textit{target}|" "|[\textit{flags}]%
[|\def\jobname{|\textit{dest}|}|]|\input{|\textit{main}|}"|
\end{center}

%%%%%%%%%%%%%%%%%%%%%%%%%%%%%%%%%%%%%%%%%%%%%%%%%%%%%%%%%%%%%%%%%%%%%%%%%%%%%%%%
\subsection{Manual Code}
\label{sec:manual}

In case one cannot be certain whether the definitions file |childdoc.def|
is installed on the target \TeX{} distribution
and one prefers not to ship it,
it is conceivable to paste a few relevant commands into the sources.

To that end, drop all statements |\input{childdoc.def}|
and perform the replacements as outlined below.
Instead of |\childdocmain{|\textit{main}|}| add the following code
to the top of the main file:
%
\begin{center}
\begin{tabular}{l}
|\||ifdefined\childdocname\endinput\||fi\newif\ifchilddoc|\\
|\edef\childdocname{\scantokens\expandafter{\jobname\noexpand}}|\\
|\def\childdocmain{|\textit{main}|}\||ifx\childdocmain\childdocname\||else|\\
|\childdoctrue\includeonly{\childdocname}\let\jobname\childdocmain\||fi|\\
\end{tabular}
\end{center}
%
Instead of |\childdocof{|\textit{main}|}| just include the main file
at the top of each child file:
%
\begin{center}
|\input{|\textit{main}|}|
\end{center}
%
A simple redirection |\childdocforward{|\textit{dest}|}| is achieved by:
%
\begin{center}
|\def\jobname{|\textit{dest}|}\input{\jobname}|
\end{center}
%
The redirection with prefix
|\childdocforwardprefix[|\textit{prefix}|]{|\textit{dest}|}|
is accomplished by:
%
\begin{center}
\begin{tabular}{l}
|{\edef\jobname{\scantokens\expandafter{\jobname\noexpand}}|\\
|\def\redirectjob |\textit{prefix}|#1~~~{\gdef\jobname{|\textit{dest}|#1}}|\\
|\expandafter\redirectjob\jobname~~~}\input{\jobname}|
\end{tabular}
\end{center}

In an alternative approach,
child documents can be compiled by a specific command line
without additional code or specific definitions:
%
\begin{center}
|... -jobname "|\textit{target}|" "|[\textit{flags}]%
|\includeonly{|\textit{dest}|}\input{|\textit{main}|}"|
\end{center}
%

%%%%%%%%%%%%%%%%%%%%%%%%%%%%%%%%%%%%%%%%%%%%%%%%%%%%%%%%%%%%%%%%%%%%%%%%%%%%%%%%
%%%%%%%%%%%%%%%%%%%%%%%%%%%%%%%%%%%%%%%%%%%%%%%%%%%%%%%%%%%%%%%%%%%%%%%%%%%%%%%%
\section{Information}

%%%%%%%%%%%%%%%%%%%%%%%%%%%%%%%%%%%%%%%%%%%%%%%%%%%%%%%%%%%%%%%%%%%%%%%%%%%%%%%%
\subsection{Copyright}

Copyright \copyright{} 2017--2018 Niklas Beisert

This work may be distributed and/or modified under the
conditions of the \LaTeX{} Project Public License, either version 1.3
of this license or (at your option) any later version.
The latest version of this license is in
  \url{http://www.latex-project.org/lppl.txt}
and version 1.3 or later is part of all distributions of \LaTeX{}
version 2005/12/01 or later.

This work has the LPPL maintenance status `maintained'.

The Current Maintainer of this work is Niklas Beisert.

This work consists of the files |README.txt|, |childdoc.ins| and |childdoc.dtx|
as well as the derived files |childdoc.def|, |cdocsamp.tex|
with |cdocsch1.tex|, |cdocsch2.tex|, |cdocspt3.tex|, |cdocspt4.tex|,
|cdocsdrf.tex|, |cdocsfn1.tex|, |cdocsfn2.tex|
as well as |childdoc.pdf|.

%%%%%%%%%%%%%%%%%%%%%%%%%%%%%%%%%%%%%%%%%%%%%%%%%%%%%%%%%%%%%%%%%%%%%%%%%%%%%%%%
\subsection{Files and Installation}

The package consists of the files:
%
\begin{center}
\begin{tabular}{ll}
    |README.txt|   & readme file \\
    |childdoc.ins| & installation file \\
    |childdoc.dtx| & source file \\
    |childdoc.def| & definition file \\
    |cdocsamp.tex| & sample main file \\
    |cdocsch1.tex| & sample include file \\
    |cdocsch2.tex| & sample include file \\
    |cdocspt3.tex| & sample part file \\
    |cdocspt4.tex| & sample part file \\
    |cdocsdrf.tex| & sample redirection file \\
    |cdocsfn1.tex| & sample redirection file \\
    |cdocsfn2.tex| & sample redirection file \\
    |childdoc.pdf| & manual
\end{tabular}
\end{center}
%
The distribution consists of the files
|README.txt|, |childdoc.ins| and |childdoc.dtx|.
%
\begin{itemize}
\item
Run (pdf)\LaTeX{} on |childdoc.dtx|
to compile the manual |childdoc.pdf| (this file).
\item
Run \LaTeX{} on |childdoc.ins| to create the definitions file |childdoc.def|
and the sample |cdocsamp.tex| with include files
|cdocsch1.tex|, |cdocsch2.tex|, |cdocspt3.tex|, |cdocspt4.tex|,
|cdocsdrf.tex|, |cdocsfn1.tex|, |cdocsfn2.tex|.
Then copy the file |childdoc.def| to an appropriate directory of your \LaTeX{}
distribution, e.g.\ \textit{texmf-root}|/tex/latex/childdoc|.
\end{itemize}

%%%%%%%%%%%%%%%%%%%%%%%%%%%%%%%%%%%%%%%%%%%%%%%%%%%%%%%%%%%%%%%%%%%%%%%%%%%%%%%%
\subsection{Related CTAN Packages}

There are several other packages which offer a similar functionality:
%
\begin{itemize}
\item
The packages
\href{http://ctan.org/pkg/docmute}{\textsf{docmute}},
\href{http://ctan.org/pkg/includex}{\textsf{includex}} and
\href{http://ctan.org/pkg/standalone}{\textsf{standalone}}
provide commands to include only the document body of
a child file thus allowing both files to be compiled individually.
\item
The packages \href{http://ctan.org/pkg/subdocs}{\textsf{subdocs}}
and \href{http://ctan.org/pkg/subfiles}{\textsf{subfiles}}
provide structures in which the main and child documents can be
encapsulated and allowing them to be compiled individually.
The inclusion mechanism is different from the conventional |\include|.
\item
The package \href{http://ctan.org/pkg/combine}{\textsf{combine}}
is an elaborate solution to combine several documents into one.
\end{itemize}
%
See also the CTAN topic \href{http://ctan.org/topic/subdocs}{\textsf{subdocs}}
for further related packages.
The present package differs from the above solutions in that
a document structure constructed with the conventional |\include| mechanism
just needs two extra commands at the top of every file
such that all constituent files can be compiled individually.

%%%%%%%%%%%%%%%%%%%%%%%%%%%%%%%%%%%%%%%%%%%%%%%%%%%%%%%%%%%%%%%%%%%%%%%%%%%%%%%%
%\subsection{Feature Suggestions}
%
%The following is a list of features which may be useful for future
%versions of this package:
%%
%\begin{itemize}
%\item
%\ldots
%\end{itemize}

%%%%%%%%%%%%%%%%%%%%%%%%%%%%%%%%%%%%%%%%%%%%%%%%%%%%%%%%%%%%%%%%%%%%%%%%%%%%%%%%
\subsection{Revision History}

%%%%%%%%%%%%%%%%%%%%%%%%%%%%%%%%%%%%%%%%
\paragraph{v2.0:} 2018/12/30

\begin{itemize}
\item
immediate forward processing
\item
added |\childdocby| mechanism
\item
manual restructured
\end{itemize}

%%%%%%%%%%%%%%%%%%%%%%%%%%%%%%%%%%%%%%%%
\paragraph{v1.6:} 2018/01/17

\begin{itemize}
\item
application for development of include files
\item
corrections to manual
\end{itemize}

%%%%%%%%%%%%%%%%%%%%%%%%%%%%%%%%%%%%%%%%
\paragraph{v1.5:} 2017/05/21

\begin{itemize}
\item
more complete structuring introduced
\item
|\childdocof| introduced
\item
|\childdoc| renamed to |\childdocmain|
\item
|\childredirect| renamed to |\childdocforward| and |\childdocforwardprefix|
and functionality expanded
\end{itemize}

%%%%%%%%%%%%%%%%%%%%%%%%%%%%%%%%%%%%%%%%
\paragraph{v1.0:} 2017/04/27

\begin{itemize}
\item
manual and install package
\item
first version published on CTAN
\end{itemize}

%%%%%%%%%%%%%%%%%%%%%%%%%%%%%%%%%%%%%%%%
\paragraph{v0.6:} 2017/04/26

\begin{itemize}
\item
redirection mechanism added
\end{itemize}

%%%%%%%%%%%%%%%%%%%%%%%%%%%%%%%%%%%%%%%%
\paragraph{v0.5:} 2017/04/26

\begin{itemize}
\item
functionality in definition file
\end{itemize}


%%%%%%%%%%%%%%%%%%%%%%%%%%%%%%%%%%%%%%%%%%%%%%%%%%%%%%%%%%%%%%%%%%%%%%%%%%%%%%%%
%%%%%%%%%%%%%%%%%%%%%%%%%%%%%%%%%%%%%%%%%%%%%%%%%%%%%%%%%%%%%%%%%%%%%%%%%%%%%%%%
%%%%%%%%%%%%%%%%%%%%%%%%%%%%%%%%%%%%%%%%%%%%%%%%%%%%%%%%%%%%%%%%%%%%%%%%%%%%%%%%
\appendix

\settowidth\MacroIndent{\rmfamily\scriptsize 000\ }

 \DocInput{childdoc.dtx}

\end{document}
%</driver>
% \fi
%
% %%%%%%%%%%%%%%%%%%%%%%%%%%%%%%%%%%%%%%%%%%%%%%%%%%%%%%%%%%%%%%%%%%%%%%%%%%%%%%
% %%%%%%%%%%%%%%%%%%%%%%%%%%%%%%%%%%%%%%%%%%%%%%%%%%%%%%%%%%%%%%%%%%%%%%%%%%%%%%
% \section{Sample}
%\iffalse
%<*samplemain>
%\fi
%
% The following presents a sample document
% with two chapters, two parts, a title page,
% a compile flag as well as three forwarding files to set the flag.
% It consists of eight |.tex| files:
% \begin{center}
% \begin{tabular}{ll}
% |cdocsamp.tex|&main file\\
% |cdocsch1.tex|&include file for chapter 1\\
% |cdocsch2.tex|&include file for chapter 2\\
% |cdocspt3.tex|&include file for part 3\\
% |cdocspt4.tex|&include file for part 4\\
% |cdocsdrf.tex|&forwarding file for main file in draft mode\\
% |cdocsfi1.tex|&forwarding file for final version of chapter 1\\
% |cdocsfi2.tex|&forwarding file for final version of chapter 2\\
% \end{tabular}
% \end{center}
% Each of the eight files can be compiled directly by the \LaTeX{} compiler.
%
% %%%%%%%%%%%%%%%%%%%%%%%%%%%%%%%%%%%%%%
% \paragraph{Main File.}
%
% The main file is called |cdocsamp.tex|.
%
% Load the \textsf{childdoc} definitions and
% declare the filename for the main document:
%    \begin{macrocode}
\input{childdoc.def}
\childdocmain{}
%    \end{macrocode}

% Optional override for |\version| flag:
%    \begin{macrocode}
%%\ifchilddoc\else\providecommand{\version}{draft}\fi
%    \end{macrocode}

% Define the default values for the |\version| flag
% (|final| for the main file and |draft| for childs):
%    \begin{macrocode}
\ifchilddoc
\providecommand{\version}{draft}
\else
\providecommand{\version}{final}
\fi
%    \end{macrocode}

% Load the standard document class:
%    \begin{macrocode}
\documentclass[12pt]{article}
%    \end{macrocode}

% Start the document body:
%    \begin{macrocode}
\begin{document}
%    \end{macrocode}

% Declare a title page.
% Print title, part of document being processed and version flag:
%    \begin{macrocode}
\addtocounter{page}{-1}
\begin{center}
{\LARGE\bfseries{}childdoc example\par}
\vspace{1cm}
\ifchilddoc
\ifchilddocmanual part\else chapter\fi:
`\childdocname' of `\childdocjob'\par
\else
main document: `\childdocjob'\par
\fi
version: \version\par
\end{center}
\newpage
%    \end{macrocode}

% Manually include selected file,
% otherwise process as usual:
%    \begin{macrocode}
\ifchilddocmanual
\section*{part `\childdocname'}
\input{\childdocname}
\else
%    \end{macrocode}

% Include the two chapters:
%    \begin{macrocode}
\include{cdocsch1}
\include{cdocsch2}
%    \end{macrocode}

% Include the two parts unless only chapters should be displayed:
%    \begin{macrocode}
\ifchilddoc\else
\section{part three}
\input{cdocspt3}
\section{part four}
\input{cdocspt4}
\fi
%    \end{macrocode}

% Process as usual until here:
%    \begin{macrocode}
\fi
%    \end{macrocode}

% End of document body:
%    \begin{macrocode}
\end{document}
%    \end{macrocode}
%\iffalse
%</samplemain>
%\fi
%
% %%%%%%%%%%%%%%%%%%%%%%%%%%%%%%%%%%%%%%
% \paragraph{Chapter Include Files.}
%
% The include files are called |cdocsch1.tex| and |cdocsch2.tex|.
%
%\iffalse
%<*samplechap1|samplechap2>
%\fi

% Optional override for |\version| flag:
%    \begin{macrocode}
%%\providecommand{\version}{final}
%    \end{macrocode}

% Include the main document:
%    \begin{macrocode}
\input{childdoc.def}
\childdocof{cdocsamp}
%    \end{macrocode}

%\iffalse
%</samplechap1|samplechap2>
%\fi
%
%\iffalse
%<*samplechap1>
%\fi
% Some text for chapter 1:
%    \begin{macrocode}
\section{one}
some text in chapter one
%    \end{macrocode}

%\iffalse
%</samplechap1>
%\fi
% Some text for chapter 2:
%\iffalse
%<*samplechap2>
%\fi
%    \begin{macrocode}
\section{two}
more text in chapter two
%    \end{macrocode}

%\iffalse
%</samplechap2>
%\fi
%
% %%%%%%%%%%%%%%%%%%%%%%%%%%%%%%%%%%%%%%
% \paragraph{Part Include Files.}
%
% The include files are called |cdocspt3.tex| and |cdocspt4.tex|.
%
%\iffalse
%<*samplepart3|samplepart4>
%\fi

% Optional override for |\version| flag:
%    \begin{macrocode}
%%\providecommand{\version}{final}
%    \end{macrocode}

% Include the main document:
%    \begin{macrocode}
\input{childdoc.def}
\childdocby{cdocsamp}
%    \end{macrocode}

%\iffalse
%</samplepart3|samplepart4>
%\fi
%
%\iffalse
%<*samplepart3>
%\fi
% Some text for part 3:
%    \begin{macrocode}
some text in part three
%    \end{macrocode}

%\iffalse
%</samplepart3>
%\fi
% Some text for part 4:
%\iffalse
%<*samplepart4>
%\fi
%    \begin{macrocode}
more text in part four
%    \end{macrocode}

%\iffalse
%</samplepart4>
%\fi
%
% %%%%%%%%%%%%%%%%%%%%%%%%%%%%%%%%%%%%%%
% \paragraph{Forwarding for a Complete Draft.}
%
% The following forwarding file |cdocsdrf.tex|
% compiles the main document in draft mode:
%\iffalse
%<*sampledraft>
%\fi
%    \begin{macrocode}
\def\version{draft}
\input{childdoc.def}
\childdocforward{cdocsamp}
%    \end{macrocode}

%\iffalse
%</sampledraft>
%\fi
%
% %%%%%%%%%%%%%%%%%%%%%%%%%%%%%%%%%%%%%%
% \paragraph{Forwarding for Final Version of the Chapters.}
%
% The following forwarding files |cdocsfn1.tex| and |cdocsfn2.tex|
% (with identical content)
% compile the final versions of the child documents
% |cdocsch1.tex| and |cdocsch2.tex|, respectively:
%\iffalse
%<*samplefinal>
%\fi
%    \begin{macrocode}
\def\version{final}
\input{childdoc.def}
\childdocforwardprefix[cdocsamp]{cdocsfn}{cdocsch}
%    \end{macrocode}

%\iffalse
%</samplefinal>
%\fi
%
% %%%%%%%%%%%%%%%%%%%%%%%%%%%%%%%%%%%%%%
% \paragraph{Command Line Processing.}
%
% The following three command lines generate the output files
% |cdocscld|, |cdocscl1| and |cdocscl2|
% which should be identical to
% |cdocsdrf|, |cdocsch1| and |cdocsfn2|, respectively:
% \begin{center}
% \begin{tabular}{l}
% |latex -jobname cdocscld \|\\
% |  "\def\version{draft}\input{childdoc.def}\childdocforward{cdocsamp}"|\\
% |latex -jobname cdocscl1 \|\\
% |  "\input{childdoc.def}\childdocforward[cdocsamp]{cdocsch1}"|\\
% |latex -jobname cdocscl2 \|\\
% |  "\def\version{final}\input{childdoc.def}\childdocforward{cdocsch2}"|
% \end{tabular}
% \end{center}
% Note that the trailing backslash on each first line
% merely continues the input to the second line
% (for convenient cut ant paste).
% Furthermore, the command |latex| can be replaced by any
% of its alternative versions such as |pdflatex|.
%
% %%%%%%%%%%%%%%%%%%%%%%%%%%%%%%%%%%%%%%%%%%%%%%%%%%%%%%%%%%%%%%%%%%%%%%%%%%%%%%
% %%%%%%%%%%%%%%%%%%%%%%%%%%%%%%%%%%%%%%%%%%%%%%%%%%%%%%%%%%%%%%%%%%%%%%%%%%%%%%
% \section{Implementation}
%\iffalse
%<*package>
%\fi
%
% This section describes the definitions file |childdoc.def|.

% The definitions cannot be loaded using |\usepackage| or |\RequirePackage|
% which has a mechanism to prevent loading a style file more than once.
% When loading the definitions by means of |\input|
% multiple instances have to be prevented manually:
%\iffalse
%This code needs to be before the `\ProvidesFile' directive
%which is defined at the beginning of this file.
%Therefore it is also placed there and commented out here.
%</package>
%<*discard>
%\fi
%    \begin{macrocode}
\ifdefined\childdocmain\endinput\fi
%    \end{macrocode}
%\iffalse
%</discard>
%<*package>
%\fi
%
% \macro{\ifchilddoc}
% \macro{\ifchilddocmanual}
% The conditional |\ifchilddoc| tells whether a
% child (true) or main (false) document is being compiled.
% The conditional |\ifchilddocmanual| tells whether
% the |\includeonly| mechanism is used (false) or
% the selection of child files must be performed manually (true).
% The definitions initialise to false:
%    \begin{macrocode}
\newif\ifchilddoc
\newif\ifchilddocmanual
%    \end{macrocode}

% \macro{\childdocname}
% \macro{\childdocjob}
% The macro |\childdocname| stores the name of the main document
% to be compiled. The macro |\childdocjob| stores the name of
% the document on which the \LaTeX{} compiler was originally invoked.
% The content of |\jobname| cannot be compared
% to filenames specified in the source due to different catcodes.
% The following code rescans |\jobname|, stores the result
% in |\childdocname| and saves a copy in |\childdocjob|:
%    \begin{macrocode}
\edef\childdocname{\scantokens\expandafter{\jobname\noexpand}}
\let\childdocjob\childdocname
%    \end{macrocode}

% \macro{\childdocdisable}
% The macro |\childdocdisable| prevents the main file
% from being processed more than once.
% At this stage, the main document command |\childdocmain|
% is assumed to be called once again where it should do nothing.
% Any subsequent call to it should prevent
% a secondary processing of the main document
% It overwrites the forwarding commands
% |\childdocof| and |\childdocforward|
% with empty macros to prevent further inclusions of the main document:
%    \begin{macrocode}
\newcommand{\childdocdisable}
{
  \renewcommand{\childdocmain}[1]{\renewcommand{\childdocmain}[1]{\endinput}}
  \renewcommand{\childdocof}[1]{}
  \renewcommand{\childdocby}[2][]{}
  \renewcommand{\childdocforward}[2][]{}
  \renewcommand{\childdocdisable}{}
}
%    \end{macrocode}

% \macro{\childdocmain}
% The macro |\childdocmain| is to be called at the top of the main file
% with nothing or the main filename (without extension) as argument.
% First, it breaks loops.
% If the argument is not empty and does not match |\childdocname|
% (which is set by the first inclusion of |childdoc.def|),
% |\ifchilddoc| is set to true, |\includeonly| is applied to the child file
% and |\jobname| is set to the main file
% (for proper handling of |.aux| files):
%    \begin{macrocode}
\newcommand{\childdocmain}[1]
{
  \childdocdisable\childdocmain{}
  \if?#1?\else
    \begingroup
      \def\childdoctmp{#1}
      \ifx\childdoctmp\childdocname
        \def\childdoctmp{}
      \else
        \def\childdoctmp
        {
          \childdoctrue
          \includeonly{\childdocname}
          \def\childdocjob{#1}
          \def\jobname{#1}
        }
      \fi
      \expandafter
    \endgroup
    \childdoctmp
  \fi
}
%    \end{macrocode}

% \macro{\childdocof}
% The command |\childdocof| redirects
% compilation to the main file |#1|.
%    \begin{macrocode}
\newcommand{\childdocof}[1]
{
  \childdocdisable
  \childdoctrue
  \includeonly{\childdocname}
  \def\jobname{#1}
  \def\childdocjob{#1}
  \input{#1}
}
%    \end{macrocode}

% \macro{\childdocby}
% The command |\childdocby| ....
%    \begin{macrocode}
\newcommand{\childdocby}[2][]
{
  \childdocdisable
  \childdoctrue
  \childdocmanualtrue
  \if?#1?\else
    \def\jobname{#2}
  \fi
  \def\childdocjob{#2}
  \input{#2}
  \endinput
}
%    \end{macrocode}

% \macro{\childdocforward}
% The command |\childdocforward| redirects
% compilation to the main file or
% (if the optional argument is given) a child file.
% Parameters are set as if the main file
% or a child file starting with |\childdocof| was compiled.
% Then compilation is handed over to the main file:
%    \begin{macrocode}
\newcommand{\childdocforward}[2][]
{
  \begingroup
    \if?#1?
      \def\childdoctmp
      {
        \def\childdocname{#2}
        \def\childdocjob{#2}
        \def\jobname{#2}
        \input{#2}
        \endinput
      }
    \else
      \def\childdoctmp
      {
        \childdocdisable
        \def\childdocname{#2}
        \childdoctrue
        \includeonly{#2}
        \def\childdocjob{#1}
        \def\jobname{#1}
        \input{#1}
        \endinput
      }
    \fi
    \expandafter
  \endgroup
  \childdoctmp
}
%    \end{macrocode}

% \macro{\childdocforwardprefix}
% The command |\childdocforwardprefix| redirects
% compilation to the main or a child file by means of a pattern.
% The prefix |#1| in the current filename is replaced by |#2|
% and the suffix of the current filename is kept
% (it is assumed that the filename does not contain the substring `|~~~|'
% which is used as a delimiter).
% Compilation is handed over to the new file by |\childdocforward|:
%    \begin{macrocode}
\newcommand{\childdocforwardprefix}[3][]
{
  \begingroup
    \def\childdocextract #2##1~~~{\def\childdoctmp{\childdocforward[#1]{#3##1}}}
    \expandafter\childdocextract\childdocname~~~
    \expandafter
  \endgroup
  \childdoctmp
}
%    \end{macrocode}

% \macro{\childdoc}
% The deprecated macro |\childdoc| is a legacy version of |\childdocmain|:
%    \begin{macrocode}
\newcommand{\childdoc}{\childdocmain}
%    \end{macrocode}

% \macro{\childdocredirect}
% The deprecated macro |\childdocredirect| is a legacy version
% of |\childdocforward| and |\childdocforwardprefix|:
%    \begin{macrocode}
\newcommand{\childdocredirect}[2][]
{
  \begingroup
    \if?#1?
      \def\childdoctmp{\childdocforward{#2}}
    \else
      \def\childdoctmp{\childdocforwardprefix{#1}{#2}}
    \fi
    \expandafter
  \endgroup
  \childdoctmp
}
%    \end{macrocode}

%\iffalse
%</package>
%\fi
%
\endinput

\childdocforwardprefix[cdocsamp]{cdocsfn}{cdocsch}
%    \end{macrocode}

%\iffalse
%</samplefinal>
%\fi
%
% %%%%%%%%%%%%%%%%%%%%%%%%%%%%%%%%%%%%%%
% \paragraph{Command Line Processing.}
%
% The following three command lines generate the output files
% |cdocscld|, |cdocscl1| and |cdocscl2|
% which should be identical to
% |cdocsdrf|, |cdocsch1| and |cdocsfn2|, respectively:
% \begin{center}
% \begin{tabular}{l}
% |latex -jobname cdocscld \|\\
% |  "\def\version{draft}% \iffalse
%
% childdoc.dtx Copyright (C) 2017-2018 Niklas Beisert
%
% This work may be distributed and/or modified under the
% conditions of the LaTeX Project Public License, either version 1.3
% of this license or (at your option) any later version.
% The latest version of this license is in
%   http://www.latex-project.org/lppl.txt
% and version 1.3 or later is part of all distributions of LaTeX
% version 2005/12/01 or later.
%
% This work has the LPPL maintenance status `maintained'.
%
% The Current Maintainer of this work is Niklas Beisert.
%
% This work consists of the files childdoc.dtx and childdoc.ins
% and the derived files childdoc.def and cdocsamp.tex with
% cdocsch1.tex, cdocsch2.tex, cdocsdrf.tex, cdocsfn1.tex, cdocsfn2.tex.
%
%<package>\ifdefined\childdocmain\endinput\fi
%<package>\ProvidesFile{childdoc.def}[2018/12/30 v2.0 child document driver]
%<samplemain>\ProvidesFile{cdocsamp.tex}[2018/12/30 v2.0 sample for childdoc]
%<*driver>
%\ProvidesFile{childdoc.drv}[2018/12/30 v2.0 childdoc reference manual file]
\PassOptionsToClass{10pt,a4paper}{article}
\documentclass{ltxdoc}

\usepackage[margin=35mm]{geometry}
\usepackage{hyperref}
\usepackage{hyperxmp}
\usepackage[usenames]{color}

\hypersetup{colorlinks=true}
\hypersetup{pdfstartview=FitH}
\hypersetup{pdfpagemode=UseNone}
\hypersetup{pdfsource={}}
\hypersetup{pdflang={en-UK}}
\hypersetup{pdfcopyright={Copyright 2017-2018 Niklas Beisert.
  This work may be distributed and/or modified under the
  conditions of the LaTeX Project Public License, either version 1.3
  of this license or (at your option) any later version.}}
\hypersetup{pdflicenseurl={http://www.latex-project.org/lppl.txt}}
\hypersetup{pdfcontactaddress={ETH Zurich, ITP, HIT K,
  Wolfgang-Pauli-Strasse 27}}
\hypersetup{pdfcontactpostcode={8093}}
\hypersetup{pdfcontactcity={Zurich}}
\hypersetup{pdfcontactcountry={Switzerland}}
\hypersetup{pdfcontactemail={nbeisert@itp.phys.ethz.ch}}
\hypersetup{pdfcontacturl={http://people.phys.ethz.ch/\xmptilde nbeisert/}}

\newcommand{\secref}[1]{\hyperref[#1]{section \ref*{#1}}}

\parskip1ex
\parindent0pt
\let\olditemize\itemize
\def\itemize{\olditemize\parskip0pt}

\begin{document}

\title{The \textsf{childdoc} Package}
\hypersetup{pdftitle={The childdoc Package}}
\author{Niklas Beisert\\[2ex]
  Institut f\"ur Theoretische Physik\\
  Eidgen\"ossische Technische Hochschule Z\"urich\\
  Wolfgang-Pauli-Strasse 27, 8093 Z\"urich, Switzerland\\[1ex]
  \href{mailto:nbeisert@itp.phys.ethz.ch}
  {\texttt{nbeisert@itp.phys.ethz.ch}}}
\hypersetup{pdfauthor={Niklas Beisert}}
\hypersetup{pdfsubject={Manual for the LaTeX2e Package childdoc}}
\date{30 December 2018, \textsf{v2.0}}
\maketitle

\begin{abstract}\noindent
\textsf{childdoc} is a \LaTeXe{} package
that enables the direct compilation
of document sections included by |\include|
to individual files.
\end{abstract}

\begingroup
\parskip0ex
\tableofcontents
\endgroup

%%%%%%%%%%%%%%%%%%%%%%%%%%%%%%%%%%%%%%%%%%%%%%%%%%%%%%%%%%%%%%%%%%%%%%%%%%%%%%%%
%%%%%%%%%%%%%%%%%%%%%%%%%%%%%%%%%%%%%%%%%%%%%%%%%%%%%%%%%%%%%%%%%%%%%%%%%%%%%%%%
\section{Introduction}

\LaTeX{} provides a mechanism to structure a large document (such as a book)
into a main file and several child files (containing the chapters)
using the |\include| command.
This mechanism is beneficial for documents
which span hundreds of pages in order to
make the source file(s) more manageable.
Moreover, compilation can be restricted to
selected child files by means of the |\includeonly| command.
The latter feature can be used to reduce the compilation time while editing
(this was significantly more useful in the earlier days of \LaTeX{})
or to generate a smaller document which is easier to navigate.
Another application of |\includeonly| is to generate
documents consisting of selected parts of the complete document.

However, there are a few drawbacks of the plain |\include| mechanism:
\begin{itemize}
\item
The child files cannot be compiled on their own,
they can only be compiled via the main file.
A naive editing environment
(such as a text editor with an option
to have the current file processed by \LaTeX)
may require one to switch to the main file before compiling;
attempting to compile the child file produces errors.
\item
The main file must be modified (each time)
to adjust the |\includeonly| command
to the present needs. This easily leaves the main file in a messy state.
\item
The generated document will always carry the filename
of the main document. This is inconvenient if
several child files are to be compiled and
to be kept for distribution.
\end{itemize}

The present package provides a simple interface
to make child files individually compilable by \LaTeX{}.
Compiling a child file then has the same effect as compiling
the main file with an |\includeonly| command
to select the appropriate child.
Moreover the generated document will carry the name of the child
rather than the main file.
This resolves all three above issues.

This feature is meant to make the editing of books,
thesis documents and lecture notes somewhat more convenient.
However, the package can also be used efficiently for
composing a series of documents (such as exercise sheets)
which are typically distributed individually.
It then assists the author in generating the individual documents
(potentially in different versions)
as well as a document containing the collected series.
Another application is in developing style files
or other kinds of included material
where compilation of the style file could redirect
to a sample or test file.

%%%%%%%%%%%%%%%%%%%%%%%%%%%%%%%%%%%%%%%%%%%%%%%%%%%%%%%%%%%%%%%%%%%%%%%%%%%%%%%%
%%%%%%%%%%%%%%%%%%%%%%%%%%%%%%%%%%%%%%%%%%%%%%%%%%%%%%%%%%%%%%%%%%%%%%%%%%%%%%%%
\section{Usage}

First of all, the package \textsf{childdoc} is \emph{not} a standard
\LaTeXe{} |.sty| style file! Therefore it needs to be invoked in
a non-standard way.

%%%%%%%%%%%%%%%%%%%%%%%%%%%%%%%%%%%%%%%%%%%%%%%%%%%%%%%%%%%%%%%%%%%%%%%%%%%%%%%%
\subsection{Included Files}
\label{sec:include}

%%%%%%%%%%%%%%%%%%%%%%%%%%%%%%%%%%%%%%%%
\DescribeMacro{\childdocmain}
To use the package, add the commands
\begin{center}
\begin{tabular}{l}
|\input{childdoc.def}|\\
|\childdocmain{}|\\
\end{tabular}
\end{center}
at the very top of the main \LaTeX{} file,
in particular \emph{before} the |\documentclass| statement!
The argument of |\childdocmain| should be left empty
(but it must be present).

%%%%%%%%%%%%%%%%%%%%%%%%%%%%%%%%%%%%%%%%
\DescribeMacro{\childdocof}
Furthermore, add the commands
\begin{center}
\begin{tabular}{l}
|\input{childdoc.def}|\\
|\childdocof{|\textit{main}|}|\\
\end{tabular}
\end{center}
at the top of every child file \textit{child}
which is included by |\include{|\textit{child}|}|
from within the main file
(or at least for those files to be compiled individually).
The argument \textit{main} must be the filename of the main file.

There are a couple of
considerations in setting up the main and child documents:

%%%%%%%%%%%%%%%%%%%%%%%%%%%%%%%%%%%%%%%%
\paragraph{Restrictions.}

Please note the following restrictions:
\begin{itemize}
\item
|\childdocmain| must be called with one argument \textit{main}
to ensure compatibility with earlier version of the package.
It must either be empty (|\childdocmain{}|)
or precisely match the filename of the main file in which it is specified.
See \secref{sec:detection} for further information.
\item
The filename \textit{main} must be specified without the |.tex| extension.
\item
The filename \textit{main} is case sensitive
(even in case-insensitive file systems)
due to internal string comparison.
\item
The argument \textit{main} should be fully expanded, it cannot be a macro.
\item
Subdirectories and special characters should be avoided in filenames.
\item
The command |\childdocmain{|\textit{main}|}| must be followed by a whitespace.
It should not be followed immediately by another command
or by a comment mark `|%|'.
This is because the \TeX{} parser reads the token immediately following
the argument of |\childdocmain| and puts it
at the beginning of every child section;
however, a white\-space is ignored.
\end{itemize}

%%%%%%%%%%%%%%%%%%%%%%%%%%%%%%%%%%%%%%%%
\paragraph{Content of Main File.}

It is advisable to place all content in the child files included by |\include|.
Any output contained in the main file will appear in all child documents
unless suppressed manually;
it cannot be suppressed automatically by the |\includeonly| directive
and thus should normally be avoided.
A method to include some content in the main file
by means of conditional processing is described in \secref{sec:conditional}.

%%%%%%%%%%%%%%%%%%%%%%%%%%%%%%%%%%%%%%%%
\paragraph{Page Numbering.}

When only a part of the document is compiled,
the appropriate numbering of pages
(as well as other status parameters)
is determined from the |.aux| files.
The latter contain information from previous passes.
However this information needs to propagate through
all intermediate child documents.
Therefore the page numbering in child documents may well
be inconsistent until the complete document is compiled at least once.

A useful (if unconventional) way to always ensure a consistent
page numbering is to restart the numbering in each child document
and denote the pages by `\textit{child}|.|\textit{page}'
where \textit{child} represents the chapter/section number of the child file.
This can be achieved by the command
|\numberwithin{page}{|\textit{child}|}|
of the \textsf{amsmath} package
where \textit{child} can be |chapter| or |section|
depending on the chosen structuring.
Alternatively, one can modify the macro |\thepage| appropriately
and reset the counter |page| at the start of each child file.

%%%%%%%%%%%%%%%%%%%%%%%%%%%%%%%%%%%%%%%%%%%%%%%%%%%%%%%%%%%%%%%%%%%%%%%%%%%%%%%%
\subsection{Conditional Processing}
\label{sec:conditional}

The package provides a mechanism to compile different versions
of a document. To customise the versions further some conditional processing
can come in handy to distinguish which version is being compiled.
The package provides two macros to describe the compilation context:

%%%%%%%%%%%%%%%%%%%%%%%%%%%%%%%%%%%%%%%%
\DescribeMacro{\ifchilddoc}
The conditional |\ifchilddoc| distinguishes between the compilation of
child documents and the main document:
%
\begin{center}
|\ifchilddoc |\textit{child-code}| |[|\||else |\textit{main-code}]| \||fi|
\end{center}

%%%%%%%%%%%%%%%%%%%%%%%%%%%%%%%%%%%%%%%%
\DescribeMacro{\childdocname}
\DescribeMacro{\childdocjob}
The macro |\childdocname| contains the filename (without extension)
of the main or child file being processed.
Note that |\childdocjob| will always contain the name of the main file.

%%%%%%%%%%%%%%%%%%%%%%%%%%%%%%%%%%%%%%%%
\paragraph{Title Page.}

Conditional processing can be used to include a title or banner page
in the main document when proper precautions are taken.
Importantly, the code in the main file should ensure that the page counter
(as well as other status parameters which are stored in the |.aux| files)
takes the same value after the conditional processing.
Otherwise the page numbers may take divergent values
depending on which part is compiled.

For example, a title page could be declared by:
%
\begin{center}
\begin{tabular}{l}
|\ifchilddoc\||else|\\
|\addtocounter{page}{-1}|\\
\textit{code for title page}\\
|\newpage|\\
|\||fi|
\end{tabular}
\end{center}
%
A banner page for the child documents can be generated by:
%
\begin{center}
\begin{tabular}{l}
|\ifchilddoc|\\
|\addtocounter{page}{-1}|\\
\textit{code for banner page}\\
|\newpage|\\
|\||fi|
\end{tabular}
\end{center}
%
Here one could write a message such as:
\begin{center}
|This is the part \childdocname{} of \childdocjob{}.|
\end{center}

%%%%%%%%%%%%%%%%%%%%%%%%%%%%%%%%%%%%%%%%%%%%%%%%%%%%%%%%%%%%%%%%%%%%%%%%%%%%%%%%
\subsection{Flags}
\label{sec:flags}

The package makes it easy to generate different versions
of the main or child documents.
To this end compilation flags can be defined
and assigned different default values.
They will be particularly useful in conjunction
with the forwarding mechanism described in \secref{sec:forward}.

For example, it may be useful to have a flag |\version|
which can be set to |draft| or |final|.
The document source will contain some conditional code
depending on the value of |\version|.
Suppose further, the flag should default to |final| for the main file
and to |draft| for child files
which is a natural assignment for editing the document.
This is achieved by placing the following code
in the preamble of the main document
(below the |\childdocmain| directive):
%
\begin{center}
\begin{tabular}{l}
|\ifchilddoc|\\
|\providecommand{\version}{draft}|\\
|\||else|\\
|\providecommand{\version}{final}|\\
|\||fi|
\end{tabular}
\end{center}
%
The definition by |\providecommand| makes sure
that previous definitions are not overwritten.
Further statements |\providecommand{\version}{...}|
can thus be added before the above code to override it.

For the main file, one might add a line
(between |\childdocmain| and the above block)
%
\begin{center}
|%\ifchilddoc\||else\providecommand{\version}{draft}\||fi|
\end{center}
%
which can be uncommented to produce a draft version.
Likewise one can add a line to the very top of a child file
(above the |\childdocof{|\textit{main}|}| directive)
%
\begin{center}
|%\providecommand{\version}{final}|
\end{center}
%
which can be uncommented to produce the final version of this child document.

%%%%%%%%%%%%%%%%%%%%%%%%%%%%%%%%%%%%%%%%%%%%%%%%%%%%%%%%%%%%%%%%%%%%%%%%%%%%%%%%
\subsection{Forwarding}
\label{sec:forward}

Different versions of the main or child documents
using compilation flags as described in \secref{sec:flags}
can be (permanently) stored in different files
for convenient compilation, viewing and distribution.
To this end, the package defines a command
to pass on compilation to a different file:

%%%%%%%%%%%%%%%%%%%%%%%%%%%%%%%%%%%%%%%%
\DescribeMacro{\childdocforward}
The command |\childdocforward| redirects processing to
another source file:
%
\begin{center}
\begin{tabular}{l}
|\input{childdoc.def}|\\
|\childdocforward[|\textit{main}|]{|\textit{dest}|}|\\
\end{tabular}
\end{center}
%
The argument \textit{dest} is the destination file
(without extension).
It should be the main file or one of the child files.
Note that further \textsf{childdoc} directives
such as |\childdocof| and |\childdocforward|
in the indicated file will be processed in this form.
The optional argument \textit{main}
passes on directly to the main file \textit{main}
while pretending to compile the child \textit{dest}.
This form behaves as if \textit{dest}
issues |\childdocof{|\textit{main}|}| right away,
and no further \textsf{childdoc} directives will be processed.

%%%%%%%%%%%%%%%%%%%%%%%%%%%%%%%%%%%%%%%%
\DescribeMacro{\...prefix}
In the alternative form |\childdocforwardprefix|,
%
\begin{center}
\begin{tabular}{l}
|\input{childdoc.def}|\\
|\childdocforwardprefix[|\textit{main}|]{|\textit{prefix}|}{|\textit{dest}|}|
\end{tabular}
\end{center}
%
the destination file is determined by a pattern
depending on the current file:
To make this work, the current file must be called
`{\textit{prefix}\hspace{0.2em}\textit{suffix}}'
with \textit{prefix} matching precisely the argument.
Processing is then passed on to the file
`{\textit{dest}\hspace{0.2em}\textit{suffix}}'.
Surely, the same effect is achieved by
directly specifying the
argument `{\textit{dest}\hspace{0.2em}\textit{suffix}}'
in the first form.
However, that requires to set up a different file
for each child. With the alternative form of the command
all these files can have exactly the same content
which simplifies setting them up and maintaining them.

For example, the following file |draft.tex|
with a compilation flag |\version| as described in \secref{sec:flags}
compiles the main document as a draft:
%
\begin{center}
\begin{tabular}{l}
|\def\version{draft}|\\
|\input{childdoc.def}|\\
|\childdocforward{|\textit{main}|}|
\end{tabular}
\end{center}
%
Likewise, the following files |final|\textit{nn}|.tex|
compile the final version of the child document
|child|\textit{nn}|.tex|:
%
\begin{center}
\begin{tabular}{l}
|\def\version{final}|\\
|\input{childdoc.def}|\\
|\childdocforwardprefix{final}{child}|
\end{tabular}
\end{center}
%

Note that when several versions of a main file and/or of each child file
are to be generated, it may be convenient to set up a |Makefile| or
shell script to automatise the process.

%%%%%%%%%%%%%%%%%%%%%%%%%%%%%%%%%%%%%%%%%%%%%%%%%%%%%%%%%%%%%%%%%%%%%%%%%%%%%%%%
\subsection{Command Line Processing}
\label{sec:commandline}

The effect of redirection files can also be achieved by invoking
the \LaTeX{} compiler with a more elaborate command line.
Most conveniently this should be done as part
of a shell script or a |Makefile|.

When using \textsf{childdoc} in the main file, the following
command lines effectively perform a redirection
(note that depending on the shell being used,
backslashes may have to be doubled: `|\|' $\to$ `|\\|'):
%
\begin{center}
|... -jobname "|\textit{target}|" |\\|"|[\textit{flags}]%
|\input{childdoc.def}\childdocforward[|\textit{main}|]{|\textit{dest}|}"|
\end{center}
%
Here \textit{target} is the name of the output file,
\textit{main} is the name of the main file
and \textit{dest} is the name of the main or child file to be processed
(all filenames without extensions).
The optional argument \textit{main} can be omitted
if \textit{main} matches \textit{dest}.
Optionally, compilation \textit{flags} can be defined via |\def| commands.
This command line makes the \TeX{} engine believe
it is compiling the file \textit{target}
whose content is specified as the latter parameter.
The provided code then forwards the processing to
\textit{main} or \textit{dest} as described in \secref{sec:forward}.

%%%%%%%%%%%%%%%%%%%%%%%%%%%%%%%%%%%%%%%%%%%%%%%%%%%%%%%%%%%%%%%%%%%%%%%%%%%%%%%%
\subsection{Include by Input}
\label{sec:input}

Including child documents by |\include| has some restrictions by design.
Most notably, the content of a child document always occupies
its own set of pages; pages cannot be shared between child documents.
Usually, this behaviour makes perfect sense
because each child document contain an essential part of the document.
However, in some situations it may be desirable to compose
a document from a collection of parts
without having mandatory page breaks between then.
For this case, the package
provides a mechanism to include parts
by |\input| which can also be processed individually.
However, by construction this mechanism
requires manual handling of the content to be output.

%%%%%%%%%%%%%%%%%%%%%%%%%%%%%%%%%%%%%%%%
\DescribeMacro{\ifchilddocmanual}
The main file should be prepared as usual, see \secref{sec:include}.
However, the document body must make a distinction
between processing of an individual part and of the main document, e.g.:
%
\begin{center}
\begin{tabular}{l}
|\ifchilddocmanual|\\
|\input{\childdocname}|\\
|\||else|\\
\textit{document body with }|\input{|\textit{part}|}|\\
|\||fi|
\end{tabular}
\end{center}
%
The conditional |\ifchilddocmanual| is true whenever
a part to be included by |\input| is being compiled,
and the name of the part is stored in |\childdocname|.

%%%%%%%%%%%%%%%%%%%%%%%%%%%%%%%%%%%%%%%%
\DescribeMacro{\childdocby}
Each part to be included by |\input| should start with:
%
\begin{center}
\begin{tabular}{l}
|\input{childdoc.def}|\\
|\childdocby{|\textit{main}|}|\\
\end{tabular}
\end{center}
%
The directive |\childdocby| is similar to |\childdocof|
described in \secref{sec:include},
but the subsequent selection of content must be done manually.
To that end, both |\ifchilddoc| and |\ifchilddocmanual|
will be true upon processing of a part,
and the name of the part is stored in |\childdocname|.
Note that |\jobname| will be set to the filename of the current part
so that each part receives an individual |.aux| file
that does not interfere with the |.aux| file(s) of the main document.
This behaviour can be altered by the alternative form
|\childdocby[*]{|\textit{main}|}| (with a non-empty optional argument)
which uses the |.aux| file of the main document
by setting |\jobname| to \textit{main}.

%%%%%%%%%%%%%%%%%%%%%%%%%%%%%%%%%%%%%%%%%%%%%%%%%%%%%%%%%%%%%%%%%%%%%%%%%%%%%%%%
\subsection{Driver Development}
\label{sec:driver}

The \textsf{childdoc} mechanism can also be use for the development
of definition files such as \LaTeX{} styles or classes.
This case differs from the above setup with multiple parts
included by |\include| in that no |\includeonly| should be invoked.
This can be achieved by starting the include file
(before |\ProvidesPackage|) with:
%
\begin{center}
\begin{tabular}{l}
|\input{childdoc.def}|\\
|\childdocforward{|\textit{main}|}|\\
\end{tabular}
\end{center}
%
or alternatively with:
%
\begin{center}
\begin{tabular}{l}
|\input{childdoc.def}|\\
|\childdocby{|\textit{main}|}|\\
\end{tabular}
\end{center}
%
Both forms have slightly different effects as described above.
The main file is prepared as usual, see \secref{sec:include}.

%%%%%%%%%%%%%%%%%%%%%%%%%%%%%%%%%%%%%%%%%%%%%%%%%%%%%%%%%%%%%%%%%%%%%%%%%%%%%%%%
\subsection{Legacy Detection}
\label{sec:detection}

The directive |\childdocmain| in the main file can detect
whether the complete document or merely a child is to be compiled
even without using the directive |\childdocof|.
This method is deprecated because it is less robust
and there is no compelling reason to use it;
it is merely provided for backward compatibility
and it may be removed in future versions.

If the detection mechanism is to be used,
it is mandatory to correctly specify
the filename of the main file as the argument of |\childdocmain|:
%
\begin{center}
\begin{tabular}{l}
|\input{childdoc.def}|\\
|\childdocmain{|\textit{main}|}|\\
\end{tabular}
\end{center}
%
If |\jobname| does not match the argument \textit{main} of |\childdocmain|,
it is assumed that |\jobname| points to the child file to be compiled.
When using |\childdocmain| with the main file specified as argument,
it suffices to start a child file
with just |\input{|\textit{main}|}|
without loading of the package and using |\childdocof|.
If instead all processing is done
with the appropriate \textsf{childdoc} directives,
the argument of \textit{main} of |\childdocmain| can be empty.

An alternative version of the command line processing described
in \secref{sec:commandline} using the detection mechanism reads:
%
\begin{center}
|... -jobname "|\textit{target}|" "|[\textit{flags}]%
[|\def\jobname{|\textit{dest}|}|]|\input{|\textit{main}|}"|
\end{center}

%%%%%%%%%%%%%%%%%%%%%%%%%%%%%%%%%%%%%%%%%%%%%%%%%%%%%%%%%%%%%%%%%%%%%%%%%%%%%%%%
\subsection{Manual Code}
\label{sec:manual}

In case one cannot be certain whether the definitions file |childdoc.def|
is installed on the target \TeX{} distribution
and one prefers not to ship it,
it is conceivable to paste a few relevant commands into the sources.

To that end, drop all statements |\input{childdoc.def}|
and perform the replacements as outlined below.
Instead of |\childdocmain{|\textit{main}|}| add the following code
to the top of the main file:
%
\begin{center}
\begin{tabular}{l}
|\||ifdefined\childdocname\endinput\||fi\newif\ifchilddoc|\\
|\edef\childdocname{\scantokens\expandafter{\jobname\noexpand}}|\\
|\def\childdocmain{|\textit{main}|}\||ifx\childdocmain\childdocname\||else|\\
|\childdoctrue\includeonly{\childdocname}\let\jobname\childdocmain\||fi|\\
\end{tabular}
\end{center}
%
Instead of |\childdocof{|\textit{main}|}| just include the main file
at the top of each child file:
%
\begin{center}
|\input{|\textit{main}|}|
\end{center}
%
A simple redirection |\childdocforward{|\textit{dest}|}| is achieved by:
%
\begin{center}
|\def\jobname{|\textit{dest}|}\input{\jobname}|
\end{center}
%
The redirection with prefix
|\childdocforwardprefix[|\textit{prefix}|]{|\textit{dest}|}|
is accomplished by:
%
\begin{center}
\begin{tabular}{l}
|{\edef\jobname{\scantokens\expandafter{\jobname\noexpand}}|\\
|\def\redirectjob |\textit{prefix}|#1~~~{\gdef\jobname{|\textit{dest}|#1}}|\\
|\expandafter\redirectjob\jobname~~~}\input{\jobname}|
\end{tabular}
\end{center}

In an alternative approach,
child documents can be compiled by a specific command line
without additional code or specific definitions:
%
\begin{center}
|... -jobname "|\textit{target}|" "|[\textit{flags}]%
|\includeonly{|\textit{dest}|}\input{|\textit{main}|}"|
\end{center}
%

%%%%%%%%%%%%%%%%%%%%%%%%%%%%%%%%%%%%%%%%%%%%%%%%%%%%%%%%%%%%%%%%%%%%%%%%%%%%%%%%
%%%%%%%%%%%%%%%%%%%%%%%%%%%%%%%%%%%%%%%%%%%%%%%%%%%%%%%%%%%%%%%%%%%%%%%%%%%%%%%%
\section{Information}

%%%%%%%%%%%%%%%%%%%%%%%%%%%%%%%%%%%%%%%%%%%%%%%%%%%%%%%%%%%%%%%%%%%%%%%%%%%%%%%%
\subsection{Copyright}

Copyright \copyright{} 2017--2018 Niklas Beisert

This work may be distributed and/or modified under the
conditions of the \LaTeX{} Project Public License, either version 1.3
of this license or (at your option) any later version.
The latest version of this license is in
  \url{http://www.latex-project.org/lppl.txt}
and version 1.3 or later is part of all distributions of \LaTeX{}
version 2005/12/01 or later.

This work has the LPPL maintenance status `maintained'.

The Current Maintainer of this work is Niklas Beisert.

This work consists of the files |README.txt|, |childdoc.ins| and |childdoc.dtx|
as well as the derived files |childdoc.def|, |cdocsamp.tex|
with |cdocsch1.tex|, |cdocsch2.tex|, |cdocspt3.tex|, |cdocspt4.tex|,
|cdocsdrf.tex|, |cdocsfn1.tex|, |cdocsfn2.tex|
as well as |childdoc.pdf|.

%%%%%%%%%%%%%%%%%%%%%%%%%%%%%%%%%%%%%%%%%%%%%%%%%%%%%%%%%%%%%%%%%%%%%%%%%%%%%%%%
\subsection{Files and Installation}

The package consists of the files:
%
\begin{center}
\begin{tabular}{ll}
    |README.txt|   & readme file \\
    |childdoc.ins| & installation file \\
    |childdoc.dtx| & source file \\
    |childdoc.def| & definition file \\
    |cdocsamp.tex| & sample main file \\
    |cdocsch1.tex| & sample include file \\
    |cdocsch2.tex| & sample include file \\
    |cdocspt3.tex| & sample part file \\
    |cdocspt4.tex| & sample part file \\
    |cdocsdrf.tex| & sample redirection file \\
    |cdocsfn1.tex| & sample redirection file \\
    |cdocsfn2.tex| & sample redirection file \\
    |childdoc.pdf| & manual
\end{tabular}
\end{center}
%
The distribution consists of the files
|README.txt|, |childdoc.ins| and |childdoc.dtx|.
%
\begin{itemize}
\item
Run (pdf)\LaTeX{} on |childdoc.dtx|
to compile the manual |childdoc.pdf| (this file).
\item
Run \LaTeX{} on |childdoc.ins| to create the definitions file |childdoc.def|
and the sample |cdocsamp.tex| with include files
|cdocsch1.tex|, |cdocsch2.tex|, |cdocspt3.tex|, |cdocspt4.tex|,
|cdocsdrf.tex|, |cdocsfn1.tex|, |cdocsfn2.tex|.
Then copy the file |childdoc.def| to an appropriate directory of your \LaTeX{}
distribution, e.g.\ \textit{texmf-root}|/tex/latex/childdoc|.
\end{itemize}

%%%%%%%%%%%%%%%%%%%%%%%%%%%%%%%%%%%%%%%%%%%%%%%%%%%%%%%%%%%%%%%%%%%%%%%%%%%%%%%%
\subsection{Related CTAN Packages}

There are several other packages which offer a similar functionality:
%
\begin{itemize}
\item
The packages
\href{http://ctan.org/pkg/docmute}{\textsf{docmute}},
\href{http://ctan.org/pkg/includex}{\textsf{includex}} and
\href{http://ctan.org/pkg/standalone}{\textsf{standalone}}
provide commands to include only the document body of
a child file thus allowing both files to be compiled individually.
\item
The packages \href{http://ctan.org/pkg/subdocs}{\textsf{subdocs}}
and \href{http://ctan.org/pkg/subfiles}{\textsf{subfiles}}
provide structures in which the main and child documents can be
encapsulated and allowing them to be compiled individually.
The inclusion mechanism is different from the conventional |\include|.
\item
The package \href{http://ctan.org/pkg/combine}{\textsf{combine}}
is an elaborate solution to combine several documents into one.
\end{itemize}
%
See also the CTAN topic \href{http://ctan.org/topic/subdocs}{\textsf{subdocs}}
for further related packages.
The present package differs from the above solutions in that
a document structure constructed with the conventional |\include| mechanism
just needs two extra commands at the top of every file
such that all constituent files can be compiled individually.

%%%%%%%%%%%%%%%%%%%%%%%%%%%%%%%%%%%%%%%%%%%%%%%%%%%%%%%%%%%%%%%%%%%%%%%%%%%%%%%%
%\subsection{Feature Suggestions}
%
%The following is a list of features which may be useful for future
%versions of this package:
%%
%\begin{itemize}
%\item
%\ldots
%\end{itemize}

%%%%%%%%%%%%%%%%%%%%%%%%%%%%%%%%%%%%%%%%%%%%%%%%%%%%%%%%%%%%%%%%%%%%%%%%%%%%%%%%
\subsection{Revision History}

%%%%%%%%%%%%%%%%%%%%%%%%%%%%%%%%%%%%%%%%
\paragraph{v2.0:} 2018/12/30

\begin{itemize}
\item
immediate forward processing
\item
added |\childdocby| mechanism
\item
manual restructured
\end{itemize}

%%%%%%%%%%%%%%%%%%%%%%%%%%%%%%%%%%%%%%%%
\paragraph{v1.6:} 2018/01/17

\begin{itemize}
\item
application for development of include files
\item
corrections to manual
\end{itemize}

%%%%%%%%%%%%%%%%%%%%%%%%%%%%%%%%%%%%%%%%
\paragraph{v1.5:} 2017/05/21

\begin{itemize}
\item
more complete structuring introduced
\item
|\childdocof| introduced
\item
|\childdoc| renamed to |\childdocmain|
\item
|\childredirect| renamed to |\childdocforward| and |\childdocforwardprefix|
and functionality expanded
\end{itemize}

%%%%%%%%%%%%%%%%%%%%%%%%%%%%%%%%%%%%%%%%
\paragraph{v1.0:} 2017/04/27

\begin{itemize}
\item
manual and install package
\item
first version published on CTAN
\end{itemize}

%%%%%%%%%%%%%%%%%%%%%%%%%%%%%%%%%%%%%%%%
\paragraph{v0.6:} 2017/04/26

\begin{itemize}
\item
redirection mechanism added
\end{itemize}

%%%%%%%%%%%%%%%%%%%%%%%%%%%%%%%%%%%%%%%%
\paragraph{v0.5:} 2017/04/26

\begin{itemize}
\item
functionality in definition file
\end{itemize}


%%%%%%%%%%%%%%%%%%%%%%%%%%%%%%%%%%%%%%%%%%%%%%%%%%%%%%%%%%%%%%%%%%%%%%%%%%%%%%%%
%%%%%%%%%%%%%%%%%%%%%%%%%%%%%%%%%%%%%%%%%%%%%%%%%%%%%%%%%%%%%%%%%%%%%%%%%%%%%%%%
%%%%%%%%%%%%%%%%%%%%%%%%%%%%%%%%%%%%%%%%%%%%%%%%%%%%%%%%%%%%%%%%%%%%%%%%%%%%%%%%
\appendix

\settowidth\MacroIndent{\rmfamily\scriptsize 000\ }

 \DocInput{childdoc.dtx}

\end{document}
%</driver>
% \fi
%
% %%%%%%%%%%%%%%%%%%%%%%%%%%%%%%%%%%%%%%%%%%%%%%%%%%%%%%%%%%%%%%%%%%%%%%%%%%%%%%
% %%%%%%%%%%%%%%%%%%%%%%%%%%%%%%%%%%%%%%%%%%%%%%%%%%%%%%%%%%%%%%%%%%%%%%%%%%%%%%
% \section{Sample}
%\iffalse
%<*samplemain>
%\fi
%
% The following presents a sample document
% with two chapters, two parts, a title page,
% a compile flag as well as three forwarding files to set the flag.
% It consists of eight |.tex| files:
% \begin{center}
% \begin{tabular}{ll}
% |cdocsamp.tex|&main file\\
% |cdocsch1.tex|&include file for chapter 1\\
% |cdocsch2.tex|&include file for chapter 2\\
% |cdocspt3.tex|&include file for part 3\\
% |cdocspt4.tex|&include file for part 4\\
% |cdocsdrf.tex|&forwarding file for main file in draft mode\\
% |cdocsfi1.tex|&forwarding file for final version of chapter 1\\
% |cdocsfi2.tex|&forwarding file for final version of chapter 2\\
% \end{tabular}
% \end{center}
% Each of the eight files can be compiled directly by the \LaTeX{} compiler.
%
% %%%%%%%%%%%%%%%%%%%%%%%%%%%%%%%%%%%%%%
% \paragraph{Main File.}
%
% The main file is called |cdocsamp.tex|.
%
% Load the \textsf{childdoc} definitions and
% declare the filename for the main document:
%    \begin{macrocode}
\input{childdoc.def}
\childdocmain{}
%    \end{macrocode}

% Optional override for |\version| flag:
%    \begin{macrocode}
%%\ifchilddoc\else\providecommand{\version}{draft}\fi
%    \end{macrocode}

% Define the default values for the |\version| flag
% (|final| for the main file and |draft| for childs):
%    \begin{macrocode}
\ifchilddoc
\providecommand{\version}{draft}
\else
\providecommand{\version}{final}
\fi
%    \end{macrocode}

% Load the standard document class:
%    \begin{macrocode}
\documentclass[12pt]{article}
%    \end{macrocode}

% Start the document body:
%    \begin{macrocode}
\begin{document}
%    \end{macrocode}

% Declare a title page.
% Print title, part of document being processed and version flag:
%    \begin{macrocode}
\addtocounter{page}{-1}
\begin{center}
{\LARGE\bfseries{}childdoc example\par}
\vspace{1cm}
\ifchilddoc
\ifchilddocmanual part\else chapter\fi:
`\childdocname' of `\childdocjob'\par
\else
main document: `\childdocjob'\par
\fi
version: \version\par
\end{center}
\newpage
%    \end{macrocode}

% Manually include selected file,
% otherwise process as usual:
%    \begin{macrocode}
\ifchilddocmanual
\section*{part `\childdocname'}
\input{\childdocname}
\else
%    \end{macrocode}

% Include the two chapters:
%    \begin{macrocode}
\include{cdocsch1}
\include{cdocsch2}
%    \end{macrocode}

% Include the two parts unless only chapters should be displayed:
%    \begin{macrocode}
\ifchilddoc\else
\section{part three}
\input{cdocspt3}
\section{part four}
\input{cdocspt4}
\fi
%    \end{macrocode}

% Process as usual until here:
%    \begin{macrocode}
\fi
%    \end{macrocode}

% End of document body:
%    \begin{macrocode}
\end{document}
%    \end{macrocode}
%\iffalse
%</samplemain>
%\fi
%
% %%%%%%%%%%%%%%%%%%%%%%%%%%%%%%%%%%%%%%
% \paragraph{Chapter Include Files.}
%
% The include files are called |cdocsch1.tex| and |cdocsch2.tex|.
%
%\iffalse
%<*samplechap1|samplechap2>
%\fi

% Optional override for |\version| flag:
%    \begin{macrocode}
%%\providecommand{\version}{final}
%    \end{macrocode}

% Include the main document:
%    \begin{macrocode}
\input{childdoc.def}
\childdocof{cdocsamp}
%    \end{macrocode}

%\iffalse
%</samplechap1|samplechap2>
%\fi
%
%\iffalse
%<*samplechap1>
%\fi
% Some text for chapter 1:
%    \begin{macrocode}
\section{one}
some text in chapter one
%    \end{macrocode}

%\iffalse
%</samplechap1>
%\fi
% Some text for chapter 2:
%\iffalse
%<*samplechap2>
%\fi
%    \begin{macrocode}
\section{two}
more text in chapter two
%    \end{macrocode}

%\iffalse
%</samplechap2>
%\fi
%
% %%%%%%%%%%%%%%%%%%%%%%%%%%%%%%%%%%%%%%
% \paragraph{Part Include Files.}
%
% The include files are called |cdocspt3.tex| and |cdocspt4.tex|.
%
%\iffalse
%<*samplepart3|samplepart4>
%\fi

% Optional override for |\version| flag:
%    \begin{macrocode}
%%\providecommand{\version}{final}
%    \end{macrocode}

% Include the main document:
%    \begin{macrocode}
\input{childdoc.def}
\childdocby{cdocsamp}
%    \end{macrocode}

%\iffalse
%</samplepart3|samplepart4>
%\fi
%
%\iffalse
%<*samplepart3>
%\fi
% Some text for part 3:
%    \begin{macrocode}
some text in part three
%    \end{macrocode}

%\iffalse
%</samplepart3>
%\fi
% Some text for part 4:
%\iffalse
%<*samplepart4>
%\fi
%    \begin{macrocode}
more text in part four
%    \end{macrocode}

%\iffalse
%</samplepart4>
%\fi
%
% %%%%%%%%%%%%%%%%%%%%%%%%%%%%%%%%%%%%%%
% \paragraph{Forwarding for a Complete Draft.}
%
% The following forwarding file |cdocsdrf.tex|
% compiles the main document in draft mode:
%\iffalse
%<*sampledraft>
%\fi
%    \begin{macrocode}
\def\version{draft}
\input{childdoc.def}
\childdocforward{cdocsamp}
%    \end{macrocode}

%\iffalse
%</sampledraft>
%\fi
%
% %%%%%%%%%%%%%%%%%%%%%%%%%%%%%%%%%%%%%%
% \paragraph{Forwarding for Final Version of the Chapters.}
%
% The following forwarding files |cdocsfn1.tex| and |cdocsfn2.tex|
% (with identical content)
% compile the final versions of the child documents
% |cdocsch1.tex| and |cdocsch2.tex|, respectively:
%\iffalse
%<*samplefinal>
%\fi
%    \begin{macrocode}
\def\version{final}
\input{childdoc.def}
\childdocforwardprefix[cdocsamp]{cdocsfn}{cdocsch}
%    \end{macrocode}

%\iffalse
%</samplefinal>
%\fi
%
% %%%%%%%%%%%%%%%%%%%%%%%%%%%%%%%%%%%%%%
% \paragraph{Command Line Processing.}
%
% The following three command lines generate the output files
% |cdocscld|, |cdocscl1| and |cdocscl2|
% which should be identical to
% |cdocsdrf|, |cdocsch1| and |cdocsfn2|, respectively:
% \begin{center}
% \begin{tabular}{l}
% |latex -jobname cdocscld \|\\
% |  "\def\version{draft}\input{childdoc.def}\childdocforward{cdocsamp}"|\\
% |latex -jobname cdocscl1 \|\\
% |  "\input{childdoc.def}\childdocforward[cdocsamp]{cdocsch1}"|\\
% |latex -jobname cdocscl2 \|\\
% |  "\def\version{final}\input{childdoc.def}\childdocforward{cdocsch2}"|
% \end{tabular}
% \end{center}
% Note that the trailing backslash on each first line
% merely continues the input to the second line
% (for convenient cut ant paste).
% Furthermore, the command |latex| can be replaced by any
% of its alternative versions such as |pdflatex|.
%
% %%%%%%%%%%%%%%%%%%%%%%%%%%%%%%%%%%%%%%%%%%%%%%%%%%%%%%%%%%%%%%%%%%%%%%%%%%%%%%
% %%%%%%%%%%%%%%%%%%%%%%%%%%%%%%%%%%%%%%%%%%%%%%%%%%%%%%%%%%%%%%%%%%%%%%%%%%%%%%
% \section{Implementation}
%\iffalse
%<*package>
%\fi
%
% This section describes the definitions file |childdoc.def|.

% The definitions cannot be loaded using |\usepackage| or |\RequirePackage|
% which has a mechanism to prevent loading a style file more than once.
% When loading the definitions by means of |\input|
% multiple instances have to be prevented manually:
%\iffalse
%This code needs to be before the `\ProvidesFile' directive
%which is defined at the beginning of this file.
%Therefore it is also placed there and commented out here.
%</package>
%<*discard>
%\fi
%    \begin{macrocode}
\ifdefined\childdocmain\endinput\fi
%    \end{macrocode}
%\iffalse
%</discard>
%<*package>
%\fi
%
% \macro{\ifchilddoc}
% \macro{\ifchilddocmanual}
% The conditional |\ifchilddoc| tells whether a
% child (true) or main (false) document is being compiled.
% The conditional |\ifchilddocmanual| tells whether
% the |\includeonly| mechanism is used (false) or
% the selection of child files must be performed manually (true).
% The definitions initialise to false:
%    \begin{macrocode}
\newif\ifchilddoc
\newif\ifchilddocmanual
%    \end{macrocode}

% \macro{\childdocname}
% \macro{\childdocjob}
% The macro |\childdocname| stores the name of the main document
% to be compiled. The macro |\childdocjob| stores the name of
% the document on which the \LaTeX{} compiler was originally invoked.
% The content of |\jobname| cannot be compared
% to filenames specified in the source due to different catcodes.
% The following code rescans |\jobname|, stores the result
% in |\childdocname| and saves a copy in |\childdocjob|:
%    \begin{macrocode}
\edef\childdocname{\scantokens\expandafter{\jobname\noexpand}}
\let\childdocjob\childdocname
%    \end{macrocode}

% \macro{\childdocdisable}
% The macro |\childdocdisable| prevents the main file
% from being processed more than once.
% At this stage, the main document command |\childdocmain|
% is assumed to be called once again where it should do nothing.
% Any subsequent call to it should prevent
% a secondary processing of the main document
% It overwrites the forwarding commands
% |\childdocof| and |\childdocforward|
% with empty macros to prevent further inclusions of the main document:
%    \begin{macrocode}
\newcommand{\childdocdisable}
{
  \renewcommand{\childdocmain}[1]{\renewcommand{\childdocmain}[1]{\endinput}}
  \renewcommand{\childdocof}[1]{}
  \renewcommand{\childdocby}[2][]{}
  \renewcommand{\childdocforward}[2][]{}
  \renewcommand{\childdocdisable}{}
}
%    \end{macrocode}

% \macro{\childdocmain}
% The macro |\childdocmain| is to be called at the top of the main file
% with nothing or the main filename (without extension) as argument.
% First, it breaks loops.
% If the argument is not empty and does not match |\childdocname|
% (which is set by the first inclusion of |childdoc.def|),
% |\ifchilddoc| is set to true, |\includeonly| is applied to the child file
% and |\jobname| is set to the main file
% (for proper handling of |.aux| files):
%    \begin{macrocode}
\newcommand{\childdocmain}[1]
{
  \childdocdisable\childdocmain{}
  \if?#1?\else
    \begingroup
      \def\childdoctmp{#1}
      \ifx\childdoctmp\childdocname
        \def\childdoctmp{}
      \else
        \def\childdoctmp
        {
          \childdoctrue
          \includeonly{\childdocname}
          \def\childdocjob{#1}
          \def\jobname{#1}
        }
      \fi
      \expandafter
    \endgroup
    \childdoctmp
  \fi
}
%    \end{macrocode}

% \macro{\childdocof}
% The command |\childdocof| redirects
% compilation to the main file |#1|.
%    \begin{macrocode}
\newcommand{\childdocof}[1]
{
  \childdocdisable
  \childdoctrue
  \includeonly{\childdocname}
  \def\jobname{#1}
  \def\childdocjob{#1}
  \input{#1}
}
%    \end{macrocode}

% \macro{\childdocby}
% The command |\childdocby| ....
%    \begin{macrocode}
\newcommand{\childdocby}[2][]
{
  \childdocdisable
  \childdoctrue
  \childdocmanualtrue
  \if?#1?\else
    \def\jobname{#2}
  \fi
  \def\childdocjob{#2}
  \input{#2}
  \endinput
}
%    \end{macrocode}

% \macro{\childdocforward}
% The command |\childdocforward| redirects
% compilation to the main file or
% (if the optional argument is given) a child file.
% Parameters are set as if the main file
% or a child file starting with |\childdocof| was compiled.
% Then compilation is handed over to the main file:
%    \begin{macrocode}
\newcommand{\childdocforward}[2][]
{
  \begingroup
    \if?#1?
      \def\childdoctmp
      {
        \def\childdocname{#2}
        \def\childdocjob{#2}
        \def\jobname{#2}
        \input{#2}
        \endinput
      }
    \else
      \def\childdoctmp
      {
        \childdocdisable
        \def\childdocname{#2}
        \childdoctrue
        \includeonly{#2}
        \def\childdocjob{#1}
        \def\jobname{#1}
        \input{#1}
        \endinput
      }
    \fi
    \expandafter
  \endgroup
  \childdoctmp
}
%    \end{macrocode}

% \macro{\childdocforwardprefix}
% The command |\childdocforwardprefix| redirects
% compilation to the main or a child file by means of a pattern.
% The prefix |#1| in the current filename is replaced by |#2|
% and the suffix of the current filename is kept
% (it is assumed that the filename does not contain the substring `|~~~|'
% which is used as a delimiter).
% Compilation is handed over to the new file by |\childdocforward|:
%    \begin{macrocode}
\newcommand{\childdocforwardprefix}[3][]
{
  \begingroup
    \def\childdocextract #2##1~~~{\def\childdoctmp{\childdocforward[#1]{#3##1}}}
    \expandafter\childdocextract\childdocname~~~
    \expandafter
  \endgroup
  \childdoctmp
}
%    \end{macrocode}

% \macro{\childdoc}
% The deprecated macro |\childdoc| is a legacy version of |\childdocmain|:
%    \begin{macrocode}
\newcommand{\childdoc}{\childdocmain}
%    \end{macrocode}

% \macro{\childdocredirect}
% The deprecated macro |\childdocredirect| is a legacy version
% of |\childdocforward| and |\childdocforwardprefix|:
%    \begin{macrocode}
\newcommand{\childdocredirect}[2][]
{
  \begingroup
    \if?#1?
      \def\childdoctmp{\childdocforward{#2}}
    \else
      \def\childdoctmp{\childdocforwardprefix{#1}{#2}}
    \fi
    \expandafter
  \endgroup
  \childdoctmp
}
%    \end{macrocode}

%\iffalse
%</package>
%\fi
%
\endinput
\childdocforward{cdocsamp}"|\\
% |latex -jobname cdocscl1 \|\\
% |  "% \iffalse
%
% childdoc.dtx Copyright (C) 2017-2018 Niklas Beisert
%
% This work may be distributed and/or modified under the
% conditions of the LaTeX Project Public License, either version 1.3
% of this license or (at your option) any later version.
% The latest version of this license is in
%   http://www.latex-project.org/lppl.txt
% and version 1.3 or later is part of all distributions of LaTeX
% version 2005/12/01 or later.
%
% This work has the LPPL maintenance status `maintained'.
%
% The Current Maintainer of this work is Niklas Beisert.
%
% This work consists of the files childdoc.dtx and childdoc.ins
% and the derived files childdoc.def and cdocsamp.tex with
% cdocsch1.tex, cdocsch2.tex, cdocsdrf.tex, cdocsfn1.tex, cdocsfn2.tex.
%
%<package>\ifdefined\childdocmain\endinput\fi
%<package>\ProvidesFile{childdoc.def}[2018/12/30 v2.0 child document driver]
%<samplemain>\ProvidesFile{cdocsamp.tex}[2018/12/30 v2.0 sample for childdoc]
%<*driver>
%\ProvidesFile{childdoc.drv}[2018/12/30 v2.0 childdoc reference manual file]
\PassOptionsToClass{10pt,a4paper}{article}
\documentclass{ltxdoc}

\usepackage[margin=35mm]{geometry}
\usepackage{hyperref}
\usepackage{hyperxmp}
\usepackage[usenames]{color}

\hypersetup{colorlinks=true}
\hypersetup{pdfstartview=FitH}
\hypersetup{pdfpagemode=UseNone}
\hypersetup{pdfsource={}}
\hypersetup{pdflang={en-UK}}
\hypersetup{pdfcopyright={Copyright 2017-2018 Niklas Beisert.
  This work may be distributed and/or modified under the
  conditions of the LaTeX Project Public License, either version 1.3
  of this license or (at your option) any later version.}}
\hypersetup{pdflicenseurl={http://www.latex-project.org/lppl.txt}}
\hypersetup{pdfcontactaddress={ETH Zurich, ITP, HIT K,
  Wolfgang-Pauli-Strasse 27}}
\hypersetup{pdfcontactpostcode={8093}}
\hypersetup{pdfcontactcity={Zurich}}
\hypersetup{pdfcontactcountry={Switzerland}}
\hypersetup{pdfcontactemail={nbeisert@itp.phys.ethz.ch}}
\hypersetup{pdfcontacturl={http://people.phys.ethz.ch/\xmptilde nbeisert/}}

\newcommand{\secref}[1]{\hyperref[#1]{section \ref*{#1}}}

\parskip1ex
\parindent0pt
\let\olditemize\itemize
\def\itemize{\olditemize\parskip0pt}

\begin{document}

\title{The \textsf{childdoc} Package}
\hypersetup{pdftitle={The childdoc Package}}
\author{Niklas Beisert\\[2ex]
  Institut f\"ur Theoretische Physik\\
  Eidgen\"ossische Technische Hochschule Z\"urich\\
  Wolfgang-Pauli-Strasse 27, 8093 Z\"urich, Switzerland\\[1ex]
  \href{mailto:nbeisert@itp.phys.ethz.ch}
  {\texttt{nbeisert@itp.phys.ethz.ch}}}
\hypersetup{pdfauthor={Niklas Beisert}}
\hypersetup{pdfsubject={Manual for the LaTeX2e Package childdoc}}
\date{30 December 2018, \textsf{v2.0}}
\maketitle

\begin{abstract}\noindent
\textsf{childdoc} is a \LaTeXe{} package
that enables the direct compilation
of document sections included by |\include|
to individual files.
\end{abstract}

\begingroup
\parskip0ex
\tableofcontents
\endgroup

%%%%%%%%%%%%%%%%%%%%%%%%%%%%%%%%%%%%%%%%%%%%%%%%%%%%%%%%%%%%%%%%%%%%%%%%%%%%%%%%
%%%%%%%%%%%%%%%%%%%%%%%%%%%%%%%%%%%%%%%%%%%%%%%%%%%%%%%%%%%%%%%%%%%%%%%%%%%%%%%%
\section{Introduction}

\LaTeX{} provides a mechanism to structure a large document (such as a book)
into a main file and several child files (containing the chapters)
using the |\include| command.
This mechanism is beneficial for documents
which span hundreds of pages in order to
make the source file(s) more manageable.
Moreover, compilation can be restricted to
selected child files by means of the |\includeonly| command.
The latter feature can be used to reduce the compilation time while editing
(this was significantly more useful in the earlier days of \LaTeX{})
or to generate a smaller document which is easier to navigate.
Another application of |\includeonly| is to generate
documents consisting of selected parts of the complete document.

However, there are a few drawbacks of the plain |\include| mechanism:
\begin{itemize}
\item
The child files cannot be compiled on their own,
they can only be compiled via the main file.
A naive editing environment
(such as a text editor with an option
to have the current file processed by \LaTeX)
may require one to switch to the main file before compiling;
attempting to compile the child file produces errors.
\item
The main file must be modified (each time)
to adjust the |\includeonly| command
to the present needs. This easily leaves the main file in a messy state.
\item
The generated document will always carry the filename
of the main document. This is inconvenient if
several child files are to be compiled and
to be kept for distribution.
\end{itemize}

The present package provides a simple interface
to make child files individually compilable by \LaTeX{}.
Compiling a child file then has the same effect as compiling
the main file with an |\includeonly| command
to select the appropriate child.
Moreover the generated document will carry the name of the child
rather than the main file.
This resolves all three above issues.

This feature is meant to make the editing of books,
thesis documents and lecture notes somewhat more convenient.
However, the package can also be used efficiently for
composing a series of documents (such as exercise sheets)
which are typically distributed individually.
It then assists the author in generating the individual documents
(potentially in different versions)
as well as a document containing the collected series.
Another application is in developing style files
or other kinds of included material
where compilation of the style file could redirect
to a sample or test file.

%%%%%%%%%%%%%%%%%%%%%%%%%%%%%%%%%%%%%%%%%%%%%%%%%%%%%%%%%%%%%%%%%%%%%%%%%%%%%%%%
%%%%%%%%%%%%%%%%%%%%%%%%%%%%%%%%%%%%%%%%%%%%%%%%%%%%%%%%%%%%%%%%%%%%%%%%%%%%%%%%
\section{Usage}

First of all, the package \textsf{childdoc} is \emph{not} a standard
\LaTeXe{} |.sty| style file! Therefore it needs to be invoked in
a non-standard way.

%%%%%%%%%%%%%%%%%%%%%%%%%%%%%%%%%%%%%%%%%%%%%%%%%%%%%%%%%%%%%%%%%%%%%%%%%%%%%%%%
\subsection{Included Files}
\label{sec:include}

%%%%%%%%%%%%%%%%%%%%%%%%%%%%%%%%%%%%%%%%
\DescribeMacro{\childdocmain}
To use the package, add the commands
\begin{center}
\begin{tabular}{l}
|\input{childdoc.def}|\\
|\childdocmain{}|\\
\end{tabular}
\end{center}
at the very top of the main \LaTeX{} file,
in particular \emph{before} the |\documentclass| statement!
The argument of |\childdocmain| should be left empty
(but it must be present).

%%%%%%%%%%%%%%%%%%%%%%%%%%%%%%%%%%%%%%%%
\DescribeMacro{\childdocof}
Furthermore, add the commands
\begin{center}
\begin{tabular}{l}
|\input{childdoc.def}|\\
|\childdocof{|\textit{main}|}|\\
\end{tabular}
\end{center}
at the top of every child file \textit{child}
which is included by |\include{|\textit{child}|}|
from within the main file
(or at least for those files to be compiled individually).
The argument \textit{main} must be the filename of the main file.

There are a couple of
considerations in setting up the main and child documents:

%%%%%%%%%%%%%%%%%%%%%%%%%%%%%%%%%%%%%%%%
\paragraph{Restrictions.}

Please note the following restrictions:
\begin{itemize}
\item
|\childdocmain| must be called with one argument \textit{main}
to ensure compatibility with earlier version of the package.
It must either be empty (|\childdocmain{}|)
or precisely match the filename of the main file in which it is specified.
See \secref{sec:detection} for further information.
\item
The filename \textit{main} must be specified without the |.tex| extension.
\item
The filename \textit{main} is case sensitive
(even in case-insensitive file systems)
due to internal string comparison.
\item
The argument \textit{main} should be fully expanded, it cannot be a macro.
\item
Subdirectories and special characters should be avoided in filenames.
\item
The command |\childdocmain{|\textit{main}|}| must be followed by a whitespace.
It should not be followed immediately by another command
or by a comment mark `|%|'.
This is because the \TeX{} parser reads the token immediately following
the argument of |\childdocmain| and puts it
at the beginning of every child section;
however, a white\-space is ignored.
\end{itemize}

%%%%%%%%%%%%%%%%%%%%%%%%%%%%%%%%%%%%%%%%
\paragraph{Content of Main File.}

It is advisable to place all content in the child files included by |\include|.
Any output contained in the main file will appear in all child documents
unless suppressed manually;
it cannot be suppressed automatically by the |\includeonly| directive
and thus should normally be avoided.
A method to include some content in the main file
by means of conditional processing is described in \secref{sec:conditional}.

%%%%%%%%%%%%%%%%%%%%%%%%%%%%%%%%%%%%%%%%
\paragraph{Page Numbering.}

When only a part of the document is compiled,
the appropriate numbering of pages
(as well as other status parameters)
is determined from the |.aux| files.
The latter contain information from previous passes.
However this information needs to propagate through
all intermediate child documents.
Therefore the page numbering in child documents may well
be inconsistent until the complete document is compiled at least once.

A useful (if unconventional) way to always ensure a consistent
page numbering is to restart the numbering in each child document
and denote the pages by `\textit{child}|.|\textit{page}'
where \textit{child} represents the chapter/section number of the child file.
This can be achieved by the command
|\numberwithin{page}{|\textit{child}|}|
of the \textsf{amsmath} package
where \textit{child} can be |chapter| or |section|
depending on the chosen structuring.
Alternatively, one can modify the macro |\thepage| appropriately
and reset the counter |page| at the start of each child file.

%%%%%%%%%%%%%%%%%%%%%%%%%%%%%%%%%%%%%%%%%%%%%%%%%%%%%%%%%%%%%%%%%%%%%%%%%%%%%%%%
\subsection{Conditional Processing}
\label{sec:conditional}

The package provides a mechanism to compile different versions
of a document. To customise the versions further some conditional processing
can come in handy to distinguish which version is being compiled.
The package provides two macros to describe the compilation context:

%%%%%%%%%%%%%%%%%%%%%%%%%%%%%%%%%%%%%%%%
\DescribeMacro{\ifchilddoc}
The conditional |\ifchilddoc| distinguishes between the compilation of
child documents and the main document:
%
\begin{center}
|\ifchilddoc |\textit{child-code}| |[|\||else |\textit{main-code}]| \||fi|
\end{center}

%%%%%%%%%%%%%%%%%%%%%%%%%%%%%%%%%%%%%%%%
\DescribeMacro{\childdocname}
\DescribeMacro{\childdocjob}
The macro |\childdocname| contains the filename (without extension)
of the main or child file being processed.
Note that |\childdocjob| will always contain the name of the main file.

%%%%%%%%%%%%%%%%%%%%%%%%%%%%%%%%%%%%%%%%
\paragraph{Title Page.}

Conditional processing can be used to include a title or banner page
in the main document when proper precautions are taken.
Importantly, the code in the main file should ensure that the page counter
(as well as other status parameters which are stored in the |.aux| files)
takes the same value after the conditional processing.
Otherwise the page numbers may take divergent values
depending on which part is compiled.

For example, a title page could be declared by:
%
\begin{center}
\begin{tabular}{l}
|\ifchilddoc\||else|\\
|\addtocounter{page}{-1}|\\
\textit{code for title page}\\
|\newpage|\\
|\||fi|
\end{tabular}
\end{center}
%
A banner page for the child documents can be generated by:
%
\begin{center}
\begin{tabular}{l}
|\ifchilddoc|\\
|\addtocounter{page}{-1}|\\
\textit{code for banner page}\\
|\newpage|\\
|\||fi|
\end{tabular}
\end{center}
%
Here one could write a message such as:
\begin{center}
|This is the part \childdocname{} of \childdocjob{}.|
\end{center}

%%%%%%%%%%%%%%%%%%%%%%%%%%%%%%%%%%%%%%%%%%%%%%%%%%%%%%%%%%%%%%%%%%%%%%%%%%%%%%%%
\subsection{Flags}
\label{sec:flags}

The package makes it easy to generate different versions
of the main or child documents.
To this end compilation flags can be defined
and assigned different default values.
They will be particularly useful in conjunction
with the forwarding mechanism described in \secref{sec:forward}.

For example, it may be useful to have a flag |\version|
which can be set to |draft| or |final|.
The document source will contain some conditional code
depending on the value of |\version|.
Suppose further, the flag should default to |final| for the main file
and to |draft| for child files
which is a natural assignment for editing the document.
This is achieved by placing the following code
in the preamble of the main document
(below the |\childdocmain| directive):
%
\begin{center}
\begin{tabular}{l}
|\ifchilddoc|\\
|\providecommand{\version}{draft}|\\
|\||else|\\
|\providecommand{\version}{final}|\\
|\||fi|
\end{tabular}
\end{center}
%
The definition by |\providecommand| makes sure
that previous definitions are not overwritten.
Further statements |\providecommand{\version}{...}|
can thus be added before the above code to override it.

For the main file, one might add a line
(between |\childdocmain| and the above block)
%
\begin{center}
|%\ifchilddoc\||else\providecommand{\version}{draft}\||fi|
\end{center}
%
which can be uncommented to produce a draft version.
Likewise one can add a line to the very top of a child file
(above the |\childdocof{|\textit{main}|}| directive)
%
\begin{center}
|%\providecommand{\version}{final}|
\end{center}
%
which can be uncommented to produce the final version of this child document.

%%%%%%%%%%%%%%%%%%%%%%%%%%%%%%%%%%%%%%%%%%%%%%%%%%%%%%%%%%%%%%%%%%%%%%%%%%%%%%%%
\subsection{Forwarding}
\label{sec:forward}

Different versions of the main or child documents
using compilation flags as described in \secref{sec:flags}
can be (permanently) stored in different files
for convenient compilation, viewing and distribution.
To this end, the package defines a command
to pass on compilation to a different file:

%%%%%%%%%%%%%%%%%%%%%%%%%%%%%%%%%%%%%%%%
\DescribeMacro{\childdocforward}
The command |\childdocforward| redirects processing to
another source file:
%
\begin{center}
\begin{tabular}{l}
|\input{childdoc.def}|\\
|\childdocforward[|\textit{main}|]{|\textit{dest}|}|\\
\end{tabular}
\end{center}
%
The argument \textit{dest} is the destination file
(without extension).
It should be the main file or one of the child files.
Note that further \textsf{childdoc} directives
such as |\childdocof| and |\childdocforward|
in the indicated file will be processed in this form.
The optional argument \textit{main}
passes on directly to the main file \textit{main}
while pretending to compile the child \textit{dest}.
This form behaves as if \textit{dest}
issues |\childdocof{|\textit{main}|}| right away,
and no further \textsf{childdoc} directives will be processed.

%%%%%%%%%%%%%%%%%%%%%%%%%%%%%%%%%%%%%%%%
\DescribeMacro{\...prefix}
In the alternative form |\childdocforwardprefix|,
%
\begin{center}
\begin{tabular}{l}
|\input{childdoc.def}|\\
|\childdocforwardprefix[|\textit{main}|]{|\textit{prefix}|}{|\textit{dest}|}|
\end{tabular}
\end{center}
%
the destination file is determined by a pattern
depending on the current file:
To make this work, the current file must be called
`{\textit{prefix}\hspace{0.2em}\textit{suffix}}'
with \textit{prefix} matching precisely the argument.
Processing is then passed on to the file
`{\textit{dest}\hspace{0.2em}\textit{suffix}}'.
Surely, the same effect is achieved by
directly specifying the
argument `{\textit{dest}\hspace{0.2em}\textit{suffix}}'
in the first form.
However, that requires to set up a different file
for each child. With the alternative form of the command
all these files can have exactly the same content
which simplifies setting them up and maintaining them.

For example, the following file |draft.tex|
with a compilation flag |\version| as described in \secref{sec:flags}
compiles the main document as a draft:
%
\begin{center}
\begin{tabular}{l}
|\def\version{draft}|\\
|\input{childdoc.def}|\\
|\childdocforward{|\textit{main}|}|
\end{tabular}
\end{center}
%
Likewise, the following files |final|\textit{nn}|.tex|
compile the final version of the child document
|child|\textit{nn}|.tex|:
%
\begin{center}
\begin{tabular}{l}
|\def\version{final}|\\
|\input{childdoc.def}|\\
|\childdocforwardprefix{final}{child}|
\end{tabular}
\end{center}
%

Note that when several versions of a main file and/or of each child file
are to be generated, it may be convenient to set up a |Makefile| or
shell script to automatise the process.

%%%%%%%%%%%%%%%%%%%%%%%%%%%%%%%%%%%%%%%%%%%%%%%%%%%%%%%%%%%%%%%%%%%%%%%%%%%%%%%%
\subsection{Command Line Processing}
\label{sec:commandline}

The effect of redirection files can also be achieved by invoking
the \LaTeX{} compiler with a more elaborate command line.
Most conveniently this should be done as part
of a shell script or a |Makefile|.

When using \textsf{childdoc} in the main file, the following
command lines effectively perform a redirection
(note that depending on the shell being used,
backslashes may have to be doubled: `|\|' $\to$ `|\\|'):
%
\begin{center}
|... -jobname "|\textit{target}|" |\\|"|[\textit{flags}]%
|\input{childdoc.def}\childdocforward[|\textit{main}|]{|\textit{dest}|}"|
\end{center}
%
Here \textit{target} is the name of the output file,
\textit{main} is the name of the main file
and \textit{dest} is the name of the main or child file to be processed
(all filenames without extensions).
The optional argument \textit{main} can be omitted
if \textit{main} matches \textit{dest}.
Optionally, compilation \textit{flags} can be defined via |\def| commands.
This command line makes the \TeX{} engine believe
it is compiling the file \textit{target}
whose content is specified as the latter parameter.
The provided code then forwards the processing to
\textit{main} or \textit{dest} as described in \secref{sec:forward}.

%%%%%%%%%%%%%%%%%%%%%%%%%%%%%%%%%%%%%%%%%%%%%%%%%%%%%%%%%%%%%%%%%%%%%%%%%%%%%%%%
\subsection{Include by Input}
\label{sec:input}

Including child documents by |\include| has some restrictions by design.
Most notably, the content of a child document always occupies
its own set of pages; pages cannot be shared between child documents.
Usually, this behaviour makes perfect sense
because each child document contain an essential part of the document.
However, in some situations it may be desirable to compose
a document from a collection of parts
without having mandatory page breaks between then.
For this case, the package
provides a mechanism to include parts
by |\input| which can also be processed individually.
However, by construction this mechanism
requires manual handling of the content to be output.

%%%%%%%%%%%%%%%%%%%%%%%%%%%%%%%%%%%%%%%%
\DescribeMacro{\ifchilddocmanual}
The main file should be prepared as usual, see \secref{sec:include}.
However, the document body must make a distinction
between processing of an individual part and of the main document, e.g.:
%
\begin{center}
\begin{tabular}{l}
|\ifchilddocmanual|\\
|\input{\childdocname}|\\
|\||else|\\
\textit{document body with }|\input{|\textit{part}|}|\\
|\||fi|
\end{tabular}
\end{center}
%
The conditional |\ifchilddocmanual| is true whenever
a part to be included by |\input| is being compiled,
and the name of the part is stored in |\childdocname|.

%%%%%%%%%%%%%%%%%%%%%%%%%%%%%%%%%%%%%%%%
\DescribeMacro{\childdocby}
Each part to be included by |\input| should start with:
%
\begin{center}
\begin{tabular}{l}
|\input{childdoc.def}|\\
|\childdocby{|\textit{main}|}|\\
\end{tabular}
\end{center}
%
The directive |\childdocby| is similar to |\childdocof|
described in \secref{sec:include},
but the subsequent selection of content must be done manually.
To that end, both |\ifchilddoc| and |\ifchilddocmanual|
will be true upon processing of a part,
and the name of the part is stored in |\childdocname|.
Note that |\jobname| will be set to the filename of the current part
so that each part receives an individual |.aux| file
that does not interfere with the |.aux| file(s) of the main document.
This behaviour can be altered by the alternative form
|\childdocby[*]{|\textit{main}|}| (with a non-empty optional argument)
which uses the |.aux| file of the main document
by setting |\jobname| to \textit{main}.

%%%%%%%%%%%%%%%%%%%%%%%%%%%%%%%%%%%%%%%%%%%%%%%%%%%%%%%%%%%%%%%%%%%%%%%%%%%%%%%%
\subsection{Driver Development}
\label{sec:driver}

The \textsf{childdoc} mechanism can also be use for the development
of definition files such as \LaTeX{} styles or classes.
This case differs from the above setup with multiple parts
included by |\include| in that no |\includeonly| should be invoked.
This can be achieved by starting the include file
(before |\ProvidesPackage|) with:
%
\begin{center}
\begin{tabular}{l}
|\input{childdoc.def}|\\
|\childdocforward{|\textit{main}|}|\\
\end{tabular}
\end{center}
%
or alternatively with:
%
\begin{center}
\begin{tabular}{l}
|\input{childdoc.def}|\\
|\childdocby{|\textit{main}|}|\\
\end{tabular}
\end{center}
%
Both forms have slightly different effects as described above.
The main file is prepared as usual, see \secref{sec:include}.

%%%%%%%%%%%%%%%%%%%%%%%%%%%%%%%%%%%%%%%%%%%%%%%%%%%%%%%%%%%%%%%%%%%%%%%%%%%%%%%%
\subsection{Legacy Detection}
\label{sec:detection}

The directive |\childdocmain| in the main file can detect
whether the complete document or merely a child is to be compiled
even without using the directive |\childdocof|.
This method is deprecated because it is less robust
and there is no compelling reason to use it;
it is merely provided for backward compatibility
and it may be removed in future versions.

If the detection mechanism is to be used,
it is mandatory to correctly specify
the filename of the main file as the argument of |\childdocmain|:
%
\begin{center}
\begin{tabular}{l}
|\input{childdoc.def}|\\
|\childdocmain{|\textit{main}|}|\\
\end{tabular}
\end{center}
%
If |\jobname| does not match the argument \textit{main} of |\childdocmain|,
it is assumed that |\jobname| points to the child file to be compiled.
When using |\childdocmain| with the main file specified as argument,
it suffices to start a child file
with just |\input{|\textit{main}|}|
without loading of the package and using |\childdocof|.
If instead all processing is done
with the appropriate \textsf{childdoc} directives,
the argument of \textit{main} of |\childdocmain| can be empty.

An alternative version of the command line processing described
in \secref{sec:commandline} using the detection mechanism reads:
%
\begin{center}
|... -jobname "|\textit{target}|" "|[\textit{flags}]%
[|\def\jobname{|\textit{dest}|}|]|\input{|\textit{main}|}"|
\end{center}

%%%%%%%%%%%%%%%%%%%%%%%%%%%%%%%%%%%%%%%%%%%%%%%%%%%%%%%%%%%%%%%%%%%%%%%%%%%%%%%%
\subsection{Manual Code}
\label{sec:manual}

In case one cannot be certain whether the definitions file |childdoc.def|
is installed on the target \TeX{} distribution
and one prefers not to ship it,
it is conceivable to paste a few relevant commands into the sources.

To that end, drop all statements |\input{childdoc.def}|
and perform the replacements as outlined below.
Instead of |\childdocmain{|\textit{main}|}| add the following code
to the top of the main file:
%
\begin{center}
\begin{tabular}{l}
|\||ifdefined\childdocname\endinput\||fi\newif\ifchilddoc|\\
|\edef\childdocname{\scantokens\expandafter{\jobname\noexpand}}|\\
|\def\childdocmain{|\textit{main}|}\||ifx\childdocmain\childdocname\||else|\\
|\childdoctrue\includeonly{\childdocname}\let\jobname\childdocmain\||fi|\\
\end{tabular}
\end{center}
%
Instead of |\childdocof{|\textit{main}|}| just include the main file
at the top of each child file:
%
\begin{center}
|\input{|\textit{main}|}|
\end{center}
%
A simple redirection |\childdocforward{|\textit{dest}|}| is achieved by:
%
\begin{center}
|\def\jobname{|\textit{dest}|}\input{\jobname}|
\end{center}
%
The redirection with prefix
|\childdocforwardprefix[|\textit{prefix}|]{|\textit{dest}|}|
is accomplished by:
%
\begin{center}
\begin{tabular}{l}
|{\edef\jobname{\scantokens\expandafter{\jobname\noexpand}}|\\
|\def\redirectjob |\textit{prefix}|#1~~~{\gdef\jobname{|\textit{dest}|#1}}|\\
|\expandafter\redirectjob\jobname~~~}\input{\jobname}|
\end{tabular}
\end{center}

In an alternative approach,
child documents can be compiled by a specific command line
without additional code or specific definitions:
%
\begin{center}
|... -jobname "|\textit{target}|" "|[\textit{flags}]%
|\includeonly{|\textit{dest}|}\input{|\textit{main}|}"|
\end{center}
%

%%%%%%%%%%%%%%%%%%%%%%%%%%%%%%%%%%%%%%%%%%%%%%%%%%%%%%%%%%%%%%%%%%%%%%%%%%%%%%%%
%%%%%%%%%%%%%%%%%%%%%%%%%%%%%%%%%%%%%%%%%%%%%%%%%%%%%%%%%%%%%%%%%%%%%%%%%%%%%%%%
\section{Information}

%%%%%%%%%%%%%%%%%%%%%%%%%%%%%%%%%%%%%%%%%%%%%%%%%%%%%%%%%%%%%%%%%%%%%%%%%%%%%%%%
\subsection{Copyright}

Copyright \copyright{} 2017--2018 Niklas Beisert

This work may be distributed and/or modified under the
conditions of the \LaTeX{} Project Public License, either version 1.3
of this license or (at your option) any later version.
The latest version of this license is in
  \url{http://www.latex-project.org/lppl.txt}
and version 1.3 or later is part of all distributions of \LaTeX{}
version 2005/12/01 or later.

This work has the LPPL maintenance status `maintained'.

The Current Maintainer of this work is Niklas Beisert.

This work consists of the files |README.txt|, |childdoc.ins| and |childdoc.dtx|
as well as the derived files |childdoc.def|, |cdocsamp.tex|
with |cdocsch1.tex|, |cdocsch2.tex|, |cdocspt3.tex|, |cdocspt4.tex|,
|cdocsdrf.tex|, |cdocsfn1.tex|, |cdocsfn2.tex|
as well as |childdoc.pdf|.

%%%%%%%%%%%%%%%%%%%%%%%%%%%%%%%%%%%%%%%%%%%%%%%%%%%%%%%%%%%%%%%%%%%%%%%%%%%%%%%%
\subsection{Files and Installation}

The package consists of the files:
%
\begin{center}
\begin{tabular}{ll}
    |README.txt|   & readme file \\
    |childdoc.ins| & installation file \\
    |childdoc.dtx| & source file \\
    |childdoc.def| & definition file \\
    |cdocsamp.tex| & sample main file \\
    |cdocsch1.tex| & sample include file \\
    |cdocsch2.tex| & sample include file \\
    |cdocspt3.tex| & sample part file \\
    |cdocspt4.tex| & sample part file \\
    |cdocsdrf.tex| & sample redirection file \\
    |cdocsfn1.tex| & sample redirection file \\
    |cdocsfn2.tex| & sample redirection file \\
    |childdoc.pdf| & manual
\end{tabular}
\end{center}
%
The distribution consists of the files
|README.txt|, |childdoc.ins| and |childdoc.dtx|.
%
\begin{itemize}
\item
Run (pdf)\LaTeX{} on |childdoc.dtx|
to compile the manual |childdoc.pdf| (this file).
\item
Run \LaTeX{} on |childdoc.ins| to create the definitions file |childdoc.def|
and the sample |cdocsamp.tex| with include files
|cdocsch1.tex|, |cdocsch2.tex|, |cdocspt3.tex|, |cdocspt4.tex|,
|cdocsdrf.tex|, |cdocsfn1.tex|, |cdocsfn2.tex|.
Then copy the file |childdoc.def| to an appropriate directory of your \LaTeX{}
distribution, e.g.\ \textit{texmf-root}|/tex/latex/childdoc|.
\end{itemize}

%%%%%%%%%%%%%%%%%%%%%%%%%%%%%%%%%%%%%%%%%%%%%%%%%%%%%%%%%%%%%%%%%%%%%%%%%%%%%%%%
\subsection{Related CTAN Packages}

There are several other packages which offer a similar functionality:
%
\begin{itemize}
\item
The packages
\href{http://ctan.org/pkg/docmute}{\textsf{docmute}},
\href{http://ctan.org/pkg/includex}{\textsf{includex}} and
\href{http://ctan.org/pkg/standalone}{\textsf{standalone}}
provide commands to include only the document body of
a child file thus allowing both files to be compiled individually.
\item
The packages \href{http://ctan.org/pkg/subdocs}{\textsf{subdocs}}
and \href{http://ctan.org/pkg/subfiles}{\textsf{subfiles}}
provide structures in which the main and child documents can be
encapsulated and allowing them to be compiled individually.
The inclusion mechanism is different from the conventional |\include|.
\item
The package \href{http://ctan.org/pkg/combine}{\textsf{combine}}
is an elaborate solution to combine several documents into one.
\end{itemize}
%
See also the CTAN topic \href{http://ctan.org/topic/subdocs}{\textsf{subdocs}}
for further related packages.
The present package differs from the above solutions in that
a document structure constructed with the conventional |\include| mechanism
just needs two extra commands at the top of every file
such that all constituent files can be compiled individually.

%%%%%%%%%%%%%%%%%%%%%%%%%%%%%%%%%%%%%%%%%%%%%%%%%%%%%%%%%%%%%%%%%%%%%%%%%%%%%%%%
%\subsection{Feature Suggestions}
%
%The following is a list of features which may be useful for future
%versions of this package:
%%
%\begin{itemize}
%\item
%\ldots
%\end{itemize}

%%%%%%%%%%%%%%%%%%%%%%%%%%%%%%%%%%%%%%%%%%%%%%%%%%%%%%%%%%%%%%%%%%%%%%%%%%%%%%%%
\subsection{Revision History}

%%%%%%%%%%%%%%%%%%%%%%%%%%%%%%%%%%%%%%%%
\paragraph{v2.0:} 2018/12/30

\begin{itemize}
\item
immediate forward processing
\item
added |\childdocby| mechanism
\item
manual restructured
\end{itemize}

%%%%%%%%%%%%%%%%%%%%%%%%%%%%%%%%%%%%%%%%
\paragraph{v1.6:} 2018/01/17

\begin{itemize}
\item
application for development of include files
\item
corrections to manual
\end{itemize}

%%%%%%%%%%%%%%%%%%%%%%%%%%%%%%%%%%%%%%%%
\paragraph{v1.5:} 2017/05/21

\begin{itemize}
\item
more complete structuring introduced
\item
|\childdocof| introduced
\item
|\childdoc| renamed to |\childdocmain|
\item
|\childredirect| renamed to |\childdocforward| and |\childdocforwardprefix|
and functionality expanded
\end{itemize}

%%%%%%%%%%%%%%%%%%%%%%%%%%%%%%%%%%%%%%%%
\paragraph{v1.0:} 2017/04/27

\begin{itemize}
\item
manual and install package
\item
first version published on CTAN
\end{itemize}

%%%%%%%%%%%%%%%%%%%%%%%%%%%%%%%%%%%%%%%%
\paragraph{v0.6:} 2017/04/26

\begin{itemize}
\item
redirection mechanism added
\end{itemize}

%%%%%%%%%%%%%%%%%%%%%%%%%%%%%%%%%%%%%%%%
\paragraph{v0.5:} 2017/04/26

\begin{itemize}
\item
functionality in definition file
\end{itemize}


%%%%%%%%%%%%%%%%%%%%%%%%%%%%%%%%%%%%%%%%%%%%%%%%%%%%%%%%%%%%%%%%%%%%%%%%%%%%%%%%
%%%%%%%%%%%%%%%%%%%%%%%%%%%%%%%%%%%%%%%%%%%%%%%%%%%%%%%%%%%%%%%%%%%%%%%%%%%%%%%%
%%%%%%%%%%%%%%%%%%%%%%%%%%%%%%%%%%%%%%%%%%%%%%%%%%%%%%%%%%%%%%%%%%%%%%%%%%%%%%%%
\appendix

\settowidth\MacroIndent{\rmfamily\scriptsize 000\ }

 \DocInput{childdoc.dtx}

\end{document}
%</driver>
% \fi
%
% %%%%%%%%%%%%%%%%%%%%%%%%%%%%%%%%%%%%%%%%%%%%%%%%%%%%%%%%%%%%%%%%%%%%%%%%%%%%%%
% %%%%%%%%%%%%%%%%%%%%%%%%%%%%%%%%%%%%%%%%%%%%%%%%%%%%%%%%%%%%%%%%%%%%%%%%%%%%%%
% \section{Sample}
%\iffalse
%<*samplemain>
%\fi
%
% The following presents a sample document
% with two chapters, two parts, a title page,
% a compile flag as well as three forwarding files to set the flag.
% It consists of eight |.tex| files:
% \begin{center}
% \begin{tabular}{ll}
% |cdocsamp.tex|&main file\\
% |cdocsch1.tex|&include file for chapter 1\\
% |cdocsch2.tex|&include file for chapter 2\\
% |cdocspt3.tex|&include file for part 3\\
% |cdocspt4.tex|&include file for part 4\\
% |cdocsdrf.tex|&forwarding file for main file in draft mode\\
% |cdocsfi1.tex|&forwarding file for final version of chapter 1\\
% |cdocsfi2.tex|&forwarding file for final version of chapter 2\\
% \end{tabular}
% \end{center}
% Each of the eight files can be compiled directly by the \LaTeX{} compiler.
%
% %%%%%%%%%%%%%%%%%%%%%%%%%%%%%%%%%%%%%%
% \paragraph{Main File.}
%
% The main file is called |cdocsamp.tex|.
%
% Load the \textsf{childdoc} definitions and
% declare the filename for the main document:
%    \begin{macrocode}
\input{childdoc.def}
\childdocmain{}
%    \end{macrocode}

% Optional override for |\version| flag:
%    \begin{macrocode}
%%\ifchilddoc\else\providecommand{\version}{draft}\fi
%    \end{macrocode}

% Define the default values for the |\version| flag
% (|final| for the main file and |draft| for childs):
%    \begin{macrocode}
\ifchilddoc
\providecommand{\version}{draft}
\else
\providecommand{\version}{final}
\fi
%    \end{macrocode}

% Load the standard document class:
%    \begin{macrocode}
\documentclass[12pt]{article}
%    \end{macrocode}

% Start the document body:
%    \begin{macrocode}
\begin{document}
%    \end{macrocode}

% Declare a title page.
% Print title, part of document being processed and version flag:
%    \begin{macrocode}
\addtocounter{page}{-1}
\begin{center}
{\LARGE\bfseries{}childdoc example\par}
\vspace{1cm}
\ifchilddoc
\ifchilddocmanual part\else chapter\fi:
`\childdocname' of `\childdocjob'\par
\else
main document: `\childdocjob'\par
\fi
version: \version\par
\end{center}
\newpage
%    \end{macrocode}

% Manually include selected file,
% otherwise process as usual:
%    \begin{macrocode}
\ifchilddocmanual
\section*{part `\childdocname'}
\input{\childdocname}
\else
%    \end{macrocode}

% Include the two chapters:
%    \begin{macrocode}
\include{cdocsch1}
\include{cdocsch2}
%    \end{macrocode}

% Include the two parts unless only chapters should be displayed:
%    \begin{macrocode}
\ifchilddoc\else
\section{part three}
\input{cdocspt3}
\section{part four}
\input{cdocspt4}
\fi
%    \end{macrocode}

% Process as usual until here:
%    \begin{macrocode}
\fi
%    \end{macrocode}

% End of document body:
%    \begin{macrocode}
\end{document}
%    \end{macrocode}
%\iffalse
%</samplemain>
%\fi
%
% %%%%%%%%%%%%%%%%%%%%%%%%%%%%%%%%%%%%%%
% \paragraph{Chapter Include Files.}
%
% The include files are called |cdocsch1.tex| and |cdocsch2.tex|.
%
%\iffalse
%<*samplechap1|samplechap2>
%\fi

% Optional override for |\version| flag:
%    \begin{macrocode}
%%\providecommand{\version}{final}
%    \end{macrocode}

% Include the main document:
%    \begin{macrocode}
\input{childdoc.def}
\childdocof{cdocsamp}
%    \end{macrocode}

%\iffalse
%</samplechap1|samplechap2>
%\fi
%
%\iffalse
%<*samplechap1>
%\fi
% Some text for chapter 1:
%    \begin{macrocode}
\section{one}
some text in chapter one
%    \end{macrocode}

%\iffalse
%</samplechap1>
%\fi
% Some text for chapter 2:
%\iffalse
%<*samplechap2>
%\fi
%    \begin{macrocode}
\section{two}
more text in chapter two
%    \end{macrocode}

%\iffalse
%</samplechap2>
%\fi
%
% %%%%%%%%%%%%%%%%%%%%%%%%%%%%%%%%%%%%%%
% \paragraph{Part Include Files.}
%
% The include files are called |cdocspt3.tex| and |cdocspt4.tex|.
%
%\iffalse
%<*samplepart3|samplepart4>
%\fi

% Optional override for |\version| flag:
%    \begin{macrocode}
%%\providecommand{\version}{final}
%    \end{macrocode}

% Include the main document:
%    \begin{macrocode}
\input{childdoc.def}
\childdocby{cdocsamp}
%    \end{macrocode}

%\iffalse
%</samplepart3|samplepart4>
%\fi
%
%\iffalse
%<*samplepart3>
%\fi
% Some text for part 3:
%    \begin{macrocode}
some text in part three
%    \end{macrocode}

%\iffalse
%</samplepart3>
%\fi
% Some text for part 4:
%\iffalse
%<*samplepart4>
%\fi
%    \begin{macrocode}
more text in part four
%    \end{macrocode}

%\iffalse
%</samplepart4>
%\fi
%
% %%%%%%%%%%%%%%%%%%%%%%%%%%%%%%%%%%%%%%
% \paragraph{Forwarding for a Complete Draft.}
%
% The following forwarding file |cdocsdrf.tex|
% compiles the main document in draft mode:
%\iffalse
%<*sampledraft>
%\fi
%    \begin{macrocode}
\def\version{draft}
\input{childdoc.def}
\childdocforward{cdocsamp}
%    \end{macrocode}

%\iffalse
%</sampledraft>
%\fi
%
% %%%%%%%%%%%%%%%%%%%%%%%%%%%%%%%%%%%%%%
% \paragraph{Forwarding for Final Version of the Chapters.}
%
% The following forwarding files |cdocsfn1.tex| and |cdocsfn2.tex|
% (with identical content)
% compile the final versions of the child documents
% |cdocsch1.tex| and |cdocsch2.tex|, respectively:
%\iffalse
%<*samplefinal>
%\fi
%    \begin{macrocode}
\def\version{final}
\input{childdoc.def}
\childdocforwardprefix[cdocsamp]{cdocsfn}{cdocsch}
%    \end{macrocode}

%\iffalse
%</samplefinal>
%\fi
%
% %%%%%%%%%%%%%%%%%%%%%%%%%%%%%%%%%%%%%%
% \paragraph{Command Line Processing.}
%
% The following three command lines generate the output files
% |cdocscld|, |cdocscl1| and |cdocscl2|
% which should be identical to
% |cdocsdrf|, |cdocsch1| and |cdocsfn2|, respectively:
% \begin{center}
% \begin{tabular}{l}
% |latex -jobname cdocscld \|\\
% |  "\def\version{draft}\input{childdoc.def}\childdocforward{cdocsamp}"|\\
% |latex -jobname cdocscl1 \|\\
% |  "\input{childdoc.def}\childdocforward[cdocsamp]{cdocsch1}"|\\
% |latex -jobname cdocscl2 \|\\
% |  "\def\version{final}\input{childdoc.def}\childdocforward{cdocsch2}"|
% \end{tabular}
% \end{center}
% Note that the trailing backslash on each first line
% merely continues the input to the second line
% (for convenient cut ant paste).
% Furthermore, the command |latex| can be replaced by any
% of its alternative versions such as |pdflatex|.
%
% %%%%%%%%%%%%%%%%%%%%%%%%%%%%%%%%%%%%%%%%%%%%%%%%%%%%%%%%%%%%%%%%%%%%%%%%%%%%%%
% %%%%%%%%%%%%%%%%%%%%%%%%%%%%%%%%%%%%%%%%%%%%%%%%%%%%%%%%%%%%%%%%%%%%%%%%%%%%%%
% \section{Implementation}
%\iffalse
%<*package>
%\fi
%
% This section describes the definitions file |childdoc.def|.

% The definitions cannot be loaded using |\usepackage| or |\RequirePackage|
% which has a mechanism to prevent loading a style file more than once.
% When loading the definitions by means of |\input|
% multiple instances have to be prevented manually:
%\iffalse
%This code needs to be before the `\ProvidesFile' directive
%which is defined at the beginning of this file.
%Therefore it is also placed there and commented out here.
%</package>
%<*discard>
%\fi
%    \begin{macrocode}
\ifdefined\childdocmain\endinput\fi
%    \end{macrocode}
%\iffalse
%</discard>
%<*package>
%\fi
%
% \macro{\ifchilddoc}
% \macro{\ifchilddocmanual}
% The conditional |\ifchilddoc| tells whether a
% child (true) or main (false) document is being compiled.
% The conditional |\ifchilddocmanual| tells whether
% the |\includeonly| mechanism is used (false) or
% the selection of child files must be performed manually (true).
% The definitions initialise to false:
%    \begin{macrocode}
\newif\ifchilddoc
\newif\ifchilddocmanual
%    \end{macrocode}

% \macro{\childdocname}
% \macro{\childdocjob}
% The macro |\childdocname| stores the name of the main document
% to be compiled. The macro |\childdocjob| stores the name of
% the document on which the \LaTeX{} compiler was originally invoked.
% The content of |\jobname| cannot be compared
% to filenames specified in the source due to different catcodes.
% The following code rescans |\jobname|, stores the result
% in |\childdocname| and saves a copy in |\childdocjob|:
%    \begin{macrocode}
\edef\childdocname{\scantokens\expandafter{\jobname\noexpand}}
\let\childdocjob\childdocname
%    \end{macrocode}

% \macro{\childdocdisable}
% The macro |\childdocdisable| prevents the main file
% from being processed more than once.
% At this stage, the main document command |\childdocmain|
% is assumed to be called once again where it should do nothing.
% Any subsequent call to it should prevent
% a secondary processing of the main document
% It overwrites the forwarding commands
% |\childdocof| and |\childdocforward|
% with empty macros to prevent further inclusions of the main document:
%    \begin{macrocode}
\newcommand{\childdocdisable}
{
  \renewcommand{\childdocmain}[1]{\renewcommand{\childdocmain}[1]{\endinput}}
  \renewcommand{\childdocof}[1]{}
  \renewcommand{\childdocby}[2][]{}
  \renewcommand{\childdocforward}[2][]{}
  \renewcommand{\childdocdisable}{}
}
%    \end{macrocode}

% \macro{\childdocmain}
% The macro |\childdocmain| is to be called at the top of the main file
% with nothing or the main filename (without extension) as argument.
% First, it breaks loops.
% If the argument is not empty and does not match |\childdocname|
% (which is set by the first inclusion of |childdoc.def|),
% |\ifchilddoc| is set to true, |\includeonly| is applied to the child file
% and |\jobname| is set to the main file
% (for proper handling of |.aux| files):
%    \begin{macrocode}
\newcommand{\childdocmain}[1]
{
  \childdocdisable\childdocmain{}
  \if?#1?\else
    \begingroup
      \def\childdoctmp{#1}
      \ifx\childdoctmp\childdocname
        \def\childdoctmp{}
      \else
        \def\childdoctmp
        {
          \childdoctrue
          \includeonly{\childdocname}
          \def\childdocjob{#1}
          \def\jobname{#1}
        }
      \fi
      \expandafter
    \endgroup
    \childdoctmp
  \fi
}
%    \end{macrocode}

% \macro{\childdocof}
% The command |\childdocof| redirects
% compilation to the main file |#1|.
%    \begin{macrocode}
\newcommand{\childdocof}[1]
{
  \childdocdisable
  \childdoctrue
  \includeonly{\childdocname}
  \def\jobname{#1}
  \def\childdocjob{#1}
  \input{#1}
}
%    \end{macrocode}

% \macro{\childdocby}
% The command |\childdocby| ....
%    \begin{macrocode}
\newcommand{\childdocby}[2][]
{
  \childdocdisable
  \childdoctrue
  \childdocmanualtrue
  \if?#1?\else
    \def\jobname{#2}
  \fi
  \def\childdocjob{#2}
  \input{#2}
  \endinput
}
%    \end{macrocode}

% \macro{\childdocforward}
% The command |\childdocforward| redirects
% compilation to the main file or
% (if the optional argument is given) a child file.
% Parameters are set as if the main file
% or a child file starting with |\childdocof| was compiled.
% Then compilation is handed over to the main file:
%    \begin{macrocode}
\newcommand{\childdocforward}[2][]
{
  \begingroup
    \if?#1?
      \def\childdoctmp
      {
        \def\childdocname{#2}
        \def\childdocjob{#2}
        \def\jobname{#2}
        \input{#2}
        \endinput
      }
    \else
      \def\childdoctmp
      {
        \childdocdisable
        \def\childdocname{#2}
        \childdoctrue
        \includeonly{#2}
        \def\childdocjob{#1}
        \def\jobname{#1}
        \input{#1}
        \endinput
      }
    \fi
    \expandafter
  \endgroup
  \childdoctmp
}
%    \end{macrocode}

% \macro{\childdocforwardprefix}
% The command |\childdocforwardprefix| redirects
% compilation to the main or a child file by means of a pattern.
% The prefix |#1| in the current filename is replaced by |#2|
% and the suffix of the current filename is kept
% (it is assumed that the filename does not contain the substring `|~~~|'
% which is used as a delimiter).
% Compilation is handed over to the new file by |\childdocforward|:
%    \begin{macrocode}
\newcommand{\childdocforwardprefix}[3][]
{
  \begingroup
    \def\childdocextract #2##1~~~{\def\childdoctmp{\childdocforward[#1]{#3##1}}}
    \expandafter\childdocextract\childdocname~~~
    \expandafter
  \endgroup
  \childdoctmp
}
%    \end{macrocode}

% \macro{\childdoc}
% The deprecated macro |\childdoc| is a legacy version of |\childdocmain|:
%    \begin{macrocode}
\newcommand{\childdoc}{\childdocmain}
%    \end{macrocode}

% \macro{\childdocredirect}
% The deprecated macro |\childdocredirect| is a legacy version
% of |\childdocforward| and |\childdocforwardprefix|:
%    \begin{macrocode}
\newcommand{\childdocredirect}[2][]
{
  \begingroup
    \if?#1?
      \def\childdoctmp{\childdocforward{#2}}
    \else
      \def\childdoctmp{\childdocforwardprefix{#1}{#2}}
    \fi
    \expandafter
  \endgroup
  \childdoctmp
}
%    \end{macrocode}

%\iffalse
%</package>
%\fi
%
\endinput
\childdocforward[cdocsamp]{cdocsch1}"|\\
% |latex -jobname cdocscl2 \|\\
% |  "\def\version{final}% \iffalse
%
% childdoc.dtx Copyright (C) 2017-2018 Niklas Beisert
%
% This work may be distributed and/or modified under the
% conditions of the LaTeX Project Public License, either version 1.3
% of this license or (at your option) any later version.
% The latest version of this license is in
%   http://www.latex-project.org/lppl.txt
% and version 1.3 or later is part of all distributions of LaTeX
% version 2005/12/01 or later.
%
% This work has the LPPL maintenance status `maintained'.
%
% The Current Maintainer of this work is Niklas Beisert.
%
% This work consists of the files childdoc.dtx and childdoc.ins
% and the derived files childdoc.def and cdocsamp.tex with
% cdocsch1.tex, cdocsch2.tex, cdocsdrf.tex, cdocsfn1.tex, cdocsfn2.tex.
%
%<package>\ifdefined\childdocmain\endinput\fi
%<package>\ProvidesFile{childdoc.def}[2018/12/30 v2.0 child document driver]
%<samplemain>\ProvidesFile{cdocsamp.tex}[2018/12/30 v2.0 sample for childdoc]
%<*driver>
%\ProvidesFile{childdoc.drv}[2018/12/30 v2.0 childdoc reference manual file]
\PassOptionsToClass{10pt,a4paper}{article}
\documentclass{ltxdoc}

\usepackage[margin=35mm]{geometry}
\usepackage{hyperref}
\usepackage{hyperxmp}
\usepackage[usenames]{color}

\hypersetup{colorlinks=true}
\hypersetup{pdfstartview=FitH}
\hypersetup{pdfpagemode=UseNone}
\hypersetup{pdfsource={}}
\hypersetup{pdflang={en-UK}}
\hypersetup{pdfcopyright={Copyright 2017-2018 Niklas Beisert.
  This work may be distributed and/or modified under the
  conditions of the LaTeX Project Public License, either version 1.3
  of this license or (at your option) any later version.}}
\hypersetup{pdflicenseurl={http://www.latex-project.org/lppl.txt}}
\hypersetup{pdfcontactaddress={ETH Zurich, ITP, HIT K,
  Wolfgang-Pauli-Strasse 27}}
\hypersetup{pdfcontactpostcode={8093}}
\hypersetup{pdfcontactcity={Zurich}}
\hypersetup{pdfcontactcountry={Switzerland}}
\hypersetup{pdfcontactemail={nbeisert@itp.phys.ethz.ch}}
\hypersetup{pdfcontacturl={http://people.phys.ethz.ch/\xmptilde nbeisert/}}

\newcommand{\secref}[1]{\hyperref[#1]{section \ref*{#1}}}

\parskip1ex
\parindent0pt
\let\olditemize\itemize
\def\itemize{\olditemize\parskip0pt}

\begin{document}

\title{The \textsf{childdoc} Package}
\hypersetup{pdftitle={The childdoc Package}}
\author{Niklas Beisert\\[2ex]
  Institut f\"ur Theoretische Physik\\
  Eidgen\"ossische Technische Hochschule Z\"urich\\
  Wolfgang-Pauli-Strasse 27, 8093 Z\"urich, Switzerland\\[1ex]
  \href{mailto:nbeisert@itp.phys.ethz.ch}
  {\texttt{nbeisert@itp.phys.ethz.ch}}}
\hypersetup{pdfauthor={Niklas Beisert}}
\hypersetup{pdfsubject={Manual for the LaTeX2e Package childdoc}}
\date{30 December 2018, \textsf{v2.0}}
\maketitle

\begin{abstract}\noindent
\textsf{childdoc} is a \LaTeXe{} package
that enables the direct compilation
of document sections included by |\include|
to individual files.
\end{abstract}

\begingroup
\parskip0ex
\tableofcontents
\endgroup

%%%%%%%%%%%%%%%%%%%%%%%%%%%%%%%%%%%%%%%%%%%%%%%%%%%%%%%%%%%%%%%%%%%%%%%%%%%%%%%%
%%%%%%%%%%%%%%%%%%%%%%%%%%%%%%%%%%%%%%%%%%%%%%%%%%%%%%%%%%%%%%%%%%%%%%%%%%%%%%%%
\section{Introduction}

\LaTeX{} provides a mechanism to structure a large document (such as a book)
into a main file and several child files (containing the chapters)
using the |\include| command.
This mechanism is beneficial for documents
which span hundreds of pages in order to
make the source file(s) more manageable.
Moreover, compilation can be restricted to
selected child files by means of the |\includeonly| command.
The latter feature can be used to reduce the compilation time while editing
(this was significantly more useful in the earlier days of \LaTeX{})
or to generate a smaller document which is easier to navigate.
Another application of |\includeonly| is to generate
documents consisting of selected parts of the complete document.

However, there are a few drawbacks of the plain |\include| mechanism:
\begin{itemize}
\item
The child files cannot be compiled on their own,
they can only be compiled via the main file.
A naive editing environment
(such as a text editor with an option
to have the current file processed by \LaTeX)
may require one to switch to the main file before compiling;
attempting to compile the child file produces errors.
\item
The main file must be modified (each time)
to adjust the |\includeonly| command
to the present needs. This easily leaves the main file in a messy state.
\item
The generated document will always carry the filename
of the main document. This is inconvenient if
several child files are to be compiled and
to be kept for distribution.
\end{itemize}

The present package provides a simple interface
to make child files individually compilable by \LaTeX{}.
Compiling a child file then has the same effect as compiling
the main file with an |\includeonly| command
to select the appropriate child.
Moreover the generated document will carry the name of the child
rather than the main file.
This resolves all three above issues.

This feature is meant to make the editing of books,
thesis documents and lecture notes somewhat more convenient.
However, the package can also be used efficiently for
composing a series of documents (such as exercise sheets)
which are typically distributed individually.
It then assists the author in generating the individual documents
(potentially in different versions)
as well as a document containing the collected series.
Another application is in developing style files
or other kinds of included material
where compilation of the style file could redirect
to a sample or test file.

%%%%%%%%%%%%%%%%%%%%%%%%%%%%%%%%%%%%%%%%%%%%%%%%%%%%%%%%%%%%%%%%%%%%%%%%%%%%%%%%
%%%%%%%%%%%%%%%%%%%%%%%%%%%%%%%%%%%%%%%%%%%%%%%%%%%%%%%%%%%%%%%%%%%%%%%%%%%%%%%%
\section{Usage}

First of all, the package \textsf{childdoc} is \emph{not} a standard
\LaTeXe{} |.sty| style file! Therefore it needs to be invoked in
a non-standard way.

%%%%%%%%%%%%%%%%%%%%%%%%%%%%%%%%%%%%%%%%%%%%%%%%%%%%%%%%%%%%%%%%%%%%%%%%%%%%%%%%
\subsection{Included Files}
\label{sec:include}

%%%%%%%%%%%%%%%%%%%%%%%%%%%%%%%%%%%%%%%%
\DescribeMacro{\childdocmain}
To use the package, add the commands
\begin{center}
\begin{tabular}{l}
|\input{childdoc.def}|\\
|\childdocmain{}|\\
\end{tabular}
\end{center}
at the very top of the main \LaTeX{} file,
in particular \emph{before} the |\documentclass| statement!
The argument of |\childdocmain| should be left empty
(but it must be present).

%%%%%%%%%%%%%%%%%%%%%%%%%%%%%%%%%%%%%%%%
\DescribeMacro{\childdocof}
Furthermore, add the commands
\begin{center}
\begin{tabular}{l}
|\input{childdoc.def}|\\
|\childdocof{|\textit{main}|}|\\
\end{tabular}
\end{center}
at the top of every child file \textit{child}
which is included by |\include{|\textit{child}|}|
from within the main file
(or at least for those files to be compiled individually).
The argument \textit{main} must be the filename of the main file.

There are a couple of
considerations in setting up the main and child documents:

%%%%%%%%%%%%%%%%%%%%%%%%%%%%%%%%%%%%%%%%
\paragraph{Restrictions.}

Please note the following restrictions:
\begin{itemize}
\item
|\childdocmain| must be called with one argument \textit{main}
to ensure compatibility with earlier version of the package.
It must either be empty (|\childdocmain{}|)
or precisely match the filename of the main file in which it is specified.
See \secref{sec:detection} for further information.
\item
The filename \textit{main} must be specified without the |.tex| extension.
\item
The filename \textit{main} is case sensitive
(even in case-insensitive file systems)
due to internal string comparison.
\item
The argument \textit{main} should be fully expanded, it cannot be a macro.
\item
Subdirectories and special characters should be avoided in filenames.
\item
The command |\childdocmain{|\textit{main}|}| must be followed by a whitespace.
It should not be followed immediately by another command
or by a comment mark `|%|'.
This is because the \TeX{} parser reads the token immediately following
the argument of |\childdocmain| and puts it
at the beginning of every child section;
however, a white\-space is ignored.
\end{itemize}

%%%%%%%%%%%%%%%%%%%%%%%%%%%%%%%%%%%%%%%%
\paragraph{Content of Main File.}

It is advisable to place all content in the child files included by |\include|.
Any output contained in the main file will appear in all child documents
unless suppressed manually;
it cannot be suppressed automatically by the |\includeonly| directive
and thus should normally be avoided.
A method to include some content in the main file
by means of conditional processing is described in \secref{sec:conditional}.

%%%%%%%%%%%%%%%%%%%%%%%%%%%%%%%%%%%%%%%%
\paragraph{Page Numbering.}

When only a part of the document is compiled,
the appropriate numbering of pages
(as well as other status parameters)
is determined from the |.aux| files.
The latter contain information from previous passes.
However this information needs to propagate through
all intermediate child documents.
Therefore the page numbering in child documents may well
be inconsistent until the complete document is compiled at least once.

A useful (if unconventional) way to always ensure a consistent
page numbering is to restart the numbering in each child document
and denote the pages by `\textit{child}|.|\textit{page}'
where \textit{child} represents the chapter/section number of the child file.
This can be achieved by the command
|\numberwithin{page}{|\textit{child}|}|
of the \textsf{amsmath} package
where \textit{child} can be |chapter| or |section|
depending on the chosen structuring.
Alternatively, one can modify the macro |\thepage| appropriately
and reset the counter |page| at the start of each child file.

%%%%%%%%%%%%%%%%%%%%%%%%%%%%%%%%%%%%%%%%%%%%%%%%%%%%%%%%%%%%%%%%%%%%%%%%%%%%%%%%
\subsection{Conditional Processing}
\label{sec:conditional}

The package provides a mechanism to compile different versions
of a document. To customise the versions further some conditional processing
can come in handy to distinguish which version is being compiled.
The package provides two macros to describe the compilation context:

%%%%%%%%%%%%%%%%%%%%%%%%%%%%%%%%%%%%%%%%
\DescribeMacro{\ifchilddoc}
The conditional |\ifchilddoc| distinguishes between the compilation of
child documents and the main document:
%
\begin{center}
|\ifchilddoc |\textit{child-code}| |[|\||else |\textit{main-code}]| \||fi|
\end{center}

%%%%%%%%%%%%%%%%%%%%%%%%%%%%%%%%%%%%%%%%
\DescribeMacro{\childdocname}
\DescribeMacro{\childdocjob}
The macro |\childdocname| contains the filename (without extension)
of the main or child file being processed.
Note that |\childdocjob| will always contain the name of the main file.

%%%%%%%%%%%%%%%%%%%%%%%%%%%%%%%%%%%%%%%%
\paragraph{Title Page.}

Conditional processing can be used to include a title or banner page
in the main document when proper precautions are taken.
Importantly, the code in the main file should ensure that the page counter
(as well as other status parameters which are stored in the |.aux| files)
takes the same value after the conditional processing.
Otherwise the page numbers may take divergent values
depending on which part is compiled.

For example, a title page could be declared by:
%
\begin{center}
\begin{tabular}{l}
|\ifchilddoc\||else|\\
|\addtocounter{page}{-1}|\\
\textit{code for title page}\\
|\newpage|\\
|\||fi|
\end{tabular}
\end{center}
%
A banner page for the child documents can be generated by:
%
\begin{center}
\begin{tabular}{l}
|\ifchilddoc|\\
|\addtocounter{page}{-1}|\\
\textit{code for banner page}\\
|\newpage|\\
|\||fi|
\end{tabular}
\end{center}
%
Here one could write a message such as:
\begin{center}
|This is the part \childdocname{} of \childdocjob{}.|
\end{center}

%%%%%%%%%%%%%%%%%%%%%%%%%%%%%%%%%%%%%%%%%%%%%%%%%%%%%%%%%%%%%%%%%%%%%%%%%%%%%%%%
\subsection{Flags}
\label{sec:flags}

The package makes it easy to generate different versions
of the main or child documents.
To this end compilation flags can be defined
and assigned different default values.
They will be particularly useful in conjunction
with the forwarding mechanism described in \secref{sec:forward}.

For example, it may be useful to have a flag |\version|
which can be set to |draft| or |final|.
The document source will contain some conditional code
depending on the value of |\version|.
Suppose further, the flag should default to |final| for the main file
and to |draft| for child files
which is a natural assignment for editing the document.
This is achieved by placing the following code
in the preamble of the main document
(below the |\childdocmain| directive):
%
\begin{center}
\begin{tabular}{l}
|\ifchilddoc|\\
|\providecommand{\version}{draft}|\\
|\||else|\\
|\providecommand{\version}{final}|\\
|\||fi|
\end{tabular}
\end{center}
%
The definition by |\providecommand| makes sure
that previous definitions are not overwritten.
Further statements |\providecommand{\version}{...}|
can thus be added before the above code to override it.

For the main file, one might add a line
(between |\childdocmain| and the above block)
%
\begin{center}
|%\ifchilddoc\||else\providecommand{\version}{draft}\||fi|
\end{center}
%
which can be uncommented to produce a draft version.
Likewise one can add a line to the very top of a child file
(above the |\childdocof{|\textit{main}|}| directive)
%
\begin{center}
|%\providecommand{\version}{final}|
\end{center}
%
which can be uncommented to produce the final version of this child document.

%%%%%%%%%%%%%%%%%%%%%%%%%%%%%%%%%%%%%%%%%%%%%%%%%%%%%%%%%%%%%%%%%%%%%%%%%%%%%%%%
\subsection{Forwarding}
\label{sec:forward}

Different versions of the main or child documents
using compilation flags as described in \secref{sec:flags}
can be (permanently) stored in different files
for convenient compilation, viewing and distribution.
To this end, the package defines a command
to pass on compilation to a different file:

%%%%%%%%%%%%%%%%%%%%%%%%%%%%%%%%%%%%%%%%
\DescribeMacro{\childdocforward}
The command |\childdocforward| redirects processing to
another source file:
%
\begin{center}
\begin{tabular}{l}
|\input{childdoc.def}|\\
|\childdocforward[|\textit{main}|]{|\textit{dest}|}|\\
\end{tabular}
\end{center}
%
The argument \textit{dest} is the destination file
(without extension).
It should be the main file or one of the child files.
Note that further \textsf{childdoc} directives
such as |\childdocof| and |\childdocforward|
in the indicated file will be processed in this form.
The optional argument \textit{main}
passes on directly to the main file \textit{main}
while pretending to compile the child \textit{dest}.
This form behaves as if \textit{dest}
issues |\childdocof{|\textit{main}|}| right away,
and no further \textsf{childdoc} directives will be processed.

%%%%%%%%%%%%%%%%%%%%%%%%%%%%%%%%%%%%%%%%
\DescribeMacro{\...prefix}
In the alternative form |\childdocforwardprefix|,
%
\begin{center}
\begin{tabular}{l}
|\input{childdoc.def}|\\
|\childdocforwardprefix[|\textit{main}|]{|\textit{prefix}|}{|\textit{dest}|}|
\end{tabular}
\end{center}
%
the destination file is determined by a pattern
depending on the current file:
To make this work, the current file must be called
`{\textit{prefix}\hspace{0.2em}\textit{suffix}}'
with \textit{prefix} matching precisely the argument.
Processing is then passed on to the file
`{\textit{dest}\hspace{0.2em}\textit{suffix}}'.
Surely, the same effect is achieved by
directly specifying the
argument `{\textit{dest}\hspace{0.2em}\textit{suffix}}'
in the first form.
However, that requires to set up a different file
for each child. With the alternative form of the command
all these files can have exactly the same content
which simplifies setting them up and maintaining them.

For example, the following file |draft.tex|
with a compilation flag |\version| as described in \secref{sec:flags}
compiles the main document as a draft:
%
\begin{center}
\begin{tabular}{l}
|\def\version{draft}|\\
|\input{childdoc.def}|\\
|\childdocforward{|\textit{main}|}|
\end{tabular}
\end{center}
%
Likewise, the following files |final|\textit{nn}|.tex|
compile the final version of the child document
|child|\textit{nn}|.tex|:
%
\begin{center}
\begin{tabular}{l}
|\def\version{final}|\\
|\input{childdoc.def}|\\
|\childdocforwardprefix{final}{child}|
\end{tabular}
\end{center}
%

Note that when several versions of a main file and/or of each child file
are to be generated, it may be convenient to set up a |Makefile| or
shell script to automatise the process.

%%%%%%%%%%%%%%%%%%%%%%%%%%%%%%%%%%%%%%%%%%%%%%%%%%%%%%%%%%%%%%%%%%%%%%%%%%%%%%%%
\subsection{Command Line Processing}
\label{sec:commandline}

The effect of redirection files can also be achieved by invoking
the \LaTeX{} compiler with a more elaborate command line.
Most conveniently this should be done as part
of a shell script or a |Makefile|.

When using \textsf{childdoc} in the main file, the following
command lines effectively perform a redirection
(note that depending on the shell being used,
backslashes may have to be doubled: `|\|' $\to$ `|\\|'):
%
\begin{center}
|... -jobname "|\textit{target}|" |\\|"|[\textit{flags}]%
|\input{childdoc.def}\childdocforward[|\textit{main}|]{|\textit{dest}|}"|
\end{center}
%
Here \textit{target} is the name of the output file,
\textit{main} is the name of the main file
and \textit{dest} is the name of the main or child file to be processed
(all filenames without extensions).
The optional argument \textit{main} can be omitted
if \textit{main} matches \textit{dest}.
Optionally, compilation \textit{flags} can be defined via |\def| commands.
This command line makes the \TeX{} engine believe
it is compiling the file \textit{target}
whose content is specified as the latter parameter.
The provided code then forwards the processing to
\textit{main} or \textit{dest} as described in \secref{sec:forward}.

%%%%%%%%%%%%%%%%%%%%%%%%%%%%%%%%%%%%%%%%%%%%%%%%%%%%%%%%%%%%%%%%%%%%%%%%%%%%%%%%
\subsection{Include by Input}
\label{sec:input}

Including child documents by |\include| has some restrictions by design.
Most notably, the content of a child document always occupies
its own set of pages; pages cannot be shared between child documents.
Usually, this behaviour makes perfect sense
because each child document contain an essential part of the document.
However, in some situations it may be desirable to compose
a document from a collection of parts
without having mandatory page breaks between then.
For this case, the package
provides a mechanism to include parts
by |\input| which can also be processed individually.
However, by construction this mechanism
requires manual handling of the content to be output.

%%%%%%%%%%%%%%%%%%%%%%%%%%%%%%%%%%%%%%%%
\DescribeMacro{\ifchilddocmanual}
The main file should be prepared as usual, see \secref{sec:include}.
However, the document body must make a distinction
between processing of an individual part and of the main document, e.g.:
%
\begin{center}
\begin{tabular}{l}
|\ifchilddocmanual|\\
|\input{\childdocname}|\\
|\||else|\\
\textit{document body with }|\input{|\textit{part}|}|\\
|\||fi|
\end{tabular}
\end{center}
%
The conditional |\ifchilddocmanual| is true whenever
a part to be included by |\input| is being compiled,
and the name of the part is stored in |\childdocname|.

%%%%%%%%%%%%%%%%%%%%%%%%%%%%%%%%%%%%%%%%
\DescribeMacro{\childdocby}
Each part to be included by |\input| should start with:
%
\begin{center}
\begin{tabular}{l}
|\input{childdoc.def}|\\
|\childdocby{|\textit{main}|}|\\
\end{tabular}
\end{center}
%
The directive |\childdocby| is similar to |\childdocof|
described in \secref{sec:include},
but the subsequent selection of content must be done manually.
To that end, both |\ifchilddoc| and |\ifchilddocmanual|
will be true upon processing of a part,
and the name of the part is stored in |\childdocname|.
Note that |\jobname| will be set to the filename of the current part
so that each part receives an individual |.aux| file
that does not interfere with the |.aux| file(s) of the main document.
This behaviour can be altered by the alternative form
|\childdocby[*]{|\textit{main}|}| (with a non-empty optional argument)
which uses the |.aux| file of the main document
by setting |\jobname| to \textit{main}.

%%%%%%%%%%%%%%%%%%%%%%%%%%%%%%%%%%%%%%%%%%%%%%%%%%%%%%%%%%%%%%%%%%%%%%%%%%%%%%%%
\subsection{Driver Development}
\label{sec:driver}

The \textsf{childdoc} mechanism can also be use for the development
of definition files such as \LaTeX{} styles or classes.
This case differs from the above setup with multiple parts
included by |\include| in that no |\includeonly| should be invoked.
This can be achieved by starting the include file
(before |\ProvidesPackage|) with:
%
\begin{center}
\begin{tabular}{l}
|\input{childdoc.def}|\\
|\childdocforward{|\textit{main}|}|\\
\end{tabular}
\end{center}
%
or alternatively with:
%
\begin{center}
\begin{tabular}{l}
|\input{childdoc.def}|\\
|\childdocby{|\textit{main}|}|\\
\end{tabular}
\end{center}
%
Both forms have slightly different effects as described above.
The main file is prepared as usual, see \secref{sec:include}.

%%%%%%%%%%%%%%%%%%%%%%%%%%%%%%%%%%%%%%%%%%%%%%%%%%%%%%%%%%%%%%%%%%%%%%%%%%%%%%%%
\subsection{Legacy Detection}
\label{sec:detection}

The directive |\childdocmain| in the main file can detect
whether the complete document or merely a child is to be compiled
even without using the directive |\childdocof|.
This method is deprecated because it is less robust
and there is no compelling reason to use it;
it is merely provided for backward compatibility
and it may be removed in future versions.

If the detection mechanism is to be used,
it is mandatory to correctly specify
the filename of the main file as the argument of |\childdocmain|:
%
\begin{center}
\begin{tabular}{l}
|\input{childdoc.def}|\\
|\childdocmain{|\textit{main}|}|\\
\end{tabular}
\end{center}
%
If |\jobname| does not match the argument \textit{main} of |\childdocmain|,
it is assumed that |\jobname| points to the child file to be compiled.
When using |\childdocmain| with the main file specified as argument,
it suffices to start a child file
with just |\input{|\textit{main}|}|
without loading of the package and using |\childdocof|.
If instead all processing is done
with the appropriate \textsf{childdoc} directives,
the argument of \textit{main} of |\childdocmain| can be empty.

An alternative version of the command line processing described
in \secref{sec:commandline} using the detection mechanism reads:
%
\begin{center}
|... -jobname "|\textit{target}|" "|[\textit{flags}]%
[|\def\jobname{|\textit{dest}|}|]|\input{|\textit{main}|}"|
\end{center}

%%%%%%%%%%%%%%%%%%%%%%%%%%%%%%%%%%%%%%%%%%%%%%%%%%%%%%%%%%%%%%%%%%%%%%%%%%%%%%%%
\subsection{Manual Code}
\label{sec:manual}

In case one cannot be certain whether the definitions file |childdoc.def|
is installed on the target \TeX{} distribution
and one prefers not to ship it,
it is conceivable to paste a few relevant commands into the sources.

To that end, drop all statements |\input{childdoc.def}|
and perform the replacements as outlined below.
Instead of |\childdocmain{|\textit{main}|}| add the following code
to the top of the main file:
%
\begin{center}
\begin{tabular}{l}
|\||ifdefined\childdocname\endinput\||fi\newif\ifchilddoc|\\
|\edef\childdocname{\scantokens\expandafter{\jobname\noexpand}}|\\
|\def\childdocmain{|\textit{main}|}\||ifx\childdocmain\childdocname\||else|\\
|\childdoctrue\includeonly{\childdocname}\let\jobname\childdocmain\||fi|\\
\end{tabular}
\end{center}
%
Instead of |\childdocof{|\textit{main}|}| just include the main file
at the top of each child file:
%
\begin{center}
|\input{|\textit{main}|}|
\end{center}
%
A simple redirection |\childdocforward{|\textit{dest}|}| is achieved by:
%
\begin{center}
|\def\jobname{|\textit{dest}|}\input{\jobname}|
\end{center}
%
The redirection with prefix
|\childdocforwardprefix[|\textit{prefix}|]{|\textit{dest}|}|
is accomplished by:
%
\begin{center}
\begin{tabular}{l}
|{\edef\jobname{\scantokens\expandafter{\jobname\noexpand}}|\\
|\def\redirectjob |\textit{prefix}|#1~~~{\gdef\jobname{|\textit{dest}|#1}}|\\
|\expandafter\redirectjob\jobname~~~}\input{\jobname}|
\end{tabular}
\end{center}

In an alternative approach,
child documents can be compiled by a specific command line
without additional code or specific definitions:
%
\begin{center}
|... -jobname "|\textit{target}|" "|[\textit{flags}]%
|\includeonly{|\textit{dest}|}\input{|\textit{main}|}"|
\end{center}
%

%%%%%%%%%%%%%%%%%%%%%%%%%%%%%%%%%%%%%%%%%%%%%%%%%%%%%%%%%%%%%%%%%%%%%%%%%%%%%%%%
%%%%%%%%%%%%%%%%%%%%%%%%%%%%%%%%%%%%%%%%%%%%%%%%%%%%%%%%%%%%%%%%%%%%%%%%%%%%%%%%
\section{Information}

%%%%%%%%%%%%%%%%%%%%%%%%%%%%%%%%%%%%%%%%%%%%%%%%%%%%%%%%%%%%%%%%%%%%%%%%%%%%%%%%
\subsection{Copyright}

Copyright \copyright{} 2017--2018 Niklas Beisert

This work may be distributed and/or modified under the
conditions of the \LaTeX{} Project Public License, either version 1.3
of this license or (at your option) any later version.
The latest version of this license is in
  \url{http://www.latex-project.org/lppl.txt}
and version 1.3 or later is part of all distributions of \LaTeX{}
version 2005/12/01 or later.

This work has the LPPL maintenance status `maintained'.

The Current Maintainer of this work is Niklas Beisert.

This work consists of the files |README.txt|, |childdoc.ins| and |childdoc.dtx|
as well as the derived files |childdoc.def|, |cdocsamp.tex|
with |cdocsch1.tex|, |cdocsch2.tex|, |cdocspt3.tex|, |cdocspt4.tex|,
|cdocsdrf.tex|, |cdocsfn1.tex|, |cdocsfn2.tex|
as well as |childdoc.pdf|.

%%%%%%%%%%%%%%%%%%%%%%%%%%%%%%%%%%%%%%%%%%%%%%%%%%%%%%%%%%%%%%%%%%%%%%%%%%%%%%%%
\subsection{Files and Installation}

The package consists of the files:
%
\begin{center}
\begin{tabular}{ll}
    |README.txt|   & readme file \\
    |childdoc.ins| & installation file \\
    |childdoc.dtx| & source file \\
    |childdoc.def| & definition file \\
    |cdocsamp.tex| & sample main file \\
    |cdocsch1.tex| & sample include file \\
    |cdocsch2.tex| & sample include file \\
    |cdocspt3.tex| & sample part file \\
    |cdocspt4.tex| & sample part file \\
    |cdocsdrf.tex| & sample redirection file \\
    |cdocsfn1.tex| & sample redirection file \\
    |cdocsfn2.tex| & sample redirection file \\
    |childdoc.pdf| & manual
\end{tabular}
\end{center}
%
The distribution consists of the files
|README.txt|, |childdoc.ins| and |childdoc.dtx|.
%
\begin{itemize}
\item
Run (pdf)\LaTeX{} on |childdoc.dtx|
to compile the manual |childdoc.pdf| (this file).
\item
Run \LaTeX{} on |childdoc.ins| to create the definitions file |childdoc.def|
and the sample |cdocsamp.tex| with include files
|cdocsch1.tex|, |cdocsch2.tex|, |cdocspt3.tex|, |cdocspt4.tex|,
|cdocsdrf.tex|, |cdocsfn1.tex|, |cdocsfn2.tex|.
Then copy the file |childdoc.def| to an appropriate directory of your \LaTeX{}
distribution, e.g.\ \textit{texmf-root}|/tex/latex/childdoc|.
\end{itemize}

%%%%%%%%%%%%%%%%%%%%%%%%%%%%%%%%%%%%%%%%%%%%%%%%%%%%%%%%%%%%%%%%%%%%%%%%%%%%%%%%
\subsection{Related CTAN Packages}

There are several other packages which offer a similar functionality:
%
\begin{itemize}
\item
The packages
\href{http://ctan.org/pkg/docmute}{\textsf{docmute}},
\href{http://ctan.org/pkg/includex}{\textsf{includex}} and
\href{http://ctan.org/pkg/standalone}{\textsf{standalone}}
provide commands to include only the document body of
a child file thus allowing both files to be compiled individually.
\item
The packages \href{http://ctan.org/pkg/subdocs}{\textsf{subdocs}}
and \href{http://ctan.org/pkg/subfiles}{\textsf{subfiles}}
provide structures in which the main and child documents can be
encapsulated and allowing them to be compiled individually.
The inclusion mechanism is different from the conventional |\include|.
\item
The package \href{http://ctan.org/pkg/combine}{\textsf{combine}}
is an elaborate solution to combine several documents into one.
\end{itemize}
%
See also the CTAN topic \href{http://ctan.org/topic/subdocs}{\textsf{subdocs}}
for further related packages.
The present package differs from the above solutions in that
a document structure constructed with the conventional |\include| mechanism
just needs two extra commands at the top of every file
such that all constituent files can be compiled individually.

%%%%%%%%%%%%%%%%%%%%%%%%%%%%%%%%%%%%%%%%%%%%%%%%%%%%%%%%%%%%%%%%%%%%%%%%%%%%%%%%
%\subsection{Feature Suggestions}
%
%The following is a list of features which may be useful for future
%versions of this package:
%%
%\begin{itemize}
%\item
%\ldots
%\end{itemize}

%%%%%%%%%%%%%%%%%%%%%%%%%%%%%%%%%%%%%%%%%%%%%%%%%%%%%%%%%%%%%%%%%%%%%%%%%%%%%%%%
\subsection{Revision History}

%%%%%%%%%%%%%%%%%%%%%%%%%%%%%%%%%%%%%%%%
\paragraph{v2.0:} 2018/12/30

\begin{itemize}
\item
immediate forward processing
\item
added |\childdocby| mechanism
\item
manual restructured
\end{itemize}

%%%%%%%%%%%%%%%%%%%%%%%%%%%%%%%%%%%%%%%%
\paragraph{v1.6:} 2018/01/17

\begin{itemize}
\item
application for development of include files
\item
corrections to manual
\end{itemize}

%%%%%%%%%%%%%%%%%%%%%%%%%%%%%%%%%%%%%%%%
\paragraph{v1.5:} 2017/05/21

\begin{itemize}
\item
more complete structuring introduced
\item
|\childdocof| introduced
\item
|\childdoc| renamed to |\childdocmain|
\item
|\childredirect| renamed to |\childdocforward| and |\childdocforwardprefix|
and functionality expanded
\end{itemize}

%%%%%%%%%%%%%%%%%%%%%%%%%%%%%%%%%%%%%%%%
\paragraph{v1.0:} 2017/04/27

\begin{itemize}
\item
manual and install package
\item
first version published on CTAN
\end{itemize}

%%%%%%%%%%%%%%%%%%%%%%%%%%%%%%%%%%%%%%%%
\paragraph{v0.6:} 2017/04/26

\begin{itemize}
\item
redirection mechanism added
\end{itemize}

%%%%%%%%%%%%%%%%%%%%%%%%%%%%%%%%%%%%%%%%
\paragraph{v0.5:} 2017/04/26

\begin{itemize}
\item
functionality in definition file
\end{itemize}


%%%%%%%%%%%%%%%%%%%%%%%%%%%%%%%%%%%%%%%%%%%%%%%%%%%%%%%%%%%%%%%%%%%%%%%%%%%%%%%%
%%%%%%%%%%%%%%%%%%%%%%%%%%%%%%%%%%%%%%%%%%%%%%%%%%%%%%%%%%%%%%%%%%%%%%%%%%%%%%%%
%%%%%%%%%%%%%%%%%%%%%%%%%%%%%%%%%%%%%%%%%%%%%%%%%%%%%%%%%%%%%%%%%%%%%%%%%%%%%%%%
\appendix

\settowidth\MacroIndent{\rmfamily\scriptsize 000\ }

 \DocInput{childdoc.dtx}

\end{document}
%</driver>
% \fi
%
% %%%%%%%%%%%%%%%%%%%%%%%%%%%%%%%%%%%%%%%%%%%%%%%%%%%%%%%%%%%%%%%%%%%%%%%%%%%%%%
% %%%%%%%%%%%%%%%%%%%%%%%%%%%%%%%%%%%%%%%%%%%%%%%%%%%%%%%%%%%%%%%%%%%%%%%%%%%%%%
% \section{Sample}
%\iffalse
%<*samplemain>
%\fi
%
% The following presents a sample document
% with two chapters, two parts, a title page,
% a compile flag as well as three forwarding files to set the flag.
% It consists of eight |.tex| files:
% \begin{center}
% \begin{tabular}{ll}
% |cdocsamp.tex|&main file\\
% |cdocsch1.tex|&include file for chapter 1\\
% |cdocsch2.tex|&include file for chapter 2\\
% |cdocspt3.tex|&include file for part 3\\
% |cdocspt4.tex|&include file for part 4\\
% |cdocsdrf.tex|&forwarding file for main file in draft mode\\
% |cdocsfi1.tex|&forwarding file for final version of chapter 1\\
% |cdocsfi2.tex|&forwarding file for final version of chapter 2\\
% \end{tabular}
% \end{center}
% Each of the eight files can be compiled directly by the \LaTeX{} compiler.
%
% %%%%%%%%%%%%%%%%%%%%%%%%%%%%%%%%%%%%%%
% \paragraph{Main File.}
%
% The main file is called |cdocsamp.tex|.
%
% Load the \textsf{childdoc} definitions and
% declare the filename for the main document:
%    \begin{macrocode}
\input{childdoc.def}
\childdocmain{}
%    \end{macrocode}

% Optional override for |\version| flag:
%    \begin{macrocode}
%%\ifchilddoc\else\providecommand{\version}{draft}\fi
%    \end{macrocode}

% Define the default values for the |\version| flag
% (|final| for the main file and |draft| for childs):
%    \begin{macrocode}
\ifchilddoc
\providecommand{\version}{draft}
\else
\providecommand{\version}{final}
\fi
%    \end{macrocode}

% Load the standard document class:
%    \begin{macrocode}
\documentclass[12pt]{article}
%    \end{macrocode}

% Start the document body:
%    \begin{macrocode}
\begin{document}
%    \end{macrocode}

% Declare a title page.
% Print title, part of document being processed and version flag:
%    \begin{macrocode}
\addtocounter{page}{-1}
\begin{center}
{\LARGE\bfseries{}childdoc example\par}
\vspace{1cm}
\ifchilddoc
\ifchilddocmanual part\else chapter\fi:
`\childdocname' of `\childdocjob'\par
\else
main document: `\childdocjob'\par
\fi
version: \version\par
\end{center}
\newpage
%    \end{macrocode}

% Manually include selected file,
% otherwise process as usual:
%    \begin{macrocode}
\ifchilddocmanual
\section*{part `\childdocname'}
\input{\childdocname}
\else
%    \end{macrocode}

% Include the two chapters:
%    \begin{macrocode}
\include{cdocsch1}
\include{cdocsch2}
%    \end{macrocode}

% Include the two parts unless only chapters should be displayed:
%    \begin{macrocode}
\ifchilddoc\else
\section{part three}
\input{cdocspt3}
\section{part four}
\input{cdocspt4}
\fi
%    \end{macrocode}

% Process as usual until here:
%    \begin{macrocode}
\fi
%    \end{macrocode}

% End of document body:
%    \begin{macrocode}
\end{document}
%    \end{macrocode}
%\iffalse
%</samplemain>
%\fi
%
% %%%%%%%%%%%%%%%%%%%%%%%%%%%%%%%%%%%%%%
% \paragraph{Chapter Include Files.}
%
% The include files are called |cdocsch1.tex| and |cdocsch2.tex|.
%
%\iffalse
%<*samplechap1|samplechap2>
%\fi

% Optional override for |\version| flag:
%    \begin{macrocode}
%%\providecommand{\version}{final}
%    \end{macrocode}

% Include the main document:
%    \begin{macrocode}
\input{childdoc.def}
\childdocof{cdocsamp}
%    \end{macrocode}

%\iffalse
%</samplechap1|samplechap2>
%\fi
%
%\iffalse
%<*samplechap1>
%\fi
% Some text for chapter 1:
%    \begin{macrocode}
\section{one}
some text in chapter one
%    \end{macrocode}

%\iffalse
%</samplechap1>
%\fi
% Some text for chapter 2:
%\iffalse
%<*samplechap2>
%\fi
%    \begin{macrocode}
\section{two}
more text in chapter two
%    \end{macrocode}

%\iffalse
%</samplechap2>
%\fi
%
% %%%%%%%%%%%%%%%%%%%%%%%%%%%%%%%%%%%%%%
% \paragraph{Part Include Files.}
%
% The include files are called |cdocspt3.tex| and |cdocspt4.tex|.
%
%\iffalse
%<*samplepart3|samplepart4>
%\fi

% Optional override for |\version| flag:
%    \begin{macrocode}
%%\providecommand{\version}{final}
%    \end{macrocode}

% Include the main document:
%    \begin{macrocode}
\input{childdoc.def}
\childdocby{cdocsamp}
%    \end{macrocode}

%\iffalse
%</samplepart3|samplepart4>
%\fi
%
%\iffalse
%<*samplepart3>
%\fi
% Some text for part 3:
%    \begin{macrocode}
some text in part three
%    \end{macrocode}

%\iffalse
%</samplepart3>
%\fi
% Some text for part 4:
%\iffalse
%<*samplepart4>
%\fi
%    \begin{macrocode}
more text in part four
%    \end{macrocode}

%\iffalse
%</samplepart4>
%\fi
%
% %%%%%%%%%%%%%%%%%%%%%%%%%%%%%%%%%%%%%%
% \paragraph{Forwarding for a Complete Draft.}
%
% The following forwarding file |cdocsdrf.tex|
% compiles the main document in draft mode:
%\iffalse
%<*sampledraft>
%\fi
%    \begin{macrocode}
\def\version{draft}
\input{childdoc.def}
\childdocforward{cdocsamp}
%    \end{macrocode}

%\iffalse
%</sampledraft>
%\fi
%
% %%%%%%%%%%%%%%%%%%%%%%%%%%%%%%%%%%%%%%
% \paragraph{Forwarding for Final Version of the Chapters.}
%
% The following forwarding files |cdocsfn1.tex| and |cdocsfn2.tex|
% (with identical content)
% compile the final versions of the child documents
% |cdocsch1.tex| and |cdocsch2.tex|, respectively:
%\iffalse
%<*samplefinal>
%\fi
%    \begin{macrocode}
\def\version{final}
\input{childdoc.def}
\childdocforwardprefix[cdocsamp]{cdocsfn}{cdocsch}
%    \end{macrocode}

%\iffalse
%</samplefinal>
%\fi
%
% %%%%%%%%%%%%%%%%%%%%%%%%%%%%%%%%%%%%%%
% \paragraph{Command Line Processing.}
%
% The following three command lines generate the output files
% |cdocscld|, |cdocscl1| and |cdocscl2|
% which should be identical to
% |cdocsdrf|, |cdocsch1| and |cdocsfn2|, respectively:
% \begin{center}
% \begin{tabular}{l}
% |latex -jobname cdocscld \|\\
% |  "\def\version{draft}\input{childdoc.def}\childdocforward{cdocsamp}"|\\
% |latex -jobname cdocscl1 \|\\
% |  "\input{childdoc.def}\childdocforward[cdocsamp]{cdocsch1}"|\\
% |latex -jobname cdocscl2 \|\\
% |  "\def\version{final}\input{childdoc.def}\childdocforward{cdocsch2}"|
% \end{tabular}
% \end{center}
% Note that the trailing backslash on each first line
% merely continues the input to the second line
% (for convenient cut ant paste).
% Furthermore, the command |latex| can be replaced by any
% of its alternative versions such as |pdflatex|.
%
% %%%%%%%%%%%%%%%%%%%%%%%%%%%%%%%%%%%%%%%%%%%%%%%%%%%%%%%%%%%%%%%%%%%%%%%%%%%%%%
% %%%%%%%%%%%%%%%%%%%%%%%%%%%%%%%%%%%%%%%%%%%%%%%%%%%%%%%%%%%%%%%%%%%%%%%%%%%%%%
% \section{Implementation}
%\iffalse
%<*package>
%\fi
%
% This section describes the definitions file |childdoc.def|.

% The definitions cannot be loaded using |\usepackage| or |\RequirePackage|
% which has a mechanism to prevent loading a style file more than once.
% When loading the definitions by means of |\input|
% multiple instances have to be prevented manually:
%\iffalse
%This code needs to be before the `\ProvidesFile' directive
%which is defined at the beginning of this file.
%Therefore it is also placed there and commented out here.
%</package>
%<*discard>
%\fi
%    \begin{macrocode}
\ifdefined\childdocmain\endinput\fi
%    \end{macrocode}
%\iffalse
%</discard>
%<*package>
%\fi
%
% \macro{\ifchilddoc}
% \macro{\ifchilddocmanual}
% The conditional |\ifchilddoc| tells whether a
% child (true) or main (false) document is being compiled.
% The conditional |\ifchilddocmanual| tells whether
% the |\includeonly| mechanism is used (false) or
% the selection of child files must be performed manually (true).
% The definitions initialise to false:
%    \begin{macrocode}
\newif\ifchilddoc
\newif\ifchilddocmanual
%    \end{macrocode}

% \macro{\childdocname}
% \macro{\childdocjob}
% The macro |\childdocname| stores the name of the main document
% to be compiled. The macro |\childdocjob| stores the name of
% the document on which the \LaTeX{} compiler was originally invoked.
% The content of |\jobname| cannot be compared
% to filenames specified in the source due to different catcodes.
% The following code rescans |\jobname|, stores the result
% in |\childdocname| and saves a copy in |\childdocjob|:
%    \begin{macrocode}
\edef\childdocname{\scantokens\expandafter{\jobname\noexpand}}
\let\childdocjob\childdocname
%    \end{macrocode}

% \macro{\childdocdisable}
% The macro |\childdocdisable| prevents the main file
% from being processed more than once.
% At this stage, the main document command |\childdocmain|
% is assumed to be called once again where it should do nothing.
% Any subsequent call to it should prevent
% a secondary processing of the main document
% It overwrites the forwarding commands
% |\childdocof| and |\childdocforward|
% with empty macros to prevent further inclusions of the main document:
%    \begin{macrocode}
\newcommand{\childdocdisable}
{
  \renewcommand{\childdocmain}[1]{\renewcommand{\childdocmain}[1]{\endinput}}
  \renewcommand{\childdocof}[1]{}
  \renewcommand{\childdocby}[2][]{}
  \renewcommand{\childdocforward}[2][]{}
  \renewcommand{\childdocdisable}{}
}
%    \end{macrocode}

% \macro{\childdocmain}
% The macro |\childdocmain| is to be called at the top of the main file
% with nothing or the main filename (without extension) as argument.
% First, it breaks loops.
% If the argument is not empty and does not match |\childdocname|
% (which is set by the first inclusion of |childdoc.def|),
% |\ifchilddoc| is set to true, |\includeonly| is applied to the child file
% and |\jobname| is set to the main file
% (for proper handling of |.aux| files):
%    \begin{macrocode}
\newcommand{\childdocmain}[1]
{
  \childdocdisable\childdocmain{}
  \if?#1?\else
    \begingroup
      \def\childdoctmp{#1}
      \ifx\childdoctmp\childdocname
        \def\childdoctmp{}
      \else
        \def\childdoctmp
        {
          \childdoctrue
          \includeonly{\childdocname}
          \def\childdocjob{#1}
          \def\jobname{#1}
        }
      \fi
      \expandafter
    \endgroup
    \childdoctmp
  \fi
}
%    \end{macrocode}

% \macro{\childdocof}
% The command |\childdocof| redirects
% compilation to the main file |#1|.
%    \begin{macrocode}
\newcommand{\childdocof}[1]
{
  \childdocdisable
  \childdoctrue
  \includeonly{\childdocname}
  \def\jobname{#1}
  \def\childdocjob{#1}
  \input{#1}
}
%    \end{macrocode}

% \macro{\childdocby}
% The command |\childdocby| ....
%    \begin{macrocode}
\newcommand{\childdocby}[2][]
{
  \childdocdisable
  \childdoctrue
  \childdocmanualtrue
  \if?#1?\else
    \def\jobname{#2}
  \fi
  \def\childdocjob{#2}
  \input{#2}
  \endinput
}
%    \end{macrocode}

% \macro{\childdocforward}
% The command |\childdocforward| redirects
% compilation to the main file or
% (if the optional argument is given) a child file.
% Parameters are set as if the main file
% or a child file starting with |\childdocof| was compiled.
% Then compilation is handed over to the main file:
%    \begin{macrocode}
\newcommand{\childdocforward}[2][]
{
  \begingroup
    \if?#1?
      \def\childdoctmp
      {
        \def\childdocname{#2}
        \def\childdocjob{#2}
        \def\jobname{#2}
        \input{#2}
        \endinput
      }
    \else
      \def\childdoctmp
      {
        \childdocdisable
        \def\childdocname{#2}
        \childdoctrue
        \includeonly{#2}
        \def\childdocjob{#1}
        \def\jobname{#1}
        \input{#1}
        \endinput
      }
    \fi
    \expandafter
  \endgroup
  \childdoctmp
}
%    \end{macrocode}

% \macro{\childdocforwardprefix}
% The command |\childdocforwardprefix| redirects
% compilation to the main or a child file by means of a pattern.
% The prefix |#1| in the current filename is replaced by |#2|
% and the suffix of the current filename is kept
% (it is assumed that the filename does not contain the substring `|~~~|'
% which is used as a delimiter).
% Compilation is handed over to the new file by |\childdocforward|:
%    \begin{macrocode}
\newcommand{\childdocforwardprefix}[3][]
{
  \begingroup
    \def\childdocextract #2##1~~~{\def\childdoctmp{\childdocforward[#1]{#3##1}}}
    \expandafter\childdocextract\childdocname~~~
    \expandafter
  \endgroup
  \childdoctmp
}
%    \end{macrocode}

% \macro{\childdoc}
% The deprecated macro |\childdoc| is a legacy version of |\childdocmain|:
%    \begin{macrocode}
\newcommand{\childdoc}{\childdocmain}
%    \end{macrocode}

% \macro{\childdocredirect}
% The deprecated macro |\childdocredirect| is a legacy version
% of |\childdocforward| and |\childdocforwardprefix|:
%    \begin{macrocode}
\newcommand{\childdocredirect}[2][]
{
  \begingroup
    \if?#1?
      \def\childdoctmp{\childdocforward{#2}}
    \else
      \def\childdoctmp{\childdocforwardprefix{#1}{#2}}
    \fi
    \expandafter
  \endgroup
  \childdoctmp
}
%    \end{macrocode}

%\iffalse
%</package>
%\fi
%
\endinput
\childdocforward{cdocsch2}"|
% \end{tabular}
% \end{center}
% Note that the trailing backslash on each first line
% merely continues the input to the second line
% (for convenient cut ant paste).
% Furthermore, the command |latex| can be replaced by any
% of its alternative versions such as |pdflatex|.
%
% %%%%%%%%%%%%%%%%%%%%%%%%%%%%%%%%%%%%%%%%%%%%%%%%%%%%%%%%%%%%%%%%%%%%%%%%%%%%%%
% %%%%%%%%%%%%%%%%%%%%%%%%%%%%%%%%%%%%%%%%%%%%%%%%%%%%%%%%%%%%%%%%%%%%%%%%%%%%%%
% \section{Implementation}
%\iffalse
%<*package>
%\fi
%
% This section describes the definitions file |childdoc.def|.

% The definitions cannot be loaded using |\usepackage| or |\RequirePackage|
% which has a mechanism to prevent loading a style file more than once.
% When loading the definitions by means of |\input|
% multiple instances have to be prevented manually:
%\iffalse
%This code needs to be before the `\ProvidesFile' directive
%which is defined at the beginning of this file.
%Therefore it is also placed there and commented out here.
%</package>
%<*discard>
%\fi
%    \begin{macrocode}
\ifdefined\childdocmain\endinput\fi
%    \end{macrocode}
%\iffalse
%</discard>
%<*package>
%\fi
%
% \macro{\ifchilddoc}
% \macro{\ifchilddocmanual}
% The conditional |\ifchilddoc| tells whether a
% child (true) or main (false) document is being compiled.
% The conditional |\ifchilddocmanual| tells whether
% the |\includeonly| mechanism is used (false) or
% the selection of child files must be performed manually (true).
% The definitions initialise to false:
%    \begin{macrocode}
\newif\ifchilddoc
\newif\ifchilddocmanual
%    \end{macrocode}

% \macro{\childdocname}
% \macro{\childdocjob}
% The macro |\childdocname| stores the name of the main document
% to be compiled. The macro |\childdocjob| stores the name of
% the document on which the \LaTeX{} compiler was originally invoked.
% The content of |\jobname| cannot be compared
% to filenames specified in the source due to different catcodes.
% The following code rescans |\jobname|, stores the result
% in |\childdocname| and saves a copy in |\childdocjob|:
%    \begin{macrocode}
\edef\childdocname{\scantokens\expandafter{\jobname\noexpand}}
\let\childdocjob\childdocname
%    \end{macrocode}

% \macro{\childdocdisable}
% The macro |\childdocdisable| prevents the main file
% from being processed more than once.
% At this stage, the main document command |\childdocmain|
% is assumed to be called once again where it should do nothing.
% Any subsequent call to it should prevent
% a secondary processing of the main document
% It overwrites the forwarding commands
% |\childdocof| and |\childdocforward|
% with empty macros to prevent further inclusions of the main document:
%    \begin{macrocode}
\newcommand{\childdocdisable}
{
  \renewcommand{\childdocmain}[1]{\renewcommand{\childdocmain}[1]{\endinput}}
  \renewcommand{\childdocof}[1]{}
  \renewcommand{\childdocby}[2][]{}
  \renewcommand{\childdocforward}[2][]{}
  \renewcommand{\childdocdisable}{}
}
%    \end{macrocode}

% \macro{\childdocmain}
% The macro |\childdocmain| is to be called at the top of the main file
% with nothing or the main filename (without extension) as argument.
% First, it breaks loops.
% If the argument is not empty and does not match |\childdocname|
% (which is set by the first inclusion of |childdoc.def|),
% |\ifchilddoc| is set to true, |\includeonly| is applied to the child file
% and |\jobname| is set to the main file
% (for proper handling of |.aux| files):
%    \begin{macrocode}
\newcommand{\childdocmain}[1]
{
  \childdocdisable\childdocmain{}
  \if?#1?\else
    \begingroup
      \def\childdoctmp{#1}
      \ifx\childdoctmp\childdocname
        \def\childdoctmp{}
      \else
        \def\childdoctmp
        {
          \childdoctrue
          \includeonly{\childdocname}
          \def\childdocjob{#1}
          \def\jobname{#1}
        }
      \fi
      \expandafter
    \endgroup
    \childdoctmp
  \fi
}
%    \end{macrocode}

% \macro{\childdocof}
% The command |\childdocof| redirects
% compilation to the main file |#1|.
%    \begin{macrocode}
\newcommand{\childdocof}[1]
{
  \childdocdisable
  \childdoctrue
  \includeonly{\childdocname}
  \def\jobname{#1}
  \def\childdocjob{#1}
  \input{#1}
}
%    \end{macrocode}

% \macro{\childdocby}
% The command |\childdocby| ....
%    \begin{macrocode}
\newcommand{\childdocby}[2][]
{
  \childdocdisable
  \childdoctrue
  \childdocmanualtrue
  \if?#1?\else
    \def\jobname{#2}
  \fi
  \def\childdocjob{#2}
  \input{#2}
  \endinput
}
%    \end{macrocode}

% \macro{\childdocforward}
% The command |\childdocforward| redirects
% compilation to the main file or
% (if the optional argument is given) a child file.
% Parameters are set as if the main file
% or a child file starting with |\childdocof| was compiled.
% Then compilation is handed over to the main file:
%    \begin{macrocode}
\newcommand{\childdocforward}[2][]
{
  \begingroup
    \if?#1?
      \def\childdoctmp
      {
        \def\childdocname{#2}
        \def\childdocjob{#2}
        \def\jobname{#2}
        \input{#2}
        \endinput
      }
    \else
      \def\childdoctmp
      {
        \childdocdisable
        \def\childdocname{#2}
        \childdoctrue
        \includeonly{#2}
        \def\childdocjob{#1}
        \def\jobname{#1}
        \input{#1}
        \endinput
      }
    \fi
    \expandafter
  \endgroup
  \childdoctmp
}
%    \end{macrocode}

% \macro{\childdocforwardprefix}
% The command |\childdocforwardprefix| redirects
% compilation to the main or a child file by means of a pattern.
% The prefix |#1| in the current filename is replaced by |#2|
% and the suffix of the current filename is kept
% (it is assumed that the filename does not contain the substring `|~~~|'
% which is used as a delimiter).
% Compilation is handed over to the new file by |\childdocforward|:
%    \begin{macrocode}
\newcommand{\childdocforwardprefix}[3][]
{
  \begingroup
    \def\childdocextract #2##1~~~{\def\childdoctmp{\childdocforward[#1]{#3##1}}}
    \expandafter\childdocextract\childdocname~~~
    \expandafter
  \endgroup
  \childdoctmp
}
%    \end{macrocode}

% \macro{\childdoc}
% The deprecated macro |\childdoc| is a legacy version of |\childdocmain|:
%    \begin{macrocode}
\newcommand{\childdoc}{\childdocmain}
%    \end{macrocode}

% \macro{\childdocredirect}
% The deprecated macro |\childdocredirect| is a legacy version
% of |\childdocforward| and |\childdocforwardprefix|:
%    \begin{macrocode}
\newcommand{\childdocredirect}[2][]
{
  \begingroup
    \if?#1?
      \def\childdoctmp{\childdocforward{#2}}
    \else
      \def\childdoctmp{\childdocforwardprefix{#1}{#2}}
    \fi
    \expandafter
  \endgroup
  \childdoctmp
}
%    \end{macrocode}

%\iffalse
%</package>
%\fi
%
\endinput
|\\
|\childdocmain{}|\\
\end{tabular}
\end{center}
at the very top of the main \LaTeX{} file,
in particular \emph{before} the |\documentclass| statement!
The argument of |\childdocmain| should be left empty
(but it must be present).

%%%%%%%%%%%%%%%%%%%%%%%%%%%%%%%%%%%%%%%%
\DescribeMacro{\childdocof}
Furthermore, add the commands
\begin{center}
\begin{tabular}{l}
|% \iffalse
%
% childdoc.dtx Copyright (C) 2017-2018 Niklas Beisert
%
% This work may be distributed and/or modified under the
% conditions of the LaTeX Project Public License, either version 1.3
% of this license or (at your option) any later version.
% The latest version of this license is in
%   http://www.latex-project.org/lppl.txt
% and version 1.3 or later is part of all distributions of LaTeX
% version 2005/12/01 or later.
%
% This work has the LPPL maintenance status `maintained'.
%
% The Current Maintainer of this work is Niklas Beisert.
%
% This work consists of the files childdoc.dtx and childdoc.ins
% and the derived files childdoc.def and cdocsamp.tex with
% cdocsch1.tex, cdocsch2.tex, cdocsdrf.tex, cdocsfn1.tex, cdocsfn2.tex.
%
%<package>\ifdefined\childdocmain\endinput\fi
%<package>\ProvidesFile{childdoc.def}[2018/12/30 v2.0 child document driver]
%<samplemain>\ProvidesFile{cdocsamp.tex}[2018/12/30 v2.0 sample for childdoc]
%<*driver>
%\ProvidesFile{childdoc.drv}[2018/12/30 v2.0 childdoc reference manual file]
\PassOptionsToClass{10pt,a4paper}{article}
\documentclass{ltxdoc}

\usepackage[margin=35mm]{geometry}
\usepackage{hyperref}
\usepackage{hyperxmp}
\usepackage[usenames]{color}

\hypersetup{colorlinks=true}
\hypersetup{pdfstartview=FitH}
\hypersetup{pdfpagemode=UseNone}
\hypersetup{pdfsource={}}
\hypersetup{pdflang={en-UK}}
\hypersetup{pdfcopyright={Copyright 2017-2018 Niklas Beisert.
  This work may be distributed and/or modified under the
  conditions of the LaTeX Project Public License, either version 1.3
  of this license or (at your option) any later version.}}
\hypersetup{pdflicenseurl={http://www.latex-project.org/lppl.txt}}
\hypersetup{pdfcontactaddress={ETH Zurich, ITP, HIT K,
  Wolfgang-Pauli-Strasse 27}}
\hypersetup{pdfcontactpostcode={8093}}
\hypersetup{pdfcontactcity={Zurich}}
\hypersetup{pdfcontactcountry={Switzerland}}
\hypersetup{pdfcontactemail={nbeisert@itp.phys.ethz.ch}}
\hypersetup{pdfcontacturl={http://people.phys.ethz.ch/\xmptilde nbeisert/}}

\newcommand{\secref}[1]{\hyperref[#1]{section \ref*{#1}}}

\parskip1ex
\parindent0pt
\let\olditemize\itemize
\def\itemize{\olditemize\parskip0pt}

\begin{document}

\title{The \textsf{childdoc} Package}
\hypersetup{pdftitle={The childdoc Package}}
\author{Niklas Beisert\\[2ex]
  Institut f\"ur Theoretische Physik\\
  Eidgen\"ossische Technische Hochschule Z\"urich\\
  Wolfgang-Pauli-Strasse 27, 8093 Z\"urich, Switzerland\\[1ex]
  \href{mailto:nbeisert@itp.phys.ethz.ch}
  {\texttt{nbeisert@itp.phys.ethz.ch}}}
\hypersetup{pdfauthor={Niklas Beisert}}
\hypersetup{pdfsubject={Manual for the LaTeX2e Package childdoc}}
\date{30 December 2018, \textsf{v2.0}}
\maketitle

\begin{abstract}\noindent
\textsf{childdoc} is a \LaTeXe{} package
that enables the direct compilation
of document sections included by |\include|
to individual files.
\end{abstract}

\begingroup
\parskip0ex
\tableofcontents
\endgroup

%%%%%%%%%%%%%%%%%%%%%%%%%%%%%%%%%%%%%%%%%%%%%%%%%%%%%%%%%%%%%%%%%%%%%%%%%%%%%%%%
%%%%%%%%%%%%%%%%%%%%%%%%%%%%%%%%%%%%%%%%%%%%%%%%%%%%%%%%%%%%%%%%%%%%%%%%%%%%%%%%
\section{Introduction}

\LaTeX{} provides a mechanism to structure a large document (such as a book)
into a main file and several child files (containing the chapters)
using the |\include| command.
This mechanism is beneficial for documents
which span hundreds of pages in order to
make the source file(s) more manageable.
Moreover, compilation can be restricted to
selected child files by means of the |\includeonly| command.
The latter feature can be used to reduce the compilation time while editing
(this was significantly more useful in the earlier days of \LaTeX{})
or to generate a smaller document which is easier to navigate.
Another application of |\includeonly| is to generate
documents consisting of selected parts of the complete document.

However, there are a few drawbacks of the plain |\include| mechanism:
\begin{itemize}
\item
The child files cannot be compiled on their own,
they can only be compiled via the main file.
A naive editing environment
(such as a text editor with an option
to have the current file processed by \LaTeX)
may require one to switch to the main file before compiling;
attempting to compile the child file produces errors.
\item
The main file must be modified (each time)
to adjust the |\includeonly| command
to the present needs. This easily leaves the main file in a messy state.
\item
The generated document will always carry the filename
of the main document. This is inconvenient if
several child files are to be compiled and
to be kept for distribution.
\end{itemize}

The present package provides a simple interface
to make child files individually compilable by \LaTeX{}.
Compiling a child file then has the same effect as compiling
the main file with an |\includeonly| command
to select the appropriate child.
Moreover the generated document will carry the name of the child
rather than the main file.
This resolves all three above issues.

This feature is meant to make the editing of books,
thesis documents and lecture notes somewhat more convenient.
However, the package can also be used efficiently for
composing a series of documents (such as exercise sheets)
which are typically distributed individually.
It then assists the author in generating the individual documents
(potentially in different versions)
as well as a document containing the collected series.
Another application is in developing style files
or other kinds of included material
where compilation of the style file could redirect
to a sample or test file.

%%%%%%%%%%%%%%%%%%%%%%%%%%%%%%%%%%%%%%%%%%%%%%%%%%%%%%%%%%%%%%%%%%%%%%%%%%%%%%%%
%%%%%%%%%%%%%%%%%%%%%%%%%%%%%%%%%%%%%%%%%%%%%%%%%%%%%%%%%%%%%%%%%%%%%%%%%%%%%%%%
\section{Usage}

First of all, the package \textsf{childdoc} is \emph{not} a standard
\LaTeXe{} |.sty| style file! Therefore it needs to be invoked in
a non-standard way.

%%%%%%%%%%%%%%%%%%%%%%%%%%%%%%%%%%%%%%%%%%%%%%%%%%%%%%%%%%%%%%%%%%%%%%%%%%%%%%%%
\subsection{Included Files}
\label{sec:include}

%%%%%%%%%%%%%%%%%%%%%%%%%%%%%%%%%%%%%%%%
\DescribeMacro{\childdocmain}
To use the package, add the commands
\begin{center}
\begin{tabular}{l}
|% \iffalse
%
% childdoc.dtx Copyright (C) 2017-2018 Niklas Beisert
%
% This work may be distributed and/or modified under the
% conditions of the LaTeX Project Public License, either version 1.3
% of this license or (at your option) any later version.
% The latest version of this license is in
%   http://www.latex-project.org/lppl.txt
% and version 1.3 or later is part of all distributions of LaTeX
% version 2005/12/01 or later.
%
% This work has the LPPL maintenance status `maintained'.
%
% The Current Maintainer of this work is Niklas Beisert.
%
% This work consists of the files childdoc.dtx and childdoc.ins
% and the derived files childdoc.def and cdocsamp.tex with
% cdocsch1.tex, cdocsch2.tex, cdocsdrf.tex, cdocsfn1.tex, cdocsfn2.tex.
%
%<package>\ifdefined\childdocmain\endinput\fi
%<package>\ProvidesFile{childdoc.def}[2018/12/30 v2.0 child document driver]
%<samplemain>\ProvidesFile{cdocsamp.tex}[2018/12/30 v2.0 sample for childdoc]
%<*driver>
%\ProvidesFile{childdoc.drv}[2018/12/30 v2.0 childdoc reference manual file]
\PassOptionsToClass{10pt,a4paper}{article}
\documentclass{ltxdoc}

\usepackage[margin=35mm]{geometry}
\usepackage{hyperref}
\usepackage{hyperxmp}
\usepackage[usenames]{color}

\hypersetup{colorlinks=true}
\hypersetup{pdfstartview=FitH}
\hypersetup{pdfpagemode=UseNone}
\hypersetup{pdfsource={}}
\hypersetup{pdflang={en-UK}}
\hypersetup{pdfcopyright={Copyright 2017-2018 Niklas Beisert.
  This work may be distributed and/or modified under the
  conditions of the LaTeX Project Public License, either version 1.3
  of this license or (at your option) any later version.}}
\hypersetup{pdflicenseurl={http://www.latex-project.org/lppl.txt}}
\hypersetup{pdfcontactaddress={ETH Zurich, ITP, HIT K,
  Wolfgang-Pauli-Strasse 27}}
\hypersetup{pdfcontactpostcode={8093}}
\hypersetup{pdfcontactcity={Zurich}}
\hypersetup{pdfcontactcountry={Switzerland}}
\hypersetup{pdfcontactemail={nbeisert@itp.phys.ethz.ch}}
\hypersetup{pdfcontacturl={http://people.phys.ethz.ch/\xmptilde nbeisert/}}

\newcommand{\secref}[1]{\hyperref[#1]{section \ref*{#1}}}

\parskip1ex
\parindent0pt
\let\olditemize\itemize
\def\itemize{\olditemize\parskip0pt}

\begin{document}

\title{The \textsf{childdoc} Package}
\hypersetup{pdftitle={The childdoc Package}}
\author{Niklas Beisert\\[2ex]
  Institut f\"ur Theoretische Physik\\
  Eidgen\"ossische Technische Hochschule Z\"urich\\
  Wolfgang-Pauli-Strasse 27, 8093 Z\"urich, Switzerland\\[1ex]
  \href{mailto:nbeisert@itp.phys.ethz.ch}
  {\texttt{nbeisert@itp.phys.ethz.ch}}}
\hypersetup{pdfauthor={Niklas Beisert}}
\hypersetup{pdfsubject={Manual for the LaTeX2e Package childdoc}}
\date{30 December 2018, \textsf{v2.0}}
\maketitle

\begin{abstract}\noindent
\textsf{childdoc} is a \LaTeXe{} package
that enables the direct compilation
of document sections included by |\include|
to individual files.
\end{abstract}

\begingroup
\parskip0ex
\tableofcontents
\endgroup

%%%%%%%%%%%%%%%%%%%%%%%%%%%%%%%%%%%%%%%%%%%%%%%%%%%%%%%%%%%%%%%%%%%%%%%%%%%%%%%%
%%%%%%%%%%%%%%%%%%%%%%%%%%%%%%%%%%%%%%%%%%%%%%%%%%%%%%%%%%%%%%%%%%%%%%%%%%%%%%%%
\section{Introduction}

\LaTeX{} provides a mechanism to structure a large document (such as a book)
into a main file and several child files (containing the chapters)
using the |\include| command.
This mechanism is beneficial for documents
which span hundreds of pages in order to
make the source file(s) more manageable.
Moreover, compilation can be restricted to
selected child files by means of the |\includeonly| command.
The latter feature can be used to reduce the compilation time while editing
(this was significantly more useful in the earlier days of \LaTeX{})
or to generate a smaller document which is easier to navigate.
Another application of |\includeonly| is to generate
documents consisting of selected parts of the complete document.

However, there are a few drawbacks of the plain |\include| mechanism:
\begin{itemize}
\item
The child files cannot be compiled on their own,
they can only be compiled via the main file.
A naive editing environment
(such as a text editor with an option
to have the current file processed by \LaTeX)
may require one to switch to the main file before compiling;
attempting to compile the child file produces errors.
\item
The main file must be modified (each time)
to adjust the |\includeonly| command
to the present needs. This easily leaves the main file in a messy state.
\item
The generated document will always carry the filename
of the main document. This is inconvenient if
several child files are to be compiled and
to be kept for distribution.
\end{itemize}

The present package provides a simple interface
to make child files individually compilable by \LaTeX{}.
Compiling a child file then has the same effect as compiling
the main file with an |\includeonly| command
to select the appropriate child.
Moreover the generated document will carry the name of the child
rather than the main file.
This resolves all three above issues.

This feature is meant to make the editing of books,
thesis documents and lecture notes somewhat more convenient.
However, the package can also be used efficiently for
composing a series of documents (such as exercise sheets)
which are typically distributed individually.
It then assists the author in generating the individual documents
(potentially in different versions)
as well as a document containing the collected series.
Another application is in developing style files
or other kinds of included material
where compilation of the style file could redirect
to a sample or test file.

%%%%%%%%%%%%%%%%%%%%%%%%%%%%%%%%%%%%%%%%%%%%%%%%%%%%%%%%%%%%%%%%%%%%%%%%%%%%%%%%
%%%%%%%%%%%%%%%%%%%%%%%%%%%%%%%%%%%%%%%%%%%%%%%%%%%%%%%%%%%%%%%%%%%%%%%%%%%%%%%%
\section{Usage}

First of all, the package \textsf{childdoc} is \emph{not} a standard
\LaTeXe{} |.sty| style file! Therefore it needs to be invoked in
a non-standard way.

%%%%%%%%%%%%%%%%%%%%%%%%%%%%%%%%%%%%%%%%%%%%%%%%%%%%%%%%%%%%%%%%%%%%%%%%%%%%%%%%
\subsection{Included Files}
\label{sec:include}

%%%%%%%%%%%%%%%%%%%%%%%%%%%%%%%%%%%%%%%%
\DescribeMacro{\childdocmain}
To use the package, add the commands
\begin{center}
\begin{tabular}{l}
|\input{childdoc.def}|\\
|\childdocmain{}|\\
\end{tabular}
\end{center}
at the very top of the main \LaTeX{} file,
in particular \emph{before} the |\documentclass| statement!
The argument of |\childdocmain| should be left empty
(but it must be present).

%%%%%%%%%%%%%%%%%%%%%%%%%%%%%%%%%%%%%%%%
\DescribeMacro{\childdocof}
Furthermore, add the commands
\begin{center}
\begin{tabular}{l}
|\input{childdoc.def}|\\
|\childdocof{|\textit{main}|}|\\
\end{tabular}
\end{center}
at the top of every child file \textit{child}
which is included by |\include{|\textit{child}|}|
from within the main file
(or at least for those files to be compiled individually).
The argument \textit{main} must be the filename of the main file.

There are a couple of
considerations in setting up the main and child documents:

%%%%%%%%%%%%%%%%%%%%%%%%%%%%%%%%%%%%%%%%
\paragraph{Restrictions.}

Please note the following restrictions:
\begin{itemize}
\item
|\childdocmain| must be called with one argument \textit{main}
to ensure compatibility with earlier version of the package.
It must either be empty (|\childdocmain{}|)
or precisely match the filename of the main file in which it is specified.
See \secref{sec:detection} for further information.
\item
The filename \textit{main} must be specified without the |.tex| extension.
\item
The filename \textit{main} is case sensitive
(even in case-insensitive file systems)
due to internal string comparison.
\item
The argument \textit{main} should be fully expanded, it cannot be a macro.
\item
Subdirectories and special characters should be avoided in filenames.
\item
The command |\childdocmain{|\textit{main}|}| must be followed by a whitespace.
It should not be followed immediately by another command
or by a comment mark `|%|'.
This is because the \TeX{} parser reads the token immediately following
the argument of |\childdocmain| and puts it
at the beginning of every child section;
however, a white\-space is ignored.
\end{itemize}

%%%%%%%%%%%%%%%%%%%%%%%%%%%%%%%%%%%%%%%%
\paragraph{Content of Main File.}

It is advisable to place all content in the child files included by |\include|.
Any output contained in the main file will appear in all child documents
unless suppressed manually;
it cannot be suppressed automatically by the |\includeonly| directive
and thus should normally be avoided.
A method to include some content in the main file
by means of conditional processing is described in \secref{sec:conditional}.

%%%%%%%%%%%%%%%%%%%%%%%%%%%%%%%%%%%%%%%%
\paragraph{Page Numbering.}

When only a part of the document is compiled,
the appropriate numbering of pages
(as well as other status parameters)
is determined from the |.aux| files.
The latter contain information from previous passes.
However this information needs to propagate through
all intermediate child documents.
Therefore the page numbering in child documents may well
be inconsistent until the complete document is compiled at least once.

A useful (if unconventional) way to always ensure a consistent
page numbering is to restart the numbering in each child document
and denote the pages by `\textit{child}|.|\textit{page}'
where \textit{child} represents the chapter/section number of the child file.
This can be achieved by the command
|\numberwithin{page}{|\textit{child}|}|
of the \textsf{amsmath} package
where \textit{child} can be |chapter| or |section|
depending on the chosen structuring.
Alternatively, one can modify the macro |\thepage| appropriately
and reset the counter |page| at the start of each child file.

%%%%%%%%%%%%%%%%%%%%%%%%%%%%%%%%%%%%%%%%%%%%%%%%%%%%%%%%%%%%%%%%%%%%%%%%%%%%%%%%
\subsection{Conditional Processing}
\label{sec:conditional}

The package provides a mechanism to compile different versions
of a document. To customise the versions further some conditional processing
can come in handy to distinguish which version is being compiled.
The package provides two macros to describe the compilation context:

%%%%%%%%%%%%%%%%%%%%%%%%%%%%%%%%%%%%%%%%
\DescribeMacro{\ifchilddoc}
The conditional |\ifchilddoc| distinguishes between the compilation of
child documents and the main document:
%
\begin{center}
|\ifchilddoc |\textit{child-code}| |[|\||else |\textit{main-code}]| \||fi|
\end{center}

%%%%%%%%%%%%%%%%%%%%%%%%%%%%%%%%%%%%%%%%
\DescribeMacro{\childdocname}
\DescribeMacro{\childdocjob}
The macro |\childdocname| contains the filename (without extension)
of the main or child file being processed.
Note that |\childdocjob| will always contain the name of the main file.

%%%%%%%%%%%%%%%%%%%%%%%%%%%%%%%%%%%%%%%%
\paragraph{Title Page.}

Conditional processing can be used to include a title or banner page
in the main document when proper precautions are taken.
Importantly, the code in the main file should ensure that the page counter
(as well as other status parameters which are stored in the |.aux| files)
takes the same value after the conditional processing.
Otherwise the page numbers may take divergent values
depending on which part is compiled.

For example, a title page could be declared by:
%
\begin{center}
\begin{tabular}{l}
|\ifchilddoc\||else|\\
|\addtocounter{page}{-1}|\\
\textit{code for title page}\\
|\newpage|\\
|\||fi|
\end{tabular}
\end{center}
%
A banner page for the child documents can be generated by:
%
\begin{center}
\begin{tabular}{l}
|\ifchilddoc|\\
|\addtocounter{page}{-1}|\\
\textit{code for banner page}\\
|\newpage|\\
|\||fi|
\end{tabular}
\end{center}
%
Here one could write a message such as:
\begin{center}
|This is the part \childdocname{} of \childdocjob{}.|
\end{center}

%%%%%%%%%%%%%%%%%%%%%%%%%%%%%%%%%%%%%%%%%%%%%%%%%%%%%%%%%%%%%%%%%%%%%%%%%%%%%%%%
\subsection{Flags}
\label{sec:flags}

The package makes it easy to generate different versions
of the main or child documents.
To this end compilation flags can be defined
and assigned different default values.
They will be particularly useful in conjunction
with the forwarding mechanism described in \secref{sec:forward}.

For example, it may be useful to have a flag |\version|
which can be set to |draft| or |final|.
The document source will contain some conditional code
depending on the value of |\version|.
Suppose further, the flag should default to |final| for the main file
and to |draft| for child files
which is a natural assignment for editing the document.
This is achieved by placing the following code
in the preamble of the main document
(below the |\childdocmain| directive):
%
\begin{center}
\begin{tabular}{l}
|\ifchilddoc|\\
|\providecommand{\version}{draft}|\\
|\||else|\\
|\providecommand{\version}{final}|\\
|\||fi|
\end{tabular}
\end{center}
%
The definition by |\providecommand| makes sure
that previous definitions are not overwritten.
Further statements |\providecommand{\version}{...}|
can thus be added before the above code to override it.

For the main file, one might add a line
(between |\childdocmain| and the above block)
%
\begin{center}
|%\ifchilddoc\||else\providecommand{\version}{draft}\||fi|
\end{center}
%
which can be uncommented to produce a draft version.
Likewise one can add a line to the very top of a child file
(above the |\childdocof{|\textit{main}|}| directive)
%
\begin{center}
|%\providecommand{\version}{final}|
\end{center}
%
which can be uncommented to produce the final version of this child document.

%%%%%%%%%%%%%%%%%%%%%%%%%%%%%%%%%%%%%%%%%%%%%%%%%%%%%%%%%%%%%%%%%%%%%%%%%%%%%%%%
\subsection{Forwarding}
\label{sec:forward}

Different versions of the main or child documents
using compilation flags as described in \secref{sec:flags}
can be (permanently) stored in different files
for convenient compilation, viewing and distribution.
To this end, the package defines a command
to pass on compilation to a different file:

%%%%%%%%%%%%%%%%%%%%%%%%%%%%%%%%%%%%%%%%
\DescribeMacro{\childdocforward}
The command |\childdocforward| redirects processing to
another source file:
%
\begin{center}
\begin{tabular}{l}
|\input{childdoc.def}|\\
|\childdocforward[|\textit{main}|]{|\textit{dest}|}|\\
\end{tabular}
\end{center}
%
The argument \textit{dest} is the destination file
(without extension).
It should be the main file or one of the child files.
Note that further \textsf{childdoc} directives
such as |\childdocof| and |\childdocforward|
in the indicated file will be processed in this form.
The optional argument \textit{main}
passes on directly to the main file \textit{main}
while pretending to compile the child \textit{dest}.
This form behaves as if \textit{dest}
issues |\childdocof{|\textit{main}|}| right away,
and no further \textsf{childdoc} directives will be processed.

%%%%%%%%%%%%%%%%%%%%%%%%%%%%%%%%%%%%%%%%
\DescribeMacro{\...prefix}
In the alternative form |\childdocforwardprefix|,
%
\begin{center}
\begin{tabular}{l}
|\input{childdoc.def}|\\
|\childdocforwardprefix[|\textit{main}|]{|\textit{prefix}|}{|\textit{dest}|}|
\end{tabular}
\end{center}
%
the destination file is determined by a pattern
depending on the current file:
To make this work, the current file must be called
`{\textit{prefix}\hspace{0.2em}\textit{suffix}}'
with \textit{prefix} matching precisely the argument.
Processing is then passed on to the file
`{\textit{dest}\hspace{0.2em}\textit{suffix}}'.
Surely, the same effect is achieved by
directly specifying the
argument `{\textit{dest}\hspace{0.2em}\textit{suffix}}'
in the first form.
However, that requires to set up a different file
for each child. With the alternative form of the command
all these files can have exactly the same content
which simplifies setting them up and maintaining them.

For example, the following file |draft.tex|
with a compilation flag |\version| as described in \secref{sec:flags}
compiles the main document as a draft:
%
\begin{center}
\begin{tabular}{l}
|\def\version{draft}|\\
|\input{childdoc.def}|\\
|\childdocforward{|\textit{main}|}|
\end{tabular}
\end{center}
%
Likewise, the following files |final|\textit{nn}|.tex|
compile the final version of the child document
|child|\textit{nn}|.tex|:
%
\begin{center}
\begin{tabular}{l}
|\def\version{final}|\\
|\input{childdoc.def}|\\
|\childdocforwardprefix{final}{child}|
\end{tabular}
\end{center}
%

Note that when several versions of a main file and/or of each child file
are to be generated, it may be convenient to set up a |Makefile| or
shell script to automatise the process.

%%%%%%%%%%%%%%%%%%%%%%%%%%%%%%%%%%%%%%%%%%%%%%%%%%%%%%%%%%%%%%%%%%%%%%%%%%%%%%%%
\subsection{Command Line Processing}
\label{sec:commandline}

The effect of redirection files can also be achieved by invoking
the \LaTeX{} compiler with a more elaborate command line.
Most conveniently this should be done as part
of a shell script or a |Makefile|.

When using \textsf{childdoc} in the main file, the following
command lines effectively perform a redirection
(note that depending on the shell being used,
backslashes may have to be doubled: `|\|' $\to$ `|\\|'):
%
\begin{center}
|... -jobname "|\textit{target}|" |\\|"|[\textit{flags}]%
|\input{childdoc.def}\childdocforward[|\textit{main}|]{|\textit{dest}|}"|
\end{center}
%
Here \textit{target} is the name of the output file,
\textit{main} is the name of the main file
and \textit{dest} is the name of the main or child file to be processed
(all filenames without extensions).
The optional argument \textit{main} can be omitted
if \textit{main} matches \textit{dest}.
Optionally, compilation \textit{flags} can be defined via |\def| commands.
This command line makes the \TeX{} engine believe
it is compiling the file \textit{target}
whose content is specified as the latter parameter.
The provided code then forwards the processing to
\textit{main} or \textit{dest} as described in \secref{sec:forward}.

%%%%%%%%%%%%%%%%%%%%%%%%%%%%%%%%%%%%%%%%%%%%%%%%%%%%%%%%%%%%%%%%%%%%%%%%%%%%%%%%
\subsection{Include by Input}
\label{sec:input}

Including child documents by |\include| has some restrictions by design.
Most notably, the content of a child document always occupies
its own set of pages; pages cannot be shared between child documents.
Usually, this behaviour makes perfect sense
because each child document contain an essential part of the document.
However, in some situations it may be desirable to compose
a document from a collection of parts
without having mandatory page breaks between then.
For this case, the package
provides a mechanism to include parts
by |\input| which can also be processed individually.
However, by construction this mechanism
requires manual handling of the content to be output.

%%%%%%%%%%%%%%%%%%%%%%%%%%%%%%%%%%%%%%%%
\DescribeMacro{\ifchilddocmanual}
The main file should be prepared as usual, see \secref{sec:include}.
However, the document body must make a distinction
between processing of an individual part and of the main document, e.g.:
%
\begin{center}
\begin{tabular}{l}
|\ifchilddocmanual|\\
|\input{\childdocname}|\\
|\||else|\\
\textit{document body with }|\input{|\textit{part}|}|\\
|\||fi|
\end{tabular}
\end{center}
%
The conditional |\ifchilddocmanual| is true whenever
a part to be included by |\input| is being compiled,
and the name of the part is stored in |\childdocname|.

%%%%%%%%%%%%%%%%%%%%%%%%%%%%%%%%%%%%%%%%
\DescribeMacro{\childdocby}
Each part to be included by |\input| should start with:
%
\begin{center}
\begin{tabular}{l}
|\input{childdoc.def}|\\
|\childdocby{|\textit{main}|}|\\
\end{tabular}
\end{center}
%
The directive |\childdocby| is similar to |\childdocof|
described in \secref{sec:include},
but the subsequent selection of content must be done manually.
To that end, both |\ifchilddoc| and |\ifchilddocmanual|
will be true upon processing of a part,
and the name of the part is stored in |\childdocname|.
Note that |\jobname| will be set to the filename of the current part
so that each part receives an individual |.aux| file
that does not interfere with the |.aux| file(s) of the main document.
This behaviour can be altered by the alternative form
|\childdocby[*]{|\textit{main}|}| (with a non-empty optional argument)
which uses the |.aux| file of the main document
by setting |\jobname| to \textit{main}.

%%%%%%%%%%%%%%%%%%%%%%%%%%%%%%%%%%%%%%%%%%%%%%%%%%%%%%%%%%%%%%%%%%%%%%%%%%%%%%%%
\subsection{Driver Development}
\label{sec:driver}

The \textsf{childdoc} mechanism can also be use for the development
of definition files such as \LaTeX{} styles or classes.
This case differs from the above setup with multiple parts
included by |\include| in that no |\includeonly| should be invoked.
This can be achieved by starting the include file
(before |\ProvidesPackage|) with:
%
\begin{center}
\begin{tabular}{l}
|\input{childdoc.def}|\\
|\childdocforward{|\textit{main}|}|\\
\end{tabular}
\end{center}
%
or alternatively with:
%
\begin{center}
\begin{tabular}{l}
|\input{childdoc.def}|\\
|\childdocby{|\textit{main}|}|\\
\end{tabular}
\end{center}
%
Both forms have slightly different effects as described above.
The main file is prepared as usual, see \secref{sec:include}.

%%%%%%%%%%%%%%%%%%%%%%%%%%%%%%%%%%%%%%%%%%%%%%%%%%%%%%%%%%%%%%%%%%%%%%%%%%%%%%%%
\subsection{Legacy Detection}
\label{sec:detection}

The directive |\childdocmain| in the main file can detect
whether the complete document or merely a child is to be compiled
even without using the directive |\childdocof|.
This method is deprecated because it is less robust
and there is no compelling reason to use it;
it is merely provided for backward compatibility
and it may be removed in future versions.

If the detection mechanism is to be used,
it is mandatory to correctly specify
the filename of the main file as the argument of |\childdocmain|:
%
\begin{center}
\begin{tabular}{l}
|\input{childdoc.def}|\\
|\childdocmain{|\textit{main}|}|\\
\end{tabular}
\end{center}
%
If |\jobname| does not match the argument \textit{main} of |\childdocmain|,
it is assumed that |\jobname| points to the child file to be compiled.
When using |\childdocmain| with the main file specified as argument,
it suffices to start a child file
with just |\input{|\textit{main}|}|
without loading of the package and using |\childdocof|.
If instead all processing is done
with the appropriate \textsf{childdoc} directives,
the argument of \textit{main} of |\childdocmain| can be empty.

An alternative version of the command line processing described
in \secref{sec:commandline} using the detection mechanism reads:
%
\begin{center}
|... -jobname "|\textit{target}|" "|[\textit{flags}]%
[|\def\jobname{|\textit{dest}|}|]|\input{|\textit{main}|}"|
\end{center}

%%%%%%%%%%%%%%%%%%%%%%%%%%%%%%%%%%%%%%%%%%%%%%%%%%%%%%%%%%%%%%%%%%%%%%%%%%%%%%%%
\subsection{Manual Code}
\label{sec:manual}

In case one cannot be certain whether the definitions file |childdoc.def|
is installed on the target \TeX{} distribution
and one prefers not to ship it,
it is conceivable to paste a few relevant commands into the sources.

To that end, drop all statements |\input{childdoc.def}|
and perform the replacements as outlined below.
Instead of |\childdocmain{|\textit{main}|}| add the following code
to the top of the main file:
%
\begin{center}
\begin{tabular}{l}
|\||ifdefined\childdocname\endinput\||fi\newif\ifchilddoc|\\
|\edef\childdocname{\scantokens\expandafter{\jobname\noexpand}}|\\
|\def\childdocmain{|\textit{main}|}\||ifx\childdocmain\childdocname\||else|\\
|\childdoctrue\includeonly{\childdocname}\let\jobname\childdocmain\||fi|\\
\end{tabular}
\end{center}
%
Instead of |\childdocof{|\textit{main}|}| just include the main file
at the top of each child file:
%
\begin{center}
|\input{|\textit{main}|}|
\end{center}
%
A simple redirection |\childdocforward{|\textit{dest}|}| is achieved by:
%
\begin{center}
|\def\jobname{|\textit{dest}|}\input{\jobname}|
\end{center}
%
The redirection with prefix
|\childdocforwardprefix[|\textit{prefix}|]{|\textit{dest}|}|
is accomplished by:
%
\begin{center}
\begin{tabular}{l}
|{\edef\jobname{\scantokens\expandafter{\jobname\noexpand}}|\\
|\def\redirectjob |\textit{prefix}|#1~~~{\gdef\jobname{|\textit{dest}|#1}}|\\
|\expandafter\redirectjob\jobname~~~}\input{\jobname}|
\end{tabular}
\end{center}

In an alternative approach,
child documents can be compiled by a specific command line
without additional code or specific definitions:
%
\begin{center}
|... -jobname "|\textit{target}|" "|[\textit{flags}]%
|\includeonly{|\textit{dest}|}\input{|\textit{main}|}"|
\end{center}
%

%%%%%%%%%%%%%%%%%%%%%%%%%%%%%%%%%%%%%%%%%%%%%%%%%%%%%%%%%%%%%%%%%%%%%%%%%%%%%%%%
%%%%%%%%%%%%%%%%%%%%%%%%%%%%%%%%%%%%%%%%%%%%%%%%%%%%%%%%%%%%%%%%%%%%%%%%%%%%%%%%
\section{Information}

%%%%%%%%%%%%%%%%%%%%%%%%%%%%%%%%%%%%%%%%%%%%%%%%%%%%%%%%%%%%%%%%%%%%%%%%%%%%%%%%
\subsection{Copyright}

Copyright \copyright{} 2017--2018 Niklas Beisert

This work may be distributed and/or modified under the
conditions of the \LaTeX{} Project Public License, either version 1.3
of this license or (at your option) any later version.
The latest version of this license is in
  \url{http://www.latex-project.org/lppl.txt}
and version 1.3 or later is part of all distributions of \LaTeX{}
version 2005/12/01 or later.

This work has the LPPL maintenance status `maintained'.

The Current Maintainer of this work is Niklas Beisert.

This work consists of the files |README.txt|, |childdoc.ins| and |childdoc.dtx|
as well as the derived files |childdoc.def|, |cdocsamp.tex|
with |cdocsch1.tex|, |cdocsch2.tex|, |cdocspt3.tex|, |cdocspt4.tex|,
|cdocsdrf.tex|, |cdocsfn1.tex|, |cdocsfn2.tex|
as well as |childdoc.pdf|.

%%%%%%%%%%%%%%%%%%%%%%%%%%%%%%%%%%%%%%%%%%%%%%%%%%%%%%%%%%%%%%%%%%%%%%%%%%%%%%%%
\subsection{Files and Installation}

The package consists of the files:
%
\begin{center}
\begin{tabular}{ll}
    |README.txt|   & readme file \\
    |childdoc.ins| & installation file \\
    |childdoc.dtx| & source file \\
    |childdoc.def| & definition file \\
    |cdocsamp.tex| & sample main file \\
    |cdocsch1.tex| & sample include file \\
    |cdocsch2.tex| & sample include file \\
    |cdocspt3.tex| & sample part file \\
    |cdocspt4.tex| & sample part file \\
    |cdocsdrf.tex| & sample redirection file \\
    |cdocsfn1.tex| & sample redirection file \\
    |cdocsfn2.tex| & sample redirection file \\
    |childdoc.pdf| & manual
\end{tabular}
\end{center}
%
The distribution consists of the files
|README.txt|, |childdoc.ins| and |childdoc.dtx|.
%
\begin{itemize}
\item
Run (pdf)\LaTeX{} on |childdoc.dtx|
to compile the manual |childdoc.pdf| (this file).
\item
Run \LaTeX{} on |childdoc.ins| to create the definitions file |childdoc.def|
and the sample |cdocsamp.tex| with include files
|cdocsch1.tex|, |cdocsch2.tex|, |cdocspt3.tex|, |cdocspt4.tex|,
|cdocsdrf.tex|, |cdocsfn1.tex|, |cdocsfn2.tex|.
Then copy the file |childdoc.def| to an appropriate directory of your \LaTeX{}
distribution, e.g.\ \textit{texmf-root}|/tex/latex/childdoc|.
\end{itemize}

%%%%%%%%%%%%%%%%%%%%%%%%%%%%%%%%%%%%%%%%%%%%%%%%%%%%%%%%%%%%%%%%%%%%%%%%%%%%%%%%
\subsection{Related CTAN Packages}

There are several other packages which offer a similar functionality:
%
\begin{itemize}
\item
The packages
\href{http://ctan.org/pkg/docmute}{\textsf{docmute}},
\href{http://ctan.org/pkg/includex}{\textsf{includex}} and
\href{http://ctan.org/pkg/standalone}{\textsf{standalone}}
provide commands to include only the document body of
a child file thus allowing both files to be compiled individually.
\item
The packages \href{http://ctan.org/pkg/subdocs}{\textsf{subdocs}}
and \href{http://ctan.org/pkg/subfiles}{\textsf{subfiles}}
provide structures in which the main and child documents can be
encapsulated and allowing them to be compiled individually.
The inclusion mechanism is different from the conventional |\include|.
\item
The package \href{http://ctan.org/pkg/combine}{\textsf{combine}}
is an elaborate solution to combine several documents into one.
\end{itemize}
%
See also the CTAN topic \href{http://ctan.org/topic/subdocs}{\textsf{subdocs}}
for further related packages.
The present package differs from the above solutions in that
a document structure constructed with the conventional |\include| mechanism
just needs two extra commands at the top of every file
such that all constituent files can be compiled individually.

%%%%%%%%%%%%%%%%%%%%%%%%%%%%%%%%%%%%%%%%%%%%%%%%%%%%%%%%%%%%%%%%%%%%%%%%%%%%%%%%
%\subsection{Feature Suggestions}
%
%The following is a list of features which may be useful for future
%versions of this package:
%%
%\begin{itemize}
%\item
%\ldots
%\end{itemize}

%%%%%%%%%%%%%%%%%%%%%%%%%%%%%%%%%%%%%%%%%%%%%%%%%%%%%%%%%%%%%%%%%%%%%%%%%%%%%%%%
\subsection{Revision History}

%%%%%%%%%%%%%%%%%%%%%%%%%%%%%%%%%%%%%%%%
\paragraph{v2.0:} 2018/12/30

\begin{itemize}
\item
immediate forward processing
\item
added |\childdocby| mechanism
\item
manual restructured
\end{itemize}

%%%%%%%%%%%%%%%%%%%%%%%%%%%%%%%%%%%%%%%%
\paragraph{v1.6:} 2018/01/17

\begin{itemize}
\item
application for development of include files
\item
corrections to manual
\end{itemize}

%%%%%%%%%%%%%%%%%%%%%%%%%%%%%%%%%%%%%%%%
\paragraph{v1.5:} 2017/05/21

\begin{itemize}
\item
more complete structuring introduced
\item
|\childdocof| introduced
\item
|\childdoc| renamed to |\childdocmain|
\item
|\childredirect| renamed to |\childdocforward| and |\childdocforwardprefix|
and functionality expanded
\end{itemize}

%%%%%%%%%%%%%%%%%%%%%%%%%%%%%%%%%%%%%%%%
\paragraph{v1.0:} 2017/04/27

\begin{itemize}
\item
manual and install package
\item
first version published on CTAN
\end{itemize}

%%%%%%%%%%%%%%%%%%%%%%%%%%%%%%%%%%%%%%%%
\paragraph{v0.6:} 2017/04/26

\begin{itemize}
\item
redirection mechanism added
\end{itemize}

%%%%%%%%%%%%%%%%%%%%%%%%%%%%%%%%%%%%%%%%
\paragraph{v0.5:} 2017/04/26

\begin{itemize}
\item
functionality in definition file
\end{itemize}


%%%%%%%%%%%%%%%%%%%%%%%%%%%%%%%%%%%%%%%%%%%%%%%%%%%%%%%%%%%%%%%%%%%%%%%%%%%%%%%%
%%%%%%%%%%%%%%%%%%%%%%%%%%%%%%%%%%%%%%%%%%%%%%%%%%%%%%%%%%%%%%%%%%%%%%%%%%%%%%%%
%%%%%%%%%%%%%%%%%%%%%%%%%%%%%%%%%%%%%%%%%%%%%%%%%%%%%%%%%%%%%%%%%%%%%%%%%%%%%%%%
\appendix

\settowidth\MacroIndent{\rmfamily\scriptsize 000\ }

 \DocInput{childdoc.dtx}

\end{document}
%</driver>
% \fi
%
% %%%%%%%%%%%%%%%%%%%%%%%%%%%%%%%%%%%%%%%%%%%%%%%%%%%%%%%%%%%%%%%%%%%%%%%%%%%%%%
% %%%%%%%%%%%%%%%%%%%%%%%%%%%%%%%%%%%%%%%%%%%%%%%%%%%%%%%%%%%%%%%%%%%%%%%%%%%%%%
% \section{Sample}
%\iffalse
%<*samplemain>
%\fi
%
% The following presents a sample document
% with two chapters, two parts, a title page,
% a compile flag as well as three forwarding files to set the flag.
% It consists of eight |.tex| files:
% \begin{center}
% \begin{tabular}{ll}
% |cdocsamp.tex|&main file\\
% |cdocsch1.tex|&include file for chapter 1\\
% |cdocsch2.tex|&include file for chapter 2\\
% |cdocspt3.tex|&include file for part 3\\
% |cdocspt4.tex|&include file for part 4\\
% |cdocsdrf.tex|&forwarding file for main file in draft mode\\
% |cdocsfi1.tex|&forwarding file for final version of chapter 1\\
% |cdocsfi2.tex|&forwarding file for final version of chapter 2\\
% \end{tabular}
% \end{center}
% Each of the eight files can be compiled directly by the \LaTeX{} compiler.
%
% %%%%%%%%%%%%%%%%%%%%%%%%%%%%%%%%%%%%%%
% \paragraph{Main File.}
%
% The main file is called |cdocsamp.tex|.
%
% Load the \textsf{childdoc} definitions and
% declare the filename for the main document:
%    \begin{macrocode}
\input{childdoc.def}
\childdocmain{}
%    \end{macrocode}

% Optional override for |\version| flag:
%    \begin{macrocode}
%%\ifchilddoc\else\providecommand{\version}{draft}\fi
%    \end{macrocode}

% Define the default values for the |\version| flag
% (|final| for the main file and |draft| for childs):
%    \begin{macrocode}
\ifchilddoc
\providecommand{\version}{draft}
\else
\providecommand{\version}{final}
\fi
%    \end{macrocode}

% Load the standard document class:
%    \begin{macrocode}
\documentclass[12pt]{article}
%    \end{macrocode}

% Start the document body:
%    \begin{macrocode}
\begin{document}
%    \end{macrocode}

% Declare a title page.
% Print title, part of document being processed and version flag:
%    \begin{macrocode}
\addtocounter{page}{-1}
\begin{center}
{\LARGE\bfseries{}childdoc example\par}
\vspace{1cm}
\ifchilddoc
\ifchilddocmanual part\else chapter\fi:
`\childdocname' of `\childdocjob'\par
\else
main document: `\childdocjob'\par
\fi
version: \version\par
\end{center}
\newpage
%    \end{macrocode}

% Manually include selected file,
% otherwise process as usual:
%    \begin{macrocode}
\ifchilddocmanual
\section*{part `\childdocname'}
\input{\childdocname}
\else
%    \end{macrocode}

% Include the two chapters:
%    \begin{macrocode}
\include{cdocsch1}
\include{cdocsch2}
%    \end{macrocode}

% Include the two parts unless only chapters should be displayed:
%    \begin{macrocode}
\ifchilddoc\else
\section{part three}
\input{cdocspt3}
\section{part four}
\input{cdocspt4}
\fi
%    \end{macrocode}

% Process as usual until here:
%    \begin{macrocode}
\fi
%    \end{macrocode}

% End of document body:
%    \begin{macrocode}
\end{document}
%    \end{macrocode}
%\iffalse
%</samplemain>
%\fi
%
% %%%%%%%%%%%%%%%%%%%%%%%%%%%%%%%%%%%%%%
% \paragraph{Chapter Include Files.}
%
% The include files are called |cdocsch1.tex| and |cdocsch2.tex|.
%
%\iffalse
%<*samplechap1|samplechap2>
%\fi

% Optional override for |\version| flag:
%    \begin{macrocode}
%%\providecommand{\version}{final}
%    \end{macrocode}

% Include the main document:
%    \begin{macrocode}
\input{childdoc.def}
\childdocof{cdocsamp}
%    \end{macrocode}

%\iffalse
%</samplechap1|samplechap2>
%\fi
%
%\iffalse
%<*samplechap1>
%\fi
% Some text for chapter 1:
%    \begin{macrocode}
\section{one}
some text in chapter one
%    \end{macrocode}

%\iffalse
%</samplechap1>
%\fi
% Some text for chapter 2:
%\iffalse
%<*samplechap2>
%\fi
%    \begin{macrocode}
\section{two}
more text in chapter two
%    \end{macrocode}

%\iffalse
%</samplechap2>
%\fi
%
% %%%%%%%%%%%%%%%%%%%%%%%%%%%%%%%%%%%%%%
% \paragraph{Part Include Files.}
%
% The include files are called |cdocspt3.tex| and |cdocspt4.tex|.
%
%\iffalse
%<*samplepart3|samplepart4>
%\fi

% Optional override for |\version| flag:
%    \begin{macrocode}
%%\providecommand{\version}{final}
%    \end{macrocode}

% Include the main document:
%    \begin{macrocode}
\input{childdoc.def}
\childdocby{cdocsamp}
%    \end{macrocode}

%\iffalse
%</samplepart3|samplepart4>
%\fi
%
%\iffalse
%<*samplepart3>
%\fi
% Some text for part 3:
%    \begin{macrocode}
some text in part three
%    \end{macrocode}

%\iffalse
%</samplepart3>
%\fi
% Some text for part 4:
%\iffalse
%<*samplepart4>
%\fi
%    \begin{macrocode}
more text in part four
%    \end{macrocode}

%\iffalse
%</samplepart4>
%\fi
%
% %%%%%%%%%%%%%%%%%%%%%%%%%%%%%%%%%%%%%%
% \paragraph{Forwarding for a Complete Draft.}
%
% The following forwarding file |cdocsdrf.tex|
% compiles the main document in draft mode:
%\iffalse
%<*sampledraft>
%\fi
%    \begin{macrocode}
\def\version{draft}
\input{childdoc.def}
\childdocforward{cdocsamp}
%    \end{macrocode}

%\iffalse
%</sampledraft>
%\fi
%
% %%%%%%%%%%%%%%%%%%%%%%%%%%%%%%%%%%%%%%
% \paragraph{Forwarding for Final Version of the Chapters.}
%
% The following forwarding files |cdocsfn1.tex| and |cdocsfn2.tex|
% (with identical content)
% compile the final versions of the child documents
% |cdocsch1.tex| and |cdocsch2.tex|, respectively:
%\iffalse
%<*samplefinal>
%\fi
%    \begin{macrocode}
\def\version{final}
\input{childdoc.def}
\childdocforwardprefix[cdocsamp]{cdocsfn}{cdocsch}
%    \end{macrocode}

%\iffalse
%</samplefinal>
%\fi
%
% %%%%%%%%%%%%%%%%%%%%%%%%%%%%%%%%%%%%%%
% \paragraph{Command Line Processing.}
%
% The following three command lines generate the output files
% |cdocscld|, |cdocscl1| and |cdocscl2|
% which should be identical to
% |cdocsdrf|, |cdocsch1| and |cdocsfn2|, respectively:
% \begin{center}
% \begin{tabular}{l}
% |latex -jobname cdocscld \|\\
% |  "\def\version{draft}\input{childdoc.def}\childdocforward{cdocsamp}"|\\
% |latex -jobname cdocscl1 \|\\
% |  "\input{childdoc.def}\childdocforward[cdocsamp]{cdocsch1}"|\\
% |latex -jobname cdocscl2 \|\\
% |  "\def\version{final}\input{childdoc.def}\childdocforward{cdocsch2}"|
% \end{tabular}
% \end{center}
% Note that the trailing backslash on each first line
% merely continues the input to the second line
% (for convenient cut ant paste).
% Furthermore, the command |latex| can be replaced by any
% of its alternative versions such as |pdflatex|.
%
% %%%%%%%%%%%%%%%%%%%%%%%%%%%%%%%%%%%%%%%%%%%%%%%%%%%%%%%%%%%%%%%%%%%%%%%%%%%%%%
% %%%%%%%%%%%%%%%%%%%%%%%%%%%%%%%%%%%%%%%%%%%%%%%%%%%%%%%%%%%%%%%%%%%%%%%%%%%%%%
% \section{Implementation}
%\iffalse
%<*package>
%\fi
%
% This section describes the definitions file |childdoc.def|.

% The definitions cannot be loaded using |\usepackage| or |\RequirePackage|
% which has a mechanism to prevent loading a style file more than once.
% When loading the definitions by means of |\input|
% multiple instances have to be prevented manually:
%\iffalse
%This code needs to be before the `\ProvidesFile' directive
%which is defined at the beginning of this file.
%Therefore it is also placed there and commented out here.
%</package>
%<*discard>
%\fi
%    \begin{macrocode}
\ifdefined\childdocmain\endinput\fi
%    \end{macrocode}
%\iffalse
%</discard>
%<*package>
%\fi
%
% \macro{\ifchilddoc}
% \macro{\ifchilddocmanual}
% The conditional |\ifchilddoc| tells whether a
% child (true) or main (false) document is being compiled.
% The conditional |\ifchilddocmanual| tells whether
% the |\includeonly| mechanism is used (false) or
% the selection of child files must be performed manually (true).
% The definitions initialise to false:
%    \begin{macrocode}
\newif\ifchilddoc
\newif\ifchilddocmanual
%    \end{macrocode}

% \macro{\childdocname}
% \macro{\childdocjob}
% The macro |\childdocname| stores the name of the main document
% to be compiled. The macro |\childdocjob| stores the name of
% the document on which the \LaTeX{} compiler was originally invoked.
% The content of |\jobname| cannot be compared
% to filenames specified in the source due to different catcodes.
% The following code rescans |\jobname|, stores the result
% in |\childdocname| and saves a copy in |\childdocjob|:
%    \begin{macrocode}
\edef\childdocname{\scantokens\expandafter{\jobname\noexpand}}
\let\childdocjob\childdocname
%    \end{macrocode}

% \macro{\childdocdisable}
% The macro |\childdocdisable| prevents the main file
% from being processed more than once.
% At this stage, the main document command |\childdocmain|
% is assumed to be called once again where it should do nothing.
% Any subsequent call to it should prevent
% a secondary processing of the main document
% It overwrites the forwarding commands
% |\childdocof| and |\childdocforward|
% with empty macros to prevent further inclusions of the main document:
%    \begin{macrocode}
\newcommand{\childdocdisable}
{
  \renewcommand{\childdocmain}[1]{\renewcommand{\childdocmain}[1]{\endinput}}
  \renewcommand{\childdocof}[1]{}
  \renewcommand{\childdocby}[2][]{}
  \renewcommand{\childdocforward}[2][]{}
  \renewcommand{\childdocdisable}{}
}
%    \end{macrocode}

% \macro{\childdocmain}
% The macro |\childdocmain| is to be called at the top of the main file
% with nothing or the main filename (without extension) as argument.
% First, it breaks loops.
% If the argument is not empty and does not match |\childdocname|
% (which is set by the first inclusion of |childdoc.def|),
% |\ifchilddoc| is set to true, |\includeonly| is applied to the child file
% and |\jobname| is set to the main file
% (for proper handling of |.aux| files):
%    \begin{macrocode}
\newcommand{\childdocmain}[1]
{
  \childdocdisable\childdocmain{}
  \if?#1?\else
    \begingroup
      \def\childdoctmp{#1}
      \ifx\childdoctmp\childdocname
        \def\childdoctmp{}
      \else
        \def\childdoctmp
        {
          \childdoctrue
          \includeonly{\childdocname}
          \def\childdocjob{#1}
          \def\jobname{#1}
        }
      \fi
      \expandafter
    \endgroup
    \childdoctmp
  \fi
}
%    \end{macrocode}

% \macro{\childdocof}
% The command |\childdocof| redirects
% compilation to the main file |#1|.
%    \begin{macrocode}
\newcommand{\childdocof}[1]
{
  \childdocdisable
  \childdoctrue
  \includeonly{\childdocname}
  \def\jobname{#1}
  \def\childdocjob{#1}
  \input{#1}
}
%    \end{macrocode}

% \macro{\childdocby}
% The command |\childdocby| ....
%    \begin{macrocode}
\newcommand{\childdocby}[2][]
{
  \childdocdisable
  \childdoctrue
  \childdocmanualtrue
  \if?#1?\else
    \def\jobname{#2}
  \fi
  \def\childdocjob{#2}
  \input{#2}
  \endinput
}
%    \end{macrocode}

% \macro{\childdocforward}
% The command |\childdocforward| redirects
% compilation to the main file or
% (if the optional argument is given) a child file.
% Parameters are set as if the main file
% or a child file starting with |\childdocof| was compiled.
% Then compilation is handed over to the main file:
%    \begin{macrocode}
\newcommand{\childdocforward}[2][]
{
  \begingroup
    \if?#1?
      \def\childdoctmp
      {
        \def\childdocname{#2}
        \def\childdocjob{#2}
        \def\jobname{#2}
        \input{#2}
        \endinput
      }
    \else
      \def\childdoctmp
      {
        \childdocdisable
        \def\childdocname{#2}
        \childdoctrue
        \includeonly{#2}
        \def\childdocjob{#1}
        \def\jobname{#1}
        \input{#1}
        \endinput
      }
    \fi
    \expandafter
  \endgroup
  \childdoctmp
}
%    \end{macrocode}

% \macro{\childdocforwardprefix}
% The command |\childdocforwardprefix| redirects
% compilation to the main or a child file by means of a pattern.
% The prefix |#1| in the current filename is replaced by |#2|
% and the suffix of the current filename is kept
% (it is assumed that the filename does not contain the substring `|~~~|'
% which is used as a delimiter).
% Compilation is handed over to the new file by |\childdocforward|:
%    \begin{macrocode}
\newcommand{\childdocforwardprefix}[3][]
{
  \begingroup
    \def\childdocextract #2##1~~~{\def\childdoctmp{\childdocforward[#1]{#3##1}}}
    \expandafter\childdocextract\childdocname~~~
    \expandafter
  \endgroup
  \childdoctmp
}
%    \end{macrocode}

% \macro{\childdoc}
% The deprecated macro |\childdoc| is a legacy version of |\childdocmain|:
%    \begin{macrocode}
\newcommand{\childdoc}{\childdocmain}
%    \end{macrocode}

% \macro{\childdocredirect}
% The deprecated macro |\childdocredirect| is a legacy version
% of |\childdocforward| and |\childdocforwardprefix|:
%    \begin{macrocode}
\newcommand{\childdocredirect}[2][]
{
  \begingroup
    \if?#1?
      \def\childdoctmp{\childdocforward{#2}}
    \else
      \def\childdoctmp{\childdocforwardprefix{#1}{#2}}
    \fi
    \expandafter
  \endgroup
  \childdoctmp
}
%    \end{macrocode}

%\iffalse
%</package>
%\fi
%
\endinput
|\\
|\childdocmain{}|\\
\end{tabular}
\end{center}
at the very top of the main \LaTeX{} file,
in particular \emph{before} the |\documentclass| statement!
The argument of |\childdocmain| should be left empty
(but it must be present).

%%%%%%%%%%%%%%%%%%%%%%%%%%%%%%%%%%%%%%%%
\DescribeMacro{\childdocof}
Furthermore, add the commands
\begin{center}
\begin{tabular}{l}
|% \iffalse
%
% childdoc.dtx Copyright (C) 2017-2018 Niklas Beisert
%
% This work may be distributed and/or modified under the
% conditions of the LaTeX Project Public License, either version 1.3
% of this license or (at your option) any later version.
% The latest version of this license is in
%   http://www.latex-project.org/lppl.txt
% and version 1.3 or later is part of all distributions of LaTeX
% version 2005/12/01 or later.
%
% This work has the LPPL maintenance status `maintained'.
%
% The Current Maintainer of this work is Niklas Beisert.
%
% This work consists of the files childdoc.dtx and childdoc.ins
% and the derived files childdoc.def and cdocsamp.tex with
% cdocsch1.tex, cdocsch2.tex, cdocsdrf.tex, cdocsfn1.tex, cdocsfn2.tex.
%
%<package>\ifdefined\childdocmain\endinput\fi
%<package>\ProvidesFile{childdoc.def}[2018/12/30 v2.0 child document driver]
%<samplemain>\ProvidesFile{cdocsamp.tex}[2018/12/30 v2.0 sample for childdoc]
%<*driver>
%\ProvidesFile{childdoc.drv}[2018/12/30 v2.0 childdoc reference manual file]
\PassOptionsToClass{10pt,a4paper}{article}
\documentclass{ltxdoc}

\usepackage[margin=35mm]{geometry}
\usepackage{hyperref}
\usepackage{hyperxmp}
\usepackage[usenames]{color}

\hypersetup{colorlinks=true}
\hypersetup{pdfstartview=FitH}
\hypersetup{pdfpagemode=UseNone}
\hypersetup{pdfsource={}}
\hypersetup{pdflang={en-UK}}
\hypersetup{pdfcopyright={Copyright 2017-2018 Niklas Beisert.
  This work may be distributed and/or modified under the
  conditions of the LaTeX Project Public License, either version 1.3
  of this license or (at your option) any later version.}}
\hypersetup{pdflicenseurl={http://www.latex-project.org/lppl.txt}}
\hypersetup{pdfcontactaddress={ETH Zurich, ITP, HIT K,
  Wolfgang-Pauli-Strasse 27}}
\hypersetup{pdfcontactpostcode={8093}}
\hypersetup{pdfcontactcity={Zurich}}
\hypersetup{pdfcontactcountry={Switzerland}}
\hypersetup{pdfcontactemail={nbeisert@itp.phys.ethz.ch}}
\hypersetup{pdfcontacturl={http://people.phys.ethz.ch/\xmptilde nbeisert/}}

\newcommand{\secref}[1]{\hyperref[#1]{section \ref*{#1}}}

\parskip1ex
\parindent0pt
\let\olditemize\itemize
\def\itemize{\olditemize\parskip0pt}

\begin{document}

\title{The \textsf{childdoc} Package}
\hypersetup{pdftitle={The childdoc Package}}
\author{Niklas Beisert\\[2ex]
  Institut f\"ur Theoretische Physik\\
  Eidgen\"ossische Technische Hochschule Z\"urich\\
  Wolfgang-Pauli-Strasse 27, 8093 Z\"urich, Switzerland\\[1ex]
  \href{mailto:nbeisert@itp.phys.ethz.ch}
  {\texttt{nbeisert@itp.phys.ethz.ch}}}
\hypersetup{pdfauthor={Niklas Beisert}}
\hypersetup{pdfsubject={Manual for the LaTeX2e Package childdoc}}
\date{30 December 2018, \textsf{v2.0}}
\maketitle

\begin{abstract}\noindent
\textsf{childdoc} is a \LaTeXe{} package
that enables the direct compilation
of document sections included by |\include|
to individual files.
\end{abstract}

\begingroup
\parskip0ex
\tableofcontents
\endgroup

%%%%%%%%%%%%%%%%%%%%%%%%%%%%%%%%%%%%%%%%%%%%%%%%%%%%%%%%%%%%%%%%%%%%%%%%%%%%%%%%
%%%%%%%%%%%%%%%%%%%%%%%%%%%%%%%%%%%%%%%%%%%%%%%%%%%%%%%%%%%%%%%%%%%%%%%%%%%%%%%%
\section{Introduction}

\LaTeX{} provides a mechanism to structure a large document (such as a book)
into a main file and several child files (containing the chapters)
using the |\include| command.
This mechanism is beneficial for documents
which span hundreds of pages in order to
make the source file(s) more manageable.
Moreover, compilation can be restricted to
selected child files by means of the |\includeonly| command.
The latter feature can be used to reduce the compilation time while editing
(this was significantly more useful in the earlier days of \LaTeX{})
or to generate a smaller document which is easier to navigate.
Another application of |\includeonly| is to generate
documents consisting of selected parts of the complete document.

However, there are a few drawbacks of the plain |\include| mechanism:
\begin{itemize}
\item
The child files cannot be compiled on their own,
they can only be compiled via the main file.
A naive editing environment
(such as a text editor with an option
to have the current file processed by \LaTeX)
may require one to switch to the main file before compiling;
attempting to compile the child file produces errors.
\item
The main file must be modified (each time)
to adjust the |\includeonly| command
to the present needs. This easily leaves the main file in a messy state.
\item
The generated document will always carry the filename
of the main document. This is inconvenient if
several child files are to be compiled and
to be kept for distribution.
\end{itemize}

The present package provides a simple interface
to make child files individually compilable by \LaTeX{}.
Compiling a child file then has the same effect as compiling
the main file with an |\includeonly| command
to select the appropriate child.
Moreover the generated document will carry the name of the child
rather than the main file.
This resolves all three above issues.

This feature is meant to make the editing of books,
thesis documents and lecture notes somewhat more convenient.
However, the package can also be used efficiently for
composing a series of documents (such as exercise sheets)
which are typically distributed individually.
It then assists the author in generating the individual documents
(potentially in different versions)
as well as a document containing the collected series.
Another application is in developing style files
or other kinds of included material
where compilation of the style file could redirect
to a sample or test file.

%%%%%%%%%%%%%%%%%%%%%%%%%%%%%%%%%%%%%%%%%%%%%%%%%%%%%%%%%%%%%%%%%%%%%%%%%%%%%%%%
%%%%%%%%%%%%%%%%%%%%%%%%%%%%%%%%%%%%%%%%%%%%%%%%%%%%%%%%%%%%%%%%%%%%%%%%%%%%%%%%
\section{Usage}

First of all, the package \textsf{childdoc} is \emph{not} a standard
\LaTeXe{} |.sty| style file! Therefore it needs to be invoked in
a non-standard way.

%%%%%%%%%%%%%%%%%%%%%%%%%%%%%%%%%%%%%%%%%%%%%%%%%%%%%%%%%%%%%%%%%%%%%%%%%%%%%%%%
\subsection{Included Files}
\label{sec:include}

%%%%%%%%%%%%%%%%%%%%%%%%%%%%%%%%%%%%%%%%
\DescribeMacro{\childdocmain}
To use the package, add the commands
\begin{center}
\begin{tabular}{l}
|\input{childdoc.def}|\\
|\childdocmain{}|\\
\end{tabular}
\end{center}
at the very top of the main \LaTeX{} file,
in particular \emph{before} the |\documentclass| statement!
The argument of |\childdocmain| should be left empty
(but it must be present).

%%%%%%%%%%%%%%%%%%%%%%%%%%%%%%%%%%%%%%%%
\DescribeMacro{\childdocof}
Furthermore, add the commands
\begin{center}
\begin{tabular}{l}
|\input{childdoc.def}|\\
|\childdocof{|\textit{main}|}|\\
\end{tabular}
\end{center}
at the top of every child file \textit{child}
which is included by |\include{|\textit{child}|}|
from within the main file
(or at least for those files to be compiled individually).
The argument \textit{main} must be the filename of the main file.

There are a couple of
considerations in setting up the main and child documents:

%%%%%%%%%%%%%%%%%%%%%%%%%%%%%%%%%%%%%%%%
\paragraph{Restrictions.}

Please note the following restrictions:
\begin{itemize}
\item
|\childdocmain| must be called with one argument \textit{main}
to ensure compatibility with earlier version of the package.
It must either be empty (|\childdocmain{}|)
or precisely match the filename of the main file in which it is specified.
See \secref{sec:detection} for further information.
\item
The filename \textit{main} must be specified without the |.tex| extension.
\item
The filename \textit{main} is case sensitive
(even in case-insensitive file systems)
due to internal string comparison.
\item
The argument \textit{main} should be fully expanded, it cannot be a macro.
\item
Subdirectories and special characters should be avoided in filenames.
\item
The command |\childdocmain{|\textit{main}|}| must be followed by a whitespace.
It should not be followed immediately by another command
or by a comment mark `|%|'.
This is because the \TeX{} parser reads the token immediately following
the argument of |\childdocmain| and puts it
at the beginning of every child section;
however, a white\-space is ignored.
\end{itemize}

%%%%%%%%%%%%%%%%%%%%%%%%%%%%%%%%%%%%%%%%
\paragraph{Content of Main File.}

It is advisable to place all content in the child files included by |\include|.
Any output contained in the main file will appear in all child documents
unless suppressed manually;
it cannot be suppressed automatically by the |\includeonly| directive
and thus should normally be avoided.
A method to include some content in the main file
by means of conditional processing is described in \secref{sec:conditional}.

%%%%%%%%%%%%%%%%%%%%%%%%%%%%%%%%%%%%%%%%
\paragraph{Page Numbering.}

When only a part of the document is compiled,
the appropriate numbering of pages
(as well as other status parameters)
is determined from the |.aux| files.
The latter contain information from previous passes.
However this information needs to propagate through
all intermediate child documents.
Therefore the page numbering in child documents may well
be inconsistent until the complete document is compiled at least once.

A useful (if unconventional) way to always ensure a consistent
page numbering is to restart the numbering in each child document
and denote the pages by `\textit{child}|.|\textit{page}'
where \textit{child} represents the chapter/section number of the child file.
This can be achieved by the command
|\numberwithin{page}{|\textit{child}|}|
of the \textsf{amsmath} package
where \textit{child} can be |chapter| or |section|
depending on the chosen structuring.
Alternatively, one can modify the macro |\thepage| appropriately
and reset the counter |page| at the start of each child file.

%%%%%%%%%%%%%%%%%%%%%%%%%%%%%%%%%%%%%%%%%%%%%%%%%%%%%%%%%%%%%%%%%%%%%%%%%%%%%%%%
\subsection{Conditional Processing}
\label{sec:conditional}

The package provides a mechanism to compile different versions
of a document. To customise the versions further some conditional processing
can come in handy to distinguish which version is being compiled.
The package provides two macros to describe the compilation context:

%%%%%%%%%%%%%%%%%%%%%%%%%%%%%%%%%%%%%%%%
\DescribeMacro{\ifchilddoc}
The conditional |\ifchilddoc| distinguishes between the compilation of
child documents and the main document:
%
\begin{center}
|\ifchilddoc |\textit{child-code}| |[|\||else |\textit{main-code}]| \||fi|
\end{center}

%%%%%%%%%%%%%%%%%%%%%%%%%%%%%%%%%%%%%%%%
\DescribeMacro{\childdocname}
\DescribeMacro{\childdocjob}
The macro |\childdocname| contains the filename (without extension)
of the main or child file being processed.
Note that |\childdocjob| will always contain the name of the main file.

%%%%%%%%%%%%%%%%%%%%%%%%%%%%%%%%%%%%%%%%
\paragraph{Title Page.}

Conditional processing can be used to include a title or banner page
in the main document when proper precautions are taken.
Importantly, the code in the main file should ensure that the page counter
(as well as other status parameters which are stored in the |.aux| files)
takes the same value after the conditional processing.
Otherwise the page numbers may take divergent values
depending on which part is compiled.

For example, a title page could be declared by:
%
\begin{center}
\begin{tabular}{l}
|\ifchilddoc\||else|\\
|\addtocounter{page}{-1}|\\
\textit{code for title page}\\
|\newpage|\\
|\||fi|
\end{tabular}
\end{center}
%
A banner page for the child documents can be generated by:
%
\begin{center}
\begin{tabular}{l}
|\ifchilddoc|\\
|\addtocounter{page}{-1}|\\
\textit{code for banner page}\\
|\newpage|\\
|\||fi|
\end{tabular}
\end{center}
%
Here one could write a message such as:
\begin{center}
|This is the part \childdocname{} of \childdocjob{}.|
\end{center}

%%%%%%%%%%%%%%%%%%%%%%%%%%%%%%%%%%%%%%%%%%%%%%%%%%%%%%%%%%%%%%%%%%%%%%%%%%%%%%%%
\subsection{Flags}
\label{sec:flags}

The package makes it easy to generate different versions
of the main or child documents.
To this end compilation flags can be defined
and assigned different default values.
They will be particularly useful in conjunction
with the forwarding mechanism described in \secref{sec:forward}.

For example, it may be useful to have a flag |\version|
which can be set to |draft| or |final|.
The document source will contain some conditional code
depending on the value of |\version|.
Suppose further, the flag should default to |final| for the main file
and to |draft| for child files
which is a natural assignment for editing the document.
This is achieved by placing the following code
in the preamble of the main document
(below the |\childdocmain| directive):
%
\begin{center}
\begin{tabular}{l}
|\ifchilddoc|\\
|\providecommand{\version}{draft}|\\
|\||else|\\
|\providecommand{\version}{final}|\\
|\||fi|
\end{tabular}
\end{center}
%
The definition by |\providecommand| makes sure
that previous definitions are not overwritten.
Further statements |\providecommand{\version}{...}|
can thus be added before the above code to override it.

For the main file, one might add a line
(between |\childdocmain| and the above block)
%
\begin{center}
|%\ifchilddoc\||else\providecommand{\version}{draft}\||fi|
\end{center}
%
which can be uncommented to produce a draft version.
Likewise one can add a line to the very top of a child file
(above the |\childdocof{|\textit{main}|}| directive)
%
\begin{center}
|%\providecommand{\version}{final}|
\end{center}
%
which can be uncommented to produce the final version of this child document.

%%%%%%%%%%%%%%%%%%%%%%%%%%%%%%%%%%%%%%%%%%%%%%%%%%%%%%%%%%%%%%%%%%%%%%%%%%%%%%%%
\subsection{Forwarding}
\label{sec:forward}

Different versions of the main or child documents
using compilation flags as described in \secref{sec:flags}
can be (permanently) stored in different files
for convenient compilation, viewing and distribution.
To this end, the package defines a command
to pass on compilation to a different file:

%%%%%%%%%%%%%%%%%%%%%%%%%%%%%%%%%%%%%%%%
\DescribeMacro{\childdocforward}
The command |\childdocforward| redirects processing to
another source file:
%
\begin{center}
\begin{tabular}{l}
|\input{childdoc.def}|\\
|\childdocforward[|\textit{main}|]{|\textit{dest}|}|\\
\end{tabular}
\end{center}
%
The argument \textit{dest} is the destination file
(without extension).
It should be the main file or one of the child files.
Note that further \textsf{childdoc} directives
such as |\childdocof| and |\childdocforward|
in the indicated file will be processed in this form.
The optional argument \textit{main}
passes on directly to the main file \textit{main}
while pretending to compile the child \textit{dest}.
This form behaves as if \textit{dest}
issues |\childdocof{|\textit{main}|}| right away,
and no further \textsf{childdoc} directives will be processed.

%%%%%%%%%%%%%%%%%%%%%%%%%%%%%%%%%%%%%%%%
\DescribeMacro{\...prefix}
In the alternative form |\childdocforwardprefix|,
%
\begin{center}
\begin{tabular}{l}
|\input{childdoc.def}|\\
|\childdocforwardprefix[|\textit{main}|]{|\textit{prefix}|}{|\textit{dest}|}|
\end{tabular}
\end{center}
%
the destination file is determined by a pattern
depending on the current file:
To make this work, the current file must be called
`{\textit{prefix}\hspace{0.2em}\textit{suffix}}'
with \textit{prefix} matching precisely the argument.
Processing is then passed on to the file
`{\textit{dest}\hspace{0.2em}\textit{suffix}}'.
Surely, the same effect is achieved by
directly specifying the
argument `{\textit{dest}\hspace{0.2em}\textit{suffix}}'
in the first form.
However, that requires to set up a different file
for each child. With the alternative form of the command
all these files can have exactly the same content
which simplifies setting them up and maintaining them.

For example, the following file |draft.tex|
with a compilation flag |\version| as described in \secref{sec:flags}
compiles the main document as a draft:
%
\begin{center}
\begin{tabular}{l}
|\def\version{draft}|\\
|\input{childdoc.def}|\\
|\childdocforward{|\textit{main}|}|
\end{tabular}
\end{center}
%
Likewise, the following files |final|\textit{nn}|.tex|
compile the final version of the child document
|child|\textit{nn}|.tex|:
%
\begin{center}
\begin{tabular}{l}
|\def\version{final}|\\
|\input{childdoc.def}|\\
|\childdocforwardprefix{final}{child}|
\end{tabular}
\end{center}
%

Note that when several versions of a main file and/or of each child file
are to be generated, it may be convenient to set up a |Makefile| or
shell script to automatise the process.

%%%%%%%%%%%%%%%%%%%%%%%%%%%%%%%%%%%%%%%%%%%%%%%%%%%%%%%%%%%%%%%%%%%%%%%%%%%%%%%%
\subsection{Command Line Processing}
\label{sec:commandline}

The effect of redirection files can also be achieved by invoking
the \LaTeX{} compiler with a more elaborate command line.
Most conveniently this should be done as part
of a shell script or a |Makefile|.

When using \textsf{childdoc} in the main file, the following
command lines effectively perform a redirection
(note that depending on the shell being used,
backslashes may have to be doubled: `|\|' $\to$ `|\\|'):
%
\begin{center}
|... -jobname "|\textit{target}|" |\\|"|[\textit{flags}]%
|\input{childdoc.def}\childdocforward[|\textit{main}|]{|\textit{dest}|}"|
\end{center}
%
Here \textit{target} is the name of the output file,
\textit{main} is the name of the main file
and \textit{dest} is the name of the main or child file to be processed
(all filenames without extensions).
The optional argument \textit{main} can be omitted
if \textit{main} matches \textit{dest}.
Optionally, compilation \textit{flags} can be defined via |\def| commands.
This command line makes the \TeX{} engine believe
it is compiling the file \textit{target}
whose content is specified as the latter parameter.
The provided code then forwards the processing to
\textit{main} or \textit{dest} as described in \secref{sec:forward}.

%%%%%%%%%%%%%%%%%%%%%%%%%%%%%%%%%%%%%%%%%%%%%%%%%%%%%%%%%%%%%%%%%%%%%%%%%%%%%%%%
\subsection{Include by Input}
\label{sec:input}

Including child documents by |\include| has some restrictions by design.
Most notably, the content of a child document always occupies
its own set of pages; pages cannot be shared between child documents.
Usually, this behaviour makes perfect sense
because each child document contain an essential part of the document.
However, in some situations it may be desirable to compose
a document from a collection of parts
without having mandatory page breaks between then.
For this case, the package
provides a mechanism to include parts
by |\input| which can also be processed individually.
However, by construction this mechanism
requires manual handling of the content to be output.

%%%%%%%%%%%%%%%%%%%%%%%%%%%%%%%%%%%%%%%%
\DescribeMacro{\ifchilddocmanual}
The main file should be prepared as usual, see \secref{sec:include}.
However, the document body must make a distinction
between processing of an individual part and of the main document, e.g.:
%
\begin{center}
\begin{tabular}{l}
|\ifchilddocmanual|\\
|\input{\childdocname}|\\
|\||else|\\
\textit{document body with }|\input{|\textit{part}|}|\\
|\||fi|
\end{tabular}
\end{center}
%
The conditional |\ifchilddocmanual| is true whenever
a part to be included by |\input| is being compiled,
and the name of the part is stored in |\childdocname|.

%%%%%%%%%%%%%%%%%%%%%%%%%%%%%%%%%%%%%%%%
\DescribeMacro{\childdocby}
Each part to be included by |\input| should start with:
%
\begin{center}
\begin{tabular}{l}
|\input{childdoc.def}|\\
|\childdocby{|\textit{main}|}|\\
\end{tabular}
\end{center}
%
The directive |\childdocby| is similar to |\childdocof|
described in \secref{sec:include},
but the subsequent selection of content must be done manually.
To that end, both |\ifchilddoc| and |\ifchilddocmanual|
will be true upon processing of a part,
and the name of the part is stored in |\childdocname|.
Note that |\jobname| will be set to the filename of the current part
so that each part receives an individual |.aux| file
that does not interfere with the |.aux| file(s) of the main document.
This behaviour can be altered by the alternative form
|\childdocby[*]{|\textit{main}|}| (with a non-empty optional argument)
which uses the |.aux| file of the main document
by setting |\jobname| to \textit{main}.

%%%%%%%%%%%%%%%%%%%%%%%%%%%%%%%%%%%%%%%%%%%%%%%%%%%%%%%%%%%%%%%%%%%%%%%%%%%%%%%%
\subsection{Driver Development}
\label{sec:driver}

The \textsf{childdoc} mechanism can also be use for the development
of definition files such as \LaTeX{} styles or classes.
This case differs from the above setup with multiple parts
included by |\include| in that no |\includeonly| should be invoked.
This can be achieved by starting the include file
(before |\ProvidesPackage|) with:
%
\begin{center}
\begin{tabular}{l}
|\input{childdoc.def}|\\
|\childdocforward{|\textit{main}|}|\\
\end{tabular}
\end{center}
%
or alternatively with:
%
\begin{center}
\begin{tabular}{l}
|\input{childdoc.def}|\\
|\childdocby{|\textit{main}|}|\\
\end{tabular}
\end{center}
%
Both forms have slightly different effects as described above.
The main file is prepared as usual, see \secref{sec:include}.

%%%%%%%%%%%%%%%%%%%%%%%%%%%%%%%%%%%%%%%%%%%%%%%%%%%%%%%%%%%%%%%%%%%%%%%%%%%%%%%%
\subsection{Legacy Detection}
\label{sec:detection}

The directive |\childdocmain| in the main file can detect
whether the complete document or merely a child is to be compiled
even without using the directive |\childdocof|.
This method is deprecated because it is less robust
and there is no compelling reason to use it;
it is merely provided for backward compatibility
and it may be removed in future versions.

If the detection mechanism is to be used,
it is mandatory to correctly specify
the filename of the main file as the argument of |\childdocmain|:
%
\begin{center}
\begin{tabular}{l}
|\input{childdoc.def}|\\
|\childdocmain{|\textit{main}|}|\\
\end{tabular}
\end{center}
%
If |\jobname| does not match the argument \textit{main} of |\childdocmain|,
it is assumed that |\jobname| points to the child file to be compiled.
When using |\childdocmain| with the main file specified as argument,
it suffices to start a child file
with just |\input{|\textit{main}|}|
without loading of the package and using |\childdocof|.
If instead all processing is done
with the appropriate \textsf{childdoc} directives,
the argument of \textit{main} of |\childdocmain| can be empty.

An alternative version of the command line processing described
in \secref{sec:commandline} using the detection mechanism reads:
%
\begin{center}
|... -jobname "|\textit{target}|" "|[\textit{flags}]%
[|\def\jobname{|\textit{dest}|}|]|\input{|\textit{main}|}"|
\end{center}

%%%%%%%%%%%%%%%%%%%%%%%%%%%%%%%%%%%%%%%%%%%%%%%%%%%%%%%%%%%%%%%%%%%%%%%%%%%%%%%%
\subsection{Manual Code}
\label{sec:manual}

In case one cannot be certain whether the definitions file |childdoc.def|
is installed on the target \TeX{} distribution
and one prefers not to ship it,
it is conceivable to paste a few relevant commands into the sources.

To that end, drop all statements |\input{childdoc.def}|
and perform the replacements as outlined below.
Instead of |\childdocmain{|\textit{main}|}| add the following code
to the top of the main file:
%
\begin{center}
\begin{tabular}{l}
|\||ifdefined\childdocname\endinput\||fi\newif\ifchilddoc|\\
|\edef\childdocname{\scantokens\expandafter{\jobname\noexpand}}|\\
|\def\childdocmain{|\textit{main}|}\||ifx\childdocmain\childdocname\||else|\\
|\childdoctrue\includeonly{\childdocname}\let\jobname\childdocmain\||fi|\\
\end{tabular}
\end{center}
%
Instead of |\childdocof{|\textit{main}|}| just include the main file
at the top of each child file:
%
\begin{center}
|\input{|\textit{main}|}|
\end{center}
%
A simple redirection |\childdocforward{|\textit{dest}|}| is achieved by:
%
\begin{center}
|\def\jobname{|\textit{dest}|}\input{\jobname}|
\end{center}
%
The redirection with prefix
|\childdocforwardprefix[|\textit{prefix}|]{|\textit{dest}|}|
is accomplished by:
%
\begin{center}
\begin{tabular}{l}
|{\edef\jobname{\scantokens\expandafter{\jobname\noexpand}}|\\
|\def\redirectjob |\textit{prefix}|#1~~~{\gdef\jobname{|\textit{dest}|#1}}|\\
|\expandafter\redirectjob\jobname~~~}\input{\jobname}|
\end{tabular}
\end{center}

In an alternative approach,
child documents can be compiled by a specific command line
without additional code or specific definitions:
%
\begin{center}
|... -jobname "|\textit{target}|" "|[\textit{flags}]%
|\includeonly{|\textit{dest}|}\input{|\textit{main}|}"|
\end{center}
%

%%%%%%%%%%%%%%%%%%%%%%%%%%%%%%%%%%%%%%%%%%%%%%%%%%%%%%%%%%%%%%%%%%%%%%%%%%%%%%%%
%%%%%%%%%%%%%%%%%%%%%%%%%%%%%%%%%%%%%%%%%%%%%%%%%%%%%%%%%%%%%%%%%%%%%%%%%%%%%%%%
\section{Information}

%%%%%%%%%%%%%%%%%%%%%%%%%%%%%%%%%%%%%%%%%%%%%%%%%%%%%%%%%%%%%%%%%%%%%%%%%%%%%%%%
\subsection{Copyright}

Copyright \copyright{} 2017--2018 Niklas Beisert

This work may be distributed and/or modified under the
conditions of the \LaTeX{} Project Public License, either version 1.3
of this license or (at your option) any later version.
The latest version of this license is in
  \url{http://www.latex-project.org/lppl.txt}
and version 1.3 or later is part of all distributions of \LaTeX{}
version 2005/12/01 or later.

This work has the LPPL maintenance status `maintained'.

The Current Maintainer of this work is Niklas Beisert.

This work consists of the files |README.txt|, |childdoc.ins| and |childdoc.dtx|
as well as the derived files |childdoc.def|, |cdocsamp.tex|
with |cdocsch1.tex|, |cdocsch2.tex|, |cdocspt3.tex|, |cdocspt4.tex|,
|cdocsdrf.tex|, |cdocsfn1.tex|, |cdocsfn2.tex|
as well as |childdoc.pdf|.

%%%%%%%%%%%%%%%%%%%%%%%%%%%%%%%%%%%%%%%%%%%%%%%%%%%%%%%%%%%%%%%%%%%%%%%%%%%%%%%%
\subsection{Files and Installation}

The package consists of the files:
%
\begin{center}
\begin{tabular}{ll}
    |README.txt|   & readme file \\
    |childdoc.ins| & installation file \\
    |childdoc.dtx| & source file \\
    |childdoc.def| & definition file \\
    |cdocsamp.tex| & sample main file \\
    |cdocsch1.tex| & sample include file \\
    |cdocsch2.tex| & sample include file \\
    |cdocspt3.tex| & sample part file \\
    |cdocspt4.tex| & sample part file \\
    |cdocsdrf.tex| & sample redirection file \\
    |cdocsfn1.tex| & sample redirection file \\
    |cdocsfn2.tex| & sample redirection file \\
    |childdoc.pdf| & manual
\end{tabular}
\end{center}
%
The distribution consists of the files
|README.txt|, |childdoc.ins| and |childdoc.dtx|.
%
\begin{itemize}
\item
Run (pdf)\LaTeX{} on |childdoc.dtx|
to compile the manual |childdoc.pdf| (this file).
\item
Run \LaTeX{} on |childdoc.ins| to create the definitions file |childdoc.def|
and the sample |cdocsamp.tex| with include files
|cdocsch1.tex|, |cdocsch2.tex|, |cdocspt3.tex|, |cdocspt4.tex|,
|cdocsdrf.tex|, |cdocsfn1.tex|, |cdocsfn2.tex|.
Then copy the file |childdoc.def| to an appropriate directory of your \LaTeX{}
distribution, e.g.\ \textit{texmf-root}|/tex/latex/childdoc|.
\end{itemize}

%%%%%%%%%%%%%%%%%%%%%%%%%%%%%%%%%%%%%%%%%%%%%%%%%%%%%%%%%%%%%%%%%%%%%%%%%%%%%%%%
\subsection{Related CTAN Packages}

There are several other packages which offer a similar functionality:
%
\begin{itemize}
\item
The packages
\href{http://ctan.org/pkg/docmute}{\textsf{docmute}},
\href{http://ctan.org/pkg/includex}{\textsf{includex}} and
\href{http://ctan.org/pkg/standalone}{\textsf{standalone}}
provide commands to include only the document body of
a child file thus allowing both files to be compiled individually.
\item
The packages \href{http://ctan.org/pkg/subdocs}{\textsf{subdocs}}
and \href{http://ctan.org/pkg/subfiles}{\textsf{subfiles}}
provide structures in which the main and child documents can be
encapsulated and allowing them to be compiled individually.
The inclusion mechanism is different from the conventional |\include|.
\item
The package \href{http://ctan.org/pkg/combine}{\textsf{combine}}
is an elaborate solution to combine several documents into one.
\end{itemize}
%
See also the CTAN topic \href{http://ctan.org/topic/subdocs}{\textsf{subdocs}}
for further related packages.
The present package differs from the above solutions in that
a document structure constructed with the conventional |\include| mechanism
just needs two extra commands at the top of every file
such that all constituent files can be compiled individually.

%%%%%%%%%%%%%%%%%%%%%%%%%%%%%%%%%%%%%%%%%%%%%%%%%%%%%%%%%%%%%%%%%%%%%%%%%%%%%%%%
%\subsection{Feature Suggestions}
%
%The following is a list of features which may be useful for future
%versions of this package:
%%
%\begin{itemize}
%\item
%\ldots
%\end{itemize}

%%%%%%%%%%%%%%%%%%%%%%%%%%%%%%%%%%%%%%%%%%%%%%%%%%%%%%%%%%%%%%%%%%%%%%%%%%%%%%%%
\subsection{Revision History}

%%%%%%%%%%%%%%%%%%%%%%%%%%%%%%%%%%%%%%%%
\paragraph{v2.0:} 2018/12/30

\begin{itemize}
\item
immediate forward processing
\item
added |\childdocby| mechanism
\item
manual restructured
\end{itemize}

%%%%%%%%%%%%%%%%%%%%%%%%%%%%%%%%%%%%%%%%
\paragraph{v1.6:} 2018/01/17

\begin{itemize}
\item
application for development of include files
\item
corrections to manual
\end{itemize}

%%%%%%%%%%%%%%%%%%%%%%%%%%%%%%%%%%%%%%%%
\paragraph{v1.5:} 2017/05/21

\begin{itemize}
\item
more complete structuring introduced
\item
|\childdocof| introduced
\item
|\childdoc| renamed to |\childdocmain|
\item
|\childredirect| renamed to |\childdocforward| and |\childdocforwardprefix|
and functionality expanded
\end{itemize}

%%%%%%%%%%%%%%%%%%%%%%%%%%%%%%%%%%%%%%%%
\paragraph{v1.0:} 2017/04/27

\begin{itemize}
\item
manual and install package
\item
first version published on CTAN
\end{itemize}

%%%%%%%%%%%%%%%%%%%%%%%%%%%%%%%%%%%%%%%%
\paragraph{v0.6:} 2017/04/26

\begin{itemize}
\item
redirection mechanism added
\end{itemize}

%%%%%%%%%%%%%%%%%%%%%%%%%%%%%%%%%%%%%%%%
\paragraph{v0.5:} 2017/04/26

\begin{itemize}
\item
functionality in definition file
\end{itemize}


%%%%%%%%%%%%%%%%%%%%%%%%%%%%%%%%%%%%%%%%%%%%%%%%%%%%%%%%%%%%%%%%%%%%%%%%%%%%%%%%
%%%%%%%%%%%%%%%%%%%%%%%%%%%%%%%%%%%%%%%%%%%%%%%%%%%%%%%%%%%%%%%%%%%%%%%%%%%%%%%%
%%%%%%%%%%%%%%%%%%%%%%%%%%%%%%%%%%%%%%%%%%%%%%%%%%%%%%%%%%%%%%%%%%%%%%%%%%%%%%%%
\appendix

\settowidth\MacroIndent{\rmfamily\scriptsize 000\ }

 \DocInput{childdoc.dtx}

\end{document}
%</driver>
% \fi
%
% %%%%%%%%%%%%%%%%%%%%%%%%%%%%%%%%%%%%%%%%%%%%%%%%%%%%%%%%%%%%%%%%%%%%%%%%%%%%%%
% %%%%%%%%%%%%%%%%%%%%%%%%%%%%%%%%%%%%%%%%%%%%%%%%%%%%%%%%%%%%%%%%%%%%%%%%%%%%%%
% \section{Sample}
%\iffalse
%<*samplemain>
%\fi
%
% The following presents a sample document
% with two chapters, two parts, a title page,
% a compile flag as well as three forwarding files to set the flag.
% It consists of eight |.tex| files:
% \begin{center}
% \begin{tabular}{ll}
% |cdocsamp.tex|&main file\\
% |cdocsch1.tex|&include file for chapter 1\\
% |cdocsch2.tex|&include file for chapter 2\\
% |cdocspt3.tex|&include file for part 3\\
% |cdocspt4.tex|&include file for part 4\\
% |cdocsdrf.tex|&forwarding file for main file in draft mode\\
% |cdocsfi1.tex|&forwarding file for final version of chapter 1\\
% |cdocsfi2.tex|&forwarding file for final version of chapter 2\\
% \end{tabular}
% \end{center}
% Each of the eight files can be compiled directly by the \LaTeX{} compiler.
%
% %%%%%%%%%%%%%%%%%%%%%%%%%%%%%%%%%%%%%%
% \paragraph{Main File.}
%
% The main file is called |cdocsamp.tex|.
%
% Load the \textsf{childdoc} definitions and
% declare the filename for the main document:
%    \begin{macrocode}
\input{childdoc.def}
\childdocmain{}
%    \end{macrocode}

% Optional override for |\version| flag:
%    \begin{macrocode}
%%\ifchilddoc\else\providecommand{\version}{draft}\fi
%    \end{macrocode}

% Define the default values for the |\version| flag
% (|final| for the main file and |draft| for childs):
%    \begin{macrocode}
\ifchilddoc
\providecommand{\version}{draft}
\else
\providecommand{\version}{final}
\fi
%    \end{macrocode}

% Load the standard document class:
%    \begin{macrocode}
\documentclass[12pt]{article}
%    \end{macrocode}

% Start the document body:
%    \begin{macrocode}
\begin{document}
%    \end{macrocode}

% Declare a title page.
% Print title, part of document being processed and version flag:
%    \begin{macrocode}
\addtocounter{page}{-1}
\begin{center}
{\LARGE\bfseries{}childdoc example\par}
\vspace{1cm}
\ifchilddoc
\ifchilddocmanual part\else chapter\fi:
`\childdocname' of `\childdocjob'\par
\else
main document: `\childdocjob'\par
\fi
version: \version\par
\end{center}
\newpage
%    \end{macrocode}

% Manually include selected file,
% otherwise process as usual:
%    \begin{macrocode}
\ifchilddocmanual
\section*{part `\childdocname'}
\input{\childdocname}
\else
%    \end{macrocode}

% Include the two chapters:
%    \begin{macrocode}
\include{cdocsch1}
\include{cdocsch2}
%    \end{macrocode}

% Include the two parts unless only chapters should be displayed:
%    \begin{macrocode}
\ifchilddoc\else
\section{part three}
\input{cdocspt3}
\section{part four}
\input{cdocspt4}
\fi
%    \end{macrocode}

% Process as usual until here:
%    \begin{macrocode}
\fi
%    \end{macrocode}

% End of document body:
%    \begin{macrocode}
\end{document}
%    \end{macrocode}
%\iffalse
%</samplemain>
%\fi
%
% %%%%%%%%%%%%%%%%%%%%%%%%%%%%%%%%%%%%%%
% \paragraph{Chapter Include Files.}
%
% The include files are called |cdocsch1.tex| and |cdocsch2.tex|.
%
%\iffalse
%<*samplechap1|samplechap2>
%\fi

% Optional override for |\version| flag:
%    \begin{macrocode}
%%\providecommand{\version}{final}
%    \end{macrocode}

% Include the main document:
%    \begin{macrocode}
\input{childdoc.def}
\childdocof{cdocsamp}
%    \end{macrocode}

%\iffalse
%</samplechap1|samplechap2>
%\fi
%
%\iffalse
%<*samplechap1>
%\fi
% Some text for chapter 1:
%    \begin{macrocode}
\section{one}
some text in chapter one
%    \end{macrocode}

%\iffalse
%</samplechap1>
%\fi
% Some text for chapter 2:
%\iffalse
%<*samplechap2>
%\fi
%    \begin{macrocode}
\section{two}
more text in chapter two
%    \end{macrocode}

%\iffalse
%</samplechap2>
%\fi
%
% %%%%%%%%%%%%%%%%%%%%%%%%%%%%%%%%%%%%%%
% \paragraph{Part Include Files.}
%
% The include files are called |cdocspt3.tex| and |cdocspt4.tex|.
%
%\iffalse
%<*samplepart3|samplepart4>
%\fi

% Optional override for |\version| flag:
%    \begin{macrocode}
%%\providecommand{\version}{final}
%    \end{macrocode}

% Include the main document:
%    \begin{macrocode}
\input{childdoc.def}
\childdocby{cdocsamp}
%    \end{macrocode}

%\iffalse
%</samplepart3|samplepart4>
%\fi
%
%\iffalse
%<*samplepart3>
%\fi
% Some text for part 3:
%    \begin{macrocode}
some text in part three
%    \end{macrocode}

%\iffalse
%</samplepart3>
%\fi
% Some text for part 4:
%\iffalse
%<*samplepart4>
%\fi
%    \begin{macrocode}
more text in part four
%    \end{macrocode}

%\iffalse
%</samplepart4>
%\fi
%
% %%%%%%%%%%%%%%%%%%%%%%%%%%%%%%%%%%%%%%
% \paragraph{Forwarding for a Complete Draft.}
%
% The following forwarding file |cdocsdrf.tex|
% compiles the main document in draft mode:
%\iffalse
%<*sampledraft>
%\fi
%    \begin{macrocode}
\def\version{draft}
\input{childdoc.def}
\childdocforward{cdocsamp}
%    \end{macrocode}

%\iffalse
%</sampledraft>
%\fi
%
% %%%%%%%%%%%%%%%%%%%%%%%%%%%%%%%%%%%%%%
% \paragraph{Forwarding for Final Version of the Chapters.}
%
% The following forwarding files |cdocsfn1.tex| and |cdocsfn2.tex|
% (with identical content)
% compile the final versions of the child documents
% |cdocsch1.tex| and |cdocsch2.tex|, respectively:
%\iffalse
%<*samplefinal>
%\fi
%    \begin{macrocode}
\def\version{final}
\input{childdoc.def}
\childdocforwardprefix[cdocsamp]{cdocsfn}{cdocsch}
%    \end{macrocode}

%\iffalse
%</samplefinal>
%\fi
%
% %%%%%%%%%%%%%%%%%%%%%%%%%%%%%%%%%%%%%%
% \paragraph{Command Line Processing.}
%
% The following three command lines generate the output files
% |cdocscld|, |cdocscl1| and |cdocscl2|
% which should be identical to
% |cdocsdrf|, |cdocsch1| and |cdocsfn2|, respectively:
% \begin{center}
% \begin{tabular}{l}
% |latex -jobname cdocscld \|\\
% |  "\def\version{draft}\input{childdoc.def}\childdocforward{cdocsamp}"|\\
% |latex -jobname cdocscl1 \|\\
% |  "\input{childdoc.def}\childdocforward[cdocsamp]{cdocsch1}"|\\
% |latex -jobname cdocscl2 \|\\
% |  "\def\version{final}\input{childdoc.def}\childdocforward{cdocsch2}"|
% \end{tabular}
% \end{center}
% Note that the trailing backslash on each first line
% merely continues the input to the second line
% (for convenient cut ant paste).
% Furthermore, the command |latex| can be replaced by any
% of its alternative versions such as |pdflatex|.
%
% %%%%%%%%%%%%%%%%%%%%%%%%%%%%%%%%%%%%%%%%%%%%%%%%%%%%%%%%%%%%%%%%%%%%%%%%%%%%%%
% %%%%%%%%%%%%%%%%%%%%%%%%%%%%%%%%%%%%%%%%%%%%%%%%%%%%%%%%%%%%%%%%%%%%%%%%%%%%%%
% \section{Implementation}
%\iffalse
%<*package>
%\fi
%
% This section describes the definitions file |childdoc.def|.

% The definitions cannot be loaded using |\usepackage| or |\RequirePackage|
% which has a mechanism to prevent loading a style file more than once.
% When loading the definitions by means of |\input|
% multiple instances have to be prevented manually:
%\iffalse
%This code needs to be before the `\ProvidesFile' directive
%which is defined at the beginning of this file.
%Therefore it is also placed there and commented out here.
%</package>
%<*discard>
%\fi
%    \begin{macrocode}
\ifdefined\childdocmain\endinput\fi
%    \end{macrocode}
%\iffalse
%</discard>
%<*package>
%\fi
%
% \macro{\ifchilddoc}
% \macro{\ifchilddocmanual}
% The conditional |\ifchilddoc| tells whether a
% child (true) or main (false) document is being compiled.
% The conditional |\ifchilddocmanual| tells whether
% the |\includeonly| mechanism is used (false) or
% the selection of child files must be performed manually (true).
% The definitions initialise to false:
%    \begin{macrocode}
\newif\ifchilddoc
\newif\ifchilddocmanual
%    \end{macrocode}

% \macro{\childdocname}
% \macro{\childdocjob}
% The macro |\childdocname| stores the name of the main document
% to be compiled. The macro |\childdocjob| stores the name of
% the document on which the \LaTeX{} compiler was originally invoked.
% The content of |\jobname| cannot be compared
% to filenames specified in the source due to different catcodes.
% The following code rescans |\jobname|, stores the result
% in |\childdocname| and saves a copy in |\childdocjob|:
%    \begin{macrocode}
\edef\childdocname{\scantokens\expandafter{\jobname\noexpand}}
\let\childdocjob\childdocname
%    \end{macrocode}

% \macro{\childdocdisable}
% The macro |\childdocdisable| prevents the main file
% from being processed more than once.
% At this stage, the main document command |\childdocmain|
% is assumed to be called once again where it should do nothing.
% Any subsequent call to it should prevent
% a secondary processing of the main document
% It overwrites the forwarding commands
% |\childdocof| and |\childdocforward|
% with empty macros to prevent further inclusions of the main document:
%    \begin{macrocode}
\newcommand{\childdocdisable}
{
  \renewcommand{\childdocmain}[1]{\renewcommand{\childdocmain}[1]{\endinput}}
  \renewcommand{\childdocof}[1]{}
  \renewcommand{\childdocby}[2][]{}
  \renewcommand{\childdocforward}[2][]{}
  \renewcommand{\childdocdisable}{}
}
%    \end{macrocode}

% \macro{\childdocmain}
% The macro |\childdocmain| is to be called at the top of the main file
% with nothing or the main filename (without extension) as argument.
% First, it breaks loops.
% If the argument is not empty and does not match |\childdocname|
% (which is set by the first inclusion of |childdoc.def|),
% |\ifchilddoc| is set to true, |\includeonly| is applied to the child file
% and |\jobname| is set to the main file
% (for proper handling of |.aux| files):
%    \begin{macrocode}
\newcommand{\childdocmain}[1]
{
  \childdocdisable\childdocmain{}
  \if?#1?\else
    \begingroup
      \def\childdoctmp{#1}
      \ifx\childdoctmp\childdocname
        \def\childdoctmp{}
      \else
        \def\childdoctmp
        {
          \childdoctrue
          \includeonly{\childdocname}
          \def\childdocjob{#1}
          \def\jobname{#1}
        }
      \fi
      \expandafter
    \endgroup
    \childdoctmp
  \fi
}
%    \end{macrocode}

% \macro{\childdocof}
% The command |\childdocof| redirects
% compilation to the main file |#1|.
%    \begin{macrocode}
\newcommand{\childdocof}[1]
{
  \childdocdisable
  \childdoctrue
  \includeonly{\childdocname}
  \def\jobname{#1}
  \def\childdocjob{#1}
  \input{#1}
}
%    \end{macrocode}

% \macro{\childdocby}
% The command |\childdocby| ....
%    \begin{macrocode}
\newcommand{\childdocby}[2][]
{
  \childdocdisable
  \childdoctrue
  \childdocmanualtrue
  \if?#1?\else
    \def\jobname{#2}
  \fi
  \def\childdocjob{#2}
  \input{#2}
  \endinput
}
%    \end{macrocode}

% \macro{\childdocforward}
% The command |\childdocforward| redirects
% compilation to the main file or
% (if the optional argument is given) a child file.
% Parameters are set as if the main file
% or a child file starting with |\childdocof| was compiled.
% Then compilation is handed over to the main file:
%    \begin{macrocode}
\newcommand{\childdocforward}[2][]
{
  \begingroup
    \if?#1?
      \def\childdoctmp
      {
        \def\childdocname{#2}
        \def\childdocjob{#2}
        \def\jobname{#2}
        \input{#2}
        \endinput
      }
    \else
      \def\childdoctmp
      {
        \childdocdisable
        \def\childdocname{#2}
        \childdoctrue
        \includeonly{#2}
        \def\childdocjob{#1}
        \def\jobname{#1}
        \input{#1}
        \endinput
      }
    \fi
    \expandafter
  \endgroup
  \childdoctmp
}
%    \end{macrocode}

% \macro{\childdocforwardprefix}
% The command |\childdocforwardprefix| redirects
% compilation to the main or a child file by means of a pattern.
% The prefix |#1| in the current filename is replaced by |#2|
% and the suffix of the current filename is kept
% (it is assumed that the filename does not contain the substring `|~~~|'
% which is used as a delimiter).
% Compilation is handed over to the new file by |\childdocforward|:
%    \begin{macrocode}
\newcommand{\childdocforwardprefix}[3][]
{
  \begingroup
    \def\childdocextract #2##1~~~{\def\childdoctmp{\childdocforward[#1]{#3##1}}}
    \expandafter\childdocextract\childdocname~~~
    \expandafter
  \endgroup
  \childdoctmp
}
%    \end{macrocode}

% \macro{\childdoc}
% The deprecated macro |\childdoc| is a legacy version of |\childdocmain|:
%    \begin{macrocode}
\newcommand{\childdoc}{\childdocmain}
%    \end{macrocode}

% \macro{\childdocredirect}
% The deprecated macro |\childdocredirect| is a legacy version
% of |\childdocforward| and |\childdocforwardprefix|:
%    \begin{macrocode}
\newcommand{\childdocredirect}[2][]
{
  \begingroup
    \if?#1?
      \def\childdoctmp{\childdocforward{#2}}
    \else
      \def\childdoctmp{\childdocforwardprefix{#1}{#2}}
    \fi
    \expandafter
  \endgroup
  \childdoctmp
}
%    \end{macrocode}

%\iffalse
%</package>
%\fi
%
\endinput
|\\
|\childdocof{|\textit{main}|}|\\
\end{tabular}
\end{center}
at the top of every child file \textit{child}
which is included by |\include{|\textit{child}|}|
from within the main file
(or at least for those files to be compiled individually).
The argument \textit{main} must be the filename of the main file.

There are a couple of
considerations in setting up the main and child documents:

%%%%%%%%%%%%%%%%%%%%%%%%%%%%%%%%%%%%%%%%
\paragraph{Restrictions.}

Please note the following restrictions:
\begin{itemize}
\item
|\childdocmain| must be called with one argument \textit{main}
to ensure compatibility with earlier version of the package.
It must either be empty (|\childdocmain{}|)
or precisely match the filename of the main file in which it is specified.
See \secref{sec:detection} for further information.
\item
The filename \textit{main} must be specified without the |.tex| extension.
\item
The filename \textit{main} is case sensitive
(even in case-insensitive file systems)
due to internal string comparison.
\item
The argument \textit{main} should be fully expanded, it cannot be a macro.
\item
Subdirectories and special characters should be avoided in filenames.
\item
The command |\childdocmain{|\textit{main}|}| must be followed by a whitespace.
It should not be followed immediately by another command
or by a comment mark `|%|'.
This is because the \TeX{} parser reads the token immediately following
the argument of |\childdocmain| and puts it
at the beginning of every child section;
however, a white\-space is ignored.
\end{itemize}

%%%%%%%%%%%%%%%%%%%%%%%%%%%%%%%%%%%%%%%%
\paragraph{Content of Main File.}

It is advisable to place all content in the child files included by |\include|.
Any output contained in the main file will appear in all child documents
unless suppressed manually;
it cannot be suppressed automatically by the |\includeonly| directive
and thus should normally be avoided.
A method to include some content in the main file
by means of conditional processing is described in \secref{sec:conditional}.

%%%%%%%%%%%%%%%%%%%%%%%%%%%%%%%%%%%%%%%%
\paragraph{Page Numbering.}

When only a part of the document is compiled,
the appropriate numbering of pages
(as well as other status parameters)
is determined from the |.aux| files.
The latter contain information from previous passes.
However this information needs to propagate through
all intermediate child documents.
Therefore the page numbering in child documents may well
be inconsistent until the complete document is compiled at least once.

A useful (if unconventional) way to always ensure a consistent
page numbering is to restart the numbering in each child document
and denote the pages by `\textit{child}|.|\textit{page}'
where \textit{child} represents the chapter/section number of the child file.
This can be achieved by the command
|\numberwithin{page}{|\textit{child}|}|
of the \textsf{amsmath} package
where \textit{child} can be |chapter| or |section|
depending on the chosen structuring.
Alternatively, one can modify the macro |\thepage| appropriately
and reset the counter |page| at the start of each child file.

%%%%%%%%%%%%%%%%%%%%%%%%%%%%%%%%%%%%%%%%%%%%%%%%%%%%%%%%%%%%%%%%%%%%%%%%%%%%%%%%
\subsection{Conditional Processing}
\label{sec:conditional}

The package provides a mechanism to compile different versions
of a document. To customise the versions further some conditional processing
can come in handy to distinguish which version is being compiled.
The package provides two macros to describe the compilation context:

%%%%%%%%%%%%%%%%%%%%%%%%%%%%%%%%%%%%%%%%
\DescribeMacro{\ifchilddoc}
The conditional |\ifchilddoc| distinguishes between the compilation of
child documents and the main document:
%
\begin{center}
|\ifchilddoc |\textit{child-code}| |[|\||else |\textit{main-code}]| \||fi|
\end{center}

%%%%%%%%%%%%%%%%%%%%%%%%%%%%%%%%%%%%%%%%
\DescribeMacro{\childdocname}
\DescribeMacro{\childdocjob}
The macro |\childdocname| contains the filename (without extension)
of the main or child file being processed.
Note that |\childdocjob| will always contain the name of the main file.

%%%%%%%%%%%%%%%%%%%%%%%%%%%%%%%%%%%%%%%%
\paragraph{Title Page.}

Conditional processing can be used to include a title or banner page
in the main document when proper precautions are taken.
Importantly, the code in the main file should ensure that the page counter
(as well as other status parameters which are stored in the |.aux| files)
takes the same value after the conditional processing.
Otherwise the page numbers may take divergent values
depending on which part is compiled.

For example, a title page could be declared by:
%
\begin{center}
\begin{tabular}{l}
|\ifchilddoc\||else|\\
|\addtocounter{page}{-1}|\\
\textit{code for title page}\\
|\newpage|\\
|\||fi|
\end{tabular}
\end{center}
%
A banner page for the child documents can be generated by:
%
\begin{center}
\begin{tabular}{l}
|\ifchilddoc|\\
|\addtocounter{page}{-1}|\\
\textit{code for banner page}\\
|\newpage|\\
|\||fi|
\end{tabular}
\end{center}
%
Here one could write a message such as:
\begin{center}
|This is the part \childdocname{} of \childdocjob{}.|
\end{center}

%%%%%%%%%%%%%%%%%%%%%%%%%%%%%%%%%%%%%%%%%%%%%%%%%%%%%%%%%%%%%%%%%%%%%%%%%%%%%%%%
\subsection{Flags}
\label{sec:flags}

The package makes it easy to generate different versions
of the main or child documents.
To this end compilation flags can be defined
and assigned different default values.
They will be particularly useful in conjunction
with the forwarding mechanism described in \secref{sec:forward}.

For example, it may be useful to have a flag |\version|
which can be set to |draft| or |final|.
The document source will contain some conditional code
depending on the value of |\version|.
Suppose further, the flag should default to |final| for the main file
and to |draft| for child files
which is a natural assignment for editing the document.
This is achieved by placing the following code
in the preamble of the main document
(below the |\childdocmain| directive):
%
\begin{center}
\begin{tabular}{l}
|\ifchilddoc|\\
|\providecommand{\version}{draft}|\\
|\||else|\\
|\providecommand{\version}{final}|\\
|\||fi|
\end{tabular}
\end{center}
%
The definition by |\providecommand| makes sure
that previous definitions are not overwritten.
Further statements |\providecommand{\version}{...}|
can thus be added before the above code to override it.

For the main file, one might add a line
(between |\childdocmain| and the above block)
%
\begin{center}
|%\ifchilddoc\||else\providecommand{\version}{draft}\||fi|
\end{center}
%
which can be uncommented to produce a draft version.
Likewise one can add a line to the very top of a child file
(above the |\childdocof{|\textit{main}|}| directive)
%
\begin{center}
|%\providecommand{\version}{final}|
\end{center}
%
which can be uncommented to produce the final version of this child document.

%%%%%%%%%%%%%%%%%%%%%%%%%%%%%%%%%%%%%%%%%%%%%%%%%%%%%%%%%%%%%%%%%%%%%%%%%%%%%%%%
\subsection{Forwarding}
\label{sec:forward}

Different versions of the main or child documents
using compilation flags as described in \secref{sec:flags}
can be (permanently) stored in different files
for convenient compilation, viewing and distribution.
To this end, the package defines a command
to pass on compilation to a different file:

%%%%%%%%%%%%%%%%%%%%%%%%%%%%%%%%%%%%%%%%
\DescribeMacro{\childdocforward}
The command |\childdocforward| redirects processing to
another source file:
%
\begin{center}
\begin{tabular}{l}
|% \iffalse
%
% childdoc.dtx Copyright (C) 2017-2018 Niklas Beisert
%
% This work may be distributed and/or modified under the
% conditions of the LaTeX Project Public License, either version 1.3
% of this license or (at your option) any later version.
% The latest version of this license is in
%   http://www.latex-project.org/lppl.txt
% and version 1.3 or later is part of all distributions of LaTeX
% version 2005/12/01 or later.
%
% This work has the LPPL maintenance status `maintained'.
%
% The Current Maintainer of this work is Niklas Beisert.
%
% This work consists of the files childdoc.dtx and childdoc.ins
% and the derived files childdoc.def and cdocsamp.tex with
% cdocsch1.tex, cdocsch2.tex, cdocsdrf.tex, cdocsfn1.tex, cdocsfn2.tex.
%
%<package>\ifdefined\childdocmain\endinput\fi
%<package>\ProvidesFile{childdoc.def}[2018/12/30 v2.0 child document driver]
%<samplemain>\ProvidesFile{cdocsamp.tex}[2018/12/30 v2.0 sample for childdoc]
%<*driver>
%\ProvidesFile{childdoc.drv}[2018/12/30 v2.0 childdoc reference manual file]
\PassOptionsToClass{10pt,a4paper}{article}
\documentclass{ltxdoc}

\usepackage[margin=35mm]{geometry}
\usepackage{hyperref}
\usepackage{hyperxmp}
\usepackage[usenames]{color}

\hypersetup{colorlinks=true}
\hypersetup{pdfstartview=FitH}
\hypersetup{pdfpagemode=UseNone}
\hypersetup{pdfsource={}}
\hypersetup{pdflang={en-UK}}
\hypersetup{pdfcopyright={Copyright 2017-2018 Niklas Beisert.
  This work may be distributed and/or modified under the
  conditions of the LaTeX Project Public License, either version 1.3
  of this license or (at your option) any later version.}}
\hypersetup{pdflicenseurl={http://www.latex-project.org/lppl.txt}}
\hypersetup{pdfcontactaddress={ETH Zurich, ITP, HIT K,
  Wolfgang-Pauli-Strasse 27}}
\hypersetup{pdfcontactpostcode={8093}}
\hypersetup{pdfcontactcity={Zurich}}
\hypersetup{pdfcontactcountry={Switzerland}}
\hypersetup{pdfcontactemail={nbeisert@itp.phys.ethz.ch}}
\hypersetup{pdfcontacturl={http://people.phys.ethz.ch/\xmptilde nbeisert/}}

\newcommand{\secref}[1]{\hyperref[#1]{section \ref*{#1}}}

\parskip1ex
\parindent0pt
\let\olditemize\itemize
\def\itemize{\olditemize\parskip0pt}

\begin{document}

\title{The \textsf{childdoc} Package}
\hypersetup{pdftitle={The childdoc Package}}
\author{Niklas Beisert\\[2ex]
  Institut f\"ur Theoretische Physik\\
  Eidgen\"ossische Technische Hochschule Z\"urich\\
  Wolfgang-Pauli-Strasse 27, 8093 Z\"urich, Switzerland\\[1ex]
  \href{mailto:nbeisert@itp.phys.ethz.ch}
  {\texttt{nbeisert@itp.phys.ethz.ch}}}
\hypersetup{pdfauthor={Niklas Beisert}}
\hypersetup{pdfsubject={Manual for the LaTeX2e Package childdoc}}
\date{30 December 2018, \textsf{v2.0}}
\maketitle

\begin{abstract}\noindent
\textsf{childdoc} is a \LaTeXe{} package
that enables the direct compilation
of document sections included by |\include|
to individual files.
\end{abstract}

\begingroup
\parskip0ex
\tableofcontents
\endgroup

%%%%%%%%%%%%%%%%%%%%%%%%%%%%%%%%%%%%%%%%%%%%%%%%%%%%%%%%%%%%%%%%%%%%%%%%%%%%%%%%
%%%%%%%%%%%%%%%%%%%%%%%%%%%%%%%%%%%%%%%%%%%%%%%%%%%%%%%%%%%%%%%%%%%%%%%%%%%%%%%%
\section{Introduction}

\LaTeX{} provides a mechanism to structure a large document (such as a book)
into a main file and several child files (containing the chapters)
using the |\include| command.
This mechanism is beneficial for documents
which span hundreds of pages in order to
make the source file(s) more manageable.
Moreover, compilation can be restricted to
selected child files by means of the |\includeonly| command.
The latter feature can be used to reduce the compilation time while editing
(this was significantly more useful in the earlier days of \LaTeX{})
or to generate a smaller document which is easier to navigate.
Another application of |\includeonly| is to generate
documents consisting of selected parts of the complete document.

However, there are a few drawbacks of the plain |\include| mechanism:
\begin{itemize}
\item
The child files cannot be compiled on their own,
they can only be compiled via the main file.
A naive editing environment
(such as a text editor with an option
to have the current file processed by \LaTeX)
may require one to switch to the main file before compiling;
attempting to compile the child file produces errors.
\item
The main file must be modified (each time)
to adjust the |\includeonly| command
to the present needs. This easily leaves the main file in a messy state.
\item
The generated document will always carry the filename
of the main document. This is inconvenient if
several child files are to be compiled and
to be kept for distribution.
\end{itemize}

The present package provides a simple interface
to make child files individually compilable by \LaTeX{}.
Compiling a child file then has the same effect as compiling
the main file with an |\includeonly| command
to select the appropriate child.
Moreover the generated document will carry the name of the child
rather than the main file.
This resolves all three above issues.

This feature is meant to make the editing of books,
thesis documents and lecture notes somewhat more convenient.
However, the package can also be used efficiently for
composing a series of documents (such as exercise sheets)
which are typically distributed individually.
It then assists the author in generating the individual documents
(potentially in different versions)
as well as a document containing the collected series.
Another application is in developing style files
or other kinds of included material
where compilation of the style file could redirect
to a sample or test file.

%%%%%%%%%%%%%%%%%%%%%%%%%%%%%%%%%%%%%%%%%%%%%%%%%%%%%%%%%%%%%%%%%%%%%%%%%%%%%%%%
%%%%%%%%%%%%%%%%%%%%%%%%%%%%%%%%%%%%%%%%%%%%%%%%%%%%%%%%%%%%%%%%%%%%%%%%%%%%%%%%
\section{Usage}

First of all, the package \textsf{childdoc} is \emph{not} a standard
\LaTeXe{} |.sty| style file! Therefore it needs to be invoked in
a non-standard way.

%%%%%%%%%%%%%%%%%%%%%%%%%%%%%%%%%%%%%%%%%%%%%%%%%%%%%%%%%%%%%%%%%%%%%%%%%%%%%%%%
\subsection{Included Files}
\label{sec:include}

%%%%%%%%%%%%%%%%%%%%%%%%%%%%%%%%%%%%%%%%
\DescribeMacro{\childdocmain}
To use the package, add the commands
\begin{center}
\begin{tabular}{l}
|\input{childdoc.def}|\\
|\childdocmain{}|\\
\end{tabular}
\end{center}
at the very top of the main \LaTeX{} file,
in particular \emph{before} the |\documentclass| statement!
The argument of |\childdocmain| should be left empty
(but it must be present).

%%%%%%%%%%%%%%%%%%%%%%%%%%%%%%%%%%%%%%%%
\DescribeMacro{\childdocof}
Furthermore, add the commands
\begin{center}
\begin{tabular}{l}
|\input{childdoc.def}|\\
|\childdocof{|\textit{main}|}|\\
\end{tabular}
\end{center}
at the top of every child file \textit{child}
which is included by |\include{|\textit{child}|}|
from within the main file
(or at least for those files to be compiled individually).
The argument \textit{main} must be the filename of the main file.

There are a couple of
considerations in setting up the main and child documents:

%%%%%%%%%%%%%%%%%%%%%%%%%%%%%%%%%%%%%%%%
\paragraph{Restrictions.}

Please note the following restrictions:
\begin{itemize}
\item
|\childdocmain| must be called with one argument \textit{main}
to ensure compatibility with earlier version of the package.
It must either be empty (|\childdocmain{}|)
or precisely match the filename of the main file in which it is specified.
See \secref{sec:detection} for further information.
\item
The filename \textit{main} must be specified without the |.tex| extension.
\item
The filename \textit{main} is case sensitive
(even in case-insensitive file systems)
due to internal string comparison.
\item
The argument \textit{main} should be fully expanded, it cannot be a macro.
\item
Subdirectories and special characters should be avoided in filenames.
\item
The command |\childdocmain{|\textit{main}|}| must be followed by a whitespace.
It should not be followed immediately by another command
or by a comment mark `|%|'.
This is because the \TeX{} parser reads the token immediately following
the argument of |\childdocmain| and puts it
at the beginning of every child section;
however, a white\-space is ignored.
\end{itemize}

%%%%%%%%%%%%%%%%%%%%%%%%%%%%%%%%%%%%%%%%
\paragraph{Content of Main File.}

It is advisable to place all content in the child files included by |\include|.
Any output contained in the main file will appear in all child documents
unless suppressed manually;
it cannot be suppressed automatically by the |\includeonly| directive
and thus should normally be avoided.
A method to include some content in the main file
by means of conditional processing is described in \secref{sec:conditional}.

%%%%%%%%%%%%%%%%%%%%%%%%%%%%%%%%%%%%%%%%
\paragraph{Page Numbering.}

When only a part of the document is compiled,
the appropriate numbering of pages
(as well as other status parameters)
is determined from the |.aux| files.
The latter contain information from previous passes.
However this information needs to propagate through
all intermediate child documents.
Therefore the page numbering in child documents may well
be inconsistent until the complete document is compiled at least once.

A useful (if unconventional) way to always ensure a consistent
page numbering is to restart the numbering in each child document
and denote the pages by `\textit{child}|.|\textit{page}'
where \textit{child} represents the chapter/section number of the child file.
This can be achieved by the command
|\numberwithin{page}{|\textit{child}|}|
of the \textsf{amsmath} package
where \textit{child} can be |chapter| or |section|
depending on the chosen structuring.
Alternatively, one can modify the macro |\thepage| appropriately
and reset the counter |page| at the start of each child file.

%%%%%%%%%%%%%%%%%%%%%%%%%%%%%%%%%%%%%%%%%%%%%%%%%%%%%%%%%%%%%%%%%%%%%%%%%%%%%%%%
\subsection{Conditional Processing}
\label{sec:conditional}

The package provides a mechanism to compile different versions
of a document. To customise the versions further some conditional processing
can come in handy to distinguish which version is being compiled.
The package provides two macros to describe the compilation context:

%%%%%%%%%%%%%%%%%%%%%%%%%%%%%%%%%%%%%%%%
\DescribeMacro{\ifchilddoc}
The conditional |\ifchilddoc| distinguishes between the compilation of
child documents and the main document:
%
\begin{center}
|\ifchilddoc |\textit{child-code}| |[|\||else |\textit{main-code}]| \||fi|
\end{center}

%%%%%%%%%%%%%%%%%%%%%%%%%%%%%%%%%%%%%%%%
\DescribeMacro{\childdocname}
\DescribeMacro{\childdocjob}
The macro |\childdocname| contains the filename (without extension)
of the main or child file being processed.
Note that |\childdocjob| will always contain the name of the main file.

%%%%%%%%%%%%%%%%%%%%%%%%%%%%%%%%%%%%%%%%
\paragraph{Title Page.}

Conditional processing can be used to include a title or banner page
in the main document when proper precautions are taken.
Importantly, the code in the main file should ensure that the page counter
(as well as other status parameters which are stored in the |.aux| files)
takes the same value after the conditional processing.
Otherwise the page numbers may take divergent values
depending on which part is compiled.

For example, a title page could be declared by:
%
\begin{center}
\begin{tabular}{l}
|\ifchilddoc\||else|\\
|\addtocounter{page}{-1}|\\
\textit{code for title page}\\
|\newpage|\\
|\||fi|
\end{tabular}
\end{center}
%
A banner page for the child documents can be generated by:
%
\begin{center}
\begin{tabular}{l}
|\ifchilddoc|\\
|\addtocounter{page}{-1}|\\
\textit{code for banner page}\\
|\newpage|\\
|\||fi|
\end{tabular}
\end{center}
%
Here one could write a message such as:
\begin{center}
|This is the part \childdocname{} of \childdocjob{}.|
\end{center}

%%%%%%%%%%%%%%%%%%%%%%%%%%%%%%%%%%%%%%%%%%%%%%%%%%%%%%%%%%%%%%%%%%%%%%%%%%%%%%%%
\subsection{Flags}
\label{sec:flags}

The package makes it easy to generate different versions
of the main or child documents.
To this end compilation flags can be defined
and assigned different default values.
They will be particularly useful in conjunction
with the forwarding mechanism described in \secref{sec:forward}.

For example, it may be useful to have a flag |\version|
which can be set to |draft| or |final|.
The document source will contain some conditional code
depending on the value of |\version|.
Suppose further, the flag should default to |final| for the main file
and to |draft| for child files
which is a natural assignment for editing the document.
This is achieved by placing the following code
in the preamble of the main document
(below the |\childdocmain| directive):
%
\begin{center}
\begin{tabular}{l}
|\ifchilddoc|\\
|\providecommand{\version}{draft}|\\
|\||else|\\
|\providecommand{\version}{final}|\\
|\||fi|
\end{tabular}
\end{center}
%
The definition by |\providecommand| makes sure
that previous definitions are not overwritten.
Further statements |\providecommand{\version}{...}|
can thus be added before the above code to override it.

For the main file, one might add a line
(between |\childdocmain| and the above block)
%
\begin{center}
|%\ifchilddoc\||else\providecommand{\version}{draft}\||fi|
\end{center}
%
which can be uncommented to produce a draft version.
Likewise one can add a line to the very top of a child file
(above the |\childdocof{|\textit{main}|}| directive)
%
\begin{center}
|%\providecommand{\version}{final}|
\end{center}
%
which can be uncommented to produce the final version of this child document.

%%%%%%%%%%%%%%%%%%%%%%%%%%%%%%%%%%%%%%%%%%%%%%%%%%%%%%%%%%%%%%%%%%%%%%%%%%%%%%%%
\subsection{Forwarding}
\label{sec:forward}

Different versions of the main or child documents
using compilation flags as described in \secref{sec:flags}
can be (permanently) stored in different files
for convenient compilation, viewing and distribution.
To this end, the package defines a command
to pass on compilation to a different file:

%%%%%%%%%%%%%%%%%%%%%%%%%%%%%%%%%%%%%%%%
\DescribeMacro{\childdocforward}
The command |\childdocforward| redirects processing to
another source file:
%
\begin{center}
\begin{tabular}{l}
|\input{childdoc.def}|\\
|\childdocforward[|\textit{main}|]{|\textit{dest}|}|\\
\end{tabular}
\end{center}
%
The argument \textit{dest} is the destination file
(without extension).
It should be the main file or one of the child files.
Note that further \textsf{childdoc} directives
such as |\childdocof| and |\childdocforward|
in the indicated file will be processed in this form.
The optional argument \textit{main}
passes on directly to the main file \textit{main}
while pretending to compile the child \textit{dest}.
This form behaves as if \textit{dest}
issues |\childdocof{|\textit{main}|}| right away,
and no further \textsf{childdoc} directives will be processed.

%%%%%%%%%%%%%%%%%%%%%%%%%%%%%%%%%%%%%%%%
\DescribeMacro{\...prefix}
In the alternative form |\childdocforwardprefix|,
%
\begin{center}
\begin{tabular}{l}
|\input{childdoc.def}|\\
|\childdocforwardprefix[|\textit{main}|]{|\textit{prefix}|}{|\textit{dest}|}|
\end{tabular}
\end{center}
%
the destination file is determined by a pattern
depending on the current file:
To make this work, the current file must be called
`{\textit{prefix}\hspace{0.2em}\textit{suffix}}'
with \textit{prefix} matching precisely the argument.
Processing is then passed on to the file
`{\textit{dest}\hspace{0.2em}\textit{suffix}}'.
Surely, the same effect is achieved by
directly specifying the
argument `{\textit{dest}\hspace{0.2em}\textit{suffix}}'
in the first form.
However, that requires to set up a different file
for each child. With the alternative form of the command
all these files can have exactly the same content
which simplifies setting them up and maintaining them.

For example, the following file |draft.tex|
with a compilation flag |\version| as described in \secref{sec:flags}
compiles the main document as a draft:
%
\begin{center}
\begin{tabular}{l}
|\def\version{draft}|\\
|\input{childdoc.def}|\\
|\childdocforward{|\textit{main}|}|
\end{tabular}
\end{center}
%
Likewise, the following files |final|\textit{nn}|.tex|
compile the final version of the child document
|child|\textit{nn}|.tex|:
%
\begin{center}
\begin{tabular}{l}
|\def\version{final}|\\
|\input{childdoc.def}|\\
|\childdocforwardprefix{final}{child}|
\end{tabular}
\end{center}
%

Note that when several versions of a main file and/or of each child file
are to be generated, it may be convenient to set up a |Makefile| or
shell script to automatise the process.

%%%%%%%%%%%%%%%%%%%%%%%%%%%%%%%%%%%%%%%%%%%%%%%%%%%%%%%%%%%%%%%%%%%%%%%%%%%%%%%%
\subsection{Command Line Processing}
\label{sec:commandline}

The effect of redirection files can also be achieved by invoking
the \LaTeX{} compiler with a more elaborate command line.
Most conveniently this should be done as part
of a shell script or a |Makefile|.

When using \textsf{childdoc} in the main file, the following
command lines effectively perform a redirection
(note that depending on the shell being used,
backslashes may have to be doubled: `|\|' $\to$ `|\\|'):
%
\begin{center}
|... -jobname "|\textit{target}|" |\\|"|[\textit{flags}]%
|\input{childdoc.def}\childdocforward[|\textit{main}|]{|\textit{dest}|}"|
\end{center}
%
Here \textit{target} is the name of the output file,
\textit{main} is the name of the main file
and \textit{dest} is the name of the main or child file to be processed
(all filenames without extensions).
The optional argument \textit{main} can be omitted
if \textit{main} matches \textit{dest}.
Optionally, compilation \textit{flags} can be defined via |\def| commands.
This command line makes the \TeX{} engine believe
it is compiling the file \textit{target}
whose content is specified as the latter parameter.
The provided code then forwards the processing to
\textit{main} or \textit{dest} as described in \secref{sec:forward}.

%%%%%%%%%%%%%%%%%%%%%%%%%%%%%%%%%%%%%%%%%%%%%%%%%%%%%%%%%%%%%%%%%%%%%%%%%%%%%%%%
\subsection{Include by Input}
\label{sec:input}

Including child documents by |\include| has some restrictions by design.
Most notably, the content of a child document always occupies
its own set of pages; pages cannot be shared between child documents.
Usually, this behaviour makes perfect sense
because each child document contain an essential part of the document.
However, in some situations it may be desirable to compose
a document from a collection of parts
without having mandatory page breaks between then.
For this case, the package
provides a mechanism to include parts
by |\input| which can also be processed individually.
However, by construction this mechanism
requires manual handling of the content to be output.

%%%%%%%%%%%%%%%%%%%%%%%%%%%%%%%%%%%%%%%%
\DescribeMacro{\ifchilddocmanual}
The main file should be prepared as usual, see \secref{sec:include}.
However, the document body must make a distinction
between processing of an individual part and of the main document, e.g.:
%
\begin{center}
\begin{tabular}{l}
|\ifchilddocmanual|\\
|\input{\childdocname}|\\
|\||else|\\
\textit{document body with }|\input{|\textit{part}|}|\\
|\||fi|
\end{tabular}
\end{center}
%
The conditional |\ifchilddocmanual| is true whenever
a part to be included by |\input| is being compiled,
and the name of the part is stored in |\childdocname|.

%%%%%%%%%%%%%%%%%%%%%%%%%%%%%%%%%%%%%%%%
\DescribeMacro{\childdocby}
Each part to be included by |\input| should start with:
%
\begin{center}
\begin{tabular}{l}
|\input{childdoc.def}|\\
|\childdocby{|\textit{main}|}|\\
\end{tabular}
\end{center}
%
The directive |\childdocby| is similar to |\childdocof|
described in \secref{sec:include},
but the subsequent selection of content must be done manually.
To that end, both |\ifchilddoc| and |\ifchilddocmanual|
will be true upon processing of a part,
and the name of the part is stored in |\childdocname|.
Note that |\jobname| will be set to the filename of the current part
so that each part receives an individual |.aux| file
that does not interfere with the |.aux| file(s) of the main document.
This behaviour can be altered by the alternative form
|\childdocby[*]{|\textit{main}|}| (with a non-empty optional argument)
which uses the |.aux| file of the main document
by setting |\jobname| to \textit{main}.

%%%%%%%%%%%%%%%%%%%%%%%%%%%%%%%%%%%%%%%%%%%%%%%%%%%%%%%%%%%%%%%%%%%%%%%%%%%%%%%%
\subsection{Driver Development}
\label{sec:driver}

The \textsf{childdoc} mechanism can also be use for the development
of definition files such as \LaTeX{} styles or classes.
This case differs from the above setup with multiple parts
included by |\include| in that no |\includeonly| should be invoked.
This can be achieved by starting the include file
(before |\ProvidesPackage|) with:
%
\begin{center}
\begin{tabular}{l}
|\input{childdoc.def}|\\
|\childdocforward{|\textit{main}|}|\\
\end{tabular}
\end{center}
%
or alternatively with:
%
\begin{center}
\begin{tabular}{l}
|\input{childdoc.def}|\\
|\childdocby{|\textit{main}|}|\\
\end{tabular}
\end{center}
%
Both forms have slightly different effects as described above.
The main file is prepared as usual, see \secref{sec:include}.

%%%%%%%%%%%%%%%%%%%%%%%%%%%%%%%%%%%%%%%%%%%%%%%%%%%%%%%%%%%%%%%%%%%%%%%%%%%%%%%%
\subsection{Legacy Detection}
\label{sec:detection}

The directive |\childdocmain| in the main file can detect
whether the complete document or merely a child is to be compiled
even without using the directive |\childdocof|.
This method is deprecated because it is less robust
and there is no compelling reason to use it;
it is merely provided for backward compatibility
and it may be removed in future versions.

If the detection mechanism is to be used,
it is mandatory to correctly specify
the filename of the main file as the argument of |\childdocmain|:
%
\begin{center}
\begin{tabular}{l}
|\input{childdoc.def}|\\
|\childdocmain{|\textit{main}|}|\\
\end{tabular}
\end{center}
%
If |\jobname| does not match the argument \textit{main} of |\childdocmain|,
it is assumed that |\jobname| points to the child file to be compiled.
When using |\childdocmain| with the main file specified as argument,
it suffices to start a child file
with just |\input{|\textit{main}|}|
without loading of the package and using |\childdocof|.
If instead all processing is done
with the appropriate \textsf{childdoc} directives,
the argument of \textit{main} of |\childdocmain| can be empty.

An alternative version of the command line processing described
in \secref{sec:commandline} using the detection mechanism reads:
%
\begin{center}
|... -jobname "|\textit{target}|" "|[\textit{flags}]%
[|\def\jobname{|\textit{dest}|}|]|\input{|\textit{main}|}"|
\end{center}

%%%%%%%%%%%%%%%%%%%%%%%%%%%%%%%%%%%%%%%%%%%%%%%%%%%%%%%%%%%%%%%%%%%%%%%%%%%%%%%%
\subsection{Manual Code}
\label{sec:manual}

In case one cannot be certain whether the definitions file |childdoc.def|
is installed on the target \TeX{} distribution
and one prefers not to ship it,
it is conceivable to paste a few relevant commands into the sources.

To that end, drop all statements |\input{childdoc.def}|
and perform the replacements as outlined below.
Instead of |\childdocmain{|\textit{main}|}| add the following code
to the top of the main file:
%
\begin{center}
\begin{tabular}{l}
|\||ifdefined\childdocname\endinput\||fi\newif\ifchilddoc|\\
|\edef\childdocname{\scantokens\expandafter{\jobname\noexpand}}|\\
|\def\childdocmain{|\textit{main}|}\||ifx\childdocmain\childdocname\||else|\\
|\childdoctrue\includeonly{\childdocname}\let\jobname\childdocmain\||fi|\\
\end{tabular}
\end{center}
%
Instead of |\childdocof{|\textit{main}|}| just include the main file
at the top of each child file:
%
\begin{center}
|\input{|\textit{main}|}|
\end{center}
%
A simple redirection |\childdocforward{|\textit{dest}|}| is achieved by:
%
\begin{center}
|\def\jobname{|\textit{dest}|}\input{\jobname}|
\end{center}
%
The redirection with prefix
|\childdocforwardprefix[|\textit{prefix}|]{|\textit{dest}|}|
is accomplished by:
%
\begin{center}
\begin{tabular}{l}
|{\edef\jobname{\scantokens\expandafter{\jobname\noexpand}}|\\
|\def\redirectjob |\textit{prefix}|#1~~~{\gdef\jobname{|\textit{dest}|#1}}|\\
|\expandafter\redirectjob\jobname~~~}\input{\jobname}|
\end{tabular}
\end{center}

In an alternative approach,
child documents can be compiled by a specific command line
without additional code or specific definitions:
%
\begin{center}
|... -jobname "|\textit{target}|" "|[\textit{flags}]%
|\includeonly{|\textit{dest}|}\input{|\textit{main}|}"|
\end{center}
%

%%%%%%%%%%%%%%%%%%%%%%%%%%%%%%%%%%%%%%%%%%%%%%%%%%%%%%%%%%%%%%%%%%%%%%%%%%%%%%%%
%%%%%%%%%%%%%%%%%%%%%%%%%%%%%%%%%%%%%%%%%%%%%%%%%%%%%%%%%%%%%%%%%%%%%%%%%%%%%%%%
\section{Information}

%%%%%%%%%%%%%%%%%%%%%%%%%%%%%%%%%%%%%%%%%%%%%%%%%%%%%%%%%%%%%%%%%%%%%%%%%%%%%%%%
\subsection{Copyright}

Copyright \copyright{} 2017--2018 Niklas Beisert

This work may be distributed and/or modified under the
conditions of the \LaTeX{} Project Public License, either version 1.3
of this license or (at your option) any later version.
The latest version of this license is in
  \url{http://www.latex-project.org/lppl.txt}
and version 1.3 or later is part of all distributions of \LaTeX{}
version 2005/12/01 or later.

This work has the LPPL maintenance status `maintained'.

The Current Maintainer of this work is Niklas Beisert.

This work consists of the files |README.txt|, |childdoc.ins| and |childdoc.dtx|
as well as the derived files |childdoc.def|, |cdocsamp.tex|
with |cdocsch1.tex|, |cdocsch2.tex|, |cdocspt3.tex|, |cdocspt4.tex|,
|cdocsdrf.tex|, |cdocsfn1.tex|, |cdocsfn2.tex|
as well as |childdoc.pdf|.

%%%%%%%%%%%%%%%%%%%%%%%%%%%%%%%%%%%%%%%%%%%%%%%%%%%%%%%%%%%%%%%%%%%%%%%%%%%%%%%%
\subsection{Files and Installation}

The package consists of the files:
%
\begin{center}
\begin{tabular}{ll}
    |README.txt|   & readme file \\
    |childdoc.ins| & installation file \\
    |childdoc.dtx| & source file \\
    |childdoc.def| & definition file \\
    |cdocsamp.tex| & sample main file \\
    |cdocsch1.tex| & sample include file \\
    |cdocsch2.tex| & sample include file \\
    |cdocspt3.tex| & sample part file \\
    |cdocspt4.tex| & sample part file \\
    |cdocsdrf.tex| & sample redirection file \\
    |cdocsfn1.tex| & sample redirection file \\
    |cdocsfn2.tex| & sample redirection file \\
    |childdoc.pdf| & manual
\end{tabular}
\end{center}
%
The distribution consists of the files
|README.txt|, |childdoc.ins| and |childdoc.dtx|.
%
\begin{itemize}
\item
Run (pdf)\LaTeX{} on |childdoc.dtx|
to compile the manual |childdoc.pdf| (this file).
\item
Run \LaTeX{} on |childdoc.ins| to create the definitions file |childdoc.def|
and the sample |cdocsamp.tex| with include files
|cdocsch1.tex|, |cdocsch2.tex|, |cdocspt3.tex|, |cdocspt4.tex|,
|cdocsdrf.tex|, |cdocsfn1.tex|, |cdocsfn2.tex|.
Then copy the file |childdoc.def| to an appropriate directory of your \LaTeX{}
distribution, e.g.\ \textit{texmf-root}|/tex/latex/childdoc|.
\end{itemize}

%%%%%%%%%%%%%%%%%%%%%%%%%%%%%%%%%%%%%%%%%%%%%%%%%%%%%%%%%%%%%%%%%%%%%%%%%%%%%%%%
\subsection{Related CTAN Packages}

There are several other packages which offer a similar functionality:
%
\begin{itemize}
\item
The packages
\href{http://ctan.org/pkg/docmute}{\textsf{docmute}},
\href{http://ctan.org/pkg/includex}{\textsf{includex}} and
\href{http://ctan.org/pkg/standalone}{\textsf{standalone}}
provide commands to include only the document body of
a child file thus allowing both files to be compiled individually.
\item
The packages \href{http://ctan.org/pkg/subdocs}{\textsf{subdocs}}
and \href{http://ctan.org/pkg/subfiles}{\textsf{subfiles}}
provide structures in which the main and child documents can be
encapsulated and allowing them to be compiled individually.
The inclusion mechanism is different from the conventional |\include|.
\item
The package \href{http://ctan.org/pkg/combine}{\textsf{combine}}
is an elaborate solution to combine several documents into one.
\end{itemize}
%
See also the CTAN topic \href{http://ctan.org/topic/subdocs}{\textsf{subdocs}}
for further related packages.
The present package differs from the above solutions in that
a document structure constructed with the conventional |\include| mechanism
just needs two extra commands at the top of every file
such that all constituent files can be compiled individually.

%%%%%%%%%%%%%%%%%%%%%%%%%%%%%%%%%%%%%%%%%%%%%%%%%%%%%%%%%%%%%%%%%%%%%%%%%%%%%%%%
%\subsection{Feature Suggestions}
%
%The following is a list of features which may be useful for future
%versions of this package:
%%
%\begin{itemize}
%\item
%\ldots
%\end{itemize}

%%%%%%%%%%%%%%%%%%%%%%%%%%%%%%%%%%%%%%%%%%%%%%%%%%%%%%%%%%%%%%%%%%%%%%%%%%%%%%%%
\subsection{Revision History}

%%%%%%%%%%%%%%%%%%%%%%%%%%%%%%%%%%%%%%%%
\paragraph{v2.0:} 2018/12/30

\begin{itemize}
\item
immediate forward processing
\item
added |\childdocby| mechanism
\item
manual restructured
\end{itemize}

%%%%%%%%%%%%%%%%%%%%%%%%%%%%%%%%%%%%%%%%
\paragraph{v1.6:} 2018/01/17

\begin{itemize}
\item
application for development of include files
\item
corrections to manual
\end{itemize}

%%%%%%%%%%%%%%%%%%%%%%%%%%%%%%%%%%%%%%%%
\paragraph{v1.5:} 2017/05/21

\begin{itemize}
\item
more complete structuring introduced
\item
|\childdocof| introduced
\item
|\childdoc| renamed to |\childdocmain|
\item
|\childredirect| renamed to |\childdocforward| and |\childdocforwardprefix|
and functionality expanded
\end{itemize}

%%%%%%%%%%%%%%%%%%%%%%%%%%%%%%%%%%%%%%%%
\paragraph{v1.0:} 2017/04/27

\begin{itemize}
\item
manual and install package
\item
first version published on CTAN
\end{itemize}

%%%%%%%%%%%%%%%%%%%%%%%%%%%%%%%%%%%%%%%%
\paragraph{v0.6:} 2017/04/26

\begin{itemize}
\item
redirection mechanism added
\end{itemize}

%%%%%%%%%%%%%%%%%%%%%%%%%%%%%%%%%%%%%%%%
\paragraph{v0.5:} 2017/04/26

\begin{itemize}
\item
functionality in definition file
\end{itemize}


%%%%%%%%%%%%%%%%%%%%%%%%%%%%%%%%%%%%%%%%%%%%%%%%%%%%%%%%%%%%%%%%%%%%%%%%%%%%%%%%
%%%%%%%%%%%%%%%%%%%%%%%%%%%%%%%%%%%%%%%%%%%%%%%%%%%%%%%%%%%%%%%%%%%%%%%%%%%%%%%%
%%%%%%%%%%%%%%%%%%%%%%%%%%%%%%%%%%%%%%%%%%%%%%%%%%%%%%%%%%%%%%%%%%%%%%%%%%%%%%%%
\appendix

\settowidth\MacroIndent{\rmfamily\scriptsize 000\ }

 \DocInput{childdoc.dtx}

\end{document}
%</driver>
% \fi
%
% %%%%%%%%%%%%%%%%%%%%%%%%%%%%%%%%%%%%%%%%%%%%%%%%%%%%%%%%%%%%%%%%%%%%%%%%%%%%%%
% %%%%%%%%%%%%%%%%%%%%%%%%%%%%%%%%%%%%%%%%%%%%%%%%%%%%%%%%%%%%%%%%%%%%%%%%%%%%%%
% \section{Sample}
%\iffalse
%<*samplemain>
%\fi
%
% The following presents a sample document
% with two chapters, two parts, a title page,
% a compile flag as well as three forwarding files to set the flag.
% It consists of eight |.tex| files:
% \begin{center}
% \begin{tabular}{ll}
% |cdocsamp.tex|&main file\\
% |cdocsch1.tex|&include file for chapter 1\\
% |cdocsch2.tex|&include file for chapter 2\\
% |cdocspt3.tex|&include file for part 3\\
% |cdocspt4.tex|&include file for part 4\\
% |cdocsdrf.tex|&forwarding file for main file in draft mode\\
% |cdocsfi1.tex|&forwarding file for final version of chapter 1\\
% |cdocsfi2.tex|&forwarding file for final version of chapter 2\\
% \end{tabular}
% \end{center}
% Each of the eight files can be compiled directly by the \LaTeX{} compiler.
%
% %%%%%%%%%%%%%%%%%%%%%%%%%%%%%%%%%%%%%%
% \paragraph{Main File.}
%
% The main file is called |cdocsamp.tex|.
%
% Load the \textsf{childdoc} definitions and
% declare the filename for the main document:
%    \begin{macrocode}
\input{childdoc.def}
\childdocmain{}
%    \end{macrocode}

% Optional override for |\version| flag:
%    \begin{macrocode}
%%\ifchilddoc\else\providecommand{\version}{draft}\fi
%    \end{macrocode}

% Define the default values for the |\version| flag
% (|final| for the main file and |draft| for childs):
%    \begin{macrocode}
\ifchilddoc
\providecommand{\version}{draft}
\else
\providecommand{\version}{final}
\fi
%    \end{macrocode}

% Load the standard document class:
%    \begin{macrocode}
\documentclass[12pt]{article}
%    \end{macrocode}

% Start the document body:
%    \begin{macrocode}
\begin{document}
%    \end{macrocode}

% Declare a title page.
% Print title, part of document being processed and version flag:
%    \begin{macrocode}
\addtocounter{page}{-1}
\begin{center}
{\LARGE\bfseries{}childdoc example\par}
\vspace{1cm}
\ifchilddoc
\ifchilddocmanual part\else chapter\fi:
`\childdocname' of `\childdocjob'\par
\else
main document: `\childdocjob'\par
\fi
version: \version\par
\end{center}
\newpage
%    \end{macrocode}

% Manually include selected file,
% otherwise process as usual:
%    \begin{macrocode}
\ifchilddocmanual
\section*{part `\childdocname'}
\input{\childdocname}
\else
%    \end{macrocode}

% Include the two chapters:
%    \begin{macrocode}
\include{cdocsch1}
\include{cdocsch2}
%    \end{macrocode}

% Include the two parts unless only chapters should be displayed:
%    \begin{macrocode}
\ifchilddoc\else
\section{part three}
\input{cdocspt3}
\section{part four}
\input{cdocspt4}
\fi
%    \end{macrocode}

% Process as usual until here:
%    \begin{macrocode}
\fi
%    \end{macrocode}

% End of document body:
%    \begin{macrocode}
\end{document}
%    \end{macrocode}
%\iffalse
%</samplemain>
%\fi
%
% %%%%%%%%%%%%%%%%%%%%%%%%%%%%%%%%%%%%%%
% \paragraph{Chapter Include Files.}
%
% The include files are called |cdocsch1.tex| and |cdocsch2.tex|.
%
%\iffalse
%<*samplechap1|samplechap2>
%\fi

% Optional override for |\version| flag:
%    \begin{macrocode}
%%\providecommand{\version}{final}
%    \end{macrocode}

% Include the main document:
%    \begin{macrocode}
\input{childdoc.def}
\childdocof{cdocsamp}
%    \end{macrocode}

%\iffalse
%</samplechap1|samplechap2>
%\fi
%
%\iffalse
%<*samplechap1>
%\fi
% Some text for chapter 1:
%    \begin{macrocode}
\section{one}
some text in chapter one
%    \end{macrocode}

%\iffalse
%</samplechap1>
%\fi
% Some text for chapter 2:
%\iffalse
%<*samplechap2>
%\fi
%    \begin{macrocode}
\section{two}
more text in chapter two
%    \end{macrocode}

%\iffalse
%</samplechap2>
%\fi
%
% %%%%%%%%%%%%%%%%%%%%%%%%%%%%%%%%%%%%%%
% \paragraph{Part Include Files.}
%
% The include files are called |cdocspt3.tex| and |cdocspt4.tex|.
%
%\iffalse
%<*samplepart3|samplepart4>
%\fi

% Optional override for |\version| flag:
%    \begin{macrocode}
%%\providecommand{\version}{final}
%    \end{macrocode}

% Include the main document:
%    \begin{macrocode}
\input{childdoc.def}
\childdocby{cdocsamp}
%    \end{macrocode}

%\iffalse
%</samplepart3|samplepart4>
%\fi
%
%\iffalse
%<*samplepart3>
%\fi
% Some text for part 3:
%    \begin{macrocode}
some text in part three
%    \end{macrocode}

%\iffalse
%</samplepart3>
%\fi
% Some text for part 4:
%\iffalse
%<*samplepart4>
%\fi
%    \begin{macrocode}
more text in part four
%    \end{macrocode}

%\iffalse
%</samplepart4>
%\fi
%
% %%%%%%%%%%%%%%%%%%%%%%%%%%%%%%%%%%%%%%
% \paragraph{Forwarding for a Complete Draft.}
%
% The following forwarding file |cdocsdrf.tex|
% compiles the main document in draft mode:
%\iffalse
%<*sampledraft>
%\fi
%    \begin{macrocode}
\def\version{draft}
\input{childdoc.def}
\childdocforward{cdocsamp}
%    \end{macrocode}

%\iffalse
%</sampledraft>
%\fi
%
% %%%%%%%%%%%%%%%%%%%%%%%%%%%%%%%%%%%%%%
% \paragraph{Forwarding for Final Version of the Chapters.}
%
% The following forwarding files |cdocsfn1.tex| and |cdocsfn2.tex|
% (with identical content)
% compile the final versions of the child documents
% |cdocsch1.tex| and |cdocsch2.tex|, respectively:
%\iffalse
%<*samplefinal>
%\fi
%    \begin{macrocode}
\def\version{final}
\input{childdoc.def}
\childdocforwardprefix[cdocsamp]{cdocsfn}{cdocsch}
%    \end{macrocode}

%\iffalse
%</samplefinal>
%\fi
%
% %%%%%%%%%%%%%%%%%%%%%%%%%%%%%%%%%%%%%%
% \paragraph{Command Line Processing.}
%
% The following three command lines generate the output files
% |cdocscld|, |cdocscl1| and |cdocscl2|
% which should be identical to
% |cdocsdrf|, |cdocsch1| and |cdocsfn2|, respectively:
% \begin{center}
% \begin{tabular}{l}
% |latex -jobname cdocscld \|\\
% |  "\def\version{draft}\input{childdoc.def}\childdocforward{cdocsamp}"|\\
% |latex -jobname cdocscl1 \|\\
% |  "\input{childdoc.def}\childdocforward[cdocsamp]{cdocsch1}"|\\
% |latex -jobname cdocscl2 \|\\
% |  "\def\version{final}\input{childdoc.def}\childdocforward{cdocsch2}"|
% \end{tabular}
% \end{center}
% Note that the trailing backslash on each first line
% merely continues the input to the second line
% (for convenient cut ant paste).
% Furthermore, the command |latex| can be replaced by any
% of its alternative versions such as |pdflatex|.
%
% %%%%%%%%%%%%%%%%%%%%%%%%%%%%%%%%%%%%%%%%%%%%%%%%%%%%%%%%%%%%%%%%%%%%%%%%%%%%%%
% %%%%%%%%%%%%%%%%%%%%%%%%%%%%%%%%%%%%%%%%%%%%%%%%%%%%%%%%%%%%%%%%%%%%%%%%%%%%%%
% \section{Implementation}
%\iffalse
%<*package>
%\fi
%
% This section describes the definitions file |childdoc.def|.

% The definitions cannot be loaded using |\usepackage| or |\RequirePackage|
% which has a mechanism to prevent loading a style file more than once.
% When loading the definitions by means of |\input|
% multiple instances have to be prevented manually:
%\iffalse
%This code needs to be before the `\ProvidesFile' directive
%which is defined at the beginning of this file.
%Therefore it is also placed there and commented out here.
%</package>
%<*discard>
%\fi
%    \begin{macrocode}
\ifdefined\childdocmain\endinput\fi
%    \end{macrocode}
%\iffalse
%</discard>
%<*package>
%\fi
%
% \macro{\ifchilddoc}
% \macro{\ifchilddocmanual}
% The conditional |\ifchilddoc| tells whether a
% child (true) or main (false) document is being compiled.
% The conditional |\ifchilddocmanual| tells whether
% the |\includeonly| mechanism is used (false) or
% the selection of child files must be performed manually (true).
% The definitions initialise to false:
%    \begin{macrocode}
\newif\ifchilddoc
\newif\ifchilddocmanual
%    \end{macrocode}

% \macro{\childdocname}
% \macro{\childdocjob}
% The macro |\childdocname| stores the name of the main document
% to be compiled. The macro |\childdocjob| stores the name of
% the document on which the \LaTeX{} compiler was originally invoked.
% The content of |\jobname| cannot be compared
% to filenames specified in the source due to different catcodes.
% The following code rescans |\jobname|, stores the result
% in |\childdocname| and saves a copy in |\childdocjob|:
%    \begin{macrocode}
\edef\childdocname{\scantokens\expandafter{\jobname\noexpand}}
\let\childdocjob\childdocname
%    \end{macrocode}

% \macro{\childdocdisable}
% The macro |\childdocdisable| prevents the main file
% from being processed more than once.
% At this stage, the main document command |\childdocmain|
% is assumed to be called once again where it should do nothing.
% Any subsequent call to it should prevent
% a secondary processing of the main document
% It overwrites the forwarding commands
% |\childdocof| and |\childdocforward|
% with empty macros to prevent further inclusions of the main document:
%    \begin{macrocode}
\newcommand{\childdocdisable}
{
  \renewcommand{\childdocmain}[1]{\renewcommand{\childdocmain}[1]{\endinput}}
  \renewcommand{\childdocof}[1]{}
  \renewcommand{\childdocby}[2][]{}
  \renewcommand{\childdocforward}[2][]{}
  \renewcommand{\childdocdisable}{}
}
%    \end{macrocode}

% \macro{\childdocmain}
% The macro |\childdocmain| is to be called at the top of the main file
% with nothing or the main filename (without extension) as argument.
% First, it breaks loops.
% If the argument is not empty and does not match |\childdocname|
% (which is set by the first inclusion of |childdoc.def|),
% |\ifchilddoc| is set to true, |\includeonly| is applied to the child file
% and |\jobname| is set to the main file
% (for proper handling of |.aux| files):
%    \begin{macrocode}
\newcommand{\childdocmain}[1]
{
  \childdocdisable\childdocmain{}
  \if?#1?\else
    \begingroup
      \def\childdoctmp{#1}
      \ifx\childdoctmp\childdocname
        \def\childdoctmp{}
      \else
        \def\childdoctmp
        {
          \childdoctrue
          \includeonly{\childdocname}
          \def\childdocjob{#1}
          \def\jobname{#1}
        }
      \fi
      \expandafter
    \endgroup
    \childdoctmp
  \fi
}
%    \end{macrocode}

% \macro{\childdocof}
% The command |\childdocof| redirects
% compilation to the main file |#1|.
%    \begin{macrocode}
\newcommand{\childdocof}[1]
{
  \childdocdisable
  \childdoctrue
  \includeonly{\childdocname}
  \def\jobname{#1}
  \def\childdocjob{#1}
  \input{#1}
}
%    \end{macrocode}

% \macro{\childdocby}
% The command |\childdocby| ....
%    \begin{macrocode}
\newcommand{\childdocby}[2][]
{
  \childdocdisable
  \childdoctrue
  \childdocmanualtrue
  \if?#1?\else
    \def\jobname{#2}
  \fi
  \def\childdocjob{#2}
  \input{#2}
  \endinput
}
%    \end{macrocode}

% \macro{\childdocforward}
% The command |\childdocforward| redirects
% compilation to the main file or
% (if the optional argument is given) a child file.
% Parameters are set as if the main file
% or a child file starting with |\childdocof| was compiled.
% Then compilation is handed over to the main file:
%    \begin{macrocode}
\newcommand{\childdocforward}[2][]
{
  \begingroup
    \if?#1?
      \def\childdoctmp
      {
        \def\childdocname{#2}
        \def\childdocjob{#2}
        \def\jobname{#2}
        \input{#2}
        \endinput
      }
    \else
      \def\childdoctmp
      {
        \childdocdisable
        \def\childdocname{#2}
        \childdoctrue
        \includeonly{#2}
        \def\childdocjob{#1}
        \def\jobname{#1}
        \input{#1}
        \endinput
      }
    \fi
    \expandafter
  \endgroup
  \childdoctmp
}
%    \end{macrocode}

% \macro{\childdocforwardprefix}
% The command |\childdocforwardprefix| redirects
% compilation to the main or a child file by means of a pattern.
% The prefix |#1| in the current filename is replaced by |#2|
% and the suffix of the current filename is kept
% (it is assumed that the filename does not contain the substring `|~~~|'
% which is used as a delimiter).
% Compilation is handed over to the new file by |\childdocforward|:
%    \begin{macrocode}
\newcommand{\childdocforwardprefix}[3][]
{
  \begingroup
    \def\childdocextract #2##1~~~{\def\childdoctmp{\childdocforward[#1]{#3##1}}}
    \expandafter\childdocextract\childdocname~~~
    \expandafter
  \endgroup
  \childdoctmp
}
%    \end{macrocode}

% \macro{\childdoc}
% The deprecated macro |\childdoc| is a legacy version of |\childdocmain|:
%    \begin{macrocode}
\newcommand{\childdoc}{\childdocmain}
%    \end{macrocode}

% \macro{\childdocredirect}
% The deprecated macro |\childdocredirect| is a legacy version
% of |\childdocforward| and |\childdocforwardprefix|:
%    \begin{macrocode}
\newcommand{\childdocredirect}[2][]
{
  \begingroup
    \if?#1?
      \def\childdoctmp{\childdocforward{#2}}
    \else
      \def\childdoctmp{\childdocforwardprefix{#1}{#2}}
    \fi
    \expandafter
  \endgroup
  \childdoctmp
}
%    \end{macrocode}

%\iffalse
%</package>
%\fi
%
\endinput
|\\
|\childdocforward[|\textit{main}|]{|\textit{dest}|}|\\
\end{tabular}
\end{center}
%
The argument \textit{dest} is the destination file
(without extension).
It should be the main file or one of the child files.
Note that further \textsf{childdoc} directives
such as |\childdocof| and |\childdocforward|
in the indicated file will be processed in this form.
The optional argument \textit{main}
passes on directly to the main file \textit{main}
while pretending to compile the child \textit{dest}.
This form behaves as if \textit{dest}
issues |\childdocof{|\textit{main}|}| right away,
and no further \textsf{childdoc} directives will be processed.

%%%%%%%%%%%%%%%%%%%%%%%%%%%%%%%%%%%%%%%%
\DescribeMacro{\...prefix}
In the alternative form |\childdocforwardprefix|,
%
\begin{center}
\begin{tabular}{l}
|% \iffalse
%
% childdoc.dtx Copyright (C) 2017-2018 Niklas Beisert
%
% This work may be distributed and/or modified under the
% conditions of the LaTeX Project Public License, either version 1.3
% of this license or (at your option) any later version.
% The latest version of this license is in
%   http://www.latex-project.org/lppl.txt
% and version 1.3 or later is part of all distributions of LaTeX
% version 2005/12/01 or later.
%
% This work has the LPPL maintenance status `maintained'.
%
% The Current Maintainer of this work is Niklas Beisert.
%
% This work consists of the files childdoc.dtx and childdoc.ins
% and the derived files childdoc.def and cdocsamp.tex with
% cdocsch1.tex, cdocsch2.tex, cdocsdrf.tex, cdocsfn1.tex, cdocsfn2.tex.
%
%<package>\ifdefined\childdocmain\endinput\fi
%<package>\ProvidesFile{childdoc.def}[2018/12/30 v2.0 child document driver]
%<samplemain>\ProvidesFile{cdocsamp.tex}[2018/12/30 v2.0 sample for childdoc]
%<*driver>
%\ProvidesFile{childdoc.drv}[2018/12/30 v2.0 childdoc reference manual file]
\PassOptionsToClass{10pt,a4paper}{article}
\documentclass{ltxdoc}

\usepackage[margin=35mm]{geometry}
\usepackage{hyperref}
\usepackage{hyperxmp}
\usepackage[usenames]{color}

\hypersetup{colorlinks=true}
\hypersetup{pdfstartview=FitH}
\hypersetup{pdfpagemode=UseNone}
\hypersetup{pdfsource={}}
\hypersetup{pdflang={en-UK}}
\hypersetup{pdfcopyright={Copyright 2017-2018 Niklas Beisert.
  This work may be distributed and/or modified under the
  conditions of the LaTeX Project Public License, either version 1.3
  of this license or (at your option) any later version.}}
\hypersetup{pdflicenseurl={http://www.latex-project.org/lppl.txt}}
\hypersetup{pdfcontactaddress={ETH Zurich, ITP, HIT K,
  Wolfgang-Pauli-Strasse 27}}
\hypersetup{pdfcontactpostcode={8093}}
\hypersetup{pdfcontactcity={Zurich}}
\hypersetup{pdfcontactcountry={Switzerland}}
\hypersetup{pdfcontactemail={nbeisert@itp.phys.ethz.ch}}
\hypersetup{pdfcontacturl={http://people.phys.ethz.ch/\xmptilde nbeisert/}}

\newcommand{\secref}[1]{\hyperref[#1]{section \ref*{#1}}}

\parskip1ex
\parindent0pt
\let\olditemize\itemize
\def\itemize{\olditemize\parskip0pt}

\begin{document}

\title{The \textsf{childdoc} Package}
\hypersetup{pdftitle={The childdoc Package}}
\author{Niklas Beisert\\[2ex]
  Institut f\"ur Theoretische Physik\\
  Eidgen\"ossische Technische Hochschule Z\"urich\\
  Wolfgang-Pauli-Strasse 27, 8093 Z\"urich, Switzerland\\[1ex]
  \href{mailto:nbeisert@itp.phys.ethz.ch}
  {\texttt{nbeisert@itp.phys.ethz.ch}}}
\hypersetup{pdfauthor={Niklas Beisert}}
\hypersetup{pdfsubject={Manual for the LaTeX2e Package childdoc}}
\date{30 December 2018, \textsf{v2.0}}
\maketitle

\begin{abstract}\noindent
\textsf{childdoc} is a \LaTeXe{} package
that enables the direct compilation
of document sections included by |\include|
to individual files.
\end{abstract}

\begingroup
\parskip0ex
\tableofcontents
\endgroup

%%%%%%%%%%%%%%%%%%%%%%%%%%%%%%%%%%%%%%%%%%%%%%%%%%%%%%%%%%%%%%%%%%%%%%%%%%%%%%%%
%%%%%%%%%%%%%%%%%%%%%%%%%%%%%%%%%%%%%%%%%%%%%%%%%%%%%%%%%%%%%%%%%%%%%%%%%%%%%%%%
\section{Introduction}

\LaTeX{} provides a mechanism to structure a large document (such as a book)
into a main file and several child files (containing the chapters)
using the |\include| command.
This mechanism is beneficial for documents
which span hundreds of pages in order to
make the source file(s) more manageable.
Moreover, compilation can be restricted to
selected child files by means of the |\includeonly| command.
The latter feature can be used to reduce the compilation time while editing
(this was significantly more useful in the earlier days of \LaTeX{})
or to generate a smaller document which is easier to navigate.
Another application of |\includeonly| is to generate
documents consisting of selected parts of the complete document.

However, there are a few drawbacks of the plain |\include| mechanism:
\begin{itemize}
\item
The child files cannot be compiled on their own,
they can only be compiled via the main file.
A naive editing environment
(such as a text editor with an option
to have the current file processed by \LaTeX)
may require one to switch to the main file before compiling;
attempting to compile the child file produces errors.
\item
The main file must be modified (each time)
to adjust the |\includeonly| command
to the present needs. This easily leaves the main file in a messy state.
\item
The generated document will always carry the filename
of the main document. This is inconvenient if
several child files are to be compiled and
to be kept for distribution.
\end{itemize}

The present package provides a simple interface
to make child files individually compilable by \LaTeX{}.
Compiling a child file then has the same effect as compiling
the main file with an |\includeonly| command
to select the appropriate child.
Moreover the generated document will carry the name of the child
rather than the main file.
This resolves all three above issues.

This feature is meant to make the editing of books,
thesis documents and lecture notes somewhat more convenient.
However, the package can also be used efficiently for
composing a series of documents (such as exercise sheets)
which are typically distributed individually.
It then assists the author in generating the individual documents
(potentially in different versions)
as well as a document containing the collected series.
Another application is in developing style files
or other kinds of included material
where compilation of the style file could redirect
to a sample or test file.

%%%%%%%%%%%%%%%%%%%%%%%%%%%%%%%%%%%%%%%%%%%%%%%%%%%%%%%%%%%%%%%%%%%%%%%%%%%%%%%%
%%%%%%%%%%%%%%%%%%%%%%%%%%%%%%%%%%%%%%%%%%%%%%%%%%%%%%%%%%%%%%%%%%%%%%%%%%%%%%%%
\section{Usage}

First of all, the package \textsf{childdoc} is \emph{not} a standard
\LaTeXe{} |.sty| style file! Therefore it needs to be invoked in
a non-standard way.

%%%%%%%%%%%%%%%%%%%%%%%%%%%%%%%%%%%%%%%%%%%%%%%%%%%%%%%%%%%%%%%%%%%%%%%%%%%%%%%%
\subsection{Included Files}
\label{sec:include}

%%%%%%%%%%%%%%%%%%%%%%%%%%%%%%%%%%%%%%%%
\DescribeMacro{\childdocmain}
To use the package, add the commands
\begin{center}
\begin{tabular}{l}
|\input{childdoc.def}|\\
|\childdocmain{}|\\
\end{tabular}
\end{center}
at the very top of the main \LaTeX{} file,
in particular \emph{before} the |\documentclass| statement!
The argument of |\childdocmain| should be left empty
(but it must be present).

%%%%%%%%%%%%%%%%%%%%%%%%%%%%%%%%%%%%%%%%
\DescribeMacro{\childdocof}
Furthermore, add the commands
\begin{center}
\begin{tabular}{l}
|\input{childdoc.def}|\\
|\childdocof{|\textit{main}|}|\\
\end{tabular}
\end{center}
at the top of every child file \textit{child}
which is included by |\include{|\textit{child}|}|
from within the main file
(or at least for those files to be compiled individually).
The argument \textit{main} must be the filename of the main file.

There are a couple of
considerations in setting up the main and child documents:

%%%%%%%%%%%%%%%%%%%%%%%%%%%%%%%%%%%%%%%%
\paragraph{Restrictions.}

Please note the following restrictions:
\begin{itemize}
\item
|\childdocmain| must be called with one argument \textit{main}
to ensure compatibility with earlier version of the package.
It must either be empty (|\childdocmain{}|)
or precisely match the filename of the main file in which it is specified.
See \secref{sec:detection} for further information.
\item
The filename \textit{main} must be specified without the |.tex| extension.
\item
The filename \textit{main} is case sensitive
(even in case-insensitive file systems)
due to internal string comparison.
\item
The argument \textit{main} should be fully expanded, it cannot be a macro.
\item
Subdirectories and special characters should be avoided in filenames.
\item
The command |\childdocmain{|\textit{main}|}| must be followed by a whitespace.
It should not be followed immediately by another command
or by a comment mark `|%|'.
This is because the \TeX{} parser reads the token immediately following
the argument of |\childdocmain| and puts it
at the beginning of every child section;
however, a white\-space is ignored.
\end{itemize}

%%%%%%%%%%%%%%%%%%%%%%%%%%%%%%%%%%%%%%%%
\paragraph{Content of Main File.}

It is advisable to place all content in the child files included by |\include|.
Any output contained in the main file will appear in all child documents
unless suppressed manually;
it cannot be suppressed automatically by the |\includeonly| directive
and thus should normally be avoided.
A method to include some content in the main file
by means of conditional processing is described in \secref{sec:conditional}.

%%%%%%%%%%%%%%%%%%%%%%%%%%%%%%%%%%%%%%%%
\paragraph{Page Numbering.}

When only a part of the document is compiled,
the appropriate numbering of pages
(as well as other status parameters)
is determined from the |.aux| files.
The latter contain information from previous passes.
However this information needs to propagate through
all intermediate child documents.
Therefore the page numbering in child documents may well
be inconsistent until the complete document is compiled at least once.

A useful (if unconventional) way to always ensure a consistent
page numbering is to restart the numbering in each child document
and denote the pages by `\textit{child}|.|\textit{page}'
where \textit{child} represents the chapter/section number of the child file.
This can be achieved by the command
|\numberwithin{page}{|\textit{child}|}|
of the \textsf{amsmath} package
where \textit{child} can be |chapter| or |section|
depending on the chosen structuring.
Alternatively, one can modify the macro |\thepage| appropriately
and reset the counter |page| at the start of each child file.

%%%%%%%%%%%%%%%%%%%%%%%%%%%%%%%%%%%%%%%%%%%%%%%%%%%%%%%%%%%%%%%%%%%%%%%%%%%%%%%%
\subsection{Conditional Processing}
\label{sec:conditional}

The package provides a mechanism to compile different versions
of a document. To customise the versions further some conditional processing
can come in handy to distinguish which version is being compiled.
The package provides two macros to describe the compilation context:

%%%%%%%%%%%%%%%%%%%%%%%%%%%%%%%%%%%%%%%%
\DescribeMacro{\ifchilddoc}
The conditional |\ifchilddoc| distinguishes between the compilation of
child documents and the main document:
%
\begin{center}
|\ifchilddoc |\textit{child-code}| |[|\||else |\textit{main-code}]| \||fi|
\end{center}

%%%%%%%%%%%%%%%%%%%%%%%%%%%%%%%%%%%%%%%%
\DescribeMacro{\childdocname}
\DescribeMacro{\childdocjob}
The macro |\childdocname| contains the filename (without extension)
of the main or child file being processed.
Note that |\childdocjob| will always contain the name of the main file.

%%%%%%%%%%%%%%%%%%%%%%%%%%%%%%%%%%%%%%%%
\paragraph{Title Page.}

Conditional processing can be used to include a title or banner page
in the main document when proper precautions are taken.
Importantly, the code in the main file should ensure that the page counter
(as well as other status parameters which are stored in the |.aux| files)
takes the same value after the conditional processing.
Otherwise the page numbers may take divergent values
depending on which part is compiled.

For example, a title page could be declared by:
%
\begin{center}
\begin{tabular}{l}
|\ifchilddoc\||else|\\
|\addtocounter{page}{-1}|\\
\textit{code for title page}\\
|\newpage|\\
|\||fi|
\end{tabular}
\end{center}
%
A banner page for the child documents can be generated by:
%
\begin{center}
\begin{tabular}{l}
|\ifchilddoc|\\
|\addtocounter{page}{-1}|\\
\textit{code for banner page}\\
|\newpage|\\
|\||fi|
\end{tabular}
\end{center}
%
Here one could write a message such as:
\begin{center}
|This is the part \childdocname{} of \childdocjob{}.|
\end{center}

%%%%%%%%%%%%%%%%%%%%%%%%%%%%%%%%%%%%%%%%%%%%%%%%%%%%%%%%%%%%%%%%%%%%%%%%%%%%%%%%
\subsection{Flags}
\label{sec:flags}

The package makes it easy to generate different versions
of the main or child documents.
To this end compilation flags can be defined
and assigned different default values.
They will be particularly useful in conjunction
with the forwarding mechanism described in \secref{sec:forward}.

For example, it may be useful to have a flag |\version|
which can be set to |draft| or |final|.
The document source will contain some conditional code
depending on the value of |\version|.
Suppose further, the flag should default to |final| for the main file
and to |draft| for child files
which is a natural assignment for editing the document.
This is achieved by placing the following code
in the preamble of the main document
(below the |\childdocmain| directive):
%
\begin{center}
\begin{tabular}{l}
|\ifchilddoc|\\
|\providecommand{\version}{draft}|\\
|\||else|\\
|\providecommand{\version}{final}|\\
|\||fi|
\end{tabular}
\end{center}
%
The definition by |\providecommand| makes sure
that previous definitions are not overwritten.
Further statements |\providecommand{\version}{...}|
can thus be added before the above code to override it.

For the main file, one might add a line
(between |\childdocmain| and the above block)
%
\begin{center}
|%\ifchilddoc\||else\providecommand{\version}{draft}\||fi|
\end{center}
%
which can be uncommented to produce a draft version.
Likewise one can add a line to the very top of a child file
(above the |\childdocof{|\textit{main}|}| directive)
%
\begin{center}
|%\providecommand{\version}{final}|
\end{center}
%
which can be uncommented to produce the final version of this child document.

%%%%%%%%%%%%%%%%%%%%%%%%%%%%%%%%%%%%%%%%%%%%%%%%%%%%%%%%%%%%%%%%%%%%%%%%%%%%%%%%
\subsection{Forwarding}
\label{sec:forward}

Different versions of the main or child documents
using compilation flags as described in \secref{sec:flags}
can be (permanently) stored in different files
for convenient compilation, viewing and distribution.
To this end, the package defines a command
to pass on compilation to a different file:

%%%%%%%%%%%%%%%%%%%%%%%%%%%%%%%%%%%%%%%%
\DescribeMacro{\childdocforward}
The command |\childdocforward| redirects processing to
another source file:
%
\begin{center}
\begin{tabular}{l}
|\input{childdoc.def}|\\
|\childdocforward[|\textit{main}|]{|\textit{dest}|}|\\
\end{tabular}
\end{center}
%
The argument \textit{dest} is the destination file
(without extension).
It should be the main file or one of the child files.
Note that further \textsf{childdoc} directives
such as |\childdocof| and |\childdocforward|
in the indicated file will be processed in this form.
The optional argument \textit{main}
passes on directly to the main file \textit{main}
while pretending to compile the child \textit{dest}.
This form behaves as if \textit{dest}
issues |\childdocof{|\textit{main}|}| right away,
and no further \textsf{childdoc} directives will be processed.

%%%%%%%%%%%%%%%%%%%%%%%%%%%%%%%%%%%%%%%%
\DescribeMacro{\...prefix}
In the alternative form |\childdocforwardprefix|,
%
\begin{center}
\begin{tabular}{l}
|\input{childdoc.def}|\\
|\childdocforwardprefix[|\textit{main}|]{|\textit{prefix}|}{|\textit{dest}|}|
\end{tabular}
\end{center}
%
the destination file is determined by a pattern
depending on the current file:
To make this work, the current file must be called
`{\textit{prefix}\hspace{0.2em}\textit{suffix}}'
with \textit{prefix} matching precisely the argument.
Processing is then passed on to the file
`{\textit{dest}\hspace{0.2em}\textit{suffix}}'.
Surely, the same effect is achieved by
directly specifying the
argument `{\textit{dest}\hspace{0.2em}\textit{suffix}}'
in the first form.
However, that requires to set up a different file
for each child. With the alternative form of the command
all these files can have exactly the same content
which simplifies setting them up and maintaining them.

For example, the following file |draft.tex|
with a compilation flag |\version| as described in \secref{sec:flags}
compiles the main document as a draft:
%
\begin{center}
\begin{tabular}{l}
|\def\version{draft}|\\
|\input{childdoc.def}|\\
|\childdocforward{|\textit{main}|}|
\end{tabular}
\end{center}
%
Likewise, the following files |final|\textit{nn}|.tex|
compile the final version of the child document
|child|\textit{nn}|.tex|:
%
\begin{center}
\begin{tabular}{l}
|\def\version{final}|\\
|\input{childdoc.def}|\\
|\childdocforwardprefix{final}{child}|
\end{tabular}
\end{center}
%

Note that when several versions of a main file and/or of each child file
are to be generated, it may be convenient to set up a |Makefile| or
shell script to automatise the process.

%%%%%%%%%%%%%%%%%%%%%%%%%%%%%%%%%%%%%%%%%%%%%%%%%%%%%%%%%%%%%%%%%%%%%%%%%%%%%%%%
\subsection{Command Line Processing}
\label{sec:commandline}

The effect of redirection files can also be achieved by invoking
the \LaTeX{} compiler with a more elaborate command line.
Most conveniently this should be done as part
of a shell script or a |Makefile|.

When using \textsf{childdoc} in the main file, the following
command lines effectively perform a redirection
(note that depending on the shell being used,
backslashes may have to be doubled: `|\|' $\to$ `|\\|'):
%
\begin{center}
|... -jobname "|\textit{target}|" |\\|"|[\textit{flags}]%
|\input{childdoc.def}\childdocforward[|\textit{main}|]{|\textit{dest}|}"|
\end{center}
%
Here \textit{target} is the name of the output file,
\textit{main} is the name of the main file
and \textit{dest} is the name of the main or child file to be processed
(all filenames without extensions).
The optional argument \textit{main} can be omitted
if \textit{main} matches \textit{dest}.
Optionally, compilation \textit{flags} can be defined via |\def| commands.
This command line makes the \TeX{} engine believe
it is compiling the file \textit{target}
whose content is specified as the latter parameter.
The provided code then forwards the processing to
\textit{main} or \textit{dest} as described in \secref{sec:forward}.

%%%%%%%%%%%%%%%%%%%%%%%%%%%%%%%%%%%%%%%%%%%%%%%%%%%%%%%%%%%%%%%%%%%%%%%%%%%%%%%%
\subsection{Include by Input}
\label{sec:input}

Including child documents by |\include| has some restrictions by design.
Most notably, the content of a child document always occupies
its own set of pages; pages cannot be shared between child documents.
Usually, this behaviour makes perfect sense
because each child document contain an essential part of the document.
However, in some situations it may be desirable to compose
a document from a collection of parts
without having mandatory page breaks between then.
For this case, the package
provides a mechanism to include parts
by |\input| which can also be processed individually.
However, by construction this mechanism
requires manual handling of the content to be output.

%%%%%%%%%%%%%%%%%%%%%%%%%%%%%%%%%%%%%%%%
\DescribeMacro{\ifchilddocmanual}
The main file should be prepared as usual, see \secref{sec:include}.
However, the document body must make a distinction
between processing of an individual part and of the main document, e.g.:
%
\begin{center}
\begin{tabular}{l}
|\ifchilddocmanual|\\
|\input{\childdocname}|\\
|\||else|\\
\textit{document body with }|\input{|\textit{part}|}|\\
|\||fi|
\end{tabular}
\end{center}
%
The conditional |\ifchilddocmanual| is true whenever
a part to be included by |\input| is being compiled,
and the name of the part is stored in |\childdocname|.

%%%%%%%%%%%%%%%%%%%%%%%%%%%%%%%%%%%%%%%%
\DescribeMacro{\childdocby}
Each part to be included by |\input| should start with:
%
\begin{center}
\begin{tabular}{l}
|\input{childdoc.def}|\\
|\childdocby{|\textit{main}|}|\\
\end{tabular}
\end{center}
%
The directive |\childdocby| is similar to |\childdocof|
described in \secref{sec:include},
but the subsequent selection of content must be done manually.
To that end, both |\ifchilddoc| and |\ifchilddocmanual|
will be true upon processing of a part,
and the name of the part is stored in |\childdocname|.
Note that |\jobname| will be set to the filename of the current part
so that each part receives an individual |.aux| file
that does not interfere with the |.aux| file(s) of the main document.
This behaviour can be altered by the alternative form
|\childdocby[*]{|\textit{main}|}| (with a non-empty optional argument)
which uses the |.aux| file of the main document
by setting |\jobname| to \textit{main}.

%%%%%%%%%%%%%%%%%%%%%%%%%%%%%%%%%%%%%%%%%%%%%%%%%%%%%%%%%%%%%%%%%%%%%%%%%%%%%%%%
\subsection{Driver Development}
\label{sec:driver}

The \textsf{childdoc} mechanism can also be use for the development
of definition files such as \LaTeX{} styles or classes.
This case differs from the above setup with multiple parts
included by |\include| in that no |\includeonly| should be invoked.
This can be achieved by starting the include file
(before |\ProvidesPackage|) with:
%
\begin{center}
\begin{tabular}{l}
|\input{childdoc.def}|\\
|\childdocforward{|\textit{main}|}|\\
\end{tabular}
\end{center}
%
or alternatively with:
%
\begin{center}
\begin{tabular}{l}
|\input{childdoc.def}|\\
|\childdocby{|\textit{main}|}|\\
\end{tabular}
\end{center}
%
Both forms have slightly different effects as described above.
The main file is prepared as usual, see \secref{sec:include}.

%%%%%%%%%%%%%%%%%%%%%%%%%%%%%%%%%%%%%%%%%%%%%%%%%%%%%%%%%%%%%%%%%%%%%%%%%%%%%%%%
\subsection{Legacy Detection}
\label{sec:detection}

The directive |\childdocmain| in the main file can detect
whether the complete document or merely a child is to be compiled
even without using the directive |\childdocof|.
This method is deprecated because it is less robust
and there is no compelling reason to use it;
it is merely provided for backward compatibility
and it may be removed in future versions.

If the detection mechanism is to be used,
it is mandatory to correctly specify
the filename of the main file as the argument of |\childdocmain|:
%
\begin{center}
\begin{tabular}{l}
|\input{childdoc.def}|\\
|\childdocmain{|\textit{main}|}|\\
\end{tabular}
\end{center}
%
If |\jobname| does not match the argument \textit{main} of |\childdocmain|,
it is assumed that |\jobname| points to the child file to be compiled.
When using |\childdocmain| with the main file specified as argument,
it suffices to start a child file
with just |\input{|\textit{main}|}|
without loading of the package and using |\childdocof|.
If instead all processing is done
with the appropriate \textsf{childdoc} directives,
the argument of \textit{main} of |\childdocmain| can be empty.

An alternative version of the command line processing described
in \secref{sec:commandline} using the detection mechanism reads:
%
\begin{center}
|... -jobname "|\textit{target}|" "|[\textit{flags}]%
[|\def\jobname{|\textit{dest}|}|]|\input{|\textit{main}|}"|
\end{center}

%%%%%%%%%%%%%%%%%%%%%%%%%%%%%%%%%%%%%%%%%%%%%%%%%%%%%%%%%%%%%%%%%%%%%%%%%%%%%%%%
\subsection{Manual Code}
\label{sec:manual}

In case one cannot be certain whether the definitions file |childdoc.def|
is installed on the target \TeX{} distribution
and one prefers not to ship it,
it is conceivable to paste a few relevant commands into the sources.

To that end, drop all statements |\input{childdoc.def}|
and perform the replacements as outlined below.
Instead of |\childdocmain{|\textit{main}|}| add the following code
to the top of the main file:
%
\begin{center}
\begin{tabular}{l}
|\||ifdefined\childdocname\endinput\||fi\newif\ifchilddoc|\\
|\edef\childdocname{\scantokens\expandafter{\jobname\noexpand}}|\\
|\def\childdocmain{|\textit{main}|}\||ifx\childdocmain\childdocname\||else|\\
|\childdoctrue\includeonly{\childdocname}\let\jobname\childdocmain\||fi|\\
\end{tabular}
\end{center}
%
Instead of |\childdocof{|\textit{main}|}| just include the main file
at the top of each child file:
%
\begin{center}
|\input{|\textit{main}|}|
\end{center}
%
A simple redirection |\childdocforward{|\textit{dest}|}| is achieved by:
%
\begin{center}
|\def\jobname{|\textit{dest}|}\input{\jobname}|
\end{center}
%
The redirection with prefix
|\childdocforwardprefix[|\textit{prefix}|]{|\textit{dest}|}|
is accomplished by:
%
\begin{center}
\begin{tabular}{l}
|{\edef\jobname{\scantokens\expandafter{\jobname\noexpand}}|\\
|\def\redirectjob |\textit{prefix}|#1~~~{\gdef\jobname{|\textit{dest}|#1}}|\\
|\expandafter\redirectjob\jobname~~~}\input{\jobname}|
\end{tabular}
\end{center}

In an alternative approach,
child documents can be compiled by a specific command line
without additional code or specific definitions:
%
\begin{center}
|... -jobname "|\textit{target}|" "|[\textit{flags}]%
|\includeonly{|\textit{dest}|}\input{|\textit{main}|}"|
\end{center}
%

%%%%%%%%%%%%%%%%%%%%%%%%%%%%%%%%%%%%%%%%%%%%%%%%%%%%%%%%%%%%%%%%%%%%%%%%%%%%%%%%
%%%%%%%%%%%%%%%%%%%%%%%%%%%%%%%%%%%%%%%%%%%%%%%%%%%%%%%%%%%%%%%%%%%%%%%%%%%%%%%%
\section{Information}

%%%%%%%%%%%%%%%%%%%%%%%%%%%%%%%%%%%%%%%%%%%%%%%%%%%%%%%%%%%%%%%%%%%%%%%%%%%%%%%%
\subsection{Copyright}

Copyright \copyright{} 2017--2018 Niklas Beisert

This work may be distributed and/or modified under the
conditions of the \LaTeX{} Project Public License, either version 1.3
of this license or (at your option) any later version.
The latest version of this license is in
  \url{http://www.latex-project.org/lppl.txt}
and version 1.3 or later is part of all distributions of \LaTeX{}
version 2005/12/01 or later.

This work has the LPPL maintenance status `maintained'.

The Current Maintainer of this work is Niklas Beisert.

This work consists of the files |README.txt|, |childdoc.ins| and |childdoc.dtx|
as well as the derived files |childdoc.def|, |cdocsamp.tex|
with |cdocsch1.tex|, |cdocsch2.tex|, |cdocspt3.tex|, |cdocspt4.tex|,
|cdocsdrf.tex|, |cdocsfn1.tex|, |cdocsfn2.tex|
as well as |childdoc.pdf|.

%%%%%%%%%%%%%%%%%%%%%%%%%%%%%%%%%%%%%%%%%%%%%%%%%%%%%%%%%%%%%%%%%%%%%%%%%%%%%%%%
\subsection{Files and Installation}

The package consists of the files:
%
\begin{center}
\begin{tabular}{ll}
    |README.txt|   & readme file \\
    |childdoc.ins| & installation file \\
    |childdoc.dtx| & source file \\
    |childdoc.def| & definition file \\
    |cdocsamp.tex| & sample main file \\
    |cdocsch1.tex| & sample include file \\
    |cdocsch2.tex| & sample include file \\
    |cdocspt3.tex| & sample part file \\
    |cdocspt4.tex| & sample part file \\
    |cdocsdrf.tex| & sample redirection file \\
    |cdocsfn1.tex| & sample redirection file \\
    |cdocsfn2.tex| & sample redirection file \\
    |childdoc.pdf| & manual
\end{tabular}
\end{center}
%
The distribution consists of the files
|README.txt|, |childdoc.ins| and |childdoc.dtx|.
%
\begin{itemize}
\item
Run (pdf)\LaTeX{} on |childdoc.dtx|
to compile the manual |childdoc.pdf| (this file).
\item
Run \LaTeX{} on |childdoc.ins| to create the definitions file |childdoc.def|
and the sample |cdocsamp.tex| with include files
|cdocsch1.tex|, |cdocsch2.tex|, |cdocspt3.tex|, |cdocspt4.tex|,
|cdocsdrf.tex|, |cdocsfn1.tex|, |cdocsfn2.tex|.
Then copy the file |childdoc.def| to an appropriate directory of your \LaTeX{}
distribution, e.g.\ \textit{texmf-root}|/tex/latex/childdoc|.
\end{itemize}

%%%%%%%%%%%%%%%%%%%%%%%%%%%%%%%%%%%%%%%%%%%%%%%%%%%%%%%%%%%%%%%%%%%%%%%%%%%%%%%%
\subsection{Related CTAN Packages}

There are several other packages which offer a similar functionality:
%
\begin{itemize}
\item
The packages
\href{http://ctan.org/pkg/docmute}{\textsf{docmute}},
\href{http://ctan.org/pkg/includex}{\textsf{includex}} and
\href{http://ctan.org/pkg/standalone}{\textsf{standalone}}
provide commands to include only the document body of
a child file thus allowing both files to be compiled individually.
\item
The packages \href{http://ctan.org/pkg/subdocs}{\textsf{subdocs}}
and \href{http://ctan.org/pkg/subfiles}{\textsf{subfiles}}
provide structures in which the main and child documents can be
encapsulated and allowing them to be compiled individually.
The inclusion mechanism is different from the conventional |\include|.
\item
The package \href{http://ctan.org/pkg/combine}{\textsf{combine}}
is an elaborate solution to combine several documents into one.
\end{itemize}
%
See also the CTAN topic \href{http://ctan.org/topic/subdocs}{\textsf{subdocs}}
for further related packages.
The present package differs from the above solutions in that
a document structure constructed with the conventional |\include| mechanism
just needs two extra commands at the top of every file
such that all constituent files can be compiled individually.

%%%%%%%%%%%%%%%%%%%%%%%%%%%%%%%%%%%%%%%%%%%%%%%%%%%%%%%%%%%%%%%%%%%%%%%%%%%%%%%%
%\subsection{Feature Suggestions}
%
%The following is a list of features which may be useful for future
%versions of this package:
%%
%\begin{itemize}
%\item
%\ldots
%\end{itemize}

%%%%%%%%%%%%%%%%%%%%%%%%%%%%%%%%%%%%%%%%%%%%%%%%%%%%%%%%%%%%%%%%%%%%%%%%%%%%%%%%
\subsection{Revision History}

%%%%%%%%%%%%%%%%%%%%%%%%%%%%%%%%%%%%%%%%
\paragraph{v2.0:} 2018/12/30

\begin{itemize}
\item
immediate forward processing
\item
added |\childdocby| mechanism
\item
manual restructured
\end{itemize}

%%%%%%%%%%%%%%%%%%%%%%%%%%%%%%%%%%%%%%%%
\paragraph{v1.6:} 2018/01/17

\begin{itemize}
\item
application for development of include files
\item
corrections to manual
\end{itemize}

%%%%%%%%%%%%%%%%%%%%%%%%%%%%%%%%%%%%%%%%
\paragraph{v1.5:} 2017/05/21

\begin{itemize}
\item
more complete structuring introduced
\item
|\childdocof| introduced
\item
|\childdoc| renamed to |\childdocmain|
\item
|\childredirect| renamed to |\childdocforward| and |\childdocforwardprefix|
and functionality expanded
\end{itemize}

%%%%%%%%%%%%%%%%%%%%%%%%%%%%%%%%%%%%%%%%
\paragraph{v1.0:} 2017/04/27

\begin{itemize}
\item
manual and install package
\item
first version published on CTAN
\end{itemize}

%%%%%%%%%%%%%%%%%%%%%%%%%%%%%%%%%%%%%%%%
\paragraph{v0.6:} 2017/04/26

\begin{itemize}
\item
redirection mechanism added
\end{itemize}

%%%%%%%%%%%%%%%%%%%%%%%%%%%%%%%%%%%%%%%%
\paragraph{v0.5:} 2017/04/26

\begin{itemize}
\item
functionality in definition file
\end{itemize}


%%%%%%%%%%%%%%%%%%%%%%%%%%%%%%%%%%%%%%%%%%%%%%%%%%%%%%%%%%%%%%%%%%%%%%%%%%%%%%%%
%%%%%%%%%%%%%%%%%%%%%%%%%%%%%%%%%%%%%%%%%%%%%%%%%%%%%%%%%%%%%%%%%%%%%%%%%%%%%%%%
%%%%%%%%%%%%%%%%%%%%%%%%%%%%%%%%%%%%%%%%%%%%%%%%%%%%%%%%%%%%%%%%%%%%%%%%%%%%%%%%
\appendix

\settowidth\MacroIndent{\rmfamily\scriptsize 000\ }

 \DocInput{childdoc.dtx}

\end{document}
%</driver>
% \fi
%
% %%%%%%%%%%%%%%%%%%%%%%%%%%%%%%%%%%%%%%%%%%%%%%%%%%%%%%%%%%%%%%%%%%%%%%%%%%%%%%
% %%%%%%%%%%%%%%%%%%%%%%%%%%%%%%%%%%%%%%%%%%%%%%%%%%%%%%%%%%%%%%%%%%%%%%%%%%%%%%
% \section{Sample}
%\iffalse
%<*samplemain>
%\fi
%
% The following presents a sample document
% with two chapters, two parts, a title page,
% a compile flag as well as three forwarding files to set the flag.
% It consists of eight |.tex| files:
% \begin{center}
% \begin{tabular}{ll}
% |cdocsamp.tex|&main file\\
% |cdocsch1.tex|&include file for chapter 1\\
% |cdocsch2.tex|&include file for chapter 2\\
% |cdocspt3.tex|&include file for part 3\\
% |cdocspt4.tex|&include file for part 4\\
% |cdocsdrf.tex|&forwarding file for main file in draft mode\\
% |cdocsfi1.tex|&forwarding file for final version of chapter 1\\
% |cdocsfi2.tex|&forwarding file for final version of chapter 2\\
% \end{tabular}
% \end{center}
% Each of the eight files can be compiled directly by the \LaTeX{} compiler.
%
% %%%%%%%%%%%%%%%%%%%%%%%%%%%%%%%%%%%%%%
% \paragraph{Main File.}
%
% The main file is called |cdocsamp.tex|.
%
% Load the \textsf{childdoc} definitions and
% declare the filename for the main document:
%    \begin{macrocode}
\input{childdoc.def}
\childdocmain{}
%    \end{macrocode}

% Optional override for |\version| flag:
%    \begin{macrocode}
%%\ifchilddoc\else\providecommand{\version}{draft}\fi
%    \end{macrocode}

% Define the default values for the |\version| flag
% (|final| for the main file and |draft| for childs):
%    \begin{macrocode}
\ifchilddoc
\providecommand{\version}{draft}
\else
\providecommand{\version}{final}
\fi
%    \end{macrocode}

% Load the standard document class:
%    \begin{macrocode}
\documentclass[12pt]{article}
%    \end{macrocode}

% Start the document body:
%    \begin{macrocode}
\begin{document}
%    \end{macrocode}

% Declare a title page.
% Print title, part of document being processed and version flag:
%    \begin{macrocode}
\addtocounter{page}{-1}
\begin{center}
{\LARGE\bfseries{}childdoc example\par}
\vspace{1cm}
\ifchilddoc
\ifchilddocmanual part\else chapter\fi:
`\childdocname' of `\childdocjob'\par
\else
main document: `\childdocjob'\par
\fi
version: \version\par
\end{center}
\newpage
%    \end{macrocode}

% Manually include selected file,
% otherwise process as usual:
%    \begin{macrocode}
\ifchilddocmanual
\section*{part `\childdocname'}
\input{\childdocname}
\else
%    \end{macrocode}

% Include the two chapters:
%    \begin{macrocode}
\include{cdocsch1}
\include{cdocsch2}
%    \end{macrocode}

% Include the two parts unless only chapters should be displayed:
%    \begin{macrocode}
\ifchilddoc\else
\section{part three}
\input{cdocspt3}
\section{part four}
\input{cdocspt4}
\fi
%    \end{macrocode}

% Process as usual until here:
%    \begin{macrocode}
\fi
%    \end{macrocode}

% End of document body:
%    \begin{macrocode}
\end{document}
%    \end{macrocode}
%\iffalse
%</samplemain>
%\fi
%
% %%%%%%%%%%%%%%%%%%%%%%%%%%%%%%%%%%%%%%
% \paragraph{Chapter Include Files.}
%
% The include files are called |cdocsch1.tex| and |cdocsch2.tex|.
%
%\iffalse
%<*samplechap1|samplechap2>
%\fi

% Optional override for |\version| flag:
%    \begin{macrocode}
%%\providecommand{\version}{final}
%    \end{macrocode}

% Include the main document:
%    \begin{macrocode}
\input{childdoc.def}
\childdocof{cdocsamp}
%    \end{macrocode}

%\iffalse
%</samplechap1|samplechap2>
%\fi
%
%\iffalse
%<*samplechap1>
%\fi
% Some text for chapter 1:
%    \begin{macrocode}
\section{one}
some text in chapter one
%    \end{macrocode}

%\iffalse
%</samplechap1>
%\fi
% Some text for chapter 2:
%\iffalse
%<*samplechap2>
%\fi
%    \begin{macrocode}
\section{two}
more text in chapter two
%    \end{macrocode}

%\iffalse
%</samplechap2>
%\fi
%
% %%%%%%%%%%%%%%%%%%%%%%%%%%%%%%%%%%%%%%
% \paragraph{Part Include Files.}
%
% The include files are called |cdocspt3.tex| and |cdocspt4.tex|.
%
%\iffalse
%<*samplepart3|samplepart4>
%\fi

% Optional override for |\version| flag:
%    \begin{macrocode}
%%\providecommand{\version}{final}
%    \end{macrocode}

% Include the main document:
%    \begin{macrocode}
\input{childdoc.def}
\childdocby{cdocsamp}
%    \end{macrocode}

%\iffalse
%</samplepart3|samplepart4>
%\fi
%
%\iffalse
%<*samplepart3>
%\fi
% Some text for part 3:
%    \begin{macrocode}
some text in part three
%    \end{macrocode}

%\iffalse
%</samplepart3>
%\fi
% Some text for part 4:
%\iffalse
%<*samplepart4>
%\fi
%    \begin{macrocode}
more text in part four
%    \end{macrocode}

%\iffalse
%</samplepart4>
%\fi
%
% %%%%%%%%%%%%%%%%%%%%%%%%%%%%%%%%%%%%%%
% \paragraph{Forwarding for a Complete Draft.}
%
% The following forwarding file |cdocsdrf.tex|
% compiles the main document in draft mode:
%\iffalse
%<*sampledraft>
%\fi
%    \begin{macrocode}
\def\version{draft}
\input{childdoc.def}
\childdocforward{cdocsamp}
%    \end{macrocode}

%\iffalse
%</sampledraft>
%\fi
%
% %%%%%%%%%%%%%%%%%%%%%%%%%%%%%%%%%%%%%%
% \paragraph{Forwarding for Final Version of the Chapters.}
%
% The following forwarding files |cdocsfn1.tex| and |cdocsfn2.tex|
% (with identical content)
% compile the final versions of the child documents
% |cdocsch1.tex| and |cdocsch2.tex|, respectively:
%\iffalse
%<*samplefinal>
%\fi
%    \begin{macrocode}
\def\version{final}
\input{childdoc.def}
\childdocforwardprefix[cdocsamp]{cdocsfn}{cdocsch}
%    \end{macrocode}

%\iffalse
%</samplefinal>
%\fi
%
% %%%%%%%%%%%%%%%%%%%%%%%%%%%%%%%%%%%%%%
% \paragraph{Command Line Processing.}
%
% The following three command lines generate the output files
% |cdocscld|, |cdocscl1| and |cdocscl2|
% which should be identical to
% |cdocsdrf|, |cdocsch1| and |cdocsfn2|, respectively:
% \begin{center}
% \begin{tabular}{l}
% |latex -jobname cdocscld \|\\
% |  "\def\version{draft}\input{childdoc.def}\childdocforward{cdocsamp}"|\\
% |latex -jobname cdocscl1 \|\\
% |  "\input{childdoc.def}\childdocforward[cdocsamp]{cdocsch1}"|\\
% |latex -jobname cdocscl2 \|\\
% |  "\def\version{final}\input{childdoc.def}\childdocforward{cdocsch2}"|
% \end{tabular}
% \end{center}
% Note that the trailing backslash on each first line
% merely continues the input to the second line
% (for convenient cut ant paste).
% Furthermore, the command |latex| can be replaced by any
% of its alternative versions such as |pdflatex|.
%
% %%%%%%%%%%%%%%%%%%%%%%%%%%%%%%%%%%%%%%%%%%%%%%%%%%%%%%%%%%%%%%%%%%%%%%%%%%%%%%
% %%%%%%%%%%%%%%%%%%%%%%%%%%%%%%%%%%%%%%%%%%%%%%%%%%%%%%%%%%%%%%%%%%%%%%%%%%%%%%
% \section{Implementation}
%\iffalse
%<*package>
%\fi
%
% This section describes the definitions file |childdoc.def|.

% The definitions cannot be loaded using |\usepackage| or |\RequirePackage|
% which has a mechanism to prevent loading a style file more than once.
% When loading the definitions by means of |\input|
% multiple instances have to be prevented manually:
%\iffalse
%This code needs to be before the `\ProvidesFile' directive
%which is defined at the beginning of this file.
%Therefore it is also placed there and commented out here.
%</package>
%<*discard>
%\fi
%    \begin{macrocode}
\ifdefined\childdocmain\endinput\fi
%    \end{macrocode}
%\iffalse
%</discard>
%<*package>
%\fi
%
% \macro{\ifchilddoc}
% \macro{\ifchilddocmanual}
% The conditional |\ifchilddoc| tells whether a
% child (true) or main (false) document is being compiled.
% The conditional |\ifchilddocmanual| tells whether
% the |\includeonly| mechanism is used (false) or
% the selection of child files must be performed manually (true).
% The definitions initialise to false:
%    \begin{macrocode}
\newif\ifchilddoc
\newif\ifchilddocmanual
%    \end{macrocode}

% \macro{\childdocname}
% \macro{\childdocjob}
% The macro |\childdocname| stores the name of the main document
% to be compiled. The macro |\childdocjob| stores the name of
% the document on which the \LaTeX{} compiler was originally invoked.
% The content of |\jobname| cannot be compared
% to filenames specified in the source due to different catcodes.
% The following code rescans |\jobname|, stores the result
% in |\childdocname| and saves a copy in |\childdocjob|:
%    \begin{macrocode}
\edef\childdocname{\scantokens\expandafter{\jobname\noexpand}}
\let\childdocjob\childdocname
%    \end{macrocode}

% \macro{\childdocdisable}
% The macro |\childdocdisable| prevents the main file
% from being processed more than once.
% At this stage, the main document command |\childdocmain|
% is assumed to be called once again where it should do nothing.
% Any subsequent call to it should prevent
% a secondary processing of the main document
% It overwrites the forwarding commands
% |\childdocof| and |\childdocforward|
% with empty macros to prevent further inclusions of the main document:
%    \begin{macrocode}
\newcommand{\childdocdisable}
{
  \renewcommand{\childdocmain}[1]{\renewcommand{\childdocmain}[1]{\endinput}}
  \renewcommand{\childdocof}[1]{}
  \renewcommand{\childdocby}[2][]{}
  \renewcommand{\childdocforward}[2][]{}
  \renewcommand{\childdocdisable}{}
}
%    \end{macrocode}

% \macro{\childdocmain}
% The macro |\childdocmain| is to be called at the top of the main file
% with nothing or the main filename (without extension) as argument.
% First, it breaks loops.
% If the argument is not empty and does not match |\childdocname|
% (which is set by the first inclusion of |childdoc.def|),
% |\ifchilddoc| is set to true, |\includeonly| is applied to the child file
% and |\jobname| is set to the main file
% (for proper handling of |.aux| files):
%    \begin{macrocode}
\newcommand{\childdocmain}[1]
{
  \childdocdisable\childdocmain{}
  \if?#1?\else
    \begingroup
      \def\childdoctmp{#1}
      \ifx\childdoctmp\childdocname
        \def\childdoctmp{}
      \else
        \def\childdoctmp
        {
          \childdoctrue
          \includeonly{\childdocname}
          \def\childdocjob{#1}
          \def\jobname{#1}
        }
      \fi
      \expandafter
    \endgroup
    \childdoctmp
  \fi
}
%    \end{macrocode}

% \macro{\childdocof}
% The command |\childdocof| redirects
% compilation to the main file |#1|.
%    \begin{macrocode}
\newcommand{\childdocof}[1]
{
  \childdocdisable
  \childdoctrue
  \includeonly{\childdocname}
  \def\jobname{#1}
  \def\childdocjob{#1}
  \input{#1}
}
%    \end{macrocode}

% \macro{\childdocby}
% The command |\childdocby| ....
%    \begin{macrocode}
\newcommand{\childdocby}[2][]
{
  \childdocdisable
  \childdoctrue
  \childdocmanualtrue
  \if?#1?\else
    \def\jobname{#2}
  \fi
  \def\childdocjob{#2}
  \input{#2}
  \endinput
}
%    \end{macrocode}

% \macro{\childdocforward}
% The command |\childdocforward| redirects
% compilation to the main file or
% (if the optional argument is given) a child file.
% Parameters are set as if the main file
% or a child file starting with |\childdocof| was compiled.
% Then compilation is handed over to the main file:
%    \begin{macrocode}
\newcommand{\childdocforward}[2][]
{
  \begingroup
    \if?#1?
      \def\childdoctmp
      {
        \def\childdocname{#2}
        \def\childdocjob{#2}
        \def\jobname{#2}
        \input{#2}
        \endinput
      }
    \else
      \def\childdoctmp
      {
        \childdocdisable
        \def\childdocname{#2}
        \childdoctrue
        \includeonly{#2}
        \def\childdocjob{#1}
        \def\jobname{#1}
        \input{#1}
        \endinput
      }
    \fi
    \expandafter
  \endgroup
  \childdoctmp
}
%    \end{macrocode}

% \macro{\childdocforwardprefix}
% The command |\childdocforwardprefix| redirects
% compilation to the main or a child file by means of a pattern.
% The prefix |#1| in the current filename is replaced by |#2|
% and the suffix of the current filename is kept
% (it is assumed that the filename does not contain the substring `|~~~|'
% which is used as a delimiter).
% Compilation is handed over to the new file by |\childdocforward|:
%    \begin{macrocode}
\newcommand{\childdocforwardprefix}[3][]
{
  \begingroup
    \def\childdocextract #2##1~~~{\def\childdoctmp{\childdocforward[#1]{#3##1}}}
    \expandafter\childdocextract\childdocname~~~
    \expandafter
  \endgroup
  \childdoctmp
}
%    \end{macrocode}

% \macro{\childdoc}
% The deprecated macro |\childdoc| is a legacy version of |\childdocmain|:
%    \begin{macrocode}
\newcommand{\childdoc}{\childdocmain}
%    \end{macrocode}

% \macro{\childdocredirect}
% The deprecated macro |\childdocredirect| is a legacy version
% of |\childdocforward| and |\childdocforwardprefix|:
%    \begin{macrocode}
\newcommand{\childdocredirect}[2][]
{
  \begingroup
    \if?#1?
      \def\childdoctmp{\childdocforward{#2}}
    \else
      \def\childdoctmp{\childdocforwardprefix{#1}{#2}}
    \fi
    \expandafter
  \endgroup
  \childdoctmp
}
%    \end{macrocode}

%\iffalse
%</package>
%\fi
%
\endinput
|\\
|\childdocforwardprefix[|\textit{main}|]{|\textit{prefix}|}{|\textit{dest}|}|
\end{tabular}
\end{center}
%
the destination file is determined by a pattern
depending on the current file:
To make this work, the current file must be called
`{\textit{prefix}\hspace{0.2em}\textit{suffix}}'
with \textit{prefix} matching precisely the argument.
Processing is then passed on to the file
`{\textit{dest}\hspace{0.2em}\textit{suffix}}'.
Surely, the same effect is achieved by
directly specifying the
argument `{\textit{dest}\hspace{0.2em}\textit{suffix}}'
in the first form.
However, that requires to set up a different file
for each child. With the alternative form of the command
all these files can have exactly the same content
which simplifies setting them up and maintaining them.

For example, the following file |draft.tex|
with a compilation flag |\version| as described in \secref{sec:flags}
compiles the main document as a draft:
%
\begin{center}
\begin{tabular}{l}
|\def\version{draft}|\\
|% \iffalse
%
% childdoc.dtx Copyright (C) 2017-2018 Niklas Beisert
%
% This work may be distributed and/or modified under the
% conditions of the LaTeX Project Public License, either version 1.3
% of this license or (at your option) any later version.
% The latest version of this license is in
%   http://www.latex-project.org/lppl.txt
% and version 1.3 or later is part of all distributions of LaTeX
% version 2005/12/01 or later.
%
% This work has the LPPL maintenance status `maintained'.
%
% The Current Maintainer of this work is Niklas Beisert.
%
% This work consists of the files childdoc.dtx and childdoc.ins
% and the derived files childdoc.def and cdocsamp.tex with
% cdocsch1.tex, cdocsch2.tex, cdocsdrf.tex, cdocsfn1.tex, cdocsfn2.tex.
%
%<package>\ifdefined\childdocmain\endinput\fi
%<package>\ProvidesFile{childdoc.def}[2018/12/30 v2.0 child document driver]
%<samplemain>\ProvidesFile{cdocsamp.tex}[2018/12/30 v2.0 sample for childdoc]
%<*driver>
%\ProvidesFile{childdoc.drv}[2018/12/30 v2.0 childdoc reference manual file]
\PassOptionsToClass{10pt,a4paper}{article}
\documentclass{ltxdoc}

\usepackage[margin=35mm]{geometry}
\usepackage{hyperref}
\usepackage{hyperxmp}
\usepackage[usenames]{color}

\hypersetup{colorlinks=true}
\hypersetup{pdfstartview=FitH}
\hypersetup{pdfpagemode=UseNone}
\hypersetup{pdfsource={}}
\hypersetup{pdflang={en-UK}}
\hypersetup{pdfcopyright={Copyright 2017-2018 Niklas Beisert.
  This work may be distributed and/or modified under the
  conditions of the LaTeX Project Public License, either version 1.3
  of this license or (at your option) any later version.}}
\hypersetup{pdflicenseurl={http://www.latex-project.org/lppl.txt}}
\hypersetup{pdfcontactaddress={ETH Zurich, ITP, HIT K,
  Wolfgang-Pauli-Strasse 27}}
\hypersetup{pdfcontactpostcode={8093}}
\hypersetup{pdfcontactcity={Zurich}}
\hypersetup{pdfcontactcountry={Switzerland}}
\hypersetup{pdfcontactemail={nbeisert@itp.phys.ethz.ch}}
\hypersetup{pdfcontacturl={http://people.phys.ethz.ch/\xmptilde nbeisert/}}

\newcommand{\secref}[1]{\hyperref[#1]{section \ref*{#1}}}

\parskip1ex
\parindent0pt
\let\olditemize\itemize
\def\itemize{\olditemize\parskip0pt}

\begin{document}

\title{The \textsf{childdoc} Package}
\hypersetup{pdftitle={The childdoc Package}}
\author{Niklas Beisert\\[2ex]
  Institut f\"ur Theoretische Physik\\
  Eidgen\"ossische Technische Hochschule Z\"urich\\
  Wolfgang-Pauli-Strasse 27, 8093 Z\"urich, Switzerland\\[1ex]
  \href{mailto:nbeisert@itp.phys.ethz.ch}
  {\texttt{nbeisert@itp.phys.ethz.ch}}}
\hypersetup{pdfauthor={Niklas Beisert}}
\hypersetup{pdfsubject={Manual for the LaTeX2e Package childdoc}}
\date{30 December 2018, \textsf{v2.0}}
\maketitle

\begin{abstract}\noindent
\textsf{childdoc} is a \LaTeXe{} package
that enables the direct compilation
of document sections included by |\include|
to individual files.
\end{abstract}

\begingroup
\parskip0ex
\tableofcontents
\endgroup

%%%%%%%%%%%%%%%%%%%%%%%%%%%%%%%%%%%%%%%%%%%%%%%%%%%%%%%%%%%%%%%%%%%%%%%%%%%%%%%%
%%%%%%%%%%%%%%%%%%%%%%%%%%%%%%%%%%%%%%%%%%%%%%%%%%%%%%%%%%%%%%%%%%%%%%%%%%%%%%%%
\section{Introduction}

\LaTeX{} provides a mechanism to structure a large document (such as a book)
into a main file and several child files (containing the chapters)
using the |\include| command.
This mechanism is beneficial for documents
which span hundreds of pages in order to
make the source file(s) more manageable.
Moreover, compilation can be restricted to
selected child files by means of the |\includeonly| command.
The latter feature can be used to reduce the compilation time while editing
(this was significantly more useful in the earlier days of \LaTeX{})
or to generate a smaller document which is easier to navigate.
Another application of |\includeonly| is to generate
documents consisting of selected parts of the complete document.

However, there are a few drawbacks of the plain |\include| mechanism:
\begin{itemize}
\item
The child files cannot be compiled on their own,
they can only be compiled via the main file.
A naive editing environment
(such as a text editor with an option
to have the current file processed by \LaTeX)
may require one to switch to the main file before compiling;
attempting to compile the child file produces errors.
\item
The main file must be modified (each time)
to adjust the |\includeonly| command
to the present needs. This easily leaves the main file in a messy state.
\item
The generated document will always carry the filename
of the main document. This is inconvenient if
several child files are to be compiled and
to be kept for distribution.
\end{itemize}

The present package provides a simple interface
to make child files individually compilable by \LaTeX{}.
Compiling a child file then has the same effect as compiling
the main file with an |\includeonly| command
to select the appropriate child.
Moreover the generated document will carry the name of the child
rather than the main file.
This resolves all three above issues.

This feature is meant to make the editing of books,
thesis documents and lecture notes somewhat more convenient.
However, the package can also be used efficiently for
composing a series of documents (such as exercise sheets)
which are typically distributed individually.
It then assists the author in generating the individual documents
(potentially in different versions)
as well as a document containing the collected series.
Another application is in developing style files
or other kinds of included material
where compilation of the style file could redirect
to a sample or test file.

%%%%%%%%%%%%%%%%%%%%%%%%%%%%%%%%%%%%%%%%%%%%%%%%%%%%%%%%%%%%%%%%%%%%%%%%%%%%%%%%
%%%%%%%%%%%%%%%%%%%%%%%%%%%%%%%%%%%%%%%%%%%%%%%%%%%%%%%%%%%%%%%%%%%%%%%%%%%%%%%%
\section{Usage}

First of all, the package \textsf{childdoc} is \emph{not} a standard
\LaTeXe{} |.sty| style file! Therefore it needs to be invoked in
a non-standard way.

%%%%%%%%%%%%%%%%%%%%%%%%%%%%%%%%%%%%%%%%%%%%%%%%%%%%%%%%%%%%%%%%%%%%%%%%%%%%%%%%
\subsection{Included Files}
\label{sec:include}

%%%%%%%%%%%%%%%%%%%%%%%%%%%%%%%%%%%%%%%%
\DescribeMacro{\childdocmain}
To use the package, add the commands
\begin{center}
\begin{tabular}{l}
|\input{childdoc.def}|\\
|\childdocmain{}|\\
\end{tabular}
\end{center}
at the very top of the main \LaTeX{} file,
in particular \emph{before} the |\documentclass| statement!
The argument of |\childdocmain| should be left empty
(but it must be present).

%%%%%%%%%%%%%%%%%%%%%%%%%%%%%%%%%%%%%%%%
\DescribeMacro{\childdocof}
Furthermore, add the commands
\begin{center}
\begin{tabular}{l}
|\input{childdoc.def}|\\
|\childdocof{|\textit{main}|}|\\
\end{tabular}
\end{center}
at the top of every child file \textit{child}
which is included by |\include{|\textit{child}|}|
from within the main file
(or at least for those files to be compiled individually).
The argument \textit{main} must be the filename of the main file.

There are a couple of
considerations in setting up the main and child documents:

%%%%%%%%%%%%%%%%%%%%%%%%%%%%%%%%%%%%%%%%
\paragraph{Restrictions.}

Please note the following restrictions:
\begin{itemize}
\item
|\childdocmain| must be called with one argument \textit{main}
to ensure compatibility with earlier version of the package.
It must either be empty (|\childdocmain{}|)
or precisely match the filename of the main file in which it is specified.
See \secref{sec:detection} for further information.
\item
The filename \textit{main} must be specified without the |.tex| extension.
\item
The filename \textit{main} is case sensitive
(even in case-insensitive file systems)
due to internal string comparison.
\item
The argument \textit{main} should be fully expanded, it cannot be a macro.
\item
Subdirectories and special characters should be avoided in filenames.
\item
The command |\childdocmain{|\textit{main}|}| must be followed by a whitespace.
It should not be followed immediately by another command
or by a comment mark `|%|'.
This is because the \TeX{} parser reads the token immediately following
the argument of |\childdocmain| and puts it
at the beginning of every child section;
however, a white\-space is ignored.
\end{itemize}

%%%%%%%%%%%%%%%%%%%%%%%%%%%%%%%%%%%%%%%%
\paragraph{Content of Main File.}

It is advisable to place all content in the child files included by |\include|.
Any output contained in the main file will appear in all child documents
unless suppressed manually;
it cannot be suppressed automatically by the |\includeonly| directive
and thus should normally be avoided.
A method to include some content in the main file
by means of conditional processing is described in \secref{sec:conditional}.

%%%%%%%%%%%%%%%%%%%%%%%%%%%%%%%%%%%%%%%%
\paragraph{Page Numbering.}

When only a part of the document is compiled,
the appropriate numbering of pages
(as well as other status parameters)
is determined from the |.aux| files.
The latter contain information from previous passes.
However this information needs to propagate through
all intermediate child documents.
Therefore the page numbering in child documents may well
be inconsistent until the complete document is compiled at least once.

A useful (if unconventional) way to always ensure a consistent
page numbering is to restart the numbering in each child document
and denote the pages by `\textit{child}|.|\textit{page}'
where \textit{child} represents the chapter/section number of the child file.
This can be achieved by the command
|\numberwithin{page}{|\textit{child}|}|
of the \textsf{amsmath} package
where \textit{child} can be |chapter| or |section|
depending on the chosen structuring.
Alternatively, one can modify the macro |\thepage| appropriately
and reset the counter |page| at the start of each child file.

%%%%%%%%%%%%%%%%%%%%%%%%%%%%%%%%%%%%%%%%%%%%%%%%%%%%%%%%%%%%%%%%%%%%%%%%%%%%%%%%
\subsection{Conditional Processing}
\label{sec:conditional}

The package provides a mechanism to compile different versions
of a document. To customise the versions further some conditional processing
can come in handy to distinguish which version is being compiled.
The package provides two macros to describe the compilation context:

%%%%%%%%%%%%%%%%%%%%%%%%%%%%%%%%%%%%%%%%
\DescribeMacro{\ifchilddoc}
The conditional |\ifchilddoc| distinguishes between the compilation of
child documents and the main document:
%
\begin{center}
|\ifchilddoc |\textit{child-code}| |[|\||else |\textit{main-code}]| \||fi|
\end{center}

%%%%%%%%%%%%%%%%%%%%%%%%%%%%%%%%%%%%%%%%
\DescribeMacro{\childdocname}
\DescribeMacro{\childdocjob}
The macro |\childdocname| contains the filename (without extension)
of the main or child file being processed.
Note that |\childdocjob| will always contain the name of the main file.

%%%%%%%%%%%%%%%%%%%%%%%%%%%%%%%%%%%%%%%%
\paragraph{Title Page.}

Conditional processing can be used to include a title or banner page
in the main document when proper precautions are taken.
Importantly, the code in the main file should ensure that the page counter
(as well as other status parameters which are stored in the |.aux| files)
takes the same value after the conditional processing.
Otherwise the page numbers may take divergent values
depending on which part is compiled.

For example, a title page could be declared by:
%
\begin{center}
\begin{tabular}{l}
|\ifchilddoc\||else|\\
|\addtocounter{page}{-1}|\\
\textit{code for title page}\\
|\newpage|\\
|\||fi|
\end{tabular}
\end{center}
%
A banner page for the child documents can be generated by:
%
\begin{center}
\begin{tabular}{l}
|\ifchilddoc|\\
|\addtocounter{page}{-1}|\\
\textit{code for banner page}\\
|\newpage|\\
|\||fi|
\end{tabular}
\end{center}
%
Here one could write a message such as:
\begin{center}
|This is the part \childdocname{} of \childdocjob{}.|
\end{center}

%%%%%%%%%%%%%%%%%%%%%%%%%%%%%%%%%%%%%%%%%%%%%%%%%%%%%%%%%%%%%%%%%%%%%%%%%%%%%%%%
\subsection{Flags}
\label{sec:flags}

The package makes it easy to generate different versions
of the main or child documents.
To this end compilation flags can be defined
and assigned different default values.
They will be particularly useful in conjunction
with the forwarding mechanism described in \secref{sec:forward}.

For example, it may be useful to have a flag |\version|
which can be set to |draft| or |final|.
The document source will contain some conditional code
depending on the value of |\version|.
Suppose further, the flag should default to |final| for the main file
and to |draft| for child files
which is a natural assignment for editing the document.
This is achieved by placing the following code
in the preamble of the main document
(below the |\childdocmain| directive):
%
\begin{center}
\begin{tabular}{l}
|\ifchilddoc|\\
|\providecommand{\version}{draft}|\\
|\||else|\\
|\providecommand{\version}{final}|\\
|\||fi|
\end{tabular}
\end{center}
%
The definition by |\providecommand| makes sure
that previous definitions are not overwritten.
Further statements |\providecommand{\version}{...}|
can thus be added before the above code to override it.

For the main file, one might add a line
(between |\childdocmain| and the above block)
%
\begin{center}
|%\ifchilddoc\||else\providecommand{\version}{draft}\||fi|
\end{center}
%
which can be uncommented to produce a draft version.
Likewise one can add a line to the very top of a child file
(above the |\childdocof{|\textit{main}|}| directive)
%
\begin{center}
|%\providecommand{\version}{final}|
\end{center}
%
which can be uncommented to produce the final version of this child document.

%%%%%%%%%%%%%%%%%%%%%%%%%%%%%%%%%%%%%%%%%%%%%%%%%%%%%%%%%%%%%%%%%%%%%%%%%%%%%%%%
\subsection{Forwarding}
\label{sec:forward}

Different versions of the main or child documents
using compilation flags as described in \secref{sec:flags}
can be (permanently) stored in different files
for convenient compilation, viewing and distribution.
To this end, the package defines a command
to pass on compilation to a different file:

%%%%%%%%%%%%%%%%%%%%%%%%%%%%%%%%%%%%%%%%
\DescribeMacro{\childdocforward}
The command |\childdocforward| redirects processing to
another source file:
%
\begin{center}
\begin{tabular}{l}
|\input{childdoc.def}|\\
|\childdocforward[|\textit{main}|]{|\textit{dest}|}|\\
\end{tabular}
\end{center}
%
The argument \textit{dest} is the destination file
(without extension).
It should be the main file or one of the child files.
Note that further \textsf{childdoc} directives
such as |\childdocof| and |\childdocforward|
in the indicated file will be processed in this form.
The optional argument \textit{main}
passes on directly to the main file \textit{main}
while pretending to compile the child \textit{dest}.
This form behaves as if \textit{dest}
issues |\childdocof{|\textit{main}|}| right away,
and no further \textsf{childdoc} directives will be processed.

%%%%%%%%%%%%%%%%%%%%%%%%%%%%%%%%%%%%%%%%
\DescribeMacro{\...prefix}
In the alternative form |\childdocforwardprefix|,
%
\begin{center}
\begin{tabular}{l}
|\input{childdoc.def}|\\
|\childdocforwardprefix[|\textit{main}|]{|\textit{prefix}|}{|\textit{dest}|}|
\end{tabular}
\end{center}
%
the destination file is determined by a pattern
depending on the current file:
To make this work, the current file must be called
`{\textit{prefix}\hspace{0.2em}\textit{suffix}}'
with \textit{prefix} matching precisely the argument.
Processing is then passed on to the file
`{\textit{dest}\hspace{0.2em}\textit{suffix}}'.
Surely, the same effect is achieved by
directly specifying the
argument `{\textit{dest}\hspace{0.2em}\textit{suffix}}'
in the first form.
However, that requires to set up a different file
for each child. With the alternative form of the command
all these files can have exactly the same content
which simplifies setting them up and maintaining them.

For example, the following file |draft.tex|
with a compilation flag |\version| as described in \secref{sec:flags}
compiles the main document as a draft:
%
\begin{center}
\begin{tabular}{l}
|\def\version{draft}|\\
|\input{childdoc.def}|\\
|\childdocforward{|\textit{main}|}|
\end{tabular}
\end{center}
%
Likewise, the following files |final|\textit{nn}|.tex|
compile the final version of the child document
|child|\textit{nn}|.tex|:
%
\begin{center}
\begin{tabular}{l}
|\def\version{final}|\\
|\input{childdoc.def}|\\
|\childdocforwardprefix{final}{child}|
\end{tabular}
\end{center}
%

Note that when several versions of a main file and/or of each child file
are to be generated, it may be convenient to set up a |Makefile| or
shell script to automatise the process.

%%%%%%%%%%%%%%%%%%%%%%%%%%%%%%%%%%%%%%%%%%%%%%%%%%%%%%%%%%%%%%%%%%%%%%%%%%%%%%%%
\subsection{Command Line Processing}
\label{sec:commandline}

The effect of redirection files can also be achieved by invoking
the \LaTeX{} compiler with a more elaborate command line.
Most conveniently this should be done as part
of a shell script or a |Makefile|.

When using \textsf{childdoc} in the main file, the following
command lines effectively perform a redirection
(note that depending on the shell being used,
backslashes may have to be doubled: `|\|' $\to$ `|\\|'):
%
\begin{center}
|... -jobname "|\textit{target}|" |\\|"|[\textit{flags}]%
|\input{childdoc.def}\childdocforward[|\textit{main}|]{|\textit{dest}|}"|
\end{center}
%
Here \textit{target} is the name of the output file,
\textit{main} is the name of the main file
and \textit{dest} is the name of the main or child file to be processed
(all filenames without extensions).
The optional argument \textit{main} can be omitted
if \textit{main} matches \textit{dest}.
Optionally, compilation \textit{flags} can be defined via |\def| commands.
This command line makes the \TeX{} engine believe
it is compiling the file \textit{target}
whose content is specified as the latter parameter.
The provided code then forwards the processing to
\textit{main} or \textit{dest} as described in \secref{sec:forward}.

%%%%%%%%%%%%%%%%%%%%%%%%%%%%%%%%%%%%%%%%%%%%%%%%%%%%%%%%%%%%%%%%%%%%%%%%%%%%%%%%
\subsection{Include by Input}
\label{sec:input}

Including child documents by |\include| has some restrictions by design.
Most notably, the content of a child document always occupies
its own set of pages; pages cannot be shared between child documents.
Usually, this behaviour makes perfect sense
because each child document contain an essential part of the document.
However, in some situations it may be desirable to compose
a document from a collection of parts
without having mandatory page breaks between then.
For this case, the package
provides a mechanism to include parts
by |\input| which can also be processed individually.
However, by construction this mechanism
requires manual handling of the content to be output.

%%%%%%%%%%%%%%%%%%%%%%%%%%%%%%%%%%%%%%%%
\DescribeMacro{\ifchilddocmanual}
The main file should be prepared as usual, see \secref{sec:include}.
However, the document body must make a distinction
between processing of an individual part and of the main document, e.g.:
%
\begin{center}
\begin{tabular}{l}
|\ifchilddocmanual|\\
|\input{\childdocname}|\\
|\||else|\\
\textit{document body with }|\input{|\textit{part}|}|\\
|\||fi|
\end{tabular}
\end{center}
%
The conditional |\ifchilddocmanual| is true whenever
a part to be included by |\input| is being compiled,
and the name of the part is stored in |\childdocname|.

%%%%%%%%%%%%%%%%%%%%%%%%%%%%%%%%%%%%%%%%
\DescribeMacro{\childdocby}
Each part to be included by |\input| should start with:
%
\begin{center}
\begin{tabular}{l}
|\input{childdoc.def}|\\
|\childdocby{|\textit{main}|}|\\
\end{tabular}
\end{center}
%
The directive |\childdocby| is similar to |\childdocof|
described in \secref{sec:include},
but the subsequent selection of content must be done manually.
To that end, both |\ifchilddoc| and |\ifchilddocmanual|
will be true upon processing of a part,
and the name of the part is stored in |\childdocname|.
Note that |\jobname| will be set to the filename of the current part
so that each part receives an individual |.aux| file
that does not interfere with the |.aux| file(s) of the main document.
This behaviour can be altered by the alternative form
|\childdocby[*]{|\textit{main}|}| (with a non-empty optional argument)
which uses the |.aux| file of the main document
by setting |\jobname| to \textit{main}.

%%%%%%%%%%%%%%%%%%%%%%%%%%%%%%%%%%%%%%%%%%%%%%%%%%%%%%%%%%%%%%%%%%%%%%%%%%%%%%%%
\subsection{Driver Development}
\label{sec:driver}

The \textsf{childdoc} mechanism can also be use for the development
of definition files such as \LaTeX{} styles or classes.
This case differs from the above setup with multiple parts
included by |\include| in that no |\includeonly| should be invoked.
This can be achieved by starting the include file
(before |\ProvidesPackage|) with:
%
\begin{center}
\begin{tabular}{l}
|\input{childdoc.def}|\\
|\childdocforward{|\textit{main}|}|\\
\end{tabular}
\end{center}
%
or alternatively with:
%
\begin{center}
\begin{tabular}{l}
|\input{childdoc.def}|\\
|\childdocby{|\textit{main}|}|\\
\end{tabular}
\end{center}
%
Both forms have slightly different effects as described above.
The main file is prepared as usual, see \secref{sec:include}.

%%%%%%%%%%%%%%%%%%%%%%%%%%%%%%%%%%%%%%%%%%%%%%%%%%%%%%%%%%%%%%%%%%%%%%%%%%%%%%%%
\subsection{Legacy Detection}
\label{sec:detection}

The directive |\childdocmain| in the main file can detect
whether the complete document or merely a child is to be compiled
even without using the directive |\childdocof|.
This method is deprecated because it is less robust
and there is no compelling reason to use it;
it is merely provided for backward compatibility
and it may be removed in future versions.

If the detection mechanism is to be used,
it is mandatory to correctly specify
the filename of the main file as the argument of |\childdocmain|:
%
\begin{center}
\begin{tabular}{l}
|\input{childdoc.def}|\\
|\childdocmain{|\textit{main}|}|\\
\end{tabular}
\end{center}
%
If |\jobname| does not match the argument \textit{main} of |\childdocmain|,
it is assumed that |\jobname| points to the child file to be compiled.
When using |\childdocmain| with the main file specified as argument,
it suffices to start a child file
with just |\input{|\textit{main}|}|
without loading of the package and using |\childdocof|.
If instead all processing is done
with the appropriate \textsf{childdoc} directives,
the argument of \textit{main} of |\childdocmain| can be empty.

An alternative version of the command line processing described
in \secref{sec:commandline} using the detection mechanism reads:
%
\begin{center}
|... -jobname "|\textit{target}|" "|[\textit{flags}]%
[|\def\jobname{|\textit{dest}|}|]|\input{|\textit{main}|}"|
\end{center}

%%%%%%%%%%%%%%%%%%%%%%%%%%%%%%%%%%%%%%%%%%%%%%%%%%%%%%%%%%%%%%%%%%%%%%%%%%%%%%%%
\subsection{Manual Code}
\label{sec:manual}

In case one cannot be certain whether the definitions file |childdoc.def|
is installed on the target \TeX{} distribution
and one prefers not to ship it,
it is conceivable to paste a few relevant commands into the sources.

To that end, drop all statements |\input{childdoc.def}|
and perform the replacements as outlined below.
Instead of |\childdocmain{|\textit{main}|}| add the following code
to the top of the main file:
%
\begin{center}
\begin{tabular}{l}
|\||ifdefined\childdocname\endinput\||fi\newif\ifchilddoc|\\
|\edef\childdocname{\scantokens\expandafter{\jobname\noexpand}}|\\
|\def\childdocmain{|\textit{main}|}\||ifx\childdocmain\childdocname\||else|\\
|\childdoctrue\includeonly{\childdocname}\let\jobname\childdocmain\||fi|\\
\end{tabular}
\end{center}
%
Instead of |\childdocof{|\textit{main}|}| just include the main file
at the top of each child file:
%
\begin{center}
|\input{|\textit{main}|}|
\end{center}
%
A simple redirection |\childdocforward{|\textit{dest}|}| is achieved by:
%
\begin{center}
|\def\jobname{|\textit{dest}|}\input{\jobname}|
\end{center}
%
The redirection with prefix
|\childdocforwardprefix[|\textit{prefix}|]{|\textit{dest}|}|
is accomplished by:
%
\begin{center}
\begin{tabular}{l}
|{\edef\jobname{\scantokens\expandafter{\jobname\noexpand}}|\\
|\def\redirectjob |\textit{prefix}|#1~~~{\gdef\jobname{|\textit{dest}|#1}}|\\
|\expandafter\redirectjob\jobname~~~}\input{\jobname}|
\end{tabular}
\end{center}

In an alternative approach,
child documents can be compiled by a specific command line
without additional code or specific definitions:
%
\begin{center}
|... -jobname "|\textit{target}|" "|[\textit{flags}]%
|\includeonly{|\textit{dest}|}\input{|\textit{main}|}"|
\end{center}
%

%%%%%%%%%%%%%%%%%%%%%%%%%%%%%%%%%%%%%%%%%%%%%%%%%%%%%%%%%%%%%%%%%%%%%%%%%%%%%%%%
%%%%%%%%%%%%%%%%%%%%%%%%%%%%%%%%%%%%%%%%%%%%%%%%%%%%%%%%%%%%%%%%%%%%%%%%%%%%%%%%
\section{Information}

%%%%%%%%%%%%%%%%%%%%%%%%%%%%%%%%%%%%%%%%%%%%%%%%%%%%%%%%%%%%%%%%%%%%%%%%%%%%%%%%
\subsection{Copyright}

Copyright \copyright{} 2017--2018 Niklas Beisert

This work may be distributed and/or modified under the
conditions of the \LaTeX{} Project Public License, either version 1.3
of this license or (at your option) any later version.
The latest version of this license is in
  \url{http://www.latex-project.org/lppl.txt}
and version 1.3 or later is part of all distributions of \LaTeX{}
version 2005/12/01 or later.

This work has the LPPL maintenance status `maintained'.

The Current Maintainer of this work is Niklas Beisert.

This work consists of the files |README.txt|, |childdoc.ins| and |childdoc.dtx|
as well as the derived files |childdoc.def|, |cdocsamp.tex|
with |cdocsch1.tex|, |cdocsch2.tex|, |cdocspt3.tex|, |cdocspt4.tex|,
|cdocsdrf.tex|, |cdocsfn1.tex|, |cdocsfn2.tex|
as well as |childdoc.pdf|.

%%%%%%%%%%%%%%%%%%%%%%%%%%%%%%%%%%%%%%%%%%%%%%%%%%%%%%%%%%%%%%%%%%%%%%%%%%%%%%%%
\subsection{Files and Installation}

The package consists of the files:
%
\begin{center}
\begin{tabular}{ll}
    |README.txt|   & readme file \\
    |childdoc.ins| & installation file \\
    |childdoc.dtx| & source file \\
    |childdoc.def| & definition file \\
    |cdocsamp.tex| & sample main file \\
    |cdocsch1.tex| & sample include file \\
    |cdocsch2.tex| & sample include file \\
    |cdocspt3.tex| & sample part file \\
    |cdocspt4.tex| & sample part file \\
    |cdocsdrf.tex| & sample redirection file \\
    |cdocsfn1.tex| & sample redirection file \\
    |cdocsfn2.tex| & sample redirection file \\
    |childdoc.pdf| & manual
\end{tabular}
\end{center}
%
The distribution consists of the files
|README.txt|, |childdoc.ins| and |childdoc.dtx|.
%
\begin{itemize}
\item
Run (pdf)\LaTeX{} on |childdoc.dtx|
to compile the manual |childdoc.pdf| (this file).
\item
Run \LaTeX{} on |childdoc.ins| to create the definitions file |childdoc.def|
and the sample |cdocsamp.tex| with include files
|cdocsch1.tex|, |cdocsch2.tex|, |cdocspt3.tex|, |cdocspt4.tex|,
|cdocsdrf.tex|, |cdocsfn1.tex|, |cdocsfn2.tex|.
Then copy the file |childdoc.def| to an appropriate directory of your \LaTeX{}
distribution, e.g.\ \textit{texmf-root}|/tex/latex/childdoc|.
\end{itemize}

%%%%%%%%%%%%%%%%%%%%%%%%%%%%%%%%%%%%%%%%%%%%%%%%%%%%%%%%%%%%%%%%%%%%%%%%%%%%%%%%
\subsection{Related CTAN Packages}

There are several other packages which offer a similar functionality:
%
\begin{itemize}
\item
The packages
\href{http://ctan.org/pkg/docmute}{\textsf{docmute}},
\href{http://ctan.org/pkg/includex}{\textsf{includex}} and
\href{http://ctan.org/pkg/standalone}{\textsf{standalone}}
provide commands to include only the document body of
a child file thus allowing both files to be compiled individually.
\item
The packages \href{http://ctan.org/pkg/subdocs}{\textsf{subdocs}}
and \href{http://ctan.org/pkg/subfiles}{\textsf{subfiles}}
provide structures in which the main and child documents can be
encapsulated and allowing them to be compiled individually.
The inclusion mechanism is different from the conventional |\include|.
\item
The package \href{http://ctan.org/pkg/combine}{\textsf{combine}}
is an elaborate solution to combine several documents into one.
\end{itemize}
%
See also the CTAN topic \href{http://ctan.org/topic/subdocs}{\textsf{subdocs}}
for further related packages.
The present package differs from the above solutions in that
a document structure constructed with the conventional |\include| mechanism
just needs two extra commands at the top of every file
such that all constituent files can be compiled individually.

%%%%%%%%%%%%%%%%%%%%%%%%%%%%%%%%%%%%%%%%%%%%%%%%%%%%%%%%%%%%%%%%%%%%%%%%%%%%%%%%
%\subsection{Feature Suggestions}
%
%The following is a list of features which may be useful for future
%versions of this package:
%%
%\begin{itemize}
%\item
%\ldots
%\end{itemize}

%%%%%%%%%%%%%%%%%%%%%%%%%%%%%%%%%%%%%%%%%%%%%%%%%%%%%%%%%%%%%%%%%%%%%%%%%%%%%%%%
\subsection{Revision History}

%%%%%%%%%%%%%%%%%%%%%%%%%%%%%%%%%%%%%%%%
\paragraph{v2.0:} 2018/12/30

\begin{itemize}
\item
immediate forward processing
\item
added |\childdocby| mechanism
\item
manual restructured
\end{itemize}

%%%%%%%%%%%%%%%%%%%%%%%%%%%%%%%%%%%%%%%%
\paragraph{v1.6:} 2018/01/17

\begin{itemize}
\item
application for development of include files
\item
corrections to manual
\end{itemize}

%%%%%%%%%%%%%%%%%%%%%%%%%%%%%%%%%%%%%%%%
\paragraph{v1.5:} 2017/05/21

\begin{itemize}
\item
more complete structuring introduced
\item
|\childdocof| introduced
\item
|\childdoc| renamed to |\childdocmain|
\item
|\childredirect| renamed to |\childdocforward| and |\childdocforwardprefix|
and functionality expanded
\end{itemize}

%%%%%%%%%%%%%%%%%%%%%%%%%%%%%%%%%%%%%%%%
\paragraph{v1.0:} 2017/04/27

\begin{itemize}
\item
manual and install package
\item
first version published on CTAN
\end{itemize}

%%%%%%%%%%%%%%%%%%%%%%%%%%%%%%%%%%%%%%%%
\paragraph{v0.6:} 2017/04/26

\begin{itemize}
\item
redirection mechanism added
\end{itemize}

%%%%%%%%%%%%%%%%%%%%%%%%%%%%%%%%%%%%%%%%
\paragraph{v0.5:} 2017/04/26

\begin{itemize}
\item
functionality in definition file
\end{itemize}


%%%%%%%%%%%%%%%%%%%%%%%%%%%%%%%%%%%%%%%%%%%%%%%%%%%%%%%%%%%%%%%%%%%%%%%%%%%%%%%%
%%%%%%%%%%%%%%%%%%%%%%%%%%%%%%%%%%%%%%%%%%%%%%%%%%%%%%%%%%%%%%%%%%%%%%%%%%%%%%%%
%%%%%%%%%%%%%%%%%%%%%%%%%%%%%%%%%%%%%%%%%%%%%%%%%%%%%%%%%%%%%%%%%%%%%%%%%%%%%%%%
\appendix

\settowidth\MacroIndent{\rmfamily\scriptsize 000\ }

 \DocInput{childdoc.dtx}

\end{document}
%</driver>
% \fi
%
% %%%%%%%%%%%%%%%%%%%%%%%%%%%%%%%%%%%%%%%%%%%%%%%%%%%%%%%%%%%%%%%%%%%%%%%%%%%%%%
% %%%%%%%%%%%%%%%%%%%%%%%%%%%%%%%%%%%%%%%%%%%%%%%%%%%%%%%%%%%%%%%%%%%%%%%%%%%%%%
% \section{Sample}
%\iffalse
%<*samplemain>
%\fi
%
% The following presents a sample document
% with two chapters, two parts, a title page,
% a compile flag as well as three forwarding files to set the flag.
% It consists of eight |.tex| files:
% \begin{center}
% \begin{tabular}{ll}
% |cdocsamp.tex|&main file\\
% |cdocsch1.tex|&include file for chapter 1\\
% |cdocsch2.tex|&include file for chapter 2\\
% |cdocspt3.tex|&include file for part 3\\
% |cdocspt4.tex|&include file for part 4\\
% |cdocsdrf.tex|&forwarding file for main file in draft mode\\
% |cdocsfi1.tex|&forwarding file for final version of chapter 1\\
% |cdocsfi2.tex|&forwarding file for final version of chapter 2\\
% \end{tabular}
% \end{center}
% Each of the eight files can be compiled directly by the \LaTeX{} compiler.
%
% %%%%%%%%%%%%%%%%%%%%%%%%%%%%%%%%%%%%%%
% \paragraph{Main File.}
%
% The main file is called |cdocsamp.tex|.
%
% Load the \textsf{childdoc} definitions and
% declare the filename for the main document:
%    \begin{macrocode}
\input{childdoc.def}
\childdocmain{}
%    \end{macrocode}

% Optional override for |\version| flag:
%    \begin{macrocode}
%%\ifchilddoc\else\providecommand{\version}{draft}\fi
%    \end{macrocode}

% Define the default values for the |\version| flag
% (|final| for the main file and |draft| for childs):
%    \begin{macrocode}
\ifchilddoc
\providecommand{\version}{draft}
\else
\providecommand{\version}{final}
\fi
%    \end{macrocode}

% Load the standard document class:
%    \begin{macrocode}
\documentclass[12pt]{article}
%    \end{macrocode}

% Start the document body:
%    \begin{macrocode}
\begin{document}
%    \end{macrocode}

% Declare a title page.
% Print title, part of document being processed and version flag:
%    \begin{macrocode}
\addtocounter{page}{-1}
\begin{center}
{\LARGE\bfseries{}childdoc example\par}
\vspace{1cm}
\ifchilddoc
\ifchilddocmanual part\else chapter\fi:
`\childdocname' of `\childdocjob'\par
\else
main document: `\childdocjob'\par
\fi
version: \version\par
\end{center}
\newpage
%    \end{macrocode}

% Manually include selected file,
% otherwise process as usual:
%    \begin{macrocode}
\ifchilddocmanual
\section*{part `\childdocname'}
\input{\childdocname}
\else
%    \end{macrocode}

% Include the two chapters:
%    \begin{macrocode}
\include{cdocsch1}
\include{cdocsch2}
%    \end{macrocode}

% Include the two parts unless only chapters should be displayed:
%    \begin{macrocode}
\ifchilddoc\else
\section{part three}
\input{cdocspt3}
\section{part four}
\input{cdocspt4}
\fi
%    \end{macrocode}

% Process as usual until here:
%    \begin{macrocode}
\fi
%    \end{macrocode}

% End of document body:
%    \begin{macrocode}
\end{document}
%    \end{macrocode}
%\iffalse
%</samplemain>
%\fi
%
% %%%%%%%%%%%%%%%%%%%%%%%%%%%%%%%%%%%%%%
% \paragraph{Chapter Include Files.}
%
% The include files are called |cdocsch1.tex| and |cdocsch2.tex|.
%
%\iffalse
%<*samplechap1|samplechap2>
%\fi

% Optional override for |\version| flag:
%    \begin{macrocode}
%%\providecommand{\version}{final}
%    \end{macrocode}

% Include the main document:
%    \begin{macrocode}
\input{childdoc.def}
\childdocof{cdocsamp}
%    \end{macrocode}

%\iffalse
%</samplechap1|samplechap2>
%\fi
%
%\iffalse
%<*samplechap1>
%\fi
% Some text for chapter 1:
%    \begin{macrocode}
\section{one}
some text in chapter one
%    \end{macrocode}

%\iffalse
%</samplechap1>
%\fi
% Some text for chapter 2:
%\iffalse
%<*samplechap2>
%\fi
%    \begin{macrocode}
\section{two}
more text in chapter two
%    \end{macrocode}

%\iffalse
%</samplechap2>
%\fi
%
% %%%%%%%%%%%%%%%%%%%%%%%%%%%%%%%%%%%%%%
% \paragraph{Part Include Files.}
%
% The include files are called |cdocspt3.tex| and |cdocspt4.tex|.
%
%\iffalse
%<*samplepart3|samplepart4>
%\fi

% Optional override for |\version| flag:
%    \begin{macrocode}
%%\providecommand{\version}{final}
%    \end{macrocode}

% Include the main document:
%    \begin{macrocode}
\input{childdoc.def}
\childdocby{cdocsamp}
%    \end{macrocode}

%\iffalse
%</samplepart3|samplepart4>
%\fi
%
%\iffalse
%<*samplepart3>
%\fi
% Some text for part 3:
%    \begin{macrocode}
some text in part three
%    \end{macrocode}

%\iffalse
%</samplepart3>
%\fi
% Some text for part 4:
%\iffalse
%<*samplepart4>
%\fi
%    \begin{macrocode}
more text in part four
%    \end{macrocode}

%\iffalse
%</samplepart4>
%\fi
%
% %%%%%%%%%%%%%%%%%%%%%%%%%%%%%%%%%%%%%%
% \paragraph{Forwarding for a Complete Draft.}
%
% The following forwarding file |cdocsdrf.tex|
% compiles the main document in draft mode:
%\iffalse
%<*sampledraft>
%\fi
%    \begin{macrocode}
\def\version{draft}
\input{childdoc.def}
\childdocforward{cdocsamp}
%    \end{macrocode}

%\iffalse
%</sampledraft>
%\fi
%
% %%%%%%%%%%%%%%%%%%%%%%%%%%%%%%%%%%%%%%
% \paragraph{Forwarding for Final Version of the Chapters.}
%
% The following forwarding files |cdocsfn1.tex| and |cdocsfn2.tex|
% (with identical content)
% compile the final versions of the child documents
% |cdocsch1.tex| and |cdocsch2.tex|, respectively:
%\iffalse
%<*samplefinal>
%\fi
%    \begin{macrocode}
\def\version{final}
\input{childdoc.def}
\childdocforwardprefix[cdocsamp]{cdocsfn}{cdocsch}
%    \end{macrocode}

%\iffalse
%</samplefinal>
%\fi
%
% %%%%%%%%%%%%%%%%%%%%%%%%%%%%%%%%%%%%%%
% \paragraph{Command Line Processing.}
%
% The following three command lines generate the output files
% |cdocscld|, |cdocscl1| and |cdocscl2|
% which should be identical to
% |cdocsdrf|, |cdocsch1| and |cdocsfn2|, respectively:
% \begin{center}
% \begin{tabular}{l}
% |latex -jobname cdocscld \|\\
% |  "\def\version{draft}\input{childdoc.def}\childdocforward{cdocsamp}"|\\
% |latex -jobname cdocscl1 \|\\
% |  "\input{childdoc.def}\childdocforward[cdocsamp]{cdocsch1}"|\\
% |latex -jobname cdocscl2 \|\\
% |  "\def\version{final}\input{childdoc.def}\childdocforward{cdocsch2}"|
% \end{tabular}
% \end{center}
% Note that the trailing backslash on each first line
% merely continues the input to the second line
% (for convenient cut ant paste).
% Furthermore, the command |latex| can be replaced by any
% of its alternative versions such as |pdflatex|.
%
% %%%%%%%%%%%%%%%%%%%%%%%%%%%%%%%%%%%%%%%%%%%%%%%%%%%%%%%%%%%%%%%%%%%%%%%%%%%%%%
% %%%%%%%%%%%%%%%%%%%%%%%%%%%%%%%%%%%%%%%%%%%%%%%%%%%%%%%%%%%%%%%%%%%%%%%%%%%%%%
% \section{Implementation}
%\iffalse
%<*package>
%\fi
%
% This section describes the definitions file |childdoc.def|.

% The definitions cannot be loaded using |\usepackage| or |\RequirePackage|
% which has a mechanism to prevent loading a style file more than once.
% When loading the definitions by means of |\input|
% multiple instances have to be prevented manually:
%\iffalse
%This code needs to be before the `\ProvidesFile' directive
%which is defined at the beginning of this file.
%Therefore it is also placed there and commented out here.
%</package>
%<*discard>
%\fi
%    \begin{macrocode}
\ifdefined\childdocmain\endinput\fi
%    \end{macrocode}
%\iffalse
%</discard>
%<*package>
%\fi
%
% \macro{\ifchilddoc}
% \macro{\ifchilddocmanual}
% The conditional |\ifchilddoc| tells whether a
% child (true) or main (false) document is being compiled.
% The conditional |\ifchilddocmanual| tells whether
% the |\includeonly| mechanism is used (false) or
% the selection of child files must be performed manually (true).
% The definitions initialise to false:
%    \begin{macrocode}
\newif\ifchilddoc
\newif\ifchilddocmanual
%    \end{macrocode}

% \macro{\childdocname}
% \macro{\childdocjob}
% The macro |\childdocname| stores the name of the main document
% to be compiled. The macro |\childdocjob| stores the name of
% the document on which the \LaTeX{} compiler was originally invoked.
% The content of |\jobname| cannot be compared
% to filenames specified in the source due to different catcodes.
% The following code rescans |\jobname|, stores the result
% in |\childdocname| and saves a copy in |\childdocjob|:
%    \begin{macrocode}
\edef\childdocname{\scantokens\expandafter{\jobname\noexpand}}
\let\childdocjob\childdocname
%    \end{macrocode}

% \macro{\childdocdisable}
% The macro |\childdocdisable| prevents the main file
% from being processed more than once.
% At this stage, the main document command |\childdocmain|
% is assumed to be called once again where it should do nothing.
% Any subsequent call to it should prevent
% a secondary processing of the main document
% It overwrites the forwarding commands
% |\childdocof| and |\childdocforward|
% with empty macros to prevent further inclusions of the main document:
%    \begin{macrocode}
\newcommand{\childdocdisable}
{
  \renewcommand{\childdocmain}[1]{\renewcommand{\childdocmain}[1]{\endinput}}
  \renewcommand{\childdocof}[1]{}
  \renewcommand{\childdocby}[2][]{}
  \renewcommand{\childdocforward}[2][]{}
  \renewcommand{\childdocdisable}{}
}
%    \end{macrocode}

% \macro{\childdocmain}
% The macro |\childdocmain| is to be called at the top of the main file
% with nothing or the main filename (without extension) as argument.
% First, it breaks loops.
% If the argument is not empty and does not match |\childdocname|
% (which is set by the first inclusion of |childdoc.def|),
% |\ifchilddoc| is set to true, |\includeonly| is applied to the child file
% and |\jobname| is set to the main file
% (for proper handling of |.aux| files):
%    \begin{macrocode}
\newcommand{\childdocmain}[1]
{
  \childdocdisable\childdocmain{}
  \if?#1?\else
    \begingroup
      \def\childdoctmp{#1}
      \ifx\childdoctmp\childdocname
        \def\childdoctmp{}
      \else
        \def\childdoctmp
        {
          \childdoctrue
          \includeonly{\childdocname}
          \def\childdocjob{#1}
          \def\jobname{#1}
        }
      \fi
      \expandafter
    \endgroup
    \childdoctmp
  \fi
}
%    \end{macrocode}

% \macro{\childdocof}
% The command |\childdocof| redirects
% compilation to the main file |#1|.
%    \begin{macrocode}
\newcommand{\childdocof}[1]
{
  \childdocdisable
  \childdoctrue
  \includeonly{\childdocname}
  \def\jobname{#1}
  \def\childdocjob{#1}
  \input{#1}
}
%    \end{macrocode}

% \macro{\childdocby}
% The command |\childdocby| ....
%    \begin{macrocode}
\newcommand{\childdocby}[2][]
{
  \childdocdisable
  \childdoctrue
  \childdocmanualtrue
  \if?#1?\else
    \def\jobname{#2}
  \fi
  \def\childdocjob{#2}
  \input{#2}
  \endinput
}
%    \end{macrocode}

% \macro{\childdocforward}
% The command |\childdocforward| redirects
% compilation to the main file or
% (if the optional argument is given) a child file.
% Parameters are set as if the main file
% or a child file starting with |\childdocof| was compiled.
% Then compilation is handed over to the main file:
%    \begin{macrocode}
\newcommand{\childdocforward}[2][]
{
  \begingroup
    \if?#1?
      \def\childdoctmp
      {
        \def\childdocname{#2}
        \def\childdocjob{#2}
        \def\jobname{#2}
        \input{#2}
        \endinput
      }
    \else
      \def\childdoctmp
      {
        \childdocdisable
        \def\childdocname{#2}
        \childdoctrue
        \includeonly{#2}
        \def\childdocjob{#1}
        \def\jobname{#1}
        \input{#1}
        \endinput
      }
    \fi
    \expandafter
  \endgroup
  \childdoctmp
}
%    \end{macrocode}

% \macro{\childdocforwardprefix}
% The command |\childdocforwardprefix| redirects
% compilation to the main or a child file by means of a pattern.
% The prefix |#1| in the current filename is replaced by |#2|
% and the suffix of the current filename is kept
% (it is assumed that the filename does not contain the substring `|~~~|'
% which is used as a delimiter).
% Compilation is handed over to the new file by |\childdocforward|:
%    \begin{macrocode}
\newcommand{\childdocforwardprefix}[3][]
{
  \begingroup
    \def\childdocextract #2##1~~~{\def\childdoctmp{\childdocforward[#1]{#3##1}}}
    \expandafter\childdocextract\childdocname~~~
    \expandafter
  \endgroup
  \childdoctmp
}
%    \end{macrocode}

% \macro{\childdoc}
% The deprecated macro |\childdoc| is a legacy version of |\childdocmain|:
%    \begin{macrocode}
\newcommand{\childdoc}{\childdocmain}
%    \end{macrocode}

% \macro{\childdocredirect}
% The deprecated macro |\childdocredirect| is a legacy version
% of |\childdocforward| and |\childdocforwardprefix|:
%    \begin{macrocode}
\newcommand{\childdocredirect}[2][]
{
  \begingroup
    \if?#1?
      \def\childdoctmp{\childdocforward{#2}}
    \else
      \def\childdoctmp{\childdocforwardprefix{#1}{#2}}
    \fi
    \expandafter
  \endgroup
  \childdoctmp
}
%    \end{macrocode}

%\iffalse
%</package>
%\fi
%
\endinput
|\\
|\childdocforward{|\textit{main}|}|
\end{tabular}
\end{center}
%
Likewise, the following files |final|\textit{nn}|.tex|
compile the final version of the child document
|child|\textit{nn}|.tex|:
%
\begin{center}
\begin{tabular}{l}
|\def\version{final}|\\
|% \iffalse
%
% childdoc.dtx Copyright (C) 2017-2018 Niklas Beisert
%
% This work may be distributed and/or modified under the
% conditions of the LaTeX Project Public License, either version 1.3
% of this license or (at your option) any later version.
% The latest version of this license is in
%   http://www.latex-project.org/lppl.txt
% and version 1.3 or later is part of all distributions of LaTeX
% version 2005/12/01 or later.
%
% This work has the LPPL maintenance status `maintained'.
%
% The Current Maintainer of this work is Niklas Beisert.
%
% This work consists of the files childdoc.dtx and childdoc.ins
% and the derived files childdoc.def and cdocsamp.tex with
% cdocsch1.tex, cdocsch2.tex, cdocsdrf.tex, cdocsfn1.tex, cdocsfn2.tex.
%
%<package>\ifdefined\childdocmain\endinput\fi
%<package>\ProvidesFile{childdoc.def}[2018/12/30 v2.0 child document driver]
%<samplemain>\ProvidesFile{cdocsamp.tex}[2018/12/30 v2.0 sample for childdoc]
%<*driver>
%\ProvidesFile{childdoc.drv}[2018/12/30 v2.0 childdoc reference manual file]
\PassOptionsToClass{10pt,a4paper}{article}
\documentclass{ltxdoc}

\usepackage[margin=35mm]{geometry}
\usepackage{hyperref}
\usepackage{hyperxmp}
\usepackage[usenames]{color}

\hypersetup{colorlinks=true}
\hypersetup{pdfstartview=FitH}
\hypersetup{pdfpagemode=UseNone}
\hypersetup{pdfsource={}}
\hypersetup{pdflang={en-UK}}
\hypersetup{pdfcopyright={Copyright 2017-2018 Niklas Beisert.
  This work may be distributed and/or modified under the
  conditions of the LaTeX Project Public License, either version 1.3
  of this license or (at your option) any later version.}}
\hypersetup{pdflicenseurl={http://www.latex-project.org/lppl.txt}}
\hypersetup{pdfcontactaddress={ETH Zurich, ITP, HIT K,
  Wolfgang-Pauli-Strasse 27}}
\hypersetup{pdfcontactpostcode={8093}}
\hypersetup{pdfcontactcity={Zurich}}
\hypersetup{pdfcontactcountry={Switzerland}}
\hypersetup{pdfcontactemail={nbeisert@itp.phys.ethz.ch}}
\hypersetup{pdfcontacturl={http://people.phys.ethz.ch/\xmptilde nbeisert/}}

\newcommand{\secref}[1]{\hyperref[#1]{section \ref*{#1}}}

\parskip1ex
\parindent0pt
\let\olditemize\itemize
\def\itemize{\olditemize\parskip0pt}

\begin{document}

\title{The \textsf{childdoc} Package}
\hypersetup{pdftitle={The childdoc Package}}
\author{Niklas Beisert\\[2ex]
  Institut f\"ur Theoretische Physik\\
  Eidgen\"ossische Technische Hochschule Z\"urich\\
  Wolfgang-Pauli-Strasse 27, 8093 Z\"urich, Switzerland\\[1ex]
  \href{mailto:nbeisert@itp.phys.ethz.ch}
  {\texttt{nbeisert@itp.phys.ethz.ch}}}
\hypersetup{pdfauthor={Niklas Beisert}}
\hypersetup{pdfsubject={Manual for the LaTeX2e Package childdoc}}
\date{30 December 2018, \textsf{v2.0}}
\maketitle

\begin{abstract}\noindent
\textsf{childdoc} is a \LaTeXe{} package
that enables the direct compilation
of document sections included by |\include|
to individual files.
\end{abstract}

\begingroup
\parskip0ex
\tableofcontents
\endgroup

%%%%%%%%%%%%%%%%%%%%%%%%%%%%%%%%%%%%%%%%%%%%%%%%%%%%%%%%%%%%%%%%%%%%%%%%%%%%%%%%
%%%%%%%%%%%%%%%%%%%%%%%%%%%%%%%%%%%%%%%%%%%%%%%%%%%%%%%%%%%%%%%%%%%%%%%%%%%%%%%%
\section{Introduction}

\LaTeX{} provides a mechanism to structure a large document (such as a book)
into a main file and several child files (containing the chapters)
using the |\include| command.
This mechanism is beneficial for documents
which span hundreds of pages in order to
make the source file(s) more manageable.
Moreover, compilation can be restricted to
selected child files by means of the |\includeonly| command.
The latter feature can be used to reduce the compilation time while editing
(this was significantly more useful in the earlier days of \LaTeX{})
or to generate a smaller document which is easier to navigate.
Another application of |\includeonly| is to generate
documents consisting of selected parts of the complete document.

However, there are a few drawbacks of the plain |\include| mechanism:
\begin{itemize}
\item
The child files cannot be compiled on their own,
they can only be compiled via the main file.
A naive editing environment
(such as a text editor with an option
to have the current file processed by \LaTeX)
may require one to switch to the main file before compiling;
attempting to compile the child file produces errors.
\item
The main file must be modified (each time)
to adjust the |\includeonly| command
to the present needs. This easily leaves the main file in a messy state.
\item
The generated document will always carry the filename
of the main document. This is inconvenient if
several child files are to be compiled and
to be kept for distribution.
\end{itemize}

The present package provides a simple interface
to make child files individually compilable by \LaTeX{}.
Compiling a child file then has the same effect as compiling
the main file with an |\includeonly| command
to select the appropriate child.
Moreover the generated document will carry the name of the child
rather than the main file.
This resolves all three above issues.

This feature is meant to make the editing of books,
thesis documents and lecture notes somewhat more convenient.
However, the package can also be used efficiently for
composing a series of documents (such as exercise sheets)
which are typically distributed individually.
It then assists the author in generating the individual documents
(potentially in different versions)
as well as a document containing the collected series.
Another application is in developing style files
or other kinds of included material
where compilation of the style file could redirect
to a sample or test file.

%%%%%%%%%%%%%%%%%%%%%%%%%%%%%%%%%%%%%%%%%%%%%%%%%%%%%%%%%%%%%%%%%%%%%%%%%%%%%%%%
%%%%%%%%%%%%%%%%%%%%%%%%%%%%%%%%%%%%%%%%%%%%%%%%%%%%%%%%%%%%%%%%%%%%%%%%%%%%%%%%
\section{Usage}

First of all, the package \textsf{childdoc} is \emph{not} a standard
\LaTeXe{} |.sty| style file! Therefore it needs to be invoked in
a non-standard way.

%%%%%%%%%%%%%%%%%%%%%%%%%%%%%%%%%%%%%%%%%%%%%%%%%%%%%%%%%%%%%%%%%%%%%%%%%%%%%%%%
\subsection{Included Files}
\label{sec:include}

%%%%%%%%%%%%%%%%%%%%%%%%%%%%%%%%%%%%%%%%
\DescribeMacro{\childdocmain}
To use the package, add the commands
\begin{center}
\begin{tabular}{l}
|\input{childdoc.def}|\\
|\childdocmain{}|\\
\end{tabular}
\end{center}
at the very top of the main \LaTeX{} file,
in particular \emph{before} the |\documentclass| statement!
The argument of |\childdocmain| should be left empty
(but it must be present).

%%%%%%%%%%%%%%%%%%%%%%%%%%%%%%%%%%%%%%%%
\DescribeMacro{\childdocof}
Furthermore, add the commands
\begin{center}
\begin{tabular}{l}
|\input{childdoc.def}|\\
|\childdocof{|\textit{main}|}|\\
\end{tabular}
\end{center}
at the top of every child file \textit{child}
which is included by |\include{|\textit{child}|}|
from within the main file
(or at least for those files to be compiled individually).
The argument \textit{main} must be the filename of the main file.

There are a couple of
considerations in setting up the main and child documents:

%%%%%%%%%%%%%%%%%%%%%%%%%%%%%%%%%%%%%%%%
\paragraph{Restrictions.}

Please note the following restrictions:
\begin{itemize}
\item
|\childdocmain| must be called with one argument \textit{main}
to ensure compatibility with earlier version of the package.
It must either be empty (|\childdocmain{}|)
or precisely match the filename of the main file in which it is specified.
See \secref{sec:detection} for further information.
\item
The filename \textit{main} must be specified without the |.tex| extension.
\item
The filename \textit{main} is case sensitive
(even in case-insensitive file systems)
due to internal string comparison.
\item
The argument \textit{main} should be fully expanded, it cannot be a macro.
\item
Subdirectories and special characters should be avoided in filenames.
\item
The command |\childdocmain{|\textit{main}|}| must be followed by a whitespace.
It should not be followed immediately by another command
or by a comment mark `|%|'.
This is because the \TeX{} parser reads the token immediately following
the argument of |\childdocmain| and puts it
at the beginning of every child section;
however, a white\-space is ignored.
\end{itemize}

%%%%%%%%%%%%%%%%%%%%%%%%%%%%%%%%%%%%%%%%
\paragraph{Content of Main File.}

It is advisable to place all content in the child files included by |\include|.
Any output contained in the main file will appear in all child documents
unless suppressed manually;
it cannot be suppressed automatically by the |\includeonly| directive
and thus should normally be avoided.
A method to include some content in the main file
by means of conditional processing is described in \secref{sec:conditional}.

%%%%%%%%%%%%%%%%%%%%%%%%%%%%%%%%%%%%%%%%
\paragraph{Page Numbering.}

When only a part of the document is compiled,
the appropriate numbering of pages
(as well as other status parameters)
is determined from the |.aux| files.
The latter contain information from previous passes.
However this information needs to propagate through
all intermediate child documents.
Therefore the page numbering in child documents may well
be inconsistent until the complete document is compiled at least once.

A useful (if unconventional) way to always ensure a consistent
page numbering is to restart the numbering in each child document
and denote the pages by `\textit{child}|.|\textit{page}'
where \textit{child} represents the chapter/section number of the child file.
This can be achieved by the command
|\numberwithin{page}{|\textit{child}|}|
of the \textsf{amsmath} package
where \textit{child} can be |chapter| or |section|
depending on the chosen structuring.
Alternatively, one can modify the macro |\thepage| appropriately
and reset the counter |page| at the start of each child file.

%%%%%%%%%%%%%%%%%%%%%%%%%%%%%%%%%%%%%%%%%%%%%%%%%%%%%%%%%%%%%%%%%%%%%%%%%%%%%%%%
\subsection{Conditional Processing}
\label{sec:conditional}

The package provides a mechanism to compile different versions
of a document. To customise the versions further some conditional processing
can come in handy to distinguish which version is being compiled.
The package provides two macros to describe the compilation context:

%%%%%%%%%%%%%%%%%%%%%%%%%%%%%%%%%%%%%%%%
\DescribeMacro{\ifchilddoc}
The conditional |\ifchilddoc| distinguishes between the compilation of
child documents and the main document:
%
\begin{center}
|\ifchilddoc |\textit{child-code}| |[|\||else |\textit{main-code}]| \||fi|
\end{center}

%%%%%%%%%%%%%%%%%%%%%%%%%%%%%%%%%%%%%%%%
\DescribeMacro{\childdocname}
\DescribeMacro{\childdocjob}
The macro |\childdocname| contains the filename (without extension)
of the main or child file being processed.
Note that |\childdocjob| will always contain the name of the main file.

%%%%%%%%%%%%%%%%%%%%%%%%%%%%%%%%%%%%%%%%
\paragraph{Title Page.}

Conditional processing can be used to include a title or banner page
in the main document when proper precautions are taken.
Importantly, the code in the main file should ensure that the page counter
(as well as other status parameters which are stored in the |.aux| files)
takes the same value after the conditional processing.
Otherwise the page numbers may take divergent values
depending on which part is compiled.

For example, a title page could be declared by:
%
\begin{center}
\begin{tabular}{l}
|\ifchilddoc\||else|\\
|\addtocounter{page}{-1}|\\
\textit{code for title page}\\
|\newpage|\\
|\||fi|
\end{tabular}
\end{center}
%
A banner page for the child documents can be generated by:
%
\begin{center}
\begin{tabular}{l}
|\ifchilddoc|\\
|\addtocounter{page}{-1}|\\
\textit{code for banner page}\\
|\newpage|\\
|\||fi|
\end{tabular}
\end{center}
%
Here one could write a message such as:
\begin{center}
|This is the part \childdocname{} of \childdocjob{}.|
\end{center}

%%%%%%%%%%%%%%%%%%%%%%%%%%%%%%%%%%%%%%%%%%%%%%%%%%%%%%%%%%%%%%%%%%%%%%%%%%%%%%%%
\subsection{Flags}
\label{sec:flags}

The package makes it easy to generate different versions
of the main or child documents.
To this end compilation flags can be defined
and assigned different default values.
They will be particularly useful in conjunction
with the forwarding mechanism described in \secref{sec:forward}.

For example, it may be useful to have a flag |\version|
which can be set to |draft| or |final|.
The document source will contain some conditional code
depending on the value of |\version|.
Suppose further, the flag should default to |final| for the main file
and to |draft| for child files
which is a natural assignment for editing the document.
This is achieved by placing the following code
in the preamble of the main document
(below the |\childdocmain| directive):
%
\begin{center}
\begin{tabular}{l}
|\ifchilddoc|\\
|\providecommand{\version}{draft}|\\
|\||else|\\
|\providecommand{\version}{final}|\\
|\||fi|
\end{tabular}
\end{center}
%
The definition by |\providecommand| makes sure
that previous definitions are not overwritten.
Further statements |\providecommand{\version}{...}|
can thus be added before the above code to override it.

For the main file, one might add a line
(between |\childdocmain| and the above block)
%
\begin{center}
|%\ifchilddoc\||else\providecommand{\version}{draft}\||fi|
\end{center}
%
which can be uncommented to produce a draft version.
Likewise one can add a line to the very top of a child file
(above the |\childdocof{|\textit{main}|}| directive)
%
\begin{center}
|%\providecommand{\version}{final}|
\end{center}
%
which can be uncommented to produce the final version of this child document.

%%%%%%%%%%%%%%%%%%%%%%%%%%%%%%%%%%%%%%%%%%%%%%%%%%%%%%%%%%%%%%%%%%%%%%%%%%%%%%%%
\subsection{Forwarding}
\label{sec:forward}

Different versions of the main or child documents
using compilation flags as described in \secref{sec:flags}
can be (permanently) stored in different files
for convenient compilation, viewing and distribution.
To this end, the package defines a command
to pass on compilation to a different file:

%%%%%%%%%%%%%%%%%%%%%%%%%%%%%%%%%%%%%%%%
\DescribeMacro{\childdocforward}
The command |\childdocforward| redirects processing to
another source file:
%
\begin{center}
\begin{tabular}{l}
|\input{childdoc.def}|\\
|\childdocforward[|\textit{main}|]{|\textit{dest}|}|\\
\end{tabular}
\end{center}
%
The argument \textit{dest} is the destination file
(without extension).
It should be the main file or one of the child files.
Note that further \textsf{childdoc} directives
such as |\childdocof| and |\childdocforward|
in the indicated file will be processed in this form.
The optional argument \textit{main}
passes on directly to the main file \textit{main}
while pretending to compile the child \textit{dest}.
This form behaves as if \textit{dest}
issues |\childdocof{|\textit{main}|}| right away,
and no further \textsf{childdoc} directives will be processed.

%%%%%%%%%%%%%%%%%%%%%%%%%%%%%%%%%%%%%%%%
\DescribeMacro{\...prefix}
In the alternative form |\childdocforwardprefix|,
%
\begin{center}
\begin{tabular}{l}
|\input{childdoc.def}|\\
|\childdocforwardprefix[|\textit{main}|]{|\textit{prefix}|}{|\textit{dest}|}|
\end{tabular}
\end{center}
%
the destination file is determined by a pattern
depending on the current file:
To make this work, the current file must be called
`{\textit{prefix}\hspace{0.2em}\textit{suffix}}'
with \textit{prefix} matching precisely the argument.
Processing is then passed on to the file
`{\textit{dest}\hspace{0.2em}\textit{suffix}}'.
Surely, the same effect is achieved by
directly specifying the
argument `{\textit{dest}\hspace{0.2em}\textit{suffix}}'
in the first form.
However, that requires to set up a different file
for each child. With the alternative form of the command
all these files can have exactly the same content
which simplifies setting them up and maintaining them.

For example, the following file |draft.tex|
with a compilation flag |\version| as described in \secref{sec:flags}
compiles the main document as a draft:
%
\begin{center}
\begin{tabular}{l}
|\def\version{draft}|\\
|\input{childdoc.def}|\\
|\childdocforward{|\textit{main}|}|
\end{tabular}
\end{center}
%
Likewise, the following files |final|\textit{nn}|.tex|
compile the final version of the child document
|child|\textit{nn}|.tex|:
%
\begin{center}
\begin{tabular}{l}
|\def\version{final}|\\
|\input{childdoc.def}|\\
|\childdocforwardprefix{final}{child}|
\end{tabular}
\end{center}
%

Note that when several versions of a main file and/or of each child file
are to be generated, it may be convenient to set up a |Makefile| or
shell script to automatise the process.

%%%%%%%%%%%%%%%%%%%%%%%%%%%%%%%%%%%%%%%%%%%%%%%%%%%%%%%%%%%%%%%%%%%%%%%%%%%%%%%%
\subsection{Command Line Processing}
\label{sec:commandline}

The effect of redirection files can also be achieved by invoking
the \LaTeX{} compiler with a more elaborate command line.
Most conveniently this should be done as part
of a shell script or a |Makefile|.

When using \textsf{childdoc} in the main file, the following
command lines effectively perform a redirection
(note that depending on the shell being used,
backslashes may have to be doubled: `|\|' $\to$ `|\\|'):
%
\begin{center}
|... -jobname "|\textit{target}|" |\\|"|[\textit{flags}]%
|\input{childdoc.def}\childdocforward[|\textit{main}|]{|\textit{dest}|}"|
\end{center}
%
Here \textit{target} is the name of the output file,
\textit{main} is the name of the main file
and \textit{dest} is the name of the main or child file to be processed
(all filenames without extensions).
The optional argument \textit{main} can be omitted
if \textit{main} matches \textit{dest}.
Optionally, compilation \textit{flags} can be defined via |\def| commands.
This command line makes the \TeX{} engine believe
it is compiling the file \textit{target}
whose content is specified as the latter parameter.
The provided code then forwards the processing to
\textit{main} or \textit{dest} as described in \secref{sec:forward}.

%%%%%%%%%%%%%%%%%%%%%%%%%%%%%%%%%%%%%%%%%%%%%%%%%%%%%%%%%%%%%%%%%%%%%%%%%%%%%%%%
\subsection{Include by Input}
\label{sec:input}

Including child documents by |\include| has some restrictions by design.
Most notably, the content of a child document always occupies
its own set of pages; pages cannot be shared between child documents.
Usually, this behaviour makes perfect sense
because each child document contain an essential part of the document.
However, in some situations it may be desirable to compose
a document from a collection of parts
without having mandatory page breaks between then.
For this case, the package
provides a mechanism to include parts
by |\input| which can also be processed individually.
However, by construction this mechanism
requires manual handling of the content to be output.

%%%%%%%%%%%%%%%%%%%%%%%%%%%%%%%%%%%%%%%%
\DescribeMacro{\ifchilddocmanual}
The main file should be prepared as usual, see \secref{sec:include}.
However, the document body must make a distinction
between processing of an individual part and of the main document, e.g.:
%
\begin{center}
\begin{tabular}{l}
|\ifchilddocmanual|\\
|\input{\childdocname}|\\
|\||else|\\
\textit{document body with }|\input{|\textit{part}|}|\\
|\||fi|
\end{tabular}
\end{center}
%
The conditional |\ifchilddocmanual| is true whenever
a part to be included by |\input| is being compiled,
and the name of the part is stored in |\childdocname|.

%%%%%%%%%%%%%%%%%%%%%%%%%%%%%%%%%%%%%%%%
\DescribeMacro{\childdocby}
Each part to be included by |\input| should start with:
%
\begin{center}
\begin{tabular}{l}
|\input{childdoc.def}|\\
|\childdocby{|\textit{main}|}|\\
\end{tabular}
\end{center}
%
The directive |\childdocby| is similar to |\childdocof|
described in \secref{sec:include},
but the subsequent selection of content must be done manually.
To that end, both |\ifchilddoc| and |\ifchilddocmanual|
will be true upon processing of a part,
and the name of the part is stored in |\childdocname|.
Note that |\jobname| will be set to the filename of the current part
so that each part receives an individual |.aux| file
that does not interfere with the |.aux| file(s) of the main document.
This behaviour can be altered by the alternative form
|\childdocby[*]{|\textit{main}|}| (with a non-empty optional argument)
which uses the |.aux| file of the main document
by setting |\jobname| to \textit{main}.

%%%%%%%%%%%%%%%%%%%%%%%%%%%%%%%%%%%%%%%%%%%%%%%%%%%%%%%%%%%%%%%%%%%%%%%%%%%%%%%%
\subsection{Driver Development}
\label{sec:driver}

The \textsf{childdoc} mechanism can also be use for the development
of definition files such as \LaTeX{} styles or classes.
This case differs from the above setup with multiple parts
included by |\include| in that no |\includeonly| should be invoked.
This can be achieved by starting the include file
(before |\ProvidesPackage|) with:
%
\begin{center}
\begin{tabular}{l}
|\input{childdoc.def}|\\
|\childdocforward{|\textit{main}|}|\\
\end{tabular}
\end{center}
%
or alternatively with:
%
\begin{center}
\begin{tabular}{l}
|\input{childdoc.def}|\\
|\childdocby{|\textit{main}|}|\\
\end{tabular}
\end{center}
%
Both forms have slightly different effects as described above.
The main file is prepared as usual, see \secref{sec:include}.

%%%%%%%%%%%%%%%%%%%%%%%%%%%%%%%%%%%%%%%%%%%%%%%%%%%%%%%%%%%%%%%%%%%%%%%%%%%%%%%%
\subsection{Legacy Detection}
\label{sec:detection}

The directive |\childdocmain| in the main file can detect
whether the complete document or merely a child is to be compiled
even without using the directive |\childdocof|.
This method is deprecated because it is less robust
and there is no compelling reason to use it;
it is merely provided for backward compatibility
and it may be removed in future versions.

If the detection mechanism is to be used,
it is mandatory to correctly specify
the filename of the main file as the argument of |\childdocmain|:
%
\begin{center}
\begin{tabular}{l}
|\input{childdoc.def}|\\
|\childdocmain{|\textit{main}|}|\\
\end{tabular}
\end{center}
%
If |\jobname| does not match the argument \textit{main} of |\childdocmain|,
it is assumed that |\jobname| points to the child file to be compiled.
When using |\childdocmain| with the main file specified as argument,
it suffices to start a child file
with just |\input{|\textit{main}|}|
without loading of the package and using |\childdocof|.
If instead all processing is done
with the appropriate \textsf{childdoc} directives,
the argument of \textit{main} of |\childdocmain| can be empty.

An alternative version of the command line processing described
in \secref{sec:commandline} using the detection mechanism reads:
%
\begin{center}
|... -jobname "|\textit{target}|" "|[\textit{flags}]%
[|\def\jobname{|\textit{dest}|}|]|\input{|\textit{main}|}"|
\end{center}

%%%%%%%%%%%%%%%%%%%%%%%%%%%%%%%%%%%%%%%%%%%%%%%%%%%%%%%%%%%%%%%%%%%%%%%%%%%%%%%%
\subsection{Manual Code}
\label{sec:manual}

In case one cannot be certain whether the definitions file |childdoc.def|
is installed on the target \TeX{} distribution
and one prefers not to ship it,
it is conceivable to paste a few relevant commands into the sources.

To that end, drop all statements |\input{childdoc.def}|
and perform the replacements as outlined below.
Instead of |\childdocmain{|\textit{main}|}| add the following code
to the top of the main file:
%
\begin{center}
\begin{tabular}{l}
|\||ifdefined\childdocname\endinput\||fi\newif\ifchilddoc|\\
|\edef\childdocname{\scantokens\expandafter{\jobname\noexpand}}|\\
|\def\childdocmain{|\textit{main}|}\||ifx\childdocmain\childdocname\||else|\\
|\childdoctrue\includeonly{\childdocname}\let\jobname\childdocmain\||fi|\\
\end{tabular}
\end{center}
%
Instead of |\childdocof{|\textit{main}|}| just include the main file
at the top of each child file:
%
\begin{center}
|\input{|\textit{main}|}|
\end{center}
%
A simple redirection |\childdocforward{|\textit{dest}|}| is achieved by:
%
\begin{center}
|\def\jobname{|\textit{dest}|}\input{\jobname}|
\end{center}
%
The redirection with prefix
|\childdocforwardprefix[|\textit{prefix}|]{|\textit{dest}|}|
is accomplished by:
%
\begin{center}
\begin{tabular}{l}
|{\edef\jobname{\scantokens\expandafter{\jobname\noexpand}}|\\
|\def\redirectjob |\textit{prefix}|#1~~~{\gdef\jobname{|\textit{dest}|#1}}|\\
|\expandafter\redirectjob\jobname~~~}\input{\jobname}|
\end{tabular}
\end{center}

In an alternative approach,
child documents can be compiled by a specific command line
without additional code or specific definitions:
%
\begin{center}
|... -jobname "|\textit{target}|" "|[\textit{flags}]%
|\includeonly{|\textit{dest}|}\input{|\textit{main}|}"|
\end{center}
%

%%%%%%%%%%%%%%%%%%%%%%%%%%%%%%%%%%%%%%%%%%%%%%%%%%%%%%%%%%%%%%%%%%%%%%%%%%%%%%%%
%%%%%%%%%%%%%%%%%%%%%%%%%%%%%%%%%%%%%%%%%%%%%%%%%%%%%%%%%%%%%%%%%%%%%%%%%%%%%%%%
\section{Information}

%%%%%%%%%%%%%%%%%%%%%%%%%%%%%%%%%%%%%%%%%%%%%%%%%%%%%%%%%%%%%%%%%%%%%%%%%%%%%%%%
\subsection{Copyright}

Copyright \copyright{} 2017--2018 Niklas Beisert

This work may be distributed and/or modified under the
conditions of the \LaTeX{} Project Public License, either version 1.3
of this license or (at your option) any later version.
The latest version of this license is in
  \url{http://www.latex-project.org/lppl.txt}
and version 1.3 or later is part of all distributions of \LaTeX{}
version 2005/12/01 or later.

This work has the LPPL maintenance status `maintained'.

The Current Maintainer of this work is Niklas Beisert.

This work consists of the files |README.txt|, |childdoc.ins| and |childdoc.dtx|
as well as the derived files |childdoc.def|, |cdocsamp.tex|
with |cdocsch1.tex|, |cdocsch2.tex|, |cdocspt3.tex|, |cdocspt4.tex|,
|cdocsdrf.tex|, |cdocsfn1.tex|, |cdocsfn2.tex|
as well as |childdoc.pdf|.

%%%%%%%%%%%%%%%%%%%%%%%%%%%%%%%%%%%%%%%%%%%%%%%%%%%%%%%%%%%%%%%%%%%%%%%%%%%%%%%%
\subsection{Files and Installation}

The package consists of the files:
%
\begin{center}
\begin{tabular}{ll}
    |README.txt|   & readme file \\
    |childdoc.ins| & installation file \\
    |childdoc.dtx| & source file \\
    |childdoc.def| & definition file \\
    |cdocsamp.tex| & sample main file \\
    |cdocsch1.tex| & sample include file \\
    |cdocsch2.tex| & sample include file \\
    |cdocspt3.tex| & sample part file \\
    |cdocspt4.tex| & sample part file \\
    |cdocsdrf.tex| & sample redirection file \\
    |cdocsfn1.tex| & sample redirection file \\
    |cdocsfn2.tex| & sample redirection file \\
    |childdoc.pdf| & manual
\end{tabular}
\end{center}
%
The distribution consists of the files
|README.txt|, |childdoc.ins| and |childdoc.dtx|.
%
\begin{itemize}
\item
Run (pdf)\LaTeX{} on |childdoc.dtx|
to compile the manual |childdoc.pdf| (this file).
\item
Run \LaTeX{} on |childdoc.ins| to create the definitions file |childdoc.def|
and the sample |cdocsamp.tex| with include files
|cdocsch1.tex|, |cdocsch2.tex|, |cdocspt3.tex|, |cdocspt4.tex|,
|cdocsdrf.tex|, |cdocsfn1.tex|, |cdocsfn2.tex|.
Then copy the file |childdoc.def| to an appropriate directory of your \LaTeX{}
distribution, e.g.\ \textit{texmf-root}|/tex/latex/childdoc|.
\end{itemize}

%%%%%%%%%%%%%%%%%%%%%%%%%%%%%%%%%%%%%%%%%%%%%%%%%%%%%%%%%%%%%%%%%%%%%%%%%%%%%%%%
\subsection{Related CTAN Packages}

There are several other packages which offer a similar functionality:
%
\begin{itemize}
\item
The packages
\href{http://ctan.org/pkg/docmute}{\textsf{docmute}},
\href{http://ctan.org/pkg/includex}{\textsf{includex}} and
\href{http://ctan.org/pkg/standalone}{\textsf{standalone}}
provide commands to include only the document body of
a child file thus allowing both files to be compiled individually.
\item
The packages \href{http://ctan.org/pkg/subdocs}{\textsf{subdocs}}
and \href{http://ctan.org/pkg/subfiles}{\textsf{subfiles}}
provide structures in which the main and child documents can be
encapsulated and allowing them to be compiled individually.
The inclusion mechanism is different from the conventional |\include|.
\item
The package \href{http://ctan.org/pkg/combine}{\textsf{combine}}
is an elaborate solution to combine several documents into one.
\end{itemize}
%
See also the CTAN topic \href{http://ctan.org/topic/subdocs}{\textsf{subdocs}}
for further related packages.
The present package differs from the above solutions in that
a document structure constructed with the conventional |\include| mechanism
just needs two extra commands at the top of every file
such that all constituent files can be compiled individually.

%%%%%%%%%%%%%%%%%%%%%%%%%%%%%%%%%%%%%%%%%%%%%%%%%%%%%%%%%%%%%%%%%%%%%%%%%%%%%%%%
%\subsection{Feature Suggestions}
%
%The following is a list of features which may be useful for future
%versions of this package:
%%
%\begin{itemize}
%\item
%\ldots
%\end{itemize}

%%%%%%%%%%%%%%%%%%%%%%%%%%%%%%%%%%%%%%%%%%%%%%%%%%%%%%%%%%%%%%%%%%%%%%%%%%%%%%%%
\subsection{Revision History}

%%%%%%%%%%%%%%%%%%%%%%%%%%%%%%%%%%%%%%%%
\paragraph{v2.0:} 2018/12/30

\begin{itemize}
\item
immediate forward processing
\item
added |\childdocby| mechanism
\item
manual restructured
\end{itemize}

%%%%%%%%%%%%%%%%%%%%%%%%%%%%%%%%%%%%%%%%
\paragraph{v1.6:} 2018/01/17

\begin{itemize}
\item
application for development of include files
\item
corrections to manual
\end{itemize}

%%%%%%%%%%%%%%%%%%%%%%%%%%%%%%%%%%%%%%%%
\paragraph{v1.5:} 2017/05/21

\begin{itemize}
\item
more complete structuring introduced
\item
|\childdocof| introduced
\item
|\childdoc| renamed to |\childdocmain|
\item
|\childredirect| renamed to |\childdocforward| and |\childdocforwardprefix|
and functionality expanded
\end{itemize}

%%%%%%%%%%%%%%%%%%%%%%%%%%%%%%%%%%%%%%%%
\paragraph{v1.0:} 2017/04/27

\begin{itemize}
\item
manual and install package
\item
first version published on CTAN
\end{itemize}

%%%%%%%%%%%%%%%%%%%%%%%%%%%%%%%%%%%%%%%%
\paragraph{v0.6:} 2017/04/26

\begin{itemize}
\item
redirection mechanism added
\end{itemize}

%%%%%%%%%%%%%%%%%%%%%%%%%%%%%%%%%%%%%%%%
\paragraph{v0.5:} 2017/04/26

\begin{itemize}
\item
functionality in definition file
\end{itemize}


%%%%%%%%%%%%%%%%%%%%%%%%%%%%%%%%%%%%%%%%%%%%%%%%%%%%%%%%%%%%%%%%%%%%%%%%%%%%%%%%
%%%%%%%%%%%%%%%%%%%%%%%%%%%%%%%%%%%%%%%%%%%%%%%%%%%%%%%%%%%%%%%%%%%%%%%%%%%%%%%%
%%%%%%%%%%%%%%%%%%%%%%%%%%%%%%%%%%%%%%%%%%%%%%%%%%%%%%%%%%%%%%%%%%%%%%%%%%%%%%%%
\appendix

\settowidth\MacroIndent{\rmfamily\scriptsize 000\ }

 \DocInput{childdoc.dtx}

\end{document}
%</driver>
% \fi
%
% %%%%%%%%%%%%%%%%%%%%%%%%%%%%%%%%%%%%%%%%%%%%%%%%%%%%%%%%%%%%%%%%%%%%%%%%%%%%%%
% %%%%%%%%%%%%%%%%%%%%%%%%%%%%%%%%%%%%%%%%%%%%%%%%%%%%%%%%%%%%%%%%%%%%%%%%%%%%%%
% \section{Sample}
%\iffalse
%<*samplemain>
%\fi
%
% The following presents a sample document
% with two chapters, two parts, a title page,
% a compile flag as well as three forwarding files to set the flag.
% It consists of eight |.tex| files:
% \begin{center}
% \begin{tabular}{ll}
% |cdocsamp.tex|&main file\\
% |cdocsch1.tex|&include file for chapter 1\\
% |cdocsch2.tex|&include file for chapter 2\\
% |cdocspt3.tex|&include file for part 3\\
% |cdocspt4.tex|&include file for part 4\\
% |cdocsdrf.tex|&forwarding file for main file in draft mode\\
% |cdocsfi1.tex|&forwarding file for final version of chapter 1\\
% |cdocsfi2.tex|&forwarding file for final version of chapter 2\\
% \end{tabular}
% \end{center}
% Each of the eight files can be compiled directly by the \LaTeX{} compiler.
%
% %%%%%%%%%%%%%%%%%%%%%%%%%%%%%%%%%%%%%%
% \paragraph{Main File.}
%
% The main file is called |cdocsamp.tex|.
%
% Load the \textsf{childdoc} definitions and
% declare the filename for the main document:
%    \begin{macrocode}
\input{childdoc.def}
\childdocmain{}
%    \end{macrocode}

% Optional override for |\version| flag:
%    \begin{macrocode}
%%\ifchilddoc\else\providecommand{\version}{draft}\fi
%    \end{macrocode}

% Define the default values for the |\version| flag
% (|final| for the main file and |draft| for childs):
%    \begin{macrocode}
\ifchilddoc
\providecommand{\version}{draft}
\else
\providecommand{\version}{final}
\fi
%    \end{macrocode}

% Load the standard document class:
%    \begin{macrocode}
\documentclass[12pt]{article}
%    \end{macrocode}

% Start the document body:
%    \begin{macrocode}
\begin{document}
%    \end{macrocode}

% Declare a title page.
% Print title, part of document being processed and version flag:
%    \begin{macrocode}
\addtocounter{page}{-1}
\begin{center}
{\LARGE\bfseries{}childdoc example\par}
\vspace{1cm}
\ifchilddoc
\ifchilddocmanual part\else chapter\fi:
`\childdocname' of `\childdocjob'\par
\else
main document: `\childdocjob'\par
\fi
version: \version\par
\end{center}
\newpage
%    \end{macrocode}

% Manually include selected file,
% otherwise process as usual:
%    \begin{macrocode}
\ifchilddocmanual
\section*{part `\childdocname'}
\input{\childdocname}
\else
%    \end{macrocode}

% Include the two chapters:
%    \begin{macrocode}
\include{cdocsch1}
\include{cdocsch2}
%    \end{macrocode}

% Include the two parts unless only chapters should be displayed:
%    \begin{macrocode}
\ifchilddoc\else
\section{part three}
\input{cdocspt3}
\section{part four}
\input{cdocspt4}
\fi
%    \end{macrocode}

% Process as usual until here:
%    \begin{macrocode}
\fi
%    \end{macrocode}

% End of document body:
%    \begin{macrocode}
\end{document}
%    \end{macrocode}
%\iffalse
%</samplemain>
%\fi
%
% %%%%%%%%%%%%%%%%%%%%%%%%%%%%%%%%%%%%%%
% \paragraph{Chapter Include Files.}
%
% The include files are called |cdocsch1.tex| and |cdocsch2.tex|.
%
%\iffalse
%<*samplechap1|samplechap2>
%\fi

% Optional override for |\version| flag:
%    \begin{macrocode}
%%\providecommand{\version}{final}
%    \end{macrocode}

% Include the main document:
%    \begin{macrocode}
\input{childdoc.def}
\childdocof{cdocsamp}
%    \end{macrocode}

%\iffalse
%</samplechap1|samplechap2>
%\fi
%
%\iffalse
%<*samplechap1>
%\fi
% Some text for chapter 1:
%    \begin{macrocode}
\section{one}
some text in chapter one
%    \end{macrocode}

%\iffalse
%</samplechap1>
%\fi
% Some text for chapter 2:
%\iffalse
%<*samplechap2>
%\fi
%    \begin{macrocode}
\section{two}
more text in chapter two
%    \end{macrocode}

%\iffalse
%</samplechap2>
%\fi
%
% %%%%%%%%%%%%%%%%%%%%%%%%%%%%%%%%%%%%%%
% \paragraph{Part Include Files.}
%
% The include files are called |cdocspt3.tex| and |cdocspt4.tex|.
%
%\iffalse
%<*samplepart3|samplepart4>
%\fi

% Optional override for |\version| flag:
%    \begin{macrocode}
%%\providecommand{\version}{final}
%    \end{macrocode}

% Include the main document:
%    \begin{macrocode}
\input{childdoc.def}
\childdocby{cdocsamp}
%    \end{macrocode}

%\iffalse
%</samplepart3|samplepart4>
%\fi
%
%\iffalse
%<*samplepart3>
%\fi
% Some text for part 3:
%    \begin{macrocode}
some text in part three
%    \end{macrocode}

%\iffalse
%</samplepart3>
%\fi
% Some text for part 4:
%\iffalse
%<*samplepart4>
%\fi
%    \begin{macrocode}
more text in part four
%    \end{macrocode}

%\iffalse
%</samplepart4>
%\fi
%
% %%%%%%%%%%%%%%%%%%%%%%%%%%%%%%%%%%%%%%
% \paragraph{Forwarding for a Complete Draft.}
%
% The following forwarding file |cdocsdrf.tex|
% compiles the main document in draft mode:
%\iffalse
%<*sampledraft>
%\fi
%    \begin{macrocode}
\def\version{draft}
\input{childdoc.def}
\childdocforward{cdocsamp}
%    \end{macrocode}

%\iffalse
%</sampledraft>
%\fi
%
% %%%%%%%%%%%%%%%%%%%%%%%%%%%%%%%%%%%%%%
% \paragraph{Forwarding for Final Version of the Chapters.}
%
% The following forwarding files |cdocsfn1.tex| and |cdocsfn2.tex|
% (with identical content)
% compile the final versions of the child documents
% |cdocsch1.tex| and |cdocsch2.tex|, respectively:
%\iffalse
%<*samplefinal>
%\fi
%    \begin{macrocode}
\def\version{final}
\input{childdoc.def}
\childdocforwardprefix[cdocsamp]{cdocsfn}{cdocsch}
%    \end{macrocode}

%\iffalse
%</samplefinal>
%\fi
%
% %%%%%%%%%%%%%%%%%%%%%%%%%%%%%%%%%%%%%%
% \paragraph{Command Line Processing.}
%
% The following three command lines generate the output files
% |cdocscld|, |cdocscl1| and |cdocscl2|
% which should be identical to
% |cdocsdrf|, |cdocsch1| and |cdocsfn2|, respectively:
% \begin{center}
% \begin{tabular}{l}
% |latex -jobname cdocscld \|\\
% |  "\def\version{draft}\input{childdoc.def}\childdocforward{cdocsamp}"|\\
% |latex -jobname cdocscl1 \|\\
% |  "\input{childdoc.def}\childdocforward[cdocsamp]{cdocsch1}"|\\
% |latex -jobname cdocscl2 \|\\
% |  "\def\version{final}\input{childdoc.def}\childdocforward{cdocsch2}"|
% \end{tabular}
% \end{center}
% Note that the trailing backslash on each first line
% merely continues the input to the second line
% (for convenient cut ant paste).
% Furthermore, the command |latex| can be replaced by any
% of its alternative versions such as |pdflatex|.
%
% %%%%%%%%%%%%%%%%%%%%%%%%%%%%%%%%%%%%%%%%%%%%%%%%%%%%%%%%%%%%%%%%%%%%%%%%%%%%%%
% %%%%%%%%%%%%%%%%%%%%%%%%%%%%%%%%%%%%%%%%%%%%%%%%%%%%%%%%%%%%%%%%%%%%%%%%%%%%%%
% \section{Implementation}
%\iffalse
%<*package>
%\fi
%
% This section describes the definitions file |childdoc.def|.

% The definitions cannot be loaded using |\usepackage| or |\RequirePackage|
% which has a mechanism to prevent loading a style file more than once.
% When loading the definitions by means of |\input|
% multiple instances have to be prevented manually:
%\iffalse
%This code needs to be before the `\ProvidesFile' directive
%which is defined at the beginning of this file.
%Therefore it is also placed there and commented out here.
%</package>
%<*discard>
%\fi
%    \begin{macrocode}
\ifdefined\childdocmain\endinput\fi
%    \end{macrocode}
%\iffalse
%</discard>
%<*package>
%\fi
%
% \macro{\ifchilddoc}
% \macro{\ifchilddocmanual}
% The conditional |\ifchilddoc| tells whether a
% child (true) or main (false) document is being compiled.
% The conditional |\ifchilddocmanual| tells whether
% the |\includeonly| mechanism is used (false) or
% the selection of child files must be performed manually (true).
% The definitions initialise to false:
%    \begin{macrocode}
\newif\ifchilddoc
\newif\ifchilddocmanual
%    \end{macrocode}

% \macro{\childdocname}
% \macro{\childdocjob}
% The macro |\childdocname| stores the name of the main document
% to be compiled. The macro |\childdocjob| stores the name of
% the document on which the \LaTeX{} compiler was originally invoked.
% The content of |\jobname| cannot be compared
% to filenames specified in the source due to different catcodes.
% The following code rescans |\jobname|, stores the result
% in |\childdocname| and saves a copy in |\childdocjob|:
%    \begin{macrocode}
\edef\childdocname{\scantokens\expandafter{\jobname\noexpand}}
\let\childdocjob\childdocname
%    \end{macrocode}

% \macro{\childdocdisable}
% The macro |\childdocdisable| prevents the main file
% from being processed more than once.
% At this stage, the main document command |\childdocmain|
% is assumed to be called once again where it should do nothing.
% Any subsequent call to it should prevent
% a secondary processing of the main document
% It overwrites the forwarding commands
% |\childdocof| and |\childdocforward|
% with empty macros to prevent further inclusions of the main document:
%    \begin{macrocode}
\newcommand{\childdocdisable}
{
  \renewcommand{\childdocmain}[1]{\renewcommand{\childdocmain}[1]{\endinput}}
  \renewcommand{\childdocof}[1]{}
  \renewcommand{\childdocby}[2][]{}
  \renewcommand{\childdocforward}[2][]{}
  \renewcommand{\childdocdisable}{}
}
%    \end{macrocode}

% \macro{\childdocmain}
% The macro |\childdocmain| is to be called at the top of the main file
% with nothing or the main filename (without extension) as argument.
% First, it breaks loops.
% If the argument is not empty and does not match |\childdocname|
% (which is set by the first inclusion of |childdoc.def|),
% |\ifchilddoc| is set to true, |\includeonly| is applied to the child file
% and |\jobname| is set to the main file
% (for proper handling of |.aux| files):
%    \begin{macrocode}
\newcommand{\childdocmain}[1]
{
  \childdocdisable\childdocmain{}
  \if?#1?\else
    \begingroup
      \def\childdoctmp{#1}
      \ifx\childdoctmp\childdocname
        \def\childdoctmp{}
      \else
        \def\childdoctmp
        {
          \childdoctrue
          \includeonly{\childdocname}
          \def\childdocjob{#1}
          \def\jobname{#1}
        }
      \fi
      \expandafter
    \endgroup
    \childdoctmp
  \fi
}
%    \end{macrocode}

% \macro{\childdocof}
% The command |\childdocof| redirects
% compilation to the main file |#1|.
%    \begin{macrocode}
\newcommand{\childdocof}[1]
{
  \childdocdisable
  \childdoctrue
  \includeonly{\childdocname}
  \def\jobname{#1}
  \def\childdocjob{#1}
  \input{#1}
}
%    \end{macrocode}

% \macro{\childdocby}
% The command |\childdocby| ....
%    \begin{macrocode}
\newcommand{\childdocby}[2][]
{
  \childdocdisable
  \childdoctrue
  \childdocmanualtrue
  \if?#1?\else
    \def\jobname{#2}
  \fi
  \def\childdocjob{#2}
  \input{#2}
  \endinput
}
%    \end{macrocode}

% \macro{\childdocforward}
% The command |\childdocforward| redirects
% compilation to the main file or
% (if the optional argument is given) a child file.
% Parameters are set as if the main file
% or a child file starting with |\childdocof| was compiled.
% Then compilation is handed over to the main file:
%    \begin{macrocode}
\newcommand{\childdocforward}[2][]
{
  \begingroup
    \if?#1?
      \def\childdoctmp
      {
        \def\childdocname{#2}
        \def\childdocjob{#2}
        \def\jobname{#2}
        \input{#2}
        \endinput
      }
    \else
      \def\childdoctmp
      {
        \childdocdisable
        \def\childdocname{#2}
        \childdoctrue
        \includeonly{#2}
        \def\childdocjob{#1}
        \def\jobname{#1}
        \input{#1}
        \endinput
      }
    \fi
    \expandafter
  \endgroup
  \childdoctmp
}
%    \end{macrocode}

% \macro{\childdocforwardprefix}
% The command |\childdocforwardprefix| redirects
% compilation to the main or a child file by means of a pattern.
% The prefix |#1| in the current filename is replaced by |#2|
% and the suffix of the current filename is kept
% (it is assumed that the filename does not contain the substring `|~~~|'
% which is used as a delimiter).
% Compilation is handed over to the new file by |\childdocforward|:
%    \begin{macrocode}
\newcommand{\childdocforwardprefix}[3][]
{
  \begingroup
    \def\childdocextract #2##1~~~{\def\childdoctmp{\childdocforward[#1]{#3##1}}}
    \expandafter\childdocextract\childdocname~~~
    \expandafter
  \endgroup
  \childdoctmp
}
%    \end{macrocode}

% \macro{\childdoc}
% The deprecated macro |\childdoc| is a legacy version of |\childdocmain|:
%    \begin{macrocode}
\newcommand{\childdoc}{\childdocmain}
%    \end{macrocode}

% \macro{\childdocredirect}
% The deprecated macro |\childdocredirect| is a legacy version
% of |\childdocforward| and |\childdocforwardprefix|:
%    \begin{macrocode}
\newcommand{\childdocredirect}[2][]
{
  \begingroup
    \if?#1?
      \def\childdoctmp{\childdocforward{#2}}
    \else
      \def\childdoctmp{\childdocforwardprefix{#1}{#2}}
    \fi
    \expandafter
  \endgroup
  \childdoctmp
}
%    \end{macrocode}

%\iffalse
%</package>
%\fi
%
\endinput
|\\
|\childdocforwardprefix{final}{child}|
\end{tabular}
\end{center}
%

Note that when several versions of a main file and/or of each child file
are to be generated, it may be convenient to set up a |Makefile| or
shell script to automatise the process.

%%%%%%%%%%%%%%%%%%%%%%%%%%%%%%%%%%%%%%%%%%%%%%%%%%%%%%%%%%%%%%%%%%%%%%%%%%%%%%%%
\subsection{Command Line Processing}
\label{sec:commandline}

The effect of redirection files can also be achieved by invoking
the \LaTeX{} compiler with a more elaborate command line.
Most conveniently this should be done as part
of a shell script or a |Makefile|.

When using \textsf{childdoc} in the main file, the following
command lines effectively perform a redirection
(note that depending on the shell being used,
backslashes may have to be doubled: `|\|' $\to$ `|\\|'):
%
\begin{center}
|... -jobname "|\textit{target}|" |\\|"|[\textit{flags}]%
|% \iffalse
%
% childdoc.dtx Copyright (C) 2017-2018 Niklas Beisert
%
% This work may be distributed and/or modified under the
% conditions of the LaTeX Project Public License, either version 1.3
% of this license or (at your option) any later version.
% The latest version of this license is in
%   http://www.latex-project.org/lppl.txt
% and version 1.3 or later is part of all distributions of LaTeX
% version 2005/12/01 or later.
%
% This work has the LPPL maintenance status `maintained'.
%
% The Current Maintainer of this work is Niklas Beisert.
%
% This work consists of the files childdoc.dtx and childdoc.ins
% and the derived files childdoc.def and cdocsamp.tex with
% cdocsch1.tex, cdocsch2.tex, cdocsdrf.tex, cdocsfn1.tex, cdocsfn2.tex.
%
%<package>\ifdefined\childdocmain\endinput\fi
%<package>\ProvidesFile{childdoc.def}[2018/12/30 v2.0 child document driver]
%<samplemain>\ProvidesFile{cdocsamp.tex}[2018/12/30 v2.0 sample for childdoc]
%<*driver>
%\ProvidesFile{childdoc.drv}[2018/12/30 v2.0 childdoc reference manual file]
\PassOptionsToClass{10pt,a4paper}{article}
\documentclass{ltxdoc}

\usepackage[margin=35mm]{geometry}
\usepackage{hyperref}
\usepackage{hyperxmp}
\usepackage[usenames]{color}

\hypersetup{colorlinks=true}
\hypersetup{pdfstartview=FitH}
\hypersetup{pdfpagemode=UseNone}
\hypersetup{pdfsource={}}
\hypersetup{pdflang={en-UK}}
\hypersetup{pdfcopyright={Copyright 2017-2018 Niklas Beisert.
  This work may be distributed and/or modified under the
  conditions of the LaTeX Project Public License, either version 1.3
  of this license or (at your option) any later version.}}
\hypersetup{pdflicenseurl={http://www.latex-project.org/lppl.txt}}
\hypersetup{pdfcontactaddress={ETH Zurich, ITP, HIT K,
  Wolfgang-Pauli-Strasse 27}}
\hypersetup{pdfcontactpostcode={8093}}
\hypersetup{pdfcontactcity={Zurich}}
\hypersetup{pdfcontactcountry={Switzerland}}
\hypersetup{pdfcontactemail={nbeisert@itp.phys.ethz.ch}}
\hypersetup{pdfcontacturl={http://people.phys.ethz.ch/\xmptilde nbeisert/}}

\newcommand{\secref}[1]{\hyperref[#1]{section \ref*{#1}}}

\parskip1ex
\parindent0pt
\let\olditemize\itemize
\def\itemize{\olditemize\parskip0pt}

\begin{document}

\title{The \textsf{childdoc} Package}
\hypersetup{pdftitle={The childdoc Package}}
\author{Niklas Beisert\\[2ex]
  Institut f\"ur Theoretische Physik\\
  Eidgen\"ossische Technische Hochschule Z\"urich\\
  Wolfgang-Pauli-Strasse 27, 8093 Z\"urich, Switzerland\\[1ex]
  \href{mailto:nbeisert@itp.phys.ethz.ch}
  {\texttt{nbeisert@itp.phys.ethz.ch}}}
\hypersetup{pdfauthor={Niklas Beisert}}
\hypersetup{pdfsubject={Manual for the LaTeX2e Package childdoc}}
\date{30 December 2018, \textsf{v2.0}}
\maketitle

\begin{abstract}\noindent
\textsf{childdoc} is a \LaTeXe{} package
that enables the direct compilation
of document sections included by |\include|
to individual files.
\end{abstract}

\begingroup
\parskip0ex
\tableofcontents
\endgroup

%%%%%%%%%%%%%%%%%%%%%%%%%%%%%%%%%%%%%%%%%%%%%%%%%%%%%%%%%%%%%%%%%%%%%%%%%%%%%%%%
%%%%%%%%%%%%%%%%%%%%%%%%%%%%%%%%%%%%%%%%%%%%%%%%%%%%%%%%%%%%%%%%%%%%%%%%%%%%%%%%
\section{Introduction}

\LaTeX{} provides a mechanism to structure a large document (such as a book)
into a main file and several child files (containing the chapters)
using the |\include| command.
This mechanism is beneficial for documents
which span hundreds of pages in order to
make the source file(s) more manageable.
Moreover, compilation can be restricted to
selected child files by means of the |\includeonly| command.
The latter feature can be used to reduce the compilation time while editing
(this was significantly more useful in the earlier days of \LaTeX{})
or to generate a smaller document which is easier to navigate.
Another application of |\includeonly| is to generate
documents consisting of selected parts of the complete document.

However, there are a few drawbacks of the plain |\include| mechanism:
\begin{itemize}
\item
The child files cannot be compiled on their own,
they can only be compiled via the main file.
A naive editing environment
(such as a text editor with an option
to have the current file processed by \LaTeX)
may require one to switch to the main file before compiling;
attempting to compile the child file produces errors.
\item
The main file must be modified (each time)
to adjust the |\includeonly| command
to the present needs. This easily leaves the main file in a messy state.
\item
The generated document will always carry the filename
of the main document. This is inconvenient if
several child files are to be compiled and
to be kept for distribution.
\end{itemize}

The present package provides a simple interface
to make child files individually compilable by \LaTeX{}.
Compiling a child file then has the same effect as compiling
the main file with an |\includeonly| command
to select the appropriate child.
Moreover the generated document will carry the name of the child
rather than the main file.
This resolves all three above issues.

This feature is meant to make the editing of books,
thesis documents and lecture notes somewhat more convenient.
However, the package can also be used efficiently for
composing a series of documents (such as exercise sheets)
which are typically distributed individually.
It then assists the author in generating the individual documents
(potentially in different versions)
as well as a document containing the collected series.
Another application is in developing style files
or other kinds of included material
where compilation of the style file could redirect
to a sample or test file.

%%%%%%%%%%%%%%%%%%%%%%%%%%%%%%%%%%%%%%%%%%%%%%%%%%%%%%%%%%%%%%%%%%%%%%%%%%%%%%%%
%%%%%%%%%%%%%%%%%%%%%%%%%%%%%%%%%%%%%%%%%%%%%%%%%%%%%%%%%%%%%%%%%%%%%%%%%%%%%%%%
\section{Usage}

First of all, the package \textsf{childdoc} is \emph{not} a standard
\LaTeXe{} |.sty| style file! Therefore it needs to be invoked in
a non-standard way.

%%%%%%%%%%%%%%%%%%%%%%%%%%%%%%%%%%%%%%%%%%%%%%%%%%%%%%%%%%%%%%%%%%%%%%%%%%%%%%%%
\subsection{Included Files}
\label{sec:include}

%%%%%%%%%%%%%%%%%%%%%%%%%%%%%%%%%%%%%%%%
\DescribeMacro{\childdocmain}
To use the package, add the commands
\begin{center}
\begin{tabular}{l}
|\input{childdoc.def}|\\
|\childdocmain{}|\\
\end{tabular}
\end{center}
at the very top of the main \LaTeX{} file,
in particular \emph{before} the |\documentclass| statement!
The argument of |\childdocmain| should be left empty
(but it must be present).

%%%%%%%%%%%%%%%%%%%%%%%%%%%%%%%%%%%%%%%%
\DescribeMacro{\childdocof}
Furthermore, add the commands
\begin{center}
\begin{tabular}{l}
|\input{childdoc.def}|\\
|\childdocof{|\textit{main}|}|\\
\end{tabular}
\end{center}
at the top of every child file \textit{child}
which is included by |\include{|\textit{child}|}|
from within the main file
(or at least for those files to be compiled individually).
The argument \textit{main} must be the filename of the main file.

There are a couple of
considerations in setting up the main and child documents:

%%%%%%%%%%%%%%%%%%%%%%%%%%%%%%%%%%%%%%%%
\paragraph{Restrictions.}

Please note the following restrictions:
\begin{itemize}
\item
|\childdocmain| must be called with one argument \textit{main}
to ensure compatibility with earlier version of the package.
It must either be empty (|\childdocmain{}|)
or precisely match the filename of the main file in which it is specified.
See \secref{sec:detection} for further information.
\item
The filename \textit{main} must be specified without the |.tex| extension.
\item
The filename \textit{main} is case sensitive
(even in case-insensitive file systems)
due to internal string comparison.
\item
The argument \textit{main} should be fully expanded, it cannot be a macro.
\item
Subdirectories and special characters should be avoided in filenames.
\item
The command |\childdocmain{|\textit{main}|}| must be followed by a whitespace.
It should not be followed immediately by another command
or by a comment mark `|%|'.
This is because the \TeX{} parser reads the token immediately following
the argument of |\childdocmain| and puts it
at the beginning of every child section;
however, a white\-space is ignored.
\end{itemize}

%%%%%%%%%%%%%%%%%%%%%%%%%%%%%%%%%%%%%%%%
\paragraph{Content of Main File.}

It is advisable to place all content in the child files included by |\include|.
Any output contained in the main file will appear in all child documents
unless suppressed manually;
it cannot be suppressed automatically by the |\includeonly| directive
and thus should normally be avoided.
A method to include some content in the main file
by means of conditional processing is described in \secref{sec:conditional}.

%%%%%%%%%%%%%%%%%%%%%%%%%%%%%%%%%%%%%%%%
\paragraph{Page Numbering.}

When only a part of the document is compiled,
the appropriate numbering of pages
(as well as other status parameters)
is determined from the |.aux| files.
The latter contain information from previous passes.
However this information needs to propagate through
all intermediate child documents.
Therefore the page numbering in child documents may well
be inconsistent until the complete document is compiled at least once.

A useful (if unconventional) way to always ensure a consistent
page numbering is to restart the numbering in each child document
and denote the pages by `\textit{child}|.|\textit{page}'
where \textit{child} represents the chapter/section number of the child file.
This can be achieved by the command
|\numberwithin{page}{|\textit{child}|}|
of the \textsf{amsmath} package
where \textit{child} can be |chapter| or |section|
depending on the chosen structuring.
Alternatively, one can modify the macro |\thepage| appropriately
and reset the counter |page| at the start of each child file.

%%%%%%%%%%%%%%%%%%%%%%%%%%%%%%%%%%%%%%%%%%%%%%%%%%%%%%%%%%%%%%%%%%%%%%%%%%%%%%%%
\subsection{Conditional Processing}
\label{sec:conditional}

The package provides a mechanism to compile different versions
of a document. To customise the versions further some conditional processing
can come in handy to distinguish which version is being compiled.
The package provides two macros to describe the compilation context:

%%%%%%%%%%%%%%%%%%%%%%%%%%%%%%%%%%%%%%%%
\DescribeMacro{\ifchilddoc}
The conditional |\ifchilddoc| distinguishes between the compilation of
child documents and the main document:
%
\begin{center}
|\ifchilddoc |\textit{child-code}| |[|\||else |\textit{main-code}]| \||fi|
\end{center}

%%%%%%%%%%%%%%%%%%%%%%%%%%%%%%%%%%%%%%%%
\DescribeMacro{\childdocname}
\DescribeMacro{\childdocjob}
The macro |\childdocname| contains the filename (without extension)
of the main or child file being processed.
Note that |\childdocjob| will always contain the name of the main file.

%%%%%%%%%%%%%%%%%%%%%%%%%%%%%%%%%%%%%%%%
\paragraph{Title Page.}

Conditional processing can be used to include a title or banner page
in the main document when proper precautions are taken.
Importantly, the code in the main file should ensure that the page counter
(as well as other status parameters which are stored in the |.aux| files)
takes the same value after the conditional processing.
Otherwise the page numbers may take divergent values
depending on which part is compiled.

For example, a title page could be declared by:
%
\begin{center}
\begin{tabular}{l}
|\ifchilddoc\||else|\\
|\addtocounter{page}{-1}|\\
\textit{code for title page}\\
|\newpage|\\
|\||fi|
\end{tabular}
\end{center}
%
A banner page for the child documents can be generated by:
%
\begin{center}
\begin{tabular}{l}
|\ifchilddoc|\\
|\addtocounter{page}{-1}|\\
\textit{code for banner page}\\
|\newpage|\\
|\||fi|
\end{tabular}
\end{center}
%
Here one could write a message such as:
\begin{center}
|This is the part \childdocname{} of \childdocjob{}.|
\end{center}

%%%%%%%%%%%%%%%%%%%%%%%%%%%%%%%%%%%%%%%%%%%%%%%%%%%%%%%%%%%%%%%%%%%%%%%%%%%%%%%%
\subsection{Flags}
\label{sec:flags}

The package makes it easy to generate different versions
of the main or child documents.
To this end compilation flags can be defined
and assigned different default values.
They will be particularly useful in conjunction
with the forwarding mechanism described in \secref{sec:forward}.

For example, it may be useful to have a flag |\version|
which can be set to |draft| or |final|.
The document source will contain some conditional code
depending on the value of |\version|.
Suppose further, the flag should default to |final| for the main file
and to |draft| for child files
which is a natural assignment for editing the document.
This is achieved by placing the following code
in the preamble of the main document
(below the |\childdocmain| directive):
%
\begin{center}
\begin{tabular}{l}
|\ifchilddoc|\\
|\providecommand{\version}{draft}|\\
|\||else|\\
|\providecommand{\version}{final}|\\
|\||fi|
\end{tabular}
\end{center}
%
The definition by |\providecommand| makes sure
that previous definitions are not overwritten.
Further statements |\providecommand{\version}{...}|
can thus be added before the above code to override it.

For the main file, one might add a line
(between |\childdocmain| and the above block)
%
\begin{center}
|%\ifchilddoc\||else\providecommand{\version}{draft}\||fi|
\end{center}
%
which can be uncommented to produce a draft version.
Likewise one can add a line to the very top of a child file
(above the |\childdocof{|\textit{main}|}| directive)
%
\begin{center}
|%\providecommand{\version}{final}|
\end{center}
%
which can be uncommented to produce the final version of this child document.

%%%%%%%%%%%%%%%%%%%%%%%%%%%%%%%%%%%%%%%%%%%%%%%%%%%%%%%%%%%%%%%%%%%%%%%%%%%%%%%%
\subsection{Forwarding}
\label{sec:forward}

Different versions of the main or child documents
using compilation flags as described in \secref{sec:flags}
can be (permanently) stored in different files
for convenient compilation, viewing and distribution.
To this end, the package defines a command
to pass on compilation to a different file:

%%%%%%%%%%%%%%%%%%%%%%%%%%%%%%%%%%%%%%%%
\DescribeMacro{\childdocforward}
The command |\childdocforward| redirects processing to
another source file:
%
\begin{center}
\begin{tabular}{l}
|\input{childdoc.def}|\\
|\childdocforward[|\textit{main}|]{|\textit{dest}|}|\\
\end{tabular}
\end{center}
%
The argument \textit{dest} is the destination file
(without extension).
It should be the main file or one of the child files.
Note that further \textsf{childdoc} directives
such as |\childdocof| and |\childdocforward|
in the indicated file will be processed in this form.
The optional argument \textit{main}
passes on directly to the main file \textit{main}
while pretending to compile the child \textit{dest}.
This form behaves as if \textit{dest}
issues |\childdocof{|\textit{main}|}| right away,
and no further \textsf{childdoc} directives will be processed.

%%%%%%%%%%%%%%%%%%%%%%%%%%%%%%%%%%%%%%%%
\DescribeMacro{\...prefix}
In the alternative form |\childdocforwardprefix|,
%
\begin{center}
\begin{tabular}{l}
|\input{childdoc.def}|\\
|\childdocforwardprefix[|\textit{main}|]{|\textit{prefix}|}{|\textit{dest}|}|
\end{tabular}
\end{center}
%
the destination file is determined by a pattern
depending on the current file:
To make this work, the current file must be called
`{\textit{prefix}\hspace{0.2em}\textit{suffix}}'
with \textit{prefix} matching precisely the argument.
Processing is then passed on to the file
`{\textit{dest}\hspace{0.2em}\textit{suffix}}'.
Surely, the same effect is achieved by
directly specifying the
argument `{\textit{dest}\hspace{0.2em}\textit{suffix}}'
in the first form.
However, that requires to set up a different file
for each child. With the alternative form of the command
all these files can have exactly the same content
which simplifies setting them up and maintaining them.

For example, the following file |draft.tex|
with a compilation flag |\version| as described in \secref{sec:flags}
compiles the main document as a draft:
%
\begin{center}
\begin{tabular}{l}
|\def\version{draft}|\\
|\input{childdoc.def}|\\
|\childdocforward{|\textit{main}|}|
\end{tabular}
\end{center}
%
Likewise, the following files |final|\textit{nn}|.tex|
compile the final version of the child document
|child|\textit{nn}|.tex|:
%
\begin{center}
\begin{tabular}{l}
|\def\version{final}|\\
|\input{childdoc.def}|\\
|\childdocforwardprefix{final}{child}|
\end{tabular}
\end{center}
%

Note that when several versions of a main file and/or of each child file
are to be generated, it may be convenient to set up a |Makefile| or
shell script to automatise the process.

%%%%%%%%%%%%%%%%%%%%%%%%%%%%%%%%%%%%%%%%%%%%%%%%%%%%%%%%%%%%%%%%%%%%%%%%%%%%%%%%
\subsection{Command Line Processing}
\label{sec:commandline}

The effect of redirection files can also be achieved by invoking
the \LaTeX{} compiler with a more elaborate command line.
Most conveniently this should be done as part
of a shell script or a |Makefile|.

When using \textsf{childdoc} in the main file, the following
command lines effectively perform a redirection
(note that depending on the shell being used,
backslashes may have to be doubled: `|\|' $\to$ `|\\|'):
%
\begin{center}
|... -jobname "|\textit{target}|" |\\|"|[\textit{flags}]%
|\input{childdoc.def}\childdocforward[|\textit{main}|]{|\textit{dest}|}"|
\end{center}
%
Here \textit{target} is the name of the output file,
\textit{main} is the name of the main file
and \textit{dest} is the name of the main or child file to be processed
(all filenames without extensions).
The optional argument \textit{main} can be omitted
if \textit{main} matches \textit{dest}.
Optionally, compilation \textit{flags} can be defined via |\def| commands.
This command line makes the \TeX{} engine believe
it is compiling the file \textit{target}
whose content is specified as the latter parameter.
The provided code then forwards the processing to
\textit{main} or \textit{dest} as described in \secref{sec:forward}.

%%%%%%%%%%%%%%%%%%%%%%%%%%%%%%%%%%%%%%%%%%%%%%%%%%%%%%%%%%%%%%%%%%%%%%%%%%%%%%%%
\subsection{Include by Input}
\label{sec:input}

Including child documents by |\include| has some restrictions by design.
Most notably, the content of a child document always occupies
its own set of pages; pages cannot be shared between child documents.
Usually, this behaviour makes perfect sense
because each child document contain an essential part of the document.
However, in some situations it may be desirable to compose
a document from a collection of parts
without having mandatory page breaks between then.
For this case, the package
provides a mechanism to include parts
by |\input| which can also be processed individually.
However, by construction this mechanism
requires manual handling of the content to be output.

%%%%%%%%%%%%%%%%%%%%%%%%%%%%%%%%%%%%%%%%
\DescribeMacro{\ifchilddocmanual}
The main file should be prepared as usual, see \secref{sec:include}.
However, the document body must make a distinction
between processing of an individual part and of the main document, e.g.:
%
\begin{center}
\begin{tabular}{l}
|\ifchilddocmanual|\\
|\input{\childdocname}|\\
|\||else|\\
\textit{document body with }|\input{|\textit{part}|}|\\
|\||fi|
\end{tabular}
\end{center}
%
The conditional |\ifchilddocmanual| is true whenever
a part to be included by |\input| is being compiled,
and the name of the part is stored in |\childdocname|.

%%%%%%%%%%%%%%%%%%%%%%%%%%%%%%%%%%%%%%%%
\DescribeMacro{\childdocby}
Each part to be included by |\input| should start with:
%
\begin{center}
\begin{tabular}{l}
|\input{childdoc.def}|\\
|\childdocby{|\textit{main}|}|\\
\end{tabular}
\end{center}
%
The directive |\childdocby| is similar to |\childdocof|
described in \secref{sec:include},
but the subsequent selection of content must be done manually.
To that end, both |\ifchilddoc| and |\ifchilddocmanual|
will be true upon processing of a part,
and the name of the part is stored in |\childdocname|.
Note that |\jobname| will be set to the filename of the current part
so that each part receives an individual |.aux| file
that does not interfere with the |.aux| file(s) of the main document.
This behaviour can be altered by the alternative form
|\childdocby[*]{|\textit{main}|}| (with a non-empty optional argument)
which uses the |.aux| file of the main document
by setting |\jobname| to \textit{main}.

%%%%%%%%%%%%%%%%%%%%%%%%%%%%%%%%%%%%%%%%%%%%%%%%%%%%%%%%%%%%%%%%%%%%%%%%%%%%%%%%
\subsection{Driver Development}
\label{sec:driver}

The \textsf{childdoc} mechanism can also be use for the development
of definition files such as \LaTeX{} styles or classes.
This case differs from the above setup with multiple parts
included by |\include| in that no |\includeonly| should be invoked.
This can be achieved by starting the include file
(before |\ProvidesPackage|) with:
%
\begin{center}
\begin{tabular}{l}
|\input{childdoc.def}|\\
|\childdocforward{|\textit{main}|}|\\
\end{tabular}
\end{center}
%
or alternatively with:
%
\begin{center}
\begin{tabular}{l}
|\input{childdoc.def}|\\
|\childdocby{|\textit{main}|}|\\
\end{tabular}
\end{center}
%
Both forms have slightly different effects as described above.
The main file is prepared as usual, see \secref{sec:include}.

%%%%%%%%%%%%%%%%%%%%%%%%%%%%%%%%%%%%%%%%%%%%%%%%%%%%%%%%%%%%%%%%%%%%%%%%%%%%%%%%
\subsection{Legacy Detection}
\label{sec:detection}

The directive |\childdocmain| in the main file can detect
whether the complete document or merely a child is to be compiled
even without using the directive |\childdocof|.
This method is deprecated because it is less robust
and there is no compelling reason to use it;
it is merely provided for backward compatibility
and it may be removed in future versions.

If the detection mechanism is to be used,
it is mandatory to correctly specify
the filename of the main file as the argument of |\childdocmain|:
%
\begin{center}
\begin{tabular}{l}
|\input{childdoc.def}|\\
|\childdocmain{|\textit{main}|}|\\
\end{tabular}
\end{center}
%
If |\jobname| does not match the argument \textit{main} of |\childdocmain|,
it is assumed that |\jobname| points to the child file to be compiled.
When using |\childdocmain| with the main file specified as argument,
it suffices to start a child file
with just |\input{|\textit{main}|}|
without loading of the package and using |\childdocof|.
If instead all processing is done
with the appropriate \textsf{childdoc} directives,
the argument of \textit{main} of |\childdocmain| can be empty.

An alternative version of the command line processing described
in \secref{sec:commandline} using the detection mechanism reads:
%
\begin{center}
|... -jobname "|\textit{target}|" "|[\textit{flags}]%
[|\def\jobname{|\textit{dest}|}|]|\input{|\textit{main}|}"|
\end{center}

%%%%%%%%%%%%%%%%%%%%%%%%%%%%%%%%%%%%%%%%%%%%%%%%%%%%%%%%%%%%%%%%%%%%%%%%%%%%%%%%
\subsection{Manual Code}
\label{sec:manual}

In case one cannot be certain whether the definitions file |childdoc.def|
is installed on the target \TeX{} distribution
and one prefers not to ship it,
it is conceivable to paste a few relevant commands into the sources.

To that end, drop all statements |\input{childdoc.def}|
and perform the replacements as outlined below.
Instead of |\childdocmain{|\textit{main}|}| add the following code
to the top of the main file:
%
\begin{center}
\begin{tabular}{l}
|\||ifdefined\childdocname\endinput\||fi\newif\ifchilddoc|\\
|\edef\childdocname{\scantokens\expandafter{\jobname\noexpand}}|\\
|\def\childdocmain{|\textit{main}|}\||ifx\childdocmain\childdocname\||else|\\
|\childdoctrue\includeonly{\childdocname}\let\jobname\childdocmain\||fi|\\
\end{tabular}
\end{center}
%
Instead of |\childdocof{|\textit{main}|}| just include the main file
at the top of each child file:
%
\begin{center}
|\input{|\textit{main}|}|
\end{center}
%
A simple redirection |\childdocforward{|\textit{dest}|}| is achieved by:
%
\begin{center}
|\def\jobname{|\textit{dest}|}\input{\jobname}|
\end{center}
%
The redirection with prefix
|\childdocforwardprefix[|\textit{prefix}|]{|\textit{dest}|}|
is accomplished by:
%
\begin{center}
\begin{tabular}{l}
|{\edef\jobname{\scantokens\expandafter{\jobname\noexpand}}|\\
|\def\redirectjob |\textit{prefix}|#1~~~{\gdef\jobname{|\textit{dest}|#1}}|\\
|\expandafter\redirectjob\jobname~~~}\input{\jobname}|
\end{tabular}
\end{center}

In an alternative approach,
child documents can be compiled by a specific command line
without additional code or specific definitions:
%
\begin{center}
|... -jobname "|\textit{target}|" "|[\textit{flags}]%
|\includeonly{|\textit{dest}|}\input{|\textit{main}|}"|
\end{center}
%

%%%%%%%%%%%%%%%%%%%%%%%%%%%%%%%%%%%%%%%%%%%%%%%%%%%%%%%%%%%%%%%%%%%%%%%%%%%%%%%%
%%%%%%%%%%%%%%%%%%%%%%%%%%%%%%%%%%%%%%%%%%%%%%%%%%%%%%%%%%%%%%%%%%%%%%%%%%%%%%%%
\section{Information}

%%%%%%%%%%%%%%%%%%%%%%%%%%%%%%%%%%%%%%%%%%%%%%%%%%%%%%%%%%%%%%%%%%%%%%%%%%%%%%%%
\subsection{Copyright}

Copyright \copyright{} 2017--2018 Niklas Beisert

This work may be distributed and/or modified under the
conditions of the \LaTeX{} Project Public License, either version 1.3
of this license or (at your option) any later version.
The latest version of this license is in
  \url{http://www.latex-project.org/lppl.txt}
and version 1.3 or later is part of all distributions of \LaTeX{}
version 2005/12/01 or later.

This work has the LPPL maintenance status `maintained'.

The Current Maintainer of this work is Niklas Beisert.

This work consists of the files |README.txt|, |childdoc.ins| and |childdoc.dtx|
as well as the derived files |childdoc.def|, |cdocsamp.tex|
with |cdocsch1.tex|, |cdocsch2.tex|, |cdocspt3.tex|, |cdocspt4.tex|,
|cdocsdrf.tex|, |cdocsfn1.tex|, |cdocsfn2.tex|
as well as |childdoc.pdf|.

%%%%%%%%%%%%%%%%%%%%%%%%%%%%%%%%%%%%%%%%%%%%%%%%%%%%%%%%%%%%%%%%%%%%%%%%%%%%%%%%
\subsection{Files and Installation}

The package consists of the files:
%
\begin{center}
\begin{tabular}{ll}
    |README.txt|   & readme file \\
    |childdoc.ins| & installation file \\
    |childdoc.dtx| & source file \\
    |childdoc.def| & definition file \\
    |cdocsamp.tex| & sample main file \\
    |cdocsch1.tex| & sample include file \\
    |cdocsch2.tex| & sample include file \\
    |cdocspt3.tex| & sample part file \\
    |cdocspt4.tex| & sample part file \\
    |cdocsdrf.tex| & sample redirection file \\
    |cdocsfn1.tex| & sample redirection file \\
    |cdocsfn2.tex| & sample redirection file \\
    |childdoc.pdf| & manual
\end{tabular}
\end{center}
%
The distribution consists of the files
|README.txt|, |childdoc.ins| and |childdoc.dtx|.
%
\begin{itemize}
\item
Run (pdf)\LaTeX{} on |childdoc.dtx|
to compile the manual |childdoc.pdf| (this file).
\item
Run \LaTeX{} on |childdoc.ins| to create the definitions file |childdoc.def|
and the sample |cdocsamp.tex| with include files
|cdocsch1.tex|, |cdocsch2.tex|, |cdocspt3.tex|, |cdocspt4.tex|,
|cdocsdrf.tex|, |cdocsfn1.tex|, |cdocsfn2.tex|.
Then copy the file |childdoc.def| to an appropriate directory of your \LaTeX{}
distribution, e.g.\ \textit{texmf-root}|/tex/latex/childdoc|.
\end{itemize}

%%%%%%%%%%%%%%%%%%%%%%%%%%%%%%%%%%%%%%%%%%%%%%%%%%%%%%%%%%%%%%%%%%%%%%%%%%%%%%%%
\subsection{Related CTAN Packages}

There are several other packages which offer a similar functionality:
%
\begin{itemize}
\item
The packages
\href{http://ctan.org/pkg/docmute}{\textsf{docmute}},
\href{http://ctan.org/pkg/includex}{\textsf{includex}} and
\href{http://ctan.org/pkg/standalone}{\textsf{standalone}}
provide commands to include only the document body of
a child file thus allowing both files to be compiled individually.
\item
The packages \href{http://ctan.org/pkg/subdocs}{\textsf{subdocs}}
and \href{http://ctan.org/pkg/subfiles}{\textsf{subfiles}}
provide structures in which the main and child documents can be
encapsulated and allowing them to be compiled individually.
The inclusion mechanism is different from the conventional |\include|.
\item
The package \href{http://ctan.org/pkg/combine}{\textsf{combine}}
is an elaborate solution to combine several documents into one.
\end{itemize}
%
See also the CTAN topic \href{http://ctan.org/topic/subdocs}{\textsf{subdocs}}
for further related packages.
The present package differs from the above solutions in that
a document structure constructed with the conventional |\include| mechanism
just needs two extra commands at the top of every file
such that all constituent files can be compiled individually.

%%%%%%%%%%%%%%%%%%%%%%%%%%%%%%%%%%%%%%%%%%%%%%%%%%%%%%%%%%%%%%%%%%%%%%%%%%%%%%%%
%\subsection{Feature Suggestions}
%
%The following is a list of features which may be useful for future
%versions of this package:
%%
%\begin{itemize}
%\item
%\ldots
%\end{itemize}

%%%%%%%%%%%%%%%%%%%%%%%%%%%%%%%%%%%%%%%%%%%%%%%%%%%%%%%%%%%%%%%%%%%%%%%%%%%%%%%%
\subsection{Revision History}

%%%%%%%%%%%%%%%%%%%%%%%%%%%%%%%%%%%%%%%%
\paragraph{v2.0:} 2018/12/30

\begin{itemize}
\item
immediate forward processing
\item
added |\childdocby| mechanism
\item
manual restructured
\end{itemize}

%%%%%%%%%%%%%%%%%%%%%%%%%%%%%%%%%%%%%%%%
\paragraph{v1.6:} 2018/01/17

\begin{itemize}
\item
application for development of include files
\item
corrections to manual
\end{itemize}

%%%%%%%%%%%%%%%%%%%%%%%%%%%%%%%%%%%%%%%%
\paragraph{v1.5:} 2017/05/21

\begin{itemize}
\item
more complete structuring introduced
\item
|\childdocof| introduced
\item
|\childdoc| renamed to |\childdocmain|
\item
|\childredirect| renamed to |\childdocforward| and |\childdocforwardprefix|
and functionality expanded
\end{itemize}

%%%%%%%%%%%%%%%%%%%%%%%%%%%%%%%%%%%%%%%%
\paragraph{v1.0:} 2017/04/27

\begin{itemize}
\item
manual and install package
\item
first version published on CTAN
\end{itemize}

%%%%%%%%%%%%%%%%%%%%%%%%%%%%%%%%%%%%%%%%
\paragraph{v0.6:} 2017/04/26

\begin{itemize}
\item
redirection mechanism added
\end{itemize}

%%%%%%%%%%%%%%%%%%%%%%%%%%%%%%%%%%%%%%%%
\paragraph{v0.5:} 2017/04/26

\begin{itemize}
\item
functionality in definition file
\end{itemize}


%%%%%%%%%%%%%%%%%%%%%%%%%%%%%%%%%%%%%%%%%%%%%%%%%%%%%%%%%%%%%%%%%%%%%%%%%%%%%%%%
%%%%%%%%%%%%%%%%%%%%%%%%%%%%%%%%%%%%%%%%%%%%%%%%%%%%%%%%%%%%%%%%%%%%%%%%%%%%%%%%
%%%%%%%%%%%%%%%%%%%%%%%%%%%%%%%%%%%%%%%%%%%%%%%%%%%%%%%%%%%%%%%%%%%%%%%%%%%%%%%%
\appendix

\settowidth\MacroIndent{\rmfamily\scriptsize 000\ }

 \DocInput{childdoc.dtx}

\end{document}
%</driver>
% \fi
%
% %%%%%%%%%%%%%%%%%%%%%%%%%%%%%%%%%%%%%%%%%%%%%%%%%%%%%%%%%%%%%%%%%%%%%%%%%%%%%%
% %%%%%%%%%%%%%%%%%%%%%%%%%%%%%%%%%%%%%%%%%%%%%%%%%%%%%%%%%%%%%%%%%%%%%%%%%%%%%%
% \section{Sample}
%\iffalse
%<*samplemain>
%\fi
%
% The following presents a sample document
% with two chapters, two parts, a title page,
% a compile flag as well as three forwarding files to set the flag.
% It consists of eight |.tex| files:
% \begin{center}
% \begin{tabular}{ll}
% |cdocsamp.tex|&main file\\
% |cdocsch1.tex|&include file for chapter 1\\
% |cdocsch2.tex|&include file for chapter 2\\
% |cdocspt3.tex|&include file for part 3\\
% |cdocspt4.tex|&include file for part 4\\
% |cdocsdrf.tex|&forwarding file for main file in draft mode\\
% |cdocsfi1.tex|&forwarding file for final version of chapter 1\\
% |cdocsfi2.tex|&forwarding file for final version of chapter 2\\
% \end{tabular}
% \end{center}
% Each of the eight files can be compiled directly by the \LaTeX{} compiler.
%
% %%%%%%%%%%%%%%%%%%%%%%%%%%%%%%%%%%%%%%
% \paragraph{Main File.}
%
% The main file is called |cdocsamp.tex|.
%
% Load the \textsf{childdoc} definitions and
% declare the filename for the main document:
%    \begin{macrocode}
\input{childdoc.def}
\childdocmain{}
%    \end{macrocode}

% Optional override for |\version| flag:
%    \begin{macrocode}
%%\ifchilddoc\else\providecommand{\version}{draft}\fi
%    \end{macrocode}

% Define the default values for the |\version| flag
% (|final| for the main file and |draft| for childs):
%    \begin{macrocode}
\ifchilddoc
\providecommand{\version}{draft}
\else
\providecommand{\version}{final}
\fi
%    \end{macrocode}

% Load the standard document class:
%    \begin{macrocode}
\documentclass[12pt]{article}
%    \end{macrocode}

% Start the document body:
%    \begin{macrocode}
\begin{document}
%    \end{macrocode}

% Declare a title page.
% Print title, part of document being processed and version flag:
%    \begin{macrocode}
\addtocounter{page}{-1}
\begin{center}
{\LARGE\bfseries{}childdoc example\par}
\vspace{1cm}
\ifchilddoc
\ifchilddocmanual part\else chapter\fi:
`\childdocname' of `\childdocjob'\par
\else
main document: `\childdocjob'\par
\fi
version: \version\par
\end{center}
\newpage
%    \end{macrocode}

% Manually include selected file,
% otherwise process as usual:
%    \begin{macrocode}
\ifchilddocmanual
\section*{part `\childdocname'}
\input{\childdocname}
\else
%    \end{macrocode}

% Include the two chapters:
%    \begin{macrocode}
\include{cdocsch1}
\include{cdocsch2}
%    \end{macrocode}

% Include the two parts unless only chapters should be displayed:
%    \begin{macrocode}
\ifchilddoc\else
\section{part three}
\input{cdocspt3}
\section{part four}
\input{cdocspt4}
\fi
%    \end{macrocode}

% Process as usual until here:
%    \begin{macrocode}
\fi
%    \end{macrocode}

% End of document body:
%    \begin{macrocode}
\end{document}
%    \end{macrocode}
%\iffalse
%</samplemain>
%\fi
%
% %%%%%%%%%%%%%%%%%%%%%%%%%%%%%%%%%%%%%%
% \paragraph{Chapter Include Files.}
%
% The include files are called |cdocsch1.tex| and |cdocsch2.tex|.
%
%\iffalse
%<*samplechap1|samplechap2>
%\fi

% Optional override for |\version| flag:
%    \begin{macrocode}
%%\providecommand{\version}{final}
%    \end{macrocode}

% Include the main document:
%    \begin{macrocode}
\input{childdoc.def}
\childdocof{cdocsamp}
%    \end{macrocode}

%\iffalse
%</samplechap1|samplechap2>
%\fi
%
%\iffalse
%<*samplechap1>
%\fi
% Some text for chapter 1:
%    \begin{macrocode}
\section{one}
some text in chapter one
%    \end{macrocode}

%\iffalse
%</samplechap1>
%\fi
% Some text for chapter 2:
%\iffalse
%<*samplechap2>
%\fi
%    \begin{macrocode}
\section{two}
more text in chapter two
%    \end{macrocode}

%\iffalse
%</samplechap2>
%\fi
%
% %%%%%%%%%%%%%%%%%%%%%%%%%%%%%%%%%%%%%%
% \paragraph{Part Include Files.}
%
% The include files are called |cdocspt3.tex| and |cdocspt4.tex|.
%
%\iffalse
%<*samplepart3|samplepart4>
%\fi

% Optional override for |\version| flag:
%    \begin{macrocode}
%%\providecommand{\version}{final}
%    \end{macrocode}

% Include the main document:
%    \begin{macrocode}
\input{childdoc.def}
\childdocby{cdocsamp}
%    \end{macrocode}

%\iffalse
%</samplepart3|samplepart4>
%\fi
%
%\iffalse
%<*samplepart3>
%\fi
% Some text for part 3:
%    \begin{macrocode}
some text in part three
%    \end{macrocode}

%\iffalse
%</samplepart3>
%\fi
% Some text for part 4:
%\iffalse
%<*samplepart4>
%\fi
%    \begin{macrocode}
more text in part four
%    \end{macrocode}

%\iffalse
%</samplepart4>
%\fi
%
% %%%%%%%%%%%%%%%%%%%%%%%%%%%%%%%%%%%%%%
% \paragraph{Forwarding for a Complete Draft.}
%
% The following forwarding file |cdocsdrf.tex|
% compiles the main document in draft mode:
%\iffalse
%<*sampledraft>
%\fi
%    \begin{macrocode}
\def\version{draft}
\input{childdoc.def}
\childdocforward{cdocsamp}
%    \end{macrocode}

%\iffalse
%</sampledraft>
%\fi
%
% %%%%%%%%%%%%%%%%%%%%%%%%%%%%%%%%%%%%%%
% \paragraph{Forwarding for Final Version of the Chapters.}
%
% The following forwarding files |cdocsfn1.tex| and |cdocsfn2.tex|
% (with identical content)
% compile the final versions of the child documents
% |cdocsch1.tex| and |cdocsch2.tex|, respectively:
%\iffalse
%<*samplefinal>
%\fi
%    \begin{macrocode}
\def\version{final}
\input{childdoc.def}
\childdocforwardprefix[cdocsamp]{cdocsfn}{cdocsch}
%    \end{macrocode}

%\iffalse
%</samplefinal>
%\fi
%
% %%%%%%%%%%%%%%%%%%%%%%%%%%%%%%%%%%%%%%
% \paragraph{Command Line Processing.}
%
% The following three command lines generate the output files
% |cdocscld|, |cdocscl1| and |cdocscl2|
% which should be identical to
% |cdocsdrf|, |cdocsch1| and |cdocsfn2|, respectively:
% \begin{center}
% \begin{tabular}{l}
% |latex -jobname cdocscld \|\\
% |  "\def\version{draft}\input{childdoc.def}\childdocforward{cdocsamp}"|\\
% |latex -jobname cdocscl1 \|\\
% |  "\input{childdoc.def}\childdocforward[cdocsamp]{cdocsch1}"|\\
% |latex -jobname cdocscl2 \|\\
% |  "\def\version{final}\input{childdoc.def}\childdocforward{cdocsch2}"|
% \end{tabular}
% \end{center}
% Note that the trailing backslash on each first line
% merely continues the input to the second line
% (for convenient cut ant paste).
% Furthermore, the command |latex| can be replaced by any
% of its alternative versions such as |pdflatex|.
%
% %%%%%%%%%%%%%%%%%%%%%%%%%%%%%%%%%%%%%%%%%%%%%%%%%%%%%%%%%%%%%%%%%%%%%%%%%%%%%%
% %%%%%%%%%%%%%%%%%%%%%%%%%%%%%%%%%%%%%%%%%%%%%%%%%%%%%%%%%%%%%%%%%%%%%%%%%%%%%%
% \section{Implementation}
%\iffalse
%<*package>
%\fi
%
% This section describes the definitions file |childdoc.def|.

% The definitions cannot be loaded using |\usepackage| or |\RequirePackage|
% which has a mechanism to prevent loading a style file more than once.
% When loading the definitions by means of |\input|
% multiple instances have to be prevented manually:
%\iffalse
%This code needs to be before the `\ProvidesFile' directive
%which is defined at the beginning of this file.
%Therefore it is also placed there and commented out here.
%</package>
%<*discard>
%\fi
%    \begin{macrocode}
\ifdefined\childdocmain\endinput\fi
%    \end{macrocode}
%\iffalse
%</discard>
%<*package>
%\fi
%
% \macro{\ifchilddoc}
% \macro{\ifchilddocmanual}
% The conditional |\ifchilddoc| tells whether a
% child (true) or main (false) document is being compiled.
% The conditional |\ifchilddocmanual| tells whether
% the |\includeonly| mechanism is used (false) or
% the selection of child files must be performed manually (true).
% The definitions initialise to false:
%    \begin{macrocode}
\newif\ifchilddoc
\newif\ifchilddocmanual
%    \end{macrocode}

% \macro{\childdocname}
% \macro{\childdocjob}
% The macro |\childdocname| stores the name of the main document
% to be compiled. The macro |\childdocjob| stores the name of
% the document on which the \LaTeX{} compiler was originally invoked.
% The content of |\jobname| cannot be compared
% to filenames specified in the source due to different catcodes.
% The following code rescans |\jobname|, stores the result
% in |\childdocname| and saves a copy in |\childdocjob|:
%    \begin{macrocode}
\edef\childdocname{\scantokens\expandafter{\jobname\noexpand}}
\let\childdocjob\childdocname
%    \end{macrocode}

% \macro{\childdocdisable}
% The macro |\childdocdisable| prevents the main file
% from being processed more than once.
% At this stage, the main document command |\childdocmain|
% is assumed to be called once again where it should do nothing.
% Any subsequent call to it should prevent
% a secondary processing of the main document
% It overwrites the forwarding commands
% |\childdocof| and |\childdocforward|
% with empty macros to prevent further inclusions of the main document:
%    \begin{macrocode}
\newcommand{\childdocdisable}
{
  \renewcommand{\childdocmain}[1]{\renewcommand{\childdocmain}[1]{\endinput}}
  \renewcommand{\childdocof}[1]{}
  \renewcommand{\childdocby}[2][]{}
  \renewcommand{\childdocforward}[2][]{}
  \renewcommand{\childdocdisable}{}
}
%    \end{macrocode}

% \macro{\childdocmain}
% The macro |\childdocmain| is to be called at the top of the main file
% with nothing or the main filename (without extension) as argument.
% First, it breaks loops.
% If the argument is not empty and does not match |\childdocname|
% (which is set by the first inclusion of |childdoc.def|),
% |\ifchilddoc| is set to true, |\includeonly| is applied to the child file
% and |\jobname| is set to the main file
% (for proper handling of |.aux| files):
%    \begin{macrocode}
\newcommand{\childdocmain}[1]
{
  \childdocdisable\childdocmain{}
  \if?#1?\else
    \begingroup
      \def\childdoctmp{#1}
      \ifx\childdoctmp\childdocname
        \def\childdoctmp{}
      \else
        \def\childdoctmp
        {
          \childdoctrue
          \includeonly{\childdocname}
          \def\childdocjob{#1}
          \def\jobname{#1}
        }
      \fi
      \expandafter
    \endgroup
    \childdoctmp
  \fi
}
%    \end{macrocode}

% \macro{\childdocof}
% The command |\childdocof| redirects
% compilation to the main file |#1|.
%    \begin{macrocode}
\newcommand{\childdocof}[1]
{
  \childdocdisable
  \childdoctrue
  \includeonly{\childdocname}
  \def\jobname{#1}
  \def\childdocjob{#1}
  \input{#1}
}
%    \end{macrocode}

% \macro{\childdocby}
% The command |\childdocby| ....
%    \begin{macrocode}
\newcommand{\childdocby}[2][]
{
  \childdocdisable
  \childdoctrue
  \childdocmanualtrue
  \if?#1?\else
    \def\jobname{#2}
  \fi
  \def\childdocjob{#2}
  \input{#2}
  \endinput
}
%    \end{macrocode}

% \macro{\childdocforward}
% The command |\childdocforward| redirects
% compilation to the main file or
% (if the optional argument is given) a child file.
% Parameters are set as if the main file
% or a child file starting with |\childdocof| was compiled.
% Then compilation is handed over to the main file:
%    \begin{macrocode}
\newcommand{\childdocforward}[2][]
{
  \begingroup
    \if?#1?
      \def\childdoctmp
      {
        \def\childdocname{#2}
        \def\childdocjob{#2}
        \def\jobname{#2}
        \input{#2}
        \endinput
      }
    \else
      \def\childdoctmp
      {
        \childdocdisable
        \def\childdocname{#2}
        \childdoctrue
        \includeonly{#2}
        \def\childdocjob{#1}
        \def\jobname{#1}
        \input{#1}
        \endinput
      }
    \fi
    \expandafter
  \endgroup
  \childdoctmp
}
%    \end{macrocode}

% \macro{\childdocforwardprefix}
% The command |\childdocforwardprefix| redirects
% compilation to the main or a child file by means of a pattern.
% The prefix |#1| in the current filename is replaced by |#2|
% and the suffix of the current filename is kept
% (it is assumed that the filename does not contain the substring `|~~~|'
% which is used as a delimiter).
% Compilation is handed over to the new file by |\childdocforward|:
%    \begin{macrocode}
\newcommand{\childdocforwardprefix}[3][]
{
  \begingroup
    \def\childdocextract #2##1~~~{\def\childdoctmp{\childdocforward[#1]{#3##1}}}
    \expandafter\childdocextract\childdocname~~~
    \expandafter
  \endgroup
  \childdoctmp
}
%    \end{macrocode}

% \macro{\childdoc}
% The deprecated macro |\childdoc| is a legacy version of |\childdocmain|:
%    \begin{macrocode}
\newcommand{\childdoc}{\childdocmain}
%    \end{macrocode}

% \macro{\childdocredirect}
% The deprecated macro |\childdocredirect| is a legacy version
% of |\childdocforward| and |\childdocforwardprefix|:
%    \begin{macrocode}
\newcommand{\childdocredirect}[2][]
{
  \begingroup
    \if?#1?
      \def\childdoctmp{\childdocforward{#2}}
    \else
      \def\childdoctmp{\childdocforwardprefix{#1}{#2}}
    \fi
    \expandafter
  \endgroup
  \childdoctmp
}
%    \end{macrocode}

%\iffalse
%</package>
%\fi
%
\endinput
\childdocforward[|\textit{main}|]{|\textit{dest}|}"|
\end{center}
%
Here \textit{target} is the name of the output file,
\textit{main} is the name of the main file
and \textit{dest} is the name of the main or child file to be processed
(all filenames without extensions).
The optional argument \textit{main} can be omitted
if \textit{main} matches \textit{dest}.
Optionally, compilation \textit{flags} can be defined via |\def| commands.
This command line makes the \TeX{} engine believe
it is compiling the file \textit{target}
whose content is specified as the latter parameter.
The provided code then forwards the processing to
\textit{main} or \textit{dest} as described in \secref{sec:forward}.

%%%%%%%%%%%%%%%%%%%%%%%%%%%%%%%%%%%%%%%%%%%%%%%%%%%%%%%%%%%%%%%%%%%%%%%%%%%%%%%%
\subsection{Include by Input}
\label{sec:input}

Including child documents by |\include| has some restrictions by design.
Most notably, the content of a child document always occupies
its own set of pages; pages cannot be shared between child documents.
Usually, this behaviour makes perfect sense
because each child document contain an essential part of the document.
However, in some situations it may be desirable to compose
a document from a collection of parts
without having mandatory page breaks between then.
For this case, the package
provides a mechanism to include parts
by |\input| which can also be processed individually.
However, by construction this mechanism
requires manual handling of the content to be output.

%%%%%%%%%%%%%%%%%%%%%%%%%%%%%%%%%%%%%%%%
\DescribeMacro{\ifchilddocmanual}
The main file should be prepared as usual, see \secref{sec:include}.
However, the document body must make a distinction
between processing of an individual part and of the main document, e.g.:
%
\begin{center}
\begin{tabular}{l}
|\ifchilddocmanual|\\
|\input{\childdocname}|\\
|\||else|\\
\textit{document body with }|\input{|\textit{part}|}|\\
|\||fi|
\end{tabular}
\end{center}
%
The conditional |\ifchilddocmanual| is true whenever
a part to be included by |\input| is being compiled,
and the name of the part is stored in |\childdocname|.

%%%%%%%%%%%%%%%%%%%%%%%%%%%%%%%%%%%%%%%%
\DescribeMacro{\childdocby}
Each part to be included by |\input| should start with:
%
\begin{center}
\begin{tabular}{l}
|% \iffalse
%
% childdoc.dtx Copyright (C) 2017-2018 Niklas Beisert
%
% This work may be distributed and/or modified under the
% conditions of the LaTeX Project Public License, either version 1.3
% of this license or (at your option) any later version.
% The latest version of this license is in
%   http://www.latex-project.org/lppl.txt
% and version 1.3 or later is part of all distributions of LaTeX
% version 2005/12/01 or later.
%
% This work has the LPPL maintenance status `maintained'.
%
% The Current Maintainer of this work is Niklas Beisert.
%
% This work consists of the files childdoc.dtx and childdoc.ins
% and the derived files childdoc.def and cdocsamp.tex with
% cdocsch1.tex, cdocsch2.tex, cdocsdrf.tex, cdocsfn1.tex, cdocsfn2.tex.
%
%<package>\ifdefined\childdocmain\endinput\fi
%<package>\ProvidesFile{childdoc.def}[2018/12/30 v2.0 child document driver]
%<samplemain>\ProvidesFile{cdocsamp.tex}[2018/12/30 v2.0 sample for childdoc]
%<*driver>
%\ProvidesFile{childdoc.drv}[2018/12/30 v2.0 childdoc reference manual file]
\PassOptionsToClass{10pt,a4paper}{article}
\documentclass{ltxdoc}

\usepackage[margin=35mm]{geometry}
\usepackage{hyperref}
\usepackage{hyperxmp}
\usepackage[usenames]{color}

\hypersetup{colorlinks=true}
\hypersetup{pdfstartview=FitH}
\hypersetup{pdfpagemode=UseNone}
\hypersetup{pdfsource={}}
\hypersetup{pdflang={en-UK}}
\hypersetup{pdfcopyright={Copyright 2017-2018 Niklas Beisert.
  This work may be distributed and/or modified under the
  conditions of the LaTeX Project Public License, either version 1.3
  of this license or (at your option) any later version.}}
\hypersetup{pdflicenseurl={http://www.latex-project.org/lppl.txt}}
\hypersetup{pdfcontactaddress={ETH Zurich, ITP, HIT K,
  Wolfgang-Pauli-Strasse 27}}
\hypersetup{pdfcontactpostcode={8093}}
\hypersetup{pdfcontactcity={Zurich}}
\hypersetup{pdfcontactcountry={Switzerland}}
\hypersetup{pdfcontactemail={nbeisert@itp.phys.ethz.ch}}
\hypersetup{pdfcontacturl={http://people.phys.ethz.ch/\xmptilde nbeisert/}}

\newcommand{\secref}[1]{\hyperref[#1]{section \ref*{#1}}}

\parskip1ex
\parindent0pt
\let\olditemize\itemize
\def\itemize{\olditemize\parskip0pt}

\begin{document}

\title{The \textsf{childdoc} Package}
\hypersetup{pdftitle={The childdoc Package}}
\author{Niklas Beisert\\[2ex]
  Institut f\"ur Theoretische Physik\\
  Eidgen\"ossische Technische Hochschule Z\"urich\\
  Wolfgang-Pauli-Strasse 27, 8093 Z\"urich, Switzerland\\[1ex]
  \href{mailto:nbeisert@itp.phys.ethz.ch}
  {\texttt{nbeisert@itp.phys.ethz.ch}}}
\hypersetup{pdfauthor={Niklas Beisert}}
\hypersetup{pdfsubject={Manual for the LaTeX2e Package childdoc}}
\date{30 December 2018, \textsf{v2.0}}
\maketitle

\begin{abstract}\noindent
\textsf{childdoc} is a \LaTeXe{} package
that enables the direct compilation
of document sections included by |\include|
to individual files.
\end{abstract}

\begingroup
\parskip0ex
\tableofcontents
\endgroup

%%%%%%%%%%%%%%%%%%%%%%%%%%%%%%%%%%%%%%%%%%%%%%%%%%%%%%%%%%%%%%%%%%%%%%%%%%%%%%%%
%%%%%%%%%%%%%%%%%%%%%%%%%%%%%%%%%%%%%%%%%%%%%%%%%%%%%%%%%%%%%%%%%%%%%%%%%%%%%%%%
\section{Introduction}

\LaTeX{} provides a mechanism to structure a large document (such as a book)
into a main file and several child files (containing the chapters)
using the |\include| command.
This mechanism is beneficial for documents
which span hundreds of pages in order to
make the source file(s) more manageable.
Moreover, compilation can be restricted to
selected child files by means of the |\includeonly| command.
The latter feature can be used to reduce the compilation time while editing
(this was significantly more useful in the earlier days of \LaTeX{})
or to generate a smaller document which is easier to navigate.
Another application of |\includeonly| is to generate
documents consisting of selected parts of the complete document.

However, there are a few drawbacks of the plain |\include| mechanism:
\begin{itemize}
\item
The child files cannot be compiled on their own,
they can only be compiled via the main file.
A naive editing environment
(such as a text editor with an option
to have the current file processed by \LaTeX)
may require one to switch to the main file before compiling;
attempting to compile the child file produces errors.
\item
The main file must be modified (each time)
to adjust the |\includeonly| command
to the present needs. This easily leaves the main file in a messy state.
\item
The generated document will always carry the filename
of the main document. This is inconvenient if
several child files are to be compiled and
to be kept for distribution.
\end{itemize}

The present package provides a simple interface
to make child files individually compilable by \LaTeX{}.
Compiling a child file then has the same effect as compiling
the main file with an |\includeonly| command
to select the appropriate child.
Moreover the generated document will carry the name of the child
rather than the main file.
This resolves all three above issues.

This feature is meant to make the editing of books,
thesis documents and lecture notes somewhat more convenient.
However, the package can also be used efficiently for
composing a series of documents (such as exercise sheets)
which are typically distributed individually.
It then assists the author in generating the individual documents
(potentially in different versions)
as well as a document containing the collected series.
Another application is in developing style files
or other kinds of included material
where compilation of the style file could redirect
to a sample or test file.

%%%%%%%%%%%%%%%%%%%%%%%%%%%%%%%%%%%%%%%%%%%%%%%%%%%%%%%%%%%%%%%%%%%%%%%%%%%%%%%%
%%%%%%%%%%%%%%%%%%%%%%%%%%%%%%%%%%%%%%%%%%%%%%%%%%%%%%%%%%%%%%%%%%%%%%%%%%%%%%%%
\section{Usage}

First of all, the package \textsf{childdoc} is \emph{not} a standard
\LaTeXe{} |.sty| style file! Therefore it needs to be invoked in
a non-standard way.

%%%%%%%%%%%%%%%%%%%%%%%%%%%%%%%%%%%%%%%%%%%%%%%%%%%%%%%%%%%%%%%%%%%%%%%%%%%%%%%%
\subsection{Included Files}
\label{sec:include}

%%%%%%%%%%%%%%%%%%%%%%%%%%%%%%%%%%%%%%%%
\DescribeMacro{\childdocmain}
To use the package, add the commands
\begin{center}
\begin{tabular}{l}
|\input{childdoc.def}|\\
|\childdocmain{}|\\
\end{tabular}
\end{center}
at the very top of the main \LaTeX{} file,
in particular \emph{before} the |\documentclass| statement!
The argument of |\childdocmain| should be left empty
(but it must be present).

%%%%%%%%%%%%%%%%%%%%%%%%%%%%%%%%%%%%%%%%
\DescribeMacro{\childdocof}
Furthermore, add the commands
\begin{center}
\begin{tabular}{l}
|\input{childdoc.def}|\\
|\childdocof{|\textit{main}|}|\\
\end{tabular}
\end{center}
at the top of every child file \textit{child}
which is included by |\include{|\textit{child}|}|
from within the main file
(or at least for those files to be compiled individually).
The argument \textit{main} must be the filename of the main file.

There are a couple of
considerations in setting up the main and child documents:

%%%%%%%%%%%%%%%%%%%%%%%%%%%%%%%%%%%%%%%%
\paragraph{Restrictions.}

Please note the following restrictions:
\begin{itemize}
\item
|\childdocmain| must be called with one argument \textit{main}
to ensure compatibility with earlier version of the package.
It must either be empty (|\childdocmain{}|)
or precisely match the filename of the main file in which it is specified.
See \secref{sec:detection} for further information.
\item
The filename \textit{main} must be specified without the |.tex| extension.
\item
The filename \textit{main} is case sensitive
(even in case-insensitive file systems)
due to internal string comparison.
\item
The argument \textit{main} should be fully expanded, it cannot be a macro.
\item
Subdirectories and special characters should be avoided in filenames.
\item
The command |\childdocmain{|\textit{main}|}| must be followed by a whitespace.
It should not be followed immediately by another command
or by a comment mark `|%|'.
This is because the \TeX{} parser reads the token immediately following
the argument of |\childdocmain| and puts it
at the beginning of every child section;
however, a white\-space is ignored.
\end{itemize}

%%%%%%%%%%%%%%%%%%%%%%%%%%%%%%%%%%%%%%%%
\paragraph{Content of Main File.}

It is advisable to place all content in the child files included by |\include|.
Any output contained in the main file will appear in all child documents
unless suppressed manually;
it cannot be suppressed automatically by the |\includeonly| directive
and thus should normally be avoided.
A method to include some content in the main file
by means of conditional processing is described in \secref{sec:conditional}.

%%%%%%%%%%%%%%%%%%%%%%%%%%%%%%%%%%%%%%%%
\paragraph{Page Numbering.}

When only a part of the document is compiled,
the appropriate numbering of pages
(as well as other status parameters)
is determined from the |.aux| files.
The latter contain information from previous passes.
However this information needs to propagate through
all intermediate child documents.
Therefore the page numbering in child documents may well
be inconsistent until the complete document is compiled at least once.

A useful (if unconventional) way to always ensure a consistent
page numbering is to restart the numbering in each child document
and denote the pages by `\textit{child}|.|\textit{page}'
where \textit{child} represents the chapter/section number of the child file.
This can be achieved by the command
|\numberwithin{page}{|\textit{child}|}|
of the \textsf{amsmath} package
where \textit{child} can be |chapter| or |section|
depending on the chosen structuring.
Alternatively, one can modify the macro |\thepage| appropriately
and reset the counter |page| at the start of each child file.

%%%%%%%%%%%%%%%%%%%%%%%%%%%%%%%%%%%%%%%%%%%%%%%%%%%%%%%%%%%%%%%%%%%%%%%%%%%%%%%%
\subsection{Conditional Processing}
\label{sec:conditional}

The package provides a mechanism to compile different versions
of a document. To customise the versions further some conditional processing
can come in handy to distinguish which version is being compiled.
The package provides two macros to describe the compilation context:

%%%%%%%%%%%%%%%%%%%%%%%%%%%%%%%%%%%%%%%%
\DescribeMacro{\ifchilddoc}
The conditional |\ifchilddoc| distinguishes between the compilation of
child documents and the main document:
%
\begin{center}
|\ifchilddoc |\textit{child-code}| |[|\||else |\textit{main-code}]| \||fi|
\end{center}

%%%%%%%%%%%%%%%%%%%%%%%%%%%%%%%%%%%%%%%%
\DescribeMacro{\childdocname}
\DescribeMacro{\childdocjob}
The macro |\childdocname| contains the filename (without extension)
of the main or child file being processed.
Note that |\childdocjob| will always contain the name of the main file.

%%%%%%%%%%%%%%%%%%%%%%%%%%%%%%%%%%%%%%%%
\paragraph{Title Page.}

Conditional processing can be used to include a title or banner page
in the main document when proper precautions are taken.
Importantly, the code in the main file should ensure that the page counter
(as well as other status parameters which are stored in the |.aux| files)
takes the same value after the conditional processing.
Otherwise the page numbers may take divergent values
depending on which part is compiled.

For example, a title page could be declared by:
%
\begin{center}
\begin{tabular}{l}
|\ifchilddoc\||else|\\
|\addtocounter{page}{-1}|\\
\textit{code for title page}\\
|\newpage|\\
|\||fi|
\end{tabular}
\end{center}
%
A banner page for the child documents can be generated by:
%
\begin{center}
\begin{tabular}{l}
|\ifchilddoc|\\
|\addtocounter{page}{-1}|\\
\textit{code for banner page}\\
|\newpage|\\
|\||fi|
\end{tabular}
\end{center}
%
Here one could write a message such as:
\begin{center}
|This is the part \childdocname{} of \childdocjob{}.|
\end{center}

%%%%%%%%%%%%%%%%%%%%%%%%%%%%%%%%%%%%%%%%%%%%%%%%%%%%%%%%%%%%%%%%%%%%%%%%%%%%%%%%
\subsection{Flags}
\label{sec:flags}

The package makes it easy to generate different versions
of the main or child documents.
To this end compilation flags can be defined
and assigned different default values.
They will be particularly useful in conjunction
with the forwarding mechanism described in \secref{sec:forward}.

For example, it may be useful to have a flag |\version|
which can be set to |draft| or |final|.
The document source will contain some conditional code
depending on the value of |\version|.
Suppose further, the flag should default to |final| for the main file
and to |draft| for child files
which is a natural assignment for editing the document.
This is achieved by placing the following code
in the preamble of the main document
(below the |\childdocmain| directive):
%
\begin{center}
\begin{tabular}{l}
|\ifchilddoc|\\
|\providecommand{\version}{draft}|\\
|\||else|\\
|\providecommand{\version}{final}|\\
|\||fi|
\end{tabular}
\end{center}
%
The definition by |\providecommand| makes sure
that previous definitions are not overwritten.
Further statements |\providecommand{\version}{...}|
can thus be added before the above code to override it.

For the main file, one might add a line
(between |\childdocmain| and the above block)
%
\begin{center}
|%\ifchilddoc\||else\providecommand{\version}{draft}\||fi|
\end{center}
%
which can be uncommented to produce a draft version.
Likewise one can add a line to the very top of a child file
(above the |\childdocof{|\textit{main}|}| directive)
%
\begin{center}
|%\providecommand{\version}{final}|
\end{center}
%
which can be uncommented to produce the final version of this child document.

%%%%%%%%%%%%%%%%%%%%%%%%%%%%%%%%%%%%%%%%%%%%%%%%%%%%%%%%%%%%%%%%%%%%%%%%%%%%%%%%
\subsection{Forwarding}
\label{sec:forward}

Different versions of the main or child documents
using compilation flags as described in \secref{sec:flags}
can be (permanently) stored in different files
for convenient compilation, viewing and distribution.
To this end, the package defines a command
to pass on compilation to a different file:

%%%%%%%%%%%%%%%%%%%%%%%%%%%%%%%%%%%%%%%%
\DescribeMacro{\childdocforward}
The command |\childdocforward| redirects processing to
another source file:
%
\begin{center}
\begin{tabular}{l}
|\input{childdoc.def}|\\
|\childdocforward[|\textit{main}|]{|\textit{dest}|}|\\
\end{tabular}
\end{center}
%
The argument \textit{dest} is the destination file
(without extension).
It should be the main file or one of the child files.
Note that further \textsf{childdoc} directives
such as |\childdocof| and |\childdocforward|
in the indicated file will be processed in this form.
The optional argument \textit{main}
passes on directly to the main file \textit{main}
while pretending to compile the child \textit{dest}.
This form behaves as if \textit{dest}
issues |\childdocof{|\textit{main}|}| right away,
and no further \textsf{childdoc} directives will be processed.

%%%%%%%%%%%%%%%%%%%%%%%%%%%%%%%%%%%%%%%%
\DescribeMacro{\...prefix}
In the alternative form |\childdocforwardprefix|,
%
\begin{center}
\begin{tabular}{l}
|\input{childdoc.def}|\\
|\childdocforwardprefix[|\textit{main}|]{|\textit{prefix}|}{|\textit{dest}|}|
\end{tabular}
\end{center}
%
the destination file is determined by a pattern
depending on the current file:
To make this work, the current file must be called
`{\textit{prefix}\hspace{0.2em}\textit{suffix}}'
with \textit{prefix} matching precisely the argument.
Processing is then passed on to the file
`{\textit{dest}\hspace{0.2em}\textit{suffix}}'.
Surely, the same effect is achieved by
directly specifying the
argument `{\textit{dest}\hspace{0.2em}\textit{suffix}}'
in the first form.
However, that requires to set up a different file
for each child. With the alternative form of the command
all these files can have exactly the same content
which simplifies setting them up and maintaining them.

For example, the following file |draft.tex|
with a compilation flag |\version| as described in \secref{sec:flags}
compiles the main document as a draft:
%
\begin{center}
\begin{tabular}{l}
|\def\version{draft}|\\
|\input{childdoc.def}|\\
|\childdocforward{|\textit{main}|}|
\end{tabular}
\end{center}
%
Likewise, the following files |final|\textit{nn}|.tex|
compile the final version of the child document
|child|\textit{nn}|.tex|:
%
\begin{center}
\begin{tabular}{l}
|\def\version{final}|\\
|\input{childdoc.def}|\\
|\childdocforwardprefix{final}{child}|
\end{tabular}
\end{center}
%

Note that when several versions of a main file and/or of each child file
are to be generated, it may be convenient to set up a |Makefile| or
shell script to automatise the process.

%%%%%%%%%%%%%%%%%%%%%%%%%%%%%%%%%%%%%%%%%%%%%%%%%%%%%%%%%%%%%%%%%%%%%%%%%%%%%%%%
\subsection{Command Line Processing}
\label{sec:commandline}

The effect of redirection files can also be achieved by invoking
the \LaTeX{} compiler with a more elaborate command line.
Most conveniently this should be done as part
of a shell script or a |Makefile|.

When using \textsf{childdoc} in the main file, the following
command lines effectively perform a redirection
(note that depending on the shell being used,
backslashes may have to be doubled: `|\|' $\to$ `|\\|'):
%
\begin{center}
|... -jobname "|\textit{target}|" |\\|"|[\textit{flags}]%
|\input{childdoc.def}\childdocforward[|\textit{main}|]{|\textit{dest}|}"|
\end{center}
%
Here \textit{target} is the name of the output file,
\textit{main} is the name of the main file
and \textit{dest} is the name of the main or child file to be processed
(all filenames without extensions).
The optional argument \textit{main} can be omitted
if \textit{main} matches \textit{dest}.
Optionally, compilation \textit{flags} can be defined via |\def| commands.
This command line makes the \TeX{} engine believe
it is compiling the file \textit{target}
whose content is specified as the latter parameter.
The provided code then forwards the processing to
\textit{main} or \textit{dest} as described in \secref{sec:forward}.

%%%%%%%%%%%%%%%%%%%%%%%%%%%%%%%%%%%%%%%%%%%%%%%%%%%%%%%%%%%%%%%%%%%%%%%%%%%%%%%%
\subsection{Include by Input}
\label{sec:input}

Including child documents by |\include| has some restrictions by design.
Most notably, the content of a child document always occupies
its own set of pages; pages cannot be shared between child documents.
Usually, this behaviour makes perfect sense
because each child document contain an essential part of the document.
However, in some situations it may be desirable to compose
a document from a collection of parts
without having mandatory page breaks between then.
For this case, the package
provides a mechanism to include parts
by |\input| which can also be processed individually.
However, by construction this mechanism
requires manual handling of the content to be output.

%%%%%%%%%%%%%%%%%%%%%%%%%%%%%%%%%%%%%%%%
\DescribeMacro{\ifchilddocmanual}
The main file should be prepared as usual, see \secref{sec:include}.
However, the document body must make a distinction
between processing of an individual part and of the main document, e.g.:
%
\begin{center}
\begin{tabular}{l}
|\ifchilddocmanual|\\
|\input{\childdocname}|\\
|\||else|\\
\textit{document body with }|\input{|\textit{part}|}|\\
|\||fi|
\end{tabular}
\end{center}
%
The conditional |\ifchilddocmanual| is true whenever
a part to be included by |\input| is being compiled,
and the name of the part is stored in |\childdocname|.

%%%%%%%%%%%%%%%%%%%%%%%%%%%%%%%%%%%%%%%%
\DescribeMacro{\childdocby}
Each part to be included by |\input| should start with:
%
\begin{center}
\begin{tabular}{l}
|\input{childdoc.def}|\\
|\childdocby{|\textit{main}|}|\\
\end{tabular}
\end{center}
%
The directive |\childdocby| is similar to |\childdocof|
described in \secref{sec:include},
but the subsequent selection of content must be done manually.
To that end, both |\ifchilddoc| and |\ifchilddocmanual|
will be true upon processing of a part,
and the name of the part is stored in |\childdocname|.
Note that |\jobname| will be set to the filename of the current part
so that each part receives an individual |.aux| file
that does not interfere with the |.aux| file(s) of the main document.
This behaviour can be altered by the alternative form
|\childdocby[*]{|\textit{main}|}| (with a non-empty optional argument)
which uses the |.aux| file of the main document
by setting |\jobname| to \textit{main}.

%%%%%%%%%%%%%%%%%%%%%%%%%%%%%%%%%%%%%%%%%%%%%%%%%%%%%%%%%%%%%%%%%%%%%%%%%%%%%%%%
\subsection{Driver Development}
\label{sec:driver}

The \textsf{childdoc} mechanism can also be use for the development
of definition files such as \LaTeX{} styles or classes.
This case differs from the above setup with multiple parts
included by |\include| in that no |\includeonly| should be invoked.
This can be achieved by starting the include file
(before |\ProvidesPackage|) with:
%
\begin{center}
\begin{tabular}{l}
|\input{childdoc.def}|\\
|\childdocforward{|\textit{main}|}|\\
\end{tabular}
\end{center}
%
or alternatively with:
%
\begin{center}
\begin{tabular}{l}
|\input{childdoc.def}|\\
|\childdocby{|\textit{main}|}|\\
\end{tabular}
\end{center}
%
Both forms have slightly different effects as described above.
The main file is prepared as usual, see \secref{sec:include}.

%%%%%%%%%%%%%%%%%%%%%%%%%%%%%%%%%%%%%%%%%%%%%%%%%%%%%%%%%%%%%%%%%%%%%%%%%%%%%%%%
\subsection{Legacy Detection}
\label{sec:detection}

The directive |\childdocmain| in the main file can detect
whether the complete document or merely a child is to be compiled
even without using the directive |\childdocof|.
This method is deprecated because it is less robust
and there is no compelling reason to use it;
it is merely provided for backward compatibility
and it may be removed in future versions.

If the detection mechanism is to be used,
it is mandatory to correctly specify
the filename of the main file as the argument of |\childdocmain|:
%
\begin{center}
\begin{tabular}{l}
|\input{childdoc.def}|\\
|\childdocmain{|\textit{main}|}|\\
\end{tabular}
\end{center}
%
If |\jobname| does not match the argument \textit{main} of |\childdocmain|,
it is assumed that |\jobname| points to the child file to be compiled.
When using |\childdocmain| with the main file specified as argument,
it suffices to start a child file
with just |\input{|\textit{main}|}|
without loading of the package and using |\childdocof|.
If instead all processing is done
with the appropriate \textsf{childdoc} directives,
the argument of \textit{main} of |\childdocmain| can be empty.

An alternative version of the command line processing described
in \secref{sec:commandline} using the detection mechanism reads:
%
\begin{center}
|... -jobname "|\textit{target}|" "|[\textit{flags}]%
[|\def\jobname{|\textit{dest}|}|]|\input{|\textit{main}|}"|
\end{center}

%%%%%%%%%%%%%%%%%%%%%%%%%%%%%%%%%%%%%%%%%%%%%%%%%%%%%%%%%%%%%%%%%%%%%%%%%%%%%%%%
\subsection{Manual Code}
\label{sec:manual}

In case one cannot be certain whether the definitions file |childdoc.def|
is installed on the target \TeX{} distribution
and one prefers not to ship it,
it is conceivable to paste a few relevant commands into the sources.

To that end, drop all statements |\input{childdoc.def}|
and perform the replacements as outlined below.
Instead of |\childdocmain{|\textit{main}|}| add the following code
to the top of the main file:
%
\begin{center}
\begin{tabular}{l}
|\||ifdefined\childdocname\endinput\||fi\newif\ifchilddoc|\\
|\edef\childdocname{\scantokens\expandafter{\jobname\noexpand}}|\\
|\def\childdocmain{|\textit{main}|}\||ifx\childdocmain\childdocname\||else|\\
|\childdoctrue\includeonly{\childdocname}\let\jobname\childdocmain\||fi|\\
\end{tabular}
\end{center}
%
Instead of |\childdocof{|\textit{main}|}| just include the main file
at the top of each child file:
%
\begin{center}
|\input{|\textit{main}|}|
\end{center}
%
A simple redirection |\childdocforward{|\textit{dest}|}| is achieved by:
%
\begin{center}
|\def\jobname{|\textit{dest}|}\input{\jobname}|
\end{center}
%
The redirection with prefix
|\childdocforwardprefix[|\textit{prefix}|]{|\textit{dest}|}|
is accomplished by:
%
\begin{center}
\begin{tabular}{l}
|{\edef\jobname{\scantokens\expandafter{\jobname\noexpand}}|\\
|\def\redirectjob |\textit{prefix}|#1~~~{\gdef\jobname{|\textit{dest}|#1}}|\\
|\expandafter\redirectjob\jobname~~~}\input{\jobname}|
\end{tabular}
\end{center}

In an alternative approach,
child documents can be compiled by a specific command line
without additional code or specific definitions:
%
\begin{center}
|... -jobname "|\textit{target}|" "|[\textit{flags}]%
|\includeonly{|\textit{dest}|}\input{|\textit{main}|}"|
\end{center}
%

%%%%%%%%%%%%%%%%%%%%%%%%%%%%%%%%%%%%%%%%%%%%%%%%%%%%%%%%%%%%%%%%%%%%%%%%%%%%%%%%
%%%%%%%%%%%%%%%%%%%%%%%%%%%%%%%%%%%%%%%%%%%%%%%%%%%%%%%%%%%%%%%%%%%%%%%%%%%%%%%%
\section{Information}

%%%%%%%%%%%%%%%%%%%%%%%%%%%%%%%%%%%%%%%%%%%%%%%%%%%%%%%%%%%%%%%%%%%%%%%%%%%%%%%%
\subsection{Copyright}

Copyright \copyright{} 2017--2018 Niklas Beisert

This work may be distributed and/or modified under the
conditions of the \LaTeX{} Project Public License, either version 1.3
of this license or (at your option) any later version.
The latest version of this license is in
  \url{http://www.latex-project.org/lppl.txt}
and version 1.3 or later is part of all distributions of \LaTeX{}
version 2005/12/01 or later.

This work has the LPPL maintenance status `maintained'.

The Current Maintainer of this work is Niklas Beisert.

This work consists of the files |README.txt|, |childdoc.ins| and |childdoc.dtx|
as well as the derived files |childdoc.def|, |cdocsamp.tex|
with |cdocsch1.tex|, |cdocsch2.tex|, |cdocspt3.tex|, |cdocspt4.tex|,
|cdocsdrf.tex|, |cdocsfn1.tex|, |cdocsfn2.tex|
as well as |childdoc.pdf|.

%%%%%%%%%%%%%%%%%%%%%%%%%%%%%%%%%%%%%%%%%%%%%%%%%%%%%%%%%%%%%%%%%%%%%%%%%%%%%%%%
\subsection{Files and Installation}

The package consists of the files:
%
\begin{center}
\begin{tabular}{ll}
    |README.txt|   & readme file \\
    |childdoc.ins| & installation file \\
    |childdoc.dtx| & source file \\
    |childdoc.def| & definition file \\
    |cdocsamp.tex| & sample main file \\
    |cdocsch1.tex| & sample include file \\
    |cdocsch2.tex| & sample include file \\
    |cdocspt3.tex| & sample part file \\
    |cdocspt4.tex| & sample part file \\
    |cdocsdrf.tex| & sample redirection file \\
    |cdocsfn1.tex| & sample redirection file \\
    |cdocsfn2.tex| & sample redirection file \\
    |childdoc.pdf| & manual
\end{tabular}
\end{center}
%
The distribution consists of the files
|README.txt|, |childdoc.ins| and |childdoc.dtx|.
%
\begin{itemize}
\item
Run (pdf)\LaTeX{} on |childdoc.dtx|
to compile the manual |childdoc.pdf| (this file).
\item
Run \LaTeX{} on |childdoc.ins| to create the definitions file |childdoc.def|
and the sample |cdocsamp.tex| with include files
|cdocsch1.tex|, |cdocsch2.tex|, |cdocspt3.tex|, |cdocspt4.tex|,
|cdocsdrf.tex|, |cdocsfn1.tex|, |cdocsfn2.tex|.
Then copy the file |childdoc.def| to an appropriate directory of your \LaTeX{}
distribution, e.g.\ \textit{texmf-root}|/tex/latex/childdoc|.
\end{itemize}

%%%%%%%%%%%%%%%%%%%%%%%%%%%%%%%%%%%%%%%%%%%%%%%%%%%%%%%%%%%%%%%%%%%%%%%%%%%%%%%%
\subsection{Related CTAN Packages}

There are several other packages which offer a similar functionality:
%
\begin{itemize}
\item
The packages
\href{http://ctan.org/pkg/docmute}{\textsf{docmute}},
\href{http://ctan.org/pkg/includex}{\textsf{includex}} and
\href{http://ctan.org/pkg/standalone}{\textsf{standalone}}
provide commands to include only the document body of
a child file thus allowing both files to be compiled individually.
\item
The packages \href{http://ctan.org/pkg/subdocs}{\textsf{subdocs}}
and \href{http://ctan.org/pkg/subfiles}{\textsf{subfiles}}
provide structures in which the main and child documents can be
encapsulated and allowing them to be compiled individually.
The inclusion mechanism is different from the conventional |\include|.
\item
The package \href{http://ctan.org/pkg/combine}{\textsf{combine}}
is an elaborate solution to combine several documents into one.
\end{itemize}
%
See also the CTAN topic \href{http://ctan.org/topic/subdocs}{\textsf{subdocs}}
for further related packages.
The present package differs from the above solutions in that
a document structure constructed with the conventional |\include| mechanism
just needs two extra commands at the top of every file
such that all constituent files can be compiled individually.

%%%%%%%%%%%%%%%%%%%%%%%%%%%%%%%%%%%%%%%%%%%%%%%%%%%%%%%%%%%%%%%%%%%%%%%%%%%%%%%%
%\subsection{Feature Suggestions}
%
%The following is a list of features which may be useful for future
%versions of this package:
%%
%\begin{itemize}
%\item
%\ldots
%\end{itemize}

%%%%%%%%%%%%%%%%%%%%%%%%%%%%%%%%%%%%%%%%%%%%%%%%%%%%%%%%%%%%%%%%%%%%%%%%%%%%%%%%
\subsection{Revision History}

%%%%%%%%%%%%%%%%%%%%%%%%%%%%%%%%%%%%%%%%
\paragraph{v2.0:} 2018/12/30

\begin{itemize}
\item
immediate forward processing
\item
added |\childdocby| mechanism
\item
manual restructured
\end{itemize}

%%%%%%%%%%%%%%%%%%%%%%%%%%%%%%%%%%%%%%%%
\paragraph{v1.6:} 2018/01/17

\begin{itemize}
\item
application for development of include files
\item
corrections to manual
\end{itemize}

%%%%%%%%%%%%%%%%%%%%%%%%%%%%%%%%%%%%%%%%
\paragraph{v1.5:} 2017/05/21

\begin{itemize}
\item
more complete structuring introduced
\item
|\childdocof| introduced
\item
|\childdoc| renamed to |\childdocmain|
\item
|\childredirect| renamed to |\childdocforward| and |\childdocforwardprefix|
and functionality expanded
\end{itemize}

%%%%%%%%%%%%%%%%%%%%%%%%%%%%%%%%%%%%%%%%
\paragraph{v1.0:} 2017/04/27

\begin{itemize}
\item
manual and install package
\item
first version published on CTAN
\end{itemize}

%%%%%%%%%%%%%%%%%%%%%%%%%%%%%%%%%%%%%%%%
\paragraph{v0.6:} 2017/04/26

\begin{itemize}
\item
redirection mechanism added
\end{itemize}

%%%%%%%%%%%%%%%%%%%%%%%%%%%%%%%%%%%%%%%%
\paragraph{v0.5:} 2017/04/26

\begin{itemize}
\item
functionality in definition file
\end{itemize}


%%%%%%%%%%%%%%%%%%%%%%%%%%%%%%%%%%%%%%%%%%%%%%%%%%%%%%%%%%%%%%%%%%%%%%%%%%%%%%%%
%%%%%%%%%%%%%%%%%%%%%%%%%%%%%%%%%%%%%%%%%%%%%%%%%%%%%%%%%%%%%%%%%%%%%%%%%%%%%%%%
%%%%%%%%%%%%%%%%%%%%%%%%%%%%%%%%%%%%%%%%%%%%%%%%%%%%%%%%%%%%%%%%%%%%%%%%%%%%%%%%
\appendix

\settowidth\MacroIndent{\rmfamily\scriptsize 000\ }

 \DocInput{childdoc.dtx}

\end{document}
%</driver>
% \fi
%
% %%%%%%%%%%%%%%%%%%%%%%%%%%%%%%%%%%%%%%%%%%%%%%%%%%%%%%%%%%%%%%%%%%%%%%%%%%%%%%
% %%%%%%%%%%%%%%%%%%%%%%%%%%%%%%%%%%%%%%%%%%%%%%%%%%%%%%%%%%%%%%%%%%%%%%%%%%%%%%
% \section{Sample}
%\iffalse
%<*samplemain>
%\fi
%
% The following presents a sample document
% with two chapters, two parts, a title page,
% a compile flag as well as three forwarding files to set the flag.
% It consists of eight |.tex| files:
% \begin{center}
% \begin{tabular}{ll}
% |cdocsamp.tex|&main file\\
% |cdocsch1.tex|&include file for chapter 1\\
% |cdocsch2.tex|&include file for chapter 2\\
% |cdocspt3.tex|&include file for part 3\\
% |cdocspt4.tex|&include file for part 4\\
% |cdocsdrf.tex|&forwarding file for main file in draft mode\\
% |cdocsfi1.tex|&forwarding file for final version of chapter 1\\
% |cdocsfi2.tex|&forwarding file for final version of chapter 2\\
% \end{tabular}
% \end{center}
% Each of the eight files can be compiled directly by the \LaTeX{} compiler.
%
% %%%%%%%%%%%%%%%%%%%%%%%%%%%%%%%%%%%%%%
% \paragraph{Main File.}
%
% The main file is called |cdocsamp.tex|.
%
% Load the \textsf{childdoc} definitions and
% declare the filename for the main document:
%    \begin{macrocode}
\input{childdoc.def}
\childdocmain{}
%    \end{macrocode}

% Optional override for |\version| flag:
%    \begin{macrocode}
%%\ifchilddoc\else\providecommand{\version}{draft}\fi
%    \end{macrocode}

% Define the default values for the |\version| flag
% (|final| for the main file and |draft| for childs):
%    \begin{macrocode}
\ifchilddoc
\providecommand{\version}{draft}
\else
\providecommand{\version}{final}
\fi
%    \end{macrocode}

% Load the standard document class:
%    \begin{macrocode}
\documentclass[12pt]{article}
%    \end{macrocode}

% Start the document body:
%    \begin{macrocode}
\begin{document}
%    \end{macrocode}

% Declare a title page.
% Print title, part of document being processed and version flag:
%    \begin{macrocode}
\addtocounter{page}{-1}
\begin{center}
{\LARGE\bfseries{}childdoc example\par}
\vspace{1cm}
\ifchilddoc
\ifchilddocmanual part\else chapter\fi:
`\childdocname' of `\childdocjob'\par
\else
main document: `\childdocjob'\par
\fi
version: \version\par
\end{center}
\newpage
%    \end{macrocode}

% Manually include selected file,
% otherwise process as usual:
%    \begin{macrocode}
\ifchilddocmanual
\section*{part `\childdocname'}
\input{\childdocname}
\else
%    \end{macrocode}

% Include the two chapters:
%    \begin{macrocode}
\include{cdocsch1}
\include{cdocsch2}
%    \end{macrocode}

% Include the two parts unless only chapters should be displayed:
%    \begin{macrocode}
\ifchilddoc\else
\section{part three}
\input{cdocspt3}
\section{part four}
\input{cdocspt4}
\fi
%    \end{macrocode}

% Process as usual until here:
%    \begin{macrocode}
\fi
%    \end{macrocode}

% End of document body:
%    \begin{macrocode}
\end{document}
%    \end{macrocode}
%\iffalse
%</samplemain>
%\fi
%
% %%%%%%%%%%%%%%%%%%%%%%%%%%%%%%%%%%%%%%
% \paragraph{Chapter Include Files.}
%
% The include files are called |cdocsch1.tex| and |cdocsch2.tex|.
%
%\iffalse
%<*samplechap1|samplechap2>
%\fi

% Optional override for |\version| flag:
%    \begin{macrocode}
%%\providecommand{\version}{final}
%    \end{macrocode}

% Include the main document:
%    \begin{macrocode}
\input{childdoc.def}
\childdocof{cdocsamp}
%    \end{macrocode}

%\iffalse
%</samplechap1|samplechap2>
%\fi
%
%\iffalse
%<*samplechap1>
%\fi
% Some text for chapter 1:
%    \begin{macrocode}
\section{one}
some text in chapter one
%    \end{macrocode}

%\iffalse
%</samplechap1>
%\fi
% Some text for chapter 2:
%\iffalse
%<*samplechap2>
%\fi
%    \begin{macrocode}
\section{two}
more text in chapter two
%    \end{macrocode}

%\iffalse
%</samplechap2>
%\fi
%
% %%%%%%%%%%%%%%%%%%%%%%%%%%%%%%%%%%%%%%
% \paragraph{Part Include Files.}
%
% The include files are called |cdocspt3.tex| and |cdocspt4.tex|.
%
%\iffalse
%<*samplepart3|samplepart4>
%\fi

% Optional override for |\version| flag:
%    \begin{macrocode}
%%\providecommand{\version}{final}
%    \end{macrocode}

% Include the main document:
%    \begin{macrocode}
\input{childdoc.def}
\childdocby{cdocsamp}
%    \end{macrocode}

%\iffalse
%</samplepart3|samplepart4>
%\fi
%
%\iffalse
%<*samplepart3>
%\fi
% Some text for part 3:
%    \begin{macrocode}
some text in part three
%    \end{macrocode}

%\iffalse
%</samplepart3>
%\fi
% Some text for part 4:
%\iffalse
%<*samplepart4>
%\fi
%    \begin{macrocode}
more text in part four
%    \end{macrocode}

%\iffalse
%</samplepart4>
%\fi
%
% %%%%%%%%%%%%%%%%%%%%%%%%%%%%%%%%%%%%%%
% \paragraph{Forwarding for a Complete Draft.}
%
% The following forwarding file |cdocsdrf.tex|
% compiles the main document in draft mode:
%\iffalse
%<*sampledraft>
%\fi
%    \begin{macrocode}
\def\version{draft}
\input{childdoc.def}
\childdocforward{cdocsamp}
%    \end{macrocode}

%\iffalse
%</sampledraft>
%\fi
%
% %%%%%%%%%%%%%%%%%%%%%%%%%%%%%%%%%%%%%%
% \paragraph{Forwarding for Final Version of the Chapters.}
%
% The following forwarding files |cdocsfn1.tex| and |cdocsfn2.tex|
% (with identical content)
% compile the final versions of the child documents
% |cdocsch1.tex| and |cdocsch2.tex|, respectively:
%\iffalse
%<*samplefinal>
%\fi
%    \begin{macrocode}
\def\version{final}
\input{childdoc.def}
\childdocforwardprefix[cdocsamp]{cdocsfn}{cdocsch}
%    \end{macrocode}

%\iffalse
%</samplefinal>
%\fi
%
% %%%%%%%%%%%%%%%%%%%%%%%%%%%%%%%%%%%%%%
% \paragraph{Command Line Processing.}
%
% The following three command lines generate the output files
% |cdocscld|, |cdocscl1| and |cdocscl2|
% which should be identical to
% |cdocsdrf|, |cdocsch1| and |cdocsfn2|, respectively:
% \begin{center}
% \begin{tabular}{l}
% |latex -jobname cdocscld \|\\
% |  "\def\version{draft}\input{childdoc.def}\childdocforward{cdocsamp}"|\\
% |latex -jobname cdocscl1 \|\\
% |  "\input{childdoc.def}\childdocforward[cdocsamp]{cdocsch1}"|\\
% |latex -jobname cdocscl2 \|\\
% |  "\def\version{final}\input{childdoc.def}\childdocforward{cdocsch2}"|
% \end{tabular}
% \end{center}
% Note that the trailing backslash on each first line
% merely continues the input to the second line
% (for convenient cut ant paste).
% Furthermore, the command |latex| can be replaced by any
% of its alternative versions such as |pdflatex|.
%
% %%%%%%%%%%%%%%%%%%%%%%%%%%%%%%%%%%%%%%%%%%%%%%%%%%%%%%%%%%%%%%%%%%%%%%%%%%%%%%
% %%%%%%%%%%%%%%%%%%%%%%%%%%%%%%%%%%%%%%%%%%%%%%%%%%%%%%%%%%%%%%%%%%%%%%%%%%%%%%
% \section{Implementation}
%\iffalse
%<*package>
%\fi
%
% This section describes the definitions file |childdoc.def|.

% The definitions cannot be loaded using |\usepackage| or |\RequirePackage|
% which has a mechanism to prevent loading a style file more than once.
% When loading the definitions by means of |\input|
% multiple instances have to be prevented manually:
%\iffalse
%This code needs to be before the `\ProvidesFile' directive
%which is defined at the beginning of this file.
%Therefore it is also placed there and commented out here.
%</package>
%<*discard>
%\fi
%    \begin{macrocode}
\ifdefined\childdocmain\endinput\fi
%    \end{macrocode}
%\iffalse
%</discard>
%<*package>
%\fi
%
% \macro{\ifchilddoc}
% \macro{\ifchilddocmanual}
% The conditional |\ifchilddoc| tells whether a
% child (true) or main (false) document is being compiled.
% The conditional |\ifchilddocmanual| tells whether
% the |\includeonly| mechanism is used (false) or
% the selection of child files must be performed manually (true).
% The definitions initialise to false:
%    \begin{macrocode}
\newif\ifchilddoc
\newif\ifchilddocmanual
%    \end{macrocode}

% \macro{\childdocname}
% \macro{\childdocjob}
% The macro |\childdocname| stores the name of the main document
% to be compiled. The macro |\childdocjob| stores the name of
% the document on which the \LaTeX{} compiler was originally invoked.
% The content of |\jobname| cannot be compared
% to filenames specified in the source due to different catcodes.
% The following code rescans |\jobname|, stores the result
% in |\childdocname| and saves a copy in |\childdocjob|:
%    \begin{macrocode}
\edef\childdocname{\scantokens\expandafter{\jobname\noexpand}}
\let\childdocjob\childdocname
%    \end{macrocode}

% \macro{\childdocdisable}
% The macro |\childdocdisable| prevents the main file
% from being processed more than once.
% At this stage, the main document command |\childdocmain|
% is assumed to be called once again where it should do nothing.
% Any subsequent call to it should prevent
% a secondary processing of the main document
% It overwrites the forwarding commands
% |\childdocof| and |\childdocforward|
% with empty macros to prevent further inclusions of the main document:
%    \begin{macrocode}
\newcommand{\childdocdisable}
{
  \renewcommand{\childdocmain}[1]{\renewcommand{\childdocmain}[1]{\endinput}}
  \renewcommand{\childdocof}[1]{}
  \renewcommand{\childdocby}[2][]{}
  \renewcommand{\childdocforward}[2][]{}
  \renewcommand{\childdocdisable}{}
}
%    \end{macrocode}

% \macro{\childdocmain}
% The macro |\childdocmain| is to be called at the top of the main file
% with nothing or the main filename (without extension) as argument.
% First, it breaks loops.
% If the argument is not empty and does not match |\childdocname|
% (which is set by the first inclusion of |childdoc.def|),
% |\ifchilddoc| is set to true, |\includeonly| is applied to the child file
% and |\jobname| is set to the main file
% (for proper handling of |.aux| files):
%    \begin{macrocode}
\newcommand{\childdocmain}[1]
{
  \childdocdisable\childdocmain{}
  \if?#1?\else
    \begingroup
      \def\childdoctmp{#1}
      \ifx\childdoctmp\childdocname
        \def\childdoctmp{}
      \else
        \def\childdoctmp
        {
          \childdoctrue
          \includeonly{\childdocname}
          \def\childdocjob{#1}
          \def\jobname{#1}
        }
      \fi
      \expandafter
    \endgroup
    \childdoctmp
  \fi
}
%    \end{macrocode}

% \macro{\childdocof}
% The command |\childdocof| redirects
% compilation to the main file |#1|.
%    \begin{macrocode}
\newcommand{\childdocof}[1]
{
  \childdocdisable
  \childdoctrue
  \includeonly{\childdocname}
  \def\jobname{#1}
  \def\childdocjob{#1}
  \input{#1}
}
%    \end{macrocode}

% \macro{\childdocby}
% The command |\childdocby| ....
%    \begin{macrocode}
\newcommand{\childdocby}[2][]
{
  \childdocdisable
  \childdoctrue
  \childdocmanualtrue
  \if?#1?\else
    \def\jobname{#2}
  \fi
  \def\childdocjob{#2}
  \input{#2}
  \endinput
}
%    \end{macrocode}

% \macro{\childdocforward}
% The command |\childdocforward| redirects
% compilation to the main file or
% (if the optional argument is given) a child file.
% Parameters are set as if the main file
% or a child file starting with |\childdocof| was compiled.
% Then compilation is handed over to the main file:
%    \begin{macrocode}
\newcommand{\childdocforward}[2][]
{
  \begingroup
    \if?#1?
      \def\childdoctmp
      {
        \def\childdocname{#2}
        \def\childdocjob{#2}
        \def\jobname{#2}
        \input{#2}
        \endinput
      }
    \else
      \def\childdoctmp
      {
        \childdocdisable
        \def\childdocname{#2}
        \childdoctrue
        \includeonly{#2}
        \def\childdocjob{#1}
        \def\jobname{#1}
        \input{#1}
        \endinput
      }
    \fi
    \expandafter
  \endgroup
  \childdoctmp
}
%    \end{macrocode}

% \macro{\childdocforwardprefix}
% The command |\childdocforwardprefix| redirects
% compilation to the main or a child file by means of a pattern.
% The prefix |#1| in the current filename is replaced by |#2|
% and the suffix of the current filename is kept
% (it is assumed that the filename does not contain the substring `|~~~|'
% which is used as a delimiter).
% Compilation is handed over to the new file by |\childdocforward|:
%    \begin{macrocode}
\newcommand{\childdocforwardprefix}[3][]
{
  \begingroup
    \def\childdocextract #2##1~~~{\def\childdoctmp{\childdocforward[#1]{#3##1}}}
    \expandafter\childdocextract\childdocname~~~
    \expandafter
  \endgroup
  \childdoctmp
}
%    \end{macrocode}

% \macro{\childdoc}
% The deprecated macro |\childdoc| is a legacy version of |\childdocmain|:
%    \begin{macrocode}
\newcommand{\childdoc}{\childdocmain}
%    \end{macrocode}

% \macro{\childdocredirect}
% The deprecated macro |\childdocredirect| is a legacy version
% of |\childdocforward| and |\childdocforwardprefix|:
%    \begin{macrocode}
\newcommand{\childdocredirect}[2][]
{
  \begingroup
    \if?#1?
      \def\childdoctmp{\childdocforward{#2}}
    \else
      \def\childdoctmp{\childdocforwardprefix{#1}{#2}}
    \fi
    \expandafter
  \endgroup
  \childdoctmp
}
%    \end{macrocode}

%\iffalse
%</package>
%\fi
%
\endinput
|\\
|\childdocby{|\textit{main}|}|\\
\end{tabular}
\end{center}
%
The directive |\childdocby| is similar to |\childdocof|
described in \secref{sec:include},
but the subsequent selection of content must be done manually.
To that end, both |\ifchilddoc| and |\ifchilddocmanual|
will be true upon processing of a part,
and the name of the part is stored in |\childdocname|.
Note that |\jobname| will be set to the filename of the current part
so that each part receives an individual |.aux| file
that does not interfere with the |.aux| file(s) of the main document.
This behaviour can be altered by the alternative form
|\childdocby[*]{|\textit{main}|}| (with a non-empty optional argument)
which uses the |.aux| file of the main document
by setting |\jobname| to \textit{main}.

%%%%%%%%%%%%%%%%%%%%%%%%%%%%%%%%%%%%%%%%%%%%%%%%%%%%%%%%%%%%%%%%%%%%%%%%%%%%%%%%
\subsection{Driver Development}
\label{sec:driver}

The \textsf{childdoc} mechanism can also be use for the development
of definition files such as \LaTeX{} styles or classes.
This case differs from the above setup with multiple parts
included by |\include| in that no |\includeonly| should be invoked.
This can be achieved by starting the include file
(before |\ProvidesPackage|) with:
%
\begin{center}
\begin{tabular}{l}
|% \iffalse
%
% childdoc.dtx Copyright (C) 2017-2018 Niklas Beisert
%
% This work may be distributed and/or modified under the
% conditions of the LaTeX Project Public License, either version 1.3
% of this license or (at your option) any later version.
% The latest version of this license is in
%   http://www.latex-project.org/lppl.txt
% and version 1.3 or later is part of all distributions of LaTeX
% version 2005/12/01 or later.
%
% This work has the LPPL maintenance status `maintained'.
%
% The Current Maintainer of this work is Niklas Beisert.
%
% This work consists of the files childdoc.dtx and childdoc.ins
% and the derived files childdoc.def and cdocsamp.tex with
% cdocsch1.tex, cdocsch2.tex, cdocsdrf.tex, cdocsfn1.tex, cdocsfn2.tex.
%
%<package>\ifdefined\childdocmain\endinput\fi
%<package>\ProvidesFile{childdoc.def}[2018/12/30 v2.0 child document driver]
%<samplemain>\ProvidesFile{cdocsamp.tex}[2018/12/30 v2.0 sample for childdoc]
%<*driver>
%\ProvidesFile{childdoc.drv}[2018/12/30 v2.0 childdoc reference manual file]
\PassOptionsToClass{10pt,a4paper}{article}
\documentclass{ltxdoc}

\usepackage[margin=35mm]{geometry}
\usepackage{hyperref}
\usepackage{hyperxmp}
\usepackage[usenames]{color}

\hypersetup{colorlinks=true}
\hypersetup{pdfstartview=FitH}
\hypersetup{pdfpagemode=UseNone}
\hypersetup{pdfsource={}}
\hypersetup{pdflang={en-UK}}
\hypersetup{pdfcopyright={Copyright 2017-2018 Niklas Beisert.
  This work may be distributed and/or modified under the
  conditions of the LaTeX Project Public License, either version 1.3
  of this license or (at your option) any later version.}}
\hypersetup{pdflicenseurl={http://www.latex-project.org/lppl.txt}}
\hypersetup{pdfcontactaddress={ETH Zurich, ITP, HIT K,
  Wolfgang-Pauli-Strasse 27}}
\hypersetup{pdfcontactpostcode={8093}}
\hypersetup{pdfcontactcity={Zurich}}
\hypersetup{pdfcontactcountry={Switzerland}}
\hypersetup{pdfcontactemail={nbeisert@itp.phys.ethz.ch}}
\hypersetup{pdfcontacturl={http://people.phys.ethz.ch/\xmptilde nbeisert/}}

\newcommand{\secref}[1]{\hyperref[#1]{section \ref*{#1}}}

\parskip1ex
\parindent0pt
\let\olditemize\itemize
\def\itemize{\olditemize\parskip0pt}

\begin{document}

\title{The \textsf{childdoc} Package}
\hypersetup{pdftitle={The childdoc Package}}
\author{Niklas Beisert\\[2ex]
  Institut f\"ur Theoretische Physik\\
  Eidgen\"ossische Technische Hochschule Z\"urich\\
  Wolfgang-Pauli-Strasse 27, 8093 Z\"urich, Switzerland\\[1ex]
  \href{mailto:nbeisert@itp.phys.ethz.ch}
  {\texttt{nbeisert@itp.phys.ethz.ch}}}
\hypersetup{pdfauthor={Niklas Beisert}}
\hypersetup{pdfsubject={Manual for the LaTeX2e Package childdoc}}
\date{30 December 2018, \textsf{v2.0}}
\maketitle

\begin{abstract}\noindent
\textsf{childdoc} is a \LaTeXe{} package
that enables the direct compilation
of document sections included by |\include|
to individual files.
\end{abstract}

\begingroup
\parskip0ex
\tableofcontents
\endgroup

%%%%%%%%%%%%%%%%%%%%%%%%%%%%%%%%%%%%%%%%%%%%%%%%%%%%%%%%%%%%%%%%%%%%%%%%%%%%%%%%
%%%%%%%%%%%%%%%%%%%%%%%%%%%%%%%%%%%%%%%%%%%%%%%%%%%%%%%%%%%%%%%%%%%%%%%%%%%%%%%%
\section{Introduction}

\LaTeX{} provides a mechanism to structure a large document (such as a book)
into a main file and several child files (containing the chapters)
using the |\include| command.
This mechanism is beneficial for documents
which span hundreds of pages in order to
make the source file(s) more manageable.
Moreover, compilation can be restricted to
selected child files by means of the |\includeonly| command.
The latter feature can be used to reduce the compilation time while editing
(this was significantly more useful in the earlier days of \LaTeX{})
or to generate a smaller document which is easier to navigate.
Another application of |\includeonly| is to generate
documents consisting of selected parts of the complete document.

However, there are a few drawbacks of the plain |\include| mechanism:
\begin{itemize}
\item
The child files cannot be compiled on their own,
they can only be compiled via the main file.
A naive editing environment
(such as a text editor with an option
to have the current file processed by \LaTeX)
may require one to switch to the main file before compiling;
attempting to compile the child file produces errors.
\item
The main file must be modified (each time)
to adjust the |\includeonly| command
to the present needs. This easily leaves the main file in a messy state.
\item
The generated document will always carry the filename
of the main document. This is inconvenient if
several child files are to be compiled and
to be kept for distribution.
\end{itemize}

The present package provides a simple interface
to make child files individually compilable by \LaTeX{}.
Compiling a child file then has the same effect as compiling
the main file with an |\includeonly| command
to select the appropriate child.
Moreover the generated document will carry the name of the child
rather than the main file.
This resolves all three above issues.

This feature is meant to make the editing of books,
thesis documents and lecture notes somewhat more convenient.
However, the package can also be used efficiently for
composing a series of documents (such as exercise sheets)
which are typically distributed individually.
It then assists the author in generating the individual documents
(potentially in different versions)
as well as a document containing the collected series.
Another application is in developing style files
or other kinds of included material
where compilation of the style file could redirect
to a sample or test file.

%%%%%%%%%%%%%%%%%%%%%%%%%%%%%%%%%%%%%%%%%%%%%%%%%%%%%%%%%%%%%%%%%%%%%%%%%%%%%%%%
%%%%%%%%%%%%%%%%%%%%%%%%%%%%%%%%%%%%%%%%%%%%%%%%%%%%%%%%%%%%%%%%%%%%%%%%%%%%%%%%
\section{Usage}

First of all, the package \textsf{childdoc} is \emph{not} a standard
\LaTeXe{} |.sty| style file! Therefore it needs to be invoked in
a non-standard way.

%%%%%%%%%%%%%%%%%%%%%%%%%%%%%%%%%%%%%%%%%%%%%%%%%%%%%%%%%%%%%%%%%%%%%%%%%%%%%%%%
\subsection{Included Files}
\label{sec:include}

%%%%%%%%%%%%%%%%%%%%%%%%%%%%%%%%%%%%%%%%
\DescribeMacro{\childdocmain}
To use the package, add the commands
\begin{center}
\begin{tabular}{l}
|\input{childdoc.def}|\\
|\childdocmain{}|\\
\end{tabular}
\end{center}
at the very top of the main \LaTeX{} file,
in particular \emph{before} the |\documentclass| statement!
The argument of |\childdocmain| should be left empty
(but it must be present).

%%%%%%%%%%%%%%%%%%%%%%%%%%%%%%%%%%%%%%%%
\DescribeMacro{\childdocof}
Furthermore, add the commands
\begin{center}
\begin{tabular}{l}
|\input{childdoc.def}|\\
|\childdocof{|\textit{main}|}|\\
\end{tabular}
\end{center}
at the top of every child file \textit{child}
which is included by |\include{|\textit{child}|}|
from within the main file
(or at least for those files to be compiled individually).
The argument \textit{main} must be the filename of the main file.

There are a couple of
considerations in setting up the main and child documents:

%%%%%%%%%%%%%%%%%%%%%%%%%%%%%%%%%%%%%%%%
\paragraph{Restrictions.}

Please note the following restrictions:
\begin{itemize}
\item
|\childdocmain| must be called with one argument \textit{main}
to ensure compatibility with earlier version of the package.
It must either be empty (|\childdocmain{}|)
or precisely match the filename of the main file in which it is specified.
See \secref{sec:detection} for further information.
\item
The filename \textit{main} must be specified without the |.tex| extension.
\item
The filename \textit{main} is case sensitive
(even in case-insensitive file systems)
due to internal string comparison.
\item
The argument \textit{main} should be fully expanded, it cannot be a macro.
\item
Subdirectories and special characters should be avoided in filenames.
\item
The command |\childdocmain{|\textit{main}|}| must be followed by a whitespace.
It should not be followed immediately by another command
or by a comment mark `|%|'.
This is because the \TeX{} parser reads the token immediately following
the argument of |\childdocmain| and puts it
at the beginning of every child section;
however, a white\-space is ignored.
\end{itemize}

%%%%%%%%%%%%%%%%%%%%%%%%%%%%%%%%%%%%%%%%
\paragraph{Content of Main File.}

It is advisable to place all content in the child files included by |\include|.
Any output contained in the main file will appear in all child documents
unless suppressed manually;
it cannot be suppressed automatically by the |\includeonly| directive
and thus should normally be avoided.
A method to include some content in the main file
by means of conditional processing is described in \secref{sec:conditional}.

%%%%%%%%%%%%%%%%%%%%%%%%%%%%%%%%%%%%%%%%
\paragraph{Page Numbering.}

When only a part of the document is compiled,
the appropriate numbering of pages
(as well as other status parameters)
is determined from the |.aux| files.
The latter contain information from previous passes.
However this information needs to propagate through
all intermediate child documents.
Therefore the page numbering in child documents may well
be inconsistent until the complete document is compiled at least once.

A useful (if unconventional) way to always ensure a consistent
page numbering is to restart the numbering in each child document
and denote the pages by `\textit{child}|.|\textit{page}'
where \textit{child} represents the chapter/section number of the child file.
This can be achieved by the command
|\numberwithin{page}{|\textit{child}|}|
of the \textsf{amsmath} package
where \textit{child} can be |chapter| or |section|
depending on the chosen structuring.
Alternatively, one can modify the macro |\thepage| appropriately
and reset the counter |page| at the start of each child file.

%%%%%%%%%%%%%%%%%%%%%%%%%%%%%%%%%%%%%%%%%%%%%%%%%%%%%%%%%%%%%%%%%%%%%%%%%%%%%%%%
\subsection{Conditional Processing}
\label{sec:conditional}

The package provides a mechanism to compile different versions
of a document. To customise the versions further some conditional processing
can come in handy to distinguish which version is being compiled.
The package provides two macros to describe the compilation context:

%%%%%%%%%%%%%%%%%%%%%%%%%%%%%%%%%%%%%%%%
\DescribeMacro{\ifchilddoc}
The conditional |\ifchilddoc| distinguishes between the compilation of
child documents and the main document:
%
\begin{center}
|\ifchilddoc |\textit{child-code}| |[|\||else |\textit{main-code}]| \||fi|
\end{center}

%%%%%%%%%%%%%%%%%%%%%%%%%%%%%%%%%%%%%%%%
\DescribeMacro{\childdocname}
\DescribeMacro{\childdocjob}
The macro |\childdocname| contains the filename (without extension)
of the main or child file being processed.
Note that |\childdocjob| will always contain the name of the main file.

%%%%%%%%%%%%%%%%%%%%%%%%%%%%%%%%%%%%%%%%
\paragraph{Title Page.}

Conditional processing can be used to include a title or banner page
in the main document when proper precautions are taken.
Importantly, the code in the main file should ensure that the page counter
(as well as other status parameters which are stored in the |.aux| files)
takes the same value after the conditional processing.
Otherwise the page numbers may take divergent values
depending on which part is compiled.

For example, a title page could be declared by:
%
\begin{center}
\begin{tabular}{l}
|\ifchilddoc\||else|\\
|\addtocounter{page}{-1}|\\
\textit{code for title page}\\
|\newpage|\\
|\||fi|
\end{tabular}
\end{center}
%
A banner page for the child documents can be generated by:
%
\begin{center}
\begin{tabular}{l}
|\ifchilddoc|\\
|\addtocounter{page}{-1}|\\
\textit{code for banner page}\\
|\newpage|\\
|\||fi|
\end{tabular}
\end{center}
%
Here one could write a message such as:
\begin{center}
|This is the part \childdocname{} of \childdocjob{}.|
\end{center}

%%%%%%%%%%%%%%%%%%%%%%%%%%%%%%%%%%%%%%%%%%%%%%%%%%%%%%%%%%%%%%%%%%%%%%%%%%%%%%%%
\subsection{Flags}
\label{sec:flags}

The package makes it easy to generate different versions
of the main or child documents.
To this end compilation flags can be defined
and assigned different default values.
They will be particularly useful in conjunction
with the forwarding mechanism described in \secref{sec:forward}.

For example, it may be useful to have a flag |\version|
which can be set to |draft| or |final|.
The document source will contain some conditional code
depending on the value of |\version|.
Suppose further, the flag should default to |final| for the main file
and to |draft| for child files
which is a natural assignment for editing the document.
This is achieved by placing the following code
in the preamble of the main document
(below the |\childdocmain| directive):
%
\begin{center}
\begin{tabular}{l}
|\ifchilddoc|\\
|\providecommand{\version}{draft}|\\
|\||else|\\
|\providecommand{\version}{final}|\\
|\||fi|
\end{tabular}
\end{center}
%
The definition by |\providecommand| makes sure
that previous definitions are not overwritten.
Further statements |\providecommand{\version}{...}|
can thus be added before the above code to override it.

For the main file, one might add a line
(between |\childdocmain| and the above block)
%
\begin{center}
|%\ifchilddoc\||else\providecommand{\version}{draft}\||fi|
\end{center}
%
which can be uncommented to produce a draft version.
Likewise one can add a line to the very top of a child file
(above the |\childdocof{|\textit{main}|}| directive)
%
\begin{center}
|%\providecommand{\version}{final}|
\end{center}
%
which can be uncommented to produce the final version of this child document.

%%%%%%%%%%%%%%%%%%%%%%%%%%%%%%%%%%%%%%%%%%%%%%%%%%%%%%%%%%%%%%%%%%%%%%%%%%%%%%%%
\subsection{Forwarding}
\label{sec:forward}

Different versions of the main or child documents
using compilation flags as described in \secref{sec:flags}
can be (permanently) stored in different files
for convenient compilation, viewing and distribution.
To this end, the package defines a command
to pass on compilation to a different file:

%%%%%%%%%%%%%%%%%%%%%%%%%%%%%%%%%%%%%%%%
\DescribeMacro{\childdocforward}
The command |\childdocforward| redirects processing to
another source file:
%
\begin{center}
\begin{tabular}{l}
|\input{childdoc.def}|\\
|\childdocforward[|\textit{main}|]{|\textit{dest}|}|\\
\end{tabular}
\end{center}
%
The argument \textit{dest} is the destination file
(without extension).
It should be the main file or one of the child files.
Note that further \textsf{childdoc} directives
such as |\childdocof| and |\childdocforward|
in the indicated file will be processed in this form.
The optional argument \textit{main}
passes on directly to the main file \textit{main}
while pretending to compile the child \textit{dest}.
This form behaves as if \textit{dest}
issues |\childdocof{|\textit{main}|}| right away,
and no further \textsf{childdoc} directives will be processed.

%%%%%%%%%%%%%%%%%%%%%%%%%%%%%%%%%%%%%%%%
\DescribeMacro{\...prefix}
In the alternative form |\childdocforwardprefix|,
%
\begin{center}
\begin{tabular}{l}
|\input{childdoc.def}|\\
|\childdocforwardprefix[|\textit{main}|]{|\textit{prefix}|}{|\textit{dest}|}|
\end{tabular}
\end{center}
%
the destination file is determined by a pattern
depending on the current file:
To make this work, the current file must be called
`{\textit{prefix}\hspace{0.2em}\textit{suffix}}'
with \textit{prefix} matching precisely the argument.
Processing is then passed on to the file
`{\textit{dest}\hspace{0.2em}\textit{suffix}}'.
Surely, the same effect is achieved by
directly specifying the
argument `{\textit{dest}\hspace{0.2em}\textit{suffix}}'
in the first form.
However, that requires to set up a different file
for each child. With the alternative form of the command
all these files can have exactly the same content
which simplifies setting them up and maintaining them.

For example, the following file |draft.tex|
with a compilation flag |\version| as described in \secref{sec:flags}
compiles the main document as a draft:
%
\begin{center}
\begin{tabular}{l}
|\def\version{draft}|\\
|\input{childdoc.def}|\\
|\childdocforward{|\textit{main}|}|
\end{tabular}
\end{center}
%
Likewise, the following files |final|\textit{nn}|.tex|
compile the final version of the child document
|child|\textit{nn}|.tex|:
%
\begin{center}
\begin{tabular}{l}
|\def\version{final}|\\
|\input{childdoc.def}|\\
|\childdocforwardprefix{final}{child}|
\end{tabular}
\end{center}
%

Note that when several versions of a main file and/or of each child file
are to be generated, it may be convenient to set up a |Makefile| or
shell script to automatise the process.

%%%%%%%%%%%%%%%%%%%%%%%%%%%%%%%%%%%%%%%%%%%%%%%%%%%%%%%%%%%%%%%%%%%%%%%%%%%%%%%%
\subsection{Command Line Processing}
\label{sec:commandline}

The effect of redirection files can also be achieved by invoking
the \LaTeX{} compiler with a more elaborate command line.
Most conveniently this should be done as part
of a shell script or a |Makefile|.

When using \textsf{childdoc} in the main file, the following
command lines effectively perform a redirection
(note that depending on the shell being used,
backslashes may have to be doubled: `|\|' $\to$ `|\\|'):
%
\begin{center}
|... -jobname "|\textit{target}|" |\\|"|[\textit{flags}]%
|\input{childdoc.def}\childdocforward[|\textit{main}|]{|\textit{dest}|}"|
\end{center}
%
Here \textit{target} is the name of the output file,
\textit{main} is the name of the main file
and \textit{dest} is the name of the main or child file to be processed
(all filenames without extensions).
The optional argument \textit{main} can be omitted
if \textit{main} matches \textit{dest}.
Optionally, compilation \textit{flags} can be defined via |\def| commands.
This command line makes the \TeX{} engine believe
it is compiling the file \textit{target}
whose content is specified as the latter parameter.
The provided code then forwards the processing to
\textit{main} or \textit{dest} as described in \secref{sec:forward}.

%%%%%%%%%%%%%%%%%%%%%%%%%%%%%%%%%%%%%%%%%%%%%%%%%%%%%%%%%%%%%%%%%%%%%%%%%%%%%%%%
\subsection{Include by Input}
\label{sec:input}

Including child documents by |\include| has some restrictions by design.
Most notably, the content of a child document always occupies
its own set of pages; pages cannot be shared between child documents.
Usually, this behaviour makes perfect sense
because each child document contain an essential part of the document.
However, in some situations it may be desirable to compose
a document from a collection of parts
without having mandatory page breaks between then.
For this case, the package
provides a mechanism to include parts
by |\input| which can also be processed individually.
However, by construction this mechanism
requires manual handling of the content to be output.

%%%%%%%%%%%%%%%%%%%%%%%%%%%%%%%%%%%%%%%%
\DescribeMacro{\ifchilddocmanual}
The main file should be prepared as usual, see \secref{sec:include}.
However, the document body must make a distinction
between processing of an individual part and of the main document, e.g.:
%
\begin{center}
\begin{tabular}{l}
|\ifchilddocmanual|\\
|\input{\childdocname}|\\
|\||else|\\
\textit{document body with }|\input{|\textit{part}|}|\\
|\||fi|
\end{tabular}
\end{center}
%
The conditional |\ifchilddocmanual| is true whenever
a part to be included by |\input| is being compiled,
and the name of the part is stored in |\childdocname|.

%%%%%%%%%%%%%%%%%%%%%%%%%%%%%%%%%%%%%%%%
\DescribeMacro{\childdocby}
Each part to be included by |\input| should start with:
%
\begin{center}
\begin{tabular}{l}
|\input{childdoc.def}|\\
|\childdocby{|\textit{main}|}|\\
\end{tabular}
\end{center}
%
The directive |\childdocby| is similar to |\childdocof|
described in \secref{sec:include},
but the subsequent selection of content must be done manually.
To that end, both |\ifchilddoc| and |\ifchilddocmanual|
will be true upon processing of a part,
and the name of the part is stored in |\childdocname|.
Note that |\jobname| will be set to the filename of the current part
so that each part receives an individual |.aux| file
that does not interfere with the |.aux| file(s) of the main document.
This behaviour can be altered by the alternative form
|\childdocby[*]{|\textit{main}|}| (with a non-empty optional argument)
which uses the |.aux| file of the main document
by setting |\jobname| to \textit{main}.

%%%%%%%%%%%%%%%%%%%%%%%%%%%%%%%%%%%%%%%%%%%%%%%%%%%%%%%%%%%%%%%%%%%%%%%%%%%%%%%%
\subsection{Driver Development}
\label{sec:driver}

The \textsf{childdoc} mechanism can also be use for the development
of definition files such as \LaTeX{} styles or classes.
This case differs from the above setup with multiple parts
included by |\include| in that no |\includeonly| should be invoked.
This can be achieved by starting the include file
(before |\ProvidesPackage|) with:
%
\begin{center}
\begin{tabular}{l}
|\input{childdoc.def}|\\
|\childdocforward{|\textit{main}|}|\\
\end{tabular}
\end{center}
%
or alternatively with:
%
\begin{center}
\begin{tabular}{l}
|\input{childdoc.def}|\\
|\childdocby{|\textit{main}|}|\\
\end{tabular}
\end{center}
%
Both forms have slightly different effects as described above.
The main file is prepared as usual, see \secref{sec:include}.

%%%%%%%%%%%%%%%%%%%%%%%%%%%%%%%%%%%%%%%%%%%%%%%%%%%%%%%%%%%%%%%%%%%%%%%%%%%%%%%%
\subsection{Legacy Detection}
\label{sec:detection}

The directive |\childdocmain| in the main file can detect
whether the complete document or merely a child is to be compiled
even without using the directive |\childdocof|.
This method is deprecated because it is less robust
and there is no compelling reason to use it;
it is merely provided for backward compatibility
and it may be removed in future versions.

If the detection mechanism is to be used,
it is mandatory to correctly specify
the filename of the main file as the argument of |\childdocmain|:
%
\begin{center}
\begin{tabular}{l}
|\input{childdoc.def}|\\
|\childdocmain{|\textit{main}|}|\\
\end{tabular}
\end{center}
%
If |\jobname| does not match the argument \textit{main} of |\childdocmain|,
it is assumed that |\jobname| points to the child file to be compiled.
When using |\childdocmain| with the main file specified as argument,
it suffices to start a child file
with just |\input{|\textit{main}|}|
without loading of the package and using |\childdocof|.
If instead all processing is done
with the appropriate \textsf{childdoc} directives,
the argument of \textit{main} of |\childdocmain| can be empty.

An alternative version of the command line processing described
in \secref{sec:commandline} using the detection mechanism reads:
%
\begin{center}
|... -jobname "|\textit{target}|" "|[\textit{flags}]%
[|\def\jobname{|\textit{dest}|}|]|\input{|\textit{main}|}"|
\end{center}

%%%%%%%%%%%%%%%%%%%%%%%%%%%%%%%%%%%%%%%%%%%%%%%%%%%%%%%%%%%%%%%%%%%%%%%%%%%%%%%%
\subsection{Manual Code}
\label{sec:manual}

In case one cannot be certain whether the definitions file |childdoc.def|
is installed on the target \TeX{} distribution
and one prefers not to ship it,
it is conceivable to paste a few relevant commands into the sources.

To that end, drop all statements |\input{childdoc.def}|
and perform the replacements as outlined below.
Instead of |\childdocmain{|\textit{main}|}| add the following code
to the top of the main file:
%
\begin{center}
\begin{tabular}{l}
|\||ifdefined\childdocname\endinput\||fi\newif\ifchilddoc|\\
|\edef\childdocname{\scantokens\expandafter{\jobname\noexpand}}|\\
|\def\childdocmain{|\textit{main}|}\||ifx\childdocmain\childdocname\||else|\\
|\childdoctrue\includeonly{\childdocname}\let\jobname\childdocmain\||fi|\\
\end{tabular}
\end{center}
%
Instead of |\childdocof{|\textit{main}|}| just include the main file
at the top of each child file:
%
\begin{center}
|\input{|\textit{main}|}|
\end{center}
%
A simple redirection |\childdocforward{|\textit{dest}|}| is achieved by:
%
\begin{center}
|\def\jobname{|\textit{dest}|}\input{\jobname}|
\end{center}
%
The redirection with prefix
|\childdocforwardprefix[|\textit{prefix}|]{|\textit{dest}|}|
is accomplished by:
%
\begin{center}
\begin{tabular}{l}
|{\edef\jobname{\scantokens\expandafter{\jobname\noexpand}}|\\
|\def\redirectjob |\textit{prefix}|#1~~~{\gdef\jobname{|\textit{dest}|#1}}|\\
|\expandafter\redirectjob\jobname~~~}\input{\jobname}|
\end{tabular}
\end{center}

In an alternative approach,
child documents can be compiled by a specific command line
without additional code or specific definitions:
%
\begin{center}
|... -jobname "|\textit{target}|" "|[\textit{flags}]%
|\includeonly{|\textit{dest}|}\input{|\textit{main}|}"|
\end{center}
%

%%%%%%%%%%%%%%%%%%%%%%%%%%%%%%%%%%%%%%%%%%%%%%%%%%%%%%%%%%%%%%%%%%%%%%%%%%%%%%%%
%%%%%%%%%%%%%%%%%%%%%%%%%%%%%%%%%%%%%%%%%%%%%%%%%%%%%%%%%%%%%%%%%%%%%%%%%%%%%%%%
\section{Information}

%%%%%%%%%%%%%%%%%%%%%%%%%%%%%%%%%%%%%%%%%%%%%%%%%%%%%%%%%%%%%%%%%%%%%%%%%%%%%%%%
\subsection{Copyright}

Copyright \copyright{} 2017--2018 Niklas Beisert

This work may be distributed and/or modified under the
conditions of the \LaTeX{} Project Public License, either version 1.3
of this license or (at your option) any later version.
The latest version of this license is in
  \url{http://www.latex-project.org/lppl.txt}
and version 1.3 or later is part of all distributions of \LaTeX{}
version 2005/12/01 or later.

This work has the LPPL maintenance status `maintained'.

The Current Maintainer of this work is Niklas Beisert.

This work consists of the files |README.txt|, |childdoc.ins| and |childdoc.dtx|
as well as the derived files |childdoc.def|, |cdocsamp.tex|
with |cdocsch1.tex|, |cdocsch2.tex|, |cdocspt3.tex|, |cdocspt4.tex|,
|cdocsdrf.tex|, |cdocsfn1.tex|, |cdocsfn2.tex|
as well as |childdoc.pdf|.

%%%%%%%%%%%%%%%%%%%%%%%%%%%%%%%%%%%%%%%%%%%%%%%%%%%%%%%%%%%%%%%%%%%%%%%%%%%%%%%%
\subsection{Files and Installation}

The package consists of the files:
%
\begin{center}
\begin{tabular}{ll}
    |README.txt|   & readme file \\
    |childdoc.ins| & installation file \\
    |childdoc.dtx| & source file \\
    |childdoc.def| & definition file \\
    |cdocsamp.tex| & sample main file \\
    |cdocsch1.tex| & sample include file \\
    |cdocsch2.tex| & sample include file \\
    |cdocspt3.tex| & sample part file \\
    |cdocspt4.tex| & sample part file \\
    |cdocsdrf.tex| & sample redirection file \\
    |cdocsfn1.tex| & sample redirection file \\
    |cdocsfn2.tex| & sample redirection file \\
    |childdoc.pdf| & manual
\end{tabular}
\end{center}
%
The distribution consists of the files
|README.txt|, |childdoc.ins| and |childdoc.dtx|.
%
\begin{itemize}
\item
Run (pdf)\LaTeX{} on |childdoc.dtx|
to compile the manual |childdoc.pdf| (this file).
\item
Run \LaTeX{} on |childdoc.ins| to create the definitions file |childdoc.def|
and the sample |cdocsamp.tex| with include files
|cdocsch1.tex|, |cdocsch2.tex|, |cdocspt3.tex|, |cdocspt4.tex|,
|cdocsdrf.tex|, |cdocsfn1.tex|, |cdocsfn2.tex|.
Then copy the file |childdoc.def| to an appropriate directory of your \LaTeX{}
distribution, e.g.\ \textit{texmf-root}|/tex/latex/childdoc|.
\end{itemize}

%%%%%%%%%%%%%%%%%%%%%%%%%%%%%%%%%%%%%%%%%%%%%%%%%%%%%%%%%%%%%%%%%%%%%%%%%%%%%%%%
\subsection{Related CTAN Packages}

There are several other packages which offer a similar functionality:
%
\begin{itemize}
\item
The packages
\href{http://ctan.org/pkg/docmute}{\textsf{docmute}},
\href{http://ctan.org/pkg/includex}{\textsf{includex}} and
\href{http://ctan.org/pkg/standalone}{\textsf{standalone}}
provide commands to include only the document body of
a child file thus allowing both files to be compiled individually.
\item
The packages \href{http://ctan.org/pkg/subdocs}{\textsf{subdocs}}
and \href{http://ctan.org/pkg/subfiles}{\textsf{subfiles}}
provide structures in which the main and child documents can be
encapsulated and allowing them to be compiled individually.
The inclusion mechanism is different from the conventional |\include|.
\item
The package \href{http://ctan.org/pkg/combine}{\textsf{combine}}
is an elaborate solution to combine several documents into one.
\end{itemize}
%
See also the CTAN topic \href{http://ctan.org/topic/subdocs}{\textsf{subdocs}}
for further related packages.
The present package differs from the above solutions in that
a document structure constructed with the conventional |\include| mechanism
just needs two extra commands at the top of every file
such that all constituent files can be compiled individually.

%%%%%%%%%%%%%%%%%%%%%%%%%%%%%%%%%%%%%%%%%%%%%%%%%%%%%%%%%%%%%%%%%%%%%%%%%%%%%%%%
%\subsection{Feature Suggestions}
%
%The following is a list of features which may be useful for future
%versions of this package:
%%
%\begin{itemize}
%\item
%\ldots
%\end{itemize}

%%%%%%%%%%%%%%%%%%%%%%%%%%%%%%%%%%%%%%%%%%%%%%%%%%%%%%%%%%%%%%%%%%%%%%%%%%%%%%%%
\subsection{Revision History}

%%%%%%%%%%%%%%%%%%%%%%%%%%%%%%%%%%%%%%%%
\paragraph{v2.0:} 2018/12/30

\begin{itemize}
\item
immediate forward processing
\item
added |\childdocby| mechanism
\item
manual restructured
\end{itemize}

%%%%%%%%%%%%%%%%%%%%%%%%%%%%%%%%%%%%%%%%
\paragraph{v1.6:} 2018/01/17

\begin{itemize}
\item
application for development of include files
\item
corrections to manual
\end{itemize}

%%%%%%%%%%%%%%%%%%%%%%%%%%%%%%%%%%%%%%%%
\paragraph{v1.5:} 2017/05/21

\begin{itemize}
\item
more complete structuring introduced
\item
|\childdocof| introduced
\item
|\childdoc| renamed to |\childdocmain|
\item
|\childredirect| renamed to |\childdocforward| and |\childdocforwardprefix|
and functionality expanded
\end{itemize}

%%%%%%%%%%%%%%%%%%%%%%%%%%%%%%%%%%%%%%%%
\paragraph{v1.0:} 2017/04/27

\begin{itemize}
\item
manual and install package
\item
first version published on CTAN
\end{itemize}

%%%%%%%%%%%%%%%%%%%%%%%%%%%%%%%%%%%%%%%%
\paragraph{v0.6:} 2017/04/26

\begin{itemize}
\item
redirection mechanism added
\end{itemize}

%%%%%%%%%%%%%%%%%%%%%%%%%%%%%%%%%%%%%%%%
\paragraph{v0.5:} 2017/04/26

\begin{itemize}
\item
functionality in definition file
\end{itemize}


%%%%%%%%%%%%%%%%%%%%%%%%%%%%%%%%%%%%%%%%%%%%%%%%%%%%%%%%%%%%%%%%%%%%%%%%%%%%%%%%
%%%%%%%%%%%%%%%%%%%%%%%%%%%%%%%%%%%%%%%%%%%%%%%%%%%%%%%%%%%%%%%%%%%%%%%%%%%%%%%%
%%%%%%%%%%%%%%%%%%%%%%%%%%%%%%%%%%%%%%%%%%%%%%%%%%%%%%%%%%%%%%%%%%%%%%%%%%%%%%%%
\appendix

\settowidth\MacroIndent{\rmfamily\scriptsize 000\ }

 \DocInput{childdoc.dtx}

\end{document}
%</driver>
% \fi
%
% %%%%%%%%%%%%%%%%%%%%%%%%%%%%%%%%%%%%%%%%%%%%%%%%%%%%%%%%%%%%%%%%%%%%%%%%%%%%%%
% %%%%%%%%%%%%%%%%%%%%%%%%%%%%%%%%%%%%%%%%%%%%%%%%%%%%%%%%%%%%%%%%%%%%%%%%%%%%%%
% \section{Sample}
%\iffalse
%<*samplemain>
%\fi
%
% The following presents a sample document
% with two chapters, two parts, a title page,
% a compile flag as well as three forwarding files to set the flag.
% It consists of eight |.tex| files:
% \begin{center}
% \begin{tabular}{ll}
% |cdocsamp.tex|&main file\\
% |cdocsch1.tex|&include file for chapter 1\\
% |cdocsch2.tex|&include file for chapter 2\\
% |cdocspt3.tex|&include file for part 3\\
% |cdocspt4.tex|&include file for part 4\\
% |cdocsdrf.tex|&forwarding file for main file in draft mode\\
% |cdocsfi1.tex|&forwarding file for final version of chapter 1\\
% |cdocsfi2.tex|&forwarding file for final version of chapter 2\\
% \end{tabular}
% \end{center}
% Each of the eight files can be compiled directly by the \LaTeX{} compiler.
%
% %%%%%%%%%%%%%%%%%%%%%%%%%%%%%%%%%%%%%%
% \paragraph{Main File.}
%
% The main file is called |cdocsamp.tex|.
%
% Load the \textsf{childdoc} definitions and
% declare the filename for the main document:
%    \begin{macrocode}
\input{childdoc.def}
\childdocmain{}
%    \end{macrocode}

% Optional override for |\version| flag:
%    \begin{macrocode}
%%\ifchilddoc\else\providecommand{\version}{draft}\fi
%    \end{macrocode}

% Define the default values for the |\version| flag
% (|final| for the main file and |draft| for childs):
%    \begin{macrocode}
\ifchilddoc
\providecommand{\version}{draft}
\else
\providecommand{\version}{final}
\fi
%    \end{macrocode}

% Load the standard document class:
%    \begin{macrocode}
\documentclass[12pt]{article}
%    \end{macrocode}

% Start the document body:
%    \begin{macrocode}
\begin{document}
%    \end{macrocode}

% Declare a title page.
% Print title, part of document being processed and version flag:
%    \begin{macrocode}
\addtocounter{page}{-1}
\begin{center}
{\LARGE\bfseries{}childdoc example\par}
\vspace{1cm}
\ifchilddoc
\ifchilddocmanual part\else chapter\fi:
`\childdocname' of `\childdocjob'\par
\else
main document: `\childdocjob'\par
\fi
version: \version\par
\end{center}
\newpage
%    \end{macrocode}

% Manually include selected file,
% otherwise process as usual:
%    \begin{macrocode}
\ifchilddocmanual
\section*{part `\childdocname'}
\input{\childdocname}
\else
%    \end{macrocode}

% Include the two chapters:
%    \begin{macrocode}
\include{cdocsch1}
\include{cdocsch2}
%    \end{macrocode}

% Include the two parts unless only chapters should be displayed:
%    \begin{macrocode}
\ifchilddoc\else
\section{part three}
\input{cdocspt3}
\section{part four}
\input{cdocspt4}
\fi
%    \end{macrocode}

% Process as usual until here:
%    \begin{macrocode}
\fi
%    \end{macrocode}

% End of document body:
%    \begin{macrocode}
\end{document}
%    \end{macrocode}
%\iffalse
%</samplemain>
%\fi
%
% %%%%%%%%%%%%%%%%%%%%%%%%%%%%%%%%%%%%%%
% \paragraph{Chapter Include Files.}
%
% The include files are called |cdocsch1.tex| and |cdocsch2.tex|.
%
%\iffalse
%<*samplechap1|samplechap2>
%\fi

% Optional override for |\version| flag:
%    \begin{macrocode}
%%\providecommand{\version}{final}
%    \end{macrocode}

% Include the main document:
%    \begin{macrocode}
\input{childdoc.def}
\childdocof{cdocsamp}
%    \end{macrocode}

%\iffalse
%</samplechap1|samplechap2>
%\fi
%
%\iffalse
%<*samplechap1>
%\fi
% Some text for chapter 1:
%    \begin{macrocode}
\section{one}
some text in chapter one
%    \end{macrocode}

%\iffalse
%</samplechap1>
%\fi
% Some text for chapter 2:
%\iffalse
%<*samplechap2>
%\fi
%    \begin{macrocode}
\section{two}
more text in chapter two
%    \end{macrocode}

%\iffalse
%</samplechap2>
%\fi
%
% %%%%%%%%%%%%%%%%%%%%%%%%%%%%%%%%%%%%%%
% \paragraph{Part Include Files.}
%
% The include files are called |cdocspt3.tex| and |cdocspt4.tex|.
%
%\iffalse
%<*samplepart3|samplepart4>
%\fi

% Optional override for |\version| flag:
%    \begin{macrocode}
%%\providecommand{\version}{final}
%    \end{macrocode}

% Include the main document:
%    \begin{macrocode}
\input{childdoc.def}
\childdocby{cdocsamp}
%    \end{macrocode}

%\iffalse
%</samplepart3|samplepart4>
%\fi
%
%\iffalse
%<*samplepart3>
%\fi
% Some text for part 3:
%    \begin{macrocode}
some text in part three
%    \end{macrocode}

%\iffalse
%</samplepart3>
%\fi
% Some text for part 4:
%\iffalse
%<*samplepart4>
%\fi
%    \begin{macrocode}
more text in part four
%    \end{macrocode}

%\iffalse
%</samplepart4>
%\fi
%
% %%%%%%%%%%%%%%%%%%%%%%%%%%%%%%%%%%%%%%
% \paragraph{Forwarding for a Complete Draft.}
%
% The following forwarding file |cdocsdrf.tex|
% compiles the main document in draft mode:
%\iffalse
%<*sampledraft>
%\fi
%    \begin{macrocode}
\def\version{draft}
\input{childdoc.def}
\childdocforward{cdocsamp}
%    \end{macrocode}

%\iffalse
%</sampledraft>
%\fi
%
% %%%%%%%%%%%%%%%%%%%%%%%%%%%%%%%%%%%%%%
% \paragraph{Forwarding for Final Version of the Chapters.}
%
% The following forwarding files |cdocsfn1.tex| and |cdocsfn2.tex|
% (with identical content)
% compile the final versions of the child documents
% |cdocsch1.tex| and |cdocsch2.tex|, respectively:
%\iffalse
%<*samplefinal>
%\fi
%    \begin{macrocode}
\def\version{final}
\input{childdoc.def}
\childdocforwardprefix[cdocsamp]{cdocsfn}{cdocsch}
%    \end{macrocode}

%\iffalse
%</samplefinal>
%\fi
%
% %%%%%%%%%%%%%%%%%%%%%%%%%%%%%%%%%%%%%%
% \paragraph{Command Line Processing.}
%
% The following three command lines generate the output files
% |cdocscld|, |cdocscl1| and |cdocscl2|
% which should be identical to
% |cdocsdrf|, |cdocsch1| and |cdocsfn2|, respectively:
% \begin{center}
% \begin{tabular}{l}
% |latex -jobname cdocscld \|\\
% |  "\def\version{draft}\input{childdoc.def}\childdocforward{cdocsamp}"|\\
% |latex -jobname cdocscl1 \|\\
% |  "\input{childdoc.def}\childdocforward[cdocsamp]{cdocsch1}"|\\
% |latex -jobname cdocscl2 \|\\
% |  "\def\version{final}\input{childdoc.def}\childdocforward{cdocsch2}"|
% \end{tabular}
% \end{center}
% Note that the trailing backslash on each first line
% merely continues the input to the second line
% (for convenient cut ant paste).
% Furthermore, the command |latex| can be replaced by any
% of its alternative versions such as |pdflatex|.
%
% %%%%%%%%%%%%%%%%%%%%%%%%%%%%%%%%%%%%%%%%%%%%%%%%%%%%%%%%%%%%%%%%%%%%%%%%%%%%%%
% %%%%%%%%%%%%%%%%%%%%%%%%%%%%%%%%%%%%%%%%%%%%%%%%%%%%%%%%%%%%%%%%%%%%%%%%%%%%%%
% \section{Implementation}
%\iffalse
%<*package>
%\fi
%
% This section describes the definitions file |childdoc.def|.

% The definitions cannot be loaded using |\usepackage| or |\RequirePackage|
% which has a mechanism to prevent loading a style file more than once.
% When loading the definitions by means of |\input|
% multiple instances have to be prevented manually:
%\iffalse
%This code needs to be before the `\ProvidesFile' directive
%which is defined at the beginning of this file.
%Therefore it is also placed there and commented out here.
%</package>
%<*discard>
%\fi
%    \begin{macrocode}
\ifdefined\childdocmain\endinput\fi
%    \end{macrocode}
%\iffalse
%</discard>
%<*package>
%\fi
%
% \macro{\ifchilddoc}
% \macro{\ifchilddocmanual}
% The conditional |\ifchilddoc| tells whether a
% child (true) or main (false) document is being compiled.
% The conditional |\ifchilddocmanual| tells whether
% the |\includeonly| mechanism is used (false) or
% the selection of child files must be performed manually (true).
% The definitions initialise to false:
%    \begin{macrocode}
\newif\ifchilddoc
\newif\ifchilddocmanual
%    \end{macrocode}

% \macro{\childdocname}
% \macro{\childdocjob}
% The macro |\childdocname| stores the name of the main document
% to be compiled. The macro |\childdocjob| stores the name of
% the document on which the \LaTeX{} compiler was originally invoked.
% The content of |\jobname| cannot be compared
% to filenames specified in the source due to different catcodes.
% The following code rescans |\jobname|, stores the result
% in |\childdocname| and saves a copy in |\childdocjob|:
%    \begin{macrocode}
\edef\childdocname{\scantokens\expandafter{\jobname\noexpand}}
\let\childdocjob\childdocname
%    \end{macrocode}

% \macro{\childdocdisable}
% The macro |\childdocdisable| prevents the main file
% from being processed more than once.
% At this stage, the main document command |\childdocmain|
% is assumed to be called once again where it should do nothing.
% Any subsequent call to it should prevent
% a secondary processing of the main document
% It overwrites the forwarding commands
% |\childdocof| and |\childdocforward|
% with empty macros to prevent further inclusions of the main document:
%    \begin{macrocode}
\newcommand{\childdocdisable}
{
  \renewcommand{\childdocmain}[1]{\renewcommand{\childdocmain}[1]{\endinput}}
  \renewcommand{\childdocof}[1]{}
  \renewcommand{\childdocby}[2][]{}
  \renewcommand{\childdocforward}[2][]{}
  \renewcommand{\childdocdisable}{}
}
%    \end{macrocode}

% \macro{\childdocmain}
% The macro |\childdocmain| is to be called at the top of the main file
% with nothing or the main filename (without extension) as argument.
% First, it breaks loops.
% If the argument is not empty and does not match |\childdocname|
% (which is set by the first inclusion of |childdoc.def|),
% |\ifchilddoc| is set to true, |\includeonly| is applied to the child file
% and |\jobname| is set to the main file
% (for proper handling of |.aux| files):
%    \begin{macrocode}
\newcommand{\childdocmain}[1]
{
  \childdocdisable\childdocmain{}
  \if?#1?\else
    \begingroup
      \def\childdoctmp{#1}
      \ifx\childdoctmp\childdocname
        \def\childdoctmp{}
      \else
        \def\childdoctmp
        {
          \childdoctrue
          \includeonly{\childdocname}
          \def\childdocjob{#1}
          \def\jobname{#1}
        }
      \fi
      \expandafter
    \endgroup
    \childdoctmp
  \fi
}
%    \end{macrocode}

% \macro{\childdocof}
% The command |\childdocof| redirects
% compilation to the main file |#1|.
%    \begin{macrocode}
\newcommand{\childdocof}[1]
{
  \childdocdisable
  \childdoctrue
  \includeonly{\childdocname}
  \def\jobname{#1}
  \def\childdocjob{#1}
  \input{#1}
}
%    \end{macrocode}

% \macro{\childdocby}
% The command |\childdocby| ....
%    \begin{macrocode}
\newcommand{\childdocby}[2][]
{
  \childdocdisable
  \childdoctrue
  \childdocmanualtrue
  \if?#1?\else
    \def\jobname{#2}
  \fi
  \def\childdocjob{#2}
  \input{#2}
  \endinput
}
%    \end{macrocode}

% \macro{\childdocforward}
% The command |\childdocforward| redirects
% compilation to the main file or
% (if the optional argument is given) a child file.
% Parameters are set as if the main file
% or a child file starting with |\childdocof| was compiled.
% Then compilation is handed over to the main file:
%    \begin{macrocode}
\newcommand{\childdocforward}[2][]
{
  \begingroup
    \if?#1?
      \def\childdoctmp
      {
        \def\childdocname{#2}
        \def\childdocjob{#2}
        \def\jobname{#2}
        \input{#2}
        \endinput
      }
    \else
      \def\childdoctmp
      {
        \childdocdisable
        \def\childdocname{#2}
        \childdoctrue
        \includeonly{#2}
        \def\childdocjob{#1}
        \def\jobname{#1}
        \input{#1}
        \endinput
      }
    \fi
    \expandafter
  \endgroup
  \childdoctmp
}
%    \end{macrocode}

% \macro{\childdocforwardprefix}
% The command |\childdocforwardprefix| redirects
% compilation to the main or a child file by means of a pattern.
% The prefix |#1| in the current filename is replaced by |#2|
% and the suffix of the current filename is kept
% (it is assumed that the filename does not contain the substring `|~~~|'
% which is used as a delimiter).
% Compilation is handed over to the new file by |\childdocforward|:
%    \begin{macrocode}
\newcommand{\childdocforwardprefix}[3][]
{
  \begingroup
    \def\childdocextract #2##1~~~{\def\childdoctmp{\childdocforward[#1]{#3##1}}}
    \expandafter\childdocextract\childdocname~~~
    \expandafter
  \endgroup
  \childdoctmp
}
%    \end{macrocode}

% \macro{\childdoc}
% The deprecated macro |\childdoc| is a legacy version of |\childdocmain|:
%    \begin{macrocode}
\newcommand{\childdoc}{\childdocmain}
%    \end{macrocode}

% \macro{\childdocredirect}
% The deprecated macro |\childdocredirect| is a legacy version
% of |\childdocforward| and |\childdocforwardprefix|:
%    \begin{macrocode}
\newcommand{\childdocredirect}[2][]
{
  \begingroup
    \if?#1?
      \def\childdoctmp{\childdocforward{#2}}
    \else
      \def\childdoctmp{\childdocforwardprefix{#1}{#2}}
    \fi
    \expandafter
  \endgroup
  \childdoctmp
}
%    \end{macrocode}

%\iffalse
%</package>
%\fi
%
\endinput
|\\
|\childdocforward{|\textit{main}|}|\\
\end{tabular}
\end{center}
%
or alternatively with:
%
\begin{center}
\begin{tabular}{l}
|% \iffalse
%
% childdoc.dtx Copyright (C) 2017-2018 Niklas Beisert
%
% This work may be distributed and/or modified under the
% conditions of the LaTeX Project Public License, either version 1.3
% of this license or (at your option) any later version.
% The latest version of this license is in
%   http://www.latex-project.org/lppl.txt
% and version 1.3 or later is part of all distributions of LaTeX
% version 2005/12/01 or later.
%
% This work has the LPPL maintenance status `maintained'.
%
% The Current Maintainer of this work is Niklas Beisert.
%
% This work consists of the files childdoc.dtx and childdoc.ins
% and the derived files childdoc.def and cdocsamp.tex with
% cdocsch1.tex, cdocsch2.tex, cdocsdrf.tex, cdocsfn1.tex, cdocsfn2.tex.
%
%<package>\ifdefined\childdocmain\endinput\fi
%<package>\ProvidesFile{childdoc.def}[2018/12/30 v2.0 child document driver]
%<samplemain>\ProvidesFile{cdocsamp.tex}[2018/12/30 v2.0 sample for childdoc]
%<*driver>
%\ProvidesFile{childdoc.drv}[2018/12/30 v2.0 childdoc reference manual file]
\PassOptionsToClass{10pt,a4paper}{article}
\documentclass{ltxdoc}

\usepackage[margin=35mm]{geometry}
\usepackage{hyperref}
\usepackage{hyperxmp}
\usepackage[usenames]{color}

\hypersetup{colorlinks=true}
\hypersetup{pdfstartview=FitH}
\hypersetup{pdfpagemode=UseNone}
\hypersetup{pdfsource={}}
\hypersetup{pdflang={en-UK}}
\hypersetup{pdfcopyright={Copyright 2017-2018 Niklas Beisert.
  This work may be distributed and/or modified under the
  conditions of the LaTeX Project Public License, either version 1.3
  of this license or (at your option) any later version.}}
\hypersetup{pdflicenseurl={http://www.latex-project.org/lppl.txt}}
\hypersetup{pdfcontactaddress={ETH Zurich, ITP, HIT K,
  Wolfgang-Pauli-Strasse 27}}
\hypersetup{pdfcontactpostcode={8093}}
\hypersetup{pdfcontactcity={Zurich}}
\hypersetup{pdfcontactcountry={Switzerland}}
\hypersetup{pdfcontactemail={nbeisert@itp.phys.ethz.ch}}
\hypersetup{pdfcontacturl={http://people.phys.ethz.ch/\xmptilde nbeisert/}}

\newcommand{\secref}[1]{\hyperref[#1]{section \ref*{#1}}}

\parskip1ex
\parindent0pt
\let\olditemize\itemize
\def\itemize{\olditemize\parskip0pt}

\begin{document}

\title{The \textsf{childdoc} Package}
\hypersetup{pdftitle={The childdoc Package}}
\author{Niklas Beisert\\[2ex]
  Institut f\"ur Theoretische Physik\\
  Eidgen\"ossische Technische Hochschule Z\"urich\\
  Wolfgang-Pauli-Strasse 27, 8093 Z\"urich, Switzerland\\[1ex]
  \href{mailto:nbeisert@itp.phys.ethz.ch}
  {\texttt{nbeisert@itp.phys.ethz.ch}}}
\hypersetup{pdfauthor={Niklas Beisert}}
\hypersetup{pdfsubject={Manual for the LaTeX2e Package childdoc}}
\date{30 December 2018, \textsf{v2.0}}
\maketitle

\begin{abstract}\noindent
\textsf{childdoc} is a \LaTeXe{} package
that enables the direct compilation
of document sections included by |\include|
to individual files.
\end{abstract}

\begingroup
\parskip0ex
\tableofcontents
\endgroup

%%%%%%%%%%%%%%%%%%%%%%%%%%%%%%%%%%%%%%%%%%%%%%%%%%%%%%%%%%%%%%%%%%%%%%%%%%%%%%%%
%%%%%%%%%%%%%%%%%%%%%%%%%%%%%%%%%%%%%%%%%%%%%%%%%%%%%%%%%%%%%%%%%%%%%%%%%%%%%%%%
\section{Introduction}

\LaTeX{} provides a mechanism to structure a large document (such as a book)
into a main file and several child files (containing the chapters)
using the |\include| command.
This mechanism is beneficial for documents
which span hundreds of pages in order to
make the source file(s) more manageable.
Moreover, compilation can be restricted to
selected child files by means of the |\includeonly| command.
The latter feature can be used to reduce the compilation time while editing
(this was significantly more useful in the earlier days of \LaTeX{})
or to generate a smaller document which is easier to navigate.
Another application of |\includeonly| is to generate
documents consisting of selected parts of the complete document.

However, there are a few drawbacks of the plain |\include| mechanism:
\begin{itemize}
\item
The child files cannot be compiled on their own,
they can only be compiled via the main file.
A naive editing environment
(such as a text editor with an option
to have the current file processed by \LaTeX)
may require one to switch to the main file before compiling;
attempting to compile the child file produces errors.
\item
The main file must be modified (each time)
to adjust the |\includeonly| command
to the present needs. This easily leaves the main file in a messy state.
\item
The generated document will always carry the filename
of the main document. This is inconvenient if
several child files are to be compiled and
to be kept for distribution.
\end{itemize}

The present package provides a simple interface
to make child files individually compilable by \LaTeX{}.
Compiling a child file then has the same effect as compiling
the main file with an |\includeonly| command
to select the appropriate child.
Moreover the generated document will carry the name of the child
rather than the main file.
This resolves all three above issues.

This feature is meant to make the editing of books,
thesis documents and lecture notes somewhat more convenient.
However, the package can also be used efficiently for
composing a series of documents (such as exercise sheets)
which are typically distributed individually.
It then assists the author in generating the individual documents
(potentially in different versions)
as well as a document containing the collected series.
Another application is in developing style files
or other kinds of included material
where compilation of the style file could redirect
to a sample or test file.

%%%%%%%%%%%%%%%%%%%%%%%%%%%%%%%%%%%%%%%%%%%%%%%%%%%%%%%%%%%%%%%%%%%%%%%%%%%%%%%%
%%%%%%%%%%%%%%%%%%%%%%%%%%%%%%%%%%%%%%%%%%%%%%%%%%%%%%%%%%%%%%%%%%%%%%%%%%%%%%%%
\section{Usage}

First of all, the package \textsf{childdoc} is \emph{not} a standard
\LaTeXe{} |.sty| style file! Therefore it needs to be invoked in
a non-standard way.

%%%%%%%%%%%%%%%%%%%%%%%%%%%%%%%%%%%%%%%%%%%%%%%%%%%%%%%%%%%%%%%%%%%%%%%%%%%%%%%%
\subsection{Included Files}
\label{sec:include}

%%%%%%%%%%%%%%%%%%%%%%%%%%%%%%%%%%%%%%%%
\DescribeMacro{\childdocmain}
To use the package, add the commands
\begin{center}
\begin{tabular}{l}
|\input{childdoc.def}|\\
|\childdocmain{}|\\
\end{tabular}
\end{center}
at the very top of the main \LaTeX{} file,
in particular \emph{before} the |\documentclass| statement!
The argument of |\childdocmain| should be left empty
(but it must be present).

%%%%%%%%%%%%%%%%%%%%%%%%%%%%%%%%%%%%%%%%
\DescribeMacro{\childdocof}
Furthermore, add the commands
\begin{center}
\begin{tabular}{l}
|\input{childdoc.def}|\\
|\childdocof{|\textit{main}|}|\\
\end{tabular}
\end{center}
at the top of every child file \textit{child}
which is included by |\include{|\textit{child}|}|
from within the main file
(or at least for those files to be compiled individually).
The argument \textit{main} must be the filename of the main file.

There are a couple of
considerations in setting up the main and child documents:

%%%%%%%%%%%%%%%%%%%%%%%%%%%%%%%%%%%%%%%%
\paragraph{Restrictions.}

Please note the following restrictions:
\begin{itemize}
\item
|\childdocmain| must be called with one argument \textit{main}
to ensure compatibility with earlier version of the package.
It must either be empty (|\childdocmain{}|)
or precisely match the filename of the main file in which it is specified.
See \secref{sec:detection} for further information.
\item
The filename \textit{main} must be specified without the |.tex| extension.
\item
The filename \textit{main} is case sensitive
(even in case-insensitive file systems)
due to internal string comparison.
\item
The argument \textit{main} should be fully expanded, it cannot be a macro.
\item
Subdirectories and special characters should be avoided in filenames.
\item
The command |\childdocmain{|\textit{main}|}| must be followed by a whitespace.
It should not be followed immediately by another command
or by a comment mark `|%|'.
This is because the \TeX{} parser reads the token immediately following
the argument of |\childdocmain| and puts it
at the beginning of every child section;
however, a white\-space is ignored.
\end{itemize}

%%%%%%%%%%%%%%%%%%%%%%%%%%%%%%%%%%%%%%%%
\paragraph{Content of Main File.}

It is advisable to place all content in the child files included by |\include|.
Any output contained in the main file will appear in all child documents
unless suppressed manually;
it cannot be suppressed automatically by the |\includeonly| directive
and thus should normally be avoided.
A method to include some content in the main file
by means of conditional processing is described in \secref{sec:conditional}.

%%%%%%%%%%%%%%%%%%%%%%%%%%%%%%%%%%%%%%%%
\paragraph{Page Numbering.}

When only a part of the document is compiled,
the appropriate numbering of pages
(as well as other status parameters)
is determined from the |.aux| files.
The latter contain information from previous passes.
However this information needs to propagate through
all intermediate child documents.
Therefore the page numbering in child documents may well
be inconsistent until the complete document is compiled at least once.

A useful (if unconventional) way to always ensure a consistent
page numbering is to restart the numbering in each child document
and denote the pages by `\textit{child}|.|\textit{page}'
where \textit{child} represents the chapter/section number of the child file.
This can be achieved by the command
|\numberwithin{page}{|\textit{child}|}|
of the \textsf{amsmath} package
where \textit{child} can be |chapter| or |section|
depending on the chosen structuring.
Alternatively, one can modify the macro |\thepage| appropriately
and reset the counter |page| at the start of each child file.

%%%%%%%%%%%%%%%%%%%%%%%%%%%%%%%%%%%%%%%%%%%%%%%%%%%%%%%%%%%%%%%%%%%%%%%%%%%%%%%%
\subsection{Conditional Processing}
\label{sec:conditional}

The package provides a mechanism to compile different versions
of a document. To customise the versions further some conditional processing
can come in handy to distinguish which version is being compiled.
The package provides two macros to describe the compilation context:

%%%%%%%%%%%%%%%%%%%%%%%%%%%%%%%%%%%%%%%%
\DescribeMacro{\ifchilddoc}
The conditional |\ifchilddoc| distinguishes between the compilation of
child documents and the main document:
%
\begin{center}
|\ifchilddoc |\textit{child-code}| |[|\||else |\textit{main-code}]| \||fi|
\end{center}

%%%%%%%%%%%%%%%%%%%%%%%%%%%%%%%%%%%%%%%%
\DescribeMacro{\childdocname}
\DescribeMacro{\childdocjob}
The macro |\childdocname| contains the filename (without extension)
of the main or child file being processed.
Note that |\childdocjob| will always contain the name of the main file.

%%%%%%%%%%%%%%%%%%%%%%%%%%%%%%%%%%%%%%%%
\paragraph{Title Page.}

Conditional processing can be used to include a title or banner page
in the main document when proper precautions are taken.
Importantly, the code in the main file should ensure that the page counter
(as well as other status parameters which are stored in the |.aux| files)
takes the same value after the conditional processing.
Otherwise the page numbers may take divergent values
depending on which part is compiled.

For example, a title page could be declared by:
%
\begin{center}
\begin{tabular}{l}
|\ifchilddoc\||else|\\
|\addtocounter{page}{-1}|\\
\textit{code for title page}\\
|\newpage|\\
|\||fi|
\end{tabular}
\end{center}
%
A banner page for the child documents can be generated by:
%
\begin{center}
\begin{tabular}{l}
|\ifchilddoc|\\
|\addtocounter{page}{-1}|\\
\textit{code for banner page}\\
|\newpage|\\
|\||fi|
\end{tabular}
\end{center}
%
Here one could write a message such as:
\begin{center}
|This is the part \childdocname{} of \childdocjob{}.|
\end{center}

%%%%%%%%%%%%%%%%%%%%%%%%%%%%%%%%%%%%%%%%%%%%%%%%%%%%%%%%%%%%%%%%%%%%%%%%%%%%%%%%
\subsection{Flags}
\label{sec:flags}

The package makes it easy to generate different versions
of the main or child documents.
To this end compilation flags can be defined
and assigned different default values.
They will be particularly useful in conjunction
with the forwarding mechanism described in \secref{sec:forward}.

For example, it may be useful to have a flag |\version|
which can be set to |draft| or |final|.
The document source will contain some conditional code
depending on the value of |\version|.
Suppose further, the flag should default to |final| for the main file
and to |draft| for child files
which is a natural assignment for editing the document.
This is achieved by placing the following code
in the preamble of the main document
(below the |\childdocmain| directive):
%
\begin{center}
\begin{tabular}{l}
|\ifchilddoc|\\
|\providecommand{\version}{draft}|\\
|\||else|\\
|\providecommand{\version}{final}|\\
|\||fi|
\end{tabular}
\end{center}
%
The definition by |\providecommand| makes sure
that previous definitions are not overwritten.
Further statements |\providecommand{\version}{...}|
can thus be added before the above code to override it.

For the main file, one might add a line
(between |\childdocmain| and the above block)
%
\begin{center}
|%\ifchilddoc\||else\providecommand{\version}{draft}\||fi|
\end{center}
%
which can be uncommented to produce a draft version.
Likewise one can add a line to the very top of a child file
(above the |\childdocof{|\textit{main}|}| directive)
%
\begin{center}
|%\providecommand{\version}{final}|
\end{center}
%
which can be uncommented to produce the final version of this child document.

%%%%%%%%%%%%%%%%%%%%%%%%%%%%%%%%%%%%%%%%%%%%%%%%%%%%%%%%%%%%%%%%%%%%%%%%%%%%%%%%
\subsection{Forwarding}
\label{sec:forward}

Different versions of the main or child documents
using compilation flags as described in \secref{sec:flags}
can be (permanently) stored in different files
for convenient compilation, viewing and distribution.
To this end, the package defines a command
to pass on compilation to a different file:

%%%%%%%%%%%%%%%%%%%%%%%%%%%%%%%%%%%%%%%%
\DescribeMacro{\childdocforward}
The command |\childdocforward| redirects processing to
another source file:
%
\begin{center}
\begin{tabular}{l}
|\input{childdoc.def}|\\
|\childdocforward[|\textit{main}|]{|\textit{dest}|}|\\
\end{tabular}
\end{center}
%
The argument \textit{dest} is the destination file
(without extension).
It should be the main file or one of the child files.
Note that further \textsf{childdoc} directives
such as |\childdocof| and |\childdocforward|
in the indicated file will be processed in this form.
The optional argument \textit{main}
passes on directly to the main file \textit{main}
while pretending to compile the child \textit{dest}.
This form behaves as if \textit{dest}
issues |\childdocof{|\textit{main}|}| right away,
and no further \textsf{childdoc} directives will be processed.

%%%%%%%%%%%%%%%%%%%%%%%%%%%%%%%%%%%%%%%%
\DescribeMacro{\...prefix}
In the alternative form |\childdocforwardprefix|,
%
\begin{center}
\begin{tabular}{l}
|\input{childdoc.def}|\\
|\childdocforwardprefix[|\textit{main}|]{|\textit{prefix}|}{|\textit{dest}|}|
\end{tabular}
\end{center}
%
the destination file is determined by a pattern
depending on the current file:
To make this work, the current file must be called
`{\textit{prefix}\hspace{0.2em}\textit{suffix}}'
with \textit{prefix} matching precisely the argument.
Processing is then passed on to the file
`{\textit{dest}\hspace{0.2em}\textit{suffix}}'.
Surely, the same effect is achieved by
directly specifying the
argument `{\textit{dest}\hspace{0.2em}\textit{suffix}}'
in the first form.
However, that requires to set up a different file
for each child. With the alternative form of the command
all these files can have exactly the same content
which simplifies setting them up and maintaining them.

For example, the following file |draft.tex|
with a compilation flag |\version| as described in \secref{sec:flags}
compiles the main document as a draft:
%
\begin{center}
\begin{tabular}{l}
|\def\version{draft}|\\
|\input{childdoc.def}|\\
|\childdocforward{|\textit{main}|}|
\end{tabular}
\end{center}
%
Likewise, the following files |final|\textit{nn}|.tex|
compile the final version of the child document
|child|\textit{nn}|.tex|:
%
\begin{center}
\begin{tabular}{l}
|\def\version{final}|\\
|\input{childdoc.def}|\\
|\childdocforwardprefix{final}{child}|
\end{tabular}
\end{center}
%

Note that when several versions of a main file and/or of each child file
are to be generated, it may be convenient to set up a |Makefile| or
shell script to automatise the process.

%%%%%%%%%%%%%%%%%%%%%%%%%%%%%%%%%%%%%%%%%%%%%%%%%%%%%%%%%%%%%%%%%%%%%%%%%%%%%%%%
\subsection{Command Line Processing}
\label{sec:commandline}

The effect of redirection files can also be achieved by invoking
the \LaTeX{} compiler with a more elaborate command line.
Most conveniently this should be done as part
of a shell script or a |Makefile|.

When using \textsf{childdoc} in the main file, the following
command lines effectively perform a redirection
(note that depending on the shell being used,
backslashes may have to be doubled: `|\|' $\to$ `|\\|'):
%
\begin{center}
|... -jobname "|\textit{target}|" |\\|"|[\textit{flags}]%
|\input{childdoc.def}\childdocforward[|\textit{main}|]{|\textit{dest}|}"|
\end{center}
%
Here \textit{target} is the name of the output file,
\textit{main} is the name of the main file
and \textit{dest} is the name of the main or child file to be processed
(all filenames without extensions).
The optional argument \textit{main} can be omitted
if \textit{main} matches \textit{dest}.
Optionally, compilation \textit{flags} can be defined via |\def| commands.
This command line makes the \TeX{} engine believe
it is compiling the file \textit{target}
whose content is specified as the latter parameter.
The provided code then forwards the processing to
\textit{main} or \textit{dest} as described in \secref{sec:forward}.

%%%%%%%%%%%%%%%%%%%%%%%%%%%%%%%%%%%%%%%%%%%%%%%%%%%%%%%%%%%%%%%%%%%%%%%%%%%%%%%%
\subsection{Include by Input}
\label{sec:input}

Including child documents by |\include| has some restrictions by design.
Most notably, the content of a child document always occupies
its own set of pages; pages cannot be shared between child documents.
Usually, this behaviour makes perfect sense
because each child document contain an essential part of the document.
However, in some situations it may be desirable to compose
a document from a collection of parts
without having mandatory page breaks between then.
For this case, the package
provides a mechanism to include parts
by |\input| which can also be processed individually.
However, by construction this mechanism
requires manual handling of the content to be output.

%%%%%%%%%%%%%%%%%%%%%%%%%%%%%%%%%%%%%%%%
\DescribeMacro{\ifchilddocmanual}
The main file should be prepared as usual, see \secref{sec:include}.
However, the document body must make a distinction
between processing of an individual part and of the main document, e.g.:
%
\begin{center}
\begin{tabular}{l}
|\ifchilddocmanual|\\
|\input{\childdocname}|\\
|\||else|\\
\textit{document body with }|\input{|\textit{part}|}|\\
|\||fi|
\end{tabular}
\end{center}
%
The conditional |\ifchilddocmanual| is true whenever
a part to be included by |\input| is being compiled,
and the name of the part is stored in |\childdocname|.

%%%%%%%%%%%%%%%%%%%%%%%%%%%%%%%%%%%%%%%%
\DescribeMacro{\childdocby}
Each part to be included by |\input| should start with:
%
\begin{center}
\begin{tabular}{l}
|\input{childdoc.def}|\\
|\childdocby{|\textit{main}|}|\\
\end{tabular}
\end{center}
%
The directive |\childdocby| is similar to |\childdocof|
described in \secref{sec:include},
but the subsequent selection of content must be done manually.
To that end, both |\ifchilddoc| and |\ifchilddocmanual|
will be true upon processing of a part,
and the name of the part is stored in |\childdocname|.
Note that |\jobname| will be set to the filename of the current part
so that each part receives an individual |.aux| file
that does not interfere with the |.aux| file(s) of the main document.
This behaviour can be altered by the alternative form
|\childdocby[*]{|\textit{main}|}| (with a non-empty optional argument)
which uses the |.aux| file of the main document
by setting |\jobname| to \textit{main}.

%%%%%%%%%%%%%%%%%%%%%%%%%%%%%%%%%%%%%%%%%%%%%%%%%%%%%%%%%%%%%%%%%%%%%%%%%%%%%%%%
\subsection{Driver Development}
\label{sec:driver}

The \textsf{childdoc} mechanism can also be use for the development
of definition files such as \LaTeX{} styles or classes.
This case differs from the above setup with multiple parts
included by |\include| in that no |\includeonly| should be invoked.
This can be achieved by starting the include file
(before |\ProvidesPackage|) with:
%
\begin{center}
\begin{tabular}{l}
|\input{childdoc.def}|\\
|\childdocforward{|\textit{main}|}|\\
\end{tabular}
\end{center}
%
or alternatively with:
%
\begin{center}
\begin{tabular}{l}
|\input{childdoc.def}|\\
|\childdocby{|\textit{main}|}|\\
\end{tabular}
\end{center}
%
Both forms have slightly different effects as described above.
The main file is prepared as usual, see \secref{sec:include}.

%%%%%%%%%%%%%%%%%%%%%%%%%%%%%%%%%%%%%%%%%%%%%%%%%%%%%%%%%%%%%%%%%%%%%%%%%%%%%%%%
\subsection{Legacy Detection}
\label{sec:detection}

The directive |\childdocmain| in the main file can detect
whether the complete document or merely a child is to be compiled
even without using the directive |\childdocof|.
This method is deprecated because it is less robust
and there is no compelling reason to use it;
it is merely provided for backward compatibility
and it may be removed in future versions.

If the detection mechanism is to be used,
it is mandatory to correctly specify
the filename of the main file as the argument of |\childdocmain|:
%
\begin{center}
\begin{tabular}{l}
|\input{childdoc.def}|\\
|\childdocmain{|\textit{main}|}|\\
\end{tabular}
\end{center}
%
If |\jobname| does not match the argument \textit{main} of |\childdocmain|,
it is assumed that |\jobname| points to the child file to be compiled.
When using |\childdocmain| with the main file specified as argument,
it suffices to start a child file
with just |\input{|\textit{main}|}|
without loading of the package and using |\childdocof|.
If instead all processing is done
with the appropriate \textsf{childdoc} directives,
the argument of \textit{main} of |\childdocmain| can be empty.

An alternative version of the command line processing described
in \secref{sec:commandline} using the detection mechanism reads:
%
\begin{center}
|... -jobname "|\textit{target}|" "|[\textit{flags}]%
[|\def\jobname{|\textit{dest}|}|]|\input{|\textit{main}|}"|
\end{center}

%%%%%%%%%%%%%%%%%%%%%%%%%%%%%%%%%%%%%%%%%%%%%%%%%%%%%%%%%%%%%%%%%%%%%%%%%%%%%%%%
\subsection{Manual Code}
\label{sec:manual}

In case one cannot be certain whether the definitions file |childdoc.def|
is installed on the target \TeX{} distribution
and one prefers not to ship it,
it is conceivable to paste a few relevant commands into the sources.

To that end, drop all statements |\input{childdoc.def}|
and perform the replacements as outlined below.
Instead of |\childdocmain{|\textit{main}|}| add the following code
to the top of the main file:
%
\begin{center}
\begin{tabular}{l}
|\||ifdefined\childdocname\endinput\||fi\newif\ifchilddoc|\\
|\edef\childdocname{\scantokens\expandafter{\jobname\noexpand}}|\\
|\def\childdocmain{|\textit{main}|}\||ifx\childdocmain\childdocname\||else|\\
|\childdoctrue\includeonly{\childdocname}\let\jobname\childdocmain\||fi|\\
\end{tabular}
\end{center}
%
Instead of |\childdocof{|\textit{main}|}| just include the main file
at the top of each child file:
%
\begin{center}
|\input{|\textit{main}|}|
\end{center}
%
A simple redirection |\childdocforward{|\textit{dest}|}| is achieved by:
%
\begin{center}
|\def\jobname{|\textit{dest}|}\input{\jobname}|
\end{center}
%
The redirection with prefix
|\childdocforwardprefix[|\textit{prefix}|]{|\textit{dest}|}|
is accomplished by:
%
\begin{center}
\begin{tabular}{l}
|{\edef\jobname{\scantokens\expandafter{\jobname\noexpand}}|\\
|\def\redirectjob |\textit{prefix}|#1~~~{\gdef\jobname{|\textit{dest}|#1}}|\\
|\expandafter\redirectjob\jobname~~~}\input{\jobname}|
\end{tabular}
\end{center}

In an alternative approach,
child documents can be compiled by a specific command line
without additional code or specific definitions:
%
\begin{center}
|... -jobname "|\textit{target}|" "|[\textit{flags}]%
|\includeonly{|\textit{dest}|}\input{|\textit{main}|}"|
\end{center}
%

%%%%%%%%%%%%%%%%%%%%%%%%%%%%%%%%%%%%%%%%%%%%%%%%%%%%%%%%%%%%%%%%%%%%%%%%%%%%%%%%
%%%%%%%%%%%%%%%%%%%%%%%%%%%%%%%%%%%%%%%%%%%%%%%%%%%%%%%%%%%%%%%%%%%%%%%%%%%%%%%%
\section{Information}

%%%%%%%%%%%%%%%%%%%%%%%%%%%%%%%%%%%%%%%%%%%%%%%%%%%%%%%%%%%%%%%%%%%%%%%%%%%%%%%%
\subsection{Copyright}

Copyright \copyright{} 2017--2018 Niklas Beisert

This work may be distributed and/or modified under the
conditions of the \LaTeX{} Project Public License, either version 1.3
of this license or (at your option) any later version.
The latest version of this license is in
  \url{http://www.latex-project.org/lppl.txt}
and version 1.3 or later is part of all distributions of \LaTeX{}
version 2005/12/01 or later.

This work has the LPPL maintenance status `maintained'.

The Current Maintainer of this work is Niklas Beisert.

This work consists of the files |README.txt|, |childdoc.ins| and |childdoc.dtx|
as well as the derived files |childdoc.def|, |cdocsamp.tex|
with |cdocsch1.tex|, |cdocsch2.tex|, |cdocspt3.tex|, |cdocspt4.tex|,
|cdocsdrf.tex|, |cdocsfn1.tex|, |cdocsfn2.tex|
as well as |childdoc.pdf|.

%%%%%%%%%%%%%%%%%%%%%%%%%%%%%%%%%%%%%%%%%%%%%%%%%%%%%%%%%%%%%%%%%%%%%%%%%%%%%%%%
\subsection{Files and Installation}

The package consists of the files:
%
\begin{center}
\begin{tabular}{ll}
    |README.txt|   & readme file \\
    |childdoc.ins| & installation file \\
    |childdoc.dtx| & source file \\
    |childdoc.def| & definition file \\
    |cdocsamp.tex| & sample main file \\
    |cdocsch1.tex| & sample include file \\
    |cdocsch2.tex| & sample include file \\
    |cdocspt3.tex| & sample part file \\
    |cdocspt4.tex| & sample part file \\
    |cdocsdrf.tex| & sample redirection file \\
    |cdocsfn1.tex| & sample redirection file \\
    |cdocsfn2.tex| & sample redirection file \\
    |childdoc.pdf| & manual
\end{tabular}
\end{center}
%
The distribution consists of the files
|README.txt|, |childdoc.ins| and |childdoc.dtx|.
%
\begin{itemize}
\item
Run (pdf)\LaTeX{} on |childdoc.dtx|
to compile the manual |childdoc.pdf| (this file).
\item
Run \LaTeX{} on |childdoc.ins| to create the definitions file |childdoc.def|
and the sample |cdocsamp.tex| with include files
|cdocsch1.tex|, |cdocsch2.tex|, |cdocspt3.tex|, |cdocspt4.tex|,
|cdocsdrf.tex|, |cdocsfn1.tex|, |cdocsfn2.tex|.
Then copy the file |childdoc.def| to an appropriate directory of your \LaTeX{}
distribution, e.g.\ \textit{texmf-root}|/tex/latex/childdoc|.
\end{itemize}

%%%%%%%%%%%%%%%%%%%%%%%%%%%%%%%%%%%%%%%%%%%%%%%%%%%%%%%%%%%%%%%%%%%%%%%%%%%%%%%%
\subsection{Related CTAN Packages}

There are several other packages which offer a similar functionality:
%
\begin{itemize}
\item
The packages
\href{http://ctan.org/pkg/docmute}{\textsf{docmute}},
\href{http://ctan.org/pkg/includex}{\textsf{includex}} and
\href{http://ctan.org/pkg/standalone}{\textsf{standalone}}
provide commands to include only the document body of
a child file thus allowing both files to be compiled individually.
\item
The packages \href{http://ctan.org/pkg/subdocs}{\textsf{subdocs}}
and \href{http://ctan.org/pkg/subfiles}{\textsf{subfiles}}
provide structures in which the main and child documents can be
encapsulated and allowing them to be compiled individually.
The inclusion mechanism is different from the conventional |\include|.
\item
The package \href{http://ctan.org/pkg/combine}{\textsf{combine}}
is an elaborate solution to combine several documents into one.
\end{itemize}
%
See also the CTAN topic \href{http://ctan.org/topic/subdocs}{\textsf{subdocs}}
for further related packages.
The present package differs from the above solutions in that
a document structure constructed with the conventional |\include| mechanism
just needs two extra commands at the top of every file
such that all constituent files can be compiled individually.

%%%%%%%%%%%%%%%%%%%%%%%%%%%%%%%%%%%%%%%%%%%%%%%%%%%%%%%%%%%%%%%%%%%%%%%%%%%%%%%%
%\subsection{Feature Suggestions}
%
%The following is a list of features which may be useful for future
%versions of this package:
%%
%\begin{itemize}
%\item
%\ldots
%\end{itemize}

%%%%%%%%%%%%%%%%%%%%%%%%%%%%%%%%%%%%%%%%%%%%%%%%%%%%%%%%%%%%%%%%%%%%%%%%%%%%%%%%
\subsection{Revision History}

%%%%%%%%%%%%%%%%%%%%%%%%%%%%%%%%%%%%%%%%
\paragraph{v2.0:} 2018/12/30

\begin{itemize}
\item
immediate forward processing
\item
added |\childdocby| mechanism
\item
manual restructured
\end{itemize}

%%%%%%%%%%%%%%%%%%%%%%%%%%%%%%%%%%%%%%%%
\paragraph{v1.6:} 2018/01/17

\begin{itemize}
\item
application for development of include files
\item
corrections to manual
\end{itemize}

%%%%%%%%%%%%%%%%%%%%%%%%%%%%%%%%%%%%%%%%
\paragraph{v1.5:} 2017/05/21

\begin{itemize}
\item
more complete structuring introduced
\item
|\childdocof| introduced
\item
|\childdoc| renamed to |\childdocmain|
\item
|\childredirect| renamed to |\childdocforward| and |\childdocforwardprefix|
and functionality expanded
\end{itemize}

%%%%%%%%%%%%%%%%%%%%%%%%%%%%%%%%%%%%%%%%
\paragraph{v1.0:} 2017/04/27

\begin{itemize}
\item
manual and install package
\item
first version published on CTAN
\end{itemize}

%%%%%%%%%%%%%%%%%%%%%%%%%%%%%%%%%%%%%%%%
\paragraph{v0.6:} 2017/04/26

\begin{itemize}
\item
redirection mechanism added
\end{itemize}

%%%%%%%%%%%%%%%%%%%%%%%%%%%%%%%%%%%%%%%%
\paragraph{v0.5:} 2017/04/26

\begin{itemize}
\item
functionality in definition file
\end{itemize}


%%%%%%%%%%%%%%%%%%%%%%%%%%%%%%%%%%%%%%%%%%%%%%%%%%%%%%%%%%%%%%%%%%%%%%%%%%%%%%%%
%%%%%%%%%%%%%%%%%%%%%%%%%%%%%%%%%%%%%%%%%%%%%%%%%%%%%%%%%%%%%%%%%%%%%%%%%%%%%%%%
%%%%%%%%%%%%%%%%%%%%%%%%%%%%%%%%%%%%%%%%%%%%%%%%%%%%%%%%%%%%%%%%%%%%%%%%%%%%%%%%
\appendix

\settowidth\MacroIndent{\rmfamily\scriptsize 000\ }

 \DocInput{childdoc.dtx}

\end{document}
%</driver>
% \fi
%
% %%%%%%%%%%%%%%%%%%%%%%%%%%%%%%%%%%%%%%%%%%%%%%%%%%%%%%%%%%%%%%%%%%%%%%%%%%%%%%
% %%%%%%%%%%%%%%%%%%%%%%%%%%%%%%%%%%%%%%%%%%%%%%%%%%%%%%%%%%%%%%%%%%%%%%%%%%%%%%
% \section{Sample}
%\iffalse
%<*samplemain>
%\fi
%
% The following presents a sample document
% with two chapters, two parts, a title page,
% a compile flag as well as three forwarding files to set the flag.
% It consists of eight |.tex| files:
% \begin{center}
% \begin{tabular}{ll}
% |cdocsamp.tex|&main file\\
% |cdocsch1.tex|&include file for chapter 1\\
% |cdocsch2.tex|&include file for chapter 2\\
% |cdocspt3.tex|&include file for part 3\\
% |cdocspt4.tex|&include file for part 4\\
% |cdocsdrf.tex|&forwarding file for main file in draft mode\\
% |cdocsfi1.tex|&forwarding file for final version of chapter 1\\
% |cdocsfi2.tex|&forwarding file for final version of chapter 2\\
% \end{tabular}
% \end{center}
% Each of the eight files can be compiled directly by the \LaTeX{} compiler.
%
% %%%%%%%%%%%%%%%%%%%%%%%%%%%%%%%%%%%%%%
% \paragraph{Main File.}
%
% The main file is called |cdocsamp.tex|.
%
% Load the \textsf{childdoc} definitions and
% declare the filename for the main document:
%    \begin{macrocode}
\input{childdoc.def}
\childdocmain{}
%    \end{macrocode}

% Optional override for |\version| flag:
%    \begin{macrocode}
%%\ifchilddoc\else\providecommand{\version}{draft}\fi
%    \end{macrocode}

% Define the default values for the |\version| flag
% (|final| for the main file and |draft| for childs):
%    \begin{macrocode}
\ifchilddoc
\providecommand{\version}{draft}
\else
\providecommand{\version}{final}
\fi
%    \end{macrocode}

% Load the standard document class:
%    \begin{macrocode}
\documentclass[12pt]{article}
%    \end{macrocode}

% Start the document body:
%    \begin{macrocode}
\begin{document}
%    \end{macrocode}

% Declare a title page.
% Print title, part of document being processed and version flag:
%    \begin{macrocode}
\addtocounter{page}{-1}
\begin{center}
{\LARGE\bfseries{}childdoc example\par}
\vspace{1cm}
\ifchilddoc
\ifchilddocmanual part\else chapter\fi:
`\childdocname' of `\childdocjob'\par
\else
main document: `\childdocjob'\par
\fi
version: \version\par
\end{center}
\newpage
%    \end{macrocode}

% Manually include selected file,
% otherwise process as usual:
%    \begin{macrocode}
\ifchilddocmanual
\section*{part `\childdocname'}
\input{\childdocname}
\else
%    \end{macrocode}

% Include the two chapters:
%    \begin{macrocode}
\include{cdocsch1}
\include{cdocsch2}
%    \end{macrocode}

% Include the two parts unless only chapters should be displayed:
%    \begin{macrocode}
\ifchilddoc\else
\section{part three}
\input{cdocspt3}
\section{part four}
\input{cdocspt4}
\fi
%    \end{macrocode}

% Process as usual until here:
%    \begin{macrocode}
\fi
%    \end{macrocode}

% End of document body:
%    \begin{macrocode}
\end{document}
%    \end{macrocode}
%\iffalse
%</samplemain>
%\fi
%
% %%%%%%%%%%%%%%%%%%%%%%%%%%%%%%%%%%%%%%
% \paragraph{Chapter Include Files.}
%
% The include files are called |cdocsch1.tex| and |cdocsch2.tex|.
%
%\iffalse
%<*samplechap1|samplechap2>
%\fi

% Optional override for |\version| flag:
%    \begin{macrocode}
%%\providecommand{\version}{final}
%    \end{macrocode}

% Include the main document:
%    \begin{macrocode}
\input{childdoc.def}
\childdocof{cdocsamp}
%    \end{macrocode}

%\iffalse
%</samplechap1|samplechap2>
%\fi
%
%\iffalse
%<*samplechap1>
%\fi
% Some text for chapter 1:
%    \begin{macrocode}
\section{one}
some text in chapter one
%    \end{macrocode}

%\iffalse
%</samplechap1>
%\fi
% Some text for chapter 2:
%\iffalse
%<*samplechap2>
%\fi
%    \begin{macrocode}
\section{two}
more text in chapter two
%    \end{macrocode}

%\iffalse
%</samplechap2>
%\fi
%
% %%%%%%%%%%%%%%%%%%%%%%%%%%%%%%%%%%%%%%
% \paragraph{Part Include Files.}
%
% The include files are called |cdocspt3.tex| and |cdocspt4.tex|.
%
%\iffalse
%<*samplepart3|samplepart4>
%\fi

% Optional override for |\version| flag:
%    \begin{macrocode}
%%\providecommand{\version}{final}
%    \end{macrocode}

% Include the main document:
%    \begin{macrocode}
\input{childdoc.def}
\childdocby{cdocsamp}
%    \end{macrocode}

%\iffalse
%</samplepart3|samplepart4>
%\fi
%
%\iffalse
%<*samplepart3>
%\fi
% Some text for part 3:
%    \begin{macrocode}
some text in part three
%    \end{macrocode}

%\iffalse
%</samplepart3>
%\fi
% Some text for part 4:
%\iffalse
%<*samplepart4>
%\fi
%    \begin{macrocode}
more text in part four
%    \end{macrocode}

%\iffalse
%</samplepart4>
%\fi
%
% %%%%%%%%%%%%%%%%%%%%%%%%%%%%%%%%%%%%%%
% \paragraph{Forwarding for a Complete Draft.}
%
% The following forwarding file |cdocsdrf.tex|
% compiles the main document in draft mode:
%\iffalse
%<*sampledraft>
%\fi
%    \begin{macrocode}
\def\version{draft}
\input{childdoc.def}
\childdocforward{cdocsamp}
%    \end{macrocode}

%\iffalse
%</sampledraft>
%\fi
%
% %%%%%%%%%%%%%%%%%%%%%%%%%%%%%%%%%%%%%%
% \paragraph{Forwarding for Final Version of the Chapters.}
%
% The following forwarding files |cdocsfn1.tex| and |cdocsfn2.tex|
% (with identical content)
% compile the final versions of the child documents
% |cdocsch1.tex| and |cdocsch2.tex|, respectively:
%\iffalse
%<*samplefinal>
%\fi
%    \begin{macrocode}
\def\version{final}
\input{childdoc.def}
\childdocforwardprefix[cdocsamp]{cdocsfn}{cdocsch}
%    \end{macrocode}

%\iffalse
%</samplefinal>
%\fi
%
% %%%%%%%%%%%%%%%%%%%%%%%%%%%%%%%%%%%%%%
% \paragraph{Command Line Processing.}
%
% The following three command lines generate the output files
% |cdocscld|, |cdocscl1| and |cdocscl2|
% which should be identical to
% |cdocsdrf|, |cdocsch1| and |cdocsfn2|, respectively:
% \begin{center}
% \begin{tabular}{l}
% |latex -jobname cdocscld \|\\
% |  "\def\version{draft}\input{childdoc.def}\childdocforward{cdocsamp}"|\\
% |latex -jobname cdocscl1 \|\\
% |  "\input{childdoc.def}\childdocforward[cdocsamp]{cdocsch1}"|\\
% |latex -jobname cdocscl2 \|\\
% |  "\def\version{final}\input{childdoc.def}\childdocforward{cdocsch2}"|
% \end{tabular}
% \end{center}
% Note that the trailing backslash on each first line
% merely continues the input to the second line
% (for convenient cut ant paste).
% Furthermore, the command |latex| can be replaced by any
% of its alternative versions such as |pdflatex|.
%
% %%%%%%%%%%%%%%%%%%%%%%%%%%%%%%%%%%%%%%%%%%%%%%%%%%%%%%%%%%%%%%%%%%%%%%%%%%%%%%
% %%%%%%%%%%%%%%%%%%%%%%%%%%%%%%%%%%%%%%%%%%%%%%%%%%%%%%%%%%%%%%%%%%%%%%%%%%%%%%
% \section{Implementation}
%\iffalse
%<*package>
%\fi
%
% This section describes the definitions file |childdoc.def|.

% The definitions cannot be loaded using |\usepackage| or |\RequirePackage|
% which has a mechanism to prevent loading a style file more than once.
% When loading the definitions by means of |\input|
% multiple instances have to be prevented manually:
%\iffalse
%This code needs to be before the `\ProvidesFile' directive
%which is defined at the beginning of this file.
%Therefore it is also placed there and commented out here.
%</package>
%<*discard>
%\fi
%    \begin{macrocode}
\ifdefined\childdocmain\endinput\fi
%    \end{macrocode}
%\iffalse
%</discard>
%<*package>
%\fi
%
% \macro{\ifchilddoc}
% \macro{\ifchilddocmanual}
% The conditional |\ifchilddoc| tells whether a
% child (true) or main (false) document is being compiled.
% The conditional |\ifchilddocmanual| tells whether
% the |\includeonly| mechanism is used (false) or
% the selection of child files must be performed manually (true).
% The definitions initialise to false:
%    \begin{macrocode}
\newif\ifchilddoc
\newif\ifchilddocmanual
%    \end{macrocode}

% \macro{\childdocname}
% \macro{\childdocjob}
% The macro |\childdocname| stores the name of the main document
% to be compiled. The macro |\childdocjob| stores the name of
% the document on which the \LaTeX{} compiler was originally invoked.
% The content of |\jobname| cannot be compared
% to filenames specified in the source due to different catcodes.
% The following code rescans |\jobname|, stores the result
% in |\childdocname| and saves a copy in |\childdocjob|:
%    \begin{macrocode}
\edef\childdocname{\scantokens\expandafter{\jobname\noexpand}}
\let\childdocjob\childdocname
%    \end{macrocode}

% \macro{\childdocdisable}
% The macro |\childdocdisable| prevents the main file
% from being processed more than once.
% At this stage, the main document command |\childdocmain|
% is assumed to be called once again where it should do nothing.
% Any subsequent call to it should prevent
% a secondary processing of the main document
% It overwrites the forwarding commands
% |\childdocof| and |\childdocforward|
% with empty macros to prevent further inclusions of the main document:
%    \begin{macrocode}
\newcommand{\childdocdisable}
{
  \renewcommand{\childdocmain}[1]{\renewcommand{\childdocmain}[1]{\endinput}}
  \renewcommand{\childdocof}[1]{}
  \renewcommand{\childdocby}[2][]{}
  \renewcommand{\childdocforward}[2][]{}
  \renewcommand{\childdocdisable}{}
}
%    \end{macrocode}

% \macro{\childdocmain}
% The macro |\childdocmain| is to be called at the top of the main file
% with nothing or the main filename (without extension) as argument.
% First, it breaks loops.
% If the argument is not empty and does not match |\childdocname|
% (which is set by the first inclusion of |childdoc.def|),
% |\ifchilddoc| is set to true, |\includeonly| is applied to the child file
% and |\jobname| is set to the main file
% (for proper handling of |.aux| files):
%    \begin{macrocode}
\newcommand{\childdocmain}[1]
{
  \childdocdisable\childdocmain{}
  \if?#1?\else
    \begingroup
      \def\childdoctmp{#1}
      \ifx\childdoctmp\childdocname
        \def\childdoctmp{}
      \else
        \def\childdoctmp
        {
          \childdoctrue
          \includeonly{\childdocname}
          \def\childdocjob{#1}
          \def\jobname{#1}
        }
      \fi
      \expandafter
    \endgroup
    \childdoctmp
  \fi
}
%    \end{macrocode}

% \macro{\childdocof}
% The command |\childdocof| redirects
% compilation to the main file |#1|.
%    \begin{macrocode}
\newcommand{\childdocof}[1]
{
  \childdocdisable
  \childdoctrue
  \includeonly{\childdocname}
  \def\jobname{#1}
  \def\childdocjob{#1}
  \input{#1}
}
%    \end{macrocode}

% \macro{\childdocby}
% The command |\childdocby| ....
%    \begin{macrocode}
\newcommand{\childdocby}[2][]
{
  \childdocdisable
  \childdoctrue
  \childdocmanualtrue
  \if?#1?\else
    \def\jobname{#2}
  \fi
  \def\childdocjob{#2}
  \input{#2}
  \endinput
}
%    \end{macrocode}

% \macro{\childdocforward}
% The command |\childdocforward| redirects
% compilation to the main file or
% (if the optional argument is given) a child file.
% Parameters are set as if the main file
% or a child file starting with |\childdocof| was compiled.
% Then compilation is handed over to the main file:
%    \begin{macrocode}
\newcommand{\childdocforward}[2][]
{
  \begingroup
    \if?#1?
      \def\childdoctmp
      {
        \def\childdocname{#2}
        \def\childdocjob{#2}
        \def\jobname{#2}
        \input{#2}
        \endinput
      }
    \else
      \def\childdoctmp
      {
        \childdocdisable
        \def\childdocname{#2}
        \childdoctrue
        \includeonly{#2}
        \def\childdocjob{#1}
        \def\jobname{#1}
        \input{#1}
        \endinput
      }
    \fi
    \expandafter
  \endgroup
  \childdoctmp
}
%    \end{macrocode}

% \macro{\childdocforwardprefix}
% The command |\childdocforwardprefix| redirects
% compilation to the main or a child file by means of a pattern.
% The prefix |#1| in the current filename is replaced by |#2|
% and the suffix of the current filename is kept
% (it is assumed that the filename does not contain the substring `|~~~|'
% which is used as a delimiter).
% Compilation is handed over to the new file by |\childdocforward|:
%    \begin{macrocode}
\newcommand{\childdocforwardprefix}[3][]
{
  \begingroup
    \def\childdocextract #2##1~~~{\def\childdoctmp{\childdocforward[#1]{#3##1}}}
    \expandafter\childdocextract\childdocname~~~
    \expandafter
  \endgroup
  \childdoctmp
}
%    \end{macrocode}

% \macro{\childdoc}
% The deprecated macro |\childdoc| is a legacy version of |\childdocmain|:
%    \begin{macrocode}
\newcommand{\childdoc}{\childdocmain}
%    \end{macrocode}

% \macro{\childdocredirect}
% The deprecated macro |\childdocredirect| is a legacy version
% of |\childdocforward| and |\childdocforwardprefix|:
%    \begin{macrocode}
\newcommand{\childdocredirect}[2][]
{
  \begingroup
    \if?#1?
      \def\childdoctmp{\childdocforward{#2}}
    \else
      \def\childdoctmp{\childdocforwardprefix{#1}{#2}}
    \fi
    \expandafter
  \endgroup
  \childdoctmp
}
%    \end{macrocode}

%\iffalse
%</package>
%\fi
%
\endinput
|\\
|\childdocby{|\textit{main}|}|\\
\end{tabular}
\end{center}
%
Both forms have slightly different effects as described above.
The main file is prepared as usual, see \secref{sec:include}.

%%%%%%%%%%%%%%%%%%%%%%%%%%%%%%%%%%%%%%%%%%%%%%%%%%%%%%%%%%%%%%%%%%%%%%%%%%%%%%%%
\subsection{Legacy Detection}
\label{sec:detection}

The directive |\childdocmain| in the main file can detect
whether the complete document or merely a child is to be compiled
even without using the directive |\childdocof|.
This method is deprecated because it is less robust
and there is no compelling reason to use it;
it is merely provided for backward compatibility
and it may be removed in future versions.

If the detection mechanism is to be used,
it is mandatory to correctly specify
the filename of the main file as the argument of |\childdocmain|:
%
\begin{center}
\begin{tabular}{l}
|% \iffalse
%
% childdoc.dtx Copyright (C) 2017-2018 Niklas Beisert
%
% This work may be distributed and/or modified under the
% conditions of the LaTeX Project Public License, either version 1.3
% of this license or (at your option) any later version.
% The latest version of this license is in
%   http://www.latex-project.org/lppl.txt
% and version 1.3 or later is part of all distributions of LaTeX
% version 2005/12/01 or later.
%
% This work has the LPPL maintenance status `maintained'.
%
% The Current Maintainer of this work is Niklas Beisert.
%
% This work consists of the files childdoc.dtx and childdoc.ins
% and the derived files childdoc.def and cdocsamp.tex with
% cdocsch1.tex, cdocsch2.tex, cdocsdrf.tex, cdocsfn1.tex, cdocsfn2.tex.
%
%<package>\ifdefined\childdocmain\endinput\fi
%<package>\ProvidesFile{childdoc.def}[2018/12/30 v2.0 child document driver]
%<samplemain>\ProvidesFile{cdocsamp.tex}[2018/12/30 v2.0 sample for childdoc]
%<*driver>
%\ProvidesFile{childdoc.drv}[2018/12/30 v2.0 childdoc reference manual file]
\PassOptionsToClass{10pt,a4paper}{article}
\documentclass{ltxdoc}

\usepackage[margin=35mm]{geometry}
\usepackage{hyperref}
\usepackage{hyperxmp}
\usepackage[usenames]{color}

\hypersetup{colorlinks=true}
\hypersetup{pdfstartview=FitH}
\hypersetup{pdfpagemode=UseNone}
\hypersetup{pdfsource={}}
\hypersetup{pdflang={en-UK}}
\hypersetup{pdfcopyright={Copyright 2017-2018 Niklas Beisert.
  This work may be distributed and/or modified under the
  conditions of the LaTeX Project Public License, either version 1.3
  of this license or (at your option) any later version.}}
\hypersetup{pdflicenseurl={http://www.latex-project.org/lppl.txt}}
\hypersetup{pdfcontactaddress={ETH Zurich, ITP, HIT K,
  Wolfgang-Pauli-Strasse 27}}
\hypersetup{pdfcontactpostcode={8093}}
\hypersetup{pdfcontactcity={Zurich}}
\hypersetup{pdfcontactcountry={Switzerland}}
\hypersetup{pdfcontactemail={nbeisert@itp.phys.ethz.ch}}
\hypersetup{pdfcontacturl={http://people.phys.ethz.ch/\xmptilde nbeisert/}}

\newcommand{\secref}[1]{\hyperref[#1]{section \ref*{#1}}}

\parskip1ex
\parindent0pt
\let\olditemize\itemize
\def\itemize{\olditemize\parskip0pt}

\begin{document}

\title{The \textsf{childdoc} Package}
\hypersetup{pdftitle={The childdoc Package}}
\author{Niklas Beisert\\[2ex]
  Institut f\"ur Theoretische Physik\\
  Eidgen\"ossische Technische Hochschule Z\"urich\\
  Wolfgang-Pauli-Strasse 27, 8093 Z\"urich, Switzerland\\[1ex]
  \href{mailto:nbeisert@itp.phys.ethz.ch}
  {\texttt{nbeisert@itp.phys.ethz.ch}}}
\hypersetup{pdfauthor={Niklas Beisert}}
\hypersetup{pdfsubject={Manual for the LaTeX2e Package childdoc}}
\date{30 December 2018, \textsf{v2.0}}
\maketitle

\begin{abstract}\noindent
\textsf{childdoc} is a \LaTeXe{} package
that enables the direct compilation
of document sections included by |\include|
to individual files.
\end{abstract}

\begingroup
\parskip0ex
\tableofcontents
\endgroup

%%%%%%%%%%%%%%%%%%%%%%%%%%%%%%%%%%%%%%%%%%%%%%%%%%%%%%%%%%%%%%%%%%%%%%%%%%%%%%%%
%%%%%%%%%%%%%%%%%%%%%%%%%%%%%%%%%%%%%%%%%%%%%%%%%%%%%%%%%%%%%%%%%%%%%%%%%%%%%%%%
\section{Introduction}

\LaTeX{} provides a mechanism to structure a large document (such as a book)
into a main file and several child files (containing the chapters)
using the |\include| command.
This mechanism is beneficial for documents
which span hundreds of pages in order to
make the source file(s) more manageable.
Moreover, compilation can be restricted to
selected child files by means of the |\includeonly| command.
The latter feature can be used to reduce the compilation time while editing
(this was significantly more useful in the earlier days of \LaTeX{})
or to generate a smaller document which is easier to navigate.
Another application of |\includeonly| is to generate
documents consisting of selected parts of the complete document.

However, there are a few drawbacks of the plain |\include| mechanism:
\begin{itemize}
\item
The child files cannot be compiled on their own,
they can only be compiled via the main file.
A naive editing environment
(such as a text editor with an option
to have the current file processed by \LaTeX)
may require one to switch to the main file before compiling;
attempting to compile the child file produces errors.
\item
The main file must be modified (each time)
to adjust the |\includeonly| command
to the present needs. This easily leaves the main file in a messy state.
\item
The generated document will always carry the filename
of the main document. This is inconvenient if
several child files are to be compiled and
to be kept for distribution.
\end{itemize}

The present package provides a simple interface
to make child files individually compilable by \LaTeX{}.
Compiling a child file then has the same effect as compiling
the main file with an |\includeonly| command
to select the appropriate child.
Moreover the generated document will carry the name of the child
rather than the main file.
This resolves all three above issues.

This feature is meant to make the editing of books,
thesis documents and lecture notes somewhat more convenient.
However, the package can also be used efficiently for
composing a series of documents (such as exercise sheets)
which are typically distributed individually.
It then assists the author in generating the individual documents
(potentially in different versions)
as well as a document containing the collected series.
Another application is in developing style files
or other kinds of included material
where compilation of the style file could redirect
to a sample or test file.

%%%%%%%%%%%%%%%%%%%%%%%%%%%%%%%%%%%%%%%%%%%%%%%%%%%%%%%%%%%%%%%%%%%%%%%%%%%%%%%%
%%%%%%%%%%%%%%%%%%%%%%%%%%%%%%%%%%%%%%%%%%%%%%%%%%%%%%%%%%%%%%%%%%%%%%%%%%%%%%%%
\section{Usage}

First of all, the package \textsf{childdoc} is \emph{not} a standard
\LaTeXe{} |.sty| style file! Therefore it needs to be invoked in
a non-standard way.

%%%%%%%%%%%%%%%%%%%%%%%%%%%%%%%%%%%%%%%%%%%%%%%%%%%%%%%%%%%%%%%%%%%%%%%%%%%%%%%%
\subsection{Included Files}
\label{sec:include}

%%%%%%%%%%%%%%%%%%%%%%%%%%%%%%%%%%%%%%%%
\DescribeMacro{\childdocmain}
To use the package, add the commands
\begin{center}
\begin{tabular}{l}
|\input{childdoc.def}|\\
|\childdocmain{}|\\
\end{tabular}
\end{center}
at the very top of the main \LaTeX{} file,
in particular \emph{before} the |\documentclass| statement!
The argument of |\childdocmain| should be left empty
(but it must be present).

%%%%%%%%%%%%%%%%%%%%%%%%%%%%%%%%%%%%%%%%
\DescribeMacro{\childdocof}
Furthermore, add the commands
\begin{center}
\begin{tabular}{l}
|\input{childdoc.def}|\\
|\childdocof{|\textit{main}|}|\\
\end{tabular}
\end{center}
at the top of every child file \textit{child}
which is included by |\include{|\textit{child}|}|
from within the main file
(or at least for those files to be compiled individually).
The argument \textit{main} must be the filename of the main file.

There are a couple of
considerations in setting up the main and child documents:

%%%%%%%%%%%%%%%%%%%%%%%%%%%%%%%%%%%%%%%%
\paragraph{Restrictions.}

Please note the following restrictions:
\begin{itemize}
\item
|\childdocmain| must be called with one argument \textit{main}
to ensure compatibility with earlier version of the package.
It must either be empty (|\childdocmain{}|)
or precisely match the filename of the main file in which it is specified.
See \secref{sec:detection} for further information.
\item
The filename \textit{main} must be specified without the |.tex| extension.
\item
The filename \textit{main} is case sensitive
(even in case-insensitive file systems)
due to internal string comparison.
\item
The argument \textit{main} should be fully expanded, it cannot be a macro.
\item
Subdirectories and special characters should be avoided in filenames.
\item
The command |\childdocmain{|\textit{main}|}| must be followed by a whitespace.
It should not be followed immediately by another command
or by a comment mark `|%|'.
This is because the \TeX{} parser reads the token immediately following
the argument of |\childdocmain| and puts it
at the beginning of every child section;
however, a white\-space is ignored.
\end{itemize}

%%%%%%%%%%%%%%%%%%%%%%%%%%%%%%%%%%%%%%%%
\paragraph{Content of Main File.}

It is advisable to place all content in the child files included by |\include|.
Any output contained in the main file will appear in all child documents
unless suppressed manually;
it cannot be suppressed automatically by the |\includeonly| directive
and thus should normally be avoided.
A method to include some content in the main file
by means of conditional processing is described in \secref{sec:conditional}.

%%%%%%%%%%%%%%%%%%%%%%%%%%%%%%%%%%%%%%%%
\paragraph{Page Numbering.}

When only a part of the document is compiled,
the appropriate numbering of pages
(as well as other status parameters)
is determined from the |.aux| files.
The latter contain information from previous passes.
However this information needs to propagate through
all intermediate child documents.
Therefore the page numbering in child documents may well
be inconsistent until the complete document is compiled at least once.

A useful (if unconventional) way to always ensure a consistent
page numbering is to restart the numbering in each child document
and denote the pages by `\textit{child}|.|\textit{page}'
where \textit{child} represents the chapter/section number of the child file.
This can be achieved by the command
|\numberwithin{page}{|\textit{child}|}|
of the \textsf{amsmath} package
where \textit{child} can be |chapter| or |section|
depending on the chosen structuring.
Alternatively, one can modify the macro |\thepage| appropriately
and reset the counter |page| at the start of each child file.

%%%%%%%%%%%%%%%%%%%%%%%%%%%%%%%%%%%%%%%%%%%%%%%%%%%%%%%%%%%%%%%%%%%%%%%%%%%%%%%%
\subsection{Conditional Processing}
\label{sec:conditional}

The package provides a mechanism to compile different versions
of a document. To customise the versions further some conditional processing
can come in handy to distinguish which version is being compiled.
The package provides two macros to describe the compilation context:

%%%%%%%%%%%%%%%%%%%%%%%%%%%%%%%%%%%%%%%%
\DescribeMacro{\ifchilddoc}
The conditional |\ifchilddoc| distinguishes between the compilation of
child documents and the main document:
%
\begin{center}
|\ifchilddoc |\textit{child-code}| |[|\||else |\textit{main-code}]| \||fi|
\end{center}

%%%%%%%%%%%%%%%%%%%%%%%%%%%%%%%%%%%%%%%%
\DescribeMacro{\childdocname}
\DescribeMacro{\childdocjob}
The macro |\childdocname| contains the filename (without extension)
of the main or child file being processed.
Note that |\childdocjob| will always contain the name of the main file.

%%%%%%%%%%%%%%%%%%%%%%%%%%%%%%%%%%%%%%%%
\paragraph{Title Page.}

Conditional processing can be used to include a title or banner page
in the main document when proper precautions are taken.
Importantly, the code in the main file should ensure that the page counter
(as well as other status parameters which are stored in the |.aux| files)
takes the same value after the conditional processing.
Otherwise the page numbers may take divergent values
depending on which part is compiled.

For example, a title page could be declared by:
%
\begin{center}
\begin{tabular}{l}
|\ifchilddoc\||else|\\
|\addtocounter{page}{-1}|\\
\textit{code for title page}\\
|\newpage|\\
|\||fi|
\end{tabular}
\end{center}
%
A banner page for the child documents can be generated by:
%
\begin{center}
\begin{tabular}{l}
|\ifchilddoc|\\
|\addtocounter{page}{-1}|\\
\textit{code for banner page}\\
|\newpage|\\
|\||fi|
\end{tabular}
\end{center}
%
Here one could write a message such as:
\begin{center}
|This is the part \childdocname{} of \childdocjob{}.|
\end{center}

%%%%%%%%%%%%%%%%%%%%%%%%%%%%%%%%%%%%%%%%%%%%%%%%%%%%%%%%%%%%%%%%%%%%%%%%%%%%%%%%
\subsection{Flags}
\label{sec:flags}

The package makes it easy to generate different versions
of the main or child documents.
To this end compilation flags can be defined
and assigned different default values.
They will be particularly useful in conjunction
with the forwarding mechanism described in \secref{sec:forward}.

For example, it may be useful to have a flag |\version|
which can be set to |draft| or |final|.
The document source will contain some conditional code
depending on the value of |\version|.
Suppose further, the flag should default to |final| for the main file
and to |draft| for child files
which is a natural assignment for editing the document.
This is achieved by placing the following code
in the preamble of the main document
(below the |\childdocmain| directive):
%
\begin{center}
\begin{tabular}{l}
|\ifchilddoc|\\
|\providecommand{\version}{draft}|\\
|\||else|\\
|\providecommand{\version}{final}|\\
|\||fi|
\end{tabular}
\end{center}
%
The definition by |\providecommand| makes sure
that previous definitions are not overwritten.
Further statements |\providecommand{\version}{...}|
can thus be added before the above code to override it.

For the main file, one might add a line
(between |\childdocmain| and the above block)
%
\begin{center}
|%\ifchilddoc\||else\providecommand{\version}{draft}\||fi|
\end{center}
%
which can be uncommented to produce a draft version.
Likewise one can add a line to the very top of a child file
(above the |\childdocof{|\textit{main}|}| directive)
%
\begin{center}
|%\providecommand{\version}{final}|
\end{center}
%
which can be uncommented to produce the final version of this child document.

%%%%%%%%%%%%%%%%%%%%%%%%%%%%%%%%%%%%%%%%%%%%%%%%%%%%%%%%%%%%%%%%%%%%%%%%%%%%%%%%
\subsection{Forwarding}
\label{sec:forward}

Different versions of the main or child documents
using compilation flags as described in \secref{sec:flags}
can be (permanently) stored in different files
for convenient compilation, viewing and distribution.
To this end, the package defines a command
to pass on compilation to a different file:

%%%%%%%%%%%%%%%%%%%%%%%%%%%%%%%%%%%%%%%%
\DescribeMacro{\childdocforward}
The command |\childdocforward| redirects processing to
another source file:
%
\begin{center}
\begin{tabular}{l}
|\input{childdoc.def}|\\
|\childdocforward[|\textit{main}|]{|\textit{dest}|}|\\
\end{tabular}
\end{center}
%
The argument \textit{dest} is the destination file
(without extension).
It should be the main file or one of the child files.
Note that further \textsf{childdoc} directives
such as |\childdocof| and |\childdocforward|
in the indicated file will be processed in this form.
The optional argument \textit{main}
passes on directly to the main file \textit{main}
while pretending to compile the child \textit{dest}.
This form behaves as if \textit{dest}
issues |\childdocof{|\textit{main}|}| right away,
and no further \textsf{childdoc} directives will be processed.

%%%%%%%%%%%%%%%%%%%%%%%%%%%%%%%%%%%%%%%%
\DescribeMacro{\...prefix}
In the alternative form |\childdocforwardprefix|,
%
\begin{center}
\begin{tabular}{l}
|\input{childdoc.def}|\\
|\childdocforwardprefix[|\textit{main}|]{|\textit{prefix}|}{|\textit{dest}|}|
\end{tabular}
\end{center}
%
the destination file is determined by a pattern
depending on the current file:
To make this work, the current file must be called
`{\textit{prefix}\hspace{0.2em}\textit{suffix}}'
with \textit{prefix} matching precisely the argument.
Processing is then passed on to the file
`{\textit{dest}\hspace{0.2em}\textit{suffix}}'.
Surely, the same effect is achieved by
directly specifying the
argument `{\textit{dest}\hspace{0.2em}\textit{suffix}}'
in the first form.
However, that requires to set up a different file
for each child. With the alternative form of the command
all these files can have exactly the same content
which simplifies setting them up and maintaining them.

For example, the following file |draft.tex|
with a compilation flag |\version| as described in \secref{sec:flags}
compiles the main document as a draft:
%
\begin{center}
\begin{tabular}{l}
|\def\version{draft}|\\
|\input{childdoc.def}|\\
|\childdocforward{|\textit{main}|}|
\end{tabular}
\end{center}
%
Likewise, the following files |final|\textit{nn}|.tex|
compile the final version of the child document
|child|\textit{nn}|.tex|:
%
\begin{center}
\begin{tabular}{l}
|\def\version{final}|\\
|\input{childdoc.def}|\\
|\childdocforwardprefix{final}{child}|
\end{tabular}
\end{center}
%

Note that when several versions of a main file and/or of each child file
are to be generated, it may be convenient to set up a |Makefile| or
shell script to automatise the process.

%%%%%%%%%%%%%%%%%%%%%%%%%%%%%%%%%%%%%%%%%%%%%%%%%%%%%%%%%%%%%%%%%%%%%%%%%%%%%%%%
\subsection{Command Line Processing}
\label{sec:commandline}

The effect of redirection files can also be achieved by invoking
the \LaTeX{} compiler with a more elaborate command line.
Most conveniently this should be done as part
of a shell script or a |Makefile|.

When using \textsf{childdoc} in the main file, the following
command lines effectively perform a redirection
(note that depending on the shell being used,
backslashes may have to be doubled: `|\|' $\to$ `|\\|'):
%
\begin{center}
|... -jobname "|\textit{target}|" |\\|"|[\textit{flags}]%
|\input{childdoc.def}\childdocforward[|\textit{main}|]{|\textit{dest}|}"|
\end{center}
%
Here \textit{target} is the name of the output file,
\textit{main} is the name of the main file
and \textit{dest} is the name of the main or child file to be processed
(all filenames without extensions).
The optional argument \textit{main} can be omitted
if \textit{main} matches \textit{dest}.
Optionally, compilation \textit{flags} can be defined via |\def| commands.
This command line makes the \TeX{} engine believe
it is compiling the file \textit{target}
whose content is specified as the latter parameter.
The provided code then forwards the processing to
\textit{main} or \textit{dest} as described in \secref{sec:forward}.

%%%%%%%%%%%%%%%%%%%%%%%%%%%%%%%%%%%%%%%%%%%%%%%%%%%%%%%%%%%%%%%%%%%%%%%%%%%%%%%%
\subsection{Include by Input}
\label{sec:input}

Including child documents by |\include| has some restrictions by design.
Most notably, the content of a child document always occupies
its own set of pages; pages cannot be shared between child documents.
Usually, this behaviour makes perfect sense
because each child document contain an essential part of the document.
However, in some situations it may be desirable to compose
a document from a collection of parts
without having mandatory page breaks between then.
For this case, the package
provides a mechanism to include parts
by |\input| which can also be processed individually.
However, by construction this mechanism
requires manual handling of the content to be output.

%%%%%%%%%%%%%%%%%%%%%%%%%%%%%%%%%%%%%%%%
\DescribeMacro{\ifchilddocmanual}
The main file should be prepared as usual, see \secref{sec:include}.
However, the document body must make a distinction
between processing of an individual part and of the main document, e.g.:
%
\begin{center}
\begin{tabular}{l}
|\ifchilddocmanual|\\
|\input{\childdocname}|\\
|\||else|\\
\textit{document body with }|\input{|\textit{part}|}|\\
|\||fi|
\end{tabular}
\end{center}
%
The conditional |\ifchilddocmanual| is true whenever
a part to be included by |\input| is being compiled,
and the name of the part is stored in |\childdocname|.

%%%%%%%%%%%%%%%%%%%%%%%%%%%%%%%%%%%%%%%%
\DescribeMacro{\childdocby}
Each part to be included by |\input| should start with:
%
\begin{center}
\begin{tabular}{l}
|\input{childdoc.def}|\\
|\childdocby{|\textit{main}|}|\\
\end{tabular}
\end{center}
%
The directive |\childdocby| is similar to |\childdocof|
described in \secref{sec:include},
but the subsequent selection of content must be done manually.
To that end, both |\ifchilddoc| and |\ifchilddocmanual|
will be true upon processing of a part,
and the name of the part is stored in |\childdocname|.
Note that |\jobname| will be set to the filename of the current part
so that each part receives an individual |.aux| file
that does not interfere with the |.aux| file(s) of the main document.
This behaviour can be altered by the alternative form
|\childdocby[*]{|\textit{main}|}| (with a non-empty optional argument)
which uses the |.aux| file of the main document
by setting |\jobname| to \textit{main}.

%%%%%%%%%%%%%%%%%%%%%%%%%%%%%%%%%%%%%%%%%%%%%%%%%%%%%%%%%%%%%%%%%%%%%%%%%%%%%%%%
\subsection{Driver Development}
\label{sec:driver}

The \textsf{childdoc} mechanism can also be use for the development
of definition files such as \LaTeX{} styles or classes.
This case differs from the above setup with multiple parts
included by |\include| in that no |\includeonly| should be invoked.
This can be achieved by starting the include file
(before |\ProvidesPackage|) with:
%
\begin{center}
\begin{tabular}{l}
|\input{childdoc.def}|\\
|\childdocforward{|\textit{main}|}|\\
\end{tabular}
\end{center}
%
or alternatively with:
%
\begin{center}
\begin{tabular}{l}
|\input{childdoc.def}|\\
|\childdocby{|\textit{main}|}|\\
\end{tabular}
\end{center}
%
Both forms have slightly different effects as described above.
The main file is prepared as usual, see \secref{sec:include}.

%%%%%%%%%%%%%%%%%%%%%%%%%%%%%%%%%%%%%%%%%%%%%%%%%%%%%%%%%%%%%%%%%%%%%%%%%%%%%%%%
\subsection{Legacy Detection}
\label{sec:detection}

The directive |\childdocmain| in the main file can detect
whether the complete document or merely a child is to be compiled
even without using the directive |\childdocof|.
This method is deprecated because it is less robust
and there is no compelling reason to use it;
it is merely provided for backward compatibility
and it may be removed in future versions.

If the detection mechanism is to be used,
it is mandatory to correctly specify
the filename of the main file as the argument of |\childdocmain|:
%
\begin{center}
\begin{tabular}{l}
|\input{childdoc.def}|\\
|\childdocmain{|\textit{main}|}|\\
\end{tabular}
\end{center}
%
If |\jobname| does not match the argument \textit{main} of |\childdocmain|,
it is assumed that |\jobname| points to the child file to be compiled.
When using |\childdocmain| with the main file specified as argument,
it suffices to start a child file
with just |\input{|\textit{main}|}|
without loading of the package and using |\childdocof|.
If instead all processing is done
with the appropriate \textsf{childdoc} directives,
the argument of \textit{main} of |\childdocmain| can be empty.

An alternative version of the command line processing described
in \secref{sec:commandline} using the detection mechanism reads:
%
\begin{center}
|... -jobname "|\textit{target}|" "|[\textit{flags}]%
[|\def\jobname{|\textit{dest}|}|]|\input{|\textit{main}|}"|
\end{center}

%%%%%%%%%%%%%%%%%%%%%%%%%%%%%%%%%%%%%%%%%%%%%%%%%%%%%%%%%%%%%%%%%%%%%%%%%%%%%%%%
\subsection{Manual Code}
\label{sec:manual}

In case one cannot be certain whether the definitions file |childdoc.def|
is installed on the target \TeX{} distribution
and one prefers not to ship it,
it is conceivable to paste a few relevant commands into the sources.

To that end, drop all statements |\input{childdoc.def}|
and perform the replacements as outlined below.
Instead of |\childdocmain{|\textit{main}|}| add the following code
to the top of the main file:
%
\begin{center}
\begin{tabular}{l}
|\||ifdefined\childdocname\endinput\||fi\newif\ifchilddoc|\\
|\edef\childdocname{\scantokens\expandafter{\jobname\noexpand}}|\\
|\def\childdocmain{|\textit{main}|}\||ifx\childdocmain\childdocname\||else|\\
|\childdoctrue\includeonly{\childdocname}\let\jobname\childdocmain\||fi|\\
\end{tabular}
\end{center}
%
Instead of |\childdocof{|\textit{main}|}| just include the main file
at the top of each child file:
%
\begin{center}
|\input{|\textit{main}|}|
\end{center}
%
A simple redirection |\childdocforward{|\textit{dest}|}| is achieved by:
%
\begin{center}
|\def\jobname{|\textit{dest}|}\input{\jobname}|
\end{center}
%
The redirection with prefix
|\childdocforwardprefix[|\textit{prefix}|]{|\textit{dest}|}|
is accomplished by:
%
\begin{center}
\begin{tabular}{l}
|{\edef\jobname{\scantokens\expandafter{\jobname\noexpand}}|\\
|\def\redirectjob |\textit{prefix}|#1~~~{\gdef\jobname{|\textit{dest}|#1}}|\\
|\expandafter\redirectjob\jobname~~~}\input{\jobname}|
\end{tabular}
\end{center}

In an alternative approach,
child documents can be compiled by a specific command line
without additional code or specific definitions:
%
\begin{center}
|... -jobname "|\textit{target}|" "|[\textit{flags}]%
|\includeonly{|\textit{dest}|}\input{|\textit{main}|}"|
\end{center}
%

%%%%%%%%%%%%%%%%%%%%%%%%%%%%%%%%%%%%%%%%%%%%%%%%%%%%%%%%%%%%%%%%%%%%%%%%%%%%%%%%
%%%%%%%%%%%%%%%%%%%%%%%%%%%%%%%%%%%%%%%%%%%%%%%%%%%%%%%%%%%%%%%%%%%%%%%%%%%%%%%%
\section{Information}

%%%%%%%%%%%%%%%%%%%%%%%%%%%%%%%%%%%%%%%%%%%%%%%%%%%%%%%%%%%%%%%%%%%%%%%%%%%%%%%%
\subsection{Copyright}

Copyright \copyright{} 2017--2018 Niklas Beisert

This work may be distributed and/or modified under the
conditions of the \LaTeX{} Project Public License, either version 1.3
of this license or (at your option) any later version.
The latest version of this license is in
  \url{http://www.latex-project.org/lppl.txt}
and version 1.3 or later is part of all distributions of \LaTeX{}
version 2005/12/01 or later.

This work has the LPPL maintenance status `maintained'.

The Current Maintainer of this work is Niklas Beisert.

This work consists of the files |README.txt|, |childdoc.ins| and |childdoc.dtx|
as well as the derived files |childdoc.def|, |cdocsamp.tex|
with |cdocsch1.tex|, |cdocsch2.tex|, |cdocspt3.tex|, |cdocspt4.tex|,
|cdocsdrf.tex|, |cdocsfn1.tex|, |cdocsfn2.tex|
as well as |childdoc.pdf|.

%%%%%%%%%%%%%%%%%%%%%%%%%%%%%%%%%%%%%%%%%%%%%%%%%%%%%%%%%%%%%%%%%%%%%%%%%%%%%%%%
\subsection{Files and Installation}

The package consists of the files:
%
\begin{center}
\begin{tabular}{ll}
    |README.txt|   & readme file \\
    |childdoc.ins| & installation file \\
    |childdoc.dtx| & source file \\
    |childdoc.def| & definition file \\
    |cdocsamp.tex| & sample main file \\
    |cdocsch1.tex| & sample include file \\
    |cdocsch2.tex| & sample include file \\
    |cdocspt3.tex| & sample part file \\
    |cdocspt4.tex| & sample part file \\
    |cdocsdrf.tex| & sample redirection file \\
    |cdocsfn1.tex| & sample redirection file \\
    |cdocsfn2.tex| & sample redirection file \\
    |childdoc.pdf| & manual
\end{tabular}
\end{center}
%
The distribution consists of the files
|README.txt|, |childdoc.ins| and |childdoc.dtx|.
%
\begin{itemize}
\item
Run (pdf)\LaTeX{} on |childdoc.dtx|
to compile the manual |childdoc.pdf| (this file).
\item
Run \LaTeX{} on |childdoc.ins| to create the definitions file |childdoc.def|
and the sample |cdocsamp.tex| with include files
|cdocsch1.tex|, |cdocsch2.tex|, |cdocspt3.tex|, |cdocspt4.tex|,
|cdocsdrf.tex|, |cdocsfn1.tex|, |cdocsfn2.tex|.
Then copy the file |childdoc.def| to an appropriate directory of your \LaTeX{}
distribution, e.g.\ \textit{texmf-root}|/tex/latex/childdoc|.
\end{itemize}

%%%%%%%%%%%%%%%%%%%%%%%%%%%%%%%%%%%%%%%%%%%%%%%%%%%%%%%%%%%%%%%%%%%%%%%%%%%%%%%%
\subsection{Related CTAN Packages}

There are several other packages which offer a similar functionality:
%
\begin{itemize}
\item
The packages
\href{http://ctan.org/pkg/docmute}{\textsf{docmute}},
\href{http://ctan.org/pkg/includex}{\textsf{includex}} and
\href{http://ctan.org/pkg/standalone}{\textsf{standalone}}
provide commands to include only the document body of
a child file thus allowing both files to be compiled individually.
\item
The packages \href{http://ctan.org/pkg/subdocs}{\textsf{subdocs}}
and \href{http://ctan.org/pkg/subfiles}{\textsf{subfiles}}
provide structures in which the main and child documents can be
encapsulated and allowing them to be compiled individually.
The inclusion mechanism is different from the conventional |\include|.
\item
The package \href{http://ctan.org/pkg/combine}{\textsf{combine}}
is an elaborate solution to combine several documents into one.
\end{itemize}
%
See also the CTAN topic \href{http://ctan.org/topic/subdocs}{\textsf{subdocs}}
for further related packages.
The present package differs from the above solutions in that
a document structure constructed with the conventional |\include| mechanism
just needs two extra commands at the top of every file
such that all constituent files can be compiled individually.

%%%%%%%%%%%%%%%%%%%%%%%%%%%%%%%%%%%%%%%%%%%%%%%%%%%%%%%%%%%%%%%%%%%%%%%%%%%%%%%%
%\subsection{Feature Suggestions}
%
%The following is a list of features which may be useful for future
%versions of this package:
%%
%\begin{itemize}
%\item
%\ldots
%\end{itemize}

%%%%%%%%%%%%%%%%%%%%%%%%%%%%%%%%%%%%%%%%%%%%%%%%%%%%%%%%%%%%%%%%%%%%%%%%%%%%%%%%
\subsection{Revision History}

%%%%%%%%%%%%%%%%%%%%%%%%%%%%%%%%%%%%%%%%
\paragraph{v2.0:} 2018/12/30

\begin{itemize}
\item
immediate forward processing
\item
added |\childdocby| mechanism
\item
manual restructured
\end{itemize}

%%%%%%%%%%%%%%%%%%%%%%%%%%%%%%%%%%%%%%%%
\paragraph{v1.6:} 2018/01/17

\begin{itemize}
\item
application for development of include files
\item
corrections to manual
\end{itemize}

%%%%%%%%%%%%%%%%%%%%%%%%%%%%%%%%%%%%%%%%
\paragraph{v1.5:} 2017/05/21

\begin{itemize}
\item
more complete structuring introduced
\item
|\childdocof| introduced
\item
|\childdoc| renamed to |\childdocmain|
\item
|\childredirect| renamed to |\childdocforward| and |\childdocforwardprefix|
and functionality expanded
\end{itemize}

%%%%%%%%%%%%%%%%%%%%%%%%%%%%%%%%%%%%%%%%
\paragraph{v1.0:} 2017/04/27

\begin{itemize}
\item
manual and install package
\item
first version published on CTAN
\end{itemize}

%%%%%%%%%%%%%%%%%%%%%%%%%%%%%%%%%%%%%%%%
\paragraph{v0.6:} 2017/04/26

\begin{itemize}
\item
redirection mechanism added
\end{itemize}

%%%%%%%%%%%%%%%%%%%%%%%%%%%%%%%%%%%%%%%%
\paragraph{v0.5:} 2017/04/26

\begin{itemize}
\item
functionality in definition file
\end{itemize}


%%%%%%%%%%%%%%%%%%%%%%%%%%%%%%%%%%%%%%%%%%%%%%%%%%%%%%%%%%%%%%%%%%%%%%%%%%%%%%%%
%%%%%%%%%%%%%%%%%%%%%%%%%%%%%%%%%%%%%%%%%%%%%%%%%%%%%%%%%%%%%%%%%%%%%%%%%%%%%%%%
%%%%%%%%%%%%%%%%%%%%%%%%%%%%%%%%%%%%%%%%%%%%%%%%%%%%%%%%%%%%%%%%%%%%%%%%%%%%%%%%
\appendix

\settowidth\MacroIndent{\rmfamily\scriptsize 000\ }

 \DocInput{childdoc.dtx}

\end{document}
%</driver>
% \fi
%
% %%%%%%%%%%%%%%%%%%%%%%%%%%%%%%%%%%%%%%%%%%%%%%%%%%%%%%%%%%%%%%%%%%%%%%%%%%%%%%
% %%%%%%%%%%%%%%%%%%%%%%%%%%%%%%%%%%%%%%%%%%%%%%%%%%%%%%%%%%%%%%%%%%%%%%%%%%%%%%
% \section{Sample}
%\iffalse
%<*samplemain>
%\fi
%
% The following presents a sample document
% with two chapters, two parts, a title page,
% a compile flag as well as three forwarding files to set the flag.
% It consists of eight |.tex| files:
% \begin{center}
% \begin{tabular}{ll}
% |cdocsamp.tex|&main file\\
% |cdocsch1.tex|&include file for chapter 1\\
% |cdocsch2.tex|&include file for chapter 2\\
% |cdocspt3.tex|&include file for part 3\\
% |cdocspt4.tex|&include file for part 4\\
% |cdocsdrf.tex|&forwarding file for main file in draft mode\\
% |cdocsfi1.tex|&forwarding file for final version of chapter 1\\
% |cdocsfi2.tex|&forwarding file for final version of chapter 2\\
% \end{tabular}
% \end{center}
% Each of the eight files can be compiled directly by the \LaTeX{} compiler.
%
% %%%%%%%%%%%%%%%%%%%%%%%%%%%%%%%%%%%%%%
% \paragraph{Main File.}
%
% The main file is called |cdocsamp.tex|.
%
% Load the \textsf{childdoc} definitions and
% declare the filename for the main document:
%    \begin{macrocode}
\input{childdoc.def}
\childdocmain{}
%    \end{macrocode}

% Optional override for |\version| flag:
%    \begin{macrocode}
%%\ifchilddoc\else\providecommand{\version}{draft}\fi
%    \end{macrocode}

% Define the default values for the |\version| flag
% (|final| for the main file and |draft| for childs):
%    \begin{macrocode}
\ifchilddoc
\providecommand{\version}{draft}
\else
\providecommand{\version}{final}
\fi
%    \end{macrocode}

% Load the standard document class:
%    \begin{macrocode}
\documentclass[12pt]{article}
%    \end{macrocode}

% Start the document body:
%    \begin{macrocode}
\begin{document}
%    \end{macrocode}

% Declare a title page.
% Print title, part of document being processed and version flag:
%    \begin{macrocode}
\addtocounter{page}{-1}
\begin{center}
{\LARGE\bfseries{}childdoc example\par}
\vspace{1cm}
\ifchilddoc
\ifchilddocmanual part\else chapter\fi:
`\childdocname' of `\childdocjob'\par
\else
main document: `\childdocjob'\par
\fi
version: \version\par
\end{center}
\newpage
%    \end{macrocode}

% Manually include selected file,
% otherwise process as usual:
%    \begin{macrocode}
\ifchilddocmanual
\section*{part `\childdocname'}
\input{\childdocname}
\else
%    \end{macrocode}

% Include the two chapters:
%    \begin{macrocode}
\include{cdocsch1}
\include{cdocsch2}
%    \end{macrocode}

% Include the two parts unless only chapters should be displayed:
%    \begin{macrocode}
\ifchilddoc\else
\section{part three}
\input{cdocspt3}
\section{part four}
\input{cdocspt4}
\fi
%    \end{macrocode}

% Process as usual until here:
%    \begin{macrocode}
\fi
%    \end{macrocode}

% End of document body:
%    \begin{macrocode}
\end{document}
%    \end{macrocode}
%\iffalse
%</samplemain>
%\fi
%
% %%%%%%%%%%%%%%%%%%%%%%%%%%%%%%%%%%%%%%
% \paragraph{Chapter Include Files.}
%
% The include files are called |cdocsch1.tex| and |cdocsch2.tex|.
%
%\iffalse
%<*samplechap1|samplechap2>
%\fi

% Optional override for |\version| flag:
%    \begin{macrocode}
%%\providecommand{\version}{final}
%    \end{macrocode}

% Include the main document:
%    \begin{macrocode}
\input{childdoc.def}
\childdocof{cdocsamp}
%    \end{macrocode}

%\iffalse
%</samplechap1|samplechap2>
%\fi
%
%\iffalse
%<*samplechap1>
%\fi
% Some text for chapter 1:
%    \begin{macrocode}
\section{one}
some text in chapter one
%    \end{macrocode}

%\iffalse
%</samplechap1>
%\fi
% Some text for chapter 2:
%\iffalse
%<*samplechap2>
%\fi
%    \begin{macrocode}
\section{two}
more text in chapter two
%    \end{macrocode}

%\iffalse
%</samplechap2>
%\fi
%
% %%%%%%%%%%%%%%%%%%%%%%%%%%%%%%%%%%%%%%
% \paragraph{Part Include Files.}
%
% The include files are called |cdocspt3.tex| and |cdocspt4.tex|.
%
%\iffalse
%<*samplepart3|samplepart4>
%\fi

% Optional override for |\version| flag:
%    \begin{macrocode}
%%\providecommand{\version}{final}
%    \end{macrocode}

% Include the main document:
%    \begin{macrocode}
\input{childdoc.def}
\childdocby{cdocsamp}
%    \end{macrocode}

%\iffalse
%</samplepart3|samplepart4>
%\fi
%
%\iffalse
%<*samplepart3>
%\fi
% Some text for part 3:
%    \begin{macrocode}
some text in part three
%    \end{macrocode}

%\iffalse
%</samplepart3>
%\fi
% Some text for part 4:
%\iffalse
%<*samplepart4>
%\fi
%    \begin{macrocode}
more text in part four
%    \end{macrocode}

%\iffalse
%</samplepart4>
%\fi
%
% %%%%%%%%%%%%%%%%%%%%%%%%%%%%%%%%%%%%%%
% \paragraph{Forwarding for a Complete Draft.}
%
% The following forwarding file |cdocsdrf.tex|
% compiles the main document in draft mode:
%\iffalse
%<*sampledraft>
%\fi
%    \begin{macrocode}
\def\version{draft}
\input{childdoc.def}
\childdocforward{cdocsamp}
%    \end{macrocode}

%\iffalse
%</sampledraft>
%\fi
%
% %%%%%%%%%%%%%%%%%%%%%%%%%%%%%%%%%%%%%%
% \paragraph{Forwarding for Final Version of the Chapters.}
%
% The following forwarding files |cdocsfn1.tex| and |cdocsfn2.tex|
% (with identical content)
% compile the final versions of the child documents
% |cdocsch1.tex| and |cdocsch2.tex|, respectively:
%\iffalse
%<*samplefinal>
%\fi
%    \begin{macrocode}
\def\version{final}
\input{childdoc.def}
\childdocforwardprefix[cdocsamp]{cdocsfn}{cdocsch}
%    \end{macrocode}

%\iffalse
%</samplefinal>
%\fi
%
% %%%%%%%%%%%%%%%%%%%%%%%%%%%%%%%%%%%%%%
% \paragraph{Command Line Processing.}
%
% The following three command lines generate the output files
% |cdocscld|, |cdocscl1| and |cdocscl2|
% which should be identical to
% |cdocsdrf|, |cdocsch1| and |cdocsfn2|, respectively:
% \begin{center}
% \begin{tabular}{l}
% |latex -jobname cdocscld \|\\
% |  "\def\version{draft}\input{childdoc.def}\childdocforward{cdocsamp}"|\\
% |latex -jobname cdocscl1 \|\\
% |  "\input{childdoc.def}\childdocforward[cdocsamp]{cdocsch1}"|\\
% |latex -jobname cdocscl2 \|\\
% |  "\def\version{final}\input{childdoc.def}\childdocforward{cdocsch2}"|
% \end{tabular}
% \end{center}
% Note that the trailing backslash on each first line
% merely continues the input to the second line
% (for convenient cut ant paste).
% Furthermore, the command |latex| can be replaced by any
% of its alternative versions such as |pdflatex|.
%
% %%%%%%%%%%%%%%%%%%%%%%%%%%%%%%%%%%%%%%%%%%%%%%%%%%%%%%%%%%%%%%%%%%%%%%%%%%%%%%
% %%%%%%%%%%%%%%%%%%%%%%%%%%%%%%%%%%%%%%%%%%%%%%%%%%%%%%%%%%%%%%%%%%%%%%%%%%%%%%
% \section{Implementation}
%\iffalse
%<*package>
%\fi
%
% This section describes the definitions file |childdoc.def|.

% The definitions cannot be loaded using |\usepackage| or |\RequirePackage|
% which has a mechanism to prevent loading a style file more than once.
% When loading the definitions by means of |\input|
% multiple instances have to be prevented manually:
%\iffalse
%This code needs to be before the `\ProvidesFile' directive
%which is defined at the beginning of this file.
%Therefore it is also placed there and commented out here.
%</package>
%<*discard>
%\fi
%    \begin{macrocode}
\ifdefined\childdocmain\endinput\fi
%    \end{macrocode}
%\iffalse
%</discard>
%<*package>
%\fi
%
% \macro{\ifchilddoc}
% \macro{\ifchilddocmanual}
% The conditional |\ifchilddoc| tells whether a
% child (true) or main (false) document is being compiled.
% The conditional |\ifchilddocmanual| tells whether
% the |\includeonly| mechanism is used (false) or
% the selection of child files must be performed manually (true).
% The definitions initialise to false:
%    \begin{macrocode}
\newif\ifchilddoc
\newif\ifchilddocmanual
%    \end{macrocode}

% \macro{\childdocname}
% \macro{\childdocjob}
% The macro |\childdocname| stores the name of the main document
% to be compiled. The macro |\childdocjob| stores the name of
% the document on which the \LaTeX{} compiler was originally invoked.
% The content of |\jobname| cannot be compared
% to filenames specified in the source due to different catcodes.
% The following code rescans |\jobname|, stores the result
% in |\childdocname| and saves a copy in |\childdocjob|:
%    \begin{macrocode}
\edef\childdocname{\scantokens\expandafter{\jobname\noexpand}}
\let\childdocjob\childdocname
%    \end{macrocode}

% \macro{\childdocdisable}
% The macro |\childdocdisable| prevents the main file
% from being processed more than once.
% At this stage, the main document command |\childdocmain|
% is assumed to be called once again where it should do nothing.
% Any subsequent call to it should prevent
% a secondary processing of the main document
% It overwrites the forwarding commands
% |\childdocof| and |\childdocforward|
% with empty macros to prevent further inclusions of the main document:
%    \begin{macrocode}
\newcommand{\childdocdisable}
{
  \renewcommand{\childdocmain}[1]{\renewcommand{\childdocmain}[1]{\endinput}}
  \renewcommand{\childdocof}[1]{}
  \renewcommand{\childdocby}[2][]{}
  \renewcommand{\childdocforward}[2][]{}
  \renewcommand{\childdocdisable}{}
}
%    \end{macrocode}

% \macro{\childdocmain}
% The macro |\childdocmain| is to be called at the top of the main file
% with nothing or the main filename (without extension) as argument.
% First, it breaks loops.
% If the argument is not empty and does not match |\childdocname|
% (which is set by the first inclusion of |childdoc.def|),
% |\ifchilddoc| is set to true, |\includeonly| is applied to the child file
% and |\jobname| is set to the main file
% (for proper handling of |.aux| files):
%    \begin{macrocode}
\newcommand{\childdocmain}[1]
{
  \childdocdisable\childdocmain{}
  \if?#1?\else
    \begingroup
      \def\childdoctmp{#1}
      \ifx\childdoctmp\childdocname
        \def\childdoctmp{}
      \else
        \def\childdoctmp
        {
          \childdoctrue
          \includeonly{\childdocname}
          \def\childdocjob{#1}
          \def\jobname{#1}
        }
      \fi
      \expandafter
    \endgroup
    \childdoctmp
  \fi
}
%    \end{macrocode}

% \macro{\childdocof}
% The command |\childdocof| redirects
% compilation to the main file |#1|.
%    \begin{macrocode}
\newcommand{\childdocof}[1]
{
  \childdocdisable
  \childdoctrue
  \includeonly{\childdocname}
  \def\jobname{#1}
  \def\childdocjob{#1}
  \input{#1}
}
%    \end{macrocode}

% \macro{\childdocby}
% The command |\childdocby| ....
%    \begin{macrocode}
\newcommand{\childdocby}[2][]
{
  \childdocdisable
  \childdoctrue
  \childdocmanualtrue
  \if?#1?\else
    \def\jobname{#2}
  \fi
  \def\childdocjob{#2}
  \input{#2}
  \endinput
}
%    \end{macrocode}

% \macro{\childdocforward}
% The command |\childdocforward| redirects
% compilation to the main file or
% (if the optional argument is given) a child file.
% Parameters are set as if the main file
% or a child file starting with |\childdocof| was compiled.
% Then compilation is handed over to the main file:
%    \begin{macrocode}
\newcommand{\childdocforward}[2][]
{
  \begingroup
    \if?#1?
      \def\childdoctmp
      {
        \def\childdocname{#2}
        \def\childdocjob{#2}
        \def\jobname{#2}
        \input{#2}
        \endinput
      }
    \else
      \def\childdoctmp
      {
        \childdocdisable
        \def\childdocname{#2}
        \childdoctrue
        \includeonly{#2}
        \def\childdocjob{#1}
        \def\jobname{#1}
        \input{#1}
        \endinput
      }
    \fi
    \expandafter
  \endgroup
  \childdoctmp
}
%    \end{macrocode}

% \macro{\childdocforwardprefix}
% The command |\childdocforwardprefix| redirects
% compilation to the main or a child file by means of a pattern.
% The prefix |#1| in the current filename is replaced by |#2|
% and the suffix of the current filename is kept
% (it is assumed that the filename does not contain the substring `|~~~|'
% which is used as a delimiter).
% Compilation is handed over to the new file by |\childdocforward|:
%    \begin{macrocode}
\newcommand{\childdocforwardprefix}[3][]
{
  \begingroup
    \def\childdocextract #2##1~~~{\def\childdoctmp{\childdocforward[#1]{#3##1}}}
    \expandafter\childdocextract\childdocname~~~
    \expandafter
  \endgroup
  \childdoctmp
}
%    \end{macrocode}

% \macro{\childdoc}
% The deprecated macro |\childdoc| is a legacy version of |\childdocmain|:
%    \begin{macrocode}
\newcommand{\childdoc}{\childdocmain}
%    \end{macrocode}

% \macro{\childdocredirect}
% The deprecated macro |\childdocredirect| is a legacy version
% of |\childdocforward| and |\childdocforwardprefix|:
%    \begin{macrocode}
\newcommand{\childdocredirect}[2][]
{
  \begingroup
    \if?#1?
      \def\childdoctmp{\childdocforward{#2}}
    \else
      \def\childdoctmp{\childdocforwardprefix{#1}{#2}}
    \fi
    \expandafter
  \endgroup
  \childdoctmp
}
%    \end{macrocode}

%\iffalse
%</package>
%\fi
%
\endinput
|\\
|\childdocmain{|\textit{main}|}|\\
\end{tabular}
\end{center}
%
If |\jobname| does not match the argument \textit{main} of |\childdocmain|,
it is assumed that |\jobname| points to the child file to be compiled.
When using |\childdocmain| with the main file specified as argument,
it suffices to start a child file
with just |\input{|\textit{main}|}|
without loading of the package and using |\childdocof|.
If instead all processing is done
with the appropriate \textsf{childdoc} directives,
the argument of \textit{main} of |\childdocmain| can be empty.

An alternative version of the command line processing described
in \secref{sec:commandline} using the detection mechanism reads:
%
\begin{center}
|... -jobname "|\textit{target}|" "|[\textit{flags}]%
[|\def\jobname{|\textit{dest}|}|]|\input{|\textit{main}|}"|
\end{center}

%%%%%%%%%%%%%%%%%%%%%%%%%%%%%%%%%%%%%%%%%%%%%%%%%%%%%%%%%%%%%%%%%%%%%%%%%%%%%%%%
\subsection{Manual Code}
\label{sec:manual}

In case one cannot be certain whether the definitions file |childdoc.def|
is installed on the target \TeX{} distribution
and one prefers not to ship it,
it is conceivable to paste a few relevant commands into the sources.

To that end, drop all statements |% \iffalse
%
% childdoc.dtx Copyright (C) 2017-2018 Niklas Beisert
%
% This work may be distributed and/or modified under the
% conditions of the LaTeX Project Public License, either version 1.3
% of this license or (at your option) any later version.
% The latest version of this license is in
%   http://www.latex-project.org/lppl.txt
% and version 1.3 or later is part of all distributions of LaTeX
% version 2005/12/01 or later.
%
% This work has the LPPL maintenance status `maintained'.
%
% The Current Maintainer of this work is Niklas Beisert.
%
% This work consists of the files childdoc.dtx and childdoc.ins
% and the derived files childdoc.def and cdocsamp.tex with
% cdocsch1.tex, cdocsch2.tex, cdocsdrf.tex, cdocsfn1.tex, cdocsfn2.tex.
%
%<package>\ifdefined\childdocmain\endinput\fi
%<package>\ProvidesFile{childdoc.def}[2018/12/30 v2.0 child document driver]
%<samplemain>\ProvidesFile{cdocsamp.tex}[2018/12/30 v2.0 sample for childdoc]
%<*driver>
%\ProvidesFile{childdoc.drv}[2018/12/30 v2.0 childdoc reference manual file]
\PassOptionsToClass{10pt,a4paper}{article}
\documentclass{ltxdoc}

\usepackage[margin=35mm]{geometry}
\usepackage{hyperref}
\usepackage{hyperxmp}
\usepackage[usenames]{color}

\hypersetup{colorlinks=true}
\hypersetup{pdfstartview=FitH}
\hypersetup{pdfpagemode=UseNone}
\hypersetup{pdfsource={}}
\hypersetup{pdflang={en-UK}}
\hypersetup{pdfcopyright={Copyright 2017-2018 Niklas Beisert.
  This work may be distributed and/or modified under the
  conditions of the LaTeX Project Public License, either version 1.3
  of this license or (at your option) any later version.}}
\hypersetup{pdflicenseurl={http://www.latex-project.org/lppl.txt}}
\hypersetup{pdfcontactaddress={ETH Zurich, ITP, HIT K,
  Wolfgang-Pauli-Strasse 27}}
\hypersetup{pdfcontactpostcode={8093}}
\hypersetup{pdfcontactcity={Zurich}}
\hypersetup{pdfcontactcountry={Switzerland}}
\hypersetup{pdfcontactemail={nbeisert@itp.phys.ethz.ch}}
\hypersetup{pdfcontacturl={http://people.phys.ethz.ch/\xmptilde nbeisert/}}

\newcommand{\secref}[1]{\hyperref[#1]{section \ref*{#1}}}

\parskip1ex
\parindent0pt
\let\olditemize\itemize
\def\itemize{\olditemize\parskip0pt}

\begin{document}

\title{The \textsf{childdoc} Package}
\hypersetup{pdftitle={The childdoc Package}}
\author{Niklas Beisert\\[2ex]
  Institut f\"ur Theoretische Physik\\
  Eidgen\"ossische Technische Hochschule Z\"urich\\
  Wolfgang-Pauli-Strasse 27, 8093 Z\"urich, Switzerland\\[1ex]
  \href{mailto:nbeisert@itp.phys.ethz.ch}
  {\texttt{nbeisert@itp.phys.ethz.ch}}}
\hypersetup{pdfauthor={Niklas Beisert}}
\hypersetup{pdfsubject={Manual for the LaTeX2e Package childdoc}}
\date{30 December 2018, \textsf{v2.0}}
\maketitle

\begin{abstract}\noindent
\textsf{childdoc} is a \LaTeXe{} package
that enables the direct compilation
of document sections included by |\include|
to individual files.
\end{abstract}

\begingroup
\parskip0ex
\tableofcontents
\endgroup

%%%%%%%%%%%%%%%%%%%%%%%%%%%%%%%%%%%%%%%%%%%%%%%%%%%%%%%%%%%%%%%%%%%%%%%%%%%%%%%%
%%%%%%%%%%%%%%%%%%%%%%%%%%%%%%%%%%%%%%%%%%%%%%%%%%%%%%%%%%%%%%%%%%%%%%%%%%%%%%%%
\section{Introduction}

\LaTeX{} provides a mechanism to structure a large document (such as a book)
into a main file and several child files (containing the chapters)
using the |\include| command.
This mechanism is beneficial for documents
which span hundreds of pages in order to
make the source file(s) more manageable.
Moreover, compilation can be restricted to
selected child files by means of the |\includeonly| command.
The latter feature can be used to reduce the compilation time while editing
(this was significantly more useful in the earlier days of \LaTeX{})
or to generate a smaller document which is easier to navigate.
Another application of |\includeonly| is to generate
documents consisting of selected parts of the complete document.

However, there are a few drawbacks of the plain |\include| mechanism:
\begin{itemize}
\item
The child files cannot be compiled on their own,
they can only be compiled via the main file.
A naive editing environment
(such as a text editor with an option
to have the current file processed by \LaTeX)
may require one to switch to the main file before compiling;
attempting to compile the child file produces errors.
\item
The main file must be modified (each time)
to adjust the |\includeonly| command
to the present needs. This easily leaves the main file in a messy state.
\item
The generated document will always carry the filename
of the main document. This is inconvenient if
several child files are to be compiled and
to be kept for distribution.
\end{itemize}

The present package provides a simple interface
to make child files individually compilable by \LaTeX{}.
Compiling a child file then has the same effect as compiling
the main file with an |\includeonly| command
to select the appropriate child.
Moreover the generated document will carry the name of the child
rather than the main file.
This resolves all three above issues.

This feature is meant to make the editing of books,
thesis documents and lecture notes somewhat more convenient.
However, the package can also be used efficiently for
composing a series of documents (such as exercise sheets)
which are typically distributed individually.
It then assists the author in generating the individual documents
(potentially in different versions)
as well as a document containing the collected series.
Another application is in developing style files
or other kinds of included material
where compilation of the style file could redirect
to a sample or test file.

%%%%%%%%%%%%%%%%%%%%%%%%%%%%%%%%%%%%%%%%%%%%%%%%%%%%%%%%%%%%%%%%%%%%%%%%%%%%%%%%
%%%%%%%%%%%%%%%%%%%%%%%%%%%%%%%%%%%%%%%%%%%%%%%%%%%%%%%%%%%%%%%%%%%%%%%%%%%%%%%%
\section{Usage}

First of all, the package \textsf{childdoc} is \emph{not} a standard
\LaTeXe{} |.sty| style file! Therefore it needs to be invoked in
a non-standard way.

%%%%%%%%%%%%%%%%%%%%%%%%%%%%%%%%%%%%%%%%%%%%%%%%%%%%%%%%%%%%%%%%%%%%%%%%%%%%%%%%
\subsection{Included Files}
\label{sec:include}

%%%%%%%%%%%%%%%%%%%%%%%%%%%%%%%%%%%%%%%%
\DescribeMacro{\childdocmain}
To use the package, add the commands
\begin{center}
\begin{tabular}{l}
|\input{childdoc.def}|\\
|\childdocmain{}|\\
\end{tabular}
\end{center}
at the very top of the main \LaTeX{} file,
in particular \emph{before} the |\documentclass| statement!
The argument of |\childdocmain| should be left empty
(but it must be present).

%%%%%%%%%%%%%%%%%%%%%%%%%%%%%%%%%%%%%%%%
\DescribeMacro{\childdocof}
Furthermore, add the commands
\begin{center}
\begin{tabular}{l}
|\input{childdoc.def}|\\
|\childdocof{|\textit{main}|}|\\
\end{tabular}
\end{center}
at the top of every child file \textit{child}
which is included by |\include{|\textit{child}|}|
from within the main file
(or at least for those files to be compiled individually).
The argument \textit{main} must be the filename of the main file.

There are a couple of
considerations in setting up the main and child documents:

%%%%%%%%%%%%%%%%%%%%%%%%%%%%%%%%%%%%%%%%
\paragraph{Restrictions.}

Please note the following restrictions:
\begin{itemize}
\item
|\childdocmain| must be called with one argument \textit{main}
to ensure compatibility with earlier version of the package.
It must either be empty (|\childdocmain{}|)
or precisely match the filename of the main file in which it is specified.
See \secref{sec:detection} for further information.
\item
The filename \textit{main} must be specified without the |.tex| extension.
\item
The filename \textit{main} is case sensitive
(even in case-insensitive file systems)
due to internal string comparison.
\item
The argument \textit{main} should be fully expanded, it cannot be a macro.
\item
Subdirectories and special characters should be avoided in filenames.
\item
The command |\childdocmain{|\textit{main}|}| must be followed by a whitespace.
It should not be followed immediately by another command
or by a comment mark `|%|'.
This is because the \TeX{} parser reads the token immediately following
the argument of |\childdocmain| and puts it
at the beginning of every child section;
however, a white\-space is ignored.
\end{itemize}

%%%%%%%%%%%%%%%%%%%%%%%%%%%%%%%%%%%%%%%%
\paragraph{Content of Main File.}

It is advisable to place all content in the child files included by |\include|.
Any output contained in the main file will appear in all child documents
unless suppressed manually;
it cannot be suppressed automatically by the |\includeonly| directive
and thus should normally be avoided.
A method to include some content in the main file
by means of conditional processing is described in \secref{sec:conditional}.

%%%%%%%%%%%%%%%%%%%%%%%%%%%%%%%%%%%%%%%%
\paragraph{Page Numbering.}

When only a part of the document is compiled,
the appropriate numbering of pages
(as well as other status parameters)
is determined from the |.aux| files.
The latter contain information from previous passes.
However this information needs to propagate through
all intermediate child documents.
Therefore the page numbering in child documents may well
be inconsistent until the complete document is compiled at least once.

A useful (if unconventional) way to always ensure a consistent
page numbering is to restart the numbering in each child document
and denote the pages by `\textit{child}|.|\textit{page}'
where \textit{child} represents the chapter/section number of the child file.
This can be achieved by the command
|\numberwithin{page}{|\textit{child}|}|
of the \textsf{amsmath} package
where \textit{child} can be |chapter| or |section|
depending on the chosen structuring.
Alternatively, one can modify the macro |\thepage| appropriately
and reset the counter |page| at the start of each child file.

%%%%%%%%%%%%%%%%%%%%%%%%%%%%%%%%%%%%%%%%%%%%%%%%%%%%%%%%%%%%%%%%%%%%%%%%%%%%%%%%
\subsection{Conditional Processing}
\label{sec:conditional}

The package provides a mechanism to compile different versions
of a document. To customise the versions further some conditional processing
can come in handy to distinguish which version is being compiled.
The package provides two macros to describe the compilation context:

%%%%%%%%%%%%%%%%%%%%%%%%%%%%%%%%%%%%%%%%
\DescribeMacro{\ifchilddoc}
The conditional |\ifchilddoc| distinguishes between the compilation of
child documents and the main document:
%
\begin{center}
|\ifchilddoc |\textit{child-code}| |[|\||else |\textit{main-code}]| \||fi|
\end{center}

%%%%%%%%%%%%%%%%%%%%%%%%%%%%%%%%%%%%%%%%
\DescribeMacro{\childdocname}
\DescribeMacro{\childdocjob}
The macro |\childdocname| contains the filename (without extension)
of the main or child file being processed.
Note that |\childdocjob| will always contain the name of the main file.

%%%%%%%%%%%%%%%%%%%%%%%%%%%%%%%%%%%%%%%%
\paragraph{Title Page.}

Conditional processing can be used to include a title or banner page
in the main document when proper precautions are taken.
Importantly, the code in the main file should ensure that the page counter
(as well as other status parameters which are stored in the |.aux| files)
takes the same value after the conditional processing.
Otherwise the page numbers may take divergent values
depending on which part is compiled.

For example, a title page could be declared by:
%
\begin{center}
\begin{tabular}{l}
|\ifchilddoc\||else|\\
|\addtocounter{page}{-1}|\\
\textit{code for title page}\\
|\newpage|\\
|\||fi|
\end{tabular}
\end{center}
%
A banner page for the child documents can be generated by:
%
\begin{center}
\begin{tabular}{l}
|\ifchilddoc|\\
|\addtocounter{page}{-1}|\\
\textit{code for banner page}\\
|\newpage|\\
|\||fi|
\end{tabular}
\end{center}
%
Here one could write a message such as:
\begin{center}
|This is the part \childdocname{} of \childdocjob{}.|
\end{center}

%%%%%%%%%%%%%%%%%%%%%%%%%%%%%%%%%%%%%%%%%%%%%%%%%%%%%%%%%%%%%%%%%%%%%%%%%%%%%%%%
\subsection{Flags}
\label{sec:flags}

The package makes it easy to generate different versions
of the main or child documents.
To this end compilation flags can be defined
and assigned different default values.
They will be particularly useful in conjunction
with the forwarding mechanism described in \secref{sec:forward}.

For example, it may be useful to have a flag |\version|
which can be set to |draft| or |final|.
The document source will contain some conditional code
depending on the value of |\version|.
Suppose further, the flag should default to |final| for the main file
and to |draft| for child files
which is a natural assignment for editing the document.
This is achieved by placing the following code
in the preamble of the main document
(below the |\childdocmain| directive):
%
\begin{center}
\begin{tabular}{l}
|\ifchilddoc|\\
|\providecommand{\version}{draft}|\\
|\||else|\\
|\providecommand{\version}{final}|\\
|\||fi|
\end{tabular}
\end{center}
%
The definition by |\providecommand| makes sure
that previous definitions are not overwritten.
Further statements |\providecommand{\version}{...}|
can thus be added before the above code to override it.

For the main file, one might add a line
(between |\childdocmain| and the above block)
%
\begin{center}
|%\ifchilddoc\||else\providecommand{\version}{draft}\||fi|
\end{center}
%
which can be uncommented to produce a draft version.
Likewise one can add a line to the very top of a child file
(above the |\childdocof{|\textit{main}|}| directive)
%
\begin{center}
|%\providecommand{\version}{final}|
\end{center}
%
which can be uncommented to produce the final version of this child document.

%%%%%%%%%%%%%%%%%%%%%%%%%%%%%%%%%%%%%%%%%%%%%%%%%%%%%%%%%%%%%%%%%%%%%%%%%%%%%%%%
\subsection{Forwarding}
\label{sec:forward}

Different versions of the main or child documents
using compilation flags as described in \secref{sec:flags}
can be (permanently) stored in different files
for convenient compilation, viewing and distribution.
To this end, the package defines a command
to pass on compilation to a different file:

%%%%%%%%%%%%%%%%%%%%%%%%%%%%%%%%%%%%%%%%
\DescribeMacro{\childdocforward}
The command |\childdocforward| redirects processing to
another source file:
%
\begin{center}
\begin{tabular}{l}
|\input{childdoc.def}|\\
|\childdocforward[|\textit{main}|]{|\textit{dest}|}|\\
\end{tabular}
\end{center}
%
The argument \textit{dest} is the destination file
(without extension).
It should be the main file or one of the child files.
Note that further \textsf{childdoc} directives
such as |\childdocof| and |\childdocforward|
in the indicated file will be processed in this form.
The optional argument \textit{main}
passes on directly to the main file \textit{main}
while pretending to compile the child \textit{dest}.
This form behaves as if \textit{dest}
issues |\childdocof{|\textit{main}|}| right away,
and no further \textsf{childdoc} directives will be processed.

%%%%%%%%%%%%%%%%%%%%%%%%%%%%%%%%%%%%%%%%
\DescribeMacro{\...prefix}
In the alternative form |\childdocforwardprefix|,
%
\begin{center}
\begin{tabular}{l}
|\input{childdoc.def}|\\
|\childdocforwardprefix[|\textit{main}|]{|\textit{prefix}|}{|\textit{dest}|}|
\end{tabular}
\end{center}
%
the destination file is determined by a pattern
depending on the current file:
To make this work, the current file must be called
`{\textit{prefix}\hspace{0.2em}\textit{suffix}}'
with \textit{prefix} matching precisely the argument.
Processing is then passed on to the file
`{\textit{dest}\hspace{0.2em}\textit{suffix}}'.
Surely, the same effect is achieved by
directly specifying the
argument `{\textit{dest}\hspace{0.2em}\textit{suffix}}'
in the first form.
However, that requires to set up a different file
for each child. With the alternative form of the command
all these files can have exactly the same content
which simplifies setting them up and maintaining them.

For example, the following file |draft.tex|
with a compilation flag |\version| as described in \secref{sec:flags}
compiles the main document as a draft:
%
\begin{center}
\begin{tabular}{l}
|\def\version{draft}|\\
|\input{childdoc.def}|\\
|\childdocforward{|\textit{main}|}|
\end{tabular}
\end{center}
%
Likewise, the following files |final|\textit{nn}|.tex|
compile the final version of the child document
|child|\textit{nn}|.tex|:
%
\begin{center}
\begin{tabular}{l}
|\def\version{final}|\\
|\input{childdoc.def}|\\
|\childdocforwardprefix{final}{child}|
\end{tabular}
\end{center}
%

Note that when several versions of a main file and/or of each child file
are to be generated, it may be convenient to set up a |Makefile| or
shell script to automatise the process.

%%%%%%%%%%%%%%%%%%%%%%%%%%%%%%%%%%%%%%%%%%%%%%%%%%%%%%%%%%%%%%%%%%%%%%%%%%%%%%%%
\subsection{Command Line Processing}
\label{sec:commandline}

The effect of redirection files can also be achieved by invoking
the \LaTeX{} compiler with a more elaborate command line.
Most conveniently this should be done as part
of a shell script or a |Makefile|.

When using \textsf{childdoc} in the main file, the following
command lines effectively perform a redirection
(note that depending on the shell being used,
backslashes may have to be doubled: `|\|' $\to$ `|\\|'):
%
\begin{center}
|... -jobname "|\textit{target}|" |\\|"|[\textit{flags}]%
|\input{childdoc.def}\childdocforward[|\textit{main}|]{|\textit{dest}|}"|
\end{center}
%
Here \textit{target} is the name of the output file,
\textit{main} is the name of the main file
and \textit{dest} is the name of the main or child file to be processed
(all filenames without extensions).
The optional argument \textit{main} can be omitted
if \textit{main} matches \textit{dest}.
Optionally, compilation \textit{flags} can be defined via |\def| commands.
This command line makes the \TeX{} engine believe
it is compiling the file \textit{target}
whose content is specified as the latter parameter.
The provided code then forwards the processing to
\textit{main} or \textit{dest} as described in \secref{sec:forward}.

%%%%%%%%%%%%%%%%%%%%%%%%%%%%%%%%%%%%%%%%%%%%%%%%%%%%%%%%%%%%%%%%%%%%%%%%%%%%%%%%
\subsection{Include by Input}
\label{sec:input}

Including child documents by |\include| has some restrictions by design.
Most notably, the content of a child document always occupies
its own set of pages; pages cannot be shared between child documents.
Usually, this behaviour makes perfect sense
because each child document contain an essential part of the document.
However, in some situations it may be desirable to compose
a document from a collection of parts
without having mandatory page breaks between then.
For this case, the package
provides a mechanism to include parts
by |\input| which can also be processed individually.
However, by construction this mechanism
requires manual handling of the content to be output.

%%%%%%%%%%%%%%%%%%%%%%%%%%%%%%%%%%%%%%%%
\DescribeMacro{\ifchilddocmanual}
The main file should be prepared as usual, see \secref{sec:include}.
However, the document body must make a distinction
between processing of an individual part and of the main document, e.g.:
%
\begin{center}
\begin{tabular}{l}
|\ifchilddocmanual|\\
|\input{\childdocname}|\\
|\||else|\\
\textit{document body with }|\input{|\textit{part}|}|\\
|\||fi|
\end{tabular}
\end{center}
%
The conditional |\ifchilddocmanual| is true whenever
a part to be included by |\input| is being compiled,
and the name of the part is stored in |\childdocname|.

%%%%%%%%%%%%%%%%%%%%%%%%%%%%%%%%%%%%%%%%
\DescribeMacro{\childdocby}
Each part to be included by |\input| should start with:
%
\begin{center}
\begin{tabular}{l}
|\input{childdoc.def}|\\
|\childdocby{|\textit{main}|}|\\
\end{tabular}
\end{center}
%
The directive |\childdocby| is similar to |\childdocof|
described in \secref{sec:include},
but the subsequent selection of content must be done manually.
To that end, both |\ifchilddoc| and |\ifchilddocmanual|
will be true upon processing of a part,
and the name of the part is stored in |\childdocname|.
Note that |\jobname| will be set to the filename of the current part
so that each part receives an individual |.aux| file
that does not interfere with the |.aux| file(s) of the main document.
This behaviour can be altered by the alternative form
|\childdocby[*]{|\textit{main}|}| (with a non-empty optional argument)
which uses the |.aux| file of the main document
by setting |\jobname| to \textit{main}.

%%%%%%%%%%%%%%%%%%%%%%%%%%%%%%%%%%%%%%%%%%%%%%%%%%%%%%%%%%%%%%%%%%%%%%%%%%%%%%%%
\subsection{Driver Development}
\label{sec:driver}

The \textsf{childdoc} mechanism can also be use for the development
of definition files such as \LaTeX{} styles or classes.
This case differs from the above setup with multiple parts
included by |\include| in that no |\includeonly| should be invoked.
This can be achieved by starting the include file
(before |\ProvidesPackage|) with:
%
\begin{center}
\begin{tabular}{l}
|\input{childdoc.def}|\\
|\childdocforward{|\textit{main}|}|\\
\end{tabular}
\end{center}
%
or alternatively with:
%
\begin{center}
\begin{tabular}{l}
|\input{childdoc.def}|\\
|\childdocby{|\textit{main}|}|\\
\end{tabular}
\end{center}
%
Both forms have slightly different effects as described above.
The main file is prepared as usual, see \secref{sec:include}.

%%%%%%%%%%%%%%%%%%%%%%%%%%%%%%%%%%%%%%%%%%%%%%%%%%%%%%%%%%%%%%%%%%%%%%%%%%%%%%%%
\subsection{Legacy Detection}
\label{sec:detection}

The directive |\childdocmain| in the main file can detect
whether the complete document or merely a child is to be compiled
even without using the directive |\childdocof|.
This method is deprecated because it is less robust
and there is no compelling reason to use it;
it is merely provided for backward compatibility
and it may be removed in future versions.

If the detection mechanism is to be used,
it is mandatory to correctly specify
the filename of the main file as the argument of |\childdocmain|:
%
\begin{center}
\begin{tabular}{l}
|\input{childdoc.def}|\\
|\childdocmain{|\textit{main}|}|\\
\end{tabular}
\end{center}
%
If |\jobname| does not match the argument \textit{main} of |\childdocmain|,
it is assumed that |\jobname| points to the child file to be compiled.
When using |\childdocmain| with the main file specified as argument,
it suffices to start a child file
with just |\input{|\textit{main}|}|
without loading of the package and using |\childdocof|.
If instead all processing is done
with the appropriate \textsf{childdoc} directives,
the argument of \textit{main} of |\childdocmain| can be empty.

An alternative version of the command line processing described
in \secref{sec:commandline} using the detection mechanism reads:
%
\begin{center}
|... -jobname "|\textit{target}|" "|[\textit{flags}]%
[|\def\jobname{|\textit{dest}|}|]|\input{|\textit{main}|}"|
\end{center}

%%%%%%%%%%%%%%%%%%%%%%%%%%%%%%%%%%%%%%%%%%%%%%%%%%%%%%%%%%%%%%%%%%%%%%%%%%%%%%%%
\subsection{Manual Code}
\label{sec:manual}

In case one cannot be certain whether the definitions file |childdoc.def|
is installed on the target \TeX{} distribution
and one prefers not to ship it,
it is conceivable to paste a few relevant commands into the sources.

To that end, drop all statements |\input{childdoc.def}|
and perform the replacements as outlined below.
Instead of |\childdocmain{|\textit{main}|}| add the following code
to the top of the main file:
%
\begin{center}
\begin{tabular}{l}
|\||ifdefined\childdocname\endinput\||fi\newif\ifchilddoc|\\
|\edef\childdocname{\scantokens\expandafter{\jobname\noexpand}}|\\
|\def\childdocmain{|\textit{main}|}\||ifx\childdocmain\childdocname\||else|\\
|\childdoctrue\includeonly{\childdocname}\let\jobname\childdocmain\||fi|\\
\end{tabular}
\end{center}
%
Instead of |\childdocof{|\textit{main}|}| just include the main file
at the top of each child file:
%
\begin{center}
|\input{|\textit{main}|}|
\end{center}
%
A simple redirection |\childdocforward{|\textit{dest}|}| is achieved by:
%
\begin{center}
|\def\jobname{|\textit{dest}|}\input{\jobname}|
\end{center}
%
The redirection with prefix
|\childdocforwardprefix[|\textit{prefix}|]{|\textit{dest}|}|
is accomplished by:
%
\begin{center}
\begin{tabular}{l}
|{\edef\jobname{\scantokens\expandafter{\jobname\noexpand}}|\\
|\def\redirectjob |\textit{prefix}|#1~~~{\gdef\jobname{|\textit{dest}|#1}}|\\
|\expandafter\redirectjob\jobname~~~}\input{\jobname}|
\end{tabular}
\end{center}

In an alternative approach,
child documents can be compiled by a specific command line
without additional code or specific definitions:
%
\begin{center}
|... -jobname "|\textit{target}|" "|[\textit{flags}]%
|\includeonly{|\textit{dest}|}\input{|\textit{main}|}"|
\end{center}
%

%%%%%%%%%%%%%%%%%%%%%%%%%%%%%%%%%%%%%%%%%%%%%%%%%%%%%%%%%%%%%%%%%%%%%%%%%%%%%%%%
%%%%%%%%%%%%%%%%%%%%%%%%%%%%%%%%%%%%%%%%%%%%%%%%%%%%%%%%%%%%%%%%%%%%%%%%%%%%%%%%
\section{Information}

%%%%%%%%%%%%%%%%%%%%%%%%%%%%%%%%%%%%%%%%%%%%%%%%%%%%%%%%%%%%%%%%%%%%%%%%%%%%%%%%
\subsection{Copyright}

Copyright \copyright{} 2017--2018 Niklas Beisert

This work may be distributed and/or modified under the
conditions of the \LaTeX{} Project Public License, either version 1.3
of this license or (at your option) any later version.
The latest version of this license is in
  \url{http://www.latex-project.org/lppl.txt}
and version 1.3 or later is part of all distributions of \LaTeX{}
version 2005/12/01 or later.

This work has the LPPL maintenance status `maintained'.

The Current Maintainer of this work is Niklas Beisert.

This work consists of the files |README.txt|, |childdoc.ins| and |childdoc.dtx|
as well as the derived files |childdoc.def|, |cdocsamp.tex|
with |cdocsch1.tex|, |cdocsch2.tex|, |cdocspt3.tex|, |cdocspt4.tex|,
|cdocsdrf.tex|, |cdocsfn1.tex|, |cdocsfn2.tex|
as well as |childdoc.pdf|.

%%%%%%%%%%%%%%%%%%%%%%%%%%%%%%%%%%%%%%%%%%%%%%%%%%%%%%%%%%%%%%%%%%%%%%%%%%%%%%%%
\subsection{Files and Installation}

The package consists of the files:
%
\begin{center}
\begin{tabular}{ll}
    |README.txt|   & readme file \\
    |childdoc.ins| & installation file \\
    |childdoc.dtx| & source file \\
    |childdoc.def| & definition file \\
    |cdocsamp.tex| & sample main file \\
    |cdocsch1.tex| & sample include file \\
    |cdocsch2.tex| & sample include file \\
    |cdocspt3.tex| & sample part file \\
    |cdocspt4.tex| & sample part file \\
    |cdocsdrf.tex| & sample redirection file \\
    |cdocsfn1.tex| & sample redirection file \\
    |cdocsfn2.tex| & sample redirection file \\
    |childdoc.pdf| & manual
\end{tabular}
\end{center}
%
The distribution consists of the files
|README.txt|, |childdoc.ins| and |childdoc.dtx|.
%
\begin{itemize}
\item
Run (pdf)\LaTeX{} on |childdoc.dtx|
to compile the manual |childdoc.pdf| (this file).
\item
Run \LaTeX{} on |childdoc.ins| to create the definitions file |childdoc.def|
and the sample |cdocsamp.tex| with include files
|cdocsch1.tex|, |cdocsch2.tex|, |cdocspt3.tex|, |cdocspt4.tex|,
|cdocsdrf.tex|, |cdocsfn1.tex|, |cdocsfn2.tex|.
Then copy the file |childdoc.def| to an appropriate directory of your \LaTeX{}
distribution, e.g.\ \textit{texmf-root}|/tex/latex/childdoc|.
\end{itemize}

%%%%%%%%%%%%%%%%%%%%%%%%%%%%%%%%%%%%%%%%%%%%%%%%%%%%%%%%%%%%%%%%%%%%%%%%%%%%%%%%
\subsection{Related CTAN Packages}

There are several other packages which offer a similar functionality:
%
\begin{itemize}
\item
The packages
\href{http://ctan.org/pkg/docmute}{\textsf{docmute}},
\href{http://ctan.org/pkg/includex}{\textsf{includex}} and
\href{http://ctan.org/pkg/standalone}{\textsf{standalone}}
provide commands to include only the document body of
a child file thus allowing both files to be compiled individually.
\item
The packages \href{http://ctan.org/pkg/subdocs}{\textsf{subdocs}}
and \href{http://ctan.org/pkg/subfiles}{\textsf{subfiles}}
provide structures in which the main and child documents can be
encapsulated and allowing them to be compiled individually.
The inclusion mechanism is different from the conventional |\include|.
\item
The package \href{http://ctan.org/pkg/combine}{\textsf{combine}}
is an elaborate solution to combine several documents into one.
\end{itemize}
%
See also the CTAN topic \href{http://ctan.org/topic/subdocs}{\textsf{subdocs}}
for further related packages.
The present package differs from the above solutions in that
a document structure constructed with the conventional |\include| mechanism
just needs two extra commands at the top of every file
such that all constituent files can be compiled individually.

%%%%%%%%%%%%%%%%%%%%%%%%%%%%%%%%%%%%%%%%%%%%%%%%%%%%%%%%%%%%%%%%%%%%%%%%%%%%%%%%
%\subsection{Feature Suggestions}
%
%The following is a list of features which may be useful for future
%versions of this package:
%%
%\begin{itemize}
%\item
%\ldots
%\end{itemize}

%%%%%%%%%%%%%%%%%%%%%%%%%%%%%%%%%%%%%%%%%%%%%%%%%%%%%%%%%%%%%%%%%%%%%%%%%%%%%%%%
\subsection{Revision History}

%%%%%%%%%%%%%%%%%%%%%%%%%%%%%%%%%%%%%%%%
\paragraph{v2.0:} 2018/12/30

\begin{itemize}
\item
immediate forward processing
\item
added |\childdocby| mechanism
\item
manual restructured
\end{itemize}

%%%%%%%%%%%%%%%%%%%%%%%%%%%%%%%%%%%%%%%%
\paragraph{v1.6:} 2018/01/17

\begin{itemize}
\item
application for development of include files
\item
corrections to manual
\end{itemize}

%%%%%%%%%%%%%%%%%%%%%%%%%%%%%%%%%%%%%%%%
\paragraph{v1.5:} 2017/05/21

\begin{itemize}
\item
more complete structuring introduced
\item
|\childdocof| introduced
\item
|\childdoc| renamed to |\childdocmain|
\item
|\childredirect| renamed to |\childdocforward| and |\childdocforwardprefix|
and functionality expanded
\end{itemize}

%%%%%%%%%%%%%%%%%%%%%%%%%%%%%%%%%%%%%%%%
\paragraph{v1.0:} 2017/04/27

\begin{itemize}
\item
manual and install package
\item
first version published on CTAN
\end{itemize}

%%%%%%%%%%%%%%%%%%%%%%%%%%%%%%%%%%%%%%%%
\paragraph{v0.6:} 2017/04/26

\begin{itemize}
\item
redirection mechanism added
\end{itemize}

%%%%%%%%%%%%%%%%%%%%%%%%%%%%%%%%%%%%%%%%
\paragraph{v0.5:} 2017/04/26

\begin{itemize}
\item
functionality in definition file
\end{itemize}


%%%%%%%%%%%%%%%%%%%%%%%%%%%%%%%%%%%%%%%%%%%%%%%%%%%%%%%%%%%%%%%%%%%%%%%%%%%%%%%%
%%%%%%%%%%%%%%%%%%%%%%%%%%%%%%%%%%%%%%%%%%%%%%%%%%%%%%%%%%%%%%%%%%%%%%%%%%%%%%%%
%%%%%%%%%%%%%%%%%%%%%%%%%%%%%%%%%%%%%%%%%%%%%%%%%%%%%%%%%%%%%%%%%%%%%%%%%%%%%%%%
\appendix

\settowidth\MacroIndent{\rmfamily\scriptsize 000\ }

 \DocInput{childdoc.dtx}

\end{document}
%</driver>
% \fi
%
% %%%%%%%%%%%%%%%%%%%%%%%%%%%%%%%%%%%%%%%%%%%%%%%%%%%%%%%%%%%%%%%%%%%%%%%%%%%%%%
% %%%%%%%%%%%%%%%%%%%%%%%%%%%%%%%%%%%%%%%%%%%%%%%%%%%%%%%%%%%%%%%%%%%%%%%%%%%%%%
% \section{Sample}
%\iffalse
%<*samplemain>
%\fi
%
% The following presents a sample document
% with two chapters, two parts, a title page,
% a compile flag as well as three forwarding files to set the flag.
% It consists of eight |.tex| files:
% \begin{center}
% \begin{tabular}{ll}
% |cdocsamp.tex|&main file\\
% |cdocsch1.tex|&include file for chapter 1\\
% |cdocsch2.tex|&include file for chapter 2\\
% |cdocspt3.tex|&include file for part 3\\
% |cdocspt4.tex|&include file for part 4\\
% |cdocsdrf.tex|&forwarding file for main file in draft mode\\
% |cdocsfi1.tex|&forwarding file for final version of chapter 1\\
% |cdocsfi2.tex|&forwarding file for final version of chapter 2\\
% \end{tabular}
% \end{center}
% Each of the eight files can be compiled directly by the \LaTeX{} compiler.
%
% %%%%%%%%%%%%%%%%%%%%%%%%%%%%%%%%%%%%%%
% \paragraph{Main File.}
%
% The main file is called |cdocsamp.tex|.
%
% Load the \textsf{childdoc} definitions and
% declare the filename for the main document:
%    \begin{macrocode}
\input{childdoc.def}
\childdocmain{}
%    \end{macrocode}

% Optional override for |\version| flag:
%    \begin{macrocode}
%%\ifchilddoc\else\providecommand{\version}{draft}\fi
%    \end{macrocode}

% Define the default values for the |\version| flag
% (|final| for the main file and |draft| for childs):
%    \begin{macrocode}
\ifchilddoc
\providecommand{\version}{draft}
\else
\providecommand{\version}{final}
\fi
%    \end{macrocode}

% Load the standard document class:
%    \begin{macrocode}
\documentclass[12pt]{article}
%    \end{macrocode}

% Start the document body:
%    \begin{macrocode}
\begin{document}
%    \end{macrocode}

% Declare a title page.
% Print title, part of document being processed and version flag:
%    \begin{macrocode}
\addtocounter{page}{-1}
\begin{center}
{\LARGE\bfseries{}childdoc example\par}
\vspace{1cm}
\ifchilddoc
\ifchilddocmanual part\else chapter\fi:
`\childdocname' of `\childdocjob'\par
\else
main document: `\childdocjob'\par
\fi
version: \version\par
\end{center}
\newpage
%    \end{macrocode}

% Manually include selected file,
% otherwise process as usual:
%    \begin{macrocode}
\ifchilddocmanual
\section*{part `\childdocname'}
\input{\childdocname}
\else
%    \end{macrocode}

% Include the two chapters:
%    \begin{macrocode}
\include{cdocsch1}
\include{cdocsch2}
%    \end{macrocode}

% Include the two parts unless only chapters should be displayed:
%    \begin{macrocode}
\ifchilddoc\else
\section{part three}
\input{cdocspt3}
\section{part four}
\input{cdocspt4}
\fi
%    \end{macrocode}

% Process as usual until here:
%    \begin{macrocode}
\fi
%    \end{macrocode}

% End of document body:
%    \begin{macrocode}
\end{document}
%    \end{macrocode}
%\iffalse
%</samplemain>
%\fi
%
% %%%%%%%%%%%%%%%%%%%%%%%%%%%%%%%%%%%%%%
% \paragraph{Chapter Include Files.}
%
% The include files are called |cdocsch1.tex| and |cdocsch2.tex|.
%
%\iffalse
%<*samplechap1|samplechap2>
%\fi

% Optional override for |\version| flag:
%    \begin{macrocode}
%%\providecommand{\version}{final}
%    \end{macrocode}

% Include the main document:
%    \begin{macrocode}
\input{childdoc.def}
\childdocof{cdocsamp}
%    \end{macrocode}

%\iffalse
%</samplechap1|samplechap2>
%\fi
%
%\iffalse
%<*samplechap1>
%\fi
% Some text for chapter 1:
%    \begin{macrocode}
\section{one}
some text in chapter one
%    \end{macrocode}

%\iffalse
%</samplechap1>
%\fi
% Some text for chapter 2:
%\iffalse
%<*samplechap2>
%\fi
%    \begin{macrocode}
\section{two}
more text in chapter two
%    \end{macrocode}

%\iffalse
%</samplechap2>
%\fi
%
% %%%%%%%%%%%%%%%%%%%%%%%%%%%%%%%%%%%%%%
% \paragraph{Part Include Files.}
%
% The include files are called |cdocspt3.tex| and |cdocspt4.tex|.
%
%\iffalse
%<*samplepart3|samplepart4>
%\fi

% Optional override for |\version| flag:
%    \begin{macrocode}
%%\providecommand{\version}{final}
%    \end{macrocode}

% Include the main document:
%    \begin{macrocode}
\input{childdoc.def}
\childdocby{cdocsamp}
%    \end{macrocode}

%\iffalse
%</samplepart3|samplepart4>
%\fi
%
%\iffalse
%<*samplepart3>
%\fi
% Some text for part 3:
%    \begin{macrocode}
some text in part three
%    \end{macrocode}

%\iffalse
%</samplepart3>
%\fi
% Some text for part 4:
%\iffalse
%<*samplepart4>
%\fi
%    \begin{macrocode}
more text in part four
%    \end{macrocode}

%\iffalse
%</samplepart4>
%\fi
%
% %%%%%%%%%%%%%%%%%%%%%%%%%%%%%%%%%%%%%%
% \paragraph{Forwarding for a Complete Draft.}
%
% The following forwarding file |cdocsdrf.tex|
% compiles the main document in draft mode:
%\iffalse
%<*sampledraft>
%\fi
%    \begin{macrocode}
\def\version{draft}
\input{childdoc.def}
\childdocforward{cdocsamp}
%    \end{macrocode}

%\iffalse
%</sampledraft>
%\fi
%
% %%%%%%%%%%%%%%%%%%%%%%%%%%%%%%%%%%%%%%
% \paragraph{Forwarding for Final Version of the Chapters.}
%
% The following forwarding files |cdocsfn1.tex| and |cdocsfn2.tex|
% (with identical content)
% compile the final versions of the child documents
% |cdocsch1.tex| and |cdocsch2.tex|, respectively:
%\iffalse
%<*samplefinal>
%\fi
%    \begin{macrocode}
\def\version{final}
\input{childdoc.def}
\childdocforwardprefix[cdocsamp]{cdocsfn}{cdocsch}
%    \end{macrocode}

%\iffalse
%</samplefinal>
%\fi
%
% %%%%%%%%%%%%%%%%%%%%%%%%%%%%%%%%%%%%%%
% \paragraph{Command Line Processing.}
%
% The following three command lines generate the output files
% |cdocscld|, |cdocscl1| and |cdocscl2|
% which should be identical to
% |cdocsdrf|, |cdocsch1| and |cdocsfn2|, respectively:
% \begin{center}
% \begin{tabular}{l}
% |latex -jobname cdocscld \|\\
% |  "\def\version{draft}\input{childdoc.def}\childdocforward{cdocsamp}"|\\
% |latex -jobname cdocscl1 \|\\
% |  "\input{childdoc.def}\childdocforward[cdocsamp]{cdocsch1}"|\\
% |latex -jobname cdocscl2 \|\\
% |  "\def\version{final}\input{childdoc.def}\childdocforward{cdocsch2}"|
% \end{tabular}
% \end{center}
% Note that the trailing backslash on each first line
% merely continues the input to the second line
% (for convenient cut ant paste).
% Furthermore, the command |latex| can be replaced by any
% of its alternative versions such as |pdflatex|.
%
% %%%%%%%%%%%%%%%%%%%%%%%%%%%%%%%%%%%%%%%%%%%%%%%%%%%%%%%%%%%%%%%%%%%%%%%%%%%%%%
% %%%%%%%%%%%%%%%%%%%%%%%%%%%%%%%%%%%%%%%%%%%%%%%%%%%%%%%%%%%%%%%%%%%%%%%%%%%%%%
% \section{Implementation}
%\iffalse
%<*package>
%\fi
%
% This section describes the definitions file |childdoc.def|.

% The definitions cannot be loaded using |\usepackage| or |\RequirePackage|
% which has a mechanism to prevent loading a style file more than once.
% When loading the definitions by means of |\input|
% multiple instances have to be prevented manually:
%\iffalse
%This code needs to be before the `\ProvidesFile' directive
%which is defined at the beginning of this file.
%Therefore it is also placed there and commented out here.
%</package>
%<*discard>
%\fi
%    \begin{macrocode}
\ifdefined\childdocmain\endinput\fi
%    \end{macrocode}
%\iffalse
%</discard>
%<*package>
%\fi
%
% \macro{\ifchilddoc}
% \macro{\ifchilddocmanual}
% The conditional |\ifchilddoc| tells whether a
% child (true) or main (false) document is being compiled.
% The conditional |\ifchilddocmanual| tells whether
% the |\includeonly| mechanism is used (false) or
% the selection of child files must be performed manually (true).
% The definitions initialise to false:
%    \begin{macrocode}
\newif\ifchilddoc
\newif\ifchilddocmanual
%    \end{macrocode}

% \macro{\childdocname}
% \macro{\childdocjob}
% The macro |\childdocname| stores the name of the main document
% to be compiled. The macro |\childdocjob| stores the name of
% the document on which the \LaTeX{} compiler was originally invoked.
% The content of |\jobname| cannot be compared
% to filenames specified in the source due to different catcodes.
% The following code rescans |\jobname|, stores the result
% in |\childdocname| and saves a copy in |\childdocjob|:
%    \begin{macrocode}
\edef\childdocname{\scantokens\expandafter{\jobname\noexpand}}
\let\childdocjob\childdocname
%    \end{macrocode}

% \macro{\childdocdisable}
% The macro |\childdocdisable| prevents the main file
% from being processed more than once.
% At this stage, the main document command |\childdocmain|
% is assumed to be called once again where it should do nothing.
% Any subsequent call to it should prevent
% a secondary processing of the main document
% It overwrites the forwarding commands
% |\childdocof| and |\childdocforward|
% with empty macros to prevent further inclusions of the main document:
%    \begin{macrocode}
\newcommand{\childdocdisable}
{
  \renewcommand{\childdocmain}[1]{\renewcommand{\childdocmain}[1]{\endinput}}
  \renewcommand{\childdocof}[1]{}
  \renewcommand{\childdocby}[2][]{}
  \renewcommand{\childdocforward}[2][]{}
  \renewcommand{\childdocdisable}{}
}
%    \end{macrocode}

% \macro{\childdocmain}
% The macro |\childdocmain| is to be called at the top of the main file
% with nothing or the main filename (without extension) as argument.
% First, it breaks loops.
% If the argument is not empty and does not match |\childdocname|
% (which is set by the first inclusion of |childdoc.def|),
% |\ifchilddoc| is set to true, |\includeonly| is applied to the child file
% and |\jobname| is set to the main file
% (for proper handling of |.aux| files):
%    \begin{macrocode}
\newcommand{\childdocmain}[1]
{
  \childdocdisable\childdocmain{}
  \if?#1?\else
    \begingroup
      \def\childdoctmp{#1}
      \ifx\childdoctmp\childdocname
        \def\childdoctmp{}
      \else
        \def\childdoctmp
        {
          \childdoctrue
          \includeonly{\childdocname}
          \def\childdocjob{#1}
          \def\jobname{#1}
        }
      \fi
      \expandafter
    \endgroup
    \childdoctmp
  \fi
}
%    \end{macrocode}

% \macro{\childdocof}
% The command |\childdocof| redirects
% compilation to the main file |#1|.
%    \begin{macrocode}
\newcommand{\childdocof}[1]
{
  \childdocdisable
  \childdoctrue
  \includeonly{\childdocname}
  \def\jobname{#1}
  \def\childdocjob{#1}
  \input{#1}
}
%    \end{macrocode}

% \macro{\childdocby}
% The command |\childdocby| ....
%    \begin{macrocode}
\newcommand{\childdocby}[2][]
{
  \childdocdisable
  \childdoctrue
  \childdocmanualtrue
  \if?#1?\else
    \def\jobname{#2}
  \fi
  \def\childdocjob{#2}
  \input{#2}
  \endinput
}
%    \end{macrocode}

% \macro{\childdocforward}
% The command |\childdocforward| redirects
% compilation to the main file or
% (if the optional argument is given) a child file.
% Parameters are set as if the main file
% or a child file starting with |\childdocof| was compiled.
% Then compilation is handed over to the main file:
%    \begin{macrocode}
\newcommand{\childdocforward}[2][]
{
  \begingroup
    \if?#1?
      \def\childdoctmp
      {
        \def\childdocname{#2}
        \def\childdocjob{#2}
        \def\jobname{#2}
        \input{#2}
        \endinput
      }
    \else
      \def\childdoctmp
      {
        \childdocdisable
        \def\childdocname{#2}
        \childdoctrue
        \includeonly{#2}
        \def\childdocjob{#1}
        \def\jobname{#1}
        \input{#1}
        \endinput
      }
    \fi
    \expandafter
  \endgroup
  \childdoctmp
}
%    \end{macrocode}

% \macro{\childdocforwardprefix}
% The command |\childdocforwardprefix| redirects
% compilation to the main or a child file by means of a pattern.
% The prefix |#1| in the current filename is replaced by |#2|
% and the suffix of the current filename is kept
% (it is assumed that the filename does not contain the substring `|~~~|'
% which is used as a delimiter).
% Compilation is handed over to the new file by |\childdocforward|:
%    \begin{macrocode}
\newcommand{\childdocforwardprefix}[3][]
{
  \begingroup
    \def\childdocextract #2##1~~~{\def\childdoctmp{\childdocforward[#1]{#3##1}}}
    \expandafter\childdocextract\childdocname~~~
    \expandafter
  \endgroup
  \childdoctmp
}
%    \end{macrocode}

% \macro{\childdoc}
% The deprecated macro |\childdoc| is a legacy version of |\childdocmain|:
%    \begin{macrocode}
\newcommand{\childdoc}{\childdocmain}
%    \end{macrocode}

% \macro{\childdocredirect}
% The deprecated macro |\childdocredirect| is a legacy version
% of |\childdocforward| and |\childdocforwardprefix|:
%    \begin{macrocode}
\newcommand{\childdocredirect}[2][]
{
  \begingroup
    \if?#1?
      \def\childdoctmp{\childdocforward{#2}}
    \else
      \def\childdoctmp{\childdocforwardprefix{#1}{#2}}
    \fi
    \expandafter
  \endgroup
  \childdoctmp
}
%    \end{macrocode}

%\iffalse
%</package>
%\fi
%
\endinput
|
and perform the replacements as outlined below.
Instead of |\childdocmain{|\textit{main}|}| add the following code
to the top of the main file:
%
\begin{center}
\begin{tabular}{l}
|\||ifdefined\childdocname\endinput\||fi\newif\ifchilddoc|\\
|\edef\childdocname{\scantokens\expandafter{\jobname\noexpand}}|\\
|\def\childdocmain{|\textit{main}|}\||ifx\childdocmain\childdocname\||else|\\
|\childdoctrue\includeonly{\childdocname}\let\jobname\childdocmain\||fi|\\
\end{tabular}
\end{center}
%
Instead of |\childdocof{|\textit{main}|}| just include the main file
at the top of each child file:
%
\begin{center}
|\input{|\textit{main}|}|
\end{center}
%
A simple redirection |\childdocforward{|\textit{dest}|}| is achieved by:
%
\begin{center}
|\def\jobname{|\textit{dest}|}\input{\jobname}|
\end{center}
%
The redirection with prefix
|\childdocforwardprefix[|\textit{prefix}|]{|\textit{dest}|}|
is accomplished by:
%
\begin{center}
\begin{tabular}{l}
|{\edef\jobname{\scantokens\expandafter{\jobname\noexpand}}|\\
|\def\redirectjob |\textit{prefix}|#1~~~{\gdef\jobname{|\textit{dest}|#1}}|\\
|\expandafter\redirectjob\jobname~~~}\input{\jobname}|
\end{tabular}
\end{center}

In an alternative approach,
child documents can be compiled by a specific command line
without additional code or specific definitions:
%
\begin{center}
|... -jobname "|\textit{target}|" "|[\textit{flags}]%
|\includeonly{|\textit{dest}|}\input{|\textit{main}|}"|
\end{center}
%

%%%%%%%%%%%%%%%%%%%%%%%%%%%%%%%%%%%%%%%%%%%%%%%%%%%%%%%%%%%%%%%%%%%%%%%%%%%%%%%%
%%%%%%%%%%%%%%%%%%%%%%%%%%%%%%%%%%%%%%%%%%%%%%%%%%%%%%%%%%%%%%%%%%%%%%%%%%%%%%%%
\section{Information}

%%%%%%%%%%%%%%%%%%%%%%%%%%%%%%%%%%%%%%%%%%%%%%%%%%%%%%%%%%%%%%%%%%%%%%%%%%%%%%%%
\subsection{Copyright}

Copyright \copyright{} 2017--2018 Niklas Beisert

This work may be distributed and/or modified under the
conditions of the \LaTeX{} Project Public License, either version 1.3
of this license or (at your option) any later version.
The latest version of this license is in
  \url{http://www.latex-project.org/lppl.txt}
and version 1.3 or later is part of all distributions of \LaTeX{}
version 2005/12/01 or later.

This work has the LPPL maintenance status `maintained'.

The Current Maintainer of this work is Niklas Beisert.

This work consists of the files |README.txt|, |childdoc.ins| and |childdoc.dtx|
as well as the derived files |childdoc.def|, |cdocsamp.tex|
with |cdocsch1.tex|, |cdocsch2.tex|, |cdocspt3.tex|, |cdocspt4.tex|,
|cdocsdrf.tex|, |cdocsfn1.tex|, |cdocsfn2.tex|
as well as |childdoc.pdf|.

%%%%%%%%%%%%%%%%%%%%%%%%%%%%%%%%%%%%%%%%%%%%%%%%%%%%%%%%%%%%%%%%%%%%%%%%%%%%%%%%
\subsection{Files and Installation}

The package consists of the files:
%
\begin{center}
\begin{tabular}{ll}
    |README.txt|   & readme file \\
    |childdoc.ins| & installation file \\
    |childdoc.dtx| & source file \\
    |childdoc.def| & definition file \\
    |cdocsamp.tex| & sample main file \\
    |cdocsch1.tex| & sample include file \\
    |cdocsch2.tex| & sample include file \\
    |cdocspt3.tex| & sample part file \\
    |cdocspt4.tex| & sample part file \\
    |cdocsdrf.tex| & sample redirection file \\
    |cdocsfn1.tex| & sample redirection file \\
    |cdocsfn2.tex| & sample redirection file \\
    |childdoc.pdf| & manual
\end{tabular}
\end{center}
%
The distribution consists of the files
|README.txt|, |childdoc.ins| and |childdoc.dtx|.
%
\begin{itemize}
\item
Run (pdf)\LaTeX{} on |childdoc.dtx|
to compile the manual |childdoc.pdf| (this file).
\item
Run \LaTeX{} on |childdoc.ins| to create the definitions file |childdoc.def|
and the sample |cdocsamp.tex| with include files
|cdocsch1.tex|, |cdocsch2.tex|, |cdocspt3.tex|, |cdocspt4.tex|,
|cdocsdrf.tex|, |cdocsfn1.tex|, |cdocsfn2.tex|.
Then copy the file |childdoc.def| to an appropriate directory of your \LaTeX{}
distribution, e.g.\ \textit{texmf-root}|/tex/latex/childdoc|.
\end{itemize}

%%%%%%%%%%%%%%%%%%%%%%%%%%%%%%%%%%%%%%%%%%%%%%%%%%%%%%%%%%%%%%%%%%%%%%%%%%%%%%%%
\subsection{Related CTAN Packages}

There are several other packages which offer a similar functionality:
%
\begin{itemize}
\item
The packages
\href{http://ctan.org/pkg/docmute}{\textsf{docmute}},
\href{http://ctan.org/pkg/includex}{\textsf{includex}} and
\href{http://ctan.org/pkg/standalone}{\textsf{standalone}}
provide commands to include only the document body of
a child file thus allowing both files to be compiled individually.
\item
The packages \href{http://ctan.org/pkg/subdocs}{\textsf{subdocs}}
and \href{http://ctan.org/pkg/subfiles}{\textsf{subfiles}}
provide structures in which the main and child documents can be
encapsulated and allowing them to be compiled individually.
The inclusion mechanism is different from the conventional |\include|.
\item
The package \href{http://ctan.org/pkg/combine}{\textsf{combine}}
is an elaborate solution to combine several documents into one.
\end{itemize}
%
See also the CTAN topic \href{http://ctan.org/topic/subdocs}{\textsf{subdocs}}
for further related packages.
The present package differs from the above solutions in that
a document structure constructed with the conventional |\include| mechanism
just needs two extra commands at the top of every file
such that all constituent files can be compiled individually.

%%%%%%%%%%%%%%%%%%%%%%%%%%%%%%%%%%%%%%%%%%%%%%%%%%%%%%%%%%%%%%%%%%%%%%%%%%%%%%%%
%\subsection{Feature Suggestions}
%
%The following is a list of features which may be useful for future
%versions of this package:
%%
%\begin{itemize}
%\item
%\ldots
%\end{itemize}

%%%%%%%%%%%%%%%%%%%%%%%%%%%%%%%%%%%%%%%%%%%%%%%%%%%%%%%%%%%%%%%%%%%%%%%%%%%%%%%%
\subsection{Revision History}

%%%%%%%%%%%%%%%%%%%%%%%%%%%%%%%%%%%%%%%%
\paragraph{v2.0:} 2018/12/30

\begin{itemize}
\item
immediate forward processing
\item
added |\childdocby| mechanism
\item
manual restructured
\end{itemize}

%%%%%%%%%%%%%%%%%%%%%%%%%%%%%%%%%%%%%%%%
\paragraph{v1.6:} 2018/01/17

\begin{itemize}
\item
application for development of include files
\item
corrections to manual
\end{itemize}

%%%%%%%%%%%%%%%%%%%%%%%%%%%%%%%%%%%%%%%%
\paragraph{v1.5:} 2017/05/21

\begin{itemize}
\item
more complete structuring introduced
\item
|\childdocof| introduced
\item
|\childdoc| renamed to |\childdocmain|
\item
|\childredirect| renamed to |\childdocforward| and |\childdocforwardprefix|
and functionality expanded
\end{itemize}

%%%%%%%%%%%%%%%%%%%%%%%%%%%%%%%%%%%%%%%%
\paragraph{v1.0:} 2017/04/27

\begin{itemize}
\item
manual and install package
\item
first version published on CTAN
\end{itemize}

%%%%%%%%%%%%%%%%%%%%%%%%%%%%%%%%%%%%%%%%
\paragraph{v0.6:} 2017/04/26

\begin{itemize}
\item
redirection mechanism added
\end{itemize}

%%%%%%%%%%%%%%%%%%%%%%%%%%%%%%%%%%%%%%%%
\paragraph{v0.5:} 2017/04/26

\begin{itemize}
\item
functionality in definition file
\end{itemize}


%%%%%%%%%%%%%%%%%%%%%%%%%%%%%%%%%%%%%%%%%%%%%%%%%%%%%%%%%%%%%%%%%%%%%%%%%%%%%%%%
%%%%%%%%%%%%%%%%%%%%%%%%%%%%%%%%%%%%%%%%%%%%%%%%%%%%%%%%%%%%%%%%%%%%%%%%%%%%%%%%
%%%%%%%%%%%%%%%%%%%%%%%%%%%%%%%%%%%%%%%%%%%%%%%%%%%%%%%%%%%%%%%%%%%%%%%%%%%%%%%%
\appendix

\settowidth\MacroIndent{\rmfamily\scriptsize 000\ }

 \DocInput{childdoc.dtx}

\end{document}
%</driver>
% \fi
%
% %%%%%%%%%%%%%%%%%%%%%%%%%%%%%%%%%%%%%%%%%%%%%%%%%%%%%%%%%%%%%%%%%%%%%%%%%%%%%%
% %%%%%%%%%%%%%%%%%%%%%%%%%%%%%%%%%%%%%%%%%%%%%%%%%%%%%%%%%%%%%%%%%%%%%%%%%%%%%%
% \section{Sample}
%\iffalse
%<*samplemain>
%\fi
%
% The following presents a sample document
% with two chapters, two parts, a title page,
% a compile flag as well as three forwarding files to set the flag.
% It consists of eight |.tex| files:
% \begin{center}
% \begin{tabular}{ll}
% |cdocsamp.tex|&main file\\
% |cdocsch1.tex|&include file for chapter 1\\
% |cdocsch2.tex|&include file for chapter 2\\
% |cdocspt3.tex|&include file for part 3\\
% |cdocspt4.tex|&include file for part 4\\
% |cdocsdrf.tex|&forwarding file for main file in draft mode\\
% |cdocsfi1.tex|&forwarding file for final version of chapter 1\\
% |cdocsfi2.tex|&forwarding file for final version of chapter 2\\
% \end{tabular}
% \end{center}
% Each of the eight files can be compiled directly by the \LaTeX{} compiler.
%
% %%%%%%%%%%%%%%%%%%%%%%%%%%%%%%%%%%%%%%
% \paragraph{Main File.}
%
% The main file is called |cdocsamp.tex|.
%
% Load the \textsf{childdoc} definitions and
% declare the filename for the main document:
%    \begin{macrocode}
% \iffalse
%
% childdoc.dtx Copyright (C) 2017-2018 Niklas Beisert
%
% This work may be distributed and/or modified under the
% conditions of the LaTeX Project Public License, either version 1.3
% of this license or (at your option) any later version.
% The latest version of this license is in
%   http://www.latex-project.org/lppl.txt
% and version 1.3 or later is part of all distributions of LaTeX
% version 2005/12/01 or later.
%
% This work has the LPPL maintenance status `maintained'.
%
% The Current Maintainer of this work is Niklas Beisert.
%
% This work consists of the files childdoc.dtx and childdoc.ins
% and the derived files childdoc.def and cdocsamp.tex with
% cdocsch1.tex, cdocsch2.tex, cdocsdrf.tex, cdocsfn1.tex, cdocsfn2.tex.
%
%<package>\ifdefined\childdocmain\endinput\fi
%<package>\ProvidesFile{childdoc.def}[2018/12/30 v2.0 child document driver]
%<samplemain>\ProvidesFile{cdocsamp.tex}[2018/12/30 v2.0 sample for childdoc]
%<*driver>
%\ProvidesFile{childdoc.drv}[2018/12/30 v2.0 childdoc reference manual file]
\PassOptionsToClass{10pt,a4paper}{article}
\documentclass{ltxdoc}

\usepackage[margin=35mm]{geometry}
\usepackage{hyperref}
\usepackage{hyperxmp}
\usepackage[usenames]{color}

\hypersetup{colorlinks=true}
\hypersetup{pdfstartview=FitH}
\hypersetup{pdfpagemode=UseNone}
\hypersetup{pdfsource={}}
\hypersetup{pdflang={en-UK}}
\hypersetup{pdfcopyright={Copyright 2017-2018 Niklas Beisert.
  This work may be distributed and/or modified under the
  conditions of the LaTeX Project Public License, either version 1.3
  of this license or (at your option) any later version.}}
\hypersetup{pdflicenseurl={http://www.latex-project.org/lppl.txt}}
\hypersetup{pdfcontactaddress={ETH Zurich, ITP, HIT K,
  Wolfgang-Pauli-Strasse 27}}
\hypersetup{pdfcontactpostcode={8093}}
\hypersetup{pdfcontactcity={Zurich}}
\hypersetup{pdfcontactcountry={Switzerland}}
\hypersetup{pdfcontactemail={nbeisert@itp.phys.ethz.ch}}
\hypersetup{pdfcontacturl={http://people.phys.ethz.ch/\xmptilde nbeisert/}}

\newcommand{\secref}[1]{\hyperref[#1]{section \ref*{#1}}}

\parskip1ex
\parindent0pt
\let\olditemize\itemize
\def\itemize{\olditemize\parskip0pt}

\begin{document}

\title{The \textsf{childdoc} Package}
\hypersetup{pdftitle={The childdoc Package}}
\author{Niklas Beisert\\[2ex]
  Institut f\"ur Theoretische Physik\\
  Eidgen\"ossische Technische Hochschule Z\"urich\\
  Wolfgang-Pauli-Strasse 27, 8093 Z\"urich, Switzerland\\[1ex]
  \href{mailto:nbeisert@itp.phys.ethz.ch}
  {\texttt{nbeisert@itp.phys.ethz.ch}}}
\hypersetup{pdfauthor={Niklas Beisert}}
\hypersetup{pdfsubject={Manual for the LaTeX2e Package childdoc}}
\date{30 December 2018, \textsf{v2.0}}
\maketitle

\begin{abstract}\noindent
\textsf{childdoc} is a \LaTeXe{} package
that enables the direct compilation
of document sections included by |\include|
to individual files.
\end{abstract}

\begingroup
\parskip0ex
\tableofcontents
\endgroup

%%%%%%%%%%%%%%%%%%%%%%%%%%%%%%%%%%%%%%%%%%%%%%%%%%%%%%%%%%%%%%%%%%%%%%%%%%%%%%%%
%%%%%%%%%%%%%%%%%%%%%%%%%%%%%%%%%%%%%%%%%%%%%%%%%%%%%%%%%%%%%%%%%%%%%%%%%%%%%%%%
\section{Introduction}

\LaTeX{} provides a mechanism to structure a large document (such as a book)
into a main file and several child files (containing the chapters)
using the |\include| command.
This mechanism is beneficial for documents
which span hundreds of pages in order to
make the source file(s) more manageable.
Moreover, compilation can be restricted to
selected child files by means of the |\includeonly| command.
The latter feature can be used to reduce the compilation time while editing
(this was significantly more useful in the earlier days of \LaTeX{})
or to generate a smaller document which is easier to navigate.
Another application of |\includeonly| is to generate
documents consisting of selected parts of the complete document.

However, there are a few drawbacks of the plain |\include| mechanism:
\begin{itemize}
\item
The child files cannot be compiled on their own,
they can only be compiled via the main file.
A naive editing environment
(such as a text editor with an option
to have the current file processed by \LaTeX)
may require one to switch to the main file before compiling;
attempting to compile the child file produces errors.
\item
The main file must be modified (each time)
to adjust the |\includeonly| command
to the present needs. This easily leaves the main file in a messy state.
\item
The generated document will always carry the filename
of the main document. This is inconvenient if
several child files are to be compiled and
to be kept for distribution.
\end{itemize}

The present package provides a simple interface
to make child files individually compilable by \LaTeX{}.
Compiling a child file then has the same effect as compiling
the main file with an |\includeonly| command
to select the appropriate child.
Moreover the generated document will carry the name of the child
rather than the main file.
This resolves all three above issues.

This feature is meant to make the editing of books,
thesis documents and lecture notes somewhat more convenient.
However, the package can also be used efficiently for
composing a series of documents (such as exercise sheets)
which are typically distributed individually.
It then assists the author in generating the individual documents
(potentially in different versions)
as well as a document containing the collected series.
Another application is in developing style files
or other kinds of included material
where compilation of the style file could redirect
to a sample or test file.

%%%%%%%%%%%%%%%%%%%%%%%%%%%%%%%%%%%%%%%%%%%%%%%%%%%%%%%%%%%%%%%%%%%%%%%%%%%%%%%%
%%%%%%%%%%%%%%%%%%%%%%%%%%%%%%%%%%%%%%%%%%%%%%%%%%%%%%%%%%%%%%%%%%%%%%%%%%%%%%%%
\section{Usage}

First of all, the package \textsf{childdoc} is \emph{not} a standard
\LaTeXe{} |.sty| style file! Therefore it needs to be invoked in
a non-standard way.

%%%%%%%%%%%%%%%%%%%%%%%%%%%%%%%%%%%%%%%%%%%%%%%%%%%%%%%%%%%%%%%%%%%%%%%%%%%%%%%%
\subsection{Included Files}
\label{sec:include}

%%%%%%%%%%%%%%%%%%%%%%%%%%%%%%%%%%%%%%%%
\DescribeMacro{\childdocmain}
To use the package, add the commands
\begin{center}
\begin{tabular}{l}
|\input{childdoc.def}|\\
|\childdocmain{}|\\
\end{tabular}
\end{center}
at the very top of the main \LaTeX{} file,
in particular \emph{before} the |\documentclass| statement!
The argument of |\childdocmain| should be left empty
(but it must be present).

%%%%%%%%%%%%%%%%%%%%%%%%%%%%%%%%%%%%%%%%
\DescribeMacro{\childdocof}
Furthermore, add the commands
\begin{center}
\begin{tabular}{l}
|\input{childdoc.def}|\\
|\childdocof{|\textit{main}|}|\\
\end{tabular}
\end{center}
at the top of every child file \textit{child}
which is included by |\include{|\textit{child}|}|
from within the main file
(or at least for those files to be compiled individually).
The argument \textit{main} must be the filename of the main file.

There are a couple of
considerations in setting up the main and child documents:

%%%%%%%%%%%%%%%%%%%%%%%%%%%%%%%%%%%%%%%%
\paragraph{Restrictions.}

Please note the following restrictions:
\begin{itemize}
\item
|\childdocmain| must be called with one argument \textit{main}
to ensure compatibility with earlier version of the package.
It must either be empty (|\childdocmain{}|)
or precisely match the filename of the main file in which it is specified.
See \secref{sec:detection} for further information.
\item
The filename \textit{main} must be specified without the |.tex| extension.
\item
The filename \textit{main} is case sensitive
(even in case-insensitive file systems)
due to internal string comparison.
\item
The argument \textit{main} should be fully expanded, it cannot be a macro.
\item
Subdirectories and special characters should be avoided in filenames.
\item
The command |\childdocmain{|\textit{main}|}| must be followed by a whitespace.
It should not be followed immediately by another command
or by a comment mark `|%|'.
This is because the \TeX{} parser reads the token immediately following
the argument of |\childdocmain| and puts it
at the beginning of every child section;
however, a white\-space is ignored.
\end{itemize}

%%%%%%%%%%%%%%%%%%%%%%%%%%%%%%%%%%%%%%%%
\paragraph{Content of Main File.}

It is advisable to place all content in the child files included by |\include|.
Any output contained in the main file will appear in all child documents
unless suppressed manually;
it cannot be suppressed automatically by the |\includeonly| directive
and thus should normally be avoided.
A method to include some content in the main file
by means of conditional processing is described in \secref{sec:conditional}.

%%%%%%%%%%%%%%%%%%%%%%%%%%%%%%%%%%%%%%%%
\paragraph{Page Numbering.}

When only a part of the document is compiled,
the appropriate numbering of pages
(as well as other status parameters)
is determined from the |.aux| files.
The latter contain information from previous passes.
However this information needs to propagate through
all intermediate child documents.
Therefore the page numbering in child documents may well
be inconsistent until the complete document is compiled at least once.

A useful (if unconventional) way to always ensure a consistent
page numbering is to restart the numbering in each child document
and denote the pages by `\textit{child}|.|\textit{page}'
where \textit{child} represents the chapter/section number of the child file.
This can be achieved by the command
|\numberwithin{page}{|\textit{child}|}|
of the \textsf{amsmath} package
where \textit{child} can be |chapter| or |section|
depending on the chosen structuring.
Alternatively, one can modify the macro |\thepage| appropriately
and reset the counter |page| at the start of each child file.

%%%%%%%%%%%%%%%%%%%%%%%%%%%%%%%%%%%%%%%%%%%%%%%%%%%%%%%%%%%%%%%%%%%%%%%%%%%%%%%%
\subsection{Conditional Processing}
\label{sec:conditional}

The package provides a mechanism to compile different versions
of a document. To customise the versions further some conditional processing
can come in handy to distinguish which version is being compiled.
The package provides two macros to describe the compilation context:

%%%%%%%%%%%%%%%%%%%%%%%%%%%%%%%%%%%%%%%%
\DescribeMacro{\ifchilddoc}
The conditional |\ifchilddoc| distinguishes between the compilation of
child documents and the main document:
%
\begin{center}
|\ifchilddoc |\textit{child-code}| |[|\||else |\textit{main-code}]| \||fi|
\end{center}

%%%%%%%%%%%%%%%%%%%%%%%%%%%%%%%%%%%%%%%%
\DescribeMacro{\childdocname}
\DescribeMacro{\childdocjob}
The macro |\childdocname| contains the filename (without extension)
of the main or child file being processed.
Note that |\childdocjob| will always contain the name of the main file.

%%%%%%%%%%%%%%%%%%%%%%%%%%%%%%%%%%%%%%%%
\paragraph{Title Page.}

Conditional processing can be used to include a title or banner page
in the main document when proper precautions are taken.
Importantly, the code in the main file should ensure that the page counter
(as well as other status parameters which are stored in the |.aux| files)
takes the same value after the conditional processing.
Otherwise the page numbers may take divergent values
depending on which part is compiled.

For example, a title page could be declared by:
%
\begin{center}
\begin{tabular}{l}
|\ifchilddoc\||else|\\
|\addtocounter{page}{-1}|\\
\textit{code for title page}\\
|\newpage|\\
|\||fi|
\end{tabular}
\end{center}
%
A banner page for the child documents can be generated by:
%
\begin{center}
\begin{tabular}{l}
|\ifchilddoc|\\
|\addtocounter{page}{-1}|\\
\textit{code for banner page}\\
|\newpage|\\
|\||fi|
\end{tabular}
\end{center}
%
Here one could write a message such as:
\begin{center}
|This is the part \childdocname{} of \childdocjob{}.|
\end{center}

%%%%%%%%%%%%%%%%%%%%%%%%%%%%%%%%%%%%%%%%%%%%%%%%%%%%%%%%%%%%%%%%%%%%%%%%%%%%%%%%
\subsection{Flags}
\label{sec:flags}

The package makes it easy to generate different versions
of the main or child documents.
To this end compilation flags can be defined
and assigned different default values.
They will be particularly useful in conjunction
with the forwarding mechanism described in \secref{sec:forward}.

For example, it may be useful to have a flag |\version|
which can be set to |draft| or |final|.
The document source will contain some conditional code
depending on the value of |\version|.
Suppose further, the flag should default to |final| for the main file
and to |draft| for child files
which is a natural assignment for editing the document.
This is achieved by placing the following code
in the preamble of the main document
(below the |\childdocmain| directive):
%
\begin{center}
\begin{tabular}{l}
|\ifchilddoc|\\
|\providecommand{\version}{draft}|\\
|\||else|\\
|\providecommand{\version}{final}|\\
|\||fi|
\end{tabular}
\end{center}
%
The definition by |\providecommand| makes sure
that previous definitions are not overwritten.
Further statements |\providecommand{\version}{...}|
can thus be added before the above code to override it.

For the main file, one might add a line
(between |\childdocmain| and the above block)
%
\begin{center}
|%\ifchilddoc\||else\providecommand{\version}{draft}\||fi|
\end{center}
%
which can be uncommented to produce a draft version.
Likewise one can add a line to the very top of a child file
(above the |\childdocof{|\textit{main}|}| directive)
%
\begin{center}
|%\providecommand{\version}{final}|
\end{center}
%
which can be uncommented to produce the final version of this child document.

%%%%%%%%%%%%%%%%%%%%%%%%%%%%%%%%%%%%%%%%%%%%%%%%%%%%%%%%%%%%%%%%%%%%%%%%%%%%%%%%
\subsection{Forwarding}
\label{sec:forward}

Different versions of the main or child documents
using compilation flags as described in \secref{sec:flags}
can be (permanently) stored in different files
for convenient compilation, viewing and distribution.
To this end, the package defines a command
to pass on compilation to a different file:

%%%%%%%%%%%%%%%%%%%%%%%%%%%%%%%%%%%%%%%%
\DescribeMacro{\childdocforward}
The command |\childdocforward| redirects processing to
another source file:
%
\begin{center}
\begin{tabular}{l}
|\input{childdoc.def}|\\
|\childdocforward[|\textit{main}|]{|\textit{dest}|}|\\
\end{tabular}
\end{center}
%
The argument \textit{dest} is the destination file
(without extension).
It should be the main file or one of the child files.
Note that further \textsf{childdoc} directives
such as |\childdocof| and |\childdocforward|
in the indicated file will be processed in this form.
The optional argument \textit{main}
passes on directly to the main file \textit{main}
while pretending to compile the child \textit{dest}.
This form behaves as if \textit{dest}
issues |\childdocof{|\textit{main}|}| right away,
and no further \textsf{childdoc} directives will be processed.

%%%%%%%%%%%%%%%%%%%%%%%%%%%%%%%%%%%%%%%%
\DescribeMacro{\...prefix}
In the alternative form |\childdocforwardprefix|,
%
\begin{center}
\begin{tabular}{l}
|\input{childdoc.def}|\\
|\childdocforwardprefix[|\textit{main}|]{|\textit{prefix}|}{|\textit{dest}|}|
\end{tabular}
\end{center}
%
the destination file is determined by a pattern
depending on the current file:
To make this work, the current file must be called
`{\textit{prefix}\hspace{0.2em}\textit{suffix}}'
with \textit{prefix} matching precisely the argument.
Processing is then passed on to the file
`{\textit{dest}\hspace{0.2em}\textit{suffix}}'.
Surely, the same effect is achieved by
directly specifying the
argument `{\textit{dest}\hspace{0.2em}\textit{suffix}}'
in the first form.
However, that requires to set up a different file
for each child. With the alternative form of the command
all these files can have exactly the same content
which simplifies setting them up and maintaining them.

For example, the following file |draft.tex|
with a compilation flag |\version| as described in \secref{sec:flags}
compiles the main document as a draft:
%
\begin{center}
\begin{tabular}{l}
|\def\version{draft}|\\
|\input{childdoc.def}|\\
|\childdocforward{|\textit{main}|}|
\end{tabular}
\end{center}
%
Likewise, the following files |final|\textit{nn}|.tex|
compile the final version of the child document
|child|\textit{nn}|.tex|:
%
\begin{center}
\begin{tabular}{l}
|\def\version{final}|\\
|\input{childdoc.def}|\\
|\childdocforwardprefix{final}{child}|
\end{tabular}
\end{center}
%

Note that when several versions of a main file and/or of each child file
are to be generated, it may be convenient to set up a |Makefile| or
shell script to automatise the process.

%%%%%%%%%%%%%%%%%%%%%%%%%%%%%%%%%%%%%%%%%%%%%%%%%%%%%%%%%%%%%%%%%%%%%%%%%%%%%%%%
\subsection{Command Line Processing}
\label{sec:commandline}

The effect of redirection files can also be achieved by invoking
the \LaTeX{} compiler with a more elaborate command line.
Most conveniently this should be done as part
of a shell script or a |Makefile|.

When using \textsf{childdoc} in the main file, the following
command lines effectively perform a redirection
(note that depending on the shell being used,
backslashes may have to be doubled: `|\|' $\to$ `|\\|'):
%
\begin{center}
|... -jobname "|\textit{target}|" |\\|"|[\textit{flags}]%
|\input{childdoc.def}\childdocforward[|\textit{main}|]{|\textit{dest}|}"|
\end{center}
%
Here \textit{target} is the name of the output file,
\textit{main} is the name of the main file
and \textit{dest} is the name of the main or child file to be processed
(all filenames without extensions).
The optional argument \textit{main} can be omitted
if \textit{main} matches \textit{dest}.
Optionally, compilation \textit{flags} can be defined via |\def| commands.
This command line makes the \TeX{} engine believe
it is compiling the file \textit{target}
whose content is specified as the latter parameter.
The provided code then forwards the processing to
\textit{main} or \textit{dest} as described in \secref{sec:forward}.

%%%%%%%%%%%%%%%%%%%%%%%%%%%%%%%%%%%%%%%%%%%%%%%%%%%%%%%%%%%%%%%%%%%%%%%%%%%%%%%%
\subsection{Include by Input}
\label{sec:input}

Including child documents by |\include| has some restrictions by design.
Most notably, the content of a child document always occupies
its own set of pages; pages cannot be shared between child documents.
Usually, this behaviour makes perfect sense
because each child document contain an essential part of the document.
However, in some situations it may be desirable to compose
a document from a collection of parts
without having mandatory page breaks between then.
For this case, the package
provides a mechanism to include parts
by |\input| which can also be processed individually.
However, by construction this mechanism
requires manual handling of the content to be output.

%%%%%%%%%%%%%%%%%%%%%%%%%%%%%%%%%%%%%%%%
\DescribeMacro{\ifchilddocmanual}
The main file should be prepared as usual, see \secref{sec:include}.
However, the document body must make a distinction
between processing of an individual part and of the main document, e.g.:
%
\begin{center}
\begin{tabular}{l}
|\ifchilddocmanual|\\
|\input{\childdocname}|\\
|\||else|\\
\textit{document body with }|\input{|\textit{part}|}|\\
|\||fi|
\end{tabular}
\end{center}
%
The conditional |\ifchilddocmanual| is true whenever
a part to be included by |\input| is being compiled,
and the name of the part is stored in |\childdocname|.

%%%%%%%%%%%%%%%%%%%%%%%%%%%%%%%%%%%%%%%%
\DescribeMacro{\childdocby}
Each part to be included by |\input| should start with:
%
\begin{center}
\begin{tabular}{l}
|\input{childdoc.def}|\\
|\childdocby{|\textit{main}|}|\\
\end{tabular}
\end{center}
%
The directive |\childdocby| is similar to |\childdocof|
described in \secref{sec:include},
but the subsequent selection of content must be done manually.
To that end, both |\ifchilddoc| and |\ifchilddocmanual|
will be true upon processing of a part,
and the name of the part is stored in |\childdocname|.
Note that |\jobname| will be set to the filename of the current part
so that each part receives an individual |.aux| file
that does not interfere with the |.aux| file(s) of the main document.
This behaviour can be altered by the alternative form
|\childdocby[*]{|\textit{main}|}| (with a non-empty optional argument)
which uses the |.aux| file of the main document
by setting |\jobname| to \textit{main}.

%%%%%%%%%%%%%%%%%%%%%%%%%%%%%%%%%%%%%%%%%%%%%%%%%%%%%%%%%%%%%%%%%%%%%%%%%%%%%%%%
\subsection{Driver Development}
\label{sec:driver}

The \textsf{childdoc} mechanism can also be use for the development
of definition files such as \LaTeX{} styles or classes.
This case differs from the above setup with multiple parts
included by |\include| in that no |\includeonly| should be invoked.
This can be achieved by starting the include file
(before |\ProvidesPackage|) with:
%
\begin{center}
\begin{tabular}{l}
|\input{childdoc.def}|\\
|\childdocforward{|\textit{main}|}|\\
\end{tabular}
\end{center}
%
or alternatively with:
%
\begin{center}
\begin{tabular}{l}
|\input{childdoc.def}|\\
|\childdocby{|\textit{main}|}|\\
\end{tabular}
\end{center}
%
Both forms have slightly different effects as described above.
The main file is prepared as usual, see \secref{sec:include}.

%%%%%%%%%%%%%%%%%%%%%%%%%%%%%%%%%%%%%%%%%%%%%%%%%%%%%%%%%%%%%%%%%%%%%%%%%%%%%%%%
\subsection{Legacy Detection}
\label{sec:detection}

The directive |\childdocmain| in the main file can detect
whether the complete document or merely a child is to be compiled
even without using the directive |\childdocof|.
This method is deprecated because it is less robust
and there is no compelling reason to use it;
it is merely provided for backward compatibility
and it may be removed in future versions.

If the detection mechanism is to be used,
it is mandatory to correctly specify
the filename of the main file as the argument of |\childdocmain|:
%
\begin{center}
\begin{tabular}{l}
|\input{childdoc.def}|\\
|\childdocmain{|\textit{main}|}|\\
\end{tabular}
\end{center}
%
If |\jobname| does not match the argument \textit{main} of |\childdocmain|,
it is assumed that |\jobname| points to the child file to be compiled.
When using |\childdocmain| with the main file specified as argument,
it suffices to start a child file
with just |\input{|\textit{main}|}|
without loading of the package and using |\childdocof|.
If instead all processing is done
with the appropriate \textsf{childdoc} directives,
the argument of \textit{main} of |\childdocmain| can be empty.

An alternative version of the command line processing described
in \secref{sec:commandline} using the detection mechanism reads:
%
\begin{center}
|... -jobname "|\textit{target}|" "|[\textit{flags}]%
[|\def\jobname{|\textit{dest}|}|]|\input{|\textit{main}|}"|
\end{center}

%%%%%%%%%%%%%%%%%%%%%%%%%%%%%%%%%%%%%%%%%%%%%%%%%%%%%%%%%%%%%%%%%%%%%%%%%%%%%%%%
\subsection{Manual Code}
\label{sec:manual}

In case one cannot be certain whether the definitions file |childdoc.def|
is installed on the target \TeX{} distribution
and one prefers not to ship it,
it is conceivable to paste a few relevant commands into the sources.

To that end, drop all statements |\input{childdoc.def}|
and perform the replacements as outlined below.
Instead of |\childdocmain{|\textit{main}|}| add the following code
to the top of the main file:
%
\begin{center}
\begin{tabular}{l}
|\||ifdefined\childdocname\endinput\||fi\newif\ifchilddoc|\\
|\edef\childdocname{\scantokens\expandafter{\jobname\noexpand}}|\\
|\def\childdocmain{|\textit{main}|}\||ifx\childdocmain\childdocname\||else|\\
|\childdoctrue\includeonly{\childdocname}\let\jobname\childdocmain\||fi|\\
\end{tabular}
\end{center}
%
Instead of |\childdocof{|\textit{main}|}| just include the main file
at the top of each child file:
%
\begin{center}
|\input{|\textit{main}|}|
\end{center}
%
A simple redirection |\childdocforward{|\textit{dest}|}| is achieved by:
%
\begin{center}
|\def\jobname{|\textit{dest}|}\input{\jobname}|
\end{center}
%
The redirection with prefix
|\childdocforwardprefix[|\textit{prefix}|]{|\textit{dest}|}|
is accomplished by:
%
\begin{center}
\begin{tabular}{l}
|{\edef\jobname{\scantokens\expandafter{\jobname\noexpand}}|\\
|\def\redirectjob |\textit{prefix}|#1~~~{\gdef\jobname{|\textit{dest}|#1}}|\\
|\expandafter\redirectjob\jobname~~~}\input{\jobname}|
\end{tabular}
\end{center}

In an alternative approach,
child documents can be compiled by a specific command line
without additional code or specific definitions:
%
\begin{center}
|... -jobname "|\textit{target}|" "|[\textit{flags}]%
|\includeonly{|\textit{dest}|}\input{|\textit{main}|}"|
\end{center}
%

%%%%%%%%%%%%%%%%%%%%%%%%%%%%%%%%%%%%%%%%%%%%%%%%%%%%%%%%%%%%%%%%%%%%%%%%%%%%%%%%
%%%%%%%%%%%%%%%%%%%%%%%%%%%%%%%%%%%%%%%%%%%%%%%%%%%%%%%%%%%%%%%%%%%%%%%%%%%%%%%%
\section{Information}

%%%%%%%%%%%%%%%%%%%%%%%%%%%%%%%%%%%%%%%%%%%%%%%%%%%%%%%%%%%%%%%%%%%%%%%%%%%%%%%%
\subsection{Copyright}

Copyright \copyright{} 2017--2018 Niklas Beisert

This work may be distributed and/or modified under the
conditions of the \LaTeX{} Project Public License, either version 1.3
of this license or (at your option) any later version.
The latest version of this license is in
  \url{http://www.latex-project.org/lppl.txt}
and version 1.3 or later is part of all distributions of \LaTeX{}
version 2005/12/01 or later.

This work has the LPPL maintenance status `maintained'.

The Current Maintainer of this work is Niklas Beisert.

This work consists of the files |README.txt|, |childdoc.ins| and |childdoc.dtx|
as well as the derived files |childdoc.def|, |cdocsamp.tex|
with |cdocsch1.tex|, |cdocsch2.tex|, |cdocspt3.tex|, |cdocspt4.tex|,
|cdocsdrf.tex|, |cdocsfn1.tex|, |cdocsfn2.tex|
as well as |childdoc.pdf|.

%%%%%%%%%%%%%%%%%%%%%%%%%%%%%%%%%%%%%%%%%%%%%%%%%%%%%%%%%%%%%%%%%%%%%%%%%%%%%%%%
\subsection{Files and Installation}

The package consists of the files:
%
\begin{center}
\begin{tabular}{ll}
    |README.txt|   & readme file \\
    |childdoc.ins| & installation file \\
    |childdoc.dtx| & source file \\
    |childdoc.def| & definition file \\
    |cdocsamp.tex| & sample main file \\
    |cdocsch1.tex| & sample include file \\
    |cdocsch2.tex| & sample include file \\
    |cdocspt3.tex| & sample part file \\
    |cdocspt4.tex| & sample part file \\
    |cdocsdrf.tex| & sample redirection file \\
    |cdocsfn1.tex| & sample redirection file \\
    |cdocsfn2.tex| & sample redirection file \\
    |childdoc.pdf| & manual
\end{tabular}
\end{center}
%
The distribution consists of the files
|README.txt|, |childdoc.ins| and |childdoc.dtx|.
%
\begin{itemize}
\item
Run (pdf)\LaTeX{} on |childdoc.dtx|
to compile the manual |childdoc.pdf| (this file).
\item
Run \LaTeX{} on |childdoc.ins| to create the definitions file |childdoc.def|
and the sample |cdocsamp.tex| with include files
|cdocsch1.tex|, |cdocsch2.tex|, |cdocspt3.tex|, |cdocspt4.tex|,
|cdocsdrf.tex|, |cdocsfn1.tex|, |cdocsfn2.tex|.
Then copy the file |childdoc.def| to an appropriate directory of your \LaTeX{}
distribution, e.g.\ \textit{texmf-root}|/tex/latex/childdoc|.
\end{itemize}

%%%%%%%%%%%%%%%%%%%%%%%%%%%%%%%%%%%%%%%%%%%%%%%%%%%%%%%%%%%%%%%%%%%%%%%%%%%%%%%%
\subsection{Related CTAN Packages}

There are several other packages which offer a similar functionality:
%
\begin{itemize}
\item
The packages
\href{http://ctan.org/pkg/docmute}{\textsf{docmute}},
\href{http://ctan.org/pkg/includex}{\textsf{includex}} and
\href{http://ctan.org/pkg/standalone}{\textsf{standalone}}
provide commands to include only the document body of
a child file thus allowing both files to be compiled individually.
\item
The packages \href{http://ctan.org/pkg/subdocs}{\textsf{subdocs}}
and \href{http://ctan.org/pkg/subfiles}{\textsf{subfiles}}
provide structures in which the main and child documents can be
encapsulated and allowing them to be compiled individually.
The inclusion mechanism is different from the conventional |\include|.
\item
The package \href{http://ctan.org/pkg/combine}{\textsf{combine}}
is an elaborate solution to combine several documents into one.
\end{itemize}
%
See also the CTAN topic \href{http://ctan.org/topic/subdocs}{\textsf{subdocs}}
for further related packages.
The present package differs from the above solutions in that
a document structure constructed with the conventional |\include| mechanism
just needs two extra commands at the top of every file
such that all constituent files can be compiled individually.

%%%%%%%%%%%%%%%%%%%%%%%%%%%%%%%%%%%%%%%%%%%%%%%%%%%%%%%%%%%%%%%%%%%%%%%%%%%%%%%%
%\subsection{Feature Suggestions}
%
%The following is a list of features which may be useful for future
%versions of this package:
%%
%\begin{itemize}
%\item
%\ldots
%\end{itemize}

%%%%%%%%%%%%%%%%%%%%%%%%%%%%%%%%%%%%%%%%%%%%%%%%%%%%%%%%%%%%%%%%%%%%%%%%%%%%%%%%
\subsection{Revision History}

%%%%%%%%%%%%%%%%%%%%%%%%%%%%%%%%%%%%%%%%
\paragraph{v2.0:} 2018/12/30

\begin{itemize}
\item
immediate forward processing
\item
added |\childdocby| mechanism
\item
manual restructured
\end{itemize}

%%%%%%%%%%%%%%%%%%%%%%%%%%%%%%%%%%%%%%%%
\paragraph{v1.6:} 2018/01/17

\begin{itemize}
\item
application for development of include files
\item
corrections to manual
\end{itemize}

%%%%%%%%%%%%%%%%%%%%%%%%%%%%%%%%%%%%%%%%
\paragraph{v1.5:} 2017/05/21

\begin{itemize}
\item
more complete structuring introduced
\item
|\childdocof| introduced
\item
|\childdoc| renamed to |\childdocmain|
\item
|\childredirect| renamed to |\childdocforward| and |\childdocforwardprefix|
and functionality expanded
\end{itemize}

%%%%%%%%%%%%%%%%%%%%%%%%%%%%%%%%%%%%%%%%
\paragraph{v1.0:} 2017/04/27

\begin{itemize}
\item
manual and install package
\item
first version published on CTAN
\end{itemize}

%%%%%%%%%%%%%%%%%%%%%%%%%%%%%%%%%%%%%%%%
\paragraph{v0.6:} 2017/04/26

\begin{itemize}
\item
redirection mechanism added
\end{itemize}

%%%%%%%%%%%%%%%%%%%%%%%%%%%%%%%%%%%%%%%%
\paragraph{v0.5:} 2017/04/26

\begin{itemize}
\item
functionality in definition file
\end{itemize}


%%%%%%%%%%%%%%%%%%%%%%%%%%%%%%%%%%%%%%%%%%%%%%%%%%%%%%%%%%%%%%%%%%%%%%%%%%%%%%%%
%%%%%%%%%%%%%%%%%%%%%%%%%%%%%%%%%%%%%%%%%%%%%%%%%%%%%%%%%%%%%%%%%%%%%%%%%%%%%%%%
%%%%%%%%%%%%%%%%%%%%%%%%%%%%%%%%%%%%%%%%%%%%%%%%%%%%%%%%%%%%%%%%%%%%%%%%%%%%%%%%
\appendix

\settowidth\MacroIndent{\rmfamily\scriptsize 000\ }

 \DocInput{childdoc.dtx}

\end{document}
%</driver>
% \fi
%
% %%%%%%%%%%%%%%%%%%%%%%%%%%%%%%%%%%%%%%%%%%%%%%%%%%%%%%%%%%%%%%%%%%%%%%%%%%%%%%
% %%%%%%%%%%%%%%%%%%%%%%%%%%%%%%%%%%%%%%%%%%%%%%%%%%%%%%%%%%%%%%%%%%%%%%%%%%%%%%
% \section{Sample}
%\iffalse
%<*samplemain>
%\fi
%
% The following presents a sample document
% with two chapters, two parts, a title page,
% a compile flag as well as three forwarding files to set the flag.
% It consists of eight |.tex| files:
% \begin{center}
% \begin{tabular}{ll}
% |cdocsamp.tex|&main file\\
% |cdocsch1.tex|&include file for chapter 1\\
% |cdocsch2.tex|&include file for chapter 2\\
% |cdocspt3.tex|&include file for part 3\\
% |cdocspt4.tex|&include file for part 4\\
% |cdocsdrf.tex|&forwarding file for main file in draft mode\\
% |cdocsfi1.tex|&forwarding file for final version of chapter 1\\
% |cdocsfi2.tex|&forwarding file for final version of chapter 2\\
% \end{tabular}
% \end{center}
% Each of the eight files can be compiled directly by the \LaTeX{} compiler.
%
% %%%%%%%%%%%%%%%%%%%%%%%%%%%%%%%%%%%%%%
% \paragraph{Main File.}
%
% The main file is called |cdocsamp.tex|.
%
% Load the \textsf{childdoc} definitions and
% declare the filename for the main document:
%    \begin{macrocode}
\input{childdoc.def}
\childdocmain{}
%    \end{macrocode}

% Optional override for |\version| flag:
%    \begin{macrocode}
%%\ifchilddoc\else\providecommand{\version}{draft}\fi
%    \end{macrocode}

% Define the default values for the |\version| flag
% (|final| for the main file and |draft| for childs):
%    \begin{macrocode}
\ifchilddoc
\providecommand{\version}{draft}
\else
\providecommand{\version}{final}
\fi
%    \end{macrocode}

% Load the standard document class:
%    \begin{macrocode}
\documentclass[12pt]{article}
%    \end{macrocode}

% Start the document body:
%    \begin{macrocode}
\begin{document}
%    \end{macrocode}

% Declare a title page.
% Print title, part of document being processed and version flag:
%    \begin{macrocode}
\addtocounter{page}{-1}
\begin{center}
{\LARGE\bfseries{}childdoc example\par}
\vspace{1cm}
\ifchilddoc
\ifchilddocmanual part\else chapter\fi:
`\childdocname' of `\childdocjob'\par
\else
main document: `\childdocjob'\par
\fi
version: \version\par
\end{center}
\newpage
%    \end{macrocode}

% Manually include selected file,
% otherwise process as usual:
%    \begin{macrocode}
\ifchilddocmanual
\section*{part `\childdocname'}
\input{\childdocname}
\else
%    \end{macrocode}

% Include the two chapters:
%    \begin{macrocode}
\include{cdocsch1}
\include{cdocsch2}
%    \end{macrocode}

% Include the two parts unless only chapters should be displayed:
%    \begin{macrocode}
\ifchilddoc\else
\section{part three}
\input{cdocspt3}
\section{part four}
\input{cdocspt4}
\fi
%    \end{macrocode}

% Process as usual until here:
%    \begin{macrocode}
\fi
%    \end{macrocode}

% End of document body:
%    \begin{macrocode}
\end{document}
%    \end{macrocode}
%\iffalse
%</samplemain>
%\fi
%
% %%%%%%%%%%%%%%%%%%%%%%%%%%%%%%%%%%%%%%
% \paragraph{Chapter Include Files.}
%
% The include files are called |cdocsch1.tex| and |cdocsch2.tex|.
%
%\iffalse
%<*samplechap1|samplechap2>
%\fi

% Optional override for |\version| flag:
%    \begin{macrocode}
%%\providecommand{\version}{final}
%    \end{macrocode}

% Include the main document:
%    \begin{macrocode}
\input{childdoc.def}
\childdocof{cdocsamp}
%    \end{macrocode}

%\iffalse
%</samplechap1|samplechap2>
%\fi
%
%\iffalse
%<*samplechap1>
%\fi
% Some text for chapter 1:
%    \begin{macrocode}
\section{one}
some text in chapter one
%    \end{macrocode}

%\iffalse
%</samplechap1>
%\fi
% Some text for chapter 2:
%\iffalse
%<*samplechap2>
%\fi
%    \begin{macrocode}
\section{two}
more text in chapter two
%    \end{macrocode}

%\iffalse
%</samplechap2>
%\fi
%
% %%%%%%%%%%%%%%%%%%%%%%%%%%%%%%%%%%%%%%
% \paragraph{Part Include Files.}
%
% The include files are called |cdocspt3.tex| and |cdocspt4.tex|.
%
%\iffalse
%<*samplepart3|samplepart4>
%\fi

% Optional override for |\version| flag:
%    \begin{macrocode}
%%\providecommand{\version}{final}
%    \end{macrocode}

% Include the main document:
%    \begin{macrocode}
\input{childdoc.def}
\childdocby{cdocsamp}
%    \end{macrocode}

%\iffalse
%</samplepart3|samplepart4>
%\fi
%
%\iffalse
%<*samplepart3>
%\fi
% Some text for part 3:
%    \begin{macrocode}
some text in part three
%    \end{macrocode}

%\iffalse
%</samplepart3>
%\fi
% Some text for part 4:
%\iffalse
%<*samplepart4>
%\fi
%    \begin{macrocode}
more text in part four
%    \end{macrocode}

%\iffalse
%</samplepart4>
%\fi
%
% %%%%%%%%%%%%%%%%%%%%%%%%%%%%%%%%%%%%%%
% \paragraph{Forwarding for a Complete Draft.}
%
% The following forwarding file |cdocsdrf.tex|
% compiles the main document in draft mode:
%\iffalse
%<*sampledraft>
%\fi
%    \begin{macrocode}
\def\version{draft}
\input{childdoc.def}
\childdocforward{cdocsamp}
%    \end{macrocode}

%\iffalse
%</sampledraft>
%\fi
%
% %%%%%%%%%%%%%%%%%%%%%%%%%%%%%%%%%%%%%%
% \paragraph{Forwarding for Final Version of the Chapters.}
%
% The following forwarding files |cdocsfn1.tex| and |cdocsfn2.tex|
% (with identical content)
% compile the final versions of the child documents
% |cdocsch1.tex| and |cdocsch2.tex|, respectively:
%\iffalse
%<*samplefinal>
%\fi
%    \begin{macrocode}
\def\version{final}
\input{childdoc.def}
\childdocforwardprefix[cdocsamp]{cdocsfn}{cdocsch}
%    \end{macrocode}

%\iffalse
%</samplefinal>
%\fi
%
% %%%%%%%%%%%%%%%%%%%%%%%%%%%%%%%%%%%%%%
% \paragraph{Command Line Processing.}
%
% The following three command lines generate the output files
% |cdocscld|, |cdocscl1| and |cdocscl2|
% which should be identical to
% |cdocsdrf|, |cdocsch1| and |cdocsfn2|, respectively:
% \begin{center}
% \begin{tabular}{l}
% |latex -jobname cdocscld \|\\
% |  "\def\version{draft}\input{childdoc.def}\childdocforward{cdocsamp}"|\\
% |latex -jobname cdocscl1 \|\\
% |  "\input{childdoc.def}\childdocforward[cdocsamp]{cdocsch1}"|\\
% |latex -jobname cdocscl2 \|\\
% |  "\def\version{final}\input{childdoc.def}\childdocforward{cdocsch2}"|
% \end{tabular}
% \end{center}
% Note that the trailing backslash on each first line
% merely continues the input to the second line
% (for convenient cut ant paste).
% Furthermore, the command |latex| can be replaced by any
% of its alternative versions such as |pdflatex|.
%
% %%%%%%%%%%%%%%%%%%%%%%%%%%%%%%%%%%%%%%%%%%%%%%%%%%%%%%%%%%%%%%%%%%%%%%%%%%%%%%
% %%%%%%%%%%%%%%%%%%%%%%%%%%%%%%%%%%%%%%%%%%%%%%%%%%%%%%%%%%%%%%%%%%%%%%%%%%%%%%
% \section{Implementation}
%\iffalse
%<*package>
%\fi
%
% This section describes the definitions file |childdoc.def|.

% The definitions cannot be loaded using |\usepackage| or |\RequirePackage|
% which has a mechanism to prevent loading a style file more than once.
% When loading the definitions by means of |\input|
% multiple instances have to be prevented manually:
%\iffalse
%This code needs to be before the `\ProvidesFile' directive
%which is defined at the beginning of this file.
%Therefore it is also placed there and commented out here.
%</package>
%<*discard>
%\fi
%    \begin{macrocode}
\ifdefined\childdocmain\endinput\fi
%    \end{macrocode}
%\iffalse
%</discard>
%<*package>
%\fi
%
% \macro{\ifchilddoc}
% \macro{\ifchilddocmanual}
% The conditional |\ifchilddoc| tells whether a
% child (true) or main (false) document is being compiled.
% The conditional |\ifchilddocmanual| tells whether
% the |\includeonly| mechanism is used (false) or
% the selection of child files must be performed manually (true).
% The definitions initialise to false:
%    \begin{macrocode}
\newif\ifchilddoc
\newif\ifchilddocmanual
%    \end{macrocode}

% \macro{\childdocname}
% \macro{\childdocjob}
% The macro |\childdocname| stores the name of the main document
% to be compiled. The macro |\childdocjob| stores the name of
% the document on which the \LaTeX{} compiler was originally invoked.
% The content of |\jobname| cannot be compared
% to filenames specified in the source due to different catcodes.
% The following code rescans |\jobname|, stores the result
% in |\childdocname| and saves a copy in |\childdocjob|:
%    \begin{macrocode}
\edef\childdocname{\scantokens\expandafter{\jobname\noexpand}}
\let\childdocjob\childdocname
%    \end{macrocode}

% \macro{\childdocdisable}
% The macro |\childdocdisable| prevents the main file
% from being processed more than once.
% At this stage, the main document command |\childdocmain|
% is assumed to be called once again where it should do nothing.
% Any subsequent call to it should prevent
% a secondary processing of the main document
% It overwrites the forwarding commands
% |\childdocof| and |\childdocforward|
% with empty macros to prevent further inclusions of the main document:
%    \begin{macrocode}
\newcommand{\childdocdisable}
{
  \renewcommand{\childdocmain}[1]{\renewcommand{\childdocmain}[1]{\endinput}}
  \renewcommand{\childdocof}[1]{}
  \renewcommand{\childdocby}[2][]{}
  \renewcommand{\childdocforward}[2][]{}
  \renewcommand{\childdocdisable}{}
}
%    \end{macrocode}

% \macro{\childdocmain}
% The macro |\childdocmain| is to be called at the top of the main file
% with nothing or the main filename (without extension) as argument.
% First, it breaks loops.
% If the argument is not empty and does not match |\childdocname|
% (which is set by the first inclusion of |childdoc.def|),
% |\ifchilddoc| is set to true, |\includeonly| is applied to the child file
% and |\jobname| is set to the main file
% (for proper handling of |.aux| files):
%    \begin{macrocode}
\newcommand{\childdocmain}[1]
{
  \childdocdisable\childdocmain{}
  \if?#1?\else
    \begingroup
      \def\childdoctmp{#1}
      \ifx\childdoctmp\childdocname
        \def\childdoctmp{}
      \else
        \def\childdoctmp
        {
          \childdoctrue
          \includeonly{\childdocname}
          \def\childdocjob{#1}
          \def\jobname{#1}
        }
      \fi
      \expandafter
    \endgroup
    \childdoctmp
  \fi
}
%    \end{macrocode}

% \macro{\childdocof}
% The command |\childdocof| redirects
% compilation to the main file |#1|.
%    \begin{macrocode}
\newcommand{\childdocof}[1]
{
  \childdocdisable
  \childdoctrue
  \includeonly{\childdocname}
  \def\jobname{#1}
  \def\childdocjob{#1}
  \input{#1}
}
%    \end{macrocode}

% \macro{\childdocby}
% The command |\childdocby| ....
%    \begin{macrocode}
\newcommand{\childdocby}[2][]
{
  \childdocdisable
  \childdoctrue
  \childdocmanualtrue
  \if?#1?\else
    \def\jobname{#2}
  \fi
  \def\childdocjob{#2}
  \input{#2}
  \endinput
}
%    \end{macrocode}

% \macro{\childdocforward}
% The command |\childdocforward| redirects
% compilation to the main file or
% (if the optional argument is given) a child file.
% Parameters are set as if the main file
% or a child file starting with |\childdocof| was compiled.
% Then compilation is handed over to the main file:
%    \begin{macrocode}
\newcommand{\childdocforward}[2][]
{
  \begingroup
    \if?#1?
      \def\childdoctmp
      {
        \def\childdocname{#2}
        \def\childdocjob{#2}
        \def\jobname{#2}
        \input{#2}
        \endinput
      }
    \else
      \def\childdoctmp
      {
        \childdocdisable
        \def\childdocname{#2}
        \childdoctrue
        \includeonly{#2}
        \def\childdocjob{#1}
        \def\jobname{#1}
        \input{#1}
        \endinput
      }
    \fi
    \expandafter
  \endgroup
  \childdoctmp
}
%    \end{macrocode}

% \macro{\childdocforwardprefix}
% The command |\childdocforwardprefix| redirects
% compilation to the main or a child file by means of a pattern.
% The prefix |#1| in the current filename is replaced by |#2|
% and the suffix of the current filename is kept
% (it is assumed that the filename does not contain the substring `|~~~|'
% which is used as a delimiter).
% Compilation is handed over to the new file by |\childdocforward|:
%    \begin{macrocode}
\newcommand{\childdocforwardprefix}[3][]
{
  \begingroup
    \def\childdocextract #2##1~~~{\def\childdoctmp{\childdocforward[#1]{#3##1}}}
    \expandafter\childdocextract\childdocname~~~
    \expandafter
  \endgroup
  \childdoctmp
}
%    \end{macrocode}

% \macro{\childdoc}
% The deprecated macro |\childdoc| is a legacy version of |\childdocmain|:
%    \begin{macrocode}
\newcommand{\childdoc}{\childdocmain}
%    \end{macrocode}

% \macro{\childdocredirect}
% The deprecated macro |\childdocredirect| is a legacy version
% of |\childdocforward| and |\childdocforwardprefix|:
%    \begin{macrocode}
\newcommand{\childdocredirect}[2][]
{
  \begingroup
    \if?#1?
      \def\childdoctmp{\childdocforward{#2}}
    \else
      \def\childdoctmp{\childdocforwardprefix{#1}{#2}}
    \fi
    \expandafter
  \endgroup
  \childdoctmp
}
%    \end{macrocode}

%\iffalse
%</package>
%\fi
%
\endinput

\childdocmain{}
%    \end{macrocode}

% Optional override for |\version| flag:
%    \begin{macrocode}
%%\ifchilddoc\else\providecommand{\version}{draft}\fi
%    \end{macrocode}

% Define the default values for the |\version| flag
% (|final| for the main file and |draft| for childs):
%    \begin{macrocode}
\ifchilddoc
\providecommand{\version}{draft}
\else
\providecommand{\version}{final}
\fi
%    \end{macrocode}

% Load the standard document class:
%    \begin{macrocode}
\documentclass[12pt]{article}
%    \end{macrocode}

% Start the document body:
%    \begin{macrocode}
\begin{document}
%    \end{macrocode}

% Declare a title page.
% Print title, part of document being processed and version flag:
%    \begin{macrocode}
\addtocounter{page}{-1}
\begin{center}
{\LARGE\bfseries{}childdoc example\par}
\vspace{1cm}
\ifchilddoc
\ifchilddocmanual part\else chapter\fi:
`\childdocname' of `\childdocjob'\par
\else
main document: `\childdocjob'\par
\fi
version: \version\par
\end{center}
\newpage
%    \end{macrocode}

% Manually include selected file,
% otherwise process as usual:
%    \begin{macrocode}
\ifchilddocmanual
\section*{part `\childdocname'}
\input{\childdocname}
\else
%    \end{macrocode}

% Include the two chapters:
%    \begin{macrocode}
\include{cdocsch1}
\include{cdocsch2}
%    \end{macrocode}

% Include the two parts unless only chapters should be displayed:
%    \begin{macrocode}
\ifchilddoc\else
\section{part three}
\input{cdocspt3}
\section{part four}
\input{cdocspt4}
\fi
%    \end{macrocode}

% Process as usual until here:
%    \begin{macrocode}
\fi
%    \end{macrocode}

% End of document body:
%    \begin{macrocode}
\end{document}
%    \end{macrocode}
%\iffalse
%</samplemain>
%\fi
%
% %%%%%%%%%%%%%%%%%%%%%%%%%%%%%%%%%%%%%%
% \paragraph{Chapter Include Files.}
%
% The include files are called |cdocsch1.tex| and |cdocsch2.tex|.
%
%\iffalse
%<*samplechap1|samplechap2>
%\fi

% Optional override for |\version| flag:
%    \begin{macrocode}
%%\providecommand{\version}{final}
%    \end{macrocode}

% Include the main document:
%    \begin{macrocode}
% \iffalse
%
% childdoc.dtx Copyright (C) 2017-2018 Niklas Beisert
%
% This work may be distributed and/or modified under the
% conditions of the LaTeX Project Public License, either version 1.3
% of this license or (at your option) any later version.
% The latest version of this license is in
%   http://www.latex-project.org/lppl.txt
% and version 1.3 or later is part of all distributions of LaTeX
% version 2005/12/01 or later.
%
% This work has the LPPL maintenance status `maintained'.
%
% The Current Maintainer of this work is Niklas Beisert.
%
% This work consists of the files childdoc.dtx and childdoc.ins
% and the derived files childdoc.def and cdocsamp.tex with
% cdocsch1.tex, cdocsch2.tex, cdocsdrf.tex, cdocsfn1.tex, cdocsfn2.tex.
%
%<package>\ifdefined\childdocmain\endinput\fi
%<package>\ProvidesFile{childdoc.def}[2018/12/30 v2.0 child document driver]
%<samplemain>\ProvidesFile{cdocsamp.tex}[2018/12/30 v2.0 sample for childdoc]
%<*driver>
%\ProvidesFile{childdoc.drv}[2018/12/30 v2.0 childdoc reference manual file]
\PassOptionsToClass{10pt,a4paper}{article}
\documentclass{ltxdoc}

\usepackage[margin=35mm]{geometry}
\usepackage{hyperref}
\usepackage{hyperxmp}
\usepackage[usenames]{color}

\hypersetup{colorlinks=true}
\hypersetup{pdfstartview=FitH}
\hypersetup{pdfpagemode=UseNone}
\hypersetup{pdfsource={}}
\hypersetup{pdflang={en-UK}}
\hypersetup{pdfcopyright={Copyright 2017-2018 Niklas Beisert.
  This work may be distributed and/or modified under the
  conditions of the LaTeX Project Public License, either version 1.3
  of this license or (at your option) any later version.}}
\hypersetup{pdflicenseurl={http://www.latex-project.org/lppl.txt}}
\hypersetup{pdfcontactaddress={ETH Zurich, ITP, HIT K,
  Wolfgang-Pauli-Strasse 27}}
\hypersetup{pdfcontactpostcode={8093}}
\hypersetup{pdfcontactcity={Zurich}}
\hypersetup{pdfcontactcountry={Switzerland}}
\hypersetup{pdfcontactemail={nbeisert@itp.phys.ethz.ch}}
\hypersetup{pdfcontacturl={http://people.phys.ethz.ch/\xmptilde nbeisert/}}

\newcommand{\secref}[1]{\hyperref[#1]{section \ref*{#1}}}

\parskip1ex
\parindent0pt
\let\olditemize\itemize
\def\itemize{\olditemize\parskip0pt}

\begin{document}

\title{The \textsf{childdoc} Package}
\hypersetup{pdftitle={The childdoc Package}}
\author{Niklas Beisert\\[2ex]
  Institut f\"ur Theoretische Physik\\
  Eidgen\"ossische Technische Hochschule Z\"urich\\
  Wolfgang-Pauli-Strasse 27, 8093 Z\"urich, Switzerland\\[1ex]
  \href{mailto:nbeisert@itp.phys.ethz.ch}
  {\texttt{nbeisert@itp.phys.ethz.ch}}}
\hypersetup{pdfauthor={Niklas Beisert}}
\hypersetup{pdfsubject={Manual for the LaTeX2e Package childdoc}}
\date{30 December 2018, \textsf{v2.0}}
\maketitle

\begin{abstract}\noindent
\textsf{childdoc} is a \LaTeXe{} package
that enables the direct compilation
of document sections included by |\include|
to individual files.
\end{abstract}

\begingroup
\parskip0ex
\tableofcontents
\endgroup

%%%%%%%%%%%%%%%%%%%%%%%%%%%%%%%%%%%%%%%%%%%%%%%%%%%%%%%%%%%%%%%%%%%%%%%%%%%%%%%%
%%%%%%%%%%%%%%%%%%%%%%%%%%%%%%%%%%%%%%%%%%%%%%%%%%%%%%%%%%%%%%%%%%%%%%%%%%%%%%%%
\section{Introduction}

\LaTeX{} provides a mechanism to structure a large document (such as a book)
into a main file and several child files (containing the chapters)
using the |\include| command.
This mechanism is beneficial for documents
which span hundreds of pages in order to
make the source file(s) more manageable.
Moreover, compilation can be restricted to
selected child files by means of the |\includeonly| command.
The latter feature can be used to reduce the compilation time while editing
(this was significantly more useful in the earlier days of \LaTeX{})
or to generate a smaller document which is easier to navigate.
Another application of |\includeonly| is to generate
documents consisting of selected parts of the complete document.

However, there are a few drawbacks of the plain |\include| mechanism:
\begin{itemize}
\item
The child files cannot be compiled on their own,
they can only be compiled via the main file.
A naive editing environment
(such as a text editor with an option
to have the current file processed by \LaTeX)
may require one to switch to the main file before compiling;
attempting to compile the child file produces errors.
\item
The main file must be modified (each time)
to adjust the |\includeonly| command
to the present needs. This easily leaves the main file in a messy state.
\item
The generated document will always carry the filename
of the main document. This is inconvenient if
several child files are to be compiled and
to be kept for distribution.
\end{itemize}

The present package provides a simple interface
to make child files individually compilable by \LaTeX{}.
Compiling a child file then has the same effect as compiling
the main file with an |\includeonly| command
to select the appropriate child.
Moreover the generated document will carry the name of the child
rather than the main file.
This resolves all three above issues.

This feature is meant to make the editing of books,
thesis documents and lecture notes somewhat more convenient.
However, the package can also be used efficiently for
composing a series of documents (such as exercise sheets)
which are typically distributed individually.
It then assists the author in generating the individual documents
(potentially in different versions)
as well as a document containing the collected series.
Another application is in developing style files
or other kinds of included material
where compilation of the style file could redirect
to a sample or test file.

%%%%%%%%%%%%%%%%%%%%%%%%%%%%%%%%%%%%%%%%%%%%%%%%%%%%%%%%%%%%%%%%%%%%%%%%%%%%%%%%
%%%%%%%%%%%%%%%%%%%%%%%%%%%%%%%%%%%%%%%%%%%%%%%%%%%%%%%%%%%%%%%%%%%%%%%%%%%%%%%%
\section{Usage}

First of all, the package \textsf{childdoc} is \emph{not} a standard
\LaTeXe{} |.sty| style file! Therefore it needs to be invoked in
a non-standard way.

%%%%%%%%%%%%%%%%%%%%%%%%%%%%%%%%%%%%%%%%%%%%%%%%%%%%%%%%%%%%%%%%%%%%%%%%%%%%%%%%
\subsection{Included Files}
\label{sec:include}

%%%%%%%%%%%%%%%%%%%%%%%%%%%%%%%%%%%%%%%%
\DescribeMacro{\childdocmain}
To use the package, add the commands
\begin{center}
\begin{tabular}{l}
|\input{childdoc.def}|\\
|\childdocmain{}|\\
\end{tabular}
\end{center}
at the very top of the main \LaTeX{} file,
in particular \emph{before} the |\documentclass| statement!
The argument of |\childdocmain| should be left empty
(but it must be present).

%%%%%%%%%%%%%%%%%%%%%%%%%%%%%%%%%%%%%%%%
\DescribeMacro{\childdocof}
Furthermore, add the commands
\begin{center}
\begin{tabular}{l}
|\input{childdoc.def}|\\
|\childdocof{|\textit{main}|}|\\
\end{tabular}
\end{center}
at the top of every child file \textit{child}
which is included by |\include{|\textit{child}|}|
from within the main file
(or at least for those files to be compiled individually).
The argument \textit{main} must be the filename of the main file.

There are a couple of
considerations in setting up the main and child documents:

%%%%%%%%%%%%%%%%%%%%%%%%%%%%%%%%%%%%%%%%
\paragraph{Restrictions.}

Please note the following restrictions:
\begin{itemize}
\item
|\childdocmain| must be called with one argument \textit{main}
to ensure compatibility with earlier version of the package.
It must either be empty (|\childdocmain{}|)
or precisely match the filename of the main file in which it is specified.
See \secref{sec:detection} for further information.
\item
The filename \textit{main} must be specified without the |.tex| extension.
\item
The filename \textit{main} is case sensitive
(even in case-insensitive file systems)
due to internal string comparison.
\item
The argument \textit{main} should be fully expanded, it cannot be a macro.
\item
Subdirectories and special characters should be avoided in filenames.
\item
The command |\childdocmain{|\textit{main}|}| must be followed by a whitespace.
It should not be followed immediately by another command
or by a comment mark `|%|'.
This is because the \TeX{} parser reads the token immediately following
the argument of |\childdocmain| and puts it
at the beginning of every child section;
however, a white\-space is ignored.
\end{itemize}

%%%%%%%%%%%%%%%%%%%%%%%%%%%%%%%%%%%%%%%%
\paragraph{Content of Main File.}

It is advisable to place all content in the child files included by |\include|.
Any output contained in the main file will appear in all child documents
unless suppressed manually;
it cannot be suppressed automatically by the |\includeonly| directive
and thus should normally be avoided.
A method to include some content in the main file
by means of conditional processing is described in \secref{sec:conditional}.

%%%%%%%%%%%%%%%%%%%%%%%%%%%%%%%%%%%%%%%%
\paragraph{Page Numbering.}

When only a part of the document is compiled,
the appropriate numbering of pages
(as well as other status parameters)
is determined from the |.aux| files.
The latter contain information from previous passes.
However this information needs to propagate through
all intermediate child documents.
Therefore the page numbering in child documents may well
be inconsistent until the complete document is compiled at least once.

A useful (if unconventional) way to always ensure a consistent
page numbering is to restart the numbering in each child document
and denote the pages by `\textit{child}|.|\textit{page}'
where \textit{child} represents the chapter/section number of the child file.
This can be achieved by the command
|\numberwithin{page}{|\textit{child}|}|
of the \textsf{amsmath} package
where \textit{child} can be |chapter| or |section|
depending on the chosen structuring.
Alternatively, one can modify the macro |\thepage| appropriately
and reset the counter |page| at the start of each child file.

%%%%%%%%%%%%%%%%%%%%%%%%%%%%%%%%%%%%%%%%%%%%%%%%%%%%%%%%%%%%%%%%%%%%%%%%%%%%%%%%
\subsection{Conditional Processing}
\label{sec:conditional}

The package provides a mechanism to compile different versions
of a document. To customise the versions further some conditional processing
can come in handy to distinguish which version is being compiled.
The package provides two macros to describe the compilation context:

%%%%%%%%%%%%%%%%%%%%%%%%%%%%%%%%%%%%%%%%
\DescribeMacro{\ifchilddoc}
The conditional |\ifchilddoc| distinguishes between the compilation of
child documents and the main document:
%
\begin{center}
|\ifchilddoc |\textit{child-code}| |[|\||else |\textit{main-code}]| \||fi|
\end{center}

%%%%%%%%%%%%%%%%%%%%%%%%%%%%%%%%%%%%%%%%
\DescribeMacro{\childdocname}
\DescribeMacro{\childdocjob}
The macro |\childdocname| contains the filename (without extension)
of the main or child file being processed.
Note that |\childdocjob| will always contain the name of the main file.

%%%%%%%%%%%%%%%%%%%%%%%%%%%%%%%%%%%%%%%%
\paragraph{Title Page.}

Conditional processing can be used to include a title or banner page
in the main document when proper precautions are taken.
Importantly, the code in the main file should ensure that the page counter
(as well as other status parameters which are stored in the |.aux| files)
takes the same value after the conditional processing.
Otherwise the page numbers may take divergent values
depending on which part is compiled.

For example, a title page could be declared by:
%
\begin{center}
\begin{tabular}{l}
|\ifchilddoc\||else|\\
|\addtocounter{page}{-1}|\\
\textit{code for title page}\\
|\newpage|\\
|\||fi|
\end{tabular}
\end{center}
%
A banner page for the child documents can be generated by:
%
\begin{center}
\begin{tabular}{l}
|\ifchilddoc|\\
|\addtocounter{page}{-1}|\\
\textit{code for banner page}\\
|\newpage|\\
|\||fi|
\end{tabular}
\end{center}
%
Here one could write a message such as:
\begin{center}
|This is the part \childdocname{} of \childdocjob{}.|
\end{center}

%%%%%%%%%%%%%%%%%%%%%%%%%%%%%%%%%%%%%%%%%%%%%%%%%%%%%%%%%%%%%%%%%%%%%%%%%%%%%%%%
\subsection{Flags}
\label{sec:flags}

The package makes it easy to generate different versions
of the main or child documents.
To this end compilation flags can be defined
and assigned different default values.
They will be particularly useful in conjunction
with the forwarding mechanism described in \secref{sec:forward}.

For example, it may be useful to have a flag |\version|
which can be set to |draft| or |final|.
The document source will contain some conditional code
depending on the value of |\version|.
Suppose further, the flag should default to |final| for the main file
and to |draft| for child files
which is a natural assignment for editing the document.
This is achieved by placing the following code
in the preamble of the main document
(below the |\childdocmain| directive):
%
\begin{center}
\begin{tabular}{l}
|\ifchilddoc|\\
|\providecommand{\version}{draft}|\\
|\||else|\\
|\providecommand{\version}{final}|\\
|\||fi|
\end{tabular}
\end{center}
%
The definition by |\providecommand| makes sure
that previous definitions are not overwritten.
Further statements |\providecommand{\version}{...}|
can thus be added before the above code to override it.

For the main file, one might add a line
(between |\childdocmain| and the above block)
%
\begin{center}
|%\ifchilddoc\||else\providecommand{\version}{draft}\||fi|
\end{center}
%
which can be uncommented to produce a draft version.
Likewise one can add a line to the very top of a child file
(above the |\childdocof{|\textit{main}|}| directive)
%
\begin{center}
|%\providecommand{\version}{final}|
\end{center}
%
which can be uncommented to produce the final version of this child document.

%%%%%%%%%%%%%%%%%%%%%%%%%%%%%%%%%%%%%%%%%%%%%%%%%%%%%%%%%%%%%%%%%%%%%%%%%%%%%%%%
\subsection{Forwarding}
\label{sec:forward}

Different versions of the main or child documents
using compilation flags as described in \secref{sec:flags}
can be (permanently) stored in different files
for convenient compilation, viewing and distribution.
To this end, the package defines a command
to pass on compilation to a different file:

%%%%%%%%%%%%%%%%%%%%%%%%%%%%%%%%%%%%%%%%
\DescribeMacro{\childdocforward}
The command |\childdocforward| redirects processing to
another source file:
%
\begin{center}
\begin{tabular}{l}
|\input{childdoc.def}|\\
|\childdocforward[|\textit{main}|]{|\textit{dest}|}|\\
\end{tabular}
\end{center}
%
The argument \textit{dest} is the destination file
(without extension).
It should be the main file or one of the child files.
Note that further \textsf{childdoc} directives
such as |\childdocof| and |\childdocforward|
in the indicated file will be processed in this form.
The optional argument \textit{main}
passes on directly to the main file \textit{main}
while pretending to compile the child \textit{dest}.
This form behaves as if \textit{dest}
issues |\childdocof{|\textit{main}|}| right away,
and no further \textsf{childdoc} directives will be processed.

%%%%%%%%%%%%%%%%%%%%%%%%%%%%%%%%%%%%%%%%
\DescribeMacro{\...prefix}
In the alternative form |\childdocforwardprefix|,
%
\begin{center}
\begin{tabular}{l}
|\input{childdoc.def}|\\
|\childdocforwardprefix[|\textit{main}|]{|\textit{prefix}|}{|\textit{dest}|}|
\end{tabular}
\end{center}
%
the destination file is determined by a pattern
depending on the current file:
To make this work, the current file must be called
`{\textit{prefix}\hspace{0.2em}\textit{suffix}}'
with \textit{prefix} matching precisely the argument.
Processing is then passed on to the file
`{\textit{dest}\hspace{0.2em}\textit{suffix}}'.
Surely, the same effect is achieved by
directly specifying the
argument `{\textit{dest}\hspace{0.2em}\textit{suffix}}'
in the first form.
However, that requires to set up a different file
for each child. With the alternative form of the command
all these files can have exactly the same content
which simplifies setting them up and maintaining them.

For example, the following file |draft.tex|
with a compilation flag |\version| as described in \secref{sec:flags}
compiles the main document as a draft:
%
\begin{center}
\begin{tabular}{l}
|\def\version{draft}|\\
|\input{childdoc.def}|\\
|\childdocforward{|\textit{main}|}|
\end{tabular}
\end{center}
%
Likewise, the following files |final|\textit{nn}|.tex|
compile the final version of the child document
|child|\textit{nn}|.tex|:
%
\begin{center}
\begin{tabular}{l}
|\def\version{final}|\\
|\input{childdoc.def}|\\
|\childdocforwardprefix{final}{child}|
\end{tabular}
\end{center}
%

Note that when several versions of a main file and/or of each child file
are to be generated, it may be convenient to set up a |Makefile| or
shell script to automatise the process.

%%%%%%%%%%%%%%%%%%%%%%%%%%%%%%%%%%%%%%%%%%%%%%%%%%%%%%%%%%%%%%%%%%%%%%%%%%%%%%%%
\subsection{Command Line Processing}
\label{sec:commandline}

The effect of redirection files can also be achieved by invoking
the \LaTeX{} compiler with a more elaborate command line.
Most conveniently this should be done as part
of a shell script or a |Makefile|.

When using \textsf{childdoc} in the main file, the following
command lines effectively perform a redirection
(note that depending on the shell being used,
backslashes may have to be doubled: `|\|' $\to$ `|\\|'):
%
\begin{center}
|... -jobname "|\textit{target}|" |\\|"|[\textit{flags}]%
|\input{childdoc.def}\childdocforward[|\textit{main}|]{|\textit{dest}|}"|
\end{center}
%
Here \textit{target} is the name of the output file,
\textit{main} is the name of the main file
and \textit{dest} is the name of the main or child file to be processed
(all filenames without extensions).
The optional argument \textit{main} can be omitted
if \textit{main} matches \textit{dest}.
Optionally, compilation \textit{flags} can be defined via |\def| commands.
This command line makes the \TeX{} engine believe
it is compiling the file \textit{target}
whose content is specified as the latter parameter.
The provided code then forwards the processing to
\textit{main} or \textit{dest} as described in \secref{sec:forward}.

%%%%%%%%%%%%%%%%%%%%%%%%%%%%%%%%%%%%%%%%%%%%%%%%%%%%%%%%%%%%%%%%%%%%%%%%%%%%%%%%
\subsection{Include by Input}
\label{sec:input}

Including child documents by |\include| has some restrictions by design.
Most notably, the content of a child document always occupies
its own set of pages; pages cannot be shared between child documents.
Usually, this behaviour makes perfect sense
because each child document contain an essential part of the document.
However, in some situations it may be desirable to compose
a document from a collection of parts
without having mandatory page breaks between then.
For this case, the package
provides a mechanism to include parts
by |\input| which can also be processed individually.
However, by construction this mechanism
requires manual handling of the content to be output.

%%%%%%%%%%%%%%%%%%%%%%%%%%%%%%%%%%%%%%%%
\DescribeMacro{\ifchilddocmanual}
The main file should be prepared as usual, see \secref{sec:include}.
However, the document body must make a distinction
between processing of an individual part and of the main document, e.g.:
%
\begin{center}
\begin{tabular}{l}
|\ifchilddocmanual|\\
|\input{\childdocname}|\\
|\||else|\\
\textit{document body with }|\input{|\textit{part}|}|\\
|\||fi|
\end{tabular}
\end{center}
%
The conditional |\ifchilddocmanual| is true whenever
a part to be included by |\input| is being compiled,
and the name of the part is stored in |\childdocname|.

%%%%%%%%%%%%%%%%%%%%%%%%%%%%%%%%%%%%%%%%
\DescribeMacro{\childdocby}
Each part to be included by |\input| should start with:
%
\begin{center}
\begin{tabular}{l}
|\input{childdoc.def}|\\
|\childdocby{|\textit{main}|}|\\
\end{tabular}
\end{center}
%
The directive |\childdocby| is similar to |\childdocof|
described in \secref{sec:include},
but the subsequent selection of content must be done manually.
To that end, both |\ifchilddoc| and |\ifchilddocmanual|
will be true upon processing of a part,
and the name of the part is stored in |\childdocname|.
Note that |\jobname| will be set to the filename of the current part
so that each part receives an individual |.aux| file
that does not interfere with the |.aux| file(s) of the main document.
This behaviour can be altered by the alternative form
|\childdocby[*]{|\textit{main}|}| (with a non-empty optional argument)
which uses the |.aux| file of the main document
by setting |\jobname| to \textit{main}.

%%%%%%%%%%%%%%%%%%%%%%%%%%%%%%%%%%%%%%%%%%%%%%%%%%%%%%%%%%%%%%%%%%%%%%%%%%%%%%%%
\subsection{Driver Development}
\label{sec:driver}

The \textsf{childdoc} mechanism can also be use for the development
of definition files such as \LaTeX{} styles or classes.
This case differs from the above setup with multiple parts
included by |\include| in that no |\includeonly| should be invoked.
This can be achieved by starting the include file
(before |\ProvidesPackage|) with:
%
\begin{center}
\begin{tabular}{l}
|\input{childdoc.def}|\\
|\childdocforward{|\textit{main}|}|\\
\end{tabular}
\end{center}
%
or alternatively with:
%
\begin{center}
\begin{tabular}{l}
|\input{childdoc.def}|\\
|\childdocby{|\textit{main}|}|\\
\end{tabular}
\end{center}
%
Both forms have slightly different effects as described above.
The main file is prepared as usual, see \secref{sec:include}.

%%%%%%%%%%%%%%%%%%%%%%%%%%%%%%%%%%%%%%%%%%%%%%%%%%%%%%%%%%%%%%%%%%%%%%%%%%%%%%%%
\subsection{Legacy Detection}
\label{sec:detection}

The directive |\childdocmain| in the main file can detect
whether the complete document or merely a child is to be compiled
even without using the directive |\childdocof|.
This method is deprecated because it is less robust
and there is no compelling reason to use it;
it is merely provided for backward compatibility
and it may be removed in future versions.

If the detection mechanism is to be used,
it is mandatory to correctly specify
the filename of the main file as the argument of |\childdocmain|:
%
\begin{center}
\begin{tabular}{l}
|\input{childdoc.def}|\\
|\childdocmain{|\textit{main}|}|\\
\end{tabular}
\end{center}
%
If |\jobname| does not match the argument \textit{main} of |\childdocmain|,
it is assumed that |\jobname| points to the child file to be compiled.
When using |\childdocmain| with the main file specified as argument,
it suffices to start a child file
with just |\input{|\textit{main}|}|
without loading of the package and using |\childdocof|.
If instead all processing is done
with the appropriate \textsf{childdoc} directives,
the argument of \textit{main} of |\childdocmain| can be empty.

An alternative version of the command line processing described
in \secref{sec:commandline} using the detection mechanism reads:
%
\begin{center}
|... -jobname "|\textit{target}|" "|[\textit{flags}]%
[|\def\jobname{|\textit{dest}|}|]|\input{|\textit{main}|}"|
\end{center}

%%%%%%%%%%%%%%%%%%%%%%%%%%%%%%%%%%%%%%%%%%%%%%%%%%%%%%%%%%%%%%%%%%%%%%%%%%%%%%%%
\subsection{Manual Code}
\label{sec:manual}

In case one cannot be certain whether the definitions file |childdoc.def|
is installed on the target \TeX{} distribution
and one prefers not to ship it,
it is conceivable to paste a few relevant commands into the sources.

To that end, drop all statements |\input{childdoc.def}|
and perform the replacements as outlined below.
Instead of |\childdocmain{|\textit{main}|}| add the following code
to the top of the main file:
%
\begin{center}
\begin{tabular}{l}
|\||ifdefined\childdocname\endinput\||fi\newif\ifchilddoc|\\
|\edef\childdocname{\scantokens\expandafter{\jobname\noexpand}}|\\
|\def\childdocmain{|\textit{main}|}\||ifx\childdocmain\childdocname\||else|\\
|\childdoctrue\includeonly{\childdocname}\let\jobname\childdocmain\||fi|\\
\end{tabular}
\end{center}
%
Instead of |\childdocof{|\textit{main}|}| just include the main file
at the top of each child file:
%
\begin{center}
|\input{|\textit{main}|}|
\end{center}
%
A simple redirection |\childdocforward{|\textit{dest}|}| is achieved by:
%
\begin{center}
|\def\jobname{|\textit{dest}|}\input{\jobname}|
\end{center}
%
The redirection with prefix
|\childdocforwardprefix[|\textit{prefix}|]{|\textit{dest}|}|
is accomplished by:
%
\begin{center}
\begin{tabular}{l}
|{\edef\jobname{\scantokens\expandafter{\jobname\noexpand}}|\\
|\def\redirectjob |\textit{prefix}|#1~~~{\gdef\jobname{|\textit{dest}|#1}}|\\
|\expandafter\redirectjob\jobname~~~}\input{\jobname}|
\end{tabular}
\end{center}

In an alternative approach,
child documents can be compiled by a specific command line
without additional code or specific definitions:
%
\begin{center}
|... -jobname "|\textit{target}|" "|[\textit{flags}]%
|\includeonly{|\textit{dest}|}\input{|\textit{main}|}"|
\end{center}
%

%%%%%%%%%%%%%%%%%%%%%%%%%%%%%%%%%%%%%%%%%%%%%%%%%%%%%%%%%%%%%%%%%%%%%%%%%%%%%%%%
%%%%%%%%%%%%%%%%%%%%%%%%%%%%%%%%%%%%%%%%%%%%%%%%%%%%%%%%%%%%%%%%%%%%%%%%%%%%%%%%
\section{Information}

%%%%%%%%%%%%%%%%%%%%%%%%%%%%%%%%%%%%%%%%%%%%%%%%%%%%%%%%%%%%%%%%%%%%%%%%%%%%%%%%
\subsection{Copyright}

Copyright \copyright{} 2017--2018 Niklas Beisert

This work may be distributed and/or modified under the
conditions of the \LaTeX{} Project Public License, either version 1.3
of this license or (at your option) any later version.
The latest version of this license is in
  \url{http://www.latex-project.org/lppl.txt}
and version 1.3 or later is part of all distributions of \LaTeX{}
version 2005/12/01 or later.

This work has the LPPL maintenance status `maintained'.

The Current Maintainer of this work is Niklas Beisert.

This work consists of the files |README.txt|, |childdoc.ins| and |childdoc.dtx|
as well as the derived files |childdoc.def|, |cdocsamp.tex|
with |cdocsch1.tex|, |cdocsch2.tex|, |cdocspt3.tex|, |cdocspt4.tex|,
|cdocsdrf.tex|, |cdocsfn1.tex|, |cdocsfn2.tex|
as well as |childdoc.pdf|.

%%%%%%%%%%%%%%%%%%%%%%%%%%%%%%%%%%%%%%%%%%%%%%%%%%%%%%%%%%%%%%%%%%%%%%%%%%%%%%%%
\subsection{Files and Installation}

The package consists of the files:
%
\begin{center}
\begin{tabular}{ll}
    |README.txt|   & readme file \\
    |childdoc.ins| & installation file \\
    |childdoc.dtx| & source file \\
    |childdoc.def| & definition file \\
    |cdocsamp.tex| & sample main file \\
    |cdocsch1.tex| & sample include file \\
    |cdocsch2.tex| & sample include file \\
    |cdocspt3.tex| & sample part file \\
    |cdocspt4.tex| & sample part file \\
    |cdocsdrf.tex| & sample redirection file \\
    |cdocsfn1.tex| & sample redirection file \\
    |cdocsfn2.tex| & sample redirection file \\
    |childdoc.pdf| & manual
\end{tabular}
\end{center}
%
The distribution consists of the files
|README.txt|, |childdoc.ins| and |childdoc.dtx|.
%
\begin{itemize}
\item
Run (pdf)\LaTeX{} on |childdoc.dtx|
to compile the manual |childdoc.pdf| (this file).
\item
Run \LaTeX{} on |childdoc.ins| to create the definitions file |childdoc.def|
and the sample |cdocsamp.tex| with include files
|cdocsch1.tex|, |cdocsch2.tex|, |cdocspt3.tex|, |cdocspt4.tex|,
|cdocsdrf.tex|, |cdocsfn1.tex|, |cdocsfn2.tex|.
Then copy the file |childdoc.def| to an appropriate directory of your \LaTeX{}
distribution, e.g.\ \textit{texmf-root}|/tex/latex/childdoc|.
\end{itemize}

%%%%%%%%%%%%%%%%%%%%%%%%%%%%%%%%%%%%%%%%%%%%%%%%%%%%%%%%%%%%%%%%%%%%%%%%%%%%%%%%
\subsection{Related CTAN Packages}

There are several other packages which offer a similar functionality:
%
\begin{itemize}
\item
The packages
\href{http://ctan.org/pkg/docmute}{\textsf{docmute}},
\href{http://ctan.org/pkg/includex}{\textsf{includex}} and
\href{http://ctan.org/pkg/standalone}{\textsf{standalone}}
provide commands to include only the document body of
a child file thus allowing both files to be compiled individually.
\item
The packages \href{http://ctan.org/pkg/subdocs}{\textsf{subdocs}}
and \href{http://ctan.org/pkg/subfiles}{\textsf{subfiles}}
provide structures in which the main and child documents can be
encapsulated and allowing them to be compiled individually.
The inclusion mechanism is different from the conventional |\include|.
\item
The package \href{http://ctan.org/pkg/combine}{\textsf{combine}}
is an elaborate solution to combine several documents into one.
\end{itemize}
%
See also the CTAN topic \href{http://ctan.org/topic/subdocs}{\textsf{subdocs}}
for further related packages.
The present package differs from the above solutions in that
a document structure constructed with the conventional |\include| mechanism
just needs two extra commands at the top of every file
such that all constituent files can be compiled individually.

%%%%%%%%%%%%%%%%%%%%%%%%%%%%%%%%%%%%%%%%%%%%%%%%%%%%%%%%%%%%%%%%%%%%%%%%%%%%%%%%
%\subsection{Feature Suggestions}
%
%The following is a list of features which may be useful for future
%versions of this package:
%%
%\begin{itemize}
%\item
%\ldots
%\end{itemize}

%%%%%%%%%%%%%%%%%%%%%%%%%%%%%%%%%%%%%%%%%%%%%%%%%%%%%%%%%%%%%%%%%%%%%%%%%%%%%%%%
\subsection{Revision History}

%%%%%%%%%%%%%%%%%%%%%%%%%%%%%%%%%%%%%%%%
\paragraph{v2.0:} 2018/12/30

\begin{itemize}
\item
immediate forward processing
\item
added |\childdocby| mechanism
\item
manual restructured
\end{itemize}

%%%%%%%%%%%%%%%%%%%%%%%%%%%%%%%%%%%%%%%%
\paragraph{v1.6:} 2018/01/17

\begin{itemize}
\item
application for development of include files
\item
corrections to manual
\end{itemize}

%%%%%%%%%%%%%%%%%%%%%%%%%%%%%%%%%%%%%%%%
\paragraph{v1.5:} 2017/05/21

\begin{itemize}
\item
more complete structuring introduced
\item
|\childdocof| introduced
\item
|\childdoc| renamed to |\childdocmain|
\item
|\childredirect| renamed to |\childdocforward| and |\childdocforwardprefix|
and functionality expanded
\end{itemize}

%%%%%%%%%%%%%%%%%%%%%%%%%%%%%%%%%%%%%%%%
\paragraph{v1.0:} 2017/04/27

\begin{itemize}
\item
manual and install package
\item
first version published on CTAN
\end{itemize}

%%%%%%%%%%%%%%%%%%%%%%%%%%%%%%%%%%%%%%%%
\paragraph{v0.6:} 2017/04/26

\begin{itemize}
\item
redirection mechanism added
\end{itemize}

%%%%%%%%%%%%%%%%%%%%%%%%%%%%%%%%%%%%%%%%
\paragraph{v0.5:} 2017/04/26

\begin{itemize}
\item
functionality in definition file
\end{itemize}


%%%%%%%%%%%%%%%%%%%%%%%%%%%%%%%%%%%%%%%%%%%%%%%%%%%%%%%%%%%%%%%%%%%%%%%%%%%%%%%%
%%%%%%%%%%%%%%%%%%%%%%%%%%%%%%%%%%%%%%%%%%%%%%%%%%%%%%%%%%%%%%%%%%%%%%%%%%%%%%%%
%%%%%%%%%%%%%%%%%%%%%%%%%%%%%%%%%%%%%%%%%%%%%%%%%%%%%%%%%%%%%%%%%%%%%%%%%%%%%%%%
\appendix

\settowidth\MacroIndent{\rmfamily\scriptsize 000\ }

 \DocInput{childdoc.dtx}

\end{document}
%</driver>
% \fi
%
% %%%%%%%%%%%%%%%%%%%%%%%%%%%%%%%%%%%%%%%%%%%%%%%%%%%%%%%%%%%%%%%%%%%%%%%%%%%%%%
% %%%%%%%%%%%%%%%%%%%%%%%%%%%%%%%%%%%%%%%%%%%%%%%%%%%%%%%%%%%%%%%%%%%%%%%%%%%%%%
% \section{Sample}
%\iffalse
%<*samplemain>
%\fi
%
% The following presents a sample document
% with two chapters, two parts, a title page,
% a compile flag as well as three forwarding files to set the flag.
% It consists of eight |.tex| files:
% \begin{center}
% \begin{tabular}{ll}
% |cdocsamp.tex|&main file\\
% |cdocsch1.tex|&include file for chapter 1\\
% |cdocsch2.tex|&include file for chapter 2\\
% |cdocspt3.tex|&include file for part 3\\
% |cdocspt4.tex|&include file for part 4\\
% |cdocsdrf.tex|&forwarding file for main file in draft mode\\
% |cdocsfi1.tex|&forwarding file for final version of chapter 1\\
% |cdocsfi2.tex|&forwarding file for final version of chapter 2\\
% \end{tabular}
% \end{center}
% Each of the eight files can be compiled directly by the \LaTeX{} compiler.
%
% %%%%%%%%%%%%%%%%%%%%%%%%%%%%%%%%%%%%%%
% \paragraph{Main File.}
%
% The main file is called |cdocsamp.tex|.
%
% Load the \textsf{childdoc} definitions and
% declare the filename for the main document:
%    \begin{macrocode}
\input{childdoc.def}
\childdocmain{}
%    \end{macrocode}

% Optional override for |\version| flag:
%    \begin{macrocode}
%%\ifchilddoc\else\providecommand{\version}{draft}\fi
%    \end{macrocode}

% Define the default values for the |\version| flag
% (|final| for the main file and |draft| for childs):
%    \begin{macrocode}
\ifchilddoc
\providecommand{\version}{draft}
\else
\providecommand{\version}{final}
\fi
%    \end{macrocode}

% Load the standard document class:
%    \begin{macrocode}
\documentclass[12pt]{article}
%    \end{macrocode}

% Start the document body:
%    \begin{macrocode}
\begin{document}
%    \end{macrocode}

% Declare a title page.
% Print title, part of document being processed and version flag:
%    \begin{macrocode}
\addtocounter{page}{-1}
\begin{center}
{\LARGE\bfseries{}childdoc example\par}
\vspace{1cm}
\ifchilddoc
\ifchilddocmanual part\else chapter\fi:
`\childdocname' of `\childdocjob'\par
\else
main document: `\childdocjob'\par
\fi
version: \version\par
\end{center}
\newpage
%    \end{macrocode}

% Manually include selected file,
% otherwise process as usual:
%    \begin{macrocode}
\ifchilddocmanual
\section*{part `\childdocname'}
\input{\childdocname}
\else
%    \end{macrocode}

% Include the two chapters:
%    \begin{macrocode}
\include{cdocsch1}
\include{cdocsch2}
%    \end{macrocode}

% Include the two parts unless only chapters should be displayed:
%    \begin{macrocode}
\ifchilddoc\else
\section{part three}
\input{cdocspt3}
\section{part four}
\input{cdocspt4}
\fi
%    \end{macrocode}

% Process as usual until here:
%    \begin{macrocode}
\fi
%    \end{macrocode}

% End of document body:
%    \begin{macrocode}
\end{document}
%    \end{macrocode}
%\iffalse
%</samplemain>
%\fi
%
% %%%%%%%%%%%%%%%%%%%%%%%%%%%%%%%%%%%%%%
% \paragraph{Chapter Include Files.}
%
% The include files are called |cdocsch1.tex| and |cdocsch2.tex|.
%
%\iffalse
%<*samplechap1|samplechap2>
%\fi

% Optional override for |\version| flag:
%    \begin{macrocode}
%%\providecommand{\version}{final}
%    \end{macrocode}

% Include the main document:
%    \begin{macrocode}
\input{childdoc.def}
\childdocof{cdocsamp}
%    \end{macrocode}

%\iffalse
%</samplechap1|samplechap2>
%\fi
%
%\iffalse
%<*samplechap1>
%\fi
% Some text for chapter 1:
%    \begin{macrocode}
\section{one}
some text in chapter one
%    \end{macrocode}

%\iffalse
%</samplechap1>
%\fi
% Some text for chapter 2:
%\iffalse
%<*samplechap2>
%\fi
%    \begin{macrocode}
\section{two}
more text in chapter two
%    \end{macrocode}

%\iffalse
%</samplechap2>
%\fi
%
% %%%%%%%%%%%%%%%%%%%%%%%%%%%%%%%%%%%%%%
% \paragraph{Part Include Files.}
%
% The include files are called |cdocspt3.tex| and |cdocspt4.tex|.
%
%\iffalse
%<*samplepart3|samplepart4>
%\fi

% Optional override for |\version| flag:
%    \begin{macrocode}
%%\providecommand{\version}{final}
%    \end{macrocode}

% Include the main document:
%    \begin{macrocode}
\input{childdoc.def}
\childdocby{cdocsamp}
%    \end{macrocode}

%\iffalse
%</samplepart3|samplepart4>
%\fi
%
%\iffalse
%<*samplepart3>
%\fi
% Some text for part 3:
%    \begin{macrocode}
some text in part three
%    \end{macrocode}

%\iffalse
%</samplepart3>
%\fi
% Some text for part 4:
%\iffalse
%<*samplepart4>
%\fi
%    \begin{macrocode}
more text in part four
%    \end{macrocode}

%\iffalse
%</samplepart4>
%\fi
%
% %%%%%%%%%%%%%%%%%%%%%%%%%%%%%%%%%%%%%%
% \paragraph{Forwarding for a Complete Draft.}
%
% The following forwarding file |cdocsdrf.tex|
% compiles the main document in draft mode:
%\iffalse
%<*sampledraft>
%\fi
%    \begin{macrocode}
\def\version{draft}
\input{childdoc.def}
\childdocforward{cdocsamp}
%    \end{macrocode}

%\iffalse
%</sampledraft>
%\fi
%
% %%%%%%%%%%%%%%%%%%%%%%%%%%%%%%%%%%%%%%
% \paragraph{Forwarding for Final Version of the Chapters.}
%
% The following forwarding files |cdocsfn1.tex| and |cdocsfn2.tex|
% (with identical content)
% compile the final versions of the child documents
% |cdocsch1.tex| and |cdocsch2.tex|, respectively:
%\iffalse
%<*samplefinal>
%\fi
%    \begin{macrocode}
\def\version{final}
\input{childdoc.def}
\childdocforwardprefix[cdocsamp]{cdocsfn}{cdocsch}
%    \end{macrocode}

%\iffalse
%</samplefinal>
%\fi
%
% %%%%%%%%%%%%%%%%%%%%%%%%%%%%%%%%%%%%%%
% \paragraph{Command Line Processing.}
%
% The following three command lines generate the output files
% |cdocscld|, |cdocscl1| and |cdocscl2|
% which should be identical to
% |cdocsdrf|, |cdocsch1| and |cdocsfn2|, respectively:
% \begin{center}
% \begin{tabular}{l}
% |latex -jobname cdocscld \|\\
% |  "\def\version{draft}\input{childdoc.def}\childdocforward{cdocsamp}"|\\
% |latex -jobname cdocscl1 \|\\
% |  "\input{childdoc.def}\childdocforward[cdocsamp]{cdocsch1}"|\\
% |latex -jobname cdocscl2 \|\\
% |  "\def\version{final}\input{childdoc.def}\childdocforward{cdocsch2}"|
% \end{tabular}
% \end{center}
% Note that the trailing backslash on each first line
% merely continues the input to the second line
% (for convenient cut ant paste).
% Furthermore, the command |latex| can be replaced by any
% of its alternative versions such as |pdflatex|.
%
% %%%%%%%%%%%%%%%%%%%%%%%%%%%%%%%%%%%%%%%%%%%%%%%%%%%%%%%%%%%%%%%%%%%%%%%%%%%%%%
% %%%%%%%%%%%%%%%%%%%%%%%%%%%%%%%%%%%%%%%%%%%%%%%%%%%%%%%%%%%%%%%%%%%%%%%%%%%%%%
% \section{Implementation}
%\iffalse
%<*package>
%\fi
%
% This section describes the definitions file |childdoc.def|.

% The definitions cannot be loaded using |\usepackage| or |\RequirePackage|
% which has a mechanism to prevent loading a style file more than once.
% When loading the definitions by means of |\input|
% multiple instances have to be prevented manually:
%\iffalse
%This code needs to be before the `\ProvidesFile' directive
%which is defined at the beginning of this file.
%Therefore it is also placed there and commented out here.
%</package>
%<*discard>
%\fi
%    \begin{macrocode}
\ifdefined\childdocmain\endinput\fi
%    \end{macrocode}
%\iffalse
%</discard>
%<*package>
%\fi
%
% \macro{\ifchilddoc}
% \macro{\ifchilddocmanual}
% The conditional |\ifchilddoc| tells whether a
% child (true) or main (false) document is being compiled.
% The conditional |\ifchilddocmanual| tells whether
% the |\includeonly| mechanism is used (false) or
% the selection of child files must be performed manually (true).
% The definitions initialise to false:
%    \begin{macrocode}
\newif\ifchilddoc
\newif\ifchilddocmanual
%    \end{macrocode}

% \macro{\childdocname}
% \macro{\childdocjob}
% The macro |\childdocname| stores the name of the main document
% to be compiled. The macro |\childdocjob| stores the name of
% the document on which the \LaTeX{} compiler was originally invoked.
% The content of |\jobname| cannot be compared
% to filenames specified in the source due to different catcodes.
% The following code rescans |\jobname|, stores the result
% in |\childdocname| and saves a copy in |\childdocjob|:
%    \begin{macrocode}
\edef\childdocname{\scantokens\expandafter{\jobname\noexpand}}
\let\childdocjob\childdocname
%    \end{macrocode}

% \macro{\childdocdisable}
% The macro |\childdocdisable| prevents the main file
% from being processed more than once.
% At this stage, the main document command |\childdocmain|
% is assumed to be called once again where it should do nothing.
% Any subsequent call to it should prevent
% a secondary processing of the main document
% It overwrites the forwarding commands
% |\childdocof| and |\childdocforward|
% with empty macros to prevent further inclusions of the main document:
%    \begin{macrocode}
\newcommand{\childdocdisable}
{
  \renewcommand{\childdocmain}[1]{\renewcommand{\childdocmain}[1]{\endinput}}
  \renewcommand{\childdocof}[1]{}
  \renewcommand{\childdocby}[2][]{}
  \renewcommand{\childdocforward}[2][]{}
  \renewcommand{\childdocdisable}{}
}
%    \end{macrocode}

% \macro{\childdocmain}
% The macro |\childdocmain| is to be called at the top of the main file
% with nothing or the main filename (without extension) as argument.
% First, it breaks loops.
% If the argument is not empty and does not match |\childdocname|
% (which is set by the first inclusion of |childdoc.def|),
% |\ifchilddoc| is set to true, |\includeonly| is applied to the child file
% and |\jobname| is set to the main file
% (for proper handling of |.aux| files):
%    \begin{macrocode}
\newcommand{\childdocmain}[1]
{
  \childdocdisable\childdocmain{}
  \if?#1?\else
    \begingroup
      \def\childdoctmp{#1}
      \ifx\childdoctmp\childdocname
        \def\childdoctmp{}
      \else
        \def\childdoctmp
        {
          \childdoctrue
          \includeonly{\childdocname}
          \def\childdocjob{#1}
          \def\jobname{#1}
        }
      \fi
      \expandafter
    \endgroup
    \childdoctmp
  \fi
}
%    \end{macrocode}

% \macro{\childdocof}
% The command |\childdocof| redirects
% compilation to the main file |#1|.
%    \begin{macrocode}
\newcommand{\childdocof}[1]
{
  \childdocdisable
  \childdoctrue
  \includeonly{\childdocname}
  \def\jobname{#1}
  \def\childdocjob{#1}
  \input{#1}
}
%    \end{macrocode}

% \macro{\childdocby}
% The command |\childdocby| ....
%    \begin{macrocode}
\newcommand{\childdocby}[2][]
{
  \childdocdisable
  \childdoctrue
  \childdocmanualtrue
  \if?#1?\else
    \def\jobname{#2}
  \fi
  \def\childdocjob{#2}
  \input{#2}
  \endinput
}
%    \end{macrocode}

% \macro{\childdocforward}
% The command |\childdocforward| redirects
% compilation to the main file or
% (if the optional argument is given) a child file.
% Parameters are set as if the main file
% or a child file starting with |\childdocof| was compiled.
% Then compilation is handed over to the main file:
%    \begin{macrocode}
\newcommand{\childdocforward}[2][]
{
  \begingroup
    \if?#1?
      \def\childdoctmp
      {
        \def\childdocname{#2}
        \def\childdocjob{#2}
        \def\jobname{#2}
        \input{#2}
        \endinput
      }
    \else
      \def\childdoctmp
      {
        \childdocdisable
        \def\childdocname{#2}
        \childdoctrue
        \includeonly{#2}
        \def\childdocjob{#1}
        \def\jobname{#1}
        \input{#1}
        \endinput
      }
    \fi
    \expandafter
  \endgroup
  \childdoctmp
}
%    \end{macrocode}

% \macro{\childdocforwardprefix}
% The command |\childdocforwardprefix| redirects
% compilation to the main or a child file by means of a pattern.
% The prefix |#1| in the current filename is replaced by |#2|
% and the suffix of the current filename is kept
% (it is assumed that the filename does not contain the substring `|~~~|'
% which is used as a delimiter).
% Compilation is handed over to the new file by |\childdocforward|:
%    \begin{macrocode}
\newcommand{\childdocforwardprefix}[3][]
{
  \begingroup
    \def\childdocextract #2##1~~~{\def\childdoctmp{\childdocforward[#1]{#3##1}}}
    \expandafter\childdocextract\childdocname~~~
    \expandafter
  \endgroup
  \childdoctmp
}
%    \end{macrocode}

% \macro{\childdoc}
% The deprecated macro |\childdoc| is a legacy version of |\childdocmain|:
%    \begin{macrocode}
\newcommand{\childdoc}{\childdocmain}
%    \end{macrocode}

% \macro{\childdocredirect}
% The deprecated macro |\childdocredirect| is a legacy version
% of |\childdocforward| and |\childdocforwardprefix|:
%    \begin{macrocode}
\newcommand{\childdocredirect}[2][]
{
  \begingroup
    \if?#1?
      \def\childdoctmp{\childdocforward{#2}}
    \else
      \def\childdoctmp{\childdocforwardprefix{#1}{#2}}
    \fi
    \expandafter
  \endgroup
  \childdoctmp
}
%    \end{macrocode}

%\iffalse
%</package>
%\fi
%
\endinput

\childdocof{cdocsamp}
%    \end{macrocode}

%\iffalse
%</samplechap1|samplechap2>
%\fi
%
%\iffalse
%<*samplechap1>
%\fi
% Some text for chapter 1:
%    \begin{macrocode}
\section{one}
some text in chapter one
%    \end{macrocode}

%\iffalse
%</samplechap1>
%\fi
% Some text for chapter 2:
%\iffalse
%<*samplechap2>
%\fi
%    \begin{macrocode}
\section{two}
more text in chapter two
%    \end{macrocode}

%\iffalse
%</samplechap2>
%\fi
%
% %%%%%%%%%%%%%%%%%%%%%%%%%%%%%%%%%%%%%%
% \paragraph{Part Include Files.}
%
% The include files are called |cdocspt3.tex| and |cdocspt4.tex|.
%
%\iffalse
%<*samplepart3|samplepart4>
%\fi

% Optional override for |\version| flag:
%    \begin{macrocode}
%%\providecommand{\version}{final}
%    \end{macrocode}

% Include the main document:
%    \begin{macrocode}
% \iffalse
%
% childdoc.dtx Copyright (C) 2017-2018 Niklas Beisert
%
% This work may be distributed and/or modified under the
% conditions of the LaTeX Project Public License, either version 1.3
% of this license or (at your option) any later version.
% The latest version of this license is in
%   http://www.latex-project.org/lppl.txt
% and version 1.3 or later is part of all distributions of LaTeX
% version 2005/12/01 or later.
%
% This work has the LPPL maintenance status `maintained'.
%
% The Current Maintainer of this work is Niklas Beisert.
%
% This work consists of the files childdoc.dtx and childdoc.ins
% and the derived files childdoc.def and cdocsamp.tex with
% cdocsch1.tex, cdocsch2.tex, cdocsdrf.tex, cdocsfn1.tex, cdocsfn2.tex.
%
%<package>\ifdefined\childdocmain\endinput\fi
%<package>\ProvidesFile{childdoc.def}[2018/12/30 v2.0 child document driver]
%<samplemain>\ProvidesFile{cdocsamp.tex}[2018/12/30 v2.0 sample for childdoc]
%<*driver>
%\ProvidesFile{childdoc.drv}[2018/12/30 v2.0 childdoc reference manual file]
\PassOptionsToClass{10pt,a4paper}{article}
\documentclass{ltxdoc}

\usepackage[margin=35mm]{geometry}
\usepackage{hyperref}
\usepackage{hyperxmp}
\usepackage[usenames]{color}

\hypersetup{colorlinks=true}
\hypersetup{pdfstartview=FitH}
\hypersetup{pdfpagemode=UseNone}
\hypersetup{pdfsource={}}
\hypersetup{pdflang={en-UK}}
\hypersetup{pdfcopyright={Copyright 2017-2018 Niklas Beisert.
  This work may be distributed and/or modified under the
  conditions of the LaTeX Project Public License, either version 1.3
  of this license or (at your option) any later version.}}
\hypersetup{pdflicenseurl={http://www.latex-project.org/lppl.txt}}
\hypersetup{pdfcontactaddress={ETH Zurich, ITP, HIT K,
  Wolfgang-Pauli-Strasse 27}}
\hypersetup{pdfcontactpostcode={8093}}
\hypersetup{pdfcontactcity={Zurich}}
\hypersetup{pdfcontactcountry={Switzerland}}
\hypersetup{pdfcontactemail={nbeisert@itp.phys.ethz.ch}}
\hypersetup{pdfcontacturl={http://people.phys.ethz.ch/\xmptilde nbeisert/}}

\newcommand{\secref}[1]{\hyperref[#1]{section \ref*{#1}}}

\parskip1ex
\parindent0pt
\let\olditemize\itemize
\def\itemize{\olditemize\parskip0pt}

\begin{document}

\title{The \textsf{childdoc} Package}
\hypersetup{pdftitle={The childdoc Package}}
\author{Niklas Beisert\\[2ex]
  Institut f\"ur Theoretische Physik\\
  Eidgen\"ossische Technische Hochschule Z\"urich\\
  Wolfgang-Pauli-Strasse 27, 8093 Z\"urich, Switzerland\\[1ex]
  \href{mailto:nbeisert@itp.phys.ethz.ch}
  {\texttt{nbeisert@itp.phys.ethz.ch}}}
\hypersetup{pdfauthor={Niklas Beisert}}
\hypersetup{pdfsubject={Manual for the LaTeX2e Package childdoc}}
\date{30 December 2018, \textsf{v2.0}}
\maketitle

\begin{abstract}\noindent
\textsf{childdoc} is a \LaTeXe{} package
that enables the direct compilation
of document sections included by |\include|
to individual files.
\end{abstract}

\begingroup
\parskip0ex
\tableofcontents
\endgroup

%%%%%%%%%%%%%%%%%%%%%%%%%%%%%%%%%%%%%%%%%%%%%%%%%%%%%%%%%%%%%%%%%%%%%%%%%%%%%%%%
%%%%%%%%%%%%%%%%%%%%%%%%%%%%%%%%%%%%%%%%%%%%%%%%%%%%%%%%%%%%%%%%%%%%%%%%%%%%%%%%
\section{Introduction}

\LaTeX{} provides a mechanism to structure a large document (such as a book)
into a main file and several child files (containing the chapters)
using the |\include| command.
This mechanism is beneficial for documents
which span hundreds of pages in order to
make the source file(s) more manageable.
Moreover, compilation can be restricted to
selected child files by means of the |\includeonly| command.
The latter feature can be used to reduce the compilation time while editing
(this was significantly more useful in the earlier days of \LaTeX{})
or to generate a smaller document which is easier to navigate.
Another application of |\includeonly| is to generate
documents consisting of selected parts of the complete document.

However, there are a few drawbacks of the plain |\include| mechanism:
\begin{itemize}
\item
The child files cannot be compiled on their own,
they can only be compiled via the main file.
A naive editing environment
(such as a text editor with an option
to have the current file processed by \LaTeX)
may require one to switch to the main file before compiling;
attempting to compile the child file produces errors.
\item
The main file must be modified (each time)
to adjust the |\includeonly| command
to the present needs. This easily leaves the main file in a messy state.
\item
The generated document will always carry the filename
of the main document. This is inconvenient if
several child files are to be compiled and
to be kept for distribution.
\end{itemize}

The present package provides a simple interface
to make child files individually compilable by \LaTeX{}.
Compiling a child file then has the same effect as compiling
the main file with an |\includeonly| command
to select the appropriate child.
Moreover the generated document will carry the name of the child
rather than the main file.
This resolves all three above issues.

This feature is meant to make the editing of books,
thesis documents and lecture notes somewhat more convenient.
However, the package can also be used efficiently for
composing a series of documents (such as exercise sheets)
which are typically distributed individually.
It then assists the author in generating the individual documents
(potentially in different versions)
as well as a document containing the collected series.
Another application is in developing style files
or other kinds of included material
where compilation of the style file could redirect
to a sample or test file.

%%%%%%%%%%%%%%%%%%%%%%%%%%%%%%%%%%%%%%%%%%%%%%%%%%%%%%%%%%%%%%%%%%%%%%%%%%%%%%%%
%%%%%%%%%%%%%%%%%%%%%%%%%%%%%%%%%%%%%%%%%%%%%%%%%%%%%%%%%%%%%%%%%%%%%%%%%%%%%%%%
\section{Usage}

First of all, the package \textsf{childdoc} is \emph{not} a standard
\LaTeXe{} |.sty| style file! Therefore it needs to be invoked in
a non-standard way.

%%%%%%%%%%%%%%%%%%%%%%%%%%%%%%%%%%%%%%%%%%%%%%%%%%%%%%%%%%%%%%%%%%%%%%%%%%%%%%%%
\subsection{Included Files}
\label{sec:include}

%%%%%%%%%%%%%%%%%%%%%%%%%%%%%%%%%%%%%%%%
\DescribeMacro{\childdocmain}
To use the package, add the commands
\begin{center}
\begin{tabular}{l}
|\input{childdoc.def}|\\
|\childdocmain{}|\\
\end{tabular}
\end{center}
at the very top of the main \LaTeX{} file,
in particular \emph{before} the |\documentclass| statement!
The argument of |\childdocmain| should be left empty
(but it must be present).

%%%%%%%%%%%%%%%%%%%%%%%%%%%%%%%%%%%%%%%%
\DescribeMacro{\childdocof}
Furthermore, add the commands
\begin{center}
\begin{tabular}{l}
|\input{childdoc.def}|\\
|\childdocof{|\textit{main}|}|\\
\end{tabular}
\end{center}
at the top of every child file \textit{child}
which is included by |\include{|\textit{child}|}|
from within the main file
(or at least for those files to be compiled individually).
The argument \textit{main} must be the filename of the main file.

There are a couple of
considerations in setting up the main and child documents:

%%%%%%%%%%%%%%%%%%%%%%%%%%%%%%%%%%%%%%%%
\paragraph{Restrictions.}

Please note the following restrictions:
\begin{itemize}
\item
|\childdocmain| must be called with one argument \textit{main}
to ensure compatibility with earlier version of the package.
It must either be empty (|\childdocmain{}|)
or precisely match the filename of the main file in which it is specified.
See \secref{sec:detection} for further information.
\item
The filename \textit{main} must be specified without the |.tex| extension.
\item
The filename \textit{main} is case sensitive
(even in case-insensitive file systems)
due to internal string comparison.
\item
The argument \textit{main} should be fully expanded, it cannot be a macro.
\item
Subdirectories and special characters should be avoided in filenames.
\item
The command |\childdocmain{|\textit{main}|}| must be followed by a whitespace.
It should not be followed immediately by another command
or by a comment mark `|%|'.
This is because the \TeX{} parser reads the token immediately following
the argument of |\childdocmain| and puts it
at the beginning of every child section;
however, a white\-space is ignored.
\end{itemize}

%%%%%%%%%%%%%%%%%%%%%%%%%%%%%%%%%%%%%%%%
\paragraph{Content of Main File.}

It is advisable to place all content in the child files included by |\include|.
Any output contained in the main file will appear in all child documents
unless suppressed manually;
it cannot be suppressed automatically by the |\includeonly| directive
and thus should normally be avoided.
A method to include some content in the main file
by means of conditional processing is described in \secref{sec:conditional}.

%%%%%%%%%%%%%%%%%%%%%%%%%%%%%%%%%%%%%%%%
\paragraph{Page Numbering.}

When only a part of the document is compiled,
the appropriate numbering of pages
(as well as other status parameters)
is determined from the |.aux| files.
The latter contain information from previous passes.
However this information needs to propagate through
all intermediate child documents.
Therefore the page numbering in child documents may well
be inconsistent until the complete document is compiled at least once.

A useful (if unconventional) way to always ensure a consistent
page numbering is to restart the numbering in each child document
and denote the pages by `\textit{child}|.|\textit{page}'
where \textit{child} represents the chapter/section number of the child file.
This can be achieved by the command
|\numberwithin{page}{|\textit{child}|}|
of the \textsf{amsmath} package
where \textit{child} can be |chapter| or |section|
depending on the chosen structuring.
Alternatively, one can modify the macro |\thepage| appropriately
and reset the counter |page| at the start of each child file.

%%%%%%%%%%%%%%%%%%%%%%%%%%%%%%%%%%%%%%%%%%%%%%%%%%%%%%%%%%%%%%%%%%%%%%%%%%%%%%%%
\subsection{Conditional Processing}
\label{sec:conditional}

The package provides a mechanism to compile different versions
of a document. To customise the versions further some conditional processing
can come in handy to distinguish which version is being compiled.
The package provides two macros to describe the compilation context:

%%%%%%%%%%%%%%%%%%%%%%%%%%%%%%%%%%%%%%%%
\DescribeMacro{\ifchilddoc}
The conditional |\ifchilddoc| distinguishes between the compilation of
child documents and the main document:
%
\begin{center}
|\ifchilddoc |\textit{child-code}| |[|\||else |\textit{main-code}]| \||fi|
\end{center}

%%%%%%%%%%%%%%%%%%%%%%%%%%%%%%%%%%%%%%%%
\DescribeMacro{\childdocname}
\DescribeMacro{\childdocjob}
The macro |\childdocname| contains the filename (without extension)
of the main or child file being processed.
Note that |\childdocjob| will always contain the name of the main file.

%%%%%%%%%%%%%%%%%%%%%%%%%%%%%%%%%%%%%%%%
\paragraph{Title Page.}

Conditional processing can be used to include a title or banner page
in the main document when proper precautions are taken.
Importantly, the code in the main file should ensure that the page counter
(as well as other status parameters which are stored in the |.aux| files)
takes the same value after the conditional processing.
Otherwise the page numbers may take divergent values
depending on which part is compiled.

For example, a title page could be declared by:
%
\begin{center}
\begin{tabular}{l}
|\ifchilddoc\||else|\\
|\addtocounter{page}{-1}|\\
\textit{code for title page}\\
|\newpage|\\
|\||fi|
\end{tabular}
\end{center}
%
A banner page for the child documents can be generated by:
%
\begin{center}
\begin{tabular}{l}
|\ifchilddoc|\\
|\addtocounter{page}{-1}|\\
\textit{code for banner page}\\
|\newpage|\\
|\||fi|
\end{tabular}
\end{center}
%
Here one could write a message such as:
\begin{center}
|This is the part \childdocname{} of \childdocjob{}.|
\end{center}

%%%%%%%%%%%%%%%%%%%%%%%%%%%%%%%%%%%%%%%%%%%%%%%%%%%%%%%%%%%%%%%%%%%%%%%%%%%%%%%%
\subsection{Flags}
\label{sec:flags}

The package makes it easy to generate different versions
of the main or child documents.
To this end compilation flags can be defined
and assigned different default values.
They will be particularly useful in conjunction
with the forwarding mechanism described in \secref{sec:forward}.

For example, it may be useful to have a flag |\version|
which can be set to |draft| or |final|.
The document source will contain some conditional code
depending on the value of |\version|.
Suppose further, the flag should default to |final| for the main file
and to |draft| for child files
which is a natural assignment for editing the document.
This is achieved by placing the following code
in the preamble of the main document
(below the |\childdocmain| directive):
%
\begin{center}
\begin{tabular}{l}
|\ifchilddoc|\\
|\providecommand{\version}{draft}|\\
|\||else|\\
|\providecommand{\version}{final}|\\
|\||fi|
\end{tabular}
\end{center}
%
The definition by |\providecommand| makes sure
that previous definitions are not overwritten.
Further statements |\providecommand{\version}{...}|
can thus be added before the above code to override it.

For the main file, one might add a line
(between |\childdocmain| and the above block)
%
\begin{center}
|%\ifchilddoc\||else\providecommand{\version}{draft}\||fi|
\end{center}
%
which can be uncommented to produce a draft version.
Likewise one can add a line to the very top of a child file
(above the |\childdocof{|\textit{main}|}| directive)
%
\begin{center}
|%\providecommand{\version}{final}|
\end{center}
%
which can be uncommented to produce the final version of this child document.

%%%%%%%%%%%%%%%%%%%%%%%%%%%%%%%%%%%%%%%%%%%%%%%%%%%%%%%%%%%%%%%%%%%%%%%%%%%%%%%%
\subsection{Forwarding}
\label{sec:forward}

Different versions of the main or child documents
using compilation flags as described in \secref{sec:flags}
can be (permanently) stored in different files
for convenient compilation, viewing and distribution.
To this end, the package defines a command
to pass on compilation to a different file:

%%%%%%%%%%%%%%%%%%%%%%%%%%%%%%%%%%%%%%%%
\DescribeMacro{\childdocforward}
The command |\childdocforward| redirects processing to
another source file:
%
\begin{center}
\begin{tabular}{l}
|\input{childdoc.def}|\\
|\childdocforward[|\textit{main}|]{|\textit{dest}|}|\\
\end{tabular}
\end{center}
%
The argument \textit{dest} is the destination file
(without extension).
It should be the main file or one of the child files.
Note that further \textsf{childdoc} directives
such as |\childdocof| and |\childdocforward|
in the indicated file will be processed in this form.
The optional argument \textit{main}
passes on directly to the main file \textit{main}
while pretending to compile the child \textit{dest}.
This form behaves as if \textit{dest}
issues |\childdocof{|\textit{main}|}| right away,
and no further \textsf{childdoc} directives will be processed.

%%%%%%%%%%%%%%%%%%%%%%%%%%%%%%%%%%%%%%%%
\DescribeMacro{\...prefix}
In the alternative form |\childdocforwardprefix|,
%
\begin{center}
\begin{tabular}{l}
|\input{childdoc.def}|\\
|\childdocforwardprefix[|\textit{main}|]{|\textit{prefix}|}{|\textit{dest}|}|
\end{tabular}
\end{center}
%
the destination file is determined by a pattern
depending on the current file:
To make this work, the current file must be called
`{\textit{prefix}\hspace{0.2em}\textit{suffix}}'
with \textit{prefix} matching precisely the argument.
Processing is then passed on to the file
`{\textit{dest}\hspace{0.2em}\textit{suffix}}'.
Surely, the same effect is achieved by
directly specifying the
argument `{\textit{dest}\hspace{0.2em}\textit{suffix}}'
in the first form.
However, that requires to set up a different file
for each child. With the alternative form of the command
all these files can have exactly the same content
which simplifies setting them up and maintaining them.

For example, the following file |draft.tex|
with a compilation flag |\version| as described in \secref{sec:flags}
compiles the main document as a draft:
%
\begin{center}
\begin{tabular}{l}
|\def\version{draft}|\\
|\input{childdoc.def}|\\
|\childdocforward{|\textit{main}|}|
\end{tabular}
\end{center}
%
Likewise, the following files |final|\textit{nn}|.tex|
compile the final version of the child document
|child|\textit{nn}|.tex|:
%
\begin{center}
\begin{tabular}{l}
|\def\version{final}|\\
|\input{childdoc.def}|\\
|\childdocforwardprefix{final}{child}|
\end{tabular}
\end{center}
%

Note that when several versions of a main file and/or of each child file
are to be generated, it may be convenient to set up a |Makefile| or
shell script to automatise the process.

%%%%%%%%%%%%%%%%%%%%%%%%%%%%%%%%%%%%%%%%%%%%%%%%%%%%%%%%%%%%%%%%%%%%%%%%%%%%%%%%
\subsection{Command Line Processing}
\label{sec:commandline}

The effect of redirection files can also be achieved by invoking
the \LaTeX{} compiler with a more elaborate command line.
Most conveniently this should be done as part
of a shell script or a |Makefile|.

When using \textsf{childdoc} in the main file, the following
command lines effectively perform a redirection
(note that depending on the shell being used,
backslashes may have to be doubled: `|\|' $\to$ `|\\|'):
%
\begin{center}
|... -jobname "|\textit{target}|" |\\|"|[\textit{flags}]%
|\input{childdoc.def}\childdocforward[|\textit{main}|]{|\textit{dest}|}"|
\end{center}
%
Here \textit{target} is the name of the output file,
\textit{main} is the name of the main file
and \textit{dest} is the name of the main or child file to be processed
(all filenames without extensions).
The optional argument \textit{main} can be omitted
if \textit{main} matches \textit{dest}.
Optionally, compilation \textit{flags} can be defined via |\def| commands.
This command line makes the \TeX{} engine believe
it is compiling the file \textit{target}
whose content is specified as the latter parameter.
The provided code then forwards the processing to
\textit{main} or \textit{dest} as described in \secref{sec:forward}.

%%%%%%%%%%%%%%%%%%%%%%%%%%%%%%%%%%%%%%%%%%%%%%%%%%%%%%%%%%%%%%%%%%%%%%%%%%%%%%%%
\subsection{Include by Input}
\label{sec:input}

Including child documents by |\include| has some restrictions by design.
Most notably, the content of a child document always occupies
its own set of pages; pages cannot be shared between child documents.
Usually, this behaviour makes perfect sense
because each child document contain an essential part of the document.
However, in some situations it may be desirable to compose
a document from a collection of parts
without having mandatory page breaks between then.
For this case, the package
provides a mechanism to include parts
by |\input| which can also be processed individually.
However, by construction this mechanism
requires manual handling of the content to be output.

%%%%%%%%%%%%%%%%%%%%%%%%%%%%%%%%%%%%%%%%
\DescribeMacro{\ifchilddocmanual}
The main file should be prepared as usual, see \secref{sec:include}.
However, the document body must make a distinction
between processing of an individual part and of the main document, e.g.:
%
\begin{center}
\begin{tabular}{l}
|\ifchilddocmanual|\\
|\input{\childdocname}|\\
|\||else|\\
\textit{document body with }|\input{|\textit{part}|}|\\
|\||fi|
\end{tabular}
\end{center}
%
The conditional |\ifchilddocmanual| is true whenever
a part to be included by |\input| is being compiled,
and the name of the part is stored in |\childdocname|.

%%%%%%%%%%%%%%%%%%%%%%%%%%%%%%%%%%%%%%%%
\DescribeMacro{\childdocby}
Each part to be included by |\input| should start with:
%
\begin{center}
\begin{tabular}{l}
|\input{childdoc.def}|\\
|\childdocby{|\textit{main}|}|\\
\end{tabular}
\end{center}
%
The directive |\childdocby| is similar to |\childdocof|
described in \secref{sec:include},
but the subsequent selection of content must be done manually.
To that end, both |\ifchilddoc| and |\ifchilddocmanual|
will be true upon processing of a part,
and the name of the part is stored in |\childdocname|.
Note that |\jobname| will be set to the filename of the current part
so that each part receives an individual |.aux| file
that does not interfere with the |.aux| file(s) of the main document.
This behaviour can be altered by the alternative form
|\childdocby[*]{|\textit{main}|}| (with a non-empty optional argument)
which uses the |.aux| file of the main document
by setting |\jobname| to \textit{main}.

%%%%%%%%%%%%%%%%%%%%%%%%%%%%%%%%%%%%%%%%%%%%%%%%%%%%%%%%%%%%%%%%%%%%%%%%%%%%%%%%
\subsection{Driver Development}
\label{sec:driver}

The \textsf{childdoc} mechanism can also be use for the development
of definition files such as \LaTeX{} styles or classes.
This case differs from the above setup with multiple parts
included by |\include| in that no |\includeonly| should be invoked.
This can be achieved by starting the include file
(before |\ProvidesPackage|) with:
%
\begin{center}
\begin{tabular}{l}
|\input{childdoc.def}|\\
|\childdocforward{|\textit{main}|}|\\
\end{tabular}
\end{center}
%
or alternatively with:
%
\begin{center}
\begin{tabular}{l}
|\input{childdoc.def}|\\
|\childdocby{|\textit{main}|}|\\
\end{tabular}
\end{center}
%
Both forms have slightly different effects as described above.
The main file is prepared as usual, see \secref{sec:include}.

%%%%%%%%%%%%%%%%%%%%%%%%%%%%%%%%%%%%%%%%%%%%%%%%%%%%%%%%%%%%%%%%%%%%%%%%%%%%%%%%
\subsection{Legacy Detection}
\label{sec:detection}

The directive |\childdocmain| in the main file can detect
whether the complete document or merely a child is to be compiled
even without using the directive |\childdocof|.
This method is deprecated because it is less robust
and there is no compelling reason to use it;
it is merely provided for backward compatibility
and it may be removed in future versions.

If the detection mechanism is to be used,
it is mandatory to correctly specify
the filename of the main file as the argument of |\childdocmain|:
%
\begin{center}
\begin{tabular}{l}
|\input{childdoc.def}|\\
|\childdocmain{|\textit{main}|}|\\
\end{tabular}
\end{center}
%
If |\jobname| does not match the argument \textit{main} of |\childdocmain|,
it is assumed that |\jobname| points to the child file to be compiled.
When using |\childdocmain| with the main file specified as argument,
it suffices to start a child file
with just |\input{|\textit{main}|}|
without loading of the package and using |\childdocof|.
If instead all processing is done
with the appropriate \textsf{childdoc} directives,
the argument of \textit{main} of |\childdocmain| can be empty.

An alternative version of the command line processing described
in \secref{sec:commandline} using the detection mechanism reads:
%
\begin{center}
|... -jobname "|\textit{target}|" "|[\textit{flags}]%
[|\def\jobname{|\textit{dest}|}|]|\input{|\textit{main}|}"|
\end{center}

%%%%%%%%%%%%%%%%%%%%%%%%%%%%%%%%%%%%%%%%%%%%%%%%%%%%%%%%%%%%%%%%%%%%%%%%%%%%%%%%
\subsection{Manual Code}
\label{sec:manual}

In case one cannot be certain whether the definitions file |childdoc.def|
is installed on the target \TeX{} distribution
and one prefers not to ship it,
it is conceivable to paste a few relevant commands into the sources.

To that end, drop all statements |\input{childdoc.def}|
and perform the replacements as outlined below.
Instead of |\childdocmain{|\textit{main}|}| add the following code
to the top of the main file:
%
\begin{center}
\begin{tabular}{l}
|\||ifdefined\childdocname\endinput\||fi\newif\ifchilddoc|\\
|\edef\childdocname{\scantokens\expandafter{\jobname\noexpand}}|\\
|\def\childdocmain{|\textit{main}|}\||ifx\childdocmain\childdocname\||else|\\
|\childdoctrue\includeonly{\childdocname}\let\jobname\childdocmain\||fi|\\
\end{tabular}
\end{center}
%
Instead of |\childdocof{|\textit{main}|}| just include the main file
at the top of each child file:
%
\begin{center}
|\input{|\textit{main}|}|
\end{center}
%
A simple redirection |\childdocforward{|\textit{dest}|}| is achieved by:
%
\begin{center}
|\def\jobname{|\textit{dest}|}\input{\jobname}|
\end{center}
%
The redirection with prefix
|\childdocforwardprefix[|\textit{prefix}|]{|\textit{dest}|}|
is accomplished by:
%
\begin{center}
\begin{tabular}{l}
|{\edef\jobname{\scantokens\expandafter{\jobname\noexpand}}|\\
|\def\redirectjob |\textit{prefix}|#1~~~{\gdef\jobname{|\textit{dest}|#1}}|\\
|\expandafter\redirectjob\jobname~~~}\input{\jobname}|
\end{tabular}
\end{center}

In an alternative approach,
child documents can be compiled by a specific command line
without additional code or specific definitions:
%
\begin{center}
|... -jobname "|\textit{target}|" "|[\textit{flags}]%
|\includeonly{|\textit{dest}|}\input{|\textit{main}|}"|
\end{center}
%

%%%%%%%%%%%%%%%%%%%%%%%%%%%%%%%%%%%%%%%%%%%%%%%%%%%%%%%%%%%%%%%%%%%%%%%%%%%%%%%%
%%%%%%%%%%%%%%%%%%%%%%%%%%%%%%%%%%%%%%%%%%%%%%%%%%%%%%%%%%%%%%%%%%%%%%%%%%%%%%%%
\section{Information}

%%%%%%%%%%%%%%%%%%%%%%%%%%%%%%%%%%%%%%%%%%%%%%%%%%%%%%%%%%%%%%%%%%%%%%%%%%%%%%%%
\subsection{Copyright}

Copyright \copyright{} 2017--2018 Niklas Beisert

This work may be distributed and/or modified under the
conditions of the \LaTeX{} Project Public License, either version 1.3
of this license or (at your option) any later version.
The latest version of this license is in
  \url{http://www.latex-project.org/lppl.txt}
and version 1.3 or later is part of all distributions of \LaTeX{}
version 2005/12/01 or later.

This work has the LPPL maintenance status `maintained'.

The Current Maintainer of this work is Niklas Beisert.

This work consists of the files |README.txt|, |childdoc.ins| and |childdoc.dtx|
as well as the derived files |childdoc.def|, |cdocsamp.tex|
with |cdocsch1.tex|, |cdocsch2.tex|, |cdocspt3.tex|, |cdocspt4.tex|,
|cdocsdrf.tex|, |cdocsfn1.tex|, |cdocsfn2.tex|
as well as |childdoc.pdf|.

%%%%%%%%%%%%%%%%%%%%%%%%%%%%%%%%%%%%%%%%%%%%%%%%%%%%%%%%%%%%%%%%%%%%%%%%%%%%%%%%
\subsection{Files and Installation}

The package consists of the files:
%
\begin{center}
\begin{tabular}{ll}
    |README.txt|   & readme file \\
    |childdoc.ins| & installation file \\
    |childdoc.dtx| & source file \\
    |childdoc.def| & definition file \\
    |cdocsamp.tex| & sample main file \\
    |cdocsch1.tex| & sample include file \\
    |cdocsch2.tex| & sample include file \\
    |cdocspt3.tex| & sample part file \\
    |cdocspt4.tex| & sample part file \\
    |cdocsdrf.tex| & sample redirection file \\
    |cdocsfn1.tex| & sample redirection file \\
    |cdocsfn2.tex| & sample redirection file \\
    |childdoc.pdf| & manual
\end{tabular}
\end{center}
%
The distribution consists of the files
|README.txt|, |childdoc.ins| and |childdoc.dtx|.
%
\begin{itemize}
\item
Run (pdf)\LaTeX{} on |childdoc.dtx|
to compile the manual |childdoc.pdf| (this file).
\item
Run \LaTeX{} on |childdoc.ins| to create the definitions file |childdoc.def|
and the sample |cdocsamp.tex| with include files
|cdocsch1.tex|, |cdocsch2.tex|, |cdocspt3.tex|, |cdocspt4.tex|,
|cdocsdrf.tex|, |cdocsfn1.tex|, |cdocsfn2.tex|.
Then copy the file |childdoc.def| to an appropriate directory of your \LaTeX{}
distribution, e.g.\ \textit{texmf-root}|/tex/latex/childdoc|.
\end{itemize}

%%%%%%%%%%%%%%%%%%%%%%%%%%%%%%%%%%%%%%%%%%%%%%%%%%%%%%%%%%%%%%%%%%%%%%%%%%%%%%%%
\subsection{Related CTAN Packages}

There are several other packages which offer a similar functionality:
%
\begin{itemize}
\item
The packages
\href{http://ctan.org/pkg/docmute}{\textsf{docmute}},
\href{http://ctan.org/pkg/includex}{\textsf{includex}} and
\href{http://ctan.org/pkg/standalone}{\textsf{standalone}}
provide commands to include only the document body of
a child file thus allowing both files to be compiled individually.
\item
The packages \href{http://ctan.org/pkg/subdocs}{\textsf{subdocs}}
and \href{http://ctan.org/pkg/subfiles}{\textsf{subfiles}}
provide structures in which the main and child documents can be
encapsulated and allowing them to be compiled individually.
The inclusion mechanism is different from the conventional |\include|.
\item
The package \href{http://ctan.org/pkg/combine}{\textsf{combine}}
is an elaborate solution to combine several documents into one.
\end{itemize}
%
See also the CTAN topic \href{http://ctan.org/topic/subdocs}{\textsf{subdocs}}
for further related packages.
The present package differs from the above solutions in that
a document structure constructed with the conventional |\include| mechanism
just needs two extra commands at the top of every file
such that all constituent files can be compiled individually.

%%%%%%%%%%%%%%%%%%%%%%%%%%%%%%%%%%%%%%%%%%%%%%%%%%%%%%%%%%%%%%%%%%%%%%%%%%%%%%%%
%\subsection{Feature Suggestions}
%
%The following is a list of features which may be useful for future
%versions of this package:
%%
%\begin{itemize}
%\item
%\ldots
%\end{itemize}

%%%%%%%%%%%%%%%%%%%%%%%%%%%%%%%%%%%%%%%%%%%%%%%%%%%%%%%%%%%%%%%%%%%%%%%%%%%%%%%%
\subsection{Revision History}

%%%%%%%%%%%%%%%%%%%%%%%%%%%%%%%%%%%%%%%%
\paragraph{v2.0:} 2018/12/30

\begin{itemize}
\item
immediate forward processing
\item
added |\childdocby| mechanism
\item
manual restructured
\end{itemize}

%%%%%%%%%%%%%%%%%%%%%%%%%%%%%%%%%%%%%%%%
\paragraph{v1.6:} 2018/01/17

\begin{itemize}
\item
application for development of include files
\item
corrections to manual
\end{itemize}

%%%%%%%%%%%%%%%%%%%%%%%%%%%%%%%%%%%%%%%%
\paragraph{v1.5:} 2017/05/21

\begin{itemize}
\item
more complete structuring introduced
\item
|\childdocof| introduced
\item
|\childdoc| renamed to |\childdocmain|
\item
|\childredirect| renamed to |\childdocforward| and |\childdocforwardprefix|
and functionality expanded
\end{itemize}

%%%%%%%%%%%%%%%%%%%%%%%%%%%%%%%%%%%%%%%%
\paragraph{v1.0:} 2017/04/27

\begin{itemize}
\item
manual and install package
\item
first version published on CTAN
\end{itemize}

%%%%%%%%%%%%%%%%%%%%%%%%%%%%%%%%%%%%%%%%
\paragraph{v0.6:} 2017/04/26

\begin{itemize}
\item
redirection mechanism added
\end{itemize}

%%%%%%%%%%%%%%%%%%%%%%%%%%%%%%%%%%%%%%%%
\paragraph{v0.5:} 2017/04/26

\begin{itemize}
\item
functionality in definition file
\end{itemize}


%%%%%%%%%%%%%%%%%%%%%%%%%%%%%%%%%%%%%%%%%%%%%%%%%%%%%%%%%%%%%%%%%%%%%%%%%%%%%%%%
%%%%%%%%%%%%%%%%%%%%%%%%%%%%%%%%%%%%%%%%%%%%%%%%%%%%%%%%%%%%%%%%%%%%%%%%%%%%%%%%
%%%%%%%%%%%%%%%%%%%%%%%%%%%%%%%%%%%%%%%%%%%%%%%%%%%%%%%%%%%%%%%%%%%%%%%%%%%%%%%%
\appendix

\settowidth\MacroIndent{\rmfamily\scriptsize 000\ }

 \DocInput{childdoc.dtx}

\end{document}
%</driver>
% \fi
%
% %%%%%%%%%%%%%%%%%%%%%%%%%%%%%%%%%%%%%%%%%%%%%%%%%%%%%%%%%%%%%%%%%%%%%%%%%%%%%%
% %%%%%%%%%%%%%%%%%%%%%%%%%%%%%%%%%%%%%%%%%%%%%%%%%%%%%%%%%%%%%%%%%%%%%%%%%%%%%%
% \section{Sample}
%\iffalse
%<*samplemain>
%\fi
%
% The following presents a sample document
% with two chapters, two parts, a title page,
% a compile flag as well as three forwarding files to set the flag.
% It consists of eight |.tex| files:
% \begin{center}
% \begin{tabular}{ll}
% |cdocsamp.tex|&main file\\
% |cdocsch1.tex|&include file for chapter 1\\
% |cdocsch2.tex|&include file for chapter 2\\
% |cdocspt3.tex|&include file for part 3\\
% |cdocspt4.tex|&include file for part 4\\
% |cdocsdrf.tex|&forwarding file for main file in draft mode\\
% |cdocsfi1.tex|&forwarding file for final version of chapter 1\\
% |cdocsfi2.tex|&forwarding file for final version of chapter 2\\
% \end{tabular}
% \end{center}
% Each of the eight files can be compiled directly by the \LaTeX{} compiler.
%
% %%%%%%%%%%%%%%%%%%%%%%%%%%%%%%%%%%%%%%
% \paragraph{Main File.}
%
% The main file is called |cdocsamp.tex|.
%
% Load the \textsf{childdoc} definitions and
% declare the filename for the main document:
%    \begin{macrocode}
\input{childdoc.def}
\childdocmain{}
%    \end{macrocode}

% Optional override for |\version| flag:
%    \begin{macrocode}
%%\ifchilddoc\else\providecommand{\version}{draft}\fi
%    \end{macrocode}

% Define the default values for the |\version| flag
% (|final| for the main file and |draft| for childs):
%    \begin{macrocode}
\ifchilddoc
\providecommand{\version}{draft}
\else
\providecommand{\version}{final}
\fi
%    \end{macrocode}

% Load the standard document class:
%    \begin{macrocode}
\documentclass[12pt]{article}
%    \end{macrocode}

% Start the document body:
%    \begin{macrocode}
\begin{document}
%    \end{macrocode}

% Declare a title page.
% Print title, part of document being processed and version flag:
%    \begin{macrocode}
\addtocounter{page}{-1}
\begin{center}
{\LARGE\bfseries{}childdoc example\par}
\vspace{1cm}
\ifchilddoc
\ifchilddocmanual part\else chapter\fi:
`\childdocname' of `\childdocjob'\par
\else
main document: `\childdocjob'\par
\fi
version: \version\par
\end{center}
\newpage
%    \end{macrocode}

% Manually include selected file,
% otherwise process as usual:
%    \begin{macrocode}
\ifchilddocmanual
\section*{part `\childdocname'}
\input{\childdocname}
\else
%    \end{macrocode}

% Include the two chapters:
%    \begin{macrocode}
\include{cdocsch1}
\include{cdocsch2}
%    \end{macrocode}

% Include the two parts unless only chapters should be displayed:
%    \begin{macrocode}
\ifchilddoc\else
\section{part three}
\input{cdocspt3}
\section{part four}
\input{cdocspt4}
\fi
%    \end{macrocode}

% Process as usual until here:
%    \begin{macrocode}
\fi
%    \end{macrocode}

% End of document body:
%    \begin{macrocode}
\end{document}
%    \end{macrocode}
%\iffalse
%</samplemain>
%\fi
%
% %%%%%%%%%%%%%%%%%%%%%%%%%%%%%%%%%%%%%%
% \paragraph{Chapter Include Files.}
%
% The include files are called |cdocsch1.tex| and |cdocsch2.tex|.
%
%\iffalse
%<*samplechap1|samplechap2>
%\fi

% Optional override for |\version| flag:
%    \begin{macrocode}
%%\providecommand{\version}{final}
%    \end{macrocode}

% Include the main document:
%    \begin{macrocode}
\input{childdoc.def}
\childdocof{cdocsamp}
%    \end{macrocode}

%\iffalse
%</samplechap1|samplechap2>
%\fi
%
%\iffalse
%<*samplechap1>
%\fi
% Some text for chapter 1:
%    \begin{macrocode}
\section{one}
some text in chapter one
%    \end{macrocode}

%\iffalse
%</samplechap1>
%\fi
% Some text for chapter 2:
%\iffalse
%<*samplechap2>
%\fi
%    \begin{macrocode}
\section{two}
more text in chapter two
%    \end{macrocode}

%\iffalse
%</samplechap2>
%\fi
%
% %%%%%%%%%%%%%%%%%%%%%%%%%%%%%%%%%%%%%%
% \paragraph{Part Include Files.}
%
% The include files are called |cdocspt3.tex| and |cdocspt4.tex|.
%
%\iffalse
%<*samplepart3|samplepart4>
%\fi

% Optional override for |\version| flag:
%    \begin{macrocode}
%%\providecommand{\version}{final}
%    \end{macrocode}

% Include the main document:
%    \begin{macrocode}
\input{childdoc.def}
\childdocby{cdocsamp}
%    \end{macrocode}

%\iffalse
%</samplepart3|samplepart4>
%\fi
%
%\iffalse
%<*samplepart3>
%\fi
% Some text for part 3:
%    \begin{macrocode}
some text in part three
%    \end{macrocode}

%\iffalse
%</samplepart3>
%\fi
% Some text for part 4:
%\iffalse
%<*samplepart4>
%\fi
%    \begin{macrocode}
more text in part four
%    \end{macrocode}

%\iffalse
%</samplepart4>
%\fi
%
% %%%%%%%%%%%%%%%%%%%%%%%%%%%%%%%%%%%%%%
% \paragraph{Forwarding for a Complete Draft.}
%
% The following forwarding file |cdocsdrf.tex|
% compiles the main document in draft mode:
%\iffalse
%<*sampledraft>
%\fi
%    \begin{macrocode}
\def\version{draft}
\input{childdoc.def}
\childdocforward{cdocsamp}
%    \end{macrocode}

%\iffalse
%</sampledraft>
%\fi
%
% %%%%%%%%%%%%%%%%%%%%%%%%%%%%%%%%%%%%%%
% \paragraph{Forwarding for Final Version of the Chapters.}
%
% The following forwarding files |cdocsfn1.tex| and |cdocsfn2.tex|
% (with identical content)
% compile the final versions of the child documents
% |cdocsch1.tex| and |cdocsch2.tex|, respectively:
%\iffalse
%<*samplefinal>
%\fi
%    \begin{macrocode}
\def\version{final}
\input{childdoc.def}
\childdocforwardprefix[cdocsamp]{cdocsfn}{cdocsch}
%    \end{macrocode}

%\iffalse
%</samplefinal>
%\fi
%
% %%%%%%%%%%%%%%%%%%%%%%%%%%%%%%%%%%%%%%
% \paragraph{Command Line Processing.}
%
% The following three command lines generate the output files
% |cdocscld|, |cdocscl1| and |cdocscl2|
% which should be identical to
% |cdocsdrf|, |cdocsch1| and |cdocsfn2|, respectively:
% \begin{center}
% \begin{tabular}{l}
% |latex -jobname cdocscld \|\\
% |  "\def\version{draft}\input{childdoc.def}\childdocforward{cdocsamp}"|\\
% |latex -jobname cdocscl1 \|\\
% |  "\input{childdoc.def}\childdocforward[cdocsamp]{cdocsch1}"|\\
% |latex -jobname cdocscl2 \|\\
% |  "\def\version{final}\input{childdoc.def}\childdocforward{cdocsch2}"|
% \end{tabular}
% \end{center}
% Note that the trailing backslash on each first line
% merely continues the input to the second line
% (for convenient cut ant paste).
% Furthermore, the command |latex| can be replaced by any
% of its alternative versions such as |pdflatex|.
%
% %%%%%%%%%%%%%%%%%%%%%%%%%%%%%%%%%%%%%%%%%%%%%%%%%%%%%%%%%%%%%%%%%%%%%%%%%%%%%%
% %%%%%%%%%%%%%%%%%%%%%%%%%%%%%%%%%%%%%%%%%%%%%%%%%%%%%%%%%%%%%%%%%%%%%%%%%%%%%%
% \section{Implementation}
%\iffalse
%<*package>
%\fi
%
% This section describes the definitions file |childdoc.def|.

% The definitions cannot be loaded using |\usepackage| or |\RequirePackage|
% which has a mechanism to prevent loading a style file more than once.
% When loading the definitions by means of |\input|
% multiple instances have to be prevented manually:
%\iffalse
%This code needs to be before the `\ProvidesFile' directive
%which is defined at the beginning of this file.
%Therefore it is also placed there and commented out here.
%</package>
%<*discard>
%\fi
%    \begin{macrocode}
\ifdefined\childdocmain\endinput\fi
%    \end{macrocode}
%\iffalse
%</discard>
%<*package>
%\fi
%
% \macro{\ifchilddoc}
% \macro{\ifchilddocmanual}
% The conditional |\ifchilddoc| tells whether a
% child (true) or main (false) document is being compiled.
% The conditional |\ifchilddocmanual| tells whether
% the |\includeonly| mechanism is used (false) or
% the selection of child files must be performed manually (true).
% The definitions initialise to false:
%    \begin{macrocode}
\newif\ifchilddoc
\newif\ifchilddocmanual
%    \end{macrocode}

% \macro{\childdocname}
% \macro{\childdocjob}
% The macro |\childdocname| stores the name of the main document
% to be compiled. The macro |\childdocjob| stores the name of
% the document on which the \LaTeX{} compiler was originally invoked.
% The content of |\jobname| cannot be compared
% to filenames specified in the source due to different catcodes.
% The following code rescans |\jobname|, stores the result
% in |\childdocname| and saves a copy in |\childdocjob|:
%    \begin{macrocode}
\edef\childdocname{\scantokens\expandafter{\jobname\noexpand}}
\let\childdocjob\childdocname
%    \end{macrocode}

% \macro{\childdocdisable}
% The macro |\childdocdisable| prevents the main file
% from being processed more than once.
% At this stage, the main document command |\childdocmain|
% is assumed to be called once again where it should do nothing.
% Any subsequent call to it should prevent
% a secondary processing of the main document
% It overwrites the forwarding commands
% |\childdocof| and |\childdocforward|
% with empty macros to prevent further inclusions of the main document:
%    \begin{macrocode}
\newcommand{\childdocdisable}
{
  \renewcommand{\childdocmain}[1]{\renewcommand{\childdocmain}[1]{\endinput}}
  \renewcommand{\childdocof}[1]{}
  \renewcommand{\childdocby}[2][]{}
  \renewcommand{\childdocforward}[2][]{}
  \renewcommand{\childdocdisable}{}
}
%    \end{macrocode}

% \macro{\childdocmain}
% The macro |\childdocmain| is to be called at the top of the main file
% with nothing or the main filename (without extension) as argument.
% First, it breaks loops.
% If the argument is not empty and does not match |\childdocname|
% (which is set by the first inclusion of |childdoc.def|),
% |\ifchilddoc| is set to true, |\includeonly| is applied to the child file
% and |\jobname| is set to the main file
% (for proper handling of |.aux| files):
%    \begin{macrocode}
\newcommand{\childdocmain}[1]
{
  \childdocdisable\childdocmain{}
  \if?#1?\else
    \begingroup
      \def\childdoctmp{#1}
      \ifx\childdoctmp\childdocname
        \def\childdoctmp{}
      \else
        \def\childdoctmp
        {
          \childdoctrue
          \includeonly{\childdocname}
          \def\childdocjob{#1}
          \def\jobname{#1}
        }
      \fi
      \expandafter
    \endgroup
    \childdoctmp
  \fi
}
%    \end{macrocode}

% \macro{\childdocof}
% The command |\childdocof| redirects
% compilation to the main file |#1|.
%    \begin{macrocode}
\newcommand{\childdocof}[1]
{
  \childdocdisable
  \childdoctrue
  \includeonly{\childdocname}
  \def\jobname{#1}
  \def\childdocjob{#1}
  \input{#1}
}
%    \end{macrocode}

% \macro{\childdocby}
% The command |\childdocby| ....
%    \begin{macrocode}
\newcommand{\childdocby}[2][]
{
  \childdocdisable
  \childdoctrue
  \childdocmanualtrue
  \if?#1?\else
    \def\jobname{#2}
  \fi
  \def\childdocjob{#2}
  \input{#2}
  \endinput
}
%    \end{macrocode}

% \macro{\childdocforward}
% The command |\childdocforward| redirects
% compilation to the main file or
% (if the optional argument is given) a child file.
% Parameters are set as if the main file
% or a child file starting with |\childdocof| was compiled.
% Then compilation is handed over to the main file:
%    \begin{macrocode}
\newcommand{\childdocforward}[2][]
{
  \begingroup
    \if?#1?
      \def\childdoctmp
      {
        \def\childdocname{#2}
        \def\childdocjob{#2}
        \def\jobname{#2}
        \input{#2}
        \endinput
      }
    \else
      \def\childdoctmp
      {
        \childdocdisable
        \def\childdocname{#2}
        \childdoctrue
        \includeonly{#2}
        \def\childdocjob{#1}
        \def\jobname{#1}
        \input{#1}
        \endinput
      }
    \fi
    \expandafter
  \endgroup
  \childdoctmp
}
%    \end{macrocode}

% \macro{\childdocforwardprefix}
% The command |\childdocforwardprefix| redirects
% compilation to the main or a child file by means of a pattern.
% The prefix |#1| in the current filename is replaced by |#2|
% and the suffix of the current filename is kept
% (it is assumed that the filename does not contain the substring `|~~~|'
% which is used as a delimiter).
% Compilation is handed over to the new file by |\childdocforward|:
%    \begin{macrocode}
\newcommand{\childdocforwardprefix}[3][]
{
  \begingroup
    \def\childdocextract #2##1~~~{\def\childdoctmp{\childdocforward[#1]{#3##1}}}
    \expandafter\childdocextract\childdocname~~~
    \expandafter
  \endgroup
  \childdoctmp
}
%    \end{macrocode}

% \macro{\childdoc}
% The deprecated macro |\childdoc| is a legacy version of |\childdocmain|:
%    \begin{macrocode}
\newcommand{\childdoc}{\childdocmain}
%    \end{macrocode}

% \macro{\childdocredirect}
% The deprecated macro |\childdocredirect| is a legacy version
% of |\childdocforward| and |\childdocforwardprefix|:
%    \begin{macrocode}
\newcommand{\childdocredirect}[2][]
{
  \begingroup
    \if?#1?
      \def\childdoctmp{\childdocforward{#2}}
    \else
      \def\childdoctmp{\childdocforwardprefix{#1}{#2}}
    \fi
    \expandafter
  \endgroup
  \childdoctmp
}
%    \end{macrocode}

%\iffalse
%</package>
%\fi
%
\endinput

\childdocby{cdocsamp}
%    \end{macrocode}

%\iffalse
%</samplepart3|samplepart4>
%\fi
%
%\iffalse
%<*samplepart3>
%\fi
% Some text for part 3:
%    \begin{macrocode}
some text in part three
%    \end{macrocode}

%\iffalse
%</samplepart3>
%\fi
% Some text for part 4:
%\iffalse
%<*samplepart4>
%\fi
%    \begin{macrocode}
more text in part four
%    \end{macrocode}

%\iffalse
%</samplepart4>
%\fi
%
% %%%%%%%%%%%%%%%%%%%%%%%%%%%%%%%%%%%%%%
% \paragraph{Forwarding for a Complete Draft.}
%
% The following forwarding file |cdocsdrf.tex|
% compiles the main document in draft mode:
%\iffalse
%<*sampledraft>
%\fi
%    \begin{macrocode}
\def\version{draft}
% \iffalse
%
% childdoc.dtx Copyright (C) 2017-2018 Niklas Beisert
%
% This work may be distributed and/or modified under the
% conditions of the LaTeX Project Public License, either version 1.3
% of this license or (at your option) any later version.
% The latest version of this license is in
%   http://www.latex-project.org/lppl.txt
% and version 1.3 or later is part of all distributions of LaTeX
% version 2005/12/01 or later.
%
% This work has the LPPL maintenance status `maintained'.
%
% The Current Maintainer of this work is Niklas Beisert.
%
% This work consists of the files childdoc.dtx and childdoc.ins
% and the derived files childdoc.def and cdocsamp.tex with
% cdocsch1.tex, cdocsch2.tex, cdocsdrf.tex, cdocsfn1.tex, cdocsfn2.tex.
%
%<package>\ifdefined\childdocmain\endinput\fi
%<package>\ProvidesFile{childdoc.def}[2018/12/30 v2.0 child document driver]
%<samplemain>\ProvidesFile{cdocsamp.tex}[2018/12/30 v2.0 sample for childdoc]
%<*driver>
%\ProvidesFile{childdoc.drv}[2018/12/30 v2.0 childdoc reference manual file]
\PassOptionsToClass{10pt,a4paper}{article}
\documentclass{ltxdoc}

\usepackage[margin=35mm]{geometry}
\usepackage{hyperref}
\usepackage{hyperxmp}
\usepackage[usenames]{color}

\hypersetup{colorlinks=true}
\hypersetup{pdfstartview=FitH}
\hypersetup{pdfpagemode=UseNone}
\hypersetup{pdfsource={}}
\hypersetup{pdflang={en-UK}}
\hypersetup{pdfcopyright={Copyright 2017-2018 Niklas Beisert.
  This work may be distributed and/or modified under the
  conditions of the LaTeX Project Public License, either version 1.3
  of this license or (at your option) any later version.}}
\hypersetup{pdflicenseurl={http://www.latex-project.org/lppl.txt}}
\hypersetup{pdfcontactaddress={ETH Zurich, ITP, HIT K,
  Wolfgang-Pauli-Strasse 27}}
\hypersetup{pdfcontactpostcode={8093}}
\hypersetup{pdfcontactcity={Zurich}}
\hypersetup{pdfcontactcountry={Switzerland}}
\hypersetup{pdfcontactemail={nbeisert@itp.phys.ethz.ch}}
\hypersetup{pdfcontacturl={http://people.phys.ethz.ch/\xmptilde nbeisert/}}

\newcommand{\secref}[1]{\hyperref[#1]{section \ref*{#1}}}

\parskip1ex
\parindent0pt
\let\olditemize\itemize
\def\itemize{\olditemize\parskip0pt}

\begin{document}

\title{The \textsf{childdoc} Package}
\hypersetup{pdftitle={The childdoc Package}}
\author{Niklas Beisert\\[2ex]
  Institut f\"ur Theoretische Physik\\
  Eidgen\"ossische Technische Hochschule Z\"urich\\
  Wolfgang-Pauli-Strasse 27, 8093 Z\"urich, Switzerland\\[1ex]
  \href{mailto:nbeisert@itp.phys.ethz.ch}
  {\texttt{nbeisert@itp.phys.ethz.ch}}}
\hypersetup{pdfauthor={Niklas Beisert}}
\hypersetup{pdfsubject={Manual for the LaTeX2e Package childdoc}}
\date{30 December 2018, \textsf{v2.0}}
\maketitle

\begin{abstract}\noindent
\textsf{childdoc} is a \LaTeXe{} package
that enables the direct compilation
of document sections included by |\include|
to individual files.
\end{abstract}

\begingroup
\parskip0ex
\tableofcontents
\endgroup

%%%%%%%%%%%%%%%%%%%%%%%%%%%%%%%%%%%%%%%%%%%%%%%%%%%%%%%%%%%%%%%%%%%%%%%%%%%%%%%%
%%%%%%%%%%%%%%%%%%%%%%%%%%%%%%%%%%%%%%%%%%%%%%%%%%%%%%%%%%%%%%%%%%%%%%%%%%%%%%%%
\section{Introduction}

\LaTeX{} provides a mechanism to structure a large document (such as a book)
into a main file and several child files (containing the chapters)
using the |\include| command.
This mechanism is beneficial for documents
which span hundreds of pages in order to
make the source file(s) more manageable.
Moreover, compilation can be restricted to
selected child files by means of the |\includeonly| command.
The latter feature can be used to reduce the compilation time while editing
(this was significantly more useful in the earlier days of \LaTeX{})
or to generate a smaller document which is easier to navigate.
Another application of |\includeonly| is to generate
documents consisting of selected parts of the complete document.

However, there are a few drawbacks of the plain |\include| mechanism:
\begin{itemize}
\item
The child files cannot be compiled on their own,
they can only be compiled via the main file.
A naive editing environment
(such as a text editor with an option
to have the current file processed by \LaTeX)
may require one to switch to the main file before compiling;
attempting to compile the child file produces errors.
\item
The main file must be modified (each time)
to adjust the |\includeonly| command
to the present needs. This easily leaves the main file in a messy state.
\item
The generated document will always carry the filename
of the main document. This is inconvenient if
several child files are to be compiled and
to be kept for distribution.
\end{itemize}

The present package provides a simple interface
to make child files individually compilable by \LaTeX{}.
Compiling a child file then has the same effect as compiling
the main file with an |\includeonly| command
to select the appropriate child.
Moreover the generated document will carry the name of the child
rather than the main file.
This resolves all three above issues.

This feature is meant to make the editing of books,
thesis documents and lecture notes somewhat more convenient.
However, the package can also be used efficiently for
composing a series of documents (such as exercise sheets)
which are typically distributed individually.
It then assists the author in generating the individual documents
(potentially in different versions)
as well as a document containing the collected series.
Another application is in developing style files
or other kinds of included material
where compilation of the style file could redirect
to a sample or test file.

%%%%%%%%%%%%%%%%%%%%%%%%%%%%%%%%%%%%%%%%%%%%%%%%%%%%%%%%%%%%%%%%%%%%%%%%%%%%%%%%
%%%%%%%%%%%%%%%%%%%%%%%%%%%%%%%%%%%%%%%%%%%%%%%%%%%%%%%%%%%%%%%%%%%%%%%%%%%%%%%%
\section{Usage}

First of all, the package \textsf{childdoc} is \emph{not} a standard
\LaTeXe{} |.sty| style file! Therefore it needs to be invoked in
a non-standard way.

%%%%%%%%%%%%%%%%%%%%%%%%%%%%%%%%%%%%%%%%%%%%%%%%%%%%%%%%%%%%%%%%%%%%%%%%%%%%%%%%
\subsection{Included Files}
\label{sec:include}

%%%%%%%%%%%%%%%%%%%%%%%%%%%%%%%%%%%%%%%%
\DescribeMacro{\childdocmain}
To use the package, add the commands
\begin{center}
\begin{tabular}{l}
|\input{childdoc.def}|\\
|\childdocmain{}|\\
\end{tabular}
\end{center}
at the very top of the main \LaTeX{} file,
in particular \emph{before} the |\documentclass| statement!
The argument of |\childdocmain| should be left empty
(but it must be present).

%%%%%%%%%%%%%%%%%%%%%%%%%%%%%%%%%%%%%%%%
\DescribeMacro{\childdocof}
Furthermore, add the commands
\begin{center}
\begin{tabular}{l}
|\input{childdoc.def}|\\
|\childdocof{|\textit{main}|}|\\
\end{tabular}
\end{center}
at the top of every child file \textit{child}
which is included by |\include{|\textit{child}|}|
from within the main file
(or at least for those files to be compiled individually).
The argument \textit{main} must be the filename of the main file.

There are a couple of
considerations in setting up the main and child documents:

%%%%%%%%%%%%%%%%%%%%%%%%%%%%%%%%%%%%%%%%
\paragraph{Restrictions.}

Please note the following restrictions:
\begin{itemize}
\item
|\childdocmain| must be called with one argument \textit{main}
to ensure compatibility with earlier version of the package.
It must either be empty (|\childdocmain{}|)
or precisely match the filename of the main file in which it is specified.
See \secref{sec:detection} for further information.
\item
The filename \textit{main} must be specified without the |.tex| extension.
\item
The filename \textit{main} is case sensitive
(even in case-insensitive file systems)
due to internal string comparison.
\item
The argument \textit{main} should be fully expanded, it cannot be a macro.
\item
Subdirectories and special characters should be avoided in filenames.
\item
The command |\childdocmain{|\textit{main}|}| must be followed by a whitespace.
It should not be followed immediately by another command
or by a comment mark `|%|'.
This is because the \TeX{} parser reads the token immediately following
the argument of |\childdocmain| and puts it
at the beginning of every child section;
however, a white\-space is ignored.
\end{itemize}

%%%%%%%%%%%%%%%%%%%%%%%%%%%%%%%%%%%%%%%%
\paragraph{Content of Main File.}

It is advisable to place all content in the child files included by |\include|.
Any output contained in the main file will appear in all child documents
unless suppressed manually;
it cannot be suppressed automatically by the |\includeonly| directive
and thus should normally be avoided.
A method to include some content in the main file
by means of conditional processing is described in \secref{sec:conditional}.

%%%%%%%%%%%%%%%%%%%%%%%%%%%%%%%%%%%%%%%%
\paragraph{Page Numbering.}

When only a part of the document is compiled,
the appropriate numbering of pages
(as well as other status parameters)
is determined from the |.aux| files.
The latter contain information from previous passes.
However this information needs to propagate through
all intermediate child documents.
Therefore the page numbering in child documents may well
be inconsistent until the complete document is compiled at least once.

A useful (if unconventional) way to always ensure a consistent
page numbering is to restart the numbering in each child document
and denote the pages by `\textit{child}|.|\textit{page}'
where \textit{child} represents the chapter/section number of the child file.
This can be achieved by the command
|\numberwithin{page}{|\textit{child}|}|
of the \textsf{amsmath} package
where \textit{child} can be |chapter| or |section|
depending on the chosen structuring.
Alternatively, one can modify the macro |\thepage| appropriately
and reset the counter |page| at the start of each child file.

%%%%%%%%%%%%%%%%%%%%%%%%%%%%%%%%%%%%%%%%%%%%%%%%%%%%%%%%%%%%%%%%%%%%%%%%%%%%%%%%
\subsection{Conditional Processing}
\label{sec:conditional}

The package provides a mechanism to compile different versions
of a document. To customise the versions further some conditional processing
can come in handy to distinguish which version is being compiled.
The package provides two macros to describe the compilation context:

%%%%%%%%%%%%%%%%%%%%%%%%%%%%%%%%%%%%%%%%
\DescribeMacro{\ifchilddoc}
The conditional |\ifchilddoc| distinguishes between the compilation of
child documents and the main document:
%
\begin{center}
|\ifchilddoc |\textit{child-code}| |[|\||else |\textit{main-code}]| \||fi|
\end{center}

%%%%%%%%%%%%%%%%%%%%%%%%%%%%%%%%%%%%%%%%
\DescribeMacro{\childdocname}
\DescribeMacro{\childdocjob}
The macro |\childdocname| contains the filename (without extension)
of the main or child file being processed.
Note that |\childdocjob| will always contain the name of the main file.

%%%%%%%%%%%%%%%%%%%%%%%%%%%%%%%%%%%%%%%%
\paragraph{Title Page.}

Conditional processing can be used to include a title or banner page
in the main document when proper precautions are taken.
Importantly, the code in the main file should ensure that the page counter
(as well as other status parameters which are stored in the |.aux| files)
takes the same value after the conditional processing.
Otherwise the page numbers may take divergent values
depending on which part is compiled.

For example, a title page could be declared by:
%
\begin{center}
\begin{tabular}{l}
|\ifchilddoc\||else|\\
|\addtocounter{page}{-1}|\\
\textit{code for title page}\\
|\newpage|\\
|\||fi|
\end{tabular}
\end{center}
%
A banner page for the child documents can be generated by:
%
\begin{center}
\begin{tabular}{l}
|\ifchilddoc|\\
|\addtocounter{page}{-1}|\\
\textit{code for banner page}\\
|\newpage|\\
|\||fi|
\end{tabular}
\end{center}
%
Here one could write a message such as:
\begin{center}
|This is the part \childdocname{} of \childdocjob{}.|
\end{center}

%%%%%%%%%%%%%%%%%%%%%%%%%%%%%%%%%%%%%%%%%%%%%%%%%%%%%%%%%%%%%%%%%%%%%%%%%%%%%%%%
\subsection{Flags}
\label{sec:flags}

The package makes it easy to generate different versions
of the main or child documents.
To this end compilation flags can be defined
and assigned different default values.
They will be particularly useful in conjunction
with the forwarding mechanism described in \secref{sec:forward}.

For example, it may be useful to have a flag |\version|
which can be set to |draft| or |final|.
The document source will contain some conditional code
depending on the value of |\version|.
Suppose further, the flag should default to |final| for the main file
and to |draft| for child files
which is a natural assignment for editing the document.
This is achieved by placing the following code
in the preamble of the main document
(below the |\childdocmain| directive):
%
\begin{center}
\begin{tabular}{l}
|\ifchilddoc|\\
|\providecommand{\version}{draft}|\\
|\||else|\\
|\providecommand{\version}{final}|\\
|\||fi|
\end{tabular}
\end{center}
%
The definition by |\providecommand| makes sure
that previous definitions are not overwritten.
Further statements |\providecommand{\version}{...}|
can thus be added before the above code to override it.

For the main file, one might add a line
(between |\childdocmain| and the above block)
%
\begin{center}
|%\ifchilddoc\||else\providecommand{\version}{draft}\||fi|
\end{center}
%
which can be uncommented to produce a draft version.
Likewise one can add a line to the very top of a child file
(above the |\childdocof{|\textit{main}|}| directive)
%
\begin{center}
|%\providecommand{\version}{final}|
\end{center}
%
which can be uncommented to produce the final version of this child document.

%%%%%%%%%%%%%%%%%%%%%%%%%%%%%%%%%%%%%%%%%%%%%%%%%%%%%%%%%%%%%%%%%%%%%%%%%%%%%%%%
\subsection{Forwarding}
\label{sec:forward}

Different versions of the main or child documents
using compilation flags as described in \secref{sec:flags}
can be (permanently) stored in different files
for convenient compilation, viewing and distribution.
To this end, the package defines a command
to pass on compilation to a different file:

%%%%%%%%%%%%%%%%%%%%%%%%%%%%%%%%%%%%%%%%
\DescribeMacro{\childdocforward}
The command |\childdocforward| redirects processing to
another source file:
%
\begin{center}
\begin{tabular}{l}
|\input{childdoc.def}|\\
|\childdocforward[|\textit{main}|]{|\textit{dest}|}|\\
\end{tabular}
\end{center}
%
The argument \textit{dest} is the destination file
(without extension).
It should be the main file or one of the child files.
Note that further \textsf{childdoc} directives
such as |\childdocof| and |\childdocforward|
in the indicated file will be processed in this form.
The optional argument \textit{main}
passes on directly to the main file \textit{main}
while pretending to compile the child \textit{dest}.
This form behaves as if \textit{dest}
issues |\childdocof{|\textit{main}|}| right away,
and no further \textsf{childdoc} directives will be processed.

%%%%%%%%%%%%%%%%%%%%%%%%%%%%%%%%%%%%%%%%
\DescribeMacro{\...prefix}
In the alternative form |\childdocforwardprefix|,
%
\begin{center}
\begin{tabular}{l}
|\input{childdoc.def}|\\
|\childdocforwardprefix[|\textit{main}|]{|\textit{prefix}|}{|\textit{dest}|}|
\end{tabular}
\end{center}
%
the destination file is determined by a pattern
depending on the current file:
To make this work, the current file must be called
`{\textit{prefix}\hspace{0.2em}\textit{suffix}}'
with \textit{prefix} matching precisely the argument.
Processing is then passed on to the file
`{\textit{dest}\hspace{0.2em}\textit{suffix}}'.
Surely, the same effect is achieved by
directly specifying the
argument `{\textit{dest}\hspace{0.2em}\textit{suffix}}'
in the first form.
However, that requires to set up a different file
for each child. With the alternative form of the command
all these files can have exactly the same content
which simplifies setting them up and maintaining them.

For example, the following file |draft.tex|
with a compilation flag |\version| as described in \secref{sec:flags}
compiles the main document as a draft:
%
\begin{center}
\begin{tabular}{l}
|\def\version{draft}|\\
|\input{childdoc.def}|\\
|\childdocforward{|\textit{main}|}|
\end{tabular}
\end{center}
%
Likewise, the following files |final|\textit{nn}|.tex|
compile the final version of the child document
|child|\textit{nn}|.tex|:
%
\begin{center}
\begin{tabular}{l}
|\def\version{final}|\\
|\input{childdoc.def}|\\
|\childdocforwardprefix{final}{child}|
\end{tabular}
\end{center}
%

Note that when several versions of a main file and/or of each child file
are to be generated, it may be convenient to set up a |Makefile| or
shell script to automatise the process.

%%%%%%%%%%%%%%%%%%%%%%%%%%%%%%%%%%%%%%%%%%%%%%%%%%%%%%%%%%%%%%%%%%%%%%%%%%%%%%%%
\subsection{Command Line Processing}
\label{sec:commandline}

The effect of redirection files can also be achieved by invoking
the \LaTeX{} compiler with a more elaborate command line.
Most conveniently this should be done as part
of a shell script or a |Makefile|.

When using \textsf{childdoc} in the main file, the following
command lines effectively perform a redirection
(note that depending on the shell being used,
backslashes may have to be doubled: `|\|' $\to$ `|\\|'):
%
\begin{center}
|... -jobname "|\textit{target}|" |\\|"|[\textit{flags}]%
|\input{childdoc.def}\childdocforward[|\textit{main}|]{|\textit{dest}|}"|
\end{center}
%
Here \textit{target} is the name of the output file,
\textit{main} is the name of the main file
and \textit{dest} is the name of the main or child file to be processed
(all filenames without extensions).
The optional argument \textit{main} can be omitted
if \textit{main} matches \textit{dest}.
Optionally, compilation \textit{flags} can be defined via |\def| commands.
This command line makes the \TeX{} engine believe
it is compiling the file \textit{target}
whose content is specified as the latter parameter.
The provided code then forwards the processing to
\textit{main} or \textit{dest} as described in \secref{sec:forward}.

%%%%%%%%%%%%%%%%%%%%%%%%%%%%%%%%%%%%%%%%%%%%%%%%%%%%%%%%%%%%%%%%%%%%%%%%%%%%%%%%
\subsection{Include by Input}
\label{sec:input}

Including child documents by |\include| has some restrictions by design.
Most notably, the content of a child document always occupies
its own set of pages; pages cannot be shared between child documents.
Usually, this behaviour makes perfect sense
because each child document contain an essential part of the document.
However, in some situations it may be desirable to compose
a document from a collection of parts
without having mandatory page breaks between then.
For this case, the package
provides a mechanism to include parts
by |\input| which can also be processed individually.
However, by construction this mechanism
requires manual handling of the content to be output.

%%%%%%%%%%%%%%%%%%%%%%%%%%%%%%%%%%%%%%%%
\DescribeMacro{\ifchilddocmanual}
The main file should be prepared as usual, see \secref{sec:include}.
However, the document body must make a distinction
between processing of an individual part and of the main document, e.g.:
%
\begin{center}
\begin{tabular}{l}
|\ifchilddocmanual|\\
|\input{\childdocname}|\\
|\||else|\\
\textit{document body with }|\input{|\textit{part}|}|\\
|\||fi|
\end{tabular}
\end{center}
%
The conditional |\ifchilddocmanual| is true whenever
a part to be included by |\input| is being compiled,
and the name of the part is stored in |\childdocname|.

%%%%%%%%%%%%%%%%%%%%%%%%%%%%%%%%%%%%%%%%
\DescribeMacro{\childdocby}
Each part to be included by |\input| should start with:
%
\begin{center}
\begin{tabular}{l}
|\input{childdoc.def}|\\
|\childdocby{|\textit{main}|}|\\
\end{tabular}
\end{center}
%
The directive |\childdocby| is similar to |\childdocof|
described in \secref{sec:include},
but the subsequent selection of content must be done manually.
To that end, both |\ifchilddoc| and |\ifchilddocmanual|
will be true upon processing of a part,
and the name of the part is stored in |\childdocname|.
Note that |\jobname| will be set to the filename of the current part
so that each part receives an individual |.aux| file
that does not interfere with the |.aux| file(s) of the main document.
This behaviour can be altered by the alternative form
|\childdocby[*]{|\textit{main}|}| (with a non-empty optional argument)
which uses the |.aux| file of the main document
by setting |\jobname| to \textit{main}.

%%%%%%%%%%%%%%%%%%%%%%%%%%%%%%%%%%%%%%%%%%%%%%%%%%%%%%%%%%%%%%%%%%%%%%%%%%%%%%%%
\subsection{Driver Development}
\label{sec:driver}

The \textsf{childdoc} mechanism can also be use for the development
of definition files such as \LaTeX{} styles or classes.
This case differs from the above setup with multiple parts
included by |\include| in that no |\includeonly| should be invoked.
This can be achieved by starting the include file
(before |\ProvidesPackage|) with:
%
\begin{center}
\begin{tabular}{l}
|\input{childdoc.def}|\\
|\childdocforward{|\textit{main}|}|\\
\end{tabular}
\end{center}
%
or alternatively with:
%
\begin{center}
\begin{tabular}{l}
|\input{childdoc.def}|\\
|\childdocby{|\textit{main}|}|\\
\end{tabular}
\end{center}
%
Both forms have slightly different effects as described above.
The main file is prepared as usual, see \secref{sec:include}.

%%%%%%%%%%%%%%%%%%%%%%%%%%%%%%%%%%%%%%%%%%%%%%%%%%%%%%%%%%%%%%%%%%%%%%%%%%%%%%%%
\subsection{Legacy Detection}
\label{sec:detection}

The directive |\childdocmain| in the main file can detect
whether the complete document or merely a child is to be compiled
even without using the directive |\childdocof|.
This method is deprecated because it is less robust
and there is no compelling reason to use it;
it is merely provided for backward compatibility
and it may be removed in future versions.

If the detection mechanism is to be used,
it is mandatory to correctly specify
the filename of the main file as the argument of |\childdocmain|:
%
\begin{center}
\begin{tabular}{l}
|\input{childdoc.def}|\\
|\childdocmain{|\textit{main}|}|\\
\end{tabular}
\end{center}
%
If |\jobname| does not match the argument \textit{main} of |\childdocmain|,
it is assumed that |\jobname| points to the child file to be compiled.
When using |\childdocmain| with the main file specified as argument,
it suffices to start a child file
with just |\input{|\textit{main}|}|
without loading of the package and using |\childdocof|.
If instead all processing is done
with the appropriate \textsf{childdoc} directives,
the argument of \textit{main} of |\childdocmain| can be empty.

An alternative version of the command line processing described
in \secref{sec:commandline} using the detection mechanism reads:
%
\begin{center}
|... -jobname "|\textit{target}|" "|[\textit{flags}]%
[|\def\jobname{|\textit{dest}|}|]|\input{|\textit{main}|}"|
\end{center}

%%%%%%%%%%%%%%%%%%%%%%%%%%%%%%%%%%%%%%%%%%%%%%%%%%%%%%%%%%%%%%%%%%%%%%%%%%%%%%%%
\subsection{Manual Code}
\label{sec:manual}

In case one cannot be certain whether the definitions file |childdoc.def|
is installed on the target \TeX{} distribution
and one prefers not to ship it,
it is conceivable to paste a few relevant commands into the sources.

To that end, drop all statements |\input{childdoc.def}|
and perform the replacements as outlined below.
Instead of |\childdocmain{|\textit{main}|}| add the following code
to the top of the main file:
%
\begin{center}
\begin{tabular}{l}
|\||ifdefined\childdocname\endinput\||fi\newif\ifchilddoc|\\
|\edef\childdocname{\scantokens\expandafter{\jobname\noexpand}}|\\
|\def\childdocmain{|\textit{main}|}\||ifx\childdocmain\childdocname\||else|\\
|\childdoctrue\includeonly{\childdocname}\let\jobname\childdocmain\||fi|\\
\end{tabular}
\end{center}
%
Instead of |\childdocof{|\textit{main}|}| just include the main file
at the top of each child file:
%
\begin{center}
|\input{|\textit{main}|}|
\end{center}
%
A simple redirection |\childdocforward{|\textit{dest}|}| is achieved by:
%
\begin{center}
|\def\jobname{|\textit{dest}|}\input{\jobname}|
\end{center}
%
The redirection with prefix
|\childdocforwardprefix[|\textit{prefix}|]{|\textit{dest}|}|
is accomplished by:
%
\begin{center}
\begin{tabular}{l}
|{\edef\jobname{\scantokens\expandafter{\jobname\noexpand}}|\\
|\def\redirectjob |\textit{prefix}|#1~~~{\gdef\jobname{|\textit{dest}|#1}}|\\
|\expandafter\redirectjob\jobname~~~}\input{\jobname}|
\end{tabular}
\end{center}

In an alternative approach,
child documents can be compiled by a specific command line
without additional code or specific definitions:
%
\begin{center}
|... -jobname "|\textit{target}|" "|[\textit{flags}]%
|\includeonly{|\textit{dest}|}\input{|\textit{main}|}"|
\end{center}
%

%%%%%%%%%%%%%%%%%%%%%%%%%%%%%%%%%%%%%%%%%%%%%%%%%%%%%%%%%%%%%%%%%%%%%%%%%%%%%%%%
%%%%%%%%%%%%%%%%%%%%%%%%%%%%%%%%%%%%%%%%%%%%%%%%%%%%%%%%%%%%%%%%%%%%%%%%%%%%%%%%
\section{Information}

%%%%%%%%%%%%%%%%%%%%%%%%%%%%%%%%%%%%%%%%%%%%%%%%%%%%%%%%%%%%%%%%%%%%%%%%%%%%%%%%
\subsection{Copyright}

Copyright \copyright{} 2017--2018 Niklas Beisert

This work may be distributed and/or modified under the
conditions of the \LaTeX{} Project Public License, either version 1.3
of this license or (at your option) any later version.
The latest version of this license is in
  \url{http://www.latex-project.org/lppl.txt}
and version 1.3 or later is part of all distributions of \LaTeX{}
version 2005/12/01 or later.

This work has the LPPL maintenance status `maintained'.

The Current Maintainer of this work is Niklas Beisert.

This work consists of the files |README.txt|, |childdoc.ins| and |childdoc.dtx|
as well as the derived files |childdoc.def|, |cdocsamp.tex|
with |cdocsch1.tex|, |cdocsch2.tex|, |cdocspt3.tex|, |cdocspt4.tex|,
|cdocsdrf.tex|, |cdocsfn1.tex|, |cdocsfn2.tex|
as well as |childdoc.pdf|.

%%%%%%%%%%%%%%%%%%%%%%%%%%%%%%%%%%%%%%%%%%%%%%%%%%%%%%%%%%%%%%%%%%%%%%%%%%%%%%%%
\subsection{Files and Installation}

The package consists of the files:
%
\begin{center}
\begin{tabular}{ll}
    |README.txt|   & readme file \\
    |childdoc.ins| & installation file \\
    |childdoc.dtx| & source file \\
    |childdoc.def| & definition file \\
    |cdocsamp.tex| & sample main file \\
    |cdocsch1.tex| & sample include file \\
    |cdocsch2.tex| & sample include file \\
    |cdocspt3.tex| & sample part file \\
    |cdocspt4.tex| & sample part file \\
    |cdocsdrf.tex| & sample redirection file \\
    |cdocsfn1.tex| & sample redirection file \\
    |cdocsfn2.tex| & sample redirection file \\
    |childdoc.pdf| & manual
\end{tabular}
\end{center}
%
The distribution consists of the files
|README.txt|, |childdoc.ins| and |childdoc.dtx|.
%
\begin{itemize}
\item
Run (pdf)\LaTeX{} on |childdoc.dtx|
to compile the manual |childdoc.pdf| (this file).
\item
Run \LaTeX{} on |childdoc.ins| to create the definitions file |childdoc.def|
and the sample |cdocsamp.tex| with include files
|cdocsch1.tex|, |cdocsch2.tex|, |cdocspt3.tex|, |cdocspt4.tex|,
|cdocsdrf.tex|, |cdocsfn1.tex|, |cdocsfn2.tex|.
Then copy the file |childdoc.def| to an appropriate directory of your \LaTeX{}
distribution, e.g.\ \textit{texmf-root}|/tex/latex/childdoc|.
\end{itemize}

%%%%%%%%%%%%%%%%%%%%%%%%%%%%%%%%%%%%%%%%%%%%%%%%%%%%%%%%%%%%%%%%%%%%%%%%%%%%%%%%
\subsection{Related CTAN Packages}

There are several other packages which offer a similar functionality:
%
\begin{itemize}
\item
The packages
\href{http://ctan.org/pkg/docmute}{\textsf{docmute}},
\href{http://ctan.org/pkg/includex}{\textsf{includex}} and
\href{http://ctan.org/pkg/standalone}{\textsf{standalone}}
provide commands to include only the document body of
a child file thus allowing both files to be compiled individually.
\item
The packages \href{http://ctan.org/pkg/subdocs}{\textsf{subdocs}}
and \href{http://ctan.org/pkg/subfiles}{\textsf{subfiles}}
provide structures in which the main and child documents can be
encapsulated and allowing them to be compiled individually.
The inclusion mechanism is different from the conventional |\include|.
\item
The package \href{http://ctan.org/pkg/combine}{\textsf{combine}}
is an elaborate solution to combine several documents into one.
\end{itemize}
%
See also the CTAN topic \href{http://ctan.org/topic/subdocs}{\textsf{subdocs}}
for further related packages.
The present package differs from the above solutions in that
a document structure constructed with the conventional |\include| mechanism
just needs two extra commands at the top of every file
such that all constituent files can be compiled individually.

%%%%%%%%%%%%%%%%%%%%%%%%%%%%%%%%%%%%%%%%%%%%%%%%%%%%%%%%%%%%%%%%%%%%%%%%%%%%%%%%
%\subsection{Feature Suggestions}
%
%The following is a list of features which may be useful for future
%versions of this package:
%%
%\begin{itemize}
%\item
%\ldots
%\end{itemize}

%%%%%%%%%%%%%%%%%%%%%%%%%%%%%%%%%%%%%%%%%%%%%%%%%%%%%%%%%%%%%%%%%%%%%%%%%%%%%%%%
\subsection{Revision History}

%%%%%%%%%%%%%%%%%%%%%%%%%%%%%%%%%%%%%%%%
\paragraph{v2.0:} 2018/12/30

\begin{itemize}
\item
immediate forward processing
\item
added |\childdocby| mechanism
\item
manual restructured
\end{itemize}

%%%%%%%%%%%%%%%%%%%%%%%%%%%%%%%%%%%%%%%%
\paragraph{v1.6:} 2018/01/17

\begin{itemize}
\item
application for development of include files
\item
corrections to manual
\end{itemize}

%%%%%%%%%%%%%%%%%%%%%%%%%%%%%%%%%%%%%%%%
\paragraph{v1.5:} 2017/05/21

\begin{itemize}
\item
more complete structuring introduced
\item
|\childdocof| introduced
\item
|\childdoc| renamed to |\childdocmain|
\item
|\childredirect| renamed to |\childdocforward| and |\childdocforwardprefix|
and functionality expanded
\end{itemize}

%%%%%%%%%%%%%%%%%%%%%%%%%%%%%%%%%%%%%%%%
\paragraph{v1.0:} 2017/04/27

\begin{itemize}
\item
manual and install package
\item
first version published on CTAN
\end{itemize}

%%%%%%%%%%%%%%%%%%%%%%%%%%%%%%%%%%%%%%%%
\paragraph{v0.6:} 2017/04/26

\begin{itemize}
\item
redirection mechanism added
\end{itemize}

%%%%%%%%%%%%%%%%%%%%%%%%%%%%%%%%%%%%%%%%
\paragraph{v0.5:} 2017/04/26

\begin{itemize}
\item
functionality in definition file
\end{itemize}


%%%%%%%%%%%%%%%%%%%%%%%%%%%%%%%%%%%%%%%%%%%%%%%%%%%%%%%%%%%%%%%%%%%%%%%%%%%%%%%%
%%%%%%%%%%%%%%%%%%%%%%%%%%%%%%%%%%%%%%%%%%%%%%%%%%%%%%%%%%%%%%%%%%%%%%%%%%%%%%%%
%%%%%%%%%%%%%%%%%%%%%%%%%%%%%%%%%%%%%%%%%%%%%%%%%%%%%%%%%%%%%%%%%%%%%%%%%%%%%%%%
\appendix

\settowidth\MacroIndent{\rmfamily\scriptsize 000\ }

 \DocInput{childdoc.dtx}

\end{document}
%</driver>
% \fi
%
% %%%%%%%%%%%%%%%%%%%%%%%%%%%%%%%%%%%%%%%%%%%%%%%%%%%%%%%%%%%%%%%%%%%%%%%%%%%%%%
% %%%%%%%%%%%%%%%%%%%%%%%%%%%%%%%%%%%%%%%%%%%%%%%%%%%%%%%%%%%%%%%%%%%%%%%%%%%%%%
% \section{Sample}
%\iffalse
%<*samplemain>
%\fi
%
% The following presents a sample document
% with two chapters, two parts, a title page,
% a compile flag as well as three forwarding files to set the flag.
% It consists of eight |.tex| files:
% \begin{center}
% \begin{tabular}{ll}
% |cdocsamp.tex|&main file\\
% |cdocsch1.tex|&include file for chapter 1\\
% |cdocsch2.tex|&include file for chapter 2\\
% |cdocspt3.tex|&include file for part 3\\
% |cdocspt4.tex|&include file for part 4\\
% |cdocsdrf.tex|&forwarding file for main file in draft mode\\
% |cdocsfi1.tex|&forwarding file for final version of chapter 1\\
% |cdocsfi2.tex|&forwarding file for final version of chapter 2\\
% \end{tabular}
% \end{center}
% Each of the eight files can be compiled directly by the \LaTeX{} compiler.
%
% %%%%%%%%%%%%%%%%%%%%%%%%%%%%%%%%%%%%%%
% \paragraph{Main File.}
%
% The main file is called |cdocsamp.tex|.
%
% Load the \textsf{childdoc} definitions and
% declare the filename for the main document:
%    \begin{macrocode}
\input{childdoc.def}
\childdocmain{}
%    \end{macrocode}

% Optional override for |\version| flag:
%    \begin{macrocode}
%%\ifchilddoc\else\providecommand{\version}{draft}\fi
%    \end{macrocode}

% Define the default values for the |\version| flag
% (|final| for the main file and |draft| for childs):
%    \begin{macrocode}
\ifchilddoc
\providecommand{\version}{draft}
\else
\providecommand{\version}{final}
\fi
%    \end{macrocode}

% Load the standard document class:
%    \begin{macrocode}
\documentclass[12pt]{article}
%    \end{macrocode}

% Start the document body:
%    \begin{macrocode}
\begin{document}
%    \end{macrocode}

% Declare a title page.
% Print title, part of document being processed and version flag:
%    \begin{macrocode}
\addtocounter{page}{-1}
\begin{center}
{\LARGE\bfseries{}childdoc example\par}
\vspace{1cm}
\ifchilddoc
\ifchilddocmanual part\else chapter\fi:
`\childdocname' of `\childdocjob'\par
\else
main document: `\childdocjob'\par
\fi
version: \version\par
\end{center}
\newpage
%    \end{macrocode}

% Manually include selected file,
% otherwise process as usual:
%    \begin{macrocode}
\ifchilddocmanual
\section*{part `\childdocname'}
\input{\childdocname}
\else
%    \end{macrocode}

% Include the two chapters:
%    \begin{macrocode}
\include{cdocsch1}
\include{cdocsch2}
%    \end{macrocode}

% Include the two parts unless only chapters should be displayed:
%    \begin{macrocode}
\ifchilddoc\else
\section{part three}
\input{cdocspt3}
\section{part four}
\input{cdocspt4}
\fi
%    \end{macrocode}

% Process as usual until here:
%    \begin{macrocode}
\fi
%    \end{macrocode}

% End of document body:
%    \begin{macrocode}
\end{document}
%    \end{macrocode}
%\iffalse
%</samplemain>
%\fi
%
% %%%%%%%%%%%%%%%%%%%%%%%%%%%%%%%%%%%%%%
% \paragraph{Chapter Include Files.}
%
% The include files are called |cdocsch1.tex| and |cdocsch2.tex|.
%
%\iffalse
%<*samplechap1|samplechap2>
%\fi

% Optional override for |\version| flag:
%    \begin{macrocode}
%%\providecommand{\version}{final}
%    \end{macrocode}

% Include the main document:
%    \begin{macrocode}
\input{childdoc.def}
\childdocof{cdocsamp}
%    \end{macrocode}

%\iffalse
%</samplechap1|samplechap2>
%\fi
%
%\iffalse
%<*samplechap1>
%\fi
% Some text for chapter 1:
%    \begin{macrocode}
\section{one}
some text in chapter one
%    \end{macrocode}

%\iffalse
%</samplechap1>
%\fi
% Some text for chapter 2:
%\iffalse
%<*samplechap2>
%\fi
%    \begin{macrocode}
\section{two}
more text in chapter two
%    \end{macrocode}

%\iffalse
%</samplechap2>
%\fi
%
% %%%%%%%%%%%%%%%%%%%%%%%%%%%%%%%%%%%%%%
% \paragraph{Part Include Files.}
%
% The include files are called |cdocspt3.tex| and |cdocspt4.tex|.
%
%\iffalse
%<*samplepart3|samplepart4>
%\fi

% Optional override for |\version| flag:
%    \begin{macrocode}
%%\providecommand{\version}{final}
%    \end{macrocode}

% Include the main document:
%    \begin{macrocode}
\input{childdoc.def}
\childdocby{cdocsamp}
%    \end{macrocode}

%\iffalse
%</samplepart3|samplepart4>
%\fi
%
%\iffalse
%<*samplepart3>
%\fi
% Some text for part 3:
%    \begin{macrocode}
some text in part three
%    \end{macrocode}

%\iffalse
%</samplepart3>
%\fi
% Some text for part 4:
%\iffalse
%<*samplepart4>
%\fi
%    \begin{macrocode}
more text in part four
%    \end{macrocode}

%\iffalse
%</samplepart4>
%\fi
%
% %%%%%%%%%%%%%%%%%%%%%%%%%%%%%%%%%%%%%%
% \paragraph{Forwarding for a Complete Draft.}
%
% The following forwarding file |cdocsdrf.tex|
% compiles the main document in draft mode:
%\iffalse
%<*sampledraft>
%\fi
%    \begin{macrocode}
\def\version{draft}
\input{childdoc.def}
\childdocforward{cdocsamp}
%    \end{macrocode}

%\iffalse
%</sampledraft>
%\fi
%
% %%%%%%%%%%%%%%%%%%%%%%%%%%%%%%%%%%%%%%
% \paragraph{Forwarding for Final Version of the Chapters.}
%
% The following forwarding files |cdocsfn1.tex| and |cdocsfn2.tex|
% (with identical content)
% compile the final versions of the child documents
% |cdocsch1.tex| and |cdocsch2.tex|, respectively:
%\iffalse
%<*samplefinal>
%\fi
%    \begin{macrocode}
\def\version{final}
\input{childdoc.def}
\childdocforwardprefix[cdocsamp]{cdocsfn}{cdocsch}
%    \end{macrocode}

%\iffalse
%</samplefinal>
%\fi
%
% %%%%%%%%%%%%%%%%%%%%%%%%%%%%%%%%%%%%%%
% \paragraph{Command Line Processing.}
%
% The following three command lines generate the output files
% |cdocscld|, |cdocscl1| and |cdocscl2|
% which should be identical to
% |cdocsdrf|, |cdocsch1| and |cdocsfn2|, respectively:
% \begin{center}
% \begin{tabular}{l}
% |latex -jobname cdocscld \|\\
% |  "\def\version{draft}\input{childdoc.def}\childdocforward{cdocsamp}"|\\
% |latex -jobname cdocscl1 \|\\
% |  "\input{childdoc.def}\childdocforward[cdocsamp]{cdocsch1}"|\\
% |latex -jobname cdocscl2 \|\\
% |  "\def\version{final}\input{childdoc.def}\childdocforward{cdocsch2}"|
% \end{tabular}
% \end{center}
% Note that the trailing backslash on each first line
% merely continues the input to the second line
% (for convenient cut ant paste).
% Furthermore, the command |latex| can be replaced by any
% of its alternative versions such as |pdflatex|.
%
% %%%%%%%%%%%%%%%%%%%%%%%%%%%%%%%%%%%%%%%%%%%%%%%%%%%%%%%%%%%%%%%%%%%%%%%%%%%%%%
% %%%%%%%%%%%%%%%%%%%%%%%%%%%%%%%%%%%%%%%%%%%%%%%%%%%%%%%%%%%%%%%%%%%%%%%%%%%%%%
% \section{Implementation}
%\iffalse
%<*package>
%\fi
%
% This section describes the definitions file |childdoc.def|.

% The definitions cannot be loaded using |\usepackage| or |\RequirePackage|
% which has a mechanism to prevent loading a style file more than once.
% When loading the definitions by means of |\input|
% multiple instances have to be prevented manually:
%\iffalse
%This code needs to be before the `\ProvidesFile' directive
%which is defined at the beginning of this file.
%Therefore it is also placed there and commented out here.
%</package>
%<*discard>
%\fi
%    \begin{macrocode}
\ifdefined\childdocmain\endinput\fi
%    \end{macrocode}
%\iffalse
%</discard>
%<*package>
%\fi
%
% \macro{\ifchilddoc}
% \macro{\ifchilddocmanual}
% The conditional |\ifchilddoc| tells whether a
% child (true) or main (false) document is being compiled.
% The conditional |\ifchilddocmanual| tells whether
% the |\includeonly| mechanism is used (false) or
% the selection of child files must be performed manually (true).
% The definitions initialise to false:
%    \begin{macrocode}
\newif\ifchilddoc
\newif\ifchilddocmanual
%    \end{macrocode}

% \macro{\childdocname}
% \macro{\childdocjob}
% The macro |\childdocname| stores the name of the main document
% to be compiled. The macro |\childdocjob| stores the name of
% the document on which the \LaTeX{} compiler was originally invoked.
% The content of |\jobname| cannot be compared
% to filenames specified in the source due to different catcodes.
% The following code rescans |\jobname|, stores the result
% in |\childdocname| and saves a copy in |\childdocjob|:
%    \begin{macrocode}
\edef\childdocname{\scantokens\expandafter{\jobname\noexpand}}
\let\childdocjob\childdocname
%    \end{macrocode}

% \macro{\childdocdisable}
% The macro |\childdocdisable| prevents the main file
% from being processed more than once.
% At this stage, the main document command |\childdocmain|
% is assumed to be called once again where it should do nothing.
% Any subsequent call to it should prevent
% a secondary processing of the main document
% It overwrites the forwarding commands
% |\childdocof| and |\childdocforward|
% with empty macros to prevent further inclusions of the main document:
%    \begin{macrocode}
\newcommand{\childdocdisable}
{
  \renewcommand{\childdocmain}[1]{\renewcommand{\childdocmain}[1]{\endinput}}
  \renewcommand{\childdocof}[1]{}
  \renewcommand{\childdocby}[2][]{}
  \renewcommand{\childdocforward}[2][]{}
  \renewcommand{\childdocdisable}{}
}
%    \end{macrocode}

% \macro{\childdocmain}
% The macro |\childdocmain| is to be called at the top of the main file
% with nothing or the main filename (without extension) as argument.
% First, it breaks loops.
% If the argument is not empty and does not match |\childdocname|
% (which is set by the first inclusion of |childdoc.def|),
% |\ifchilddoc| is set to true, |\includeonly| is applied to the child file
% and |\jobname| is set to the main file
% (for proper handling of |.aux| files):
%    \begin{macrocode}
\newcommand{\childdocmain}[1]
{
  \childdocdisable\childdocmain{}
  \if?#1?\else
    \begingroup
      \def\childdoctmp{#1}
      \ifx\childdoctmp\childdocname
        \def\childdoctmp{}
      \else
        \def\childdoctmp
        {
          \childdoctrue
          \includeonly{\childdocname}
          \def\childdocjob{#1}
          \def\jobname{#1}
        }
      \fi
      \expandafter
    \endgroup
    \childdoctmp
  \fi
}
%    \end{macrocode}

% \macro{\childdocof}
% The command |\childdocof| redirects
% compilation to the main file |#1|.
%    \begin{macrocode}
\newcommand{\childdocof}[1]
{
  \childdocdisable
  \childdoctrue
  \includeonly{\childdocname}
  \def\jobname{#1}
  \def\childdocjob{#1}
  \input{#1}
}
%    \end{macrocode}

% \macro{\childdocby}
% The command |\childdocby| ....
%    \begin{macrocode}
\newcommand{\childdocby}[2][]
{
  \childdocdisable
  \childdoctrue
  \childdocmanualtrue
  \if?#1?\else
    \def\jobname{#2}
  \fi
  \def\childdocjob{#2}
  \input{#2}
  \endinput
}
%    \end{macrocode}

% \macro{\childdocforward}
% The command |\childdocforward| redirects
% compilation to the main file or
% (if the optional argument is given) a child file.
% Parameters are set as if the main file
% or a child file starting with |\childdocof| was compiled.
% Then compilation is handed over to the main file:
%    \begin{macrocode}
\newcommand{\childdocforward}[2][]
{
  \begingroup
    \if?#1?
      \def\childdoctmp
      {
        \def\childdocname{#2}
        \def\childdocjob{#2}
        \def\jobname{#2}
        \input{#2}
        \endinput
      }
    \else
      \def\childdoctmp
      {
        \childdocdisable
        \def\childdocname{#2}
        \childdoctrue
        \includeonly{#2}
        \def\childdocjob{#1}
        \def\jobname{#1}
        \input{#1}
        \endinput
      }
    \fi
    \expandafter
  \endgroup
  \childdoctmp
}
%    \end{macrocode}

% \macro{\childdocforwardprefix}
% The command |\childdocforwardprefix| redirects
% compilation to the main or a child file by means of a pattern.
% The prefix |#1| in the current filename is replaced by |#2|
% and the suffix of the current filename is kept
% (it is assumed that the filename does not contain the substring `|~~~|'
% which is used as a delimiter).
% Compilation is handed over to the new file by |\childdocforward|:
%    \begin{macrocode}
\newcommand{\childdocforwardprefix}[3][]
{
  \begingroup
    \def\childdocextract #2##1~~~{\def\childdoctmp{\childdocforward[#1]{#3##1}}}
    \expandafter\childdocextract\childdocname~~~
    \expandafter
  \endgroup
  \childdoctmp
}
%    \end{macrocode}

% \macro{\childdoc}
% The deprecated macro |\childdoc| is a legacy version of |\childdocmain|:
%    \begin{macrocode}
\newcommand{\childdoc}{\childdocmain}
%    \end{macrocode}

% \macro{\childdocredirect}
% The deprecated macro |\childdocredirect| is a legacy version
% of |\childdocforward| and |\childdocforwardprefix|:
%    \begin{macrocode}
\newcommand{\childdocredirect}[2][]
{
  \begingroup
    \if?#1?
      \def\childdoctmp{\childdocforward{#2}}
    \else
      \def\childdoctmp{\childdocforwardprefix{#1}{#2}}
    \fi
    \expandafter
  \endgroup
  \childdoctmp
}
%    \end{macrocode}

%\iffalse
%</package>
%\fi
%
\endinput

\childdocforward{cdocsamp}
%    \end{macrocode}

%\iffalse
%</sampledraft>
%\fi
%
% %%%%%%%%%%%%%%%%%%%%%%%%%%%%%%%%%%%%%%
% \paragraph{Forwarding for Final Version of the Chapters.}
%
% The following forwarding files |cdocsfn1.tex| and |cdocsfn2.tex|
% (with identical content)
% compile the final versions of the child documents
% |cdocsch1.tex| and |cdocsch2.tex|, respectively:
%\iffalse
%<*samplefinal>
%\fi
%    \begin{macrocode}
\def\version{final}
% \iffalse
%
% childdoc.dtx Copyright (C) 2017-2018 Niklas Beisert
%
% This work may be distributed and/or modified under the
% conditions of the LaTeX Project Public License, either version 1.3
% of this license or (at your option) any later version.
% The latest version of this license is in
%   http://www.latex-project.org/lppl.txt
% and version 1.3 or later is part of all distributions of LaTeX
% version 2005/12/01 or later.
%
% This work has the LPPL maintenance status `maintained'.
%
% The Current Maintainer of this work is Niklas Beisert.
%
% This work consists of the files childdoc.dtx and childdoc.ins
% and the derived files childdoc.def and cdocsamp.tex with
% cdocsch1.tex, cdocsch2.tex, cdocsdrf.tex, cdocsfn1.tex, cdocsfn2.tex.
%
%<package>\ifdefined\childdocmain\endinput\fi
%<package>\ProvidesFile{childdoc.def}[2018/12/30 v2.0 child document driver]
%<samplemain>\ProvidesFile{cdocsamp.tex}[2018/12/30 v2.0 sample for childdoc]
%<*driver>
%\ProvidesFile{childdoc.drv}[2018/12/30 v2.0 childdoc reference manual file]
\PassOptionsToClass{10pt,a4paper}{article}
\documentclass{ltxdoc}

\usepackage[margin=35mm]{geometry}
\usepackage{hyperref}
\usepackage{hyperxmp}
\usepackage[usenames]{color}

\hypersetup{colorlinks=true}
\hypersetup{pdfstartview=FitH}
\hypersetup{pdfpagemode=UseNone}
\hypersetup{pdfsource={}}
\hypersetup{pdflang={en-UK}}
\hypersetup{pdfcopyright={Copyright 2017-2018 Niklas Beisert.
  This work may be distributed and/or modified under the
  conditions of the LaTeX Project Public License, either version 1.3
  of this license or (at your option) any later version.}}
\hypersetup{pdflicenseurl={http://www.latex-project.org/lppl.txt}}
\hypersetup{pdfcontactaddress={ETH Zurich, ITP, HIT K,
  Wolfgang-Pauli-Strasse 27}}
\hypersetup{pdfcontactpostcode={8093}}
\hypersetup{pdfcontactcity={Zurich}}
\hypersetup{pdfcontactcountry={Switzerland}}
\hypersetup{pdfcontactemail={nbeisert@itp.phys.ethz.ch}}
\hypersetup{pdfcontacturl={http://people.phys.ethz.ch/\xmptilde nbeisert/}}

\newcommand{\secref}[1]{\hyperref[#1]{section \ref*{#1}}}

\parskip1ex
\parindent0pt
\let\olditemize\itemize
\def\itemize{\olditemize\parskip0pt}

\begin{document}

\title{The \textsf{childdoc} Package}
\hypersetup{pdftitle={The childdoc Package}}
\author{Niklas Beisert\\[2ex]
  Institut f\"ur Theoretische Physik\\
  Eidgen\"ossische Technische Hochschule Z\"urich\\
  Wolfgang-Pauli-Strasse 27, 8093 Z\"urich, Switzerland\\[1ex]
  \href{mailto:nbeisert@itp.phys.ethz.ch}
  {\texttt{nbeisert@itp.phys.ethz.ch}}}
\hypersetup{pdfauthor={Niklas Beisert}}
\hypersetup{pdfsubject={Manual for the LaTeX2e Package childdoc}}
\date{30 December 2018, \textsf{v2.0}}
\maketitle

\begin{abstract}\noindent
\textsf{childdoc} is a \LaTeXe{} package
that enables the direct compilation
of document sections included by |\include|
to individual files.
\end{abstract}

\begingroup
\parskip0ex
\tableofcontents
\endgroup

%%%%%%%%%%%%%%%%%%%%%%%%%%%%%%%%%%%%%%%%%%%%%%%%%%%%%%%%%%%%%%%%%%%%%%%%%%%%%%%%
%%%%%%%%%%%%%%%%%%%%%%%%%%%%%%%%%%%%%%%%%%%%%%%%%%%%%%%%%%%%%%%%%%%%%%%%%%%%%%%%
\section{Introduction}

\LaTeX{} provides a mechanism to structure a large document (such as a book)
into a main file and several child files (containing the chapters)
using the |\include| command.
This mechanism is beneficial for documents
which span hundreds of pages in order to
make the source file(s) more manageable.
Moreover, compilation can be restricted to
selected child files by means of the |\includeonly| command.
The latter feature can be used to reduce the compilation time while editing
(this was significantly more useful in the earlier days of \LaTeX{})
or to generate a smaller document which is easier to navigate.
Another application of |\includeonly| is to generate
documents consisting of selected parts of the complete document.

However, there are a few drawbacks of the plain |\include| mechanism:
\begin{itemize}
\item
The child files cannot be compiled on their own,
they can only be compiled via the main file.
A naive editing environment
(such as a text editor with an option
to have the current file processed by \LaTeX)
may require one to switch to the main file before compiling;
attempting to compile the child file produces errors.
\item
The main file must be modified (each time)
to adjust the |\includeonly| command
to the present needs. This easily leaves the main file in a messy state.
\item
The generated document will always carry the filename
of the main document. This is inconvenient if
several child files are to be compiled and
to be kept for distribution.
\end{itemize}

The present package provides a simple interface
to make child files individually compilable by \LaTeX{}.
Compiling a child file then has the same effect as compiling
the main file with an |\includeonly| command
to select the appropriate child.
Moreover the generated document will carry the name of the child
rather than the main file.
This resolves all three above issues.

This feature is meant to make the editing of books,
thesis documents and lecture notes somewhat more convenient.
However, the package can also be used efficiently for
composing a series of documents (such as exercise sheets)
which are typically distributed individually.
It then assists the author in generating the individual documents
(potentially in different versions)
as well as a document containing the collected series.
Another application is in developing style files
or other kinds of included material
where compilation of the style file could redirect
to a sample or test file.

%%%%%%%%%%%%%%%%%%%%%%%%%%%%%%%%%%%%%%%%%%%%%%%%%%%%%%%%%%%%%%%%%%%%%%%%%%%%%%%%
%%%%%%%%%%%%%%%%%%%%%%%%%%%%%%%%%%%%%%%%%%%%%%%%%%%%%%%%%%%%%%%%%%%%%%%%%%%%%%%%
\section{Usage}

First of all, the package \textsf{childdoc} is \emph{not} a standard
\LaTeXe{} |.sty| style file! Therefore it needs to be invoked in
a non-standard way.

%%%%%%%%%%%%%%%%%%%%%%%%%%%%%%%%%%%%%%%%%%%%%%%%%%%%%%%%%%%%%%%%%%%%%%%%%%%%%%%%
\subsection{Included Files}
\label{sec:include}

%%%%%%%%%%%%%%%%%%%%%%%%%%%%%%%%%%%%%%%%
\DescribeMacro{\childdocmain}
To use the package, add the commands
\begin{center}
\begin{tabular}{l}
|\input{childdoc.def}|\\
|\childdocmain{}|\\
\end{tabular}
\end{center}
at the very top of the main \LaTeX{} file,
in particular \emph{before} the |\documentclass| statement!
The argument of |\childdocmain| should be left empty
(but it must be present).

%%%%%%%%%%%%%%%%%%%%%%%%%%%%%%%%%%%%%%%%
\DescribeMacro{\childdocof}
Furthermore, add the commands
\begin{center}
\begin{tabular}{l}
|\input{childdoc.def}|\\
|\childdocof{|\textit{main}|}|\\
\end{tabular}
\end{center}
at the top of every child file \textit{child}
which is included by |\include{|\textit{child}|}|
from within the main file
(or at least for those files to be compiled individually).
The argument \textit{main} must be the filename of the main file.

There are a couple of
considerations in setting up the main and child documents:

%%%%%%%%%%%%%%%%%%%%%%%%%%%%%%%%%%%%%%%%
\paragraph{Restrictions.}

Please note the following restrictions:
\begin{itemize}
\item
|\childdocmain| must be called with one argument \textit{main}
to ensure compatibility with earlier version of the package.
It must either be empty (|\childdocmain{}|)
or precisely match the filename of the main file in which it is specified.
See \secref{sec:detection} for further information.
\item
The filename \textit{main} must be specified without the |.tex| extension.
\item
The filename \textit{main} is case sensitive
(even in case-insensitive file systems)
due to internal string comparison.
\item
The argument \textit{main} should be fully expanded, it cannot be a macro.
\item
Subdirectories and special characters should be avoided in filenames.
\item
The command |\childdocmain{|\textit{main}|}| must be followed by a whitespace.
It should not be followed immediately by another command
or by a comment mark `|%|'.
This is because the \TeX{} parser reads the token immediately following
the argument of |\childdocmain| and puts it
at the beginning of every child section;
however, a white\-space is ignored.
\end{itemize}

%%%%%%%%%%%%%%%%%%%%%%%%%%%%%%%%%%%%%%%%
\paragraph{Content of Main File.}

It is advisable to place all content in the child files included by |\include|.
Any output contained in the main file will appear in all child documents
unless suppressed manually;
it cannot be suppressed automatically by the |\includeonly| directive
and thus should normally be avoided.
A method to include some content in the main file
by means of conditional processing is described in \secref{sec:conditional}.

%%%%%%%%%%%%%%%%%%%%%%%%%%%%%%%%%%%%%%%%
\paragraph{Page Numbering.}

When only a part of the document is compiled,
the appropriate numbering of pages
(as well as other status parameters)
is determined from the |.aux| files.
The latter contain information from previous passes.
However this information needs to propagate through
all intermediate child documents.
Therefore the page numbering in child documents may well
be inconsistent until the complete document is compiled at least once.

A useful (if unconventional) way to always ensure a consistent
page numbering is to restart the numbering in each child document
and denote the pages by `\textit{child}|.|\textit{page}'
where \textit{child} represents the chapter/section number of the child file.
This can be achieved by the command
|\numberwithin{page}{|\textit{child}|}|
of the \textsf{amsmath} package
where \textit{child} can be |chapter| or |section|
depending on the chosen structuring.
Alternatively, one can modify the macro |\thepage| appropriately
and reset the counter |page| at the start of each child file.

%%%%%%%%%%%%%%%%%%%%%%%%%%%%%%%%%%%%%%%%%%%%%%%%%%%%%%%%%%%%%%%%%%%%%%%%%%%%%%%%
\subsection{Conditional Processing}
\label{sec:conditional}

The package provides a mechanism to compile different versions
of a document. To customise the versions further some conditional processing
can come in handy to distinguish which version is being compiled.
The package provides two macros to describe the compilation context:

%%%%%%%%%%%%%%%%%%%%%%%%%%%%%%%%%%%%%%%%
\DescribeMacro{\ifchilddoc}
The conditional |\ifchilddoc| distinguishes between the compilation of
child documents and the main document:
%
\begin{center}
|\ifchilddoc |\textit{child-code}| |[|\||else |\textit{main-code}]| \||fi|
\end{center}

%%%%%%%%%%%%%%%%%%%%%%%%%%%%%%%%%%%%%%%%
\DescribeMacro{\childdocname}
\DescribeMacro{\childdocjob}
The macro |\childdocname| contains the filename (without extension)
of the main or child file being processed.
Note that |\childdocjob| will always contain the name of the main file.

%%%%%%%%%%%%%%%%%%%%%%%%%%%%%%%%%%%%%%%%
\paragraph{Title Page.}

Conditional processing can be used to include a title or banner page
in the main document when proper precautions are taken.
Importantly, the code in the main file should ensure that the page counter
(as well as other status parameters which are stored in the |.aux| files)
takes the same value after the conditional processing.
Otherwise the page numbers may take divergent values
depending on which part is compiled.

For example, a title page could be declared by:
%
\begin{center}
\begin{tabular}{l}
|\ifchilddoc\||else|\\
|\addtocounter{page}{-1}|\\
\textit{code for title page}\\
|\newpage|\\
|\||fi|
\end{tabular}
\end{center}
%
A banner page for the child documents can be generated by:
%
\begin{center}
\begin{tabular}{l}
|\ifchilddoc|\\
|\addtocounter{page}{-1}|\\
\textit{code for banner page}\\
|\newpage|\\
|\||fi|
\end{tabular}
\end{center}
%
Here one could write a message such as:
\begin{center}
|This is the part \childdocname{} of \childdocjob{}.|
\end{center}

%%%%%%%%%%%%%%%%%%%%%%%%%%%%%%%%%%%%%%%%%%%%%%%%%%%%%%%%%%%%%%%%%%%%%%%%%%%%%%%%
\subsection{Flags}
\label{sec:flags}

The package makes it easy to generate different versions
of the main or child documents.
To this end compilation flags can be defined
and assigned different default values.
They will be particularly useful in conjunction
with the forwarding mechanism described in \secref{sec:forward}.

For example, it may be useful to have a flag |\version|
which can be set to |draft| or |final|.
The document source will contain some conditional code
depending on the value of |\version|.
Suppose further, the flag should default to |final| for the main file
and to |draft| for child files
which is a natural assignment for editing the document.
This is achieved by placing the following code
in the preamble of the main document
(below the |\childdocmain| directive):
%
\begin{center}
\begin{tabular}{l}
|\ifchilddoc|\\
|\providecommand{\version}{draft}|\\
|\||else|\\
|\providecommand{\version}{final}|\\
|\||fi|
\end{tabular}
\end{center}
%
The definition by |\providecommand| makes sure
that previous definitions are not overwritten.
Further statements |\providecommand{\version}{...}|
can thus be added before the above code to override it.

For the main file, one might add a line
(between |\childdocmain| and the above block)
%
\begin{center}
|%\ifchilddoc\||else\providecommand{\version}{draft}\||fi|
\end{center}
%
which can be uncommented to produce a draft version.
Likewise one can add a line to the very top of a child file
(above the |\childdocof{|\textit{main}|}| directive)
%
\begin{center}
|%\providecommand{\version}{final}|
\end{center}
%
which can be uncommented to produce the final version of this child document.

%%%%%%%%%%%%%%%%%%%%%%%%%%%%%%%%%%%%%%%%%%%%%%%%%%%%%%%%%%%%%%%%%%%%%%%%%%%%%%%%
\subsection{Forwarding}
\label{sec:forward}

Different versions of the main or child documents
using compilation flags as described in \secref{sec:flags}
can be (permanently) stored in different files
for convenient compilation, viewing and distribution.
To this end, the package defines a command
to pass on compilation to a different file:

%%%%%%%%%%%%%%%%%%%%%%%%%%%%%%%%%%%%%%%%
\DescribeMacro{\childdocforward}
The command |\childdocforward| redirects processing to
another source file:
%
\begin{center}
\begin{tabular}{l}
|\input{childdoc.def}|\\
|\childdocforward[|\textit{main}|]{|\textit{dest}|}|\\
\end{tabular}
\end{center}
%
The argument \textit{dest} is the destination file
(without extension).
It should be the main file or one of the child files.
Note that further \textsf{childdoc} directives
such as |\childdocof| and |\childdocforward|
in the indicated file will be processed in this form.
The optional argument \textit{main}
passes on directly to the main file \textit{main}
while pretending to compile the child \textit{dest}.
This form behaves as if \textit{dest}
issues |\childdocof{|\textit{main}|}| right away,
and no further \textsf{childdoc} directives will be processed.

%%%%%%%%%%%%%%%%%%%%%%%%%%%%%%%%%%%%%%%%
\DescribeMacro{\...prefix}
In the alternative form |\childdocforwardprefix|,
%
\begin{center}
\begin{tabular}{l}
|\input{childdoc.def}|\\
|\childdocforwardprefix[|\textit{main}|]{|\textit{prefix}|}{|\textit{dest}|}|
\end{tabular}
\end{center}
%
the destination file is determined by a pattern
depending on the current file:
To make this work, the current file must be called
`{\textit{prefix}\hspace{0.2em}\textit{suffix}}'
with \textit{prefix} matching precisely the argument.
Processing is then passed on to the file
`{\textit{dest}\hspace{0.2em}\textit{suffix}}'.
Surely, the same effect is achieved by
directly specifying the
argument `{\textit{dest}\hspace{0.2em}\textit{suffix}}'
in the first form.
However, that requires to set up a different file
for each child. With the alternative form of the command
all these files can have exactly the same content
which simplifies setting them up and maintaining them.

For example, the following file |draft.tex|
with a compilation flag |\version| as described in \secref{sec:flags}
compiles the main document as a draft:
%
\begin{center}
\begin{tabular}{l}
|\def\version{draft}|\\
|\input{childdoc.def}|\\
|\childdocforward{|\textit{main}|}|
\end{tabular}
\end{center}
%
Likewise, the following files |final|\textit{nn}|.tex|
compile the final version of the child document
|child|\textit{nn}|.tex|:
%
\begin{center}
\begin{tabular}{l}
|\def\version{final}|\\
|\input{childdoc.def}|\\
|\childdocforwardprefix{final}{child}|
\end{tabular}
\end{center}
%

Note that when several versions of a main file and/or of each child file
are to be generated, it may be convenient to set up a |Makefile| or
shell script to automatise the process.

%%%%%%%%%%%%%%%%%%%%%%%%%%%%%%%%%%%%%%%%%%%%%%%%%%%%%%%%%%%%%%%%%%%%%%%%%%%%%%%%
\subsection{Command Line Processing}
\label{sec:commandline}

The effect of redirection files can also be achieved by invoking
the \LaTeX{} compiler with a more elaborate command line.
Most conveniently this should be done as part
of a shell script or a |Makefile|.

When using \textsf{childdoc} in the main file, the following
command lines effectively perform a redirection
(note that depending on the shell being used,
backslashes may have to be doubled: `|\|' $\to$ `|\\|'):
%
\begin{center}
|... -jobname "|\textit{target}|" |\\|"|[\textit{flags}]%
|\input{childdoc.def}\childdocforward[|\textit{main}|]{|\textit{dest}|}"|
\end{center}
%
Here \textit{target} is the name of the output file,
\textit{main} is the name of the main file
and \textit{dest} is the name of the main or child file to be processed
(all filenames without extensions).
The optional argument \textit{main} can be omitted
if \textit{main} matches \textit{dest}.
Optionally, compilation \textit{flags} can be defined via |\def| commands.
This command line makes the \TeX{} engine believe
it is compiling the file \textit{target}
whose content is specified as the latter parameter.
The provided code then forwards the processing to
\textit{main} or \textit{dest} as described in \secref{sec:forward}.

%%%%%%%%%%%%%%%%%%%%%%%%%%%%%%%%%%%%%%%%%%%%%%%%%%%%%%%%%%%%%%%%%%%%%%%%%%%%%%%%
\subsection{Include by Input}
\label{sec:input}

Including child documents by |\include| has some restrictions by design.
Most notably, the content of a child document always occupies
its own set of pages; pages cannot be shared between child documents.
Usually, this behaviour makes perfect sense
because each child document contain an essential part of the document.
However, in some situations it may be desirable to compose
a document from a collection of parts
without having mandatory page breaks between then.
For this case, the package
provides a mechanism to include parts
by |\input| which can also be processed individually.
However, by construction this mechanism
requires manual handling of the content to be output.

%%%%%%%%%%%%%%%%%%%%%%%%%%%%%%%%%%%%%%%%
\DescribeMacro{\ifchilddocmanual}
The main file should be prepared as usual, see \secref{sec:include}.
However, the document body must make a distinction
between processing of an individual part and of the main document, e.g.:
%
\begin{center}
\begin{tabular}{l}
|\ifchilddocmanual|\\
|\input{\childdocname}|\\
|\||else|\\
\textit{document body with }|\input{|\textit{part}|}|\\
|\||fi|
\end{tabular}
\end{center}
%
The conditional |\ifchilddocmanual| is true whenever
a part to be included by |\input| is being compiled,
and the name of the part is stored in |\childdocname|.

%%%%%%%%%%%%%%%%%%%%%%%%%%%%%%%%%%%%%%%%
\DescribeMacro{\childdocby}
Each part to be included by |\input| should start with:
%
\begin{center}
\begin{tabular}{l}
|\input{childdoc.def}|\\
|\childdocby{|\textit{main}|}|\\
\end{tabular}
\end{center}
%
The directive |\childdocby| is similar to |\childdocof|
described in \secref{sec:include},
but the subsequent selection of content must be done manually.
To that end, both |\ifchilddoc| and |\ifchilddocmanual|
will be true upon processing of a part,
and the name of the part is stored in |\childdocname|.
Note that |\jobname| will be set to the filename of the current part
so that each part receives an individual |.aux| file
that does not interfere with the |.aux| file(s) of the main document.
This behaviour can be altered by the alternative form
|\childdocby[*]{|\textit{main}|}| (with a non-empty optional argument)
which uses the |.aux| file of the main document
by setting |\jobname| to \textit{main}.

%%%%%%%%%%%%%%%%%%%%%%%%%%%%%%%%%%%%%%%%%%%%%%%%%%%%%%%%%%%%%%%%%%%%%%%%%%%%%%%%
\subsection{Driver Development}
\label{sec:driver}

The \textsf{childdoc} mechanism can also be use for the development
of definition files such as \LaTeX{} styles or classes.
This case differs from the above setup with multiple parts
included by |\include| in that no |\includeonly| should be invoked.
This can be achieved by starting the include file
(before |\ProvidesPackage|) with:
%
\begin{center}
\begin{tabular}{l}
|\input{childdoc.def}|\\
|\childdocforward{|\textit{main}|}|\\
\end{tabular}
\end{center}
%
or alternatively with:
%
\begin{center}
\begin{tabular}{l}
|\input{childdoc.def}|\\
|\childdocby{|\textit{main}|}|\\
\end{tabular}
\end{center}
%
Both forms have slightly different effects as described above.
The main file is prepared as usual, see \secref{sec:include}.

%%%%%%%%%%%%%%%%%%%%%%%%%%%%%%%%%%%%%%%%%%%%%%%%%%%%%%%%%%%%%%%%%%%%%%%%%%%%%%%%
\subsection{Legacy Detection}
\label{sec:detection}

The directive |\childdocmain| in the main file can detect
whether the complete document or merely a child is to be compiled
even without using the directive |\childdocof|.
This method is deprecated because it is less robust
and there is no compelling reason to use it;
it is merely provided for backward compatibility
and it may be removed in future versions.

If the detection mechanism is to be used,
it is mandatory to correctly specify
the filename of the main file as the argument of |\childdocmain|:
%
\begin{center}
\begin{tabular}{l}
|\input{childdoc.def}|\\
|\childdocmain{|\textit{main}|}|\\
\end{tabular}
\end{center}
%
If |\jobname| does not match the argument \textit{main} of |\childdocmain|,
it is assumed that |\jobname| points to the child file to be compiled.
When using |\childdocmain| with the main file specified as argument,
it suffices to start a child file
with just |\input{|\textit{main}|}|
without loading of the package and using |\childdocof|.
If instead all processing is done
with the appropriate \textsf{childdoc} directives,
the argument of \textit{main} of |\childdocmain| can be empty.

An alternative version of the command line processing described
in \secref{sec:commandline} using the detection mechanism reads:
%
\begin{center}
|... -jobname "|\textit{target}|" "|[\textit{flags}]%
[|\def\jobname{|\textit{dest}|}|]|\input{|\textit{main}|}"|
\end{center}

%%%%%%%%%%%%%%%%%%%%%%%%%%%%%%%%%%%%%%%%%%%%%%%%%%%%%%%%%%%%%%%%%%%%%%%%%%%%%%%%
\subsection{Manual Code}
\label{sec:manual}

In case one cannot be certain whether the definitions file |childdoc.def|
is installed on the target \TeX{} distribution
and one prefers not to ship it,
it is conceivable to paste a few relevant commands into the sources.

To that end, drop all statements |\input{childdoc.def}|
and perform the replacements as outlined below.
Instead of |\childdocmain{|\textit{main}|}| add the following code
to the top of the main file:
%
\begin{center}
\begin{tabular}{l}
|\||ifdefined\childdocname\endinput\||fi\newif\ifchilddoc|\\
|\edef\childdocname{\scantokens\expandafter{\jobname\noexpand}}|\\
|\def\childdocmain{|\textit{main}|}\||ifx\childdocmain\childdocname\||else|\\
|\childdoctrue\includeonly{\childdocname}\let\jobname\childdocmain\||fi|\\
\end{tabular}
\end{center}
%
Instead of |\childdocof{|\textit{main}|}| just include the main file
at the top of each child file:
%
\begin{center}
|\input{|\textit{main}|}|
\end{center}
%
A simple redirection |\childdocforward{|\textit{dest}|}| is achieved by:
%
\begin{center}
|\def\jobname{|\textit{dest}|}\input{\jobname}|
\end{center}
%
The redirection with prefix
|\childdocforwardprefix[|\textit{prefix}|]{|\textit{dest}|}|
is accomplished by:
%
\begin{center}
\begin{tabular}{l}
|{\edef\jobname{\scantokens\expandafter{\jobname\noexpand}}|\\
|\def\redirectjob |\textit{prefix}|#1~~~{\gdef\jobname{|\textit{dest}|#1}}|\\
|\expandafter\redirectjob\jobname~~~}\input{\jobname}|
\end{tabular}
\end{center}

In an alternative approach,
child documents can be compiled by a specific command line
without additional code or specific definitions:
%
\begin{center}
|... -jobname "|\textit{target}|" "|[\textit{flags}]%
|\includeonly{|\textit{dest}|}\input{|\textit{main}|}"|
\end{center}
%

%%%%%%%%%%%%%%%%%%%%%%%%%%%%%%%%%%%%%%%%%%%%%%%%%%%%%%%%%%%%%%%%%%%%%%%%%%%%%%%%
%%%%%%%%%%%%%%%%%%%%%%%%%%%%%%%%%%%%%%%%%%%%%%%%%%%%%%%%%%%%%%%%%%%%%%%%%%%%%%%%
\section{Information}

%%%%%%%%%%%%%%%%%%%%%%%%%%%%%%%%%%%%%%%%%%%%%%%%%%%%%%%%%%%%%%%%%%%%%%%%%%%%%%%%
\subsection{Copyright}

Copyright \copyright{} 2017--2018 Niklas Beisert

This work may be distributed and/or modified under the
conditions of the \LaTeX{} Project Public License, either version 1.3
of this license or (at your option) any later version.
The latest version of this license is in
  \url{http://www.latex-project.org/lppl.txt}
and version 1.3 or later is part of all distributions of \LaTeX{}
version 2005/12/01 or later.

This work has the LPPL maintenance status `maintained'.

The Current Maintainer of this work is Niklas Beisert.

This work consists of the files |README.txt|, |childdoc.ins| and |childdoc.dtx|
as well as the derived files |childdoc.def|, |cdocsamp.tex|
with |cdocsch1.tex|, |cdocsch2.tex|, |cdocspt3.tex|, |cdocspt4.tex|,
|cdocsdrf.tex|, |cdocsfn1.tex|, |cdocsfn2.tex|
as well as |childdoc.pdf|.

%%%%%%%%%%%%%%%%%%%%%%%%%%%%%%%%%%%%%%%%%%%%%%%%%%%%%%%%%%%%%%%%%%%%%%%%%%%%%%%%
\subsection{Files and Installation}

The package consists of the files:
%
\begin{center}
\begin{tabular}{ll}
    |README.txt|   & readme file \\
    |childdoc.ins| & installation file \\
    |childdoc.dtx| & source file \\
    |childdoc.def| & definition file \\
    |cdocsamp.tex| & sample main file \\
    |cdocsch1.tex| & sample include file \\
    |cdocsch2.tex| & sample include file \\
    |cdocspt3.tex| & sample part file \\
    |cdocspt4.tex| & sample part file \\
    |cdocsdrf.tex| & sample redirection file \\
    |cdocsfn1.tex| & sample redirection file \\
    |cdocsfn2.tex| & sample redirection file \\
    |childdoc.pdf| & manual
\end{tabular}
\end{center}
%
The distribution consists of the files
|README.txt|, |childdoc.ins| and |childdoc.dtx|.
%
\begin{itemize}
\item
Run (pdf)\LaTeX{} on |childdoc.dtx|
to compile the manual |childdoc.pdf| (this file).
\item
Run \LaTeX{} on |childdoc.ins| to create the definitions file |childdoc.def|
and the sample |cdocsamp.tex| with include files
|cdocsch1.tex|, |cdocsch2.tex|, |cdocspt3.tex|, |cdocspt4.tex|,
|cdocsdrf.tex|, |cdocsfn1.tex|, |cdocsfn2.tex|.
Then copy the file |childdoc.def| to an appropriate directory of your \LaTeX{}
distribution, e.g.\ \textit{texmf-root}|/tex/latex/childdoc|.
\end{itemize}

%%%%%%%%%%%%%%%%%%%%%%%%%%%%%%%%%%%%%%%%%%%%%%%%%%%%%%%%%%%%%%%%%%%%%%%%%%%%%%%%
\subsection{Related CTAN Packages}

There are several other packages which offer a similar functionality:
%
\begin{itemize}
\item
The packages
\href{http://ctan.org/pkg/docmute}{\textsf{docmute}},
\href{http://ctan.org/pkg/includex}{\textsf{includex}} and
\href{http://ctan.org/pkg/standalone}{\textsf{standalone}}
provide commands to include only the document body of
a child file thus allowing both files to be compiled individually.
\item
The packages \href{http://ctan.org/pkg/subdocs}{\textsf{subdocs}}
and \href{http://ctan.org/pkg/subfiles}{\textsf{subfiles}}
provide structures in which the main and child documents can be
encapsulated and allowing them to be compiled individually.
The inclusion mechanism is different from the conventional |\include|.
\item
The package \href{http://ctan.org/pkg/combine}{\textsf{combine}}
is an elaborate solution to combine several documents into one.
\end{itemize}
%
See also the CTAN topic \href{http://ctan.org/topic/subdocs}{\textsf{subdocs}}
for further related packages.
The present package differs from the above solutions in that
a document structure constructed with the conventional |\include| mechanism
just needs two extra commands at the top of every file
such that all constituent files can be compiled individually.

%%%%%%%%%%%%%%%%%%%%%%%%%%%%%%%%%%%%%%%%%%%%%%%%%%%%%%%%%%%%%%%%%%%%%%%%%%%%%%%%
%\subsection{Feature Suggestions}
%
%The following is a list of features which may be useful for future
%versions of this package:
%%
%\begin{itemize}
%\item
%\ldots
%\end{itemize}

%%%%%%%%%%%%%%%%%%%%%%%%%%%%%%%%%%%%%%%%%%%%%%%%%%%%%%%%%%%%%%%%%%%%%%%%%%%%%%%%
\subsection{Revision History}

%%%%%%%%%%%%%%%%%%%%%%%%%%%%%%%%%%%%%%%%
\paragraph{v2.0:} 2018/12/30

\begin{itemize}
\item
immediate forward processing
\item
added |\childdocby| mechanism
\item
manual restructured
\end{itemize}

%%%%%%%%%%%%%%%%%%%%%%%%%%%%%%%%%%%%%%%%
\paragraph{v1.6:} 2018/01/17

\begin{itemize}
\item
application for development of include files
\item
corrections to manual
\end{itemize}

%%%%%%%%%%%%%%%%%%%%%%%%%%%%%%%%%%%%%%%%
\paragraph{v1.5:} 2017/05/21

\begin{itemize}
\item
more complete structuring introduced
\item
|\childdocof| introduced
\item
|\childdoc| renamed to |\childdocmain|
\item
|\childredirect| renamed to |\childdocforward| and |\childdocforwardprefix|
and functionality expanded
\end{itemize}

%%%%%%%%%%%%%%%%%%%%%%%%%%%%%%%%%%%%%%%%
\paragraph{v1.0:} 2017/04/27

\begin{itemize}
\item
manual and install package
\item
first version published on CTAN
\end{itemize}

%%%%%%%%%%%%%%%%%%%%%%%%%%%%%%%%%%%%%%%%
\paragraph{v0.6:} 2017/04/26

\begin{itemize}
\item
redirection mechanism added
\end{itemize}

%%%%%%%%%%%%%%%%%%%%%%%%%%%%%%%%%%%%%%%%
\paragraph{v0.5:} 2017/04/26

\begin{itemize}
\item
functionality in definition file
\end{itemize}


%%%%%%%%%%%%%%%%%%%%%%%%%%%%%%%%%%%%%%%%%%%%%%%%%%%%%%%%%%%%%%%%%%%%%%%%%%%%%%%%
%%%%%%%%%%%%%%%%%%%%%%%%%%%%%%%%%%%%%%%%%%%%%%%%%%%%%%%%%%%%%%%%%%%%%%%%%%%%%%%%
%%%%%%%%%%%%%%%%%%%%%%%%%%%%%%%%%%%%%%%%%%%%%%%%%%%%%%%%%%%%%%%%%%%%%%%%%%%%%%%%
\appendix

\settowidth\MacroIndent{\rmfamily\scriptsize 000\ }

 \DocInput{childdoc.dtx}

\end{document}
%</driver>
% \fi
%
% %%%%%%%%%%%%%%%%%%%%%%%%%%%%%%%%%%%%%%%%%%%%%%%%%%%%%%%%%%%%%%%%%%%%%%%%%%%%%%
% %%%%%%%%%%%%%%%%%%%%%%%%%%%%%%%%%%%%%%%%%%%%%%%%%%%%%%%%%%%%%%%%%%%%%%%%%%%%%%
% \section{Sample}
%\iffalse
%<*samplemain>
%\fi
%
% The following presents a sample document
% with two chapters, two parts, a title page,
% a compile flag as well as three forwarding files to set the flag.
% It consists of eight |.tex| files:
% \begin{center}
% \begin{tabular}{ll}
% |cdocsamp.tex|&main file\\
% |cdocsch1.tex|&include file for chapter 1\\
% |cdocsch2.tex|&include file for chapter 2\\
% |cdocspt3.tex|&include file for part 3\\
% |cdocspt4.tex|&include file for part 4\\
% |cdocsdrf.tex|&forwarding file for main file in draft mode\\
% |cdocsfi1.tex|&forwarding file for final version of chapter 1\\
% |cdocsfi2.tex|&forwarding file for final version of chapter 2\\
% \end{tabular}
% \end{center}
% Each of the eight files can be compiled directly by the \LaTeX{} compiler.
%
% %%%%%%%%%%%%%%%%%%%%%%%%%%%%%%%%%%%%%%
% \paragraph{Main File.}
%
% The main file is called |cdocsamp.tex|.
%
% Load the \textsf{childdoc} definitions and
% declare the filename for the main document:
%    \begin{macrocode}
\input{childdoc.def}
\childdocmain{}
%    \end{macrocode}

% Optional override for |\version| flag:
%    \begin{macrocode}
%%\ifchilddoc\else\providecommand{\version}{draft}\fi
%    \end{macrocode}

% Define the default values for the |\version| flag
% (|final| for the main file and |draft| for childs):
%    \begin{macrocode}
\ifchilddoc
\providecommand{\version}{draft}
\else
\providecommand{\version}{final}
\fi
%    \end{macrocode}

% Load the standard document class:
%    \begin{macrocode}
\documentclass[12pt]{article}
%    \end{macrocode}

% Start the document body:
%    \begin{macrocode}
\begin{document}
%    \end{macrocode}

% Declare a title page.
% Print title, part of document being processed and version flag:
%    \begin{macrocode}
\addtocounter{page}{-1}
\begin{center}
{\LARGE\bfseries{}childdoc example\par}
\vspace{1cm}
\ifchilddoc
\ifchilddocmanual part\else chapter\fi:
`\childdocname' of `\childdocjob'\par
\else
main document: `\childdocjob'\par
\fi
version: \version\par
\end{center}
\newpage
%    \end{macrocode}

% Manually include selected file,
% otherwise process as usual:
%    \begin{macrocode}
\ifchilddocmanual
\section*{part `\childdocname'}
\input{\childdocname}
\else
%    \end{macrocode}

% Include the two chapters:
%    \begin{macrocode}
\include{cdocsch1}
\include{cdocsch2}
%    \end{macrocode}

% Include the two parts unless only chapters should be displayed:
%    \begin{macrocode}
\ifchilddoc\else
\section{part three}
\input{cdocspt3}
\section{part four}
\input{cdocspt4}
\fi
%    \end{macrocode}

% Process as usual until here:
%    \begin{macrocode}
\fi
%    \end{macrocode}

% End of document body:
%    \begin{macrocode}
\end{document}
%    \end{macrocode}
%\iffalse
%</samplemain>
%\fi
%
% %%%%%%%%%%%%%%%%%%%%%%%%%%%%%%%%%%%%%%
% \paragraph{Chapter Include Files.}
%
% The include files are called |cdocsch1.tex| and |cdocsch2.tex|.
%
%\iffalse
%<*samplechap1|samplechap2>
%\fi

% Optional override for |\version| flag:
%    \begin{macrocode}
%%\providecommand{\version}{final}
%    \end{macrocode}

% Include the main document:
%    \begin{macrocode}
\input{childdoc.def}
\childdocof{cdocsamp}
%    \end{macrocode}

%\iffalse
%</samplechap1|samplechap2>
%\fi
%
%\iffalse
%<*samplechap1>
%\fi
% Some text for chapter 1:
%    \begin{macrocode}
\section{one}
some text in chapter one
%    \end{macrocode}

%\iffalse
%</samplechap1>
%\fi
% Some text for chapter 2:
%\iffalse
%<*samplechap2>
%\fi
%    \begin{macrocode}
\section{two}
more text in chapter two
%    \end{macrocode}

%\iffalse
%</samplechap2>
%\fi
%
% %%%%%%%%%%%%%%%%%%%%%%%%%%%%%%%%%%%%%%
% \paragraph{Part Include Files.}
%
% The include files are called |cdocspt3.tex| and |cdocspt4.tex|.
%
%\iffalse
%<*samplepart3|samplepart4>
%\fi

% Optional override for |\version| flag:
%    \begin{macrocode}
%%\providecommand{\version}{final}
%    \end{macrocode}

% Include the main document:
%    \begin{macrocode}
\input{childdoc.def}
\childdocby{cdocsamp}
%    \end{macrocode}

%\iffalse
%</samplepart3|samplepart4>
%\fi
%
%\iffalse
%<*samplepart3>
%\fi
% Some text for part 3:
%    \begin{macrocode}
some text in part three
%    \end{macrocode}

%\iffalse
%</samplepart3>
%\fi
% Some text for part 4:
%\iffalse
%<*samplepart4>
%\fi
%    \begin{macrocode}
more text in part four
%    \end{macrocode}

%\iffalse
%</samplepart4>
%\fi
%
% %%%%%%%%%%%%%%%%%%%%%%%%%%%%%%%%%%%%%%
% \paragraph{Forwarding for a Complete Draft.}
%
% The following forwarding file |cdocsdrf.tex|
% compiles the main document in draft mode:
%\iffalse
%<*sampledraft>
%\fi
%    \begin{macrocode}
\def\version{draft}
\input{childdoc.def}
\childdocforward{cdocsamp}
%    \end{macrocode}

%\iffalse
%</sampledraft>
%\fi
%
% %%%%%%%%%%%%%%%%%%%%%%%%%%%%%%%%%%%%%%
% \paragraph{Forwarding for Final Version of the Chapters.}
%
% The following forwarding files |cdocsfn1.tex| and |cdocsfn2.tex|
% (with identical content)
% compile the final versions of the child documents
% |cdocsch1.tex| and |cdocsch2.tex|, respectively:
%\iffalse
%<*samplefinal>
%\fi
%    \begin{macrocode}
\def\version{final}
\input{childdoc.def}
\childdocforwardprefix[cdocsamp]{cdocsfn}{cdocsch}
%    \end{macrocode}

%\iffalse
%</samplefinal>
%\fi
%
% %%%%%%%%%%%%%%%%%%%%%%%%%%%%%%%%%%%%%%
% \paragraph{Command Line Processing.}
%
% The following three command lines generate the output files
% |cdocscld|, |cdocscl1| and |cdocscl2|
% which should be identical to
% |cdocsdrf|, |cdocsch1| and |cdocsfn2|, respectively:
% \begin{center}
% \begin{tabular}{l}
% |latex -jobname cdocscld \|\\
% |  "\def\version{draft}\input{childdoc.def}\childdocforward{cdocsamp}"|\\
% |latex -jobname cdocscl1 \|\\
% |  "\input{childdoc.def}\childdocforward[cdocsamp]{cdocsch1}"|\\
% |latex -jobname cdocscl2 \|\\
% |  "\def\version{final}\input{childdoc.def}\childdocforward{cdocsch2}"|
% \end{tabular}
% \end{center}
% Note that the trailing backslash on each first line
% merely continues the input to the second line
% (for convenient cut ant paste).
% Furthermore, the command |latex| can be replaced by any
% of its alternative versions such as |pdflatex|.
%
% %%%%%%%%%%%%%%%%%%%%%%%%%%%%%%%%%%%%%%%%%%%%%%%%%%%%%%%%%%%%%%%%%%%%%%%%%%%%%%
% %%%%%%%%%%%%%%%%%%%%%%%%%%%%%%%%%%%%%%%%%%%%%%%%%%%%%%%%%%%%%%%%%%%%%%%%%%%%%%
% \section{Implementation}
%\iffalse
%<*package>
%\fi
%
% This section describes the definitions file |childdoc.def|.

% The definitions cannot be loaded using |\usepackage| or |\RequirePackage|
% which has a mechanism to prevent loading a style file more than once.
% When loading the definitions by means of |\input|
% multiple instances have to be prevented manually:
%\iffalse
%This code needs to be before the `\ProvidesFile' directive
%which is defined at the beginning of this file.
%Therefore it is also placed there and commented out here.
%</package>
%<*discard>
%\fi
%    \begin{macrocode}
\ifdefined\childdocmain\endinput\fi
%    \end{macrocode}
%\iffalse
%</discard>
%<*package>
%\fi
%
% \macro{\ifchilddoc}
% \macro{\ifchilddocmanual}
% The conditional |\ifchilddoc| tells whether a
% child (true) or main (false) document is being compiled.
% The conditional |\ifchilddocmanual| tells whether
% the |\includeonly| mechanism is used (false) or
% the selection of child files must be performed manually (true).
% The definitions initialise to false:
%    \begin{macrocode}
\newif\ifchilddoc
\newif\ifchilddocmanual
%    \end{macrocode}

% \macro{\childdocname}
% \macro{\childdocjob}
% The macro |\childdocname| stores the name of the main document
% to be compiled. The macro |\childdocjob| stores the name of
% the document on which the \LaTeX{} compiler was originally invoked.
% The content of |\jobname| cannot be compared
% to filenames specified in the source due to different catcodes.
% The following code rescans |\jobname|, stores the result
% in |\childdocname| and saves a copy in |\childdocjob|:
%    \begin{macrocode}
\edef\childdocname{\scantokens\expandafter{\jobname\noexpand}}
\let\childdocjob\childdocname
%    \end{macrocode}

% \macro{\childdocdisable}
% The macro |\childdocdisable| prevents the main file
% from being processed more than once.
% At this stage, the main document command |\childdocmain|
% is assumed to be called once again where it should do nothing.
% Any subsequent call to it should prevent
% a secondary processing of the main document
% It overwrites the forwarding commands
% |\childdocof| and |\childdocforward|
% with empty macros to prevent further inclusions of the main document:
%    \begin{macrocode}
\newcommand{\childdocdisable}
{
  \renewcommand{\childdocmain}[1]{\renewcommand{\childdocmain}[1]{\endinput}}
  \renewcommand{\childdocof}[1]{}
  \renewcommand{\childdocby}[2][]{}
  \renewcommand{\childdocforward}[2][]{}
  \renewcommand{\childdocdisable}{}
}
%    \end{macrocode}

% \macro{\childdocmain}
% The macro |\childdocmain| is to be called at the top of the main file
% with nothing or the main filename (without extension) as argument.
% First, it breaks loops.
% If the argument is not empty and does not match |\childdocname|
% (which is set by the first inclusion of |childdoc.def|),
% |\ifchilddoc| is set to true, |\includeonly| is applied to the child file
% and |\jobname| is set to the main file
% (for proper handling of |.aux| files):
%    \begin{macrocode}
\newcommand{\childdocmain}[1]
{
  \childdocdisable\childdocmain{}
  \if?#1?\else
    \begingroup
      \def\childdoctmp{#1}
      \ifx\childdoctmp\childdocname
        \def\childdoctmp{}
      \else
        \def\childdoctmp
        {
          \childdoctrue
          \includeonly{\childdocname}
          \def\childdocjob{#1}
          \def\jobname{#1}
        }
      \fi
      \expandafter
    \endgroup
    \childdoctmp
  \fi
}
%    \end{macrocode}

% \macro{\childdocof}
% The command |\childdocof| redirects
% compilation to the main file |#1|.
%    \begin{macrocode}
\newcommand{\childdocof}[1]
{
  \childdocdisable
  \childdoctrue
  \includeonly{\childdocname}
  \def\jobname{#1}
  \def\childdocjob{#1}
  \input{#1}
}
%    \end{macrocode}

% \macro{\childdocby}
% The command |\childdocby| ....
%    \begin{macrocode}
\newcommand{\childdocby}[2][]
{
  \childdocdisable
  \childdoctrue
  \childdocmanualtrue
  \if?#1?\else
    \def\jobname{#2}
  \fi
  \def\childdocjob{#2}
  \input{#2}
  \endinput
}
%    \end{macrocode}

% \macro{\childdocforward}
% The command |\childdocforward| redirects
% compilation to the main file or
% (if the optional argument is given) a child file.
% Parameters are set as if the main file
% or a child file starting with |\childdocof| was compiled.
% Then compilation is handed over to the main file:
%    \begin{macrocode}
\newcommand{\childdocforward}[2][]
{
  \begingroup
    \if?#1?
      \def\childdoctmp
      {
        \def\childdocname{#2}
        \def\childdocjob{#2}
        \def\jobname{#2}
        \input{#2}
        \endinput
      }
    \else
      \def\childdoctmp
      {
        \childdocdisable
        \def\childdocname{#2}
        \childdoctrue
        \includeonly{#2}
        \def\childdocjob{#1}
        \def\jobname{#1}
        \input{#1}
        \endinput
      }
    \fi
    \expandafter
  \endgroup
  \childdoctmp
}
%    \end{macrocode}

% \macro{\childdocforwardprefix}
% The command |\childdocforwardprefix| redirects
% compilation to the main or a child file by means of a pattern.
% The prefix |#1| in the current filename is replaced by |#2|
% and the suffix of the current filename is kept
% (it is assumed that the filename does not contain the substring `|~~~|'
% which is used as a delimiter).
% Compilation is handed over to the new file by |\childdocforward|:
%    \begin{macrocode}
\newcommand{\childdocforwardprefix}[3][]
{
  \begingroup
    \def\childdocextract #2##1~~~{\def\childdoctmp{\childdocforward[#1]{#3##1}}}
    \expandafter\childdocextract\childdocname~~~
    \expandafter
  \endgroup
  \childdoctmp
}
%    \end{macrocode}

% \macro{\childdoc}
% The deprecated macro |\childdoc| is a legacy version of |\childdocmain|:
%    \begin{macrocode}
\newcommand{\childdoc}{\childdocmain}
%    \end{macrocode}

% \macro{\childdocredirect}
% The deprecated macro |\childdocredirect| is a legacy version
% of |\childdocforward| and |\childdocforwardprefix|:
%    \begin{macrocode}
\newcommand{\childdocredirect}[2][]
{
  \begingroup
    \if?#1?
      \def\childdoctmp{\childdocforward{#2}}
    \else
      \def\childdoctmp{\childdocforwardprefix{#1}{#2}}
    \fi
    \expandafter
  \endgroup
  \childdoctmp
}
%    \end{macrocode}

%\iffalse
%</package>
%\fi
%
\endinput

\childdocforwardprefix[cdocsamp]{cdocsfn}{cdocsch}
%    \end{macrocode}

%\iffalse
%</samplefinal>
%\fi
%
% %%%%%%%%%%%%%%%%%%%%%%%%%%%%%%%%%%%%%%
% \paragraph{Command Line Processing.}
%
% The following three command lines generate the output files
% |cdocscld|, |cdocscl1| and |cdocscl2|
% which should be identical to
% |cdocsdrf|, |cdocsch1| and |cdocsfn2|, respectively:
% \begin{center}
% \begin{tabular}{l}
% |latex -jobname cdocscld \|\\
% |  "\def\version{draft}% \iffalse
%
% childdoc.dtx Copyright (C) 2017-2018 Niklas Beisert
%
% This work may be distributed and/or modified under the
% conditions of the LaTeX Project Public License, either version 1.3
% of this license or (at your option) any later version.
% The latest version of this license is in
%   http://www.latex-project.org/lppl.txt
% and version 1.3 or later is part of all distributions of LaTeX
% version 2005/12/01 or later.
%
% This work has the LPPL maintenance status `maintained'.
%
% The Current Maintainer of this work is Niklas Beisert.
%
% This work consists of the files childdoc.dtx and childdoc.ins
% and the derived files childdoc.def and cdocsamp.tex with
% cdocsch1.tex, cdocsch2.tex, cdocsdrf.tex, cdocsfn1.tex, cdocsfn2.tex.
%
%<package>\ifdefined\childdocmain\endinput\fi
%<package>\ProvidesFile{childdoc.def}[2018/12/30 v2.0 child document driver]
%<samplemain>\ProvidesFile{cdocsamp.tex}[2018/12/30 v2.0 sample for childdoc]
%<*driver>
%\ProvidesFile{childdoc.drv}[2018/12/30 v2.0 childdoc reference manual file]
\PassOptionsToClass{10pt,a4paper}{article}
\documentclass{ltxdoc}

\usepackage[margin=35mm]{geometry}
\usepackage{hyperref}
\usepackage{hyperxmp}
\usepackage[usenames]{color}

\hypersetup{colorlinks=true}
\hypersetup{pdfstartview=FitH}
\hypersetup{pdfpagemode=UseNone}
\hypersetup{pdfsource={}}
\hypersetup{pdflang={en-UK}}
\hypersetup{pdfcopyright={Copyright 2017-2018 Niklas Beisert.
  This work may be distributed and/or modified under the
  conditions of the LaTeX Project Public License, either version 1.3
  of this license or (at your option) any later version.}}
\hypersetup{pdflicenseurl={http://www.latex-project.org/lppl.txt}}
\hypersetup{pdfcontactaddress={ETH Zurich, ITP, HIT K,
  Wolfgang-Pauli-Strasse 27}}
\hypersetup{pdfcontactpostcode={8093}}
\hypersetup{pdfcontactcity={Zurich}}
\hypersetup{pdfcontactcountry={Switzerland}}
\hypersetup{pdfcontactemail={nbeisert@itp.phys.ethz.ch}}
\hypersetup{pdfcontacturl={http://people.phys.ethz.ch/\xmptilde nbeisert/}}

\newcommand{\secref}[1]{\hyperref[#1]{section \ref*{#1}}}

\parskip1ex
\parindent0pt
\let\olditemize\itemize
\def\itemize{\olditemize\parskip0pt}

\begin{document}

\title{The \textsf{childdoc} Package}
\hypersetup{pdftitle={The childdoc Package}}
\author{Niklas Beisert\\[2ex]
  Institut f\"ur Theoretische Physik\\
  Eidgen\"ossische Technische Hochschule Z\"urich\\
  Wolfgang-Pauli-Strasse 27, 8093 Z\"urich, Switzerland\\[1ex]
  \href{mailto:nbeisert@itp.phys.ethz.ch}
  {\texttt{nbeisert@itp.phys.ethz.ch}}}
\hypersetup{pdfauthor={Niklas Beisert}}
\hypersetup{pdfsubject={Manual for the LaTeX2e Package childdoc}}
\date{30 December 2018, \textsf{v2.0}}
\maketitle

\begin{abstract}\noindent
\textsf{childdoc} is a \LaTeXe{} package
that enables the direct compilation
of document sections included by |\include|
to individual files.
\end{abstract}

\begingroup
\parskip0ex
\tableofcontents
\endgroup

%%%%%%%%%%%%%%%%%%%%%%%%%%%%%%%%%%%%%%%%%%%%%%%%%%%%%%%%%%%%%%%%%%%%%%%%%%%%%%%%
%%%%%%%%%%%%%%%%%%%%%%%%%%%%%%%%%%%%%%%%%%%%%%%%%%%%%%%%%%%%%%%%%%%%%%%%%%%%%%%%
\section{Introduction}

\LaTeX{} provides a mechanism to structure a large document (such as a book)
into a main file and several child files (containing the chapters)
using the |\include| command.
This mechanism is beneficial for documents
which span hundreds of pages in order to
make the source file(s) more manageable.
Moreover, compilation can be restricted to
selected child files by means of the |\includeonly| command.
The latter feature can be used to reduce the compilation time while editing
(this was significantly more useful in the earlier days of \LaTeX{})
or to generate a smaller document which is easier to navigate.
Another application of |\includeonly| is to generate
documents consisting of selected parts of the complete document.

However, there are a few drawbacks of the plain |\include| mechanism:
\begin{itemize}
\item
The child files cannot be compiled on their own,
they can only be compiled via the main file.
A naive editing environment
(such as a text editor with an option
to have the current file processed by \LaTeX)
may require one to switch to the main file before compiling;
attempting to compile the child file produces errors.
\item
The main file must be modified (each time)
to adjust the |\includeonly| command
to the present needs. This easily leaves the main file in a messy state.
\item
The generated document will always carry the filename
of the main document. This is inconvenient if
several child files are to be compiled and
to be kept for distribution.
\end{itemize}

The present package provides a simple interface
to make child files individually compilable by \LaTeX{}.
Compiling a child file then has the same effect as compiling
the main file with an |\includeonly| command
to select the appropriate child.
Moreover the generated document will carry the name of the child
rather than the main file.
This resolves all three above issues.

This feature is meant to make the editing of books,
thesis documents and lecture notes somewhat more convenient.
However, the package can also be used efficiently for
composing a series of documents (such as exercise sheets)
which are typically distributed individually.
It then assists the author in generating the individual documents
(potentially in different versions)
as well as a document containing the collected series.
Another application is in developing style files
or other kinds of included material
where compilation of the style file could redirect
to a sample or test file.

%%%%%%%%%%%%%%%%%%%%%%%%%%%%%%%%%%%%%%%%%%%%%%%%%%%%%%%%%%%%%%%%%%%%%%%%%%%%%%%%
%%%%%%%%%%%%%%%%%%%%%%%%%%%%%%%%%%%%%%%%%%%%%%%%%%%%%%%%%%%%%%%%%%%%%%%%%%%%%%%%
\section{Usage}

First of all, the package \textsf{childdoc} is \emph{not} a standard
\LaTeXe{} |.sty| style file! Therefore it needs to be invoked in
a non-standard way.

%%%%%%%%%%%%%%%%%%%%%%%%%%%%%%%%%%%%%%%%%%%%%%%%%%%%%%%%%%%%%%%%%%%%%%%%%%%%%%%%
\subsection{Included Files}
\label{sec:include}

%%%%%%%%%%%%%%%%%%%%%%%%%%%%%%%%%%%%%%%%
\DescribeMacro{\childdocmain}
To use the package, add the commands
\begin{center}
\begin{tabular}{l}
|\input{childdoc.def}|\\
|\childdocmain{}|\\
\end{tabular}
\end{center}
at the very top of the main \LaTeX{} file,
in particular \emph{before} the |\documentclass| statement!
The argument of |\childdocmain| should be left empty
(but it must be present).

%%%%%%%%%%%%%%%%%%%%%%%%%%%%%%%%%%%%%%%%
\DescribeMacro{\childdocof}
Furthermore, add the commands
\begin{center}
\begin{tabular}{l}
|\input{childdoc.def}|\\
|\childdocof{|\textit{main}|}|\\
\end{tabular}
\end{center}
at the top of every child file \textit{child}
which is included by |\include{|\textit{child}|}|
from within the main file
(or at least for those files to be compiled individually).
The argument \textit{main} must be the filename of the main file.

There are a couple of
considerations in setting up the main and child documents:

%%%%%%%%%%%%%%%%%%%%%%%%%%%%%%%%%%%%%%%%
\paragraph{Restrictions.}

Please note the following restrictions:
\begin{itemize}
\item
|\childdocmain| must be called with one argument \textit{main}
to ensure compatibility with earlier version of the package.
It must either be empty (|\childdocmain{}|)
or precisely match the filename of the main file in which it is specified.
See \secref{sec:detection} for further information.
\item
The filename \textit{main} must be specified without the |.tex| extension.
\item
The filename \textit{main} is case sensitive
(even in case-insensitive file systems)
due to internal string comparison.
\item
The argument \textit{main} should be fully expanded, it cannot be a macro.
\item
Subdirectories and special characters should be avoided in filenames.
\item
The command |\childdocmain{|\textit{main}|}| must be followed by a whitespace.
It should not be followed immediately by another command
or by a comment mark `|%|'.
This is because the \TeX{} parser reads the token immediately following
the argument of |\childdocmain| and puts it
at the beginning of every child section;
however, a white\-space is ignored.
\end{itemize}

%%%%%%%%%%%%%%%%%%%%%%%%%%%%%%%%%%%%%%%%
\paragraph{Content of Main File.}

It is advisable to place all content in the child files included by |\include|.
Any output contained in the main file will appear in all child documents
unless suppressed manually;
it cannot be suppressed automatically by the |\includeonly| directive
and thus should normally be avoided.
A method to include some content in the main file
by means of conditional processing is described in \secref{sec:conditional}.

%%%%%%%%%%%%%%%%%%%%%%%%%%%%%%%%%%%%%%%%
\paragraph{Page Numbering.}

When only a part of the document is compiled,
the appropriate numbering of pages
(as well as other status parameters)
is determined from the |.aux| files.
The latter contain information from previous passes.
However this information needs to propagate through
all intermediate child documents.
Therefore the page numbering in child documents may well
be inconsistent until the complete document is compiled at least once.

A useful (if unconventional) way to always ensure a consistent
page numbering is to restart the numbering in each child document
and denote the pages by `\textit{child}|.|\textit{page}'
where \textit{child} represents the chapter/section number of the child file.
This can be achieved by the command
|\numberwithin{page}{|\textit{child}|}|
of the \textsf{amsmath} package
where \textit{child} can be |chapter| or |section|
depending on the chosen structuring.
Alternatively, one can modify the macro |\thepage| appropriately
and reset the counter |page| at the start of each child file.

%%%%%%%%%%%%%%%%%%%%%%%%%%%%%%%%%%%%%%%%%%%%%%%%%%%%%%%%%%%%%%%%%%%%%%%%%%%%%%%%
\subsection{Conditional Processing}
\label{sec:conditional}

The package provides a mechanism to compile different versions
of a document. To customise the versions further some conditional processing
can come in handy to distinguish which version is being compiled.
The package provides two macros to describe the compilation context:

%%%%%%%%%%%%%%%%%%%%%%%%%%%%%%%%%%%%%%%%
\DescribeMacro{\ifchilddoc}
The conditional |\ifchilddoc| distinguishes between the compilation of
child documents and the main document:
%
\begin{center}
|\ifchilddoc |\textit{child-code}| |[|\||else |\textit{main-code}]| \||fi|
\end{center}

%%%%%%%%%%%%%%%%%%%%%%%%%%%%%%%%%%%%%%%%
\DescribeMacro{\childdocname}
\DescribeMacro{\childdocjob}
The macro |\childdocname| contains the filename (without extension)
of the main or child file being processed.
Note that |\childdocjob| will always contain the name of the main file.

%%%%%%%%%%%%%%%%%%%%%%%%%%%%%%%%%%%%%%%%
\paragraph{Title Page.}

Conditional processing can be used to include a title or banner page
in the main document when proper precautions are taken.
Importantly, the code in the main file should ensure that the page counter
(as well as other status parameters which are stored in the |.aux| files)
takes the same value after the conditional processing.
Otherwise the page numbers may take divergent values
depending on which part is compiled.

For example, a title page could be declared by:
%
\begin{center}
\begin{tabular}{l}
|\ifchilddoc\||else|\\
|\addtocounter{page}{-1}|\\
\textit{code for title page}\\
|\newpage|\\
|\||fi|
\end{tabular}
\end{center}
%
A banner page for the child documents can be generated by:
%
\begin{center}
\begin{tabular}{l}
|\ifchilddoc|\\
|\addtocounter{page}{-1}|\\
\textit{code for banner page}\\
|\newpage|\\
|\||fi|
\end{tabular}
\end{center}
%
Here one could write a message such as:
\begin{center}
|This is the part \childdocname{} of \childdocjob{}.|
\end{center}

%%%%%%%%%%%%%%%%%%%%%%%%%%%%%%%%%%%%%%%%%%%%%%%%%%%%%%%%%%%%%%%%%%%%%%%%%%%%%%%%
\subsection{Flags}
\label{sec:flags}

The package makes it easy to generate different versions
of the main or child documents.
To this end compilation flags can be defined
and assigned different default values.
They will be particularly useful in conjunction
with the forwarding mechanism described in \secref{sec:forward}.

For example, it may be useful to have a flag |\version|
which can be set to |draft| or |final|.
The document source will contain some conditional code
depending on the value of |\version|.
Suppose further, the flag should default to |final| for the main file
and to |draft| for child files
which is a natural assignment for editing the document.
This is achieved by placing the following code
in the preamble of the main document
(below the |\childdocmain| directive):
%
\begin{center}
\begin{tabular}{l}
|\ifchilddoc|\\
|\providecommand{\version}{draft}|\\
|\||else|\\
|\providecommand{\version}{final}|\\
|\||fi|
\end{tabular}
\end{center}
%
The definition by |\providecommand| makes sure
that previous definitions are not overwritten.
Further statements |\providecommand{\version}{...}|
can thus be added before the above code to override it.

For the main file, one might add a line
(between |\childdocmain| and the above block)
%
\begin{center}
|%\ifchilddoc\||else\providecommand{\version}{draft}\||fi|
\end{center}
%
which can be uncommented to produce a draft version.
Likewise one can add a line to the very top of a child file
(above the |\childdocof{|\textit{main}|}| directive)
%
\begin{center}
|%\providecommand{\version}{final}|
\end{center}
%
which can be uncommented to produce the final version of this child document.

%%%%%%%%%%%%%%%%%%%%%%%%%%%%%%%%%%%%%%%%%%%%%%%%%%%%%%%%%%%%%%%%%%%%%%%%%%%%%%%%
\subsection{Forwarding}
\label{sec:forward}

Different versions of the main or child documents
using compilation flags as described in \secref{sec:flags}
can be (permanently) stored in different files
for convenient compilation, viewing and distribution.
To this end, the package defines a command
to pass on compilation to a different file:

%%%%%%%%%%%%%%%%%%%%%%%%%%%%%%%%%%%%%%%%
\DescribeMacro{\childdocforward}
The command |\childdocforward| redirects processing to
another source file:
%
\begin{center}
\begin{tabular}{l}
|\input{childdoc.def}|\\
|\childdocforward[|\textit{main}|]{|\textit{dest}|}|\\
\end{tabular}
\end{center}
%
The argument \textit{dest} is the destination file
(without extension).
It should be the main file or one of the child files.
Note that further \textsf{childdoc} directives
such as |\childdocof| and |\childdocforward|
in the indicated file will be processed in this form.
The optional argument \textit{main}
passes on directly to the main file \textit{main}
while pretending to compile the child \textit{dest}.
This form behaves as if \textit{dest}
issues |\childdocof{|\textit{main}|}| right away,
and no further \textsf{childdoc} directives will be processed.

%%%%%%%%%%%%%%%%%%%%%%%%%%%%%%%%%%%%%%%%
\DescribeMacro{\...prefix}
In the alternative form |\childdocforwardprefix|,
%
\begin{center}
\begin{tabular}{l}
|\input{childdoc.def}|\\
|\childdocforwardprefix[|\textit{main}|]{|\textit{prefix}|}{|\textit{dest}|}|
\end{tabular}
\end{center}
%
the destination file is determined by a pattern
depending on the current file:
To make this work, the current file must be called
`{\textit{prefix}\hspace{0.2em}\textit{suffix}}'
with \textit{prefix} matching precisely the argument.
Processing is then passed on to the file
`{\textit{dest}\hspace{0.2em}\textit{suffix}}'.
Surely, the same effect is achieved by
directly specifying the
argument `{\textit{dest}\hspace{0.2em}\textit{suffix}}'
in the first form.
However, that requires to set up a different file
for each child. With the alternative form of the command
all these files can have exactly the same content
which simplifies setting them up and maintaining them.

For example, the following file |draft.tex|
with a compilation flag |\version| as described in \secref{sec:flags}
compiles the main document as a draft:
%
\begin{center}
\begin{tabular}{l}
|\def\version{draft}|\\
|\input{childdoc.def}|\\
|\childdocforward{|\textit{main}|}|
\end{tabular}
\end{center}
%
Likewise, the following files |final|\textit{nn}|.tex|
compile the final version of the child document
|child|\textit{nn}|.tex|:
%
\begin{center}
\begin{tabular}{l}
|\def\version{final}|\\
|\input{childdoc.def}|\\
|\childdocforwardprefix{final}{child}|
\end{tabular}
\end{center}
%

Note that when several versions of a main file and/or of each child file
are to be generated, it may be convenient to set up a |Makefile| or
shell script to automatise the process.

%%%%%%%%%%%%%%%%%%%%%%%%%%%%%%%%%%%%%%%%%%%%%%%%%%%%%%%%%%%%%%%%%%%%%%%%%%%%%%%%
\subsection{Command Line Processing}
\label{sec:commandline}

The effect of redirection files can also be achieved by invoking
the \LaTeX{} compiler with a more elaborate command line.
Most conveniently this should be done as part
of a shell script or a |Makefile|.

When using \textsf{childdoc} in the main file, the following
command lines effectively perform a redirection
(note that depending on the shell being used,
backslashes may have to be doubled: `|\|' $\to$ `|\\|'):
%
\begin{center}
|... -jobname "|\textit{target}|" |\\|"|[\textit{flags}]%
|\input{childdoc.def}\childdocforward[|\textit{main}|]{|\textit{dest}|}"|
\end{center}
%
Here \textit{target} is the name of the output file,
\textit{main} is the name of the main file
and \textit{dest} is the name of the main or child file to be processed
(all filenames without extensions).
The optional argument \textit{main} can be omitted
if \textit{main} matches \textit{dest}.
Optionally, compilation \textit{flags} can be defined via |\def| commands.
This command line makes the \TeX{} engine believe
it is compiling the file \textit{target}
whose content is specified as the latter parameter.
The provided code then forwards the processing to
\textit{main} or \textit{dest} as described in \secref{sec:forward}.

%%%%%%%%%%%%%%%%%%%%%%%%%%%%%%%%%%%%%%%%%%%%%%%%%%%%%%%%%%%%%%%%%%%%%%%%%%%%%%%%
\subsection{Include by Input}
\label{sec:input}

Including child documents by |\include| has some restrictions by design.
Most notably, the content of a child document always occupies
its own set of pages; pages cannot be shared between child documents.
Usually, this behaviour makes perfect sense
because each child document contain an essential part of the document.
However, in some situations it may be desirable to compose
a document from a collection of parts
without having mandatory page breaks between then.
For this case, the package
provides a mechanism to include parts
by |\input| which can also be processed individually.
However, by construction this mechanism
requires manual handling of the content to be output.

%%%%%%%%%%%%%%%%%%%%%%%%%%%%%%%%%%%%%%%%
\DescribeMacro{\ifchilddocmanual}
The main file should be prepared as usual, see \secref{sec:include}.
However, the document body must make a distinction
between processing of an individual part and of the main document, e.g.:
%
\begin{center}
\begin{tabular}{l}
|\ifchilddocmanual|\\
|\input{\childdocname}|\\
|\||else|\\
\textit{document body with }|\input{|\textit{part}|}|\\
|\||fi|
\end{tabular}
\end{center}
%
The conditional |\ifchilddocmanual| is true whenever
a part to be included by |\input| is being compiled,
and the name of the part is stored in |\childdocname|.

%%%%%%%%%%%%%%%%%%%%%%%%%%%%%%%%%%%%%%%%
\DescribeMacro{\childdocby}
Each part to be included by |\input| should start with:
%
\begin{center}
\begin{tabular}{l}
|\input{childdoc.def}|\\
|\childdocby{|\textit{main}|}|\\
\end{tabular}
\end{center}
%
The directive |\childdocby| is similar to |\childdocof|
described in \secref{sec:include},
but the subsequent selection of content must be done manually.
To that end, both |\ifchilddoc| and |\ifchilddocmanual|
will be true upon processing of a part,
and the name of the part is stored in |\childdocname|.
Note that |\jobname| will be set to the filename of the current part
so that each part receives an individual |.aux| file
that does not interfere with the |.aux| file(s) of the main document.
This behaviour can be altered by the alternative form
|\childdocby[*]{|\textit{main}|}| (with a non-empty optional argument)
which uses the |.aux| file of the main document
by setting |\jobname| to \textit{main}.

%%%%%%%%%%%%%%%%%%%%%%%%%%%%%%%%%%%%%%%%%%%%%%%%%%%%%%%%%%%%%%%%%%%%%%%%%%%%%%%%
\subsection{Driver Development}
\label{sec:driver}

The \textsf{childdoc} mechanism can also be use for the development
of definition files such as \LaTeX{} styles or classes.
This case differs from the above setup with multiple parts
included by |\include| in that no |\includeonly| should be invoked.
This can be achieved by starting the include file
(before |\ProvidesPackage|) with:
%
\begin{center}
\begin{tabular}{l}
|\input{childdoc.def}|\\
|\childdocforward{|\textit{main}|}|\\
\end{tabular}
\end{center}
%
or alternatively with:
%
\begin{center}
\begin{tabular}{l}
|\input{childdoc.def}|\\
|\childdocby{|\textit{main}|}|\\
\end{tabular}
\end{center}
%
Both forms have slightly different effects as described above.
The main file is prepared as usual, see \secref{sec:include}.

%%%%%%%%%%%%%%%%%%%%%%%%%%%%%%%%%%%%%%%%%%%%%%%%%%%%%%%%%%%%%%%%%%%%%%%%%%%%%%%%
\subsection{Legacy Detection}
\label{sec:detection}

The directive |\childdocmain| in the main file can detect
whether the complete document or merely a child is to be compiled
even without using the directive |\childdocof|.
This method is deprecated because it is less robust
and there is no compelling reason to use it;
it is merely provided for backward compatibility
and it may be removed in future versions.

If the detection mechanism is to be used,
it is mandatory to correctly specify
the filename of the main file as the argument of |\childdocmain|:
%
\begin{center}
\begin{tabular}{l}
|\input{childdoc.def}|\\
|\childdocmain{|\textit{main}|}|\\
\end{tabular}
\end{center}
%
If |\jobname| does not match the argument \textit{main} of |\childdocmain|,
it is assumed that |\jobname| points to the child file to be compiled.
When using |\childdocmain| with the main file specified as argument,
it suffices to start a child file
with just |\input{|\textit{main}|}|
without loading of the package and using |\childdocof|.
If instead all processing is done
with the appropriate \textsf{childdoc} directives,
the argument of \textit{main} of |\childdocmain| can be empty.

An alternative version of the command line processing described
in \secref{sec:commandline} using the detection mechanism reads:
%
\begin{center}
|... -jobname "|\textit{target}|" "|[\textit{flags}]%
[|\def\jobname{|\textit{dest}|}|]|\input{|\textit{main}|}"|
\end{center}

%%%%%%%%%%%%%%%%%%%%%%%%%%%%%%%%%%%%%%%%%%%%%%%%%%%%%%%%%%%%%%%%%%%%%%%%%%%%%%%%
\subsection{Manual Code}
\label{sec:manual}

In case one cannot be certain whether the definitions file |childdoc.def|
is installed on the target \TeX{} distribution
and one prefers not to ship it,
it is conceivable to paste a few relevant commands into the sources.

To that end, drop all statements |\input{childdoc.def}|
and perform the replacements as outlined below.
Instead of |\childdocmain{|\textit{main}|}| add the following code
to the top of the main file:
%
\begin{center}
\begin{tabular}{l}
|\||ifdefined\childdocname\endinput\||fi\newif\ifchilddoc|\\
|\edef\childdocname{\scantokens\expandafter{\jobname\noexpand}}|\\
|\def\childdocmain{|\textit{main}|}\||ifx\childdocmain\childdocname\||else|\\
|\childdoctrue\includeonly{\childdocname}\let\jobname\childdocmain\||fi|\\
\end{tabular}
\end{center}
%
Instead of |\childdocof{|\textit{main}|}| just include the main file
at the top of each child file:
%
\begin{center}
|\input{|\textit{main}|}|
\end{center}
%
A simple redirection |\childdocforward{|\textit{dest}|}| is achieved by:
%
\begin{center}
|\def\jobname{|\textit{dest}|}\input{\jobname}|
\end{center}
%
The redirection with prefix
|\childdocforwardprefix[|\textit{prefix}|]{|\textit{dest}|}|
is accomplished by:
%
\begin{center}
\begin{tabular}{l}
|{\edef\jobname{\scantokens\expandafter{\jobname\noexpand}}|\\
|\def\redirectjob |\textit{prefix}|#1~~~{\gdef\jobname{|\textit{dest}|#1}}|\\
|\expandafter\redirectjob\jobname~~~}\input{\jobname}|
\end{tabular}
\end{center}

In an alternative approach,
child documents can be compiled by a specific command line
without additional code or specific definitions:
%
\begin{center}
|... -jobname "|\textit{target}|" "|[\textit{flags}]%
|\includeonly{|\textit{dest}|}\input{|\textit{main}|}"|
\end{center}
%

%%%%%%%%%%%%%%%%%%%%%%%%%%%%%%%%%%%%%%%%%%%%%%%%%%%%%%%%%%%%%%%%%%%%%%%%%%%%%%%%
%%%%%%%%%%%%%%%%%%%%%%%%%%%%%%%%%%%%%%%%%%%%%%%%%%%%%%%%%%%%%%%%%%%%%%%%%%%%%%%%
\section{Information}

%%%%%%%%%%%%%%%%%%%%%%%%%%%%%%%%%%%%%%%%%%%%%%%%%%%%%%%%%%%%%%%%%%%%%%%%%%%%%%%%
\subsection{Copyright}

Copyright \copyright{} 2017--2018 Niklas Beisert

This work may be distributed and/or modified under the
conditions of the \LaTeX{} Project Public License, either version 1.3
of this license or (at your option) any later version.
The latest version of this license is in
  \url{http://www.latex-project.org/lppl.txt}
and version 1.3 or later is part of all distributions of \LaTeX{}
version 2005/12/01 or later.

This work has the LPPL maintenance status `maintained'.

The Current Maintainer of this work is Niklas Beisert.

This work consists of the files |README.txt|, |childdoc.ins| and |childdoc.dtx|
as well as the derived files |childdoc.def|, |cdocsamp.tex|
with |cdocsch1.tex|, |cdocsch2.tex|, |cdocspt3.tex|, |cdocspt4.tex|,
|cdocsdrf.tex|, |cdocsfn1.tex|, |cdocsfn2.tex|
as well as |childdoc.pdf|.

%%%%%%%%%%%%%%%%%%%%%%%%%%%%%%%%%%%%%%%%%%%%%%%%%%%%%%%%%%%%%%%%%%%%%%%%%%%%%%%%
\subsection{Files and Installation}

The package consists of the files:
%
\begin{center}
\begin{tabular}{ll}
    |README.txt|   & readme file \\
    |childdoc.ins| & installation file \\
    |childdoc.dtx| & source file \\
    |childdoc.def| & definition file \\
    |cdocsamp.tex| & sample main file \\
    |cdocsch1.tex| & sample include file \\
    |cdocsch2.tex| & sample include file \\
    |cdocspt3.tex| & sample part file \\
    |cdocspt4.tex| & sample part file \\
    |cdocsdrf.tex| & sample redirection file \\
    |cdocsfn1.tex| & sample redirection file \\
    |cdocsfn2.tex| & sample redirection file \\
    |childdoc.pdf| & manual
\end{tabular}
\end{center}
%
The distribution consists of the files
|README.txt|, |childdoc.ins| and |childdoc.dtx|.
%
\begin{itemize}
\item
Run (pdf)\LaTeX{} on |childdoc.dtx|
to compile the manual |childdoc.pdf| (this file).
\item
Run \LaTeX{} on |childdoc.ins| to create the definitions file |childdoc.def|
and the sample |cdocsamp.tex| with include files
|cdocsch1.tex|, |cdocsch2.tex|, |cdocspt3.tex|, |cdocspt4.tex|,
|cdocsdrf.tex|, |cdocsfn1.tex|, |cdocsfn2.tex|.
Then copy the file |childdoc.def| to an appropriate directory of your \LaTeX{}
distribution, e.g.\ \textit{texmf-root}|/tex/latex/childdoc|.
\end{itemize}

%%%%%%%%%%%%%%%%%%%%%%%%%%%%%%%%%%%%%%%%%%%%%%%%%%%%%%%%%%%%%%%%%%%%%%%%%%%%%%%%
\subsection{Related CTAN Packages}

There are several other packages which offer a similar functionality:
%
\begin{itemize}
\item
The packages
\href{http://ctan.org/pkg/docmute}{\textsf{docmute}},
\href{http://ctan.org/pkg/includex}{\textsf{includex}} and
\href{http://ctan.org/pkg/standalone}{\textsf{standalone}}
provide commands to include only the document body of
a child file thus allowing both files to be compiled individually.
\item
The packages \href{http://ctan.org/pkg/subdocs}{\textsf{subdocs}}
and \href{http://ctan.org/pkg/subfiles}{\textsf{subfiles}}
provide structures in which the main and child documents can be
encapsulated and allowing them to be compiled individually.
The inclusion mechanism is different from the conventional |\include|.
\item
The package \href{http://ctan.org/pkg/combine}{\textsf{combine}}
is an elaborate solution to combine several documents into one.
\end{itemize}
%
See also the CTAN topic \href{http://ctan.org/topic/subdocs}{\textsf{subdocs}}
for further related packages.
The present package differs from the above solutions in that
a document structure constructed with the conventional |\include| mechanism
just needs two extra commands at the top of every file
such that all constituent files can be compiled individually.

%%%%%%%%%%%%%%%%%%%%%%%%%%%%%%%%%%%%%%%%%%%%%%%%%%%%%%%%%%%%%%%%%%%%%%%%%%%%%%%%
%\subsection{Feature Suggestions}
%
%The following is a list of features which may be useful for future
%versions of this package:
%%
%\begin{itemize}
%\item
%\ldots
%\end{itemize}

%%%%%%%%%%%%%%%%%%%%%%%%%%%%%%%%%%%%%%%%%%%%%%%%%%%%%%%%%%%%%%%%%%%%%%%%%%%%%%%%
\subsection{Revision History}

%%%%%%%%%%%%%%%%%%%%%%%%%%%%%%%%%%%%%%%%
\paragraph{v2.0:} 2018/12/30

\begin{itemize}
\item
immediate forward processing
\item
added |\childdocby| mechanism
\item
manual restructured
\end{itemize}

%%%%%%%%%%%%%%%%%%%%%%%%%%%%%%%%%%%%%%%%
\paragraph{v1.6:} 2018/01/17

\begin{itemize}
\item
application for development of include files
\item
corrections to manual
\end{itemize}

%%%%%%%%%%%%%%%%%%%%%%%%%%%%%%%%%%%%%%%%
\paragraph{v1.5:} 2017/05/21

\begin{itemize}
\item
more complete structuring introduced
\item
|\childdocof| introduced
\item
|\childdoc| renamed to |\childdocmain|
\item
|\childredirect| renamed to |\childdocforward| and |\childdocforwardprefix|
and functionality expanded
\end{itemize}

%%%%%%%%%%%%%%%%%%%%%%%%%%%%%%%%%%%%%%%%
\paragraph{v1.0:} 2017/04/27

\begin{itemize}
\item
manual and install package
\item
first version published on CTAN
\end{itemize}

%%%%%%%%%%%%%%%%%%%%%%%%%%%%%%%%%%%%%%%%
\paragraph{v0.6:} 2017/04/26

\begin{itemize}
\item
redirection mechanism added
\end{itemize}

%%%%%%%%%%%%%%%%%%%%%%%%%%%%%%%%%%%%%%%%
\paragraph{v0.5:} 2017/04/26

\begin{itemize}
\item
functionality in definition file
\end{itemize}


%%%%%%%%%%%%%%%%%%%%%%%%%%%%%%%%%%%%%%%%%%%%%%%%%%%%%%%%%%%%%%%%%%%%%%%%%%%%%%%%
%%%%%%%%%%%%%%%%%%%%%%%%%%%%%%%%%%%%%%%%%%%%%%%%%%%%%%%%%%%%%%%%%%%%%%%%%%%%%%%%
%%%%%%%%%%%%%%%%%%%%%%%%%%%%%%%%%%%%%%%%%%%%%%%%%%%%%%%%%%%%%%%%%%%%%%%%%%%%%%%%
\appendix

\settowidth\MacroIndent{\rmfamily\scriptsize 000\ }

 \DocInput{childdoc.dtx}

\end{document}
%</driver>
% \fi
%
% %%%%%%%%%%%%%%%%%%%%%%%%%%%%%%%%%%%%%%%%%%%%%%%%%%%%%%%%%%%%%%%%%%%%%%%%%%%%%%
% %%%%%%%%%%%%%%%%%%%%%%%%%%%%%%%%%%%%%%%%%%%%%%%%%%%%%%%%%%%%%%%%%%%%%%%%%%%%%%
% \section{Sample}
%\iffalse
%<*samplemain>
%\fi
%
% The following presents a sample document
% with two chapters, two parts, a title page,
% a compile flag as well as three forwarding files to set the flag.
% It consists of eight |.tex| files:
% \begin{center}
% \begin{tabular}{ll}
% |cdocsamp.tex|&main file\\
% |cdocsch1.tex|&include file for chapter 1\\
% |cdocsch2.tex|&include file for chapter 2\\
% |cdocspt3.tex|&include file for part 3\\
% |cdocspt4.tex|&include file for part 4\\
% |cdocsdrf.tex|&forwarding file for main file in draft mode\\
% |cdocsfi1.tex|&forwarding file for final version of chapter 1\\
% |cdocsfi2.tex|&forwarding file for final version of chapter 2\\
% \end{tabular}
% \end{center}
% Each of the eight files can be compiled directly by the \LaTeX{} compiler.
%
% %%%%%%%%%%%%%%%%%%%%%%%%%%%%%%%%%%%%%%
% \paragraph{Main File.}
%
% The main file is called |cdocsamp.tex|.
%
% Load the \textsf{childdoc} definitions and
% declare the filename for the main document:
%    \begin{macrocode}
\input{childdoc.def}
\childdocmain{}
%    \end{macrocode}

% Optional override for |\version| flag:
%    \begin{macrocode}
%%\ifchilddoc\else\providecommand{\version}{draft}\fi
%    \end{macrocode}

% Define the default values for the |\version| flag
% (|final| for the main file and |draft| for childs):
%    \begin{macrocode}
\ifchilddoc
\providecommand{\version}{draft}
\else
\providecommand{\version}{final}
\fi
%    \end{macrocode}

% Load the standard document class:
%    \begin{macrocode}
\documentclass[12pt]{article}
%    \end{macrocode}

% Start the document body:
%    \begin{macrocode}
\begin{document}
%    \end{macrocode}

% Declare a title page.
% Print title, part of document being processed and version flag:
%    \begin{macrocode}
\addtocounter{page}{-1}
\begin{center}
{\LARGE\bfseries{}childdoc example\par}
\vspace{1cm}
\ifchilddoc
\ifchilddocmanual part\else chapter\fi:
`\childdocname' of `\childdocjob'\par
\else
main document: `\childdocjob'\par
\fi
version: \version\par
\end{center}
\newpage
%    \end{macrocode}

% Manually include selected file,
% otherwise process as usual:
%    \begin{macrocode}
\ifchilddocmanual
\section*{part `\childdocname'}
\input{\childdocname}
\else
%    \end{macrocode}

% Include the two chapters:
%    \begin{macrocode}
\include{cdocsch1}
\include{cdocsch2}
%    \end{macrocode}

% Include the two parts unless only chapters should be displayed:
%    \begin{macrocode}
\ifchilddoc\else
\section{part three}
\input{cdocspt3}
\section{part four}
\input{cdocspt4}
\fi
%    \end{macrocode}

% Process as usual until here:
%    \begin{macrocode}
\fi
%    \end{macrocode}

% End of document body:
%    \begin{macrocode}
\end{document}
%    \end{macrocode}
%\iffalse
%</samplemain>
%\fi
%
% %%%%%%%%%%%%%%%%%%%%%%%%%%%%%%%%%%%%%%
% \paragraph{Chapter Include Files.}
%
% The include files are called |cdocsch1.tex| and |cdocsch2.tex|.
%
%\iffalse
%<*samplechap1|samplechap2>
%\fi

% Optional override for |\version| flag:
%    \begin{macrocode}
%%\providecommand{\version}{final}
%    \end{macrocode}

% Include the main document:
%    \begin{macrocode}
\input{childdoc.def}
\childdocof{cdocsamp}
%    \end{macrocode}

%\iffalse
%</samplechap1|samplechap2>
%\fi
%
%\iffalse
%<*samplechap1>
%\fi
% Some text for chapter 1:
%    \begin{macrocode}
\section{one}
some text in chapter one
%    \end{macrocode}

%\iffalse
%</samplechap1>
%\fi
% Some text for chapter 2:
%\iffalse
%<*samplechap2>
%\fi
%    \begin{macrocode}
\section{two}
more text in chapter two
%    \end{macrocode}

%\iffalse
%</samplechap2>
%\fi
%
% %%%%%%%%%%%%%%%%%%%%%%%%%%%%%%%%%%%%%%
% \paragraph{Part Include Files.}
%
% The include files are called |cdocspt3.tex| and |cdocspt4.tex|.
%
%\iffalse
%<*samplepart3|samplepart4>
%\fi

% Optional override for |\version| flag:
%    \begin{macrocode}
%%\providecommand{\version}{final}
%    \end{macrocode}

% Include the main document:
%    \begin{macrocode}
\input{childdoc.def}
\childdocby{cdocsamp}
%    \end{macrocode}

%\iffalse
%</samplepart3|samplepart4>
%\fi
%
%\iffalse
%<*samplepart3>
%\fi
% Some text for part 3:
%    \begin{macrocode}
some text in part three
%    \end{macrocode}

%\iffalse
%</samplepart3>
%\fi
% Some text for part 4:
%\iffalse
%<*samplepart4>
%\fi
%    \begin{macrocode}
more text in part four
%    \end{macrocode}

%\iffalse
%</samplepart4>
%\fi
%
% %%%%%%%%%%%%%%%%%%%%%%%%%%%%%%%%%%%%%%
% \paragraph{Forwarding for a Complete Draft.}
%
% The following forwarding file |cdocsdrf.tex|
% compiles the main document in draft mode:
%\iffalse
%<*sampledraft>
%\fi
%    \begin{macrocode}
\def\version{draft}
\input{childdoc.def}
\childdocforward{cdocsamp}
%    \end{macrocode}

%\iffalse
%</sampledraft>
%\fi
%
% %%%%%%%%%%%%%%%%%%%%%%%%%%%%%%%%%%%%%%
% \paragraph{Forwarding for Final Version of the Chapters.}
%
% The following forwarding files |cdocsfn1.tex| and |cdocsfn2.tex|
% (with identical content)
% compile the final versions of the child documents
% |cdocsch1.tex| and |cdocsch2.tex|, respectively:
%\iffalse
%<*samplefinal>
%\fi
%    \begin{macrocode}
\def\version{final}
\input{childdoc.def}
\childdocforwardprefix[cdocsamp]{cdocsfn}{cdocsch}
%    \end{macrocode}

%\iffalse
%</samplefinal>
%\fi
%
% %%%%%%%%%%%%%%%%%%%%%%%%%%%%%%%%%%%%%%
% \paragraph{Command Line Processing.}
%
% The following three command lines generate the output files
% |cdocscld|, |cdocscl1| and |cdocscl2|
% which should be identical to
% |cdocsdrf|, |cdocsch1| and |cdocsfn2|, respectively:
% \begin{center}
% \begin{tabular}{l}
% |latex -jobname cdocscld \|\\
% |  "\def\version{draft}\input{childdoc.def}\childdocforward{cdocsamp}"|\\
% |latex -jobname cdocscl1 \|\\
% |  "\input{childdoc.def}\childdocforward[cdocsamp]{cdocsch1}"|\\
% |latex -jobname cdocscl2 \|\\
% |  "\def\version{final}\input{childdoc.def}\childdocforward{cdocsch2}"|
% \end{tabular}
% \end{center}
% Note that the trailing backslash on each first line
% merely continues the input to the second line
% (for convenient cut ant paste).
% Furthermore, the command |latex| can be replaced by any
% of its alternative versions such as |pdflatex|.
%
% %%%%%%%%%%%%%%%%%%%%%%%%%%%%%%%%%%%%%%%%%%%%%%%%%%%%%%%%%%%%%%%%%%%%%%%%%%%%%%
% %%%%%%%%%%%%%%%%%%%%%%%%%%%%%%%%%%%%%%%%%%%%%%%%%%%%%%%%%%%%%%%%%%%%%%%%%%%%%%
% \section{Implementation}
%\iffalse
%<*package>
%\fi
%
% This section describes the definitions file |childdoc.def|.

% The definitions cannot be loaded using |\usepackage| or |\RequirePackage|
% which has a mechanism to prevent loading a style file more than once.
% When loading the definitions by means of |\input|
% multiple instances have to be prevented manually:
%\iffalse
%This code needs to be before the `\ProvidesFile' directive
%which is defined at the beginning of this file.
%Therefore it is also placed there and commented out here.
%</package>
%<*discard>
%\fi
%    \begin{macrocode}
\ifdefined\childdocmain\endinput\fi
%    \end{macrocode}
%\iffalse
%</discard>
%<*package>
%\fi
%
% \macro{\ifchilddoc}
% \macro{\ifchilddocmanual}
% The conditional |\ifchilddoc| tells whether a
% child (true) or main (false) document is being compiled.
% The conditional |\ifchilddocmanual| tells whether
% the |\includeonly| mechanism is used (false) or
% the selection of child files must be performed manually (true).
% The definitions initialise to false:
%    \begin{macrocode}
\newif\ifchilddoc
\newif\ifchilddocmanual
%    \end{macrocode}

% \macro{\childdocname}
% \macro{\childdocjob}
% The macro |\childdocname| stores the name of the main document
% to be compiled. The macro |\childdocjob| stores the name of
% the document on which the \LaTeX{} compiler was originally invoked.
% The content of |\jobname| cannot be compared
% to filenames specified in the source due to different catcodes.
% The following code rescans |\jobname|, stores the result
% in |\childdocname| and saves a copy in |\childdocjob|:
%    \begin{macrocode}
\edef\childdocname{\scantokens\expandafter{\jobname\noexpand}}
\let\childdocjob\childdocname
%    \end{macrocode}

% \macro{\childdocdisable}
% The macro |\childdocdisable| prevents the main file
% from being processed more than once.
% At this stage, the main document command |\childdocmain|
% is assumed to be called once again where it should do nothing.
% Any subsequent call to it should prevent
% a secondary processing of the main document
% It overwrites the forwarding commands
% |\childdocof| and |\childdocforward|
% with empty macros to prevent further inclusions of the main document:
%    \begin{macrocode}
\newcommand{\childdocdisable}
{
  \renewcommand{\childdocmain}[1]{\renewcommand{\childdocmain}[1]{\endinput}}
  \renewcommand{\childdocof}[1]{}
  \renewcommand{\childdocby}[2][]{}
  \renewcommand{\childdocforward}[2][]{}
  \renewcommand{\childdocdisable}{}
}
%    \end{macrocode}

% \macro{\childdocmain}
% The macro |\childdocmain| is to be called at the top of the main file
% with nothing or the main filename (without extension) as argument.
% First, it breaks loops.
% If the argument is not empty and does not match |\childdocname|
% (which is set by the first inclusion of |childdoc.def|),
% |\ifchilddoc| is set to true, |\includeonly| is applied to the child file
% and |\jobname| is set to the main file
% (for proper handling of |.aux| files):
%    \begin{macrocode}
\newcommand{\childdocmain}[1]
{
  \childdocdisable\childdocmain{}
  \if?#1?\else
    \begingroup
      \def\childdoctmp{#1}
      \ifx\childdoctmp\childdocname
        \def\childdoctmp{}
      \else
        \def\childdoctmp
        {
          \childdoctrue
          \includeonly{\childdocname}
          \def\childdocjob{#1}
          \def\jobname{#1}
        }
      \fi
      \expandafter
    \endgroup
    \childdoctmp
  \fi
}
%    \end{macrocode}

% \macro{\childdocof}
% The command |\childdocof| redirects
% compilation to the main file |#1|.
%    \begin{macrocode}
\newcommand{\childdocof}[1]
{
  \childdocdisable
  \childdoctrue
  \includeonly{\childdocname}
  \def\jobname{#1}
  \def\childdocjob{#1}
  \input{#1}
}
%    \end{macrocode}

% \macro{\childdocby}
% The command |\childdocby| ....
%    \begin{macrocode}
\newcommand{\childdocby}[2][]
{
  \childdocdisable
  \childdoctrue
  \childdocmanualtrue
  \if?#1?\else
    \def\jobname{#2}
  \fi
  \def\childdocjob{#2}
  \input{#2}
  \endinput
}
%    \end{macrocode}

% \macro{\childdocforward}
% The command |\childdocforward| redirects
% compilation to the main file or
% (if the optional argument is given) a child file.
% Parameters are set as if the main file
% or a child file starting with |\childdocof| was compiled.
% Then compilation is handed over to the main file:
%    \begin{macrocode}
\newcommand{\childdocforward}[2][]
{
  \begingroup
    \if?#1?
      \def\childdoctmp
      {
        \def\childdocname{#2}
        \def\childdocjob{#2}
        \def\jobname{#2}
        \input{#2}
        \endinput
      }
    \else
      \def\childdoctmp
      {
        \childdocdisable
        \def\childdocname{#2}
        \childdoctrue
        \includeonly{#2}
        \def\childdocjob{#1}
        \def\jobname{#1}
        \input{#1}
        \endinput
      }
    \fi
    \expandafter
  \endgroup
  \childdoctmp
}
%    \end{macrocode}

% \macro{\childdocforwardprefix}
% The command |\childdocforwardprefix| redirects
% compilation to the main or a child file by means of a pattern.
% The prefix |#1| in the current filename is replaced by |#2|
% and the suffix of the current filename is kept
% (it is assumed that the filename does not contain the substring `|~~~|'
% which is used as a delimiter).
% Compilation is handed over to the new file by |\childdocforward|:
%    \begin{macrocode}
\newcommand{\childdocforwardprefix}[3][]
{
  \begingroup
    \def\childdocextract #2##1~~~{\def\childdoctmp{\childdocforward[#1]{#3##1}}}
    \expandafter\childdocextract\childdocname~~~
    \expandafter
  \endgroup
  \childdoctmp
}
%    \end{macrocode}

% \macro{\childdoc}
% The deprecated macro |\childdoc| is a legacy version of |\childdocmain|:
%    \begin{macrocode}
\newcommand{\childdoc}{\childdocmain}
%    \end{macrocode}

% \macro{\childdocredirect}
% The deprecated macro |\childdocredirect| is a legacy version
% of |\childdocforward| and |\childdocforwardprefix|:
%    \begin{macrocode}
\newcommand{\childdocredirect}[2][]
{
  \begingroup
    \if?#1?
      \def\childdoctmp{\childdocforward{#2}}
    \else
      \def\childdoctmp{\childdocforwardprefix{#1}{#2}}
    \fi
    \expandafter
  \endgroup
  \childdoctmp
}
%    \end{macrocode}

%\iffalse
%</package>
%\fi
%
\endinput
\childdocforward{cdocsamp}"|\\
% |latex -jobname cdocscl1 \|\\
% |  "% \iffalse
%
% childdoc.dtx Copyright (C) 2017-2018 Niklas Beisert
%
% This work may be distributed and/or modified under the
% conditions of the LaTeX Project Public License, either version 1.3
% of this license or (at your option) any later version.
% The latest version of this license is in
%   http://www.latex-project.org/lppl.txt
% and version 1.3 or later is part of all distributions of LaTeX
% version 2005/12/01 or later.
%
% This work has the LPPL maintenance status `maintained'.
%
% The Current Maintainer of this work is Niklas Beisert.
%
% This work consists of the files childdoc.dtx and childdoc.ins
% and the derived files childdoc.def and cdocsamp.tex with
% cdocsch1.tex, cdocsch2.tex, cdocsdrf.tex, cdocsfn1.tex, cdocsfn2.tex.
%
%<package>\ifdefined\childdocmain\endinput\fi
%<package>\ProvidesFile{childdoc.def}[2018/12/30 v2.0 child document driver]
%<samplemain>\ProvidesFile{cdocsamp.tex}[2018/12/30 v2.0 sample for childdoc]
%<*driver>
%\ProvidesFile{childdoc.drv}[2018/12/30 v2.0 childdoc reference manual file]
\PassOptionsToClass{10pt,a4paper}{article}
\documentclass{ltxdoc}

\usepackage[margin=35mm]{geometry}
\usepackage{hyperref}
\usepackage{hyperxmp}
\usepackage[usenames]{color}

\hypersetup{colorlinks=true}
\hypersetup{pdfstartview=FitH}
\hypersetup{pdfpagemode=UseNone}
\hypersetup{pdfsource={}}
\hypersetup{pdflang={en-UK}}
\hypersetup{pdfcopyright={Copyright 2017-2018 Niklas Beisert.
  This work may be distributed and/or modified under the
  conditions of the LaTeX Project Public License, either version 1.3
  of this license or (at your option) any later version.}}
\hypersetup{pdflicenseurl={http://www.latex-project.org/lppl.txt}}
\hypersetup{pdfcontactaddress={ETH Zurich, ITP, HIT K,
  Wolfgang-Pauli-Strasse 27}}
\hypersetup{pdfcontactpostcode={8093}}
\hypersetup{pdfcontactcity={Zurich}}
\hypersetup{pdfcontactcountry={Switzerland}}
\hypersetup{pdfcontactemail={nbeisert@itp.phys.ethz.ch}}
\hypersetup{pdfcontacturl={http://people.phys.ethz.ch/\xmptilde nbeisert/}}

\newcommand{\secref}[1]{\hyperref[#1]{section \ref*{#1}}}

\parskip1ex
\parindent0pt
\let\olditemize\itemize
\def\itemize{\olditemize\parskip0pt}

\begin{document}

\title{The \textsf{childdoc} Package}
\hypersetup{pdftitle={The childdoc Package}}
\author{Niklas Beisert\\[2ex]
  Institut f\"ur Theoretische Physik\\
  Eidgen\"ossische Technische Hochschule Z\"urich\\
  Wolfgang-Pauli-Strasse 27, 8093 Z\"urich, Switzerland\\[1ex]
  \href{mailto:nbeisert@itp.phys.ethz.ch}
  {\texttt{nbeisert@itp.phys.ethz.ch}}}
\hypersetup{pdfauthor={Niklas Beisert}}
\hypersetup{pdfsubject={Manual for the LaTeX2e Package childdoc}}
\date{30 December 2018, \textsf{v2.0}}
\maketitle

\begin{abstract}\noindent
\textsf{childdoc} is a \LaTeXe{} package
that enables the direct compilation
of document sections included by |\include|
to individual files.
\end{abstract}

\begingroup
\parskip0ex
\tableofcontents
\endgroup

%%%%%%%%%%%%%%%%%%%%%%%%%%%%%%%%%%%%%%%%%%%%%%%%%%%%%%%%%%%%%%%%%%%%%%%%%%%%%%%%
%%%%%%%%%%%%%%%%%%%%%%%%%%%%%%%%%%%%%%%%%%%%%%%%%%%%%%%%%%%%%%%%%%%%%%%%%%%%%%%%
\section{Introduction}

\LaTeX{} provides a mechanism to structure a large document (such as a book)
into a main file and several child files (containing the chapters)
using the |\include| command.
This mechanism is beneficial for documents
which span hundreds of pages in order to
make the source file(s) more manageable.
Moreover, compilation can be restricted to
selected child files by means of the |\includeonly| command.
The latter feature can be used to reduce the compilation time while editing
(this was significantly more useful in the earlier days of \LaTeX{})
or to generate a smaller document which is easier to navigate.
Another application of |\includeonly| is to generate
documents consisting of selected parts of the complete document.

However, there are a few drawbacks of the plain |\include| mechanism:
\begin{itemize}
\item
The child files cannot be compiled on their own,
they can only be compiled via the main file.
A naive editing environment
(such as a text editor with an option
to have the current file processed by \LaTeX)
may require one to switch to the main file before compiling;
attempting to compile the child file produces errors.
\item
The main file must be modified (each time)
to adjust the |\includeonly| command
to the present needs. This easily leaves the main file in a messy state.
\item
The generated document will always carry the filename
of the main document. This is inconvenient if
several child files are to be compiled and
to be kept for distribution.
\end{itemize}

The present package provides a simple interface
to make child files individually compilable by \LaTeX{}.
Compiling a child file then has the same effect as compiling
the main file with an |\includeonly| command
to select the appropriate child.
Moreover the generated document will carry the name of the child
rather than the main file.
This resolves all three above issues.

This feature is meant to make the editing of books,
thesis documents and lecture notes somewhat more convenient.
However, the package can also be used efficiently for
composing a series of documents (such as exercise sheets)
which are typically distributed individually.
It then assists the author in generating the individual documents
(potentially in different versions)
as well as a document containing the collected series.
Another application is in developing style files
or other kinds of included material
where compilation of the style file could redirect
to a sample or test file.

%%%%%%%%%%%%%%%%%%%%%%%%%%%%%%%%%%%%%%%%%%%%%%%%%%%%%%%%%%%%%%%%%%%%%%%%%%%%%%%%
%%%%%%%%%%%%%%%%%%%%%%%%%%%%%%%%%%%%%%%%%%%%%%%%%%%%%%%%%%%%%%%%%%%%%%%%%%%%%%%%
\section{Usage}

First of all, the package \textsf{childdoc} is \emph{not} a standard
\LaTeXe{} |.sty| style file! Therefore it needs to be invoked in
a non-standard way.

%%%%%%%%%%%%%%%%%%%%%%%%%%%%%%%%%%%%%%%%%%%%%%%%%%%%%%%%%%%%%%%%%%%%%%%%%%%%%%%%
\subsection{Included Files}
\label{sec:include}

%%%%%%%%%%%%%%%%%%%%%%%%%%%%%%%%%%%%%%%%
\DescribeMacro{\childdocmain}
To use the package, add the commands
\begin{center}
\begin{tabular}{l}
|\input{childdoc.def}|\\
|\childdocmain{}|\\
\end{tabular}
\end{center}
at the very top of the main \LaTeX{} file,
in particular \emph{before} the |\documentclass| statement!
The argument of |\childdocmain| should be left empty
(but it must be present).

%%%%%%%%%%%%%%%%%%%%%%%%%%%%%%%%%%%%%%%%
\DescribeMacro{\childdocof}
Furthermore, add the commands
\begin{center}
\begin{tabular}{l}
|\input{childdoc.def}|\\
|\childdocof{|\textit{main}|}|\\
\end{tabular}
\end{center}
at the top of every child file \textit{child}
which is included by |\include{|\textit{child}|}|
from within the main file
(or at least for those files to be compiled individually).
The argument \textit{main} must be the filename of the main file.

There are a couple of
considerations in setting up the main and child documents:

%%%%%%%%%%%%%%%%%%%%%%%%%%%%%%%%%%%%%%%%
\paragraph{Restrictions.}

Please note the following restrictions:
\begin{itemize}
\item
|\childdocmain| must be called with one argument \textit{main}
to ensure compatibility with earlier version of the package.
It must either be empty (|\childdocmain{}|)
or precisely match the filename of the main file in which it is specified.
See \secref{sec:detection} for further information.
\item
The filename \textit{main} must be specified without the |.tex| extension.
\item
The filename \textit{main} is case sensitive
(even in case-insensitive file systems)
due to internal string comparison.
\item
The argument \textit{main} should be fully expanded, it cannot be a macro.
\item
Subdirectories and special characters should be avoided in filenames.
\item
The command |\childdocmain{|\textit{main}|}| must be followed by a whitespace.
It should not be followed immediately by another command
or by a comment mark `|%|'.
This is because the \TeX{} parser reads the token immediately following
the argument of |\childdocmain| and puts it
at the beginning of every child section;
however, a white\-space is ignored.
\end{itemize}

%%%%%%%%%%%%%%%%%%%%%%%%%%%%%%%%%%%%%%%%
\paragraph{Content of Main File.}

It is advisable to place all content in the child files included by |\include|.
Any output contained in the main file will appear in all child documents
unless suppressed manually;
it cannot be suppressed automatically by the |\includeonly| directive
and thus should normally be avoided.
A method to include some content in the main file
by means of conditional processing is described in \secref{sec:conditional}.

%%%%%%%%%%%%%%%%%%%%%%%%%%%%%%%%%%%%%%%%
\paragraph{Page Numbering.}

When only a part of the document is compiled,
the appropriate numbering of pages
(as well as other status parameters)
is determined from the |.aux| files.
The latter contain information from previous passes.
However this information needs to propagate through
all intermediate child documents.
Therefore the page numbering in child documents may well
be inconsistent until the complete document is compiled at least once.

A useful (if unconventional) way to always ensure a consistent
page numbering is to restart the numbering in each child document
and denote the pages by `\textit{child}|.|\textit{page}'
where \textit{child} represents the chapter/section number of the child file.
This can be achieved by the command
|\numberwithin{page}{|\textit{child}|}|
of the \textsf{amsmath} package
where \textit{child} can be |chapter| or |section|
depending on the chosen structuring.
Alternatively, one can modify the macro |\thepage| appropriately
and reset the counter |page| at the start of each child file.

%%%%%%%%%%%%%%%%%%%%%%%%%%%%%%%%%%%%%%%%%%%%%%%%%%%%%%%%%%%%%%%%%%%%%%%%%%%%%%%%
\subsection{Conditional Processing}
\label{sec:conditional}

The package provides a mechanism to compile different versions
of a document. To customise the versions further some conditional processing
can come in handy to distinguish which version is being compiled.
The package provides two macros to describe the compilation context:

%%%%%%%%%%%%%%%%%%%%%%%%%%%%%%%%%%%%%%%%
\DescribeMacro{\ifchilddoc}
The conditional |\ifchilddoc| distinguishes between the compilation of
child documents and the main document:
%
\begin{center}
|\ifchilddoc |\textit{child-code}| |[|\||else |\textit{main-code}]| \||fi|
\end{center}

%%%%%%%%%%%%%%%%%%%%%%%%%%%%%%%%%%%%%%%%
\DescribeMacro{\childdocname}
\DescribeMacro{\childdocjob}
The macro |\childdocname| contains the filename (without extension)
of the main or child file being processed.
Note that |\childdocjob| will always contain the name of the main file.

%%%%%%%%%%%%%%%%%%%%%%%%%%%%%%%%%%%%%%%%
\paragraph{Title Page.}

Conditional processing can be used to include a title or banner page
in the main document when proper precautions are taken.
Importantly, the code in the main file should ensure that the page counter
(as well as other status parameters which are stored in the |.aux| files)
takes the same value after the conditional processing.
Otherwise the page numbers may take divergent values
depending on which part is compiled.

For example, a title page could be declared by:
%
\begin{center}
\begin{tabular}{l}
|\ifchilddoc\||else|\\
|\addtocounter{page}{-1}|\\
\textit{code for title page}\\
|\newpage|\\
|\||fi|
\end{tabular}
\end{center}
%
A banner page for the child documents can be generated by:
%
\begin{center}
\begin{tabular}{l}
|\ifchilddoc|\\
|\addtocounter{page}{-1}|\\
\textit{code for banner page}\\
|\newpage|\\
|\||fi|
\end{tabular}
\end{center}
%
Here one could write a message such as:
\begin{center}
|This is the part \childdocname{} of \childdocjob{}.|
\end{center}

%%%%%%%%%%%%%%%%%%%%%%%%%%%%%%%%%%%%%%%%%%%%%%%%%%%%%%%%%%%%%%%%%%%%%%%%%%%%%%%%
\subsection{Flags}
\label{sec:flags}

The package makes it easy to generate different versions
of the main or child documents.
To this end compilation flags can be defined
and assigned different default values.
They will be particularly useful in conjunction
with the forwarding mechanism described in \secref{sec:forward}.

For example, it may be useful to have a flag |\version|
which can be set to |draft| or |final|.
The document source will contain some conditional code
depending on the value of |\version|.
Suppose further, the flag should default to |final| for the main file
and to |draft| for child files
which is a natural assignment for editing the document.
This is achieved by placing the following code
in the preamble of the main document
(below the |\childdocmain| directive):
%
\begin{center}
\begin{tabular}{l}
|\ifchilddoc|\\
|\providecommand{\version}{draft}|\\
|\||else|\\
|\providecommand{\version}{final}|\\
|\||fi|
\end{tabular}
\end{center}
%
The definition by |\providecommand| makes sure
that previous definitions are not overwritten.
Further statements |\providecommand{\version}{...}|
can thus be added before the above code to override it.

For the main file, one might add a line
(between |\childdocmain| and the above block)
%
\begin{center}
|%\ifchilddoc\||else\providecommand{\version}{draft}\||fi|
\end{center}
%
which can be uncommented to produce a draft version.
Likewise one can add a line to the very top of a child file
(above the |\childdocof{|\textit{main}|}| directive)
%
\begin{center}
|%\providecommand{\version}{final}|
\end{center}
%
which can be uncommented to produce the final version of this child document.

%%%%%%%%%%%%%%%%%%%%%%%%%%%%%%%%%%%%%%%%%%%%%%%%%%%%%%%%%%%%%%%%%%%%%%%%%%%%%%%%
\subsection{Forwarding}
\label{sec:forward}

Different versions of the main or child documents
using compilation flags as described in \secref{sec:flags}
can be (permanently) stored in different files
for convenient compilation, viewing and distribution.
To this end, the package defines a command
to pass on compilation to a different file:

%%%%%%%%%%%%%%%%%%%%%%%%%%%%%%%%%%%%%%%%
\DescribeMacro{\childdocforward}
The command |\childdocforward| redirects processing to
another source file:
%
\begin{center}
\begin{tabular}{l}
|\input{childdoc.def}|\\
|\childdocforward[|\textit{main}|]{|\textit{dest}|}|\\
\end{tabular}
\end{center}
%
The argument \textit{dest} is the destination file
(without extension).
It should be the main file or one of the child files.
Note that further \textsf{childdoc} directives
such as |\childdocof| and |\childdocforward|
in the indicated file will be processed in this form.
The optional argument \textit{main}
passes on directly to the main file \textit{main}
while pretending to compile the child \textit{dest}.
This form behaves as if \textit{dest}
issues |\childdocof{|\textit{main}|}| right away,
and no further \textsf{childdoc} directives will be processed.

%%%%%%%%%%%%%%%%%%%%%%%%%%%%%%%%%%%%%%%%
\DescribeMacro{\...prefix}
In the alternative form |\childdocforwardprefix|,
%
\begin{center}
\begin{tabular}{l}
|\input{childdoc.def}|\\
|\childdocforwardprefix[|\textit{main}|]{|\textit{prefix}|}{|\textit{dest}|}|
\end{tabular}
\end{center}
%
the destination file is determined by a pattern
depending on the current file:
To make this work, the current file must be called
`{\textit{prefix}\hspace{0.2em}\textit{suffix}}'
with \textit{prefix} matching precisely the argument.
Processing is then passed on to the file
`{\textit{dest}\hspace{0.2em}\textit{suffix}}'.
Surely, the same effect is achieved by
directly specifying the
argument `{\textit{dest}\hspace{0.2em}\textit{suffix}}'
in the first form.
However, that requires to set up a different file
for each child. With the alternative form of the command
all these files can have exactly the same content
which simplifies setting them up and maintaining them.

For example, the following file |draft.tex|
with a compilation flag |\version| as described in \secref{sec:flags}
compiles the main document as a draft:
%
\begin{center}
\begin{tabular}{l}
|\def\version{draft}|\\
|\input{childdoc.def}|\\
|\childdocforward{|\textit{main}|}|
\end{tabular}
\end{center}
%
Likewise, the following files |final|\textit{nn}|.tex|
compile the final version of the child document
|child|\textit{nn}|.tex|:
%
\begin{center}
\begin{tabular}{l}
|\def\version{final}|\\
|\input{childdoc.def}|\\
|\childdocforwardprefix{final}{child}|
\end{tabular}
\end{center}
%

Note that when several versions of a main file and/or of each child file
are to be generated, it may be convenient to set up a |Makefile| or
shell script to automatise the process.

%%%%%%%%%%%%%%%%%%%%%%%%%%%%%%%%%%%%%%%%%%%%%%%%%%%%%%%%%%%%%%%%%%%%%%%%%%%%%%%%
\subsection{Command Line Processing}
\label{sec:commandline}

The effect of redirection files can also be achieved by invoking
the \LaTeX{} compiler with a more elaborate command line.
Most conveniently this should be done as part
of a shell script or a |Makefile|.

When using \textsf{childdoc} in the main file, the following
command lines effectively perform a redirection
(note that depending on the shell being used,
backslashes may have to be doubled: `|\|' $\to$ `|\\|'):
%
\begin{center}
|... -jobname "|\textit{target}|" |\\|"|[\textit{flags}]%
|\input{childdoc.def}\childdocforward[|\textit{main}|]{|\textit{dest}|}"|
\end{center}
%
Here \textit{target} is the name of the output file,
\textit{main} is the name of the main file
and \textit{dest} is the name of the main or child file to be processed
(all filenames without extensions).
The optional argument \textit{main} can be omitted
if \textit{main} matches \textit{dest}.
Optionally, compilation \textit{flags} can be defined via |\def| commands.
This command line makes the \TeX{} engine believe
it is compiling the file \textit{target}
whose content is specified as the latter parameter.
The provided code then forwards the processing to
\textit{main} or \textit{dest} as described in \secref{sec:forward}.

%%%%%%%%%%%%%%%%%%%%%%%%%%%%%%%%%%%%%%%%%%%%%%%%%%%%%%%%%%%%%%%%%%%%%%%%%%%%%%%%
\subsection{Include by Input}
\label{sec:input}

Including child documents by |\include| has some restrictions by design.
Most notably, the content of a child document always occupies
its own set of pages; pages cannot be shared between child documents.
Usually, this behaviour makes perfect sense
because each child document contain an essential part of the document.
However, in some situations it may be desirable to compose
a document from a collection of parts
without having mandatory page breaks between then.
For this case, the package
provides a mechanism to include parts
by |\input| which can also be processed individually.
However, by construction this mechanism
requires manual handling of the content to be output.

%%%%%%%%%%%%%%%%%%%%%%%%%%%%%%%%%%%%%%%%
\DescribeMacro{\ifchilddocmanual}
The main file should be prepared as usual, see \secref{sec:include}.
However, the document body must make a distinction
between processing of an individual part and of the main document, e.g.:
%
\begin{center}
\begin{tabular}{l}
|\ifchilddocmanual|\\
|\input{\childdocname}|\\
|\||else|\\
\textit{document body with }|\input{|\textit{part}|}|\\
|\||fi|
\end{tabular}
\end{center}
%
The conditional |\ifchilddocmanual| is true whenever
a part to be included by |\input| is being compiled,
and the name of the part is stored in |\childdocname|.

%%%%%%%%%%%%%%%%%%%%%%%%%%%%%%%%%%%%%%%%
\DescribeMacro{\childdocby}
Each part to be included by |\input| should start with:
%
\begin{center}
\begin{tabular}{l}
|\input{childdoc.def}|\\
|\childdocby{|\textit{main}|}|\\
\end{tabular}
\end{center}
%
The directive |\childdocby| is similar to |\childdocof|
described in \secref{sec:include},
but the subsequent selection of content must be done manually.
To that end, both |\ifchilddoc| and |\ifchilddocmanual|
will be true upon processing of a part,
and the name of the part is stored in |\childdocname|.
Note that |\jobname| will be set to the filename of the current part
so that each part receives an individual |.aux| file
that does not interfere with the |.aux| file(s) of the main document.
This behaviour can be altered by the alternative form
|\childdocby[*]{|\textit{main}|}| (with a non-empty optional argument)
which uses the |.aux| file of the main document
by setting |\jobname| to \textit{main}.

%%%%%%%%%%%%%%%%%%%%%%%%%%%%%%%%%%%%%%%%%%%%%%%%%%%%%%%%%%%%%%%%%%%%%%%%%%%%%%%%
\subsection{Driver Development}
\label{sec:driver}

The \textsf{childdoc} mechanism can also be use for the development
of definition files such as \LaTeX{} styles or classes.
This case differs from the above setup with multiple parts
included by |\include| in that no |\includeonly| should be invoked.
This can be achieved by starting the include file
(before |\ProvidesPackage|) with:
%
\begin{center}
\begin{tabular}{l}
|\input{childdoc.def}|\\
|\childdocforward{|\textit{main}|}|\\
\end{tabular}
\end{center}
%
or alternatively with:
%
\begin{center}
\begin{tabular}{l}
|\input{childdoc.def}|\\
|\childdocby{|\textit{main}|}|\\
\end{tabular}
\end{center}
%
Both forms have slightly different effects as described above.
The main file is prepared as usual, see \secref{sec:include}.

%%%%%%%%%%%%%%%%%%%%%%%%%%%%%%%%%%%%%%%%%%%%%%%%%%%%%%%%%%%%%%%%%%%%%%%%%%%%%%%%
\subsection{Legacy Detection}
\label{sec:detection}

The directive |\childdocmain| in the main file can detect
whether the complete document or merely a child is to be compiled
even without using the directive |\childdocof|.
This method is deprecated because it is less robust
and there is no compelling reason to use it;
it is merely provided for backward compatibility
and it may be removed in future versions.

If the detection mechanism is to be used,
it is mandatory to correctly specify
the filename of the main file as the argument of |\childdocmain|:
%
\begin{center}
\begin{tabular}{l}
|\input{childdoc.def}|\\
|\childdocmain{|\textit{main}|}|\\
\end{tabular}
\end{center}
%
If |\jobname| does not match the argument \textit{main} of |\childdocmain|,
it is assumed that |\jobname| points to the child file to be compiled.
When using |\childdocmain| with the main file specified as argument,
it suffices to start a child file
with just |\input{|\textit{main}|}|
without loading of the package and using |\childdocof|.
If instead all processing is done
with the appropriate \textsf{childdoc} directives,
the argument of \textit{main} of |\childdocmain| can be empty.

An alternative version of the command line processing described
in \secref{sec:commandline} using the detection mechanism reads:
%
\begin{center}
|... -jobname "|\textit{target}|" "|[\textit{flags}]%
[|\def\jobname{|\textit{dest}|}|]|\input{|\textit{main}|}"|
\end{center}

%%%%%%%%%%%%%%%%%%%%%%%%%%%%%%%%%%%%%%%%%%%%%%%%%%%%%%%%%%%%%%%%%%%%%%%%%%%%%%%%
\subsection{Manual Code}
\label{sec:manual}

In case one cannot be certain whether the definitions file |childdoc.def|
is installed on the target \TeX{} distribution
and one prefers not to ship it,
it is conceivable to paste a few relevant commands into the sources.

To that end, drop all statements |\input{childdoc.def}|
and perform the replacements as outlined below.
Instead of |\childdocmain{|\textit{main}|}| add the following code
to the top of the main file:
%
\begin{center}
\begin{tabular}{l}
|\||ifdefined\childdocname\endinput\||fi\newif\ifchilddoc|\\
|\edef\childdocname{\scantokens\expandafter{\jobname\noexpand}}|\\
|\def\childdocmain{|\textit{main}|}\||ifx\childdocmain\childdocname\||else|\\
|\childdoctrue\includeonly{\childdocname}\let\jobname\childdocmain\||fi|\\
\end{tabular}
\end{center}
%
Instead of |\childdocof{|\textit{main}|}| just include the main file
at the top of each child file:
%
\begin{center}
|\input{|\textit{main}|}|
\end{center}
%
A simple redirection |\childdocforward{|\textit{dest}|}| is achieved by:
%
\begin{center}
|\def\jobname{|\textit{dest}|}\input{\jobname}|
\end{center}
%
The redirection with prefix
|\childdocforwardprefix[|\textit{prefix}|]{|\textit{dest}|}|
is accomplished by:
%
\begin{center}
\begin{tabular}{l}
|{\edef\jobname{\scantokens\expandafter{\jobname\noexpand}}|\\
|\def\redirectjob |\textit{prefix}|#1~~~{\gdef\jobname{|\textit{dest}|#1}}|\\
|\expandafter\redirectjob\jobname~~~}\input{\jobname}|
\end{tabular}
\end{center}

In an alternative approach,
child documents can be compiled by a specific command line
without additional code or specific definitions:
%
\begin{center}
|... -jobname "|\textit{target}|" "|[\textit{flags}]%
|\includeonly{|\textit{dest}|}\input{|\textit{main}|}"|
\end{center}
%

%%%%%%%%%%%%%%%%%%%%%%%%%%%%%%%%%%%%%%%%%%%%%%%%%%%%%%%%%%%%%%%%%%%%%%%%%%%%%%%%
%%%%%%%%%%%%%%%%%%%%%%%%%%%%%%%%%%%%%%%%%%%%%%%%%%%%%%%%%%%%%%%%%%%%%%%%%%%%%%%%
\section{Information}

%%%%%%%%%%%%%%%%%%%%%%%%%%%%%%%%%%%%%%%%%%%%%%%%%%%%%%%%%%%%%%%%%%%%%%%%%%%%%%%%
\subsection{Copyright}

Copyright \copyright{} 2017--2018 Niklas Beisert

This work may be distributed and/or modified under the
conditions of the \LaTeX{} Project Public License, either version 1.3
of this license or (at your option) any later version.
The latest version of this license is in
  \url{http://www.latex-project.org/lppl.txt}
and version 1.3 or later is part of all distributions of \LaTeX{}
version 2005/12/01 or later.

This work has the LPPL maintenance status `maintained'.

The Current Maintainer of this work is Niklas Beisert.

This work consists of the files |README.txt|, |childdoc.ins| and |childdoc.dtx|
as well as the derived files |childdoc.def|, |cdocsamp.tex|
with |cdocsch1.tex|, |cdocsch2.tex|, |cdocspt3.tex|, |cdocspt4.tex|,
|cdocsdrf.tex|, |cdocsfn1.tex|, |cdocsfn2.tex|
as well as |childdoc.pdf|.

%%%%%%%%%%%%%%%%%%%%%%%%%%%%%%%%%%%%%%%%%%%%%%%%%%%%%%%%%%%%%%%%%%%%%%%%%%%%%%%%
\subsection{Files and Installation}

The package consists of the files:
%
\begin{center}
\begin{tabular}{ll}
    |README.txt|   & readme file \\
    |childdoc.ins| & installation file \\
    |childdoc.dtx| & source file \\
    |childdoc.def| & definition file \\
    |cdocsamp.tex| & sample main file \\
    |cdocsch1.tex| & sample include file \\
    |cdocsch2.tex| & sample include file \\
    |cdocspt3.tex| & sample part file \\
    |cdocspt4.tex| & sample part file \\
    |cdocsdrf.tex| & sample redirection file \\
    |cdocsfn1.tex| & sample redirection file \\
    |cdocsfn2.tex| & sample redirection file \\
    |childdoc.pdf| & manual
\end{tabular}
\end{center}
%
The distribution consists of the files
|README.txt|, |childdoc.ins| and |childdoc.dtx|.
%
\begin{itemize}
\item
Run (pdf)\LaTeX{} on |childdoc.dtx|
to compile the manual |childdoc.pdf| (this file).
\item
Run \LaTeX{} on |childdoc.ins| to create the definitions file |childdoc.def|
and the sample |cdocsamp.tex| with include files
|cdocsch1.tex|, |cdocsch2.tex|, |cdocspt3.tex|, |cdocspt4.tex|,
|cdocsdrf.tex|, |cdocsfn1.tex|, |cdocsfn2.tex|.
Then copy the file |childdoc.def| to an appropriate directory of your \LaTeX{}
distribution, e.g.\ \textit{texmf-root}|/tex/latex/childdoc|.
\end{itemize}

%%%%%%%%%%%%%%%%%%%%%%%%%%%%%%%%%%%%%%%%%%%%%%%%%%%%%%%%%%%%%%%%%%%%%%%%%%%%%%%%
\subsection{Related CTAN Packages}

There are several other packages which offer a similar functionality:
%
\begin{itemize}
\item
The packages
\href{http://ctan.org/pkg/docmute}{\textsf{docmute}},
\href{http://ctan.org/pkg/includex}{\textsf{includex}} and
\href{http://ctan.org/pkg/standalone}{\textsf{standalone}}
provide commands to include only the document body of
a child file thus allowing both files to be compiled individually.
\item
The packages \href{http://ctan.org/pkg/subdocs}{\textsf{subdocs}}
and \href{http://ctan.org/pkg/subfiles}{\textsf{subfiles}}
provide structures in which the main and child documents can be
encapsulated and allowing them to be compiled individually.
The inclusion mechanism is different from the conventional |\include|.
\item
The package \href{http://ctan.org/pkg/combine}{\textsf{combine}}
is an elaborate solution to combine several documents into one.
\end{itemize}
%
See also the CTAN topic \href{http://ctan.org/topic/subdocs}{\textsf{subdocs}}
for further related packages.
The present package differs from the above solutions in that
a document structure constructed with the conventional |\include| mechanism
just needs two extra commands at the top of every file
such that all constituent files can be compiled individually.

%%%%%%%%%%%%%%%%%%%%%%%%%%%%%%%%%%%%%%%%%%%%%%%%%%%%%%%%%%%%%%%%%%%%%%%%%%%%%%%%
%\subsection{Feature Suggestions}
%
%The following is a list of features which may be useful for future
%versions of this package:
%%
%\begin{itemize}
%\item
%\ldots
%\end{itemize}

%%%%%%%%%%%%%%%%%%%%%%%%%%%%%%%%%%%%%%%%%%%%%%%%%%%%%%%%%%%%%%%%%%%%%%%%%%%%%%%%
\subsection{Revision History}

%%%%%%%%%%%%%%%%%%%%%%%%%%%%%%%%%%%%%%%%
\paragraph{v2.0:} 2018/12/30

\begin{itemize}
\item
immediate forward processing
\item
added |\childdocby| mechanism
\item
manual restructured
\end{itemize}

%%%%%%%%%%%%%%%%%%%%%%%%%%%%%%%%%%%%%%%%
\paragraph{v1.6:} 2018/01/17

\begin{itemize}
\item
application for development of include files
\item
corrections to manual
\end{itemize}

%%%%%%%%%%%%%%%%%%%%%%%%%%%%%%%%%%%%%%%%
\paragraph{v1.5:} 2017/05/21

\begin{itemize}
\item
more complete structuring introduced
\item
|\childdocof| introduced
\item
|\childdoc| renamed to |\childdocmain|
\item
|\childredirect| renamed to |\childdocforward| and |\childdocforwardprefix|
and functionality expanded
\end{itemize}

%%%%%%%%%%%%%%%%%%%%%%%%%%%%%%%%%%%%%%%%
\paragraph{v1.0:} 2017/04/27

\begin{itemize}
\item
manual and install package
\item
first version published on CTAN
\end{itemize}

%%%%%%%%%%%%%%%%%%%%%%%%%%%%%%%%%%%%%%%%
\paragraph{v0.6:} 2017/04/26

\begin{itemize}
\item
redirection mechanism added
\end{itemize}

%%%%%%%%%%%%%%%%%%%%%%%%%%%%%%%%%%%%%%%%
\paragraph{v0.5:} 2017/04/26

\begin{itemize}
\item
functionality in definition file
\end{itemize}


%%%%%%%%%%%%%%%%%%%%%%%%%%%%%%%%%%%%%%%%%%%%%%%%%%%%%%%%%%%%%%%%%%%%%%%%%%%%%%%%
%%%%%%%%%%%%%%%%%%%%%%%%%%%%%%%%%%%%%%%%%%%%%%%%%%%%%%%%%%%%%%%%%%%%%%%%%%%%%%%%
%%%%%%%%%%%%%%%%%%%%%%%%%%%%%%%%%%%%%%%%%%%%%%%%%%%%%%%%%%%%%%%%%%%%%%%%%%%%%%%%
\appendix

\settowidth\MacroIndent{\rmfamily\scriptsize 000\ }

 \DocInput{childdoc.dtx}

\end{document}
%</driver>
% \fi
%
% %%%%%%%%%%%%%%%%%%%%%%%%%%%%%%%%%%%%%%%%%%%%%%%%%%%%%%%%%%%%%%%%%%%%%%%%%%%%%%
% %%%%%%%%%%%%%%%%%%%%%%%%%%%%%%%%%%%%%%%%%%%%%%%%%%%%%%%%%%%%%%%%%%%%%%%%%%%%%%
% \section{Sample}
%\iffalse
%<*samplemain>
%\fi
%
% The following presents a sample document
% with two chapters, two parts, a title page,
% a compile flag as well as three forwarding files to set the flag.
% It consists of eight |.tex| files:
% \begin{center}
% \begin{tabular}{ll}
% |cdocsamp.tex|&main file\\
% |cdocsch1.tex|&include file for chapter 1\\
% |cdocsch2.tex|&include file for chapter 2\\
% |cdocspt3.tex|&include file for part 3\\
% |cdocspt4.tex|&include file for part 4\\
% |cdocsdrf.tex|&forwarding file for main file in draft mode\\
% |cdocsfi1.tex|&forwarding file for final version of chapter 1\\
% |cdocsfi2.tex|&forwarding file for final version of chapter 2\\
% \end{tabular}
% \end{center}
% Each of the eight files can be compiled directly by the \LaTeX{} compiler.
%
% %%%%%%%%%%%%%%%%%%%%%%%%%%%%%%%%%%%%%%
% \paragraph{Main File.}
%
% The main file is called |cdocsamp.tex|.
%
% Load the \textsf{childdoc} definitions and
% declare the filename for the main document:
%    \begin{macrocode}
\input{childdoc.def}
\childdocmain{}
%    \end{macrocode}

% Optional override for |\version| flag:
%    \begin{macrocode}
%%\ifchilddoc\else\providecommand{\version}{draft}\fi
%    \end{macrocode}

% Define the default values for the |\version| flag
% (|final| for the main file and |draft| for childs):
%    \begin{macrocode}
\ifchilddoc
\providecommand{\version}{draft}
\else
\providecommand{\version}{final}
\fi
%    \end{macrocode}

% Load the standard document class:
%    \begin{macrocode}
\documentclass[12pt]{article}
%    \end{macrocode}

% Start the document body:
%    \begin{macrocode}
\begin{document}
%    \end{macrocode}

% Declare a title page.
% Print title, part of document being processed and version flag:
%    \begin{macrocode}
\addtocounter{page}{-1}
\begin{center}
{\LARGE\bfseries{}childdoc example\par}
\vspace{1cm}
\ifchilddoc
\ifchilddocmanual part\else chapter\fi:
`\childdocname' of `\childdocjob'\par
\else
main document: `\childdocjob'\par
\fi
version: \version\par
\end{center}
\newpage
%    \end{macrocode}

% Manually include selected file,
% otherwise process as usual:
%    \begin{macrocode}
\ifchilddocmanual
\section*{part `\childdocname'}
\input{\childdocname}
\else
%    \end{macrocode}

% Include the two chapters:
%    \begin{macrocode}
\include{cdocsch1}
\include{cdocsch2}
%    \end{macrocode}

% Include the two parts unless only chapters should be displayed:
%    \begin{macrocode}
\ifchilddoc\else
\section{part three}
\input{cdocspt3}
\section{part four}
\input{cdocspt4}
\fi
%    \end{macrocode}

% Process as usual until here:
%    \begin{macrocode}
\fi
%    \end{macrocode}

% End of document body:
%    \begin{macrocode}
\end{document}
%    \end{macrocode}
%\iffalse
%</samplemain>
%\fi
%
% %%%%%%%%%%%%%%%%%%%%%%%%%%%%%%%%%%%%%%
% \paragraph{Chapter Include Files.}
%
% The include files are called |cdocsch1.tex| and |cdocsch2.tex|.
%
%\iffalse
%<*samplechap1|samplechap2>
%\fi

% Optional override for |\version| flag:
%    \begin{macrocode}
%%\providecommand{\version}{final}
%    \end{macrocode}

% Include the main document:
%    \begin{macrocode}
\input{childdoc.def}
\childdocof{cdocsamp}
%    \end{macrocode}

%\iffalse
%</samplechap1|samplechap2>
%\fi
%
%\iffalse
%<*samplechap1>
%\fi
% Some text for chapter 1:
%    \begin{macrocode}
\section{one}
some text in chapter one
%    \end{macrocode}

%\iffalse
%</samplechap1>
%\fi
% Some text for chapter 2:
%\iffalse
%<*samplechap2>
%\fi
%    \begin{macrocode}
\section{two}
more text in chapter two
%    \end{macrocode}

%\iffalse
%</samplechap2>
%\fi
%
% %%%%%%%%%%%%%%%%%%%%%%%%%%%%%%%%%%%%%%
% \paragraph{Part Include Files.}
%
% The include files are called |cdocspt3.tex| and |cdocspt4.tex|.
%
%\iffalse
%<*samplepart3|samplepart4>
%\fi

% Optional override for |\version| flag:
%    \begin{macrocode}
%%\providecommand{\version}{final}
%    \end{macrocode}

% Include the main document:
%    \begin{macrocode}
\input{childdoc.def}
\childdocby{cdocsamp}
%    \end{macrocode}

%\iffalse
%</samplepart3|samplepart4>
%\fi
%
%\iffalse
%<*samplepart3>
%\fi
% Some text for part 3:
%    \begin{macrocode}
some text in part three
%    \end{macrocode}

%\iffalse
%</samplepart3>
%\fi
% Some text for part 4:
%\iffalse
%<*samplepart4>
%\fi
%    \begin{macrocode}
more text in part four
%    \end{macrocode}

%\iffalse
%</samplepart4>
%\fi
%
% %%%%%%%%%%%%%%%%%%%%%%%%%%%%%%%%%%%%%%
% \paragraph{Forwarding for a Complete Draft.}
%
% The following forwarding file |cdocsdrf.tex|
% compiles the main document in draft mode:
%\iffalse
%<*sampledraft>
%\fi
%    \begin{macrocode}
\def\version{draft}
\input{childdoc.def}
\childdocforward{cdocsamp}
%    \end{macrocode}

%\iffalse
%</sampledraft>
%\fi
%
% %%%%%%%%%%%%%%%%%%%%%%%%%%%%%%%%%%%%%%
% \paragraph{Forwarding for Final Version of the Chapters.}
%
% The following forwarding files |cdocsfn1.tex| and |cdocsfn2.tex|
% (with identical content)
% compile the final versions of the child documents
% |cdocsch1.tex| and |cdocsch2.tex|, respectively:
%\iffalse
%<*samplefinal>
%\fi
%    \begin{macrocode}
\def\version{final}
\input{childdoc.def}
\childdocforwardprefix[cdocsamp]{cdocsfn}{cdocsch}
%    \end{macrocode}

%\iffalse
%</samplefinal>
%\fi
%
% %%%%%%%%%%%%%%%%%%%%%%%%%%%%%%%%%%%%%%
% \paragraph{Command Line Processing.}
%
% The following three command lines generate the output files
% |cdocscld|, |cdocscl1| and |cdocscl2|
% which should be identical to
% |cdocsdrf|, |cdocsch1| and |cdocsfn2|, respectively:
% \begin{center}
% \begin{tabular}{l}
% |latex -jobname cdocscld \|\\
% |  "\def\version{draft}\input{childdoc.def}\childdocforward{cdocsamp}"|\\
% |latex -jobname cdocscl1 \|\\
% |  "\input{childdoc.def}\childdocforward[cdocsamp]{cdocsch1}"|\\
% |latex -jobname cdocscl2 \|\\
% |  "\def\version{final}\input{childdoc.def}\childdocforward{cdocsch2}"|
% \end{tabular}
% \end{center}
% Note that the trailing backslash on each first line
% merely continues the input to the second line
% (for convenient cut ant paste).
% Furthermore, the command |latex| can be replaced by any
% of its alternative versions such as |pdflatex|.
%
% %%%%%%%%%%%%%%%%%%%%%%%%%%%%%%%%%%%%%%%%%%%%%%%%%%%%%%%%%%%%%%%%%%%%%%%%%%%%%%
% %%%%%%%%%%%%%%%%%%%%%%%%%%%%%%%%%%%%%%%%%%%%%%%%%%%%%%%%%%%%%%%%%%%%%%%%%%%%%%
% \section{Implementation}
%\iffalse
%<*package>
%\fi
%
% This section describes the definitions file |childdoc.def|.

% The definitions cannot be loaded using |\usepackage| or |\RequirePackage|
% which has a mechanism to prevent loading a style file more than once.
% When loading the definitions by means of |\input|
% multiple instances have to be prevented manually:
%\iffalse
%This code needs to be before the `\ProvidesFile' directive
%which is defined at the beginning of this file.
%Therefore it is also placed there and commented out here.
%</package>
%<*discard>
%\fi
%    \begin{macrocode}
\ifdefined\childdocmain\endinput\fi
%    \end{macrocode}
%\iffalse
%</discard>
%<*package>
%\fi
%
% \macro{\ifchilddoc}
% \macro{\ifchilddocmanual}
% The conditional |\ifchilddoc| tells whether a
% child (true) or main (false) document is being compiled.
% The conditional |\ifchilddocmanual| tells whether
% the |\includeonly| mechanism is used (false) or
% the selection of child files must be performed manually (true).
% The definitions initialise to false:
%    \begin{macrocode}
\newif\ifchilddoc
\newif\ifchilddocmanual
%    \end{macrocode}

% \macro{\childdocname}
% \macro{\childdocjob}
% The macro |\childdocname| stores the name of the main document
% to be compiled. The macro |\childdocjob| stores the name of
% the document on which the \LaTeX{} compiler was originally invoked.
% The content of |\jobname| cannot be compared
% to filenames specified in the source due to different catcodes.
% The following code rescans |\jobname|, stores the result
% in |\childdocname| and saves a copy in |\childdocjob|:
%    \begin{macrocode}
\edef\childdocname{\scantokens\expandafter{\jobname\noexpand}}
\let\childdocjob\childdocname
%    \end{macrocode}

% \macro{\childdocdisable}
% The macro |\childdocdisable| prevents the main file
% from being processed more than once.
% At this stage, the main document command |\childdocmain|
% is assumed to be called once again where it should do nothing.
% Any subsequent call to it should prevent
% a secondary processing of the main document
% It overwrites the forwarding commands
% |\childdocof| and |\childdocforward|
% with empty macros to prevent further inclusions of the main document:
%    \begin{macrocode}
\newcommand{\childdocdisable}
{
  \renewcommand{\childdocmain}[1]{\renewcommand{\childdocmain}[1]{\endinput}}
  \renewcommand{\childdocof}[1]{}
  \renewcommand{\childdocby}[2][]{}
  \renewcommand{\childdocforward}[2][]{}
  \renewcommand{\childdocdisable}{}
}
%    \end{macrocode}

% \macro{\childdocmain}
% The macro |\childdocmain| is to be called at the top of the main file
% with nothing or the main filename (without extension) as argument.
% First, it breaks loops.
% If the argument is not empty and does not match |\childdocname|
% (which is set by the first inclusion of |childdoc.def|),
% |\ifchilddoc| is set to true, |\includeonly| is applied to the child file
% and |\jobname| is set to the main file
% (for proper handling of |.aux| files):
%    \begin{macrocode}
\newcommand{\childdocmain}[1]
{
  \childdocdisable\childdocmain{}
  \if?#1?\else
    \begingroup
      \def\childdoctmp{#1}
      \ifx\childdoctmp\childdocname
        \def\childdoctmp{}
      \else
        \def\childdoctmp
        {
          \childdoctrue
          \includeonly{\childdocname}
          \def\childdocjob{#1}
          \def\jobname{#1}
        }
      \fi
      \expandafter
    \endgroup
    \childdoctmp
  \fi
}
%    \end{macrocode}

% \macro{\childdocof}
% The command |\childdocof| redirects
% compilation to the main file |#1|.
%    \begin{macrocode}
\newcommand{\childdocof}[1]
{
  \childdocdisable
  \childdoctrue
  \includeonly{\childdocname}
  \def\jobname{#1}
  \def\childdocjob{#1}
  \input{#1}
}
%    \end{macrocode}

% \macro{\childdocby}
% The command |\childdocby| ....
%    \begin{macrocode}
\newcommand{\childdocby}[2][]
{
  \childdocdisable
  \childdoctrue
  \childdocmanualtrue
  \if?#1?\else
    \def\jobname{#2}
  \fi
  \def\childdocjob{#2}
  \input{#2}
  \endinput
}
%    \end{macrocode}

% \macro{\childdocforward}
% The command |\childdocforward| redirects
% compilation to the main file or
% (if the optional argument is given) a child file.
% Parameters are set as if the main file
% or a child file starting with |\childdocof| was compiled.
% Then compilation is handed over to the main file:
%    \begin{macrocode}
\newcommand{\childdocforward}[2][]
{
  \begingroup
    \if?#1?
      \def\childdoctmp
      {
        \def\childdocname{#2}
        \def\childdocjob{#2}
        \def\jobname{#2}
        \input{#2}
        \endinput
      }
    \else
      \def\childdoctmp
      {
        \childdocdisable
        \def\childdocname{#2}
        \childdoctrue
        \includeonly{#2}
        \def\childdocjob{#1}
        \def\jobname{#1}
        \input{#1}
        \endinput
      }
    \fi
    \expandafter
  \endgroup
  \childdoctmp
}
%    \end{macrocode}

% \macro{\childdocforwardprefix}
% The command |\childdocforwardprefix| redirects
% compilation to the main or a child file by means of a pattern.
% The prefix |#1| in the current filename is replaced by |#2|
% and the suffix of the current filename is kept
% (it is assumed that the filename does not contain the substring `|~~~|'
% which is used as a delimiter).
% Compilation is handed over to the new file by |\childdocforward|:
%    \begin{macrocode}
\newcommand{\childdocforwardprefix}[3][]
{
  \begingroup
    \def\childdocextract #2##1~~~{\def\childdoctmp{\childdocforward[#1]{#3##1}}}
    \expandafter\childdocextract\childdocname~~~
    \expandafter
  \endgroup
  \childdoctmp
}
%    \end{macrocode}

% \macro{\childdoc}
% The deprecated macro |\childdoc| is a legacy version of |\childdocmain|:
%    \begin{macrocode}
\newcommand{\childdoc}{\childdocmain}
%    \end{macrocode}

% \macro{\childdocredirect}
% The deprecated macro |\childdocredirect| is a legacy version
% of |\childdocforward| and |\childdocforwardprefix|:
%    \begin{macrocode}
\newcommand{\childdocredirect}[2][]
{
  \begingroup
    \if?#1?
      \def\childdoctmp{\childdocforward{#2}}
    \else
      \def\childdoctmp{\childdocforwardprefix{#1}{#2}}
    \fi
    \expandafter
  \endgroup
  \childdoctmp
}
%    \end{macrocode}

%\iffalse
%</package>
%\fi
%
\endinput
\childdocforward[cdocsamp]{cdocsch1}"|\\
% |latex -jobname cdocscl2 \|\\
% |  "\def\version{final}% \iffalse
%
% childdoc.dtx Copyright (C) 2017-2018 Niklas Beisert
%
% This work may be distributed and/or modified under the
% conditions of the LaTeX Project Public License, either version 1.3
% of this license or (at your option) any later version.
% The latest version of this license is in
%   http://www.latex-project.org/lppl.txt
% and version 1.3 or later is part of all distributions of LaTeX
% version 2005/12/01 or later.
%
% This work has the LPPL maintenance status `maintained'.
%
% The Current Maintainer of this work is Niklas Beisert.
%
% This work consists of the files childdoc.dtx and childdoc.ins
% and the derived files childdoc.def and cdocsamp.tex with
% cdocsch1.tex, cdocsch2.tex, cdocsdrf.tex, cdocsfn1.tex, cdocsfn2.tex.
%
%<package>\ifdefined\childdocmain\endinput\fi
%<package>\ProvidesFile{childdoc.def}[2018/12/30 v2.0 child document driver]
%<samplemain>\ProvidesFile{cdocsamp.tex}[2018/12/30 v2.0 sample for childdoc]
%<*driver>
%\ProvidesFile{childdoc.drv}[2018/12/30 v2.0 childdoc reference manual file]
\PassOptionsToClass{10pt,a4paper}{article}
\documentclass{ltxdoc}

\usepackage[margin=35mm]{geometry}
\usepackage{hyperref}
\usepackage{hyperxmp}
\usepackage[usenames]{color}

\hypersetup{colorlinks=true}
\hypersetup{pdfstartview=FitH}
\hypersetup{pdfpagemode=UseNone}
\hypersetup{pdfsource={}}
\hypersetup{pdflang={en-UK}}
\hypersetup{pdfcopyright={Copyright 2017-2018 Niklas Beisert.
  This work may be distributed and/or modified under the
  conditions of the LaTeX Project Public License, either version 1.3
  of this license or (at your option) any later version.}}
\hypersetup{pdflicenseurl={http://www.latex-project.org/lppl.txt}}
\hypersetup{pdfcontactaddress={ETH Zurich, ITP, HIT K,
  Wolfgang-Pauli-Strasse 27}}
\hypersetup{pdfcontactpostcode={8093}}
\hypersetup{pdfcontactcity={Zurich}}
\hypersetup{pdfcontactcountry={Switzerland}}
\hypersetup{pdfcontactemail={nbeisert@itp.phys.ethz.ch}}
\hypersetup{pdfcontacturl={http://people.phys.ethz.ch/\xmptilde nbeisert/}}

\newcommand{\secref}[1]{\hyperref[#1]{section \ref*{#1}}}

\parskip1ex
\parindent0pt
\let\olditemize\itemize
\def\itemize{\olditemize\parskip0pt}

\begin{document}

\title{The \textsf{childdoc} Package}
\hypersetup{pdftitle={The childdoc Package}}
\author{Niklas Beisert\\[2ex]
  Institut f\"ur Theoretische Physik\\
  Eidgen\"ossische Technische Hochschule Z\"urich\\
  Wolfgang-Pauli-Strasse 27, 8093 Z\"urich, Switzerland\\[1ex]
  \href{mailto:nbeisert@itp.phys.ethz.ch}
  {\texttt{nbeisert@itp.phys.ethz.ch}}}
\hypersetup{pdfauthor={Niklas Beisert}}
\hypersetup{pdfsubject={Manual for the LaTeX2e Package childdoc}}
\date{30 December 2018, \textsf{v2.0}}
\maketitle

\begin{abstract}\noindent
\textsf{childdoc} is a \LaTeXe{} package
that enables the direct compilation
of document sections included by |\include|
to individual files.
\end{abstract}

\begingroup
\parskip0ex
\tableofcontents
\endgroup

%%%%%%%%%%%%%%%%%%%%%%%%%%%%%%%%%%%%%%%%%%%%%%%%%%%%%%%%%%%%%%%%%%%%%%%%%%%%%%%%
%%%%%%%%%%%%%%%%%%%%%%%%%%%%%%%%%%%%%%%%%%%%%%%%%%%%%%%%%%%%%%%%%%%%%%%%%%%%%%%%
\section{Introduction}

\LaTeX{} provides a mechanism to structure a large document (such as a book)
into a main file and several child files (containing the chapters)
using the |\include| command.
This mechanism is beneficial for documents
which span hundreds of pages in order to
make the source file(s) more manageable.
Moreover, compilation can be restricted to
selected child files by means of the |\includeonly| command.
The latter feature can be used to reduce the compilation time while editing
(this was significantly more useful in the earlier days of \LaTeX{})
or to generate a smaller document which is easier to navigate.
Another application of |\includeonly| is to generate
documents consisting of selected parts of the complete document.

However, there are a few drawbacks of the plain |\include| mechanism:
\begin{itemize}
\item
The child files cannot be compiled on their own,
they can only be compiled via the main file.
A naive editing environment
(such as a text editor with an option
to have the current file processed by \LaTeX)
may require one to switch to the main file before compiling;
attempting to compile the child file produces errors.
\item
The main file must be modified (each time)
to adjust the |\includeonly| command
to the present needs. This easily leaves the main file in a messy state.
\item
The generated document will always carry the filename
of the main document. This is inconvenient if
several child files are to be compiled and
to be kept for distribution.
\end{itemize}

The present package provides a simple interface
to make child files individually compilable by \LaTeX{}.
Compiling a child file then has the same effect as compiling
the main file with an |\includeonly| command
to select the appropriate child.
Moreover the generated document will carry the name of the child
rather than the main file.
This resolves all three above issues.

This feature is meant to make the editing of books,
thesis documents and lecture notes somewhat more convenient.
However, the package can also be used efficiently for
composing a series of documents (such as exercise sheets)
which are typically distributed individually.
It then assists the author in generating the individual documents
(potentially in different versions)
as well as a document containing the collected series.
Another application is in developing style files
or other kinds of included material
where compilation of the style file could redirect
to a sample or test file.

%%%%%%%%%%%%%%%%%%%%%%%%%%%%%%%%%%%%%%%%%%%%%%%%%%%%%%%%%%%%%%%%%%%%%%%%%%%%%%%%
%%%%%%%%%%%%%%%%%%%%%%%%%%%%%%%%%%%%%%%%%%%%%%%%%%%%%%%%%%%%%%%%%%%%%%%%%%%%%%%%
\section{Usage}

First of all, the package \textsf{childdoc} is \emph{not} a standard
\LaTeXe{} |.sty| style file! Therefore it needs to be invoked in
a non-standard way.

%%%%%%%%%%%%%%%%%%%%%%%%%%%%%%%%%%%%%%%%%%%%%%%%%%%%%%%%%%%%%%%%%%%%%%%%%%%%%%%%
\subsection{Included Files}
\label{sec:include}

%%%%%%%%%%%%%%%%%%%%%%%%%%%%%%%%%%%%%%%%
\DescribeMacro{\childdocmain}
To use the package, add the commands
\begin{center}
\begin{tabular}{l}
|\input{childdoc.def}|\\
|\childdocmain{}|\\
\end{tabular}
\end{center}
at the very top of the main \LaTeX{} file,
in particular \emph{before} the |\documentclass| statement!
The argument of |\childdocmain| should be left empty
(but it must be present).

%%%%%%%%%%%%%%%%%%%%%%%%%%%%%%%%%%%%%%%%
\DescribeMacro{\childdocof}
Furthermore, add the commands
\begin{center}
\begin{tabular}{l}
|\input{childdoc.def}|\\
|\childdocof{|\textit{main}|}|\\
\end{tabular}
\end{center}
at the top of every child file \textit{child}
which is included by |\include{|\textit{child}|}|
from within the main file
(or at least for those files to be compiled individually).
The argument \textit{main} must be the filename of the main file.

There are a couple of
considerations in setting up the main and child documents:

%%%%%%%%%%%%%%%%%%%%%%%%%%%%%%%%%%%%%%%%
\paragraph{Restrictions.}

Please note the following restrictions:
\begin{itemize}
\item
|\childdocmain| must be called with one argument \textit{main}
to ensure compatibility with earlier version of the package.
It must either be empty (|\childdocmain{}|)
or precisely match the filename of the main file in which it is specified.
See \secref{sec:detection} for further information.
\item
The filename \textit{main} must be specified without the |.tex| extension.
\item
The filename \textit{main} is case sensitive
(even in case-insensitive file systems)
due to internal string comparison.
\item
The argument \textit{main} should be fully expanded, it cannot be a macro.
\item
Subdirectories and special characters should be avoided in filenames.
\item
The command |\childdocmain{|\textit{main}|}| must be followed by a whitespace.
It should not be followed immediately by another command
or by a comment mark `|%|'.
This is because the \TeX{} parser reads the token immediately following
the argument of |\childdocmain| and puts it
at the beginning of every child section;
however, a white\-space is ignored.
\end{itemize}

%%%%%%%%%%%%%%%%%%%%%%%%%%%%%%%%%%%%%%%%
\paragraph{Content of Main File.}

It is advisable to place all content in the child files included by |\include|.
Any output contained in the main file will appear in all child documents
unless suppressed manually;
it cannot be suppressed automatically by the |\includeonly| directive
and thus should normally be avoided.
A method to include some content in the main file
by means of conditional processing is described in \secref{sec:conditional}.

%%%%%%%%%%%%%%%%%%%%%%%%%%%%%%%%%%%%%%%%
\paragraph{Page Numbering.}

When only a part of the document is compiled,
the appropriate numbering of pages
(as well as other status parameters)
is determined from the |.aux| files.
The latter contain information from previous passes.
However this information needs to propagate through
all intermediate child documents.
Therefore the page numbering in child documents may well
be inconsistent until the complete document is compiled at least once.

A useful (if unconventional) way to always ensure a consistent
page numbering is to restart the numbering in each child document
and denote the pages by `\textit{child}|.|\textit{page}'
where \textit{child} represents the chapter/section number of the child file.
This can be achieved by the command
|\numberwithin{page}{|\textit{child}|}|
of the \textsf{amsmath} package
where \textit{child} can be |chapter| or |section|
depending on the chosen structuring.
Alternatively, one can modify the macro |\thepage| appropriately
and reset the counter |page| at the start of each child file.

%%%%%%%%%%%%%%%%%%%%%%%%%%%%%%%%%%%%%%%%%%%%%%%%%%%%%%%%%%%%%%%%%%%%%%%%%%%%%%%%
\subsection{Conditional Processing}
\label{sec:conditional}

The package provides a mechanism to compile different versions
of a document. To customise the versions further some conditional processing
can come in handy to distinguish which version is being compiled.
The package provides two macros to describe the compilation context:

%%%%%%%%%%%%%%%%%%%%%%%%%%%%%%%%%%%%%%%%
\DescribeMacro{\ifchilddoc}
The conditional |\ifchilddoc| distinguishes between the compilation of
child documents and the main document:
%
\begin{center}
|\ifchilddoc |\textit{child-code}| |[|\||else |\textit{main-code}]| \||fi|
\end{center}

%%%%%%%%%%%%%%%%%%%%%%%%%%%%%%%%%%%%%%%%
\DescribeMacro{\childdocname}
\DescribeMacro{\childdocjob}
The macro |\childdocname| contains the filename (without extension)
of the main or child file being processed.
Note that |\childdocjob| will always contain the name of the main file.

%%%%%%%%%%%%%%%%%%%%%%%%%%%%%%%%%%%%%%%%
\paragraph{Title Page.}

Conditional processing can be used to include a title or banner page
in the main document when proper precautions are taken.
Importantly, the code in the main file should ensure that the page counter
(as well as other status parameters which are stored in the |.aux| files)
takes the same value after the conditional processing.
Otherwise the page numbers may take divergent values
depending on which part is compiled.

For example, a title page could be declared by:
%
\begin{center}
\begin{tabular}{l}
|\ifchilddoc\||else|\\
|\addtocounter{page}{-1}|\\
\textit{code for title page}\\
|\newpage|\\
|\||fi|
\end{tabular}
\end{center}
%
A banner page for the child documents can be generated by:
%
\begin{center}
\begin{tabular}{l}
|\ifchilddoc|\\
|\addtocounter{page}{-1}|\\
\textit{code for banner page}\\
|\newpage|\\
|\||fi|
\end{tabular}
\end{center}
%
Here one could write a message such as:
\begin{center}
|This is the part \childdocname{} of \childdocjob{}.|
\end{center}

%%%%%%%%%%%%%%%%%%%%%%%%%%%%%%%%%%%%%%%%%%%%%%%%%%%%%%%%%%%%%%%%%%%%%%%%%%%%%%%%
\subsection{Flags}
\label{sec:flags}

The package makes it easy to generate different versions
of the main or child documents.
To this end compilation flags can be defined
and assigned different default values.
They will be particularly useful in conjunction
with the forwarding mechanism described in \secref{sec:forward}.

For example, it may be useful to have a flag |\version|
which can be set to |draft| or |final|.
The document source will contain some conditional code
depending on the value of |\version|.
Suppose further, the flag should default to |final| for the main file
and to |draft| for child files
which is a natural assignment for editing the document.
This is achieved by placing the following code
in the preamble of the main document
(below the |\childdocmain| directive):
%
\begin{center}
\begin{tabular}{l}
|\ifchilddoc|\\
|\providecommand{\version}{draft}|\\
|\||else|\\
|\providecommand{\version}{final}|\\
|\||fi|
\end{tabular}
\end{center}
%
The definition by |\providecommand| makes sure
that previous definitions are not overwritten.
Further statements |\providecommand{\version}{...}|
can thus be added before the above code to override it.

For the main file, one might add a line
(between |\childdocmain| and the above block)
%
\begin{center}
|%\ifchilddoc\||else\providecommand{\version}{draft}\||fi|
\end{center}
%
which can be uncommented to produce a draft version.
Likewise one can add a line to the very top of a child file
(above the |\childdocof{|\textit{main}|}| directive)
%
\begin{center}
|%\providecommand{\version}{final}|
\end{center}
%
which can be uncommented to produce the final version of this child document.

%%%%%%%%%%%%%%%%%%%%%%%%%%%%%%%%%%%%%%%%%%%%%%%%%%%%%%%%%%%%%%%%%%%%%%%%%%%%%%%%
\subsection{Forwarding}
\label{sec:forward}

Different versions of the main or child documents
using compilation flags as described in \secref{sec:flags}
can be (permanently) stored in different files
for convenient compilation, viewing and distribution.
To this end, the package defines a command
to pass on compilation to a different file:

%%%%%%%%%%%%%%%%%%%%%%%%%%%%%%%%%%%%%%%%
\DescribeMacro{\childdocforward}
The command |\childdocforward| redirects processing to
another source file:
%
\begin{center}
\begin{tabular}{l}
|\input{childdoc.def}|\\
|\childdocforward[|\textit{main}|]{|\textit{dest}|}|\\
\end{tabular}
\end{center}
%
The argument \textit{dest} is the destination file
(without extension).
It should be the main file or one of the child files.
Note that further \textsf{childdoc} directives
such as |\childdocof| and |\childdocforward|
in the indicated file will be processed in this form.
The optional argument \textit{main}
passes on directly to the main file \textit{main}
while pretending to compile the child \textit{dest}.
This form behaves as if \textit{dest}
issues |\childdocof{|\textit{main}|}| right away,
and no further \textsf{childdoc} directives will be processed.

%%%%%%%%%%%%%%%%%%%%%%%%%%%%%%%%%%%%%%%%
\DescribeMacro{\...prefix}
In the alternative form |\childdocforwardprefix|,
%
\begin{center}
\begin{tabular}{l}
|\input{childdoc.def}|\\
|\childdocforwardprefix[|\textit{main}|]{|\textit{prefix}|}{|\textit{dest}|}|
\end{tabular}
\end{center}
%
the destination file is determined by a pattern
depending on the current file:
To make this work, the current file must be called
`{\textit{prefix}\hspace{0.2em}\textit{suffix}}'
with \textit{prefix} matching precisely the argument.
Processing is then passed on to the file
`{\textit{dest}\hspace{0.2em}\textit{suffix}}'.
Surely, the same effect is achieved by
directly specifying the
argument `{\textit{dest}\hspace{0.2em}\textit{suffix}}'
in the first form.
However, that requires to set up a different file
for each child. With the alternative form of the command
all these files can have exactly the same content
which simplifies setting them up and maintaining them.

For example, the following file |draft.tex|
with a compilation flag |\version| as described in \secref{sec:flags}
compiles the main document as a draft:
%
\begin{center}
\begin{tabular}{l}
|\def\version{draft}|\\
|\input{childdoc.def}|\\
|\childdocforward{|\textit{main}|}|
\end{tabular}
\end{center}
%
Likewise, the following files |final|\textit{nn}|.tex|
compile the final version of the child document
|child|\textit{nn}|.tex|:
%
\begin{center}
\begin{tabular}{l}
|\def\version{final}|\\
|\input{childdoc.def}|\\
|\childdocforwardprefix{final}{child}|
\end{tabular}
\end{center}
%

Note that when several versions of a main file and/or of each child file
are to be generated, it may be convenient to set up a |Makefile| or
shell script to automatise the process.

%%%%%%%%%%%%%%%%%%%%%%%%%%%%%%%%%%%%%%%%%%%%%%%%%%%%%%%%%%%%%%%%%%%%%%%%%%%%%%%%
\subsection{Command Line Processing}
\label{sec:commandline}

The effect of redirection files can also be achieved by invoking
the \LaTeX{} compiler with a more elaborate command line.
Most conveniently this should be done as part
of a shell script or a |Makefile|.

When using \textsf{childdoc} in the main file, the following
command lines effectively perform a redirection
(note that depending on the shell being used,
backslashes may have to be doubled: `|\|' $\to$ `|\\|'):
%
\begin{center}
|... -jobname "|\textit{target}|" |\\|"|[\textit{flags}]%
|\input{childdoc.def}\childdocforward[|\textit{main}|]{|\textit{dest}|}"|
\end{center}
%
Here \textit{target} is the name of the output file,
\textit{main} is the name of the main file
and \textit{dest} is the name of the main or child file to be processed
(all filenames without extensions).
The optional argument \textit{main} can be omitted
if \textit{main} matches \textit{dest}.
Optionally, compilation \textit{flags} can be defined via |\def| commands.
This command line makes the \TeX{} engine believe
it is compiling the file \textit{target}
whose content is specified as the latter parameter.
The provided code then forwards the processing to
\textit{main} or \textit{dest} as described in \secref{sec:forward}.

%%%%%%%%%%%%%%%%%%%%%%%%%%%%%%%%%%%%%%%%%%%%%%%%%%%%%%%%%%%%%%%%%%%%%%%%%%%%%%%%
\subsection{Include by Input}
\label{sec:input}

Including child documents by |\include| has some restrictions by design.
Most notably, the content of a child document always occupies
its own set of pages; pages cannot be shared between child documents.
Usually, this behaviour makes perfect sense
because each child document contain an essential part of the document.
However, in some situations it may be desirable to compose
a document from a collection of parts
without having mandatory page breaks between then.
For this case, the package
provides a mechanism to include parts
by |\input| which can also be processed individually.
However, by construction this mechanism
requires manual handling of the content to be output.

%%%%%%%%%%%%%%%%%%%%%%%%%%%%%%%%%%%%%%%%
\DescribeMacro{\ifchilddocmanual}
The main file should be prepared as usual, see \secref{sec:include}.
However, the document body must make a distinction
between processing of an individual part and of the main document, e.g.:
%
\begin{center}
\begin{tabular}{l}
|\ifchilddocmanual|\\
|\input{\childdocname}|\\
|\||else|\\
\textit{document body with }|\input{|\textit{part}|}|\\
|\||fi|
\end{tabular}
\end{center}
%
The conditional |\ifchilddocmanual| is true whenever
a part to be included by |\input| is being compiled,
and the name of the part is stored in |\childdocname|.

%%%%%%%%%%%%%%%%%%%%%%%%%%%%%%%%%%%%%%%%
\DescribeMacro{\childdocby}
Each part to be included by |\input| should start with:
%
\begin{center}
\begin{tabular}{l}
|\input{childdoc.def}|\\
|\childdocby{|\textit{main}|}|\\
\end{tabular}
\end{center}
%
The directive |\childdocby| is similar to |\childdocof|
described in \secref{sec:include},
but the subsequent selection of content must be done manually.
To that end, both |\ifchilddoc| and |\ifchilddocmanual|
will be true upon processing of a part,
and the name of the part is stored in |\childdocname|.
Note that |\jobname| will be set to the filename of the current part
so that each part receives an individual |.aux| file
that does not interfere with the |.aux| file(s) of the main document.
This behaviour can be altered by the alternative form
|\childdocby[*]{|\textit{main}|}| (with a non-empty optional argument)
which uses the |.aux| file of the main document
by setting |\jobname| to \textit{main}.

%%%%%%%%%%%%%%%%%%%%%%%%%%%%%%%%%%%%%%%%%%%%%%%%%%%%%%%%%%%%%%%%%%%%%%%%%%%%%%%%
\subsection{Driver Development}
\label{sec:driver}

The \textsf{childdoc} mechanism can also be use for the development
of definition files such as \LaTeX{} styles or classes.
This case differs from the above setup with multiple parts
included by |\include| in that no |\includeonly| should be invoked.
This can be achieved by starting the include file
(before |\ProvidesPackage|) with:
%
\begin{center}
\begin{tabular}{l}
|\input{childdoc.def}|\\
|\childdocforward{|\textit{main}|}|\\
\end{tabular}
\end{center}
%
or alternatively with:
%
\begin{center}
\begin{tabular}{l}
|\input{childdoc.def}|\\
|\childdocby{|\textit{main}|}|\\
\end{tabular}
\end{center}
%
Both forms have slightly different effects as described above.
The main file is prepared as usual, see \secref{sec:include}.

%%%%%%%%%%%%%%%%%%%%%%%%%%%%%%%%%%%%%%%%%%%%%%%%%%%%%%%%%%%%%%%%%%%%%%%%%%%%%%%%
\subsection{Legacy Detection}
\label{sec:detection}

The directive |\childdocmain| in the main file can detect
whether the complete document or merely a child is to be compiled
even without using the directive |\childdocof|.
This method is deprecated because it is less robust
and there is no compelling reason to use it;
it is merely provided for backward compatibility
and it may be removed in future versions.

If the detection mechanism is to be used,
it is mandatory to correctly specify
the filename of the main file as the argument of |\childdocmain|:
%
\begin{center}
\begin{tabular}{l}
|\input{childdoc.def}|\\
|\childdocmain{|\textit{main}|}|\\
\end{tabular}
\end{center}
%
If |\jobname| does not match the argument \textit{main} of |\childdocmain|,
it is assumed that |\jobname| points to the child file to be compiled.
When using |\childdocmain| with the main file specified as argument,
it suffices to start a child file
with just |\input{|\textit{main}|}|
without loading of the package and using |\childdocof|.
If instead all processing is done
with the appropriate \textsf{childdoc} directives,
the argument of \textit{main} of |\childdocmain| can be empty.

An alternative version of the command line processing described
in \secref{sec:commandline} using the detection mechanism reads:
%
\begin{center}
|... -jobname "|\textit{target}|" "|[\textit{flags}]%
[|\def\jobname{|\textit{dest}|}|]|\input{|\textit{main}|}"|
\end{center}

%%%%%%%%%%%%%%%%%%%%%%%%%%%%%%%%%%%%%%%%%%%%%%%%%%%%%%%%%%%%%%%%%%%%%%%%%%%%%%%%
\subsection{Manual Code}
\label{sec:manual}

In case one cannot be certain whether the definitions file |childdoc.def|
is installed on the target \TeX{} distribution
and one prefers not to ship it,
it is conceivable to paste a few relevant commands into the sources.

To that end, drop all statements |\input{childdoc.def}|
and perform the replacements as outlined below.
Instead of |\childdocmain{|\textit{main}|}| add the following code
to the top of the main file:
%
\begin{center}
\begin{tabular}{l}
|\||ifdefined\childdocname\endinput\||fi\newif\ifchilddoc|\\
|\edef\childdocname{\scantokens\expandafter{\jobname\noexpand}}|\\
|\def\childdocmain{|\textit{main}|}\||ifx\childdocmain\childdocname\||else|\\
|\childdoctrue\includeonly{\childdocname}\let\jobname\childdocmain\||fi|\\
\end{tabular}
\end{center}
%
Instead of |\childdocof{|\textit{main}|}| just include the main file
at the top of each child file:
%
\begin{center}
|\input{|\textit{main}|}|
\end{center}
%
A simple redirection |\childdocforward{|\textit{dest}|}| is achieved by:
%
\begin{center}
|\def\jobname{|\textit{dest}|}\input{\jobname}|
\end{center}
%
The redirection with prefix
|\childdocforwardprefix[|\textit{prefix}|]{|\textit{dest}|}|
is accomplished by:
%
\begin{center}
\begin{tabular}{l}
|{\edef\jobname{\scantokens\expandafter{\jobname\noexpand}}|\\
|\def\redirectjob |\textit{prefix}|#1~~~{\gdef\jobname{|\textit{dest}|#1}}|\\
|\expandafter\redirectjob\jobname~~~}\input{\jobname}|
\end{tabular}
\end{center}

In an alternative approach,
child documents can be compiled by a specific command line
without additional code or specific definitions:
%
\begin{center}
|... -jobname "|\textit{target}|" "|[\textit{flags}]%
|\includeonly{|\textit{dest}|}\input{|\textit{main}|}"|
\end{center}
%

%%%%%%%%%%%%%%%%%%%%%%%%%%%%%%%%%%%%%%%%%%%%%%%%%%%%%%%%%%%%%%%%%%%%%%%%%%%%%%%%
%%%%%%%%%%%%%%%%%%%%%%%%%%%%%%%%%%%%%%%%%%%%%%%%%%%%%%%%%%%%%%%%%%%%%%%%%%%%%%%%
\section{Information}

%%%%%%%%%%%%%%%%%%%%%%%%%%%%%%%%%%%%%%%%%%%%%%%%%%%%%%%%%%%%%%%%%%%%%%%%%%%%%%%%
\subsection{Copyright}

Copyright \copyright{} 2017--2018 Niklas Beisert

This work may be distributed and/or modified under the
conditions of the \LaTeX{} Project Public License, either version 1.3
of this license or (at your option) any later version.
The latest version of this license is in
  \url{http://www.latex-project.org/lppl.txt}
and version 1.3 or later is part of all distributions of \LaTeX{}
version 2005/12/01 or later.

This work has the LPPL maintenance status `maintained'.

The Current Maintainer of this work is Niklas Beisert.

This work consists of the files |README.txt|, |childdoc.ins| and |childdoc.dtx|
as well as the derived files |childdoc.def|, |cdocsamp.tex|
with |cdocsch1.tex|, |cdocsch2.tex|, |cdocspt3.tex|, |cdocspt4.tex|,
|cdocsdrf.tex|, |cdocsfn1.tex|, |cdocsfn2.tex|
as well as |childdoc.pdf|.

%%%%%%%%%%%%%%%%%%%%%%%%%%%%%%%%%%%%%%%%%%%%%%%%%%%%%%%%%%%%%%%%%%%%%%%%%%%%%%%%
\subsection{Files and Installation}

The package consists of the files:
%
\begin{center}
\begin{tabular}{ll}
    |README.txt|   & readme file \\
    |childdoc.ins| & installation file \\
    |childdoc.dtx| & source file \\
    |childdoc.def| & definition file \\
    |cdocsamp.tex| & sample main file \\
    |cdocsch1.tex| & sample include file \\
    |cdocsch2.tex| & sample include file \\
    |cdocspt3.tex| & sample part file \\
    |cdocspt4.tex| & sample part file \\
    |cdocsdrf.tex| & sample redirection file \\
    |cdocsfn1.tex| & sample redirection file \\
    |cdocsfn2.tex| & sample redirection file \\
    |childdoc.pdf| & manual
\end{tabular}
\end{center}
%
The distribution consists of the files
|README.txt|, |childdoc.ins| and |childdoc.dtx|.
%
\begin{itemize}
\item
Run (pdf)\LaTeX{} on |childdoc.dtx|
to compile the manual |childdoc.pdf| (this file).
\item
Run \LaTeX{} on |childdoc.ins| to create the definitions file |childdoc.def|
and the sample |cdocsamp.tex| with include files
|cdocsch1.tex|, |cdocsch2.tex|, |cdocspt3.tex|, |cdocspt4.tex|,
|cdocsdrf.tex|, |cdocsfn1.tex|, |cdocsfn2.tex|.
Then copy the file |childdoc.def| to an appropriate directory of your \LaTeX{}
distribution, e.g.\ \textit{texmf-root}|/tex/latex/childdoc|.
\end{itemize}

%%%%%%%%%%%%%%%%%%%%%%%%%%%%%%%%%%%%%%%%%%%%%%%%%%%%%%%%%%%%%%%%%%%%%%%%%%%%%%%%
\subsection{Related CTAN Packages}

There are several other packages which offer a similar functionality:
%
\begin{itemize}
\item
The packages
\href{http://ctan.org/pkg/docmute}{\textsf{docmute}},
\href{http://ctan.org/pkg/includex}{\textsf{includex}} and
\href{http://ctan.org/pkg/standalone}{\textsf{standalone}}
provide commands to include only the document body of
a child file thus allowing both files to be compiled individually.
\item
The packages \href{http://ctan.org/pkg/subdocs}{\textsf{subdocs}}
and \href{http://ctan.org/pkg/subfiles}{\textsf{subfiles}}
provide structures in which the main and child documents can be
encapsulated and allowing them to be compiled individually.
The inclusion mechanism is different from the conventional |\include|.
\item
The package \href{http://ctan.org/pkg/combine}{\textsf{combine}}
is an elaborate solution to combine several documents into one.
\end{itemize}
%
See also the CTAN topic \href{http://ctan.org/topic/subdocs}{\textsf{subdocs}}
for further related packages.
The present package differs from the above solutions in that
a document structure constructed with the conventional |\include| mechanism
just needs two extra commands at the top of every file
such that all constituent files can be compiled individually.

%%%%%%%%%%%%%%%%%%%%%%%%%%%%%%%%%%%%%%%%%%%%%%%%%%%%%%%%%%%%%%%%%%%%%%%%%%%%%%%%
%\subsection{Feature Suggestions}
%
%The following is a list of features which may be useful for future
%versions of this package:
%%
%\begin{itemize}
%\item
%\ldots
%\end{itemize}

%%%%%%%%%%%%%%%%%%%%%%%%%%%%%%%%%%%%%%%%%%%%%%%%%%%%%%%%%%%%%%%%%%%%%%%%%%%%%%%%
\subsection{Revision History}

%%%%%%%%%%%%%%%%%%%%%%%%%%%%%%%%%%%%%%%%
\paragraph{v2.0:} 2018/12/30

\begin{itemize}
\item
immediate forward processing
\item
added |\childdocby| mechanism
\item
manual restructured
\end{itemize}

%%%%%%%%%%%%%%%%%%%%%%%%%%%%%%%%%%%%%%%%
\paragraph{v1.6:} 2018/01/17

\begin{itemize}
\item
application for development of include files
\item
corrections to manual
\end{itemize}

%%%%%%%%%%%%%%%%%%%%%%%%%%%%%%%%%%%%%%%%
\paragraph{v1.5:} 2017/05/21

\begin{itemize}
\item
more complete structuring introduced
\item
|\childdocof| introduced
\item
|\childdoc| renamed to |\childdocmain|
\item
|\childredirect| renamed to |\childdocforward| and |\childdocforwardprefix|
and functionality expanded
\end{itemize}

%%%%%%%%%%%%%%%%%%%%%%%%%%%%%%%%%%%%%%%%
\paragraph{v1.0:} 2017/04/27

\begin{itemize}
\item
manual and install package
\item
first version published on CTAN
\end{itemize}

%%%%%%%%%%%%%%%%%%%%%%%%%%%%%%%%%%%%%%%%
\paragraph{v0.6:} 2017/04/26

\begin{itemize}
\item
redirection mechanism added
\end{itemize}

%%%%%%%%%%%%%%%%%%%%%%%%%%%%%%%%%%%%%%%%
\paragraph{v0.5:} 2017/04/26

\begin{itemize}
\item
functionality in definition file
\end{itemize}


%%%%%%%%%%%%%%%%%%%%%%%%%%%%%%%%%%%%%%%%%%%%%%%%%%%%%%%%%%%%%%%%%%%%%%%%%%%%%%%%
%%%%%%%%%%%%%%%%%%%%%%%%%%%%%%%%%%%%%%%%%%%%%%%%%%%%%%%%%%%%%%%%%%%%%%%%%%%%%%%%
%%%%%%%%%%%%%%%%%%%%%%%%%%%%%%%%%%%%%%%%%%%%%%%%%%%%%%%%%%%%%%%%%%%%%%%%%%%%%%%%
\appendix

\settowidth\MacroIndent{\rmfamily\scriptsize 000\ }

 \DocInput{childdoc.dtx}

\end{document}
%</driver>
% \fi
%
% %%%%%%%%%%%%%%%%%%%%%%%%%%%%%%%%%%%%%%%%%%%%%%%%%%%%%%%%%%%%%%%%%%%%%%%%%%%%%%
% %%%%%%%%%%%%%%%%%%%%%%%%%%%%%%%%%%%%%%%%%%%%%%%%%%%%%%%%%%%%%%%%%%%%%%%%%%%%%%
% \section{Sample}
%\iffalse
%<*samplemain>
%\fi
%
% The following presents a sample document
% with two chapters, two parts, a title page,
% a compile flag as well as three forwarding files to set the flag.
% It consists of eight |.tex| files:
% \begin{center}
% \begin{tabular}{ll}
% |cdocsamp.tex|&main file\\
% |cdocsch1.tex|&include file for chapter 1\\
% |cdocsch2.tex|&include file for chapter 2\\
% |cdocspt3.tex|&include file for part 3\\
% |cdocspt4.tex|&include file for part 4\\
% |cdocsdrf.tex|&forwarding file for main file in draft mode\\
% |cdocsfi1.tex|&forwarding file for final version of chapter 1\\
% |cdocsfi2.tex|&forwarding file for final version of chapter 2\\
% \end{tabular}
% \end{center}
% Each of the eight files can be compiled directly by the \LaTeX{} compiler.
%
% %%%%%%%%%%%%%%%%%%%%%%%%%%%%%%%%%%%%%%
% \paragraph{Main File.}
%
% The main file is called |cdocsamp.tex|.
%
% Load the \textsf{childdoc} definitions and
% declare the filename for the main document:
%    \begin{macrocode}
\input{childdoc.def}
\childdocmain{}
%    \end{macrocode}

% Optional override for |\version| flag:
%    \begin{macrocode}
%%\ifchilddoc\else\providecommand{\version}{draft}\fi
%    \end{macrocode}

% Define the default values for the |\version| flag
% (|final| for the main file and |draft| for childs):
%    \begin{macrocode}
\ifchilddoc
\providecommand{\version}{draft}
\else
\providecommand{\version}{final}
\fi
%    \end{macrocode}

% Load the standard document class:
%    \begin{macrocode}
\documentclass[12pt]{article}
%    \end{macrocode}

% Start the document body:
%    \begin{macrocode}
\begin{document}
%    \end{macrocode}

% Declare a title page.
% Print title, part of document being processed and version flag:
%    \begin{macrocode}
\addtocounter{page}{-1}
\begin{center}
{\LARGE\bfseries{}childdoc example\par}
\vspace{1cm}
\ifchilddoc
\ifchilddocmanual part\else chapter\fi:
`\childdocname' of `\childdocjob'\par
\else
main document: `\childdocjob'\par
\fi
version: \version\par
\end{center}
\newpage
%    \end{macrocode}

% Manually include selected file,
% otherwise process as usual:
%    \begin{macrocode}
\ifchilddocmanual
\section*{part `\childdocname'}
\input{\childdocname}
\else
%    \end{macrocode}

% Include the two chapters:
%    \begin{macrocode}
\include{cdocsch1}
\include{cdocsch2}
%    \end{macrocode}

% Include the two parts unless only chapters should be displayed:
%    \begin{macrocode}
\ifchilddoc\else
\section{part three}
\input{cdocspt3}
\section{part four}
\input{cdocspt4}
\fi
%    \end{macrocode}

% Process as usual until here:
%    \begin{macrocode}
\fi
%    \end{macrocode}

% End of document body:
%    \begin{macrocode}
\end{document}
%    \end{macrocode}
%\iffalse
%</samplemain>
%\fi
%
% %%%%%%%%%%%%%%%%%%%%%%%%%%%%%%%%%%%%%%
% \paragraph{Chapter Include Files.}
%
% The include files are called |cdocsch1.tex| and |cdocsch2.tex|.
%
%\iffalse
%<*samplechap1|samplechap2>
%\fi

% Optional override for |\version| flag:
%    \begin{macrocode}
%%\providecommand{\version}{final}
%    \end{macrocode}

% Include the main document:
%    \begin{macrocode}
\input{childdoc.def}
\childdocof{cdocsamp}
%    \end{macrocode}

%\iffalse
%</samplechap1|samplechap2>
%\fi
%
%\iffalse
%<*samplechap1>
%\fi
% Some text for chapter 1:
%    \begin{macrocode}
\section{one}
some text in chapter one
%    \end{macrocode}

%\iffalse
%</samplechap1>
%\fi
% Some text for chapter 2:
%\iffalse
%<*samplechap2>
%\fi
%    \begin{macrocode}
\section{two}
more text in chapter two
%    \end{macrocode}

%\iffalse
%</samplechap2>
%\fi
%
% %%%%%%%%%%%%%%%%%%%%%%%%%%%%%%%%%%%%%%
% \paragraph{Part Include Files.}
%
% The include files are called |cdocspt3.tex| and |cdocspt4.tex|.
%
%\iffalse
%<*samplepart3|samplepart4>
%\fi

% Optional override for |\version| flag:
%    \begin{macrocode}
%%\providecommand{\version}{final}
%    \end{macrocode}

% Include the main document:
%    \begin{macrocode}
\input{childdoc.def}
\childdocby{cdocsamp}
%    \end{macrocode}

%\iffalse
%</samplepart3|samplepart4>
%\fi
%
%\iffalse
%<*samplepart3>
%\fi
% Some text for part 3:
%    \begin{macrocode}
some text in part three
%    \end{macrocode}

%\iffalse
%</samplepart3>
%\fi
% Some text for part 4:
%\iffalse
%<*samplepart4>
%\fi
%    \begin{macrocode}
more text in part four
%    \end{macrocode}

%\iffalse
%</samplepart4>
%\fi
%
% %%%%%%%%%%%%%%%%%%%%%%%%%%%%%%%%%%%%%%
% \paragraph{Forwarding for a Complete Draft.}
%
% The following forwarding file |cdocsdrf.tex|
% compiles the main document in draft mode:
%\iffalse
%<*sampledraft>
%\fi
%    \begin{macrocode}
\def\version{draft}
\input{childdoc.def}
\childdocforward{cdocsamp}
%    \end{macrocode}

%\iffalse
%</sampledraft>
%\fi
%
% %%%%%%%%%%%%%%%%%%%%%%%%%%%%%%%%%%%%%%
% \paragraph{Forwarding for Final Version of the Chapters.}
%
% The following forwarding files |cdocsfn1.tex| and |cdocsfn2.tex|
% (with identical content)
% compile the final versions of the child documents
% |cdocsch1.tex| and |cdocsch2.tex|, respectively:
%\iffalse
%<*samplefinal>
%\fi
%    \begin{macrocode}
\def\version{final}
\input{childdoc.def}
\childdocforwardprefix[cdocsamp]{cdocsfn}{cdocsch}
%    \end{macrocode}

%\iffalse
%</samplefinal>
%\fi
%
% %%%%%%%%%%%%%%%%%%%%%%%%%%%%%%%%%%%%%%
% \paragraph{Command Line Processing.}
%
% The following three command lines generate the output files
% |cdocscld|, |cdocscl1| and |cdocscl2|
% which should be identical to
% |cdocsdrf|, |cdocsch1| and |cdocsfn2|, respectively:
% \begin{center}
% \begin{tabular}{l}
% |latex -jobname cdocscld \|\\
% |  "\def\version{draft}\input{childdoc.def}\childdocforward{cdocsamp}"|\\
% |latex -jobname cdocscl1 \|\\
% |  "\input{childdoc.def}\childdocforward[cdocsamp]{cdocsch1}"|\\
% |latex -jobname cdocscl2 \|\\
% |  "\def\version{final}\input{childdoc.def}\childdocforward{cdocsch2}"|
% \end{tabular}
% \end{center}
% Note that the trailing backslash on each first line
% merely continues the input to the second line
% (for convenient cut ant paste).
% Furthermore, the command |latex| can be replaced by any
% of its alternative versions such as |pdflatex|.
%
% %%%%%%%%%%%%%%%%%%%%%%%%%%%%%%%%%%%%%%%%%%%%%%%%%%%%%%%%%%%%%%%%%%%%%%%%%%%%%%
% %%%%%%%%%%%%%%%%%%%%%%%%%%%%%%%%%%%%%%%%%%%%%%%%%%%%%%%%%%%%%%%%%%%%%%%%%%%%%%
% \section{Implementation}
%\iffalse
%<*package>
%\fi
%
% This section describes the definitions file |childdoc.def|.

% The definitions cannot be loaded using |\usepackage| or |\RequirePackage|
% which has a mechanism to prevent loading a style file more than once.
% When loading the definitions by means of |\input|
% multiple instances have to be prevented manually:
%\iffalse
%This code needs to be before the `\ProvidesFile' directive
%which is defined at the beginning of this file.
%Therefore it is also placed there and commented out here.
%</package>
%<*discard>
%\fi
%    \begin{macrocode}
\ifdefined\childdocmain\endinput\fi
%    \end{macrocode}
%\iffalse
%</discard>
%<*package>
%\fi
%
% \macro{\ifchilddoc}
% \macro{\ifchilddocmanual}
% The conditional |\ifchilddoc| tells whether a
% child (true) or main (false) document is being compiled.
% The conditional |\ifchilddocmanual| tells whether
% the |\includeonly| mechanism is used (false) or
% the selection of child files must be performed manually (true).
% The definitions initialise to false:
%    \begin{macrocode}
\newif\ifchilddoc
\newif\ifchilddocmanual
%    \end{macrocode}

% \macro{\childdocname}
% \macro{\childdocjob}
% The macro |\childdocname| stores the name of the main document
% to be compiled. The macro |\childdocjob| stores the name of
% the document on which the \LaTeX{} compiler was originally invoked.
% The content of |\jobname| cannot be compared
% to filenames specified in the source due to different catcodes.
% The following code rescans |\jobname|, stores the result
% in |\childdocname| and saves a copy in |\childdocjob|:
%    \begin{macrocode}
\edef\childdocname{\scantokens\expandafter{\jobname\noexpand}}
\let\childdocjob\childdocname
%    \end{macrocode}

% \macro{\childdocdisable}
% The macro |\childdocdisable| prevents the main file
% from being processed more than once.
% At this stage, the main document command |\childdocmain|
% is assumed to be called once again where it should do nothing.
% Any subsequent call to it should prevent
% a secondary processing of the main document
% It overwrites the forwarding commands
% |\childdocof| and |\childdocforward|
% with empty macros to prevent further inclusions of the main document:
%    \begin{macrocode}
\newcommand{\childdocdisable}
{
  \renewcommand{\childdocmain}[1]{\renewcommand{\childdocmain}[1]{\endinput}}
  \renewcommand{\childdocof}[1]{}
  \renewcommand{\childdocby}[2][]{}
  \renewcommand{\childdocforward}[2][]{}
  \renewcommand{\childdocdisable}{}
}
%    \end{macrocode}

% \macro{\childdocmain}
% The macro |\childdocmain| is to be called at the top of the main file
% with nothing or the main filename (without extension) as argument.
% First, it breaks loops.
% If the argument is not empty and does not match |\childdocname|
% (which is set by the first inclusion of |childdoc.def|),
% |\ifchilddoc| is set to true, |\includeonly| is applied to the child file
% and |\jobname| is set to the main file
% (for proper handling of |.aux| files):
%    \begin{macrocode}
\newcommand{\childdocmain}[1]
{
  \childdocdisable\childdocmain{}
  \if?#1?\else
    \begingroup
      \def\childdoctmp{#1}
      \ifx\childdoctmp\childdocname
        \def\childdoctmp{}
      \else
        \def\childdoctmp
        {
          \childdoctrue
          \includeonly{\childdocname}
          \def\childdocjob{#1}
          \def\jobname{#1}
        }
      \fi
      \expandafter
    \endgroup
    \childdoctmp
  \fi
}
%    \end{macrocode}

% \macro{\childdocof}
% The command |\childdocof| redirects
% compilation to the main file |#1|.
%    \begin{macrocode}
\newcommand{\childdocof}[1]
{
  \childdocdisable
  \childdoctrue
  \includeonly{\childdocname}
  \def\jobname{#1}
  \def\childdocjob{#1}
  \input{#1}
}
%    \end{macrocode}

% \macro{\childdocby}
% The command |\childdocby| ....
%    \begin{macrocode}
\newcommand{\childdocby}[2][]
{
  \childdocdisable
  \childdoctrue
  \childdocmanualtrue
  \if?#1?\else
    \def\jobname{#2}
  \fi
  \def\childdocjob{#2}
  \input{#2}
  \endinput
}
%    \end{macrocode}

% \macro{\childdocforward}
% The command |\childdocforward| redirects
% compilation to the main file or
% (if the optional argument is given) a child file.
% Parameters are set as if the main file
% or a child file starting with |\childdocof| was compiled.
% Then compilation is handed over to the main file:
%    \begin{macrocode}
\newcommand{\childdocforward}[2][]
{
  \begingroup
    \if?#1?
      \def\childdoctmp
      {
        \def\childdocname{#2}
        \def\childdocjob{#2}
        \def\jobname{#2}
        \input{#2}
        \endinput
      }
    \else
      \def\childdoctmp
      {
        \childdocdisable
        \def\childdocname{#2}
        \childdoctrue
        \includeonly{#2}
        \def\childdocjob{#1}
        \def\jobname{#1}
        \input{#1}
        \endinput
      }
    \fi
    \expandafter
  \endgroup
  \childdoctmp
}
%    \end{macrocode}

% \macro{\childdocforwardprefix}
% The command |\childdocforwardprefix| redirects
% compilation to the main or a child file by means of a pattern.
% The prefix |#1| in the current filename is replaced by |#2|
% and the suffix of the current filename is kept
% (it is assumed that the filename does not contain the substring `|~~~|'
% which is used as a delimiter).
% Compilation is handed over to the new file by |\childdocforward|:
%    \begin{macrocode}
\newcommand{\childdocforwardprefix}[3][]
{
  \begingroup
    \def\childdocextract #2##1~~~{\def\childdoctmp{\childdocforward[#1]{#3##1}}}
    \expandafter\childdocextract\childdocname~~~
    \expandafter
  \endgroup
  \childdoctmp
}
%    \end{macrocode}

% \macro{\childdoc}
% The deprecated macro |\childdoc| is a legacy version of |\childdocmain|:
%    \begin{macrocode}
\newcommand{\childdoc}{\childdocmain}
%    \end{macrocode}

% \macro{\childdocredirect}
% The deprecated macro |\childdocredirect| is a legacy version
% of |\childdocforward| and |\childdocforwardprefix|:
%    \begin{macrocode}
\newcommand{\childdocredirect}[2][]
{
  \begingroup
    \if?#1?
      \def\childdoctmp{\childdocforward{#2}}
    \else
      \def\childdoctmp{\childdocforwardprefix{#1}{#2}}
    \fi
    \expandafter
  \endgroup
  \childdoctmp
}
%    \end{macrocode}

%\iffalse
%</package>
%\fi
%
\endinput
\childdocforward{cdocsch2}"|
% \end{tabular}
% \end{center}
% Note that the trailing backslash on each first line
% merely continues the input to the second line
% (for convenient cut ant paste).
% Furthermore, the command |latex| can be replaced by any
% of its alternative versions such as |pdflatex|.
%
% %%%%%%%%%%%%%%%%%%%%%%%%%%%%%%%%%%%%%%%%%%%%%%%%%%%%%%%%%%%%%%%%%%%%%%%%%%%%%%
% %%%%%%%%%%%%%%%%%%%%%%%%%%%%%%%%%%%%%%%%%%%%%%%%%%%%%%%%%%%%%%%%%%%%%%%%%%%%%%
% \section{Implementation}
%\iffalse
%<*package>
%\fi
%
% This section describes the definitions file |childdoc.def|.

% The definitions cannot be loaded using |\usepackage| or |\RequirePackage|
% which has a mechanism to prevent loading a style file more than once.
% When loading the definitions by means of |\input|
% multiple instances have to be prevented manually:
%\iffalse
%This code needs to be before the `\ProvidesFile' directive
%which is defined at the beginning of this file.
%Therefore it is also placed there and commented out here.
%</package>
%<*discard>
%\fi
%    \begin{macrocode}
\ifdefined\childdocmain\endinput\fi
%    \end{macrocode}
%\iffalse
%</discard>
%<*package>
%\fi
%
% \macro{\ifchilddoc}
% \macro{\ifchilddocmanual}
% The conditional |\ifchilddoc| tells whether a
% child (true) or main (false) document is being compiled.
% The conditional |\ifchilddocmanual| tells whether
% the |\includeonly| mechanism is used (false) or
% the selection of child files must be performed manually (true).
% The definitions initialise to false:
%    \begin{macrocode}
\newif\ifchilddoc
\newif\ifchilddocmanual
%    \end{macrocode}

% \macro{\childdocname}
% \macro{\childdocjob}
% The macro |\childdocname| stores the name of the main document
% to be compiled. The macro |\childdocjob| stores the name of
% the document on which the \LaTeX{} compiler was originally invoked.
% The content of |\jobname| cannot be compared
% to filenames specified in the source due to different catcodes.
% The following code rescans |\jobname|, stores the result
% in |\childdocname| and saves a copy in |\childdocjob|:
%    \begin{macrocode}
\edef\childdocname{\scantokens\expandafter{\jobname\noexpand}}
\let\childdocjob\childdocname
%    \end{macrocode}

% \macro{\childdocdisable}
% The macro |\childdocdisable| prevents the main file
% from being processed more than once.
% At this stage, the main document command |\childdocmain|
% is assumed to be called once again where it should do nothing.
% Any subsequent call to it should prevent
% a secondary processing of the main document
% It overwrites the forwarding commands
% |\childdocof| and |\childdocforward|
% with empty macros to prevent further inclusions of the main document:
%    \begin{macrocode}
\newcommand{\childdocdisable}
{
  \renewcommand{\childdocmain}[1]{\renewcommand{\childdocmain}[1]{\endinput}}
  \renewcommand{\childdocof}[1]{}
  \renewcommand{\childdocby}[2][]{}
  \renewcommand{\childdocforward}[2][]{}
  \renewcommand{\childdocdisable}{}
}
%    \end{macrocode}

% \macro{\childdocmain}
% The macro |\childdocmain| is to be called at the top of the main file
% with nothing or the main filename (without extension) as argument.
% First, it breaks loops.
% If the argument is not empty and does not match |\childdocname|
% (which is set by the first inclusion of |childdoc.def|),
% |\ifchilddoc| is set to true, |\includeonly| is applied to the child file
% and |\jobname| is set to the main file
% (for proper handling of |.aux| files):
%    \begin{macrocode}
\newcommand{\childdocmain}[1]
{
  \childdocdisable\childdocmain{}
  \if?#1?\else
    \begingroup
      \def\childdoctmp{#1}
      \ifx\childdoctmp\childdocname
        \def\childdoctmp{}
      \else
        \def\childdoctmp
        {
          \childdoctrue
          \includeonly{\childdocname}
          \def\childdocjob{#1}
          \def\jobname{#1}
        }
      \fi
      \expandafter
    \endgroup
    \childdoctmp
  \fi
}
%    \end{macrocode}

% \macro{\childdocof}
% The command |\childdocof| redirects
% compilation to the main file |#1|.
%    \begin{macrocode}
\newcommand{\childdocof}[1]
{
  \childdocdisable
  \childdoctrue
  \includeonly{\childdocname}
  \def\jobname{#1}
  \def\childdocjob{#1}
  \input{#1}
}
%    \end{macrocode}

% \macro{\childdocby}
% The command |\childdocby| ....
%    \begin{macrocode}
\newcommand{\childdocby}[2][]
{
  \childdocdisable
  \childdoctrue
  \childdocmanualtrue
  \if?#1?\else
    \def\jobname{#2}
  \fi
  \def\childdocjob{#2}
  \input{#2}
  \endinput
}
%    \end{macrocode}

% \macro{\childdocforward}
% The command |\childdocforward| redirects
% compilation to the main file or
% (if the optional argument is given) a child file.
% Parameters are set as if the main file
% or a child file starting with |\childdocof| was compiled.
% Then compilation is handed over to the main file:
%    \begin{macrocode}
\newcommand{\childdocforward}[2][]
{
  \begingroup
    \if?#1?
      \def\childdoctmp
      {
        \def\childdocname{#2}
        \def\childdocjob{#2}
        \def\jobname{#2}
        \input{#2}
        \endinput
      }
    \else
      \def\childdoctmp
      {
        \childdocdisable
        \def\childdocname{#2}
        \childdoctrue
        \includeonly{#2}
        \def\childdocjob{#1}
        \def\jobname{#1}
        \input{#1}
        \endinput
      }
    \fi
    \expandafter
  \endgroup
  \childdoctmp
}
%    \end{macrocode}

% \macro{\childdocforwardprefix}
% The command |\childdocforwardprefix| redirects
% compilation to the main or a child file by means of a pattern.
% The prefix |#1| in the current filename is replaced by |#2|
% and the suffix of the current filename is kept
% (it is assumed that the filename does not contain the substring `|~~~|'
% which is used as a delimiter).
% Compilation is handed over to the new file by |\childdocforward|:
%    \begin{macrocode}
\newcommand{\childdocforwardprefix}[3][]
{
  \begingroup
    \def\childdocextract #2##1~~~{\def\childdoctmp{\childdocforward[#1]{#3##1}}}
    \expandafter\childdocextract\childdocname~~~
    \expandafter
  \endgroup
  \childdoctmp
}
%    \end{macrocode}

% \macro{\childdoc}
% The deprecated macro |\childdoc| is a legacy version of |\childdocmain|:
%    \begin{macrocode}
\newcommand{\childdoc}{\childdocmain}
%    \end{macrocode}

% \macro{\childdocredirect}
% The deprecated macro |\childdocredirect| is a legacy version
% of |\childdocforward| and |\childdocforwardprefix|:
%    \begin{macrocode}
\newcommand{\childdocredirect}[2][]
{
  \begingroup
    \if?#1?
      \def\childdoctmp{\childdocforward{#2}}
    \else
      \def\childdoctmp{\childdocforwardprefix{#1}{#2}}
    \fi
    \expandafter
  \endgroup
  \childdoctmp
}
%    \end{macrocode}

%\iffalse
%</package>
%\fi
%
\endinput
|\\
|\childdocof{|\textit{main}|}|\\
\end{tabular}
\end{center}
at the top of every child file \textit{child}
which is included by |\include{|\textit{child}|}|
from within the main file
(or at least for those files to be compiled individually).
The argument \textit{main} must be the filename of the main file.

There are a couple of
considerations in setting up the main and child documents:

%%%%%%%%%%%%%%%%%%%%%%%%%%%%%%%%%%%%%%%%
\paragraph{Restrictions.}

Please note the following restrictions:
\begin{itemize}
\item
|\childdocmain| must be called with one argument \textit{main}
to ensure compatibility with earlier version of the package.
It must either be empty (|\childdocmain{}|)
or precisely match the filename of the main file in which it is specified.
See \secref{sec:detection} for further information.
\item
The filename \textit{main} must be specified without the |.tex| extension.
\item
The filename \textit{main} is case sensitive
(even in case-insensitive file systems)
due to internal string comparison.
\item
The argument \textit{main} should be fully expanded, it cannot be a macro.
\item
Subdirectories and special characters should be avoided in filenames.
\item
The command |\childdocmain{|\textit{main}|}| must be followed by a whitespace.
It should not be followed immediately by another command
or by a comment mark `|%|'.
This is because the \TeX{} parser reads the token immediately following
the argument of |\childdocmain| and puts it
at the beginning of every child section;
however, a white\-space is ignored.
\end{itemize}

%%%%%%%%%%%%%%%%%%%%%%%%%%%%%%%%%%%%%%%%
\paragraph{Content of Main File.}

It is advisable to place all content in the child files included by |\include|.
Any output contained in the main file will appear in all child documents
unless suppressed manually;
it cannot be suppressed automatically by the |\includeonly| directive
and thus should normally be avoided.
A method to include some content in the main file
by means of conditional processing is described in \secref{sec:conditional}.

%%%%%%%%%%%%%%%%%%%%%%%%%%%%%%%%%%%%%%%%
\paragraph{Page Numbering.}

When only a part of the document is compiled,
the appropriate numbering of pages
(as well as other status parameters)
is determined from the |.aux| files.
The latter contain information from previous passes.
However this information needs to propagate through
all intermediate child documents.
Therefore the page numbering in child documents may well
be inconsistent until the complete document is compiled at least once.

A useful (if unconventional) way to always ensure a consistent
page numbering is to restart the numbering in each child document
and denote the pages by `\textit{child}|.|\textit{page}'
where \textit{child} represents the chapter/section number of the child file.
This can be achieved by the command
|\numberwithin{page}{|\textit{child}|}|
of the \textsf{amsmath} package
where \textit{child} can be |chapter| or |section|
depending on the chosen structuring.
Alternatively, one can modify the macro |\thepage| appropriately
and reset the counter |page| at the start of each child file.

%%%%%%%%%%%%%%%%%%%%%%%%%%%%%%%%%%%%%%%%%%%%%%%%%%%%%%%%%%%%%%%%%%%%%%%%%%%%%%%%
\subsection{Conditional Processing}
\label{sec:conditional}

The package provides a mechanism to compile different versions
of a document. To customise the versions further some conditional processing
can come in handy to distinguish which version is being compiled.
The package provides two macros to describe the compilation context:

%%%%%%%%%%%%%%%%%%%%%%%%%%%%%%%%%%%%%%%%
\DescribeMacro{\ifchilddoc}
The conditional |\ifchilddoc| distinguishes between the compilation of
child documents and the main document:
%
\begin{center}
|\ifchilddoc |\textit{child-code}| |[|\||else |\textit{main-code}]| \||fi|
\end{center}

%%%%%%%%%%%%%%%%%%%%%%%%%%%%%%%%%%%%%%%%
\DescribeMacro{\childdocname}
\DescribeMacro{\childdocjob}
The macro |\childdocname| contains the filename (without extension)
of the main or child file being processed.
Note that |\childdocjob| will always contain the name of the main file.

%%%%%%%%%%%%%%%%%%%%%%%%%%%%%%%%%%%%%%%%
\paragraph{Title Page.}

Conditional processing can be used to include a title or banner page
in the main document when proper precautions are taken.
Importantly, the code in the main file should ensure that the page counter
(as well as other status parameters which are stored in the |.aux| files)
takes the same value after the conditional processing.
Otherwise the page numbers may take divergent values
depending on which part is compiled.

For example, a title page could be declared by:
%
\begin{center}
\begin{tabular}{l}
|\ifchilddoc\||else|\\
|\addtocounter{page}{-1}|\\
\textit{code for title page}\\
|\newpage|\\
|\||fi|
\end{tabular}
\end{center}
%
A banner page for the child documents can be generated by:
%
\begin{center}
\begin{tabular}{l}
|\ifchilddoc|\\
|\addtocounter{page}{-1}|\\
\textit{code for banner page}\\
|\newpage|\\
|\||fi|
\end{tabular}
\end{center}
%
Here one could write a message such as:
\begin{center}
|This is the part \childdocname{} of \childdocjob{}.|
\end{center}

%%%%%%%%%%%%%%%%%%%%%%%%%%%%%%%%%%%%%%%%%%%%%%%%%%%%%%%%%%%%%%%%%%%%%%%%%%%%%%%%
\subsection{Flags}
\label{sec:flags}

The package makes it easy to generate different versions
of the main or child documents.
To this end compilation flags can be defined
and assigned different default values.
They will be particularly useful in conjunction
with the forwarding mechanism described in \secref{sec:forward}.

For example, it may be useful to have a flag |\version|
which can be set to |draft| or |final|.
The document source will contain some conditional code
depending on the value of |\version|.
Suppose further, the flag should default to |final| for the main file
and to |draft| for child files
which is a natural assignment for editing the document.
This is achieved by placing the following code
in the preamble of the main document
(below the |\childdocmain| directive):
%
\begin{center}
\begin{tabular}{l}
|\ifchilddoc|\\
|\providecommand{\version}{draft}|\\
|\||else|\\
|\providecommand{\version}{final}|\\
|\||fi|
\end{tabular}
\end{center}
%
The definition by |\providecommand| makes sure
that previous definitions are not overwritten.
Further statements |\providecommand{\version}{...}|
can thus be added before the above code to override it.

For the main file, one might add a line
(between |\childdocmain| and the above block)
%
\begin{center}
|%\ifchilddoc\||else\providecommand{\version}{draft}\||fi|
\end{center}
%
which can be uncommented to produce a draft version.
Likewise one can add a line to the very top of a child file
(above the |\childdocof{|\textit{main}|}| directive)
%
\begin{center}
|%\providecommand{\version}{final}|
\end{center}
%
which can be uncommented to produce the final version of this child document.

%%%%%%%%%%%%%%%%%%%%%%%%%%%%%%%%%%%%%%%%%%%%%%%%%%%%%%%%%%%%%%%%%%%%%%%%%%%%%%%%
\subsection{Forwarding}
\label{sec:forward}

Different versions of the main or child documents
using compilation flags as described in \secref{sec:flags}
can be (permanently) stored in different files
for convenient compilation, viewing and distribution.
To this end, the package defines a command
to pass on compilation to a different file:

%%%%%%%%%%%%%%%%%%%%%%%%%%%%%%%%%%%%%%%%
\DescribeMacro{\childdocforward}
The command |\childdocforward| redirects processing to
another source file:
%
\begin{center}
\begin{tabular}{l}
|% \iffalse
%
% childdoc.dtx Copyright (C) 2017-2018 Niklas Beisert
%
% This work may be distributed and/or modified under the
% conditions of the LaTeX Project Public License, either version 1.3
% of this license or (at your option) any later version.
% The latest version of this license is in
%   http://www.latex-project.org/lppl.txt
% and version 1.3 or later is part of all distributions of LaTeX
% version 2005/12/01 or later.
%
% This work has the LPPL maintenance status `maintained'.
%
% The Current Maintainer of this work is Niklas Beisert.
%
% This work consists of the files childdoc.dtx and childdoc.ins
% and the derived files childdoc.def and cdocsamp.tex with
% cdocsch1.tex, cdocsch2.tex, cdocsdrf.tex, cdocsfn1.tex, cdocsfn2.tex.
%
%<package>\ifdefined\childdocmain\endinput\fi
%<package>\ProvidesFile{childdoc.def}[2018/12/30 v2.0 child document driver]
%<samplemain>\ProvidesFile{cdocsamp.tex}[2018/12/30 v2.0 sample for childdoc]
%<*driver>
%\ProvidesFile{childdoc.drv}[2018/12/30 v2.0 childdoc reference manual file]
\PassOptionsToClass{10pt,a4paper}{article}
\documentclass{ltxdoc}

\usepackage[margin=35mm]{geometry}
\usepackage{hyperref}
\usepackage{hyperxmp}
\usepackage[usenames]{color}

\hypersetup{colorlinks=true}
\hypersetup{pdfstartview=FitH}
\hypersetup{pdfpagemode=UseNone}
\hypersetup{pdfsource={}}
\hypersetup{pdflang={en-UK}}
\hypersetup{pdfcopyright={Copyright 2017-2018 Niklas Beisert.
  This work may be distributed and/or modified under the
  conditions of the LaTeX Project Public License, either version 1.3
  of this license or (at your option) any later version.}}
\hypersetup{pdflicenseurl={http://www.latex-project.org/lppl.txt}}
\hypersetup{pdfcontactaddress={ETH Zurich, ITP, HIT K,
  Wolfgang-Pauli-Strasse 27}}
\hypersetup{pdfcontactpostcode={8093}}
\hypersetup{pdfcontactcity={Zurich}}
\hypersetup{pdfcontactcountry={Switzerland}}
\hypersetup{pdfcontactemail={nbeisert@itp.phys.ethz.ch}}
\hypersetup{pdfcontacturl={http://people.phys.ethz.ch/\xmptilde nbeisert/}}

\newcommand{\secref}[1]{\hyperref[#1]{section \ref*{#1}}}

\parskip1ex
\parindent0pt
\let\olditemize\itemize
\def\itemize{\olditemize\parskip0pt}

\begin{document}

\title{The \textsf{childdoc} Package}
\hypersetup{pdftitle={The childdoc Package}}
\author{Niklas Beisert\\[2ex]
  Institut f\"ur Theoretische Physik\\
  Eidgen\"ossische Technische Hochschule Z\"urich\\
  Wolfgang-Pauli-Strasse 27, 8093 Z\"urich, Switzerland\\[1ex]
  \href{mailto:nbeisert@itp.phys.ethz.ch}
  {\texttt{nbeisert@itp.phys.ethz.ch}}}
\hypersetup{pdfauthor={Niklas Beisert}}
\hypersetup{pdfsubject={Manual for the LaTeX2e Package childdoc}}
\date{30 December 2018, \textsf{v2.0}}
\maketitle

\begin{abstract}\noindent
\textsf{childdoc} is a \LaTeXe{} package
that enables the direct compilation
of document sections included by |\include|
to individual files.
\end{abstract}

\begingroup
\parskip0ex
\tableofcontents
\endgroup

%%%%%%%%%%%%%%%%%%%%%%%%%%%%%%%%%%%%%%%%%%%%%%%%%%%%%%%%%%%%%%%%%%%%%%%%%%%%%%%%
%%%%%%%%%%%%%%%%%%%%%%%%%%%%%%%%%%%%%%%%%%%%%%%%%%%%%%%%%%%%%%%%%%%%%%%%%%%%%%%%
\section{Introduction}

\LaTeX{} provides a mechanism to structure a large document (such as a book)
into a main file and several child files (containing the chapters)
using the |\include| command.
This mechanism is beneficial for documents
which span hundreds of pages in order to
make the source file(s) more manageable.
Moreover, compilation can be restricted to
selected child files by means of the |\includeonly| command.
The latter feature can be used to reduce the compilation time while editing
(this was significantly more useful in the earlier days of \LaTeX{})
or to generate a smaller document which is easier to navigate.
Another application of |\includeonly| is to generate
documents consisting of selected parts of the complete document.

However, there are a few drawbacks of the plain |\include| mechanism:
\begin{itemize}
\item
The child files cannot be compiled on their own,
they can only be compiled via the main file.
A naive editing environment
(such as a text editor with an option
to have the current file processed by \LaTeX)
may require one to switch to the main file before compiling;
attempting to compile the child file produces errors.
\item
The main file must be modified (each time)
to adjust the |\includeonly| command
to the present needs. This easily leaves the main file in a messy state.
\item
The generated document will always carry the filename
of the main document. This is inconvenient if
several child files are to be compiled and
to be kept for distribution.
\end{itemize}

The present package provides a simple interface
to make child files individually compilable by \LaTeX{}.
Compiling a child file then has the same effect as compiling
the main file with an |\includeonly| command
to select the appropriate child.
Moreover the generated document will carry the name of the child
rather than the main file.
This resolves all three above issues.

This feature is meant to make the editing of books,
thesis documents and lecture notes somewhat more convenient.
However, the package can also be used efficiently for
composing a series of documents (such as exercise sheets)
which are typically distributed individually.
It then assists the author in generating the individual documents
(potentially in different versions)
as well as a document containing the collected series.
Another application is in developing style files
or other kinds of included material
where compilation of the style file could redirect
to a sample or test file.

%%%%%%%%%%%%%%%%%%%%%%%%%%%%%%%%%%%%%%%%%%%%%%%%%%%%%%%%%%%%%%%%%%%%%%%%%%%%%%%%
%%%%%%%%%%%%%%%%%%%%%%%%%%%%%%%%%%%%%%%%%%%%%%%%%%%%%%%%%%%%%%%%%%%%%%%%%%%%%%%%
\section{Usage}

First of all, the package \textsf{childdoc} is \emph{not} a standard
\LaTeXe{} |.sty| style file! Therefore it needs to be invoked in
a non-standard way.

%%%%%%%%%%%%%%%%%%%%%%%%%%%%%%%%%%%%%%%%%%%%%%%%%%%%%%%%%%%%%%%%%%%%%%%%%%%%%%%%
\subsection{Included Files}
\label{sec:include}

%%%%%%%%%%%%%%%%%%%%%%%%%%%%%%%%%%%%%%%%
\DescribeMacro{\childdocmain}
To use the package, add the commands
\begin{center}
\begin{tabular}{l}
|% \iffalse
%
% childdoc.dtx Copyright (C) 2017-2018 Niklas Beisert
%
% This work may be distributed and/or modified under the
% conditions of the LaTeX Project Public License, either version 1.3
% of this license or (at your option) any later version.
% The latest version of this license is in
%   http://www.latex-project.org/lppl.txt
% and version 1.3 or later is part of all distributions of LaTeX
% version 2005/12/01 or later.
%
% This work has the LPPL maintenance status `maintained'.
%
% The Current Maintainer of this work is Niklas Beisert.
%
% This work consists of the files childdoc.dtx and childdoc.ins
% and the derived files childdoc.def and cdocsamp.tex with
% cdocsch1.tex, cdocsch2.tex, cdocsdrf.tex, cdocsfn1.tex, cdocsfn2.tex.
%
%<package>\ifdefined\childdocmain\endinput\fi
%<package>\ProvidesFile{childdoc.def}[2018/12/30 v2.0 child document driver]
%<samplemain>\ProvidesFile{cdocsamp.tex}[2018/12/30 v2.0 sample for childdoc]
%<*driver>
%\ProvidesFile{childdoc.drv}[2018/12/30 v2.0 childdoc reference manual file]
\PassOptionsToClass{10pt,a4paper}{article}
\documentclass{ltxdoc}

\usepackage[margin=35mm]{geometry}
\usepackage{hyperref}
\usepackage{hyperxmp}
\usepackage[usenames]{color}

\hypersetup{colorlinks=true}
\hypersetup{pdfstartview=FitH}
\hypersetup{pdfpagemode=UseNone}
\hypersetup{pdfsource={}}
\hypersetup{pdflang={en-UK}}
\hypersetup{pdfcopyright={Copyright 2017-2018 Niklas Beisert.
  This work may be distributed and/or modified under the
  conditions of the LaTeX Project Public License, either version 1.3
  of this license or (at your option) any later version.}}
\hypersetup{pdflicenseurl={http://www.latex-project.org/lppl.txt}}
\hypersetup{pdfcontactaddress={ETH Zurich, ITP, HIT K,
  Wolfgang-Pauli-Strasse 27}}
\hypersetup{pdfcontactpostcode={8093}}
\hypersetup{pdfcontactcity={Zurich}}
\hypersetup{pdfcontactcountry={Switzerland}}
\hypersetup{pdfcontactemail={nbeisert@itp.phys.ethz.ch}}
\hypersetup{pdfcontacturl={http://people.phys.ethz.ch/\xmptilde nbeisert/}}

\newcommand{\secref}[1]{\hyperref[#1]{section \ref*{#1}}}

\parskip1ex
\parindent0pt
\let\olditemize\itemize
\def\itemize{\olditemize\parskip0pt}

\begin{document}

\title{The \textsf{childdoc} Package}
\hypersetup{pdftitle={The childdoc Package}}
\author{Niklas Beisert\\[2ex]
  Institut f\"ur Theoretische Physik\\
  Eidgen\"ossische Technische Hochschule Z\"urich\\
  Wolfgang-Pauli-Strasse 27, 8093 Z\"urich, Switzerland\\[1ex]
  \href{mailto:nbeisert@itp.phys.ethz.ch}
  {\texttt{nbeisert@itp.phys.ethz.ch}}}
\hypersetup{pdfauthor={Niklas Beisert}}
\hypersetup{pdfsubject={Manual for the LaTeX2e Package childdoc}}
\date{30 December 2018, \textsf{v2.0}}
\maketitle

\begin{abstract}\noindent
\textsf{childdoc} is a \LaTeXe{} package
that enables the direct compilation
of document sections included by |\include|
to individual files.
\end{abstract}

\begingroup
\parskip0ex
\tableofcontents
\endgroup

%%%%%%%%%%%%%%%%%%%%%%%%%%%%%%%%%%%%%%%%%%%%%%%%%%%%%%%%%%%%%%%%%%%%%%%%%%%%%%%%
%%%%%%%%%%%%%%%%%%%%%%%%%%%%%%%%%%%%%%%%%%%%%%%%%%%%%%%%%%%%%%%%%%%%%%%%%%%%%%%%
\section{Introduction}

\LaTeX{} provides a mechanism to structure a large document (such as a book)
into a main file and several child files (containing the chapters)
using the |\include| command.
This mechanism is beneficial for documents
which span hundreds of pages in order to
make the source file(s) more manageable.
Moreover, compilation can be restricted to
selected child files by means of the |\includeonly| command.
The latter feature can be used to reduce the compilation time while editing
(this was significantly more useful in the earlier days of \LaTeX{})
or to generate a smaller document which is easier to navigate.
Another application of |\includeonly| is to generate
documents consisting of selected parts of the complete document.

However, there are a few drawbacks of the plain |\include| mechanism:
\begin{itemize}
\item
The child files cannot be compiled on their own,
they can only be compiled via the main file.
A naive editing environment
(such as a text editor with an option
to have the current file processed by \LaTeX)
may require one to switch to the main file before compiling;
attempting to compile the child file produces errors.
\item
The main file must be modified (each time)
to adjust the |\includeonly| command
to the present needs. This easily leaves the main file in a messy state.
\item
The generated document will always carry the filename
of the main document. This is inconvenient if
several child files are to be compiled and
to be kept for distribution.
\end{itemize}

The present package provides a simple interface
to make child files individually compilable by \LaTeX{}.
Compiling a child file then has the same effect as compiling
the main file with an |\includeonly| command
to select the appropriate child.
Moreover the generated document will carry the name of the child
rather than the main file.
This resolves all three above issues.

This feature is meant to make the editing of books,
thesis documents and lecture notes somewhat more convenient.
However, the package can also be used efficiently for
composing a series of documents (such as exercise sheets)
which are typically distributed individually.
It then assists the author in generating the individual documents
(potentially in different versions)
as well as a document containing the collected series.
Another application is in developing style files
or other kinds of included material
where compilation of the style file could redirect
to a sample or test file.

%%%%%%%%%%%%%%%%%%%%%%%%%%%%%%%%%%%%%%%%%%%%%%%%%%%%%%%%%%%%%%%%%%%%%%%%%%%%%%%%
%%%%%%%%%%%%%%%%%%%%%%%%%%%%%%%%%%%%%%%%%%%%%%%%%%%%%%%%%%%%%%%%%%%%%%%%%%%%%%%%
\section{Usage}

First of all, the package \textsf{childdoc} is \emph{not} a standard
\LaTeXe{} |.sty| style file! Therefore it needs to be invoked in
a non-standard way.

%%%%%%%%%%%%%%%%%%%%%%%%%%%%%%%%%%%%%%%%%%%%%%%%%%%%%%%%%%%%%%%%%%%%%%%%%%%%%%%%
\subsection{Included Files}
\label{sec:include}

%%%%%%%%%%%%%%%%%%%%%%%%%%%%%%%%%%%%%%%%
\DescribeMacro{\childdocmain}
To use the package, add the commands
\begin{center}
\begin{tabular}{l}
|\input{childdoc.def}|\\
|\childdocmain{}|\\
\end{tabular}
\end{center}
at the very top of the main \LaTeX{} file,
in particular \emph{before} the |\documentclass| statement!
The argument of |\childdocmain| should be left empty
(but it must be present).

%%%%%%%%%%%%%%%%%%%%%%%%%%%%%%%%%%%%%%%%
\DescribeMacro{\childdocof}
Furthermore, add the commands
\begin{center}
\begin{tabular}{l}
|\input{childdoc.def}|\\
|\childdocof{|\textit{main}|}|\\
\end{tabular}
\end{center}
at the top of every child file \textit{child}
which is included by |\include{|\textit{child}|}|
from within the main file
(or at least for those files to be compiled individually).
The argument \textit{main} must be the filename of the main file.

There are a couple of
considerations in setting up the main and child documents:

%%%%%%%%%%%%%%%%%%%%%%%%%%%%%%%%%%%%%%%%
\paragraph{Restrictions.}

Please note the following restrictions:
\begin{itemize}
\item
|\childdocmain| must be called with one argument \textit{main}
to ensure compatibility with earlier version of the package.
It must either be empty (|\childdocmain{}|)
or precisely match the filename of the main file in which it is specified.
See \secref{sec:detection} for further information.
\item
The filename \textit{main} must be specified without the |.tex| extension.
\item
The filename \textit{main} is case sensitive
(even in case-insensitive file systems)
due to internal string comparison.
\item
The argument \textit{main} should be fully expanded, it cannot be a macro.
\item
Subdirectories and special characters should be avoided in filenames.
\item
The command |\childdocmain{|\textit{main}|}| must be followed by a whitespace.
It should not be followed immediately by another command
or by a comment mark `|%|'.
This is because the \TeX{} parser reads the token immediately following
the argument of |\childdocmain| and puts it
at the beginning of every child section;
however, a white\-space is ignored.
\end{itemize}

%%%%%%%%%%%%%%%%%%%%%%%%%%%%%%%%%%%%%%%%
\paragraph{Content of Main File.}

It is advisable to place all content in the child files included by |\include|.
Any output contained in the main file will appear in all child documents
unless suppressed manually;
it cannot be suppressed automatically by the |\includeonly| directive
and thus should normally be avoided.
A method to include some content in the main file
by means of conditional processing is described in \secref{sec:conditional}.

%%%%%%%%%%%%%%%%%%%%%%%%%%%%%%%%%%%%%%%%
\paragraph{Page Numbering.}

When only a part of the document is compiled,
the appropriate numbering of pages
(as well as other status parameters)
is determined from the |.aux| files.
The latter contain information from previous passes.
However this information needs to propagate through
all intermediate child documents.
Therefore the page numbering in child documents may well
be inconsistent until the complete document is compiled at least once.

A useful (if unconventional) way to always ensure a consistent
page numbering is to restart the numbering in each child document
and denote the pages by `\textit{child}|.|\textit{page}'
where \textit{child} represents the chapter/section number of the child file.
This can be achieved by the command
|\numberwithin{page}{|\textit{child}|}|
of the \textsf{amsmath} package
where \textit{child} can be |chapter| or |section|
depending on the chosen structuring.
Alternatively, one can modify the macro |\thepage| appropriately
and reset the counter |page| at the start of each child file.

%%%%%%%%%%%%%%%%%%%%%%%%%%%%%%%%%%%%%%%%%%%%%%%%%%%%%%%%%%%%%%%%%%%%%%%%%%%%%%%%
\subsection{Conditional Processing}
\label{sec:conditional}

The package provides a mechanism to compile different versions
of a document. To customise the versions further some conditional processing
can come in handy to distinguish which version is being compiled.
The package provides two macros to describe the compilation context:

%%%%%%%%%%%%%%%%%%%%%%%%%%%%%%%%%%%%%%%%
\DescribeMacro{\ifchilddoc}
The conditional |\ifchilddoc| distinguishes between the compilation of
child documents and the main document:
%
\begin{center}
|\ifchilddoc |\textit{child-code}| |[|\||else |\textit{main-code}]| \||fi|
\end{center}

%%%%%%%%%%%%%%%%%%%%%%%%%%%%%%%%%%%%%%%%
\DescribeMacro{\childdocname}
\DescribeMacro{\childdocjob}
The macro |\childdocname| contains the filename (without extension)
of the main or child file being processed.
Note that |\childdocjob| will always contain the name of the main file.

%%%%%%%%%%%%%%%%%%%%%%%%%%%%%%%%%%%%%%%%
\paragraph{Title Page.}

Conditional processing can be used to include a title or banner page
in the main document when proper precautions are taken.
Importantly, the code in the main file should ensure that the page counter
(as well as other status parameters which are stored in the |.aux| files)
takes the same value after the conditional processing.
Otherwise the page numbers may take divergent values
depending on which part is compiled.

For example, a title page could be declared by:
%
\begin{center}
\begin{tabular}{l}
|\ifchilddoc\||else|\\
|\addtocounter{page}{-1}|\\
\textit{code for title page}\\
|\newpage|\\
|\||fi|
\end{tabular}
\end{center}
%
A banner page for the child documents can be generated by:
%
\begin{center}
\begin{tabular}{l}
|\ifchilddoc|\\
|\addtocounter{page}{-1}|\\
\textit{code for banner page}\\
|\newpage|\\
|\||fi|
\end{tabular}
\end{center}
%
Here one could write a message such as:
\begin{center}
|This is the part \childdocname{} of \childdocjob{}.|
\end{center}

%%%%%%%%%%%%%%%%%%%%%%%%%%%%%%%%%%%%%%%%%%%%%%%%%%%%%%%%%%%%%%%%%%%%%%%%%%%%%%%%
\subsection{Flags}
\label{sec:flags}

The package makes it easy to generate different versions
of the main or child documents.
To this end compilation flags can be defined
and assigned different default values.
They will be particularly useful in conjunction
with the forwarding mechanism described in \secref{sec:forward}.

For example, it may be useful to have a flag |\version|
which can be set to |draft| or |final|.
The document source will contain some conditional code
depending on the value of |\version|.
Suppose further, the flag should default to |final| for the main file
and to |draft| for child files
which is a natural assignment for editing the document.
This is achieved by placing the following code
in the preamble of the main document
(below the |\childdocmain| directive):
%
\begin{center}
\begin{tabular}{l}
|\ifchilddoc|\\
|\providecommand{\version}{draft}|\\
|\||else|\\
|\providecommand{\version}{final}|\\
|\||fi|
\end{tabular}
\end{center}
%
The definition by |\providecommand| makes sure
that previous definitions are not overwritten.
Further statements |\providecommand{\version}{...}|
can thus be added before the above code to override it.

For the main file, one might add a line
(between |\childdocmain| and the above block)
%
\begin{center}
|%\ifchilddoc\||else\providecommand{\version}{draft}\||fi|
\end{center}
%
which can be uncommented to produce a draft version.
Likewise one can add a line to the very top of a child file
(above the |\childdocof{|\textit{main}|}| directive)
%
\begin{center}
|%\providecommand{\version}{final}|
\end{center}
%
which can be uncommented to produce the final version of this child document.

%%%%%%%%%%%%%%%%%%%%%%%%%%%%%%%%%%%%%%%%%%%%%%%%%%%%%%%%%%%%%%%%%%%%%%%%%%%%%%%%
\subsection{Forwarding}
\label{sec:forward}

Different versions of the main or child documents
using compilation flags as described in \secref{sec:flags}
can be (permanently) stored in different files
for convenient compilation, viewing and distribution.
To this end, the package defines a command
to pass on compilation to a different file:

%%%%%%%%%%%%%%%%%%%%%%%%%%%%%%%%%%%%%%%%
\DescribeMacro{\childdocforward}
The command |\childdocforward| redirects processing to
another source file:
%
\begin{center}
\begin{tabular}{l}
|\input{childdoc.def}|\\
|\childdocforward[|\textit{main}|]{|\textit{dest}|}|\\
\end{tabular}
\end{center}
%
The argument \textit{dest} is the destination file
(without extension).
It should be the main file or one of the child files.
Note that further \textsf{childdoc} directives
such as |\childdocof| and |\childdocforward|
in the indicated file will be processed in this form.
The optional argument \textit{main}
passes on directly to the main file \textit{main}
while pretending to compile the child \textit{dest}.
This form behaves as if \textit{dest}
issues |\childdocof{|\textit{main}|}| right away,
and no further \textsf{childdoc} directives will be processed.

%%%%%%%%%%%%%%%%%%%%%%%%%%%%%%%%%%%%%%%%
\DescribeMacro{\...prefix}
In the alternative form |\childdocforwardprefix|,
%
\begin{center}
\begin{tabular}{l}
|\input{childdoc.def}|\\
|\childdocforwardprefix[|\textit{main}|]{|\textit{prefix}|}{|\textit{dest}|}|
\end{tabular}
\end{center}
%
the destination file is determined by a pattern
depending on the current file:
To make this work, the current file must be called
`{\textit{prefix}\hspace{0.2em}\textit{suffix}}'
with \textit{prefix} matching precisely the argument.
Processing is then passed on to the file
`{\textit{dest}\hspace{0.2em}\textit{suffix}}'.
Surely, the same effect is achieved by
directly specifying the
argument `{\textit{dest}\hspace{0.2em}\textit{suffix}}'
in the first form.
However, that requires to set up a different file
for each child. With the alternative form of the command
all these files can have exactly the same content
which simplifies setting them up and maintaining them.

For example, the following file |draft.tex|
with a compilation flag |\version| as described in \secref{sec:flags}
compiles the main document as a draft:
%
\begin{center}
\begin{tabular}{l}
|\def\version{draft}|\\
|\input{childdoc.def}|\\
|\childdocforward{|\textit{main}|}|
\end{tabular}
\end{center}
%
Likewise, the following files |final|\textit{nn}|.tex|
compile the final version of the child document
|child|\textit{nn}|.tex|:
%
\begin{center}
\begin{tabular}{l}
|\def\version{final}|\\
|\input{childdoc.def}|\\
|\childdocforwardprefix{final}{child}|
\end{tabular}
\end{center}
%

Note that when several versions of a main file and/or of each child file
are to be generated, it may be convenient to set up a |Makefile| or
shell script to automatise the process.

%%%%%%%%%%%%%%%%%%%%%%%%%%%%%%%%%%%%%%%%%%%%%%%%%%%%%%%%%%%%%%%%%%%%%%%%%%%%%%%%
\subsection{Command Line Processing}
\label{sec:commandline}

The effect of redirection files can also be achieved by invoking
the \LaTeX{} compiler with a more elaborate command line.
Most conveniently this should be done as part
of a shell script or a |Makefile|.

When using \textsf{childdoc} in the main file, the following
command lines effectively perform a redirection
(note that depending on the shell being used,
backslashes may have to be doubled: `|\|' $\to$ `|\\|'):
%
\begin{center}
|... -jobname "|\textit{target}|" |\\|"|[\textit{flags}]%
|\input{childdoc.def}\childdocforward[|\textit{main}|]{|\textit{dest}|}"|
\end{center}
%
Here \textit{target} is the name of the output file,
\textit{main} is the name of the main file
and \textit{dest} is the name of the main or child file to be processed
(all filenames without extensions).
The optional argument \textit{main} can be omitted
if \textit{main} matches \textit{dest}.
Optionally, compilation \textit{flags} can be defined via |\def| commands.
This command line makes the \TeX{} engine believe
it is compiling the file \textit{target}
whose content is specified as the latter parameter.
The provided code then forwards the processing to
\textit{main} or \textit{dest} as described in \secref{sec:forward}.

%%%%%%%%%%%%%%%%%%%%%%%%%%%%%%%%%%%%%%%%%%%%%%%%%%%%%%%%%%%%%%%%%%%%%%%%%%%%%%%%
\subsection{Include by Input}
\label{sec:input}

Including child documents by |\include| has some restrictions by design.
Most notably, the content of a child document always occupies
its own set of pages; pages cannot be shared between child documents.
Usually, this behaviour makes perfect sense
because each child document contain an essential part of the document.
However, in some situations it may be desirable to compose
a document from a collection of parts
without having mandatory page breaks between then.
For this case, the package
provides a mechanism to include parts
by |\input| which can also be processed individually.
However, by construction this mechanism
requires manual handling of the content to be output.

%%%%%%%%%%%%%%%%%%%%%%%%%%%%%%%%%%%%%%%%
\DescribeMacro{\ifchilddocmanual}
The main file should be prepared as usual, see \secref{sec:include}.
However, the document body must make a distinction
between processing of an individual part and of the main document, e.g.:
%
\begin{center}
\begin{tabular}{l}
|\ifchilddocmanual|\\
|\input{\childdocname}|\\
|\||else|\\
\textit{document body with }|\input{|\textit{part}|}|\\
|\||fi|
\end{tabular}
\end{center}
%
The conditional |\ifchilddocmanual| is true whenever
a part to be included by |\input| is being compiled,
and the name of the part is stored in |\childdocname|.

%%%%%%%%%%%%%%%%%%%%%%%%%%%%%%%%%%%%%%%%
\DescribeMacro{\childdocby}
Each part to be included by |\input| should start with:
%
\begin{center}
\begin{tabular}{l}
|\input{childdoc.def}|\\
|\childdocby{|\textit{main}|}|\\
\end{tabular}
\end{center}
%
The directive |\childdocby| is similar to |\childdocof|
described in \secref{sec:include},
but the subsequent selection of content must be done manually.
To that end, both |\ifchilddoc| and |\ifchilddocmanual|
will be true upon processing of a part,
and the name of the part is stored in |\childdocname|.
Note that |\jobname| will be set to the filename of the current part
so that each part receives an individual |.aux| file
that does not interfere with the |.aux| file(s) of the main document.
This behaviour can be altered by the alternative form
|\childdocby[*]{|\textit{main}|}| (with a non-empty optional argument)
which uses the |.aux| file of the main document
by setting |\jobname| to \textit{main}.

%%%%%%%%%%%%%%%%%%%%%%%%%%%%%%%%%%%%%%%%%%%%%%%%%%%%%%%%%%%%%%%%%%%%%%%%%%%%%%%%
\subsection{Driver Development}
\label{sec:driver}

The \textsf{childdoc} mechanism can also be use for the development
of definition files such as \LaTeX{} styles or classes.
This case differs from the above setup with multiple parts
included by |\include| in that no |\includeonly| should be invoked.
This can be achieved by starting the include file
(before |\ProvidesPackage|) with:
%
\begin{center}
\begin{tabular}{l}
|\input{childdoc.def}|\\
|\childdocforward{|\textit{main}|}|\\
\end{tabular}
\end{center}
%
or alternatively with:
%
\begin{center}
\begin{tabular}{l}
|\input{childdoc.def}|\\
|\childdocby{|\textit{main}|}|\\
\end{tabular}
\end{center}
%
Both forms have slightly different effects as described above.
The main file is prepared as usual, see \secref{sec:include}.

%%%%%%%%%%%%%%%%%%%%%%%%%%%%%%%%%%%%%%%%%%%%%%%%%%%%%%%%%%%%%%%%%%%%%%%%%%%%%%%%
\subsection{Legacy Detection}
\label{sec:detection}

The directive |\childdocmain| in the main file can detect
whether the complete document or merely a child is to be compiled
even without using the directive |\childdocof|.
This method is deprecated because it is less robust
and there is no compelling reason to use it;
it is merely provided for backward compatibility
and it may be removed in future versions.

If the detection mechanism is to be used,
it is mandatory to correctly specify
the filename of the main file as the argument of |\childdocmain|:
%
\begin{center}
\begin{tabular}{l}
|\input{childdoc.def}|\\
|\childdocmain{|\textit{main}|}|\\
\end{tabular}
\end{center}
%
If |\jobname| does not match the argument \textit{main} of |\childdocmain|,
it is assumed that |\jobname| points to the child file to be compiled.
When using |\childdocmain| with the main file specified as argument,
it suffices to start a child file
with just |\input{|\textit{main}|}|
without loading of the package and using |\childdocof|.
If instead all processing is done
with the appropriate \textsf{childdoc} directives,
the argument of \textit{main} of |\childdocmain| can be empty.

An alternative version of the command line processing described
in \secref{sec:commandline} using the detection mechanism reads:
%
\begin{center}
|... -jobname "|\textit{target}|" "|[\textit{flags}]%
[|\def\jobname{|\textit{dest}|}|]|\input{|\textit{main}|}"|
\end{center}

%%%%%%%%%%%%%%%%%%%%%%%%%%%%%%%%%%%%%%%%%%%%%%%%%%%%%%%%%%%%%%%%%%%%%%%%%%%%%%%%
\subsection{Manual Code}
\label{sec:manual}

In case one cannot be certain whether the definitions file |childdoc.def|
is installed on the target \TeX{} distribution
and one prefers not to ship it,
it is conceivable to paste a few relevant commands into the sources.

To that end, drop all statements |\input{childdoc.def}|
and perform the replacements as outlined below.
Instead of |\childdocmain{|\textit{main}|}| add the following code
to the top of the main file:
%
\begin{center}
\begin{tabular}{l}
|\||ifdefined\childdocname\endinput\||fi\newif\ifchilddoc|\\
|\edef\childdocname{\scantokens\expandafter{\jobname\noexpand}}|\\
|\def\childdocmain{|\textit{main}|}\||ifx\childdocmain\childdocname\||else|\\
|\childdoctrue\includeonly{\childdocname}\let\jobname\childdocmain\||fi|\\
\end{tabular}
\end{center}
%
Instead of |\childdocof{|\textit{main}|}| just include the main file
at the top of each child file:
%
\begin{center}
|\input{|\textit{main}|}|
\end{center}
%
A simple redirection |\childdocforward{|\textit{dest}|}| is achieved by:
%
\begin{center}
|\def\jobname{|\textit{dest}|}\input{\jobname}|
\end{center}
%
The redirection with prefix
|\childdocforwardprefix[|\textit{prefix}|]{|\textit{dest}|}|
is accomplished by:
%
\begin{center}
\begin{tabular}{l}
|{\edef\jobname{\scantokens\expandafter{\jobname\noexpand}}|\\
|\def\redirectjob |\textit{prefix}|#1~~~{\gdef\jobname{|\textit{dest}|#1}}|\\
|\expandafter\redirectjob\jobname~~~}\input{\jobname}|
\end{tabular}
\end{center}

In an alternative approach,
child documents can be compiled by a specific command line
without additional code or specific definitions:
%
\begin{center}
|... -jobname "|\textit{target}|" "|[\textit{flags}]%
|\includeonly{|\textit{dest}|}\input{|\textit{main}|}"|
\end{center}
%

%%%%%%%%%%%%%%%%%%%%%%%%%%%%%%%%%%%%%%%%%%%%%%%%%%%%%%%%%%%%%%%%%%%%%%%%%%%%%%%%
%%%%%%%%%%%%%%%%%%%%%%%%%%%%%%%%%%%%%%%%%%%%%%%%%%%%%%%%%%%%%%%%%%%%%%%%%%%%%%%%
\section{Information}

%%%%%%%%%%%%%%%%%%%%%%%%%%%%%%%%%%%%%%%%%%%%%%%%%%%%%%%%%%%%%%%%%%%%%%%%%%%%%%%%
\subsection{Copyright}

Copyright \copyright{} 2017--2018 Niklas Beisert

This work may be distributed and/or modified under the
conditions of the \LaTeX{} Project Public License, either version 1.3
of this license or (at your option) any later version.
The latest version of this license is in
  \url{http://www.latex-project.org/lppl.txt}
and version 1.3 or later is part of all distributions of \LaTeX{}
version 2005/12/01 or later.

This work has the LPPL maintenance status `maintained'.

The Current Maintainer of this work is Niklas Beisert.

This work consists of the files |README.txt|, |childdoc.ins| and |childdoc.dtx|
as well as the derived files |childdoc.def|, |cdocsamp.tex|
with |cdocsch1.tex|, |cdocsch2.tex|, |cdocspt3.tex|, |cdocspt4.tex|,
|cdocsdrf.tex|, |cdocsfn1.tex|, |cdocsfn2.tex|
as well as |childdoc.pdf|.

%%%%%%%%%%%%%%%%%%%%%%%%%%%%%%%%%%%%%%%%%%%%%%%%%%%%%%%%%%%%%%%%%%%%%%%%%%%%%%%%
\subsection{Files and Installation}

The package consists of the files:
%
\begin{center}
\begin{tabular}{ll}
    |README.txt|   & readme file \\
    |childdoc.ins| & installation file \\
    |childdoc.dtx| & source file \\
    |childdoc.def| & definition file \\
    |cdocsamp.tex| & sample main file \\
    |cdocsch1.tex| & sample include file \\
    |cdocsch2.tex| & sample include file \\
    |cdocspt3.tex| & sample part file \\
    |cdocspt4.tex| & sample part file \\
    |cdocsdrf.tex| & sample redirection file \\
    |cdocsfn1.tex| & sample redirection file \\
    |cdocsfn2.tex| & sample redirection file \\
    |childdoc.pdf| & manual
\end{tabular}
\end{center}
%
The distribution consists of the files
|README.txt|, |childdoc.ins| and |childdoc.dtx|.
%
\begin{itemize}
\item
Run (pdf)\LaTeX{} on |childdoc.dtx|
to compile the manual |childdoc.pdf| (this file).
\item
Run \LaTeX{} on |childdoc.ins| to create the definitions file |childdoc.def|
and the sample |cdocsamp.tex| with include files
|cdocsch1.tex|, |cdocsch2.tex|, |cdocspt3.tex|, |cdocspt4.tex|,
|cdocsdrf.tex|, |cdocsfn1.tex|, |cdocsfn2.tex|.
Then copy the file |childdoc.def| to an appropriate directory of your \LaTeX{}
distribution, e.g.\ \textit{texmf-root}|/tex/latex/childdoc|.
\end{itemize}

%%%%%%%%%%%%%%%%%%%%%%%%%%%%%%%%%%%%%%%%%%%%%%%%%%%%%%%%%%%%%%%%%%%%%%%%%%%%%%%%
\subsection{Related CTAN Packages}

There are several other packages which offer a similar functionality:
%
\begin{itemize}
\item
The packages
\href{http://ctan.org/pkg/docmute}{\textsf{docmute}},
\href{http://ctan.org/pkg/includex}{\textsf{includex}} and
\href{http://ctan.org/pkg/standalone}{\textsf{standalone}}
provide commands to include only the document body of
a child file thus allowing both files to be compiled individually.
\item
The packages \href{http://ctan.org/pkg/subdocs}{\textsf{subdocs}}
and \href{http://ctan.org/pkg/subfiles}{\textsf{subfiles}}
provide structures in which the main and child documents can be
encapsulated and allowing them to be compiled individually.
The inclusion mechanism is different from the conventional |\include|.
\item
The package \href{http://ctan.org/pkg/combine}{\textsf{combine}}
is an elaborate solution to combine several documents into one.
\end{itemize}
%
See also the CTAN topic \href{http://ctan.org/topic/subdocs}{\textsf{subdocs}}
for further related packages.
The present package differs from the above solutions in that
a document structure constructed with the conventional |\include| mechanism
just needs two extra commands at the top of every file
such that all constituent files can be compiled individually.

%%%%%%%%%%%%%%%%%%%%%%%%%%%%%%%%%%%%%%%%%%%%%%%%%%%%%%%%%%%%%%%%%%%%%%%%%%%%%%%%
%\subsection{Feature Suggestions}
%
%The following is a list of features which may be useful for future
%versions of this package:
%%
%\begin{itemize}
%\item
%\ldots
%\end{itemize}

%%%%%%%%%%%%%%%%%%%%%%%%%%%%%%%%%%%%%%%%%%%%%%%%%%%%%%%%%%%%%%%%%%%%%%%%%%%%%%%%
\subsection{Revision History}

%%%%%%%%%%%%%%%%%%%%%%%%%%%%%%%%%%%%%%%%
\paragraph{v2.0:} 2018/12/30

\begin{itemize}
\item
immediate forward processing
\item
added |\childdocby| mechanism
\item
manual restructured
\end{itemize}

%%%%%%%%%%%%%%%%%%%%%%%%%%%%%%%%%%%%%%%%
\paragraph{v1.6:} 2018/01/17

\begin{itemize}
\item
application for development of include files
\item
corrections to manual
\end{itemize}

%%%%%%%%%%%%%%%%%%%%%%%%%%%%%%%%%%%%%%%%
\paragraph{v1.5:} 2017/05/21

\begin{itemize}
\item
more complete structuring introduced
\item
|\childdocof| introduced
\item
|\childdoc| renamed to |\childdocmain|
\item
|\childredirect| renamed to |\childdocforward| and |\childdocforwardprefix|
and functionality expanded
\end{itemize}

%%%%%%%%%%%%%%%%%%%%%%%%%%%%%%%%%%%%%%%%
\paragraph{v1.0:} 2017/04/27

\begin{itemize}
\item
manual and install package
\item
first version published on CTAN
\end{itemize}

%%%%%%%%%%%%%%%%%%%%%%%%%%%%%%%%%%%%%%%%
\paragraph{v0.6:} 2017/04/26

\begin{itemize}
\item
redirection mechanism added
\end{itemize}

%%%%%%%%%%%%%%%%%%%%%%%%%%%%%%%%%%%%%%%%
\paragraph{v0.5:} 2017/04/26

\begin{itemize}
\item
functionality in definition file
\end{itemize}


%%%%%%%%%%%%%%%%%%%%%%%%%%%%%%%%%%%%%%%%%%%%%%%%%%%%%%%%%%%%%%%%%%%%%%%%%%%%%%%%
%%%%%%%%%%%%%%%%%%%%%%%%%%%%%%%%%%%%%%%%%%%%%%%%%%%%%%%%%%%%%%%%%%%%%%%%%%%%%%%%
%%%%%%%%%%%%%%%%%%%%%%%%%%%%%%%%%%%%%%%%%%%%%%%%%%%%%%%%%%%%%%%%%%%%%%%%%%%%%%%%
\appendix

\settowidth\MacroIndent{\rmfamily\scriptsize 000\ }

 \DocInput{childdoc.dtx}

\end{document}
%</driver>
% \fi
%
% %%%%%%%%%%%%%%%%%%%%%%%%%%%%%%%%%%%%%%%%%%%%%%%%%%%%%%%%%%%%%%%%%%%%%%%%%%%%%%
% %%%%%%%%%%%%%%%%%%%%%%%%%%%%%%%%%%%%%%%%%%%%%%%%%%%%%%%%%%%%%%%%%%%%%%%%%%%%%%
% \section{Sample}
%\iffalse
%<*samplemain>
%\fi
%
% The following presents a sample document
% with two chapters, two parts, a title page,
% a compile flag as well as three forwarding files to set the flag.
% It consists of eight |.tex| files:
% \begin{center}
% \begin{tabular}{ll}
% |cdocsamp.tex|&main file\\
% |cdocsch1.tex|&include file for chapter 1\\
% |cdocsch2.tex|&include file for chapter 2\\
% |cdocspt3.tex|&include file for part 3\\
% |cdocspt4.tex|&include file for part 4\\
% |cdocsdrf.tex|&forwarding file for main file in draft mode\\
% |cdocsfi1.tex|&forwarding file for final version of chapter 1\\
% |cdocsfi2.tex|&forwarding file for final version of chapter 2\\
% \end{tabular}
% \end{center}
% Each of the eight files can be compiled directly by the \LaTeX{} compiler.
%
% %%%%%%%%%%%%%%%%%%%%%%%%%%%%%%%%%%%%%%
% \paragraph{Main File.}
%
% The main file is called |cdocsamp.tex|.
%
% Load the \textsf{childdoc} definitions and
% declare the filename for the main document:
%    \begin{macrocode}
\input{childdoc.def}
\childdocmain{}
%    \end{macrocode}

% Optional override for |\version| flag:
%    \begin{macrocode}
%%\ifchilddoc\else\providecommand{\version}{draft}\fi
%    \end{macrocode}

% Define the default values for the |\version| flag
% (|final| for the main file and |draft| for childs):
%    \begin{macrocode}
\ifchilddoc
\providecommand{\version}{draft}
\else
\providecommand{\version}{final}
\fi
%    \end{macrocode}

% Load the standard document class:
%    \begin{macrocode}
\documentclass[12pt]{article}
%    \end{macrocode}

% Start the document body:
%    \begin{macrocode}
\begin{document}
%    \end{macrocode}

% Declare a title page.
% Print title, part of document being processed and version flag:
%    \begin{macrocode}
\addtocounter{page}{-1}
\begin{center}
{\LARGE\bfseries{}childdoc example\par}
\vspace{1cm}
\ifchilddoc
\ifchilddocmanual part\else chapter\fi:
`\childdocname' of `\childdocjob'\par
\else
main document: `\childdocjob'\par
\fi
version: \version\par
\end{center}
\newpage
%    \end{macrocode}

% Manually include selected file,
% otherwise process as usual:
%    \begin{macrocode}
\ifchilddocmanual
\section*{part `\childdocname'}
\input{\childdocname}
\else
%    \end{macrocode}

% Include the two chapters:
%    \begin{macrocode}
\include{cdocsch1}
\include{cdocsch2}
%    \end{macrocode}

% Include the two parts unless only chapters should be displayed:
%    \begin{macrocode}
\ifchilddoc\else
\section{part three}
\input{cdocspt3}
\section{part four}
\input{cdocspt4}
\fi
%    \end{macrocode}

% Process as usual until here:
%    \begin{macrocode}
\fi
%    \end{macrocode}

% End of document body:
%    \begin{macrocode}
\end{document}
%    \end{macrocode}
%\iffalse
%</samplemain>
%\fi
%
% %%%%%%%%%%%%%%%%%%%%%%%%%%%%%%%%%%%%%%
% \paragraph{Chapter Include Files.}
%
% The include files are called |cdocsch1.tex| and |cdocsch2.tex|.
%
%\iffalse
%<*samplechap1|samplechap2>
%\fi

% Optional override for |\version| flag:
%    \begin{macrocode}
%%\providecommand{\version}{final}
%    \end{macrocode}

% Include the main document:
%    \begin{macrocode}
\input{childdoc.def}
\childdocof{cdocsamp}
%    \end{macrocode}

%\iffalse
%</samplechap1|samplechap2>
%\fi
%
%\iffalse
%<*samplechap1>
%\fi
% Some text for chapter 1:
%    \begin{macrocode}
\section{one}
some text in chapter one
%    \end{macrocode}

%\iffalse
%</samplechap1>
%\fi
% Some text for chapter 2:
%\iffalse
%<*samplechap2>
%\fi
%    \begin{macrocode}
\section{two}
more text in chapter two
%    \end{macrocode}

%\iffalse
%</samplechap2>
%\fi
%
% %%%%%%%%%%%%%%%%%%%%%%%%%%%%%%%%%%%%%%
% \paragraph{Part Include Files.}
%
% The include files are called |cdocspt3.tex| and |cdocspt4.tex|.
%
%\iffalse
%<*samplepart3|samplepart4>
%\fi

% Optional override for |\version| flag:
%    \begin{macrocode}
%%\providecommand{\version}{final}
%    \end{macrocode}

% Include the main document:
%    \begin{macrocode}
\input{childdoc.def}
\childdocby{cdocsamp}
%    \end{macrocode}

%\iffalse
%</samplepart3|samplepart4>
%\fi
%
%\iffalse
%<*samplepart3>
%\fi
% Some text for part 3:
%    \begin{macrocode}
some text in part three
%    \end{macrocode}

%\iffalse
%</samplepart3>
%\fi
% Some text for part 4:
%\iffalse
%<*samplepart4>
%\fi
%    \begin{macrocode}
more text in part four
%    \end{macrocode}

%\iffalse
%</samplepart4>
%\fi
%
% %%%%%%%%%%%%%%%%%%%%%%%%%%%%%%%%%%%%%%
% \paragraph{Forwarding for a Complete Draft.}
%
% The following forwarding file |cdocsdrf.tex|
% compiles the main document in draft mode:
%\iffalse
%<*sampledraft>
%\fi
%    \begin{macrocode}
\def\version{draft}
\input{childdoc.def}
\childdocforward{cdocsamp}
%    \end{macrocode}

%\iffalse
%</sampledraft>
%\fi
%
% %%%%%%%%%%%%%%%%%%%%%%%%%%%%%%%%%%%%%%
% \paragraph{Forwarding for Final Version of the Chapters.}
%
% The following forwarding files |cdocsfn1.tex| and |cdocsfn2.tex|
% (with identical content)
% compile the final versions of the child documents
% |cdocsch1.tex| and |cdocsch2.tex|, respectively:
%\iffalse
%<*samplefinal>
%\fi
%    \begin{macrocode}
\def\version{final}
\input{childdoc.def}
\childdocforwardprefix[cdocsamp]{cdocsfn}{cdocsch}
%    \end{macrocode}

%\iffalse
%</samplefinal>
%\fi
%
% %%%%%%%%%%%%%%%%%%%%%%%%%%%%%%%%%%%%%%
% \paragraph{Command Line Processing.}
%
% The following three command lines generate the output files
% |cdocscld|, |cdocscl1| and |cdocscl2|
% which should be identical to
% |cdocsdrf|, |cdocsch1| and |cdocsfn2|, respectively:
% \begin{center}
% \begin{tabular}{l}
% |latex -jobname cdocscld \|\\
% |  "\def\version{draft}\input{childdoc.def}\childdocforward{cdocsamp}"|\\
% |latex -jobname cdocscl1 \|\\
% |  "\input{childdoc.def}\childdocforward[cdocsamp]{cdocsch1}"|\\
% |latex -jobname cdocscl2 \|\\
% |  "\def\version{final}\input{childdoc.def}\childdocforward{cdocsch2}"|
% \end{tabular}
% \end{center}
% Note that the trailing backslash on each first line
% merely continues the input to the second line
% (for convenient cut ant paste).
% Furthermore, the command |latex| can be replaced by any
% of its alternative versions such as |pdflatex|.
%
% %%%%%%%%%%%%%%%%%%%%%%%%%%%%%%%%%%%%%%%%%%%%%%%%%%%%%%%%%%%%%%%%%%%%%%%%%%%%%%
% %%%%%%%%%%%%%%%%%%%%%%%%%%%%%%%%%%%%%%%%%%%%%%%%%%%%%%%%%%%%%%%%%%%%%%%%%%%%%%
% \section{Implementation}
%\iffalse
%<*package>
%\fi
%
% This section describes the definitions file |childdoc.def|.

% The definitions cannot be loaded using |\usepackage| or |\RequirePackage|
% which has a mechanism to prevent loading a style file more than once.
% When loading the definitions by means of |\input|
% multiple instances have to be prevented manually:
%\iffalse
%This code needs to be before the `\ProvidesFile' directive
%which is defined at the beginning of this file.
%Therefore it is also placed there and commented out here.
%</package>
%<*discard>
%\fi
%    \begin{macrocode}
\ifdefined\childdocmain\endinput\fi
%    \end{macrocode}
%\iffalse
%</discard>
%<*package>
%\fi
%
% \macro{\ifchilddoc}
% \macro{\ifchilddocmanual}
% The conditional |\ifchilddoc| tells whether a
% child (true) or main (false) document is being compiled.
% The conditional |\ifchilddocmanual| tells whether
% the |\includeonly| mechanism is used (false) or
% the selection of child files must be performed manually (true).
% The definitions initialise to false:
%    \begin{macrocode}
\newif\ifchilddoc
\newif\ifchilddocmanual
%    \end{macrocode}

% \macro{\childdocname}
% \macro{\childdocjob}
% The macro |\childdocname| stores the name of the main document
% to be compiled. The macro |\childdocjob| stores the name of
% the document on which the \LaTeX{} compiler was originally invoked.
% The content of |\jobname| cannot be compared
% to filenames specified in the source due to different catcodes.
% The following code rescans |\jobname|, stores the result
% in |\childdocname| and saves a copy in |\childdocjob|:
%    \begin{macrocode}
\edef\childdocname{\scantokens\expandafter{\jobname\noexpand}}
\let\childdocjob\childdocname
%    \end{macrocode}

% \macro{\childdocdisable}
% The macro |\childdocdisable| prevents the main file
% from being processed more than once.
% At this stage, the main document command |\childdocmain|
% is assumed to be called once again where it should do nothing.
% Any subsequent call to it should prevent
% a secondary processing of the main document
% It overwrites the forwarding commands
% |\childdocof| and |\childdocforward|
% with empty macros to prevent further inclusions of the main document:
%    \begin{macrocode}
\newcommand{\childdocdisable}
{
  \renewcommand{\childdocmain}[1]{\renewcommand{\childdocmain}[1]{\endinput}}
  \renewcommand{\childdocof}[1]{}
  \renewcommand{\childdocby}[2][]{}
  \renewcommand{\childdocforward}[2][]{}
  \renewcommand{\childdocdisable}{}
}
%    \end{macrocode}

% \macro{\childdocmain}
% The macro |\childdocmain| is to be called at the top of the main file
% with nothing or the main filename (without extension) as argument.
% First, it breaks loops.
% If the argument is not empty and does not match |\childdocname|
% (which is set by the first inclusion of |childdoc.def|),
% |\ifchilddoc| is set to true, |\includeonly| is applied to the child file
% and |\jobname| is set to the main file
% (for proper handling of |.aux| files):
%    \begin{macrocode}
\newcommand{\childdocmain}[1]
{
  \childdocdisable\childdocmain{}
  \if?#1?\else
    \begingroup
      \def\childdoctmp{#1}
      \ifx\childdoctmp\childdocname
        \def\childdoctmp{}
      \else
        \def\childdoctmp
        {
          \childdoctrue
          \includeonly{\childdocname}
          \def\childdocjob{#1}
          \def\jobname{#1}
        }
      \fi
      \expandafter
    \endgroup
    \childdoctmp
  \fi
}
%    \end{macrocode}

% \macro{\childdocof}
% The command |\childdocof| redirects
% compilation to the main file |#1|.
%    \begin{macrocode}
\newcommand{\childdocof}[1]
{
  \childdocdisable
  \childdoctrue
  \includeonly{\childdocname}
  \def\jobname{#1}
  \def\childdocjob{#1}
  \input{#1}
}
%    \end{macrocode}

% \macro{\childdocby}
% The command |\childdocby| ....
%    \begin{macrocode}
\newcommand{\childdocby}[2][]
{
  \childdocdisable
  \childdoctrue
  \childdocmanualtrue
  \if?#1?\else
    \def\jobname{#2}
  \fi
  \def\childdocjob{#2}
  \input{#2}
  \endinput
}
%    \end{macrocode}

% \macro{\childdocforward}
% The command |\childdocforward| redirects
% compilation to the main file or
% (if the optional argument is given) a child file.
% Parameters are set as if the main file
% or a child file starting with |\childdocof| was compiled.
% Then compilation is handed over to the main file:
%    \begin{macrocode}
\newcommand{\childdocforward}[2][]
{
  \begingroup
    \if?#1?
      \def\childdoctmp
      {
        \def\childdocname{#2}
        \def\childdocjob{#2}
        \def\jobname{#2}
        \input{#2}
        \endinput
      }
    \else
      \def\childdoctmp
      {
        \childdocdisable
        \def\childdocname{#2}
        \childdoctrue
        \includeonly{#2}
        \def\childdocjob{#1}
        \def\jobname{#1}
        \input{#1}
        \endinput
      }
    \fi
    \expandafter
  \endgroup
  \childdoctmp
}
%    \end{macrocode}

% \macro{\childdocforwardprefix}
% The command |\childdocforwardprefix| redirects
% compilation to the main or a child file by means of a pattern.
% The prefix |#1| in the current filename is replaced by |#2|
% and the suffix of the current filename is kept
% (it is assumed that the filename does not contain the substring `|~~~|'
% which is used as a delimiter).
% Compilation is handed over to the new file by |\childdocforward|:
%    \begin{macrocode}
\newcommand{\childdocforwardprefix}[3][]
{
  \begingroup
    \def\childdocextract #2##1~~~{\def\childdoctmp{\childdocforward[#1]{#3##1}}}
    \expandafter\childdocextract\childdocname~~~
    \expandafter
  \endgroup
  \childdoctmp
}
%    \end{macrocode}

% \macro{\childdoc}
% The deprecated macro |\childdoc| is a legacy version of |\childdocmain|:
%    \begin{macrocode}
\newcommand{\childdoc}{\childdocmain}
%    \end{macrocode}

% \macro{\childdocredirect}
% The deprecated macro |\childdocredirect| is a legacy version
% of |\childdocforward| and |\childdocforwardprefix|:
%    \begin{macrocode}
\newcommand{\childdocredirect}[2][]
{
  \begingroup
    \if?#1?
      \def\childdoctmp{\childdocforward{#2}}
    \else
      \def\childdoctmp{\childdocforwardprefix{#1}{#2}}
    \fi
    \expandafter
  \endgroup
  \childdoctmp
}
%    \end{macrocode}

%\iffalse
%</package>
%\fi
%
\endinput
|\\
|\childdocmain{}|\\
\end{tabular}
\end{center}
at the very top of the main \LaTeX{} file,
in particular \emph{before} the |\documentclass| statement!
The argument of |\childdocmain| should be left empty
(but it must be present).

%%%%%%%%%%%%%%%%%%%%%%%%%%%%%%%%%%%%%%%%
\DescribeMacro{\childdocof}
Furthermore, add the commands
\begin{center}
\begin{tabular}{l}
|% \iffalse
%
% childdoc.dtx Copyright (C) 2017-2018 Niklas Beisert
%
% This work may be distributed and/or modified under the
% conditions of the LaTeX Project Public License, either version 1.3
% of this license or (at your option) any later version.
% The latest version of this license is in
%   http://www.latex-project.org/lppl.txt
% and version 1.3 or later is part of all distributions of LaTeX
% version 2005/12/01 or later.
%
% This work has the LPPL maintenance status `maintained'.
%
% The Current Maintainer of this work is Niklas Beisert.
%
% This work consists of the files childdoc.dtx and childdoc.ins
% and the derived files childdoc.def and cdocsamp.tex with
% cdocsch1.tex, cdocsch2.tex, cdocsdrf.tex, cdocsfn1.tex, cdocsfn2.tex.
%
%<package>\ifdefined\childdocmain\endinput\fi
%<package>\ProvidesFile{childdoc.def}[2018/12/30 v2.0 child document driver]
%<samplemain>\ProvidesFile{cdocsamp.tex}[2018/12/30 v2.0 sample for childdoc]
%<*driver>
%\ProvidesFile{childdoc.drv}[2018/12/30 v2.0 childdoc reference manual file]
\PassOptionsToClass{10pt,a4paper}{article}
\documentclass{ltxdoc}

\usepackage[margin=35mm]{geometry}
\usepackage{hyperref}
\usepackage{hyperxmp}
\usepackage[usenames]{color}

\hypersetup{colorlinks=true}
\hypersetup{pdfstartview=FitH}
\hypersetup{pdfpagemode=UseNone}
\hypersetup{pdfsource={}}
\hypersetup{pdflang={en-UK}}
\hypersetup{pdfcopyright={Copyright 2017-2018 Niklas Beisert.
  This work may be distributed and/or modified under the
  conditions of the LaTeX Project Public License, either version 1.3
  of this license or (at your option) any later version.}}
\hypersetup{pdflicenseurl={http://www.latex-project.org/lppl.txt}}
\hypersetup{pdfcontactaddress={ETH Zurich, ITP, HIT K,
  Wolfgang-Pauli-Strasse 27}}
\hypersetup{pdfcontactpostcode={8093}}
\hypersetup{pdfcontactcity={Zurich}}
\hypersetup{pdfcontactcountry={Switzerland}}
\hypersetup{pdfcontactemail={nbeisert@itp.phys.ethz.ch}}
\hypersetup{pdfcontacturl={http://people.phys.ethz.ch/\xmptilde nbeisert/}}

\newcommand{\secref}[1]{\hyperref[#1]{section \ref*{#1}}}

\parskip1ex
\parindent0pt
\let\olditemize\itemize
\def\itemize{\olditemize\parskip0pt}

\begin{document}

\title{The \textsf{childdoc} Package}
\hypersetup{pdftitle={The childdoc Package}}
\author{Niklas Beisert\\[2ex]
  Institut f\"ur Theoretische Physik\\
  Eidgen\"ossische Technische Hochschule Z\"urich\\
  Wolfgang-Pauli-Strasse 27, 8093 Z\"urich, Switzerland\\[1ex]
  \href{mailto:nbeisert@itp.phys.ethz.ch}
  {\texttt{nbeisert@itp.phys.ethz.ch}}}
\hypersetup{pdfauthor={Niklas Beisert}}
\hypersetup{pdfsubject={Manual for the LaTeX2e Package childdoc}}
\date{30 December 2018, \textsf{v2.0}}
\maketitle

\begin{abstract}\noindent
\textsf{childdoc} is a \LaTeXe{} package
that enables the direct compilation
of document sections included by |\include|
to individual files.
\end{abstract}

\begingroup
\parskip0ex
\tableofcontents
\endgroup

%%%%%%%%%%%%%%%%%%%%%%%%%%%%%%%%%%%%%%%%%%%%%%%%%%%%%%%%%%%%%%%%%%%%%%%%%%%%%%%%
%%%%%%%%%%%%%%%%%%%%%%%%%%%%%%%%%%%%%%%%%%%%%%%%%%%%%%%%%%%%%%%%%%%%%%%%%%%%%%%%
\section{Introduction}

\LaTeX{} provides a mechanism to structure a large document (such as a book)
into a main file and several child files (containing the chapters)
using the |\include| command.
This mechanism is beneficial for documents
which span hundreds of pages in order to
make the source file(s) more manageable.
Moreover, compilation can be restricted to
selected child files by means of the |\includeonly| command.
The latter feature can be used to reduce the compilation time while editing
(this was significantly more useful in the earlier days of \LaTeX{})
or to generate a smaller document which is easier to navigate.
Another application of |\includeonly| is to generate
documents consisting of selected parts of the complete document.

However, there are a few drawbacks of the plain |\include| mechanism:
\begin{itemize}
\item
The child files cannot be compiled on their own,
they can only be compiled via the main file.
A naive editing environment
(such as a text editor with an option
to have the current file processed by \LaTeX)
may require one to switch to the main file before compiling;
attempting to compile the child file produces errors.
\item
The main file must be modified (each time)
to adjust the |\includeonly| command
to the present needs. This easily leaves the main file in a messy state.
\item
The generated document will always carry the filename
of the main document. This is inconvenient if
several child files are to be compiled and
to be kept for distribution.
\end{itemize}

The present package provides a simple interface
to make child files individually compilable by \LaTeX{}.
Compiling a child file then has the same effect as compiling
the main file with an |\includeonly| command
to select the appropriate child.
Moreover the generated document will carry the name of the child
rather than the main file.
This resolves all three above issues.

This feature is meant to make the editing of books,
thesis documents and lecture notes somewhat more convenient.
However, the package can also be used efficiently for
composing a series of documents (such as exercise sheets)
which are typically distributed individually.
It then assists the author in generating the individual documents
(potentially in different versions)
as well as a document containing the collected series.
Another application is in developing style files
or other kinds of included material
where compilation of the style file could redirect
to a sample or test file.

%%%%%%%%%%%%%%%%%%%%%%%%%%%%%%%%%%%%%%%%%%%%%%%%%%%%%%%%%%%%%%%%%%%%%%%%%%%%%%%%
%%%%%%%%%%%%%%%%%%%%%%%%%%%%%%%%%%%%%%%%%%%%%%%%%%%%%%%%%%%%%%%%%%%%%%%%%%%%%%%%
\section{Usage}

First of all, the package \textsf{childdoc} is \emph{not} a standard
\LaTeXe{} |.sty| style file! Therefore it needs to be invoked in
a non-standard way.

%%%%%%%%%%%%%%%%%%%%%%%%%%%%%%%%%%%%%%%%%%%%%%%%%%%%%%%%%%%%%%%%%%%%%%%%%%%%%%%%
\subsection{Included Files}
\label{sec:include}

%%%%%%%%%%%%%%%%%%%%%%%%%%%%%%%%%%%%%%%%
\DescribeMacro{\childdocmain}
To use the package, add the commands
\begin{center}
\begin{tabular}{l}
|\input{childdoc.def}|\\
|\childdocmain{}|\\
\end{tabular}
\end{center}
at the very top of the main \LaTeX{} file,
in particular \emph{before} the |\documentclass| statement!
The argument of |\childdocmain| should be left empty
(but it must be present).

%%%%%%%%%%%%%%%%%%%%%%%%%%%%%%%%%%%%%%%%
\DescribeMacro{\childdocof}
Furthermore, add the commands
\begin{center}
\begin{tabular}{l}
|\input{childdoc.def}|\\
|\childdocof{|\textit{main}|}|\\
\end{tabular}
\end{center}
at the top of every child file \textit{child}
which is included by |\include{|\textit{child}|}|
from within the main file
(or at least for those files to be compiled individually).
The argument \textit{main} must be the filename of the main file.

There are a couple of
considerations in setting up the main and child documents:

%%%%%%%%%%%%%%%%%%%%%%%%%%%%%%%%%%%%%%%%
\paragraph{Restrictions.}

Please note the following restrictions:
\begin{itemize}
\item
|\childdocmain| must be called with one argument \textit{main}
to ensure compatibility with earlier version of the package.
It must either be empty (|\childdocmain{}|)
or precisely match the filename of the main file in which it is specified.
See \secref{sec:detection} for further information.
\item
The filename \textit{main} must be specified without the |.tex| extension.
\item
The filename \textit{main} is case sensitive
(even in case-insensitive file systems)
due to internal string comparison.
\item
The argument \textit{main} should be fully expanded, it cannot be a macro.
\item
Subdirectories and special characters should be avoided in filenames.
\item
The command |\childdocmain{|\textit{main}|}| must be followed by a whitespace.
It should not be followed immediately by another command
or by a comment mark `|%|'.
This is because the \TeX{} parser reads the token immediately following
the argument of |\childdocmain| and puts it
at the beginning of every child section;
however, a white\-space is ignored.
\end{itemize}

%%%%%%%%%%%%%%%%%%%%%%%%%%%%%%%%%%%%%%%%
\paragraph{Content of Main File.}

It is advisable to place all content in the child files included by |\include|.
Any output contained in the main file will appear in all child documents
unless suppressed manually;
it cannot be suppressed automatically by the |\includeonly| directive
and thus should normally be avoided.
A method to include some content in the main file
by means of conditional processing is described in \secref{sec:conditional}.

%%%%%%%%%%%%%%%%%%%%%%%%%%%%%%%%%%%%%%%%
\paragraph{Page Numbering.}

When only a part of the document is compiled,
the appropriate numbering of pages
(as well as other status parameters)
is determined from the |.aux| files.
The latter contain information from previous passes.
However this information needs to propagate through
all intermediate child documents.
Therefore the page numbering in child documents may well
be inconsistent until the complete document is compiled at least once.

A useful (if unconventional) way to always ensure a consistent
page numbering is to restart the numbering in each child document
and denote the pages by `\textit{child}|.|\textit{page}'
where \textit{child} represents the chapter/section number of the child file.
This can be achieved by the command
|\numberwithin{page}{|\textit{child}|}|
of the \textsf{amsmath} package
where \textit{child} can be |chapter| or |section|
depending on the chosen structuring.
Alternatively, one can modify the macro |\thepage| appropriately
and reset the counter |page| at the start of each child file.

%%%%%%%%%%%%%%%%%%%%%%%%%%%%%%%%%%%%%%%%%%%%%%%%%%%%%%%%%%%%%%%%%%%%%%%%%%%%%%%%
\subsection{Conditional Processing}
\label{sec:conditional}

The package provides a mechanism to compile different versions
of a document. To customise the versions further some conditional processing
can come in handy to distinguish which version is being compiled.
The package provides two macros to describe the compilation context:

%%%%%%%%%%%%%%%%%%%%%%%%%%%%%%%%%%%%%%%%
\DescribeMacro{\ifchilddoc}
The conditional |\ifchilddoc| distinguishes between the compilation of
child documents and the main document:
%
\begin{center}
|\ifchilddoc |\textit{child-code}| |[|\||else |\textit{main-code}]| \||fi|
\end{center}

%%%%%%%%%%%%%%%%%%%%%%%%%%%%%%%%%%%%%%%%
\DescribeMacro{\childdocname}
\DescribeMacro{\childdocjob}
The macro |\childdocname| contains the filename (without extension)
of the main or child file being processed.
Note that |\childdocjob| will always contain the name of the main file.

%%%%%%%%%%%%%%%%%%%%%%%%%%%%%%%%%%%%%%%%
\paragraph{Title Page.}

Conditional processing can be used to include a title or banner page
in the main document when proper precautions are taken.
Importantly, the code in the main file should ensure that the page counter
(as well as other status parameters which are stored in the |.aux| files)
takes the same value after the conditional processing.
Otherwise the page numbers may take divergent values
depending on which part is compiled.

For example, a title page could be declared by:
%
\begin{center}
\begin{tabular}{l}
|\ifchilddoc\||else|\\
|\addtocounter{page}{-1}|\\
\textit{code for title page}\\
|\newpage|\\
|\||fi|
\end{tabular}
\end{center}
%
A banner page for the child documents can be generated by:
%
\begin{center}
\begin{tabular}{l}
|\ifchilddoc|\\
|\addtocounter{page}{-1}|\\
\textit{code for banner page}\\
|\newpage|\\
|\||fi|
\end{tabular}
\end{center}
%
Here one could write a message such as:
\begin{center}
|This is the part \childdocname{} of \childdocjob{}.|
\end{center}

%%%%%%%%%%%%%%%%%%%%%%%%%%%%%%%%%%%%%%%%%%%%%%%%%%%%%%%%%%%%%%%%%%%%%%%%%%%%%%%%
\subsection{Flags}
\label{sec:flags}

The package makes it easy to generate different versions
of the main or child documents.
To this end compilation flags can be defined
and assigned different default values.
They will be particularly useful in conjunction
with the forwarding mechanism described in \secref{sec:forward}.

For example, it may be useful to have a flag |\version|
which can be set to |draft| or |final|.
The document source will contain some conditional code
depending on the value of |\version|.
Suppose further, the flag should default to |final| for the main file
and to |draft| for child files
which is a natural assignment for editing the document.
This is achieved by placing the following code
in the preamble of the main document
(below the |\childdocmain| directive):
%
\begin{center}
\begin{tabular}{l}
|\ifchilddoc|\\
|\providecommand{\version}{draft}|\\
|\||else|\\
|\providecommand{\version}{final}|\\
|\||fi|
\end{tabular}
\end{center}
%
The definition by |\providecommand| makes sure
that previous definitions are not overwritten.
Further statements |\providecommand{\version}{...}|
can thus be added before the above code to override it.

For the main file, one might add a line
(between |\childdocmain| and the above block)
%
\begin{center}
|%\ifchilddoc\||else\providecommand{\version}{draft}\||fi|
\end{center}
%
which can be uncommented to produce a draft version.
Likewise one can add a line to the very top of a child file
(above the |\childdocof{|\textit{main}|}| directive)
%
\begin{center}
|%\providecommand{\version}{final}|
\end{center}
%
which can be uncommented to produce the final version of this child document.

%%%%%%%%%%%%%%%%%%%%%%%%%%%%%%%%%%%%%%%%%%%%%%%%%%%%%%%%%%%%%%%%%%%%%%%%%%%%%%%%
\subsection{Forwarding}
\label{sec:forward}

Different versions of the main or child documents
using compilation flags as described in \secref{sec:flags}
can be (permanently) stored in different files
for convenient compilation, viewing and distribution.
To this end, the package defines a command
to pass on compilation to a different file:

%%%%%%%%%%%%%%%%%%%%%%%%%%%%%%%%%%%%%%%%
\DescribeMacro{\childdocforward}
The command |\childdocforward| redirects processing to
another source file:
%
\begin{center}
\begin{tabular}{l}
|\input{childdoc.def}|\\
|\childdocforward[|\textit{main}|]{|\textit{dest}|}|\\
\end{tabular}
\end{center}
%
The argument \textit{dest} is the destination file
(without extension).
It should be the main file or one of the child files.
Note that further \textsf{childdoc} directives
such as |\childdocof| and |\childdocforward|
in the indicated file will be processed in this form.
The optional argument \textit{main}
passes on directly to the main file \textit{main}
while pretending to compile the child \textit{dest}.
This form behaves as if \textit{dest}
issues |\childdocof{|\textit{main}|}| right away,
and no further \textsf{childdoc} directives will be processed.

%%%%%%%%%%%%%%%%%%%%%%%%%%%%%%%%%%%%%%%%
\DescribeMacro{\...prefix}
In the alternative form |\childdocforwardprefix|,
%
\begin{center}
\begin{tabular}{l}
|\input{childdoc.def}|\\
|\childdocforwardprefix[|\textit{main}|]{|\textit{prefix}|}{|\textit{dest}|}|
\end{tabular}
\end{center}
%
the destination file is determined by a pattern
depending on the current file:
To make this work, the current file must be called
`{\textit{prefix}\hspace{0.2em}\textit{suffix}}'
with \textit{prefix} matching precisely the argument.
Processing is then passed on to the file
`{\textit{dest}\hspace{0.2em}\textit{suffix}}'.
Surely, the same effect is achieved by
directly specifying the
argument `{\textit{dest}\hspace{0.2em}\textit{suffix}}'
in the first form.
However, that requires to set up a different file
for each child. With the alternative form of the command
all these files can have exactly the same content
which simplifies setting them up and maintaining them.

For example, the following file |draft.tex|
with a compilation flag |\version| as described in \secref{sec:flags}
compiles the main document as a draft:
%
\begin{center}
\begin{tabular}{l}
|\def\version{draft}|\\
|\input{childdoc.def}|\\
|\childdocforward{|\textit{main}|}|
\end{tabular}
\end{center}
%
Likewise, the following files |final|\textit{nn}|.tex|
compile the final version of the child document
|child|\textit{nn}|.tex|:
%
\begin{center}
\begin{tabular}{l}
|\def\version{final}|\\
|\input{childdoc.def}|\\
|\childdocforwardprefix{final}{child}|
\end{tabular}
\end{center}
%

Note that when several versions of a main file and/or of each child file
are to be generated, it may be convenient to set up a |Makefile| or
shell script to automatise the process.

%%%%%%%%%%%%%%%%%%%%%%%%%%%%%%%%%%%%%%%%%%%%%%%%%%%%%%%%%%%%%%%%%%%%%%%%%%%%%%%%
\subsection{Command Line Processing}
\label{sec:commandline}

The effect of redirection files can also be achieved by invoking
the \LaTeX{} compiler with a more elaborate command line.
Most conveniently this should be done as part
of a shell script or a |Makefile|.

When using \textsf{childdoc} in the main file, the following
command lines effectively perform a redirection
(note that depending on the shell being used,
backslashes may have to be doubled: `|\|' $\to$ `|\\|'):
%
\begin{center}
|... -jobname "|\textit{target}|" |\\|"|[\textit{flags}]%
|\input{childdoc.def}\childdocforward[|\textit{main}|]{|\textit{dest}|}"|
\end{center}
%
Here \textit{target} is the name of the output file,
\textit{main} is the name of the main file
and \textit{dest} is the name of the main or child file to be processed
(all filenames without extensions).
The optional argument \textit{main} can be omitted
if \textit{main} matches \textit{dest}.
Optionally, compilation \textit{flags} can be defined via |\def| commands.
This command line makes the \TeX{} engine believe
it is compiling the file \textit{target}
whose content is specified as the latter parameter.
The provided code then forwards the processing to
\textit{main} or \textit{dest} as described in \secref{sec:forward}.

%%%%%%%%%%%%%%%%%%%%%%%%%%%%%%%%%%%%%%%%%%%%%%%%%%%%%%%%%%%%%%%%%%%%%%%%%%%%%%%%
\subsection{Include by Input}
\label{sec:input}

Including child documents by |\include| has some restrictions by design.
Most notably, the content of a child document always occupies
its own set of pages; pages cannot be shared between child documents.
Usually, this behaviour makes perfect sense
because each child document contain an essential part of the document.
However, in some situations it may be desirable to compose
a document from a collection of parts
without having mandatory page breaks between then.
For this case, the package
provides a mechanism to include parts
by |\input| which can also be processed individually.
However, by construction this mechanism
requires manual handling of the content to be output.

%%%%%%%%%%%%%%%%%%%%%%%%%%%%%%%%%%%%%%%%
\DescribeMacro{\ifchilddocmanual}
The main file should be prepared as usual, see \secref{sec:include}.
However, the document body must make a distinction
between processing of an individual part and of the main document, e.g.:
%
\begin{center}
\begin{tabular}{l}
|\ifchilddocmanual|\\
|\input{\childdocname}|\\
|\||else|\\
\textit{document body with }|\input{|\textit{part}|}|\\
|\||fi|
\end{tabular}
\end{center}
%
The conditional |\ifchilddocmanual| is true whenever
a part to be included by |\input| is being compiled,
and the name of the part is stored in |\childdocname|.

%%%%%%%%%%%%%%%%%%%%%%%%%%%%%%%%%%%%%%%%
\DescribeMacro{\childdocby}
Each part to be included by |\input| should start with:
%
\begin{center}
\begin{tabular}{l}
|\input{childdoc.def}|\\
|\childdocby{|\textit{main}|}|\\
\end{tabular}
\end{center}
%
The directive |\childdocby| is similar to |\childdocof|
described in \secref{sec:include},
but the subsequent selection of content must be done manually.
To that end, both |\ifchilddoc| and |\ifchilddocmanual|
will be true upon processing of a part,
and the name of the part is stored in |\childdocname|.
Note that |\jobname| will be set to the filename of the current part
so that each part receives an individual |.aux| file
that does not interfere with the |.aux| file(s) of the main document.
This behaviour can be altered by the alternative form
|\childdocby[*]{|\textit{main}|}| (with a non-empty optional argument)
which uses the |.aux| file of the main document
by setting |\jobname| to \textit{main}.

%%%%%%%%%%%%%%%%%%%%%%%%%%%%%%%%%%%%%%%%%%%%%%%%%%%%%%%%%%%%%%%%%%%%%%%%%%%%%%%%
\subsection{Driver Development}
\label{sec:driver}

The \textsf{childdoc} mechanism can also be use for the development
of definition files such as \LaTeX{} styles or classes.
This case differs from the above setup with multiple parts
included by |\include| in that no |\includeonly| should be invoked.
This can be achieved by starting the include file
(before |\ProvidesPackage|) with:
%
\begin{center}
\begin{tabular}{l}
|\input{childdoc.def}|\\
|\childdocforward{|\textit{main}|}|\\
\end{tabular}
\end{center}
%
or alternatively with:
%
\begin{center}
\begin{tabular}{l}
|\input{childdoc.def}|\\
|\childdocby{|\textit{main}|}|\\
\end{tabular}
\end{center}
%
Both forms have slightly different effects as described above.
The main file is prepared as usual, see \secref{sec:include}.

%%%%%%%%%%%%%%%%%%%%%%%%%%%%%%%%%%%%%%%%%%%%%%%%%%%%%%%%%%%%%%%%%%%%%%%%%%%%%%%%
\subsection{Legacy Detection}
\label{sec:detection}

The directive |\childdocmain| in the main file can detect
whether the complete document or merely a child is to be compiled
even without using the directive |\childdocof|.
This method is deprecated because it is less robust
and there is no compelling reason to use it;
it is merely provided for backward compatibility
and it may be removed in future versions.

If the detection mechanism is to be used,
it is mandatory to correctly specify
the filename of the main file as the argument of |\childdocmain|:
%
\begin{center}
\begin{tabular}{l}
|\input{childdoc.def}|\\
|\childdocmain{|\textit{main}|}|\\
\end{tabular}
\end{center}
%
If |\jobname| does not match the argument \textit{main} of |\childdocmain|,
it is assumed that |\jobname| points to the child file to be compiled.
When using |\childdocmain| with the main file specified as argument,
it suffices to start a child file
with just |\input{|\textit{main}|}|
without loading of the package and using |\childdocof|.
If instead all processing is done
with the appropriate \textsf{childdoc} directives,
the argument of \textit{main} of |\childdocmain| can be empty.

An alternative version of the command line processing described
in \secref{sec:commandline} using the detection mechanism reads:
%
\begin{center}
|... -jobname "|\textit{target}|" "|[\textit{flags}]%
[|\def\jobname{|\textit{dest}|}|]|\input{|\textit{main}|}"|
\end{center}

%%%%%%%%%%%%%%%%%%%%%%%%%%%%%%%%%%%%%%%%%%%%%%%%%%%%%%%%%%%%%%%%%%%%%%%%%%%%%%%%
\subsection{Manual Code}
\label{sec:manual}

In case one cannot be certain whether the definitions file |childdoc.def|
is installed on the target \TeX{} distribution
and one prefers not to ship it,
it is conceivable to paste a few relevant commands into the sources.

To that end, drop all statements |\input{childdoc.def}|
and perform the replacements as outlined below.
Instead of |\childdocmain{|\textit{main}|}| add the following code
to the top of the main file:
%
\begin{center}
\begin{tabular}{l}
|\||ifdefined\childdocname\endinput\||fi\newif\ifchilddoc|\\
|\edef\childdocname{\scantokens\expandafter{\jobname\noexpand}}|\\
|\def\childdocmain{|\textit{main}|}\||ifx\childdocmain\childdocname\||else|\\
|\childdoctrue\includeonly{\childdocname}\let\jobname\childdocmain\||fi|\\
\end{tabular}
\end{center}
%
Instead of |\childdocof{|\textit{main}|}| just include the main file
at the top of each child file:
%
\begin{center}
|\input{|\textit{main}|}|
\end{center}
%
A simple redirection |\childdocforward{|\textit{dest}|}| is achieved by:
%
\begin{center}
|\def\jobname{|\textit{dest}|}\input{\jobname}|
\end{center}
%
The redirection with prefix
|\childdocforwardprefix[|\textit{prefix}|]{|\textit{dest}|}|
is accomplished by:
%
\begin{center}
\begin{tabular}{l}
|{\edef\jobname{\scantokens\expandafter{\jobname\noexpand}}|\\
|\def\redirectjob |\textit{prefix}|#1~~~{\gdef\jobname{|\textit{dest}|#1}}|\\
|\expandafter\redirectjob\jobname~~~}\input{\jobname}|
\end{tabular}
\end{center}

In an alternative approach,
child documents can be compiled by a specific command line
without additional code or specific definitions:
%
\begin{center}
|... -jobname "|\textit{target}|" "|[\textit{flags}]%
|\includeonly{|\textit{dest}|}\input{|\textit{main}|}"|
\end{center}
%

%%%%%%%%%%%%%%%%%%%%%%%%%%%%%%%%%%%%%%%%%%%%%%%%%%%%%%%%%%%%%%%%%%%%%%%%%%%%%%%%
%%%%%%%%%%%%%%%%%%%%%%%%%%%%%%%%%%%%%%%%%%%%%%%%%%%%%%%%%%%%%%%%%%%%%%%%%%%%%%%%
\section{Information}

%%%%%%%%%%%%%%%%%%%%%%%%%%%%%%%%%%%%%%%%%%%%%%%%%%%%%%%%%%%%%%%%%%%%%%%%%%%%%%%%
\subsection{Copyright}

Copyright \copyright{} 2017--2018 Niklas Beisert

This work may be distributed and/or modified under the
conditions of the \LaTeX{} Project Public License, either version 1.3
of this license or (at your option) any later version.
The latest version of this license is in
  \url{http://www.latex-project.org/lppl.txt}
and version 1.3 or later is part of all distributions of \LaTeX{}
version 2005/12/01 or later.

This work has the LPPL maintenance status `maintained'.

The Current Maintainer of this work is Niklas Beisert.

This work consists of the files |README.txt|, |childdoc.ins| and |childdoc.dtx|
as well as the derived files |childdoc.def|, |cdocsamp.tex|
with |cdocsch1.tex|, |cdocsch2.tex|, |cdocspt3.tex|, |cdocspt4.tex|,
|cdocsdrf.tex|, |cdocsfn1.tex|, |cdocsfn2.tex|
as well as |childdoc.pdf|.

%%%%%%%%%%%%%%%%%%%%%%%%%%%%%%%%%%%%%%%%%%%%%%%%%%%%%%%%%%%%%%%%%%%%%%%%%%%%%%%%
\subsection{Files and Installation}

The package consists of the files:
%
\begin{center}
\begin{tabular}{ll}
    |README.txt|   & readme file \\
    |childdoc.ins| & installation file \\
    |childdoc.dtx| & source file \\
    |childdoc.def| & definition file \\
    |cdocsamp.tex| & sample main file \\
    |cdocsch1.tex| & sample include file \\
    |cdocsch2.tex| & sample include file \\
    |cdocspt3.tex| & sample part file \\
    |cdocspt4.tex| & sample part file \\
    |cdocsdrf.tex| & sample redirection file \\
    |cdocsfn1.tex| & sample redirection file \\
    |cdocsfn2.tex| & sample redirection file \\
    |childdoc.pdf| & manual
\end{tabular}
\end{center}
%
The distribution consists of the files
|README.txt|, |childdoc.ins| and |childdoc.dtx|.
%
\begin{itemize}
\item
Run (pdf)\LaTeX{} on |childdoc.dtx|
to compile the manual |childdoc.pdf| (this file).
\item
Run \LaTeX{} on |childdoc.ins| to create the definitions file |childdoc.def|
and the sample |cdocsamp.tex| with include files
|cdocsch1.tex|, |cdocsch2.tex|, |cdocspt3.tex|, |cdocspt4.tex|,
|cdocsdrf.tex|, |cdocsfn1.tex|, |cdocsfn2.tex|.
Then copy the file |childdoc.def| to an appropriate directory of your \LaTeX{}
distribution, e.g.\ \textit{texmf-root}|/tex/latex/childdoc|.
\end{itemize}

%%%%%%%%%%%%%%%%%%%%%%%%%%%%%%%%%%%%%%%%%%%%%%%%%%%%%%%%%%%%%%%%%%%%%%%%%%%%%%%%
\subsection{Related CTAN Packages}

There are several other packages which offer a similar functionality:
%
\begin{itemize}
\item
The packages
\href{http://ctan.org/pkg/docmute}{\textsf{docmute}},
\href{http://ctan.org/pkg/includex}{\textsf{includex}} and
\href{http://ctan.org/pkg/standalone}{\textsf{standalone}}
provide commands to include only the document body of
a child file thus allowing both files to be compiled individually.
\item
The packages \href{http://ctan.org/pkg/subdocs}{\textsf{subdocs}}
and \href{http://ctan.org/pkg/subfiles}{\textsf{subfiles}}
provide structures in which the main and child documents can be
encapsulated and allowing them to be compiled individually.
The inclusion mechanism is different from the conventional |\include|.
\item
The package \href{http://ctan.org/pkg/combine}{\textsf{combine}}
is an elaborate solution to combine several documents into one.
\end{itemize}
%
See also the CTAN topic \href{http://ctan.org/topic/subdocs}{\textsf{subdocs}}
for further related packages.
The present package differs from the above solutions in that
a document structure constructed with the conventional |\include| mechanism
just needs two extra commands at the top of every file
such that all constituent files can be compiled individually.

%%%%%%%%%%%%%%%%%%%%%%%%%%%%%%%%%%%%%%%%%%%%%%%%%%%%%%%%%%%%%%%%%%%%%%%%%%%%%%%%
%\subsection{Feature Suggestions}
%
%The following is a list of features which may be useful for future
%versions of this package:
%%
%\begin{itemize}
%\item
%\ldots
%\end{itemize}

%%%%%%%%%%%%%%%%%%%%%%%%%%%%%%%%%%%%%%%%%%%%%%%%%%%%%%%%%%%%%%%%%%%%%%%%%%%%%%%%
\subsection{Revision History}

%%%%%%%%%%%%%%%%%%%%%%%%%%%%%%%%%%%%%%%%
\paragraph{v2.0:} 2018/12/30

\begin{itemize}
\item
immediate forward processing
\item
added |\childdocby| mechanism
\item
manual restructured
\end{itemize}

%%%%%%%%%%%%%%%%%%%%%%%%%%%%%%%%%%%%%%%%
\paragraph{v1.6:} 2018/01/17

\begin{itemize}
\item
application for development of include files
\item
corrections to manual
\end{itemize}

%%%%%%%%%%%%%%%%%%%%%%%%%%%%%%%%%%%%%%%%
\paragraph{v1.5:} 2017/05/21

\begin{itemize}
\item
more complete structuring introduced
\item
|\childdocof| introduced
\item
|\childdoc| renamed to |\childdocmain|
\item
|\childredirect| renamed to |\childdocforward| and |\childdocforwardprefix|
and functionality expanded
\end{itemize}

%%%%%%%%%%%%%%%%%%%%%%%%%%%%%%%%%%%%%%%%
\paragraph{v1.0:} 2017/04/27

\begin{itemize}
\item
manual and install package
\item
first version published on CTAN
\end{itemize}

%%%%%%%%%%%%%%%%%%%%%%%%%%%%%%%%%%%%%%%%
\paragraph{v0.6:} 2017/04/26

\begin{itemize}
\item
redirection mechanism added
\end{itemize}

%%%%%%%%%%%%%%%%%%%%%%%%%%%%%%%%%%%%%%%%
\paragraph{v0.5:} 2017/04/26

\begin{itemize}
\item
functionality in definition file
\end{itemize}


%%%%%%%%%%%%%%%%%%%%%%%%%%%%%%%%%%%%%%%%%%%%%%%%%%%%%%%%%%%%%%%%%%%%%%%%%%%%%%%%
%%%%%%%%%%%%%%%%%%%%%%%%%%%%%%%%%%%%%%%%%%%%%%%%%%%%%%%%%%%%%%%%%%%%%%%%%%%%%%%%
%%%%%%%%%%%%%%%%%%%%%%%%%%%%%%%%%%%%%%%%%%%%%%%%%%%%%%%%%%%%%%%%%%%%%%%%%%%%%%%%
\appendix

\settowidth\MacroIndent{\rmfamily\scriptsize 000\ }

 \DocInput{childdoc.dtx}

\end{document}
%</driver>
% \fi
%
% %%%%%%%%%%%%%%%%%%%%%%%%%%%%%%%%%%%%%%%%%%%%%%%%%%%%%%%%%%%%%%%%%%%%%%%%%%%%%%
% %%%%%%%%%%%%%%%%%%%%%%%%%%%%%%%%%%%%%%%%%%%%%%%%%%%%%%%%%%%%%%%%%%%%%%%%%%%%%%
% \section{Sample}
%\iffalse
%<*samplemain>
%\fi
%
% The following presents a sample document
% with two chapters, two parts, a title page,
% a compile flag as well as three forwarding files to set the flag.
% It consists of eight |.tex| files:
% \begin{center}
% \begin{tabular}{ll}
% |cdocsamp.tex|&main file\\
% |cdocsch1.tex|&include file for chapter 1\\
% |cdocsch2.tex|&include file for chapter 2\\
% |cdocspt3.tex|&include file for part 3\\
% |cdocspt4.tex|&include file for part 4\\
% |cdocsdrf.tex|&forwarding file for main file in draft mode\\
% |cdocsfi1.tex|&forwarding file for final version of chapter 1\\
% |cdocsfi2.tex|&forwarding file for final version of chapter 2\\
% \end{tabular}
% \end{center}
% Each of the eight files can be compiled directly by the \LaTeX{} compiler.
%
% %%%%%%%%%%%%%%%%%%%%%%%%%%%%%%%%%%%%%%
% \paragraph{Main File.}
%
% The main file is called |cdocsamp.tex|.
%
% Load the \textsf{childdoc} definitions and
% declare the filename for the main document:
%    \begin{macrocode}
\input{childdoc.def}
\childdocmain{}
%    \end{macrocode}

% Optional override for |\version| flag:
%    \begin{macrocode}
%%\ifchilddoc\else\providecommand{\version}{draft}\fi
%    \end{macrocode}

% Define the default values for the |\version| flag
% (|final| for the main file and |draft| for childs):
%    \begin{macrocode}
\ifchilddoc
\providecommand{\version}{draft}
\else
\providecommand{\version}{final}
\fi
%    \end{macrocode}

% Load the standard document class:
%    \begin{macrocode}
\documentclass[12pt]{article}
%    \end{macrocode}

% Start the document body:
%    \begin{macrocode}
\begin{document}
%    \end{macrocode}

% Declare a title page.
% Print title, part of document being processed and version flag:
%    \begin{macrocode}
\addtocounter{page}{-1}
\begin{center}
{\LARGE\bfseries{}childdoc example\par}
\vspace{1cm}
\ifchilddoc
\ifchilddocmanual part\else chapter\fi:
`\childdocname' of `\childdocjob'\par
\else
main document: `\childdocjob'\par
\fi
version: \version\par
\end{center}
\newpage
%    \end{macrocode}

% Manually include selected file,
% otherwise process as usual:
%    \begin{macrocode}
\ifchilddocmanual
\section*{part `\childdocname'}
\input{\childdocname}
\else
%    \end{macrocode}

% Include the two chapters:
%    \begin{macrocode}
\include{cdocsch1}
\include{cdocsch2}
%    \end{macrocode}

% Include the two parts unless only chapters should be displayed:
%    \begin{macrocode}
\ifchilddoc\else
\section{part three}
\input{cdocspt3}
\section{part four}
\input{cdocspt4}
\fi
%    \end{macrocode}

% Process as usual until here:
%    \begin{macrocode}
\fi
%    \end{macrocode}

% End of document body:
%    \begin{macrocode}
\end{document}
%    \end{macrocode}
%\iffalse
%</samplemain>
%\fi
%
% %%%%%%%%%%%%%%%%%%%%%%%%%%%%%%%%%%%%%%
% \paragraph{Chapter Include Files.}
%
% The include files are called |cdocsch1.tex| and |cdocsch2.tex|.
%
%\iffalse
%<*samplechap1|samplechap2>
%\fi

% Optional override for |\version| flag:
%    \begin{macrocode}
%%\providecommand{\version}{final}
%    \end{macrocode}

% Include the main document:
%    \begin{macrocode}
\input{childdoc.def}
\childdocof{cdocsamp}
%    \end{macrocode}

%\iffalse
%</samplechap1|samplechap2>
%\fi
%
%\iffalse
%<*samplechap1>
%\fi
% Some text for chapter 1:
%    \begin{macrocode}
\section{one}
some text in chapter one
%    \end{macrocode}

%\iffalse
%</samplechap1>
%\fi
% Some text for chapter 2:
%\iffalse
%<*samplechap2>
%\fi
%    \begin{macrocode}
\section{two}
more text in chapter two
%    \end{macrocode}

%\iffalse
%</samplechap2>
%\fi
%
% %%%%%%%%%%%%%%%%%%%%%%%%%%%%%%%%%%%%%%
% \paragraph{Part Include Files.}
%
% The include files are called |cdocspt3.tex| and |cdocspt4.tex|.
%
%\iffalse
%<*samplepart3|samplepart4>
%\fi

% Optional override for |\version| flag:
%    \begin{macrocode}
%%\providecommand{\version}{final}
%    \end{macrocode}

% Include the main document:
%    \begin{macrocode}
\input{childdoc.def}
\childdocby{cdocsamp}
%    \end{macrocode}

%\iffalse
%</samplepart3|samplepart4>
%\fi
%
%\iffalse
%<*samplepart3>
%\fi
% Some text for part 3:
%    \begin{macrocode}
some text in part three
%    \end{macrocode}

%\iffalse
%</samplepart3>
%\fi
% Some text for part 4:
%\iffalse
%<*samplepart4>
%\fi
%    \begin{macrocode}
more text in part four
%    \end{macrocode}

%\iffalse
%</samplepart4>
%\fi
%
% %%%%%%%%%%%%%%%%%%%%%%%%%%%%%%%%%%%%%%
% \paragraph{Forwarding for a Complete Draft.}
%
% The following forwarding file |cdocsdrf.tex|
% compiles the main document in draft mode:
%\iffalse
%<*sampledraft>
%\fi
%    \begin{macrocode}
\def\version{draft}
\input{childdoc.def}
\childdocforward{cdocsamp}
%    \end{macrocode}

%\iffalse
%</sampledraft>
%\fi
%
% %%%%%%%%%%%%%%%%%%%%%%%%%%%%%%%%%%%%%%
% \paragraph{Forwarding for Final Version of the Chapters.}
%
% The following forwarding files |cdocsfn1.tex| and |cdocsfn2.tex|
% (with identical content)
% compile the final versions of the child documents
% |cdocsch1.tex| and |cdocsch2.tex|, respectively:
%\iffalse
%<*samplefinal>
%\fi
%    \begin{macrocode}
\def\version{final}
\input{childdoc.def}
\childdocforwardprefix[cdocsamp]{cdocsfn}{cdocsch}
%    \end{macrocode}

%\iffalse
%</samplefinal>
%\fi
%
% %%%%%%%%%%%%%%%%%%%%%%%%%%%%%%%%%%%%%%
% \paragraph{Command Line Processing.}
%
% The following three command lines generate the output files
% |cdocscld|, |cdocscl1| and |cdocscl2|
% which should be identical to
% |cdocsdrf|, |cdocsch1| and |cdocsfn2|, respectively:
% \begin{center}
% \begin{tabular}{l}
% |latex -jobname cdocscld \|\\
% |  "\def\version{draft}\input{childdoc.def}\childdocforward{cdocsamp}"|\\
% |latex -jobname cdocscl1 \|\\
% |  "\input{childdoc.def}\childdocforward[cdocsamp]{cdocsch1}"|\\
% |latex -jobname cdocscl2 \|\\
% |  "\def\version{final}\input{childdoc.def}\childdocforward{cdocsch2}"|
% \end{tabular}
% \end{center}
% Note that the trailing backslash on each first line
% merely continues the input to the second line
% (for convenient cut ant paste).
% Furthermore, the command |latex| can be replaced by any
% of its alternative versions such as |pdflatex|.
%
% %%%%%%%%%%%%%%%%%%%%%%%%%%%%%%%%%%%%%%%%%%%%%%%%%%%%%%%%%%%%%%%%%%%%%%%%%%%%%%
% %%%%%%%%%%%%%%%%%%%%%%%%%%%%%%%%%%%%%%%%%%%%%%%%%%%%%%%%%%%%%%%%%%%%%%%%%%%%%%
% \section{Implementation}
%\iffalse
%<*package>
%\fi
%
% This section describes the definitions file |childdoc.def|.

% The definitions cannot be loaded using |\usepackage| or |\RequirePackage|
% which has a mechanism to prevent loading a style file more than once.
% When loading the definitions by means of |\input|
% multiple instances have to be prevented manually:
%\iffalse
%This code needs to be before the `\ProvidesFile' directive
%which is defined at the beginning of this file.
%Therefore it is also placed there and commented out here.
%</package>
%<*discard>
%\fi
%    \begin{macrocode}
\ifdefined\childdocmain\endinput\fi
%    \end{macrocode}
%\iffalse
%</discard>
%<*package>
%\fi
%
% \macro{\ifchilddoc}
% \macro{\ifchilddocmanual}
% The conditional |\ifchilddoc| tells whether a
% child (true) or main (false) document is being compiled.
% The conditional |\ifchilddocmanual| tells whether
% the |\includeonly| mechanism is used (false) or
% the selection of child files must be performed manually (true).
% The definitions initialise to false:
%    \begin{macrocode}
\newif\ifchilddoc
\newif\ifchilddocmanual
%    \end{macrocode}

% \macro{\childdocname}
% \macro{\childdocjob}
% The macro |\childdocname| stores the name of the main document
% to be compiled. The macro |\childdocjob| stores the name of
% the document on which the \LaTeX{} compiler was originally invoked.
% The content of |\jobname| cannot be compared
% to filenames specified in the source due to different catcodes.
% The following code rescans |\jobname|, stores the result
% in |\childdocname| and saves a copy in |\childdocjob|:
%    \begin{macrocode}
\edef\childdocname{\scantokens\expandafter{\jobname\noexpand}}
\let\childdocjob\childdocname
%    \end{macrocode}

% \macro{\childdocdisable}
% The macro |\childdocdisable| prevents the main file
% from being processed more than once.
% At this stage, the main document command |\childdocmain|
% is assumed to be called once again where it should do nothing.
% Any subsequent call to it should prevent
% a secondary processing of the main document
% It overwrites the forwarding commands
% |\childdocof| and |\childdocforward|
% with empty macros to prevent further inclusions of the main document:
%    \begin{macrocode}
\newcommand{\childdocdisable}
{
  \renewcommand{\childdocmain}[1]{\renewcommand{\childdocmain}[1]{\endinput}}
  \renewcommand{\childdocof}[1]{}
  \renewcommand{\childdocby}[2][]{}
  \renewcommand{\childdocforward}[2][]{}
  \renewcommand{\childdocdisable}{}
}
%    \end{macrocode}

% \macro{\childdocmain}
% The macro |\childdocmain| is to be called at the top of the main file
% with nothing or the main filename (without extension) as argument.
% First, it breaks loops.
% If the argument is not empty and does not match |\childdocname|
% (which is set by the first inclusion of |childdoc.def|),
% |\ifchilddoc| is set to true, |\includeonly| is applied to the child file
% and |\jobname| is set to the main file
% (for proper handling of |.aux| files):
%    \begin{macrocode}
\newcommand{\childdocmain}[1]
{
  \childdocdisable\childdocmain{}
  \if?#1?\else
    \begingroup
      \def\childdoctmp{#1}
      \ifx\childdoctmp\childdocname
        \def\childdoctmp{}
      \else
        \def\childdoctmp
        {
          \childdoctrue
          \includeonly{\childdocname}
          \def\childdocjob{#1}
          \def\jobname{#1}
        }
      \fi
      \expandafter
    \endgroup
    \childdoctmp
  \fi
}
%    \end{macrocode}

% \macro{\childdocof}
% The command |\childdocof| redirects
% compilation to the main file |#1|.
%    \begin{macrocode}
\newcommand{\childdocof}[1]
{
  \childdocdisable
  \childdoctrue
  \includeonly{\childdocname}
  \def\jobname{#1}
  \def\childdocjob{#1}
  \input{#1}
}
%    \end{macrocode}

% \macro{\childdocby}
% The command |\childdocby| ....
%    \begin{macrocode}
\newcommand{\childdocby}[2][]
{
  \childdocdisable
  \childdoctrue
  \childdocmanualtrue
  \if?#1?\else
    \def\jobname{#2}
  \fi
  \def\childdocjob{#2}
  \input{#2}
  \endinput
}
%    \end{macrocode}

% \macro{\childdocforward}
% The command |\childdocforward| redirects
% compilation to the main file or
% (if the optional argument is given) a child file.
% Parameters are set as if the main file
% or a child file starting with |\childdocof| was compiled.
% Then compilation is handed over to the main file:
%    \begin{macrocode}
\newcommand{\childdocforward}[2][]
{
  \begingroup
    \if?#1?
      \def\childdoctmp
      {
        \def\childdocname{#2}
        \def\childdocjob{#2}
        \def\jobname{#2}
        \input{#2}
        \endinput
      }
    \else
      \def\childdoctmp
      {
        \childdocdisable
        \def\childdocname{#2}
        \childdoctrue
        \includeonly{#2}
        \def\childdocjob{#1}
        \def\jobname{#1}
        \input{#1}
        \endinput
      }
    \fi
    \expandafter
  \endgroup
  \childdoctmp
}
%    \end{macrocode}

% \macro{\childdocforwardprefix}
% The command |\childdocforwardprefix| redirects
% compilation to the main or a child file by means of a pattern.
% The prefix |#1| in the current filename is replaced by |#2|
% and the suffix of the current filename is kept
% (it is assumed that the filename does not contain the substring `|~~~|'
% which is used as a delimiter).
% Compilation is handed over to the new file by |\childdocforward|:
%    \begin{macrocode}
\newcommand{\childdocforwardprefix}[3][]
{
  \begingroup
    \def\childdocextract #2##1~~~{\def\childdoctmp{\childdocforward[#1]{#3##1}}}
    \expandafter\childdocextract\childdocname~~~
    \expandafter
  \endgroup
  \childdoctmp
}
%    \end{macrocode}

% \macro{\childdoc}
% The deprecated macro |\childdoc| is a legacy version of |\childdocmain|:
%    \begin{macrocode}
\newcommand{\childdoc}{\childdocmain}
%    \end{macrocode}

% \macro{\childdocredirect}
% The deprecated macro |\childdocredirect| is a legacy version
% of |\childdocforward| and |\childdocforwardprefix|:
%    \begin{macrocode}
\newcommand{\childdocredirect}[2][]
{
  \begingroup
    \if?#1?
      \def\childdoctmp{\childdocforward{#2}}
    \else
      \def\childdoctmp{\childdocforwardprefix{#1}{#2}}
    \fi
    \expandafter
  \endgroup
  \childdoctmp
}
%    \end{macrocode}

%\iffalse
%</package>
%\fi
%
\endinput
|\\
|\childdocof{|\textit{main}|}|\\
\end{tabular}
\end{center}
at the top of every child file \textit{child}
which is included by |\include{|\textit{child}|}|
from within the main file
(or at least for those files to be compiled individually).
The argument \textit{main} must be the filename of the main file.

There are a couple of
considerations in setting up the main and child documents:

%%%%%%%%%%%%%%%%%%%%%%%%%%%%%%%%%%%%%%%%
\paragraph{Restrictions.}

Please note the following restrictions:
\begin{itemize}
\item
|\childdocmain| must be called with one argument \textit{main}
to ensure compatibility with earlier version of the package.
It must either be empty (|\childdocmain{}|)
or precisely match the filename of the main file in which it is specified.
See \secref{sec:detection} for further information.
\item
The filename \textit{main} must be specified without the |.tex| extension.
\item
The filename \textit{main} is case sensitive
(even in case-insensitive file systems)
due to internal string comparison.
\item
The argument \textit{main} should be fully expanded, it cannot be a macro.
\item
Subdirectories and special characters should be avoided in filenames.
\item
The command |\childdocmain{|\textit{main}|}| must be followed by a whitespace.
It should not be followed immediately by another command
or by a comment mark `|%|'.
This is because the \TeX{} parser reads the token immediately following
the argument of |\childdocmain| and puts it
at the beginning of every child section;
however, a white\-space is ignored.
\end{itemize}

%%%%%%%%%%%%%%%%%%%%%%%%%%%%%%%%%%%%%%%%
\paragraph{Content of Main File.}

It is advisable to place all content in the child files included by |\include|.
Any output contained in the main file will appear in all child documents
unless suppressed manually;
it cannot be suppressed automatically by the |\includeonly| directive
and thus should normally be avoided.
A method to include some content in the main file
by means of conditional processing is described in \secref{sec:conditional}.

%%%%%%%%%%%%%%%%%%%%%%%%%%%%%%%%%%%%%%%%
\paragraph{Page Numbering.}

When only a part of the document is compiled,
the appropriate numbering of pages
(as well as other status parameters)
is determined from the |.aux| files.
The latter contain information from previous passes.
However this information needs to propagate through
all intermediate child documents.
Therefore the page numbering in child documents may well
be inconsistent until the complete document is compiled at least once.

A useful (if unconventional) way to always ensure a consistent
page numbering is to restart the numbering in each child document
and denote the pages by `\textit{child}|.|\textit{page}'
where \textit{child} represents the chapter/section number of the child file.
This can be achieved by the command
|\numberwithin{page}{|\textit{child}|}|
of the \textsf{amsmath} package
where \textit{child} can be |chapter| or |section|
depending on the chosen structuring.
Alternatively, one can modify the macro |\thepage| appropriately
and reset the counter |page| at the start of each child file.

%%%%%%%%%%%%%%%%%%%%%%%%%%%%%%%%%%%%%%%%%%%%%%%%%%%%%%%%%%%%%%%%%%%%%%%%%%%%%%%%
\subsection{Conditional Processing}
\label{sec:conditional}

The package provides a mechanism to compile different versions
of a document. To customise the versions further some conditional processing
can come in handy to distinguish which version is being compiled.
The package provides two macros to describe the compilation context:

%%%%%%%%%%%%%%%%%%%%%%%%%%%%%%%%%%%%%%%%
\DescribeMacro{\ifchilddoc}
The conditional |\ifchilddoc| distinguishes between the compilation of
child documents and the main document:
%
\begin{center}
|\ifchilddoc |\textit{child-code}| |[|\||else |\textit{main-code}]| \||fi|
\end{center}

%%%%%%%%%%%%%%%%%%%%%%%%%%%%%%%%%%%%%%%%
\DescribeMacro{\childdocname}
\DescribeMacro{\childdocjob}
The macro |\childdocname| contains the filename (without extension)
of the main or child file being processed.
Note that |\childdocjob| will always contain the name of the main file.

%%%%%%%%%%%%%%%%%%%%%%%%%%%%%%%%%%%%%%%%
\paragraph{Title Page.}

Conditional processing can be used to include a title or banner page
in the main document when proper precautions are taken.
Importantly, the code in the main file should ensure that the page counter
(as well as other status parameters which are stored in the |.aux| files)
takes the same value after the conditional processing.
Otherwise the page numbers may take divergent values
depending on which part is compiled.

For example, a title page could be declared by:
%
\begin{center}
\begin{tabular}{l}
|\ifchilddoc\||else|\\
|\addtocounter{page}{-1}|\\
\textit{code for title page}\\
|\newpage|\\
|\||fi|
\end{tabular}
\end{center}
%
A banner page for the child documents can be generated by:
%
\begin{center}
\begin{tabular}{l}
|\ifchilddoc|\\
|\addtocounter{page}{-1}|\\
\textit{code for banner page}\\
|\newpage|\\
|\||fi|
\end{tabular}
\end{center}
%
Here one could write a message such as:
\begin{center}
|This is the part \childdocname{} of \childdocjob{}.|
\end{center}

%%%%%%%%%%%%%%%%%%%%%%%%%%%%%%%%%%%%%%%%%%%%%%%%%%%%%%%%%%%%%%%%%%%%%%%%%%%%%%%%
\subsection{Flags}
\label{sec:flags}

The package makes it easy to generate different versions
of the main or child documents.
To this end compilation flags can be defined
and assigned different default values.
They will be particularly useful in conjunction
with the forwarding mechanism described in \secref{sec:forward}.

For example, it may be useful to have a flag |\version|
which can be set to |draft| or |final|.
The document source will contain some conditional code
depending on the value of |\version|.
Suppose further, the flag should default to |final| for the main file
and to |draft| for child files
which is a natural assignment for editing the document.
This is achieved by placing the following code
in the preamble of the main document
(below the |\childdocmain| directive):
%
\begin{center}
\begin{tabular}{l}
|\ifchilddoc|\\
|\providecommand{\version}{draft}|\\
|\||else|\\
|\providecommand{\version}{final}|\\
|\||fi|
\end{tabular}
\end{center}
%
The definition by |\providecommand| makes sure
that previous definitions are not overwritten.
Further statements |\providecommand{\version}{...}|
can thus be added before the above code to override it.

For the main file, one might add a line
(between |\childdocmain| and the above block)
%
\begin{center}
|%\ifchilddoc\||else\providecommand{\version}{draft}\||fi|
\end{center}
%
which can be uncommented to produce a draft version.
Likewise one can add a line to the very top of a child file
(above the |\childdocof{|\textit{main}|}| directive)
%
\begin{center}
|%\providecommand{\version}{final}|
\end{center}
%
which can be uncommented to produce the final version of this child document.

%%%%%%%%%%%%%%%%%%%%%%%%%%%%%%%%%%%%%%%%%%%%%%%%%%%%%%%%%%%%%%%%%%%%%%%%%%%%%%%%
\subsection{Forwarding}
\label{sec:forward}

Different versions of the main or child documents
using compilation flags as described in \secref{sec:flags}
can be (permanently) stored in different files
for convenient compilation, viewing and distribution.
To this end, the package defines a command
to pass on compilation to a different file:

%%%%%%%%%%%%%%%%%%%%%%%%%%%%%%%%%%%%%%%%
\DescribeMacro{\childdocforward}
The command |\childdocforward| redirects processing to
another source file:
%
\begin{center}
\begin{tabular}{l}
|% \iffalse
%
% childdoc.dtx Copyright (C) 2017-2018 Niklas Beisert
%
% This work may be distributed and/or modified under the
% conditions of the LaTeX Project Public License, either version 1.3
% of this license or (at your option) any later version.
% The latest version of this license is in
%   http://www.latex-project.org/lppl.txt
% and version 1.3 or later is part of all distributions of LaTeX
% version 2005/12/01 or later.
%
% This work has the LPPL maintenance status `maintained'.
%
% The Current Maintainer of this work is Niklas Beisert.
%
% This work consists of the files childdoc.dtx and childdoc.ins
% and the derived files childdoc.def and cdocsamp.tex with
% cdocsch1.tex, cdocsch2.tex, cdocsdrf.tex, cdocsfn1.tex, cdocsfn2.tex.
%
%<package>\ifdefined\childdocmain\endinput\fi
%<package>\ProvidesFile{childdoc.def}[2018/12/30 v2.0 child document driver]
%<samplemain>\ProvidesFile{cdocsamp.tex}[2018/12/30 v2.0 sample for childdoc]
%<*driver>
%\ProvidesFile{childdoc.drv}[2018/12/30 v2.0 childdoc reference manual file]
\PassOptionsToClass{10pt,a4paper}{article}
\documentclass{ltxdoc}

\usepackage[margin=35mm]{geometry}
\usepackage{hyperref}
\usepackage{hyperxmp}
\usepackage[usenames]{color}

\hypersetup{colorlinks=true}
\hypersetup{pdfstartview=FitH}
\hypersetup{pdfpagemode=UseNone}
\hypersetup{pdfsource={}}
\hypersetup{pdflang={en-UK}}
\hypersetup{pdfcopyright={Copyright 2017-2018 Niklas Beisert.
  This work may be distributed and/or modified under the
  conditions of the LaTeX Project Public License, either version 1.3
  of this license or (at your option) any later version.}}
\hypersetup{pdflicenseurl={http://www.latex-project.org/lppl.txt}}
\hypersetup{pdfcontactaddress={ETH Zurich, ITP, HIT K,
  Wolfgang-Pauli-Strasse 27}}
\hypersetup{pdfcontactpostcode={8093}}
\hypersetup{pdfcontactcity={Zurich}}
\hypersetup{pdfcontactcountry={Switzerland}}
\hypersetup{pdfcontactemail={nbeisert@itp.phys.ethz.ch}}
\hypersetup{pdfcontacturl={http://people.phys.ethz.ch/\xmptilde nbeisert/}}

\newcommand{\secref}[1]{\hyperref[#1]{section \ref*{#1}}}

\parskip1ex
\parindent0pt
\let\olditemize\itemize
\def\itemize{\olditemize\parskip0pt}

\begin{document}

\title{The \textsf{childdoc} Package}
\hypersetup{pdftitle={The childdoc Package}}
\author{Niklas Beisert\\[2ex]
  Institut f\"ur Theoretische Physik\\
  Eidgen\"ossische Technische Hochschule Z\"urich\\
  Wolfgang-Pauli-Strasse 27, 8093 Z\"urich, Switzerland\\[1ex]
  \href{mailto:nbeisert@itp.phys.ethz.ch}
  {\texttt{nbeisert@itp.phys.ethz.ch}}}
\hypersetup{pdfauthor={Niklas Beisert}}
\hypersetup{pdfsubject={Manual for the LaTeX2e Package childdoc}}
\date{30 December 2018, \textsf{v2.0}}
\maketitle

\begin{abstract}\noindent
\textsf{childdoc} is a \LaTeXe{} package
that enables the direct compilation
of document sections included by |\include|
to individual files.
\end{abstract}

\begingroup
\parskip0ex
\tableofcontents
\endgroup

%%%%%%%%%%%%%%%%%%%%%%%%%%%%%%%%%%%%%%%%%%%%%%%%%%%%%%%%%%%%%%%%%%%%%%%%%%%%%%%%
%%%%%%%%%%%%%%%%%%%%%%%%%%%%%%%%%%%%%%%%%%%%%%%%%%%%%%%%%%%%%%%%%%%%%%%%%%%%%%%%
\section{Introduction}

\LaTeX{} provides a mechanism to structure a large document (such as a book)
into a main file and several child files (containing the chapters)
using the |\include| command.
This mechanism is beneficial for documents
which span hundreds of pages in order to
make the source file(s) more manageable.
Moreover, compilation can be restricted to
selected child files by means of the |\includeonly| command.
The latter feature can be used to reduce the compilation time while editing
(this was significantly more useful in the earlier days of \LaTeX{})
or to generate a smaller document which is easier to navigate.
Another application of |\includeonly| is to generate
documents consisting of selected parts of the complete document.

However, there are a few drawbacks of the plain |\include| mechanism:
\begin{itemize}
\item
The child files cannot be compiled on their own,
they can only be compiled via the main file.
A naive editing environment
(such as a text editor with an option
to have the current file processed by \LaTeX)
may require one to switch to the main file before compiling;
attempting to compile the child file produces errors.
\item
The main file must be modified (each time)
to adjust the |\includeonly| command
to the present needs. This easily leaves the main file in a messy state.
\item
The generated document will always carry the filename
of the main document. This is inconvenient if
several child files are to be compiled and
to be kept for distribution.
\end{itemize}

The present package provides a simple interface
to make child files individually compilable by \LaTeX{}.
Compiling a child file then has the same effect as compiling
the main file with an |\includeonly| command
to select the appropriate child.
Moreover the generated document will carry the name of the child
rather than the main file.
This resolves all three above issues.

This feature is meant to make the editing of books,
thesis documents and lecture notes somewhat more convenient.
However, the package can also be used efficiently for
composing a series of documents (such as exercise sheets)
which are typically distributed individually.
It then assists the author in generating the individual documents
(potentially in different versions)
as well as a document containing the collected series.
Another application is in developing style files
or other kinds of included material
where compilation of the style file could redirect
to a sample or test file.

%%%%%%%%%%%%%%%%%%%%%%%%%%%%%%%%%%%%%%%%%%%%%%%%%%%%%%%%%%%%%%%%%%%%%%%%%%%%%%%%
%%%%%%%%%%%%%%%%%%%%%%%%%%%%%%%%%%%%%%%%%%%%%%%%%%%%%%%%%%%%%%%%%%%%%%%%%%%%%%%%
\section{Usage}

First of all, the package \textsf{childdoc} is \emph{not} a standard
\LaTeXe{} |.sty| style file! Therefore it needs to be invoked in
a non-standard way.

%%%%%%%%%%%%%%%%%%%%%%%%%%%%%%%%%%%%%%%%%%%%%%%%%%%%%%%%%%%%%%%%%%%%%%%%%%%%%%%%
\subsection{Included Files}
\label{sec:include}

%%%%%%%%%%%%%%%%%%%%%%%%%%%%%%%%%%%%%%%%
\DescribeMacro{\childdocmain}
To use the package, add the commands
\begin{center}
\begin{tabular}{l}
|\input{childdoc.def}|\\
|\childdocmain{}|\\
\end{tabular}
\end{center}
at the very top of the main \LaTeX{} file,
in particular \emph{before} the |\documentclass| statement!
The argument of |\childdocmain| should be left empty
(but it must be present).

%%%%%%%%%%%%%%%%%%%%%%%%%%%%%%%%%%%%%%%%
\DescribeMacro{\childdocof}
Furthermore, add the commands
\begin{center}
\begin{tabular}{l}
|\input{childdoc.def}|\\
|\childdocof{|\textit{main}|}|\\
\end{tabular}
\end{center}
at the top of every child file \textit{child}
which is included by |\include{|\textit{child}|}|
from within the main file
(or at least for those files to be compiled individually).
The argument \textit{main} must be the filename of the main file.

There are a couple of
considerations in setting up the main and child documents:

%%%%%%%%%%%%%%%%%%%%%%%%%%%%%%%%%%%%%%%%
\paragraph{Restrictions.}

Please note the following restrictions:
\begin{itemize}
\item
|\childdocmain| must be called with one argument \textit{main}
to ensure compatibility with earlier version of the package.
It must either be empty (|\childdocmain{}|)
or precisely match the filename of the main file in which it is specified.
See \secref{sec:detection} for further information.
\item
The filename \textit{main} must be specified without the |.tex| extension.
\item
The filename \textit{main} is case sensitive
(even in case-insensitive file systems)
due to internal string comparison.
\item
The argument \textit{main} should be fully expanded, it cannot be a macro.
\item
Subdirectories and special characters should be avoided in filenames.
\item
The command |\childdocmain{|\textit{main}|}| must be followed by a whitespace.
It should not be followed immediately by another command
or by a comment mark `|%|'.
This is because the \TeX{} parser reads the token immediately following
the argument of |\childdocmain| and puts it
at the beginning of every child section;
however, a white\-space is ignored.
\end{itemize}

%%%%%%%%%%%%%%%%%%%%%%%%%%%%%%%%%%%%%%%%
\paragraph{Content of Main File.}

It is advisable to place all content in the child files included by |\include|.
Any output contained in the main file will appear in all child documents
unless suppressed manually;
it cannot be suppressed automatically by the |\includeonly| directive
and thus should normally be avoided.
A method to include some content in the main file
by means of conditional processing is described in \secref{sec:conditional}.

%%%%%%%%%%%%%%%%%%%%%%%%%%%%%%%%%%%%%%%%
\paragraph{Page Numbering.}

When only a part of the document is compiled,
the appropriate numbering of pages
(as well as other status parameters)
is determined from the |.aux| files.
The latter contain information from previous passes.
However this information needs to propagate through
all intermediate child documents.
Therefore the page numbering in child documents may well
be inconsistent until the complete document is compiled at least once.

A useful (if unconventional) way to always ensure a consistent
page numbering is to restart the numbering in each child document
and denote the pages by `\textit{child}|.|\textit{page}'
where \textit{child} represents the chapter/section number of the child file.
This can be achieved by the command
|\numberwithin{page}{|\textit{child}|}|
of the \textsf{amsmath} package
where \textit{child} can be |chapter| or |section|
depending on the chosen structuring.
Alternatively, one can modify the macro |\thepage| appropriately
and reset the counter |page| at the start of each child file.

%%%%%%%%%%%%%%%%%%%%%%%%%%%%%%%%%%%%%%%%%%%%%%%%%%%%%%%%%%%%%%%%%%%%%%%%%%%%%%%%
\subsection{Conditional Processing}
\label{sec:conditional}

The package provides a mechanism to compile different versions
of a document. To customise the versions further some conditional processing
can come in handy to distinguish which version is being compiled.
The package provides two macros to describe the compilation context:

%%%%%%%%%%%%%%%%%%%%%%%%%%%%%%%%%%%%%%%%
\DescribeMacro{\ifchilddoc}
The conditional |\ifchilddoc| distinguishes between the compilation of
child documents and the main document:
%
\begin{center}
|\ifchilddoc |\textit{child-code}| |[|\||else |\textit{main-code}]| \||fi|
\end{center}

%%%%%%%%%%%%%%%%%%%%%%%%%%%%%%%%%%%%%%%%
\DescribeMacro{\childdocname}
\DescribeMacro{\childdocjob}
The macro |\childdocname| contains the filename (without extension)
of the main or child file being processed.
Note that |\childdocjob| will always contain the name of the main file.

%%%%%%%%%%%%%%%%%%%%%%%%%%%%%%%%%%%%%%%%
\paragraph{Title Page.}

Conditional processing can be used to include a title or banner page
in the main document when proper precautions are taken.
Importantly, the code in the main file should ensure that the page counter
(as well as other status parameters which are stored in the |.aux| files)
takes the same value after the conditional processing.
Otherwise the page numbers may take divergent values
depending on which part is compiled.

For example, a title page could be declared by:
%
\begin{center}
\begin{tabular}{l}
|\ifchilddoc\||else|\\
|\addtocounter{page}{-1}|\\
\textit{code for title page}\\
|\newpage|\\
|\||fi|
\end{tabular}
\end{center}
%
A banner page for the child documents can be generated by:
%
\begin{center}
\begin{tabular}{l}
|\ifchilddoc|\\
|\addtocounter{page}{-1}|\\
\textit{code for banner page}\\
|\newpage|\\
|\||fi|
\end{tabular}
\end{center}
%
Here one could write a message such as:
\begin{center}
|This is the part \childdocname{} of \childdocjob{}.|
\end{center}

%%%%%%%%%%%%%%%%%%%%%%%%%%%%%%%%%%%%%%%%%%%%%%%%%%%%%%%%%%%%%%%%%%%%%%%%%%%%%%%%
\subsection{Flags}
\label{sec:flags}

The package makes it easy to generate different versions
of the main or child documents.
To this end compilation flags can be defined
and assigned different default values.
They will be particularly useful in conjunction
with the forwarding mechanism described in \secref{sec:forward}.

For example, it may be useful to have a flag |\version|
which can be set to |draft| or |final|.
The document source will contain some conditional code
depending on the value of |\version|.
Suppose further, the flag should default to |final| for the main file
and to |draft| for child files
which is a natural assignment for editing the document.
This is achieved by placing the following code
in the preamble of the main document
(below the |\childdocmain| directive):
%
\begin{center}
\begin{tabular}{l}
|\ifchilddoc|\\
|\providecommand{\version}{draft}|\\
|\||else|\\
|\providecommand{\version}{final}|\\
|\||fi|
\end{tabular}
\end{center}
%
The definition by |\providecommand| makes sure
that previous definitions are not overwritten.
Further statements |\providecommand{\version}{...}|
can thus be added before the above code to override it.

For the main file, one might add a line
(between |\childdocmain| and the above block)
%
\begin{center}
|%\ifchilddoc\||else\providecommand{\version}{draft}\||fi|
\end{center}
%
which can be uncommented to produce a draft version.
Likewise one can add a line to the very top of a child file
(above the |\childdocof{|\textit{main}|}| directive)
%
\begin{center}
|%\providecommand{\version}{final}|
\end{center}
%
which can be uncommented to produce the final version of this child document.

%%%%%%%%%%%%%%%%%%%%%%%%%%%%%%%%%%%%%%%%%%%%%%%%%%%%%%%%%%%%%%%%%%%%%%%%%%%%%%%%
\subsection{Forwarding}
\label{sec:forward}

Different versions of the main or child documents
using compilation flags as described in \secref{sec:flags}
can be (permanently) stored in different files
for convenient compilation, viewing and distribution.
To this end, the package defines a command
to pass on compilation to a different file:

%%%%%%%%%%%%%%%%%%%%%%%%%%%%%%%%%%%%%%%%
\DescribeMacro{\childdocforward}
The command |\childdocforward| redirects processing to
another source file:
%
\begin{center}
\begin{tabular}{l}
|\input{childdoc.def}|\\
|\childdocforward[|\textit{main}|]{|\textit{dest}|}|\\
\end{tabular}
\end{center}
%
The argument \textit{dest} is the destination file
(without extension).
It should be the main file or one of the child files.
Note that further \textsf{childdoc} directives
such as |\childdocof| and |\childdocforward|
in the indicated file will be processed in this form.
The optional argument \textit{main}
passes on directly to the main file \textit{main}
while pretending to compile the child \textit{dest}.
This form behaves as if \textit{dest}
issues |\childdocof{|\textit{main}|}| right away,
and no further \textsf{childdoc} directives will be processed.

%%%%%%%%%%%%%%%%%%%%%%%%%%%%%%%%%%%%%%%%
\DescribeMacro{\...prefix}
In the alternative form |\childdocforwardprefix|,
%
\begin{center}
\begin{tabular}{l}
|\input{childdoc.def}|\\
|\childdocforwardprefix[|\textit{main}|]{|\textit{prefix}|}{|\textit{dest}|}|
\end{tabular}
\end{center}
%
the destination file is determined by a pattern
depending on the current file:
To make this work, the current file must be called
`{\textit{prefix}\hspace{0.2em}\textit{suffix}}'
with \textit{prefix} matching precisely the argument.
Processing is then passed on to the file
`{\textit{dest}\hspace{0.2em}\textit{suffix}}'.
Surely, the same effect is achieved by
directly specifying the
argument `{\textit{dest}\hspace{0.2em}\textit{suffix}}'
in the first form.
However, that requires to set up a different file
for each child. With the alternative form of the command
all these files can have exactly the same content
which simplifies setting them up and maintaining them.

For example, the following file |draft.tex|
with a compilation flag |\version| as described in \secref{sec:flags}
compiles the main document as a draft:
%
\begin{center}
\begin{tabular}{l}
|\def\version{draft}|\\
|\input{childdoc.def}|\\
|\childdocforward{|\textit{main}|}|
\end{tabular}
\end{center}
%
Likewise, the following files |final|\textit{nn}|.tex|
compile the final version of the child document
|child|\textit{nn}|.tex|:
%
\begin{center}
\begin{tabular}{l}
|\def\version{final}|\\
|\input{childdoc.def}|\\
|\childdocforwardprefix{final}{child}|
\end{tabular}
\end{center}
%

Note that when several versions of a main file and/or of each child file
are to be generated, it may be convenient to set up a |Makefile| or
shell script to automatise the process.

%%%%%%%%%%%%%%%%%%%%%%%%%%%%%%%%%%%%%%%%%%%%%%%%%%%%%%%%%%%%%%%%%%%%%%%%%%%%%%%%
\subsection{Command Line Processing}
\label{sec:commandline}

The effect of redirection files can also be achieved by invoking
the \LaTeX{} compiler with a more elaborate command line.
Most conveniently this should be done as part
of a shell script or a |Makefile|.

When using \textsf{childdoc} in the main file, the following
command lines effectively perform a redirection
(note that depending on the shell being used,
backslashes may have to be doubled: `|\|' $\to$ `|\\|'):
%
\begin{center}
|... -jobname "|\textit{target}|" |\\|"|[\textit{flags}]%
|\input{childdoc.def}\childdocforward[|\textit{main}|]{|\textit{dest}|}"|
\end{center}
%
Here \textit{target} is the name of the output file,
\textit{main} is the name of the main file
and \textit{dest} is the name of the main or child file to be processed
(all filenames without extensions).
The optional argument \textit{main} can be omitted
if \textit{main} matches \textit{dest}.
Optionally, compilation \textit{flags} can be defined via |\def| commands.
This command line makes the \TeX{} engine believe
it is compiling the file \textit{target}
whose content is specified as the latter parameter.
The provided code then forwards the processing to
\textit{main} or \textit{dest} as described in \secref{sec:forward}.

%%%%%%%%%%%%%%%%%%%%%%%%%%%%%%%%%%%%%%%%%%%%%%%%%%%%%%%%%%%%%%%%%%%%%%%%%%%%%%%%
\subsection{Include by Input}
\label{sec:input}

Including child documents by |\include| has some restrictions by design.
Most notably, the content of a child document always occupies
its own set of pages; pages cannot be shared between child documents.
Usually, this behaviour makes perfect sense
because each child document contain an essential part of the document.
However, in some situations it may be desirable to compose
a document from a collection of parts
without having mandatory page breaks between then.
For this case, the package
provides a mechanism to include parts
by |\input| which can also be processed individually.
However, by construction this mechanism
requires manual handling of the content to be output.

%%%%%%%%%%%%%%%%%%%%%%%%%%%%%%%%%%%%%%%%
\DescribeMacro{\ifchilddocmanual}
The main file should be prepared as usual, see \secref{sec:include}.
However, the document body must make a distinction
between processing of an individual part and of the main document, e.g.:
%
\begin{center}
\begin{tabular}{l}
|\ifchilddocmanual|\\
|\input{\childdocname}|\\
|\||else|\\
\textit{document body with }|\input{|\textit{part}|}|\\
|\||fi|
\end{tabular}
\end{center}
%
The conditional |\ifchilddocmanual| is true whenever
a part to be included by |\input| is being compiled,
and the name of the part is stored in |\childdocname|.

%%%%%%%%%%%%%%%%%%%%%%%%%%%%%%%%%%%%%%%%
\DescribeMacro{\childdocby}
Each part to be included by |\input| should start with:
%
\begin{center}
\begin{tabular}{l}
|\input{childdoc.def}|\\
|\childdocby{|\textit{main}|}|\\
\end{tabular}
\end{center}
%
The directive |\childdocby| is similar to |\childdocof|
described in \secref{sec:include},
but the subsequent selection of content must be done manually.
To that end, both |\ifchilddoc| and |\ifchilddocmanual|
will be true upon processing of a part,
and the name of the part is stored in |\childdocname|.
Note that |\jobname| will be set to the filename of the current part
so that each part receives an individual |.aux| file
that does not interfere with the |.aux| file(s) of the main document.
This behaviour can be altered by the alternative form
|\childdocby[*]{|\textit{main}|}| (with a non-empty optional argument)
which uses the |.aux| file of the main document
by setting |\jobname| to \textit{main}.

%%%%%%%%%%%%%%%%%%%%%%%%%%%%%%%%%%%%%%%%%%%%%%%%%%%%%%%%%%%%%%%%%%%%%%%%%%%%%%%%
\subsection{Driver Development}
\label{sec:driver}

The \textsf{childdoc} mechanism can also be use for the development
of definition files such as \LaTeX{} styles or classes.
This case differs from the above setup with multiple parts
included by |\include| in that no |\includeonly| should be invoked.
This can be achieved by starting the include file
(before |\ProvidesPackage|) with:
%
\begin{center}
\begin{tabular}{l}
|\input{childdoc.def}|\\
|\childdocforward{|\textit{main}|}|\\
\end{tabular}
\end{center}
%
or alternatively with:
%
\begin{center}
\begin{tabular}{l}
|\input{childdoc.def}|\\
|\childdocby{|\textit{main}|}|\\
\end{tabular}
\end{center}
%
Both forms have slightly different effects as described above.
The main file is prepared as usual, see \secref{sec:include}.

%%%%%%%%%%%%%%%%%%%%%%%%%%%%%%%%%%%%%%%%%%%%%%%%%%%%%%%%%%%%%%%%%%%%%%%%%%%%%%%%
\subsection{Legacy Detection}
\label{sec:detection}

The directive |\childdocmain| in the main file can detect
whether the complete document or merely a child is to be compiled
even without using the directive |\childdocof|.
This method is deprecated because it is less robust
and there is no compelling reason to use it;
it is merely provided for backward compatibility
and it may be removed in future versions.

If the detection mechanism is to be used,
it is mandatory to correctly specify
the filename of the main file as the argument of |\childdocmain|:
%
\begin{center}
\begin{tabular}{l}
|\input{childdoc.def}|\\
|\childdocmain{|\textit{main}|}|\\
\end{tabular}
\end{center}
%
If |\jobname| does not match the argument \textit{main} of |\childdocmain|,
it is assumed that |\jobname| points to the child file to be compiled.
When using |\childdocmain| with the main file specified as argument,
it suffices to start a child file
with just |\input{|\textit{main}|}|
without loading of the package and using |\childdocof|.
If instead all processing is done
with the appropriate \textsf{childdoc} directives,
the argument of \textit{main} of |\childdocmain| can be empty.

An alternative version of the command line processing described
in \secref{sec:commandline} using the detection mechanism reads:
%
\begin{center}
|... -jobname "|\textit{target}|" "|[\textit{flags}]%
[|\def\jobname{|\textit{dest}|}|]|\input{|\textit{main}|}"|
\end{center}

%%%%%%%%%%%%%%%%%%%%%%%%%%%%%%%%%%%%%%%%%%%%%%%%%%%%%%%%%%%%%%%%%%%%%%%%%%%%%%%%
\subsection{Manual Code}
\label{sec:manual}

In case one cannot be certain whether the definitions file |childdoc.def|
is installed on the target \TeX{} distribution
and one prefers not to ship it,
it is conceivable to paste a few relevant commands into the sources.

To that end, drop all statements |\input{childdoc.def}|
and perform the replacements as outlined below.
Instead of |\childdocmain{|\textit{main}|}| add the following code
to the top of the main file:
%
\begin{center}
\begin{tabular}{l}
|\||ifdefined\childdocname\endinput\||fi\newif\ifchilddoc|\\
|\edef\childdocname{\scantokens\expandafter{\jobname\noexpand}}|\\
|\def\childdocmain{|\textit{main}|}\||ifx\childdocmain\childdocname\||else|\\
|\childdoctrue\includeonly{\childdocname}\let\jobname\childdocmain\||fi|\\
\end{tabular}
\end{center}
%
Instead of |\childdocof{|\textit{main}|}| just include the main file
at the top of each child file:
%
\begin{center}
|\input{|\textit{main}|}|
\end{center}
%
A simple redirection |\childdocforward{|\textit{dest}|}| is achieved by:
%
\begin{center}
|\def\jobname{|\textit{dest}|}\input{\jobname}|
\end{center}
%
The redirection with prefix
|\childdocforwardprefix[|\textit{prefix}|]{|\textit{dest}|}|
is accomplished by:
%
\begin{center}
\begin{tabular}{l}
|{\edef\jobname{\scantokens\expandafter{\jobname\noexpand}}|\\
|\def\redirectjob |\textit{prefix}|#1~~~{\gdef\jobname{|\textit{dest}|#1}}|\\
|\expandafter\redirectjob\jobname~~~}\input{\jobname}|
\end{tabular}
\end{center}

In an alternative approach,
child documents can be compiled by a specific command line
without additional code or specific definitions:
%
\begin{center}
|... -jobname "|\textit{target}|" "|[\textit{flags}]%
|\includeonly{|\textit{dest}|}\input{|\textit{main}|}"|
\end{center}
%

%%%%%%%%%%%%%%%%%%%%%%%%%%%%%%%%%%%%%%%%%%%%%%%%%%%%%%%%%%%%%%%%%%%%%%%%%%%%%%%%
%%%%%%%%%%%%%%%%%%%%%%%%%%%%%%%%%%%%%%%%%%%%%%%%%%%%%%%%%%%%%%%%%%%%%%%%%%%%%%%%
\section{Information}

%%%%%%%%%%%%%%%%%%%%%%%%%%%%%%%%%%%%%%%%%%%%%%%%%%%%%%%%%%%%%%%%%%%%%%%%%%%%%%%%
\subsection{Copyright}

Copyright \copyright{} 2017--2018 Niklas Beisert

This work may be distributed and/or modified under the
conditions of the \LaTeX{} Project Public License, either version 1.3
of this license or (at your option) any later version.
The latest version of this license is in
  \url{http://www.latex-project.org/lppl.txt}
and version 1.3 or later is part of all distributions of \LaTeX{}
version 2005/12/01 or later.

This work has the LPPL maintenance status `maintained'.

The Current Maintainer of this work is Niklas Beisert.

This work consists of the files |README.txt|, |childdoc.ins| and |childdoc.dtx|
as well as the derived files |childdoc.def|, |cdocsamp.tex|
with |cdocsch1.tex|, |cdocsch2.tex|, |cdocspt3.tex|, |cdocspt4.tex|,
|cdocsdrf.tex|, |cdocsfn1.tex|, |cdocsfn2.tex|
as well as |childdoc.pdf|.

%%%%%%%%%%%%%%%%%%%%%%%%%%%%%%%%%%%%%%%%%%%%%%%%%%%%%%%%%%%%%%%%%%%%%%%%%%%%%%%%
\subsection{Files and Installation}

The package consists of the files:
%
\begin{center}
\begin{tabular}{ll}
    |README.txt|   & readme file \\
    |childdoc.ins| & installation file \\
    |childdoc.dtx| & source file \\
    |childdoc.def| & definition file \\
    |cdocsamp.tex| & sample main file \\
    |cdocsch1.tex| & sample include file \\
    |cdocsch2.tex| & sample include file \\
    |cdocspt3.tex| & sample part file \\
    |cdocspt4.tex| & sample part file \\
    |cdocsdrf.tex| & sample redirection file \\
    |cdocsfn1.tex| & sample redirection file \\
    |cdocsfn2.tex| & sample redirection file \\
    |childdoc.pdf| & manual
\end{tabular}
\end{center}
%
The distribution consists of the files
|README.txt|, |childdoc.ins| and |childdoc.dtx|.
%
\begin{itemize}
\item
Run (pdf)\LaTeX{} on |childdoc.dtx|
to compile the manual |childdoc.pdf| (this file).
\item
Run \LaTeX{} on |childdoc.ins| to create the definitions file |childdoc.def|
and the sample |cdocsamp.tex| with include files
|cdocsch1.tex|, |cdocsch2.tex|, |cdocspt3.tex|, |cdocspt4.tex|,
|cdocsdrf.tex|, |cdocsfn1.tex|, |cdocsfn2.tex|.
Then copy the file |childdoc.def| to an appropriate directory of your \LaTeX{}
distribution, e.g.\ \textit{texmf-root}|/tex/latex/childdoc|.
\end{itemize}

%%%%%%%%%%%%%%%%%%%%%%%%%%%%%%%%%%%%%%%%%%%%%%%%%%%%%%%%%%%%%%%%%%%%%%%%%%%%%%%%
\subsection{Related CTAN Packages}

There are several other packages which offer a similar functionality:
%
\begin{itemize}
\item
The packages
\href{http://ctan.org/pkg/docmute}{\textsf{docmute}},
\href{http://ctan.org/pkg/includex}{\textsf{includex}} and
\href{http://ctan.org/pkg/standalone}{\textsf{standalone}}
provide commands to include only the document body of
a child file thus allowing both files to be compiled individually.
\item
The packages \href{http://ctan.org/pkg/subdocs}{\textsf{subdocs}}
and \href{http://ctan.org/pkg/subfiles}{\textsf{subfiles}}
provide structures in which the main and child documents can be
encapsulated and allowing them to be compiled individually.
The inclusion mechanism is different from the conventional |\include|.
\item
The package \href{http://ctan.org/pkg/combine}{\textsf{combine}}
is an elaborate solution to combine several documents into one.
\end{itemize}
%
See also the CTAN topic \href{http://ctan.org/topic/subdocs}{\textsf{subdocs}}
for further related packages.
The present package differs from the above solutions in that
a document structure constructed with the conventional |\include| mechanism
just needs two extra commands at the top of every file
such that all constituent files can be compiled individually.

%%%%%%%%%%%%%%%%%%%%%%%%%%%%%%%%%%%%%%%%%%%%%%%%%%%%%%%%%%%%%%%%%%%%%%%%%%%%%%%%
%\subsection{Feature Suggestions}
%
%The following is a list of features which may be useful for future
%versions of this package:
%%
%\begin{itemize}
%\item
%\ldots
%\end{itemize}

%%%%%%%%%%%%%%%%%%%%%%%%%%%%%%%%%%%%%%%%%%%%%%%%%%%%%%%%%%%%%%%%%%%%%%%%%%%%%%%%
\subsection{Revision History}

%%%%%%%%%%%%%%%%%%%%%%%%%%%%%%%%%%%%%%%%
\paragraph{v2.0:} 2018/12/30

\begin{itemize}
\item
immediate forward processing
\item
added |\childdocby| mechanism
\item
manual restructured
\end{itemize}

%%%%%%%%%%%%%%%%%%%%%%%%%%%%%%%%%%%%%%%%
\paragraph{v1.6:} 2018/01/17

\begin{itemize}
\item
application for development of include files
\item
corrections to manual
\end{itemize}

%%%%%%%%%%%%%%%%%%%%%%%%%%%%%%%%%%%%%%%%
\paragraph{v1.5:} 2017/05/21

\begin{itemize}
\item
more complete structuring introduced
\item
|\childdocof| introduced
\item
|\childdoc| renamed to |\childdocmain|
\item
|\childredirect| renamed to |\childdocforward| and |\childdocforwardprefix|
and functionality expanded
\end{itemize}

%%%%%%%%%%%%%%%%%%%%%%%%%%%%%%%%%%%%%%%%
\paragraph{v1.0:} 2017/04/27

\begin{itemize}
\item
manual and install package
\item
first version published on CTAN
\end{itemize}

%%%%%%%%%%%%%%%%%%%%%%%%%%%%%%%%%%%%%%%%
\paragraph{v0.6:} 2017/04/26

\begin{itemize}
\item
redirection mechanism added
\end{itemize}

%%%%%%%%%%%%%%%%%%%%%%%%%%%%%%%%%%%%%%%%
\paragraph{v0.5:} 2017/04/26

\begin{itemize}
\item
functionality in definition file
\end{itemize}


%%%%%%%%%%%%%%%%%%%%%%%%%%%%%%%%%%%%%%%%%%%%%%%%%%%%%%%%%%%%%%%%%%%%%%%%%%%%%%%%
%%%%%%%%%%%%%%%%%%%%%%%%%%%%%%%%%%%%%%%%%%%%%%%%%%%%%%%%%%%%%%%%%%%%%%%%%%%%%%%%
%%%%%%%%%%%%%%%%%%%%%%%%%%%%%%%%%%%%%%%%%%%%%%%%%%%%%%%%%%%%%%%%%%%%%%%%%%%%%%%%
\appendix

\settowidth\MacroIndent{\rmfamily\scriptsize 000\ }

 \DocInput{childdoc.dtx}

\end{document}
%</driver>
% \fi
%
% %%%%%%%%%%%%%%%%%%%%%%%%%%%%%%%%%%%%%%%%%%%%%%%%%%%%%%%%%%%%%%%%%%%%%%%%%%%%%%
% %%%%%%%%%%%%%%%%%%%%%%%%%%%%%%%%%%%%%%%%%%%%%%%%%%%%%%%%%%%%%%%%%%%%%%%%%%%%%%
% \section{Sample}
%\iffalse
%<*samplemain>
%\fi
%
% The following presents a sample document
% with two chapters, two parts, a title page,
% a compile flag as well as three forwarding files to set the flag.
% It consists of eight |.tex| files:
% \begin{center}
% \begin{tabular}{ll}
% |cdocsamp.tex|&main file\\
% |cdocsch1.tex|&include file for chapter 1\\
% |cdocsch2.tex|&include file for chapter 2\\
% |cdocspt3.tex|&include file for part 3\\
% |cdocspt4.tex|&include file for part 4\\
% |cdocsdrf.tex|&forwarding file for main file in draft mode\\
% |cdocsfi1.tex|&forwarding file for final version of chapter 1\\
% |cdocsfi2.tex|&forwarding file for final version of chapter 2\\
% \end{tabular}
% \end{center}
% Each of the eight files can be compiled directly by the \LaTeX{} compiler.
%
% %%%%%%%%%%%%%%%%%%%%%%%%%%%%%%%%%%%%%%
% \paragraph{Main File.}
%
% The main file is called |cdocsamp.tex|.
%
% Load the \textsf{childdoc} definitions and
% declare the filename for the main document:
%    \begin{macrocode}
\input{childdoc.def}
\childdocmain{}
%    \end{macrocode}

% Optional override for |\version| flag:
%    \begin{macrocode}
%%\ifchilddoc\else\providecommand{\version}{draft}\fi
%    \end{macrocode}

% Define the default values for the |\version| flag
% (|final| for the main file and |draft| for childs):
%    \begin{macrocode}
\ifchilddoc
\providecommand{\version}{draft}
\else
\providecommand{\version}{final}
\fi
%    \end{macrocode}

% Load the standard document class:
%    \begin{macrocode}
\documentclass[12pt]{article}
%    \end{macrocode}

% Start the document body:
%    \begin{macrocode}
\begin{document}
%    \end{macrocode}

% Declare a title page.
% Print title, part of document being processed and version flag:
%    \begin{macrocode}
\addtocounter{page}{-1}
\begin{center}
{\LARGE\bfseries{}childdoc example\par}
\vspace{1cm}
\ifchilddoc
\ifchilddocmanual part\else chapter\fi:
`\childdocname' of `\childdocjob'\par
\else
main document: `\childdocjob'\par
\fi
version: \version\par
\end{center}
\newpage
%    \end{macrocode}

% Manually include selected file,
% otherwise process as usual:
%    \begin{macrocode}
\ifchilddocmanual
\section*{part `\childdocname'}
\input{\childdocname}
\else
%    \end{macrocode}

% Include the two chapters:
%    \begin{macrocode}
\include{cdocsch1}
\include{cdocsch2}
%    \end{macrocode}

% Include the two parts unless only chapters should be displayed:
%    \begin{macrocode}
\ifchilddoc\else
\section{part three}
\input{cdocspt3}
\section{part four}
\input{cdocspt4}
\fi
%    \end{macrocode}

% Process as usual until here:
%    \begin{macrocode}
\fi
%    \end{macrocode}

% End of document body:
%    \begin{macrocode}
\end{document}
%    \end{macrocode}
%\iffalse
%</samplemain>
%\fi
%
% %%%%%%%%%%%%%%%%%%%%%%%%%%%%%%%%%%%%%%
% \paragraph{Chapter Include Files.}
%
% The include files are called |cdocsch1.tex| and |cdocsch2.tex|.
%
%\iffalse
%<*samplechap1|samplechap2>
%\fi

% Optional override for |\version| flag:
%    \begin{macrocode}
%%\providecommand{\version}{final}
%    \end{macrocode}

% Include the main document:
%    \begin{macrocode}
\input{childdoc.def}
\childdocof{cdocsamp}
%    \end{macrocode}

%\iffalse
%</samplechap1|samplechap2>
%\fi
%
%\iffalse
%<*samplechap1>
%\fi
% Some text for chapter 1:
%    \begin{macrocode}
\section{one}
some text in chapter one
%    \end{macrocode}

%\iffalse
%</samplechap1>
%\fi
% Some text for chapter 2:
%\iffalse
%<*samplechap2>
%\fi
%    \begin{macrocode}
\section{two}
more text in chapter two
%    \end{macrocode}

%\iffalse
%</samplechap2>
%\fi
%
% %%%%%%%%%%%%%%%%%%%%%%%%%%%%%%%%%%%%%%
% \paragraph{Part Include Files.}
%
% The include files are called |cdocspt3.tex| and |cdocspt4.tex|.
%
%\iffalse
%<*samplepart3|samplepart4>
%\fi

% Optional override for |\version| flag:
%    \begin{macrocode}
%%\providecommand{\version}{final}
%    \end{macrocode}

% Include the main document:
%    \begin{macrocode}
\input{childdoc.def}
\childdocby{cdocsamp}
%    \end{macrocode}

%\iffalse
%</samplepart3|samplepart4>
%\fi
%
%\iffalse
%<*samplepart3>
%\fi
% Some text for part 3:
%    \begin{macrocode}
some text in part three
%    \end{macrocode}

%\iffalse
%</samplepart3>
%\fi
% Some text for part 4:
%\iffalse
%<*samplepart4>
%\fi
%    \begin{macrocode}
more text in part four
%    \end{macrocode}

%\iffalse
%</samplepart4>
%\fi
%
% %%%%%%%%%%%%%%%%%%%%%%%%%%%%%%%%%%%%%%
% \paragraph{Forwarding for a Complete Draft.}
%
% The following forwarding file |cdocsdrf.tex|
% compiles the main document in draft mode:
%\iffalse
%<*sampledraft>
%\fi
%    \begin{macrocode}
\def\version{draft}
\input{childdoc.def}
\childdocforward{cdocsamp}
%    \end{macrocode}

%\iffalse
%</sampledraft>
%\fi
%
% %%%%%%%%%%%%%%%%%%%%%%%%%%%%%%%%%%%%%%
% \paragraph{Forwarding for Final Version of the Chapters.}
%
% The following forwarding files |cdocsfn1.tex| and |cdocsfn2.tex|
% (with identical content)
% compile the final versions of the child documents
% |cdocsch1.tex| and |cdocsch2.tex|, respectively:
%\iffalse
%<*samplefinal>
%\fi
%    \begin{macrocode}
\def\version{final}
\input{childdoc.def}
\childdocforwardprefix[cdocsamp]{cdocsfn}{cdocsch}
%    \end{macrocode}

%\iffalse
%</samplefinal>
%\fi
%
% %%%%%%%%%%%%%%%%%%%%%%%%%%%%%%%%%%%%%%
% \paragraph{Command Line Processing.}
%
% The following three command lines generate the output files
% |cdocscld|, |cdocscl1| and |cdocscl2|
% which should be identical to
% |cdocsdrf|, |cdocsch1| and |cdocsfn2|, respectively:
% \begin{center}
% \begin{tabular}{l}
% |latex -jobname cdocscld \|\\
% |  "\def\version{draft}\input{childdoc.def}\childdocforward{cdocsamp}"|\\
% |latex -jobname cdocscl1 \|\\
% |  "\input{childdoc.def}\childdocforward[cdocsamp]{cdocsch1}"|\\
% |latex -jobname cdocscl2 \|\\
% |  "\def\version{final}\input{childdoc.def}\childdocforward{cdocsch2}"|
% \end{tabular}
% \end{center}
% Note that the trailing backslash on each first line
% merely continues the input to the second line
% (for convenient cut ant paste).
% Furthermore, the command |latex| can be replaced by any
% of its alternative versions such as |pdflatex|.
%
% %%%%%%%%%%%%%%%%%%%%%%%%%%%%%%%%%%%%%%%%%%%%%%%%%%%%%%%%%%%%%%%%%%%%%%%%%%%%%%
% %%%%%%%%%%%%%%%%%%%%%%%%%%%%%%%%%%%%%%%%%%%%%%%%%%%%%%%%%%%%%%%%%%%%%%%%%%%%%%
% \section{Implementation}
%\iffalse
%<*package>
%\fi
%
% This section describes the definitions file |childdoc.def|.

% The definitions cannot be loaded using |\usepackage| or |\RequirePackage|
% which has a mechanism to prevent loading a style file more than once.
% When loading the definitions by means of |\input|
% multiple instances have to be prevented manually:
%\iffalse
%This code needs to be before the `\ProvidesFile' directive
%which is defined at the beginning of this file.
%Therefore it is also placed there and commented out here.
%</package>
%<*discard>
%\fi
%    \begin{macrocode}
\ifdefined\childdocmain\endinput\fi
%    \end{macrocode}
%\iffalse
%</discard>
%<*package>
%\fi
%
% \macro{\ifchilddoc}
% \macro{\ifchilddocmanual}
% The conditional |\ifchilddoc| tells whether a
% child (true) or main (false) document is being compiled.
% The conditional |\ifchilddocmanual| tells whether
% the |\includeonly| mechanism is used (false) or
% the selection of child files must be performed manually (true).
% The definitions initialise to false:
%    \begin{macrocode}
\newif\ifchilddoc
\newif\ifchilddocmanual
%    \end{macrocode}

% \macro{\childdocname}
% \macro{\childdocjob}
% The macro |\childdocname| stores the name of the main document
% to be compiled. The macro |\childdocjob| stores the name of
% the document on which the \LaTeX{} compiler was originally invoked.
% The content of |\jobname| cannot be compared
% to filenames specified in the source due to different catcodes.
% The following code rescans |\jobname|, stores the result
% in |\childdocname| and saves a copy in |\childdocjob|:
%    \begin{macrocode}
\edef\childdocname{\scantokens\expandafter{\jobname\noexpand}}
\let\childdocjob\childdocname
%    \end{macrocode}

% \macro{\childdocdisable}
% The macro |\childdocdisable| prevents the main file
% from being processed more than once.
% At this stage, the main document command |\childdocmain|
% is assumed to be called once again where it should do nothing.
% Any subsequent call to it should prevent
% a secondary processing of the main document
% It overwrites the forwarding commands
% |\childdocof| and |\childdocforward|
% with empty macros to prevent further inclusions of the main document:
%    \begin{macrocode}
\newcommand{\childdocdisable}
{
  \renewcommand{\childdocmain}[1]{\renewcommand{\childdocmain}[1]{\endinput}}
  \renewcommand{\childdocof}[1]{}
  \renewcommand{\childdocby}[2][]{}
  \renewcommand{\childdocforward}[2][]{}
  \renewcommand{\childdocdisable}{}
}
%    \end{macrocode}

% \macro{\childdocmain}
% The macro |\childdocmain| is to be called at the top of the main file
% with nothing or the main filename (without extension) as argument.
% First, it breaks loops.
% If the argument is not empty and does not match |\childdocname|
% (which is set by the first inclusion of |childdoc.def|),
% |\ifchilddoc| is set to true, |\includeonly| is applied to the child file
% and |\jobname| is set to the main file
% (for proper handling of |.aux| files):
%    \begin{macrocode}
\newcommand{\childdocmain}[1]
{
  \childdocdisable\childdocmain{}
  \if?#1?\else
    \begingroup
      \def\childdoctmp{#1}
      \ifx\childdoctmp\childdocname
        \def\childdoctmp{}
      \else
        \def\childdoctmp
        {
          \childdoctrue
          \includeonly{\childdocname}
          \def\childdocjob{#1}
          \def\jobname{#1}
        }
      \fi
      \expandafter
    \endgroup
    \childdoctmp
  \fi
}
%    \end{macrocode}

% \macro{\childdocof}
% The command |\childdocof| redirects
% compilation to the main file |#1|.
%    \begin{macrocode}
\newcommand{\childdocof}[1]
{
  \childdocdisable
  \childdoctrue
  \includeonly{\childdocname}
  \def\jobname{#1}
  \def\childdocjob{#1}
  \input{#1}
}
%    \end{macrocode}

% \macro{\childdocby}
% The command |\childdocby| ....
%    \begin{macrocode}
\newcommand{\childdocby}[2][]
{
  \childdocdisable
  \childdoctrue
  \childdocmanualtrue
  \if?#1?\else
    \def\jobname{#2}
  \fi
  \def\childdocjob{#2}
  \input{#2}
  \endinput
}
%    \end{macrocode}

% \macro{\childdocforward}
% The command |\childdocforward| redirects
% compilation to the main file or
% (if the optional argument is given) a child file.
% Parameters are set as if the main file
% or a child file starting with |\childdocof| was compiled.
% Then compilation is handed over to the main file:
%    \begin{macrocode}
\newcommand{\childdocforward}[2][]
{
  \begingroup
    \if?#1?
      \def\childdoctmp
      {
        \def\childdocname{#2}
        \def\childdocjob{#2}
        \def\jobname{#2}
        \input{#2}
        \endinput
      }
    \else
      \def\childdoctmp
      {
        \childdocdisable
        \def\childdocname{#2}
        \childdoctrue
        \includeonly{#2}
        \def\childdocjob{#1}
        \def\jobname{#1}
        \input{#1}
        \endinput
      }
    \fi
    \expandafter
  \endgroup
  \childdoctmp
}
%    \end{macrocode}

% \macro{\childdocforwardprefix}
% The command |\childdocforwardprefix| redirects
% compilation to the main or a child file by means of a pattern.
% The prefix |#1| in the current filename is replaced by |#2|
% and the suffix of the current filename is kept
% (it is assumed that the filename does not contain the substring `|~~~|'
% which is used as a delimiter).
% Compilation is handed over to the new file by |\childdocforward|:
%    \begin{macrocode}
\newcommand{\childdocforwardprefix}[3][]
{
  \begingroup
    \def\childdocextract #2##1~~~{\def\childdoctmp{\childdocforward[#1]{#3##1}}}
    \expandafter\childdocextract\childdocname~~~
    \expandafter
  \endgroup
  \childdoctmp
}
%    \end{macrocode}

% \macro{\childdoc}
% The deprecated macro |\childdoc| is a legacy version of |\childdocmain|:
%    \begin{macrocode}
\newcommand{\childdoc}{\childdocmain}
%    \end{macrocode}

% \macro{\childdocredirect}
% The deprecated macro |\childdocredirect| is a legacy version
% of |\childdocforward| and |\childdocforwardprefix|:
%    \begin{macrocode}
\newcommand{\childdocredirect}[2][]
{
  \begingroup
    \if?#1?
      \def\childdoctmp{\childdocforward{#2}}
    \else
      \def\childdoctmp{\childdocforwardprefix{#1}{#2}}
    \fi
    \expandafter
  \endgroup
  \childdoctmp
}
%    \end{macrocode}

%\iffalse
%</package>
%\fi
%
\endinput
|\\
|\childdocforward[|\textit{main}|]{|\textit{dest}|}|\\
\end{tabular}
\end{center}
%
The argument \textit{dest} is the destination file
(without extension).
It should be the main file or one of the child files.
Note that further \textsf{childdoc} directives
such as |\childdocof| and |\childdocforward|
in the indicated file will be processed in this form.
The optional argument \textit{main}
passes on directly to the main file \textit{main}
while pretending to compile the child \textit{dest}.
This form behaves as if \textit{dest}
issues |\childdocof{|\textit{main}|}| right away,
and no further \textsf{childdoc} directives will be processed.

%%%%%%%%%%%%%%%%%%%%%%%%%%%%%%%%%%%%%%%%
\DescribeMacro{\...prefix}
In the alternative form |\childdocforwardprefix|,
%
\begin{center}
\begin{tabular}{l}
|% \iffalse
%
% childdoc.dtx Copyright (C) 2017-2018 Niklas Beisert
%
% This work may be distributed and/or modified under the
% conditions of the LaTeX Project Public License, either version 1.3
% of this license or (at your option) any later version.
% The latest version of this license is in
%   http://www.latex-project.org/lppl.txt
% and version 1.3 or later is part of all distributions of LaTeX
% version 2005/12/01 or later.
%
% This work has the LPPL maintenance status `maintained'.
%
% The Current Maintainer of this work is Niklas Beisert.
%
% This work consists of the files childdoc.dtx and childdoc.ins
% and the derived files childdoc.def and cdocsamp.tex with
% cdocsch1.tex, cdocsch2.tex, cdocsdrf.tex, cdocsfn1.tex, cdocsfn2.tex.
%
%<package>\ifdefined\childdocmain\endinput\fi
%<package>\ProvidesFile{childdoc.def}[2018/12/30 v2.0 child document driver]
%<samplemain>\ProvidesFile{cdocsamp.tex}[2018/12/30 v2.0 sample for childdoc]
%<*driver>
%\ProvidesFile{childdoc.drv}[2018/12/30 v2.0 childdoc reference manual file]
\PassOptionsToClass{10pt,a4paper}{article}
\documentclass{ltxdoc}

\usepackage[margin=35mm]{geometry}
\usepackage{hyperref}
\usepackage{hyperxmp}
\usepackage[usenames]{color}

\hypersetup{colorlinks=true}
\hypersetup{pdfstartview=FitH}
\hypersetup{pdfpagemode=UseNone}
\hypersetup{pdfsource={}}
\hypersetup{pdflang={en-UK}}
\hypersetup{pdfcopyright={Copyright 2017-2018 Niklas Beisert.
  This work may be distributed and/or modified under the
  conditions of the LaTeX Project Public License, either version 1.3
  of this license or (at your option) any later version.}}
\hypersetup{pdflicenseurl={http://www.latex-project.org/lppl.txt}}
\hypersetup{pdfcontactaddress={ETH Zurich, ITP, HIT K,
  Wolfgang-Pauli-Strasse 27}}
\hypersetup{pdfcontactpostcode={8093}}
\hypersetup{pdfcontactcity={Zurich}}
\hypersetup{pdfcontactcountry={Switzerland}}
\hypersetup{pdfcontactemail={nbeisert@itp.phys.ethz.ch}}
\hypersetup{pdfcontacturl={http://people.phys.ethz.ch/\xmptilde nbeisert/}}

\newcommand{\secref}[1]{\hyperref[#1]{section \ref*{#1}}}

\parskip1ex
\parindent0pt
\let\olditemize\itemize
\def\itemize{\olditemize\parskip0pt}

\begin{document}

\title{The \textsf{childdoc} Package}
\hypersetup{pdftitle={The childdoc Package}}
\author{Niklas Beisert\\[2ex]
  Institut f\"ur Theoretische Physik\\
  Eidgen\"ossische Technische Hochschule Z\"urich\\
  Wolfgang-Pauli-Strasse 27, 8093 Z\"urich, Switzerland\\[1ex]
  \href{mailto:nbeisert@itp.phys.ethz.ch}
  {\texttt{nbeisert@itp.phys.ethz.ch}}}
\hypersetup{pdfauthor={Niklas Beisert}}
\hypersetup{pdfsubject={Manual for the LaTeX2e Package childdoc}}
\date{30 December 2018, \textsf{v2.0}}
\maketitle

\begin{abstract}\noindent
\textsf{childdoc} is a \LaTeXe{} package
that enables the direct compilation
of document sections included by |\include|
to individual files.
\end{abstract}

\begingroup
\parskip0ex
\tableofcontents
\endgroup

%%%%%%%%%%%%%%%%%%%%%%%%%%%%%%%%%%%%%%%%%%%%%%%%%%%%%%%%%%%%%%%%%%%%%%%%%%%%%%%%
%%%%%%%%%%%%%%%%%%%%%%%%%%%%%%%%%%%%%%%%%%%%%%%%%%%%%%%%%%%%%%%%%%%%%%%%%%%%%%%%
\section{Introduction}

\LaTeX{} provides a mechanism to structure a large document (such as a book)
into a main file and several child files (containing the chapters)
using the |\include| command.
This mechanism is beneficial for documents
which span hundreds of pages in order to
make the source file(s) more manageable.
Moreover, compilation can be restricted to
selected child files by means of the |\includeonly| command.
The latter feature can be used to reduce the compilation time while editing
(this was significantly more useful in the earlier days of \LaTeX{})
or to generate a smaller document which is easier to navigate.
Another application of |\includeonly| is to generate
documents consisting of selected parts of the complete document.

However, there are a few drawbacks of the plain |\include| mechanism:
\begin{itemize}
\item
The child files cannot be compiled on their own,
they can only be compiled via the main file.
A naive editing environment
(such as a text editor with an option
to have the current file processed by \LaTeX)
may require one to switch to the main file before compiling;
attempting to compile the child file produces errors.
\item
The main file must be modified (each time)
to adjust the |\includeonly| command
to the present needs. This easily leaves the main file in a messy state.
\item
The generated document will always carry the filename
of the main document. This is inconvenient if
several child files are to be compiled and
to be kept for distribution.
\end{itemize}

The present package provides a simple interface
to make child files individually compilable by \LaTeX{}.
Compiling a child file then has the same effect as compiling
the main file with an |\includeonly| command
to select the appropriate child.
Moreover the generated document will carry the name of the child
rather than the main file.
This resolves all three above issues.

This feature is meant to make the editing of books,
thesis documents and lecture notes somewhat more convenient.
However, the package can also be used efficiently for
composing a series of documents (such as exercise sheets)
which are typically distributed individually.
It then assists the author in generating the individual documents
(potentially in different versions)
as well as a document containing the collected series.
Another application is in developing style files
or other kinds of included material
where compilation of the style file could redirect
to a sample or test file.

%%%%%%%%%%%%%%%%%%%%%%%%%%%%%%%%%%%%%%%%%%%%%%%%%%%%%%%%%%%%%%%%%%%%%%%%%%%%%%%%
%%%%%%%%%%%%%%%%%%%%%%%%%%%%%%%%%%%%%%%%%%%%%%%%%%%%%%%%%%%%%%%%%%%%%%%%%%%%%%%%
\section{Usage}

First of all, the package \textsf{childdoc} is \emph{not} a standard
\LaTeXe{} |.sty| style file! Therefore it needs to be invoked in
a non-standard way.

%%%%%%%%%%%%%%%%%%%%%%%%%%%%%%%%%%%%%%%%%%%%%%%%%%%%%%%%%%%%%%%%%%%%%%%%%%%%%%%%
\subsection{Included Files}
\label{sec:include}

%%%%%%%%%%%%%%%%%%%%%%%%%%%%%%%%%%%%%%%%
\DescribeMacro{\childdocmain}
To use the package, add the commands
\begin{center}
\begin{tabular}{l}
|\input{childdoc.def}|\\
|\childdocmain{}|\\
\end{tabular}
\end{center}
at the very top of the main \LaTeX{} file,
in particular \emph{before} the |\documentclass| statement!
The argument of |\childdocmain| should be left empty
(but it must be present).

%%%%%%%%%%%%%%%%%%%%%%%%%%%%%%%%%%%%%%%%
\DescribeMacro{\childdocof}
Furthermore, add the commands
\begin{center}
\begin{tabular}{l}
|\input{childdoc.def}|\\
|\childdocof{|\textit{main}|}|\\
\end{tabular}
\end{center}
at the top of every child file \textit{child}
which is included by |\include{|\textit{child}|}|
from within the main file
(or at least for those files to be compiled individually).
The argument \textit{main} must be the filename of the main file.

There are a couple of
considerations in setting up the main and child documents:

%%%%%%%%%%%%%%%%%%%%%%%%%%%%%%%%%%%%%%%%
\paragraph{Restrictions.}

Please note the following restrictions:
\begin{itemize}
\item
|\childdocmain| must be called with one argument \textit{main}
to ensure compatibility with earlier version of the package.
It must either be empty (|\childdocmain{}|)
or precisely match the filename of the main file in which it is specified.
See \secref{sec:detection} for further information.
\item
The filename \textit{main} must be specified without the |.tex| extension.
\item
The filename \textit{main} is case sensitive
(even in case-insensitive file systems)
due to internal string comparison.
\item
The argument \textit{main} should be fully expanded, it cannot be a macro.
\item
Subdirectories and special characters should be avoided in filenames.
\item
The command |\childdocmain{|\textit{main}|}| must be followed by a whitespace.
It should not be followed immediately by another command
or by a comment mark `|%|'.
This is because the \TeX{} parser reads the token immediately following
the argument of |\childdocmain| and puts it
at the beginning of every child section;
however, a white\-space is ignored.
\end{itemize}

%%%%%%%%%%%%%%%%%%%%%%%%%%%%%%%%%%%%%%%%
\paragraph{Content of Main File.}

It is advisable to place all content in the child files included by |\include|.
Any output contained in the main file will appear in all child documents
unless suppressed manually;
it cannot be suppressed automatically by the |\includeonly| directive
and thus should normally be avoided.
A method to include some content in the main file
by means of conditional processing is described in \secref{sec:conditional}.

%%%%%%%%%%%%%%%%%%%%%%%%%%%%%%%%%%%%%%%%
\paragraph{Page Numbering.}

When only a part of the document is compiled,
the appropriate numbering of pages
(as well as other status parameters)
is determined from the |.aux| files.
The latter contain information from previous passes.
However this information needs to propagate through
all intermediate child documents.
Therefore the page numbering in child documents may well
be inconsistent until the complete document is compiled at least once.

A useful (if unconventional) way to always ensure a consistent
page numbering is to restart the numbering in each child document
and denote the pages by `\textit{child}|.|\textit{page}'
where \textit{child} represents the chapter/section number of the child file.
This can be achieved by the command
|\numberwithin{page}{|\textit{child}|}|
of the \textsf{amsmath} package
where \textit{child} can be |chapter| or |section|
depending on the chosen structuring.
Alternatively, one can modify the macro |\thepage| appropriately
and reset the counter |page| at the start of each child file.

%%%%%%%%%%%%%%%%%%%%%%%%%%%%%%%%%%%%%%%%%%%%%%%%%%%%%%%%%%%%%%%%%%%%%%%%%%%%%%%%
\subsection{Conditional Processing}
\label{sec:conditional}

The package provides a mechanism to compile different versions
of a document. To customise the versions further some conditional processing
can come in handy to distinguish which version is being compiled.
The package provides two macros to describe the compilation context:

%%%%%%%%%%%%%%%%%%%%%%%%%%%%%%%%%%%%%%%%
\DescribeMacro{\ifchilddoc}
The conditional |\ifchilddoc| distinguishes between the compilation of
child documents and the main document:
%
\begin{center}
|\ifchilddoc |\textit{child-code}| |[|\||else |\textit{main-code}]| \||fi|
\end{center}

%%%%%%%%%%%%%%%%%%%%%%%%%%%%%%%%%%%%%%%%
\DescribeMacro{\childdocname}
\DescribeMacro{\childdocjob}
The macro |\childdocname| contains the filename (without extension)
of the main or child file being processed.
Note that |\childdocjob| will always contain the name of the main file.

%%%%%%%%%%%%%%%%%%%%%%%%%%%%%%%%%%%%%%%%
\paragraph{Title Page.}

Conditional processing can be used to include a title or banner page
in the main document when proper precautions are taken.
Importantly, the code in the main file should ensure that the page counter
(as well as other status parameters which are stored in the |.aux| files)
takes the same value after the conditional processing.
Otherwise the page numbers may take divergent values
depending on which part is compiled.

For example, a title page could be declared by:
%
\begin{center}
\begin{tabular}{l}
|\ifchilddoc\||else|\\
|\addtocounter{page}{-1}|\\
\textit{code for title page}\\
|\newpage|\\
|\||fi|
\end{tabular}
\end{center}
%
A banner page for the child documents can be generated by:
%
\begin{center}
\begin{tabular}{l}
|\ifchilddoc|\\
|\addtocounter{page}{-1}|\\
\textit{code for banner page}\\
|\newpage|\\
|\||fi|
\end{tabular}
\end{center}
%
Here one could write a message such as:
\begin{center}
|This is the part \childdocname{} of \childdocjob{}.|
\end{center}

%%%%%%%%%%%%%%%%%%%%%%%%%%%%%%%%%%%%%%%%%%%%%%%%%%%%%%%%%%%%%%%%%%%%%%%%%%%%%%%%
\subsection{Flags}
\label{sec:flags}

The package makes it easy to generate different versions
of the main or child documents.
To this end compilation flags can be defined
and assigned different default values.
They will be particularly useful in conjunction
with the forwarding mechanism described in \secref{sec:forward}.

For example, it may be useful to have a flag |\version|
which can be set to |draft| or |final|.
The document source will contain some conditional code
depending on the value of |\version|.
Suppose further, the flag should default to |final| for the main file
and to |draft| for child files
which is a natural assignment for editing the document.
This is achieved by placing the following code
in the preamble of the main document
(below the |\childdocmain| directive):
%
\begin{center}
\begin{tabular}{l}
|\ifchilddoc|\\
|\providecommand{\version}{draft}|\\
|\||else|\\
|\providecommand{\version}{final}|\\
|\||fi|
\end{tabular}
\end{center}
%
The definition by |\providecommand| makes sure
that previous definitions are not overwritten.
Further statements |\providecommand{\version}{...}|
can thus be added before the above code to override it.

For the main file, one might add a line
(between |\childdocmain| and the above block)
%
\begin{center}
|%\ifchilddoc\||else\providecommand{\version}{draft}\||fi|
\end{center}
%
which can be uncommented to produce a draft version.
Likewise one can add a line to the very top of a child file
(above the |\childdocof{|\textit{main}|}| directive)
%
\begin{center}
|%\providecommand{\version}{final}|
\end{center}
%
which can be uncommented to produce the final version of this child document.

%%%%%%%%%%%%%%%%%%%%%%%%%%%%%%%%%%%%%%%%%%%%%%%%%%%%%%%%%%%%%%%%%%%%%%%%%%%%%%%%
\subsection{Forwarding}
\label{sec:forward}

Different versions of the main or child documents
using compilation flags as described in \secref{sec:flags}
can be (permanently) stored in different files
for convenient compilation, viewing and distribution.
To this end, the package defines a command
to pass on compilation to a different file:

%%%%%%%%%%%%%%%%%%%%%%%%%%%%%%%%%%%%%%%%
\DescribeMacro{\childdocforward}
The command |\childdocforward| redirects processing to
another source file:
%
\begin{center}
\begin{tabular}{l}
|\input{childdoc.def}|\\
|\childdocforward[|\textit{main}|]{|\textit{dest}|}|\\
\end{tabular}
\end{center}
%
The argument \textit{dest} is the destination file
(without extension).
It should be the main file or one of the child files.
Note that further \textsf{childdoc} directives
such as |\childdocof| and |\childdocforward|
in the indicated file will be processed in this form.
The optional argument \textit{main}
passes on directly to the main file \textit{main}
while pretending to compile the child \textit{dest}.
This form behaves as if \textit{dest}
issues |\childdocof{|\textit{main}|}| right away,
and no further \textsf{childdoc} directives will be processed.

%%%%%%%%%%%%%%%%%%%%%%%%%%%%%%%%%%%%%%%%
\DescribeMacro{\...prefix}
In the alternative form |\childdocforwardprefix|,
%
\begin{center}
\begin{tabular}{l}
|\input{childdoc.def}|\\
|\childdocforwardprefix[|\textit{main}|]{|\textit{prefix}|}{|\textit{dest}|}|
\end{tabular}
\end{center}
%
the destination file is determined by a pattern
depending on the current file:
To make this work, the current file must be called
`{\textit{prefix}\hspace{0.2em}\textit{suffix}}'
with \textit{prefix} matching precisely the argument.
Processing is then passed on to the file
`{\textit{dest}\hspace{0.2em}\textit{suffix}}'.
Surely, the same effect is achieved by
directly specifying the
argument `{\textit{dest}\hspace{0.2em}\textit{suffix}}'
in the first form.
However, that requires to set up a different file
for each child. With the alternative form of the command
all these files can have exactly the same content
which simplifies setting them up and maintaining them.

For example, the following file |draft.tex|
with a compilation flag |\version| as described in \secref{sec:flags}
compiles the main document as a draft:
%
\begin{center}
\begin{tabular}{l}
|\def\version{draft}|\\
|\input{childdoc.def}|\\
|\childdocforward{|\textit{main}|}|
\end{tabular}
\end{center}
%
Likewise, the following files |final|\textit{nn}|.tex|
compile the final version of the child document
|child|\textit{nn}|.tex|:
%
\begin{center}
\begin{tabular}{l}
|\def\version{final}|\\
|\input{childdoc.def}|\\
|\childdocforwardprefix{final}{child}|
\end{tabular}
\end{center}
%

Note that when several versions of a main file and/or of each child file
are to be generated, it may be convenient to set up a |Makefile| or
shell script to automatise the process.

%%%%%%%%%%%%%%%%%%%%%%%%%%%%%%%%%%%%%%%%%%%%%%%%%%%%%%%%%%%%%%%%%%%%%%%%%%%%%%%%
\subsection{Command Line Processing}
\label{sec:commandline}

The effect of redirection files can also be achieved by invoking
the \LaTeX{} compiler with a more elaborate command line.
Most conveniently this should be done as part
of a shell script or a |Makefile|.

When using \textsf{childdoc} in the main file, the following
command lines effectively perform a redirection
(note that depending on the shell being used,
backslashes may have to be doubled: `|\|' $\to$ `|\\|'):
%
\begin{center}
|... -jobname "|\textit{target}|" |\\|"|[\textit{flags}]%
|\input{childdoc.def}\childdocforward[|\textit{main}|]{|\textit{dest}|}"|
\end{center}
%
Here \textit{target} is the name of the output file,
\textit{main} is the name of the main file
and \textit{dest} is the name of the main or child file to be processed
(all filenames without extensions).
The optional argument \textit{main} can be omitted
if \textit{main} matches \textit{dest}.
Optionally, compilation \textit{flags} can be defined via |\def| commands.
This command line makes the \TeX{} engine believe
it is compiling the file \textit{target}
whose content is specified as the latter parameter.
The provided code then forwards the processing to
\textit{main} or \textit{dest} as described in \secref{sec:forward}.

%%%%%%%%%%%%%%%%%%%%%%%%%%%%%%%%%%%%%%%%%%%%%%%%%%%%%%%%%%%%%%%%%%%%%%%%%%%%%%%%
\subsection{Include by Input}
\label{sec:input}

Including child documents by |\include| has some restrictions by design.
Most notably, the content of a child document always occupies
its own set of pages; pages cannot be shared between child documents.
Usually, this behaviour makes perfect sense
because each child document contain an essential part of the document.
However, in some situations it may be desirable to compose
a document from a collection of parts
without having mandatory page breaks between then.
For this case, the package
provides a mechanism to include parts
by |\input| which can also be processed individually.
However, by construction this mechanism
requires manual handling of the content to be output.

%%%%%%%%%%%%%%%%%%%%%%%%%%%%%%%%%%%%%%%%
\DescribeMacro{\ifchilddocmanual}
The main file should be prepared as usual, see \secref{sec:include}.
However, the document body must make a distinction
between processing of an individual part and of the main document, e.g.:
%
\begin{center}
\begin{tabular}{l}
|\ifchilddocmanual|\\
|\input{\childdocname}|\\
|\||else|\\
\textit{document body with }|\input{|\textit{part}|}|\\
|\||fi|
\end{tabular}
\end{center}
%
The conditional |\ifchilddocmanual| is true whenever
a part to be included by |\input| is being compiled,
and the name of the part is stored in |\childdocname|.

%%%%%%%%%%%%%%%%%%%%%%%%%%%%%%%%%%%%%%%%
\DescribeMacro{\childdocby}
Each part to be included by |\input| should start with:
%
\begin{center}
\begin{tabular}{l}
|\input{childdoc.def}|\\
|\childdocby{|\textit{main}|}|\\
\end{tabular}
\end{center}
%
The directive |\childdocby| is similar to |\childdocof|
described in \secref{sec:include},
but the subsequent selection of content must be done manually.
To that end, both |\ifchilddoc| and |\ifchilddocmanual|
will be true upon processing of a part,
and the name of the part is stored in |\childdocname|.
Note that |\jobname| will be set to the filename of the current part
so that each part receives an individual |.aux| file
that does not interfere with the |.aux| file(s) of the main document.
This behaviour can be altered by the alternative form
|\childdocby[*]{|\textit{main}|}| (with a non-empty optional argument)
which uses the |.aux| file of the main document
by setting |\jobname| to \textit{main}.

%%%%%%%%%%%%%%%%%%%%%%%%%%%%%%%%%%%%%%%%%%%%%%%%%%%%%%%%%%%%%%%%%%%%%%%%%%%%%%%%
\subsection{Driver Development}
\label{sec:driver}

The \textsf{childdoc} mechanism can also be use for the development
of definition files such as \LaTeX{} styles or classes.
This case differs from the above setup with multiple parts
included by |\include| in that no |\includeonly| should be invoked.
This can be achieved by starting the include file
(before |\ProvidesPackage|) with:
%
\begin{center}
\begin{tabular}{l}
|\input{childdoc.def}|\\
|\childdocforward{|\textit{main}|}|\\
\end{tabular}
\end{center}
%
or alternatively with:
%
\begin{center}
\begin{tabular}{l}
|\input{childdoc.def}|\\
|\childdocby{|\textit{main}|}|\\
\end{tabular}
\end{center}
%
Both forms have slightly different effects as described above.
The main file is prepared as usual, see \secref{sec:include}.

%%%%%%%%%%%%%%%%%%%%%%%%%%%%%%%%%%%%%%%%%%%%%%%%%%%%%%%%%%%%%%%%%%%%%%%%%%%%%%%%
\subsection{Legacy Detection}
\label{sec:detection}

The directive |\childdocmain| in the main file can detect
whether the complete document or merely a child is to be compiled
even without using the directive |\childdocof|.
This method is deprecated because it is less robust
and there is no compelling reason to use it;
it is merely provided for backward compatibility
and it may be removed in future versions.

If the detection mechanism is to be used,
it is mandatory to correctly specify
the filename of the main file as the argument of |\childdocmain|:
%
\begin{center}
\begin{tabular}{l}
|\input{childdoc.def}|\\
|\childdocmain{|\textit{main}|}|\\
\end{tabular}
\end{center}
%
If |\jobname| does not match the argument \textit{main} of |\childdocmain|,
it is assumed that |\jobname| points to the child file to be compiled.
When using |\childdocmain| with the main file specified as argument,
it suffices to start a child file
with just |\input{|\textit{main}|}|
without loading of the package and using |\childdocof|.
If instead all processing is done
with the appropriate \textsf{childdoc} directives,
the argument of \textit{main} of |\childdocmain| can be empty.

An alternative version of the command line processing described
in \secref{sec:commandline} using the detection mechanism reads:
%
\begin{center}
|... -jobname "|\textit{target}|" "|[\textit{flags}]%
[|\def\jobname{|\textit{dest}|}|]|\input{|\textit{main}|}"|
\end{center}

%%%%%%%%%%%%%%%%%%%%%%%%%%%%%%%%%%%%%%%%%%%%%%%%%%%%%%%%%%%%%%%%%%%%%%%%%%%%%%%%
\subsection{Manual Code}
\label{sec:manual}

In case one cannot be certain whether the definitions file |childdoc.def|
is installed on the target \TeX{} distribution
and one prefers not to ship it,
it is conceivable to paste a few relevant commands into the sources.

To that end, drop all statements |\input{childdoc.def}|
and perform the replacements as outlined below.
Instead of |\childdocmain{|\textit{main}|}| add the following code
to the top of the main file:
%
\begin{center}
\begin{tabular}{l}
|\||ifdefined\childdocname\endinput\||fi\newif\ifchilddoc|\\
|\edef\childdocname{\scantokens\expandafter{\jobname\noexpand}}|\\
|\def\childdocmain{|\textit{main}|}\||ifx\childdocmain\childdocname\||else|\\
|\childdoctrue\includeonly{\childdocname}\let\jobname\childdocmain\||fi|\\
\end{tabular}
\end{center}
%
Instead of |\childdocof{|\textit{main}|}| just include the main file
at the top of each child file:
%
\begin{center}
|\input{|\textit{main}|}|
\end{center}
%
A simple redirection |\childdocforward{|\textit{dest}|}| is achieved by:
%
\begin{center}
|\def\jobname{|\textit{dest}|}\input{\jobname}|
\end{center}
%
The redirection with prefix
|\childdocforwardprefix[|\textit{prefix}|]{|\textit{dest}|}|
is accomplished by:
%
\begin{center}
\begin{tabular}{l}
|{\edef\jobname{\scantokens\expandafter{\jobname\noexpand}}|\\
|\def\redirectjob |\textit{prefix}|#1~~~{\gdef\jobname{|\textit{dest}|#1}}|\\
|\expandafter\redirectjob\jobname~~~}\input{\jobname}|
\end{tabular}
\end{center}

In an alternative approach,
child documents can be compiled by a specific command line
without additional code or specific definitions:
%
\begin{center}
|... -jobname "|\textit{target}|" "|[\textit{flags}]%
|\includeonly{|\textit{dest}|}\input{|\textit{main}|}"|
\end{center}
%

%%%%%%%%%%%%%%%%%%%%%%%%%%%%%%%%%%%%%%%%%%%%%%%%%%%%%%%%%%%%%%%%%%%%%%%%%%%%%%%%
%%%%%%%%%%%%%%%%%%%%%%%%%%%%%%%%%%%%%%%%%%%%%%%%%%%%%%%%%%%%%%%%%%%%%%%%%%%%%%%%
\section{Information}

%%%%%%%%%%%%%%%%%%%%%%%%%%%%%%%%%%%%%%%%%%%%%%%%%%%%%%%%%%%%%%%%%%%%%%%%%%%%%%%%
\subsection{Copyright}

Copyright \copyright{} 2017--2018 Niklas Beisert

This work may be distributed and/or modified under the
conditions of the \LaTeX{} Project Public License, either version 1.3
of this license or (at your option) any later version.
The latest version of this license is in
  \url{http://www.latex-project.org/lppl.txt}
and version 1.3 or later is part of all distributions of \LaTeX{}
version 2005/12/01 or later.

This work has the LPPL maintenance status `maintained'.

The Current Maintainer of this work is Niklas Beisert.

This work consists of the files |README.txt|, |childdoc.ins| and |childdoc.dtx|
as well as the derived files |childdoc.def|, |cdocsamp.tex|
with |cdocsch1.tex|, |cdocsch2.tex|, |cdocspt3.tex|, |cdocspt4.tex|,
|cdocsdrf.tex|, |cdocsfn1.tex|, |cdocsfn2.tex|
as well as |childdoc.pdf|.

%%%%%%%%%%%%%%%%%%%%%%%%%%%%%%%%%%%%%%%%%%%%%%%%%%%%%%%%%%%%%%%%%%%%%%%%%%%%%%%%
\subsection{Files and Installation}

The package consists of the files:
%
\begin{center}
\begin{tabular}{ll}
    |README.txt|   & readme file \\
    |childdoc.ins| & installation file \\
    |childdoc.dtx| & source file \\
    |childdoc.def| & definition file \\
    |cdocsamp.tex| & sample main file \\
    |cdocsch1.tex| & sample include file \\
    |cdocsch2.tex| & sample include file \\
    |cdocspt3.tex| & sample part file \\
    |cdocspt4.tex| & sample part file \\
    |cdocsdrf.tex| & sample redirection file \\
    |cdocsfn1.tex| & sample redirection file \\
    |cdocsfn2.tex| & sample redirection file \\
    |childdoc.pdf| & manual
\end{tabular}
\end{center}
%
The distribution consists of the files
|README.txt|, |childdoc.ins| and |childdoc.dtx|.
%
\begin{itemize}
\item
Run (pdf)\LaTeX{} on |childdoc.dtx|
to compile the manual |childdoc.pdf| (this file).
\item
Run \LaTeX{} on |childdoc.ins| to create the definitions file |childdoc.def|
and the sample |cdocsamp.tex| with include files
|cdocsch1.tex|, |cdocsch2.tex|, |cdocspt3.tex|, |cdocspt4.tex|,
|cdocsdrf.tex|, |cdocsfn1.tex|, |cdocsfn2.tex|.
Then copy the file |childdoc.def| to an appropriate directory of your \LaTeX{}
distribution, e.g.\ \textit{texmf-root}|/tex/latex/childdoc|.
\end{itemize}

%%%%%%%%%%%%%%%%%%%%%%%%%%%%%%%%%%%%%%%%%%%%%%%%%%%%%%%%%%%%%%%%%%%%%%%%%%%%%%%%
\subsection{Related CTAN Packages}

There are several other packages which offer a similar functionality:
%
\begin{itemize}
\item
The packages
\href{http://ctan.org/pkg/docmute}{\textsf{docmute}},
\href{http://ctan.org/pkg/includex}{\textsf{includex}} and
\href{http://ctan.org/pkg/standalone}{\textsf{standalone}}
provide commands to include only the document body of
a child file thus allowing both files to be compiled individually.
\item
The packages \href{http://ctan.org/pkg/subdocs}{\textsf{subdocs}}
and \href{http://ctan.org/pkg/subfiles}{\textsf{subfiles}}
provide structures in which the main and child documents can be
encapsulated and allowing them to be compiled individually.
The inclusion mechanism is different from the conventional |\include|.
\item
The package \href{http://ctan.org/pkg/combine}{\textsf{combine}}
is an elaborate solution to combine several documents into one.
\end{itemize}
%
See also the CTAN topic \href{http://ctan.org/topic/subdocs}{\textsf{subdocs}}
for further related packages.
The present package differs from the above solutions in that
a document structure constructed with the conventional |\include| mechanism
just needs two extra commands at the top of every file
such that all constituent files can be compiled individually.

%%%%%%%%%%%%%%%%%%%%%%%%%%%%%%%%%%%%%%%%%%%%%%%%%%%%%%%%%%%%%%%%%%%%%%%%%%%%%%%%
%\subsection{Feature Suggestions}
%
%The following is a list of features which may be useful for future
%versions of this package:
%%
%\begin{itemize}
%\item
%\ldots
%\end{itemize}

%%%%%%%%%%%%%%%%%%%%%%%%%%%%%%%%%%%%%%%%%%%%%%%%%%%%%%%%%%%%%%%%%%%%%%%%%%%%%%%%
\subsection{Revision History}

%%%%%%%%%%%%%%%%%%%%%%%%%%%%%%%%%%%%%%%%
\paragraph{v2.0:} 2018/12/30

\begin{itemize}
\item
immediate forward processing
\item
added |\childdocby| mechanism
\item
manual restructured
\end{itemize}

%%%%%%%%%%%%%%%%%%%%%%%%%%%%%%%%%%%%%%%%
\paragraph{v1.6:} 2018/01/17

\begin{itemize}
\item
application for development of include files
\item
corrections to manual
\end{itemize}

%%%%%%%%%%%%%%%%%%%%%%%%%%%%%%%%%%%%%%%%
\paragraph{v1.5:} 2017/05/21

\begin{itemize}
\item
more complete structuring introduced
\item
|\childdocof| introduced
\item
|\childdoc| renamed to |\childdocmain|
\item
|\childredirect| renamed to |\childdocforward| and |\childdocforwardprefix|
and functionality expanded
\end{itemize}

%%%%%%%%%%%%%%%%%%%%%%%%%%%%%%%%%%%%%%%%
\paragraph{v1.0:} 2017/04/27

\begin{itemize}
\item
manual and install package
\item
first version published on CTAN
\end{itemize}

%%%%%%%%%%%%%%%%%%%%%%%%%%%%%%%%%%%%%%%%
\paragraph{v0.6:} 2017/04/26

\begin{itemize}
\item
redirection mechanism added
\end{itemize}

%%%%%%%%%%%%%%%%%%%%%%%%%%%%%%%%%%%%%%%%
\paragraph{v0.5:} 2017/04/26

\begin{itemize}
\item
functionality in definition file
\end{itemize}


%%%%%%%%%%%%%%%%%%%%%%%%%%%%%%%%%%%%%%%%%%%%%%%%%%%%%%%%%%%%%%%%%%%%%%%%%%%%%%%%
%%%%%%%%%%%%%%%%%%%%%%%%%%%%%%%%%%%%%%%%%%%%%%%%%%%%%%%%%%%%%%%%%%%%%%%%%%%%%%%%
%%%%%%%%%%%%%%%%%%%%%%%%%%%%%%%%%%%%%%%%%%%%%%%%%%%%%%%%%%%%%%%%%%%%%%%%%%%%%%%%
\appendix

\settowidth\MacroIndent{\rmfamily\scriptsize 000\ }

 \DocInput{childdoc.dtx}

\end{document}
%</driver>
% \fi
%
% %%%%%%%%%%%%%%%%%%%%%%%%%%%%%%%%%%%%%%%%%%%%%%%%%%%%%%%%%%%%%%%%%%%%%%%%%%%%%%
% %%%%%%%%%%%%%%%%%%%%%%%%%%%%%%%%%%%%%%%%%%%%%%%%%%%%%%%%%%%%%%%%%%%%%%%%%%%%%%
% \section{Sample}
%\iffalse
%<*samplemain>
%\fi
%
% The following presents a sample document
% with two chapters, two parts, a title page,
% a compile flag as well as three forwarding files to set the flag.
% It consists of eight |.tex| files:
% \begin{center}
% \begin{tabular}{ll}
% |cdocsamp.tex|&main file\\
% |cdocsch1.tex|&include file for chapter 1\\
% |cdocsch2.tex|&include file for chapter 2\\
% |cdocspt3.tex|&include file for part 3\\
% |cdocspt4.tex|&include file for part 4\\
% |cdocsdrf.tex|&forwarding file for main file in draft mode\\
% |cdocsfi1.tex|&forwarding file for final version of chapter 1\\
% |cdocsfi2.tex|&forwarding file for final version of chapter 2\\
% \end{tabular}
% \end{center}
% Each of the eight files can be compiled directly by the \LaTeX{} compiler.
%
% %%%%%%%%%%%%%%%%%%%%%%%%%%%%%%%%%%%%%%
% \paragraph{Main File.}
%
% The main file is called |cdocsamp.tex|.
%
% Load the \textsf{childdoc} definitions and
% declare the filename for the main document:
%    \begin{macrocode}
\input{childdoc.def}
\childdocmain{}
%    \end{macrocode}

% Optional override for |\version| flag:
%    \begin{macrocode}
%%\ifchilddoc\else\providecommand{\version}{draft}\fi
%    \end{macrocode}

% Define the default values for the |\version| flag
% (|final| for the main file and |draft| for childs):
%    \begin{macrocode}
\ifchilddoc
\providecommand{\version}{draft}
\else
\providecommand{\version}{final}
\fi
%    \end{macrocode}

% Load the standard document class:
%    \begin{macrocode}
\documentclass[12pt]{article}
%    \end{macrocode}

% Start the document body:
%    \begin{macrocode}
\begin{document}
%    \end{macrocode}

% Declare a title page.
% Print title, part of document being processed and version flag:
%    \begin{macrocode}
\addtocounter{page}{-1}
\begin{center}
{\LARGE\bfseries{}childdoc example\par}
\vspace{1cm}
\ifchilddoc
\ifchilddocmanual part\else chapter\fi:
`\childdocname' of `\childdocjob'\par
\else
main document: `\childdocjob'\par
\fi
version: \version\par
\end{center}
\newpage
%    \end{macrocode}

% Manually include selected file,
% otherwise process as usual:
%    \begin{macrocode}
\ifchilddocmanual
\section*{part `\childdocname'}
\input{\childdocname}
\else
%    \end{macrocode}

% Include the two chapters:
%    \begin{macrocode}
\include{cdocsch1}
\include{cdocsch2}
%    \end{macrocode}

% Include the two parts unless only chapters should be displayed:
%    \begin{macrocode}
\ifchilddoc\else
\section{part three}
\input{cdocspt3}
\section{part four}
\input{cdocspt4}
\fi
%    \end{macrocode}

% Process as usual until here:
%    \begin{macrocode}
\fi
%    \end{macrocode}

% End of document body:
%    \begin{macrocode}
\end{document}
%    \end{macrocode}
%\iffalse
%</samplemain>
%\fi
%
% %%%%%%%%%%%%%%%%%%%%%%%%%%%%%%%%%%%%%%
% \paragraph{Chapter Include Files.}
%
% The include files are called |cdocsch1.tex| and |cdocsch2.tex|.
%
%\iffalse
%<*samplechap1|samplechap2>
%\fi

% Optional override for |\version| flag:
%    \begin{macrocode}
%%\providecommand{\version}{final}
%    \end{macrocode}

% Include the main document:
%    \begin{macrocode}
\input{childdoc.def}
\childdocof{cdocsamp}
%    \end{macrocode}

%\iffalse
%</samplechap1|samplechap2>
%\fi
%
%\iffalse
%<*samplechap1>
%\fi
% Some text for chapter 1:
%    \begin{macrocode}
\section{one}
some text in chapter one
%    \end{macrocode}

%\iffalse
%</samplechap1>
%\fi
% Some text for chapter 2:
%\iffalse
%<*samplechap2>
%\fi
%    \begin{macrocode}
\section{two}
more text in chapter two
%    \end{macrocode}

%\iffalse
%</samplechap2>
%\fi
%
% %%%%%%%%%%%%%%%%%%%%%%%%%%%%%%%%%%%%%%
% \paragraph{Part Include Files.}
%
% The include files are called |cdocspt3.tex| and |cdocspt4.tex|.
%
%\iffalse
%<*samplepart3|samplepart4>
%\fi

% Optional override for |\version| flag:
%    \begin{macrocode}
%%\providecommand{\version}{final}
%    \end{macrocode}

% Include the main document:
%    \begin{macrocode}
\input{childdoc.def}
\childdocby{cdocsamp}
%    \end{macrocode}

%\iffalse
%</samplepart3|samplepart4>
%\fi
%
%\iffalse
%<*samplepart3>
%\fi
% Some text for part 3:
%    \begin{macrocode}
some text in part three
%    \end{macrocode}

%\iffalse
%</samplepart3>
%\fi
% Some text for part 4:
%\iffalse
%<*samplepart4>
%\fi
%    \begin{macrocode}
more text in part four
%    \end{macrocode}

%\iffalse
%</samplepart4>
%\fi
%
% %%%%%%%%%%%%%%%%%%%%%%%%%%%%%%%%%%%%%%
% \paragraph{Forwarding for a Complete Draft.}
%
% The following forwarding file |cdocsdrf.tex|
% compiles the main document in draft mode:
%\iffalse
%<*sampledraft>
%\fi
%    \begin{macrocode}
\def\version{draft}
\input{childdoc.def}
\childdocforward{cdocsamp}
%    \end{macrocode}

%\iffalse
%</sampledraft>
%\fi
%
% %%%%%%%%%%%%%%%%%%%%%%%%%%%%%%%%%%%%%%
% \paragraph{Forwarding for Final Version of the Chapters.}
%
% The following forwarding files |cdocsfn1.tex| and |cdocsfn2.tex|
% (with identical content)
% compile the final versions of the child documents
% |cdocsch1.tex| and |cdocsch2.tex|, respectively:
%\iffalse
%<*samplefinal>
%\fi
%    \begin{macrocode}
\def\version{final}
\input{childdoc.def}
\childdocforwardprefix[cdocsamp]{cdocsfn}{cdocsch}
%    \end{macrocode}

%\iffalse
%</samplefinal>
%\fi
%
% %%%%%%%%%%%%%%%%%%%%%%%%%%%%%%%%%%%%%%
% \paragraph{Command Line Processing.}
%
% The following three command lines generate the output files
% |cdocscld|, |cdocscl1| and |cdocscl2|
% which should be identical to
% |cdocsdrf|, |cdocsch1| and |cdocsfn2|, respectively:
% \begin{center}
% \begin{tabular}{l}
% |latex -jobname cdocscld \|\\
% |  "\def\version{draft}\input{childdoc.def}\childdocforward{cdocsamp}"|\\
% |latex -jobname cdocscl1 \|\\
% |  "\input{childdoc.def}\childdocforward[cdocsamp]{cdocsch1}"|\\
% |latex -jobname cdocscl2 \|\\
% |  "\def\version{final}\input{childdoc.def}\childdocforward{cdocsch2}"|
% \end{tabular}
% \end{center}
% Note that the trailing backslash on each first line
% merely continues the input to the second line
% (for convenient cut ant paste).
% Furthermore, the command |latex| can be replaced by any
% of its alternative versions such as |pdflatex|.
%
% %%%%%%%%%%%%%%%%%%%%%%%%%%%%%%%%%%%%%%%%%%%%%%%%%%%%%%%%%%%%%%%%%%%%%%%%%%%%%%
% %%%%%%%%%%%%%%%%%%%%%%%%%%%%%%%%%%%%%%%%%%%%%%%%%%%%%%%%%%%%%%%%%%%%%%%%%%%%%%
% \section{Implementation}
%\iffalse
%<*package>
%\fi
%
% This section describes the definitions file |childdoc.def|.

% The definitions cannot be loaded using |\usepackage| or |\RequirePackage|
% which has a mechanism to prevent loading a style file more than once.
% When loading the definitions by means of |\input|
% multiple instances have to be prevented manually:
%\iffalse
%This code needs to be before the `\ProvidesFile' directive
%which is defined at the beginning of this file.
%Therefore it is also placed there and commented out here.
%</package>
%<*discard>
%\fi
%    \begin{macrocode}
\ifdefined\childdocmain\endinput\fi
%    \end{macrocode}
%\iffalse
%</discard>
%<*package>
%\fi
%
% \macro{\ifchilddoc}
% \macro{\ifchilddocmanual}
% The conditional |\ifchilddoc| tells whether a
% child (true) or main (false) document is being compiled.
% The conditional |\ifchilddocmanual| tells whether
% the |\includeonly| mechanism is used (false) or
% the selection of child files must be performed manually (true).
% The definitions initialise to false:
%    \begin{macrocode}
\newif\ifchilddoc
\newif\ifchilddocmanual
%    \end{macrocode}

% \macro{\childdocname}
% \macro{\childdocjob}
% The macro |\childdocname| stores the name of the main document
% to be compiled. The macro |\childdocjob| stores the name of
% the document on which the \LaTeX{} compiler was originally invoked.
% The content of |\jobname| cannot be compared
% to filenames specified in the source due to different catcodes.
% The following code rescans |\jobname|, stores the result
% in |\childdocname| and saves a copy in |\childdocjob|:
%    \begin{macrocode}
\edef\childdocname{\scantokens\expandafter{\jobname\noexpand}}
\let\childdocjob\childdocname
%    \end{macrocode}

% \macro{\childdocdisable}
% The macro |\childdocdisable| prevents the main file
% from being processed more than once.
% At this stage, the main document command |\childdocmain|
% is assumed to be called once again where it should do nothing.
% Any subsequent call to it should prevent
% a secondary processing of the main document
% It overwrites the forwarding commands
% |\childdocof| and |\childdocforward|
% with empty macros to prevent further inclusions of the main document:
%    \begin{macrocode}
\newcommand{\childdocdisable}
{
  \renewcommand{\childdocmain}[1]{\renewcommand{\childdocmain}[1]{\endinput}}
  \renewcommand{\childdocof}[1]{}
  \renewcommand{\childdocby}[2][]{}
  \renewcommand{\childdocforward}[2][]{}
  \renewcommand{\childdocdisable}{}
}
%    \end{macrocode}

% \macro{\childdocmain}
% The macro |\childdocmain| is to be called at the top of the main file
% with nothing or the main filename (without extension) as argument.
% First, it breaks loops.
% If the argument is not empty and does not match |\childdocname|
% (which is set by the first inclusion of |childdoc.def|),
% |\ifchilddoc| is set to true, |\includeonly| is applied to the child file
% and |\jobname| is set to the main file
% (for proper handling of |.aux| files):
%    \begin{macrocode}
\newcommand{\childdocmain}[1]
{
  \childdocdisable\childdocmain{}
  \if?#1?\else
    \begingroup
      \def\childdoctmp{#1}
      \ifx\childdoctmp\childdocname
        \def\childdoctmp{}
      \else
        \def\childdoctmp
        {
          \childdoctrue
          \includeonly{\childdocname}
          \def\childdocjob{#1}
          \def\jobname{#1}
        }
      \fi
      \expandafter
    \endgroup
    \childdoctmp
  \fi
}
%    \end{macrocode}

% \macro{\childdocof}
% The command |\childdocof| redirects
% compilation to the main file |#1|.
%    \begin{macrocode}
\newcommand{\childdocof}[1]
{
  \childdocdisable
  \childdoctrue
  \includeonly{\childdocname}
  \def\jobname{#1}
  \def\childdocjob{#1}
  \input{#1}
}
%    \end{macrocode}

% \macro{\childdocby}
% The command |\childdocby| ....
%    \begin{macrocode}
\newcommand{\childdocby}[2][]
{
  \childdocdisable
  \childdoctrue
  \childdocmanualtrue
  \if?#1?\else
    \def\jobname{#2}
  \fi
  \def\childdocjob{#2}
  \input{#2}
  \endinput
}
%    \end{macrocode}

% \macro{\childdocforward}
% The command |\childdocforward| redirects
% compilation to the main file or
% (if the optional argument is given) a child file.
% Parameters are set as if the main file
% or a child file starting with |\childdocof| was compiled.
% Then compilation is handed over to the main file:
%    \begin{macrocode}
\newcommand{\childdocforward}[2][]
{
  \begingroup
    \if?#1?
      \def\childdoctmp
      {
        \def\childdocname{#2}
        \def\childdocjob{#2}
        \def\jobname{#2}
        \input{#2}
        \endinput
      }
    \else
      \def\childdoctmp
      {
        \childdocdisable
        \def\childdocname{#2}
        \childdoctrue
        \includeonly{#2}
        \def\childdocjob{#1}
        \def\jobname{#1}
        \input{#1}
        \endinput
      }
    \fi
    \expandafter
  \endgroup
  \childdoctmp
}
%    \end{macrocode}

% \macro{\childdocforwardprefix}
% The command |\childdocforwardprefix| redirects
% compilation to the main or a child file by means of a pattern.
% The prefix |#1| in the current filename is replaced by |#2|
% and the suffix of the current filename is kept
% (it is assumed that the filename does not contain the substring `|~~~|'
% which is used as a delimiter).
% Compilation is handed over to the new file by |\childdocforward|:
%    \begin{macrocode}
\newcommand{\childdocforwardprefix}[3][]
{
  \begingroup
    \def\childdocextract #2##1~~~{\def\childdoctmp{\childdocforward[#1]{#3##1}}}
    \expandafter\childdocextract\childdocname~~~
    \expandafter
  \endgroup
  \childdoctmp
}
%    \end{macrocode}

% \macro{\childdoc}
% The deprecated macro |\childdoc| is a legacy version of |\childdocmain|:
%    \begin{macrocode}
\newcommand{\childdoc}{\childdocmain}
%    \end{macrocode}

% \macro{\childdocredirect}
% The deprecated macro |\childdocredirect| is a legacy version
% of |\childdocforward| and |\childdocforwardprefix|:
%    \begin{macrocode}
\newcommand{\childdocredirect}[2][]
{
  \begingroup
    \if?#1?
      \def\childdoctmp{\childdocforward{#2}}
    \else
      \def\childdoctmp{\childdocforwardprefix{#1}{#2}}
    \fi
    \expandafter
  \endgroup
  \childdoctmp
}
%    \end{macrocode}

%\iffalse
%</package>
%\fi
%
\endinput
|\\
|\childdocforwardprefix[|\textit{main}|]{|\textit{prefix}|}{|\textit{dest}|}|
\end{tabular}
\end{center}
%
the destination file is determined by a pattern
depending on the current file:
To make this work, the current file must be called
`{\textit{prefix}\hspace{0.2em}\textit{suffix}}'
with \textit{prefix} matching precisely the argument.
Processing is then passed on to the file
`{\textit{dest}\hspace{0.2em}\textit{suffix}}'.
Surely, the same effect is achieved by
directly specifying the
argument `{\textit{dest}\hspace{0.2em}\textit{suffix}}'
in the first form.
However, that requires to set up a different file
for each child. With the alternative form of the command
all these files can have exactly the same content
which simplifies setting them up and maintaining them.

For example, the following file |draft.tex|
with a compilation flag |\version| as described in \secref{sec:flags}
compiles the main document as a draft:
%
\begin{center}
\begin{tabular}{l}
|\def\version{draft}|\\
|% \iffalse
%
% childdoc.dtx Copyright (C) 2017-2018 Niklas Beisert
%
% This work may be distributed and/or modified under the
% conditions of the LaTeX Project Public License, either version 1.3
% of this license or (at your option) any later version.
% The latest version of this license is in
%   http://www.latex-project.org/lppl.txt
% and version 1.3 or later is part of all distributions of LaTeX
% version 2005/12/01 or later.
%
% This work has the LPPL maintenance status `maintained'.
%
% The Current Maintainer of this work is Niklas Beisert.
%
% This work consists of the files childdoc.dtx and childdoc.ins
% and the derived files childdoc.def and cdocsamp.tex with
% cdocsch1.tex, cdocsch2.tex, cdocsdrf.tex, cdocsfn1.tex, cdocsfn2.tex.
%
%<package>\ifdefined\childdocmain\endinput\fi
%<package>\ProvidesFile{childdoc.def}[2018/12/30 v2.0 child document driver]
%<samplemain>\ProvidesFile{cdocsamp.tex}[2018/12/30 v2.0 sample for childdoc]
%<*driver>
%\ProvidesFile{childdoc.drv}[2018/12/30 v2.0 childdoc reference manual file]
\PassOptionsToClass{10pt,a4paper}{article}
\documentclass{ltxdoc}

\usepackage[margin=35mm]{geometry}
\usepackage{hyperref}
\usepackage{hyperxmp}
\usepackage[usenames]{color}

\hypersetup{colorlinks=true}
\hypersetup{pdfstartview=FitH}
\hypersetup{pdfpagemode=UseNone}
\hypersetup{pdfsource={}}
\hypersetup{pdflang={en-UK}}
\hypersetup{pdfcopyright={Copyright 2017-2018 Niklas Beisert.
  This work may be distributed and/or modified under the
  conditions of the LaTeX Project Public License, either version 1.3
  of this license or (at your option) any later version.}}
\hypersetup{pdflicenseurl={http://www.latex-project.org/lppl.txt}}
\hypersetup{pdfcontactaddress={ETH Zurich, ITP, HIT K,
  Wolfgang-Pauli-Strasse 27}}
\hypersetup{pdfcontactpostcode={8093}}
\hypersetup{pdfcontactcity={Zurich}}
\hypersetup{pdfcontactcountry={Switzerland}}
\hypersetup{pdfcontactemail={nbeisert@itp.phys.ethz.ch}}
\hypersetup{pdfcontacturl={http://people.phys.ethz.ch/\xmptilde nbeisert/}}

\newcommand{\secref}[1]{\hyperref[#1]{section \ref*{#1}}}

\parskip1ex
\parindent0pt
\let\olditemize\itemize
\def\itemize{\olditemize\parskip0pt}

\begin{document}

\title{The \textsf{childdoc} Package}
\hypersetup{pdftitle={The childdoc Package}}
\author{Niklas Beisert\\[2ex]
  Institut f\"ur Theoretische Physik\\
  Eidgen\"ossische Technische Hochschule Z\"urich\\
  Wolfgang-Pauli-Strasse 27, 8093 Z\"urich, Switzerland\\[1ex]
  \href{mailto:nbeisert@itp.phys.ethz.ch}
  {\texttt{nbeisert@itp.phys.ethz.ch}}}
\hypersetup{pdfauthor={Niklas Beisert}}
\hypersetup{pdfsubject={Manual for the LaTeX2e Package childdoc}}
\date{30 December 2018, \textsf{v2.0}}
\maketitle

\begin{abstract}\noindent
\textsf{childdoc} is a \LaTeXe{} package
that enables the direct compilation
of document sections included by |\include|
to individual files.
\end{abstract}

\begingroup
\parskip0ex
\tableofcontents
\endgroup

%%%%%%%%%%%%%%%%%%%%%%%%%%%%%%%%%%%%%%%%%%%%%%%%%%%%%%%%%%%%%%%%%%%%%%%%%%%%%%%%
%%%%%%%%%%%%%%%%%%%%%%%%%%%%%%%%%%%%%%%%%%%%%%%%%%%%%%%%%%%%%%%%%%%%%%%%%%%%%%%%
\section{Introduction}

\LaTeX{} provides a mechanism to structure a large document (such as a book)
into a main file and several child files (containing the chapters)
using the |\include| command.
This mechanism is beneficial for documents
which span hundreds of pages in order to
make the source file(s) more manageable.
Moreover, compilation can be restricted to
selected child files by means of the |\includeonly| command.
The latter feature can be used to reduce the compilation time while editing
(this was significantly more useful in the earlier days of \LaTeX{})
or to generate a smaller document which is easier to navigate.
Another application of |\includeonly| is to generate
documents consisting of selected parts of the complete document.

However, there are a few drawbacks of the plain |\include| mechanism:
\begin{itemize}
\item
The child files cannot be compiled on their own,
they can only be compiled via the main file.
A naive editing environment
(such as a text editor with an option
to have the current file processed by \LaTeX)
may require one to switch to the main file before compiling;
attempting to compile the child file produces errors.
\item
The main file must be modified (each time)
to adjust the |\includeonly| command
to the present needs. This easily leaves the main file in a messy state.
\item
The generated document will always carry the filename
of the main document. This is inconvenient if
several child files are to be compiled and
to be kept for distribution.
\end{itemize}

The present package provides a simple interface
to make child files individually compilable by \LaTeX{}.
Compiling a child file then has the same effect as compiling
the main file with an |\includeonly| command
to select the appropriate child.
Moreover the generated document will carry the name of the child
rather than the main file.
This resolves all three above issues.

This feature is meant to make the editing of books,
thesis documents and lecture notes somewhat more convenient.
However, the package can also be used efficiently for
composing a series of documents (such as exercise sheets)
which are typically distributed individually.
It then assists the author in generating the individual documents
(potentially in different versions)
as well as a document containing the collected series.
Another application is in developing style files
or other kinds of included material
where compilation of the style file could redirect
to a sample or test file.

%%%%%%%%%%%%%%%%%%%%%%%%%%%%%%%%%%%%%%%%%%%%%%%%%%%%%%%%%%%%%%%%%%%%%%%%%%%%%%%%
%%%%%%%%%%%%%%%%%%%%%%%%%%%%%%%%%%%%%%%%%%%%%%%%%%%%%%%%%%%%%%%%%%%%%%%%%%%%%%%%
\section{Usage}

First of all, the package \textsf{childdoc} is \emph{not} a standard
\LaTeXe{} |.sty| style file! Therefore it needs to be invoked in
a non-standard way.

%%%%%%%%%%%%%%%%%%%%%%%%%%%%%%%%%%%%%%%%%%%%%%%%%%%%%%%%%%%%%%%%%%%%%%%%%%%%%%%%
\subsection{Included Files}
\label{sec:include}

%%%%%%%%%%%%%%%%%%%%%%%%%%%%%%%%%%%%%%%%
\DescribeMacro{\childdocmain}
To use the package, add the commands
\begin{center}
\begin{tabular}{l}
|\input{childdoc.def}|\\
|\childdocmain{}|\\
\end{tabular}
\end{center}
at the very top of the main \LaTeX{} file,
in particular \emph{before} the |\documentclass| statement!
The argument of |\childdocmain| should be left empty
(but it must be present).

%%%%%%%%%%%%%%%%%%%%%%%%%%%%%%%%%%%%%%%%
\DescribeMacro{\childdocof}
Furthermore, add the commands
\begin{center}
\begin{tabular}{l}
|\input{childdoc.def}|\\
|\childdocof{|\textit{main}|}|\\
\end{tabular}
\end{center}
at the top of every child file \textit{child}
which is included by |\include{|\textit{child}|}|
from within the main file
(or at least for those files to be compiled individually).
The argument \textit{main} must be the filename of the main file.

There are a couple of
considerations in setting up the main and child documents:

%%%%%%%%%%%%%%%%%%%%%%%%%%%%%%%%%%%%%%%%
\paragraph{Restrictions.}

Please note the following restrictions:
\begin{itemize}
\item
|\childdocmain| must be called with one argument \textit{main}
to ensure compatibility with earlier version of the package.
It must either be empty (|\childdocmain{}|)
or precisely match the filename of the main file in which it is specified.
See \secref{sec:detection} for further information.
\item
The filename \textit{main} must be specified without the |.tex| extension.
\item
The filename \textit{main} is case sensitive
(even in case-insensitive file systems)
due to internal string comparison.
\item
The argument \textit{main} should be fully expanded, it cannot be a macro.
\item
Subdirectories and special characters should be avoided in filenames.
\item
The command |\childdocmain{|\textit{main}|}| must be followed by a whitespace.
It should not be followed immediately by another command
or by a comment mark `|%|'.
This is because the \TeX{} parser reads the token immediately following
the argument of |\childdocmain| and puts it
at the beginning of every child section;
however, a white\-space is ignored.
\end{itemize}

%%%%%%%%%%%%%%%%%%%%%%%%%%%%%%%%%%%%%%%%
\paragraph{Content of Main File.}

It is advisable to place all content in the child files included by |\include|.
Any output contained in the main file will appear in all child documents
unless suppressed manually;
it cannot be suppressed automatically by the |\includeonly| directive
and thus should normally be avoided.
A method to include some content in the main file
by means of conditional processing is described in \secref{sec:conditional}.

%%%%%%%%%%%%%%%%%%%%%%%%%%%%%%%%%%%%%%%%
\paragraph{Page Numbering.}

When only a part of the document is compiled,
the appropriate numbering of pages
(as well as other status parameters)
is determined from the |.aux| files.
The latter contain information from previous passes.
However this information needs to propagate through
all intermediate child documents.
Therefore the page numbering in child documents may well
be inconsistent until the complete document is compiled at least once.

A useful (if unconventional) way to always ensure a consistent
page numbering is to restart the numbering in each child document
and denote the pages by `\textit{child}|.|\textit{page}'
where \textit{child} represents the chapter/section number of the child file.
This can be achieved by the command
|\numberwithin{page}{|\textit{child}|}|
of the \textsf{amsmath} package
where \textit{child} can be |chapter| or |section|
depending on the chosen structuring.
Alternatively, one can modify the macro |\thepage| appropriately
and reset the counter |page| at the start of each child file.

%%%%%%%%%%%%%%%%%%%%%%%%%%%%%%%%%%%%%%%%%%%%%%%%%%%%%%%%%%%%%%%%%%%%%%%%%%%%%%%%
\subsection{Conditional Processing}
\label{sec:conditional}

The package provides a mechanism to compile different versions
of a document. To customise the versions further some conditional processing
can come in handy to distinguish which version is being compiled.
The package provides two macros to describe the compilation context:

%%%%%%%%%%%%%%%%%%%%%%%%%%%%%%%%%%%%%%%%
\DescribeMacro{\ifchilddoc}
The conditional |\ifchilddoc| distinguishes between the compilation of
child documents and the main document:
%
\begin{center}
|\ifchilddoc |\textit{child-code}| |[|\||else |\textit{main-code}]| \||fi|
\end{center}

%%%%%%%%%%%%%%%%%%%%%%%%%%%%%%%%%%%%%%%%
\DescribeMacro{\childdocname}
\DescribeMacro{\childdocjob}
The macro |\childdocname| contains the filename (without extension)
of the main or child file being processed.
Note that |\childdocjob| will always contain the name of the main file.

%%%%%%%%%%%%%%%%%%%%%%%%%%%%%%%%%%%%%%%%
\paragraph{Title Page.}

Conditional processing can be used to include a title or banner page
in the main document when proper precautions are taken.
Importantly, the code in the main file should ensure that the page counter
(as well as other status parameters which are stored in the |.aux| files)
takes the same value after the conditional processing.
Otherwise the page numbers may take divergent values
depending on which part is compiled.

For example, a title page could be declared by:
%
\begin{center}
\begin{tabular}{l}
|\ifchilddoc\||else|\\
|\addtocounter{page}{-1}|\\
\textit{code for title page}\\
|\newpage|\\
|\||fi|
\end{tabular}
\end{center}
%
A banner page for the child documents can be generated by:
%
\begin{center}
\begin{tabular}{l}
|\ifchilddoc|\\
|\addtocounter{page}{-1}|\\
\textit{code for banner page}\\
|\newpage|\\
|\||fi|
\end{tabular}
\end{center}
%
Here one could write a message such as:
\begin{center}
|This is the part \childdocname{} of \childdocjob{}.|
\end{center}

%%%%%%%%%%%%%%%%%%%%%%%%%%%%%%%%%%%%%%%%%%%%%%%%%%%%%%%%%%%%%%%%%%%%%%%%%%%%%%%%
\subsection{Flags}
\label{sec:flags}

The package makes it easy to generate different versions
of the main or child documents.
To this end compilation flags can be defined
and assigned different default values.
They will be particularly useful in conjunction
with the forwarding mechanism described in \secref{sec:forward}.

For example, it may be useful to have a flag |\version|
which can be set to |draft| or |final|.
The document source will contain some conditional code
depending on the value of |\version|.
Suppose further, the flag should default to |final| for the main file
and to |draft| for child files
which is a natural assignment for editing the document.
This is achieved by placing the following code
in the preamble of the main document
(below the |\childdocmain| directive):
%
\begin{center}
\begin{tabular}{l}
|\ifchilddoc|\\
|\providecommand{\version}{draft}|\\
|\||else|\\
|\providecommand{\version}{final}|\\
|\||fi|
\end{tabular}
\end{center}
%
The definition by |\providecommand| makes sure
that previous definitions are not overwritten.
Further statements |\providecommand{\version}{...}|
can thus be added before the above code to override it.

For the main file, one might add a line
(between |\childdocmain| and the above block)
%
\begin{center}
|%\ifchilddoc\||else\providecommand{\version}{draft}\||fi|
\end{center}
%
which can be uncommented to produce a draft version.
Likewise one can add a line to the very top of a child file
(above the |\childdocof{|\textit{main}|}| directive)
%
\begin{center}
|%\providecommand{\version}{final}|
\end{center}
%
which can be uncommented to produce the final version of this child document.

%%%%%%%%%%%%%%%%%%%%%%%%%%%%%%%%%%%%%%%%%%%%%%%%%%%%%%%%%%%%%%%%%%%%%%%%%%%%%%%%
\subsection{Forwarding}
\label{sec:forward}

Different versions of the main or child documents
using compilation flags as described in \secref{sec:flags}
can be (permanently) stored in different files
for convenient compilation, viewing and distribution.
To this end, the package defines a command
to pass on compilation to a different file:

%%%%%%%%%%%%%%%%%%%%%%%%%%%%%%%%%%%%%%%%
\DescribeMacro{\childdocforward}
The command |\childdocforward| redirects processing to
another source file:
%
\begin{center}
\begin{tabular}{l}
|\input{childdoc.def}|\\
|\childdocforward[|\textit{main}|]{|\textit{dest}|}|\\
\end{tabular}
\end{center}
%
The argument \textit{dest} is the destination file
(without extension).
It should be the main file or one of the child files.
Note that further \textsf{childdoc} directives
such as |\childdocof| and |\childdocforward|
in the indicated file will be processed in this form.
The optional argument \textit{main}
passes on directly to the main file \textit{main}
while pretending to compile the child \textit{dest}.
This form behaves as if \textit{dest}
issues |\childdocof{|\textit{main}|}| right away,
and no further \textsf{childdoc} directives will be processed.

%%%%%%%%%%%%%%%%%%%%%%%%%%%%%%%%%%%%%%%%
\DescribeMacro{\...prefix}
In the alternative form |\childdocforwardprefix|,
%
\begin{center}
\begin{tabular}{l}
|\input{childdoc.def}|\\
|\childdocforwardprefix[|\textit{main}|]{|\textit{prefix}|}{|\textit{dest}|}|
\end{tabular}
\end{center}
%
the destination file is determined by a pattern
depending on the current file:
To make this work, the current file must be called
`{\textit{prefix}\hspace{0.2em}\textit{suffix}}'
with \textit{prefix} matching precisely the argument.
Processing is then passed on to the file
`{\textit{dest}\hspace{0.2em}\textit{suffix}}'.
Surely, the same effect is achieved by
directly specifying the
argument `{\textit{dest}\hspace{0.2em}\textit{suffix}}'
in the first form.
However, that requires to set up a different file
for each child. With the alternative form of the command
all these files can have exactly the same content
which simplifies setting them up and maintaining them.

For example, the following file |draft.tex|
with a compilation flag |\version| as described in \secref{sec:flags}
compiles the main document as a draft:
%
\begin{center}
\begin{tabular}{l}
|\def\version{draft}|\\
|\input{childdoc.def}|\\
|\childdocforward{|\textit{main}|}|
\end{tabular}
\end{center}
%
Likewise, the following files |final|\textit{nn}|.tex|
compile the final version of the child document
|child|\textit{nn}|.tex|:
%
\begin{center}
\begin{tabular}{l}
|\def\version{final}|\\
|\input{childdoc.def}|\\
|\childdocforwardprefix{final}{child}|
\end{tabular}
\end{center}
%

Note that when several versions of a main file and/or of each child file
are to be generated, it may be convenient to set up a |Makefile| or
shell script to automatise the process.

%%%%%%%%%%%%%%%%%%%%%%%%%%%%%%%%%%%%%%%%%%%%%%%%%%%%%%%%%%%%%%%%%%%%%%%%%%%%%%%%
\subsection{Command Line Processing}
\label{sec:commandline}

The effect of redirection files can also be achieved by invoking
the \LaTeX{} compiler with a more elaborate command line.
Most conveniently this should be done as part
of a shell script or a |Makefile|.

When using \textsf{childdoc} in the main file, the following
command lines effectively perform a redirection
(note that depending on the shell being used,
backslashes may have to be doubled: `|\|' $\to$ `|\\|'):
%
\begin{center}
|... -jobname "|\textit{target}|" |\\|"|[\textit{flags}]%
|\input{childdoc.def}\childdocforward[|\textit{main}|]{|\textit{dest}|}"|
\end{center}
%
Here \textit{target} is the name of the output file,
\textit{main} is the name of the main file
and \textit{dest} is the name of the main or child file to be processed
(all filenames without extensions).
The optional argument \textit{main} can be omitted
if \textit{main} matches \textit{dest}.
Optionally, compilation \textit{flags} can be defined via |\def| commands.
This command line makes the \TeX{} engine believe
it is compiling the file \textit{target}
whose content is specified as the latter parameter.
The provided code then forwards the processing to
\textit{main} or \textit{dest} as described in \secref{sec:forward}.

%%%%%%%%%%%%%%%%%%%%%%%%%%%%%%%%%%%%%%%%%%%%%%%%%%%%%%%%%%%%%%%%%%%%%%%%%%%%%%%%
\subsection{Include by Input}
\label{sec:input}

Including child documents by |\include| has some restrictions by design.
Most notably, the content of a child document always occupies
its own set of pages; pages cannot be shared between child documents.
Usually, this behaviour makes perfect sense
because each child document contain an essential part of the document.
However, in some situations it may be desirable to compose
a document from a collection of parts
without having mandatory page breaks between then.
For this case, the package
provides a mechanism to include parts
by |\input| which can also be processed individually.
However, by construction this mechanism
requires manual handling of the content to be output.

%%%%%%%%%%%%%%%%%%%%%%%%%%%%%%%%%%%%%%%%
\DescribeMacro{\ifchilddocmanual}
The main file should be prepared as usual, see \secref{sec:include}.
However, the document body must make a distinction
between processing of an individual part and of the main document, e.g.:
%
\begin{center}
\begin{tabular}{l}
|\ifchilddocmanual|\\
|\input{\childdocname}|\\
|\||else|\\
\textit{document body with }|\input{|\textit{part}|}|\\
|\||fi|
\end{tabular}
\end{center}
%
The conditional |\ifchilddocmanual| is true whenever
a part to be included by |\input| is being compiled,
and the name of the part is stored in |\childdocname|.

%%%%%%%%%%%%%%%%%%%%%%%%%%%%%%%%%%%%%%%%
\DescribeMacro{\childdocby}
Each part to be included by |\input| should start with:
%
\begin{center}
\begin{tabular}{l}
|\input{childdoc.def}|\\
|\childdocby{|\textit{main}|}|\\
\end{tabular}
\end{center}
%
The directive |\childdocby| is similar to |\childdocof|
described in \secref{sec:include},
but the subsequent selection of content must be done manually.
To that end, both |\ifchilddoc| and |\ifchilddocmanual|
will be true upon processing of a part,
and the name of the part is stored in |\childdocname|.
Note that |\jobname| will be set to the filename of the current part
so that each part receives an individual |.aux| file
that does not interfere with the |.aux| file(s) of the main document.
This behaviour can be altered by the alternative form
|\childdocby[*]{|\textit{main}|}| (with a non-empty optional argument)
which uses the |.aux| file of the main document
by setting |\jobname| to \textit{main}.

%%%%%%%%%%%%%%%%%%%%%%%%%%%%%%%%%%%%%%%%%%%%%%%%%%%%%%%%%%%%%%%%%%%%%%%%%%%%%%%%
\subsection{Driver Development}
\label{sec:driver}

The \textsf{childdoc} mechanism can also be use for the development
of definition files such as \LaTeX{} styles or classes.
This case differs from the above setup with multiple parts
included by |\include| in that no |\includeonly| should be invoked.
This can be achieved by starting the include file
(before |\ProvidesPackage|) with:
%
\begin{center}
\begin{tabular}{l}
|\input{childdoc.def}|\\
|\childdocforward{|\textit{main}|}|\\
\end{tabular}
\end{center}
%
or alternatively with:
%
\begin{center}
\begin{tabular}{l}
|\input{childdoc.def}|\\
|\childdocby{|\textit{main}|}|\\
\end{tabular}
\end{center}
%
Both forms have slightly different effects as described above.
The main file is prepared as usual, see \secref{sec:include}.

%%%%%%%%%%%%%%%%%%%%%%%%%%%%%%%%%%%%%%%%%%%%%%%%%%%%%%%%%%%%%%%%%%%%%%%%%%%%%%%%
\subsection{Legacy Detection}
\label{sec:detection}

The directive |\childdocmain| in the main file can detect
whether the complete document or merely a child is to be compiled
even without using the directive |\childdocof|.
This method is deprecated because it is less robust
and there is no compelling reason to use it;
it is merely provided for backward compatibility
and it may be removed in future versions.

If the detection mechanism is to be used,
it is mandatory to correctly specify
the filename of the main file as the argument of |\childdocmain|:
%
\begin{center}
\begin{tabular}{l}
|\input{childdoc.def}|\\
|\childdocmain{|\textit{main}|}|\\
\end{tabular}
\end{center}
%
If |\jobname| does not match the argument \textit{main} of |\childdocmain|,
it is assumed that |\jobname| points to the child file to be compiled.
When using |\childdocmain| with the main file specified as argument,
it suffices to start a child file
with just |\input{|\textit{main}|}|
without loading of the package and using |\childdocof|.
If instead all processing is done
with the appropriate \textsf{childdoc} directives,
the argument of \textit{main} of |\childdocmain| can be empty.

An alternative version of the command line processing described
in \secref{sec:commandline} using the detection mechanism reads:
%
\begin{center}
|... -jobname "|\textit{target}|" "|[\textit{flags}]%
[|\def\jobname{|\textit{dest}|}|]|\input{|\textit{main}|}"|
\end{center}

%%%%%%%%%%%%%%%%%%%%%%%%%%%%%%%%%%%%%%%%%%%%%%%%%%%%%%%%%%%%%%%%%%%%%%%%%%%%%%%%
\subsection{Manual Code}
\label{sec:manual}

In case one cannot be certain whether the definitions file |childdoc.def|
is installed on the target \TeX{} distribution
and one prefers not to ship it,
it is conceivable to paste a few relevant commands into the sources.

To that end, drop all statements |\input{childdoc.def}|
and perform the replacements as outlined below.
Instead of |\childdocmain{|\textit{main}|}| add the following code
to the top of the main file:
%
\begin{center}
\begin{tabular}{l}
|\||ifdefined\childdocname\endinput\||fi\newif\ifchilddoc|\\
|\edef\childdocname{\scantokens\expandafter{\jobname\noexpand}}|\\
|\def\childdocmain{|\textit{main}|}\||ifx\childdocmain\childdocname\||else|\\
|\childdoctrue\includeonly{\childdocname}\let\jobname\childdocmain\||fi|\\
\end{tabular}
\end{center}
%
Instead of |\childdocof{|\textit{main}|}| just include the main file
at the top of each child file:
%
\begin{center}
|\input{|\textit{main}|}|
\end{center}
%
A simple redirection |\childdocforward{|\textit{dest}|}| is achieved by:
%
\begin{center}
|\def\jobname{|\textit{dest}|}\input{\jobname}|
\end{center}
%
The redirection with prefix
|\childdocforwardprefix[|\textit{prefix}|]{|\textit{dest}|}|
is accomplished by:
%
\begin{center}
\begin{tabular}{l}
|{\edef\jobname{\scantokens\expandafter{\jobname\noexpand}}|\\
|\def\redirectjob |\textit{prefix}|#1~~~{\gdef\jobname{|\textit{dest}|#1}}|\\
|\expandafter\redirectjob\jobname~~~}\input{\jobname}|
\end{tabular}
\end{center}

In an alternative approach,
child documents can be compiled by a specific command line
without additional code or specific definitions:
%
\begin{center}
|... -jobname "|\textit{target}|" "|[\textit{flags}]%
|\includeonly{|\textit{dest}|}\input{|\textit{main}|}"|
\end{center}
%

%%%%%%%%%%%%%%%%%%%%%%%%%%%%%%%%%%%%%%%%%%%%%%%%%%%%%%%%%%%%%%%%%%%%%%%%%%%%%%%%
%%%%%%%%%%%%%%%%%%%%%%%%%%%%%%%%%%%%%%%%%%%%%%%%%%%%%%%%%%%%%%%%%%%%%%%%%%%%%%%%
\section{Information}

%%%%%%%%%%%%%%%%%%%%%%%%%%%%%%%%%%%%%%%%%%%%%%%%%%%%%%%%%%%%%%%%%%%%%%%%%%%%%%%%
\subsection{Copyright}

Copyright \copyright{} 2017--2018 Niklas Beisert

This work may be distributed and/or modified under the
conditions of the \LaTeX{} Project Public License, either version 1.3
of this license or (at your option) any later version.
The latest version of this license is in
  \url{http://www.latex-project.org/lppl.txt}
and version 1.3 or later is part of all distributions of \LaTeX{}
version 2005/12/01 or later.

This work has the LPPL maintenance status `maintained'.

The Current Maintainer of this work is Niklas Beisert.

This work consists of the files |README.txt|, |childdoc.ins| and |childdoc.dtx|
as well as the derived files |childdoc.def|, |cdocsamp.tex|
with |cdocsch1.tex|, |cdocsch2.tex|, |cdocspt3.tex|, |cdocspt4.tex|,
|cdocsdrf.tex|, |cdocsfn1.tex|, |cdocsfn2.tex|
as well as |childdoc.pdf|.

%%%%%%%%%%%%%%%%%%%%%%%%%%%%%%%%%%%%%%%%%%%%%%%%%%%%%%%%%%%%%%%%%%%%%%%%%%%%%%%%
\subsection{Files and Installation}

The package consists of the files:
%
\begin{center}
\begin{tabular}{ll}
    |README.txt|   & readme file \\
    |childdoc.ins| & installation file \\
    |childdoc.dtx| & source file \\
    |childdoc.def| & definition file \\
    |cdocsamp.tex| & sample main file \\
    |cdocsch1.tex| & sample include file \\
    |cdocsch2.tex| & sample include file \\
    |cdocspt3.tex| & sample part file \\
    |cdocspt4.tex| & sample part file \\
    |cdocsdrf.tex| & sample redirection file \\
    |cdocsfn1.tex| & sample redirection file \\
    |cdocsfn2.tex| & sample redirection file \\
    |childdoc.pdf| & manual
\end{tabular}
\end{center}
%
The distribution consists of the files
|README.txt|, |childdoc.ins| and |childdoc.dtx|.
%
\begin{itemize}
\item
Run (pdf)\LaTeX{} on |childdoc.dtx|
to compile the manual |childdoc.pdf| (this file).
\item
Run \LaTeX{} on |childdoc.ins| to create the definitions file |childdoc.def|
and the sample |cdocsamp.tex| with include files
|cdocsch1.tex|, |cdocsch2.tex|, |cdocspt3.tex|, |cdocspt4.tex|,
|cdocsdrf.tex|, |cdocsfn1.tex|, |cdocsfn2.tex|.
Then copy the file |childdoc.def| to an appropriate directory of your \LaTeX{}
distribution, e.g.\ \textit{texmf-root}|/tex/latex/childdoc|.
\end{itemize}

%%%%%%%%%%%%%%%%%%%%%%%%%%%%%%%%%%%%%%%%%%%%%%%%%%%%%%%%%%%%%%%%%%%%%%%%%%%%%%%%
\subsection{Related CTAN Packages}

There are several other packages which offer a similar functionality:
%
\begin{itemize}
\item
The packages
\href{http://ctan.org/pkg/docmute}{\textsf{docmute}},
\href{http://ctan.org/pkg/includex}{\textsf{includex}} and
\href{http://ctan.org/pkg/standalone}{\textsf{standalone}}
provide commands to include only the document body of
a child file thus allowing both files to be compiled individually.
\item
The packages \href{http://ctan.org/pkg/subdocs}{\textsf{subdocs}}
and \href{http://ctan.org/pkg/subfiles}{\textsf{subfiles}}
provide structures in which the main and child documents can be
encapsulated and allowing them to be compiled individually.
The inclusion mechanism is different from the conventional |\include|.
\item
The package \href{http://ctan.org/pkg/combine}{\textsf{combine}}
is an elaborate solution to combine several documents into one.
\end{itemize}
%
See also the CTAN topic \href{http://ctan.org/topic/subdocs}{\textsf{subdocs}}
for further related packages.
The present package differs from the above solutions in that
a document structure constructed with the conventional |\include| mechanism
just needs two extra commands at the top of every file
such that all constituent files can be compiled individually.

%%%%%%%%%%%%%%%%%%%%%%%%%%%%%%%%%%%%%%%%%%%%%%%%%%%%%%%%%%%%%%%%%%%%%%%%%%%%%%%%
%\subsection{Feature Suggestions}
%
%The following is a list of features which may be useful for future
%versions of this package:
%%
%\begin{itemize}
%\item
%\ldots
%\end{itemize}

%%%%%%%%%%%%%%%%%%%%%%%%%%%%%%%%%%%%%%%%%%%%%%%%%%%%%%%%%%%%%%%%%%%%%%%%%%%%%%%%
\subsection{Revision History}

%%%%%%%%%%%%%%%%%%%%%%%%%%%%%%%%%%%%%%%%
\paragraph{v2.0:} 2018/12/30

\begin{itemize}
\item
immediate forward processing
\item
added |\childdocby| mechanism
\item
manual restructured
\end{itemize}

%%%%%%%%%%%%%%%%%%%%%%%%%%%%%%%%%%%%%%%%
\paragraph{v1.6:} 2018/01/17

\begin{itemize}
\item
application for development of include files
\item
corrections to manual
\end{itemize}

%%%%%%%%%%%%%%%%%%%%%%%%%%%%%%%%%%%%%%%%
\paragraph{v1.5:} 2017/05/21

\begin{itemize}
\item
more complete structuring introduced
\item
|\childdocof| introduced
\item
|\childdoc| renamed to |\childdocmain|
\item
|\childredirect| renamed to |\childdocforward| and |\childdocforwardprefix|
and functionality expanded
\end{itemize}

%%%%%%%%%%%%%%%%%%%%%%%%%%%%%%%%%%%%%%%%
\paragraph{v1.0:} 2017/04/27

\begin{itemize}
\item
manual and install package
\item
first version published on CTAN
\end{itemize}

%%%%%%%%%%%%%%%%%%%%%%%%%%%%%%%%%%%%%%%%
\paragraph{v0.6:} 2017/04/26

\begin{itemize}
\item
redirection mechanism added
\end{itemize}

%%%%%%%%%%%%%%%%%%%%%%%%%%%%%%%%%%%%%%%%
\paragraph{v0.5:} 2017/04/26

\begin{itemize}
\item
functionality in definition file
\end{itemize}


%%%%%%%%%%%%%%%%%%%%%%%%%%%%%%%%%%%%%%%%%%%%%%%%%%%%%%%%%%%%%%%%%%%%%%%%%%%%%%%%
%%%%%%%%%%%%%%%%%%%%%%%%%%%%%%%%%%%%%%%%%%%%%%%%%%%%%%%%%%%%%%%%%%%%%%%%%%%%%%%%
%%%%%%%%%%%%%%%%%%%%%%%%%%%%%%%%%%%%%%%%%%%%%%%%%%%%%%%%%%%%%%%%%%%%%%%%%%%%%%%%
\appendix

\settowidth\MacroIndent{\rmfamily\scriptsize 000\ }

 \DocInput{childdoc.dtx}

\end{document}
%</driver>
% \fi
%
% %%%%%%%%%%%%%%%%%%%%%%%%%%%%%%%%%%%%%%%%%%%%%%%%%%%%%%%%%%%%%%%%%%%%%%%%%%%%%%
% %%%%%%%%%%%%%%%%%%%%%%%%%%%%%%%%%%%%%%%%%%%%%%%%%%%%%%%%%%%%%%%%%%%%%%%%%%%%%%
% \section{Sample}
%\iffalse
%<*samplemain>
%\fi
%
% The following presents a sample document
% with two chapters, two parts, a title page,
% a compile flag as well as three forwarding files to set the flag.
% It consists of eight |.tex| files:
% \begin{center}
% \begin{tabular}{ll}
% |cdocsamp.tex|&main file\\
% |cdocsch1.tex|&include file for chapter 1\\
% |cdocsch2.tex|&include file for chapter 2\\
% |cdocspt3.tex|&include file for part 3\\
% |cdocspt4.tex|&include file for part 4\\
% |cdocsdrf.tex|&forwarding file for main file in draft mode\\
% |cdocsfi1.tex|&forwarding file for final version of chapter 1\\
% |cdocsfi2.tex|&forwarding file for final version of chapter 2\\
% \end{tabular}
% \end{center}
% Each of the eight files can be compiled directly by the \LaTeX{} compiler.
%
% %%%%%%%%%%%%%%%%%%%%%%%%%%%%%%%%%%%%%%
% \paragraph{Main File.}
%
% The main file is called |cdocsamp.tex|.
%
% Load the \textsf{childdoc} definitions and
% declare the filename for the main document:
%    \begin{macrocode}
\input{childdoc.def}
\childdocmain{}
%    \end{macrocode}

% Optional override for |\version| flag:
%    \begin{macrocode}
%%\ifchilddoc\else\providecommand{\version}{draft}\fi
%    \end{macrocode}

% Define the default values for the |\version| flag
% (|final| for the main file and |draft| for childs):
%    \begin{macrocode}
\ifchilddoc
\providecommand{\version}{draft}
\else
\providecommand{\version}{final}
\fi
%    \end{macrocode}

% Load the standard document class:
%    \begin{macrocode}
\documentclass[12pt]{article}
%    \end{macrocode}

% Start the document body:
%    \begin{macrocode}
\begin{document}
%    \end{macrocode}

% Declare a title page.
% Print title, part of document being processed and version flag:
%    \begin{macrocode}
\addtocounter{page}{-1}
\begin{center}
{\LARGE\bfseries{}childdoc example\par}
\vspace{1cm}
\ifchilddoc
\ifchilddocmanual part\else chapter\fi:
`\childdocname' of `\childdocjob'\par
\else
main document: `\childdocjob'\par
\fi
version: \version\par
\end{center}
\newpage
%    \end{macrocode}

% Manually include selected file,
% otherwise process as usual:
%    \begin{macrocode}
\ifchilddocmanual
\section*{part `\childdocname'}
\input{\childdocname}
\else
%    \end{macrocode}

% Include the two chapters:
%    \begin{macrocode}
\include{cdocsch1}
\include{cdocsch2}
%    \end{macrocode}

% Include the two parts unless only chapters should be displayed:
%    \begin{macrocode}
\ifchilddoc\else
\section{part three}
\input{cdocspt3}
\section{part four}
\input{cdocspt4}
\fi
%    \end{macrocode}

% Process as usual until here:
%    \begin{macrocode}
\fi
%    \end{macrocode}

% End of document body:
%    \begin{macrocode}
\end{document}
%    \end{macrocode}
%\iffalse
%</samplemain>
%\fi
%
% %%%%%%%%%%%%%%%%%%%%%%%%%%%%%%%%%%%%%%
% \paragraph{Chapter Include Files.}
%
% The include files are called |cdocsch1.tex| and |cdocsch2.tex|.
%
%\iffalse
%<*samplechap1|samplechap2>
%\fi

% Optional override for |\version| flag:
%    \begin{macrocode}
%%\providecommand{\version}{final}
%    \end{macrocode}

% Include the main document:
%    \begin{macrocode}
\input{childdoc.def}
\childdocof{cdocsamp}
%    \end{macrocode}

%\iffalse
%</samplechap1|samplechap2>
%\fi
%
%\iffalse
%<*samplechap1>
%\fi
% Some text for chapter 1:
%    \begin{macrocode}
\section{one}
some text in chapter one
%    \end{macrocode}

%\iffalse
%</samplechap1>
%\fi
% Some text for chapter 2:
%\iffalse
%<*samplechap2>
%\fi
%    \begin{macrocode}
\section{two}
more text in chapter two
%    \end{macrocode}

%\iffalse
%</samplechap2>
%\fi
%
% %%%%%%%%%%%%%%%%%%%%%%%%%%%%%%%%%%%%%%
% \paragraph{Part Include Files.}
%
% The include files are called |cdocspt3.tex| and |cdocspt4.tex|.
%
%\iffalse
%<*samplepart3|samplepart4>
%\fi

% Optional override for |\version| flag:
%    \begin{macrocode}
%%\providecommand{\version}{final}
%    \end{macrocode}

% Include the main document:
%    \begin{macrocode}
\input{childdoc.def}
\childdocby{cdocsamp}
%    \end{macrocode}

%\iffalse
%</samplepart3|samplepart4>
%\fi
%
%\iffalse
%<*samplepart3>
%\fi
% Some text for part 3:
%    \begin{macrocode}
some text in part three
%    \end{macrocode}

%\iffalse
%</samplepart3>
%\fi
% Some text for part 4:
%\iffalse
%<*samplepart4>
%\fi
%    \begin{macrocode}
more text in part four
%    \end{macrocode}

%\iffalse
%</samplepart4>
%\fi
%
% %%%%%%%%%%%%%%%%%%%%%%%%%%%%%%%%%%%%%%
% \paragraph{Forwarding for a Complete Draft.}
%
% The following forwarding file |cdocsdrf.tex|
% compiles the main document in draft mode:
%\iffalse
%<*sampledraft>
%\fi
%    \begin{macrocode}
\def\version{draft}
\input{childdoc.def}
\childdocforward{cdocsamp}
%    \end{macrocode}

%\iffalse
%</sampledraft>
%\fi
%
% %%%%%%%%%%%%%%%%%%%%%%%%%%%%%%%%%%%%%%
% \paragraph{Forwarding for Final Version of the Chapters.}
%
% The following forwarding files |cdocsfn1.tex| and |cdocsfn2.tex|
% (with identical content)
% compile the final versions of the child documents
% |cdocsch1.tex| and |cdocsch2.tex|, respectively:
%\iffalse
%<*samplefinal>
%\fi
%    \begin{macrocode}
\def\version{final}
\input{childdoc.def}
\childdocforwardprefix[cdocsamp]{cdocsfn}{cdocsch}
%    \end{macrocode}

%\iffalse
%</samplefinal>
%\fi
%
% %%%%%%%%%%%%%%%%%%%%%%%%%%%%%%%%%%%%%%
% \paragraph{Command Line Processing.}
%
% The following three command lines generate the output files
% |cdocscld|, |cdocscl1| and |cdocscl2|
% which should be identical to
% |cdocsdrf|, |cdocsch1| and |cdocsfn2|, respectively:
% \begin{center}
% \begin{tabular}{l}
% |latex -jobname cdocscld \|\\
% |  "\def\version{draft}\input{childdoc.def}\childdocforward{cdocsamp}"|\\
% |latex -jobname cdocscl1 \|\\
% |  "\input{childdoc.def}\childdocforward[cdocsamp]{cdocsch1}"|\\
% |latex -jobname cdocscl2 \|\\
% |  "\def\version{final}\input{childdoc.def}\childdocforward{cdocsch2}"|
% \end{tabular}
% \end{center}
% Note that the trailing backslash on each first line
% merely continues the input to the second line
% (for convenient cut ant paste).
% Furthermore, the command |latex| can be replaced by any
% of its alternative versions such as |pdflatex|.
%
% %%%%%%%%%%%%%%%%%%%%%%%%%%%%%%%%%%%%%%%%%%%%%%%%%%%%%%%%%%%%%%%%%%%%%%%%%%%%%%
% %%%%%%%%%%%%%%%%%%%%%%%%%%%%%%%%%%%%%%%%%%%%%%%%%%%%%%%%%%%%%%%%%%%%%%%%%%%%%%
% \section{Implementation}
%\iffalse
%<*package>
%\fi
%
% This section describes the definitions file |childdoc.def|.

% The definitions cannot be loaded using |\usepackage| or |\RequirePackage|
% which has a mechanism to prevent loading a style file more than once.
% When loading the definitions by means of |\input|
% multiple instances have to be prevented manually:
%\iffalse
%This code needs to be before the `\ProvidesFile' directive
%which is defined at the beginning of this file.
%Therefore it is also placed there and commented out here.
%</package>
%<*discard>
%\fi
%    \begin{macrocode}
\ifdefined\childdocmain\endinput\fi
%    \end{macrocode}
%\iffalse
%</discard>
%<*package>
%\fi
%
% \macro{\ifchilddoc}
% \macro{\ifchilddocmanual}
% The conditional |\ifchilddoc| tells whether a
% child (true) or main (false) document is being compiled.
% The conditional |\ifchilddocmanual| tells whether
% the |\includeonly| mechanism is used (false) or
% the selection of child files must be performed manually (true).
% The definitions initialise to false:
%    \begin{macrocode}
\newif\ifchilddoc
\newif\ifchilddocmanual
%    \end{macrocode}

% \macro{\childdocname}
% \macro{\childdocjob}
% The macro |\childdocname| stores the name of the main document
% to be compiled. The macro |\childdocjob| stores the name of
% the document on which the \LaTeX{} compiler was originally invoked.
% The content of |\jobname| cannot be compared
% to filenames specified in the source due to different catcodes.
% The following code rescans |\jobname|, stores the result
% in |\childdocname| and saves a copy in |\childdocjob|:
%    \begin{macrocode}
\edef\childdocname{\scantokens\expandafter{\jobname\noexpand}}
\let\childdocjob\childdocname
%    \end{macrocode}

% \macro{\childdocdisable}
% The macro |\childdocdisable| prevents the main file
% from being processed more than once.
% At this stage, the main document command |\childdocmain|
% is assumed to be called once again where it should do nothing.
% Any subsequent call to it should prevent
% a secondary processing of the main document
% It overwrites the forwarding commands
% |\childdocof| and |\childdocforward|
% with empty macros to prevent further inclusions of the main document:
%    \begin{macrocode}
\newcommand{\childdocdisable}
{
  \renewcommand{\childdocmain}[1]{\renewcommand{\childdocmain}[1]{\endinput}}
  \renewcommand{\childdocof}[1]{}
  \renewcommand{\childdocby}[2][]{}
  \renewcommand{\childdocforward}[2][]{}
  \renewcommand{\childdocdisable}{}
}
%    \end{macrocode}

% \macro{\childdocmain}
% The macro |\childdocmain| is to be called at the top of the main file
% with nothing or the main filename (without extension) as argument.
% First, it breaks loops.
% If the argument is not empty and does not match |\childdocname|
% (which is set by the first inclusion of |childdoc.def|),
% |\ifchilddoc| is set to true, |\includeonly| is applied to the child file
% and |\jobname| is set to the main file
% (for proper handling of |.aux| files):
%    \begin{macrocode}
\newcommand{\childdocmain}[1]
{
  \childdocdisable\childdocmain{}
  \if?#1?\else
    \begingroup
      \def\childdoctmp{#1}
      \ifx\childdoctmp\childdocname
        \def\childdoctmp{}
      \else
        \def\childdoctmp
        {
          \childdoctrue
          \includeonly{\childdocname}
          \def\childdocjob{#1}
          \def\jobname{#1}
        }
      \fi
      \expandafter
    \endgroup
    \childdoctmp
  \fi
}
%    \end{macrocode}

% \macro{\childdocof}
% The command |\childdocof| redirects
% compilation to the main file |#1|.
%    \begin{macrocode}
\newcommand{\childdocof}[1]
{
  \childdocdisable
  \childdoctrue
  \includeonly{\childdocname}
  \def\jobname{#1}
  \def\childdocjob{#1}
  \input{#1}
}
%    \end{macrocode}

% \macro{\childdocby}
% The command |\childdocby| ....
%    \begin{macrocode}
\newcommand{\childdocby}[2][]
{
  \childdocdisable
  \childdoctrue
  \childdocmanualtrue
  \if?#1?\else
    \def\jobname{#2}
  \fi
  \def\childdocjob{#2}
  \input{#2}
  \endinput
}
%    \end{macrocode}

% \macro{\childdocforward}
% The command |\childdocforward| redirects
% compilation to the main file or
% (if the optional argument is given) a child file.
% Parameters are set as if the main file
% or a child file starting with |\childdocof| was compiled.
% Then compilation is handed over to the main file:
%    \begin{macrocode}
\newcommand{\childdocforward}[2][]
{
  \begingroup
    \if?#1?
      \def\childdoctmp
      {
        \def\childdocname{#2}
        \def\childdocjob{#2}
        \def\jobname{#2}
        \input{#2}
        \endinput
      }
    \else
      \def\childdoctmp
      {
        \childdocdisable
        \def\childdocname{#2}
        \childdoctrue
        \includeonly{#2}
        \def\childdocjob{#1}
        \def\jobname{#1}
        \input{#1}
        \endinput
      }
    \fi
    \expandafter
  \endgroup
  \childdoctmp
}
%    \end{macrocode}

% \macro{\childdocforwardprefix}
% The command |\childdocforwardprefix| redirects
% compilation to the main or a child file by means of a pattern.
% The prefix |#1| in the current filename is replaced by |#2|
% and the suffix of the current filename is kept
% (it is assumed that the filename does not contain the substring `|~~~|'
% which is used as a delimiter).
% Compilation is handed over to the new file by |\childdocforward|:
%    \begin{macrocode}
\newcommand{\childdocforwardprefix}[3][]
{
  \begingroup
    \def\childdocextract #2##1~~~{\def\childdoctmp{\childdocforward[#1]{#3##1}}}
    \expandafter\childdocextract\childdocname~~~
    \expandafter
  \endgroup
  \childdoctmp
}
%    \end{macrocode}

% \macro{\childdoc}
% The deprecated macro |\childdoc| is a legacy version of |\childdocmain|:
%    \begin{macrocode}
\newcommand{\childdoc}{\childdocmain}
%    \end{macrocode}

% \macro{\childdocredirect}
% The deprecated macro |\childdocredirect| is a legacy version
% of |\childdocforward| and |\childdocforwardprefix|:
%    \begin{macrocode}
\newcommand{\childdocredirect}[2][]
{
  \begingroup
    \if?#1?
      \def\childdoctmp{\childdocforward{#2}}
    \else
      \def\childdoctmp{\childdocforwardprefix{#1}{#2}}
    \fi
    \expandafter
  \endgroup
  \childdoctmp
}
%    \end{macrocode}

%\iffalse
%</package>
%\fi
%
\endinput
|\\
|\childdocforward{|\textit{main}|}|
\end{tabular}
\end{center}
%
Likewise, the following files |final|\textit{nn}|.tex|
compile the final version of the child document
|child|\textit{nn}|.tex|:
%
\begin{center}
\begin{tabular}{l}
|\def\version{final}|\\
|% \iffalse
%
% childdoc.dtx Copyright (C) 2017-2018 Niklas Beisert
%
% This work may be distributed and/or modified under the
% conditions of the LaTeX Project Public License, either version 1.3
% of this license or (at your option) any later version.
% The latest version of this license is in
%   http://www.latex-project.org/lppl.txt
% and version 1.3 or later is part of all distributions of LaTeX
% version 2005/12/01 or later.
%
% This work has the LPPL maintenance status `maintained'.
%
% The Current Maintainer of this work is Niklas Beisert.
%
% This work consists of the files childdoc.dtx and childdoc.ins
% and the derived files childdoc.def and cdocsamp.tex with
% cdocsch1.tex, cdocsch2.tex, cdocsdrf.tex, cdocsfn1.tex, cdocsfn2.tex.
%
%<package>\ifdefined\childdocmain\endinput\fi
%<package>\ProvidesFile{childdoc.def}[2018/12/30 v2.0 child document driver]
%<samplemain>\ProvidesFile{cdocsamp.tex}[2018/12/30 v2.0 sample for childdoc]
%<*driver>
%\ProvidesFile{childdoc.drv}[2018/12/30 v2.0 childdoc reference manual file]
\PassOptionsToClass{10pt,a4paper}{article}
\documentclass{ltxdoc}

\usepackage[margin=35mm]{geometry}
\usepackage{hyperref}
\usepackage{hyperxmp}
\usepackage[usenames]{color}

\hypersetup{colorlinks=true}
\hypersetup{pdfstartview=FitH}
\hypersetup{pdfpagemode=UseNone}
\hypersetup{pdfsource={}}
\hypersetup{pdflang={en-UK}}
\hypersetup{pdfcopyright={Copyright 2017-2018 Niklas Beisert.
  This work may be distributed and/or modified under the
  conditions of the LaTeX Project Public License, either version 1.3
  of this license or (at your option) any later version.}}
\hypersetup{pdflicenseurl={http://www.latex-project.org/lppl.txt}}
\hypersetup{pdfcontactaddress={ETH Zurich, ITP, HIT K,
  Wolfgang-Pauli-Strasse 27}}
\hypersetup{pdfcontactpostcode={8093}}
\hypersetup{pdfcontactcity={Zurich}}
\hypersetup{pdfcontactcountry={Switzerland}}
\hypersetup{pdfcontactemail={nbeisert@itp.phys.ethz.ch}}
\hypersetup{pdfcontacturl={http://people.phys.ethz.ch/\xmptilde nbeisert/}}

\newcommand{\secref}[1]{\hyperref[#1]{section \ref*{#1}}}

\parskip1ex
\parindent0pt
\let\olditemize\itemize
\def\itemize{\olditemize\parskip0pt}

\begin{document}

\title{The \textsf{childdoc} Package}
\hypersetup{pdftitle={The childdoc Package}}
\author{Niklas Beisert\\[2ex]
  Institut f\"ur Theoretische Physik\\
  Eidgen\"ossische Technische Hochschule Z\"urich\\
  Wolfgang-Pauli-Strasse 27, 8093 Z\"urich, Switzerland\\[1ex]
  \href{mailto:nbeisert@itp.phys.ethz.ch}
  {\texttt{nbeisert@itp.phys.ethz.ch}}}
\hypersetup{pdfauthor={Niklas Beisert}}
\hypersetup{pdfsubject={Manual for the LaTeX2e Package childdoc}}
\date{30 December 2018, \textsf{v2.0}}
\maketitle

\begin{abstract}\noindent
\textsf{childdoc} is a \LaTeXe{} package
that enables the direct compilation
of document sections included by |\include|
to individual files.
\end{abstract}

\begingroup
\parskip0ex
\tableofcontents
\endgroup

%%%%%%%%%%%%%%%%%%%%%%%%%%%%%%%%%%%%%%%%%%%%%%%%%%%%%%%%%%%%%%%%%%%%%%%%%%%%%%%%
%%%%%%%%%%%%%%%%%%%%%%%%%%%%%%%%%%%%%%%%%%%%%%%%%%%%%%%%%%%%%%%%%%%%%%%%%%%%%%%%
\section{Introduction}

\LaTeX{} provides a mechanism to structure a large document (such as a book)
into a main file and several child files (containing the chapters)
using the |\include| command.
This mechanism is beneficial for documents
which span hundreds of pages in order to
make the source file(s) more manageable.
Moreover, compilation can be restricted to
selected child files by means of the |\includeonly| command.
The latter feature can be used to reduce the compilation time while editing
(this was significantly more useful in the earlier days of \LaTeX{})
or to generate a smaller document which is easier to navigate.
Another application of |\includeonly| is to generate
documents consisting of selected parts of the complete document.

However, there are a few drawbacks of the plain |\include| mechanism:
\begin{itemize}
\item
The child files cannot be compiled on their own,
they can only be compiled via the main file.
A naive editing environment
(such as a text editor with an option
to have the current file processed by \LaTeX)
may require one to switch to the main file before compiling;
attempting to compile the child file produces errors.
\item
The main file must be modified (each time)
to adjust the |\includeonly| command
to the present needs. This easily leaves the main file in a messy state.
\item
The generated document will always carry the filename
of the main document. This is inconvenient if
several child files are to be compiled and
to be kept for distribution.
\end{itemize}

The present package provides a simple interface
to make child files individually compilable by \LaTeX{}.
Compiling a child file then has the same effect as compiling
the main file with an |\includeonly| command
to select the appropriate child.
Moreover the generated document will carry the name of the child
rather than the main file.
This resolves all three above issues.

This feature is meant to make the editing of books,
thesis documents and lecture notes somewhat more convenient.
However, the package can also be used efficiently for
composing a series of documents (such as exercise sheets)
which are typically distributed individually.
It then assists the author in generating the individual documents
(potentially in different versions)
as well as a document containing the collected series.
Another application is in developing style files
or other kinds of included material
where compilation of the style file could redirect
to a sample or test file.

%%%%%%%%%%%%%%%%%%%%%%%%%%%%%%%%%%%%%%%%%%%%%%%%%%%%%%%%%%%%%%%%%%%%%%%%%%%%%%%%
%%%%%%%%%%%%%%%%%%%%%%%%%%%%%%%%%%%%%%%%%%%%%%%%%%%%%%%%%%%%%%%%%%%%%%%%%%%%%%%%
\section{Usage}

First of all, the package \textsf{childdoc} is \emph{not} a standard
\LaTeXe{} |.sty| style file! Therefore it needs to be invoked in
a non-standard way.

%%%%%%%%%%%%%%%%%%%%%%%%%%%%%%%%%%%%%%%%%%%%%%%%%%%%%%%%%%%%%%%%%%%%%%%%%%%%%%%%
\subsection{Included Files}
\label{sec:include}

%%%%%%%%%%%%%%%%%%%%%%%%%%%%%%%%%%%%%%%%
\DescribeMacro{\childdocmain}
To use the package, add the commands
\begin{center}
\begin{tabular}{l}
|\input{childdoc.def}|\\
|\childdocmain{}|\\
\end{tabular}
\end{center}
at the very top of the main \LaTeX{} file,
in particular \emph{before} the |\documentclass| statement!
The argument of |\childdocmain| should be left empty
(but it must be present).

%%%%%%%%%%%%%%%%%%%%%%%%%%%%%%%%%%%%%%%%
\DescribeMacro{\childdocof}
Furthermore, add the commands
\begin{center}
\begin{tabular}{l}
|\input{childdoc.def}|\\
|\childdocof{|\textit{main}|}|\\
\end{tabular}
\end{center}
at the top of every child file \textit{child}
which is included by |\include{|\textit{child}|}|
from within the main file
(or at least for those files to be compiled individually).
The argument \textit{main} must be the filename of the main file.

There are a couple of
considerations in setting up the main and child documents:

%%%%%%%%%%%%%%%%%%%%%%%%%%%%%%%%%%%%%%%%
\paragraph{Restrictions.}

Please note the following restrictions:
\begin{itemize}
\item
|\childdocmain| must be called with one argument \textit{main}
to ensure compatibility with earlier version of the package.
It must either be empty (|\childdocmain{}|)
or precisely match the filename of the main file in which it is specified.
See \secref{sec:detection} for further information.
\item
The filename \textit{main} must be specified without the |.tex| extension.
\item
The filename \textit{main} is case sensitive
(even in case-insensitive file systems)
due to internal string comparison.
\item
The argument \textit{main} should be fully expanded, it cannot be a macro.
\item
Subdirectories and special characters should be avoided in filenames.
\item
The command |\childdocmain{|\textit{main}|}| must be followed by a whitespace.
It should not be followed immediately by another command
or by a comment mark `|%|'.
This is because the \TeX{} parser reads the token immediately following
the argument of |\childdocmain| and puts it
at the beginning of every child section;
however, a white\-space is ignored.
\end{itemize}

%%%%%%%%%%%%%%%%%%%%%%%%%%%%%%%%%%%%%%%%
\paragraph{Content of Main File.}

It is advisable to place all content in the child files included by |\include|.
Any output contained in the main file will appear in all child documents
unless suppressed manually;
it cannot be suppressed automatically by the |\includeonly| directive
and thus should normally be avoided.
A method to include some content in the main file
by means of conditional processing is described in \secref{sec:conditional}.

%%%%%%%%%%%%%%%%%%%%%%%%%%%%%%%%%%%%%%%%
\paragraph{Page Numbering.}

When only a part of the document is compiled,
the appropriate numbering of pages
(as well as other status parameters)
is determined from the |.aux| files.
The latter contain information from previous passes.
However this information needs to propagate through
all intermediate child documents.
Therefore the page numbering in child documents may well
be inconsistent until the complete document is compiled at least once.

A useful (if unconventional) way to always ensure a consistent
page numbering is to restart the numbering in each child document
and denote the pages by `\textit{child}|.|\textit{page}'
where \textit{child} represents the chapter/section number of the child file.
This can be achieved by the command
|\numberwithin{page}{|\textit{child}|}|
of the \textsf{amsmath} package
where \textit{child} can be |chapter| or |section|
depending on the chosen structuring.
Alternatively, one can modify the macro |\thepage| appropriately
and reset the counter |page| at the start of each child file.

%%%%%%%%%%%%%%%%%%%%%%%%%%%%%%%%%%%%%%%%%%%%%%%%%%%%%%%%%%%%%%%%%%%%%%%%%%%%%%%%
\subsection{Conditional Processing}
\label{sec:conditional}

The package provides a mechanism to compile different versions
of a document. To customise the versions further some conditional processing
can come in handy to distinguish which version is being compiled.
The package provides two macros to describe the compilation context:

%%%%%%%%%%%%%%%%%%%%%%%%%%%%%%%%%%%%%%%%
\DescribeMacro{\ifchilddoc}
The conditional |\ifchilddoc| distinguishes between the compilation of
child documents and the main document:
%
\begin{center}
|\ifchilddoc |\textit{child-code}| |[|\||else |\textit{main-code}]| \||fi|
\end{center}

%%%%%%%%%%%%%%%%%%%%%%%%%%%%%%%%%%%%%%%%
\DescribeMacro{\childdocname}
\DescribeMacro{\childdocjob}
The macro |\childdocname| contains the filename (without extension)
of the main or child file being processed.
Note that |\childdocjob| will always contain the name of the main file.

%%%%%%%%%%%%%%%%%%%%%%%%%%%%%%%%%%%%%%%%
\paragraph{Title Page.}

Conditional processing can be used to include a title or banner page
in the main document when proper precautions are taken.
Importantly, the code in the main file should ensure that the page counter
(as well as other status parameters which are stored in the |.aux| files)
takes the same value after the conditional processing.
Otherwise the page numbers may take divergent values
depending on which part is compiled.

For example, a title page could be declared by:
%
\begin{center}
\begin{tabular}{l}
|\ifchilddoc\||else|\\
|\addtocounter{page}{-1}|\\
\textit{code for title page}\\
|\newpage|\\
|\||fi|
\end{tabular}
\end{center}
%
A banner page for the child documents can be generated by:
%
\begin{center}
\begin{tabular}{l}
|\ifchilddoc|\\
|\addtocounter{page}{-1}|\\
\textit{code for banner page}\\
|\newpage|\\
|\||fi|
\end{tabular}
\end{center}
%
Here one could write a message such as:
\begin{center}
|This is the part \childdocname{} of \childdocjob{}.|
\end{center}

%%%%%%%%%%%%%%%%%%%%%%%%%%%%%%%%%%%%%%%%%%%%%%%%%%%%%%%%%%%%%%%%%%%%%%%%%%%%%%%%
\subsection{Flags}
\label{sec:flags}

The package makes it easy to generate different versions
of the main or child documents.
To this end compilation flags can be defined
and assigned different default values.
They will be particularly useful in conjunction
with the forwarding mechanism described in \secref{sec:forward}.

For example, it may be useful to have a flag |\version|
which can be set to |draft| or |final|.
The document source will contain some conditional code
depending on the value of |\version|.
Suppose further, the flag should default to |final| for the main file
and to |draft| for child files
which is a natural assignment for editing the document.
This is achieved by placing the following code
in the preamble of the main document
(below the |\childdocmain| directive):
%
\begin{center}
\begin{tabular}{l}
|\ifchilddoc|\\
|\providecommand{\version}{draft}|\\
|\||else|\\
|\providecommand{\version}{final}|\\
|\||fi|
\end{tabular}
\end{center}
%
The definition by |\providecommand| makes sure
that previous definitions are not overwritten.
Further statements |\providecommand{\version}{...}|
can thus be added before the above code to override it.

For the main file, one might add a line
(between |\childdocmain| and the above block)
%
\begin{center}
|%\ifchilddoc\||else\providecommand{\version}{draft}\||fi|
\end{center}
%
which can be uncommented to produce a draft version.
Likewise one can add a line to the very top of a child file
(above the |\childdocof{|\textit{main}|}| directive)
%
\begin{center}
|%\providecommand{\version}{final}|
\end{center}
%
which can be uncommented to produce the final version of this child document.

%%%%%%%%%%%%%%%%%%%%%%%%%%%%%%%%%%%%%%%%%%%%%%%%%%%%%%%%%%%%%%%%%%%%%%%%%%%%%%%%
\subsection{Forwarding}
\label{sec:forward}

Different versions of the main or child documents
using compilation flags as described in \secref{sec:flags}
can be (permanently) stored in different files
for convenient compilation, viewing and distribution.
To this end, the package defines a command
to pass on compilation to a different file:

%%%%%%%%%%%%%%%%%%%%%%%%%%%%%%%%%%%%%%%%
\DescribeMacro{\childdocforward}
The command |\childdocforward| redirects processing to
another source file:
%
\begin{center}
\begin{tabular}{l}
|\input{childdoc.def}|\\
|\childdocforward[|\textit{main}|]{|\textit{dest}|}|\\
\end{tabular}
\end{center}
%
The argument \textit{dest} is the destination file
(without extension).
It should be the main file or one of the child files.
Note that further \textsf{childdoc} directives
such as |\childdocof| and |\childdocforward|
in the indicated file will be processed in this form.
The optional argument \textit{main}
passes on directly to the main file \textit{main}
while pretending to compile the child \textit{dest}.
This form behaves as if \textit{dest}
issues |\childdocof{|\textit{main}|}| right away,
and no further \textsf{childdoc} directives will be processed.

%%%%%%%%%%%%%%%%%%%%%%%%%%%%%%%%%%%%%%%%
\DescribeMacro{\...prefix}
In the alternative form |\childdocforwardprefix|,
%
\begin{center}
\begin{tabular}{l}
|\input{childdoc.def}|\\
|\childdocforwardprefix[|\textit{main}|]{|\textit{prefix}|}{|\textit{dest}|}|
\end{tabular}
\end{center}
%
the destination file is determined by a pattern
depending on the current file:
To make this work, the current file must be called
`{\textit{prefix}\hspace{0.2em}\textit{suffix}}'
with \textit{prefix} matching precisely the argument.
Processing is then passed on to the file
`{\textit{dest}\hspace{0.2em}\textit{suffix}}'.
Surely, the same effect is achieved by
directly specifying the
argument `{\textit{dest}\hspace{0.2em}\textit{suffix}}'
in the first form.
However, that requires to set up a different file
for each child. With the alternative form of the command
all these files can have exactly the same content
which simplifies setting them up and maintaining them.

For example, the following file |draft.tex|
with a compilation flag |\version| as described in \secref{sec:flags}
compiles the main document as a draft:
%
\begin{center}
\begin{tabular}{l}
|\def\version{draft}|\\
|\input{childdoc.def}|\\
|\childdocforward{|\textit{main}|}|
\end{tabular}
\end{center}
%
Likewise, the following files |final|\textit{nn}|.tex|
compile the final version of the child document
|child|\textit{nn}|.tex|:
%
\begin{center}
\begin{tabular}{l}
|\def\version{final}|\\
|\input{childdoc.def}|\\
|\childdocforwardprefix{final}{child}|
\end{tabular}
\end{center}
%

Note that when several versions of a main file and/or of each child file
are to be generated, it may be convenient to set up a |Makefile| or
shell script to automatise the process.

%%%%%%%%%%%%%%%%%%%%%%%%%%%%%%%%%%%%%%%%%%%%%%%%%%%%%%%%%%%%%%%%%%%%%%%%%%%%%%%%
\subsection{Command Line Processing}
\label{sec:commandline}

The effect of redirection files can also be achieved by invoking
the \LaTeX{} compiler with a more elaborate command line.
Most conveniently this should be done as part
of a shell script or a |Makefile|.

When using \textsf{childdoc} in the main file, the following
command lines effectively perform a redirection
(note that depending on the shell being used,
backslashes may have to be doubled: `|\|' $\to$ `|\\|'):
%
\begin{center}
|... -jobname "|\textit{target}|" |\\|"|[\textit{flags}]%
|\input{childdoc.def}\childdocforward[|\textit{main}|]{|\textit{dest}|}"|
\end{center}
%
Here \textit{target} is the name of the output file,
\textit{main} is the name of the main file
and \textit{dest} is the name of the main or child file to be processed
(all filenames without extensions).
The optional argument \textit{main} can be omitted
if \textit{main} matches \textit{dest}.
Optionally, compilation \textit{flags} can be defined via |\def| commands.
This command line makes the \TeX{} engine believe
it is compiling the file \textit{target}
whose content is specified as the latter parameter.
The provided code then forwards the processing to
\textit{main} or \textit{dest} as described in \secref{sec:forward}.

%%%%%%%%%%%%%%%%%%%%%%%%%%%%%%%%%%%%%%%%%%%%%%%%%%%%%%%%%%%%%%%%%%%%%%%%%%%%%%%%
\subsection{Include by Input}
\label{sec:input}

Including child documents by |\include| has some restrictions by design.
Most notably, the content of a child document always occupies
its own set of pages; pages cannot be shared between child documents.
Usually, this behaviour makes perfect sense
because each child document contain an essential part of the document.
However, in some situations it may be desirable to compose
a document from a collection of parts
without having mandatory page breaks between then.
For this case, the package
provides a mechanism to include parts
by |\input| which can also be processed individually.
However, by construction this mechanism
requires manual handling of the content to be output.

%%%%%%%%%%%%%%%%%%%%%%%%%%%%%%%%%%%%%%%%
\DescribeMacro{\ifchilddocmanual}
The main file should be prepared as usual, see \secref{sec:include}.
However, the document body must make a distinction
between processing of an individual part and of the main document, e.g.:
%
\begin{center}
\begin{tabular}{l}
|\ifchilddocmanual|\\
|\input{\childdocname}|\\
|\||else|\\
\textit{document body with }|\input{|\textit{part}|}|\\
|\||fi|
\end{tabular}
\end{center}
%
The conditional |\ifchilddocmanual| is true whenever
a part to be included by |\input| is being compiled,
and the name of the part is stored in |\childdocname|.

%%%%%%%%%%%%%%%%%%%%%%%%%%%%%%%%%%%%%%%%
\DescribeMacro{\childdocby}
Each part to be included by |\input| should start with:
%
\begin{center}
\begin{tabular}{l}
|\input{childdoc.def}|\\
|\childdocby{|\textit{main}|}|\\
\end{tabular}
\end{center}
%
The directive |\childdocby| is similar to |\childdocof|
described in \secref{sec:include},
but the subsequent selection of content must be done manually.
To that end, both |\ifchilddoc| and |\ifchilddocmanual|
will be true upon processing of a part,
and the name of the part is stored in |\childdocname|.
Note that |\jobname| will be set to the filename of the current part
so that each part receives an individual |.aux| file
that does not interfere with the |.aux| file(s) of the main document.
This behaviour can be altered by the alternative form
|\childdocby[*]{|\textit{main}|}| (with a non-empty optional argument)
which uses the |.aux| file of the main document
by setting |\jobname| to \textit{main}.

%%%%%%%%%%%%%%%%%%%%%%%%%%%%%%%%%%%%%%%%%%%%%%%%%%%%%%%%%%%%%%%%%%%%%%%%%%%%%%%%
\subsection{Driver Development}
\label{sec:driver}

The \textsf{childdoc} mechanism can also be use for the development
of definition files such as \LaTeX{} styles or classes.
This case differs from the above setup with multiple parts
included by |\include| in that no |\includeonly| should be invoked.
This can be achieved by starting the include file
(before |\ProvidesPackage|) with:
%
\begin{center}
\begin{tabular}{l}
|\input{childdoc.def}|\\
|\childdocforward{|\textit{main}|}|\\
\end{tabular}
\end{center}
%
or alternatively with:
%
\begin{center}
\begin{tabular}{l}
|\input{childdoc.def}|\\
|\childdocby{|\textit{main}|}|\\
\end{tabular}
\end{center}
%
Both forms have slightly different effects as described above.
The main file is prepared as usual, see \secref{sec:include}.

%%%%%%%%%%%%%%%%%%%%%%%%%%%%%%%%%%%%%%%%%%%%%%%%%%%%%%%%%%%%%%%%%%%%%%%%%%%%%%%%
\subsection{Legacy Detection}
\label{sec:detection}

The directive |\childdocmain| in the main file can detect
whether the complete document or merely a child is to be compiled
even without using the directive |\childdocof|.
This method is deprecated because it is less robust
and there is no compelling reason to use it;
it is merely provided for backward compatibility
and it may be removed in future versions.

If the detection mechanism is to be used,
it is mandatory to correctly specify
the filename of the main file as the argument of |\childdocmain|:
%
\begin{center}
\begin{tabular}{l}
|\input{childdoc.def}|\\
|\childdocmain{|\textit{main}|}|\\
\end{tabular}
\end{center}
%
If |\jobname| does not match the argument \textit{main} of |\childdocmain|,
it is assumed that |\jobname| points to the child file to be compiled.
When using |\childdocmain| with the main file specified as argument,
it suffices to start a child file
with just |\input{|\textit{main}|}|
without loading of the package and using |\childdocof|.
If instead all processing is done
with the appropriate \textsf{childdoc} directives,
the argument of \textit{main} of |\childdocmain| can be empty.

An alternative version of the command line processing described
in \secref{sec:commandline} using the detection mechanism reads:
%
\begin{center}
|... -jobname "|\textit{target}|" "|[\textit{flags}]%
[|\def\jobname{|\textit{dest}|}|]|\input{|\textit{main}|}"|
\end{center}

%%%%%%%%%%%%%%%%%%%%%%%%%%%%%%%%%%%%%%%%%%%%%%%%%%%%%%%%%%%%%%%%%%%%%%%%%%%%%%%%
\subsection{Manual Code}
\label{sec:manual}

In case one cannot be certain whether the definitions file |childdoc.def|
is installed on the target \TeX{} distribution
and one prefers not to ship it,
it is conceivable to paste a few relevant commands into the sources.

To that end, drop all statements |\input{childdoc.def}|
and perform the replacements as outlined below.
Instead of |\childdocmain{|\textit{main}|}| add the following code
to the top of the main file:
%
\begin{center}
\begin{tabular}{l}
|\||ifdefined\childdocname\endinput\||fi\newif\ifchilddoc|\\
|\edef\childdocname{\scantokens\expandafter{\jobname\noexpand}}|\\
|\def\childdocmain{|\textit{main}|}\||ifx\childdocmain\childdocname\||else|\\
|\childdoctrue\includeonly{\childdocname}\let\jobname\childdocmain\||fi|\\
\end{tabular}
\end{center}
%
Instead of |\childdocof{|\textit{main}|}| just include the main file
at the top of each child file:
%
\begin{center}
|\input{|\textit{main}|}|
\end{center}
%
A simple redirection |\childdocforward{|\textit{dest}|}| is achieved by:
%
\begin{center}
|\def\jobname{|\textit{dest}|}\input{\jobname}|
\end{center}
%
The redirection with prefix
|\childdocforwardprefix[|\textit{prefix}|]{|\textit{dest}|}|
is accomplished by:
%
\begin{center}
\begin{tabular}{l}
|{\edef\jobname{\scantokens\expandafter{\jobname\noexpand}}|\\
|\def\redirectjob |\textit{prefix}|#1~~~{\gdef\jobname{|\textit{dest}|#1}}|\\
|\expandafter\redirectjob\jobname~~~}\input{\jobname}|
\end{tabular}
\end{center}

In an alternative approach,
child documents can be compiled by a specific command line
without additional code or specific definitions:
%
\begin{center}
|... -jobname "|\textit{target}|" "|[\textit{flags}]%
|\includeonly{|\textit{dest}|}\input{|\textit{main}|}"|
\end{center}
%

%%%%%%%%%%%%%%%%%%%%%%%%%%%%%%%%%%%%%%%%%%%%%%%%%%%%%%%%%%%%%%%%%%%%%%%%%%%%%%%%
%%%%%%%%%%%%%%%%%%%%%%%%%%%%%%%%%%%%%%%%%%%%%%%%%%%%%%%%%%%%%%%%%%%%%%%%%%%%%%%%
\section{Information}

%%%%%%%%%%%%%%%%%%%%%%%%%%%%%%%%%%%%%%%%%%%%%%%%%%%%%%%%%%%%%%%%%%%%%%%%%%%%%%%%
\subsection{Copyright}

Copyright \copyright{} 2017--2018 Niklas Beisert

This work may be distributed and/or modified under the
conditions of the \LaTeX{} Project Public License, either version 1.3
of this license or (at your option) any later version.
The latest version of this license is in
  \url{http://www.latex-project.org/lppl.txt}
and version 1.3 or later is part of all distributions of \LaTeX{}
version 2005/12/01 or later.

This work has the LPPL maintenance status `maintained'.

The Current Maintainer of this work is Niklas Beisert.

This work consists of the files |README.txt|, |childdoc.ins| and |childdoc.dtx|
as well as the derived files |childdoc.def|, |cdocsamp.tex|
with |cdocsch1.tex|, |cdocsch2.tex|, |cdocspt3.tex|, |cdocspt4.tex|,
|cdocsdrf.tex|, |cdocsfn1.tex|, |cdocsfn2.tex|
as well as |childdoc.pdf|.

%%%%%%%%%%%%%%%%%%%%%%%%%%%%%%%%%%%%%%%%%%%%%%%%%%%%%%%%%%%%%%%%%%%%%%%%%%%%%%%%
\subsection{Files and Installation}

The package consists of the files:
%
\begin{center}
\begin{tabular}{ll}
    |README.txt|   & readme file \\
    |childdoc.ins| & installation file \\
    |childdoc.dtx| & source file \\
    |childdoc.def| & definition file \\
    |cdocsamp.tex| & sample main file \\
    |cdocsch1.tex| & sample include file \\
    |cdocsch2.tex| & sample include file \\
    |cdocspt3.tex| & sample part file \\
    |cdocspt4.tex| & sample part file \\
    |cdocsdrf.tex| & sample redirection file \\
    |cdocsfn1.tex| & sample redirection file \\
    |cdocsfn2.tex| & sample redirection file \\
    |childdoc.pdf| & manual
\end{tabular}
\end{center}
%
The distribution consists of the files
|README.txt|, |childdoc.ins| and |childdoc.dtx|.
%
\begin{itemize}
\item
Run (pdf)\LaTeX{} on |childdoc.dtx|
to compile the manual |childdoc.pdf| (this file).
\item
Run \LaTeX{} on |childdoc.ins| to create the definitions file |childdoc.def|
and the sample |cdocsamp.tex| with include files
|cdocsch1.tex|, |cdocsch2.tex|, |cdocspt3.tex|, |cdocspt4.tex|,
|cdocsdrf.tex|, |cdocsfn1.tex|, |cdocsfn2.tex|.
Then copy the file |childdoc.def| to an appropriate directory of your \LaTeX{}
distribution, e.g.\ \textit{texmf-root}|/tex/latex/childdoc|.
\end{itemize}

%%%%%%%%%%%%%%%%%%%%%%%%%%%%%%%%%%%%%%%%%%%%%%%%%%%%%%%%%%%%%%%%%%%%%%%%%%%%%%%%
\subsection{Related CTAN Packages}

There are several other packages which offer a similar functionality:
%
\begin{itemize}
\item
The packages
\href{http://ctan.org/pkg/docmute}{\textsf{docmute}},
\href{http://ctan.org/pkg/includex}{\textsf{includex}} and
\href{http://ctan.org/pkg/standalone}{\textsf{standalone}}
provide commands to include only the document body of
a child file thus allowing both files to be compiled individually.
\item
The packages \href{http://ctan.org/pkg/subdocs}{\textsf{subdocs}}
and \href{http://ctan.org/pkg/subfiles}{\textsf{subfiles}}
provide structures in which the main and child documents can be
encapsulated and allowing them to be compiled individually.
The inclusion mechanism is different from the conventional |\include|.
\item
The package \href{http://ctan.org/pkg/combine}{\textsf{combine}}
is an elaborate solution to combine several documents into one.
\end{itemize}
%
See also the CTAN topic \href{http://ctan.org/topic/subdocs}{\textsf{subdocs}}
for further related packages.
The present package differs from the above solutions in that
a document structure constructed with the conventional |\include| mechanism
just needs two extra commands at the top of every file
such that all constituent files can be compiled individually.

%%%%%%%%%%%%%%%%%%%%%%%%%%%%%%%%%%%%%%%%%%%%%%%%%%%%%%%%%%%%%%%%%%%%%%%%%%%%%%%%
%\subsection{Feature Suggestions}
%
%The following is a list of features which may be useful for future
%versions of this package:
%%
%\begin{itemize}
%\item
%\ldots
%\end{itemize}

%%%%%%%%%%%%%%%%%%%%%%%%%%%%%%%%%%%%%%%%%%%%%%%%%%%%%%%%%%%%%%%%%%%%%%%%%%%%%%%%
\subsection{Revision History}

%%%%%%%%%%%%%%%%%%%%%%%%%%%%%%%%%%%%%%%%
\paragraph{v2.0:} 2018/12/30

\begin{itemize}
\item
immediate forward processing
\item
added |\childdocby| mechanism
\item
manual restructured
\end{itemize}

%%%%%%%%%%%%%%%%%%%%%%%%%%%%%%%%%%%%%%%%
\paragraph{v1.6:} 2018/01/17

\begin{itemize}
\item
application for development of include files
\item
corrections to manual
\end{itemize}

%%%%%%%%%%%%%%%%%%%%%%%%%%%%%%%%%%%%%%%%
\paragraph{v1.5:} 2017/05/21

\begin{itemize}
\item
more complete structuring introduced
\item
|\childdocof| introduced
\item
|\childdoc| renamed to |\childdocmain|
\item
|\childredirect| renamed to |\childdocforward| and |\childdocforwardprefix|
and functionality expanded
\end{itemize}

%%%%%%%%%%%%%%%%%%%%%%%%%%%%%%%%%%%%%%%%
\paragraph{v1.0:} 2017/04/27

\begin{itemize}
\item
manual and install package
\item
first version published on CTAN
\end{itemize}

%%%%%%%%%%%%%%%%%%%%%%%%%%%%%%%%%%%%%%%%
\paragraph{v0.6:} 2017/04/26

\begin{itemize}
\item
redirection mechanism added
\end{itemize}

%%%%%%%%%%%%%%%%%%%%%%%%%%%%%%%%%%%%%%%%
\paragraph{v0.5:} 2017/04/26

\begin{itemize}
\item
functionality in definition file
\end{itemize}


%%%%%%%%%%%%%%%%%%%%%%%%%%%%%%%%%%%%%%%%%%%%%%%%%%%%%%%%%%%%%%%%%%%%%%%%%%%%%%%%
%%%%%%%%%%%%%%%%%%%%%%%%%%%%%%%%%%%%%%%%%%%%%%%%%%%%%%%%%%%%%%%%%%%%%%%%%%%%%%%%
%%%%%%%%%%%%%%%%%%%%%%%%%%%%%%%%%%%%%%%%%%%%%%%%%%%%%%%%%%%%%%%%%%%%%%%%%%%%%%%%
\appendix

\settowidth\MacroIndent{\rmfamily\scriptsize 000\ }

 \DocInput{childdoc.dtx}

\end{document}
%</driver>
% \fi
%
% %%%%%%%%%%%%%%%%%%%%%%%%%%%%%%%%%%%%%%%%%%%%%%%%%%%%%%%%%%%%%%%%%%%%%%%%%%%%%%
% %%%%%%%%%%%%%%%%%%%%%%%%%%%%%%%%%%%%%%%%%%%%%%%%%%%%%%%%%%%%%%%%%%%%%%%%%%%%%%
% \section{Sample}
%\iffalse
%<*samplemain>
%\fi
%
% The following presents a sample document
% with two chapters, two parts, a title page,
% a compile flag as well as three forwarding files to set the flag.
% It consists of eight |.tex| files:
% \begin{center}
% \begin{tabular}{ll}
% |cdocsamp.tex|&main file\\
% |cdocsch1.tex|&include file for chapter 1\\
% |cdocsch2.tex|&include file for chapter 2\\
% |cdocspt3.tex|&include file for part 3\\
% |cdocspt4.tex|&include file for part 4\\
% |cdocsdrf.tex|&forwarding file for main file in draft mode\\
% |cdocsfi1.tex|&forwarding file for final version of chapter 1\\
% |cdocsfi2.tex|&forwarding file for final version of chapter 2\\
% \end{tabular}
% \end{center}
% Each of the eight files can be compiled directly by the \LaTeX{} compiler.
%
% %%%%%%%%%%%%%%%%%%%%%%%%%%%%%%%%%%%%%%
% \paragraph{Main File.}
%
% The main file is called |cdocsamp.tex|.
%
% Load the \textsf{childdoc} definitions and
% declare the filename for the main document:
%    \begin{macrocode}
\input{childdoc.def}
\childdocmain{}
%    \end{macrocode}

% Optional override for |\version| flag:
%    \begin{macrocode}
%%\ifchilddoc\else\providecommand{\version}{draft}\fi
%    \end{macrocode}

% Define the default values for the |\version| flag
% (|final| for the main file and |draft| for childs):
%    \begin{macrocode}
\ifchilddoc
\providecommand{\version}{draft}
\else
\providecommand{\version}{final}
\fi
%    \end{macrocode}

% Load the standard document class:
%    \begin{macrocode}
\documentclass[12pt]{article}
%    \end{macrocode}

% Start the document body:
%    \begin{macrocode}
\begin{document}
%    \end{macrocode}

% Declare a title page.
% Print title, part of document being processed and version flag:
%    \begin{macrocode}
\addtocounter{page}{-1}
\begin{center}
{\LARGE\bfseries{}childdoc example\par}
\vspace{1cm}
\ifchilddoc
\ifchilddocmanual part\else chapter\fi:
`\childdocname' of `\childdocjob'\par
\else
main document: `\childdocjob'\par
\fi
version: \version\par
\end{center}
\newpage
%    \end{macrocode}

% Manually include selected file,
% otherwise process as usual:
%    \begin{macrocode}
\ifchilddocmanual
\section*{part `\childdocname'}
\input{\childdocname}
\else
%    \end{macrocode}

% Include the two chapters:
%    \begin{macrocode}
\include{cdocsch1}
\include{cdocsch2}
%    \end{macrocode}

% Include the two parts unless only chapters should be displayed:
%    \begin{macrocode}
\ifchilddoc\else
\section{part three}
\input{cdocspt3}
\section{part four}
\input{cdocspt4}
\fi
%    \end{macrocode}

% Process as usual until here:
%    \begin{macrocode}
\fi
%    \end{macrocode}

% End of document body:
%    \begin{macrocode}
\end{document}
%    \end{macrocode}
%\iffalse
%</samplemain>
%\fi
%
% %%%%%%%%%%%%%%%%%%%%%%%%%%%%%%%%%%%%%%
% \paragraph{Chapter Include Files.}
%
% The include files are called |cdocsch1.tex| and |cdocsch2.tex|.
%
%\iffalse
%<*samplechap1|samplechap2>
%\fi

% Optional override for |\version| flag:
%    \begin{macrocode}
%%\providecommand{\version}{final}
%    \end{macrocode}

% Include the main document:
%    \begin{macrocode}
\input{childdoc.def}
\childdocof{cdocsamp}
%    \end{macrocode}

%\iffalse
%</samplechap1|samplechap2>
%\fi
%
%\iffalse
%<*samplechap1>
%\fi
% Some text for chapter 1:
%    \begin{macrocode}
\section{one}
some text in chapter one
%    \end{macrocode}

%\iffalse
%</samplechap1>
%\fi
% Some text for chapter 2:
%\iffalse
%<*samplechap2>
%\fi
%    \begin{macrocode}
\section{two}
more text in chapter two
%    \end{macrocode}

%\iffalse
%</samplechap2>
%\fi
%
% %%%%%%%%%%%%%%%%%%%%%%%%%%%%%%%%%%%%%%
% \paragraph{Part Include Files.}
%
% The include files are called |cdocspt3.tex| and |cdocspt4.tex|.
%
%\iffalse
%<*samplepart3|samplepart4>
%\fi

% Optional override for |\version| flag:
%    \begin{macrocode}
%%\providecommand{\version}{final}
%    \end{macrocode}

% Include the main document:
%    \begin{macrocode}
\input{childdoc.def}
\childdocby{cdocsamp}
%    \end{macrocode}

%\iffalse
%</samplepart3|samplepart4>
%\fi
%
%\iffalse
%<*samplepart3>
%\fi
% Some text for part 3:
%    \begin{macrocode}
some text in part three
%    \end{macrocode}

%\iffalse
%</samplepart3>
%\fi
% Some text for part 4:
%\iffalse
%<*samplepart4>
%\fi
%    \begin{macrocode}
more text in part four
%    \end{macrocode}

%\iffalse
%</samplepart4>
%\fi
%
% %%%%%%%%%%%%%%%%%%%%%%%%%%%%%%%%%%%%%%
% \paragraph{Forwarding for a Complete Draft.}
%
% The following forwarding file |cdocsdrf.tex|
% compiles the main document in draft mode:
%\iffalse
%<*sampledraft>
%\fi
%    \begin{macrocode}
\def\version{draft}
\input{childdoc.def}
\childdocforward{cdocsamp}
%    \end{macrocode}

%\iffalse
%</sampledraft>
%\fi
%
% %%%%%%%%%%%%%%%%%%%%%%%%%%%%%%%%%%%%%%
% \paragraph{Forwarding for Final Version of the Chapters.}
%
% The following forwarding files |cdocsfn1.tex| and |cdocsfn2.tex|
% (with identical content)
% compile the final versions of the child documents
% |cdocsch1.tex| and |cdocsch2.tex|, respectively:
%\iffalse
%<*samplefinal>
%\fi
%    \begin{macrocode}
\def\version{final}
\input{childdoc.def}
\childdocforwardprefix[cdocsamp]{cdocsfn}{cdocsch}
%    \end{macrocode}

%\iffalse
%</samplefinal>
%\fi
%
% %%%%%%%%%%%%%%%%%%%%%%%%%%%%%%%%%%%%%%
% \paragraph{Command Line Processing.}
%
% The following three command lines generate the output files
% |cdocscld|, |cdocscl1| and |cdocscl2|
% which should be identical to
% |cdocsdrf|, |cdocsch1| and |cdocsfn2|, respectively:
% \begin{center}
% \begin{tabular}{l}
% |latex -jobname cdocscld \|\\
% |  "\def\version{draft}\input{childdoc.def}\childdocforward{cdocsamp}"|\\
% |latex -jobname cdocscl1 \|\\
% |  "\input{childdoc.def}\childdocforward[cdocsamp]{cdocsch1}"|\\
% |latex -jobname cdocscl2 \|\\
% |  "\def\version{final}\input{childdoc.def}\childdocforward{cdocsch2}"|
% \end{tabular}
% \end{center}
% Note that the trailing backslash on each first line
% merely continues the input to the second line
% (for convenient cut ant paste).
% Furthermore, the command |latex| can be replaced by any
% of its alternative versions such as |pdflatex|.
%
% %%%%%%%%%%%%%%%%%%%%%%%%%%%%%%%%%%%%%%%%%%%%%%%%%%%%%%%%%%%%%%%%%%%%%%%%%%%%%%
% %%%%%%%%%%%%%%%%%%%%%%%%%%%%%%%%%%%%%%%%%%%%%%%%%%%%%%%%%%%%%%%%%%%%%%%%%%%%%%
% \section{Implementation}
%\iffalse
%<*package>
%\fi
%
% This section describes the definitions file |childdoc.def|.

% The definitions cannot be loaded using |\usepackage| or |\RequirePackage|
% which has a mechanism to prevent loading a style file more than once.
% When loading the definitions by means of |\input|
% multiple instances have to be prevented manually:
%\iffalse
%This code needs to be before the `\ProvidesFile' directive
%which is defined at the beginning of this file.
%Therefore it is also placed there and commented out here.
%</package>
%<*discard>
%\fi
%    \begin{macrocode}
\ifdefined\childdocmain\endinput\fi
%    \end{macrocode}
%\iffalse
%</discard>
%<*package>
%\fi
%
% \macro{\ifchilddoc}
% \macro{\ifchilddocmanual}
% The conditional |\ifchilddoc| tells whether a
% child (true) or main (false) document is being compiled.
% The conditional |\ifchilddocmanual| tells whether
% the |\includeonly| mechanism is used (false) or
% the selection of child files must be performed manually (true).
% The definitions initialise to false:
%    \begin{macrocode}
\newif\ifchilddoc
\newif\ifchilddocmanual
%    \end{macrocode}

% \macro{\childdocname}
% \macro{\childdocjob}
% The macro |\childdocname| stores the name of the main document
% to be compiled. The macro |\childdocjob| stores the name of
% the document on which the \LaTeX{} compiler was originally invoked.
% The content of |\jobname| cannot be compared
% to filenames specified in the source due to different catcodes.
% The following code rescans |\jobname|, stores the result
% in |\childdocname| and saves a copy in |\childdocjob|:
%    \begin{macrocode}
\edef\childdocname{\scantokens\expandafter{\jobname\noexpand}}
\let\childdocjob\childdocname
%    \end{macrocode}

% \macro{\childdocdisable}
% The macro |\childdocdisable| prevents the main file
% from being processed more than once.
% At this stage, the main document command |\childdocmain|
% is assumed to be called once again where it should do nothing.
% Any subsequent call to it should prevent
% a secondary processing of the main document
% It overwrites the forwarding commands
% |\childdocof| and |\childdocforward|
% with empty macros to prevent further inclusions of the main document:
%    \begin{macrocode}
\newcommand{\childdocdisable}
{
  \renewcommand{\childdocmain}[1]{\renewcommand{\childdocmain}[1]{\endinput}}
  \renewcommand{\childdocof}[1]{}
  \renewcommand{\childdocby}[2][]{}
  \renewcommand{\childdocforward}[2][]{}
  \renewcommand{\childdocdisable}{}
}
%    \end{macrocode}

% \macro{\childdocmain}
% The macro |\childdocmain| is to be called at the top of the main file
% with nothing or the main filename (without extension) as argument.
% First, it breaks loops.
% If the argument is not empty and does not match |\childdocname|
% (which is set by the first inclusion of |childdoc.def|),
% |\ifchilddoc| is set to true, |\includeonly| is applied to the child file
% and |\jobname| is set to the main file
% (for proper handling of |.aux| files):
%    \begin{macrocode}
\newcommand{\childdocmain}[1]
{
  \childdocdisable\childdocmain{}
  \if?#1?\else
    \begingroup
      \def\childdoctmp{#1}
      \ifx\childdoctmp\childdocname
        \def\childdoctmp{}
      \else
        \def\childdoctmp
        {
          \childdoctrue
          \includeonly{\childdocname}
          \def\childdocjob{#1}
          \def\jobname{#1}
        }
      \fi
      \expandafter
    \endgroup
    \childdoctmp
  \fi
}
%    \end{macrocode}

% \macro{\childdocof}
% The command |\childdocof| redirects
% compilation to the main file |#1|.
%    \begin{macrocode}
\newcommand{\childdocof}[1]
{
  \childdocdisable
  \childdoctrue
  \includeonly{\childdocname}
  \def\jobname{#1}
  \def\childdocjob{#1}
  \input{#1}
}
%    \end{macrocode}

% \macro{\childdocby}
% The command |\childdocby| ....
%    \begin{macrocode}
\newcommand{\childdocby}[2][]
{
  \childdocdisable
  \childdoctrue
  \childdocmanualtrue
  \if?#1?\else
    \def\jobname{#2}
  \fi
  \def\childdocjob{#2}
  \input{#2}
  \endinput
}
%    \end{macrocode}

% \macro{\childdocforward}
% The command |\childdocforward| redirects
% compilation to the main file or
% (if the optional argument is given) a child file.
% Parameters are set as if the main file
% or a child file starting with |\childdocof| was compiled.
% Then compilation is handed over to the main file:
%    \begin{macrocode}
\newcommand{\childdocforward}[2][]
{
  \begingroup
    \if?#1?
      \def\childdoctmp
      {
        \def\childdocname{#2}
        \def\childdocjob{#2}
        \def\jobname{#2}
        \input{#2}
        \endinput
      }
    \else
      \def\childdoctmp
      {
        \childdocdisable
        \def\childdocname{#2}
        \childdoctrue
        \includeonly{#2}
        \def\childdocjob{#1}
        \def\jobname{#1}
        \input{#1}
        \endinput
      }
    \fi
    \expandafter
  \endgroup
  \childdoctmp
}
%    \end{macrocode}

% \macro{\childdocforwardprefix}
% The command |\childdocforwardprefix| redirects
% compilation to the main or a child file by means of a pattern.
% The prefix |#1| in the current filename is replaced by |#2|
% and the suffix of the current filename is kept
% (it is assumed that the filename does not contain the substring `|~~~|'
% which is used as a delimiter).
% Compilation is handed over to the new file by |\childdocforward|:
%    \begin{macrocode}
\newcommand{\childdocforwardprefix}[3][]
{
  \begingroup
    \def\childdocextract #2##1~~~{\def\childdoctmp{\childdocforward[#1]{#3##1}}}
    \expandafter\childdocextract\childdocname~~~
    \expandafter
  \endgroup
  \childdoctmp
}
%    \end{macrocode}

% \macro{\childdoc}
% The deprecated macro |\childdoc| is a legacy version of |\childdocmain|:
%    \begin{macrocode}
\newcommand{\childdoc}{\childdocmain}
%    \end{macrocode}

% \macro{\childdocredirect}
% The deprecated macro |\childdocredirect| is a legacy version
% of |\childdocforward| and |\childdocforwardprefix|:
%    \begin{macrocode}
\newcommand{\childdocredirect}[2][]
{
  \begingroup
    \if?#1?
      \def\childdoctmp{\childdocforward{#2}}
    \else
      \def\childdoctmp{\childdocforwardprefix{#1}{#2}}
    \fi
    \expandafter
  \endgroup
  \childdoctmp
}
%    \end{macrocode}

%\iffalse
%</package>
%\fi
%
\endinput
|\\
|\childdocforwardprefix{final}{child}|
\end{tabular}
\end{center}
%

Note that when several versions of a main file and/or of each child file
are to be generated, it may be convenient to set up a |Makefile| or
shell script to automatise the process.

%%%%%%%%%%%%%%%%%%%%%%%%%%%%%%%%%%%%%%%%%%%%%%%%%%%%%%%%%%%%%%%%%%%%%%%%%%%%%%%%
\subsection{Command Line Processing}
\label{sec:commandline}

The effect of redirection files can also be achieved by invoking
the \LaTeX{} compiler with a more elaborate command line.
Most conveniently this should be done as part
of a shell script or a |Makefile|.

When using \textsf{childdoc} in the main file, the following
command lines effectively perform a redirection
(note that depending on the shell being used,
backslashes may have to be doubled: `|\|' $\to$ `|\\|'):
%
\begin{center}
|... -jobname "|\textit{target}|" |\\|"|[\textit{flags}]%
|% \iffalse
%
% childdoc.dtx Copyright (C) 2017-2018 Niklas Beisert
%
% This work may be distributed and/or modified under the
% conditions of the LaTeX Project Public License, either version 1.3
% of this license or (at your option) any later version.
% The latest version of this license is in
%   http://www.latex-project.org/lppl.txt
% and version 1.3 or later is part of all distributions of LaTeX
% version 2005/12/01 or later.
%
% This work has the LPPL maintenance status `maintained'.
%
% The Current Maintainer of this work is Niklas Beisert.
%
% This work consists of the files childdoc.dtx and childdoc.ins
% and the derived files childdoc.def and cdocsamp.tex with
% cdocsch1.tex, cdocsch2.tex, cdocsdrf.tex, cdocsfn1.tex, cdocsfn2.tex.
%
%<package>\ifdefined\childdocmain\endinput\fi
%<package>\ProvidesFile{childdoc.def}[2018/12/30 v2.0 child document driver]
%<samplemain>\ProvidesFile{cdocsamp.tex}[2018/12/30 v2.0 sample for childdoc]
%<*driver>
%\ProvidesFile{childdoc.drv}[2018/12/30 v2.0 childdoc reference manual file]
\PassOptionsToClass{10pt,a4paper}{article}
\documentclass{ltxdoc}

\usepackage[margin=35mm]{geometry}
\usepackage{hyperref}
\usepackage{hyperxmp}
\usepackage[usenames]{color}

\hypersetup{colorlinks=true}
\hypersetup{pdfstartview=FitH}
\hypersetup{pdfpagemode=UseNone}
\hypersetup{pdfsource={}}
\hypersetup{pdflang={en-UK}}
\hypersetup{pdfcopyright={Copyright 2017-2018 Niklas Beisert.
  This work may be distributed and/or modified under the
  conditions of the LaTeX Project Public License, either version 1.3
  of this license or (at your option) any later version.}}
\hypersetup{pdflicenseurl={http://www.latex-project.org/lppl.txt}}
\hypersetup{pdfcontactaddress={ETH Zurich, ITP, HIT K,
  Wolfgang-Pauli-Strasse 27}}
\hypersetup{pdfcontactpostcode={8093}}
\hypersetup{pdfcontactcity={Zurich}}
\hypersetup{pdfcontactcountry={Switzerland}}
\hypersetup{pdfcontactemail={nbeisert@itp.phys.ethz.ch}}
\hypersetup{pdfcontacturl={http://people.phys.ethz.ch/\xmptilde nbeisert/}}

\newcommand{\secref}[1]{\hyperref[#1]{section \ref*{#1}}}

\parskip1ex
\parindent0pt
\let\olditemize\itemize
\def\itemize{\olditemize\parskip0pt}

\begin{document}

\title{The \textsf{childdoc} Package}
\hypersetup{pdftitle={The childdoc Package}}
\author{Niklas Beisert\\[2ex]
  Institut f\"ur Theoretische Physik\\
  Eidgen\"ossische Technische Hochschule Z\"urich\\
  Wolfgang-Pauli-Strasse 27, 8093 Z\"urich, Switzerland\\[1ex]
  \href{mailto:nbeisert@itp.phys.ethz.ch}
  {\texttt{nbeisert@itp.phys.ethz.ch}}}
\hypersetup{pdfauthor={Niklas Beisert}}
\hypersetup{pdfsubject={Manual for the LaTeX2e Package childdoc}}
\date{30 December 2018, \textsf{v2.0}}
\maketitle

\begin{abstract}\noindent
\textsf{childdoc} is a \LaTeXe{} package
that enables the direct compilation
of document sections included by |\include|
to individual files.
\end{abstract}

\begingroup
\parskip0ex
\tableofcontents
\endgroup

%%%%%%%%%%%%%%%%%%%%%%%%%%%%%%%%%%%%%%%%%%%%%%%%%%%%%%%%%%%%%%%%%%%%%%%%%%%%%%%%
%%%%%%%%%%%%%%%%%%%%%%%%%%%%%%%%%%%%%%%%%%%%%%%%%%%%%%%%%%%%%%%%%%%%%%%%%%%%%%%%
\section{Introduction}

\LaTeX{} provides a mechanism to structure a large document (such as a book)
into a main file and several child files (containing the chapters)
using the |\include| command.
This mechanism is beneficial for documents
which span hundreds of pages in order to
make the source file(s) more manageable.
Moreover, compilation can be restricted to
selected child files by means of the |\includeonly| command.
The latter feature can be used to reduce the compilation time while editing
(this was significantly more useful in the earlier days of \LaTeX{})
or to generate a smaller document which is easier to navigate.
Another application of |\includeonly| is to generate
documents consisting of selected parts of the complete document.

However, there are a few drawbacks of the plain |\include| mechanism:
\begin{itemize}
\item
The child files cannot be compiled on their own,
they can only be compiled via the main file.
A naive editing environment
(such as a text editor with an option
to have the current file processed by \LaTeX)
may require one to switch to the main file before compiling;
attempting to compile the child file produces errors.
\item
The main file must be modified (each time)
to adjust the |\includeonly| command
to the present needs. This easily leaves the main file in a messy state.
\item
The generated document will always carry the filename
of the main document. This is inconvenient if
several child files are to be compiled and
to be kept for distribution.
\end{itemize}

The present package provides a simple interface
to make child files individually compilable by \LaTeX{}.
Compiling a child file then has the same effect as compiling
the main file with an |\includeonly| command
to select the appropriate child.
Moreover the generated document will carry the name of the child
rather than the main file.
This resolves all three above issues.

This feature is meant to make the editing of books,
thesis documents and lecture notes somewhat more convenient.
However, the package can also be used efficiently for
composing a series of documents (such as exercise sheets)
which are typically distributed individually.
It then assists the author in generating the individual documents
(potentially in different versions)
as well as a document containing the collected series.
Another application is in developing style files
or other kinds of included material
where compilation of the style file could redirect
to a sample or test file.

%%%%%%%%%%%%%%%%%%%%%%%%%%%%%%%%%%%%%%%%%%%%%%%%%%%%%%%%%%%%%%%%%%%%%%%%%%%%%%%%
%%%%%%%%%%%%%%%%%%%%%%%%%%%%%%%%%%%%%%%%%%%%%%%%%%%%%%%%%%%%%%%%%%%%%%%%%%%%%%%%
\section{Usage}

First of all, the package \textsf{childdoc} is \emph{not} a standard
\LaTeXe{} |.sty| style file! Therefore it needs to be invoked in
a non-standard way.

%%%%%%%%%%%%%%%%%%%%%%%%%%%%%%%%%%%%%%%%%%%%%%%%%%%%%%%%%%%%%%%%%%%%%%%%%%%%%%%%
\subsection{Included Files}
\label{sec:include}

%%%%%%%%%%%%%%%%%%%%%%%%%%%%%%%%%%%%%%%%
\DescribeMacro{\childdocmain}
To use the package, add the commands
\begin{center}
\begin{tabular}{l}
|\input{childdoc.def}|\\
|\childdocmain{}|\\
\end{tabular}
\end{center}
at the very top of the main \LaTeX{} file,
in particular \emph{before} the |\documentclass| statement!
The argument of |\childdocmain| should be left empty
(but it must be present).

%%%%%%%%%%%%%%%%%%%%%%%%%%%%%%%%%%%%%%%%
\DescribeMacro{\childdocof}
Furthermore, add the commands
\begin{center}
\begin{tabular}{l}
|\input{childdoc.def}|\\
|\childdocof{|\textit{main}|}|\\
\end{tabular}
\end{center}
at the top of every child file \textit{child}
which is included by |\include{|\textit{child}|}|
from within the main file
(or at least for those files to be compiled individually).
The argument \textit{main} must be the filename of the main file.

There are a couple of
considerations in setting up the main and child documents:

%%%%%%%%%%%%%%%%%%%%%%%%%%%%%%%%%%%%%%%%
\paragraph{Restrictions.}

Please note the following restrictions:
\begin{itemize}
\item
|\childdocmain| must be called with one argument \textit{main}
to ensure compatibility with earlier version of the package.
It must either be empty (|\childdocmain{}|)
or precisely match the filename of the main file in which it is specified.
See \secref{sec:detection} for further information.
\item
The filename \textit{main} must be specified without the |.tex| extension.
\item
The filename \textit{main} is case sensitive
(even in case-insensitive file systems)
due to internal string comparison.
\item
The argument \textit{main} should be fully expanded, it cannot be a macro.
\item
Subdirectories and special characters should be avoided in filenames.
\item
The command |\childdocmain{|\textit{main}|}| must be followed by a whitespace.
It should not be followed immediately by another command
or by a comment mark `|%|'.
This is because the \TeX{} parser reads the token immediately following
the argument of |\childdocmain| and puts it
at the beginning of every child section;
however, a white\-space is ignored.
\end{itemize}

%%%%%%%%%%%%%%%%%%%%%%%%%%%%%%%%%%%%%%%%
\paragraph{Content of Main File.}

It is advisable to place all content in the child files included by |\include|.
Any output contained in the main file will appear in all child documents
unless suppressed manually;
it cannot be suppressed automatically by the |\includeonly| directive
and thus should normally be avoided.
A method to include some content in the main file
by means of conditional processing is described in \secref{sec:conditional}.

%%%%%%%%%%%%%%%%%%%%%%%%%%%%%%%%%%%%%%%%
\paragraph{Page Numbering.}

When only a part of the document is compiled,
the appropriate numbering of pages
(as well as other status parameters)
is determined from the |.aux| files.
The latter contain information from previous passes.
However this information needs to propagate through
all intermediate child documents.
Therefore the page numbering in child documents may well
be inconsistent until the complete document is compiled at least once.

A useful (if unconventional) way to always ensure a consistent
page numbering is to restart the numbering in each child document
and denote the pages by `\textit{child}|.|\textit{page}'
where \textit{child} represents the chapter/section number of the child file.
This can be achieved by the command
|\numberwithin{page}{|\textit{child}|}|
of the \textsf{amsmath} package
where \textit{child} can be |chapter| or |section|
depending on the chosen structuring.
Alternatively, one can modify the macro |\thepage| appropriately
and reset the counter |page| at the start of each child file.

%%%%%%%%%%%%%%%%%%%%%%%%%%%%%%%%%%%%%%%%%%%%%%%%%%%%%%%%%%%%%%%%%%%%%%%%%%%%%%%%
\subsection{Conditional Processing}
\label{sec:conditional}

The package provides a mechanism to compile different versions
of a document. To customise the versions further some conditional processing
can come in handy to distinguish which version is being compiled.
The package provides two macros to describe the compilation context:

%%%%%%%%%%%%%%%%%%%%%%%%%%%%%%%%%%%%%%%%
\DescribeMacro{\ifchilddoc}
The conditional |\ifchilddoc| distinguishes between the compilation of
child documents and the main document:
%
\begin{center}
|\ifchilddoc |\textit{child-code}| |[|\||else |\textit{main-code}]| \||fi|
\end{center}

%%%%%%%%%%%%%%%%%%%%%%%%%%%%%%%%%%%%%%%%
\DescribeMacro{\childdocname}
\DescribeMacro{\childdocjob}
The macro |\childdocname| contains the filename (without extension)
of the main or child file being processed.
Note that |\childdocjob| will always contain the name of the main file.

%%%%%%%%%%%%%%%%%%%%%%%%%%%%%%%%%%%%%%%%
\paragraph{Title Page.}

Conditional processing can be used to include a title or banner page
in the main document when proper precautions are taken.
Importantly, the code in the main file should ensure that the page counter
(as well as other status parameters which are stored in the |.aux| files)
takes the same value after the conditional processing.
Otherwise the page numbers may take divergent values
depending on which part is compiled.

For example, a title page could be declared by:
%
\begin{center}
\begin{tabular}{l}
|\ifchilddoc\||else|\\
|\addtocounter{page}{-1}|\\
\textit{code for title page}\\
|\newpage|\\
|\||fi|
\end{tabular}
\end{center}
%
A banner page for the child documents can be generated by:
%
\begin{center}
\begin{tabular}{l}
|\ifchilddoc|\\
|\addtocounter{page}{-1}|\\
\textit{code for banner page}\\
|\newpage|\\
|\||fi|
\end{tabular}
\end{center}
%
Here one could write a message such as:
\begin{center}
|This is the part \childdocname{} of \childdocjob{}.|
\end{center}

%%%%%%%%%%%%%%%%%%%%%%%%%%%%%%%%%%%%%%%%%%%%%%%%%%%%%%%%%%%%%%%%%%%%%%%%%%%%%%%%
\subsection{Flags}
\label{sec:flags}

The package makes it easy to generate different versions
of the main or child documents.
To this end compilation flags can be defined
and assigned different default values.
They will be particularly useful in conjunction
with the forwarding mechanism described in \secref{sec:forward}.

For example, it may be useful to have a flag |\version|
which can be set to |draft| or |final|.
The document source will contain some conditional code
depending on the value of |\version|.
Suppose further, the flag should default to |final| for the main file
and to |draft| for child files
which is a natural assignment for editing the document.
This is achieved by placing the following code
in the preamble of the main document
(below the |\childdocmain| directive):
%
\begin{center}
\begin{tabular}{l}
|\ifchilddoc|\\
|\providecommand{\version}{draft}|\\
|\||else|\\
|\providecommand{\version}{final}|\\
|\||fi|
\end{tabular}
\end{center}
%
The definition by |\providecommand| makes sure
that previous definitions are not overwritten.
Further statements |\providecommand{\version}{...}|
can thus be added before the above code to override it.

For the main file, one might add a line
(between |\childdocmain| and the above block)
%
\begin{center}
|%\ifchilddoc\||else\providecommand{\version}{draft}\||fi|
\end{center}
%
which can be uncommented to produce a draft version.
Likewise one can add a line to the very top of a child file
(above the |\childdocof{|\textit{main}|}| directive)
%
\begin{center}
|%\providecommand{\version}{final}|
\end{center}
%
which can be uncommented to produce the final version of this child document.

%%%%%%%%%%%%%%%%%%%%%%%%%%%%%%%%%%%%%%%%%%%%%%%%%%%%%%%%%%%%%%%%%%%%%%%%%%%%%%%%
\subsection{Forwarding}
\label{sec:forward}

Different versions of the main or child documents
using compilation flags as described in \secref{sec:flags}
can be (permanently) stored in different files
for convenient compilation, viewing and distribution.
To this end, the package defines a command
to pass on compilation to a different file:

%%%%%%%%%%%%%%%%%%%%%%%%%%%%%%%%%%%%%%%%
\DescribeMacro{\childdocforward}
The command |\childdocforward| redirects processing to
another source file:
%
\begin{center}
\begin{tabular}{l}
|\input{childdoc.def}|\\
|\childdocforward[|\textit{main}|]{|\textit{dest}|}|\\
\end{tabular}
\end{center}
%
The argument \textit{dest} is the destination file
(without extension).
It should be the main file or one of the child files.
Note that further \textsf{childdoc} directives
such as |\childdocof| and |\childdocforward|
in the indicated file will be processed in this form.
The optional argument \textit{main}
passes on directly to the main file \textit{main}
while pretending to compile the child \textit{dest}.
This form behaves as if \textit{dest}
issues |\childdocof{|\textit{main}|}| right away,
and no further \textsf{childdoc} directives will be processed.

%%%%%%%%%%%%%%%%%%%%%%%%%%%%%%%%%%%%%%%%
\DescribeMacro{\...prefix}
In the alternative form |\childdocforwardprefix|,
%
\begin{center}
\begin{tabular}{l}
|\input{childdoc.def}|\\
|\childdocforwardprefix[|\textit{main}|]{|\textit{prefix}|}{|\textit{dest}|}|
\end{tabular}
\end{center}
%
the destination file is determined by a pattern
depending on the current file:
To make this work, the current file must be called
`{\textit{prefix}\hspace{0.2em}\textit{suffix}}'
with \textit{prefix} matching precisely the argument.
Processing is then passed on to the file
`{\textit{dest}\hspace{0.2em}\textit{suffix}}'.
Surely, the same effect is achieved by
directly specifying the
argument `{\textit{dest}\hspace{0.2em}\textit{suffix}}'
in the first form.
However, that requires to set up a different file
for each child. With the alternative form of the command
all these files can have exactly the same content
which simplifies setting them up and maintaining them.

For example, the following file |draft.tex|
with a compilation flag |\version| as described in \secref{sec:flags}
compiles the main document as a draft:
%
\begin{center}
\begin{tabular}{l}
|\def\version{draft}|\\
|\input{childdoc.def}|\\
|\childdocforward{|\textit{main}|}|
\end{tabular}
\end{center}
%
Likewise, the following files |final|\textit{nn}|.tex|
compile the final version of the child document
|child|\textit{nn}|.tex|:
%
\begin{center}
\begin{tabular}{l}
|\def\version{final}|\\
|\input{childdoc.def}|\\
|\childdocforwardprefix{final}{child}|
\end{tabular}
\end{center}
%

Note that when several versions of a main file and/or of each child file
are to be generated, it may be convenient to set up a |Makefile| or
shell script to automatise the process.

%%%%%%%%%%%%%%%%%%%%%%%%%%%%%%%%%%%%%%%%%%%%%%%%%%%%%%%%%%%%%%%%%%%%%%%%%%%%%%%%
\subsection{Command Line Processing}
\label{sec:commandline}

The effect of redirection files can also be achieved by invoking
the \LaTeX{} compiler with a more elaborate command line.
Most conveniently this should be done as part
of a shell script or a |Makefile|.

When using \textsf{childdoc} in the main file, the following
command lines effectively perform a redirection
(note that depending on the shell being used,
backslashes may have to be doubled: `|\|' $\to$ `|\\|'):
%
\begin{center}
|... -jobname "|\textit{target}|" |\\|"|[\textit{flags}]%
|\input{childdoc.def}\childdocforward[|\textit{main}|]{|\textit{dest}|}"|
\end{center}
%
Here \textit{target} is the name of the output file,
\textit{main} is the name of the main file
and \textit{dest} is the name of the main or child file to be processed
(all filenames without extensions).
The optional argument \textit{main} can be omitted
if \textit{main} matches \textit{dest}.
Optionally, compilation \textit{flags} can be defined via |\def| commands.
This command line makes the \TeX{} engine believe
it is compiling the file \textit{target}
whose content is specified as the latter parameter.
The provided code then forwards the processing to
\textit{main} or \textit{dest} as described in \secref{sec:forward}.

%%%%%%%%%%%%%%%%%%%%%%%%%%%%%%%%%%%%%%%%%%%%%%%%%%%%%%%%%%%%%%%%%%%%%%%%%%%%%%%%
\subsection{Include by Input}
\label{sec:input}

Including child documents by |\include| has some restrictions by design.
Most notably, the content of a child document always occupies
its own set of pages; pages cannot be shared between child documents.
Usually, this behaviour makes perfect sense
because each child document contain an essential part of the document.
However, in some situations it may be desirable to compose
a document from a collection of parts
without having mandatory page breaks between then.
For this case, the package
provides a mechanism to include parts
by |\input| which can also be processed individually.
However, by construction this mechanism
requires manual handling of the content to be output.

%%%%%%%%%%%%%%%%%%%%%%%%%%%%%%%%%%%%%%%%
\DescribeMacro{\ifchilddocmanual}
The main file should be prepared as usual, see \secref{sec:include}.
However, the document body must make a distinction
between processing of an individual part and of the main document, e.g.:
%
\begin{center}
\begin{tabular}{l}
|\ifchilddocmanual|\\
|\input{\childdocname}|\\
|\||else|\\
\textit{document body with }|\input{|\textit{part}|}|\\
|\||fi|
\end{tabular}
\end{center}
%
The conditional |\ifchilddocmanual| is true whenever
a part to be included by |\input| is being compiled,
and the name of the part is stored in |\childdocname|.

%%%%%%%%%%%%%%%%%%%%%%%%%%%%%%%%%%%%%%%%
\DescribeMacro{\childdocby}
Each part to be included by |\input| should start with:
%
\begin{center}
\begin{tabular}{l}
|\input{childdoc.def}|\\
|\childdocby{|\textit{main}|}|\\
\end{tabular}
\end{center}
%
The directive |\childdocby| is similar to |\childdocof|
described in \secref{sec:include},
but the subsequent selection of content must be done manually.
To that end, both |\ifchilddoc| and |\ifchilddocmanual|
will be true upon processing of a part,
and the name of the part is stored in |\childdocname|.
Note that |\jobname| will be set to the filename of the current part
so that each part receives an individual |.aux| file
that does not interfere with the |.aux| file(s) of the main document.
This behaviour can be altered by the alternative form
|\childdocby[*]{|\textit{main}|}| (with a non-empty optional argument)
which uses the |.aux| file of the main document
by setting |\jobname| to \textit{main}.

%%%%%%%%%%%%%%%%%%%%%%%%%%%%%%%%%%%%%%%%%%%%%%%%%%%%%%%%%%%%%%%%%%%%%%%%%%%%%%%%
\subsection{Driver Development}
\label{sec:driver}

The \textsf{childdoc} mechanism can also be use for the development
of definition files such as \LaTeX{} styles or classes.
This case differs from the above setup with multiple parts
included by |\include| in that no |\includeonly| should be invoked.
This can be achieved by starting the include file
(before |\ProvidesPackage|) with:
%
\begin{center}
\begin{tabular}{l}
|\input{childdoc.def}|\\
|\childdocforward{|\textit{main}|}|\\
\end{tabular}
\end{center}
%
or alternatively with:
%
\begin{center}
\begin{tabular}{l}
|\input{childdoc.def}|\\
|\childdocby{|\textit{main}|}|\\
\end{tabular}
\end{center}
%
Both forms have slightly different effects as described above.
The main file is prepared as usual, see \secref{sec:include}.

%%%%%%%%%%%%%%%%%%%%%%%%%%%%%%%%%%%%%%%%%%%%%%%%%%%%%%%%%%%%%%%%%%%%%%%%%%%%%%%%
\subsection{Legacy Detection}
\label{sec:detection}

The directive |\childdocmain| in the main file can detect
whether the complete document or merely a child is to be compiled
even without using the directive |\childdocof|.
This method is deprecated because it is less robust
and there is no compelling reason to use it;
it is merely provided for backward compatibility
and it may be removed in future versions.

If the detection mechanism is to be used,
it is mandatory to correctly specify
the filename of the main file as the argument of |\childdocmain|:
%
\begin{center}
\begin{tabular}{l}
|\input{childdoc.def}|\\
|\childdocmain{|\textit{main}|}|\\
\end{tabular}
\end{center}
%
If |\jobname| does not match the argument \textit{main} of |\childdocmain|,
it is assumed that |\jobname| points to the child file to be compiled.
When using |\childdocmain| with the main file specified as argument,
it suffices to start a child file
with just |\input{|\textit{main}|}|
without loading of the package and using |\childdocof|.
If instead all processing is done
with the appropriate \textsf{childdoc} directives,
the argument of \textit{main} of |\childdocmain| can be empty.

An alternative version of the command line processing described
in \secref{sec:commandline} using the detection mechanism reads:
%
\begin{center}
|... -jobname "|\textit{target}|" "|[\textit{flags}]%
[|\def\jobname{|\textit{dest}|}|]|\input{|\textit{main}|}"|
\end{center}

%%%%%%%%%%%%%%%%%%%%%%%%%%%%%%%%%%%%%%%%%%%%%%%%%%%%%%%%%%%%%%%%%%%%%%%%%%%%%%%%
\subsection{Manual Code}
\label{sec:manual}

In case one cannot be certain whether the definitions file |childdoc.def|
is installed on the target \TeX{} distribution
and one prefers not to ship it,
it is conceivable to paste a few relevant commands into the sources.

To that end, drop all statements |\input{childdoc.def}|
and perform the replacements as outlined below.
Instead of |\childdocmain{|\textit{main}|}| add the following code
to the top of the main file:
%
\begin{center}
\begin{tabular}{l}
|\||ifdefined\childdocname\endinput\||fi\newif\ifchilddoc|\\
|\edef\childdocname{\scantokens\expandafter{\jobname\noexpand}}|\\
|\def\childdocmain{|\textit{main}|}\||ifx\childdocmain\childdocname\||else|\\
|\childdoctrue\includeonly{\childdocname}\let\jobname\childdocmain\||fi|\\
\end{tabular}
\end{center}
%
Instead of |\childdocof{|\textit{main}|}| just include the main file
at the top of each child file:
%
\begin{center}
|\input{|\textit{main}|}|
\end{center}
%
A simple redirection |\childdocforward{|\textit{dest}|}| is achieved by:
%
\begin{center}
|\def\jobname{|\textit{dest}|}\input{\jobname}|
\end{center}
%
The redirection with prefix
|\childdocforwardprefix[|\textit{prefix}|]{|\textit{dest}|}|
is accomplished by:
%
\begin{center}
\begin{tabular}{l}
|{\edef\jobname{\scantokens\expandafter{\jobname\noexpand}}|\\
|\def\redirectjob |\textit{prefix}|#1~~~{\gdef\jobname{|\textit{dest}|#1}}|\\
|\expandafter\redirectjob\jobname~~~}\input{\jobname}|
\end{tabular}
\end{center}

In an alternative approach,
child documents can be compiled by a specific command line
without additional code or specific definitions:
%
\begin{center}
|... -jobname "|\textit{target}|" "|[\textit{flags}]%
|\includeonly{|\textit{dest}|}\input{|\textit{main}|}"|
\end{center}
%

%%%%%%%%%%%%%%%%%%%%%%%%%%%%%%%%%%%%%%%%%%%%%%%%%%%%%%%%%%%%%%%%%%%%%%%%%%%%%%%%
%%%%%%%%%%%%%%%%%%%%%%%%%%%%%%%%%%%%%%%%%%%%%%%%%%%%%%%%%%%%%%%%%%%%%%%%%%%%%%%%
\section{Information}

%%%%%%%%%%%%%%%%%%%%%%%%%%%%%%%%%%%%%%%%%%%%%%%%%%%%%%%%%%%%%%%%%%%%%%%%%%%%%%%%
\subsection{Copyright}

Copyright \copyright{} 2017--2018 Niklas Beisert

This work may be distributed and/or modified under the
conditions of the \LaTeX{} Project Public License, either version 1.3
of this license or (at your option) any later version.
The latest version of this license is in
  \url{http://www.latex-project.org/lppl.txt}
and version 1.3 or later is part of all distributions of \LaTeX{}
version 2005/12/01 or later.

This work has the LPPL maintenance status `maintained'.

The Current Maintainer of this work is Niklas Beisert.

This work consists of the files |README.txt|, |childdoc.ins| and |childdoc.dtx|
as well as the derived files |childdoc.def|, |cdocsamp.tex|
with |cdocsch1.tex|, |cdocsch2.tex|, |cdocspt3.tex|, |cdocspt4.tex|,
|cdocsdrf.tex|, |cdocsfn1.tex|, |cdocsfn2.tex|
as well as |childdoc.pdf|.

%%%%%%%%%%%%%%%%%%%%%%%%%%%%%%%%%%%%%%%%%%%%%%%%%%%%%%%%%%%%%%%%%%%%%%%%%%%%%%%%
\subsection{Files and Installation}

The package consists of the files:
%
\begin{center}
\begin{tabular}{ll}
    |README.txt|   & readme file \\
    |childdoc.ins| & installation file \\
    |childdoc.dtx| & source file \\
    |childdoc.def| & definition file \\
    |cdocsamp.tex| & sample main file \\
    |cdocsch1.tex| & sample include file \\
    |cdocsch2.tex| & sample include file \\
    |cdocspt3.tex| & sample part file \\
    |cdocspt4.tex| & sample part file \\
    |cdocsdrf.tex| & sample redirection file \\
    |cdocsfn1.tex| & sample redirection file \\
    |cdocsfn2.tex| & sample redirection file \\
    |childdoc.pdf| & manual
\end{tabular}
\end{center}
%
The distribution consists of the files
|README.txt|, |childdoc.ins| and |childdoc.dtx|.
%
\begin{itemize}
\item
Run (pdf)\LaTeX{} on |childdoc.dtx|
to compile the manual |childdoc.pdf| (this file).
\item
Run \LaTeX{} on |childdoc.ins| to create the definitions file |childdoc.def|
and the sample |cdocsamp.tex| with include files
|cdocsch1.tex|, |cdocsch2.tex|, |cdocspt3.tex|, |cdocspt4.tex|,
|cdocsdrf.tex|, |cdocsfn1.tex|, |cdocsfn2.tex|.
Then copy the file |childdoc.def| to an appropriate directory of your \LaTeX{}
distribution, e.g.\ \textit{texmf-root}|/tex/latex/childdoc|.
\end{itemize}

%%%%%%%%%%%%%%%%%%%%%%%%%%%%%%%%%%%%%%%%%%%%%%%%%%%%%%%%%%%%%%%%%%%%%%%%%%%%%%%%
\subsection{Related CTAN Packages}

There are several other packages which offer a similar functionality:
%
\begin{itemize}
\item
The packages
\href{http://ctan.org/pkg/docmute}{\textsf{docmute}},
\href{http://ctan.org/pkg/includex}{\textsf{includex}} and
\href{http://ctan.org/pkg/standalone}{\textsf{standalone}}
provide commands to include only the document body of
a child file thus allowing both files to be compiled individually.
\item
The packages \href{http://ctan.org/pkg/subdocs}{\textsf{subdocs}}
and \href{http://ctan.org/pkg/subfiles}{\textsf{subfiles}}
provide structures in which the main and child documents can be
encapsulated and allowing them to be compiled individually.
The inclusion mechanism is different from the conventional |\include|.
\item
The package \href{http://ctan.org/pkg/combine}{\textsf{combine}}
is an elaborate solution to combine several documents into one.
\end{itemize}
%
See also the CTAN topic \href{http://ctan.org/topic/subdocs}{\textsf{subdocs}}
for further related packages.
The present package differs from the above solutions in that
a document structure constructed with the conventional |\include| mechanism
just needs two extra commands at the top of every file
such that all constituent files can be compiled individually.

%%%%%%%%%%%%%%%%%%%%%%%%%%%%%%%%%%%%%%%%%%%%%%%%%%%%%%%%%%%%%%%%%%%%%%%%%%%%%%%%
%\subsection{Feature Suggestions}
%
%The following is a list of features which may be useful for future
%versions of this package:
%%
%\begin{itemize}
%\item
%\ldots
%\end{itemize}

%%%%%%%%%%%%%%%%%%%%%%%%%%%%%%%%%%%%%%%%%%%%%%%%%%%%%%%%%%%%%%%%%%%%%%%%%%%%%%%%
\subsection{Revision History}

%%%%%%%%%%%%%%%%%%%%%%%%%%%%%%%%%%%%%%%%
\paragraph{v2.0:} 2018/12/30

\begin{itemize}
\item
immediate forward processing
\item
added |\childdocby| mechanism
\item
manual restructured
\end{itemize}

%%%%%%%%%%%%%%%%%%%%%%%%%%%%%%%%%%%%%%%%
\paragraph{v1.6:} 2018/01/17

\begin{itemize}
\item
application for development of include files
\item
corrections to manual
\end{itemize}

%%%%%%%%%%%%%%%%%%%%%%%%%%%%%%%%%%%%%%%%
\paragraph{v1.5:} 2017/05/21

\begin{itemize}
\item
more complete structuring introduced
\item
|\childdocof| introduced
\item
|\childdoc| renamed to |\childdocmain|
\item
|\childredirect| renamed to |\childdocforward| and |\childdocforwardprefix|
and functionality expanded
\end{itemize}

%%%%%%%%%%%%%%%%%%%%%%%%%%%%%%%%%%%%%%%%
\paragraph{v1.0:} 2017/04/27

\begin{itemize}
\item
manual and install package
\item
first version published on CTAN
\end{itemize}

%%%%%%%%%%%%%%%%%%%%%%%%%%%%%%%%%%%%%%%%
\paragraph{v0.6:} 2017/04/26

\begin{itemize}
\item
redirection mechanism added
\end{itemize}

%%%%%%%%%%%%%%%%%%%%%%%%%%%%%%%%%%%%%%%%
\paragraph{v0.5:} 2017/04/26

\begin{itemize}
\item
functionality in definition file
\end{itemize}


%%%%%%%%%%%%%%%%%%%%%%%%%%%%%%%%%%%%%%%%%%%%%%%%%%%%%%%%%%%%%%%%%%%%%%%%%%%%%%%%
%%%%%%%%%%%%%%%%%%%%%%%%%%%%%%%%%%%%%%%%%%%%%%%%%%%%%%%%%%%%%%%%%%%%%%%%%%%%%%%%
%%%%%%%%%%%%%%%%%%%%%%%%%%%%%%%%%%%%%%%%%%%%%%%%%%%%%%%%%%%%%%%%%%%%%%%%%%%%%%%%
\appendix

\settowidth\MacroIndent{\rmfamily\scriptsize 000\ }

 \DocInput{childdoc.dtx}

\end{document}
%</driver>
% \fi
%
% %%%%%%%%%%%%%%%%%%%%%%%%%%%%%%%%%%%%%%%%%%%%%%%%%%%%%%%%%%%%%%%%%%%%%%%%%%%%%%
% %%%%%%%%%%%%%%%%%%%%%%%%%%%%%%%%%%%%%%%%%%%%%%%%%%%%%%%%%%%%%%%%%%%%%%%%%%%%%%
% \section{Sample}
%\iffalse
%<*samplemain>
%\fi
%
% The following presents a sample document
% with two chapters, two parts, a title page,
% a compile flag as well as three forwarding files to set the flag.
% It consists of eight |.tex| files:
% \begin{center}
% \begin{tabular}{ll}
% |cdocsamp.tex|&main file\\
% |cdocsch1.tex|&include file for chapter 1\\
% |cdocsch2.tex|&include file for chapter 2\\
% |cdocspt3.tex|&include file for part 3\\
% |cdocspt4.tex|&include file for part 4\\
% |cdocsdrf.tex|&forwarding file for main file in draft mode\\
% |cdocsfi1.tex|&forwarding file for final version of chapter 1\\
% |cdocsfi2.tex|&forwarding file for final version of chapter 2\\
% \end{tabular}
% \end{center}
% Each of the eight files can be compiled directly by the \LaTeX{} compiler.
%
% %%%%%%%%%%%%%%%%%%%%%%%%%%%%%%%%%%%%%%
% \paragraph{Main File.}
%
% The main file is called |cdocsamp.tex|.
%
% Load the \textsf{childdoc} definitions and
% declare the filename for the main document:
%    \begin{macrocode}
\input{childdoc.def}
\childdocmain{}
%    \end{macrocode}

% Optional override for |\version| flag:
%    \begin{macrocode}
%%\ifchilddoc\else\providecommand{\version}{draft}\fi
%    \end{macrocode}

% Define the default values for the |\version| flag
% (|final| for the main file and |draft| for childs):
%    \begin{macrocode}
\ifchilddoc
\providecommand{\version}{draft}
\else
\providecommand{\version}{final}
\fi
%    \end{macrocode}

% Load the standard document class:
%    \begin{macrocode}
\documentclass[12pt]{article}
%    \end{macrocode}

% Start the document body:
%    \begin{macrocode}
\begin{document}
%    \end{macrocode}

% Declare a title page.
% Print title, part of document being processed and version flag:
%    \begin{macrocode}
\addtocounter{page}{-1}
\begin{center}
{\LARGE\bfseries{}childdoc example\par}
\vspace{1cm}
\ifchilddoc
\ifchilddocmanual part\else chapter\fi:
`\childdocname' of `\childdocjob'\par
\else
main document: `\childdocjob'\par
\fi
version: \version\par
\end{center}
\newpage
%    \end{macrocode}

% Manually include selected file,
% otherwise process as usual:
%    \begin{macrocode}
\ifchilddocmanual
\section*{part `\childdocname'}
\input{\childdocname}
\else
%    \end{macrocode}

% Include the two chapters:
%    \begin{macrocode}
\include{cdocsch1}
\include{cdocsch2}
%    \end{macrocode}

% Include the two parts unless only chapters should be displayed:
%    \begin{macrocode}
\ifchilddoc\else
\section{part three}
\input{cdocspt3}
\section{part four}
\input{cdocspt4}
\fi
%    \end{macrocode}

% Process as usual until here:
%    \begin{macrocode}
\fi
%    \end{macrocode}

% End of document body:
%    \begin{macrocode}
\end{document}
%    \end{macrocode}
%\iffalse
%</samplemain>
%\fi
%
% %%%%%%%%%%%%%%%%%%%%%%%%%%%%%%%%%%%%%%
% \paragraph{Chapter Include Files.}
%
% The include files are called |cdocsch1.tex| and |cdocsch2.tex|.
%
%\iffalse
%<*samplechap1|samplechap2>
%\fi

% Optional override for |\version| flag:
%    \begin{macrocode}
%%\providecommand{\version}{final}
%    \end{macrocode}

% Include the main document:
%    \begin{macrocode}
\input{childdoc.def}
\childdocof{cdocsamp}
%    \end{macrocode}

%\iffalse
%</samplechap1|samplechap2>
%\fi
%
%\iffalse
%<*samplechap1>
%\fi
% Some text for chapter 1:
%    \begin{macrocode}
\section{one}
some text in chapter one
%    \end{macrocode}

%\iffalse
%</samplechap1>
%\fi
% Some text for chapter 2:
%\iffalse
%<*samplechap2>
%\fi
%    \begin{macrocode}
\section{two}
more text in chapter two
%    \end{macrocode}

%\iffalse
%</samplechap2>
%\fi
%
% %%%%%%%%%%%%%%%%%%%%%%%%%%%%%%%%%%%%%%
% \paragraph{Part Include Files.}
%
% The include files are called |cdocspt3.tex| and |cdocspt4.tex|.
%
%\iffalse
%<*samplepart3|samplepart4>
%\fi

% Optional override for |\version| flag:
%    \begin{macrocode}
%%\providecommand{\version}{final}
%    \end{macrocode}

% Include the main document:
%    \begin{macrocode}
\input{childdoc.def}
\childdocby{cdocsamp}
%    \end{macrocode}

%\iffalse
%</samplepart3|samplepart4>
%\fi
%
%\iffalse
%<*samplepart3>
%\fi
% Some text for part 3:
%    \begin{macrocode}
some text in part three
%    \end{macrocode}

%\iffalse
%</samplepart3>
%\fi
% Some text for part 4:
%\iffalse
%<*samplepart4>
%\fi
%    \begin{macrocode}
more text in part four
%    \end{macrocode}

%\iffalse
%</samplepart4>
%\fi
%
% %%%%%%%%%%%%%%%%%%%%%%%%%%%%%%%%%%%%%%
% \paragraph{Forwarding for a Complete Draft.}
%
% The following forwarding file |cdocsdrf.tex|
% compiles the main document in draft mode:
%\iffalse
%<*sampledraft>
%\fi
%    \begin{macrocode}
\def\version{draft}
\input{childdoc.def}
\childdocforward{cdocsamp}
%    \end{macrocode}

%\iffalse
%</sampledraft>
%\fi
%
% %%%%%%%%%%%%%%%%%%%%%%%%%%%%%%%%%%%%%%
% \paragraph{Forwarding for Final Version of the Chapters.}
%
% The following forwarding files |cdocsfn1.tex| and |cdocsfn2.tex|
% (with identical content)
% compile the final versions of the child documents
% |cdocsch1.tex| and |cdocsch2.tex|, respectively:
%\iffalse
%<*samplefinal>
%\fi
%    \begin{macrocode}
\def\version{final}
\input{childdoc.def}
\childdocforwardprefix[cdocsamp]{cdocsfn}{cdocsch}
%    \end{macrocode}

%\iffalse
%</samplefinal>
%\fi
%
% %%%%%%%%%%%%%%%%%%%%%%%%%%%%%%%%%%%%%%
% \paragraph{Command Line Processing.}
%
% The following three command lines generate the output files
% |cdocscld|, |cdocscl1| and |cdocscl2|
% which should be identical to
% |cdocsdrf|, |cdocsch1| and |cdocsfn2|, respectively:
% \begin{center}
% \begin{tabular}{l}
% |latex -jobname cdocscld \|\\
% |  "\def\version{draft}\input{childdoc.def}\childdocforward{cdocsamp}"|\\
% |latex -jobname cdocscl1 \|\\
% |  "\input{childdoc.def}\childdocforward[cdocsamp]{cdocsch1}"|\\
% |latex -jobname cdocscl2 \|\\
% |  "\def\version{final}\input{childdoc.def}\childdocforward{cdocsch2}"|
% \end{tabular}
% \end{center}
% Note that the trailing backslash on each first line
% merely continues the input to the second line
% (for convenient cut ant paste).
% Furthermore, the command |latex| can be replaced by any
% of its alternative versions such as |pdflatex|.
%
% %%%%%%%%%%%%%%%%%%%%%%%%%%%%%%%%%%%%%%%%%%%%%%%%%%%%%%%%%%%%%%%%%%%%%%%%%%%%%%
% %%%%%%%%%%%%%%%%%%%%%%%%%%%%%%%%%%%%%%%%%%%%%%%%%%%%%%%%%%%%%%%%%%%%%%%%%%%%%%
% \section{Implementation}
%\iffalse
%<*package>
%\fi
%
% This section describes the definitions file |childdoc.def|.

% The definitions cannot be loaded using |\usepackage| or |\RequirePackage|
% which has a mechanism to prevent loading a style file more than once.
% When loading the definitions by means of |\input|
% multiple instances have to be prevented manually:
%\iffalse
%This code needs to be before the `\ProvidesFile' directive
%which is defined at the beginning of this file.
%Therefore it is also placed there and commented out here.
%</package>
%<*discard>
%\fi
%    \begin{macrocode}
\ifdefined\childdocmain\endinput\fi
%    \end{macrocode}
%\iffalse
%</discard>
%<*package>
%\fi
%
% \macro{\ifchilddoc}
% \macro{\ifchilddocmanual}
% The conditional |\ifchilddoc| tells whether a
% child (true) or main (false) document is being compiled.
% The conditional |\ifchilddocmanual| tells whether
% the |\includeonly| mechanism is used (false) or
% the selection of child files must be performed manually (true).
% The definitions initialise to false:
%    \begin{macrocode}
\newif\ifchilddoc
\newif\ifchilddocmanual
%    \end{macrocode}

% \macro{\childdocname}
% \macro{\childdocjob}
% The macro |\childdocname| stores the name of the main document
% to be compiled. The macro |\childdocjob| stores the name of
% the document on which the \LaTeX{} compiler was originally invoked.
% The content of |\jobname| cannot be compared
% to filenames specified in the source due to different catcodes.
% The following code rescans |\jobname|, stores the result
% in |\childdocname| and saves a copy in |\childdocjob|:
%    \begin{macrocode}
\edef\childdocname{\scantokens\expandafter{\jobname\noexpand}}
\let\childdocjob\childdocname
%    \end{macrocode}

% \macro{\childdocdisable}
% The macro |\childdocdisable| prevents the main file
% from being processed more than once.
% At this stage, the main document command |\childdocmain|
% is assumed to be called once again where it should do nothing.
% Any subsequent call to it should prevent
% a secondary processing of the main document
% It overwrites the forwarding commands
% |\childdocof| and |\childdocforward|
% with empty macros to prevent further inclusions of the main document:
%    \begin{macrocode}
\newcommand{\childdocdisable}
{
  \renewcommand{\childdocmain}[1]{\renewcommand{\childdocmain}[1]{\endinput}}
  \renewcommand{\childdocof}[1]{}
  \renewcommand{\childdocby}[2][]{}
  \renewcommand{\childdocforward}[2][]{}
  \renewcommand{\childdocdisable}{}
}
%    \end{macrocode}

% \macro{\childdocmain}
% The macro |\childdocmain| is to be called at the top of the main file
% with nothing or the main filename (without extension) as argument.
% First, it breaks loops.
% If the argument is not empty and does not match |\childdocname|
% (which is set by the first inclusion of |childdoc.def|),
% |\ifchilddoc| is set to true, |\includeonly| is applied to the child file
% and |\jobname| is set to the main file
% (for proper handling of |.aux| files):
%    \begin{macrocode}
\newcommand{\childdocmain}[1]
{
  \childdocdisable\childdocmain{}
  \if?#1?\else
    \begingroup
      \def\childdoctmp{#1}
      \ifx\childdoctmp\childdocname
        \def\childdoctmp{}
      \else
        \def\childdoctmp
        {
          \childdoctrue
          \includeonly{\childdocname}
          \def\childdocjob{#1}
          \def\jobname{#1}
        }
      \fi
      \expandafter
    \endgroup
    \childdoctmp
  \fi
}
%    \end{macrocode}

% \macro{\childdocof}
% The command |\childdocof| redirects
% compilation to the main file |#1|.
%    \begin{macrocode}
\newcommand{\childdocof}[1]
{
  \childdocdisable
  \childdoctrue
  \includeonly{\childdocname}
  \def\jobname{#1}
  \def\childdocjob{#1}
  \input{#1}
}
%    \end{macrocode}

% \macro{\childdocby}
% The command |\childdocby| ....
%    \begin{macrocode}
\newcommand{\childdocby}[2][]
{
  \childdocdisable
  \childdoctrue
  \childdocmanualtrue
  \if?#1?\else
    \def\jobname{#2}
  \fi
  \def\childdocjob{#2}
  \input{#2}
  \endinput
}
%    \end{macrocode}

% \macro{\childdocforward}
% The command |\childdocforward| redirects
% compilation to the main file or
% (if the optional argument is given) a child file.
% Parameters are set as if the main file
% or a child file starting with |\childdocof| was compiled.
% Then compilation is handed over to the main file:
%    \begin{macrocode}
\newcommand{\childdocforward}[2][]
{
  \begingroup
    \if?#1?
      \def\childdoctmp
      {
        \def\childdocname{#2}
        \def\childdocjob{#2}
        \def\jobname{#2}
        \input{#2}
        \endinput
      }
    \else
      \def\childdoctmp
      {
        \childdocdisable
        \def\childdocname{#2}
        \childdoctrue
        \includeonly{#2}
        \def\childdocjob{#1}
        \def\jobname{#1}
        \input{#1}
        \endinput
      }
    \fi
    \expandafter
  \endgroup
  \childdoctmp
}
%    \end{macrocode}

% \macro{\childdocforwardprefix}
% The command |\childdocforwardprefix| redirects
% compilation to the main or a child file by means of a pattern.
% The prefix |#1| in the current filename is replaced by |#2|
% and the suffix of the current filename is kept
% (it is assumed that the filename does not contain the substring `|~~~|'
% which is used as a delimiter).
% Compilation is handed over to the new file by |\childdocforward|:
%    \begin{macrocode}
\newcommand{\childdocforwardprefix}[3][]
{
  \begingroup
    \def\childdocextract #2##1~~~{\def\childdoctmp{\childdocforward[#1]{#3##1}}}
    \expandafter\childdocextract\childdocname~~~
    \expandafter
  \endgroup
  \childdoctmp
}
%    \end{macrocode}

% \macro{\childdoc}
% The deprecated macro |\childdoc| is a legacy version of |\childdocmain|:
%    \begin{macrocode}
\newcommand{\childdoc}{\childdocmain}
%    \end{macrocode}

% \macro{\childdocredirect}
% The deprecated macro |\childdocredirect| is a legacy version
% of |\childdocforward| and |\childdocforwardprefix|:
%    \begin{macrocode}
\newcommand{\childdocredirect}[2][]
{
  \begingroup
    \if?#1?
      \def\childdoctmp{\childdocforward{#2}}
    \else
      \def\childdoctmp{\childdocforwardprefix{#1}{#2}}
    \fi
    \expandafter
  \endgroup
  \childdoctmp
}
%    \end{macrocode}

%\iffalse
%</package>
%\fi
%
\endinput
\childdocforward[|\textit{main}|]{|\textit{dest}|}"|
\end{center}
%
Here \textit{target} is the name of the output file,
\textit{main} is the name of the main file
and \textit{dest} is the name of the main or child file to be processed
(all filenames without extensions).
The optional argument \textit{main} can be omitted
if \textit{main} matches \textit{dest}.
Optionally, compilation \textit{flags} can be defined via |\def| commands.
This command line makes the \TeX{} engine believe
it is compiling the file \textit{target}
whose content is specified as the latter parameter.
The provided code then forwards the processing to
\textit{main} or \textit{dest} as described in \secref{sec:forward}.

%%%%%%%%%%%%%%%%%%%%%%%%%%%%%%%%%%%%%%%%%%%%%%%%%%%%%%%%%%%%%%%%%%%%%%%%%%%%%%%%
\subsection{Include by Input}
\label{sec:input}

Including child documents by |\include| has some restrictions by design.
Most notably, the content of a child document always occupies
its own set of pages; pages cannot be shared between child documents.
Usually, this behaviour makes perfect sense
because each child document contain an essential part of the document.
However, in some situations it may be desirable to compose
a document from a collection of parts
without having mandatory page breaks between then.
For this case, the package
provides a mechanism to include parts
by |\input| which can also be processed individually.
However, by construction this mechanism
requires manual handling of the content to be output.

%%%%%%%%%%%%%%%%%%%%%%%%%%%%%%%%%%%%%%%%
\DescribeMacro{\ifchilddocmanual}
The main file should be prepared as usual, see \secref{sec:include}.
However, the document body must make a distinction
between processing of an individual part and of the main document, e.g.:
%
\begin{center}
\begin{tabular}{l}
|\ifchilddocmanual|\\
|\input{\childdocname}|\\
|\||else|\\
\textit{document body with }|\input{|\textit{part}|}|\\
|\||fi|
\end{tabular}
\end{center}
%
The conditional |\ifchilddocmanual| is true whenever
a part to be included by |\input| is being compiled,
and the name of the part is stored in |\childdocname|.

%%%%%%%%%%%%%%%%%%%%%%%%%%%%%%%%%%%%%%%%
\DescribeMacro{\childdocby}
Each part to be included by |\input| should start with:
%
\begin{center}
\begin{tabular}{l}
|% \iffalse
%
% childdoc.dtx Copyright (C) 2017-2018 Niklas Beisert
%
% This work may be distributed and/or modified under the
% conditions of the LaTeX Project Public License, either version 1.3
% of this license or (at your option) any later version.
% The latest version of this license is in
%   http://www.latex-project.org/lppl.txt
% and version 1.3 or later is part of all distributions of LaTeX
% version 2005/12/01 or later.
%
% This work has the LPPL maintenance status `maintained'.
%
% The Current Maintainer of this work is Niklas Beisert.
%
% This work consists of the files childdoc.dtx and childdoc.ins
% and the derived files childdoc.def and cdocsamp.tex with
% cdocsch1.tex, cdocsch2.tex, cdocsdrf.tex, cdocsfn1.tex, cdocsfn2.tex.
%
%<package>\ifdefined\childdocmain\endinput\fi
%<package>\ProvidesFile{childdoc.def}[2018/12/30 v2.0 child document driver]
%<samplemain>\ProvidesFile{cdocsamp.tex}[2018/12/30 v2.0 sample for childdoc]
%<*driver>
%\ProvidesFile{childdoc.drv}[2018/12/30 v2.0 childdoc reference manual file]
\PassOptionsToClass{10pt,a4paper}{article}
\documentclass{ltxdoc}

\usepackage[margin=35mm]{geometry}
\usepackage{hyperref}
\usepackage{hyperxmp}
\usepackage[usenames]{color}

\hypersetup{colorlinks=true}
\hypersetup{pdfstartview=FitH}
\hypersetup{pdfpagemode=UseNone}
\hypersetup{pdfsource={}}
\hypersetup{pdflang={en-UK}}
\hypersetup{pdfcopyright={Copyright 2017-2018 Niklas Beisert.
  This work may be distributed and/or modified under the
  conditions of the LaTeX Project Public License, either version 1.3
  of this license or (at your option) any later version.}}
\hypersetup{pdflicenseurl={http://www.latex-project.org/lppl.txt}}
\hypersetup{pdfcontactaddress={ETH Zurich, ITP, HIT K,
  Wolfgang-Pauli-Strasse 27}}
\hypersetup{pdfcontactpostcode={8093}}
\hypersetup{pdfcontactcity={Zurich}}
\hypersetup{pdfcontactcountry={Switzerland}}
\hypersetup{pdfcontactemail={nbeisert@itp.phys.ethz.ch}}
\hypersetup{pdfcontacturl={http://people.phys.ethz.ch/\xmptilde nbeisert/}}

\newcommand{\secref}[1]{\hyperref[#1]{section \ref*{#1}}}

\parskip1ex
\parindent0pt
\let\olditemize\itemize
\def\itemize{\olditemize\parskip0pt}

\begin{document}

\title{The \textsf{childdoc} Package}
\hypersetup{pdftitle={The childdoc Package}}
\author{Niklas Beisert\\[2ex]
  Institut f\"ur Theoretische Physik\\
  Eidgen\"ossische Technische Hochschule Z\"urich\\
  Wolfgang-Pauli-Strasse 27, 8093 Z\"urich, Switzerland\\[1ex]
  \href{mailto:nbeisert@itp.phys.ethz.ch}
  {\texttt{nbeisert@itp.phys.ethz.ch}}}
\hypersetup{pdfauthor={Niklas Beisert}}
\hypersetup{pdfsubject={Manual for the LaTeX2e Package childdoc}}
\date{30 December 2018, \textsf{v2.0}}
\maketitle

\begin{abstract}\noindent
\textsf{childdoc} is a \LaTeXe{} package
that enables the direct compilation
of document sections included by |\include|
to individual files.
\end{abstract}

\begingroup
\parskip0ex
\tableofcontents
\endgroup

%%%%%%%%%%%%%%%%%%%%%%%%%%%%%%%%%%%%%%%%%%%%%%%%%%%%%%%%%%%%%%%%%%%%%%%%%%%%%%%%
%%%%%%%%%%%%%%%%%%%%%%%%%%%%%%%%%%%%%%%%%%%%%%%%%%%%%%%%%%%%%%%%%%%%%%%%%%%%%%%%
\section{Introduction}

\LaTeX{} provides a mechanism to structure a large document (such as a book)
into a main file and several child files (containing the chapters)
using the |\include| command.
This mechanism is beneficial for documents
which span hundreds of pages in order to
make the source file(s) more manageable.
Moreover, compilation can be restricted to
selected child files by means of the |\includeonly| command.
The latter feature can be used to reduce the compilation time while editing
(this was significantly more useful in the earlier days of \LaTeX{})
or to generate a smaller document which is easier to navigate.
Another application of |\includeonly| is to generate
documents consisting of selected parts of the complete document.

However, there are a few drawbacks of the plain |\include| mechanism:
\begin{itemize}
\item
The child files cannot be compiled on their own,
they can only be compiled via the main file.
A naive editing environment
(such as a text editor with an option
to have the current file processed by \LaTeX)
may require one to switch to the main file before compiling;
attempting to compile the child file produces errors.
\item
The main file must be modified (each time)
to adjust the |\includeonly| command
to the present needs. This easily leaves the main file in a messy state.
\item
The generated document will always carry the filename
of the main document. This is inconvenient if
several child files are to be compiled and
to be kept for distribution.
\end{itemize}

The present package provides a simple interface
to make child files individually compilable by \LaTeX{}.
Compiling a child file then has the same effect as compiling
the main file with an |\includeonly| command
to select the appropriate child.
Moreover the generated document will carry the name of the child
rather than the main file.
This resolves all three above issues.

This feature is meant to make the editing of books,
thesis documents and lecture notes somewhat more convenient.
However, the package can also be used efficiently for
composing a series of documents (such as exercise sheets)
which are typically distributed individually.
It then assists the author in generating the individual documents
(potentially in different versions)
as well as a document containing the collected series.
Another application is in developing style files
or other kinds of included material
where compilation of the style file could redirect
to a sample or test file.

%%%%%%%%%%%%%%%%%%%%%%%%%%%%%%%%%%%%%%%%%%%%%%%%%%%%%%%%%%%%%%%%%%%%%%%%%%%%%%%%
%%%%%%%%%%%%%%%%%%%%%%%%%%%%%%%%%%%%%%%%%%%%%%%%%%%%%%%%%%%%%%%%%%%%%%%%%%%%%%%%
\section{Usage}

First of all, the package \textsf{childdoc} is \emph{not} a standard
\LaTeXe{} |.sty| style file! Therefore it needs to be invoked in
a non-standard way.

%%%%%%%%%%%%%%%%%%%%%%%%%%%%%%%%%%%%%%%%%%%%%%%%%%%%%%%%%%%%%%%%%%%%%%%%%%%%%%%%
\subsection{Included Files}
\label{sec:include}

%%%%%%%%%%%%%%%%%%%%%%%%%%%%%%%%%%%%%%%%
\DescribeMacro{\childdocmain}
To use the package, add the commands
\begin{center}
\begin{tabular}{l}
|\input{childdoc.def}|\\
|\childdocmain{}|\\
\end{tabular}
\end{center}
at the very top of the main \LaTeX{} file,
in particular \emph{before} the |\documentclass| statement!
The argument of |\childdocmain| should be left empty
(but it must be present).

%%%%%%%%%%%%%%%%%%%%%%%%%%%%%%%%%%%%%%%%
\DescribeMacro{\childdocof}
Furthermore, add the commands
\begin{center}
\begin{tabular}{l}
|\input{childdoc.def}|\\
|\childdocof{|\textit{main}|}|\\
\end{tabular}
\end{center}
at the top of every child file \textit{child}
which is included by |\include{|\textit{child}|}|
from within the main file
(or at least for those files to be compiled individually).
The argument \textit{main} must be the filename of the main file.

There are a couple of
considerations in setting up the main and child documents:

%%%%%%%%%%%%%%%%%%%%%%%%%%%%%%%%%%%%%%%%
\paragraph{Restrictions.}

Please note the following restrictions:
\begin{itemize}
\item
|\childdocmain| must be called with one argument \textit{main}
to ensure compatibility with earlier version of the package.
It must either be empty (|\childdocmain{}|)
or precisely match the filename of the main file in which it is specified.
See \secref{sec:detection} for further information.
\item
The filename \textit{main} must be specified without the |.tex| extension.
\item
The filename \textit{main} is case sensitive
(even in case-insensitive file systems)
due to internal string comparison.
\item
The argument \textit{main} should be fully expanded, it cannot be a macro.
\item
Subdirectories and special characters should be avoided in filenames.
\item
The command |\childdocmain{|\textit{main}|}| must be followed by a whitespace.
It should not be followed immediately by another command
or by a comment mark `|%|'.
This is because the \TeX{} parser reads the token immediately following
the argument of |\childdocmain| and puts it
at the beginning of every child section;
however, a white\-space is ignored.
\end{itemize}

%%%%%%%%%%%%%%%%%%%%%%%%%%%%%%%%%%%%%%%%
\paragraph{Content of Main File.}

It is advisable to place all content in the child files included by |\include|.
Any output contained in the main file will appear in all child documents
unless suppressed manually;
it cannot be suppressed automatically by the |\includeonly| directive
and thus should normally be avoided.
A method to include some content in the main file
by means of conditional processing is described in \secref{sec:conditional}.

%%%%%%%%%%%%%%%%%%%%%%%%%%%%%%%%%%%%%%%%
\paragraph{Page Numbering.}

When only a part of the document is compiled,
the appropriate numbering of pages
(as well as other status parameters)
is determined from the |.aux| files.
The latter contain information from previous passes.
However this information needs to propagate through
all intermediate child documents.
Therefore the page numbering in child documents may well
be inconsistent until the complete document is compiled at least once.

A useful (if unconventional) way to always ensure a consistent
page numbering is to restart the numbering in each child document
and denote the pages by `\textit{child}|.|\textit{page}'
where \textit{child} represents the chapter/section number of the child file.
This can be achieved by the command
|\numberwithin{page}{|\textit{child}|}|
of the \textsf{amsmath} package
where \textit{child} can be |chapter| or |section|
depending on the chosen structuring.
Alternatively, one can modify the macro |\thepage| appropriately
and reset the counter |page| at the start of each child file.

%%%%%%%%%%%%%%%%%%%%%%%%%%%%%%%%%%%%%%%%%%%%%%%%%%%%%%%%%%%%%%%%%%%%%%%%%%%%%%%%
\subsection{Conditional Processing}
\label{sec:conditional}

The package provides a mechanism to compile different versions
of a document. To customise the versions further some conditional processing
can come in handy to distinguish which version is being compiled.
The package provides two macros to describe the compilation context:

%%%%%%%%%%%%%%%%%%%%%%%%%%%%%%%%%%%%%%%%
\DescribeMacro{\ifchilddoc}
The conditional |\ifchilddoc| distinguishes between the compilation of
child documents and the main document:
%
\begin{center}
|\ifchilddoc |\textit{child-code}| |[|\||else |\textit{main-code}]| \||fi|
\end{center}

%%%%%%%%%%%%%%%%%%%%%%%%%%%%%%%%%%%%%%%%
\DescribeMacro{\childdocname}
\DescribeMacro{\childdocjob}
The macro |\childdocname| contains the filename (without extension)
of the main or child file being processed.
Note that |\childdocjob| will always contain the name of the main file.

%%%%%%%%%%%%%%%%%%%%%%%%%%%%%%%%%%%%%%%%
\paragraph{Title Page.}

Conditional processing can be used to include a title or banner page
in the main document when proper precautions are taken.
Importantly, the code in the main file should ensure that the page counter
(as well as other status parameters which are stored in the |.aux| files)
takes the same value after the conditional processing.
Otherwise the page numbers may take divergent values
depending on which part is compiled.

For example, a title page could be declared by:
%
\begin{center}
\begin{tabular}{l}
|\ifchilddoc\||else|\\
|\addtocounter{page}{-1}|\\
\textit{code for title page}\\
|\newpage|\\
|\||fi|
\end{tabular}
\end{center}
%
A banner page for the child documents can be generated by:
%
\begin{center}
\begin{tabular}{l}
|\ifchilddoc|\\
|\addtocounter{page}{-1}|\\
\textit{code for banner page}\\
|\newpage|\\
|\||fi|
\end{tabular}
\end{center}
%
Here one could write a message such as:
\begin{center}
|This is the part \childdocname{} of \childdocjob{}.|
\end{center}

%%%%%%%%%%%%%%%%%%%%%%%%%%%%%%%%%%%%%%%%%%%%%%%%%%%%%%%%%%%%%%%%%%%%%%%%%%%%%%%%
\subsection{Flags}
\label{sec:flags}

The package makes it easy to generate different versions
of the main or child documents.
To this end compilation flags can be defined
and assigned different default values.
They will be particularly useful in conjunction
with the forwarding mechanism described in \secref{sec:forward}.

For example, it may be useful to have a flag |\version|
which can be set to |draft| or |final|.
The document source will contain some conditional code
depending on the value of |\version|.
Suppose further, the flag should default to |final| for the main file
and to |draft| for child files
which is a natural assignment for editing the document.
This is achieved by placing the following code
in the preamble of the main document
(below the |\childdocmain| directive):
%
\begin{center}
\begin{tabular}{l}
|\ifchilddoc|\\
|\providecommand{\version}{draft}|\\
|\||else|\\
|\providecommand{\version}{final}|\\
|\||fi|
\end{tabular}
\end{center}
%
The definition by |\providecommand| makes sure
that previous definitions are not overwritten.
Further statements |\providecommand{\version}{...}|
can thus be added before the above code to override it.

For the main file, one might add a line
(between |\childdocmain| and the above block)
%
\begin{center}
|%\ifchilddoc\||else\providecommand{\version}{draft}\||fi|
\end{center}
%
which can be uncommented to produce a draft version.
Likewise one can add a line to the very top of a child file
(above the |\childdocof{|\textit{main}|}| directive)
%
\begin{center}
|%\providecommand{\version}{final}|
\end{center}
%
which can be uncommented to produce the final version of this child document.

%%%%%%%%%%%%%%%%%%%%%%%%%%%%%%%%%%%%%%%%%%%%%%%%%%%%%%%%%%%%%%%%%%%%%%%%%%%%%%%%
\subsection{Forwarding}
\label{sec:forward}

Different versions of the main or child documents
using compilation flags as described in \secref{sec:flags}
can be (permanently) stored in different files
for convenient compilation, viewing and distribution.
To this end, the package defines a command
to pass on compilation to a different file:

%%%%%%%%%%%%%%%%%%%%%%%%%%%%%%%%%%%%%%%%
\DescribeMacro{\childdocforward}
The command |\childdocforward| redirects processing to
another source file:
%
\begin{center}
\begin{tabular}{l}
|\input{childdoc.def}|\\
|\childdocforward[|\textit{main}|]{|\textit{dest}|}|\\
\end{tabular}
\end{center}
%
The argument \textit{dest} is the destination file
(without extension).
It should be the main file or one of the child files.
Note that further \textsf{childdoc} directives
such as |\childdocof| and |\childdocforward|
in the indicated file will be processed in this form.
The optional argument \textit{main}
passes on directly to the main file \textit{main}
while pretending to compile the child \textit{dest}.
This form behaves as if \textit{dest}
issues |\childdocof{|\textit{main}|}| right away,
and no further \textsf{childdoc} directives will be processed.

%%%%%%%%%%%%%%%%%%%%%%%%%%%%%%%%%%%%%%%%
\DescribeMacro{\...prefix}
In the alternative form |\childdocforwardprefix|,
%
\begin{center}
\begin{tabular}{l}
|\input{childdoc.def}|\\
|\childdocforwardprefix[|\textit{main}|]{|\textit{prefix}|}{|\textit{dest}|}|
\end{tabular}
\end{center}
%
the destination file is determined by a pattern
depending on the current file:
To make this work, the current file must be called
`{\textit{prefix}\hspace{0.2em}\textit{suffix}}'
with \textit{prefix} matching precisely the argument.
Processing is then passed on to the file
`{\textit{dest}\hspace{0.2em}\textit{suffix}}'.
Surely, the same effect is achieved by
directly specifying the
argument `{\textit{dest}\hspace{0.2em}\textit{suffix}}'
in the first form.
However, that requires to set up a different file
for each child. With the alternative form of the command
all these files can have exactly the same content
which simplifies setting them up and maintaining them.

For example, the following file |draft.tex|
with a compilation flag |\version| as described in \secref{sec:flags}
compiles the main document as a draft:
%
\begin{center}
\begin{tabular}{l}
|\def\version{draft}|\\
|\input{childdoc.def}|\\
|\childdocforward{|\textit{main}|}|
\end{tabular}
\end{center}
%
Likewise, the following files |final|\textit{nn}|.tex|
compile the final version of the child document
|child|\textit{nn}|.tex|:
%
\begin{center}
\begin{tabular}{l}
|\def\version{final}|\\
|\input{childdoc.def}|\\
|\childdocforwardprefix{final}{child}|
\end{tabular}
\end{center}
%

Note that when several versions of a main file and/or of each child file
are to be generated, it may be convenient to set up a |Makefile| or
shell script to automatise the process.

%%%%%%%%%%%%%%%%%%%%%%%%%%%%%%%%%%%%%%%%%%%%%%%%%%%%%%%%%%%%%%%%%%%%%%%%%%%%%%%%
\subsection{Command Line Processing}
\label{sec:commandline}

The effect of redirection files can also be achieved by invoking
the \LaTeX{} compiler with a more elaborate command line.
Most conveniently this should be done as part
of a shell script or a |Makefile|.

When using \textsf{childdoc} in the main file, the following
command lines effectively perform a redirection
(note that depending on the shell being used,
backslashes may have to be doubled: `|\|' $\to$ `|\\|'):
%
\begin{center}
|... -jobname "|\textit{target}|" |\\|"|[\textit{flags}]%
|\input{childdoc.def}\childdocforward[|\textit{main}|]{|\textit{dest}|}"|
\end{center}
%
Here \textit{target} is the name of the output file,
\textit{main} is the name of the main file
and \textit{dest} is the name of the main or child file to be processed
(all filenames without extensions).
The optional argument \textit{main} can be omitted
if \textit{main} matches \textit{dest}.
Optionally, compilation \textit{flags} can be defined via |\def| commands.
This command line makes the \TeX{} engine believe
it is compiling the file \textit{target}
whose content is specified as the latter parameter.
The provided code then forwards the processing to
\textit{main} or \textit{dest} as described in \secref{sec:forward}.

%%%%%%%%%%%%%%%%%%%%%%%%%%%%%%%%%%%%%%%%%%%%%%%%%%%%%%%%%%%%%%%%%%%%%%%%%%%%%%%%
\subsection{Include by Input}
\label{sec:input}

Including child documents by |\include| has some restrictions by design.
Most notably, the content of a child document always occupies
its own set of pages; pages cannot be shared between child documents.
Usually, this behaviour makes perfect sense
because each child document contain an essential part of the document.
However, in some situations it may be desirable to compose
a document from a collection of parts
without having mandatory page breaks between then.
For this case, the package
provides a mechanism to include parts
by |\input| which can also be processed individually.
However, by construction this mechanism
requires manual handling of the content to be output.

%%%%%%%%%%%%%%%%%%%%%%%%%%%%%%%%%%%%%%%%
\DescribeMacro{\ifchilddocmanual}
The main file should be prepared as usual, see \secref{sec:include}.
However, the document body must make a distinction
between processing of an individual part and of the main document, e.g.:
%
\begin{center}
\begin{tabular}{l}
|\ifchilddocmanual|\\
|\input{\childdocname}|\\
|\||else|\\
\textit{document body with }|\input{|\textit{part}|}|\\
|\||fi|
\end{tabular}
\end{center}
%
The conditional |\ifchilddocmanual| is true whenever
a part to be included by |\input| is being compiled,
and the name of the part is stored in |\childdocname|.

%%%%%%%%%%%%%%%%%%%%%%%%%%%%%%%%%%%%%%%%
\DescribeMacro{\childdocby}
Each part to be included by |\input| should start with:
%
\begin{center}
\begin{tabular}{l}
|\input{childdoc.def}|\\
|\childdocby{|\textit{main}|}|\\
\end{tabular}
\end{center}
%
The directive |\childdocby| is similar to |\childdocof|
described in \secref{sec:include},
but the subsequent selection of content must be done manually.
To that end, both |\ifchilddoc| and |\ifchilddocmanual|
will be true upon processing of a part,
and the name of the part is stored in |\childdocname|.
Note that |\jobname| will be set to the filename of the current part
so that each part receives an individual |.aux| file
that does not interfere with the |.aux| file(s) of the main document.
This behaviour can be altered by the alternative form
|\childdocby[*]{|\textit{main}|}| (with a non-empty optional argument)
which uses the |.aux| file of the main document
by setting |\jobname| to \textit{main}.

%%%%%%%%%%%%%%%%%%%%%%%%%%%%%%%%%%%%%%%%%%%%%%%%%%%%%%%%%%%%%%%%%%%%%%%%%%%%%%%%
\subsection{Driver Development}
\label{sec:driver}

The \textsf{childdoc} mechanism can also be use for the development
of definition files such as \LaTeX{} styles or classes.
This case differs from the above setup with multiple parts
included by |\include| in that no |\includeonly| should be invoked.
This can be achieved by starting the include file
(before |\ProvidesPackage|) with:
%
\begin{center}
\begin{tabular}{l}
|\input{childdoc.def}|\\
|\childdocforward{|\textit{main}|}|\\
\end{tabular}
\end{center}
%
or alternatively with:
%
\begin{center}
\begin{tabular}{l}
|\input{childdoc.def}|\\
|\childdocby{|\textit{main}|}|\\
\end{tabular}
\end{center}
%
Both forms have slightly different effects as described above.
The main file is prepared as usual, see \secref{sec:include}.

%%%%%%%%%%%%%%%%%%%%%%%%%%%%%%%%%%%%%%%%%%%%%%%%%%%%%%%%%%%%%%%%%%%%%%%%%%%%%%%%
\subsection{Legacy Detection}
\label{sec:detection}

The directive |\childdocmain| in the main file can detect
whether the complete document or merely a child is to be compiled
even without using the directive |\childdocof|.
This method is deprecated because it is less robust
and there is no compelling reason to use it;
it is merely provided for backward compatibility
and it may be removed in future versions.

If the detection mechanism is to be used,
it is mandatory to correctly specify
the filename of the main file as the argument of |\childdocmain|:
%
\begin{center}
\begin{tabular}{l}
|\input{childdoc.def}|\\
|\childdocmain{|\textit{main}|}|\\
\end{tabular}
\end{center}
%
If |\jobname| does not match the argument \textit{main} of |\childdocmain|,
it is assumed that |\jobname| points to the child file to be compiled.
When using |\childdocmain| with the main file specified as argument,
it suffices to start a child file
with just |\input{|\textit{main}|}|
without loading of the package and using |\childdocof|.
If instead all processing is done
with the appropriate \textsf{childdoc} directives,
the argument of \textit{main} of |\childdocmain| can be empty.

An alternative version of the command line processing described
in \secref{sec:commandline} using the detection mechanism reads:
%
\begin{center}
|... -jobname "|\textit{target}|" "|[\textit{flags}]%
[|\def\jobname{|\textit{dest}|}|]|\input{|\textit{main}|}"|
\end{center}

%%%%%%%%%%%%%%%%%%%%%%%%%%%%%%%%%%%%%%%%%%%%%%%%%%%%%%%%%%%%%%%%%%%%%%%%%%%%%%%%
\subsection{Manual Code}
\label{sec:manual}

In case one cannot be certain whether the definitions file |childdoc.def|
is installed on the target \TeX{} distribution
and one prefers not to ship it,
it is conceivable to paste a few relevant commands into the sources.

To that end, drop all statements |\input{childdoc.def}|
and perform the replacements as outlined below.
Instead of |\childdocmain{|\textit{main}|}| add the following code
to the top of the main file:
%
\begin{center}
\begin{tabular}{l}
|\||ifdefined\childdocname\endinput\||fi\newif\ifchilddoc|\\
|\edef\childdocname{\scantokens\expandafter{\jobname\noexpand}}|\\
|\def\childdocmain{|\textit{main}|}\||ifx\childdocmain\childdocname\||else|\\
|\childdoctrue\includeonly{\childdocname}\let\jobname\childdocmain\||fi|\\
\end{tabular}
\end{center}
%
Instead of |\childdocof{|\textit{main}|}| just include the main file
at the top of each child file:
%
\begin{center}
|\input{|\textit{main}|}|
\end{center}
%
A simple redirection |\childdocforward{|\textit{dest}|}| is achieved by:
%
\begin{center}
|\def\jobname{|\textit{dest}|}\input{\jobname}|
\end{center}
%
The redirection with prefix
|\childdocforwardprefix[|\textit{prefix}|]{|\textit{dest}|}|
is accomplished by:
%
\begin{center}
\begin{tabular}{l}
|{\edef\jobname{\scantokens\expandafter{\jobname\noexpand}}|\\
|\def\redirectjob |\textit{prefix}|#1~~~{\gdef\jobname{|\textit{dest}|#1}}|\\
|\expandafter\redirectjob\jobname~~~}\input{\jobname}|
\end{tabular}
\end{center}

In an alternative approach,
child documents can be compiled by a specific command line
without additional code or specific definitions:
%
\begin{center}
|... -jobname "|\textit{target}|" "|[\textit{flags}]%
|\includeonly{|\textit{dest}|}\input{|\textit{main}|}"|
\end{center}
%

%%%%%%%%%%%%%%%%%%%%%%%%%%%%%%%%%%%%%%%%%%%%%%%%%%%%%%%%%%%%%%%%%%%%%%%%%%%%%%%%
%%%%%%%%%%%%%%%%%%%%%%%%%%%%%%%%%%%%%%%%%%%%%%%%%%%%%%%%%%%%%%%%%%%%%%%%%%%%%%%%
\section{Information}

%%%%%%%%%%%%%%%%%%%%%%%%%%%%%%%%%%%%%%%%%%%%%%%%%%%%%%%%%%%%%%%%%%%%%%%%%%%%%%%%
\subsection{Copyright}

Copyright \copyright{} 2017--2018 Niklas Beisert

This work may be distributed and/or modified under the
conditions of the \LaTeX{} Project Public License, either version 1.3
of this license or (at your option) any later version.
The latest version of this license is in
  \url{http://www.latex-project.org/lppl.txt}
and version 1.3 or later is part of all distributions of \LaTeX{}
version 2005/12/01 or later.

This work has the LPPL maintenance status `maintained'.

The Current Maintainer of this work is Niklas Beisert.

This work consists of the files |README.txt|, |childdoc.ins| and |childdoc.dtx|
as well as the derived files |childdoc.def|, |cdocsamp.tex|
with |cdocsch1.tex|, |cdocsch2.tex|, |cdocspt3.tex|, |cdocspt4.tex|,
|cdocsdrf.tex|, |cdocsfn1.tex|, |cdocsfn2.tex|
as well as |childdoc.pdf|.

%%%%%%%%%%%%%%%%%%%%%%%%%%%%%%%%%%%%%%%%%%%%%%%%%%%%%%%%%%%%%%%%%%%%%%%%%%%%%%%%
\subsection{Files and Installation}

The package consists of the files:
%
\begin{center}
\begin{tabular}{ll}
    |README.txt|   & readme file \\
    |childdoc.ins| & installation file \\
    |childdoc.dtx| & source file \\
    |childdoc.def| & definition file \\
    |cdocsamp.tex| & sample main file \\
    |cdocsch1.tex| & sample include file \\
    |cdocsch2.tex| & sample include file \\
    |cdocspt3.tex| & sample part file \\
    |cdocspt4.tex| & sample part file \\
    |cdocsdrf.tex| & sample redirection file \\
    |cdocsfn1.tex| & sample redirection file \\
    |cdocsfn2.tex| & sample redirection file \\
    |childdoc.pdf| & manual
\end{tabular}
\end{center}
%
The distribution consists of the files
|README.txt|, |childdoc.ins| and |childdoc.dtx|.
%
\begin{itemize}
\item
Run (pdf)\LaTeX{} on |childdoc.dtx|
to compile the manual |childdoc.pdf| (this file).
\item
Run \LaTeX{} on |childdoc.ins| to create the definitions file |childdoc.def|
and the sample |cdocsamp.tex| with include files
|cdocsch1.tex|, |cdocsch2.tex|, |cdocspt3.tex|, |cdocspt4.tex|,
|cdocsdrf.tex|, |cdocsfn1.tex|, |cdocsfn2.tex|.
Then copy the file |childdoc.def| to an appropriate directory of your \LaTeX{}
distribution, e.g.\ \textit{texmf-root}|/tex/latex/childdoc|.
\end{itemize}

%%%%%%%%%%%%%%%%%%%%%%%%%%%%%%%%%%%%%%%%%%%%%%%%%%%%%%%%%%%%%%%%%%%%%%%%%%%%%%%%
\subsection{Related CTAN Packages}

There are several other packages which offer a similar functionality:
%
\begin{itemize}
\item
The packages
\href{http://ctan.org/pkg/docmute}{\textsf{docmute}},
\href{http://ctan.org/pkg/includex}{\textsf{includex}} and
\href{http://ctan.org/pkg/standalone}{\textsf{standalone}}
provide commands to include only the document body of
a child file thus allowing both files to be compiled individually.
\item
The packages \href{http://ctan.org/pkg/subdocs}{\textsf{subdocs}}
and \href{http://ctan.org/pkg/subfiles}{\textsf{subfiles}}
provide structures in which the main and child documents can be
encapsulated and allowing them to be compiled individually.
The inclusion mechanism is different from the conventional |\include|.
\item
The package \href{http://ctan.org/pkg/combine}{\textsf{combine}}
is an elaborate solution to combine several documents into one.
\end{itemize}
%
See also the CTAN topic \href{http://ctan.org/topic/subdocs}{\textsf{subdocs}}
for further related packages.
The present package differs from the above solutions in that
a document structure constructed with the conventional |\include| mechanism
just needs two extra commands at the top of every file
such that all constituent files can be compiled individually.

%%%%%%%%%%%%%%%%%%%%%%%%%%%%%%%%%%%%%%%%%%%%%%%%%%%%%%%%%%%%%%%%%%%%%%%%%%%%%%%%
%\subsection{Feature Suggestions}
%
%The following is a list of features which may be useful for future
%versions of this package:
%%
%\begin{itemize}
%\item
%\ldots
%\end{itemize}

%%%%%%%%%%%%%%%%%%%%%%%%%%%%%%%%%%%%%%%%%%%%%%%%%%%%%%%%%%%%%%%%%%%%%%%%%%%%%%%%
\subsection{Revision History}

%%%%%%%%%%%%%%%%%%%%%%%%%%%%%%%%%%%%%%%%
\paragraph{v2.0:} 2018/12/30

\begin{itemize}
\item
immediate forward processing
\item
added |\childdocby| mechanism
\item
manual restructured
\end{itemize}

%%%%%%%%%%%%%%%%%%%%%%%%%%%%%%%%%%%%%%%%
\paragraph{v1.6:} 2018/01/17

\begin{itemize}
\item
application for development of include files
\item
corrections to manual
\end{itemize}

%%%%%%%%%%%%%%%%%%%%%%%%%%%%%%%%%%%%%%%%
\paragraph{v1.5:} 2017/05/21

\begin{itemize}
\item
more complete structuring introduced
\item
|\childdocof| introduced
\item
|\childdoc| renamed to |\childdocmain|
\item
|\childredirect| renamed to |\childdocforward| and |\childdocforwardprefix|
and functionality expanded
\end{itemize}

%%%%%%%%%%%%%%%%%%%%%%%%%%%%%%%%%%%%%%%%
\paragraph{v1.0:} 2017/04/27

\begin{itemize}
\item
manual and install package
\item
first version published on CTAN
\end{itemize}

%%%%%%%%%%%%%%%%%%%%%%%%%%%%%%%%%%%%%%%%
\paragraph{v0.6:} 2017/04/26

\begin{itemize}
\item
redirection mechanism added
\end{itemize}

%%%%%%%%%%%%%%%%%%%%%%%%%%%%%%%%%%%%%%%%
\paragraph{v0.5:} 2017/04/26

\begin{itemize}
\item
functionality in definition file
\end{itemize}


%%%%%%%%%%%%%%%%%%%%%%%%%%%%%%%%%%%%%%%%%%%%%%%%%%%%%%%%%%%%%%%%%%%%%%%%%%%%%%%%
%%%%%%%%%%%%%%%%%%%%%%%%%%%%%%%%%%%%%%%%%%%%%%%%%%%%%%%%%%%%%%%%%%%%%%%%%%%%%%%%
%%%%%%%%%%%%%%%%%%%%%%%%%%%%%%%%%%%%%%%%%%%%%%%%%%%%%%%%%%%%%%%%%%%%%%%%%%%%%%%%
\appendix

\settowidth\MacroIndent{\rmfamily\scriptsize 000\ }

 \DocInput{childdoc.dtx}

\end{document}
%</driver>
% \fi
%
% %%%%%%%%%%%%%%%%%%%%%%%%%%%%%%%%%%%%%%%%%%%%%%%%%%%%%%%%%%%%%%%%%%%%%%%%%%%%%%
% %%%%%%%%%%%%%%%%%%%%%%%%%%%%%%%%%%%%%%%%%%%%%%%%%%%%%%%%%%%%%%%%%%%%%%%%%%%%%%
% \section{Sample}
%\iffalse
%<*samplemain>
%\fi
%
% The following presents a sample document
% with two chapters, two parts, a title page,
% a compile flag as well as three forwarding files to set the flag.
% It consists of eight |.tex| files:
% \begin{center}
% \begin{tabular}{ll}
% |cdocsamp.tex|&main file\\
% |cdocsch1.tex|&include file for chapter 1\\
% |cdocsch2.tex|&include file for chapter 2\\
% |cdocspt3.tex|&include file for part 3\\
% |cdocspt4.tex|&include file for part 4\\
% |cdocsdrf.tex|&forwarding file for main file in draft mode\\
% |cdocsfi1.tex|&forwarding file for final version of chapter 1\\
% |cdocsfi2.tex|&forwarding file for final version of chapter 2\\
% \end{tabular}
% \end{center}
% Each of the eight files can be compiled directly by the \LaTeX{} compiler.
%
% %%%%%%%%%%%%%%%%%%%%%%%%%%%%%%%%%%%%%%
% \paragraph{Main File.}
%
% The main file is called |cdocsamp.tex|.
%
% Load the \textsf{childdoc} definitions and
% declare the filename for the main document:
%    \begin{macrocode}
\input{childdoc.def}
\childdocmain{}
%    \end{macrocode}

% Optional override for |\version| flag:
%    \begin{macrocode}
%%\ifchilddoc\else\providecommand{\version}{draft}\fi
%    \end{macrocode}

% Define the default values for the |\version| flag
% (|final| for the main file and |draft| for childs):
%    \begin{macrocode}
\ifchilddoc
\providecommand{\version}{draft}
\else
\providecommand{\version}{final}
\fi
%    \end{macrocode}

% Load the standard document class:
%    \begin{macrocode}
\documentclass[12pt]{article}
%    \end{macrocode}

% Start the document body:
%    \begin{macrocode}
\begin{document}
%    \end{macrocode}

% Declare a title page.
% Print title, part of document being processed and version flag:
%    \begin{macrocode}
\addtocounter{page}{-1}
\begin{center}
{\LARGE\bfseries{}childdoc example\par}
\vspace{1cm}
\ifchilddoc
\ifchilddocmanual part\else chapter\fi:
`\childdocname' of `\childdocjob'\par
\else
main document: `\childdocjob'\par
\fi
version: \version\par
\end{center}
\newpage
%    \end{macrocode}

% Manually include selected file,
% otherwise process as usual:
%    \begin{macrocode}
\ifchilddocmanual
\section*{part `\childdocname'}
\input{\childdocname}
\else
%    \end{macrocode}

% Include the two chapters:
%    \begin{macrocode}
\include{cdocsch1}
\include{cdocsch2}
%    \end{macrocode}

% Include the two parts unless only chapters should be displayed:
%    \begin{macrocode}
\ifchilddoc\else
\section{part three}
\input{cdocspt3}
\section{part four}
\input{cdocspt4}
\fi
%    \end{macrocode}

% Process as usual until here:
%    \begin{macrocode}
\fi
%    \end{macrocode}

% End of document body:
%    \begin{macrocode}
\end{document}
%    \end{macrocode}
%\iffalse
%</samplemain>
%\fi
%
% %%%%%%%%%%%%%%%%%%%%%%%%%%%%%%%%%%%%%%
% \paragraph{Chapter Include Files.}
%
% The include files are called |cdocsch1.tex| and |cdocsch2.tex|.
%
%\iffalse
%<*samplechap1|samplechap2>
%\fi

% Optional override for |\version| flag:
%    \begin{macrocode}
%%\providecommand{\version}{final}
%    \end{macrocode}

% Include the main document:
%    \begin{macrocode}
\input{childdoc.def}
\childdocof{cdocsamp}
%    \end{macrocode}

%\iffalse
%</samplechap1|samplechap2>
%\fi
%
%\iffalse
%<*samplechap1>
%\fi
% Some text for chapter 1:
%    \begin{macrocode}
\section{one}
some text in chapter one
%    \end{macrocode}

%\iffalse
%</samplechap1>
%\fi
% Some text for chapter 2:
%\iffalse
%<*samplechap2>
%\fi
%    \begin{macrocode}
\section{two}
more text in chapter two
%    \end{macrocode}

%\iffalse
%</samplechap2>
%\fi
%
% %%%%%%%%%%%%%%%%%%%%%%%%%%%%%%%%%%%%%%
% \paragraph{Part Include Files.}
%
% The include files are called |cdocspt3.tex| and |cdocspt4.tex|.
%
%\iffalse
%<*samplepart3|samplepart4>
%\fi

% Optional override for |\version| flag:
%    \begin{macrocode}
%%\providecommand{\version}{final}
%    \end{macrocode}

% Include the main document:
%    \begin{macrocode}
\input{childdoc.def}
\childdocby{cdocsamp}
%    \end{macrocode}

%\iffalse
%</samplepart3|samplepart4>
%\fi
%
%\iffalse
%<*samplepart3>
%\fi
% Some text for part 3:
%    \begin{macrocode}
some text in part three
%    \end{macrocode}

%\iffalse
%</samplepart3>
%\fi
% Some text for part 4:
%\iffalse
%<*samplepart4>
%\fi
%    \begin{macrocode}
more text in part four
%    \end{macrocode}

%\iffalse
%</samplepart4>
%\fi
%
% %%%%%%%%%%%%%%%%%%%%%%%%%%%%%%%%%%%%%%
% \paragraph{Forwarding for a Complete Draft.}
%
% The following forwarding file |cdocsdrf.tex|
% compiles the main document in draft mode:
%\iffalse
%<*sampledraft>
%\fi
%    \begin{macrocode}
\def\version{draft}
\input{childdoc.def}
\childdocforward{cdocsamp}
%    \end{macrocode}

%\iffalse
%</sampledraft>
%\fi
%
% %%%%%%%%%%%%%%%%%%%%%%%%%%%%%%%%%%%%%%
% \paragraph{Forwarding for Final Version of the Chapters.}
%
% The following forwarding files |cdocsfn1.tex| and |cdocsfn2.tex|
% (with identical content)
% compile the final versions of the child documents
% |cdocsch1.tex| and |cdocsch2.tex|, respectively:
%\iffalse
%<*samplefinal>
%\fi
%    \begin{macrocode}
\def\version{final}
\input{childdoc.def}
\childdocforwardprefix[cdocsamp]{cdocsfn}{cdocsch}
%    \end{macrocode}

%\iffalse
%</samplefinal>
%\fi
%
% %%%%%%%%%%%%%%%%%%%%%%%%%%%%%%%%%%%%%%
% \paragraph{Command Line Processing.}
%
% The following three command lines generate the output files
% |cdocscld|, |cdocscl1| and |cdocscl2|
% which should be identical to
% |cdocsdrf|, |cdocsch1| and |cdocsfn2|, respectively:
% \begin{center}
% \begin{tabular}{l}
% |latex -jobname cdocscld \|\\
% |  "\def\version{draft}\input{childdoc.def}\childdocforward{cdocsamp}"|\\
% |latex -jobname cdocscl1 \|\\
% |  "\input{childdoc.def}\childdocforward[cdocsamp]{cdocsch1}"|\\
% |latex -jobname cdocscl2 \|\\
% |  "\def\version{final}\input{childdoc.def}\childdocforward{cdocsch2}"|
% \end{tabular}
% \end{center}
% Note that the trailing backslash on each first line
% merely continues the input to the second line
% (for convenient cut ant paste).
% Furthermore, the command |latex| can be replaced by any
% of its alternative versions such as |pdflatex|.
%
% %%%%%%%%%%%%%%%%%%%%%%%%%%%%%%%%%%%%%%%%%%%%%%%%%%%%%%%%%%%%%%%%%%%%%%%%%%%%%%
% %%%%%%%%%%%%%%%%%%%%%%%%%%%%%%%%%%%%%%%%%%%%%%%%%%%%%%%%%%%%%%%%%%%%%%%%%%%%%%
% \section{Implementation}
%\iffalse
%<*package>
%\fi
%
% This section describes the definitions file |childdoc.def|.

% The definitions cannot be loaded using |\usepackage| or |\RequirePackage|
% which has a mechanism to prevent loading a style file more than once.
% When loading the definitions by means of |\input|
% multiple instances have to be prevented manually:
%\iffalse
%This code needs to be before the `\ProvidesFile' directive
%which is defined at the beginning of this file.
%Therefore it is also placed there and commented out here.
%</package>
%<*discard>
%\fi
%    \begin{macrocode}
\ifdefined\childdocmain\endinput\fi
%    \end{macrocode}
%\iffalse
%</discard>
%<*package>
%\fi
%
% \macro{\ifchilddoc}
% \macro{\ifchilddocmanual}
% The conditional |\ifchilddoc| tells whether a
% child (true) or main (false) document is being compiled.
% The conditional |\ifchilddocmanual| tells whether
% the |\includeonly| mechanism is used (false) or
% the selection of child files must be performed manually (true).
% The definitions initialise to false:
%    \begin{macrocode}
\newif\ifchilddoc
\newif\ifchilddocmanual
%    \end{macrocode}

% \macro{\childdocname}
% \macro{\childdocjob}
% The macro |\childdocname| stores the name of the main document
% to be compiled. The macro |\childdocjob| stores the name of
% the document on which the \LaTeX{} compiler was originally invoked.
% The content of |\jobname| cannot be compared
% to filenames specified in the source due to different catcodes.
% The following code rescans |\jobname|, stores the result
% in |\childdocname| and saves a copy in |\childdocjob|:
%    \begin{macrocode}
\edef\childdocname{\scantokens\expandafter{\jobname\noexpand}}
\let\childdocjob\childdocname
%    \end{macrocode}

% \macro{\childdocdisable}
% The macro |\childdocdisable| prevents the main file
% from being processed more than once.
% At this stage, the main document command |\childdocmain|
% is assumed to be called once again where it should do nothing.
% Any subsequent call to it should prevent
% a secondary processing of the main document
% It overwrites the forwarding commands
% |\childdocof| and |\childdocforward|
% with empty macros to prevent further inclusions of the main document:
%    \begin{macrocode}
\newcommand{\childdocdisable}
{
  \renewcommand{\childdocmain}[1]{\renewcommand{\childdocmain}[1]{\endinput}}
  \renewcommand{\childdocof}[1]{}
  \renewcommand{\childdocby}[2][]{}
  \renewcommand{\childdocforward}[2][]{}
  \renewcommand{\childdocdisable}{}
}
%    \end{macrocode}

% \macro{\childdocmain}
% The macro |\childdocmain| is to be called at the top of the main file
% with nothing or the main filename (without extension) as argument.
% First, it breaks loops.
% If the argument is not empty and does not match |\childdocname|
% (which is set by the first inclusion of |childdoc.def|),
% |\ifchilddoc| is set to true, |\includeonly| is applied to the child file
% and |\jobname| is set to the main file
% (for proper handling of |.aux| files):
%    \begin{macrocode}
\newcommand{\childdocmain}[1]
{
  \childdocdisable\childdocmain{}
  \if?#1?\else
    \begingroup
      \def\childdoctmp{#1}
      \ifx\childdoctmp\childdocname
        \def\childdoctmp{}
      \else
        \def\childdoctmp
        {
          \childdoctrue
          \includeonly{\childdocname}
          \def\childdocjob{#1}
          \def\jobname{#1}
        }
      \fi
      \expandafter
    \endgroup
    \childdoctmp
  \fi
}
%    \end{macrocode}

% \macro{\childdocof}
% The command |\childdocof| redirects
% compilation to the main file |#1|.
%    \begin{macrocode}
\newcommand{\childdocof}[1]
{
  \childdocdisable
  \childdoctrue
  \includeonly{\childdocname}
  \def\jobname{#1}
  \def\childdocjob{#1}
  \input{#1}
}
%    \end{macrocode}

% \macro{\childdocby}
% The command |\childdocby| ....
%    \begin{macrocode}
\newcommand{\childdocby}[2][]
{
  \childdocdisable
  \childdoctrue
  \childdocmanualtrue
  \if?#1?\else
    \def\jobname{#2}
  \fi
  \def\childdocjob{#2}
  \input{#2}
  \endinput
}
%    \end{macrocode}

% \macro{\childdocforward}
% The command |\childdocforward| redirects
% compilation to the main file or
% (if the optional argument is given) a child file.
% Parameters are set as if the main file
% or a child file starting with |\childdocof| was compiled.
% Then compilation is handed over to the main file:
%    \begin{macrocode}
\newcommand{\childdocforward}[2][]
{
  \begingroup
    \if?#1?
      \def\childdoctmp
      {
        \def\childdocname{#2}
        \def\childdocjob{#2}
        \def\jobname{#2}
        \input{#2}
        \endinput
      }
    \else
      \def\childdoctmp
      {
        \childdocdisable
        \def\childdocname{#2}
        \childdoctrue
        \includeonly{#2}
        \def\childdocjob{#1}
        \def\jobname{#1}
        \input{#1}
        \endinput
      }
    \fi
    \expandafter
  \endgroup
  \childdoctmp
}
%    \end{macrocode}

% \macro{\childdocforwardprefix}
% The command |\childdocforwardprefix| redirects
% compilation to the main or a child file by means of a pattern.
% The prefix |#1| in the current filename is replaced by |#2|
% and the suffix of the current filename is kept
% (it is assumed that the filename does not contain the substring `|~~~|'
% which is used as a delimiter).
% Compilation is handed over to the new file by |\childdocforward|:
%    \begin{macrocode}
\newcommand{\childdocforwardprefix}[3][]
{
  \begingroup
    \def\childdocextract #2##1~~~{\def\childdoctmp{\childdocforward[#1]{#3##1}}}
    \expandafter\childdocextract\childdocname~~~
    \expandafter
  \endgroup
  \childdoctmp
}
%    \end{macrocode}

% \macro{\childdoc}
% The deprecated macro |\childdoc| is a legacy version of |\childdocmain|:
%    \begin{macrocode}
\newcommand{\childdoc}{\childdocmain}
%    \end{macrocode}

% \macro{\childdocredirect}
% The deprecated macro |\childdocredirect| is a legacy version
% of |\childdocforward| and |\childdocforwardprefix|:
%    \begin{macrocode}
\newcommand{\childdocredirect}[2][]
{
  \begingroup
    \if?#1?
      \def\childdoctmp{\childdocforward{#2}}
    \else
      \def\childdoctmp{\childdocforwardprefix{#1}{#2}}
    \fi
    \expandafter
  \endgroup
  \childdoctmp
}
%    \end{macrocode}

%\iffalse
%</package>
%\fi
%
\endinput
|\\
|\childdocby{|\textit{main}|}|\\
\end{tabular}
\end{center}
%
The directive |\childdocby| is similar to |\childdocof|
described in \secref{sec:include},
but the subsequent selection of content must be done manually.
To that end, both |\ifchilddoc| and |\ifchilddocmanual|
will be true upon processing of a part,
and the name of the part is stored in |\childdocname|.
Note that |\jobname| will be set to the filename of the current part
so that each part receives an individual |.aux| file
that does not interfere with the |.aux| file(s) of the main document.
This behaviour can be altered by the alternative form
|\childdocby[*]{|\textit{main}|}| (with a non-empty optional argument)
which uses the |.aux| file of the main document
by setting |\jobname| to \textit{main}.

%%%%%%%%%%%%%%%%%%%%%%%%%%%%%%%%%%%%%%%%%%%%%%%%%%%%%%%%%%%%%%%%%%%%%%%%%%%%%%%%
\subsection{Driver Development}
\label{sec:driver}

The \textsf{childdoc} mechanism can also be use for the development
of definition files such as \LaTeX{} styles or classes.
This case differs from the above setup with multiple parts
included by |\include| in that no |\includeonly| should be invoked.
This can be achieved by starting the include file
(before |\ProvidesPackage|) with:
%
\begin{center}
\begin{tabular}{l}
|% \iffalse
%
% childdoc.dtx Copyright (C) 2017-2018 Niklas Beisert
%
% This work may be distributed and/or modified under the
% conditions of the LaTeX Project Public License, either version 1.3
% of this license or (at your option) any later version.
% The latest version of this license is in
%   http://www.latex-project.org/lppl.txt
% and version 1.3 or later is part of all distributions of LaTeX
% version 2005/12/01 or later.
%
% This work has the LPPL maintenance status `maintained'.
%
% The Current Maintainer of this work is Niklas Beisert.
%
% This work consists of the files childdoc.dtx and childdoc.ins
% and the derived files childdoc.def and cdocsamp.tex with
% cdocsch1.tex, cdocsch2.tex, cdocsdrf.tex, cdocsfn1.tex, cdocsfn2.tex.
%
%<package>\ifdefined\childdocmain\endinput\fi
%<package>\ProvidesFile{childdoc.def}[2018/12/30 v2.0 child document driver]
%<samplemain>\ProvidesFile{cdocsamp.tex}[2018/12/30 v2.0 sample for childdoc]
%<*driver>
%\ProvidesFile{childdoc.drv}[2018/12/30 v2.0 childdoc reference manual file]
\PassOptionsToClass{10pt,a4paper}{article}
\documentclass{ltxdoc}

\usepackage[margin=35mm]{geometry}
\usepackage{hyperref}
\usepackage{hyperxmp}
\usepackage[usenames]{color}

\hypersetup{colorlinks=true}
\hypersetup{pdfstartview=FitH}
\hypersetup{pdfpagemode=UseNone}
\hypersetup{pdfsource={}}
\hypersetup{pdflang={en-UK}}
\hypersetup{pdfcopyright={Copyright 2017-2018 Niklas Beisert.
  This work may be distributed and/or modified under the
  conditions of the LaTeX Project Public License, either version 1.3
  of this license or (at your option) any later version.}}
\hypersetup{pdflicenseurl={http://www.latex-project.org/lppl.txt}}
\hypersetup{pdfcontactaddress={ETH Zurich, ITP, HIT K,
  Wolfgang-Pauli-Strasse 27}}
\hypersetup{pdfcontactpostcode={8093}}
\hypersetup{pdfcontactcity={Zurich}}
\hypersetup{pdfcontactcountry={Switzerland}}
\hypersetup{pdfcontactemail={nbeisert@itp.phys.ethz.ch}}
\hypersetup{pdfcontacturl={http://people.phys.ethz.ch/\xmptilde nbeisert/}}

\newcommand{\secref}[1]{\hyperref[#1]{section \ref*{#1}}}

\parskip1ex
\parindent0pt
\let\olditemize\itemize
\def\itemize{\olditemize\parskip0pt}

\begin{document}

\title{The \textsf{childdoc} Package}
\hypersetup{pdftitle={The childdoc Package}}
\author{Niklas Beisert\\[2ex]
  Institut f\"ur Theoretische Physik\\
  Eidgen\"ossische Technische Hochschule Z\"urich\\
  Wolfgang-Pauli-Strasse 27, 8093 Z\"urich, Switzerland\\[1ex]
  \href{mailto:nbeisert@itp.phys.ethz.ch}
  {\texttt{nbeisert@itp.phys.ethz.ch}}}
\hypersetup{pdfauthor={Niklas Beisert}}
\hypersetup{pdfsubject={Manual for the LaTeX2e Package childdoc}}
\date{30 December 2018, \textsf{v2.0}}
\maketitle

\begin{abstract}\noindent
\textsf{childdoc} is a \LaTeXe{} package
that enables the direct compilation
of document sections included by |\include|
to individual files.
\end{abstract}

\begingroup
\parskip0ex
\tableofcontents
\endgroup

%%%%%%%%%%%%%%%%%%%%%%%%%%%%%%%%%%%%%%%%%%%%%%%%%%%%%%%%%%%%%%%%%%%%%%%%%%%%%%%%
%%%%%%%%%%%%%%%%%%%%%%%%%%%%%%%%%%%%%%%%%%%%%%%%%%%%%%%%%%%%%%%%%%%%%%%%%%%%%%%%
\section{Introduction}

\LaTeX{} provides a mechanism to structure a large document (such as a book)
into a main file and several child files (containing the chapters)
using the |\include| command.
This mechanism is beneficial for documents
which span hundreds of pages in order to
make the source file(s) more manageable.
Moreover, compilation can be restricted to
selected child files by means of the |\includeonly| command.
The latter feature can be used to reduce the compilation time while editing
(this was significantly more useful in the earlier days of \LaTeX{})
or to generate a smaller document which is easier to navigate.
Another application of |\includeonly| is to generate
documents consisting of selected parts of the complete document.

However, there are a few drawbacks of the plain |\include| mechanism:
\begin{itemize}
\item
The child files cannot be compiled on their own,
they can only be compiled via the main file.
A naive editing environment
(such as a text editor with an option
to have the current file processed by \LaTeX)
may require one to switch to the main file before compiling;
attempting to compile the child file produces errors.
\item
The main file must be modified (each time)
to adjust the |\includeonly| command
to the present needs. This easily leaves the main file in a messy state.
\item
The generated document will always carry the filename
of the main document. This is inconvenient if
several child files are to be compiled and
to be kept for distribution.
\end{itemize}

The present package provides a simple interface
to make child files individually compilable by \LaTeX{}.
Compiling a child file then has the same effect as compiling
the main file with an |\includeonly| command
to select the appropriate child.
Moreover the generated document will carry the name of the child
rather than the main file.
This resolves all three above issues.

This feature is meant to make the editing of books,
thesis documents and lecture notes somewhat more convenient.
However, the package can also be used efficiently for
composing a series of documents (such as exercise sheets)
which are typically distributed individually.
It then assists the author in generating the individual documents
(potentially in different versions)
as well as a document containing the collected series.
Another application is in developing style files
or other kinds of included material
where compilation of the style file could redirect
to a sample or test file.

%%%%%%%%%%%%%%%%%%%%%%%%%%%%%%%%%%%%%%%%%%%%%%%%%%%%%%%%%%%%%%%%%%%%%%%%%%%%%%%%
%%%%%%%%%%%%%%%%%%%%%%%%%%%%%%%%%%%%%%%%%%%%%%%%%%%%%%%%%%%%%%%%%%%%%%%%%%%%%%%%
\section{Usage}

First of all, the package \textsf{childdoc} is \emph{not} a standard
\LaTeXe{} |.sty| style file! Therefore it needs to be invoked in
a non-standard way.

%%%%%%%%%%%%%%%%%%%%%%%%%%%%%%%%%%%%%%%%%%%%%%%%%%%%%%%%%%%%%%%%%%%%%%%%%%%%%%%%
\subsection{Included Files}
\label{sec:include}

%%%%%%%%%%%%%%%%%%%%%%%%%%%%%%%%%%%%%%%%
\DescribeMacro{\childdocmain}
To use the package, add the commands
\begin{center}
\begin{tabular}{l}
|\input{childdoc.def}|\\
|\childdocmain{}|\\
\end{tabular}
\end{center}
at the very top of the main \LaTeX{} file,
in particular \emph{before} the |\documentclass| statement!
The argument of |\childdocmain| should be left empty
(but it must be present).

%%%%%%%%%%%%%%%%%%%%%%%%%%%%%%%%%%%%%%%%
\DescribeMacro{\childdocof}
Furthermore, add the commands
\begin{center}
\begin{tabular}{l}
|\input{childdoc.def}|\\
|\childdocof{|\textit{main}|}|\\
\end{tabular}
\end{center}
at the top of every child file \textit{child}
which is included by |\include{|\textit{child}|}|
from within the main file
(or at least for those files to be compiled individually).
The argument \textit{main} must be the filename of the main file.

There are a couple of
considerations in setting up the main and child documents:

%%%%%%%%%%%%%%%%%%%%%%%%%%%%%%%%%%%%%%%%
\paragraph{Restrictions.}

Please note the following restrictions:
\begin{itemize}
\item
|\childdocmain| must be called with one argument \textit{main}
to ensure compatibility with earlier version of the package.
It must either be empty (|\childdocmain{}|)
or precisely match the filename of the main file in which it is specified.
See \secref{sec:detection} for further information.
\item
The filename \textit{main} must be specified without the |.tex| extension.
\item
The filename \textit{main} is case sensitive
(even in case-insensitive file systems)
due to internal string comparison.
\item
The argument \textit{main} should be fully expanded, it cannot be a macro.
\item
Subdirectories and special characters should be avoided in filenames.
\item
The command |\childdocmain{|\textit{main}|}| must be followed by a whitespace.
It should not be followed immediately by another command
or by a comment mark `|%|'.
This is because the \TeX{} parser reads the token immediately following
the argument of |\childdocmain| and puts it
at the beginning of every child section;
however, a white\-space is ignored.
\end{itemize}

%%%%%%%%%%%%%%%%%%%%%%%%%%%%%%%%%%%%%%%%
\paragraph{Content of Main File.}

It is advisable to place all content in the child files included by |\include|.
Any output contained in the main file will appear in all child documents
unless suppressed manually;
it cannot be suppressed automatically by the |\includeonly| directive
and thus should normally be avoided.
A method to include some content in the main file
by means of conditional processing is described in \secref{sec:conditional}.

%%%%%%%%%%%%%%%%%%%%%%%%%%%%%%%%%%%%%%%%
\paragraph{Page Numbering.}

When only a part of the document is compiled,
the appropriate numbering of pages
(as well as other status parameters)
is determined from the |.aux| files.
The latter contain information from previous passes.
However this information needs to propagate through
all intermediate child documents.
Therefore the page numbering in child documents may well
be inconsistent until the complete document is compiled at least once.

A useful (if unconventional) way to always ensure a consistent
page numbering is to restart the numbering in each child document
and denote the pages by `\textit{child}|.|\textit{page}'
where \textit{child} represents the chapter/section number of the child file.
This can be achieved by the command
|\numberwithin{page}{|\textit{child}|}|
of the \textsf{amsmath} package
where \textit{child} can be |chapter| or |section|
depending on the chosen structuring.
Alternatively, one can modify the macro |\thepage| appropriately
and reset the counter |page| at the start of each child file.

%%%%%%%%%%%%%%%%%%%%%%%%%%%%%%%%%%%%%%%%%%%%%%%%%%%%%%%%%%%%%%%%%%%%%%%%%%%%%%%%
\subsection{Conditional Processing}
\label{sec:conditional}

The package provides a mechanism to compile different versions
of a document. To customise the versions further some conditional processing
can come in handy to distinguish which version is being compiled.
The package provides two macros to describe the compilation context:

%%%%%%%%%%%%%%%%%%%%%%%%%%%%%%%%%%%%%%%%
\DescribeMacro{\ifchilddoc}
The conditional |\ifchilddoc| distinguishes between the compilation of
child documents and the main document:
%
\begin{center}
|\ifchilddoc |\textit{child-code}| |[|\||else |\textit{main-code}]| \||fi|
\end{center}

%%%%%%%%%%%%%%%%%%%%%%%%%%%%%%%%%%%%%%%%
\DescribeMacro{\childdocname}
\DescribeMacro{\childdocjob}
The macro |\childdocname| contains the filename (without extension)
of the main or child file being processed.
Note that |\childdocjob| will always contain the name of the main file.

%%%%%%%%%%%%%%%%%%%%%%%%%%%%%%%%%%%%%%%%
\paragraph{Title Page.}

Conditional processing can be used to include a title or banner page
in the main document when proper precautions are taken.
Importantly, the code in the main file should ensure that the page counter
(as well as other status parameters which are stored in the |.aux| files)
takes the same value after the conditional processing.
Otherwise the page numbers may take divergent values
depending on which part is compiled.

For example, a title page could be declared by:
%
\begin{center}
\begin{tabular}{l}
|\ifchilddoc\||else|\\
|\addtocounter{page}{-1}|\\
\textit{code for title page}\\
|\newpage|\\
|\||fi|
\end{tabular}
\end{center}
%
A banner page for the child documents can be generated by:
%
\begin{center}
\begin{tabular}{l}
|\ifchilddoc|\\
|\addtocounter{page}{-1}|\\
\textit{code for banner page}\\
|\newpage|\\
|\||fi|
\end{tabular}
\end{center}
%
Here one could write a message such as:
\begin{center}
|This is the part \childdocname{} of \childdocjob{}.|
\end{center}

%%%%%%%%%%%%%%%%%%%%%%%%%%%%%%%%%%%%%%%%%%%%%%%%%%%%%%%%%%%%%%%%%%%%%%%%%%%%%%%%
\subsection{Flags}
\label{sec:flags}

The package makes it easy to generate different versions
of the main or child documents.
To this end compilation flags can be defined
and assigned different default values.
They will be particularly useful in conjunction
with the forwarding mechanism described in \secref{sec:forward}.

For example, it may be useful to have a flag |\version|
which can be set to |draft| or |final|.
The document source will contain some conditional code
depending on the value of |\version|.
Suppose further, the flag should default to |final| for the main file
and to |draft| for child files
which is a natural assignment for editing the document.
This is achieved by placing the following code
in the preamble of the main document
(below the |\childdocmain| directive):
%
\begin{center}
\begin{tabular}{l}
|\ifchilddoc|\\
|\providecommand{\version}{draft}|\\
|\||else|\\
|\providecommand{\version}{final}|\\
|\||fi|
\end{tabular}
\end{center}
%
The definition by |\providecommand| makes sure
that previous definitions are not overwritten.
Further statements |\providecommand{\version}{...}|
can thus be added before the above code to override it.

For the main file, one might add a line
(between |\childdocmain| and the above block)
%
\begin{center}
|%\ifchilddoc\||else\providecommand{\version}{draft}\||fi|
\end{center}
%
which can be uncommented to produce a draft version.
Likewise one can add a line to the very top of a child file
(above the |\childdocof{|\textit{main}|}| directive)
%
\begin{center}
|%\providecommand{\version}{final}|
\end{center}
%
which can be uncommented to produce the final version of this child document.

%%%%%%%%%%%%%%%%%%%%%%%%%%%%%%%%%%%%%%%%%%%%%%%%%%%%%%%%%%%%%%%%%%%%%%%%%%%%%%%%
\subsection{Forwarding}
\label{sec:forward}

Different versions of the main or child documents
using compilation flags as described in \secref{sec:flags}
can be (permanently) stored in different files
for convenient compilation, viewing and distribution.
To this end, the package defines a command
to pass on compilation to a different file:

%%%%%%%%%%%%%%%%%%%%%%%%%%%%%%%%%%%%%%%%
\DescribeMacro{\childdocforward}
The command |\childdocforward| redirects processing to
another source file:
%
\begin{center}
\begin{tabular}{l}
|\input{childdoc.def}|\\
|\childdocforward[|\textit{main}|]{|\textit{dest}|}|\\
\end{tabular}
\end{center}
%
The argument \textit{dest} is the destination file
(without extension).
It should be the main file or one of the child files.
Note that further \textsf{childdoc} directives
such as |\childdocof| and |\childdocforward|
in the indicated file will be processed in this form.
The optional argument \textit{main}
passes on directly to the main file \textit{main}
while pretending to compile the child \textit{dest}.
This form behaves as if \textit{dest}
issues |\childdocof{|\textit{main}|}| right away,
and no further \textsf{childdoc} directives will be processed.

%%%%%%%%%%%%%%%%%%%%%%%%%%%%%%%%%%%%%%%%
\DescribeMacro{\...prefix}
In the alternative form |\childdocforwardprefix|,
%
\begin{center}
\begin{tabular}{l}
|\input{childdoc.def}|\\
|\childdocforwardprefix[|\textit{main}|]{|\textit{prefix}|}{|\textit{dest}|}|
\end{tabular}
\end{center}
%
the destination file is determined by a pattern
depending on the current file:
To make this work, the current file must be called
`{\textit{prefix}\hspace{0.2em}\textit{suffix}}'
with \textit{prefix} matching precisely the argument.
Processing is then passed on to the file
`{\textit{dest}\hspace{0.2em}\textit{suffix}}'.
Surely, the same effect is achieved by
directly specifying the
argument `{\textit{dest}\hspace{0.2em}\textit{suffix}}'
in the first form.
However, that requires to set up a different file
for each child. With the alternative form of the command
all these files can have exactly the same content
which simplifies setting them up and maintaining them.

For example, the following file |draft.tex|
with a compilation flag |\version| as described in \secref{sec:flags}
compiles the main document as a draft:
%
\begin{center}
\begin{tabular}{l}
|\def\version{draft}|\\
|\input{childdoc.def}|\\
|\childdocforward{|\textit{main}|}|
\end{tabular}
\end{center}
%
Likewise, the following files |final|\textit{nn}|.tex|
compile the final version of the child document
|child|\textit{nn}|.tex|:
%
\begin{center}
\begin{tabular}{l}
|\def\version{final}|\\
|\input{childdoc.def}|\\
|\childdocforwardprefix{final}{child}|
\end{tabular}
\end{center}
%

Note that when several versions of a main file and/or of each child file
are to be generated, it may be convenient to set up a |Makefile| or
shell script to automatise the process.

%%%%%%%%%%%%%%%%%%%%%%%%%%%%%%%%%%%%%%%%%%%%%%%%%%%%%%%%%%%%%%%%%%%%%%%%%%%%%%%%
\subsection{Command Line Processing}
\label{sec:commandline}

The effect of redirection files can also be achieved by invoking
the \LaTeX{} compiler with a more elaborate command line.
Most conveniently this should be done as part
of a shell script or a |Makefile|.

When using \textsf{childdoc} in the main file, the following
command lines effectively perform a redirection
(note that depending on the shell being used,
backslashes may have to be doubled: `|\|' $\to$ `|\\|'):
%
\begin{center}
|... -jobname "|\textit{target}|" |\\|"|[\textit{flags}]%
|\input{childdoc.def}\childdocforward[|\textit{main}|]{|\textit{dest}|}"|
\end{center}
%
Here \textit{target} is the name of the output file,
\textit{main} is the name of the main file
and \textit{dest} is the name of the main or child file to be processed
(all filenames without extensions).
The optional argument \textit{main} can be omitted
if \textit{main} matches \textit{dest}.
Optionally, compilation \textit{flags} can be defined via |\def| commands.
This command line makes the \TeX{} engine believe
it is compiling the file \textit{target}
whose content is specified as the latter parameter.
The provided code then forwards the processing to
\textit{main} or \textit{dest} as described in \secref{sec:forward}.

%%%%%%%%%%%%%%%%%%%%%%%%%%%%%%%%%%%%%%%%%%%%%%%%%%%%%%%%%%%%%%%%%%%%%%%%%%%%%%%%
\subsection{Include by Input}
\label{sec:input}

Including child documents by |\include| has some restrictions by design.
Most notably, the content of a child document always occupies
its own set of pages; pages cannot be shared between child documents.
Usually, this behaviour makes perfect sense
because each child document contain an essential part of the document.
However, in some situations it may be desirable to compose
a document from a collection of parts
without having mandatory page breaks between then.
For this case, the package
provides a mechanism to include parts
by |\input| which can also be processed individually.
However, by construction this mechanism
requires manual handling of the content to be output.

%%%%%%%%%%%%%%%%%%%%%%%%%%%%%%%%%%%%%%%%
\DescribeMacro{\ifchilddocmanual}
The main file should be prepared as usual, see \secref{sec:include}.
However, the document body must make a distinction
between processing of an individual part and of the main document, e.g.:
%
\begin{center}
\begin{tabular}{l}
|\ifchilddocmanual|\\
|\input{\childdocname}|\\
|\||else|\\
\textit{document body with }|\input{|\textit{part}|}|\\
|\||fi|
\end{tabular}
\end{center}
%
The conditional |\ifchilddocmanual| is true whenever
a part to be included by |\input| is being compiled,
and the name of the part is stored in |\childdocname|.

%%%%%%%%%%%%%%%%%%%%%%%%%%%%%%%%%%%%%%%%
\DescribeMacro{\childdocby}
Each part to be included by |\input| should start with:
%
\begin{center}
\begin{tabular}{l}
|\input{childdoc.def}|\\
|\childdocby{|\textit{main}|}|\\
\end{tabular}
\end{center}
%
The directive |\childdocby| is similar to |\childdocof|
described in \secref{sec:include},
but the subsequent selection of content must be done manually.
To that end, both |\ifchilddoc| and |\ifchilddocmanual|
will be true upon processing of a part,
and the name of the part is stored in |\childdocname|.
Note that |\jobname| will be set to the filename of the current part
so that each part receives an individual |.aux| file
that does not interfere with the |.aux| file(s) of the main document.
This behaviour can be altered by the alternative form
|\childdocby[*]{|\textit{main}|}| (with a non-empty optional argument)
which uses the |.aux| file of the main document
by setting |\jobname| to \textit{main}.

%%%%%%%%%%%%%%%%%%%%%%%%%%%%%%%%%%%%%%%%%%%%%%%%%%%%%%%%%%%%%%%%%%%%%%%%%%%%%%%%
\subsection{Driver Development}
\label{sec:driver}

The \textsf{childdoc} mechanism can also be use for the development
of definition files such as \LaTeX{} styles or classes.
This case differs from the above setup with multiple parts
included by |\include| in that no |\includeonly| should be invoked.
This can be achieved by starting the include file
(before |\ProvidesPackage|) with:
%
\begin{center}
\begin{tabular}{l}
|\input{childdoc.def}|\\
|\childdocforward{|\textit{main}|}|\\
\end{tabular}
\end{center}
%
or alternatively with:
%
\begin{center}
\begin{tabular}{l}
|\input{childdoc.def}|\\
|\childdocby{|\textit{main}|}|\\
\end{tabular}
\end{center}
%
Both forms have slightly different effects as described above.
The main file is prepared as usual, see \secref{sec:include}.

%%%%%%%%%%%%%%%%%%%%%%%%%%%%%%%%%%%%%%%%%%%%%%%%%%%%%%%%%%%%%%%%%%%%%%%%%%%%%%%%
\subsection{Legacy Detection}
\label{sec:detection}

The directive |\childdocmain| in the main file can detect
whether the complete document or merely a child is to be compiled
even without using the directive |\childdocof|.
This method is deprecated because it is less robust
and there is no compelling reason to use it;
it is merely provided for backward compatibility
and it may be removed in future versions.

If the detection mechanism is to be used,
it is mandatory to correctly specify
the filename of the main file as the argument of |\childdocmain|:
%
\begin{center}
\begin{tabular}{l}
|\input{childdoc.def}|\\
|\childdocmain{|\textit{main}|}|\\
\end{tabular}
\end{center}
%
If |\jobname| does not match the argument \textit{main} of |\childdocmain|,
it is assumed that |\jobname| points to the child file to be compiled.
When using |\childdocmain| with the main file specified as argument,
it suffices to start a child file
with just |\input{|\textit{main}|}|
without loading of the package and using |\childdocof|.
If instead all processing is done
with the appropriate \textsf{childdoc} directives,
the argument of \textit{main} of |\childdocmain| can be empty.

An alternative version of the command line processing described
in \secref{sec:commandline} using the detection mechanism reads:
%
\begin{center}
|... -jobname "|\textit{target}|" "|[\textit{flags}]%
[|\def\jobname{|\textit{dest}|}|]|\input{|\textit{main}|}"|
\end{center}

%%%%%%%%%%%%%%%%%%%%%%%%%%%%%%%%%%%%%%%%%%%%%%%%%%%%%%%%%%%%%%%%%%%%%%%%%%%%%%%%
\subsection{Manual Code}
\label{sec:manual}

In case one cannot be certain whether the definitions file |childdoc.def|
is installed on the target \TeX{} distribution
and one prefers not to ship it,
it is conceivable to paste a few relevant commands into the sources.

To that end, drop all statements |\input{childdoc.def}|
and perform the replacements as outlined below.
Instead of |\childdocmain{|\textit{main}|}| add the following code
to the top of the main file:
%
\begin{center}
\begin{tabular}{l}
|\||ifdefined\childdocname\endinput\||fi\newif\ifchilddoc|\\
|\edef\childdocname{\scantokens\expandafter{\jobname\noexpand}}|\\
|\def\childdocmain{|\textit{main}|}\||ifx\childdocmain\childdocname\||else|\\
|\childdoctrue\includeonly{\childdocname}\let\jobname\childdocmain\||fi|\\
\end{tabular}
\end{center}
%
Instead of |\childdocof{|\textit{main}|}| just include the main file
at the top of each child file:
%
\begin{center}
|\input{|\textit{main}|}|
\end{center}
%
A simple redirection |\childdocforward{|\textit{dest}|}| is achieved by:
%
\begin{center}
|\def\jobname{|\textit{dest}|}\input{\jobname}|
\end{center}
%
The redirection with prefix
|\childdocforwardprefix[|\textit{prefix}|]{|\textit{dest}|}|
is accomplished by:
%
\begin{center}
\begin{tabular}{l}
|{\edef\jobname{\scantokens\expandafter{\jobname\noexpand}}|\\
|\def\redirectjob |\textit{prefix}|#1~~~{\gdef\jobname{|\textit{dest}|#1}}|\\
|\expandafter\redirectjob\jobname~~~}\input{\jobname}|
\end{tabular}
\end{center}

In an alternative approach,
child documents can be compiled by a specific command line
without additional code or specific definitions:
%
\begin{center}
|... -jobname "|\textit{target}|" "|[\textit{flags}]%
|\includeonly{|\textit{dest}|}\input{|\textit{main}|}"|
\end{center}
%

%%%%%%%%%%%%%%%%%%%%%%%%%%%%%%%%%%%%%%%%%%%%%%%%%%%%%%%%%%%%%%%%%%%%%%%%%%%%%%%%
%%%%%%%%%%%%%%%%%%%%%%%%%%%%%%%%%%%%%%%%%%%%%%%%%%%%%%%%%%%%%%%%%%%%%%%%%%%%%%%%
\section{Information}

%%%%%%%%%%%%%%%%%%%%%%%%%%%%%%%%%%%%%%%%%%%%%%%%%%%%%%%%%%%%%%%%%%%%%%%%%%%%%%%%
\subsection{Copyright}

Copyright \copyright{} 2017--2018 Niklas Beisert

This work may be distributed and/or modified under the
conditions of the \LaTeX{} Project Public License, either version 1.3
of this license or (at your option) any later version.
The latest version of this license is in
  \url{http://www.latex-project.org/lppl.txt}
and version 1.3 or later is part of all distributions of \LaTeX{}
version 2005/12/01 or later.

This work has the LPPL maintenance status `maintained'.

The Current Maintainer of this work is Niklas Beisert.

This work consists of the files |README.txt|, |childdoc.ins| and |childdoc.dtx|
as well as the derived files |childdoc.def|, |cdocsamp.tex|
with |cdocsch1.tex|, |cdocsch2.tex|, |cdocspt3.tex|, |cdocspt4.tex|,
|cdocsdrf.tex|, |cdocsfn1.tex|, |cdocsfn2.tex|
as well as |childdoc.pdf|.

%%%%%%%%%%%%%%%%%%%%%%%%%%%%%%%%%%%%%%%%%%%%%%%%%%%%%%%%%%%%%%%%%%%%%%%%%%%%%%%%
\subsection{Files and Installation}

The package consists of the files:
%
\begin{center}
\begin{tabular}{ll}
    |README.txt|   & readme file \\
    |childdoc.ins| & installation file \\
    |childdoc.dtx| & source file \\
    |childdoc.def| & definition file \\
    |cdocsamp.tex| & sample main file \\
    |cdocsch1.tex| & sample include file \\
    |cdocsch2.tex| & sample include file \\
    |cdocspt3.tex| & sample part file \\
    |cdocspt4.tex| & sample part file \\
    |cdocsdrf.tex| & sample redirection file \\
    |cdocsfn1.tex| & sample redirection file \\
    |cdocsfn2.tex| & sample redirection file \\
    |childdoc.pdf| & manual
\end{tabular}
\end{center}
%
The distribution consists of the files
|README.txt|, |childdoc.ins| and |childdoc.dtx|.
%
\begin{itemize}
\item
Run (pdf)\LaTeX{} on |childdoc.dtx|
to compile the manual |childdoc.pdf| (this file).
\item
Run \LaTeX{} on |childdoc.ins| to create the definitions file |childdoc.def|
and the sample |cdocsamp.tex| with include files
|cdocsch1.tex|, |cdocsch2.tex|, |cdocspt3.tex|, |cdocspt4.tex|,
|cdocsdrf.tex|, |cdocsfn1.tex|, |cdocsfn2.tex|.
Then copy the file |childdoc.def| to an appropriate directory of your \LaTeX{}
distribution, e.g.\ \textit{texmf-root}|/tex/latex/childdoc|.
\end{itemize}

%%%%%%%%%%%%%%%%%%%%%%%%%%%%%%%%%%%%%%%%%%%%%%%%%%%%%%%%%%%%%%%%%%%%%%%%%%%%%%%%
\subsection{Related CTAN Packages}

There are several other packages which offer a similar functionality:
%
\begin{itemize}
\item
The packages
\href{http://ctan.org/pkg/docmute}{\textsf{docmute}},
\href{http://ctan.org/pkg/includex}{\textsf{includex}} and
\href{http://ctan.org/pkg/standalone}{\textsf{standalone}}
provide commands to include only the document body of
a child file thus allowing both files to be compiled individually.
\item
The packages \href{http://ctan.org/pkg/subdocs}{\textsf{subdocs}}
and \href{http://ctan.org/pkg/subfiles}{\textsf{subfiles}}
provide structures in which the main and child documents can be
encapsulated and allowing them to be compiled individually.
The inclusion mechanism is different from the conventional |\include|.
\item
The package \href{http://ctan.org/pkg/combine}{\textsf{combine}}
is an elaborate solution to combine several documents into one.
\end{itemize}
%
See also the CTAN topic \href{http://ctan.org/topic/subdocs}{\textsf{subdocs}}
for further related packages.
The present package differs from the above solutions in that
a document structure constructed with the conventional |\include| mechanism
just needs two extra commands at the top of every file
such that all constituent files can be compiled individually.

%%%%%%%%%%%%%%%%%%%%%%%%%%%%%%%%%%%%%%%%%%%%%%%%%%%%%%%%%%%%%%%%%%%%%%%%%%%%%%%%
%\subsection{Feature Suggestions}
%
%The following is a list of features which may be useful for future
%versions of this package:
%%
%\begin{itemize}
%\item
%\ldots
%\end{itemize}

%%%%%%%%%%%%%%%%%%%%%%%%%%%%%%%%%%%%%%%%%%%%%%%%%%%%%%%%%%%%%%%%%%%%%%%%%%%%%%%%
\subsection{Revision History}

%%%%%%%%%%%%%%%%%%%%%%%%%%%%%%%%%%%%%%%%
\paragraph{v2.0:} 2018/12/30

\begin{itemize}
\item
immediate forward processing
\item
added |\childdocby| mechanism
\item
manual restructured
\end{itemize}

%%%%%%%%%%%%%%%%%%%%%%%%%%%%%%%%%%%%%%%%
\paragraph{v1.6:} 2018/01/17

\begin{itemize}
\item
application for development of include files
\item
corrections to manual
\end{itemize}

%%%%%%%%%%%%%%%%%%%%%%%%%%%%%%%%%%%%%%%%
\paragraph{v1.5:} 2017/05/21

\begin{itemize}
\item
more complete structuring introduced
\item
|\childdocof| introduced
\item
|\childdoc| renamed to |\childdocmain|
\item
|\childredirect| renamed to |\childdocforward| and |\childdocforwardprefix|
and functionality expanded
\end{itemize}

%%%%%%%%%%%%%%%%%%%%%%%%%%%%%%%%%%%%%%%%
\paragraph{v1.0:} 2017/04/27

\begin{itemize}
\item
manual and install package
\item
first version published on CTAN
\end{itemize}

%%%%%%%%%%%%%%%%%%%%%%%%%%%%%%%%%%%%%%%%
\paragraph{v0.6:} 2017/04/26

\begin{itemize}
\item
redirection mechanism added
\end{itemize}

%%%%%%%%%%%%%%%%%%%%%%%%%%%%%%%%%%%%%%%%
\paragraph{v0.5:} 2017/04/26

\begin{itemize}
\item
functionality in definition file
\end{itemize}


%%%%%%%%%%%%%%%%%%%%%%%%%%%%%%%%%%%%%%%%%%%%%%%%%%%%%%%%%%%%%%%%%%%%%%%%%%%%%%%%
%%%%%%%%%%%%%%%%%%%%%%%%%%%%%%%%%%%%%%%%%%%%%%%%%%%%%%%%%%%%%%%%%%%%%%%%%%%%%%%%
%%%%%%%%%%%%%%%%%%%%%%%%%%%%%%%%%%%%%%%%%%%%%%%%%%%%%%%%%%%%%%%%%%%%%%%%%%%%%%%%
\appendix

\settowidth\MacroIndent{\rmfamily\scriptsize 000\ }

 \DocInput{childdoc.dtx}

\end{document}
%</driver>
% \fi
%
% %%%%%%%%%%%%%%%%%%%%%%%%%%%%%%%%%%%%%%%%%%%%%%%%%%%%%%%%%%%%%%%%%%%%%%%%%%%%%%
% %%%%%%%%%%%%%%%%%%%%%%%%%%%%%%%%%%%%%%%%%%%%%%%%%%%%%%%%%%%%%%%%%%%%%%%%%%%%%%
% \section{Sample}
%\iffalse
%<*samplemain>
%\fi
%
% The following presents a sample document
% with two chapters, two parts, a title page,
% a compile flag as well as three forwarding files to set the flag.
% It consists of eight |.tex| files:
% \begin{center}
% \begin{tabular}{ll}
% |cdocsamp.tex|&main file\\
% |cdocsch1.tex|&include file for chapter 1\\
% |cdocsch2.tex|&include file for chapter 2\\
% |cdocspt3.tex|&include file for part 3\\
% |cdocspt4.tex|&include file for part 4\\
% |cdocsdrf.tex|&forwarding file for main file in draft mode\\
% |cdocsfi1.tex|&forwarding file for final version of chapter 1\\
% |cdocsfi2.tex|&forwarding file for final version of chapter 2\\
% \end{tabular}
% \end{center}
% Each of the eight files can be compiled directly by the \LaTeX{} compiler.
%
% %%%%%%%%%%%%%%%%%%%%%%%%%%%%%%%%%%%%%%
% \paragraph{Main File.}
%
% The main file is called |cdocsamp.tex|.
%
% Load the \textsf{childdoc} definitions and
% declare the filename for the main document:
%    \begin{macrocode}
\input{childdoc.def}
\childdocmain{}
%    \end{macrocode}

% Optional override for |\version| flag:
%    \begin{macrocode}
%%\ifchilddoc\else\providecommand{\version}{draft}\fi
%    \end{macrocode}

% Define the default values for the |\version| flag
% (|final| for the main file and |draft| for childs):
%    \begin{macrocode}
\ifchilddoc
\providecommand{\version}{draft}
\else
\providecommand{\version}{final}
\fi
%    \end{macrocode}

% Load the standard document class:
%    \begin{macrocode}
\documentclass[12pt]{article}
%    \end{macrocode}

% Start the document body:
%    \begin{macrocode}
\begin{document}
%    \end{macrocode}

% Declare a title page.
% Print title, part of document being processed and version flag:
%    \begin{macrocode}
\addtocounter{page}{-1}
\begin{center}
{\LARGE\bfseries{}childdoc example\par}
\vspace{1cm}
\ifchilddoc
\ifchilddocmanual part\else chapter\fi:
`\childdocname' of `\childdocjob'\par
\else
main document: `\childdocjob'\par
\fi
version: \version\par
\end{center}
\newpage
%    \end{macrocode}

% Manually include selected file,
% otherwise process as usual:
%    \begin{macrocode}
\ifchilddocmanual
\section*{part `\childdocname'}
\input{\childdocname}
\else
%    \end{macrocode}

% Include the two chapters:
%    \begin{macrocode}
\include{cdocsch1}
\include{cdocsch2}
%    \end{macrocode}

% Include the two parts unless only chapters should be displayed:
%    \begin{macrocode}
\ifchilddoc\else
\section{part three}
\input{cdocspt3}
\section{part four}
\input{cdocspt4}
\fi
%    \end{macrocode}

% Process as usual until here:
%    \begin{macrocode}
\fi
%    \end{macrocode}

% End of document body:
%    \begin{macrocode}
\end{document}
%    \end{macrocode}
%\iffalse
%</samplemain>
%\fi
%
% %%%%%%%%%%%%%%%%%%%%%%%%%%%%%%%%%%%%%%
% \paragraph{Chapter Include Files.}
%
% The include files are called |cdocsch1.tex| and |cdocsch2.tex|.
%
%\iffalse
%<*samplechap1|samplechap2>
%\fi

% Optional override for |\version| flag:
%    \begin{macrocode}
%%\providecommand{\version}{final}
%    \end{macrocode}

% Include the main document:
%    \begin{macrocode}
\input{childdoc.def}
\childdocof{cdocsamp}
%    \end{macrocode}

%\iffalse
%</samplechap1|samplechap2>
%\fi
%
%\iffalse
%<*samplechap1>
%\fi
% Some text for chapter 1:
%    \begin{macrocode}
\section{one}
some text in chapter one
%    \end{macrocode}

%\iffalse
%</samplechap1>
%\fi
% Some text for chapter 2:
%\iffalse
%<*samplechap2>
%\fi
%    \begin{macrocode}
\section{two}
more text in chapter two
%    \end{macrocode}

%\iffalse
%</samplechap2>
%\fi
%
% %%%%%%%%%%%%%%%%%%%%%%%%%%%%%%%%%%%%%%
% \paragraph{Part Include Files.}
%
% The include files are called |cdocspt3.tex| and |cdocspt4.tex|.
%
%\iffalse
%<*samplepart3|samplepart4>
%\fi

% Optional override for |\version| flag:
%    \begin{macrocode}
%%\providecommand{\version}{final}
%    \end{macrocode}

% Include the main document:
%    \begin{macrocode}
\input{childdoc.def}
\childdocby{cdocsamp}
%    \end{macrocode}

%\iffalse
%</samplepart3|samplepart4>
%\fi
%
%\iffalse
%<*samplepart3>
%\fi
% Some text for part 3:
%    \begin{macrocode}
some text in part three
%    \end{macrocode}

%\iffalse
%</samplepart3>
%\fi
% Some text for part 4:
%\iffalse
%<*samplepart4>
%\fi
%    \begin{macrocode}
more text in part four
%    \end{macrocode}

%\iffalse
%</samplepart4>
%\fi
%
% %%%%%%%%%%%%%%%%%%%%%%%%%%%%%%%%%%%%%%
% \paragraph{Forwarding for a Complete Draft.}
%
% The following forwarding file |cdocsdrf.tex|
% compiles the main document in draft mode:
%\iffalse
%<*sampledraft>
%\fi
%    \begin{macrocode}
\def\version{draft}
\input{childdoc.def}
\childdocforward{cdocsamp}
%    \end{macrocode}

%\iffalse
%</sampledraft>
%\fi
%
% %%%%%%%%%%%%%%%%%%%%%%%%%%%%%%%%%%%%%%
% \paragraph{Forwarding for Final Version of the Chapters.}
%
% The following forwarding files |cdocsfn1.tex| and |cdocsfn2.tex|
% (with identical content)
% compile the final versions of the child documents
% |cdocsch1.tex| and |cdocsch2.tex|, respectively:
%\iffalse
%<*samplefinal>
%\fi
%    \begin{macrocode}
\def\version{final}
\input{childdoc.def}
\childdocforwardprefix[cdocsamp]{cdocsfn}{cdocsch}
%    \end{macrocode}

%\iffalse
%</samplefinal>
%\fi
%
% %%%%%%%%%%%%%%%%%%%%%%%%%%%%%%%%%%%%%%
% \paragraph{Command Line Processing.}
%
% The following three command lines generate the output files
% |cdocscld|, |cdocscl1| and |cdocscl2|
% which should be identical to
% |cdocsdrf|, |cdocsch1| and |cdocsfn2|, respectively:
% \begin{center}
% \begin{tabular}{l}
% |latex -jobname cdocscld \|\\
% |  "\def\version{draft}\input{childdoc.def}\childdocforward{cdocsamp}"|\\
% |latex -jobname cdocscl1 \|\\
% |  "\input{childdoc.def}\childdocforward[cdocsamp]{cdocsch1}"|\\
% |latex -jobname cdocscl2 \|\\
% |  "\def\version{final}\input{childdoc.def}\childdocforward{cdocsch2}"|
% \end{tabular}
% \end{center}
% Note that the trailing backslash on each first line
% merely continues the input to the second line
% (for convenient cut ant paste).
% Furthermore, the command |latex| can be replaced by any
% of its alternative versions such as |pdflatex|.
%
% %%%%%%%%%%%%%%%%%%%%%%%%%%%%%%%%%%%%%%%%%%%%%%%%%%%%%%%%%%%%%%%%%%%%%%%%%%%%%%
% %%%%%%%%%%%%%%%%%%%%%%%%%%%%%%%%%%%%%%%%%%%%%%%%%%%%%%%%%%%%%%%%%%%%%%%%%%%%%%
% \section{Implementation}
%\iffalse
%<*package>
%\fi
%
% This section describes the definitions file |childdoc.def|.

% The definitions cannot be loaded using |\usepackage| or |\RequirePackage|
% which has a mechanism to prevent loading a style file more than once.
% When loading the definitions by means of |\input|
% multiple instances have to be prevented manually:
%\iffalse
%This code needs to be before the `\ProvidesFile' directive
%which is defined at the beginning of this file.
%Therefore it is also placed there and commented out here.
%</package>
%<*discard>
%\fi
%    \begin{macrocode}
\ifdefined\childdocmain\endinput\fi
%    \end{macrocode}
%\iffalse
%</discard>
%<*package>
%\fi
%
% \macro{\ifchilddoc}
% \macro{\ifchilddocmanual}
% The conditional |\ifchilddoc| tells whether a
% child (true) or main (false) document is being compiled.
% The conditional |\ifchilddocmanual| tells whether
% the |\includeonly| mechanism is used (false) or
% the selection of child files must be performed manually (true).
% The definitions initialise to false:
%    \begin{macrocode}
\newif\ifchilddoc
\newif\ifchilddocmanual
%    \end{macrocode}

% \macro{\childdocname}
% \macro{\childdocjob}
% The macro |\childdocname| stores the name of the main document
% to be compiled. The macro |\childdocjob| stores the name of
% the document on which the \LaTeX{} compiler was originally invoked.
% The content of |\jobname| cannot be compared
% to filenames specified in the source due to different catcodes.
% The following code rescans |\jobname|, stores the result
% in |\childdocname| and saves a copy in |\childdocjob|:
%    \begin{macrocode}
\edef\childdocname{\scantokens\expandafter{\jobname\noexpand}}
\let\childdocjob\childdocname
%    \end{macrocode}

% \macro{\childdocdisable}
% The macro |\childdocdisable| prevents the main file
% from being processed more than once.
% At this stage, the main document command |\childdocmain|
% is assumed to be called once again where it should do nothing.
% Any subsequent call to it should prevent
% a secondary processing of the main document
% It overwrites the forwarding commands
% |\childdocof| and |\childdocforward|
% with empty macros to prevent further inclusions of the main document:
%    \begin{macrocode}
\newcommand{\childdocdisable}
{
  \renewcommand{\childdocmain}[1]{\renewcommand{\childdocmain}[1]{\endinput}}
  \renewcommand{\childdocof}[1]{}
  \renewcommand{\childdocby}[2][]{}
  \renewcommand{\childdocforward}[2][]{}
  \renewcommand{\childdocdisable}{}
}
%    \end{macrocode}

% \macro{\childdocmain}
% The macro |\childdocmain| is to be called at the top of the main file
% with nothing or the main filename (without extension) as argument.
% First, it breaks loops.
% If the argument is not empty and does not match |\childdocname|
% (which is set by the first inclusion of |childdoc.def|),
% |\ifchilddoc| is set to true, |\includeonly| is applied to the child file
% and |\jobname| is set to the main file
% (for proper handling of |.aux| files):
%    \begin{macrocode}
\newcommand{\childdocmain}[1]
{
  \childdocdisable\childdocmain{}
  \if?#1?\else
    \begingroup
      \def\childdoctmp{#1}
      \ifx\childdoctmp\childdocname
        \def\childdoctmp{}
      \else
        \def\childdoctmp
        {
          \childdoctrue
          \includeonly{\childdocname}
          \def\childdocjob{#1}
          \def\jobname{#1}
        }
      \fi
      \expandafter
    \endgroup
    \childdoctmp
  \fi
}
%    \end{macrocode}

% \macro{\childdocof}
% The command |\childdocof| redirects
% compilation to the main file |#1|.
%    \begin{macrocode}
\newcommand{\childdocof}[1]
{
  \childdocdisable
  \childdoctrue
  \includeonly{\childdocname}
  \def\jobname{#1}
  \def\childdocjob{#1}
  \input{#1}
}
%    \end{macrocode}

% \macro{\childdocby}
% The command |\childdocby| ....
%    \begin{macrocode}
\newcommand{\childdocby}[2][]
{
  \childdocdisable
  \childdoctrue
  \childdocmanualtrue
  \if?#1?\else
    \def\jobname{#2}
  \fi
  \def\childdocjob{#2}
  \input{#2}
  \endinput
}
%    \end{macrocode}

% \macro{\childdocforward}
% The command |\childdocforward| redirects
% compilation to the main file or
% (if the optional argument is given) a child file.
% Parameters are set as if the main file
% or a child file starting with |\childdocof| was compiled.
% Then compilation is handed over to the main file:
%    \begin{macrocode}
\newcommand{\childdocforward}[2][]
{
  \begingroup
    \if?#1?
      \def\childdoctmp
      {
        \def\childdocname{#2}
        \def\childdocjob{#2}
        \def\jobname{#2}
        \input{#2}
        \endinput
      }
    \else
      \def\childdoctmp
      {
        \childdocdisable
        \def\childdocname{#2}
        \childdoctrue
        \includeonly{#2}
        \def\childdocjob{#1}
        \def\jobname{#1}
        \input{#1}
        \endinput
      }
    \fi
    \expandafter
  \endgroup
  \childdoctmp
}
%    \end{macrocode}

% \macro{\childdocforwardprefix}
% The command |\childdocforwardprefix| redirects
% compilation to the main or a child file by means of a pattern.
% The prefix |#1| in the current filename is replaced by |#2|
% and the suffix of the current filename is kept
% (it is assumed that the filename does not contain the substring `|~~~|'
% which is used as a delimiter).
% Compilation is handed over to the new file by |\childdocforward|:
%    \begin{macrocode}
\newcommand{\childdocforwardprefix}[3][]
{
  \begingroup
    \def\childdocextract #2##1~~~{\def\childdoctmp{\childdocforward[#1]{#3##1}}}
    \expandafter\childdocextract\childdocname~~~
    \expandafter
  \endgroup
  \childdoctmp
}
%    \end{macrocode}

% \macro{\childdoc}
% The deprecated macro |\childdoc| is a legacy version of |\childdocmain|:
%    \begin{macrocode}
\newcommand{\childdoc}{\childdocmain}
%    \end{macrocode}

% \macro{\childdocredirect}
% The deprecated macro |\childdocredirect| is a legacy version
% of |\childdocforward| and |\childdocforwardprefix|:
%    \begin{macrocode}
\newcommand{\childdocredirect}[2][]
{
  \begingroup
    \if?#1?
      \def\childdoctmp{\childdocforward{#2}}
    \else
      \def\childdoctmp{\childdocforwardprefix{#1}{#2}}
    \fi
    \expandafter
  \endgroup
  \childdoctmp
}
%    \end{macrocode}

%\iffalse
%</package>
%\fi
%
\endinput
|\\
|\childdocforward{|\textit{main}|}|\\
\end{tabular}
\end{center}
%
or alternatively with:
%
\begin{center}
\begin{tabular}{l}
|% \iffalse
%
% childdoc.dtx Copyright (C) 2017-2018 Niklas Beisert
%
% This work may be distributed and/or modified under the
% conditions of the LaTeX Project Public License, either version 1.3
% of this license or (at your option) any later version.
% The latest version of this license is in
%   http://www.latex-project.org/lppl.txt
% and version 1.3 or later is part of all distributions of LaTeX
% version 2005/12/01 or later.
%
% This work has the LPPL maintenance status `maintained'.
%
% The Current Maintainer of this work is Niklas Beisert.
%
% This work consists of the files childdoc.dtx and childdoc.ins
% and the derived files childdoc.def and cdocsamp.tex with
% cdocsch1.tex, cdocsch2.tex, cdocsdrf.tex, cdocsfn1.tex, cdocsfn2.tex.
%
%<package>\ifdefined\childdocmain\endinput\fi
%<package>\ProvidesFile{childdoc.def}[2018/12/30 v2.0 child document driver]
%<samplemain>\ProvidesFile{cdocsamp.tex}[2018/12/30 v2.0 sample for childdoc]
%<*driver>
%\ProvidesFile{childdoc.drv}[2018/12/30 v2.0 childdoc reference manual file]
\PassOptionsToClass{10pt,a4paper}{article}
\documentclass{ltxdoc}

\usepackage[margin=35mm]{geometry}
\usepackage{hyperref}
\usepackage{hyperxmp}
\usepackage[usenames]{color}

\hypersetup{colorlinks=true}
\hypersetup{pdfstartview=FitH}
\hypersetup{pdfpagemode=UseNone}
\hypersetup{pdfsource={}}
\hypersetup{pdflang={en-UK}}
\hypersetup{pdfcopyright={Copyright 2017-2018 Niklas Beisert.
  This work may be distributed and/or modified under the
  conditions of the LaTeX Project Public License, either version 1.3
  of this license or (at your option) any later version.}}
\hypersetup{pdflicenseurl={http://www.latex-project.org/lppl.txt}}
\hypersetup{pdfcontactaddress={ETH Zurich, ITP, HIT K,
  Wolfgang-Pauli-Strasse 27}}
\hypersetup{pdfcontactpostcode={8093}}
\hypersetup{pdfcontactcity={Zurich}}
\hypersetup{pdfcontactcountry={Switzerland}}
\hypersetup{pdfcontactemail={nbeisert@itp.phys.ethz.ch}}
\hypersetup{pdfcontacturl={http://people.phys.ethz.ch/\xmptilde nbeisert/}}

\newcommand{\secref}[1]{\hyperref[#1]{section \ref*{#1}}}

\parskip1ex
\parindent0pt
\let\olditemize\itemize
\def\itemize{\olditemize\parskip0pt}

\begin{document}

\title{The \textsf{childdoc} Package}
\hypersetup{pdftitle={The childdoc Package}}
\author{Niklas Beisert\\[2ex]
  Institut f\"ur Theoretische Physik\\
  Eidgen\"ossische Technische Hochschule Z\"urich\\
  Wolfgang-Pauli-Strasse 27, 8093 Z\"urich, Switzerland\\[1ex]
  \href{mailto:nbeisert@itp.phys.ethz.ch}
  {\texttt{nbeisert@itp.phys.ethz.ch}}}
\hypersetup{pdfauthor={Niklas Beisert}}
\hypersetup{pdfsubject={Manual for the LaTeX2e Package childdoc}}
\date{30 December 2018, \textsf{v2.0}}
\maketitle

\begin{abstract}\noindent
\textsf{childdoc} is a \LaTeXe{} package
that enables the direct compilation
of document sections included by |\include|
to individual files.
\end{abstract}

\begingroup
\parskip0ex
\tableofcontents
\endgroup

%%%%%%%%%%%%%%%%%%%%%%%%%%%%%%%%%%%%%%%%%%%%%%%%%%%%%%%%%%%%%%%%%%%%%%%%%%%%%%%%
%%%%%%%%%%%%%%%%%%%%%%%%%%%%%%%%%%%%%%%%%%%%%%%%%%%%%%%%%%%%%%%%%%%%%%%%%%%%%%%%
\section{Introduction}

\LaTeX{} provides a mechanism to structure a large document (such as a book)
into a main file and several child files (containing the chapters)
using the |\include| command.
This mechanism is beneficial for documents
which span hundreds of pages in order to
make the source file(s) more manageable.
Moreover, compilation can be restricted to
selected child files by means of the |\includeonly| command.
The latter feature can be used to reduce the compilation time while editing
(this was significantly more useful in the earlier days of \LaTeX{})
or to generate a smaller document which is easier to navigate.
Another application of |\includeonly| is to generate
documents consisting of selected parts of the complete document.

However, there are a few drawbacks of the plain |\include| mechanism:
\begin{itemize}
\item
The child files cannot be compiled on their own,
they can only be compiled via the main file.
A naive editing environment
(such as a text editor with an option
to have the current file processed by \LaTeX)
may require one to switch to the main file before compiling;
attempting to compile the child file produces errors.
\item
The main file must be modified (each time)
to adjust the |\includeonly| command
to the present needs. This easily leaves the main file in a messy state.
\item
The generated document will always carry the filename
of the main document. This is inconvenient if
several child files are to be compiled and
to be kept for distribution.
\end{itemize}

The present package provides a simple interface
to make child files individually compilable by \LaTeX{}.
Compiling a child file then has the same effect as compiling
the main file with an |\includeonly| command
to select the appropriate child.
Moreover the generated document will carry the name of the child
rather than the main file.
This resolves all three above issues.

This feature is meant to make the editing of books,
thesis documents and lecture notes somewhat more convenient.
However, the package can also be used efficiently for
composing a series of documents (such as exercise sheets)
which are typically distributed individually.
It then assists the author in generating the individual documents
(potentially in different versions)
as well as a document containing the collected series.
Another application is in developing style files
or other kinds of included material
where compilation of the style file could redirect
to a sample or test file.

%%%%%%%%%%%%%%%%%%%%%%%%%%%%%%%%%%%%%%%%%%%%%%%%%%%%%%%%%%%%%%%%%%%%%%%%%%%%%%%%
%%%%%%%%%%%%%%%%%%%%%%%%%%%%%%%%%%%%%%%%%%%%%%%%%%%%%%%%%%%%%%%%%%%%%%%%%%%%%%%%
\section{Usage}

First of all, the package \textsf{childdoc} is \emph{not} a standard
\LaTeXe{} |.sty| style file! Therefore it needs to be invoked in
a non-standard way.

%%%%%%%%%%%%%%%%%%%%%%%%%%%%%%%%%%%%%%%%%%%%%%%%%%%%%%%%%%%%%%%%%%%%%%%%%%%%%%%%
\subsection{Included Files}
\label{sec:include}

%%%%%%%%%%%%%%%%%%%%%%%%%%%%%%%%%%%%%%%%
\DescribeMacro{\childdocmain}
To use the package, add the commands
\begin{center}
\begin{tabular}{l}
|\input{childdoc.def}|\\
|\childdocmain{}|\\
\end{tabular}
\end{center}
at the very top of the main \LaTeX{} file,
in particular \emph{before} the |\documentclass| statement!
The argument of |\childdocmain| should be left empty
(but it must be present).

%%%%%%%%%%%%%%%%%%%%%%%%%%%%%%%%%%%%%%%%
\DescribeMacro{\childdocof}
Furthermore, add the commands
\begin{center}
\begin{tabular}{l}
|\input{childdoc.def}|\\
|\childdocof{|\textit{main}|}|\\
\end{tabular}
\end{center}
at the top of every child file \textit{child}
which is included by |\include{|\textit{child}|}|
from within the main file
(or at least for those files to be compiled individually).
The argument \textit{main} must be the filename of the main file.

There are a couple of
considerations in setting up the main and child documents:

%%%%%%%%%%%%%%%%%%%%%%%%%%%%%%%%%%%%%%%%
\paragraph{Restrictions.}

Please note the following restrictions:
\begin{itemize}
\item
|\childdocmain| must be called with one argument \textit{main}
to ensure compatibility with earlier version of the package.
It must either be empty (|\childdocmain{}|)
or precisely match the filename of the main file in which it is specified.
See \secref{sec:detection} for further information.
\item
The filename \textit{main} must be specified without the |.tex| extension.
\item
The filename \textit{main} is case sensitive
(even in case-insensitive file systems)
due to internal string comparison.
\item
The argument \textit{main} should be fully expanded, it cannot be a macro.
\item
Subdirectories and special characters should be avoided in filenames.
\item
The command |\childdocmain{|\textit{main}|}| must be followed by a whitespace.
It should not be followed immediately by another command
or by a comment mark `|%|'.
This is because the \TeX{} parser reads the token immediately following
the argument of |\childdocmain| and puts it
at the beginning of every child section;
however, a white\-space is ignored.
\end{itemize}

%%%%%%%%%%%%%%%%%%%%%%%%%%%%%%%%%%%%%%%%
\paragraph{Content of Main File.}

It is advisable to place all content in the child files included by |\include|.
Any output contained in the main file will appear in all child documents
unless suppressed manually;
it cannot be suppressed automatically by the |\includeonly| directive
and thus should normally be avoided.
A method to include some content in the main file
by means of conditional processing is described in \secref{sec:conditional}.

%%%%%%%%%%%%%%%%%%%%%%%%%%%%%%%%%%%%%%%%
\paragraph{Page Numbering.}

When only a part of the document is compiled,
the appropriate numbering of pages
(as well as other status parameters)
is determined from the |.aux| files.
The latter contain information from previous passes.
However this information needs to propagate through
all intermediate child documents.
Therefore the page numbering in child documents may well
be inconsistent until the complete document is compiled at least once.

A useful (if unconventional) way to always ensure a consistent
page numbering is to restart the numbering in each child document
and denote the pages by `\textit{child}|.|\textit{page}'
where \textit{child} represents the chapter/section number of the child file.
This can be achieved by the command
|\numberwithin{page}{|\textit{child}|}|
of the \textsf{amsmath} package
where \textit{child} can be |chapter| or |section|
depending on the chosen structuring.
Alternatively, one can modify the macro |\thepage| appropriately
and reset the counter |page| at the start of each child file.

%%%%%%%%%%%%%%%%%%%%%%%%%%%%%%%%%%%%%%%%%%%%%%%%%%%%%%%%%%%%%%%%%%%%%%%%%%%%%%%%
\subsection{Conditional Processing}
\label{sec:conditional}

The package provides a mechanism to compile different versions
of a document. To customise the versions further some conditional processing
can come in handy to distinguish which version is being compiled.
The package provides two macros to describe the compilation context:

%%%%%%%%%%%%%%%%%%%%%%%%%%%%%%%%%%%%%%%%
\DescribeMacro{\ifchilddoc}
The conditional |\ifchilddoc| distinguishes between the compilation of
child documents and the main document:
%
\begin{center}
|\ifchilddoc |\textit{child-code}| |[|\||else |\textit{main-code}]| \||fi|
\end{center}

%%%%%%%%%%%%%%%%%%%%%%%%%%%%%%%%%%%%%%%%
\DescribeMacro{\childdocname}
\DescribeMacro{\childdocjob}
The macro |\childdocname| contains the filename (without extension)
of the main or child file being processed.
Note that |\childdocjob| will always contain the name of the main file.

%%%%%%%%%%%%%%%%%%%%%%%%%%%%%%%%%%%%%%%%
\paragraph{Title Page.}

Conditional processing can be used to include a title or banner page
in the main document when proper precautions are taken.
Importantly, the code in the main file should ensure that the page counter
(as well as other status parameters which are stored in the |.aux| files)
takes the same value after the conditional processing.
Otherwise the page numbers may take divergent values
depending on which part is compiled.

For example, a title page could be declared by:
%
\begin{center}
\begin{tabular}{l}
|\ifchilddoc\||else|\\
|\addtocounter{page}{-1}|\\
\textit{code for title page}\\
|\newpage|\\
|\||fi|
\end{tabular}
\end{center}
%
A banner page for the child documents can be generated by:
%
\begin{center}
\begin{tabular}{l}
|\ifchilddoc|\\
|\addtocounter{page}{-1}|\\
\textit{code for banner page}\\
|\newpage|\\
|\||fi|
\end{tabular}
\end{center}
%
Here one could write a message such as:
\begin{center}
|This is the part \childdocname{} of \childdocjob{}.|
\end{center}

%%%%%%%%%%%%%%%%%%%%%%%%%%%%%%%%%%%%%%%%%%%%%%%%%%%%%%%%%%%%%%%%%%%%%%%%%%%%%%%%
\subsection{Flags}
\label{sec:flags}

The package makes it easy to generate different versions
of the main or child documents.
To this end compilation flags can be defined
and assigned different default values.
They will be particularly useful in conjunction
with the forwarding mechanism described in \secref{sec:forward}.

For example, it may be useful to have a flag |\version|
which can be set to |draft| or |final|.
The document source will contain some conditional code
depending on the value of |\version|.
Suppose further, the flag should default to |final| for the main file
and to |draft| for child files
which is a natural assignment for editing the document.
This is achieved by placing the following code
in the preamble of the main document
(below the |\childdocmain| directive):
%
\begin{center}
\begin{tabular}{l}
|\ifchilddoc|\\
|\providecommand{\version}{draft}|\\
|\||else|\\
|\providecommand{\version}{final}|\\
|\||fi|
\end{tabular}
\end{center}
%
The definition by |\providecommand| makes sure
that previous definitions are not overwritten.
Further statements |\providecommand{\version}{...}|
can thus be added before the above code to override it.

For the main file, one might add a line
(between |\childdocmain| and the above block)
%
\begin{center}
|%\ifchilddoc\||else\providecommand{\version}{draft}\||fi|
\end{center}
%
which can be uncommented to produce a draft version.
Likewise one can add a line to the very top of a child file
(above the |\childdocof{|\textit{main}|}| directive)
%
\begin{center}
|%\providecommand{\version}{final}|
\end{center}
%
which can be uncommented to produce the final version of this child document.

%%%%%%%%%%%%%%%%%%%%%%%%%%%%%%%%%%%%%%%%%%%%%%%%%%%%%%%%%%%%%%%%%%%%%%%%%%%%%%%%
\subsection{Forwarding}
\label{sec:forward}

Different versions of the main or child documents
using compilation flags as described in \secref{sec:flags}
can be (permanently) stored in different files
for convenient compilation, viewing and distribution.
To this end, the package defines a command
to pass on compilation to a different file:

%%%%%%%%%%%%%%%%%%%%%%%%%%%%%%%%%%%%%%%%
\DescribeMacro{\childdocforward}
The command |\childdocforward| redirects processing to
another source file:
%
\begin{center}
\begin{tabular}{l}
|\input{childdoc.def}|\\
|\childdocforward[|\textit{main}|]{|\textit{dest}|}|\\
\end{tabular}
\end{center}
%
The argument \textit{dest} is the destination file
(without extension).
It should be the main file or one of the child files.
Note that further \textsf{childdoc} directives
such as |\childdocof| and |\childdocforward|
in the indicated file will be processed in this form.
The optional argument \textit{main}
passes on directly to the main file \textit{main}
while pretending to compile the child \textit{dest}.
This form behaves as if \textit{dest}
issues |\childdocof{|\textit{main}|}| right away,
and no further \textsf{childdoc} directives will be processed.

%%%%%%%%%%%%%%%%%%%%%%%%%%%%%%%%%%%%%%%%
\DescribeMacro{\...prefix}
In the alternative form |\childdocforwardprefix|,
%
\begin{center}
\begin{tabular}{l}
|\input{childdoc.def}|\\
|\childdocforwardprefix[|\textit{main}|]{|\textit{prefix}|}{|\textit{dest}|}|
\end{tabular}
\end{center}
%
the destination file is determined by a pattern
depending on the current file:
To make this work, the current file must be called
`{\textit{prefix}\hspace{0.2em}\textit{suffix}}'
with \textit{prefix} matching precisely the argument.
Processing is then passed on to the file
`{\textit{dest}\hspace{0.2em}\textit{suffix}}'.
Surely, the same effect is achieved by
directly specifying the
argument `{\textit{dest}\hspace{0.2em}\textit{suffix}}'
in the first form.
However, that requires to set up a different file
for each child. With the alternative form of the command
all these files can have exactly the same content
which simplifies setting them up and maintaining them.

For example, the following file |draft.tex|
with a compilation flag |\version| as described in \secref{sec:flags}
compiles the main document as a draft:
%
\begin{center}
\begin{tabular}{l}
|\def\version{draft}|\\
|\input{childdoc.def}|\\
|\childdocforward{|\textit{main}|}|
\end{tabular}
\end{center}
%
Likewise, the following files |final|\textit{nn}|.tex|
compile the final version of the child document
|child|\textit{nn}|.tex|:
%
\begin{center}
\begin{tabular}{l}
|\def\version{final}|\\
|\input{childdoc.def}|\\
|\childdocforwardprefix{final}{child}|
\end{tabular}
\end{center}
%

Note that when several versions of a main file and/or of each child file
are to be generated, it may be convenient to set up a |Makefile| or
shell script to automatise the process.

%%%%%%%%%%%%%%%%%%%%%%%%%%%%%%%%%%%%%%%%%%%%%%%%%%%%%%%%%%%%%%%%%%%%%%%%%%%%%%%%
\subsection{Command Line Processing}
\label{sec:commandline}

The effect of redirection files can also be achieved by invoking
the \LaTeX{} compiler with a more elaborate command line.
Most conveniently this should be done as part
of a shell script or a |Makefile|.

When using \textsf{childdoc} in the main file, the following
command lines effectively perform a redirection
(note that depending on the shell being used,
backslashes may have to be doubled: `|\|' $\to$ `|\\|'):
%
\begin{center}
|... -jobname "|\textit{target}|" |\\|"|[\textit{flags}]%
|\input{childdoc.def}\childdocforward[|\textit{main}|]{|\textit{dest}|}"|
\end{center}
%
Here \textit{target} is the name of the output file,
\textit{main} is the name of the main file
and \textit{dest} is the name of the main or child file to be processed
(all filenames without extensions).
The optional argument \textit{main} can be omitted
if \textit{main} matches \textit{dest}.
Optionally, compilation \textit{flags} can be defined via |\def| commands.
This command line makes the \TeX{} engine believe
it is compiling the file \textit{target}
whose content is specified as the latter parameter.
The provided code then forwards the processing to
\textit{main} or \textit{dest} as described in \secref{sec:forward}.

%%%%%%%%%%%%%%%%%%%%%%%%%%%%%%%%%%%%%%%%%%%%%%%%%%%%%%%%%%%%%%%%%%%%%%%%%%%%%%%%
\subsection{Include by Input}
\label{sec:input}

Including child documents by |\include| has some restrictions by design.
Most notably, the content of a child document always occupies
its own set of pages; pages cannot be shared between child documents.
Usually, this behaviour makes perfect sense
because each child document contain an essential part of the document.
However, in some situations it may be desirable to compose
a document from a collection of parts
without having mandatory page breaks between then.
For this case, the package
provides a mechanism to include parts
by |\input| which can also be processed individually.
However, by construction this mechanism
requires manual handling of the content to be output.

%%%%%%%%%%%%%%%%%%%%%%%%%%%%%%%%%%%%%%%%
\DescribeMacro{\ifchilddocmanual}
The main file should be prepared as usual, see \secref{sec:include}.
However, the document body must make a distinction
between processing of an individual part and of the main document, e.g.:
%
\begin{center}
\begin{tabular}{l}
|\ifchilddocmanual|\\
|\input{\childdocname}|\\
|\||else|\\
\textit{document body with }|\input{|\textit{part}|}|\\
|\||fi|
\end{tabular}
\end{center}
%
The conditional |\ifchilddocmanual| is true whenever
a part to be included by |\input| is being compiled,
and the name of the part is stored in |\childdocname|.

%%%%%%%%%%%%%%%%%%%%%%%%%%%%%%%%%%%%%%%%
\DescribeMacro{\childdocby}
Each part to be included by |\input| should start with:
%
\begin{center}
\begin{tabular}{l}
|\input{childdoc.def}|\\
|\childdocby{|\textit{main}|}|\\
\end{tabular}
\end{center}
%
The directive |\childdocby| is similar to |\childdocof|
described in \secref{sec:include},
but the subsequent selection of content must be done manually.
To that end, both |\ifchilddoc| and |\ifchilddocmanual|
will be true upon processing of a part,
and the name of the part is stored in |\childdocname|.
Note that |\jobname| will be set to the filename of the current part
so that each part receives an individual |.aux| file
that does not interfere with the |.aux| file(s) of the main document.
This behaviour can be altered by the alternative form
|\childdocby[*]{|\textit{main}|}| (with a non-empty optional argument)
which uses the |.aux| file of the main document
by setting |\jobname| to \textit{main}.

%%%%%%%%%%%%%%%%%%%%%%%%%%%%%%%%%%%%%%%%%%%%%%%%%%%%%%%%%%%%%%%%%%%%%%%%%%%%%%%%
\subsection{Driver Development}
\label{sec:driver}

The \textsf{childdoc} mechanism can also be use for the development
of definition files such as \LaTeX{} styles or classes.
This case differs from the above setup with multiple parts
included by |\include| in that no |\includeonly| should be invoked.
This can be achieved by starting the include file
(before |\ProvidesPackage|) with:
%
\begin{center}
\begin{tabular}{l}
|\input{childdoc.def}|\\
|\childdocforward{|\textit{main}|}|\\
\end{tabular}
\end{center}
%
or alternatively with:
%
\begin{center}
\begin{tabular}{l}
|\input{childdoc.def}|\\
|\childdocby{|\textit{main}|}|\\
\end{tabular}
\end{center}
%
Both forms have slightly different effects as described above.
The main file is prepared as usual, see \secref{sec:include}.

%%%%%%%%%%%%%%%%%%%%%%%%%%%%%%%%%%%%%%%%%%%%%%%%%%%%%%%%%%%%%%%%%%%%%%%%%%%%%%%%
\subsection{Legacy Detection}
\label{sec:detection}

The directive |\childdocmain| in the main file can detect
whether the complete document or merely a child is to be compiled
even without using the directive |\childdocof|.
This method is deprecated because it is less robust
and there is no compelling reason to use it;
it is merely provided for backward compatibility
and it may be removed in future versions.

If the detection mechanism is to be used,
it is mandatory to correctly specify
the filename of the main file as the argument of |\childdocmain|:
%
\begin{center}
\begin{tabular}{l}
|\input{childdoc.def}|\\
|\childdocmain{|\textit{main}|}|\\
\end{tabular}
\end{center}
%
If |\jobname| does not match the argument \textit{main} of |\childdocmain|,
it is assumed that |\jobname| points to the child file to be compiled.
When using |\childdocmain| with the main file specified as argument,
it suffices to start a child file
with just |\input{|\textit{main}|}|
without loading of the package and using |\childdocof|.
If instead all processing is done
with the appropriate \textsf{childdoc} directives,
the argument of \textit{main} of |\childdocmain| can be empty.

An alternative version of the command line processing described
in \secref{sec:commandline} using the detection mechanism reads:
%
\begin{center}
|... -jobname "|\textit{target}|" "|[\textit{flags}]%
[|\def\jobname{|\textit{dest}|}|]|\input{|\textit{main}|}"|
\end{center}

%%%%%%%%%%%%%%%%%%%%%%%%%%%%%%%%%%%%%%%%%%%%%%%%%%%%%%%%%%%%%%%%%%%%%%%%%%%%%%%%
\subsection{Manual Code}
\label{sec:manual}

In case one cannot be certain whether the definitions file |childdoc.def|
is installed on the target \TeX{} distribution
and one prefers not to ship it,
it is conceivable to paste a few relevant commands into the sources.

To that end, drop all statements |\input{childdoc.def}|
and perform the replacements as outlined below.
Instead of |\childdocmain{|\textit{main}|}| add the following code
to the top of the main file:
%
\begin{center}
\begin{tabular}{l}
|\||ifdefined\childdocname\endinput\||fi\newif\ifchilddoc|\\
|\edef\childdocname{\scantokens\expandafter{\jobname\noexpand}}|\\
|\def\childdocmain{|\textit{main}|}\||ifx\childdocmain\childdocname\||else|\\
|\childdoctrue\includeonly{\childdocname}\let\jobname\childdocmain\||fi|\\
\end{tabular}
\end{center}
%
Instead of |\childdocof{|\textit{main}|}| just include the main file
at the top of each child file:
%
\begin{center}
|\input{|\textit{main}|}|
\end{center}
%
A simple redirection |\childdocforward{|\textit{dest}|}| is achieved by:
%
\begin{center}
|\def\jobname{|\textit{dest}|}\input{\jobname}|
\end{center}
%
The redirection with prefix
|\childdocforwardprefix[|\textit{prefix}|]{|\textit{dest}|}|
is accomplished by:
%
\begin{center}
\begin{tabular}{l}
|{\edef\jobname{\scantokens\expandafter{\jobname\noexpand}}|\\
|\def\redirectjob |\textit{prefix}|#1~~~{\gdef\jobname{|\textit{dest}|#1}}|\\
|\expandafter\redirectjob\jobname~~~}\input{\jobname}|
\end{tabular}
\end{center}

In an alternative approach,
child documents can be compiled by a specific command line
without additional code or specific definitions:
%
\begin{center}
|... -jobname "|\textit{target}|" "|[\textit{flags}]%
|\includeonly{|\textit{dest}|}\input{|\textit{main}|}"|
\end{center}
%

%%%%%%%%%%%%%%%%%%%%%%%%%%%%%%%%%%%%%%%%%%%%%%%%%%%%%%%%%%%%%%%%%%%%%%%%%%%%%%%%
%%%%%%%%%%%%%%%%%%%%%%%%%%%%%%%%%%%%%%%%%%%%%%%%%%%%%%%%%%%%%%%%%%%%%%%%%%%%%%%%
\section{Information}

%%%%%%%%%%%%%%%%%%%%%%%%%%%%%%%%%%%%%%%%%%%%%%%%%%%%%%%%%%%%%%%%%%%%%%%%%%%%%%%%
\subsection{Copyright}

Copyright \copyright{} 2017--2018 Niklas Beisert

This work may be distributed and/or modified under the
conditions of the \LaTeX{} Project Public License, either version 1.3
of this license or (at your option) any later version.
The latest version of this license is in
  \url{http://www.latex-project.org/lppl.txt}
and version 1.3 or later is part of all distributions of \LaTeX{}
version 2005/12/01 or later.

This work has the LPPL maintenance status `maintained'.

The Current Maintainer of this work is Niklas Beisert.

This work consists of the files |README.txt|, |childdoc.ins| and |childdoc.dtx|
as well as the derived files |childdoc.def|, |cdocsamp.tex|
with |cdocsch1.tex|, |cdocsch2.tex|, |cdocspt3.tex|, |cdocspt4.tex|,
|cdocsdrf.tex|, |cdocsfn1.tex|, |cdocsfn2.tex|
as well as |childdoc.pdf|.

%%%%%%%%%%%%%%%%%%%%%%%%%%%%%%%%%%%%%%%%%%%%%%%%%%%%%%%%%%%%%%%%%%%%%%%%%%%%%%%%
\subsection{Files and Installation}

The package consists of the files:
%
\begin{center}
\begin{tabular}{ll}
    |README.txt|   & readme file \\
    |childdoc.ins| & installation file \\
    |childdoc.dtx| & source file \\
    |childdoc.def| & definition file \\
    |cdocsamp.tex| & sample main file \\
    |cdocsch1.tex| & sample include file \\
    |cdocsch2.tex| & sample include file \\
    |cdocspt3.tex| & sample part file \\
    |cdocspt4.tex| & sample part file \\
    |cdocsdrf.tex| & sample redirection file \\
    |cdocsfn1.tex| & sample redirection file \\
    |cdocsfn2.tex| & sample redirection file \\
    |childdoc.pdf| & manual
\end{tabular}
\end{center}
%
The distribution consists of the files
|README.txt|, |childdoc.ins| and |childdoc.dtx|.
%
\begin{itemize}
\item
Run (pdf)\LaTeX{} on |childdoc.dtx|
to compile the manual |childdoc.pdf| (this file).
\item
Run \LaTeX{} on |childdoc.ins| to create the definitions file |childdoc.def|
and the sample |cdocsamp.tex| with include files
|cdocsch1.tex|, |cdocsch2.tex|, |cdocspt3.tex|, |cdocspt4.tex|,
|cdocsdrf.tex|, |cdocsfn1.tex|, |cdocsfn2.tex|.
Then copy the file |childdoc.def| to an appropriate directory of your \LaTeX{}
distribution, e.g.\ \textit{texmf-root}|/tex/latex/childdoc|.
\end{itemize}

%%%%%%%%%%%%%%%%%%%%%%%%%%%%%%%%%%%%%%%%%%%%%%%%%%%%%%%%%%%%%%%%%%%%%%%%%%%%%%%%
\subsection{Related CTAN Packages}

There are several other packages which offer a similar functionality:
%
\begin{itemize}
\item
The packages
\href{http://ctan.org/pkg/docmute}{\textsf{docmute}},
\href{http://ctan.org/pkg/includex}{\textsf{includex}} and
\href{http://ctan.org/pkg/standalone}{\textsf{standalone}}
provide commands to include only the document body of
a child file thus allowing both files to be compiled individually.
\item
The packages \href{http://ctan.org/pkg/subdocs}{\textsf{subdocs}}
and \href{http://ctan.org/pkg/subfiles}{\textsf{subfiles}}
provide structures in which the main and child documents can be
encapsulated and allowing them to be compiled individually.
The inclusion mechanism is different from the conventional |\include|.
\item
The package \href{http://ctan.org/pkg/combine}{\textsf{combine}}
is an elaborate solution to combine several documents into one.
\end{itemize}
%
See also the CTAN topic \href{http://ctan.org/topic/subdocs}{\textsf{subdocs}}
for further related packages.
The present package differs from the above solutions in that
a document structure constructed with the conventional |\include| mechanism
just needs two extra commands at the top of every file
such that all constituent files can be compiled individually.

%%%%%%%%%%%%%%%%%%%%%%%%%%%%%%%%%%%%%%%%%%%%%%%%%%%%%%%%%%%%%%%%%%%%%%%%%%%%%%%%
%\subsection{Feature Suggestions}
%
%The following is a list of features which may be useful for future
%versions of this package:
%%
%\begin{itemize}
%\item
%\ldots
%\end{itemize}

%%%%%%%%%%%%%%%%%%%%%%%%%%%%%%%%%%%%%%%%%%%%%%%%%%%%%%%%%%%%%%%%%%%%%%%%%%%%%%%%
\subsection{Revision History}

%%%%%%%%%%%%%%%%%%%%%%%%%%%%%%%%%%%%%%%%
\paragraph{v2.0:} 2018/12/30

\begin{itemize}
\item
immediate forward processing
\item
added |\childdocby| mechanism
\item
manual restructured
\end{itemize}

%%%%%%%%%%%%%%%%%%%%%%%%%%%%%%%%%%%%%%%%
\paragraph{v1.6:} 2018/01/17

\begin{itemize}
\item
application for development of include files
\item
corrections to manual
\end{itemize}

%%%%%%%%%%%%%%%%%%%%%%%%%%%%%%%%%%%%%%%%
\paragraph{v1.5:} 2017/05/21

\begin{itemize}
\item
more complete structuring introduced
\item
|\childdocof| introduced
\item
|\childdoc| renamed to |\childdocmain|
\item
|\childredirect| renamed to |\childdocforward| and |\childdocforwardprefix|
and functionality expanded
\end{itemize}

%%%%%%%%%%%%%%%%%%%%%%%%%%%%%%%%%%%%%%%%
\paragraph{v1.0:} 2017/04/27

\begin{itemize}
\item
manual and install package
\item
first version published on CTAN
\end{itemize}

%%%%%%%%%%%%%%%%%%%%%%%%%%%%%%%%%%%%%%%%
\paragraph{v0.6:} 2017/04/26

\begin{itemize}
\item
redirection mechanism added
\end{itemize}

%%%%%%%%%%%%%%%%%%%%%%%%%%%%%%%%%%%%%%%%
\paragraph{v0.5:} 2017/04/26

\begin{itemize}
\item
functionality in definition file
\end{itemize}


%%%%%%%%%%%%%%%%%%%%%%%%%%%%%%%%%%%%%%%%%%%%%%%%%%%%%%%%%%%%%%%%%%%%%%%%%%%%%%%%
%%%%%%%%%%%%%%%%%%%%%%%%%%%%%%%%%%%%%%%%%%%%%%%%%%%%%%%%%%%%%%%%%%%%%%%%%%%%%%%%
%%%%%%%%%%%%%%%%%%%%%%%%%%%%%%%%%%%%%%%%%%%%%%%%%%%%%%%%%%%%%%%%%%%%%%%%%%%%%%%%
\appendix

\settowidth\MacroIndent{\rmfamily\scriptsize 000\ }

 \DocInput{childdoc.dtx}

\end{document}
%</driver>
% \fi
%
% %%%%%%%%%%%%%%%%%%%%%%%%%%%%%%%%%%%%%%%%%%%%%%%%%%%%%%%%%%%%%%%%%%%%%%%%%%%%%%
% %%%%%%%%%%%%%%%%%%%%%%%%%%%%%%%%%%%%%%%%%%%%%%%%%%%%%%%%%%%%%%%%%%%%%%%%%%%%%%
% \section{Sample}
%\iffalse
%<*samplemain>
%\fi
%
% The following presents a sample document
% with two chapters, two parts, a title page,
% a compile flag as well as three forwarding files to set the flag.
% It consists of eight |.tex| files:
% \begin{center}
% \begin{tabular}{ll}
% |cdocsamp.tex|&main file\\
% |cdocsch1.tex|&include file for chapter 1\\
% |cdocsch2.tex|&include file for chapter 2\\
% |cdocspt3.tex|&include file for part 3\\
% |cdocspt4.tex|&include file for part 4\\
% |cdocsdrf.tex|&forwarding file for main file in draft mode\\
% |cdocsfi1.tex|&forwarding file for final version of chapter 1\\
% |cdocsfi2.tex|&forwarding file for final version of chapter 2\\
% \end{tabular}
% \end{center}
% Each of the eight files can be compiled directly by the \LaTeX{} compiler.
%
% %%%%%%%%%%%%%%%%%%%%%%%%%%%%%%%%%%%%%%
% \paragraph{Main File.}
%
% The main file is called |cdocsamp.tex|.
%
% Load the \textsf{childdoc} definitions and
% declare the filename for the main document:
%    \begin{macrocode}
\input{childdoc.def}
\childdocmain{}
%    \end{macrocode}

% Optional override for |\version| flag:
%    \begin{macrocode}
%%\ifchilddoc\else\providecommand{\version}{draft}\fi
%    \end{macrocode}

% Define the default values for the |\version| flag
% (|final| for the main file and |draft| for childs):
%    \begin{macrocode}
\ifchilddoc
\providecommand{\version}{draft}
\else
\providecommand{\version}{final}
\fi
%    \end{macrocode}

% Load the standard document class:
%    \begin{macrocode}
\documentclass[12pt]{article}
%    \end{macrocode}

% Start the document body:
%    \begin{macrocode}
\begin{document}
%    \end{macrocode}

% Declare a title page.
% Print title, part of document being processed and version flag:
%    \begin{macrocode}
\addtocounter{page}{-1}
\begin{center}
{\LARGE\bfseries{}childdoc example\par}
\vspace{1cm}
\ifchilddoc
\ifchilddocmanual part\else chapter\fi:
`\childdocname' of `\childdocjob'\par
\else
main document: `\childdocjob'\par
\fi
version: \version\par
\end{center}
\newpage
%    \end{macrocode}

% Manually include selected file,
% otherwise process as usual:
%    \begin{macrocode}
\ifchilddocmanual
\section*{part `\childdocname'}
\input{\childdocname}
\else
%    \end{macrocode}

% Include the two chapters:
%    \begin{macrocode}
\include{cdocsch1}
\include{cdocsch2}
%    \end{macrocode}

% Include the two parts unless only chapters should be displayed:
%    \begin{macrocode}
\ifchilddoc\else
\section{part three}
\input{cdocspt3}
\section{part four}
\input{cdocspt4}
\fi
%    \end{macrocode}

% Process as usual until here:
%    \begin{macrocode}
\fi
%    \end{macrocode}

% End of document body:
%    \begin{macrocode}
\end{document}
%    \end{macrocode}
%\iffalse
%</samplemain>
%\fi
%
% %%%%%%%%%%%%%%%%%%%%%%%%%%%%%%%%%%%%%%
% \paragraph{Chapter Include Files.}
%
% The include files are called |cdocsch1.tex| and |cdocsch2.tex|.
%
%\iffalse
%<*samplechap1|samplechap2>
%\fi

% Optional override for |\version| flag:
%    \begin{macrocode}
%%\providecommand{\version}{final}
%    \end{macrocode}

% Include the main document:
%    \begin{macrocode}
\input{childdoc.def}
\childdocof{cdocsamp}
%    \end{macrocode}

%\iffalse
%</samplechap1|samplechap2>
%\fi
%
%\iffalse
%<*samplechap1>
%\fi
% Some text for chapter 1:
%    \begin{macrocode}
\section{one}
some text in chapter one
%    \end{macrocode}

%\iffalse
%</samplechap1>
%\fi
% Some text for chapter 2:
%\iffalse
%<*samplechap2>
%\fi
%    \begin{macrocode}
\section{two}
more text in chapter two
%    \end{macrocode}

%\iffalse
%</samplechap2>
%\fi
%
% %%%%%%%%%%%%%%%%%%%%%%%%%%%%%%%%%%%%%%
% \paragraph{Part Include Files.}
%
% The include files are called |cdocspt3.tex| and |cdocspt4.tex|.
%
%\iffalse
%<*samplepart3|samplepart4>
%\fi

% Optional override for |\version| flag:
%    \begin{macrocode}
%%\providecommand{\version}{final}
%    \end{macrocode}

% Include the main document:
%    \begin{macrocode}
\input{childdoc.def}
\childdocby{cdocsamp}
%    \end{macrocode}

%\iffalse
%</samplepart3|samplepart4>
%\fi
%
%\iffalse
%<*samplepart3>
%\fi
% Some text for part 3:
%    \begin{macrocode}
some text in part three
%    \end{macrocode}

%\iffalse
%</samplepart3>
%\fi
% Some text for part 4:
%\iffalse
%<*samplepart4>
%\fi
%    \begin{macrocode}
more text in part four
%    \end{macrocode}

%\iffalse
%</samplepart4>
%\fi
%
% %%%%%%%%%%%%%%%%%%%%%%%%%%%%%%%%%%%%%%
% \paragraph{Forwarding for a Complete Draft.}
%
% The following forwarding file |cdocsdrf.tex|
% compiles the main document in draft mode:
%\iffalse
%<*sampledraft>
%\fi
%    \begin{macrocode}
\def\version{draft}
\input{childdoc.def}
\childdocforward{cdocsamp}
%    \end{macrocode}

%\iffalse
%</sampledraft>
%\fi
%
% %%%%%%%%%%%%%%%%%%%%%%%%%%%%%%%%%%%%%%
% \paragraph{Forwarding for Final Version of the Chapters.}
%
% The following forwarding files |cdocsfn1.tex| and |cdocsfn2.tex|
% (with identical content)
% compile the final versions of the child documents
% |cdocsch1.tex| and |cdocsch2.tex|, respectively:
%\iffalse
%<*samplefinal>
%\fi
%    \begin{macrocode}
\def\version{final}
\input{childdoc.def}
\childdocforwardprefix[cdocsamp]{cdocsfn}{cdocsch}
%    \end{macrocode}

%\iffalse
%</samplefinal>
%\fi
%
% %%%%%%%%%%%%%%%%%%%%%%%%%%%%%%%%%%%%%%
% \paragraph{Command Line Processing.}
%
% The following three command lines generate the output files
% |cdocscld|, |cdocscl1| and |cdocscl2|
% which should be identical to
% |cdocsdrf|, |cdocsch1| and |cdocsfn2|, respectively:
% \begin{center}
% \begin{tabular}{l}
% |latex -jobname cdocscld \|\\
% |  "\def\version{draft}\input{childdoc.def}\childdocforward{cdocsamp}"|\\
% |latex -jobname cdocscl1 \|\\
% |  "\input{childdoc.def}\childdocforward[cdocsamp]{cdocsch1}"|\\
% |latex -jobname cdocscl2 \|\\
% |  "\def\version{final}\input{childdoc.def}\childdocforward{cdocsch2}"|
% \end{tabular}
% \end{center}
% Note that the trailing backslash on each first line
% merely continues the input to the second line
% (for convenient cut ant paste).
% Furthermore, the command |latex| can be replaced by any
% of its alternative versions such as |pdflatex|.
%
% %%%%%%%%%%%%%%%%%%%%%%%%%%%%%%%%%%%%%%%%%%%%%%%%%%%%%%%%%%%%%%%%%%%%%%%%%%%%%%
% %%%%%%%%%%%%%%%%%%%%%%%%%%%%%%%%%%%%%%%%%%%%%%%%%%%%%%%%%%%%%%%%%%%%%%%%%%%%%%
% \section{Implementation}
%\iffalse
%<*package>
%\fi
%
% This section describes the definitions file |childdoc.def|.

% The definitions cannot be loaded using |\usepackage| or |\RequirePackage|
% which has a mechanism to prevent loading a style file more than once.
% When loading the definitions by means of |\input|
% multiple instances have to be prevented manually:
%\iffalse
%This code needs to be before the `\ProvidesFile' directive
%which is defined at the beginning of this file.
%Therefore it is also placed there and commented out here.
%</package>
%<*discard>
%\fi
%    \begin{macrocode}
\ifdefined\childdocmain\endinput\fi
%    \end{macrocode}
%\iffalse
%</discard>
%<*package>
%\fi
%
% \macro{\ifchilddoc}
% \macro{\ifchilddocmanual}
% The conditional |\ifchilddoc| tells whether a
% child (true) or main (false) document is being compiled.
% The conditional |\ifchilddocmanual| tells whether
% the |\includeonly| mechanism is used (false) or
% the selection of child files must be performed manually (true).
% The definitions initialise to false:
%    \begin{macrocode}
\newif\ifchilddoc
\newif\ifchilddocmanual
%    \end{macrocode}

% \macro{\childdocname}
% \macro{\childdocjob}
% The macro |\childdocname| stores the name of the main document
% to be compiled. The macro |\childdocjob| stores the name of
% the document on which the \LaTeX{} compiler was originally invoked.
% The content of |\jobname| cannot be compared
% to filenames specified in the source due to different catcodes.
% The following code rescans |\jobname|, stores the result
% in |\childdocname| and saves a copy in |\childdocjob|:
%    \begin{macrocode}
\edef\childdocname{\scantokens\expandafter{\jobname\noexpand}}
\let\childdocjob\childdocname
%    \end{macrocode}

% \macro{\childdocdisable}
% The macro |\childdocdisable| prevents the main file
% from being processed more than once.
% At this stage, the main document command |\childdocmain|
% is assumed to be called once again where it should do nothing.
% Any subsequent call to it should prevent
% a secondary processing of the main document
% It overwrites the forwarding commands
% |\childdocof| and |\childdocforward|
% with empty macros to prevent further inclusions of the main document:
%    \begin{macrocode}
\newcommand{\childdocdisable}
{
  \renewcommand{\childdocmain}[1]{\renewcommand{\childdocmain}[1]{\endinput}}
  \renewcommand{\childdocof}[1]{}
  \renewcommand{\childdocby}[2][]{}
  \renewcommand{\childdocforward}[2][]{}
  \renewcommand{\childdocdisable}{}
}
%    \end{macrocode}

% \macro{\childdocmain}
% The macro |\childdocmain| is to be called at the top of the main file
% with nothing or the main filename (without extension) as argument.
% First, it breaks loops.
% If the argument is not empty and does not match |\childdocname|
% (which is set by the first inclusion of |childdoc.def|),
% |\ifchilddoc| is set to true, |\includeonly| is applied to the child file
% and |\jobname| is set to the main file
% (for proper handling of |.aux| files):
%    \begin{macrocode}
\newcommand{\childdocmain}[1]
{
  \childdocdisable\childdocmain{}
  \if?#1?\else
    \begingroup
      \def\childdoctmp{#1}
      \ifx\childdoctmp\childdocname
        \def\childdoctmp{}
      \else
        \def\childdoctmp
        {
          \childdoctrue
          \includeonly{\childdocname}
          \def\childdocjob{#1}
          \def\jobname{#1}
        }
      \fi
      \expandafter
    \endgroup
    \childdoctmp
  \fi
}
%    \end{macrocode}

% \macro{\childdocof}
% The command |\childdocof| redirects
% compilation to the main file |#1|.
%    \begin{macrocode}
\newcommand{\childdocof}[1]
{
  \childdocdisable
  \childdoctrue
  \includeonly{\childdocname}
  \def\jobname{#1}
  \def\childdocjob{#1}
  \input{#1}
}
%    \end{macrocode}

% \macro{\childdocby}
% The command |\childdocby| ....
%    \begin{macrocode}
\newcommand{\childdocby}[2][]
{
  \childdocdisable
  \childdoctrue
  \childdocmanualtrue
  \if?#1?\else
    \def\jobname{#2}
  \fi
  \def\childdocjob{#2}
  \input{#2}
  \endinput
}
%    \end{macrocode}

% \macro{\childdocforward}
% The command |\childdocforward| redirects
% compilation to the main file or
% (if the optional argument is given) a child file.
% Parameters are set as if the main file
% or a child file starting with |\childdocof| was compiled.
% Then compilation is handed over to the main file:
%    \begin{macrocode}
\newcommand{\childdocforward}[2][]
{
  \begingroup
    \if?#1?
      \def\childdoctmp
      {
        \def\childdocname{#2}
        \def\childdocjob{#2}
        \def\jobname{#2}
        \input{#2}
        \endinput
      }
    \else
      \def\childdoctmp
      {
        \childdocdisable
        \def\childdocname{#2}
        \childdoctrue
        \includeonly{#2}
        \def\childdocjob{#1}
        \def\jobname{#1}
        \input{#1}
        \endinput
      }
    \fi
    \expandafter
  \endgroup
  \childdoctmp
}
%    \end{macrocode}

% \macro{\childdocforwardprefix}
% The command |\childdocforwardprefix| redirects
% compilation to the main or a child file by means of a pattern.
% The prefix |#1| in the current filename is replaced by |#2|
% and the suffix of the current filename is kept
% (it is assumed that the filename does not contain the substring `|~~~|'
% which is used as a delimiter).
% Compilation is handed over to the new file by |\childdocforward|:
%    \begin{macrocode}
\newcommand{\childdocforwardprefix}[3][]
{
  \begingroup
    \def\childdocextract #2##1~~~{\def\childdoctmp{\childdocforward[#1]{#3##1}}}
    \expandafter\childdocextract\childdocname~~~
    \expandafter
  \endgroup
  \childdoctmp
}
%    \end{macrocode}

% \macro{\childdoc}
% The deprecated macro |\childdoc| is a legacy version of |\childdocmain|:
%    \begin{macrocode}
\newcommand{\childdoc}{\childdocmain}
%    \end{macrocode}

% \macro{\childdocredirect}
% The deprecated macro |\childdocredirect| is a legacy version
% of |\childdocforward| and |\childdocforwardprefix|:
%    \begin{macrocode}
\newcommand{\childdocredirect}[2][]
{
  \begingroup
    \if?#1?
      \def\childdoctmp{\childdocforward{#2}}
    \else
      \def\childdoctmp{\childdocforwardprefix{#1}{#2}}
    \fi
    \expandafter
  \endgroup
  \childdoctmp
}
%    \end{macrocode}

%\iffalse
%</package>
%\fi
%
\endinput
|\\
|\childdocby{|\textit{main}|}|\\
\end{tabular}
\end{center}
%
Both forms have slightly different effects as described above.
The main file is prepared as usual, see \secref{sec:include}.

%%%%%%%%%%%%%%%%%%%%%%%%%%%%%%%%%%%%%%%%%%%%%%%%%%%%%%%%%%%%%%%%%%%%%%%%%%%%%%%%
\subsection{Legacy Detection}
\label{sec:detection}

The directive |\childdocmain| in the main file can detect
whether the complete document or merely a child is to be compiled
even without using the directive |\childdocof|.
This method is deprecated because it is less robust
and there is no compelling reason to use it;
it is merely provided for backward compatibility
and it may be removed in future versions.

If the detection mechanism is to be used,
it is mandatory to correctly specify
the filename of the main file as the argument of |\childdocmain|:
%
\begin{center}
\begin{tabular}{l}
|% \iffalse
%
% childdoc.dtx Copyright (C) 2017-2018 Niklas Beisert
%
% This work may be distributed and/or modified under the
% conditions of the LaTeX Project Public License, either version 1.3
% of this license or (at your option) any later version.
% The latest version of this license is in
%   http://www.latex-project.org/lppl.txt
% and version 1.3 or later is part of all distributions of LaTeX
% version 2005/12/01 or later.
%
% This work has the LPPL maintenance status `maintained'.
%
% The Current Maintainer of this work is Niklas Beisert.
%
% This work consists of the files childdoc.dtx and childdoc.ins
% and the derived files childdoc.def and cdocsamp.tex with
% cdocsch1.tex, cdocsch2.tex, cdocsdrf.tex, cdocsfn1.tex, cdocsfn2.tex.
%
%<package>\ifdefined\childdocmain\endinput\fi
%<package>\ProvidesFile{childdoc.def}[2018/12/30 v2.0 child document driver]
%<samplemain>\ProvidesFile{cdocsamp.tex}[2018/12/30 v2.0 sample for childdoc]
%<*driver>
%\ProvidesFile{childdoc.drv}[2018/12/30 v2.0 childdoc reference manual file]
\PassOptionsToClass{10pt,a4paper}{article}
\documentclass{ltxdoc}

\usepackage[margin=35mm]{geometry}
\usepackage{hyperref}
\usepackage{hyperxmp}
\usepackage[usenames]{color}

\hypersetup{colorlinks=true}
\hypersetup{pdfstartview=FitH}
\hypersetup{pdfpagemode=UseNone}
\hypersetup{pdfsource={}}
\hypersetup{pdflang={en-UK}}
\hypersetup{pdfcopyright={Copyright 2017-2018 Niklas Beisert.
  This work may be distributed and/or modified under the
  conditions of the LaTeX Project Public License, either version 1.3
  of this license or (at your option) any later version.}}
\hypersetup{pdflicenseurl={http://www.latex-project.org/lppl.txt}}
\hypersetup{pdfcontactaddress={ETH Zurich, ITP, HIT K,
  Wolfgang-Pauli-Strasse 27}}
\hypersetup{pdfcontactpostcode={8093}}
\hypersetup{pdfcontactcity={Zurich}}
\hypersetup{pdfcontactcountry={Switzerland}}
\hypersetup{pdfcontactemail={nbeisert@itp.phys.ethz.ch}}
\hypersetup{pdfcontacturl={http://people.phys.ethz.ch/\xmptilde nbeisert/}}

\newcommand{\secref}[1]{\hyperref[#1]{section \ref*{#1}}}

\parskip1ex
\parindent0pt
\let\olditemize\itemize
\def\itemize{\olditemize\parskip0pt}

\begin{document}

\title{The \textsf{childdoc} Package}
\hypersetup{pdftitle={The childdoc Package}}
\author{Niklas Beisert\\[2ex]
  Institut f\"ur Theoretische Physik\\
  Eidgen\"ossische Technische Hochschule Z\"urich\\
  Wolfgang-Pauli-Strasse 27, 8093 Z\"urich, Switzerland\\[1ex]
  \href{mailto:nbeisert@itp.phys.ethz.ch}
  {\texttt{nbeisert@itp.phys.ethz.ch}}}
\hypersetup{pdfauthor={Niklas Beisert}}
\hypersetup{pdfsubject={Manual for the LaTeX2e Package childdoc}}
\date{30 December 2018, \textsf{v2.0}}
\maketitle

\begin{abstract}\noindent
\textsf{childdoc} is a \LaTeXe{} package
that enables the direct compilation
of document sections included by |\include|
to individual files.
\end{abstract}

\begingroup
\parskip0ex
\tableofcontents
\endgroup

%%%%%%%%%%%%%%%%%%%%%%%%%%%%%%%%%%%%%%%%%%%%%%%%%%%%%%%%%%%%%%%%%%%%%%%%%%%%%%%%
%%%%%%%%%%%%%%%%%%%%%%%%%%%%%%%%%%%%%%%%%%%%%%%%%%%%%%%%%%%%%%%%%%%%%%%%%%%%%%%%
\section{Introduction}

\LaTeX{} provides a mechanism to structure a large document (such as a book)
into a main file and several child files (containing the chapters)
using the |\include| command.
This mechanism is beneficial for documents
which span hundreds of pages in order to
make the source file(s) more manageable.
Moreover, compilation can be restricted to
selected child files by means of the |\includeonly| command.
The latter feature can be used to reduce the compilation time while editing
(this was significantly more useful in the earlier days of \LaTeX{})
or to generate a smaller document which is easier to navigate.
Another application of |\includeonly| is to generate
documents consisting of selected parts of the complete document.

However, there are a few drawbacks of the plain |\include| mechanism:
\begin{itemize}
\item
The child files cannot be compiled on their own,
they can only be compiled via the main file.
A naive editing environment
(such as a text editor with an option
to have the current file processed by \LaTeX)
may require one to switch to the main file before compiling;
attempting to compile the child file produces errors.
\item
The main file must be modified (each time)
to adjust the |\includeonly| command
to the present needs. This easily leaves the main file in a messy state.
\item
The generated document will always carry the filename
of the main document. This is inconvenient if
several child files are to be compiled and
to be kept for distribution.
\end{itemize}

The present package provides a simple interface
to make child files individually compilable by \LaTeX{}.
Compiling a child file then has the same effect as compiling
the main file with an |\includeonly| command
to select the appropriate child.
Moreover the generated document will carry the name of the child
rather than the main file.
This resolves all three above issues.

This feature is meant to make the editing of books,
thesis documents and lecture notes somewhat more convenient.
However, the package can also be used efficiently for
composing a series of documents (such as exercise sheets)
which are typically distributed individually.
It then assists the author in generating the individual documents
(potentially in different versions)
as well as a document containing the collected series.
Another application is in developing style files
or other kinds of included material
where compilation of the style file could redirect
to a sample or test file.

%%%%%%%%%%%%%%%%%%%%%%%%%%%%%%%%%%%%%%%%%%%%%%%%%%%%%%%%%%%%%%%%%%%%%%%%%%%%%%%%
%%%%%%%%%%%%%%%%%%%%%%%%%%%%%%%%%%%%%%%%%%%%%%%%%%%%%%%%%%%%%%%%%%%%%%%%%%%%%%%%
\section{Usage}

First of all, the package \textsf{childdoc} is \emph{not} a standard
\LaTeXe{} |.sty| style file! Therefore it needs to be invoked in
a non-standard way.

%%%%%%%%%%%%%%%%%%%%%%%%%%%%%%%%%%%%%%%%%%%%%%%%%%%%%%%%%%%%%%%%%%%%%%%%%%%%%%%%
\subsection{Included Files}
\label{sec:include}

%%%%%%%%%%%%%%%%%%%%%%%%%%%%%%%%%%%%%%%%
\DescribeMacro{\childdocmain}
To use the package, add the commands
\begin{center}
\begin{tabular}{l}
|\input{childdoc.def}|\\
|\childdocmain{}|\\
\end{tabular}
\end{center}
at the very top of the main \LaTeX{} file,
in particular \emph{before} the |\documentclass| statement!
The argument of |\childdocmain| should be left empty
(but it must be present).

%%%%%%%%%%%%%%%%%%%%%%%%%%%%%%%%%%%%%%%%
\DescribeMacro{\childdocof}
Furthermore, add the commands
\begin{center}
\begin{tabular}{l}
|\input{childdoc.def}|\\
|\childdocof{|\textit{main}|}|\\
\end{tabular}
\end{center}
at the top of every child file \textit{child}
which is included by |\include{|\textit{child}|}|
from within the main file
(or at least for those files to be compiled individually).
The argument \textit{main} must be the filename of the main file.

There are a couple of
considerations in setting up the main and child documents:

%%%%%%%%%%%%%%%%%%%%%%%%%%%%%%%%%%%%%%%%
\paragraph{Restrictions.}

Please note the following restrictions:
\begin{itemize}
\item
|\childdocmain| must be called with one argument \textit{main}
to ensure compatibility with earlier version of the package.
It must either be empty (|\childdocmain{}|)
or precisely match the filename of the main file in which it is specified.
See \secref{sec:detection} for further information.
\item
The filename \textit{main} must be specified without the |.tex| extension.
\item
The filename \textit{main} is case sensitive
(even in case-insensitive file systems)
due to internal string comparison.
\item
The argument \textit{main} should be fully expanded, it cannot be a macro.
\item
Subdirectories and special characters should be avoided in filenames.
\item
The command |\childdocmain{|\textit{main}|}| must be followed by a whitespace.
It should not be followed immediately by another command
or by a comment mark `|%|'.
This is because the \TeX{} parser reads the token immediately following
the argument of |\childdocmain| and puts it
at the beginning of every child section;
however, a white\-space is ignored.
\end{itemize}

%%%%%%%%%%%%%%%%%%%%%%%%%%%%%%%%%%%%%%%%
\paragraph{Content of Main File.}

It is advisable to place all content in the child files included by |\include|.
Any output contained in the main file will appear in all child documents
unless suppressed manually;
it cannot be suppressed automatically by the |\includeonly| directive
and thus should normally be avoided.
A method to include some content in the main file
by means of conditional processing is described in \secref{sec:conditional}.

%%%%%%%%%%%%%%%%%%%%%%%%%%%%%%%%%%%%%%%%
\paragraph{Page Numbering.}

When only a part of the document is compiled,
the appropriate numbering of pages
(as well as other status parameters)
is determined from the |.aux| files.
The latter contain information from previous passes.
However this information needs to propagate through
all intermediate child documents.
Therefore the page numbering in child documents may well
be inconsistent until the complete document is compiled at least once.

A useful (if unconventional) way to always ensure a consistent
page numbering is to restart the numbering in each child document
and denote the pages by `\textit{child}|.|\textit{page}'
where \textit{child} represents the chapter/section number of the child file.
This can be achieved by the command
|\numberwithin{page}{|\textit{child}|}|
of the \textsf{amsmath} package
where \textit{child} can be |chapter| or |section|
depending on the chosen structuring.
Alternatively, one can modify the macro |\thepage| appropriately
and reset the counter |page| at the start of each child file.

%%%%%%%%%%%%%%%%%%%%%%%%%%%%%%%%%%%%%%%%%%%%%%%%%%%%%%%%%%%%%%%%%%%%%%%%%%%%%%%%
\subsection{Conditional Processing}
\label{sec:conditional}

The package provides a mechanism to compile different versions
of a document. To customise the versions further some conditional processing
can come in handy to distinguish which version is being compiled.
The package provides two macros to describe the compilation context:

%%%%%%%%%%%%%%%%%%%%%%%%%%%%%%%%%%%%%%%%
\DescribeMacro{\ifchilddoc}
The conditional |\ifchilddoc| distinguishes between the compilation of
child documents and the main document:
%
\begin{center}
|\ifchilddoc |\textit{child-code}| |[|\||else |\textit{main-code}]| \||fi|
\end{center}

%%%%%%%%%%%%%%%%%%%%%%%%%%%%%%%%%%%%%%%%
\DescribeMacro{\childdocname}
\DescribeMacro{\childdocjob}
The macro |\childdocname| contains the filename (without extension)
of the main or child file being processed.
Note that |\childdocjob| will always contain the name of the main file.

%%%%%%%%%%%%%%%%%%%%%%%%%%%%%%%%%%%%%%%%
\paragraph{Title Page.}

Conditional processing can be used to include a title or banner page
in the main document when proper precautions are taken.
Importantly, the code in the main file should ensure that the page counter
(as well as other status parameters which are stored in the |.aux| files)
takes the same value after the conditional processing.
Otherwise the page numbers may take divergent values
depending on which part is compiled.

For example, a title page could be declared by:
%
\begin{center}
\begin{tabular}{l}
|\ifchilddoc\||else|\\
|\addtocounter{page}{-1}|\\
\textit{code for title page}\\
|\newpage|\\
|\||fi|
\end{tabular}
\end{center}
%
A banner page for the child documents can be generated by:
%
\begin{center}
\begin{tabular}{l}
|\ifchilddoc|\\
|\addtocounter{page}{-1}|\\
\textit{code for banner page}\\
|\newpage|\\
|\||fi|
\end{tabular}
\end{center}
%
Here one could write a message such as:
\begin{center}
|This is the part \childdocname{} of \childdocjob{}.|
\end{center}

%%%%%%%%%%%%%%%%%%%%%%%%%%%%%%%%%%%%%%%%%%%%%%%%%%%%%%%%%%%%%%%%%%%%%%%%%%%%%%%%
\subsection{Flags}
\label{sec:flags}

The package makes it easy to generate different versions
of the main or child documents.
To this end compilation flags can be defined
and assigned different default values.
They will be particularly useful in conjunction
with the forwarding mechanism described in \secref{sec:forward}.

For example, it may be useful to have a flag |\version|
which can be set to |draft| or |final|.
The document source will contain some conditional code
depending on the value of |\version|.
Suppose further, the flag should default to |final| for the main file
and to |draft| for child files
which is a natural assignment for editing the document.
This is achieved by placing the following code
in the preamble of the main document
(below the |\childdocmain| directive):
%
\begin{center}
\begin{tabular}{l}
|\ifchilddoc|\\
|\providecommand{\version}{draft}|\\
|\||else|\\
|\providecommand{\version}{final}|\\
|\||fi|
\end{tabular}
\end{center}
%
The definition by |\providecommand| makes sure
that previous definitions are not overwritten.
Further statements |\providecommand{\version}{...}|
can thus be added before the above code to override it.

For the main file, one might add a line
(between |\childdocmain| and the above block)
%
\begin{center}
|%\ifchilddoc\||else\providecommand{\version}{draft}\||fi|
\end{center}
%
which can be uncommented to produce a draft version.
Likewise one can add a line to the very top of a child file
(above the |\childdocof{|\textit{main}|}| directive)
%
\begin{center}
|%\providecommand{\version}{final}|
\end{center}
%
which can be uncommented to produce the final version of this child document.

%%%%%%%%%%%%%%%%%%%%%%%%%%%%%%%%%%%%%%%%%%%%%%%%%%%%%%%%%%%%%%%%%%%%%%%%%%%%%%%%
\subsection{Forwarding}
\label{sec:forward}

Different versions of the main or child documents
using compilation flags as described in \secref{sec:flags}
can be (permanently) stored in different files
for convenient compilation, viewing and distribution.
To this end, the package defines a command
to pass on compilation to a different file:

%%%%%%%%%%%%%%%%%%%%%%%%%%%%%%%%%%%%%%%%
\DescribeMacro{\childdocforward}
The command |\childdocforward| redirects processing to
another source file:
%
\begin{center}
\begin{tabular}{l}
|\input{childdoc.def}|\\
|\childdocforward[|\textit{main}|]{|\textit{dest}|}|\\
\end{tabular}
\end{center}
%
The argument \textit{dest} is the destination file
(without extension).
It should be the main file or one of the child files.
Note that further \textsf{childdoc} directives
such as |\childdocof| and |\childdocforward|
in the indicated file will be processed in this form.
The optional argument \textit{main}
passes on directly to the main file \textit{main}
while pretending to compile the child \textit{dest}.
This form behaves as if \textit{dest}
issues |\childdocof{|\textit{main}|}| right away,
and no further \textsf{childdoc} directives will be processed.

%%%%%%%%%%%%%%%%%%%%%%%%%%%%%%%%%%%%%%%%
\DescribeMacro{\...prefix}
In the alternative form |\childdocforwardprefix|,
%
\begin{center}
\begin{tabular}{l}
|\input{childdoc.def}|\\
|\childdocforwardprefix[|\textit{main}|]{|\textit{prefix}|}{|\textit{dest}|}|
\end{tabular}
\end{center}
%
the destination file is determined by a pattern
depending on the current file:
To make this work, the current file must be called
`{\textit{prefix}\hspace{0.2em}\textit{suffix}}'
with \textit{prefix} matching precisely the argument.
Processing is then passed on to the file
`{\textit{dest}\hspace{0.2em}\textit{suffix}}'.
Surely, the same effect is achieved by
directly specifying the
argument `{\textit{dest}\hspace{0.2em}\textit{suffix}}'
in the first form.
However, that requires to set up a different file
for each child. With the alternative form of the command
all these files can have exactly the same content
which simplifies setting them up and maintaining them.

For example, the following file |draft.tex|
with a compilation flag |\version| as described in \secref{sec:flags}
compiles the main document as a draft:
%
\begin{center}
\begin{tabular}{l}
|\def\version{draft}|\\
|\input{childdoc.def}|\\
|\childdocforward{|\textit{main}|}|
\end{tabular}
\end{center}
%
Likewise, the following files |final|\textit{nn}|.tex|
compile the final version of the child document
|child|\textit{nn}|.tex|:
%
\begin{center}
\begin{tabular}{l}
|\def\version{final}|\\
|\input{childdoc.def}|\\
|\childdocforwardprefix{final}{child}|
\end{tabular}
\end{center}
%

Note that when several versions of a main file and/or of each child file
are to be generated, it may be convenient to set up a |Makefile| or
shell script to automatise the process.

%%%%%%%%%%%%%%%%%%%%%%%%%%%%%%%%%%%%%%%%%%%%%%%%%%%%%%%%%%%%%%%%%%%%%%%%%%%%%%%%
\subsection{Command Line Processing}
\label{sec:commandline}

The effect of redirection files can also be achieved by invoking
the \LaTeX{} compiler with a more elaborate command line.
Most conveniently this should be done as part
of a shell script or a |Makefile|.

When using \textsf{childdoc} in the main file, the following
command lines effectively perform a redirection
(note that depending on the shell being used,
backslashes may have to be doubled: `|\|' $\to$ `|\\|'):
%
\begin{center}
|... -jobname "|\textit{target}|" |\\|"|[\textit{flags}]%
|\input{childdoc.def}\childdocforward[|\textit{main}|]{|\textit{dest}|}"|
\end{center}
%
Here \textit{target} is the name of the output file,
\textit{main} is the name of the main file
and \textit{dest} is the name of the main or child file to be processed
(all filenames without extensions).
The optional argument \textit{main} can be omitted
if \textit{main} matches \textit{dest}.
Optionally, compilation \textit{flags} can be defined via |\def| commands.
This command line makes the \TeX{} engine believe
it is compiling the file \textit{target}
whose content is specified as the latter parameter.
The provided code then forwards the processing to
\textit{main} or \textit{dest} as described in \secref{sec:forward}.

%%%%%%%%%%%%%%%%%%%%%%%%%%%%%%%%%%%%%%%%%%%%%%%%%%%%%%%%%%%%%%%%%%%%%%%%%%%%%%%%
\subsection{Include by Input}
\label{sec:input}

Including child documents by |\include| has some restrictions by design.
Most notably, the content of a child document always occupies
its own set of pages; pages cannot be shared between child documents.
Usually, this behaviour makes perfect sense
because each child document contain an essential part of the document.
However, in some situations it may be desirable to compose
a document from a collection of parts
without having mandatory page breaks between then.
For this case, the package
provides a mechanism to include parts
by |\input| which can also be processed individually.
However, by construction this mechanism
requires manual handling of the content to be output.

%%%%%%%%%%%%%%%%%%%%%%%%%%%%%%%%%%%%%%%%
\DescribeMacro{\ifchilddocmanual}
The main file should be prepared as usual, see \secref{sec:include}.
However, the document body must make a distinction
between processing of an individual part and of the main document, e.g.:
%
\begin{center}
\begin{tabular}{l}
|\ifchilddocmanual|\\
|\input{\childdocname}|\\
|\||else|\\
\textit{document body with }|\input{|\textit{part}|}|\\
|\||fi|
\end{tabular}
\end{center}
%
The conditional |\ifchilddocmanual| is true whenever
a part to be included by |\input| is being compiled,
and the name of the part is stored in |\childdocname|.

%%%%%%%%%%%%%%%%%%%%%%%%%%%%%%%%%%%%%%%%
\DescribeMacro{\childdocby}
Each part to be included by |\input| should start with:
%
\begin{center}
\begin{tabular}{l}
|\input{childdoc.def}|\\
|\childdocby{|\textit{main}|}|\\
\end{tabular}
\end{center}
%
The directive |\childdocby| is similar to |\childdocof|
described in \secref{sec:include},
but the subsequent selection of content must be done manually.
To that end, both |\ifchilddoc| and |\ifchilddocmanual|
will be true upon processing of a part,
and the name of the part is stored in |\childdocname|.
Note that |\jobname| will be set to the filename of the current part
so that each part receives an individual |.aux| file
that does not interfere with the |.aux| file(s) of the main document.
This behaviour can be altered by the alternative form
|\childdocby[*]{|\textit{main}|}| (with a non-empty optional argument)
which uses the |.aux| file of the main document
by setting |\jobname| to \textit{main}.

%%%%%%%%%%%%%%%%%%%%%%%%%%%%%%%%%%%%%%%%%%%%%%%%%%%%%%%%%%%%%%%%%%%%%%%%%%%%%%%%
\subsection{Driver Development}
\label{sec:driver}

The \textsf{childdoc} mechanism can also be use for the development
of definition files such as \LaTeX{} styles or classes.
This case differs from the above setup with multiple parts
included by |\include| in that no |\includeonly| should be invoked.
This can be achieved by starting the include file
(before |\ProvidesPackage|) with:
%
\begin{center}
\begin{tabular}{l}
|\input{childdoc.def}|\\
|\childdocforward{|\textit{main}|}|\\
\end{tabular}
\end{center}
%
or alternatively with:
%
\begin{center}
\begin{tabular}{l}
|\input{childdoc.def}|\\
|\childdocby{|\textit{main}|}|\\
\end{tabular}
\end{center}
%
Both forms have slightly different effects as described above.
The main file is prepared as usual, see \secref{sec:include}.

%%%%%%%%%%%%%%%%%%%%%%%%%%%%%%%%%%%%%%%%%%%%%%%%%%%%%%%%%%%%%%%%%%%%%%%%%%%%%%%%
\subsection{Legacy Detection}
\label{sec:detection}

The directive |\childdocmain| in the main file can detect
whether the complete document or merely a child is to be compiled
even without using the directive |\childdocof|.
This method is deprecated because it is less robust
and there is no compelling reason to use it;
it is merely provided for backward compatibility
and it may be removed in future versions.

If the detection mechanism is to be used,
it is mandatory to correctly specify
the filename of the main file as the argument of |\childdocmain|:
%
\begin{center}
\begin{tabular}{l}
|\input{childdoc.def}|\\
|\childdocmain{|\textit{main}|}|\\
\end{tabular}
\end{center}
%
If |\jobname| does not match the argument \textit{main} of |\childdocmain|,
it is assumed that |\jobname| points to the child file to be compiled.
When using |\childdocmain| with the main file specified as argument,
it suffices to start a child file
with just |\input{|\textit{main}|}|
without loading of the package and using |\childdocof|.
If instead all processing is done
with the appropriate \textsf{childdoc} directives,
the argument of \textit{main} of |\childdocmain| can be empty.

An alternative version of the command line processing described
in \secref{sec:commandline} using the detection mechanism reads:
%
\begin{center}
|... -jobname "|\textit{target}|" "|[\textit{flags}]%
[|\def\jobname{|\textit{dest}|}|]|\input{|\textit{main}|}"|
\end{center}

%%%%%%%%%%%%%%%%%%%%%%%%%%%%%%%%%%%%%%%%%%%%%%%%%%%%%%%%%%%%%%%%%%%%%%%%%%%%%%%%
\subsection{Manual Code}
\label{sec:manual}

In case one cannot be certain whether the definitions file |childdoc.def|
is installed on the target \TeX{} distribution
and one prefers not to ship it,
it is conceivable to paste a few relevant commands into the sources.

To that end, drop all statements |\input{childdoc.def}|
and perform the replacements as outlined below.
Instead of |\childdocmain{|\textit{main}|}| add the following code
to the top of the main file:
%
\begin{center}
\begin{tabular}{l}
|\||ifdefined\childdocname\endinput\||fi\newif\ifchilddoc|\\
|\edef\childdocname{\scantokens\expandafter{\jobname\noexpand}}|\\
|\def\childdocmain{|\textit{main}|}\||ifx\childdocmain\childdocname\||else|\\
|\childdoctrue\includeonly{\childdocname}\let\jobname\childdocmain\||fi|\\
\end{tabular}
\end{center}
%
Instead of |\childdocof{|\textit{main}|}| just include the main file
at the top of each child file:
%
\begin{center}
|\input{|\textit{main}|}|
\end{center}
%
A simple redirection |\childdocforward{|\textit{dest}|}| is achieved by:
%
\begin{center}
|\def\jobname{|\textit{dest}|}\input{\jobname}|
\end{center}
%
The redirection with prefix
|\childdocforwardprefix[|\textit{prefix}|]{|\textit{dest}|}|
is accomplished by:
%
\begin{center}
\begin{tabular}{l}
|{\edef\jobname{\scantokens\expandafter{\jobname\noexpand}}|\\
|\def\redirectjob |\textit{prefix}|#1~~~{\gdef\jobname{|\textit{dest}|#1}}|\\
|\expandafter\redirectjob\jobname~~~}\input{\jobname}|
\end{tabular}
\end{center}

In an alternative approach,
child documents can be compiled by a specific command line
without additional code or specific definitions:
%
\begin{center}
|... -jobname "|\textit{target}|" "|[\textit{flags}]%
|\includeonly{|\textit{dest}|}\input{|\textit{main}|}"|
\end{center}
%

%%%%%%%%%%%%%%%%%%%%%%%%%%%%%%%%%%%%%%%%%%%%%%%%%%%%%%%%%%%%%%%%%%%%%%%%%%%%%%%%
%%%%%%%%%%%%%%%%%%%%%%%%%%%%%%%%%%%%%%%%%%%%%%%%%%%%%%%%%%%%%%%%%%%%%%%%%%%%%%%%
\section{Information}

%%%%%%%%%%%%%%%%%%%%%%%%%%%%%%%%%%%%%%%%%%%%%%%%%%%%%%%%%%%%%%%%%%%%%%%%%%%%%%%%
\subsection{Copyright}

Copyright \copyright{} 2017--2018 Niklas Beisert

This work may be distributed and/or modified under the
conditions of the \LaTeX{} Project Public License, either version 1.3
of this license or (at your option) any later version.
The latest version of this license is in
  \url{http://www.latex-project.org/lppl.txt}
and version 1.3 or later is part of all distributions of \LaTeX{}
version 2005/12/01 or later.

This work has the LPPL maintenance status `maintained'.

The Current Maintainer of this work is Niklas Beisert.

This work consists of the files |README.txt|, |childdoc.ins| and |childdoc.dtx|
as well as the derived files |childdoc.def|, |cdocsamp.tex|
with |cdocsch1.tex|, |cdocsch2.tex|, |cdocspt3.tex|, |cdocspt4.tex|,
|cdocsdrf.tex|, |cdocsfn1.tex|, |cdocsfn2.tex|
as well as |childdoc.pdf|.

%%%%%%%%%%%%%%%%%%%%%%%%%%%%%%%%%%%%%%%%%%%%%%%%%%%%%%%%%%%%%%%%%%%%%%%%%%%%%%%%
\subsection{Files and Installation}

The package consists of the files:
%
\begin{center}
\begin{tabular}{ll}
    |README.txt|   & readme file \\
    |childdoc.ins| & installation file \\
    |childdoc.dtx| & source file \\
    |childdoc.def| & definition file \\
    |cdocsamp.tex| & sample main file \\
    |cdocsch1.tex| & sample include file \\
    |cdocsch2.tex| & sample include file \\
    |cdocspt3.tex| & sample part file \\
    |cdocspt4.tex| & sample part file \\
    |cdocsdrf.tex| & sample redirection file \\
    |cdocsfn1.tex| & sample redirection file \\
    |cdocsfn2.tex| & sample redirection file \\
    |childdoc.pdf| & manual
\end{tabular}
\end{center}
%
The distribution consists of the files
|README.txt|, |childdoc.ins| and |childdoc.dtx|.
%
\begin{itemize}
\item
Run (pdf)\LaTeX{} on |childdoc.dtx|
to compile the manual |childdoc.pdf| (this file).
\item
Run \LaTeX{} on |childdoc.ins| to create the definitions file |childdoc.def|
and the sample |cdocsamp.tex| with include files
|cdocsch1.tex|, |cdocsch2.tex|, |cdocspt3.tex|, |cdocspt4.tex|,
|cdocsdrf.tex|, |cdocsfn1.tex|, |cdocsfn2.tex|.
Then copy the file |childdoc.def| to an appropriate directory of your \LaTeX{}
distribution, e.g.\ \textit{texmf-root}|/tex/latex/childdoc|.
\end{itemize}

%%%%%%%%%%%%%%%%%%%%%%%%%%%%%%%%%%%%%%%%%%%%%%%%%%%%%%%%%%%%%%%%%%%%%%%%%%%%%%%%
\subsection{Related CTAN Packages}

There are several other packages which offer a similar functionality:
%
\begin{itemize}
\item
The packages
\href{http://ctan.org/pkg/docmute}{\textsf{docmute}},
\href{http://ctan.org/pkg/includex}{\textsf{includex}} and
\href{http://ctan.org/pkg/standalone}{\textsf{standalone}}
provide commands to include only the document body of
a child file thus allowing both files to be compiled individually.
\item
The packages \href{http://ctan.org/pkg/subdocs}{\textsf{subdocs}}
and \href{http://ctan.org/pkg/subfiles}{\textsf{subfiles}}
provide structures in which the main and child documents can be
encapsulated and allowing them to be compiled individually.
The inclusion mechanism is different from the conventional |\include|.
\item
The package \href{http://ctan.org/pkg/combine}{\textsf{combine}}
is an elaborate solution to combine several documents into one.
\end{itemize}
%
See also the CTAN topic \href{http://ctan.org/topic/subdocs}{\textsf{subdocs}}
for further related packages.
The present package differs from the above solutions in that
a document structure constructed with the conventional |\include| mechanism
just needs two extra commands at the top of every file
such that all constituent files can be compiled individually.

%%%%%%%%%%%%%%%%%%%%%%%%%%%%%%%%%%%%%%%%%%%%%%%%%%%%%%%%%%%%%%%%%%%%%%%%%%%%%%%%
%\subsection{Feature Suggestions}
%
%The following is a list of features which may be useful for future
%versions of this package:
%%
%\begin{itemize}
%\item
%\ldots
%\end{itemize}

%%%%%%%%%%%%%%%%%%%%%%%%%%%%%%%%%%%%%%%%%%%%%%%%%%%%%%%%%%%%%%%%%%%%%%%%%%%%%%%%
\subsection{Revision History}

%%%%%%%%%%%%%%%%%%%%%%%%%%%%%%%%%%%%%%%%
\paragraph{v2.0:} 2018/12/30

\begin{itemize}
\item
immediate forward processing
\item
added |\childdocby| mechanism
\item
manual restructured
\end{itemize}

%%%%%%%%%%%%%%%%%%%%%%%%%%%%%%%%%%%%%%%%
\paragraph{v1.6:} 2018/01/17

\begin{itemize}
\item
application for development of include files
\item
corrections to manual
\end{itemize}

%%%%%%%%%%%%%%%%%%%%%%%%%%%%%%%%%%%%%%%%
\paragraph{v1.5:} 2017/05/21

\begin{itemize}
\item
more complete structuring introduced
\item
|\childdocof| introduced
\item
|\childdoc| renamed to |\childdocmain|
\item
|\childredirect| renamed to |\childdocforward| and |\childdocforwardprefix|
and functionality expanded
\end{itemize}

%%%%%%%%%%%%%%%%%%%%%%%%%%%%%%%%%%%%%%%%
\paragraph{v1.0:} 2017/04/27

\begin{itemize}
\item
manual and install package
\item
first version published on CTAN
\end{itemize}

%%%%%%%%%%%%%%%%%%%%%%%%%%%%%%%%%%%%%%%%
\paragraph{v0.6:} 2017/04/26

\begin{itemize}
\item
redirection mechanism added
\end{itemize}

%%%%%%%%%%%%%%%%%%%%%%%%%%%%%%%%%%%%%%%%
\paragraph{v0.5:} 2017/04/26

\begin{itemize}
\item
functionality in definition file
\end{itemize}


%%%%%%%%%%%%%%%%%%%%%%%%%%%%%%%%%%%%%%%%%%%%%%%%%%%%%%%%%%%%%%%%%%%%%%%%%%%%%%%%
%%%%%%%%%%%%%%%%%%%%%%%%%%%%%%%%%%%%%%%%%%%%%%%%%%%%%%%%%%%%%%%%%%%%%%%%%%%%%%%%
%%%%%%%%%%%%%%%%%%%%%%%%%%%%%%%%%%%%%%%%%%%%%%%%%%%%%%%%%%%%%%%%%%%%%%%%%%%%%%%%
\appendix

\settowidth\MacroIndent{\rmfamily\scriptsize 000\ }

 \DocInput{childdoc.dtx}

\end{document}
%</driver>
% \fi
%
% %%%%%%%%%%%%%%%%%%%%%%%%%%%%%%%%%%%%%%%%%%%%%%%%%%%%%%%%%%%%%%%%%%%%%%%%%%%%%%
% %%%%%%%%%%%%%%%%%%%%%%%%%%%%%%%%%%%%%%%%%%%%%%%%%%%%%%%%%%%%%%%%%%%%%%%%%%%%%%
% \section{Sample}
%\iffalse
%<*samplemain>
%\fi
%
% The following presents a sample document
% with two chapters, two parts, a title page,
% a compile flag as well as three forwarding files to set the flag.
% It consists of eight |.tex| files:
% \begin{center}
% \begin{tabular}{ll}
% |cdocsamp.tex|&main file\\
% |cdocsch1.tex|&include file for chapter 1\\
% |cdocsch2.tex|&include file for chapter 2\\
% |cdocspt3.tex|&include file for part 3\\
% |cdocspt4.tex|&include file for part 4\\
% |cdocsdrf.tex|&forwarding file for main file in draft mode\\
% |cdocsfi1.tex|&forwarding file for final version of chapter 1\\
% |cdocsfi2.tex|&forwarding file for final version of chapter 2\\
% \end{tabular}
% \end{center}
% Each of the eight files can be compiled directly by the \LaTeX{} compiler.
%
% %%%%%%%%%%%%%%%%%%%%%%%%%%%%%%%%%%%%%%
% \paragraph{Main File.}
%
% The main file is called |cdocsamp.tex|.
%
% Load the \textsf{childdoc} definitions and
% declare the filename for the main document:
%    \begin{macrocode}
\input{childdoc.def}
\childdocmain{}
%    \end{macrocode}

% Optional override for |\version| flag:
%    \begin{macrocode}
%%\ifchilddoc\else\providecommand{\version}{draft}\fi
%    \end{macrocode}

% Define the default values for the |\version| flag
% (|final| for the main file and |draft| for childs):
%    \begin{macrocode}
\ifchilddoc
\providecommand{\version}{draft}
\else
\providecommand{\version}{final}
\fi
%    \end{macrocode}

% Load the standard document class:
%    \begin{macrocode}
\documentclass[12pt]{article}
%    \end{macrocode}

% Start the document body:
%    \begin{macrocode}
\begin{document}
%    \end{macrocode}

% Declare a title page.
% Print title, part of document being processed and version flag:
%    \begin{macrocode}
\addtocounter{page}{-1}
\begin{center}
{\LARGE\bfseries{}childdoc example\par}
\vspace{1cm}
\ifchilddoc
\ifchilddocmanual part\else chapter\fi:
`\childdocname' of `\childdocjob'\par
\else
main document: `\childdocjob'\par
\fi
version: \version\par
\end{center}
\newpage
%    \end{macrocode}

% Manually include selected file,
% otherwise process as usual:
%    \begin{macrocode}
\ifchilddocmanual
\section*{part `\childdocname'}
\input{\childdocname}
\else
%    \end{macrocode}

% Include the two chapters:
%    \begin{macrocode}
\include{cdocsch1}
\include{cdocsch2}
%    \end{macrocode}

% Include the two parts unless only chapters should be displayed:
%    \begin{macrocode}
\ifchilddoc\else
\section{part three}
\input{cdocspt3}
\section{part four}
\input{cdocspt4}
\fi
%    \end{macrocode}

% Process as usual until here:
%    \begin{macrocode}
\fi
%    \end{macrocode}

% End of document body:
%    \begin{macrocode}
\end{document}
%    \end{macrocode}
%\iffalse
%</samplemain>
%\fi
%
% %%%%%%%%%%%%%%%%%%%%%%%%%%%%%%%%%%%%%%
% \paragraph{Chapter Include Files.}
%
% The include files are called |cdocsch1.tex| and |cdocsch2.tex|.
%
%\iffalse
%<*samplechap1|samplechap2>
%\fi

% Optional override for |\version| flag:
%    \begin{macrocode}
%%\providecommand{\version}{final}
%    \end{macrocode}

% Include the main document:
%    \begin{macrocode}
\input{childdoc.def}
\childdocof{cdocsamp}
%    \end{macrocode}

%\iffalse
%</samplechap1|samplechap2>
%\fi
%
%\iffalse
%<*samplechap1>
%\fi
% Some text for chapter 1:
%    \begin{macrocode}
\section{one}
some text in chapter one
%    \end{macrocode}

%\iffalse
%</samplechap1>
%\fi
% Some text for chapter 2:
%\iffalse
%<*samplechap2>
%\fi
%    \begin{macrocode}
\section{two}
more text in chapter two
%    \end{macrocode}

%\iffalse
%</samplechap2>
%\fi
%
% %%%%%%%%%%%%%%%%%%%%%%%%%%%%%%%%%%%%%%
% \paragraph{Part Include Files.}
%
% The include files are called |cdocspt3.tex| and |cdocspt4.tex|.
%
%\iffalse
%<*samplepart3|samplepart4>
%\fi

% Optional override for |\version| flag:
%    \begin{macrocode}
%%\providecommand{\version}{final}
%    \end{macrocode}

% Include the main document:
%    \begin{macrocode}
\input{childdoc.def}
\childdocby{cdocsamp}
%    \end{macrocode}

%\iffalse
%</samplepart3|samplepart4>
%\fi
%
%\iffalse
%<*samplepart3>
%\fi
% Some text for part 3:
%    \begin{macrocode}
some text in part three
%    \end{macrocode}

%\iffalse
%</samplepart3>
%\fi
% Some text for part 4:
%\iffalse
%<*samplepart4>
%\fi
%    \begin{macrocode}
more text in part four
%    \end{macrocode}

%\iffalse
%</samplepart4>
%\fi
%
% %%%%%%%%%%%%%%%%%%%%%%%%%%%%%%%%%%%%%%
% \paragraph{Forwarding for a Complete Draft.}
%
% The following forwarding file |cdocsdrf.tex|
% compiles the main document in draft mode:
%\iffalse
%<*sampledraft>
%\fi
%    \begin{macrocode}
\def\version{draft}
\input{childdoc.def}
\childdocforward{cdocsamp}
%    \end{macrocode}

%\iffalse
%</sampledraft>
%\fi
%
% %%%%%%%%%%%%%%%%%%%%%%%%%%%%%%%%%%%%%%
% \paragraph{Forwarding for Final Version of the Chapters.}
%
% The following forwarding files |cdocsfn1.tex| and |cdocsfn2.tex|
% (with identical content)
% compile the final versions of the child documents
% |cdocsch1.tex| and |cdocsch2.tex|, respectively:
%\iffalse
%<*samplefinal>
%\fi
%    \begin{macrocode}
\def\version{final}
\input{childdoc.def}
\childdocforwardprefix[cdocsamp]{cdocsfn}{cdocsch}
%    \end{macrocode}

%\iffalse
%</samplefinal>
%\fi
%
% %%%%%%%%%%%%%%%%%%%%%%%%%%%%%%%%%%%%%%
% \paragraph{Command Line Processing.}
%
% The following three command lines generate the output files
% |cdocscld|, |cdocscl1| and |cdocscl2|
% which should be identical to
% |cdocsdrf|, |cdocsch1| and |cdocsfn2|, respectively:
% \begin{center}
% \begin{tabular}{l}
% |latex -jobname cdocscld \|\\
% |  "\def\version{draft}\input{childdoc.def}\childdocforward{cdocsamp}"|\\
% |latex -jobname cdocscl1 \|\\
% |  "\input{childdoc.def}\childdocforward[cdocsamp]{cdocsch1}"|\\
% |latex -jobname cdocscl2 \|\\
% |  "\def\version{final}\input{childdoc.def}\childdocforward{cdocsch2}"|
% \end{tabular}
% \end{center}
% Note that the trailing backslash on each first line
% merely continues the input to the second line
% (for convenient cut ant paste).
% Furthermore, the command |latex| can be replaced by any
% of its alternative versions such as |pdflatex|.
%
% %%%%%%%%%%%%%%%%%%%%%%%%%%%%%%%%%%%%%%%%%%%%%%%%%%%%%%%%%%%%%%%%%%%%%%%%%%%%%%
% %%%%%%%%%%%%%%%%%%%%%%%%%%%%%%%%%%%%%%%%%%%%%%%%%%%%%%%%%%%%%%%%%%%%%%%%%%%%%%
% \section{Implementation}
%\iffalse
%<*package>
%\fi
%
% This section describes the definitions file |childdoc.def|.

% The definitions cannot be loaded using |\usepackage| or |\RequirePackage|
% which has a mechanism to prevent loading a style file more than once.
% When loading the definitions by means of |\input|
% multiple instances have to be prevented manually:
%\iffalse
%This code needs to be before the `\ProvidesFile' directive
%which is defined at the beginning of this file.
%Therefore it is also placed there and commented out here.
%</package>
%<*discard>
%\fi
%    \begin{macrocode}
\ifdefined\childdocmain\endinput\fi
%    \end{macrocode}
%\iffalse
%</discard>
%<*package>
%\fi
%
% \macro{\ifchilddoc}
% \macro{\ifchilddocmanual}
% The conditional |\ifchilddoc| tells whether a
% child (true) or main (false) document is being compiled.
% The conditional |\ifchilddocmanual| tells whether
% the |\includeonly| mechanism is used (false) or
% the selection of child files must be performed manually (true).
% The definitions initialise to false:
%    \begin{macrocode}
\newif\ifchilddoc
\newif\ifchilddocmanual
%    \end{macrocode}

% \macro{\childdocname}
% \macro{\childdocjob}
% The macro |\childdocname| stores the name of the main document
% to be compiled. The macro |\childdocjob| stores the name of
% the document on which the \LaTeX{} compiler was originally invoked.
% The content of |\jobname| cannot be compared
% to filenames specified in the source due to different catcodes.
% The following code rescans |\jobname|, stores the result
% in |\childdocname| and saves a copy in |\childdocjob|:
%    \begin{macrocode}
\edef\childdocname{\scantokens\expandafter{\jobname\noexpand}}
\let\childdocjob\childdocname
%    \end{macrocode}

% \macro{\childdocdisable}
% The macro |\childdocdisable| prevents the main file
% from being processed more than once.
% At this stage, the main document command |\childdocmain|
% is assumed to be called once again where it should do nothing.
% Any subsequent call to it should prevent
% a secondary processing of the main document
% It overwrites the forwarding commands
% |\childdocof| and |\childdocforward|
% with empty macros to prevent further inclusions of the main document:
%    \begin{macrocode}
\newcommand{\childdocdisable}
{
  \renewcommand{\childdocmain}[1]{\renewcommand{\childdocmain}[1]{\endinput}}
  \renewcommand{\childdocof}[1]{}
  \renewcommand{\childdocby}[2][]{}
  \renewcommand{\childdocforward}[2][]{}
  \renewcommand{\childdocdisable}{}
}
%    \end{macrocode}

% \macro{\childdocmain}
% The macro |\childdocmain| is to be called at the top of the main file
% with nothing or the main filename (without extension) as argument.
% First, it breaks loops.
% If the argument is not empty and does not match |\childdocname|
% (which is set by the first inclusion of |childdoc.def|),
% |\ifchilddoc| is set to true, |\includeonly| is applied to the child file
% and |\jobname| is set to the main file
% (for proper handling of |.aux| files):
%    \begin{macrocode}
\newcommand{\childdocmain}[1]
{
  \childdocdisable\childdocmain{}
  \if?#1?\else
    \begingroup
      \def\childdoctmp{#1}
      \ifx\childdoctmp\childdocname
        \def\childdoctmp{}
      \else
        \def\childdoctmp
        {
          \childdoctrue
          \includeonly{\childdocname}
          \def\childdocjob{#1}
          \def\jobname{#1}
        }
      \fi
      \expandafter
    \endgroup
    \childdoctmp
  \fi
}
%    \end{macrocode}

% \macro{\childdocof}
% The command |\childdocof| redirects
% compilation to the main file |#1|.
%    \begin{macrocode}
\newcommand{\childdocof}[1]
{
  \childdocdisable
  \childdoctrue
  \includeonly{\childdocname}
  \def\jobname{#1}
  \def\childdocjob{#1}
  \input{#1}
}
%    \end{macrocode}

% \macro{\childdocby}
% The command |\childdocby| ....
%    \begin{macrocode}
\newcommand{\childdocby}[2][]
{
  \childdocdisable
  \childdoctrue
  \childdocmanualtrue
  \if?#1?\else
    \def\jobname{#2}
  \fi
  \def\childdocjob{#2}
  \input{#2}
  \endinput
}
%    \end{macrocode}

% \macro{\childdocforward}
% The command |\childdocforward| redirects
% compilation to the main file or
% (if the optional argument is given) a child file.
% Parameters are set as if the main file
% or a child file starting with |\childdocof| was compiled.
% Then compilation is handed over to the main file:
%    \begin{macrocode}
\newcommand{\childdocforward}[2][]
{
  \begingroup
    \if?#1?
      \def\childdoctmp
      {
        \def\childdocname{#2}
        \def\childdocjob{#2}
        \def\jobname{#2}
        \input{#2}
        \endinput
      }
    \else
      \def\childdoctmp
      {
        \childdocdisable
        \def\childdocname{#2}
        \childdoctrue
        \includeonly{#2}
        \def\childdocjob{#1}
        \def\jobname{#1}
        \input{#1}
        \endinput
      }
    \fi
    \expandafter
  \endgroup
  \childdoctmp
}
%    \end{macrocode}

% \macro{\childdocforwardprefix}
% The command |\childdocforwardprefix| redirects
% compilation to the main or a child file by means of a pattern.
% The prefix |#1| in the current filename is replaced by |#2|
% and the suffix of the current filename is kept
% (it is assumed that the filename does not contain the substring `|~~~|'
% which is used as a delimiter).
% Compilation is handed over to the new file by |\childdocforward|:
%    \begin{macrocode}
\newcommand{\childdocforwardprefix}[3][]
{
  \begingroup
    \def\childdocextract #2##1~~~{\def\childdoctmp{\childdocforward[#1]{#3##1}}}
    \expandafter\childdocextract\childdocname~~~
    \expandafter
  \endgroup
  \childdoctmp
}
%    \end{macrocode}

% \macro{\childdoc}
% The deprecated macro |\childdoc| is a legacy version of |\childdocmain|:
%    \begin{macrocode}
\newcommand{\childdoc}{\childdocmain}
%    \end{macrocode}

% \macro{\childdocredirect}
% The deprecated macro |\childdocredirect| is a legacy version
% of |\childdocforward| and |\childdocforwardprefix|:
%    \begin{macrocode}
\newcommand{\childdocredirect}[2][]
{
  \begingroup
    \if?#1?
      \def\childdoctmp{\childdocforward{#2}}
    \else
      \def\childdoctmp{\childdocforwardprefix{#1}{#2}}
    \fi
    \expandafter
  \endgroup
  \childdoctmp
}
%    \end{macrocode}

%\iffalse
%</package>
%\fi
%
\endinput
|\\
|\childdocmain{|\textit{main}|}|\\
\end{tabular}
\end{center}
%
If |\jobname| does not match the argument \textit{main} of |\childdocmain|,
it is assumed that |\jobname| points to the child file to be compiled.
When using |\childdocmain| with the main file specified as argument,
it suffices to start a child file
with just |\input{|\textit{main}|}|
without loading of the package and using |\childdocof|.
If instead all processing is done
with the appropriate \textsf{childdoc} directives,
the argument of \textit{main} of |\childdocmain| can be empty.

An alternative version of the command line processing described
in \secref{sec:commandline} using the detection mechanism reads:
%
\begin{center}
|... -jobname "|\textit{target}|" "|[\textit{flags}]%
[|\def\jobname{|\textit{dest}|}|]|\input{|\textit{main}|}"|
\end{center}

%%%%%%%%%%%%%%%%%%%%%%%%%%%%%%%%%%%%%%%%%%%%%%%%%%%%%%%%%%%%%%%%%%%%%%%%%%%%%%%%
\subsection{Manual Code}
\label{sec:manual}

In case one cannot be certain whether the definitions file |childdoc.def|
is installed on the target \TeX{} distribution
and one prefers not to ship it,
it is conceivable to paste a few relevant commands into the sources.

To that end, drop all statements |% \iffalse
%
% childdoc.dtx Copyright (C) 2017-2018 Niklas Beisert
%
% This work may be distributed and/or modified under the
% conditions of the LaTeX Project Public License, either version 1.3
% of this license or (at your option) any later version.
% The latest version of this license is in
%   http://www.latex-project.org/lppl.txt
% and version 1.3 or later is part of all distributions of LaTeX
% version 2005/12/01 or later.
%
% This work has the LPPL maintenance status `maintained'.
%
% The Current Maintainer of this work is Niklas Beisert.
%
% This work consists of the files childdoc.dtx and childdoc.ins
% and the derived files childdoc.def and cdocsamp.tex with
% cdocsch1.tex, cdocsch2.tex, cdocsdrf.tex, cdocsfn1.tex, cdocsfn2.tex.
%
%<package>\ifdefined\childdocmain\endinput\fi
%<package>\ProvidesFile{childdoc.def}[2018/12/30 v2.0 child document driver]
%<samplemain>\ProvidesFile{cdocsamp.tex}[2018/12/30 v2.0 sample for childdoc]
%<*driver>
%\ProvidesFile{childdoc.drv}[2018/12/30 v2.0 childdoc reference manual file]
\PassOptionsToClass{10pt,a4paper}{article}
\documentclass{ltxdoc}

\usepackage[margin=35mm]{geometry}
\usepackage{hyperref}
\usepackage{hyperxmp}
\usepackage[usenames]{color}

\hypersetup{colorlinks=true}
\hypersetup{pdfstartview=FitH}
\hypersetup{pdfpagemode=UseNone}
\hypersetup{pdfsource={}}
\hypersetup{pdflang={en-UK}}
\hypersetup{pdfcopyright={Copyright 2017-2018 Niklas Beisert.
  This work may be distributed and/or modified under the
  conditions of the LaTeX Project Public License, either version 1.3
  of this license or (at your option) any later version.}}
\hypersetup{pdflicenseurl={http://www.latex-project.org/lppl.txt}}
\hypersetup{pdfcontactaddress={ETH Zurich, ITP, HIT K,
  Wolfgang-Pauli-Strasse 27}}
\hypersetup{pdfcontactpostcode={8093}}
\hypersetup{pdfcontactcity={Zurich}}
\hypersetup{pdfcontactcountry={Switzerland}}
\hypersetup{pdfcontactemail={nbeisert@itp.phys.ethz.ch}}
\hypersetup{pdfcontacturl={http://people.phys.ethz.ch/\xmptilde nbeisert/}}

\newcommand{\secref}[1]{\hyperref[#1]{section \ref*{#1}}}

\parskip1ex
\parindent0pt
\let\olditemize\itemize
\def\itemize{\olditemize\parskip0pt}

\begin{document}

\title{The \textsf{childdoc} Package}
\hypersetup{pdftitle={The childdoc Package}}
\author{Niklas Beisert\\[2ex]
  Institut f\"ur Theoretische Physik\\
  Eidgen\"ossische Technische Hochschule Z\"urich\\
  Wolfgang-Pauli-Strasse 27, 8093 Z\"urich, Switzerland\\[1ex]
  \href{mailto:nbeisert@itp.phys.ethz.ch}
  {\texttt{nbeisert@itp.phys.ethz.ch}}}
\hypersetup{pdfauthor={Niklas Beisert}}
\hypersetup{pdfsubject={Manual for the LaTeX2e Package childdoc}}
\date{30 December 2018, \textsf{v2.0}}
\maketitle

\begin{abstract}\noindent
\textsf{childdoc} is a \LaTeXe{} package
that enables the direct compilation
of document sections included by |\include|
to individual files.
\end{abstract}

\begingroup
\parskip0ex
\tableofcontents
\endgroup

%%%%%%%%%%%%%%%%%%%%%%%%%%%%%%%%%%%%%%%%%%%%%%%%%%%%%%%%%%%%%%%%%%%%%%%%%%%%%%%%
%%%%%%%%%%%%%%%%%%%%%%%%%%%%%%%%%%%%%%%%%%%%%%%%%%%%%%%%%%%%%%%%%%%%%%%%%%%%%%%%
\section{Introduction}

\LaTeX{} provides a mechanism to structure a large document (such as a book)
into a main file and several child files (containing the chapters)
using the |\include| command.
This mechanism is beneficial for documents
which span hundreds of pages in order to
make the source file(s) more manageable.
Moreover, compilation can be restricted to
selected child files by means of the |\includeonly| command.
The latter feature can be used to reduce the compilation time while editing
(this was significantly more useful in the earlier days of \LaTeX{})
or to generate a smaller document which is easier to navigate.
Another application of |\includeonly| is to generate
documents consisting of selected parts of the complete document.

However, there are a few drawbacks of the plain |\include| mechanism:
\begin{itemize}
\item
The child files cannot be compiled on their own,
they can only be compiled via the main file.
A naive editing environment
(such as a text editor with an option
to have the current file processed by \LaTeX)
may require one to switch to the main file before compiling;
attempting to compile the child file produces errors.
\item
The main file must be modified (each time)
to adjust the |\includeonly| command
to the present needs. This easily leaves the main file in a messy state.
\item
The generated document will always carry the filename
of the main document. This is inconvenient if
several child files are to be compiled and
to be kept for distribution.
\end{itemize}

The present package provides a simple interface
to make child files individually compilable by \LaTeX{}.
Compiling a child file then has the same effect as compiling
the main file with an |\includeonly| command
to select the appropriate child.
Moreover the generated document will carry the name of the child
rather than the main file.
This resolves all three above issues.

This feature is meant to make the editing of books,
thesis documents and lecture notes somewhat more convenient.
However, the package can also be used efficiently for
composing a series of documents (such as exercise sheets)
which are typically distributed individually.
It then assists the author in generating the individual documents
(potentially in different versions)
as well as a document containing the collected series.
Another application is in developing style files
or other kinds of included material
where compilation of the style file could redirect
to a sample or test file.

%%%%%%%%%%%%%%%%%%%%%%%%%%%%%%%%%%%%%%%%%%%%%%%%%%%%%%%%%%%%%%%%%%%%%%%%%%%%%%%%
%%%%%%%%%%%%%%%%%%%%%%%%%%%%%%%%%%%%%%%%%%%%%%%%%%%%%%%%%%%%%%%%%%%%%%%%%%%%%%%%
\section{Usage}

First of all, the package \textsf{childdoc} is \emph{not} a standard
\LaTeXe{} |.sty| style file! Therefore it needs to be invoked in
a non-standard way.

%%%%%%%%%%%%%%%%%%%%%%%%%%%%%%%%%%%%%%%%%%%%%%%%%%%%%%%%%%%%%%%%%%%%%%%%%%%%%%%%
\subsection{Included Files}
\label{sec:include}

%%%%%%%%%%%%%%%%%%%%%%%%%%%%%%%%%%%%%%%%
\DescribeMacro{\childdocmain}
To use the package, add the commands
\begin{center}
\begin{tabular}{l}
|\input{childdoc.def}|\\
|\childdocmain{}|\\
\end{tabular}
\end{center}
at the very top of the main \LaTeX{} file,
in particular \emph{before} the |\documentclass| statement!
The argument of |\childdocmain| should be left empty
(but it must be present).

%%%%%%%%%%%%%%%%%%%%%%%%%%%%%%%%%%%%%%%%
\DescribeMacro{\childdocof}
Furthermore, add the commands
\begin{center}
\begin{tabular}{l}
|\input{childdoc.def}|\\
|\childdocof{|\textit{main}|}|\\
\end{tabular}
\end{center}
at the top of every child file \textit{child}
which is included by |\include{|\textit{child}|}|
from within the main file
(or at least for those files to be compiled individually).
The argument \textit{main} must be the filename of the main file.

There are a couple of
considerations in setting up the main and child documents:

%%%%%%%%%%%%%%%%%%%%%%%%%%%%%%%%%%%%%%%%
\paragraph{Restrictions.}

Please note the following restrictions:
\begin{itemize}
\item
|\childdocmain| must be called with one argument \textit{main}
to ensure compatibility with earlier version of the package.
It must either be empty (|\childdocmain{}|)
or precisely match the filename of the main file in which it is specified.
See \secref{sec:detection} for further information.
\item
The filename \textit{main} must be specified without the |.tex| extension.
\item
The filename \textit{main} is case sensitive
(even in case-insensitive file systems)
due to internal string comparison.
\item
The argument \textit{main} should be fully expanded, it cannot be a macro.
\item
Subdirectories and special characters should be avoided in filenames.
\item
The command |\childdocmain{|\textit{main}|}| must be followed by a whitespace.
It should not be followed immediately by another command
or by a comment mark `|%|'.
This is because the \TeX{} parser reads the token immediately following
the argument of |\childdocmain| and puts it
at the beginning of every child section;
however, a white\-space is ignored.
\end{itemize}

%%%%%%%%%%%%%%%%%%%%%%%%%%%%%%%%%%%%%%%%
\paragraph{Content of Main File.}

It is advisable to place all content in the child files included by |\include|.
Any output contained in the main file will appear in all child documents
unless suppressed manually;
it cannot be suppressed automatically by the |\includeonly| directive
and thus should normally be avoided.
A method to include some content in the main file
by means of conditional processing is described in \secref{sec:conditional}.

%%%%%%%%%%%%%%%%%%%%%%%%%%%%%%%%%%%%%%%%
\paragraph{Page Numbering.}

When only a part of the document is compiled,
the appropriate numbering of pages
(as well as other status parameters)
is determined from the |.aux| files.
The latter contain information from previous passes.
However this information needs to propagate through
all intermediate child documents.
Therefore the page numbering in child documents may well
be inconsistent until the complete document is compiled at least once.

A useful (if unconventional) way to always ensure a consistent
page numbering is to restart the numbering in each child document
and denote the pages by `\textit{child}|.|\textit{page}'
where \textit{child} represents the chapter/section number of the child file.
This can be achieved by the command
|\numberwithin{page}{|\textit{child}|}|
of the \textsf{amsmath} package
where \textit{child} can be |chapter| or |section|
depending on the chosen structuring.
Alternatively, one can modify the macro |\thepage| appropriately
and reset the counter |page| at the start of each child file.

%%%%%%%%%%%%%%%%%%%%%%%%%%%%%%%%%%%%%%%%%%%%%%%%%%%%%%%%%%%%%%%%%%%%%%%%%%%%%%%%
\subsection{Conditional Processing}
\label{sec:conditional}

The package provides a mechanism to compile different versions
of a document. To customise the versions further some conditional processing
can come in handy to distinguish which version is being compiled.
The package provides two macros to describe the compilation context:

%%%%%%%%%%%%%%%%%%%%%%%%%%%%%%%%%%%%%%%%
\DescribeMacro{\ifchilddoc}
The conditional |\ifchilddoc| distinguishes between the compilation of
child documents and the main document:
%
\begin{center}
|\ifchilddoc |\textit{child-code}| |[|\||else |\textit{main-code}]| \||fi|
\end{center}

%%%%%%%%%%%%%%%%%%%%%%%%%%%%%%%%%%%%%%%%
\DescribeMacro{\childdocname}
\DescribeMacro{\childdocjob}
The macro |\childdocname| contains the filename (without extension)
of the main or child file being processed.
Note that |\childdocjob| will always contain the name of the main file.

%%%%%%%%%%%%%%%%%%%%%%%%%%%%%%%%%%%%%%%%
\paragraph{Title Page.}

Conditional processing can be used to include a title or banner page
in the main document when proper precautions are taken.
Importantly, the code in the main file should ensure that the page counter
(as well as other status parameters which are stored in the |.aux| files)
takes the same value after the conditional processing.
Otherwise the page numbers may take divergent values
depending on which part is compiled.

For example, a title page could be declared by:
%
\begin{center}
\begin{tabular}{l}
|\ifchilddoc\||else|\\
|\addtocounter{page}{-1}|\\
\textit{code for title page}\\
|\newpage|\\
|\||fi|
\end{tabular}
\end{center}
%
A banner page for the child documents can be generated by:
%
\begin{center}
\begin{tabular}{l}
|\ifchilddoc|\\
|\addtocounter{page}{-1}|\\
\textit{code for banner page}\\
|\newpage|\\
|\||fi|
\end{tabular}
\end{center}
%
Here one could write a message such as:
\begin{center}
|This is the part \childdocname{} of \childdocjob{}.|
\end{center}

%%%%%%%%%%%%%%%%%%%%%%%%%%%%%%%%%%%%%%%%%%%%%%%%%%%%%%%%%%%%%%%%%%%%%%%%%%%%%%%%
\subsection{Flags}
\label{sec:flags}

The package makes it easy to generate different versions
of the main or child documents.
To this end compilation flags can be defined
and assigned different default values.
They will be particularly useful in conjunction
with the forwarding mechanism described in \secref{sec:forward}.

For example, it may be useful to have a flag |\version|
which can be set to |draft| or |final|.
The document source will contain some conditional code
depending on the value of |\version|.
Suppose further, the flag should default to |final| for the main file
and to |draft| for child files
which is a natural assignment for editing the document.
This is achieved by placing the following code
in the preamble of the main document
(below the |\childdocmain| directive):
%
\begin{center}
\begin{tabular}{l}
|\ifchilddoc|\\
|\providecommand{\version}{draft}|\\
|\||else|\\
|\providecommand{\version}{final}|\\
|\||fi|
\end{tabular}
\end{center}
%
The definition by |\providecommand| makes sure
that previous definitions are not overwritten.
Further statements |\providecommand{\version}{...}|
can thus be added before the above code to override it.

For the main file, one might add a line
(between |\childdocmain| and the above block)
%
\begin{center}
|%\ifchilddoc\||else\providecommand{\version}{draft}\||fi|
\end{center}
%
which can be uncommented to produce a draft version.
Likewise one can add a line to the very top of a child file
(above the |\childdocof{|\textit{main}|}| directive)
%
\begin{center}
|%\providecommand{\version}{final}|
\end{center}
%
which can be uncommented to produce the final version of this child document.

%%%%%%%%%%%%%%%%%%%%%%%%%%%%%%%%%%%%%%%%%%%%%%%%%%%%%%%%%%%%%%%%%%%%%%%%%%%%%%%%
\subsection{Forwarding}
\label{sec:forward}

Different versions of the main or child documents
using compilation flags as described in \secref{sec:flags}
can be (permanently) stored in different files
for convenient compilation, viewing and distribution.
To this end, the package defines a command
to pass on compilation to a different file:

%%%%%%%%%%%%%%%%%%%%%%%%%%%%%%%%%%%%%%%%
\DescribeMacro{\childdocforward}
The command |\childdocforward| redirects processing to
another source file:
%
\begin{center}
\begin{tabular}{l}
|\input{childdoc.def}|\\
|\childdocforward[|\textit{main}|]{|\textit{dest}|}|\\
\end{tabular}
\end{center}
%
The argument \textit{dest} is the destination file
(without extension).
It should be the main file or one of the child files.
Note that further \textsf{childdoc} directives
such as |\childdocof| and |\childdocforward|
in the indicated file will be processed in this form.
The optional argument \textit{main}
passes on directly to the main file \textit{main}
while pretending to compile the child \textit{dest}.
This form behaves as if \textit{dest}
issues |\childdocof{|\textit{main}|}| right away,
and no further \textsf{childdoc} directives will be processed.

%%%%%%%%%%%%%%%%%%%%%%%%%%%%%%%%%%%%%%%%
\DescribeMacro{\...prefix}
In the alternative form |\childdocforwardprefix|,
%
\begin{center}
\begin{tabular}{l}
|\input{childdoc.def}|\\
|\childdocforwardprefix[|\textit{main}|]{|\textit{prefix}|}{|\textit{dest}|}|
\end{tabular}
\end{center}
%
the destination file is determined by a pattern
depending on the current file:
To make this work, the current file must be called
`{\textit{prefix}\hspace{0.2em}\textit{suffix}}'
with \textit{prefix} matching precisely the argument.
Processing is then passed on to the file
`{\textit{dest}\hspace{0.2em}\textit{suffix}}'.
Surely, the same effect is achieved by
directly specifying the
argument `{\textit{dest}\hspace{0.2em}\textit{suffix}}'
in the first form.
However, that requires to set up a different file
for each child. With the alternative form of the command
all these files can have exactly the same content
which simplifies setting them up and maintaining them.

For example, the following file |draft.tex|
with a compilation flag |\version| as described in \secref{sec:flags}
compiles the main document as a draft:
%
\begin{center}
\begin{tabular}{l}
|\def\version{draft}|\\
|\input{childdoc.def}|\\
|\childdocforward{|\textit{main}|}|
\end{tabular}
\end{center}
%
Likewise, the following files |final|\textit{nn}|.tex|
compile the final version of the child document
|child|\textit{nn}|.tex|:
%
\begin{center}
\begin{tabular}{l}
|\def\version{final}|\\
|\input{childdoc.def}|\\
|\childdocforwardprefix{final}{child}|
\end{tabular}
\end{center}
%

Note that when several versions of a main file and/or of each child file
are to be generated, it may be convenient to set up a |Makefile| or
shell script to automatise the process.

%%%%%%%%%%%%%%%%%%%%%%%%%%%%%%%%%%%%%%%%%%%%%%%%%%%%%%%%%%%%%%%%%%%%%%%%%%%%%%%%
\subsection{Command Line Processing}
\label{sec:commandline}

The effect of redirection files can also be achieved by invoking
the \LaTeX{} compiler with a more elaborate command line.
Most conveniently this should be done as part
of a shell script or a |Makefile|.

When using \textsf{childdoc} in the main file, the following
command lines effectively perform a redirection
(note that depending on the shell being used,
backslashes may have to be doubled: `|\|' $\to$ `|\\|'):
%
\begin{center}
|... -jobname "|\textit{target}|" |\\|"|[\textit{flags}]%
|\input{childdoc.def}\childdocforward[|\textit{main}|]{|\textit{dest}|}"|
\end{center}
%
Here \textit{target} is the name of the output file,
\textit{main} is the name of the main file
and \textit{dest} is the name of the main or child file to be processed
(all filenames without extensions).
The optional argument \textit{main} can be omitted
if \textit{main} matches \textit{dest}.
Optionally, compilation \textit{flags} can be defined via |\def| commands.
This command line makes the \TeX{} engine believe
it is compiling the file \textit{target}
whose content is specified as the latter parameter.
The provided code then forwards the processing to
\textit{main} or \textit{dest} as described in \secref{sec:forward}.

%%%%%%%%%%%%%%%%%%%%%%%%%%%%%%%%%%%%%%%%%%%%%%%%%%%%%%%%%%%%%%%%%%%%%%%%%%%%%%%%
\subsection{Include by Input}
\label{sec:input}

Including child documents by |\include| has some restrictions by design.
Most notably, the content of a child document always occupies
its own set of pages; pages cannot be shared between child documents.
Usually, this behaviour makes perfect sense
because each child document contain an essential part of the document.
However, in some situations it may be desirable to compose
a document from a collection of parts
without having mandatory page breaks between then.
For this case, the package
provides a mechanism to include parts
by |\input| which can also be processed individually.
However, by construction this mechanism
requires manual handling of the content to be output.

%%%%%%%%%%%%%%%%%%%%%%%%%%%%%%%%%%%%%%%%
\DescribeMacro{\ifchilddocmanual}
The main file should be prepared as usual, see \secref{sec:include}.
However, the document body must make a distinction
between processing of an individual part and of the main document, e.g.:
%
\begin{center}
\begin{tabular}{l}
|\ifchilddocmanual|\\
|\input{\childdocname}|\\
|\||else|\\
\textit{document body with }|\input{|\textit{part}|}|\\
|\||fi|
\end{tabular}
\end{center}
%
The conditional |\ifchilddocmanual| is true whenever
a part to be included by |\input| is being compiled,
and the name of the part is stored in |\childdocname|.

%%%%%%%%%%%%%%%%%%%%%%%%%%%%%%%%%%%%%%%%
\DescribeMacro{\childdocby}
Each part to be included by |\input| should start with:
%
\begin{center}
\begin{tabular}{l}
|\input{childdoc.def}|\\
|\childdocby{|\textit{main}|}|\\
\end{tabular}
\end{center}
%
The directive |\childdocby| is similar to |\childdocof|
described in \secref{sec:include},
but the subsequent selection of content must be done manually.
To that end, both |\ifchilddoc| and |\ifchilddocmanual|
will be true upon processing of a part,
and the name of the part is stored in |\childdocname|.
Note that |\jobname| will be set to the filename of the current part
so that each part receives an individual |.aux| file
that does not interfere with the |.aux| file(s) of the main document.
This behaviour can be altered by the alternative form
|\childdocby[*]{|\textit{main}|}| (with a non-empty optional argument)
which uses the |.aux| file of the main document
by setting |\jobname| to \textit{main}.

%%%%%%%%%%%%%%%%%%%%%%%%%%%%%%%%%%%%%%%%%%%%%%%%%%%%%%%%%%%%%%%%%%%%%%%%%%%%%%%%
\subsection{Driver Development}
\label{sec:driver}

The \textsf{childdoc} mechanism can also be use for the development
of definition files such as \LaTeX{} styles or classes.
This case differs from the above setup with multiple parts
included by |\include| in that no |\includeonly| should be invoked.
This can be achieved by starting the include file
(before |\ProvidesPackage|) with:
%
\begin{center}
\begin{tabular}{l}
|\input{childdoc.def}|\\
|\childdocforward{|\textit{main}|}|\\
\end{tabular}
\end{center}
%
or alternatively with:
%
\begin{center}
\begin{tabular}{l}
|\input{childdoc.def}|\\
|\childdocby{|\textit{main}|}|\\
\end{tabular}
\end{center}
%
Both forms have slightly different effects as described above.
The main file is prepared as usual, see \secref{sec:include}.

%%%%%%%%%%%%%%%%%%%%%%%%%%%%%%%%%%%%%%%%%%%%%%%%%%%%%%%%%%%%%%%%%%%%%%%%%%%%%%%%
\subsection{Legacy Detection}
\label{sec:detection}

The directive |\childdocmain| in the main file can detect
whether the complete document or merely a child is to be compiled
even without using the directive |\childdocof|.
This method is deprecated because it is less robust
and there is no compelling reason to use it;
it is merely provided for backward compatibility
and it may be removed in future versions.

If the detection mechanism is to be used,
it is mandatory to correctly specify
the filename of the main file as the argument of |\childdocmain|:
%
\begin{center}
\begin{tabular}{l}
|\input{childdoc.def}|\\
|\childdocmain{|\textit{main}|}|\\
\end{tabular}
\end{center}
%
If |\jobname| does not match the argument \textit{main} of |\childdocmain|,
it is assumed that |\jobname| points to the child file to be compiled.
When using |\childdocmain| with the main file specified as argument,
it suffices to start a child file
with just |\input{|\textit{main}|}|
without loading of the package and using |\childdocof|.
If instead all processing is done
with the appropriate \textsf{childdoc} directives,
the argument of \textit{main} of |\childdocmain| can be empty.

An alternative version of the command line processing described
in \secref{sec:commandline} using the detection mechanism reads:
%
\begin{center}
|... -jobname "|\textit{target}|" "|[\textit{flags}]%
[|\def\jobname{|\textit{dest}|}|]|\input{|\textit{main}|}"|
\end{center}

%%%%%%%%%%%%%%%%%%%%%%%%%%%%%%%%%%%%%%%%%%%%%%%%%%%%%%%%%%%%%%%%%%%%%%%%%%%%%%%%
\subsection{Manual Code}
\label{sec:manual}

In case one cannot be certain whether the definitions file |childdoc.def|
is installed on the target \TeX{} distribution
and one prefers not to ship it,
it is conceivable to paste a few relevant commands into the sources.

To that end, drop all statements |\input{childdoc.def}|
and perform the replacements as outlined below.
Instead of |\childdocmain{|\textit{main}|}| add the following code
to the top of the main file:
%
\begin{center}
\begin{tabular}{l}
|\||ifdefined\childdocname\endinput\||fi\newif\ifchilddoc|\\
|\edef\childdocname{\scantokens\expandafter{\jobname\noexpand}}|\\
|\def\childdocmain{|\textit{main}|}\||ifx\childdocmain\childdocname\||else|\\
|\childdoctrue\includeonly{\childdocname}\let\jobname\childdocmain\||fi|\\
\end{tabular}
\end{center}
%
Instead of |\childdocof{|\textit{main}|}| just include the main file
at the top of each child file:
%
\begin{center}
|\input{|\textit{main}|}|
\end{center}
%
A simple redirection |\childdocforward{|\textit{dest}|}| is achieved by:
%
\begin{center}
|\def\jobname{|\textit{dest}|}\input{\jobname}|
\end{center}
%
The redirection with prefix
|\childdocforwardprefix[|\textit{prefix}|]{|\textit{dest}|}|
is accomplished by:
%
\begin{center}
\begin{tabular}{l}
|{\edef\jobname{\scantokens\expandafter{\jobname\noexpand}}|\\
|\def\redirectjob |\textit{prefix}|#1~~~{\gdef\jobname{|\textit{dest}|#1}}|\\
|\expandafter\redirectjob\jobname~~~}\input{\jobname}|
\end{tabular}
\end{center}

In an alternative approach,
child documents can be compiled by a specific command line
without additional code or specific definitions:
%
\begin{center}
|... -jobname "|\textit{target}|" "|[\textit{flags}]%
|\includeonly{|\textit{dest}|}\input{|\textit{main}|}"|
\end{center}
%

%%%%%%%%%%%%%%%%%%%%%%%%%%%%%%%%%%%%%%%%%%%%%%%%%%%%%%%%%%%%%%%%%%%%%%%%%%%%%%%%
%%%%%%%%%%%%%%%%%%%%%%%%%%%%%%%%%%%%%%%%%%%%%%%%%%%%%%%%%%%%%%%%%%%%%%%%%%%%%%%%
\section{Information}

%%%%%%%%%%%%%%%%%%%%%%%%%%%%%%%%%%%%%%%%%%%%%%%%%%%%%%%%%%%%%%%%%%%%%%%%%%%%%%%%
\subsection{Copyright}

Copyright \copyright{} 2017--2018 Niklas Beisert

This work may be distributed and/or modified under the
conditions of the \LaTeX{} Project Public License, either version 1.3
of this license or (at your option) any later version.
The latest version of this license is in
  \url{http://www.latex-project.org/lppl.txt}
and version 1.3 or later is part of all distributions of \LaTeX{}
version 2005/12/01 or later.

This work has the LPPL maintenance status `maintained'.

The Current Maintainer of this work is Niklas Beisert.

This work consists of the files |README.txt|, |childdoc.ins| and |childdoc.dtx|
as well as the derived files |childdoc.def|, |cdocsamp.tex|
with |cdocsch1.tex|, |cdocsch2.tex|, |cdocspt3.tex|, |cdocspt4.tex|,
|cdocsdrf.tex|, |cdocsfn1.tex|, |cdocsfn2.tex|
as well as |childdoc.pdf|.

%%%%%%%%%%%%%%%%%%%%%%%%%%%%%%%%%%%%%%%%%%%%%%%%%%%%%%%%%%%%%%%%%%%%%%%%%%%%%%%%
\subsection{Files and Installation}

The package consists of the files:
%
\begin{center}
\begin{tabular}{ll}
    |README.txt|   & readme file \\
    |childdoc.ins| & installation file \\
    |childdoc.dtx| & source file \\
    |childdoc.def| & definition file \\
    |cdocsamp.tex| & sample main file \\
    |cdocsch1.tex| & sample include file \\
    |cdocsch2.tex| & sample include file \\
    |cdocspt3.tex| & sample part file \\
    |cdocspt4.tex| & sample part file \\
    |cdocsdrf.tex| & sample redirection file \\
    |cdocsfn1.tex| & sample redirection file \\
    |cdocsfn2.tex| & sample redirection file \\
    |childdoc.pdf| & manual
\end{tabular}
\end{center}
%
The distribution consists of the files
|README.txt|, |childdoc.ins| and |childdoc.dtx|.
%
\begin{itemize}
\item
Run (pdf)\LaTeX{} on |childdoc.dtx|
to compile the manual |childdoc.pdf| (this file).
\item
Run \LaTeX{} on |childdoc.ins| to create the definitions file |childdoc.def|
and the sample |cdocsamp.tex| with include files
|cdocsch1.tex|, |cdocsch2.tex|, |cdocspt3.tex|, |cdocspt4.tex|,
|cdocsdrf.tex|, |cdocsfn1.tex|, |cdocsfn2.tex|.
Then copy the file |childdoc.def| to an appropriate directory of your \LaTeX{}
distribution, e.g.\ \textit{texmf-root}|/tex/latex/childdoc|.
\end{itemize}

%%%%%%%%%%%%%%%%%%%%%%%%%%%%%%%%%%%%%%%%%%%%%%%%%%%%%%%%%%%%%%%%%%%%%%%%%%%%%%%%
\subsection{Related CTAN Packages}

There are several other packages which offer a similar functionality:
%
\begin{itemize}
\item
The packages
\href{http://ctan.org/pkg/docmute}{\textsf{docmute}},
\href{http://ctan.org/pkg/includex}{\textsf{includex}} and
\href{http://ctan.org/pkg/standalone}{\textsf{standalone}}
provide commands to include only the document body of
a child file thus allowing both files to be compiled individually.
\item
The packages \href{http://ctan.org/pkg/subdocs}{\textsf{subdocs}}
and \href{http://ctan.org/pkg/subfiles}{\textsf{subfiles}}
provide structures in which the main and child documents can be
encapsulated and allowing them to be compiled individually.
The inclusion mechanism is different from the conventional |\include|.
\item
The package \href{http://ctan.org/pkg/combine}{\textsf{combine}}
is an elaborate solution to combine several documents into one.
\end{itemize}
%
See also the CTAN topic \href{http://ctan.org/topic/subdocs}{\textsf{subdocs}}
for further related packages.
The present package differs from the above solutions in that
a document structure constructed with the conventional |\include| mechanism
just needs two extra commands at the top of every file
such that all constituent files can be compiled individually.

%%%%%%%%%%%%%%%%%%%%%%%%%%%%%%%%%%%%%%%%%%%%%%%%%%%%%%%%%%%%%%%%%%%%%%%%%%%%%%%%
%\subsection{Feature Suggestions}
%
%The following is a list of features which may be useful for future
%versions of this package:
%%
%\begin{itemize}
%\item
%\ldots
%\end{itemize}

%%%%%%%%%%%%%%%%%%%%%%%%%%%%%%%%%%%%%%%%%%%%%%%%%%%%%%%%%%%%%%%%%%%%%%%%%%%%%%%%
\subsection{Revision History}

%%%%%%%%%%%%%%%%%%%%%%%%%%%%%%%%%%%%%%%%
\paragraph{v2.0:} 2018/12/30

\begin{itemize}
\item
immediate forward processing
\item
added |\childdocby| mechanism
\item
manual restructured
\end{itemize}

%%%%%%%%%%%%%%%%%%%%%%%%%%%%%%%%%%%%%%%%
\paragraph{v1.6:} 2018/01/17

\begin{itemize}
\item
application for development of include files
\item
corrections to manual
\end{itemize}

%%%%%%%%%%%%%%%%%%%%%%%%%%%%%%%%%%%%%%%%
\paragraph{v1.5:} 2017/05/21

\begin{itemize}
\item
more complete structuring introduced
\item
|\childdocof| introduced
\item
|\childdoc| renamed to |\childdocmain|
\item
|\childredirect| renamed to |\childdocforward| and |\childdocforwardprefix|
and functionality expanded
\end{itemize}

%%%%%%%%%%%%%%%%%%%%%%%%%%%%%%%%%%%%%%%%
\paragraph{v1.0:} 2017/04/27

\begin{itemize}
\item
manual and install package
\item
first version published on CTAN
\end{itemize}

%%%%%%%%%%%%%%%%%%%%%%%%%%%%%%%%%%%%%%%%
\paragraph{v0.6:} 2017/04/26

\begin{itemize}
\item
redirection mechanism added
\end{itemize}

%%%%%%%%%%%%%%%%%%%%%%%%%%%%%%%%%%%%%%%%
\paragraph{v0.5:} 2017/04/26

\begin{itemize}
\item
functionality in definition file
\end{itemize}


%%%%%%%%%%%%%%%%%%%%%%%%%%%%%%%%%%%%%%%%%%%%%%%%%%%%%%%%%%%%%%%%%%%%%%%%%%%%%%%%
%%%%%%%%%%%%%%%%%%%%%%%%%%%%%%%%%%%%%%%%%%%%%%%%%%%%%%%%%%%%%%%%%%%%%%%%%%%%%%%%
%%%%%%%%%%%%%%%%%%%%%%%%%%%%%%%%%%%%%%%%%%%%%%%%%%%%%%%%%%%%%%%%%%%%%%%%%%%%%%%%
\appendix

\settowidth\MacroIndent{\rmfamily\scriptsize 000\ }

 \DocInput{childdoc.dtx}

\end{document}
%</driver>
% \fi
%
% %%%%%%%%%%%%%%%%%%%%%%%%%%%%%%%%%%%%%%%%%%%%%%%%%%%%%%%%%%%%%%%%%%%%%%%%%%%%%%
% %%%%%%%%%%%%%%%%%%%%%%%%%%%%%%%%%%%%%%%%%%%%%%%%%%%%%%%%%%%%%%%%%%%%%%%%%%%%%%
% \section{Sample}
%\iffalse
%<*samplemain>
%\fi
%
% The following presents a sample document
% with two chapters, two parts, a title page,
% a compile flag as well as three forwarding files to set the flag.
% It consists of eight |.tex| files:
% \begin{center}
% \begin{tabular}{ll}
% |cdocsamp.tex|&main file\\
% |cdocsch1.tex|&include file for chapter 1\\
% |cdocsch2.tex|&include file for chapter 2\\
% |cdocspt3.tex|&include file for part 3\\
% |cdocspt4.tex|&include file for part 4\\
% |cdocsdrf.tex|&forwarding file for main file in draft mode\\
% |cdocsfi1.tex|&forwarding file for final version of chapter 1\\
% |cdocsfi2.tex|&forwarding file for final version of chapter 2\\
% \end{tabular}
% \end{center}
% Each of the eight files can be compiled directly by the \LaTeX{} compiler.
%
% %%%%%%%%%%%%%%%%%%%%%%%%%%%%%%%%%%%%%%
% \paragraph{Main File.}
%
% The main file is called |cdocsamp.tex|.
%
% Load the \textsf{childdoc} definitions and
% declare the filename for the main document:
%    \begin{macrocode}
\input{childdoc.def}
\childdocmain{}
%    \end{macrocode}

% Optional override for |\version| flag:
%    \begin{macrocode}
%%\ifchilddoc\else\providecommand{\version}{draft}\fi
%    \end{macrocode}

% Define the default values for the |\version| flag
% (|final| for the main file and |draft| for childs):
%    \begin{macrocode}
\ifchilddoc
\providecommand{\version}{draft}
\else
\providecommand{\version}{final}
\fi
%    \end{macrocode}

% Load the standard document class:
%    \begin{macrocode}
\documentclass[12pt]{article}
%    \end{macrocode}

% Start the document body:
%    \begin{macrocode}
\begin{document}
%    \end{macrocode}

% Declare a title page.
% Print title, part of document being processed and version flag:
%    \begin{macrocode}
\addtocounter{page}{-1}
\begin{center}
{\LARGE\bfseries{}childdoc example\par}
\vspace{1cm}
\ifchilddoc
\ifchilddocmanual part\else chapter\fi:
`\childdocname' of `\childdocjob'\par
\else
main document: `\childdocjob'\par
\fi
version: \version\par
\end{center}
\newpage
%    \end{macrocode}

% Manually include selected file,
% otherwise process as usual:
%    \begin{macrocode}
\ifchilddocmanual
\section*{part `\childdocname'}
\input{\childdocname}
\else
%    \end{macrocode}

% Include the two chapters:
%    \begin{macrocode}
\include{cdocsch1}
\include{cdocsch2}
%    \end{macrocode}

% Include the two parts unless only chapters should be displayed:
%    \begin{macrocode}
\ifchilddoc\else
\section{part three}
\input{cdocspt3}
\section{part four}
\input{cdocspt4}
\fi
%    \end{macrocode}

% Process as usual until here:
%    \begin{macrocode}
\fi
%    \end{macrocode}

% End of document body:
%    \begin{macrocode}
\end{document}
%    \end{macrocode}
%\iffalse
%</samplemain>
%\fi
%
% %%%%%%%%%%%%%%%%%%%%%%%%%%%%%%%%%%%%%%
% \paragraph{Chapter Include Files.}
%
% The include files are called |cdocsch1.tex| and |cdocsch2.tex|.
%
%\iffalse
%<*samplechap1|samplechap2>
%\fi

% Optional override for |\version| flag:
%    \begin{macrocode}
%%\providecommand{\version}{final}
%    \end{macrocode}

% Include the main document:
%    \begin{macrocode}
\input{childdoc.def}
\childdocof{cdocsamp}
%    \end{macrocode}

%\iffalse
%</samplechap1|samplechap2>
%\fi
%
%\iffalse
%<*samplechap1>
%\fi
% Some text for chapter 1:
%    \begin{macrocode}
\section{one}
some text in chapter one
%    \end{macrocode}

%\iffalse
%</samplechap1>
%\fi
% Some text for chapter 2:
%\iffalse
%<*samplechap2>
%\fi
%    \begin{macrocode}
\section{two}
more text in chapter two
%    \end{macrocode}

%\iffalse
%</samplechap2>
%\fi
%
% %%%%%%%%%%%%%%%%%%%%%%%%%%%%%%%%%%%%%%
% \paragraph{Part Include Files.}
%
% The include files are called |cdocspt3.tex| and |cdocspt4.tex|.
%
%\iffalse
%<*samplepart3|samplepart4>
%\fi

% Optional override for |\version| flag:
%    \begin{macrocode}
%%\providecommand{\version}{final}
%    \end{macrocode}

% Include the main document:
%    \begin{macrocode}
\input{childdoc.def}
\childdocby{cdocsamp}
%    \end{macrocode}

%\iffalse
%</samplepart3|samplepart4>
%\fi
%
%\iffalse
%<*samplepart3>
%\fi
% Some text for part 3:
%    \begin{macrocode}
some text in part three
%    \end{macrocode}

%\iffalse
%</samplepart3>
%\fi
% Some text for part 4:
%\iffalse
%<*samplepart4>
%\fi
%    \begin{macrocode}
more text in part four
%    \end{macrocode}

%\iffalse
%</samplepart4>
%\fi
%
% %%%%%%%%%%%%%%%%%%%%%%%%%%%%%%%%%%%%%%
% \paragraph{Forwarding for a Complete Draft.}
%
% The following forwarding file |cdocsdrf.tex|
% compiles the main document in draft mode:
%\iffalse
%<*sampledraft>
%\fi
%    \begin{macrocode}
\def\version{draft}
\input{childdoc.def}
\childdocforward{cdocsamp}
%    \end{macrocode}

%\iffalse
%</sampledraft>
%\fi
%
% %%%%%%%%%%%%%%%%%%%%%%%%%%%%%%%%%%%%%%
% \paragraph{Forwarding for Final Version of the Chapters.}
%
% The following forwarding files |cdocsfn1.tex| and |cdocsfn2.tex|
% (with identical content)
% compile the final versions of the child documents
% |cdocsch1.tex| and |cdocsch2.tex|, respectively:
%\iffalse
%<*samplefinal>
%\fi
%    \begin{macrocode}
\def\version{final}
\input{childdoc.def}
\childdocforwardprefix[cdocsamp]{cdocsfn}{cdocsch}
%    \end{macrocode}

%\iffalse
%</samplefinal>
%\fi
%
% %%%%%%%%%%%%%%%%%%%%%%%%%%%%%%%%%%%%%%
% \paragraph{Command Line Processing.}
%
% The following three command lines generate the output files
% |cdocscld|, |cdocscl1| and |cdocscl2|
% which should be identical to
% |cdocsdrf|, |cdocsch1| and |cdocsfn2|, respectively:
% \begin{center}
% \begin{tabular}{l}
% |latex -jobname cdocscld \|\\
% |  "\def\version{draft}\input{childdoc.def}\childdocforward{cdocsamp}"|\\
% |latex -jobname cdocscl1 \|\\
% |  "\input{childdoc.def}\childdocforward[cdocsamp]{cdocsch1}"|\\
% |latex -jobname cdocscl2 \|\\
% |  "\def\version{final}\input{childdoc.def}\childdocforward{cdocsch2}"|
% \end{tabular}
% \end{center}
% Note that the trailing backslash on each first line
% merely continues the input to the second line
% (for convenient cut ant paste).
% Furthermore, the command |latex| can be replaced by any
% of its alternative versions such as |pdflatex|.
%
% %%%%%%%%%%%%%%%%%%%%%%%%%%%%%%%%%%%%%%%%%%%%%%%%%%%%%%%%%%%%%%%%%%%%%%%%%%%%%%
% %%%%%%%%%%%%%%%%%%%%%%%%%%%%%%%%%%%%%%%%%%%%%%%%%%%%%%%%%%%%%%%%%%%%%%%%%%%%%%
% \section{Implementation}
%\iffalse
%<*package>
%\fi
%
% This section describes the definitions file |childdoc.def|.

% The definitions cannot be loaded using |\usepackage| or |\RequirePackage|
% which has a mechanism to prevent loading a style file more than once.
% When loading the definitions by means of |\input|
% multiple instances have to be prevented manually:
%\iffalse
%This code needs to be before the `\ProvidesFile' directive
%which is defined at the beginning of this file.
%Therefore it is also placed there and commented out here.
%</package>
%<*discard>
%\fi
%    \begin{macrocode}
\ifdefined\childdocmain\endinput\fi
%    \end{macrocode}
%\iffalse
%</discard>
%<*package>
%\fi
%
% \macro{\ifchilddoc}
% \macro{\ifchilddocmanual}
% The conditional |\ifchilddoc| tells whether a
% child (true) or main (false) document is being compiled.
% The conditional |\ifchilddocmanual| tells whether
% the |\includeonly| mechanism is used (false) or
% the selection of child files must be performed manually (true).
% The definitions initialise to false:
%    \begin{macrocode}
\newif\ifchilddoc
\newif\ifchilddocmanual
%    \end{macrocode}

% \macro{\childdocname}
% \macro{\childdocjob}
% The macro |\childdocname| stores the name of the main document
% to be compiled. The macro |\childdocjob| stores the name of
% the document on which the \LaTeX{} compiler was originally invoked.
% The content of |\jobname| cannot be compared
% to filenames specified in the source due to different catcodes.
% The following code rescans |\jobname|, stores the result
% in |\childdocname| and saves a copy in |\childdocjob|:
%    \begin{macrocode}
\edef\childdocname{\scantokens\expandafter{\jobname\noexpand}}
\let\childdocjob\childdocname
%    \end{macrocode}

% \macro{\childdocdisable}
% The macro |\childdocdisable| prevents the main file
% from being processed more than once.
% At this stage, the main document command |\childdocmain|
% is assumed to be called once again where it should do nothing.
% Any subsequent call to it should prevent
% a secondary processing of the main document
% It overwrites the forwarding commands
% |\childdocof| and |\childdocforward|
% with empty macros to prevent further inclusions of the main document:
%    \begin{macrocode}
\newcommand{\childdocdisable}
{
  \renewcommand{\childdocmain}[1]{\renewcommand{\childdocmain}[1]{\endinput}}
  \renewcommand{\childdocof}[1]{}
  \renewcommand{\childdocby}[2][]{}
  \renewcommand{\childdocforward}[2][]{}
  \renewcommand{\childdocdisable}{}
}
%    \end{macrocode}

% \macro{\childdocmain}
% The macro |\childdocmain| is to be called at the top of the main file
% with nothing or the main filename (without extension) as argument.
% First, it breaks loops.
% If the argument is not empty and does not match |\childdocname|
% (which is set by the first inclusion of |childdoc.def|),
% |\ifchilddoc| is set to true, |\includeonly| is applied to the child file
% and |\jobname| is set to the main file
% (for proper handling of |.aux| files):
%    \begin{macrocode}
\newcommand{\childdocmain}[1]
{
  \childdocdisable\childdocmain{}
  \if?#1?\else
    \begingroup
      \def\childdoctmp{#1}
      \ifx\childdoctmp\childdocname
        \def\childdoctmp{}
      \else
        \def\childdoctmp
        {
          \childdoctrue
          \includeonly{\childdocname}
          \def\childdocjob{#1}
          \def\jobname{#1}
        }
      \fi
      \expandafter
    \endgroup
    \childdoctmp
  \fi
}
%    \end{macrocode}

% \macro{\childdocof}
% The command |\childdocof| redirects
% compilation to the main file |#1|.
%    \begin{macrocode}
\newcommand{\childdocof}[1]
{
  \childdocdisable
  \childdoctrue
  \includeonly{\childdocname}
  \def\jobname{#1}
  \def\childdocjob{#1}
  \input{#1}
}
%    \end{macrocode}

% \macro{\childdocby}
% The command |\childdocby| ....
%    \begin{macrocode}
\newcommand{\childdocby}[2][]
{
  \childdocdisable
  \childdoctrue
  \childdocmanualtrue
  \if?#1?\else
    \def\jobname{#2}
  \fi
  \def\childdocjob{#2}
  \input{#2}
  \endinput
}
%    \end{macrocode}

% \macro{\childdocforward}
% The command |\childdocforward| redirects
% compilation to the main file or
% (if the optional argument is given) a child file.
% Parameters are set as if the main file
% or a child file starting with |\childdocof| was compiled.
% Then compilation is handed over to the main file:
%    \begin{macrocode}
\newcommand{\childdocforward}[2][]
{
  \begingroup
    \if?#1?
      \def\childdoctmp
      {
        \def\childdocname{#2}
        \def\childdocjob{#2}
        \def\jobname{#2}
        \input{#2}
        \endinput
      }
    \else
      \def\childdoctmp
      {
        \childdocdisable
        \def\childdocname{#2}
        \childdoctrue
        \includeonly{#2}
        \def\childdocjob{#1}
        \def\jobname{#1}
        \input{#1}
        \endinput
      }
    \fi
    \expandafter
  \endgroup
  \childdoctmp
}
%    \end{macrocode}

% \macro{\childdocforwardprefix}
% The command |\childdocforwardprefix| redirects
% compilation to the main or a child file by means of a pattern.
% The prefix |#1| in the current filename is replaced by |#2|
% and the suffix of the current filename is kept
% (it is assumed that the filename does not contain the substring `|~~~|'
% which is used as a delimiter).
% Compilation is handed over to the new file by |\childdocforward|:
%    \begin{macrocode}
\newcommand{\childdocforwardprefix}[3][]
{
  \begingroup
    \def\childdocextract #2##1~~~{\def\childdoctmp{\childdocforward[#1]{#3##1}}}
    \expandafter\childdocextract\childdocname~~~
    \expandafter
  \endgroup
  \childdoctmp
}
%    \end{macrocode}

% \macro{\childdoc}
% The deprecated macro |\childdoc| is a legacy version of |\childdocmain|:
%    \begin{macrocode}
\newcommand{\childdoc}{\childdocmain}
%    \end{macrocode}

% \macro{\childdocredirect}
% The deprecated macro |\childdocredirect| is a legacy version
% of |\childdocforward| and |\childdocforwardprefix|:
%    \begin{macrocode}
\newcommand{\childdocredirect}[2][]
{
  \begingroup
    \if?#1?
      \def\childdoctmp{\childdocforward{#2}}
    \else
      \def\childdoctmp{\childdocforwardprefix{#1}{#2}}
    \fi
    \expandafter
  \endgroup
  \childdoctmp
}
%    \end{macrocode}

%\iffalse
%</package>
%\fi
%
\endinput
|
and perform the replacements as outlined below.
Instead of |\childdocmain{|\textit{main}|}| add the following code
to the top of the main file:
%
\begin{center}
\begin{tabular}{l}
|\||ifdefined\childdocname\endinput\||fi\newif\ifchilddoc|\\
|\edef\childdocname{\scantokens\expandafter{\jobname\noexpand}}|\\
|\def\childdocmain{|\textit{main}|}\||ifx\childdocmain\childdocname\||else|\\
|\childdoctrue\includeonly{\childdocname}\let\jobname\childdocmain\||fi|\\
\end{tabular}
\end{center}
%
Instead of |\childdocof{|\textit{main}|}| just include the main file
at the top of each child file:
%
\begin{center}
|\input{|\textit{main}|}|
\end{center}
%
A simple redirection |\childdocforward{|\textit{dest}|}| is achieved by:
%
\begin{center}
|\def\jobname{|\textit{dest}|}\input{\jobname}|
\end{center}
%
The redirection with prefix
|\childdocforwardprefix[|\textit{prefix}|]{|\textit{dest}|}|
is accomplished by:
%
\begin{center}
\begin{tabular}{l}
|{\edef\jobname{\scantokens\expandafter{\jobname\noexpand}}|\\
|\def\redirectjob |\textit{prefix}|#1~~~{\gdef\jobname{|\textit{dest}|#1}}|\\
|\expandafter\redirectjob\jobname~~~}\input{\jobname}|
\end{tabular}
\end{center}

In an alternative approach,
child documents can be compiled by a specific command line
without additional code or specific definitions:
%
\begin{center}
|... -jobname "|\textit{target}|" "|[\textit{flags}]%
|\includeonly{|\textit{dest}|}\input{|\textit{main}|}"|
\end{center}
%

%%%%%%%%%%%%%%%%%%%%%%%%%%%%%%%%%%%%%%%%%%%%%%%%%%%%%%%%%%%%%%%%%%%%%%%%%%%%%%%%
%%%%%%%%%%%%%%%%%%%%%%%%%%%%%%%%%%%%%%%%%%%%%%%%%%%%%%%%%%%%%%%%%%%%%%%%%%%%%%%%
\section{Information}

%%%%%%%%%%%%%%%%%%%%%%%%%%%%%%%%%%%%%%%%%%%%%%%%%%%%%%%%%%%%%%%%%%%%%%%%%%%%%%%%
\subsection{Copyright}

Copyright \copyright{} 2017--2018 Niklas Beisert

This work may be distributed and/or modified under the
conditions of the \LaTeX{} Project Public License, either version 1.3
of this license or (at your option) any later version.
The latest version of this license is in
  \url{http://www.latex-project.org/lppl.txt}
and version 1.3 or later is part of all distributions of \LaTeX{}
version 2005/12/01 or later.

This work has the LPPL maintenance status `maintained'.

The Current Maintainer of this work is Niklas Beisert.

This work consists of the files |README.txt|, |childdoc.ins| and |childdoc.dtx|
as well as the derived files |childdoc.def|, |cdocsamp.tex|
with |cdocsch1.tex|, |cdocsch2.tex|, |cdocspt3.tex|, |cdocspt4.tex|,
|cdocsdrf.tex|, |cdocsfn1.tex|, |cdocsfn2.tex|
as well as |childdoc.pdf|.

%%%%%%%%%%%%%%%%%%%%%%%%%%%%%%%%%%%%%%%%%%%%%%%%%%%%%%%%%%%%%%%%%%%%%%%%%%%%%%%%
\subsection{Files and Installation}

The package consists of the files:
%
\begin{center}
\begin{tabular}{ll}
    |README.txt|   & readme file \\
    |childdoc.ins| & installation file \\
    |childdoc.dtx| & source file \\
    |childdoc.def| & definition file \\
    |cdocsamp.tex| & sample main file \\
    |cdocsch1.tex| & sample include file \\
    |cdocsch2.tex| & sample include file \\
    |cdocspt3.tex| & sample part file \\
    |cdocspt4.tex| & sample part file \\
    |cdocsdrf.tex| & sample redirection file \\
    |cdocsfn1.tex| & sample redirection file \\
    |cdocsfn2.tex| & sample redirection file \\
    |childdoc.pdf| & manual
\end{tabular}
\end{center}
%
The distribution consists of the files
|README.txt|, |childdoc.ins| and |childdoc.dtx|.
%
\begin{itemize}
\item
Run (pdf)\LaTeX{} on |childdoc.dtx|
to compile the manual |childdoc.pdf| (this file).
\item
Run \LaTeX{} on |childdoc.ins| to create the definitions file |childdoc.def|
and the sample |cdocsamp.tex| with include files
|cdocsch1.tex|, |cdocsch2.tex|, |cdocspt3.tex|, |cdocspt4.tex|,
|cdocsdrf.tex|, |cdocsfn1.tex|, |cdocsfn2.tex|.
Then copy the file |childdoc.def| to an appropriate directory of your \LaTeX{}
distribution, e.g.\ \textit{texmf-root}|/tex/latex/childdoc|.
\end{itemize}

%%%%%%%%%%%%%%%%%%%%%%%%%%%%%%%%%%%%%%%%%%%%%%%%%%%%%%%%%%%%%%%%%%%%%%%%%%%%%%%%
\subsection{Related CTAN Packages}

There are several other packages which offer a similar functionality:
%
\begin{itemize}
\item
The packages
\href{http://ctan.org/pkg/docmute}{\textsf{docmute}},
\href{http://ctan.org/pkg/includex}{\textsf{includex}} and
\href{http://ctan.org/pkg/standalone}{\textsf{standalone}}
provide commands to include only the document body of
a child file thus allowing both files to be compiled individually.
\item
The packages \href{http://ctan.org/pkg/subdocs}{\textsf{subdocs}}
and \href{http://ctan.org/pkg/subfiles}{\textsf{subfiles}}
provide structures in which the main and child documents can be
encapsulated and allowing them to be compiled individually.
The inclusion mechanism is different from the conventional |\include|.
\item
The package \href{http://ctan.org/pkg/combine}{\textsf{combine}}
is an elaborate solution to combine several documents into one.
\end{itemize}
%
See also the CTAN topic \href{http://ctan.org/topic/subdocs}{\textsf{subdocs}}
for further related packages.
The present package differs from the above solutions in that
a document structure constructed with the conventional |\include| mechanism
just needs two extra commands at the top of every file
such that all constituent files can be compiled individually.

%%%%%%%%%%%%%%%%%%%%%%%%%%%%%%%%%%%%%%%%%%%%%%%%%%%%%%%%%%%%%%%%%%%%%%%%%%%%%%%%
%\subsection{Feature Suggestions}
%
%The following is a list of features which may be useful for future
%versions of this package:
%%
%\begin{itemize}
%\item
%\ldots
%\end{itemize}

%%%%%%%%%%%%%%%%%%%%%%%%%%%%%%%%%%%%%%%%%%%%%%%%%%%%%%%%%%%%%%%%%%%%%%%%%%%%%%%%
\subsection{Revision History}

%%%%%%%%%%%%%%%%%%%%%%%%%%%%%%%%%%%%%%%%
\paragraph{v2.0:} 2018/12/30

\begin{itemize}
\item
immediate forward processing
\item
added |\childdocby| mechanism
\item
manual restructured
\end{itemize}

%%%%%%%%%%%%%%%%%%%%%%%%%%%%%%%%%%%%%%%%
\paragraph{v1.6:} 2018/01/17

\begin{itemize}
\item
application for development of include files
\item
corrections to manual
\end{itemize}

%%%%%%%%%%%%%%%%%%%%%%%%%%%%%%%%%%%%%%%%
\paragraph{v1.5:} 2017/05/21

\begin{itemize}
\item
more complete structuring introduced
\item
|\childdocof| introduced
\item
|\childdoc| renamed to |\childdocmain|
\item
|\childredirect| renamed to |\childdocforward| and |\childdocforwardprefix|
and functionality expanded
\end{itemize}

%%%%%%%%%%%%%%%%%%%%%%%%%%%%%%%%%%%%%%%%
\paragraph{v1.0:} 2017/04/27

\begin{itemize}
\item
manual and install package
\item
first version published on CTAN
\end{itemize}

%%%%%%%%%%%%%%%%%%%%%%%%%%%%%%%%%%%%%%%%
\paragraph{v0.6:} 2017/04/26

\begin{itemize}
\item
redirection mechanism added
\end{itemize}

%%%%%%%%%%%%%%%%%%%%%%%%%%%%%%%%%%%%%%%%
\paragraph{v0.5:} 2017/04/26

\begin{itemize}
\item
functionality in definition file
\end{itemize}


%%%%%%%%%%%%%%%%%%%%%%%%%%%%%%%%%%%%%%%%%%%%%%%%%%%%%%%%%%%%%%%%%%%%%%%%%%%%%%%%
%%%%%%%%%%%%%%%%%%%%%%%%%%%%%%%%%%%%%%%%%%%%%%%%%%%%%%%%%%%%%%%%%%%%%%%%%%%%%%%%
%%%%%%%%%%%%%%%%%%%%%%%%%%%%%%%%%%%%%%%%%%%%%%%%%%%%%%%%%%%%%%%%%%%%%%%%%%%%%%%%
\appendix

\settowidth\MacroIndent{\rmfamily\scriptsize 000\ }

 \DocInput{childdoc.dtx}

\end{document}
%</driver>
% \fi
%
% %%%%%%%%%%%%%%%%%%%%%%%%%%%%%%%%%%%%%%%%%%%%%%%%%%%%%%%%%%%%%%%%%%%%%%%%%%%%%%
% %%%%%%%%%%%%%%%%%%%%%%%%%%%%%%%%%%%%%%%%%%%%%%%%%%%%%%%%%%%%%%%%%%%%%%%%%%%%%%
% \section{Sample}
%\iffalse
%<*samplemain>
%\fi
%
% The following presents a sample document
% with two chapters, two parts, a title page,
% a compile flag as well as three forwarding files to set the flag.
% It consists of eight |.tex| files:
% \begin{center}
% \begin{tabular}{ll}
% |cdocsamp.tex|&main file\\
% |cdocsch1.tex|&include file for chapter 1\\
% |cdocsch2.tex|&include file for chapter 2\\
% |cdocspt3.tex|&include file for part 3\\
% |cdocspt4.tex|&include file for part 4\\
% |cdocsdrf.tex|&forwarding file for main file in draft mode\\
% |cdocsfi1.tex|&forwarding file for final version of chapter 1\\
% |cdocsfi2.tex|&forwarding file for final version of chapter 2\\
% \end{tabular}
% \end{center}
% Each of the eight files can be compiled directly by the \LaTeX{} compiler.
%
% %%%%%%%%%%%%%%%%%%%%%%%%%%%%%%%%%%%%%%
% \paragraph{Main File.}
%
% The main file is called |cdocsamp.tex|.
%
% Load the \textsf{childdoc} definitions and
% declare the filename for the main document:
%    \begin{macrocode}
% \iffalse
%
% childdoc.dtx Copyright (C) 2017-2018 Niklas Beisert
%
% This work may be distributed and/or modified under the
% conditions of the LaTeX Project Public License, either version 1.3
% of this license or (at your option) any later version.
% The latest version of this license is in
%   http://www.latex-project.org/lppl.txt
% and version 1.3 or later is part of all distributions of LaTeX
% version 2005/12/01 or later.
%
% This work has the LPPL maintenance status `maintained'.
%
% The Current Maintainer of this work is Niklas Beisert.
%
% This work consists of the files childdoc.dtx and childdoc.ins
% and the derived files childdoc.def and cdocsamp.tex with
% cdocsch1.tex, cdocsch2.tex, cdocsdrf.tex, cdocsfn1.tex, cdocsfn2.tex.
%
%<package>\ifdefined\childdocmain\endinput\fi
%<package>\ProvidesFile{childdoc.def}[2018/12/30 v2.0 child document driver]
%<samplemain>\ProvidesFile{cdocsamp.tex}[2018/12/30 v2.0 sample for childdoc]
%<*driver>
%\ProvidesFile{childdoc.drv}[2018/12/30 v2.0 childdoc reference manual file]
\PassOptionsToClass{10pt,a4paper}{article}
\documentclass{ltxdoc}

\usepackage[margin=35mm]{geometry}
\usepackage{hyperref}
\usepackage{hyperxmp}
\usepackage[usenames]{color}

\hypersetup{colorlinks=true}
\hypersetup{pdfstartview=FitH}
\hypersetup{pdfpagemode=UseNone}
\hypersetup{pdfsource={}}
\hypersetup{pdflang={en-UK}}
\hypersetup{pdfcopyright={Copyright 2017-2018 Niklas Beisert.
  This work may be distributed and/or modified under the
  conditions of the LaTeX Project Public License, either version 1.3
  of this license or (at your option) any later version.}}
\hypersetup{pdflicenseurl={http://www.latex-project.org/lppl.txt}}
\hypersetup{pdfcontactaddress={ETH Zurich, ITP, HIT K,
  Wolfgang-Pauli-Strasse 27}}
\hypersetup{pdfcontactpostcode={8093}}
\hypersetup{pdfcontactcity={Zurich}}
\hypersetup{pdfcontactcountry={Switzerland}}
\hypersetup{pdfcontactemail={nbeisert@itp.phys.ethz.ch}}
\hypersetup{pdfcontacturl={http://people.phys.ethz.ch/\xmptilde nbeisert/}}

\newcommand{\secref}[1]{\hyperref[#1]{section \ref*{#1}}}

\parskip1ex
\parindent0pt
\let\olditemize\itemize
\def\itemize{\olditemize\parskip0pt}

\begin{document}

\title{The \textsf{childdoc} Package}
\hypersetup{pdftitle={The childdoc Package}}
\author{Niklas Beisert\\[2ex]
  Institut f\"ur Theoretische Physik\\
  Eidgen\"ossische Technische Hochschule Z\"urich\\
  Wolfgang-Pauli-Strasse 27, 8093 Z\"urich, Switzerland\\[1ex]
  \href{mailto:nbeisert@itp.phys.ethz.ch}
  {\texttt{nbeisert@itp.phys.ethz.ch}}}
\hypersetup{pdfauthor={Niklas Beisert}}
\hypersetup{pdfsubject={Manual for the LaTeX2e Package childdoc}}
\date{30 December 2018, \textsf{v2.0}}
\maketitle

\begin{abstract}\noindent
\textsf{childdoc} is a \LaTeXe{} package
that enables the direct compilation
of document sections included by |\include|
to individual files.
\end{abstract}

\begingroup
\parskip0ex
\tableofcontents
\endgroup

%%%%%%%%%%%%%%%%%%%%%%%%%%%%%%%%%%%%%%%%%%%%%%%%%%%%%%%%%%%%%%%%%%%%%%%%%%%%%%%%
%%%%%%%%%%%%%%%%%%%%%%%%%%%%%%%%%%%%%%%%%%%%%%%%%%%%%%%%%%%%%%%%%%%%%%%%%%%%%%%%
\section{Introduction}

\LaTeX{} provides a mechanism to structure a large document (such as a book)
into a main file and several child files (containing the chapters)
using the |\include| command.
This mechanism is beneficial for documents
which span hundreds of pages in order to
make the source file(s) more manageable.
Moreover, compilation can be restricted to
selected child files by means of the |\includeonly| command.
The latter feature can be used to reduce the compilation time while editing
(this was significantly more useful in the earlier days of \LaTeX{})
or to generate a smaller document which is easier to navigate.
Another application of |\includeonly| is to generate
documents consisting of selected parts of the complete document.

However, there are a few drawbacks of the plain |\include| mechanism:
\begin{itemize}
\item
The child files cannot be compiled on their own,
they can only be compiled via the main file.
A naive editing environment
(such as a text editor with an option
to have the current file processed by \LaTeX)
may require one to switch to the main file before compiling;
attempting to compile the child file produces errors.
\item
The main file must be modified (each time)
to adjust the |\includeonly| command
to the present needs. This easily leaves the main file in a messy state.
\item
The generated document will always carry the filename
of the main document. This is inconvenient if
several child files are to be compiled and
to be kept for distribution.
\end{itemize}

The present package provides a simple interface
to make child files individually compilable by \LaTeX{}.
Compiling a child file then has the same effect as compiling
the main file with an |\includeonly| command
to select the appropriate child.
Moreover the generated document will carry the name of the child
rather than the main file.
This resolves all three above issues.

This feature is meant to make the editing of books,
thesis documents and lecture notes somewhat more convenient.
However, the package can also be used efficiently for
composing a series of documents (such as exercise sheets)
which are typically distributed individually.
It then assists the author in generating the individual documents
(potentially in different versions)
as well as a document containing the collected series.
Another application is in developing style files
or other kinds of included material
where compilation of the style file could redirect
to a sample or test file.

%%%%%%%%%%%%%%%%%%%%%%%%%%%%%%%%%%%%%%%%%%%%%%%%%%%%%%%%%%%%%%%%%%%%%%%%%%%%%%%%
%%%%%%%%%%%%%%%%%%%%%%%%%%%%%%%%%%%%%%%%%%%%%%%%%%%%%%%%%%%%%%%%%%%%%%%%%%%%%%%%
\section{Usage}

First of all, the package \textsf{childdoc} is \emph{not} a standard
\LaTeXe{} |.sty| style file! Therefore it needs to be invoked in
a non-standard way.

%%%%%%%%%%%%%%%%%%%%%%%%%%%%%%%%%%%%%%%%%%%%%%%%%%%%%%%%%%%%%%%%%%%%%%%%%%%%%%%%
\subsection{Included Files}
\label{sec:include}

%%%%%%%%%%%%%%%%%%%%%%%%%%%%%%%%%%%%%%%%
\DescribeMacro{\childdocmain}
To use the package, add the commands
\begin{center}
\begin{tabular}{l}
|\input{childdoc.def}|\\
|\childdocmain{}|\\
\end{tabular}
\end{center}
at the very top of the main \LaTeX{} file,
in particular \emph{before} the |\documentclass| statement!
The argument of |\childdocmain| should be left empty
(but it must be present).

%%%%%%%%%%%%%%%%%%%%%%%%%%%%%%%%%%%%%%%%
\DescribeMacro{\childdocof}
Furthermore, add the commands
\begin{center}
\begin{tabular}{l}
|\input{childdoc.def}|\\
|\childdocof{|\textit{main}|}|\\
\end{tabular}
\end{center}
at the top of every child file \textit{child}
which is included by |\include{|\textit{child}|}|
from within the main file
(or at least for those files to be compiled individually).
The argument \textit{main} must be the filename of the main file.

There are a couple of
considerations in setting up the main and child documents:

%%%%%%%%%%%%%%%%%%%%%%%%%%%%%%%%%%%%%%%%
\paragraph{Restrictions.}

Please note the following restrictions:
\begin{itemize}
\item
|\childdocmain| must be called with one argument \textit{main}
to ensure compatibility with earlier version of the package.
It must either be empty (|\childdocmain{}|)
or precisely match the filename of the main file in which it is specified.
See \secref{sec:detection} for further information.
\item
The filename \textit{main} must be specified without the |.tex| extension.
\item
The filename \textit{main} is case sensitive
(even in case-insensitive file systems)
due to internal string comparison.
\item
The argument \textit{main} should be fully expanded, it cannot be a macro.
\item
Subdirectories and special characters should be avoided in filenames.
\item
The command |\childdocmain{|\textit{main}|}| must be followed by a whitespace.
It should not be followed immediately by another command
or by a comment mark `|%|'.
This is because the \TeX{} parser reads the token immediately following
the argument of |\childdocmain| and puts it
at the beginning of every child section;
however, a white\-space is ignored.
\end{itemize}

%%%%%%%%%%%%%%%%%%%%%%%%%%%%%%%%%%%%%%%%
\paragraph{Content of Main File.}

It is advisable to place all content in the child files included by |\include|.
Any output contained in the main file will appear in all child documents
unless suppressed manually;
it cannot be suppressed automatically by the |\includeonly| directive
and thus should normally be avoided.
A method to include some content in the main file
by means of conditional processing is described in \secref{sec:conditional}.

%%%%%%%%%%%%%%%%%%%%%%%%%%%%%%%%%%%%%%%%
\paragraph{Page Numbering.}

When only a part of the document is compiled,
the appropriate numbering of pages
(as well as other status parameters)
is determined from the |.aux| files.
The latter contain information from previous passes.
However this information needs to propagate through
all intermediate child documents.
Therefore the page numbering in child documents may well
be inconsistent until the complete document is compiled at least once.

A useful (if unconventional) way to always ensure a consistent
page numbering is to restart the numbering in each child document
and denote the pages by `\textit{child}|.|\textit{page}'
where \textit{child} represents the chapter/section number of the child file.
This can be achieved by the command
|\numberwithin{page}{|\textit{child}|}|
of the \textsf{amsmath} package
where \textit{child} can be |chapter| or |section|
depending on the chosen structuring.
Alternatively, one can modify the macro |\thepage| appropriately
and reset the counter |page| at the start of each child file.

%%%%%%%%%%%%%%%%%%%%%%%%%%%%%%%%%%%%%%%%%%%%%%%%%%%%%%%%%%%%%%%%%%%%%%%%%%%%%%%%
\subsection{Conditional Processing}
\label{sec:conditional}

The package provides a mechanism to compile different versions
of a document. To customise the versions further some conditional processing
can come in handy to distinguish which version is being compiled.
The package provides two macros to describe the compilation context:

%%%%%%%%%%%%%%%%%%%%%%%%%%%%%%%%%%%%%%%%
\DescribeMacro{\ifchilddoc}
The conditional |\ifchilddoc| distinguishes between the compilation of
child documents and the main document:
%
\begin{center}
|\ifchilddoc |\textit{child-code}| |[|\||else |\textit{main-code}]| \||fi|
\end{center}

%%%%%%%%%%%%%%%%%%%%%%%%%%%%%%%%%%%%%%%%
\DescribeMacro{\childdocname}
\DescribeMacro{\childdocjob}
The macro |\childdocname| contains the filename (without extension)
of the main or child file being processed.
Note that |\childdocjob| will always contain the name of the main file.

%%%%%%%%%%%%%%%%%%%%%%%%%%%%%%%%%%%%%%%%
\paragraph{Title Page.}

Conditional processing can be used to include a title or banner page
in the main document when proper precautions are taken.
Importantly, the code in the main file should ensure that the page counter
(as well as other status parameters which are stored in the |.aux| files)
takes the same value after the conditional processing.
Otherwise the page numbers may take divergent values
depending on which part is compiled.

For example, a title page could be declared by:
%
\begin{center}
\begin{tabular}{l}
|\ifchilddoc\||else|\\
|\addtocounter{page}{-1}|\\
\textit{code for title page}\\
|\newpage|\\
|\||fi|
\end{tabular}
\end{center}
%
A banner page for the child documents can be generated by:
%
\begin{center}
\begin{tabular}{l}
|\ifchilddoc|\\
|\addtocounter{page}{-1}|\\
\textit{code for banner page}\\
|\newpage|\\
|\||fi|
\end{tabular}
\end{center}
%
Here one could write a message such as:
\begin{center}
|This is the part \childdocname{} of \childdocjob{}.|
\end{center}

%%%%%%%%%%%%%%%%%%%%%%%%%%%%%%%%%%%%%%%%%%%%%%%%%%%%%%%%%%%%%%%%%%%%%%%%%%%%%%%%
\subsection{Flags}
\label{sec:flags}

The package makes it easy to generate different versions
of the main or child documents.
To this end compilation flags can be defined
and assigned different default values.
They will be particularly useful in conjunction
with the forwarding mechanism described in \secref{sec:forward}.

For example, it may be useful to have a flag |\version|
which can be set to |draft| or |final|.
The document source will contain some conditional code
depending on the value of |\version|.
Suppose further, the flag should default to |final| for the main file
and to |draft| for child files
which is a natural assignment for editing the document.
This is achieved by placing the following code
in the preamble of the main document
(below the |\childdocmain| directive):
%
\begin{center}
\begin{tabular}{l}
|\ifchilddoc|\\
|\providecommand{\version}{draft}|\\
|\||else|\\
|\providecommand{\version}{final}|\\
|\||fi|
\end{tabular}
\end{center}
%
The definition by |\providecommand| makes sure
that previous definitions are not overwritten.
Further statements |\providecommand{\version}{...}|
can thus be added before the above code to override it.

For the main file, one might add a line
(between |\childdocmain| and the above block)
%
\begin{center}
|%\ifchilddoc\||else\providecommand{\version}{draft}\||fi|
\end{center}
%
which can be uncommented to produce a draft version.
Likewise one can add a line to the very top of a child file
(above the |\childdocof{|\textit{main}|}| directive)
%
\begin{center}
|%\providecommand{\version}{final}|
\end{center}
%
which can be uncommented to produce the final version of this child document.

%%%%%%%%%%%%%%%%%%%%%%%%%%%%%%%%%%%%%%%%%%%%%%%%%%%%%%%%%%%%%%%%%%%%%%%%%%%%%%%%
\subsection{Forwarding}
\label{sec:forward}

Different versions of the main or child documents
using compilation flags as described in \secref{sec:flags}
can be (permanently) stored in different files
for convenient compilation, viewing and distribution.
To this end, the package defines a command
to pass on compilation to a different file:

%%%%%%%%%%%%%%%%%%%%%%%%%%%%%%%%%%%%%%%%
\DescribeMacro{\childdocforward}
The command |\childdocforward| redirects processing to
another source file:
%
\begin{center}
\begin{tabular}{l}
|\input{childdoc.def}|\\
|\childdocforward[|\textit{main}|]{|\textit{dest}|}|\\
\end{tabular}
\end{center}
%
The argument \textit{dest} is the destination file
(without extension).
It should be the main file or one of the child files.
Note that further \textsf{childdoc} directives
such as |\childdocof| and |\childdocforward|
in the indicated file will be processed in this form.
The optional argument \textit{main}
passes on directly to the main file \textit{main}
while pretending to compile the child \textit{dest}.
This form behaves as if \textit{dest}
issues |\childdocof{|\textit{main}|}| right away,
and no further \textsf{childdoc} directives will be processed.

%%%%%%%%%%%%%%%%%%%%%%%%%%%%%%%%%%%%%%%%
\DescribeMacro{\...prefix}
In the alternative form |\childdocforwardprefix|,
%
\begin{center}
\begin{tabular}{l}
|\input{childdoc.def}|\\
|\childdocforwardprefix[|\textit{main}|]{|\textit{prefix}|}{|\textit{dest}|}|
\end{tabular}
\end{center}
%
the destination file is determined by a pattern
depending on the current file:
To make this work, the current file must be called
`{\textit{prefix}\hspace{0.2em}\textit{suffix}}'
with \textit{prefix} matching precisely the argument.
Processing is then passed on to the file
`{\textit{dest}\hspace{0.2em}\textit{suffix}}'.
Surely, the same effect is achieved by
directly specifying the
argument `{\textit{dest}\hspace{0.2em}\textit{suffix}}'
in the first form.
However, that requires to set up a different file
for each child. With the alternative form of the command
all these files can have exactly the same content
which simplifies setting them up and maintaining them.

For example, the following file |draft.tex|
with a compilation flag |\version| as described in \secref{sec:flags}
compiles the main document as a draft:
%
\begin{center}
\begin{tabular}{l}
|\def\version{draft}|\\
|\input{childdoc.def}|\\
|\childdocforward{|\textit{main}|}|
\end{tabular}
\end{center}
%
Likewise, the following files |final|\textit{nn}|.tex|
compile the final version of the child document
|child|\textit{nn}|.tex|:
%
\begin{center}
\begin{tabular}{l}
|\def\version{final}|\\
|\input{childdoc.def}|\\
|\childdocforwardprefix{final}{child}|
\end{tabular}
\end{center}
%

Note that when several versions of a main file and/or of each child file
are to be generated, it may be convenient to set up a |Makefile| or
shell script to automatise the process.

%%%%%%%%%%%%%%%%%%%%%%%%%%%%%%%%%%%%%%%%%%%%%%%%%%%%%%%%%%%%%%%%%%%%%%%%%%%%%%%%
\subsection{Command Line Processing}
\label{sec:commandline}

The effect of redirection files can also be achieved by invoking
the \LaTeX{} compiler with a more elaborate command line.
Most conveniently this should be done as part
of a shell script or a |Makefile|.

When using \textsf{childdoc} in the main file, the following
command lines effectively perform a redirection
(note that depending on the shell being used,
backslashes may have to be doubled: `|\|' $\to$ `|\\|'):
%
\begin{center}
|... -jobname "|\textit{target}|" |\\|"|[\textit{flags}]%
|\input{childdoc.def}\childdocforward[|\textit{main}|]{|\textit{dest}|}"|
\end{center}
%
Here \textit{target} is the name of the output file,
\textit{main} is the name of the main file
and \textit{dest} is the name of the main or child file to be processed
(all filenames without extensions).
The optional argument \textit{main} can be omitted
if \textit{main} matches \textit{dest}.
Optionally, compilation \textit{flags} can be defined via |\def| commands.
This command line makes the \TeX{} engine believe
it is compiling the file \textit{target}
whose content is specified as the latter parameter.
The provided code then forwards the processing to
\textit{main} or \textit{dest} as described in \secref{sec:forward}.

%%%%%%%%%%%%%%%%%%%%%%%%%%%%%%%%%%%%%%%%%%%%%%%%%%%%%%%%%%%%%%%%%%%%%%%%%%%%%%%%
\subsection{Include by Input}
\label{sec:input}

Including child documents by |\include| has some restrictions by design.
Most notably, the content of a child document always occupies
its own set of pages; pages cannot be shared between child documents.
Usually, this behaviour makes perfect sense
because each child document contain an essential part of the document.
However, in some situations it may be desirable to compose
a document from a collection of parts
without having mandatory page breaks between then.
For this case, the package
provides a mechanism to include parts
by |\input| which can also be processed individually.
However, by construction this mechanism
requires manual handling of the content to be output.

%%%%%%%%%%%%%%%%%%%%%%%%%%%%%%%%%%%%%%%%
\DescribeMacro{\ifchilddocmanual}
The main file should be prepared as usual, see \secref{sec:include}.
However, the document body must make a distinction
between processing of an individual part and of the main document, e.g.:
%
\begin{center}
\begin{tabular}{l}
|\ifchilddocmanual|\\
|\input{\childdocname}|\\
|\||else|\\
\textit{document body with }|\input{|\textit{part}|}|\\
|\||fi|
\end{tabular}
\end{center}
%
The conditional |\ifchilddocmanual| is true whenever
a part to be included by |\input| is being compiled,
and the name of the part is stored in |\childdocname|.

%%%%%%%%%%%%%%%%%%%%%%%%%%%%%%%%%%%%%%%%
\DescribeMacro{\childdocby}
Each part to be included by |\input| should start with:
%
\begin{center}
\begin{tabular}{l}
|\input{childdoc.def}|\\
|\childdocby{|\textit{main}|}|\\
\end{tabular}
\end{center}
%
The directive |\childdocby| is similar to |\childdocof|
described in \secref{sec:include},
but the subsequent selection of content must be done manually.
To that end, both |\ifchilddoc| and |\ifchilddocmanual|
will be true upon processing of a part,
and the name of the part is stored in |\childdocname|.
Note that |\jobname| will be set to the filename of the current part
so that each part receives an individual |.aux| file
that does not interfere with the |.aux| file(s) of the main document.
This behaviour can be altered by the alternative form
|\childdocby[*]{|\textit{main}|}| (with a non-empty optional argument)
which uses the |.aux| file of the main document
by setting |\jobname| to \textit{main}.

%%%%%%%%%%%%%%%%%%%%%%%%%%%%%%%%%%%%%%%%%%%%%%%%%%%%%%%%%%%%%%%%%%%%%%%%%%%%%%%%
\subsection{Driver Development}
\label{sec:driver}

The \textsf{childdoc} mechanism can also be use for the development
of definition files such as \LaTeX{} styles or classes.
This case differs from the above setup with multiple parts
included by |\include| in that no |\includeonly| should be invoked.
This can be achieved by starting the include file
(before |\ProvidesPackage|) with:
%
\begin{center}
\begin{tabular}{l}
|\input{childdoc.def}|\\
|\childdocforward{|\textit{main}|}|\\
\end{tabular}
\end{center}
%
or alternatively with:
%
\begin{center}
\begin{tabular}{l}
|\input{childdoc.def}|\\
|\childdocby{|\textit{main}|}|\\
\end{tabular}
\end{center}
%
Both forms have slightly different effects as described above.
The main file is prepared as usual, see \secref{sec:include}.

%%%%%%%%%%%%%%%%%%%%%%%%%%%%%%%%%%%%%%%%%%%%%%%%%%%%%%%%%%%%%%%%%%%%%%%%%%%%%%%%
\subsection{Legacy Detection}
\label{sec:detection}

The directive |\childdocmain| in the main file can detect
whether the complete document or merely a child is to be compiled
even without using the directive |\childdocof|.
This method is deprecated because it is less robust
and there is no compelling reason to use it;
it is merely provided for backward compatibility
and it may be removed in future versions.

If the detection mechanism is to be used,
it is mandatory to correctly specify
the filename of the main file as the argument of |\childdocmain|:
%
\begin{center}
\begin{tabular}{l}
|\input{childdoc.def}|\\
|\childdocmain{|\textit{main}|}|\\
\end{tabular}
\end{center}
%
If |\jobname| does not match the argument \textit{main} of |\childdocmain|,
it is assumed that |\jobname| points to the child file to be compiled.
When using |\childdocmain| with the main file specified as argument,
it suffices to start a child file
with just |\input{|\textit{main}|}|
without loading of the package and using |\childdocof|.
If instead all processing is done
with the appropriate \textsf{childdoc} directives,
the argument of \textit{main} of |\childdocmain| can be empty.

An alternative version of the command line processing described
in \secref{sec:commandline} using the detection mechanism reads:
%
\begin{center}
|... -jobname "|\textit{target}|" "|[\textit{flags}]%
[|\def\jobname{|\textit{dest}|}|]|\input{|\textit{main}|}"|
\end{center}

%%%%%%%%%%%%%%%%%%%%%%%%%%%%%%%%%%%%%%%%%%%%%%%%%%%%%%%%%%%%%%%%%%%%%%%%%%%%%%%%
\subsection{Manual Code}
\label{sec:manual}

In case one cannot be certain whether the definitions file |childdoc.def|
is installed on the target \TeX{} distribution
and one prefers not to ship it,
it is conceivable to paste a few relevant commands into the sources.

To that end, drop all statements |\input{childdoc.def}|
and perform the replacements as outlined below.
Instead of |\childdocmain{|\textit{main}|}| add the following code
to the top of the main file:
%
\begin{center}
\begin{tabular}{l}
|\||ifdefined\childdocname\endinput\||fi\newif\ifchilddoc|\\
|\edef\childdocname{\scantokens\expandafter{\jobname\noexpand}}|\\
|\def\childdocmain{|\textit{main}|}\||ifx\childdocmain\childdocname\||else|\\
|\childdoctrue\includeonly{\childdocname}\let\jobname\childdocmain\||fi|\\
\end{tabular}
\end{center}
%
Instead of |\childdocof{|\textit{main}|}| just include the main file
at the top of each child file:
%
\begin{center}
|\input{|\textit{main}|}|
\end{center}
%
A simple redirection |\childdocforward{|\textit{dest}|}| is achieved by:
%
\begin{center}
|\def\jobname{|\textit{dest}|}\input{\jobname}|
\end{center}
%
The redirection with prefix
|\childdocforwardprefix[|\textit{prefix}|]{|\textit{dest}|}|
is accomplished by:
%
\begin{center}
\begin{tabular}{l}
|{\edef\jobname{\scantokens\expandafter{\jobname\noexpand}}|\\
|\def\redirectjob |\textit{prefix}|#1~~~{\gdef\jobname{|\textit{dest}|#1}}|\\
|\expandafter\redirectjob\jobname~~~}\input{\jobname}|
\end{tabular}
\end{center}

In an alternative approach,
child documents can be compiled by a specific command line
without additional code or specific definitions:
%
\begin{center}
|... -jobname "|\textit{target}|" "|[\textit{flags}]%
|\includeonly{|\textit{dest}|}\input{|\textit{main}|}"|
\end{center}
%

%%%%%%%%%%%%%%%%%%%%%%%%%%%%%%%%%%%%%%%%%%%%%%%%%%%%%%%%%%%%%%%%%%%%%%%%%%%%%%%%
%%%%%%%%%%%%%%%%%%%%%%%%%%%%%%%%%%%%%%%%%%%%%%%%%%%%%%%%%%%%%%%%%%%%%%%%%%%%%%%%
\section{Information}

%%%%%%%%%%%%%%%%%%%%%%%%%%%%%%%%%%%%%%%%%%%%%%%%%%%%%%%%%%%%%%%%%%%%%%%%%%%%%%%%
\subsection{Copyright}

Copyright \copyright{} 2017--2018 Niklas Beisert

This work may be distributed and/or modified under the
conditions of the \LaTeX{} Project Public License, either version 1.3
of this license or (at your option) any later version.
The latest version of this license is in
  \url{http://www.latex-project.org/lppl.txt}
and version 1.3 or later is part of all distributions of \LaTeX{}
version 2005/12/01 or later.

This work has the LPPL maintenance status `maintained'.

The Current Maintainer of this work is Niklas Beisert.

This work consists of the files |README.txt|, |childdoc.ins| and |childdoc.dtx|
as well as the derived files |childdoc.def|, |cdocsamp.tex|
with |cdocsch1.tex|, |cdocsch2.tex|, |cdocspt3.tex|, |cdocspt4.tex|,
|cdocsdrf.tex|, |cdocsfn1.tex|, |cdocsfn2.tex|
as well as |childdoc.pdf|.

%%%%%%%%%%%%%%%%%%%%%%%%%%%%%%%%%%%%%%%%%%%%%%%%%%%%%%%%%%%%%%%%%%%%%%%%%%%%%%%%
\subsection{Files and Installation}

The package consists of the files:
%
\begin{center}
\begin{tabular}{ll}
    |README.txt|   & readme file \\
    |childdoc.ins| & installation file \\
    |childdoc.dtx| & source file \\
    |childdoc.def| & definition file \\
    |cdocsamp.tex| & sample main file \\
    |cdocsch1.tex| & sample include file \\
    |cdocsch2.tex| & sample include file \\
    |cdocspt3.tex| & sample part file \\
    |cdocspt4.tex| & sample part file \\
    |cdocsdrf.tex| & sample redirection file \\
    |cdocsfn1.tex| & sample redirection file \\
    |cdocsfn2.tex| & sample redirection file \\
    |childdoc.pdf| & manual
\end{tabular}
\end{center}
%
The distribution consists of the files
|README.txt|, |childdoc.ins| and |childdoc.dtx|.
%
\begin{itemize}
\item
Run (pdf)\LaTeX{} on |childdoc.dtx|
to compile the manual |childdoc.pdf| (this file).
\item
Run \LaTeX{} on |childdoc.ins| to create the definitions file |childdoc.def|
and the sample |cdocsamp.tex| with include files
|cdocsch1.tex|, |cdocsch2.tex|, |cdocspt3.tex|, |cdocspt4.tex|,
|cdocsdrf.tex|, |cdocsfn1.tex|, |cdocsfn2.tex|.
Then copy the file |childdoc.def| to an appropriate directory of your \LaTeX{}
distribution, e.g.\ \textit{texmf-root}|/tex/latex/childdoc|.
\end{itemize}

%%%%%%%%%%%%%%%%%%%%%%%%%%%%%%%%%%%%%%%%%%%%%%%%%%%%%%%%%%%%%%%%%%%%%%%%%%%%%%%%
\subsection{Related CTAN Packages}

There are several other packages which offer a similar functionality:
%
\begin{itemize}
\item
The packages
\href{http://ctan.org/pkg/docmute}{\textsf{docmute}},
\href{http://ctan.org/pkg/includex}{\textsf{includex}} and
\href{http://ctan.org/pkg/standalone}{\textsf{standalone}}
provide commands to include only the document body of
a child file thus allowing both files to be compiled individually.
\item
The packages \href{http://ctan.org/pkg/subdocs}{\textsf{subdocs}}
and \href{http://ctan.org/pkg/subfiles}{\textsf{subfiles}}
provide structures in which the main and child documents can be
encapsulated and allowing them to be compiled individually.
The inclusion mechanism is different from the conventional |\include|.
\item
The package \href{http://ctan.org/pkg/combine}{\textsf{combine}}
is an elaborate solution to combine several documents into one.
\end{itemize}
%
See also the CTAN topic \href{http://ctan.org/topic/subdocs}{\textsf{subdocs}}
for further related packages.
The present package differs from the above solutions in that
a document structure constructed with the conventional |\include| mechanism
just needs two extra commands at the top of every file
such that all constituent files can be compiled individually.

%%%%%%%%%%%%%%%%%%%%%%%%%%%%%%%%%%%%%%%%%%%%%%%%%%%%%%%%%%%%%%%%%%%%%%%%%%%%%%%%
%\subsection{Feature Suggestions}
%
%The following is a list of features which may be useful for future
%versions of this package:
%%
%\begin{itemize}
%\item
%\ldots
%\end{itemize}

%%%%%%%%%%%%%%%%%%%%%%%%%%%%%%%%%%%%%%%%%%%%%%%%%%%%%%%%%%%%%%%%%%%%%%%%%%%%%%%%
\subsection{Revision History}

%%%%%%%%%%%%%%%%%%%%%%%%%%%%%%%%%%%%%%%%
\paragraph{v2.0:} 2018/12/30

\begin{itemize}
\item
immediate forward processing
\item
added |\childdocby| mechanism
\item
manual restructured
\end{itemize}

%%%%%%%%%%%%%%%%%%%%%%%%%%%%%%%%%%%%%%%%
\paragraph{v1.6:} 2018/01/17

\begin{itemize}
\item
application for development of include files
\item
corrections to manual
\end{itemize}

%%%%%%%%%%%%%%%%%%%%%%%%%%%%%%%%%%%%%%%%
\paragraph{v1.5:} 2017/05/21

\begin{itemize}
\item
more complete structuring introduced
\item
|\childdocof| introduced
\item
|\childdoc| renamed to |\childdocmain|
\item
|\childredirect| renamed to |\childdocforward| and |\childdocforwardprefix|
and functionality expanded
\end{itemize}

%%%%%%%%%%%%%%%%%%%%%%%%%%%%%%%%%%%%%%%%
\paragraph{v1.0:} 2017/04/27

\begin{itemize}
\item
manual and install package
\item
first version published on CTAN
\end{itemize}

%%%%%%%%%%%%%%%%%%%%%%%%%%%%%%%%%%%%%%%%
\paragraph{v0.6:} 2017/04/26

\begin{itemize}
\item
redirection mechanism added
\end{itemize}

%%%%%%%%%%%%%%%%%%%%%%%%%%%%%%%%%%%%%%%%
\paragraph{v0.5:} 2017/04/26

\begin{itemize}
\item
functionality in definition file
\end{itemize}


%%%%%%%%%%%%%%%%%%%%%%%%%%%%%%%%%%%%%%%%%%%%%%%%%%%%%%%%%%%%%%%%%%%%%%%%%%%%%%%%
%%%%%%%%%%%%%%%%%%%%%%%%%%%%%%%%%%%%%%%%%%%%%%%%%%%%%%%%%%%%%%%%%%%%%%%%%%%%%%%%
%%%%%%%%%%%%%%%%%%%%%%%%%%%%%%%%%%%%%%%%%%%%%%%%%%%%%%%%%%%%%%%%%%%%%%%%%%%%%%%%
\appendix

\settowidth\MacroIndent{\rmfamily\scriptsize 000\ }

 \DocInput{childdoc.dtx}

\end{document}
%</driver>
% \fi
%
% %%%%%%%%%%%%%%%%%%%%%%%%%%%%%%%%%%%%%%%%%%%%%%%%%%%%%%%%%%%%%%%%%%%%%%%%%%%%%%
% %%%%%%%%%%%%%%%%%%%%%%%%%%%%%%%%%%%%%%%%%%%%%%%%%%%%%%%%%%%%%%%%%%%%%%%%%%%%%%
% \section{Sample}
%\iffalse
%<*samplemain>
%\fi
%
% The following presents a sample document
% with two chapters, two parts, a title page,
% a compile flag as well as three forwarding files to set the flag.
% It consists of eight |.tex| files:
% \begin{center}
% \begin{tabular}{ll}
% |cdocsamp.tex|&main file\\
% |cdocsch1.tex|&include file for chapter 1\\
% |cdocsch2.tex|&include file for chapter 2\\
% |cdocspt3.tex|&include file for part 3\\
% |cdocspt4.tex|&include file for part 4\\
% |cdocsdrf.tex|&forwarding file for main file in draft mode\\
% |cdocsfi1.tex|&forwarding file for final version of chapter 1\\
% |cdocsfi2.tex|&forwarding file for final version of chapter 2\\
% \end{tabular}
% \end{center}
% Each of the eight files can be compiled directly by the \LaTeX{} compiler.
%
% %%%%%%%%%%%%%%%%%%%%%%%%%%%%%%%%%%%%%%
% \paragraph{Main File.}
%
% The main file is called |cdocsamp.tex|.
%
% Load the \textsf{childdoc} definitions and
% declare the filename for the main document:
%    \begin{macrocode}
\input{childdoc.def}
\childdocmain{}
%    \end{macrocode}

% Optional override for |\version| flag:
%    \begin{macrocode}
%%\ifchilddoc\else\providecommand{\version}{draft}\fi
%    \end{macrocode}

% Define the default values for the |\version| flag
% (|final| for the main file and |draft| for childs):
%    \begin{macrocode}
\ifchilddoc
\providecommand{\version}{draft}
\else
\providecommand{\version}{final}
\fi
%    \end{macrocode}

% Load the standard document class:
%    \begin{macrocode}
\documentclass[12pt]{article}
%    \end{macrocode}

% Start the document body:
%    \begin{macrocode}
\begin{document}
%    \end{macrocode}

% Declare a title page.
% Print title, part of document being processed and version flag:
%    \begin{macrocode}
\addtocounter{page}{-1}
\begin{center}
{\LARGE\bfseries{}childdoc example\par}
\vspace{1cm}
\ifchilddoc
\ifchilddocmanual part\else chapter\fi:
`\childdocname' of `\childdocjob'\par
\else
main document: `\childdocjob'\par
\fi
version: \version\par
\end{center}
\newpage
%    \end{macrocode}

% Manually include selected file,
% otherwise process as usual:
%    \begin{macrocode}
\ifchilddocmanual
\section*{part `\childdocname'}
\input{\childdocname}
\else
%    \end{macrocode}

% Include the two chapters:
%    \begin{macrocode}
\include{cdocsch1}
\include{cdocsch2}
%    \end{macrocode}

% Include the two parts unless only chapters should be displayed:
%    \begin{macrocode}
\ifchilddoc\else
\section{part three}
\input{cdocspt3}
\section{part four}
\input{cdocspt4}
\fi
%    \end{macrocode}

% Process as usual until here:
%    \begin{macrocode}
\fi
%    \end{macrocode}

% End of document body:
%    \begin{macrocode}
\end{document}
%    \end{macrocode}
%\iffalse
%</samplemain>
%\fi
%
% %%%%%%%%%%%%%%%%%%%%%%%%%%%%%%%%%%%%%%
% \paragraph{Chapter Include Files.}
%
% The include files are called |cdocsch1.tex| and |cdocsch2.tex|.
%
%\iffalse
%<*samplechap1|samplechap2>
%\fi

% Optional override for |\version| flag:
%    \begin{macrocode}
%%\providecommand{\version}{final}
%    \end{macrocode}

% Include the main document:
%    \begin{macrocode}
\input{childdoc.def}
\childdocof{cdocsamp}
%    \end{macrocode}

%\iffalse
%</samplechap1|samplechap2>
%\fi
%
%\iffalse
%<*samplechap1>
%\fi
% Some text for chapter 1:
%    \begin{macrocode}
\section{one}
some text in chapter one
%    \end{macrocode}

%\iffalse
%</samplechap1>
%\fi
% Some text for chapter 2:
%\iffalse
%<*samplechap2>
%\fi
%    \begin{macrocode}
\section{two}
more text in chapter two
%    \end{macrocode}

%\iffalse
%</samplechap2>
%\fi
%
% %%%%%%%%%%%%%%%%%%%%%%%%%%%%%%%%%%%%%%
% \paragraph{Part Include Files.}
%
% The include files are called |cdocspt3.tex| and |cdocspt4.tex|.
%
%\iffalse
%<*samplepart3|samplepart4>
%\fi

% Optional override for |\version| flag:
%    \begin{macrocode}
%%\providecommand{\version}{final}
%    \end{macrocode}

% Include the main document:
%    \begin{macrocode}
\input{childdoc.def}
\childdocby{cdocsamp}
%    \end{macrocode}

%\iffalse
%</samplepart3|samplepart4>
%\fi
%
%\iffalse
%<*samplepart3>
%\fi
% Some text for part 3:
%    \begin{macrocode}
some text in part three
%    \end{macrocode}

%\iffalse
%</samplepart3>
%\fi
% Some text for part 4:
%\iffalse
%<*samplepart4>
%\fi
%    \begin{macrocode}
more text in part four
%    \end{macrocode}

%\iffalse
%</samplepart4>
%\fi
%
% %%%%%%%%%%%%%%%%%%%%%%%%%%%%%%%%%%%%%%
% \paragraph{Forwarding for a Complete Draft.}
%
% The following forwarding file |cdocsdrf.tex|
% compiles the main document in draft mode:
%\iffalse
%<*sampledraft>
%\fi
%    \begin{macrocode}
\def\version{draft}
\input{childdoc.def}
\childdocforward{cdocsamp}
%    \end{macrocode}

%\iffalse
%</sampledraft>
%\fi
%
% %%%%%%%%%%%%%%%%%%%%%%%%%%%%%%%%%%%%%%
% \paragraph{Forwarding for Final Version of the Chapters.}
%
% The following forwarding files |cdocsfn1.tex| and |cdocsfn2.tex|
% (with identical content)
% compile the final versions of the child documents
% |cdocsch1.tex| and |cdocsch2.tex|, respectively:
%\iffalse
%<*samplefinal>
%\fi
%    \begin{macrocode}
\def\version{final}
\input{childdoc.def}
\childdocforwardprefix[cdocsamp]{cdocsfn}{cdocsch}
%    \end{macrocode}

%\iffalse
%</samplefinal>
%\fi
%
% %%%%%%%%%%%%%%%%%%%%%%%%%%%%%%%%%%%%%%
% \paragraph{Command Line Processing.}
%
% The following three command lines generate the output files
% |cdocscld|, |cdocscl1| and |cdocscl2|
% which should be identical to
% |cdocsdrf|, |cdocsch1| and |cdocsfn2|, respectively:
% \begin{center}
% \begin{tabular}{l}
% |latex -jobname cdocscld \|\\
% |  "\def\version{draft}\input{childdoc.def}\childdocforward{cdocsamp}"|\\
% |latex -jobname cdocscl1 \|\\
% |  "\input{childdoc.def}\childdocforward[cdocsamp]{cdocsch1}"|\\
% |latex -jobname cdocscl2 \|\\
% |  "\def\version{final}\input{childdoc.def}\childdocforward{cdocsch2}"|
% \end{tabular}
% \end{center}
% Note that the trailing backslash on each first line
% merely continues the input to the second line
% (for convenient cut ant paste).
% Furthermore, the command |latex| can be replaced by any
% of its alternative versions such as |pdflatex|.
%
% %%%%%%%%%%%%%%%%%%%%%%%%%%%%%%%%%%%%%%%%%%%%%%%%%%%%%%%%%%%%%%%%%%%%%%%%%%%%%%
% %%%%%%%%%%%%%%%%%%%%%%%%%%%%%%%%%%%%%%%%%%%%%%%%%%%%%%%%%%%%%%%%%%%%%%%%%%%%%%
% \section{Implementation}
%\iffalse
%<*package>
%\fi
%
% This section describes the definitions file |childdoc.def|.

% The definitions cannot be loaded using |\usepackage| or |\RequirePackage|
% which has a mechanism to prevent loading a style file more than once.
% When loading the definitions by means of |\input|
% multiple instances have to be prevented manually:
%\iffalse
%This code needs to be before the `\ProvidesFile' directive
%which is defined at the beginning of this file.
%Therefore it is also placed there and commented out here.
%</package>
%<*discard>
%\fi
%    \begin{macrocode}
\ifdefined\childdocmain\endinput\fi
%    \end{macrocode}
%\iffalse
%</discard>
%<*package>
%\fi
%
% \macro{\ifchilddoc}
% \macro{\ifchilddocmanual}
% The conditional |\ifchilddoc| tells whether a
% child (true) or main (false) document is being compiled.
% The conditional |\ifchilddocmanual| tells whether
% the |\includeonly| mechanism is used (false) or
% the selection of child files must be performed manually (true).
% The definitions initialise to false:
%    \begin{macrocode}
\newif\ifchilddoc
\newif\ifchilddocmanual
%    \end{macrocode}

% \macro{\childdocname}
% \macro{\childdocjob}
% The macro |\childdocname| stores the name of the main document
% to be compiled. The macro |\childdocjob| stores the name of
% the document on which the \LaTeX{} compiler was originally invoked.
% The content of |\jobname| cannot be compared
% to filenames specified in the source due to different catcodes.
% The following code rescans |\jobname|, stores the result
% in |\childdocname| and saves a copy in |\childdocjob|:
%    \begin{macrocode}
\edef\childdocname{\scantokens\expandafter{\jobname\noexpand}}
\let\childdocjob\childdocname
%    \end{macrocode}

% \macro{\childdocdisable}
% The macro |\childdocdisable| prevents the main file
% from being processed more than once.
% At this stage, the main document command |\childdocmain|
% is assumed to be called once again where it should do nothing.
% Any subsequent call to it should prevent
% a secondary processing of the main document
% It overwrites the forwarding commands
% |\childdocof| and |\childdocforward|
% with empty macros to prevent further inclusions of the main document:
%    \begin{macrocode}
\newcommand{\childdocdisable}
{
  \renewcommand{\childdocmain}[1]{\renewcommand{\childdocmain}[1]{\endinput}}
  \renewcommand{\childdocof}[1]{}
  \renewcommand{\childdocby}[2][]{}
  \renewcommand{\childdocforward}[2][]{}
  \renewcommand{\childdocdisable}{}
}
%    \end{macrocode}

% \macro{\childdocmain}
% The macro |\childdocmain| is to be called at the top of the main file
% with nothing or the main filename (without extension) as argument.
% First, it breaks loops.
% If the argument is not empty and does not match |\childdocname|
% (which is set by the first inclusion of |childdoc.def|),
% |\ifchilddoc| is set to true, |\includeonly| is applied to the child file
% and |\jobname| is set to the main file
% (for proper handling of |.aux| files):
%    \begin{macrocode}
\newcommand{\childdocmain}[1]
{
  \childdocdisable\childdocmain{}
  \if?#1?\else
    \begingroup
      \def\childdoctmp{#1}
      \ifx\childdoctmp\childdocname
        \def\childdoctmp{}
      \else
        \def\childdoctmp
        {
          \childdoctrue
          \includeonly{\childdocname}
          \def\childdocjob{#1}
          \def\jobname{#1}
        }
      \fi
      \expandafter
    \endgroup
    \childdoctmp
  \fi
}
%    \end{macrocode}

% \macro{\childdocof}
% The command |\childdocof| redirects
% compilation to the main file |#1|.
%    \begin{macrocode}
\newcommand{\childdocof}[1]
{
  \childdocdisable
  \childdoctrue
  \includeonly{\childdocname}
  \def\jobname{#1}
  \def\childdocjob{#1}
  \input{#1}
}
%    \end{macrocode}

% \macro{\childdocby}
% The command |\childdocby| ....
%    \begin{macrocode}
\newcommand{\childdocby}[2][]
{
  \childdocdisable
  \childdoctrue
  \childdocmanualtrue
  \if?#1?\else
    \def\jobname{#2}
  \fi
  \def\childdocjob{#2}
  \input{#2}
  \endinput
}
%    \end{macrocode}

% \macro{\childdocforward}
% The command |\childdocforward| redirects
% compilation to the main file or
% (if the optional argument is given) a child file.
% Parameters are set as if the main file
% or a child file starting with |\childdocof| was compiled.
% Then compilation is handed over to the main file:
%    \begin{macrocode}
\newcommand{\childdocforward}[2][]
{
  \begingroup
    \if?#1?
      \def\childdoctmp
      {
        \def\childdocname{#2}
        \def\childdocjob{#2}
        \def\jobname{#2}
        \input{#2}
        \endinput
      }
    \else
      \def\childdoctmp
      {
        \childdocdisable
        \def\childdocname{#2}
        \childdoctrue
        \includeonly{#2}
        \def\childdocjob{#1}
        \def\jobname{#1}
        \input{#1}
        \endinput
      }
    \fi
    \expandafter
  \endgroup
  \childdoctmp
}
%    \end{macrocode}

% \macro{\childdocforwardprefix}
% The command |\childdocforwardprefix| redirects
% compilation to the main or a child file by means of a pattern.
% The prefix |#1| in the current filename is replaced by |#2|
% and the suffix of the current filename is kept
% (it is assumed that the filename does not contain the substring `|~~~|'
% which is used as a delimiter).
% Compilation is handed over to the new file by |\childdocforward|:
%    \begin{macrocode}
\newcommand{\childdocforwardprefix}[3][]
{
  \begingroup
    \def\childdocextract #2##1~~~{\def\childdoctmp{\childdocforward[#1]{#3##1}}}
    \expandafter\childdocextract\childdocname~~~
    \expandafter
  \endgroup
  \childdoctmp
}
%    \end{macrocode}

% \macro{\childdoc}
% The deprecated macro |\childdoc| is a legacy version of |\childdocmain|:
%    \begin{macrocode}
\newcommand{\childdoc}{\childdocmain}
%    \end{macrocode}

% \macro{\childdocredirect}
% The deprecated macro |\childdocredirect| is a legacy version
% of |\childdocforward| and |\childdocforwardprefix|:
%    \begin{macrocode}
\newcommand{\childdocredirect}[2][]
{
  \begingroup
    \if?#1?
      \def\childdoctmp{\childdocforward{#2}}
    \else
      \def\childdoctmp{\childdocforwardprefix{#1}{#2}}
    \fi
    \expandafter
  \endgroup
  \childdoctmp
}
%    \end{macrocode}

%\iffalse
%</package>
%\fi
%
\endinput

\childdocmain{}
%    \end{macrocode}

% Optional override for |\version| flag:
%    \begin{macrocode}
%%\ifchilddoc\else\providecommand{\version}{draft}\fi
%    \end{macrocode}

% Define the default values for the |\version| flag
% (|final| for the main file and |draft| for childs):
%    \begin{macrocode}
\ifchilddoc
\providecommand{\version}{draft}
\else
\providecommand{\version}{final}
\fi
%    \end{macrocode}

% Load the standard document class:
%    \begin{macrocode}
\documentclass[12pt]{article}
%    \end{macrocode}

% Start the document body:
%    \begin{macrocode}
\begin{document}
%    \end{macrocode}

% Declare a title page.
% Print title, part of document being processed and version flag:
%    \begin{macrocode}
\addtocounter{page}{-1}
\begin{center}
{\LARGE\bfseries{}childdoc example\par}
\vspace{1cm}
\ifchilddoc
\ifchilddocmanual part\else chapter\fi:
`\childdocname' of `\childdocjob'\par
\else
main document: `\childdocjob'\par
\fi
version: \version\par
\end{center}
\newpage
%    \end{macrocode}

% Manually include selected file,
% otherwise process as usual:
%    \begin{macrocode}
\ifchilddocmanual
\section*{part `\childdocname'}
\input{\childdocname}
\else
%    \end{macrocode}

% Include the two chapters:
%    \begin{macrocode}
\include{cdocsch1}
\include{cdocsch2}
%    \end{macrocode}

% Include the two parts unless only chapters should be displayed:
%    \begin{macrocode}
\ifchilddoc\else
\section{part three}
\input{cdocspt3}
\section{part four}
\input{cdocspt4}
\fi
%    \end{macrocode}

% Process as usual until here:
%    \begin{macrocode}
\fi
%    \end{macrocode}

% End of document body:
%    \begin{macrocode}
\end{document}
%    \end{macrocode}
%\iffalse
%</samplemain>
%\fi
%
% %%%%%%%%%%%%%%%%%%%%%%%%%%%%%%%%%%%%%%
% \paragraph{Chapter Include Files.}
%
% The include files are called |cdocsch1.tex| and |cdocsch2.tex|.
%
%\iffalse
%<*samplechap1|samplechap2>
%\fi

% Optional override for |\version| flag:
%    \begin{macrocode}
%%\providecommand{\version}{final}
%    \end{macrocode}

% Include the main document:
%    \begin{macrocode}
% \iffalse
%
% childdoc.dtx Copyright (C) 2017-2018 Niklas Beisert
%
% This work may be distributed and/or modified under the
% conditions of the LaTeX Project Public License, either version 1.3
% of this license or (at your option) any later version.
% The latest version of this license is in
%   http://www.latex-project.org/lppl.txt
% and version 1.3 or later is part of all distributions of LaTeX
% version 2005/12/01 or later.
%
% This work has the LPPL maintenance status `maintained'.
%
% The Current Maintainer of this work is Niklas Beisert.
%
% This work consists of the files childdoc.dtx and childdoc.ins
% and the derived files childdoc.def and cdocsamp.tex with
% cdocsch1.tex, cdocsch2.tex, cdocsdrf.tex, cdocsfn1.tex, cdocsfn2.tex.
%
%<package>\ifdefined\childdocmain\endinput\fi
%<package>\ProvidesFile{childdoc.def}[2018/12/30 v2.0 child document driver]
%<samplemain>\ProvidesFile{cdocsamp.tex}[2018/12/30 v2.0 sample for childdoc]
%<*driver>
%\ProvidesFile{childdoc.drv}[2018/12/30 v2.0 childdoc reference manual file]
\PassOptionsToClass{10pt,a4paper}{article}
\documentclass{ltxdoc}

\usepackage[margin=35mm]{geometry}
\usepackage{hyperref}
\usepackage{hyperxmp}
\usepackage[usenames]{color}

\hypersetup{colorlinks=true}
\hypersetup{pdfstartview=FitH}
\hypersetup{pdfpagemode=UseNone}
\hypersetup{pdfsource={}}
\hypersetup{pdflang={en-UK}}
\hypersetup{pdfcopyright={Copyright 2017-2018 Niklas Beisert.
  This work may be distributed and/or modified under the
  conditions of the LaTeX Project Public License, either version 1.3
  of this license or (at your option) any later version.}}
\hypersetup{pdflicenseurl={http://www.latex-project.org/lppl.txt}}
\hypersetup{pdfcontactaddress={ETH Zurich, ITP, HIT K,
  Wolfgang-Pauli-Strasse 27}}
\hypersetup{pdfcontactpostcode={8093}}
\hypersetup{pdfcontactcity={Zurich}}
\hypersetup{pdfcontactcountry={Switzerland}}
\hypersetup{pdfcontactemail={nbeisert@itp.phys.ethz.ch}}
\hypersetup{pdfcontacturl={http://people.phys.ethz.ch/\xmptilde nbeisert/}}

\newcommand{\secref}[1]{\hyperref[#1]{section \ref*{#1}}}

\parskip1ex
\parindent0pt
\let\olditemize\itemize
\def\itemize{\olditemize\parskip0pt}

\begin{document}

\title{The \textsf{childdoc} Package}
\hypersetup{pdftitle={The childdoc Package}}
\author{Niklas Beisert\\[2ex]
  Institut f\"ur Theoretische Physik\\
  Eidgen\"ossische Technische Hochschule Z\"urich\\
  Wolfgang-Pauli-Strasse 27, 8093 Z\"urich, Switzerland\\[1ex]
  \href{mailto:nbeisert@itp.phys.ethz.ch}
  {\texttt{nbeisert@itp.phys.ethz.ch}}}
\hypersetup{pdfauthor={Niklas Beisert}}
\hypersetup{pdfsubject={Manual for the LaTeX2e Package childdoc}}
\date{30 December 2018, \textsf{v2.0}}
\maketitle

\begin{abstract}\noindent
\textsf{childdoc} is a \LaTeXe{} package
that enables the direct compilation
of document sections included by |\include|
to individual files.
\end{abstract}

\begingroup
\parskip0ex
\tableofcontents
\endgroup

%%%%%%%%%%%%%%%%%%%%%%%%%%%%%%%%%%%%%%%%%%%%%%%%%%%%%%%%%%%%%%%%%%%%%%%%%%%%%%%%
%%%%%%%%%%%%%%%%%%%%%%%%%%%%%%%%%%%%%%%%%%%%%%%%%%%%%%%%%%%%%%%%%%%%%%%%%%%%%%%%
\section{Introduction}

\LaTeX{} provides a mechanism to structure a large document (such as a book)
into a main file and several child files (containing the chapters)
using the |\include| command.
This mechanism is beneficial for documents
which span hundreds of pages in order to
make the source file(s) more manageable.
Moreover, compilation can be restricted to
selected child files by means of the |\includeonly| command.
The latter feature can be used to reduce the compilation time while editing
(this was significantly more useful in the earlier days of \LaTeX{})
or to generate a smaller document which is easier to navigate.
Another application of |\includeonly| is to generate
documents consisting of selected parts of the complete document.

However, there are a few drawbacks of the plain |\include| mechanism:
\begin{itemize}
\item
The child files cannot be compiled on their own,
they can only be compiled via the main file.
A naive editing environment
(such as a text editor with an option
to have the current file processed by \LaTeX)
may require one to switch to the main file before compiling;
attempting to compile the child file produces errors.
\item
The main file must be modified (each time)
to adjust the |\includeonly| command
to the present needs. This easily leaves the main file in a messy state.
\item
The generated document will always carry the filename
of the main document. This is inconvenient if
several child files are to be compiled and
to be kept for distribution.
\end{itemize}

The present package provides a simple interface
to make child files individually compilable by \LaTeX{}.
Compiling a child file then has the same effect as compiling
the main file with an |\includeonly| command
to select the appropriate child.
Moreover the generated document will carry the name of the child
rather than the main file.
This resolves all three above issues.

This feature is meant to make the editing of books,
thesis documents and lecture notes somewhat more convenient.
However, the package can also be used efficiently for
composing a series of documents (such as exercise sheets)
which are typically distributed individually.
It then assists the author in generating the individual documents
(potentially in different versions)
as well as a document containing the collected series.
Another application is in developing style files
or other kinds of included material
where compilation of the style file could redirect
to a sample or test file.

%%%%%%%%%%%%%%%%%%%%%%%%%%%%%%%%%%%%%%%%%%%%%%%%%%%%%%%%%%%%%%%%%%%%%%%%%%%%%%%%
%%%%%%%%%%%%%%%%%%%%%%%%%%%%%%%%%%%%%%%%%%%%%%%%%%%%%%%%%%%%%%%%%%%%%%%%%%%%%%%%
\section{Usage}

First of all, the package \textsf{childdoc} is \emph{not} a standard
\LaTeXe{} |.sty| style file! Therefore it needs to be invoked in
a non-standard way.

%%%%%%%%%%%%%%%%%%%%%%%%%%%%%%%%%%%%%%%%%%%%%%%%%%%%%%%%%%%%%%%%%%%%%%%%%%%%%%%%
\subsection{Included Files}
\label{sec:include}

%%%%%%%%%%%%%%%%%%%%%%%%%%%%%%%%%%%%%%%%
\DescribeMacro{\childdocmain}
To use the package, add the commands
\begin{center}
\begin{tabular}{l}
|\input{childdoc.def}|\\
|\childdocmain{}|\\
\end{tabular}
\end{center}
at the very top of the main \LaTeX{} file,
in particular \emph{before} the |\documentclass| statement!
The argument of |\childdocmain| should be left empty
(but it must be present).

%%%%%%%%%%%%%%%%%%%%%%%%%%%%%%%%%%%%%%%%
\DescribeMacro{\childdocof}
Furthermore, add the commands
\begin{center}
\begin{tabular}{l}
|\input{childdoc.def}|\\
|\childdocof{|\textit{main}|}|\\
\end{tabular}
\end{center}
at the top of every child file \textit{child}
which is included by |\include{|\textit{child}|}|
from within the main file
(or at least for those files to be compiled individually).
The argument \textit{main} must be the filename of the main file.

There are a couple of
considerations in setting up the main and child documents:

%%%%%%%%%%%%%%%%%%%%%%%%%%%%%%%%%%%%%%%%
\paragraph{Restrictions.}

Please note the following restrictions:
\begin{itemize}
\item
|\childdocmain| must be called with one argument \textit{main}
to ensure compatibility with earlier version of the package.
It must either be empty (|\childdocmain{}|)
or precisely match the filename of the main file in which it is specified.
See \secref{sec:detection} for further information.
\item
The filename \textit{main} must be specified without the |.tex| extension.
\item
The filename \textit{main} is case sensitive
(even in case-insensitive file systems)
due to internal string comparison.
\item
The argument \textit{main} should be fully expanded, it cannot be a macro.
\item
Subdirectories and special characters should be avoided in filenames.
\item
The command |\childdocmain{|\textit{main}|}| must be followed by a whitespace.
It should not be followed immediately by another command
or by a comment mark `|%|'.
This is because the \TeX{} parser reads the token immediately following
the argument of |\childdocmain| and puts it
at the beginning of every child section;
however, a white\-space is ignored.
\end{itemize}

%%%%%%%%%%%%%%%%%%%%%%%%%%%%%%%%%%%%%%%%
\paragraph{Content of Main File.}

It is advisable to place all content in the child files included by |\include|.
Any output contained in the main file will appear in all child documents
unless suppressed manually;
it cannot be suppressed automatically by the |\includeonly| directive
and thus should normally be avoided.
A method to include some content in the main file
by means of conditional processing is described in \secref{sec:conditional}.

%%%%%%%%%%%%%%%%%%%%%%%%%%%%%%%%%%%%%%%%
\paragraph{Page Numbering.}

When only a part of the document is compiled,
the appropriate numbering of pages
(as well as other status parameters)
is determined from the |.aux| files.
The latter contain information from previous passes.
However this information needs to propagate through
all intermediate child documents.
Therefore the page numbering in child documents may well
be inconsistent until the complete document is compiled at least once.

A useful (if unconventional) way to always ensure a consistent
page numbering is to restart the numbering in each child document
and denote the pages by `\textit{child}|.|\textit{page}'
where \textit{child} represents the chapter/section number of the child file.
This can be achieved by the command
|\numberwithin{page}{|\textit{child}|}|
of the \textsf{amsmath} package
where \textit{child} can be |chapter| or |section|
depending on the chosen structuring.
Alternatively, one can modify the macro |\thepage| appropriately
and reset the counter |page| at the start of each child file.

%%%%%%%%%%%%%%%%%%%%%%%%%%%%%%%%%%%%%%%%%%%%%%%%%%%%%%%%%%%%%%%%%%%%%%%%%%%%%%%%
\subsection{Conditional Processing}
\label{sec:conditional}

The package provides a mechanism to compile different versions
of a document. To customise the versions further some conditional processing
can come in handy to distinguish which version is being compiled.
The package provides two macros to describe the compilation context:

%%%%%%%%%%%%%%%%%%%%%%%%%%%%%%%%%%%%%%%%
\DescribeMacro{\ifchilddoc}
The conditional |\ifchilddoc| distinguishes between the compilation of
child documents and the main document:
%
\begin{center}
|\ifchilddoc |\textit{child-code}| |[|\||else |\textit{main-code}]| \||fi|
\end{center}

%%%%%%%%%%%%%%%%%%%%%%%%%%%%%%%%%%%%%%%%
\DescribeMacro{\childdocname}
\DescribeMacro{\childdocjob}
The macro |\childdocname| contains the filename (without extension)
of the main or child file being processed.
Note that |\childdocjob| will always contain the name of the main file.

%%%%%%%%%%%%%%%%%%%%%%%%%%%%%%%%%%%%%%%%
\paragraph{Title Page.}

Conditional processing can be used to include a title or banner page
in the main document when proper precautions are taken.
Importantly, the code in the main file should ensure that the page counter
(as well as other status parameters which are stored in the |.aux| files)
takes the same value after the conditional processing.
Otherwise the page numbers may take divergent values
depending on which part is compiled.

For example, a title page could be declared by:
%
\begin{center}
\begin{tabular}{l}
|\ifchilddoc\||else|\\
|\addtocounter{page}{-1}|\\
\textit{code for title page}\\
|\newpage|\\
|\||fi|
\end{tabular}
\end{center}
%
A banner page for the child documents can be generated by:
%
\begin{center}
\begin{tabular}{l}
|\ifchilddoc|\\
|\addtocounter{page}{-1}|\\
\textit{code for banner page}\\
|\newpage|\\
|\||fi|
\end{tabular}
\end{center}
%
Here one could write a message such as:
\begin{center}
|This is the part \childdocname{} of \childdocjob{}.|
\end{center}

%%%%%%%%%%%%%%%%%%%%%%%%%%%%%%%%%%%%%%%%%%%%%%%%%%%%%%%%%%%%%%%%%%%%%%%%%%%%%%%%
\subsection{Flags}
\label{sec:flags}

The package makes it easy to generate different versions
of the main or child documents.
To this end compilation flags can be defined
and assigned different default values.
They will be particularly useful in conjunction
with the forwarding mechanism described in \secref{sec:forward}.

For example, it may be useful to have a flag |\version|
which can be set to |draft| or |final|.
The document source will contain some conditional code
depending on the value of |\version|.
Suppose further, the flag should default to |final| for the main file
and to |draft| for child files
which is a natural assignment for editing the document.
This is achieved by placing the following code
in the preamble of the main document
(below the |\childdocmain| directive):
%
\begin{center}
\begin{tabular}{l}
|\ifchilddoc|\\
|\providecommand{\version}{draft}|\\
|\||else|\\
|\providecommand{\version}{final}|\\
|\||fi|
\end{tabular}
\end{center}
%
The definition by |\providecommand| makes sure
that previous definitions are not overwritten.
Further statements |\providecommand{\version}{...}|
can thus be added before the above code to override it.

For the main file, one might add a line
(between |\childdocmain| and the above block)
%
\begin{center}
|%\ifchilddoc\||else\providecommand{\version}{draft}\||fi|
\end{center}
%
which can be uncommented to produce a draft version.
Likewise one can add a line to the very top of a child file
(above the |\childdocof{|\textit{main}|}| directive)
%
\begin{center}
|%\providecommand{\version}{final}|
\end{center}
%
which can be uncommented to produce the final version of this child document.

%%%%%%%%%%%%%%%%%%%%%%%%%%%%%%%%%%%%%%%%%%%%%%%%%%%%%%%%%%%%%%%%%%%%%%%%%%%%%%%%
\subsection{Forwarding}
\label{sec:forward}

Different versions of the main or child documents
using compilation flags as described in \secref{sec:flags}
can be (permanently) stored in different files
for convenient compilation, viewing and distribution.
To this end, the package defines a command
to pass on compilation to a different file:

%%%%%%%%%%%%%%%%%%%%%%%%%%%%%%%%%%%%%%%%
\DescribeMacro{\childdocforward}
The command |\childdocforward| redirects processing to
another source file:
%
\begin{center}
\begin{tabular}{l}
|\input{childdoc.def}|\\
|\childdocforward[|\textit{main}|]{|\textit{dest}|}|\\
\end{tabular}
\end{center}
%
The argument \textit{dest} is the destination file
(without extension).
It should be the main file or one of the child files.
Note that further \textsf{childdoc} directives
such as |\childdocof| and |\childdocforward|
in the indicated file will be processed in this form.
The optional argument \textit{main}
passes on directly to the main file \textit{main}
while pretending to compile the child \textit{dest}.
This form behaves as if \textit{dest}
issues |\childdocof{|\textit{main}|}| right away,
and no further \textsf{childdoc} directives will be processed.

%%%%%%%%%%%%%%%%%%%%%%%%%%%%%%%%%%%%%%%%
\DescribeMacro{\...prefix}
In the alternative form |\childdocforwardprefix|,
%
\begin{center}
\begin{tabular}{l}
|\input{childdoc.def}|\\
|\childdocforwardprefix[|\textit{main}|]{|\textit{prefix}|}{|\textit{dest}|}|
\end{tabular}
\end{center}
%
the destination file is determined by a pattern
depending on the current file:
To make this work, the current file must be called
`{\textit{prefix}\hspace{0.2em}\textit{suffix}}'
with \textit{prefix} matching precisely the argument.
Processing is then passed on to the file
`{\textit{dest}\hspace{0.2em}\textit{suffix}}'.
Surely, the same effect is achieved by
directly specifying the
argument `{\textit{dest}\hspace{0.2em}\textit{suffix}}'
in the first form.
However, that requires to set up a different file
for each child. With the alternative form of the command
all these files can have exactly the same content
which simplifies setting them up and maintaining them.

For example, the following file |draft.tex|
with a compilation flag |\version| as described in \secref{sec:flags}
compiles the main document as a draft:
%
\begin{center}
\begin{tabular}{l}
|\def\version{draft}|\\
|\input{childdoc.def}|\\
|\childdocforward{|\textit{main}|}|
\end{tabular}
\end{center}
%
Likewise, the following files |final|\textit{nn}|.tex|
compile the final version of the child document
|child|\textit{nn}|.tex|:
%
\begin{center}
\begin{tabular}{l}
|\def\version{final}|\\
|\input{childdoc.def}|\\
|\childdocforwardprefix{final}{child}|
\end{tabular}
\end{center}
%

Note that when several versions of a main file and/or of each child file
are to be generated, it may be convenient to set up a |Makefile| or
shell script to automatise the process.

%%%%%%%%%%%%%%%%%%%%%%%%%%%%%%%%%%%%%%%%%%%%%%%%%%%%%%%%%%%%%%%%%%%%%%%%%%%%%%%%
\subsection{Command Line Processing}
\label{sec:commandline}

The effect of redirection files can also be achieved by invoking
the \LaTeX{} compiler with a more elaborate command line.
Most conveniently this should be done as part
of a shell script or a |Makefile|.

When using \textsf{childdoc} in the main file, the following
command lines effectively perform a redirection
(note that depending on the shell being used,
backslashes may have to be doubled: `|\|' $\to$ `|\\|'):
%
\begin{center}
|... -jobname "|\textit{target}|" |\\|"|[\textit{flags}]%
|\input{childdoc.def}\childdocforward[|\textit{main}|]{|\textit{dest}|}"|
\end{center}
%
Here \textit{target} is the name of the output file,
\textit{main} is the name of the main file
and \textit{dest} is the name of the main or child file to be processed
(all filenames without extensions).
The optional argument \textit{main} can be omitted
if \textit{main} matches \textit{dest}.
Optionally, compilation \textit{flags} can be defined via |\def| commands.
This command line makes the \TeX{} engine believe
it is compiling the file \textit{target}
whose content is specified as the latter parameter.
The provided code then forwards the processing to
\textit{main} or \textit{dest} as described in \secref{sec:forward}.

%%%%%%%%%%%%%%%%%%%%%%%%%%%%%%%%%%%%%%%%%%%%%%%%%%%%%%%%%%%%%%%%%%%%%%%%%%%%%%%%
\subsection{Include by Input}
\label{sec:input}

Including child documents by |\include| has some restrictions by design.
Most notably, the content of a child document always occupies
its own set of pages; pages cannot be shared between child documents.
Usually, this behaviour makes perfect sense
because each child document contain an essential part of the document.
However, in some situations it may be desirable to compose
a document from a collection of parts
without having mandatory page breaks between then.
For this case, the package
provides a mechanism to include parts
by |\input| which can also be processed individually.
However, by construction this mechanism
requires manual handling of the content to be output.

%%%%%%%%%%%%%%%%%%%%%%%%%%%%%%%%%%%%%%%%
\DescribeMacro{\ifchilddocmanual}
The main file should be prepared as usual, see \secref{sec:include}.
However, the document body must make a distinction
between processing of an individual part and of the main document, e.g.:
%
\begin{center}
\begin{tabular}{l}
|\ifchilddocmanual|\\
|\input{\childdocname}|\\
|\||else|\\
\textit{document body with }|\input{|\textit{part}|}|\\
|\||fi|
\end{tabular}
\end{center}
%
The conditional |\ifchilddocmanual| is true whenever
a part to be included by |\input| is being compiled,
and the name of the part is stored in |\childdocname|.

%%%%%%%%%%%%%%%%%%%%%%%%%%%%%%%%%%%%%%%%
\DescribeMacro{\childdocby}
Each part to be included by |\input| should start with:
%
\begin{center}
\begin{tabular}{l}
|\input{childdoc.def}|\\
|\childdocby{|\textit{main}|}|\\
\end{tabular}
\end{center}
%
The directive |\childdocby| is similar to |\childdocof|
described in \secref{sec:include},
but the subsequent selection of content must be done manually.
To that end, both |\ifchilddoc| and |\ifchilddocmanual|
will be true upon processing of a part,
and the name of the part is stored in |\childdocname|.
Note that |\jobname| will be set to the filename of the current part
so that each part receives an individual |.aux| file
that does not interfere with the |.aux| file(s) of the main document.
This behaviour can be altered by the alternative form
|\childdocby[*]{|\textit{main}|}| (with a non-empty optional argument)
which uses the |.aux| file of the main document
by setting |\jobname| to \textit{main}.

%%%%%%%%%%%%%%%%%%%%%%%%%%%%%%%%%%%%%%%%%%%%%%%%%%%%%%%%%%%%%%%%%%%%%%%%%%%%%%%%
\subsection{Driver Development}
\label{sec:driver}

The \textsf{childdoc} mechanism can also be use for the development
of definition files such as \LaTeX{} styles or classes.
This case differs from the above setup with multiple parts
included by |\include| in that no |\includeonly| should be invoked.
This can be achieved by starting the include file
(before |\ProvidesPackage|) with:
%
\begin{center}
\begin{tabular}{l}
|\input{childdoc.def}|\\
|\childdocforward{|\textit{main}|}|\\
\end{tabular}
\end{center}
%
or alternatively with:
%
\begin{center}
\begin{tabular}{l}
|\input{childdoc.def}|\\
|\childdocby{|\textit{main}|}|\\
\end{tabular}
\end{center}
%
Both forms have slightly different effects as described above.
The main file is prepared as usual, see \secref{sec:include}.

%%%%%%%%%%%%%%%%%%%%%%%%%%%%%%%%%%%%%%%%%%%%%%%%%%%%%%%%%%%%%%%%%%%%%%%%%%%%%%%%
\subsection{Legacy Detection}
\label{sec:detection}

The directive |\childdocmain| in the main file can detect
whether the complete document or merely a child is to be compiled
even without using the directive |\childdocof|.
This method is deprecated because it is less robust
and there is no compelling reason to use it;
it is merely provided for backward compatibility
and it may be removed in future versions.

If the detection mechanism is to be used,
it is mandatory to correctly specify
the filename of the main file as the argument of |\childdocmain|:
%
\begin{center}
\begin{tabular}{l}
|\input{childdoc.def}|\\
|\childdocmain{|\textit{main}|}|\\
\end{tabular}
\end{center}
%
If |\jobname| does not match the argument \textit{main} of |\childdocmain|,
it is assumed that |\jobname| points to the child file to be compiled.
When using |\childdocmain| with the main file specified as argument,
it suffices to start a child file
with just |\input{|\textit{main}|}|
without loading of the package and using |\childdocof|.
If instead all processing is done
with the appropriate \textsf{childdoc} directives,
the argument of \textit{main} of |\childdocmain| can be empty.

An alternative version of the command line processing described
in \secref{sec:commandline} using the detection mechanism reads:
%
\begin{center}
|... -jobname "|\textit{target}|" "|[\textit{flags}]%
[|\def\jobname{|\textit{dest}|}|]|\input{|\textit{main}|}"|
\end{center}

%%%%%%%%%%%%%%%%%%%%%%%%%%%%%%%%%%%%%%%%%%%%%%%%%%%%%%%%%%%%%%%%%%%%%%%%%%%%%%%%
\subsection{Manual Code}
\label{sec:manual}

In case one cannot be certain whether the definitions file |childdoc.def|
is installed on the target \TeX{} distribution
and one prefers not to ship it,
it is conceivable to paste a few relevant commands into the sources.

To that end, drop all statements |\input{childdoc.def}|
and perform the replacements as outlined below.
Instead of |\childdocmain{|\textit{main}|}| add the following code
to the top of the main file:
%
\begin{center}
\begin{tabular}{l}
|\||ifdefined\childdocname\endinput\||fi\newif\ifchilddoc|\\
|\edef\childdocname{\scantokens\expandafter{\jobname\noexpand}}|\\
|\def\childdocmain{|\textit{main}|}\||ifx\childdocmain\childdocname\||else|\\
|\childdoctrue\includeonly{\childdocname}\let\jobname\childdocmain\||fi|\\
\end{tabular}
\end{center}
%
Instead of |\childdocof{|\textit{main}|}| just include the main file
at the top of each child file:
%
\begin{center}
|\input{|\textit{main}|}|
\end{center}
%
A simple redirection |\childdocforward{|\textit{dest}|}| is achieved by:
%
\begin{center}
|\def\jobname{|\textit{dest}|}\input{\jobname}|
\end{center}
%
The redirection with prefix
|\childdocforwardprefix[|\textit{prefix}|]{|\textit{dest}|}|
is accomplished by:
%
\begin{center}
\begin{tabular}{l}
|{\edef\jobname{\scantokens\expandafter{\jobname\noexpand}}|\\
|\def\redirectjob |\textit{prefix}|#1~~~{\gdef\jobname{|\textit{dest}|#1}}|\\
|\expandafter\redirectjob\jobname~~~}\input{\jobname}|
\end{tabular}
\end{center}

In an alternative approach,
child documents can be compiled by a specific command line
without additional code or specific definitions:
%
\begin{center}
|... -jobname "|\textit{target}|" "|[\textit{flags}]%
|\includeonly{|\textit{dest}|}\input{|\textit{main}|}"|
\end{center}
%

%%%%%%%%%%%%%%%%%%%%%%%%%%%%%%%%%%%%%%%%%%%%%%%%%%%%%%%%%%%%%%%%%%%%%%%%%%%%%%%%
%%%%%%%%%%%%%%%%%%%%%%%%%%%%%%%%%%%%%%%%%%%%%%%%%%%%%%%%%%%%%%%%%%%%%%%%%%%%%%%%
\section{Information}

%%%%%%%%%%%%%%%%%%%%%%%%%%%%%%%%%%%%%%%%%%%%%%%%%%%%%%%%%%%%%%%%%%%%%%%%%%%%%%%%
\subsection{Copyright}

Copyright \copyright{} 2017--2018 Niklas Beisert

This work may be distributed and/or modified under the
conditions of the \LaTeX{} Project Public License, either version 1.3
of this license or (at your option) any later version.
The latest version of this license is in
  \url{http://www.latex-project.org/lppl.txt}
and version 1.3 or later is part of all distributions of \LaTeX{}
version 2005/12/01 or later.

This work has the LPPL maintenance status `maintained'.

The Current Maintainer of this work is Niklas Beisert.

This work consists of the files |README.txt|, |childdoc.ins| and |childdoc.dtx|
as well as the derived files |childdoc.def|, |cdocsamp.tex|
with |cdocsch1.tex|, |cdocsch2.tex|, |cdocspt3.tex|, |cdocspt4.tex|,
|cdocsdrf.tex|, |cdocsfn1.tex|, |cdocsfn2.tex|
as well as |childdoc.pdf|.

%%%%%%%%%%%%%%%%%%%%%%%%%%%%%%%%%%%%%%%%%%%%%%%%%%%%%%%%%%%%%%%%%%%%%%%%%%%%%%%%
\subsection{Files and Installation}

The package consists of the files:
%
\begin{center}
\begin{tabular}{ll}
    |README.txt|   & readme file \\
    |childdoc.ins| & installation file \\
    |childdoc.dtx| & source file \\
    |childdoc.def| & definition file \\
    |cdocsamp.tex| & sample main file \\
    |cdocsch1.tex| & sample include file \\
    |cdocsch2.tex| & sample include file \\
    |cdocspt3.tex| & sample part file \\
    |cdocspt4.tex| & sample part file \\
    |cdocsdrf.tex| & sample redirection file \\
    |cdocsfn1.tex| & sample redirection file \\
    |cdocsfn2.tex| & sample redirection file \\
    |childdoc.pdf| & manual
\end{tabular}
\end{center}
%
The distribution consists of the files
|README.txt|, |childdoc.ins| and |childdoc.dtx|.
%
\begin{itemize}
\item
Run (pdf)\LaTeX{} on |childdoc.dtx|
to compile the manual |childdoc.pdf| (this file).
\item
Run \LaTeX{} on |childdoc.ins| to create the definitions file |childdoc.def|
and the sample |cdocsamp.tex| with include files
|cdocsch1.tex|, |cdocsch2.tex|, |cdocspt3.tex|, |cdocspt4.tex|,
|cdocsdrf.tex|, |cdocsfn1.tex|, |cdocsfn2.tex|.
Then copy the file |childdoc.def| to an appropriate directory of your \LaTeX{}
distribution, e.g.\ \textit{texmf-root}|/tex/latex/childdoc|.
\end{itemize}

%%%%%%%%%%%%%%%%%%%%%%%%%%%%%%%%%%%%%%%%%%%%%%%%%%%%%%%%%%%%%%%%%%%%%%%%%%%%%%%%
\subsection{Related CTAN Packages}

There are several other packages which offer a similar functionality:
%
\begin{itemize}
\item
The packages
\href{http://ctan.org/pkg/docmute}{\textsf{docmute}},
\href{http://ctan.org/pkg/includex}{\textsf{includex}} and
\href{http://ctan.org/pkg/standalone}{\textsf{standalone}}
provide commands to include only the document body of
a child file thus allowing both files to be compiled individually.
\item
The packages \href{http://ctan.org/pkg/subdocs}{\textsf{subdocs}}
and \href{http://ctan.org/pkg/subfiles}{\textsf{subfiles}}
provide structures in which the main and child documents can be
encapsulated and allowing them to be compiled individually.
The inclusion mechanism is different from the conventional |\include|.
\item
The package \href{http://ctan.org/pkg/combine}{\textsf{combine}}
is an elaborate solution to combine several documents into one.
\end{itemize}
%
See also the CTAN topic \href{http://ctan.org/topic/subdocs}{\textsf{subdocs}}
for further related packages.
The present package differs from the above solutions in that
a document structure constructed with the conventional |\include| mechanism
just needs two extra commands at the top of every file
such that all constituent files can be compiled individually.

%%%%%%%%%%%%%%%%%%%%%%%%%%%%%%%%%%%%%%%%%%%%%%%%%%%%%%%%%%%%%%%%%%%%%%%%%%%%%%%%
%\subsection{Feature Suggestions}
%
%The following is a list of features which may be useful for future
%versions of this package:
%%
%\begin{itemize}
%\item
%\ldots
%\end{itemize}

%%%%%%%%%%%%%%%%%%%%%%%%%%%%%%%%%%%%%%%%%%%%%%%%%%%%%%%%%%%%%%%%%%%%%%%%%%%%%%%%
\subsection{Revision History}

%%%%%%%%%%%%%%%%%%%%%%%%%%%%%%%%%%%%%%%%
\paragraph{v2.0:} 2018/12/30

\begin{itemize}
\item
immediate forward processing
\item
added |\childdocby| mechanism
\item
manual restructured
\end{itemize}

%%%%%%%%%%%%%%%%%%%%%%%%%%%%%%%%%%%%%%%%
\paragraph{v1.6:} 2018/01/17

\begin{itemize}
\item
application for development of include files
\item
corrections to manual
\end{itemize}

%%%%%%%%%%%%%%%%%%%%%%%%%%%%%%%%%%%%%%%%
\paragraph{v1.5:} 2017/05/21

\begin{itemize}
\item
more complete structuring introduced
\item
|\childdocof| introduced
\item
|\childdoc| renamed to |\childdocmain|
\item
|\childredirect| renamed to |\childdocforward| and |\childdocforwardprefix|
and functionality expanded
\end{itemize}

%%%%%%%%%%%%%%%%%%%%%%%%%%%%%%%%%%%%%%%%
\paragraph{v1.0:} 2017/04/27

\begin{itemize}
\item
manual and install package
\item
first version published on CTAN
\end{itemize}

%%%%%%%%%%%%%%%%%%%%%%%%%%%%%%%%%%%%%%%%
\paragraph{v0.6:} 2017/04/26

\begin{itemize}
\item
redirection mechanism added
\end{itemize}

%%%%%%%%%%%%%%%%%%%%%%%%%%%%%%%%%%%%%%%%
\paragraph{v0.5:} 2017/04/26

\begin{itemize}
\item
functionality in definition file
\end{itemize}


%%%%%%%%%%%%%%%%%%%%%%%%%%%%%%%%%%%%%%%%%%%%%%%%%%%%%%%%%%%%%%%%%%%%%%%%%%%%%%%%
%%%%%%%%%%%%%%%%%%%%%%%%%%%%%%%%%%%%%%%%%%%%%%%%%%%%%%%%%%%%%%%%%%%%%%%%%%%%%%%%
%%%%%%%%%%%%%%%%%%%%%%%%%%%%%%%%%%%%%%%%%%%%%%%%%%%%%%%%%%%%%%%%%%%%%%%%%%%%%%%%
\appendix

\settowidth\MacroIndent{\rmfamily\scriptsize 000\ }

 \DocInput{childdoc.dtx}

\end{document}
%</driver>
% \fi
%
% %%%%%%%%%%%%%%%%%%%%%%%%%%%%%%%%%%%%%%%%%%%%%%%%%%%%%%%%%%%%%%%%%%%%%%%%%%%%%%
% %%%%%%%%%%%%%%%%%%%%%%%%%%%%%%%%%%%%%%%%%%%%%%%%%%%%%%%%%%%%%%%%%%%%%%%%%%%%%%
% \section{Sample}
%\iffalse
%<*samplemain>
%\fi
%
% The following presents a sample document
% with two chapters, two parts, a title page,
% a compile flag as well as three forwarding files to set the flag.
% It consists of eight |.tex| files:
% \begin{center}
% \begin{tabular}{ll}
% |cdocsamp.tex|&main file\\
% |cdocsch1.tex|&include file for chapter 1\\
% |cdocsch2.tex|&include file for chapter 2\\
% |cdocspt3.tex|&include file for part 3\\
% |cdocspt4.tex|&include file for part 4\\
% |cdocsdrf.tex|&forwarding file for main file in draft mode\\
% |cdocsfi1.tex|&forwarding file for final version of chapter 1\\
% |cdocsfi2.tex|&forwarding file for final version of chapter 2\\
% \end{tabular}
% \end{center}
% Each of the eight files can be compiled directly by the \LaTeX{} compiler.
%
% %%%%%%%%%%%%%%%%%%%%%%%%%%%%%%%%%%%%%%
% \paragraph{Main File.}
%
% The main file is called |cdocsamp.tex|.
%
% Load the \textsf{childdoc} definitions and
% declare the filename for the main document:
%    \begin{macrocode}
\input{childdoc.def}
\childdocmain{}
%    \end{macrocode}

% Optional override for |\version| flag:
%    \begin{macrocode}
%%\ifchilddoc\else\providecommand{\version}{draft}\fi
%    \end{macrocode}

% Define the default values for the |\version| flag
% (|final| for the main file and |draft| for childs):
%    \begin{macrocode}
\ifchilddoc
\providecommand{\version}{draft}
\else
\providecommand{\version}{final}
\fi
%    \end{macrocode}

% Load the standard document class:
%    \begin{macrocode}
\documentclass[12pt]{article}
%    \end{macrocode}

% Start the document body:
%    \begin{macrocode}
\begin{document}
%    \end{macrocode}

% Declare a title page.
% Print title, part of document being processed and version flag:
%    \begin{macrocode}
\addtocounter{page}{-1}
\begin{center}
{\LARGE\bfseries{}childdoc example\par}
\vspace{1cm}
\ifchilddoc
\ifchilddocmanual part\else chapter\fi:
`\childdocname' of `\childdocjob'\par
\else
main document: `\childdocjob'\par
\fi
version: \version\par
\end{center}
\newpage
%    \end{macrocode}

% Manually include selected file,
% otherwise process as usual:
%    \begin{macrocode}
\ifchilddocmanual
\section*{part `\childdocname'}
\input{\childdocname}
\else
%    \end{macrocode}

% Include the two chapters:
%    \begin{macrocode}
\include{cdocsch1}
\include{cdocsch2}
%    \end{macrocode}

% Include the two parts unless only chapters should be displayed:
%    \begin{macrocode}
\ifchilddoc\else
\section{part three}
\input{cdocspt3}
\section{part four}
\input{cdocspt4}
\fi
%    \end{macrocode}

% Process as usual until here:
%    \begin{macrocode}
\fi
%    \end{macrocode}

% End of document body:
%    \begin{macrocode}
\end{document}
%    \end{macrocode}
%\iffalse
%</samplemain>
%\fi
%
% %%%%%%%%%%%%%%%%%%%%%%%%%%%%%%%%%%%%%%
% \paragraph{Chapter Include Files.}
%
% The include files are called |cdocsch1.tex| and |cdocsch2.tex|.
%
%\iffalse
%<*samplechap1|samplechap2>
%\fi

% Optional override for |\version| flag:
%    \begin{macrocode}
%%\providecommand{\version}{final}
%    \end{macrocode}

% Include the main document:
%    \begin{macrocode}
\input{childdoc.def}
\childdocof{cdocsamp}
%    \end{macrocode}

%\iffalse
%</samplechap1|samplechap2>
%\fi
%
%\iffalse
%<*samplechap1>
%\fi
% Some text for chapter 1:
%    \begin{macrocode}
\section{one}
some text in chapter one
%    \end{macrocode}

%\iffalse
%</samplechap1>
%\fi
% Some text for chapter 2:
%\iffalse
%<*samplechap2>
%\fi
%    \begin{macrocode}
\section{two}
more text in chapter two
%    \end{macrocode}

%\iffalse
%</samplechap2>
%\fi
%
% %%%%%%%%%%%%%%%%%%%%%%%%%%%%%%%%%%%%%%
% \paragraph{Part Include Files.}
%
% The include files are called |cdocspt3.tex| and |cdocspt4.tex|.
%
%\iffalse
%<*samplepart3|samplepart4>
%\fi

% Optional override for |\version| flag:
%    \begin{macrocode}
%%\providecommand{\version}{final}
%    \end{macrocode}

% Include the main document:
%    \begin{macrocode}
\input{childdoc.def}
\childdocby{cdocsamp}
%    \end{macrocode}

%\iffalse
%</samplepart3|samplepart4>
%\fi
%
%\iffalse
%<*samplepart3>
%\fi
% Some text for part 3:
%    \begin{macrocode}
some text in part three
%    \end{macrocode}

%\iffalse
%</samplepart3>
%\fi
% Some text for part 4:
%\iffalse
%<*samplepart4>
%\fi
%    \begin{macrocode}
more text in part four
%    \end{macrocode}

%\iffalse
%</samplepart4>
%\fi
%
% %%%%%%%%%%%%%%%%%%%%%%%%%%%%%%%%%%%%%%
% \paragraph{Forwarding for a Complete Draft.}
%
% The following forwarding file |cdocsdrf.tex|
% compiles the main document in draft mode:
%\iffalse
%<*sampledraft>
%\fi
%    \begin{macrocode}
\def\version{draft}
\input{childdoc.def}
\childdocforward{cdocsamp}
%    \end{macrocode}

%\iffalse
%</sampledraft>
%\fi
%
% %%%%%%%%%%%%%%%%%%%%%%%%%%%%%%%%%%%%%%
% \paragraph{Forwarding for Final Version of the Chapters.}
%
% The following forwarding files |cdocsfn1.tex| and |cdocsfn2.tex|
% (with identical content)
% compile the final versions of the child documents
% |cdocsch1.tex| and |cdocsch2.tex|, respectively:
%\iffalse
%<*samplefinal>
%\fi
%    \begin{macrocode}
\def\version{final}
\input{childdoc.def}
\childdocforwardprefix[cdocsamp]{cdocsfn}{cdocsch}
%    \end{macrocode}

%\iffalse
%</samplefinal>
%\fi
%
% %%%%%%%%%%%%%%%%%%%%%%%%%%%%%%%%%%%%%%
% \paragraph{Command Line Processing.}
%
% The following three command lines generate the output files
% |cdocscld|, |cdocscl1| and |cdocscl2|
% which should be identical to
% |cdocsdrf|, |cdocsch1| and |cdocsfn2|, respectively:
% \begin{center}
% \begin{tabular}{l}
% |latex -jobname cdocscld \|\\
% |  "\def\version{draft}\input{childdoc.def}\childdocforward{cdocsamp}"|\\
% |latex -jobname cdocscl1 \|\\
% |  "\input{childdoc.def}\childdocforward[cdocsamp]{cdocsch1}"|\\
% |latex -jobname cdocscl2 \|\\
% |  "\def\version{final}\input{childdoc.def}\childdocforward{cdocsch2}"|
% \end{tabular}
% \end{center}
% Note that the trailing backslash on each first line
% merely continues the input to the second line
% (for convenient cut ant paste).
% Furthermore, the command |latex| can be replaced by any
% of its alternative versions such as |pdflatex|.
%
% %%%%%%%%%%%%%%%%%%%%%%%%%%%%%%%%%%%%%%%%%%%%%%%%%%%%%%%%%%%%%%%%%%%%%%%%%%%%%%
% %%%%%%%%%%%%%%%%%%%%%%%%%%%%%%%%%%%%%%%%%%%%%%%%%%%%%%%%%%%%%%%%%%%%%%%%%%%%%%
% \section{Implementation}
%\iffalse
%<*package>
%\fi
%
% This section describes the definitions file |childdoc.def|.

% The definitions cannot be loaded using |\usepackage| or |\RequirePackage|
% which has a mechanism to prevent loading a style file more than once.
% When loading the definitions by means of |\input|
% multiple instances have to be prevented manually:
%\iffalse
%This code needs to be before the `\ProvidesFile' directive
%which is defined at the beginning of this file.
%Therefore it is also placed there and commented out here.
%</package>
%<*discard>
%\fi
%    \begin{macrocode}
\ifdefined\childdocmain\endinput\fi
%    \end{macrocode}
%\iffalse
%</discard>
%<*package>
%\fi
%
% \macro{\ifchilddoc}
% \macro{\ifchilddocmanual}
% The conditional |\ifchilddoc| tells whether a
% child (true) or main (false) document is being compiled.
% The conditional |\ifchilddocmanual| tells whether
% the |\includeonly| mechanism is used (false) or
% the selection of child files must be performed manually (true).
% The definitions initialise to false:
%    \begin{macrocode}
\newif\ifchilddoc
\newif\ifchilddocmanual
%    \end{macrocode}

% \macro{\childdocname}
% \macro{\childdocjob}
% The macro |\childdocname| stores the name of the main document
% to be compiled. The macro |\childdocjob| stores the name of
% the document on which the \LaTeX{} compiler was originally invoked.
% The content of |\jobname| cannot be compared
% to filenames specified in the source due to different catcodes.
% The following code rescans |\jobname|, stores the result
% in |\childdocname| and saves a copy in |\childdocjob|:
%    \begin{macrocode}
\edef\childdocname{\scantokens\expandafter{\jobname\noexpand}}
\let\childdocjob\childdocname
%    \end{macrocode}

% \macro{\childdocdisable}
% The macro |\childdocdisable| prevents the main file
% from being processed more than once.
% At this stage, the main document command |\childdocmain|
% is assumed to be called once again where it should do nothing.
% Any subsequent call to it should prevent
% a secondary processing of the main document
% It overwrites the forwarding commands
% |\childdocof| and |\childdocforward|
% with empty macros to prevent further inclusions of the main document:
%    \begin{macrocode}
\newcommand{\childdocdisable}
{
  \renewcommand{\childdocmain}[1]{\renewcommand{\childdocmain}[1]{\endinput}}
  \renewcommand{\childdocof}[1]{}
  \renewcommand{\childdocby}[2][]{}
  \renewcommand{\childdocforward}[2][]{}
  \renewcommand{\childdocdisable}{}
}
%    \end{macrocode}

% \macro{\childdocmain}
% The macro |\childdocmain| is to be called at the top of the main file
% with nothing or the main filename (without extension) as argument.
% First, it breaks loops.
% If the argument is not empty and does not match |\childdocname|
% (which is set by the first inclusion of |childdoc.def|),
% |\ifchilddoc| is set to true, |\includeonly| is applied to the child file
% and |\jobname| is set to the main file
% (for proper handling of |.aux| files):
%    \begin{macrocode}
\newcommand{\childdocmain}[1]
{
  \childdocdisable\childdocmain{}
  \if?#1?\else
    \begingroup
      \def\childdoctmp{#1}
      \ifx\childdoctmp\childdocname
        \def\childdoctmp{}
      \else
        \def\childdoctmp
        {
          \childdoctrue
          \includeonly{\childdocname}
          \def\childdocjob{#1}
          \def\jobname{#1}
        }
      \fi
      \expandafter
    \endgroup
    \childdoctmp
  \fi
}
%    \end{macrocode}

% \macro{\childdocof}
% The command |\childdocof| redirects
% compilation to the main file |#1|.
%    \begin{macrocode}
\newcommand{\childdocof}[1]
{
  \childdocdisable
  \childdoctrue
  \includeonly{\childdocname}
  \def\jobname{#1}
  \def\childdocjob{#1}
  \input{#1}
}
%    \end{macrocode}

% \macro{\childdocby}
% The command |\childdocby| ....
%    \begin{macrocode}
\newcommand{\childdocby}[2][]
{
  \childdocdisable
  \childdoctrue
  \childdocmanualtrue
  \if?#1?\else
    \def\jobname{#2}
  \fi
  \def\childdocjob{#2}
  \input{#2}
  \endinput
}
%    \end{macrocode}

% \macro{\childdocforward}
% The command |\childdocforward| redirects
% compilation to the main file or
% (if the optional argument is given) a child file.
% Parameters are set as if the main file
% or a child file starting with |\childdocof| was compiled.
% Then compilation is handed over to the main file:
%    \begin{macrocode}
\newcommand{\childdocforward}[2][]
{
  \begingroup
    \if?#1?
      \def\childdoctmp
      {
        \def\childdocname{#2}
        \def\childdocjob{#2}
        \def\jobname{#2}
        \input{#2}
        \endinput
      }
    \else
      \def\childdoctmp
      {
        \childdocdisable
        \def\childdocname{#2}
        \childdoctrue
        \includeonly{#2}
        \def\childdocjob{#1}
        \def\jobname{#1}
        \input{#1}
        \endinput
      }
    \fi
    \expandafter
  \endgroup
  \childdoctmp
}
%    \end{macrocode}

% \macro{\childdocforwardprefix}
% The command |\childdocforwardprefix| redirects
% compilation to the main or a child file by means of a pattern.
% The prefix |#1| in the current filename is replaced by |#2|
% and the suffix of the current filename is kept
% (it is assumed that the filename does not contain the substring `|~~~|'
% which is used as a delimiter).
% Compilation is handed over to the new file by |\childdocforward|:
%    \begin{macrocode}
\newcommand{\childdocforwardprefix}[3][]
{
  \begingroup
    \def\childdocextract #2##1~~~{\def\childdoctmp{\childdocforward[#1]{#3##1}}}
    \expandafter\childdocextract\childdocname~~~
    \expandafter
  \endgroup
  \childdoctmp
}
%    \end{macrocode}

% \macro{\childdoc}
% The deprecated macro |\childdoc| is a legacy version of |\childdocmain|:
%    \begin{macrocode}
\newcommand{\childdoc}{\childdocmain}
%    \end{macrocode}

% \macro{\childdocredirect}
% The deprecated macro |\childdocredirect| is a legacy version
% of |\childdocforward| and |\childdocforwardprefix|:
%    \begin{macrocode}
\newcommand{\childdocredirect}[2][]
{
  \begingroup
    \if?#1?
      \def\childdoctmp{\childdocforward{#2}}
    \else
      \def\childdoctmp{\childdocforwardprefix{#1}{#2}}
    \fi
    \expandafter
  \endgroup
  \childdoctmp
}
%    \end{macrocode}

%\iffalse
%</package>
%\fi
%
\endinput

\childdocof{cdocsamp}
%    \end{macrocode}

%\iffalse
%</samplechap1|samplechap2>
%\fi
%
%\iffalse
%<*samplechap1>
%\fi
% Some text for chapter 1:
%    \begin{macrocode}
\section{one}
some text in chapter one
%    \end{macrocode}

%\iffalse
%</samplechap1>
%\fi
% Some text for chapter 2:
%\iffalse
%<*samplechap2>
%\fi
%    \begin{macrocode}
\section{two}
more text in chapter two
%    \end{macrocode}

%\iffalse
%</samplechap2>
%\fi
%
% %%%%%%%%%%%%%%%%%%%%%%%%%%%%%%%%%%%%%%
% \paragraph{Part Include Files.}
%
% The include files are called |cdocspt3.tex| and |cdocspt4.tex|.
%
%\iffalse
%<*samplepart3|samplepart4>
%\fi

% Optional override for |\version| flag:
%    \begin{macrocode}
%%\providecommand{\version}{final}
%    \end{macrocode}

% Include the main document:
%    \begin{macrocode}
% \iffalse
%
% childdoc.dtx Copyright (C) 2017-2018 Niklas Beisert
%
% This work may be distributed and/or modified under the
% conditions of the LaTeX Project Public License, either version 1.3
% of this license or (at your option) any later version.
% The latest version of this license is in
%   http://www.latex-project.org/lppl.txt
% and version 1.3 or later is part of all distributions of LaTeX
% version 2005/12/01 or later.
%
% This work has the LPPL maintenance status `maintained'.
%
% The Current Maintainer of this work is Niklas Beisert.
%
% This work consists of the files childdoc.dtx and childdoc.ins
% and the derived files childdoc.def and cdocsamp.tex with
% cdocsch1.tex, cdocsch2.tex, cdocsdrf.tex, cdocsfn1.tex, cdocsfn2.tex.
%
%<package>\ifdefined\childdocmain\endinput\fi
%<package>\ProvidesFile{childdoc.def}[2018/12/30 v2.0 child document driver]
%<samplemain>\ProvidesFile{cdocsamp.tex}[2018/12/30 v2.0 sample for childdoc]
%<*driver>
%\ProvidesFile{childdoc.drv}[2018/12/30 v2.0 childdoc reference manual file]
\PassOptionsToClass{10pt,a4paper}{article}
\documentclass{ltxdoc}

\usepackage[margin=35mm]{geometry}
\usepackage{hyperref}
\usepackage{hyperxmp}
\usepackage[usenames]{color}

\hypersetup{colorlinks=true}
\hypersetup{pdfstartview=FitH}
\hypersetup{pdfpagemode=UseNone}
\hypersetup{pdfsource={}}
\hypersetup{pdflang={en-UK}}
\hypersetup{pdfcopyright={Copyright 2017-2018 Niklas Beisert.
  This work may be distributed and/or modified under the
  conditions of the LaTeX Project Public License, either version 1.3
  of this license or (at your option) any later version.}}
\hypersetup{pdflicenseurl={http://www.latex-project.org/lppl.txt}}
\hypersetup{pdfcontactaddress={ETH Zurich, ITP, HIT K,
  Wolfgang-Pauli-Strasse 27}}
\hypersetup{pdfcontactpostcode={8093}}
\hypersetup{pdfcontactcity={Zurich}}
\hypersetup{pdfcontactcountry={Switzerland}}
\hypersetup{pdfcontactemail={nbeisert@itp.phys.ethz.ch}}
\hypersetup{pdfcontacturl={http://people.phys.ethz.ch/\xmptilde nbeisert/}}

\newcommand{\secref}[1]{\hyperref[#1]{section \ref*{#1}}}

\parskip1ex
\parindent0pt
\let\olditemize\itemize
\def\itemize{\olditemize\parskip0pt}

\begin{document}

\title{The \textsf{childdoc} Package}
\hypersetup{pdftitle={The childdoc Package}}
\author{Niklas Beisert\\[2ex]
  Institut f\"ur Theoretische Physik\\
  Eidgen\"ossische Technische Hochschule Z\"urich\\
  Wolfgang-Pauli-Strasse 27, 8093 Z\"urich, Switzerland\\[1ex]
  \href{mailto:nbeisert@itp.phys.ethz.ch}
  {\texttt{nbeisert@itp.phys.ethz.ch}}}
\hypersetup{pdfauthor={Niklas Beisert}}
\hypersetup{pdfsubject={Manual for the LaTeX2e Package childdoc}}
\date{30 December 2018, \textsf{v2.0}}
\maketitle

\begin{abstract}\noindent
\textsf{childdoc} is a \LaTeXe{} package
that enables the direct compilation
of document sections included by |\include|
to individual files.
\end{abstract}

\begingroup
\parskip0ex
\tableofcontents
\endgroup

%%%%%%%%%%%%%%%%%%%%%%%%%%%%%%%%%%%%%%%%%%%%%%%%%%%%%%%%%%%%%%%%%%%%%%%%%%%%%%%%
%%%%%%%%%%%%%%%%%%%%%%%%%%%%%%%%%%%%%%%%%%%%%%%%%%%%%%%%%%%%%%%%%%%%%%%%%%%%%%%%
\section{Introduction}

\LaTeX{} provides a mechanism to structure a large document (such as a book)
into a main file and several child files (containing the chapters)
using the |\include| command.
This mechanism is beneficial for documents
which span hundreds of pages in order to
make the source file(s) more manageable.
Moreover, compilation can be restricted to
selected child files by means of the |\includeonly| command.
The latter feature can be used to reduce the compilation time while editing
(this was significantly more useful in the earlier days of \LaTeX{})
or to generate a smaller document which is easier to navigate.
Another application of |\includeonly| is to generate
documents consisting of selected parts of the complete document.

However, there are a few drawbacks of the plain |\include| mechanism:
\begin{itemize}
\item
The child files cannot be compiled on their own,
they can only be compiled via the main file.
A naive editing environment
(such as a text editor with an option
to have the current file processed by \LaTeX)
may require one to switch to the main file before compiling;
attempting to compile the child file produces errors.
\item
The main file must be modified (each time)
to adjust the |\includeonly| command
to the present needs. This easily leaves the main file in a messy state.
\item
The generated document will always carry the filename
of the main document. This is inconvenient if
several child files are to be compiled and
to be kept for distribution.
\end{itemize}

The present package provides a simple interface
to make child files individually compilable by \LaTeX{}.
Compiling a child file then has the same effect as compiling
the main file with an |\includeonly| command
to select the appropriate child.
Moreover the generated document will carry the name of the child
rather than the main file.
This resolves all three above issues.

This feature is meant to make the editing of books,
thesis documents and lecture notes somewhat more convenient.
However, the package can also be used efficiently for
composing a series of documents (such as exercise sheets)
which are typically distributed individually.
It then assists the author in generating the individual documents
(potentially in different versions)
as well as a document containing the collected series.
Another application is in developing style files
or other kinds of included material
where compilation of the style file could redirect
to a sample or test file.

%%%%%%%%%%%%%%%%%%%%%%%%%%%%%%%%%%%%%%%%%%%%%%%%%%%%%%%%%%%%%%%%%%%%%%%%%%%%%%%%
%%%%%%%%%%%%%%%%%%%%%%%%%%%%%%%%%%%%%%%%%%%%%%%%%%%%%%%%%%%%%%%%%%%%%%%%%%%%%%%%
\section{Usage}

First of all, the package \textsf{childdoc} is \emph{not} a standard
\LaTeXe{} |.sty| style file! Therefore it needs to be invoked in
a non-standard way.

%%%%%%%%%%%%%%%%%%%%%%%%%%%%%%%%%%%%%%%%%%%%%%%%%%%%%%%%%%%%%%%%%%%%%%%%%%%%%%%%
\subsection{Included Files}
\label{sec:include}

%%%%%%%%%%%%%%%%%%%%%%%%%%%%%%%%%%%%%%%%
\DescribeMacro{\childdocmain}
To use the package, add the commands
\begin{center}
\begin{tabular}{l}
|\input{childdoc.def}|\\
|\childdocmain{}|\\
\end{tabular}
\end{center}
at the very top of the main \LaTeX{} file,
in particular \emph{before} the |\documentclass| statement!
The argument of |\childdocmain| should be left empty
(but it must be present).

%%%%%%%%%%%%%%%%%%%%%%%%%%%%%%%%%%%%%%%%
\DescribeMacro{\childdocof}
Furthermore, add the commands
\begin{center}
\begin{tabular}{l}
|\input{childdoc.def}|\\
|\childdocof{|\textit{main}|}|\\
\end{tabular}
\end{center}
at the top of every child file \textit{child}
which is included by |\include{|\textit{child}|}|
from within the main file
(or at least for those files to be compiled individually).
The argument \textit{main} must be the filename of the main file.

There are a couple of
considerations in setting up the main and child documents:

%%%%%%%%%%%%%%%%%%%%%%%%%%%%%%%%%%%%%%%%
\paragraph{Restrictions.}

Please note the following restrictions:
\begin{itemize}
\item
|\childdocmain| must be called with one argument \textit{main}
to ensure compatibility with earlier version of the package.
It must either be empty (|\childdocmain{}|)
or precisely match the filename of the main file in which it is specified.
See \secref{sec:detection} for further information.
\item
The filename \textit{main} must be specified without the |.tex| extension.
\item
The filename \textit{main} is case sensitive
(even in case-insensitive file systems)
due to internal string comparison.
\item
The argument \textit{main} should be fully expanded, it cannot be a macro.
\item
Subdirectories and special characters should be avoided in filenames.
\item
The command |\childdocmain{|\textit{main}|}| must be followed by a whitespace.
It should not be followed immediately by another command
or by a comment mark `|%|'.
This is because the \TeX{} parser reads the token immediately following
the argument of |\childdocmain| and puts it
at the beginning of every child section;
however, a white\-space is ignored.
\end{itemize}

%%%%%%%%%%%%%%%%%%%%%%%%%%%%%%%%%%%%%%%%
\paragraph{Content of Main File.}

It is advisable to place all content in the child files included by |\include|.
Any output contained in the main file will appear in all child documents
unless suppressed manually;
it cannot be suppressed automatically by the |\includeonly| directive
and thus should normally be avoided.
A method to include some content in the main file
by means of conditional processing is described in \secref{sec:conditional}.

%%%%%%%%%%%%%%%%%%%%%%%%%%%%%%%%%%%%%%%%
\paragraph{Page Numbering.}

When only a part of the document is compiled,
the appropriate numbering of pages
(as well as other status parameters)
is determined from the |.aux| files.
The latter contain information from previous passes.
However this information needs to propagate through
all intermediate child documents.
Therefore the page numbering in child documents may well
be inconsistent until the complete document is compiled at least once.

A useful (if unconventional) way to always ensure a consistent
page numbering is to restart the numbering in each child document
and denote the pages by `\textit{child}|.|\textit{page}'
where \textit{child} represents the chapter/section number of the child file.
This can be achieved by the command
|\numberwithin{page}{|\textit{child}|}|
of the \textsf{amsmath} package
where \textit{child} can be |chapter| or |section|
depending on the chosen structuring.
Alternatively, one can modify the macro |\thepage| appropriately
and reset the counter |page| at the start of each child file.

%%%%%%%%%%%%%%%%%%%%%%%%%%%%%%%%%%%%%%%%%%%%%%%%%%%%%%%%%%%%%%%%%%%%%%%%%%%%%%%%
\subsection{Conditional Processing}
\label{sec:conditional}

The package provides a mechanism to compile different versions
of a document. To customise the versions further some conditional processing
can come in handy to distinguish which version is being compiled.
The package provides two macros to describe the compilation context:

%%%%%%%%%%%%%%%%%%%%%%%%%%%%%%%%%%%%%%%%
\DescribeMacro{\ifchilddoc}
The conditional |\ifchilddoc| distinguishes between the compilation of
child documents and the main document:
%
\begin{center}
|\ifchilddoc |\textit{child-code}| |[|\||else |\textit{main-code}]| \||fi|
\end{center}

%%%%%%%%%%%%%%%%%%%%%%%%%%%%%%%%%%%%%%%%
\DescribeMacro{\childdocname}
\DescribeMacro{\childdocjob}
The macro |\childdocname| contains the filename (without extension)
of the main or child file being processed.
Note that |\childdocjob| will always contain the name of the main file.

%%%%%%%%%%%%%%%%%%%%%%%%%%%%%%%%%%%%%%%%
\paragraph{Title Page.}

Conditional processing can be used to include a title or banner page
in the main document when proper precautions are taken.
Importantly, the code in the main file should ensure that the page counter
(as well as other status parameters which are stored in the |.aux| files)
takes the same value after the conditional processing.
Otherwise the page numbers may take divergent values
depending on which part is compiled.

For example, a title page could be declared by:
%
\begin{center}
\begin{tabular}{l}
|\ifchilddoc\||else|\\
|\addtocounter{page}{-1}|\\
\textit{code for title page}\\
|\newpage|\\
|\||fi|
\end{tabular}
\end{center}
%
A banner page for the child documents can be generated by:
%
\begin{center}
\begin{tabular}{l}
|\ifchilddoc|\\
|\addtocounter{page}{-1}|\\
\textit{code for banner page}\\
|\newpage|\\
|\||fi|
\end{tabular}
\end{center}
%
Here one could write a message such as:
\begin{center}
|This is the part \childdocname{} of \childdocjob{}.|
\end{center}

%%%%%%%%%%%%%%%%%%%%%%%%%%%%%%%%%%%%%%%%%%%%%%%%%%%%%%%%%%%%%%%%%%%%%%%%%%%%%%%%
\subsection{Flags}
\label{sec:flags}

The package makes it easy to generate different versions
of the main or child documents.
To this end compilation flags can be defined
and assigned different default values.
They will be particularly useful in conjunction
with the forwarding mechanism described in \secref{sec:forward}.

For example, it may be useful to have a flag |\version|
which can be set to |draft| or |final|.
The document source will contain some conditional code
depending on the value of |\version|.
Suppose further, the flag should default to |final| for the main file
and to |draft| for child files
which is a natural assignment for editing the document.
This is achieved by placing the following code
in the preamble of the main document
(below the |\childdocmain| directive):
%
\begin{center}
\begin{tabular}{l}
|\ifchilddoc|\\
|\providecommand{\version}{draft}|\\
|\||else|\\
|\providecommand{\version}{final}|\\
|\||fi|
\end{tabular}
\end{center}
%
The definition by |\providecommand| makes sure
that previous definitions are not overwritten.
Further statements |\providecommand{\version}{...}|
can thus be added before the above code to override it.

For the main file, one might add a line
(between |\childdocmain| and the above block)
%
\begin{center}
|%\ifchilddoc\||else\providecommand{\version}{draft}\||fi|
\end{center}
%
which can be uncommented to produce a draft version.
Likewise one can add a line to the very top of a child file
(above the |\childdocof{|\textit{main}|}| directive)
%
\begin{center}
|%\providecommand{\version}{final}|
\end{center}
%
which can be uncommented to produce the final version of this child document.

%%%%%%%%%%%%%%%%%%%%%%%%%%%%%%%%%%%%%%%%%%%%%%%%%%%%%%%%%%%%%%%%%%%%%%%%%%%%%%%%
\subsection{Forwarding}
\label{sec:forward}

Different versions of the main or child documents
using compilation flags as described in \secref{sec:flags}
can be (permanently) stored in different files
for convenient compilation, viewing and distribution.
To this end, the package defines a command
to pass on compilation to a different file:

%%%%%%%%%%%%%%%%%%%%%%%%%%%%%%%%%%%%%%%%
\DescribeMacro{\childdocforward}
The command |\childdocforward| redirects processing to
another source file:
%
\begin{center}
\begin{tabular}{l}
|\input{childdoc.def}|\\
|\childdocforward[|\textit{main}|]{|\textit{dest}|}|\\
\end{tabular}
\end{center}
%
The argument \textit{dest} is the destination file
(without extension).
It should be the main file or one of the child files.
Note that further \textsf{childdoc} directives
such as |\childdocof| and |\childdocforward|
in the indicated file will be processed in this form.
The optional argument \textit{main}
passes on directly to the main file \textit{main}
while pretending to compile the child \textit{dest}.
This form behaves as if \textit{dest}
issues |\childdocof{|\textit{main}|}| right away,
and no further \textsf{childdoc} directives will be processed.

%%%%%%%%%%%%%%%%%%%%%%%%%%%%%%%%%%%%%%%%
\DescribeMacro{\...prefix}
In the alternative form |\childdocforwardprefix|,
%
\begin{center}
\begin{tabular}{l}
|\input{childdoc.def}|\\
|\childdocforwardprefix[|\textit{main}|]{|\textit{prefix}|}{|\textit{dest}|}|
\end{tabular}
\end{center}
%
the destination file is determined by a pattern
depending on the current file:
To make this work, the current file must be called
`{\textit{prefix}\hspace{0.2em}\textit{suffix}}'
with \textit{prefix} matching precisely the argument.
Processing is then passed on to the file
`{\textit{dest}\hspace{0.2em}\textit{suffix}}'.
Surely, the same effect is achieved by
directly specifying the
argument `{\textit{dest}\hspace{0.2em}\textit{suffix}}'
in the first form.
However, that requires to set up a different file
for each child. With the alternative form of the command
all these files can have exactly the same content
which simplifies setting them up and maintaining them.

For example, the following file |draft.tex|
with a compilation flag |\version| as described in \secref{sec:flags}
compiles the main document as a draft:
%
\begin{center}
\begin{tabular}{l}
|\def\version{draft}|\\
|\input{childdoc.def}|\\
|\childdocforward{|\textit{main}|}|
\end{tabular}
\end{center}
%
Likewise, the following files |final|\textit{nn}|.tex|
compile the final version of the child document
|child|\textit{nn}|.tex|:
%
\begin{center}
\begin{tabular}{l}
|\def\version{final}|\\
|\input{childdoc.def}|\\
|\childdocforwardprefix{final}{child}|
\end{tabular}
\end{center}
%

Note that when several versions of a main file and/or of each child file
are to be generated, it may be convenient to set up a |Makefile| or
shell script to automatise the process.

%%%%%%%%%%%%%%%%%%%%%%%%%%%%%%%%%%%%%%%%%%%%%%%%%%%%%%%%%%%%%%%%%%%%%%%%%%%%%%%%
\subsection{Command Line Processing}
\label{sec:commandline}

The effect of redirection files can also be achieved by invoking
the \LaTeX{} compiler with a more elaborate command line.
Most conveniently this should be done as part
of a shell script or a |Makefile|.

When using \textsf{childdoc} in the main file, the following
command lines effectively perform a redirection
(note that depending on the shell being used,
backslashes may have to be doubled: `|\|' $\to$ `|\\|'):
%
\begin{center}
|... -jobname "|\textit{target}|" |\\|"|[\textit{flags}]%
|\input{childdoc.def}\childdocforward[|\textit{main}|]{|\textit{dest}|}"|
\end{center}
%
Here \textit{target} is the name of the output file,
\textit{main} is the name of the main file
and \textit{dest} is the name of the main or child file to be processed
(all filenames without extensions).
The optional argument \textit{main} can be omitted
if \textit{main} matches \textit{dest}.
Optionally, compilation \textit{flags} can be defined via |\def| commands.
This command line makes the \TeX{} engine believe
it is compiling the file \textit{target}
whose content is specified as the latter parameter.
The provided code then forwards the processing to
\textit{main} or \textit{dest} as described in \secref{sec:forward}.

%%%%%%%%%%%%%%%%%%%%%%%%%%%%%%%%%%%%%%%%%%%%%%%%%%%%%%%%%%%%%%%%%%%%%%%%%%%%%%%%
\subsection{Include by Input}
\label{sec:input}

Including child documents by |\include| has some restrictions by design.
Most notably, the content of a child document always occupies
its own set of pages; pages cannot be shared between child documents.
Usually, this behaviour makes perfect sense
because each child document contain an essential part of the document.
However, in some situations it may be desirable to compose
a document from a collection of parts
without having mandatory page breaks between then.
For this case, the package
provides a mechanism to include parts
by |\input| which can also be processed individually.
However, by construction this mechanism
requires manual handling of the content to be output.

%%%%%%%%%%%%%%%%%%%%%%%%%%%%%%%%%%%%%%%%
\DescribeMacro{\ifchilddocmanual}
The main file should be prepared as usual, see \secref{sec:include}.
However, the document body must make a distinction
between processing of an individual part and of the main document, e.g.:
%
\begin{center}
\begin{tabular}{l}
|\ifchilddocmanual|\\
|\input{\childdocname}|\\
|\||else|\\
\textit{document body with }|\input{|\textit{part}|}|\\
|\||fi|
\end{tabular}
\end{center}
%
The conditional |\ifchilddocmanual| is true whenever
a part to be included by |\input| is being compiled,
and the name of the part is stored in |\childdocname|.

%%%%%%%%%%%%%%%%%%%%%%%%%%%%%%%%%%%%%%%%
\DescribeMacro{\childdocby}
Each part to be included by |\input| should start with:
%
\begin{center}
\begin{tabular}{l}
|\input{childdoc.def}|\\
|\childdocby{|\textit{main}|}|\\
\end{tabular}
\end{center}
%
The directive |\childdocby| is similar to |\childdocof|
described in \secref{sec:include},
but the subsequent selection of content must be done manually.
To that end, both |\ifchilddoc| and |\ifchilddocmanual|
will be true upon processing of a part,
and the name of the part is stored in |\childdocname|.
Note that |\jobname| will be set to the filename of the current part
so that each part receives an individual |.aux| file
that does not interfere with the |.aux| file(s) of the main document.
This behaviour can be altered by the alternative form
|\childdocby[*]{|\textit{main}|}| (with a non-empty optional argument)
which uses the |.aux| file of the main document
by setting |\jobname| to \textit{main}.

%%%%%%%%%%%%%%%%%%%%%%%%%%%%%%%%%%%%%%%%%%%%%%%%%%%%%%%%%%%%%%%%%%%%%%%%%%%%%%%%
\subsection{Driver Development}
\label{sec:driver}

The \textsf{childdoc} mechanism can also be use for the development
of definition files such as \LaTeX{} styles or classes.
This case differs from the above setup with multiple parts
included by |\include| in that no |\includeonly| should be invoked.
This can be achieved by starting the include file
(before |\ProvidesPackage|) with:
%
\begin{center}
\begin{tabular}{l}
|\input{childdoc.def}|\\
|\childdocforward{|\textit{main}|}|\\
\end{tabular}
\end{center}
%
or alternatively with:
%
\begin{center}
\begin{tabular}{l}
|\input{childdoc.def}|\\
|\childdocby{|\textit{main}|}|\\
\end{tabular}
\end{center}
%
Both forms have slightly different effects as described above.
The main file is prepared as usual, see \secref{sec:include}.

%%%%%%%%%%%%%%%%%%%%%%%%%%%%%%%%%%%%%%%%%%%%%%%%%%%%%%%%%%%%%%%%%%%%%%%%%%%%%%%%
\subsection{Legacy Detection}
\label{sec:detection}

The directive |\childdocmain| in the main file can detect
whether the complete document or merely a child is to be compiled
even without using the directive |\childdocof|.
This method is deprecated because it is less robust
and there is no compelling reason to use it;
it is merely provided for backward compatibility
and it may be removed in future versions.

If the detection mechanism is to be used,
it is mandatory to correctly specify
the filename of the main file as the argument of |\childdocmain|:
%
\begin{center}
\begin{tabular}{l}
|\input{childdoc.def}|\\
|\childdocmain{|\textit{main}|}|\\
\end{tabular}
\end{center}
%
If |\jobname| does not match the argument \textit{main} of |\childdocmain|,
it is assumed that |\jobname| points to the child file to be compiled.
When using |\childdocmain| with the main file specified as argument,
it suffices to start a child file
with just |\input{|\textit{main}|}|
without loading of the package and using |\childdocof|.
If instead all processing is done
with the appropriate \textsf{childdoc} directives,
the argument of \textit{main} of |\childdocmain| can be empty.

An alternative version of the command line processing described
in \secref{sec:commandline} using the detection mechanism reads:
%
\begin{center}
|... -jobname "|\textit{target}|" "|[\textit{flags}]%
[|\def\jobname{|\textit{dest}|}|]|\input{|\textit{main}|}"|
\end{center}

%%%%%%%%%%%%%%%%%%%%%%%%%%%%%%%%%%%%%%%%%%%%%%%%%%%%%%%%%%%%%%%%%%%%%%%%%%%%%%%%
\subsection{Manual Code}
\label{sec:manual}

In case one cannot be certain whether the definitions file |childdoc.def|
is installed on the target \TeX{} distribution
and one prefers not to ship it,
it is conceivable to paste a few relevant commands into the sources.

To that end, drop all statements |\input{childdoc.def}|
and perform the replacements as outlined below.
Instead of |\childdocmain{|\textit{main}|}| add the following code
to the top of the main file:
%
\begin{center}
\begin{tabular}{l}
|\||ifdefined\childdocname\endinput\||fi\newif\ifchilddoc|\\
|\edef\childdocname{\scantokens\expandafter{\jobname\noexpand}}|\\
|\def\childdocmain{|\textit{main}|}\||ifx\childdocmain\childdocname\||else|\\
|\childdoctrue\includeonly{\childdocname}\let\jobname\childdocmain\||fi|\\
\end{tabular}
\end{center}
%
Instead of |\childdocof{|\textit{main}|}| just include the main file
at the top of each child file:
%
\begin{center}
|\input{|\textit{main}|}|
\end{center}
%
A simple redirection |\childdocforward{|\textit{dest}|}| is achieved by:
%
\begin{center}
|\def\jobname{|\textit{dest}|}\input{\jobname}|
\end{center}
%
The redirection with prefix
|\childdocforwardprefix[|\textit{prefix}|]{|\textit{dest}|}|
is accomplished by:
%
\begin{center}
\begin{tabular}{l}
|{\edef\jobname{\scantokens\expandafter{\jobname\noexpand}}|\\
|\def\redirectjob |\textit{prefix}|#1~~~{\gdef\jobname{|\textit{dest}|#1}}|\\
|\expandafter\redirectjob\jobname~~~}\input{\jobname}|
\end{tabular}
\end{center}

In an alternative approach,
child documents can be compiled by a specific command line
without additional code or specific definitions:
%
\begin{center}
|... -jobname "|\textit{target}|" "|[\textit{flags}]%
|\includeonly{|\textit{dest}|}\input{|\textit{main}|}"|
\end{center}
%

%%%%%%%%%%%%%%%%%%%%%%%%%%%%%%%%%%%%%%%%%%%%%%%%%%%%%%%%%%%%%%%%%%%%%%%%%%%%%%%%
%%%%%%%%%%%%%%%%%%%%%%%%%%%%%%%%%%%%%%%%%%%%%%%%%%%%%%%%%%%%%%%%%%%%%%%%%%%%%%%%
\section{Information}

%%%%%%%%%%%%%%%%%%%%%%%%%%%%%%%%%%%%%%%%%%%%%%%%%%%%%%%%%%%%%%%%%%%%%%%%%%%%%%%%
\subsection{Copyright}

Copyright \copyright{} 2017--2018 Niklas Beisert

This work may be distributed and/or modified under the
conditions of the \LaTeX{} Project Public License, either version 1.3
of this license or (at your option) any later version.
The latest version of this license is in
  \url{http://www.latex-project.org/lppl.txt}
and version 1.3 or later is part of all distributions of \LaTeX{}
version 2005/12/01 or later.

This work has the LPPL maintenance status `maintained'.

The Current Maintainer of this work is Niklas Beisert.

This work consists of the files |README.txt|, |childdoc.ins| and |childdoc.dtx|
as well as the derived files |childdoc.def|, |cdocsamp.tex|
with |cdocsch1.tex|, |cdocsch2.tex|, |cdocspt3.tex|, |cdocspt4.tex|,
|cdocsdrf.tex|, |cdocsfn1.tex|, |cdocsfn2.tex|
as well as |childdoc.pdf|.

%%%%%%%%%%%%%%%%%%%%%%%%%%%%%%%%%%%%%%%%%%%%%%%%%%%%%%%%%%%%%%%%%%%%%%%%%%%%%%%%
\subsection{Files and Installation}

The package consists of the files:
%
\begin{center}
\begin{tabular}{ll}
    |README.txt|   & readme file \\
    |childdoc.ins| & installation file \\
    |childdoc.dtx| & source file \\
    |childdoc.def| & definition file \\
    |cdocsamp.tex| & sample main file \\
    |cdocsch1.tex| & sample include file \\
    |cdocsch2.tex| & sample include file \\
    |cdocspt3.tex| & sample part file \\
    |cdocspt4.tex| & sample part file \\
    |cdocsdrf.tex| & sample redirection file \\
    |cdocsfn1.tex| & sample redirection file \\
    |cdocsfn2.tex| & sample redirection file \\
    |childdoc.pdf| & manual
\end{tabular}
\end{center}
%
The distribution consists of the files
|README.txt|, |childdoc.ins| and |childdoc.dtx|.
%
\begin{itemize}
\item
Run (pdf)\LaTeX{} on |childdoc.dtx|
to compile the manual |childdoc.pdf| (this file).
\item
Run \LaTeX{} on |childdoc.ins| to create the definitions file |childdoc.def|
and the sample |cdocsamp.tex| with include files
|cdocsch1.tex|, |cdocsch2.tex|, |cdocspt3.tex|, |cdocspt4.tex|,
|cdocsdrf.tex|, |cdocsfn1.tex|, |cdocsfn2.tex|.
Then copy the file |childdoc.def| to an appropriate directory of your \LaTeX{}
distribution, e.g.\ \textit{texmf-root}|/tex/latex/childdoc|.
\end{itemize}

%%%%%%%%%%%%%%%%%%%%%%%%%%%%%%%%%%%%%%%%%%%%%%%%%%%%%%%%%%%%%%%%%%%%%%%%%%%%%%%%
\subsection{Related CTAN Packages}

There are several other packages which offer a similar functionality:
%
\begin{itemize}
\item
The packages
\href{http://ctan.org/pkg/docmute}{\textsf{docmute}},
\href{http://ctan.org/pkg/includex}{\textsf{includex}} and
\href{http://ctan.org/pkg/standalone}{\textsf{standalone}}
provide commands to include only the document body of
a child file thus allowing both files to be compiled individually.
\item
The packages \href{http://ctan.org/pkg/subdocs}{\textsf{subdocs}}
and \href{http://ctan.org/pkg/subfiles}{\textsf{subfiles}}
provide structures in which the main and child documents can be
encapsulated and allowing them to be compiled individually.
The inclusion mechanism is different from the conventional |\include|.
\item
The package \href{http://ctan.org/pkg/combine}{\textsf{combine}}
is an elaborate solution to combine several documents into one.
\end{itemize}
%
See also the CTAN topic \href{http://ctan.org/topic/subdocs}{\textsf{subdocs}}
for further related packages.
The present package differs from the above solutions in that
a document structure constructed with the conventional |\include| mechanism
just needs two extra commands at the top of every file
such that all constituent files can be compiled individually.

%%%%%%%%%%%%%%%%%%%%%%%%%%%%%%%%%%%%%%%%%%%%%%%%%%%%%%%%%%%%%%%%%%%%%%%%%%%%%%%%
%\subsection{Feature Suggestions}
%
%The following is a list of features which may be useful for future
%versions of this package:
%%
%\begin{itemize}
%\item
%\ldots
%\end{itemize}

%%%%%%%%%%%%%%%%%%%%%%%%%%%%%%%%%%%%%%%%%%%%%%%%%%%%%%%%%%%%%%%%%%%%%%%%%%%%%%%%
\subsection{Revision History}

%%%%%%%%%%%%%%%%%%%%%%%%%%%%%%%%%%%%%%%%
\paragraph{v2.0:} 2018/12/30

\begin{itemize}
\item
immediate forward processing
\item
added |\childdocby| mechanism
\item
manual restructured
\end{itemize}

%%%%%%%%%%%%%%%%%%%%%%%%%%%%%%%%%%%%%%%%
\paragraph{v1.6:} 2018/01/17

\begin{itemize}
\item
application for development of include files
\item
corrections to manual
\end{itemize}

%%%%%%%%%%%%%%%%%%%%%%%%%%%%%%%%%%%%%%%%
\paragraph{v1.5:} 2017/05/21

\begin{itemize}
\item
more complete structuring introduced
\item
|\childdocof| introduced
\item
|\childdoc| renamed to |\childdocmain|
\item
|\childredirect| renamed to |\childdocforward| and |\childdocforwardprefix|
and functionality expanded
\end{itemize}

%%%%%%%%%%%%%%%%%%%%%%%%%%%%%%%%%%%%%%%%
\paragraph{v1.0:} 2017/04/27

\begin{itemize}
\item
manual and install package
\item
first version published on CTAN
\end{itemize}

%%%%%%%%%%%%%%%%%%%%%%%%%%%%%%%%%%%%%%%%
\paragraph{v0.6:} 2017/04/26

\begin{itemize}
\item
redirection mechanism added
\end{itemize}

%%%%%%%%%%%%%%%%%%%%%%%%%%%%%%%%%%%%%%%%
\paragraph{v0.5:} 2017/04/26

\begin{itemize}
\item
functionality in definition file
\end{itemize}


%%%%%%%%%%%%%%%%%%%%%%%%%%%%%%%%%%%%%%%%%%%%%%%%%%%%%%%%%%%%%%%%%%%%%%%%%%%%%%%%
%%%%%%%%%%%%%%%%%%%%%%%%%%%%%%%%%%%%%%%%%%%%%%%%%%%%%%%%%%%%%%%%%%%%%%%%%%%%%%%%
%%%%%%%%%%%%%%%%%%%%%%%%%%%%%%%%%%%%%%%%%%%%%%%%%%%%%%%%%%%%%%%%%%%%%%%%%%%%%%%%
\appendix

\settowidth\MacroIndent{\rmfamily\scriptsize 000\ }

 \DocInput{childdoc.dtx}

\end{document}
%</driver>
% \fi
%
% %%%%%%%%%%%%%%%%%%%%%%%%%%%%%%%%%%%%%%%%%%%%%%%%%%%%%%%%%%%%%%%%%%%%%%%%%%%%%%
% %%%%%%%%%%%%%%%%%%%%%%%%%%%%%%%%%%%%%%%%%%%%%%%%%%%%%%%%%%%%%%%%%%%%%%%%%%%%%%
% \section{Sample}
%\iffalse
%<*samplemain>
%\fi
%
% The following presents a sample document
% with two chapters, two parts, a title page,
% a compile flag as well as three forwarding files to set the flag.
% It consists of eight |.tex| files:
% \begin{center}
% \begin{tabular}{ll}
% |cdocsamp.tex|&main file\\
% |cdocsch1.tex|&include file for chapter 1\\
% |cdocsch2.tex|&include file for chapter 2\\
% |cdocspt3.tex|&include file for part 3\\
% |cdocspt4.tex|&include file for part 4\\
% |cdocsdrf.tex|&forwarding file for main file in draft mode\\
% |cdocsfi1.tex|&forwarding file for final version of chapter 1\\
% |cdocsfi2.tex|&forwarding file for final version of chapter 2\\
% \end{tabular}
% \end{center}
% Each of the eight files can be compiled directly by the \LaTeX{} compiler.
%
% %%%%%%%%%%%%%%%%%%%%%%%%%%%%%%%%%%%%%%
% \paragraph{Main File.}
%
% The main file is called |cdocsamp.tex|.
%
% Load the \textsf{childdoc} definitions and
% declare the filename for the main document:
%    \begin{macrocode}
\input{childdoc.def}
\childdocmain{}
%    \end{macrocode}

% Optional override for |\version| flag:
%    \begin{macrocode}
%%\ifchilddoc\else\providecommand{\version}{draft}\fi
%    \end{macrocode}

% Define the default values for the |\version| flag
% (|final| for the main file and |draft| for childs):
%    \begin{macrocode}
\ifchilddoc
\providecommand{\version}{draft}
\else
\providecommand{\version}{final}
\fi
%    \end{macrocode}

% Load the standard document class:
%    \begin{macrocode}
\documentclass[12pt]{article}
%    \end{macrocode}

% Start the document body:
%    \begin{macrocode}
\begin{document}
%    \end{macrocode}

% Declare a title page.
% Print title, part of document being processed and version flag:
%    \begin{macrocode}
\addtocounter{page}{-1}
\begin{center}
{\LARGE\bfseries{}childdoc example\par}
\vspace{1cm}
\ifchilddoc
\ifchilddocmanual part\else chapter\fi:
`\childdocname' of `\childdocjob'\par
\else
main document: `\childdocjob'\par
\fi
version: \version\par
\end{center}
\newpage
%    \end{macrocode}

% Manually include selected file,
% otherwise process as usual:
%    \begin{macrocode}
\ifchilddocmanual
\section*{part `\childdocname'}
\input{\childdocname}
\else
%    \end{macrocode}

% Include the two chapters:
%    \begin{macrocode}
\include{cdocsch1}
\include{cdocsch2}
%    \end{macrocode}

% Include the two parts unless only chapters should be displayed:
%    \begin{macrocode}
\ifchilddoc\else
\section{part three}
\input{cdocspt3}
\section{part four}
\input{cdocspt4}
\fi
%    \end{macrocode}

% Process as usual until here:
%    \begin{macrocode}
\fi
%    \end{macrocode}

% End of document body:
%    \begin{macrocode}
\end{document}
%    \end{macrocode}
%\iffalse
%</samplemain>
%\fi
%
% %%%%%%%%%%%%%%%%%%%%%%%%%%%%%%%%%%%%%%
% \paragraph{Chapter Include Files.}
%
% The include files are called |cdocsch1.tex| and |cdocsch2.tex|.
%
%\iffalse
%<*samplechap1|samplechap2>
%\fi

% Optional override for |\version| flag:
%    \begin{macrocode}
%%\providecommand{\version}{final}
%    \end{macrocode}

% Include the main document:
%    \begin{macrocode}
\input{childdoc.def}
\childdocof{cdocsamp}
%    \end{macrocode}

%\iffalse
%</samplechap1|samplechap2>
%\fi
%
%\iffalse
%<*samplechap1>
%\fi
% Some text for chapter 1:
%    \begin{macrocode}
\section{one}
some text in chapter one
%    \end{macrocode}

%\iffalse
%</samplechap1>
%\fi
% Some text for chapter 2:
%\iffalse
%<*samplechap2>
%\fi
%    \begin{macrocode}
\section{two}
more text in chapter two
%    \end{macrocode}

%\iffalse
%</samplechap2>
%\fi
%
% %%%%%%%%%%%%%%%%%%%%%%%%%%%%%%%%%%%%%%
% \paragraph{Part Include Files.}
%
% The include files are called |cdocspt3.tex| and |cdocspt4.tex|.
%
%\iffalse
%<*samplepart3|samplepart4>
%\fi

% Optional override for |\version| flag:
%    \begin{macrocode}
%%\providecommand{\version}{final}
%    \end{macrocode}

% Include the main document:
%    \begin{macrocode}
\input{childdoc.def}
\childdocby{cdocsamp}
%    \end{macrocode}

%\iffalse
%</samplepart3|samplepart4>
%\fi
%
%\iffalse
%<*samplepart3>
%\fi
% Some text for part 3:
%    \begin{macrocode}
some text in part three
%    \end{macrocode}

%\iffalse
%</samplepart3>
%\fi
% Some text for part 4:
%\iffalse
%<*samplepart4>
%\fi
%    \begin{macrocode}
more text in part four
%    \end{macrocode}

%\iffalse
%</samplepart4>
%\fi
%
% %%%%%%%%%%%%%%%%%%%%%%%%%%%%%%%%%%%%%%
% \paragraph{Forwarding for a Complete Draft.}
%
% The following forwarding file |cdocsdrf.tex|
% compiles the main document in draft mode:
%\iffalse
%<*sampledraft>
%\fi
%    \begin{macrocode}
\def\version{draft}
\input{childdoc.def}
\childdocforward{cdocsamp}
%    \end{macrocode}

%\iffalse
%</sampledraft>
%\fi
%
% %%%%%%%%%%%%%%%%%%%%%%%%%%%%%%%%%%%%%%
% \paragraph{Forwarding for Final Version of the Chapters.}
%
% The following forwarding files |cdocsfn1.tex| and |cdocsfn2.tex|
% (with identical content)
% compile the final versions of the child documents
% |cdocsch1.tex| and |cdocsch2.tex|, respectively:
%\iffalse
%<*samplefinal>
%\fi
%    \begin{macrocode}
\def\version{final}
\input{childdoc.def}
\childdocforwardprefix[cdocsamp]{cdocsfn}{cdocsch}
%    \end{macrocode}

%\iffalse
%</samplefinal>
%\fi
%
% %%%%%%%%%%%%%%%%%%%%%%%%%%%%%%%%%%%%%%
% \paragraph{Command Line Processing.}
%
% The following three command lines generate the output files
% |cdocscld|, |cdocscl1| and |cdocscl2|
% which should be identical to
% |cdocsdrf|, |cdocsch1| and |cdocsfn2|, respectively:
% \begin{center}
% \begin{tabular}{l}
% |latex -jobname cdocscld \|\\
% |  "\def\version{draft}\input{childdoc.def}\childdocforward{cdocsamp}"|\\
% |latex -jobname cdocscl1 \|\\
% |  "\input{childdoc.def}\childdocforward[cdocsamp]{cdocsch1}"|\\
% |latex -jobname cdocscl2 \|\\
% |  "\def\version{final}\input{childdoc.def}\childdocforward{cdocsch2}"|
% \end{tabular}
% \end{center}
% Note that the trailing backslash on each first line
% merely continues the input to the second line
% (for convenient cut ant paste).
% Furthermore, the command |latex| can be replaced by any
% of its alternative versions such as |pdflatex|.
%
% %%%%%%%%%%%%%%%%%%%%%%%%%%%%%%%%%%%%%%%%%%%%%%%%%%%%%%%%%%%%%%%%%%%%%%%%%%%%%%
% %%%%%%%%%%%%%%%%%%%%%%%%%%%%%%%%%%%%%%%%%%%%%%%%%%%%%%%%%%%%%%%%%%%%%%%%%%%%%%
% \section{Implementation}
%\iffalse
%<*package>
%\fi
%
% This section describes the definitions file |childdoc.def|.

% The definitions cannot be loaded using |\usepackage| or |\RequirePackage|
% which has a mechanism to prevent loading a style file more than once.
% When loading the definitions by means of |\input|
% multiple instances have to be prevented manually:
%\iffalse
%This code needs to be before the `\ProvidesFile' directive
%which is defined at the beginning of this file.
%Therefore it is also placed there and commented out here.
%</package>
%<*discard>
%\fi
%    \begin{macrocode}
\ifdefined\childdocmain\endinput\fi
%    \end{macrocode}
%\iffalse
%</discard>
%<*package>
%\fi
%
% \macro{\ifchilddoc}
% \macro{\ifchilddocmanual}
% The conditional |\ifchilddoc| tells whether a
% child (true) or main (false) document is being compiled.
% The conditional |\ifchilddocmanual| tells whether
% the |\includeonly| mechanism is used (false) or
% the selection of child files must be performed manually (true).
% The definitions initialise to false:
%    \begin{macrocode}
\newif\ifchilddoc
\newif\ifchilddocmanual
%    \end{macrocode}

% \macro{\childdocname}
% \macro{\childdocjob}
% The macro |\childdocname| stores the name of the main document
% to be compiled. The macro |\childdocjob| stores the name of
% the document on which the \LaTeX{} compiler was originally invoked.
% The content of |\jobname| cannot be compared
% to filenames specified in the source due to different catcodes.
% The following code rescans |\jobname|, stores the result
% in |\childdocname| and saves a copy in |\childdocjob|:
%    \begin{macrocode}
\edef\childdocname{\scantokens\expandafter{\jobname\noexpand}}
\let\childdocjob\childdocname
%    \end{macrocode}

% \macro{\childdocdisable}
% The macro |\childdocdisable| prevents the main file
% from being processed more than once.
% At this stage, the main document command |\childdocmain|
% is assumed to be called once again where it should do nothing.
% Any subsequent call to it should prevent
% a secondary processing of the main document
% It overwrites the forwarding commands
% |\childdocof| and |\childdocforward|
% with empty macros to prevent further inclusions of the main document:
%    \begin{macrocode}
\newcommand{\childdocdisable}
{
  \renewcommand{\childdocmain}[1]{\renewcommand{\childdocmain}[1]{\endinput}}
  \renewcommand{\childdocof}[1]{}
  \renewcommand{\childdocby}[2][]{}
  \renewcommand{\childdocforward}[2][]{}
  \renewcommand{\childdocdisable}{}
}
%    \end{macrocode}

% \macro{\childdocmain}
% The macro |\childdocmain| is to be called at the top of the main file
% with nothing or the main filename (without extension) as argument.
% First, it breaks loops.
% If the argument is not empty and does not match |\childdocname|
% (which is set by the first inclusion of |childdoc.def|),
% |\ifchilddoc| is set to true, |\includeonly| is applied to the child file
% and |\jobname| is set to the main file
% (for proper handling of |.aux| files):
%    \begin{macrocode}
\newcommand{\childdocmain}[1]
{
  \childdocdisable\childdocmain{}
  \if?#1?\else
    \begingroup
      \def\childdoctmp{#1}
      \ifx\childdoctmp\childdocname
        \def\childdoctmp{}
      \else
        \def\childdoctmp
        {
          \childdoctrue
          \includeonly{\childdocname}
          \def\childdocjob{#1}
          \def\jobname{#1}
        }
      \fi
      \expandafter
    \endgroup
    \childdoctmp
  \fi
}
%    \end{macrocode}

% \macro{\childdocof}
% The command |\childdocof| redirects
% compilation to the main file |#1|.
%    \begin{macrocode}
\newcommand{\childdocof}[1]
{
  \childdocdisable
  \childdoctrue
  \includeonly{\childdocname}
  \def\jobname{#1}
  \def\childdocjob{#1}
  \input{#1}
}
%    \end{macrocode}

% \macro{\childdocby}
% The command |\childdocby| ....
%    \begin{macrocode}
\newcommand{\childdocby}[2][]
{
  \childdocdisable
  \childdoctrue
  \childdocmanualtrue
  \if?#1?\else
    \def\jobname{#2}
  \fi
  \def\childdocjob{#2}
  \input{#2}
  \endinput
}
%    \end{macrocode}

% \macro{\childdocforward}
% The command |\childdocforward| redirects
% compilation to the main file or
% (if the optional argument is given) a child file.
% Parameters are set as if the main file
% or a child file starting with |\childdocof| was compiled.
% Then compilation is handed over to the main file:
%    \begin{macrocode}
\newcommand{\childdocforward}[2][]
{
  \begingroup
    \if?#1?
      \def\childdoctmp
      {
        \def\childdocname{#2}
        \def\childdocjob{#2}
        \def\jobname{#2}
        \input{#2}
        \endinput
      }
    \else
      \def\childdoctmp
      {
        \childdocdisable
        \def\childdocname{#2}
        \childdoctrue
        \includeonly{#2}
        \def\childdocjob{#1}
        \def\jobname{#1}
        \input{#1}
        \endinput
      }
    \fi
    \expandafter
  \endgroup
  \childdoctmp
}
%    \end{macrocode}

% \macro{\childdocforwardprefix}
% The command |\childdocforwardprefix| redirects
% compilation to the main or a child file by means of a pattern.
% The prefix |#1| in the current filename is replaced by |#2|
% and the suffix of the current filename is kept
% (it is assumed that the filename does not contain the substring `|~~~|'
% which is used as a delimiter).
% Compilation is handed over to the new file by |\childdocforward|:
%    \begin{macrocode}
\newcommand{\childdocforwardprefix}[3][]
{
  \begingroup
    \def\childdocextract #2##1~~~{\def\childdoctmp{\childdocforward[#1]{#3##1}}}
    \expandafter\childdocextract\childdocname~~~
    \expandafter
  \endgroup
  \childdoctmp
}
%    \end{macrocode}

% \macro{\childdoc}
% The deprecated macro |\childdoc| is a legacy version of |\childdocmain|:
%    \begin{macrocode}
\newcommand{\childdoc}{\childdocmain}
%    \end{macrocode}

% \macro{\childdocredirect}
% The deprecated macro |\childdocredirect| is a legacy version
% of |\childdocforward| and |\childdocforwardprefix|:
%    \begin{macrocode}
\newcommand{\childdocredirect}[2][]
{
  \begingroup
    \if?#1?
      \def\childdoctmp{\childdocforward{#2}}
    \else
      \def\childdoctmp{\childdocforwardprefix{#1}{#2}}
    \fi
    \expandafter
  \endgroup
  \childdoctmp
}
%    \end{macrocode}

%\iffalse
%</package>
%\fi
%
\endinput

\childdocby{cdocsamp}
%    \end{macrocode}

%\iffalse
%</samplepart3|samplepart4>
%\fi
%
%\iffalse
%<*samplepart3>
%\fi
% Some text for part 3:
%    \begin{macrocode}
some text in part three
%    \end{macrocode}

%\iffalse
%</samplepart3>
%\fi
% Some text for part 4:
%\iffalse
%<*samplepart4>
%\fi
%    \begin{macrocode}
more text in part four
%    \end{macrocode}

%\iffalse
%</samplepart4>
%\fi
%
% %%%%%%%%%%%%%%%%%%%%%%%%%%%%%%%%%%%%%%
% \paragraph{Forwarding for a Complete Draft.}
%
% The following forwarding file |cdocsdrf.tex|
% compiles the main document in draft mode:
%\iffalse
%<*sampledraft>
%\fi
%    \begin{macrocode}
\def\version{draft}
% \iffalse
%
% childdoc.dtx Copyright (C) 2017-2018 Niklas Beisert
%
% This work may be distributed and/or modified under the
% conditions of the LaTeX Project Public License, either version 1.3
% of this license or (at your option) any later version.
% The latest version of this license is in
%   http://www.latex-project.org/lppl.txt
% and version 1.3 or later is part of all distributions of LaTeX
% version 2005/12/01 or later.
%
% This work has the LPPL maintenance status `maintained'.
%
% The Current Maintainer of this work is Niklas Beisert.
%
% This work consists of the files childdoc.dtx and childdoc.ins
% and the derived files childdoc.def and cdocsamp.tex with
% cdocsch1.tex, cdocsch2.tex, cdocsdrf.tex, cdocsfn1.tex, cdocsfn2.tex.
%
%<package>\ifdefined\childdocmain\endinput\fi
%<package>\ProvidesFile{childdoc.def}[2018/12/30 v2.0 child document driver]
%<samplemain>\ProvidesFile{cdocsamp.tex}[2018/12/30 v2.0 sample for childdoc]
%<*driver>
%\ProvidesFile{childdoc.drv}[2018/12/30 v2.0 childdoc reference manual file]
\PassOptionsToClass{10pt,a4paper}{article}
\documentclass{ltxdoc}

\usepackage[margin=35mm]{geometry}
\usepackage{hyperref}
\usepackage{hyperxmp}
\usepackage[usenames]{color}

\hypersetup{colorlinks=true}
\hypersetup{pdfstartview=FitH}
\hypersetup{pdfpagemode=UseNone}
\hypersetup{pdfsource={}}
\hypersetup{pdflang={en-UK}}
\hypersetup{pdfcopyright={Copyright 2017-2018 Niklas Beisert.
  This work may be distributed and/or modified under the
  conditions of the LaTeX Project Public License, either version 1.3
  of this license or (at your option) any later version.}}
\hypersetup{pdflicenseurl={http://www.latex-project.org/lppl.txt}}
\hypersetup{pdfcontactaddress={ETH Zurich, ITP, HIT K,
  Wolfgang-Pauli-Strasse 27}}
\hypersetup{pdfcontactpostcode={8093}}
\hypersetup{pdfcontactcity={Zurich}}
\hypersetup{pdfcontactcountry={Switzerland}}
\hypersetup{pdfcontactemail={nbeisert@itp.phys.ethz.ch}}
\hypersetup{pdfcontacturl={http://people.phys.ethz.ch/\xmptilde nbeisert/}}

\newcommand{\secref}[1]{\hyperref[#1]{section \ref*{#1}}}

\parskip1ex
\parindent0pt
\let\olditemize\itemize
\def\itemize{\olditemize\parskip0pt}

\begin{document}

\title{The \textsf{childdoc} Package}
\hypersetup{pdftitle={The childdoc Package}}
\author{Niklas Beisert\\[2ex]
  Institut f\"ur Theoretische Physik\\
  Eidgen\"ossische Technische Hochschule Z\"urich\\
  Wolfgang-Pauli-Strasse 27, 8093 Z\"urich, Switzerland\\[1ex]
  \href{mailto:nbeisert@itp.phys.ethz.ch}
  {\texttt{nbeisert@itp.phys.ethz.ch}}}
\hypersetup{pdfauthor={Niklas Beisert}}
\hypersetup{pdfsubject={Manual for the LaTeX2e Package childdoc}}
\date{30 December 2018, \textsf{v2.0}}
\maketitle

\begin{abstract}\noindent
\textsf{childdoc} is a \LaTeXe{} package
that enables the direct compilation
of document sections included by |\include|
to individual files.
\end{abstract}

\begingroup
\parskip0ex
\tableofcontents
\endgroup

%%%%%%%%%%%%%%%%%%%%%%%%%%%%%%%%%%%%%%%%%%%%%%%%%%%%%%%%%%%%%%%%%%%%%%%%%%%%%%%%
%%%%%%%%%%%%%%%%%%%%%%%%%%%%%%%%%%%%%%%%%%%%%%%%%%%%%%%%%%%%%%%%%%%%%%%%%%%%%%%%
\section{Introduction}

\LaTeX{} provides a mechanism to structure a large document (such as a book)
into a main file and several child files (containing the chapters)
using the |\include| command.
This mechanism is beneficial for documents
which span hundreds of pages in order to
make the source file(s) more manageable.
Moreover, compilation can be restricted to
selected child files by means of the |\includeonly| command.
The latter feature can be used to reduce the compilation time while editing
(this was significantly more useful in the earlier days of \LaTeX{})
or to generate a smaller document which is easier to navigate.
Another application of |\includeonly| is to generate
documents consisting of selected parts of the complete document.

However, there are a few drawbacks of the plain |\include| mechanism:
\begin{itemize}
\item
The child files cannot be compiled on their own,
they can only be compiled via the main file.
A naive editing environment
(such as a text editor with an option
to have the current file processed by \LaTeX)
may require one to switch to the main file before compiling;
attempting to compile the child file produces errors.
\item
The main file must be modified (each time)
to adjust the |\includeonly| command
to the present needs. This easily leaves the main file in a messy state.
\item
The generated document will always carry the filename
of the main document. This is inconvenient if
several child files are to be compiled and
to be kept for distribution.
\end{itemize}

The present package provides a simple interface
to make child files individually compilable by \LaTeX{}.
Compiling a child file then has the same effect as compiling
the main file with an |\includeonly| command
to select the appropriate child.
Moreover the generated document will carry the name of the child
rather than the main file.
This resolves all three above issues.

This feature is meant to make the editing of books,
thesis documents and lecture notes somewhat more convenient.
However, the package can also be used efficiently for
composing a series of documents (such as exercise sheets)
which are typically distributed individually.
It then assists the author in generating the individual documents
(potentially in different versions)
as well as a document containing the collected series.
Another application is in developing style files
or other kinds of included material
where compilation of the style file could redirect
to a sample or test file.

%%%%%%%%%%%%%%%%%%%%%%%%%%%%%%%%%%%%%%%%%%%%%%%%%%%%%%%%%%%%%%%%%%%%%%%%%%%%%%%%
%%%%%%%%%%%%%%%%%%%%%%%%%%%%%%%%%%%%%%%%%%%%%%%%%%%%%%%%%%%%%%%%%%%%%%%%%%%%%%%%
\section{Usage}

First of all, the package \textsf{childdoc} is \emph{not} a standard
\LaTeXe{} |.sty| style file! Therefore it needs to be invoked in
a non-standard way.

%%%%%%%%%%%%%%%%%%%%%%%%%%%%%%%%%%%%%%%%%%%%%%%%%%%%%%%%%%%%%%%%%%%%%%%%%%%%%%%%
\subsection{Included Files}
\label{sec:include}

%%%%%%%%%%%%%%%%%%%%%%%%%%%%%%%%%%%%%%%%
\DescribeMacro{\childdocmain}
To use the package, add the commands
\begin{center}
\begin{tabular}{l}
|\input{childdoc.def}|\\
|\childdocmain{}|\\
\end{tabular}
\end{center}
at the very top of the main \LaTeX{} file,
in particular \emph{before} the |\documentclass| statement!
The argument of |\childdocmain| should be left empty
(but it must be present).

%%%%%%%%%%%%%%%%%%%%%%%%%%%%%%%%%%%%%%%%
\DescribeMacro{\childdocof}
Furthermore, add the commands
\begin{center}
\begin{tabular}{l}
|\input{childdoc.def}|\\
|\childdocof{|\textit{main}|}|\\
\end{tabular}
\end{center}
at the top of every child file \textit{child}
which is included by |\include{|\textit{child}|}|
from within the main file
(or at least for those files to be compiled individually).
The argument \textit{main} must be the filename of the main file.

There are a couple of
considerations in setting up the main and child documents:

%%%%%%%%%%%%%%%%%%%%%%%%%%%%%%%%%%%%%%%%
\paragraph{Restrictions.}

Please note the following restrictions:
\begin{itemize}
\item
|\childdocmain| must be called with one argument \textit{main}
to ensure compatibility with earlier version of the package.
It must either be empty (|\childdocmain{}|)
or precisely match the filename of the main file in which it is specified.
See \secref{sec:detection} for further information.
\item
The filename \textit{main} must be specified without the |.tex| extension.
\item
The filename \textit{main} is case sensitive
(even in case-insensitive file systems)
due to internal string comparison.
\item
The argument \textit{main} should be fully expanded, it cannot be a macro.
\item
Subdirectories and special characters should be avoided in filenames.
\item
The command |\childdocmain{|\textit{main}|}| must be followed by a whitespace.
It should not be followed immediately by another command
or by a comment mark `|%|'.
This is because the \TeX{} parser reads the token immediately following
the argument of |\childdocmain| and puts it
at the beginning of every child section;
however, a white\-space is ignored.
\end{itemize}

%%%%%%%%%%%%%%%%%%%%%%%%%%%%%%%%%%%%%%%%
\paragraph{Content of Main File.}

It is advisable to place all content in the child files included by |\include|.
Any output contained in the main file will appear in all child documents
unless suppressed manually;
it cannot be suppressed automatically by the |\includeonly| directive
and thus should normally be avoided.
A method to include some content in the main file
by means of conditional processing is described in \secref{sec:conditional}.

%%%%%%%%%%%%%%%%%%%%%%%%%%%%%%%%%%%%%%%%
\paragraph{Page Numbering.}

When only a part of the document is compiled,
the appropriate numbering of pages
(as well as other status parameters)
is determined from the |.aux| files.
The latter contain information from previous passes.
However this information needs to propagate through
all intermediate child documents.
Therefore the page numbering in child documents may well
be inconsistent until the complete document is compiled at least once.

A useful (if unconventional) way to always ensure a consistent
page numbering is to restart the numbering in each child document
and denote the pages by `\textit{child}|.|\textit{page}'
where \textit{child} represents the chapter/section number of the child file.
This can be achieved by the command
|\numberwithin{page}{|\textit{child}|}|
of the \textsf{amsmath} package
where \textit{child} can be |chapter| or |section|
depending on the chosen structuring.
Alternatively, one can modify the macro |\thepage| appropriately
and reset the counter |page| at the start of each child file.

%%%%%%%%%%%%%%%%%%%%%%%%%%%%%%%%%%%%%%%%%%%%%%%%%%%%%%%%%%%%%%%%%%%%%%%%%%%%%%%%
\subsection{Conditional Processing}
\label{sec:conditional}

The package provides a mechanism to compile different versions
of a document. To customise the versions further some conditional processing
can come in handy to distinguish which version is being compiled.
The package provides two macros to describe the compilation context:

%%%%%%%%%%%%%%%%%%%%%%%%%%%%%%%%%%%%%%%%
\DescribeMacro{\ifchilddoc}
The conditional |\ifchilddoc| distinguishes between the compilation of
child documents and the main document:
%
\begin{center}
|\ifchilddoc |\textit{child-code}| |[|\||else |\textit{main-code}]| \||fi|
\end{center}

%%%%%%%%%%%%%%%%%%%%%%%%%%%%%%%%%%%%%%%%
\DescribeMacro{\childdocname}
\DescribeMacro{\childdocjob}
The macro |\childdocname| contains the filename (without extension)
of the main or child file being processed.
Note that |\childdocjob| will always contain the name of the main file.

%%%%%%%%%%%%%%%%%%%%%%%%%%%%%%%%%%%%%%%%
\paragraph{Title Page.}

Conditional processing can be used to include a title or banner page
in the main document when proper precautions are taken.
Importantly, the code in the main file should ensure that the page counter
(as well as other status parameters which are stored in the |.aux| files)
takes the same value after the conditional processing.
Otherwise the page numbers may take divergent values
depending on which part is compiled.

For example, a title page could be declared by:
%
\begin{center}
\begin{tabular}{l}
|\ifchilddoc\||else|\\
|\addtocounter{page}{-1}|\\
\textit{code for title page}\\
|\newpage|\\
|\||fi|
\end{tabular}
\end{center}
%
A banner page for the child documents can be generated by:
%
\begin{center}
\begin{tabular}{l}
|\ifchilddoc|\\
|\addtocounter{page}{-1}|\\
\textit{code for banner page}\\
|\newpage|\\
|\||fi|
\end{tabular}
\end{center}
%
Here one could write a message such as:
\begin{center}
|This is the part \childdocname{} of \childdocjob{}.|
\end{center}

%%%%%%%%%%%%%%%%%%%%%%%%%%%%%%%%%%%%%%%%%%%%%%%%%%%%%%%%%%%%%%%%%%%%%%%%%%%%%%%%
\subsection{Flags}
\label{sec:flags}

The package makes it easy to generate different versions
of the main or child documents.
To this end compilation flags can be defined
and assigned different default values.
They will be particularly useful in conjunction
with the forwarding mechanism described in \secref{sec:forward}.

For example, it may be useful to have a flag |\version|
which can be set to |draft| or |final|.
The document source will contain some conditional code
depending on the value of |\version|.
Suppose further, the flag should default to |final| for the main file
and to |draft| for child files
which is a natural assignment for editing the document.
This is achieved by placing the following code
in the preamble of the main document
(below the |\childdocmain| directive):
%
\begin{center}
\begin{tabular}{l}
|\ifchilddoc|\\
|\providecommand{\version}{draft}|\\
|\||else|\\
|\providecommand{\version}{final}|\\
|\||fi|
\end{tabular}
\end{center}
%
The definition by |\providecommand| makes sure
that previous definitions are not overwritten.
Further statements |\providecommand{\version}{...}|
can thus be added before the above code to override it.

For the main file, one might add a line
(between |\childdocmain| and the above block)
%
\begin{center}
|%\ifchilddoc\||else\providecommand{\version}{draft}\||fi|
\end{center}
%
which can be uncommented to produce a draft version.
Likewise one can add a line to the very top of a child file
(above the |\childdocof{|\textit{main}|}| directive)
%
\begin{center}
|%\providecommand{\version}{final}|
\end{center}
%
which can be uncommented to produce the final version of this child document.

%%%%%%%%%%%%%%%%%%%%%%%%%%%%%%%%%%%%%%%%%%%%%%%%%%%%%%%%%%%%%%%%%%%%%%%%%%%%%%%%
\subsection{Forwarding}
\label{sec:forward}

Different versions of the main or child documents
using compilation flags as described in \secref{sec:flags}
can be (permanently) stored in different files
for convenient compilation, viewing and distribution.
To this end, the package defines a command
to pass on compilation to a different file:

%%%%%%%%%%%%%%%%%%%%%%%%%%%%%%%%%%%%%%%%
\DescribeMacro{\childdocforward}
The command |\childdocforward| redirects processing to
another source file:
%
\begin{center}
\begin{tabular}{l}
|\input{childdoc.def}|\\
|\childdocforward[|\textit{main}|]{|\textit{dest}|}|\\
\end{tabular}
\end{center}
%
The argument \textit{dest} is the destination file
(without extension).
It should be the main file or one of the child files.
Note that further \textsf{childdoc} directives
such as |\childdocof| and |\childdocforward|
in the indicated file will be processed in this form.
The optional argument \textit{main}
passes on directly to the main file \textit{main}
while pretending to compile the child \textit{dest}.
This form behaves as if \textit{dest}
issues |\childdocof{|\textit{main}|}| right away,
and no further \textsf{childdoc} directives will be processed.

%%%%%%%%%%%%%%%%%%%%%%%%%%%%%%%%%%%%%%%%
\DescribeMacro{\...prefix}
In the alternative form |\childdocforwardprefix|,
%
\begin{center}
\begin{tabular}{l}
|\input{childdoc.def}|\\
|\childdocforwardprefix[|\textit{main}|]{|\textit{prefix}|}{|\textit{dest}|}|
\end{tabular}
\end{center}
%
the destination file is determined by a pattern
depending on the current file:
To make this work, the current file must be called
`{\textit{prefix}\hspace{0.2em}\textit{suffix}}'
with \textit{prefix} matching precisely the argument.
Processing is then passed on to the file
`{\textit{dest}\hspace{0.2em}\textit{suffix}}'.
Surely, the same effect is achieved by
directly specifying the
argument `{\textit{dest}\hspace{0.2em}\textit{suffix}}'
in the first form.
However, that requires to set up a different file
for each child. With the alternative form of the command
all these files can have exactly the same content
which simplifies setting them up and maintaining them.

For example, the following file |draft.tex|
with a compilation flag |\version| as described in \secref{sec:flags}
compiles the main document as a draft:
%
\begin{center}
\begin{tabular}{l}
|\def\version{draft}|\\
|\input{childdoc.def}|\\
|\childdocforward{|\textit{main}|}|
\end{tabular}
\end{center}
%
Likewise, the following files |final|\textit{nn}|.tex|
compile the final version of the child document
|child|\textit{nn}|.tex|:
%
\begin{center}
\begin{tabular}{l}
|\def\version{final}|\\
|\input{childdoc.def}|\\
|\childdocforwardprefix{final}{child}|
\end{tabular}
\end{center}
%

Note that when several versions of a main file and/or of each child file
are to be generated, it may be convenient to set up a |Makefile| or
shell script to automatise the process.

%%%%%%%%%%%%%%%%%%%%%%%%%%%%%%%%%%%%%%%%%%%%%%%%%%%%%%%%%%%%%%%%%%%%%%%%%%%%%%%%
\subsection{Command Line Processing}
\label{sec:commandline}

The effect of redirection files can also be achieved by invoking
the \LaTeX{} compiler with a more elaborate command line.
Most conveniently this should be done as part
of a shell script or a |Makefile|.

When using \textsf{childdoc} in the main file, the following
command lines effectively perform a redirection
(note that depending on the shell being used,
backslashes may have to be doubled: `|\|' $\to$ `|\\|'):
%
\begin{center}
|... -jobname "|\textit{target}|" |\\|"|[\textit{flags}]%
|\input{childdoc.def}\childdocforward[|\textit{main}|]{|\textit{dest}|}"|
\end{center}
%
Here \textit{target} is the name of the output file,
\textit{main} is the name of the main file
and \textit{dest} is the name of the main or child file to be processed
(all filenames without extensions).
The optional argument \textit{main} can be omitted
if \textit{main} matches \textit{dest}.
Optionally, compilation \textit{flags} can be defined via |\def| commands.
This command line makes the \TeX{} engine believe
it is compiling the file \textit{target}
whose content is specified as the latter parameter.
The provided code then forwards the processing to
\textit{main} or \textit{dest} as described in \secref{sec:forward}.

%%%%%%%%%%%%%%%%%%%%%%%%%%%%%%%%%%%%%%%%%%%%%%%%%%%%%%%%%%%%%%%%%%%%%%%%%%%%%%%%
\subsection{Include by Input}
\label{sec:input}

Including child documents by |\include| has some restrictions by design.
Most notably, the content of a child document always occupies
its own set of pages; pages cannot be shared between child documents.
Usually, this behaviour makes perfect sense
because each child document contain an essential part of the document.
However, in some situations it may be desirable to compose
a document from a collection of parts
without having mandatory page breaks between then.
For this case, the package
provides a mechanism to include parts
by |\input| which can also be processed individually.
However, by construction this mechanism
requires manual handling of the content to be output.

%%%%%%%%%%%%%%%%%%%%%%%%%%%%%%%%%%%%%%%%
\DescribeMacro{\ifchilddocmanual}
The main file should be prepared as usual, see \secref{sec:include}.
However, the document body must make a distinction
between processing of an individual part and of the main document, e.g.:
%
\begin{center}
\begin{tabular}{l}
|\ifchilddocmanual|\\
|\input{\childdocname}|\\
|\||else|\\
\textit{document body with }|\input{|\textit{part}|}|\\
|\||fi|
\end{tabular}
\end{center}
%
The conditional |\ifchilddocmanual| is true whenever
a part to be included by |\input| is being compiled,
and the name of the part is stored in |\childdocname|.

%%%%%%%%%%%%%%%%%%%%%%%%%%%%%%%%%%%%%%%%
\DescribeMacro{\childdocby}
Each part to be included by |\input| should start with:
%
\begin{center}
\begin{tabular}{l}
|\input{childdoc.def}|\\
|\childdocby{|\textit{main}|}|\\
\end{tabular}
\end{center}
%
The directive |\childdocby| is similar to |\childdocof|
described in \secref{sec:include},
but the subsequent selection of content must be done manually.
To that end, both |\ifchilddoc| and |\ifchilddocmanual|
will be true upon processing of a part,
and the name of the part is stored in |\childdocname|.
Note that |\jobname| will be set to the filename of the current part
so that each part receives an individual |.aux| file
that does not interfere with the |.aux| file(s) of the main document.
This behaviour can be altered by the alternative form
|\childdocby[*]{|\textit{main}|}| (with a non-empty optional argument)
which uses the |.aux| file of the main document
by setting |\jobname| to \textit{main}.

%%%%%%%%%%%%%%%%%%%%%%%%%%%%%%%%%%%%%%%%%%%%%%%%%%%%%%%%%%%%%%%%%%%%%%%%%%%%%%%%
\subsection{Driver Development}
\label{sec:driver}

The \textsf{childdoc} mechanism can also be use for the development
of definition files such as \LaTeX{} styles or classes.
This case differs from the above setup with multiple parts
included by |\include| in that no |\includeonly| should be invoked.
This can be achieved by starting the include file
(before |\ProvidesPackage|) with:
%
\begin{center}
\begin{tabular}{l}
|\input{childdoc.def}|\\
|\childdocforward{|\textit{main}|}|\\
\end{tabular}
\end{center}
%
or alternatively with:
%
\begin{center}
\begin{tabular}{l}
|\input{childdoc.def}|\\
|\childdocby{|\textit{main}|}|\\
\end{tabular}
\end{center}
%
Both forms have slightly different effects as described above.
The main file is prepared as usual, see \secref{sec:include}.

%%%%%%%%%%%%%%%%%%%%%%%%%%%%%%%%%%%%%%%%%%%%%%%%%%%%%%%%%%%%%%%%%%%%%%%%%%%%%%%%
\subsection{Legacy Detection}
\label{sec:detection}

The directive |\childdocmain| in the main file can detect
whether the complete document or merely a child is to be compiled
even without using the directive |\childdocof|.
This method is deprecated because it is less robust
and there is no compelling reason to use it;
it is merely provided for backward compatibility
and it may be removed in future versions.

If the detection mechanism is to be used,
it is mandatory to correctly specify
the filename of the main file as the argument of |\childdocmain|:
%
\begin{center}
\begin{tabular}{l}
|\input{childdoc.def}|\\
|\childdocmain{|\textit{main}|}|\\
\end{tabular}
\end{center}
%
If |\jobname| does not match the argument \textit{main} of |\childdocmain|,
it is assumed that |\jobname| points to the child file to be compiled.
When using |\childdocmain| with the main file specified as argument,
it suffices to start a child file
with just |\input{|\textit{main}|}|
without loading of the package and using |\childdocof|.
If instead all processing is done
with the appropriate \textsf{childdoc} directives,
the argument of \textit{main} of |\childdocmain| can be empty.

An alternative version of the command line processing described
in \secref{sec:commandline} using the detection mechanism reads:
%
\begin{center}
|... -jobname "|\textit{target}|" "|[\textit{flags}]%
[|\def\jobname{|\textit{dest}|}|]|\input{|\textit{main}|}"|
\end{center}

%%%%%%%%%%%%%%%%%%%%%%%%%%%%%%%%%%%%%%%%%%%%%%%%%%%%%%%%%%%%%%%%%%%%%%%%%%%%%%%%
\subsection{Manual Code}
\label{sec:manual}

In case one cannot be certain whether the definitions file |childdoc.def|
is installed on the target \TeX{} distribution
and one prefers not to ship it,
it is conceivable to paste a few relevant commands into the sources.

To that end, drop all statements |\input{childdoc.def}|
and perform the replacements as outlined below.
Instead of |\childdocmain{|\textit{main}|}| add the following code
to the top of the main file:
%
\begin{center}
\begin{tabular}{l}
|\||ifdefined\childdocname\endinput\||fi\newif\ifchilddoc|\\
|\edef\childdocname{\scantokens\expandafter{\jobname\noexpand}}|\\
|\def\childdocmain{|\textit{main}|}\||ifx\childdocmain\childdocname\||else|\\
|\childdoctrue\includeonly{\childdocname}\let\jobname\childdocmain\||fi|\\
\end{tabular}
\end{center}
%
Instead of |\childdocof{|\textit{main}|}| just include the main file
at the top of each child file:
%
\begin{center}
|\input{|\textit{main}|}|
\end{center}
%
A simple redirection |\childdocforward{|\textit{dest}|}| is achieved by:
%
\begin{center}
|\def\jobname{|\textit{dest}|}\input{\jobname}|
\end{center}
%
The redirection with prefix
|\childdocforwardprefix[|\textit{prefix}|]{|\textit{dest}|}|
is accomplished by:
%
\begin{center}
\begin{tabular}{l}
|{\edef\jobname{\scantokens\expandafter{\jobname\noexpand}}|\\
|\def\redirectjob |\textit{prefix}|#1~~~{\gdef\jobname{|\textit{dest}|#1}}|\\
|\expandafter\redirectjob\jobname~~~}\input{\jobname}|
\end{tabular}
\end{center}

In an alternative approach,
child documents can be compiled by a specific command line
without additional code or specific definitions:
%
\begin{center}
|... -jobname "|\textit{target}|" "|[\textit{flags}]%
|\includeonly{|\textit{dest}|}\input{|\textit{main}|}"|
\end{center}
%

%%%%%%%%%%%%%%%%%%%%%%%%%%%%%%%%%%%%%%%%%%%%%%%%%%%%%%%%%%%%%%%%%%%%%%%%%%%%%%%%
%%%%%%%%%%%%%%%%%%%%%%%%%%%%%%%%%%%%%%%%%%%%%%%%%%%%%%%%%%%%%%%%%%%%%%%%%%%%%%%%
\section{Information}

%%%%%%%%%%%%%%%%%%%%%%%%%%%%%%%%%%%%%%%%%%%%%%%%%%%%%%%%%%%%%%%%%%%%%%%%%%%%%%%%
\subsection{Copyright}

Copyright \copyright{} 2017--2018 Niklas Beisert

This work may be distributed and/or modified under the
conditions of the \LaTeX{} Project Public License, either version 1.3
of this license or (at your option) any later version.
The latest version of this license is in
  \url{http://www.latex-project.org/lppl.txt}
and version 1.3 or later is part of all distributions of \LaTeX{}
version 2005/12/01 or later.

This work has the LPPL maintenance status `maintained'.

The Current Maintainer of this work is Niklas Beisert.

This work consists of the files |README.txt|, |childdoc.ins| and |childdoc.dtx|
as well as the derived files |childdoc.def|, |cdocsamp.tex|
with |cdocsch1.tex|, |cdocsch2.tex|, |cdocspt3.tex|, |cdocspt4.tex|,
|cdocsdrf.tex|, |cdocsfn1.tex|, |cdocsfn2.tex|
as well as |childdoc.pdf|.

%%%%%%%%%%%%%%%%%%%%%%%%%%%%%%%%%%%%%%%%%%%%%%%%%%%%%%%%%%%%%%%%%%%%%%%%%%%%%%%%
\subsection{Files and Installation}

The package consists of the files:
%
\begin{center}
\begin{tabular}{ll}
    |README.txt|   & readme file \\
    |childdoc.ins| & installation file \\
    |childdoc.dtx| & source file \\
    |childdoc.def| & definition file \\
    |cdocsamp.tex| & sample main file \\
    |cdocsch1.tex| & sample include file \\
    |cdocsch2.tex| & sample include file \\
    |cdocspt3.tex| & sample part file \\
    |cdocspt4.tex| & sample part file \\
    |cdocsdrf.tex| & sample redirection file \\
    |cdocsfn1.tex| & sample redirection file \\
    |cdocsfn2.tex| & sample redirection file \\
    |childdoc.pdf| & manual
\end{tabular}
\end{center}
%
The distribution consists of the files
|README.txt|, |childdoc.ins| and |childdoc.dtx|.
%
\begin{itemize}
\item
Run (pdf)\LaTeX{} on |childdoc.dtx|
to compile the manual |childdoc.pdf| (this file).
\item
Run \LaTeX{} on |childdoc.ins| to create the definitions file |childdoc.def|
and the sample |cdocsamp.tex| with include files
|cdocsch1.tex|, |cdocsch2.tex|, |cdocspt3.tex|, |cdocspt4.tex|,
|cdocsdrf.tex|, |cdocsfn1.tex|, |cdocsfn2.tex|.
Then copy the file |childdoc.def| to an appropriate directory of your \LaTeX{}
distribution, e.g.\ \textit{texmf-root}|/tex/latex/childdoc|.
\end{itemize}

%%%%%%%%%%%%%%%%%%%%%%%%%%%%%%%%%%%%%%%%%%%%%%%%%%%%%%%%%%%%%%%%%%%%%%%%%%%%%%%%
\subsection{Related CTAN Packages}

There are several other packages which offer a similar functionality:
%
\begin{itemize}
\item
The packages
\href{http://ctan.org/pkg/docmute}{\textsf{docmute}},
\href{http://ctan.org/pkg/includex}{\textsf{includex}} and
\href{http://ctan.org/pkg/standalone}{\textsf{standalone}}
provide commands to include only the document body of
a child file thus allowing both files to be compiled individually.
\item
The packages \href{http://ctan.org/pkg/subdocs}{\textsf{subdocs}}
and \href{http://ctan.org/pkg/subfiles}{\textsf{subfiles}}
provide structures in which the main and child documents can be
encapsulated and allowing them to be compiled individually.
The inclusion mechanism is different from the conventional |\include|.
\item
The package \href{http://ctan.org/pkg/combine}{\textsf{combine}}
is an elaborate solution to combine several documents into one.
\end{itemize}
%
See also the CTAN topic \href{http://ctan.org/topic/subdocs}{\textsf{subdocs}}
for further related packages.
The present package differs from the above solutions in that
a document structure constructed with the conventional |\include| mechanism
just needs two extra commands at the top of every file
such that all constituent files can be compiled individually.

%%%%%%%%%%%%%%%%%%%%%%%%%%%%%%%%%%%%%%%%%%%%%%%%%%%%%%%%%%%%%%%%%%%%%%%%%%%%%%%%
%\subsection{Feature Suggestions}
%
%The following is a list of features which may be useful for future
%versions of this package:
%%
%\begin{itemize}
%\item
%\ldots
%\end{itemize}

%%%%%%%%%%%%%%%%%%%%%%%%%%%%%%%%%%%%%%%%%%%%%%%%%%%%%%%%%%%%%%%%%%%%%%%%%%%%%%%%
\subsection{Revision History}

%%%%%%%%%%%%%%%%%%%%%%%%%%%%%%%%%%%%%%%%
\paragraph{v2.0:} 2018/12/30

\begin{itemize}
\item
immediate forward processing
\item
added |\childdocby| mechanism
\item
manual restructured
\end{itemize}

%%%%%%%%%%%%%%%%%%%%%%%%%%%%%%%%%%%%%%%%
\paragraph{v1.6:} 2018/01/17

\begin{itemize}
\item
application for development of include files
\item
corrections to manual
\end{itemize}

%%%%%%%%%%%%%%%%%%%%%%%%%%%%%%%%%%%%%%%%
\paragraph{v1.5:} 2017/05/21

\begin{itemize}
\item
more complete structuring introduced
\item
|\childdocof| introduced
\item
|\childdoc| renamed to |\childdocmain|
\item
|\childredirect| renamed to |\childdocforward| and |\childdocforwardprefix|
and functionality expanded
\end{itemize}

%%%%%%%%%%%%%%%%%%%%%%%%%%%%%%%%%%%%%%%%
\paragraph{v1.0:} 2017/04/27

\begin{itemize}
\item
manual and install package
\item
first version published on CTAN
\end{itemize}

%%%%%%%%%%%%%%%%%%%%%%%%%%%%%%%%%%%%%%%%
\paragraph{v0.6:} 2017/04/26

\begin{itemize}
\item
redirection mechanism added
\end{itemize}

%%%%%%%%%%%%%%%%%%%%%%%%%%%%%%%%%%%%%%%%
\paragraph{v0.5:} 2017/04/26

\begin{itemize}
\item
functionality in definition file
\end{itemize}


%%%%%%%%%%%%%%%%%%%%%%%%%%%%%%%%%%%%%%%%%%%%%%%%%%%%%%%%%%%%%%%%%%%%%%%%%%%%%%%%
%%%%%%%%%%%%%%%%%%%%%%%%%%%%%%%%%%%%%%%%%%%%%%%%%%%%%%%%%%%%%%%%%%%%%%%%%%%%%%%%
%%%%%%%%%%%%%%%%%%%%%%%%%%%%%%%%%%%%%%%%%%%%%%%%%%%%%%%%%%%%%%%%%%%%%%%%%%%%%%%%
\appendix

\settowidth\MacroIndent{\rmfamily\scriptsize 000\ }

 \DocInput{childdoc.dtx}

\end{document}
%</driver>
% \fi
%
% %%%%%%%%%%%%%%%%%%%%%%%%%%%%%%%%%%%%%%%%%%%%%%%%%%%%%%%%%%%%%%%%%%%%%%%%%%%%%%
% %%%%%%%%%%%%%%%%%%%%%%%%%%%%%%%%%%%%%%%%%%%%%%%%%%%%%%%%%%%%%%%%%%%%%%%%%%%%%%
% \section{Sample}
%\iffalse
%<*samplemain>
%\fi
%
% The following presents a sample document
% with two chapters, two parts, a title page,
% a compile flag as well as three forwarding files to set the flag.
% It consists of eight |.tex| files:
% \begin{center}
% \begin{tabular}{ll}
% |cdocsamp.tex|&main file\\
% |cdocsch1.tex|&include file for chapter 1\\
% |cdocsch2.tex|&include file for chapter 2\\
% |cdocspt3.tex|&include file for part 3\\
% |cdocspt4.tex|&include file for part 4\\
% |cdocsdrf.tex|&forwarding file for main file in draft mode\\
% |cdocsfi1.tex|&forwarding file for final version of chapter 1\\
% |cdocsfi2.tex|&forwarding file for final version of chapter 2\\
% \end{tabular}
% \end{center}
% Each of the eight files can be compiled directly by the \LaTeX{} compiler.
%
% %%%%%%%%%%%%%%%%%%%%%%%%%%%%%%%%%%%%%%
% \paragraph{Main File.}
%
% The main file is called |cdocsamp.tex|.
%
% Load the \textsf{childdoc} definitions and
% declare the filename for the main document:
%    \begin{macrocode}
\input{childdoc.def}
\childdocmain{}
%    \end{macrocode}

% Optional override for |\version| flag:
%    \begin{macrocode}
%%\ifchilddoc\else\providecommand{\version}{draft}\fi
%    \end{macrocode}

% Define the default values for the |\version| flag
% (|final| for the main file and |draft| for childs):
%    \begin{macrocode}
\ifchilddoc
\providecommand{\version}{draft}
\else
\providecommand{\version}{final}
\fi
%    \end{macrocode}

% Load the standard document class:
%    \begin{macrocode}
\documentclass[12pt]{article}
%    \end{macrocode}

% Start the document body:
%    \begin{macrocode}
\begin{document}
%    \end{macrocode}

% Declare a title page.
% Print title, part of document being processed and version flag:
%    \begin{macrocode}
\addtocounter{page}{-1}
\begin{center}
{\LARGE\bfseries{}childdoc example\par}
\vspace{1cm}
\ifchilddoc
\ifchilddocmanual part\else chapter\fi:
`\childdocname' of `\childdocjob'\par
\else
main document: `\childdocjob'\par
\fi
version: \version\par
\end{center}
\newpage
%    \end{macrocode}

% Manually include selected file,
% otherwise process as usual:
%    \begin{macrocode}
\ifchilddocmanual
\section*{part `\childdocname'}
\input{\childdocname}
\else
%    \end{macrocode}

% Include the two chapters:
%    \begin{macrocode}
\include{cdocsch1}
\include{cdocsch2}
%    \end{macrocode}

% Include the two parts unless only chapters should be displayed:
%    \begin{macrocode}
\ifchilddoc\else
\section{part three}
\input{cdocspt3}
\section{part four}
\input{cdocspt4}
\fi
%    \end{macrocode}

% Process as usual until here:
%    \begin{macrocode}
\fi
%    \end{macrocode}

% End of document body:
%    \begin{macrocode}
\end{document}
%    \end{macrocode}
%\iffalse
%</samplemain>
%\fi
%
% %%%%%%%%%%%%%%%%%%%%%%%%%%%%%%%%%%%%%%
% \paragraph{Chapter Include Files.}
%
% The include files are called |cdocsch1.tex| and |cdocsch2.tex|.
%
%\iffalse
%<*samplechap1|samplechap2>
%\fi

% Optional override for |\version| flag:
%    \begin{macrocode}
%%\providecommand{\version}{final}
%    \end{macrocode}

% Include the main document:
%    \begin{macrocode}
\input{childdoc.def}
\childdocof{cdocsamp}
%    \end{macrocode}

%\iffalse
%</samplechap1|samplechap2>
%\fi
%
%\iffalse
%<*samplechap1>
%\fi
% Some text for chapter 1:
%    \begin{macrocode}
\section{one}
some text in chapter one
%    \end{macrocode}

%\iffalse
%</samplechap1>
%\fi
% Some text for chapter 2:
%\iffalse
%<*samplechap2>
%\fi
%    \begin{macrocode}
\section{two}
more text in chapter two
%    \end{macrocode}

%\iffalse
%</samplechap2>
%\fi
%
% %%%%%%%%%%%%%%%%%%%%%%%%%%%%%%%%%%%%%%
% \paragraph{Part Include Files.}
%
% The include files are called |cdocspt3.tex| and |cdocspt4.tex|.
%
%\iffalse
%<*samplepart3|samplepart4>
%\fi

% Optional override for |\version| flag:
%    \begin{macrocode}
%%\providecommand{\version}{final}
%    \end{macrocode}

% Include the main document:
%    \begin{macrocode}
\input{childdoc.def}
\childdocby{cdocsamp}
%    \end{macrocode}

%\iffalse
%</samplepart3|samplepart4>
%\fi
%
%\iffalse
%<*samplepart3>
%\fi
% Some text for part 3:
%    \begin{macrocode}
some text in part three
%    \end{macrocode}

%\iffalse
%</samplepart3>
%\fi
% Some text for part 4:
%\iffalse
%<*samplepart4>
%\fi
%    \begin{macrocode}
more text in part four
%    \end{macrocode}

%\iffalse
%</samplepart4>
%\fi
%
% %%%%%%%%%%%%%%%%%%%%%%%%%%%%%%%%%%%%%%
% \paragraph{Forwarding for a Complete Draft.}
%
% The following forwarding file |cdocsdrf.tex|
% compiles the main document in draft mode:
%\iffalse
%<*sampledraft>
%\fi
%    \begin{macrocode}
\def\version{draft}
\input{childdoc.def}
\childdocforward{cdocsamp}
%    \end{macrocode}

%\iffalse
%</sampledraft>
%\fi
%
% %%%%%%%%%%%%%%%%%%%%%%%%%%%%%%%%%%%%%%
% \paragraph{Forwarding for Final Version of the Chapters.}
%
% The following forwarding files |cdocsfn1.tex| and |cdocsfn2.tex|
% (with identical content)
% compile the final versions of the child documents
% |cdocsch1.tex| and |cdocsch2.tex|, respectively:
%\iffalse
%<*samplefinal>
%\fi
%    \begin{macrocode}
\def\version{final}
\input{childdoc.def}
\childdocforwardprefix[cdocsamp]{cdocsfn}{cdocsch}
%    \end{macrocode}

%\iffalse
%</samplefinal>
%\fi
%
% %%%%%%%%%%%%%%%%%%%%%%%%%%%%%%%%%%%%%%
% \paragraph{Command Line Processing.}
%
% The following three command lines generate the output files
% |cdocscld|, |cdocscl1| and |cdocscl2|
% which should be identical to
% |cdocsdrf|, |cdocsch1| and |cdocsfn2|, respectively:
% \begin{center}
% \begin{tabular}{l}
% |latex -jobname cdocscld \|\\
% |  "\def\version{draft}\input{childdoc.def}\childdocforward{cdocsamp}"|\\
% |latex -jobname cdocscl1 \|\\
% |  "\input{childdoc.def}\childdocforward[cdocsamp]{cdocsch1}"|\\
% |latex -jobname cdocscl2 \|\\
% |  "\def\version{final}\input{childdoc.def}\childdocforward{cdocsch2}"|
% \end{tabular}
% \end{center}
% Note that the trailing backslash on each first line
% merely continues the input to the second line
% (for convenient cut ant paste).
% Furthermore, the command |latex| can be replaced by any
% of its alternative versions such as |pdflatex|.
%
% %%%%%%%%%%%%%%%%%%%%%%%%%%%%%%%%%%%%%%%%%%%%%%%%%%%%%%%%%%%%%%%%%%%%%%%%%%%%%%
% %%%%%%%%%%%%%%%%%%%%%%%%%%%%%%%%%%%%%%%%%%%%%%%%%%%%%%%%%%%%%%%%%%%%%%%%%%%%%%
% \section{Implementation}
%\iffalse
%<*package>
%\fi
%
% This section describes the definitions file |childdoc.def|.

% The definitions cannot be loaded using |\usepackage| or |\RequirePackage|
% which has a mechanism to prevent loading a style file more than once.
% When loading the definitions by means of |\input|
% multiple instances have to be prevented manually:
%\iffalse
%This code needs to be before the `\ProvidesFile' directive
%which is defined at the beginning of this file.
%Therefore it is also placed there and commented out here.
%</package>
%<*discard>
%\fi
%    \begin{macrocode}
\ifdefined\childdocmain\endinput\fi
%    \end{macrocode}
%\iffalse
%</discard>
%<*package>
%\fi
%
% \macro{\ifchilddoc}
% \macro{\ifchilddocmanual}
% The conditional |\ifchilddoc| tells whether a
% child (true) or main (false) document is being compiled.
% The conditional |\ifchilddocmanual| tells whether
% the |\includeonly| mechanism is used (false) or
% the selection of child files must be performed manually (true).
% The definitions initialise to false:
%    \begin{macrocode}
\newif\ifchilddoc
\newif\ifchilddocmanual
%    \end{macrocode}

% \macro{\childdocname}
% \macro{\childdocjob}
% The macro |\childdocname| stores the name of the main document
% to be compiled. The macro |\childdocjob| stores the name of
% the document on which the \LaTeX{} compiler was originally invoked.
% The content of |\jobname| cannot be compared
% to filenames specified in the source due to different catcodes.
% The following code rescans |\jobname|, stores the result
% in |\childdocname| and saves a copy in |\childdocjob|:
%    \begin{macrocode}
\edef\childdocname{\scantokens\expandafter{\jobname\noexpand}}
\let\childdocjob\childdocname
%    \end{macrocode}

% \macro{\childdocdisable}
% The macro |\childdocdisable| prevents the main file
% from being processed more than once.
% At this stage, the main document command |\childdocmain|
% is assumed to be called once again where it should do nothing.
% Any subsequent call to it should prevent
% a secondary processing of the main document
% It overwrites the forwarding commands
% |\childdocof| and |\childdocforward|
% with empty macros to prevent further inclusions of the main document:
%    \begin{macrocode}
\newcommand{\childdocdisable}
{
  \renewcommand{\childdocmain}[1]{\renewcommand{\childdocmain}[1]{\endinput}}
  \renewcommand{\childdocof}[1]{}
  \renewcommand{\childdocby}[2][]{}
  \renewcommand{\childdocforward}[2][]{}
  \renewcommand{\childdocdisable}{}
}
%    \end{macrocode}

% \macro{\childdocmain}
% The macro |\childdocmain| is to be called at the top of the main file
% with nothing or the main filename (without extension) as argument.
% First, it breaks loops.
% If the argument is not empty and does not match |\childdocname|
% (which is set by the first inclusion of |childdoc.def|),
% |\ifchilddoc| is set to true, |\includeonly| is applied to the child file
% and |\jobname| is set to the main file
% (for proper handling of |.aux| files):
%    \begin{macrocode}
\newcommand{\childdocmain}[1]
{
  \childdocdisable\childdocmain{}
  \if?#1?\else
    \begingroup
      \def\childdoctmp{#1}
      \ifx\childdoctmp\childdocname
        \def\childdoctmp{}
      \else
        \def\childdoctmp
        {
          \childdoctrue
          \includeonly{\childdocname}
          \def\childdocjob{#1}
          \def\jobname{#1}
        }
      \fi
      \expandafter
    \endgroup
    \childdoctmp
  \fi
}
%    \end{macrocode}

% \macro{\childdocof}
% The command |\childdocof| redirects
% compilation to the main file |#1|.
%    \begin{macrocode}
\newcommand{\childdocof}[1]
{
  \childdocdisable
  \childdoctrue
  \includeonly{\childdocname}
  \def\jobname{#1}
  \def\childdocjob{#1}
  \input{#1}
}
%    \end{macrocode}

% \macro{\childdocby}
% The command |\childdocby| ....
%    \begin{macrocode}
\newcommand{\childdocby}[2][]
{
  \childdocdisable
  \childdoctrue
  \childdocmanualtrue
  \if?#1?\else
    \def\jobname{#2}
  \fi
  \def\childdocjob{#2}
  \input{#2}
  \endinput
}
%    \end{macrocode}

% \macro{\childdocforward}
% The command |\childdocforward| redirects
% compilation to the main file or
% (if the optional argument is given) a child file.
% Parameters are set as if the main file
% or a child file starting with |\childdocof| was compiled.
% Then compilation is handed over to the main file:
%    \begin{macrocode}
\newcommand{\childdocforward}[2][]
{
  \begingroup
    \if?#1?
      \def\childdoctmp
      {
        \def\childdocname{#2}
        \def\childdocjob{#2}
        \def\jobname{#2}
        \input{#2}
        \endinput
      }
    \else
      \def\childdoctmp
      {
        \childdocdisable
        \def\childdocname{#2}
        \childdoctrue
        \includeonly{#2}
        \def\childdocjob{#1}
        \def\jobname{#1}
        \input{#1}
        \endinput
      }
    \fi
    \expandafter
  \endgroup
  \childdoctmp
}
%    \end{macrocode}

% \macro{\childdocforwardprefix}
% The command |\childdocforwardprefix| redirects
% compilation to the main or a child file by means of a pattern.
% The prefix |#1| in the current filename is replaced by |#2|
% and the suffix of the current filename is kept
% (it is assumed that the filename does not contain the substring `|~~~|'
% which is used as a delimiter).
% Compilation is handed over to the new file by |\childdocforward|:
%    \begin{macrocode}
\newcommand{\childdocforwardprefix}[3][]
{
  \begingroup
    \def\childdocextract #2##1~~~{\def\childdoctmp{\childdocforward[#1]{#3##1}}}
    \expandafter\childdocextract\childdocname~~~
    \expandafter
  \endgroup
  \childdoctmp
}
%    \end{macrocode}

% \macro{\childdoc}
% The deprecated macro |\childdoc| is a legacy version of |\childdocmain|:
%    \begin{macrocode}
\newcommand{\childdoc}{\childdocmain}
%    \end{macrocode}

% \macro{\childdocredirect}
% The deprecated macro |\childdocredirect| is a legacy version
% of |\childdocforward| and |\childdocforwardprefix|:
%    \begin{macrocode}
\newcommand{\childdocredirect}[2][]
{
  \begingroup
    \if?#1?
      \def\childdoctmp{\childdocforward{#2}}
    \else
      \def\childdoctmp{\childdocforwardprefix{#1}{#2}}
    \fi
    \expandafter
  \endgroup
  \childdoctmp
}
%    \end{macrocode}

%\iffalse
%</package>
%\fi
%
\endinput

\childdocforward{cdocsamp}
%    \end{macrocode}

%\iffalse
%</sampledraft>
%\fi
%
% %%%%%%%%%%%%%%%%%%%%%%%%%%%%%%%%%%%%%%
% \paragraph{Forwarding for Final Version of the Chapters.}
%
% The following forwarding files |cdocsfn1.tex| and |cdocsfn2.tex|
% (with identical content)
% compile the final versions of the child documents
% |cdocsch1.tex| and |cdocsch2.tex|, respectively:
%\iffalse
%<*samplefinal>
%\fi
%    \begin{macrocode}
\def\version{final}
% \iffalse
%
% childdoc.dtx Copyright (C) 2017-2018 Niklas Beisert
%
% This work may be distributed and/or modified under the
% conditions of the LaTeX Project Public License, either version 1.3
% of this license or (at your option) any later version.
% The latest version of this license is in
%   http://www.latex-project.org/lppl.txt
% and version 1.3 or later is part of all distributions of LaTeX
% version 2005/12/01 or later.
%
% This work has the LPPL maintenance status `maintained'.
%
% The Current Maintainer of this work is Niklas Beisert.
%
% This work consists of the files childdoc.dtx and childdoc.ins
% and the derived files childdoc.def and cdocsamp.tex with
% cdocsch1.tex, cdocsch2.tex, cdocsdrf.tex, cdocsfn1.tex, cdocsfn2.tex.
%
%<package>\ifdefined\childdocmain\endinput\fi
%<package>\ProvidesFile{childdoc.def}[2018/12/30 v2.0 child document driver]
%<samplemain>\ProvidesFile{cdocsamp.tex}[2018/12/30 v2.0 sample for childdoc]
%<*driver>
%\ProvidesFile{childdoc.drv}[2018/12/30 v2.0 childdoc reference manual file]
\PassOptionsToClass{10pt,a4paper}{article}
\documentclass{ltxdoc}

\usepackage[margin=35mm]{geometry}
\usepackage{hyperref}
\usepackage{hyperxmp}
\usepackage[usenames]{color}

\hypersetup{colorlinks=true}
\hypersetup{pdfstartview=FitH}
\hypersetup{pdfpagemode=UseNone}
\hypersetup{pdfsource={}}
\hypersetup{pdflang={en-UK}}
\hypersetup{pdfcopyright={Copyright 2017-2018 Niklas Beisert.
  This work may be distributed and/or modified under the
  conditions of the LaTeX Project Public License, either version 1.3
  of this license or (at your option) any later version.}}
\hypersetup{pdflicenseurl={http://www.latex-project.org/lppl.txt}}
\hypersetup{pdfcontactaddress={ETH Zurich, ITP, HIT K,
  Wolfgang-Pauli-Strasse 27}}
\hypersetup{pdfcontactpostcode={8093}}
\hypersetup{pdfcontactcity={Zurich}}
\hypersetup{pdfcontactcountry={Switzerland}}
\hypersetup{pdfcontactemail={nbeisert@itp.phys.ethz.ch}}
\hypersetup{pdfcontacturl={http://people.phys.ethz.ch/\xmptilde nbeisert/}}

\newcommand{\secref}[1]{\hyperref[#1]{section \ref*{#1}}}

\parskip1ex
\parindent0pt
\let\olditemize\itemize
\def\itemize{\olditemize\parskip0pt}

\begin{document}

\title{The \textsf{childdoc} Package}
\hypersetup{pdftitle={The childdoc Package}}
\author{Niklas Beisert\\[2ex]
  Institut f\"ur Theoretische Physik\\
  Eidgen\"ossische Technische Hochschule Z\"urich\\
  Wolfgang-Pauli-Strasse 27, 8093 Z\"urich, Switzerland\\[1ex]
  \href{mailto:nbeisert@itp.phys.ethz.ch}
  {\texttt{nbeisert@itp.phys.ethz.ch}}}
\hypersetup{pdfauthor={Niklas Beisert}}
\hypersetup{pdfsubject={Manual for the LaTeX2e Package childdoc}}
\date{30 December 2018, \textsf{v2.0}}
\maketitle

\begin{abstract}\noindent
\textsf{childdoc} is a \LaTeXe{} package
that enables the direct compilation
of document sections included by |\include|
to individual files.
\end{abstract}

\begingroup
\parskip0ex
\tableofcontents
\endgroup

%%%%%%%%%%%%%%%%%%%%%%%%%%%%%%%%%%%%%%%%%%%%%%%%%%%%%%%%%%%%%%%%%%%%%%%%%%%%%%%%
%%%%%%%%%%%%%%%%%%%%%%%%%%%%%%%%%%%%%%%%%%%%%%%%%%%%%%%%%%%%%%%%%%%%%%%%%%%%%%%%
\section{Introduction}

\LaTeX{} provides a mechanism to structure a large document (such as a book)
into a main file and several child files (containing the chapters)
using the |\include| command.
This mechanism is beneficial for documents
which span hundreds of pages in order to
make the source file(s) more manageable.
Moreover, compilation can be restricted to
selected child files by means of the |\includeonly| command.
The latter feature can be used to reduce the compilation time while editing
(this was significantly more useful in the earlier days of \LaTeX{})
or to generate a smaller document which is easier to navigate.
Another application of |\includeonly| is to generate
documents consisting of selected parts of the complete document.

However, there are a few drawbacks of the plain |\include| mechanism:
\begin{itemize}
\item
The child files cannot be compiled on their own,
they can only be compiled via the main file.
A naive editing environment
(such as a text editor with an option
to have the current file processed by \LaTeX)
may require one to switch to the main file before compiling;
attempting to compile the child file produces errors.
\item
The main file must be modified (each time)
to adjust the |\includeonly| command
to the present needs. This easily leaves the main file in a messy state.
\item
The generated document will always carry the filename
of the main document. This is inconvenient if
several child files are to be compiled and
to be kept for distribution.
\end{itemize}

The present package provides a simple interface
to make child files individually compilable by \LaTeX{}.
Compiling a child file then has the same effect as compiling
the main file with an |\includeonly| command
to select the appropriate child.
Moreover the generated document will carry the name of the child
rather than the main file.
This resolves all three above issues.

This feature is meant to make the editing of books,
thesis documents and lecture notes somewhat more convenient.
However, the package can also be used efficiently for
composing a series of documents (such as exercise sheets)
which are typically distributed individually.
It then assists the author in generating the individual documents
(potentially in different versions)
as well as a document containing the collected series.
Another application is in developing style files
or other kinds of included material
where compilation of the style file could redirect
to a sample or test file.

%%%%%%%%%%%%%%%%%%%%%%%%%%%%%%%%%%%%%%%%%%%%%%%%%%%%%%%%%%%%%%%%%%%%%%%%%%%%%%%%
%%%%%%%%%%%%%%%%%%%%%%%%%%%%%%%%%%%%%%%%%%%%%%%%%%%%%%%%%%%%%%%%%%%%%%%%%%%%%%%%
\section{Usage}

First of all, the package \textsf{childdoc} is \emph{not} a standard
\LaTeXe{} |.sty| style file! Therefore it needs to be invoked in
a non-standard way.

%%%%%%%%%%%%%%%%%%%%%%%%%%%%%%%%%%%%%%%%%%%%%%%%%%%%%%%%%%%%%%%%%%%%%%%%%%%%%%%%
\subsection{Included Files}
\label{sec:include}

%%%%%%%%%%%%%%%%%%%%%%%%%%%%%%%%%%%%%%%%
\DescribeMacro{\childdocmain}
To use the package, add the commands
\begin{center}
\begin{tabular}{l}
|\input{childdoc.def}|\\
|\childdocmain{}|\\
\end{tabular}
\end{center}
at the very top of the main \LaTeX{} file,
in particular \emph{before} the |\documentclass| statement!
The argument of |\childdocmain| should be left empty
(but it must be present).

%%%%%%%%%%%%%%%%%%%%%%%%%%%%%%%%%%%%%%%%
\DescribeMacro{\childdocof}
Furthermore, add the commands
\begin{center}
\begin{tabular}{l}
|\input{childdoc.def}|\\
|\childdocof{|\textit{main}|}|\\
\end{tabular}
\end{center}
at the top of every child file \textit{child}
which is included by |\include{|\textit{child}|}|
from within the main file
(or at least for those files to be compiled individually).
The argument \textit{main} must be the filename of the main file.

There are a couple of
considerations in setting up the main and child documents:

%%%%%%%%%%%%%%%%%%%%%%%%%%%%%%%%%%%%%%%%
\paragraph{Restrictions.}

Please note the following restrictions:
\begin{itemize}
\item
|\childdocmain| must be called with one argument \textit{main}
to ensure compatibility with earlier version of the package.
It must either be empty (|\childdocmain{}|)
or precisely match the filename of the main file in which it is specified.
See \secref{sec:detection} for further information.
\item
The filename \textit{main} must be specified without the |.tex| extension.
\item
The filename \textit{main} is case sensitive
(even in case-insensitive file systems)
due to internal string comparison.
\item
The argument \textit{main} should be fully expanded, it cannot be a macro.
\item
Subdirectories and special characters should be avoided in filenames.
\item
The command |\childdocmain{|\textit{main}|}| must be followed by a whitespace.
It should not be followed immediately by another command
or by a comment mark `|%|'.
This is because the \TeX{} parser reads the token immediately following
the argument of |\childdocmain| and puts it
at the beginning of every child section;
however, a white\-space is ignored.
\end{itemize}

%%%%%%%%%%%%%%%%%%%%%%%%%%%%%%%%%%%%%%%%
\paragraph{Content of Main File.}

It is advisable to place all content in the child files included by |\include|.
Any output contained in the main file will appear in all child documents
unless suppressed manually;
it cannot be suppressed automatically by the |\includeonly| directive
and thus should normally be avoided.
A method to include some content in the main file
by means of conditional processing is described in \secref{sec:conditional}.

%%%%%%%%%%%%%%%%%%%%%%%%%%%%%%%%%%%%%%%%
\paragraph{Page Numbering.}

When only a part of the document is compiled,
the appropriate numbering of pages
(as well as other status parameters)
is determined from the |.aux| files.
The latter contain information from previous passes.
However this information needs to propagate through
all intermediate child documents.
Therefore the page numbering in child documents may well
be inconsistent until the complete document is compiled at least once.

A useful (if unconventional) way to always ensure a consistent
page numbering is to restart the numbering in each child document
and denote the pages by `\textit{child}|.|\textit{page}'
where \textit{child} represents the chapter/section number of the child file.
This can be achieved by the command
|\numberwithin{page}{|\textit{child}|}|
of the \textsf{amsmath} package
where \textit{child} can be |chapter| or |section|
depending on the chosen structuring.
Alternatively, one can modify the macro |\thepage| appropriately
and reset the counter |page| at the start of each child file.

%%%%%%%%%%%%%%%%%%%%%%%%%%%%%%%%%%%%%%%%%%%%%%%%%%%%%%%%%%%%%%%%%%%%%%%%%%%%%%%%
\subsection{Conditional Processing}
\label{sec:conditional}

The package provides a mechanism to compile different versions
of a document. To customise the versions further some conditional processing
can come in handy to distinguish which version is being compiled.
The package provides two macros to describe the compilation context:

%%%%%%%%%%%%%%%%%%%%%%%%%%%%%%%%%%%%%%%%
\DescribeMacro{\ifchilddoc}
The conditional |\ifchilddoc| distinguishes between the compilation of
child documents and the main document:
%
\begin{center}
|\ifchilddoc |\textit{child-code}| |[|\||else |\textit{main-code}]| \||fi|
\end{center}

%%%%%%%%%%%%%%%%%%%%%%%%%%%%%%%%%%%%%%%%
\DescribeMacro{\childdocname}
\DescribeMacro{\childdocjob}
The macro |\childdocname| contains the filename (without extension)
of the main or child file being processed.
Note that |\childdocjob| will always contain the name of the main file.

%%%%%%%%%%%%%%%%%%%%%%%%%%%%%%%%%%%%%%%%
\paragraph{Title Page.}

Conditional processing can be used to include a title or banner page
in the main document when proper precautions are taken.
Importantly, the code in the main file should ensure that the page counter
(as well as other status parameters which are stored in the |.aux| files)
takes the same value after the conditional processing.
Otherwise the page numbers may take divergent values
depending on which part is compiled.

For example, a title page could be declared by:
%
\begin{center}
\begin{tabular}{l}
|\ifchilddoc\||else|\\
|\addtocounter{page}{-1}|\\
\textit{code for title page}\\
|\newpage|\\
|\||fi|
\end{tabular}
\end{center}
%
A banner page for the child documents can be generated by:
%
\begin{center}
\begin{tabular}{l}
|\ifchilddoc|\\
|\addtocounter{page}{-1}|\\
\textit{code for banner page}\\
|\newpage|\\
|\||fi|
\end{tabular}
\end{center}
%
Here one could write a message such as:
\begin{center}
|This is the part \childdocname{} of \childdocjob{}.|
\end{center}

%%%%%%%%%%%%%%%%%%%%%%%%%%%%%%%%%%%%%%%%%%%%%%%%%%%%%%%%%%%%%%%%%%%%%%%%%%%%%%%%
\subsection{Flags}
\label{sec:flags}

The package makes it easy to generate different versions
of the main or child documents.
To this end compilation flags can be defined
and assigned different default values.
They will be particularly useful in conjunction
with the forwarding mechanism described in \secref{sec:forward}.

For example, it may be useful to have a flag |\version|
which can be set to |draft| or |final|.
The document source will contain some conditional code
depending on the value of |\version|.
Suppose further, the flag should default to |final| for the main file
and to |draft| for child files
which is a natural assignment for editing the document.
This is achieved by placing the following code
in the preamble of the main document
(below the |\childdocmain| directive):
%
\begin{center}
\begin{tabular}{l}
|\ifchilddoc|\\
|\providecommand{\version}{draft}|\\
|\||else|\\
|\providecommand{\version}{final}|\\
|\||fi|
\end{tabular}
\end{center}
%
The definition by |\providecommand| makes sure
that previous definitions are not overwritten.
Further statements |\providecommand{\version}{...}|
can thus be added before the above code to override it.

For the main file, one might add a line
(between |\childdocmain| and the above block)
%
\begin{center}
|%\ifchilddoc\||else\providecommand{\version}{draft}\||fi|
\end{center}
%
which can be uncommented to produce a draft version.
Likewise one can add a line to the very top of a child file
(above the |\childdocof{|\textit{main}|}| directive)
%
\begin{center}
|%\providecommand{\version}{final}|
\end{center}
%
which can be uncommented to produce the final version of this child document.

%%%%%%%%%%%%%%%%%%%%%%%%%%%%%%%%%%%%%%%%%%%%%%%%%%%%%%%%%%%%%%%%%%%%%%%%%%%%%%%%
\subsection{Forwarding}
\label{sec:forward}

Different versions of the main or child documents
using compilation flags as described in \secref{sec:flags}
can be (permanently) stored in different files
for convenient compilation, viewing and distribution.
To this end, the package defines a command
to pass on compilation to a different file:

%%%%%%%%%%%%%%%%%%%%%%%%%%%%%%%%%%%%%%%%
\DescribeMacro{\childdocforward}
The command |\childdocforward| redirects processing to
another source file:
%
\begin{center}
\begin{tabular}{l}
|\input{childdoc.def}|\\
|\childdocforward[|\textit{main}|]{|\textit{dest}|}|\\
\end{tabular}
\end{center}
%
The argument \textit{dest} is the destination file
(without extension).
It should be the main file or one of the child files.
Note that further \textsf{childdoc} directives
such as |\childdocof| and |\childdocforward|
in the indicated file will be processed in this form.
The optional argument \textit{main}
passes on directly to the main file \textit{main}
while pretending to compile the child \textit{dest}.
This form behaves as if \textit{dest}
issues |\childdocof{|\textit{main}|}| right away,
and no further \textsf{childdoc} directives will be processed.

%%%%%%%%%%%%%%%%%%%%%%%%%%%%%%%%%%%%%%%%
\DescribeMacro{\...prefix}
In the alternative form |\childdocforwardprefix|,
%
\begin{center}
\begin{tabular}{l}
|\input{childdoc.def}|\\
|\childdocforwardprefix[|\textit{main}|]{|\textit{prefix}|}{|\textit{dest}|}|
\end{tabular}
\end{center}
%
the destination file is determined by a pattern
depending on the current file:
To make this work, the current file must be called
`{\textit{prefix}\hspace{0.2em}\textit{suffix}}'
with \textit{prefix} matching precisely the argument.
Processing is then passed on to the file
`{\textit{dest}\hspace{0.2em}\textit{suffix}}'.
Surely, the same effect is achieved by
directly specifying the
argument `{\textit{dest}\hspace{0.2em}\textit{suffix}}'
in the first form.
However, that requires to set up a different file
for each child. With the alternative form of the command
all these files can have exactly the same content
which simplifies setting them up and maintaining them.

For example, the following file |draft.tex|
with a compilation flag |\version| as described in \secref{sec:flags}
compiles the main document as a draft:
%
\begin{center}
\begin{tabular}{l}
|\def\version{draft}|\\
|\input{childdoc.def}|\\
|\childdocforward{|\textit{main}|}|
\end{tabular}
\end{center}
%
Likewise, the following files |final|\textit{nn}|.tex|
compile the final version of the child document
|child|\textit{nn}|.tex|:
%
\begin{center}
\begin{tabular}{l}
|\def\version{final}|\\
|\input{childdoc.def}|\\
|\childdocforwardprefix{final}{child}|
\end{tabular}
\end{center}
%

Note that when several versions of a main file and/or of each child file
are to be generated, it may be convenient to set up a |Makefile| or
shell script to automatise the process.

%%%%%%%%%%%%%%%%%%%%%%%%%%%%%%%%%%%%%%%%%%%%%%%%%%%%%%%%%%%%%%%%%%%%%%%%%%%%%%%%
\subsection{Command Line Processing}
\label{sec:commandline}

The effect of redirection files can also be achieved by invoking
the \LaTeX{} compiler with a more elaborate command line.
Most conveniently this should be done as part
of a shell script or a |Makefile|.

When using \textsf{childdoc} in the main file, the following
command lines effectively perform a redirection
(note that depending on the shell being used,
backslashes may have to be doubled: `|\|' $\to$ `|\\|'):
%
\begin{center}
|... -jobname "|\textit{target}|" |\\|"|[\textit{flags}]%
|\input{childdoc.def}\childdocforward[|\textit{main}|]{|\textit{dest}|}"|
\end{center}
%
Here \textit{target} is the name of the output file,
\textit{main} is the name of the main file
and \textit{dest} is the name of the main or child file to be processed
(all filenames without extensions).
The optional argument \textit{main} can be omitted
if \textit{main} matches \textit{dest}.
Optionally, compilation \textit{flags} can be defined via |\def| commands.
This command line makes the \TeX{} engine believe
it is compiling the file \textit{target}
whose content is specified as the latter parameter.
The provided code then forwards the processing to
\textit{main} or \textit{dest} as described in \secref{sec:forward}.

%%%%%%%%%%%%%%%%%%%%%%%%%%%%%%%%%%%%%%%%%%%%%%%%%%%%%%%%%%%%%%%%%%%%%%%%%%%%%%%%
\subsection{Include by Input}
\label{sec:input}

Including child documents by |\include| has some restrictions by design.
Most notably, the content of a child document always occupies
its own set of pages; pages cannot be shared between child documents.
Usually, this behaviour makes perfect sense
because each child document contain an essential part of the document.
However, in some situations it may be desirable to compose
a document from a collection of parts
without having mandatory page breaks between then.
For this case, the package
provides a mechanism to include parts
by |\input| which can also be processed individually.
However, by construction this mechanism
requires manual handling of the content to be output.

%%%%%%%%%%%%%%%%%%%%%%%%%%%%%%%%%%%%%%%%
\DescribeMacro{\ifchilddocmanual}
The main file should be prepared as usual, see \secref{sec:include}.
However, the document body must make a distinction
between processing of an individual part and of the main document, e.g.:
%
\begin{center}
\begin{tabular}{l}
|\ifchilddocmanual|\\
|\input{\childdocname}|\\
|\||else|\\
\textit{document body with }|\input{|\textit{part}|}|\\
|\||fi|
\end{tabular}
\end{center}
%
The conditional |\ifchilddocmanual| is true whenever
a part to be included by |\input| is being compiled,
and the name of the part is stored in |\childdocname|.

%%%%%%%%%%%%%%%%%%%%%%%%%%%%%%%%%%%%%%%%
\DescribeMacro{\childdocby}
Each part to be included by |\input| should start with:
%
\begin{center}
\begin{tabular}{l}
|\input{childdoc.def}|\\
|\childdocby{|\textit{main}|}|\\
\end{tabular}
\end{center}
%
The directive |\childdocby| is similar to |\childdocof|
described in \secref{sec:include},
but the subsequent selection of content must be done manually.
To that end, both |\ifchilddoc| and |\ifchilddocmanual|
will be true upon processing of a part,
and the name of the part is stored in |\childdocname|.
Note that |\jobname| will be set to the filename of the current part
so that each part receives an individual |.aux| file
that does not interfere with the |.aux| file(s) of the main document.
This behaviour can be altered by the alternative form
|\childdocby[*]{|\textit{main}|}| (with a non-empty optional argument)
which uses the |.aux| file of the main document
by setting |\jobname| to \textit{main}.

%%%%%%%%%%%%%%%%%%%%%%%%%%%%%%%%%%%%%%%%%%%%%%%%%%%%%%%%%%%%%%%%%%%%%%%%%%%%%%%%
\subsection{Driver Development}
\label{sec:driver}

The \textsf{childdoc} mechanism can also be use for the development
of definition files such as \LaTeX{} styles or classes.
This case differs from the above setup with multiple parts
included by |\include| in that no |\includeonly| should be invoked.
This can be achieved by starting the include file
(before |\ProvidesPackage|) with:
%
\begin{center}
\begin{tabular}{l}
|\input{childdoc.def}|\\
|\childdocforward{|\textit{main}|}|\\
\end{tabular}
\end{center}
%
or alternatively with:
%
\begin{center}
\begin{tabular}{l}
|\input{childdoc.def}|\\
|\childdocby{|\textit{main}|}|\\
\end{tabular}
\end{center}
%
Both forms have slightly different effects as described above.
The main file is prepared as usual, see \secref{sec:include}.

%%%%%%%%%%%%%%%%%%%%%%%%%%%%%%%%%%%%%%%%%%%%%%%%%%%%%%%%%%%%%%%%%%%%%%%%%%%%%%%%
\subsection{Legacy Detection}
\label{sec:detection}

The directive |\childdocmain| in the main file can detect
whether the complete document or merely a child is to be compiled
even without using the directive |\childdocof|.
This method is deprecated because it is less robust
and there is no compelling reason to use it;
it is merely provided for backward compatibility
and it may be removed in future versions.

If the detection mechanism is to be used,
it is mandatory to correctly specify
the filename of the main file as the argument of |\childdocmain|:
%
\begin{center}
\begin{tabular}{l}
|\input{childdoc.def}|\\
|\childdocmain{|\textit{main}|}|\\
\end{tabular}
\end{center}
%
If |\jobname| does not match the argument \textit{main} of |\childdocmain|,
it is assumed that |\jobname| points to the child file to be compiled.
When using |\childdocmain| with the main file specified as argument,
it suffices to start a child file
with just |\input{|\textit{main}|}|
without loading of the package and using |\childdocof|.
If instead all processing is done
with the appropriate \textsf{childdoc} directives,
the argument of \textit{main} of |\childdocmain| can be empty.

An alternative version of the command line processing described
in \secref{sec:commandline} using the detection mechanism reads:
%
\begin{center}
|... -jobname "|\textit{target}|" "|[\textit{flags}]%
[|\def\jobname{|\textit{dest}|}|]|\input{|\textit{main}|}"|
\end{center}

%%%%%%%%%%%%%%%%%%%%%%%%%%%%%%%%%%%%%%%%%%%%%%%%%%%%%%%%%%%%%%%%%%%%%%%%%%%%%%%%
\subsection{Manual Code}
\label{sec:manual}

In case one cannot be certain whether the definitions file |childdoc.def|
is installed on the target \TeX{} distribution
and one prefers not to ship it,
it is conceivable to paste a few relevant commands into the sources.

To that end, drop all statements |\input{childdoc.def}|
and perform the replacements as outlined below.
Instead of |\childdocmain{|\textit{main}|}| add the following code
to the top of the main file:
%
\begin{center}
\begin{tabular}{l}
|\||ifdefined\childdocname\endinput\||fi\newif\ifchilddoc|\\
|\edef\childdocname{\scantokens\expandafter{\jobname\noexpand}}|\\
|\def\childdocmain{|\textit{main}|}\||ifx\childdocmain\childdocname\||else|\\
|\childdoctrue\includeonly{\childdocname}\let\jobname\childdocmain\||fi|\\
\end{tabular}
\end{center}
%
Instead of |\childdocof{|\textit{main}|}| just include the main file
at the top of each child file:
%
\begin{center}
|\input{|\textit{main}|}|
\end{center}
%
A simple redirection |\childdocforward{|\textit{dest}|}| is achieved by:
%
\begin{center}
|\def\jobname{|\textit{dest}|}\input{\jobname}|
\end{center}
%
The redirection with prefix
|\childdocforwardprefix[|\textit{prefix}|]{|\textit{dest}|}|
is accomplished by:
%
\begin{center}
\begin{tabular}{l}
|{\edef\jobname{\scantokens\expandafter{\jobname\noexpand}}|\\
|\def\redirectjob |\textit{prefix}|#1~~~{\gdef\jobname{|\textit{dest}|#1}}|\\
|\expandafter\redirectjob\jobname~~~}\input{\jobname}|
\end{tabular}
\end{center}

In an alternative approach,
child documents can be compiled by a specific command line
without additional code or specific definitions:
%
\begin{center}
|... -jobname "|\textit{target}|" "|[\textit{flags}]%
|\includeonly{|\textit{dest}|}\input{|\textit{main}|}"|
\end{center}
%

%%%%%%%%%%%%%%%%%%%%%%%%%%%%%%%%%%%%%%%%%%%%%%%%%%%%%%%%%%%%%%%%%%%%%%%%%%%%%%%%
%%%%%%%%%%%%%%%%%%%%%%%%%%%%%%%%%%%%%%%%%%%%%%%%%%%%%%%%%%%%%%%%%%%%%%%%%%%%%%%%
\section{Information}

%%%%%%%%%%%%%%%%%%%%%%%%%%%%%%%%%%%%%%%%%%%%%%%%%%%%%%%%%%%%%%%%%%%%%%%%%%%%%%%%
\subsection{Copyright}

Copyright \copyright{} 2017--2018 Niklas Beisert

This work may be distributed and/or modified under the
conditions of the \LaTeX{} Project Public License, either version 1.3
of this license or (at your option) any later version.
The latest version of this license is in
  \url{http://www.latex-project.org/lppl.txt}
and version 1.3 or later is part of all distributions of \LaTeX{}
version 2005/12/01 or later.

This work has the LPPL maintenance status `maintained'.

The Current Maintainer of this work is Niklas Beisert.

This work consists of the files |README.txt|, |childdoc.ins| and |childdoc.dtx|
as well as the derived files |childdoc.def|, |cdocsamp.tex|
with |cdocsch1.tex|, |cdocsch2.tex|, |cdocspt3.tex|, |cdocspt4.tex|,
|cdocsdrf.tex|, |cdocsfn1.tex|, |cdocsfn2.tex|
as well as |childdoc.pdf|.

%%%%%%%%%%%%%%%%%%%%%%%%%%%%%%%%%%%%%%%%%%%%%%%%%%%%%%%%%%%%%%%%%%%%%%%%%%%%%%%%
\subsection{Files and Installation}

The package consists of the files:
%
\begin{center}
\begin{tabular}{ll}
    |README.txt|   & readme file \\
    |childdoc.ins| & installation file \\
    |childdoc.dtx| & source file \\
    |childdoc.def| & definition file \\
    |cdocsamp.tex| & sample main file \\
    |cdocsch1.tex| & sample include file \\
    |cdocsch2.tex| & sample include file \\
    |cdocspt3.tex| & sample part file \\
    |cdocspt4.tex| & sample part file \\
    |cdocsdrf.tex| & sample redirection file \\
    |cdocsfn1.tex| & sample redirection file \\
    |cdocsfn2.tex| & sample redirection file \\
    |childdoc.pdf| & manual
\end{tabular}
\end{center}
%
The distribution consists of the files
|README.txt|, |childdoc.ins| and |childdoc.dtx|.
%
\begin{itemize}
\item
Run (pdf)\LaTeX{} on |childdoc.dtx|
to compile the manual |childdoc.pdf| (this file).
\item
Run \LaTeX{} on |childdoc.ins| to create the definitions file |childdoc.def|
and the sample |cdocsamp.tex| with include files
|cdocsch1.tex|, |cdocsch2.tex|, |cdocspt3.tex|, |cdocspt4.tex|,
|cdocsdrf.tex|, |cdocsfn1.tex|, |cdocsfn2.tex|.
Then copy the file |childdoc.def| to an appropriate directory of your \LaTeX{}
distribution, e.g.\ \textit{texmf-root}|/tex/latex/childdoc|.
\end{itemize}

%%%%%%%%%%%%%%%%%%%%%%%%%%%%%%%%%%%%%%%%%%%%%%%%%%%%%%%%%%%%%%%%%%%%%%%%%%%%%%%%
\subsection{Related CTAN Packages}

There are several other packages which offer a similar functionality:
%
\begin{itemize}
\item
The packages
\href{http://ctan.org/pkg/docmute}{\textsf{docmute}},
\href{http://ctan.org/pkg/includex}{\textsf{includex}} and
\href{http://ctan.org/pkg/standalone}{\textsf{standalone}}
provide commands to include only the document body of
a child file thus allowing both files to be compiled individually.
\item
The packages \href{http://ctan.org/pkg/subdocs}{\textsf{subdocs}}
and \href{http://ctan.org/pkg/subfiles}{\textsf{subfiles}}
provide structures in which the main and child documents can be
encapsulated and allowing them to be compiled individually.
The inclusion mechanism is different from the conventional |\include|.
\item
The package \href{http://ctan.org/pkg/combine}{\textsf{combine}}
is an elaborate solution to combine several documents into one.
\end{itemize}
%
See also the CTAN topic \href{http://ctan.org/topic/subdocs}{\textsf{subdocs}}
for further related packages.
The present package differs from the above solutions in that
a document structure constructed with the conventional |\include| mechanism
just needs two extra commands at the top of every file
such that all constituent files can be compiled individually.

%%%%%%%%%%%%%%%%%%%%%%%%%%%%%%%%%%%%%%%%%%%%%%%%%%%%%%%%%%%%%%%%%%%%%%%%%%%%%%%%
%\subsection{Feature Suggestions}
%
%The following is a list of features which may be useful for future
%versions of this package:
%%
%\begin{itemize}
%\item
%\ldots
%\end{itemize}

%%%%%%%%%%%%%%%%%%%%%%%%%%%%%%%%%%%%%%%%%%%%%%%%%%%%%%%%%%%%%%%%%%%%%%%%%%%%%%%%
\subsection{Revision History}

%%%%%%%%%%%%%%%%%%%%%%%%%%%%%%%%%%%%%%%%
\paragraph{v2.0:} 2018/12/30

\begin{itemize}
\item
immediate forward processing
\item
added |\childdocby| mechanism
\item
manual restructured
\end{itemize}

%%%%%%%%%%%%%%%%%%%%%%%%%%%%%%%%%%%%%%%%
\paragraph{v1.6:} 2018/01/17

\begin{itemize}
\item
application for development of include files
\item
corrections to manual
\end{itemize}

%%%%%%%%%%%%%%%%%%%%%%%%%%%%%%%%%%%%%%%%
\paragraph{v1.5:} 2017/05/21

\begin{itemize}
\item
more complete structuring introduced
\item
|\childdocof| introduced
\item
|\childdoc| renamed to |\childdocmain|
\item
|\childredirect| renamed to |\childdocforward| and |\childdocforwardprefix|
and functionality expanded
\end{itemize}

%%%%%%%%%%%%%%%%%%%%%%%%%%%%%%%%%%%%%%%%
\paragraph{v1.0:} 2017/04/27

\begin{itemize}
\item
manual and install package
\item
first version published on CTAN
\end{itemize}

%%%%%%%%%%%%%%%%%%%%%%%%%%%%%%%%%%%%%%%%
\paragraph{v0.6:} 2017/04/26

\begin{itemize}
\item
redirection mechanism added
\end{itemize}

%%%%%%%%%%%%%%%%%%%%%%%%%%%%%%%%%%%%%%%%
\paragraph{v0.5:} 2017/04/26

\begin{itemize}
\item
functionality in definition file
\end{itemize}


%%%%%%%%%%%%%%%%%%%%%%%%%%%%%%%%%%%%%%%%%%%%%%%%%%%%%%%%%%%%%%%%%%%%%%%%%%%%%%%%
%%%%%%%%%%%%%%%%%%%%%%%%%%%%%%%%%%%%%%%%%%%%%%%%%%%%%%%%%%%%%%%%%%%%%%%%%%%%%%%%
%%%%%%%%%%%%%%%%%%%%%%%%%%%%%%%%%%%%%%%%%%%%%%%%%%%%%%%%%%%%%%%%%%%%%%%%%%%%%%%%
\appendix

\settowidth\MacroIndent{\rmfamily\scriptsize 000\ }

 \DocInput{childdoc.dtx}

\end{document}
%</driver>
% \fi
%
% %%%%%%%%%%%%%%%%%%%%%%%%%%%%%%%%%%%%%%%%%%%%%%%%%%%%%%%%%%%%%%%%%%%%%%%%%%%%%%
% %%%%%%%%%%%%%%%%%%%%%%%%%%%%%%%%%%%%%%%%%%%%%%%%%%%%%%%%%%%%%%%%%%%%%%%%%%%%%%
% \section{Sample}
%\iffalse
%<*samplemain>
%\fi
%
% The following presents a sample document
% with two chapters, two parts, a title page,
% a compile flag as well as three forwarding files to set the flag.
% It consists of eight |.tex| files:
% \begin{center}
% \begin{tabular}{ll}
% |cdocsamp.tex|&main file\\
% |cdocsch1.tex|&include file for chapter 1\\
% |cdocsch2.tex|&include file for chapter 2\\
% |cdocspt3.tex|&include file for part 3\\
% |cdocspt4.tex|&include file for part 4\\
% |cdocsdrf.tex|&forwarding file for main file in draft mode\\
% |cdocsfi1.tex|&forwarding file for final version of chapter 1\\
% |cdocsfi2.tex|&forwarding file for final version of chapter 2\\
% \end{tabular}
% \end{center}
% Each of the eight files can be compiled directly by the \LaTeX{} compiler.
%
% %%%%%%%%%%%%%%%%%%%%%%%%%%%%%%%%%%%%%%
% \paragraph{Main File.}
%
% The main file is called |cdocsamp.tex|.
%
% Load the \textsf{childdoc} definitions and
% declare the filename for the main document:
%    \begin{macrocode}
\input{childdoc.def}
\childdocmain{}
%    \end{macrocode}

% Optional override for |\version| flag:
%    \begin{macrocode}
%%\ifchilddoc\else\providecommand{\version}{draft}\fi
%    \end{macrocode}

% Define the default values for the |\version| flag
% (|final| for the main file and |draft| for childs):
%    \begin{macrocode}
\ifchilddoc
\providecommand{\version}{draft}
\else
\providecommand{\version}{final}
\fi
%    \end{macrocode}

% Load the standard document class:
%    \begin{macrocode}
\documentclass[12pt]{article}
%    \end{macrocode}

% Start the document body:
%    \begin{macrocode}
\begin{document}
%    \end{macrocode}

% Declare a title page.
% Print title, part of document being processed and version flag:
%    \begin{macrocode}
\addtocounter{page}{-1}
\begin{center}
{\LARGE\bfseries{}childdoc example\par}
\vspace{1cm}
\ifchilddoc
\ifchilddocmanual part\else chapter\fi:
`\childdocname' of `\childdocjob'\par
\else
main document: `\childdocjob'\par
\fi
version: \version\par
\end{center}
\newpage
%    \end{macrocode}

% Manually include selected file,
% otherwise process as usual:
%    \begin{macrocode}
\ifchilddocmanual
\section*{part `\childdocname'}
\input{\childdocname}
\else
%    \end{macrocode}

% Include the two chapters:
%    \begin{macrocode}
\include{cdocsch1}
\include{cdocsch2}
%    \end{macrocode}

% Include the two parts unless only chapters should be displayed:
%    \begin{macrocode}
\ifchilddoc\else
\section{part three}
\input{cdocspt3}
\section{part four}
\input{cdocspt4}
\fi
%    \end{macrocode}

% Process as usual until here:
%    \begin{macrocode}
\fi
%    \end{macrocode}

% End of document body:
%    \begin{macrocode}
\end{document}
%    \end{macrocode}
%\iffalse
%</samplemain>
%\fi
%
% %%%%%%%%%%%%%%%%%%%%%%%%%%%%%%%%%%%%%%
% \paragraph{Chapter Include Files.}
%
% The include files are called |cdocsch1.tex| and |cdocsch2.tex|.
%
%\iffalse
%<*samplechap1|samplechap2>
%\fi

% Optional override for |\version| flag:
%    \begin{macrocode}
%%\providecommand{\version}{final}
%    \end{macrocode}

% Include the main document:
%    \begin{macrocode}
\input{childdoc.def}
\childdocof{cdocsamp}
%    \end{macrocode}

%\iffalse
%</samplechap1|samplechap2>
%\fi
%
%\iffalse
%<*samplechap1>
%\fi
% Some text for chapter 1:
%    \begin{macrocode}
\section{one}
some text in chapter one
%    \end{macrocode}

%\iffalse
%</samplechap1>
%\fi
% Some text for chapter 2:
%\iffalse
%<*samplechap2>
%\fi
%    \begin{macrocode}
\section{two}
more text in chapter two
%    \end{macrocode}

%\iffalse
%</samplechap2>
%\fi
%
% %%%%%%%%%%%%%%%%%%%%%%%%%%%%%%%%%%%%%%
% \paragraph{Part Include Files.}
%
% The include files are called |cdocspt3.tex| and |cdocspt4.tex|.
%
%\iffalse
%<*samplepart3|samplepart4>
%\fi

% Optional override for |\version| flag:
%    \begin{macrocode}
%%\providecommand{\version}{final}
%    \end{macrocode}

% Include the main document:
%    \begin{macrocode}
\input{childdoc.def}
\childdocby{cdocsamp}
%    \end{macrocode}

%\iffalse
%</samplepart3|samplepart4>
%\fi
%
%\iffalse
%<*samplepart3>
%\fi
% Some text for part 3:
%    \begin{macrocode}
some text in part three
%    \end{macrocode}

%\iffalse
%</samplepart3>
%\fi
% Some text for part 4:
%\iffalse
%<*samplepart4>
%\fi
%    \begin{macrocode}
more text in part four
%    \end{macrocode}

%\iffalse
%</samplepart4>
%\fi
%
% %%%%%%%%%%%%%%%%%%%%%%%%%%%%%%%%%%%%%%
% \paragraph{Forwarding for a Complete Draft.}
%
% The following forwarding file |cdocsdrf.tex|
% compiles the main document in draft mode:
%\iffalse
%<*sampledraft>
%\fi
%    \begin{macrocode}
\def\version{draft}
\input{childdoc.def}
\childdocforward{cdocsamp}
%    \end{macrocode}

%\iffalse
%</sampledraft>
%\fi
%
% %%%%%%%%%%%%%%%%%%%%%%%%%%%%%%%%%%%%%%
% \paragraph{Forwarding for Final Version of the Chapters.}
%
% The following forwarding files |cdocsfn1.tex| and |cdocsfn2.tex|
% (with identical content)
% compile the final versions of the child documents
% |cdocsch1.tex| and |cdocsch2.tex|, respectively:
%\iffalse
%<*samplefinal>
%\fi
%    \begin{macrocode}
\def\version{final}
\input{childdoc.def}
\childdocforwardprefix[cdocsamp]{cdocsfn}{cdocsch}
%    \end{macrocode}

%\iffalse
%</samplefinal>
%\fi
%
% %%%%%%%%%%%%%%%%%%%%%%%%%%%%%%%%%%%%%%
% \paragraph{Command Line Processing.}
%
% The following three command lines generate the output files
% |cdocscld|, |cdocscl1| and |cdocscl2|
% which should be identical to
% |cdocsdrf|, |cdocsch1| and |cdocsfn2|, respectively:
% \begin{center}
% \begin{tabular}{l}
% |latex -jobname cdocscld \|\\
% |  "\def\version{draft}\input{childdoc.def}\childdocforward{cdocsamp}"|\\
% |latex -jobname cdocscl1 \|\\
% |  "\input{childdoc.def}\childdocforward[cdocsamp]{cdocsch1}"|\\
% |latex -jobname cdocscl2 \|\\
% |  "\def\version{final}\input{childdoc.def}\childdocforward{cdocsch2}"|
% \end{tabular}
% \end{center}
% Note that the trailing backslash on each first line
% merely continues the input to the second line
% (for convenient cut ant paste).
% Furthermore, the command |latex| can be replaced by any
% of its alternative versions such as |pdflatex|.
%
% %%%%%%%%%%%%%%%%%%%%%%%%%%%%%%%%%%%%%%%%%%%%%%%%%%%%%%%%%%%%%%%%%%%%%%%%%%%%%%
% %%%%%%%%%%%%%%%%%%%%%%%%%%%%%%%%%%%%%%%%%%%%%%%%%%%%%%%%%%%%%%%%%%%%%%%%%%%%%%
% \section{Implementation}
%\iffalse
%<*package>
%\fi
%
% This section describes the definitions file |childdoc.def|.

% The definitions cannot be loaded using |\usepackage| or |\RequirePackage|
% which has a mechanism to prevent loading a style file more than once.
% When loading the definitions by means of |\input|
% multiple instances have to be prevented manually:
%\iffalse
%This code needs to be before the `\ProvidesFile' directive
%which is defined at the beginning of this file.
%Therefore it is also placed there and commented out here.
%</package>
%<*discard>
%\fi
%    \begin{macrocode}
\ifdefined\childdocmain\endinput\fi
%    \end{macrocode}
%\iffalse
%</discard>
%<*package>
%\fi
%
% \macro{\ifchilddoc}
% \macro{\ifchilddocmanual}
% The conditional |\ifchilddoc| tells whether a
% child (true) or main (false) document is being compiled.
% The conditional |\ifchilddocmanual| tells whether
% the |\includeonly| mechanism is used (false) or
% the selection of child files must be performed manually (true).
% The definitions initialise to false:
%    \begin{macrocode}
\newif\ifchilddoc
\newif\ifchilddocmanual
%    \end{macrocode}

% \macro{\childdocname}
% \macro{\childdocjob}
% The macro |\childdocname| stores the name of the main document
% to be compiled. The macro |\childdocjob| stores the name of
% the document on which the \LaTeX{} compiler was originally invoked.
% The content of |\jobname| cannot be compared
% to filenames specified in the source due to different catcodes.
% The following code rescans |\jobname|, stores the result
% in |\childdocname| and saves a copy in |\childdocjob|:
%    \begin{macrocode}
\edef\childdocname{\scantokens\expandafter{\jobname\noexpand}}
\let\childdocjob\childdocname
%    \end{macrocode}

% \macro{\childdocdisable}
% The macro |\childdocdisable| prevents the main file
% from being processed more than once.
% At this stage, the main document command |\childdocmain|
% is assumed to be called once again where it should do nothing.
% Any subsequent call to it should prevent
% a secondary processing of the main document
% It overwrites the forwarding commands
% |\childdocof| and |\childdocforward|
% with empty macros to prevent further inclusions of the main document:
%    \begin{macrocode}
\newcommand{\childdocdisable}
{
  \renewcommand{\childdocmain}[1]{\renewcommand{\childdocmain}[1]{\endinput}}
  \renewcommand{\childdocof}[1]{}
  \renewcommand{\childdocby}[2][]{}
  \renewcommand{\childdocforward}[2][]{}
  \renewcommand{\childdocdisable}{}
}
%    \end{macrocode}

% \macro{\childdocmain}
% The macro |\childdocmain| is to be called at the top of the main file
% with nothing or the main filename (without extension) as argument.
% First, it breaks loops.
% If the argument is not empty and does not match |\childdocname|
% (which is set by the first inclusion of |childdoc.def|),
% |\ifchilddoc| is set to true, |\includeonly| is applied to the child file
% and |\jobname| is set to the main file
% (for proper handling of |.aux| files):
%    \begin{macrocode}
\newcommand{\childdocmain}[1]
{
  \childdocdisable\childdocmain{}
  \if?#1?\else
    \begingroup
      \def\childdoctmp{#1}
      \ifx\childdoctmp\childdocname
        \def\childdoctmp{}
      \else
        \def\childdoctmp
        {
          \childdoctrue
          \includeonly{\childdocname}
          \def\childdocjob{#1}
          \def\jobname{#1}
        }
      \fi
      \expandafter
    \endgroup
    \childdoctmp
  \fi
}
%    \end{macrocode}

% \macro{\childdocof}
% The command |\childdocof| redirects
% compilation to the main file |#1|.
%    \begin{macrocode}
\newcommand{\childdocof}[1]
{
  \childdocdisable
  \childdoctrue
  \includeonly{\childdocname}
  \def\jobname{#1}
  \def\childdocjob{#1}
  \input{#1}
}
%    \end{macrocode}

% \macro{\childdocby}
% The command |\childdocby| ....
%    \begin{macrocode}
\newcommand{\childdocby}[2][]
{
  \childdocdisable
  \childdoctrue
  \childdocmanualtrue
  \if?#1?\else
    \def\jobname{#2}
  \fi
  \def\childdocjob{#2}
  \input{#2}
  \endinput
}
%    \end{macrocode}

% \macro{\childdocforward}
% The command |\childdocforward| redirects
% compilation to the main file or
% (if the optional argument is given) a child file.
% Parameters are set as if the main file
% or a child file starting with |\childdocof| was compiled.
% Then compilation is handed over to the main file:
%    \begin{macrocode}
\newcommand{\childdocforward}[2][]
{
  \begingroup
    \if?#1?
      \def\childdoctmp
      {
        \def\childdocname{#2}
        \def\childdocjob{#2}
        \def\jobname{#2}
        \input{#2}
        \endinput
      }
    \else
      \def\childdoctmp
      {
        \childdocdisable
        \def\childdocname{#2}
        \childdoctrue
        \includeonly{#2}
        \def\childdocjob{#1}
        \def\jobname{#1}
        \input{#1}
        \endinput
      }
    \fi
    \expandafter
  \endgroup
  \childdoctmp
}
%    \end{macrocode}

% \macro{\childdocforwardprefix}
% The command |\childdocforwardprefix| redirects
% compilation to the main or a child file by means of a pattern.
% The prefix |#1| in the current filename is replaced by |#2|
% and the suffix of the current filename is kept
% (it is assumed that the filename does not contain the substring `|~~~|'
% which is used as a delimiter).
% Compilation is handed over to the new file by |\childdocforward|:
%    \begin{macrocode}
\newcommand{\childdocforwardprefix}[3][]
{
  \begingroup
    \def\childdocextract #2##1~~~{\def\childdoctmp{\childdocforward[#1]{#3##1}}}
    \expandafter\childdocextract\childdocname~~~
    \expandafter
  \endgroup
  \childdoctmp
}
%    \end{macrocode}

% \macro{\childdoc}
% The deprecated macro |\childdoc| is a legacy version of |\childdocmain|:
%    \begin{macrocode}
\newcommand{\childdoc}{\childdocmain}
%    \end{macrocode}

% \macro{\childdocredirect}
% The deprecated macro |\childdocredirect| is a legacy version
% of |\childdocforward| and |\childdocforwardprefix|:
%    \begin{macrocode}
\newcommand{\childdocredirect}[2][]
{
  \begingroup
    \if?#1?
      \def\childdoctmp{\childdocforward{#2}}
    \else
      \def\childdoctmp{\childdocforwardprefix{#1}{#2}}
    \fi
    \expandafter
  \endgroup
  \childdoctmp
}
%    \end{macrocode}

%\iffalse
%</package>
%\fi
%
\endinput

\childdocforwardprefix[cdocsamp]{cdocsfn}{cdocsch}
%    \end{macrocode}

%\iffalse
%</samplefinal>
%\fi
%
% %%%%%%%%%%%%%%%%%%%%%%%%%%%%%%%%%%%%%%
% \paragraph{Command Line Processing.}
%
% The following three command lines generate the output files
% |cdocscld|, |cdocscl1| and |cdocscl2|
% which should be identical to
% |cdocsdrf|, |cdocsch1| and |cdocsfn2|, respectively:
% \begin{center}
% \begin{tabular}{l}
% |latex -jobname cdocscld \|\\
% |  "\def\version{draft}% \iffalse
%
% childdoc.dtx Copyright (C) 2017-2018 Niklas Beisert
%
% This work may be distributed and/or modified under the
% conditions of the LaTeX Project Public License, either version 1.3
% of this license or (at your option) any later version.
% The latest version of this license is in
%   http://www.latex-project.org/lppl.txt
% and version 1.3 or later is part of all distributions of LaTeX
% version 2005/12/01 or later.
%
% This work has the LPPL maintenance status `maintained'.
%
% The Current Maintainer of this work is Niklas Beisert.
%
% This work consists of the files childdoc.dtx and childdoc.ins
% and the derived files childdoc.def and cdocsamp.tex with
% cdocsch1.tex, cdocsch2.tex, cdocsdrf.tex, cdocsfn1.tex, cdocsfn2.tex.
%
%<package>\ifdefined\childdocmain\endinput\fi
%<package>\ProvidesFile{childdoc.def}[2018/12/30 v2.0 child document driver]
%<samplemain>\ProvidesFile{cdocsamp.tex}[2018/12/30 v2.0 sample for childdoc]
%<*driver>
%\ProvidesFile{childdoc.drv}[2018/12/30 v2.0 childdoc reference manual file]
\PassOptionsToClass{10pt,a4paper}{article}
\documentclass{ltxdoc}

\usepackage[margin=35mm]{geometry}
\usepackage{hyperref}
\usepackage{hyperxmp}
\usepackage[usenames]{color}

\hypersetup{colorlinks=true}
\hypersetup{pdfstartview=FitH}
\hypersetup{pdfpagemode=UseNone}
\hypersetup{pdfsource={}}
\hypersetup{pdflang={en-UK}}
\hypersetup{pdfcopyright={Copyright 2017-2018 Niklas Beisert.
  This work may be distributed and/or modified under the
  conditions of the LaTeX Project Public License, either version 1.3
  of this license or (at your option) any later version.}}
\hypersetup{pdflicenseurl={http://www.latex-project.org/lppl.txt}}
\hypersetup{pdfcontactaddress={ETH Zurich, ITP, HIT K,
  Wolfgang-Pauli-Strasse 27}}
\hypersetup{pdfcontactpostcode={8093}}
\hypersetup{pdfcontactcity={Zurich}}
\hypersetup{pdfcontactcountry={Switzerland}}
\hypersetup{pdfcontactemail={nbeisert@itp.phys.ethz.ch}}
\hypersetup{pdfcontacturl={http://people.phys.ethz.ch/\xmptilde nbeisert/}}

\newcommand{\secref}[1]{\hyperref[#1]{section \ref*{#1}}}

\parskip1ex
\parindent0pt
\let\olditemize\itemize
\def\itemize{\olditemize\parskip0pt}

\begin{document}

\title{The \textsf{childdoc} Package}
\hypersetup{pdftitle={The childdoc Package}}
\author{Niklas Beisert\\[2ex]
  Institut f\"ur Theoretische Physik\\
  Eidgen\"ossische Technische Hochschule Z\"urich\\
  Wolfgang-Pauli-Strasse 27, 8093 Z\"urich, Switzerland\\[1ex]
  \href{mailto:nbeisert@itp.phys.ethz.ch}
  {\texttt{nbeisert@itp.phys.ethz.ch}}}
\hypersetup{pdfauthor={Niklas Beisert}}
\hypersetup{pdfsubject={Manual for the LaTeX2e Package childdoc}}
\date{30 December 2018, \textsf{v2.0}}
\maketitle

\begin{abstract}\noindent
\textsf{childdoc} is a \LaTeXe{} package
that enables the direct compilation
of document sections included by |\include|
to individual files.
\end{abstract}

\begingroup
\parskip0ex
\tableofcontents
\endgroup

%%%%%%%%%%%%%%%%%%%%%%%%%%%%%%%%%%%%%%%%%%%%%%%%%%%%%%%%%%%%%%%%%%%%%%%%%%%%%%%%
%%%%%%%%%%%%%%%%%%%%%%%%%%%%%%%%%%%%%%%%%%%%%%%%%%%%%%%%%%%%%%%%%%%%%%%%%%%%%%%%
\section{Introduction}

\LaTeX{} provides a mechanism to structure a large document (such as a book)
into a main file and several child files (containing the chapters)
using the |\include| command.
This mechanism is beneficial for documents
which span hundreds of pages in order to
make the source file(s) more manageable.
Moreover, compilation can be restricted to
selected child files by means of the |\includeonly| command.
The latter feature can be used to reduce the compilation time while editing
(this was significantly more useful in the earlier days of \LaTeX{})
or to generate a smaller document which is easier to navigate.
Another application of |\includeonly| is to generate
documents consisting of selected parts of the complete document.

However, there are a few drawbacks of the plain |\include| mechanism:
\begin{itemize}
\item
The child files cannot be compiled on their own,
they can only be compiled via the main file.
A naive editing environment
(such as a text editor with an option
to have the current file processed by \LaTeX)
may require one to switch to the main file before compiling;
attempting to compile the child file produces errors.
\item
The main file must be modified (each time)
to adjust the |\includeonly| command
to the present needs. This easily leaves the main file in a messy state.
\item
The generated document will always carry the filename
of the main document. This is inconvenient if
several child files are to be compiled and
to be kept for distribution.
\end{itemize}

The present package provides a simple interface
to make child files individually compilable by \LaTeX{}.
Compiling a child file then has the same effect as compiling
the main file with an |\includeonly| command
to select the appropriate child.
Moreover the generated document will carry the name of the child
rather than the main file.
This resolves all three above issues.

This feature is meant to make the editing of books,
thesis documents and lecture notes somewhat more convenient.
However, the package can also be used efficiently for
composing a series of documents (such as exercise sheets)
which are typically distributed individually.
It then assists the author in generating the individual documents
(potentially in different versions)
as well as a document containing the collected series.
Another application is in developing style files
or other kinds of included material
where compilation of the style file could redirect
to a sample or test file.

%%%%%%%%%%%%%%%%%%%%%%%%%%%%%%%%%%%%%%%%%%%%%%%%%%%%%%%%%%%%%%%%%%%%%%%%%%%%%%%%
%%%%%%%%%%%%%%%%%%%%%%%%%%%%%%%%%%%%%%%%%%%%%%%%%%%%%%%%%%%%%%%%%%%%%%%%%%%%%%%%
\section{Usage}

First of all, the package \textsf{childdoc} is \emph{not} a standard
\LaTeXe{} |.sty| style file! Therefore it needs to be invoked in
a non-standard way.

%%%%%%%%%%%%%%%%%%%%%%%%%%%%%%%%%%%%%%%%%%%%%%%%%%%%%%%%%%%%%%%%%%%%%%%%%%%%%%%%
\subsection{Included Files}
\label{sec:include}

%%%%%%%%%%%%%%%%%%%%%%%%%%%%%%%%%%%%%%%%
\DescribeMacro{\childdocmain}
To use the package, add the commands
\begin{center}
\begin{tabular}{l}
|\input{childdoc.def}|\\
|\childdocmain{}|\\
\end{tabular}
\end{center}
at the very top of the main \LaTeX{} file,
in particular \emph{before} the |\documentclass| statement!
The argument of |\childdocmain| should be left empty
(but it must be present).

%%%%%%%%%%%%%%%%%%%%%%%%%%%%%%%%%%%%%%%%
\DescribeMacro{\childdocof}
Furthermore, add the commands
\begin{center}
\begin{tabular}{l}
|\input{childdoc.def}|\\
|\childdocof{|\textit{main}|}|\\
\end{tabular}
\end{center}
at the top of every child file \textit{child}
which is included by |\include{|\textit{child}|}|
from within the main file
(or at least for those files to be compiled individually).
The argument \textit{main} must be the filename of the main file.

There are a couple of
considerations in setting up the main and child documents:

%%%%%%%%%%%%%%%%%%%%%%%%%%%%%%%%%%%%%%%%
\paragraph{Restrictions.}

Please note the following restrictions:
\begin{itemize}
\item
|\childdocmain| must be called with one argument \textit{main}
to ensure compatibility with earlier version of the package.
It must either be empty (|\childdocmain{}|)
or precisely match the filename of the main file in which it is specified.
See \secref{sec:detection} for further information.
\item
The filename \textit{main} must be specified without the |.tex| extension.
\item
The filename \textit{main} is case sensitive
(even in case-insensitive file systems)
due to internal string comparison.
\item
The argument \textit{main} should be fully expanded, it cannot be a macro.
\item
Subdirectories and special characters should be avoided in filenames.
\item
The command |\childdocmain{|\textit{main}|}| must be followed by a whitespace.
It should not be followed immediately by another command
or by a comment mark `|%|'.
This is because the \TeX{} parser reads the token immediately following
the argument of |\childdocmain| and puts it
at the beginning of every child section;
however, a white\-space is ignored.
\end{itemize}

%%%%%%%%%%%%%%%%%%%%%%%%%%%%%%%%%%%%%%%%
\paragraph{Content of Main File.}

It is advisable to place all content in the child files included by |\include|.
Any output contained in the main file will appear in all child documents
unless suppressed manually;
it cannot be suppressed automatically by the |\includeonly| directive
and thus should normally be avoided.
A method to include some content in the main file
by means of conditional processing is described in \secref{sec:conditional}.

%%%%%%%%%%%%%%%%%%%%%%%%%%%%%%%%%%%%%%%%
\paragraph{Page Numbering.}

When only a part of the document is compiled,
the appropriate numbering of pages
(as well as other status parameters)
is determined from the |.aux| files.
The latter contain information from previous passes.
However this information needs to propagate through
all intermediate child documents.
Therefore the page numbering in child documents may well
be inconsistent until the complete document is compiled at least once.

A useful (if unconventional) way to always ensure a consistent
page numbering is to restart the numbering in each child document
and denote the pages by `\textit{child}|.|\textit{page}'
where \textit{child} represents the chapter/section number of the child file.
This can be achieved by the command
|\numberwithin{page}{|\textit{child}|}|
of the \textsf{amsmath} package
where \textit{child} can be |chapter| or |section|
depending on the chosen structuring.
Alternatively, one can modify the macro |\thepage| appropriately
and reset the counter |page| at the start of each child file.

%%%%%%%%%%%%%%%%%%%%%%%%%%%%%%%%%%%%%%%%%%%%%%%%%%%%%%%%%%%%%%%%%%%%%%%%%%%%%%%%
\subsection{Conditional Processing}
\label{sec:conditional}

The package provides a mechanism to compile different versions
of a document. To customise the versions further some conditional processing
can come in handy to distinguish which version is being compiled.
The package provides two macros to describe the compilation context:

%%%%%%%%%%%%%%%%%%%%%%%%%%%%%%%%%%%%%%%%
\DescribeMacro{\ifchilddoc}
The conditional |\ifchilddoc| distinguishes between the compilation of
child documents and the main document:
%
\begin{center}
|\ifchilddoc |\textit{child-code}| |[|\||else |\textit{main-code}]| \||fi|
\end{center}

%%%%%%%%%%%%%%%%%%%%%%%%%%%%%%%%%%%%%%%%
\DescribeMacro{\childdocname}
\DescribeMacro{\childdocjob}
The macro |\childdocname| contains the filename (without extension)
of the main or child file being processed.
Note that |\childdocjob| will always contain the name of the main file.

%%%%%%%%%%%%%%%%%%%%%%%%%%%%%%%%%%%%%%%%
\paragraph{Title Page.}

Conditional processing can be used to include a title or banner page
in the main document when proper precautions are taken.
Importantly, the code in the main file should ensure that the page counter
(as well as other status parameters which are stored in the |.aux| files)
takes the same value after the conditional processing.
Otherwise the page numbers may take divergent values
depending on which part is compiled.

For example, a title page could be declared by:
%
\begin{center}
\begin{tabular}{l}
|\ifchilddoc\||else|\\
|\addtocounter{page}{-1}|\\
\textit{code for title page}\\
|\newpage|\\
|\||fi|
\end{tabular}
\end{center}
%
A banner page for the child documents can be generated by:
%
\begin{center}
\begin{tabular}{l}
|\ifchilddoc|\\
|\addtocounter{page}{-1}|\\
\textit{code for banner page}\\
|\newpage|\\
|\||fi|
\end{tabular}
\end{center}
%
Here one could write a message such as:
\begin{center}
|This is the part \childdocname{} of \childdocjob{}.|
\end{center}

%%%%%%%%%%%%%%%%%%%%%%%%%%%%%%%%%%%%%%%%%%%%%%%%%%%%%%%%%%%%%%%%%%%%%%%%%%%%%%%%
\subsection{Flags}
\label{sec:flags}

The package makes it easy to generate different versions
of the main or child documents.
To this end compilation flags can be defined
and assigned different default values.
They will be particularly useful in conjunction
with the forwarding mechanism described in \secref{sec:forward}.

For example, it may be useful to have a flag |\version|
which can be set to |draft| or |final|.
The document source will contain some conditional code
depending on the value of |\version|.
Suppose further, the flag should default to |final| for the main file
and to |draft| for child files
which is a natural assignment for editing the document.
This is achieved by placing the following code
in the preamble of the main document
(below the |\childdocmain| directive):
%
\begin{center}
\begin{tabular}{l}
|\ifchilddoc|\\
|\providecommand{\version}{draft}|\\
|\||else|\\
|\providecommand{\version}{final}|\\
|\||fi|
\end{tabular}
\end{center}
%
The definition by |\providecommand| makes sure
that previous definitions are not overwritten.
Further statements |\providecommand{\version}{...}|
can thus be added before the above code to override it.

For the main file, one might add a line
(between |\childdocmain| and the above block)
%
\begin{center}
|%\ifchilddoc\||else\providecommand{\version}{draft}\||fi|
\end{center}
%
which can be uncommented to produce a draft version.
Likewise one can add a line to the very top of a child file
(above the |\childdocof{|\textit{main}|}| directive)
%
\begin{center}
|%\providecommand{\version}{final}|
\end{center}
%
which can be uncommented to produce the final version of this child document.

%%%%%%%%%%%%%%%%%%%%%%%%%%%%%%%%%%%%%%%%%%%%%%%%%%%%%%%%%%%%%%%%%%%%%%%%%%%%%%%%
\subsection{Forwarding}
\label{sec:forward}

Different versions of the main or child documents
using compilation flags as described in \secref{sec:flags}
can be (permanently) stored in different files
for convenient compilation, viewing and distribution.
To this end, the package defines a command
to pass on compilation to a different file:

%%%%%%%%%%%%%%%%%%%%%%%%%%%%%%%%%%%%%%%%
\DescribeMacro{\childdocforward}
The command |\childdocforward| redirects processing to
another source file:
%
\begin{center}
\begin{tabular}{l}
|\input{childdoc.def}|\\
|\childdocforward[|\textit{main}|]{|\textit{dest}|}|\\
\end{tabular}
\end{center}
%
The argument \textit{dest} is the destination file
(without extension).
It should be the main file or one of the child files.
Note that further \textsf{childdoc} directives
such as |\childdocof| and |\childdocforward|
in the indicated file will be processed in this form.
The optional argument \textit{main}
passes on directly to the main file \textit{main}
while pretending to compile the child \textit{dest}.
This form behaves as if \textit{dest}
issues |\childdocof{|\textit{main}|}| right away,
and no further \textsf{childdoc} directives will be processed.

%%%%%%%%%%%%%%%%%%%%%%%%%%%%%%%%%%%%%%%%
\DescribeMacro{\...prefix}
In the alternative form |\childdocforwardprefix|,
%
\begin{center}
\begin{tabular}{l}
|\input{childdoc.def}|\\
|\childdocforwardprefix[|\textit{main}|]{|\textit{prefix}|}{|\textit{dest}|}|
\end{tabular}
\end{center}
%
the destination file is determined by a pattern
depending on the current file:
To make this work, the current file must be called
`{\textit{prefix}\hspace{0.2em}\textit{suffix}}'
with \textit{prefix} matching precisely the argument.
Processing is then passed on to the file
`{\textit{dest}\hspace{0.2em}\textit{suffix}}'.
Surely, the same effect is achieved by
directly specifying the
argument `{\textit{dest}\hspace{0.2em}\textit{suffix}}'
in the first form.
However, that requires to set up a different file
for each child. With the alternative form of the command
all these files can have exactly the same content
which simplifies setting them up and maintaining them.

For example, the following file |draft.tex|
with a compilation flag |\version| as described in \secref{sec:flags}
compiles the main document as a draft:
%
\begin{center}
\begin{tabular}{l}
|\def\version{draft}|\\
|\input{childdoc.def}|\\
|\childdocforward{|\textit{main}|}|
\end{tabular}
\end{center}
%
Likewise, the following files |final|\textit{nn}|.tex|
compile the final version of the child document
|child|\textit{nn}|.tex|:
%
\begin{center}
\begin{tabular}{l}
|\def\version{final}|\\
|\input{childdoc.def}|\\
|\childdocforwardprefix{final}{child}|
\end{tabular}
\end{center}
%

Note that when several versions of a main file and/or of each child file
are to be generated, it may be convenient to set up a |Makefile| or
shell script to automatise the process.

%%%%%%%%%%%%%%%%%%%%%%%%%%%%%%%%%%%%%%%%%%%%%%%%%%%%%%%%%%%%%%%%%%%%%%%%%%%%%%%%
\subsection{Command Line Processing}
\label{sec:commandline}

The effect of redirection files can also be achieved by invoking
the \LaTeX{} compiler with a more elaborate command line.
Most conveniently this should be done as part
of a shell script or a |Makefile|.

When using \textsf{childdoc} in the main file, the following
command lines effectively perform a redirection
(note that depending on the shell being used,
backslashes may have to be doubled: `|\|' $\to$ `|\\|'):
%
\begin{center}
|... -jobname "|\textit{target}|" |\\|"|[\textit{flags}]%
|\input{childdoc.def}\childdocforward[|\textit{main}|]{|\textit{dest}|}"|
\end{center}
%
Here \textit{target} is the name of the output file,
\textit{main} is the name of the main file
and \textit{dest} is the name of the main or child file to be processed
(all filenames without extensions).
The optional argument \textit{main} can be omitted
if \textit{main} matches \textit{dest}.
Optionally, compilation \textit{flags} can be defined via |\def| commands.
This command line makes the \TeX{} engine believe
it is compiling the file \textit{target}
whose content is specified as the latter parameter.
The provided code then forwards the processing to
\textit{main} or \textit{dest} as described in \secref{sec:forward}.

%%%%%%%%%%%%%%%%%%%%%%%%%%%%%%%%%%%%%%%%%%%%%%%%%%%%%%%%%%%%%%%%%%%%%%%%%%%%%%%%
\subsection{Include by Input}
\label{sec:input}

Including child documents by |\include| has some restrictions by design.
Most notably, the content of a child document always occupies
its own set of pages; pages cannot be shared between child documents.
Usually, this behaviour makes perfect sense
because each child document contain an essential part of the document.
However, in some situations it may be desirable to compose
a document from a collection of parts
without having mandatory page breaks between then.
For this case, the package
provides a mechanism to include parts
by |\input| which can also be processed individually.
However, by construction this mechanism
requires manual handling of the content to be output.

%%%%%%%%%%%%%%%%%%%%%%%%%%%%%%%%%%%%%%%%
\DescribeMacro{\ifchilddocmanual}
The main file should be prepared as usual, see \secref{sec:include}.
However, the document body must make a distinction
between processing of an individual part and of the main document, e.g.:
%
\begin{center}
\begin{tabular}{l}
|\ifchilddocmanual|\\
|\input{\childdocname}|\\
|\||else|\\
\textit{document body with }|\input{|\textit{part}|}|\\
|\||fi|
\end{tabular}
\end{center}
%
The conditional |\ifchilddocmanual| is true whenever
a part to be included by |\input| is being compiled,
and the name of the part is stored in |\childdocname|.

%%%%%%%%%%%%%%%%%%%%%%%%%%%%%%%%%%%%%%%%
\DescribeMacro{\childdocby}
Each part to be included by |\input| should start with:
%
\begin{center}
\begin{tabular}{l}
|\input{childdoc.def}|\\
|\childdocby{|\textit{main}|}|\\
\end{tabular}
\end{center}
%
The directive |\childdocby| is similar to |\childdocof|
described in \secref{sec:include},
but the subsequent selection of content must be done manually.
To that end, both |\ifchilddoc| and |\ifchilddocmanual|
will be true upon processing of a part,
and the name of the part is stored in |\childdocname|.
Note that |\jobname| will be set to the filename of the current part
so that each part receives an individual |.aux| file
that does not interfere with the |.aux| file(s) of the main document.
This behaviour can be altered by the alternative form
|\childdocby[*]{|\textit{main}|}| (with a non-empty optional argument)
which uses the |.aux| file of the main document
by setting |\jobname| to \textit{main}.

%%%%%%%%%%%%%%%%%%%%%%%%%%%%%%%%%%%%%%%%%%%%%%%%%%%%%%%%%%%%%%%%%%%%%%%%%%%%%%%%
\subsection{Driver Development}
\label{sec:driver}

The \textsf{childdoc} mechanism can also be use for the development
of definition files such as \LaTeX{} styles or classes.
This case differs from the above setup with multiple parts
included by |\include| in that no |\includeonly| should be invoked.
This can be achieved by starting the include file
(before |\ProvidesPackage|) with:
%
\begin{center}
\begin{tabular}{l}
|\input{childdoc.def}|\\
|\childdocforward{|\textit{main}|}|\\
\end{tabular}
\end{center}
%
or alternatively with:
%
\begin{center}
\begin{tabular}{l}
|\input{childdoc.def}|\\
|\childdocby{|\textit{main}|}|\\
\end{tabular}
\end{center}
%
Both forms have slightly different effects as described above.
The main file is prepared as usual, see \secref{sec:include}.

%%%%%%%%%%%%%%%%%%%%%%%%%%%%%%%%%%%%%%%%%%%%%%%%%%%%%%%%%%%%%%%%%%%%%%%%%%%%%%%%
\subsection{Legacy Detection}
\label{sec:detection}

The directive |\childdocmain| in the main file can detect
whether the complete document or merely a child is to be compiled
even without using the directive |\childdocof|.
This method is deprecated because it is less robust
and there is no compelling reason to use it;
it is merely provided for backward compatibility
and it may be removed in future versions.

If the detection mechanism is to be used,
it is mandatory to correctly specify
the filename of the main file as the argument of |\childdocmain|:
%
\begin{center}
\begin{tabular}{l}
|\input{childdoc.def}|\\
|\childdocmain{|\textit{main}|}|\\
\end{tabular}
\end{center}
%
If |\jobname| does not match the argument \textit{main} of |\childdocmain|,
it is assumed that |\jobname| points to the child file to be compiled.
When using |\childdocmain| with the main file specified as argument,
it suffices to start a child file
with just |\input{|\textit{main}|}|
without loading of the package and using |\childdocof|.
If instead all processing is done
with the appropriate \textsf{childdoc} directives,
the argument of \textit{main} of |\childdocmain| can be empty.

An alternative version of the command line processing described
in \secref{sec:commandline} using the detection mechanism reads:
%
\begin{center}
|... -jobname "|\textit{target}|" "|[\textit{flags}]%
[|\def\jobname{|\textit{dest}|}|]|\input{|\textit{main}|}"|
\end{center}

%%%%%%%%%%%%%%%%%%%%%%%%%%%%%%%%%%%%%%%%%%%%%%%%%%%%%%%%%%%%%%%%%%%%%%%%%%%%%%%%
\subsection{Manual Code}
\label{sec:manual}

In case one cannot be certain whether the definitions file |childdoc.def|
is installed on the target \TeX{} distribution
and one prefers not to ship it,
it is conceivable to paste a few relevant commands into the sources.

To that end, drop all statements |\input{childdoc.def}|
and perform the replacements as outlined below.
Instead of |\childdocmain{|\textit{main}|}| add the following code
to the top of the main file:
%
\begin{center}
\begin{tabular}{l}
|\||ifdefined\childdocname\endinput\||fi\newif\ifchilddoc|\\
|\edef\childdocname{\scantokens\expandafter{\jobname\noexpand}}|\\
|\def\childdocmain{|\textit{main}|}\||ifx\childdocmain\childdocname\||else|\\
|\childdoctrue\includeonly{\childdocname}\let\jobname\childdocmain\||fi|\\
\end{tabular}
\end{center}
%
Instead of |\childdocof{|\textit{main}|}| just include the main file
at the top of each child file:
%
\begin{center}
|\input{|\textit{main}|}|
\end{center}
%
A simple redirection |\childdocforward{|\textit{dest}|}| is achieved by:
%
\begin{center}
|\def\jobname{|\textit{dest}|}\input{\jobname}|
\end{center}
%
The redirection with prefix
|\childdocforwardprefix[|\textit{prefix}|]{|\textit{dest}|}|
is accomplished by:
%
\begin{center}
\begin{tabular}{l}
|{\edef\jobname{\scantokens\expandafter{\jobname\noexpand}}|\\
|\def\redirectjob |\textit{prefix}|#1~~~{\gdef\jobname{|\textit{dest}|#1}}|\\
|\expandafter\redirectjob\jobname~~~}\input{\jobname}|
\end{tabular}
\end{center}

In an alternative approach,
child documents can be compiled by a specific command line
without additional code or specific definitions:
%
\begin{center}
|... -jobname "|\textit{target}|" "|[\textit{flags}]%
|\includeonly{|\textit{dest}|}\input{|\textit{main}|}"|
\end{center}
%

%%%%%%%%%%%%%%%%%%%%%%%%%%%%%%%%%%%%%%%%%%%%%%%%%%%%%%%%%%%%%%%%%%%%%%%%%%%%%%%%
%%%%%%%%%%%%%%%%%%%%%%%%%%%%%%%%%%%%%%%%%%%%%%%%%%%%%%%%%%%%%%%%%%%%%%%%%%%%%%%%
\section{Information}

%%%%%%%%%%%%%%%%%%%%%%%%%%%%%%%%%%%%%%%%%%%%%%%%%%%%%%%%%%%%%%%%%%%%%%%%%%%%%%%%
\subsection{Copyright}

Copyright \copyright{} 2017--2018 Niklas Beisert

This work may be distributed and/or modified under the
conditions of the \LaTeX{} Project Public License, either version 1.3
of this license or (at your option) any later version.
The latest version of this license is in
  \url{http://www.latex-project.org/lppl.txt}
and version 1.3 or later is part of all distributions of \LaTeX{}
version 2005/12/01 or later.

This work has the LPPL maintenance status `maintained'.

The Current Maintainer of this work is Niklas Beisert.

This work consists of the files |README.txt|, |childdoc.ins| and |childdoc.dtx|
as well as the derived files |childdoc.def|, |cdocsamp.tex|
with |cdocsch1.tex|, |cdocsch2.tex|, |cdocspt3.tex|, |cdocspt4.tex|,
|cdocsdrf.tex|, |cdocsfn1.tex|, |cdocsfn2.tex|
as well as |childdoc.pdf|.

%%%%%%%%%%%%%%%%%%%%%%%%%%%%%%%%%%%%%%%%%%%%%%%%%%%%%%%%%%%%%%%%%%%%%%%%%%%%%%%%
\subsection{Files and Installation}

The package consists of the files:
%
\begin{center}
\begin{tabular}{ll}
    |README.txt|   & readme file \\
    |childdoc.ins| & installation file \\
    |childdoc.dtx| & source file \\
    |childdoc.def| & definition file \\
    |cdocsamp.tex| & sample main file \\
    |cdocsch1.tex| & sample include file \\
    |cdocsch2.tex| & sample include file \\
    |cdocspt3.tex| & sample part file \\
    |cdocspt4.tex| & sample part file \\
    |cdocsdrf.tex| & sample redirection file \\
    |cdocsfn1.tex| & sample redirection file \\
    |cdocsfn2.tex| & sample redirection file \\
    |childdoc.pdf| & manual
\end{tabular}
\end{center}
%
The distribution consists of the files
|README.txt|, |childdoc.ins| and |childdoc.dtx|.
%
\begin{itemize}
\item
Run (pdf)\LaTeX{} on |childdoc.dtx|
to compile the manual |childdoc.pdf| (this file).
\item
Run \LaTeX{} on |childdoc.ins| to create the definitions file |childdoc.def|
and the sample |cdocsamp.tex| with include files
|cdocsch1.tex|, |cdocsch2.tex|, |cdocspt3.tex|, |cdocspt4.tex|,
|cdocsdrf.tex|, |cdocsfn1.tex|, |cdocsfn2.tex|.
Then copy the file |childdoc.def| to an appropriate directory of your \LaTeX{}
distribution, e.g.\ \textit{texmf-root}|/tex/latex/childdoc|.
\end{itemize}

%%%%%%%%%%%%%%%%%%%%%%%%%%%%%%%%%%%%%%%%%%%%%%%%%%%%%%%%%%%%%%%%%%%%%%%%%%%%%%%%
\subsection{Related CTAN Packages}

There are several other packages which offer a similar functionality:
%
\begin{itemize}
\item
The packages
\href{http://ctan.org/pkg/docmute}{\textsf{docmute}},
\href{http://ctan.org/pkg/includex}{\textsf{includex}} and
\href{http://ctan.org/pkg/standalone}{\textsf{standalone}}
provide commands to include only the document body of
a child file thus allowing both files to be compiled individually.
\item
The packages \href{http://ctan.org/pkg/subdocs}{\textsf{subdocs}}
and \href{http://ctan.org/pkg/subfiles}{\textsf{subfiles}}
provide structures in which the main and child documents can be
encapsulated and allowing them to be compiled individually.
The inclusion mechanism is different from the conventional |\include|.
\item
The package \href{http://ctan.org/pkg/combine}{\textsf{combine}}
is an elaborate solution to combine several documents into one.
\end{itemize}
%
See also the CTAN topic \href{http://ctan.org/topic/subdocs}{\textsf{subdocs}}
for further related packages.
The present package differs from the above solutions in that
a document structure constructed with the conventional |\include| mechanism
just needs two extra commands at the top of every file
such that all constituent files can be compiled individually.

%%%%%%%%%%%%%%%%%%%%%%%%%%%%%%%%%%%%%%%%%%%%%%%%%%%%%%%%%%%%%%%%%%%%%%%%%%%%%%%%
%\subsection{Feature Suggestions}
%
%The following is a list of features which may be useful for future
%versions of this package:
%%
%\begin{itemize}
%\item
%\ldots
%\end{itemize}

%%%%%%%%%%%%%%%%%%%%%%%%%%%%%%%%%%%%%%%%%%%%%%%%%%%%%%%%%%%%%%%%%%%%%%%%%%%%%%%%
\subsection{Revision History}

%%%%%%%%%%%%%%%%%%%%%%%%%%%%%%%%%%%%%%%%
\paragraph{v2.0:} 2018/12/30

\begin{itemize}
\item
immediate forward processing
\item
added |\childdocby| mechanism
\item
manual restructured
\end{itemize}

%%%%%%%%%%%%%%%%%%%%%%%%%%%%%%%%%%%%%%%%
\paragraph{v1.6:} 2018/01/17

\begin{itemize}
\item
application for development of include files
\item
corrections to manual
\end{itemize}

%%%%%%%%%%%%%%%%%%%%%%%%%%%%%%%%%%%%%%%%
\paragraph{v1.5:} 2017/05/21

\begin{itemize}
\item
more complete structuring introduced
\item
|\childdocof| introduced
\item
|\childdoc| renamed to |\childdocmain|
\item
|\childredirect| renamed to |\childdocforward| and |\childdocforwardprefix|
and functionality expanded
\end{itemize}

%%%%%%%%%%%%%%%%%%%%%%%%%%%%%%%%%%%%%%%%
\paragraph{v1.0:} 2017/04/27

\begin{itemize}
\item
manual and install package
\item
first version published on CTAN
\end{itemize}

%%%%%%%%%%%%%%%%%%%%%%%%%%%%%%%%%%%%%%%%
\paragraph{v0.6:} 2017/04/26

\begin{itemize}
\item
redirection mechanism added
\end{itemize}

%%%%%%%%%%%%%%%%%%%%%%%%%%%%%%%%%%%%%%%%
\paragraph{v0.5:} 2017/04/26

\begin{itemize}
\item
functionality in definition file
\end{itemize}


%%%%%%%%%%%%%%%%%%%%%%%%%%%%%%%%%%%%%%%%%%%%%%%%%%%%%%%%%%%%%%%%%%%%%%%%%%%%%%%%
%%%%%%%%%%%%%%%%%%%%%%%%%%%%%%%%%%%%%%%%%%%%%%%%%%%%%%%%%%%%%%%%%%%%%%%%%%%%%%%%
%%%%%%%%%%%%%%%%%%%%%%%%%%%%%%%%%%%%%%%%%%%%%%%%%%%%%%%%%%%%%%%%%%%%%%%%%%%%%%%%
\appendix

\settowidth\MacroIndent{\rmfamily\scriptsize 000\ }

 \DocInput{childdoc.dtx}

\end{document}
%</driver>
% \fi
%
% %%%%%%%%%%%%%%%%%%%%%%%%%%%%%%%%%%%%%%%%%%%%%%%%%%%%%%%%%%%%%%%%%%%%%%%%%%%%%%
% %%%%%%%%%%%%%%%%%%%%%%%%%%%%%%%%%%%%%%%%%%%%%%%%%%%%%%%%%%%%%%%%%%%%%%%%%%%%%%
% \section{Sample}
%\iffalse
%<*samplemain>
%\fi
%
% The following presents a sample document
% with two chapters, two parts, a title page,
% a compile flag as well as three forwarding files to set the flag.
% It consists of eight |.tex| files:
% \begin{center}
% \begin{tabular}{ll}
% |cdocsamp.tex|&main file\\
% |cdocsch1.tex|&include file for chapter 1\\
% |cdocsch2.tex|&include file for chapter 2\\
% |cdocspt3.tex|&include file for part 3\\
% |cdocspt4.tex|&include file for part 4\\
% |cdocsdrf.tex|&forwarding file for main file in draft mode\\
% |cdocsfi1.tex|&forwarding file for final version of chapter 1\\
% |cdocsfi2.tex|&forwarding file for final version of chapter 2\\
% \end{tabular}
% \end{center}
% Each of the eight files can be compiled directly by the \LaTeX{} compiler.
%
% %%%%%%%%%%%%%%%%%%%%%%%%%%%%%%%%%%%%%%
% \paragraph{Main File.}
%
% The main file is called |cdocsamp.tex|.
%
% Load the \textsf{childdoc} definitions and
% declare the filename for the main document:
%    \begin{macrocode}
\input{childdoc.def}
\childdocmain{}
%    \end{macrocode}

% Optional override for |\version| flag:
%    \begin{macrocode}
%%\ifchilddoc\else\providecommand{\version}{draft}\fi
%    \end{macrocode}

% Define the default values for the |\version| flag
% (|final| for the main file and |draft| for childs):
%    \begin{macrocode}
\ifchilddoc
\providecommand{\version}{draft}
\else
\providecommand{\version}{final}
\fi
%    \end{macrocode}

% Load the standard document class:
%    \begin{macrocode}
\documentclass[12pt]{article}
%    \end{macrocode}

% Start the document body:
%    \begin{macrocode}
\begin{document}
%    \end{macrocode}

% Declare a title page.
% Print title, part of document being processed and version flag:
%    \begin{macrocode}
\addtocounter{page}{-1}
\begin{center}
{\LARGE\bfseries{}childdoc example\par}
\vspace{1cm}
\ifchilddoc
\ifchilddocmanual part\else chapter\fi:
`\childdocname' of `\childdocjob'\par
\else
main document: `\childdocjob'\par
\fi
version: \version\par
\end{center}
\newpage
%    \end{macrocode}

% Manually include selected file,
% otherwise process as usual:
%    \begin{macrocode}
\ifchilddocmanual
\section*{part `\childdocname'}
\input{\childdocname}
\else
%    \end{macrocode}

% Include the two chapters:
%    \begin{macrocode}
\include{cdocsch1}
\include{cdocsch2}
%    \end{macrocode}

% Include the two parts unless only chapters should be displayed:
%    \begin{macrocode}
\ifchilddoc\else
\section{part three}
\input{cdocspt3}
\section{part four}
\input{cdocspt4}
\fi
%    \end{macrocode}

% Process as usual until here:
%    \begin{macrocode}
\fi
%    \end{macrocode}

% End of document body:
%    \begin{macrocode}
\end{document}
%    \end{macrocode}
%\iffalse
%</samplemain>
%\fi
%
% %%%%%%%%%%%%%%%%%%%%%%%%%%%%%%%%%%%%%%
% \paragraph{Chapter Include Files.}
%
% The include files are called |cdocsch1.tex| and |cdocsch2.tex|.
%
%\iffalse
%<*samplechap1|samplechap2>
%\fi

% Optional override for |\version| flag:
%    \begin{macrocode}
%%\providecommand{\version}{final}
%    \end{macrocode}

% Include the main document:
%    \begin{macrocode}
\input{childdoc.def}
\childdocof{cdocsamp}
%    \end{macrocode}

%\iffalse
%</samplechap1|samplechap2>
%\fi
%
%\iffalse
%<*samplechap1>
%\fi
% Some text for chapter 1:
%    \begin{macrocode}
\section{one}
some text in chapter one
%    \end{macrocode}

%\iffalse
%</samplechap1>
%\fi
% Some text for chapter 2:
%\iffalse
%<*samplechap2>
%\fi
%    \begin{macrocode}
\section{two}
more text in chapter two
%    \end{macrocode}

%\iffalse
%</samplechap2>
%\fi
%
% %%%%%%%%%%%%%%%%%%%%%%%%%%%%%%%%%%%%%%
% \paragraph{Part Include Files.}
%
% The include files are called |cdocspt3.tex| and |cdocspt4.tex|.
%
%\iffalse
%<*samplepart3|samplepart4>
%\fi

% Optional override for |\version| flag:
%    \begin{macrocode}
%%\providecommand{\version}{final}
%    \end{macrocode}

% Include the main document:
%    \begin{macrocode}
\input{childdoc.def}
\childdocby{cdocsamp}
%    \end{macrocode}

%\iffalse
%</samplepart3|samplepart4>
%\fi
%
%\iffalse
%<*samplepart3>
%\fi
% Some text for part 3:
%    \begin{macrocode}
some text in part three
%    \end{macrocode}

%\iffalse
%</samplepart3>
%\fi
% Some text for part 4:
%\iffalse
%<*samplepart4>
%\fi
%    \begin{macrocode}
more text in part four
%    \end{macrocode}

%\iffalse
%</samplepart4>
%\fi
%
% %%%%%%%%%%%%%%%%%%%%%%%%%%%%%%%%%%%%%%
% \paragraph{Forwarding for a Complete Draft.}
%
% The following forwarding file |cdocsdrf.tex|
% compiles the main document in draft mode:
%\iffalse
%<*sampledraft>
%\fi
%    \begin{macrocode}
\def\version{draft}
\input{childdoc.def}
\childdocforward{cdocsamp}
%    \end{macrocode}

%\iffalse
%</sampledraft>
%\fi
%
% %%%%%%%%%%%%%%%%%%%%%%%%%%%%%%%%%%%%%%
% \paragraph{Forwarding for Final Version of the Chapters.}
%
% The following forwarding files |cdocsfn1.tex| and |cdocsfn2.tex|
% (with identical content)
% compile the final versions of the child documents
% |cdocsch1.tex| and |cdocsch2.tex|, respectively:
%\iffalse
%<*samplefinal>
%\fi
%    \begin{macrocode}
\def\version{final}
\input{childdoc.def}
\childdocforwardprefix[cdocsamp]{cdocsfn}{cdocsch}
%    \end{macrocode}

%\iffalse
%</samplefinal>
%\fi
%
% %%%%%%%%%%%%%%%%%%%%%%%%%%%%%%%%%%%%%%
% \paragraph{Command Line Processing.}
%
% The following three command lines generate the output files
% |cdocscld|, |cdocscl1| and |cdocscl2|
% which should be identical to
% |cdocsdrf|, |cdocsch1| and |cdocsfn2|, respectively:
% \begin{center}
% \begin{tabular}{l}
% |latex -jobname cdocscld \|\\
% |  "\def\version{draft}\input{childdoc.def}\childdocforward{cdocsamp}"|\\
% |latex -jobname cdocscl1 \|\\
% |  "\input{childdoc.def}\childdocforward[cdocsamp]{cdocsch1}"|\\
% |latex -jobname cdocscl2 \|\\
% |  "\def\version{final}\input{childdoc.def}\childdocforward{cdocsch2}"|
% \end{tabular}
% \end{center}
% Note that the trailing backslash on each first line
% merely continues the input to the second line
% (for convenient cut ant paste).
% Furthermore, the command |latex| can be replaced by any
% of its alternative versions such as |pdflatex|.
%
% %%%%%%%%%%%%%%%%%%%%%%%%%%%%%%%%%%%%%%%%%%%%%%%%%%%%%%%%%%%%%%%%%%%%%%%%%%%%%%
% %%%%%%%%%%%%%%%%%%%%%%%%%%%%%%%%%%%%%%%%%%%%%%%%%%%%%%%%%%%%%%%%%%%%%%%%%%%%%%
% \section{Implementation}
%\iffalse
%<*package>
%\fi
%
% This section describes the definitions file |childdoc.def|.

% The definitions cannot be loaded using |\usepackage| or |\RequirePackage|
% which has a mechanism to prevent loading a style file more than once.
% When loading the definitions by means of |\input|
% multiple instances have to be prevented manually:
%\iffalse
%This code needs to be before the `\ProvidesFile' directive
%which is defined at the beginning of this file.
%Therefore it is also placed there and commented out here.
%</package>
%<*discard>
%\fi
%    \begin{macrocode}
\ifdefined\childdocmain\endinput\fi
%    \end{macrocode}
%\iffalse
%</discard>
%<*package>
%\fi
%
% \macro{\ifchilddoc}
% \macro{\ifchilddocmanual}
% The conditional |\ifchilddoc| tells whether a
% child (true) or main (false) document is being compiled.
% The conditional |\ifchilddocmanual| tells whether
% the |\includeonly| mechanism is used (false) or
% the selection of child files must be performed manually (true).
% The definitions initialise to false:
%    \begin{macrocode}
\newif\ifchilddoc
\newif\ifchilddocmanual
%    \end{macrocode}

% \macro{\childdocname}
% \macro{\childdocjob}
% The macro |\childdocname| stores the name of the main document
% to be compiled. The macro |\childdocjob| stores the name of
% the document on which the \LaTeX{} compiler was originally invoked.
% The content of |\jobname| cannot be compared
% to filenames specified in the source due to different catcodes.
% The following code rescans |\jobname|, stores the result
% in |\childdocname| and saves a copy in |\childdocjob|:
%    \begin{macrocode}
\edef\childdocname{\scantokens\expandafter{\jobname\noexpand}}
\let\childdocjob\childdocname
%    \end{macrocode}

% \macro{\childdocdisable}
% The macro |\childdocdisable| prevents the main file
% from being processed more than once.
% At this stage, the main document command |\childdocmain|
% is assumed to be called once again where it should do nothing.
% Any subsequent call to it should prevent
% a secondary processing of the main document
% It overwrites the forwarding commands
% |\childdocof| and |\childdocforward|
% with empty macros to prevent further inclusions of the main document:
%    \begin{macrocode}
\newcommand{\childdocdisable}
{
  \renewcommand{\childdocmain}[1]{\renewcommand{\childdocmain}[1]{\endinput}}
  \renewcommand{\childdocof}[1]{}
  \renewcommand{\childdocby}[2][]{}
  \renewcommand{\childdocforward}[2][]{}
  \renewcommand{\childdocdisable}{}
}
%    \end{macrocode}

% \macro{\childdocmain}
% The macro |\childdocmain| is to be called at the top of the main file
% with nothing or the main filename (without extension) as argument.
% First, it breaks loops.
% If the argument is not empty and does not match |\childdocname|
% (which is set by the first inclusion of |childdoc.def|),
% |\ifchilddoc| is set to true, |\includeonly| is applied to the child file
% and |\jobname| is set to the main file
% (for proper handling of |.aux| files):
%    \begin{macrocode}
\newcommand{\childdocmain}[1]
{
  \childdocdisable\childdocmain{}
  \if?#1?\else
    \begingroup
      \def\childdoctmp{#1}
      \ifx\childdoctmp\childdocname
        \def\childdoctmp{}
      \else
        \def\childdoctmp
        {
          \childdoctrue
          \includeonly{\childdocname}
          \def\childdocjob{#1}
          \def\jobname{#1}
        }
      \fi
      \expandafter
    \endgroup
    \childdoctmp
  \fi
}
%    \end{macrocode}

% \macro{\childdocof}
% The command |\childdocof| redirects
% compilation to the main file |#1|.
%    \begin{macrocode}
\newcommand{\childdocof}[1]
{
  \childdocdisable
  \childdoctrue
  \includeonly{\childdocname}
  \def\jobname{#1}
  \def\childdocjob{#1}
  \input{#1}
}
%    \end{macrocode}

% \macro{\childdocby}
% The command |\childdocby| ....
%    \begin{macrocode}
\newcommand{\childdocby}[2][]
{
  \childdocdisable
  \childdoctrue
  \childdocmanualtrue
  \if?#1?\else
    \def\jobname{#2}
  \fi
  \def\childdocjob{#2}
  \input{#2}
  \endinput
}
%    \end{macrocode}

% \macro{\childdocforward}
% The command |\childdocforward| redirects
% compilation to the main file or
% (if the optional argument is given) a child file.
% Parameters are set as if the main file
% or a child file starting with |\childdocof| was compiled.
% Then compilation is handed over to the main file:
%    \begin{macrocode}
\newcommand{\childdocforward}[2][]
{
  \begingroup
    \if?#1?
      \def\childdoctmp
      {
        \def\childdocname{#2}
        \def\childdocjob{#2}
        \def\jobname{#2}
        \input{#2}
        \endinput
      }
    \else
      \def\childdoctmp
      {
        \childdocdisable
        \def\childdocname{#2}
        \childdoctrue
        \includeonly{#2}
        \def\childdocjob{#1}
        \def\jobname{#1}
        \input{#1}
        \endinput
      }
    \fi
    \expandafter
  \endgroup
  \childdoctmp
}
%    \end{macrocode}

% \macro{\childdocforwardprefix}
% The command |\childdocforwardprefix| redirects
% compilation to the main or a child file by means of a pattern.
% The prefix |#1| in the current filename is replaced by |#2|
% and the suffix of the current filename is kept
% (it is assumed that the filename does not contain the substring `|~~~|'
% which is used as a delimiter).
% Compilation is handed over to the new file by |\childdocforward|:
%    \begin{macrocode}
\newcommand{\childdocforwardprefix}[3][]
{
  \begingroup
    \def\childdocextract #2##1~~~{\def\childdoctmp{\childdocforward[#1]{#3##1}}}
    \expandafter\childdocextract\childdocname~~~
    \expandafter
  \endgroup
  \childdoctmp
}
%    \end{macrocode}

% \macro{\childdoc}
% The deprecated macro |\childdoc| is a legacy version of |\childdocmain|:
%    \begin{macrocode}
\newcommand{\childdoc}{\childdocmain}
%    \end{macrocode}

% \macro{\childdocredirect}
% The deprecated macro |\childdocredirect| is a legacy version
% of |\childdocforward| and |\childdocforwardprefix|:
%    \begin{macrocode}
\newcommand{\childdocredirect}[2][]
{
  \begingroup
    \if?#1?
      \def\childdoctmp{\childdocforward{#2}}
    \else
      \def\childdoctmp{\childdocforwardprefix{#1}{#2}}
    \fi
    \expandafter
  \endgroup
  \childdoctmp
}
%    \end{macrocode}

%\iffalse
%</package>
%\fi
%
\endinput
\childdocforward{cdocsamp}"|\\
% |latex -jobname cdocscl1 \|\\
% |  "% \iffalse
%
% childdoc.dtx Copyright (C) 2017-2018 Niklas Beisert
%
% This work may be distributed and/or modified under the
% conditions of the LaTeX Project Public License, either version 1.3
% of this license or (at your option) any later version.
% The latest version of this license is in
%   http://www.latex-project.org/lppl.txt
% and version 1.3 or later is part of all distributions of LaTeX
% version 2005/12/01 or later.
%
% This work has the LPPL maintenance status `maintained'.
%
% The Current Maintainer of this work is Niklas Beisert.
%
% This work consists of the files childdoc.dtx and childdoc.ins
% and the derived files childdoc.def and cdocsamp.tex with
% cdocsch1.tex, cdocsch2.tex, cdocsdrf.tex, cdocsfn1.tex, cdocsfn2.tex.
%
%<package>\ifdefined\childdocmain\endinput\fi
%<package>\ProvidesFile{childdoc.def}[2018/12/30 v2.0 child document driver]
%<samplemain>\ProvidesFile{cdocsamp.tex}[2018/12/30 v2.0 sample for childdoc]
%<*driver>
%\ProvidesFile{childdoc.drv}[2018/12/30 v2.0 childdoc reference manual file]
\PassOptionsToClass{10pt,a4paper}{article}
\documentclass{ltxdoc}

\usepackage[margin=35mm]{geometry}
\usepackage{hyperref}
\usepackage{hyperxmp}
\usepackage[usenames]{color}

\hypersetup{colorlinks=true}
\hypersetup{pdfstartview=FitH}
\hypersetup{pdfpagemode=UseNone}
\hypersetup{pdfsource={}}
\hypersetup{pdflang={en-UK}}
\hypersetup{pdfcopyright={Copyright 2017-2018 Niklas Beisert.
  This work may be distributed and/or modified under the
  conditions of the LaTeX Project Public License, either version 1.3
  of this license or (at your option) any later version.}}
\hypersetup{pdflicenseurl={http://www.latex-project.org/lppl.txt}}
\hypersetup{pdfcontactaddress={ETH Zurich, ITP, HIT K,
  Wolfgang-Pauli-Strasse 27}}
\hypersetup{pdfcontactpostcode={8093}}
\hypersetup{pdfcontactcity={Zurich}}
\hypersetup{pdfcontactcountry={Switzerland}}
\hypersetup{pdfcontactemail={nbeisert@itp.phys.ethz.ch}}
\hypersetup{pdfcontacturl={http://people.phys.ethz.ch/\xmptilde nbeisert/}}

\newcommand{\secref}[1]{\hyperref[#1]{section \ref*{#1}}}

\parskip1ex
\parindent0pt
\let\olditemize\itemize
\def\itemize{\olditemize\parskip0pt}

\begin{document}

\title{The \textsf{childdoc} Package}
\hypersetup{pdftitle={The childdoc Package}}
\author{Niklas Beisert\\[2ex]
  Institut f\"ur Theoretische Physik\\
  Eidgen\"ossische Technische Hochschule Z\"urich\\
  Wolfgang-Pauli-Strasse 27, 8093 Z\"urich, Switzerland\\[1ex]
  \href{mailto:nbeisert@itp.phys.ethz.ch}
  {\texttt{nbeisert@itp.phys.ethz.ch}}}
\hypersetup{pdfauthor={Niklas Beisert}}
\hypersetup{pdfsubject={Manual for the LaTeX2e Package childdoc}}
\date{30 December 2018, \textsf{v2.0}}
\maketitle

\begin{abstract}\noindent
\textsf{childdoc} is a \LaTeXe{} package
that enables the direct compilation
of document sections included by |\include|
to individual files.
\end{abstract}

\begingroup
\parskip0ex
\tableofcontents
\endgroup

%%%%%%%%%%%%%%%%%%%%%%%%%%%%%%%%%%%%%%%%%%%%%%%%%%%%%%%%%%%%%%%%%%%%%%%%%%%%%%%%
%%%%%%%%%%%%%%%%%%%%%%%%%%%%%%%%%%%%%%%%%%%%%%%%%%%%%%%%%%%%%%%%%%%%%%%%%%%%%%%%
\section{Introduction}

\LaTeX{} provides a mechanism to structure a large document (such as a book)
into a main file and several child files (containing the chapters)
using the |\include| command.
This mechanism is beneficial for documents
which span hundreds of pages in order to
make the source file(s) more manageable.
Moreover, compilation can be restricted to
selected child files by means of the |\includeonly| command.
The latter feature can be used to reduce the compilation time while editing
(this was significantly more useful in the earlier days of \LaTeX{})
or to generate a smaller document which is easier to navigate.
Another application of |\includeonly| is to generate
documents consisting of selected parts of the complete document.

However, there are a few drawbacks of the plain |\include| mechanism:
\begin{itemize}
\item
The child files cannot be compiled on their own,
they can only be compiled via the main file.
A naive editing environment
(such as a text editor with an option
to have the current file processed by \LaTeX)
may require one to switch to the main file before compiling;
attempting to compile the child file produces errors.
\item
The main file must be modified (each time)
to adjust the |\includeonly| command
to the present needs. This easily leaves the main file in a messy state.
\item
The generated document will always carry the filename
of the main document. This is inconvenient if
several child files are to be compiled and
to be kept for distribution.
\end{itemize}

The present package provides a simple interface
to make child files individually compilable by \LaTeX{}.
Compiling a child file then has the same effect as compiling
the main file with an |\includeonly| command
to select the appropriate child.
Moreover the generated document will carry the name of the child
rather than the main file.
This resolves all three above issues.

This feature is meant to make the editing of books,
thesis documents and lecture notes somewhat more convenient.
However, the package can also be used efficiently for
composing a series of documents (such as exercise sheets)
which are typically distributed individually.
It then assists the author in generating the individual documents
(potentially in different versions)
as well as a document containing the collected series.
Another application is in developing style files
or other kinds of included material
where compilation of the style file could redirect
to a sample or test file.

%%%%%%%%%%%%%%%%%%%%%%%%%%%%%%%%%%%%%%%%%%%%%%%%%%%%%%%%%%%%%%%%%%%%%%%%%%%%%%%%
%%%%%%%%%%%%%%%%%%%%%%%%%%%%%%%%%%%%%%%%%%%%%%%%%%%%%%%%%%%%%%%%%%%%%%%%%%%%%%%%
\section{Usage}

First of all, the package \textsf{childdoc} is \emph{not} a standard
\LaTeXe{} |.sty| style file! Therefore it needs to be invoked in
a non-standard way.

%%%%%%%%%%%%%%%%%%%%%%%%%%%%%%%%%%%%%%%%%%%%%%%%%%%%%%%%%%%%%%%%%%%%%%%%%%%%%%%%
\subsection{Included Files}
\label{sec:include}

%%%%%%%%%%%%%%%%%%%%%%%%%%%%%%%%%%%%%%%%
\DescribeMacro{\childdocmain}
To use the package, add the commands
\begin{center}
\begin{tabular}{l}
|\input{childdoc.def}|\\
|\childdocmain{}|\\
\end{tabular}
\end{center}
at the very top of the main \LaTeX{} file,
in particular \emph{before} the |\documentclass| statement!
The argument of |\childdocmain| should be left empty
(but it must be present).

%%%%%%%%%%%%%%%%%%%%%%%%%%%%%%%%%%%%%%%%
\DescribeMacro{\childdocof}
Furthermore, add the commands
\begin{center}
\begin{tabular}{l}
|\input{childdoc.def}|\\
|\childdocof{|\textit{main}|}|\\
\end{tabular}
\end{center}
at the top of every child file \textit{child}
which is included by |\include{|\textit{child}|}|
from within the main file
(or at least for those files to be compiled individually).
The argument \textit{main} must be the filename of the main file.

There are a couple of
considerations in setting up the main and child documents:

%%%%%%%%%%%%%%%%%%%%%%%%%%%%%%%%%%%%%%%%
\paragraph{Restrictions.}

Please note the following restrictions:
\begin{itemize}
\item
|\childdocmain| must be called with one argument \textit{main}
to ensure compatibility with earlier version of the package.
It must either be empty (|\childdocmain{}|)
or precisely match the filename of the main file in which it is specified.
See \secref{sec:detection} for further information.
\item
The filename \textit{main} must be specified without the |.tex| extension.
\item
The filename \textit{main} is case sensitive
(even in case-insensitive file systems)
due to internal string comparison.
\item
The argument \textit{main} should be fully expanded, it cannot be a macro.
\item
Subdirectories and special characters should be avoided in filenames.
\item
The command |\childdocmain{|\textit{main}|}| must be followed by a whitespace.
It should not be followed immediately by another command
or by a comment mark `|%|'.
This is because the \TeX{} parser reads the token immediately following
the argument of |\childdocmain| and puts it
at the beginning of every child section;
however, a white\-space is ignored.
\end{itemize}

%%%%%%%%%%%%%%%%%%%%%%%%%%%%%%%%%%%%%%%%
\paragraph{Content of Main File.}

It is advisable to place all content in the child files included by |\include|.
Any output contained in the main file will appear in all child documents
unless suppressed manually;
it cannot be suppressed automatically by the |\includeonly| directive
and thus should normally be avoided.
A method to include some content in the main file
by means of conditional processing is described in \secref{sec:conditional}.

%%%%%%%%%%%%%%%%%%%%%%%%%%%%%%%%%%%%%%%%
\paragraph{Page Numbering.}

When only a part of the document is compiled,
the appropriate numbering of pages
(as well as other status parameters)
is determined from the |.aux| files.
The latter contain information from previous passes.
However this information needs to propagate through
all intermediate child documents.
Therefore the page numbering in child documents may well
be inconsistent until the complete document is compiled at least once.

A useful (if unconventional) way to always ensure a consistent
page numbering is to restart the numbering in each child document
and denote the pages by `\textit{child}|.|\textit{page}'
where \textit{child} represents the chapter/section number of the child file.
This can be achieved by the command
|\numberwithin{page}{|\textit{child}|}|
of the \textsf{amsmath} package
where \textit{child} can be |chapter| or |section|
depending on the chosen structuring.
Alternatively, one can modify the macro |\thepage| appropriately
and reset the counter |page| at the start of each child file.

%%%%%%%%%%%%%%%%%%%%%%%%%%%%%%%%%%%%%%%%%%%%%%%%%%%%%%%%%%%%%%%%%%%%%%%%%%%%%%%%
\subsection{Conditional Processing}
\label{sec:conditional}

The package provides a mechanism to compile different versions
of a document. To customise the versions further some conditional processing
can come in handy to distinguish which version is being compiled.
The package provides two macros to describe the compilation context:

%%%%%%%%%%%%%%%%%%%%%%%%%%%%%%%%%%%%%%%%
\DescribeMacro{\ifchilddoc}
The conditional |\ifchilddoc| distinguishes between the compilation of
child documents and the main document:
%
\begin{center}
|\ifchilddoc |\textit{child-code}| |[|\||else |\textit{main-code}]| \||fi|
\end{center}

%%%%%%%%%%%%%%%%%%%%%%%%%%%%%%%%%%%%%%%%
\DescribeMacro{\childdocname}
\DescribeMacro{\childdocjob}
The macro |\childdocname| contains the filename (without extension)
of the main or child file being processed.
Note that |\childdocjob| will always contain the name of the main file.

%%%%%%%%%%%%%%%%%%%%%%%%%%%%%%%%%%%%%%%%
\paragraph{Title Page.}

Conditional processing can be used to include a title or banner page
in the main document when proper precautions are taken.
Importantly, the code in the main file should ensure that the page counter
(as well as other status parameters which are stored in the |.aux| files)
takes the same value after the conditional processing.
Otherwise the page numbers may take divergent values
depending on which part is compiled.

For example, a title page could be declared by:
%
\begin{center}
\begin{tabular}{l}
|\ifchilddoc\||else|\\
|\addtocounter{page}{-1}|\\
\textit{code for title page}\\
|\newpage|\\
|\||fi|
\end{tabular}
\end{center}
%
A banner page for the child documents can be generated by:
%
\begin{center}
\begin{tabular}{l}
|\ifchilddoc|\\
|\addtocounter{page}{-1}|\\
\textit{code for banner page}\\
|\newpage|\\
|\||fi|
\end{tabular}
\end{center}
%
Here one could write a message such as:
\begin{center}
|This is the part \childdocname{} of \childdocjob{}.|
\end{center}

%%%%%%%%%%%%%%%%%%%%%%%%%%%%%%%%%%%%%%%%%%%%%%%%%%%%%%%%%%%%%%%%%%%%%%%%%%%%%%%%
\subsection{Flags}
\label{sec:flags}

The package makes it easy to generate different versions
of the main or child documents.
To this end compilation flags can be defined
and assigned different default values.
They will be particularly useful in conjunction
with the forwarding mechanism described in \secref{sec:forward}.

For example, it may be useful to have a flag |\version|
which can be set to |draft| or |final|.
The document source will contain some conditional code
depending on the value of |\version|.
Suppose further, the flag should default to |final| for the main file
and to |draft| for child files
which is a natural assignment for editing the document.
This is achieved by placing the following code
in the preamble of the main document
(below the |\childdocmain| directive):
%
\begin{center}
\begin{tabular}{l}
|\ifchilddoc|\\
|\providecommand{\version}{draft}|\\
|\||else|\\
|\providecommand{\version}{final}|\\
|\||fi|
\end{tabular}
\end{center}
%
The definition by |\providecommand| makes sure
that previous definitions are not overwritten.
Further statements |\providecommand{\version}{...}|
can thus be added before the above code to override it.

For the main file, one might add a line
(between |\childdocmain| and the above block)
%
\begin{center}
|%\ifchilddoc\||else\providecommand{\version}{draft}\||fi|
\end{center}
%
which can be uncommented to produce a draft version.
Likewise one can add a line to the very top of a child file
(above the |\childdocof{|\textit{main}|}| directive)
%
\begin{center}
|%\providecommand{\version}{final}|
\end{center}
%
which can be uncommented to produce the final version of this child document.

%%%%%%%%%%%%%%%%%%%%%%%%%%%%%%%%%%%%%%%%%%%%%%%%%%%%%%%%%%%%%%%%%%%%%%%%%%%%%%%%
\subsection{Forwarding}
\label{sec:forward}

Different versions of the main or child documents
using compilation flags as described in \secref{sec:flags}
can be (permanently) stored in different files
for convenient compilation, viewing and distribution.
To this end, the package defines a command
to pass on compilation to a different file:

%%%%%%%%%%%%%%%%%%%%%%%%%%%%%%%%%%%%%%%%
\DescribeMacro{\childdocforward}
The command |\childdocforward| redirects processing to
another source file:
%
\begin{center}
\begin{tabular}{l}
|\input{childdoc.def}|\\
|\childdocforward[|\textit{main}|]{|\textit{dest}|}|\\
\end{tabular}
\end{center}
%
The argument \textit{dest} is the destination file
(without extension).
It should be the main file or one of the child files.
Note that further \textsf{childdoc} directives
such as |\childdocof| and |\childdocforward|
in the indicated file will be processed in this form.
The optional argument \textit{main}
passes on directly to the main file \textit{main}
while pretending to compile the child \textit{dest}.
This form behaves as if \textit{dest}
issues |\childdocof{|\textit{main}|}| right away,
and no further \textsf{childdoc} directives will be processed.

%%%%%%%%%%%%%%%%%%%%%%%%%%%%%%%%%%%%%%%%
\DescribeMacro{\...prefix}
In the alternative form |\childdocforwardprefix|,
%
\begin{center}
\begin{tabular}{l}
|\input{childdoc.def}|\\
|\childdocforwardprefix[|\textit{main}|]{|\textit{prefix}|}{|\textit{dest}|}|
\end{tabular}
\end{center}
%
the destination file is determined by a pattern
depending on the current file:
To make this work, the current file must be called
`{\textit{prefix}\hspace{0.2em}\textit{suffix}}'
with \textit{prefix} matching precisely the argument.
Processing is then passed on to the file
`{\textit{dest}\hspace{0.2em}\textit{suffix}}'.
Surely, the same effect is achieved by
directly specifying the
argument `{\textit{dest}\hspace{0.2em}\textit{suffix}}'
in the first form.
However, that requires to set up a different file
for each child. With the alternative form of the command
all these files can have exactly the same content
which simplifies setting them up and maintaining them.

For example, the following file |draft.tex|
with a compilation flag |\version| as described in \secref{sec:flags}
compiles the main document as a draft:
%
\begin{center}
\begin{tabular}{l}
|\def\version{draft}|\\
|\input{childdoc.def}|\\
|\childdocforward{|\textit{main}|}|
\end{tabular}
\end{center}
%
Likewise, the following files |final|\textit{nn}|.tex|
compile the final version of the child document
|child|\textit{nn}|.tex|:
%
\begin{center}
\begin{tabular}{l}
|\def\version{final}|\\
|\input{childdoc.def}|\\
|\childdocforwardprefix{final}{child}|
\end{tabular}
\end{center}
%

Note that when several versions of a main file and/or of each child file
are to be generated, it may be convenient to set up a |Makefile| or
shell script to automatise the process.

%%%%%%%%%%%%%%%%%%%%%%%%%%%%%%%%%%%%%%%%%%%%%%%%%%%%%%%%%%%%%%%%%%%%%%%%%%%%%%%%
\subsection{Command Line Processing}
\label{sec:commandline}

The effect of redirection files can also be achieved by invoking
the \LaTeX{} compiler with a more elaborate command line.
Most conveniently this should be done as part
of a shell script or a |Makefile|.

When using \textsf{childdoc} in the main file, the following
command lines effectively perform a redirection
(note that depending on the shell being used,
backslashes may have to be doubled: `|\|' $\to$ `|\\|'):
%
\begin{center}
|... -jobname "|\textit{target}|" |\\|"|[\textit{flags}]%
|\input{childdoc.def}\childdocforward[|\textit{main}|]{|\textit{dest}|}"|
\end{center}
%
Here \textit{target} is the name of the output file,
\textit{main} is the name of the main file
and \textit{dest} is the name of the main or child file to be processed
(all filenames without extensions).
The optional argument \textit{main} can be omitted
if \textit{main} matches \textit{dest}.
Optionally, compilation \textit{flags} can be defined via |\def| commands.
This command line makes the \TeX{} engine believe
it is compiling the file \textit{target}
whose content is specified as the latter parameter.
The provided code then forwards the processing to
\textit{main} or \textit{dest} as described in \secref{sec:forward}.

%%%%%%%%%%%%%%%%%%%%%%%%%%%%%%%%%%%%%%%%%%%%%%%%%%%%%%%%%%%%%%%%%%%%%%%%%%%%%%%%
\subsection{Include by Input}
\label{sec:input}

Including child documents by |\include| has some restrictions by design.
Most notably, the content of a child document always occupies
its own set of pages; pages cannot be shared between child documents.
Usually, this behaviour makes perfect sense
because each child document contain an essential part of the document.
However, in some situations it may be desirable to compose
a document from a collection of parts
without having mandatory page breaks between then.
For this case, the package
provides a mechanism to include parts
by |\input| which can also be processed individually.
However, by construction this mechanism
requires manual handling of the content to be output.

%%%%%%%%%%%%%%%%%%%%%%%%%%%%%%%%%%%%%%%%
\DescribeMacro{\ifchilddocmanual}
The main file should be prepared as usual, see \secref{sec:include}.
However, the document body must make a distinction
between processing of an individual part and of the main document, e.g.:
%
\begin{center}
\begin{tabular}{l}
|\ifchilddocmanual|\\
|\input{\childdocname}|\\
|\||else|\\
\textit{document body with }|\input{|\textit{part}|}|\\
|\||fi|
\end{tabular}
\end{center}
%
The conditional |\ifchilddocmanual| is true whenever
a part to be included by |\input| is being compiled,
and the name of the part is stored in |\childdocname|.

%%%%%%%%%%%%%%%%%%%%%%%%%%%%%%%%%%%%%%%%
\DescribeMacro{\childdocby}
Each part to be included by |\input| should start with:
%
\begin{center}
\begin{tabular}{l}
|\input{childdoc.def}|\\
|\childdocby{|\textit{main}|}|\\
\end{tabular}
\end{center}
%
The directive |\childdocby| is similar to |\childdocof|
described in \secref{sec:include},
but the subsequent selection of content must be done manually.
To that end, both |\ifchilddoc| and |\ifchilddocmanual|
will be true upon processing of a part,
and the name of the part is stored in |\childdocname|.
Note that |\jobname| will be set to the filename of the current part
so that each part receives an individual |.aux| file
that does not interfere with the |.aux| file(s) of the main document.
This behaviour can be altered by the alternative form
|\childdocby[*]{|\textit{main}|}| (with a non-empty optional argument)
which uses the |.aux| file of the main document
by setting |\jobname| to \textit{main}.

%%%%%%%%%%%%%%%%%%%%%%%%%%%%%%%%%%%%%%%%%%%%%%%%%%%%%%%%%%%%%%%%%%%%%%%%%%%%%%%%
\subsection{Driver Development}
\label{sec:driver}

The \textsf{childdoc} mechanism can also be use for the development
of definition files such as \LaTeX{} styles or classes.
This case differs from the above setup with multiple parts
included by |\include| in that no |\includeonly| should be invoked.
This can be achieved by starting the include file
(before |\ProvidesPackage|) with:
%
\begin{center}
\begin{tabular}{l}
|\input{childdoc.def}|\\
|\childdocforward{|\textit{main}|}|\\
\end{tabular}
\end{center}
%
or alternatively with:
%
\begin{center}
\begin{tabular}{l}
|\input{childdoc.def}|\\
|\childdocby{|\textit{main}|}|\\
\end{tabular}
\end{center}
%
Both forms have slightly different effects as described above.
The main file is prepared as usual, see \secref{sec:include}.

%%%%%%%%%%%%%%%%%%%%%%%%%%%%%%%%%%%%%%%%%%%%%%%%%%%%%%%%%%%%%%%%%%%%%%%%%%%%%%%%
\subsection{Legacy Detection}
\label{sec:detection}

The directive |\childdocmain| in the main file can detect
whether the complete document or merely a child is to be compiled
even without using the directive |\childdocof|.
This method is deprecated because it is less robust
and there is no compelling reason to use it;
it is merely provided for backward compatibility
and it may be removed in future versions.

If the detection mechanism is to be used,
it is mandatory to correctly specify
the filename of the main file as the argument of |\childdocmain|:
%
\begin{center}
\begin{tabular}{l}
|\input{childdoc.def}|\\
|\childdocmain{|\textit{main}|}|\\
\end{tabular}
\end{center}
%
If |\jobname| does not match the argument \textit{main} of |\childdocmain|,
it is assumed that |\jobname| points to the child file to be compiled.
When using |\childdocmain| with the main file specified as argument,
it suffices to start a child file
with just |\input{|\textit{main}|}|
without loading of the package and using |\childdocof|.
If instead all processing is done
with the appropriate \textsf{childdoc} directives,
the argument of \textit{main} of |\childdocmain| can be empty.

An alternative version of the command line processing described
in \secref{sec:commandline} using the detection mechanism reads:
%
\begin{center}
|... -jobname "|\textit{target}|" "|[\textit{flags}]%
[|\def\jobname{|\textit{dest}|}|]|\input{|\textit{main}|}"|
\end{center}

%%%%%%%%%%%%%%%%%%%%%%%%%%%%%%%%%%%%%%%%%%%%%%%%%%%%%%%%%%%%%%%%%%%%%%%%%%%%%%%%
\subsection{Manual Code}
\label{sec:manual}

In case one cannot be certain whether the definitions file |childdoc.def|
is installed on the target \TeX{} distribution
and one prefers not to ship it,
it is conceivable to paste a few relevant commands into the sources.

To that end, drop all statements |\input{childdoc.def}|
and perform the replacements as outlined below.
Instead of |\childdocmain{|\textit{main}|}| add the following code
to the top of the main file:
%
\begin{center}
\begin{tabular}{l}
|\||ifdefined\childdocname\endinput\||fi\newif\ifchilddoc|\\
|\edef\childdocname{\scantokens\expandafter{\jobname\noexpand}}|\\
|\def\childdocmain{|\textit{main}|}\||ifx\childdocmain\childdocname\||else|\\
|\childdoctrue\includeonly{\childdocname}\let\jobname\childdocmain\||fi|\\
\end{tabular}
\end{center}
%
Instead of |\childdocof{|\textit{main}|}| just include the main file
at the top of each child file:
%
\begin{center}
|\input{|\textit{main}|}|
\end{center}
%
A simple redirection |\childdocforward{|\textit{dest}|}| is achieved by:
%
\begin{center}
|\def\jobname{|\textit{dest}|}\input{\jobname}|
\end{center}
%
The redirection with prefix
|\childdocforwardprefix[|\textit{prefix}|]{|\textit{dest}|}|
is accomplished by:
%
\begin{center}
\begin{tabular}{l}
|{\edef\jobname{\scantokens\expandafter{\jobname\noexpand}}|\\
|\def\redirectjob |\textit{prefix}|#1~~~{\gdef\jobname{|\textit{dest}|#1}}|\\
|\expandafter\redirectjob\jobname~~~}\input{\jobname}|
\end{tabular}
\end{center}

In an alternative approach,
child documents can be compiled by a specific command line
without additional code or specific definitions:
%
\begin{center}
|... -jobname "|\textit{target}|" "|[\textit{flags}]%
|\includeonly{|\textit{dest}|}\input{|\textit{main}|}"|
\end{center}
%

%%%%%%%%%%%%%%%%%%%%%%%%%%%%%%%%%%%%%%%%%%%%%%%%%%%%%%%%%%%%%%%%%%%%%%%%%%%%%%%%
%%%%%%%%%%%%%%%%%%%%%%%%%%%%%%%%%%%%%%%%%%%%%%%%%%%%%%%%%%%%%%%%%%%%%%%%%%%%%%%%
\section{Information}

%%%%%%%%%%%%%%%%%%%%%%%%%%%%%%%%%%%%%%%%%%%%%%%%%%%%%%%%%%%%%%%%%%%%%%%%%%%%%%%%
\subsection{Copyright}

Copyright \copyright{} 2017--2018 Niklas Beisert

This work may be distributed and/or modified under the
conditions of the \LaTeX{} Project Public License, either version 1.3
of this license or (at your option) any later version.
The latest version of this license is in
  \url{http://www.latex-project.org/lppl.txt}
and version 1.3 or later is part of all distributions of \LaTeX{}
version 2005/12/01 or later.

This work has the LPPL maintenance status `maintained'.

The Current Maintainer of this work is Niklas Beisert.

This work consists of the files |README.txt|, |childdoc.ins| and |childdoc.dtx|
as well as the derived files |childdoc.def|, |cdocsamp.tex|
with |cdocsch1.tex|, |cdocsch2.tex|, |cdocspt3.tex|, |cdocspt4.tex|,
|cdocsdrf.tex|, |cdocsfn1.tex|, |cdocsfn2.tex|
as well as |childdoc.pdf|.

%%%%%%%%%%%%%%%%%%%%%%%%%%%%%%%%%%%%%%%%%%%%%%%%%%%%%%%%%%%%%%%%%%%%%%%%%%%%%%%%
\subsection{Files and Installation}

The package consists of the files:
%
\begin{center}
\begin{tabular}{ll}
    |README.txt|   & readme file \\
    |childdoc.ins| & installation file \\
    |childdoc.dtx| & source file \\
    |childdoc.def| & definition file \\
    |cdocsamp.tex| & sample main file \\
    |cdocsch1.tex| & sample include file \\
    |cdocsch2.tex| & sample include file \\
    |cdocspt3.tex| & sample part file \\
    |cdocspt4.tex| & sample part file \\
    |cdocsdrf.tex| & sample redirection file \\
    |cdocsfn1.tex| & sample redirection file \\
    |cdocsfn2.tex| & sample redirection file \\
    |childdoc.pdf| & manual
\end{tabular}
\end{center}
%
The distribution consists of the files
|README.txt|, |childdoc.ins| and |childdoc.dtx|.
%
\begin{itemize}
\item
Run (pdf)\LaTeX{} on |childdoc.dtx|
to compile the manual |childdoc.pdf| (this file).
\item
Run \LaTeX{} on |childdoc.ins| to create the definitions file |childdoc.def|
and the sample |cdocsamp.tex| with include files
|cdocsch1.tex|, |cdocsch2.tex|, |cdocspt3.tex|, |cdocspt4.tex|,
|cdocsdrf.tex|, |cdocsfn1.tex|, |cdocsfn2.tex|.
Then copy the file |childdoc.def| to an appropriate directory of your \LaTeX{}
distribution, e.g.\ \textit{texmf-root}|/tex/latex/childdoc|.
\end{itemize}

%%%%%%%%%%%%%%%%%%%%%%%%%%%%%%%%%%%%%%%%%%%%%%%%%%%%%%%%%%%%%%%%%%%%%%%%%%%%%%%%
\subsection{Related CTAN Packages}

There are several other packages which offer a similar functionality:
%
\begin{itemize}
\item
The packages
\href{http://ctan.org/pkg/docmute}{\textsf{docmute}},
\href{http://ctan.org/pkg/includex}{\textsf{includex}} and
\href{http://ctan.org/pkg/standalone}{\textsf{standalone}}
provide commands to include only the document body of
a child file thus allowing both files to be compiled individually.
\item
The packages \href{http://ctan.org/pkg/subdocs}{\textsf{subdocs}}
and \href{http://ctan.org/pkg/subfiles}{\textsf{subfiles}}
provide structures in which the main and child documents can be
encapsulated and allowing them to be compiled individually.
The inclusion mechanism is different from the conventional |\include|.
\item
The package \href{http://ctan.org/pkg/combine}{\textsf{combine}}
is an elaborate solution to combine several documents into one.
\end{itemize}
%
See also the CTAN topic \href{http://ctan.org/topic/subdocs}{\textsf{subdocs}}
for further related packages.
The present package differs from the above solutions in that
a document structure constructed with the conventional |\include| mechanism
just needs two extra commands at the top of every file
such that all constituent files can be compiled individually.

%%%%%%%%%%%%%%%%%%%%%%%%%%%%%%%%%%%%%%%%%%%%%%%%%%%%%%%%%%%%%%%%%%%%%%%%%%%%%%%%
%\subsection{Feature Suggestions}
%
%The following is a list of features which may be useful for future
%versions of this package:
%%
%\begin{itemize}
%\item
%\ldots
%\end{itemize}

%%%%%%%%%%%%%%%%%%%%%%%%%%%%%%%%%%%%%%%%%%%%%%%%%%%%%%%%%%%%%%%%%%%%%%%%%%%%%%%%
\subsection{Revision History}

%%%%%%%%%%%%%%%%%%%%%%%%%%%%%%%%%%%%%%%%
\paragraph{v2.0:} 2018/12/30

\begin{itemize}
\item
immediate forward processing
\item
added |\childdocby| mechanism
\item
manual restructured
\end{itemize}

%%%%%%%%%%%%%%%%%%%%%%%%%%%%%%%%%%%%%%%%
\paragraph{v1.6:} 2018/01/17

\begin{itemize}
\item
application for development of include files
\item
corrections to manual
\end{itemize}

%%%%%%%%%%%%%%%%%%%%%%%%%%%%%%%%%%%%%%%%
\paragraph{v1.5:} 2017/05/21

\begin{itemize}
\item
more complete structuring introduced
\item
|\childdocof| introduced
\item
|\childdoc| renamed to |\childdocmain|
\item
|\childredirect| renamed to |\childdocforward| and |\childdocforwardprefix|
and functionality expanded
\end{itemize}

%%%%%%%%%%%%%%%%%%%%%%%%%%%%%%%%%%%%%%%%
\paragraph{v1.0:} 2017/04/27

\begin{itemize}
\item
manual and install package
\item
first version published on CTAN
\end{itemize}

%%%%%%%%%%%%%%%%%%%%%%%%%%%%%%%%%%%%%%%%
\paragraph{v0.6:} 2017/04/26

\begin{itemize}
\item
redirection mechanism added
\end{itemize}

%%%%%%%%%%%%%%%%%%%%%%%%%%%%%%%%%%%%%%%%
\paragraph{v0.5:} 2017/04/26

\begin{itemize}
\item
functionality in definition file
\end{itemize}


%%%%%%%%%%%%%%%%%%%%%%%%%%%%%%%%%%%%%%%%%%%%%%%%%%%%%%%%%%%%%%%%%%%%%%%%%%%%%%%%
%%%%%%%%%%%%%%%%%%%%%%%%%%%%%%%%%%%%%%%%%%%%%%%%%%%%%%%%%%%%%%%%%%%%%%%%%%%%%%%%
%%%%%%%%%%%%%%%%%%%%%%%%%%%%%%%%%%%%%%%%%%%%%%%%%%%%%%%%%%%%%%%%%%%%%%%%%%%%%%%%
\appendix

\settowidth\MacroIndent{\rmfamily\scriptsize 000\ }

 \DocInput{childdoc.dtx}

\end{document}
%</driver>
% \fi
%
% %%%%%%%%%%%%%%%%%%%%%%%%%%%%%%%%%%%%%%%%%%%%%%%%%%%%%%%%%%%%%%%%%%%%%%%%%%%%%%
% %%%%%%%%%%%%%%%%%%%%%%%%%%%%%%%%%%%%%%%%%%%%%%%%%%%%%%%%%%%%%%%%%%%%%%%%%%%%%%
% \section{Sample}
%\iffalse
%<*samplemain>
%\fi
%
% The following presents a sample document
% with two chapters, two parts, a title page,
% a compile flag as well as three forwarding files to set the flag.
% It consists of eight |.tex| files:
% \begin{center}
% \begin{tabular}{ll}
% |cdocsamp.tex|&main file\\
% |cdocsch1.tex|&include file for chapter 1\\
% |cdocsch2.tex|&include file for chapter 2\\
% |cdocspt3.tex|&include file for part 3\\
% |cdocspt4.tex|&include file for part 4\\
% |cdocsdrf.tex|&forwarding file for main file in draft mode\\
% |cdocsfi1.tex|&forwarding file for final version of chapter 1\\
% |cdocsfi2.tex|&forwarding file for final version of chapter 2\\
% \end{tabular}
% \end{center}
% Each of the eight files can be compiled directly by the \LaTeX{} compiler.
%
% %%%%%%%%%%%%%%%%%%%%%%%%%%%%%%%%%%%%%%
% \paragraph{Main File.}
%
% The main file is called |cdocsamp.tex|.
%
% Load the \textsf{childdoc} definitions and
% declare the filename for the main document:
%    \begin{macrocode}
\input{childdoc.def}
\childdocmain{}
%    \end{macrocode}

% Optional override for |\version| flag:
%    \begin{macrocode}
%%\ifchilddoc\else\providecommand{\version}{draft}\fi
%    \end{macrocode}

% Define the default values for the |\version| flag
% (|final| for the main file and |draft| for childs):
%    \begin{macrocode}
\ifchilddoc
\providecommand{\version}{draft}
\else
\providecommand{\version}{final}
\fi
%    \end{macrocode}

% Load the standard document class:
%    \begin{macrocode}
\documentclass[12pt]{article}
%    \end{macrocode}

% Start the document body:
%    \begin{macrocode}
\begin{document}
%    \end{macrocode}

% Declare a title page.
% Print title, part of document being processed and version flag:
%    \begin{macrocode}
\addtocounter{page}{-1}
\begin{center}
{\LARGE\bfseries{}childdoc example\par}
\vspace{1cm}
\ifchilddoc
\ifchilddocmanual part\else chapter\fi:
`\childdocname' of `\childdocjob'\par
\else
main document: `\childdocjob'\par
\fi
version: \version\par
\end{center}
\newpage
%    \end{macrocode}

% Manually include selected file,
% otherwise process as usual:
%    \begin{macrocode}
\ifchilddocmanual
\section*{part `\childdocname'}
\input{\childdocname}
\else
%    \end{macrocode}

% Include the two chapters:
%    \begin{macrocode}
\include{cdocsch1}
\include{cdocsch2}
%    \end{macrocode}

% Include the two parts unless only chapters should be displayed:
%    \begin{macrocode}
\ifchilddoc\else
\section{part three}
\input{cdocspt3}
\section{part four}
\input{cdocspt4}
\fi
%    \end{macrocode}

% Process as usual until here:
%    \begin{macrocode}
\fi
%    \end{macrocode}

% End of document body:
%    \begin{macrocode}
\end{document}
%    \end{macrocode}
%\iffalse
%</samplemain>
%\fi
%
% %%%%%%%%%%%%%%%%%%%%%%%%%%%%%%%%%%%%%%
% \paragraph{Chapter Include Files.}
%
% The include files are called |cdocsch1.tex| and |cdocsch2.tex|.
%
%\iffalse
%<*samplechap1|samplechap2>
%\fi

% Optional override for |\version| flag:
%    \begin{macrocode}
%%\providecommand{\version}{final}
%    \end{macrocode}

% Include the main document:
%    \begin{macrocode}
\input{childdoc.def}
\childdocof{cdocsamp}
%    \end{macrocode}

%\iffalse
%</samplechap1|samplechap2>
%\fi
%
%\iffalse
%<*samplechap1>
%\fi
% Some text for chapter 1:
%    \begin{macrocode}
\section{one}
some text in chapter one
%    \end{macrocode}

%\iffalse
%</samplechap1>
%\fi
% Some text for chapter 2:
%\iffalse
%<*samplechap2>
%\fi
%    \begin{macrocode}
\section{two}
more text in chapter two
%    \end{macrocode}

%\iffalse
%</samplechap2>
%\fi
%
% %%%%%%%%%%%%%%%%%%%%%%%%%%%%%%%%%%%%%%
% \paragraph{Part Include Files.}
%
% The include files are called |cdocspt3.tex| and |cdocspt4.tex|.
%
%\iffalse
%<*samplepart3|samplepart4>
%\fi

% Optional override for |\version| flag:
%    \begin{macrocode}
%%\providecommand{\version}{final}
%    \end{macrocode}

% Include the main document:
%    \begin{macrocode}
\input{childdoc.def}
\childdocby{cdocsamp}
%    \end{macrocode}

%\iffalse
%</samplepart3|samplepart4>
%\fi
%
%\iffalse
%<*samplepart3>
%\fi
% Some text for part 3:
%    \begin{macrocode}
some text in part three
%    \end{macrocode}

%\iffalse
%</samplepart3>
%\fi
% Some text for part 4:
%\iffalse
%<*samplepart4>
%\fi
%    \begin{macrocode}
more text in part four
%    \end{macrocode}

%\iffalse
%</samplepart4>
%\fi
%
% %%%%%%%%%%%%%%%%%%%%%%%%%%%%%%%%%%%%%%
% \paragraph{Forwarding for a Complete Draft.}
%
% The following forwarding file |cdocsdrf.tex|
% compiles the main document in draft mode:
%\iffalse
%<*sampledraft>
%\fi
%    \begin{macrocode}
\def\version{draft}
\input{childdoc.def}
\childdocforward{cdocsamp}
%    \end{macrocode}

%\iffalse
%</sampledraft>
%\fi
%
% %%%%%%%%%%%%%%%%%%%%%%%%%%%%%%%%%%%%%%
% \paragraph{Forwarding for Final Version of the Chapters.}
%
% The following forwarding files |cdocsfn1.tex| and |cdocsfn2.tex|
% (with identical content)
% compile the final versions of the child documents
% |cdocsch1.tex| and |cdocsch2.tex|, respectively:
%\iffalse
%<*samplefinal>
%\fi
%    \begin{macrocode}
\def\version{final}
\input{childdoc.def}
\childdocforwardprefix[cdocsamp]{cdocsfn}{cdocsch}
%    \end{macrocode}

%\iffalse
%</samplefinal>
%\fi
%
% %%%%%%%%%%%%%%%%%%%%%%%%%%%%%%%%%%%%%%
% \paragraph{Command Line Processing.}
%
% The following three command lines generate the output files
% |cdocscld|, |cdocscl1| and |cdocscl2|
% which should be identical to
% |cdocsdrf|, |cdocsch1| and |cdocsfn2|, respectively:
% \begin{center}
% \begin{tabular}{l}
% |latex -jobname cdocscld \|\\
% |  "\def\version{draft}\input{childdoc.def}\childdocforward{cdocsamp}"|\\
% |latex -jobname cdocscl1 \|\\
% |  "\input{childdoc.def}\childdocforward[cdocsamp]{cdocsch1}"|\\
% |latex -jobname cdocscl2 \|\\
% |  "\def\version{final}\input{childdoc.def}\childdocforward{cdocsch2}"|
% \end{tabular}
% \end{center}
% Note that the trailing backslash on each first line
% merely continues the input to the second line
% (for convenient cut ant paste).
% Furthermore, the command |latex| can be replaced by any
% of its alternative versions such as |pdflatex|.
%
% %%%%%%%%%%%%%%%%%%%%%%%%%%%%%%%%%%%%%%%%%%%%%%%%%%%%%%%%%%%%%%%%%%%%%%%%%%%%%%
% %%%%%%%%%%%%%%%%%%%%%%%%%%%%%%%%%%%%%%%%%%%%%%%%%%%%%%%%%%%%%%%%%%%%%%%%%%%%%%
% \section{Implementation}
%\iffalse
%<*package>
%\fi
%
% This section describes the definitions file |childdoc.def|.

% The definitions cannot be loaded using |\usepackage| or |\RequirePackage|
% which has a mechanism to prevent loading a style file more than once.
% When loading the definitions by means of |\input|
% multiple instances have to be prevented manually:
%\iffalse
%This code needs to be before the `\ProvidesFile' directive
%which is defined at the beginning of this file.
%Therefore it is also placed there and commented out here.
%</package>
%<*discard>
%\fi
%    \begin{macrocode}
\ifdefined\childdocmain\endinput\fi
%    \end{macrocode}
%\iffalse
%</discard>
%<*package>
%\fi
%
% \macro{\ifchilddoc}
% \macro{\ifchilddocmanual}
% The conditional |\ifchilddoc| tells whether a
% child (true) or main (false) document is being compiled.
% The conditional |\ifchilddocmanual| tells whether
% the |\includeonly| mechanism is used (false) or
% the selection of child files must be performed manually (true).
% The definitions initialise to false:
%    \begin{macrocode}
\newif\ifchilddoc
\newif\ifchilddocmanual
%    \end{macrocode}

% \macro{\childdocname}
% \macro{\childdocjob}
% The macro |\childdocname| stores the name of the main document
% to be compiled. The macro |\childdocjob| stores the name of
% the document on which the \LaTeX{} compiler was originally invoked.
% The content of |\jobname| cannot be compared
% to filenames specified in the source due to different catcodes.
% The following code rescans |\jobname|, stores the result
% in |\childdocname| and saves a copy in |\childdocjob|:
%    \begin{macrocode}
\edef\childdocname{\scantokens\expandafter{\jobname\noexpand}}
\let\childdocjob\childdocname
%    \end{macrocode}

% \macro{\childdocdisable}
% The macro |\childdocdisable| prevents the main file
% from being processed more than once.
% At this stage, the main document command |\childdocmain|
% is assumed to be called once again where it should do nothing.
% Any subsequent call to it should prevent
% a secondary processing of the main document
% It overwrites the forwarding commands
% |\childdocof| and |\childdocforward|
% with empty macros to prevent further inclusions of the main document:
%    \begin{macrocode}
\newcommand{\childdocdisable}
{
  \renewcommand{\childdocmain}[1]{\renewcommand{\childdocmain}[1]{\endinput}}
  \renewcommand{\childdocof}[1]{}
  \renewcommand{\childdocby}[2][]{}
  \renewcommand{\childdocforward}[2][]{}
  \renewcommand{\childdocdisable}{}
}
%    \end{macrocode}

% \macro{\childdocmain}
% The macro |\childdocmain| is to be called at the top of the main file
% with nothing or the main filename (without extension) as argument.
% First, it breaks loops.
% If the argument is not empty and does not match |\childdocname|
% (which is set by the first inclusion of |childdoc.def|),
% |\ifchilddoc| is set to true, |\includeonly| is applied to the child file
% and |\jobname| is set to the main file
% (for proper handling of |.aux| files):
%    \begin{macrocode}
\newcommand{\childdocmain}[1]
{
  \childdocdisable\childdocmain{}
  \if?#1?\else
    \begingroup
      \def\childdoctmp{#1}
      \ifx\childdoctmp\childdocname
        \def\childdoctmp{}
      \else
        \def\childdoctmp
        {
          \childdoctrue
          \includeonly{\childdocname}
          \def\childdocjob{#1}
          \def\jobname{#1}
        }
      \fi
      \expandafter
    \endgroup
    \childdoctmp
  \fi
}
%    \end{macrocode}

% \macro{\childdocof}
% The command |\childdocof| redirects
% compilation to the main file |#1|.
%    \begin{macrocode}
\newcommand{\childdocof}[1]
{
  \childdocdisable
  \childdoctrue
  \includeonly{\childdocname}
  \def\jobname{#1}
  \def\childdocjob{#1}
  \input{#1}
}
%    \end{macrocode}

% \macro{\childdocby}
% The command |\childdocby| ....
%    \begin{macrocode}
\newcommand{\childdocby}[2][]
{
  \childdocdisable
  \childdoctrue
  \childdocmanualtrue
  \if?#1?\else
    \def\jobname{#2}
  \fi
  \def\childdocjob{#2}
  \input{#2}
  \endinput
}
%    \end{macrocode}

% \macro{\childdocforward}
% The command |\childdocforward| redirects
% compilation to the main file or
% (if the optional argument is given) a child file.
% Parameters are set as if the main file
% or a child file starting with |\childdocof| was compiled.
% Then compilation is handed over to the main file:
%    \begin{macrocode}
\newcommand{\childdocforward}[2][]
{
  \begingroup
    \if?#1?
      \def\childdoctmp
      {
        \def\childdocname{#2}
        \def\childdocjob{#2}
        \def\jobname{#2}
        \input{#2}
        \endinput
      }
    \else
      \def\childdoctmp
      {
        \childdocdisable
        \def\childdocname{#2}
        \childdoctrue
        \includeonly{#2}
        \def\childdocjob{#1}
        \def\jobname{#1}
        \input{#1}
        \endinput
      }
    \fi
    \expandafter
  \endgroup
  \childdoctmp
}
%    \end{macrocode}

% \macro{\childdocforwardprefix}
% The command |\childdocforwardprefix| redirects
% compilation to the main or a child file by means of a pattern.
% The prefix |#1| in the current filename is replaced by |#2|
% and the suffix of the current filename is kept
% (it is assumed that the filename does not contain the substring `|~~~|'
% which is used as a delimiter).
% Compilation is handed over to the new file by |\childdocforward|:
%    \begin{macrocode}
\newcommand{\childdocforwardprefix}[3][]
{
  \begingroup
    \def\childdocextract #2##1~~~{\def\childdoctmp{\childdocforward[#1]{#3##1}}}
    \expandafter\childdocextract\childdocname~~~
    \expandafter
  \endgroup
  \childdoctmp
}
%    \end{macrocode}

% \macro{\childdoc}
% The deprecated macro |\childdoc| is a legacy version of |\childdocmain|:
%    \begin{macrocode}
\newcommand{\childdoc}{\childdocmain}
%    \end{macrocode}

% \macro{\childdocredirect}
% The deprecated macro |\childdocredirect| is a legacy version
% of |\childdocforward| and |\childdocforwardprefix|:
%    \begin{macrocode}
\newcommand{\childdocredirect}[2][]
{
  \begingroup
    \if?#1?
      \def\childdoctmp{\childdocforward{#2}}
    \else
      \def\childdoctmp{\childdocforwardprefix{#1}{#2}}
    \fi
    \expandafter
  \endgroup
  \childdoctmp
}
%    \end{macrocode}

%\iffalse
%</package>
%\fi
%
\endinput
\childdocforward[cdocsamp]{cdocsch1}"|\\
% |latex -jobname cdocscl2 \|\\
% |  "\def\version{final}% \iffalse
%
% childdoc.dtx Copyright (C) 2017-2018 Niklas Beisert
%
% This work may be distributed and/or modified under the
% conditions of the LaTeX Project Public License, either version 1.3
% of this license or (at your option) any later version.
% The latest version of this license is in
%   http://www.latex-project.org/lppl.txt
% and version 1.3 or later is part of all distributions of LaTeX
% version 2005/12/01 or later.
%
% This work has the LPPL maintenance status `maintained'.
%
% The Current Maintainer of this work is Niklas Beisert.
%
% This work consists of the files childdoc.dtx and childdoc.ins
% and the derived files childdoc.def and cdocsamp.tex with
% cdocsch1.tex, cdocsch2.tex, cdocsdrf.tex, cdocsfn1.tex, cdocsfn2.tex.
%
%<package>\ifdefined\childdocmain\endinput\fi
%<package>\ProvidesFile{childdoc.def}[2018/12/30 v2.0 child document driver]
%<samplemain>\ProvidesFile{cdocsamp.tex}[2018/12/30 v2.0 sample for childdoc]
%<*driver>
%\ProvidesFile{childdoc.drv}[2018/12/30 v2.0 childdoc reference manual file]
\PassOptionsToClass{10pt,a4paper}{article}
\documentclass{ltxdoc}

\usepackage[margin=35mm]{geometry}
\usepackage{hyperref}
\usepackage{hyperxmp}
\usepackage[usenames]{color}

\hypersetup{colorlinks=true}
\hypersetup{pdfstartview=FitH}
\hypersetup{pdfpagemode=UseNone}
\hypersetup{pdfsource={}}
\hypersetup{pdflang={en-UK}}
\hypersetup{pdfcopyright={Copyright 2017-2018 Niklas Beisert.
  This work may be distributed and/or modified under the
  conditions of the LaTeX Project Public License, either version 1.3
  of this license or (at your option) any later version.}}
\hypersetup{pdflicenseurl={http://www.latex-project.org/lppl.txt}}
\hypersetup{pdfcontactaddress={ETH Zurich, ITP, HIT K,
  Wolfgang-Pauli-Strasse 27}}
\hypersetup{pdfcontactpostcode={8093}}
\hypersetup{pdfcontactcity={Zurich}}
\hypersetup{pdfcontactcountry={Switzerland}}
\hypersetup{pdfcontactemail={nbeisert@itp.phys.ethz.ch}}
\hypersetup{pdfcontacturl={http://people.phys.ethz.ch/\xmptilde nbeisert/}}

\newcommand{\secref}[1]{\hyperref[#1]{section \ref*{#1}}}

\parskip1ex
\parindent0pt
\let\olditemize\itemize
\def\itemize{\olditemize\parskip0pt}

\begin{document}

\title{The \textsf{childdoc} Package}
\hypersetup{pdftitle={The childdoc Package}}
\author{Niklas Beisert\\[2ex]
  Institut f\"ur Theoretische Physik\\
  Eidgen\"ossische Technische Hochschule Z\"urich\\
  Wolfgang-Pauli-Strasse 27, 8093 Z\"urich, Switzerland\\[1ex]
  \href{mailto:nbeisert@itp.phys.ethz.ch}
  {\texttt{nbeisert@itp.phys.ethz.ch}}}
\hypersetup{pdfauthor={Niklas Beisert}}
\hypersetup{pdfsubject={Manual for the LaTeX2e Package childdoc}}
\date{30 December 2018, \textsf{v2.0}}
\maketitle

\begin{abstract}\noindent
\textsf{childdoc} is a \LaTeXe{} package
that enables the direct compilation
of document sections included by |\include|
to individual files.
\end{abstract}

\begingroup
\parskip0ex
\tableofcontents
\endgroup

%%%%%%%%%%%%%%%%%%%%%%%%%%%%%%%%%%%%%%%%%%%%%%%%%%%%%%%%%%%%%%%%%%%%%%%%%%%%%%%%
%%%%%%%%%%%%%%%%%%%%%%%%%%%%%%%%%%%%%%%%%%%%%%%%%%%%%%%%%%%%%%%%%%%%%%%%%%%%%%%%
\section{Introduction}

\LaTeX{} provides a mechanism to structure a large document (such as a book)
into a main file and several child files (containing the chapters)
using the |\include| command.
This mechanism is beneficial for documents
which span hundreds of pages in order to
make the source file(s) more manageable.
Moreover, compilation can be restricted to
selected child files by means of the |\includeonly| command.
The latter feature can be used to reduce the compilation time while editing
(this was significantly more useful in the earlier days of \LaTeX{})
or to generate a smaller document which is easier to navigate.
Another application of |\includeonly| is to generate
documents consisting of selected parts of the complete document.

However, there are a few drawbacks of the plain |\include| mechanism:
\begin{itemize}
\item
The child files cannot be compiled on their own,
they can only be compiled via the main file.
A naive editing environment
(such as a text editor with an option
to have the current file processed by \LaTeX)
may require one to switch to the main file before compiling;
attempting to compile the child file produces errors.
\item
The main file must be modified (each time)
to adjust the |\includeonly| command
to the present needs. This easily leaves the main file in a messy state.
\item
The generated document will always carry the filename
of the main document. This is inconvenient if
several child files are to be compiled and
to be kept for distribution.
\end{itemize}

The present package provides a simple interface
to make child files individually compilable by \LaTeX{}.
Compiling a child file then has the same effect as compiling
the main file with an |\includeonly| command
to select the appropriate child.
Moreover the generated document will carry the name of the child
rather than the main file.
This resolves all three above issues.

This feature is meant to make the editing of books,
thesis documents and lecture notes somewhat more convenient.
However, the package can also be used efficiently for
composing a series of documents (such as exercise sheets)
which are typically distributed individually.
It then assists the author in generating the individual documents
(potentially in different versions)
as well as a document containing the collected series.
Another application is in developing style files
or other kinds of included material
where compilation of the style file could redirect
to a sample or test file.

%%%%%%%%%%%%%%%%%%%%%%%%%%%%%%%%%%%%%%%%%%%%%%%%%%%%%%%%%%%%%%%%%%%%%%%%%%%%%%%%
%%%%%%%%%%%%%%%%%%%%%%%%%%%%%%%%%%%%%%%%%%%%%%%%%%%%%%%%%%%%%%%%%%%%%%%%%%%%%%%%
\section{Usage}

First of all, the package \textsf{childdoc} is \emph{not} a standard
\LaTeXe{} |.sty| style file! Therefore it needs to be invoked in
a non-standard way.

%%%%%%%%%%%%%%%%%%%%%%%%%%%%%%%%%%%%%%%%%%%%%%%%%%%%%%%%%%%%%%%%%%%%%%%%%%%%%%%%
\subsection{Included Files}
\label{sec:include}

%%%%%%%%%%%%%%%%%%%%%%%%%%%%%%%%%%%%%%%%
\DescribeMacro{\childdocmain}
To use the package, add the commands
\begin{center}
\begin{tabular}{l}
|\input{childdoc.def}|\\
|\childdocmain{}|\\
\end{tabular}
\end{center}
at the very top of the main \LaTeX{} file,
in particular \emph{before} the |\documentclass| statement!
The argument of |\childdocmain| should be left empty
(but it must be present).

%%%%%%%%%%%%%%%%%%%%%%%%%%%%%%%%%%%%%%%%
\DescribeMacro{\childdocof}
Furthermore, add the commands
\begin{center}
\begin{tabular}{l}
|\input{childdoc.def}|\\
|\childdocof{|\textit{main}|}|\\
\end{tabular}
\end{center}
at the top of every child file \textit{child}
which is included by |\include{|\textit{child}|}|
from within the main file
(or at least for those files to be compiled individually).
The argument \textit{main} must be the filename of the main file.

There are a couple of
considerations in setting up the main and child documents:

%%%%%%%%%%%%%%%%%%%%%%%%%%%%%%%%%%%%%%%%
\paragraph{Restrictions.}

Please note the following restrictions:
\begin{itemize}
\item
|\childdocmain| must be called with one argument \textit{main}
to ensure compatibility with earlier version of the package.
It must either be empty (|\childdocmain{}|)
or precisely match the filename of the main file in which it is specified.
See \secref{sec:detection} for further information.
\item
The filename \textit{main} must be specified without the |.tex| extension.
\item
The filename \textit{main} is case sensitive
(even in case-insensitive file systems)
due to internal string comparison.
\item
The argument \textit{main} should be fully expanded, it cannot be a macro.
\item
Subdirectories and special characters should be avoided in filenames.
\item
The command |\childdocmain{|\textit{main}|}| must be followed by a whitespace.
It should not be followed immediately by another command
or by a comment mark `|%|'.
This is because the \TeX{} parser reads the token immediately following
the argument of |\childdocmain| and puts it
at the beginning of every child section;
however, a white\-space is ignored.
\end{itemize}

%%%%%%%%%%%%%%%%%%%%%%%%%%%%%%%%%%%%%%%%
\paragraph{Content of Main File.}

It is advisable to place all content in the child files included by |\include|.
Any output contained in the main file will appear in all child documents
unless suppressed manually;
it cannot be suppressed automatically by the |\includeonly| directive
and thus should normally be avoided.
A method to include some content in the main file
by means of conditional processing is described in \secref{sec:conditional}.

%%%%%%%%%%%%%%%%%%%%%%%%%%%%%%%%%%%%%%%%
\paragraph{Page Numbering.}

When only a part of the document is compiled,
the appropriate numbering of pages
(as well as other status parameters)
is determined from the |.aux| files.
The latter contain information from previous passes.
However this information needs to propagate through
all intermediate child documents.
Therefore the page numbering in child documents may well
be inconsistent until the complete document is compiled at least once.

A useful (if unconventional) way to always ensure a consistent
page numbering is to restart the numbering in each child document
and denote the pages by `\textit{child}|.|\textit{page}'
where \textit{child} represents the chapter/section number of the child file.
This can be achieved by the command
|\numberwithin{page}{|\textit{child}|}|
of the \textsf{amsmath} package
where \textit{child} can be |chapter| or |section|
depending on the chosen structuring.
Alternatively, one can modify the macro |\thepage| appropriately
and reset the counter |page| at the start of each child file.

%%%%%%%%%%%%%%%%%%%%%%%%%%%%%%%%%%%%%%%%%%%%%%%%%%%%%%%%%%%%%%%%%%%%%%%%%%%%%%%%
\subsection{Conditional Processing}
\label{sec:conditional}

The package provides a mechanism to compile different versions
of a document. To customise the versions further some conditional processing
can come in handy to distinguish which version is being compiled.
The package provides two macros to describe the compilation context:

%%%%%%%%%%%%%%%%%%%%%%%%%%%%%%%%%%%%%%%%
\DescribeMacro{\ifchilddoc}
The conditional |\ifchilddoc| distinguishes between the compilation of
child documents and the main document:
%
\begin{center}
|\ifchilddoc |\textit{child-code}| |[|\||else |\textit{main-code}]| \||fi|
\end{center}

%%%%%%%%%%%%%%%%%%%%%%%%%%%%%%%%%%%%%%%%
\DescribeMacro{\childdocname}
\DescribeMacro{\childdocjob}
The macro |\childdocname| contains the filename (without extension)
of the main or child file being processed.
Note that |\childdocjob| will always contain the name of the main file.

%%%%%%%%%%%%%%%%%%%%%%%%%%%%%%%%%%%%%%%%
\paragraph{Title Page.}

Conditional processing can be used to include a title or banner page
in the main document when proper precautions are taken.
Importantly, the code in the main file should ensure that the page counter
(as well as other status parameters which are stored in the |.aux| files)
takes the same value after the conditional processing.
Otherwise the page numbers may take divergent values
depending on which part is compiled.

For example, a title page could be declared by:
%
\begin{center}
\begin{tabular}{l}
|\ifchilddoc\||else|\\
|\addtocounter{page}{-1}|\\
\textit{code for title page}\\
|\newpage|\\
|\||fi|
\end{tabular}
\end{center}
%
A banner page for the child documents can be generated by:
%
\begin{center}
\begin{tabular}{l}
|\ifchilddoc|\\
|\addtocounter{page}{-1}|\\
\textit{code for banner page}\\
|\newpage|\\
|\||fi|
\end{tabular}
\end{center}
%
Here one could write a message such as:
\begin{center}
|This is the part \childdocname{} of \childdocjob{}.|
\end{center}

%%%%%%%%%%%%%%%%%%%%%%%%%%%%%%%%%%%%%%%%%%%%%%%%%%%%%%%%%%%%%%%%%%%%%%%%%%%%%%%%
\subsection{Flags}
\label{sec:flags}

The package makes it easy to generate different versions
of the main or child documents.
To this end compilation flags can be defined
and assigned different default values.
They will be particularly useful in conjunction
with the forwarding mechanism described in \secref{sec:forward}.

For example, it may be useful to have a flag |\version|
which can be set to |draft| or |final|.
The document source will contain some conditional code
depending on the value of |\version|.
Suppose further, the flag should default to |final| for the main file
and to |draft| for child files
which is a natural assignment for editing the document.
This is achieved by placing the following code
in the preamble of the main document
(below the |\childdocmain| directive):
%
\begin{center}
\begin{tabular}{l}
|\ifchilddoc|\\
|\providecommand{\version}{draft}|\\
|\||else|\\
|\providecommand{\version}{final}|\\
|\||fi|
\end{tabular}
\end{center}
%
The definition by |\providecommand| makes sure
that previous definitions are not overwritten.
Further statements |\providecommand{\version}{...}|
can thus be added before the above code to override it.

For the main file, one might add a line
(between |\childdocmain| and the above block)
%
\begin{center}
|%\ifchilddoc\||else\providecommand{\version}{draft}\||fi|
\end{center}
%
which can be uncommented to produce a draft version.
Likewise one can add a line to the very top of a child file
(above the |\childdocof{|\textit{main}|}| directive)
%
\begin{center}
|%\providecommand{\version}{final}|
\end{center}
%
which can be uncommented to produce the final version of this child document.

%%%%%%%%%%%%%%%%%%%%%%%%%%%%%%%%%%%%%%%%%%%%%%%%%%%%%%%%%%%%%%%%%%%%%%%%%%%%%%%%
\subsection{Forwarding}
\label{sec:forward}

Different versions of the main or child documents
using compilation flags as described in \secref{sec:flags}
can be (permanently) stored in different files
for convenient compilation, viewing and distribution.
To this end, the package defines a command
to pass on compilation to a different file:

%%%%%%%%%%%%%%%%%%%%%%%%%%%%%%%%%%%%%%%%
\DescribeMacro{\childdocforward}
The command |\childdocforward| redirects processing to
another source file:
%
\begin{center}
\begin{tabular}{l}
|\input{childdoc.def}|\\
|\childdocforward[|\textit{main}|]{|\textit{dest}|}|\\
\end{tabular}
\end{center}
%
The argument \textit{dest} is the destination file
(without extension).
It should be the main file or one of the child files.
Note that further \textsf{childdoc} directives
such as |\childdocof| and |\childdocforward|
in the indicated file will be processed in this form.
The optional argument \textit{main}
passes on directly to the main file \textit{main}
while pretending to compile the child \textit{dest}.
This form behaves as if \textit{dest}
issues |\childdocof{|\textit{main}|}| right away,
and no further \textsf{childdoc} directives will be processed.

%%%%%%%%%%%%%%%%%%%%%%%%%%%%%%%%%%%%%%%%
\DescribeMacro{\...prefix}
In the alternative form |\childdocforwardprefix|,
%
\begin{center}
\begin{tabular}{l}
|\input{childdoc.def}|\\
|\childdocforwardprefix[|\textit{main}|]{|\textit{prefix}|}{|\textit{dest}|}|
\end{tabular}
\end{center}
%
the destination file is determined by a pattern
depending on the current file:
To make this work, the current file must be called
`{\textit{prefix}\hspace{0.2em}\textit{suffix}}'
with \textit{prefix} matching precisely the argument.
Processing is then passed on to the file
`{\textit{dest}\hspace{0.2em}\textit{suffix}}'.
Surely, the same effect is achieved by
directly specifying the
argument `{\textit{dest}\hspace{0.2em}\textit{suffix}}'
in the first form.
However, that requires to set up a different file
for each child. With the alternative form of the command
all these files can have exactly the same content
which simplifies setting them up and maintaining them.

For example, the following file |draft.tex|
with a compilation flag |\version| as described in \secref{sec:flags}
compiles the main document as a draft:
%
\begin{center}
\begin{tabular}{l}
|\def\version{draft}|\\
|\input{childdoc.def}|\\
|\childdocforward{|\textit{main}|}|
\end{tabular}
\end{center}
%
Likewise, the following files |final|\textit{nn}|.tex|
compile the final version of the child document
|child|\textit{nn}|.tex|:
%
\begin{center}
\begin{tabular}{l}
|\def\version{final}|\\
|\input{childdoc.def}|\\
|\childdocforwardprefix{final}{child}|
\end{tabular}
\end{center}
%

Note that when several versions of a main file and/or of each child file
are to be generated, it may be convenient to set up a |Makefile| or
shell script to automatise the process.

%%%%%%%%%%%%%%%%%%%%%%%%%%%%%%%%%%%%%%%%%%%%%%%%%%%%%%%%%%%%%%%%%%%%%%%%%%%%%%%%
\subsection{Command Line Processing}
\label{sec:commandline}

The effect of redirection files can also be achieved by invoking
the \LaTeX{} compiler with a more elaborate command line.
Most conveniently this should be done as part
of a shell script or a |Makefile|.

When using \textsf{childdoc} in the main file, the following
command lines effectively perform a redirection
(note that depending on the shell being used,
backslashes may have to be doubled: `|\|' $\to$ `|\\|'):
%
\begin{center}
|... -jobname "|\textit{target}|" |\\|"|[\textit{flags}]%
|\input{childdoc.def}\childdocforward[|\textit{main}|]{|\textit{dest}|}"|
\end{center}
%
Here \textit{target} is the name of the output file,
\textit{main} is the name of the main file
and \textit{dest} is the name of the main or child file to be processed
(all filenames without extensions).
The optional argument \textit{main} can be omitted
if \textit{main} matches \textit{dest}.
Optionally, compilation \textit{flags} can be defined via |\def| commands.
This command line makes the \TeX{} engine believe
it is compiling the file \textit{target}
whose content is specified as the latter parameter.
The provided code then forwards the processing to
\textit{main} or \textit{dest} as described in \secref{sec:forward}.

%%%%%%%%%%%%%%%%%%%%%%%%%%%%%%%%%%%%%%%%%%%%%%%%%%%%%%%%%%%%%%%%%%%%%%%%%%%%%%%%
\subsection{Include by Input}
\label{sec:input}

Including child documents by |\include| has some restrictions by design.
Most notably, the content of a child document always occupies
its own set of pages; pages cannot be shared between child documents.
Usually, this behaviour makes perfect sense
because each child document contain an essential part of the document.
However, in some situations it may be desirable to compose
a document from a collection of parts
without having mandatory page breaks between then.
For this case, the package
provides a mechanism to include parts
by |\input| which can also be processed individually.
However, by construction this mechanism
requires manual handling of the content to be output.

%%%%%%%%%%%%%%%%%%%%%%%%%%%%%%%%%%%%%%%%
\DescribeMacro{\ifchilddocmanual}
The main file should be prepared as usual, see \secref{sec:include}.
However, the document body must make a distinction
between processing of an individual part and of the main document, e.g.:
%
\begin{center}
\begin{tabular}{l}
|\ifchilddocmanual|\\
|\input{\childdocname}|\\
|\||else|\\
\textit{document body with }|\input{|\textit{part}|}|\\
|\||fi|
\end{tabular}
\end{center}
%
The conditional |\ifchilddocmanual| is true whenever
a part to be included by |\input| is being compiled,
and the name of the part is stored in |\childdocname|.

%%%%%%%%%%%%%%%%%%%%%%%%%%%%%%%%%%%%%%%%
\DescribeMacro{\childdocby}
Each part to be included by |\input| should start with:
%
\begin{center}
\begin{tabular}{l}
|\input{childdoc.def}|\\
|\childdocby{|\textit{main}|}|\\
\end{tabular}
\end{center}
%
The directive |\childdocby| is similar to |\childdocof|
described in \secref{sec:include},
but the subsequent selection of content must be done manually.
To that end, both |\ifchilddoc| and |\ifchilddocmanual|
will be true upon processing of a part,
and the name of the part is stored in |\childdocname|.
Note that |\jobname| will be set to the filename of the current part
so that each part receives an individual |.aux| file
that does not interfere with the |.aux| file(s) of the main document.
This behaviour can be altered by the alternative form
|\childdocby[*]{|\textit{main}|}| (with a non-empty optional argument)
which uses the |.aux| file of the main document
by setting |\jobname| to \textit{main}.

%%%%%%%%%%%%%%%%%%%%%%%%%%%%%%%%%%%%%%%%%%%%%%%%%%%%%%%%%%%%%%%%%%%%%%%%%%%%%%%%
\subsection{Driver Development}
\label{sec:driver}

The \textsf{childdoc} mechanism can also be use for the development
of definition files such as \LaTeX{} styles or classes.
This case differs from the above setup with multiple parts
included by |\include| in that no |\includeonly| should be invoked.
This can be achieved by starting the include file
(before |\ProvidesPackage|) with:
%
\begin{center}
\begin{tabular}{l}
|\input{childdoc.def}|\\
|\childdocforward{|\textit{main}|}|\\
\end{tabular}
\end{center}
%
or alternatively with:
%
\begin{center}
\begin{tabular}{l}
|\input{childdoc.def}|\\
|\childdocby{|\textit{main}|}|\\
\end{tabular}
\end{center}
%
Both forms have slightly different effects as described above.
The main file is prepared as usual, see \secref{sec:include}.

%%%%%%%%%%%%%%%%%%%%%%%%%%%%%%%%%%%%%%%%%%%%%%%%%%%%%%%%%%%%%%%%%%%%%%%%%%%%%%%%
\subsection{Legacy Detection}
\label{sec:detection}

The directive |\childdocmain| in the main file can detect
whether the complete document or merely a child is to be compiled
even without using the directive |\childdocof|.
This method is deprecated because it is less robust
and there is no compelling reason to use it;
it is merely provided for backward compatibility
and it may be removed in future versions.

If the detection mechanism is to be used,
it is mandatory to correctly specify
the filename of the main file as the argument of |\childdocmain|:
%
\begin{center}
\begin{tabular}{l}
|\input{childdoc.def}|\\
|\childdocmain{|\textit{main}|}|\\
\end{tabular}
\end{center}
%
If |\jobname| does not match the argument \textit{main} of |\childdocmain|,
it is assumed that |\jobname| points to the child file to be compiled.
When using |\childdocmain| with the main file specified as argument,
it suffices to start a child file
with just |\input{|\textit{main}|}|
without loading of the package and using |\childdocof|.
If instead all processing is done
with the appropriate \textsf{childdoc} directives,
the argument of \textit{main} of |\childdocmain| can be empty.

An alternative version of the command line processing described
in \secref{sec:commandline} using the detection mechanism reads:
%
\begin{center}
|... -jobname "|\textit{target}|" "|[\textit{flags}]%
[|\def\jobname{|\textit{dest}|}|]|\input{|\textit{main}|}"|
\end{center}

%%%%%%%%%%%%%%%%%%%%%%%%%%%%%%%%%%%%%%%%%%%%%%%%%%%%%%%%%%%%%%%%%%%%%%%%%%%%%%%%
\subsection{Manual Code}
\label{sec:manual}

In case one cannot be certain whether the definitions file |childdoc.def|
is installed on the target \TeX{} distribution
and one prefers not to ship it,
it is conceivable to paste a few relevant commands into the sources.

To that end, drop all statements |\input{childdoc.def}|
and perform the replacements as outlined below.
Instead of |\childdocmain{|\textit{main}|}| add the following code
to the top of the main file:
%
\begin{center}
\begin{tabular}{l}
|\||ifdefined\childdocname\endinput\||fi\newif\ifchilddoc|\\
|\edef\childdocname{\scantokens\expandafter{\jobname\noexpand}}|\\
|\def\childdocmain{|\textit{main}|}\||ifx\childdocmain\childdocname\||else|\\
|\childdoctrue\includeonly{\childdocname}\let\jobname\childdocmain\||fi|\\
\end{tabular}
\end{center}
%
Instead of |\childdocof{|\textit{main}|}| just include the main file
at the top of each child file:
%
\begin{center}
|\input{|\textit{main}|}|
\end{center}
%
A simple redirection |\childdocforward{|\textit{dest}|}| is achieved by:
%
\begin{center}
|\def\jobname{|\textit{dest}|}\input{\jobname}|
\end{center}
%
The redirection with prefix
|\childdocforwardprefix[|\textit{prefix}|]{|\textit{dest}|}|
is accomplished by:
%
\begin{center}
\begin{tabular}{l}
|{\edef\jobname{\scantokens\expandafter{\jobname\noexpand}}|\\
|\def\redirectjob |\textit{prefix}|#1~~~{\gdef\jobname{|\textit{dest}|#1}}|\\
|\expandafter\redirectjob\jobname~~~}\input{\jobname}|
\end{tabular}
\end{center}

In an alternative approach,
child documents can be compiled by a specific command line
without additional code or specific definitions:
%
\begin{center}
|... -jobname "|\textit{target}|" "|[\textit{flags}]%
|\includeonly{|\textit{dest}|}\input{|\textit{main}|}"|
\end{center}
%

%%%%%%%%%%%%%%%%%%%%%%%%%%%%%%%%%%%%%%%%%%%%%%%%%%%%%%%%%%%%%%%%%%%%%%%%%%%%%%%%
%%%%%%%%%%%%%%%%%%%%%%%%%%%%%%%%%%%%%%%%%%%%%%%%%%%%%%%%%%%%%%%%%%%%%%%%%%%%%%%%
\section{Information}

%%%%%%%%%%%%%%%%%%%%%%%%%%%%%%%%%%%%%%%%%%%%%%%%%%%%%%%%%%%%%%%%%%%%%%%%%%%%%%%%
\subsection{Copyright}

Copyright \copyright{} 2017--2018 Niklas Beisert

This work may be distributed and/or modified under the
conditions of the \LaTeX{} Project Public License, either version 1.3
of this license or (at your option) any later version.
The latest version of this license is in
  \url{http://www.latex-project.org/lppl.txt}
and version 1.3 or later is part of all distributions of \LaTeX{}
version 2005/12/01 or later.

This work has the LPPL maintenance status `maintained'.

The Current Maintainer of this work is Niklas Beisert.

This work consists of the files |README.txt|, |childdoc.ins| and |childdoc.dtx|
as well as the derived files |childdoc.def|, |cdocsamp.tex|
with |cdocsch1.tex|, |cdocsch2.tex|, |cdocspt3.tex|, |cdocspt4.tex|,
|cdocsdrf.tex|, |cdocsfn1.tex|, |cdocsfn2.tex|
as well as |childdoc.pdf|.

%%%%%%%%%%%%%%%%%%%%%%%%%%%%%%%%%%%%%%%%%%%%%%%%%%%%%%%%%%%%%%%%%%%%%%%%%%%%%%%%
\subsection{Files and Installation}

The package consists of the files:
%
\begin{center}
\begin{tabular}{ll}
    |README.txt|   & readme file \\
    |childdoc.ins| & installation file \\
    |childdoc.dtx| & source file \\
    |childdoc.def| & definition file \\
    |cdocsamp.tex| & sample main file \\
    |cdocsch1.tex| & sample include file \\
    |cdocsch2.tex| & sample include file \\
    |cdocspt3.tex| & sample part file \\
    |cdocspt4.tex| & sample part file \\
    |cdocsdrf.tex| & sample redirection file \\
    |cdocsfn1.tex| & sample redirection file \\
    |cdocsfn2.tex| & sample redirection file \\
    |childdoc.pdf| & manual
\end{tabular}
\end{center}
%
The distribution consists of the files
|README.txt|, |childdoc.ins| and |childdoc.dtx|.
%
\begin{itemize}
\item
Run (pdf)\LaTeX{} on |childdoc.dtx|
to compile the manual |childdoc.pdf| (this file).
\item
Run \LaTeX{} on |childdoc.ins| to create the definitions file |childdoc.def|
and the sample |cdocsamp.tex| with include files
|cdocsch1.tex|, |cdocsch2.tex|, |cdocspt3.tex|, |cdocspt4.tex|,
|cdocsdrf.tex|, |cdocsfn1.tex|, |cdocsfn2.tex|.
Then copy the file |childdoc.def| to an appropriate directory of your \LaTeX{}
distribution, e.g.\ \textit{texmf-root}|/tex/latex/childdoc|.
\end{itemize}

%%%%%%%%%%%%%%%%%%%%%%%%%%%%%%%%%%%%%%%%%%%%%%%%%%%%%%%%%%%%%%%%%%%%%%%%%%%%%%%%
\subsection{Related CTAN Packages}

There are several other packages which offer a similar functionality:
%
\begin{itemize}
\item
The packages
\href{http://ctan.org/pkg/docmute}{\textsf{docmute}},
\href{http://ctan.org/pkg/includex}{\textsf{includex}} and
\href{http://ctan.org/pkg/standalone}{\textsf{standalone}}
provide commands to include only the document body of
a child file thus allowing both files to be compiled individually.
\item
The packages \href{http://ctan.org/pkg/subdocs}{\textsf{subdocs}}
and \href{http://ctan.org/pkg/subfiles}{\textsf{subfiles}}
provide structures in which the main and child documents can be
encapsulated and allowing them to be compiled individually.
The inclusion mechanism is different from the conventional |\include|.
\item
The package \href{http://ctan.org/pkg/combine}{\textsf{combine}}
is an elaborate solution to combine several documents into one.
\end{itemize}
%
See also the CTAN topic \href{http://ctan.org/topic/subdocs}{\textsf{subdocs}}
for further related packages.
The present package differs from the above solutions in that
a document structure constructed with the conventional |\include| mechanism
just needs two extra commands at the top of every file
such that all constituent files can be compiled individually.

%%%%%%%%%%%%%%%%%%%%%%%%%%%%%%%%%%%%%%%%%%%%%%%%%%%%%%%%%%%%%%%%%%%%%%%%%%%%%%%%
%\subsection{Feature Suggestions}
%
%The following is a list of features which may be useful for future
%versions of this package:
%%
%\begin{itemize}
%\item
%\ldots
%\end{itemize}

%%%%%%%%%%%%%%%%%%%%%%%%%%%%%%%%%%%%%%%%%%%%%%%%%%%%%%%%%%%%%%%%%%%%%%%%%%%%%%%%
\subsection{Revision History}

%%%%%%%%%%%%%%%%%%%%%%%%%%%%%%%%%%%%%%%%
\paragraph{v2.0:} 2018/12/30

\begin{itemize}
\item
immediate forward processing
\item
added |\childdocby| mechanism
\item
manual restructured
\end{itemize}

%%%%%%%%%%%%%%%%%%%%%%%%%%%%%%%%%%%%%%%%
\paragraph{v1.6:} 2018/01/17

\begin{itemize}
\item
application for development of include files
\item
corrections to manual
\end{itemize}

%%%%%%%%%%%%%%%%%%%%%%%%%%%%%%%%%%%%%%%%
\paragraph{v1.5:} 2017/05/21

\begin{itemize}
\item
more complete structuring introduced
\item
|\childdocof| introduced
\item
|\childdoc| renamed to |\childdocmain|
\item
|\childredirect| renamed to |\childdocforward| and |\childdocforwardprefix|
and functionality expanded
\end{itemize}

%%%%%%%%%%%%%%%%%%%%%%%%%%%%%%%%%%%%%%%%
\paragraph{v1.0:} 2017/04/27

\begin{itemize}
\item
manual and install package
\item
first version published on CTAN
\end{itemize}

%%%%%%%%%%%%%%%%%%%%%%%%%%%%%%%%%%%%%%%%
\paragraph{v0.6:} 2017/04/26

\begin{itemize}
\item
redirection mechanism added
\end{itemize}

%%%%%%%%%%%%%%%%%%%%%%%%%%%%%%%%%%%%%%%%
\paragraph{v0.5:} 2017/04/26

\begin{itemize}
\item
functionality in definition file
\end{itemize}


%%%%%%%%%%%%%%%%%%%%%%%%%%%%%%%%%%%%%%%%%%%%%%%%%%%%%%%%%%%%%%%%%%%%%%%%%%%%%%%%
%%%%%%%%%%%%%%%%%%%%%%%%%%%%%%%%%%%%%%%%%%%%%%%%%%%%%%%%%%%%%%%%%%%%%%%%%%%%%%%%
%%%%%%%%%%%%%%%%%%%%%%%%%%%%%%%%%%%%%%%%%%%%%%%%%%%%%%%%%%%%%%%%%%%%%%%%%%%%%%%%
\appendix

\settowidth\MacroIndent{\rmfamily\scriptsize 000\ }

 \DocInput{childdoc.dtx}

\end{document}
%</driver>
% \fi
%
% %%%%%%%%%%%%%%%%%%%%%%%%%%%%%%%%%%%%%%%%%%%%%%%%%%%%%%%%%%%%%%%%%%%%%%%%%%%%%%
% %%%%%%%%%%%%%%%%%%%%%%%%%%%%%%%%%%%%%%%%%%%%%%%%%%%%%%%%%%%%%%%%%%%%%%%%%%%%%%
% \section{Sample}
%\iffalse
%<*samplemain>
%\fi
%
% The following presents a sample document
% with two chapters, two parts, a title page,
% a compile flag as well as three forwarding files to set the flag.
% It consists of eight |.tex| files:
% \begin{center}
% \begin{tabular}{ll}
% |cdocsamp.tex|&main file\\
% |cdocsch1.tex|&include file for chapter 1\\
% |cdocsch2.tex|&include file for chapter 2\\
% |cdocspt3.tex|&include file for part 3\\
% |cdocspt4.tex|&include file for part 4\\
% |cdocsdrf.tex|&forwarding file for main file in draft mode\\
% |cdocsfi1.tex|&forwarding file for final version of chapter 1\\
% |cdocsfi2.tex|&forwarding file for final version of chapter 2\\
% \end{tabular}
% \end{center}
% Each of the eight files can be compiled directly by the \LaTeX{} compiler.
%
% %%%%%%%%%%%%%%%%%%%%%%%%%%%%%%%%%%%%%%
% \paragraph{Main File.}
%
% The main file is called |cdocsamp.tex|.
%
% Load the \textsf{childdoc} definitions and
% declare the filename for the main document:
%    \begin{macrocode}
\input{childdoc.def}
\childdocmain{}
%    \end{macrocode}

% Optional override for |\version| flag:
%    \begin{macrocode}
%%\ifchilddoc\else\providecommand{\version}{draft}\fi
%    \end{macrocode}

% Define the default values for the |\version| flag
% (|final| for the main file and |draft| for childs):
%    \begin{macrocode}
\ifchilddoc
\providecommand{\version}{draft}
\else
\providecommand{\version}{final}
\fi
%    \end{macrocode}

% Load the standard document class:
%    \begin{macrocode}
\documentclass[12pt]{article}
%    \end{macrocode}

% Start the document body:
%    \begin{macrocode}
\begin{document}
%    \end{macrocode}

% Declare a title page.
% Print title, part of document being processed and version flag:
%    \begin{macrocode}
\addtocounter{page}{-1}
\begin{center}
{\LARGE\bfseries{}childdoc example\par}
\vspace{1cm}
\ifchilddoc
\ifchilddocmanual part\else chapter\fi:
`\childdocname' of `\childdocjob'\par
\else
main document: `\childdocjob'\par
\fi
version: \version\par
\end{center}
\newpage
%    \end{macrocode}

% Manually include selected file,
% otherwise process as usual:
%    \begin{macrocode}
\ifchilddocmanual
\section*{part `\childdocname'}
\input{\childdocname}
\else
%    \end{macrocode}

% Include the two chapters:
%    \begin{macrocode}
\include{cdocsch1}
\include{cdocsch2}
%    \end{macrocode}

% Include the two parts unless only chapters should be displayed:
%    \begin{macrocode}
\ifchilddoc\else
\section{part three}
\input{cdocspt3}
\section{part four}
\input{cdocspt4}
\fi
%    \end{macrocode}

% Process as usual until here:
%    \begin{macrocode}
\fi
%    \end{macrocode}

% End of document body:
%    \begin{macrocode}
\end{document}
%    \end{macrocode}
%\iffalse
%</samplemain>
%\fi
%
% %%%%%%%%%%%%%%%%%%%%%%%%%%%%%%%%%%%%%%
% \paragraph{Chapter Include Files.}
%
% The include files are called |cdocsch1.tex| and |cdocsch2.tex|.
%
%\iffalse
%<*samplechap1|samplechap2>
%\fi

% Optional override for |\version| flag:
%    \begin{macrocode}
%%\providecommand{\version}{final}
%    \end{macrocode}

% Include the main document:
%    \begin{macrocode}
\input{childdoc.def}
\childdocof{cdocsamp}
%    \end{macrocode}

%\iffalse
%</samplechap1|samplechap2>
%\fi
%
%\iffalse
%<*samplechap1>
%\fi
% Some text for chapter 1:
%    \begin{macrocode}
\section{one}
some text in chapter one
%    \end{macrocode}

%\iffalse
%</samplechap1>
%\fi
% Some text for chapter 2:
%\iffalse
%<*samplechap2>
%\fi
%    \begin{macrocode}
\section{two}
more text in chapter two
%    \end{macrocode}

%\iffalse
%</samplechap2>
%\fi
%
% %%%%%%%%%%%%%%%%%%%%%%%%%%%%%%%%%%%%%%
% \paragraph{Part Include Files.}
%
% The include files are called |cdocspt3.tex| and |cdocspt4.tex|.
%
%\iffalse
%<*samplepart3|samplepart4>
%\fi

% Optional override for |\version| flag:
%    \begin{macrocode}
%%\providecommand{\version}{final}
%    \end{macrocode}

% Include the main document:
%    \begin{macrocode}
\input{childdoc.def}
\childdocby{cdocsamp}
%    \end{macrocode}

%\iffalse
%</samplepart3|samplepart4>
%\fi
%
%\iffalse
%<*samplepart3>
%\fi
% Some text for part 3:
%    \begin{macrocode}
some text in part three
%    \end{macrocode}

%\iffalse
%</samplepart3>
%\fi
% Some text for part 4:
%\iffalse
%<*samplepart4>
%\fi
%    \begin{macrocode}
more text in part four
%    \end{macrocode}

%\iffalse
%</samplepart4>
%\fi
%
% %%%%%%%%%%%%%%%%%%%%%%%%%%%%%%%%%%%%%%
% \paragraph{Forwarding for a Complete Draft.}
%
% The following forwarding file |cdocsdrf.tex|
% compiles the main document in draft mode:
%\iffalse
%<*sampledraft>
%\fi
%    \begin{macrocode}
\def\version{draft}
\input{childdoc.def}
\childdocforward{cdocsamp}
%    \end{macrocode}

%\iffalse
%</sampledraft>
%\fi
%
% %%%%%%%%%%%%%%%%%%%%%%%%%%%%%%%%%%%%%%
% \paragraph{Forwarding for Final Version of the Chapters.}
%
% The following forwarding files |cdocsfn1.tex| and |cdocsfn2.tex|
% (with identical content)
% compile the final versions of the child documents
% |cdocsch1.tex| and |cdocsch2.tex|, respectively:
%\iffalse
%<*samplefinal>
%\fi
%    \begin{macrocode}
\def\version{final}
\input{childdoc.def}
\childdocforwardprefix[cdocsamp]{cdocsfn}{cdocsch}
%    \end{macrocode}

%\iffalse
%</samplefinal>
%\fi
%
% %%%%%%%%%%%%%%%%%%%%%%%%%%%%%%%%%%%%%%
% \paragraph{Command Line Processing.}
%
% The following three command lines generate the output files
% |cdocscld|, |cdocscl1| and |cdocscl2|
% which should be identical to
% |cdocsdrf|, |cdocsch1| and |cdocsfn2|, respectively:
% \begin{center}
% \begin{tabular}{l}
% |latex -jobname cdocscld \|\\
% |  "\def\version{draft}\input{childdoc.def}\childdocforward{cdocsamp}"|\\
% |latex -jobname cdocscl1 \|\\
% |  "\input{childdoc.def}\childdocforward[cdocsamp]{cdocsch1}"|\\
% |latex -jobname cdocscl2 \|\\
% |  "\def\version{final}\input{childdoc.def}\childdocforward{cdocsch2}"|
% \end{tabular}
% \end{center}
% Note that the trailing backslash on each first line
% merely continues the input to the second line
% (for convenient cut ant paste).
% Furthermore, the command |latex| can be replaced by any
% of its alternative versions such as |pdflatex|.
%
% %%%%%%%%%%%%%%%%%%%%%%%%%%%%%%%%%%%%%%%%%%%%%%%%%%%%%%%%%%%%%%%%%%%%%%%%%%%%%%
% %%%%%%%%%%%%%%%%%%%%%%%%%%%%%%%%%%%%%%%%%%%%%%%%%%%%%%%%%%%%%%%%%%%%%%%%%%%%%%
% \section{Implementation}
%\iffalse
%<*package>
%\fi
%
% This section describes the definitions file |childdoc.def|.

% The definitions cannot be loaded using |\usepackage| or |\RequirePackage|
% which has a mechanism to prevent loading a style file more than once.
% When loading the definitions by means of |\input|
% multiple instances have to be prevented manually:
%\iffalse
%This code needs to be before the `\ProvidesFile' directive
%which is defined at the beginning of this file.
%Therefore it is also placed there and commented out here.
%</package>
%<*discard>
%\fi
%    \begin{macrocode}
\ifdefined\childdocmain\endinput\fi
%    \end{macrocode}
%\iffalse
%</discard>
%<*package>
%\fi
%
% \macro{\ifchilddoc}
% \macro{\ifchilddocmanual}
% The conditional |\ifchilddoc| tells whether a
% child (true) or main (false) document is being compiled.
% The conditional |\ifchilddocmanual| tells whether
% the |\includeonly| mechanism is used (false) or
% the selection of child files must be performed manually (true).
% The definitions initialise to false:
%    \begin{macrocode}
\newif\ifchilddoc
\newif\ifchilddocmanual
%    \end{macrocode}

% \macro{\childdocname}
% \macro{\childdocjob}
% The macro |\childdocname| stores the name of the main document
% to be compiled. The macro |\childdocjob| stores the name of
% the document on which the \LaTeX{} compiler was originally invoked.
% The content of |\jobname| cannot be compared
% to filenames specified in the source due to different catcodes.
% The following code rescans |\jobname|, stores the result
% in |\childdocname| and saves a copy in |\childdocjob|:
%    \begin{macrocode}
\edef\childdocname{\scantokens\expandafter{\jobname\noexpand}}
\let\childdocjob\childdocname
%    \end{macrocode}

% \macro{\childdocdisable}
% The macro |\childdocdisable| prevents the main file
% from being processed more than once.
% At this stage, the main document command |\childdocmain|
% is assumed to be called once again where it should do nothing.
% Any subsequent call to it should prevent
% a secondary processing of the main document
% It overwrites the forwarding commands
% |\childdocof| and |\childdocforward|
% with empty macros to prevent further inclusions of the main document:
%    \begin{macrocode}
\newcommand{\childdocdisable}
{
  \renewcommand{\childdocmain}[1]{\renewcommand{\childdocmain}[1]{\endinput}}
  \renewcommand{\childdocof}[1]{}
  \renewcommand{\childdocby}[2][]{}
  \renewcommand{\childdocforward}[2][]{}
  \renewcommand{\childdocdisable}{}
}
%    \end{macrocode}

% \macro{\childdocmain}
% The macro |\childdocmain| is to be called at the top of the main file
% with nothing or the main filename (without extension) as argument.
% First, it breaks loops.
% If the argument is not empty and does not match |\childdocname|
% (which is set by the first inclusion of |childdoc.def|),
% |\ifchilddoc| is set to true, |\includeonly| is applied to the child file
% and |\jobname| is set to the main file
% (for proper handling of |.aux| files):
%    \begin{macrocode}
\newcommand{\childdocmain}[1]
{
  \childdocdisable\childdocmain{}
  \if?#1?\else
    \begingroup
      \def\childdoctmp{#1}
      \ifx\childdoctmp\childdocname
        \def\childdoctmp{}
      \else
        \def\childdoctmp
        {
          \childdoctrue
          \includeonly{\childdocname}
          \def\childdocjob{#1}
          \def\jobname{#1}
        }
      \fi
      \expandafter
    \endgroup
    \childdoctmp
  \fi
}
%    \end{macrocode}

% \macro{\childdocof}
% The command |\childdocof| redirects
% compilation to the main file |#1|.
%    \begin{macrocode}
\newcommand{\childdocof}[1]
{
  \childdocdisable
  \childdoctrue
  \includeonly{\childdocname}
  \def\jobname{#1}
  \def\childdocjob{#1}
  \input{#1}
}
%    \end{macrocode}

% \macro{\childdocby}
% The command |\childdocby| ....
%    \begin{macrocode}
\newcommand{\childdocby}[2][]
{
  \childdocdisable
  \childdoctrue
  \childdocmanualtrue
  \if?#1?\else
    \def\jobname{#2}
  \fi
  \def\childdocjob{#2}
  \input{#2}
  \endinput
}
%    \end{macrocode}

% \macro{\childdocforward}
% The command |\childdocforward| redirects
% compilation to the main file or
% (if the optional argument is given) a child file.
% Parameters are set as if the main file
% or a child file starting with |\childdocof| was compiled.
% Then compilation is handed over to the main file:
%    \begin{macrocode}
\newcommand{\childdocforward}[2][]
{
  \begingroup
    \if?#1?
      \def\childdoctmp
      {
        \def\childdocname{#2}
        \def\childdocjob{#2}
        \def\jobname{#2}
        \input{#2}
        \endinput
      }
    \else
      \def\childdoctmp
      {
        \childdocdisable
        \def\childdocname{#2}
        \childdoctrue
        \includeonly{#2}
        \def\childdocjob{#1}
        \def\jobname{#1}
        \input{#1}
        \endinput
      }
    \fi
    \expandafter
  \endgroup
  \childdoctmp
}
%    \end{macrocode}

% \macro{\childdocforwardprefix}
% The command |\childdocforwardprefix| redirects
% compilation to the main or a child file by means of a pattern.
% The prefix |#1| in the current filename is replaced by |#2|
% and the suffix of the current filename is kept
% (it is assumed that the filename does not contain the substring `|~~~|'
% which is used as a delimiter).
% Compilation is handed over to the new file by |\childdocforward|:
%    \begin{macrocode}
\newcommand{\childdocforwardprefix}[3][]
{
  \begingroup
    \def\childdocextract #2##1~~~{\def\childdoctmp{\childdocforward[#1]{#3##1}}}
    \expandafter\childdocextract\childdocname~~~
    \expandafter
  \endgroup
  \childdoctmp
}
%    \end{macrocode}

% \macro{\childdoc}
% The deprecated macro |\childdoc| is a legacy version of |\childdocmain|:
%    \begin{macrocode}
\newcommand{\childdoc}{\childdocmain}
%    \end{macrocode}

% \macro{\childdocredirect}
% The deprecated macro |\childdocredirect| is a legacy version
% of |\childdocforward| and |\childdocforwardprefix|:
%    \begin{macrocode}
\newcommand{\childdocredirect}[2][]
{
  \begingroup
    \if?#1?
      \def\childdoctmp{\childdocforward{#2}}
    \else
      \def\childdoctmp{\childdocforwardprefix{#1}{#2}}
    \fi
    \expandafter
  \endgroup
  \childdoctmp
}
%    \end{macrocode}

%\iffalse
%</package>
%\fi
%
\endinput
\childdocforward{cdocsch2}"|
% \end{tabular}
% \end{center}
% Note that the trailing backslash on each first line
% merely continues the input to the second line
% (for convenient cut ant paste).
% Furthermore, the command |latex| can be replaced by any
% of its alternative versions such as |pdflatex|.
%
% %%%%%%%%%%%%%%%%%%%%%%%%%%%%%%%%%%%%%%%%%%%%%%%%%%%%%%%%%%%%%%%%%%%%%%%%%%%%%%
% %%%%%%%%%%%%%%%%%%%%%%%%%%%%%%%%%%%%%%%%%%%%%%%%%%%%%%%%%%%%%%%%%%%%%%%%%%%%%%
% \section{Implementation}
%\iffalse
%<*package>
%\fi
%
% This section describes the definitions file |childdoc.def|.

% The definitions cannot be loaded using |\usepackage| or |\RequirePackage|
% which has a mechanism to prevent loading a style file more than once.
% When loading the definitions by means of |\input|
% multiple instances have to be prevented manually:
%\iffalse
%This code needs to be before the `\ProvidesFile' directive
%which is defined at the beginning of this file.
%Therefore it is also placed there and commented out here.
%</package>
%<*discard>
%\fi
%    \begin{macrocode}
\ifdefined\childdocmain\endinput\fi
%    \end{macrocode}
%\iffalse
%</discard>
%<*package>
%\fi
%
% \macro{\ifchilddoc}
% \macro{\ifchilddocmanual}
% The conditional |\ifchilddoc| tells whether a
% child (true) or main (false) document is being compiled.
% The conditional |\ifchilddocmanual| tells whether
% the |\includeonly| mechanism is used (false) or
% the selection of child files must be performed manually (true).
% The definitions initialise to false:
%    \begin{macrocode}
\newif\ifchilddoc
\newif\ifchilddocmanual
%    \end{macrocode}

% \macro{\childdocname}
% \macro{\childdocjob}
% The macro |\childdocname| stores the name of the main document
% to be compiled. The macro |\childdocjob| stores the name of
% the document on which the \LaTeX{} compiler was originally invoked.
% The content of |\jobname| cannot be compared
% to filenames specified in the source due to different catcodes.
% The following code rescans |\jobname|, stores the result
% in |\childdocname| and saves a copy in |\childdocjob|:
%    \begin{macrocode}
\edef\childdocname{\scantokens\expandafter{\jobname\noexpand}}
\let\childdocjob\childdocname
%    \end{macrocode}

% \macro{\childdocdisable}
% The macro |\childdocdisable| prevents the main file
% from being processed more than once.
% At this stage, the main document command |\childdocmain|
% is assumed to be called once again where it should do nothing.
% Any subsequent call to it should prevent
% a secondary processing of the main document
% It overwrites the forwarding commands
% |\childdocof| and |\childdocforward|
% with empty macros to prevent further inclusions of the main document:
%    \begin{macrocode}
\newcommand{\childdocdisable}
{
  \renewcommand{\childdocmain}[1]{\renewcommand{\childdocmain}[1]{\endinput}}
  \renewcommand{\childdocof}[1]{}
  \renewcommand{\childdocby}[2][]{}
  \renewcommand{\childdocforward}[2][]{}
  \renewcommand{\childdocdisable}{}
}
%    \end{macrocode}

% \macro{\childdocmain}
% The macro |\childdocmain| is to be called at the top of the main file
% with nothing or the main filename (without extension) as argument.
% First, it breaks loops.
% If the argument is not empty and does not match |\childdocname|
% (which is set by the first inclusion of |childdoc.def|),
% |\ifchilddoc| is set to true, |\includeonly| is applied to the child file
% and |\jobname| is set to the main file
% (for proper handling of |.aux| files):
%    \begin{macrocode}
\newcommand{\childdocmain}[1]
{
  \childdocdisable\childdocmain{}
  \if?#1?\else
    \begingroup
      \def\childdoctmp{#1}
      \ifx\childdoctmp\childdocname
        \def\childdoctmp{}
      \else
        \def\childdoctmp
        {
          \childdoctrue
          \includeonly{\childdocname}
          \def\childdocjob{#1}
          \def\jobname{#1}
        }
      \fi
      \expandafter
    \endgroup
    \childdoctmp
  \fi
}
%    \end{macrocode}

% \macro{\childdocof}
% The command |\childdocof| redirects
% compilation to the main file |#1|.
%    \begin{macrocode}
\newcommand{\childdocof}[1]
{
  \childdocdisable
  \childdoctrue
  \includeonly{\childdocname}
  \def\jobname{#1}
  \def\childdocjob{#1}
  \input{#1}
}
%    \end{macrocode}

% \macro{\childdocby}
% The command |\childdocby| ....
%    \begin{macrocode}
\newcommand{\childdocby}[2][]
{
  \childdocdisable
  \childdoctrue
  \childdocmanualtrue
  \if?#1?\else
    \def\jobname{#2}
  \fi
  \def\childdocjob{#2}
  \input{#2}
  \endinput
}
%    \end{macrocode}

% \macro{\childdocforward}
% The command |\childdocforward| redirects
% compilation to the main file or
% (if the optional argument is given) a child file.
% Parameters are set as if the main file
% or a child file starting with |\childdocof| was compiled.
% Then compilation is handed over to the main file:
%    \begin{macrocode}
\newcommand{\childdocforward}[2][]
{
  \begingroup
    \if?#1?
      \def\childdoctmp
      {
        \def\childdocname{#2}
        \def\childdocjob{#2}
        \def\jobname{#2}
        \input{#2}
        \endinput
      }
    \else
      \def\childdoctmp
      {
        \childdocdisable
        \def\childdocname{#2}
        \childdoctrue
        \includeonly{#2}
        \def\childdocjob{#1}
        \def\jobname{#1}
        \input{#1}
        \endinput
      }
    \fi
    \expandafter
  \endgroup
  \childdoctmp
}
%    \end{macrocode}

% \macro{\childdocforwardprefix}
% The command |\childdocforwardprefix| redirects
% compilation to the main or a child file by means of a pattern.
% The prefix |#1| in the current filename is replaced by |#2|
% and the suffix of the current filename is kept
% (it is assumed that the filename does not contain the substring `|~~~|'
% which is used as a delimiter).
% Compilation is handed over to the new file by |\childdocforward|:
%    \begin{macrocode}
\newcommand{\childdocforwardprefix}[3][]
{
  \begingroup
    \def\childdocextract #2##1~~~{\def\childdoctmp{\childdocforward[#1]{#3##1}}}
    \expandafter\childdocextract\childdocname~~~
    \expandafter
  \endgroup
  \childdoctmp
}
%    \end{macrocode}

% \macro{\childdoc}
% The deprecated macro |\childdoc| is a legacy version of |\childdocmain|:
%    \begin{macrocode}
\newcommand{\childdoc}{\childdocmain}
%    \end{macrocode}

% \macro{\childdocredirect}
% The deprecated macro |\childdocredirect| is a legacy version
% of |\childdocforward| and |\childdocforwardprefix|:
%    \begin{macrocode}
\newcommand{\childdocredirect}[2][]
{
  \begingroup
    \if?#1?
      \def\childdoctmp{\childdocforward{#2}}
    \else
      \def\childdoctmp{\childdocforwardprefix{#1}{#2}}
    \fi
    \expandafter
  \endgroup
  \childdoctmp
}
%    \end{macrocode}

%\iffalse
%</package>
%\fi
%
\endinput
|\\
|\childdocforward[|\textit{main}|]{|\textit{dest}|}|\\
\end{tabular}
\end{center}
%
The argument \textit{dest} is the destination file
(without extension).
It should be the main file or one of the child files.
Note that further \textsf{childdoc} directives
such as |\childdocof| and |\childdocforward|
in the indicated file will be processed in this form.
The optional argument \textit{main}
passes on directly to the main file \textit{main}
while pretending to compile the child \textit{dest}.
This form behaves as if \textit{dest}
issues |\childdocof{|\textit{main}|}| right away,
and no further \textsf{childdoc} directives will be processed.

%%%%%%%%%%%%%%%%%%%%%%%%%%%%%%%%%%%%%%%%
\DescribeMacro{\...prefix}
In the alternative form |\childdocforwardprefix|,
%
\begin{center}
\begin{tabular}{l}
|% \iffalse
%
% childdoc.dtx Copyright (C) 2017-2018 Niklas Beisert
%
% This work may be distributed and/or modified under the
% conditions of the LaTeX Project Public License, either version 1.3
% of this license or (at your option) any later version.
% The latest version of this license is in
%   http://www.latex-project.org/lppl.txt
% and version 1.3 or later is part of all distributions of LaTeX
% version 2005/12/01 or later.
%
% This work has the LPPL maintenance status `maintained'.
%
% The Current Maintainer of this work is Niklas Beisert.
%
% This work consists of the files childdoc.dtx and childdoc.ins
% and the derived files childdoc.def and cdocsamp.tex with
% cdocsch1.tex, cdocsch2.tex, cdocsdrf.tex, cdocsfn1.tex, cdocsfn2.tex.
%
%<package>\ifdefined\childdocmain\endinput\fi
%<package>\ProvidesFile{childdoc.def}[2018/12/30 v2.0 child document driver]
%<samplemain>\ProvidesFile{cdocsamp.tex}[2018/12/30 v2.0 sample for childdoc]
%<*driver>
%\ProvidesFile{childdoc.drv}[2018/12/30 v2.0 childdoc reference manual file]
\PassOptionsToClass{10pt,a4paper}{article}
\documentclass{ltxdoc}

\usepackage[margin=35mm]{geometry}
\usepackage{hyperref}
\usepackage{hyperxmp}
\usepackage[usenames]{color}

\hypersetup{colorlinks=true}
\hypersetup{pdfstartview=FitH}
\hypersetup{pdfpagemode=UseNone}
\hypersetup{pdfsource={}}
\hypersetup{pdflang={en-UK}}
\hypersetup{pdfcopyright={Copyright 2017-2018 Niklas Beisert.
  This work may be distributed and/or modified under the
  conditions of the LaTeX Project Public License, either version 1.3
  of this license or (at your option) any later version.}}
\hypersetup{pdflicenseurl={http://www.latex-project.org/lppl.txt}}
\hypersetup{pdfcontactaddress={ETH Zurich, ITP, HIT K,
  Wolfgang-Pauli-Strasse 27}}
\hypersetup{pdfcontactpostcode={8093}}
\hypersetup{pdfcontactcity={Zurich}}
\hypersetup{pdfcontactcountry={Switzerland}}
\hypersetup{pdfcontactemail={nbeisert@itp.phys.ethz.ch}}
\hypersetup{pdfcontacturl={http://people.phys.ethz.ch/\xmptilde nbeisert/}}

\newcommand{\secref}[1]{\hyperref[#1]{section \ref*{#1}}}

\parskip1ex
\parindent0pt
\let\olditemize\itemize
\def\itemize{\olditemize\parskip0pt}

\begin{document}

\title{The \textsf{childdoc} Package}
\hypersetup{pdftitle={The childdoc Package}}
\author{Niklas Beisert\\[2ex]
  Institut f\"ur Theoretische Physik\\
  Eidgen\"ossische Technische Hochschule Z\"urich\\
  Wolfgang-Pauli-Strasse 27, 8093 Z\"urich, Switzerland\\[1ex]
  \href{mailto:nbeisert@itp.phys.ethz.ch}
  {\texttt{nbeisert@itp.phys.ethz.ch}}}
\hypersetup{pdfauthor={Niklas Beisert}}
\hypersetup{pdfsubject={Manual for the LaTeX2e Package childdoc}}
\date{30 December 2018, \textsf{v2.0}}
\maketitle

\begin{abstract}\noindent
\textsf{childdoc} is a \LaTeXe{} package
that enables the direct compilation
of document sections included by |\include|
to individual files.
\end{abstract}

\begingroup
\parskip0ex
\tableofcontents
\endgroup

%%%%%%%%%%%%%%%%%%%%%%%%%%%%%%%%%%%%%%%%%%%%%%%%%%%%%%%%%%%%%%%%%%%%%%%%%%%%%%%%
%%%%%%%%%%%%%%%%%%%%%%%%%%%%%%%%%%%%%%%%%%%%%%%%%%%%%%%%%%%%%%%%%%%%%%%%%%%%%%%%
\section{Introduction}

\LaTeX{} provides a mechanism to structure a large document (such as a book)
into a main file and several child files (containing the chapters)
using the |\include| command.
This mechanism is beneficial for documents
which span hundreds of pages in order to
make the source file(s) more manageable.
Moreover, compilation can be restricted to
selected child files by means of the |\includeonly| command.
The latter feature can be used to reduce the compilation time while editing
(this was significantly more useful in the earlier days of \LaTeX{})
or to generate a smaller document which is easier to navigate.
Another application of |\includeonly| is to generate
documents consisting of selected parts of the complete document.

However, there are a few drawbacks of the plain |\include| mechanism:
\begin{itemize}
\item
The child files cannot be compiled on their own,
they can only be compiled via the main file.
A naive editing environment
(such as a text editor with an option
to have the current file processed by \LaTeX)
may require one to switch to the main file before compiling;
attempting to compile the child file produces errors.
\item
The main file must be modified (each time)
to adjust the |\includeonly| command
to the present needs. This easily leaves the main file in a messy state.
\item
The generated document will always carry the filename
of the main document. This is inconvenient if
several child files are to be compiled and
to be kept for distribution.
\end{itemize}

The present package provides a simple interface
to make child files individually compilable by \LaTeX{}.
Compiling a child file then has the same effect as compiling
the main file with an |\includeonly| command
to select the appropriate child.
Moreover the generated document will carry the name of the child
rather than the main file.
This resolves all three above issues.

This feature is meant to make the editing of books,
thesis documents and lecture notes somewhat more convenient.
However, the package can also be used efficiently for
composing a series of documents (such as exercise sheets)
which are typically distributed individually.
It then assists the author in generating the individual documents
(potentially in different versions)
as well as a document containing the collected series.
Another application is in developing style files
or other kinds of included material
where compilation of the style file could redirect
to a sample or test file.

%%%%%%%%%%%%%%%%%%%%%%%%%%%%%%%%%%%%%%%%%%%%%%%%%%%%%%%%%%%%%%%%%%%%%%%%%%%%%%%%
%%%%%%%%%%%%%%%%%%%%%%%%%%%%%%%%%%%%%%%%%%%%%%%%%%%%%%%%%%%%%%%%%%%%%%%%%%%%%%%%
\section{Usage}

First of all, the package \textsf{childdoc} is \emph{not} a standard
\LaTeXe{} |.sty| style file! Therefore it needs to be invoked in
a non-standard way.

%%%%%%%%%%%%%%%%%%%%%%%%%%%%%%%%%%%%%%%%%%%%%%%%%%%%%%%%%%%%%%%%%%%%%%%%%%%%%%%%
\subsection{Included Files}
\label{sec:include}

%%%%%%%%%%%%%%%%%%%%%%%%%%%%%%%%%%%%%%%%
\DescribeMacro{\childdocmain}
To use the package, add the commands
\begin{center}
\begin{tabular}{l}
|% \iffalse
%
% childdoc.dtx Copyright (C) 2017-2018 Niklas Beisert
%
% This work may be distributed and/or modified under the
% conditions of the LaTeX Project Public License, either version 1.3
% of this license or (at your option) any later version.
% The latest version of this license is in
%   http://www.latex-project.org/lppl.txt
% and version 1.3 or later is part of all distributions of LaTeX
% version 2005/12/01 or later.
%
% This work has the LPPL maintenance status `maintained'.
%
% The Current Maintainer of this work is Niklas Beisert.
%
% This work consists of the files childdoc.dtx and childdoc.ins
% and the derived files childdoc.def and cdocsamp.tex with
% cdocsch1.tex, cdocsch2.tex, cdocsdrf.tex, cdocsfn1.tex, cdocsfn2.tex.
%
%<package>\ifdefined\childdocmain\endinput\fi
%<package>\ProvidesFile{childdoc.def}[2018/12/30 v2.0 child document driver]
%<samplemain>\ProvidesFile{cdocsamp.tex}[2018/12/30 v2.0 sample for childdoc]
%<*driver>
%\ProvidesFile{childdoc.drv}[2018/12/30 v2.0 childdoc reference manual file]
\PassOptionsToClass{10pt,a4paper}{article}
\documentclass{ltxdoc}

\usepackage[margin=35mm]{geometry}
\usepackage{hyperref}
\usepackage{hyperxmp}
\usepackage[usenames]{color}

\hypersetup{colorlinks=true}
\hypersetup{pdfstartview=FitH}
\hypersetup{pdfpagemode=UseNone}
\hypersetup{pdfsource={}}
\hypersetup{pdflang={en-UK}}
\hypersetup{pdfcopyright={Copyright 2017-2018 Niklas Beisert.
  This work may be distributed and/or modified under the
  conditions of the LaTeX Project Public License, either version 1.3
  of this license or (at your option) any later version.}}
\hypersetup{pdflicenseurl={http://www.latex-project.org/lppl.txt}}
\hypersetup{pdfcontactaddress={ETH Zurich, ITP, HIT K,
  Wolfgang-Pauli-Strasse 27}}
\hypersetup{pdfcontactpostcode={8093}}
\hypersetup{pdfcontactcity={Zurich}}
\hypersetup{pdfcontactcountry={Switzerland}}
\hypersetup{pdfcontactemail={nbeisert@itp.phys.ethz.ch}}
\hypersetup{pdfcontacturl={http://people.phys.ethz.ch/\xmptilde nbeisert/}}

\newcommand{\secref}[1]{\hyperref[#1]{section \ref*{#1}}}

\parskip1ex
\parindent0pt
\let\olditemize\itemize
\def\itemize{\olditemize\parskip0pt}

\begin{document}

\title{The \textsf{childdoc} Package}
\hypersetup{pdftitle={The childdoc Package}}
\author{Niklas Beisert\\[2ex]
  Institut f\"ur Theoretische Physik\\
  Eidgen\"ossische Technische Hochschule Z\"urich\\
  Wolfgang-Pauli-Strasse 27, 8093 Z\"urich, Switzerland\\[1ex]
  \href{mailto:nbeisert@itp.phys.ethz.ch}
  {\texttt{nbeisert@itp.phys.ethz.ch}}}
\hypersetup{pdfauthor={Niklas Beisert}}
\hypersetup{pdfsubject={Manual for the LaTeX2e Package childdoc}}
\date{30 December 2018, \textsf{v2.0}}
\maketitle

\begin{abstract}\noindent
\textsf{childdoc} is a \LaTeXe{} package
that enables the direct compilation
of document sections included by |\include|
to individual files.
\end{abstract}

\begingroup
\parskip0ex
\tableofcontents
\endgroup

%%%%%%%%%%%%%%%%%%%%%%%%%%%%%%%%%%%%%%%%%%%%%%%%%%%%%%%%%%%%%%%%%%%%%%%%%%%%%%%%
%%%%%%%%%%%%%%%%%%%%%%%%%%%%%%%%%%%%%%%%%%%%%%%%%%%%%%%%%%%%%%%%%%%%%%%%%%%%%%%%
\section{Introduction}

\LaTeX{} provides a mechanism to structure a large document (such as a book)
into a main file and several child files (containing the chapters)
using the |\include| command.
This mechanism is beneficial for documents
which span hundreds of pages in order to
make the source file(s) more manageable.
Moreover, compilation can be restricted to
selected child files by means of the |\includeonly| command.
The latter feature can be used to reduce the compilation time while editing
(this was significantly more useful in the earlier days of \LaTeX{})
or to generate a smaller document which is easier to navigate.
Another application of |\includeonly| is to generate
documents consisting of selected parts of the complete document.

However, there are a few drawbacks of the plain |\include| mechanism:
\begin{itemize}
\item
The child files cannot be compiled on their own,
they can only be compiled via the main file.
A naive editing environment
(such as a text editor with an option
to have the current file processed by \LaTeX)
may require one to switch to the main file before compiling;
attempting to compile the child file produces errors.
\item
The main file must be modified (each time)
to adjust the |\includeonly| command
to the present needs. This easily leaves the main file in a messy state.
\item
The generated document will always carry the filename
of the main document. This is inconvenient if
several child files are to be compiled and
to be kept for distribution.
\end{itemize}

The present package provides a simple interface
to make child files individually compilable by \LaTeX{}.
Compiling a child file then has the same effect as compiling
the main file with an |\includeonly| command
to select the appropriate child.
Moreover the generated document will carry the name of the child
rather than the main file.
This resolves all three above issues.

This feature is meant to make the editing of books,
thesis documents and lecture notes somewhat more convenient.
However, the package can also be used efficiently for
composing a series of documents (such as exercise sheets)
which are typically distributed individually.
It then assists the author in generating the individual documents
(potentially in different versions)
as well as a document containing the collected series.
Another application is in developing style files
or other kinds of included material
where compilation of the style file could redirect
to a sample or test file.

%%%%%%%%%%%%%%%%%%%%%%%%%%%%%%%%%%%%%%%%%%%%%%%%%%%%%%%%%%%%%%%%%%%%%%%%%%%%%%%%
%%%%%%%%%%%%%%%%%%%%%%%%%%%%%%%%%%%%%%%%%%%%%%%%%%%%%%%%%%%%%%%%%%%%%%%%%%%%%%%%
\section{Usage}

First of all, the package \textsf{childdoc} is \emph{not} a standard
\LaTeXe{} |.sty| style file! Therefore it needs to be invoked in
a non-standard way.

%%%%%%%%%%%%%%%%%%%%%%%%%%%%%%%%%%%%%%%%%%%%%%%%%%%%%%%%%%%%%%%%%%%%%%%%%%%%%%%%
\subsection{Included Files}
\label{sec:include}

%%%%%%%%%%%%%%%%%%%%%%%%%%%%%%%%%%%%%%%%
\DescribeMacro{\childdocmain}
To use the package, add the commands
\begin{center}
\begin{tabular}{l}
|\input{childdoc.def}|\\
|\childdocmain{}|\\
\end{tabular}
\end{center}
at the very top of the main \LaTeX{} file,
in particular \emph{before} the |\documentclass| statement!
The argument of |\childdocmain| should be left empty
(but it must be present).

%%%%%%%%%%%%%%%%%%%%%%%%%%%%%%%%%%%%%%%%
\DescribeMacro{\childdocof}
Furthermore, add the commands
\begin{center}
\begin{tabular}{l}
|\input{childdoc.def}|\\
|\childdocof{|\textit{main}|}|\\
\end{tabular}
\end{center}
at the top of every child file \textit{child}
which is included by |\include{|\textit{child}|}|
from within the main file
(or at least for those files to be compiled individually).
The argument \textit{main} must be the filename of the main file.

There are a couple of
considerations in setting up the main and child documents:

%%%%%%%%%%%%%%%%%%%%%%%%%%%%%%%%%%%%%%%%
\paragraph{Restrictions.}

Please note the following restrictions:
\begin{itemize}
\item
|\childdocmain| must be called with one argument \textit{main}
to ensure compatibility with earlier version of the package.
It must either be empty (|\childdocmain{}|)
or precisely match the filename of the main file in which it is specified.
See \secref{sec:detection} for further information.
\item
The filename \textit{main} must be specified without the |.tex| extension.
\item
The filename \textit{main} is case sensitive
(even in case-insensitive file systems)
due to internal string comparison.
\item
The argument \textit{main} should be fully expanded, it cannot be a macro.
\item
Subdirectories and special characters should be avoided in filenames.
\item
The command |\childdocmain{|\textit{main}|}| must be followed by a whitespace.
It should not be followed immediately by another command
or by a comment mark `|%|'.
This is because the \TeX{} parser reads the token immediately following
the argument of |\childdocmain| and puts it
at the beginning of every child section;
however, a white\-space is ignored.
\end{itemize}

%%%%%%%%%%%%%%%%%%%%%%%%%%%%%%%%%%%%%%%%
\paragraph{Content of Main File.}

It is advisable to place all content in the child files included by |\include|.
Any output contained in the main file will appear in all child documents
unless suppressed manually;
it cannot be suppressed automatically by the |\includeonly| directive
and thus should normally be avoided.
A method to include some content in the main file
by means of conditional processing is described in \secref{sec:conditional}.

%%%%%%%%%%%%%%%%%%%%%%%%%%%%%%%%%%%%%%%%
\paragraph{Page Numbering.}

When only a part of the document is compiled,
the appropriate numbering of pages
(as well as other status parameters)
is determined from the |.aux| files.
The latter contain information from previous passes.
However this information needs to propagate through
all intermediate child documents.
Therefore the page numbering in child documents may well
be inconsistent until the complete document is compiled at least once.

A useful (if unconventional) way to always ensure a consistent
page numbering is to restart the numbering in each child document
and denote the pages by `\textit{child}|.|\textit{page}'
where \textit{child} represents the chapter/section number of the child file.
This can be achieved by the command
|\numberwithin{page}{|\textit{child}|}|
of the \textsf{amsmath} package
where \textit{child} can be |chapter| or |section|
depending on the chosen structuring.
Alternatively, one can modify the macro |\thepage| appropriately
and reset the counter |page| at the start of each child file.

%%%%%%%%%%%%%%%%%%%%%%%%%%%%%%%%%%%%%%%%%%%%%%%%%%%%%%%%%%%%%%%%%%%%%%%%%%%%%%%%
\subsection{Conditional Processing}
\label{sec:conditional}

The package provides a mechanism to compile different versions
of a document. To customise the versions further some conditional processing
can come in handy to distinguish which version is being compiled.
The package provides two macros to describe the compilation context:

%%%%%%%%%%%%%%%%%%%%%%%%%%%%%%%%%%%%%%%%
\DescribeMacro{\ifchilddoc}
The conditional |\ifchilddoc| distinguishes between the compilation of
child documents and the main document:
%
\begin{center}
|\ifchilddoc |\textit{child-code}| |[|\||else |\textit{main-code}]| \||fi|
\end{center}

%%%%%%%%%%%%%%%%%%%%%%%%%%%%%%%%%%%%%%%%
\DescribeMacro{\childdocname}
\DescribeMacro{\childdocjob}
The macro |\childdocname| contains the filename (without extension)
of the main or child file being processed.
Note that |\childdocjob| will always contain the name of the main file.

%%%%%%%%%%%%%%%%%%%%%%%%%%%%%%%%%%%%%%%%
\paragraph{Title Page.}

Conditional processing can be used to include a title or banner page
in the main document when proper precautions are taken.
Importantly, the code in the main file should ensure that the page counter
(as well as other status parameters which are stored in the |.aux| files)
takes the same value after the conditional processing.
Otherwise the page numbers may take divergent values
depending on which part is compiled.

For example, a title page could be declared by:
%
\begin{center}
\begin{tabular}{l}
|\ifchilddoc\||else|\\
|\addtocounter{page}{-1}|\\
\textit{code for title page}\\
|\newpage|\\
|\||fi|
\end{tabular}
\end{center}
%
A banner page for the child documents can be generated by:
%
\begin{center}
\begin{tabular}{l}
|\ifchilddoc|\\
|\addtocounter{page}{-1}|\\
\textit{code for banner page}\\
|\newpage|\\
|\||fi|
\end{tabular}
\end{center}
%
Here one could write a message such as:
\begin{center}
|This is the part \childdocname{} of \childdocjob{}.|
\end{center}

%%%%%%%%%%%%%%%%%%%%%%%%%%%%%%%%%%%%%%%%%%%%%%%%%%%%%%%%%%%%%%%%%%%%%%%%%%%%%%%%
\subsection{Flags}
\label{sec:flags}

The package makes it easy to generate different versions
of the main or child documents.
To this end compilation flags can be defined
and assigned different default values.
They will be particularly useful in conjunction
with the forwarding mechanism described in \secref{sec:forward}.

For example, it may be useful to have a flag |\version|
which can be set to |draft| or |final|.
The document source will contain some conditional code
depending on the value of |\version|.
Suppose further, the flag should default to |final| for the main file
and to |draft| for child files
which is a natural assignment for editing the document.
This is achieved by placing the following code
in the preamble of the main document
(below the |\childdocmain| directive):
%
\begin{center}
\begin{tabular}{l}
|\ifchilddoc|\\
|\providecommand{\version}{draft}|\\
|\||else|\\
|\providecommand{\version}{final}|\\
|\||fi|
\end{tabular}
\end{center}
%
The definition by |\providecommand| makes sure
that previous definitions are not overwritten.
Further statements |\providecommand{\version}{...}|
can thus be added before the above code to override it.

For the main file, one might add a line
(between |\childdocmain| and the above block)
%
\begin{center}
|%\ifchilddoc\||else\providecommand{\version}{draft}\||fi|
\end{center}
%
which can be uncommented to produce a draft version.
Likewise one can add a line to the very top of a child file
(above the |\childdocof{|\textit{main}|}| directive)
%
\begin{center}
|%\providecommand{\version}{final}|
\end{center}
%
which can be uncommented to produce the final version of this child document.

%%%%%%%%%%%%%%%%%%%%%%%%%%%%%%%%%%%%%%%%%%%%%%%%%%%%%%%%%%%%%%%%%%%%%%%%%%%%%%%%
\subsection{Forwarding}
\label{sec:forward}

Different versions of the main or child documents
using compilation flags as described in \secref{sec:flags}
can be (permanently) stored in different files
for convenient compilation, viewing and distribution.
To this end, the package defines a command
to pass on compilation to a different file:

%%%%%%%%%%%%%%%%%%%%%%%%%%%%%%%%%%%%%%%%
\DescribeMacro{\childdocforward}
The command |\childdocforward| redirects processing to
another source file:
%
\begin{center}
\begin{tabular}{l}
|\input{childdoc.def}|\\
|\childdocforward[|\textit{main}|]{|\textit{dest}|}|\\
\end{tabular}
\end{center}
%
The argument \textit{dest} is the destination file
(without extension).
It should be the main file or one of the child files.
Note that further \textsf{childdoc} directives
such as |\childdocof| and |\childdocforward|
in the indicated file will be processed in this form.
The optional argument \textit{main}
passes on directly to the main file \textit{main}
while pretending to compile the child \textit{dest}.
This form behaves as if \textit{dest}
issues |\childdocof{|\textit{main}|}| right away,
and no further \textsf{childdoc} directives will be processed.

%%%%%%%%%%%%%%%%%%%%%%%%%%%%%%%%%%%%%%%%
\DescribeMacro{\...prefix}
In the alternative form |\childdocforwardprefix|,
%
\begin{center}
\begin{tabular}{l}
|\input{childdoc.def}|\\
|\childdocforwardprefix[|\textit{main}|]{|\textit{prefix}|}{|\textit{dest}|}|
\end{tabular}
\end{center}
%
the destination file is determined by a pattern
depending on the current file:
To make this work, the current file must be called
`{\textit{prefix}\hspace{0.2em}\textit{suffix}}'
with \textit{prefix} matching precisely the argument.
Processing is then passed on to the file
`{\textit{dest}\hspace{0.2em}\textit{suffix}}'.
Surely, the same effect is achieved by
directly specifying the
argument `{\textit{dest}\hspace{0.2em}\textit{suffix}}'
in the first form.
However, that requires to set up a different file
for each child. With the alternative form of the command
all these files can have exactly the same content
which simplifies setting them up and maintaining them.

For example, the following file |draft.tex|
with a compilation flag |\version| as described in \secref{sec:flags}
compiles the main document as a draft:
%
\begin{center}
\begin{tabular}{l}
|\def\version{draft}|\\
|\input{childdoc.def}|\\
|\childdocforward{|\textit{main}|}|
\end{tabular}
\end{center}
%
Likewise, the following files |final|\textit{nn}|.tex|
compile the final version of the child document
|child|\textit{nn}|.tex|:
%
\begin{center}
\begin{tabular}{l}
|\def\version{final}|\\
|\input{childdoc.def}|\\
|\childdocforwardprefix{final}{child}|
\end{tabular}
\end{center}
%

Note that when several versions of a main file and/or of each child file
are to be generated, it may be convenient to set up a |Makefile| or
shell script to automatise the process.

%%%%%%%%%%%%%%%%%%%%%%%%%%%%%%%%%%%%%%%%%%%%%%%%%%%%%%%%%%%%%%%%%%%%%%%%%%%%%%%%
\subsection{Command Line Processing}
\label{sec:commandline}

The effect of redirection files can also be achieved by invoking
the \LaTeX{} compiler with a more elaborate command line.
Most conveniently this should be done as part
of a shell script or a |Makefile|.

When using \textsf{childdoc} in the main file, the following
command lines effectively perform a redirection
(note that depending on the shell being used,
backslashes may have to be doubled: `|\|' $\to$ `|\\|'):
%
\begin{center}
|... -jobname "|\textit{target}|" |\\|"|[\textit{flags}]%
|\input{childdoc.def}\childdocforward[|\textit{main}|]{|\textit{dest}|}"|
\end{center}
%
Here \textit{target} is the name of the output file,
\textit{main} is the name of the main file
and \textit{dest} is the name of the main or child file to be processed
(all filenames without extensions).
The optional argument \textit{main} can be omitted
if \textit{main} matches \textit{dest}.
Optionally, compilation \textit{flags} can be defined via |\def| commands.
This command line makes the \TeX{} engine believe
it is compiling the file \textit{target}
whose content is specified as the latter parameter.
The provided code then forwards the processing to
\textit{main} or \textit{dest} as described in \secref{sec:forward}.

%%%%%%%%%%%%%%%%%%%%%%%%%%%%%%%%%%%%%%%%%%%%%%%%%%%%%%%%%%%%%%%%%%%%%%%%%%%%%%%%
\subsection{Include by Input}
\label{sec:input}

Including child documents by |\include| has some restrictions by design.
Most notably, the content of a child document always occupies
its own set of pages; pages cannot be shared between child documents.
Usually, this behaviour makes perfect sense
because each child document contain an essential part of the document.
However, in some situations it may be desirable to compose
a document from a collection of parts
without having mandatory page breaks between then.
For this case, the package
provides a mechanism to include parts
by |\input| which can also be processed individually.
However, by construction this mechanism
requires manual handling of the content to be output.

%%%%%%%%%%%%%%%%%%%%%%%%%%%%%%%%%%%%%%%%
\DescribeMacro{\ifchilddocmanual}
The main file should be prepared as usual, see \secref{sec:include}.
However, the document body must make a distinction
between processing of an individual part and of the main document, e.g.:
%
\begin{center}
\begin{tabular}{l}
|\ifchilddocmanual|\\
|\input{\childdocname}|\\
|\||else|\\
\textit{document body with }|\input{|\textit{part}|}|\\
|\||fi|
\end{tabular}
\end{center}
%
The conditional |\ifchilddocmanual| is true whenever
a part to be included by |\input| is being compiled,
and the name of the part is stored in |\childdocname|.

%%%%%%%%%%%%%%%%%%%%%%%%%%%%%%%%%%%%%%%%
\DescribeMacro{\childdocby}
Each part to be included by |\input| should start with:
%
\begin{center}
\begin{tabular}{l}
|\input{childdoc.def}|\\
|\childdocby{|\textit{main}|}|\\
\end{tabular}
\end{center}
%
The directive |\childdocby| is similar to |\childdocof|
described in \secref{sec:include},
but the subsequent selection of content must be done manually.
To that end, both |\ifchilddoc| and |\ifchilddocmanual|
will be true upon processing of a part,
and the name of the part is stored in |\childdocname|.
Note that |\jobname| will be set to the filename of the current part
so that each part receives an individual |.aux| file
that does not interfere with the |.aux| file(s) of the main document.
This behaviour can be altered by the alternative form
|\childdocby[*]{|\textit{main}|}| (with a non-empty optional argument)
which uses the |.aux| file of the main document
by setting |\jobname| to \textit{main}.

%%%%%%%%%%%%%%%%%%%%%%%%%%%%%%%%%%%%%%%%%%%%%%%%%%%%%%%%%%%%%%%%%%%%%%%%%%%%%%%%
\subsection{Driver Development}
\label{sec:driver}

The \textsf{childdoc} mechanism can also be use for the development
of definition files such as \LaTeX{} styles or classes.
This case differs from the above setup with multiple parts
included by |\include| in that no |\includeonly| should be invoked.
This can be achieved by starting the include file
(before |\ProvidesPackage|) with:
%
\begin{center}
\begin{tabular}{l}
|\input{childdoc.def}|\\
|\childdocforward{|\textit{main}|}|\\
\end{tabular}
\end{center}
%
or alternatively with:
%
\begin{center}
\begin{tabular}{l}
|\input{childdoc.def}|\\
|\childdocby{|\textit{main}|}|\\
\end{tabular}
\end{center}
%
Both forms have slightly different effects as described above.
The main file is prepared as usual, see \secref{sec:include}.

%%%%%%%%%%%%%%%%%%%%%%%%%%%%%%%%%%%%%%%%%%%%%%%%%%%%%%%%%%%%%%%%%%%%%%%%%%%%%%%%
\subsection{Legacy Detection}
\label{sec:detection}

The directive |\childdocmain| in the main file can detect
whether the complete document or merely a child is to be compiled
even without using the directive |\childdocof|.
This method is deprecated because it is less robust
and there is no compelling reason to use it;
it is merely provided for backward compatibility
and it may be removed in future versions.

If the detection mechanism is to be used,
it is mandatory to correctly specify
the filename of the main file as the argument of |\childdocmain|:
%
\begin{center}
\begin{tabular}{l}
|\input{childdoc.def}|\\
|\childdocmain{|\textit{main}|}|\\
\end{tabular}
\end{center}
%
If |\jobname| does not match the argument \textit{main} of |\childdocmain|,
it is assumed that |\jobname| points to the child file to be compiled.
When using |\childdocmain| with the main file specified as argument,
it suffices to start a child file
with just |\input{|\textit{main}|}|
without loading of the package and using |\childdocof|.
If instead all processing is done
with the appropriate \textsf{childdoc} directives,
the argument of \textit{main} of |\childdocmain| can be empty.

An alternative version of the command line processing described
in \secref{sec:commandline} using the detection mechanism reads:
%
\begin{center}
|... -jobname "|\textit{target}|" "|[\textit{flags}]%
[|\def\jobname{|\textit{dest}|}|]|\input{|\textit{main}|}"|
\end{center}

%%%%%%%%%%%%%%%%%%%%%%%%%%%%%%%%%%%%%%%%%%%%%%%%%%%%%%%%%%%%%%%%%%%%%%%%%%%%%%%%
\subsection{Manual Code}
\label{sec:manual}

In case one cannot be certain whether the definitions file |childdoc.def|
is installed on the target \TeX{} distribution
and one prefers not to ship it,
it is conceivable to paste a few relevant commands into the sources.

To that end, drop all statements |\input{childdoc.def}|
and perform the replacements as outlined below.
Instead of |\childdocmain{|\textit{main}|}| add the following code
to the top of the main file:
%
\begin{center}
\begin{tabular}{l}
|\||ifdefined\childdocname\endinput\||fi\newif\ifchilddoc|\\
|\edef\childdocname{\scantokens\expandafter{\jobname\noexpand}}|\\
|\def\childdocmain{|\textit{main}|}\||ifx\childdocmain\childdocname\||else|\\
|\childdoctrue\includeonly{\childdocname}\let\jobname\childdocmain\||fi|\\
\end{tabular}
\end{center}
%
Instead of |\childdocof{|\textit{main}|}| just include the main file
at the top of each child file:
%
\begin{center}
|\input{|\textit{main}|}|
\end{center}
%
A simple redirection |\childdocforward{|\textit{dest}|}| is achieved by:
%
\begin{center}
|\def\jobname{|\textit{dest}|}\input{\jobname}|
\end{center}
%
The redirection with prefix
|\childdocforwardprefix[|\textit{prefix}|]{|\textit{dest}|}|
is accomplished by:
%
\begin{center}
\begin{tabular}{l}
|{\edef\jobname{\scantokens\expandafter{\jobname\noexpand}}|\\
|\def\redirectjob |\textit{prefix}|#1~~~{\gdef\jobname{|\textit{dest}|#1}}|\\
|\expandafter\redirectjob\jobname~~~}\input{\jobname}|
\end{tabular}
\end{center}

In an alternative approach,
child documents can be compiled by a specific command line
without additional code or specific definitions:
%
\begin{center}
|... -jobname "|\textit{target}|" "|[\textit{flags}]%
|\includeonly{|\textit{dest}|}\input{|\textit{main}|}"|
\end{center}
%

%%%%%%%%%%%%%%%%%%%%%%%%%%%%%%%%%%%%%%%%%%%%%%%%%%%%%%%%%%%%%%%%%%%%%%%%%%%%%%%%
%%%%%%%%%%%%%%%%%%%%%%%%%%%%%%%%%%%%%%%%%%%%%%%%%%%%%%%%%%%%%%%%%%%%%%%%%%%%%%%%
\section{Information}

%%%%%%%%%%%%%%%%%%%%%%%%%%%%%%%%%%%%%%%%%%%%%%%%%%%%%%%%%%%%%%%%%%%%%%%%%%%%%%%%
\subsection{Copyright}

Copyright \copyright{} 2017--2018 Niklas Beisert

This work may be distributed and/or modified under the
conditions of the \LaTeX{} Project Public License, either version 1.3
of this license or (at your option) any later version.
The latest version of this license is in
  \url{http://www.latex-project.org/lppl.txt}
and version 1.3 or later is part of all distributions of \LaTeX{}
version 2005/12/01 or later.

This work has the LPPL maintenance status `maintained'.

The Current Maintainer of this work is Niklas Beisert.

This work consists of the files |README.txt|, |childdoc.ins| and |childdoc.dtx|
as well as the derived files |childdoc.def|, |cdocsamp.tex|
with |cdocsch1.tex|, |cdocsch2.tex|, |cdocspt3.tex|, |cdocspt4.tex|,
|cdocsdrf.tex|, |cdocsfn1.tex|, |cdocsfn2.tex|
as well as |childdoc.pdf|.

%%%%%%%%%%%%%%%%%%%%%%%%%%%%%%%%%%%%%%%%%%%%%%%%%%%%%%%%%%%%%%%%%%%%%%%%%%%%%%%%
\subsection{Files and Installation}

The package consists of the files:
%
\begin{center}
\begin{tabular}{ll}
    |README.txt|   & readme file \\
    |childdoc.ins| & installation file \\
    |childdoc.dtx| & source file \\
    |childdoc.def| & definition file \\
    |cdocsamp.tex| & sample main file \\
    |cdocsch1.tex| & sample include file \\
    |cdocsch2.tex| & sample include file \\
    |cdocspt3.tex| & sample part file \\
    |cdocspt4.tex| & sample part file \\
    |cdocsdrf.tex| & sample redirection file \\
    |cdocsfn1.tex| & sample redirection file \\
    |cdocsfn2.tex| & sample redirection file \\
    |childdoc.pdf| & manual
\end{tabular}
\end{center}
%
The distribution consists of the files
|README.txt|, |childdoc.ins| and |childdoc.dtx|.
%
\begin{itemize}
\item
Run (pdf)\LaTeX{} on |childdoc.dtx|
to compile the manual |childdoc.pdf| (this file).
\item
Run \LaTeX{} on |childdoc.ins| to create the definitions file |childdoc.def|
and the sample |cdocsamp.tex| with include files
|cdocsch1.tex|, |cdocsch2.tex|, |cdocspt3.tex|, |cdocspt4.tex|,
|cdocsdrf.tex|, |cdocsfn1.tex|, |cdocsfn2.tex|.
Then copy the file |childdoc.def| to an appropriate directory of your \LaTeX{}
distribution, e.g.\ \textit{texmf-root}|/tex/latex/childdoc|.
\end{itemize}

%%%%%%%%%%%%%%%%%%%%%%%%%%%%%%%%%%%%%%%%%%%%%%%%%%%%%%%%%%%%%%%%%%%%%%%%%%%%%%%%
\subsection{Related CTAN Packages}

There are several other packages which offer a similar functionality:
%
\begin{itemize}
\item
The packages
\href{http://ctan.org/pkg/docmute}{\textsf{docmute}},
\href{http://ctan.org/pkg/includex}{\textsf{includex}} and
\href{http://ctan.org/pkg/standalone}{\textsf{standalone}}
provide commands to include only the document body of
a child file thus allowing both files to be compiled individually.
\item
The packages \href{http://ctan.org/pkg/subdocs}{\textsf{subdocs}}
and \href{http://ctan.org/pkg/subfiles}{\textsf{subfiles}}
provide structures in which the main and child documents can be
encapsulated and allowing them to be compiled individually.
The inclusion mechanism is different from the conventional |\include|.
\item
The package \href{http://ctan.org/pkg/combine}{\textsf{combine}}
is an elaborate solution to combine several documents into one.
\end{itemize}
%
See also the CTAN topic \href{http://ctan.org/topic/subdocs}{\textsf{subdocs}}
for further related packages.
The present package differs from the above solutions in that
a document structure constructed with the conventional |\include| mechanism
just needs two extra commands at the top of every file
such that all constituent files can be compiled individually.

%%%%%%%%%%%%%%%%%%%%%%%%%%%%%%%%%%%%%%%%%%%%%%%%%%%%%%%%%%%%%%%%%%%%%%%%%%%%%%%%
%\subsection{Feature Suggestions}
%
%The following is a list of features which may be useful for future
%versions of this package:
%%
%\begin{itemize}
%\item
%\ldots
%\end{itemize}

%%%%%%%%%%%%%%%%%%%%%%%%%%%%%%%%%%%%%%%%%%%%%%%%%%%%%%%%%%%%%%%%%%%%%%%%%%%%%%%%
\subsection{Revision History}

%%%%%%%%%%%%%%%%%%%%%%%%%%%%%%%%%%%%%%%%
\paragraph{v2.0:} 2018/12/30

\begin{itemize}
\item
immediate forward processing
\item
added |\childdocby| mechanism
\item
manual restructured
\end{itemize}

%%%%%%%%%%%%%%%%%%%%%%%%%%%%%%%%%%%%%%%%
\paragraph{v1.6:} 2018/01/17

\begin{itemize}
\item
application for development of include files
\item
corrections to manual
\end{itemize}

%%%%%%%%%%%%%%%%%%%%%%%%%%%%%%%%%%%%%%%%
\paragraph{v1.5:} 2017/05/21

\begin{itemize}
\item
more complete structuring introduced
\item
|\childdocof| introduced
\item
|\childdoc| renamed to |\childdocmain|
\item
|\childredirect| renamed to |\childdocforward| and |\childdocforwardprefix|
and functionality expanded
\end{itemize}

%%%%%%%%%%%%%%%%%%%%%%%%%%%%%%%%%%%%%%%%
\paragraph{v1.0:} 2017/04/27

\begin{itemize}
\item
manual and install package
\item
first version published on CTAN
\end{itemize}

%%%%%%%%%%%%%%%%%%%%%%%%%%%%%%%%%%%%%%%%
\paragraph{v0.6:} 2017/04/26

\begin{itemize}
\item
redirection mechanism added
\end{itemize}

%%%%%%%%%%%%%%%%%%%%%%%%%%%%%%%%%%%%%%%%
\paragraph{v0.5:} 2017/04/26

\begin{itemize}
\item
functionality in definition file
\end{itemize}


%%%%%%%%%%%%%%%%%%%%%%%%%%%%%%%%%%%%%%%%%%%%%%%%%%%%%%%%%%%%%%%%%%%%%%%%%%%%%%%%
%%%%%%%%%%%%%%%%%%%%%%%%%%%%%%%%%%%%%%%%%%%%%%%%%%%%%%%%%%%%%%%%%%%%%%%%%%%%%%%%
%%%%%%%%%%%%%%%%%%%%%%%%%%%%%%%%%%%%%%%%%%%%%%%%%%%%%%%%%%%%%%%%%%%%%%%%%%%%%%%%
\appendix

\settowidth\MacroIndent{\rmfamily\scriptsize 000\ }

 \DocInput{childdoc.dtx}

\end{document}
%</driver>
% \fi
%
% %%%%%%%%%%%%%%%%%%%%%%%%%%%%%%%%%%%%%%%%%%%%%%%%%%%%%%%%%%%%%%%%%%%%%%%%%%%%%%
% %%%%%%%%%%%%%%%%%%%%%%%%%%%%%%%%%%%%%%%%%%%%%%%%%%%%%%%%%%%%%%%%%%%%%%%%%%%%%%
% \section{Sample}
%\iffalse
%<*samplemain>
%\fi
%
% The following presents a sample document
% with two chapters, two parts, a title page,
% a compile flag as well as three forwarding files to set the flag.
% It consists of eight |.tex| files:
% \begin{center}
% \begin{tabular}{ll}
% |cdocsamp.tex|&main file\\
% |cdocsch1.tex|&include file for chapter 1\\
% |cdocsch2.tex|&include file for chapter 2\\
% |cdocspt3.tex|&include file for part 3\\
% |cdocspt4.tex|&include file for part 4\\
% |cdocsdrf.tex|&forwarding file for main file in draft mode\\
% |cdocsfi1.tex|&forwarding file for final version of chapter 1\\
% |cdocsfi2.tex|&forwarding file for final version of chapter 2\\
% \end{tabular}
% \end{center}
% Each of the eight files can be compiled directly by the \LaTeX{} compiler.
%
% %%%%%%%%%%%%%%%%%%%%%%%%%%%%%%%%%%%%%%
% \paragraph{Main File.}
%
% The main file is called |cdocsamp.tex|.
%
% Load the \textsf{childdoc} definitions and
% declare the filename for the main document:
%    \begin{macrocode}
\input{childdoc.def}
\childdocmain{}
%    \end{macrocode}

% Optional override for |\version| flag:
%    \begin{macrocode}
%%\ifchilddoc\else\providecommand{\version}{draft}\fi
%    \end{macrocode}

% Define the default values for the |\version| flag
% (|final| for the main file and |draft| for childs):
%    \begin{macrocode}
\ifchilddoc
\providecommand{\version}{draft}
\else
\providecommand{\version}{final}
\fi
%    \end{macrocode}

% Load the standard document class:
%    \begin{macrocode}
\documentclass[12pt]{article}
%    \end{macrocode}

% Start the document body:
%    \begin{macrocode}
\begin{document}
%    \end{macrocode}

% Declare a title page.
% Print title, part of document being processed and version flag:
%    \begin{macrocode}
\addtocounter{page}{-1}
\begin{center}
{\LARGE\bfseries{}childdoc example\par}
\vspace{1cm}
\ifchilddoc
\ifchilddocmanual part\else chapter\fi:
`\childdocname' of `\childdocjob'\par
\else
main document: `\childdocjob'\par
\fi
version: \version\par
\end{center}
\newpage
%    \end{macrocode}

% Manually include selected file,
% otherwise process as usual:
%    \begin{macrocode}
\ifchilddocmanual
\section*{part `\childdocname'}
\input{\childdocname}
\else
%    \end{macrocode}

% Include the two chapters:
%    \begin{macrocode}
\include{cdocsch1}
\include{cdocsch2}
%    \end{macrocode}

% Include the two parts unless only chapters should be displayed:
%    \begin{macrocode}
\ifchilddoc\else
\section{part three}
\input{cdocspt3}
\section{part four}
\input{cdocspt4}
\fi
%    \end{macrocode}

% Process as usual until here:
%    \begin{macrocode}
\fi
%    \end{macrocode}

% End of document body:
%    \begin{macrocode}
\end{document}
%    \end{macrocode}
%\iffalse
%</samplemain>
%\fi
%
% %%%%%%%%%%%%%%%%%%%%%%%%%%%%%%%%%%%%%%
% \paragraph{Chapter Include Files.}
%
% The include files are called |cdocsch1.tex| and |cdocsch2.tex|.
%
%\iffalse
%<*samplechap1|samplechap2>
%\fi

% Optional override for |\version| flag:
%    \begin{macrocode}
%%\providecommand{\version}{final}
%    \end{macrocode}

% Include the main document:
%    \begin{macrocode}
\input{childdoc.def}
\childdocof{cdocsamp}
%    \end{macrocode}

%\iffalse
%</samplechap1|samplechap2>
%\fi
%
%\iffalse
%<*samplechap1>
%\fi
% Some text for chapter 1:
%    \begin{macrocode}
\section{one}
some text in chapter one
%    \end{macrocode}

%\iffalse
%</samplechap1>
%\fi
% Some text for chapter 2:
%\iffalse
%<*samplechap2>
%\fi
%    \begin{macrocode}
\section{two}
more text in chapter two
%    \end{macrocode}

%\iffalse
%</samplechap2>
%\fi
%
% %%%%%%%%%%%%%%%%%%%%%%%%%%%%%%%%%%%%%%
% \paragraph{Part Include Files.}
%
% The include files are called |cdocspt3.tex| and |cdocspt4.tex|.
%
%\iffalse
%<*samplepart3|samplepart4>
%\fi

% Optional override for |\version| flag:
%    \begin{macrocode}
%%\providecommand{\version}{final}
%    \end{macrocode}

% Include the main document:
%    \begin{macrocode}
\input{childdoc.def}
\childdocby{cdocsamp}
%    \end{macrocode}

%\iffalse
%</samplepart3|samplepart4>
%\fi
%
%\iffalse
%<*samplepart3>
%\fi
% Some text for part 3:
%    \begin{macrocode}
some text in part three
%    \end{macrocode}

%\iffalse
%</samplepart3>
%\fi
% Some text for part 4:
%\iffalse
%<*samplepart4>
%\fi
%    \begin{macrocode}
more text in part four
%    \end{macrocode}

%\iffalse
%</samplepart4>
%\fi
%
% %%%%%%%%%%%%%%%%%%%%%%%%%%%%%%%%%%%%%%
% \paragraph{Forwarding for a Complete Draft.}
%
% The following forwarding file |cdocsdrf.tex|
% compiles the main document in draft mode:
%\iffalse
%<*sampledraft>
%\fi
%    \begin{macrocode}
\def\version{draft}
\input{childdoc.def}
\childdocforward{cdocsamp}
%    \end{macrocode}

%\iffalse
%</sampledraft>
%\fi
%
% %%%%%%%%%%%%%%%%%%%%%%%%%%%%%%%%%%%%%%
% \paragraph{Forwarding for Final Version of the Chapters.}
%
% The following forwarding files |cdocsfn1.tex| and |cdocsfn2.tex|
% (with identical content)
% compile the final versions of the child documents
% |cdocsch1.tex| and |cdocsch2.tex|, respectively:
%\iffalse
%<*samplefinal>
%\fi
%    \begin{macrocode}
\def\version{final}
\input{childdoc.def}
\childdocforwardprefix[cdocsamp]{cdocsfn}{cdocsch}
%    \end{macrocode}

%\iffalse
%</samplefinal>
%\fi
%
% %%%%%%%%%%%%%%%%%%%%%%%%%%%%%%%%%%%%%%
% \paragraph{Command Line Processing.}
%
% The following three command lines generate the output files
% |cdocscld|, |cdocscl1| and |cdocscl2|
% which should be identical to
% |cdocsdrf|, |cdocsch1| and |cdocsfn2|, respectively:
% \begin{center}
% \begin{tabular}{l}
% |latex -jobname cdocscld \|\\
% |  "\def\version{draft}\input{childdoc.def}\childdocforward{cdocsamp}"|\\
% |latex -jobname cdocscl1 \|\\
% |  "\input{childdoc.def}\childdocforward[cdocsamp]{cdocsch1}"|\\
% |latex -jobname cdocscl2 \|\\
% |  "\def\version{final}\input{childdoc.def}\childdocforward{cdocsch2}"|
% \end{tabular}
% \end{center}
% Note that the trailing backslash on each first line
% merely continues the input to the second line
% (for convenient cut ant paste).
% Furthermore, the command |latex| can be replaced by any
% of its alternative versions such as |pdflatex|.
%
% %%%%%%%%%%%%%%%%%%%%%%%%%%%%%%%%%%%%%%%%%%%%%%%%%%%%%%%%%%%%%%%%%%%%%%%%%%%%%%
% %%%%%%%%%%%%%%%%%%%%%%%%%%%%%%%%%%%%%%%%%%%%%%%%%%%%%%%%%%%%%%%%%%%%%%%%%%%%%%
% \section{Implementation}
%\iffalse
%<*package>
%\fi
%
% This section describes the definitions file |childdoc.def|.

% The definitions cannot be loaded using |\usepackage| or |\RequirePackage|
% which has a mechanism to prevent loading a style file more than once.
% When loading the definitions by means of |\input|
% multiple instances have to be prevented manually:
%\iffalse
%This code needs to be before the `\ProvidesFile' directive
%which is defined at the beginning of this file.
%Therefore it is also placed there and commented out here.
%</package>
%<*discard>
%\fi
%    \begin{macrocode}
\ifdefined\childdocmain\endinput\fi
%    \end{macrocode}
%\iffalse
%</discard>
%<*package>
%\fi
%
% \macro{\ifchilddoc}
% \macro{\ifchilddocmanual}
% The conditional |\ifchilddoc| tells whether a
% child (true) or main (false) document is being compiled.
% The conditional |\ifchilddocmanual| tells whether
% the |\includeonly| mechanism is used (false) or
% the selection of child files must be performed manually (true).
% The definitions initialise to false:
%    \begin{macrocode}
\newif\ifchilddoc
\newif\ifchilddocmanual
%    \end{macrocode}

% \macro{\childdocname}
% \macro{\childdocjob}
% The macro |\childdocname| stores the name of the main document
% to be compiled. The macro |\childdocjob| stores the name of
% the document on which the \LaTeX{} compiler was originally invoked.
% The content of |\jobname| cannot be compared
% to filenames specified in the source due to different catcodes.
% The following code rescans |\jobname|, stores the result
% in |\childdocname| and saves a copy in |\childdocjob|:
%    \begin{macrocode}
\edef\childdocname{\scantokens\expandafter{\jobname\noexpand}}
\let\childdocjob\childdocname
%    \end{macrocode}

% \macro{\childdocdisable}
% The macro |\childdocdisable| prevents the main file
% from being processed more than once.
% At this stage, the main document command |\childdocmain|
% is assumed to be called once again where it should do nothing.
% Any subsequent call to it should prevent
% a secondary processing of the main document
% It overwrites the forwarding commands
% |\childdocof| and |\childdocforward|
% with empty macros to prevent further inclusions of the main document:
%    \begin{macrocode}
\newcommand{\childdocdisable}
{
  \renewcommand{\childdocmain}[1]{\renewcommand{\childdocmain}[1]{\endinput}}
  \renewcommand{\childdocof}[1]{}
  \renewcommand{\childdocby}[2][]{}
  \renewcommand{\childdocforward}[2][]{}
  \renewcommand{\childdocdisable}{}
}
%    \end{macrocode}

% \macro{\childdocmain}
% The macro |\childdocmain| is to be called at the top of the main file
% with nothing or the main filename (without extension) as argument.
% First, it breaks loops.
% If the argument is not empty and does not match |\childdocname|
% (which is set by the first inclusion of |childdoc.def|),
% |\ifchilddoc| is set to true, |\includeonly| is applied to the child file
% and |\jobname| is set to the main file
% (for proper handling of |.aux| files):
%    \begin{macrocode}
\newcommand{\childdocmain}[1]
{
  \childdocdisable\childdocmain{}
  \if?#1?\else
    \begingroup
      \def\childdoctmp{#1}
      \ifx\childdoctmp\childdocname
        \def\childdoctmp{}
      \else
        \def\childdoctmp
        {
          \childdoctrue
          \includeonly{\childdocname}
          \def\childdocjob{#1}
          \def\jobname{#1}
        }
      \fi
      \expandafter
    \endgroup
    \childdoctmp
  \fi
}
%    \end{macrocode}

% \macro{\childdocof}
% The command |\childdocof| redirects
% compilation to the main file |#1|.
%    \begin{macrocode}
\newcommand{\childdocof}[1]
{
  \childdocdisable
  \childdoctrue
  \includeonly{\childdocname}
  \def\jobname{#1}
  \def\childdocjob{#1}
  \input{#1}
}
%    \end{macrocode}

% \macro{\childdocby}
% The command |\childdocby| ....
%    \begin{macrocode}
\newcommand{\childdocby}[2][]
{
  \childdocdisable
  \childdoctrue
  \childdocmanualtrue
  \if?#1?\else
    \def\jobname{#2}
  \fi
  \def\childdocjob{#2}
  \input{#2}
  \endinput
}
%    \end{macrocode}

% \macro{\childdocforward}
% The command |\childdocforward| redirects
% compilation to the main file or
% (if the optional argument is given) a child file.
% Parameters are set as if the main file
% or a child file starting with |\childdocof| was compiled.
% Then compilation is handed over to the main file:
%    \begin{macrocode}
\newcommand{\childdocforward}[2][]
{
  \begingroup
    \if?#1?
      \def\childdoctmp
      {
        \def\childdocname{#2}
        \def\childdocjob{#2}
        \def\jobname{#2}
        \input{#2}
        \endinput
      }
    \else
      \def\childdoctmp
      {
        \childdocdisable
        \def\childdocname{#2}
        \childdoctrue
        \includeonly{#2}
        \def\childdocjob{#1}
        \def\jobname{#1}
        \input{#1}
        \endinput
      }
    \fi
    \expandafter
  \endgroup
  \childdoctmp
}
%    \end{macrocode}

% \macro{\childdocforwardprefix}
% The command |\childdocforwardprefix| redirects
% compilation to the main or a child file by means of a pattern.
% The prefix |#1| in the current filename is replaced by |#2|
% and the suffix of the current filename is kept
% (it is assumed that the filename does not contain the substring `|~~~|'
% which is used as a delimiter).
% Compilation is handed over to the new file by |\childdocforward|:
%    \begin{macrocode}
\newcommand{\childdocforwardprefix}[3][]
{
  \begingroup
    \def\childdocextract #2##1~~~{\def\childdoctmp{\childdocforward[#1]{#3##1}}}
    \expandafter\childdocextract\childdocname~~~
    \expandafter
  \endgroup
  \childdoctmp
}
%    \end{macrocode}

% \macro{\childdoc}
% The deprecated macro |\childdoc| is a legacy version of |\childdocmain|:
%    \begin{macrocode}
\newcommand{\childdoc}{\childdocmain}
%    \end{macrocode}

% \macro{\childdocredirect}
% The deprecated macro |\childdocredirect| is a legacy version
% of |\childdocforward| and |\childdocforwardprefix|:
%    \begin{macrocode}
\newcommand{\childdocredirect}[2][]
{
  \begingroup
    \if?#1?
      \def\childdoctmp{\childdocforward{#2}}
    \else
      \def\childdoctmp{\childdocforwardprefix{#1}{#2}}
    \fi
    \expandafter
  \endgroup
  \childdoctmp
}
%    \end{macrocode}

%\iffalse
%</package>
%\fi
%
\endinput
|\\
|\childdocmain{}|\\
\end{tabular}
\end{center}
at the very top of the main \LaTeX{} file,
in particular \emph{before} the |\documentclass| statement!
The argument of |\childdocmain| should be left empty
(but it must be present).

%%%%%%%%%%%%%%%%%%%%%%%%%%%%%%%%%%%%%%%%
\DescribeMacro{\childdocof}
Furthermore, add the commands
\begin{center}
\begin{tabular}{l}
|% \iffalse
%
% childdoc.dtx Copyright (C) 2017-2018 Niklas Beisert
%
% This work may be distributed and/or modified under the
% conditions of the LaTeX Project Public License, either version 1.3
% of this license or (at your option) any later version.
% The latest version of this license is in
%   http://www.latex-project.org/lppl.txt
% and version 1.3 or later is part of all distributions of LaTeX
% version 2005/12/01 or later.
%
% This work has the LPPL maintenance status `maintained'.
%
% The Current Maintainer of this work is Niklas Beisert.
%
% This work consists of the files childdoc.dtx and childdoc.ins
% and the derived files childdoc.def and cdocsamp.tex with
% cdocsch1.tex, cdocsch2.tex, cdocsdrf.tex, cdocsfn1.tex, cdocsfn2.tex.
%
%<package>\ifdefined\childdocmain\endinput\fi
%<package>\ProvidesFile{childdoc.def}[2018/12/30 v2.0 child document driver]
%<samplemain>\ProvidesFile{cdocsamp.tex}[2018/12/30 v2.0 sample for childdoc]
%<*driver>
%\ProvidesFile{childdoc.drv}[2018/12/30 v2.0 childdoc reference manual file]
\PassOptionsToClass{10pt,a4paper}{article}
\documentclass{ltxdoc}

\usepackage[margin=35mm]{geometry}
\usepackage{hyperref}
\usepackage{hyperxmp}
\usepackage[usenames]{color}

\hypersetup{colorlinks=true}
\hypersetup{pdfstartview=FitH}
\hypersetup{pdfpagemode=UseNone}
\hypersetup{pdfsource={}}
\hypersetup{pdflang={en-UK}}
\hypersetup{pdfcopyright={Copyright 2017-2018 Niklas Beisert.
  This work may be distributed and/or modified under the
  conditions of the LaTeX Project Public License, either version 1.3
  of this license or (at your option) any later version.}}
\hypersetup{pdflicenseurl={http://www.latex-project.org/lppl.txt}}
\hypersetup{pdfcontactaddress={ETH Zurich, ITP, HIT K,
  Wolfgang-Pauli-Strasse 27}}
\hypersetup{pdfcontactpostcode={8093}}
\hypersetup{pdfcontactcity={Zurich}}
\hypersetup{pdfcontactcountry={Switzerland}}
\hypersetup{pdfcontactemail={nbeisert@itp.phys.ethz.ch}}
\hypersetup{pdfcontacturl={http://people.phys.ethz.ch/\xmptilde nbeisert/}}

\newcommand{\secref}[1]{\hyperref[#1]{section \ref*{#1}}}

\parskip1ex
\parindent0pt
\let\olditemize\itemize
\def\itemize{\olditemize\parskip0pt}

\begin{document}

\title{The \textsf{childdoc} Package}
\hypersetup{pdftitle={The childdoc Package}}
\author{Niklas Beisert\\[2ex]
  Institut f\"ur Theoretische Physik\\
  Eidgen\"ossische Technische Hochschule Z\"urich\\
  Wolfgang-Pauli-Strasse 27, 8093 Z\"urich, Switzerland\\[1ex]
  \href{mailto:nbeisert@itp.phys.ethz.ch}
  {\texttt{nbeisert@itp.phys.ethz.ch}}}
\hypersetup{pdfauthor={Niklas Beisert}}
\hypersetup{pdfsubject={Manual for the LaTeX2e Package childdoc}}
\date{30 December 2018, \textsf{v2.0}}
\maketitle

\begin{abstract}\noindent
\textsf{childdoc} is a \LaTeXe{} package
that enables the direct compilation
of document sections included by |\include|
to individual files.
\end{abstract}

\begingroup
\parskip0ex
\tableofcontents
\endgroup

%%%%%%%%%%%%%%%%%%%%%%%%%%%%%%%%%%%%%%%%%%%%%%%%%%%%%%%%%%%%%%%%%%%%%%%%%%%%%%%%
%%%%%%%%%%%%%%%%%%%%%%%%%%%%%%%%%%%%%%%%%%%%%%%%%%%%%%%%%%%%%%%%%%%%%%%%%%%%%%%%
\section{Introduction}

\LaTeX{} provides a mechanism to structure a large document (such as a book)
into a main file and several child files (containing the chapters)
using the |\include| command.
This mechanism is beneficial for documents
which span hundreds of pages in order to
make the source file(s) more manageable.
Moreover, compilation can be restricted to
selected child files by means of the |\includeonly| command.
The latter feature can be used to reduce the compilation time while editing
(this was significantly more useful in the earlier days of \LaTeX{})
or to generate a smaller document which is easier to navigate.
Another application of |\includeonly| is to generate
documents consisting of selected parts of the complete document.

However, there are a few drawbacks of the plain |\include| mechanism:
\begin{itemize}
\item
The child files cannot be compiled on their own,
they can only be compiled via the main file.
A naive editing environment
(such as a text editor with an option
to have the current file processed by \LaTeX)
may require one to switch to the main file before compiling;
attempting to compile the child file produces errors.
\item
The main file must be modified (each time)
to adjust the |\includeonly| command
to the present needs. This easily leaves the main file in a messy state.
\item
The generated document will always carry the filename
of the main document. This is inconvenient if
several child files are to be compiled and
to be kept for distribution.
\end{itemize}

The present package provides a simple interface
to make child files individually compilable by \LaTeX{}.
Compiling a child file then has the same effect as compiling
the main file with an |\includeonly| command
to select the appropriate child.
Moreover the generated document will carry the name of the child
rather than the main file.
This resolves all three above issues.

This feature is meant to make the editing of books,
thesis documents and lecture notes somewhat more convenient.
However, the package can also be used efficiently for
composing a series of documents (such as exercise sheets)
which are typically distributed individually.
It then assists the author in generating the individual documents
(potentially in different versions)
as well as a document containing the collected series.
Another application is in developing style files
or other kinds of included material
where compilation of the style file could redirect
to a sample or test file.

%%%%%%%%%%%%%%%%%%%%%%%%%%%%%%%%%%%%%%%%%%%%%%%%%%%%%%%%%%%%%%%%%%%%%%%%%%%%%%%%
%%%%%%%%%%%%%%%%%%%%%%%%%%%%%%%%%%%%%%%%%%%%%%%%%%%%%%%%%%%%%%%%%%%%%%%%%%%%%%%%
\section{Usage}

First of all, the package \textsf{childdoc} is \emph{not} a standard
\LaTeXe{} |.sty| style file! Therefore it needs to be invoked in
a non-standard way.

%%%%%%%%%%%%%%%%%%%%%%%%%%%%%%%%%%%%%%%%%%%%%%%%%%%%%%%%%%%%%%%%%%%%%%%%%%%%%%%%
\subsection{Included Files}
\label{sec:include}

%%%%%%%%%%%%%%%%%%%%%%%%%%%%%%%%%%%%%%%%
\DescribeMacro{\childdocmain}
To use the package, add the commands
\begin{center}
\begin{tabular}{l}
|\input{childdoc.def}|\\
|\childdocmain{}|\\
\end{tabular}
\end{center}
at the very top of the main \LaTeX{} file,
in particular \emph{before} the |\documentclass| statement!
The argument of |\childdocmain| should be left empty
(but it must be present).

%%%%%%%%%%%%%%%%%%%%%%%%%%%%%%%%%%%%%%%%
\DescribeMacro{\childdocof}
Furthermore, add the commands
\begin{center}
\begin{tabular}{l}
|\input{childdoc.def}|\\
|\childdocof{|\textit{main}|}|\\
\end{tabular}
\end{center}
at the top of every child file \textit{child}
which is included by |\include{|\textit{child}|}|
from within the main file
(or at least for those files to be compiled individually).
The argument \textit{main} must be the filename of the main file.

There are a couple of
considerations in setting up the main and child documents:

%%%%%%%%%%%%%%%%%%%%%%%%%%%%%%%%%%%%%%%%
\paragraph{Restrictions.}

Please note the following restrictions:
\begin{itemize}
\item
|\childdocmain| must be called with one argument \textit{main}
to ensure compatibility with earlier version of the package.
It must either be empty (|\childdocmain{}|)
or precisely match the filename of the main file in which it is specified.
See \secref{sec:detection} for further information.
\item
The filename \textit{main} must be specified without the |.tex| extension.
\item
The filename \textit{main} is case sensitive
(even in case-insensitive file systems)
due to internal string comparison.
\item
The argument \textit{main} should be fully expanded, it cannot be a macro.
\item
Subdirectories and special characters should be avoided in filenames.
\item
The command |\childdocmain{|\textit{main}|}| must be followed by a whitespace.
It should not be followed immediately by another command
or by a comment mark `|%|'.
This is because the \TeX{} parser reads the token immediately following
the argument of |\childdocmain| and puts it
at the beginning of every child section;
however, a white\-space is ignored.
\end{itemize}

%%%%%%%%%%%%%%%%%%%%%%%%%%%%%%%%%%%%%%%%
\paragraph{Content of Main File.}

It is advisable to place all content in the child files included by |\include|.
Any output contained in the main file will appear in all child documents
unless suppressed manually;
it cannot be suppressed automatically by the |\includeonly| directive
and thus should normally be avoided.
A method to include some content in the main file
by means of conditional processing is described in \secref{sec:conditional}.

%%%%%%%%%%%%%%%%%%%%%%%%%%%%%%%%%%%%%%%%
\paragraph{Page Numbering.}

When only a part of the document is compiled,
the appropriate numbering of pages
(as well as other status parameters)
is determined from the |.aux| files.
The latter contain information from previous passes.
However this information needs to propagate through
all intermediate child documents.
Therefore the page numbering in child documents may well
be inconsistent until the complete document is compiled at least once.

A useful (if unconventional) way to always ensure a consistent
page numbering is to restart the numbering in each child document
and denote the pages by `\textit{child}|.|\textit{page}'
where \textit{child} represents the chapter/section number of the child file.
This can be achieved by the command
|\numberwithin{page}{|\textit{child}|}|
of the \textsf{amsmath} package
where \textit{child} can be |chapter| or |section|
depending on the chosen structuring.
Alternatively, one can modify the macro |\thepage| appropriately
and reset the counter |page| at the start of each child file.

%%%%%%%%%%%%%%%%%%%%%%%%%%%%%%%%%%%%%%%%%%%%%%%%%%%%%%%%%%%%%%%%%%%%%%%%%%%%%%%%
\subsection{Conditional Processing}
\label{sec:conditional}

The package provides a mechanism to compile different versions
of a document. To customise the versions further some conditional processing
can come in handy to distinguish which version is being compiled.
The package provides two macros to describe the compilation context:

%%%%%%%%%%%%%%%%%%%%%%%%%%%%%%%%%%%%%%%%
\DescribeMacro{\ifchilddoc}
The conditional |\ifchilddoc| distinguishes between the compilation of
child documents and the main document:
%
\begin{center}
|\ifchilddoc |\textit{child-code}| |[|\||else |\textit{main-code}]| \||fi|
\end{center}

%%%%%%%%%%%%%%%%%%%%%%%%%%%%%%%%%%%%%%%%
\DescribeMacro{\childdocname}
\DescribeMacro{\childdocjob}
The macro |\childdocname| contains the filename (without extension)
of the main or child file being processed.
Note that |\childdocjob| will always contain the name of the main file.

%%%%%%%%%%%%%%%%%%%%%%%%%%%%%%%%%%%%%%%%
\paragraph{Title Page.}

Conditional processing can be used to include a title or banner page
in the main document when proper precautions are taken.
Importantly, the code in the main file should ensure that the page counter
(as well as other status parameters which are stored in the |.aux| files)
takes the same value after the conditional processing.
Otherwise the page numbers may take divergent values
depending on which part is compiled.

For example, a title page could be declared by:
%
\begin{center}
\begin{tabular}{l}
|\ifchilddoc\||else|\\
|\addtocounter{page}{-1}|\\
\textit{code for title page}\\
|\newpage|\\
|\||fi|
\end{tabular}
\end{center}
%
A banner page for the child documents can be generated by:
%
\begin{center}
\begin{tabular}{l}
|\ifchilddoc|\\
|\addtocounter{page}{-1}|\\
\textit{code for banner page}\\
|\newpage|\\
|\||fi|
\end{tabular}
\end{center}
%
Here one could write a message such as:
\begin{center}
|This is the part \childdocname{} of \childdocjob{}.|
\end{center}

%%%%%%%%%%%%%%%%%%%%%%%%%%%%%%%%%%%%%%%%%%%%%%%%%%%%%%%%%%%%%%%%%%%%%%%%%%%%%%%%
\subsection{Flags}
\label{sec:flags}

The package makes it easy to generate different versions
of the main or child documents.
To this end compilation flags can be defined
and assigned different default values.
They will be particularly useful in conjunction
with the forwarding mechanism described in \secref{sec:forward}.

For example, it may be useful to have a flag |\version|
which can be set to |draft| or |final|.
The document source will contain some conditional code
depending on the value of |\version|.
Suppose further, the flag should default to |final| for the main file
and to |draft| for child files
which is a natural assignment for editing the document.
This is achieved by placing the following code
in the preamble of the main document
(below the |\childdocmain| directive):
%
\begin{center}
\begin{tabular}{l}
|\ifchilddoc|\\
|\providecommand{\version}{draft}|\\
|\||else|\\
|\providecommand{\version}{final}|\\
|\||fi|
\end{tabular}
\end{center}
%
The definition by |\providecommand| makes sure
that previous definitions are not overwritten.
Further statements |\providecommand{\version}{...}|
can thus be added before the above code to override it.

For the main file, one might add a line
(between |\childdocmain| and the above block)
%
\begin{center}
|%\ifchilddoc\||else\providecommand{\version}{draft}\||fi|
\end{center}
%
which can be uncommented to produce a draft version.
Likewise one can add a line to the very top of a child file
(above the |\childdocof{|\textit{main}|}| directive)
%
\begin{center}
|%\providecommand{\version}{final}|
\end{center}
%
which can be uncommented to produce the final version of this child document.

%%%%%%%%%%%%%%%%%%%%%%%%%%%%%%%%%%%%%%%%%%%%%%%%%%%%%%%%%%%%%%%%%%%%%%%%%%%%%%%%
\subsection{Forwarding}
\label{sec:forward}

Different versions of the main or child documents
using compilation flags as described in \secref{sec:flags}
can be (permanently) stored in different files
for convenient compilation, viewing and distribution.
To this end, the package defines a command
to pass on compilation to a different file:

%%%%%%%%%%%%%%%%%%%%%%%%%%%%%%%%%%%%%%%%
\DescribeMacro{\childdocforward}
The command |\childdocforward| redirects processing to
another source file:
%
\begin{center}
\begin{tabular}{l}
|\input{childdoc.def}|\\
|\childdocforward[|\textit{main}|]{|\textit{dest}|}|\\
\end{tabular}
\end{center}
%
The argument \textit{dest} is the destination file
(without extension).
It should be the main file or one of the child files.
Note that further \textsf{childdoc} directives
such as |\childdocof| and |\childdocforward|
in the indicated file will be processed in this form.
The optional argument \textit{main}
passes on directly to the main file \textit{main}
while pretending to compile the child \textit{dest}.
This form behaves as if \textit{dest}
issues |\childdocof{|\textit{main}|}| right away,
and no further \textsf{childdoc} directives will be processed.

%%%%%%%%%%%%%%%%%%%%%%%%%%%%%%%%%%%%%%%%
\DescribeMacro{\...prefix}
In the alternative form |\childdocforwardprefix|,
%
\begin{center}
\begin{tabular}{l}
|\input{childdoc.def}|\\
|\childdocforwardprefix[|\textit{main}|]{|\textit{prefix}|}{|\textit{dest}|}|
\end{tabular}
\end{center}
%
the destination file is determined by a pattern
depending on the current file:
To make this work, the current file must be called
`{\textit{prefix}\hspace{0.2em}\textit{suffix}}'
with \textit{prefix} matching precisely the argument.
Processing is then passed on to the file
`{\textit{dest}\hspace{0.2em}\textit{suffix}}'.
Surely, the same effect is achieved by
directly specifying the
argument `{\textit{dest}\hspace{0.2em}\textit{suffix}}'
in the first form.
However, that requires to set up a different file
for each child. With the alternative form of the command
all these files can have exactly the same content
which simplifies setting them up and maintaining them.

For example, the following file |draft.tex|
with a compilation flag |\version| as described in \secref{sec:flags}
compiles the main document as a draft:
%
\begin{center}
\begin{tabular}{l}
|\def\version{draft}|\\
|\input{childdoc.def}|\\
|\childdocforward{|\textit{main}|}|
\end{tabular}
\end{center}
%
Likewise, the following files |final|\textit{nn}|.tex|
compile the final version of the child document
|child|\textit{nn}|.tex|:
%
\begin{center}
\begin{tabular}{l}
|\def\version{final}|\\
|\input{childdoc.def}|\\
|\childdocforwardprefix{final}{child}|
\end{tabular}
\end{center}
%

Note that when several versions of a main file and/or of each child file
are to be generated, it may be convenient to set up a |Makefile| or
shell script to automatise the process.

%%%%%%%%%%%%%%%%%%%%%%%%%%%%%%%%%%%%%%%%%%%%%%%%%%%%%%%%%%%%%%%%%%%%%%%%%%%%%%%%
\subsection{Command Line Processing}
\label{sec:commandline}

The effect of redirection files can also be achieved by invoking
the \LaTeX{} compiler with a more elaborate command line.
Most conveniently this should be done as part
of a shell script or a |Makefile|.

When using \textsf{childdoc} in the main file, the following
command lines effectively perform a redirection
(note that depending on the shell being used,
backslashes may have to be doubled: `|\|' $\to$ `|\\|'):
%
\begin{center}
|... -jobname "|\textit{target}|" |\\|"|[\textit{flags}]%
|\input{childdoc.def}\childdocforward[|\textit{main}|]{|\textit{dest}|}"|
\end{center}
%
Here \textit{target} is the name of the output file,
\textit{main} is the name of the main file
and \textit{dest} is the name of the main or child file to be processed
(all filenames without extensions).
The optional argument \textit{main} can be omitted
if \textit{main} matches \textit{dest}.
Optionally, compilation \textit{flags} can be defined via |\def| commands.
This command line makes the \TeX{} engine believe
it is compiling the file \textit{target}
whose content is specified as the latter parameter.
The provided code then forwards the processing to
\textit{main} or \textit{dest} as described in \secref{sec:forward}.

%%%%%%%%%%%%%%%%%%%%%%%%%%%%%%%%%%%%%%%%%%%%%%%%%%%%%%%%%%%%%%%%%%%%%%%%%%%%%%%%
\subsection{Include by Input}
\label{sec:input}

Including child documents by |\include| has some restrictions by design.
Most notably, the content of a child document always occupies
its own set of pages; pages cannot be shared between child documents.
Usually, this behaviour makes perfect sense
because each child document contain an essential part of the document.
However, in some situations it may be desirable to compose
a document from a collection of parts
without having mandatory page breaks between then.
For this case, the package
provides a mechanism to include parts
by |\input| which can also be processed individually.
However, by construction this mechanism
requires manual handling of the content to be output.

%%%%%%%%%%%%%%%%%%%%%%%%%%%%%%%%%%%%%%%%
\DescribeMacro{\ifchilddocmanual}
The main file should be prepared as usual, see \secref{sec:include}.
However, the document body must make a distinction
between processing of an individual part and of the main document, e.g.:
%
\begin{center}
\begin{tabular}{l}
|\ifchilddocmanual|\\
|\input{\childdocname}|\\
|\||else|\\
\textit{document body with }|\input{|\textit{part}|}|\\
|\||fi|
\end{tabular}
\end{center}
%
The conditional |\ifchilddocmanual| is true whenever
a part to be included by |\input| is being compiled,
and the name of the part is stored in |\childdocname|.

%%%%%%%%%%%%%%%%%%%%%%%%%%%%%%%%%%%%%%%%
\DescribeMacro{\childdocby}
Each part to be included by |\input| should start with:
%
\begin{center}
\begin{tabular}{l}
|\input{childdoc.def}|\\
|\childdocby{|\textit{main}|}|\\
\end{tabular}
\end{center}
%
The directive |\childdocby| is similar to |\childdocof|
described in \secref{sec:include},
but the subsequent selection of content must be done manually.
To that end, both |\ifchilddoc| and |\ifchilddocmanual|
will be true upon processing of a part,
and the name of the part is stored in |\childdocname|.
Note that |\jobname| will be set to the filename of the current part
so that each part receives an individual |.aux| file
that does not interfere with the |.aux| file(s) of the main document.
This behaviour can be altered by the alternative form
|\childdocby[*]{|\textit{main}|}| (with a non-empty optional argument)
which uses the |.aux| file of the main document
by setting |\jobname| to \textit{main}.

%%%%%%%%%%%%%%%%%%%%%%%%%%%%%%%%%%%%%%%%%%%%%%%%%%%%%%%%%%%%%%%%%%%%%%%%%%%%%%%%
\subsection{Driver Development}
\label{sec:driver}

The \textsf{childdoc} mechanism can also be use for the development
of definition files such as \LaTeX{} styles or classes.
This case differs from the above setup with multiple parts
included by |\include| in that no |\includeonly| should be invoked.
This can be achieved by starting the include file
(before |\ProvidesPackage|) with:
%
\begin{center}
\begin{tabular}{l}
|\input{childdoc.def}|\\
|\childdocforward{|\textit{main}|}|\\
\end{tabular}
\end{center}
%
or alternatively with:
%
\begin{center}
\begin{tabular}{l}
|\input{childdoc.def}|\\
|\childdocby{|\textit{main}|}|\\
\end{tabular}
\end{center}
%
Both forms have slightly different effects as described above.
The main file is prepared as usual, see \secref{sec:include}.

%%%%%%%%%%%%%%%%%%%%%%%%%%%%%%%%%%%%%%%%%%%%%%%%%%%%%%%%%%%%%%%%%%%%%%%%%%%%%%%%
\subsection{Legacy Detection}
\label{sec:detection}

The directive |\childdocmain| in the main file can detect
whether the complete document or merely a child is to be compiled
even without using the directive |\childdocof|.
This method is deprecated because it is less robust
and there is no compelling reason to use it;
it is merely provided for backward compatibility
and it may be removed in future versions.

If the detection mechanism is to be used,
it is mandatory to correctly specify
the filename of the main file as the argument of |\childdocmain|:
%
\begin{center}
\begin{tabular}{l}
|\input{childdoc.def}|\\
|\childdocmain{|\textit{main}|}|\\
\end{tabular}
\end{center}
%
If |\jobname| does not match the argument \textit{main} of |\childdocmain|,
it is assumed that |\jobname| points to the child file to be compiled.
When using |\childdocmain| with the main file specified as argument,
it suffices to start a child file
with just |\input{|\textit{main}|}|
without loading of the package and using |\childdocof|.
If instead all processing is done
with the appropriate \textsf{childdoc} directives,
the argument of \textit{main} of |\childdocmain| can be empty.

An alternative version of the command line processing described
in \secref{sec:commandline} using the detection mechanism reads:
%
\begin{center}
|... -jobname "|\textit{target}|" "|[\textit{flags}]%
[|\def\jobname{|\textit{dest}|}|]|\input{|\textit{main}|}"|
\end{center}

%%%%%%%%%%%%%%%%%%%%%%%%%%%%%%%%%%%%%%%%%%%%%%%%%%%%%%%%%%%%%%%%%%%%%%%%%%%%%%%%
\subsection{Manual Code}
\label{sec:manual}

In case one cannot be certain whether the definitions file |childdoc.def|
is installed on the target \TeX{} distribution
and one prefers not to ship it,
it is conceivable to paste a few relevant commands into the sources.

To that end, drop all statements |\input{childdoc.def}|
and perform the replacements as outlined below.
Instead of |\childdocmain{|\textit{main}|}| add the following code
to the top of the main file:
%
\begin{center}
\begin{tabular}{l}
|\||ifdefined\childdocname\endinput\||fi\newif\ifchilddoc|\\
|\edef\childdocname{\scantokens\expandafter{\jobname\noexpand}}|\\
|\def\childdocmain{|\textit{main}|}\||ifx\childdocmain\childdocname\||else|\\
|\childdoctrue\includeonly{\childdocname}\let\jobname\childdocmain\||fi|\\
\end{tabular}
\end{center}
%
Instead of |\childdocof{|\textit{main}|}| just include the main file
at the top of each child file:
%
\begin{center}
|\input{|\textit{main}|}|
\end{center}
%
A simple redirection |\childdocforward{|\textit{dest}|}| is achieved by:
%
\begin{center}
|\def\jobname{|\textit{dest}|}\input{\jobname}|
\end{center}
%
The redirection with prefix
|\childdocforwardprefix[|\textit{prefix}|]{|\textit{dest}|}|
is accomplished by:
%
\begin{center}
\begin{tabular}{l}
|{\edef\jobname{\scantokens\expandafter{\jobname\noexpand}}|\\
|\def\redirectjob |\textit{prefix}|#1~~~{\gdef\jobname{|\textit{dest}|#1}}|\\
|\expandafter\redirectjob\jobname~~~}\input{\jobname}|
\end{tabular}
\end{center}

In an alternative approach,
child documents can be compiled by a specific command line
without additional code or specific definitions:
%
\begin{center}
|... -jobname "|\textit{target}|" "|[\textit{flags}]%
|\includeonly{|\textit{dest}|}\input{|\textit{main}|}"|
\end{center}
%

%%%%%%%%%%%%%%%%%%%%%%%%%%%%%%%%%%%%%%%%%%%%%%%%%%%%%%%%%%%%%%%%%%%%%%%%%%%%%%%%
%%%%%%%%%%%%%%%%%%%%%%%%%%%%%%%%%%%%%%%%%%%%%%%%%%%%%%%%%%%%%%%%%%%%%%%%%%%%%%%%
\section{Information}

%%%%%%%%%%%%%%%%%%%%%%%%%%%%%%%%%%%%%%%%%%%%%%%%%%%%%%%%%%%%%%%%%%%%%%%%%%%%%%%%
\subsection{Copyright}

Copyright \copyright{} 2017--2018 Niklas Beisert

This work may be distributed and/or modified under the
conditions of the \LaTeX{} Project Public License, either version 1.3
of this license or (at your option) any later version.
The latest version of this license is in
  \url{http://www.latex-project.org/lppl.txt}
and version 1.3 or later is part of all distributions of \LaTeX{}
version 2005/12/01 or later.

This work has the LPPL maintenance status `maintained'.

The Current Maintainer of this work is Niklas Beisert.

This work consists of the files |README.txt|, |childdoc.ins| and |childdoc.dtx|
as well as the derived files |childdoc.def|, |cdocsamp.tex|
with |cdocsch1.tex|, |cdocsch2.tex|, |cdocspt3.tex|, |cdocspt4.tex|,
|cdocsdrf.tex|, |cdocsfn1.tex|, |cdocsfn2.tex|
as well as |childdoc.pdf|.

%%%%%%%%%%%%%%%%%%%%%%%%%%%%%%%%%%%%%%%%%%%%%%%%%%%%%%%%%%%%%%%%%%%%%%%%%%%%%%%%
\subsection{Files and Installation}

The package consists of the files:
%
\begin{center}
\begin{tabular}{ll}
    |README.txt|   & readme file \\
    |childdoc.ins| & installation file \\
    |childdoc.dtx| & source file \\
    |childdoc.def| & definition file \\
    |cdocsamp.tex| & sample main file \\
    |cdocsch1.tex| & sample include file \\
    |cdocsch2.tex| & sample include file \\
    |cdocspt3.tex| & sample part file \\
    |cdocspt4.tex| & sample part file \\
    |cdocsdrf.tex| & sample redirection file \\
    |cdocsfn1.tex| & sample redirection file \\
    |cdocsfn2.tex| & sample redirection file \\
    |childdoc.pdf| & manual
\end{tabular}
\end{center}
%
The distribution consists of the files
|README.txt|, |childdoc.ins| and |childdoc.dtx|.
%
\begin{itemize}
\item
Run (pdf)\LaTeX{} on |childdoc.dtx|
to compile the manual |childdoc.pdf| (this file).
\item
Run \LaTeX{} on |childdoc.ins| to create the definitions file |childdoc.def|
and the sample |cdocsamp.tex| with include files
|cdocsch1.tex|, |cdocsch2.tex|, |cdocspt3.tex|, |cdocspt4.tex|,
|cdocsdrf.tex|, |cdocsfn1.tex|, |cdocsfn2.tex|.
Then copy the file |childdoc.def| to an appropriate directory of your \LaTeX{}
distribution, e.g.\ \textit{texmf-root}|/tex/latex/childdoc|.
\end{itemize}

%%%%%%%%%%%%%%%%%%%%%%%%%%%%%%%%%%%%%%%%%%%%%%%%%%%%%%%%%%%%%%%%%%%%%%%%%%%%%%%%
\subsection{Related CTAN Packages}

There are several other packages which offer a similar functionality:
%
\begin{itemize}
\item
The packages
\href{http://ctan.org/pkg/docmute}{\textsf{docmute}},
\href{http://ctan.org/pkg/includex}{\textsf{includex}} and
\href{http://ctan.org/pkg/standalone}{\textsf{standalone}}
provide commands to include only the document body of
a child file thus allowing both files to be compiled individually.
\item
The packages \href{http://ctan.org/pkg/subdocs}{\textsf{subdocs}}
and \href{http://ctan.org/pkg/subfiles}{\textsf{subfiles}}
provide structures in which the main and child documents can be
encapsulated and allowing them to be compiled individually.
The inclusion mechanism is different from the conventional |\include|.
\item
The package \href{http://ctan.org/pkg/combine}{\textsf{combine}}
is an elaborate solution to combine several documents into one.
\end{itemize}
%
See also the CTAN topic \href{http://ctan.org/topic/subdocs}{\textsf{subdocs}}
for further related packages.
The present package differs from the above solutions in that
a document structure constructed with the conventional |\include| mechanism
just needs two extra commands at the top of every file
such that all constituent files can be compiled individually.

%%%%%%%%%%%%%%%%%%%%%%%%%%%%%%%%%%%%%%%%%%%%%%%%%%%%%%%%%%%%%%%%%%%%%%%%%%%%%%%%
%\subsection{Feature Suggestions}
%
%The following is a list of features which may be useful for future
%versions of this package:
%%
%\begin{itemize}
%\item
%\ldots
%\end{itemize}

%%%%%%%%%%%%%%%%%%%%%%%%%%%%%%%%%%%%%%%%%%%%%%%%%%%%%%%%%%%%%%%%%%%%%%%%%%%%%%%%
\subsection{Revision History}

%%%%%%%%%%%%%%%%%%%%%%%%%%%%%%%%%%%%%%%%
\paragraph{v2.0:} 2018/12/30

\begin{itemize}
\item
immediate forward processing
\item
added |\childdocby| mechanism
\item
manual restructured
\end{itemize}

%%%%%%%%%%%%%%%%%%%%%%%%%%%%%%%%%%%%%%%%
\paragraph{v1.6:} 2018/01/17

\begin{itemize}
\item
application for development of include files
\item
corrections to manual
\end{itemize}

%%%%%%%%%%%%%%%%%%%%%%%%%%%%%%%%%%%%%%%%
\paragraph{v1.5:} 2017/05/21

\begin{itemize}
\item
more complete structuring introduced
\item
|\childdocof| introduced
\item
|\childdoc| renamed to |\childdocmain|
\item
|\childredirect| renamed to |\childdocforward| and |\childdocforwardprefix|
and functionality expanded
\end{itemize}

%%%%%%%%%%%%%%%%%%%%%%%%%%%%%%%%%%%%%%%%
\paragraph{v1.0:} 2017/04/27

\begin{itemize}
\item
manual and install package
\item
first version published on CTAN
\end{itemize}

%%%%%%%%%%%%%%%%%%%%%%%%%%%%%%%%%%%%%%%%
\paragraph{v0.6:} 2017/04/26

\begin{itemize}
\item
redirection mechanism added
\end{itemize}

%%%%%%%%%%%%%%%%%%%%%%%%%%%%%%%%%%%%%%%%
\paragraph{v0.5:} 2017/04/26

\begin{itemize}
\item
functionality in definition file
\end{itemize}


%%%%%%%%%%%%%%%%%%%%%%%%%%%%%%%%%%%%%%%%%%%%%%%%%%%%%%%%%%%%%%%%%%%%%%%%%%%%%%%%
%%%%%%%%%%%%%%%%%%%%%%%%%%%%%%%%%%%%%%%%%%%%%%%%%%%%%%%%%%%%%%%%%%%%%%%%%%%%%%%%
%%%%%%%%%%%%%%%%%%%%%%%%%%%%%%%%%%%%%%%%%%%%%%%%%%%%%%%%%%%%%%%%%%%%%%%%%%%%%%%%
\appendix

\settowidth\MacroIndent{\rmfamily\scriptsize 000\ }

 \DocInput{childdoc.dtx}

\end{document}
%</driver>
% \fi
%
% %%%%%%%%%%%%%%%%%%%%%%%%%%%%%%%%%%%%%%%%%%%%%%%%%%%%%%%%%%%%%%%%%%%%%%%%%%%%%%
% %%%%%%%%%%%%%%%%%%%%%%%%%%%%%%%%%%%%%%%%%%%%%%%%%%%%%%%%%%%%%%%%%%%%%%%%%%%%%%
% \section{Sample}
%\iffalse
%<*samplemain>
%\fi
%
% The following presents a sample document
% with two chapters, two parts, a title page,
% a compile flag as well as three forwarding files to set the flag.
% It consists of eight |.tex| files:
% \begin{center}
% \begin{tabular}{ll}
% |cdocsamp.tex|&main file\\
% |cdocsch1.tex|&include file for chapter 1\\
% |cdocsch2.tex|&include file for chapter 2\\
% |cdocspt3.tex|&include file for part 3\\
% |cdocspt4.tex|&include file for part 4\\
% |cdocsdrf.tex|&forwarding file for main file in draft mode\\
% |cdocsfi1.tex|&forwarding file for final version of chapter 1\\
% |cdocsfi2.tex|&forwarding file for final version of chapter 2\\
% \end{tabular}
% \end{center}
% Each of the eight files can be compiled directly by the \LaTeX{} compiler.
%
% %%%%%%%%%%%%%%%%%%%%%%%%%%%%%%%%%%%%%%
% \paragraph{Main File.}
%
% The main file is called |cdocsamp.tex|.
%
% Load the \textsf{childdoc} definitions and
% declare the filename for the main document:
%    \begin{macrocode}
\input{childdoc.def}
\childdocmain{}
%    \end{macrocode}

% Optional override for |\version| flag:
%    \begin{macrocode}
%%\ifchilddoc\else\providecommand{\version}{draft}\fi
%    \end{macrocode}

% Define the default values for the |\version| flag
% (|final| for the main file and |draft| for childs):
%    \begin{macrocode}
\ifchilddoc
\providecommand{\version}{draft}
\else
\providecommand{\version}{final}
\fi
%    \end{macrocode}

% Load the standard document class:
%    \begin{macrocode}
\documentclass[12pt]{article}
%    \end{macrocode}

% Start the document body:
%    \begin{macrocode}
\begin{document}
%    \end{macrocode}

% Declare a title page.
% Print title, part of document being processed and version flag:
%    \begin{macrocode}
\addtocounter{page}{-1}
\begin{center}
{\LARGE\bfseries{}childdoc example\par}
\vspace{1cm}
\ifchilddoc
\ifchilddocmanual part\else chapter\fi:
`\childdocname' of `\childdocjob'\par
\else
main document: `\childdocjob'\par
\fi
version: \version\par
\end{center}
\newpage
%    \end{macrocode}

% Manually include selected file,
% otherwise process as usual:
%    \begin{macrocode}
\ifchilddocmanual
\section*{part `\childdocname'}
\input{\childdocname}
\else
%    \end{macrocode}

% Include the two chapters:
%    \begin{macrocode}
\include{cdocsch1}
\include{cdocsch2}
%    \end{macrocode}

% Include the two parts unless only chapters should be displayed:
%    \begin{macrocode}
\ifchilddoc\else
\section{part three}
\input{cdocspt3}
\section{part four}
\input{cdocspt4}
\fi
%    \end{macrocode}

% Process as usual until here:
%    \begin{macrocode}
\fi
%    \end{macrocode}

% End of document body:
%    \begin{macrocode}
\end{document}
%    \end{macrocode}
%\iffalse
%</samplemain>
%\fi
%
% %%%%%%%%%%%%%%%%%%%%%%%%%%%%%%%%%%%%%%
% \paragraph{Chapter Include Files.}
%
% The include files are called |cdocsch1.tex| and |cdocsch2.tex|.
%
%\iffalse
%<*samplechap1|samplechap2>
%\fi

% Optional override for |\version| flag:
%    \begin{macrocode}
%%\providecommand{\version}{final}
%    \end{macrocode}

% Include the main document:
%    \begin{macrocode}
\input{childdoc.def}
\childdocof{cdocsamp}
%    \end{macrocode}

%\iffalse
%</samplechap1|samplechap2>
%\fi
%
%\iffalse
%<*samplechap1>
%\fi
% Some text for chapter 1:
%    \begin{macrocode}
\section{one}
some text in chapter one
%    \end{macrocode}

%\iffalse
%</samplechap1>
%\fi
% Some text for chapter 2:
%\iffalse
%<*samplechap2>
%\fi
%    \begin{macrocode}
\section{two}
more text in chapter two
%    \end{macrocode}

%\iffalse
%</samplechap2>
%\fi
%
% %%%%%%%%%%%%%%%%%%%%%%%%%%%%%%%%%%%%%%
% \paragraph{Part Include Files.}
%
% The include files are called |cdocspt3.tex| and |cdocspt4.tex|.
%
%\iffalse
%<*samplepart3|samplepart4>
%\fi

% Optional override for |\version| flag:
%    \begin{macrocode}
%%\providecommand{\version}{final}
%    \end{macrocode}

% Include the main document:
%    \begin{macrocode}
\input{childdoc.def}
\childdocby{cdocsamp}
%    \end{macrocode}

%\iffalse
%</samplepart3|samplepart4>
%\fi
%
%\iffalse
%<*samplepart3>
%\fi
% Some text for part 3:
%    \begin{macrocode}
some text in part three
%    \end{macrocode}

%\iffalse
%</samplepart3>
%\fi
% Some text for part 4:
%\iffalse
%<*samplepart4>
%\fi
%    \begin{macrocode}
more text in part four
%    \end{macrocode}

%\iffalse
%</samplepart4>
%\fi
%
% %%%%%%%%%%%%%%%%%%%%%%%%%%%%%%%%%%%%%%
% \paragraph{Forwarding for a Complete Draft.}
%
% The following forwarding file |cdocsdrf.tex|
% compiles the main document in draft mode:
%\iffalse
%<*sampledraft>
%\fi
%    \begin{macrocode}
\def\version{draft}
\input{childdoc.def}
\childdocforward{cdocsamp}
%    \end{macrocode}

%\iffalse
%</sampledraft>
%\fi
%
% %%%%%%%%%%%%%%%%%%%%%%%%%%%%%%%%%%%%%%
% \paragraph{Forwarding for Final Version of the Chapters.}
%
% The following forwarding files |cdocsfn1.tex| and |cdocsfn2.tex|
% (with identical content)
% compile the final versions of the child documents
% |cdocsch1.tex| and |cdocsch2.tex|, respectively:
%\iffalse
%<*samplefinal>
%\fi
%    \begin{macrocode}
\def\version{final}
\input{childdoc.def}
\childdocforwardprefix[cdocsamp]{cdocsfn}{cdocsch}
%    \end{macrocode}

%\iffalse
%</samplefinal>
%\fi
%
% %%%%%%%%%%%%%%%%%%%%%%%%%%%%%%%%%%%%%%
% \paragraph{Command Line Processing.}
%
% The following three command lines generate the output files
% |cdocscld|, |cdocscl1| and |cdocscl2|
% which should be identical to
% |cdocsdrf|, |cdocsch1| and |cdocsfn2|, respectively:
% \begin{center}
% \begin{tabular}{l}
% |latex -jobname cdocscld \|\\
% |  "\def\version{draft}\input{childdoc.def}\childdocforward{cdocsamp}"|\\
% |latex -jobname cdocscl1 \|\\
% |  "\input{childdoc.def}\childdocforward[cdocsamp]{cdocsch1}"|\\
% |latex -jobname cdocscl2 \|\\
% |  "\def\version{final}\input{childdoc.def}\childdocforward{cdocsch2}"|
% \end{tabular}
% \end{center}
% Note that the trailing backslash on each first line
% merely continues the input to the second line
% (for convenient cut ant paste).
% Furthermore, the command |latex| can be replaced by any
% of its alternative versions such as |pdflatex|.
%
% %%%%%%%%%%%%%%%%%%%%%%%%%%%%%%%%%%%%%%%%%%%%%%%%%%%%%%%%%%%%%%%%%%%%%%%%%%%%%%
% %%%%%%%%%%%%%%%%%%%%%%%%%%%%%%%%%%%%%%%%%%%%%%%%%%%%%%%%%%%%%%%%%%%%%%%%%%%%%%
% \section{Implementation}
%\iffalse
%<*package>
%\fi
%
% This section describes the definitions file |childdoc.def|.

% The definitions cannot be loaded using |\usepackage| or |\RequirePackage|
% which has a mechanism to prevent loading a style file more than once.
% When loading the definitions by means of |\input|
% multiple instances have to be prevented manually:
%\iffalse
%This code needs to be before the `\ProvidesFile' directive
%which is defined at the beginning of this file.
%Therefore it is also placed there and commented out here.
%</package>
%<*discard>
%\fi
%    \begin{macrocode}
\ifdefined\childdocmain\endinput\fi
%    \end{macrocode}
%\iffalse
%</discard>
%<*package>
%\fi
%
% \macro{\ifchilddoc}
% \macro{\ifchilddocmanual}
% The conditional |\ifchilddoc| tells whether a
% child (true) or main (false) document is being compiled.
% The conditional |\ifchilddocmanual| tells whether
% the |\includeonly| mechanism is used (false) or
% the selection of child files must be performed manually (true).
% The definitions initialise to false:
%    \begin{macrocode}
\newif\ifchilddoc
\newif\ifchilddocmanual
%    \end{macrocode}

% \macro{\childdocname}
% \macro{\childdocjob}
% The macro |\childdocname| stores the name of the main document
% to be compiled. The macro |\childdocjob| stores the name of
% the document on which the \LaTeX{} compiler was originally invoked.
% The content of |\jobname| cannot be compared
% to filenames specified in the source due to different catcodes.
% The following code rescans |\jobname|, stores the result
% in |\childdocname| and saves a copy in |\childdocjob|:
%    \begin{macrocode}
\edef\childdocname{\scantokens\expandafter{\jobname\noexpand}}
\let\childdocjob\childdocname
%    \end{macrocode}

% \macro{\childdocdisable}
% The macro |\childdocdisable| prevents the main file
% from being processed more than once.
% At this stage, the main document command |\childdocmain|
% is assumed to be called once again where it should do nothing.
% Any subsequent call to it should prevent
% a secondary processing of the main document
% It overwrites the forwarding commands
% |\childdocof| and |\childdocforward|
% with empty macros to prevent further inclusions of the main document:
%    \begin{macrocode}
\newcommand{\childdocdisable}
{
  \renewcommand{\childdocmain}[1]{\renewcommand{\childdocmain}[1]{\endinput}}
  \renewcommand{\childdocof}[1]{}
  \renewcommand{\childdocby}[2][]{}
  \renewcommand{\childdocforward}[2][]{}
  \renewcommand{\childdocdisable}{}
}
%    \end{macrocode}

% \macro{\childdocmain}
% The macro |\childdocmain| is to be called at the top of the main file
% with nothing or the main filename (without extension) as argument.
% First, it breaks loops.
% If the argument is not empty and does not match |\childdocname|
% (which is set by the first inclusion of |childdoc.def|),
% |\ifchilddoc| is set to true, |\includeonly| is applied to the child file
% and |\jobname| is set to the main file
% (for proper handling of |.aux| files):
%    \begin{macrocode}
\newcommand{\childdocmain}[1]
{
  \childdocdisable\childdocmain{}
  \if?#1?\else
    \begingroup
      \def\childdoctmp{#1}
      \ifx\childdoctmp\childdocname
        \def\childdoctmp{}
      \else
        \def\childdoctmp
        {
          \childdoctrue
          \includeonly{\childdocname}
          \def\childdocjob{#1}
          \def\jobname{#1}
        }
      \fi
      \expandafter
    \endgroup
    \childdoctmp
  \fi
}
%    \end{macrocode}

% \macro{\childdocof}
% The command |\childdocof| redirects
% compilation to the main file |#1|.
%    \begin{macrocode}
\newcommand{\childdocof}[1]
{
  \childdocdisable
  \childdoctrue
  \includeonly{\childdocname}
  \def\jobname{#1}
  \def\childdocjob{#1}
  \input{#1}
}
%    \end{macrocode}

% \macro{\childdocby}
% The command |\childdocby| ....
%    \begin{macrocode}
\newcommand{\childdocby}[2][]
{
  \childdocdisable
  \childdoctrue
  \childdocmanualtrue
  \if?#1?\else
    \def\jobname{#2}
  \fi
  \def\childdocjob{#2}
  \input{#2}
  \endinput
}
%    \end{macrocode}

% \macro{\childdocforward}
% The command |\childdocforward| redirects
% compilation to the main file or
% (if the optional argument is given) a child file.
% Parameters are set as if the main file
% or a child file starting with |\childdocof| was compiled.
% Then compilation is handed over to the main file:
%    \begin{macrocode}
\newcommand{\childdocforward}[2][]
{
  \begingroup
    \if?#1?
      \def\childdoctmp
      {
        \def\childdocname{#2}
        \def\childdocjob{#2}
        \def\jobname{#2}
        \input{#2}
        \endinput
      }
    \else
      \def\childdoctmp
      {
        \childdocdisable
        \def\childdocname{#2}
        \childdoctrue
        \includeonly{#2}
        \def\childdocjob{#1}
        \def\jobname{#1}
        \input{#1}
        \endinput
      }
    \fi
    \expandafter
  \endgroup
  \childdoctmp
}
%    \end{macrocode}

% \macro{\childdocforwardprefix}
% The command |\childdocforwardprefix| redirects
% compilation to the main or a child file by means of a pattern.
% The prefix |#1| in the current filename is replaced by |#2|
% and the suffix of the current filename is kept
% (it is assumed that the filename does not contain the substring `|~~~|'
% which is used as a delimiter).
% Compilation is handed over to the new file by |\childdocforward|:
%    \begin{macrocode}
\newcommand{\childdocforwardprefix}[3][]
{
  \begingroup
    \def\childdocextract #2##1~~~{\def\childdoctmp{\childdocforward[#1]{#3##1}}}
    \expandafter\childdocextract\childdocname~~~
    \expandafter
  \endgroup
  \childdoctmp
}
%    \end{macrocode}

% \macro{\childdoc}
% The deprecated macro |\childdoc| is a legacy version of |\childdocmain|:
%    \begin{macrocode}
\newcommand{\childdoc}{\childdocmain}
%    \end{macrocode}

% \macro{\childdocredirect}
% The deprecated macro |\childdocredirect| is a legacy version
% of |\childdocforward| and |\childdocforwardprefix|:
%    \begin{macrocode}
\newcommand{\childdocredirect}[2][]
{
  \begingroup
    \if?#1?
      \def\childdoctmp{\childdocforward{#2}}
    \else
      \def\childdoctmp{\childdocforwardprefix{#1}{#2}}
    \fi
    \expandafter
  \endgroup
  \childdoctmp
}
%    \end{macrocode}

%\iffalse
%</package>
%\fi
%
\endinput
|\\
|\childdocof{|\textit{main}|}|\\
\end{tabular}
\end{center}
at the top of every child file \textit{child}
which is included by |\include{|\textit{child}|}|
from within the main file
(or at least for those files to be compiled individually).
The argument \textit{main} must be the filename of the main file.

There are a couple of
considerations in setting up the main and child documents:

%%%%%%%%%%%%%%%%%%%%%%%%%%%%%%%%%%%%%%%%
\paragraph{Restrictions.}

Please note the following restrictions:
\begin{itemize}
\item
|\childdocmain| must be called with one argument \textit{main}
to ensure compatibility with earlier version of the package.
It must either be empty (|\childdocmain{}|)
or precisely match the filename of the main file in which it is specified.
See \secref{sec:detection} for further information.
\item
The filename \textit{main} must be specified without the |.tex| extension.
\item
The filename \textit{main} is case sensitive
(even in case-insensitive file systems)
due to internal string comparison.
\item
The argument \textit{main} should be fully expanded, it cannot be a macro.
\item
Subdirectories and special characters should be avoided in filenames.
\item
The command |\childdocmain{|\textit{main}|}| must be followed by a whitespace.
It should not be followed immediately by another command
or by a comment mark `|%|'.
This is because the \TeX{} parser reads the token immediately following
the argument of |\childdocmain| and puts it
at the beginning of every child section;
however, a white\-space is ignored.
\end{itemize}

%%%%%%%%%%%%%%%%%%%%%%%%%%%%%%%%%%%%%%%%
\paragraph{Content of Main File.}

It is advisable to place all content in the child files included by |\include|.
Any output contained in the main file will appear in all child documents
unless suppressed manually;
it cannot be suppressed automatically by the |\includeonly| directive
and thus should normally be avoided.
A method to include some content in the main file
by means of conditional processing is described in \secref{sec:conditional}.

%%%%%%%%%%%%%%%%%%%%%%%%%%%%%%%%%%%%%%%%
\paragraph{Page Numbering.}

When only a part of the document is compiled,
the appropriate numbering of pages
(as well as other status parameters)
is determined from the |.aux| files.
The latter contain information from previous passes.
However this information needs to propagate through
all intermediate child documents.
Therefore the page numbering in child documents may well
be inconsistent until the complete document is compiled at least once.

A useful (if unconventional) way to always ensure a consistent
page numbering is to restart the numbering in each child document
and denote the pages by `\textit{child}|.|\textit{page}'
where \textit{child} represents the chapter/section number of the child file.
This can be achieved by the command
|\numberwithin{page}{|\textit{child}|}|
of the \textsf{amsmath} package
where \textit{child} can be |chapter| or |section|
depending on the chosen structuring.
Alternatively, one can modify the macro |\thepage| appropriately
and reset the counter |page| at the start of each child file.

%%%%%%%%%%%%%%%%%%%%%%%%%%%%%%%%%%%%%%%%%%%%%%%%%%%%%%%%%%%%%%%%%%%%%%%%%%%%%%%%
\subsection{Conditional Processing}
\label{sec:conditional}

The package provides a mechanism to compile different versions
of a document. To customise the versions further some conditional processing
can come in handy to distinguish which version is being compiled.
The package provides two macros to describe the compilation context:

%%%%%%%%%%%%%%%%%%%%%%%%%%%%%%%%%%%%%%%%
\DescribeMacro{\ifchilddoc}
The conditional |\ifchilddoc| distinguishes between the compilation of
child documents and the main document:
%
\begin{center}
|\ifchilddoc |\textit{child-code}| |[|\||else |\textit{main-code}]| \||fi|
\end{center}

%%%%%%%%%%%%%%%%%%%%%%%%%%%%%%%%%%%%%%%%
\DescribeMacro{\childdocname}
\DescribeMacro{\childdocjob}
The macro |\childdocname| contains the filename (without extension)
of the main or child file being processed.
Note that |\childdocjob| will always contain the name of the main file.

%%%%%%%%%%%%%%%%%%%%%%%%%%%%%%%%%%%%%%%%
\paragraph{Title Page.}

Conditional processing can be used to include a title or banner page
in the main document when proper precautions are taken.
Importantly, the code in the main file should ensure that the page counter
(as well as other status parameters which are stored in the |.aux| files)
takes the same value after the conditional processing.
Otherwise the page numbers may take divergent values
depending on which part is compiled.

For example, a title page could be declared by:
%
\begin{center}
\begin{tabular}{l}
|\ifchilddoc\||else|\\
|\addtocounter{page}{-1}|\\
\textit{code for title page}\\
|\newpage|\\
|\||fi|
\end{tabular}
\end{center}
%
A banner page for the child documents can be generated by:
%
\begin{center}
\begin{tabular}{l}
|\ifchilddoc|\\
|\addtocounter{page}{-1}|\\
\textit{code for banner page}\\
|\newpage|\\
|\||fi|
\end{tabular}
\end{center}
%
Here one could write a message such as:
\begin{center}
|This is the part \childdocname{} of \childdocjob{}.|
\end{center}

%%%%%%%%%%%%%%%%%%%%%%%%%%%%%%%%%%%%%%%%%%%%%%%%%%%%%%%%%%%%%%%%%%%%%%%%%%%%%%%%
\subsection{Flags}
\label{sec:flags}

The package makes it easy to generate different versions
of the main or child documents.
To this end compilation flags can be defined
and assigned different default values.
They will be particularly useful in conjunction
with the forwarding mechanism described in \secref{sec:forward}.

For example, it may be useful to have a flag |\version|
which can be set to |draft| or |final|.
The document source will contain some conditional code
depending on the value of |\version|.
Suppose further, the flag should default to |final| for the main file
and to |draft| for child files
which is a natural assignment for editing the document.
This is achieved by placing the following code
in the preamble of the main document
(below the |\childdocmain| directive):
%
\begin{center}
\begin{tabular}{l}
|\ifchilddoc|\\
|\providecommand{\version}{draft}|\\
|\||else|\\
|\providecommand{\version}{final}|\\
|\||fi|
\end{tabular}
\end{center}
%
The definition by |\providecommand| makes sure
that previous definitions are not overwritten.
Further statements |\providecommand{\version}{...}|
can thus be added before the above code to override it.

For the main file, one might add a line
(between |\childdocmain| and the above block)
%
\begin{center}
|%\ifchilddoc\||else\providecommand{\version}{draft}\||fi|
\end{center}
%
which can be uncommented to produce a draft version.
Likewise one can add a line to the very top of a child file
(above the |\childdocof{|\textit{main}|}| directive)
%
\begin{center}
|%\providecommand{\version}{final}|
\end{center}
%
which can be uncommented to produce the final version of this child document.

%%%%%%%%%%%%%%%%%%%%%%%%%%%%%%%%%%%%%%%%%%%%%%%%%%%%%%%%%%%%%%%%%%%%%%%%%%%%%%%%
\subsection{Forwarding}
\label{sec:forward}

Different versions of the main or child documents
using compilation flags as described in \secref{sec:flags}
can be (permanently) stored in different files
for convenient compilation, viewing and distribution.
To this end, the package defines a command
to pass on compilation to a different file:

%%%%%%%%%%%%%%%%%%%%%%%%%%%%%%%%%%%%%%%%
\DescribeMacro{\childdocforward}
The command |\childdocforward| redirects processing to
another source file:
%
\begin{center}
\begin{tabular}{l}
|% \iffalse
%
% childdoc.dtx Copyright (C) 2017-2018 Niklas Beisert
%
% This work may be distributed and/or modified under the
% conditions of the LaTeX Project Public License, either version 1.3
% of this license or (at your option) any later version.
% The latest version of this license is in
%   http://www.latex-project.org/lppl.txt
% and version 1.3 or later is part of all distributions of LaTeX
% version 2005/12/01 or later.
%
% This work has the LPPL maintenance status `maintained'.
%
% The Current Maintainer of this work is Niklas Beisert.
%
% This work consists of the files childdoc.dtx and childdoc.ins
% and the derived files childdoc.def and cdocsamp.tex with
% cdocsch1.tex, cdocsch2.tex, cdocsdrf.tex, cdocsfn1.tex, cdocsfn2.tex.
%
%<package>\ifdefined\childdocmain\endinput\fi
%<package>\ProvidesFile{childdoc.def}[2018/12/30 v2.0 child document driver]
%<samplemain>\ProvidesFile{cdocsamp.tex}[2018/12/30 v2.0 sample for childdoc]
%<*driver>
%\ProvidesFile{childdoc.drv}[2018/12/30 v2.0 childdoc reference manual file]
\PassOptionsToClass{10pt,a4paper}{article}
\documentclass{ltxdoc}

\usepackage[margin=35mm]{geometry}
\usepackage{hyperref}
\usepackage{hyperxmp}
\usepackage[usenames]{color}

\hypersetup{colorlinks=true}
\hypersetup{pdfstartview=FitH}
\hypersetup{pdfpagemode=UseNone}
\hypersetup{pdfsource={}}
\hypersetup{pdflang={en-UK}}
\hypersetup{pdfcopyright={Copyright 2017-2018 Niklas Beisert.
  This work may be distributed and/or modified under the
  conditions of the LaTeX Project Public License, either version 1.3
  of this license or (at your option) any later version.}}
\hypersetup{pdflicenseurl={http://www.latex-project.org/lppl.txt}}
\hypersetup{pdfcontactaddress={ETH Zurich, ITP, HIT K,
  Wolfgang-Pauli-Strasse 27}}
\hypersetup{pdfcontactpostcode={8093}}
\hypersetup{pdfcontactcity={Zurich}}
\hypersetup{pdfcontactcountry={Switzerland}}
\hypersetup{pdfcontactemail={nbeisert@itp.phys.ethz.ch}}
\hypersetup{pdfcontacturl={http://people.phys.ethz.ch/\xmptilde nbeisert/}}

\newcommand{\secref}[1]{\hyperref[#1]{section \ref*{#1}}}

\parskip1ex
\parindent0pt
\let\olditemize\itemize
\def\itemize{\olditemize\parskip0pt}

\begin{document}

\title{The \textsf{childdoc} Package}
\hypersetup{pdftitle={The childdoc Package}}
\author{Niklas Beisert\\[2ex]
  Institut f\"ur Theoretische Physik\\
  Eidgen\"ossische Technische Hochschule Z\"urich\\
  Wolfgang-Pauli-Strasse 27, 8093 Z\"urich, Switzerland\\[1ex]
  \href{mailto:nbeisert@itp.phys.ethz.ch}
  {\texttt{nbeisert@itp.phys.ethz.ch}}}
\hypersetup{pdfauthor={Niklas Beisert}}
\hypersetup{pdfsubject={Manual for the LaTeX2e Package childdoc}}
\date{30 December 2018, \textsf{v2.0}}
\maketitle

\begin{abstract}\noindent
\textsf{childdoc} is a \LaTeXe{} package
that enables the direct compilation
of document sections included by |\include|
to individual files.
\end{abstract}

\begingroup
\parskip0ex
\tableofcontents
\endgroup

%%%%%%%%%%%%%%%%%%%%%%%%%%%%%%%%%%%%%%%%%%%%%%%%%%%%%%%%%%%%%%%%%%%%%%%%%%%%%%%%
%%%%%%%%%%%%%%%%%%%%%%%%%%%%%%%%%%%%%%%%%%%%%%%%%%%%%%%%%%%%%%%%%%%%%%%%%%%%%%%%
\section{Introduction}

\LaTeX{} provides a mechanism to structure a large document (such as a book)
into a main file and several child files (containing the chapters)
using the |\include| command.
This mechanism is beneficial for documents
which span hundreds of pages in order to
make the source file(s) more manageable.
Moreover, compilation can be restricted to
selected child files by means of the |\includeonly| command.
The latter feature can be used to reduce the compilation time while editing
(this was significantly more useful in the earlier days of \LaTeX{})
or to generate a smaller document which is easier to navigate.
Another application of |\includeonly| is to generate
documents consisting of selected parts of the complete document.

However, there are a few drawbacks of the plain |\include| mechanism:
\begin{itemize}
\item
The child files cannot be compiled on their own,
they can only be compiled via the main file.
A naive editing environment
(such as a text editor with an option
to have the current file processed by \LaTeX)
may require one to switch to the main file before compiling;
attempting to compile the child file produces errors.
\item
The main file must be modified (each time)
to adjust the |\includeonly| command
to the present needs. This easily leaves the main file in a messy state.
\item
The generated document will always carry the filename
of the main document. This is inconvenient if
several child files are to be compiled and
to be kept for distribution.
\end{itemize}

The present package provides a simple interface
to make child files individually compilable by \LaTeX{}.
Compiling a child file then has the same effect as compiling
the main file with an |\includeonly| command
to select the appropriate child.
Moreover the generated document will carry the name of the child
rather than the main file.
This resolves all three above issues.

This feature is meant to make the editing of books,
thesis documents and lecture notes somewhat more convenient.
However, the package can also be used efficiently for
composing a series of documents (such as exercise sheets)
which are typically distributed individually.
It then assists the author in generating the individual documents
(potentially in different versions)
as well as a document containing the collected series.
Another application is in developing style files
or other kinds of included material
where compilation of the style file could redirect
to a sample or test file.

%%%%%%%%%%%%%%%%%%%%%%%%%%%%%%%%%%%%%%%%%%%%%%%%%%%%%%%%%%%%%%%%%%%%%%%%%%%%%%%%
%%%%%%%%%%%%%%%%%%%%%%%%%%%%%%%%%%%%%%%%%%%%%%%%%%%%%%%%%%%%%%%%%%%%%%%%%%%%%%%%
\section{Usage}

First of all, the package \textsf{childdoc} is \emph{not} a standard
\LaTeXe{} |.sty| style file! Therefore it needs to be invoked in
a non-standard way.

%%%%%%%%%%%%%%%%%%%%%%%%%%%%%%%%%%%%%%%%%%%%%%%%%%%%%%%%%%%%%%%%%%%%%%%%%%%%%%%%
\subsection{Included Files}
\label{sec:include}

%%%%%%%%%%%%%%%%%%%%%%%%%%%%%%%%%%%%%%%%
\DescribeMacro{\childdocmain}
To use the package, add the commands
\begin{center}
\begin{tabular}{l}
|\input{childdoc.def}|\\
|\childdocmain{}|\\
\end{tabular}
\end{center}
at the very top of the main \LaTeX{} file,
in particular \emph{before} the |\documentclass| statement!
The argument of |\childdocmain| should be left empty
(but it must be present).

%%%%%%%%%%%%%%%%%%%%%%%%%%%%%%%%%%%%%%%%
\DescribeMacro{\childdocof}
Furthermore, add the commands
\begin{center}
\begin{tabular}{l}
|\input{childdoc.def}|\\
|\childdocof{|\textit{main}|}|\\
\end{tabular}
\end{center}
at the top of every child file \textit{child}
which is included by |\include{|\textit{child}|}|
from within the main file
(or at least for those files to be compiled individually).
The argument \textit{main} must be the filename of the main file.

There are a couple of
considerations in setting up the main and child documents:

%%%%%%%%%%%%%%%%%%%%%%%%%%%%%%%%%%%%%%%%
\paragraph{Restrictions.}

Please note the following restrictions:
\begin{itemize}
\item
|\childdocmain| must be called with one argument \textit{main}
to ensure compatibility with earlier version of the package.
It must either be empty (|\childdocmain{}|)
or precisely match the filename of the main file in which it is specified.
See \secref{sec:detection} for further information.
\item
The filename \textit{main} must be specified without the |.tex| extension.
\item
The filename \textit{main} is case sensitive
(even in case-insensitive file systems)
due to internal string comparison.
\item
The argument \textit{main} should be fully expanded, it cannot be a macro.
\item
Subdirectories and special characters should be avoided in filenames.
\item
The command |\childdocmain{|\textit{main}|}| must be followed by a whitespace.
It should not be followed immediately by another command
or by a comment mark `|%|'.
This is because the \TeX{} parser reads the token immediately following
the argument of |\childdocmain| and puts it
at the beginning of every child section;
however, a white\-space is ignored.
\end{itemize}

%%%%%%%%%%%%%%%%%%%%%%%%%%%%%%%%%%%%%%%%
\paragraph{Content of Main File.}

It is advisable to place all content in the child files included by |\include|.
Any output contained in the main file will appear in all child documents
unless suppressed manually;
it cannot be suppressed automatically by the |\includeonly| directive
and thus should normally be avoided.
A method to include some content in the main file
by means of conditional processing is described in \secref{sec:conditional}.

%%%%%%%%%%%%%%%%%%%%%%%%%%%%%%%%%%%%%%%%
\paragraph{Page Numbering.}

When only a part of the document is compiled,
the appropriate numbering of pages
(as well as other status parameters)
is determined from the |.aux| files.
The latter contain information from previous passes.
However this information needs to propagate through
all intermediate child documents.
Therefore the page numbering in child documents may well
be inconsistent until the complete document is compiled at least once.

A useful (if unconventional) way to always ensure a consistent
page numbering is to restart the numbering in each child document
and denote the pages by `\textit{child}|.|\textit{page}'
where \textit{child} represents the chapter/section number of the child file.
This can be achieved by the command
|\numberwithin{page}{|\textit{child}|}|
of the \textsf{amsmath} package
where \textit{child} can be |chapter| or |section|
depending on the chosen structuring.
Alternatively, one can modify the macro |\thepage| appropriately
and reset the counter |page| at the start of each child file.

%%%%%%%%%%%%%%%%%%%%%%%%%%%%%%%%%%%%%%%%%%%%%%%%%%%%%%%%%%%%%%%%%%%%%%%%%%%%%%%%
\subsection{Conditional Processing}
\label{sec:conditional}

The package provides a mechanism to compile different versions
of a document. To customise the versions further some conditional processing
can come in handy to distinguish which version is being compiled.
The package provides two macros to describe the compilation context:

%%%%%%%%%%%%%%%%%%%%%%%%%%%%%%%%%%%%%%%%
\DescribeMacro{\ifchilddoc}
The conditional |\ifchilddoc| distinguishes between the compilation of
child documents and the main document:
%
\begin{center}
|\ifchilddoc |\textit{child-code}| |[|\||else |\textit{main-code}]| \||fi|
\end{center}

%%%%%%%%%%%%%%%%%%%%%%%%%%%%%%%%%%%%%%%%
\DescribeMacro{\childdocname}
\DescribeMacro{\childdocjob}
The macro |\childdocname| contains the filename (without extension)
of the main or child file being processed.
Note that |\childdocjob| will always contain the name of the main file.

%%%%%%%%%%%%%%%%%%%%%%%%%%%%%%%%%%%%%%%%
\paragraph{Title Page.}

Conditional processing can be used to include a title or banner page
in the main document when proper precautions are taken.
Importantly, the code in the main file should ensure that the page counter
(as well as other status parameters which are stored in the |.aux| files)
takes the same value after the conditional processing.
Otherwise the page numbers may take divergent values
depending on which part is compiled.

For example, a title page could be declared by:
%
\begin{center}
\begin{tabular}{l}
|\ifchilddoc\||else|\\
|\addtocounter{page}{-1}|\\
\textit{code for title page}\\
|\newpage|\\
|\||fi|
\end{tabular}
\end{center}
%
A banner page for the child documents can be generated by:
%
\begin{center}
\begin{tabular}{l}
|\ifchilddoc|\\
|\addtocounter{page}{-1}|\\
\textit{code for banner page}\\
|\newpage|\\
|\||fi|
\end{tabular}
\end{center}
%
Here one could write a message such as:
\begin{center}
|This is the part \childdocname{} of \childdocjob{}.|
\end{center}

%%%%%%%%%%%%%%%%%%%%%%%%%%%%%%%%%%%%%%%%%%%%%%%%%%%%%%%%%%%%%%%%%%%%%%%%%%%%%%%%
\subsection{Flags}
\label{sec:flags}

The package makes it easy to generate different versions
of the main or child documents.
To this end compilation flags can be defined
and assigned different default values.
They will be particularly useful in conjunction
with the forwarding mechanism described in \secref{sec:forward}.

For example, it may be useful to have a flag |\version|
which can be set to |draft| or |final|.
The document source will contain some conditional code
depending on the value of |\version|.
Suppose further, the flag should default to |final| for the main file
and to |draft| for child files
which is a natural assignment for editing the document.
This is achieved by placing the following code
in the preamble of the main document
(below the |\childdocmain| directive):
%
\begin{center}
\begin{tabular}{l}
|\ifchilddoc|\\
|\providecommand{\version}{draft}|\\
|\||else|\\
|\providecommand{\version}{final}|\\
|\||fi|
\end{tabular}
\end{center}
%
The definition by |\providecommand| makes sure
that previous definitions are not overwritten.
Further statements |\providecommand{\version}{...}|
can thus be added before the above code to override it.

For the main file, one might add a line
(between |\childdocmain| and the above block)
%
\begin{center}
|%\ifchilddoc\||else\providecommand{\version}{draft}\||fi|
\end{center}
%
which can be uncommented to produce a draft version.
Likewise one can add a line to the very top of a child file
(above the |\childdocof{|\textit{main}|}| directive)
%
\begin{center}
|%\providecommand{\version}{final}|
\end{center}
%
which can be uncommented to produce the final version of this child document.

%%%%%%%%%%%%%%%%%%%%%%%%%%%%%%%%%%%%%%%%%%%%%%%%%%%%%%%%%%%%%%%%%%%%%%%%%%%%%%%%
\subsection{Forwarding}
\label{sec:forward}

Different versions of the main or child documents
using compilation flags as described in \secref{sec:flags}
can be (permanently) stored in different files
for convenient compilation, viewing and distribution.
To this end, the package defines a command
to pass on compilation to a different file:

%%%%%%%%%%%%%%%%%%%%%%%%%%%%%%%%%%%%%%%%
\DescribeMacro{\childdocforward}
The command |\childdocforward| redirects processing to
another source file:
%
\begin{center}
\begin{tabular}{l}
|\input{childdoc.def}|\\
|\childdocforward[|\textit{main}|]{|\textit{dest}|}|\\
\end{tabular}
\end{center}
%
The argument \textit{dest} is the destination file
(without extension).
It should be the main file or one of the child files.
Note that further \textsf{childdoc} directives
such as |\childdocof| and |\childdocforward|
in the indicated file will be processed in this form.
The optional argument \textit{main}
passes on directly to the main file \textit{main}
while pretending to compile the child \textit{dest}.
This form behaves as if \textit{dest}
issues |\childdocof{|\textit{main}|}| right away,
and no further \textsf{childdoc} directives will be processed.

%%%%%%%%%%%%%%%%%%%%%%%%%%%%%%%%%%%%%%%%
\DescribeMacro{\...prefix}
In the alternative form |\childdocforwardprefix|,
%
\begin{center}
\begin{tabular}{l}
|\input{childdoc.def}|\\
|\childdocforwardprefix[|\textit{main}|]{|\textit{prefix}|}{|\textit{dest}|}|
\end{tabular}
\end{center}
%
the destination file is determined by a pattern
depending on the current file:
To make this work, the current file must be called
`{\textit{prefix}\hspace{0.2em}\textit{suffix}}'
with \textit{prefix} matching precisely the argument.
Processing is then passed on to the file
`{\textit{dest}\hspace{0.2em}\textit{suffix}}'.
Surely, the same effect is achieved by
directly specifying the
argument `{\textit{dest}\hspace{0.2em}\textit{suffix}}'
in the first form.
However, that requires to set up a different file
for each child. With the alternative form of the command
all these files can have exactly the same content
which simplifies setting them up and maintaining them.

For example, the following file |draft.tex|
with a compilation flag |\version| as described in \secref{sec:flags}
compiles the main document as a draft:
%
\begin{center}
\begin{tabular}{l}
|\def\version{draft}|\\
|\input{childdoc.def}|\\
|\childdocforward{|\textit{main}|}|
\end{tabular}
\end{center}
%
Likewise, the following files |final|\textit{nn}|.tex|
compile the final version of the child document
|child|\textit{nn}|.tex|:
%
\begin{center}
\begin{tabular}{l}
|\def\version{final}|\\
|\input{childdoc.def}|\\
|\childdocforwardprefix{final}{child}|
\end{tabular}
\end{center}
%

Note that when several versions of a main file and/or of each child file
are to be generated, it may be convenient to set up a |Makefile| or
shell script to automatise the process.

%%%%%%%%%%%%%%%%%%%%%%%%%%%%%%%%%%%%%%%%%%%%%%%%%%%%%%%%%%%%%%%%%%%%%%%%%%%%%%%%
\subsection{Command Line Processing}
\label{sec:commandline}

The effect of redirection files can also be achieved by invoking
the \LaTeX{} compiler with a more elaborate command line.
Most conveniently this should be done as part
of a shell script or a |Makefile|.

When using \textsf{childdoc} in the main file, the following
command lines effectively perform a redirection
(note that depending on the shell being used,
backslashes may have to be doubled: `|\|' $\to$ `|\\|'):
%
\begin{center}
|... -jobname "|\textit{target}|" |\\|"|[\textit{flags}]%
|\input{childdoc.def}\childdocforward[|\textit{main}|]{|\textit{dest}|}"|
\end{center}
%
Here \textit{target} is the name of the output file,
\textit{main} is the name of the main file
and \textit{dest} is the name of the main or child file to be processed
(all filenames without extensions).
The optional argument \textit{main} can be omitted
if \textit{main} matches \textit{dest}.
Optionally, compilation \textit{flags} can be defined via |\def| commands.
This command line makes the \TeX{} engine believe
it is compiling the file \textit{target}
whose content is specified as the latter parameter.
The provided code then forwards the processing to
\textit{main} or \textit{dest} as described in \secref{sec:forward}.

%%%%%%%%%%%%%%%%%%%%%%%%%%%%%%%%%%%%%%%%%%%%%%%%%%%%%%%%%%%%%%%%%%%%%%%%%%%%%%%%
\subsection{Include by Input}
\label{sec:input}

Including child documents by |\include| has some restrictions by design.
Most notably, the content of a child document always occupies
its own set of pages; pages cannot be shared between child documents.
Usually, this behaviour makes perfect sense
because each child document contain an essential part of the document.
However, in some situations it may be desirable to compose
a document from a collection of parts
without having mandatory page breaks between then.
For this case, the package
provides a mechanism to include parts
by |\input| which can also be processed individually.
However, by construction this mechanism
requires manual handling of the content to be output.

%%%%%%%%%%%%%%%%%%%%%%%%%%%%%%%%%%%%%%%%
\DescribeMacro{\ifchilddocmanual}
The main file should be prepared as usual, see \secref{sec:include}.
However, the document body must make a distinction
between processing of an individual part and of the main document, e.g.:
%
\begin{center}
\begin{tabular}{l}
|\ifchilddocmanual|\\
|\input{\childdocname}|\\
|\||else|\\
\textit{document body with }|\input{|\textit{part}|}|\\
|\||fi|
\end{tabular}
\end{center}
%
The conditional |\ifchilddocmanual| is true whenever
a part to be included by |\input| is being compiled,
and the name of the part is stored in |\childdocname|.

%%%%%%%%%%%%%%%%%%%%%%%%%%%%%%%%%%%%%%%%
\DescribeMacro{\childdocby}
Each part to be included by |\input| should start with:
%
\begin{center}
\begin{tabular}{l}
|\input{childdoc.def}|\\
|\childdocby{|\textit{main}|}|\\
\end{tabular}
\end{center}
%
The directive |\childdocby| is similar to |\childdocof|
described in \secref{sec:include},
but the subsequent selection of content must be done manually.
To that end, both |\ifchilddoc| and |\ifchilddocmanual|
will be true upon processing of a part,
and the name of the part is stored in |\childdocname|.
Note that |\jobname| will be set to the filename of the current part
so that each part receives an individual |.aux| file
that does not interfere with the |.aux| file(s) of the main document.
This behaviour can be altered by the alternative form
|\childdocby[*]{|\textit{main}|}| (with a non-empty optional argument)
which uses the |.aux| file of the main document
by setting |\jobname| to \textit{main}.

%%%%%%%%%%%%%%%%%%%%%%%%%%%%%%%%%%%%%%%%%%%%%%%%%%%%%%%%%%%%%%%%%%%%%%%%%%%%%%%%
\subsection{Driver Development}
\label{sec:driver}

The \textsf{childdoc} mechanism can also be use for the development
of definition files such as \LaTeX{} styles or classes.
This case differs from the above setup with multiple parts
included by |\include| in that no |\includeonly| should be invoked.
This can be achieved by starting the include file
(before |\ProvidesPackage|) with:
%
\begin{center}
\begin{tabular}{l}
|\input{childdoc.def}|\\
|\childdocforward{|\textit{main}|}|\\
\end{tabular}
\end{center}
%
or alternatively with:
%
\begin{center}
\begin{tabular}{l}
|\input{childdoc.def}|\\
|\childdocby{|\textit{main}|}|\\
\end{tabular}
\end{center}
%
Both forms have slightly different effects as described above.
The main file is prepared as usual, see \secref{sec:include}.

%%%%%%%%%%%%%%%%%%%%%%%%%%%%%%%%%%%%%%%%%%%%%%%%%%%%%%%%%%%%%%%%%%%%%%%%%%%%%%%%
\subsection{Legacy Detection}
\label{sec:detection}

The directive |\childdocmain| in the main file can detect
whether the complete document or merely a child is to be compiled
even without using the directive |\childdocof|.
This method is deprecated because it is less robust
and there is no compelling reason to use it;
it is merely provided for backward compatibility
and it may be removed in future versions.

If the detection mechanism is to be used,
it is mandatory to correctly specify
the filename of the main file as the argument of |\childdocmain|:
%
\begin{center}
\begin{tabular}{l}
|\input{childdoc.def}|\\
|\childdocmain{|\textit{main}|}|\\
\end{tabular}
\end{center}
%
If |\jobname| does not match the argument \textit{main} of |\childdocmain|,
it is assumed that |\jobname| points to the child file to be compiled.
When using |\childdocmain| with the main file specified as argument,
it suffices to start a child file
with just |\input{|\textit{main}|}|
without loading of the package and using |\childdocof|.
If instead all processing is done
with the appropriate \textsf{childdoc} directives,
the argument of \textit{main} of |\childdocmain| can be empty.

An alternative version of the command line processing described
in \secref{sec:commandline} using the detection mechanism reads:
%
\begin{center}
|... -jobname "|\textit{target}|" "|[\textit{flags}]%
[|\def\jobname{|\textit{dest}|}|]|\input{|\textit{main}|}"|
\end{center}

%%%%%%%%%%%%%%%%%%%%%%%%%%%%%%%%%%%%%%%%%%%%%%%%%%%%%%%%%%%%%%%%%%%%%%%%%%%%%%%%
\subsection{Manual Code}
\label{sec:manual}

In case one cannot be certain whether the definitions file |childdoc.def|
is installed on the target \TeX{} distribution
and one prefers not to ship it,
it is conceivable to paste a few relevant commands into the sources.

To that end, drop all statements |\input{childdoc.def}|
and perform the replacements as outlined below.
Instead of |\childdocmain{|\textit{main}|}| add the following code
to the top of the main file:
%
\begin{center}
\begin{tabular}{l}
|\||ifdefined\childdocname\endinput\||fi\newif\ifchilddoc|\\
|\edef\childdocname{\scantokens\expandafter{\jobname\noexpand}}|\\
|\def\childdocmain{|\textit{main}|}\||ifx\childdocmain\childdocname\||else|\\
|\childdoctrue\includeonly{\childdocname}\let\jobname\childdocmain\||fi|\\
\end{tabular}
\end{center}
%
Instead of |\childdocof{|\textit{main}|}| just include the main file
at the top of each child file:
%
\begin{center}
|\input{|\textit{main}|}|
\end{center}
%
A simple redirection |\childdocforward{|\textit{dest}|}| is achieved by:
%
\begin{center}
|\def\jobname{|\textit{dest}|}\input{\jobname}|
\end{center}
%
The redirection with prefix
|\childdocforwardprefix[|\textit{prefix}|]{|\textit{dest}|}|
is accomplished by:
%
\begin{center}
\begin{tabular}{l}
|{\edef\jobname{\scantokens\expandafter{\jobname\noexpand}}|\\
|\def\redirectjob |\textit{prefix}|#1~~~{\gdef\jobname{|\textit{dest}|#1}}|\\
|\expandafter\redirectjob\jobname~~~}\input{\jobname}|
\end{tabular}
\end{center}

In an alternative approach,
child documents can be compiled by a specific command line
without additional code or specific definitions:
%
\begin{center}
|... -jobname "|\textit{target}|" "|[\textit{flags}]%
|\includeonly{|\textit{dest}|}\input{|\textit{main}|}"|
\end{center}
%

%%%%%%%%%%%%%%%%%%%%%%%%%%%%%%%%%%%%%%%%%%%%%%%%%%%%%%%%%%%%%%%%%%%%%%%%%%%%%%%%
%%%%%%%%%%%%%%%%%%%%%%%%%%%%%%%%%%%%%%%%%%%%%%%%%%%%%%%%%%%%%%%%%%%%%%%%%%%%%%%%
\section{Information}

%%%%%%%%%%%%%%%%%%%%%%%%%%%%%%%%%%%%%%%%%%%%%%%%%%%%%%%%%%%%%%%%%%%%%%%%%%%%%%%%
\subsection{Copyright}

Copyright \copyright{} 2017--2018 Niklas Beisert

This work may be distributed and/or modified under the
conditions of the \LaTeX{} Project Public License, either version 1.3
of this license or (at your option) any later version.
The latest version of this license is in
  \url{http://www.latex-project.org/lppl.txt}
and version 1.3 or later is part of all distributions of \LaTeX{}
version 2005/12/01 or later.

This work has the LPPL maintenance status `maintained'.

The Current Maintainer of this work is Niklas Beisert.

This work consists of the files |README.txt|, |childdoc.ins| and |childdoc.dtx|
as well as the derived files |childdoc.def|, |cdocsamp.tex|
with |cdocsch1.tex|, |cdocsch2.tex|, |cdocspt3.tex|, |cdocspt4.tex|,
|cdocsdrf.tex|, |cdocsfn1.tex|, |cdocsfn2.tex|
as well as |childdoc.pdf|.

%%%%%%%%%%%%%%%%%%%%%%%%%%%%%%%%%%%%%%%%%%%%%%%%%%%%%%%%%%%%%%%%%%%%%%%%%%%%%%%%
\subsection{Files and Installation}

The package consists of the files:
%
\begin{center}
\begin{tabular}{ll}
    |README.txt|   & readme file \\
    |childdoc.ins| & installation file \\
    |childdoc.dtx| & source file \\
    |childdoc.def| & definition file \\
    |cdocsamp.tex| & sample main file \\
    |cdocsch1.tex| & sample include file \\
    |cdocsch2.tex| & sample include file \\
    |cdocspt3.tex| & sample part file \\
    |cdocspt4.tex| & sample part file \\
    |cdocsdrf.tex| & sample redirection file \\
    |cdocsfn1.tex| & sample redirection file \\
    |cdocsfn2.tex| & sample redirection file \\
    |childdoc.pdf| & manual
\end{tabular}
\end{center}
%
The distribution consists of the files
|README.txt|, |childdoc.ins| and |childdoc.dtx|.
%
\begin{itemize}
\item
Run (pdf)\LaTeX{} on |childdoc.dtx|
to compile the manual |childdoc.pdf| (this file).
\item
Run \LaTeX{} on |childdoc.ins| to create the definitions file |childdoc.def|
and the sample |cdocsamp.tex| with include files
|cdocsch1.tex|, |cdocsch2.tex|, |cdocspt3.tex|, |cdocspt4.tex|,
|cdocsdrf.tex|, |cdocsfn1.tex|, |cdocsfn2.tex|.
Then copy the file |childdoc.def| to an appropriate directory of your \LaTeX{}
distribution, e.g.\ \textit{texmf-root}|/tex/latex/childdoc|.
\end{itemize}

%%%%%%%%%%%%%%%%%%%%%%%%%%%%%%%%%%%%%%%%%%%%%%%%%%%%%%%%%%%%%%%%%%%%%%%%%%%%%%%%
\subsection{Related CTAN Packages}

There are several other packages which offer a similar functionality:
%
\begin{itemize}
\item
The packages
\href{http://ctan.org/pkg/docmute}{\textsf{docmute}},
\href{http://ctan.org/pkg/includex}{\textsf{includex}} and
\href{http://ctan.org/pkg/standalone}{\textsf{standalone}}
provide commands to include only the document body of
a child file thus allowing both files to be compiled individually.
\item
The packages \href{http://ctan.org/pkg/subdocs}{\textsf{subdocs}}
and \href{http://ctan.org/pkg/subfiles}{\textsf{subfiles}}
provide structures in which the main and child documents can be
encapsulated and allowing them to be compiled individually.
The inclusion mechanism is different from the conventional |\include|.
\item
The package \href{http://ctan.org/pkg/combine}{\textsf{combine}}
is an elaborate solution to combine several documents into one.
\end{itemize}
%
See also the CTAN topic \href{http://ctan.org/topic/subdocs}{\textsf{subdocs}}
for further related packages.
The present package differs from the above solutions in that
a document structure constructed with the conventional |\include| mechanism
just needs two extra commands at the top of every file
such that all constituent files can be compiled individually.

%%%%%%%%%%%%%%%%%%%%%%%%%%%%%%%%%%%%%%%%%%%%%%%%%%%%%%%%%%%%%%%%%%%%%%%%%%%%%%%%
%\subsection{Feature Suggestions}
%
%The following is a list of features which may be useful for future
%versions of this package:
%%
%\begin{itemize}
%\item
%\ldots
%\end{itemize}

%%%%%%%%%%%%%%%%%%%%%%%%%%%%%%%%%%%%%%%%%%%%%%%%%%%%%%%%%%%%%%%%%%%%%%%%%%%%%%%%
\subsection{Revision History}

%%%%%%%%%%%%%%%%%%%%%%%%%%%%%%%%%%%%%%%%
\paragraph{v2.0:} 2018/12/30

\begin{itemize}
\item
immediate forward processing
\item
added |\childdocby| mechanism
\item
manual restructured
\end{itemize}

%%%%%%%%%%%%%%%%%%%%%%%%%%%%%%%%%%%%%%%%
\paragraph{v1.6:} 2018/01/17

\begin{itemize}
\item
application for development of include files
\item
corrections to manual
\end{itemize}

%%%%%%%%%%%%%%%%%%%%%%%%%%%%%%%%%%%%%%%%
\paragraph{v1.5:} 2017/05/21

\begin{itemize}
\item
more complete structuring introduced
\item
|\childdocof| introduced
\item
|\childdoc| renamed to |\childdocmain|
\item
|\childredirect| renamed to |\childdocforward| and |\childdocforwardprefix|
and functionality expanded
\end{itemize}

%%%%%%%%%%%%%%%%%%%%%%%%%%%%%%%%%%%%%%%%
\paragraph{v1.0:} 2017/04/27

\begin{itemize}
\item
manual and install package
\item
first version published on CTAN
\end{itemize}

%%%%%%%%%%%%%%%%%%%%%%%%%%%%%%%%%%%%%%%%
\paragraph{v0.6:} 2017/04/26

\begin{itemize}
\item
redirection mechanism added
\end{itemize}

%%%%%%%%%%%%%%%%%%%%%%%%%%%%%%%%%%%%%%%%
\paragraph{v0.5:} 2017/04/26

\begin{itemize}
\item
functionality in definition file
\end{itemize}


%%%%%%%%%%%%%%%%%%%%%%%%%%%%%%%%%%%%%%%%%%%%%%%%%%%%%%%%%%%%%%%%%%%%%%%%%%%%%%%%
%%%%%%%%%%%%%%%%%%%%%%%%%%%%%%%%%%%%%%%%%%%%%%%%%%%%%%%%%%%%%%%%%%%%%%%%%%%%%%%%
%%%%%%%%%%%%%%%%%%%%%%%%%%%%%%%%%%%%%%%%%%%%%%%%%%%%%%%%%%%%%%%%%%%%%%%%%%%%%%%%
\appendix

\settowidth\MacroIndent{\rmfamily\scriptsize 000\ }

 \DocInput{childdoc.dtx}

\end{document}
%</driver>
% \fi
%
% %%%%%%%%%%%%%%%%%%%%%%%%%%%%%%%%%%%%%%%%%%%%%%%%%%%%%%%%%%%%%%%%%%%%%%%%%%%%%%
% %%%%%%%%%%%%%%%%%%%%%%%%%%%%%%%%%%%%%%%%%%%%%%%%%%%%%%%%%%%%%%%%%%%%%%%%%%%%%%
% \section{Sample}
%\iffalse
%<*samplemain>
%\fi
%
% The following presents a sample document
% with two chapters, two parts, a title page,
% a compile flag as well as three forwarding files to set the flag.
% It consists of eight |.tex| files:
% \begin{center}
% \begin{tabular}{ll}
% |cdocsamp.tex|&main file\\
% |cdocsch1.tex|&include file for chapter 1\\
% |cdocsch2.tex|&include file for chapter 2\\
% |cdocspt3.tex|&include file for part 3\\
% |cdocspt4.tex|&include file for part 4\\
% |cdocsdrf.tex|&forwarding file for main file in draft mode\\
% |cdocsfi1.tex|&forwarding file for final version of chapter 1\\
% |cdocsfi2.tex|&forwarding file for final version of chapter 2\\
% \end{tabular}
% \end{center}
% Each of the eight files can be compiled directly by the \LaTeX{} compiler.
%
% %%%%%%%%%%%%%%%%%%%%%%%%%%%%%%%%%%%%%%
% \paragraph{Main File.}
%
% The main file is called |cdocsamp.tex|.
%
% Load the \textsf{childdoc} definitions and
% declare the filename for the main document:
%    \begin{macrocode}
\input{childdoc.def}
\childdocmain{}
%    \end{macrocode}

% Optional override for |\version| flag:
%    \begin{macrocode}
%%\ifchilddoc\else\providecommand{\version}{draft}\fi
%    \end{macrocode}

% Define the default values for the |\version| flag
% (|final| for the main file and |draft| for childs):
%    \begin{macrocode}
\ifchilddoc
\providecommand{\version}{draft}
\else
\providecommand{\version}{final}
\fi
%    \end{macrocode}

% Load the standard document class:
%    \begin{macrocode}
\documentclass[12pt]{article}
%    \end{macrocode}

% Start the document body:
%    \begin{macrocode}
\begin{document}
%    \end{macrocode}

% Declare a title page.
% Print title, part of document being processed and version flag:
%    \begin{macrocode}
\addtocounter{page}{-1}
\begin{center}
{\LARGE\bfseries{}childdoc example\par}
\vspace{1cm}
\ifchilddoc
\ifchilddocmanual part\else chapter\fi:
`\childdocname' of `\childdocjob'\par
\else
main document: `\childdocjob'\par
\fi
version: \version\par
\end{center}
\newpage
%    \end{macrocode}

% Manually include selected file,
% otherwise process as usual:
%    \begin{macrocode}
\ifchilddocmanual
\section*{part `\childdocname'}
\input{\childdocname}
\else
%    \end{macrocode}

% Include the two chapters:
%    \begin{macrocode}
\include{cdocsch1}
\include{cdocsch2}
%    \end{macrocode}

% Include the two parts unless only chapters should be displayed:
%    \begin{macrocode}
\ifchilddoc\else
\section{part three}
\input{cdocspt3}
\section{part four}
\input{cdocspt4}
\fi
%    \end{macrocode}

% Process as usual until here:
%    \begin{macrocode}
\fi
%    \end{macrocode}

% End of document body:
%    \begin{macrocode}
\end{document}
%    \end{macrocode}
%\iffalse
%</samplemain>
%\fi
%
% %%%%%%%%%%%%%%%%%%%%%%%%%%%%%%%%%%%%%%
% \paragraph{Chapter Include Files.}
%
% The include files are called |cdocsch1.tex| and |cdocsch2.tex|.
%
%\iffalse
%<*samplechap1|samplechap2>
%\fi

% Optional override for |\version| flag:
%    \begin{macrocode}
%%\providecommand{\version}{final}
%    \end{macrocode}

% Include the main document:
%    \begin{macrocode}
\input{childdoc.def}
\childdocof{cdocsamp}
%    \end{macrocode}

%\iffalse
%</samplechap1|samplechap2>
%\fi
%
%\iffalse
%<*samplechap1>
%\fi
% Some text for chapter 1:
%    \begin{macrocode}
\section{one}
some text in chapter one
%    \end{macrocode}

%\iffalse
%</samplechap1>
%\fi
% Some text for chapter 2:
%\iffalse
%<*samplechap2>
%\fi
%    \begin{macrocode}
\section{two}
more text in chapter two
%    \end{macrocode}

%\iffalse
%</samplechap2>
%\fi
%
% %%%%%%%%%%%%%%%%%%%%%%%%%%%%%%%%%%%%%%
% \paragraph{Part Include Files.}
%
% The include files are called |cdocspt3.tex| and |cdocspt4.tex|.
%
%\iffalse
%<*samplepart3|samplepart4>
%\fi

% Optional override for |\version| flag:
%    \begin{macrocode}
%%\providecommand{\version}{final}
%    \end{macrocode}

% Include the main document:
%    \begin{macrocode}
\input{childdoc.def}
\childdocby{cdocsamp}
%    \end{macrocode}

%\iffalse
%</samplepart3|samplepart4>
%\fi
%
%\iffalse
%<*samplepart3>
%\fi
% Some text for part 3:
%    \begin{macrocode}
some text in part three
%    \end{macrocode}

%\iffalse
%</samplepart3>
%\fi
% Some text for part 4:
%\iffalse
%<*samplepart4>
%\fi
%    \begin{macrocode}
more text in part four
%    \end{macrocode}

%\iffalse
%</samplepart4>
%\fi
%
% %%%%%%%%%%%%%%%%%%%%%%%%%%%%%%%%%%%%%%
% \paragraph{Forwarding for a Complete Draft.}
%
% The following forwarding file |cdocsdrf.tex|
% compiles the main document in draft mode:
%\iffalse
%<*sampledraft>
%\fi
%    \begin{macrocode}
\def\version{draft}
\input{childdoc.def}
\childdocforward{cdocsamp}
%    \end{macrocode}

%\iffalse
%</sampledraft>
%\fi
%
% %%%%%%%%%%%%%%%%%%%%%%%%%%%%%%%%%%%%%%
% \paragraph{Forwarding for Final Version of the Chapters.}
%
% The following forwarding files |cdocsfn1.tex| and |cdocsfn2.tex|
% (with identical content)
% compile the final versions of the child documents
% |cdocsch1.tex| and |cdocsch2.tex|, respectively:
%\iffalse
%<*samplefinal>
%\fi
%    \begin{macrocode}
\def\version{final}
\input{childdoc.def}
\childdocforwardprefix[cdocsamp]{cdocsfn}{cdocsch}
%    \end{macrocode}

%\iffalse
%</samplefinal>
%\fi
%
% %%%%%%%%%%%%%%%%%%%%%%%%%%%%%%%%%%%%%%
% \paragraph{Command Line Processing.}
%
% The following three command lines generate the output files
% |cdocscld|, |cdocscl1| and |cdocscl2|
% which should be identical to
% |cdocsdrf|, |cdocsch1| and |cdocsfn2|, respectively:
% \begin{center}
% \begin{tabular}{l}
% |latex -jobname cdocscld \|\\
% |  "\def\version{draft}\input{childdoc.def}\childdocforward{cdocsamp}"|\\
% |latex -jobname cdocscl1 \|\\
% |  "\input{childdoc.def}\childdocforward[cdocsamp]{cdocsch1}"|\\
% |latex -jobname cdocscl2 \|\\
% |  "\def\version{final}\input{childdoc.def}\childdocforward{cdocsch2}"|
% \end{tabular}
% \end{center}
% Note that the trailing backslash on each first line
% merely continues the input to the second line
% (for convenient cut ant paste).
% Furthermore, the command |latex| can be replaced by any
% of its alternative versions such as |pdflatex|.
%
% %%%%%%%%%%%%%%%%%%%%%%%%%%%%%%%%%%%%%%%%%%%%%%%%%%%%%%%%%%%%%%%%%%%%%%%%%%%%%%
% %%%%%%%%%%%%%%%%%%%%%%%%%%%%%%%%%%%%%%%%%%%%%%%%%%%%%%%%%%%%%%%%%%%%%%%%%%%%%%
% \section{Implementation}
%\iffalse
%<*package>
%\fi
%
% This section describes the definitions file |childdoc.def|.

% The definitions cannot be loaded using |\usepackage| or |\RequirePackage|
% which has a mechanism to prevent loading a style file more than once.
% When loading the definitions by means of |\input|
% multiple instances have to be prevented manually:
%\iffalse
%This code needs to be before the `\ProvidesFile' directive
%which is defined at the beginning of this file.
%Therefore it is also placed there and commented out here.
%</package>
%<*discard>
%\fi
%    \begin{macrocode}
\ifdefined\childdocmain\endinput\fi
%    \end{macrocode}
%\iffalse
%</discard>
%<*package>
%\fi
%
% \macro{\ifchilddoc}
% \macro{\ifchilddocmanual}
% The conditional |\ifchilddoc| tells whether a
% child (true) or main (false) document is being compiled.
% The conditional |\ifchilddocmanual| tells whether
% the |\includeonly| mechanism is used (false) or
% the selection of child files must be performed manually (true).
% The definitions initialise to false:
%    \begin{macrocode}
\newif\ifchilddoc
\newif\ifchilddocmanual
%    \end{macrocode}

% \macro{\childdocname}
% \macro{\childdocjob}
% The macro |\childdocname| stores the name of the main document
% to be compiled. The macro |\childdocjob| stores the name of
% the document on which the \LaTeX{} compiler was originally invoked.
% The content of |\jobname| cannot be compared
% to filenames specified in the source due to different catcodes.
% The following code rescans |\jobname|, stores the result
% in |\childdocname| and saves a copy in |\childdocjob|:
%    \begin{macrocode}
\edef\childdocname{\scantokens\expandafter{\jobname\noexpand}}
\let\childdocjob\childdocname
%    \end{macrocode}

% \macro{\childdocdisable}
% The macro |\childdocdisable| prevents the main file
% from being processed more than once.
% At this stage, the main document command |\childdocmain|
% is assumed to be called once again where it should do nothing.
% Any subsequent call to it should prevent
% a secondary processing of the main document
% It overwrites the forwarding commands
% |\childdocof| and |\childdocforward|
% with empty macros to prevent further inclusions of the main document:
%    \begin{macrocode}
\newcommand{\childdocdisable}
{
  \renewcommand{\childdocmain}[1]{\renewcommand{\childdocmain}[1]{\endinput}}
  \renewcommand{\childdocof}[1]{}
  \renewcommand{\childdocby}[2][]{}
  \renewcommand{\childdocforward}[2][]{}
  \renewcommand{\childdocdisable}{}
}
%    \end{macrocode}

% \macro{\childdocmain}
% The macro |\childdocmain| is to be called at the top of the main file
% with nothing or the main filename (without extension) as argument.
% First, it breaks loops.
% If the argument is not empty and does not match |\childdocname|
% (which is set by the first inclusion of |childdoc.def|),
% |\ifchilddoc| is set to true, |\includeonly| is applied to the child file
% and |\jobname| is set to the main file
% (for proper handling of |.aux| files):
%    \begin{macrocode}
\newcommand{\childdocmain}[1]
{
  \childdocdisable\childdocmain{}
  \if?#1?\else
    \begingroup
      \def\childdoctmp{#1}
      \ifx\childdoctmp\childdocname
        \def\childdoctmp{}
      \else
        \def\childdoctmp
        {
          \childdoctrue
          \includeonly{\childdocname}
          \def\childdocjob{#1}
          \def\jobname{#1}
        }
      \fi
      \expandafter
    \endgroup
    \childdoctmp
  \fi
}
%    \end{macrocode}

% \macro{\childdocof}
% The command |\childdocof| redirects
% compilation to the main file |#1|.
%    \begin{macrocode}
\newcommand{\childdocof}[1]
{
  \childdocdisable
  \childdoctrue
  \includeonly{\childdocname}
  \def\jobname{#1}
  \def\childdocjob{#1}
  \input{#1}
}
%    \end{macrocode}

% \macro{\childdocby}
% The command |\childdocby| ....
%    \begin{macrocode}
\newcommand{\childdocby}[2][]
{
  \childdocdisable
  \childdoctrue
  \childdocmanualtrue
  \if?#1?\else
    \def\jobname{#2}
  \fi
  \def\childdocjob{#2}
  \input{#2}
  \endinput
}
%    \end{macrocode}

% \macro{\childdocforward}
% The command |\childdocforward| redirects
% compilation to the main file or
% (if the optional argument is given) a child file.
% Parameters are set as if the main file
% or a child file starting with |\childdocof| was compiled.
% Then compilation is handed over to the main file:
%    \begin{macrocode}
\newcommand{\childdocforward}[2][]
{
  \begingroup
    \if?#1?
      \def\childdoctmp
      {
        \def\childdocname{#2}
        \def\childdocjob{#2}
        \def\jobname{#2}
        \input{#2}
        \endinput
      }
    \else
      \def\childdoctmp
      {
        \childdocdisable
        \def\childdocname{#2}
        \childdoctrue
        \includeonly{#2}
        \def\childdocjob{#1}
        \def\jobname{#1}
        \input{#1}
        \endinput
      }
    \fi
    \expandafter
  \endgroup
  \childdoctmp
}
%    \end{macrocode}

% \macro{\childdocforwardprefix}
% The command |\childdocforwardprefix| redirects
% compilation to the main or a child file by means of a pattern.
% The prefix |#1| in the current filename is replaced by |#2|
% and the suffix of the current filename is kept
% (it is assumed that the filename does not contain the substring `|~~~|'
% which is used as a delimiter).
% Compilation is handed over to the new file by |\childdocforward|:
%    \begin{macrocode}
\newcommand{\childdocforwardprefix}[3][]
{
  \begingroup
    \def\childdocextract #2##1~~~{\def\childdoctmp{\childdocforward[#1]{#3##1}}}
    \expandafter\childdocextract\childdocname~~~
    \expandafter
  \endgroup
  \childdoctmp
}
%    \end{macrocode}

% \macro{\childdoc}
% The deprecated macro |\childdoc| is a legacy version of |\childdocmain|:
%    \begin{macrocode}
\newcommand{\childdoc}{\childdocmain}
%    \end{macrocode}

% \macro{\childdocredirect}
% The deprecated macro |\childdocredirect| is a legacy version
% of |\childdocforward| and |\childdocforwardprefix|:
%    \begin{macrocode}
\newcommand{\childdocredirect}[2][]
{
  \begingroup
    \if?#1?
      \def\childdoctmp{\childdocforward{#2}}
    \else
      \def\childdoctmp{\childdocforwardprefix{#1}{#2}}
    \fi
    \expandafter
  \endgroup
  \childdoctmp
}
%    \end{macrocode}

%\iffalse
%</package>
%\fi
%
\endinput
|\\
|\childdocforward[|\textit{main}|]{|\textit{dest}|}|\\
\end{tabular}
\end{center}
%
The argument \textit{dest} is the destination file
(without extension).
It should be the main file or one of the child files.
Note that further \textsf{childdoc} directives
such as |\childdocof| and |\childdocforward|
in the indicated file will be processed in this form.
The optional argument \textit{main}
passes on directly to the main file \textit{main}
while pretending to compile the child \textit{dest}.
This form behaves as if \textit{dest}
issues |\childdocof{|\textit{main}|}| right away,
and no further \textsf{childdoc} directives will be processed.

%%%%%%%%%%%%%%%%%%%%%%%%%%%%%%%%%%%%%%%%
\DescribeMacro{\...prefix}
In the alternative form |\childdocforwardprefix|,
%
\begin{center}
\begin{tabular}{l}
|% \iffalse
%
% childdoc.dtx Copyright (C) 2017-2018 Niklas Beisert
%
% This work may be distributed and/or modified under the
% conditions of the LaTeX Project Public License, either version 1.3
% of this license or (at your option) any later version.
% The latest version of this license is in
%   http://www.latex-project.org/lppl.txt
% and version 1.3 or later is part of all distributions of LaTeX
% version 2005/12/01 or later.
%
% This work has the LPPL maintenance status `maintained'.
%
% The Current Maintainer of this work is Niklas Beisert.
%
% This work consists of the files childdoc.dtx and childdoc.ins
% and the derived files childdoc.def and cdocsamp.tex with
% cdocsch1.tex, cdocsch2.tex, cdocsdrf.tex, cdocsfn1.tex, cdocsfn2.tex.
%
%<package>\ifdefined\childdocmain\endinput\fi
%<package>\ProvidesFile{childdoc.def}[2018/12/30 v2.0 child document driver]
%<samplemain>\ProvidesFile{cdocsamp.tex}[2018/12/30 v2.0 sample for childdoc]
%<*driver>
%\ProvidesFile{childdoc.drv}[2018/12/30 v2.0 childdoc reference manual file]
\PassOptionsToClass{10pt,a4paper}{article}
\documentclass{ltxdoc}

\usepackage[margin=35mm]{geometry}
\usepackage{hyperref}
\usepackage{hyperxmp}
\usepackage[usenames]{color}

\hypersetup{colorlinks=true}
\hypersetup{pdfstartview=FitH}
\hypersetup{pdfpagemode=UseNone}
\hypersetup{pdfsource={}}
\hypersetup{pdflang={en-UK}}
\hypersetup{pdfcopyright={Copyright 2017-2018 Niklas Beisert.
  This work may be distributed and/or modified under the
  conditions of the LaTeX Project Public License, either version 1.3
  of this license or (at your option) any later version.}}
\hypersetup{pdflicenseurl={http://www.latex-project.org/lppl.txt}}
\hypersetup{pdfcontactaddress={ETH Zurich, ITP, HIT K,
  Wolfgang-Pauli-Strasse 27}}
\hypersetup{pdfcontactpostcode={8093}}
\hypersetup{pdfcontactcity={Zurich}}
\hypersetup{pdfcontactcountry={Switzerland}}
\hypersetup{pdfcontactemail={nbeisert@itp.phys.ethz.ch}}
\hypersetup{pdfcontacturl={http://people.phys.ethz.ch/\xmptilde nbeisert/}}

\newcommand{\secref}[1]{\hyperref[#1]{section \ref*{#1}}}

\parskip1ex
\parindent0pt
\let\olditemize\itemize
\def\itemize{\olditemize\parskip0pt}

\begin{document}

\title{The \textsf{childdoc} Package}
\hypersetup{pdftitle={The childdoc Package}}
\author{Niklas Beisert\\[2ex]
  Institut f\"ur Theoretische Physik\\
  Eidgen\"ossische Technische Hochschule Z\"urich\\
  Wolfgang-Pauli-Strasse 27, 8093 Z\"urich, Switzerland\\[1ex]
  \href{mailto:nbeisert@itp.phys.ethz.ch}
  {\texttt{nbeisert@itp.phys.ethz.ch}}}
\hypersetup{pdfauthor={Niklas Beisert}}
\hypersetup{pdfsubject={Manual for the LaTeX2e Package childdoc}}
\date{30 December 2018, \textsf{v2.0}}
\maketitle

\begin{abstract}\noindent
\textsf{childdoc} is a \LaTeXe{} package
that enables the direct compilation
of document sections included by |\include|
to individual files.
\end{abstract}

\begingroup
\parskip0ex
\tableofcontents
\endgroup

%%%%%%%%%%%%%%%%%%%%%%%%%%%%%%%%%%%%%%%%%%%%%%%%%%%%%%%%%%%%%%%%%%%%%%%%%%%%%%%%
%%%%%%%%%%%%%%%%%%%%%%%%%%%%%%%%%%%%%%%%%%%%%%%%%%%%%%%%%%%%%%%%%%%%%%%%%%%%%%%%
\section{Introduction}

\LaTeX{} provides a mechanism to structure a large document (such as a book)
into a main file and several child files (containing the chapters)
using the |\include| command.
This mechanism is beneficial for documents
which span hundreds of pages in order to
make the source file(s) more manageable.
Moreover, compilation can be restricted to
selected child files by means of the |\includeonly| command.
The latter feature can be used to reduce the compilation time while editing
(this was significantly more useful in the earlier days of \LaTeX{})
or to generate a smaller document which is easier to navigate.
Another application of |\includeonly| is to generate
documents consisting of selected parts of the complete document.

However, there are a few drawbacks of the plain |\include| mechanism:
\begin{itemize}
\item
The child files cannot be compiled on their own,
they can only be compiled via the main file.
A naive editing environment
(such as a text editor with an option
to have the current file processed by \LaTeX)
may require one to switch to the main file before compiling;
attempting to compile the child file produces errors.
\item
The main file must be modified (each time)
to adjust the |\includeonly| command
to the present needs. This easily leaves the main file in a messy state.
\item
The generated document will always carry the filename
of the main document. This is inconvenient if
several child files are to be compiled and
to be kept for distribution.
\end{itemize}

The present package provides a simple interface
to make child files individually compilable by \LaTeX{}.
Compiling a child file then has the same effect as compiling
the main file with an |\includeonly| command
to select the appropriate child.
Moreover the generated document will carry the name of the child
rather than the main file.
This resolves all three above issues.

This feature is meant to make the editing of books,
thesis documents and lecture notes somewhat more convenient.
However, the package can also be used efficiently for
composing a series of documents (such as exercise sheets)
which are typically distributed individually.
It then assists the author in generating the individual documents
(potentially in different versions)
as well as a document containing the collected series.
Another application is in developing style files
or other kinds of included material
where compilation of the style file could redirect
to a sample or test file.

%%%%%%%%%%%%%%%%%%%%%%%%%%%%%%%%%%%%%%%%%%%%%%%%%%%%%%%%%%%%%%%%%%%%%%%%%%%%%%%%
%%%%%%%%%%%%%%%%%%%%%%%%%%%%%%%%%%%%%%%%%%%%%%%%%%%%%%%%%%%%%%%%%%%%%%%%%%%%%%%%
\section{Usage}

First of all, the package \textsf{childdoc} is \emph{not} a standard
\LaTeXe{} |.sty| style file! Therefore it needs to be invoked in
a non-standard way.

%%%%%%%%%%%%%%%%%%%%%%%%%%%%%%%%%%%%%%%%%%%%%%%%%%%%%%%%%%%%%%%%%%%%%%%%%%%%%%%%
\subsection{Included Files}
\label{sec:include}

%%%%%%%%%%%%%%%%%%%%%%%%%%%%%%%%%%%%%%%%
\DescribeMacro{\childdocmain}
To use the package, add the commands
\begin{center}
\begin{tabular}{l}
|\input{childdoc.def}|\\
|\childdocmain{}|\\
\end{tabular}
\end{center}
at the very top of the main \LaTeX{} file,
in particular \emph{before} the |\documentclass| statement!
The argument of |\childdocmain| should be left empty
(but it must be present).

%%%%%%%%%%%%%%%%%%%%%%%%%%%%%%%%%%%%%%%%
\DescribeMacro{\childdocof}
Furthermore, add the commands
\begin{center}
\begin{tabular}{l}
|\input{childdoc.def}|\\
|\childdocof{|\textit{main}|}|\\
\end{tabular}
\end{center}
at the top of every child file \textit{child}
which is included by |\include{|\textit{child}|}|
from within the main file
(or at least for those files to be compiled individually).
The argument \textit{main} must be the filename of the main file.

There are a couple of
considerations in setting up the main and child documents:

%%%%%%%%%%%%%%%%%%%%%%%%%%%%%%%%%%%%%%%%
\paragraph{Restrictions.}

Please note the following restrictions:
\begin{itemize}
\item
|\childdocmain| must be called with one argument \textit{main}
to ensure compatibility with earlier version of the package.
It must either be empty (|\childdocmain{}|)
or precisely match the filename of the main file in which it is specified.
See \secref{sec:detection} for further information.
\item
The filename \textit{main} must be specified without the |.tex| extension.
\item
The filename \textit{main} is case sensitive
(even in case-insensitive file systems)
due to internal string comparison.
\item
The argument \textit{main} should be fully expanded, it cannot be a macro.
\item
Subdirectories and special characters should be avoided in filenames.
\item
The command |\childdocmain{|\textit{main}|}| must be followed by a whitespace.
It should not be followed immediately by another command
or by a comment mark `|%|'.
This is because the \TeX{} parser reads the token immediately following
the argument of |\childdocmain| and puts it
at the beginning of every child section;
however, a white\-space is ignored.
\end{itemize}

%%%%%%%%%%%%%%%%%%%%%%%%%%%%%%%%%%%%%%%%
\paragraph{Content of Main File.}

It is advisable to place all content in the child files included by |\include|.
Any output contained in the main file will appear in all child documents
unless suppressed manually;
it cannot be suppressed automatically by the |\includeonly| directive
and thus should normally be avoided.
A method to include some content in the main file
by means of conditional processing is described in \secref{sec:conditional}.

%%%%%%%%%%%%%%%%%%%%%%%%%%%%%%%%%%%%%%%%
\paragraph{Page Numbering.}

When only a part of the document is compiled,
the appropriate numbering of pages
(as well as other status parameters)
is determined from the |.aux| files.
The latter contain information from previous passes.
However this information needs to propagate through
all intermediate child documents.
Therefore the page numbering in child documents may well
be inconsistent until the complete document is compiled at least once.

A useful (if unconventional) way to always ensure a consistent
page numbering is to restart the numbering in each child document
and denote the pages by `\textit{child}|.|\textit{page}'
where \textit{child} represents the chapter/section number of the child file.
This can be achieved by the command
|\numberwithin{page}{|\textit{child}|}|
of the \textsf{amsmath} package
where \textit{child} can be |chapter| or |section|
depending on the chosen structuring.
Alternatively, one can modify the macro |\thepage| appropriately
and reset the counter |page| at the start of each child file.

%%%%%%%%%%%%%%%%%%%%%%%%%%%%%%%%%%%%%%%%%%%%%%%%%%%%%%%%%%%%%%%%%%%%%%%%%%%%%%%%
\subsection{Conditional Processing}
\label{sec:conditional}

The package provides a mechanism to compile different versions
of a document. To customise the versions further some conditional processing
can come in handy to distinguish which version is being compiled.
The package provides two macros to describe the compilation context:

%%%%%%%%%%%%%%%%%%%%%%%%%%%%%%%%%%%%%%%%
\DescribeMacro{\ifchilddoc}
The conditional |\ifchilddoc| distinguishes between the compilation of
child documents and the main document:
%
\begin{center}
|\ifchilddoc |\textit{child-code}| |[|\||else |\textit{main-code}]| \||fi|
\end{center}

%%%%%%%%%%%%%%%%%%%%%%%%%%%%%%%%%%%%%%%%
\DescribeMacro{\childdocname}
\DescribeMacro{\childdocjob}
The macro |\childdocname| contains the filename (without extension)
of the main or child file being processed.
Note that |\childdocjob| will always contain the name of the main file.

%%%%%%%%%%%%%%%%%%%%%%%%%%%%%%%%%%%%%%%%
\paragraph{Title Page.}

Conditional processing can be used to include a title or banner page
in the main document when proper precautions are taken.
Importantly, the code in the main file should ensure that the page counter
(as well as other status parameters which are stored in the |.aux| files)
takes the same value after the conditional processing.
Otherwise the page numbers may take divergent values
depending on which part is compiled.

For example, a title page could be declared by:
%
\begin{center}
\begin{tabular}{l}
|\ifchilddoc\||else|\\
|\addtocounter{page}{-1}|\\
\textit{code for title page}\\
|\newpage|\\
|\||fi|
\end{tabular}
\end{center}
%
A banner page for the child documents can be generated by:
%
\begin{center}
\begin{tabular}{l}
|\ifchilddoc|\\
|\addtocounter{page}{-1}|\\
\textit{code for banner page}\\
|\newpage|\\
|\||fi|
\end{tabular}
\end{center}
%
Here one could write a message such as:
\begin{center}
|This is the part \childdocname{} of \childdocjob{}.|
\end{center}

%%%%%%%%%%%%%%%%%%%%%%%%%%%%%%%%%%%%%%%%%%%%%%%%%%%%%%%%%%%%%%%%%%%%%%%%%%%%%%%%
\subsection{Flags}
\label{sec:flags}

The package makes it easy to generate different versions
of the main or child documents.
To this end compilation flags can be defined
and assigned different default values.
They will be particularly useful in conjunction
with the forwarding mechanism described in \secref{sec:forward}.

For example, it may be useful to have a flag |\version|
which can be set to |draft| or |final|.
The document source will contain some conditional code
depending on the value of |\version|.
Suppose further, the flag should default to |final| for the main file
and to |draft| for child files
which is a natural assignment for editing the document.
This is achieved by placing the following code
in the preamble of the main document
(below the |\childdocmain| directive):
%
\begin{center}
\begin{tabular}{l}
|\ifchilddoc|\\
|\providecommand{\version}{draft}|\\
|\||else|\\
|\providecommand{\version}{final}|\\
|\||fi|
\end{tabular}
\end{center}
%
The definition by |\providecommand| makes sure
that previous definitions are not overwritten.
Further statements |\providecommand{\version}{...}|
can thus be added before the above code to override it.

For the main file, one might add a line
(between |\childdocmain| and the above block)
%
\begin{center}
|%\ifchilddoc\||else\providecommand{\version}{draft}\||fi|
\end{center}
%
which can be uncommented to produce a draft version.
Likewise one can add a line to the very top of a child file
(above the |\childdocof{|\textit{main}|}| directive)
%
\begin{center}
|%\providecommand{\version}{final}|
\end{center}
%
which can be uncommented to produce the final version of this child document.

%%%%%%%%%%%%%%%%%%%%%%%%%%%%%%%%%%%%%%%%%%%%%%%%%%%%%%%%%%%%%%%%%%%%%%%%%%%%%%%%
\subsection{Forwarding}
\label{sec:forward}

Different versions of the main or child documents
using compilation flags as described in \secref{sec:flags}
can be (permanently) stored in different files
for convenient compilation, viewing and distribution.
To this end, the package defines a command
to pass on compilation to a different file:

%%%%%%%%%%%%%%%%%%%%%%%%%%%%%%%%%%%%%%%%
\DescribeMacro{\childdocforward}
The command |\childdocforward| redirects processing to
another source file:
%
\begin{center}
\begin{tabular}{l}
|\input{childdoc.def}|\\
|\childdocforward[|\textit{main}|]{|\textit{dest}|}|\\
\end{tabular}
\end{center}
%
The argument \textit{dest} is the destination file
(without extension).
It should be the main file or one of the child files.
Note that further \textsf{childdoc} directives
such as |\childdocof| and |\childdocforward|
in the indicated file will be processed in this form.
The optional argument \textit{main}
passes on directly to the main file \textit{main}
while pretending to compile the child \textit{dest}.
This form behaves as if \textit{dest}
issues |\childdocof{|\textit{main}|}| right away,
and no further \textsf{childdoc} directives will be processed.

%%%%%%%%%%%%%%%%%%%%%%%%%%%%%%%%%%%%%%%%
\DescribeMacro{\...prefix}
In the alternative form |\childdocforwardprefix|,
%
\begin{center}
\begin{tabular}{l}
|\input{childdoc.def}|\\
|\childdocforwardprefix[|\textit{main}|]{|\textit{prefix}|}{|\textit{dest}|}|
\end{tabular}
\end{center}
%
the destination file is determined by a pattern
depending on the current file:
To make this work, the current file must be called
`{\textit{prefix}\hspace{0.2em}\textit{suffix}}'
with \textit{prefix} matching precisely the argument.
Processing is then passed on to the file
`{\textit{dest}\hspace{0.2em}\textit{suffix}}'.
Surely, the same effect is achieved by
directly specifying the
argument `{\textit{dest}\hspace{0.2em}\textit{suffix}}'
in the first form.
However, that requires to set up a different file
for each child. With the alternative form of the command
all these files can have exactly the same content
which simplifies setting them up and maintaining them.

For example, the following file |draft.tex|
with a compilation flag |\version| as described in \secref{sec:flags}
compiles the main document as a draft:
%
\begin{center}
\begin{tabular}{l}
|\def\version{draft}|\\
|\input{childdoc.def}|\\
|\childdocforward{|\textit{main}|}|
\end{tabular}
\end{center}
%
Likewise, the following files |final|\textit{nn}|.tex|
compile the final version of the child document
|child|\textit{nn}|.tex|:
%
\begin{center}
\begin{tabular}{l}
|\def\version{final}|\\
|\input{childdoc.def}|\\
|\childdocforwardprefix{final}{child}|
\end{tabular}
\end{center}
%

Note that when several versions of a main file and/or of each child file
are to be generated, it may be convenient to set up a |Makefile| or
shell script to automatise the process.

%%%%%%%%%%%%%%%%%%%%%%%%%%%%%%%%%%%%%%%%%%%%%%%%%%%%%%%%%%%%%%%%%%%%%%%%%%%%%%%%
\subsection{Command Line Processing}
\label{sec:commandline}

The effect of redirection files can also be achieved by invoking
the \LaTeX{} compiler with a more elaborate command line.
Most conveniently this should be done as part
of a shell script or a |Makefile|.

When using \textsf{childdoc} in the main file, the following
command lines effectively perform a redirection
(note that depending on the shell being used,
backslashes may have to be doubled: `|\|' $\to$ `|\\|'):
%
\begin{center}
|... -jobname "|\textit{target}|" |\\|"|[\textit{flags}]%
|\input{childdoc.def}\childdocforward[|\textit{main}|]{|\textit{dest}|}"|
\end{center}
%
Here \textit{target} is the name of the output file,
\textit{main} is the name of the main file
and \textit{dest} is the name of the main or child file to be processed
(all filenames without extensions).
The optional argument \textit{main} can be omitted
if \textit{main} matches \textit{dest}.
Optionally, compilation \textit{flags} can be defined via |\def| commands.
This command line makes the \TeX{} engine believe
it is compiling the file \textit{target}
whose content is specified as the latter parameter.
The provided code then forwards the processing to
\textit{main} or \textit{dest} as described in \secref{sec:forward}.

%%%%%%%%%%%%%%%%%%%%%%%%%%%%%%%%%%%%%%%%%%%%%%%%%%%%%%%%%%%%%%%%%%%%%%%%%%%%%%%%
\subsection{Include by Input}
\label{sec:input}

Including child documents by |\include| has some restrictions by design.
Most notably, the content of a child document always occupies
its own set of pages; pages cannot be shared between child documents.
Usually, this behaviour makes perfect sense
because each child document contain an essential part of the document.
However, in some situations it may be desirable to compose
a document from a collection of parts
without having mandatory page breaks between then.
For this case, the package
provides a mechanism to include parts
by |\input| which can also be processed individually.
However, by construction this mechanism
requires manual handling of the content to be output.

%%%%%%%%%%%%%%%%%%%%%%%%%%%%%%%%%%%%%%%%
\DescribeMacro{\ifchilddocmanual}
The main file should be prepared as usual, see \secref{sec:include}.
However, the document body must make a distinction
between processing of an individual part and of the main document, e.g.:
%
\begin{center}
\begin{tabular}{l}
|\ifchilddocmanual|\\
|\input{\childdocname}|\\
|\||else|\\
\textit{document body with }|\input{|\textit{part}|}|\\
|\||fi|
\end{tabular}
\end{center}
%
The conditional |\ifchilddocmanual| is true whenever
a part to be included by |\input| is being compiled,
and the name of the part is stored in |\childdocname|.

%%%%%%%%%%%%%%%%%%%%%%%%%%%%%%%%%%%%%%%%
\DescribeMacro{\childdocby}
Each part to be included by |\input| should start with:
%
\begin{center}
\begin{tabular}{l}
|\input{childdoc.def}|\\
|\childdocby{|\textit{main}|}|\\
\end{tabular}
\end{center}
%
The directive |\childdocby| is similar to |\childdocof|
described in \secref{sec:include},
but the subsequent selection of content must be done manually.
To that end, both |\ifchilddoc| and |\ifchilddocmanual|
will be true upon processing of a part,
and the name of the part is stored in |\childdocname|.
Note that |\jobname| will be set to the filename of the current part
so that each part receives an individual |.aux| file
that does not interfere with the |.aux| file(s) of the main document.
This behaviour can be altered by the alternative form
|\childdocby[*]{|\textit{main}|}| (with a non-empty optional argument)
which uses the |.aux| file of the main document
by setting |\jobname| to \textit{main}.

%%%%%%%%%%%%%%%%%%%%%%%%%%%%%%%%%%%%%%%%%%%%%%%%%%%%%%%%%%%%%%%%%%%%%%%%%%%%%%%%
\subsection{Driver Development}
\label{sec:driver}

The \textsf{childdoc} mechanism can also be use for the development
of definition files such as \LaTeX{} styles or classes.
This case differs from the above setup with multiple parts
included by |\include| in that no |\includeonly| should be invoked.
This can be achieved by starting the include file
(before |\ProvidesPackage|) with:
%
\begin{center}
\begin{tabular}{l}
|\input{childdoc.def}|\\
|\childdocforward{|\textit{main}|}|\\
\end{tabular}
\end{center}
%
or alternatively with:
%
\begin{center}
\begin{tabular}{l}
|\input{childdoc.def}|\\
|\childdocby{|\textit{main}|}|\\
\end{tabular}
\end{center}
%
Both forms have slightly different effects as described above.
The main file is prepared as usual, see \secref{sec:include}.

%%%%%%%%%%%%%%%%%%%%%%%%%%%%%%%%%%%%%%%%%%%%%%%%%%%%%%%%%%%%%%%%%%%%%%%%%%%%%%%%
\subsection{Legacy Detection}
\label{sec:detection}

The directive |\childdocmain| in the main file can detect
whether the complete document or merely a child is to be compiled
even without using the directive |\childdocof|.
This method is deprecated because it is less robust
and there is no compelling reason to use it;
it is merely provided for backward compatibility
and it may be removed in future versions.

If the detection mechanism is to be used,
it is mandatory to correctly specify
the filename of the main file as the argument of |\childdocmain|:
%
\begin{center}
\begin{tabular}{l}
|\input{childdoc.def}|\\
|\childdocmain{|\textit{main}|}|\\
\end{tabular}
\end{center}
%
If |\jobname| does not match the argument \textit{main} of |\childdocmain|,
it is assumed that |\jobname| points to the child file to be compiled.
When using |\childdocmain| with the main file specified as argument,
it suffices to start a child file
with just |\input{|\textit{main}|}|
without loading of the package and using |\childdocof|.
If instead all processing is done
with the appropriate \textsf{childdoc} directives,
the argument of \textit{main} of |\childdocmain| can be empty.

An alternative version of the command line processing described
in \secref{sec:commandline} using the detection mechanism reads:
%
\begin{center}
|... -jobname "|\textit{target}|" "|[\textit{flags}]%
[|\def\jobname{|\textit{dest}|}|]|\input{|\textit{main}|}"|
\end{center}

%%%%%%%%%%%%%%%%%%%%%%%%%%%%%%%%%%%%%%%%%%%%%%%%%%%%%%%%%%%%%%%%%%%%%%%%%%%%%%%%
\subsection{Manual Code}
\label{sec:manual}

In case one cannot be certain whether the definitions file |childdoc.def|
is installed on the target \TeX{} distribution
and one prefers not to ship it,
it is conceivable to paste a few relevant commands into the sources.

To that end, drop all statements |\input{childdoc.def}|
and perform the replacements as outlined below.
Instead of |\childdocmain{|\textit{main}|}| add the following code
to the top of the main file:
%
\begin{center}
\begin{tabular}{l}
|\||ifdefined\childdocname\endinput\||fi\newif\ifchilddoc|\\
|\edef\childdocname{\scantokens\expandafter{\jobname\noexpand}}|\\
|\def\childdocmain{|\textit{main}|}\||ifx\childdocmain\childdocname\||else|\\
|\childdoctrue\includeonly{\childdocname}\let\jobname\childdocmain\||fi|\\
\end{tabular}
\end{center}
%
Instead of |\childdocof{|\textit{main}|}| just include the main file
at the top of each child file:
%
\begin{center}
|\input{|\textit{main}|}|
\end{center}
%
A simple redirection |\childdocforward{|\textit{dest}|}| is achieved by:
%
\begin{center}
|\def\jobname{|\textit{dest}|}\input{\jobname}|
\end{center}
%
The redirection with prefix
|\childdocforwardprefix[|\textit{prefix}|]{|\textit{dest}|}|
is accomplished by:
%
\begin{center}
\begin{tabular}{l}
|{\edef\jobname{\scantokens\expandafter{\jobname\noexpand}}|\\
|\def\redirectjob |\textit{prefix}|#1~~~{\gdef\jobname{|\textit{dest}|#1}}|\\
|\expandafter\redirectjob\jobname~~~}\input{\jobname}|
\end{tabular}
\end{center}

In an alternative approach,
child documents can be compiled by a specific command line
without additional code or specific definitions:
%
\begin{center}
|... -jobname "|\textit{target}|" "|[\textit{flags}]%
|\includeonly{|\textit{dest}|}\input{|\textit{main}|}"|
\end{center}
%

%%%%%%%%%%%%%%%%%%%%%%%%%%%%%%%%%%%%%%%%%%%%%%%%%%%%%%%%%%%%%%%%%%%%%%%%%%%%%%%%
%%%%%%%%%%%%%%%%%%%%%%%%%%%%%%%%%%%%%%%%%%%%%%%%%%%%%%%%%%%%%%%%%%%%%%%%%%%%%%%%
\section{Information}

%%%%%%%%%%%%%%%%%%%%%%%%%%%%%%%%%%%%%%%%%%%%%%%%%%%%%%%%%%%%%%%%%%%%%%%%%%%%%%%%
\subsection{Copyright}

Copyright \copyright{} 2017--2018 Niklas Beisert

This work may be distributed and/or modified under the
conditions of the \LaTeX{} Project Public License, either version 1.3
of this license or (at your option) any later version.
The latest version of this license is in
  \url{http://www.latex-project.org/lppl.txt}
and version 1.3 or later is part of all distributions of \LaTeX{}
version 2005/12/01 or later.

This work has the LPPL maintenance status `maintained'.

The Current Maintainer of this work is Niklas Beisert.

This work consists of the files |README.txt|, |childdoc.ins| and |childdoc.dtx|
as well as the derived files |childdoc.def|, |cdocsamp.tex|
with |cdocsch1.tex|, |cdocsch2.tex|, |cdocspt3.tex|, |cdocspt4.tex|,
|cdocsdrf.tex|, |cdocsfn1.tex|, |cdocsfn2.tex|
as well as |childdoc.pdf|.

%%%%%%%%%%%%%%%%%%%%%%%%%%%%%%%%%%%%%%%%%%%%%%%%%%%%%%%%%%%%%%%%%%%%%%%%%%%%%%%%
\subsection{Files and Installation}

The package consists of the files:
%
\begin{center}
\begin{tabular}{ll}
    |README.txt|   & readme file \\
    |childdoc.ins| & installation file \\
    |childdoc.dtx| & source file \\
    |childdoc.def| & definition file \\
    |cdocsamp.tex| & sample main file \\
    |cdocsch1.tex| & sample include file \\
    |cdocsch2.tex| & sample include file \\
    |cdocspt3.tex| & sample part file \\
    |cdocspt4.tex| & sample part file \\
    |cdocsdrf.tex| & sample redirection file \\
    |cdocsfn1.tex| & sample redirection file \\
    |cdocsfn2.tex| & sample redirection file \\
    |childdoc.pdf| & manual
\end{tabular}
\end{center}
%
The distribution consists of the files
|README.txt|, |childdoc.ins| and |childdoc.dtx|.
%
\begin{itemize}
\item
Run (pdf)\LaTeX{} on |childdoc.dtx|
to compile the manual |childdoc.pdf| (this file).
\item
Run \LaTeX{} on |childdoc.ins| to create the definitions file |childdoc.def|
and the sample |cdocsamp.tex| with include files
|cdocsch1.tex|, |cdocsch2.tex|, |cdocspt3.tex|, |cdocspt4.tex|,
|cdocsdrf.tex|, |cdocsfn1.tex|, |cdocsfn2.tex|.
Then copy the file |childdoc.def| to an appropriate directory of your \LaTeX{}
distribution, e.g.\ \textit{texmf-root}|/tex/latex/childdoc|.
\end{itemize}

%%%%%%%%%%%%%%%%%%%%%%%%%%%%%%%%%%%%%%%%%%%%%%%%%%%%%%%%%%%%%%%%%%%%%%%%%%%%%%%%
\subsection{Related CTAN Packages}

There are several other packages which offer a similar functionality:
%
\begin{itemize}
\item
The packages
\href{http://ctan.org/pkg/docmute}{\textsf{docmute}},
\href{http://ctan.org/pkg/includex}{\textsf{includex}} and
\href{http://ctan.org/pkg/standalone}{\textsf{standalone}}
provide commands to include only the document body of
a child file thus allowing both files to be compiled individually.
\item
The packages \href{http://ctan.org/pkg/subdocs}{\textsf{subdocs}}
and \href{http://ctan.org/pkg/subfiles}{\textsf{subfiles}}
provide structures in which the main and child documents can be
encapsulated and allowing them to be compiled individually.
The inclusion mechanism is different from the conventional |\include|.
\item
The package \href{http://ctan.org/pkg/combine}{\textsf{combine}}
is an elaborate solution to combine several documents into one.
\end{itemize}
%
See also the CTAN topic \href{http://ctan.org/topic/subdocs}{\textsf{subdocs}}
for further related packages.
The present package differs from the above solutions in that
a document structure constructed with the conventional |\include| mechanism
just needs two extra commands at the top of every file
such that all constituent files can be compiled individually.

%%%%%%%%%%%%%%%%%%%%%%%%%%%%%%%%%%%%%%%%%%%%%%%%%%%%%%%%%%%%%%%%%%%%%%%%%%%%%%%%
%\subsection{Feature Suggestions}
%
%The following is a list of features which may be useful for future
%versions of this package:
%%
%\begin{itemize}
%\item
%\ldots
%\end{itemize}

%%%%%%%%%%%%%%%%%%%%%%%%%%%%%%%%%%%%%%%%%%%%%%%%%%%%%%%%%%%%%%%%%%%%%%%%%%%%%%%%
\subsection{Revision History}

%%%%%%%%%%%%%%%%%%%%%%%%%%%%%%%%%%%%%%%%
\paragraph{v2.0:} 2018/12/30

\begin{itemize}
\item
immediate forward processing
\item
added |\childdocby| mechanism
\item
manual restructured
\end{itemize}

%%%%%%%%%%%%%%%%%%%%%%%%%%%%%%%%%%%%%%%%
\paragraph{v1.6:} 2018/01/17

\begin{itemize}
\item
application for development of include files
\item
corrections to manual
\end{itemize}

%%%%%%%%%%%%%%%%%%%%%%%%%%%%%%%%%%%%%%%%
\paragraph{v1.5:} 2017/05/21

\begin{itemize}
\item
more complete structuring introduced
\item
|\childdocof| introduced
\item
|\childdoc| renamed to |\childdocmain|
\item
|\childredirect| renamed to |\childdocforward| and |\childdocforwardprefix|
and functionality expanded
\end{itemize}

%%%%%%%%%%%%%%%%%%%%%%%%%%%%%%%%%%%%%%%%
\paragraph{v1.0:} 2017/04/27

\begin{itemize}
\item
manual and install package
\item
first version published on CTAN
\end{itemize}

%%%%%%%%%%%%%%%%%%%%%%%%%%%%%%%%%%%%%%%%
\paragraph{v0.6:} 2017/04/26

\begin{itemize}
\item
redirection mechanism added
\end{itemize}

%%%%%%%%%%%%%%%%%%%%%%%%%%%%%%%%%%%%%%%%
\paragraph{v0.5:} 2017/04/26

\begin{itemize}
\item
functionality in definition file
\end{itemize}


%%%%%%%%%%%%%%%%%%%%%%%%%%%%%%%%%%%%%%%%%%%%%%%%%%%%%%%%%%%%%%%%%%%%%%%%%%%%%%%%
%%%%%%%%%%%%%%%%%%%%%%%%%%%%%%%%%%%%%%%%%%%%%%%%%%%%%%%%%%%%%%%%%%%%%%%%%%%%%%%%
%%%%%%%%%%%%%%%%%%%%%%%%%%%%%%%%%%%%%%%%%%%%%%%%%%%%%%%%%%%%%%%%%%%%%%%%%%%%%%%%
\appendix

\settowidth\MacroIndent{\rmfamily\scriptsize 000\ }

 \DocInput{childdoc.dtx}

\end{document}
%</driver>
% \fi
%
% %%%%%%%%%%%%%%%%%%%%%%%%%%%%%%%%%%%%%%%%%%%%%%%%%%%%%%%%%%%%%%%%%%%%%%%%%%%%%%
% %%%%%%%%%%%%%%%%%%%%%%%%%%%%%%%%%%%%%%%%%%%%%%%%%%%%%%%%%%%%%%%%%%%%%%%%%%%%%%
% \section{Sample}
%\iffalse
%<*samplemain>
%\fi
%
% The following presents a sample document
% with two chapters, two parts, a title page,
% a compile flag as well as three forwarding files to set the flag.
% It consists of eight |.tex| files:
% \begin{center}
% \begin{tabular}{ll}
% |cdocsamp.tex|&main file\\
% |cdocsch1.tex|&include file for chapter 1\\
% |cdocsch2.tex|&include file for chapter 2\\
% |cdocspt3.tex|&include file for part 3\\
% |cdocspt4.tex|&include file for part 4\\
% |cdocsdrf.tex|&forwarding file for main file in draft mode\\
% |cdocsfi1.tex|&forwarding file for final version of chapter 1\\
% |cdocsfi2.tex|&forwarding file for final version of chapter 2\\
% \end{tabular}
% \end{center}
% Each of the eight files can be compiled directly by the \LaTeX{} compiler.
%
% %%%%%%%%%%%%%%%%%%%%%%%%%%%%%%%%%%%%%%
% \paragraph{Main File.}
%
% The main file is called |cdocsamp.tex|.
%
% Load the \textsf{childdoc} definitions and
% declare the filename for the main document:
%    \begin{macrocode}
\input{childdoc.def}
\childdocmain{}
%    \end{macrocode}

% Optional override for |\version| flag:
%    \begin{macrocode}
%%\ifchilddoc\else\providecommand{\version}{draft}\fi
%    \end{macrocode}

% Define the default values for the |\version| flag
% (|final| for the main file and |draft| for childs):
%    \begin{macrocode}
\ifchilddoc
\providecommand{\version}{draft}
\else
\providecommand{\version}{final}
\fi
%    \end{macrocode}

% Load the standard document class:
%    \begin{macrocode}
\documentclass[12pt]{article}
%    \end{macrocode}

% Start the document body:
%    \begin{macrocode}
\begin{document}
%    \end{macrocode}

% Declare a title page.
% Print title, part of document being processed and version flag:
%    \begin{macrocode}
\addtocounter{page}{-1}
\begin{center}
{\LARGE\bfseries{}childdoc example\par}
\vspace{1cm}
\ifchilddoc
\ifchilddocmanual part\else chapter\fi:
`\childdocname' of `\childdocjob'\par
\else
main document: `\childdocjob'\par
\fi
version: \version\par
\end{center}
\newpage
%    \end{macrocode}

% Manually include selected file,
% otherwise process as usual:
%    \begin{macrocode}
\ifchilddocmanual
\section*{part `\childdocname'}
\input{\childdocname}
\else
%    \end{macrocode}

% Include the two chapters:
%    \begin{macrocode}
\include{cdocsch1}
\include{cdocsch2}
%    \end{macrocode}

% Include the two parts unless only chapters should be displayed:
%    \begin{macrocode}
\ifchilddoc\else
\section{part three}
\input{cdocspt3}
\section{part four}
\input{cdocspt4}
\fi
%    \end{macrocode}

% Process as usual until here:
%    \begin{macrocode}
\fi
%    \end{macrocode}

% End of document body:
%    \begin{macrocode}
\end{document}
%    \end{macrocode}
%\iffalse
%</samplemain>
%\fi
%
% %%%%%%%%%%%%%%%%%%%%%%%%%%%%%%%%%%%%%%
% \paragraph{Chapter Include Files.}
%
% The include files are called |cdocsch1.tex| and |cdocsch2.tex|.
%
%\iffalse
%<*samplechap1|samplechap2>
%\fi

% Optional override for |\version| flag:
%    \begin{macrocode}
%%\providecommand{\version}{final}
%    \end{macrocode}

% Include the main document:
%    \begin{macrocode}
\input{childdoc.def}
\childdocof{cdocsamp}
%    \end{macrocode}

%\iffalse
%</samplechap1|samplechap2>
%\fi
%
%\iffalse
%<*samplechap1>
%\fi
% Some text for chapter 1:
%    \begin{macrocode}
\section{one}
some text in chapter one
%    \end{macrocode}

%\iffalse
%</samplechap1>
%\fi
% Some text for chapter 2:
%\iffalse
%<*samplechap2>
%\fi
%    \begin{macrocode}
\section{two}
more text in chapter two
%    \end{macrocode}

%\iffalse
%</samplechap2>
%\fi
%
% %%%%%%%%%%%%%%%%%%%%%%%%%%%%%%%%%%%%%%
% \paragraph{Part Include Files.}
%
% The include files are called |cdocspt3.tex| and |cdocspt4.tex|.
%
%\iffalse
%<*samplepart3|samplepart4>
%\fi

% Optional override for |\version| flag:
%    \begin{macrocode}
%%\providecommand{\version}{final}
%    \end{macrocode}

% Include the main document:
%    \begin{macrocode}
\input{childdoc.def}
\childdocby{cdocsamp}
%    \end{macrocode}

%\iffalse
%</samplepart3|samplepart4>
%\fi
%
%\iffalse
%<*samplepart3>
%\fi
% Some text for part 3:
%    \begin{macrocode}
some text in part three
%    \end{macrocode}

%\iffalse
%</samplepart3>
%\fi
% Some text for part 4:
%\iffalse
%<*samplepart4>
%\fi
%    \begin{macrocode}
more text in part four
%    \end{macrocode}

%\iffalse
%</samplepart4>
%\fi
%
% %%%%%%%%%%%%%%%%%%%%%%%%%%%%%%%%%%%%%%
% \paragraph{Forwarding for a Complete Draft.}
%
% The following forwarding file |cdocsdrf.tex|
% compiles the main document in draft mode:
%\iffalse
%<*sampledraft>
%\fi
%    \begin{macrocode}
\def\version{draft}
\input{childdoc.def}
\childdocforward{cdocsamp}
%    \end{macrocode}

%\iffalse
%</sampledraft>
%\fi
%
% %%%%%%%%%%%%%%%%%%%%%%%%%%%%%%%%%%%%%%
% \paragraph{Forwarding for Final Version of the Chapters.}
%
% The following forwarding files |cdocsfn1.tex| and |cdocsfn2.tex|
% (with identical content)
% compile the final versions of the child documents
% |cdocsch1.tex| and |cdocsch2.tex|, respectively:
%\iffalse
%<*samplefinal>
%\fi
%    \begin{macrocode}
\def\version{final}
\input{childdoc.def}
\childdocforwardprefix[cdocsamp]{cdocsfn}{cdocsch}
%    \end{macrocode}

%\iffalse
%</samplefinal>
%\fi
%
% %%%%%%%%%%%%%%%%%%%%%%%%%%%%%%%%%%%%%%
% \paragraph{Command Line Processing.}
%
% The following three command lines generate the output files
% |cdocscld|, |cdocscl1| and |cdocscl2|
% which should be identical to
% |cdocsdrf|, |cdocsch1| and |cdocsfn2|, respectively:
% \begin{center}
% \begin{tabular}{l}
% |latex -jobname cdocscld \|\\
% |  "\def\version{draft}\input{childdoc.def}\childdocforward{cdocsamp}"|\\
% |latex -jobname cdocscl1 \|\\
% |  "\input{childdoc.def}\childdocforward[cdocsamp]{cdocsch1}"|\\
% |latex -jobname cdocscl2 \|\\
% |  "\def\version{final}\input{childdoc.def}\childdocforward{cdocsch2}"|
% \end{tabular}
% \end{center}
% Note that the trailing backslash on each first line
% merely continues the input to the second line
% (for convenient cut ant paste).
% Furthermore, the command |latex| can be replaced by any
% of its alternative versions such as |pdflatex|.
%
% %%%%%%%%%%%%%%%%%%%%%%%%%%%%%%%%%%%%%%%%%%%%%%%%%%%%%%%%%%%%%%%%%%%%%%%%%%%%%%
% %%%%%%%%%%%%%%%%%%%%%%%%%%%%%%%%%%%%%%%%%%%%%%%%%%%%%%%%%%%%%%%%%%%%%%%%%%%%%%
% \section{Implementation}
%\iffalse
%<*package>
%\fi
%
% This section describes the definitions file |childdoc.def|.

% The definitions cannot be loaded using |\usepackage| or |\RequirePackage|
% which has a mechanism to prevent loading a style file more than once.
% When loading the definitions by means of |\input|
% multiple instances have to be prevented manually:
%\iffalse
%This code needs to be before the `\ProvidesFile' directive
%which is defined at the beginning of this file.
%Therefore it is also placed there and commented out here.
%</package>
%<*discard>
%\fi
%    \begin{macrocode}
\ifdefined\childdocmain\endinput\fi
%    \end{macrocode}
%\iffalse
%</discard>
%<*package>
%\fi
%
% \macro{\ifchilddoc}
% \macro{\ifchilddocmanual}
% The conditional |\ifchilddoc| tells whether a
% child (true) or main (false) document is being compiled.
% The conditional |\ifchilddocmanual| tells whether
% the |\includeonly| mechanism is used (false) or
% the selection of child files must be performed manually (true).
% The definitions initialise to false:
%    \begin{macrocode}
\newif\ifchilddoc
\newif\ifchilddocmanual
%    \end{macrocode}

% \macro{\childdocname}
% \macro{\childdocjob}
% The macro |\childdocname| stores the name of the main document
% to be compiled. The macro |\childdocjob| stores the name of
% the document on which the \LaTeX{} compiler was originally invoked.
% The content of |\jobname| cannot be compared
% to filenames specified in the source due to different catcodes.
% The following code rescans |\jobname|, stores the result
% in |\childdocname| and saves a copy in |\childdocjob|:
%    \begin{macrocode}
\edef\childdocname{\scantokens\expandafter{\jobname\noexpand}}
\let\childdocjob\childdocname
%    \end{macrocode}

% \macro{\childdocdisable}
% The macro |\childdocdisable| prevents the main file
% from being processed more than once.
% At this stage, the main document command |\childdocmain|
% is assumed to be called once again where it should do nothing.
% Any subsequent call to it should prevent
% a secondary processing of the main document
% It overwrites the forwarding commands
% |\childdocof| and |\childdocforward|
% with empty macros to prevent further inclusions of the main document:
%    \begin{macrocode}
\newcommand{\childdocdisable}
{
  \renewcommand{\childdocmain}[1]{\renewcommand{\childdocmain}[1]{\endinput}}
  \renewcommand{\childdocof}[1]{}
  \renewcommand{\childdocby}[2][]{}
  \renewcommand{\childdocforward}[2][]{}
  \renewcommand{\childdocdisable}{}
}
%    \end{macrocode}

% \macro{\childdocmain}
% The macro |\childdocmain| is to be called at the top of the main file
% with nothing or the main filename (without extension) as argument.
% First, it breaks loops.
% If the argument is not empty and does not match |\childdocname|
% (which is set by the first inclusion of |childdoc.def|),
% |\ifchilddoc| is set to true, |\includeonly| is applied to the child file
% and |\jobname| is set to the main file
% (for proper handling of |.aux| files):
%    \begin{macrocode}
\newcommand{\childdocmain}[1]
{
  \childdocdisable\childdocmain{}
  \if?#1?\else
    \begingroup
      \def\childdoctmp{#1}
      \ifx\childdoctmp\childdocname
        \def\childdoctmp{}
      \else
        \def\childdoctmp
        {
          \childdoctrue
          \includeonly{\childdocname}
          \def\childdocjob{#1}
          \def\jobname{#1}
        }
      \fi
      \expandafter
    \endgroup
    \childdoctmp
  \fi
}
%    \end{macrocode}

% \macro{\childdocof}
% The command |\childdocof| redirects
% compilation to the main file |#1|.
%    \begin{macrocode}
\newcommand{\childdocof}[1]
{
  \childdocdisable
  \childdoctrue
  \includeonly{\childdocname}
  \def\jobname{#1}
  \def\childdocjob{#1}
  \input{#1}
}
%    \end{macrocode}

% \macro{\childdocby}
% The command |\childdocby| ....
%    \begin{macrocode}
\newcommand{\childdocby}[2][]
{
  \childdocdisable
  \childdoctrue
  \childdocmanualtrue
  \if?#1?\else
    \def\jobname{#2}
  \fi
  \def\childdocjob{#2}
  \input{#2}
  \endinput
}
%    \end{macrocode}

% \macro{\childdocforward}
% The command |\childdocforward| redirects
% compilation to the main file or
% (if the optional argument is given) a child file.
% Parameters are set as if the main file
% or a child file starting with |\childdocof| was compiled.
% Then compilation is handed over to the main file:
%    \begin{macrocode}
\newcommand{\childdocforward}[2][]
{
  \begingroup
    \if?#1?
      \def\childdoctmp
      {
        \def\childdocname{#2}
        \def\childdocjob{#2}
        \def\jobname{#2}
        \input{#2}
        \endinput
      }
    \else
      \def\childdoctmp
      {
        \childdocdisable
        \def\childdocname{#2}
        \childdoctrue
        \includeonly{#2}
        \def\childdocjob{#1}
        \def\jobname{#1}
        \input{#1}
        \endinput
      }
    \fi
    \expandafter
  \endgroup
  \childdoctmp
}
%    \end{macrocode}

% \macro{\childdocforwardprefix}
% The command |\childdocforwardprefix| redirects
% compilation to the main or a child file by means of a pattern.
% The prefix |#1| in the current filename is replaced by |#2|
% and the suffix of the current filename is kept
% (it is assumed that the filename does not contain the substring `|~~~|'
% which is used as a delimiter).
% Compilation is handed over to the new file by |\childdocforward|:
%    \begin{macrocode}
\newcommand{\childdocforwardprefix}[3][]
{
  \begingroup
    \def\childdocextract #2##1~~~{\def\childdoctmp{\childdocforward[#1]{#3##1}}}
    \expandafter\childdocextract\childdocname~~~
    \expandafter
  \endgroup
  \childdoctmp
}
%    \end{macrocode}

% \macro{\childdoc}
% The deprecated macro |\childdoc| is a legacy version of |\childdocmain|:
%    \begin{macrocode}
\newcommand{\childdoc}{\childdocmain}
%    \end{macrocode}

% \macro{\childdocredirect}
% The deprecated macro |\childdocredirect| is a legacy version
% of |\childdocforward| and |\childdocforwardprefix|:
%    \begin{macrocode}
\newcommand{\childdocredirect}[2][]
{
  \begingroup
    \if?#1?
      \def\childdoctmp{\childdocforward{#2}}
    \else
      \def\childdoctmp{\childdocforwardprefix{#1}{#2}}
    \fi
    \expandafter
  \endgroup
  \childdoctmp
}
%    \end{macrocode}

%\iffalse
%</package>
%\fi
%
\endinput
|\\
|\childdocforwardprefix[|\textit{main}|]{|\textit{prefix}|}{|\textit{dest}|}|
\end{tabular}
\end{center}
%
the destination file is determined by a pattern
depending on the current file:
To make this work, the current file must be called
`{\textit{prefix}\hspace{0.2em}\textit{suffix}}'
with \textit{prefix} matching precisely the argument.
Processing is then passed on to the file
`{\textit{dest}\hspace{0.2em}\textit{suffix}}'.
Surely, the same effect is achieved by
directly specifying the
argument `{\textit{dest}\hspace{0.2em}\textit{suffix}}'
in the first form.
However, that requires to set up a different file
for each child. With the alternative form of the command
all these files can have exactly the same content
which simplifies setting them up and maintaining them.

For example, the following file |draft.tex|
with a compilation flag |\version| as described in \secref{sec:flags}
compiles the main document as a draft:
%
\begin{center}
\begin{tabular}{l}
|\def\version{draft}|\\
|% \iffalse
%
% childdoc.dtx Copyright (C) 2017-2018 Niklas Beisert
%
% This work may be distributed and/or modified under the
% conditions of the LaTeX Project Public License, either version 1.3
% of this license or (at your option) any later version.
% The latest version of this license is in
%   http://www.latex-project.org/lppl.txt
% and version 1.3 or later is part of all distributions of LaTeX
% version 2005/12/01 or later.
%
% This work has the LPPL maintenance status `maintained'.
%
% The Current Maintainer of this work is Niklas Beisert.
%
% This work consists of the files childdoc.dtx and childdoc.ins
% and the derived files childdoc.def and cdocsamp.tex with
% cdocsch1.tex, cdocsch2.tex, cdocsdrf.tex, cdocsfn1.tex, cdocsfn2.tex.
%
%<package>\ifdefined\childdocmain\endinput\fi
%<package>\ProvidesFile{childdoc.def}[2018/12/30 v2.0 child document driver]
%<samplemain>\ProvidesFile{cdocsamp.tex}[2018/12/30 v2.0 sample for childdoc]
%<*driver>
%\ProvidesFile{childdoc.drv}[2018/12/30 v2.0 childdoc reference manual file]
\PassOptionsToClass{10pt,a4paper}{article}
\documentclass{ltxdoc}

\usepackage[margin=35mm]{geometry}
\usepackage{hyperref}
\usepackage{hyperxmp}
\usepackage[usenames]{color}

\hypersetup{colorlinks=true}
\hypersetup{pdfstartview=FitH}
\hypersetup{pdfpagemode=UseNone}
\hypersetup{pdfsource={}}
\hypersetup{pdflang={en-UK}}
\hypersetup{pdfcopyright={Copyright 2017-2018 Niklas Beisert.
  This work may be distributed and/or modified under the
  conditions of the LaTeX Project Public License, either version 1.3
  of this license or (at your option) any later version.}}
\hypersetup{pdflicenseurl={http://www.latex-project.org/lppl.txt}}
\hypersetup{pdfcontactaddress={ETH Zurich, ITP, HIT K,
  Wolfgang-Pauli-Strasse 27}}
\hypersetup{pdfcontactpostcode={8093}}
\hypersetup{pdfcontactcity={Zurich}}
\hypersetup{pdfcontactcountry={Switzerland}}
\hypersetup{pdfcontactemail={nbeisert@itp.phys.ethz.ch}}
\hypersetup{pdfcontacturl={http://people.phys.ethz.ch/\xmptilde nbeisert/}}

\newcommand{\secref}[1]{\hyperref[#1]{section \ref*{#1}}}

\parskip1ex
\parindent0pt
\let\olditemize\itemize
\def\itemize{\olditemize\parskip0pt}

\begin{document}

\title{The \textsf{childdoc} Package}
\hypersetup{pdftitle={The childdoc Package}}
\author{Niklas Beisert\\[2ex]
  Institut f\"ur Theoretische Physik\\
  Eidgen\"ossische Technische Hochschule Z\"urich\\
  Wolfgang-Pauli-Strasse 27, 8093 Z\"urich, Switzerland\\[1ex]
  \href{mailto:nbeisert@itp.phys.ethz.ch}
  {\texttt{nbeisert@itp.phys.ethz.ch}}}
\hypersetup{pdfauthor={Niklas Beisert}}
\hypersetup{pdfsubject={Manual for the LaTeX2e Package childdoc}}
\date{30 December 2018, \textsf{v2.0}}
\maketitle

\begin{abstract}\noindent
\textsf{childdoc} is a \LaTeXe{} package
that enables the direct compilation
of document sections included by |\include|
to individual files.
\end{abstract}

\begingroup
\parskip0ex
\tableofcontents
\endgroup

%%%%%%%%%%%%%%%%%%%%%%%%%%%%%%%%%%%%%%%%%%%%%%%%%%%%%%%%%%%%%%%%%%%%%%%%%%%%%%%%
%%%%%%%%%%%%%%%%%%%%%%%%%%%%%%%%%%%%%%%%%%%%%%%%%%%%%%%%%%%%%%%%%%%%%%%%%%%%%%%%
\section{Introduction}

\LaTeX{} provides a mechanism to structure a large document (such as a book)
into a main file and several child files (containing the chapters)
using the |\include| command.
This mechanism is beneficial for documents
which span hundreds of pages in order to
make the source file(s) more manageable.
Moreover, compilation can be restricted to
selected child files by means of the |\includeonly| command.
The latter feature can be used to reduce the compilation time while editing
(this was significantly more useful in the earlier days of \LaTeX{})
or to generate a smaller document which is easier to navigate.
Another application of |\includeonly| is to generate
documents consisting of selected parts of the complete document.

However, there are a few drawbacks of the plain |\include| mechanism:
\begin{itemize}
\item
The child files cannot be compiled on their own,
they can only be compiled via the main file.
A naive editing environment
(such as a text editor with an option
to have the current file processed by \LaTeX)
may require one to switch to the main file before compiling;
attempting to compile the child file produces errors.
\item
The main file must be modified (each time)
to adjust the |\includeonly| command
to the present needs. This easily leaves the main file in a messy state.
\item
The generated document will always carry the filename
of the main document. This is inconvenient if
several child files are to be compiled and
to be kept for distribution.
\end{itemize}

The present package provides a simple interface
to make child files individually compilable by \LaTeX{}.
Compiling a child file then has the same effect as compiling
the main file with an |\includeonly| command
to select the appropriate child.
Moreover the generated document will carry the name of the child
rather than the main file.
This resolves all three above issues.

This feature is meant to make the editing of books,
thesis documents and lecture notes somewhat more convenient.
However, the package can also be used efficiently for
composing a series of documents (such as exercise sheets)
which are typically distributed individually.
It then assists the author in generating the individual documents
(potentially in different versions)
as well as a document containing the collected series.
Another application is in developing style files
or other kinds of included material
where compilation of the style file could redirect
to a sample or test file.

%%%%%%%%%%%%%%%%%%%%%%%%%%%%%%%%%%%%%%%%%%%%%%%%%%%%%%%%%%%%%%%%%%%%%%%%%%%%%%%%
%%%%%%%%%%%%%%%%%%%%%%%%%%%%%%%%%%%%%%%%%%%%%%%%%%%%%%%%%%%%%%%%%%%%%%%%%%%%%%%%
\section{Usage}

First of all, the package \textsf{childdoc} is \emph{not} a standard
\LaTeXe{} |.sty| style file! Therefore it needs to be invoked in
a non-standard way.

%%%%%%%%%%%%%%%%%%%%%%%%%%%%%%%%%%%%%%%%%%%%%%%%%%%%%%%%%%%%%%%%%%%%%%%%%%%%%%%%
\subsection{Included Files}
\label{sec:include}

%%%%%%%%%%%%%%%%%%%%%%%%%%%%%%%%%%%%%%%%
\DescribeMacro{\childdocmain}
To use the package, add the commands
\begin{center}
\begin{tabular}{l}
|\input{childdoc.def}|\\
|\childdocmain{}|\\
\end{tabular}
\end{center}
at the very top of the main \LaTeX{} file,
in particular \emph{before} the |\documentclass| statement!
The argument of |\childdocmain| should be left empty
(but it must be present).

%%%%%%%%%%%%%%%%%%%%%%%%%%%%%%%%%%%%%%%%
\DescribeMacro{\childdocof}
Furthermore, add the commands
\begin{center}
\begin{tabular}{l}
|\input{childdoc.def}|\\
|\childdocof{|\textit{main}|}|\\
\end{tabular}
\end{center}
at the top of every child file \textit{child}
which is included by |\include{|\textit{child}|}|
from within the main file
(or at least for those files to be compiled individually).
The argument \textit{main} must be the filename of the main file.

There are a couple of
considerations in setting up the main and child documents:

%%%%%%%%%%%%%%%%%%%%%%%%%%%%%%%%%%%%%%%%
\paragraph{Restrictions.}

Please note the following restrictions:
\begin{itemize}
\item
|\childdocmain| must be called with one argument \textit{main}
to ensure compatibility with earlier version of the package.
It must either be empty (|\childdocmain{}|)
or precisely match the filename of the main file in which it is specified.
See \secref{sec:detection} for further information.
\item
The filename \textit{main} must be specified without the |.tex| extension.
\item
The filename \textit{main} is case sensitive
(even in case-insensitive file systems)
due to internal string comparison.
\item
The argument \textit{main} should be fully expanded, it cannot be a macro.
\item
Subdirectories and special characters should be avoided in filenames.
\item
The command |\childdocmain{|\textit{main}|}| must be followed by a whitespace.
It should not be followed immediately by another command
or by a comment mark `|%|'.
This is because the \TeX{} parser reads the token immediately following
the argument of |\childdocmain| and puts it
at the beginning of every child section;
however, a white\-space is ignored.
\end{itemize}

%%%%%%%%%%%%%%%%%%%%%%%%%%%%%%%%%%%%%%%%
\paragraph{Content of Main File.}

It is advisable to place all content in the child files included by |\include|.
Any output contained in the main file will appear in all child documents
unless suppressed manually;
it cannot be suppressed automatically by the |\includeonly| directive
and thus should normally be avoided.
A method to include some content in the main file
by means of conditional processing is described in \secref{sec:conditional}.

%%%%%%%%%%%%%%%%%%%%%%%%%%%%%%%%%%%%%%%%
\paragraph{Page Numbering.}

When only a part of the document is compiled,
the appropriate numbering of pages
(as well as other status parameters)
is determined from the |.aux| files.
The latter contain information from previous passes.
However this information needs to propagate through
all intermediate child documents.
Therefore the page numbering in child documents may well
be inconsistent until the complete document is compiled at least once.

A useful (if unconventional) way to always ensure a consistent
page numbering is to restart the numbering in each child document
and denote the pages by `\textit{child}|.|\textit{page}'
where \textit{child} represents the chapter/section number of the child file.
This can be achieved by the command
|\numberwithin{page}{|\textit{child}|}|
of the \textsf{amsmath} package
where \textit{child} can be |chapter| or |section|
depending on the chosen structuring.
Alternatively, one can modify the macro |\thepage| appropriately
and reset the counter |page| at the start of each child file.

%%%%%%%%%%%%%%%%%%%%%%%%%%%%%%%%%%%%%%%%%%%%%%%%%%%%%%%%%%%%%%%%%%%%%%%%%%%%%%%%
\subsection{Conditional Processing}
\label{sec:conditional}

The package provides a mechanism to compile different versions
of a document. To customise the versions further some conditional processing
can come in handy to distinguish which version is being compiled.
The package provides two macros to describe the compilation context:

%%%%%%%%%%%%%%%%%%%%%%%%%%%%%%%%%%%%%%%%
\DescribeMacro{\ifchilddoc}
The conditional |\ifchilddoc| distinguishes between the compilation of
child documents and the main document:
%
\begin{center}
|\ifchilddoc |\textit{child-code}| |[|\||else |\textit{main-code}]| \||fi|
\end{center}

%%%%%%%%%%%%%%%%%%%%%%%%%%%%%%%%%%%%%%%%
\DescribeMacro{\childdocname}
\DescribeMacro{\childdocjob}
The macro |\childdocname| contains the filename (without extension)
of the main or child file being processed.
Note that |\childdocjob| will always contain the name of the main file.

%%%%%%%%%%%%%%%%%%%%%%%%%%%%%%%%%%%%%%%%
\paragraph{Title Page.}

Conditional processing can be used to include a title or banner page
in the main document when proper precautions are taken.
Importantly, the code in the main file should ensure that the page counter
(as well as other status parameters which are stored in the |.aux| files)
takes the same value after the conditional processing.
Otherwise the page numbers may take divergent values
depending on which part is compiled.

For example, a title page could be declared by:
%
\begin{center}
\begin{tabular}{l}
|\ifchilddoc\||else|\\
|\addtocounter{page}{-1}|\\
\textit{code for title page}\\
|\newpage|\\
|\||fi|
\end{tabular}
\end{center}
%
A banner page for the child documents can be generated by:
%
\begin{center}
\begin{tabular}{l}
|\ifchilddoc|\\
|\addtocounter{page}{-1}|\\
\textit{code for banner page}\\
|\newpage|\\
|\||fi|
\end{tabular}
\end{center}
%
Here one could write a message such as:
\begin{center}
|This is the part \childdocname{} of \childdocjob{}.|
\end{center}

%%%%%%%%%%%%%%%%%%%%%%%%%%%%%%%%%%%%%%%%%%%%%%%%%%%%%%%%%%%%%%%%%%%%%%%%%%%%%%%%
\subsection{Flags}
\label{sec:flags}

The package makes it easy to generate different versions
of the main or child documents.
To this end compilation flags can be defined
and assigned different default values.
They will be particularly useful in conjunction
with the forwarding mechanism described in \secref{sec:forward}.

For example, it may be useful to have a flag |\version|
which can be set to |draft| or |final|.
The document source will contain some conditional code
depending on the value of |\version|.
Suppose further, the flag should default to |final| for the main file
and to |draft| for child files
which is a natural assignment for editing the document.
This is achieved by placing the following code
in the preamble of the main document
(below the |\childdocmain| directive):
%
\begin{center}
\begin{tabular}{l}
|\ifchilddoc|\\
|\providecommand{\version}{draft}|\\
|\||else|\\
|\providecommand{\version}{final}|\\
|\||fi|
\end{tabular}
\end{center}
%
The definition by |\providecommand| makes sure
that previous definitions are not overwritten.
Further statements |\providecommand{\version}{...}|
can thus be added before the above code to override it.

For the main file, one might add a line
(between |\childdocmain| and the above block)
%
\begin{center}
|%\ifchilddoc\||else\providecommand{\version}{draft}\||fi|
\end{center}
%
which can be uncommented to produce a draft version.
Likewise one can add a line to the very top of a child file
(above the |\childdocof{|\textit{main}|}| directive)
%
\begin{center}
|%\providecommand{\version}{final}|
\end{center}
%
which can be uncommented to produce the final version of this child document.

%%%%%%%%%%%%%%%%%%%%%%%%%%%%%%%%%%%%%%%%%%%%%%%%%%%%%%%%%%%%%%%%%%%%%%%%%%%%%%%%
\subsection{Forwarding}
\label{sec:forward}

Different versions of the main or child documents
using compilation flags as described in \secref{sec:flags}
can be (permanently) stored in different files
for convenient compilation, viewing and distribution.
To this end, the package defines a command
to pass on compilation to a different file:

%%%%%%%%%%%%%%%%%%%%%%%%%%%%%%%%%%%%%%%%
\DescribeMacro{\childdocforward}
The command |\childdocforward| redirects processing to
another source file:
%
\begin{center}
\begin{tabular}{l}
|\input{childdoc.def}|\\
|\childdocforward[|\textit{main}|]{|\textit{dest}|}|\\
\end{tabular}
\end{center}
%
The argument \textit{dest} is the destination file
(without extension).
It should be the main file or one of the child files.
Note that further \textsf{childdoc} directives
such as |\childdocof| and |\childdocforward|
in the indicated file will be processed in this form.
The optional argument \textit{main}
passes on directly to the main file \textit{main}
while pretending to compile the child \textit{dest}.
This form behaves as if \textit{dest}
issues |\childdocof{|\textit{main}|}| right away,
and no further \textsf{childdoc} directives will be processed.

%%%%%%%%%%%%%%%%%%%%%%%%%%%%%%%%%%%%%%%%
\DescribeMacro{\...prefix}
In the alternative form |\childdocforwardprefix|,
%
\begin{center}
\begin{tabular}{l}
|\input{childdoc.def}|\\
|\childdocforwardprefix[|\textit{main}|]{|\textit{prefix}|}{|\textit{dest}|}|
\end{tabular}
\end{center}
%
the destination file is determined by a pattern
depending on the current file:
To make this work, the current file must be called
`{\textit{prefix}\hspace{0.2em}\textit{suffix}}'
with \textit{prefix} matching precisely the argument.
Processing is then passed on to the file
`{\textit{dest}\hspace{0.2em}\textit{suffix}}'.
Surely, the same effect is achieved by
directly specifying the
argument `{\textit{dest}\hspace{0.2em}\textit{suffix}}'
in the first form.
However, that requires to set up a different file
for each child. With the alternative form of the command
all these files can have exactly the same content
which simplifies setting them up and maintaining them.

For example, the following file |draft.tex|
with a compilation flag |\version| as described in \secref{sec:flags}
compiles the main document as a draft:
%
\begin{center}
\begin{tabular}{l}
|\def\version{draft}|\\
|\input{childdoc.def}|\\
|\childdocforward{|\textit{main}|}|
\end{tabular}
\end{center}
%
Likewise, the following files |final|\textit{nn}|.tex|
compile the final version of the child document
|child|\textit{nn}|.tex|:
%
\begin{center}
\begin{tabular}{l}
|\def\version{final}|\\
|\input{childdoc.def}|\\
|\childdocforwardprefix{final}{child}|
\end{tabular}
\end{center}
%

Note that when several versions of a main file and/or of each child file
are to be generated, it may be convenient to set up a |Makefile| or
shell script to automatise the process.

%%%%%%%%%%%%%%%%%%%%%%%%%%%%%%%%%%%%%%%%%%%%%%%%%%%%%%%%%%%%%%%%%%%%%%%%%%%%%%%%
\subsection{Command Line Processing}
\label{sec:commandline}

The effect of redirection files can also be achieved by invoking
the \LaTeX{} compiler with a more elaborate command line.
Most conveniently this should be done as part
of a shell script or a |Makefile|.

When using \textsf{childdoc} in the main file, the following
command lines effectively perform a redirection
(note that depending on the shell being used,
backslashes may have to be doubled: `|\|' $\to$ `|\\|'):
%
\begin{center}
|... -jobname "|\textit{target}|" |\\|"|[\textit{flags}]%
|\input{childdoc.def}\childdocforward[|\textit{main}|]{|\textit{dest}|}"|
\end{center}
%
Here \textit{target} is the name of the output file,
\textit{main} is the name of the main file
and \textit{dest} is the name of the main or child file to be processed
(all filenames without extensions).
The optional argument \textit{main} can be omitted
if \textit{main} matches \textit{dest}.
Optionally, compilation \textit{flags} can be defined via |\def| commands.
This command line makes the \TeX{} engine believe
it is compiling the file \textit{target}
whose content is specified as the latter parameter.
The provided code then forwards the processing to
\textit{main} or \textit{dest} as described in \secref{sec:forward}.

%%%%%%%%%%%%%%%%%%%%%%%%%%%%%%%%%%%%%%%%%%%%%%%%%%%%%%%%%%%%%%%%%%%%%%%%%%%%%%%%
\subsection{Include by Input}
\label{sec:input}

Including child documents by |\include| has some restrictions by design.
Most notably, the content of a child document always occupies
its own set of pages; pages cannot be shared between child documents.
Usually, this behaviour makes perfect sense
because each child document contain an essential part of the document.
However, in some situations it may be desirable to compose
a document from a collection of parts
without having mandatory page breaks between then.
For this case, the package
provides a mechanism to include parts
by |\input| which can also be processed individually.
However, by construction this mechanism
requires manual handling of the content to be output.

%%%%%%%%%%%%%%%%%%%%%%%%%%%%%%%%%%%%%%%%
\DescribeMacro{\ifchilddocmanual}
The main file should be prepared as usual, see \secref{sec:include}.
However, the document body must make a distinction
between processing of an individual part and of the main document, e.g.:
%
\begin{center}
\begin{tabular}{l}
|\ifchilddocmanual|\\
|\input{\childdocname}|\\
|\||else|\\
\textit{document body with }|\input{|\textit{part}|}|\\
|\||fi|
\end{tabular}
\end{center}
%
The conditional |\ifchilddocmanual| is true whenever
a part to be included by |\input| is being compiled,
and the name of the part is stored in |\childdocname|.

%%%%%%%%%%%%%%%%%%%%%%%%%%%%%%%%%%%%%%%%
\DescribeMacro{\childdocby}
Each part to be included by |\input| should start with:
%
\begin{center}
\begin{tabular}{l}
|\input{childdoc.def}|\\
|\childdocby{|\textit{main}|}|\\
\end{tabular}
\end{center}
%
The directive |\childdocby| is similar to |\childdocof|
described in \secref{sec:include},
but the subsequent selection of content must be done manually.
To that end, both |\ifchilddoc| and |\ifchilddocmanual|
will be true upon processing of a part,
and the name of the part is stored in |\childdocname|.
Note that |\jobname| will be set to the filename of the current part
so that each part receives an individual |.aux| file
that does not interfere with the |.aux| file(s) of the main document.
This behaviour can be altered by the alternative form
|\childdocby[*]{|\textit{main}|}| (with a non-empty optional argument)
which uses the |.aux| file of the main document
by setting |\jobname| to \textit{main}.

%%%%%%%%%%%%%%%%%%%%%%%%%%%%%%%%%%%%%%%%%%%%%%%%%%%%%%%%%%%%%%%%%%%%%%%%%%%%%%%%
\subsection{Driver Development}
\label{sec:driver}

The \textsf{childdoc} mechanism can also be use for the development
of definition files such as \LaTeX{} styles or classes.
This case differs from the above setup with multiple parts
included by |\include| in that no |\includeonly| should be invoked.
This can be achieved by starting the include file
(before |\ProvidesPackage|) with:
%
\begin{center}
\begin{tabular}{l}
|\input{childdoc.def}|\\
|\childdocforward{|\textit{main}|}|\\
\end{tabular}
\end{center}
%
or alternatively with:
%
\begin{center}
\begin{tabular}{l}
|\input{childdoc.def}|\\
|\childdocby{|\textit{main}|}|\\
\end{tabular}
\end{center}
%
Both forms have slightly different effects as described above.
The main file is prepared as usual, see \secref{sec:include}.

%%%%%%%%%%%%%%%%%%%%%%%%%%%%%%%%%%%%%%%%%%%%%%%%%%%%%%%%%%%%%%%%%%%%%%%%%%%%%%%%
\subsection{Legacy Detection}
\label{sec:detection}

The directive |\childdocmain| in the main file can detect
whether the complete document or merely a child is to be compiled
even without using the directive |\childdocof|.
This method is deprecated because it is less robust
and there is no compelling reason to use it;
it is merely provided for backward compatibility
and it may be removed in future versions.

If the detection mechanism is to be used,
it is mandatory to correctly specify
the filename of the main file as the argument of |\childdocmain|:
%
\begin{center}
\begin{tabular}{l}
|\input{childdoc.def}|\\
|\childdocmain{|\textit{main}|}|\\
\end{tabular}
\end{center}
%
If |\jobname| does not match the argument \textit{main} of |\childdocmain|,
it is assumed that |\jobname| points to the child file to be compiled.
When using |\childdocmain| with the main file specified as argument,
it suffices to start a child file
with just |\input{|\textit{main}|}|
without loading of the package and using |\childdocof|.
If instead all processing is done
with the appropriate \textsf{childdoc} directives,
the argument of \textit{main} of |\childdocmain| can be empty.

An alternative version of the command line processing described
in \secref{sec:commandline} using the detection mechanism reads:
%
\begin{center}
|... -jobname "|\textit{target}|" "|[\textit{flags}]%
[|\def\jobname{|\textit{dest}|}|]|\input{|\textit{main}|}"|
\end{center}

%%%%%%%%%%%%%%%%%%%%%%%%%%%%%%%%%%%%%%%%%%%%%%%%%%%%%%%%%%%%%%%%%%%%%%%%%%%%%%%%
\subsection{Manual Code}
\label{sec:manual}

In case one cannot be certain whether the definitions file |childdoc.def|
is installed on the target \TeX{} distribution
and one prefers not to ship it,
it is conceivable to paste a few relevant commands into the sources.

To that end, drop all statements |\input{childdoc.def}|
and perform the replacements as outlined below.
Instead of |\childdocmain{|\textit{main}|}| add the following code
to the top of the main file:
%
\begin{center}
\begin{tabular}{l}
|\||ifdefined\childdocname\endinput\||fi\newif\ifchilddoc|\\
|\edef\childdocname{\scantokens\expandafter{\jobname\noexpand}}|\\
|\def\childdocmain{|\textit{main}|}\||ifx\childdocmain\childdocname\||else|\\
|\childdoctrue\includeonly{\childdocname}\let\jobname\childdocmain\||fi|\\
\end{tabular}
\end{center}
%
Instead of |\childdocof{|\textit{main}|}| just include the main file
at the top of each child file:
%
\begin{center}
|\input{|\textit{main}|}|
\end{center}
%
A simple redirection |\childdocforward{|\textit{dest}|}| is achieved by:
%
\begin{center}
|\def\jobname{|\textit{dest}|}\input{\jobname}|
\end{center}
%
The redirection with prefix
|\childdocforwardprefix[|\textit{prefix}|]{|\textit{dest}|}|
is accomplished by:
%
\begin{center}
\begin{tabular}{l}
|{\edef\jobname{\scantokens\expandafter{\jobname\noexpand}}|\\
|\def\redirectjob |\textit{prefix}|#1~~~{\gdef\jobname{|\textit{dest}|#1}}|\\
|\expandafter\redirectjob\jobname~~~}\input{\jobname}|
\end{tabular}
\end{center}

In an alternative approach,
child documents can be compiled by a specific command line
without additional code or specific definitions:
%
\begin{center}
|... -jobname "|\textit{target}|" "|[\textit{flags}]%
|\includeonly{|\textit{dest}|}\input{|\textit{main}|}"|
\end{center}
%

%%%%%%%%%%%%%%%%%%%%%%%%%%%%%%%%%%%%%%%%%%%%%%%%%%%%%%%%%%%%%%%%%%%%%%%%%%%%%%%%
%%%%%%%%%%%%%%%%%%%%%%%%%%%%%%%%%%%%%%%%%%%%%%%%%%%%%%%%%%%%%%%%%%%%%%%%%%%%%%%%
\section{Information}

%%%%%%%%%%%%%%%%%%%%%%%%%%%%%%%%%%%%%%%%%%%%%%%%%%%%%%%%%%%%%%%%%%%%%%%%%%%%%%%%
\subsection{Copyright}

Copyright \copyright{} 2017--2018 Niklas Beisert

This work may be distributed and/or modified under the
conditions of the \LaTeX{} Project Public License, either version 1.3
of this license or (at your option) any later version.
The latest version of this license is in
  \url{http://www.latex-project.org/lppl.txt}
and version 1.3 or later is part of all distributions of \LaTeX{}
version 2005/12/01 or later.

This work has the LPPL maintenance status `maintained'.

The Current Maintainer of this work is Niklas Beisert.

This work consists of the files |README.txt|, |childdoc.ins| and |childdoc.dtx|
as well as the derived files |childdoc.def|, |cdocsamp.tex|
with |cdocsch1.tex|, |cdocsch2.tex|, |cdocspt3.tex|, |cdocspt4.tex|,
|cdocsdrf.tex|, |cdocsfn1.tex|, |cdocsfn2.tex|
as well as |childdoc.pdf|.

%%%%%%%%%%%%%%%%%%%%%%%%%%%%%%%%%%%%%%%%%%%%%%%%%%%%%%%%%%%%%%%%%%%%%%%%%%%%%%%%
\subsection{Files and Installation}

The package consists of the files:
%
\begin{center}
\begin{tabular}{ll}
    |README.txt|   & readme file \\
    |childdoc.ins| & installation file \\
    |childdoc.dtx| & source file \\
    |childdoc.def| & definition file \\
    |cdocsamp.tex| & sample main file \\
    |cdocsch1.tex| & sample include file \\
    |cdocsch2.tex| & sample include file \\
    |cdocspt3.tex| & sample part file \\
    |cdocspt4.tex| & sample part file \\
    |cdocsdrf.tex| & sample redirection file \\
    |cdocsfn1.tex| & sample redirection file \\
    |cdocsfn2.tex| & sample redirection file \\
    |childdoc.pdf| & manual
\end{tabular}
\end{center}
%
The distribution consists of the files
|README.txt|, |childdoc.ins| and |childdoc.dtx|.
%
\begin{itemize}
\item
Run (pdf)\LaTeX{} on |childdoc.dtx|
to compile the manual |childdoc.pdf| (this file).
\item
Run \LaTeX{} on |childdoc.ins| to create the definitions file |childdoc.def|
and the sample |cdocsamp.tex| with include files
|cdocsch1.tex|, |cdocsch2.tex|, |cdocspt3.tex|, |cdocspt4.tex|,
|cdocsdrf.tex|, |cdocsfn1.tex|, |cdocsfn2.tex|.
Then copy the file |childdoc.def| to an appropriate directory of your \LaTeX{}
distribution, e.g.\ \textit{texmf-root}|/tex/latex/childdoc|.
\end{itemize}

%%%%%%%%%%%%%%%%%%%%%%%%%%%%%%%%%%%%%%%%%%%%%%%%%%%%%%%%%%%%%%%%%%%%%%%%%%%%%%%%
\subsection{Related CTAN Packages}

There are several other packages which offer a similar functionality:
%
\begin{itemize}
\item
The packages
\href{http://ctan.org/pkg/docmute}{\textsf{docmute}},
\href{http://ctan.org/pkg/includex}{\textsf{includex}} and
\href{http://ctan.org/pkg/standalone}{\textsf{standalone}}
provide commands to include only the document body of
a child file thus allowing both files to be compiled individually.
\item
The packages \href{http://ctan.org/pkg/subdocs}{\textsf{subdocs}}
and \href{http://ctan.org/pkg/subfiles}{\textsf{subfiles}}
provide structures in which the main and child documents can be
encapsulated and allowing them to be compiled individually.
The inclusion mechanism is different from the conventional |\include|.
\item
The package \href{http://ctan.org/pkg/combine}{\textsf{combine}}
is an elaborate solution to combine several documents into one.
\end{itemize}
%
See also the CTAN topic \href{http://ctan.org/topic/subdocs}{\textsf{subdocs}}
for further related packages.
The present package differs from the above solutions in that
a document structure constructed with the conventional |\include| mechanism
just needs two extra commands at the top of every file
such that all constituent files can be compiled individually.

%%%%%%%%%%%%%%%%%%%%%%%%%%%%%%%%%%%%%%%%%%%%%%%%%%%%%%%%%%%%%%%%%%%%%%%%%%%%%%%%
%\subsection{Feature Suggestions}
%
%The following is a list of features which may be useful for future
%versions of this package:
%%
%\begin{itemize}
%\item
%\ldots
%\end{itemize}

%%%%%%%%%%%%%%%%%%%%%%%%%%%%%%%%%%%%%%%%%%%%%%%%%%%%%%%%%%%%%%%%%%%%%%%%%%%%%%%%
\subsection{Revision History}

%%%%%%%%%%%%%%%%%%%%%%%%%%%%%%%%%%%%%%%%
\paragraph{v2.0:} 2018/12/30

\begin{itemize}
\item
immediate forward processing
\item
added |\childdocby| mechanism
\item
manual restructured
\end{itemize}

%%%%%%%%%%%%%%%%%%%%%%%%%%%%%%%%%%%%%%%%
\paragraph{v1.6:} 2018/01/17

\begin{itemize}
\item
application for development of include files
\item
corrections to manual
\end{itemize}

%%%%%%%%%%%%%%%%%%%%%%%%%%%%%%%%%%%%%%%%
\paragraph{v1.5:} 2017/05/21

\begin{itemize}
\item
more complete structuring introduced
\item
|\childdocof| introduced
\item
|\childdoc| renamed to |\childdocmain|
\item
|\childredirect| renamed to |\childdocforward| and |\childdocforwardprefix|
and functionality expanded
\end{itemize}

%%%%%%%%%%%%%%%%%%%%%%%%%%%%%%%%%%%%%%%%
\paragraph{v1.0:} 2017/04/27

\begin{itemize}
\item
manual and install package
\item
first version published on CTAN
\end{itemize}

%%%%%%%%%%%%%%%%%%%%%%%%%%%%%%%%%%%%%%%%
\paragraph{v0.6:} 2017/04/26

\begin{itemize}
\item
redirection mechanism added
\end{itemize}

%%%%%%%%%%%%%%%%%%%%%%%%%%%%%%%%%%%%%%%%
\paragraph{v0.5:} 2017/04/26

\begin{itemize}
\item
functionality in definition file
\end{itemize}


%%%%%%%%%%%%%%%%%%%%%%%%%%%%%%%%%%%%%%%%%%%%%%%%%%%%%%%%%%%%%%%%%%%%%%%%%%%%%%%%
%%%%%%%%%%%%%%%%%%%%%%%%%%%%%%%%%%%%%%%%%%%%%%%%%%%%%%%%%%%%%%%%%%%%%%%%%%%%%%%%
%%%%%%%%%%%%%%%%%%%%%%%%%%%%%%%%%%%%%%%%%%%%%%%%%%%%%%%%%%%%%%%%%%%%%%%%%%%%%%%%
\appendix

\settowidth\MacroIndent{\rmfamily\scriptsize 000\ }

 \DocInput{childdoc.dtx}

\end{document}
%</driver>
% \fi
%
% %%%%%%%%%%%%%%%%%%%%%%%%%%%%%%%%%%%%%%%%%%%%%%%%%%%%%%%%%%%%%%%%%%%%%%%%%%%%%%
% %%%%%%%%%%%%%%%%%%%%%%%%%%%%%%%%%%%%%%%%%%%%%%%%%%%%%%%%%%%%%%%%%%%%%%%%%%%%%%
% \section{Sample}
%\iffalse
%<*samplemain>
%\fi
%
% The following presents a sample document
% with two chapters, two parts, a title page,
% a compile flag as well as three forwarding files to set the flag.
% It consists of eight |.tex| files:
% \begin{center}
% \begin{tabular}{ll}
% |cdocsamp.tex|&main file\\
% |cdocsch1.tex|&include file for chapter 1\\
% |cdocsch2.tex|&include file for chapter 2\\
% |cdocspt3.tex|&include file for part 3\\
% |cdocspt4.tex|&include file for part 4\\
% |cdocsdrf.tex|&forwarding file for main file in draft mode\\
% |cdocsfi1.tex|&forwarding file for final version of chapter 1\\
% |cdocsfi2.tex|&forwarding file for final version of chapter 2\\
% \end{tabular}
% \end{center}
% Each of the eight files can be compiled directly by the \LaTeX{} compiler.
%
% %%%%%%%%%%%%%%%%%%%%%%%%%%%%%%%%%%%%%%
% \paragraph{Main File.}
%
% The main file is called |cdocsamp.tex|.
%
% Load the \textsf{childdoc} definitions and
% declare the filename for the main document:
%    \begin{macrocode}
\input{childdoc.def}
\childdocmain{}
%    \end{macrocode}

% Optional override for |\version| flag:
%    \begin{macrocode}
%%\ifchilddoc\else\providecommand{\version}{draft}\fi
%    \end{macrocode}

% Define the default values for the |\version| flag
% (|final| for the main file and |draft| for childs):
%    \begin{macrocode}
\ifchilddoc
\providecommand{\version}{draft}
\else
\providecommand{\version}{final}
\fi
%    \end{macrocode}

% Load the standard document class:
%    \begin{macrocode}
\documentclass[12pt]{article}
%    \end{macrocode}

% Start the document body:
%    \begin{macrocode}
\begin{document}
%    \end{macrocode}

% Declare a title page.
% Print title, part of document being processed and version flag:
%    \begin{macrocode}
\addtocounter{page}{-1}
\begin{center}
{\LARGE\bfseries{}childdoc example\par}
\vspace{1cm}
\ifchilddoc
\ifchilddocmanual part\else chapter\fi:
`\childdocname' of `\childdocjob'\par
\else
main document: `\childdocjob'\par
\fi
version: \version\par
\end{center}
\newpage
%    \end{macrocode}

% Manually include selected file,
% otherwise process as usual:
%    \begin{macrocode}
\ifchilddocmanual
\section*{part `\childdocname'}
\input{\childdocname}
\else
%    \end{macrocode}

% Include the two chapters:
%    \begin{macrocode}
\include{cdocsch1}
\include{cdocsch2}
%    \end{macrocode}

% Include the two parts unless only chapters should be displayed:
%    \begin{macrocode}
\ifchilddoc\else
\section{part three}
\input{cdocspt3}
\section{part four}
\input{cdocspt4}
\fi
%    \end{macrocode}

% Process as usual until here:
%    \begin{macrocode}
\fi
%    \end{macrocode}

% End of document body:
%    \begin{macrocode}
\end{document}
%    \end{macrocode}
%\iffalse
%</samplemain>
%\fi
%
% %%%%%%%%%%%%%%%%%%%%%%%%%%%%%%%%%%%%%%
% \paragraph{Chapter Include Files.}
%
% The include files are called |cdocsch1.tex| and |cdocsch2.tex|.
%
%\iffalse
%<*samplechap1|samplechap2>
%\fi

% Optional override for |\version| flag:
%    \begin{macrocode}
%%\providecommand{\version}{final}
%    \end{macrocode}

% Include the main document:
%    \begin{macrocode}
\input{childdoc.def}
\childdocof{cdocsamp}
%    \end{macrocode}

%\iffalse
%</samplechap1|samplechap2>
%\fi
%
%\iffalse
%<*samplechap1>
%\fi
% Some text for chapter 1:
%    \begin{macrocode}
\section{one}
some text in chapter one
%    \end{macrocode}

%\iffalse
%</samplechap1>
%\fi
% Some text for chapter 2:
%\iffalse
%<*samplechap2>
%\fi
%    \begin{macrocode}
\section{two}
more text in chapter two
%    \end{macrocode}

%\iffalse
%</samplechap2>
%\fi
%
% %%%%%%%%%%%%%%%%%%%%%%%%%%%%%%%%%%%%%%
% \paragraph{Part Include Files.}
%
% The include files are called |cdocspt3.tex| and |cdocspt4.tex|.
%
%\iffalse
%<*samplepart3|samplepart4>
%\fi

% Optional override for |\version| flag:
%    \begin{macrocode}
%%\providecommand{\version}{final}
%    \end{macrocode}

% Include the main document:
%    \begin{macrocode}
\input{childdoc.def}
\childdocby{cdocsamp}
%    \end{macrocode}

%\iffalse
%</samplepart3|samplepart4>
%\fi
%
%\iffalse
%<*samplepart3>
%\fi
% Some text for part 3:
%    \begin{macrocode}
some text in part three
%    \end{macrocode}

%\iffalse
%</samplepart3>
%\fi
% Some text for part 4:
%\iffalse
%<*samplepart4>
%\fi
%    \begin{macrocode}
more text in part four
%    \end{macrocode}

%\iffalse
%</samplepart4>
%\fi
%
% %%%%%%%%%%%%%%%%%%%%%%%%%%%%%%%%%%%%%%
% \paragraph{Forwarding for a Complete Draft.}
%
% The following forwarding file |cdocsdrf.tex|
% compiles the main document in draft mode:
%\iffalse
%<*sampledraft>
%\fi
%    \begin{macrocode}
\def\version{draft}
\input{childdoc.def}
\childdocforward{cdocsamp}
%    \end{macrocode}

%\iffalse
%</sampledraft>
%\fi
%
% %%%%%%%%%%%%%%%%%%%%%%%%%%%%%%%%%%%%%%
% \paragraph{Forwarding for Final Version of the Chapters.}
%
% The following forwarding files |cdocsfn1.tex| and |cdocsfn2.tex|
% (with identical content)
% compile the final versions of the child documents
% |cdocsch1.tex| and |cdocsch2.tex|, respectively:
%\iffalse
%<*samplefinal>
%\fi
%    \begin{macrocode}
\def\version{final}
\input{childdoc.def}
\childdocforwardprefix[cdocsamp]{cdocsfn}{cdocsch}
%    \end{macrocode}

%\iffalse
%</samplefinal>
%\fi
%
% %%%%%%%%%%%%%%%%%%%%%%%%%%%%%%%%%%%%%%
% \paragraph{Command Line Processing.}
%
% The following three command lines generate the output files
% |cdocscld|, |cdocscl1| and |cdocscl2|
% which should be identical to
% |cdocsdrf|, |cdocsch1| and |cdocsfn2|, respectively:
% \begin{center}
% \begin{tabular}{l}
% |latex -jobname cdocscld \|\\
% |  "\def\version{draft}\input{childdoc.def}\childdocforward{cdocsamp}"|\\
% |latex -jobname cdocscl1 \|\\
% |  "\input{childdoc.def}\childdocforward[cdocsamp]{cdocsch1}"|\\
% |latex -jobname cdocscl2 \|\\
% |  "\def\version{final}\input{childdoc.def}\childdocforward{cdocsch2}"|
% \end{tabular}
% \end{center}
% Note that the trailing backslash on each first line
% merely continues the input to the second line
% (for convenient cut ant paste).
% Furthermore, the command |latex| can be replaced by any
% of its alternative versions such as |pdflatex|.
%
% %%%%%%%%%%%%%%%%%%%%%%%%%%%%%%%%%%%%%%%%%%%%%%%%%%%%%%%%%%%%%%%%%%%%%%%%%%%%%%
% %%%%%%%%%%%%%%%%%%%%%%%%%%%%%%%%%%%%%%%%%%%%%%%%%%%%%%%%%%%%%%%%%%%%%%%%%%%%%%
% \section{Implementation}
%\iffalse
%<*package>
%\fi
%
% This section describes the definitions file |childdoc.def|.

% The definitions cannot be loaded using |\usepackage| or |\RequirePackage|
% which has a mechanism to prevent loading a style file more than once.
% When loading the definitions by means of |\input|
% multiple instances have to be prevented manually:
%\iffalse
%This code needs to be before the `\ProvidesFile' directive
%which is defined at the beginning of this file.
%Therefore it is also placed there and commented out here.
%</package>
%<*discard>
%\fi
%    \begin{macrocode}
\ifdefined\childdocmain\endinput\fi
%    \end{macrocode}
%\iffalse
%</discard>
%<*package>
%\fi
%
% \macro{\ifchilddoc}
% \macro{\ifchilddocmanual}
% The conditional |\ifchilddoc| tells whether a
% child (true) or main (false) document is being compiled.
% The conditional |\ifchilddocmanual| tells whether
% the |\includeonly| mechanism is used (false) or
% the selection of child files must be performed manually (true).
% The definitions initialise to false:
%    \begin{macrocode}
\newif\ifchilddoc
\newif\ifchilddocmanual
%    \end{macrocode}

% \macro{\childdocname}
% \macro{\childdocjob}
% The macro |\childdocname| stores the name of the main document
% to be compiled. The macro |\childdocjob| stores the name of
% the document on which the \LaTeX{} compiler was originally invoked.
% The content of |\jobname| cannot be compared
% to filenames specified in the source due to different catcodes.
% The following code rescans |\jobname|, stores the result
% in |\childdocname| and saves a copy in |\childdocjob|:
%    \begin{macrocode}
\edef\childdocname{\scantokens\expandafter{\jobname\noexpand}}
\let\childdocjob\childdocname
%    \end{macrocode}

% \macro{\childdocdisable}
% The macro |\childdocdisable| prevents the main file
% from being processed more than once.
% At this stage, the main document command |\childdocmain|
% is assumed to be called once again where it should do nothing.
% Any subsequent call to it should prevent
% a secondary processing of the main document
% It overwrites the forwarding commands
% |\childdocof| and |\childdocforward|
% with empty macros to prevent further inclusions of the main document:
%    \begin{macrocode}
\newcommand{\childdocdisable}
{
  \renewcommand{\childdocmain}[1]{\renewcommand{\childdocmain}[1]{\endinput}}
  \renewcommand{\childdocof}[1]{}
  \renewcommand{\childdocby}[2][]{}
  \renewcommand{\childdocforward}[2][]{}
  \renewcommand{\childdocdisable}{}
}
%    \end{macrocode}

% \macro{\childdocmain}
% The macro |\childdocmain| is to be called at the top of the main file
% with nothing or the main filename (without extension) as argument.
% First, it breaks loops.
% If the argument is not empty and does not match |\childdocname|
% (which is set by the first inclusion of |childdoc.def|),
% |\ifchilddoc| is set to true, |\includeonly| is applied to the child file
% and |\jobname| is set to the main file
% (for proper handling of |.aux| files):
%    \begin{macrocode}
\newcommand{\childdocmain}[1]
{
  \childdocdisable\childdocmain{}
  \if?#1?\else
    \begingroup
      \def\childdoctmp{#1}
      \ifx\childdoctmp\childdocname
        \def\childdoctmp{}
      \else
        \def\childdoctmp
        {
          \childdoctrue
          \includeonly{\childdocname}
          \def\childdocjob{#1}
          \def\jobname{#1}
        }
      \fi
      \expandafter
    \endgroup
    \childdoctmp
  \fi
}
%    \end{macrocode}

% \macro{\childdocof}
% The command |\childdocof| redirects
% compilation to the main file |#1|.
%    \begin{macrocode}
\newcommand{\childdocof}[1]
{
  \childdocdisable
  \childdoctrue
  \includeonly{\childdocname}
  \def\jobname{#1}
  \def\childdocjob{#1}
  \input{#1}
}
%    \end{macrocode}

% \macro{\childdocby}
% The command |\childdocby| ....
%    \begin{macrocode}
\newcommand{\childdocby}[2][]
{
  \childdocdisable
  \childdoctrue
  \childdocmanualtrue
  \if?#1?\else
    \def\jobname{#2}
  \fi
  \def\childdocjob{#2}
  \input{#2}
  \endinput
}
%    \end{macrocode}

% \macro{\childdocforward}
% The command |\childdocforward| redirects
% compilation to the main file or
% (if the optional argument is given) a child file.
% Parameters are set as if the main file
% or a child file starting with |\childdocof| was compiled.
% Then compilation is handed over to the main file:
%    \begin{macrocode}
\newcommand{\childdocforward}[2][]
{
  \begingroup
    \if?#1?
      \def\childdoctmp
      {
        \def\childdocname{#2}
        \def\childdocjob{#2}
        \def\jobname{#2}
        \input{#2}
        \endinput
      }
    \else
      \def\childdoctmp
      {
        \childdocdisable
        \def\childdocname{#2}
        \childdoctrue
        \includeonly{#2}
        \def\childdocjob{#1}
        \def\jobname{#1}
        \input{#1}
        \endinput
      }
    \fi
    \expandafter
  \endgroup
  \childdoctmp
}
%    \end{macrocode}

% \macro{\childdocforwardprefix}
% The command |\childdocforwardprefix| redirects
% compilation to the main or a child file by means of a pattern.
% The prefix |#1| in the current filename is replaced by |#2|
% and the suffix of the current filename is kept
% (it is assumed that the filename does not contain the substring `|~~~|'
% which is used as a delimiter).
% Compilation is handed over to the new file by |\childdocforward|:
%    \begin{macrocode}
\newcommand{\childdocforwardprefix}[3][]
{
  \begingroup
    \def\childdocextract #2##1~~~{\def\childdoctmp{\childdocforward[#1]{#3##1}}}
    \expandafter\childdocextract\childdocname~~~
    \expandafter
  \endgroup
  \childdoctmp
}
%    \end{macrocode}

% \macro{\childdoc}
% The deprecated macro |\childdoc| is a legacy version of |\childdocmain|:
%    \begin{macrocode}
\newcommand{\childdoc}{\childdocmain}
%    \end{macrocode}

% \macro{\childdocredirect}
% The deprecated macro |\childdocredirect| is a legacy version
% of |\childdocforward| and |\childdocforwardprefix|:
%    \begin{macrocode}
\newcommand{\childdocredirect}[2][]
{
  \begingroup
    \if?#1?
      \def\childdoctmp{\childdocforward{#2}}
    \else
      \def\childdoctmp{\childdocforwardprefix{#1}{#2}}
    \fi
    \expandafter
  \endgroup
  \childdoctmp
}
%    \end{macrocode}

%\iffalse
%</package>
%\fi
%
\endinput
|\\
|\childdocforward{|\textit{main}|}|
\end{tabular}
\end{center}
%
Likewise, the following files |final|\textit{nn}|.tex|
compile the final version of the child document
|child|\textit{nn}|.tex|:
%
\begin{center}
\begin{tabular}{l}
|\def\version{final}|\\
|% \iffalse
%
% childdoc.dtx Copyright (C) 2017-2018 Niklas Beisert
%
% This work may be distributed and/or modified under the
% conditions of the LaTeX Project Public License, either version 1.3
% of this license or (at your option) any later version.
% The latest version of this license is in
%   http://www.latex-project.org/lppl.txt
% and version 1.3 or later is part of all distributions of LaTeX
% version 2005/12/01 or later.
%
% This work has the LPPL maintenance status `maintained'.
%
% The Current Maintainer of this work is Niklas Beisert.
%
% This work consists of the files childdoc.dtx and childdoc.ins
% and the derived files childdoc.def and cdocsamp.tex with
% cdocsch1.tex, cdocsch2.tex, cdocsdrf.tex, cdocsfn1.tex, cdocsfn2.tex.
%
%<package>\ifdefined\childdocmain\endinput\fi
%<package>\ProvidesFile{childdoc.def}[2018/12/30 v2.0 child document driver]
%<samplemain>\ProvidesFile{cdocsamp.tex}[2018/12/30 v2.0 sample for childdoc]
%<*driver>
%\ProvidesFile{childdoc.drv}[2018/12/30 v2.0 childdoc reference manual file]
\PassOptionsToClass{10pt,a4paper}{article}
\documentclass{ltxdoc}

\usepackage[margin=35mm]{geometry}
\usepackage{hyperref}
\usepackage{hyperxmp}
\usepackage[usenames]{color}

\hypersetup{colorlinks=true}
\hypersetup{pdfstartview=FitH}
\hypersetup{pdfpagemode=UseNone}
\hypersetup{pdfsource={}}
\hypersetup{pdflang={en-UK}}
\hypersetup{pdfcopyright={Copyright 2017-2018 Niklas Beisert.
  This work may be distributed and/or modified under the
  conditions of the LaTeX Project Public License, either version 1.3
  of this license or (at your option) any later version.}}
\hypersetup{pdflicenseurl={http://www.latex-project.org/lppl.txt}}
\hypersetup{pdfcontactaddress={ETH Zurich, ITP, HIT K,
  Wolfgang-Pauli-Strasse 27}}
\hypersetup{pdfcontactpostcode={8093}}
\hypersetup{pdfcontactcity={Zurich}}
\hypersetup{pdfcontactcountry={Switzerland}}
\hypersetup{pdfcontactemail={nbeisert@itp.phys.ethz.ch}}
\hypersetup{pdfcontacturl={http://people.phys.ethz.ch/\xmptilde nbeisert/}}

\newcommand{\secref}[1]{\hyperref[#1]{section \ref*{#1}}}

\parskip1ex
\parindent0pt
\let\olditemize\itemize
\def\itemize{\olditemize\parskip0pt}

\begin{document}

\title{The \textsf{childdoc} Package}
\hypersetup{pdftitle={The childdoc Package}}
\author{Niklas Beisert\\[2ex]
  Institut f\"ur Theoretische Physik\\
  Eidgen\"ossische Technische Hochschule Z\"urich\\
  Wolfgang-Pauli-Strasse 27, 8093 Z\"urich, Switzerland\\[1ex]
  \href{mailto:nbeisert@itp.phys.ethz.ch}
  {\texttt{nbeisert@itp.phys.ethz.ch}}}
\hypersetup{pdfauthor={Niklas Beisert}}
\hypersetup{pdfsubject={Manual for the LaTeX2e Package childdoc}}
\date{30 December 2018, \textsf{v2.0}}
\maketitle

\begin{abstract}\noindent
\textsf{childdoc} is a \LaTeXe{} package
that enables the direct compilation
of document sections included by |\include|
to individual files.
\end{abstract}

\begingroup
\parskip0ex
\tableofcontents
\endgroup

%%%%%%%%%%%%%%%%%%%%%%%%%%%%%%%%%%%%%%%%%%%%%%%%%%%%%%%%%%%%%%%%%%%%%%%%%%%%%%%%
%%%%%%%%%%%%%%%%%%%%%%%%%%%%%%%%%%%%%%%%%%%%%%%%%%%%%%%%%%%%%%%%%%%%%%%%%%%%%%%%
\section{Introduction}

\LaTeX{} provides a mechanism to structure a large document (such as a book)
into a main file and several child files (containing the chapters)
using the |\include| command.
This mechanism is beneficial for documents
which span hundreds of pages in order to
make the source file(s) more manageable.
Moreover, compilation can be restricted to
selected child files by means of the |\includeonly| command.
The latter feature can be used to reduce the compilation time while editing
(this was significantly more useful in the earlier days of \LaTeX{})
or to generate a smaller document which is easier to navigate.
Another application of |\includeonly| is to generate
documents consisting of selected parts of the complete document.

However, there are a few drawbacks of the plain |\include| mechanism:
\begin{itemize}
\item
The child files cannot be compiled on their own,
they can only be compiled via the main file.
A naive editing environment
(such as a text editor with an option
to have the current file processed by \LaTeX)
may require one to switch to the main file before compiling;
attempting to compile the child file produces errors.
\item
The main file must be modified (each time)
to adjust the |\includeonly| command
to the present needs. This easily leaves the main file in a messy state.
\item
The generated document will always carry the filename
of the main document. This is inconvenient if
several child files are to be compiled and
to be kept for distribution.
\end{itemize}

The present package provides a simple interface
to make child files individually compilable by \LaTeX{}.
Compiling a child file then has the same effect as compiling
the main file with an |\includeonly| command
to select the appropriate child.
Moreover the generated document will carry the name of the child
rather than the main file.
This resolves all three above issues.

This feature is meant to make the editing of books,
thesis documents and lecture notes somewhat more convenient.
However, the package can also be used efficiently for
composing a series of documents (such as exercise sheets)
which are typically distributed individually.
It then assists the author in generating the individual documents
(potentially in different versions)
as well as a document containing the collected series.
Another application is in developing style files
or other kinds of included material
where compilation of the style file could redirect
to a sample or test file.

%%%%%%%%%%%%%%%%%%%%%%%%%%%%%%%%%%%%%%%%%%%%%%%%%%%%%%%%%%%%%%%%%%%%%%%%%%%%%%%%
%%%%%%%%%%%%%%%%%%%%%%%%%%%%%%%%%%%%%%%%%%%%%%%%%%%%%%%%%%%%%%%%%%%%%%%%%%%%%%%%
\section{Usage}

First of all, the package \textsf{childdoc} is \emph{not} a standard
\LaTeXe{} |.sty| style file! Therefore it needs to be invoked in
a non-standard way.

%%%%%%%%%%%%%%%%%%%%%%%%%%%%%%%%%%%%%%%%%%%%%%%%%%%%%%%%%%%%%%%%%%%%%%%%%%%%%%%%
\subsection{Included Files}
\label{sec:include}

%%%%%%%%%%%%%%%%%%%%%%%%%%%%%%%%%%%%%%%%
\DescribeMacro{\childdocmain}
To use the package, add the commands
\begin{center}
\begin{tabular}{l}
|\input{childdoc.def}|\\
|\childdocmain{}|\\
\end{tabular}
\end{center}
at the very top of the main \LaTeX{} file,
in particular \emph{before} the |\documentclass| statement!
The argument of |\childdocmain| should be left empty
(but it must be present).

%%%%%%%%%%%%%%%%%%%%%%%%%%%%%%%%%%%%%%%%
\DescribeMacro{\childdocof}
Furthermore, add the commands
\begin{center}
\begin{tabular}{l}
|\input{childdoc.def}|\\
|\childdocof{|\textit{main}|}|\\
\end{tabular}
\end{center}
at the top of every child file \textit{child}
which is included by |\include{|\textit{child}|}|
from within the main file
(or at least for those files to be compiled individually).
The argument \textit{main} must be the filename of the main file.

There are a couple of
considerations in setting up the main and child documents:

%%%%%%%%%%%%%%%%%%%%%%%%%%%%%%%%%%%%%%%%
\paragraph{Restrictions.}

Please note the following restrictions:
\begin{itemize}
\item
|\childdocmain| must be called with one argument \textit{main}
to ensure compatibility with earlier version of the package.
It must either be empty (|\childdocmain{}|)
or precisely match the filename of the main file in which it is specified.
See \secref{sec:detection} for further information.
\item
The filename \textit{main} must be specified without the |.tex| extension.
\item
The filename \textit{main} is case sensitive
(even in case-insensitive file systems)
due to internal string comparison.
\item
The argument \textit{main} should be fully expanded, it cannot be a macro.
\item
Subdirectories and special characters should be avoided in filenames.
\item
The command |\childdocmain{|\textit{main}|}| must be followed by a whitespace.
It should not be followed immediately by another command
or by a comment mark `|%|'.
This is because the \TeX{} parser reads the token immediately following
the argument of |\childdocmain| and puts it
at the beginning of every child section;
however, a white\-space is ignored.
\end{itemize}

%%%%%%%%%%%%%%%%%%%%%%%%%%%%%%%%%%%%%%%%
\paragraph{Content of Main File.}

It is advisable to place all content in the child files included by |\include|.
Any output contained in the main file will appear in all child documents
unless suppressed manually;
it cannot be suppressed automatically by the |\includeonly| directive
and thus should normally be avoided.
A method to include some content in the main file
by means of conditional processing is described in \secref{sec:conditional}.

%%%%%%%%%%%%%%%%%%%%%%%%%%%%%%%%%%%%%%%%
\paragraph{Page Numbering.}

When only a part of the document is compiled,
the appropriate numbering of pages
(as well as other status parameters)
is determined from the |.aux| files.
The latter contain information from previous passes.
However this information needs to propagate through
all intermediate child documents.
Therefore the page numbering in child documents may well
be inconsistent until the complete document is compiled at least once.

A useful (if unconventional) way to always ensure a consistent
page numbering is to restart the numbering in each child document
and denote the pages by `\textit{child}|.|\textit{page}'
where \textit{child} represents the chapter/section number of the child file.
This can be achieved by the command
|\numberwithin{page}{|\textit{child}|}|
of the \textsf{amsmath} package
where \textit{child} can be |chapter| or |section|
depending on the chosen structuring.
Alternatively, one can modify the macro |\thepage| appropriately
and reset the counter |page| at the start of each child file.

%%%%%%%%%%%%%%%%%%%%%%%%%%%%%%%%%%%%%%%%%%%%%%%%%%%%%%%%%%%%%%%%%%%%%%%%%%%%%%%%
\subsection{Conditional Processing}
\label{sec:conditional}

The package provides a mechanism to compile different versions
of a document. To customise the versions further some conditional processing
can come in handy to distinguish which version is being compiled.
The package provides two macros to describe the compilation context:

%%%%%%%%%%%%%%%%%%%%%%%%%%%%%%%%%%%%%%%%
\DescribeMacro{\ifchilddoc}
The conditional |\ifchilddoc| distinguishes between the compilation of
child documents and the main document:
%
\begin{center}
|\ifchilddoc |\textit{child-code}| |[|\||else |\textit{main-code}]| \||fi|
\end{center}

%%%%%%%%%%%%%%%%%%%%%%%%%%%%%%%%%%%%%%%%
\DescribeMacro{\childdocname}
\DescribeMacro{\childdocjob}
The macro |\childdocname| contains the filename (without extension)
of the main or child file being processed.
Note that |\childdocjob| will always contain the name of the main file.

%%%%%%%%%%%%%%%%%%%%%%%%%%%%%%%%%%%%%%%%
\paragraph{Title Page.}

Conditional processing can be used to include a title or banner page
in the main document when proper precautions are taken.
Importantly, the code in the main file should ensure that the page counter
(as well as other status parameters which are stored in the |.aux| files)
takes the same value after the conditional processing.
Otherwise the page numbers may take divergent values
depending on which part is compiled.

For example, a title page could be declared by:
%
\begin{center}
\begin{tabular}{l}
|\ifchilddoc\||else|\\
|\addtocounter{page}{-1}|\\
\textit{code for title page}\\
|\newpage|\\
|\||fi|
\end{tabular}
\end{center}
%
A banner page for the child documents can be generated by:
%
\begin{center}
\begin{tabular}{l}
|\ifchilddoc|\\
|\addtocounter{page}{-1}|\\
\textit{code for banner page}\\
|\newpage|\\
|\||fi|
\end{tabular}
\end{center}
%
Here one could write a message such as:
\begin{center}
|This is the part \childdocname{} of \childdocjob{}.|
\end{center}

%%%%%%%%%%%%%%%%%%%%%%%%%%%%%%%%%%%%%%%%%%%%%%%%%%%%%%%%%%%%%%%%%%%%%%%%%%%%%%%%
\subsection{Flags}
\label{sec:flags}

The package makes it easy to generate different versions
of the main or child documents.
To this end compilation flags can be defined
and assigned different default values.
They will be particularly useful in conjunction
with the forwarding mechanism described in \secref{sec:forward}.

For example, it may be useful to have a flag |\version|
which can be set to |draft| or |final|.
The document source will contain some conditional code
depending on the value of |\version|.
Suppose further, the flag should default to |final| for the main file
and to |draft| for child files
which is a natural assignment for editing the document.
This is achieved by placing the following code
in the preamble of the main document
(below the |\childdocmain| directive):
%
\begin{center}
\begin{tabular}{l}
|\ifchilddoc|\\
|\providecommand{\version}{draft}|\\
|\||else|\\
|\providecommand{\version}{final}|\\
|\||fi|
\end{tabular}
\end{center}
%
The definition by |\providecommand| makes sure
that previous definitions are not overwritten.
Further statements |\providecommand{\version}{...}|
can thus be added before the above code to override it.

For the main file, one might add a line
(between |\childdocmain| and the above block)
%
\begin{center}
|%\ifchilddoc\||else\providecommand{\version}{draft}\||fi|
\end{center}
%
which can be uncommented to produce a draft version.
Likewise one can add a line to the very top of a child file
(above the |\childdocof{|\textit{main}|}| directive)
%
\begin{center}
|%\providecommand{\version}{final}|
\end{center}
%
which can be uncommented to produce the final version of this child document.

%%%%%%%%%%%%%%%%%%%%%%%%%%%%%%%%%%%%%%%%%%%%%%%%%%%%%%%%%%%%%%%%%%%%%%%%%%%%%%%%
\subsection{Forwarding}
\label{sec:forward}

Different versions of the main or child documents
using compilation flags as described in \secref{sec:flags}
can be (permanently) stored in different files
for convenient compilation, viewing and distribution.
To this end, the package defines a command
to pass on compilation to a different file:

%%%%%%%%%%%%%%%%%%%%%%%%%%%%%%%%%%%%%%%%
\DescribeMacro{\childdocforward}
The command |\childdocforward| redirects processing to
another source file:
%
\begin{center}
\begin{tabular}{l}
|\input{childdoc.def}|\\
|\childdocforward[|\textit{main}|]{|\textit{dest}|}|\\
\end{tabular}
\end{center}
%
The argument \textit{dest} is the destination file
(without extension).
It should be the main file or one of the child files.
Note that further \textsf{childdoc} directives
such as |\childdocof| and |\childdocforward|
in the indicated file will be processed in this form.
The optional argument \textit{main}
passes on directly to the main file \textit{main}
while pretending to compile the child \textit{dest}.
This form behaves as if \textit{dest}
issues |\childdocof{|\textit{main}|}| right away,
and no further \textsf{childdoc} directives will be processed.

%%%%%%%%%%%%%%%%%%%%%%%%%%%%%%%%%%%%%%%%
\DescribeMacro{\...prefix}
In the alternative form |\childdocforwardprefix|,
%
\begin{center}
\begin{tabular}{l}
|\input{childdoc.def}|\\
|\childdocforwardprefix[|\textit{main}|]{|\textit{prefix}|}{|\textit{dest}|}|
\end{tabular}
\end{center}
%
the destination file is determined by a pattern
depending on the current file:
To make this work, the current file must be called
`{\textit{prefix}\hspace{0.2em}\textit{suffix}}'
with \textit{prefix} matching precisely the argument.
Processing is then passed on to the file
`{\textit{dest}\hspace{0.2em}\textit{suffix}}'.
Surely, the same effect is achieved by
directly specifying the
argument `{\textit{dest}\hspace{0.2em}\textit{suffix}}'
in the first form.
However, that requires to set up a different file
for each child. With the alternative form of the command
all these files can have exactly the same content
which simplifies setting them up and maintaining them.

For example, the following file |draft.tex|
with a compilation flag |\version| as described in \secref{sec:flags}
compiles the main document as a draft:
%
\begin{center}
\begin{tabular}{l}
|\def\version{draft}|\\
|\input{childdoc.def}|\\
|\childdocforward{|\textit{main}|}|
\end{tabular}
\end{center}
%
Likewise, the following files |final|\textit{nn}|.tex|
compile the final version of the child document
|child|\textit{nn}|.tex|:
%
\begin{center}
\begin{tabular}{l}
|\def\version{final}|\\
|\input{childdoc.def}|\\
|\childdocforwardprefix{final}{child}|
\end{tabular}
\end{center}
%

Note that when several versions of a main file and/or of each child file
are to be generated, it may be convenient to set up a |Makefile| or
shell script to automatise the process.

%%%%%%%%%%%%%%%%%%%%%%%%%%%%%%%%%%%%%%%%%%%%%%%%%%%%%%%%%%%%%%%%%%%%%%%%%%%%%%%%
\subsection{Command Line Processing}
\label{sec:commandline}

The effect of redirection files can also be achieved by invoking
the \LaTeX{} compiler with a more elaborate command line.
Most conveniently this should be done as part
of a shell script or a |Makefile|.

When using \textsf{childdoc} in the main file, the following
command lines effectively perform a redirection
(note that depending on the shell being used,
backslashes may have to be doubled: `|\|' $\to$ `|\\|'):
%
\begin{center}
|... -jobname "|\textit{target}|" |\\|"|[\textit{flags}]%
|\input{childdoc.def}\childdocforward[|\textit{main}|]{|\textit{dest}|}"|
\end{center}
%
Here \textit{target} is the name of the output file,
\textit{main} is the name of the main file
and \textit{dest} is the name of the main or child file to be processed
(all filenames without extensions).
The optional argument \textit{main} can be omitted
if \textit{main} matches \textit{dest}.
Optionally, compilation \textit{flags} can be defined via |\def| commands.
This command line makes the \TeX{} engine believe
it is compiling the file \textit{target}
whose content is specified as the latter parameter.
The provided code then forwards the processing to
\textit{main} or \textit{dest} as described in \secref{sec:forward}.

%%%%%%%%%%%%%%%%%%%%%%%%%%%%%%%%%%%%%%%%%%%%%%%%%%%%%%%%%%%%%%%%%%%%%%%%%%%%%%%%
\subsection{Include by Input}
\label{sec:input}

Including child documents by |\include| has some restrictions by design.
Most notably, the content of a child document always occupies
its own set of pages; pages cannot be shared between child documents.
Usually, this behaviour makes perfect sense
because each child document contain an essential part of the document.
However, in some situations it may be desirable to compose
a document from a collection of parts
without having mandatory page breaks between then.
For this case, the package
provides a mechanism to include parts
by |\input| which can also be processed individually.
However, by construction this mechanism
requires manual handling of the content to be output.

%%%%%%%%%%%%%%%%%%%%%%%%%%%%%%%%%%%%%%%%
\DescribeMacro{\ifchilddocmanual}
The main file should be prepared as usual, see \secref{sec:include}.
However, the document body must make a distinction
between processing of an individual part and of the main document, e.g.:
%
\begin{center}
\begin{tabular}{l}
|\ifchilddocmanual|\\
|\input{\childdocname}|\\
|\||else|\\
\textit{document body with }|\input{|\textit{part}|}|\\
|\||fi|
\end{tabular}
\end{center}
%
The conditional |\ifchilddocmanual| is true whenever
a part to be included by |\input| is being compiled,
and the name of the part is stored in |\childdocname|.

%%%%%%%%%%%%%%%%%%%%%%%%%%%%%%%%%%%%%%%%
\DescribeMacro{\childdocby}
Each part to be included by |\input| should start with:
%
\begin{center}
\begin{tabular}{l}
|\input{childdoc.def}|\\
|\childdocby{|\textit{main}|}|\\
\end{tabular}
\end{center}
%
The directive |\childdocby| is similar to |\childdocof|
described in \secref{sec:include},
but the subsequent selection of content must be done manually.
To that end, both |\ifchilddoc| and |\ifchilddocmanual|
will be true upon processing of a part,
and the name of the part is stored in |\childdocname|.
Note that |\jobname| will be set to the filename of the current part
so that each part receives an individual |.aux| file
that does not interfere with the |.aux| file(s) of the main document.
This behaviour can be altered by the alternative form
|\childdocby[*]{|\textit{main}|}| (with a non-empty optional argument)
which uses the |.aux| file of the main document
by setting |\jobname| to \textit{main}.

%%%%%%%%%%%%%%%%%%%%%%%%%%%%%%%%%%%%%%%%%%%%%%%%%%%%%%%%%%%%%%%%%%%%%%%%%%%%%%%%
\subsection{Driver Development}
\label{sec:driver}

The \textsf{childdoc} mechanism can also be use for the development
of definition files such as \LaTeX{} styles or classes.
This case differs from the above setup with multiple parts
included by |\include| in that no |\includeonly| should be invoked.
This can be achieved by starting the include file
(before |\ProvidesPackage|) with:
%
\begin{center}
\begin{tabular}{l}
|\input{childdoc.def}|\\
|\childdocforward{|\textit{main}|}|\\
\end{tabular}
\end{center}
%
or alternatively with:
%
\begin{center}
\begin{tabular}{l}
|\input{childdoc.def}|\\
|\childdocby{|\textit{main}|}|\\
\end{tabular}
\end{center}
%
Both forms have slightly different effects as described above.
The main file is prepared as usual, see \secref{sec:include}.

%%%%%%%%%%%%%%%%%%%%%%%%%%%%%%%%%%%%%%%%%%%%%%%%%%%%%%%%%%%%%%%%%%%%%%%%%%%%%%%%
\subsection{Legacy Detection}
\label{sec:detection}

The directive |\childdocmain| in the main file can detect
whether the complete document or merely a child is to be compiled
even without using the directive |\childdocof|.
This method is deprecated because it is less robust
and there is no compelling reason to use it;
it is merely provided for backward compatibility
and it may be removed in future versions.

If the detection mechanism is to be used,
it is mandatory to correctly specify
the filename of the main file as the argument of |\childdocmain|:
%
\begin{center}
\begin{tabular}{l}
|\input{childdoc.def}|\\
|\childdocmain{|\textit{main}|}|\\
\end{tabular}
\end{center}
%
If |\jobname| does not match the argument \textit{main} of |\childdocmain|,
it is assumed that |\jobname| points to the child file to be compiled.
When using |\childdocmain| with the main file specified as argument,
it suffices to start a child file
with just |\input{|\textit{main}|}|
without loading of the package and using |\childdocof|.
If instead all processing is done
with the appropriate \textsf{childdoc} directives,
the argument of \textit{main} of |\childdocmain| can be empty.

An alternative version of the command line processing described
in \secref{sec:commandline} using the detection mechanism reads:
%
\begin{center}
|... -jobname "|\textit{target}|" "|[\textit{flags}]%
[|\def\jobname{|\textit{dest}|}|]|\input{|\textit{main}|}"|
\end{center}

%%%%%%%%%%%%%%%%%%%%%%%%%%%%%%%%%%%%%%%%%%%%%%%%%%%%%%%%%%%%%%%%%%%%%%%%%%%%%%%%
\subsection{Manual Code}
\label{sec:manual}

In case one cannot be certain whether the definitions file |childdoc.def|
is installed on the target \TeX{} distribution
and one prefers not to ship it,
it is conceivable to paste a few relevant commands into the sources.

To that end, drop all statements |\input{childdoc.def}|
and perform the replacements as outlined below.
Instead of |\childdocmain{|\textit{main}|}| add the following code
to the top of the main file:
%
\begin{center}
\begin{tabular}{l}
|\||ifdefined\childdocname\endinput\||fi\newif\ifchilddoc|\\
|\edef\childdocname{\scantokens\expandafter{\jobname\noexpand}}|\\
|\def\childdocmain{|\textit{main}|}\||ifx\childdocmain\childdocname\||else|\\
|\childdoctrue\includeonly{\childdocname}\let\jobname\childdocmain\||fi|\\
\end{tabular}
\end{center}
%
Instead of |\childdocof{|\textit{main}|}| just include the main file
at the top of each child file:
%
\begin{center}
|\input{|\textit{main}|}|
\end{center}
%
A simple redirection |\childdocforward{|\textit{dest}|}| is achieved by:
%
\begin{center}
|\def\jobname{|\textit{dest}|}\input{\jobname}|
\end{center}
%
The redirection with prefix
|\childdocforwardprefix[|\textit{prefix}|]{|\textit{dest}|}|
is accomplished by:
%
\begin{center}
\begin{tabular}{l}
|{\edef\jobname{\scantokens\expandafter{\jobname\noexpand}}|\\
|\def\redirectjob |\textit{prefix}|#1~~~{\gdef\jobname{|\textit{dest}|#1}}|\\
|\expandafter\redirectjob\jobname~~~}\input{\jobname}|
\end{tabular}
\end{center}

In an alternative approach,
child documents can be compiled by a specific command line
without additional code or specific definitions:
%
\begin{center}
|... -jobname "|\textit{target}|" "|[\textit{flags}]%
|\includeonly{|\textit{dest}|}\input{|\textit{main}|}"|
\end{center}
%

%%%%%%%%%%%%%%%%%%%%%%%%%%%%%%%%%%%%%%%%%%%%%%%%%%%%%%%%%%%%%%%%%%%%%%%%%%%%%%%%
%%%%%%%%%%%%%%%%%%%%%%%%%%%%%%%%%%%%%%%%%%%%%%%%%%%%%%%%%%%%%%%%%%%%%%%%%%%%%%%%
\section{Information}

%%%%%%%%%%%%%%%%%%%%%%%%%%%%%%%%%%%%%%%%%%%%%%%%%%%%%%%%%%%%%%%%%%%%%%%%%%%%%%%%
\subsection{Copyright}

Copyright \copyright{} 2017--2018 Niklas Beisert

This work may be distributed and/or modified under the
conditions of the \LaTeX{} Project Public License, either version 1.3
of this license or (at your option) any later version.
The latest version of this license is in
  \url{http://www.latex-project.org/lppl.txt}
and version 1.3 or later is part of all distributions of \LaTeX{}
version 2005/12/01 or later.

This work has the LPPL maintenance status `maintained'.

The Current Maintainer of this work is Niklas Beisert.

This work consists of the files |README.txt|, |childdoc.ins| and |childdoc.dtx|
as well as the derived files |childdoc.def|, |cdocsamp.tex|
with |cdocsch1.tex|, |cdocsch2.tex|, |cdocspt3.tex|, |cdocspt4.tex|,
|cdocsdrf.tex|, |cdocsfn1.tex|, |cdocsfn2.tex|
as well as |childdoc.pdf|.

%%%%%%%%%%%%%%%%%%%%%%%%%%%%%%%%%%%%%%%%%%%%%%%%%%%%%%%%%%%%%%%%%%%%%%%%%%%%%%%%
\subsection{Files and Installation}

The package consists of the files:
%
\begin{center}
\begin{tabular}{ll}
    |README.txt|   & readme file \\
    |childdoc.ins| & installation file \\
    |childdoc.dtx| & source file \\
    |childdoc.def| & definition file \\
    |cdocsamp.tex| & sample main file \\
    |cdocsch1.tex| & sample include file \\
    |cdocsch2.tex| & sample include file \\
    |cdocspt3.tex| & sample part file \\
    |cdocspt4.tex| & sample part file \\
    |cdocsdrf.tex| & sample redirection file \\
    |cdocsfn1.tex| & sample redirection file \\
    |cdocsfn2.tex| & sample redirection file \\
    |childdoc.pdf| & manual
\end{tabular}
\end{center}
%
The distribution consists of the files
|README.txt|, |childdoc.ins| and |childdoc.dtx|.
%
\begin{itemize}
\item
Run (pdf)\LaTeX{} on |childdoc.dtx|
to compile the manual |childdoc.pdf| (this file).
\item
Run \LaTeX{} on |childdoc.ins| to create the definitions file |childdoc.def|
and the sample |cdocsamp.tex| with include files
|cdocsch1.tex|, |cdocsch2.tex|, |cdocspt3.tex|, |cdocspt4.tex|,
|cdocsdrf.tex|, |cdocsfn1.tex|, |cdocsfn2.tex|.
Then copy the file |childdoc.def| to an appropriate directory of your \LaTeX{}
distribution, e.g.\ \textit{texmf-root}|/tex/latex/childdoc|.
\end{itemize}

%%%%%%%%%%%%%%%%%%%%%%%%%%%%%%%%%%%%%%%%%%%%%%%%%%%%%%%%%%%%%%%%%%%%%%%%%%%%%%%%
\subsection{Related CTAN Packages}

There are several other packages which offer a similar functionality:
%
\begin{itemize}
\item
The packages
\href{http://ctan.org/pkg/docmute}{\textsf{docmute}},
\href{http://ctan.org/pkg/includex}{\textsf{includex}} and
\href{http://ctan.org/pkg/standalone}{\textsf{standalone}}
provide commands to include only the document body of
a child file thus allowing both files to be compiled individually.
\item
The packages \href{http://ctan.org/pkg/subdocs}{\textsf{subdocs}}
and \href{http://ctan.org/pkg/subfiles}{\textsf{subfiles}}
provide structures in which the main and child documents can be
encapsulated and allowing them to be compiled individually.
The inclusion mechanism is different from the conventional |\include|.
\item
The package \href{http://ctan.org/pkg/combine}{\textsf{combine}}
is an elaborate solution to combine several documents into one.
\end{itemize}
%
See also the CTAN topic \href{http://ctan.org/topic/subdocs}{\textsf{subdocs}}
for further related packages.
The present package differs from the above solutions in that
a document structure constructed with the conventional |\include| mechanism
just needs two extra commands at the top of every file
such that all constituent files can be compiled individually.

%%%%%%%%%%%%%%%%%%%%%%%%%%%%%%%%%%%%%%%%%%%%%%%%%%%%%%%%%%%%%%%%%%%%%%%%%%%%%%%%
%\subsection{Feature Suggestions}
%
%The following is a list of features which may be useful for future
%versions of this package:
%%
%\begin{itemize}
%\item
%\ldots
%\end{itemize}

%%%%%%%%%%%%%%%%%%%%%%%%%%%%%%%%%%%%%%%%%%%%%%%%%%%%%%%%%%%%%%%%%%%%%%%%%%%%%%%%
\subsection{Revision History}

%%%%%%%%%%%%%%%%%%%%%%%%%%%%%%%%%%%%%%%%
\paragraph{v2.0:} 2018/12/30

\begin{itemize}
\item
immediate forward processing
\item
added |\childdocby| mechanism
\item
manual restructured
\end{itemize}

%%%%%%%%%%%%%%%%%%%%%%%%%%%%%%%%%%%%%%%%
\paragraph{v1.6:} 2018/01/17

\begin{itemize}
\item
application for development of include files
\item
corrections to manual
\end{itemize}

%%%%%%%%%%%%%%%%%%%%%%%%%%%%%%%%%%%%%%%%
\paragraph{v1.5:} 2017/05/21

\begin{itemize}
\item
more complete structuring introduced
\item
|\childdocof| introduced
\item
|\childdoc| renamed to |\childdocmain|
\item
|\childredirect| renamed to |\childdocforward| and |\childdocforwardprefix|
and functionality expanded
\end{itemize}

%%%%%%%%%%%%%%%%%%%%%%%%%%%%%%%%%%%%%%%%
\paragraph{v1.0:} 2017/04/27

\begin{itemize}
\item
manual and install package
\item
first version published on CTAN
\end{itemize}

%%%%%%%%%%%%%%%%%%%%%%%%%%%%%%%%%%%%%%%%
\paragraph{v0.6:} 2017/04/26

\begin{itemize}
\item
redirection mechanism added
\end{itemize}

%%%%%%%%%%%%%%%%%%%%%%%%%%%%%%%%%%%%%%%%
\paragraph{v0.5:} 2017/04/26

\begin{itemize}
\item
functionality in definition file
\end{itemize}


%%%%%%%%%%%%%%%%%%%%%%%%%%%%%%%%%%%%%%%%%%%%%%%%%%%%%%%%%%%%%%%%%%%%%%%%%%%%%%%%
%%%%%%%%%%%%%%%%%%%%%%%%%%%%%%%%%%%%%%%%%%%%%%%%%%%%%%%%%%%%%%%%%%%%%%%%%%%%%%%%
%%%%%%%%%%%%%%%%%%%%%%%%%%%%%%%%%%%%%%%%%%%%%%%%%%%%%%%%%%%%%%%%%%%%%%%%%%%%%%%%
\appendix

\settowidth\MacroIndent{\rmfamily\scriptsize 000\ }

 \DocInput{childdoc.dtx}

\end{document}
%</driver>
% \fi
%
% %%%%%%%%%%%%%%%%%%%%%%%%%%%%%%%%%%%%%%%%%%%%%%%%%%%%%%%%%%%%%%%%%%%%%%%%%%%%%%
% %%%%%%%%%%%%%%%%%%%%%%%%%%%%%%%%%%%%%%%%%%%%%%%%%%%%%%%%%%%%%%%%%%%%%%%%%%%%%%
% \section{Sample}
%\iffalse
%<*samplemain>
%\fi
%
% The following presents a sample document
% with two chapters, two parts, a title page,
% a compile flag as well as three forwarding files to set the flag.
% It consists of eight |.tex| files:
% \begin{center}
% \begin{tabular}{ll}
% |cdocsamp.tex|&main file\\
% |cdocsch1.tex|&include file for chapter 1\\
% |cdocsch2.tex|&include file for chapter 2\\
% |cdocspt3.tex|&include file for part 3\\
% |cdocspt4.tex|&include file for part 4\\
% |cdocsdrf.tex|&forwarding file for main file in draft mode\\
% |cdocsfi1.tex|&forwarding file for final version of chapter 1\\
% |cdocsfi2.tex|&forwarding file for final version of chapter 2\\
% \end{tabular}
% \end{center}
% Each of the eight files can be compiled directly by the \LaTeX{} compiler.
%
% %%%%%%%%%%%%%%%%%%%%%%%%%%%%%%%%%%%%%%
% \paragraph{Main File.}
%
% The main file is called |cdocsamp.tex|.
%
% Load the \textsf{childdoc} definitions and
% declare the filename for the main document:
%    \begin{macrocode}
\input{childdoc.def}
\childdocmain{}
%    \end{macrocode}

% Optional override for |\version| flag:
%    \begin{macrocode}
%%\ifchilddoc\else\providecommand{\version}{draft}\fi
%    \end{macrocode}

% Define the default values for the |\version| flag
% (|final| for the main file and |draft| for childs):
%    \begin{macrocode}
\ifchilddoc
\providecommand{\version}{draft}
\else
\providecommand{\version}{final}
\fi
%    \end{macrocode}

% Load the standard document class:
%    \begin{macrocode}
\documentclass[12pt]{article}
%    \end{macrocode}

% Start the document body:
%    \begin{macrocode}
\begin{document}
%    \end{macrocode}

% Declare a title page.
% Print title, part of document being processed and version flag:
%    \begin{macrocode}
\addtocounter{page}{-1}
\begin{center}
{\LARGE\bfseries{}childdoc example\par}
\vspace{1cm}
\ifchilddoc
\ifchilddocmanual part\else chapter\fi:
`\childdocname' of `\childdocjob'\par
\else
main document: `\childdocjob'\par
\fi
version: \version\par
\end{center}
\newpage
%    \end{macrocode}

% Manually include selected file,
% otherwise process as usual:
%    \begin{macrocode}
\ifchilddocmanual
\section*{part `\childdocname'}
\input{\childdocname}
\else
%    \end{macrocode}

% Include the two chapters:
%    \begin{macrocode}
\include{cdocsch1}
\include{cdocsch2}
%    \end{macrocode}

% Include the two parts unless only chapters should be displayed:
%    \begin{macrocode}
\ifchilddoc\else
\section{part three}
\input{cdocspt3}
\section{part four}
\input{cdocspt4}
\fi
%    \end{macrocode}

% Process as usual until here:
%    \begin{macrocode}
\fi
%    \end{macrocode}

% End of document body:
%    \begin{macrocode}
\end{document}
%    \end{macrocode}
%\iffalse
%</samplemain>
%\fi
%
% %%%%%%%%%%%%%%%%%%%%%%%%%%%%%%%%%%%%%%
% \paragraph{Chapter Include Files.}
%
% The include files are called |cdocsch1.tex| and |cdocsch2.tex|.
%
%\iffalse
%<*samplechap1|samplechap2>
%\fi

% Optional override for |\version| flag:
%    \begin{macrocode}
%%\providecommand{\version}{final}
%    \end{macrocode}

% Include the main document:
%    \begin{macrocode}
\input{childdoc.def}
\childdocof{cdocsamp}
%    \end{macrocode}

%\iffalse
%</samplechap1|samplechap2>
%\fi
%
%\iffalse
%<*samplechap1>
%\fi
% Some text for chapter 1:
%    \begin{macrocode}
\section{one}
some text in chapter one
%    \end{macrocode}

%\iffalse
%</samplechap1>
%\fi
% Some text for chapter 2:
%\iffalse
%<*samplechap2>
%\fi
%    \begin{macrocode}
\section{two}
more text in chapter two
%    \end{macrocode}

%\iffalse
%</samplechap2>
%\fi
%
% %%%%%%%%%%%%%%%%%%%%%%%%%%%%%%%%%%%%%%
% \paragraph{Part Include Files.}
%
% The include files are called |cdocspt3.tex| and |cdocspt4.tex|.
%
%\iffalse
%<*samplepart3|samplepart4>
%\fi

% Optional override for |\version| flag:
%    \begin{macrocode}
%%\providecommand{\version}{final}
%    \end{macrocode}

% Include the main document:
%    \begin{macrocode}
\input{childdoc.def}
\childdocby{cdocsamp}
%    \end{macrocode}

%\iffalse
%</samplepart3|samplepart4>
%\fi
%
%\iffalse
%<*samplepart3>
%\fi
% Some text for part 3:
%    \begin{macrocode}
some text in part three
%    \end{macrocode}

%\iffalse
%</samplepart3>
%\fi
% Some text for part 4:
%\iffalse
%<*samplepart4>
%\fi
%    \begin{macrocode}
more text in part four
%    \end{macrocode}

%\iffalse
%</samplepart4>
%\fi
%
% %%%%%%%%%%%%%%%%%%%%%%%%%%%%%%%%%%%%%%
% \paragraph{Forwarding for a Complete Draft.}
%
% The following forwarding file |cdocsdrf.tex|
% compiles the main document in draft mode:
%\iffalse
%<*sampledraft>
%\fi
%    \begin{macrocode}
\def\version{draft}
\input{childdoc.def}
\childdocforward{cdocsamp}
%    \end{macrocode}

%\iffalse
%</sampledraft>
%\fi
%
% %%%%%%%%%%%%%%%%%%%%%%%%%%%%%%%%%%%%%%
% \paragraph{Forwarding for Final Version of the Chapters.}
%
% The following forwarding files |cdocsfn1.tex| and |cdocsfn2.tex|
% (with identical content)
% compile the final versions of the child documents
% |cdocsch1.tex| and |cdocsch2.tex|, respectively:
%\iffalse
%<*samplefinal>
%\fi
%    \begin{macrocode}
\def\version{final}
\input{childdoc.def}
\childdocforwardprefix[cdocsamp]{cdocsfn}{cdocsch}
%    \end{macrocode}

%\iffalse
%</samplefinal>
%\fi
%
% %%%%%%%%%%%%%%%%%%%%%%%%%%%%%%%%%%%%%%
% \paragraph{Command Line Processing.}
%
% The following three command lines generate the output files
% |cdocscld|, |cdocscl1| and |cdocscl2|
% which should be identical to
% |cdocsdrf|, |cdocsch1| and |cdocsfn2|, respectively:
% \begin{center}
% \begin{tabular}{l}
% |latex -jobname cdocscld \|\\
% |  "\def\version{draft}\input{childdoc.def}\childdocforward{cdocsamp}"|\\
% |latex -jobname cdocscl1 \|\\
% |  "\input{childdoc.def}\childdocforward[cdocsamp]{cdocsch1}"|\\
% |latex -jobname cdocscl2 \|\\
% |  "\def\version{final}\input{childdoc.def}\childdocforward{cdocsch2}"|
% \end{tabular}
% \end{center}
% Note that the trailing backslash on each first line
% merely continues the input to the second line
% (for convenient cut ant paste).
% Furthermore, the command |latex| can be replaced by any
% of its alternative versions such as |pdflatex|.
%
% %%%%%%%%%%%%%%%%%%%%%%%%%%%%%%%%%%%%%%%%%%%%%%%%%%%%%%%%%%%%%%%%%%%%%%%%%%%%%%
% %%%%%%%%%%%%%%%%%%%%%%%%%%%%%%%%%%%%%%%%%%%%%%%%%%%%%%%%%%%%%%%%%%%%%%%%%%%%%%
% \section{Implementation}
%\iffalse
%<*package>
%\fi
%
% This section describes the definitions file |childdoc.def|.

% The definitions cannot be loaded using |\usepackage| or |\RequirePackage|
% which has a mechanism to prevent loading a style file more than once.
% When loading the definitions by means of |\input|
% multiple instances have to be prevented manually:
%\iffalse
%This code needs to be before the `\ProvidesFile' directive
%which is defined at the beginning of this file.
%Therefore it is also placed there and commented out here.
%</package>
%<*discard>
%\fi
%    \begin{macrocode}
\ifdefined\childdocmain\endinput\fi
%    \end{macrocode}
%\iffalse
%</discard>
%<*package>
%\fi
%
% \macro{\ifchilddoc}
% \macro{\ifchilddocmanual}
% The conditional |\ifchilddoc| tells whether a
% child (true) or main (false) document is being compiled.
% The conditional |\ifchilddocmanual| tells whether
% the |\includeonly| mechanism is used (false) or
% the selection of child files must be performed manually (true).
% The definitions initialise to false:
%    \begin{macrocode}
\newif\ifchilddoc
\newif\ifchilddocmanual
%    \end{macrocode}

% \macro{\childdocname}
% \macro{\childdocjob}
% The macro |\childdocname| stores the name of the main document
% to be compiled. The macro |\childdocjob| stores the name of
% the document on which the \LaTeX{} compiler was originally invoked.
% The content of |\jobname| cannot be compared
% to filenames specified in the source due to different catcodes.
% The following code rescans |\jobname|, stores the result
% in |\childdocname| and saves a copy in |\childdocjob|:
%    \begin{macrocode}
\edef\childdocname{\scantokens\expandafter{\jobname\noexpand}}
\let\childdocjob\childdocname
%    \end{macrocode}

% \macro{\childdocdisable}
% The macro |\childdocdisable| prevents the main file
% from being processed more than once.
% At this stage, the main document command |\childdocmain|
% is assumed to be called once again where it should do nothing.
% Any subsequent call to it should prevent
% a secondary processing of the main document
% It overwrites the forwarding commands
% |\childdocof| and |\childdocforward|
% with empty macros to prevent further inclusions of the main document:
%    \begin{macrocode}
\newcommand{\childdocdisable}
{
  \renewcommand{\childdocmain}[1]{\renewcommand{\childdocmain}[1]{\endinput}}
  \renewcommand{\childdocof}[1]{}
  \renewcommand{\childdocby}[2][]{}
  \renewcommand{\childdocforward}[2][]{}
  \renewcommand{\childdocdisable}{}
}
%    \end{macrocode}

% \macro{\childdocmain}
% The macro |\childdocmain| is to be called at the top of the main file
% with nothing or the main filename (without extension) as argument.
% First, it breaks loops.
% If the argument is not empty and does not match |\childdocname|
% (which is set by the first inclusion of |childdoc.def|),
% |\ifchilddoc| is set to true, |\includeonly| is applied to the child file
% and |\jobname| is set to the main file
% (for proper handling of |.aux| files):
%    \begin{macrocode}
\newcommand{\childdocmain}[1]
{
  \childdocdisable\childdocmain{}
  \if?#1?\else
    \begingroup
      \def\childdoctmp{#1}
      \ifx\childdoctmp\childdocname
        \def\childdoctmp{}
      \else
        \def\childdoctmp
        {
          \childdoctrue
          \includeonly{\childdocname}
          \def\childdocjob{#1}
          \def\jobname{#1}
        }
      \fi
      \expandafter
    \endgroup
    \childdoctmp
  \fi
}
%    \end{macrocode}

% \macro{\childdocof}
% The command |\childdocof| redirects
% compilation to the main file |#1|.
%    \begin{macrocode}
\newcommand{\childdocof}[1]
{
  \childdocdisable
  \childdoctrue
  \includeonly{\childdocname}
  \def\jobname{#1}
  \def\childdocjob{#1}
  \input{#1}
}
%    \end{macrocode}

% \macro{\childdocby}
% The command |\childdocby| ....
%    \begin{macrocode}
\newcommand{\childdocby}[2][]
{
  \childdocdisable
  \childdoctrue
  \childdocmanualtrue
  \if?#1?\else
    \def\jobname{#2}
  \fi
  \def\childdocjob{#2}
  \input{#2}
  \endinput
}
%    \end{macrocode}

% \macro{\childdocforward}
% The command |\childdocforward| redirects
% compilation to the main file or
% (if the optional argument is given) a child file.
% Parameters are set as if the main file
% or a child file starting with |\childdocof| was compiled.
% Then compilation is handed over to the main file:
%    \begin{macrocode}
\newcommand{\childdocforward}[2][]
{
  \begingroup
    \if?#1?
      \def\childdoctmp
      {
        \def\childdocname{#2}
        \def\childdocjob{#2}
        \def\jobname{#2}
        \input{#2}
        \endinput
      }
    \else
      \def\childdoctmp
      {
        \childdocdisable
        \def\childdocname{#2}
        \childdoctrue
        \includeonly{#2}
        \def\childdocjob{#1}
        \def\jobname{#1}
        \input{#1}
        \endinput
      }
    \fi
    \expandafter
  \endgroup
  \childdoctmp
}
%    \end{macrocode}

% \macro{\childdocforwardprefix}
% The command |\childdocforwardprefix| redirects
% compilation to the main or a child file by means of a pattern.
% The prefix |#1| in the current filename is replaced by |#2|
% and the suffix of the current filename is kept
% (it is assumed that the filename does not contain the substring `|~~~|'
% which is used as a delimiter).
% Compilation is handed over to the new file by |\childdocforward|:
%    \begin{macrocode}
\newcommand{\childdocforwardprefix}[3][]
{
  \begingroup
    \def\childdocextract #2##1~~~{\def\childdoctmp{\childdocforward[#1]{#3##1}}}
    \expandafter\childdocextract\childdocname~~~
    \expandafter
  \endgroup
  \childdoctmp
}
%    \end{macrocode}

% \macro{\childdoc}
% The deprecated macro |\childdoc| is a legacy version of |\childdocmain|:
%    \begin{macrocode}
\newcommand{\childdoc}{\childdocmain}
%    \end{macrocode}

% \macro{\childdocredirect}
% The deprecated macro |\childdocredirect| is a legacy version
% of |\childdocforward| and |\childdocforwardprefix|:
%    \begin{macrocode}
\newcommand{\childdocredirect}[2][]
{
  \begingroup
    \if?#1?
      \def\childdoctmp{\childdocforward{#2}}
    \else
      \def\childdoctmp{\childdocforwardprefix{#1}{#2}}
    \fi
    \expandafter
  \endgroup
  \childdoctmp
}
%    \end{macrocode}

%\iffalse
%</package>
%\fi
%
\endinput
|\\
|\childdocforwardprefix{final}{child}|
\end{tabular}
\end{center}
%

Note that when several versions of a main file and/or of each child file
are to be generated, it may be convenient to set up a |Makefile| or
shell script to automatise the process.

%%%%%%%%%%%%%%%%%%%%%%%%%%%%%%%%%%%%%%%%%%%%%%%%%%%%%%%%%%%%%%%%%%%%%%%%%%%%%%%%
\subsection{Command Line Processing}
\label{sec:commandline}

The effect of redirection files can also be achieved by invoking
the \LaTeX{} compiler with a more elaborate command line.
Most conveniently this should be done as part
of a shell script or a |Makefile|.

When using \textsf{childdoc} in the main file, the following
command lines effectively perform a redirection
(note that depending on the shell being used,
backslashes may have to be doubled: `|\|' $\to$ `|\\|'):
%
\begin{center}
|... -jobname "|\textit{target}|" |\\|"|[\textit{flags}]%
|% \iffalse
%
% childdoc.dtx Copyright (C) 2017-2018 Niklas Beisert
%
% This work may be distributed and/or modified under the
% conditions of the LaTeX Project Public License, either version 1.3
% of this license or (at your option) any later version.
% The latest version of this license is in
%   http://www.latex-project.org/lppl.txt
% and version 1.3 or later is part of all distributions of LaTeX
% version 2005/12/01 or later.
%
% This work has the LPPL maintenance status `maintained'.
%
% The Current Maintainer of this work is Niklas Beisert.
%
% This work consists of the files childdoc.dtx and childdoc.ins
% and the derived files childdoc.def and cdocsamp.tex with
% cdocsch1.tex, cdocsch2.tex, cdocsdrf.tex, cdocsfn1.tex, cdocsfn2.tex.
%
%<package>\ifdefined\childdocmain\endinput\fi
%<package>\ProvidesFile{childdoc.def}[2018/12/30 v2.0 child document driver]
%<samplemain>\ProvidesFile{cdocsamp.tex}[2018/12/30 v2.0 sample for childdoc]
%<*driver>
%\ProvidesFile{childdoc.drv}[2018/12/30 v2.0 childdoc reference manual file]
\PassOptionsToClass{10pt,a4paper}{article}
\documentclass{ltxdoc}

\usepackage[margin=35mm]{geometry}
\usepackage{hyperref}
\usepackage{hyperxmp}
\usepackage[usenames]{color}

\hypersetup{colorlinks=true}
\hypersetup{pdfstartview=FitH}
\hypersetup{pdfpagemode=UseNone}
\hypersetup{pdfsource={}}
\hypersetup{pdflang={en-UK}}
\hypersetup{pdfcopyright={Copyright 2017-2018 Niklas Beisert.
  This work may be distributed and/or modified under the
  conditions of the LaTeX Project Public License, either version 1.3
  of this license or (at your option) any later version.}}
\hypersetup{pdflicenseurl={http://www.latex-project.org/lppl.txt}}
\hypersetup{pdfcontactaddress={ETH Zurich, ITP, HIT K,
  Wolfgang-Pauli-Strasse 27}}
\hypersetup{pdfcontactpostcode={8093}}
\hypersetup{pdfcontactcity={Zurich}}
\hypersetup{pdfcontactcountry={Switzerland}}
\hypersetup{pdfcontactemail={nbeisert@itp.phys.ethz.ch}}
\hypersetup{pdfcontacturl={http://people.phys.ethz.ch/\xmptilde nbeisert/}}

\newcommand{\secref}[1]{\hyperref[#1]{section \ref*{#1}}}

\parskip1ex
\parindent0pt
\let\olditemize\itemize
\def\itemize{\olditemize\parskip0pt}

\begin{document}

\title{The \textsf{childdoc} Package}
\hypersetup{pdftitle={The childdoc Package}}
\author{Niklas Beisert\\[2ex]
  Institut f\"ur Theoretische Physik\\
  Eidgen\"ossische Technische Hochschule Z\"urich\\
  Wolfgang-Pauli-Strasse 27, 8093 Z\"urich, Switzerland\\[1ex]
  \href{mailto:nbeisert@itp.phys.ethz.ch}
  {\texttt{nbeisert@itp.phys.ethz.ch}}}
\hypersetup{pdfauthor={Niklas Beisert}}
\hypersetup{pdfsubject={Manual for the LaTeX2e Package childdoc}}
\date{30 December 2018, \textsf{v2.0}}
\maketitle

\begin{abstract}\noindent
\textsf{childdoc} is a \LaTeXe{} package
that enables the direct compilation
of document sections included by |\include|
to individual files.
\end{abstract}

\begingroup
\parskip0ex
\tableofcontents
\endgroup

%%%%%%%%%%%%%%%%%%%%%%%%%%%%%%%%%%%%%%%%%%%%%%%%%%%%%%%%%%%%%%%%%%%%%%%%%%%%%%%%
%%%%%%%%%%%%%%%%%%%%%%%%%%%%%%%%%%%%%%%%%%%%%%%%%%%%%%%%%%%%%%%%%%%%%%%%%%%%%%%%
\section{Introduction}

\LaTeX{} provides a mechanism to structure a large document (such as a book)
into a main file and several child files (containing the chapters)
using the |\include| command.
This mechanism is beneficial for documents
which span hundreds of pages in order to
make the source file(s) more manageable.
Moreover, compilation can be restricted to
selected child files by means of the |\includeonly| command.
The latter feature can be used to reduce the compilation time while editing
(this was significantly more useful in the earlier days of \LaTeX{})
or to generate a smaller document which is easier to navigate.
Another application of |\includeonly| is to generate
documents consisting of selected parts of the complete document.

However, there are a few drawbacks of the plain |\include| mechanism:
\begin{itemize}
\item
The child files cannot be compiled on their own,
they can only be compiled via the main file.
A naive editing environment
(such as a text editor with an option
to have the current file processed by \LaTeX)
may require one to switch to the main file before compiling;
attempting to compile the child file produces errors.
\item
The main file must be modified (each time)
to adjust the |\includeonly| command
to the present needs. This easily leaves the main file in a messy state.
\item
The generated document will always carry the filename
of the main document. This is inconvenient if
several child files are to be compiled and
to be kept for distribution.
\end{itemize}

The present package provides a simple interface
to make child files individually compilable by \LaTeX{}.
Compiling a child file then has the same effect as compiling
the main file with an |\includeonly| command
to select the appropriate child.
Moreover the generated document will carry the name of the child
rather than the main file.
This resolves all three above issues.

This feature is meant to make the editing of books,
thesis documents and lecture notes somewhat more convenient.
However, the package can also be used efficiently for
composing a series of documents (such as exercise sheets)
which are typically distributed individually.
It then assists the author in generating the individual documents
(potentially in different versions)
as well as a document containing the collected series.
Another application is in developing style files
or other kinds of included material
where compilation of the style file could redirect
to a sample or test file.

%%%%%%%%%%%%%%%%%%%%%%%%%%%%%%%%%%%%%%%%%%%%%%%%%%%%%%%%%%%%%%%%%%%%%%%%%%%%%%%%
%%%%%%%%%%%%%%%%%%%%%%%%%%%%%%%%%%%%%%%%%%%%%%%%%%%%%%%%%%%%%%%%%%%%%%%%%%%%%%%%
\section{Usage}

First of all, the package \textsf{childdoc} is \emph{not} a standard
\LaTeXe{} |.sty| style file! Therefore it needs to be invoked in
a non-standard way.

%%%%%%%%%%%%%%%%%%%%%%%%%%%%%%%%%%%%%%%%%%%%%%%%%%%%%%%%%%%%%%%%%%%%%%%%%%%%%%%%
\subsection{Included Files}
\label{sec:include}

%%%%%%%%%%%%%%%%%%%%%%%%%%%%%%%%%%%%%%%%
\DescribeMacro{\childdocmain}
To use the package, add the commands
\begin{center}
\begin{tabular}{l}
|\input{childdoc.def}|\\
|\childdocmain{}|\\
\end{tabular}
\end{center}
at the very top of the main \LaTeX{} file,
in particular \emph{before} the |\documentclass| statement!
The argument of |\childdocmain| should be left empty
(but it must be present).

%%%%%%%%%%%%%%%%%%%%%%%%%%%%%%%%%%%%%%%%
\DescribeMacro{\childdocof}
Furthermore, add the commands
\begin{center}
\begin{tabular}{l}
|\input{childdoc.def}|\\
|\childdocof{|\textit{main}|}|\\
\end{tabular}
\end{center}
at the top of every child file \textit{child}
which is included by |\include{|\textit{child}|}|
from within the main file
(or at least for those files to be compiled individually).
The argument \textit{main} must be the filename of the main file.

There are a couple of
considerations in setting up the main and child documents:

%%%%%%%%%%%%%%%%%%%%%%%%%%%%%%%%%%%%%%%%
\paragraph{Restrictions.}

Please note the following restrictions:
\begin{itemize}
\item
|\childdocmain| must be called with one argument \textit{main}
to ensure compatibility with earlier version of the package.
It must either be empty (|\childdocmain{}|)
or precisely match the filename of the main file in which it is specified.
See \secref{sec:detection} for further information.
\item
The filename \textit{main} must be specified without the |.tex| extension.
\item
The filename \textit{main} is case sensitive
(even in case-insensitive file systems)
due to internal string comparison.
\item
The argument \textit{main} should be fully expanded, it cannot be a macro.
\item
Subdirectories and special characters should be avoided in filenames.
\item
The command |\childdocmain{|\textit{main}|}| must be followed by a whitespace.
It should not be followed immediately by another command
or by a comment mark `|%|'.
This is because the \TeX{} parser reads the token immediately following
the argument of |\childdocmain| and puts it
at the beginning of every child section;
however, a white\-space is ignored.
\end{itemize}

%%%%%%%%%%%%%%%%%%%%%%%%%%%%%%%%%%%%%%%%
\paragraph{Content of Main File.}

It is advisable to place all content in the child files included by |\include|.
Any output contained in the main file will appear in all child documents
unless suppressed manually;
it cannot be suppressed automatically by the |\includeonly| directive
and thus should normally be avoided.
A method to include some content in the main file
by means of conditional processing is described in \secref{sec:conditional}.

%%%%%%%%%%%%%%%%%%%%%%%%%%%%%%%%%%%%%%%%
\paragraph{Page Numbering.}

When only a part of the document is compiled,
the appropriate numbering of pages
(as well as other status parameters)
is determined from the |.aux| files.
The latter contain information from previous passes.
However this information needs to propagate through
all intermediate child documents.
Therefore the page numbering in child documents may well
be inconsistent until the complete document is compiled at least once.

A useful (if unconventional) way to always ensure a consistent
page numbering is to restart the numbering in each child document
and denote the pages by `\textit{child}|.|\textit{page}'
where \textit{child} represents the chapter/section number of the child file.
This can be achieved by the command
|\numberwithin{page}{|\textit{child}|}|
of the \textsf{amsmath} package
where \textit{child} can be |chapter| or |section|
depending on the chosen structuring.
Alternatively, one can modify the macro |\thepage| appropriately
and reset the counter |page| at the start of each child file.

%%%%%%%%%%%%%%%%%%%%%%%%%%%%%%%%%%%%%%%%%%%%%%%%%%%%%%%%%%%%%%%%%%%%%%%%%%%%%%%%
\subsection{Conditional Processing}
\label{sec:conditional}

The package provides a mechanism to compile different versions
of a document. To customise the versions further some conditional processing
can come in handy to distinguish which version is being compiled.
The package provides two macros to describe the compilation context:

%%%%%%%%%%%%%%%%%%%%%%%%%%%%%%%%%%%%%%%%
\DescribeMacro{\ifchilddoc}
The conditional |\ifchilddoc| distinguishes between the compilation of
child documents and the main document:
%
\begin{center}
|\ifchilddoc |\textit{child-code}| |[|\||else |\textit{main-code}]| \||fi|
\end{center}

%%%%%%%%%%%%%%%%%%%%%%%%%%%%%%%%%%%%%%%%
\DescribeMacro{\childdocname}
\DescribeMacro{\childdocjob}
The macro |\childdocname| contains the filename (without extension)
of the main or child file being processed.
Note that |\childdocjob| will always contain the name of the main file.

%%%%%%%%%%%%%%%%%%%%%%%%%%%%%%%%%%%%%%%%
\paragraph{Title Page.}

Conditional processing can be used to include a title or banner page
in the main document when proper precautions are taken.
Importantly, the code in the main file should ensure that the page counter
(as well as other status parameters which are stored in the |.aux| files)
takes the same value after the conditional processing.
Otherwise the page numbers may take divergent values
depending on which part is compiled.

For example, a title page could be declared by:
%
\begin{center}
\begin{tabular}{l}
|\ifchilddoc\||else|\\
|\addtocounter{page}{-1}|\\
\textit{code for title page}\\
|\newpage|\\
|\||fi|
\end{tabular}
\end{center}
%
A banner page for the child documents can be generated by:
%
\begin{center}
\begin{tabular}{l}
|\ifchilddoc|\\
|\addtocounter{page}{-1}|\\
\textit{code for banner page}\\
|\newpage|\\
|\||fi|
\end{tabular}
\end{center}
%
Here one could write a message such as:
\begin{center}
|This is the part \childdocname{} of \childdocjob{}.|
\end{center}

%%%%%%%%%%%%%%%%%%%%%%%%%%%%%%%%%%%%%%%%%%%%%%%%%%%%%%%%%%%%%%%%%%%%%%%%%%%%%%%%
\subsection{Flags}
\label{sec:flags}

The package makes it easy to generate different versions
of the main or child documents.
To this end compilation flags can be defined
and assigned different default values.
They will be particularly useful in conjunction
with the forwarding mechanism described in \secref{sec:forward}.

For example, it may be useful to have a flag |\version|
which can be set to |draft| or |final|.
The document source will contain some conditional code
depending on the value of |\version|.
Suppose further, the flag should default to |final| for the main file
and to |draft| for child files
which is a natural assignment for editing the document.
This is achieved by placing the following code
in the preamble of the main document
(below the |\childdocmain| directive):
%
\begin{center}
\begin{tabular}{l}
|\ifchilddoc|\\
|\providecommand{\version}{draft}|\\
|\||else|\\
|\providecommand{\version}{final}|\\
|\||fi|
\end{tabular}
\end{center}
%
The definition by |\providecommand| makes sure
that previous definitions are not overwritten.
Further statements |\providecommand{\version}{...}|
can thus be added before the above code to override it.

For the main file, one might add a line
(between |\childdocmain| and the above block)
%
\begin{center}
|%\ifchilddoc\||else\providecommand{\version}{draft}\||fi|
\end{center}
%
which can be uncommented to produce a draft version.
Likewise one can add a line to the very top of a child file
(above the |\childdocof{|\textit{main}|}| directive)
%
\begin{center}
|%\providecommand{\version}{final}|
\end{center}
%
which can be uncommented to produce the final version of this child document.

%%%%%%%%%%%%%%%%%%%%%%%%%%%%%%%%%%%%%%%%%%%%%%%%%%%%%%%%%%%%%%%%%%%%%%%%%%%%%%%%
\subsection{Forwarding}
\label{sec:forward}

Different versions of the main or child documents
using compilation flags as described in \secref{sec:flags}
can be (permanently) stored in different files
for convenient compilation, viewing and distribution.
To this end, the package defines a command
to pass on compilation to a different file:

%%%%%%%%%%%%%%%%%%%%%%%%%%%%%%%%%%%%%%%%
\DescribeMacro{\childdocforward}
The command |\childdocforward| redirects processing to
another source file:
%
\begin{center}
\begin{tabular}{l}
|\input{childdoc.def}|\\
|\childdocforward[|\textit{main}|]{|\textit{dest}|}|\\
\end{tabular}
\end{center}
%
The argument \textit{dest} is the destination file
(without extension).
It should be the main file or one of the child files.
Note that further \textsf{childdoc} directives
such as |\childdocof| and |\childdocforward|
in the indicated file will be processed in this form.
The optional argument \textit{main}
passes on directly to the main file \textit{main}
while pretending to compile the child \textit{dest}.
This form behaves as if \textit{dest}
issues |\childdocof{|\textit{main}|}| right away,
and no further \textsf{childdoc} directives will be processed.

%%%%%%%%%%%%%%%%%%%%%%%%%%%%%%%%%%%%%%%%
\DescribeMacro{\...prefix}
In the alternative form |\childdocforwardprefix|,
%
\begin{center}
\begin{tabular}{l}
|\input{childdoc.def}|\\
|\childdocforwardprefix[|\textit{main}|]{|\textit{prefix}|}{|\textit{dest}|}|
\end{tabular}
\end{center}
%
the destination file is determined by a pattern
depending on the current file:
To make this work, the current file must be called
`{\textit{prefix}\hspace{0.2em}\textit{suffix}}'
with \textit{prefix} matching precisely the argument.
Processing is then passed on to the file
`{\textit{dest}\hspace{0.2em}\textit{suffix}}'.
Surely, the same effect is achieved by
directly specifying the
argument `{\textit{dest}\hspace{0.2em}\textit{suffix}}'
in the first form.
However, that requires to set up a different file
for each child. With the alternative form of the command
all these files can have exactly the same content
which simplifies setting them up and maintaining them.

For example, the following file |draft.tex|
with a compilation flag |\version| as described in \secref{sec:flags}
compiles the main document as a draft:
%
\begin{center}
\begin{tabular}{l}
|\def\version{draft}|\\
|\input{childdoc.def}|\\
|\childdocforward{|\textit{main}|}|
\end{tabular}
\end{center}
%
Likewise, the following files |final|\textit{nn}|.tex|
compile the final version of the child document
|child|\textit{nn}|.tex|:
%
\begin{center}
\begin{tabular}{l}
|\def\version{final}|\\
|\input{childdoc.def}|\\
|\childdocforwardprefix{final}{child}|
\end{tabular}
\end{center}
%

Note that when several versions of a main file and/or of each child file
are to be generated, it may be convenient to set up a |Makefile| or
shell script to automatise the process.

%%%%%%%%%%%%%%%%%%%%%%%%%%%%%%%%%%%%%%%%%%%%%%%%%%%%%%%%%%%%%%%%%%%%%%%%%%%%%%%%
\subsection{Command Line Processing}
\label{sec:commandline}

The effect of redirection files can also be achieved by invoking
the \LaTeX{} compiler with a more elaborate command line.
Most conveniently this should be done as part
of a shell script or a |Makefile|.

When using \textsf{childdoc} in the main file, the following
command lines effectively perform a redirection
(note that depending on the shell being used,
backslashes may have to be doubled: `|\|' $\to$ `|\\|'):
%
\begin{center}
|... -jobname "|\textit{target}|" |\\|"|[\textit{flags}]%
|\input{childdoc.def}\childdocforward[|\textit{main}|]{|\textit{dest}|}"|
\end{center}
%
Here \textit{target} is the name of the output file,
\textit{main} is the name of the main file
and \textit{dest} is the name of the main or child file to be processed
(all filenames without extensions).
The optional argument \textit{main} can be omitted
if \textit{main} matches \textit{dest}.
Optionally, compilation \textit{flags} can be defined via |\def| commands.
This command line makes the \TeX{} engine believe
it is compiling the file \textit{target}
whose content is specified as the latter parameter.
The provided code then forwards the processing to
\textit{main} or \textit{dest} as described in \secref{sec:forward}.

%%%%%%%%%%%%%%%%%%%%%%%%%%%%%%%%%%%%%%%%%%%%%%%%%%%%%%%%%%%%%%%%%%%%%%%%%%%%%%%%
\subsection{Include by Input}
\label{sec:input}

Including child documents by |\include| has some restrictions by design.
Most notably, the content of a child document always occupies
its own set of pages; pages cannot be shared between child documents.
Usually, this behaviour makes perfect sense
because each child document contain an essential part of the document.
However, in some situations it may be desirable to compose
a document from a collection of parts
without having mandatory page breaks between then.
For this case, the package
provides a mechanism to include parts
by |\input| which can also be processed individually.
However, by construction this mechanism
requires manual handling of the content to be output.

%%%%%%%%%%%%%%%%%%%%%%%%%%%%%%%%%%%%%%%%
\DescribeMacro{\ifchilddocmanual}
The main file should be prepared as usual, see \secref{sec:include}.
However, the document body must make a distinction
between processing of an individual part and of the main document, e.g.:
%
\begin{center}
\begin{tabular}{l}
|\ifchilddocmanual|\\
|\input{\childdocname}|\\
|\||else|\\
\textit{document body with }|\input{|\textit{part}|}|\\
|\||fi|
\end{tabular}
\end{center}
%
The conditional |\ifchilddocmanual| is true whenever
a part to be included by |\input| is being compiled,
and the name of the part is stored in |\childdocname|.

%%%%%%%%%%%%%%%%%%%%%%%%%%%%%%%%%%%%%%%%
\DescribeMacro{\childdocby}
Each part to be included by |\input| should start with:
%
\begin{center}
\begin{tabular}{l}
|\input{childdoc.def}|\\
|\childdocby{|\textit{main}|}|\\
\end{tabular}
\end{center}
%
The directive |\childdocby| is similar to |\childdocof|
described in \secref{sec:include},
but the subsequent selection of content must be done manually.
To that end, both |\ifchilddoc| and |\ifchilddocmanual|
will be true upon processing of a part,
and the name of the part is stored in |\childdocname|.
Note that |\jobname| will be set to the filename of the current part
so that each part receives an individual |.aux| file
that does not interfere with the |.aux| file(s) of the main document.
This behaviour can be altered by the alternative form
|\childdocby[*]{|\textit{main}|}| (with a non-empty optional argument)
which uses the |.aux| file of the main document
by setting |\jobname| to \textit{main}.

%%%%%%%%%%%%%%%%%%%%%%%%%%%%%%%%%%%%%%%%%%%%%%%%%%%%%%%%%%%%%%%%%%%%%%%%%%%%%%%%
\subsection{Driver Development}
\label{sec:driver}

The \textsf{childdoc} mechanism can also be use for the development
of definition files such as \LaTeX{} styles or classes.
This case differs from the above setup with multiple parts
included by |\include| in that no |\includeonly| should be invoked.
This can be achieved by starting the include file
(before |\ProvidesPackage|) with:
%
\begin{center}
\begin{tabular}{l}
|\input{childdoc.def}|\\
|\childdocforward{|\textit{main}|}|\\
\end{tabular}
\end{center}
%
or alternatively with:
%
\begin{center}
\begin{tabular}{l}
|\input{childdoc.def}|\\
|\childdocby{|\textit{main}|}|\\
\end{tabular}
\end{center}
%
Both forms have slightly different effects as described above.
The main file is prepared as usual, see \secref{sec:include}.

%%%%%%%%%%%%%%%%%%%%%%%%%%%%%%%%%%%%%%%%%%%%%%%%%%%%%%%%%%%%%%%%%%%%%%%%%%%%%%%%
\subsection{Legacy Detection}
\label{sec:detection}

The directive |\childdocmain| in the main file can detect
whether the complete document or merely a child is to be compiled
even without using the directive |\childdocof|.
This method is deprecated because it is less robust
and there is no compelling reason to use it;
it is merely provided for backward compatibility
and it may be removed in future versions.

If the detection mechanism is to be used,
it is mandatory to correctly specify
the filename of the main file as the argument of |\childdocmain|:
%
\begin{center}
\begin{tabular}{l}
|\input{childdoc.def}|\\
|\childdocmain{|\textit{main}|}|\\
\end{tabular}
\end{center}
%
If |\jobname| does not match the argument \textit{main} of |\childdocmain|,
it is assumed that |\jobname| points to the child file to be compiled.
When using |\childdocmain| with the main file specified as argument,
it suffices to start a child file
with just |\input{|\textit{main}|}|
without loading of the package and using |\childdocof|.
If instead all processing is done
with the appropriate \textsf{childdoc} directives,
the argument of \textit{main} of |\childdocmain| can be empty.

An alternative version of the command line processing described
in \secref{sec:commandline} using the detection mechanism reads:
%
\begin{center}
|... -jobname "|\textit{target}|" "|[\textit{flags}]%
[|\def\jobname{|\textit{dest}|}|]|\input{|\textit{main}|}"|
\end{center}

%%%%%%%%%%%%%%%%%%%%%%%%%%%%%%%%%%%%%%%%%%%%%%%%%%%%%%%%%%%%%%%%%%%%%%%%%%%%%%%%
\subsection{Manual Code}
\label{sec:manual}

In case one cannot be certain whether the definitions file |childdoc.def|
is installed on the target \TeX{} distribution
and one prefers not to ship it,
it is conceivable to paste a few relevant commands into the sources.

To that end, drop all statements |\input{childdoc.def}|
and perform the replacements as outlined below.
Instead of |\childdocmain{|\textit{main}|}| add the following code
to the top of the main file:
%
\begin{center}
\begin{tabular}{l}
|\||ifdefined\childdocname\endinput\||fi\newif\ifchilddoc|\\
|\edef\childdocname{\scantokens\expandafter{\jobname\noexpand}}|\\
|\def\childdocmain{|\textit{main}|}\||ifx\childdocmain\childdocname\||else|\\
|\childdoctrue\includeonly{\childdocname}\let\jobname\childdocmain\||fi|\\
\end{tabular}
\end{center}
%
Instead of |\childdocof{|\textit{main}|}| just include the main file
at the top of each child file:
%
\begin{center}
|\input{|\textit{main}|}|
\end{center}
%
A simple redirection |\childdocforward{|\textit{dest}|}| is achieved by:
%
\begin{center}
|\def\jobname{|\textit{dest}|}\input{\jobname}|
\end{center}
%
The redirection with prefix
|\childdocforwardprefix[|\textit{prefix}|]{|\textit{dest}|}|
is accomplished by:
%
\begin{center}
\begin{tabular}{l}
|{\edef\jobname{\scantokens\expandafter{\jobname\noexpand}}|\\
|\def\redirectjob |\textit{prefix}|#1~~~{\gdef\jobname{|\textit{dest}|#1}}|\\
|\expandafter\redirectjob\jobname~~~}\input{\jobname}|
\end{tabular}
\end{center}

In an alternative approach,
child documents can be compiled by a specific command line
without additional code or specific definitions:
%
\begin{center}
|... -jobname "|\textit{target}|" "|[\textit{flags}]%
|\includeonly{|\textit{dest}|}\input{|\textit{main}|}"|
\end{center}
%

%%%%%%%%%%%%%%%%%%%%%%%%%%%%%%%%%%%%%%%%%%%%%%%%%%%%%%%%%%%%%%%%%%%%%%%%%%%%%%%%
%%%%%%%%%%%%%%%%%%%%%%%%%%%%%%%%%%%%%%%%%%%%%%%%%%%%%%%%%%%%%%%%%%%%%%%%%%%%%%%%
\section{Information}

%%%%%%%%%%%%%%%%%%%%%%%%%%%%%%%%%%%%%%%%%%%%%%%%%%%%%%%%%%%%%%%%%%%%%%%%%%%%%%%%
\subsection{Copyright}

Copyright \copyright{} 2017--2018 Niklas Beisert

This work may be distributed and/or modified under the
conditions of the \LaTeX{} Project Public License, either version 1.3
of this license or (at your option) any later version.
The latest version of this license is in
  \url{http://www.latex-project.org/lppl.txt}
and version 1.3 or later is part of all distributions of \LaTeX{}
version 2005/12/01 or later.

This work has the LPPL maintenance status `maintained'.

The Current Maintainer of this work is Niklas Beisert.

This work consists of the files |README.txt|, |childdoc.ins| and |childdoc.dtx|
as well as the derived files |childdoc.def|, |cdocsamp.tex|
with |cdocsch1.tex|, |cdocsch2.tex|, |cdocspt3.tex|, |cdocspt4.tex|,
|cdocsdrf.tex|, |cdocsfn1.tex|, |cdocsfn2.tex|
as well as |childdoc.pdf|.

%%%%%%%%%%%%%%%%%%%%%%%%%%%%%%%%%%%%%%%%%%%%%%%%%%%%%%%%%%%%%%%%%%%%%%%%%%%%%%%%
\subsection{Files and Installation}

The package consists of the files:
%
\begin{center}
\begin{tabular}{ll}
    |README.txt|   & readme file \\
    |childdoc.ins| & installation file \\
    |childdoc.dtx| & source file \\
    |childdoc.def| & definition file \\
    |cdocsamp.tex| & sample main file \\
    |cdocsch1.tex| & sample include file \\
    |cdocsch2.tex| & sample include file \\
    |cdocspt3.tex| & sample part file \\
    |cdocspt4.tex| & sample part file \\
    |cdocsdrf.tex| & sample redirection file \\
    |cdocsfn1.tex| & sample redirection file \\
    |cdocsfn2.tex| & sample redirection file \\
    |childdoc.pdf| & manual
\end{tabular}
\end{center}
%
The distribution consists of the files
|README.txt|, |childdoc.ins| and |childdoc.dtx|.
%
\begin{itemize}
\item
Run (pdf)\LaTeX{} on |childdoc.dtx|
to compile the manual |childdoc.pdf| (this file).
\item
Run \LaTeX{} on |childdoc.ins| to create the definitions file |childdoc.def|
and the sample |cdocsamp.tex| with include files
|cdocsch1.tex|, |cdocsch2.tex|, |cdocspt3.tex|, |cdocspt4.tex|,
|cdocsdrf.tex|, |cdocsfn1.tex|, |cdocsfn2.tex|.
Then copy the file |childdoc.def| to an appropriate directory of your \LaTeX{}
distribution, e.g.\ \textit{texmf-root}|/tex/latex/childdoc|.
\end{itemize}

%%%%%%%%%%%%%%%%%%%%%%%%%%%%%%%%%%%%%%%%%%%%%%%%%%%%%%%%%%%%%%%%%%%%%%%%%%%%%%%%
\subsection{Related CTAN Packages}

There are several other packages which offer a similar functionality:
%
\begin{itemize}
\item
The packages
\href{http://ctan.org/pkg/docmute}{\textsf{docmute}},
\href{http://ctan.org/pkg/includex}{\textsf{includex}} and
\href{http://ctan.org/pkg/standalone}{\textsf{standalone}}
provide commands to include only the document body of
a child file thus allowing both files to be compiled individually.
\item
The packages \href{http://ctan.org/pkg/subdocs}{\textsf{subdocs}}
and \href{http://ctan.org/pkg/subfiles}{\textsf{subfiles}}
provide structures in which the main and child documents can be
encapsulated and allowing them to be compiled individually.
The inclusion mechanism is different from the conventional |\include|.
\item
The package \href{http://ctan.org/pkg/combine}{\textsf{combine}}
is an elaborate solution to combine several documents into one.
\end{itemize}
%
See also the CTAN topic \href{http://ctan.org/topic/subdocs}{\textsf{subdocs}}
for further related packages.
The present package differs from the above solutions in that
a document structure constructed with the conventional |\include| mechanism
just needs two extra commands at the top of every file
such that all constituent files can be compiled individually.

%%%%%%%%%%%%%%%%%%%%%%%%%%%%%%%%%%%%%%%%%%%%%%%%%%%%%%%%%%%%%%%%%%%%%%%%%%%%%%%%
%\subsection{Feature Suggestions}
%
%The following is a list of features which may be useful for future
%versions of this package:
%%
%\begin{itemize}
%\item
%\ldots
%\end{itemize}

%%%%%%%%%%%%%%%%%%%%%%%%%%%%%%%%%%%%%%%%%%%%%%%%%%%%%%%%%%%%%%%%%%%%%%%%%%%%%%%%
\subsection{Revision History}

%%%%%%%%%%%%%%%%%%%%%%%%%%%%%%%%%%%%%%%%
\paragraph{v2.0:} 2018/12/30

\begin{itemize}
\item
immediate forward processing
\item
added |\childdocby| mechanism
\item
manual restructured
\end{itemize}

%%%%%%%%%%%%%%%%%%%%%%%%%%%%%%%%%%%%%%%%
\paragraph{v1.6:} 2018/01/17

\begin{itemize}
\item
application for development of include files
\item
corrections to manual
\end{itemize}

%%%%%%%%%%%%%%%%%%%%%%%%%%%%%%%%%%%%%%%%
\paragraph{v1.5:} 2017/05/21

\begin{itemize}
\item
more complete structuring introduced
\item
|\childdocof| introduced
\item
|\childdoc| renamed to |\childdocmain|
\item
|\childredirect| renamed to |\childdocforward| and |\childdocforwardprefix|
and functionality expanded
\end{itemize}

%%%%%%%%%%%%%%%%%%%%%%%%%%%%%%%%%%%%%%%%
\paragraph{v1.0:} 2017/04/27

\begin{itemize}
\item
manual and install package
\item
first version published on CTAN
\end{itemize}

%%%%%%%%%%%%%%%%%%%%%%%%%%%%%%%%%%%%%%%%
\paragraph{v0.6:} 2017/04/26

\begin{itemize}
\item
redirection mechanism added
\end{itemize}

%%%%%%%%%%%%%%%%%%%%%%%%%%%%%%%%%%%%%%%%
\paragraph{v0.5:} 2017/04/26

\begin{itemize}
\item
functionality in definition file
\end{itemize}


%%%%%%%%%%%%%%%%%%%%%%%%%%%%%%%%%%%%%%%%%%%%%%%%%%%%%%%%%%%%%%%%%%%%%%%%%%%%%%%%
%%%%%%%%%%%%%%%%%%%%%%%%%%%%%%%%%%%%%%%%%%%%%%%%%%%%%%%%%%%%%%%%%%%%%%%%%%%%%%%%
%%%%%%%%%%%%%%%%%%%%%%%%%%%%%%%%%%%%%%%%%%%%%%%%%%%%%%%%%%%%%%%%%%%%%%%%%%%%%%%%
\appendix

\settowidth\MacroIndent{\rmfamily\scriptsize 000\ }

 \DocInput{childdoc.dtx}

\end{document}
%</driver>
% \fi
%
% %%%%%%%%%%%%%%%%%%%%%%%%%%%%%%%%%%%%%%%%%%%%%%%%%%%%%%%%%%%%%%%%%%%%%%%%%%%%%%
% %%%%%%%%%%%%%%%%%%%%%%%%%%%%%%%%%%%%%%%%%%%%%%%%%%%%%%%%%%%%%%%%%%%%%%%%%%%%%%
% \section{Sample}
%\iffalse
%<*samplemain>
%\fi
%
% The following presents a sample document
% with two chapters, two parts, a title page,
% a compile flag as well as three forwarding files to set the flag.
% It consists of eight |.tex| files:
% \begin{center}
% \begin{tabular}{ll}
% |cdocsamp.tex|&main file\\
% |cdocsch1.tex|&include file for chapter 1\\
% |cdocsch2.tex|&include file for chapter 2\\
% |cdocspt3.tex|&include file for part 3\\
% |cdocspt4.tex|&include file for part 4\\
% |cdocsdrf.tex|&forwarding file for main file in draft mode\\
% |cdocsfi1.tex|&forwarding file for final version of chapter 1\\
% |cdocsfi2.tex|&forwarding file for final version of chapter 2\\
% \end{tabular}
% \end{center}
% Each of the eight files can be compiled directly by the \LaTeX{} compiler.
%
% %%%%%%%%%%%%%%%%%%%%%%%%%%%%%%%%%%%%%%
% \paragraph{Main File.}
%
% The main file is called |cdocsamp.tex|.
%
% Load the \textsf{childdoc} definitions and
% declare the filename for the main document:
%    \begin{macrocode}
\input{childdoc.def}
\childdocmain{}
%    \end{macrocode}

% Optional override for |\version| flag:
%    \begin{macrocode}
%%\ifchilddoc\else\providecommand{\version}{draft}\fi
%    \end{macrocode}

% Define the default values for the |\version| flag
% (|final| for the main file and |draft| for childs):
%    \begin{macrocode}
\ifchilddoc
\providecommand{\version}{draft}
\else
\providecommand{\version}{final}
\fi
%    \end{macrocode}

% Load the standard document class:
%    \begin{macrocode}
\documentclass[12pt]{article}
%    \end{macrocode}

% Start the document body:
%    \begin{macrocode}
\begin{document}
%    \end{macrocode}

% Declare a title page.
% Print title, part of document being processed and version flag:
%    \begin{macrocode}
\addtocounter{page}{-1}
\begin{center}
{\LARGE\bfseries{}childdoc example\par}
\vspace{1cm}
\ifchilddoc
\ifchilddocmanual part\else chapter\fi:
`\childdocname' of `\childdocjob'\par
\else
main document: `\childdocjob'\par
\fi
version: \version\par
\end{center}
\newpage
%    \end{macrocode}

% Manually include selected file,
% otherwise process as usual:
%    \begin{macrocode}
\ifchilddocmanual
\section*{part `\childdocname'}
\input{\childdocname}
\else
%    \end{macrocode}

% Include the two chapters:
%    \begin{macrocode}
\include{cdocsch1}
\include{cdocsch2}
%    \end{macrocode}

% Include the two parts unless only chapters should be displayed:
%    \begin{macrocode}
\ifchilddoc\else
\section{part three}
\input{cdocspt3}
\section{part four}
\input{cdocspt4}
\fi
%    \end{macrocode}

% Process as usual until here:
%    \begin{macrocode}
\fi
%    \end{macrocode}

% End of document body:
%    \begin{macrocode}
\end{document}
%    \end{macrocode}
%\iffalse
%</samplemain>
%\fi
%
% %%%%%%%%%%%%%%%%%%%%%%%%%%%%%%%%%%%%%%
% \paragraph{Chapter Include Files.}
%
% The include files are called |cdocsch1.tex| and |cdocsch2.tex|.
%
%\iffalse
%<*samplechap1|samplechap2>
%\fi

% Optional override for |\version| flag:
%    \begin{macrocode}
%%\providecommand{\version}{final}
%    \end{macrocode}

% Include the main document:
%    \begin{macrocode}
\input{childdoc.def}
\childdocof{cdocsamp}
%    \end{macrocode}

%\iffalse
%</samplechap1|samplechap2>
%\fi
%
%\iffalse
%<*samplechap1>
%\fi
% Some text for chapter 1:
%    \begin{macrocode}
\section{one}
some text in chapter one
%    \end{macrocode}

%\iffalse
%</samplechap1>
%\fi
% Some text for chapter 2:
%\iffalse
%<*samplechap2>
%\fi
%    \begin{macrocode}
\section{two}
more text in chapter two
%    \end{macrocode}

%\iffalse
%</samplechap2>
%\fi
%
% %%%%%%%%%%%%%%%%%%%%%%%%%%%%%%%%%%%%%%
% \paragraph{Part Include Files.}
%
% The include files are called |cdocspt3.tex| and |cdocspt4.tex|.
%
%\iffalse
%<*samplepart3|samplepart4>
%\fi

% Optional override for |\version| flag:
%    \begin{macrocode}
%%\providecommand{\version}{final}
%    \end{macrocode}

% Include the main document:
%    \begin{macrocode}
\input{childdoc.def}
\childdocby{cdocsamp}
%    \end{macrocode}

%\iffalse
%</samplepart3|samplepart4>
%\fi
%
%\iffalse
%<*samplepart3>
%\fi
% Some text for part 3:
%    \begin{macrocode}
some text in part three
%    \end{macrocode}

%\iffalse
%</samplepart3>
%\fi
% Some text for part 4:
%\iffalse
%<*samplepart4>
%\fi
%    \begin{macrocode}
more text in part four
%    \end{macrocode}

%\iffalse
%</samplepart4>
%\fi
%
% %%%%%%%%%%%%%%%%%%%%%%%%%%%%%%%%%%%%%%
% \paragraph{Forwarding for a Complete Draft.}
%
% The following forwarding file |cdocsdrf.tex|
% compiles the main document in draft mode:
%\iffalse
%<*sampledraft>
%\fi
%    \begin{macrocode}
\def\version{draft}
\input{childdoc.def}
\childdocforward{cdocsamp}
%    \end{macrocode}

%\iffalse
%</sampledraft>
%\fi
%
% %%%%%%%%%%%%%%%%%%%%%%%%%%%%%%%%%%%%%%
% \paragraph{Forwarding for Final Version of the Chapters.}
%
% The following forwarding files |cdocsfn1.tex| and |cdocsfn2.tex|
% (with identical content)
% compile the final versions of the child documents
% |cdocsch1.tex| and |cdocsch2.tex|, respectively:
%\iffalse
%<*samplefinal>
%\fi
%    \begin{macrocode}
\def\version{final}
\input{childdoc.def}
\childdocforwardprefix[cdocsamp]{cdocsfn}{cdocsch}
%    \end{macrocode}

%\iffalse
%</samplefinal>
%\fi
%
% %%%%%%%%%%%%%%%%%%%%%%%%%%%%%%%%%%%%%%
% \paragraph{Command Line Processing.}
%
% The following three command lines generate the output files
% |cdocscld|, |cdocscl1| and |cdocscl2|
% which should be identical to
% |cdocsdrf|, |cdocsch1| and |cdocsfn2|, respectively:
% \begin{center}
% \begin{tabular}{l}
% |latex -jobname cdocscld \|\\
% |  "\def\version{draft}\input{childdoc.def}\childdocforward{cdocsamp}"|\\
% |latex -jobname cdocscl1 \|\\
% |  "\input{childdoc.def}\childdocforward[cdocsamp]{cdocsch1}"|\\
% |latex -jobname cdocscl2 \|\\
% |  "\def\version{final}\input{childdoc.def}\childdocforward{cdocsch2}"|
% \end{tabular}
% \end{center}
% Note that the trailing backslash on each first line
% merely continues the input to the second line
% (for convenient cut ant paste).
% Furthermore, the command |latex| can be replaced by any
% of its alternative versions such as |pdflatex|.
%
% %%%%%%%%%%%%%%%%%%%%%%%%%%%%%%%%%%%%%%%%%%%%%%%%%%%%%%%%%%%%%%%%%%%%%%%%%%%%%%
% %%%%%%%%%%%%%%%%%%%%%%%%%%%%%%%%%%%%%%%%%%%%%%%%%%%%%%%%%%%%%%%%%%%%%%%%%%%%%%
% \section{Implementation}
%\iffalse
%<*package>
%\fi
%
% This section describes the definitions file |childdoc.def|.

% The definitions cannot be loaded using |\usepackage| or |\RequirePackage|
% which has a mechanism to prevent loading a style file more than once.
% When loading the definitions by means of |\input|
% multiple instances have to be prevented manually:
%\iffalse
%This code needs to be before the `\ProvidesFile' directive
%which is defined at the beginning of this file.
%Therefore it is also placed there and commented out here.
%</package>
%<*discard>
%\fi
%    \begin{macrocode}
\ifdefined\childdocmain\endinput\fi
%    \end{macrocode}
%\iffalse
%</discard>
%<*package>
%\fi
%
% \macro{\ifchilddoc}
% \macro{\ifchilddocmanual}
% The conditional |\ifchilddoc| tells whether a
% child (true) or main (false) document is being compiled.
% The conditional |\ifchilddocmanual| tells whether
% the |\includeonly| mechanism is used (false) or
% the selection of child files must be performed manually (true).
% The definitions initialise to false:
%    \begin{macrocode}
\newif\ifchilddoc
\newif\ifchilddocmanual
%    \end{macrocode}

% \macro{\childdocname}
% \macro{\childdocjob}
% The macro |\childdocname| stores the name of the main document
% to be compiled. The macro |\childdocjob| stores the name of
% the document on which the \LaTeX{} compiler was originally invoked.
% The content of |\jobname| cannot be compared
% to filenames specified in the source due to different catcodes.
% The following code rescans |\jobname|, stores the result
% in |\childdocname| and saves a copy in |\childdocjob|:
%    \begin{macrocode}
\edef\childdocname{\scantokens\expandafter{\jobname\noexpand}}
\let\childdocjob\childdocname
%    \end{macrocode}

% \macro{\childdocdisable}
% The macro |\childdocdisable| prevents the main file
% from being processed more than once.
% At this stage, the main document command |\childdocmain|
% is assumed to be called once again where it should do nothing.
% Any subsequent call to it should prevent
% a secondary processing of the main document
% It overwrites the forwarding commands
% |\childdocof| and |\childdocforward|
% with empty macros to prevent further inclusions of the main document:
%    \begin{macrocode}
\newcommand{\childdocdisable}
{
  \renewcommand{\childdocmain}[1]{\renewcommand{\childdocmain}[1]{\endinput}}
  \renewcommand{\childdocof}[1]{}
  \renewcommand{\childdocby}[2][]{}
  \renewcommand{\childdocforward}[2][]{}
  \renewcommand{\childdocdisable}{}
}
%    \end{macrocode}

% \macro{\childdocmain}
% The macro |\childdocmain| is to be called at the top of the main file
% with nothing or the main filename (without extension) as argument.
% First, it breaks loops.
% If the argument is not empty and does not match |\childdocname|
% (which is set by the first inclusion of |childdoc.def|),
% |\ifchilddoc| is set to true, |\includeonly| is applied to the child file
% and |\jobname| is set to the main file
% (for proper handling of |.aux| files):
%    \begin{macrocode}
\newcommand{\childdocmain}[1]
{
  \childdocdisable\childdocmain{}
  \if?#1?\else
    \begingroup
      \def\childdoctmp{#1}
      \ifx\childdoctmp\childdocname
        \def\childdoctmp{}
      \else
        \def\childdoctmp
        {
          \childdoctrue
          \includeonly{\childdocname}
          \def\childdocjob{#1}
          \def\jobname{#1}
        }
      \fi
      \expandafter
    \endgroup
    \childdoctmp
  \fi
}
%    \end{macrocode}

% \macro{\childdocof}
% The command |\childdocof| redirects
% compilation to the main file |#1|.
%    \begin{macrocode}
\newcommand{\childdocof}[1]
{
  \childdocdisable
  \childdoctrue
  \includeonly{\childdocname}
  \def\jobname{#1}
  \def\childdocjob{#1}
  \input{#1}
}
%    \end{macrocode}

% \macro{\childdocby}
% The command |\childdocby| ....
%    \begin{macrocode}
\newcommand{\childdocby}[2][]
{
  \childdocdisable
  \childdoctrue
  \childdocmanualtrue
  \if?#1?\else
    \def\jobname{#2}
  \fi
  \def\childdocjob{#2}
  \input{#2}
  \endinput
}
%    \end{macrocode}

% \macro{\childdocforward}
% The command |\childdocforward| redirects
% compilation to the main file or
% (if the optional argument is given) a child file.
% Parameters are set as if the main file
% or a child file starting with |\childdocof| was compiled.
% Then compilation is handed over to the main file:
%    \begin{macrocode}
\newcommand{\childdocforward}[2][]
{
  \begingroup
    \if?#1?
      \def\childdoctmp
      {
        \def\childdocname{#2}
        \def\childdocjob{#2}
        \def\jobname{#2}
        \input{#2}
        \endinput
      }
    \else
      \def\childdoctmp
      {
        \childdocdisable
        \def\childdocname{#2}
        \childdoctrue
        \includeonly{#2}
        \def\childdocjob{#1}
        \def\jobname{#1}
        \input{#1}
        \endinput
      }
    \fi
    \expandafter
  \endgroup
  \childdoctmp
}
%    \end{macrocode}

% \macro{\childdocforwardprefix}
% The command |\childdocforwardprefix| redirects
% compilation to the main or a child file by means of a pattern.
% The prefix |#1| in the current filename is replaced by |#2|
% and the suffix of the current filename is kept
% (it is assumed that the filename does not contain the substring `|~~~|'
% which is used as a delimiter).
% Compilation is handed over to the new file by |\childdocforward|:
%    \begin{macrocode}
\newcommand{\childdocforwardprefix}[3][]
{
  \begingroup
    \def\childdocextract #2##1~~~{\def\childdoctmp{\childdocforward[#1]{#3##1}}}
    \expandafter\childdocextract\childdocname~~~
    \expandafter
  \endgroup
  \childdoctmp
}
%    \end{macrocode}

% \macro{\childdoc}
% The deprecated macro |\childdoc| is a legacy version of |\childdocmain|:
%    \begin{macrocode}
\newcommand{\childdoc}{\childdocmain}
%    \end{macrocode}

% \macro{\childdocredirect}
% The deprecated macro |\childdocredirect| is a legacy version
% of |\childdocforward| and |\childdocforwardprefix|:
%    \begin{macrocode}
\newcommand{\childdocredirect}[2][]
{
  \begingroup
    \if?#1?
      \def\childdoctmp{\childdocforward{#2}}
    \else
      \def\childdoctmp{\childdocforwardprefix{#1}{#2}}
    \fi
    \expandafter
  \endgroup
  \childdoctmp
}
%    \end{macrocode}

%\iffalse
%</package>
%\fi
%
\endinput
\childdocforward[|\textit{main}|]{|\textit{dest}|}"|
\end{center}
%
Here \textit{target} is the name of the output file,
\textit{main} is the name of the main file
and \textit{dest} is the name of the main or child file to be processed
(all filenames without extensions).
The optional argument \textit{main} can be omitted
if \textit{main} matches \textit{dest}.
Optionally, compilation \textit{flags} can be defined via |\def| commands.
This command line makes the \TeX{} engine believe
it is compiling the file \textit{target}
whose content is specified as the latter parameter.
The provided code then forwards the processing to
\textit{main} or \textit{dest} as described in \secref{sec:forward}.

%%%%%%%%%%%%%%%%%%%%%%%%%%%%%%%%%%%%%%%%%%%%%%%%%%%%%%%%%%%%%%%%%%%%%%%%%%%%%%%%
\subsection{Include by Input}
\label{sec:input}

Including child documents by |\include| has some restrictions by design.
Most notably, the content of a child document always occupies
its own set of pages; pages cannot be shared between child documents.
Usually, this behaviour makes perfect sense
because each child document contain an essential part of the document.
However, in some situations it may be desirable to compose
a document from a collection of parts
without having mandatory page breaks between then.
For this case, the package
provides a mechanism to include parts
by |\input| which can also be processed individually.
However, by construction this mechanism
requires manual handling of the content to be output.

%%%%%%%%%%%%%%%%%%%%%%%%%%%%%%%%%%%%%%%%
\DescribeMacro{\ifchilddocmanual}
The main file should be prepared as usual, see \secref{sec:include}.
However, the document body must make a distinction
between processing of an individual part and of the main document, e.g.:
%
\begin{center}
\begin{tabular}{l}
|\ifchilddocmanual|\\
|\input{\childdocname}|\\
|\||else|\\
\textit{document body with }|\input{|\textit{part}|}|\\
|\||fi|
\end{tabular}
\end{center}
%
The conditional |\ifchilddocmanual| is true whenever
a part to be included by |\input| is being compiled,
and the name of the part is stored in |\childdocname|.

%%%%%%%%%%%%%%%%%%%%%%%%%%%%%%%%%%%%%%%%
\DescribeMacro{\childdocby}
Each part to be included by |\input| should start with:
%
\begin{center}
\begin{tabular}{l}
|% \iffalse
%
% childdoc.dtx Copyright (C) 2017-2018 Niklas Beisert
%
% This work may be distributed and/or modified under the
% conditions of the LaTeX Project Public License, either version 1.3
% of this license or (at your option) any later version.
% The latest version of this license is in
%   http://www.latex-project.org/lppl.txt
% and version 1.3 or later is part of all distributions of LaTeX
% version 2005/12/01 or later.
%
% This work has the LPPL maintenance status `maintained'.
%
% The Current Maintainer of this work is Niklas Beisert.
%
% This work consists of the files childdoc.dtx and childdoc.ins
% and the derived files childdoc.def and cdocsamp.tex with
% cdocsch1.tex, cdocsch2.tex, cdocsdrf.tex, cdocsfn1.tex, cdocsfn2.tex.
%
%<package>\ifdefined\childdocmain\endinput\fi
%<package>\ProvidesFile{childdoc.def}[2018/12/30 v2.0 child document driver]
%<samplemain>\ProvidesFile{cdocsamp.tex}[2018/12/30 v2.0 sample for childdoc]
%<*driver>
%\ProvidesFile{childdoc.drv}[2018/12/30 v2.0 childdoc reference manual file]
\PassOptionsToClass{10pt,a4paper}{article}
\documentclass{ltxdoc}

\usepackage[margin=35mm]{geometry}
\usepackage{hyperref}
\usepackage{hyperxmp}
\usepackage[usenames]{color}

\hypersetup{colorlinks=true}
\hypersetup{pdfstartview=FitH}
\hypersetup{pdfpagemode=UseNone}
\hypersetup{pdfsource={}}
\hypersetup{pdflang={en-UK}}
\hypersetup{pdfcopyright={Copyright 2017-2018 Niklas Beisert.
  This work may be distributed and/or modified under the
  conditions of the LaTeX Project Public License, either version 1.3
  of this license or (at your option) any later version.}}
\hypersetup{pdflicenseurl={http://www.latex-project.org/lppl.txt}}
\hypersetup{pdfcontactaddress={ETH Zurich, ITP, HIT K,
  Wolfgang-Pauli-Strasse 27}}
\hypersetup{pdfcontactpostcode={8093}}
\hypersetup{pdfcontactcity={Zurich}}
\hypersetup{pdfcontactcountry={Switzerland}}
\hypersetup{pdfcontactemail={nbeisert@itp.phys.ethz.ch}}
\hypersetup{pdfcontacturl={http://people.phys.ethz.ch/\xmptilde nbeisert/}}

\newcommand{\secref}[1]{\hyperref[#1]{section \ref*{#1}}}

\parskip1ex
\parindent0pt
\let\olditemize\itemize
\def\itemize{\olditemize\parskip0pt}

\begin{document}

\title{The \textsf{childdoc} Package}
\hypersetup{pdftitle={The childdoc Package}}
\author{Niklas Beisert\\[2ex]
  Institut f\"ur Theoretische Physik\\
  Eidgen\"ossische Technische Hochschule Z\"urich\\
  Wolfgang-Pauli-Strasse 27, 8093 Z\"urich, Switzerland\\[1ex]
  \href{mailto:nbeisert@itp.phys.ethz.ch}
  {\texttt{nbeisert@itp.phys.ethz.ch}}}
\hypersetup{pdfauthor={Niklas Beisert}}
\hypersetup{pdfsubject={Manual for the LaTeX2e Package childdoc}}
\date{30 December 2018, \textsf{v2.0}}
\maketitle

\begin{abstract}\noindent
\textsf{childdoc} is a \LaTeXe{} package
that enables the direct compilation
of document sections included by |\include|
to individual files.
\end{abstract}

\begingroup
\parskip0ex
\tableofcontents
\endgroup

%%%%%%%%%%%%%%%%%%%%%%%%%%%%%%%%%%%%%%%%%%%%%%%%%%%%%%%%%%%%%%%%%%%%%%%%%%%%%%%%
%%%%%%%%%%%%%%%%%%%%%%%%%%%%%%%%%%%%%%%%%%%%%%%%%%%%%%%%%%%%%%%%%%%%%%%%%%%%%%%%
\section{Introduction}

\LaTeX{} provides a mechanism to structure a large document (such as a book)
into a main file and several child files (containing the chapters)
using the |\include| command.
This mechanism is beneficial for documents
which span hundreds of pages in order to
make the source file(s) more manageable.
Moreover, compilation can be restricted to
selected child files by means of the |\includeonly| command.
The latter feature can be used to reduce the compilation time while editing
(this was significantly more useful in the earlier days of \LaTeX{})
or to generate a smaller document which is easier to navigate.
Another application of |\includeonly| is to generate
documents consisting of selected parts of the complete document.

However, there are a few drawbacks of the plain |\include| mechanism:
\begin{itemize}
\item
The child files cannot be compiled on their own,
they can only be compiled via the main file.
A naive editing environment
(such as a text editor with an option
to have the current file processed by \LaTeX)
may require one to switch to the main file before compiling;
attempting to compile the child file produces errors.
\item
The main file must be modified (each time)
to adjust the |\includeonly| command
to the present needs. This easily leaves the main file in a messy state.
\item
The generated document will always carry the filename
of the main document. This is inconvenient if
several child files are to be compiled and
to be kept for distribution.
\end{itemize}

The present package provides a simple interface
to make child files individually compilable by \LaTeX{}.
Compiling a child file then has the same effect as compiling
the main file with an |\includeonly| command
to select the appropriate child.
Moreover the generated document will carry the name of the child
rather than the main file.
This resolves all three above issues.

This feature is meant to make the editing of books,
thesis documents and lecture notes somewhat more convenient.
However, the package can also be used efficiently for
composing a series of documents (such as exercise sheets)
which are typically distributed individually.
It then assists the author in generating the individual documents
(potentially in different versions)
as well as a document containing the collected series.
Another application is in developing style files
or other kinds of included material
where compilation of the style file could redirect
to a sample or test file.

%%%%%%%%%%%%%%%%%%%%%%%%%%%%%%%%%%%%%%%%%%%%%%%%%%%%%%%%%%%%%%%%%%%%%%%%%%%%%%%%
%%%%%%%%%%%%%%%%%%%%%%%%%%%%%%%%%%%%%%%%%%%%%%%%%%%%%%%%%%%%%%%%%%%%%%%%%%%%%%%%
\section{Usage}

First of all, the package \textsf{childdoc} is \emph{not} a standard
\LaTeXe{} |.sty| style file! Therefore it needs to be invoked in
a non-standard way.

%%%%%%%%%%%%%%%%%%%%%%%%%%%%%%%%%%%%%%%%%%%%%%%%%%%%%%%%%%%%%%%%%%%%%%%%%%%%%%%%
\subsection{Included Files}
\label{sec:include}

%%%%%%%%%%%%%%%%%%%%%%%%%%%%%%%%%%%%%%%%
\DescribeMacro{\childdocmain}
To use the package, add the commands
\begin{center}
\begin{tabular}{l}
|\input{childdoc.def}|\\
|\childdocmain{}|\\
\end{tabular}
\end{center}
at the very top of the main \LaTeX{} file,
in particular \emph{before} the |\documentclass| statement!
The argument of |\childdocmain| should be left empty
(but it must be present).

%%%%%%%%%%%%%%%%%%%%%%%%%%%%%%%%%%%%%%%%
\DescribeMacro{\childdocof}
Furthermore, add the commands
\begin{center}
\begin{tabular}{l}
|\input{childdoc.def}|\\
|\childdocof{|\textit{main}|}|\\
\end{tabular}
\end{center}
at the top of every child file \textit{child}
which is included by |\include{|\textit{child}|}|
from within the main file
(or at least for those files to be compiled individually).
The argument \textit{main} must be the filename of the main file.

There are a couple of
considerations in setting up the main and child documents:

%%%%%%%%%%%%%%%%%%%%%%%%%%%%%%%%%%%%%%%%
\paragraph{Restrictions.}

Please note the following restrictions:
\begin{itemize}
\item
|\childdocmain| must be called with one argument \textit{main}
to ensure compatibility with earlier version of the package.
It must either be empty (|\childdocmain{}|)
or precisely match the filename of the main file in which it is specified.
See \secref{sec:detection} for further information.
\item
The filename \textit{main} must be specified without the |.tex| extension.
\item
The filename \textit{main} is case sensitive
(even in case-insensitive file systems)
due to internal string comparison.
\item
The argument \textit{main} should be fully expanded, it cannot be a macro.
\item
Subdirectories and special characters should be avoided in filenames.
\item
The command |\childdocmain{|\textit{main}|}| must be followed by a whitespace.
It should not be followed immediately by another command
or by a comment mark `|%|'.
This is because the \TeX{} parser reads the token immediately following
the argument of |\childdocmain| and puts it
at the beginning of every child section;
however, a white\-space is ignored.
\end{itemize}

%%%%%%%%%%%%%%%%%%%%%%%%%%%%%%%%%%%%%%%%
\paragraph{Content of Main File.}

It is advisable to place all content in the child files included by |\include|.
Any output contained in the main file will appear in all child documents
unless suppressed manually;
it cannot be suppressed automatically by the |\includeonly| directive
and thus should normally be avoided.
A method to include some content in the main file
by means of conditional processing is described in \secref{sec:conditional}.

%%%%%%%%%%%%%%%%%%%%%%%%%%%%%%%%%%%%%%%%
\paragraph{Page Numbering.}

When only a part of the document is compiled,
the appropriate numbering of pages
(as well as other status parameters)
is determined from the |.aux| files.
The latter contain information from previous passes.
However this information needs to propagate through
all intermediate child documents.
Therefore the page numbering in child documents may well
be inconsistent until the complete document is compiled at least once.

A useful (if unconventional) way to always ensure a consistent
page numbering is to restart the numbering in each child document
and denote the pages by `\textit{child}|.|\textit{page}'
where \textit{child} represents the chapter/section number of the child file.
This can be achieved by the command
|\numberwithin{page}{|\textit{child}|}|
of the \textsf{amsmath} package
where \textit{child} can be |chapter| or |section|
depending on the chosen structuring.
Alternatively, one can modify the macro |\thepage| appropriately
and reset the counter |page| at the start of each child file.

%%%%%%%%%%%%%%%%%%%%%%%%%%%%%%%%%%%%%%%%%%%%%%%%%%%%%%%%%%%%%%%%%%%%%%%%%%%%%%%%
\subsection{Conditional Processing}
\label{sec:conditional}

The package provides a mechanism to compile different versions
of a document. To customise the versions further some conditional processing
can come in handy to distinguish which version is being compiled.
The package provides two macros to describe the compilation context:

%%%%%%%%%%%%%%%%%%%%%%%%%%%%%%%%%%%%%%%%
\DescribeMacro{\ifchilddoc}
The conditional |\ifchilddoc| distinguishes between the compilation of
child documents and the main document:
%
\begin{center}
|\ifchilddoc |\textit{child-code}| |[|\||else |\textit{main-code}]| \||fi|
\end{center}

%%%%%%%%%%%%%%%%%%%%%%%%%%%%%%%%%%%%%%%%
\DescribeMacro{\childdocname}
\DescribeMacro{\childdocjob}
The macro |\childdocname| contains the filename (without extension)
of the main or child file being processed.
Note that |\childdocjob| will always contain the name of the main file.

%%%%%%%%%%%%%%%%%%%%%%%%%%%%%%%%%%%%%%%%
\paragraph{Title Page.}

Conditional processing can be used to include a title or banner page
in the main document when proper precautions are taken.
Importantly, the code in the main file should ensure that the page counter
(as well as other status parameters which are stored in the |.aux| files)
takes the same value after the conditional processing.
Otherwise the page numbers may take divergent values
depending on which part is compiled.

For example, a title page could be declared by:
%
\begin{center}
\begin{tabular}{l}
|\ifchilddoc\||else|\\
|\addtocounter{page}{-1}|\\
\textit{code for title page}\\
|\newpage|\\
|\||fi|
\end{tabular}
\end{center}
%
A banner page for the child documents can be generated by:
%
\begin{center}
\begin{tabular}{l}
|\ifchilddoc|\\
|\addtocounter{page}{-1}|\\
\textit{code for banner page}\\
|\newpage|\\
|\||fi|
\end{tabular}
\end{center}
%
Here one could write a message such as:
\begin{center}
|This is the part \childdocname{} of \childdocjob{}.|
\end{center}

%%%%%%%%%%%%%%%%%%%%%%%%%%%%%%%%%%%%%%%%%%%%%%%%%%%%%%%%%%%%%%%%%%%%%%%%%%%%%%%%
\subsection{Flags}
\label{sec:flags}

The package makes it easy to generate different versions
of the main or child documents.
To this end compilation flags can be defined
and assigned different default values.
They will be particularly useful in conjunction
with the forwarding mechanism described in \secref{sec:forward}.

For example, it may be useful to have a flag |\version|
which can be set to |draft| or |final|.
The document source will contain some conditional code
depending on the value of |\version|.
Suppose further, the flag should default to |final| for the main file
and to |draft| for child files
which is a natural assignment for editing the document.
This is achieved by placing the following code
in the preamble of the main document
(below the |\childdocmain| directive):
%
\begin{center}
\begin{tabular}{l}
|\ifchilddoc|\\
|\providecommand{\version}{draft}|\\
|\||else|\\
|\providecommand{\version}{final}|\\
|\||fi|
\end{tabular}
\end{center}
%
The definition by |\providecommand| makes sure
that previous definitions are not overwritten.
Further statements |\providecommand{\version}{...}|
can thus be added before the above code to override it.

For the main file, one might add a line
(between |\childdocmain| and the above block)
%
\begin{center}
|%\ifchilddoc\||else\providecommand{\version}{draft}\||fi|
\end{center}
%
which can be uncommented to produce a draft version.
Likewise one can add a line to the very top of a child file
(above the |\childdocof{|\textit{main}|}| directive)
%
\begin{center}
|%\providecommand{\version}{final}|
\end{center}
%
which can be uncommented to produce the final version of this child document.

%%%%%%%%%%%%%%%%%%%%%%%%%%%%%%%%%%%%%%%%%%%%%%%%%%%%%%%%%%%%%%%%%%%%%%%%%%%%%%%%
\subsection{Forwarding}
\label{sec:forward}

Different versions of the main or child documents
using compilation flags as described in \secref{sec:flags}
can be (permanently) stored in different files
for convenient compilation, viewing and distribution.
To this end, the package defines a command
to pass on compilation to a different file:

%%%%%%%%%%%%%%%%%%%%%%%%%%%%%%%%%%%%%%%%
\DescribeMacro{\childdocforward}
The command |\childdocforward| redirects processing to
another source file:
%
\begin{center}
\begin{tabular}{l}
|\input{childdoc.def}|\\
|\childdocforward[|\textit{main}|]{|\textit{dest}|}|\\
\end{tabular}
\end{center}
%
The argument \textit{dest} is the destination file
(without extension).
It should be the main file or one of the child files.
Note that further \textsf{childdoc} directives
such as |\childdocof| and |\childdocforward|
in the indicated file will be processed in this form.
The optional argument \textit{main}
passes on directly to the main file \textit{main}
while pretending to compile the child \textit{dest}.
This form behaves as if \textit{dest}
issues |\childdocof{|\textit{main}|}| right away,
and no further \textsf{childdoc} directives will be processed.

%%%%%%%%%%%%%%%%%%%%%%%%%%%%%%%%%%%%%%%%
\DescribeMacro{\...prefix}
In the alternative form |\childdocforwardprefix|,
%
\begin{center}
\begin{tabular}{l}
|\input{childdoc.def}|\\
|\childdocforwardprefix[|\textit{main}|]{|\textit{prefix}|}{|\textit{dest}|}|
\end{tabular}
\end{center}
%
the destination file is determined by a pattern
depending on the current file:
To make this work, the current file must be called
`{\textit{prefix}\hspace{0.2em}\textit{suffix}}'
with \textit{prefix} matching precisely the argument.
Processing is then passed on to the file
`{\textit{dest}\hspace{0.2em}\textit{suffix}}'.
Surely, the same effect is achieved by
directly specifying the
argument `{\textit{dest}\hspace{0.2em}\textit{suffix}}'
in the first form.
However, that requires to set up a different file
for each child. With the alternative form of the command
all these files can have exactly the same content
which simplifies setting them up and maintaining them.

For example, the following file |draft.tex|
with a compilation flag |\version| as described in \secref{sec:flags}
compiles the main document as a draft:
%
\begin{center}
\begin{tabular}{l}
|\def\version{draft}|\\
|\input{childdoc.def}|\\
|\childdocforward{|\textit{main}|}|
\end{tabular}
\end{center}
%
Likewise, the following files |final|\textit{nn}|.tex|
compile the final version of the child document
|child|\textit{nn}|.tex|:
%
\begin{center}
\begin{tabular}{l}
|\def\version{final}|\\
|\input{childdoc.def}|\\
|\childdocforwardprefix{final}{child}|
\end{tabular}
\end{center}
%

Note that when several versions of a main file and/or of each child file
are to be generated, it may be convenient to set up a |Makefile| or
shell script to automatise the process.

%%%%%%%%%%%%%%%%%%%%%%%%%%%%%%%%%%%%%%%%%%%%%%%%%%%%%%%%%%%%%%%%%%%%%%%%%%%%%%%%
\subsection{Command Line Processing}
\label{sec:commandline}

The effect of redirection files can also be achieved by invoking
the \LaTeX{} compiler with a more elaborate command line.
Most conveniently this should be done as part
of a shell script or a |Makefile|.

When using \textsf{childdoc} in the main file, the following
command lines effectively perform a redirection
(note that depending on the shell being used,
backslashes may have to be doubled: `|\|' $\to$ `|\\|'):
%
\begin{center}
|... -jobname "|\textit{target}|" |\\|"|[\textit{flags}]%
|\input{childdoc.def}\childdocforward[|\textit{main}|]{|\textit{dest}|}"|
\end{center}
%
Here \textit{target} is the name of the output file,
\textit{main} is the name of the main file
and \textit{dest} is the name of the main or child file to be processed
(all filenames without extensions).
The optional argument \textit{main} can be omitted
if \textit{main} matches \textit{dest}.
Optionally, compilation \textit{flags} can be defined via |\def| commands.
This command line makes the \TeX{} engine believe
it is compiling the file \textit{target}
whose content is specified as the latter parameter.
The provided code then forwards the processing to
\textit{main} or \textit{dest} as described in \secref{sec:forward}.

%%%%%%%%%%%%%%%%%%%%%%%%%%%%%%%%%%%%%%%%%%%%%%%%%%%%%%%%%%%%%%%%%%%%%%%%%%%%%%%%
\subsection{Include by Input}
\label{sec:input}

Including child documents by |\include| has some restrictions by design.
Most notably, the content of a child document always occupies
its own set of pages; pages cannot be shared between child documents.
Usually, this behaviour makes perfect sense
because each child document contain an essential part of the document.
However, in some situations it may be desirable to compose
a document from a collection of parts
without having mandatory page breaks between then.
For this case, the package
provides a mechanism to include parts
by |\input| which can also be processed individually.
However, by construction this mechanism
requires manual handling of the content to be output.

%%%%%%%%%%%%%%%%%%%%%%%%%%%%%%%%%%%%%%%%
\DescribeMacro{\ifchilddocmanual}
The main file should be prepared as usual, see \secref{sec:include}.
However, the document body must make a distinction
between processing of an individual part and of the main document, e.g.:
%
\begin{center}
\begin{tabular}{l}
|\ifchilddocmanual|\\
|\input{\childdocname}|\\
|\||else|\\
\textit{document body with }|\input{|\textit{part}|}|\\
|\||fi|
\end{tabular}
\end{center}
%
The conditional |\ifchilddocmanual| is true whenever
a part to be included by |\input| is being compiled,
and the name of the part is stored in |\childdocname|.

%%%%%%%%%%%%%%%%%%%%%%%%%%%%%%%%%%%%%%%%
\DescribeMacro{\childdocby}
Each part to be included by |\input| should start with:
%
\begin{center}
\begin{tabular}{l}
|\input{childdoc.def}|\\
|\childdocby{|\textit{main}|}|\\
\end{tabular}
\end{center}
%
The directive |\childdocby| is similar to |\childdocof|
described in \secref{sec:include},
but the subsequent selection of content must be done manually.
To that end, both |\ifchilddoc| and |\ifchilddocmanual|
will be true upon processing of a part,
and the name of the part is stored in |\childdocname|.
Note that |\jobname| will be set to the filename of the current part
so that each part receives an individual |.aux| file
that does not interfere with the |.aux| file(s) of the main document.
This behaviour can be altered by the alternative form
|\childdocby[*]{|\textit{main}|}| (with a non-empty optional argument)
which uses the |.aux| file of the main document
by setting |\jobname| to \textit{main}.

%%%%%%%%%%%%%%%%%%%%%%%%%%%%%%%%%%%%%%%%%%%%%%%%%%%%%%%%%%%%%%%%%%%%%%%%%%%%%%%%
\subsection{Driver Development}
\label{sec:driver}

The \textsf{childdoc} mechanism can also be use for the development
of definition files such as \LaTeX{} styles or classes.
This case differs from the above setup with multiple parts
included by |\include| in that no |\includeonly| should be invoked.
This can be achieved by starting the include file
(before |\ProvidesPackage|) with:
%
\begin{center}
\begin{tabular}{l}
|\input{childdoc.def}|\\
|\childdocforward{|\textit{main}|}|\\
\end{tabular}
\end{center}
%
or alternatively with:
%
\begin{center}
\begin{tabular}{l}
|\input{childdoc.def}|\\
|\childdocby{|\textit{main}|}|\\
\end{tabular}
\end{center}
%
Both forms have slightly different effects as described above.
The main file is prepared as usual, see \secref{sec:include}.

%%%%%%%%%%%%%%%%%%%%%%%%%%%%%%%%%%%%%%%%%%%%%%%%%%%%%%%%%%%%%%%%%%%%%%%%%%%%%%%%
\subsection{Legacy Detection}
\label{sec:detection}

The directive |\childdocmain| in the main file can detect
whether the complete document or merely a child is to be compiled
even without using the directive |\childdocof|.
This method is deprecated because it is less robust
and there is no compelling reason to use it;
it is merely provided for backward compatibility
and it may be removed in future versions.

If the detection mechanism is to be used,
it is mandatory to correctly specify
the filename of the main file as the argument of |\childdocmain|:
%
\begin{center}
\begin{tabular}{l}
|\input{childdoc.def}|\\
|\childdocmain{|\textit{main}|}|\\
\end{tabular}
\end{center}
%
If |\jobname| does not match the argument \textit{main} of |\childdocmain|,
it is assumed that |\jobname| points to the child file to be compiled.
When using |\childdocmain| with the main file specified as argument,
it suffices to start a child file
with just |\input{|\textit{main}|}|
without loading of the package and using |\childdocof|.
If instead all processing is done
with the appropriate \textsf{childdoc} directives,
the argument of \textit{main} of |\childdocmain| can be empty.

An alternative version of the command line processing described
in \secref{sec:commandline} using the detection mechanism reads:
%
\begin{center}
|... -jobname "|\textit{target}|" "|[\textit{flags}]%
[|\def\jobname{|\textit{dest}|}|]|\input{|\textit{main}|}"|
\end{center}

%%%%%%%%%%%%%%%%%%%%%%%%%%%%%%%%%%%%%%%%%%%%%%%%%%%%%%%%%%%%%%%%%%%%%%%%%%%%%%%%
\subsection{Manual Code}
\label{sec:manual}

In case one cannot be certain whether the definitions file |childdoc.def|
is installed on the target \TeX{} distribution
and one prefers not to ship it,
it is conceivable to paste a few relevant commands into the sources.

To that end, drop all statements |\input{childdoc.def}|
and perform the replacements as outlined below.
Instead of |\childdocmain{|\textit{main}|}| add the following code
to the top of the main file:
%
\begin{center}
\begin{tabular}{l}
|\||ifdefined\childdocname\endinput\||fi\newif\ifchilddoc|\\
|\edef\childdocname{\scantokens\expandafter{\jobname\noexpand}}|\\
|\def\childdocmain{|\textit{main}|}\||ifx\childdocmain\childdocname\||else|\\
|\childdoctrue\includeonly{\childdocname}\let\jobname\childdocmain\||fi|\\
\end{tabular}
\end{center}
%
Instead of |\childdocof{|\textit{main}|}| just include the main file
at the top of each child file:
%
\begin{center}
|\input{|\textit{main}|}|
\end{center}
%
A simple redirection |\childdocforward{|\textit{dest}|}| is achieved by:
%
\begin{center}
|\def\jobname{|\textit{dest}|}\input{\jobname}|
\end{center}
%
The redirection with prefix
|\childdocforwardprefix[|\textit{prefix}|]{|\textit{dest}|}|
is accomplished by:
%
\begin{center}
\begin{tabular}{l}
|{\edef\jobname{\scantokens\expandafter{\jobname\noexpand}}|\\
|\def\redirectjob |\textit{prefix}|#1~~~{\gdef\jobname{|\textit{dest}|#1}}|\\
|\expandafter\redirectjob\jobname~~~}\input{\jobname}|
\end{tabular}
\end{center}

In an alternative approach,
child documents can be compiled by a specific command line
without additional code or specific definitions:
%
\begin{center}
|... -jobname "|\textit{target}|" "|[\textit{flags}]%
|\includeonly{|\textit{dest}|}\input{|\textit{main}|}"|
\end{center}
%

%%%%%%%%%%%%%%%%%%%%%%%%%%%%%%%%%%%%%%%%%%%%%%%%%%%%%%%%%%%%%%%%%%%%%%%%%%%%%%%%
%%%%%%%%%%%%%%%%%%%%%%%%%%%%%%%%%%%%%%%%%%%%%%%%%%%%%%%%%%%%%%%%%%%%%%%%%%%%%%%%
\section{Information}

%%%%%%%%%%%%%%%%%%%%%%%%%%%%%%%%%%%%%%%%%%%%%%%%%%%%%%%%%%%%%%%%%%%%%%%%%%%%%%%%
\subsection{Copyright}

Copyright \copyright{} 2017--2018 Niklas Beisert

This work may be distributed and/or modified under the
conditions of the \LaTeX{} Project Public License, either version 1.3
of this license or (at your option) any later version.
The latest version of this license is in
  \url{http://www.latex-project.org/lppl.txt}
and version 1.3 or later is part of all distributions of \LaTeX{}
version 2005/12/01 or later.

This work has the LPPL maintenance status `maintained'.

The Current Maintainer of this work is Niklas Beisert.

This work consists of the files |README.txt|, |childdoc.ins| and |childdoc.dtx|
as well as the derived files |childdoc.def|, |cdocsamp.tex|
with |cdocsch1.tex|, |cdocsch2.tex|, |cdocspt3.tex|, |cdocspt4.tex|,
|cdocsdrf.tex|, |cdocsfn1.tex|, |cdocsfn2.tex|
as well as |childdoc.pdf|.

%%%%%%%%%%%%%%%%%%%%%%%%%%%%%%%%%%%%%%%%%%%%%%%%%%%%%%%%%%%%%%%%%%%%%%%%%%%%%%%%
\subsection{Files and Installation}

The package consists of the files:
%
\begin{center}
\begin{tabular}{ll}
    |README.txt|   & readme file \\
    |childdoc.ins| & installation file \\
    |childdoc.dtx| & source file \\
    |childdoc.def| & definition file \\
    |cdocsamp.tex| & sample main file \\
    |cdocsch1.tex| & sample include file \\
    |cdocsch2.tex| & sample include file \\
    |cdocspt3.tex| & sample part file \\
    |cdocspt4.tex| & sample part file \\
    |cdocsdrf.tex| & sample redirection file \\
    |cdocsfn1.tex| & sample redirection file \\
    |cdocsfn2.tex| & sample redirection file \\
    |childdoc.pdf| & manual
\end{tabular}
\end{center}
%
The distribution consists of the files
|README.txt|, |childdoc.ins| and |childdoc.dtx|.
%
\begin{itemize}
\item
Run (pdf)\LaTeX{} on |childdoc.dtx|
to compile the manual |childdoc.pdf| (this file).
\item
Run \LaTeX{} on |childdoc.ins| to create the definitions file |childdoc.def|
and the sample |cdocsamp.tex| with include files
|cdocsch1.tex|, |cdocsch2.tex|, |cdocspt3.tex|, |cdocspt4.tex|,
|cdocsdrf.tex|, |cdocsfn1.tex|, |cdocsfn2.tex|.
Then copy the file |childdoc.def| to an appropriate directory of your \LaTeX{}
distribution, e.g.\ \textit{texmf-root}|/tex/latex/childdoc|.
\end{itemize}

%%%%%%%%%%%%%%%%%%%%%%%%%%%%%%%%%%%%%%%%%%%%%%%%%%%%%%%%%%%%%%%%%%%%%%%%%%%%%%%%
\subsection{Related CTAN Packages}

There are several other packages which offer a similar functionality:
%
\begin{itemize}
\item
The packages
\href{http://ctan.org/pkg/docmute}{\textsf{docmute}},
\href{http://ctan.org/pkg/includex}{\textsf{includex}} and
\href{http://ctan.org/pkg/standalone}{\textsf{standalone}}
provide commands to include only the document body of
a child file thus allowing both files to be compiled individually.
\item
The packages \href{http://ctan.org/pkg/subdocs}{\textsf{subdocs}}
and \href{http://ctan.org/pkg/subfiles}{\textsf{subfiles}}
provide structures in which the main and child documents can be
encapsulated and allowing them to be compiled individually.
The inclusion mechanism is different from the conventional |\include|.
\item
The package \href{http://ctan.org/pkg/combine}{\textsf{combine}}
is an elaborate solution to combine several documents into one.
\end{itemize}
%
See also the CTAN topic \href{http://ctan.org/topic/subdocs}{\textsf{subdocs}}
for further related packages.
The present package differs from the above solutions in that
a document structure constructed with the conventional |\include| mechanism
just needs two extra commands at the top of every file
such that all constituent files can be compiled individually.

%%%%%%%%%%%%%%%%%%%%%%%%%%%%%%%%%%%%%%%%%%%%%%%%%%%%%%%%%%%%%%%%%%%%%%%%%%%%%%%%
%\subsection{Feature Suggestions}
%
%The following is a list of features which may be useful for future
%versions of this package:
%%
%\begin{itemize}
%\item
%\ldots
%\end{itemize}

%%%%%%%%%%%%%%%%%%%%%%%%%%%%%%%%%%%%%%%%%%%%%%%%%%%%%%%%%%%%%%%%%%%%%%%%%%%%%%%%
\subsection{Revision History}

%%%%%%%%%%%%%%%%%%%%%%%%%%%%%%%%%%%%%%%%
\paragraph{v2.0:} 2018/12/30

\begin{itemize}
\item
immediate forward processing
\item
added |\childdocby| mechanism
\item
manual restructured
\end{itemize}

%%%%%%%%%%%%%%%%%%%%%%%%%%%%%%%%%%%%%%%%
\paragraph{v1.6:} 2018/01/17

\begin{itemize}
\item
application for development of include files
\item
corrections to manual
\end{itemize}

%%%%%%%%%%%%%%%%%%%%%%%%%%%%%%%%%%%%%%%%
\paragraph{v1.5:} 2017/05/21

\begin{itemize}
\item
more complete structuring introduced
\item
|\childdocof| introduced
\item
|\childdoc| renamed to |\childdocmain|
\item
|\childredirect| renamed to |\childdocforward| and |\childdocforwardprefix|
and functionality expanded
\end{itemize}

%%%%%%%%%%%%%%%%%%%%%%%%%%%%%%%%%%%%%%%%
\paragraph{v1.0:} 2017/04/27

\begin{itemize}
\item
manual and install package
\item
first version published on CTAN
\end{itemize}

%%%%%%%%%%%%%%%%%%%%%%%%%%%%%%%%%%%%%%%%
\paragraph{v0.6:} 2017/04/26

\begin{itemize}
\item
redirection mechanism added
\end{itemize}

%%%%%%%%%%%%%%%%%%%%%%%%%%%%%%%%%%%%%%%%
\paragraph{v0.5:} 2017/04/26

\begin{itemize}
\item
functionality in definition file
\end{itemize}


%%%%%%%%%%%%%%%%%%%%%%%%%%%%%%%%%%%%%%%%%%%%%%%%%%%%%%%%%%%%%%%%%%%%%%%%%%%%%%%%
%%%%%%%%%%%%%%%%%%%%%%%%%%%%%%%%%%%%%%%%%%%%%%%%%%%%%%%%%%%%%%%%%%%%%%%%%%%%%%%%
%%%%%%%%%%%%%%%%%%%%%%%%%%%%%%%%%%%%%%%%%%%%%%%%%%%%%%%%%%%%%%%%%%%%%%%%%%%%%%%%
\appendix

\settowidth\MacroIndent{\rmfamily\scriptsize 000\ }

 \DocInput{childdoc.dtx}

\end{document}
%</driver>
% \fi
%
% %%%%%%%%%%%%%%%%%%%%%%%%%%%%%%%%%%%%%%%%%%%%%%%%%%%%%%%%%%%%%%%%%%%%%%%%%%%%%%
% %%%%%%%%%%%%%%%%%%%%%%%%%%%%%%%%%%%%%%%%%%%%%%%%%%%%%%%%%%%%%%%%%%%%%%%%%%%%%%
% \section{Sample}
%\iffalse
%<*samplemain>
%\fi
%
% The following presents a sample document
% with two chapters, two parts, a title page,
% a compile flag as well as three forwarding files to set the flag.
% It consists of eight |.tex| files:
% \begin{center}
% \begin{tabular}{ll}
% |cdocsamp.tex|&main file\\
% |cdocsch1.tex|&include file for chapter 1\\
% |cdocsch2.tex|&include file for chapter 2\\
% |cdocspt3.tex|&include file for part 3\\
% |cdocspt4.tex|&include file for part 4\\
% |cdocsdrf.tex|&forwarding file for main file in draft mode\\
% |cdocsfi1.tex|&forwarding file for final version of chapter 1\\
% |cdocsfi2.tex|&forwarding file for final version of chapter 2\\
% \end{tabular}
% \end{center}
% Each of the eight files can be compiled directly by the \LaTeX{} compiler.
%
% %%%%%%%%%%%%%%%%%%%%%%%%%%%%%%%%%%%%%%
% \paragraph{Main File.}
%
% The main file is called |cdocsamp.tex|.
%
% Load the \textsf{childdoc} definitions and
% declare the filename for the main document:
%    \begin{macrocode}
\input{childdoc.def}
\childdocmain{}
%    \end{macrocode}

% Optional override for |\version| flag:
%    \begin{macrocode}
%%\ifchilddoc\else\providecommand{\version}{draft}\fi
%    \end{macrocode}

% Define the default values for the |\version| flag
% (|final| for the main file and |draft| for childs):
%    \begin{macrocode}
\ifchilddoc
\providecommand{\version}{draft}
\else
\providecommand{\version}{final}
\fi
%    \end{macrocode}

% Load the standard document class:
%    \begin{macrocode}
\documentclass[12pt]{article}
%    \end{macrocode}

% Start the document body:
%    \begin{macrocode}
\begin{document}
%    \end{macrocode}

% Declare a title page.
% Print title, part of document being processed and version flag:
%    \begin{macrocode}
\addtocounter{page}{-1}
\begin{center}
{\LARGE\bfseries{}childdoc example\par}
\vspace{1cm}
\ifchilddoc
\ifchilddocmanual part\else chapter\fi:
`\childdocname' of `\childdocjob'\par
\else
main document: `\childdocjob'\par
\fi
version: \version\par
\end{center}
\newpage
%    \end{macrocode}

% Manually include selected file,
% otherwise process as usual:
%    \begin{macrocode}
\ifchilddocmanual
\section*{part `\childdocname'}
\input{\childdocname}
\else
%    \end{macrocode}

% Include the two chapters:
%    \begin{macrocode}
\include{cdocsch1}
\include{cdocsch2}
%    \end{macrocode}

% Include the two parts unless only chapters should be displayed:
%    \begin{macrocode}
\ifchilddoc\else
\section{part three}
\input{cdocspt3}
\section{part four}
\input{cdocspt4}
\fi
%    \end{macrocode}

% Process as usual until here:
%    \begin{macrocode}
\fi
%    \end{macrocode}

% End of document body:
%    \begin{macrocode}
\end{document}
%    \end{macrocode}
%\iffalse
%</samplemain>
%\fi
%
% %%%%%%%%%%%%%%%%%%%%%%%%%%%%%%%%%%%%%%
% \paragraph{Chapter Include Files.}
%
% The include files are called |cdocsch1.tex| and |cdocsch2.tex|.
%
%\iffalse
%<*samplechap1|samplechap2>
%\fi

% Optional override for |\version| flag:
%    \begin{macrocode}
%%\providecommand{\version}{final}
%    \end{macrocode}

% Include the main document:
%    \begin{macrocode}
\input{childdoc.def}
\childdocof{cdocsamp}
%    \end{macrocode}

%\iffalse
%</samplechap1|samplechap2>
%\fi
%
%\iffalse
%<*samplechap1>
%\fi
% Some text for chapter 1:
%    \begin{macrocode}
\section{one}
some text in chapter one
%    \end{macrocode}

%\iffalse
%</samplechap1>
%\fi
% Some text for chapter 2:
%\iffalse
%<*samplechap2>
%\fi
%    \begin{macrocode}
\section{two}
more text in chapter two
%    \end{macrocode}

%\iffalse
%</samplechap2>
%\fi
%
% %%%%%%%%%%%%%%%%%%%%%%%%%%%%%%%%%%%%%%
% \paragraph{Part Include Files.}
%
% The include files are called |cdocspt3.tex| and |cdocspt4.tex|.
%
%\iffalse
%<*samplepart3|samplepart4>
%\fi

% Optional override for |\version| flag:
%    \begin{macrocode}
%%\providecommand{\version}{final}
%    \end{macrocode}

% Include the main document:
%    \begin{macrocode}
\input{childdoc.def}
\childdocby{cdocsamp}
%    \end{macrocode}

%\iffalse
%</samplepart3|samplepart4>
%\fi
%
%\iffalse
%<*samplepart3>
%\fi
% Some text for part 3:
%    \begin{macrocode}
some text in part three
%    \end{macrocode}

%\iffalse
%</samplepart3>
%\fi
% Some text for part 4:
%\iffalse
%<*samplepart4>
%\fi
%    \begin{macrocode}
more text in part four
%    \end{macrocode}

%\iffalse
%</samplepart4>
%\fi
%
% %%%%%%%%%%%%%%%%%%%%%%%%%%%%%%%%%%%%%%
% \paragraph{Forwarding for a Complete Draft.}
%
% The following forwarding file |cdocsdrf.tex|
% compiles the main document in draft mode:
%\iffalse
%<*sampledraft>
%\fi
%    \begin{macrocode}
\def\version{draft}
\input{childdoc.def}
\childdocforward{cdocsamp}
%    \end{macrocode}

%\iffalse
%</sampledraft>
%\fi
%
% %%%%%%%%%%%%%%%%%%%%%%%%%%%%%%%%%%%%%%
% \paragraph{Forwarding for Final Version of the Chapters.}
%
% The following forwarding files |cdocsfn1.tex| and |cdocsfn2.tex|
% (with identical content)
% compile the final versions of the child documents
% |cdocsch1.tex| and |cdocsch2.tex|, respectively:
%\iffalse
%<*samplefinal>
%\fi
%    \begin{macrocode}
\def\version{final}
\input{childdoc.def}
\childdocforwardprefix[cdocsamp]{cdocsfn}{cdocsch}
%    \end{macrocode}

%\iffalse
%</samplefinal>
%\fi
%
% %%%%%%%%%%%%%%%%%%%%%%%%%%%%%%%%%%%%%%
% \paragraph{Command Line Processing.}
%
% The following three command lines generate the output files
% |cdocscld|, |cdocscl1| and |cdocscl2|
% which should be identical to
% |cdocsdrf|, |cdocsch1| and |cdocsfn2|, respectively:
% \begin{center}
% \begin{tabular}{l}
% |latex -jobname cdocscld \|\\
% |  "\def\version{draft}\input{childdoc.def}\childdocforward{cdocsamp}"|\\
% |latex -jobname cdocscl1 \|\\
% |  "\input{childdoc.def}\childdocforward[cdocsamp]{cdocsch1}"|\\
% |latex -jobname cdocscl2 \|\\
% |  "\def\version{final}\input{childdoc.def}\childdocforward{cdocsch2}"|
% \end{tabular}
% \end{center}
% Note that the trailing backslash on each first line
% merely continues the input to the second line
% (for convenient cut ant paste).
% Furthermore, the command |latex| can be replaced by any
% of its alternative versions such as |pdflatex|.
%
% %%%%%%%%%%%%%%%%%%%%%%%%%%%%%%%%%%%%%%%%%%%%%%%%%%%%%%%%%%%%%%%%%%%%%%%%%%%%%%
% %%%%%%%%%%%%%%%%%%%%%%%%%%%%%%%%%%%%%%%%%%%%%%%%%%%%%%%%%%%%%%%%%%%%%%%%%%%%%%
% \section{Implementation}
%\iffalse
%<*package>
%\fi
%
% This section describes the definitions file |childdoc.def|.

% The definitions cannot be loaded using |\usepackage| or |\RequirePackage|
% which has a mechanism to prevent loading a style file more than once.
% When loading the definitions by means of |\input|
% multiple instances have to be prevented manually:
%\iffalse
%This code needs to be before the `\ProvidesFile' directive
%which is defined at the beginning of this file.
%Therefore it is also placed there and commented out here.
%</package>
%<*discard>
%\fi
%    \begin{macrocode}
\ifdefined\childdocmain\endinput\fi
%    \end{macrocode}
%\iffalse
%</discard>
%<*package>
%\fi
%
% \macro{\ifchilddoc}
% \macro{\ifchilddocmanual}
% The conditional |\ifchilddoc| tells whether a
% child (true) or main (false) document is being compiled.
% The conditional |\ifchilddocmanual| tells whether
% the |\includeonly| mechanism is used (false) or
% the selection of child files must be performed manually (true).
% The definitions initialise to false:
%    \begin{macrocode}
\newif\ifchilddoc
\newif\ifchilddocmanual
%    \end{macrocode}

% \macro{\childdocname}
% \macro{\childdocjob}
% The macro |\childdocname| stores the name of the main document
% to be compiled. The macro |\childdocjob| stores the name of
% the document on which the \LaTeX{} compiler was originally invoked.
% The content of |\jobname| cannot be compared
% to filenames specified in the source due to different catcodes.
% The following code rescans |\jobname|, stores the result
% in |\childdocname| and saves a copy in |\childdocjob|:
%    \begin{macrocode}
\edef\childdocname{\scantokens\expandafter{\jobname\noexpand}}
\let\childdocjob\childdocname
%    \end{macrocode}

% \macro{\childdocdisable}
% The macro |\childdocdisable| prevents the main file
% from being processed more than once.
% At this stage, the main document command |\childdocmain|
% is assumed to be called once again where it should do nothing.
% Any subsequent call to it should prevent
% a secondary processing of the main document
% It overwrites the forwarding commands
% |\childdocof| and |\childdocforward|
% with empty macros to prevent further inclusions of the main document:
%    \begin{macrocode}
\newcommand{\childdocdisable}
{
  \renewcommand{\childdocmain}[1]{\renewcommand{\childdocmain}[1]{\endinput}}
  \renewcommand{\childdocof}[1]{}
  \renewcommand{\childdocby}[2][]{}
  \renewcommand{\childdocforward}[2][]{}
  \renewcommand{\childdocdisable}{}
}
%    \end{macrocode}

% \macro{\childdocmain}
% The macro |\childdocmain| is to be called at the top of the main file
% with nothing or the main filename (without extension) as argument.
% First, it breaks loops.
% If the argument is not empty and does not match |\childdocname|
% (which is set by the first inclusion of |childdoc.def|),
% |\ifchilddoc| is set to true, |\includeonly| is applied to the child file
% and |\jobname| is set to the main file
% (for proper handling of |.aux| files):
%    \begin{macrocode}
\newcommand{\childdocmain}[1]
{
  \childdocdisable\childdocmain{}
  \if?#1?\else
    \begingroup
      \def\childdoctmp{#1}
      \ifx\childdoctmp\childdocname
        \def\childdoctmp{}
      \else
        \def\childdoctmp
        {
          \childdoctrue
          \includeonly{\childdocname}
          \def\childdocjob{#1}
          \def\jobname{#1}
        }
      \fi
      \expandafter
    \endgroup
    \childdoctmp
  \fi
}
%    \end{macrocode}

% \macro{\childdocof}
% The command |\childdocof| redirects
% compilation to the main file |#1|.
%    \begin{macrocode}
\newcommand{\childdocof}[1]
{
  \childdocdisable
  \childdoctrue
  \includeonly{\childdocname}
  \def\jobname{#1}
  \def\childdocjob{#1}
  \input{#1}
}
%    \end{macrocode}

% \macro{\childdocby}
% The command |\childdocby| ....
%    \begin{macrocode}
\newcommand{\childdocby}[2][]
{
  \childdocdisable
  \childdoctrue
  \childdocmanualtrue
  \if?#1?\else
    \def\jobname{#2}
  \fi
  \def\childdocjob{#2}
  \input{#2}
  \endinput
}
%    \end{macrocode}

% \macro{\childdocforward}
% The command |\childdocforward| redirects
% compilation to the main file or
% (if the optional argument is given) a child file.
% Parameters are set as if the main file
% or a child file starting with |\childdocof| was compiled.
% Then compilation is handed over to the main file:
%    \begin{macrocode}
\newcommand{\childdocforward}[2][]
{
  \begingroup
    \if?#1?
      \def\childdoctmp
      {
        \def\childdocname{#2}
        \def\childdocjob{#2}
        \def\jobname{#2}
        \input{#2}
        \endinput
      }
    \else
      \def\childdoctmp
      {
        \childdocdisable
        \def\childdocname{#2}
        \childdoctrue
        \includeonly{#2}
        \def\childdocjob{#1}
        \def\jobname{#1}
        \input{#1}
        \endinput
      }
    \fi
    \expandafter
  \endgroup
  \childdoctmp
}
%    \end{macrocode}

% \macro{\childdocforwardprefix}
% The command |\childdocforwardprefix| redirects
% compilation to the main or a child file by means of a pattern.
% The prefix |#1| in the current filename is replaced by |#2|
% and the suffix of the current filename is kept
% (it is assumed that the filename does not contain the substring `|~~~|'
% which is used as a delimiter).
% Compilation is handed over to the new file by |\childdocforward|:
%    \begin{macrocode}
\newcommand{\childdocforwardprefix}[3][]
{
  \begingroup
    \def\childdocextract #2##1~~~{\def\childdoctmp{\childdocforward[#1]{#3##1}}}
    \expandafter\childdocextract\childdocname~~~
    \expandafter
  \endgroup
  \childdoctmp
}
%    \end{macrocode}

% \macro{\childdoc}
% The deprecated macro |\childdoc| is a legacy version of |\childdocmain|:
%    \begin{macrocode}
\newcommand{\childdoc}{\childdocmain}
%    \end{macrocode}

% \macro{\childdocredirect}
% The deprecated macro |\childdocredirect| is a legacy version
% of |\childdocforward| and |\childdocforwardprefix|:
%    \begin{macrocode}
\newcommand{\childdocredirect}[2][]
{
  \begingroup
    \if?#1?
      \def\childdoctmp{\childdocforward{#2}}
    \else
      \def\childdoctmp{\childdocforwardprefix{#1}{#2}}
    \fi
    \expandafter
  \endgroup
  \childdoctmp
}
%    \end{macrocode}

%\iffalse
%</package>
%\fi
%
\endinput
|\\
|\childdocby{|\textit{main}|}|\\
\end{tabular}
\end{center}
%
The directive |\childdocby| is similar to |\childdocof|
described in \secref{sec:include},
but the subsequent selection of content must be done manually.
To that end, both |\ifchilddoc| and |\ifchilddocmanual|
will be true upon processing of a part,
and the name of the part is stored in |\childdocname|.
Note that |\jobname| will be set to the filename of the current part
so that each part receives an individual |.aux| file
that does not interfere with the |.aux| file(s) of the main document.
This behaviour can be altered by the alternative form
|\childdocby[*]{|\textit{main}|}| (with a non-empty optional argument)
which uses the |.aux| file of the main document
by setting |\jobname| to \textit{main}.

%%%%%%%%%%%%%%%%%%%%%%%%%%%%%%%%%%%%%%%%%%%%%%%%%%%%%%%%%%%%%%%%%%%%%%%%%%%%%%%%
\subsection{Driver Development}
\label{sec:driver}

The \textsf{childdoc} mechanism can also be use for the development
of definition files such as \LaTeX{} styles or classes.
This case differs from the above setup with multiple parts
included by |\include| in that no |\includeonly| should be invoked.
This can be achieved by starting the include file
(before |\ProvidesPackage|) with:
%
\begin{center}
\begin{tabular}{l}
|% \iffalse
%
% childdoc.dtx Copyright (C) 2017-2018 Niklas Beisert
%
% This work may be distributed and/or modified under the
% conditions of the LaTeX Project Public License, either version 1.3
% of this license or (at your option) any later version.
% The latest version of this license is in
%   http://www.latex-project.org/lppl.txt
% and version 1.3 or later is part of all distributions of LaTeX
% version 2005/12/01 or later.
%
% This work has the LPPL maintenance status `maintained'.
%
% The Current Maintainer of this work is Niklas Beisert.
%
% This work consists of the files childdoc.dtx and childdoc.ins
% and the derived files childdoc.def and cdocsamp.tex with
% cdocsch1.tex, cdocsch2.tex, cdocsdrf.tex, cdocsfn1.tex, cdocsfn2.tex.
%
%<package>\ifdefined\childdocmain\endinput\fi
%<package>\ProvidesFile{childdoc.def}[2018/12/30 v2.0 child document driver]
%<samplemain>\ProvidesFile{cdocsamp.tex}[2018/12/30 v2.0 sample for childdoc]
%<*driver>
%\ProvidesFile{childdoc.drv}[2018/12/30 v2.0 childdoc reference manual file]
\PassOptionsToClass{10pt,a4paper}{article}
\documentclass{ltxdoc}

\usepackage[margin=35mm]{geometry}
\usepackage{hyperref}
\usepackage{hyperxmp}
\usepackage[usenames]{color}

\hypersetup{colorlinks=true}
\hypersetup{pdfstartview=FitH}
\hypersetup{pdfpagemode=UseNone}
\hypersetup{pdfsource={}}
\hypersetup{pdflang={en-UK}}
\hypersetup{pdfcopyright={Copyright 2017-2018 Niklas Beisert.
  This work may be distributed and/or modified under the
  conditions of the LaTeX Project Public License, either version 1.3
  of this license or (at your option) any later version.}}
\hypersetup{pdflicenseurl={http://www.latex-project.org/lppl.txt}}
\hypersetup{pdfcontactaddress={ETH Zurich, ITP, HIT K,
  Wolfgang-Pauli-Strasse 27}}
\hypersetup{pdfcontactpostcode={8093}}
\hypersetup{pdfcontactcity={Zurich}}
\hypersetup{pdfcontactcountry={Switzerland}}
\hypersetup{pdfcontactemail={nbeisert@itp.phys.ethz.ch}}
\hypersetup{pdfcontacturl={http://people.phys.ethz.ch/\xmptilde nbeisert/}}

\newcommand{\secref}[1]{\hyperref[#1]{section \ref*{#1}}}

\parskip1ex
\parindent0pt
\let\olditemize\itemize
\def\itemize{\olditemize\parskip0pt}

\begin{document}

\title{The \textsf{childdoc} Package}
\hypersetup{pdftitle={The childdoc Package}}
\author{Niklas Beisert\\[2ex]
  Institut f\"ur Theoretische Physik\\
  Eidgen\"ossische Technische Hochschule Z\"urich\\
  Wolfgang-Pauli-Strasse 27, 8093 Z\"urich, Switzerland\\[1ex]
  \href{mailto:nbeisert@itp.phys.ethz.ch}
  {\texttt{nbeisert@itp.phys.ethz.ch}}}
\hypersetup{pdfauthor={Niklas Beisert}}
\hypersetup{pdfsubject={Manual for the LaTeX2e Package childdoc}}
\date{30 December 2018, \textsf{v2.0}}
\maketitle

\begin{abstract}\noindent
\textsf{childdoc} is a \LaTeXe{} package
that enables the direct compilation
of document sections included by |\include|
to individual files.
\end{abstract}

\begingroup
\parskip0ex
\tableofcontents
\endgroup

%%%%%%%%%%%%%%%%%%%%%%%%%%%%%%%%%%%%%%%%%%%%%%%%%%%%%%%%%%%%%%%%%%%%%%%%%%%%%%%%
%%%%%%%%%%%%%%%%%%%%%%%%%%%%%%%%%%%%%%%%%%%%%%%%%%%%%%%%%%%%%%%%%%%%%%%%%%%%%%%%
\section{Introduction}

\LaTeX{} provides a mechanism to structure a large document (such as a book)
into a main file and several child files (containing the chapters)
using the |\include| command.
This mechanism is beneficial for documents
which span hundreds of pages in order to
make the source file(s) more manageable.
Moreover, compilation can be restricted to
selected child files by means of the |\includeonly| command.
The latter feature can be used to reduce the compilation time while editing
(this was significantly more useful in the earlier days of \LaTeX{})
or to generate a smaller document which is easier to navigate.
Another application of |\includeonly| is to generate
documents consisting of selected parts of the complete document.

However, there are a few drawbacks of the plain |\include| mechanism:
\begin{itemize}
\item
The child files cannot be compiled on their own,
they can only be compiled via the main file.
A naive editing environment
(such as a text editor with an option
to have the current file processed by \LaTeX)
may require one to switch to the main file before compiling;
attempting to compile the child file produces errors.
\item
The main file must be modified (each time)
to adjust the |\includeonly| command
to the present needs. This easily leaves the main file in a messy state.
\item
The generated document will always carry the filename
of the main document. This is inconvenient if
several child files are to be compiled and
to be kept for distribution.
\end{itemize}

The present package provides a simple interface
to make child files individually compilable by \LaTeX{}.
Compiling a child file then has the same effect as compiling
the main file with an |\includeonly| command
to select the appropriate child.
Moreover the generated document will carry the name of the child
rather than the main file.
This resolves all three above issues.

This feature is meant to make the editing of books,
thesis documents and lecture notes somewhat more convenient.
However, the package can also be used efficiently for
composing a series of documents (such as exercise sheets)
which are typically distributed individually.
It then assists the author in generating the individual documents
(potentially in different versions)
as well as a document containing the collected series.
Another application is in developing style files
or other kinds of included material
where compilation of the style file could redirect
to a sample or test file.

%%%%%%%%%%%%%%%%%%%%%%%%%%%%%%%%%%%%%%%%%%%%%%%%%%%%%%%%%%%%%%%%%%%%%%%%%%%%%%%%
%%%%%%%%%%%%%%%%%%%%%%%%%%%%%%%%%%%%%%%%%%%%%%%%%%%%%%%%%%%%%%%%%%%%%%%%%%%%%%%%
\section{Usage}

First of all, the package \textsf{childdoc} is \emph{not} a standard
\LaTeXe{} |.sty| style file! Therefore it needs to be invoked in
a non-standard way.

%%%%%%%%%%%%%%%%%%%%%%%%%%%%%%%%%%%%%%%%%%%%%%%%%%%%%%%%%%%%%%%%%%%%%%%%%%%%%%%%
\subsection{Included Files}
\label{sec:include}

%%%%%%%%%%%%%%%%%%%%%%%%%%%%%%%%%%%%%%%%
\DescribeMacro{\childdocmain}
To use the package, add the commands
\begin{center}
\begin{tabular}{l}
|\input{childdoc.def}|\\
|\childdocmain{}|\\
\end{tabular}
\end{center}
at the very top of the main \LaTeX{} file,
in particular \emph{before} the |\documentclass| statement!
The argument of |\childdocmain| should be left empty
(but it must be present).

%%%%%%%%%%%%%%%%%%%%%%%%%%%%%%%%%%%%%%%%
\DescribeMacro{\childdocof}
Furthermore, add the commands
\begin{center}
\begin{tabular}{l}
|\input{childdoc.def}|\\
|\childdocof{|\textit{main}|}|\\
\end{tabular}
\end{center}
at the top of every child file \textit{child}
which is included by |\include{|\textit{child}|}|
from within the main file
(or at least for those files to be compiled individually).
The argument \textit{main} must be the filename of the main file.

There are a couple of
considerations in setting up the main and child documents:

%%%%%%%%%%%%%%%%%%%%%%%%%%%%%%%%%%%%%%%%
\paragraph{Restrictions.}

Please note the following restrictions:
\begin{itemize}
\item
|\childdocmain| must be called with one argument \textit{main}
to ensure compatibility with earlier version of the package.
It must either be empty (|\childdocmain{}|)
or precisely match the filename of the main file in which it is specified.
See \secref{sec:detection} for further information.
\item
The filename \textit{main} must be specified without the |.tex| extension.
\item
The filename \textit{main} is case sensitive
(even in case-insensitive file systems)
due to internal string comparison.
\item
The argument \textit{main} should be fully expanded, it cannot be a macro.
\item
Subdirectories and special characters should be avoided in filenames.
\item
The command |\childdocmain{|\textit{main}|}| must be followed by a whitespace.
It should not be followed immediately by another command
or by a comment mark `|%|'.
This is because the \TeX{} parser reads the token immediately following
the argument of |\childdocmain| and puts it
at the beginning of every child section;
however, a white\-space is ignored.
\end{itemize}

%%%%%%%%%%%%%%%%%%%%%%%%%%%%%%%%%%%%%%%%
\paragraph{Content of Main File.}

It is advisable to place all content in the child files included by |\include|.
Any output contained in the main file will appear in all child documents
unless suppressed manually;
it cannot be suppressed automatically by the |\includeonly| directive
and thus should normally be avoided.
A method to include some content in the main file
by means of conditional processing is described in \secref{sec:conditional}.

%%%%%%%%%%%%%%%%%%%%%%%%%%%%%%%%%%%%%%%%
\paragraph{Page Numbering.}

When only a part of the document is compiled,
the appropriate numbering of pages
(as well as other status parameters)
is determined from the |.aux| files.
The latter contain information from previous passes.
However this information needs to propagate through
all intermediate child documents.
Therefore the page numbering in child documents may well
be inconsistent until the complete document is compiled at least once.

A useful (if unconventional) way to always ensure a consistent
page numbering is to restart the numbering in each child document
and denote the pages by `\textit{child}|.|\textit{page}'
where \textit{child} represents the chapter/section number of the child file.
This can be achieved by the command
|\numberwithin{page}{|\textit{child}|}|
of the \textsf{amsmath} package
where \textit{child} can be |chapter| or |section|
depending on the chosen structuring.
Alternatively, one can modify the macro |\thepage| appropriately
and reset the counter |page| at the start of each child file.

%%%%%%%%%%%%%%%%%%%%%%%%%%%%%%%%%%%%%%%%%%%%%%%%%%%%%%%%%%%%%%%%%%%%%%%%%%%%%%%%
\subsection{Conditional Processing}
\label{sec:conditional}

The package provides a mechanism to compile different versions
of a document. To customise the versions further some conditional processing
can come in handy to distinguish which version is being compiled.
The package provides two macros to describe the compilation context:

%%%%%%%%%%%%%%%%%%%%%%%%%%%%%%%%%%%%%%%%
\DescribeMacro{\ifchilddoc}
The conditional |\ifchilddoc| distinguishes between the compilation of
child documents and the main document:
%
\begin{center}
|\ifchilddoc |\textit{child-code}| |[|\||else |\textit{main-code}]| \||fi|
\end{center}

%%%%%%%%%%%%%%%%%%%%%%%%%%%%%%%%%%%%%%%%
\DescribeMacro{\childdocname}
\DescribeMacro{\childdocjob}
The macro |\childdocname| contains the filename (without extension)
of the main or child file being processed.
Note that |\childdocjob| will always contain the name of the main file.

%%%%%%%%%%%%%%%%%%%%%%%%%%%%%%%%%%%%%%%%
\paragraph{Title Page.}

Conditional processing can be used to include a title or banner page
in the main document when proper precautions are taken.
Importantly, the code in the main file should ensure that the page counter
(as well as other status parameters which are stored in the |.aux| files)
takes the same value after the conditional processing.
Otherwise the page numbers may take divergent values
depending on which part is compiled.

For example, a title page could be declared by:
%
\begin{center}
\begin{tabular}{l}
|\ifchilddoc\||else|\\
|\addtocounter{page}{-1}|\\
\textit{code for title page}\\
|\newpage|\\
|\||fi|
\end{tabular}
\end{center}
%
A banner page for the child documents can be generated by:
%
\begin{center}
\begin{tabular}{l}
|\ifchilddoc|\\
|\addtocounter{page}{-1}|\\
\textit{code for banner page}\\
|\newpage|\\
|\||fi|
\end{tabular}
\end{center}
%
Here one could write a message such as:
\begin{center}
|This is the part \childdocname{} of \childdocjob{}.|
\end{center}

%%%%%%%%%%%%%%%%%%%%%%%%%%%%%%%%%%%%%%%%%%%%%%%%%%%%%%%%%%%%%%%%%%%%%%%%%%%%%%%%
\subsection{Flags}
\label{sec:flags}

The package makes it easy to generate different versions
of the main or child documents.
To this end compilation flags can be defined
and assigned different default values.
They will be particularly useful in conjunction
with the forwarding mechanism described in \secref{sec:forward}.

For example, it may be useful to have a flag |\version|
which can be set to |draft| or |final|.
The document source will contain some conditional code
depending on the value of |\version|.
Suppose further, the flag should default to |final| for the main file
and to |draft| for child files
which is a natural assignment for editing the document.
This is achieved by placing the following code
in the preamble of the main document
(below the |\childdocmain| directive):
%
\begin{center}
\begin{tabular}{l}
|\ifchilddoc|\\
|\providecommand{\version}{draft}|\\
|\||else|\\
|\providecommand{\version}{final}|\\
|\||fi|
\end{tabular}
\end{center}
%
The definition by |\providecommand| makes sure
that previous definitions are not overwritten.
Further statements |\providecommand{\version}{...}|
can thus be added before the above code to override it.

For the main file, one might add a line
(between |\childdocmain| and the above block)
%
\begin{center}
|%\ifchilddoc\||else\providecommand{\version}{draft}\||fi|
\end{center}
%
which can be uncommented to produce a draft version.
Likewise one can add a line to the very top of a child file
(above the |\childdocof{|\textit{main}|}| directive)
%
\begin{center}
|%\providecommand{\version}{final}|
\end{center}
%
which can be uncommented to produce the final version of this child document.

%%%%%%%%%%%%%%%%%%%%%%%%%%%%%%%%%%%%%%%%%%%%%%%%%%%%%%%%%%%%%%%%%%%%%%%%%%%%%%%%
\subsection{Forwarding}
\label{sec:forward}

Different versions of the main or child documents
using compilation flags as described in \secref{sec:flags}
can be (permanently) stored in different files
for convenient compilation, viewing and distribution.
To this end, the package defines a command
to pass on compilation to a different file:

%%%%%%%%%%%%%%%%%%%%%%%%%%%%%%%%%%%%%%%%
\DescribeMacro{\childdocforward}
The command |\childdocforward| redirects processing to
another source file:
%
\begin{center}
\begin{tabular}{l}
|\input{childdoc.def}|\\
|\childdocforward[|\textit{main}|]{|\textit{dest}|}|\\
\end{tabular}
\end{center}
%
The argument \textit{dest} is the destination file
(without extension).
It should be the main file or one of the child files.
Note that further \textsf{childdoc} directives
such as |\childdocof| and |\childdocforward|
in the indicated file will be processed in this form.
The optional argument \textit{main}
passes on directly to the main file \textit{main}
while pretending to compile the child \textit{dest}.
This form behaves as if \textit{dest}
issues |\childdocof{|\textit{main}|}| right away,
and no further \textsf{childdoc} directives will be processed.

%%%%%%%%%%%%%%%%%%%%%%%%%%%%%%%%%%%%%%%%
\DescribeMacro{\...prefix}
In the alternative form |\childdocforwardprefix|,
%
\begin{center}
\begin{tabular}{l}
|\input{childdoc.def}|\\
|\childdocforwardprefix[|\textit{main}|]{|\textit{prefix}|}{|\textit{dest}|}|
\end{tabular}
\end{center}
%
the destination file is determined by a pattern
depending on the current file:
To make this work, the current file must be called
`{\textit{prefix}\hspace{0.2em}\textit{suffix}}'
with \textit{prefix} matching precisely the argument.
Processing is then passed on to the file
`{\textit{dest}\hspace{0.2em}\textit{suffix}}'.
Surely, the same effect is achieved by
directly specifying the
argument `{\textit{dest}\hspace{0.2em}\textit{suffix}}'
in the first form.
However, that requires to set up a different file
for each child. With the alternative form of the command
all these files can have exactly the same content
which simplifies setting them up and maintaining them.

For example, the following file |draft.tex|
with a compilation flag |\version| as described in \secref{sec:flags}
compiles the main document as a draft:
%
\begin{center}
\begin{tabular}{l}
|\def\version{draft}|\\
|\input{childdoc.def}|\\
|\childdocforward{|\textit{main}|}|
\end{tabular}
\end{center}
%
Likewise, the following files |final|\textit{nn}|.tex|
compile the final version of the child document
|child|\textit{nn}|.tex|:
%
\begin{center}
\begin{tabular}{l}
|\def\version{final}|\\
|\input{childdoc.def}|\\
|\childdocforwardprefix{final}{child}|
\end{tabular}
\end{center}
%

Note that when several versions of a main file and/or of each child file
are to be generated, it may be convenient to set up a |Makefile| or
shell script to automatise the process.

%%%%%%%%%%%%%%%%%%%%%%%%%%%%%%%%%%%%%%%%%%%%%%%%%%%%%%%%%%%%%%%%%%%%%%%%%%%%%%%%
\subsection{Command Line Processing}
\label{sec:commandline}

The effect of redirection files can also be achieved by invoking
the \LaTeX{} compiler with a more elaborate command line.
Most conveniently this should be done as part
of a shell script or a |Makefile|.

When using \textsf{childdoc} in the main file, the following
command lines effectively perform a redirection
(note that depending on the shell being used,
backslashes may have to be doubled: `|\|' $\to$ `|\\|'):
%
\begin{center}
|... -jobname "|\textit{target}|" |\\|"|[\textit{flags}]%
|\input{childdoc.def}\childdocforward[|\textit{main}|]{|\textit{dest}|}"|
\end{center}
%
Here \textit{target} is the name of the output file,
\textit{main} is the name of the main file
and \textit{dest} is the name of the main or child file to be processed
(all filenames without extensions).
The optional argument \textit{main} can be omitted
if \textit{main} matches \textit{dest}.
Optionally, compilation \textit{flags} can be defined via |\def| commands.
This command line makes the \TeX{} engine believe
it is compiling the file \textit{target}
whose content is specified as the latter parameter.
The provided code then forwards the processing to
\textit{main} or \textit{dest} as described in \secref{sec:forward}.

%%%%%%%%%%%%%%%%%%%%%%%%%%%%%%%%%%%%%%%%%%%%%%%%%%%%%%%%%%%%%%%%%%%%%%%%%%%%%%%%
\subsection{Include by Input}
\label{sec:input}

Including child documents by |\include| has some restrictions by design.
Most notably, the content of a child document always occupies
its own set of pages; pages cannot be shared between child documents.
Usually, this behaviour makes perfect sense
because each child document contain an essential part of the document.
However, in some situations it may be desirable to compose
a document from a collection of parts
without having mandatory page breaks between then.
For this case, the package
provides a mechanism to include parts
by |\input| which can also be processed individually.
However, by construction this mechanism
requires manual handling of the content to be output.

%%%%%%%%%%%%%%%%%%%%%%%%%%%%%%%%%%%%%%%%
\DescribeMacro{\ifchilddocmanual}
The main file should be prepared as usual, see \secref{sec:include}.
However, the document body must make a distinction
between processing of an individual part and of the main document, e.g.:
%
\begin{center}
\begin{tabular}{l}
|\ifchilddocmanual|\\
|\input{\childdocname}|\\
|\||else|\\
\textit{document body with }|\input{|\textit{part}|}|\\
|\||fi|
\end{tabular}
\end{center}
%
The conditional |\ifchilddocmanual| is true whenever
a part to be included by |\input| is being compiled,
and the name of the part is stored in |\childdocname|.

%%%%%%%%%%%%%%%%%%%%%%%%%%%%%%%%%%%%%%%%
\DescribeMacro{\childdocby}
Each part to be included by |\input| should start with:
%
\begin{center}
\begin{tabular}{l}
|\input{childdoc.def}|\\
|\childdocby{|\textit{main}|}|\\
\end{tabular}
\end{center}
%
The directive |\childdocby| is similar to |\childdocof|
described in \secref{sec:include},
but the subsequent selection of content must be done manually.
To that end, both |\ifchilddoc| and |\ifchilddocmanual|
will be true upon processing of a part,
and the name of the part is stored in |\childdocname|.
Note that |\jobname| will be set to the filename of the current part
so that each part receives an individual |.aux| file
that does not interfere with the |.aux| file(s) of the main document.
This behaviour can be altered by the alternative form
|\childdocby[*]{|\textit{main}|}| (with a non-empty optional argument)
which uses the |.aux| file of the main document
by setting |\jobname| to \textit{main}.

%%%%%%%%%%%%%%%%%%%%%%%%%%%%%%%%%%%%%%%%%%%%%%%%%%%%%%%%%%%%%%%%%%%%%%%%%%%%%%%%
\subsection{Driver Development}
\label{sec:driver}

The \textsf{childdoc} mechanism can also be use for the development
of definition files such as \LaTeX{} styles or classes.
This case differs from the above setup with multiple parts
included by |\include| in that no |\includeonly| should be invoked.
This can be achieved by starting the include file
(before |\ProvidesPackage|) with:
%
\begin{center}
\begin{tabular}{l}
|\input{childdoc.def}|\\
|\childdocforward{|\textit{main}|}|\\
\end{tabular}
\end{center}
%
or alternatively with:
%
\begin{center}
\begin{tabular}{l}
|\input{childdoc.def}|\\
|\childdocby{|\textit{main}|}|\\
\end{tabular}
\end{center}
%
Both forms have slightly different effects as described above.
The main file is prepared as usual, see \secref{sec:include}.

%%%%%%%%%%%%%%%%%%%%%%%%%%%%%%%%%%%%%%%%%%%%%%%%%%%%%%%%%%%%%%%%%%%%%%%%%%%%%%%%
\subsection{Legacy Detection}
\label{sec:detection}

The directive |\childdocmain| in the main file can detect
whether the complete document or merely a child is to be compiled
even without using the directive |\childdocof|.
This method is deprecated because it is less robust
and there is no compelling reason to use it;
it is merely provided for backward compatibility
and it may be removed in future versions.

If the detection mechanism is to be used,
it is mandatory to correctly specify
the filename of the main file as the argument of |\childdocmain|:
%
\begin{center}
\begin{tabular}{l}
|\input{childdoc.def}|\\
|\childdocmain{|\textit{main}|}|\\
\end{tabular}
\end{center}
%
If |\jobname| does not match the argument \textit{main} of |\childdocmain|,
it is assumed that |\jobname| points to the child file to be compiled.
When using |\childdocmain| with the main file specified as argument,
it suffices to start a child file
with just |\input{|\textit{main}|}|
without loading of the package and using |\childdocof|.
If instead all processing is done
with the appropriate \textsf{childdoc} directives,
the argument of \textit{main} of |\childdocmain| can be empty.

An alternative version of the command line processing described
in \secref{sec:commandline} using the detection mechanism reads:
%
\begin{center}
|... -jobname "|\textit{target}|" "|[\textit{flags}]%
[|\def\jobname{|\textit{dest}|}|]|\input{|\textit{main}|}"|
\end{center}

%%%%%%%%%%%%%%%%%%%%%%%%%%%%%%%%%%%%%%%%%%%%%%%%%%%%%%%%%%%%%%%%%%%%%%%%%%%%%%%%
\subsection{Manual Code}
\label{sec:manual}

In case one cannot be certain whether the definitions file |childdoc.def|
is installed on the target \TeX{} distribution
and one prefers not to ship it,
it is conceivable to paste a few relevant commands into the sources.

To that end, drop all statements |\input{childdoc.def}|
and perform the replacements as outlined below.
Instead of |\childdocmain{|\textit{main}|}| add the following code
to the top of the main file:
%
\begin{center}
\begin{tabular}{l}
|\||ifdefined\childdocname\endinput\||fi\newif\ifchilddoc|\\
|\edef\childdocname{\scantokens\expandafter{\jobname\noexpand}}|\\
|\def\childdocmain{|\textit{main}|}\||ifx\childdocmain\childdocname\||else|\\
|\childdoctrue\includeonly{\childdocname}\let\jobname\childdocmain\||fi|\\
\end{tabular}
\end{center}
%
Instead of |\childdocof{|\textit{main}|}| just include the main file
at the top of each child file:
%
\begin{center}
|\input{|\textit{main}|}|
\end{center}
%
A simple redirection |\childdocforward{|\textit{dest}|}| is achieved by:
%
\begin{center}
|\def\jobname{|\textit{dest}|}\input{\jobname}|
\end{center}
%
The redirection with prefix
|\childdocforwardprefix[|\textit{prefix}|]{|\textit{dest}|}|
is accomplished by:
%
\begin{center}
\begin{tabular}{l}
|{\edef\jobname{\scantokens\expandafter{\jobname\noexpand}}|\\
|\def\redirectjob |\textit{prefix}|#1~~~{\gdef\jobname{|\textit{dest}|#1}}|\\
|\expandafter\redirectjob\jobname~~~}\input{\jobname}|
\end{tabular}
\end{center}

In an alternative approach,
child documents can be compiled by a specific command line
without additional code or specific definitions:
%
\begin{center}
|... -jobname "|\textit{target}|" "|[\textit{flags}]%
|\includeonly{|\textit{dest}|}\input{|\textit{main}|}"|
\end{center}
%

%%%%%%%%%%%%%%%%%%%%%%%%%%%%%%%%%%%%%%%%%%%%%%%%%%%%%%%%%%%%%%%%%%%%%%%%%%%%%%%%
%%%%%%%%%%%%%%%%%%%%%%%%%%%%%%%%%%%%%%%%%%%%%%%%%%%%%%%%%%%%%%%%%%%%%%%%%%%%%%%%
\section{Information}

%%%%%%%%%%%%%%%%%%%%%%%%%%%%%%%%%%%%%%%%%%%%%%%%%%%%%%%%%%%%%%%%%%%%%%%%%%%%%%%%
\subsection{Copyright}

Copyright \copyright{} 2017--2018 Niklas Beisert

This work may be distributed and/or modified under the
conditions of the \LaTeX{} Project Public License, either version 1.3
of this license or (at your option) any later version.
The latest version of this license is in
  \url{http://www.latex-project.org/lppl.txt}
and version 1.3 or later is part of all distributions of \LaTeX{}
version 2005/12/01 or later.

This work has the LPPL maintenance status `maintained'.

The Current Maintainer of this work is Niklas Beisert.

This work consists of the files |README.txt|, |childdoc.ins| and |childdoc.dtx|
as well as the derived files |childdoc.def|, |cdocsamp.tex|
with |cdocsch1.tex|, |cdocsch2.tex|, |cdocspt3.tex|, |cdocspt4.tex|,
|cdocsdrf.tex|, |cdocsfn1.tex|, |cdocsfn2.tex|
as well as |childdoc.pdf|.

%%%%%%%%%%%%%%%%%%%%%%%%%%%%%%%%%%%%%%%%%%%%%%%%%%%%%%%%%%%%%%%%%%%%%%%%%%%%%%%%
\subsection{Files and Installation}

The package consists of the files:
%
\begin{center}
\begin{tabular}{ll}
    |README.txt|   & readme file \\
    |childdoc.ins| & installation file \\
    |childdoc.dtx| & source file \\
    |childdoc.def| & definition file \\
    |cdocsamp.tex| & sample main file \\
    |cdocsch1.tex| & sample include file \\
    |cdocsch2.tex| & sample include file \\
    |cdocspt3.tex| & sample part file \\
    |cdocspt4.tex| & sample part file \\
    |cdocsdrf.tex| & sample redirection file \\
    |cdocsfn1.tex| & sample redirection file \\
    |cdocsfn2.tex| & sample redirection file \\
    |childdoc.pdf| & manual
\end{tabular}
\end{center}
%
The distribution consists of the files
|README.txt|, |childdoc.ins| and |childdoc.dtx|.
%
\begin{itemize}
\item
Run (pdf)\LaTeX{} on |childdoc.dtx|
to compile the manual |childdoc.pdf| (this file).
\item
Run \LaTeX{} on |childdoc.ins| to create the definitions file |childdoc.def|
and the sample |cdocsamp.tex| with include files
|cdocsch1.tex|, |cdocsch2.tex|, |cdocspt3.tex|, |cdocspt4.tex|,
|cdocsdrf.tex|, |cdocsfn1.tex|, |cdocsfn2.tex|.
Then copy the file |childdoc.def| to an appropriate directory of your \LaTeX{}
distribution, e.g.\ \textit{texmf-root}|/tex/latex/childdoc|.
\end{itemize}

%%%%%%%%%%%%%%%%%%%%%%%%%%%%%%%%%%%%%%%%%%%%%%%%%%%%%%%%%%%%%%%%%%%%%%%%%%%%%%%%
\subsection{Related CTAN Packages}

There are several other packages which offer a similar functionality:
%
\begin{itemize}
\item
The packages
\href{http://ctan.org/pkg/docmute}{\textsf{docmute}},
\href{http://ctan.org/pkg/includex}{\textsf{includex}} and
\href{http://ctan.org/pkg/standalone}{\textsf{standalone}}
provide commands to include only the document body of
a child file thus allowing both files to be compiled individually.
\item
The packages \href{http://ctan.org/pkg/subdocs}{\textsf{subdocs}}
and \href{http://ctan.org/pkg/subfiles}{\textsf{subfiles}}
provide structures in which the main and child documents can be
encapsulated and allowing them to be compiled individually.
The inclusion mechanism is different from the conventional |\include|.
\item
The package \href{http://ctan.org/pkg/combine}{\textsf{combine}}
is an elaborate solution to combine several documents into one.
\end{itemize}
%
See also the CTAN topic \href{http://ctan.org/topic/subdocs}{\textsf{subdocs}}
for further related packages.
The present package differs from the above solutions in that
a document structure constructed with the conventional |\include| mechanism
just needs two extra commands at the top of every file
such that all constituent files can be compiled individually.

%%%%%%%%%%%%%%%%%%%%%%%%%%%%%%%%%%%%%%%%%%%%%%%%%%%%%%%%%%%%%%%%%%%%%%%%%%%%%%%%
%\subsection{Feature Suggestions}
%
%The following is a list of features which may be useful for future
%versions of this package:
%%
%\begin{itemize}
%\item
%\ldots
%\end{itemize}

%%%%%%%%%%%%%%%%%%%%%%%%%%%%%%%%%%%%%%%%%%%%%%%%%%%%%%%%%%%%%%%%%%%%%%%%%%%%%%%%
\subsection{Revision History}

%%%%%%%%%%%%%%%%%%%%%%%%%%%%%%%%%%%%%%%%
\paragraph{v2.0:} 2018/12/30

\begin{itemize}
\item
immediate forward processing
\item
added |\childdocby| mechanism
\item
manual restructured
\end{itemize}

%%%%%%%%%%%%%%%%%%%%%%%%%%%%%%%%%%%%%%%%
\paragraph{v1.6:} 2018/01/17

\begin{itemize}
\item
application for development of include files
\item
corrections to manual
\end{itemize}

%%%%%%%%%%%%%%%%%%%%%%%%%%%%%%%%%%%%%%%%
\paragraph{v1.5:} 2017/05/21

\begin{itemize}
\item
more complete structuring introduced
\item
|\childdocof| introduced
\item
|\childdoc| renamed to |\childdocmain|
\item
|\childredirect| renamed to |\childdocforward| and |\childdocforwardprefix|
and functionality expanded
\end{itemize}

%%%%%%%%%%%%%%%%%%%%%%%%%%%%%%%%%%%%%%%%
\paragraph{v1.0:} 2017/04/27

\begin{itemize}
\item
manual and install package
\item
first version published on CTAN
\end{itemize}

%%%%%%%%%%%%%%%%%%%%%%%%%%%%%%%%%%%%%%%%
\paragraph{v0.6:} 2017/04/26

\begin{itemize}
\item
redirection mechanism added
\end{itemize}

%%%%%%%%%%%%%%%%%%%%%%%%%%%%%%%%%%%%%%%%
\paragraph{v0.5:} 2017/04/26

\begin{itemize}
\item
functionality in definition file
\end{itemize}


%%%%%%%%%%%%%%%%%%%%%%%%%%%%%%%%%%%%%%%%%%%%%%%%%%%%%%%%%%%%%%%%%%%%%%%%%%%%%%%%
%%%%%%%%%%%%%%%%%%%%%%%%%%%%%%%%%%%%%%%%%%%%%%%%%%%%%%%%%%%%%%%%%%%%%%%%%%%%%%%%
%%%%%%%%%%%%%%%%%%%%%%%%%%%%%%%%%%%%%%%%%%%%%%%%%%%%%%%%%%%%%%%%%%%%%%%%%%%%%%%%
\appendix

\settowidth\MacroIndent{\rmfamily\scriptsize 000\ }

 \DocInput{childdoc.dtx}

\end{document}
%</driver>
% \fi
%
% %%%%%%%%%%%%%%%%%%%%%%%%%%%%%%%%%%%%%%%%%%%%%%%%%%%%%%%%%%%%%%%%%%%%%%%%%%%%%%
% %%%%%%%%%%%%%%%%%%%%%%%%%%%%%%%%%%%%%%%%%%%%%%%%%%%%%%%%%%%%%%%%%%%%%%%%%%%%%%
% \section{Sample}
%\iffalse
%<*samplemain>
%\fi
%
% The following presents a sample document
% with two chapters, two parts, a title page,
% a compile flag as well as three forwarding files to set the flag.
% It consists of eight |.tex| files:
% \begin{center}
% \begin{tabular}{ll}
% |cdocsamp.tex|&main file\\
% |cdocsch1.tex|&include file for chapter 1\\
% |cdocsch2.tex|&include file for chapter 2\\
% |cdocspt3.tex|&include file for part 3\\
% |cdocspt4.tex|&include file for part 4\\
% |cdocsdrf.tex|&forwarding file for main file in draft mode\\
% |cdocsfi1.tex|&forwarding file for final version of chapter 1\\
% |cdocsfi2.tex|&forwarding file for final version of chapter 2\\
% \end{tabular}
% \end{center}
% Each of the eight files can be compiled directly by the \LaTeX{} compiler.
%
% %%%%%%%%%%%%%%%%%%%%%%%%%%%%%%%%%%%%%%
% \paragraph{Main File.}
%
% The main file is called |cdocsamp.tex|.
%
% Load the \textsf{childdoc} definitions and
% declare the filename for the main document:
%    \begin{macrocode}
\input{childdoc.def}
\childdocmain{}
%    \end{macrocode}

% Optional override for |\version| flag:
%    \begin{macrocode}
%%\ifchilddoc\else\providecommand{\version}{draft}\fi
%    \end{macrocode}

% Define the default values for the |\version| flag
% (|final| for the main file and |draft| for childs):
%    \begin{macrocode}
\ifchilddoc
\providecommand{\version}{draft}
\else
\providecommand{\version}{final}
\fi
%    \end{macrocode}

% Load the standard document class:
%    \begin{macrocode}
\documentclass[12pt]{article}
%    \end{macrocode}

% Start the document body:
%    \begin{macrocode}
\begin{document}
%    \end{macrocode}

% Declare a title page.
% Print title, part of document being processed and version flag:
%    \begin{macrocode}
\addtocounter{page}{-1}
\begin{center}
{\LARGE\bfseries{}childdoc example\par}
\vspace{1cm}
\ifchilddoc
\ifchilddocmanual part\else chapter\fi:
`\childdocname' of `\childdocjob'\par
\else
main document: `\childdocjob'\par
\fi
version: \version\par
\end{center}
\newpage
%    \end{macrocode}

% Manually include selected file,
% otherwise process as usual:
%    \begin{macrocode}
\ifchilddocmanual
\section*{part `\childdocname'}
\input{\childdocname}
\else
%    \end{macrocode}

% Include the two chapters:
%    \begin{macrocode}
\include{cdocsch1}
\include{cdocsch2}
%    \end{macrocode}

% Include the two parts unless only chapters should be displayed:
%    \begin{macrocode}
\ifchilddoc\else
\section{part three}
\input{cdocspt3}
\section{part four}
\input{cdocspt4}
\fi
%    \end{macrocode}

% Process as usual until here:
%    \begin{macrocode}
\fi
%    \end{macrocode}

% End of document body:
%    \begin{macrocode}
\end{document}
%    \end{macrocode}
%\iffalse
%</samplemain>
%\fi
%
% %%%%%%%%%%%%%%%%%%%%%%%%%%%%%%%%%%%%%%
% \paragraph{Chapter Include Files.}
%
% The include files are called |cdocsch1.tex| and |cdocsch2.tex|.
%
%\iffalse
%<*samplechap1|samplechap2>
%\fi

% Optional override for |\version| flag:
%    \begin{macrocode}
%%\providecommand{\version}{final}
%    \end{macrocode}

% Include the main document:
%    \begin{macrocode}
\input{childdoc.def}
\childdocof{cdocsamp}
%    \end{macrocode}

%\iffalse
%</samplechap1|samplechap2>
%\fi
%
%\iffalse
%<*samplechap1>
%\fi
% Some text for chapter 1:
%    \begin{macrocode}
\section{one}
some text in chapter one
%    \end{macrocode}

%\iffalse
%</samplechap1>
%\fi
% Some text for chapter 2:
%\iffalse
%<*samplechap2>
%\fi
%    \begin{macrocode}
\section{two}
more text in chapter two
%    \end{macrocode}

%\iffalse
%</samplechap2>
%\fi
%
% %%%%%%%%%%%%%%%%%%%%%%%%%%%%%%%%%%%%%%
% \paragraph{Part Include Files.}
%
% The include files are called |cdocspt3.tex| and |cdocspt4.tex|.
%
%\iffalse
%<*samplepart3|samplepart4>
%\fi

% Optional override for |\version| flag:
%    \begin{macrocode}
%%\providecommand{\version}{final}
%    \end{macrocode}

% Include the main document:
%    \begin{macrocode}
\input{childdoc.def}
\childdocby{cdocsamp}
%    \end{macrocode}

%\iffalse
%</samplepart3|samplepart4>
%\fi
%
%\iffalse
%<*samplepart3>
%\fi
% Some text for part 3:
%    \begin{macrocode}
some text in part three
%    \end{macrocode}

%\iffalse
%</samplepart3>
%\fi
% Some text for part 4:
%\iffalse
%<*samplepart4>
%\fi
%    \begin{macrocode}
more text in part four
%    \end{macrocode}

%\iffalse
%</samplepart4>
%\fi
%
% %%%%%%%%%%%%%%%%%%%%%%%%%%%%%%%%%%%%%%
% \paragraph{Forwarding for a Complete Draft.}
%
% The following forwarding file |cdocsdrf.tex|
% compiles the main document in draft mode:
%\iffalse
%<*sampledraft>
%\fi
%    \begin{macrocode}
\def\version{draft}
\input{childdoc.def}
\childdocforward{cdocsamp}
%    \end{macrocode}

%\iffalse
%</sampledraft>
%\fi
%
% %%%%%%%%%%%%%%%%%%%%%%%%%%%%%%%%%%%%%%
% \paragraph{Forwarding for Final Version of the Chapters.}
%
% The following forwarding files |cdocsfn1.tex| and |cdocsfn2.tex|
% (with identical content)
% compile the final versions of the child documents
% |cdocsch1.tex| and |cdocsch2.tex|, respectively:
%\iffalse
%<*samplefinal>
%\fi
%    \begin{macrocode}
\def\version{final}
\input{childdoc.def}
\childdocforwardprefix[cdocsamp]{cdocsfn}{cdocsch}
%    \end{macrocode}

%\iffalse
%</samplefinal>
%\fi
%
% %%%%%%%%%%%%%%%%%%%%%%%%%%%%%%%%%%%%%%
% \paragraph{Command Line Processing.}
%
% The following three command lines generate the output files
% |cdocscld|, |cdocscl1| and |cdocscl2|
% which should be identical to
% |cdocsdrf|, |cdocsch1| and |cdocsfn2|, respectively:
% \begin{center}
% \begin{tabular}{l}
% |latex -jobname cdocscld \|\\
% |  "\def\version{draft}\input{childdoc.def}\childdocforward{cdocsamp}"|\\
% |latex -jobname cdocscl1 \|\\
% |  "\input{childdoc.def}\childdocforward[cdocsamp]{cdocsch1}"|\\
% |latex -jobname cdocscl2 \|\\
% |  "\def\version{final}\input{childdoc.def}\childdocforward{cdocsch2}"|
% \end{tabular}
% \end{center}
% Note that the trailing backslash on each first line
% merely continues the input to the second line
% (for convenient cut ant paste).
% Furthermore, the command |latex| can be replaced by any
% of its alternative versions such as |pdflatex|.
%
% %%%%%%%%%%%%%%%%%%%%%%%%%%%%%%%%%%%%%%%%%%%%%%%%%%%%%%%%%%%%%%%%%%%%%%%%%%%%%%
% %%%%%%%%%%%%%%%%%%%%%%%%%%%%%%%%%%%%%%%%%%%%%%%%%%%%%%%%%%%%%%%%%%%%%%%%%%%%%%
% \section{Implementation}
%\iffalse
%<*package>
%\fi
%
% This section describes the definitions file |childdoc.def|.

% The definitions cannot be loaded using |\usepackage| or |\RequirePackage|
% which has a mechanism to prevent loading a style file more than once.
% When loading the definitions by means of |\input|
% multiple instances have to be prevented manually:
%\iffalse
%This code needs to be before the `\ProvidesFile' directive
%which is defined at the beginning of this file.
%Therefore it is also placed there and commented out here.
%</package>
%<*discard>
%\fi
%    \begin{macrocode}
\ifdefined\childdocmain\endinput\fi
%    \end{macrocode}
%\iffalse
%</discard>
%<*package>
%\fi
%
% \macro{\ifchilddoc}
% \macro{\ifchilddocmanual}
% The conditional |\ifchilddoc| tells whether a
% child (true) or main (false) document is being compiled.
% The conditional |\ifchilddocmanual| tells whether
% the |\includeonly| mechanism is used (false) or
% the selection of child files must be performed manually (true).
% The definitions initialise to false:
%    \begin{macrocode}
\newif\ifchilddoc
\newif\ifchilddocmanual
%    \end{macrocode}

% \macro{\childdocname}
% \macro{\childdocjob}
% The macro |\childdocname| stores the name of the main document
% to be compiled. The macro |\childdocjob| stores the name of
% the document on which the \LaTeX{} compiler was originally invoked.
% The content of |\jobname| cannot be compared
% to filenames specified in the source due to different catcodes.
% The following code rescans |\jobname|, stores the result
% in |\childdocname| and saves a copy in |\childdocjob|:
%    \begin{macrocode}
\edef\childdocname{\scantokens\expandafter{\jobname\noexpand}}
\let\childdocjob\childdocname
%    \end{macrocode}

% \macro{\childdocdisable}
% The macro |\childdocdisable| prevents the main file
% from being processed more than once.
% At this stage, the main document command |\childdocmain|
% is assumed to be called once again where it should do nothing.
% Any subsequent call to it should prevent
% a secondary processing of the main document
% It overwrites the forwarding commands
% |\childdocof| and |\childdocforward|
% with empty macros to prevent further inclusions of the main document:
%    \begin{macrocode}
\newcommand{\childdocdisable}
{
  \renewcommand{\childdocmain}[1]{\renewcommand{\childdocmain}[1]{\endinput}}
  \renewcommand{\childdocof}[1]{}
  \renewcommand{\childdocby}[2][]{}
  \renewcommand{\childdocforward}[2][]{}
  \renewcommand{\childdocdisable}{}
}
%    \end{macrocode}

% \macro{\childdocmain}
% The macro |\childdocmain| is to be called at the top of the main file
% with nothing or the main filename (without extension) as argument.
% First, it breaks loops.
% If the argument is not empty and does not match |\childdocname|
% (which is set by the first inclusion of |childdoc.def|),
% |\ifchilddoc| is set to true, |\includeonly| is applied to the child file
% and |\jobname| is set to the main file
% (for proper handling of |.aux| files):
%    \begin{macrocode}
\newcommand{\childdocmain}[1]
{
  \childdocdisable\childdocmain{}
  \if?#1?\else
    \begingroup
      \def\childdoctmp{#1}
      \ifx\childdoctmp\childdocname
        \def\childdoctmp{}
      \else
        \def\childdoctmp
        {
          \childdoctrue
          \includeonly{\childdocname}
          \def\childdocjob{#1}
          \def\jobname{#1}
        }
      \fi
      \expandafter
    \endgroup
    \childdoctmp
  \fi
}
%    \end{macrocode}

% \macro{\childdocof}
% The command |\childdocof| redirects
% compilation to the main file |#1|.
%    \begin{macrocode}
\newcommand{\childdocof}[1]
{
  \childdocdisable
  \childdoctrue
  \includeonly{\childdocname}
  \def\jobname{#1}
  \def\childdocjob{#1}
  \input{#1}
}
%    \end{macrocode}

% \macro{\childdocby}
% The command |\childdocby| ....
%    \begin{macrocode}
\newcommand{\childdocby}[2][]
{
  \childdocdisable
  \childdoctrue
  \childdocmanualtrue
  \if?#1?\else
    \def\jobname{#2}
  \fi
  \def\childdocjob{#2}
  \input{#2}
  \endinput
}
%    \end{macrocode}

% \macro{\childdocforward}
% The command |\childdocforward| redirects
% compilation to the main file or
% (if the optional argument is given) a child file.
% Parameters are set as if the main file
% or a child file starting with |\childdocof| was compiled.
% Then compilation is handed over to the main file:
%    \begin{macrocode}
\newcommand{\childdocforward}[2][]
{
  \begingroup
    \if?#1?
      \def\childdoctmp
      {
        \def\childdocname{#2}
        \def\childdocjob{#2}
        \def\jobname{#2}
        \input{#2}
        \endinput
      }
    \else
      \def\childdoctmp
      {
        \childdocdisable
        \def\childdocname{#2}
        \childdoctrue
        \includeonly{#2}
        \def\childdocjob{#1}
        \def\jobname{#1}
        \input{#1}
        \endinput
      }
    \fi
    \expandafter
  \endgroup
  \childdoctmp
}
%    \end{macrocode}

% \macro{\childdocforwardprefix}
% The command |\childdocforwardprefix| redirects
% compilation to the main or a child file by means of a pattern.
% The prefix |#1| in the current filename is replaced by |#2|
% and the suffix of the current filename is kept
% (it is assumed that the filename does not contain the substring `|~~~|'
% which is used as a delimiter).
% Compilation is handed over to the new file by |\childdocforward|:
%    \begin{macrocode}
\newcommand{\childdocforwardprefix}[3][]
{
  \begingroup
    \def\childdocextract #2##1~~~{\def\childdoctmp{\childdocforward[#1]{#3##1}}}
    \expandafter\childdocextract\childdocname~~~
    \expandafter
  \endgroup
  \childdoctmp
}
%    \end{macrocode}

% \macro{\childdoc}
% The deprecated macro |\childdoc| is a legacy version of |\childdocmain|:
%    \begin{macrocode}
\newcommand{\childdoc}{\childdocmain}
%    \end{macrocode}

% \macro{\childdocredirect}
% The deprecated macro |\childdocredirect| is a legacy version
% of |\childdocforward| and |\childdocforwardprefix|:
%    \begin{macrocode}
\newcommand{\childdocredirect}[2][]
{
  \begingroup
    \if?#1?
      \def\childdoctmp{\childdocforward{#2}}
    \else
      \def\childdoctmp{\childdocforwardprefix{#1}{#2}}
    \fi
    \expandafter
  \endgroup
  \childdoctmp
}
%    \end{macrocode}

%\iffalse
%</package>
%\fi
%
\endinput
|\\
|\childdocforward{|\textit{main}|}|\\
\end{tabular}
\end{center}
%
or alternatively with:
%
\begin{center}
\begin{tabular}{l}
|% \iffalse
%
% childdoc.dtx Copyright (C) 2017-2018 Niklas Beisert
%
% This work may be distributed and/or modified under the
% conditions of the LaTeX Project Public License, either version 1.3
% of this license or (at your option) any later version.
% The latest version of this license is in
%   http://www.latex-project.org/lppl.txt
% and version 1.3 or later is part of all distributions of LaTeX
% version 2005/12/01 or later.
%
% This work has the LPPL maintenance status `maintained'.
%
% The Current Maintainer of this work is Niklas Beisert.
%
% This work consists of the files childdoc.dtx and childdoc.ins
% and the derived files childdoc.def and cdocsamp.tex with
% cdocsch1.tex, cdocsch2.tex, cdocsdrf.tex, cdocsfn1.tex, cdocsfn2.tex.
%
%<package>\ifdefined\childdocmain\endinput\fi
%<package>\ProvidesFile{childdoc.def}[2018/12/30 v2.0 child document driver]
%<samplemain>\ProvidesFile{cdocsamp.tex}[2018/12/30 v2.0 sample for childdoc]
%<*driver>
%\ProvidesFile{childdoc.drv}[2018/12/30 v2.0 childdoc reference manual file]
\PassOptionsToClass{10pt,a4paper}{article}
\documentclass{ltxdoc}

\usepackage[margin=35mm]{geometry}
\usepackage{hyperref}
\usepackage{hyperxmp}
\usepackage[usenames]{color}

\hypersetup{colorlinks=true}
\hypersetup{pdfstartview=FitH}
\hypersetup{pdfpagemode=UseNone}
\hypersetup{pdfsource={}}
\hypersetup{pdflang={en-UK}}
\hypersetup{pdfcopyright={Copyright 2017-2018 Niklas Beisert.
  This work may be distributed and/or modified under the
  conditions of the LaTeX Project Public License, either version 1.3
  of this license or (at your option) any later version.}}
\hypersetup{pdflicenseurl={http://www.latex-project.org/lppl.txt}}
\hypersetup{pdfcontactaddress={ETH Zurich, ITP, HIT K,
  Wolfgang-Pauli-Strasse 27}}
\hypersetup{pdfcontactpostcode={8093}}
\hypersetup{pdfcontactcity={Zurich}}
\hypersetup{pdfcontactcountry={Switzerland}}
\hypersetup{pdfcontactemail={nbeisert@itp.phys.ethz.ch}}
\hypersetup{pdfcontacturl={http://people.phys.ethz.ch/\xmptilde nbeisert/}}

\newcommand{\secref}[1]{\hyperref[#1]{section \ref*{#1}}}

\parskip1ex
\parindent0pt
\let\olditemize\itemize
\def\itemize{\olditemize\parskip0pt}

\begin{document}

\title{The \textsf{childdoc} Package}
\hypersetup{pdftitle={The childdoc Package}}
\author{Niklas Beisert\\[2ex]
  Institut f\"ur Theoretische Physik\\
  Eidgen\"ossische Technische Hochschule Z\"urich\\
  Wolfgang-Pauli-Strasse 27, 8093 Z\"urich, Switzerland\\[1ex]
  \href{mailto:nbeisert@itp.phys.ethz.ch}
  {\texttt{nbeisert@itp.phys.ethz.ch}}}
\hypersetup{pdfauthor={Niklas Beisert}}
\hypersetup{pdfsubject={Manual for the LaTeX2e Package childdoc}}
\date{30 December 2018, \textsf{v2.0}}
\maketitle

\begin{abstract}\noindent
\textsf{childdoc} is a \LaTeXe{} package
that enables the direct compilation
of document sections included by |\include|
to individual files.
\end{abstract}

\begingroup
\parskip0ex
\tableofcontents
\endgroup

%%%%%%%%%%%%%%%%%%%%%%%%%%%%%%%%%%%%%%%%%%%%%%%%%%%%%%%%%%%%%%%%%%%%%%%%%%%%%%%%
%%%%%%%%%%%%%%%%%%%%%%%%%%%%%%%%%%%%%%%%%%%%%%%%%%%%%%%%%%%%%%%%%%%%%%%%%%%%%%%%
\section{Introduction}

\LaTeX{} provides a mechanism to structure a large document (such as a book)
into a main file and several child files (containing the chapters)
using the |\include| command.
This mechanism is beneficial for documents
which span hundreds of pages in order to
make the source file(s) more manageable.
Moreover, compilation can be restricted to
selected child files by means of the |\includeonly| command.
The latter feature can be used to reduce the compilation time while editing
(this was significantly more useful in the earlier days of \LaTeX{})
or to generate a smaller document which is easier to navigate.
Another application of |\includeonly| is to generate
documents consisting of selected parts of the complete document.

However, there are a few drawbacks of the plain |\include| mechanism:
\begin{itemize}
\item
The child files cannot be compiled on their own,
they can only be compiled via the main file.
A naive editing environment
(such as a text editor with an option
to have the current file processed by \LaTeX)
may require one to switch to the main file before compiling;
attempting to compile the child file produces errors.
\item
The main file must be modified (each time)
to adjust the |\includeonly| command
to the present needs. This easily leaves the main file in a messy state.
\item
The generated document will always carry the filename
of the main document. This is inconvenient if
several child files are to be compiled and
to be kept for distribution.
\end{itemize}

The present package provides a simple interface
to make child files individually compilable by \LaTeX{}.
Compiling a child file then has the same effect as compiling
the main file with an |\includeonly| command
to select the appropriate child.
Moreover the generated document will carry the name of the child
rather than the main file.
This resolves all three above issues.

This feature is meant to make the editing of books,
thesis documents and lecture notes somewhat more convenient.
However, the package can also be used efficiently for
composing a series of documents (such as exercise sheets)
which are typically distributed individually.
It then assists the author in generating the individual documents
(potentially in different versions)
as well as a document containing the collected series.
Another application is in developing style files
or other kinds of included material
where compilation of the style file could redirect
to a sample or test file.

%%%%%%%%%%%%%%%%%%%%%%%%%%%%%%%%%%%%%%%%%%%%%%%%%%%%%%%%%%%%%%%%%%%%%%%%%%%%%%%%
%%%%%%%%%%%%%%%%%%%%%%%%%%%%%%%%%%%%%%%%%%%%%%%%%%%%%%%%%%%%%%%%%%%%%%%%%%%%%%%%
\section{Usage}

First of all, the package \textsf{childdoc} is \emph{not} a standard
\LaTeXe{} |.sty| style file! Therefore it needs to be invoked in
a non-standard way.

%%%%%%%%%%%%%%%%%%%%%%%%%%%%%%%%%%%%%%%%%%%%%%%%%%%%%%%%%%%%%%%%%%%%%%%%%%%%%%%%
\subsection{Included Files}
\label{sec:include}

%%%%%%%%%%%%%%%%%%%%%%%%%%%%%%%%%%%%%%%%
\DescribeMacro{\childdocmain}
To use the package, add the commands
\begin{center}
\begin{tabular}{l}
|\input{childdoc.def}|\\
|\childdocmain{}|\\
\end{tabular}
\end{center}
at the very top of the main \LaTeX{} file,
in particular \emph{before} the |\documentclass| statement!
The argument of |\childdocmain| should be left empty
(but it must be present).

%%%%%%%%%%%%%%%%%%%%%%%%%%%%%%%%%%%%%%%%
\DescribeMacro{\childdocof}
Furthermore, add the commands
\begin{center}
\begin{tabular}{l}
|\input{childdoc.def}|\\
|\childdocof{|\textit{main}|}|\\
\end{tabular}
\end{center}
at the top of every child file \textit{child}
which is included by |\include{|\textit{child}|}|
from within the main file
(or at least for those files to be compiled individually).
The argument \textit{main} must be the filename of the main file.

There are a couple of
considerations in setting up the main and child documents:

%%%%%%%%%%%%%%%%%%%%%%%%%%%%%%%%%%%%%%%%
\paragraph{Restrictions.}

Please note the following restrictions:
\begin{itemize}
\item
|\childdocmain| must be called with one argument \textit{main}
to ensure compatibility with earlier version of the package.
It must either be empty (|\childdocmain{}|)
or precisely match the filename of the main file in which it is specified.
See \secref{sec:detection} for further information.
\item
The filename \textit{main} must be specified without the |.tex| extension.
\item
The filename \textit{main} is case sensitive
(even in case-insensitive file systems)
due to internal string comparison.
\item
The argument \textit{main} should be fully expanded, it cannot be a macro.
\item
Subdirectories and special characters should be avoided in filenames.
\item
The command |\childdocmain{|\textit{main}|}| must be followed by a whitespace.
It should not be followed immediately by another command
or by a comment mark `|%|'.
This is because the \TeX{} parser reads the token immediately following
the argument of |\childdocmain| and puts it
at the beginning of every child section;
however, a white\-space is ignored.
\end{itemize}

%%%%%%%%%%%%%%%%%%%%%%%%%%%%%%%%%%%%%%%%
\paragraph{Content of Main File.}

It is advisable to place all content in the child files included by |\include|.
Any output contained in the main file will appear in all child documents
unless suppressed manually;
it cannot be suppressed automatically by the |\includeonly| directive
and thus should normally be avoided.
A method to include some content in the main file
by means of conditional processing is described in \secref{sec:conditional}.

%%%%%%%%%%%%%%%%%%%%%%%%%%%%%%%%%%%%%%%%
\paragraph{Page Numbering.}

When only a part of the document is compiled,
the appropriate numbering of pages
(as well as other status parameters)
is determined from the |.aux| files.
The latter contain information from previous passes.
However this information needs to propagate through
all intermediate child documents.
Therefore the page numbering in child documents may well
be inconsistent until the complete document is compiled at least once.

A useful (if unconventional) way to always ensure a consistent
page numbering is to restart the numbering in each child document
and denote the pages by `\textit{child}|.|\textit{page}'
where \textit{child} represents the chapter/section number of the child file.
This can be achieved by the command
|\numberwithin{page}{|\textit{child}|}|
of the \textsf{amsmath} package
where \textit{child} can be |chapter| or |section|
depending on the chosen structuring.
Alternatively, one can modify the macro |\thepage| appropriately
and reset the counter |page| at the start of each child file.

%%%%%%%%%%%%%%%%%%%%%%%%%%%%%%%%%%%%%%%%%%%%%%%%%%%%%%%%%%%%%%%%%%%%%%%%%%%%%%%%
\subsection{Conditional Processing}
\label{sec:conditional}

The package provides a mechanism to compile different versions
of a document. To customise the versions further some conditional processing
can come in handy to distinguish which version is being compiled.
The package provides two macros to describe the compilation context:

%%%%%%%%%%%%%%%%%%%%%%%%%%%%%%%%%%%%%%%%
\DescribeMacro{\ifchilddoc}
The conditional |\ifchilddoc| distinguishes between the compilation of
child documents and the main document:
%
\begin{center}
|\ifchilddoc |\textit{child-code}| |[|\||else |\textit{main-code}]| \||fi|
\end{center}

%%%%%%%%%%%%%%%%%%%%%%%%%%%%%%%%%%%%%%%%
\DescribeMacro{\childdocname}
\DescribeMacro{\childdocjob}
The macro |\childdocname| contains the filename (without extension)
of the main or child file being processed.
Note that |\childdocjob| will always contain the name of the main file.

%%%%%%%%%%%%%%%%%%%%%%%%%%%%%%%%%%%%%%%%
\paragraph{Title Page.}

Conditional processing can be used to include a title or banner page
in the main document when proper precautions are taken.
Importantly, the code in the main file should ensure that the page counter
(as well as other status parameters which are stored in the |.aux| files)
takes the same value after the conditional processing.
Otherwise the page numbers may take divergent values
depending on which part is compiled.

For example, a title page could be declared by:
%
\begin{center}
\begin{tabular}{l}
|\ifchilddoc\||else|\\
|\addtocounter{page}{-1}|\\
\textit{code for title page}\\
|\newpage|\\
|\||fi|
\end{tabular}
\end{center}
%
A banner page for the child documents can be generated by:
%
\begin{center}
\begin{tabular}{l}
|\ifchilddoc|\\
|\addtocounter{page}{-1}|\\
\textit{code for banner page}\\
|\newpage|\\
|\||fi|
\end{tabular}
\end{center}
%
Here one could write a message such as:
\begin{center}
|This is the part \childdocname{} of \childdocjob{}.|
\end{center}

%%%%%%%%%%%%%%%%%%%%%%%%%%%%%%%%%%%%%%%%%%%%%%%%%%%%%%%%%%%%%%%%%%%%%%%%%%%%%%%%
\subsection{Flags}
\label{sec:flags}

The package makes it easy to generate different versions
of the main or child documents.
To this end compilation flags can be defined
and assigned different default values.
They will be particularly useful in conjunction
with the forwarding mechanism described in \secref{sec:forward}.

For example, it may be useful to have a flag |\version|
which can be set to |draft| or |final|.
The document source will contain some conditional code
depending on the value of |\version|.
Suppose further, the flag should default to |final| for the main file
and to |draft| for child files
which is a natural assignment for editing the document.
This is achieved by placing the following code
in the preamble of the main document
(below the |\childdocmain| directive):
%
\begin{center}
\begin{tabular}{l}
|\ifchilddoc|\\
|\providecommand{\version}{draft}|\\
|\||else|\\
|\providecommand{\version}{final}|\\
|\||fi|
\end{tabular}
\end{center}
%
The definition by |\providecommand| makes sure
that previous definitions are not overwritten.
Further statements |\providecommand{\version}{...}|
can thus be added before the above code to override it.

For the main file, one might add a line
(between |\childdocmain| and the above block)
%
\begin{center}
|%\ifchilddoc\||else\providecommand{\version}{draft}\||fi|
\end{center}
%
which can be uncommented to produce a draft version.
Likewise one can add a line to the very top of a child file
(above the |\childdocof{|\textit{main}|}| directive)
%
\begin{center}
|%\providecommand{\version}{final}|
\end{center}
%
which can be uncommented to produce the final version of this child document.

%%%%%%%%%%%%%%%%%%%%%%%%%%%%%%%%%%%%%%%%%%%%%%%%%%%%%%%%%%%%%%%%%%%%%%%%%%%%%%%%
\subsection{Forwarding}
\label{sec:forward}

Different versions of the main or child documents
using compilation flags as described in \secref{sec:flags}
can be (permanently) stored in different files
for convenient compilation, viewing and distribution.
To this end, the package defines a command
to pass on compilation to a different file:

%%%%%%%%%%%%%%%%%%%%%%%%%%%%%%%%%%%%%%%%
\DescribeMacro{\childdocforward}
The command |\childdocforward| redirects processing to
another source file:
%
\begin{center}
\begin{tabular}{l}
|\input{childdoc.def}|\\
|\childdocforward[|\textit{main}|]{|\textit{dest}|}|\\
\end{tabular}
\end{center}
%
The argument \textit{dest} is the destination file
(without extension).
It should be the main file or one of the child files.
Note that further \textsf{childdoc} directives
such as |\childdocof| and |\childdocforward|
in the indicated file will be processed in this form.
The optional argument \textit{main}
passes on directly to the main file \textit{main}
while pretending to compile the child \textit{dest}.
This form behaves as if \textit{dest}
issues |\childdocof{|\textit{main}|}| right away,
and no further \textsf{childdoc} directives will be processed.

%%%%%%%%%%%%%%%%%%%%%%%%%%%%%%%%%%%%%%%%
\DescribeMacro{\...prefix}
In the alternative form |\childdocforwardprefix|,
%
\begin{center}
\begin{tabular}{l}
|\input{childdoc.def}|\\
|\childdocforwardprefix[|\textit{main}|]{|\textit{prefix}|}{|\textit{dest}|}|
\end{tabular}
\end{center}
%
the destination file is determined by a pattern
depending on the current file:
To make this work, the current file must be called
`{\textit{prefix}\hspace{0.2em}\textit{suffix}}'
with \textit{prefix} matching precisely the argument.
Processing is then passed on to the file
`{\textit{dest}\hspace{0.2em}\textit{suffix}}'.
Surely, the same effect is achieved by
directly specifying the
argument `{\textit{dest}\hspace{0.2em}\textit{suffix}}'
in the first form.
However, that requires to set up a different file
for each child. With the alternative form of the command
all these files can have exactly the same content
which simplifies setting them up and maintaining them.

For example, the following file |draft.tex|
with a compilation flag |\version| as described in \secref{sec:flags}
compiles the main document as a draft:
%
\begin{center}
\begin{tabular}{l}
|\def\version{draft}|\\
|\input{childdoc.def}|\\
|\childdocforward{|\textit{main}|}|
\end{tabular}
\end{center}
%
Likewise, the following files |final|\textit{nn}|.tex|
compile the final version of the child document
|child|\textit{nn}|.tex|:
%
\begin{center}
\begin{tabular}{l}
|\def\version{final}|\\
|\input{childdoc.def}|\\
|\childdocforwardprefix{final}{child}|
\end{tabular}
\end{center}
%

Note that when several versions of a main file and/or of each child file
are to be generated, it may be convenient to set up a |Makefile| or
shell script to automatise the process.

%%%%%%%%%%%%%%%%%%%%%%%%%%%%%%%%%%%%%%%%%%%%%%%%%%%%%%%%%%%%%%%%%%%%%%%%%%%%%%%%
\subsection{Command Line Processing}
\label{sec:commandline}

The effect of redirection files can also be achieved by invoking
the \LaTeX{} compiler with a more elaborate command line.
Most conveniently this should be done as part
of a shell script or a |Makefile|.

When using \textsf{childdoc} in the main file, the following
command lines effectively perform a redirection
(note that depending on the shell being used,
backslashes may have to be doubled: `|\|' $\to$ `|\\|'):
%
\begin{center}
|... -jobname "|\textit{target}|" |\\|"|[\textit{flags}]%
|\input{childdoc.def}\childdocforward[|\textit{main}|]{|\textit{dest}|}"|
\end{center}
%
Here \textit{target} is the name of the output file,
\textit{main} is the name of the main file
and \textit{dest} is the name of the main or child file to be processed
(all filenames without extensions).
The optional argument \textit{main} can be omitted
if \textit{main} matches \textit{dest}.
Optionally, compilation \textit{flags} can be defined via |\def| commands.
This command line makes the \TeX{} engine believe
it is compiling the file \textit{target}
whose content is specified as the latter parameter.
The provided code then forwards the processing to
\textit{main} or \textit{dest} as described in \secref{sec:forward}.

%%%%%%%%%%%%%%%%%%%%%%%%%%%%%%%%%%%%%%%%%%%%%%%%%%%%%%%%%%%%%%%%%%%%%%%%%%%%%%%%
\subsection{Include by Input}
\label{sec:input}

Including child documents by |\include| has some restrictions by design.
Most notably, the content of a child document always occupies
its own set of pages; pages cannot be shared between child documents.
Usually, this behaviour makes perfect sense
because each child document contain an essential part of the document.
However, in some situations it may be desirable to compose
a document from a collection of parts
without having mandatory page breaks between then.
For this case, the package
provides a mechanism to include parts
by |\input| which can also be processed individually.
However, by construction this mechanism
requires manual handling of the content to be output.

%%%%%%%%%%%%%%%%%%%%%%%%%%%%%%%%%%%%%%%%
\DescribeMacro{\ifchilddocmanual}
The main file should be prepared as usual, see \secref{sec:include}.
However, the document body must make a distinction
between processing of an individual part and of the main document, e.g.:
%
\begin{center}
\begin{tabular}{l}
|\ifchilddocmanual|\\
|\input{\childdocname}|\\
|\||else|\\
\textit{document body with }|\input{|\textit{part}|}|\\
|\||fi|
\end{tabular}
\end{center}
%
The conditional |\ifchilddocmanual| is true whenever
a part to be included by |\input| is being compiled,
and the name of the part is stored in |\childdocname|.

%%%%%%%%%%%%%%%%%%%%%%%%%%%%%%%%%%%%%%%%
\DescribeMacro{\childdocby}
Each part to be included by |\input| should start with:
%
\begin{center}
\begin{tabular}{l}
|\input{childdoc.def}|\\
|\childdocby{|\textit{main}|}|\\
\end{tabular}
\end{center}
%
The directive |\childdocby| is similar to |\childdocof|
described in \secref{sec:include},
but the subsequent selection of content must be done manually.
To that end, both |\ifchilddoc| and |\ifchilddocmanual|
will be true upon processing of a part,
and the name of the part is stored in |\childdocname|.
Note that |\jobname| will be set to the filename of the current part
so that each part receives an individual |.aux| file
that does not interfere with the |.aux| file(s) of the main document.
This behaviour can be altered by the alternative form
|\childdocby[*]{|\textit{main}|}| (with a non-empty optional argument)
which uses the |.aux| file of the main document
by setting |\jobname| to \textit{main}.

%%%%%%%%%%%%%%%%%%%%%%%%%%%%%%%%%%%%%%%%%%%%%%%%%%%%%%%%%%%%%%%%%%%%%%%%%%%%%%%%
\subsection{Driver Development}
\label{sec:driver}

The \textsf{childdoc} mechanism can also be use for the development
of definition files such as \LaTeX{} styles or classes.
This case differs from the above setup with multiple parts
included by |\include| in that no |\includeonly| should be invoked.
This can be achieved by starting the include file
(before |\ProvidesPackage|) with:
%
\begin{center}
\begin{tabular}{l}
|\input{childdoc.def}|\\
|\childdocforward{|\textit{main}|}|\\
\end{tabular}
\end{center}
%
or alternatively with:
%
\begin{center}
\begin{tabular}{l}
|\input{childdoc.def}|\\
|\childdocby{|\textit{main}|}|\\
\end{tabular}
\end{center}
%
Both forms have slightly different effects as described above.
The main file is prepared as usual, see \secref{sec:include}.

%%%%%%%%%%%%%%%%%%%%%%%%%%%%%%%%%%%%%%%%%%%%%%%%%%%%%%%%%%%%%%%%%%%%%%%%%%%%%%%%
\subsection{Legacy Detection}
\label{sec:detection}

The directive |\childdocmain| in the main file can detect
whether the complete document or merely a child is to be compiled
even without using the directive |\childdocof|.
This method is deprecated because it is less robust
and there is no compelling reason to use it;
it is merely provided for backward compatibility
and it may be removed in future versions.

If the detection mechanism is to be used,
it is mandatory to correctly specify
the filename of the main file as the argument of |\childdocmain|:
%
\begin{center}
\begin{tabular}{l}
|\input{childdoc.def}|\\
|\childdocmain{|\textit{main}|}|\\
\end{tabular}
\end{center}
%
If |\jobname| does not match the argument \textit{main} of |\childdocmain|,
it is assumed that |\jobname| points to the child file to be compiled.
When using |\childdocmain| with the main file specified as argument,
it suffices to start a child file
with just |\input{|\textit{main}|}|
without loading of the package and using |\childdocof|.
If instead all processing is done
with the appropriate \textsf{childdoc} directives,
the argument of \textit{main} of |\childdocmain| can be empty.

An alternative version of the command line processing described
in \secref{sec:commandline} using the detection mechanism reads:
%
\begin{center}
|... -jobname "|\textit{target}|" "|[\textit{flags}]%
[|\def\jobname{|\textit{dest}|}|]|\input{|\textit{main}|}"|
\end{center}

%%%%%%%%%%%%%%%%%%%%%%%%%%%%%%%%%%%%%%%%%%%%%%%%%%%%%%%%%%%%%%%%%%%%%%%%%%%%%%%%
\subsection{Manual Code}
\label{sec:manual}

In case one cannot be certain whether the definitions file |childdoc.def|
is installed on the target \TeX{} distribution
and one prefers not to ship it,
it is conceivable to paste a few relevant commands into the sources.

To that end, drop all statements |\input{childdoc.def}|
and perform the replacements as outlined below.
Instead of |\childdocmain{|\textit{main}|}| add the following code
to the top of the main file:
%
\begin{center}
\begin{tabular}{l}
|\||ifdefined\childdocname\endinput\||fi\newif\ifchilddoc|\\
|\edef\childdocname{\scantokens\expandafter{\jobname\noexpand}}|\\
|\def\childdocmain{|\textit{main}|}\||ifx\childdocmain\childdocname\||else|\\
|\childdoctrue\includeonly{\childdocname}\let\jobname\childdocmain\||fi|\\
\end{tabular}
\end{center}
%
Instead of |\childdocof{|\textit{main}|}| just include the main file
at the top of each child file:
%
\begin{center}
|\input{|\textit{main}|}|
\end{center}
%
A simple redirection |\childdocforward{|\textit{dest}|}| is achieved by:
%
\begin{center}
|\def\jobname{|\textit{dest}|}\input{\jobname}|
\end{center}
%
The redirection with prefix
|\childdocforwardprefix[|\textit{prefix}|]{|\textit{dest}|}|
is accomplished by:
%
\begin{center}
\begin{tabular}{l}
|{\edef\jobname{\scantokens\expandafter{\jobname\noexpand}}|\\
|\def\redirectjob |\textit{prefix}|#1~~~{\gdef\jobname{|\textit{dest}|#1}}|\\
|\expandafter\redirectjob\jobname~~~}\input{\jobname}|
\end{tabular}
\end{center}

In an alternative approach,
child documents can be compiled by a specific command line
without additional code or specific definitions:
%
\begin{center}
|... -jobname "|\textit{target}|" "|[\textit{flags}]%
|\includeonly{|\textit{dest}|}\input{|\textit{main}|}"|
\end{center}
%

%%%%%%%%%%%%%%%%%%%%%%%%%%%%%%%%%%%%%%%%%%%%%%%%%%%%%%%%%%%%%%%%%%%%%%%%%%%%%%%%
%%%%%%%%%%%%%%%%%%%%%%%%%%%%%%%%%%%%%%%%%%%%%%%%%%%%%%%%%%%%%%%%%%%%%%%%%%%%%%%%
\section{Information}

%%%%%%%%%%%%%%%%%%%%%%%%%%%%%%%%%%%%%%%%%%%%%%%%%%%%%%%%%%%%%%%%%%%%%%%%%%%%%%%%
\subsection{Copyright}

Copyright \copyright{} 2017--2018 Niklas Beisert

This work may be distributed and/or modified under the
conditions of the \LaTeX{} Project Public License, either version 1.3
of this license or (at your option) any later version.
The latest version of this license is in
  \url{http://www.latex-project.org/lppl.txt}
and version 1.3 or later is part of all distributions of \LaTeX{}
version 2005/12/01 or later.

This work has the LPPL maintenance status `maintained'.

The Current Maintainer of this work is Niklas Beisert.

This work consists of the files |README.txt|, |childdoc.ins| and |childdoc.dtx|
as well as the derived files |childdoc.def|, |cdocsamp.tex|
with |cdocsch1.tex|, |cdocsch2.tex|, |cdocspt3.tex|, |cdocspt4.tex|,
|cdocsdrf.tex|, |cdocsfn1.tex|, |cdocsfn2.tex|
as well as |childdoc.pdf|.

%%%%%%%%%%%%%%%%%%%%%%%%%%%%%%%%%%%%%%%%%%%%%%%%%%%%%%%%%%%%%%%%%%%%%%%%%%%%%%%%
\subsection{Files and Installation}

The package consists of the files:
%
\begin{center}
\begin{tabular}{ll}
    |README.txt|   & readme file \\
    |childdoc.ins| & installation file \\
    |childdoc.dtx| & source file \\
    |childdoc.def| & definition file \\
    |cdocsamp.tex| & sample main file \\
    |cdocsch1.tex| & sample include file \\
    |cdocsch2.tex| & sample include file \\
    |cdocspt3.tex| & sample part file \\
    |cdocspt4.tex| & sample part file \\
    |cdocsdrf.tex| & sample redirection file \\
    |cdocsfn1.tex| & sample redirection file \\
    |cdocsfn2.tex| & sample redirection file \\
    |childdoc.pdf| & manual
\end{tabular}
\end{center}
%
The distribution consists of the files
|README.txt|, |childdoc.ins| and |childdoc.dtx|.
%
\begin{itemize}
\item
Run (pdf)\LaTeX{} on |childdoc.dtx|
to compile the manual |childdoc.pdf| (this file).
\item
Run \LaTeX{} on |childdoc.ins| to create the definitions file |childdoc.def|
and the sample |cdocsamp.tex| with include files
|cdocsch1.tex|, |cdocsch2.tex|, |cdocspt3.tex|, |cdocspt4.tex|,
|cdocsdrf.tex|, |cdocsfn1.tex|, |cdocsfn2.tex|.
Then copy the file |childdoc.def| to an appropriate directory of your \LaTeX{}
distribution, e.g.\ \textit{texmf-root}|/tex/latex/childdoc|.
\end{itemize}

%%%%%%%%%%%%%%%%%%%%%%%%%%%%%%%%%%%%%%%%%%%%%%%%%%%%%%%%%%%%%%%%%%%%%%%%%%%%%%%%
\subsection{Related CTAN Packages}

There are several other packages which offer a similar functionality:
%
\begin{itemize}
\item
The packages
\href{http://ctan.org/pkg/docmute}{\textsf{docmute}},
\href{http://ctan.org/pkg/includex}{\textsf{includex}} and
\href{http://ctan.org/pkg/standalone}{\textsf{standalone}}
provide commands to include only the document body of
a child file thus allowing both files to be compiled individually.
\item
The packages \href{http://ctan.org/pkg/subdocs}{\textsf{subdocs}}
and \href{http://ctan.org/pkg/subfiles}{\textsf{subfiles}}
provide structures in which the main and child documents can be
encapsulated and allowing them to be compiled individually.
The inclusion mechanism is different from the conventional |\include|.
\item
The package \href{http://ctan.org/pkg/combine}{\textsf{combine}}
is an elaborate solution to combine several documents into one.
\end{itemize}
%
See also the CTAN topic \href{http://ctan.org/topic/subdocs}{\textsf{subdocs}}
for further related packages.
The present package differs from the above solutions in that
a document structure constructed with the conventional |\include| mechanism
just needs two extra commands at the top of every file
such that all constituent files can be compiled individually.

%%%%%%%%%%%%%%%%%%%%%%%%%%%%%%%%%%%%%%%%%%%%%%%%%%%%%%%%%%%%%%%%%%%%%%%%%%%%%%%%
%\subsection{Feature Suggestions}
%
%The following is a list of features which may be useful for future
%versions of this package:
%%
%\begin{itemize}
%\item
%\ldots
%\end{itemize}

%%%%%%%%%%%%%%%%%%%%%%%%%%%%%%%%%%%%%%%%%%%%%%%%%%%%%%%%%%%%%%%%%%%%%%%%%%%%%%%%
\subsection{Revision History}

%%%%%%%%%%%%%%%%%%%%%%%%%%%%%%%%%%%%%%%%
\paragraph{v2.0:} 2018/12/30

\begin{itemize}
\item
immediate forward processing
\item
added |\childdocby| mechanism
\item
manual restructured
\end{itemize}

%%%%%%%%%%%%%%%%%%%%%%%%%%%%%%%%%%%%%%%%
\paragraph{v1.6:} 2018/01/17

\begin{itemize}
\item
application for development of include files
\item
corrections to manual
\end{itemize}

%%%%%%%%%%%%%%%%%%%%%%%%%%%%%%%%%%%%%%%%
\paragraph{v1.5:} 2017/05/21

\begin{itemize}
\item
more complete structuring introduced
\item
|\childdocof| introduced
\item
|\childdoc| renamed to |\childdocmain|
\item
|\childredirect| renamed to |\childdocforward| and |\childdocforwardprefix|
and functionality expanded
\end{itemize}

%%%%%%%%%%%%%%%%%%%%%%%%%%%%%%%%%%%%%%%%
\paragraph{v1.0:} 2017/04/27

\begin{itemize}
\item
manual and install package
\item
first version published on CTAN
\end{itemize}

%%%%%%%%%%%%%%%%%%%%%%%%%%%%%%%%%%%%%%%%
\paragraph{v0.6:} 2017/04/26

\begin{itemize}
\item
redirection mechanism added
\end{itemize}

%%%%%%%%%%%%%%%%%%%%%%%%%%%%%%%%%%%%%%%%
\paragraph{v0.5:} 2017/04/26

\begin{itemize}
\item
functionality in definition file
\end{itemize}


%%%%%%%%%%%%%%%%%%%%%%%%%%%%%%%%%%%%%%%%%%%%%%%%%%%%%%%%%%%%%%%%%%%%%%%%%%%%%%%%
%%%%%%%%%%%%%%%%%%%%%%%%%%%%%%%%%%%%%%%%%%%%%%%%%%%%%%%%%%%%%%%%%%%%%%%%%%%%%%%%
%%%%%%%%%%%%%%%%%%%%%%%%%%%%%%%%%%%%%%%%%%%%%%%%%%%%%%%%%%%%%%%%%%%%%%%%%%%%%%%%
\appendix

\settowidth\MacroIndent{\rmfamily\scriptsize 000\ }

 \DocInput{childdoc.dtx}

\end{document}
%</driver>
% \fi
%
% %%%%%%%%%%%%%%%%%%%%%%%%%%%%%%%%%%%%%%%%%%%%%%%%%%%%%%%%%%%%%%%%%%%%%%%%%%%%%%
% %%%%%%%%%%%%%%%%%%%%%%%%%%%%%%%%%%%%%%%%%%%%%%%%%%%%%%%%%%%%%%%%%%%%%%%%%%%%%%
% \section{Sample}
%\iffalse
%<*samplemain>
%\fi
%
% The following presents a sample document
% with two chapters, two parts, a title page,
% a compile flag as well as three forwarding files to set the flag.
% It consists of eight |.tex| files:
% \begin{center}
% \begin{tabular}{ll}
% |cdocsamp.tex|&main file\\
% |cdocsch1.tex|&include file for chapter 1\\
% |cdocsch2.tex|&include file for chapter 2\\
% |cdocspt3.tex|&include file for part 3\\
% |cdocspt4.tex|&include file for part 4\\
% |cdocsdrf.tex|&forwarding file for main file in draft mode\\
% |cdocsfi1.tex|&forwarding file for final version of chapter 1\\
% |cdocsfi2.tex|&forwarding file for final version of chapter 2\\
% \end{tabular}
% \end{center}
% Each of the eight files can be compiled directly by the \LaTeX{} compiler.
%
% %%%%%%%%%%%%%%%%%%%%%%%%%%%%%%%%%%%%%%
% \paragraph{Main File.}
%
% The main file is called |cdocsamp.tex|.
%
% Load the \textsf{childdoc} definitions and
% declare the filename for the main document:
%    \begin{macrocode}
\input{childdoc.def}
\childdocmain{}
%    \end{macrocode}

% Optional override for |\version| flag:
%    \begin{macrocode}
%%\ifchilddoc\else\providecommand{\version}{draft}\fi
%    \end{macrocode}

% Define the default values for the |\version| flag
% (|final| for the main file and |draft| for childs):
%    \begin{macrocode}
\ifchilddoc
\providecommand{\version}{draft}
\else
\providecommand{\version}{final}
\fi
%    \end{macrocode}

% Load the standard document class:
%    \begin{macrocode}
\documentclass[12pt]{article}
%    \end{macrocode}

% Start the document body:
%    \begin{macrocode}
\begin{document}
%    \end{macrocode}

% Declare a title page.
% Print title, part of document being processed and version flag:
%    \begin{macrocode}
\addtocounter{page}{-1}
\begin{center}
{\LARGE\bfseries{}childdoc example\par}
\vspace{1cm}
\ifchilddoc
\ifchilddocmanual part\else chapter\fi:
`\childdocname' of `\childdocjob'\par
\else
main document: `\childdocjob'\par
\fi
version: \version\par
\end{center}
\newpage
%    \end{macrocode}

% Manually include selected file,
% otherwise process as usual:
%    \begin{macrocode}
\ifchilddocmanual
\section*{part `\childdocname'}
\input{\childdocname}
\else
%    \end{macrocode}

% Include the two chapters:
%    \begin{macrocode}
\include{cdocsch1}
\include{cdocsch2}
%    \end{macrocode}

% Include the two parts unless only chapters should be displayed:
%    \begin{macrocode}
\ifchilddoc\else
\section{part three}
\input{cdocspt3}
\section{part four}
\input{cdocspt4}
\fi
%    \end{macrocode}

% Process as usual until here:
%    \begin{macrocode}
\fi
%    \end{macrocode}

% End of document body:
%    \begin{macrocode}
\end{document}
%    \end{macrocode}
%\iffalse
%</samplemain>
%\fi
%
% %%%%%%%%%%%%%%%%%%%%%%%%%%%%%%%%%%%%%%
% \paragraph{Chapter Include Files.}
%
% The include files are called |cdocsch1.tex| and |cdocsch2.tex|.
%
%\iffalse
%<*samplechap1|samplechap2>
%\fi

% Optional override for |\version| flag:
%    \begin{macrocode}
%%\providecommand{\version}{final}
%    \end{macrocode}

% Include the main document:
%    \begin{macrocode}
\input{childdoc.def}
\childdocof{cdocsamp}
%    \end{macrocode}

%\iffalse
%</samplechap1|samplechap2>
%\fi
%
%\iffalse
%<*samplechap1>
%\fi
% Some text for chapter 1:
%    \begin{macrocode}
\section{one}
some text in chapter one
%    \end{macrocode}

%\iffalse
%</samplechap1>
%\fi
% Some text for chapter 2:
%\iffalse
%<*samplechap2>
%\fi
%    \begin{macrocode}
\section{two}
more text in chapter two
%    \end{macrocode}

%\iffalse
%</samplechap2>
%\fi
%
% %%%%%%%%%%%%%%%%%%%%%%%%%%%%%%%%%%%%%%
% \paragraph{Part Include Files.}
%
% The include files are called |cdocspt3.tex| and |cdocspt4.tex|.
%
%\iffalse
%<*samplepart3|samplepart4>
%\fi

% Optional override for |\version| flag:
%    \begin{macrocode}
%%\providecommand{\version}{final}
%    \end{macrocode}

% Include the main document:
%    \begin{macrocode}
\input{childdoc.def}
\childdocby{cdocsamp}
%    \end{macrocode}

%\iffalse
%</samplepart3|samplepart4>
%\fi
%
%\iffalse
%<*samplepart3>
%\fi
% Some text for part 3:
%    \begin{macrocode}
some text in part three
%    \end{macrocode}

%\iffalse
%</samplepart3>
%\fi
% Some text for part 4:
%\iffalse
%<*samplepart4>
%\fi
%    \begin{macrocode}
more text in part four
%    \end{macrocode}

%\iffalse
%</samplepart4>
%\fi
%
% %%%%%%%%%%%%%%%%%%%%%%%%%%%%%%%%%%%%%%
% \paragraph{Forwarding for a Complete Draft.}
%
% The following forwarding file |cdocsdrf.tex|
% compiles the main document in draft mode:
%\iffalse
%<*sampledraft>
%\fi
%    \begin{macrocode}
\def\version{draft}
\input{childdoc.def}
\childdocforward{cdocsamp}
%    \end{macrocode}

%\iffalse
%</sampledraft>
%\fi
%
% %%%%%%%%%%%%%%%%%%%%%%%%%%%%%%%%%%%%%%
% \paragraph{Forwarding for Final Version of the Chapters.}
%
% The following forwarding files |cdocsfn1.tex| and |cdocsfn2.tex|
% (with identical content)
% compile the final versions of the child documents
% |cdocsch1.tex| and |cdocsch2.tex|, respectively:
%\iffalse
%<*samplefinal>
%\fi
%    \begin{macrocode}
\def\version{final}
\input{childdoc.def}
\childdocforwardprefix[cdocsamp]{cdocsfn}{cdocsch}
%    \end{macrocode}

%\iffalse
%</samplefinal>
%\fi
%
% %%%%%%%%%%%%%%%%%%%%%%%%%%%%%%%%%%%%%%
% \paragraph{Command Line Processing.}
%
% The following three command lines generate the output files
% |cdocscld|, |cdocscl1| and |cdocscl2|
% which should be identical to
% |cdocsdrf|, |cdocsch1| and |cdocsfn2|, respectively:
% \begin{center}
% \begin{tabular}{l}
% |latex -jobname cdocscld \|\\
% |  "\def\version{draft}\input{childdoc.def}\childdocforward{cdocsamp}"|\\
% |latex -jobname cdocscl1 \|\\
% |  "\input{childdoc.def}\childdocforward[cdocsamp]{cdocsch1}"|\\
% |latex -jobname cdocscl2 \|\\
% |  "\def\version{final}\input{childdoc.def}\childdocforward{cdocsch2}"|
% \end{tabular}
% \end{center}
% Note that the trailing backslash on each first line
% merely continues the input to the second line
% (for convenient cut ant paste).
% Furthermore, the command |latex| can be replaced by any
% of its alternative versions such as |pdflatex|.
%
% %%%%%%%%%%%%%%%%%%%%%%%%%%%%%%%%%%%%%%%%%%%%%%%%%%%%%%%%%%%%%%%%%%%%%%%%%%%%%%
% %%%%%%%%%%%%%%%%%%%%%%%%%%%%%%%%%%%%%%%%%%%%%%%%%%%%%%%%%%%%%%%%%%%%%%%%%%%%%%
% \section{Implementation}
%\iffalse
%<*package>
%\fi
%
% This section describes the definitions file |childdoc.def|.

% The definitions cannot be loaded using |\usepackage| or |\RequirePackage|
% which has a mechanism to prevent loading a style file more than once.
% When loading the definitions by means of |\input|
% multiple instances have to be prevented manually:
%\iffalse
%This code needs to be before the `\ProvidesFile' directive
%which is defined at the beginning of this file.
%Therefore it is also placed there and commented out here.
%</package>
%<*discard>
%\fi
%    \begin{macrocode}
\ifdefined\childdocmain\endinput\fi
%    \end{macrocode}
%\iffalse
%</discard>
%<*package>
%\fi
%
% \macro{\ifchilddoc}
% \macro{\ifchilddocmanual}
% The conditional |\ifchilddoc| tells whether a
% child (true) or main (false) document is being compiled.
% The conditional |\ifchilddocmanual| tells whether
% the |\includeonly| mechanism is used (false) or
% the selection of child files must be performed manually (true).
% The definitions initialise to false:
%    \begin{macrocode}
\newif\ifchilddoc
\newif\ifchilddocmanual
%    \end{macrocode}

% \macro{\childdocname}
% \macro{\childdocjob}
% The macro |\childdocname| stores the name of the main document
% to be compiled. The macro |\childdocjob| stores the name of
% the document on which the \LaTeX{} compiler was originally invoked.
% The content of |\jobname| cannot be compared
% to filenames specified in the source due to different catcodes.
% The following code rescans |\jobname|, stores the result
% in |\childdocname| and saves a copy in |\childdocjob|:
%    \begin{macrocode}
\edef\childdocname{\scantokens\expandafter{\jobname\noexpand}}
\let\childdocjob\childdocname
%    \end{macrocode}

% \macro{\childdocdisable}
% The macro |\childdocdisable| prevents the main file
% from being processed more than once.
% At this stage, the main document command |\childdocmain|
% is assumed to be called once again where it should do nothing.
% Any subsequent call to it should prevent
% a secondary processing of the main document
% It overwrites the forwarding commands
% |\childdocof| and |\childdocforward|
% with empty macros to prevent further inclusions of the main document:
%    \begin{macrocode}
\newcommand{\childdocdisable}
{
  \renewcommand{\childdocmain}[1]{\renewcommand{\childdocmain}[1]{\endinput}}
  \renewcommand{\childdocof}[1]{}
  \renewcommand{\childdocby}[2][]{}
  \renewcommand{\childdocforward}[2][]{}
  \renewcommand{\childdocdisable}{}
}
%    \end{macrocode}

% \macro{\childdocmain}
% The macro |\childdocmain| is to be called at the top of the main file
% with nothing or the main filename (without extension) as argument.
% First, it breaks loops.
% If the argument is not empty and does not match |\childdocname|
% (which is set by the first inclusion of |childdoc.def|),
% |\ifchilddoc| is set to true, |\includeonly| is applied to the child file
% and |\jobname| is set to the main file
% (for proper handling of |.aux| files):
%    \begin{macrocode}
\newcommand{\childdocmain}[1]
{
  \childdocdisable\childdocmain{}
  \if?#1?\else
    \begingroup
      \def\childdoctmp{#1}
      \ifx\childdoctmp\childdocname
        \def\childdoctmp{}
      \else
        \def\childdoctmp
        {
          \childdoctrue
          \includeonly{\childdocname}
          \def\childdocjob{#1}
          \def\jobname{#1}
        }
      \fi
      \expandafter
    \endgroup
    \childdoctmp
  \fi
}
%    \end{macrocode}

% \macro{\childdocof}
% The command |\childdocof| redirects
% compilation to the main file |#1|.
%    \begin{macrocode}
\newcommand{\childdocof}[1]
{
  \childdocdisable
  \childdoctrue
  \includeonly{\childdocname}
  \def\jobname{#1}
  \def\childdocjob{#1}
  \input{#1}
}
%    \end{macrocode}

% \macro{\childdocby}
% The command |\childdocby| ....
%    \begin{macrocode}
\newcommand{\childdocby}[2][]
{
  \childdocdisable
  \childdoctrue
  \childdocmanualtrue
  \if?#1?\else
    \def\jobname{#2}
  \fi
  \def\childdocjob{#2}
  \input{#2}
  \endinput
}
%    \end{macrocode}

% \macro{\childdocforward}
% The command |\childdocforward| redirects
% compilation to the main file or
% (if the optional argument is given) a child file.
% Parameters are set as if the main file
% or a child file starting with |\childdocof| was compiled.
% Then compilation is handed over to the main file:
%    \begin{macrocode}
\newcommand{\childdocforward}[2][]
{
  \begingroup
    \if?#1?
      \def\childdoctmp
      {
        \def\childdocname{#2}
        \def\childdocjob{#2}
        \def\jobname{#2}
        \input{#2}
        \endinput
      }
    \else
      \def\childdoctmp
      {
        \childdocdisable
        \def\childdocname{#2}
        \childdoctrue
        \includeonly{#2}
        \def\childdocjob{#1}
        \def\jobname{#1}
        \input{#1}
        \endinput
      }
    \fi
    \expandafter
  \endgroup
  \childdoctmp
}
%    \end{macrocode}

% \macro{\childdocforwardprefix}
% The command |\childdocforwardprefix| redirects
% compilation to the main or a child file by means of a pattern.
% The prefix |#1| in the current filename is replaced by |#2|
% and the suffix of the current filename is kept
% (it is assumed that the filename does not contain the substring `|~~~|'
% which is used as a delimiter).
% Compilation is handed over to the new file by |\childdocforward|:
%    \begin{macrocode}
\newcommand{\childdocforwardprefix}[3][]
{
  \begingroup
    \def\childdocextract #2##1~~~{\def\childdoctmp{\childdocforward[#1]{#3##1}}}
    \expandafter\childdocextract\childdocname~~~
    \expandafter
  \endgroup
  \childdoctmp
}
%    \end{macrocode}

% \macro{\childdoc}
% The deprecated macro |\childdoc| is a legacy version of |\childdocmain|:
%    \begin{macrocode}
\newcommand{\childdoc}{\childdocmain}
%    \end{macrocode}

% \macro{\childdocredirect}
% The deprecated macro |\childdocredirect| is a legacy version
% of |\childdocforward| and |\childdocforwardprefix|:
%    \begin{macrocode}
\newcommand{\childdocredirect}[2][]
{
  \begingroup
    \if?#1?
      \def\childdoctmp{\childdocforward{#2}}
    \else
      \def\childdoctmp{\childdocforwardprefix{#1}{#2}}
    \fi
    \expandafter
  \endgroup
  \childdoctmp
}
%    \end{macrocode}

%\iffalse
%</package>
%\fi
%
\endinput
|\\
|\childdocby{|\textit{main}|}|\\
\end{tabular}
\end{center}
%
Both forms have slightly different effects as described above.
The main file is prepared as usual, see \secref{sec:include}.

%%%%%%%%%%%%%%%%%%%%%%%%%%%%%%%%%%%%%%%%%%%%%%%%%%%%%%%%%%%%%%%%%%%%%%%%%%%%%%%%
\subsection{Legacy Detection}
\label{sec:detection}

The directive |\childdocmain| in the main file can detect
whether the complete document or merely a child is to be compiled
even without using the directive |\childdocof|.
This method is deprecated because it is less robust
and there is no compelling reason to use it;
it is merely provided for backward compatibility
and it may be removed in future versions.

If the detection mechanism is to be used,
it is mandatory to correctly specify
the filename of the main file as the argument of |\childdocmain|:
%
\begin{center}
\begin{tabular}{l}
|% \iffalse
%
% childdoc.dtx Copyright (C) 2017-2018 Niklas Beisert
%
% This work may be distributed and/or modified under the
% conditions of the LaTeX Project Public License, either version 1.3
% of this license or (at your option) any later version.
% The latest version of this license is in
%   http://www.latex-project.org/lppl.txt
% and version 1.3 or later is part of all distributions of LaTeX
% version 2005/12/01 or later.
%
% This work has the LPPL maintenance status `maintained'.
%
% The Current Maintainer of this work is Niklas Beisert.
%
% This work consists of the files childdoc.dtx and childdoc.ins
% and the derived files childdoc.def and cdocsamp.tex with
% cdocsch1.tex, cdocsch2.tex, cdocsdrf.tex, cdocsfn1.tex, cdocsfn2.tex.
%
%<package>\ifdefined\childdocmain\endinput\fi
%<package>\ProvidesFile{childdoc.def}[2018/12/30 v2.0 child document driver]
%<samplemain>\ProvidesFile{cdocsamp.tex}[2018/12/30 v2.0 sample for childdoc]
%<*driver>
%\ProvidesFile{childdoc.drv}[2018/12/30 v2.0 childdoc reference manual file]
\PassOptionsToClass{10pt,a4paper}{article}
\documentclass{ltxdoc}

\usepackage[margin=35mm]{geometry}
\usepackage{hyperref}
\usepackage{hyperxmp}
\usepackage[usenames]{color}

\hypersetup{colorlinks=true}
\hypersetup{pdfstartview=FitH}
\hypersetup{pdfpagemode=UseNone}
\hypersetup{pdfsource={}}
\hypersetup{pdflang={en-UK}}
\hypersetup{pdfcopyright={Copyright 2017-2018 Niklas Beisert.
  This work may be distributed and/or modified under the
  conditions of the LaTeX Project Public License, either version 1.3
  of this license or (at your option) any later version.}}
\hypersetup{pdflicenseurl={http://www.latex-project.org/lppl.txt}}
\hypersetup{pdfcontactaddress={ETH Zurich, ITP, HIT K,
  Wolfgang-Pauli-Strasse 27}}
\hypersetup{pdfcontactpostcode={8093}}
\hypersetup{pdfcontactcity={Zurich}}
\hypersetup{pdfcontactcountry={Switzerland}}
\hypersetup{pdfcontactemail={nbeisert@itp.phys.ethz.ch}}
\hypersetup{pdfcontacturl={http://people.phys.ethz.ch/\xmptilde nbeisert/}}

\newcommand{\secref}[1]{\hyperref[#1]{section \ref*{#1}}}

\parskip1ex
\parindent0pt
\let\olditemize\itemize
\def\itemize{\olditemize\parskip0pt}

\begin{document}

\title{The \textsf{childdoc} Package}
\hypersetup{pdftitle={The childdoc Package}}
\author{Niklas Beisert\\[2ex]
  Institut f\"ur Theoretische Physik\\
  Eidgen\"ossische Technische Hochschule Z\"urich\\
  Wolfgang-Pauli-Strasse 27, 8093 Z\"urich, Switzerland\\[1ex]
  \href{mailto:nbeisert@itp.phys.ethz.ch}
  {\texttt{nbeisert@itp.phys.ethz.ch}}}
\hypersetup{pdfauthor={Niklas Beisert}}
\hypersetup{pdfsubject={Manual for the LaTeX2e Package childdoc}}
\date{30 December 2018, \textsf{v2.0}}
\maketitle

\begin{abstract}\noindent
\textsf{childdoc} is a \LaTeXe{} package
that enables the direct compilation
of document sections included by |\include|
to individual files.
\end{abstract}

\begingroup
\parskip0ex
\tableofcontents
\endgroup

%%%%%%%%%%%%%%%%%%%%%%%%%%%%%%%%%%%%%%%%%%%%%%%%%%%%%%%%%%%%%%%%%%%%%%%%%%%%%%%%
%%%%%%%%%%%%%%%%%%%%%%%%%%%%%%%%%%%%%%%%%%%%%%%%%%%%%%%%%%%%%%%%%%%%%%%%%%%%%%%%
\section{Introduction}

\LaTeX{} provides a mechanism to structure a large document (such as a book)
into a main file and several child files (containing the chapters)
using the |\include| command.
This mechanism is beneficial for documents
which span hundreds of pages in order to
make the source file(s) more manageable.
Moreover, compilation can be restricted to
selected child files by means of the |\includeonly| command.
The latter feature can be used to reduce the compilation time while editing
(this was significantly more useful in the earlier days of \LaTeX{})
or to generate a smaller document which is easier to navigate.
Another application of |\includeonly| is to generate
documents consisting of selected parts of the complete document.

However, there are a few drawbacks of the plain |\include| mechanism:
\begin{itemize}
\item
The child files cannot be compiled on their own,
they can only be compiled via the main file.
A naive editing environment
(such as a text editor with an option
to have the current file processed by \LaTeX)
may require one to switch to the main file before compiling;
attempting to compile the child file produces errors.
\item
The main file must be modified (each time)
to adjust the |\includeonly| command
to the present needs. This easily leaves the main file in a messy state.
\item
The generated document will always carry the filename
of the main document. This is inconvenient if
several child files are to be compiled and
to be kept for distribution.
\end{itemize}

The present package provides a simple interface
to make child files individually compilable by \LaTeX{}.
Compiling a child file then has the same effect as compiling
the main file with an |\includeonly| command
to select the appropriate child.
Moreover the generated document will carry the name of the child
rather than the main file.
This resolves all three above issues.

This feature is meant to make the editing of books,
thesis documents and lecture notes somewhat more convenient.
However, the package can also be used efficiently for
composing a series of documents (such as exercise sheets)
which are typically distributed individually.
It then assists the author in generating the individual documents
(potentially in different versions)
as well as a document containing the collected series.
Another application is in developing style files
or other kinds of included material
where compilation of the style file could redirect
to a sample or test file.

%%%%%%%%%%%%%%%%%%%%%%%%%%%%%%%%%%%%%%%%%%%%%%%%%%%%%%%%%%%%%%%%%%%%%%%%%%%%%%%%
%%%%%%%%%%%%%%%%%%%%%%%%%%%%%%%%%%%%%%%%%%%%%%%%%%%%%%%%%%%%%%%%%%%%%%%%%%%%%%%%
\section{Usage}

First of all, the package \textsf{childdoc} is \emph{not} a standard
\LaTeXe{} |.sty| style file! Therefore it needs to be invoked in
a non-standard way.

%%%%%%%%%%%%%%%%%%%%%%%%%%%%%%%%%%%%%%%%%%%%%%%%%%%%%%%%%%%%%%%%%%%%%%%%%%%%%%%%
\subsection{Included Files}
\label{sec:include}

%%%%%%%%%%%%%%%%%%%%%%%%%%%%%%%%%%%%%%%%
\DescribeMacro{\childdocmain}
To use the package, add the commands
\begin{center}
\begin{tabular}{l}
|\input{childdoc.def}|\\
|\childdocmain{}|\\
\end{tabular}
\end{center}
at the very top of the main \LaTeX{} file,
in particular \emph{before} the |\documentclass| statement!
The argument of |\childdocmain| should be left empty
(but it must be present).

%%%%%%%%%%%%%%%%%%%%%%%%%%%%%%%%%%%%%%%%
\DescribeMacro{\childdocof}
Furthermore, add the commands
\begin{center}
\begin{tabular}{l}
|\input{childdoc.def}|\\
|\childdocof{|\textit{main}|}|\\
\end{tabular}
\end{center}
at the top of every child file \textit{child}
which is included by |\include{|\textit{child}|}|
from within the main file
(or at least for those files to be compiled individually).
The argument \textit{main} must be the filename of the main file.

There are a couple of
considerations in setting up the main and child documents:

%%%%%%%%%%%%%%%%%%%%%%%%%%%%%%%%%%%%%%%%
\paragraph{Restrictions.}

Please note the following restrictions:
\begin{itemize}
\item
|\childdocmain| must be called with one argument \textit{main}
to ensure compatibility with earlier version of the package.
It must either be empty (|\childdocmain{}|)
or precisely match the filename of the main file in which it is specified.
See \secref{sec:detection} for further information.
\item
The filename \textit{main} must be specified without the |.tex| extension.
\item
The filename \textit{main} is case sensitive
(even in case-insensitive file systems)
due to internal string comparison.
\item
The argument \textit{main} should be fully expanded, it cannot be a macro.
\item
Subdirectories and special characters should be avoided in filenames.
\item
The command |\childdocmain{|\textit{main}|}| must be followed by a whitespace.
It should not be followed immediately by another command
or by a comment mark `|%|'.
This is because the \TeX{} parser reads the token immediately following
the argument of |\childdocmain| and puts it
at the beginning of every child section;
however, a white\-space is ignored.
\end{itemize}

%%%%%%%%%%%%%%%%%%%%%%%%%%%%%%%%%%%%%%%%
\paragraph{Content of Main File.}

It is advisable to place all content in the child files included by |\include|.
Any output contained in the main file will appear in all child documents
unless suppressed manually;
it cannot be suppressed automatically by the |\includeonly| directive
and thus should normally be avoided.
A method to include some content in the main file
by means of conditional processing is described in \secref{sec:conditional}.

%%%%%%%%%%%%%%%%%%%%%%%%%%%%%%%%%%%%%%%%
\paragraph{Page Numbering.}

When only a part of the document is compiled,
the appropriate numbering of pages
(as well as other status parameters)
is determined from the |.aux| files.
The latter contain information from previous passes.
However this information needs to propagate through
all intermediate child documents.
Therefore the page numbering in child documents may well
be inconsistent until the complete document is compiled at least once.

A useful (if unconventional) way to always ensure a consistent
page numbering is to restart the numbering in each child document
and denote the pages by `\textit{child}|.|\textit{page}'
where \textit{child} represents the chapter/section number of the child file.
This can be achieved by the command
|\numberwithin{page}{|\textit{child}|}|
of the \textsf{amsmath} package
where \textit{child} can be |chapter| or |section|
depending on the chosen structuring.
Alternatively, one can modify the macro |\thepage| appropriately
and reset the counter |page| at the start of each child file.

%%%%%%%%%%%%%%%%%%%%%%%%%%%%%%%%%%%%%%%%%%%%%%%%%%%%%%%%%%%%%%%%%%%%%%%%%%%%%%%%
\subsection{Conditional Processing}
\label{sec:conditional}

The package provides a mechanism to compile different versions
of a document. To customise the versions further some conditional processing
can come in handy to distinguish which version is being compiled.
The package provides two macros to describe the compilation context:

%%%%%%%%%%%%%%%%%%%%%%%%%%%%%%%%%%%%%%%%
\DescribeMacro{\ifchilddoc}
The conditional |\ifchilddoc| distinguishes between the compilation of
child documents and the main document:
%
\begin{center}
|\ifchilddoc |\textit{child-code}| |[|\||else |\textit{main-code}]| \||fi|
\end{center}

%%%%%%%%%%%%%%%%%%%%%%%%%%%%%%%%%%%%%%%%
\DescribeMacro{\childdocname}
\DescribeMacro{\childdocjob}
The macro |\childdocname| contains the filename (without extension)
of the main or child file being processed.
Note that |\childdocjob| will always contain the name of the main file.

%%%%%%%%%%%%%%%%%%%%%%%%%%%%%%%%%%%%%%%%
\paragraph{Title Page.}

Conditional processing can be used to include a title or banner page
in the main document when proper precautions are taken.
Importantly, the code in the main file should ensure that the page counter
(as well as other status parameters which are stored in the |.aux| files)
takes the same value after the conditional processing.
Otherwise the page numbers may take divergent values
depending on which part is compiled.

For example, a title page could be declared by:
%
\begin{center}
\begin{tabular}{l}
|\ifchilddoc\||else|\\
|\addtocounter{page}{-1}|\\
\textit{code for title page}\\
|\newpage|\\
|\||fi|
\end{tabular}
\end{center}
%
A banner page for the child documents can be generated by:
%
\begin{center}
\begin{tabular}{l}
|\ifchilddoc|\\
|\addtocounter{page}{-1}|\\
\textit{code for banner page}\\
|\newpage|\\
|\||fi|
\end{tabular}
\end{center}
%
Here one could write a message such as:
\begin{center}
|This is the part \childdocname{} of \childdocjob{}.|
\end{center}

%%%%%%%%%%%%%%%%%%%%%%%%%%%%%%%%%%%%%%%%%%%%%%%%%%%%%%%%%%%%%%%%%%%%%%%%%%%%%%%%
\subsection{Flags}
\label{sec:flags}

The package makes it easy to generate different versions
of the main or child documents.
To this end compilation flags can be defined
and assigned different default values.
They will be particularly useful in conjunction
with the forwarding mechanism described in \secref{sec:forward}.

For example, it may be useful to have a flag |\version|
which can be set to |draft| or |final|.
The document source will contain some conditional code
depending on the value of |\version|.
Suppose further, the flag should default to |final| for the main file
and to |draft| for child files
which is a natural assignment for editing the document.
This is achieved by placing the following code
in the preamble of the main document
(below the |\childdocmain| directive):
%
\begin{center}
\begin{tabular}{l}
|\ifchilddoc|\\
|\providecommand{\version}{draft}|\\
|\||else|\\
|\providecommand{\version}{final}|\\
|\||fi|
\end{tabular}
\end{center}
%
The definition by |\providecommand| makes sure
that previous definitions are not overwritten.
Further statements |\providecommand{\version}{...}|
can thus be added before the above code to override it.

For the main file, one might add a line
(between |\childdocmain| and the above block)
%
\begin{center}
|%\ifchilddoc\||else\providecommand{\version}{draft}\||fi|
\end{center}
%
which can be uncommented to produce a draft version.
Likewise one can add a line to the very top of a child file
(above the |\childdocof{|\textit{main}|}| directive)
%
\begin{center}
|%\providecommand{\version}{final}|
\end{center}
%
which can be uncommented to produce the final version of this child document.

%%%%%%%%%%%%%%%%%%%%%%%%%%%%%%%%%%%%%%%%%%%%%%%%%%%%%%%%%%%%%%%%%%%%%%%%%%%%%%%%
\subsection{Forwarding}
\label{sec:forward}

Different versions of the main or child documents
using compilation flags as described in \secref{sec:flags}
can be (permanently) stored in different files
for convenient compilation, viewing and distribution.
To this end, the package defines a command
to pass on compilation to a different file:

%%%%%%%%%%%%%%%%%%%%%%%%%%%%%%%%%%%%%%%%
\DescribeMacro{\childdocforward}
The command |\childdocforward| redirects processing to
another source file:
%
\begin{center}
\begin{tabular}{l}
|\input{childdoc.def}|\\
|\childdocforward[|\textit{main}|]{|\textit{dest}|}|\\
\end{tabular}
\end{center}
%
The argument \textit{dest} is the destination file
(without extension).
It should be the main file or one of the child files.
Note that further \textsf{childdoc} directives
such as |\childdocof| and |\childdocforward|
in the indicated file will be processed in this form.
The optional argument \textit{main}
passes on directly to the main file \textit{main}
while pretending to compile the child \textit{dest}.
This form behaves as if \textit{dest}
issues |\childdocof{|\textit{main}|}| right away,
and no further \textsf{childdoc} directives will be processed.

%%%%%%%%%%%%%%%%%%%%%%%%%%%%%%%%%%%%%%%%
\DescribeMacro{\...prefix}
In the alternative form |\childdocforwardprefix|,
%
\begin{center}
\begin{tabular}{l}
|\input{childdoc.def}|\\
|\childdocforwardprefix[|\textit{main}|]{|\textit{prefix}|}{|\textit{dest}|}|
\end{tabular}
\end{center}
%
the destination file is determined by a pattern
depending on the current file:
To make this work, the current file must be called
`{\textit{prefix}\hspace{0.2em}\textit{suffix}}'
with \textit{prefix} matching precisely the argument.
Processing is then passed on to the file
`{\textit{dest}\hspace{0.2em}\textit{suffix}}'.
Surely, the same effect is achieved by
directly specifying the
argument `{\textit{dest}\hspace{0.2em}\textit{suffix}}'
in the first form.
However, that requires to set up a different file
for each child. With the alternative form of the command
all these files can have exactly the same content
which simplifies setting them up and maintaining them.

For example, the following file |draft.tex|
with a compilation flag |\version| as described in \secref{sec:flags}
compiles the main document as a draft:
%
\begin{center}
\begin{tabular}{l}
|\def\version{draft}|\\
|\input{childdoc.def}|\\
|\childdocforward{|\textit{main}|}|
\end{tabular}
\end{center}
%
Likewise, the following files |final|\textit{nn}|.tex|
compile the final version of the child document
|child|\textit{nn}|.tex|:
%
\begin{center}
\begin{tabular}{l}
|\def\version{final}|\\
|\input{childdoc.def}|\\
|\childdocforwardprefix{final}{child}|
\end{tabular}
\end{center}
%

Note that when several versions of a main file and/or of each child file
are to be generated, it may be convenient to set up a |Makefile| or
shell script to automatise the process.

%%%%%%%%%%%%%%%%%%%%%%%%%%%%%%%%%%%%%%%%%%%%%%%%%%%%%%%%%%%%%%%%%%%%%%%%%%%%%%%%
\subsection{Command Line Processing}
\label{sec:commandline}

The effect of redirection files can also be achieved by invoking
the \LaTeX{} compiler with a more elaborate command line.
Most conveniently this should be done as part
of a shell script or a |Makefile|.

When using \textsf{childdoc} in the main file, the following
command lines effectively perform a redirection
(note that depending on the shell being used,
backslashes may have to be doubled: `|\|' $\to$ `|\\|'):
%
\begin{center}
|... -jobname "|\textit{target}|" |\\|"|[\textit{flags}]%
|\input{childdoc.def}\childdocforward[|\textit{main}|]{|\textit{dest}|}"|
\end{center}
%
Here \textit{target} is the name of the output file,
\textit{main} is the name of the main file
and \textit{dest} is the name of the main or child file to be processed
(all filenames without extensions).
The optional argument \textit{main} can be omitted
if \textit{main} matches \textit{dest}.
Optionally, compilation \textit{flags} can be defined via |\def| commands.
This command line makes the \TeX{} engine believe
it is compiling the file \textit{target}
whose content is specified as the latter parameter.
The provided code then forwards the processing to
\textit{main} or \textit{dest} as described in \secref{sec:forward}.

%%%%%%%%%%%%%%%%%%%%%%%%%%%%%%%%%%%%%%%%%%%%%%%%%%%%%%%%%%%%%%%%%%%%%%%%%%%%%%%%
\subsection{Include by Input}
\label{sec:input}

Including child documents by |\include| has some restrictions by design.
Most notably, the content of a child document always occupies
its own set of pages; pages cannot be shared between child documents.
Usually, this behaviour makes perfect sense
because each child document contain an essential part of the document.
However, in some situations it may be desirable to compose
a document from a collection of parts
without having mandatory page breaks between then.
For this case, the package
provides a mechanism to include parts
by |\input| which can also be processed individually.
However, by construction this mechanism
requires manual handling of the content to be output.

%%%%%%%%%%%%%%%%%%%%%%%%%%%%%%%%%%%%%%%%
\DescribeMacro{\ifchilddocmanual}
The main file should be prepared as usual, see \secref{sec:include}.
However, the document body must make a distinction
between processing of an individual part and of the main document, e.g.:
%
\begin{center}
\begin{tabular}{l}
|\ifchilddocmanual|\\
|\input{\childdocname}|\\
|\||else|\\
\textit{document body with }|\input{|\textit{part}|}|\\
|\||fi|
\end{tabular}
\end{center}
%
The conditional |\ifchilddocmanual| is true whenever
a part to be included by |\input| is being compiled,
and the name of the part is stored in |\childdocname|.

%%%%%%%%%%%%%%%%%%%%%%%%%%%%%%%%%%%%%%%%
\DescribeMacro{\childdocby}
Each part to be included by |\input| should start with:
%
\begin{center}
\begin{tabular}{l}
|\input{childdoc.def}|\\
|\childdocby{|\textit{main}|}|\\
\end{tabular}
\end{center}
%
The directive |\childdocby| is similar to |\childdocof|
described in \secref{sec:include},
but the subsequent selection of content must be done manually.
To that end, both |\ifchilddoc| and |\ifchilddocmanual|
will be true upon processing of a part,
and the name of the part is stored in |\childdocname|.
Note that |\jobname| will be set to the filename of the current part
so that each part receives an individual |.aux| file
that does not interfere with the |.aux| file(s) of the main document.
This behaviour can be altered by the alternative form
|\childdocby[*]{|\textit{main}|}| (with a non-empty optional argument)
which uses the |.aux| file of the main document
by setting |\jobname| to \textit{main}.

%%%%%%%%%%%%%%%%%%%%%%%%%%%%%%%%%%%%%%%%%%%%%%%%%%%%%%%%%%%%%%%%%%%%%%%%%%%%%%%%
\subsection{Driver Development}
\label{sec:driver}

The \textsf{childdoc} mechanism can also be use for the development
of definition files such as \LaTeX{} styles or classes.
This case differs from the above setup with multiple parts
included by |\include| in that no |\includeonly| should be invoked.
This can be achieved by starting the include file
(before |\ProvidesPackage|) with:
%
\begin{center}
\begin{tabular}{l}
|\input{childdoc.def}|\\
|\childdocforward{|\textit{main}|}|\\
\end{tabular}
\end{center}
%
or alternatively with:
%
\begin{center}
\begin{tabular}{l}
|\input{childdoc.def}|\\
|\childdocby{|\textit{main}|}|\\
\end{tabular}
\end{center}
%
Both forms have slightly different effects as described above.
The main file is prepared as usual, see \secref{sec:include}.

%%%%%%%%%%%%%%%%%%%%%%%%%%%%%%%%%%%%%%%%%%%%%%%%%%%%%%%%%%%%%%%%%%%%%%%%%%%%%%%%
\subsection{Legacy Detection}
\label{sec:detection}

The directive |\childdocmain| in the main file can detect
whether the complete document or merely a child is to be compiled
even without using the directive |\childdocof|.
This method is deprecated because it is less robust
and there is no compelling reason to use it;
it is merely provided for backward compatibility
and it may be removed in future versions.

If the detection mechanism is to be used,
it is mandatory to correctly specify
the filename of the main file as the argument of |\childdocmain|:
%
\begin{center}
\begin{tabular}{l}
|\input{childdoc.def}|\\
|\childdocmain{|\textit{main}|}|\\
\end{tabular}
\end{center}
%
If |\jobname| does not match the argument \textit{main} of |\childdocmain|,
it is assumed that |\jobname| points to the child file to be compiled.
When using |\childdocmain| with the main file specified as argument,
it suffices to start a child file
with just |\input{|\textit{main}|}|
without loading of the package and using |\childdocof|.
If instead all processing is done
with the appropriate \textsf{childdoc} directives,
the argument of \textit{main} of |\childdocmain| can be empty.

An alternative version of the command line processing described
in \secref{sec:commandline} using the detection mechanism reads:
%
\begin{center}
|... -jobname "|\textit{target}|" "|[\textit{flags}]%
[|\def\jobname{|\textit{dest}|}|]|\input{|\textit{main}|}"|
\end{center}

%%%%%%%%%%%%%%%%%%%%%%%%%%%%%%%%%%%%%%%%%%%%%%%%%%%%%%%%%%%%%%%%%%%%%%%%%%%%%%%%
\subsection{Manual Code}
\label{sec:manual}

In case one cannot be certain whether the definitions file |childdoc.def|
is installed on the target \TeX{} distribution
and one prefers not to ship it,
it is conceivable to paste a few relevant commands into the sources.

To that end, drop all statements |\input{childdoc.def}|
and perform the replacements as outlined below.
Instead of |\childdocmain{|\textit{main}|}| add the following code
to the top of the main file:
%
\begin{center}
\begin{tabular}{l}
|\||ifdefined\childdocname\endinput\||fi\newif\ifchilddoc|\\
|\edef\childdocname{\scantokens\expandafter{\jobname\noexpand}}|\\
|\def\childdocmain{|\textit{main}|}\||ifx\childdocmain\childdocname\||else|\\
|\childdoctrue\includeonly{\childdocname}\let\jobname\childdocmain\||fi|\\
\end{tabular}
\end{center}
%
Instead of |\childdocof{|\textit{main}|}| just include the main file
at the top of each child file:
%
\begin{center}
|\input{|\textit{main}|}|
\end{center}
%
A simple redirection |\childdocforward{|\textit{dest}|}| is achieved by:
%
\begin{center}
|\def\jobname{|\textit{dest}|}\input{\jobname}|
\end{center}
%
The redirection with prefix
|\childdocforwardprefix[|\textit{prefix}|]{|\textit{dest}|}|
is accomplished by:
%
\begin{center}
\begin{tabular}{l}
|{\edef\jobname{\scantokens\expandafter{\jobname\noexpand}}|\\
|\def\redirectjob |\textit{prefix}|#1~~~{\gdef\jobname{|\textit{dest}|#1}}|\\
|\expandafter\redirectjob\jobname~~~}\input{\jobname}|
\end{tabular}
\end{center}

In an alternative approach,
child documents can be compiled by a specific command line
without additional code or specific definitions:
%
\begin{center}
|... -jobname "|\textit{target}|" "|[\textit{flags}]%
|\includeonly{|\textit{dest}|}\input{|\textit{main}|}"|
\end{center}
%

%%%%%%%%%%%%%%%%%%%%%%%%%%%%%%%%%%%%%%%%%%%%%%%%%%%%%%%%%%%%%%%%%%%%%%%%%%%%%%%%
%%%%%%%%%%%%%%%%%%%%%%%%%%%%%%%%%%%%%%%%%%%%%%%%%%%%%%%%%%%%%%%%%%%%%%%%%%%%%%%%
\section{Information}

%%%%%%%%%%%%%%%%%%%%%%%%%%%%%%%%%%%%%%%%%%%%%%%%%%%%%%%%%%%%%%%%%%%%%%%%%%%%%%%%
\subsection{Copyright}

Copyright \copyright{} 2017--2018 Niklas Beisert

This work may be distributed and/or modified under the
conditions of the \LaTeX{} Project Public License, either version 1.3
of this license or (at your option) any later version.
The latest version of this license is in
  \url{http://www.latex-project.org/lppl.txt}
and version 1.3 or later is part of all distributions of \LaTeX{}
version 2005/12/01 or later.

This work has the LPPL maintenance status `maintained'.

The Current Maintainer of this work is Niklas Beisert.

This work consists of the files |README.txt|, |childdoc.ins| and |childdoc.dtx|
as well as the derived files |childdoc.def|, |cdocsamp.tex|
with |cdocsch1.tex|, |cdocsch2.tex|, |cdocspt3.tex|, |cdocspt4.tex|,
|cdocsdrf.tex|, |cdocsfn1.tex|, |cdocsfn2.tex|
as well as |childdoc.pdf|.

%%%%%%%%%%%%%%%%%%%%%%%%%%%%%%%%%%%%%%%%%%%%%%%%%%%%%%%%%%%%%%%%%%%%%%%%%%%%%%%%
\subsection{Files and Installation}

The package consists of the files:
%
\begin{center}
\begin{tabular}{ll}
    |README.txt|   & readme file \\
    |childdoc.ins| & installation file \\
    |childdoc.dtx| & source file \\
    |childdoc.def| & definition file \\
    |cdocsamp.tex| & sample main file \\
    |cdocsch1.tex| & sample include file \\
    |cdocsch2.tex| & sample include file \\
    |cdocspt3.tex| & sample part file \\
    |cdocspt4.tex| & sample part file \\
    |cdocsdrf.tex| & sample redirection file \\
    |cdocsfn1.tex| & sample redirection file \\
    |cdocsfn2.tex| & sample redirection file \\
    |childdoc.pdf| & manual
\end{tabular}
\end{center}
%
The distribution consists of the files
|README.txt|, |childdoc.ins| and |childdoc.dtx|.
%
\begin{itemize}
\item
Run (pdf)\LaTeX{} on |childdoc.dtx|
to compile the manual |childdoc.pdf| (this file).
\item
Run \LaTeX{} on |childdoc.ins| to create the definitions file |childdoc.def|
and the sample |cdocsamp.tex| with include files
|cdocsch1.tex|, |cdocsch2.tex|, |cdocspt3.tex|, |cdocspt4.tex|,
|cdocsdrf.tex|, |cdocsfn1.tex|, |cdocsfn2.tex|.
Then copy the file |childdoc.def| to an appropriate directory of your \LaTeX{}
distribution, e.g.\ \textit{texmf-root}|/tex/latex/childdoc|.
\end{itemize}

%%%%%%%%%%%%%%%%%%%%%%%%%%%%%%%%%%%%%%%%%%%%%%%%%%%%%%%%%%%%%%%%%%%%%%%%%%%%%%%%
\subsection{Related CTAN Packages}

There are several other packages which offer a similar functionality:
%
\begin{itemize}
\item
The packages
\href{http://ctan.org/pkg/docmute}{\textsf{docmute}},
\href{http://ctan.org/pkg/includex}{\textsf{includex}} and
\href{http://ctan.org/pkg/standalone}{\textsf{standalone}}
provide commands to include only the document body of
a child file thus allowing both files to be compiled individually.
\item
The packages \href{http://ctan.org/pkg/subdocs}{\textsf{subdocs}}
and \href{http://ctan.org/pkg/subfiles}{\textsf{subfiles}}
provide structures in which the main and child documents can be
encapsulated and allowing them to be compiled individually.
The inclusion mechanism is different from the conventional |\include|.
\item
The package \href{http://ctan.org/pkg/combine}{\textsf{combine}}
is an elaborate solution to combine several documents into one.
\end{itemize}
%
See also the CTAN topic \href{http://ctan.org/topic/subdocs}{\textsf{subdocs}}
for further related packages.
The present package differs from the above solutions in that
a document structure constructed with the conventional |\include| mechanism
just needs two extra commands at the top of every file
such that all constituent files can be compiled individually.

%%%%%%%%%%%%%%%%%%%%%%%%%%%%%%%%%%%%%%%%%%%%%%%%%%%%%%%%%%%%%%%%%%%%%%%%%%%%%%%%
%\subsection{Feature Suggestions}
%
%The following is a list of features which may be useful for future
%versions of this package:
%%
%\begin{itemize}
%\item
%\ldots
%\end{itemize}

%%%%%%%%%%%%%%%%%%%%%%%%%%%%%%%%%%%%%%%%%%%%%%%%%%%%%%%%%%%%%%%%%%%%%%%%%%%%%%%%
\subsection{Revision History}

%%%%%%%%%%%%%%%%%%%%%%%%%%%%%%%%%%%%%%%%
\paragraph{v2.0:} 2018/12/30

\begin{itemize}
\item
immediate forward processing
\item
added |\childdocby| mechanism
\item
manual restructured
\end{itemize}

%%%%%%%%%%%%%%%%%%%%%%%%%%%%%%%%%%%%%%%%
\paragraph{v1.6:} 2018/01/17

\begin{itemize}
\item
application for development of include files
\item
corrections to manual
\end{itemize}

%%%%%%%%%%%%%%%%%%%%%%%%%%%%%%%%%%%%%%%%
\paragraph{v1.5:} 2017/05/21

\begin{itemize}
\item
more complete structuring introduced
\item
|\childdocof| introduced
\item
|\childdoc| renamed to |\childdocmain|
\item
|\childredirect| renamed to |\childdocforward| and |\childdocforwardprefix|
and functionality expanded
\end{itemize}

%%%%%%%%%%%%%%%%%%%%%%%%%%%%%%%%%%%%%%%%
\paragraph{v1.0:} 2017/04/27

\begin{itemize}
\item
manual and install package
\item
first version published on CTAN
\end{itemize}

%%%%%%%%%%%%%%%%%%%%%%%%%%%%%%%%%%%%%%%%
\paragraph{v0.6:} 2017/04/26

\begin{itemize}
\item
redirection mechanism added
\end{itemize}

%%%%%%%%%%%%%%%%%%%%%%%%%%%%%%%%%%%%%%%%
\paragraph{v0.5:} 2017/04/26

\begin{itemize}
\item
functionality in definition file
\end{itemize}


%%%%%%%%%%%%%%%%%%%%%%%%%%%%%%%%%%%%%%%%%%%%%%%%%%%%%%%%%%%%%%%%%%%%%%%%%%%%%%%%
%%%%%%%%%%%%%%%%%%%%%%%%%%%%%%%%%%%%%%%%%%%%%%%%%%%%%%%%%%%%%%%%%%%%%%%%%%%%%%%%
%%%%%%%%%%%%%%%%%%%%%%%%%%%%%%%%%%%%%%%%%%%%%%%%%%%%%%%%%%%%%%%%%%%%%%%%%%%%%%%%
\appendix

\settowidth\MacroIndent{\rmfamily\scriptsize 000\ }

 \DocInput{childdoc.dtx}

\end{document}
%</driver>
% \fi
%
% %%%%%%%%%%%%%%%%%%%%%%%%%%%%%%%%%%%%%%%%%%%%%%%%%%%%%%%%%%%%%%%%%%%%%%%%%%%%%%
% %%%%%%%%%%%%%%%%%%%%%%%%%%%%%%%%%%%%%%%%%%%%%%%%%%%%%%%%%%%%%%%%%%%%%%%%%%%%%%
% \section{Sample}
%\iffalse
%<*samplemain>
%\fi
%
% The following presents a sample document
% with two chapters, two parts, a title page,
% a compile flag as well as three forwarding files to set the flag.
% It consists of eight |.tex| files:
% \begin{center}
% \begin{tabular}{ll}
% |cdocsamp.tex|&main file\\
% |cdocsch1.tex|&include file for chapter 1\\
% |cdocsch2.tex|&include file for chapter 2\\
% |cdocspt3.tex|&include file for part 3\\
% |cdocspt4.tex|&include file for part 4\\
% |cdocsdrf.tex|&forwarding file for main file in draft mode\\
% |cdocsfi1.tex|&forwarding file for final version of chapter 1\\
% |cdocsfi2.tex|&forwarding file for final version of chapter 2\\
% \end{tabular}
% \end{center}
% Each of the eight files can be compiled directly by the \LaTeX{} compiler.
%
% %%%%%%%%%%%%%%%%%%%%%%%%%%%%%%%%%%%%%%
% \paragraph{Main File.}
%
% The main file is called |cdocsamp.tex|.
%
% Load the \textsf{childdoc} definitions and
% declare the filename for the main document:
%    \begin{macrocode}
\input{childdoc.def}
\childdocmain{}
%    \end{macrocode}

% Optional override for |\version| flag:
%    \begin{macrocode}
%%\ifchilddoc\else\providecommand{\version}{draft}\fi
%    \end{macrocode}

% Define the default values for the |\version| flag
% (|final| for the main file and |draft| for childs):
%    \begin{macrocode}
\ifchilddoc
\providecommand{\version}{draft}
\else
\providecommand{\version}{final}
\fi
%    \end{macrocode}

% Load the standard document class:
%    \begin{macrocode}
\documentclass[12pt]{article}
%    \end{macrocode}

% Start the document body:
%    \begin{macrocode}
\begin{document}
%    \end{macrocode}

% Declare a title page.
% Print title, part of document being processed and version flag:
%    \begin{macrocode}
\addtocounter{page}{-1}
\begin{center}
{\LARGE\bfseries{}childdoc example\par}
\vspace{1cm}
\ifchilddoc
\ifchilddocmanual part\else chapter\fi:
`\childdocname' of `\childdocjob'\par
\else
main document: `\childdocjob'\par
\fi
version: \version\par
\end{center}
\newpage
%    \end{macrocode}

% Manually include selected file,
% otherwise process as usual:
%    \begin{macrocode}
\ifchilddocmanual
\section*{part `\childdocname'}
\input{\childdocname}
\else
%    \end{macrocode}

% Include the two chapters:
%    \begin{macrocode}
\include{cdocsch1}
\include{cdocsch2}
%    \end{macrocode}

% Include the two parts unless only chapters should be displayed:
%    \begin{macrocode}
\ifchilddoc\else
\section{part three}
\input{cdocspt3}
\section{part four}
\input{cdocspt4}
\fi
%    \end{macrocode}

% Process as usual until here:
%    \begin{macrocode}
\fi
%    \end{macrocode}

% End of document body:
%    \begin{macrocode}
\end{document}
%    \end{macrocode}
%\iffalse
%</samplemain>
%\fi
%
% %%%%%%%%%%%%%%%%%%%%%%%%%%%%%%%%%%%%%%
% \paragraph{Chapter Include Files.}
%
% The include files are called |cdocsch1.tex| and |cdocsch2.tex|.
%
%\iffalse
%<*samplechap1|samplechap2>
%\fi

% Optional override for |\version| flag:
%    \begin{macrocode}
%%\providecommand{\version}{final}
%    \end{macrocode}

% Include the main document:
%    \begin{macrocode}
\input{childdoc.def}
\childdocof{cdocsamp}
%    \end{macrocode}

%\iffalse
%</samplechap1|samplechap2>
%\fi
%
%\iffalse
%<*samplechap1>
%\fi
% Some text for chapter 1:
%    \begin{macrocode}
\section{one}
some text in chapter one
%    \end{macrocode}

%\iffalse
%</samplechap1>
%\fi
% Some text for chapter 2:
%\iffalse
%<*samplechap2>
%\fi
%    \begin{macrocode}
\section{two}
more text in chapter two
%    \end{macrocode}

%\iffalse
%</samplechap2>
%\fi
%
% %%%%%%%%%%%%%%%%%%%%%%%%%%%%%%%%%%%%%%
% \paragraph{Part Include Files.}
%
% The include files are called |cdocspt3.tex| and |cdocspt4.tex|.
%
%\iffalse
%<*samplepart3|samplepart4>
%\fi

% Optional override for |\version| flag:
%    \begin{macrocode}
%%\providecommand{\version}{final}
%    \end{macrocode}

% Include the main document:
%    \begin{macrocode}
\input{childdoc.def}
\childdocby{cdocsamp}
%    \end{macrocode}

%\iffalse
%</samplepart3|samplepart4>
%\fi
%
%\iffalse
%<*samplepart3>
%\fi
% Some text for part 3:
%    \begin{macrocode}
some text in part three
%    \end{macrocode}

%\iffalse
%</samplepart3>
%\fi
% Some text for part 4:
%\iffalse
%<*samplepart4>
%\fi
%    \begin{macrocode}
more text in part four
%    \end{macrocode}

%\iffalse
%</samplepart4>
%\fi
%
% %%%%%%%%%%%%%%%%%%%%%%%%%%%%%%%%%%%%%%
% \paragraph{Forwarding for a Complete Draft.}
%
% The following forwarding file |cdocsdrf.tex|
% compiles the main document in draft mode:
%\iffalse
%<*sampledraft>
%\fi
%    \begin{macrocode}
\def\version{draft}
\input{childdoc.def}
\childdocforward{cdocsamp}
%    \end{macrocode}

%\iffalse
%</sampledraft>
%\fi
%
% %%%%%%%%%%%%%%%%%%%%%%%%%%%%%%%%%%%%%%
% \paragraph{Forwarding for Final Version of the Chapters.}
%
% The following forwarding files |cdocsfn1.tex| and |cdocsfn2.tex|
% (with identical content)
% compile the final versions of the child documents
% |cdocsch1.tex| and |cdocsch2.tex|, respectively:
%\iffalse
%<*samplefinal>
%\fi
%    \begin{macrocode}
\def\version{final}
\input{childdoc.def}
\childdocforwardprefix[cdocsamp]{cdocsfn}{cdocsch}
%    \end{macrocode}

%\iffalse
%</samplefinal>
%\fi
%
% %%%%%%%%%%%%%%%%%%%%%%%%%%%%%%%%%%%%%%
% \paragraph{Command Line Processing.}
%
% The following three command lines generate the output files
% |cdocscld|, |cdocscl1| and |cdocscl2|
% which should be identical to
% |cdocsdrf|, |cdocsch1| and |cdocsfn2|, respectively:
% \begin{center}
% \begin{tabular}{l}
% |latex -jobname cdocscld \|\\
% |  "\def\version{draft}\input{childdoc.def}\childdocforward{cdocsamp}"|\\
% |latex -jobname cdocscl1 \|\\
% |  "\input{childdoc.def}\childdocforward[cdocsamp]{cdocsch1}"|\\
% |latex -jobname cdocscl2 \|\\
% |  "\def\version{final}\input{childdoc.def}\childdocforward{cdocsch2}"|
% \end{tabular}
% \end{center}
% Note that the trailing backslash on each first line
% merely continues the input to the second line
% (for convenient cut ant paste).
% Furthermore, the command |latex| can be replaced by any
% of its alternative versions such as |pdflatex|.
%
% %%%%%%%%%%%%%%%%%%%%%%%%%%%%%%%%%%%%%%%%%%%%%%%%%%%%%%%%%%%%%%%%%%%%%%%%%%%%%%
% %%%%%%%%%%%%%%%%%%%%%%%%%%%%%%%%%%%%%%%%%%%%%%%%%%%%%%%%%%%%%%%%%%%%%%%%%%%%%%
% \section{Implementation}
%\iffalse
%<*package>
%\fi
%
% This section describes the definitions file |childdoc.def|.

% The definitions cannot be loaded using |\usepackage| or |\RequirePackage|
% which has a mechanism to prevent loading a style file more than once.
% When loading the definitions by means of |\input|
% multiple instances have to be prevented manually:
%\iffalse
%This code needs to be before the `\ProvidesFile' directive
%which is defined at the beginning of this file.
%Therefore it is also placed there and commented out here.
%</package>
%<*discard>
%\fi
%    \begin{macrocode}
\ifdefined\childdocmain\endinput\fi
%    \end{macrocode}
%\iffalse
%</discard>
%<*package>
%\fi
%
% \macro{\ifchilddoc}
% \macro{\ifchilddocmanual}
% The conditional |\ifchilddoc| tells whether a
% child (true) or main (false) document is being compiled.
% The conditional |\ifchilddocmanual| tells whether
% the |\includeonly| mechanism is used (false) or
% the selection of child files must be performed manually (true).
% The definitions initialise to false:
%    \begin{macrocode}
\newif\ifchilddoc
\newif\ifchilddocmanual
%    \end{macrocode}

% \macro{\childdocname}
% \macro{\childdocjob}
% The macro |\childdocname| stores the name of the main document
% to be compiled. The macro |\childdocjob| stores the name of
% the document on which the \LaTeX{} compiler was originally invoked.
% The content of |\jobname| cannot be compared
% to filenames specified in the source due to different catcodes.
% The following code rescans |\jobname|, stores the result
% in |\childdocname| and saves a copy in |\childdocjob|:
%    \begin{macrocode}
\edef\childdocname{\scantokens\expandafter{\jobname\noexpand}}
\let\childdocjob\childdocname
%    \end{macrocode}

% \macro{\childdocdisable}
% The macro |\childdocdisable| prevents the main file
% from being processed more than once.
% At this stage, the main document command |\childdocmain|
% is assumed to be called once again where it should do nothing.
% Any subsequent call to it should prevent
% a secondary processing of the main document
% It overwrites the forwarding commands
% |\childdocof| and |\childdocforward|
% with empty macros to prevent further inclusions of the main document:
%    \begin{macrocode}
\newcommand{\childdocdisable}
{
  \renewcommand{\childdocmain}[1]{\renewcommand{\childdocmain}[1]{\endinput}}
  \renewcommand{\childdocof}[1]{}
  \renewcommand{\childdocby}[2][]{}
  \renewcommand{\childdocforward}[2][]{}
  \renewcommand{\childdocdisable}{}
}
%    \end{macrocode}

% \macro{\childdocmain}
% The macro |\childdocmain| is to be called at the top of the main file
% with nothing or the main filename (without extension) as argument.
% First, it breaks loops.
% If the argument is not empty and does not match |\childdocname|
% (which is set by the first inclusion of |childdoc.def|),
% |\ifchilddoc| is set to true, |\includeonly| is applied to the child file
% and |\jobname| is set to the main file
% (for proper handling of |.aux| files):
%    \begin{macrocode}
\newcommand{\childdocmain}[1]
{
  \childdocdisable\childdocmain{}
  \if?#1?\else
    \begingroup
      \def\childdoctmp{#1}
      \ifx\childdoctmp\childdocname
        \def\childdoctmp{}
      \else
        \def\childdoctmp
        {
          \childdoctrue
          \includeonly{\childdocname}
          \def\childdocjob{#1}
          \def\jobname{#1}
        }
      \fi
      \expandafter
    \endgroup
    \childdoctmp
  \fi
}
%    \end{macrocode}

% \macro{\childdocof}
% The command |\childdocof| redirects
% compilation to the main file |#1|.
%    \begin{macrocode}
\newcommand{\childdocof}[1]
{
  \childdocdisable
  \childdoctrue
  \includeonly{\childdocname}
  \def\jobname{#1}
  \def\childdocjob{#1}
  \input{#1}
}
%    \end{macrocode}

% \macro{\childdocby}
% The command |\childdocby| ....
%    \begin{macrocode}
\newcommand{\childdocby}[2][]
{
  \childdocdisable
  \childdoctrue
  \childdocmanualtrue
  \if?#1?\else
    \def\jobname{#2}
  \fi
  \def\childdocjob{#2}
  \input{#2}
  \endinput
}
%    \end{macrocode}

% \macro{\childdocforward}
% The command |\childdocforward| redirects
% compilation to the main file or
% (if the optional argument is given) a child file.
% Parameters are set as if the main file
% or a child file starting with |\childdocof| was compiled.
% Then compilation is handed over to the main file:
%    \begin{macrocode}
\newcommand{\childdocforward}[2][]
{
  \begingroup
    \if?#1?
      \def\childdoctmp
      {
        \def\childdocname{#2}
        \def\childdocjob{#2}
        \def\jobname{#2}
        \input{#2}
        \endinput
      }
    \else
      \def\childdoctmp
      {
        \childdocdisable
        \def\childdocname{#2}
        \childdoctrue
        \includeonly{#2}
        \def\childdocjob{#1}
        \def\jobname{#1}
        \input{#1}
        \endinput
      }
    \fi
    \expandafter
  \endgroup
  \childdoctmp
}
%    \end{macrocode}

% \macro{\childdocforwardprefix}
% The command |\childdocforwardprefix| redirects
% compilation to the main or a child file by means of a pattern.
% The prefix |#1| in the current filename is replaced by |#2|
% and the suffix of the current filename is kept
% (it is assumed that the filename does not contain the substring `|~~~|'
% which is used as a delimiter).
% Compilation is handed over to the new file by |\childdocforward|:
%    \begin{macrocode}
\newcommand{\childdocforwardprefix}[3][]
{
  \begingroup
    \def\childdocextract #2##1~~~{\def\childdoctmp{\childdocforward[#1]{#3##1}}}
    \expandafter\childdocextract\childdocname~~~
    \expandafter
  \endgroup
  \childdoctmp
}
%    \end{macrocode}

% \macro{\childdoc}
% The deprecated macro |\childdoc| is a legacy version of |\childdocmain|:
%    \begin{macrocode}
\newcommand{\childdoc}{\childdocmain}
%    \end{macrocode}

% \macro{\childdocredirect}
% The deprecated macro |\childdocredirect| is a legacy version
% of |\childdocforward| and |\childdocforwardprefix|:
%    \begin{macrocode}
\newcommand{\childdocredirect}[2][]
{
  \begingroup
    \if?#1?
      \def\childdoctmp{\childdocforward{#2}}
    \else
      \def\childdoctmp{\childdocforwardprefix{#1}{#2}}
    \fi
    \expandafter
  \endgroup
  \childdoctmp
}
%    \end{macrocode}

%\iffalse
%</package>
%\fi
%
\endinput
|\\
|\childdocmain{|\textit{main}|}|\\
\end{tabular}
\end{center}
%
If |\jobname| does not match the argument \textit{main} of |\childdocmain|,
it is assumed that |\jobname| points to the child file to be compiled.
When using |\childdocmain| with the main file specified as argument,
it suffices to start a child file
with just |\input{|\textit{main}|}|
without loading of the package and using |\childdocof|.
If instead all processing is done
with the appropriate \textsf{childdoc} directives,
the argument of \textit{main} of |\childdocmain| can be empty.

An alternative version of the command line processing described
in \secref{sec:commandline} using the detection mechanism reads:
%
\begin{center}
|... -jobname "|\textit{target}|" "|[\textit{flags}]%
[|\def\jobname{|\textit{dest}|}|]|\input{|\textit{main}|}"|
\end{center}

%%%%%%%%%%%%%%%%%%%%%%%%%%%%%%%%%%%%%%%%%%%%%%%%%%%%%%%%%%%%%%%%%%%%%%%%%%%%%%%%
\subsection{Manual Code}
\label{sec:manual}

In case one cannot be certain whether the definitions file |childdoc.def|
is installed on the target \TeX{} distribution
and one prefers not to ship it,
it is conceivable to paste a few relevant commands into the sources.

To that end, drop all statements |% \iffalse
%
% childdoc.dtx Copyright (C) 2017-2018 Niklas Beisert
%
% This work may be distributed and/or modified under the
% conditions of the LaTeX Project Public License, either version 1.3
% of this license or (at your option) any later version.
% The latest version of this license is in
%   http://www.latex-project.org/lppl.txt
% and version 1.3 or later is part of all distributions of LaTeX
% version 2005/12/01 or later.
%
% This work has the LPPL maintenance status `maintained'.
%
% The Current Maintainer of this work is Niklas Beisert.
%
% This work consists of the files childdoc.dtx and childdoc.ins
% and the derived files childdoc.def and cdocsamp.tex with
% cdocsch1.tex, cdocsch2.tex, cdocsdrf.tex, cdocsfn1.tex, cdocsfn2.tex.
%
%<package>\ifdefined\childdocmain\endinput\fi
%<package>\ProvidesFile{childdoc.def}[2018/12/30 v2.0 child document driver]
%<samplemain>\ProvidesFile{cdocsamp.tex}[2018/12/30 v2.0 sample for childdoc]
%<*driver>
%\ProvidesFile{childdoc.drv}[2018/12/30 v2.0 childdoc reference manual file]
\PassOptionsToClass{10pt,a4paper}{article}
\documentclass{ltxdoc}

\usepackage[margin=35mm]{geometry}
\usepackage{hyperref}
\usepackage{hyperxmp}
\usepackage[usenames]{color}

\hypersetup{colorlinks=true}
\hypersetup{pdfstartview=FitH}
\hypersetup{pdfpagemode=UseNone}
\hypersetup{pdfsource={}}
\hypersetup{pdflang={en-UK}}
\hypersetup{pdfcopyright={Copyright 2017-2018 Niklas Beisert.
  This work may be distributed and/or modified under the
  conditions of the LaTeX Project Public License, either version 1.3
  of this license or (at your option) any later version.}}
\hypersetup{pdflicenseurl={http://www.latex-project.org/lppl.txt}}
\hypersetup{pdfcontactaddress={ETH Zurich, ITP, HIT K,
  Wolfgang-Pauli-Strasse 27}}
\hypersetup{pdfcontactpostcode={8093}}
\hypersetup{pdfcontactcity={Zurich}}
\hypersetup{pdfcontactcountry={Switzerland}}
\hypersetup{pdfcontactemail={nbeisert@itp.phys.ethz.ch}}
\hypersetup{pdfcontacturl={http://people.phys.ethz.ch/\xmptilde nbeisert/}}

\newcommand{\secref}[1]{\hyperref[#1]{section \ref*{#1}}}

\parskip1ex
\parindent0pt
\let\olditemize\itemize
\def\itemize{\olditemize\parskip0pt}

\begin{document}

\title{The \textsf{childdoc} Package}
\hypersetup{pdftitle={The childdoc Package}}
\author{Niklas Beisert\\[2ex]
  Institut f\"ur Theoretische Physik\\
  Eidgen\"ossische Technische Hochschule Z\"urich\\
  Wolfgang-Pauli-Strasse 27, 8093 Z\"urich, Switzerland\\[1ex]
  \href{mailto:nbeisert@itp.phys.ethz.ch}
  {\texttt{nbeisert@itp.phys.ethz.ch}}}
\hypersetup{pdfauthor={Niklas Beisert}}
\hypersetup{pdfsubject={Manual for the LaTeX2e Package childdoc}}
\date{30 December 2018, \textsf{v2.0}}
\maketitle

\begin{abstract}\noindent
\textsf{childdoc} is a \LaTeXe{} package
that enables the direct compilation
of document sections included by |\include|
to individual files.
\end{abstract}

\begingroup
\parskip0ex
\tableofcontents
\endgroup

%%%%%%%%%%%%%%%%%%%%%%%%%%%%%%%%%%%%%%%%%%%%%%%%%%%%%%%%%%%%%%%%%%%%%%%%%%%%%%%%
%%%%%%%%%%%%%%%%%%%%%%%%%%%%%%%%%%%%%%%%%%%%%%%%%%%%%%%%%%%%%%%%%%%%%%%%%%%%%%%%
\section{Introduction}

\LaTeX{} provides a mechanism to structure a large document (such as a book)
into a main file and several child files (containing the chapters)
using the |\include| command.
This mechanism is beneficial for documents
which span hundreds of pages in order to
make the source file(s) more manageable.
Moreover, compilation can be restricted to
selected child files by means of the |\includeonly| command.
The latter feature can be used to reduce the compilation time while editing
(this was significantly more useful in the earlier days of \LaTeX{})
or to generate a smaller document which is easier to navigate.
Another application of |\includeonly| is to generate
documents consisting of selected parts of the complete document.

However, there are a few drawbacks of the plain |\include| mechanism:
\begin{itemize}
\item
The child files cannot be compiled on their own,
they can only be compiled via the main file.
A naive editing environment
(such as a text editor with an option
to have the current file processed by \LaTeX)
may require one to switch to the main file before compiling;
attempting to compile the child file produces errors.
\item
The main file must be modified (each time)
to adjust the |\includeonly| command
to the present needs. This easily leaves the main file in a messy state.
\item
The generated document will always carry the filename
of the main document. This is inconvenient if
several child files are to be compiled and
to be kept for distribution.
\end{itemize}

The present package provides a simple interface
to make child files individually compilable by \LaTeX{}.
Compiling a child file then has the same effect as compiling
the main file with an |\includeonly| command
to select the appropriate child.
Moreover the generated document will carry the name of the child
rather than the main file.
This resolves all three above issues.

This feature is meant to make the editing of books,
thesis documents and lecture notes somewhat more convenient.
However, the package can also be used efficiently for
composing a series of documents (such as exercise sheets)
which are typically distributed individually.
It then assists the author in generating the individual documents
(potentially in different versions)
as well as a document containing the collected series.
Another application is in developing style files
or other kinds of included material
where compilation of the style file could redirect
to a sample or test file.

%%%%%%%%%%%%%%%%%%%%%%%%%%%%%%%%%%%%%%%%%%%%%%%%%%%%%%%%%%%%%%%%%%%%%%%%%%%%%%%%
%%%%%%%%%%%%%%%%%%%%%%%%%%%%%%%%%%%%%%%%%%%%%%%%%%%%%%%%%%%%%%%%%%%%%%%%%%%%%%%%
\section{Usage}

First of all, the package \textsf{childdoc} is \emph{not} a standard
\LaTeXe{} |.sty| style file! Therefore it needs to be invoked in
a non-standard way.

%%%%%%%%%%%%%%%%%%%%%%%%%%%%%%%%%%%%%%%%%%%%%%%%%%%%%%%%%%%%%%%%%%%%%%%%%%%%%%%%
\subsection{Included Files}
\label{sec:include}

%%%%%%%%%%%%%%%%%%%%%%%%%%%%%%%%%%%%%%%%
\DescribeMacro{\childdocmain}
To use the package, add the commands
\begin{center}
\begin{tabular}{l}
|\input{childdoc.def}|\\
|\childdocmain{}|\\
\end{tabular}
\end{center}
at the very top of the main \LaTeX{} file,
in particular \emph{before} the |\documentclass| statement!
The argument of |\childdocmain| should be left empty
(but it must be present).

%%%%%%%%%%%%%%%%%%%%%%%%%%%%%%%%%%%%%%%%
\DescribeMacro{\childdocof}
Furthermore, add the commands
\begin{center}
\begin{tabular}{l}
|\input{childdoc.def}|\\
|\childdocof{|\textit{main}|}|\\
\end{tabular}
\end{center}
at the top of every child file \textit{child}
which is included by |\include{|\textit{child}|}|
from within the main file
(or at least for those files to be compiled individually).
The argument \textit{main} must be the filename of the main file.

There are a couple of
considerations in setting up the main and child documents:

%%%%%%%%%%%%%%%%%%%%%%%%%%%%%%%%%%%%%%%%
\paragraph{Restrictions.}

Please note the following restrictions:
\begin{itemize}
\item
|\childdocmain| must be called with one argument \textit{main}
to ensure compatibility with earlier version of the package.
It must either be empty (|\childdocmain{}|)
or precisely match the filename of the main file in which it is specified.
See \secref{sec:detection} for further information.
\item
The filename \textit{main} must be specified without the |.tex| extension.
\item
The filename \textit{main} is case sensitive
(even in case-insensitive file systems)
due to internal string comparison.
\item
The argument \textit{main} should be fully expanded, it cannot be a macro.
\item
Subdirectories and special characters should be avoided in filenames.
\item
The command |\childdocmain{|\textit{main}|}| must be followed by a whitespace.
It should not be followed immediately by another command
or by a comment mark `|%|'.
This is because the \TeX{} parser reads the token immediately following
the argument of |\childdocmain| and puts it
at the beginning of every child section;
however, a white\-space is ignored.
\end{itemize}

%%%%%%%%%%%%%%%%%%%%%%%%%%%%%%%%%%%%%%%%
\paragraph{Content of Main File.}

It is advisable to place all content in the child files included by |\include|.
Any output contained in the main file will appear in all child documents
unless suppressed manually;
it cannot be suppressed automatically by the |\includeonly| directive
and thus should normally be avoided.
A method to include some content in the main file
by means of conditional processing is described in \secref{sec:conditional}.

%%%%%%%%%%%%%%%%%%%%%%%%%%%%%%%%%%%%%%%%
\paragraph{Page Numbering.}

When only a part of the document is compiled,
the appropriate numbering of pages
(as well as other status parameters)
is determined from the |.aux| files.
The latter contain information from previous passes.
However this information needs to propagate through
all intermediate child documents.
Therefore the page numbering in child documents may well
be inconsistent until the complete document is compiled at least once.

A useful (if unconventional) way to always ensure a consistent
page numbering is to restart the numbering in each child document
and denote the pages by `\textit{child}|.|\textit{page}'
where \textit{child} represents the chapter/section number of the child file.
This can be achieved by the command
|\numberwithin{page}{|\textit{child}|}|
of the \textsf{amsmath} package
where \textit{child} can be |chapter| or |section|
depending on the chosen structuring.
Alternatively, one can modify the macro |\thepage| appropriately
and reset the counter |page| at the start of each child file.

%%%%%%%%%%%%%%%%%%%%%%%%%%%%%%%%%%%%%%%%%%%%%%%%%%%%%%%%%%%%%%%%%%%%%%%%%%%%%%%%
\subsection{Conditional Processing}
\label{sec:conditional}

The package provides a mechanism to compile different versions
of a document. To customise the versions further some conditional processing
can come in handy to distinguish which version is being compiled.
The package provides two macros to describe the compilation context:

%%%%%%%%%%%%%%%%%%%%%%%%%%%%%%%%%%%%%%%%
\DescribeMacro{\ifchilddoc}
The conditional |\ifchilddoc| distinguishes between the compilation of
child documents and the main document:
%
\begin{center}
|\ifchilddoc |\textit{child-code}| |[|\||else |\textit{main-code}]| \||fi|
\end{center}

%%%%%%%%%%%%%%%%%%%%%%%%%%%%%%%%%%%%%%%%
\DescribeMacro{\childdocname}
\DescribeMacro{\childdocjob}
The macro |\childdocname| contains the filename (without extension)
of the main or child file being processed.
Note that |\childdocjob| will always contain the name of the main file.

%%%%%%%%%%%%%%%%%%%%%%%%%%%%%%%%%%%%%%%%
\paragraph{Title Page.}

Conditional processing can be used to include a title or banner page
in the main document when proper precautions are taken.
Importantly, the code in the main file should ensure that the page counter
(as well as other status parameters which are stored in the |.aux| files)
takes the same value after the conditional processing.
Otherwise the page numbers may take divergent values
depending on which part is compiled.

For example, a title page could be declared by:
%
\begin{center}
\begin{tabular}{l}
|\ifchilddoc\||else|\\
|\addtocounter{page}{-1}|\\
\textit{code for title page}\\
|\newpage|\\
|\||fi|
\end{tabular}
\end{center}
%
A banner page for the child documents can be generated by:
%
\begin{center}
\begin{tabular}{l}
|\ifchilddoc|\\
|\addtocounter{page}{-1}|\\
\textit{code for banner page}\\
|\newpage|\\
|\||fi|
\end{tabular}
\end{center}
%
Here one could write a message such as:
\begin{center}
|This is the part \childdocname{} of \childdocjob{}.|
\end{center}

%%%%%%%%%%%%%%%%%%%%%%%%%%%%%%%%%%%%%%%%%%%%%%%%%%%%%%%%%%%%%%%%%%%%%%%%%%%%%%%%
\subsection{Flags}
\label{sec:flags}

The package makes it easy to generate different versions
of the main or child documents.
To this end compilation flags can be defined
and assigned different default values.
They will be particularly useful in conjunction
with the forwarding mechanism described in \secref{sec:forward}.

For example, it may be useful to have a flag |\version|
which can be set to |draft| or |final|.
The document source will contain some conditional code
depending on the value of |\version|.
Suppose further, the flag should default to |final| for the main file
and to |draft| for child files
which is a natural assignment for editing the document.
This is achieved by placing the following code
in the preamble of the main document
(below the |\childdocmain| directive):
%
\begin{center}
\begin{tabular}{l}
|\ifchilddoc|\\
|\providecommand{\version}{draft}|\\
|\||else|\\
|\providecommand{\version}{final}|\\
|\||fi|
\end{tabular}
\end{center}
%
The definition by |\providecommand| makes sure
that previous definitions are not overwritten.
Further statements |\providecommand{\version}{...}|
can thus be added before the above code to override it.

For the main file, one might add a line
(between |\childdocmain| and the above block)
%
\begin{center}
|%\ifchilddoc\||else\providecommand{\version}{draft}\||fi|
\end{center}
%
which can be uncommented to produce a draft version.
Likewise one can add a line to the very top of a child file
(above the |\childdocof{|\textit{main}|}| directive)
%
\begin{center}
|%\providecommand{\version}{final}|
\end{center}
%
which can be uncommented to produce the final version of this child document.

%%%%%%%%%%%%%%%%%%%%%%%%%%%%%%%%%%%%%%%%%%%%%%%%%%%%%%%%%%%%%%%%%%%%%%%%%%%%%%%%
\subsection{Forwarding}
\label{sec:forward}

Different versions of the main or child documents
using compilation flags as described in \secref{sec:flags}
can be (permanently) stored in different files
for convenient compilation, viewing and distribution.
To this end, the package defines a command
to pass on compilation to a different file:

%%%%%%%%%%%%%%%%%%%%%%%%%%%%%%%%%%%%%%%%
\DescribeMacro{\childdocforward}
The command |\childdocforward| redirects processing to
another source file:
%
\begin{center}
\begin{tabular}{l}
|\input{childdoc.def}|\\
|\childdocforward[|\textit{main}|]{|\textit{dest}|}|\\
\end{tabular}
\end{center}
%
The argument \textit{dest} is the destination file
(without extension).
It should be the main file or one of the child files.
Note that further \textsf{childdoc} directives
such as |\childdocof| and |\childdocforward|
in the indicated file will be processed in this form.
The optional argument \textit{main}
passes on directly to the main file \textit{main}
while pretending to compile the child \textit{dest}.
This form behaves as if \textit{dest}
issues |\childdocof{|\textit{main}|}| right away,
and no further \textsf{childdoc} directives will be processed.

%%%%%%%%%%%%%%%%%%%%%%%%%%%%%%%%%%%%%%%%
\DescribeMacro{\...prefix}
In the alternative form |\childdocforwardprefix|,
%
\begin{center}
\begin{tabular}{l}
|\input{childdoc.def}|\\
|\childdocforwardprefix[|\textit{main}|]{|\textit{prefix}|}{|\textit{dest}|}|
\end{tabular}
\end{center}
%
the destination file is determined by a pattern
depending on the current file:
To make this work, the current file must be called
`{\textit{prefix}\hspace{0.2em}\textit{suffix}}'
with \textit{prefix} matching precisely the argument.
Processing is then passed on to the file
`{\textit{dest}\hspace{0.2em}\textit{suffix}}'.
Surely, the same effect is achieved by
directly specifying the
argument `{\textit{dest}\hspace{0.2em}\textit{suffix}}'
in the first form.
However, that requires to set up a different file
for each child. With the alternative form of the command
all these files can have exactly the same content
which simplifies setting them up and maintaining them.

For example, the following file |draft.tex|
with a compilation flag |\version| as described in \secref{sec:flags}
compiles the main document as a draft:
%
\begin{center}
\begin{tabular}{l}
|\def\version{draft}|\\
|\input{childdoc.def}|\\
|\childdocforward{|\textit{main}|}|
\end{tabular}
\end{center}
%
Likewise, the following files |final|\textit{nn}|.tex|
compile the final version of the child document
|child|\textit{nn}|.tex|:
%
\begin{center}
\begin{tabular}{l}
|\def\version{final}|\\
|\input{childdoc.def}|\\
|\childdocforwardprefix{final}{child}|
\end{tabular}
\end{center}
%

Note that when several versions of a main file and/or of each child file
are to be generated, it may be convenient to set up a |Makefile| or
shell script to automatise the process.

%%%%%%%%%%%%%%%%%%%%%%%%%%%%%%%%%%%%%%%%%%%%%%%%%%%%%%%%%%%%%%%%%%%%%%%%%%%%%%%%
\subsection{Command Line Processing}
\label{sec:commandline}

The effect of redirection files can also be achieved by invoking
the \LaTeX{} compiler with a more elaborate command line.
Most conveniently this should be done as part
of a shell script or a |Makefile|.

When using \textsf{childdoc} in the main file, the following
command lines effectively perform a redirection
(note that depending on the shell being used,
backslashes may have to be doubled: `|\|' $\to$ `|\\|'):
%
\begin{center}
|... -jobname "|\textit{target}|" |\\|"|[\textit{flags}]%
|\input{childdoc.def}\childdocforward[|\textit{main}|]{|\textit{dest}|}"|
\end{center}
%
Here \textit{target} is the name of the output file,
\textit{main} is the name of the main file
and \textit{dest} is the name of the main or child file to be processed
(all filenames without extensions).
The optional argument \textit{main} can be omitted
if \textit{main} matches \textit{dest}.
Optionally, compilation \textit{flags} can be defined via |\def| commands.
This command line makes the \TeX{} engine believe
it is compiling the file \textit{target}
whose content is specified as the latter parameter.
The provided code then forwards the processing to
\textit{main} or \textit{dest} as described in \secref{sec:forward}.

%%%%%%%%%%%%%%%%%%%%%%%%%%%%%%%%%%%%%%%%%%%%%%%%%%%%%%%%%%%%%%%%%%%%%%%%%%%%%%%%
\subsection{Include by Input}
\label{sec:input}

Including child documents by |\include| has some restrictions by design.
Most notably, the content of a child document always occupies
its own set of pages; pages cannot be shared between child documents.
Usually, this behaviour makes perfect sense
because each child document contain an essential part of the document.
However, in some situations it may be desirable to compose
a document from a collection of parts
without having mandatory page breaks between then.
For this case, the package
provides a mechanism to include parts
by |\input| which can also be processed individually.
However, by construction this mechanism
requires manual handling of the content to be output.

%%%%%%%%%%%%%%%%%%%%%%%%%%%%%%%%%%%%%%%%
\DescribeMacro{\ifchilddocmanual}
The main file should be prepared as usual, see \secref{sec:include}.
However, the document body must make a distinction
between processing of an individual part and of the main document, e.g.:
%
\begin{center}
\begin{tabular}{l}
|\ifchilddocmanual|\\
|\input{\childdocname}|\\
|\||else|\\
\textit{document body with }|\input{|\textit{part}|}|\\
|\||fi|
\end{tabular}
\end{center}
%
The conditional |\ifchilddocmanual| is true whenever
a part to be included by |\input| is being compiled,
and the name of the part is stored in |\childdocname|.

%%%%%%%%%%%%%%%%%%%%%%%%%%%%%%%%%%%%%%%%
\DescribeMacro{\childdocby}
Each part to be included by |\input| should start with:
%
\begin{center}
\begin{tabular}{l}
|\input{childdoc.def}|\\
|\childdocby{|\textit{main}|}|\\
\end{tabular}
\end{center}
%
The directive |\childdocby| is similar to |\childdocof|
described in \secref{sec:include},
but the subsequent selection of content must be done manually.
To that end, both |\ifchilddoc| and |\ifchilddocmanual|
will be true upon processing of a part,
and the name of the part is stored in |\childdocname|.
Note that |\jobname| will be set to the filename of the current part
so that each part receives an individual |.aux| file
that does not interfere with the |.aux| file(s) of the main document.
This behaviour can be altered by the alternative form
|\childdocby[*]{|\textit{main}|}| (with a non-empty optional argument)
which uses the |.aux| file of the main document
by setting |\jobname| to \textit{main}.

%%%%%%%%%%%%%%%%%%%%%%%%%%%%%%%%%%%%%%%%%%%%%%%%%%%%%%%%%%%%%%%%%%%%%%%%%%%%%%%%
\subsection{Driver Development}
\label{sec:driver}

The \textsf{childdoc} mechanism can also be use for the development
of definition files such as \LaTeX{} styles or classes.
This case differs from the above setup with multiple parts
included by |\include| in that no |\includeonly| should be invoked.
This can be achieved by starting the include file
(before |\ProvidesPackage|) with:
%
\begin{center}
\begin{tabular}{l}
|\input{childdoc.def}|\\
|\childdocforward{|\textit{main}|}|\\
\end{tabular}
\end{center}
%
or alternatively with:
%
\begin{center}
\begin{tabular}{l}
|\input{childdoc.def}|\\
|\childdocby{|\textit{main}|}|\\
\end{tabular}
\end{center}
%
Both forms have slightly different effects as described above.
The main file is prepared as usual, see \secref{sec:include}.

%%%%%%%%%%%%%%%%%%%%%%%%%%%%%%%%%%%%%%%%%%%%%%%%%%%%%%%%%%%%%%%%%%%%%%%%%%%%%%%%
\subsection{Legacy Detection}
\label{sec:detection}

The directive |\childdocmain| in the main file can detect
whether the complete document or merely a child is to be compiled
even without using the directive |\childdocof|.
This method is deprecated because it is less robust
and there is no compelling reason to use it;
it is merely provided for backward compatibility
and it may be removed in future versions.

If the detection mechanism is to be used,
it is mandatory to correctly specify
the filename of the main file as the argument of |\childdocmain|:
%
\begin{center}
\begin{tabular}{l}
|\input{childdoc.def}|\\
|\childdocmain{|\textit{main}|}|\\
\end{tabular}
\end{center}
%
If |\jobname| does not match the argument \textit{main} of |\childdocmain|,
it is assumed that |\jobname| points to the child file to be compiled.
When using |\childdocmain| with the main file specified as argument,
it suffices to start a child file
with just |\input{|\textit{main}|}|
without loading of the package and using |\childdocof|.
If instead all processing is done
with the appropriate \textsf{childdoc} directives,
the argument of \textit{main} of |\childdocmain| can be empty.

An alternative version of the command line processing described
in \secref{sec:commandline} using the detection mechanism reads:
%
\begin{center}
|... -jobname "|\textit{target}|" "|[\textit{flags}]%
[|\def\jobname{|\textit{dest}|}|]|\input{|\textit{main}|}"|
\end{center}

%%%%%%%%%%%%%%%%%%%%%%%%%%%%%%%%%%%%%%%%%%%%%%%%%%%%%%%%%%%%%%%%%%%%%%%%%%%%%%%%
\subsection{Manual Code}
\label{sec:manual}

In case one cannot be certain whether the definitions file |childdoc.def|
is installed on the target \TeX{} distribution
and one prefers not to ship it,
it is conceivable to paste a few relevant commands into the sources.

To that end, drop all statements |\input{childdoc.def}|
and perform the replacements as outlined below.
Instead of |\childdocmain{|\textit{main}|}| add the following code
to the top of the main file:
%
\begin{center}
\begin{tabular}{l}
|\||ifdefined\childdocname\endinput\||fi\newif\ifchilddoc|\\
|\edef\childdocname{\scantokens\expandafter{\jobname\noexpand}}|\\
|\def\childdocmain{|\textit{main}|}\||ifx\childdocmain\childdocname\||else|\\
|\childdoctrue\includeonly{\childdocname}\let\jobname\childdocmain\||fi|\\
\end{tabular}
\end{center}
%
Instead of |\childdocof{|\textit{main}|}| just include the main file
at the top of each child file:
%
\begin{center}
|\input{|\textit{main}|}|
\end{center}
%
A simple redirection |\childdocforward{|\textit{dest}|}| is achieved by:
%
\begin{center}
|\def\jobname{|\textit{dest}|}\input{\jobname}|
\end{center}
%
The redirection with prefix
|\childdocforwardprefix[|\textit{prefix}|]{|\textit{dest}|}|
is accomplished by:
%
\begin{center}
\begin{tabular}{l}
|{\edef\jobname{\scantokens\expandafter{\jobname\noexpand}}|\\
|\def\redirectjob |\textit{prefix}|#1~~~{\gdef\jobname{|\textit{dest}|#1}}|\\
|\expandafter\redirectjob\jobname~~~}\input{\jobname}|
\end{tabular}
\end{center}

In an alternative approach,
child documents can be compiled by a specific command line
without additional code or specific definitions:
%
\begin{center}
|... -jobname "|\textit{target}|" "|[\textit{flags}]%
|\includeonly{|\textit{dest}|}\input{|\textit{main}|}"|
\end{center}
%

%%%%%%%%%%%%%%%%%%%%%%%%%%%%%%%%%%%%%%%%%%%%%%%%%%%%%%%%%%%%%%%%%%%%%%%%%%%%%%%%
%%%%%%%%%%%%%%%%%%%%%%%%%%%%%%%%%%%%%%%%%%%%%%%%%%%%%%%%%%%%%%%%%%%%%%%%%%%%%%%%
\section{Information}

%%%%%%%%%%%%%%%%%%%%%%%%%%%%%%%%%%%%%%%%%%%%%%%%%%%%%%%%%%%%%%%%%%%%%%%%%%%%%%%%
\subsection{Copyright}

Copyright \copyright{} 2017--2018 Niklas Beisert

This work may be distributed and/or modified under the
conditions of the \LaTeX{} Project Public License, either version 1.3
of this license or (at your option) any later version.
The latest version of this license is in
  \url{http://www.latex-project.org/lppl.txt}
and version 1.3 or later is part of all distributions of \LaTeX{}
version 2005/12/01 or later.

This work has the LPPL maintenance status `maintained'.

The Current Maintainer of this work is Niklas Beisert.

This work consists of the files |README.txt|, |childdoc.ins| and |childdoc.dtx|
as well as the derived files |childdoc.def|, |cdocsamp.tex|
with |cdocsch1.tex|, |cdocsch2.tex|, |cdocspt3.tex|, |cdocspt4.tex|,
|cdocsdrf.tex|, |cdocsfn1.tex|, |cdocsfn2.tex|
as well as |childdoc.pdf|.

%%%%%%%%%%%%%%%%%%%%%%%%%%%%%%%%%%%%%%%%%%%%%%%%%%%%%%%%%%%%%%%%%%%%%%%%%%%%%%%%
\subsection{Files and Installation}

The package consists of the files:
%
\begin{center}
\begin{tabular}{ll}
    |README.txt|   & readme file \\
    |childdoc.ins| & installation file \\
    |childdoc.dtx| & source file \\
    |childdoc.def| & definition file \\
    |cdocsamp.tex| & sample main file \\
    |cdocsch1.tex| & sample include file \\
    |cdocsch2.tex| & sample include file \\
    |cdocspt3.tex| & sample part file \\
    |cdocspt4.tex| & sample part file \\
    |cdocsdrf.tex| & sample redirection file \\
    |cdocsfn1.tex| & sample redirection file \\
    |cdocsfn2.tex| & sample redirection file \\
    |childdoc.pdf| & manual
\end{tabular}
\end{center}
%
The distribution consists of the files
|README.txt|, |childdoc.ins| and |childdoc.dtx|.
%
\begin{itemize}
\item
Run (pdf)\LaTeX{} on |childdoc.dtx|
to compile the manual |childdoc.pdf| (this file).
\item
Run \LaTeX{} on |childdoc.ins| to create the definitions file |childdoc.def|
and the sample |cdocsamp.tex| with include files
|cdocsch1.tex|, |cdocsch2.tex|, |cdocspt3.tex|, |cdocspt4.tex|,
|cdocsdrf.tex|, |cdocsfn1.tex|, |cdocsfn2.tex|.
Then copy the file |childdoc.def| to an appropriate directory of your \LaTeX{}
distribution, e.g.\ \textit{texmf-root}|/tex/latex/childdoc|.
\end{itemize}

%%%%%%%%%%%%%%%%%%%%%%%%%%%%%%%%%%%%%%%%%%%%%%%%%%%%%%%%%%%%%%%%%%%%%%%%%%%%%%%%
\subsection{Related CTAN Packages}

There are several other packages which offer a similar functionality:
%
\begin{itemize}
\item
The packages
\href{http://ctan.org/pkg/docmute}{\textsf{docmute}},
\href{http://ctan.org/pkg/includex}{\textsf{includex}} and
\href{http://ctan.org/pkg/standalone}{\textsf{standalone}}
provide commands to include only the document body of
a child file thus allowing both files to be compiled individually.
\item
The packages \href{http://ctan.org/pkg/subdocs}{\textsf{subdocs}}
and \href{http://ctan.org/pkg/subfiles}{\textsf{subfiles}}
provide structures in which the main and child documents can be
encapsulated and allowing them to be compiled individually.
The inclusion mechanism is different from the conventional |\include|.
\item
The package \href{http://ctan.org/pkg/combine}{\textsf{combine}}
is an elaborate solution to combine several documents into one.
\end{itemize}
%
See also the CTAN topic \href{http://ctan.org/topic/subdocs}{\textsf{subdocs}}
for further related packages.
The present package differs from the above solutions in that
a document structure constructed with the conventional |\include| mechanism
just needs two extra commands at the top of every file
such that all constituent files can be compiled individually.

%%%%%%%%%%%%%%%%%%%%%%%%%%%%%%%%%%%%%%%%%%%%%%%%%%%%%%%%%%%%%%%%%%%%%%%%%%%%%%%%
%\subsection{Feature Suggestions}
%
%The following is a list of features which may be useful for future
%versions of this package:
%%
%\begin{itemize}
%\item
%\ldots
%\end{itemize}

%%%%%%%%%%%%%%%%%%%%%%%%%%%%%%%%%%%%%%%%%%%%%%%%%%%%%%%%%%%%%%%%%%%%%%%%%%%%%%%%
\subsection{Revision History}

%%%%%%%%%%%%%%%%%%%%%%%%%%%%%%%%%%%%%%%%
\paragraph{v2.0:} 2018/12/30

\begin{itemize}
\item
immediate forward processing
\item
added |\childdocby| mechanism
\item
manual restructured
\end{itemize}

%%%%%%%%%%%%%%%%%%%%%%%%%%%%%%%%%%%%%%%%
\paragraph{v1.6:} 2018/01/17

\begin{itemize}
\item
application for development of include files
\item
corrections to manual
\end{itemize}

%%%%%%%%%%%%%%%%%%%%%%%%%%%%%%%%%%%%%%%%
\paragraph{v1.5:} 2017/05/21

\begin{itemize}
\item
more complete structuring introduced
\item
|\childdocof| introduced
\item
|\childdoc| renamed to |\childdocmain|
\item
|\childredirect| renamed to |\childdocforward| and |\childdocforwardprefix|
and functionality expanded
\end{itemize}

%%%%%%%%%%%%%%%%%%%%%%%%%%%%%%%%%%%%%%%%
\paragraph{v1.0:} 2017/04/27

\begin{itemize}
\item
manual and install package
\item
first version published on CTAN
\end{itemize}

%%%%%%%%%%%%%%%%%%%%%%%%%%%%%%%%%%%%%%%%
\paragraph{v0.6:} 2017/04/26

\begin{itemize}
\item
redirection mechanism added
\end{itemize}

%%%%%%%%%%%%%%%%%%%%%%%%%%%%%%%%%%%%%%%%
\paragraph{v0.5:} 2017/04/26

\begin{itemize}
\item
functionality in definition file
\end{itemize}


%%%%%%%%%%%%%%%%%%%%%%%%%%%%%%%%%%%%%%%%%%%%%%%%%%%%%%%%%%%%%%%%%%%%%%%%%%%%%%%%
%%%%%%%%%%%%%%%%%%%%%%%%%%%%%%%%%%%%%%%%%%%%%%%%%%%%%%%%%%%%%%%%%%%%%%%%%%%%%%%%
%%%%%%%%%%%%%%%%%%%%%%%%%%%%%%%%%%%%%%%%%%%%%%%%%%%%%%%%%%%%%%%%%%%%%%%%%%%%%%%%
\appendix

\settowidth\MacroIndent{\rmfamily\scriptsize 000\ }

 \DocInput{childdoc.dtx}

\end{document}
%</driver>
% \fi
%
% %%%%%%%%%%%%%%%%%%%%%%%%%%%%%%%%%%%%%%%%%%%%%%%%%%%%%%%%%%%%%%%%%%%%%%%%%%%%%%
% %%%%%%%%%%%%%%%%%%%%%%%%%%%%%%%%%%%%%%%%%%%%%%%%%%%%%%%%%%%%%%%%%%%%%%%%%%%%%%
% \section{Sample}
%\iffalse
%<*samplemain>
%\fi
%
% The following presents a sample document
% with two chapters, two parts, a title page,
% a compile flag as well as three forwarding files to set the flag.
% It consists of eight |.tex| files:
% \begin{center}
% \begin{tabular}{ll}
% |cdocsamp.tex|&main file\\
% |cdocsch1.tex|&include file for chapter 1\\
% |cdocsch2.tex|&include file for chapter 2\\
% |cdocspt3.tex|&include file for part 3\\
% |cdocspt4.tex|&include file for part 4\\
% |cdocsdrf.tex|&forwarding file for main file in draft mode\\
% |cdocsfi1.tex|&forwarding file for final version of chapter 1\\
% |cdocsfi2.tex|&forwarding file for final version of chapter 2\\
% \end{tabular}
% \end{center}
% Each of the eight files can be compiled directly by the \LaTeX{} compiler.
%
% %%%%%%%%%%%%%%%%%%%%%%%%%%%%%%%%%%%%%%
% \paragraph{Main File.}
%
% The main file is called |cdocsamp.tex|.
%
% Load the \textsf{childdoc} definitions and
% declare the filename for the main document:
%    \begin{macrocode}
\input{childdoc.def}
\childdocmain{}
%    \end{macrocode}

% Optional override for |\version| flag:
%    \begin{macrocode}
%%\ifchilddoc\else\providecommand{\version}{draft}\fi
%    \end{macrocode}

% Define the default values for the |\version| flag
% (|final| for the main file and |draft| for childs):
%    \begin{macrocode}
\ifchilddoc
\providecommand{\version}{draft}
\else
\providecommand{\version}{final}
\fi
%    \end{macrocode}

% Load the standard document class:
%    \begin{macrocode}
\documentclass[12pt]{article}
%    \end{macrocode}

% Start the document body:
%    \begin{macrocode}
\begin{document}
%    \end{macrocode}

% Declare a title page.
% Print title, part of document being processed and version flag:
%    \begin{macrocode}
\addtocounter{page}{-1}
\begin{center}
{\LARGE\bfseries{}childdoc example\par}
\vspace{1cm}
\ifchilddoc
\ifchilddocmanual part\else chapter\fi:
`\childdocname' of `\childdocjob'\par
\else
main document: `\childdocjob'\par
\fi
version: \version\par
\end{center}
\newpage
%    \end{macrocode}

% Manually include selected file,
% otherwise process as usual:
%    \begin{macrocode}
\ifchilddocmanual
\section*{part `\childdocname'}
\input{\childdocname}
\else
%    \end{macrocode}

% Include the two chapters:
%    \begin{macrocode}
\include{cdocsch1}
\include{cdocsch2}
%    \end{macrocode}

% Include the two parts unless only chapters should be displayed:
%    \begin{macrocode}
\ifchilddoc\else
\section{part three}
\input{cdocspt3}
\section{part four}
\input{cdocspt4}
\fi
%    \end{macrocode}

% Process as usual until here:
%    \begin{macrocode}
\fi
%    \end{macrocode}

% End of document body:
%    \begin{macrocode}
\end{document}
%    \end{macrocode}
%\iffalse
%</samplemain>
%\fi
%
% %%%%%%%%%%%%%%%%%%%%%%%%%%%%%%%%%%%%%%
% \paragraph{Chapter Include Files.}
%
% The include files are called |cdocsch1.tex| and |cdocsch2.tex|.
%
%\iffalse
%<*samplechap1|samplechap2>
%\fi

% Optional override for |\version| flag:
%    \begin{macrocode}
%%\providecommand{\version}{final}
%    \end{macrocode}

% Include the main document:
%    \begin{macrocode}
\input{childdoc.def}
\childdocof{cdocsamp}
%    \end{macrocode}

%\iffalse
%</samplechap1|samplechap2>
%\fi
%
%\iffalse
%<*samplechap1>
%\fi
% Some text for chapter 1:
%    \begin{macrocode}
\section{one}
some text in chapter one
%    \end{macrocode}

%\iffalse
%</samplechap1>
%\fi
% Some text for chapter 2:
%\iffalse
%<*samplechap2>
%\fi
%    \begin{macrocode}
\section{two}
more text in chapter two
%    \end{macrocode}

%\iffalse
%</samplechap2>
%\fi
%
% %%%%%%%%%%%%%%%%%%%%%%%%%%%%%%%%%%%%%%
% \paragraph{Part Include Files.}
%
% The include files are called |cdocspt3.tex| and |cdocspt4.tex|.
%
%\iffalse
%<*samplepart3|samplepart4>
%\fi

% Optional override for |\version| flag:
%    \begin{macrocode}
%%\providecommand{\version}{final}
%    \end{macrocode}

% Include the main document:
%    \begin{macrocode}
\input{childdoc.def}
\childdocby{cdocsamp}
%    \end{macrocode}

%\iffalse
%</samplepart3|samplepart4>
%\fi
%
%\iffalse
%<*samplepart3>
%\fi
% Some text for part 3:
%    \begin{macrocode}
some text in part three
%    \end{macrocode}

%\iffalse
%</samplepart3>
%\fi
% Some text for part 4:
%\iffalse
%<*samplepart4>
%\fi
%    \begin{macrocode}
more text in part four
%    \end{macrocode}

%\iffalse
%</samplepart4>
%\fi
%
% %%%%%%%%%%%%%%%%%%%%%%%%%%%%%%%%%%%%%%
% \paragraph{Forwarding for a Complete Draft.}
%
% The following forwarding file |cdocsdrf.tex|
% compiles the main document in draft mode:
%\iffalse
%<*sampledraft>
%\fi
%    \begin{macrocode}
\def\version{draft}
\input{childdoc.def}
\childdocforward{cdocsamp}
%    \end{macrocode}

%\iffalse
%</sampledraft>
%\fi
%
% %%%%%%%%%%%%%%%%%%%%%%%%%%%%%%%%%%%%%%
% \paragraph{Forwarding for Final Version of the Chapters.}
%
% The following forwarding files |cdocsfn1.tex| and |cdocsfn2.tex|
% (with identical content)
% compile the final versions of the child documents
% |cdocsch1.tex| and |cdocsch2.tex|, respectively:
%\iffalse
%<*samplefinal>
%\fi
%    \begin{macrocode}
\def\version{final}
\input{childdoc.def}
\childdocforwardprefix[cdocsamp]{cdocsfn}{cdocsch}
%    \end{macrocode}

%\iffalse
%</samplefinal>
%\fi
%
% %%%%%%%%%%%%%%%%%%%%%%%%%%%%%%%%%%%%%%
% \paragraph{Command Line Processing.}
%
% The following three command lines generate the output files
% |cdocscld|, |cdocscl1| and |cdocscl2|
% which should be identical to
% |cdocsdrf|, |cdocsch1| and |cdocsfn2|, respectively:
% \begin{center}
% \begin{tabular}{l}
% |latex -jobname cdocscld \|\\
% |  "\def\version{draft}\input{childdoc.def}\childdocforward{cdocsamp}"|\\
% |latex -jobname cdocscl1 \|\\
% |  "\input{childdoc.def}\childdocforward[cdocsamp]{cdocsch1}"|\\
% |latex -jobname cdocscl2 \|\\
% |  "\def\version{final}\input{childdoc.def}\childdocforward{cdocsch2}"|
% \end{tabular}
% \end{center}
% Note that the trailing backslash on each first line
% merely continues the input to the second line
% (for convenient cut ant paste).
% Furthermore, the command |latex| can be replaced by any
% of its alternative versions such as |pdflatex|.
%
% %%%%%%%%%%%%%%%%%%%%%%%%%%%%%%%%%%%%%%%%%%%%%%%%%%%%%%%%%%%%%%%%%%%%%%%%%%%%%%
% %%%%%%%%%%%%%%%%%%%%%%%%%%%%%%%%%%%%%%%%%%%%%%%%%%%%%%%%%%%%%%%%%%%%%%%%%%%%%%
% \section{Implementation}
%\iffalse
%<*package>
%\fi
%
% This section describes the definitions file |childdoc.def|.

% The definitions cannot be loaded using |\usepackage| or |\RequirePackage|
% which has a mechanism to prevent loading a style file more than once.
% When loading the definitions by means of |\input|
% multiple instances have to be prevented manually:
%\iffalse
%This code needs to be before the `\ProvidesFile' directive
%which is defined at the beginning of this file.
%Therefore it is also placed there and commented out here.
%</package>
%<*discard>
%\fi
%    \begin{macrocode}
\ifdefined\childdocmain\endinput\fi
%    \end{macrocode}
%\iffalse
%</discard>
%<*package>
%\fi
%
% \macro{\ifchilddoc}
% \macro{\ifchilddocmanual}
% The conditional |\ifchilddoc| tells whether a
% child (true) or main (false) document is being compiled.
% The conditional |\ifchilddocmanual| tells whether
% the |\includeonly| mechanism is used (false) or
% the selection of child files must be performed manually (true).
% The definitions initialise to false:
%    \begin{macrocode}
\newif\ifchilddoc
\newif\ifchilddocmanual
%    \end{macrocode}

% \macro{\childdocname}
% \macro{\childdocjob}
% The macro |\childdocname| stores the name of the main document
% to be compiled. The macro |\childdocjob| stores the name of
% the document on which the \LaTeX{} compiler was originally invoked.
% The content of |\jobname| cannot be compared
% to filenames specified in the source due to different catcodes.
% The following code rescans |\jobname|, stores the result
% in |\childdocname| and saves a copy in |\childdocjob|:
%    \begin{macrocode}
\edef\childdocname{\scantokens\expandafter{\jobname\noexpand}}
\let\childdocjob\childdocname
%    \end{macrocode}

% \macro{\childdocdisable}
% The macro |\childdocdisable| prevents the main file
% from being processed more than once.
% At this stage, the main document command |\childdocmain|
% is assumed to be called once again where it should do nothing.
% Any subsequent call to it should prevent
% a secondary processing of the main document
% It overwrites the forwarding commands
% |\childdocof| and |\childdocforward|
% with empty macros to prevent further inclusions of the main document:
%    \begin{macrocode}
\newcommand{\childdocdisable}
{
  \renewcommand{\childdocmain}[1]{\renewcommand{\childdocmain}[1]{\endinput}}
  \renewcommand{\childdocof}[1]{}
  \renewcommand{\childdocby}[2][]{}
  \renewcommand{\childdocforward}[2][]{}
  \renewcommand{\childdocdisable}{}
}
%    \end{macrocode}

% \macro{\childdocmain}
% The macro |\childdocmain| is to be called at the top of the main file
% with nothing or the main filename (without extension) as argument.
% First, it breaks loops.
% If the argument is not empty and does not match |\childdocname|
% (which is set by the first inclusion of |childdoc.def|),
% |\ifchilddoc| is set to true, |\includeonly| is applied to the child file
% and |\jobname| is set to the main file
% (for proper handling of |.aux| files):
%    \begin{macrocode}
\newcommand{\childdocmain}[1]
{
  \childdocdisable\childdocmain{}
  \if?#1?\else
    \begingroup
      \def\childdoctmp{#1}
      \ifx\childdoctmp\childdocname
        \def\childdoctmp{}
      \else
        \def\childdoctmp
        {
          \childdoctrue
          \includeonly{\childdocname}
          \def\childdocjob{#1}
          \def\jobname{#1}
        }
      \fi
      \expandafter
    \endgroup
    \childdoctmp
  \fi
}
%    \end{macrocode}

% \macro{\childdocof}
% The command |\childdocof| redirects
% compilation to the main file |#1|.
%    \begin{macrocode}
\newcommand{\childdocof}[1]
{
  \childdocdisable
  \childdoctrue
  \includeonly{\childdocname}
  \def\jobname{#1}
  \def\childdocjob{#1}
  \input{#1}
}
%    \end{macrocode}

% \macro{\childdocby}
% The command |\childdocby| ....
%    \begin{macrocode}
\newcommand{\childdocby}[2][]
{
  \childdocdisable
  \childdoctrue
  \childdocmanualtrue
  \if?#1?\else
    \def\jobname{#2}
  \fi
  \def\childdocjob{#2}
  \input{#2}
  \endinput
}
%    \end{macrocode}

% \macro{\childdocforward}
% The command |\childdocforward| redirects
% compilation to the main file or
% (if the optional argument is given) a child file.
% Parameters are set as if the main file
% or a child file starting with |\childdocof| was compiled.
% Then compilation is handed over to the main file:
%    \begin{macrocode}
\newcommand{\childdocforward}[2][]
{
  \begingroup
    \if?#1?
      \def\childdoctmp
      {
        \def\childdocname{#2}
        \def\childdocjob{#2}
        \def\jobname{#2}
        \input{#2}
        \endinput
      }
    \else
      \def\childdoctmp
      {
        \childdocdisable
        \def\childdocname{#2}
        \childdoctrue
        \includeonly{#2}
        \def\childdocjob{#1}
        \def\jobname{#1}
        \input{#1}
        \endinput
      }
    \fi
    \expandafter
  \endgroup
  \childdoctmp
}
%    \end{macrocode}

% \macro{\childdocforwardprefix}
% The command |\childdocforwardprefix| redirects
% compilation to the main or a child file by means of a pattern.
% The prefix |#1| in the current filename is replaced by |#2|
% and the suffix of the current filename is kept
% (it is assumed that the filename does not contain the substring `|~~~|'
% which is used as a delimiter).
% Compilation is handed over to the new file by |\childdocforward|:
%    \begin{macrocode}
\newcommand{\childdocforwardprefix}[3][]
{
  \begingroup
    \def\childdocextract #2##1~~~{\def\childdoctmp{\childdocforward[#1]{#3##1}}}
    \expandafter\childdocextract\childdocname~~~
    \expandafter
  \endgroup
  \childdoctmp
}
%    \end{macrocode}

% \macro{\childdoc}
% The deprecated macro |\childdoc| is a legacy version of |\childdocmain|:
%    \begin{macrocode}
\newcommand{\childdoc}{\childdocmain}
%    \end{macrocode}

% \macro{\childdocredirect}
% The deprecated macro |\childdocredirect| is a legacy version
% of |\childdocforward| and |\childdocforwardprefix|:
%    \begin{macrocode}
\newcommand{\childdocredirect}[2][]
{
  \begingroup
    \if?#1?
      \def\childdoctmp{\childdocforward{#2}}
    \else
      \def\childdoctmp{\childdocforwardprefix{#1}{#2}}
    \fi
    \expandafter
  \endgroup
  \childdoctmp
}
%    \end{macrocode}

%\iffalse
%</package>
%\fi
%
\endinput
|
and perform the replacements as outlined below.
Instead of |\childdocmain{|\textit{main}|}| add the following code
to the top of the main file:
%
\begin{center}
\begin{tabular}{l}
|\||ifdefined\childdocname\endinput\||fi\newif\ifchilddoc|\\
|\edef\childdocname{\scantokens\expandafter{\jobname\noexpand}}|\\
|\def\childdocmain{|\textit{main}|}\||ifx\childdocmain\childdocname\||else|\\
|\childdoctrue\includeonly{\childdocname}\let\jobname\childdocmain\||fi|\\
\end{tabular}
\end{center}
%
Instead of |\childdocof{|\textit{main}|}| just include the main file
at the top of each child file:
%
\begin{center}
|\input{|\textit{main}|}|
\end{center}
%
A simple redirection |\childdocforward{|\textit{dest}|}| is achieved by:
%
\begin{center}
|\def\jobname{|\textit{dest}|}\input{\jobname}|
\end{center}
%
The redirection with prefix
|\childdocforwardprefix[|\textit{prefix}|]{|\textit{dest}|}|
is accomplished by:
%
\begin{center}
\begin{tabular}{l}
|{\edef\jobname{\scantokens\expandafter{\jobname\noexpand}}|\\
|\def\redirectjob |\textit{prefix}|#1~~~{\gdef\jobname{|\textit{dest}|#1}}|\\
|\expandafter\redirectjob\jobname~~~}\input{\jobname}|
\end{tabular}
\end{center}

In an alternative approach,
child documents can be compiled by a specific command line
without additional code or specific definitions:
%
\begin{center}
|... -jobname "|\textit{target}|" "|[\textit{flags}]%
|\includeonly{|\textit{dest}|}\input{|\textit{main}|}"|
\end{center}
%

%%%%%%%%%%%%%%%%%%%%%%%%%%%%%%%%%%%%%%%%%%%%%%%%%%%%%%%%%%%%%%%%%%%%%%%%%%%%%%%%
%%%%%%%%%%%%%%%%%%%%%%%%%%%%%%%%%%%%%%%%%%%%%%%%%%%%%%%%%%%%%%%%%%%%%%%%%%%%%%%%
\section{Information}

%%%%%%%%%%%%%%%%%%%%%%%%%%%%%%%%%%%%%%%%%%%%%%%%%%%%%%%%%%%%%%%%%%%%%%%%%%%%%%%%
\subsection{Copyright}

Copyright \copyright{} 2017--2018 Niklas Beisert

This work may be distributed and/or modified under the
conditions of the \LaTeX{} Project Public License, either version 1.3
of this license or (at your option) any later version.
The latest version of this license is in
  \url{http://www.latex-project.org/lppl.txt}
and version 1.3 or later is part of all distributions of \LaTeX{}
version 2005/12/01 or later.

This work has the LPPL maintenance status `maintained'.

The Current Maintainer of this work is Niklas Beisert.

This work consists of the files |README.txt|, |childdoc.ins| and |childdoc.dtx|
as well as the derived files |childdoc.def|, |cdocsamp.tex|
with |cdocsch1.tex|, |cdocsch2.tex|, |cdocspt3.tex|, |cdocspt4.tex|,
|cdocsdrf.tex|, |cdocsfn1.tex|, |cdocsfn2.tex|
as well as |childdoc.pdf|.

%%%%%%%%%%%%%%%%%%%%%%%%%%%%%%%%%%%%%%%%%%%%%%%%%%%%%%%%%%%%%%%%%%%%%%%%%%%%%%%%
\subsection{Files and Installation}

The package consists of the files:
%
\begin{center}
\begin{tabular}{ll}
    |README.txt|   & readme file \\
    |childdoc.ins| & installation file \\
    |childdoc.dtx| & source file \\
    |childdoc.def| & definition file \\
    |cdocsamp.tex| & sample main file \\
    |cdocsch1.tex| & sample include file \\
    |cdocsch2.tex| & sample include file \\
    |cdocspt3.tex| & sample part file \\
    |cdocspt4.tex| & sample part file \\
    |cdocsdrf.tex| & sample redirection file \\
    |cdocsfn1.tex| & sample redirection file \\
    |cdocsfn2.tex| & sample redirection file \\
    |childdoc.pdf| & manual
\end{tabular}
\end{center}
%
The distribution consists of the files
|README.txt|, |childdoc.ins| and |childdoc.dtx|.
%
\begin{itemize}
\item
Run (pdf)\LaTeX{} on |childdoc.dtx|
to compile the manual |childdoc.pdf| (this file).
\item
Run \LaTeX{} on |childdoc.ins| to create the definitions file |childdoc.def|
and the sample |cdocsamp.tex| with include files
|cdocsch1.tex|, |cdocsch2.tex|, |cdocspt3.tex|, |cdocspt4.tex|,
|cdocsdrf.tex|, |cdocsfn1.tex|, |cdocsfn2.tex|.
Then copy the file |childdoc.def| to an appropriate directory of your \LaTeX{}
distribution, e.g.\ \textit{texmf-root}|/tex/latex/childdoc|.
\end{itemize}

%%%%%%%%%%%%%%%%%%%%%%%%%%%%%%%%%%%%%%%%%%%%%%%%%%%%%%%%%%%%%%%%%%%%%%%%%%%%%%%%
\subsection{Related CTAN Packages}

There are several other packages which offer a similar functionality:
%
\begin{itemize}
\item
The packages
\href{http://ctan.org/pkg/docmute}{\textsf{docmute}},
\href{http://ctan.org/pkg/includex}{\textsf{includex}} and
\href{http://ctan.org/pkg/standalone}{\textsf{standalone}}
provide commands to include only the document body of
a child file thus allowing both files to be compiled individually.
\item
The packages \href{http://ctan.org/pkg/subdocs}{\textsf{subdocs}}
and \href{http://ctan.org/pkg/subfiles}{\textsf{subfiles}}
provide structures in which the main and child documents can be
encapsulated and allowing them to be compiled individually.
The inclusion mechanism is different from the conventional |\include|.
\item
The package \href{http://ctan.org/pkg/combine}{\textsf{combine}}
is an elaborate solution to combine several documents into one.
\end{itemize}
%
See also the CTAN topic \href{http://ctan.org/topic/subdocs}{\textsf{subdocs}}
for further related packages.
The present package differs from the above solutions in that
a document structure constructed with the conventional |\include| mechanism
just needs two extra commands at the top of every file
such that all constituent files can be compiled individually.

%%%%%%%%%%%%%%%%%%%%%%%%%%%%%%%%%%%%%%%%%%%%%%%%%%%%%%%%%%%%%%%%%%%%%%%%%%%%%%%%
%\subsection{Feature Suggestions}
%
%The following is a list of features which may be useful for future
%versions of this package:
%%
%\begin{itemize}
%\item
%\ldots
%\end{itemize}

%%%%%%%%%%%%%%%%%%%%%%%%%%%%%%%%%%%%%%%%%%%%%%%%%%%%%%%%%%%%%%%%%%%%%%%%%%%%%%%%
\subsection{Revision History}

%%%%%%%%%%%%%%%%%%%%%%%%%%%%%%%%%%%%%%%%
\paragraph{v2.0:} 2018/12/30

\begin{itemize}
\item
immediate forward processing
\item
added |\childdocby| mechanism
\item
manual restructured
\end{itemize}

%%%%%%%%%%%%%%%%%%%%%%%%%%%%%%%%%%%%%%%%
\paragraph{v1.6:} 2018/01/17

\begin{itemize}
\item
application for development of include files
\item
corrections to manual
\end{itemize}

%%%%%%%%%%%%%%%%%%%%%%%%%%%%%%%%%%%%%%%%
\paragraph{v1.5:} 2017/05/21

\begin{itemize}
\item
more complete structuring introduced
\item
|\childdocof| introduced
\item
|\childdoc| renamed to |\childdocmain|
\item
|\childredirect| renamed to |\childdocforward| and |\childdocforwardprefix|
and functionality expanded
\end{itemize}

%%%%%%%%%%%%%%%%%%%%%%%%%%%%%%%%%%%%%%%%
\paragraph{v1.0:} 2017/04/27

\begin{itemize}
\item
manual and install package
\item
first version published on CTAN
\end{itemize}

%%%%%%%%%%%%%%%%%%%%%%%%%%%%%%%%%%%%%%%%
\paragraph{v0.6:} 2017/04/26

\begin{itemize}
\item
redirection mechanism added
\end{itemize}

%%%%%%%%%%%%%%%%%%%%%%%%%%%%%%%%%%%%%%%%
\paragraph{v0.5:} 2017/04/26

\begin{itemize}
\item
functionality in definition file
\end{itemize}


%%%%%%%%%%%%%%%%%%%%%%%%%%%%%%%%%%%%%%%%%%%%%%%%%%%%%%%%%%%%%%%%%%%%%%%%%%%%%%%%
%%%%%%%%%%%%%%%%%%%%%%%%%%%%%%%%%%%%%%%%%%%%%%%%%%%%%%%%%%%%%%%%%%%%%%%%%%%%%%%%
%%%%%%%%%%%%%%%%%%%%%%%%%%%%%%%%%%%%%%%%%%%%%%%%%%%%%%%%%%%%%%%%%%%%%%%%%%%%%%%%
\appendix

\settowidth\MacroIndent{\rmfamily\scriptsize 000\ }

 \DocInput{childdoc.dtx}

\end{document}
%</driver>
% \fi
%
% %%%%%%%%%%%%%%%%%%%%%%%%%%%%%%%%%%%%%%%%%%%%%%%%%%%%%%%%%%%%%%%%%%%%%%%%%%%%%%
% %%%%%%%%%%%%%%%%%%%%%%%%%%%%%%%%%%%%%%%%%%%%%%%%%%%%%%%%%%%%%%%%%%%%%%%%%%%%%%
% \section{Sample}
%\iffalse
%<*samplemain>
%\fi
%
% The following presents a sample document
% with two chapters, two parts, a title page,
% a compile flag as well as three forwarding files to set the flag.
% It consists of eight |.tex| files:
% \begin{center}
% \begin{tabular}{ll}
% |cdocsamp.tex|&main file\\
% |cdocsch1.tex|&include file for chapter 1\\
% |cdocsch2.tex|&include file for chapter 2\\
% |cdocspt3.tex|&include file for part 3\\
% |cdocspt4.tex|&include file for part 4\\
% |cdocsdrf.tex|&forwarding file for main file in draft mode\\
% |cdocsfi1.tex|&forwarding file for final version of chapter 1\\
% |cdocsfi2.tex|&forwarding file for final version of chapter 2\\
% \end{tabular}
% \end{center}
% Each of the eight files can be compiled directly by the \LaTeX{} compiler.
%
% %%%%%%%%%%%%%%%%%%%%%%%%%%%%%%%%%%%%%%
% \paragraph{Main File.}
%
% The main file is called |cdocsamp.tex|.
%
% Load the \textsf{childdoc} definitions and
% declare the filename for the main document:
%    \begin{macrocode}
% \iffalse
%
% childdoc.dtx Copyright (C) 2017-2018 Niklas Beisert
%
% This work may be distributed and/or modified under the
% conditions of the LaTeX Project Public License, either version 1.3
% of this license or (at your option) any later version.
% The latest version of this license is in
%   http://www.latex-project.org/lppl.txt
% and version 1.3 or later is part of all distributions of LaTeX
% version 2005/12/01 or later.
%
% This work has the LPPL maintenance status `maintained'.
%
% The Current Maintainer of this work is Niklas Beisert.
%
% This work consists of the files childdoc.dtx and childdoc.ins
% and the derived files childdoc.def and cdocsamp.tex with
% cdocsch1.tex, cdocsch2.tex, cdocsdrf.tex, cdocsfn1.tex, cdocsfn2.tex.
%
%<package>\ifdefined\childdocmain\endinput\fi
%<package>\ProvidesFile{childdoc.def}[2018/12/30 v2.0 child document driver]
%<samplemain>\ProvidesFile{cdocsamp.tex}[2018/12/30 v2.0 sample for childdoc]
%<*driver>
%\ProvidesFile{childdoc.drv}[2018/12/30 v2.0 childdoc reference manual file]
\PassOptionsToClass{10pt,a4paper}{article}
\documentclass{ltxdoc}

\usepackage[margin=35mm]{geometry}
\usepackage{hyperref}
\usepackage{hyperxmp}
\usepackage[usenames]{color}

\hypersetup{colorlinks=true}
\hypersetup{pdfstartview=FitH}
\hypersetup{pdfpagemode=UseNone}
\hypersetup{pdfsource={}}
\hypersetup{pdflang={en-UK}}
\hypersetup{pdfcopyright={Copyright 2017-2018 Niklas Beisert.
  This work may be distributed and/or modified under the
  conditions of the LaTeX Project Public License, either version 1.3
  of this license or (at your option) any later version.}}
\hypersetup{pdflicenseurl={http://www.latex-project.org/lppl.txt}}
\hypersetup{pdfcontactaddress={ETH Zurich, ITP, HIT K,
  Wolfgang-Pauli-Strasse 27}}
\hypersetup{pdfcontactpostcode={8093}}
\hypersetup{pdfcontactcity={Zurich}}
\hypersetup{pdfcontactcountry={Switzerland}}
\hypersetup{pdfcontactemail={nbeisert@itp.phys.ethz.ch}}
\hypersetup{pdfcontacturl={http://people.phys.ethz.ch/\xmptilde nbeisert/}}

\newcommand{\secref}[1]{\hyperref[#1]{section \ref*{#1}}}

\parskip1ex
\parindent0pt
\let\olditemize\itemize
\def\itemize{\olditemize\parskip0pt}

\begin{document}

\title{The \textsf{childdoc} Package}
\hypersetup{pdftitle={The childdoc Package}}
\author{Niklas Beisert\\[2ex]
  Institut f\"ur Theoretische Physik\\
  Eidgen\"ossische Technische Hochschule Z\"urich\\
  Wolfgang-Pauli-Strasse 27, 8093 Z\"urich, Switzerland\\[1ex]
  \href{mailto:nbeisert@itp.phys.ethz.ch}
  {\texttt{nbeisert@itp.phys.ethz.ch}}}
\hypersetup{pdfauthor={Niklas Beisert}}
\hypersetup{pdfsubject={Manual for the LaTeX2e Package childdoc}}
\date{30 December 2018, \textsf{v2.0}}
\maketitle

\begin{abstract}\noindent
\textsf{childdoc} is a \LaTeXe{} package
that enables the direct compilation
of document sections included by |\include|
to individual files.
\end{abstract}

\begingroup
\parskip0ex
\tableofcontents
\endgroup

%%%%%%%%%%%%%%%%%%%%%%%%%%%%%%%%%%%%%%%%%%%%%%%%%%%%%%%%%%%%%%%%%%%%%%%%%%%%%%%%
%%%%%%%%%%%%%%%%%%%%%%%%%%%%%%%%%%%%%%%%%%%%%%%%%%%%%%%%%%%%%%%%%%%%%%%%%%%%%%%%
\section{Introduction}

\LaTeX{} provides a mechanism to structure a large document (such as a book)
into a main file and several child files (containing the chapters)
using the |\include| command.
This mechanism is beneficial for documents
which span hundreds of pages in order to
make the source file(s) more manageable.
Moreover, compilation can be restricted to
selected child files by means of the |\includeonly| command.
The latter feature can be used to reduce the compilation time while editing
(this was significantly more useful in the earlier days of \LaTeX{})
or to generate a smaller document which is easier to navigate.
Another application of |\includeonly| is to generate
documents consisting of selected parts of the complete document.

However, there are a few drawbacks of the plain |\include| mechanism:
\begin{itemize}
\item
The child files cannot be compiled on their own,
they can only be compiled via the main file.
A naive editing environment
(such as a text editor with an option
to have the current file processed by \LaTeX)
may require one to switch to the main file before compiling;
attempting to compile the child file produces errors.
\item
The main file must be modified (each time)
to adjust the |\includeonly| command
to the present needs. This easily leaves the main file in a messy state.
\item
The generated document will always carry the filename
of the main document. This is inconvenient if
several child files are to be compiled and
to be kept for distribution.
\end{itemize}

The present package provides a simple interface
to make child files individually compilable by \LaTeX{}.
Compiling a child file then has the same effect as compiling
the main file with an |\includeonly| command
to select the appropriate child.
Moreover the generated document will carry the name of the child
rather than the main file.
This resolves all three above issues.

This feature is meant to make the editing of books,
thesis documents and lecture notes somewhat more convenient.
However, the package can also be used efficiently for
composing a series of documents (such as exercise sheets)
which are typically distributed individually.
It then assists the author in generating the individual documents
(potentially in different versions)
as well as a document containing the collected series.
Another application is in developing style files
or other kinds of included material
where compilation of the style file could redirect
to a sample or test file.

%%%%%%%%%%%%%%%%%%%%%%%%%%%%%%%%%%%%%%%%%%%%%%%%%%%%%%%%%%%%%%%%%%%%%%%%%%%%%%%%
%%%%%%%%%%%%%%%%%%%%%%%%%%%%%%%%%%%%%%%%%%%%%%%%%%%%%%%%%%%%%%%%%%%%%%%%%%%%%%%%
\section{Usage}

First of all, the package \textsf{childdoc} is \emph{not} a standard
\LaTeXe{} |.sty| style file! Therefore it needs to be invoked in
a non-standard way.

%%%%%%%%%%%%%%%%%%%%%%%%%%%%%%%%%%%%%%%%%%%%%%%%%%%%%%%%%%%%%%%%%%%%%%%%%%%%%%%%
\subsection{Included Files}
\label{sec:include}

%%%%%%%%%%%%%%%%%%%%%%%%%%%%%%%%%%%%%%%%
\DescribeMacro{\childdocmain}
To use the package, add the commands
\begin{center}
\begin{tabular}{l}
|\input{childdoc.def}|\\
|\childdocmain{}|\\
\end{tabular}
\end{center}
at the very top of the main \LaTeX{} file,
in particular \emph{before} the |\documentclass| statement!
The argument of |\childdocmain| should be left empty
(but it must be present).

%%%%%%%%%%%%%%%%%%%%%%%%%%%%%%%%%%%%%%%%
\DescribeMacro{\childdocof}
Furthermore, add the commands
\begin{center}
\begin{tabular}{l}
|\input{childdoc.def}|\\
|\childdocof{|\textit{main}|}|\\
\end{tabular}
\end{center}
at the top of every child file \textit{child}
which is included by |\include{|\textit{child}|}|
from within the main file
(or at least for those files to be compiled individually).
The argument \textit{main} must be the filename of the main file.

There are a couple of
considerations in setting up the main and child documents:

%%%%%%%%%%%%%%%%%%%%%%%%%%%%%%%%%%%%%%%%
\paragraph{Restrictions.}

Please note the following restrictions:
\begin{itemize}
\item
|\childdocmain| must be called with one argument \textit{main}
to ensure compatibility with earlier version of the package.
It must either be empty (|\childdocmain{}|)
or precisely match the filename of the main file in which it is specified.
See \secref{sec:detection} for further information.
\item
The filename \textit{main} must be specified without the |.tex| extension.
\item
The filename \textit{main} is case sensitive
(even in case-insensitive file systems)
due to internal string comparison.
\item
The argument \textit{main} should be fully expanded, it cannot be a macro.
\item
Subdirectories and special characters should be avoided in filenames.
\item
The command |\childdocmain{|\textit{main}|}| must be followed by a whitespace.
It should not be followed immediately by another command
or by a comment mark `|%|'.
This is because the \TeX{} parser reads the token immediately following
the argument of |\childdocmain| and puts it
at the beginning of every child section;
however, a white\-space is ignored.
\end{itemize}

%%%%%%%%%%%%%%%%%%%%%%%%%%%%%%%%%%%%%%%%
\paragraph{Content of Main File.}

It is advisable to place all content in the child files included by |\include|.
Any output contained in the main file will appear in all child documents
unless suppressed manually;
it cannot be suppressed automatically by the |\includeonly| directive
and thus should normally be avoided.
A method to include some content in the main file
by means of conditional processing is described in \secref{sec:conditional}.

%%%%%%%%%%%%%%%%%%%%%%%%%%%%%%%%%%%%%%%%
\paragraph{Page Numbering.}

When only a part of the document is compiled,
the appropriate numbering of pages
(as well as other status parameters)
is determined from the |.aux| files.
The latter contain information from previous passes.
However this information needs to propagate through
all intermediate child documents.
Therefore the page numbering in child documents may well
be inconsistent until the complete document is compiled at least once.

A useful (if unconventional) way to always ensure a consistent
page numbering is to restart the numbering in each child document
and denote the pages by `\textit{child}|.|\textit{page}'
where \textit{child} represents the chapter/section number of the child file.
This can be achieved by the command
|\numberwithin{page}{|\textit{child}|}|
of the \textsf{amsmath} package
where \textit{child} can be |chapter| or |section|
depending on the chosen structuring.
Alternatively, one can modify the macro |\thepage| appropriately
and reset the counter |page| at the start of each child file.

%%%%%%%%%%%%%%%%%%%%%%%%%%%%%%%%%%%%%%%%%%%%%%%%%%%%%%%%%%%%%%%%%%%%%%%%%%%%%%%%
\subsection{Conditional Processing}
\label{sec:conditional}

The package provides a mechanism to compile different versions
of a document. To customise the versions further some conditional processing
can come in handy to distinguish which version is being compiled.
The package provides two macros to describe the compilation context:

%%%%%%%%%%%%%%%%%%%%%%%%%%%%%%%%%%%%%%%%
\DescribeMacro{\ifchilddoc}
The conditional |\ifchilddoc| distinguishes between the compilation of
child documents and the main document:
%
\begin{center}
|\ifchilddoc |\textit{child-code}| |[|\||else |\textit{main-code}]| \||fi|
\end{center}

%%%%%%%%%%%%%%%%%%%%%%%%%%%%%%%%%%%%%%%%
\DescribeMacro{\childdocname}
\DescribeMacro{\childdocjob}
The macro |\childdocname| contains the filename (without extension)
of the main or child file being processed.
Note that |\childdocjob| will always contain the name of the main file.

%%%%%%%%%%%%%%%%%%%%%%%%%%%%%%%%%%%%%%%%
\paragraph{Title Page.}

Conditional processing can be used to include a title or banner page
in the main document when proper precautions are taken.
Importantly, the code in the main file should ensure that the page counter
(as well as other status parameters which are stored in the |.aux| files)
takes the same value after the conditional processing.
Otherwise the page numbers may take divergent values
depending on which part is compiled.

For example, a title page could be declared by:
%
\begin{center}
\begin{tabular}{l}
|\ifchilddoc\||else|\\
|\addtocounter{page}{-1}|\\
\textit{code for title page}\\
|\newpage|\\
|\||fi|
\end{tabular}
\end{center}
%
A banner page for the child documents can be generated by:
%
\begin{center}
\begin{tabular}{l}
|\ifchilddoc|\\
|\addtocounter{page}{-1}|\\
\textit{code for banner page}\\
|\newpage|\\
|\||fi|
\end{tabular}
\end{center}
%
Here one could write a message such as:
\begin{center}
|This is the part \childdocname{} of \childdocjob{}.|
\end{center}

%%%%%%%%%%%%%%%%%%%%%%%%%%%%%%%%%%%%%%%%%%%%%%%%%%%%%%%%%%%%%%%%%%%%%%%%%%%%%%%%
\subsection{Flags}
\label{sec:flags}

The package makes it easy to generate different versions
of the main or child documents.
To this end compilation flags can be defined
and assigned different default values.
They will be particularly useful in conjunction
with the forwarding mechanism described in \secref{sec:forward}.

For example, it may be useful to have a flag |\version|
which can be set to |draft| or |final|.
The document source will contain some conditional code
depending on the value of |\version|.
Suppose further, the flag should default to |final| for the main file
and to |draft| for child files
which is a natural assignment for editing the document.
This is achieved by placing the following code
in the preamble of the main document
(below the |\childdocmain| directive):
%
\begin{center}
\begin{tabular}{l}
|\ifchilddoc|\\
|\providecommand{\version}{draft}|\\
|\||else|\\
|\providecommand{\version}{final}|\\
|\||fi|
\end{tabular}
\end{center}
%
The definition by |\providecommand| makes sure
that previous definitions are not overwritten.
Further statements |\providecommand{\version}{...}|
can thus be added before the above code to override it.

For the main file, one might add a line
(between |\childdocmain| and the above block)
%
\begin{center}
|%\ifchilddoc\||else\providecommand{\version}{draft}\||fi|
\end{center}
%
which can be uncommented to produce a draft version.
Likewise one can add a line to the very top of a child file
(above the |\childdocof{|\textit{main}|}| directive)
%
\begin{center}
|%\providecommand{\version}{final}|
\end{center}
%
which can be uncommented to produce the final version of this child document.

%%%%%%%%%%%%%%%%%%%%%%%%%%%%%%%%%%%%%%%%%%%%%%%%%%%%%%%%%%%%%%%%%%%%%%%%%%%%%%%%
\subsection{Forwarding}
\label{sec:forward}

Different versions of the main or child documents
using compilation flags as described in \secref{sec:flags}
can be (permanently) stored in different files
for convenient compilation, viewing and distribution.
To this end, the package defines a command
to pass on compilation to a different file:

%%%%%%%%%%%%%%%%%%%%%%%%%%%%%%%%%%%%%%%%
\DescribeMacro{\childdocforward}
The command |\childdocforward| redirects processing to
another source file:
%
\begin{center}
\begin{tabular}{l}
|\input{childdoc.def}|\\
|\childdocforward[|\textit{main}|]{|\textit{dest}|}|\\
\end{tabular}
\end{center}
%
The argument \textit{dest} is the destination file
(without extension).
It should be the main file or one of the child files.
Note that further \textsf{childdoc} directives
such as |\childdocof| and |\childdocforward|
in the indicated file will be processed in this form.
The optional argument \textit{main}
passes on directly to the main file \textit{main}
while pretending to compile the child \textit{dest}.
This form behaves as if \textit{dest}
issues |\childdocof{|\textit{main}|}| right away,
and no further \textsf{childdoc} directives will be processed.

%%%%%%%%%%%%%%%%%%%%%%%%%%%%%%%%%%%%%%%%
\DescribeMacro{\...prefix}
In the alternative form |\childdocforwardprefix|,
%
\begin{center}
\begin{tabular}{l}
|\input{childdoc.def}|\\
|\childdocforwardprefix[|\textit{main}|]{|\textit{prefix}|}{|\textit{dest}|}|
\end{tabular}
\end{center}
%
the destination file is determined by a pattern
depending on the current file:
To make this work, the current file must be called
`{\textit{prefix}\hspace{0.2em}\textit{suffix}}'
with \textit{prefix} matching precisely the argument.
Processing is then passed on to the file
`{\textit{dest}\hspace{0.2em}\textit{suffix}}'.
Surely, the same effect is achieved by
directly specifying the
argument `{\textit{dest}\hspace{0.2em}\textit{suffix}}'
in the first form.
However, that requires to set up a different file
for each child. With the alternative form of the command
all these files can have exactly the same content
which simplifies setting them up and maintaining them.

For example, the following file |draft.tex|
with a compilation flag |\version| as described in \secref{sec:flags}
compiles the main document as a draft:
%
\begin{center}
\begin{tabular}{l}
|\def\version{draft}|\\
|\input{childdoc.def}|\\
|\childdocforward{|\textit{main}|}|
\end{tabular}
\end{center}
%
Likewise, the following files |final|\textit{nn}|.tex|
compile the final version of the child document
|child|\textit{nn}|.tex|:
%
\begin{center}
\begin{tabular}{l}
|\def\version{final}|\\
|\input{childdoc.def}|\\
|\childdocforwardprefix{final}{child}|
\end{tabular}
\end{center}
%

Note that when several versions of a main file and/or of each child file
are to be generated, it may be convenient to set up a |Makefile| or
shell script to automatise the process.

%%%%%%%%%%%%%%%%%%%%%%%%%%%%%%%%%%%%%%%%%%%%%%%%%%%%%%%%%%%%%%%%%%%%%%%%%%%%%%%%
\subsection{Command Line Processing}
\label{sec:commandline}

The effect of redirection files can also be achieved by invoking
the \LaTeX{} compiler with a more elaborate command line.
Most conveniently this should be done as part
of a shell script or a |Makefile|.

When using \textsf{childdoc} in the main file, the following
command lines effectively perform a redirection
(note that depending on the shell being used,
backslashes may have to be doubled: `|\|' $\to$ `|\\|'):
%
\begin{center}
|... -jobname "|\textit{target}|" |\\|"|[\textit{flags}]%
|\input{childdoc.def}\childdocforward[|\textit{main}|]{|\textit{dest}|}"|
\end{center}
%
Here \textit{target} is the name of the output file,
\textit{main} is the name of the main file
and \textit{dest} is the name of the main or child file to be processed
(all filenames without extensions).
The optional argument \textit{main} can be omitted
if \textit{main} matches \textit{dest}.
Optionally, compilation \textit{flags} can be defined via |\def| commands.
This command line makes the \TeX{} engine believe
it is compiling the file \textit{target}
whose content is specified as the latter parameter.
The provided code then forwards the processing to
\textit{main} or \textit{dest} as described in \secref{sec:forward}.

%%%%%%%%%%%%%%%%%%%%%%%%%%%%%%%%%%%%%%%%%%%%%%%%%%%%%%%%%%%%%%%%%%%%%%%%%%%%%%%%
\subsection{Include by Input}
\label{sec:input}

Including child documents by |\include| has some restrictions by design.
Most notably, the content of a child document always occupies
its own set of pages; pages cannot be shared between child documents.
Usually, this behaviour makes perfect sense
because each child document contain an essential part of the document.
However, in some situations it may be desirable to compose
a document from a collection of parts
without having mandatory page breaks between then.
For this case, the package
provides a mechanism to include parts
by |\input| which can also be processed individually.
However, by construction this mechanism
requires manual handling of the content to be output.

%%%%%%%%%%%%%%%%%%%%%%%%%%%%%%%%%%%%%%%%
\DescribeMacro{\ifchilddocmanual}
The main file should be prepared as usual, see \secref{sec:include}.
However, the document body must make a distinction
between processing of an individual part and of the main document, e.g.:
%
\begin{center}
\begin{tabular}{l}
|\ifchilddocmanual|\\
|\input{\childdocname}|\\
|\||else|\\
\textit{document body with }|\input{|\textit{part}|}|\\
|\||fi|
\end{tabular}
\end{center}
%
The conditional |\ifchilddocmanual| is true whenever
a part to be included by |\input| is being compiled,
and the name of the part is stored in |\childdocname|.

%%%%%%%%%%%%%%%%%%%%%%%%%%%%%%%%%%%%%%%%
\DescribeMacro{\childdocby}
Each part to be included by |\input| should start with:
%
\begin{center}
\begin{tabular}{l}
|\input{childdoc.def}|\\
|\childdocby{|\textit{main}|}|\\
\end{tabular}
\end{center}
%
The directive |\childdocby| is similar to |\childdocof|
described in \secref{sec:include},
but the subsequent selection of content must be done manually.
To that end, both |\ifchilddoc| and |\ifchilddocmanual|
will be true upon processing of a part,
and the name of the part is stored in |\childdocname|.
Note that |\jobname| will be set to the filename of the current part
so that each part receives an individual |.aux| file
that does not interfere with the |.aux| file(s) of the main document.
This behaviour can be altered by the alternative form
|\childdocby[*]{|\textit{main}|}| (with a non-empty optional argument)
which uses the |.aux| file of the main document
by setting |\jobname| to \textit{main}.

%%%%%%%%%%%%%%%%%%%%%%%%%%%%%%%%%%%%%%%%%%%%%%%%%%%%%%%%%%%%%%%%%%%%%%%%%%%%%%%%
\subsection{Driver Development}
\label{sec:driver}

The \textsf{childdoc} mechanism can also be use for the development
of definition files such as \LaTeX{} styles or classes.
This case differs from the above setup with multiple parts
included by |\include| in that no |\includeonly| should be invoked.
This can be achieved by starting the include file
(before |\ProvidesPackage|) with:
%
\begin{center}
\begin{tabular}{l}
|\input{childdoc.def}|\\
|\childdocforward{|\textit{main}|}|\\
\end{tabular}
\end{center}
%
or alternatively with:
%
\begin{center}
\begin{tabular}{l}
|\input{childdoc.def}|\\
|\childdocby{|\textit{main}|}|\\
\end{tabular}
\end{center}
%
Both forms have slightly different effects as described above.
The main file is prepared as usual, see \secref{sec:include}.

%%%%%%%%%%%%%%%%%%%%%%%%%%%%%%%%%%%%%%%%%%%%%%%%%%%%%%%%%%%%%%%%%%%%%%%%%%%%%%%%
\subsection{Legacy Detection}
\label{sec:detection}

The directive |\childdocmain| in the main file can detect
whether the complete document or merely a child is to be compiled
even without using the directive |\childdocof|.
This method is deprecated because it is less robust
and there is no compelling reason to use it;
it is merely provided for backward compatibility
and it may be removed in future versions.

If the detection mechanism is to be used,
it is mandatory to correctly specify
the filename of the main file as the argument of |\childdocmain|:
%
\begin{center}
\begin{tabular}{l}
|\input{childdoc.def}|\\
|\childdocmain{|\textit{main}|}|\\
\end{tabular}
\end{center}
%
If |\jobname| does not match the argument \textit{main} of |\childdocmain|,
it is assumed that |\jobname| points to the child file to be compiled.
When using |\childdocmain| with the main file specified as argument,
it suffices to start a child file
with just |\input{|\textit{main}|}|
without loading of the package and using |\childdocof|.
If instead all processing is done
with the appropriate \textsf{childdoc} directives,
the argument of \textit{main} of |\childdocmain| can be empty.

An alternative version of the command line processing described
in \secref{sec:commandline} using the detection mechanism reads:
%
\begin{center}
|... -jobname "|\textit{target}|" "|[\textit{flags}]%
[|\def\jobname{|\textit{dest}|}|]|\input{|\textit{main}|}"|
\end{center}

%%%%%%%%%%%%%%%%%%%%%%%%%%%%%%%%%%%%%%%%%%%%%%%%%%%%%%%%%%%%%%%%%%%%%%%%%%%%%%%%
\subsection{Manual Code}
\label{sec:manual}

In case one cannot be certain whether the definitions file |childdoc.def|
is installed on the target \TeX{} distribution
and one prefers not to ship it,
it is conceivable to paste a few relevant commands into the sources.

To that end, drop all statements |\input{childdoc.def}|
and perform the replacements as outlined below.
Instead of |\childdocmain{|\textit{main}|}| add the following code
to the top of the main file:
%
\begin{center}
\begin{tabular}{l}
|\||ifdefined\childdocname\endinput\||fi\newif\ifchilddoc|\\
|\edef\childdocname{\scantokens\expandafter{\jobname\noexpand}}|\\
|\def\childdocmain{|\textit{main}|}\||ifx\childdocmain\childdocname\||else|\\
|\childdoctrue\includeonly{\childdocname}\let\jobname\childdocmain\||fi|\\
\end{tabular}
\end{center}
%
Instead of |\childdocof{|\textit{main}|}| just include the main file
at the top of each child file:
%
\begin{center}
|\input{|\textit{main}|}|
\end{center}
%
A simple redirection |\childdocforward{|\textit{dest}|}| is achieved by:
%
\begin{center}
|\def\jobname{|\textit{dest}|}\input{\jobname}|
\end{center}
%
The redirection with prefix
|\childdocforwardprefix[|\textit{prefix}|]{|\textit{dest}|}|
is accomplished by:
%
\begin{center}
\begin{tabular}{l}
|{\edef\jobname{\scantokens\expandafter{\jobname\noexpand}}|\\
|\def\redirectjob |\textit{prefix}|#1~~~{\gdef\jobname{|\textit{dest}|#1}}|\\
|\expandafter\redirectjob\jobname~~~}\input{\jobname}|
\end{tabular}
\end{center}

In an alternative approach,
child documents can be compiled by a specific command line
without additional code or specific definitions:
%
\begin{center}
|... -jobname "|\textit{target}|" "|[\textit{flags}]%
|\includeonly{|\textit{dest}|}\input{|\textit{main}|}"|
\end{center}
%

%%%%%%%%%%%%%%%%%%%%%%%%%%%%%%%%%%%%%%%%%%%%%%%%%%%%%%%%%%%%%%%%%%%%%%%%%%%%%%%%
%%%%%%%%%%%%%%%%%%%%%%%%%%%%%%%%%%%%%%%%%%%%%%%%%%%%%%%%%%%%%%%%%%%%%%%%%%%%%%%%
\section{Information}

%%%%%%%%%%%%%%%%%%%%%%%%%%%%%%%%%%%%%%%%%%%%%%%%%%%%%%%%%%%%%%%%%%%%%%%%%%%%%%%%
\subsection{Copyright}

Copyright \copyright{} 2017--2018 Niklas Beisert

This work may be distributed and/or modified under the
conditions of the \LaTeX{} Project Public License, either version 1.3
of this license or (at your option) any later version.
The latest version of this license is in
  \url{http://www.latex-project.org/lppl.txt}
and version 1.3 or later is part of all distributions of \LaTeX{}
version 2005/12/01 or later.

This work has the LPPL maintenance status `maintained'.

The Current Maintainer of this work is Niklas Beisert.

This work consists of the files |README.txt|, |childdoc.ins| and |childdoc.dtx|
as well as the derived files |childdoc.def|, |cdocsamp.tex|
with |cdocsch1.tex|, |cdocsch2.tex|, |cdocspt3.tex|, |cdocspt4.tex|,
|cdocsdrf.tex|, |cdocsfn1.tex|, |cdocsfn2.tex|
as well as |childdoc.pdf|.

%%%%%%%%%%%%%%%%%%%%%%%%%%%%%%%%%%%%%%%%%%%%%%%%%%%%%%%%%%%%%%%%%%%%%%%%%%%%%%%%
\subsection{Files and Installation}

The package consists of the files:
%
\begin{center}
\begin{tabular}{ll}
    |README.txt|   & readme file \\
    |childdoc.ins| & installation file \\
    |childdoc.dtx| & source file \\
    |childdoc.def| & definition file \\
    |cdocsamp.tex| & sample main file \\
    |cdocsch1.tex| & sample include file \\
    |cdocsch2.tex| & sample include file \\
    |cdocspt3.tex| & sample part file \\
    |cdocspt4.tex| & sample part file \\
    |cdocsdrf.tex| & sample redirection file \\
    |cdocsfn1.tex| & sample redirection file \\
    |cdocsfn2.tex| & sample redirection file \\
    |childdoc.pdf| & manual
\end{tabular}
\end{center}
%
The distribution consists of the files
|README.txt|, |childdoc.ins| and |childdoc.dtx|.
%
\begin{itemize}
\item
Run (pdf)\LaTeX{} on |childdoc.dtx|
to compile the manual |childdoc.pdf| (this file).
\item
Run \LaTeX{} on |childdoc.ins| to create the definitions file |childdoc.def|
and the sample |cdocsamp.tex| with include files
|cdocsch1.tex|, |cdocsch2.tex|, |cdocspt3.tex|, |cdocspt4.tex|,
|cdocsdrf.tex|, |cdocsfn1.tex|, |cdocsfn2.tex|.
Then copy the file |childdoc.def| to an appropriate directory of your \LaTeX{}
distribution, e.g.\ \textit{texmf-root}|/tex/latex/childdoc|.
\end{itemize}

%%%%%%%%%%%%%%%%%%%%%%%%%%%%%%%%%%%%%%%%%%%%%%%%%%%%%%%%%%%%%%%%%%%%%%%%%%%%%%%%
\subsection{Related CTAN Packages}

There are several other packages which offer a similar functionality:
%
\begin{itemize}
\item
The packages
\href{http://ctan.org/pkg/docmute}{\textsf{docmute}},
\href{http://ctan.org/pkg/includex}{\textsf{includex}} and
\href{http://ctan.org/pkg/standalone}{\textsf{standalone}}
provide commands to include only the document body of
a child file thus allowing both files to be compiled individually.
\item
The packages \href{http://ctan.org/pkg/subdocs}{\textsf{subdocs}}
and \href{http://ctan.org/pkg/subfiles}{\textsf{subfiles}}
provide structures in which the main and child documents can be
encapsulated and allowing them to be compiled individually.
The inclusion mechanism is different from the conventional |\include|.
\item
The package \href{http://ctan.org/pkg/combine}{\textsf{combine}}
is an elaborate solution to combine several documents into one.
\end{itemize}
%
See also the CTAN topic \href{http://ctan.org/topic/subdocs}{\textsf{subdocs}}
for further related packages.
The present package differs from the above solutions in that
a document structure constructed with the conventional |\include| mechanism
just needs two extra commands at the top of every file
such that all constituent files can be compiled individually.

%%%%%%%%%%%%%%%%%%%%%%%%%%%%%%%%%%%%%%%%%%%%%%%%%%%%%%%%%%%%%%%%%%%%%%%%%%%%%%%%
%\subsection{Feature Suggestions}
%
%The following is a list of features which may be useful for future
%versions of this package:
%%
%\begin{itemize}
%\item
%\ldots
%\end{itemize}

%%%%%%%%%%%%%%%%%%%%%%%%%%%%%%%%%%%%%%%%%%%%%%%%%%%%%%%%%%%%%%%%%%%%%%%%%%%%%%%%
\subsection{Revision History}

%%%%%%%%%%%%%%%%%%%%%%%%%%%%%%%%%%%%%%%%
\paragraph{v2.0:} 2018/12/30

\begin{itemize}
\item
immediate forward processing
\item
added |\childdocby| mechanism
\item
manual restructured
\end{itemize}

%%%%%%%%%%%%%%%%%%%%%%%%%%%%%%%%%%%%%%%%
\paragraph{v1.6:} 2018/01/17

\begin{itemize}
\item
application for development of include files
\item
corrections to manual
\end{itemize}

%%%%%%%%%%%%%%%%%%%%%%%%%%%%%%%%%%%%%%%%
\paragraph{v1.5:} 2017/05/21

\begin{itemize}
\item
more complete structuring introduced
\item
|\childdocof| introduced
\item
|\childdoc| renamed to |\childdocmain|
\item
|\childredirect| renamed to |\childdocforward| and |\childdocforwardprefix|
and functionality expanded
\end{itemize}

%%%%%%%%%%%%%%%%%%%%%%%%%%%%%%%%%%%%%%%%
\paragraph{v1.0:} 2017/04/27

\begin{itemize}
\item
manual and install package
\item
first version published on CTAN
\end{itemize}

%%%%%%%%%%%%%%%%%%%%%%%%%%%%%%%%%%%%%%%%
\paragraph{v0.6:} 2017/04/26

\begin{itemize}
\item
redirection mechanism added
\end{itemize}

%%%%%%%%%%%%%%%%%%%%%%%%%%%%%%%%%%%%%%%%
\paragraph{v0.5:} 2017/04/26

\begin{itemize}
\item
functionality in definition file
\end{itemize}


%%%%%%%%%%%%%%%%%%%%%%%%%%%%%%%%%%%%%%%%%%%%%%%%%%%%%%%%%%%%%%%%%%%%%%%%%%%%%%%%
%%%%%%%%%%%%%%%%%%%%%%%%%%%%%%%%%%%%%%%%%%%%%%%%%%%%%%%%%%%%%%%%%%%%%%%%%%%%%%%%
%%%%%%%%%%%%%%%%%%%%%%%%%%%%%%%%%%%%%%%%%%%%%%%%%%%%%%%%%%%%%%%%%%%%%%%%%%%%%%%%
\appendix

\settowidth\MacroIndent{\rmfamily\scriptsize 000\ }

 \DocInput{childdoc.dtx}

\end{document}
%</driver>
% \fi
%
% %%%%%%%%%%%%%%%%%%%%%%%%%%%%%%%%%%%%%%%%%%%%%%%%%%%%%%%%%%%%%%%%%%%%%%%%%%%%%%
% %%%%%%%%%%%%%%%%%%%%%%%%%%%%%%%%%%%%%%%%%%%%%%%%%%%%%%%%%%%%%%%%%%%%%%%%%%%%%%
% \section{Sample}
%\iffalse
%<*samplemain>
%\fi
%
% The following presents a sample document
% with two chapters, two parts, a title page,
% a compile flag as well as three forwarding files to set the flag.
% It consists of eight |.tex| files:
% \begin{center}
% \begin{tabular}{ll}
% |cdocsamp.tex|&main file\\
% |cdocsch1.tex|&include file for chapter 1\\
% |cdocsch2.tex|&include file for chapter 2\\
% |cdocspt3.tex|&include file for part 3\\
% |cdocspt4.tex|&include file for part 4\\
% |cdocsdrf.tex|&forwarding file for main file in draft mode\\
% |cdocsfi1.tex|&forwarding file for final version of chapter 1\\
% |cdocsfi2.tex|&forwarding file for final version of chapter 2\\
% \end{tabular}
% \end{center}
% Each of the eight files can be compiled directly by the \LaTeX{} compiler.
%
% %%%%%%%%%%%%%%%%%%%%%%%%%%%%%%%%%%%%%%
% \paragraph{Main File.}
%
% The main file is called |cdocsamp.tex|.
%
% Load the \textsf{childdoc} definitions and
% declare the filename for the main document:
%    \begin{macrocode}
\input{childdoc.def}
\childdocmain{}
%    \end{macrocode}

% Optional override for |\version| flag:
%    \begin{macrocode}
%%\ifchilddoc\else\providecommand{\version}{draft}\fi
%    \end{macrocode}

% Define the default values for the |\version| flag
% (|final| for the main file and |draft| for childs):
%    \begin{macrocode}
\ifchilddoc
\providecommand{\version}{draft}
\else
\providecommand{\version}{final}
\fi
%    \end{macrocode}

% Load the standard document class:
%    \begin{macrocode}
\documentclass[12pt]{article}
%    \end{macrocode}

% Start the document body:
%    \begin{macrocode}
\begin{document}
%    \end{macrocode}

% Declare a title page.
% Print title, part of document being processed and version flag:
%    \begin{macrocode}
\addtocounter{page}{-1}
\begin{center}
{\LARGE\bfseries{}childdoc example\par}
\vspace{1cm}
\ifchilddoc
\ifchilddocmanual part\else chapter\fi:
`\childdocname' of `\childdocjob'\par
\else
main document: `\childdocjob'\par
\fi
version: \version\par
\end{center}
\newpage
%    \end{macrocode}

% Manually include selected file,
% otherwise process as usual:
%    \begin{macrocode}
\ifchilddocmanual
\section*{part `\childdocname'}
\input{\childdocname}
\else
%    \end{macrocode}

% Include the two chapters:
%    \begin{macrocode}
\include{cdocsch1}
\include{cdocsch2}
%    \end{macrocode}

% Include the two parts unless only chapters should be displayed:
%    \begin{macrocode}
\ifchilddoc\else
\section{part three}
\input{cdocspt3}
\section{part four}
\input{cdocspt4}
\fi
%    \end{macrocode}

% Process as usual until here:
%    \begin{macrocode}
\fi
%    \end{macrocode}

% End of document body:
%    \begin{macrocode}
\end{document}
%    \end{macrocode}
%\iffalse
%</samplemain>
%\fi
%
% %%%%%%%%%%%%%%%%%%%%%%%%%%%%%%%%%%%%%%
% \paragraph{Chapter Include Files.}
%
% The include files are called |cdocsch1.tex| and |cdocsch2.tex|.
%
%\iffalse
%<*samplechap1|samplechap2>
%\fi

% Optional override for |\version| flag:
%    \begin{macrocode}
%%\providecommand{\version}{final}
%    \end{macrocode}

% Include the main document:
%    \begin{macrocode}
\input{childdoc.def}
\childdocof{cdocsamp}
%    \end{macrocode}

%\iffalse
%</samplechap1|samplechap2>
%\fi
%
%\iffalse
%<*samplechap1>
%\fi
% Some text for chapter 1:
%    \begin{macrocode}
\section{one}
some text in chapter one
%    \end{macrocode}

%\iffalse
%</samplechap1>
%\fi
% Some text for chapter 2:
%\iffalse
%<*samplechap2>
%\fi
%    \begin{macrocode}
\section{two}
more text in chapter two
%    \end{macrocode}

%\iffalse
%</samplechap2>
%\fi
%
% %%%%%%%%%%%%%%%%%%%%%%%%%%%%%%%%%%%%%%
% \paragraph{Part Include Files.}
%
% The include files are called |cdocspt3.tex| and |cdocspt4.tex|.
%
%\iffalse
%<*samplepart3|samplepart4>
%\fi

% Optional override for |\version| flag:
%    \begin{macrocode}
%%\providecommand{\version}{final}
%    \end{macrocode}

% Include the main document:
%    \begin{macrocode}
\input{childdoc.def}
\childdocby{cdocsamp}
%    \end{macrocode}

%\iffalse
%</samplepart3|samplepart4>
%\fi
%
%\iffalse
%<*samplepart3>
%\fi
% Some text for part 3:
%    \begin{macrocode}
some text in part three
%    \end{macrocode}

%\iffalse
%</samplepart3>
%\fi
% Some text for part 4:
%\iffalse
%<*samplepart4>
%\fi
%    \begin{macrocode}
more text in part four
%    \end{macrocode}

%\iffalse
%</samplepart4>
%\fi
%
% %%%%%%%%%%%%%%%%%%%%%%%%%%%%%%%%%%%%%%
% \paragraph{Forwarding for a Complete Draft.}
%
% The following forwarding file |cdocsdrf.tex|
% compiles the main document in draft mode:
%\iffalse
%<*sampledraft>
%\fi
%    \begin{macrocode}
\def\version{draft}
\input{childdoc.def}
\childdocforward{cdocsamp}
%    \end{macrocode}

%\iffalse
%</sampledraft>
%\fi
%
% %%%%%%%%%%%%%%%%%%%%%%%%%%%%%%%%%%%%%%
% \paragraph{Forwarding for Final Version of the Chapters.}
%
% The following forwarding files |cdocsfn1.tex| and |cdocsfn2.tex|
% (with identical content)
% compile the final versions of the child documents
% |cdocsch1.tex| and |cdocsch2.tex|, respectively:
%\iffalse
%<*samplefinal>
%\fi
%    \begin{macrocode}
\def\version{final}
\input{childdoc.def}
\childdocforwardprefix[cdocsamp]{cdocsfn}{cdocsch}
%    \end{macrocode}

%\iffalse
%</samplefinal>
%\fi
%
% %%%%%%%%%%%%%%%%%%%%%%%%%%%%%%%%%%%%%%
% \paragraph{Command Line Processing.}
%
% The following three command lines generate the output files
% |cdocscld|, |cdocscl1| and |cdocscl2|
% which should be identical to
% |cdocsdrf|, |cdocsch1| and |cdocsfn2|, respectively:
% \begin{center}
% \begin{tabular}{l}
% |latex -jobname cdocscld \|\\
% |  "\def\version{draft}\input{childdoc.def}\childdocforward{cdocsamp}"|\\
% |latex -jobname cdocscl1 \|\\
% |  "\input{childdoc.def}\childdocforward[cdocsamp]{cdocsch1}"|\\
% |latex -jobname cdocscl2 \|\\
% |  "\def\version{final}\input{childdoc.def}\childdocforward{cdocsch2}"|
% \end{tabular}
% \end{center}
% Note that the trailing backslash on each first line
% merely continues the input to the second line
% (for convenient cut ant paste).
% Furthermore, the command |latex| can be replaced by any
% of its alternative versions such as |pdflatex|.
%
% %%%%%%%%%%%%%%%%%%%%%%%%%%%%%%%%%%%%%%%%%%%%%%%%%%%%%%%%%%%%%%%%%%%%%%%%%%%%%%
% %%%%%%%%%%%%%%%%%%%%%%%%%%%%%%%%%%%%%%%%%%%%%%%%%%%%%%%%%%%%%%%%%%%%%%%%%%%%%%
% \section{Implementation}
%\iffalse
%<*package>
%\fi
%
% This section describes the definitions file |childdoc.def|.

% The definitions cannot be loaded using |\usepackage| or |\RequirePackage|
% which has a mechanism to prevent loading a style file more than once.
% When loading the definitions by means of |\input|
% multiple instances have to be prevented manually:
%\iffalse
%This code needs to be before the `\ProvidesFile' directive
%which is defined at the beginning of this file.
%Therefore it is also placed there and commented out here.
%</package>
%<*discard>
%\fi
%    \begin{macrocode}
\ifdefined\childdocmain\endinput\fi
%    \end{macrocode}
%\iffalse
%</discard>
%<*package>
%\fi
%
% \macro{\ifchilddoc}
% \macro{\ifchilddocmanual}
% The conditional |\ifchilddoc| tells whether a
% child (true) or main (false) document is being compiled.
% The conditional |\ifchilddocmanual| tells whether
% the |\includeonly| mechanism is used (false) or
% the selection of child files must be performed manually (true).
% The definitions initialise to false:
%    \begin{macrocode}
\newif\ifchilddoc
\newif\ifchilddocmanual
%    \end{macrocode}

% \macro{\childdocname}
% \macro{\childdocjob}
% The macro |\childdocname| stores the name of the main document
% to be compiled. The macro |\childdocjob| stores the name of
% the document on which the \LaTeX{} compiler was originally invoked.
% The content of |\jobname| cannot be compared
% to filenames specified in the source due to different catcodes.
% The following code rescans |\jobname|, stores the result
% in |\childdocname| and saves a copy in |\childdocjob|:
%    \begin{macrocode}
\edef\childdocname{\scantokens\expandafter{\jobname\noexpand}}
\let\childdocjob\childdocname
%    \end{macrocode}

% \macro{\childdocdisable}
% The macro |\childdocdisable| prevents the main file
% from being processed more than once.
% At this stage, the main document command |\childdocmain|
% is assumed to be called once again where it should do nothing.
% Any subsequent call to it should prevent
% a secondary processing of the main document
% It overwrites the forwarding commands
% |\childdocof| and |\childdocforward|
% with empty macros to prevent further inclusions of the main document:
%    \begin{macrocode}
\newcommand{\childdocdisable}
{
  \renewcommand{\childdocmain}[1]{\renewcommand{\childdocmain}[1]{\endinput}}
  \renewcommand{\childdocof}[1]{}
  \renewcommand{\childdocby}[2][]{}
  \renewcommand{\childdocforward}[2][]{}
  \renewcommand{\childdocdisable}{}
}
%    \end{macrocode}

% \macro{\childdocmain}
% The macro |\childdocmain| is to be called at the top of the main file
% with nothing or the main filename (without extension) as argument.
% First, it breaks loops.
% If the argument is not empty and does not match |\childdocname|
% (which is set by the first inclusion of |childdoc.def|),
% |\ifchilddoc| is set to true, |\includeonly| is applied to the child file
% and |\jobname| is set to the main file
% (for proper handling of |.aux| files):
%    \begin{macrocode}
\newcommand{\childdocmain}[1]
{
  \childdocdisable\childdocmain{}
  \if?#1?\else
    \begingroup
      \def\childdoctmp{#1}
      \ifx\childdoctmp\childdocname
        \def\childdoctmp{}
      \else
        \def\childdoctmp
        {
          \childdoctrue
          \includeonly{\childdocname}
          \def\childdocjob{#1}
          \def\jobname{#1}
        }
      \fi
      \expandafter
    \endgroup
    \childdoctmp
  \fi
}
%    \end{macrocode}

% \macro{\childdocof}
% The command |\childdocof| redirects
% compilation to the main file |#1|.
%    \begin{macrocode}
\newcommand{\childdocof}[1]
{
  \childdocdisable
  \childdoctrue
  \includeonly{\childdocname}
  \def\jobname{#1}
  \def\childdocjob{#1}
  \input{#1}
}
%    \end{macrocode}

% \macro{\childdocby}
% The command |\childdocby| ....
%    \begin{macrocode}
\newcommand{\childdocby}[2][]
{
  \childdocdisable
  \childdoctrue
  \childdocmanualtrue
  \if?#1?\else
    \def\jobname{#2}
  \fi
  \def\childdocjob{#2}
  \input{#2}
  \endinput
}
%    \end{macrocode}

% \macro{\childdocforward}
% The command |\childdocforward| redirects
% compilation to the main file or
% (if the optional argument is given) a child file.
% Parameters are set as if the main file
% or a child file starting with |\childdocof| was compiled.
% Then compilation is handed over to the main file:
%    \begin{macrocode}
\newcommand{\childdocforward}[2][]
{
  \begingroup
    \if?#1?
      \def\childdoctmp
      {
        \def\childdocname{#2}
        \def\childdocjob{#2}
        \def\jobname{#2}
        \input{#2}
        \endinput
      }
    \else
      \def\childdoctmp
      {
        \childdocdisable
        \def\childdocname{#2}
        \childdoctrue
        \includeonly{#2}
        \def\childdocjob{#1}
        \def\jobname{#1}
        \input{#1}
        \endinput
      }
    \fi
    \expandafter
  \endgroup
  \childdoctmp
}
%    \end{macrocode}

% \macro{\childdocforwardprefix}
% The command |\childdocforwardprefix| redirects
% compilation to the main or a child file by means of a pattern.
% The prefix |#1| in the current filename is replaced by |#2|
% and the suffix of the current filename is kept
% (it is assumed that the filename does not contain the substring `|~~~|'
% which is used as a delimiter).
% Compilation is handed over to the new file by |\childdocforward|:
%    \begin{macrocode}
\newcommand{\childdocforwardprefix}[3][]
{
  \begingroup
    \def\childdocextract #2##1~~~{\def\childdoctmp{\childdocforward[#1]{#3##1}}}
    \expandafter\childdocextract\childdocname~~~
    \expandafter
  \endgroup
  \childdoctmp
}
%    \end{macrocode}

% \macro{\childdoc}
% The deprecated macro |\childdoc| is a legacy version of |\childdocmain|:
%    \begin{macrocode}
\newcommand{\childdoc}{\childdocmain}
%    \end{macrocode}

% \macro{\childdocredirect}
% The deprecated macro |\childdocredirect| is a legacy version
% of |\childdocforward| and |\childdocforwardprefix|:
%    \begin{macrocode}
\newcommand{\childdocredirect}[2][]
{
  \begingroup
    \if?#1?
      \def\childdoctmp{\childdocforward{#2}}
    \else
      \def\childdoctmp{\childdocforwardprefix{#1}{#2}}
    \fi
    \expandafter
  \endgroup
  \childdoctmp
}
%    \end{macrocode}

%\iffalse
%</package>
%\fi
%
\endinput

\childdocmain{}
%    \end{macrocode}

% Optional override for |\version| flag:
%    \begin{macrocode}
%%\ifchilddoc\else\providecommand{\version}{draft}\fi
%    \end{macrocode}

% Define the default values for the |\version| flag
% (|final| for the main file and |draft| for childs):
%    \begin{macrocode}
\ifchilddoc
\providecommand{\version}{draft}
\else
\providecommand{\version}{final}
\fi
%    \end{macrocode}

% Load the standard document class:
%    \begin{macrocode}
\documentclass[12pt]{article}
%    \end{macrocode}

% Start the document body:
%    \begin{macrocode}
\begin{document}
%    \end{macrocode}

% Declare a title page.
% Print title, part of document being processed and version flag:
%    \begin{macrocode}
\addtocounter{page}{-1}
\begin{center}
{\LARGE\bfseries{}childdoc example\par}
\vspace{1cm}
\ifchilddoc
\ifchilddocmanual part\else chapter\fi:
`\childdocname' of `\childdocjob'\par
\else
main document: `\childdocjob'\par
\fi
version: \version\par
\end{center}
\newpage
%    \end{macrocode}

% Manually include selected file,
% otherwise process as usual:
%    \begin{macrocode}
\ifchilddocmanual
\section*{part `\childdocname'}
\input{\childdocname}
\else
%    \end{macrocode}

% Include the two chapters:
%    \begin{macrocode}
\include{cdocsch1}
\include{cdocsch2}
%    \end{macrocode}

% Include the two parts unless only chapters should be displayed:
%    \begin{macrocode}
\ifchilddoc\else
\section{part three}
\input{cdocspt3}
\section{part four}
\input{cdocspt4}
\fi
%    \end{macrocode}

% Process as usual until here:
%    \begin{macrocode}
\fi
%    \end{macrocode}

% End of document body:
%    \begin{macrocode}
\end{document}
%    \end{macrocode}
%\iffalse
%</samplemain>
%\fi
%
% %%%%%%%%%%%%%%%%%%%%%%%%%%%%%%%%%%%%%%
% \paragraph{Chapter Include Files.}
%
% The include files are called |cdocsch1.tex| and |cdocsch2.tex|.
%
%\iffalse
%<*samplechap1|samplechap2>
%\fi

% Optional override for |\version| flag:
%    \begin{macrocode}
%%\providecommand{\version}{final}
%    \end{macrocode}

% Include the main document:
%    \begin{macrocode}
% \iffalse
%
% childdoc.dtx Copyright (C) 2017-2018 Niklas Beisert
%
% This work may be distributed and/or modified under the
% conditions of the LaTeX Project Public License, either version 1.3
% of this license or (at your option) any later version.
% The latest version of this license is in
%   http://www.latex-project.org/lppl.txt
% and version 1.3 or later is part of all distributions of LaTeX
% version 2005/12/01 or later.
%
% This work has the LPPL maintenance status `maintained'.
%
% The Current Maintainer of this work is Niklas Beisert.
%
% This work consists of the files childdoc.dtx and childdoc.ins
% and the derived files childdoc.def and cdocsamp.tex with
% cdocsch1.tex, cdocsch2.tex, cdocsdrf.tex, cdocsfn1.tex, cdocsfn2.tex.
%
%<package>\ifdefined\childdocmain\endinput\fi
%<package>\ProvidesFile{childdoc.def}[2018/12/30 v2.0 child document driver]
%<samplemain>\ProvidesFile{cdocsamp.tex}[2018/12/30 v2.0 sample for childdoc]
%<*driver>
%\ProvidesFile{childdoc.drv}[2018/12/30 v2.0 childdoc reference manual file]
\PassOptionsToClass{10pt,a4paper}{article}
\documentclass{ltxdoc}

\usepackage[margin=35mm]{geometry}
\usepackage{hyperref}
\usepackage{hyperxmp}
\usepackage[usenames]{color}

\hypersetup{colorlinks=true}
\hypersetup{pdfstartview=FitH}
\hypersetup{pdfpagemode=UseNone}
\hypersetup{pdfsource={}}
\hypersetup{pdflang={en-UK}}
\hypersetup{pdfcopyright={Copyright 2017-2018 Niklas Beisert.
  This work may be distributed and/or modified under the
  conditions of the LaTeX Project Public License, either version 1.3
  of this license or (at your option) any later version.}}
\hypersetup{pdflicenseurl={http://www.latex-project.org/lppl.txt}}
\hypersetup{pdfcontactaddress={ETH Zurich, ITP, HIT K,
  Wolfgang-Pauli-Strasse 27}}
\hypersetup{pdfcontactpostcode={8093}}
\hypersetup{pdfcontactcity={Zurich}}
\hypersetup{pdfcontactcountry={Switzerland}}
\hypersetup{pdfcontactemail={nbeisert@itp.phys.ethz.ch}}
\hypersetup{pdfcontacturl={http://people.phys.ethz.ch/\xmptilde nbeisert/}}

\newcommand{\secref}[1]{\hyperref[#1]{section \ref*{#1}}}

\parskip1ex
\parindent0pt
\let\olditemize\itemize
\def\itemize{\olditemize\parskip0pt}

\begin{document}

\title{The \textsf{childdoc} Package}
\hypersetup{pdftitle={The childdoc Package}}
\author{Niklas Beisert\\[2ex]
  Institut f\"ur Theoretische Physik\\
  Eidgen\"ossische Technische Hochschule Z\"urich\\
  Wolfgang-Pauli-Strasse 27, 8093 Z\"urich, Switzerland\\[1ex]
  \href{mailto:nbeisert@itp.phys.ethz.ch}
  {\texttt{nbeisert@itp.phys.ethz.ch}}}
\hypersetup{pdfauthor={Niklas Beisert}}
\hypersetup{pdfsubject={Manual for the LaTeX2e Package childdoc}}
\date{30 December 2018, \textsf{v2.0}}
\maketitle

\begin{abstract}\noindent
\textsf{childdoc} is a \LaTeXe{} package
that enables the direct compilation
of document sections included by |\include|
to individual files.
\end{abstract}

\begingroup
\parskip0ex
\tableofcontents
\endgroup

%%%%%%%%%%%%%%%%%%%%%%%%%%%%%%%%%%%%%%%%%%%%%%%%%%%%%%%%%%%%%%%%%%%%%%%%%%%%%%%%
%%%%%%%%%%%%%%%%%%%%%%%%%%%%%%%%%%%%%%%%%%%%%%%%%%%%%%%%%%%%%%%%%%%%%%%%%%%%%%%%
\section{Introduction}

\LaTeX{} provides a mechanism to structure a large document (such as a book)
into a main file and several child files (containing the chapters)
using the |\include| command.
This mechanism is beneficial for documents
which span hundreds of pages in order to
make the source file(s) more manageable.
Moreover, compilation can be restricted to
selected child files by means of the |\includeonly| command.
The latter feature can be used to reduce the compilation time while editing
(this was significantly more useful in the earlier days of \LaTeX{})
or to generate a smaller document which is easier to navigate.
Another application of |\includeonly| is to generate
documents consisting of selected parts of the complete document.

However, there are a few drawbacks of the plain |\include| mechanism:
\begin{itemize}
\item
The child files cannot be compiled on their own,
they can only be compiled via the main file.
A naive editing environment
(such as a text editor with an option
to have the current file processed by \LaTeX)
may require one to switch to the main file before compiling;
attempting to compile the child file produces errors.
\item
The main file must be modified (each time)
to adjust the |\includeonly| command
to the present needs. This easily leaves the main file in a messy state.
\item
The generated document will always carry the filename
of the main document. This is inconvenient if
several child files are to be compiled and
to be kept for distribution.
\end{itemize}

The present package provides a simple interface
to make child files individually compilable by \LaTeX{}.
Compiling a child file then has the same effect as compiling
the main file with an |\includeonly| command
to select the appropriate child.
Moreover the generated document will carry the name of the child
rather than the main file.
This resolves all three above issues.

This feature is meant to make the editing of books,
thesis documents and lecture notes somewhat more convenient.
However, the package can also be used efficiently for
composing a series of documents (such as exercise sheets)
which are typically distributed individually.
It then assists the author in generating the individual documents
(potentially in different versions)
as well as a document containing the collected series.
Another application is in developing style files
or other kinds of included material
where compilation of the style file could redirect
to a sample or test file.

%%%%%%%%%%%%%%%%%%%%%%%%%%%%%%%%%%%%%%%%%%%%%%%%%%%%%%%%%%%%%%%%%%%%%%%%%%%%%%%%
%%%%%%%%%%%%%%%%%%%%%%%%%%%%%%%%%%%%%%%%%%%%%%%%%%%%%%%%%%%%%%%%%%%%%%%%%%%%%%%%
\section{Usage}

First of all, the package \textsf{childdoc} is \emph{not} a standard
\LaTeXe{} |.sty| style file! Therefore it needs to be invoked in
a non-standard way.

%%%%%%%%%%%%%%%%%%%%%%%%%%%%%%%%%%%%%%%%%%%%%%%%%%%%%%%%%%%%%%%%%%%%%%%%%%%%%%%%
\subsection{Included Files}
\label{sec:include}

%%%%%%%%%%%%%%%%%%%%%%%%%%%%%%%%%%%%%%%%
\DescribeMacro{\childdocmain}
To use the package, add the commands
\begin{center}
\begin{tabular}{l}
|\input{childdoc.def}|\\
|\childdocmain{}|\\
\end{tabular}
\end{center}
at the very top of the main \LaTeX{} file,
in particular \emph{before} the |\documentclass| statement!
The argument of |\childdocmain| should be left empty
(but it must be present).

%%%%%%%%%%%%%%%%%%%%%%%%%%%%%%%%%%%%%%%%
\DescribeMacro{\childdocof}
Furthermore, add the commands
\begin{center}
\begin{tabular}{l}
|\input{childdoc.def}|\\
|\childdocof{|\textit{main}|}|\\
\end{tabular}
\end{center}
at the top of every child file \textit{child}
which is included by |\include{|\textit{child}|}|
from within the main file
(or at least for those files to be compiled individually).
The argument \textit{main} must be the filename of the main file.

There are a couple of
considerations in setting up the main and child documents:

%%%%%%%%%%%%%%%%%%%%%%%%%%%%%%%%%%%%%%%%
\paragraph{Restrictions.}

Please note the following restrictions:
\begin{itemize}
\item
|\childdocmain| must be called with one argument \textit{main}
to ensure compatibility with earlier version of the package.
It must either be empty (|\childdocmain{}|)
or precisely match the filename of the main file in which it is specified.
See \secref{sec:detection} for further information.
\item
The filename \textit{main} must be specified without the |.tex| extension.
\item
The filename \textit{main} is case sensitive
(even in case-insensitive file systems)
due to internal string comparison.
\item
The argument \textit{main} should be fully expanded, it cannot be a macro.
\item
Subdirectories and special characters should be avoided in filenames.
\item
The command |\childdocmain{|\textit{main}|}| must be followed by a whitespace.
It should not be followed immediately by another command
or by a comment mark `|%|'.
This is because the \TeX{} parser reads the token immediately following
the argument of |\childdocmain| and puts it
at the beginning of every child section;
however, a white\-space is ignored.
\end{itemize}

%%%%%%%%%%%%%%%%%%%%%%%%%%%%%%%%%%%%%%%%
\paragraph{Content of Main File.}

It is advisable to place all content in the child files included by |\include|.
Any output contained in the main file will appear in all child documents
unless suppressed manually;
it cannot be suppressed automatically by the |\includeonly| directive
and thus should normally be avoided.
A method to include some content in the main file
by means of conditional processing is described in \secref{sec:conditional}.

%%%%%%%%%%%%%%%%%%%%%%%%%%%%%%%%%%%%%%%%
\paragraph{Page Numbering.}

When only a part of the document is compiled,
the appropriate numbering of pages
(as well as other status parameters)
is determined from the |.aux| files.
The latter contain information from previous passes.
However this information needs to propagate through
all intermediate child documents.
Therefore the page numbering in child documents may well
be inconsistent until the complete document is compiled at least once.

A useful (if unconventional) way to always ensure a consistent
page numbering is to restart the numbering in each child document
and denote the pages by `\textit{child}|.|\textit{page}'
where \textit{child} represents the chapter/section number of the child file.
This can be achieved by the command
|\numberwithin{page}{|\textit{child}|}|
of the \textsf{amsmath} package
where \textit{child} can be |chapter| or |section|
depending on the chosen structuring.
Alternatively, one can modify the macro |\thepage| appropriately
and reset the counter |page| at the start of each child file.

%%%%%%%%%%%%%%%%%%%%%%%%%%%%%%%%%%%%%%%%%%%%%%%%%%%%%%%%%%%%%%%%%%%%%%%%%%%%%%%%
\subsection{Conditional Processing}
\label{sec:conditional}

The package provides a mechanism to compile different versions
of a document. To customise the versions further some conditional processing
can come in handy to distinguish which version is being compiled.
The package provides two macros to describe the compilation context:

%%%%%%%%%%%%%%%%%%%%%%%%%%%%%%%%%%%%%%%%
\DescribeMacro{\ifchilddoc}
The conditional |\ifchilddoc| distinguishes between the compilation of
child documents and the main document:
%
\begin{center}
|\ifchilddoc |\textit{child-code}| |[|\||else |\textit{main-code}]| \||fi|
\end{center}

%%%%%%%%%%%%%%%%%%%%%%%%%%%%%%%%%%%%%%%%
\DescribeMacro{\childdocname}
\DescribeMacro{\childdocjob}
The macro |\childdocname| contains the filename (without extension)
of the main or child file being processed.
Note that |\childdocjob| will always contain the name of the main file.

%%%%%%%%%%%%%%%%%%%%%%%%%%%%%%%%%%%%%%%%
\paragraph{Title Page.}

Conditional processing can be used to include a title or banner page
in the main document when proper precautions are taken.
Importantly, the code in the main file should ensure that the page counter
(as well as other status parameters which are stored in the |.aux| files)
takes the same value after the conditional processing.
Otherwise the page numbers may take divergent values
depending on which part is compiled.

For example, a title page could be declared by:
%
\begin{center}
\begin{tabular}{l}
|\ifchilddoc\||else|\\
|\addtocounter{page}{-1}|\\
\textit{code for title page}\\
|\newpage|\\
|\||fi|
\end{tabular}
\end{center}
%
A banner page for the child documents can be generated by:
%
\begin{center}
\begin{tabular}{l}
|\ifchilddoc|\\
|\addtocounter{page}{-1}|\\
\textit{code for banner page}\\
|\newpage|\\
|\||fi|
\end{tabular}
\end{center}
%
Here one could write a message such as:
\begin{center}
|This is the part \childdocname{} of \childdocjob{}.|
\end{center}

%%%%%%%%%%%%%%%%%%%%%%%%%%%%%%%%%%%%%%%%%%%%%%%%%%%%%%%%%%%%%%%%%%%%%%%%%%%%%%%%
\subsection{Flags}
\label{sec:flags}

The package makes it easy to generate different versions
of the main or child documents.
To this end compilation flags can be defined
and assigned different default values.
They will be particularly useful in conjunction
with the forwarding mechanism described in \secref{sec:forward}.

For example, it may be useful to have a flag |\version|
which can be set to |draft| or |final|.
The document source will contain some conditional code
depending on the value of |\version|.
Suppose further, the flag should default to |final| for the main file
and to |draft| for child files
which is a natural assignment for editing the document.
This is achieved by placing the following code
in the preamble of the main document
(below the |\childdocmain| directive):
%
\begin{center}
\begin{tabular}{l}
|\ifchilddoc|\\
|\providecommand{\version}{draft}|\\
|\||else|\\
|\providecommand{\version}{final}|\\
|\||fi|
\end{tabular}
\end{center}
%
The definition by |\providecommand| makes sure
that previous definitions are not overwritten.
Further statements |\providecommand{\version}{...}|
can thus be added before the above code to override it.

For the main file, one might add a line
(between |\childdocmain| and the above block)
%
\begin{center}
|%\ifchilddoc\||else\providecommand{\version}{draft}\||fi|
\end{center}
%
which can be uncommented to produce a draft version.
Likewise one can add a line to the very top of a child file
(above the |\childdocof{|\textit{main}|}| directive)
%
\begin{center}
|%\providecommand{\version}{final}|
\end{center}
%
which can be uncommented to produce the final version of this child document.

%%%%%%%%%%%%%%%%%%%%%%%%%%%%%%%%%%%%%%%%%%%%%%%%%%%%%%%%%%%%%%%%%%%%%%%%%%%%%%%%
\subsection{Forwarding}
\label{sec:forward}

Different versions of the main or child documents
using compilation flags as described in \secref{sec:flags}
can be (permanently) stored in different files
for convenient compilation, viewing and distribution.
To this end, the package defines a command
to pass on compilation to a different file:

%%%%%%%%%%%%%%%%%%%%%%%%%%%%%%%%%%%%%%%%
\DescribeMacro{\childdocforward}
The command |\childdocforward| redirects processing to
another source file:
%
\begin{center}
\begin{tabular}{l}
|\input{childdoc.def}|\\
|\childdocforward[|\textit{main}|]{|\textit{dest}|}|\\
\end{tabular}
\end{center}
%
The argument \textit{dest} is the destination file
(without extension).
It should be the main file or one of the child files.
Note that further \textsf{childdoc} directives
such as |\childdocof| and |\childdocforward|
in the indicated file will be processed in this form.
The optional argument \textit{main}
passes on directly to the main file \textit{main}
while pretending to compile the child \textit{dest}.
This form behaves as if \textit{dest}
issues |\childdocof{|\textit{main}|}| right away,
and no further \textsf{childdoc} directives will be processed.

%%%%%%%%%%%%%%%%%%%%%%%%%%%%%%%%%%%%%%%%
\DescribeMacro{\...prefix}
In the alternative form |\childdocforwardprefix|,
%
\begin{center}
\begin{tabular}{l}
|\input{childdoc.def}|\\
|\childdocforwardprefix[|\textit{main}|]{|\textit{prefix}|}{|\textit{dest}|}|
\end{tabular}
\end{center}
%
the destination file is determined by a pattern
depending on the current file:
To make this work, the current file must be called
`{\textit{prefix}\hspace{0.2em}\textit{suffix}}'
with \textit{prefix} matching precisely the argument.
Processing is then passed on to the file
`{\textit{dest}\hspace{0.2em}\textit{suffix}}'.
Surely, the same effect is achieved by
directly specifying the
argument `{\textit{dest}\hspace{0.2em}\textit{suffix}}'
in the first form.
However, that requires to set up a different file
for each child. With the alternative form of the command
all these files can have exactly the same content
which simplifies setting them up and maintaining them.

For example, the following file |draft.tex|
with a compilation flag |\version| as described in \secref{sec:flags}
compiles the main document as a draft:
%
\begin{center}
\begin{tabular}{l}
|\def\version{draft}|\\
|\input{childdoc.def}|\\
|\childdocforward{|\textit{main}|}|
\end{tabular}
\end{center}
%
Likewise, the following files |final|\textit{nn}|.tex|
compile the final version of the child document
|child|\textit{nn}|.tex|:
%
\begin{center}
\begin{tabular}{l}
|\def\version{final}|\\
|\input{childdoc.def}|\\
|\childdocforwardprefix{final}{child}|
\end{tabular}
\end{center}
%

Note that when several versions of a main file and/or of each child file
are to be generated, it may be convenient to set up a |Makefile| or
shell script to automatise the process.

%%%%%%%%%%%%%%%%%%%%%%%%%%%%%%%%%%%%%%%%%%%%%%%%%%%%%%%%%%%%%%%%%%%%%%%%%%%%%%%%
\subsection{Command Line Processing}
\label{sec:commandline}

The effect of redirection files can also be achieved by invoking
the \LaTeX{} compiler with a more elaborate command line.
Most conveniently this should be done as part
of a shell script or a |Makefile|.

When using \textsf{childdoc} in the main file, the following
command lines effectively perform a redirection
(note that depending on the shell being used,
backslashes may have to be doubled: `|\|' $\to$ `|\\|'):
%
\begin{center}
|... -jobname "|\textit{target}|" |\\|"|[\textit{flags}]%
|\input{childdoc.def}\childdocforward[|\textit{main}|]{|\textit{dest}|}"|
\end{center}
%
Here \textit{target} is the name of the output file,
\textit{main} is the name of the main file
and \textit{dest} is the name of the main or child file to be processed
(all filenames without extensions).
The optional argument \textit{main} can be omitted
if \textit{main} matches \textit{dest}.
Optionally, compilation \textit{flags} can be defined via |\def| commands.
This command line makes the \TeX{} engine believe
it is compiling the file \textit{target}
whose content is specified as the latter parameter.
The provided code then forwards the processing to
\textit{main} or \textit{dest} as described in \secref{sec:forward}.

%%%%%%%%%%%%%%%%%%%%%%%%%%%%%%%%%%%%%%%%%%%%%%%%%%%%%%%%%%%%%%%%%%%%%%%%%%%%%%%%
\subsection{Include by Input}
\label{sec:input}

Including child documents by |\include| has some restrictions by design.
Most notably, the content of a child document always occupies
its own set of pages; pages cannot be shared between child documents.
Usually, this behaviour makes perfect sense
because each child document contain an essential part of the document.
However, in some situations it may be desirable to compose
a document from a collection of parts
without having mandatory page breaks between then.
For this case, the package
provides a mechanism to include parts
by |\input| which can also be processed individually.
However, by construction this mechanism
requires manual handling of the content to be output.

%%%%%%%%%%%%%%%%%%%%%%%%%%%%%%%%%%%%%%%%
\DescribeMacro{\ifchilddocmanual}
The main file should be prepared as usual, see \secref{sec:include}.
However, the document body must make a distinction
between processing of an individual part and of the main document, e.g.:
%
\begin{center}
\begin{tabular}{l}
|\ifchilddocmanual|\\
|\input{\childdocname}|\\
|\||else|\\
\textit{document body with }|\input{|\textit{part}|}|\\
|\||fi|
\end{tabular}
\end{center}
%
The conditional |\ifchilddocmanual| is true whenever
a part to be included by |\input| is being compiled,
and the name of the part is stored in |\childdocname|.

%%%%%%%%%%%%%%%%%%%%%%%%%%%%%%%%%%%%%%%%
\DescribeMacro{\childdocby}
Each part to be included by |\input| should start with:
%
\begin{center}
\begin{tabular}{l}
|\input{childdoc.def}|\\
|\childdocby{|\textit{main}|}|\\
\end{tabular}
\end{center}
%
The directive |\childdocby| is similar to |\childdocof|
described in \secref{sec:include},
but the subsequent selection of content must be done manually.
To that end, both |\ifchilddoc| and |\ifchilddocmanual|
will be true upon processing of a part,
and the name of the part is stored in |\childdocname|.
Note that |\jobname| will be set to the filename of the current part
so that each part receives an individual |.aux| file
that does not interfere with the |.aux| file(s) of the main document.
This behaviour can be altered by the alternative form
|\childdocby[*]{|\textit{main}|}| (with a non-empty optional argument)
which uses the |.aux| file of the main document
by setting |\jobname| to \textit{main}.

%%%%%%%%%%%%%%%%%%%%%%%%%%%%%%%%%%%%%%%%%%%%%%%%%%%%%%%%%%%%%%%%%%%%%%%%%%%%%%%%
\subsection{Driver Development}
\label{sec:driver}

The \textsf{childdoc} mechanism can also be use for the development
of definition files such as \LaTeX{} styles or classes.
This case differs from the above setup with multiple parts
included by |\include| in that no |\includeonly| should be invoked.
This can be achieved by starting the include file
(before |\ProvidesPackage|) with:
%
\begin{center}
\begin{tabular}{l}
|\input{childdoc.def}|\\
|\childdocforward{|\textit{main}|}|\\
\end{tabular}
\end{center}
%
or alternatively with:
%
\begin{center}
\begin{tabular}{l}
|\input{childdoc.def}|\\
|\childdocby{|\textit{main}|}|\\
\end{tabular}
\end{center}
%
Both forms have slightly different effects as described above.
The main file is prepared as usual, see \secref{sec:include}.

%%%%%%%%%%%%%%%%%%%%%%%%%%%%%%%%%%%%%%%%%%%%%%%%%%%%%%%%%%%%%%%%%%%%%%%%%%%%%%%%
\subsection{Legacy Detection}
\label{sec:detection}

The directive |\childdocmain| in the main file can detect
whether the complete document or merely a child is to be compiled
even without using the directive |\childdocof|.
This method is deprecated because it is less robust
and there is no compelling reason to use it;
it is merely provided for backward compatibility
and it may be removed in future versions.

If the detection mechanism is to be used,
it is mandatory to correctly specify
the filename of the main file as the argument of |\childdocmain|:
%
\begin{center}
\begin{tabular}{l}
|\input{childdoc.def}|\\
|\childdocmain{|\textit{main}|}|\\
\end{tabular}
\end{center}
%
If |\jobname| does not match the argument \textit{main} of |\childdocmain|,
it is assumed that |\jobname| points to the child file to be compiled.
When using |\childdocmain| with the main file specified as argument,
it suffices to start a child file
with just |\input{|\textit{main}|}|
without loading of the package and using |\childdocof|.
If instead all processing is done
with the appropriate \textsf{childdoc} directives,
the argument of \textit{main} of |\childdocmain| can be empty.

An alternative version of the command line processing described
in \secref{sec:commandline} using the detection mechanism reads:
%
\begin{center}
|... -jobname "|\textit{target}|" "|[\textit{flags}]%
[|\def\jobname{|\textit{dest}|}|]|\input{|\textit{main}|}"|
\end{center}

%%%%%%%%%%%%%%%%%%%%%%%%%%%%%%%%%%%%%%%%%%%%%%%%%%%%%%%%%%%%%%%%%%%%%%%%%%%%%%%%
\subsection{Manual Code}
\label{sec:manual}

In case one cannot be certain whether the definitions file |childdoc.def|
is installed on the target \TeX{} distribution
and one prefers not to ship it,
it is conceivable to paste a few relevant commands into the sources.

To that end, drop all statements |\input{childdoc.def}|
and perform the replacements as outlined below.
Instead of |\childdocmain{|\textit{main}|}| add the following code
to the top of the main file:
%
\begin{center}
\begin{tabular}{l}
|\||ifdefined\childdocname\endinput\||fi\newif\ifchilddoc|\\
|\edef\childdocname{\scantokens\expandafter{\jobname\noexpand}}|\\
|\def\childdocmain{|\textit{main}|}\||ifx\childdocmain\childdocname\||else|\\
|\childdoctrue\includeonly{\childdocname}\let\jobname\childdocmain\||fi|\\
\end{tabular}
\end{center}
%
Instead of |\childdocof{|\textit{main}|}| just include the main file
at the top of each child file:
%
\begin{center}
|\input{|\textit{main}|}|
\end{center}
%
A simple redirection |\childdocforward{|\textit{dest}|}| is achieved by:
%
\begin{center}
|\def\jobname{|\textit{dest}|}\input{\jobname}|
\end{center}
%
The redirection with prefix
|\childdocforwardprefix[|\textit{prefix}|]{|\textit{dest}|}|
is accomplished by:
%
\begin{center}
\begin{tabular}{l}
|{\edef\jobname{\scantokens\expandafter{\jobname\noexpand}}|\\
|\def\redirectjob |\textit{prefix}|#1~~~{\gdef\jobname{|\textit{dest}|#1}}|\\
|\expandafter\redirectjob\jobname~~~}\input{\jobname}|
\end{tabular}
\end{center}

In an alternative approach,
child documents can be compiled by a specific command line
without additional code or specific definitions:
%
\begin{center}
|... -jobname "|\textit{target}|" "|[\textit{flags}]%
|\includeonly{|\textit{dest}|}\input{|\textit{main}|}"|
\end{center}
%

%%%%%%%%%%%%%%%%%%%%%%%%%%%%%%%%%%%%%%%%%%%%%%%%%%%%%%%%%%%%%%%%%%%%%%%%%%%%%%%%
%%%%%%%%%%%%%%%%%%%%%%%%%%%%%%%%%%%%%%%%%%%%%%%%%%%%%%%%%%%%%%%%%%%%%%%%%%%%%%%%
\section{Information}

%%%%%%%%%%%%%%%%%%%%%%%%%%%%%%%%%%%%%%%%%%%%%%%%%%%%%%%%%%%%%%%%%%%%%%%%%%%%%%%%
\subsection{Copyright}

Copyright \copyright{} 2017--2018 Niklas Beisert

This work may be distributed and/or modified under the
conditions of the \LaTeX{} Project Public License, either version 1.3
of this license or (at your option) any later version.
The latest version of this license is in
  \url{http://www.latex-project.org/lppl.txt}
and version 1.3 or later is part of all distributions of \LaTeX{}
version 2005/12/01 or later.

This work has the LPPL maintenance status `maintained'.

The Current Maintainer of this work is Niklas Beisert.

This work consists of the files |README.txt|, |childdoc.ins| and |childdoc.dtx|
as well as the derived files |childdoc.def|, |cdocsamp.tex|
with |cdocsch1.tex|, |cdocsch2.tex|, |cdocspt3.tex|, |cdocspt4.tex|,
|cdocsdrf.tex|, |cdocsfn1.tex|, |cdocsfn2.tex|
as well as |childdoc.pdf|.

%%%%%%%%%%%%%%%%%%%%%%%%%%%%%%%%%%%%%%%%%%%%%%%%%%%%%%%%%%%%%%%%%%%%%%%%%%%%%%%%
\subsection{Files and Installation}

The package consists of the files:
%
\begin{center}
\begin{tabular}{ll}
    |README.txt|   & readme file \\
    |childdoc.ins| & installation file \\
    |childdoc.dtx| & source file \\
    |childdoc.def| & definition file \\
    |cdocsamp.tex| & sample main file \\
    |cdocsch1.tex| & sample include file \\
    |cdocsch2.tex| & sample include file \\
    |cdocspt3.tex| & sample part file \\
    |cdocspt4.tex| & sample part file \\
    |cdocsdrf.tex| & sample redirection file \\
    |cdocsfn1.tex| & sample redirection file \\
    |cdocsfn2.tex| & sample redirection file \\
    |childdoc.pdf| & manual
\end{tabular}
\end{center}
%
The distribution consists of the files
|README.txt|, |childdoc.ins| and |childdoc.dtx|.
%
\begin{itemize}
\item
Run (pdf)\LaTeX{} on |childdoc.dtx|
to compile the manual |childdoc.pdf| (this file).
\item
Run \LaTeX{} on |childdoc.ins| to create the definitions file |childdoc.def|
and the sample |cdocsamp.tex| with include files
|cdocsch1.tex|, |cdocsch2.tex|, |cdocspt3.tex|, |cdocspt4.tex|,
|cdocsdrf.tex|, |cdocsfn1.tex|, |cdocsfn2.tex|.
Then copy the file |childdoc.def| to an appropriate directory of your \LaTeX{}
distribution, e.g.\ \textit{texmf-root}|/tex/latex/childdoc|.
\end{itemize}

%%%%%%%%%%%%%%%%%%%%%%%%%%%%%%%%%%%%%%%%%%%%%%%%%%%%%%%%%%%%%%%%%%%%%%%%%%%%%%%%
\subsection{Related CTAN Packages}

There are several other packages which offer a similar functionality:
%
\begin{itemize}
\item
The packages
\href{http://ctan.org/pkg/docmute}{\textsf{docmute}},
\href{http://ctan.org/pkg/includex}{\textsf{includex}} and
\href{http://ctan.org/pkg/standalone}{\textsf{standalone}}
provide commands to include only the document body of
a child file thus allowing both files to be compiled individually.
\item
The packages \href{http://ctan.org/pkg/subdocs}{\textsf{subdocs}}
and \href{http://ctan.org/pkg/subfiles}{\textsf{subfiles}}
provide structures in which the main and child documents can be
encapsulated and allowing them to be compiled individually.
The inclusion mechanism is different from the conventional |\include|.
\item
The package \href{http://ctan.org/pkg/combine}{\textsf{combine}}
is an elaborate solution to combine several documents into one.
\end{itemize}
%
See also the CTAN topic \href{http://ctan.org/topic/subdocs}{\textsf{subdocs}}
for further related packages.
The present package differs from the above solutions in that
a document structure constructed with the conventional |\include| mechanism
just needs two extra commands at the top of every file
such that all constituent files can be compiled individually.

%%%%%%%%%%%%%%%%%%%%%%%%%%%%%%%%%%%%%%%%%%%%%%%%%%%%%%%%%%%%%%%%%%%%%%%%%%%%%%%%
%\subsection{Feature Suggestions}
%
%The following is a list of features which may be useful for future
%versions of this package:
%%
%\begin{itemize}
%\item
%\ldots
%\end{itemize}

%%%%%%%%%%%%%%%%%%%%%%%%%%%%%%%%%%%%%%%%%%%%%%%%%%%%%%%%%%%%%%%%%%%%%%%%%%%%%%%%
\subsection{Revision History}

%%%%%%%%%%%%%%%%%%%%%%%%%%%%%%%%%%%%%%%%
\paragraph{v2.0:} 2018/12/30

\begin{itemize}
\item
immediate forward processing
\item
added |\childdocby| mechanism
\item
manual restructured
\end{itemize}

%%%%%%%%%%%%%%%%%%%%%%%%%%%%%%%%%%%%%%%%
\paragraph{v1.6:} 2018/01/17

\begin{itemize}
\item
application for development of include files
\item
corrections to manual
\end{itemize}

%%%%%%%%%%%%%%%%%%%%%%%%%%%%%%%%%%%%%%%%
\paragraph{v1.5:} 2017/05/21

\begin{itemize}
\item
more complete structuring introduced
\item
|\childdocof| introduced
\item
|\childdoc| renamed to |\childdocmain|
\item
|\childredirect| renamed to |\childdocforward| and |\childdocforwardprefix|
and functionality expanded
\end{itemize}

%%%%%%%%%%%%%%%%%%%%%%%%%%%%%%%%%%%%%%%%
\paragraph{v1.0:} 2017/04/27

\begin{itemize}
\item
manual and install package
\item
first version published on CTAN
\end{itemize}

%%%%%%%%%%%%%%%%%%%%%%%%%%%%%%%%%%%%%%%%
\paragraph{v0.6:} 2017/04/26

\begin{itemize}
\item
redirection mechanism added
\end{itemize}

%%%%%%%%%%%%%%%%%%%%%%%%%%%%%%%%%%%%%%%%
\paragraph{v0.5:} 2017/04/26

\begin{itemize}
\item
functionality in definition file
\end{itemize}


%%%%%%%%%%%%%%%%%%%%%%%%%%%%%%%%%%%%%%%%%%%%%%%%%%%%%%%%%%%%%%%%%%%%%%%%%%%%%%%%
%%%%%%%%%%%%%%%%%%%%%%%%%%%%%%%%%%%%%%%%%%%%%%%%%%%%%%%%%%%%%%%%%%%%%%%%%%%%%%%%
%%%%%%%%%%%%%%%%%%%%%%%%%%%%%%%%%%%%%%%%%%%%%%%%%%%%%%%%%%%%%%%%%%%%%%%%%%%%%%%%
\appendix

\settowidth\MacroIndent{\rmfamily\scriptsize 000\ }

 \DocInput{childdoc.dtx}

\end{document}
%</driver>
% \fi
%
% %%%%%%%%%%%%%%%%%%%%%%%%%%%%%%%%%%%%%%%%%%%%%%%%%%%%%%%%%%%%%%%%%%%%%%%%%%%%%%
% %%%%%%%%%%%%%%%%%%%%%%%%%%%%%%%%%%%%%%%%%%%%%%%%%%%%%%%%%%%%%%%%%%%%%%%%%%%%%%
% \section{Sample}
%\iffalse
%<*samplemain>
%\fi
%
% The following presents a sample document
% with two chapters, two parts, a title page,
% a compile flag as well as three forwarding files to set the flag.
% It consists of eight |.tex| files:
% \begin{center}
% \begin{tabular}{ll}
% |cdocsamp.tex|&main file\\
% |cdocsch1.tex|&include file for chapter 1\\
% |cdocsch2.tex|&include file for chapter 2\\
% |cdocspt3.tex|&include file for part 3\\
% |cdocspt4.tex|&include file for part 4\\
% |cdocsdrf.tex|&forwarding file for main file in draft mode\\
% |cdocsfi1.tex|&forwarding file for final version of chapter 1\\
% |cdocsfi2.tex|&forwarding file for final version of chapter 2\\
% \end{tabular}
% \end{center}
% Each of the eight files can be compiled directly by the \LaTeX{} compiler.
%
% %%%%%%%%%%%%%%%%%%%%%%%%%%%%%%%%%%%%%%
% \paragraph{Main File.}
%
% The main file is called |cdocsamp.tex|.
%
% Load the \textsf{childdoc} definitions and
% declare the filename for the main document:
%    \begin{macrocode}
\input{childdoc.def}
\childdocmain{}
%    \end{macrocode}

% Optional override for |\version| flag:
%    \begin{macrocode}
%%\ifchilddoc\else\providecommand{\version}{draft}\fi
%    \end{macrocode}

% Define the default values for the |\version| flag
% (|final| for the main file and |draft| for childs):
%    \begin{macrocode}
\ifchilddoc
\providecommand{\version}{draft}
\else
\providecommand{\version}{final}
\fi
%    \end{macrocode}

% Load the standard document class:
%    \begin{macrocode}
\documentclass[12pt]{article}
%    \end{macrocode}

% Start the document body:
%    \begin{macrocode}
\begin{document}
%    \end{macrocode}

% Declare a title page.
% Print title, part of document being processed and version flag:
%    \begin{macrocode}
\addtocounter{page}{-1}
\begin{center}
{\LARGE\bfseries{}childdoc example\par}
\vspace{1cm}
\ifchilddoc
\ifchilddocmanual part\else chapter\fi:
`\childdocname' of `\childdocjob'\par
\else
main document: `\childdocjob'\par
\fi
version: \version\par
\end{center}
\newpage
%    \end{macrocode}

% Manually include selected file,
% otherwise process as usual:
%    \begin{macrocode}
\ifchilddocmanual
\section*{part `\childdocname'}
\input{\childdocname}
\else
%    \end{macrocode}

% Include the two chapters:
%    \begin{macrocode}
\include{cdocsch1}
\include{cdocsch2}
%    \end{macrocode}

% Include the two parts unless only chapters should be displayed:
%    \begin{macrocode}
\ifchilddoc\else
\section{part three}
\input{cdocspt3}
\section{part four}
\input{cdocspt4}
\fi
%    \end{macrocode}

% Process as usual until here:
%    \begin{macrocode}
\fi
%    \end{macrocode}

% End of document body:
%    \begin{macrocode}
\end{document}
%    \end{macrocode}
%\iffalse
%</samplemain>
%\fi
%
% %%%%%%%%%%%%%%%%%%%%%%%%%%%%%%%%%%%%%%
% \paragraph{Chapter Include Files.}
%
% The include files are called |cdocsch1.tex| and |cdocsch2.tex|.
%
%\iffalse
%<*samplechap1|samplechap2>
%\fi

% Optional override for |\version| flag:
%    \begin{macrocode}
%%\providecommand{\version}{final}
%    \end{macrocode}

% Include the main document:
%    \begin{macrocode}
\input{childdoc.def}
\childdocof{cdocsamp}
%    \end{macrocode}

%\iffalse
%</samplechap1|samplechap2>
%\fi
%
%\iffalse
%<*samplechap1>
%\fi
% Some text for chapter 1:
%    \begin{macrocode}
\section{one}
some text in chapter one
%    \end{macrocode}

%\iffalse
%</samplechap1>
%\fi
% Some text for chapter 2:
%\iffalse
%<*samplechap2>
%\fi
%    \begin{macrocode}
\section{two}
more text in chapter two
%    \end{macrocode}

%\iffalse
%</samplechap2>
%\fi
%
% %%%%%%%%%%%%%%%%%%%%%%%%%%%%%%%%%%%%%%
% \paragraph{Part Include Files.}
%
% The include files are called |cdocspt3.tex| and |cdocspt4.tex|.
%
%\iffalse
%<*samplepart3|samplepart4>
%\fi

% Optional override for |\version| flag:
%    \begin{macrocode}
%%\providecommand{\version}{final}
%    \end{macrocode}

% Include the main document:
%    \begin{macrocode}
\input{childdoc.def}
\childdocby{cdocsamp}
%    \end{macrocode}

%\iffalse
%</samplepart3|samplepart4>
%\fi
%
%\iffalse
%<*samplepart3>
%\fi
% Some text for part 3:
%    \begin{macrocode}
some text in part three
%    \end{macrocode}

%\iffalse
%</samplepart3>
%\fi
% Some text for part 4:
%\iffalse
%<*samplepart4>
%\fi
%    \begin{macrocode}
more text in part four
%    \end{macrocode}

%\iffalse
%</samplepart4>
%\fi
%
% %%%%%%%%%%%%%%%%%%%%%%%%%%%%%%%%%%%%%%
% \paragraph{Forwarding for a Complete Draft.}
%
% The following forwarding file |cdocsdrf.tex|
% compiles the main document in draft mode:
%\iffalse
%<*sampledraft>
%\fi
%    \begin{macrocode}
\def\version{draft}
\input{childdoc.def}
\childdocforward{cdocsamp}
%    \end{macrocode}

%\iffalse
%</sampledraft>
%\fi
%
% %%%%%%%%%%%%%%%%%%%%%%%%%%%%%%%%%%%%%%
% \paragraph{Forwarding for Final Version of the Chapters.}
%
% The following forwarding files |cdocsfn1.tex| and |cdocsfn2.tex|
% (with identical content)
% compile the final versions of the child documents
% |cdocsch1.tex| and |cdocsch2.tex|, respectively:
%\iffalse
%<*samplefinal>
%\fi
%    \begin{macrocode}
\def\version{final}
\input{childdoc.def}
\childdocforwardprefix[cdocsamp]{cdocsfn}{cdocsch}
%    \end{macrocode}

%\iffalse
%</samplefinal>
%\fi
%
% %%%%%%%%%%%%%%%%%%%%%%%%%%%%%%%%%%%%%%
% \paragraph{Command Line Processing.}
%
% The following three command lines generate the output files
% |cdocscld|, |cdocscl1| and |cdocscl2|
% which should be identical to
% |cdocsdrf|, |cdocsch1| and |cdocsfn2|, respectively:
% \begin{center}
% \begin{tabular}{l}
% |latex -jobname cdocscld \|\\
% |  "\def\version{draft}\input{childdoc.def}\childdocforward{cdocsamp}"|\\
% |latex -jobname cdocscl1 \|\\
% |  "\input{childdoc.def}\childdocforward[cdocsamp]{cdocsch1}"|\\
% |latex -jobname cdocscl2 \|\\
% |  "\def\version{final}\input{childdoc.def}\childdocforward{cdocsch2}"|
% \end{tabular}
% \end{center}
% Note that the trailing backslash on each first line
% merely continues the input to the second line
% (for convenient cut ant paste).
% Furthermore, the command |latex| can be replaced by any
% of its alternative versions such as |pdflatex|.
%
% %%%%%%%%%%%%%%%%%%%%%%%%%%%%%%%%%%%%%%%%%%%%%%%%%%%%%%%%%%%%%%%%%%%%%%%%%%%%%%
% %%%%%%%%%%%%%%%%%%%%%%%%%%%%%%%%%%%%%%%%%%%%%%%%%%%%%%%%%%%%%%%%%%%%%%%%%%%%%%
% \section{Implementation}
%\iffalse
%<*package>
%\fi
%
% This section describes the definitions file |childdoc.def|.

% The definitions cannot be loaded using |\usepackage| or |\RequirePackage|
% which has a mechanism to prevent loading a style file more than once.
% When loading the definitions by means of |\input|
% multiple instances have to be prevented manually:
%\iffalse
%This code needs to be before the `\ProvidesFile' directive
%which is defined at the beginning of this file.
%Therefore it is also placed there and commented out here.
%</package>
%<*discard>
%\fi
%    \begin{macrocode}
\ifdefined\childdocmain\endinput\fi
%    \end{macrocode}
%\iffalse
%</discard>
%<*package>
%\fi
%
% \macro{\ifchilddoc}
% \macro{\ifchilddocmanual}
% The conditional |\ifchilddoc| tells whether a
% child (true) or main (false) document is being compiled.
% The conditional |\ifchilddocmanual| tells whether
% the |\includeonly| mechanism is used (false) or
% the selection of child files must be performed manually (true).
% The definitions initialise to false:
%    \begin{macrocode}
\newif\ifchilddoc
\newif\ifchilddocmanual
%    \end{macrocode}

% \macro{\childdocname}
% \macro{\childdocjob}
% The macro |\childdocname| stores the name of the main document
% to be compiled. The macro |\childdocjob| stores the name of
% the document on which the \LaTeX{} compiler was originally invoked.
% The content of |\jobname| cannot be compared
% to filenames specified in the source due to different catcodes.
% The following code rescans |\jobname|, stores the result
% in |\childdocname| and saves a copy in |\childdocjob|:
%    \begin{macrocode}
\edef\childdocname{\scantokens\expandafter{\jobname\noexpand}}
\let\childdocjob\childdocname
%    \end{macrocode}

% \macro{\childdocdisable}
% The macro |\childdocdisable| prevents the main file
% from being processed more than once.
% At this stage, the main document command |\childdocmain|
% is assumed to be called once again where it should do nothing.
% Any subsequent call to it should prevent
% a secondary processing of the main document
% It overwrites the forwarding commands
% |\childdocof| and |\childdocforward|
% with empty macros to prevent further inclusions of the main document:
%    \begin{macrocode}
\newcommand{\childdocdisable}
{
  \renewcommand{\childdocmain}[1]{\renewcommand{\childdocmain}[1]{\endinput}}
  \renewcommand{\childdocof}[1]{}
  \renewcommand{\childdocby}[2][]{}
  \renewcommand{\childdocforward}[2][]{}
  \renewcommand{\childdocdisable}{}
}
%    \end{macrocode}

% \macro{\childdocmain}
% The macro |\childdocmain| is to be called at the top of the main file
% with nothing or the main filename (without extension) as argument.
% First, it breaks loops.
% If the argument is not empty and does not match |\childdocname|
% (which is set by the first inclusion of |childdoc.def|),
% |\ifchilddoc| is set to true, |\includeonly| is applied to the child file
% and |\jobname| is set to the main file
% (for proper handling of |.aux| files):
%    \begin{macrocode}
\newcommand{\childdocmain}[1]
{
  \childdocdisable\childdocmain{}
  \if?#1?\else
    \begingroup
      \def\childdoctmp{#1}
      \ifx\childdoctmp\childdocname
        \def\childdoctmp{}
      \else
        \def\childdoctmp
        {
          \childdoctrue
          \includeonly{\childdocname}
          \def\childdocjob{#1}
          \def\jobname{#1}
        }
      \fi
      \expandafter
    \endgroup
    \childdoctmp
  \fi
}
%    \end{macrocode}

% \macro{\childdocof}
% The command |\childdocof| redirects
% compilation to the main file |#1|.
%    \begin{macrocode}
\newcommand{\childdocof}[1]
{
  \childdocdisable
  \childdoctrue
  \includeonly{\childdocname}
  \def\jobname{#1}
  \def\childdocjob{#1}
  \input{#1}
}
%    \end{macrocode}

% \macro{\childdocby}
% The command |\childdocby| ....
%    \begin{macrocode}
\newcommand{\childdocby}[2][]
{
  \childdocdisable
  \childdoctrue
  \childdocmanualtrue
  \if?#1?\else
    \def\jobname{#2}
  \fi
  \def\childdocjob{#2}
  \input{#2}
  \endinput
}
%    \end{macrocode}

% \macro{\childdocforward}
% The command |\childdocforward| redirects
% compilation to the main file or
% (if the optional argument is given) a child file.
% Parameters are set as if the main file
% or a child file starting with |\childdocof| was compiled.
% Then compilation is handed over to the main file:
%    \begin{macrocode}
\newcommand{\childdocforward}[2][]
{
  \begingroup
    \if?#1?
      \def\childdoctmp
      {
        \def\childdocname{#2}
        \def\childdocjob{#2}
        \def\jobname{#2}
        \input{#2}
        \endinput
      }
    \else
      \def\childdoctmp
      {
        \childdocdisable
        \def\childdocname{#2}
        \childdoctrue
        \includeonly{#2}
        \def\childdocjob{#1}
        \def\jobname{#1}
        \input{#1}
        \endinput
      }
    \fi
    \expandafter
  \endgroup
  \childdoctmp
}
%    \end{macrocode}

% \macro{\childdocforwardprefix}
% The command |\childdocforwardprefix| redirects
% compilation to the main or a child file by means of a pattern.
% The prefix |#1| in the current filename is replaced by |#2|
% and the suffix of the current filename is kept
% (it is assumed that the filename does not contain the substring `|~~~|'
% which is used as a delimiter).
% Compilation is handed over to the new file by |\childdocforward|:
%    \begin{macrocode}
\newcommand{\childdocforwardprefix}[3][]
{
  \begingroup
    \def\childdocextract #2##1~~~{\def\childdoctmp{\childdocforward[#1]{#3##1}}}
    \expandafter\childdocextract\childdocname~~~
    \expandafter
  \endgroup
  \childdoctmp
}
%    \end{macrocode}

% \macro{\childdoc}
% The deprecated macro |\childdoc| is a legacy version of |\childdocmain|:
%    \begin{macrocode}
\newcommand{\childdoc}{\childdocmain}
%    \end{macrocode}

% \macro{\childdocredirect}
% The deprecated macro |\childdocredirect| is a legacy version
% of |\childdocforward| and |\childdocforwardprefix|:
%    \begin{macrocode}
\newcommand{\childdocredirect}[2][]
{
  \begingroup
    \if?#1?
      \def\childdoctmp{\childdocforward{#2}}
    \else
      \def\childdoctmp{\childdocforwardprefix{#1}{#2}}
    \fi
    \expandafter
  \endgroup
  \childdoctmp
}
%    \end{macrocode}

%\iffalse
%</package>
%\fi
%
\endinput

\childdocof{cdocsamp}
%    \end{macrocode}

%\iffalse
%</samplechap1|samplechap2>
%\fi
%
%\iffalse
%<*samplechap1>
%\fi
% Some text for chapter 1:
%    \begin{macrocode}
\section{one}
some text in chapter one
%    \end{macrocode}

%\iffalse
%</samplechap1>
%\fi
% Some text for chapter 2:
%\iffalse
%<*samplechap2>
%\fi
%    \begin{macrocode}
\section{two}
more text in chapter two
%    \end{macrocode}

%\iffalse
%</samplechap2>
%\fi
%
% %%%%%%%%%%%%%%%%%%%%%%%%%%%%%%%%%%%%%%
% \paragraph{Part Include Files.}
%
% The include files are called |cdocspt3.tex| and |cdocspt4.tex|.
%
%\iffalse
%<*samplepart3|samplepart4>
%\fi

% Optional override for |\version| flag:
%    \begin{macrocode}
%%\providecommand{\version}{final}
%    \end{macrocode}

% Include the main document:
%    \begin{macrocode}
% \iffalse
%
% childdoc.dtx Copyright (C) 2017-2018 Niklas Beisert
%
% This work may be distributed and/or modified under the
% conditions of the LaTeX Project Public License, either version 1.3
% of this license or (at your option) any later version.
% The latest version of this license is in
%   http://www.latex-project.org/lppl.txt
% and version 1.3 or later is part of all distributions of LaTeX
% version 2005/12/01 or later.
%
% This work has the LPPL maintenance status `maintained'.
%
% The Current Maintainer of this work is Niklas Beisert.
%
% This work consists of the files childdoc.dtx and childdoc.ins
% and the derived files childdoc.def and cdocsamp.tex with
% cdocsch1.tex, cdocsch2.tex, cdocsdrf.tex, cdocsfn1.tex, cdocsfn2.tex.
%
%<package>\ifdefined\childdocmain\endinput\fi
%<package>\ProvidesFile{childdoc.def}[2018/12/30 v2.0 child document driver]
%<samplemain>\ProvidesFile{cdocsamp.tex}[2018/12/30 v2.0 sample for childdoc]
%<*driver>
%\ProvidesFile{childdoc.drv}[2018/12/30 v2.0 childdoc reference manual file]
\PassOptionsToClass{10pt,a4paper}{article}
\documentclass{ltxdoc}

\usepackage[margin=35mm]{geometry}
\usepackage{hyperref}
\usepackage{hyperxmp}
\usepackage[usenames]{color}

\hypersetup{colorlinks=true}
\hypersetup{pdfstartview=FitH}
\hypersetup{pdfpagemode=UseNone}
\hypersetup{pdfsource={}}
\hypersetup{pdflang={en-UK}}
\hypersetup{pdfcopyright={Copyright 2017-2018 Niklas Beisert.
  This work may be distributed and/or modified under the
  conditions of the LaTeX Project Public License, either version 1.3
  of this license or (at your option) any later version.}}
\hypersetup{pdflicenseurl={http://www.latex-project.org/lppl.txt}}
\hypersetup{pdfcontactaddress={ETH Zurich, ITP, HIT K,
  Wolfgang-Pauli-Strasse 27}}
\hypersetup{pdfcontactpostcode={8093}}
\hypersetup{pdfcontactcity={Zurich}}
\hypersetup{pdfcontactcountry={Switzerland}}
\hypersetup{pdfcontactemail={nbeisert@itp.phys.ethz.ch}}
\hypersetup{pdfcontacturl={http://people.phys.ethz.ch/\xmptilde nbeisert/}}

\newcommand{\secref}[1]{\hyperref[#1]{section \ref*{#1}}}

\parskip1ex
\parindent0pt
\let\olditemize\itemize
\def\itemize{\olditemize\parskip0pt}

\begin{document}

\title{The \textsf{childdoc} Package}
\hypersetup{pdftitle={The childdoc Package}}
\author{Niklas Beisert\\[2ex]
  Institut f\"ur Theoretische Physik\\
  Eidgen\"ossische Technische Hochschule Z\"urich\\
  Wolfgang-Pauli-Strasse 27, 8093 Z\"urich, Switzerland\\[1ex]
  \href{mailto:nbeisert@itp.phys.ethz.ch}
  {\texttt{nbeisert@itp.phys.ethz.ch}}}
\hypersetup{pdfauthor={Niklas Beisert}}
\hypersetup{pdfsubject={Manual for the LaTeX2e Package childdoc}}
\date{30 December 2018, \textsf{v2.0}}
\maketitle

\begin{abstract}\noindent
\textsf{childdoc} is a \LaTeXe{} package
that enables the direct compilation
of document sections included by |\include|
to individual files.
\end{abstract}

\begingroup
\parskip0ex
\tableofcontents
\endgroup

%%%%%%%%%%%%%%%%%%%%%%%%%%%%%%%%%%%%%%%%%%%%%%%%%%%%%%%%%%%%%%%%%%%%%%%%%%%%%%%%
%%%%%%%%%%%%%%%%%%%%%%%%%%%%%%%%%%%%%%%%%%%%%%%%%%%%%%%%%%%%%%%%%%%%%%%%%%%%%%%%
\section{Introduction}

\LaTeX{} provides a mechanism to structure a large document (such as a book)
into a main file and several child files (containing the chapters)
using the |\include| command.
This mechanism is beneficial for documents
which span hundreds of pages in order to
make the source file(s) more manageable.
Moreover, compilation can be restricted to
selected child files by means of the |\includeonly| command.
The latter feature can be used to reduce the compilation time while editing
(this was significantly more useful in the earlier days of \LaTeX{})
or to generate a smaller document which is easier to navigate.
Another application of |\includeonly| is to generate
documents consisting of selected parts of the complete document.

However, there are a few drawbacks of the plain |\include| mechanism:
\begin{itemize}
\item
The child files cannot be compiled on their own,
they can only be compiled via the main file.
A naive editing environment
(such as a text editor with an option
to have the current file processed by \LaTeX)
may require one to switch to the main file before compiling;
attempting to compile the child file produces errors.
\item
The main file must be modified (each time)
to adjust the |\includeonly| command
to the present needs. This easily leaves the main file in a messy state.
\item
The generated document will always carry the filename
of the main document. This is inconvenient if
several child files are to be compiled and
to be kept for distribution.
\end{itemize}

The present package provides a simple interface
to make child files individually compilable by \LaTeX{}.
Compiling a child file then has the same effect as compiling
the main file with an |\includeonly| command
to select the appropriate child.
Moreover the generated document will carry the name of the child
rather than the main file.
This resolves all three above issues.

This feature is meant to make the editing of books,
thesis documents and lecture notes somewhat more convenient.
However, the package can also be used efficiently for
composing a series of documents (such as exercise sheets)
which are typically distributed individually.
It then assists the author in generating the individual documents
(potentially in different versions)
as well as a document containing the collected series.
Another application is in developing style files
or other kinds of included material
where compilation of the style file could redirect
to a sample or test file.

%%%%%%%%%%%%%%%%%%%%%%%%%%%%%%%%%%%%%%%%%%%%%%%%%%%%%%%%%%%%%%%%%%%%%%%%%%%%%%%%
%%%%%%%%%%%%%%%%%%%%%%%%%%%%%%%%%%%%%%%%%%%%%%%%%%%%%%%%%%%%%%%%%%%%%%%%%%%%%%%%
\section{Usage}

First of all, the package \textsf{childdoc} is \emph{not} a standard
\LaTeXe{} |.sty| style file! Therefore it needs to be invoked in
a non-standard way.

%%%%%%%%%%%%%%%%%%%%%%%%%%%%%%%%%%%%%%%%%%%%%%%%%%%%%%%%%%%%%%%%%%%%%%%%%%%%%%%%
\subsection{Included Files}
\label{sec:include}

%%%%%%%%%%%%%%%%%%%%%%%%%%%%%%%%%%%%%%%%
\DescribeMacro{\childdocmain}
To use the package, add the commands
\begin{center}
\begin{tabular}{l}
|\input{childdoc.def}|\\
|\childdocmain{}|\\
\end{tabular}
\end{center}
at the very top of the main \LaTeX{} file,
in particular \emph{before} the |\documentclass| statement!
The argument of |\childdocmain| should be left empty
(but it must be present).

%%%%%%%%%%%%%%%%%%%%%%%%%%%%%%%%%%%%%%%%
\DescribeMacro{\childdocof}
Furthermore, add the commands
\begin{center}
\begin{tabular}{l}
|\input{childdoc.def}|\\
|\childdocof{|\textit{main}|}|\\
\end{tabular}
\end{center}
at the top of every child file \textit{child}
which is included by |\include{|\textit{child}|}|
from within the main file
(or at least for those files to be compiled individually).
The argument \textit{main} must be the filename of the main file.

There are a couple of
considerations in setting up the main and child documents:

%%%%%%%%%%%%%%%%%%%%%%%%%%%%%%%%%%%%%%%%
\paragraph{Restrictions.}

Please note the following restrictions:
\begin{itemize}
\item
|\childdocmain| must be called with one argument \textit{main}
to ensure compatibility with earlier version of the package.
It must either be empty (|\childdocmain{}|)
or precisely match the filename of the main file in which it is specified.
See \secref{sec:detection} for further information.
\item
The filename \textit{main} must be specified without the |.tex| extension.
\item
The filename \textit{main} is case sensitive
(even in case-insensitive file systems)
due to internal string comparison.
\item
The argument \textit{main} should be fully expanded, it cannot be a macro.
\item
Subdirectories and special characters should be avoided in filenames.
\item
The command |\childdocmain{|\textit{main}|}| must be followed by a whitespace.
It should not be followed immediately by another command
or by a comment mark `|%|'.
This is because the \TeX{} parser reads the token immediately following
the argument of |\childdocmain| and puts it
at the beginning of every child section;
however, a white\-space is ignored.
\end{itemize}

%%%%%%%%%%%%%%%%%%%%%%%%%%%%%%%%%%%%%%%%
\paragraph{Content of Main File.}

It is advisable to place all content in the child files included by |\include|.
Any output contained in the main file will appear in all child documents
unless suppressed manually;
it cannot be suppressed automatically by the |\includeonly| directive
and thus should normally be avoided.
A method to include some content in the main file
by means of conditional processing is described in \secref{sec:conditional}.

%%%%%%%%%%%%%%%%%%%%%%%%%%%%%%%%%%%%%%%%
\paragraph{Page Numbering.}

When only a part of the document is compiled,
the appropriate numbering of pages
(as well as other status parameters)
is determined from the |.aux| files.
The latter contain information from previous passes.
However this information needs to propagate through
all intermediate child documents.
Therefore the page numbering in child documents may well
be inconsistent until the complete document is compiled at least once.

A useful (if unconventional) way to always ensure a consistent
page numbering is to restart the numbering in each child document
and denote the pages by `\textit{child}|.|\textit{page}'
where \textit{child} represents the chapter/section number of the child file.
This can be achieved by the command
|\numberwithin{page}{|\textit{child}|}|
of the \textsf{amsmath} package
where \textit{child} can be |chapter| or |section|
depending on the chosen structuring.
Alternatively, one can modify the macro |\thepage| appropriately
and reset the counter |page| at the start of each child file.

%%%%%%%%%%%%%%%%%%%%%%%%%%%%%%%%%%%%%%%%%%%%%%%%%%%%%%%%%%%%%%%%%%%%%%%%%%%%%%%%
\subsection{Conditional Processing}
\label{sec:conditional}

The package provides a mechanism to compile different versions
of a document. To customise the versions further some conditional processing
can come in handy to distinguish which version is being compiled.
The package provides two macros to describe the compilation context:

%%%%%%%%%%%%%%%%%%%%%%%%%%%%%%%%%%%%%%%%
\DescribeMacro{\ifchilddoc}
The conditional |\ifchilddoc| distinguishes between the compilation of
child documents and the main document:
%
\begin{center}
|\ifchilddoc |\textit{child-code}| |[|\||else |\textit{main-code}]| \||fi|
\end{center}

%%%%%%%%%%%%%%%%%%%%%%%%%%%%%%%%%%%%%%%%
\DescribeMacro{\childdocname}
\DescribeMacro{\childdocjob}
The macro |\childdocname| contains the filename (without extension)
of the main or child file being processed.
Note that |\childdocjob| will always contain the name of the main file.

%%%%%%%%%%%%%%%%%%%%%%%%%%%%%%%%%%%%%%%%
\paragraph{Title Page.}

Conditional processing can be used to include a title or banner page
in the main document when proper precautions are taken.
Importantly, the code in the main file should ensure that the page counter
(as well as other status parameters which are stored in the |.aux| files)
takes the same value after the conditional processing.
Otherwise the page numbers may take divergent values
depending on which part is compiled.

For example, a title page could be declared by:
%
\begin{center}
\begin{tabular}{l}
|\ifchilddoc\||else|\\
|\addtocounter{page}{-1}|\\
\textit{code for title page}\\
|\newpage|\\
|\||fi|
\end{tabular}
\end{center}
%
A banner page for the child documents can be generated by:
%
\begin{center}
\begin{tabular}{l}
|\ifchilddoc|\\
|\addtocounter{page}{-1}|\\
\textit{code for banner page}\\
|\newpage|\\
|\||fi|
\end{tabular}
\end{center}
%
Here one could write a message such as:
\begin{center}
|This is the part \childdocname{} of \childdocjob{}.|
\end{center}

%%%%%%%%%%%%%%%%%%%%%%%%%%%%%%%%%%%%%%%%%%%%%%%%%%%%%%%%%%%%%%%%%%%%%%%%%%%%%%%%
\subsection{Flags}
\label{sec:flags}

The package makes it easy to generate different versions
of the main or child documents.
To this end compilation flags can be defined
and assigned different default values.
They will be particularly useful in conjunction
with the forwarding mechanism described in \secref{sec:forward}.

For example, it may be useful to have a flag |\version|
which can be set to |draft| or |final|.
The document source will contain some conditional code
depending on the value of |\version|.
Suppose further, the flag should default to |final| for the main file
and to |draft| for child files
which is a natural assignment for editing the document.
This is achieved by placing the following code
in the preamble of the main document
(below the |\childdocmain| directive):
%
\begin{center}
\begin{tabular}{l}
|\ifchilddoc|\\
|\providecommand{\version}{draft}|\\
|\||else|\\
|\providecommand{\version}{final}|\\
|\||fi|
\end{tabular}
\end{center}
%
The definition by |\providecommand| makes sure
that previous definitions are not overwritten.
Further statements |\providecommand{\version}{...}|
can thus be added before the above code to override it.

For the main file, one might add a line
(between |\childdocmain| and the above block)
%
\begin{center}
|%\ifchilddoc\||else\providecommand{\version}{draft}\||fi|
\end{center}
%
which can be uncommented to produce a draft version.
Likewise one can add a line to the very top of a child file
(above the |\childdocof{|\textit{main}|}| directive)
%
\begin{center}
|%\providecommand{\version}{final}|
\end{center}
%
which can be uncommented to produce the final version of this child document.

%%%%%%%%%%%%%%%%%%%%%%%%%%%%%%%%%%%%%%%%%%%%%%%%%%%%%%%%%%%%%%%%%%%%%%%%%%%%%%%%
\subsection{Forwarding}
\label{sec:forward}

Different versions of the main or child documents
using compilation flags as described in \secref{sec:flags}
can be (permanently) stored in different files
for convenient compilation, viewing and distribution.
To this end, the package defines a command
to pass on compilation to a different file:

%%%%%%%%%%%%%%%%%%%%%%%%%%%%%%%%%%%%%%%%
\DescribeMacro{\childdocforward}
The command |\childdocforward| redirects processing to
another source file:
%
\begin{center}
\begin{tabular}{l}
|\input{childdoc.def}|\\
|\childdocforward[|\textit{main}|]{|\textit{dest}|}|\\
\end{tabular}
\end{center}
%
The argument \textit{dest} is the destination file
(without extension).
It should be the main file or one of the child files.
Note that further \textsf{childdoc} directives
such as |\childdocof| and |\childdocforward|
in the indicated file will be processed in this form.
The optional argument \textit{main}
passes on directly to the main file \textit{main}
while pretending to compile the child \textit{dest}.
This form behaves as if \textit{dest}
issues |\childdocof{|\textit{main}|}| right away,
and no further \textsf{childdoc} directives will be processed.

%%%%%%%%%%%%%%%%%%%%%%%%%%%%%%%%%%%%%%%%
\DescribeMacro{\...prefix}
In the alternative form |\childdocforwardprefix|,
%
\begin{center}
\begin{tabular}{l}
|\input{childdoc.def}|\\
|\childdocforwardprefix[|\textit{main}|]{|\textit{prefix}|}{|\textit{dest}|}|
\end{tabular}
\end{center}
%
the destination file is determined by a pattern
depending on the current file:
To make this work, the current file must be called
`{\textit{prefix}\hspace{0.2em}\textit{suffix}}'
with \textit{prefix} matching precisely the argument.
Processing is then passed on to the file
`{\textit{dest}\hspace{0.2em}\textit{suffix}}'.
Surely, the same effect is achieved by
directly specifying the
argument `{\textit{dest}\hspace{0.2em}\textit{suffix}}'
in the first form.
However, that requires to set up a different file
for each child. With the alternative form of the command
all these files can have exactly the same content
which simplifies setting them up and maintaining them.

For example, the following file |draft.tex|
with a compilation flag |\version| as described in \secref{sec:flags}
compiles the main document as a draft:
%
\begin{center}
\begin{tabular}{l}
|\def\version{draft}|\\
|\input{childdoc.def}|\\
|\childdocforward{|\textit{main}|}|
\end{tabular}
\end{center}
%
Likewise, the following files |final|\textit{nn}|.tex|
compile the final version of the child document
|child|\textit{nn}|.tex|:
%
\begin{center}
\begin{tabular}{l}
|\def\version{final}|\\
|\input{childdoc.def}|\\
|\childdocforwardprefix{final}{child}|
\end{tabular}
\end{center}
%

Note that when several versions of a main file and/or of each child file
are to be generated, it may be convenient to set up a |Makefile| or
shell script to automatise the process.

%%%%%%%%%%%%%%%%%%%%%%%%%%%%%%%%%%%%%%%%%%%%%%%%%%%%%%%%%%%%%%%%%%%%%%%%%%%%%%%%
\subsection{Command Line Processing}
\label{sec:commandline}

The effect of redirection files can also be achieved by invoking
the \LaTeX{} compiler with a more elaborate command line.
Most conveniently this should be done as part
of a shell script or a |Makefile|.

When using \textsf{childdoc} in the main file, the following
command lines effectively perform a redirection
(note that depending on the shell being used,
backslashes may have to be doubled: `|\|' $\to$ `|\\|'):
%
\begin{center}
|... -jobname "|\textit{target}|" |\\|"|[\textit{flags}]%
|\input{childdoc.def}\childdocforward[|\textit{main}|]{|\textit{dest}|}"|
\end{center}
%
Here \textit{target} is the name of the output file,
\textit{main} is the name of the main file
and \textit{dest} is the name of the main or child file to be processed
(all filenames without extensions).
The optional argument \textit{main} can be omitted
if \textit{main} matches \textit{dest}.
Optionally, compilation \textit{flags} can be defined via |\def| commands.
This command line makes the \TeX{} engine believe
it is compiling the file \textit{target}
whose content is specified as the latter parameter.
The provided code then forwards the processing to
\textit{main} or \textit{dest} as described in \secref{sec:forward}.

%%%%%%%%%%%%%%%%%%%%%%%%%%%%%%%%%%%%%%%%%%%%%%%%%%%%%%%%%%%%%%%%%%%%%%%%%%%%%%%%
\subsection{Include by Input}
\label{sec:input}

Including child documents by |\include| has some restrictions by design.
Most notably, the content of a child document always occupies
its own set of pages; pages cannot be shared between child documents.
Usually, this behaviour makes perfect sense
because each child document contain an essential part of the document.
However, in some situations it may be desirable to compose
a document from a collection of parts
without having mandatory page breaks between then.
For this case, the package
provides a mechanism to include parts
by |\input| which can also be processed individually.
However, by construction this mechanism
requires manual handling of the content to be output.

%%%%%%%%%%%%%%%%%%%%%%%%%%%%%%%%%%%%%%%%
\DescribeMacro{\ifchilddocmanual}
The main file should be prepared as usual, see \secref{sec:include}.
However, the document body must make a distinction
between processing of an individual part and of the main document, e.g.:
%
\begin{center}
\begin{tabular}{l}
|\ifchilddocmanual|\\
|\input{\childdocname}|\\
|\||else|\\
\textit{document body with }|\input{|\textit{part}|}|\\
|\||fi|
\end{tabular}
\end{center}
%
The conditional |\ifchilddocmanual| is true whenever
a part to be included by |\input| is being compiled,
and the name of the part is stored in |\childdocname|.

%%%%%%%%%%%%%%%%%%%%%%%%%%%%%%%%%%%%%%%%
\DescribeMacro{\childdocby}
Each part to be included by |\input| should start with:
%
\begin{center}
\begin{tabular}{l}
|\input{childdoc.def}|\\
|\childdocby{|\textit{main}|}|\\
\end{tabular}
\end{center}
%
The directive |\childdocby| is similar to |\childdocof|
described in \secref{sec:include},
but the subsequent selection of content must be done manually.
To that end, both |\ifchilddoc| and |\ifchilddocmanual|
will be true upon processing of a part,
and the name of the part is stored in |\childdocname|.
Note that |\jobname| will be set to the filename of the current part
so that each part receives an individual |.aux| file
that does not interfere with the |.aux| file(s) of the main document.
This behaviour can be altered by the alternative form
|\childdocby[*]{|\textit{main}|}| (with a non-empty optional argument)
which uses the |.aux| file of the main document
by setting |\jobname| to \textit{main}.

%%%%%%%%%%%%%%%%%%%%%%%%%%%%%%%%%%%%%%%%%%%%%%%%%%%%%%%%%%%%%%%%%%%%%%%%%%%%%%%%
\subsection{Driver Development}
\label{sec:driver}

The \textsf{childdoc} mechanism can also be use for the development
of definition files such as \LaTeX{} styles or classes.
This case differs from the above setup with multiple parts
included by |\include| in that no |\includeonly| should be invoked.
This can be achieved by starting the include file
(before |\ProvidesPackage|) with:
%
\begin{center}
\begin{tabular}{l}
|\input{childdoc.def}|\\
|\childdocforward{|\textit{main}|}|\\
\end{tabular}
\end{center}
%
or alternatively with:
%
\begin{center}
\begin{tabular}{l}
|\input{childdoc.def}|\\
|\childdocby{|\textit{main}|}|\\
\end{tabular}
\end{center}
%
Both forms have slightly different effects as described above.
The main file is prepared as usual, see \secref{sec:include}.

%%%%%%%%%%%%%%%%%%%%%%%%%%%%%%%%%%%%%%%%%%%%%%%%%%%%%%%%%%%%%%%%%%%%%%%%%%%%%%%%
\subsection{Legacy Detection}
\label{sec:detection}

The directive |\childdocmain| in the main file can detect
whether the complete document or merely a child is to be compiled
even without using the directive |\childdocof|.
This method is deprecated because it is less robust
and there is no compelling reason to use it;
it is merely provided for backward compatibility
and it may be removed in future versions.

If the detection mechanism is to be used,
it is mandatory to correctly specify
the filename of the main file as the argument of |\childdocmain|:
%
\begin{center}
\begin{tabular}{l}
|\input{childdoc.def}|\\
|\childdocmain{|\textit{main}|}|\\
\end{tabular}
\end{center}
%
If |\jobname| does not match the argument \textit{main} of |\childdocmain|,
it is assumed that |\jobname| points to the child file to be compiled.
When using |\childdocmain| with the main file specified as argument,
it suffices to start a child file
with just |\input{|\textit{main}|}|
without loading of the package and using |\childdocof|.
If instead all processing is done
with the appropriate \textsf{childdoc} directives,
the argument of \textit{main} of |\childdocmain| can be empty.

An alternative version of the command line processing described
in \secref{sec:commandline} using the detection mechanism reads:
%
\begin{center}
|... -jobname "|\textit{target}|" "|[\textit{flags}]%
[|\def\jobname{|\textit{dest}|}|]|\input{|\textit{main}|}"|
\end{center}

%%%%%%%%%%%%%%%%%%%%%%%%%%%%%%%%%%%%%%%%%%%%%%%%%%%%%%%%%%%%%%%%%%%%%%%%%%%%%%%%
\subsection{Manual Code}
\label{sec:manual}

In case one cannot be certain whether the definitions file |childdoc.def|
is installed on the target \TeX{} distribution
and one prefers not to ship it,
it is conceivable to paste a few relevant commands into the sources.

To that end, drop all statements |\input{childdoc.def}|
and perform the replacements as outlined below.
Instead of |\childdocmain{|\textit{main}|}| add the following code
to the top of the main file:
%
\begin{center}
\begin{tabular}{l}
|\||ifdefined\childdocname\endinput\||fi\newif\ifchilddoc|\\
|\edef\childdocname{\scantokens\expandafter{\jobname\noexpand}}|\\
|\def\childdocmain{|\textit{main}|}\||ifx\childdocmain\childdocname\||else|\\
|\childdoctrue\includeonly{\childdocname}\let\jobname\childdocmain\||fi|\\
\end{tabular}
\end{center}
%
Instead of |\childdocof{|\textit{main}|}| just include the main file
at the top of each child file:
%
\begin{center}
|\input{|\textit{main}|}|
\end{center}
%
A simple redirection |\childdocforward{|\textit{dest}|}| is achieved by:
%
\begin{center}
|\def\jobname{|\textit{dest}|}\input{\jobname}|
\end{center}
%
The redirection with prefix
|\childdocforwardprefix[|\textit{prefix}|]{|\textit{dest}|}|
is accomplished by:
%
\begin{center}
\begin{tabular}{l}
|{\edef\jobname{\scantokens\expandafter{\jobname\noexpand}}|\\
|\def\redirectjob |\textit{prefix}|#1~~~{\gdef\jobname{|\textit{dest}|#1}}|\\
|\expandafter\redirectjob\jobname~~~}\input{\jobname}|
\end{tabular}
\end{center}

In an alternative approach,
child documents can be compiled by a specific command line
without additional code or specific definitions:
%
\begin{center}
|... -jobname "|\textit{target}|" "|[\textit{flags}]%
|\includeonly{|\textit{dest}|}\input{|\textit{main}|}"|
\end{center}
%

%%%%%%%%%%%%%%%%%%%%%%%%%%%%%%%%%%%%%%%%%%%%%%%%%%%%%%%%%%%%%%%%%%%%%%%%%%%%%%%%
%%%%%%%%%%%%%%%%%%%%%%%%%%%%%%%%%%%%%%%%%%%%%%%%%%%%%%%%%%%%%%%%%%%%%%%%%%%%%%%%
\section{Information}

%%%%%%%%%%%%%%%%%%%%%%%%%%%%%%%%%%%%%%%%%%%%%%%%%%%%%%%%%%%%%%%%%%%%%%%%%%%%%%%%
\subsection{Copyright}

Copyright \copyright{} 2017--2018 Niklas Beisert

This work may be distributed and/or modified under the
conditions of the \LaTeX{} Project Public License, either version 1.3
of this license or (at your option) any later version.
The latest version of this license is in
  \url{http://www.latex-project.org/lppl.txt}
and version 1.3 or later is part of all distributions of \LaTeX{}
version 2005/12/01 or later.

This work has the LPPL maintenance status `maintained'.

The Current Maintainer of this work is Niklas Beisert.

This work consists of the files |README.txt|, |childdoc.ins| and |childdoc.dtx|
as well as the derived files |childdoc.def|, |cdocsamp.tex|
with |cdocsch1.tex|, |cdocsch2.tex|, |cdocspt3.tex|, |cdocspt4.tex|,
|cdocsdrf.tex|, |cdocsfn1.tex|, |cdocsfn2.tex|
as well as |childdoc.pdf|.

%%%%%%%%%%%%%%%%%%%%%%%%%%%%%%%%%%%%%%%%%%%%%%%%%%%%%%%%%%%%%%%%%%%%%%%%%%%%%%%%
\subsection{Files and Installation}

The package consists of the files:
%
\begin{center}
\begin{tabular}{ll}
    |README.txt|   & readme file \\
    |childdoc.ins| & installation file \\
    |childdoc.dtx| & source file \\
    |childdoc.def| & definition file \\
    |cdocsamp.tex| & sample main file \\
    |cdocsch1.tex| & sample include file \\
    |cdocsch2.tex| & sample include file \\
    |cdocspt3.tex| & sample part file \\
    |cdocspt4.tex| & sample part file \\
    |cdocsdrf.tex| & sample redirection file \\
    |cdocsfn1.tex| & sample redirection file \\
    |cdocsfn2.tex| & sample redirection file \\
    |childdoc.pdf| & manual
\end{tabular}
\end{center}
%
The distribution consists of the files
|README.txt|, |childdoc.ins| and |childdoc.dtx|.
%
\begin{itemize}
\item
Run (pdf)\LaTeX{} on |childdoc.dtx|
to compile the manual |childdoc.pdf| (this file).
\item
Run \LaTeX{} on |childdoc.ins| to create the definitions file |childdoc.def|
and the sample |cdocsamp.tex| with include files
|cdocsch1.tex|, |cdocsch2.tex|, |cdocspt3.tex|, |cdocspt4.tex|,
|cdocsdrf.tex|, |cdocsfn1.tex|, |cdocsfn2.tex|.
Then copy the file |childdoc.def| to an appropriate directory of your \LaTeX{}
distribution, e.g.\ \textit{texmf-root}|/tex/latex/childdoc|.
\end{itemize}

%%%%%%%%%%%%%%%%%%%%%%%%%%%%%%%%%%%%%%%%%%%%%%%%%%%%%%%%%%%%%%%%%%%%%%%%%%%%%%%%
\subsection{Related CTAN Packages}

There are several other packages which offer a similar functionality:
%
\begin{itemize}
\item
The packages
\href{http://ctan.org/pkg/docmute}{\textsf{docmute}},
\href{http://ctan.org/pkg/includex}{\textsf{includex}} and
\href{http://ctan.org/pkg/standalone}{\textsf{standalone}}
provide commands to include only the document body of
a child file thus allowing both files to be compiled individually.
\item
The packages \href{http://ctan.org/pkg/subdocs}{\textsf{subdocs}}
and \href{http://ctan.org/pkg/subfiles}{\textsf{subfiles}}
provide structures in which the main and child documents can be
encapsulated and allowing them to be compiled individually.
The inclusion mechanism is different from the conventional |\include|.
\item
The package \href{http://ctan.org/pkg/combine}{\textsf{combine}}
is an elaborate solution to combine several documents into one.
\end{itemize}
%
See also the CTAN topic \href{http://ctan.org/topic/subdocs}{\textsf{subdocs}}
for further related packages.
The present package differs from the above solutions in that
a document structure constructed with the conventional |\include| mechanism
just needs two extra commands at the top of every file
such that all constituent files can be compiled individually.

%%%%%%%%%%%%%%%%%%%%%%%%%%%%%%%%%%%%%%%%%%%%%%%%%%%%%%%%%%%%%%%%%%%%%%%%%%%%%%%%
%\subsection{Feature Suggestions}
%
%The following is a list of features which may be useful for future
%versions of this package:
%%
%\begin{itemize}
%\item
%\ldots
%\end{itemize}

%%%%%%%%%%%%%%%%%%%%%%%%%%%%%%%%%%%%%%%%%%%%%%%%%%%%%%%%%%%%%%%%%%%%%%%%%%%%%%%%
\subsection{Revision History}

%%%%%%%%%%%%%%%%%%%%%%%%%%%%%%%%%%%%%%%%
\paragraph{v2.0:} 2018/12/30

\begin{itemize}
\item
immediate forward processing
\item
added |\childdocby| mechanism
\item
manual restructured
\end{itemize}

%%%%%%%%%%%%%%%%%%%%%%%%%%%%%%%%%%%%%%%%
\paragraph{v1.6:} 2018/01/17

\begin{itemize}
\item
application for development of include files
\item
corrections to manual
\end{itemize}

%%%%%%%%%%%%%%%%%%%%%%%%%%%%%%%%%%%%%%%%
\paragraph{v1.5:} 2017/05/21

\begin{itemize}
\item
more complete structuring introduced
\item
|\childdocof| introduced
\item
|\childdoc| renamed to |\childdocmain|
\item
|\childredirect| renamed to |\childdocforward| and |\childdocforwardprefix|
and functionality expanded
\end{itemize}

%%%%%%%%%%%%%%%%%%%%%%%%%%%%%%%%%%%%%%%%
\paragraph{v1.0:} 2017/04/27

\begin{itemize}
\item
manual and install package
\item
first version published on CTAN
\end{itemize}

%%%%%%%%%%%%%%%%%%%%%%%%%%%%%%%%%%%%%%%%
\paragraph{v0.6:} 2017/04/26

\begin{itemize}
\item
redirection mechanism added
\end{itemize}

%%%%%%%%%%%%%%%%%%%%%%%%%%%%%%%%%%%%%%%%
\paragraph{v0.5:} 2017/04/26

\begin{itemize}
\item
functionality in definition file
\end{itemize}


%%%%%%%%%%%%%%%%%%%%%%%%%%%%%%%%%%%%%%%%%%%%%%%%%%%%%%%%%%%%%%%%%%%%%%%%%%%%%%%%
%%%%%%%%%%%%%%%%%%%%%%%%%%%%%%%%%%%%%%%%%%%%%%%%%%%%%%%%%%%%%%%%%%%%%%%%%%%%%%%%
%%%%%%%%%%%%%%%%%%%%%%%%%%%%%%%%%%%%%%%%%%%%%%%%%%%%%%%%%%%%%%%%%%%%%%%%%%%%%%%%
\appendix

\settowidth\MacroIndent{\rmfamily\scriptsize 000\ }

 \DocInput{childdoc.dtx}

\end{document}
%</driver>
% \fi
%
% %%%%%%%%%%%%%%%%%%%%%%%%%%%%%%%%%%%%%%%%%%%%%%%%%%%%%%%%%%%%%%%%%%%%%%%%%%%%%%
% %%%%%%%%%%%%%%%%%%%%%%%%%%%%%%%%%%%%%%%%%%%%%%%%%%%%%%%%%%%%%%%%%%%%%%%%%%%%%%
% \section{Sample}
%\iffalse
%<*samplemain>
%\fi
%
% The following presents a sample document
% with two chapters, two parts, a title page,
% a compile flag as well as three forwarding files to set the flag.
% It consists of eight |.tex| files:
% \begin{center}
% \begin{tabular}{ll}
% |cdocsamp.tex|&main file\\
% |cdocsch1.tex|&include file for chapter 1\\
% |cdocsch2.tex|&include file for chapter 2\\
% |cdocspt3.tex|&include file for part 3\\
% |cdocspt4.tex|&include file for part 4\\
% |cdocsdrf.tex|&forwarding file for main file in draft mode\\
% |cdocsfi1.tex|&forwarding file for final version of chapter 1\\
% |cdocsfi2.tex|&forwarding file for final version of chapter 2\\
% \end{tabular}
% \end{center}
% Each of the eight files can be compiled directly by the \LaTeX{} compiler.
%
% %%%%%%%%%%%%%%%%%%%%%%%%%%%%%%%%%%%%%%
% \paragraph{Main File.}
%
% The main file is called |cdocsamp.tex|.
%
% Load the \textsf{childdoc} definitions and
% declare the filename for the main document:
%    \begin{macrocode}
\input{childdoc.def}
\childdocmain{}
%    \end{macrocode}

% Optional override for |\version| flag:
%    \begin{macrocode}
%%\ifchilddoc\else\providecommand{\version}{draft}\fi
%    \end{macrocode}

% Define the default values for the |\version| flag
% (|final| for the main file and |draft| for childs):
%    \begin{macrocode}
\ifchilddoc
\providecommand{\version}{draft}
\else
\providecommand{\version}{final}
\fi
%    \end{macrocode}

% Load the standard document class:
%    \begin{macrocode}
\documentclass[12pt]{article}
%    \end{macrocode}

% Start the document body:
%    \begin{macrocode}
\begin{document}
%    \end{macrocode}

% Declare a title page.
% Print title, part of document being processed and version flag:
%    \begin{macrocode}
\addtocounter{page}{-1}
\begin{center}
{\LARGE\bfseries{}childdoc example\par}
\vspace{1cm}
\ifchilddoc
\ifchilddocmanual part\else chapter\fi:
`\childdocname' of `\childdocjob'\par
\else
main document: `\childdocjob'\par
\fi
version: \version\par
\end{center}
\newpage
%    \end{macrocode}

% Manually include selected file,
% otherwise process as usual:
%    \begin{macrocode}
\ifchilddocmanual
\section*{part `\childdocname'}
\input{\childdocname}
\else
%    \end{macrocode}

% Include the two chapters:
%    \begin{macrocode}
\include{cdocsch1}
\include{cdocsch2}
%    \end{macrocode}

% Include the two parts unless only chapters should be displayed:
%    \begin{macrocode}
\ifchilddoc\else
\section{part three}
\input{cdocspt3}
\section{part four}
\input{cdocspt4}
\fi
%    \end{macrocode}

% Process as usual until here:
%    \begin{macrocode}
\fi
%    \end{macrocode}

% End of document body:
%    \begin{macrocode}
\end{document}
%    \end{macrocode}
%\iffalse
%</samplemain>
%\fi
%
% %%%%%%%%%%%%%%%%%%%%%%%%%%%%%%%%%%%%%%
% \paragraph{Chapter Include Files.}
%
% The include files are called |cdocsch1.tex| and |cdocsch2.tex|.
%
%\iffalse
%<*samplechap1|samplechap2>
%\fi

% Optional override for |\version| flag:
%    \begin{macrocode}
%%\providecommand{\version}{final}
%    \end{macrocode}

% Include the main document:
%    \begin{macrocode}
\input{childdoc.def}
\childdocof{cdocsamp}
%    \end{macrocode}

%\iffalse
%</samplechap1|samplechap2>
%\fi
%
%\iffalse
%<*samplechap1>
%\fi
% Some text for chapter 1:
%    \begin{macrocode}
\section{one}
some text in chapter one
%    \end{macrocode}

%\iffalse
%</samplechap1>
%\fi
% Some text for chapter 2:
%\iffalse
%<*samplechap2>
%\fi
%    \begin{macrocode}
\section{two}
more text in chapter two
%    \end{macrocode}

%\iffalse
%</samplechap2>
%\fi
%
% %%%%%%%%%%%%%%%%%%%%%%%%%%%%%%%%%%%%%%
% \paragraph{Part Include Files.}
%
% The include files are called |cdocspt3.tex| and |cdocspt4.tex|.
%
%\iffalse
%<*samplepart3|samplepart4>
%\fi

% Optional override for |\version| flag:
%    \begin{macrocode}
%%\providecommand{\version}{final}
%    \end{macrocode}

% Include the main document:
%    \begin{macrocode}
\input{childdoc.def}
\childdocby{cdocsamp}
%    \end{macrocode}

%\iffalse
%</samplepart3|samplepart4>
%\fi
%
%\iffalse
%<*samplepart3>
%\fi
% Some text for part 3:
%    \begin{macrocode}
some text in part three
%    \end{macrocode}

%\iffalse
%</samplepart3>
%\fi
% Some text for part 4:
%\iffalse
%<*samplepart4>
%\fi
%    \begin{macrocode}
more text in part four
%    \end{macrocode}

%\iffalse
%</samplepart4>
%\fi
%
% %%%%%%%%%%%%%%%%%%%%%%%%%%%%%%%%%%%%%%
% \paragraph{Forwarding for a Complete Draft.}
%
% The following forwarding file |cdocsdrf.tex|
% compiles the main document in draft mode:
%\iffalse
%<*sampledraft>
%\fi
%    \begin{macrocode}
\def\version{draft}
\input{childdoc.def}
\childdocforward{cdocsamp}
%    \end{macrocode}

%\iffalse
%</sampledraft>
%\fi
%
% %%%%%%%%%%%%%%%%%%%%%%%%%%%%%%%%%%%%%%
% \paragraph{Forwarding for Final Version of the Chapters.}
%
% The following forwarding files |cdocsfn1.tex| and |cdocsfn2.tex|
% (with identical content)
% compile the final versions of the child documents
% |cdocsch1.tex| and |cdocsch2.tex|, respectively:
%\iffalse
%<*samplefinal>
%\fi
%    \begin{macrocode}
\def\version{final}
\input{childdoc.def}
\childdocforwardprefix[cdocsamp]{cdocsfn}{cdocsch}
%    \end{macrocode}

%\iffalse
%</samplefinal>
%\fi
%
% %%%%%%%%%%%%%%%%%%%%%%%%%%%%%%%%%%%%%%
% \paragraph{Command Line Processing.}
%
% The following three command lines generate the output files
% |cdocscld|, |cdocscl1| and |cdocscl2|
% which should be identical to
% |cdocsdrf|, |cdocsch1| and |cdocsfn2|, respectively:
% \begin{center}
% \begin{tabular}{l}
% |latex -jobname cdocscld \|\\
% |  "\def\version{draft}\input{childdoc.def}\childdocforward{cdocsamp}"|\\
% |latex -jobname cdocscl1 \|\\
% |  "\input{childdoc.def}\childdocforward[cdocsamp]{cdocsch1}"|\\
% |latex -jobname cdocscl2 \|\\
% |  "\def\version{final}\input{childdoc.def}\childdocforward{cdocsch2}"|
% \end{tabular}
% \end{center}
% Note that the trailing backslash on each first line
% merely continues the input to the second line
% (for convenient cut ant paste).
% Furthermore, the command |latex| can be replaced by any
% of its alternative versions such as |pdflatex|.
%
% %%%%%%%%%%%%%%%%%%%%%%%%%%%%%%%%%%%%%%%%%%%%%%%%%%%%%%%%%%%%%%%%%%%%%%%%%%%%%%
% %%%%%%%%%%%%%%%%%%%%%%%%%%%%%%%%%%%%%%%%%%%%%%%%%%%%%%%%%%%%%%%%%%%%%%%%%%%%%%
% \section{Implementation}
%\iffalse
%<*package>
%\fi
%
% This section describes the definitions file |childdoc.def|.

% The definitions cannot be loaded using |\usepackage| or |\RequirePackage|
% which has a mechanism to prevent loading a style file more than once.
% When loading the definitions by means of |\input|
% multiple instances have to be prevented manually:
%\iffalse
%This code needs to be before the `\ProvidesFile' directive
%which is defined at the beginning of this file.
%Therefore it is also placed there and commented out here.
%</package>
%<*discard>
%\fi
%    \begin{macrocode}
\ifdefined\childdocmain\endinput\fi
%    \end{macrocode}
%\iffalse
%</discard>
%<*package>
%\fi
%
% \macro{\ifchilddoc}
% \macro{\ifchilddocmanual}
% The conditional |\ifchilddoc| tells whether a
% child (true) or main (false) document is being compiled.
% The conditional |\ifchilddocmanual| tells whether
% the |\includeonly| mechanism is used (false) or
% the selection of child files must be performed manually (true).
% The definitions initialise to false:
%    \begin{macrocode}
\newif\ifchilddoc
\newif\ifchilddocmanual
%    \end{macrocode}

% \macro{\childdocname}
% \macro{\childdocjob}
% The macro |\childdocname| stores the name of the main document
% to be compiled. The macro |\childdocjob| stores the name of
% the document on which the \LaTeX{} compiler was originally invoked.
% The content of |\jobname| cannot be compared
% to filenames specified in the source due to different catcodes.
% The following code rescans |\jobname|, stores the result
% in |\childdocname| and saves a copy in |\childdocjob|:
%    \begin{macrocode}
\edef\childdocname{\scantokens\expandafter{\jobname\noexpand}}
\let\childdocjob\childdocname
%    \end{macrocode}

% \macro{\childdocdisable}
% The macro |\childdocdisable| prevents the main file
% from being processed more than once.
% At this stage, the main document command |\childdocmain|
% is assumed to be called once again where it should do nothing.
% Any subsequent call to it should prevent
% a secondary processing of the main document
% It overwrites the forwarding commands
% |\childdocof| and |\childdocforward|
% with empty macros to prevent further inclusions of the main document:
%    \begin{macrocode}
\newcommand{\childdocdisable}
{
  \renewcommand{\childdocmain}[1]{\renewcommand{\childdocmain}[1]{\endinput}}
  \renewcommand{\childdocof}[1]{}
  \renewcommand{\childdocby}[2][]{}
  \renewcommand{\childdocforward}[2][]{}
  \renewcommand{\childdocdisable}{}
}
%    \end{macrocode}

% \macro{\childdocmain}
% The macro |\childdocmain| is to be called at the top of the main file
% with nothing or the main filename (without extension) as argument.
% First, it breaks loops.
% If the argument is not empty and does not match |\childdocname|
% (which is set by the first inclusion of |childdoc.def|),
% |\ifchilddoc| is set to true, |\includeonly| is applied to the child file
% and |\jobname| is set to the main file
% (for proper handling of |.aux| files):
%    \begin{macrocode}
\newcommand{\childdocmain}[1]
{
  \childdocdisable\childdocmain{}
  \if?#1?\else
    \begingroup
      \def\childdoctmp{#1}
      \ifx\childdoctmp\childdocname
        \def\childdoctmp{}
      \else
        \def\childdoctmp
        {
          \childdoctrue
          \includeonly{\childdocname}
          \def\childdocjob{#1}
          \def\jobname{#1}
        }
      \fi
      \expandafter
    \endgroup
    \childdoctmp
  \fi
}
%    \end{macrocode}

% \macro{\childdocof}
% The command |\childdocof| redirects
% compilation to the main file |#1|.
%    \begin{macrocode}
\newcommand{\childdocof}[1]
{
  \childdocdisable
  \childdoctrue
  \includeonly{\childdocname}
  \def\jobname{#1}
  \def\childdocjob{#1}
  \input{#1}
}
%    \end{macrocode}

% \macro{\childdocby}
% The command |\childdocby| ....
%    \begin{macrocode}
\newcommand{\childdocby}[2][]
{
  \childdocdisable
  \childdoctrue
  \childdocmanualtrue
  \if?#1?\else
    \def\jobname{#2}
  \fi
  \def\childdocjob{#2}
  \input{#2}
  \endinput
}
%    \end{macrocode}

% \macro{\childdocforward}
% The command |\childdocforward| redirects
% compilation to the main file or
% (if the optional argument is given) a child file.
% Parameters are set as if the main file
% or a child file starting with |\childdocof| was compiled.
% Then compilation is handed over to the main file:
%    \begin{macrocode}
\newcommand{\childdocforward}[2][]
{
  \begingroup
    \if?#1?
      \def\childdoctmp
      {
        \def\childdocname{#2}
        \def\childdocjob{#2}
        \def\jobname{#2}
        \input{#2}
        \endinput
      }
    \else
      \def\childdoctmp
      {
        \childdocdisable
        \def\childdocname{#2}
        \childdoctrue
        \includeonly{#2}
        \def\childdocjob{#1}
        \def\jobname{#1}
        \input{#1}
        \endinput
      }
    \fi
    \expandafter
  \endgroup
  \childdoctmp
}
%    \end{macrocode}

% \macro{\childdocforwardprefix}
% The command |\childdocforwardprefix| redirects
% compilation to the main or a child file by means of a pattern.
% The prefix |#1| in the current filename is replaced by |#2|
% and the suffix of the current filename is kept
% (it is assumed that the filename does not contain the substring `|~~~|'
% which is used as a delimiter).
% Compilation is handed over to the new file by |\childdocforward|:
%    \begin{macrocode}
\newcommand{\childdocforwardprefix}[3][]
{
  \begingroup
    \def\childdocextract #2##1~~~{\def\childdoctmp{\childdocforward[#1]{#3##1}}}
    \expandafter\childdocextract\childdocname~~~
    \expandafter
  \endgroup
  \childdoctmp
}
%    \end{macrocode}

% \macro{\childdoc}
% The deprecated macro |\childdoc| is a legacy version of |\childdocmain|:
%    \begin{macrocode}
\newcommand{\childdoc}{\childdocmain}
%    \end{macrocode}

% \macro{\childdocredirect}
% The deprecated macro |\childdocredirect| is a legacy version
% of |\childdocforward| and |\childdocforwardprefix|:
%    \begin{macrocode}
\newcommand{\childdocredirect}[2][]
{
  \begingroup
    \if?#1?
      \def\childdoctmp{\childdocforward{#2}}
    \else
      \def\childdoctmp{\childdocforwardprefix{#1}{#2}}
    \fi
    \expandafter
  \endgroup
  \childdoctmp
}
%    \end{macrocode}

%\iffalse
%</package>
%\fi
%
\endinput

\childdocby{cdocsamp}
%    \end{macrocode}

%\iffalse
%</samplepart3|samplepart4>
%\fi
%
%\iffalse
%<*samplepart3>
%\fi
% Some text for part 3:
%    \begin{macrocode}
some text in part three
%    \end{macrocode}

%\iffalse
%</samplepart3>
%\fi
% Some text for part 4:
%\iffalse
%<*samplepart4>
%\fi
%    \begin{macrocode}
more text in part four
%    \end{macrocode}

%\iffalse
%</samplepart4>
%\fi
%
% %%%%%%%%%%%%%%%%%%%%%%%%%%%%%%%%%%%%%%
% \paragraph{Forwarding for a Complete Draft.}
%
% The following forwarding file |cdocsdrf.tex|
% compiles the main document in draft mode:
%\iffalse
%<*sampledraft>
%\fi
%    \begin{macrocode}
\def\version{draft}
% \iffalse
%
% childdoc.dtx Copyright (C) 2017-2018 Niklas Beisert
%
% This work may be distributed and/or modified under the
% conditions of the LaTeX Project Public License, either version 1.3
% of this license or (at your option) any later version.
% The latest version of this license is in
%   http://www.latex-project.org/lppl.txt
% and version 1.3 or later is part of all distributions of LaTeX
% version 2005/12/01 or later.
%
% This work has the LPPL maintenance status `maintained'.
%
% The Current Maintainer of this work is Niklas Beisert.
%
% This work consists of the files childdoc.dtx and childdoc.ins
% and the derived files childdoc.def and cdocsamp.tex with
% cdocsch1.tex, cdocsch2.tex, cdocsdrf.tex, cdocsfn1.tex, cdocsfn2.tex.
%
%<package>\ifdefined\childdocmain\endinput\fi
%<package>\ProvidesFile{childdoc.def}[2018/12/30 v2.0 child document driver]
%<samplemain>\ProvidesFile{cdocsamp.tex}[2018/12/30 v2.0 sample for childdoc]
%<*driver>
%\ProvidesFile{childdoc.drv}[2018/12/30 v2.0 childdoc reference manual file]
\PassOptionsToClass{10pt,a4paper}{article}
\documentclass{ltxdoc}

\usepackage[margin=35mm]{geometry}
\usepackage{hyperref}
\usepackage{hyperxmp}
\usepackage[usenames]{color}

\hypersetup{colorlinks=true}
\hypersetup{pdfstartview=FitH}
\hypersetup{pdfpagemode=UseNone}
\hypersetup{pdfsource={}}
\hypersetup{pdflang={en-UK}}
\hypersetup{pdfcopyright={Copyright 2017-2018 Niklas Beisert.
  This work may be distributed and/or modified under the
  conditions of the LaTeX Project Public License, either version 1.3
  of this license or (at your option) any later version.}}
\hypersetup{pdflicenseurl={http://www.latex-project.org/lppl.txt}}
\hypersetup{pdfcontactaddress={ETH Zurich, ITP, HIT K,
  Wolfgang-Pauli-Strasse 27}}
\hypersetup{pdfcontactpostcode={8093}}
\hypersetup{pdfcontactcity={Zurich}}
\hypersetup{pdfcontactcountry={Switzerland}}
\hypersetup{pdfcontactemail={nbeisert@itp.phys.ethz.ch}}
\hypersetup{pdfcontacturl={http://people.phys.ethz.ch/\xmptilde nbeisert/}}

\newcommand{\secref}[1]{\hyperref[#1]{section \ref*{#1}}}

\parskip1ex
\parindent0pt
\let\olditemize\itemize
\def\itemize{\olditemize\parskip0pt}

\begin{document}

\title{The \textsf{childdoc} Package}
\hypersetup{pdftitle={The childdoc Package}}
\author{Niklas Beisert\\[2ex]
  Institut f\"ur Theoretische Physik\\
  Eidgen\"ossische Technische Hochschule Z\"urich\\
  Wolfgang-Pauli-Strasse 27, 8093 Z\"urich, Switzerland\\[1ex]
  \href{mailto:nbeisert@itp.phys.ethz.ch}
  {\texttt{nbeisert@itp.phys.ethz.ch}}}
\hypersetup{pdfauthor={Niklas Beisert}}
\hypersetup{pdfsubject={Manual for the LaTeX2e Package childdoc}}
\date{30 December 2018, \textsf{v2.0}}
\maketitle

\begin{abstract}\noindent
\textsf{childdoc} is a \LaTeXe{} package
that enables the direct compilation
of document sections included by |\include|
to individual files.
\end{abstract}

\begingroup
\parskip0ex
\tableofcontents
\endgroup

%%%%%%%%%%%%%%%%%%%%%%%%%%%%%%%%%%%%%%%%%%%%%%%%%%%%%%%%%%%%%%%%%%%%%%%%%%%%%%%%
%%%%%%%%%%%%%%%%%%%%%%%%%%%%%%%%%%%%%%%%%%%%%%%%%%%%%%%%%%%%%%%%%%%%%%%%%%%%%%%%
\section{Introduction}

\LaTeX{} provides a mechanism to structure a large document (such as a book)
into a main file and several child files (containing the chapters)
using the |\include| command.
This mechanism is beneficial for documents
which span hundreds of pages in order to
make the source file(s) more manageable.
Moreover, compilation can be restricted to
selected child files by means of the |\includeonly| command.
The latter feature can be used to reduce the compilation time while editing
(this was significantly more useful in the earlier days of \LaTeX{})
or to generate a smaller document which is easier to navigate.
Another application of |\includeonly| is to generate
documents consisting of selected parts of the complete document.

However, there are a few drawbacks of the plain |\include| mechanism:
\begin{itemize}
\item
The child files cannot be compiled on their own,
they can only be compiled via the main file.
A naive editing environment
(such as a text editor with an option
to have the current file processed by \LaTeX)
may require one to switch to the main file before compiling;
attempting to compile the child file produces errors.
\item
The main file must be modified (each time)
to adjust the |\includeonly| command
to the present needs. This easily leaves the main file in a messy state.
\item
The generated document will always carry the filename
of the main document. This is inconvenient if
several child files are to be compiled and
to be kept for distribution.
\end{itemize}

The present package provides a simple interface
to make child files individually compilable by \LaTeX{}.
Compiling a child file then has the same effect as compiling
the main file with an |\includeonly| command
to select the appropriate child.
Moreover the generated document will carry the name of the child
rather than the main file.
This resolves all three above issues.

This feature is meant to make the editing of books,
thesis documents and lecture notes somewhat more convenient.
However, the package can also be used efficiently for
composing a series of documents (such as exercise sheets)
which are typically distributed individually.
It then assists the author in generating the individual documents
(potentially in different versions)
as well as a document containing the collected series.
Another application is in developing style files
or other kinds of included material
where compilation of the style file could redirect
to a sample or test file.

%%%%%%%%%%%%%%%%%%%%%%%%%%%%%%%%%%%%%%%%%%%%%%%%%%%%%%%%%%%%%%%%%%%%%%%%%%%%%%%%
%%%%%%%%%%%%%%%%%%%%%%%%%%%%%%%%%%%%%%%%%%%%%%%%%%%%%%%%%%%%%%%%%%%%%%%%%%%%%%%%
\section{Usage}

First of all, the package \textsf{childdoc} is \emph{not} a standard
\LaTeXe{} |.sty| style file! Therefore it needs to be invoked in
a non-standard way.

%%%%%%%%%%%%%%%%%%%%%%%%%%%%%%%%%%%%%%%%%%%%%%%%%%%%%%%%%%%%%%%%%%%%%%%%%%%%%%%%
\subsection{Included Files}
\label{sec:include}

%%%%%%%%%%%%%%%%%%%%%%%%%%%%%%%%%%%%%%%%
\DescribeMacro{\childdocmain}
To use the package, add the commands
\begin{center}
\begin{tabular}{l}
|\input{childdoc.def}|\\
|\childdocmain{}|\\
\end{tabular}
\end{center}
at the very top of the main \LaTeX{} file,
in particular \emph{before} the |\documentclass| statement!
The argument of |\childdocmain| should be left empty
(but it must be present).

%%%%%%%%%%%%%%%%%%%%%%%%%%%%%%%%%%%%%%%%
\DescribeMacro{\childdocof}
Furthermore, add the commands
\begin{center}
\begin{tabular}{l}
|\input{childdoc.def}|\\
|\childdocof{|\textit{main}|}|\\
\end{tabular}
\end{center}
at the top of every child file \textit{child}
which is included by |\include{|\textit{child}|}|
from within the main file
(or at least for those files to be compiled individually).
The argument \textit{main} must be the filename of the main file.

There are a couple of
considerations in setting up the main and child documents:

%%%%%%%%%%%%%%%%%%%%%%%%%%%%%%%%%%%%%%%%
\paragraph{Restrictions.}

Please note the following restrictions:
\begin{itemize}
\item
|\childdocmain| must be called with one argument \textit{main}
to ensure compatibility with earlier version of the package.
It must either be empty (|\childdocmain{}|)
or precisely match the filename of the main file in which it is specified.
See \secref{sec:detection} for further information.
\item
The filename \textit{main} must be specified without the |.tex| extension.
\item
The filename \textit{main} is case sensitive
(even in case-insensitive file systems)
due to internal string comparison.
\item
The argument \textit{main} should be fully expanded, it cannot be a macro.
\item
Subdirectories and special characters should be avoided in filenames.
\item
The command |\childdocmain{|\textit{main}|}| must be followed by a whitespace.
It should not be followed immediately by another command
or by a comment mark `|%|'.
This is because the \TeX{} parser reads the token immediately following
the argument of |\childdocmain| and puts it
at the beginning of every child section;
however, a white\-space is ignored.
\end{itemize}

%%%%%%%%%%%%%%%%%%%%%%%%%%%%%%%%%%%%%%%%
\paragraph{Content of Main File.}

It is advisable to place all content in the child files included by |\include|.
Any output contained in the main file will appear in all child documents
unless suppressed manually;
it cannot be suppressed automatically by the |\includeonly| directive
and thus should normally be avoided.
A method to include some content in the main file
by means of conditional processing is described in \secref{sec:conditional}.

%%%%%%%%%%%%%%%%%%%%%%%%%%%%%%%%%%%%%%%%
\paragraph{Page Numbering.}

When only a part of the document is compiled,
the appropriate numbering of pages
(as well as other status parameters)
is determined from the |.aux| files.
The latter contain information from previous passes.
However this information needs to propagate through
all intermediate child documents.
Therefore the page numbering in child documents may well
be inconsistent until the complete document is compiled at least once.

A useful (if unconventional) way to always ensure a consistent
page numbering is to restart the numbering in each child document
and denote the pages by `\textit{child}|.|\textit{page}'
where \textit{child} represents the chapter/section number of the child file.
This can be achieved by the command
|\numberwithin{page}{|\textit{child}|}|
of the \textsf{amsmath} package
where \textit{child} can be |chapter| or |section|
depending on the chosen structuring.
Alternatively, one can modify the macro |\thepage| appropriately
and reset the counter |page| at the start of each child file.

%%%%%%%%%%%%%%%%%%%%%%%%%%%%%%%%%%%%%%%%%%%%%%%%%%%%%%%%%%%%%%%%%%%%%%%%%%%%%%%%
\subsection{Conditional Processing}
\label{sec:conditional}

The package provides a mechanism to compile different versions
of a document. To customise the versions further some conditional processing
can come in handy to distinguish which version is being compiled.
The package provides two macros to describe the compilation context:

%%%%%%%%%%%%%%%%%%%%%%%%%%%%%%%%%%%%%%%%
\DescribeMacro{\ifchilddoc}
The conditional |\ifchilddoc| distinguishes between the compilation of
child documents and the main document:
%
\begin{center}
|\ifchilddoc |\textit{child-code}| |[|\||else |\textit{main-code}]| \||fi|
\end{center}

%%%%%%%%%%%%%%%%%%%%%%%%%%%%%%%%%%%%%%%%
\DescribeMacro{\childdocname}
\DescribeMacro{\childdocjob}
The macro |\childdocname| contains the filename (without extension)
of the main or child file being processed.
Note that |\childdocjob| will always contain the name of the main file.

%%%%%%%%%%%%%%%%%%%%%%%%%%%%%%%%%%%%%%%%
\paragraph{Title Page.}

Conditional processing can be used to include a title or banner page
in the main document when proper precautions are taken.
Importantly, the code in the main file should ensure that the page counter
(as well as other status parameters which are stored in the |.aux| files)
takes the same value after the conditional processing.
Otherwise the page numbers may take divergent values
depending on which part is compiled.

For example, a title page could be declared by:
%
\begin{center}
\begin{tabular}{l}
|\ifchilddoc\||else|\\
|\addtocounter{page}{-1}|\\
\textit{code for title page}\\
|\newpage|\\
|\||fi|
\end{tabular}
\end{center}
%
A banner page for the child documents can be generated by:
%
\begin{center}
\begin{tabular}{l}
|\ifchilddoc|\\
|\addtocounter{page}{-1}|\\
\textit{code for banner page}\\
|\newpage|\\
|\||fi|
\end{tabular}
\end{center}
%
Here one could write a message such as:
\begin{center}
|This is the part \childdocname{} of \childdocjob{}.|
\end{center}

%%%%%%%%%%%%%%%%%%%%%%%%%%%%%%%%%%%%%%%%%%%%%%%%%%%%%%%%%%%%%%%%%%%%%%%%%%%%%%%%
\subsection{Flags}
\label{sec:flags}

The package makes it easy to generate different versions
of the main or child documents.
To this end compilation flags can be defined
and assigned different default values.
They will be particularly useful in conjunction
with the forwarding mechanism described in \secref{sec:forward}.

For example, it may be useful to have a flag |\version|
which can be set to |draft| or |final|.
The document source will contain some conditional code
depending on the value of |\version|.
Suppose further, the flag should default to |final| for the main file
and to |draft| for child files
which is a natural assignment for editing the document.
This is achieved by placing the following code
in the preamble of the main document
(below the |\childdocmain| directive):
%
\begin{center}
\begin{tabular}{l}
|\ifchilddoc|\\
|\providecommand{\version}{draft}|\\
|\||else|\\
|\providecommand{\version}{final}|\\
|\||fi|
\end{tabular}
\end{center}
%
The definition by |\providecommand| makes sure
that previous definitions are not overwritten.
Further statements |\providecommand{\version}{...}|
can thus be added before the above code to override it.

For the main file, one might add a line
(between |\childdocmain| and the above block)
%
\begin{center}
|%\ifchilddoc\||else\providecommand{\version}{draft}\||fi|
\end{center}
%
which can be uncommented to produce a draft version.
Likewise one can add a line to the very top of a child file
(above the |\childdocof{|\textit{main}|}| directive)
%
\begin{center}
|%\providecommand{\version}{final}|
\end{center}
%
which can be uncommented to produce the final version of this child document.

%%%%%%%%%%%%%%%%%%%%%%%%%%%%%%%%%%%%%%%%%%%%%%%%%%%%%%%%%%%%%%%%%%%%%%%%%%%%%%%%
\subsection{Forwarding}
\label{sec:forward}

Different versions of the main or child documents
using compilation flags as described in \secref{sec:flags}
can be (permanently) stored in different files
for convenient compilation, viewing and distribution.
To this end, the package defines a command
to pass on compilation to a different file:

%%%%%%%%%%%%%%%%%%%%%%%%%%%%%%%%%%%%%%%%
\DescribeMacro{\childdocforward}
The command |\childdocforward| redirects processing to
another source file:
%
\begin{center}
\begin{tabular}{l}
|\input{childdoc.def}|\\
|\childdocforward[|\textit{main}|]{|\textit{dest}|}|\\
\end{tabular}
\end{center}
%
The argument \textit{dest} is the destination file
(without extension).
It should be the main file or one of the child files.
Note that further \textsf{childdoc} directives
such as |\childdocof| and |\childdocforward|
in the indicated file will be processed in this form.
The optional argument \textit{main}
passes on directly to the main file \textit{main}
while pretending to compile the child \textit{dest}.
This form behaves as if \textit{dest}
issues |\childdocof{|\textit{main}|}| right away,
and no further \textsf{childdoc} directives will be processed.

%%%%%%%%%%%%%%%%%%%%%%%%%%%%%%%%%%%%%%%%
\DescribeMacro{\...prefix}
In the alternative form |\childdocforwardprefix|,
%
\begin{center}
\begin{tabular}{l}
|\input{childdoc.def}|\\
|\childdocforwardprefix[|\textit{main}|]{|\textit{prefix}|}{|\textit{dest}|}|
\end{tabular}
\end{center}
%
the destination file is determined by a pattern
depending on the current file:
To make this work, the current file must be called
`{\textit{prefix}\hspace{0.2em}\textit{suffix}}'
with \textit{prefix} matching precisely the argument.
Processing is then passed on to the file
`{\textit{dest}\hspace{0.2em}\textit{suffix}}'.
Surely, the same effect is achieved by
directly specifying the
argument `{\textit{dest}\hspace{0.2em}\textit{suffix}}'
in the first form.
However, that requires to set up a different file
for each child. With the alternative form of the command
all these files can have exactly the same content
which simplifies setting them up and maintaining them.

For example, the following file |draft.tex|
with a compilation flag |\version| as described in \secref{sec:flags}
compiles the main document as a draft:
%
\begin{center}
\begin{tabular}{l}
|\def\version{draft}|\\
|\input{childdoc.def}|\\
|\childdocforward{|\textit{main}|}|
\end{tabular}
\end{center}
%
Likewise, the following files |final|\textit{nn}|.tex|
compile the final version of the child document
|child|\textit{nn}|.tex|:
%
\begin{center}
\begin{tabular}{l}
|\def\version{final}|\\
|\input{childdoc.def}|\\
|\childdocforwardprefix{final}{child}|
\end{tabular}
\end{center}
%

Note that when several versions of a main file and/or of each child file
are to be generated, it may be convenient to set up a |Makefile| or
shell script to automatise the process.

%%%%%%%%%%%%%%%%%%%%%%%%%%%%%%%%%%%%%%%%%%%%%%%%%%%%%%%%%%%%%%%%%%%%%%%%%%%%%%%%
\subsection{Command Line Processing}
\label{sec:commandline}

The effect of redirection files can also be achieved by invoking
the \LaTeX{} compiler with a more elaborate command line.
Most conveniently this should be done as part
of a shell script or a |Makefile|.

When using \textsf{childdoc} in the main file, the following
command lines effectively perform a redirection
(note that depending on the shell being used,
backslashes may have to be doubled: `|\|' $\to$ `|\\|'):
%
\begin{center}
|... -jobname "|\textit{target}|" |\\|"|[\textit{flags}]%
|\input{childdoc.def}\childdocforward[|\textit{main}|]{|\textit{dest}|}"|
\end{center}
%
Here \textit{target} is the name of the output file,
\textit{main} is the name of the main file
and \textit{dest} is the name of the main or child file to be processed
(all filenames without extensions).
The optional argument \textit{main} can be omitted
if \textit{main} matches \textit{dest}.
Optionally, compilation \textit{flags} can be defined via |\def| commands.
This command line makes the \TeX{} engine believe
it is compiling the file \textit{target}
whose content is specified as the latter parameter.
The provided code then forwards the processing to
\textit{main} or \textit{dest} as described in \secref{sec:forward}.

%%%%%%%%%%%%%%%%%%%%%%%%%%%%%%%%%%%%%%%%%%%%%%%%%%%%%%%%%%%%%%%%%%%%%%%%%%%%%%%%
\subsection{Include by Input}
\label{sec:input}

Including child documents by |\include| has some restrictions by design.
Most notably, the content of a child document always occupies
its own set of pages; pages cannot be shared between child documents.
Usually, this behaviour makes perfect sense
because each child document contain an essential part of the document.
However, in some situations it may be desirable to compose
a document from a collection of parts
without having mandatory page breaks between then.
For this case, the package
provides a mechanism to include parts
by |\input| which can also be processed individually.
However, by construction this mechanism
requires manual handling of the content to be output.

%%%%%%%%%%%%%%%%%%%%%%%%%%%%%%%%%%%%%%%%
\DescribeMacro{\ifchilddocmanual}
The main file should be prepared as usual, see \secref{sec:include}.
However, the document body must make a distinction
between processing of an individual part and of the main document, e.g.:
%
\begin{center}
\begin{tabular}{l}
|\ifchilddocmanual|\\
|\input{\childdocname}|\\
|\||else|\\
\textit{document body with }|\input{|\textit{part}|}|\\
|\||fi|
\end{tabular}
\end{center}
%
The conditional |\ifchilddocmanual| is true whenever
a part to be included by |\input| is being compiled,
and the name of the part is stored in |\childdocname|.

%%%%%%%%%%%%%%%%%%%%%%%%%%%%%%%%%%%%%%%%
\DescribeMacro{\childdocby}
Each part to be included by |\input| should start with:
%
\begin{center}
\begin{tabular}{l}
|\input{childdoc.def}|\\
|\childdocby{|\textit{main}|}|\\
\end{tabular}
\end{center}
%
The directive |\childdocby| is similar to |\childdocof|
described in \secref{sec:include},
but the subsequent selection of content must be done manually.
To that end, both |\ifchilddoc| and |\ifchilddocmanual|
will be true upon processing of a part,
and the name of the part is stored in |\childdocname|.
Note that |\jobname| will be set to the filename of the current part
so that each part receives an individual |.aux| file
that does not interfere with the |.aux| file(s) of the main document.
This behaviour can be altered by the alternative form
|\childdocby[*]{|\textit{main}|}| (with a non-empty optional argument)
which uses the |.aux| file of the main document
by setting |\jobname| to \textit{main}.

%%%%%%%%%%%%%%%%%%%%%%%%%%%%%%%%%%%%%%%%%%%%%%%%%%%%%%%%%%%%%%%%%%%%%%%%%%%%%%%%
\subsection{Driver Development}
\label{sec:driver}

The \textsf{childdoc} mechanism can also be use for the development
of definition files such as \LaTeX{} styles or classes.
This case differs from the above setup with multiple parts
included by |\include| in that no |\includeonly| should be invoked.
This can be achieved by starting the include file
(before |\ProvidesPackage|) with:
%
\begin{center}
\begin{tabular}{l}
|\input{childdoc.def}|\\
|\childdocforward{|\textit{main}|}|\\
\end{tabular}
\end{center}
%
or alternatively with:
%
\begin{center}
\begin{tabular}{l}
|\input{childdoc.def}|\\
|\childdocby{|\textit{main}|}|\\
\end{tabular}
\end{center}
%
Both forms have slightly different effects as described above.
The main file is prepared as usual, see \secref{sec:include}.

%%%%%%%%%%%%%%%%%%%%%%%%%%%%%%%%%%%%%%%%%%%%%%%%%%%%%%%%%%%%%%%%%%%%%%%%%%%%%%%%
\subsection{Legacy Detection}
\label{sec:detection}

The directive |\childdocmain| in the main file can detect
whether the complete document or merely a child is to be compiled
even without using the directive |\childdocof|.
This method is deprecated because it is less robust
and there is no compelling reason to use it;
it is merely provided for backward compatibility
and it may be removed in future versions.

If the detection mechanism is to be used,
it is mandatory to correctly specify
the filename of the main file as the argument of |\childdocmain|:
%
\begin{center}
\begin{tabular}{l}
|\input{childdoc.def}|\\
|\childdocmain{|\textit{main}|}|\\
\end{tabular}
\end{center}
%
If |\jobname| does not match the argument \textit{main} of |\childdocmain|,
it is assumed that |\jobname| points to the child file to be compiled.
When using |\childdocmain| with the main file specified as argument,
it suffices to start a child file
with just |\input{|\textit{main}|}|
without loading of the package and using |\childdocof|.
If instead all processing is done
with the appropriate \textsf{childdoc} directives,
the argument of \textit{main} of |\childdocmain| can be empty.

An alternative version of the command line processing described
in \secref{sec:commandline} using the detection mechanism reads:
%
\begin{center}
|... -jobname "|\textit{target}|" "|[\textit{flags}]%
[|\def\jobname{|\textit{dest}|}|]|\input{|\textit{main}|}"|
\end{center}

%%%%%%%%%%%%%%%%%%%%%%%%%%%%%%%%%%%%%%%%%%%%%%%%%%%%%%%%%%%%%%%%%%%%%%%%%%%%%%%%
\subsection{Manual Code}
\label{sec:manual}

In case one cannot be certain whether the definitions file |childdoc.def|
is installed on the target \TeX{} distribution
and one prefers not to ship it,
it is conceivable to paste a few relevant commands into the sources.

To that end, drop all statements |\input{childdoc.def}|
and perform the replacements as outlined below.
Instead of |\childdocmain{|\textit{main}|}| add the following code
to the top of the main file:
%
\begin{center}
\begin{tabular}{l}
|\||ifdefined\childdocname\endinput\||fi\newif\ifchilddoc|\\
|\edef\childdocname{\scantokens\expandafter{\jobname\noexpand}}|\\
|\def\childdocmain{|\textit{main}|}\||ifx\childdocmain\childdocname\||else|\\
|\childdoctrue\includeonly{\childdocname}\let\jobname\childdocmain\||fi|\\
\end{tabular}
\end{center}
%
Instead of |\childdocof{|\textit{main}|}| just include the main file
at the top of each child file:
%
\begin{center}
|\input{|\textit{main}|}|
\end{center}
%
A simple redirection |\childdocforward{|\textit{dest}|}| is achieved by:
%
\begin{center}
|\def\jobname{|\textit{dest}|}\input{\jobname}|
\end{center}
%
The redirection with prefix
|\childdocforwardprefix[|\textit{prefix}|]{|\textit{dest}|}|
is accomplished by:
%
\begin{center}
\begin{tabular}{l}
|{\edef\jobname{\scantokens\expandafter{\jobname\noexpand}}|\\
|\def\redirectjob |\textit{prefix}|#1~~~{\gdef\jobname{|\textit{dest}|#1}}|\\
|\expandafter\redirectjob\jobname~~~}\input{\jobname}|
\end{tabular}
\end{center}

In an alternative approach,
child documents can be compiled by a specific command line
without additional code or specific definitions:
%
\begin{center}
|... -jobname "|\textit{target}|" "|[\textit{flags}]%
|\includeonly{|\textit{dest}|}\input{|\textit{main}|}"|
\end{center}
%

%%%%%%%%%%%%%%%%%%%%%%%%%%%%%%%%%%%%%%%%%%%%%%%%%%%%%%%%%%%%%%%%%%%%%%%%%%%%%%%%
%%%%%%%%%%%%%%%%%%%%%%%%%%%%%%%%%%%%%%%%%%%%%%%%%%%%%%%%%%%%%%%%%%%%%%%%%%%%%%%%
\section{Information}

%%%%%%%%%%%%%%%%%%%%%%%%%%%%%%%%%%%%%%%%%%%%%%%%%%%%%%%%%%%%%%%%%%%%%%%%%%%%%%%%
\subsection{Copyright}

Copyright \copyright{} 2017--2018 Niklas Beisert

This work may be distributed and/or modified under the
conditions of the \LaTeX{} Project Public License, either version 1.3
of this license or (at your option) any later version.
The latest version of this license is in
  \url{http://www.latex-project.org/lppl.txt}
and version 1.3 or later is part of all distributions of \LaTeX{}
version 2005/12/01 or later.

This work has the LPPL maintenance status `maintained'.

The Current Maintainer of this work is Niklas Beisert.

This work consists of the files |README.txt|, |childdoc.ins| and |childdoc.dtx|
as well as the derived files |childdoc.def|, |cdocsamp.tex|
with |cdocsch1.tex|, |cdocsch2.tex|, |cdocspt3.tex|, |cdocspt4.tex|,
|cdocsdrf.tex|, |cdocsfn1.tex|, |cdocsfn2.tex|
as well as |childdoc.pdf|.

%%%%%%%%%%%%%%%%%%%%%%%%%%%%%%%%%%%%%%%%%%%%%%%%%%%%%%%%%%%%%%%%%%%%%%%%%%%%%%%%
\subsection{Files and Installation}

The package consists of the files:
%
\begin{center}
\begin{tabular}{ll}
    |README.txt|   & readme file \\
    |childdoc.ins| & installation file \\
    |childdoc.dtx| & source file \\
    |childdoc.def| & definition file \\
    |cdocsamp.tex| & sample main file \\
    |cdocsch1.tex| & sample include file \\
    |cdocsch2.tex| & sample include file \\
    |cdocspt3.tex| & sample part file \\
    |cdocspt4.tex| & sample part file \\
    |cdocsdrf.tex| & sample redirection file \\
    |cdocsfn1.tex| & sample redirection file \\
    |cdocsfn2.tex| & sample redirection file \\
    |childdoc.pdf| & manual
\end{tabular}
\end{center}
%
The distribution consists of the files
|README.txt|, |childdoc.ins| and |childdoc.dtx|.
%
\begin{itemize}
\item
Run (pdf)\LaTeX{} on |childdoc.dtx|
to compile the manual |childdoc.pdf| (this file).
\item
Run \LaTeX{} on |childdoc.ins| to create the definitions file |childdoc.def|
and the sample |cdocsamp.tex| with include files
|cdocsch1.tex|, |cdocsch2.tex|, |cdocspt3.tex|, |cdocspt4.tex|,
|cdocsdrf.tex|, |cdocsfn1.tex|, |cdocsfn2.tex|.
Then copy the file |childdoc.def| to an appropriate directory of your \LaTeX{}
distribution, e.g.\ \textit{texmf-root}|/tex/latex/childdoc|.
\end{itemize}

%%%%%%%%%%%%%%%%%%%%%%%%%%%%%%%%%%%%%%%%%%%%%%%%%%%%%%%%%%%%%%%%%%%%%%%%%%%%%%%%
\subsection{Related CTAN Packages}

There are several other packages which offer a similar functionality:
%
\begin{itemize}
\item
The packages
\href{http://ctan.org/pkg/docmute}{\textsf{docmute}},
\href{http://ctan.org/pkg/includex}{\textsf{includex}} and
\href{http://ctan.org/pkg/standalone}{\textsf{standalone}}
provide commands to include only the document body of
a child file thus allowing both files to be compiled individually.
\item
The packages \href{http://ctan.org/pkg/subdocs}{\textsf{subdocs}}
and \href{http://ctan.org/pkg/subfiles}{\textsf{subfiles}}
provide structures in which the main and child documents can be
encapsulated and allowing them to be compiled individually.
The inclusion mechanism is different from the conventional |\include|.
\item
The package \href{http://ctan.org/pkg/combine}{\textsf{combine}}
is an elaborate solution to combine several documents into one.
\end{itemize}
%
See also the CTAN topic \href{http://ctan.org/topic/subdocs}{\textsf{subdocs}}
for further related packages.
The present package differs from the above solutions in that
a document structure constructed with the conventional |\include| mechanism
just needs two extra commands at the top of every file
such that all constituent files can be compiled individually.

%%%%%%%%%%%%%%%%%%%%%%%%%%%%%%%%%%%%%%%%%%%%%%%%%%%%%%%%%%%%%%%%%%%%%%%%%%%%%%%%
%\subsection{Feature Suggestions}
%
%The following is a list of features which may be useful for future
%versions of this package:
%%
%\begin{itemize}
%\item
%\ldots
%\end{itemize}

%%%%%%%%%%%%%%%%%%%%%%%%%%%%%%%%%%%%%%%%%%%%%%%%%%%%%%%%%%%%%%%%%%%%%%%%%%%%%%%%
\subsection{Revision History}

%%%%%%%%%%%%%%%%%%%%%%%%%%%%%%%%%%%%%%%%
\paragraph{v2.0:} 2018/12/30

\begin{itemize}
\item
immediate forward processing
\item
added |\childdocby| mechanism
\item
manual restructured
\end{itemize}

%%%%%%%%%%%%%%%%%%%%%%%%%%%%%%%%%%%%%%%%
\paragraph{v1.6:} 2018/01/17

\begin{itemize}
\item
application for development of include files
\item
corrections to manual
\end{itemize}

%%%%%%%%%%%%%%%%%%%%%%%%%%%%%%%%%%%%%%%%
\paragraph{v1.5:} 2017/05/21

\begin{itemize}
\item
more complete structuring introduced
\item
|\childdocof| introduced
\item
|\childdoc| renamed to |\childdocmain|
\item
|\childredirect| renamed to |\childdocforward| and |\childdocforwardprefix|
and functionality expanded
\end{itemize}

%%%%%%%%%%%%%%%%%%%%%%%%%%%%%%%%%%%%%%%%
\paragraph{v1.0:} 2017/04/27

\begin{itemize}
\item
manual and install package
\item
first version published on CTAN
\end{itemize}

%%%%%%%%%%%%%%%%%%%%%%%%%%%%%%%%%%%%%%%%
\paragraph{v0.6:} 2017/04/26

\begin{itemize}
\item
redirection mechanism added
\end{itemize}

%%%%%%%%%%%%%%%%%%%%%%%%%%%%%%%%%%%%%%%%
\paragraph{v0.5:} 2017/04/26

\begin{itemize}
\item
functionality in definition file
\end{itemize}


%%%%%%%%%%%%%%%%%%%%%%%%%%%%%%%%%%%%%%%%%%%%%%%%%%%%%%%%%%%%%%%%%%%%%%%%%%%%%%%%
%%%%%%%%%%%%%%%%%%%%%%%%%%%%%%%%%%%%%%%%%%%%%%%%%%%%%%%%%%%%%%%%%%%%%%%%%%%%%%%%
%%%%%%%%%%%%%%%%%%%%%%%%%%%%%%%%%%%%%%%%%%%%%%%%%%%%%%%%%%%%%%%%%%%%%%%%%%%%%%%%
\appendix

\settowidth\MacroIndent{\rmfamily\scriptsize 000\ }

 \DocInput{childdoc.dtx}

\end{document}
%</driver>
% \fi
%
% %%%%%%%%%%%%%%%%%%%%%%%%%%%%%%%%%%%%%%%%%%%%%%%%%%%%%%%%%%%%%%%%%%%%%%%%%%%%%%
% %%%%%%%%%%%%%%%%%%%%%%%%%%%%%%%%%%%%%%%%%%%%%%%%%%%%%%%%%%%%%%%%%%%%%%%%%%%%%%
% \section{Sample}
%\iffalse
%<*samplemain>
%\fi
%
% The following presents a sample document
% with two chapters, two parts, a title page,
% a compile flag as well as three forwarding files to set the flag.
% It consists of eight |.tex| files:
% \begin{center}
% \begin{tabular}{ll}
% |cdocsamp.tex|&main file\\
% |cdocsch1.tex|&include file for chapter 1\\
% |cdocsch2.tex|&include file for chapter 2\\
% |cdocspt3.tex|&include file for part 3\\
% |cdocspt4.tex|&include file for part 4\\
% |cdocsdrf.tex|&forwarding file for main file in draft mode\\
% |cdocsfi1.tex|&forwarding file for final version of chapter 1\\
% |cdocsfi2.tex|&forwarding file for final version of chapter 2\\
% \end{tabular}
% \end{center}
% Each of the eight files can be compiled directly by the \LaTeX{} compiler.
%
% %%%%%%%%%%%%%%%%%%%%%%%%%%%%%%%%%%%%%%
% \paragraph{Main File.}
%
% The main file is called |cdocsamp.tex|.
%
% Load the \textsf{childdoc} definitions and
% declare the filename for the main document:
%    \begin{macrocode}
\input{childdoc.def}
\childdocmain{}
%    \end{macrocode}

% Optional override for |\version| flag:
%    \begin{macrocode}
%%\ifchilddoc\else\providecommand{\version}{draft}\fi
%    \end{macrocode}

% Define the default values for the |\version| flag
% (|final| for the main file and |draft| for childs):
%    \begin{macrocode}
\ifchilddoc
\providecommand{\version}{draft}
\else
\providecommand{\version}{final}
\fi
%    \end{macrocode}

% Load the standard document class:
%    \begin{macrocode}
\documentclass[12pt]{article}
%    \end{macrocode}

% Start the document body:
%    \begin{macrocode}
\begin{document}
%    \end{macrocode}

% Declare a title page.
% Print title, part of document being processed and version flag:
%    \begin{macrocode}
\addtocounter{page}{-1}
\begin{center}
{\LARGE\bfseries{}childdoc example\par}
\vspace{1cm}
\ifchilddoc
\ifchilddocmanual part\else chapter\fi:
`\childdocname' of `\childdocjob'\par
\else
main document: `\childdocjob'\par
\fi
version: \version\par
\end{center}
\newpage
%    \end{macrocode}

% Manually include selected file,
% otherwise process as usual:
%    \begin{macrocode}
\ifchilddocmanual
\section*{part `\childdocname'}
\input{\childdocname}
\else
%    \end{macrocode}

% Include the two chapters:
%    \begin{macrocode}
\include{cdocsch1}
\include{cdocsch2}
%    \end{macrocode}

% Include the two parts unless only chapters should be displayed:
%    \begin{macrocode}
\ifchilddoc\else
\section{part three}
\input{cdocspt3}
\section{part four}
\input{cdocspt4}
\fi
%    \end{macrocode}

% Process as usual until here:
%    \begin{macrocode}
\fi
%    \end{macrocode}

% End of document body:
%    \begin{macrocode}
\end{document}
%    \end{macrocode}
%\iffalse
%</samplemain>
%\fi
%
% %%%%%%%%%%%%%%%%%%%%%%%%%%%%%%%%%%%%%%
% \paragraph{Chapter Include Files.}
%
% The include files are called |cdocsch1.tex| and |cdocsch2.tex|.
%
%\iffalse
%<*samplechap1|samplechap2>
%\fi

% Optional override for |\version| flag:
%    \begin{macrocode}
%%\providecommand{\version}{final}
%    \end{macrocode}

% Include the main document:
%    \begin{macrocode}
\input{childdoc.def}
\childdocof{cdocsamp}
%    \end{macrocode}

%\iffalse
%</samplechap1|samplechap2>
%\fi
%
%\iffalse
%<*samplechap1>
%\fi
% Some text for chapter 1:
%    \begin{macrocode}
\section{one}
some text in chapter one
%    \end{macrocode}

%\iffalse
%</samplechap1>
%\fi
% Some text for chapter 2:
%\iffalse
%<*samplechap2>
%\fi
%    \begin{macrocode}
\section{two}
more text in chapter two
%    \end{macrocode}

%\iffalse
%</samplechap2>
%\fi
%
% %%%%%%%%%%%%%%%%%%%%%%%%%%%%%%%%%%%%%%
% \paragraph{Part Include Files.}
%
% The include files are called |cdocspt3.tex| and |cdocspt4.tex|.
%
%\iffalse
%<*samplepart3|samplepart4>
%\fi

% Optional override for |\version| flag:
%    \begin{macrocode}
%%\providecommand{\version}{final}
%    \end{macrocode}

% Include the main document:
%    \begin{macrocode}
\input{childdoc.def}
\childdocby{cdocsamp}
%    \end{macrocode}

%\iffalse
%</samplepart3|samplepart4>
%\fi
%
%\iffalse
%<*samplepart3>
%\fi
% Some text for part 3:
%    \begin{macrocode}
some text in part three
%    \end{macrocode}

%\iffalse
%</samplepart3>
%\fi
% Some text for part 4:
%\iffalse
%<*samplepart4>
%\fi
%    \begin{macrocode}
more text in part four
%    \end{macrocode}

%\iffalse
%</samplepart4>
%\fi
%
% %%%%%%%%%%%%%%%%%%%%%%%%%%%%%%%%%%%%%%
% \paragraph{Forwarding for a Complete Draft.}
%
% The following forwarding file |cdocsdrf.tex|
% compiles the main document in draft mode:
%\iffalse
%<*sampledraft>
%\fi
%    \begin{macrocode}
\def\version{draft}
\input{childdoc.def}
\childdocforward{cdocsamp}
%    \end{macrocode}

%\iffalse
%</sampledraft>
%\fi
%
% %%%%%%%%%%%%%%%%%%%%%%%%%%%%%%%%%%%%%%
% \paragraph{Forwarding for Final Version of the Chapters.}
%
% The following forwarding files |cdocsfn1.tex| and |cdocsfn2.tex|
% (with identical content)
% compile the final versions of the child documents
% |cdocsch1.tex| and |cdocsch2.tex|, respectively:
%\iffalse
%<*samplefinal>
%\fi
%    \begin{macrocode}
\def\version{final}
\input{childdoc.def}
\childdocforwardprefix[cdocsamp]{cdocsfn}{cdocsch}
%    \end{macrocode}

%\iffalse
%</samplefinal>
%\fi
%
% %%%%%%%%%%%%%%%%%%%%%%%%%%%%%%%%%%%%%%
% \paragraph{Command Line Processing.}
%
% The following three command lines generate the output files
% |cdocscld|, |cdocscl1| and |cdocscl2|
% which should be identical to
% |cdocsdrf|, |cdocsch1| and |cdocsfn2|, respectively:
% \begin{center}
% \begin{tabular}{l}
% |latex -jobname cdocscld \|\\
% |  "\def\version{draft}\input{childdoc.def}\childdocforward{cdocsamp}"|\\
% |latex -jobname cdocscl1 \|\\
% |  "\input{childdoc.def}\childdocforward[cdocsamp]{cdocsch1}"|\\
% |latex -jobname cdocscl2 \|\\
% |  "\def\version{final}\input{childdoc.def}\childdocforward{cdocsch2}"|
% \end{tabular}
% \end{center}
% Note that the trailing backslash on each first line
% merely continues the input to the second line
% (for convenient cut ant paste).
% Furthermore, the command |latex| can be replaced by any
% of its alternative versions such as |pdflatex|.
%
% %%%%%%%%%%%%%%%%%%%%%%%%%%%%%%%%%%%%%%%%%%%%%%%%%%%%%%%%%%%%%%%%%%%%%%%%%%%%%%
% %%%%%%%%%%%%%%%%%%%%%%%%%%%%%%%%%%%%%%%%%%%%%%%%%%%%%%%%%%%%%%%%%%%%%%%%%%%%%%
% \section{Implementation}
%\iffalse
%<*package>
%\fi
%
% This section describes the definitions file |childdoc.def|.

% The definitions cannot be loaded using |\usepackage| or |\RequirePackage|
% which has a mechanism to prevent loading a style file more than once.
% When loading the definitions by means of |\input|
% multiple instances have to be prevented manually:
%\iffalse
%This code needs to be before the `\ProvidesFile' directive
%which is defined at the beginning of this file.
%Therefore it is also placed there and commented out here.
%</package>
%<*discard>
%\fi
%    \begin{macrocode}
\ifdefined\childdocmain\endinput\fi
%    \end{macrocode}
%\iffalse
%</discard>
%<*package>
%\fi
%
% \macro{\ifchilddoc}
% \macro{\ifchilddocmanual}
% The conditional |\ifchilddoc| tells whether a
% child (true) or main (false) document is being compiled.
% The conditional |\ifchilddocmanual| tells whether
% the |\includeonly| mechanism is used (false) or
% the selection of child files must be performed manually (true).
% The definitions initialise to false:
%    \begin{macrocode}
\newif\ifchilddoc
\newif\ifchilddocmanual
%    \end{macrocode}

% \macro{\childdocname}
% \macro{\childdocjob}
% The macro |\childdocname| stores the name of the main document
% to be compiled. The macro |\childdocjob| stores the name of
% the document on which the \LaTeX{} compiler was originally invoked.
% The content of |\jobname| cannot be compared
% to filenames specified in the source due to different catcodes.
% The following code rescans |\jobname|, stores the result
% in |\childdocname| and saves a copy in |\childdocjob|:
%    \begin{macrocode}
\edef\childdocname{\scantokens\expandafter{\jobname\noexpand}}
\let\childdocjob\childdocname
%    \end{macrocode}

% \macro{\childdocdisable}
% The macro |\childdocdisable| prevents the main file
% from being processed more than once.
% At this stage, the main document command |\childdocmain|
% is assumed to be called once again where it should do nothing.
% Any subsequent call to it should prevent
% a secondary processing of the main document
% It overwrites the forwarding commands
% |\childdocof| and |\childdocforward|
% with empty macros to prevent further inclusions of the main document:
%    \begin{macrocode}
\newcommand{\childdocdisable}
{
  \renewcommand{\childdocmain}[1]{\renewcommand{\childdocmain}[1]{\endinput}}
  \renewcommand{\childdocof}[1]{}
  \renewcommand{\childdocby}[2][]{}
  \renewcommand{\childdocforward}[2][]{}
  \renewcommand{\childdocdisable}{}
}
%    \end{macrocode}

% \macro{\childdocmain}
% The macro |\childdocmain| is to be called at the top of the main file
% with nothing or the main filename (without extension) as argument.
% First, it breaks loops.
% If the argument is not empty and does not match |\childdocname|
% (which is set by the first inclusion of |childdoc.def|),
% |\ifchilddoc| is set to true, |\includeonly| is applied to the child file
% and |\jobname| is set to the main file
% (for proper handling of |.aux| files):
%    \begin{macrocode}
\newcommand{\childdocmain}[1]
{
  \childdocdisable\childdocmain{}
  \if?#1?\else
    \begingroup
      \def\childdoctmp{#1}
      \ifx\childdoctmp\childdocname
        \def\childdoctmp{}
      \else
        \def\childdoctmp
        {
          \childdoctrue
          \includeonly{\childdocname}
          \def\childdocjob{#1}
          \def\jobname{#1}
        }
      \fi
      \expandafter
    \endgroup
    \childdoctmp
  \fi
}
%    \end{macrocode}

% \macro{\childdocof}
% The command |\childdocof| redirects
% compilation to the main file |#1|.
%    \begin{macrocode}
\newcommand{\childdocof}[1]
{
  \childdocdisable
  \childdoctrue
  \includeonly{\childdocname}
  \def\jobname{#1}
  \def\childdocjob{#1}
  \input{#1}
}
%    \end{macrocode}

% \macro{\childdocby}
% The command |\childdocby| ....
%    \begin{macrocode}
\newcommand{\childdocby}[2][]
{
  \childdocdisable
  \childdoctrue
  \childdocmanualtrue
  \if?#1?\else
    \def\jobname{#2}
  \fi
  \def\childdocjob{#2}
  \input{#2}
  \endinput
}
%    \end{macrocode}

% \macro{\childdocforward}
% The command |\childdocforward| redirects
% compilation to the main file or
% (if the optional argument is given) a child file.
% Parameters are set as if the main file
% or a child file starting with |\childdocof| was compiled.
% Then compilation is handed over to the main file:
%    \begin{macrocode}
\newcommand{\childdocforward}[2][]
{
  \begingroup
    \if?#1?
      \def\childdoctmp
      {
        \def\childdocname{#2}
        \def\childdocjob{#2}
        \def\jobname{#2}
        \input{#2}
        \endinput
      }
    \else
      \def\childdoctmp
      {
        \childdocdisable
        \def\childdocname{#2}
        \childdoctrue
        \includeonly{#2}
        \def\childdocjob{#1}
        \def\jobname{#1}
        \input{#1}
        \endinput
      }
    \fi
    \expandafter
  \endgroup
  \childdoctmp
}
%    \end{macrocode}

% \macro{\childdocforwardprefix}
% The command |\childdocforwardprefix| redirects
% compilation to the main or a child file by means of a pattern.
% The prefix |#1| in the current filename is replaced by |#2|
% and the suffix of the current filename is kept
% (it is assumed that the filename does not contain the substring `|~~~|'
% which is used as a delimiter).
% Compilation is handed over to the new file by |\childdocforward|:
%    \begin{macrocode}
\newcommand{\childdocforwardprefix}[3][]
{
  \begingroup
    \def\childdocextract #2##1~~~{\def\childdoctmp{\childdocforward[#1]{#3##1}}}
    \expandafter\childdocextract\childdocname~~~
    \expandafter
  \endgroup
  \childdoctmp
}
%    \end{macrocode}

% \macro{\childdoc}
% The deprecated macro |\childdoc| is a legacy version of |\childdocmain|:
%    \begin{macrocode}
\newcommand{\childdoc}{\childdocmain}
%    \end{macrocode}

% \macro{\childdocredirect}
% The deprecated macro |\childdocredirect| is a legacy version
% of |\childdocforward| and |\childdocforwardprefix|:
%    \begin{macrocode}
\newcommand{\childdocredirect}[2][]
{
  \begingroup
    \if?#1?
      \def\childdoctmp{\childdocforward{#2}}
    \else
      \def\childdoctmp{\childdocforwardprefix{#1}{#2}}
    \fi
    \expandafter
  \endgroup
  \childdoctmp
}
%    \end{macrocode}

%\iffalse
%</package>
%\fi
%
\endinput

\childdocforward{cdocsamp}
%    \end{macrocode}

%\iffalse
%</sampledraft>
%\fi
%
% %%%%%%%%%%%%%%%%%%%%%%%%%%%%%%%%%%%%%%
% \paragraph{Forwarding for Final Version of the Chapters.}
%
% The following forwarding files |cdocsfn1.tex| and |cdocsfn2.tex|
% (with identical content)
% compile the final versions of the child documents
% |cdocsch1.tex| and |cdocsch2.tex|, respectively:
%\iffalse
%<*samplefinal>
%\fi
%    \begin{macrocode}
\def\version{final}
% \iffalse
%
% childdoc.dtx Copyright (C) 2017-2018 Niklas Beisert
%
% This work may be distributed and/or modified under the
% conditions of the LaTeX Project Public License, either version 1.3
% of this license or (at your option) any later version.
% The latest version of this license is in
%   http://www.latex-project.org/lppl.txt
% and version 1.3 or later is part of all distributions of LaTeX
% version 2005/12/01 or later.
%
% This work has the LPPL maintenance status `maintained'.
%
% The Current Maintainer of this work is Niklas Beisert.
%
% This work consists of the files childdoc.dtx and childdoc.ins
% and the derived files childdoc.def and cdocsamp.tex with
% cdocsch1.tex, cdocsch2.tex, cdocsdrf.tex, cdocsfn1.tex, cdocsfn2.tex.
%
%<package>\ifdefined\childdocmain\endinput\fi
%<package>\ProvidesFile{childdoc.def}[2018/12/30 v2.0 child document driver]
%<samplemain>\ProvidesFile{cdocsamp.tex}[2018/12/30 v2.0 sample for childdoc]
%<*driver>
%\ProvidesFile{childdoc.drv}[2018/12/30 v2.0 childdoc reference manual file]
\PassOptionsToClass{10pt,a4paper}{article}
\documentclass{ltxdoc}

\usepackage[margin=35mm]{geometry}
\usepackage{hyperref}
\usepackage{hyperxmp}
\usepackage[usenames]{color}

\hypersetup{colorlinks=true}
\hypersetup{pdfstartview=FitH}
\hypersetup{pdfpagemode=UseNone}
\hypersetup{pdfsource={}}
\hypersetup{pdflang={en-UK}}
\hypersetup{pdfcopyright={Copyright 2017-2018 Niklas Beisert.
  This work may be distributed and/or modified under the
  conditions of the LaTeX Project Public License, either version 1.3
  of this license or (at your option) any later version.}}
\hypersetup{pdflicenseurl={http://www.latex-project.org/lppl.txt}}
\hypersetup{pdfcontactaddress={ETH Zurich, ITP, HIT K,
  Wolfgang-Pauli-Strasse 27}}
\hypersetup{pdfcontactpostcode={8093}}
\hypersetup{pdfcontactcity={Zurich}}
\hypersetup{pdfcontactcountry={Switzerland}}
\hypersetup{pdfcontactemail={nbeisert@itp.phys.ethz.ch}}
\hypersetup{pdfcontacturl={http://people.phys.ethz.ch/\xmptilde nbeisert/}}

\newcommand{\secref}[1]{\hyperref[#1]{section \ref*{#1}}}

\parskip1ex
\parindent0pt
\let\olditemize\itemize
\def\itemize{\olditemize\parskip0pt}

\begin{document}

\title{The \textsf{childdoc} Package}
\hypersetup{pdftitle={The childdoc Package}}
\author{Niklas Beisert\\[2ex]
  Institut f\"ur Theoretische Physik\\
  Eidgen\"ossische Technische Hochschule Z\"urich\\
  Wolfgang-Pauli-Strasse 27, 8093 Z\"urich, Switzerland\\[1ex]
  \href{mailto:nbeisert@itp.phys.ethz.ch}
  {\texttt{nbeisert@itp.phys.ethz.ch}}}
\hypersetup{pdfauthor={Niklas Beisert}}
\hypersetup{pdfsubject={Manual for the LaTeX2e Package childdoc}}
\date{30 December 2018, \textsf{v2.0}}
\maketitle

\begin{abstract}\noindent
\textsf{childdoc} is a \LaTeXe{} package
that enables the direct compilation
of document sections included by |\include|
to individual files.
\end{abstract}

\begingroup
\parskip0ex
\tableofcontents
\endgroup

%%%%%%%%%%%%%%%%%%%%%%%%%%%%%%%%%%%%%%%%%%%%%%%%%%%%%%%%%%%%%%%%%%%%%%%%%%%%%%%%
%%%%%%%%%%%%%%%%%%%%%%%%%%%%%%%%%%%%%%%%%%%%%%%%%%%%%%%%%%%%%%%%%%%%%%%%%%%%%%%%
\section{Introduction}

\LaTeX{} provides a mechanism to structure a large document (such as a book)
into a main file and several child files (containing the chapters)
using the |\include| command.
This mechanism is beneficial for documents
which span hundreds of pages in order to
make the source file(s) more manageable.
Moreover, compilation can be restricted to
selected child files by means of the |\includeonly| command.
The latter feature can be used to reduce the compilation time while editing
(this was significantly more useful in the earlier days of \LaTeX{})
or to generate a smaller document which is easier to navigate.
Another application of |\includeonly| is to generate
documents consisting of selected parts of the complete document.

However, there are a few drawbacks of the plain |\include| mechanism:
\begin{itemize}
\item
The child files cannot be compiled on their own,
they can only be compiled via the main file.
A naive editing environment
(such as a text editor with an option
to have the current file processed by \LaTeX)
may require one to switch to the main file before compiling;
attempting to compile the child file produces errors.
\item
The main file must be modified (each time)
to adjust the |\includeonly| command
to the present needs. This easily leaves the main file in a messy state.
\item
The generated document will always carry the filename
of the main document. This is inconvenient if
several child files are to be compiled and
to be kept for distribution.
\end{itemize}

The present package provides a simple interface
to make child files individually compilable by \LaTeX{}.
Compiling a child file then has the same effect as compiling
the main file with an |\includeonly| command
to select the appropriate child.
Moreover the generated document will carry the name of the child
rather than the main file.
This resolves all three above issues.

This feature is meant to make the editing of books,
thesis documents and lecture notes somewhat more convenient.
However, the package can also be used efficiently for
composing a series of documents (such as exercise sheets)
which are typically distributed individually.
It then assists the author in generating the individual documents
(potentially in different versions)
as well as a document containing the collected series.
Another application is in developing style files
or other kinds of included material
where compilation of the style file could redirect
to a sample or test file.

%%%%%%%%%%%%%%%%%%%%%%%%%%%%%%%%%%%%%%%%%%%%%%%%%%%%%%%%%%%%%%%%%%%%%%%%%%%%%%%%
%%%%%%%%%%%%%%%%%%%%%%%%%%%%%%%%%%%%%%%%%%%%%%%%%%%%%%%%%%%%%%%%%%%%%%%%%%%%%%%%
\section{Usage}

First of all, the package \textsf{childdoc} is \emph{not} a standard
\LaTeXe{} |.sty| style file! Therefore it needs to be invoked in
a non-standard way.

%%%%%%%%%%%%%%%%%%%%%%%%%%%%%%%%%%%%%%%%%%%%%%%%%%%%%%%%%%%%%%%%%%%%%%%%%%%%%%%%
\subsection{Included Files}
\label{sec:include}

%%%%%%%%%%%%%%%%%%%%%%%%%%%%%%%%%%%%%%%%
\DescribeMacro{\childdocmain}
To use the package, add the commands
\begin{center}
\begin{tabular}{l}
|\input{childdoc.def}|\\
|\childdocmain{}|\\
\end{tabular}
\end{center}
at the very top of the main \LaTeX{} file,
in particular \emph{before} the |\documentclass| statement!
The argument of |\childdocmain| should be left empty
(but it must be present).

%%%%%%%%%%%%%%%%%%%%%%%%%%%%%%%%%%%%%%%%
\DescribeMacro{\childdocof}
Furthermore, add the commands
\begin{center}
\begin{tabular}{l}
|\input{childdoc.def}|\\
|\childdocof{|\textit{main}|}|\\
\end{tabular}
\end{center}
at the top of every child file \textit{child}
which is included by |\include{|\textit{child}|}|
from within the main file
(or at least for those files to be compiled individually).
The argument \textit{main} must be the filename of the main file.

There are a couple of
considerations in setting up the main and child documents:

%%%%%%%%%%%%%%%%%%%%%%%%%%%%%%%%%%%%%%%%
\paragraph{Restrictions.}

Please note the following restrictions:
\begin{itemize}
\item
|\childdocmain| must be called with one argument \textit{main}
to ensure compatibility with earlier version of the package.
It must either be empty (|\childdocmain{}|)
or precisely match the filename of the main file in which it is specified.
See \secref{sec:detection} for further information.
\item
The filename \textit{main} must be specified without the |.tex| extension.
\item
The filename \textit{main} is case sensitive
(even in case-insensitive file systems)
due to internal string comparison.
\item
The argument \textit{main} should be fully expanded, it cannot be a macro.
\item
Subdirectories and special characters should be avoided in filenames.
\item
The command |\childdocmain{|\textit{main}|}| must be followed by a whitespace.
It should not be followed immediately by another command
or by a comment mark `|%|'.
This is because the \TeX{} parser reads the token immediately following
the argument of |\childdocmain| and puts it
at the beginning of every child section;
however, a white\-space is ignored.
\end{itemize}

%%%%%%%%%%%%%%%%%%%%%%%%%%%%%%%%%%%%%%%%
\paragraph{Content of Main File.}

It is advisable to place all content in the child files included by |\include|.
Any output contained in the main file will appear in all child documents
unless suppressed manually;
it cannot be suppressed automatically by the |\includeonly| directive
and thus should normally be avoided.
A method to include some content in the main file
by means of conditional processing is described in \secref{sec:conditional}.

%%%%%%%%%%%%%%%%%%%%%%%%%%%%%%%%%%%%%%%%
\paragraph{Page Numbering.}

When only a part of the document is compiled,
the appropriate numbering of pages
(as well as other status parameters)
is determined from the |.aux| files.
The latter contain information from previous passes.
However this information needs to propagate through
all intermediate child documents.
Therefore the page numbering in child documents may well
be inconsistent until the complete document is compiled at least once.

A useful (if unconventional) way to always ensure a consistent
page numbering is to restart the numbering in each child document
and denote the pages by `\textit{child}|.|\textit{page}'
where \textit{child} represents the chapter/section number of the child file.
This can be achieved by the command
|\numberwithin{page}{|\textit{child}|}|
of the \textsf{amsmath} package
where \textit{child} can be |chapter| or |section|
depending on the chosen structuring.
Alternatively, one can modify the macro |\thepage| appropriately
and reset the counter |page| at the start of each child file.

%%%%%%%%%%%%%%%%%%%%%%%%%%%%%%%%%%%%%%%%%%%%%%%%%%%%%%%%%%%%%%%%%%%%%%%%%%%%%%%%
\subsection{Conditional Processing}
\label{sec:conditional}

The package provides a mechanism to compile different versions
of a document. To customise the versions further some conditional processing
can come in handy to distinguish which version is being compiled.
The package provides two macros to describe the compilation context:

%%%%%%%%%%%%%%%%%%%%%%%%%%%%%%%%%%%%%%%%
\DescribeMacro{\ifchilddoc}
The conditional |\ifchilddoc| distinguishes between the compilation of
child documents and the main document:
%
\begin{center}
|\ifchilddoc |\textit{child-code}| |[|\||else |\textit{main-code}]| \||fi|
\end{center}

%%%%%%%%%%%%%%%%%%%%%%%%%%%%%%%%%%%%%%%%
\DescribeMacro{\childdocname}
\DescribeMacro{\childdocjob}
The macro |\childdocname| contains the filename (without extension)
of the main or child file being processed.
Note that |\childdocjob| will always contain the name of the main file.

%%%%%%%%%%%%%%%%%%%%%%%%%%%%%%%%%%%%%%%%
\paragraph{Title Page.}

Conditional processing can be used to include a title or banner page
in the main document when proper precautions are taken.
Importantly, the code in the main file should ensure that the page counter
(as well as other status parameters which are stored in the |.aux| files)
takes the same value after the conditional processing.
Otherwise the page numbers may take divergent values
depending on which part is compiled.

For example, a title page could be declared by:
%
\begin{center}
\begin{tabular}{l}
|\ifchilddoc\||else|\\
|\addtocounter{page}{-1}|\\
\textit{code for title page}\\
|\newpage|\\
|\||fi|
\end{tabular}
\end{center}
%
A banner page for the child documents can be generated by:
%
\begin{center}
\begin{tabular}{l}
|\ifchilddoc|\\
|\addtocounter{page}{-1}|\\
\textit{code for banner page}\\
|\newpage|\\
|\||fi|
\end{tabular}
\end{center}
%
Here one could write a message such as:
\begin{center}
|This is the part \childdocname{} of \childdocjob{}.|
\end{center}

%%%%%%%%%%%%%%%%%%%%%%%%%%%%%%%%%%%%%%%%%%%%%%%%%%%%%%%%%%%%%%%%%%%%%%%%%%%%%%%%
\subsection{Flags}
\label{sec:flags}

The package makes it easy to generate different versions
of the main or child documents.
To this end compilation flags can be defined
and assigned different default values.
They will be particularly useful in conjunction
with the forwarding mechanism described in \secref{sec:forward}.

For example, it may be useful to have a flag |\version|
which can be set to |draft| or |final|.
The document source will contain some conditional code
depending on the value of |\version|.
Suppose further, the flag should default to |final| for the main file
and to |draft| for child files
which is a natural assignment for editing the document.
This is achieved by placing the following code
in the preamble of the main document
(below the |\childdocmain| directive):
%
\begin{center}
\begin{tabular}{l}
|\ifchilddoc|\\
|\providecommand{\version}{draft}|\\
|\||else|\\
|\providecommand{\version}{final}|\\
|\||fi|
\end{tabular}
\end{center}
%
The definition by |\providecommand| makes sure
that previous definitions are not overwritten.
Further statements |\providecommand{\version}{...}|
can thus be added before the above code to override it.

For the main file, one might add a line
(between |\childdocmain| and the above block)
%
\begin{center}
|%\ifchilddoc\||else\providecommand{\version}{draft}\||fi|
\end{center}
%
which can be uncommented to produce a draft version.
Likewise one can add a line to the very top of a child file
(above the |\childdocof{|\textit{main}|}| directive)
%
\begin{center}
|%\providecommand{\version}{final}|
\end{center}
%
which can be uncommented to produce the final version of this child document.

%%%%%%%%%%%%%%%%%%%%%%%%%%%%%%%%%%%%%%%%%%%%%%%%%%%%%%%%%%%%%%%%%%%%%%%%%%%%%%%%
\subsection{Forwarding}
\label{sec:forward}

Different versions of the main or child documents
using compilation flags as described in \secref{sec:flags}
can be (permanently) stored in different files
for convenient compilation, viewing and distribution.
To this end, the package defines a command
to pass on compilation to a different file:

%%%%%%%%%%%%%%%%%%%%%%%%%%%%%%%%%%%%%%%%
\DescribeMacro{\childdocforward}
The command |\childdocforward| redirects processing to
another source file:
%
\begin{center}
\begin{tabular}{l}
|\input{childdoc.def}|\\
|\childdocforward[|\textit{main}|]{|\textit{dest}|}|\\
\end{tabular}
\end{center}
%
The argument \textit{dest} is the destination file
(without extension).
It should be the main file or one of the child files.
Note that further \textsf{childdoc} directives
such as |\childdocof| and |\childdocforward|
in the indicated file will be processed in this form.
The optional argument \textit{main}
passes on directly to the main file \textit{main}
while pretending to compile the child \textit{dest}.
This form behaves as if \textit{dest}
issues |\childdocof{|\textit{main}|}| right away,
and no further \textsf{childdoc} directives will be processed.

%%%%%%%%%%%%%%%%%%%%%%%%%%%%%%%%%%%%%%%%
\DescribeMacro{\...prefix}
In the alternative form |\childdocforwardprefix|,
%
\begin{center}
\begin{tabular}{l}
|\input{childdoc.def}|\\
|\childdocforwardprefix[|\textit{main}|]{|\textit{prefix}|}{|\textit{dest}|}|
\end{tabular}
\end{center}
%
the destination file is determined by a pattern
depending on the current file:
To make this work, the current file must be called
`{\textit{prefix}\hspace{0.2em}\textit{suffix}}'
with \textit{prefix} matching precisely the argument.
Processing is then passed on to the file
`{\textit{dest}\hspace{0.2em}\textit{suffix}}'.
Surely, the same effect is achieved by
directly specifying the
argument `{\textit{dest}\hspace{0.2em}\textit{suffix}}'
in the first form.
However, that requires to set up a different file
for each child. With the alternative form of the command
all these files can have exactly the same content
which simplifies setting them up and maintaining them.

For example, the following file |draft.tex|
with a compilation flag |\version| as described in \secref{sec:flags}
compiles the main document as a draft:
%
\begin{center}
\begin{tabular}{l}
|\def\version{draft}|\\
|\input{childdoc.def}|\\
|\childdocforward{|\textit{main}|}|
\end{tabular}
\end{center}
%
Likewise, the following files |final|\textit{nn}|.tex|
compile the final version of the child document
|child|\textit{nn}|.tex|:
%
\begin{center}
\begin{tabular}{l}
|\def\version{final}|\\
|\input{childdoc.def}|\\
|\childdocforwardprefix{final}{child}|
\end{tabular}
\end{center}
%

Note that when several versions of a main file and/or of each child file
are to be generated, it may be convenient to set up a |Makefile| or
shell script to automatise the process.

%%%%%%%%%%%%%%%%%%%%%%%%%%%%%%%%%%%%%%%%%%%%%%%%%%%%%%%%%%%%%%%%%%%%%%%%%%%%%%%%
\subsection{Command Line Processing}
\label{sec:commandline}

The effect of redirection files can also be achieved by invoking
the \LaTeX{} compiler with a more elaborate command line.
Most conveniently this should be done as part
of a shell script or a |Makefile|.

When using \textsf{childdoc} in the main file, the following
command lines effectively perform a redirection
(note that depending on the shell being used,
backslashes may have to be doubled: `|\|' $\to$ `|\\|'):
%
\begin{center}
|... -jobname "|\textit{target}|" |\\|"|[\textit{flags}]%
|\input{childdoc.def}\childdocforward[|\textit{main}|]{|\textit{dest}|}"|
\end{center}
%
Here \textit{target} is the name of the output file,
\textit{main} is the name of the main file
and \textit{dest} is the name of the main or child file to be processed
(all filenames without extensions).
The optional argument \textit{main} can be omitted
if \textit{main} matches \textit{dest}.
Optionally, compilation \textit{flags} can be defined via |\def| commands.
This command line makes the \TeX{} engine believe
it is compiling the file \textit{target}
whose content is specified as the latter parameter.
The provided code then forwards the processing to
\textit{main} or \textit{dest} as described in \secref{sec:forward}.

%%%%%%%%%%%%%%%%%%%%%%%%%%%%%%%%%%%%%%%%%%%%%%%%%%%%%%%%%%%%%%%%%%%%%%%%%%%%%%%%
\subsection{Include by Input}
\label{sec:input}

Including child documents by |\include| has some restrictions by design.
Most notably, the content of a child document always occupies
its own set of pages; pages cannot be shared between child documents.
Usually, this behaviour makes perfect sense
because each child document contain an essential part of the document.
However, in some situations it may be desirable to compose
a document from a collection of parts
without having mandatory page breaks between then.
For this case, the package
provides a mechanism to include parts
by |\input| which can also be processed individually.
However, by construction this mechanism
requires manual handling of the content to be output.

%%%%%%%%%%%%%%%%%%%%%%%%%%%%%%%%%%%%%%%%
\DescribeMacro{\ifchilddocmanual}
The main file should be prepared as usual, see \secref{sec:include}.
However, the document body must make a distinction
between processing of an individual part and of the main document, e.g.:
%
\begin{center}
\begin{tabular}{l}
|\ifchilddocmanual|\\
|\input{\childdocname}|\\
|\||else|\\
\textit{document body with }|\input{|\textit{part}|}|\\
|\||fi|
\end{tabular}
\end{center}
%
The conditional |\ifchilddocmanual| is true whenever
a part to be included by |\input| is being compiled,
and the name of the part is stored in |\childdocname|.

%%%%%%%%%%%%%%%%%%%%%%%%%%%%%%%%%%%%%%%%
\DescribeMacro{\childdocby}
Each part to be included by |\input| should start with:
%
\begin{center}
\begin{tabular}{l}
|\input{childdoc.def}|\\
|\childdocby{|\textit{main}|}|\\
\end{tabular}
\end{center}
%
The directive |\childdocby| is similar to |\childdocof|
described in \secref{sec:include},
but the subsequent selection of content must be done manually.
To that end, both |\ifchilddoc| and |\ifchilddocmanual|
will be true upon processing of a part,
and the name of the part is stored in |\childdocname|.
Note that |\jobname| will be set to the filename of the current part
so that each part receives an individual |.aux| file
that does not interfere with the |.aux| file(s) of the main document.
This behaviour can be altered by the alternative form
|\childdocby[*]{|\textit{main}|}| (with a non-empty optional argument)
which uses the |.aux| file of the main document
by setting |\jobname| to \textit{main}.

%%%%%%%%%%%%%%%%%%%%%%%%%%%%%%%%%%%%%%%%%%%%%%%%%%%%%%%%%%%%%%%%%%%%%%%%%%%%%%%%
\subsection{Driver Development}
\label{sec:driver}

The \textsf{childdoc} mechanism can also be use for the development
of definition files such as \LaTeX{} styles or classes.
This case differs from the above setup with multiple parts
included by |\include| in that no |\includeonly| should be invoked.
This can be achieved by starting the include file
(before |\ProvidesPackage|) with:
%
\begin{center}
\begin{tabular}{l}
|\input{childdoc.def}|\\
|\childdocforward{|\textit{main}|}|\\
\end{tabular}
\end{center}
%
or alternatively with:
%
\begin{center}
\begin{tabular}{l}
|\input{childdoc.def}|\\
|\childdocby{|\textit{main}|}|\\
\end{tabular}
\end{center}
%
Both forms have slightly different effects as described above.
The main file is prepared as usual, see \secref{sec:include}.

%%%%%%%%%%%%%%%%%%%%%%%%%%%%%%%%%%%%%%%%%%%%%%%%%%%%%%%%%%%%%%%%%%%%%%%%%%%%%%%%
\subsection{Legacy Detection}
\label{sec:detection}

The directive |\childdocmain| in the main file can detect
whether the complete document or merely a child is to be compiled
even without using the directive |\childdocof|.
This method is deprecated because it is less robust
and there is no compelling reason to use it;
it is merely provided for backward compatibility
and it may be removed in future versions.

If the detection mechanism is to be used,
it is mandatory to correctly specify
the filename of the main file as the argument of |\childdocmain|:
%
\begin{center}
\begin{tabular}{l}
|\input{childdoc.def}|\\
|\childdocmain{|\textit{main}|}|\\
\end{tabular}
\end{center}
%
If |\jobname| does not match the argument \textit{main} of |\childdocmain|,
it is assumed that |\jobname| points to the child file to be compiled.
When using |\childdocmain| with the main file specified as argument,
it suffices to start a child file
with just |\input{|\textit{main}|}|
without loading of the package and using |\childdocof|.
If instead all processing is done
with the appropriate \textsf{childdoc} directives,
the argument of \textit{main} of |\childdocmain| can be empty.

An alternative version of the command line processing described
in \secref{sec:commandline} using the detection mechanism reads:
%
\begin{center}
|... -jobname "|\textit{target}|" "|[\textit{flags}]%
[|\def\jobname{|\textit{dest}|}|]|\input{|\textit{main}|}"|
\end{center}

%%%%%%%%%%%%%%%%%%%%%%%%%%%%%%%%%%%%%%%%%%%%%%%%%%%%%%%%%%%%%%%%%%%%%%%%%%%%%%%%
\subsection{Manual Code}
\label{sec:manual}

In case one cannot be certain whether the definitions file |childdoc.def|
is installed on the target \TeX{} distribution
and one prefers not to ship it,
it is conceivable to paste a few relevant commands into the sources.

To that end, drop all statements |\input{childdoc.def}|
and perform the replacements as outlined below.
Instead of |\childdocmain{|\textit{main}|}| add the following code
to the top of the main file:
%
\begin{center}
\begin{tabular}{l}
|\||ifdefined\childdocname\endinput\||fi\newif\ifchilddoc|\\
|\edef\childdocname{\scantokens\expandafter{\jobname\noexpand}}|\\
|\def\childdocmain{|\textit{main}|}\||ifx\childdocmain\childdocname\||else|\\
|\childdoctrue\includeonly{\childdocname}\let\jobname\childdocmain\||fi|\\
\end{tabular}
\end{center}
%
Instead of |\childdocof{|\textit{main}|}| just include the main file
at the top of each child file:
%
\begin{center}
|\input{|\textit{main}|}|
\end{center}
%
A simple redirection |\childdocforward{|\textit{dest}|}| is achieved by:
%
\begin{center}
|\def\jobname{|\textit{dest}|}\input{\jobname}|
\end{center}
%
The redirection with prefix
|\childdocforwardprefix[|\textit{prefix}|]{|\textit{dest}|}|
is accomplished by:
%
\begin{center}
\begin{tabular}{l}
|{\edef\jobname{\scantokens\expandafter{\jobname\noexpand}}|\\
|\def\redirectjob |\textit{prefix}|#1~~~{\gdef\jobname{|\textit{dest}|#1}}|\\
|\expandafter\redirectjob\jobname~~~}\input{\jobname}|
\end{tabular}
\end{center}

In an alternative approach,
child documents can be compiled by a specific command line
without additional code or specific definitions:
%
\begin{center}
|... -jobname "|\textit{target}|" "|[\textit{flags}]%
|\includeonly{|\textit{dest}|}\input{|\textit{main}|}"|
\end{center}
%

%%%%%%%%%%%%%%%%%%%%%%%%%%%%%%%%%%%%%%%%%%%%%%%%%%%%%%%%%%%%%%%%%%%%%%%%%%%%%%%%
%%%%%%%%%%%%%%%%%%%%%%%%%%%%%%%%%%%%%%%%%%%%%%%%%%%%%%%%%%%%%%%%%%%%%%%%%%%%%%%%
\section{Information}

%%%%%%%%%%%%%%%%%%%%%%%%%%%%%%%%%%%%%%%%%%%%%%%%%%%%%%%%%%%%%%%%%%%%%%%%%%%%%%%%
\subsection{Copyright}

Copyright \copyright{} 2017--2018 Niklas Beisert

This work may be distributed and/or modified under the
conditions of the \LaTeX{} Project Public License, either version 1.3
of this license or (at your option) any later version.
The latest version of this license is in
  \url{http://www.latex-project.org/lppl.txt}
and version 1.3 or later is part of all distributions of \LaTeX{}
version 2005/12/01 or later.

This work has the LPPL maintenance status `maintained'.

The Current Maintainer of this work is Niklas Beisert.

This work consists of the files |README.txt|, |childdoc.ins| and |childdoc.dtx|
as well as the derived files |childdoc.def|, |cdocsamp.tex|
with |cdocsch1.tex|, |cdocsch2.tex|, |cdocspt3.tex|, |cdocspt4.tex|,
|cdocsdrf.tex|, |cdocsfn1.tex|, |cdocsfn2.tex|
as well as |childdoc.pdf|.

%%%%%%%%%%%%%%%%%%%%%%%%%%%%%%%%%%%%%%%%%%%%%%%%%%%%%%%%%%%%%%%%%%%%%%%%%%%%%%%%
\subsection{Files and Installation}

The package consists of the files:
%
\begin{center}
\begin{tabular}{ll}
    |README.txt|   & readme file \\
    |childdoc.ins| & installation file \\
    |childdoc.dtx| & source file \\
    |childdoc.def| & definition file \\
    |cdocsamp.tex| & sample main file \\
    |cdocsch1.tex| & sample include file \\
    |cdocsch2.tex| & sample include file \\
    |cdocspt3.tex| & sample part file \\
    |cdocspt4.tex| & sample part file \\
    |cdocsdrf.tex| & sample redirection file \\
    |cdocsfn1.tex| & sample redirection file \\
    |cdocsfn2.tex| & sample redirection file \\
    |childdoc.pdf| & manual
\end{tabular}
\end{center}
%
The distribution consists of the files
|README.txt|, |childdoc.ins| and |childdoc.dtx|.
%
\begin{itemize}
\item
Run (pdf)\LaTeX{} on |childdoc.dtx|
to compile the manual |childdoc.pdf| (this file).
\item
Run \LaTeX{} on |childdoc.ins| to create the definitions file |childdoc.def|
and the sample |cdocsamp.tex| with include files
|cdocsch1.tex|, |cdocsch2.tex|, |cdocspt3.tex|, |cdocspt4.tex|,
|cdocsdrf.tex|, |cdocsfn1.tex|, |cdocsfn2.tex|.
Then copy the file |childdoc.def| to an appropriate directory of your \LaTeX{}
distribution, e.g.\ \textit{texmf-root}|/tex/latex/childdoc|.
\end{itemize}

%%%%%%%%%%%%%%%%%%%%%%%%%%%%%%%%%%%%%%%%%%%%%%%%%%%%%%%%%%%%%%%%%%%%%%%%%%%%%%%%
\subsection{Related CTAN Packages}

There are several other packages which offer a similar functionality:
%
\begin{itemize}
\item
The packages
\href{http://ctan.org/pkg/docmute}{\textsf{docmute}},
\href{http://ctan.org/pkg/includex}{\textsf{includex}} and
\href{http://ctan.org/pkg/standalone}{\textsf{standalone}}
provide commands to include only the document body of
a child file thus allowing both files to be compiled individually.
\item
The packages \href{http://ctan.org/pkg/subdocs}{\textsf{subdocs}}
and \href{http://ctan.org/pkg/subfiles}{\textsf{subfiles}}
provide structures in which the main and child documents can be
encapsulated and allowing them to be compiled individually.
The inclusion mechanism is different from the conventional |\include|.
\item
The package \href{http://ctan.org/pkg/combine}{\textsf{combine}}
is an elaborate solution to combine several documents into one.
\end{itemize}
%
See also the CTAN topic \href{http://ctan.org/topic/subdocs}{\textsf{subdocs}}
for further related packages.
The present package differs from the above solutions in that
a document structure constructed with the conventional |\include| mechanism
just needs two extra commands at the top of every file
such that all constituent files can be compiled individually.

%%%%%%%%%%%%%%%%%%%%%%%%%%%%%%%%%%%%%%%%%%%%%%%%%%%%%%%%%%%%%%%%%%%%%%%%%%%%%%%%
%\subsection{Feature Suggestions}
%
%The following is a list of features which may be useful for future
%versions of this package:
%%
%\begin{itemize}
%\item
%\ldots
%\end{itemize}

%%%%%%%%%%%%%%%%%%%%%%%%%%%%%%%%%%%%%%%%%%%%%%%%%%%%%%%%%%%%%%%%%%%%%%%%%%%%%%%%
\subsection{Revision History}

%%%%%%%%%%%%%%%%%%%%%%%%%%%%%%%%%%%%%%%%
\paragraph{v2.0:} 2018/12/30

\begin{itemize}
\item
immediate forward processing
\item
added |\childdocby| mechanism
\item
manual restructured
\end{itemize}

%%%%%%%%%%%%%%%%%%%%%%%%%%%%%%%%%%%%%%%%
\paragraph{v1.6:} 2018/01/17

\begin{itemize}
\item
application for development of include files
\item
corrections to manual
\end{itemize}

%%%%%%%%%%%%%%%%%%%%%%%%%%%%%%%%%%%%%%%%
\paragraph{v1.5:} 2017/05/21

\begin{itemize}
\item
more complete structuring introduced
\item
|\childdocof| introduced
\item
|\childdoc| renamed to |\childdocmain|
\item
|\childredirect| renamed to |\childdocforward| and |\childdocforwardprefix|
and functionality expanded
\end{itemize}

%%%%%%%%%%%%%%%%%%%%%%%%%%%%%%%%%%%%%%%%
\paragraph{v1.0:} 2017/04/27

\begin{itemize}
\item
manual and install package
\item
first version published on CTAN
\end{itemize}

%%%%%%%%%%%%%%%%%%%%%%%%%%%%%%%%%%%%%%%%
\paragraph{v0.6:} 2017/04/26

\begin{itemize}
\item
redirection mechanism added
\end{itemize}

%%%%%%%%%%%%%%%%%%%%%%%%%%%%%%%%%%%%%%%%
\paragraph{v0.5:} 2017/04/26

\begin{itemize}
\item
functionality in definition file
\end{itemize}


%%%%%%%%%%%%%%%%%%%%%%%%%%%%%%%%%%%%%%%%%%%%%%%%%%%%%%%%%%%%%%%%%%%%%%%%%%%%%%%%
%%%%%%%%%%%%%%%%%%%%%%%%%%%%%%%%%%%%%%%%%%%%%%%%%%%%%%%%%%%%%%%%%%%%%%%%%%%%%%%%
%%%%%%%%%%%%%%%%%%%%%%%%%%%%%%%%%%%%%%%%%%%%%%%%%%%%%%%%%%%%%%%%%%%%%%%%%%%%%%%%
\appendix

\settowidth\MacroIndent{\rmfamily\scriptsize 000\ }

 \DocInput{childdoc.dtx}

\end{document}
%</driver>
% \fi
%
% %%%%%%%%%%%%%%%%%%%%%%%%%%%%%%%%%%%%%%%%%%%%%%%%%%%%%%%%%%%%%%%%%%%%%%%%%%%%%%
% %%%%%%%%%%%%%%%%%%%%%%%%%%%%%%%%%%%%%%%%%%%%%%%%%%%%%%%%%%%%%%%%%%%%%%%%%%%%%%
% \section{Sample}
%\iffalse
%<*samplemain>
%\fi
%
% The following presents a sample document
% with two chapters, two parts, a title page,
% a compile flag as well as three forwarding files to set the flag.
% It consists of eight |.tex| files:
% \begin{center}
% \begin{tabular}{ll}
% |cdocsamp.tex|&main file\\
% |cdocsch1.tex|&include file for chapter 1\\
% |cdocsch2.tex|&include file for chapter 2\\
% |cdocspt3.tex|&include file for part 3\\
% |cdocspt4.tex|&include file for part 4\\
% |cdocsdrf.tex|&forwarding file for main file in draft mode\\
% |cdocsfi1.tex|&forwarding file for final version of chapter 1\\
% |cdocsfi2.tex|&forwarding file for final version of chapter 2\\
% \end{tabular}
% \end{center}
% Each of the eight files can be compiled directly by the \LaTeX{} compiler.
%
% %%%%%%%%%%%%%%%%%%%%%%%%%%%%%%%%%%%%%%
% \paragraph{Main File.}
%
% The main file is called |cdocsamp.tex|.
%
% Load the \textsf{childdoc} definitions and
% declare the filename for the main document:
%    \begin{macrocode}
\input{childdoc.def}
\childdocmain{}
%    \end{macrocode}

% Optional override for |\version| flag:
%    \begin{macrocode}
%%\ifchilddoc\else\providecommand{\version}{draft}\fi
%    \end{macrocode}

% Define the default values for the |\version| flag
% (|final| for the main file and |draft| for childs):
%    \begin{macrocode}
\ifchilddoc
\providecommand{\version}{draft}
\else
\providecommand{\version}{final}
\fi
%    \end{macrocode}

% Load the standard document class:
%    \begin{macrocode}
\documentclass[12pt]{article}
%    \end{macrocode}

% Start the document body:
%    \begin{macrocode}
\begin{document}
%    \end{macrocode}

% Declare a title page.
% Print title, part of document being processed and version flag:
%    \begin{macrocode}
\addtocounter{page}{-1}
\begin{center}
{\LARGE\bfseries{}childdoc example\par}
\vspace{1cm}
\ifchilddoc
\ifchilddocmanual part\else chapter\fi:
`\childdocname' of `\childdocjob'\par
\else
main document: `\childdocjob'\par
\fi
version: \version\par
\end{center}
\newpage
%    \end{macrocode}

% Manually include selected file,
% otherwise process as usual:
%    \begin{macrocode}
\ifchilddocmanual
\section*{part `\childdocname'}
\input{\childdocname}
\else
%    \end{macrocode}

% Include the two chapters:
%    \begin{macrocode}
\include{cdocsch1}
\include{cdocsch2}
%    \end{macrocode}

% Include the two parts unless only chapters should be displayed:
%    \begin{macrocode}
\ifchilddoc\else
\section{part three}
\input{cdocspt3}
\section{part four}
\input{cdocspt4}
\fi
%    \end{macrocode}

% Process as usual until here:
%    \begin{macrocode}
\fi
%    \end{macrocode}

% End of document body:
%    \begin{macrocode}
\end{document}
%    \end{macrocode}
%\iffalse
%</samplemain>
%\fi
%
% %%%%%%%%%%%%%%%%%%%%%%%%%%%%%%%%%%%%%%
% \paragraph{Chapter Include Files.}
%
% The include files are called |cdocsch1.tex| and |cdocsch2.tex|.
%
%\iffalse
%<*samplechap1|samplechap2>
%\fi

% Optional override for |\version| flag:
%    \begin{macrocode}
%%\providecommand{\version}{final}
%    \end{macrocode}

% Include the main document:
%    \begin{macrocode}
\input{childdoc.def}
\childdocof{cdocsamp}
%    \end{macrocode}

%\iffalse
%</samplechap1|samplechap2>
%\fi
%
%\iffalse
%<*samplechap1>
%\fi
% Some text for chapter 1:
%    \begin{macrocode}
\section{one}
some text in chapter one
%    \end{macrocode}

%\iffalse
%</samplechap1>
%\fi
% Some text for chapter 2:
%\iffalse
%<*samplechap2>
%\fi
%    \begin{macrocode}
\section{two}
more text in chapter two
%    \end{macrocode}

%\iffalse
%</samplechap2>
%\fi
%
% %%%%%%%%%%%%%%%%%%%%%%%%%%%%%%%%%%%%%%
% \paragraph{Part Include Files.}
%
% The include files are called |cdocspt3.tex| and |cdocspt4.tex|.
%
%\iffalse
%<*samplepart3|samplepart4>
%\fi

% Optional override for |\version| flag:
%    \begin{macrocode}
%%\providecommand{\version}{final}
%    \end{macrocode}

% Include the main document:
%    \begin{macrocode}
\input{childdoc.def}
\childdocby{cdocsamp}
%    \end{macrocode}

%\iffalse
%</samplepart3|samplepart4>
%\fi
%
%\iffalse
%<*samplepart3>
%\fi
% Some text for part 3:
%    \begin{macrocode}
some text in part three
%    \end{macrocode}

%\iffalse
%</samplepart3>
%\fi
% Some text for part 4:
%\iffalse
%<*samplepart4>
%\fi
%    \begin{macrocode}
more text in part four
%    \end{macrocode}

%\iffalse
%</samplepart4>
%\fi
%
% %%%%%%%%%%%%%%%%%%%%%%%%%%%%%%%%%%%%%%
% \paragraph{Forwarding for a Complete Draft.}
%
% The following forwarding file |cdocsdrf.tex|
% compiles the main document in draft mode:
%\iffalse
%<*sampledraft>
%\fi
%    \begin{macrocode}
\def\version{draft}
\input{childdoc.def}
\childdocforward{cdocsamp}
%    \end{macrocode}

%\iffalse
%</sampledraft>
%\fi
%
% %%%%%%%%%%%%%%%%%%%%%%%%%%%%%%%%%%%%%%
% \paragraph{Forwarding for Final Version of the Chapters.}
%
% The following forwarding files |cdocsfn1.tex| and |cdocsfn2.tex|
% (with identical content)
% compile the final versions of the child documents
% |cdocsch1.tex| and |cdocsch2.tex|, respectively:
%\iffalse
%<*samplefinal>
%\fi
%    \begin{macrocode}
\def\version{final}
\input{childdoc.def}
\childdocforwardprefix[cdocsamp]{cdocsfn}{cdocsch}
%    \end{macrocode}

%\iffalse
%</samplefinal>
%\fi
%
% %%%%%%%%%%%%%%%%%%%%%%%%%%%%%%%%%%%%%%
% \paragraph{Command Line Processing.}
%
% The following three command lines generate the output files
% |cdocscld|, |cdocscl1| and |cdocscl2|
% which should be identical to
% |cdocsdrf|, |cdocsch1| and |cdocsfn2|, respectively:
% \begin{center}
% \begin{tabular}{l}
% |latex -jobname cdocscld \|\\
% |  "\def\version{draft}\input{childdoc.def}\childdocforward{cdocsamp}"|\\
% |latex -jobname cdocscl1 \|\\
% |  "\input{childdoc.def}\childdocforward[cdocsamp]{cdocsch1}"|\\
% |latex -jobname cdocscl2 \|\\
% |  "\def\version{final}\input{childdoc.def}\childdocforward{cdocsch2}"|
% \end{tabular}
% \end{center}
% Note that the trailing backslash on each first line
% merely continues the input to the second line
% (for convenient cut ant paste).
% Furthermore, the command |latex| can be replaced by any
% of its alternative versions such as |pdflatex|.
%
% %%%%%%%%%%%%%%%%%%%%%%%%%%%%%%%%%%%%%%%%%%%%%%%%%%%%%%%%%%%%%%%%%%%%%%%%%%%%%%
% %%%%%%%%%%%%%%%%%%%%%%%%%%%%%%%%%%%%%%%%%%%%%%%%%%%%%%%%%%%%%%%%%%%%%%%%%%%%%%
% \section{Implementation}
%\iffalse
%<*package>
%\fi
%
% This section describes the definitions file |childdoc.def|.

% The definitions cannot be loaded using |\usepackage| or |\RequirePackage|
% which has a mechanism to prevent loading a style file more than once.
% When loading the definitions by means of |\input|
% multiple instances have to be prevented manually:
%\iffalse
%This code needs to be before the `\ProvidesFile' directive
%which is defined at the beginning of this file.
%Therefore it is also placed there and commented out here.
%</package>
%<*discard>
%\fi
%    \begin{macrocode}
\ifdefined\childdocmain\endinput\fi
%    \end{macrocode}
%\iffalse
%</discard>
%<*package>
%\fi
%
% \macro{\ifchilddoc}
% \macro{\ifchilddocmanual}
% The conditional |\ifchilddoc| tells whether a
% child (true) or main (false) document is being compiled.
% The conditional |\ifchilddocmanual| tells whether
% the |\includeonly| mechanism is used (false) or
% the selection of child files must be performed manually (true).
% The definitions initialise to false:
%    \begin{macrocode}
\newif\ifchilddoc
\newif\ifchilddocmanual
%    \end{macrocode}

% \macro{\childdocname}
% \macro{\childdocjob}
% The macro |\childdocname| stores the name of the main document
% to be compiled. The macro |\childdocjob| stores the name of
% the document on which the \LaTeX{} compiler was originally invoked.
% The content of |\jobname| cannot be compared
% to filenames specified in the source due to different catcodes.
% The following code rescans |\jobname|, stores the result
% in |\childdocname| and saves a copy in |\childdocjob|:
%    \begin{macrocode}
\edef\childdocname{\scantokens\expandafter{\jobname\noexpand}}
\let\childdocjob\childdocname
%    \end{macrocode}

% \macro{\childdocdisable}
% The macro |\childdocdisable| prevents the main file
% from being processed more than once.
% At this stage, the main document command |\childdocmain|
% is assumed to be called once again where it should do nothing.
% Any subsequent call to it should prevent
% a secondary processing of the main document
% It overwrites the forwarding commands
% |\childdocof| and |\childdocforward|
% with empty macros to prevent further inclusions of the main document:
%    \begin{macrocode}
\newcommand{\childdocdisable}
{
  \renewcommand{\childdocmain}[1]{\renewcommand{\childdocmain}[1]{\endinput}}
  \renewcommand{\childdocof}[1]{}
  \renewcommand{\childdocby}[2][]{}
  \renewcommand{\childdocforward}[2][]{}
  \renewcommand{\childdocdisable}{}
}
%    \end{macrocode}

% \macro{\childdocmain}
% The macro |\childdocmain| is to be called at the top of the main file
% with nothing or the main filename (without extension) as argument.
% First, it breaks loops.
% If the argument is not empty and does not match |\childdocname|
% (which is set by the first inclusion of |childdoc.def|),
% |\ifchilddoc| is set to true, |\includeonly| is applied to the child file
% and |\jobname| is set to the main file
% (for proper handling of |.aux| files):
%    \begin{macrocode}
\newcommand{\childdocmain}[1]
{
  \childdocdisable\childdocmain{}
  \if?#1?\else
    \begingroup
      \def\childdoctmp{#1}
      \ifx\childdoctmp\childdocname
        \def\childdoctmp{}
      \else
        \def\childdoctmp
        {
          \childdoctrue
          \includeonly{\childdocname}
          \def\childdocjob{#1}
          \def\jobname{#1}
        }
      \fi
      \expandafter
    \endgroup
    \childdoctmp
  \fi
}
%    \end{macrocode}

% \macro{\childdocof}
% The command |\childdocof| redirects
% compilation to the main file |#1|.
%    \begin{macrocode}
\newcommand{\childdocof}[1]
{
  \childdocdisable
  \childdoctrue
  \includeonly{\childdocname}
  \def\jobname{#1}
  \def\childdocjob{#1}
  \input{#1}
}
%    \end{macrocode}

% \macro{\childdocby}
% The command |\childdocby| ....
%    \begin{macrocode}
\newcommand{\childdocby}[2][]
{
  \childdocdisable
  \childdoctrue
  \childdocmanualtrue
  \if?#1?\else
    \def\jobname{#2}
  \fi
  \def\childdocjob{#2}
  \input{#2}
  \endinput
}
%    \end{macrocode}

% \macro{\childdocforward}
% The command |\childdocforward| redirects
% compilation to the main file or
% (if the optional argument is given) a child file.
% Parameters are set as if the main file
% or a child file starting with |\childdocof| was compiled.
% Then compilation is handed over to the main file:
%    \begin{macrocode}
\newcommand{\childdocforward}[2][]
{
  \begingroup
    \if?#1?
      \def\childdoctmp
      {
        \def\childdocname{#2}
        \def\childdocjob{#2}
        \def\jobname{#2}
        \input{#2}
        \endinput
      }
    \else
      \def\childdoctmp
      {
        \childdocdisable
        \def\childdocname{#2}
        \childdoctrue
        \includeonly{#2}
        \def\childdocjob{#1}
        \def\jobname{#1}
        \input{#1}
        \endinput
      }
    \fi
    \expandafter
  \endgroup
  \childdoctmp
}
%    \end{macrocode}

% \macro{\childdocforwardprefix}
% The command |\childdocforwardprefix| redirects
% compilation to the main or a child file by means of a pattern.
% The prefix |#1| in the current filename is replaced by |#2|
% and the suffix of the current filename is kept
% (it is assumed that the filename does not contain the substring `|~~~|'
% which is used as a delimiter).
% Compilation is handed over to the new file by |\childdocforward|:
%    \begin{macrocode}
\newcommand{\childdocforwardprefix}[3][]
{
  \begingroup
    \def\childdocextract #2##1~~~{\def\childdoctmp{\childdocforward[#1]{#3##1}}}
    \expandafter\childdocextract\childdocname~~~
    \expandafter
  \endgroup
  \childdoctmp
}
%    \end{macrocode}

% \macro{\childdoc}
% The deprecated macro |\childdoc| is a legacy version of |\childdocmain|:
%    \begin{macrocode}
\newcommand{\childdoc}{\childdocmain}
%    \end{macrocode}

% \macro{\childdocredirect}
% The deprecated macro |\childdocredirect| is a legacy version
% of |\childdocforward| and |\childdocforwardprefix|:
%    \begin{macrocode}
\newcommand{\childdocredirect}[2][]
{
  \begingroup
    \if?#1?
      \def\childdoctmp{\childdocforward{#2}}
    \else
      \def\childdoctmp{\childdocforwardprefix{#1}{#2}}
    \fi
    \expandafter
  \endgroup
  \childdoctmp
}
%    \end{macrocode}

%\iffalse
%</package>
%\fi
%
\endinput

\childdocforwardprefix[cdocsamp]{cdocsfn}{cdocsch}
%    \end{macrocode}

%\iffalse
%</samplefinal>
%\fi
%
% %%%%%%%%%%%%%%%%%%%%%%%%%%%%%%%%%%%%%%
% \paragraph{Command Line Processing.}
%
% The following three command lines generate the output files
% |cdocscld|, |cdocscl1| and |cdocscl2|
% which should be identical to
% |cdocsdrf|, |cdocsch1| and |cdocsfn2|, respectively:
% \begin{center}
% \begin{tabular}{l}
% |latex -jobname cdocscld \|\\
% |  "\def\version{draft}% \iffalse
%
% childdoc.dtx Copyright (C) 2017-2018 Niklas Beisert
%
% This work may be distributed and/or modified under the
% conditions of the LaTeX Project Public License, either version 1.3
% of this license or (at your option) any later version.
% The latest version of this license is in
%   http://www.latex-project.org/lppl.txt
% and version 1.3 or later is part of all distributions of LaTeX
% version 2005/12/01 or later.
%
% This work has the LPPL maintenance status `maintained'.
%
% The Current Maintainer of this work is Niklas Beisert.
%
% This work consists of the files childdoc.dtx and childdoc.ins
% and the derived files childdoc.def and cdocsamp.tex with
% cdocsch1.tex, cdocsch2.tex, cdocsdrf.tex, cdocsfn1.tex, cdocsfn2.tex.
%
%<package>\ifdefined\childdocmain\endinput\fi
%<package>\ProvidesFile{childdoc.def}[2018/12/30 v2.0 child document driver]
%<samplemain>\ProvidesFile{cdocsamp.tex}[2018/12/30 v2.0 sample for childdoc]
%<*driver>
%\ProvidesFile{childdoc.drv}[2018/12/30 v2.0 childdoc reference manual file]
\PassOptionsToClass{10pt,a4paper}{article}
\documentclass{ltxdoc}

\usepackage[margin=35mm]{geometry}
\usepackage{hyperref}
\usepackage{hyperxmp}
\usepackage[usenames]{color}

\hypersetup{colorlinks=true}
\hypersetup{pdfstartview=FitH}
\hypersetup{pdfpagemode=UseNone}
\hypersetup{pdfsource={}}
\hypersetup{pdflang={en-UK}}
\hypersetup{pdfcopyright={Copyright 2017-2018 Niklas Beisert.
  This work may be distributed and/or modified under the
  conditions of the LaTeX Project Public License, either version 1.3
  of this license or (at your option) any later version.}}
\hypersetup{pdflicenseurl={http://www.latex-project.org/lppl.txt}}
\hypersetup{pdfcontactaddress={ETH Zurich, ITP, HIT K,
  Wolfgang-Pauli-Strasse 27}}
\hypersetup{pdfcontactpostcode={8093}}
\hypersetup{pdfcontactcity={Zurich}}
\hypersetup{pdfcontactcountry={Switzerland}}
\hypersetup{pdfcontactemail={nbeisert@itp.phys.ethz.ch}}
\hypersetup{pdfcontacturl={http://people.phys.ethz.ch/\xmptilde nbeisert/}}

\newcommand{\secref}[1]{\hyperref[#1]{section \ref*{#1}}}

\parskip1ex
\parindent0pt
\let\olditemize\itemize
\def\itemize{\olditemize\parskip0pt}

\begin{document}

\title{The \textsf{childdoc} Package}
\hypersetup{pdftitle={The childdoc Package}}
\author{Niklas Beisert\\[2ex]
  Institut f\"ur Theoretische Physik\\
  Eidgen\"ossische Technische Hochschule Z\"urich\\
  Wolfgang-Pauli-Strasse 27, 8093 Z\"urich, Switzerland\\[1ex]
  \href{mailto:nbeisert@itp.phys.ethz.ch}
  {\texttt{nbeisert@itp.phys.ethz.ch}}}
\hypersetup{pdfauthor={Niklas Beisert}}
\hypersetup{pdfsubject={Manual for the LaTeX2e Package childdoc}}
\date{30 December 2018, \textsf{v2.0}}
\maketitle

\begin{abstract}\noindent
\textsf{childdoc} is a \LaTeXe{} package
that enables the direct compilation
of document sections included by |\include|
to individual files.
\end{abstract}

\begingroup
\parskip0ex
\tableofcontents
\endgroup

%%%%%%%%%%%%%%%%%%%%%%%%%%%%%%%%%%%%%%%%%%%%%%%%%%%%%%%%%%%%%%%%%%%%%%%%%%%%%%%%
%%%%%%%%%%%%%%%%%%%%%%%%%%%%%%%%%%%%%%%%%%%%%%%%%%%%%%%%%%%%%%%%%%%%%%%%%%%%%%%%
\section{Introduction}

\LaTeX{} provides a mechanism to structure a large document (such as a book)
into a main file and several child files (containing the chapters)
using the |\include| command.
This mechanism is beneficial for documents
which span hundreds of pages in order to
make the source file(s) more manageable.
Moreover, compilation can be restricted to
selected child files by means of the |\includeonly| command.
The latter feature can be used to reduce the compilation time while editing
(this was significantly more useful in the earlier days of \LaTeX{})
or to generate a smaller document which is easier to navigate.
Another application of |\includeonly| is to generate
documents consisting of selected parts of the complete document.

However, there are a few drawbacks of the plain |\include| mechanism:
\begin{itemize}
\item
The child files cannot be compiled on their own,
they can only be compiled via the main file.
A naive editing environment
(such as a text editor with an option
to have the current file processed by \LaTeX)
may require one to switch to the main file before compiling;
attempting to compile the child file produces errors.
\item
The main file must be modified (each time)
to adjust the |\includeonly| command
to the present needs. This easily leaves the main file in a messy state.
\item
The generated document will always carry the filename
of the main document. This is inconvenient if
several child files are to be compiled and
to be kept for distribution.
\end{itemize}

The present package provides a simple interface
to make child files individually compilable by \LaTeX{}.
Compiling a child file then has the same effect as compiling
the main file with an |\includeonly| command
to select the appropriate child.
Moreover the generated document will carry the name of the child
rather than the main file.
This resolves all three above issues.

This feature is meant to make the editing of books,
thesis documents and lecture notes somewhat more convenient.
However, the package can also be used efficiently for
composing a series of documents (such as exercise sheets)
which are typically distributed individually.
It then assists the author in generating the individual documents
(potentially in different versions)
as well as a document containing the collected series.
Another application is in developing style files
or other kinds of included material
where compilation of the style file could redirect
to a sample or test file.

%%%%%%%%%%%%%%%%%%%%%%%%%%%%%%%%%%%%%%%%%%%%%%%%%%%%%%%%%%%%%%%%%%%%%%%%%%%%%%%%
%%%%%%%%%%%%%%%%%%%%%%%%%%%%%%%%%%%%%%%%%%%%%%%%%%%%%%%%%%%%%%%%%%%%%%%%%%%%%%%%
\section{Usage}

First of all, the package \textsf{childdoc} is \emph{not} a standard
\LaTeXe{} |.sty| style file! Therefore it needs to be invoked in
a non-standard way.

%%%%%%%%%%%%%%%%%%%%%%%%%%%%%%%%%%%%%%%%%%%%%%%%%%%%%%%%%%%%%%%%%%%%%%%%%%%%%%%%
\subsection{Included Files}
\label{sec:include}

%%%%%%%%%%%%%%%%%%%%%%%%%%%%%%%%%%%%%%%%
\DescribeMacro{\childdocmain}
To use the package, add the commands
\begin{center}
\begin{tabular}{l}
|\input{childdoc.def}|\\
|\childdocmain{}|\\
\end{tabular}
\end{center}
at the very top of the main \LaTeX{} file,
in particular \emph{before} the |\documentclass| statement!
The argument of |\childdocmain| should be left empty
(but it must be present).

%%%%%%%%%%%%%%%%%%%%%%%%%%%%%%%%%%%%%%%%
\DescribeMacro{\childdocof}
Furthermore, add the commands
\begin{center}
\begin{tabular}{l}
|\input{childdoc.def}|\\
|\childdocof{|\textit{main}|}|\\
\end{tabular}
\end{center}
at the top of every child file \textit{child}
which is included by |\include{|\textit{child}|}|
from within the main file
(or at least for those files to be compiled individually).
The argument \textit{main} must be the filename of the main file.

There are a couple of
considerations in setting up the main and child documents:

%%%%%%%%%%%%%%%%%%%%%%%%%%%%%%%%%%%%%%%%
\paragraph{Restrictions.}

Please note the following restrictions:
\begin{itemize}
\item
|\childdocmain| must be called with one argument \textit{main}
to ensure compatibility with earlier version of the package.
It must either be empty (|\childdocmain{}|)
or precisely match the filename of the main file in which it is specified.
See \secref{sec:detection} for further information.
\item
The filename \textit{main} must be specified without the |.tex| extension.
\item
The filename \textit{main} is case sensitive
(even in case-insensitive file systems)
due to internal string comparison.
\item
The argument \textit{main} should be fully expanded, it cannot be a macro.
\item
Subdirectories and special characters should be avoided in filenames.
\item
The command |\childdocmain{|\textit{main}|}| must be followed by a whitespace.
It should not be followed immediately by another command
or by a comment mark `|%|'.
This is because the \TeX{} parser reads the token immediately following
the argument of |\childdocmain| and puts it
at the beginning of every child section;
however, a white\-space is ignored.
\end{itemize}

%%%%%%%%%%%%%%%%%%%%%%%%%%%%%%%%%%%%%%%%
\paragraph{Content of Main File.}

It is advisable to place all content in the child files included by |\include|.
Any output contained in the main file will appear in all child documents
unless suppressed manually;
it cannot be suppressed automatically by the |\includeonly| directive
and thus should normally be avoided.
A method to include some content in the main file
by means of conditional processing is described in \secref{sec:conditional}.

%%%%%%%%%%%%%%%%%%%%%%%%%%%%%%%%%%%%%%%%
\paragraph{Page Numbering.}

When only a part of the document is compiled,
the appropriate numbering of pages
(as well as other status parameters)
is determined from the |.aux| files.
The latter contain information from previous passes.
However this information needs to propagate through
all intermediate child documents.
Therefore the page numbering in child documents may well
be inconsistent until the complete document is compiled at least once.

A useful (if unconventional) way to always ensure a consistent
page numbering is to restart the numbering in each child document
and denote the pages by `\textit{child}|.|\textit{page}'
where \textit{child} represents the chapter/section number of the child file.
This can be achieved by the command
|\numberwithin{page}{|\textit{child}|}|
of the \textsf{amsmath} package
where \textit{child} can be |chapter| or |section|
depending on the chosen structuring.
Alternatively, one can modify the macro |\thepage| appropriately
and reset the counter |page| at the start of each child file.

%%%%%%%%%%%%%%%%%%%%%%%%%%%%%%%%%%%%%%%%%%%%%%%%%%%%%%%%%%%%%%%%%%%%%%%%%%%%%%%%
\subsection{Conditional Processing}
\label{sec:conditional}

The package provides a mechanism to compile different versions
of a document. To customise the versions further some conditional processing
can come in handy to distinguish which version is being compiled.
The package provides two macros to describe the compilation context:

%%%%%%%%%%%%%%%%%%%%%%%%%%%%%%%%%%%%%%%%
\DescribeMacro{\ifchilddoc}
The conditional |\ifchilddoc| distinguishes between the compilation of
child documents and the main document:
%
\begin{center}
|\ifchilddoc |\textit{child-code}| |[|\||else |\textit{main-code}]| \||fi|
\end{center}

%%%%%%%%%%%%%%%%%%%%%%%%%%%%%%%%%%%%%%%%
\DescribeMacro{\childdocname}
\DescribeMacro{\childdocjob}
The macro |\childdocname| contains the filename (without extension)
of the main or child file being processed.
Note that |\childdocjob| will always contain the name of the main file.

%%%%%%%%%%%%%%%%%%%%%%%%%%%%%%%%%%%%%%%%
\paragraph{Title Page.}

Conditional processing can be used to include a title or banner page
in the main document when proper precautions are taken.
Importantly, the code in the main file should ensure that the page counter
(as well as other status parameters which are stored in the |.aux| files)
takes the same value after the conditional processing.
Otherwise the page numbers may take divergent values
depending on which part is compiled.

For example, a title page could be declared by:
%
\begin{center}
\begin{tabular}{l}
|\ifchilddoc\||else|\\
|\addtocounter{page}{-1}|\\
\textit{code for title page}\\
|\newpage|\\
|\||fi|
\end{tabular}
\end{center}
%
A banner page for the child documents can be generated by:
%
\begin{center}
\begin{tabular}{l}
|\ifchilddoc|\\
|\addtocounter{page}{-1}|\\
\textit{code for banner page}\\
|\newpage|\\
|\||fi|
\end{tabular}
\end{center}
%
Here one could write a message such as:
\begin{center}
|This is the part \childdocname{} of \childdocjob{}.|
\end{center}

%%%%%%%%%%%%%%%%%%%%%%%%%%%%%%%%%%%%%%%%%%%%%%%%%%%%%%%%%%%%%%%%%%%%%%%%%%%%%%%%
\subsection{Flags}
\label{sec:flags}

The package makes it easy to generate different versions
of the main or child documents.
To this end compilation flags can be defined
and assigned different default values.
They will be particularly useful in conjunction
with the forwarding mechanism described in \secref{sec:forward}.

For example, it may be useful to have a flag |\version|
which can be set to |draft| or |final|.
The document source will contain some conditional code
depending on the value of |\version|.
Suppose further, the flag should default to |final| for the main file
and to |draft| for child files
which is a natural assignment for editing the document.
This is achieved by placing the following code
in the preamble of the main document
(below the |\childdocmain| directive):
%
\begin{center}
\begin{tabular}{l}
|\ifchilddoc|\\
|\providecommand{\version}{draft}|\\
|\||else|\\
|\providecommand{\version}{final}|\\
|\||fi|
\end{tabular}
\end{center}
%
The definition by |\providecommand| makes sure
that previous definitions are not overwritten.
Further statements |\providecommand{\version}{...}|
can thus be added before the above code to override it.

For the main file, one might add a line
(between |\childdocmain| and the above block)
%
\begin{center}
|%\ifchilddoc\||else\providecommand{\version}{draft}\||fi|
\end{center}
%
which can be uncommented to produce a draft version.
Likewise one can add a line to the very top of a child file
(above the |\childdocof{|\textit{main}|}| directive)
%
\begin{center}
|%\providecommand{\version}{final}|
\end{center}
%
which can be uncommented to produce the final version of this child document.

%%%%%%%%%%%%%%%%%%%%%%%%%%%%%%%%%%%%%%%%%%%%%%%%%%%%%%%%%%%%%%%%%%%%%%%%%%%%%%%%
\subsection{Forwarding}
\label{sec:forward}

Different versions of the main or child documents
using compilation flags as described in \secref{sec:flags}
can be (permanently) stored in different files
for convenient compilation, viewing and distribution.
To this end, the package defines a command
to pass on compilation to a different file:

%%%%%%%%%%%%%%%%%%%%%%%%%%%%%%%%%%%%%%%%
\DescribeMacro{\childdocforward}
The command |\childdocforward| redirects processing to
another source file:
%
\begin{center}
\begin{tabular}{l}
|\input{childdoc.def}|\\
|\childdocforward[|\textit{main}|]{|\textit{dest}|}|\\
\end{tabular}
\end{center}
%
The argument \textit{dest} is the destination file
(without extension).
It should be the main file or one of the child files.
Note that further \textsf{childdoc} directives
such as |\childdocof| and |\childdocforward|
in the indicated file will be processed in this form.
The optional argument \textit{main}
passes on directly to the main file \textit{main}
while pretending to compile the child \textit{dest}.
This form behaves as if \textit{dest}
issues |\childdocof{|\textit{main}|}| right away,
and no further \textsf{childdoc} directives will be processed.

%%%%%%%%%%%%%%%%%%%%%%%%%%%%%%%%%%%%%%%%
\DescribeMacro{\...prefix}
In the alternative form |\childdocforwardprefix|,
%
\begin{center}
\begin{tabular}{l}
|\input{childdoc.def}|\\
|\childdocforwardprefix[|\textit{main}|]{|\textit{prefix}|}{|\textit{dest}|}|
\end{tabular}
\end{center}
%
the destination file is determined by a pattern
depending on the current file:
To make this work, the current file must be called
`{\textit{prefix}\hspace{0.2em}\textit{suffix}}'
with \textit{prefix} matching precisely the argument.
Processing is then passed on to the file
`{\textit{dest}\hspace{0.2em}\textit{suffix}}'.
Surely, the same effect is achieved by
directly specifying the
argument `{\textit{dest}\hspace{0.2em}\textit{suffix}}'
in the first form.
However, that requires to set up a different file
for each child. With the alternative form of the command
all these files can have exactly the same content
which simplifies setting them up and maintaining them.

For example, the following file |draft.tex|
with a compilation flag |\version| as described in \secref{sec:flags}
compiles the main document as a draft:
%
\begin{center}
\begin{tabular}{l}
|\def\version{draft}|\\
|\input{childdoc.def}|\\
|\childdocforward{|\textit{main}|}|
\end{tabular}
\end{center}
%
Likewise, the following files |final|\textit{nn}|.tex|
compile the final version of the child document
|child|\textit{nn}|.tex|:
%
\begin{center}
\begin{tabular}{l}
|\def\version{final}|\\
|\input{childdoc.def}|\\
|\childdocforwardprefix{final}{child}|
\end{tabular}
\end{center}
%

Note that when several versions of a main file and/or of each child file
are to be generated, it may be convenient to set up a |Makefile| or
shell script to automatise the process.

%%%%%%%%%%%%%%%%%%%%%%%%%%%%%%%%%%%%%%%%%%%%%%%%%%%%%%%%%%%%%%%%%%%%%%%%%%%%%%%%
\subsection{Command Line Processing}
\label{sec:commandline}

The effect of redirection files can also be achieved by invoking
the \LaTeX{} compiler with a more elaborate command line.
Most conveniently this should be done as part
of a shell script or a |Makefile|.

When using \textsf{childdoc} in the main file, the following
command lines effectively perform a redirection
(note that depending on the shell being used,
backslashes may have to be doubled: `|\|' $\to$ `|\\|'):
%
\begin{center}
|... -jobname "|\textit{target}|" |\\|"|[\textit{flags}]%
|\input{childdoc.def}\childdocforward[|\textit{main}|]{|\textit{dest}|}"|
\end{center}
%
Here \textit{target} is the name of the output file,
\textit{main} is the name of the main file
and \textit{dest} is the name of the main or child file to be processed
(all filenames without extensions).
The optional argument \textit{main} can be omitted
if \textit{main} matches \textit{dest}.
Optionally, compilation \textit{flags} can be defined via |\def| commands.
This command line makes the \TeX{} engine believe
it is compiling the file \textit{target}
whose content is specified as the latter parameter.
The provided code then forwards the processing to
\textit{main} or \textit{dest} as described in \secref{sec:forward}.

%%%%%%%%%%%%%%%%%%%%%%%%%%%%%%%%%%%%%%%%%%%%%%%%%%%%%%%%%%%%%%%%%%%%%%%%%%%%%%%%
\subsection{Include by Input}
\label{sec:input}

Including child documents by |\include| has some restrictions by design.
Most notably, the content of a child document always occupies
its own set of pages; pages cannot be shared between child documents.
Usually, this behaviour makes perfect sense
because each child document contain an essential part of the document.
However, in some situations it may be desirable to compose
a document from a collection of parts
without having mandatory page breaks between then.
For this case, the package
provides a mechanism to include parts
by |\input| which can also be processed individually.
However, by construction this mechanism
requires manual handling of the content to be output.

%%%%%%%%%%%%%%%%%%%%%%%%%%%%%%%%%%%%%%%%
\DescribeMacro{\ifchilddocmanual}
The main file should be prepared as usual, see \secref{sec:include}.
However, the document body must make a distinction
between processing of an individual part and of the main document, e.g.:
%
\begin{center}
\begin{tabular}{l}
|\ifchilddocmanual|\\
|\input{\childdocname}|\\
|\||else|\\
\textit{document body with }|\input{|\textit{part}|}|\\
|\||fi|
\end{tabular}
\end{center}
%
The conditional |\ifchilddocmanual| is true whenever
a part to be included by |\input| is being compiled,
and the name of the part is stored in |\childdocname|.

%%%%%%%%%%%%%%%%%%%%%%%%%%%%%%%%%%%%%%%%
\DescribeMacro{\childdocby}
Each part to be included by |\input| should start with:
%
\begin{center}
\begin{tabular}{l}
|\input{childdoc.def}|\\
|\childdocby{|\textit{main}|}|\\
\end{tabular}
\end{center}
%
The directive |\childdocby| is similar to |\childdocof|
described in \secref{sec:include},
but the subsequent selection of content must be done manually.
To that end, both |\ifchilddoc| and |\ifchilddocmanual|
will be true upon processing of a part,
and the name of the part is stored in |\childdocname|.
Note that |\jobname| will be set to the filename of the current part
so that each part receives an individual |.aux| file
that does not interfere with the |.aux| file(s) of the main document.
This behaviour can be altered by the alternative form
|\childdocby[*]{|\textit{main}|}| (with a non-empty optional argument)
which uses the |.aux| file of the main document
by setting |\jobname| to \textit{main}.

%%%%%%%%%%%%%%%%%%%%%%%%%%%%%%%%%%%%%%%%%%%%%%%%%%%%%%%%%%%%%%%%%%%%%%%%%%%%%%%%
\subsection{Driver Development}
\label{sec:driver}

The \textsf{childdoc} mechanism can also be use for the development
of definition files such as \LaTeX{} styles or classes.
This case differs from the above setup with multiple parts
included by |\include| in that no |\includeonly| should be invoked.
This can be achieved by starting the include file
(before |\ProvidesPackage|) with:
%
\begin{center}
\begin{tabular}{l}
|\input{childdoc.def}|\\
|\childdocforward{|\textit{main}|}|\\
\end{tabular}
\end{center}
%
or alternatively with:
%
\begin{center}
\begin{tabular}{l}
|\input{childdoc.def}|\\
|\childdocby{|\textit{main}|}|\\
\end{tabular}
\end{center}
%
Both forms have slightly different effects as described above.
The main file is prepared as usual, see \secref{sec:include}.

%%%%%%%%%%%%%%%%%%%%%%%%%%%%%%%%%%%%%%%%%%%%%%%%%%%%%%%%%%%%%%%%%%%%%%%%%%%%%%%%
\subsection{Legacy Detection}
\label{sec:detection}

The directive |\childdocmain| in the main file can detect
whether the complete document or merely a child is to be compiled
even without using the directive |\childdocof|.
This method is deprecated because it is less robust
and there is no compelling reason to use it;
it is merely provided for backward compatibility
and it may be removed in future versions.

If the detection mechanism is to be used,
it is mandatory to correctly specify
the filename of the main file as the argument of |\childdocmain|:
%
\begin{center}
\begin{tabular}{l}
|\input{childdoc.def}|\\
|\childdocmain{|\textit{main}|}|\\
\end{tabular}
\end{center}
%
If |\jobname| does not match the argument \textit{main} of |\childdocmain|,
it is assumed that |\jobname| points to the child file to be compiled.
When using |\childdocmain| with the main file specified as argument,
it suffices to start a child file
with just |\input{|\textit{main}|}|
without loading of the package and using |\childdocof|.
If instead all processing is done
with the appropriate \textsf{childdoc} directives,
the argument of \textit{main} of |\childdocmain| can be empty.

An alternative version of the command line processing described
in \secref{sec:commandline} using the detection mechanism reads:
%
\begin{center}
|... -jobname "|\textit{target}|" "|[\textit{flags}]%
[|\def\jobname{|\textit{dest}|}|]|\input{|\textit{main}|}"|
\end{center}

%%%%%%%%%%%%%%%%%%%%%%%%%%%%%%%%%%%%%%%%%%%%%%%%%%%%%%%%%%%%%%%%%%%%%%%%%%%%%%%%
\subsection{Manual Code}
\label{sec:manual}

In case one cannot be certain whether the definitions file |childdoc.def|
is installed on the target \TeX{} distribution
and one prefers not to ship it,
it is conceivable to paste a few relevant commands into the sources.

To that end, drop all statements |\input{childdoc.def}|
and perform the replacements as outlined below.
Instead of |\childdocmain{|\textit{main}|}| add the following code
to the top of the main file:
%
\begin{center}
\begin{tabular}{l}
|\||ifdefined\childdocname\endinput\||fi\newif\ifchilddoc|\\
|\edef\childdocname{\scantokens\expandafter{\jobname\noexpand}}|\\
|\def\childdocmain{|\textit{main}|}\||ifx\childdocmain\childdocname\||else|\\
|\childdoctrue\includeonly{\childdocname}\let\jobname\childdocmain\||fi|\\
\end{tabular}
\end{center}
%
Instead of |\childdocof{|\textit{main}|}| just include the main file
at the top of each child file:
%
\begin{center}
|\input{|\textit{main}|}|
\end{center}
%
A simple redirection |\childdocforward{|\textit{dest}|}| is achieved by:
%
\begin{center}
|\def\jobname{|\textit{dest}|}\input{\jobname}|
\end{center}
%
The redirection with prefix
|\childdocforwardprefix[|\textit{prefix}|]{|\textit{dest}|}|
is accomplished by:
%
\begin{center}
\begin{tabular}{l}
|{\edef\jobname{\scantokens\expandafter{\jobname\noexpand}}|\\
|\def\redirectjob |\textit{prefix}|#1~~~{\gdef\jobname{|\textit{dest}|#1}}|\\
|\expandafter\redirectjob\jobname~~~}\input{\jobname}|
\end{tabular}
\end{center}

In an alternative approach,
child documents can be compiled by a specific command line
without additional code or specific definitions:
%
\begin{center}
|... -jobname "|\textit{target}|" "|[\textit{flags}]%
|\includeonly{|\textit{dest}|}\input{|\textit{main}|}"|
\end{center}
%

%%%%%%%%%%%%%%%%%%%%%%%%%%%%%%%%%%%%%%%%%%%%%%%%%%%%%%%%%%%%%%%%%%%%%%%%%%%%%%%%
%%%%%%%%%%%%%%%%%%%%%%%%%%%%%%%%%%%%%%%%%%%%%%%%%%%%%%%%%%%%%%%%%%%%%%%%%%%%%%%%
\section{Information}

%%%%%%%%%%%%%%%%%%%%%%%%%%%%%%%%%%%%%%%%%%%%%%%%%%%%%%%%%%%%%%%%%%%%%%%%%%%%%%%%
\subsection{Copyright}

Copyright \copyright{} 2017--2018 Niklas Beisert

This work may be distributed and/or modified under the
conditions of the \LaTeX{} Project Public License, either version 1.3
of this license or (at your option) any later version.
The latest version of this license is in
  \url{http://www.latex-project.org/lppl.txt}
and version 1.3 or later is part of all distributions of \LaTeX{}
version 2005/12/01 or later.

This work has the LPPL maintenance status `maintained'.

The Current Maintainer of this work is Niklas Beisert.

This work consists of the files |README.txt|, |childdoc.ins| and |childdoc.dtx|
as well as the derived files |childdoc.def|, |cdocsamp.tex|
with |cdocsch1.tex|, |cdocsch2.tex|, |cdocspt3.tex|, |cdocspt4.tex|,
|cdocsdrf.tex|, |cdocsfn1.tex|, |cdocsfn2.tex|
as well as |childdoc.pdf|.

%%%%%%%%%%%%%%%%%%%%%%%%%%%%%%%%%%%%%%%%%%%%%%%%%%%%%%%%%%%%%%%%%%%%%%%%%%%%%%%%
\subsection{Files and Installation}

The package consists of the files:
%
\begin{center}
\begin{tabular}{ll}
    |README.txt|   & readme file \\
    |childdoc.ins| & installation file \\
    |childdoc.dtx| & source file \\
    |childdoc.def| & definition file \\
    |cdocsamp.tex| & sample main file \\
    |cdocsch1.tex| & sample include file \\
    |cdocsch2.tex| & sample include file \\
    |cdocspt3.tex| & sample part file \\
    |cdocspt4.tex| & sample part file \\
    |cdocsdrf.tex| & sample redirection file \\
    |cdocsfn1.tex| & sample redirection file \\
    |cdocsfn2.tex| & sample redirection file \\
    |childdoc.pdf| & manual
\end{tabular}
\end{center}
%
The distribution consists of the files
|README.txt|, |childdoc.ins| and |childdoc.dtx|.
%
\begin{itemize}
\item
Run (pdf)\LaTeX{} on |childdoc.dtx|
to compile the manual |childdoc.pdf| (this file).
\item
Run \LaTeX{} on |childdoc.ins| to create the definitions file |childdoc.def|
and the sample |cdocsamp.tex| with include files
|cdocsch1.tex|, |cdocsch2.tex|, |cdocspt3.tex|, |cdocspt4.tex|,
|cdocsdrf.tex|, |cdocsfn1.tex|, |cdocsfn2.tex|.
Then copy the file |childdoc.def| to an appropriate directory of your \LaTeX{}
distribution, e.g.\ \textit{texmf-root}|/tex/latex/childdoc|.
\end{itemize}

%%%%%%%%%%%%%%%%%%%%%%%%%%%%%%%%%%%%%%%%%%%%%%%%%%%%%%%%%%%%%%%%%%%%%%%%%%%%%%%%
\subsection{Related CTAN Packages}

There are several other packages which offer a similar functionality:
%
\begin{itemize}
\item
The packages
\href{http://ctan.org/pkg/docmute}{\textsf{docmute}},
\href{http://ctan.org/pkg/includex}{\textsf{includex}} and
\href{http://ctan.org/pkg/standalone}{\textsf{standalone}}
provide commands to include only the document body of
a child file thus allowing both files to be compiled individually.
\item
The packages \href{http://ctan.org/pkg/subdocs}{\textsf{subdocs}}
and \href{http://ctan.org/pkg/subfiles}{\textsf{subfiles}}
provide structures in which the main and child documents can be
encapsulated and allowing them to be compiled individually.
The inclusion mechanism is different from the conventional |\include|.
\item
The package \href{http://ctan.org/pkg/combine}{\textsf{combine}}
is an elaborate solution to combine several documents into one.
\end{itemize}
%
See also the CTAN topic \href{http://ctan.org/topic/subdocs}{\textsf{subdocs}}
for further related packages.
The present package differs from the above solutions in that
a document structure constructed with the conventional |\include| mechanism
just needs two extra commands at the top of every file
such that all constituent files can be compiled individually.

%%%%%%%%%%%%%%%%%%%%%%%%%%%%%%%%%%%%%%%%%%%%%%%%%%%%%%%%%%%%%%%%%%%%%%%%%%%%%%%%
%\subsection{Feature Suggestions}
%
%The following is a list of features which may be useful for future
%versions of this package:
%%
%\begin{itemize}
%\item
%\ldots
%\end{itemize}

%%%%%%%%%%%%%%%%%%%%%%%%%%%%%%%%%%%%%%%%%%%%%%%%%%%%%%%%%%%%%%%%%%%%%%%%%%%%%%%%
\subsection{Revision History}

%%%%%%%%%%%%%%%%%%%%%%%%%%%%%%%%%%%%%%%%
\paragraph{v2.0:} 2018/12/30

\begin{itemize}
\item
immediate forward processing
\item
added |\childdocby| mechanism
\item
manual restructured
\end{itemize}

%%%%%%%%%%%%%%%%%%%%%%%%%%%%%%%%%%%%%%%%
\paragraph{v1.6:} 2018/01/17

\begin{itemize}
\item
application for development of include files
\item
corrections to manual
\end{itemize}

%%%%%%%%%%%%%%%%%%%%%%%%%%%%%%%%%%%%%%%%
\paragraph{v1.5:} 2017/05/21

\begin{itemize}
\item
more complete structuring introduced
\item
|\childdocof| introduced
\item
|\childdoc| renamed to |\childdocmain|
\item
|\childredirect| renamed to |\childdocforward| and |\childdocforwardprefix|
and functionality expanded
\end{itemize}

%%%%%%%%%%%%%%%%%%%%%%%%%%%%%%%%%%%%%%%%
\paragraph{v1.0:} 2017/04/27

\begin{itemize}
\item
manual and install package
\item
first version published on CTAN
\end{itemize}

%%%%%%%%%%%%%%%%%%%%%%%%%%%%%%%%%%%%%%%%
\paragraph{v0.6:} 2017/04/26

\begin{itemize}
\item
redirection mechanism added
\end{itemize}

%%%%%%%%%%%%%%%%%%%%%%%%%%%%%%%%%%%%%%%%
\paragraph{v0.5:} 2017/04/26

\begin{itemize}
\item
functionality in definition file
\end{itemize}


%%%%%%%%%%%%%%%%%%%%%%%%%%%%%%%%%%%%%%%%%%%%%%%%%%%%%%%%%%%%%%%%%%%%%%%%%%%%%%%%
%%%%%%%%%%%%%%%%%%%%%%%%%%%%%%%%%%%%%%%%%%%%%%%%%%%%%%%%%%%%%%%%%%%%%%%%%%%%%%%%
%%%%%%%%%%%%%%%%%%%%%%%%%%%%%%%%%%%%%%%%%%%%%%%%%%%%%%%%%%%%%%%%%%%%%%%%%%%%%%%%
\appendix

\settowidth\MacroIndent{\rmfamily\scriptsize 000\ }

 \DocInput{childdoc.dtx}

\end{document}
%</driver>
% \fi
%
% %%%%%%%%%%%%%%%%%%%%%%%%%%%%%%%%%%%%%%%%%%%%%%%%%%%%%%%%%%%%%%%%%%%%%%%%%%%%%%
% %%%%%%%%%%%%%%%%%%%%%%%%%%%%%%%%%%%%%%%%%%%%%%%%%%%%%%%%%%%%%%%%%%%%%%%%%%%%%%
% \section{Sample}
%\iffalse
%<*samplemain>
%\fi
%
% The following presents a sample document
% with two chapters, two parts, a title page,
% a compile flag as well as three forwarding files to set the flag.
% It consists of eight |.tex| files:
% \begin{center}
% \begin{tabular}{ll}
% |cdocsamp.tex|&main file\\
% |cdocsch1.tex|&include file for chapter 1\\
% |cdocsch2.tex|&include file for chapter 2\\
% |cdocspt3.tex|&include file for part 3\\
% |cdocspt4.tex|&include file for part 4\\
% |cdocsdrf.tex|&forwarding file for main file in draft mode\\
% |cdocsfi1.tex|&forwarding file for final version of chapter 1\\
% |cdocsfi2.tex|&forwarding file for final version of chapter 2\\
% \end{tabular}
% \end{center}
% Each of the eight files can be compiled directly by the \LaTeX{} compiler.
%
% %%%%%%%%%%%%%%%%%%%%%%%%%%%%%%%%%%%%%%
% \paragraph{Main File.}
%
% The main file is called |cdocsamp.tex|.
%
% Load the \textsf{childdoc} definitions and
% declare the filename for the main document:
%    \begin{macrocode}
\input{childdoc.def}
\childdocmain{}
%    \end{macrocode}

% Optional override for |\version| flag:
%    \begin{macrocode}
%%\ifchilddoc\else\providecommand{\version}{draft}\fi
%    \end{macrocode}

% Define the default values for the |\version| flag
% (|final| for the main file and |draft| for childs):
%    \begin{macrocode}
\ifchilddoc
\providecommand{\version}{draft}
\else
\providecommand{\version}{final}
\fi
%    \end{macrocode}

% Load the standard document class:
%    \begin{macrocode}
\documentclass[12pt]{article}
%    \end{macrocode}

% Start the document body:
%    \begin{macrocode}
\begin{document}
%    \end{macrocode}

% Declare a title page.
% Print title, part of document being processed and version flag:
%    \begin{macrocode}
\addtocounter{page}{-1}
\begin{center}
{\LARGE\bfseries{}childdoc example\par}
\vspace{1cm}
\ifchilddoc
\ifchilddocmanual part\else chapter\fi:
`\childdocname' of `\childdocjob'\par
\else
main document: `\childdocjob'\par
\fi
version: \version\par
\end{center}
\newpage
%    \end{macrocode}

% Manually include selected file,
% otherwise process as usual:
%    \begin{macrocode}
\ifchilddocmanual
\section*{part `\childdocname'}
\input{\childdocname}
\else
%    \end{macrocode}

% Include the two chapters:
%    \begin{macrocode}
\include{cdocsch1}
\include{cdocsch2}
%    \end{macrocode}

% Include the two parts unless only chapters should be displayed:
%    \begin{macrocode}
\ifchilddoc\else
\section{part three}
\input{cdocspt3}
\section{part four}
\input{cdocspt4}
\fi
%    \end{macrocode}

% Process as usual until here:
%    \begin{macrocode}
\fi
%    \end{macrocode}

% End of document body:
%    \begin{macrocode}
\end{document}
%    \end{macrocode}
%\iffalse
%</samplemain>
%\fi
%
% %%%%%%%%%%%%%%%%%%%%%%%%%%%%%%%%%%%%%%
% \paragraph{Chapter Include Files.}
%
% The include files are called |cdocsch1.tex| and |cdocsch2.tex|.
%
%\iffalse
%<*samplechap1|samplechap2>
%\fi

% Optional override for |\version| flag:
%    \begin{macrocode}
%%\providecommand{\version}{final}
%    \end{macrocode}

% Include the main document:
%    \begin{macrocode}
\input{childdoc.def}
\childdocof{cdocsamp}
%    \end{macrocode}

%\iffalse
%</samplechap1|samplechap2>
%\fi
%
%\iffalse
%<*samplechap1>
%\fi
% Some text for chapter 1:
%    \begin{macrocode}
\section{one}
some text in chapter one
%    \end{macrocode}

%\iffalse
%</samplechap1>
%\fi
% Some text for chapter 2:
%\iffalse
%<*samplechap2>
%\fi
%    \begin{macrocode}
\section{two}
more text in chapter two
%    \end{macrocode}

%\iffalse
%</samplechap2>
%\fi
%
% %%%%%%%%%%%%%%%%%%%%%%%%%%%%%%%%%%%%%%
% \paragraph{Part Include Files.}
%
% The include files are called |cdocspt3.tex| and |cdocspt4.tex|.
%
%\iffalse
%<*samplepart3|samplepart4>
%\fi

% Optional override for |\version| flag:
%    \begin{macrocode}
%%\providecommand{\version}{final}
%    \end{macrocode}

% Include the main document:
%    \begin{macrocode}
\input{childdoc.def}
\childdocby{cdocsamp}
%    \end{macrocode}

%\iffalse
%</samplepart3|samplepart4>
%\fi
%
%\iffalse
%<*samplepart3>
%\fi
% Some text for part 3:
%    \begin{macrocode}
some text in part three
%    \end{macrocode}

%\iffalse
%</samplepart3>
%\fi
% Some text for part 4:
%\iffalse
%<*samplepart4>
%\fi
%    \begin{macrocode}
more text in part four
%    \end{macrocode}

%\iffalse
%</samplepart4>
%\fi
%
% %%%%%%%%%%%%%%%%%%%%%%%%%%%%%%%%%%%%%%
% \paragraph{Forwarding for a Complete Draft.}
%
% The following forwarding file |cdocsdrf.tex|
% compiles the main document in draft mode:
%\iffalse
%<*sampledraft>
%\fi
%    \begin{macrocode}
\def\version{draft}
\input{childdoc.def}
\childdocforward{cdocsamp}
%    \end{macrocode}

%\iffalse
%</sampledraft>
%\fi
%
% %%%%%%%%%%%%%%%%%%%%%%%%%%%%%%%%%%%%%%
% \paragraph{Forwarding for Final Version of the Chapters.}
%
% The following forwarding files |cdocsfn1.tex| and |cdocsfn2.tex|
% (with identical content)
% compile the final versions of the child documents
% |cdocsch1.tex| and |cdocsch2.tex|, respectively:
%\iffalse
%<*samplefinal>
%\fi
%    \begin{macrocode}
\def\version{final}
\input{childdoc.def}
\childdocforwardprefix[cdocsamp]{cdocsfn}{cdocsch}
%    \end{macrocode}

%\iffalse
%</samplefinal>
%\fi
%
% %%%%%%%%%%%%%%%%%%%%%%%%%%%%%%%%%%%%%%
% \paragraph{Command Line Processing.}
%
% The following three command lines generate the output files
% |cdocscld|, |cdocscl1| and |cdocscl2|
% which should be identical to
% |cdocsdrf|, |cdocsch1| and |cdocsfn2|, respectively:
% \begin{center}
% \begin{tabular}{l}
% |latex -jobname cdocscld \|\\
% |  "\def\version{draft}\input{childdoc.def}\childdocforward{cdocsamp}"|\\
% |latex -jobname cdocscl1 \|\\
% |  "\input{childdoc.def}\childdocforward[cdocsamp]{cdocsch1}"|\\
% |latex -jobname cdocscl2 \|\\
% |  "\def\version{final}\input{childdoc.def}\childdocforward{cdocsch2}"|
% \end{tabular}
% \end{center}
% Note that the trailing backslash on each first line
% merely continues the input to the second line
% (for convenient cut ant paste).
% Furthermore, the command |latex| can be replaced by any
% of its alternative versions such as |pdflatex|.
%
% %%%%%%%%%%%%%%%%%%%%%%%%%%%%%%%%%%%%%%%%%%%%%%%%%%%%%%%%%%%%%%%%%%%%%%%%%%%%%%
% %%%%%%%%%%%%%%%%%%%%%%%%%%%%%%%%%%%%%%%%%%%%%%%%%%%%%%%%%%%%%%%%%%%%%%%%%%%%%%
% \section{Implementation}
%\iffalse
%<*package>
%\fi
%
% This section describes the definitions file |childdoc.def|.

% The definitions cannot be loaded using |\usepackage| or |\RequirePackage|
% which has a mechanism to prevent loading a style file more than once.
% When loading the definitions by means of |\input|
% multiple instances have to be prevented manually:
%\iffalse
%This code needs to be before the `\ProvidesFile' directive
%which is defined at the beginning of this file.
%Therefore it is also placed there and commented out here.
%</package>
%<*discard>
%\fi
%    \begin{macrocode}
\ifdefined\childdocmain\endinput\fi
%    \end{macrocode}
%\iffalse
%</discard>
%<*package>
%\fi
%
% \macro{\ifchilddoc}
% \macro{\ifchilddocmanual}
% The conditional |\ifchilddoc| tells whether a
% child (true) or main (false) document is being compiled.
% The conditional |\ifchilddocmanual| tells whether
% the |\includeonly| mechanism is used (false) or
% the selection of child files must be performed manually (true).
% The definitions initialise to false:
%    \begin{macrocode}
\newif\ifchilddoc
\newif\ifchilddocmanual
%    \end{macrocode}

% \macro{\childdocname}
% \macro{\childdocjob}
% The macro |\childdocname| stores the name of the main document
% to be compiled. The macro |\childdocjob| stores the name of
% the document on which the \LaTeX{} compiler was originally invoked.
% The content of |\jobname| cannot be compared
% to filenames specified in the source due to different catcodes.
% The following code rescans |\jobname|, stores the result
% in |\childdocname| and saves a copy in |\childdocjob|:
%    \begin{macrocode}
\edef\childdocname{\scantokens\expandafter{\jobname\noexpand}}
\let\childdocjob\childdocname
%    \end{macrocode}

% \macro{\childdocdisable}
% The macro |\childdocdisable| prevents the main file
% from being processed more than once.
% At this stage, the main document command |\childdocmain|
% is assumed to be called once again where it should do nothing.
% Any subsequent call to it should prevent
% a secondary processing of the main document
% It overwrites the forwarding commands
% |\childdocof| and |\childdocforward|
% with empty macros to prevent further inclusions of the main document:
%    \begin{macrocode}
\newcommand{\childdocdisable}
{
  \renewcommand{\childdocmain}[1]{\renewcommand{\childdocmain}[1]{\endinput}}
  \renewcommand{\childdocof}[1]{}
  \renewcommand{\childdocby}[2][]{}
  \renewcommand{\childdocforward}[2][]{}
  \renewcommand{\childdocdisable}{}
}
%    \end{macrocode}

% \macro{\childdocmain}
% The macro |\childdocmain| is to be called at the top of the main file
% with nothing or the main filename (without extension) as argument.
% First, it breaks loops.
% If the argument is not empty and does not match |\childdocname|
% (which is set by the first inclusion of |childdoc.def|),
% |\ifchilddoc| is set to true, |\includeonly| is applied to the child file
% and |\jobname| is set to the main file
% (for proper handling of |.aux| files):
%    \begin{macrocode}
\newcommand{\childdocmain}[1]
{
  \childdocdisable\childdocmain{}
  \if?#1?\else
    \begingroup
      \def\childdoctmp{#1}
      \ifx\childdoctmp\childdocname
        \def\childdoctmp{}
      \else
        \def\childdoctmp
        {
          \childdoctrue
          \includeonly{\childdocname}
          \def\childdocjob{#1}
          \def\jobname{#1}
        }
      \fi
      \expandafter
    \endgroup
    \childdoctmp
  \fi
}
%    \end{macrocode}

% \macro{\childdocof}
% The command |\childdocof| redirects
% compilation to the main file |#1|.
%    \begin{macrocode}
\newcommand{\childdocof}[1]
{
  \childdocdisable
  \childdoctrue
  \includeonly{\childdocname}
  \def\jobname{#1}
  \def\childdocjob{#1}
  \input{#1}
}
%    \end{macrocode}

% \macro{\childdocby}
% The command |\childdocby| ....
%    \begin{macrocode}
\newcommand{\childdocby}[2][]
{
  \childdocdisable
  \childdoctrue
  \childdocmanualtrue
  \if?#1?\else
    \def\jobname{#2}
  \fi
  \def\childdocjob{#2}
  \input{#2}
  \endinput
}
%    \end{macrocode}

% \macro{\childdocforward}
% The command |\childdocforward| redirects
% compilation to the main file or
% (if the optional argument is given) a child file.
% Parameters are set as if the main file
% or a child file starting with |\childdocof| was compiled.
% Then compilation is handed over to the main file:
%    \begin{macrocode}
\newcommand{\childdocforward}[2][]
{
  \begingroup
    \if?#1?
      \def\childdoctmp
      {
        \def\childdocname{#2}
        \def\childdocjob{#2}
        \def\jobname{#2}
        \input{#2}
        \endinput
      }
    \else
      \def\childdoctmp
      {
        \childdocdisable
        \def\childdocname{#2}
        \childdoctrue
        \includeonly{#2}
        \def\childdocjob{#1}
        \def\jobname{#1}
        \input{#1}
        \endinput
      }
    \fi
    \expandafter
  \endgroup
  \childdoctmp
}
%    \end{macrocode}

% \macro{\childdocforwardprefix}
% The command |\childdocforwardprefix| redirects
% compilation to the main or a child file by means of a pattern.
% The prefix |#1| in the current filename is replaced by |#2|
% and the suffix of the current filename is kept
% (it is assumed that the filename does not contain the substring `|~~~|'
% which is used as a delimiter).
% Compilation is handed over to the new file by |\childdocforward|:
%    \begin{macrocode}
\newcommand{\childdocforwardprefix}[3][]
{
  \begingroup
    \def\childdocextract #2##1~~~{\def\childdoctmp{\childdocforward[#1]{#3##1}}}
    \expandafter\childdocextract\childdocname~~~
    \expandafter
  \endgroup
  \childdoctmp
}
%    \end{macrocode}

% \macro{\childdoc}
% The deprecated macro |\childdoc| is a legacy version of |\childdocmain|:
%    \begin{macrocode}
\newcommand{\childdoc}{\childdocmain}
%    \end{macrocode}

% \macro{\childdocredirect}
% The deprecated macro |\childdocredirect| is a legacy version
% of |\childdocforward| and |\childdocforwardprefix|:
%    \begin{macrocode}
\newcommand{\childdocredirect}[2][]
{
  \begingroup
    \if?#1?
      \def\childdoctmp{\childdocforward{#2}}
    \else
      \def\childdoctmp{\childdocforwardprefix{#1}{#2}}
    \fi
    \expandafter
  \endgroup
  \childdoctmp
}
%    \end{macrocode}

%\iffalse
%</package>
%\fi
%
\endinput
\childdocforward{cdocsamp}"|\\
% |latex -jobname cdocscl1 \|\\
% |  "% \iffalse
%
% childdoc.dtx Copyright (C) 2017-2018 Niklas Beisert
%
% This work may be distributed and/or modified under the
% conditions of the LaTeX Project Public License, either version 1.3
% of this license or (at your option) any later version.
% The latest version of this license is in
%   http://www.latex-project.org/lppl.txt
% and version 1.3 or later is part of all distributions of LaTeX
% version 2005/12/01 or later.
%
% This work has the LPPL maintenance status `maintained'.
%
% The Current Maintainer of this work is Niklas Beisert.
%
% This work consists of the files childdoc.dtx and childdoc.ins
% and the derived files childdoc.def and cdocsamp.tex with
% cdocsch1.tex, cdocsch2.tex, cdocsdrf.tex, cdocsfn1.tex, cdocsfn2.tex.
%
%<package>\ifdefined\childdocmain\endinput\fi
%<package>\ProvidesFile{childdoc.def}[2018/12/30 v2.0 child document driver]
%<samplemain>\ProvidesFile{cdocsamp.tex}[2018/12/30 v2.0 sample for childdoc]
%<*driver>
%\ProvidesFile{childdoc.drv}[2018/12/30 v2.0 childdoc reference manual file]
\PassOptionsToClass{10pt,a4paper}{article}
\documentclass{ltxdoc}

\usepackage[margin=35mm]{geometry}
\usepackage{hyperref}
\usepackage{hyperxmp}
\usepackage[usenames]{color}

\hypersetup{colorlinks=true}
\hypersetup{pdfstartview=FitH}
\hypersetup{pdfpagemode=UseNone}
\hypersetup{pdfsource={}}
\hypersetup{pdflang={en-UK}}
\hypersetup{pdfcopyright={Copyright 2017-2018 Niklas Beisert.
  This work may be distributed and/or modified under the
  conditions of the LaTeX Project Public License, either version 1.3
  of this license or (at your option) any later version.}}
\hypersetup{pdflicenseurl={http://www.latex-project.org/lppl.txt}}
\hypersetup{pdfcontactaddress={ETH Zurich, ITP, HIT K,
  Wolfgang-Pauli-Strasse 27}}
\hypersetup{pdfcontactpostcode={8093}}
\hypersetup{pdfcontactcity={Zurich}}
\hypersetup{pdfcontactcountry={Switzerland}}
\hypersetup{pdfcontactemail={nbeisert@itp.phys.ethz.ch}}
\hypersetup{pdfcontacturl={http://people.phys.ethz.ch/\xmptilde nbeisert/}}

\newcommand{\secref}[1]{\hyperref[#1]{section \ref*{#1}}}

\parskip1ex
\parindent0pt
\let\olditemize\itemize
\def\itemize{\olditemize\parskip0pt}

\begin{document}

\title{The \textsf{childdoc} Package}
\hypersetup{pdftitle={The childdoc Package}}
\author{Niklas Beisert\\[2ex]
  Institut f\"ur Theoretische Physik\\
  Eidgen\"ossische Technische Hochschule Z\"urich\\
  Wolfgang-Pauli-Strasse 27, 8093 Z\"urich, Switzerland\\[1ex]
  \href{mailto:nbeisert@itp.phys.ethz.ch}
  {\texttt{nbeisert@itp.phys.ethz.ch}}}
\hypersetup{pdfauthor={Niklas Beisert}}
\hypersetup{pdfsubject={Manual for the LaTeX2e Package childdoc}}
\date{30 December 2018, \textsf{v2.0}}
\maketitle

\begin{abstract}\noindent
\textsf{childdoc} is a \LaTeXe{} package
that enables the direct compilation
of document sections included by |\include|
to individual files.
\end{abstract}

\begingroup
\parskip0ex
\tableofcontents
\endgroup

%%%%%%%%%%%%%%%%%%%%%%%%%%%%%%%%%%%%%%%%%%%%%%%%%%%%%%%%%%%%%%%%%%%%%%%%%%%%%%%%
%%%%%%%%%%%%%%%%%%%%%%%%%%%%%%%%%%%%%%%%%%%%%%%%%%%%%%%%%%%%%%%%%%%%%%%%%%%%%%%%
\section{Introduction}

\LaTeX{} provides a mechanism to structure a large document (such as a book)
into a main file and several child files (containing the chapters)
using the |\include| command.
This mechanism is beneficial for documents
which span hundreds of pages in order to
make the source file(s) more manageable.
Moreover, compilation can be restricted to
selected child files by means of the |\includeonly| command.
The latter feature can be used to reduce the compilation time while editing
(this was significantly more useful in the earlier days of \LaTeX{})
or to generate a smaller document which is easier to navigate.
Another application of |\includeonly| is to generate
documents consisting of selected parts of the complete document.

However, there are a few drawbacks of the plain |\include| mechanism:
\begin{itemize}
\item
The child files cannot be compiled on their own,
they can only be compiled via the main file.
A naive editing environment
(such as a text editor with an option
to have the current file processed by \LaTeX)
may require one to switch to the main file before compiling;
attempting to compile the child file produces errors.
\item
The main file must be modified (each time)
to adjust the |\includeonly| command
to the present needs. This easily leaves the main file in a messy state.
\item
The generated document will always carry the filename
of the main document. This is inconvenient if
several child files are to be compiled and
to be kept for distribution.
\end{itemize}

The present package provides a simple interface
to make child files individually compilable by \LaTeX{}.
Compiling a child file then has the same effect as compiling
the main file with an |\includeonly| command
to select the appropriate child.
Moreover the generated document will carry the name of the child
rather than the main file.
This resolves all three above issues.

This feature is meant to make the editing of books,
thesis documents and lecture notes somewhat more convenient.
However, the package can also be used efficiently for
composing a series of documents (such as exercise sheets)
which are typically distributed individually.
It then assists the author in generating the individual documents
(potentially in different versions)
as well as a document containing the collected series.
Another application is in developing style files
or other kinds of included material
where compilation of the style file could redirect
to a sample or test file.

%%%%%%%%%%%%%%%%%%%%%%%%%%%%%%%%%%%%%%%%%%%%%%%%%%%%%%%%%%%%%%%%%%%%%%%%%%%%%%%%
%%%%%%%%%%%%%%%%%%%%%%%%%%%%%%%%%%%%%%%%%%%%%%%%%%%%%%%%%%%%%%%%%%%%%%%%%%%%%%%%
\section{Usage}

First of all, the package \textsf{childdoc} is \emph{not} a standard
\LaTeXe{} |.sty| style file! Therefore it needs to be invoked in
a non-standard way.

%%%%%%%%%%%%%%%%%%%%%%%%%%%%%%%%%%%%%%%%%%%%%%%%%%%%%%%%%%%%%%%%%%%%%%%%%%%%%%%%
\subsection{Included Files}
\label{sec:include}

%%%%%%%%%%%%%%%%%%%%%%%%%%%%%%%%%%%%%%%%
\DescribeMacro{\childdocmain}
To use the package, add the commands
\begin{center}
\begin{tabular}{l}
|\input{childdoc.def}|\\
|\childdocmain{}|\\
\end{tabular}
\end{center}
at the very top of the main \LaTeX{} file,
in particular \emph{before} the |\documentclass| statement!
The argument of |\childdocmain| should be left empty
(but it must be present).

%%%%%%%%%%%%%%%%%%%%%%%%%%%%%%%%%%%%%%%%
\DescribeMacro{\childdocof}
Furthermore, add the commands
\begin{center}
\begin{tabular}{l}
|\input{childdoc.def}|\\
|\childdocof{|\textit{main}|}|\\
\end{tabular}
\end{center}
at the top of every child file \textit{child}
which is included by |\include{|\textit{child}|}|
from within the main file
(or at least for those files to be compiled individually).
The argument \textit{main} must be the filename of the main file.

There are a couple of
considerations in setting up the main and child documents:

%%%%%%%%%%%%%%%%%%%%%%%%%%%%%%%%%%%%%%%%
\paragraph{Restrictions.}

Please note the following restrictions:
\begin{itemize}
\item
|\childdocmain| must be called with one argument \textit{main}
to ensure compatibility with earlier version of the package.
It must either be empty (|\childdocmain{}|)
or precisely match the filename of the main file in which it is specified.
See \secref{sec:detection} for further information.
\item
The filename \textit{main} must be specified without the |.tex| extension.
\item
The filename \textit{main} is case sensitive
(even in case-insensitive file systems)
due to internal string comparison.
\item
The argument \textit{main} should be fully expanded, it cannot be a macro.
\item
Subdirectories and special characters should be avoided in filenames.
\item
The command |\childdocmain{|\textit{main}|}| must be followed by a whitespace.
It should not be followed immediately by another command
or by a comment mark `|%|'.
This is because the \TeX{} parser reads the token immediately following
the argument of |\childdocmain| and puts it
at the beginning of every child section;
however, a white\-space is ignored.
\end{itemize}

%%%%%%%%%%%%%%%%%%%%%%%%%%%%%%%%%%%%%%%%
\paragraph{Content of Main File.}

It is advisable to place all content in the child files included by |\include|.
Any output contained in the main file will appear in all child documents
unless suppressed manually;
it cannot be suppressed automatically by the |\includeonly| directive
and thus should normally be avoided.
A method to include some content in the main file
by means of conditional processing is described in \secref{sec:conditional}.

%%%%%%%%%%%%%%%%%%%%%%%%%%%%%%%%%%%%%%%%
\paragraph{Page Numbering.}

When only a part of the document is compiled,
the appropriate numbering of pages
(as well as other status parameters)
is determined from the |.aux| files.
The latter contain information from previous passes.
However this information needs to propagate through
all intermediate child documents.
Therefore the page numbering in child documents may well
be inconsistent until the complete document is compiled at least once.

A useful (if unconventional) way to always ensure a consistent
page numbering is to restart the numbering in each child document
and denote the pages by `\textit{child}|.|\textit{page}'
where \textit{child} represents the chapter/section number of the child file.
This can be achieved by the command
|\numberwithin{page}{|\textit{child}|}|
of the \textsf{amsmath} package
where \textit{child} can be |chapter| or |section|
depending on the chosen structuring.
Alternatively, one can modify the macro |\thepage| appropriately
and reset the counter |page| at the start of each child file.

%%%%%%%%%%%%%%%%%%%%%%%%%%%%%%%%%%%%%%%%%%%%%%%%%%%%%%%%%%%%%%%%%%%%%%%%%%%%%%%%
\subsection{Conditional Processing}
\label{sec:conditional}

The package provides a mechanism to compile different versions
of a document. To customise the versions further some conditional processing
can come in handy to distinguish which version is being compiled.
The package provides two macros to describe the compilation context:

%%%%%%%%%%%%%%%%%%%%%%%%%%%%%%%%%%%%%%%%
\DescribeMacro{\ifchilddoc}
The conditional |\ifchilddoc| distinguishes between the compilation of
child documents and the main document:
%
\begin{center}
|\ifchilddoc |\textit{child-code}| |[|\||else |\textit{main-code}]| \||fi|
\end{center}

%%%%%%%%%%%%%%%%%%%%%%%%%%%%%%%%%%%%%%%%
\DescribeMacro{\childdocname}
\DescribeMacro{\childdocjob}
The macro |\childdocname| contains the filename (without extension)
of the main or child file being processed.
Note that |\childdocjob| will always contain the name of the main file.

%%%%%%%%%%%%%%%%%%%%%%%%%%%%%%%%%%%%%%%%
\paragraph{Title Page.}

Conditional processing can be used to include a title or banner page
in the main document when proper precautions are taken.
Importantly, the code in the main file should ensure that the page counter
(as well as other status parameters which are stored in the |.aux| files)
takes the same value after the conditional processing.
Otherwise the page numbers may take divergent values
depending on which part is compiled.

For example, a title page could be declared by:
%
\begin{center}
\begin{tabular}{l}
|\ifchilddoc\||else|\\
|\addtocounter{page}{-1}|\\
\textit{code for title page}\\
|\newpage|\\
|\||fi|
\end{tabular}
\end{center}
%
A banner page for the child documents can be generated by:
%
\begin{center}
\begin{tabular}{l}
|\ifchilddoc|\\
|\addtocounter{page}{-1}|\\
\textit{code for banner page}\\
|\newpage|\\
|\||fi|
\end{tabular}
\end{center}
%
Here one could write a message such as:
\begin{center}
|This is the part \childdocname{} of \childdocjob{}.|
\end{center}

%%%%%%%%%%%%%%%%%%%%%%%%%%%%%%%%%%%%%%%%%%%%%%%%%%%%%%%%%%%%%%%%%%%%%%%%%%%%%%%%
\subsection{Flags}
\label{sec:flags}

The package makes it easy to generate different versions
of the main or child documents.
To this end compilation flags can be defined
and assigned different default values.
They will be particularly useful in conjunction
with the forwarding mechanism described in \secref{sec:forward}.

For example, it may be useful to have a flag |\version|
which can be set to |draft| or |final|.
The document source will contain some conditional code
depending on the value of |\version|.
Suppose further, the flag should default to |final| for the main file
and to |draft| for child files
which is a natural assignment for editing the document.
This is achieved by placing the following code
in the preamble of the main document
(below the |\childdocmain| directive):
%
\begin{center}
\begin{tabular}{l}
|\ifchilddoc|\\
|\providecommand{\version}{draft}|\\
|\||else|\\
|\providecommand{\version}{final}|\\
|\||fi|
\end{tabular}
\end{center}
%
The definition by |\providecommand| makes sure
that previous definitions are not overwritten.
Further statements |\providecommand{\version}{...}|
can thus be added before the above code to override it.

For the main file, one might add a line
(between |\childdocmain| and the above block)
%
\begin{center}
|%\ifchilddoc\||else\providecommand{\version}{draft}\||fi|
\end{center}
%
which can be uncommented to produce a draft version.
Likewise one can add a line to the very top of a child file
(above the |\childdocof{|\textit{main}|}| directive)
%
\begin{center}
|%\providecommand{\version}{final}|
\end{center}
%
which can be uncommented to produce the final version of this child document.

%%%%%%%%%%%%%%%%%%%%%%%%%%%%%%%%%%%%%%%%%%%%%%%%%%%%%%%%%%%%%%%%%%%%%%%%%%%%%%%%
\subsection{Forwarding}
\label{sec:forward}

Different versions of the main or child documents
using compilation flags as described in \secref{sec:flags}
can be (permanently) stored in different files
for convenient compilation, viewing and distribution.
To this end, the package defines a command
to pass on compilation to a different file:

%%%%%%%%%%%%%%%%%%%%%%%%%%%%%%%%%%%%%%%%
\DescribeMacro{\childdocforward}
The command |\childdocforward| redirects processing to
another source file:
%
\begin{center}
\begin{tabular}{l}
|\input{childdoc.def}|\\
|\childdocforward[|\textit{main}|]{|\textit{dest}|}|\\
\end{tabular}
\end{center}
%
The argument \textit{dest} is the destination file
(without extension).
It should be the main file or one of the child files.
Note that further \textsf{childdoc} directives
such as |\childdocof| and |\childdocforward|
in the indicated file will be processed in this form.
The optional argument \textit{main}
passes on directly to the main file \textit{main}
while pretending to compile the child \textit{dest}.
This form behaves as if \textit{dest}
issues |\childdocof{|\textit{main}|}| right away,
and no further \textsf{childdoc} directives will be processed.

%%%%%%%%%%%%%%%%%%%%%%%%%%%%%%%%%%%%%%%%
\DescribeMacro{\...prefix}
In the alternative form |\childdocforwardprefix|,
%
\begin{center}
\begin{tabular}{l}
|\input{childdoc.def}|\\
|\childdocforwardprefix[|\textit{main}|]{|\textit{prefix}|}{|\textit{dest}|}|
\end{tabular}
\end{center}
%
the destination file is determined by a pattern
depending on the current file:
To make this work, the current file must be called
`{\textit{prefix}\hspace{0.2em}\textit{suffix}}'
with \textit{prefix} matching precisely the argument.
Processing is then passed on to the file
`{\textit{dest}\hspace{0.2em}\textit{suffix}}'.
Surely, the same effect is achieved by
directly specifying the
argument `{\textit{dest}\hspace{0.2em}\textit{suffix}}'
in the first form.
However, that requires to set up a different file
for each child. With the alternative form of the command
all these files can have exactly the same content
which simplifies setting them up and maintaining them.

For example, the following file |draft.tex|
with a compilation flag |\version| as described in \secref{sec:flags}
compiles the main document as a draft:
%
\begin{center}
\begin{tabular}{l}
|\def\version{draft}|\\
|\input{childdoc.def}|\\
|\childdocforward{|\textit{main}|}|
\end{tabular}
\end{center}
%
Likewise, the following files |final|\textit{nn}|.tex|
compile the final version of the child document
|child|\textit{nn}|.tex|:
%
\begin{center}
\begin{tabular}{l}
|\def\version{final}|\\
|\input{childdoc.def}|\\
|\childdocforwardprefix{final}{child}|
\end{tabular}
\end{center}
%

Note that when several versions of a main file and/or of each child file
are to be generated, it may be convenient to set up a |Makefile| or
shell script to automatise the process.

%%%%%%%%%%%%%%%%%%%%%%%%%%%%%%%%%%%%%%%%%%%%%%%%%%%%%%%%%%%%%%%%%%%%%%%%%%%%%%%%
\subsection{Command Line Processing}
\label{sec:commandline}

The effect of redirection files can also be achieved by invoking
the \LaTeX{} compiler with a more elaborate command line.
Most conveniently this should be done as part
of a shell script or a |Makefile|.

When using \textsf{childdoc} in the main file, the following
command lines effectively perform a redirection
(note that depending on the shell being used,
backslashes may have to be doubled: `|\|' $\to$ `|\\|'):
%
\begin{center}
|... -jobname "|\textit{target}|" |\\|"|[\textit{flags}]%
|\input{childdoc.def}\childdocforward[|\textit{main}|]{|\textit{dest}|}"|
\end{center}
%
Here \textit{target} is the name of the output file,
\textit{main} is the name of the main file
and \textit{dest} is the name of the main or child file to be processed
(all filenames without extensions).
The optional argument \textit{main} can be omitted
if \textit{main} matches \textit{dest}.
Optionally, compilation \textit{flags} can be defined via |\def| commands.
This command line makes the \TeX{} engine believe
it is compiling the file \textit{target}
whose content is specified as the latter parameter.
The provided code then forwards the processing to
\textit{main} or \textit{dest} as described in \secref{sec:forward}.

%%%%%%%%%%%%%%%%%%%%%%%%%%%%%%%%%%%%%%%%%%%%%%%%%%%%%%%%%%%%%%%%%%%%%%%%%%%%%%%%
\subsection{Include by Input}
\label{sec:input}

Including child documents by |\include| has some restrictions by design.
Most notably, the content of a child document always occupies
its own set of pages; pages cannot be shared between child documents.
Usually, this behaviour makes perfect sense
because each child document contain an essential part of the document.
However, in some situations it may be desirable to compose
a document from a collection of parts
without having mandatory page breaks between then.
For this case, the package
provides a mechanism to include parts
by |\input| which can also be processed individually.
However, by construction this mechanism
requires manual handling of the content to be output.

%%%%%%%%%%%%%%%%%%%%%%%%%%%%%%%%%%%%%%%%
\DescribeMacro{\ifchilddocmanual}
The main file should be prepared as usual, see \secref{sec:include}.
However, the document body must make a distinction
between processing of an individual part and of the main document, e.g.:
%
\begin{center}
\begin{tabular}{l}
|\ifchilddocmanual|\\
|\input{\childdocname}|\\
|\||else|\\
\textit{document body with }|\input{|\textit{part}|}|\\
|\||fi|
\end{tabular}
\end{center}
%
The conditional |\ifchilddocmanual| is true whenever
a part to be included by |\input| is being compiled,
and the name of the part is stored in |\childdocname|.

%%%%%%%%%%%%%%%%%%%%%%%%%%%%%%%%%%%%%%%%
\DescribeMacro{\childdocby}
Each part to be included by |\input| should start with:
%
\begin{center}
\begin{tabular}{l}
|\input{childdoc.def}|\\
|\childdocby{|\textit{main}|}|\\
\end{tabular}
\end{center}
%
The directive |\childdocby| is similar to |\childdocof|
described in \secref{sec:include},
but the subsequent selection of content must be done manually.
To that end, both |\ifchilddoc| and |\ifchilddocmanual|
will be true upon processing of a part,
and the name of the part is stored in |\childdocname|.
Note that |\jobname| will be set to the filename of the current part
so that each part receives an individual |.aux| file
that does not interfere with the |.aux| file(s) of the main document.
This behaviour can be altered by the alternative form
|\childdocby[*]{|\textit{main}|}| (with a non-empty optional argument)
which uses the |.aux| file of the main document
by setting |\jobname| to \textit{main}.

%%%%%%%%%%%%%%%%%%%%%%%%%%%%%%%%%%%%%%%%%%%%%%%%%%%%%%%%%%%%%%%%%%%%%%%%%%%%%%%%
\subsection{Driver Development}
\label{sec:driver}

The \textsf{childdoc} mechanism can also be use for the development
of definition files such as \LaTeX{} styles or classes.
This case differs from the above setup with multiple parts
included by |\include| in that no |\includeonly| should be invoked.
This can be achieved by starting the include file
(before |\ProvidesPackage|) with:
%
\begin{center}
\begin{tabular}{l}
|\input{childdoc.def}|\\
|\childdocforward{|\textit{main}|}|\\
\end{tabular}
\end{center}
%
or alternatively with:
%
\begin{center}
\begin{tabular}{l}
|\input{childdoc.def}|\\
|\childdocby{|\textit{main}|}|\\
\end{tabular}
\end{center}
%
Both forms have slightly different effects as described above.
The main file is prepared as usual, see \secref{sec:include}.

%%%%%%%%%%%%%%%%%%%%%%%%%%%%%%%%%%%%%%%%%%%%%%%%%%%%%%%%%%%%%%%%%%%%%%%%%%%%%%%%
\subsection{Legacy Detection}
\label{sec:detection}

The directive |\childdocmain| in the main file can detect
whether the complete document or merely a child is to be compiled
even without using the directive |\childdocof|.
This method is deprecated because it is less robust
and there is no compelling reason to use it;
it is merely provided for backward compatibility
and it may be removed in future versions.

If the detection mechanism is to be used,
it is mandatory to correctly specify
the filename of the main file as the argument of |\childdocmain|:
%
\begin{center}
\begin{tabular}{l}
|\input{childdoc.def}|\\
|\childdocmain{|\textit{main}|}|\\
\end{tabular}
\end{center}
%
If |\jobname| does not match the argument \textit{main} of |\childdocmain|,
it is assumed that |\jobname| points to the child file to be compiled.
When using |\childdocmain| with the main file specified as argument,
it suffices to start a child file
with just |\input{|\textit{main}|}|
without loading of the package and using |\childdocof|.
If instead all processing is done
with the appropriate \textsf{childdoc} directives,
the argument of \textit{main} of |\childdocmain| can be empty.

An alternative version of the command line processing described
in \secref{sec:commandline} using the detection mechanism reads:
%
\begin{center}
|... -jobname "|\textit{target}|" "|[\textit{flags}]%
[|\def\jobname{|\textit{dest}|}|]|\input{|\textit{main}|}"|
\end{center}

%%%%%%%%%%%%%%%%%%%%%%%%%%%%%%%%%%%%%%%%%%%%%%%%%%%%%%%%%%%%%%%%%%%%%%%%%%%%%%%%
\subsection{Manual Code}
\label{sec:manual}

In case one cannot be certain whether the definitions file |childdoc.def|
is installed on the target \TeX{} distribution
and one prefers not to ship it,
it is conceivable to paste a few relevant commands into the sources.

To that end, drop all statements |\input{childdoc.def}|
and perform the replacements as outlined below.
Instead of |\childdocmain{|\textit{main}|}| add the following code
to the top of the main file:
%
\begin{center}
\begin{tabular}{l}
|\||ifdefined\childdocname\endinput\||fi\newif\ifchilddoc|\\
|\edef\childdocname{\scantokens\expandafter{\jobname\noexpand}}|\\
|\def\childdocmain{|\textit{main}|}\||ifx\childdocmain\childdocname\||else|\\
|\childdoctrue\includeonly{\childdocname}\let\jobname\childdocmain\||fi|\\
\end{tabular}
\end{center}
%
Instead of |\childdocof{|\textit{main}|}| just include the main file
at the top of each child file:
%
\begin{center}
|\input{|\textit{main}|}|
\end{center}
%
A simple redirection |\childdocforward{|\textit{dest}|}| is achieved by:
%
\begin{center}
|\def\jobname{|\textit{dest}|}\input{\jobname}|
\end{center}
%
The redirection with prefix
|\childdocforwardprefix[|\textit{prefix}|]{|\textit{dest}|}|
is accomplished by:
%
\begin{center}
\begin{tabular}{l}
|{\edef\jobname{\scantokens\expandafter{\jobname\noexpand}}|\\
|\def\redirectjob |\textit{prefix}|#1~~~{\gdef\jobname{|\textit{dest}|#1}}|\\
|\expandafter\redirectjob\jobname~~~}\input{\jobname}|
\end{tabular}
\end{center}

In an alternative approach,
child documents can be compiled by a specific command line
without additional code or specific definitions:
%
\begin{center}
|... -jobname "|\textit{target}|" "|[\textit{flags}]%
|\includeonly{|\textit{dest}|}\input{|\textit{main}|}"|
\end{center}
%

%%%%%%%%%%%%%%%%%%%%%%%%%%%%%%%%%%%%%%%%%%%%%%%%%%%%%%%%%%%%%%%%%%%%%%%%%%%%%%%%
%%%%%%%%%%%%%%%%%%%%%%%%%%%%%%%%%%%%%%%%%%%%%%%%%%%%%%%%%%%%%%%%%%%%%%%%%%%%%%%%
\section{Information}

%%%%%%%%%%%%%%%%%%%%%%%%%%%%%%%%%%%%%%%%%%%%%%%%%%%%%%%%%%%%%%%%%%%%%%%%%%%%%%%%
\subsection{Copyright}

Copyright \copyright{} 2017--2018 Niklas Beisert

This work may be distributed and/or modified under the
conditions of the \LaTeX{} Project Public License, either version 1.3
of this license or (at your option) any later version.
The latest version of this license is in
  \url{http://www.latex-project.org/lppl.txt}
and version 1.3 or later is part of all distributions of \LaTeX{}
version 2005/12/01 or later.

This work has the LPPL maintenance status `maintained'.

The Current Maintainer of this work is Niklas Beisert.

This work consists of the files |README.txt|, |childdoc.ins| and |childdoc.dtx|
as well as the derived files |childdoc.def|, |cdocsamp.tex|
with |cdocsch1.tex|, |cdocsch2.tex|, |cdocspt3.tex|, |cdocspt4.tex|,
|cdocsdrf.tex|, |cdocsfn1.tex|, |cdocsfn2.tex|
as well as |childdoc.pdf|.

%%%%%%%%%%%%%%%%%%%%%%%%%%%%%%%%%%%%%%%%%%%%%%%%%%%%%%%%%%%%%%%%%%%%%%%%%%%%%%%%
\subsection{Files and Installation}

The package consists of the files:
%
\begin{center}
\begin{tabular}{ll}
    |README.txt|   & readme file \\
    |childdoc.ins| & installation file \\
    |childdoc.dtx| & source file \\
    |childdoc.def| & definition file \\
    |cdocsamp.tex| & sample main file \\
    |cdocsch1.tex| & sample include file \\
    |cdocsch2.tex| & sample include file \\
    |cdocspt3.tex| & sample part file \\
    |cdocspt4.tex| & sample part file \\
    |cdocsdrf.tex| & sample redirection file \\
    |cdocsfn1.tex| & sample redirection file \\
    |cdocsfn2.tex| & sample redirection file \\
    |childdoc.pdf| & manual
\end{tabular}
\end{center}
%
The distribution consists of the files
|README.txt|, |childdoc.ins| and |childdoc.dtx|.
%
\begin{itemize}
\item
Run (pdf)\LaTeX{} on |childdoc.dtx|
to compile the manual |childdoc.pdf| (this file).
\item
Run \LaTeX{} on |childdoc.ins| to create the definitions file |childdoc.def|
and the sample |cdocsamp.tex| with include files
|cdocsch1.tex|, |cdocsch2.tex|, |cdocspt3.tex|, |cdocspt4.tex|,
|cdocsdrf.tex|, |cdocsfn1.tex|, |cdocsfn2.tex|.
Then copy the file |childdoc.def| to an appropriate directory of your \LaTeX{}
distribution, e.g.\ \textit{texmf-root}|/tex/latex/childdoc|.
\end{itemize}

%%%%%%%%%%%%%%%%%%%%%%%%%%%%%%%%%%%%%%%%%%%%%%%%%%%%%%%%%%%%%%%%%%%%%%%%%%%%%%%%
\subsection{Related CTAN Packages}

There are several other packages which offer a similar functionality:
%
\begin{itemize}
\item
The packages
\href{http://ctan.org/pkg/docmute}{\textsf{docmute}},
\href{http://ctan.org/pkg/includex}{\textsf{includex}} and
\href{http://ctan.org/pkg/standalone}{\textsf{standalone}}
provide commands to include only the document body of
a child file thus allowing both files to be compiled individually.
\item
The packages \href{http://ctan.org/pkg/subdocs}{\textsf{subdocs}}
and \href{http://ctan.org/pkg/subfiles}{\textsf{subfiles}}
provide structures in which the main and child documents can be
encapsulated and allowing them to be compiled individually.
The inclusion mechanism is different from the conventional |\include|.
\item
The package \href{http://ctan.org/pkg/combine}{\textsf{combine}}
is an elaborate solution to combine several documents into one.
\end{itemize}
%
See also the CTAN topic \href{http://ctan.org/topic/subdocs}{\textsf{subdocs}}
for further related packages.
The present package differs from the above solutions in that
a document structure constructed with the conventional |\include| mechanism
just needs two extra commands at the top of every file
such that all constituent files can be compiled individually.

%%%%%%%%%%%%%%%%%%%%%%%%%%%%%%%%%%%%%%%%%%%%%%%%%%%%%%%%%%%%%%%%%%%%%%%%%%%%%%%%
%\subsection{Feature Suggestions}
%
%The following is a list of features which may be useful for future
%versions of this package:
%%
%\begin{itemize}
%\item
%\ldots
%\end{itemize}

%%%%%%%%%%%%%%%%%%%%%%%%%%%%%%%%%%%%%%%%%%%%%%%%%%%%%%%%%%%%%%%%%%%%%%%%%%%%%%%%
\subsection{Revision History}

%%%%%%%%%%%%%%%%%%%%%%%%%%%%%%%%%%%%%%%%
\paragraph{v2.0:} 2018/12/30

\begin{itemize}
\item
immediate forward processing
\item
added |\childdocby| mechanism
\item
manual restructured
\end{itemize}

%%%%%%%%%%%%%%%%%%%%%%%%%%%%%%%%%%%%%%%%
\paragraph{v1.6:} 2018/01/17

\begin{itemize}
\item
application for development of include files
\item
corrections to manual
\end{itemize}

%%%%%%%%%%%%%%%%%%%%%%%%%%%%%%%%%%%%%%%%
\paragraph{v1.5:} 2017/05/21

\begin{itemize}
\item
more complete structuring introduced
\item
|\childdocof| introduced
\item
|\childdoc| renamed to |\childdocmain|
\item
|\childredirect| renamed to |\childdocforward| and |\childdocforwardprefix|
and functionality expanded
\end{itemize}

%%%%%%%%%%%%%%%%%%%%%%%%%%%%%%%%%%%%%%%%
\paragraph{v1.0:} 2017/04/27

\begin{itemize}
\item
manual and install package
\item
first version published on CTAN
\end{itemize}

%%%%%%%%%%%%%%%%%%%%%%%%%%%%%%%%%%%%%%%%
\paragraph{v0.6:} 2017/04/26

\begin{itemize}
\item
redirection mechanism added
\end{itemize}

%%%%%%%%%%%%%%%%%%%%%%%%%%%%%%%%%%%%%%%%
\paragraph{v0.5:} 2017/04/26

\begin{itemize}
\item
functionality in definition file
\end{itemize}


%%%%%%%%%%%%%%%%%%%%%%%%%%%%%%%%%%%%%%%%%%%%%%%%%%%%%%%%%%%%%%%%%%%%%%%%%%%%%%%%
%%%%%%%%%%%%%%%%%%%%%%%%%%%%%%%%%%%%%%%%%%%%%%%%%%%%%%%%%%%%%%%%%%%%%%%%%%%%%%%%
%%%%%%%%%%%%%%%%%%%%%%%%%%%%%%%%%%%%%%%%%%%%%%%%%%%%%%%%%%%%%%%%%%%%%%%%%%%%%%%%
\appendix

\settowidth\MacroIndent{\rmfamily\scriptsize 000\ }

 \DocInput{childdoc.dtx}

\end{document}
%</driver>
% \fi
%
% %%%%%%%%%%%%%%%%%%%%%%%%%%%%%%%%%%%%%%%%%%%%%%%%%%%%%%%%%%%%%%%%%%%%%%%%%%%%%%
% %%%%%%%%%%%%%%%%%%%%%%%%%%%%%%%%%%%%%%%%%%%%%%%%%%%%%%%%%%%%%%%%%%%%%%%%%%%%%%
% \section{Sample}
%\iffalse
%<*samplemain>
%\fi
%
% The following presents a sample document
% with two chapters, two parts, a title page,
% a compile flag as well as three forwarding files to set the flag.
% It consists of eight |.tex| files:
% \begin{center}
% \begin{tabular}{ll}
% |cdocsamp.tex|&main file\\
% |cdocsch1.tex|&include file for chapter 1\\
% |cdocsch2.tex|&include file for chapter 2\\
% |cdocspt3.tex|&include file for part 3\\
% |cdocspt4.tex|&include file for part 4\\
% |cdocsdrf.tex|&forwarding file for main file in draft mode\\
% |cdocsfi1.tex|&forwarding file for final version of chapter 1\\
% |cdocsfi2.tex|&forwarding file for final version of chapter 2\\
% \end{tabular}
% \end{center}
% Each of the eight files can be compiled directly by the \LaTeX{} compiler.
%
% %%%%%%%%%%%%%%%%%%%%%%%%%%%%%%%%%%%%%%
% \paragraph{Main File.}
%
% The main file is called |cdocsamp.tex|.
%
% Load the \textsf{childdoc} definitions and
% declare the filename for the main document:
%    \begin{macrocode}
\input{childdoc.def}
\childdocmain{}
%    \end{macrocode}

% Optional override for |\version| flag:
%    \begin{macrocode}
%%\ifchilddoc\else\providecommand{\version}{draft}\fi
%    \end{macrocode}

% Define the default values for the |\version| flag
% (|final| for the main file and |draft| for childs):
%    \begin{macrocode}
\ifchilddoc
\providecommand{\version}{draft}
\else
\providecommand{\version}{final}
\fi
%    \end{macrocode}

% Load the standard document class:
%    \begin{macrocode}
\documentclass[12pt]{article}
%    \end{macrocode}

% Start the document body:
%    \begin{macrocode}
\begin{document}
%    \end{macrocode}

% Declare a title page.
% Print title, part of document being processed and version flag:
%    \begin{macrocode}
\addtocounter{page}{-1}
\begin{center}
{\LARGE\bfseries{}childdoc example\par}
\vspace{1cm}
\ifchilddoc
\ifchilddocmanual part\else chapter\fi:
`\childdocname' of `\childdocjob'\par
\else
main document: `\childdocjob'\par
\fi
version: \version\par
\end{center}
\newpage
%    \end{macrocode}

% Manually include selected file,
% otherwise process as usual:
%    \begin{macrocode}
\ifchilddocmanual
\section*{part `\childdocname'}
\input{\childdocname}
\else
%    \end{macrocode}

% Include the two chapters:
%    \begin{macrocode}
\include{cdocsch1}
\include{cdocsch2}
%    \end{macrocode}

% Include the two parts unless only chapters should be displayed:
%    \begin{macrocode}
\ifchilddoc\else
\section{part three}
\input{cdocspt3}
\section{part four}
\input{cdocspt4}
\fi
%    \end{macrocode}

% Process as usual until here:
%    \begin{macrocode}
\fi
%    \end{macrocode}

% End of document body:
%    \begin{macrocode}
\end{document}
%    \end{macrocode}
%\iffalse
%</samplemain>
%\fi
%
% %%%%%%%%%%%%%%%%%%%%%%%%%%%%%%%%%%%%%%
% \paragraph{Chapter Include Files.}
%
% The include files are called |cdocsch1.tex| and |cdocsch2.tex|.
%
%\iffalse
%<*samplechap1|samplechap2>
%\fi

% Optional override for |\version| flag:
%    \begin{macrocode}
%%\providecommand{\version}{final}
%    \end{macrocode}

% Include the main document:
%    \begin{macrocode}
\input{childdoc.def}
\childdocof{cdocsamp}
%    \end{macrocode}

%\iffalse
%</samplechap1|samplechap2>
%\fi
%
%\iffalse
%<*samplechap1>
%\fi
% Some text for chapter 1:
%    \begin{macrocode}
\section{one}
some text in chapter one
%    \end{macrocode}

%\iffalse
%</samplechap1>
%\fi
% Some text for chapter 2:
%\iffalse
%<*samplechap2>
%\fi
%    \begin{macrocode}
\section{two}
more text in chapter two
%    \end{macrocode}

%\iffalse
%</samplechap2>
%\fi
%
% %%%%%%%%%%%%%%%%%%%%%%%%%%%%%%%%%%%%%%
% \paragraph{Part Include Files.}
%
% The include files are called |cdocspt3.tex| and |cdocspt4.tex|.
%
%\iffalse
%<*samplepart3|samplepart4>
%\fi

% Optional override for |\version| flag:
%    \begin{macrocode}
%%\providecommand{\version}{final}
%    \end{macrocode}

% Include the main document:
%    \begin{macrocode}
\input{childdoc.def}
\childdocby{cdocsamp}
%    \end{macrocode}

%\iffalse
%</samplepart3|samplepart4>
%\fi
%
%\iffalse
%<*samplepart3>
%\fi
% Some text for part 3:
%    \begin{macrocode}
some text in part three
%    \end{macrocode}

%\iffalse
%</samplepart3>
%\fi
% Some text for part 4:
%\iffalse
%<*samplepart4>
%\fi
%    \begin{macrocode}
more text in part four
%    \end{macrocode}

%\iffalse
%</samplepart4>
%\fi
%
% %%%%%%%%%%%%%%%%%%%%%%%%%%%%%%%%%%%%%%
% \paragraph{Forwarding for a Complete Draft.}
%
% The following forwarding file |cdocsdrf.tex|
% compiles the main document in draft mode:
%\iffalse
%<*sampledraft>
%\fi
%    \begin{macrocode}
\def\version{draft}
\input{childdoc.def}
\childdocforward{cdocsamp}
%    \end{macrocode}

%\iffalse
%</sampledraft>
%\fi
%
% %%%%%%%%%%%%%%%%%%%%%%%%%%%%%%%%%%%%%%
% \paragraph{Forwarding for Final Version of the Chapters.}
%
% The following forwarding files |cdocsfn1.tex| and |cdocsfn2.tex|
% (with identical content)
% compile the final versions of the child documents
% |cdocsch1.tex| and |cdocsch2.tex|, respectively:
%\iffalse
%<*samplefinal>
%\fi
%    \begin{macrocode}
\def\version{final}
\input{childdoc.def}
\childdocforwardprefix[cdocsamp]{cdocsfn}{cdocsch}
%    \end{macrocode}

%\iffalse
%</samplefinal>
%\fi
%
% %%%%%%%%%%%%%%%%%%%%%%%%%%%%%%%%%%%%%%
% \paragraph{Command Line Processing.}
%
% The following three command lines generate the output files
% |cdocscld|, |cdocscl1| and |cdocscl2|
% which should be identical to
% |cdocsdrf|, |cdocsch1| and |cdocsfn2|, respectively:
% \begin{center}
% \begin{tabular}{l}
% |latex -jobname cdocscld \|\\
% |  "\def\version{draft}\input{childdoc.def}\childdocforward{cdocsamp}"|\\
% |latex -jobname cdocscl1 \|\\
% |  "\input{childdoc.def}\childdocforward[cdocsamp]{cdocsch1}"|\\
% |latex -jobname cdocscl2 \|\\
% |  "\def\version{final}\input{childdoc.def}\childdocforward{cdocsch2}"|
% \end{tabular}
% \end{center}
% Note that the trailing backslash on each first line
% merely continues the input to the second line
% (for convenient cut ant paste).
% Furthermore, the command |latex| can be replaced by any
% of its alternative versions such as |pdflatex|.
%
% %%%%%%%%%%%%%%%%%%%%%%%%%%%%%%%%%%%%%%%%%%%%%%%%%%%%%%%%%%%%%%%%%%%%%%%%%%%%%%
% %%%%%%%%%%%%%%%%%%%%%%%%%%%%%%%%%%%%%%%%%%%%%%%%%%%%%%%%%%%%%%%%%%%%%%%%%%%%%%
% \section{Implementation}
%\iffalse
%<*package>
%\fi
%
% This section describes the definitions file |childdoc.def|.

% The definitions cannot be loaded using |\usepackage| or |\RequirePackage|
% which has a mechanism to prevent loading a style file more than once.
% When loading the definitions by means of |\input|
% multiple instances have to be prevented manually:
%\iffalse
%This code needs to be before the `\ProvidesFile' directive
%which is defined at the beginning of this file.
%Therefore it is also placed there and commented out here.
%</package>
%<*discard>
%\fi
%    \begin{macrocode}
\ifdefined\childdocmain\endinput\fi
%    \end{macrocode}
%\iffalse
%</discard>
%<*package>
%\fi
%
% \macro{\ifchilddoc}
% \macro{\ifchilddocmanual}
% The conditional |\ifchilddoc| tells whether a
% child (true) or main (false) document is being compiled.
% The conditional |\ifchilddocmanual| tells whether
% the |\includeonly| mechanism is used (false) or
% the selection of child files must be performed manually (true).
% The definitions initialise to false:
%    \begin{macrocode}
\newif\ifchilddoc
\newif\ifchilddocmanual
%    \end{macrocode}

% \macro{\childdocname}
% \macro{\childdocjob}
% The macro |\childdocname| stores the name of the main document
% to be compiled. The macro |\childdocjob| stores the name of
% the document on which the \LaTeX{} compiler was originally invoked.
% The content of |\jobname| cannot be compared
% to filenames specified in the source due to different catcodes.
% The following code rescans |\jobname|, stores the result
% in |\childdocname| and saves a copy in |\childdocjob|:
%    \begin{macrocode}
\edef\childdocname{\scantokens\expandafter{\jobname\noexpand}}
\let\childdocjob\childdocname
%    \end{macrocode}

% \macro{\childdocdisable}
% The macro |\childdocdisable| prevents the main file
% from being processed more than once.
% At this stage, the main document command |\childdocmain|
% is assumed to be called once again where it should do nothing.
% Any subsequent call to it should prevent
% a secondary processing of the main document
% It overwrites the forwarding commands
% |\childdocof| and |\childdocforward|
% with empty macros to prevent further inclusions of the main document:
%    \begin{macrocode}
\newcommand{\childdocdisable}
{
  \renewcommand{\childdocmain}[1]{\renewcommand{\childdocmain}[1]{\endinput}}
  \renewcommand{\childdocof}[1]{}
  \renewcommand{\childdocby}[2][]{}
  \renewcommand{\childdocforward}[2][]{}
  \renewcommand{\childdocdisable}{}
}
%    \end{macrocode}

% \macro{\childdocmain}
% The macro |\childdocmain| is to be called at the top of the main file
% with nothing or the main filename (without extension) as argument.
% First, it breaks loops.
% If the argument is not empty and does not match |\childdocname|
% (which is set by the first inclusion of |childdoc.def|),
% |\ifchilddoc| is set to true, |\includeonly| is applied to the child file
% and |\jobname| is set to the main file
% (for proper handling of |.aux| files):
%    \begin{macrocode}
\newcommand{\childdocmain}[1]
{
  \childdocdisable\childdocmain{}
  \if?#1?\else
    \begingroup
      \def\childdoctmp{#1}
      \ifx\childdoctmp\childdocname
        \def\childdoctmp{}
      \else
        \def\childdoctmp
        {
          \childdoctrue
          \includeonly{\childdocname}
          \def\childdocjob{#1}
          \def\jobname{#1}
        }
      \fi
      \expandafter
    \endgroup
    \childdoctmp
  \fi
}
%    \end{macrocode}

% \macro{\childdocof}
% The command |\childdocof| redirects
% compilation to the main file |#1|.
%    \begin{macrocode}
\newcommand{\childdocof}[1]
{
  \childdocdisable
  \childdoctrue
  \includeonly{\childdocname}
  \def\jobname{#1}
  \def\childdocjob{#1}
  \input{#1}
}
%    \end{macrocode}

% \macro{\childdocby}
% The command |\childdocby| ....
%    \begin{macrocode}
\newcommand{\childdocby}[2][]
{
  \childdocdisable
  \childdoctrue
  \childdocmanualtrue
  \if?#1?\else
    \def\jobname{#2}
  \fi
  \def\childdocjob{#2}
  \input{#2}
  \endinput
}
%    \end{macrocode}

% \macro{\childdocforward}
% The command |\childdocforward| redirects
% compilation to the main file or
% (if the optional argument is given) a child file.
% Parameters are set as if the main file
% or a child file starting with |\childdocof| was compiled.
% Then compilation is handed over to the main file:
%    \begin{macrocode}
\newcommand{\childdocforward}[2][]
{
  \begingroup
    \if?#1?
      \def\childdoctmp
      {
        \def\childdocname{#2}
        \def\childdocjob{#2}
        \def\jobname{#2}
        \input{#2}
        \endinput
      }
    \else
      \def\childdoctmp
      {
        \childdocdisable
        \def\childdocname{#2}
        \childdoctrue
        \includeonly{#2}
        \def\childdocjob{#1}
        \def\jobname{#1}
        \input{#1}
        \endinput
      }
    \fi
    \expandafter
  \endgroup
  \childdoctmp
}
%    \end{macrocode}

% \macro{\childdocforwardprefix}
% The command |\childdocforwardprefix| redirects
% compilation to the main or a child file by means of a pattern.
% The prefix |#1| in the current filename is replaced by |#2|
% and the suffix of the current filename is kept
% (it is assumed that the filename does not contain the substring `|~~~|'
% which is used as a delimiter).
% Compilation is handed over to the new file by |\childdocforward|:
%    \begin{macrocode}
\newcommand{\childdocforwardprefix}[3][]
{
  \begingroup
    \def\childdocextract #2##1~~~{\def\childdoctmp{\childdocforward[#1]{#3##1}}}
    \expandafter\childdocextract\childdocname~~~
    \expandafter
  \endgroup
  \childdoctmp
}
%    \end{macrocode}

% \macro{\childdoc}
% The deprecated macro |\childdoc| is a legacy version of |\childdocmain|:
%    \begin{macrocode}
\newcommand{\childdoc}{\childdocmain}
%    \end{macrocode}

% \macro{\childdocredirect}
% The deprecated macro |\childdocredirect| is a legacy version
% of |\childdocforward| and |\childdocforwardprefix|:
%    \begin{macrocode}
\newcommand{\childdocredirect}[2][]
{
  \begingroup
    \if?#1?
      \def\childdoctmp{\childdocforward{#2}}
    \else
      \def\childdoctmp{\childdocforwardprefix{#1}{#2}}
    \fi
    \expandafter
  \endgroup
  \childdoctmp
}
%    \end{macrocode}

%\iffalse
%</package>
%\fi
%
\endinput
\childdocforward[cdocsamp]{cdocsch1}"|\\
% |latex -jobname cdocscl2 \|\\
% |  "\def\version{final}% \iffalse
%
% childdoc.dtx Copyright (C) 2017-2018 Niklas Beisert
%
% This work may be distributed and/or modified under the
% conditions of the LaTeX Project Public License, either version 1.3
% of this license or (at your option) any later version.
% The latest version of this license is in
%   http://www.latex-project.org/lppl.txt
% and version 1.3 or later is part of all distributions of LaTeX
% version 2005/12/01 or later.
%
% This work has the LPPL maintenance status `maintained'.
%
% The Current Maintainer of this work is Niklas Beisert.
%
% This work consists of the files childdoc.dtx and childdoc.ins
% and the derived files childdoc.def and cdocsamp.tex with
% cdocsch1.tex, cdocsch2.tex, cdocsdrf.tex, cdocsfn1.tex, cdocsfn2.tex.
%
%<package>\ifdefined\childdocmain\endinput\fi
%<package>\ProvidesFile{childdoc.def}[2018/12/30 v2.0 child document driver]
%<samplemain>\ProvidesFile{cdocsamp.tex}[2018/12/30 v2.0 sample for childdoc]
%<*driver>
%\ProvidesFile{childdoc.drv}[2018/12/30 v2.0 childdoc reference manual file]
\PassOptionsToClass{10pt,a4paper}{article}
\documentclass{ltxdoc}

\usepackage[margin=35mm]{geometry}
\usepackage{hyperref}
\usepackage{hyperxmp}
\usepackage[usenames]{color}

\hypersetup{colorlinks=true}
\hypersetup{pdfstartview=FitH}
\hypersetup{pdfpagemode=UseNone}
\hypersetup{pdfsource={}}
\hypersetup{pdflang={en-UK}}
\hypersetup{pdfcopyright={Copyright 2017-2018 Niklas Beisert.
  This work may be distributed and/or modified under the
  conditions of the LaTeX Project Public License, either version 1.3
  of this license or (at your option) any later version.}}
\hypersetup{pdflicenseurl={http://www.latex-project.org/lppl.txt}}
\hypersetup{pdfcontactaddress={ETH Zurich, ITP, HIT K,
  Wolfgang-Pauli-Strasse 27}}
\hypersetup{pdfcontactpostcode={8093}}
\hypersetup{pdfcontactcity={Zurich}}
\hypersetup{pdfcontactcountry={Switzerland}}
\hypersetup{pdfcontactemail={nbeisert@itp.phys.ethz.ch}}
\hypersetup{pdfcontacturl={http://people.phys.ethz.ch/\xmptilde nbeisert/}}

\newcommand{\secref}[1]{\hyperref[#1]{section \ref*{#1}}}

\parskip1ex
\parindent0pt
\let\olditemize\itemize
\def\itemize{\olditemize\parskip0pt}

\begin{document}

\title{The \textsf{childdoc} Package}
\hypersetup{pdftitle={The childdoc Package}}
\author{Niklas Beisert\\[2ex]
  Institut f\"ur Theoretische Physik\\
  Eidgen\"ossische Technische Hochschule Z\"urich\\
  Wolfgang-Pauli-Strasse 27, 8093 Z\"urich, Switzerland\\[1ex]
  \href{mailto:nbeisert@itp.phys.ethz.ch}
  {\texttt{nbeisert@itp.phys.ethz.ch}}}
\hypersetup{pdfauthor={Niklas Beisert}}
\hypersetup{pdfsubject={Manual for the LaTeX2e Package childdoc}}
\date{30 December 2018, \textsf{v2.0}}
\maketitle

\begin{abstract}\noindent
\textsf{childdoc} is a \LaTeXe{} package
that enables the direct compilation
of document sections included by |\include|
to individual files.
\end{abstract}

\begingroup
\parskip0ex
\tableofcontents
\endgroup

%%%%%%%%%%%%%%%%%%%%%%%%%%%%%%%%%%%%%%%%%%%%%%%%%%%%%%%%%%%%%%%%%%%%%%%%%%%%%%%%
%%%%%%%%%%%%%%%%%%%%%%%%%%%%%%%%%%%%%%%%%%%%%%%%%%%%%%%%%%%%%%%%%%%%%%%%%%%%%%%%
\section{Introduction}

\LaTeX{} provides a mechanism to structure a large document (such as a book)
into a main file and several child files (containing the chapters)
using the |\include| command.
This mechanism is beneficial for documents
which span hundreds of pages in order to
make the source file(s) more manageable.
Moreover, compilation can be restricted to
selected child files by means of the |\includeonly| command.
The latter feature can be used to reduce the compilation time while editing
(this was significantly more useful in the earlier days of \LaTeX{})
or to generate a smaller document which is easier to navigate.
Another application of |\includeonly| is to generate
documents consisting of selected parts of the complete document.

However, there are a few drawbacks of the plain |\include| mechanism:
\begin{itemize}
\item
The child files cannot be compiled on their own,
they can only be compiled via the main file.
A naive editing environment
(such as a text editor with an option
to have the current file processed by \LaTeX)
may require one to switch to the main file before compiling;
attempting to compile the child file produces errors.
\item
The main file must be modified (each time)
to adjust the |\includeonly| command
to the present needs. This easily leaves the main file in a messy state.
\item
The generated document will always carry the filename
of the main document. This is inconvenient if
several child files are to be compiled and
to be kept for distribution.
\end{itemize}

The present package provides a simple interface
to make child files individually compilable by \LaTeX{}.
Compiling a child file then has the same effect as compiling
the main file with an |\includeonly| command
to select the appropriate child.
Moreover the generated document will carry the name of the child
rather than the main file.
This resolves all three above issues.

This feature is meant to make the editing of books,
thesis documents and lecture notes somewhat more convenient.
However, the package can also be used efficiently for
composing a series of documents (such as exercise sheets)
which are typically distributed individually.
It then assists the author in generating the individual documents
(potentially in different versions)
as well as a document containing the collected series.
Another application is in developing style files
or other kinds of included material
where compilation of the style file could redirect
to a sample or test file.

%%%%%%%%%%%%%%%%%%%%%%%%%%%%%%%%%%%%%%%%%%%%%%%%%%%%%%%%%%%%%%%%%%%%%%%%%%%%%%%%
%%%%%%%%%%%%%%%%%%%%%%%%%%%%%%%%%%%%%%%%%%%%%%%%%%%%%%%%%%%%%%%%%%%%%%%%%%%%%%%%
\section{Usage}

First of all, the package \textsf{childdoc} is \emph{not} a standard
\LaTeXe{} |.sty| style file! Therefore it needs to be invoked in
a non-standard way.

%%%%%%%%%%%%%%%%%%%%%%%%%%%%%%%%%%%%%%%%%%%%%%%%%%%%%%%%%%%%%%%%%%%%%%%%%%%%%%%%
\subsection{Included Files}
\label{sec:include}

%%%%%%%%%%%%%%%%%%%%%%%%%%%%%%%%%%%%%%%%
\DescribeMacro{\childdocmain}
To use the package, add the commands
\begin{center}
\begin{tabular}{l}
|\input{childdoc.def}|\\
|\childdocmain{}|\\
\end{tabular}
\end{center}
at the very top of the main \LaTeX{} file,
in particular \emph{before} the |\documentclass| statement!
The argument of |\childdocmain| should be left empty
(but it must be present).

%%%%%%%%%%%%%%%%%%%%%%%%%%%%%%%%%%%%%%%%
\DescribeMacro{\childdocof}
Furthermore, add the commands
\begin{center}
\begin{tabular}{l}
|\input{childdoc.def}|\\
|\childdocof{|\textit{main}|}|\\
\end{tabular}
\end{center}
at the top of every child file \textit{child}
which is included by |\include{|\textit{child}|}|
from within the main file
(or at least for those files to be compiled individually).
The argument \textit{main} must be the filename of the main file.

There are a couple of
considerations in setting up the main and child documents:

%%%%%%%%%%%%%%%%%%%%%%%%%%%%%%%%%%%%%%%%
\paragraph{Restrictions.}

Please note the following restrictions:
\begin{itemize}
\item
|\childdocmain| must be called with one argument \textit{main}
to ensure compatibility with earlier version of the package.
It must either be empty (|\childdocmain{}|)
or precisely match the filename of the main file in which it is specified.
See \secref{sec:detection} for further information.
\item
The filename \textit{main} must be specified without the |.tex| extension.
\item
The filename \textit{main} is case sensitive
(even in case-insensitive file systems)
due to internal string comparison.
\item
The argument \textit{main} should be fully expanded, it cannot be a macro.
\item
Subdirectories and special characters should be avoided in filenames.
\item
The command |\childdocmain{|\textit{main}|}| must be followed by a whitespace.
It should not be followed immediately by another command
or by a comment mark `|%|'.
This is because the \TeX{} parser reads the token immediately following
the argument of |\childdocmain| and puts it
at the beginning of every child section;
however, a white\-space is ignored.
\end{itemize}

%%%%%%%%%%%%%%%%%%%%%%%%%%%%%%%%%%%%%%%%
\paragraph{Content of Main File.}

It is advisable to place all content in the child files included by |\include|.
Any output contained in the main file will appear in all child documents
unless suppressed manually;
it cannot be suppressed automatically by the |\includeonly| directive
and thus should normally be avoided.
A method to include some content in the main file
by means of conditional processing is described in \secref{sec:conditional}.

%%%%%%%%%%%%%%%%%%%%%%%%%%%%%%%%%%%%%%%%
\paragraph{Page Numbering.}

When only a part of the document is compiled,
the appropriate numbering of pages
(as well as other status parameters)
is determined from the |.aux| files.
The latter contain information from previous passes.
However this information needs to propagate through
all intermediate child documents.
Therefore the page numbering in child documents may well
be inconsistent until the complete document is compiled at least once.

A useful (if unconventional) way to always ensure a consistent
page numbering is to restart the numbering in each child document
and denote the pages by `\textit{child}|.|\textit{page}'
where \textit{child} represents the chapter/section number of the child file.
This can be achieved by the command
|\numberwithin{page}{|\textit{child}|}|
of the \textsf{amsmath} package
where \textit{child} can be |chapter| or |section|
depending on the chosen structuring.
Alternatively, one can modify the macro |\thepage| appropriately
and reset the counter |page| at the start of each child file.

%%%%%%%%%%%%%%%%%%%%%%%%%%%%%%%%%%%%%%%%%%%%%%%%%%%%%%%%%%%%%%%%%%%%%%%%%%%%%%%%
\subsection{Conditional Processing}
\label{sec:conditional}

The package provides a mechanism to compile different versions
of a document. To customise the versions further some conditional processing
can come in handy to distinguish which version is being compiled.
The package provides two macros to describe the compilation context:

%%%%%%%%%%%%%%%%%%%%%%%%%%%%%%%%%%%%%%%%
\DescribeMacro{\ifchilddoc}
The conditional |\ifchilddoc| distinguishes between the compilation of
child documents and the main document:
%
\begin{center}
|\ifchilddoc |\textit{child-code}| |[|\||else |\textit{main-code}]| \||fi|
\end{center}

%%%%%%%%%%%%%%%%%%%%%%%%%%%%%%%%%%%%%%%%
\DescribeMacro{\childdocname}
\DescribeMacro{\childdocjob}
The macro |\childdocname| contains the filename (without extension)
of the main or child file being processed.
Note that |\childdocjob| will always contain the name of the main file.

%%%%%%%%%%%%%%%%%%%%%%%%%%%%%%%%%%%%%%%%
\paragraph{Title Page.}

Conditional processing can be used to include a title or banner page
in the main document when proper precautions are taken.
Importantly, the code in the main file should ensure that the page counter
(as well as other status parameters which are stored in the |.aux| files)
takes the same value after the conditional processing.
Otherwise the page numbers may take divergent values
depending on which part is compiled.

For example, a title page could be declared by:
%
\begin{center}
\begin{tabular}{l}
|\ifchilddoc\||else|\\
|\addtocounter{page}{-1}|\\
\textit{code for title page}\\
|\newpage|\\
|\||fi|
\end{tabular}
\end{center}
%
A banner page for the child documents can be generated by:
%
\begin{center}
\begin{tabular}{l}
|\ifchilddoc|\\
|\addtocounter{page}{-1}|\\
\textit{code for banner page}\\
|\newpage|\\
|\||fi|
\end{tabular}
\end{center}
%
Here one could write a message such as:
\begin{center}
|This is the part \childdocname{} of \childdocjob{}.|
\end{center}

%%%%%%%%%%%%%%%%%%%%%%%%%%%%%%%%%%%%%%%%%%%%%%%%%%%%%%%%%%%%%%%%%%%%%%%%%%%%%%%%
\subsection{Flags}
\label{sec:flags}

The package makes it easy to generate different versions
of the main or child documents.
To this end compilation flags can be defined
and assigned different default values.
They will be particularly useful in conjunction
with the forwarding mechanism described in \secref{sec:forward}.

For example, it may be useful to have a flag |\version|
which can be set to |draft| or |final|.
The document source will contain some conditional code
depending on the value of |\version|.
Suppose further, the flag should default to |final| for the main file
and to |draft| for child files
which is a natural assignment for editing the document.
This is achieved by placing the following code
in the preamble of the main document
(below the |\childdocmain| directive):
%
\begin{center}
\begin{tabular}{l}
|\ifchilddoc|\\
|\providecommand{\version}{draft}|\\
|\||else|\\
|\providecommand{\version}{final}|\\
|\||fi|
\end{tabular}
\end{center}
%
The definition by |\providecommand| makes sure
that previous definitions are not overwritten.
Further statements |\providecommand{\version}{...}|
can thus be added before the above code to override it.

For the main file, one might add a line
(between |\childdocmain| and the above block)
%
\begin{center}
|%\ifchilddoc\||else\providecommand{\version}{draft}\||fi|
\end{center}
%
which can be uncommented to produce a draft version.
Likewise one can add a line to the very top of a child file
(above the |\childdocof{|\textit{main}|}| directive)
%
\begin{center}
|%\providecommand{\version}{final}|
\end{center}
%
which can be uncommented to produce the final version of this child document.

%%%%%%%%%%%%%%%%%%%%%%%%%%%%%%%%%%%%%%%%%%%%%%%%%%%%%%%%%%%%%%%%%%%%%%%%%%%%%%%%
\subsection{Forwarding}
\label{sec:forward}

Different versions of the main or child documents
using compilation flags as described in \secref{sec:flags}
can be (permanently) stored in different files
for convenient compilation, viewing and distribution.
To this end, the package defines a command
to pass on compilation to a different file:

%%%%%%%%%%%%%%%%%%%%%%%%%%%%%%%%%%%%%%%%
\DescribeMacro{\childdocforward}
The command |\childdocforward| redirects processing to
another source file:
%
\begin{center}
\begin{tabular}{l}
|\input{childdoc.def}|\\
|\childdocforward[|\textit{main}|]{|\textit{dest}|}|\\
\end{tabular}
\end{center}
%
The argument \textit{dest} is the destination file
(without extension).
It should be the main file or one of the child files.
Note that further \textsf{childdoc} directives
such as |\childdocof| and |\childdocforward|
in the indicated file will be processed in this form.
The optional argument \textit{main}
passes on directly to the main file \textit{main}
while pretending to compile the child \textit{dest}.
This form behaves as if \textit{dest}
issues |\childdocof{|\textit{main}|}| right away,
and no further \textsf{childdoc} directives will be processed.

%%%%%%%%%%%%%%%%%%%%%%%%%%%%%%%%%%%%%%%%
\DescribeMacro{\...prefix}
In the alternative form |\childdocforwardprefix|,
%
\begin{center}
\begin{tabular}{l}
|\input{childdoc.def}|\\
|\childdocforwardprefix[|\textit{main}|]{|\textit{prefix}|}{|\textit{dest}|}|
\end{tabular}
\end{center}
%
the destination file is determined by a pattern
depending on the current file:
To make this work, the current file must be called
`{\textit{prefix}\hspace{0.2em}\textit{suffix}}'
with \textit{prefix} matching precisely the argument.
Processing is then passed on to the file
`{\textit{dest}\hspace{0.2em}\textit{suffix}}'.
Surely, the same effect is achieved by
directly specifying the
argument `{\textit{dest}\hspace{0.2em}\textit{suffix}}'
in the first form.
However, that requires to set up a different file
for each child. With the alternative form of the command
all these files can have exactly the same content
which simplifies setting them up and maintaining them.

For example, the following file |draft.tex|
with a compilation flag |\version| as described in \secref{sec:flags}
compiles the main document as a draft:
%
\begin{center}
\begin{tabular}{l}
|\def\version{draft}|\\
|\input{childdoc.def}|\\
|\childdocforward{|\textit{main}|}|
\end{tabular}
\end{center}
%
Likewise, the following files |final|\textit{nn}|.tex|
compile the final version of the child document
|child|\textit{nn}|.tex|:
%
\begin{center}
\begin{tabular}{l}
|\def\version{final}|\\
|\input{childdoc.def}|\\
|\childdocforwardprefix{final}{child}|
\end{tabular}
\end{center}
%

Note that when several versions of a main file and/or of each child file
are to be generated, it may be convenient to set up a |Makefile| or
shell script to automatise the process.

%%%%%%%%%%%%%%%%%%%%%%%%%%%%%%%%%%%%%%%%%%%%%%%%%%%%%%%%%%%%%%%%%%%%%%%%%%%%%%%%
\subsection{Command Line Processing}
\label{sec:commandline}

The effect of redirection files can also be achieved by invoking
the \LaTeX{} compiler with a more elaborate command line.
Most conveniently this should be done as part
of a shell script or a |Makefile|.

When using \textsf{childdoc} in the main file, the following
command lines effectively perform a redirection
(note that depending on the shell being used,
backslashes may have to be doubled: `|\|' $\to$ `|\\|'):
%
\begin{center}
|... -jobname "|\textit{target}|" |\\|"|[\textit{flags}]%
|\input{childdoc.def}\childdocforward[|\textit{main}|]{|\textit{dest}|}"|
\end{center}
%
Here \textit{target} is the name of the output file,
\textit{main} is the name of the main file
and \textit{dest} is the name of the main or child file to be processed
(all filenames without extensions).
The optional argument \textit{main} can be omitted
if \textit{main} matches \textit{dest}.
Optionally, compilation \textit{flags} can be defined via |\def| commands.
This command line makes the \TeX{} engine believe
it is compiling the file \textit{target}
whose content is specified as the latter parameter.
The provided code then forwards the processing to
\textit{main} or \textit{dest} as described in \secref{sec:forward}.

%%%%%%%%%%%%%%%%%%%%%%%%%%%%%%%%%%%%%%%%%%%%%%%%%%%%%%%%%%%%%%%%%%%%%%%%%%%%%%%%
\subsection{Include by Input}
\label{sec:input}

Including child documents by |\include| has some restrictions by design.
Most notably, the content of a child document always occupies
its own set of pages; pages cannot be shared between child documents.
Usually, this behaviour makes perfect sense
because each child document contain an essential part of the document.
However, in some situations it may be desirable to compose
a document from a collection of parts
without having mandatory page breaks between then.
For this case, the package
provides a mechanism to include parts
by |\input| which can also be processed individually.
However, by construction this mechanism
requires manual handling of the content to be output.

%%%%%%%%%%%%%%%%%%%%%%%%%%%%%%%%%%%%%%%%
\DescribeMacro{\ifchilddocmanual}
The main file should be prepared as usual, see \secref{sec:include}.
However, the document body must make a distinction
between processing of an individual part and of the main document, e.g.:
%
\begin{center}
\begin{tabular}{l}
|\ifchilddocmanual|\\
|\input{\childdocname}|\\
|\||else|\\
\textit{document body with }|\input{|\textit{part}|}|\\
|\||fi|
\end{tabular}
\end{center}
%
The conditional |\ifchilddocmanual| is true whenever
a part to be included by |\input| is being compiled,
and the name of the part is stored in |\childdocname|.

%%%%%%%%%%%%%%%%%%%%%%%%%%%%%%%%%%%%%%%%
\DescribeMacro{\childdocby}
Each part to be included by |\input| should start with:
%
\begin{center}
\begin{tabular}{l}
|\input{childdoc.def}|\\
|\childdocby{|\textit{main}|}|\\
\end{tabular}
\end{center}
%
The directive |\childdocby| is similar to |\childdocof|
described in \secref{sec:include},
but the subsequent selection of content must be done manually.
To that end, both |\ifchilddoc| and |\ifchilddocmanual|
will be true upon processing of a part,
and the name of the part is stored in |\childdocname|.
Note that |\jobname| will be set to the filename of the current part
so that each part receives an individual |.aux| file
that does not interfere with the |.aux| file(s) of the main document.
This behaviour can be altered by the alternative form
|\childdocby[*]{|\textit{main}|}| (with a non-empty optional argument)
which uses the |.aux| file of the main document
by setting |\jobname| to \textit{main}.

%%%%%%%%%%%%%%%%%%%%%%%%%%%%%%%%%%%%%%%%%%%%%%%%%%%%%%%%%%%%%%%%%%%%%%%%%%%%%%%%
\subsection{Driver Development}
\label{sec:driver}

The \textsf{childdoc} mechanism can also be use for the development
of definition files such as \LaTeX{} styles or classes.
This case differs from the above setup with multiple parts
included by |\include| in that no |\includeonly| should be invoked.
This can be achieved by starting the include file
(before |\ProvidesPackage|) with:
%
\begin{center}
\begin{tabular}{l}
|\input{childdoc.def}|\\
|\childdocforward{|\textit{main}|}|\\
\end{tabular}
\end{center}
%
or alternatively with:
%
\begin{center}
\begin{tabular}{l}
|\input{childdoc.def}|\\
|\childdocby{|\textit{main}|}|\\
\end{tabular}
\end{center}
%
Both forms have slightly different effects as described above.
The main file is prepared as usual, see \secref{sec:include}.

%%%%%%%%%%%%%%%%%%%%%%%%%%%%%%%%%%%%%%%%%%%%%%%%%%%%%%%%%%%%%%%%%%%%%%%%%%%%%%%%
\subsection{Legacy Detection}
\label{sec:detection}

The directive |\childdocmain| in the main file can detect
whether the complete document or merely a child is to be compiled
even without using the directive |\childdocof|.
This method is deprecated because it is less robust
and there is no compelling reason to use it;
it is merely provided for backward compatibility
and it may be removed in future versions.

If the detection mechanism is to be used,
it is mandatory to correctly specify
the filename of the main file as the argument of |\childdocmain|:
%
\begin{center}
\begin{tabular}{l}
|\input{childdoc.def}|\\
|\childdocmain{|\textit{main}|}|\\
\end{tabular}
\end{center}
%
If |\jobname| does not match the argument \textit{main} of |\childdocmain|,
it is assumed that |\jobname| points to the child file to be compiled.
When using |\childdocmain| with the main file specified as argument,
it suffices to start a child file
with just |\input{|\textit{main}|}|
without loading of the package and using |\childdocof|.
If instead all processing is done
with the appropriate \textsf{childdoc} directives,
the argument of \textit{main} of |\childdocmain| can be empty.

An alternative version of the command line processing described
in \secref{sec:commandline} using the detection mechanism reads:
%
\begin{center}
|... -jobname "|\textit{target}|" "|[\textit{flags}]%
[|\def\jobname{|\textit{dest}|}|]|\input{|\textit{main}|}"|
\end{center}

%%%%%%%%%%%%%%%%%%%%%%%%%%%%%%%%%%%%%%%%%%%%%%%%%%%%%%%%%%%%%%%%%%%%%%%%%%%%%%%%
\subsection{Manual Code}
\label{sec:manual}

In case one cannot be certain whether the definitions file |childdoc.def|
is installed on the target \TeX{} distribution
and one prefers not to ship it,
it is conceivable to paste a few relevant commands into the sources.

To that end, drop all statements |\input{childdoc.def}|
and perform the replacements as outlined below.
Instead of |\childdocmain{|\textit{main}|}| add the following code
to the top of the main file:
%
\begin{center}
\begin{tabular}{l}
|\||ifdefined\childdocname\endinput\||fi\newif\ifchilddoc|\\
|\edef\childdocname{\scantokens\expandafter{\jobname\noexpand}}|\\
|\def\childdocmain{|\textit{main}|}\||ifx\childdocmain\childdocname\||else|\\
|\childdoctrue\includeonly{\childdocname}\let\jobname\childdocmain\||fi|\\
\end{tabular}
\end{center}
%
Instead of |\childdocof{|\textit{main}|}| just include the main file
at the top of each child file:
%
\begin{center}
|\input{|\textit{main}|}|
\end{center}
%
A simple redirection |\childdocforward{|\textit{dest}|}| is achieved by:
%
\begin{center}
|\def\jobname{|\textit{dest}|}\input{\jobname}|
\end{center}
%
The redirection with prefix
|\childdocforwardprefix[|\textit{prefix}|]{|\textit{dest}|}|
is accomplished by:
%
\begin{center}
\begin{tabular}{l}
|{\edef\jobname{\scantokens\expandafter{\jobname\noexpand}}|\\
|\def\redirectjob |\textit{prefix}|#1~~~{\gdef\jobname{|\textit{dest}|#1}}|\\
|\expandafter\redirectjob\jobname~~~}\input{\jobname}|
\end{tabular}
\end{center}

In an alternative approach,
child documents can be compiled by a specific command line
without additional code or specific definitions:
%
\begin{center}
|... -jobname "|\textit{target}|" "|[\textit{flags}]%
|\includeonly{|\textit{dest}|}\input{|\textit{main}|}"|
\end{center}
%

%%%%%%%%%%%%%%%%%%%%%%%%%%%%%%%%%%%%%%%%%%%%%%%%%%%%%%%%%%%%%%%%%%%%%%%%%%%%%%%%
%%%%%%%%%%%%%%%%%%%%%%%%%%%%%%%%%%%%%%%%%%%%%%%%%%%%%%%%%%%%%%%%%%%%%%%%%%%%%%%%
\section{Information}

%%%%%%%%%%%%%%%%%%%%%%%%%%%%%%%%%%%%%%%%%%%%%%%%%%%%%%%%%%%%%%%%%%%%%%%%%%%%%%%%
\subsection{Copyright}

Copyright \copyright{} 2017--2018 Niklas Beisert

This work may be distributed and/or modified under the
conditions of the \LaTeX{} Project Public License, either version 1.3
of this license or (at your option) any later version.
The latest version of this license is in
  \url{http://www.latex-project.org/lppl.txt}
and version 1.3 or later is part of all distributions of \LaTeX{}
version 2005/12/01 or later.

This work has the LPPL maintenance status `maintained'.

The Current Maintainer of this work is Niklas Beisert.

This work consists of the files |README.txt|, |childdoc.ins| and |childdoc.dtx|
as well as the derived files |childdoc.def|, |cdocsamp.tex|
with |cdocsch1.tex|, |cdocsch2.tex|, |cdocspt3.tex|, |cdocspt4.tex|,
|cdocsdrf.tex|, |cdocsfn1.tex|, |cdocsfn2.tex|
as well as |childdoc.pdf|.

%%%%%%%%%%%%%%%%%%%%%%%%%%%%%%%%%%%%%%%%%%%%%%%%%%%%%%%%%%%%%%%%%%%%%%%%%%%%%%%%
\subsection{Files and Installation}

The package consists of the files:
%
\begin{center}
\begin{tabular}{ll}
    |README.txt|   & readme file \\
    |childdoc.ins| & installation file \\
    |childdoc.dtx| & source file \\
    |childdoc.def| & definition file \\
    |cdocsamp.tex| & sample main file \\
    |cdocsch1.tex| & sample include file \\
    |cdocsch2.tex| & sample include file \\
    |cdocspt3.tex| & sample part file \\
    |cdocspt4.tex| & sample part file \\
    |cdocsdrf.tex| & sample redirection file \\
    |cdocsfn1.tex| & sample redirection file \\
    |cdocsfn2.tex| & sample redirection file \\
    |childdoc.pdf| & manual
\end{tabular}
\end{center}
%
The distribution consists of the files
|README.txt|, |childdoc.ins| and |childdoc.dtx|.
%
\begin{itemize}
\item
Run (pdf)\LaTeX{} on |childdoc.dtx|
to compile the manual |childdoc.pdf| (this file).
\item
Run \LaTeX{} on |childdoc.ins| to create the definitions file |childdoc.def|
and the sample |cdocsamp.tex| with include files
|cdocsch1.tex|, |cdocsch2.tex|, |cdocspt3.tex|, |cdocspt4.tex|,
|cdocsdrf.tex|, |cdocsfn1.tex|, |cdocsfn2.tex|.
Then copy the file |childdoc.def| to an appropriate directory of your \LaTeX{}
distribution, e.g.\ \textit{texmf-root}|/tex/latex/childdoc|.
\end{itemize}

%%%%%%%%%%%%%%%%%%%%%%%%%%%%%%%%%%%%%%%%%%%%%%%%%%%%%%%%%%%%%%%%%%%%%%%%%%%%%%%%
\subsection{Related CTAN Packages}

There are several other packages which offer a similar functionality:
%
\begin{itemize}
\item
The packages
\href{http://ctan.org/pkg/docmute}{\textsf{docmute}},
\href{http://ctan.org/pkg/includex}{\textsf{includex}} and
\href{http://ctan.org/pkg/standalone}{\textsf{standalone}}
provide commands to include only the document body of
a child file thus allowing both files to be compiled individually.
\item
The packages \href{http://ctan.org/pkg/subdocs}{\textsf{subdocs}}
and \href{http://ctan.org/pkg/subfiles}{\textsf{subfiles}}
provide structures in which the main and child documents can be
encapsulated and allowing them to be compiled individually.
The inclusion mechanism is different from the conventional |\include|.
\item
The package \href{http://ctan.org/pkg/combine}{\textsf{combine}}
is an elaborate solution to combine several documents into one.
\end{itemize}
%
See also the CTAN topic \href{http://ctan.org/topic/subdocs}{\textsf{subdocs}}
for further related packages.
The present package differs from the above solutions in that
a document structure constructed with the conventional |\include| mechanism
just needs two extra commands at the top of every file
such that all constituent files can be compiled individually.

%%%%%%%%%%%%%%%%%%%%%%%%%%%%%%%%%%%%%%%%%%%%%%%%%%%%%%%%%%%%%%%%%%%%%%%%%%%%%%%%
%\subsection{Feature Suggestions}
%
%The following is a list of features which may be useful for future
%versions of this package:
%%
%\begin{itemize}
%\item
%\ldots
%\end{itemize}

%%%%%%%%%%%%%%%%%%%%%%%%%%%%%%%%%%%%%%%%%%%%%%%%%%%%%%%%%%%%%%%%%%%%%%%%%%%%%%%%
\subsection{Revision History}

%%%%%%%%%%%%%%%%%%%%%%%%%%%%%%%%%%%%%%%%
\paragraph{v2.0:} 2018/12/30

\begin{itemize}
\item
immediate forward processing
\item
added |\childdocby| mechanism
\item
manual restructured
\end{itemize}

%%%%%%%%%%%%%%%%%%%%%%%%%%%%%%%%%%%%%%%%
\paragraph{v1.6:} 2018/01/17

\begin{itemize}
\item
application for development of include files
\item
corrections to manual
\end{itemize}

%%%%%%%%%%%%%%%%%%%%%%%%%%%%%%%%%%%%%%%%
\paragraph{v1.5:} 2017/05/21

\begin{itemize}
\item
more complete structuring introduced
\item
|\childdocof| introduced
\item
|\childdoc| renamed to |\childdocmain|
\item
|\childredirect| renamed to |\childdocforward| and |\childdocforwardprefix|
and functionality expanded
\end{itemize}

%%%%%%%%%%%%%%%%%%%%%%%%%%%%%%%%%%%%%%%%
\paragraph{v1.0:} 2017/04/27

\begin{itemize}
\item
manual and install package
\item
first version published on CTAN
\end{itemize}

%%%%%%%%%%%%%%%%%%%%%%%%%%%%%%%%%%%%%%%%
\paragraph{v0.6:} 2017/04/26

\begin{itemize}
\item
redirection mechanism added
\end{itemize}

%%%%%%%%%%%%%%%%%%%%%%%%%%%%%%%%%%%%%%%%
\paragraph{v0.5:} 2017/04/26

\begin{itemize}
\item
functionality in definition file
\end{itemize}


%%%%%%%%%%%%%%%%%%%%%%%%%%%%%%%%%%%%%%%%%%%%%%%%%%%%%%%%%%%%%%%%%%%%%%%%%%%%%%%%
%%%%%%%%%%%%%%%%%%%%%%%%%%%%%%%%%%%%%%%%%%%%%%%%%%%%%%%%%%%%%%%%%%%%%%%%%%%%%%%%
%%%%%%%%%%%%%%%%%%%%%%%%%%%%%%%%%%%%%%%%%%%%%%%%%%%%%%%%%%%%%%%%%%%%%%%%%%%%%%%%
\appendix

\settowidth\MacroIndent{\rmfamily\scriptsize 000\ }

 \DocInput{childdoc.dtx}

\end{document}
%</driver>
% \fi
%
% %%%%%%%%%%%%%%%%%%%%%%%%%%%%%%%%%%%%%%%%%%%%%%%%%%%%%%%%%%%%%%%%%%%%%%%%%%%%%%
% %%%%%%%%%%%%%%%%%%%%%%%%%%%%%%%%%%%%%%%%%%%%%%%%%%%%%%%%%%%%%%%%%%%%%%%%%%%%%%
% \section{Sample}
%\iffalse
%<*samplemain>
%\fi
%
% The following presents a sample document
% with two chapters, two parts, a title page,
% a compile flag as well as three forwarding files to set the flag.
% It consists of eight |.tex| files:
% \begin{center}
% \begin{tabular}{ll}
% |cdocsamp.tex|&main file\\
% |cdocsch1.tex|&include file for chapter 1\\
% |cdocsch2.tex|&include file for chapter 2\\
% |cdocspt3.tex|&include file for part 3\\
% |cdocspt4.tex|&include file for part 4\\
% |cdocsdrf.tex|&forwarding file for main file in draft mode\\
% |cdocsfi1.tex|&forwarding file for final version of chapter 1\\
% |cdocsfi2.tex|&forwarding file for final version of chapter 2\\
% \end{tabular}
% \end{center}
% Each of the eight files can be compiled directly by the \LaTeX{} compiler.
%
% %%%%%%%%%%%%%%%%%%%%%%%%%%%%%%%%%%%%%%
% \paragraph{Main File.}
%
% The main file is called |cdocsamp.tex|.
%
% Load the \textsf{childdoc} definitions and
% declare the filename for the main document:
%    \begin{macrocode}
\input{childdoc.def}
\childdocmain{}
%    \end{macrocode}

% Optional override for |\version| flag:
%    \begin{macrocode}
%%\ifchilddoc\else\providecommand{\version}{draft}\fi
%    \end{macrocode}

% Define the default values for the |\version| flag
% (|final| for the main file and |draft| for childs):
%    \begin{macrocode}
\ifchilddoc
\providecommand{\version}{draft}
\else
\providecommand{\version}{final}
\fi
%    \end{macrocode}

% Load the standard document class:
%    \begin{macrocode}
\documentclass[12pt]{article}
%    \end{macrocode}

% Start the document body:
%    \begin{macrocode}
\begin{document}
%    \end{macrocode}

% Declare a title page.
% Print title, part of document being processed and version flag:
%    \begin{macrocode}
\addtocounter{page}{-1}
\begin{center}
{\LARGE\bfseries{}childdoc example\par}
\vspace{1cm}
\ifchilddoc
\ifchilddocmanual part\else chapter\fi:
`\childdocname' of `\childdocjob'\par
\else
main document: `\childdocjob'\par
\fi
version: \version\par
\end{center}
\newpage
%    \end{macrocode}

% Manually include selected file,
% otherwise process as usual:
%    \begin{macrocode}
\ifchilddocmanual
\section*{part `\childdocname'}
\input{\childdocname}
\else
%    \end{macrocode}

% Include the two chapters:
%    \begin{macrocode}
\include{cdocsch1}
\include{cdocsch2}
%    \end{macrocode}

% Include the two parts unless only chapters should be displayed:
%    \begin{macrocode}
\ifchilddoc\else
\section{part three}
\input{cdocspt3}
\section{part four}
\input{cdocspt4}
\fi
%    \end{macrocode}

% Process as usual until here:
%    \begin{macrocode}
\fi
%    \end{macrocode}

% End of document body:
%    \begin{macrocode}
\end{document}
%    \end{macrocode}
%\iffalse
%</samplemain>
%\fi
%
% %%%%%%%%%%%%%%%%%%%%%%%%%%%%%%%%%%%%%%
% \paragraph{Chapter Include Files.}
%
% The include files are called |cdocsch1.tex| and |cdocsch2.tex|.
%
%\iffalse
%<*samplechap1|samplechap2>
%\fi

% Optional override for |\version| flag:
%    \begin{macrocode}
%%\providecommand{\version}{final}
%    \end{macrocode}

% Include the main document:
%    \begin{macrocode}
\input{childdoc.def}
\childdocof{cdocsamp}
%    \end{macrocode}

%\iffalse
%</samplechap1|samplechap2>
%\fi
%
%\iffalse
%<*samplechap1>
%\fi
% Some text for chapter 1:
%    \begin{macrocode}
\section{one}
some text in chapter one
%    \end{macrocode}

%\iffalse
%</samplechap1>
%\fi
% Some text for chapter 2:
%\iffalse
%<*samplechap2>
%\fi
%    \begin{macrocode}
\section{two}
more text in chapter two
%    \end{macrocode}

%\iffalse
%</samplechap2>
%\fi
%
% %%%%%%%%%%%%%%%%%%%%%%%%%%%%%%%%%%%%%%
% \paragraph{Part Include Files.}
%
% The include files are called |cdocspt3.tex| and |cdocspt4.tex|.
%
%\iffalse
%<*samplepart3|samplepart4>
%\fi

% Optional override for |\version| flag:
%    \begin{macrocode}
%%\providecommand{\version}{final}
%    \end{macrocode}

% Include the main document:
%    \begin{macrocode}
\input{childdoc.def}
\childdocby{cdocsamp}
%    \end{macrocode}

%\iffalse
%</samplepart3|samplepart4>
%\fi
%
%\iffalse
%<*samplepart3>
%\fi
% Some text for part 3:
%    \begin{macrocode}
some text in part three
%    \end{macrocode}

%\iffalse
%</samplepart3>
%\fi
% Some text for part 4:
%\iffalse
%<*samplepart4>
%\fi
%    \begin{macrocode}
more text in part four
%    \end{macrocode}

%\iffalse
%</samplepart4>
%\fi
%
% %%%%%%%%%%%%%%%%%%%%%%%%%%%%%%%%%%%%%%
% \paragraph{Forwarding for a Complete Draft.}
%
% The following forwarding file |cdocsdrf.tex|
% compiles the main document in draft mode:
%\iffalse
%<*sampledraft>
%\fi
%    \begin{macrocode}
\def\version{draft}
\input{childdoc.def}
\childdocforward{cdocsamp}
%    \end{macrocode}

%\iffalse
%</sampledraft>
%\fi
%
% %%%%%%%%%%%%%%%%%%%%%%%%%%%%%%%%%%%%%%
% \paragraph{Forwarding for Final Version of the Chapters.}
%
% The following forwarding files |cdocsfn1.tex| and |cdocsfn2.tex|
% (with identical content)
% compile the final versions of the child documents
% |cdocsch1.tex| and |cdocsch2.tex|, respectively:
%\iffalse
%<*samplefinal>
%\fi
%    \begin{macrocode}
\def\version{final}
\input{childdoc.def}
\childdocforwardprefix[cdocsamp]{cdocsfn}{cdocsch}
%    \end{macrocode}

%\iffalse
%</samplefinal>
%\fi
%
% %%%%%%%%%%%%%%%%%%%%%%%%%%%%%%%%%%%%%%
% \paragraph{Command Line Processing.}
%
% The following three command lines generate the output files
% |cdocscld|, |cdocscl1| and |cdocscl2|
% which should be identical to
% |cdocsdrf|, |cdocsch1| and |cdocsfn2|, respectively:
% \begin{center}
% \begin{tabular}{l}
% |latex -jobname cdocscld \|\\
% |  "\def\version{draft}\input{childdoc.def}\childdocforward{cdocsamp}"|\\
% |latex -jobname cdocscl1 \|\\
% |  "\input{childdoc.def}\childdocforward[cdocsamp]{cdocsch1}"|\\
% |latex -jobname cdocscl2 \|\\
% |  "\def\version{final}\input{childdoc.def}\childdocforward{cdocsch2}"|
% \end{tabular}
% \end{center}
% Note that the trailing backslash on each first line
% merely continues the input to the second line
% (for convenient cut ant paste).
% Furthermore, the command |latex| can be replaced by any
% of its alternative versions such as |pdflatex|.
%
% %%%%%%%%%%%%%%%%%%%%%%%%%%%%%%%%%%%%%%%%%%%%%%%%%%%%%%%%%%%%%%%%%%%%%%%%%%%%%%
% %%%%%%%%%%%%%%%%%%%%%%%%%%%%%%%%%%%%%%%%%%%%%%%%%%%%%%%%%%%%%%%%%%%%%%%%%%%%%%
% \section{Implementation}
%\iffalse
%<*package>
%\fi
%
% This section describes the definitions file |childdoc.def|.

% The definitions cannot be loaded using |\usepackage| or |\RequirePackage|
% which has a mechanism to prevent loading a style file more than once.
% When loading the definitions by means of |\input|
% multiple instances have to be prevented manually:
%\iffalse
%This code needs to be before the `\ProvidesFile' directive
%which is defined at the beginning of this file.
%Therefore it is also placed there and commented out here.
%</package>
%<*discard>
%\fi
%    \begin{macrocode}
\ifdefined\childdocmain\endinput\fi
%    \end{macrocode}
%\iffalse
%</discard>
%<*package>
%\fi
%
% \macro{\ifchilddoc}
% \macro{\ifchilddocmanual}
% The conditional |\ifchilddoc| tells whether a
% child (true) or main (false) document is being compiled.
% The conditional |\ifchilddocmanual| tells whether
% the |\includeonly| mechanism is used (false) or
% the selection of child files must be performed manually (true).
% The definitions initialise to false:
%    \begin{macrocode}
\newif\ifchilddoc
\newif\ifchilddocmanual
%    \end{macrocode}

% \macro{\childdocname}
% \macro{\childdocjob}
% The macro |\childdocname| stores the name of the main document
% to be compiled. The macro |\childdocjob| stores the name of
% the document on which the \LaTeX{} compiler was originally invoked.
% The content of |\jobname| cannot be compared
% to filenames specified in the source due to different catcodes.
% The following code rescans |\jobname|, stores the result
% in |\childdocname| and saves a copy in |\childdocjob|:
%    \begin{macrocode}
\edef\childdocname{\scantokens\expandafter{\jobname\noexpand}}
\let\childdocjob\childdocname
%    \end{macrocode}

% \macro{\childdocdisable}
% The macro |\childdocdisable| prevents the main file
% from being processed more than once.
% At this stage, the main document command |\childdocmain|
% is assumed to be called once again where it should do nothing.
% Any subsequent call to it should prevent
% a secondary processing of the main document
% It overwrites the forwarding commands
% |\childdocof| and |\childdocforward|
% with empty macros to prevent further inclusions of the main document:
%    \begin{macrocode}
\newcommand{\childdocdisable}
{
  \renewcommand{\childdocmain}[1]{\renewcommand{\childdocmain}[1]{\endinput}}
  \renewcommand{\childdocof}[1]{}
  \renewcommand{\childdocby}[2][]{}
  \renewcommand{\childdocforward}[2][]{}
  \renewcommand{\childdocdisable}{}
}
%    \end{macrocode}

% \macro{\childdocmain}
% The macro |\childdocmain| is to be called at the top of the main file
% with nothing or the main filename (without extension) as argument.
% First, it breaks loops.
% If the argument is not empty and does not match |\childdocname|
% (which is set by the first inclusion of |childdoc.def|),
% |\ifchilddoc| is set to true, |\includeonly| is applied to the child file
% and |\jobname| is set to the main file
% (for proper handling of |.aux| files):
%    \begin{macrocode}
\newcommand{\childdocmain}[1]
{
  \childdocdisable\childdocmain{}
  \if?#1?\else
    \begingroup
      \def\childdoctmp{#1}
      \ifx\childdoctmp\childdocname
        \def\childdoctmp{}
      \else
        \def\childdoctmp
        {
          \childdoctrue
          \includeonly{\childdocname}
          \def\childdocjob{#1}
          \def\jobname{#1}
        }
      \fi
      \expandafter
    \endgroup
    \childdoctmp
  \fi
}
%    \end{macrocode}

% \macro{\childdocof}
% The command |\childdocof| redirects
% compilation to the main file |#1|.
%    \begin{macrocode}
\newcommand{\childdocof}[1]
{
  \childdocdisable
  \childdoctrue
  \includeonly{\childdocname}
  \def\jobname{#1}
  \def\childdocjob{#1}
  \input{#1}
}
%    \end{macrocode}

% \macro{\childdocby}
% The command |\childdocby| ....
%    \begin{macrocode}
\newcommand{\childdocby}[2][]
{
  \childdocdisable
  \childdoctrue
  \childdocmanualtrue
  \if?#1?\else
    \def\jobname{#2}
  \fi
  \def\childdocjob{#2}
  \input{#2}
  \endinput
}
%    \end{macrocode}

% \macro{\childdocforward}
% The command |\childdocforward| redirects
% compilation to the main file or
% (if the optional argument is given) a child file.
% Parameters are set as if the main file
% or a child file starting with |\childdocof| was compiled.
% Then compilation is handed over to the main file:
%    \begin{macrocode}
\newcommand{\childdocforward}[2][]
{
  \begingroup
    \if?#1?
      \def\childdoctmp
      {
        \def\childdocname{#2}
        \def\childdocjob{#2}
        \def\jobname{#2}
        \input{#2}
        \endinput
      }
    \else
      \def\childdoctmp
      {
        \childdocdisable
        \def\childdocname{#2}
        \childdoctrue
        \includeonly{#2}
        \def\childdocjob{#1}
        \def\jobname{#1}
        \input{#1}
        \endinput
      }
    \fi
    \expandafter
  \endgroup
  \childdoctmp
}
%    \end{macrocode}

% \macro{\childdocforwardprefix}
% The command |\childdocforwardprefix| redirects
% compilation to the main or a child file by means of a pattern.
% The prefix |#1| in the current filename is replaced by |#2|
% and the suffix of the current filename is kept
% (it is assumed that the filename does not contain the substring `|~~~|'
% which is used as a delimiter).
% Compilation is handed over to the new file by |\childdocforward|:
%    \begin{macrocode}
\newcommand{\childdocforwardprefix}[3][]
{
  \begingroup
    \def\childdocextract #2##1~~~{\def\childdoctmp{\childdocforward[#1]{#3##1}}}
    \expandafter\childdocextract\childdocname~~~
    \expandafter
  \endgroup
  \childdoctmp
}
%    \end{macrocode}

% \macro{\childdoc}
% The deprecated macro |\childdoc| is a legacy version of |\childdocmain|:
%    \begin{macrocode}
\newcommand{\childdoc}{\childdocmain}
%    \end{macrocode}

% \macro{\childdocredirect}
% The deprecated macro |\childdocredirect| is a legacy version
% of |\childdocforward| and |\childdocforwardprefix|:
%    \begin{macrocode}
\newcommand{\childdocredirect}[2][]
{
  \begingroup
    \if?#1?
      \def\childdoctmp{\childdocforward{#2}}
    \else
      \def\childdoctmp{\childdocforwardprefix{#1}{#2}}
    \fi
    \expandafter
  \endgroup
  \childdoctmp
}
%    \end{macrocode}

%\iffalse
%</package>
%\fi
%
\endinput
\childdocforward{cdocsch2}"|
% \end{tabular}
% \end{center}
% Note that the trailing backslash on each first line
% merely continues the input to the second line
% (for convenient cut ant paste).
% Furthermore, the command |latex| can be replaced by any
% of its alternative versions such as |pdflatex|.
%
% %%%%%%%%%%%%%%%%%%%%%%%%%%%%%%%%%%%%%%%%%%%%%%%%%%%%%%%%%%%%%%%%%%%%%%%%%%%%%%
% %%%%%%%%%%%%%%%%%%%%%%%%%%%%%%%%%%%%%%%%%%%%%%%%%%%%%%%%%%%%%%%%%%%%%%%%%%%%%%
% \section{Implementation}
%\iffalse
%<*package>
%\fi
%
% This section describes the definitions file |childdoc.def|.

% The definitions cannot be loaded using |\usepackage| or |\RequirePackage|
% which has a mechanism to prevent loading a style file more than once.
% When loading the definitions by means of |\input|
% multiple instances have to be prevented manually:
%\iffalse
%This code needs to be before the `\ProvidesFile' directive
%which is defined at the beginning of this file.
%Therefore it is also placed there and commented out here.
%</package>
%<*discard>
%\fi
%    \begin{macrocode}
\ifdefined\childdocmain\endinput\fi
%    \end{macrocode}
%\iffalse
%</discard>
%<*package>
%\fi
%
% \macro{\ifchilddoc}
% \macro{\ifchilddocmanual}
% The conditional |\ifchilddoc| tells whether a
% child (true) or main (false) document is being compiled.
% The conditional |\ifchilddocmanual| tells whether
% the |\includeonly| mechanism is used (false) or
% the selection of child files must be performed manually (true).
% The definitions initialise to false:
%    \begin{macrocode}
\newif\ifchilddoc
\newif\ifchilddocmanual
%    \end{macrocode}

% \macro{\childdocname}
% \macro{\childdocjob}
% The macro |\childdocname| stores the name of the main document
% to be compiled. The macro |\childdocjob| stores the name of
% the document on which the \LaTeX{} compiler was originally invoked.
% The content of |\jobname| cannot be compared
% to filenames specified in the source due to different catcodes.
% The following code rescans |\jobname|, stores the result
% in |\childdocname| and saves a copy in |\childdocjob|:
%    \begin{macrocode}
\edef\childdocname{\scantokens\expandafter{\jobname\noexpand}}
\let\childdocjob\childdocname
%    \end{macrocode}

% \macro{\childdocdisable}
% The macro |\childdocdisable| prevents the main file
% from being processed more than once.
% At this stage, the main document command |\childdocmain|
% is assumed to be called once again where it should do nothing.
% Any subsequent call to it should prevent
% a secondary processing of the main document
% It overwrites the forwarding commands
% |\childdocof| and |\childdocforward|
% with empty macros to prevent further inclusions of the main document:
%    \begin{macrocode}
\newcommand{\childdocdisable}
{
  \renewcommand{\childdocmain}[1]{\renewcommand{\childdocmain}[1]{\endinput}}
  \renewcommand{\childdocof}[1]{}
  \renewcommand{\childdocby}[2][]{}
  \renewcommand{\childdocforward}[2][]{}
  \renewcommand{\childdocdisable}{}
}
%    \end{macrocode}

% \macro{\childdocmain}
% The macro |\childdocmain| is to be called at the top of the main file
% with nothing or the main filename (without extension) as argument.
% First, it breaks loops.
% If the argument is not empty and does not match |\childdocname|
% (which is set by the first inclusion of |childdoc.def|),
% |\ifchilddoc| is set to true, |\includeonly| is applied to the child file
% and |\jobname| is set to the main file
% (for proper handling of |.aux| files):
%    \begin{macrocode}
\newcommand{\childdocmain}[1]
{
  \childdocdisable\childdocmain{}
  \if?#1?\else
    \begingroup
      \def\childdoctmp{#1}
      \ifx\childdoctmp\childdocname
        \def\childdoctmp{}
      \else
        \def\childdoctmp
        {
          \childdoctrue
          \includeonly{\childdocname}
          \def\childdocjob{#1}
          \def\jobname{#1}
        }
      \fi
      \expandafter
    \endgroup
    \childdoctmp
  \fi
}
%    \end{macrocode}

% \macro{\childdocof}
% The command |\childdocof| redirects
% compilation to the main file |#1|.
%    \begin{macrocode}
\newcommand{\childdocof}[1]
{
  \childdocdisable
  \childdoctrue
  \includeonly{\childdocname}
  \def\jobname{#1}
  \def\childdocjob{#1}
  \input{#1}
}
%    \end{macrocode}

% \macro{\childdocby}
% The command |\childdocby| ....
%    \begin{macrocode}
\newcommand{\childdocby}[2][]
{
  \childdocdisable
  \childdoctrue
  \childdocmanualtrue
  \if?#1?\else
    \def\jobname{#2}
  \fi
  \def\childdocjob{#2}
  \input{#2}
  \endinput
}
%    \end{macrocode}

% \macro{\childdocforward}
% The command |\childdocforward| redirects
% compilation to the main file or
% (if the optional argument is given) a child file.
% Parameters are set as if the main file
% or a child file starting with |\childdocof| was compiled.
% Then compilation is handed over to the main file:
%    \begin{macrocode}
\newcommand{\childdocforward}[2][]
{
  \begingroup
    \if?#1?
      \def\childdoctmp
      {
        \def\childdocname{#2}
        \def\childdocjob{#2}
        \def\jobname{#2}
        \input{#2}
        \endinput
      }
    \else
      \def\childdoctmp
      {
        \childdocdisable
        \def\childdocname{#2}
        \childdoctrue
        \includeonly{#2}
        \def\childdocjob{#1}
        \def\jobname{#1}
        \input{#1}
        \endinput
      }
    \fi
    \expandafter
  \endgroup
  \childdoctmp
}
%    \end{macrocode}

% \macro{\childdocforwardprefix}
% The command |\childdocforwardprefix| redirects
% compilation to the main or a child file by means of a pattern.
% The prefix |#1| in the current filename is replaced by |#2|
% and the suffix of the current filename is kept
% (it is assumed that the filename does not contain the substring `|~~~|'
% which is used as a delimiter).
% Compilation is handed over to the new file by |\childdocforward|:
%    \begin{macrocode}
\newcommand{\childdocforwardprefix}[3][]
{
  \begingroup
    \def\childdocextract #2##1~~~{\def\childdoctmp{\childdocforward[#1]{#3##1}}}
    \expandafter\childdocextract\childdocname~~~
    \expandafter
  \endgroup
  \childdoctmp
}
%    \end{macrocode}

% \macro{\childdoc}
% The deprecated macro |\childdoc| is a legacy version of |\childdocmain|:
%    \begin{macrocode}
\newcommand{\childdoc}{\childdocmain}
%    \end{macrocode}

% \macro{\childdocredirect}
% The deprecated macro |\childdocredirect| is a legacy version
% of |\childdocforward| and |\childdocforwardprefix|:
%    \begin{macrocode}
\newcommand{\childdocredirect}[2][]
{
  \begingroup
    \if?#1?
      \def\childdoctmp{\childdocforward{#2}}
    \else
      \def\childdoctmp{\childdocforwardprefix{#1}{#2}}
    \fi
    \expandafter
  \endgroup
  \childdoctmp
}
%    \end{macrocode}

%\iffalse
%</package>
%\fi
%
\endinput
|\\
|\childdocforwardprefix[|\textit{main}|]{|\textit{prefix}|}{|\textit{dest}|}|
\end{tabular}
\end{center}
%
the destination file is determined by a pattern
depending on the current file:
To make this work, the current file must be called
`{\textit{prefix}\hspace{0.2em}\textit{suffix}}'
with \textit{prefix} matching precisely the argument.
Processing is then passed on to the file
`{\textit{dest}\hspace{0.2em}\textit{suffix}}'.
Surely, the same effect is achieved by
directly specifying the
argument `{\textit{dest}\hspace{0.2em}\textit{suffix}}'
in the first form.
However, that requires to set up a different file
for each child. With the alternative form of the command
all these files can have exactly the same content
which simplifies setting them up and maintaining them.

For example, the following file |draft.tex|
with a compilation flag |\version| as described in \secref{sec:flags}
compiles the main document as a draft:
%
\begin{center}
\begin{tabular}{l}
|\def\version{draft}|\\
|% \iffalse
%
% childdoc.dtx Copyright (C) 2017-2018 Niklas Beisert
%
% This work may be distributed and/or modified under the
% conditions of the LaTeX Project Public License, either version 1.3
% of this license or (at your option) any later version.
% The latest version of this license is in
%   http://www.latex-project.org/lppl.txt
% and version 1.3 or later is part of all distributions of LaTeX
% version 2005/12/01 or later.
%
% This work has the LPPL maintenance status `maintained'.
%
% The Current Maintainer of this work is Niklas Beisert.
%
% This work consists of the files childdoc.dtx and childdoc.ins
% and the derived files childdoc.def and cdocsamp.tex with
% cdocsch1.tex, cdocsch2.tex, cdocsdrf.tex, cdocsfn1.tex, cdocsfn2.tex.
%
%<package>\ifdefined\childdocmain\endinput\fi
%<package>\ProvidesFile{childdoc.def}[2018/12/30 v2.0 child document driver]
%<samplemain>\ProvidesFile{cdocsamp.tex}[2018/12/30 v2.0 sample for childdoc]
%<*driver>
%\ProvidesFile{childdoc.drv}[2018/12/30 v2.0 childdoc reference manual file]
\PassOptionsToClass{10pt,a4paper}{article}
\documentclass{ltxdoc}

\usepackage[margin=35mm]{geometry}
\usepackage{hyperref}
\usepackage{hyperxmp}
\usepackage[usenames]{color}

\hypersetup{colorlinks=true}
\hypersetup{pdfstartview=FitH}
\hypersetup{pdfpagemode=UseNone}
\hypersetup{pdfsource={}}
\hypersetup{pdflang={en-UK}}
\hypersetup{pdfcopyright={Copyright 2017-2018 Niklas Beisert.
  This work may be distributed and/or modified under the
  conditions of the LaTeX Project Public License, either version 1.3
  of this license or (at your option) any later version.}}
\hypersetup{pdflicenseurl={http://www.latex-project.org/lppl.txt}}
\hypersetup{pdfcontactaddress={ETH Zurich, ITP, HIT K,
  Wolfgang-Pauli-Strasse 27}}
\hypersetup{pdfcontactpostcode={8093}}
\hypersetup{pdfcontactcity={Zurich}}
\hypersetup{pdfcontactcountry={Switzerland}}
\hypersetup{pdfcontactemail={nbeisert@itp.phys.ethz.ch}}
\hypersetup{pdfcontacturl={http://people.phys.ethz.ch/\xmptilde nbeisert/}}

\newcommand{\secref}[1]{\hyperref[#1]{section \ref*{#1}}}

\parskip1ex
\parindent0pt
\let\olditemize\itemize
\def\itemize{\olditemize\parskip0pt}

\begin{document}

\title{The \textsf{childdoc} Package}
\hypersetup{pdftitle={The childdoc Package}}
\author{Niklas Beisert\\[2ex]
  Institut f\"ur Theoretische Physik\\
  Eidgen\"ossische Technische Hochschule Z\"urich\\
  Wolfgang-Pauli-Strasse 27, 8093 Z\"urich, Switzerland\\[1ex]
  \href{mailto:nbeisert@itp.phys.ethz.ch}
  {\texttt{nbeisert@itp.phys.ethz.ch}}}
\hypersetup{pdfauthor={Niklas Beisert}}
\hypersetup{pdfsubject={Manual for the LaTeX2e Package childdoc}}
\date{30 December 2018, \textsf{v2.0}}
\maketitle

\begin{abstract}\noindent
\textsf{childdoc} is a \LaTeXe{} package
that enables the direct compilation
of document sections included by |\include|
to individual files.
\end{abstract}

\begingroup
\parskip0ex
\tableofcontents
\endgroup

%%%%%%%%%%%%%%%%%%%%%%%%%%%%%%%%%%%%%%%%%%%%%%%%%%%%%%%%%%%%%%%%%%%%%%%%%%%%%%%%
%%%%%%%%%%%%%%%%%%%%%%%%%%%%%%%%%%%%%%%%%%%%%%%%%%%%%%%%%%%%%%%%%%%%%%%%%%%%%%%%
\section{Introduction}

\LaTeX{} provides a mechanism to structure a large document (such as a book)
into a main file and several child files (containing the chapters)
using the |\include| command.
This mechanism is beneficial for documents
which span hundreds of pages in order to
make the source file(s) more manageable.
Moreover, compilation can be restricted to
selected child files by means of the |\includeonly| command.
The latter feature can be used to reduce the compilation time while editing
(this was significantly more useful in the earlier days of \LaTeX{})
or to generate a smaller document which is easier to navigate.
Another application of |\includeonly| is to generate
documents consisting of selected parts of the complete document.

However, there are a few drawbacks of the plain |\include| mechanism:
\begin{itemize}
\item
The child files cannot be compiled on their own,
they can only be compiled via the main file.
A naive editing environment
(such as a text editor with an option
to have the current file processed by \LaTeX)
may require one to switch to the main file before compiling;
attempting to compile the child file produces errors.
\item
The main file must be modified (each time)
to adjust the |\includeonly| command
to the present needs. This easily leaves the main file in a messy state.
\item
The generated document will always carry the filename
of the main document. This is inconvenient if
several child files are to be compiled and
to be kept for distribution.
\end{itemize}

The present package provides a simple interface
to make child files individually compilable by \LaTeX{}.
Compiling a child file then has the same effect as compiling
the main file with an |\includeonly| command
to select the appropriate child.
Moreover the generated document will carry the name of the child
rather than the main file.
This resolves all three above issues.

This feature is meant to make the editing of books,
thesis documents and lecture notes somewhat more convenient.
However, the package can also be used efficiently for
composing a series of documents (such as exercise sheets)
which are typically distributed individually.
It then assists the author in generating the individual documents
(potentially in different versions)
as well as a document containing the collected series.
Another application is in developing style files
or other kinds of included material
where compilation of the style file could redirect
to a sample or test file.

%%%%%%%%%%%%%%%%%%%%%%%%%%%%%%%%%%%%%%%%%%%%%%%%%%%%%%%%%%%%%%%%%%%%%%%%%%%%%%%%
%%%%%%%%%%%%%%%%%%%%%%%%%%%%%%%%%%%%%%%%%%%%%%%%%%%%%%%%%%%%%%%%%%%%%%%%%%%%%%%%
\section{Usage}

First of all, the package \textsf{childdoc} is \emph{not} a standard
\LaTeXe{} |.sty| style file! Therefore it needs to be invoked in
a non-standard way.

%%%%%%%%%%%%%%%%%%%%%%%%%%%%%%%%%%%%%%%%%%%%%%%%%%%%%%%%%%%%%%%%%%%%%%%%%%%%%%%%
\subsection{Included Files}
\label{sec:include}

%%%%%%%%%%%%%%%%%%%%%%%%%%%%%%%%%%%%%%%%
\DescribeMacro{\childdocmain}
To use the package, add the commands
\begin{center}
\begin{tabular}{l}
|% \iffalse
%
% childdoc.dtx Copyright (C) 2017-2018 Niklas Beisert
%
% This work may be distributed and/or modified under the
% conditions of the LaTeX Project Public License, either version 1.3
% of this license or (at your option) any later version.
% The latest version of this license is in
%   http://www.latex-project.org/lppl.txt
% and version 1.3 or later is part of all distributions of LaTeX
% version 2005/12/01 or later.
%
% This work has the LPPL maintenance status `maintained'.
%
% The Current Maintainer of this work is Niklas Beisert.
%
% This work consists of the files childdoc.dtx and childdoc.ins
% and the derived files childdoc.def and cdocsamp.tex with
% cdocsch1.tex, cdocsch2.tex, cdocsdrf.tex, cdocsfn1.tex, cdocsfn2.tex.
%
%<package>\ifdefined\childdocmain\endinput\fi
%<package>\ProvidesFile{childdoc.def}[2018/12/30 v2.0 child document driver]
%<samplemain>\ProvidesFile{cdocsamp.tex}[2018/12/30 v2.0 sample for childdoc]
%<*driver>
%\ProvidesFile{childdoc.drv}[2018/12/30 v2.0 childdoc reference manual file]
\PassOptionsToClass{10pt,a4paper}{article}
\documentclass{ltxdoc}

\usepackage[margin=35mm]{geometry}
\usepackage{hyperref}
\usepackage{hyperxmp}
\usepackage[usenames]{color}

\hypersetup{colorlinks=true}
\hypersetup{pdfstartview=FitH}
\hypersetup{pdfpagemode=UseNone}
\hypersetup{pdfsource={}}
\hypersetup{pdflang={en-UK}}
\hypersetup{pdfcopyright={Copyright 2017-2018 Niklas Beisert.
  This work may be distributed and/or modified under the
  conditions of the LaTeX Project Public License, either version 1.3
  of this license or (at your option) any later version.}}
\hypersetup{pdflicenseurl={http://www.latex-project.org/lppl.txt}}
\hypersetup{pdfcontactaddress={ETH Zurich, ITP, HIT K,
  Wolfgang-Pauli-Strasse 27}}
\hypersetup{pdfcontactpostcode={8093}}
\hypersetup{pdfcontactcity={Zurich}}
\hypersetup{pdfcontactcountry={Switzerland}}
\hypersetup{pdfcontactemail={nbeisert@itp.phys.ethz.ch}}
\hypersetup{pdfcontacturl={http://people.phys.ethz.ch/\xmptilde nbeisert/}}

\newcommand{\secref}[1]{\hyperref[#1]{section \ref*{#1}}}

\parskip1ex
\parindent0pt
\let\olditemize\itemize
\def\itemize{\olditemize\parskip0pt}

\begin{document}

\title{The \textsf{childdoc} Package}
\hypersetup{pdftitle={The childdoc Package}}
\author{Niklas Beisert\\[2ex]
  Institut f\"ur Theoretische Physik\\
  Eidgen\"ossische Technische Hochschule Z\"urich\\
  Wolfgang-Pauli-Strasse 27, 8093 Z\"urich, Switzerland\\[1ex]
  \href{mailto:nbeisert@itp.phys.ethz.ch}
  {\texttt{nbeisert@itp.phys.ethz.ch}}}
\hypersetup{pdfauthor={Niklas Beisert}}
\hypersetup{pdfsubject={Manual for the LaTeX2e Package childdoc}}
\date{30 December 2018, \textsf{v2.0}}
\maketitle

\begin{abstract}\noindent
\textsf{childdoc} is a \LaTeXe{} package
that enables the direct compilation
of document sections included by |\include|
to individual files.
\end{abstract}

\begingroup
\parskip0ex
\tableofcontents
\endgroup

%%%%%%%%%%%%%%%%%%%%%%%%%%%%%%%%%%%%%%%%%%%%%%%%%%%%%%%%%%%%%%%%%%%%%%%%%%%%%%%%
%%%%%%%%%%%%%%%%%%%%%%%%%%%%%%%%%%%%%%%%%%%%%%%%%%%%%%%%%%%%%%%%%%%%%%%%%%%%%%%%
\section{Introduction}

\LaTeX{} provides a mechanism to structure a large document (such as a book)
into a main file and several child files (containing the chapters)
using the |\include| command.
This mechanism is beneficial for documents
which span hundreds of pages in order to
make the source file(s) more manageable.
Moreover, compilation can be restricted to
selected child files by means of the |\includeonly| command.
The latter feature can be used to reduce the compilation time while editing
(this was significantly more useful in the earlier days of \LaTeX{})
or to generate a smaller document which is easier to navigate.
Another application of |\includeonly| is to generate
documents consisting of selected parts of the complete document.

However, there are a few drawbacks of the plain |\include| mechanism:
\begin{itemize}
\item
The child files cannot be compiled on their own,
they can only be compiled via the main file.
A naive editing environment
(such as a text editor with an option
to have the current file processed by \LaTeX)
may require one to switch to the main file before compiling;
attempting to compile the child file produces errors.
\item
The main file must be modified (each time)
to adjust the |\includeonly| command
to the present needs. This easily leaves the main file in a messy state.
\item
The generated document will always carry the filename
of the main document. This is inconvenient if
several child files are to be compiled and
to be kept for distribution.
\end{itemize}

The present package provides a simple interface
to make child files individually compilable by \LaTeX{}.
Compiling a child file then has the same effect as compiling
the main file with an |\includeonly| command
to select the appropriate child.
Moreover the generated document will carry the name of the child
rather than the main file.
This resolves all three above issues.

This feature is meant to make the editing of books,
thesis documents and lecture notes somewhat more convenient.
However, the package can also be used efficiently for
composing a series of documents (such as exercise sheets)
which are typically distributed individually.
It then assists the author in generating the individual documents
(potentially in different versions)
as well as a document containing the collected series.
Another application is in developing style files
or other kinds of included material
where compilation of the style file could redirect
to a sample or test file.

%%%%%%%%%%%%%%%%%%%%%%%%%%%%%%%%%%%%%%%%%%%%%%%%%%%%%%%%%%%%%%%%%%%%%%%%%%%%%%%%
%%%%%%%%%%%%%%%%%%%%%%%%%%%%%%%%%%%%%%%%%%%%%%%%%%%%%%%%%%%%%%%%%%%%%%%%%%%%%%%%
\section{Usage}

First of all, the package \textsf{childdoc} is \emph{not} a standard
\LaTeXe{} |.sty| style file! Therefore it needs to be invoked in
a non-standard way.

%%%%%%%%%%%%%%%%%%%%%%%%%%%%%%%%%%%%%%%%%%%%%%%%%%%%%%%%%%%%%%%%%%%%%%%%%%%%%%%%
\subsection{Included Files}
\label{sec:include}

%%%%%%%%%%%%%%%%%%%%%%%%%%%%%%%%%%%%%%%%
\DescribeMacro{\childdocmain}
To use the package, add the commands
\begin{center}
\begin{tabular}{l}
|\input{childdoc.def}|\\
|\childdocmain{}|\\
\end{tabular}
\end{center}
at the very top of the main \LaTeX{} file,
in particular \emph{before} the |\documentclass| statement!
The argument of |\childdocmain| should be left empty
(but it must be present).

%%%%%%%%%%%%%%%%%%%%%%%%%%%%%%%%%%%%%%%%
\DescribeMacro{\childdocof}
Furthermore, add the commands
\begin{center}
\begin{tabular}{l}
|\input{childdoc.def}|\\
|\childdocof{|\textit{main}|}|\\
\end{tabular}
\end{center}
at the top of every child file \textit{child}
which is included by |\include{|\textit{child}|}|
from within the main file
(or at least for those files to be compiled individually).
The argument \textit{main} must be the filename of the main file.

There are a couple of
considerations in setting up the main and child documents:

%%%%%%%%%%%%%%%%%%%%%%%%%%%%%%%%%%%%%%%%
\paragraph{Restrictions.}

Please note the following restrictions:
\begin{itemize}
\item
|\childdocmain| must be called with one argument \textit{main}
to ensure compatibility with earlier version of the package.
It must either be empty (|\childdocmain{}|)
or precisely match the filename of the main file in which it is specified.
See \secref{sec:detection} for further information.
\item
The filename \textit{main} must be specified without the |.tex| extension.
\item
The filename \textit{main} is case sensitive
(even in case-insensitive file systems)
due to internal string comparison.
\item
The argument \textit{main} should be fully expanded, it cannot be a macro.
\item
Subdirectories and special characters should be avoided in filenames.
\item
The command |\childdocmain{|\textit{main}|}| must be followed by a whitespace.
It should not be followed immediately by another command
or by a comment mark `|%|'.
This is because the \TeX{} parser reads the token immediately following
the argument of |\childdocmain| and puts it
at the beginning of every child section;
however, a white\-space is ignored.
\end{itemize}

%%%%%%%%%%%%%%%%%%%%%%%%%%%%%%%%%%%%%%%%
\paragraph{Content of Main File.}

It is advisable to place all content in the child files included by |\include|.
Any output contained in the main file will appear in all child documents
unless suppressed manually;
it cannot be suppressed automatically by the |\includeonly| directive
and thus should normally be avoided.
A method to include some content in the main file
by means of conditional processing is described in \secref{sec:conditional}.

%%%%%%%%%%%%%%%%%%%%%%%%%%%%%%%%%%%%%%%%
\paragraph{Page Numbering.}

When only a part of the document is compiled,
the appropriate numbering of pages
(as well as other status parameters)
is determined from the |.aux| files.
The latter contain information from previous passes.
However this information needs to propagate through
all intermediate child documents.
Therefore the page numbering in child documents may well
be inconsistent until the complete document is compiled at least once.

A useful (if unconventional) way to always ensure a consistent
page numbering is to restart the numbering in each child document
and denote the pages by `\textit{child}|.|\textit{page}'
where \textit{child} represents the chapter/section number of the child file.
This can be achieved by the command
|\numberwithin{page}{|\textit{child}|}|
of the \textsf{amsmath} package
where \textit{child} can be |chapter| or |section|
depending on the chosen structuring.
Alternatively, one can modify the macro |\thepage| appropriately
and reset the counter |page| at the start of each child file.

%%%%%%%%%%%%%%%%%%%%%%%%%%%%%%%%%%%%%%%%%%%%%%%%%%%%%%%%%%%%%%%%%%%%%%%%%%%%%%%%
\subsection{Conditional Processing}
\label{sec:conditional}

The package provides a mechanism to compile different versions
of a document. To customise the versions further some conditional processing
can come in handy to distinguish which version is being compiled.
The package provides two macros to describe the compilation context:

%%%%%%%%%%%%%%%%%%%%%%%%%%%%%%%%%%%%%%%%
\DescribeMacro{\ifchilddoc}
The conditional |\ifchilddoc| distinguishes between the compilation of
child documents and the main document:
%
\begin{center}
|\ifchilddoc |\textit{child-code}| |[|\||else |\textit{main-code}]| \||fi|
\end{center}

%%%%%%%%%%%%%%%%%%%%%%%%%%%%%%%%%%%%%%%%
\DescribeMacro{\childdocname}
\DescribeMacro{\childdocjob}
The macro |\childdocname| contains the filename (without extension)
of the main or child file being processed.
Note that |\childdocjob| will always contain the name of the main file.

%%%%%%%%%%%%%%%%%%%%%%%%%%%%%%%%%%%%%%%%
\paragraph{Title Page.}

Conditional processing can be used to include a title or banner page
in the main document when proper precautions are taken.
Importantly, the code in the main file should ensure that the page counter
(as well as other status parameters which are stored in the |.aux| files)
takes the same value after the conditional processing.
Otherwise the page numbers may take divergent values
depending on which part is compiled.

For example, a title page could be declared by:
%
\begin{center}
\begin{tabular}{l}
|\ifchilddoc\||else|\\
|\addtocounter{page}{-1}|\\
\textit{code for title page}\\
|\newpage|\\
|\||fi|
\end{tabular}
\end{center}
%
A banner page for the child documents can be generated by:
%
\begin{center}
\begin{tabular}{l}
|\ifchilddoc|\\
|\addtocounter{page}{-1}|\\
\textit{code for banner page}\\
|\newpage|\\
|\||fi|
\end{tabular}
\end{center}
%
Here one could write a message such as:
\begin{center}
|This is the part \childdocname{} of \childdocjob{}.|
\end{center}

%%%%%%%%%%%%%%%%%%%%%%%%%%%%%%%%%%%%%%%%%%%%%%%%%%%%%%%%%%%%%%%%%%%%%%%%%%%%%%%%
\subsection{Flags}
\label{sec:flags}

The package makes it easy to generate different versions
of the main or child documents.
To this end compilation flags can be defined
and assigned different default values.
They will be particularly useful in conjunction
with the forwarding mechanism described in \secref{sec:forward}.

For example, it may be useful to have a flag |\version|
which can be set to |draft| or |final|.
The document source will contain some conditional code
depending on the value of |\version|.
Suppose further, the flag should default to |final| for the main file
and to |draft| for child files
which is a natural assignment for editing the document.
This is achieved by placing the following code
in the preamble of the main document
(below the |\childdocmain| directive):
%
\begin{center}
\begin{tabular}{l}
|\ifchilddoc|\\
|\providecommand{\version}{draft}|\\
|\||else|\\
|\providecommand{\version}{final}|\\
|\||fi|
\end{tabular}
\end{center}
%
The definition by |\providecommand| makes sure
that previous definitions are not overwritten.
Further statements |\providecommand{\version}{...}|
can thus be added before the above code to override it.

For the main file, one might add a line
(between |\childdocmain| and the above block)
%
\begin{center}
|%\ifchilddoc\||else\providecommand{\version}{draft}\||fi|
\end{center}
%
which can be uncommented to produce a draft version.
Likewise one can add a line to the very top of a child file
(above the |\childdocof{|\textit{main}|}| directive)
%
\begin{center}
|%\providecommand{\version}{final}|
\end{center}
%
which can be uncommented to produce the final version of this child document.

%%%%%%%%%%%%%%%%%%%%%%%%%%%%%%%%%%%%%%%%%%%%%%%%%%%%%%%%%%%%%%%%%%%%%%%%%%%%%%%%
\subsection{Forwarding}
\label{sec:forward}

Different versions of the main or child documents
using compilation flags as described in \secref{sec:flags}
can be (permanently) stored in different files
for convenient compilation, viewing and distribution.
To this end, the package defines a command
to pass on compilation to a different file:

%%%%%%%%%%%%%%%%%%%%%%%%%%%%%%%%%%%%%%%%
\DescribeMacro{\childdocforward}
The command |\childdocforward| redirects processing to
another source file:
%
\begin{center}
\begin{tabular}{l}
|\input{childdoc.def}|\\
|\childdocforward[|\textit{main}|]{|\textit{dest}|}|\\
\end{tabular}
\end{center}
%
The argument \textit{dest} is the destination file
(without extension).
It should be the main file or one of the child files.
Note that further \textsf{childdoc} directives
such as |\childdocof| and |\childdocforward|
in the indicated file will be processed in this form.
The optional argument \textit{main}
passes on directly to the main file \textit{main}
while pretending to compile the child \textit{dest}.
This form behaves as if \textit{dest}
issues |\childdocof{|\textit{main}|}| right away,
and no further \textsf{childdoc} directives will be processed.

%%%%%%%%%%%%%%%%%%%%%%%%%%%%%%%%%%%%%%%%
\DescribeMacro{\...prefix}
In the alternative form |\childdocforwardprefix|,
%
\begin{center}
\begin{tabular}{l}
|\input{childdoc.def}|\\
|\childdocforwardprefix[|\textit{main}|]{|\textit{prefix}|}{|\textit{dest}|}|
\end{tabular}
\end{center}
%
the destination file is determined by a pattern
depending on the current file:
To make this work, the current file must be called
`{\textit{prefix}\hspace{0.2em}\textit{suffix}}'
with \textit{prefix} matching precisely the argument.
Processing is then passed on to the file
`{\textit{dest}\hspace{0.2em}\textit{suffix}}'.
Surely, the same effect is achieved by
directly specifying the
argument `{\textit{dest}\hspace{0.2em}\textit{suffix}}'
in the first form.
However, that requires to set up a different file
for each child. With the alternative form of the command
all these files can have exactly the same content
which simplifies setting them up and maintaining them.

For example, the following file |draft.tex|
with a compilation flag |\version| as described in \secref{sec:flags}
compiles the main document as a draft:
%
\begin{center}
\begin{tabular}{l}
|\def\version{draft}|\\
|\input{childdoc.def}|\\
|\childdocforward{|\textit{main}|}|
\end{tabular}
\end{center}
%
Likewise, the following files |final|\textit{nn}|.tex|
compile the final version of the child document
|child|\textit{nn}|.tex|:
%
\begin{center}
\begin{tabular}{l}
|\def\version{final}|\\
|\input{childdoc.def}|\\
|\childdocforwardprefix{final}{child}|
\end{tabular}
\end{center}
%

Note that when several versions of a main file and/or of each child file
are to be generated, it may be convenient to set up a |Makefile| or
shell script to automatise the process.

%%%%%%%%%%%%%%%%%%%%%%%%%%%%%%%%%%%%%%%%%%%%%%%%%%%%%%%%%%%%%%%%%%%%%%%%%%%%%%%%
\subsection{Command Line Processing}
\label{sec:commandline}

The effect of redirection files can also be achieved by invoking
the \LaTeX{} compiler with a more elaborate command line.
Most conveniently this should be done as part
of a shell script or a |Makefile|.

When using \textsf{childdoc} in the main file, the following
command lines effectively perform a redirection
(note that depending on the shell being used,
backslashes may have to be doubled: `|\|' $\to$ `|\\|'):
%
\begin{center}
|... -jobname "|\textit{target}|" |\\|"|[\textit{flags}]%
|\input{childdoc.def}\childdocforward[|\textit{main}|]{|\textit{dest}|}"|
\end{center}
%
Here \textit{target} is the name of the output file,
\textit{main} is the name of the main file
and \textit{dest} is the name of the main or child file to be processed
(all filenames without extensions).
The optional argument \textit{main} can be omitted
if \textit{main} matches \textit{dest}.
Optionally, compilation \textit{flags} can be defined via |\def| commands.
This command line makes the \TeX{} engine believe
it is compiling the file \textit{target}
whose content is specified as the latter parameter.
The provided code then forwards the processing to
\textit{main} or \textit{dest} as described in \secref{sec:forward}.

%%%%%%%%%%%%%%%%%%%%%%%%%%%%%%%%%%%%%%%%%%%%%%%%%%%%%%%%%%%%%%%%%%%%%%%%%%%%%%%%
\subsection{Include by Input}
\label{sec:input}

Including child documents by |\include| has some restrictions by design.
Most notably, the content of a child document always occupies
its own set of pages; pages cannot be shared between child documents.
Usually, this behaviour makes perfect sense
because each child document contain an essential part of the document.
However, in some situations it may be desirable to compose
a document from a collection of parts
without having mandatory page breaks between then.
For this case, the package
provides a mechanism to include parts
by |\input| which can also be processed individually.
However, by construction this mechanism
requires manual handling of the content to be output.

%%%%%%%%%%%%%%%%%%%%%%%%%%%%%%%%%%%%%%%%
\DescribeMacro{\ifchilddocmanual}
The main file should be prepared as usual, see \secref{sec:include}.
However, the document body must make a distinction
between processing of an individual part and of the main document, e.g.:
%
\begin{center}
\begin{tabular}{l}
|\ifchilddocmanual|\\
|\input{\childdocname}|\\
|\||else|\\
\textit{document body with }|\input{|\textit{part}|}|\\
|\||fi|
\end{tabular}
\end{center}
%
The conditional |\ifchilddocmanual| is true whenever
a part to be included by |\input| is being compiled,
and the name of the part is stored in |\childdocname|.

%%%%%%%%%%%%%%%%%%%%%%%%%%%%%%%%%%%%%%%%
\DescribeMacro{\childdocby}
Each part to be included by |\input| should start with:
%
\begin{center}
\begin{tabular}{l}
|\input{childdoc.def}|\\
|\childdocby{|\textit{main}|}|\\
\end{tabular}
\end{center}
%
The directive |\childdocby| is similar to |\childdocof|
described in \secref{sec:include},
but the subsequent selection of content must be done manually.
To that end, both |\ifchilddoc| and |\ifchilddocmanual|
will be true upon processing of a part,
and the name of the part is stored in |\childdocname|.
Note that |\jobname| will be set to the filename of the current part
so that each part receives an individual |.aux| file
that does not interfere with the |.aux| file(s) of the main document.
This behaviour can be altered by the alternative form
|\childdocby[*]{|\textit{main}|}| (with a non-empty optional argument)
which uses the |.aux| file of the main document
by setting |\jobname| to \textit{main}.

%%%%%%%%%%%%%%%%%%%%%%%%%%%%%%%%%%%%%%%%%%%%%%%%%%%%%%%%%%%%%%%%%%%%%%%%%%%%%%%%
\subsection{Driver Development}
\label{sec:driver}

The \textsf{childdoc} mechanism can also be use for the development
of definition files such as \LaTeX{} styles or classes.
This case differs from the above setup with multiple parts
included by |\include| in that no |\includeonly| should be invoked.
This can be achieved by starting the include file
(before |\ProvidesPackage|) with:
%
\begin{center}
\begin{tabular}{l}
|\input{childdoc.def}|\\
|\childdocforward{|\textit{main}|}|\\
\end{tabular}
\end{center}
%
or alternatively with:
%
\begin{center}
\begin{tabular}{l}
|\input{childdoc.def}|\\
|\childdocby{|\textit{main}|}|\\
\end{tabular}
\end{center}
%
Both forms have slightly different effects as described above.
The main file is prepared as usual, see \secref{sec:include}.

%%%%%%%%%%%%%%%%%%%%%%%%%%%%%%%%%%%%%%%%%%%%%%%%%%%%%%%%%%%%%%%%%%%%%%%%%%%%%%%%
\subsection{Legacy Detection}
\label{sec:detection}

The directive |\childdocmain| in the main file can detect
whether the complete document or merely a child is to be compiled
even without using the directive |\childdocof|.
This method is deprecated because it is less robust
and there is no compelling reason to use it;
it is merely provided for backward compatibility
and it may be removed in future versions.

If the detection mechanism is to be used,
it is mandatory to correctly specify
the filename of the main file as the argument of |\childdocmain|:
%
\begin{center}
\begin{tabular}{l}
|\input{childdoc.def}|\\
|\childdocmain{|\textit{main}|}|\\
\end{tabular}
\end{center}
%
If |\jobname| does not match the argument \textit{main} of |\childdocmain|,
it is assumed that |\jobname| points to the child file to be compiled.
When using |\childdocmain| with the main file specified as argument,
it suffices to start a child file
with just |\input{|\textit{main}|}|
without loading of the package and using |\childdocof|.
If instead all processing is done
with the appropriate \textsf{childdoc} directives,
the argument of \textit{main} of |\childdocmain| can be empty.

An alternative version of the command line processing described
in \secref{sec:commandline} using the detection mechanism reads:
%
\begin{center}
|... -jobname "|\textit{target}|" "|[\textit{flags}]%
[|\def\jobname{|\textit{dest}|}|]|\input{|\textit{main}|}"|
\end{center}

%%%%%%%%%%%%%%%%%%%%%%%%%%%%%%%%%%%%%%%%%%%%%%%%%%%%%%%%%%%%%%%%%%%%%%%%%%%%%%%%
\subsection{Manual Code}
\label{sec:manual}

In case one cannot be certain whether the definitions file |childdoc.def|
is installed on the target \TeX{} distribution
and one prefers not to ship it,
it is conceivable to paste a few relevant commands into the sources.

To that end, drop all statements |\input{childdoc.def}|
and perform the replacements as outlined below.
Instead of |\childdocmain{|\textit{main}|}| add the following code
to the top of the main file:
%
\begin{center}
\begin{tabular}{l}
|\||ifdefined\childdocname\endinput\||fi\newif\ifchilddoc|\\
|\edef\childdocname{\scantokens\expandafter{\jobname\noexpand}}|\\
|\def\childdocmain{|\textit{main}|}\||ifx\childdocmain\childdocname\||else|\\
|\childdoctrue\includeonly{\childdocname}\let\jobname\childdocmain\||fi|\\
\end{tabular}
\end{center}
%
Instead of |\childdocof{|\textit{main}|}| just include the main file
at the top of each child file:
%
\begin{center}
|\input{|\textit{main}|}|
\end{center}
%
A simple redirection |\childdocforward{|\textit{dest}|}| is achieved by:
%
\begin{center}
|\def\jobname{|\textit{dest}|}\input{\jobname}|
\end{center}
%
The redirection with prefix
|\childdocforwardprefix[|\textit{prefix}|]{|\textit{dest}|}|
is accomplished by:
%
\begin{center}
\begin{tabular}{l}
|{\edef\jobname{\scantokens\expandafter{\jobname\noexpand}}|\\
|\def\redirectjob |\textit{prefix}|#1~~~{\gdef\jobname{|\textit{dest}|#1}}|\\
|\expandafter\redirectjob\jobname~~~}\input{\jobname}|
\end{tabular}
\end{center}

In an alternative approach,
child documents can be compiled by a specific command line
without additional code or specific definitions:
%
\begin{center}
|... -jobname "|\textit{target}|" "|[\textit{flags}]%
|\includeonly{|\textit{dest}|}\input{|\textit{main}|}"|
\end{center}
%

%%%%%%%%%%%%%%%%%%%%%%%%%%%%%%%%%%%%%%%%%%%%%%%%%%%%%%%%%%%%%%%%%%%%%%%%%%%%%%%%
%%%%%%%%%%%%%%%%%%%%%%%%%%%%%%%%%%%%%%%%%%%%%%%%%%%%%%%%%%%%%%%%%%%%%%%%%%%%%%%%
\section{Information}

%%%%%%%%%%%%%%%%%%%%%%%%%%%%%%%%%%%%%%%%%%%%%%%%%%%%%%%%%%%%%%%%%%%%%%%%%%%%%%%%
\subsection{Copyright}

Copyright \copyright{} 2017--2018 Niklas Beisert

This work may be distributed and/or modified under the
conditions of the \LaTeX{} Project Public License, either version 1.3
of this license or (at your option) any later version.
The latest version of this license is in
  \url{http://www.latex-project.org/lppl.txt}
and version 1.3 or later is part of all distributions of \LaTeX{}
version 2005/12/01 or later.

This work has the LPPL maintenance status `maintained'.

The Current Maintainer of this work is Niklas Beisert.

This work consists of the files |README.txt|, |childdoc.ins| and |childdoc.dtx|
as well as the derived files |childdoc.def|, |cdocsamp.tex|
with |cdocsch1.tex|, |cdocsch2.tex|, |cdocspt3.tex|, |cdocspt4.tex|,
|cdocsdrf.tex|, |cdocsfn1.tex|, |cdocsfn2.tex|
as well as |childdoc.pdf|.

%%%%%%%%%%%%%%%%%%%%%%%%%%%%%%%%%%%%%%%%%%%%%%%%%%%%%%%%%%%%%%%%%%%%%%%%%%%%%%%%
\subsection{Files and Installation}

The package consists of the files:
%
\begin{center}
\begin{tabular}{ll}
    |README.txt|   & readme file \\
    |childdoc.ins| & installation file \\
    |childdoc.dtx| & source file \\
    |childdoc.def| & definition file \\
    |cdocsamp.tex| & sample main file \\
    |cdocsch1.tex| & sample include file \\
    |cdocsch2.tex| & sample include file \\
    |cdocspt3.tex| & sample part file \\
    |cdocspt4.tex| & sample part file \\
    |cdocsdrf.tex| & sample redirection file \\
    |cdocsfn1.tex| & sample redirection file \\
    |cdocsfn2.tex| & sample redirection file \\
    |childdoc.pdf| & manual
\end{tabular}
\end{center}
%
The distribution consists of the files
|README.txt|, |childdoc.ins| and |childdoc.dtx|.
%
\begin{itemize}
\item
Run (pdf)\LaTeX{} on |childdoc.dtx|
to compile the manual |childdoc.pdf| (this file).
\item
Run \LaTeX{} on |childdoc.ins| to create the definitions file |childdoc.def|
and the sample |cdocsamp.tex| with include files
|cdocsch1.tex|, |cdocsch2.tex|, |cdocspt3.tex|, |cdocspt4.tex|,
|cdocsdrf.tex|, |cdocsfn1.tex|, |cdocsfn2.tex|.
Then copy the file |childdoc.def| to an appropriate directory of your \LaTeX{}
distribution, e.g.\ \textit{texmf-root}|/tex/latex/childdoc|.
\end{itemize}

%%%%%%%%%%%%%%%%%%%%%%%%%%%%%%%%%%%%%%%%%%%%%%%%%%%%%%%%%%%%%%%%%%%%%%%%%%%%%%%%
\subsection{Related CTAN Packages}

There are several other packages which offer a similar functionality:
%
\begin{itemize}
\item
The packages
\href{http://ctan.org/pkg/docmute}{\textsf{docmute}},
\href{http://ctan.org/pkg/includex}{\textsf{includex}} and
\href{http://ctan.org/pkg/standalone}{\textsf{standalone}}
provide commands to include only the document body of
a child file thus allowing both files to be compiled individually.
\item
The packages \href{http://ctan.org/pkg/subdocs}{\textsf{subdocs}}
and \href{http://ctan.org/pkg/subfiles}{\textsf{subfiles}}
provide structures in which the main and child documents can be
encapsulated and allowing them to be compiled individually.
The inclusion mechanism is different from the conventional |\include|.
\item
The package \href{http://ctan.org/pkg/combine}{\textsf{combine}}
is an elaborate solution to combine several documents into one.
\end{itemize}
%
See also the CTAN topic \href{http://ctan.org/topic/subdocs}{\textsf{subdocs}}
for further related packages.
The present package differs from the above solutions in that
a document structure constructed with the conventional |\include| mechanism
just needs two extra commands at the top of every file
such that all constituent files can be compiled individually.

%%%%%%%%%%%%%%%%%%%%%%%%%%%%%%%%%%%%%%%%%%%%%%%%%%%%%%%%%%%%%%%%%%%%%%%%%%%%%%%%
%\subsection{Feature Suggestions}
%
%The following is a list of features which may be useful for future
%versions of this package:
%%
%\begin{itemize}
%\item
%\ldots
%\end{itemize}

%%%%%%%%%%%%%%%%%%%%%%%%%%%%%%%%%%%%%%%%%%%%%%%%%%%%%%%%%%%%%%%%%%%%%%%%%%%%%%%%
\subsection{Revision History}

%%%%%%%%%%%%%%%%%%%%%%%%%%%%%%%%%%%%%%%%
\paragraph{v2.0:} 2018/12/30

\begin{itemize}
\item
immediate forward processing
\item
added |\childdocby| mechanism
\item
manual restructured
\end{itemize}

%%%%%%%%%%%%%%%%%%%%%%%%%%%%%%%%%%%%%%%%
\paragraph{v1.6:} 2018/01/17

\begin{itemize}
\item
application for development of include files
\item
corrections to manual
\end{itemize}

%%%%%%%%%%%%%%%%%%%%%%%%%%%%%%%%%%%%%%%%
\paragraph{v1.5:} 2017/05/21

\begin{itemize}
\item
more complete structuring introduced
\item
|\childdocof| introduced
\item
|\childdoc| renamed to |\childdocmain|
\item
|\childredirect| renamed to |\childdocforward| and |\childdocforwardprefix|
and functionality expanded
\end{itemize}

%%%%%%%%%%%%%%%%%%%%%%%%%%%%%%%%%%%%%%%%
\paragraph{v1.0:} 2017/04/27

\begin{itemize}
\item
manual and install package
\item
first version published on CTAN
\end{itemize}

%%%%%%%%%%%%%%%%%%%%%%%%%%%%%%%%%%%%%%%%
\paragraph{v0.6:} 2017/04/26

\begin{itemize}
\item
redirection mechanism added
\end{itemize}

%%%%%%%%%%%%%%%%%%%%%%%%%%%%%%%%%%%%%%%%
\paragraph{v0.5:} 2017/04/26

\begin{itemize}
\item
functionality in definition file
\end{itemize}


%%%%%%%%%%%%%%%%%%%%%%%%%%%%%%%%%%%%%%%%%%%%%%%%%%%%%%%%%%%%%%%%%%%%%%%%%%%%%%%%
%%%%%%%%%%%%%%%%%%%%%%%%%%%%%%%%%%%%%%%%%%%%%%%%%%%%%%%%%%%%%%%%%%%%%%%%%%%%%%%%
%%%%%%%%%%%%%%%%%%%%%%%%%%%%%%%%%%%%%%%%%%%%%%%%%%%%%%%%%%%%%%%%%%%%%%%%%%%%%%%%
\appendix

\settowidth\MacroIndent{\rmfamily\scriptsize 000\ }

 \DocInput{childdoc.dtx}

\end{document}
%</driver>
% \fi
%
% %%%%%%%%%%%%%%%%%%%%%%%%%%%%%%%%%%%%%%%%%%%%%%%%%%%%%%%%%%%%%%%%%%%%%%%%%%%%%%
% %%%%%%%%%%%%%%%%%%%%%%%%%%%%%%%%%%%%%%%%%%%%%%%%%%%%%%%%%%%%%%%%%%%%%%%%%%%%%%
% \section{Sample}
%\iffalse
%<*samplemain>
%\fi
%
% The following presents a sample document
% with two chapters, two parts, a title page,
% a compile flag as well as three forwarding files to set the flag.
% It consists of eight |.tex| files:
% \begin{center}
% \begin{tabular}{ll}
% |cdocsamp.tex|&main file\\
% |cdocsch1.tex|&include file for chapter 1\\
% |cdocsch2.tex|&include file for chapter 2\\
% |cdocspt3.tex|&include file for part 3\\
% |cdocspt4.tex|&include file for part 4\\
% |cdocsdrf.tex|&forwarding file for main file in draft mode\\
% |cdocsfi1.tex|&forwarding file for final version of chapter 1\\
% |cdocsfi2.tex|&forwarding file for final version of chapter 2\\
% \end{tabular}
% \end{center}
% Each of the eight files can be compiled directly by the \LaTeX{} compiler.
%
% %%%%%%%%%%%%%%%%%%%%%%%%%%%%%%%%%%%%%%
% \paragraph{Main File.}
%
% The main file is called |cdocsamp.tex|.
%
% Load the \textsf{childdoc} definitions and
% declare the filename for the main document:
%    \begin{macrocode}
\input{childdoc.def}
\childdocmain{}
%    \end{macrocode}

% Optional override for |\version| flag:
%    \begin{macrocode}
%%\ifchilddoc\else\providecommand{\version}{draft}\fi
%    \end{macrocode}

% Define the default values for the |\version| flag
% (|final| for the main file and |draft| for childs):
%    \begin{macrocode}
\ifchilddoc
\providecommand{\version}{draft}
\else
\providecommand{\version}{final}
\fi
%    \end{macrocode}

% Load the standard document class:
%    \begin{macrocode}
\documentclass[12pt]{article}
%    \end{macrocode}

% Start the document body:
%    \begin{macrocode}
\begin{document}
%    \end{macrocode}

% Declare a title page.
% Print title, part of document being processed and version flag:
%    \begin{macrocode}
\addtocounter{page}{-1}
\begin{center}
{\LARGE\bfseries{}childdoc example\par}
\vspace{1cm}
\ifchilddoc
\ifchilddocmanual part\else chapter\fi:
`\childdocname' of `\childdocjob'\par
\else
main document: `\childdocjob'\par
\fi
version: \version\par
\end{center}
\newpage
%    \end{macrocode}

% Manually include selected file,
% otherwise process as usual:
%    \begin{macrocode}
\ifchilddocmanual
\section*{part `\childdocname'}
\input{\childdocname}
\else
%    \end{macrocode}

% Include the two chapters:
%    \begin{macrocode}
\include{cdocsch1}
\include{cdocsch2}
%    \end{macrocode}

% Include the two parts unless only chapters should be displayed:
%    \begin{macrocode}
\ifchilddoc\else
\section{part three}
\input{cdocspt3}
\section{part four}
\input{cdocspt4}
\fi
%    \end{macrocode}

% Process as usual until here:
%    \begin{macrocode}
\fi
%    \end{macrocode}

% End of document body:
%    \begin{macrocode}
\end{document}
%    \end{macrocode}
%\iffalse
%</samplemain>
%\fi
%
% %%%%%%%%%%%%%%%%%%%%%%%%%%%%%%%%%%%%%%
% \paragraph{Chapter Include Files.}
%
% The include files are called |cdocsch1.tex| and |cdocsch2.tex|.
%
%\iffalse
%<*samplechap1|samplechap2>
%\fi

% Optional override for |\version| flag:
%    \begin{macrocode}
%%\providecommand{\version}{final}
%    \end{macrocode}

% Include the main document:
%    \begin{macrocode}
\input{childdoc.def}
\childdocof{cdocsamp}
%    \end{macrocode}

%\iffalse
%</samplechap1|samplechap2>
%\fi
%
%\iffalse
%<*samplechap1>
%\fi
% Some text for chapter 1:
%    \begin{macrocode}
\section{one}
some text in chapter one
%    \end{macrocode}

%\iffalse
%</samplechap1>
%\fi
% Some text for chapter 2:
%\iffalse
%<*samplechap2>
%\fi
%    \begin{macrocode}
\section{two}
more text in chapter two
%    \end{macrocode}

%\iffalse
%</samplechap2>
%\fi
%
% %%%%%%%%%%%%%%%%%%%%%%%%%%%%%%%%%%%%%%
% \paragraph{Part Include Files.}
%
% The include files are called |cdocspt3.tex| and |cdocspt4.tex|.
%
%\iffalse
%<*samplepart3|samplepart4>
%\fi

% Optional override for |\version| flag:
%    \begin{macrocode}
%%\providecommand{\version}{final}
%    \end{macrocode}

% Include the main document:
%    \begin{macrocode}
\input{childdoc.def}
\childdocby{cdocsamp}
%    \end{macrocode}

%\iffalse
%</samplepart3|samplepart4>
%\fi
%
%\iffalse
%<*samplepart3>
%\fi
% Some text for part 3:
%    \begin{macrocode}
some text in part three
%    \end{macrocode}

%\iffalse
%</samplepart3>
%\fi
% Some text for part 4:
%\iffalse
%<*samplepart4>
%\fi
%    \begin{macrocode}
more text in part four
%    \end{macrocode}

%\iffalse
%</samplepart4>
%\fi
%
% %%%%%%%%%%%%%%%%%%%%%%%%%%%%%%%%%%%%%%
% \paragraph{Forwarding for a Complete Draft.}
%
% The following forwarding file |cdocsdrf.tex|
% compiles the main document in draft mode:
%\iffalse
%<*sampledraft>
%\fi
%    \begin{macrocode}
\def\version{draft}
\input{childdoc.def}
\childdocforward{cdocsamp}
%    \end{macrocode}

%\iffalse
%</sampledraft>
%\fi
%
% %%%%%%%%%%%%%%%%%%%%%%%%%%%%%%%%%%%%%%
% \paragraph{Forwarding for Final Version of the Chapters.}
%
% The following forwarding files |cdocsfn1.tex| and |cdocsfn2.tex|
% (with identical content)
% compile the final versions of the child documents
% |cdocsch1.tex| and |cdocsch2.tex|, respectively:
%\iffalse
%<*samplefinal>
%\fi
%    \begin{macrocode}
\def\version{final}
\input{childdoc.def}
\childdocforwardprefix[cdocsamp]{cdocsfn}{cdocsch}
%    \end{macrocode}

%\iffalse
%</samplefinal>
%\fi
%
% %%%%%%%%%%%%%%%%%%%%%%%%%%%%%%%%%%%%%%
% \paragraph{Command Line Processing.}
%
% The following three command lines generate the output files
% |cdocscld|, |cdocscl1| and |cdocscl2|
% which should be identical to
% |cdocsdrf|, |cdocsch1| and |cdocsfn2|, respectively:
% \begin{center}
% \begin{tabular}{l}
% |latex -jobname cdocscld \|\\
% |  "\def\version{draft}\input{childdoc.def}\childdocforward{cdocsamp}"|\\
% |latex -jobname cdocscl1 \|\\
% |  "\input{childdoc.def}\childdocforward[cdocsamp]{cdocsch1}"|\\
% |latex -jobname cdocscl2 \|\\
% |  "\def\version{final}\input{childdoc.def}\childdocforward{cdocsch2}"|
% \end{tabular}
% \end{center}
% Note that the trailing backslash on each first line
% merely continues the input to the second line
% (for convenient cut ant paste).
% Furthermore, the command |latex| can be replaced by any
% of its alternative versions such as |pdflatex|.
%
% %%%%%%%%%%%%%%%%%%%%%%%%%%%%%%%%%%%%%%%%%%%%%%%%%%%%%%%%%%%%%%%%%%%%%%%%%%%%%%
% %%%%%%%%%%%%%%%%%%%%%%%%%%%%%%%%%%%%%%%%%%%%%%%%%%%%%%%%%%%%%%%%%%%%%%%%%%%%%%
% \section{Implementation}
%\iffalse
%<*package>
%\fi
%
% This section describes the definitions file |childdoc.def|.

% The definitions cannot be loaded using |\usepackage| or |\RequirePackage|
% which has a mechanism to prevent loading a style file more than once.
% When loading the definitions by means of |\input|
% multiple instances have to be prevented manually:
%\iffalse
%This code needs to be before the `\ProvidesFile' directive
%which is defined at the beginning of this file.
%Therefore it is also placed there and commented out here.
%</package>
%<*discard>
%\fi
%    \begin{macrocode}
\ifdefined\childdocmain\endinput\fi
%    \end{macrocode}
%\iffalse
%</discard>
%<*package>
%\fi
%
% \macro{\ifchilddoc}
% \macro{\ifchilddocmanual}
% The conditional |\ifchilddoc| tells whether a
% child (true) or main (false) document is being compiled.
% The conditional |\ifchilddocmanual| tells whether
% the |\includeonly| mechanism is used (false) or
% the selection of child files must be performed manually (true).
% The definitions initialise to false:
%    \begin{macrocode}
\newif\ifchilddoc
\newif\ifchilddocmanual
%    \end{macrocode}

% \macro{\childdocname}
% \macro{\childdocjob}
% The macro |\childdocname| stores the name of the main document
% to be compiled. The macro |\childdocjob| stores the name of
% the document on which the \LaTeX{} compiler was originally invoked.
% The content of |\jobname| cannot be compared
% to filenames specified in the source due to different catcodes.
% The following code rescans |\jobname|, stores the result
% in |\childdocname| and saves a copy in |\childdocjob|:
%    \begin{macrocode}
\edef\childdocname{\scantokens\expandafter{\jobname\noexpand}}
\let\childdocjob\childdocname
%    \end{macrocode}

% \macro{\childdocdisable}
% The macro |\childdocdisable| prevents the main file
% from being processed more than once.
% At this stage, the main document command |\childdocmain|
% is assumed to be called once again where it should do nothing.
% Any subsequent call to it should prevent
% a secondary processing of the main document
% It overwrites the forwarding commands
% |\childdocof| and |\childdocforward|
% with empty macros to prevent further inclusions of the main document:
%    \begin{macrocode}
\newcommand{\childdocdisable}
{
  \renewcommand{\childdocmain}[1]{\renewcommand{\childdocmain}[1]{\endinput}}
  \renewcommand{\childdocof}[1]{}
  \renewcommand{\childdocby}[2][]{}
  \renewcommand{\childdocforward}[2][]{}
  \renewcommand{\childdocdisable}{}
}
%    \end{macrocode}

% \macro{\childdocmain}
% The macro |\childdocmain| is to be called at the top of the main file
% with nothing or the main filename (without extension) as argument.
% First, it breaks loops.
% If the argument is not empty and does not match |\childdocname|
% (which is set by the first inclusion of |childdoc.def|),
% |\ifchilddoc| is set to true, |\includeonly| is applied to the child file
% and |\jobname| is set to the main file
% (for proper handling of |.aux| files):
%    \begin{macrocode}
\newcommand{\childdocmain}[1]
{
  \childdocdisable\childdocmain{}
  \if?#1?\else
    \begingroup
      \def\childdoctmp{#1}
      \ifx\childdoctmp\childdocname
        \def\childdoctmp{}
      \else
        \def\childdoctmp
        {
          \childdoctrue
          \includeonly{\childdocname}
          \def\childdocjob{#1}
          \def\jobname{#1}
        }
      \fi
      \expandafter
    \endgroup
    \childdoctmp
  \fi
}
%    \end{macrocode}

% \macro{\childdocof}
% The command |\childdocof| redirects
% compilation to the main file |#1|.
%    \begin{macrocode}
\newcommand{\childdocof}[1]
{
  \childdocdisable
  \childdoctrue
  \includeonly{\childdocname}
  \def\jobname{#1}
  \def\childdocjob{#1}
  \input{#1}
}
%    \end{macrocode}

% \macro{\childdocby}
% The command |\childdocby| ....
%    \begin{macrocode}
\newcommand{\childdocby}[2][]
{
  \childdocdisable
  \childdoctrue
  \childdocmanualtrue
  \if?#1?\else
    \def\jobname{#2}
  \fi
  \def\childdocjob{#2}
  \input{#2}
  \endinput
}
%    \end{macrocode}

% \macro{\childdocforward}
% The command |\childdocforward| redirects
% compilation to the main file or
% (if the optional argument is given) a child file.
% Parameters are set as if the main file
% or a child file starting with |\childdocof| was compiled.
% Then compilation is handed over to the main file:
%    \begin{macrocode}
\newcommand{\childdocforward}[2][]
{
  \begingroup
    \if?#1?
      \def\childdoctmp
      {
        \def\childdocname{#2}
        \def\childdocjob{#2}
        \def\jobname{#2}
        \input{#2}
        \endinput
      }
    \else
      \def\childdoctmp
      {
        \childdocdisable
        \def\childdocname{#2}
        \childdoctrue
        \includeonly{#2}
        \def\childdocjob{#1}
        \def\jobname{#1}
        \input{#1}
        \endinput
      }
    \fi
    \expandafter
  \endgroup
  \childdoctmp
}
%    \end{macrocode}

% \macro{\childdocforwardprefix}
% The command |\childdocforwardprefix| redirects
% compilation to the main or a child file by means of a pattern.
% The prefix |#1| in the current filename is replaced by |#2|
% and the suffix of the current filename is kept
% (it is assumed that the filename does not contain the substring `|~~~|'
% which is used as a delimiter).
% Compilation is handed over to the new file by |\childdocforward|:
%    \begin{macrocode}
\newcommand{\childdocforwardprefix}[3][]
{
  \begingroup
    \def\childdocextract #2##1~~~{\def\childdoctmp{\childdocforward[#1]{#3##1}}}
    \expandafter\childdocextract\childdocname~~~
    \expandafter
  \endgroup
  \childdoctmp
}
%    \end{macrocode}

% \macro{\childdoc}
% The deprecated macro |\childdoc| is a legacy version of |\childdocmain|:
%    \begin{macrocode}
\newcommand{\childdoc}{\childdocmain}
%    \end{macrocode}

% \macro{\childdocredirect}
% The deprecated macro |\childdocredirect| is a legacy version
% of |\childdocforward| and |\childdocforwardprefix|:
%    \begin{macrocode}
\newcommand{\childdocredirect}[2][]
{
  \begingroup
    \if?#1?
      \def\childdoctmp{\childdocforward{#2}}
    \else
      \def\childdoctmp{\childdocforwardprefix{#1}{#2}}
    \fi
    \expandafter
  \endgroup
  \childdoctmp
}
%    \end{macrocode}

%\iffalse
%</package>
%\fi
%
\endinput
|\\
|\childdocmain{}|\\
\end{tabular}
\end{center}
at the very top of the main \LaTeX{} file,
in particular \emph{before} the |\documentclass| statement!
The argument of |\childdocmain| should be left empty
(but it must be present).

%%%%%%%%%%%%%%%%%%%%%%%%%%%%%%%%%%%%%%%%
\DescribeMacro{\childdocof}
Furthermore, add the commands
\begin{center}
\begin{tabular}{l}
|% \iffalse
%
% childdoc.dtx Copyright (C) 2017-2018 Niklas Beisert
%
% This work may be distributed and/or modified under the
% conditions of the LaTeX Project Public License, either version 1.3
% of this license or (at your option) any later version.
% The latest version of this license is in
%   http://www.latex-project.org/lppl.txt
% and version 1.3 or later is part of all distributions of LaTeX
% version 2005/12/01 or later.
%
% This work has the LPPL maintenance status `maintained'.
%
% The Current Maintainer of this work is Niklas Beisert.
%
% This work consists of the files childdoc.dtx and childdoc.ins
% and the derived files childdoc.def and cdocsamp.tex with
% cdocsch1.tex, cdocsch2.tex, cdocsdrf.tex, cdocsfn1.tex, cdocsfn2.tex.
%
%<package>\ifdefined\childdocmain\endinput\fi
%<package>\ProvidesFile{childdoc.def}[2018/12/30 v2.0 child document driver]
%<samplemain>\ProvidesFile{cdocsamp.tex}[2018/12/30 v2.0 sample for childdoc]
%<*driver>
%\ProvidesFile{childdoc.drv}[2018/12/30 v2.0 childdoc reference manual file]
\PassOptionsToClass{10pt,a4paper}{article}
\documentclass{ltxdoc}

\usepackage[margin=35mm]{geometry}
\usepackage{hyperref}
\usepackage{hyperxmp}
\usepackage[usenames]{color}

\hypersetup{colorlinks=true}
\hypersetup{pdfstartview=FitH}
\hypersetup{pdfpagemode=UseNone}
\hypersetup{pdfsource={}}
\hypersetup{pdflang={en-UK}}
\hypersetup{pdfcopyright={Copyright 2017-2018 Niklas Beisert.
  This work may be distributed and/or modified under the
  conditions of the LaTeX Project Public License, either version 1.3
  of this license or (at your option) any later version.}}
\hypersetup{pdflicenseurl={http://www.latex-project.org/lppl.txt}}
\hypersetup{pdfcontactaddress={ETH Zurich, ITP, HIT K,
  Wolfgang-Pauli-Strasse 27}}
\hypersetup{pdfcontactpostcode={8093}}
\hypersetup{pdfcontactcity={Zurich}}
\hypersetup{pdfcontactcountry={Switzerland}}
\hypersetup{pdfcontactemail={nbeisert@itp.phys.ethz.ch}}
\hypersetup{pdfcontacturl={http://people.phys.ethz.ch/\xmptilde nbeisert/}}

\newcommand{\secref}[1]{\hyperref[#1]{section \ref*{#1}}}

\parskip1ex
\parindent0pt
\let\olditemize\itemize
\def\itemize{\olditemize\parskip0pt}

\begin{document}

\title{The \textsf{childdoc} Package}
\hypersetup{pdftitle={The childdoc Package}}
\author{Niklas Beisert\\[2ex]
  Institut f\"ur Theoretische Physik\\
  Eidgen\"ossische Technische Hochschule Z\"urich\\
  Wolfgang-Pauli-Strasse 27, 8093 Z\"urich, Switzerland\\[1ex]
  \href{mailto:nbeisert@itp.phys.ethz.ch}
  {\texttt{nbeisert@itp.phys.ethz.ch}}}
\hypersetup{pdfauthor={Niklas Beisert}}
\hypersetup{pdfsubject={Manual for the LaTeX2e Package childdoc}}
\date{30 December 2018, \textsf{v2.0}}
\maketitle

\begin{abstract}\noindent
\textsf{childdoc} is a \LaTeXe{} package
that enables the direct compilation
of document sections included by |\include|
to individual files.
\end{abstract}

\begingroup
\parskip0ex
\tableofcontents
\endgroup

%%%%%%%%%%%%%%%%%%%%%%%%%%%%%%%%%%%%%%%%%%%%%%%%%%%%%%%%%%%%%%%%%%%%%%%%%%%%%%%%
%%%%%%%%%%%%%%%%%%%%%%%%%%%%%%%%%%%%%%%%%%%%%%%%%%%%%%%%%%%%%%%%%%%%%%%%%%%%%%%%
\section{Introduction}

\LaTeX{} provides a mechanism to structure a large document (such as a book)
into a main file and several child files (containing the chapters)
using the |\include| command.
This mechanism is beneficial for documents
which span hundreds of pages in order to
make the source file(s) more manageable.
Moreover, compilation can be restricted to
selected child files by means of the |\includeonly| command.
The latter feature can be used to reduce the compilation time while editing
(this was significantly more useful in the earlier days of \LaTeX{})
or to generate a smaller document which is easier to navigate.
Another application of |\includeonly| is to generate
documents consisting of selected parts of the complete document.

However, there are a few drawbacks of the plain |\include| mechanism:
\begin{itemize}
\item
The child files cannot be compiled on their own,
they can only be compiled via the main file.
A naive editing environment
(such as a text editor with an option
to have the current file processed by \LaTeX)
may require one to switch to the main file before compiling;
attempting to compile the child file produces errors.
\item
The main file must be modified (each time)
to adjust the |\includeonly| command
to the present needs. This easily leaves the main file in a messy state.
\item
The generated document will always carry the filename
of the main document. This is inconvenient if
several child files are to be compiled and
to be kept for distribution.
\end{itemize}

The present package provides a simple interface
to make child files individually compilable by \LaTeX{}.
Compiling a child file then has the same effect as compiling
the main file with an |\includeonly| command
to select the appropriate child.
Moreover the generated document will carry the name of the child
rather than the main file.
This resolves all three above issues.

This feature is meant to make the editing of books,
thesis documents and lecture notes somewhat more convenient.
However, the package can also be used efficiently for
composing a series of documents (such as exercise sheets)
which are typically distributed individually.
It then assists the author in generating the individual documents
(potentially in different versions)
as well as a document containing the collected series.
Another application is in developing style files
or other kinds of included material
where compilation of the style file could redirect
to a sample or test file.

%%%%%%%%%%%%%%%%%%%%%%%%%%%%%%%%%%%%%%%%%%%%%%%%%%%%%%%%%%%%%%%%%%%%%%%%%%%%%%%%
%%%%%%%%%%%%%%%%%%%%%%%%%%%%%%%%%%%%%%%%%%%%%%%%%%%%%%%%%%%%%%%%%%%%%%%%%%%%%%%%
\section{Usage}

First of all, the package \textsf{childdoc} is \emph{not} a standard
\LaTeXe{} |.sty| style file! Therefore it needs to be invoked in
a non-standard way.

%%%%%%%%%%%%%%%%%%%%%%%%%%%%%%%%%%%%%%%%%%%%%%%%%%%%%%%%%%%%%%%%%%%%%%%%%%%%%%%%
\subsection{Included Files}
\label{sec:include}

%%%%%%%%%%%%%%%%%%%%%%%%%%%%%%%%%%%%%%%%
\DescribeMacro{\childdocmain}
To use the package, add the commands
\begin{center}
\begin{tabular}{l}
|\input{childdoc.def}|\\
|\childdocmain{}|\\
\end{tabular}
\end{center}
at the very top of the main \LaTeX{} file,
in particular \emph{before} the |\documentclass| statement!
The argument of |\childdocmain| should be left empty
(but it must be present).

%%%%%%%%%%%%%%%%%%%%%%%%%%%%%%%%%%%%%%%%
\DescribeMacro{\childdocof}
Furthermore, add the commands
\begin{center}
\begin{tabular}{l}
|\input{childdoc.def}|\\
|\childdocof{|\textit{main}|}|\\
\end{tabular}
\end{center}
at the top of every child file \textit{child}
which is included by |\include{|\textit{child}|}|
from within the main file
(or at least for those files to be compiled individually).
The argument \textit{main} must be the filename of the main file.

There are a couple of
considerations in setting up the main and child documents:

%%%%%%%%%%%%%%%%%%%%%%%%%%%%%%%%%%%%%%%%
\paragraph{Restrictions.}

Please note the following restrictions:
\begin{itemize}
\item
|\childdocmain| must be called with one argument \textit{main}
to ensure compatibility with earlier version of the package.
It must either be empty (|\childdocmain{}|)
or precisely match the filename of the main file in which it is specified.
See \secref{sec:detection} for further information.
\item
The filename \textit{main} must be specified without the |.tex| extension.
\item
The filename \textit{main} is case sensitive
(even in case-insensitive file systems)
due to internal string comparison.
\item
The argument \textit{main} should be fully expanded, it cannot be a macro.
\item
Subdirectories and special characters should be avoided in filenames.
\item
The command |\childdocmain{|\textit{main}|}| must be followed by a whitespace.
It should not be followed immediately by another command
or by a comment mark `|%|'.
This is because the \TeX{} parser reads the token immediately following
the argument of |\childdocmain| and puts it
at the beginning of every child section;
however, a white\-space is ignored.
\end{itemize}

%%%%%%%%%%%%%%%%%%%%%%%%%%%%%%%%%%%%%%%%
\paragraph{Content of Main File.}

It is advisable to place all content in the child files included by |\include|.
Any output contained in the main file will appear in all child documents
unless suppressed manually;
it cannot be suppressed automatically by the |\includeonly| directive
and thus should normally be avoided.
A method to include some content in the main file
by means of conditional processing is described in \secref{sec:conditional}.

%%%%%%%%%%%%%%%%%%%%%%%%%%%%%%%%%%%%%%%%
\paragraph{Page Numbering.}

When only a part of the document is compiled,
the appropriate numbering of pages
(as well as other status parameters)
is determined from the |.aux| files.
The latter contain information from previous passes.
However this information needs to propagate through
all intermediate child documents.
Therefore the page numbering in child documents may well
be inconsistent until the complete document is compiled at least once.

A useful (if unconventional) way to always ensure a consistent
page numbering is to restart the numbering in each child document
and denote the pages by `\textit{child}|.|\textit{page}'
where \textit{child} represents the chapter/section number of the child file.
This can be achieved by the command
|\numberwithin{page}{|\textit{child}|}|
of the \textsf{amsmath} package
where \textit{child} can be |chapter| or |section|
depending on the chosen structuring.
Alternatively, one can modify the macro |\thepage| appropriately
and reset the counter |page| at the start of each child file.

%%%%%%%%%%%%%%%%%%%%%%%%%%%%%%%%%%%%%%%%%%%%%%%%%%%%%%%%%%%%%%%%%%%%%%%%%%%%%%%%
\subsection{Conditional Processing}
\label{sec:conditional}

The package provides a mechanism to compile different versions
of a document. To customise the versions further some conditional processing
can come in handy to distinguish which version is being compiled.
The package provides two macros to describe the compilation context:

%%%%%%%%%%%%%%%%%%%%%%%%%%%%%%%%%%%%%%%%
\DescribeMacro{\ifchilddoc}
The conditional |\ifchilddoc| distinguishes between the compilation of
child documents and the main document:
%
\begin{center}
|\ifchilddoc |\textit{child-code}| |[|\||else |\textit{main-code}]| \||fi|
\end{center}

%%%%%%%%%%%%%%%%%%%%%%%%%%%%%%%%%%%%%%%%
\DescribeMacro{\childdocname}
\DescribeMacro{\childdocjob}
The macro |\childdocname| contains the filename (without extension)
of the main or child file being processed.
Note that |\childdocjob| will always contain the name of the main file.

%%%%%%%%%%%%%%%%%%%%%%%%%%%%%%%%%%%%%%%%
\paragraph{Title Page.}

Conditional processing can be used to include a title or banner page
in the main document when proper precautions are taken.
Importantly, the code in the main file should ensure that the page counter
(as well as other status parameters which are stored in the |.aux| files)
takes the same value after the conditional processing.
Otherwise the page numbers may take divergent values
depending on which part is compiled.

For example, a title page could be declared by:
%
\begin{center}
\begin{tabular}{l}
|\ifchilddoc\||else|\\
|\addtocounter{page}{-1}|\\
\textit{code for title page}\\
|\newpage|\\
|\||fi|
\end{tabular}
\end{center}
%
A banner page for the child documents can be generated by:
%
\begin{center}
\begin{tabular}{l}
|\ifchilddoc|\\
|\addtocounter{page}{-1}|\\
\textit{code for banner page}\\
|\newpage|\\
|\||fi|
\end{tabular}
\end{center}
%
Here one could write a message such as:
\begin{center}
|This is the part \childdocname{} of \childdocjob{}.|
\end{center}

%%%%%%%%%%%%%%%%%%%%%%%%%%%%%%%%%%%%%%%%%%%%%%%%%%%%%%%%%%%%%%%%%%%%%%%%%%%%%%%%
\subsection{Flags}
\label{sec:flags}

The package makes it easy to generate different versions
of the main or child documents.
To this end compilation flags can be defined
and assigned different default values.
They will be particularly useful in conjunction
with the forwarding mechanism described in \secref{sec:forward}.

For example, it may be useful to have a flag |\version|
which can be set to |draft| or |final|.
The document source will contain some conditional code
depending on the value of |\version|.
Suppose further, the flag should default to |final| for the main file
and to |draft| for child files
which is a natural assignment for editing the document.
This is achieved by placing the following code
in the preamble of the main document
(below the |\childdocmain| directive):
%
\begin{center}
\begin{tabular}{l}
|\ifchilddoc|\\
|\providecommand{\version}{draft}|\\
|\||else|\\
|\providecommand{\version}{final}|\\
|\||fi|
\end{tabular}
\end{center}
%
The definition by |\providecommand| makes sure
that previous definitions are not overwritten.
Further statements |\providecommand{\version}{...}|
can thus be added before the above code to override it.

For the main file, one might add a line
(between |\childdocmain| and the above block)
%
\begin{center}
|%\ifchilddoc\||else\providecommand{\version}{draft}\||fi|
\end{center}
%
which can be uncommented to produce a draft version.
Likewise one can add a line to the very top of a child file
(above the |\childdocof{|\textit{main}|}| directive)
%
\begin{center}
|%\providecommand{\version}{final}|
\end{center}
%
which can be uncommented to produce the final version of this child document.

%%%%%%%%%%%%%%%%%%%%%%%%%%%%%%%%%%%%%%%%%%%%%%%%%%%%%%%%%%%%%%%%%%%%%%%%%%%%%%%%
\subsection{Forwarding}
\label{sec:forward}

Different versions of the main or child documents
using compilation flags as described in \secref{sec:flags}
can be (permanently) stored in different files
for convenient compilation, viewing and distribution.
To this end, the package defines a command
to pass on compilation to a different file:

%%%%%%%%%%%%%%%%%%%%%%%%%%%%%%%%%%%%%%%%
\DescribeMacro{\childdocforward}
The command |\childdocforward| redirects processing to
another source file:
%
\begin{center}
\begin{tabular}{l}
|\input{childdoc.def}|\\
|\childdocforward[|\textit{main}|]{|\textit{dest}|}|\\
\end{tabular}
\end{center}
%
The argument \textit{dest} is the destination file
(without extension).
It should be the main file or one of the child files.
Note that further \textsf{childdoc} directives
such as |\childdocof| and |\childdocforward|
in the indicated file will be processed in this form.
The optional argument \textit{main}
passes on directly to the main file \textit{main}
while pretending to compile the child \textit{dest}.
This form behaves as if \textit{dest}
issues |\childdocof{|\textit{main}|}| right away,
and no further \textsf{childdoc} directives will be processed.

%%%%%%%%%%%%%%%%%%%%%%%%%%%%%%%%%%%%%%%%
\DescribeMacro{\...prefix}
In the alternative form |\childdocforwardprefix|,
%
\begin{center}
\begin{tabular}{l}
|\input{childdoc.def}|\\
|\childdocforwardprefix[|\textit{main}|]{|\textit{prefix}|}{|\textit{dest}|}|
\end{tabular}
\end{center}
%
the destination file is determined by a pattern
depending on the current file:
To make this work, the current file must be called
`{\textit{prefix}\hspace{0.2em}\textit{suffix}}'
with \textit{prefix} matching precisely the argument.
Processing is then passed on to the file
`{\textit{dest}\hspace{0.2em}\textit{suffix}}'.
Surely, the same effect is achieved by
directly specifying the
argument `{\textit{dest}\hspace{0.2em}\textit{suffix}}'
in the first form.
However, that requires to set up a different file
for each child. With the alternative form of the command
all these files can have exactly the same content
which simplifies setting them up and maintaining them.

For example, the following file |draft.tex|
with a compilation flag |\version| as described in \secref{sec:flags}
compiles the main document as a draft:
%
\begin{center}
\begin{tabular}{l}
|\def\version{draft}|\\
|\input{childdoc.def}|\\
|\childdocforward{|\textit{main}|}|
\end{tabular}
\end{center}
%
Likewise, the following files |final|\textit{nn}|.tex|
compile the final version of the child document
|child|\textit{nn}|.tex|:
%
\begin{center}
\begin{tabular}{l}
|\def\version{final}|\\
|\input{childdoc.def}|\\
|\childdocforwardprefix{final}{child}|
\end{tabular}
\end{center}
%

Note that when several versions of a main file and/or of each child file
are to be generated, it may be convenient to set up a |Makefile| or
shell script to automatise the process.

%%%%%%%%%%%%%%%%%%%%%%%%%%%%%%%%%%%%%%%%%%%%%%%%%%%%%%%%%%%%%%%%%%%%%%%%%%%%%%%%
\subsection{Command Line Processing}
\label{sec:commandline}

The effect of redirection files can also be achieved by invoking
the \LaTeX{} compiler with a more elaborate command line.
Most conveniently this should be done as part
of a shell script or a |Makefile|.

When using \textsf{childdoc} in the main file, the following
command lines effectively perform a redirection
(note that depending on the shell being used,
backslashes may have to be doubled: `|\|' $\to$ `|\\|'):
%
\begin{center}
|... -jobname "|\textit{target}|" |\\|"|[\textit{flags}]%
|\input{childdoc.def}\childdocforward[|\textit{main}|]{|\textit{dest}|}"|
\end{center}
%
Here \textit{target} is the name of the output file,
\textit{main} is the name of the main file
and \textit{dest} is the name of the main or child file to be processed
(all filenames without extensions).
The optional argument \textit{main} can be omitted
if \textit{main} matches \textit{dest}.
Optionally, compilation \textit{flags} can be defined via |\def| commands.
This command line makes the \TeX{} engine believe
it is compiling the file \textit{target}
whose content is specified as the latter parameter.
The provided code then forwards the processing to
\textit{main} or \textit{dest} as described in \secref{sec:forward}.

%%%%%%%%%%%%%%%%%%%%%%%%%%%%%%%%%%%%%%%%%%%%%%%%%%%%%%%%%%%%%%%%%%%%%%%%%%%%%%%%
\subsection{Include by Input}
\label{sec:input}

Including child documents by |\include| has some restrictions by design.
Most notably, the content of a child document always occupies
its own set of pages; pages cannot be shared between child documents.
Usually, this behaviour makes perfect sense
because each child document contain an essential part of the document.
However, in some situations it may be desirable to compose
a document from a collection of parts
without having mandatory page breaks between then.
For this case, the package
provides a mechanism to include parts
by |\input| which can also be processed individually.
However, by construction this mechanism
requires manual handling of the content to be output.

%%%%%%%%%%%%%%%%%%%%%%%%%%%%%%%%%%%%%%%%
\DescribeMacro{\ifchilddocmanual}
The main file should be prepared as usual, see \secref{sec:include}.
However, the document body must make a distinction
between processing of an individual part and of the main document, e.g.:
%
\begin{center}
\begin{tabular}{l}
|\ifchilddocmanual|\\
|\input{\childdocname}|\\
|\||else|\\
\textit{document body with }|\input{|\textit{part}|}|\\
|\||fi|
\end{tabular}
\end{center}
%
The conditional |\ifchilddocmanual| is true whenever
a part to be included by |\input| is being compiled,
and the name of the part is stored in |\childdocname|.

%%%%%%%%%%%%%%%%%%%%%%%%%%%%%%%%%%%%%%%%
\DescribeMacro{\childdocby}
Each part to be included by |\input| should start with:
%
\begin{center}
\begin{tabular}{l}
|\input{childdoc.def}|\\
|\childdocby{|\textit{main}|}|\\
\end{tabular}
\end{center}
%
The directive |\childdocby| is similar to |\childdocof|
described in \secref{sec:include},
but the subsequent selection of content must be done manually.
To that end, both |\ifchilddoc| and |\ifchilddocmanual|
will be true upon processing of a part,
and the name of the part is stored in |\childdocname|.
Note that |\jobname| will be set to the filename of the current part
so that each part receives an individual |.aux| file
that does not interfere with the |.aux| file(s) of the main document.
This behaviour can be altered by the alternative form
|\childdocby[*]{|\textit{main}|}| (with a non-empty optional argument)
which uses the |.aux| file of the main document
by setting |\jobname| to \textit{main}.

%%%%%%%%%%%%%%%%%%%%%%%%%%%%%%%%%%%%%%%%%%%%%%%%%%%%%%%%%%%%%%%%%%%%%%%%%%%%%%%%
\subsection{Driver Development}
\label{sec:driver}

The \textsf{childdoc} mechanism can also be use for the development
of definition files such as \LaTeX{} styles or classes.
This case differs from the above setup with multiple parts
included by |\include| in that no |\includeonly| should be invoked.
This can be achieved by starting the include file
(before |\ProvidesPackage|) with:
%
\begin{center}
\begin{tabular}{l}
|\input{childdoc.def}|\\
|\childdocforward{|\textit{main}|}|\\
\end{tabular}
\end{center}
%
or alternatively with:
%
\begin{center}
\begin{tabular}{l}
|\input{childdoc.def}|\\
|\childdocby{|\textit{main}|}|\\
\end{tabular}
\end{center}
%
Both forms have slightly different effects as described above.
The main file is prepared as usual, see \secref{sec:include}.

%%%%%%%%%%%%%%%%%%%%%%%%%%%%%%%%%%%%%%%%%%%%%%%%%%%%%%%%%%%%%%%%%%%%%%%%%%%%%%%%
\subsection{Legacy Detection}
\label{sec:detection}

The directive |\childdocmain| in the main file can detect
whether the complete document or merely a child is to be compiled
even without using the directive |\childdocof|.
This method is deprecated because it is less robust
and there is no compelling reason to use it;
it is merely provided for backward compatibility
and it may be removed in future versions.

If the detection mechanism is to be used,
it is mandatory to correctly specify
the filename of the main file as the argument of |\childdocmain|:
%
\begin{center}
\begin{tabular}{l}
|\input{childdoc.def}|\\
|\childdocmain{|\textit{main}|}|\\
\end{tabular}
\end{center}
%
If |\jobname| does not match the argument \textit{main} of |\childdocmain|,
it is assumed that |\jobname| points to the child file to be compiled.
When using |\childdocmain| with the main file specified as argument,
it suffices to start a child file
with just |\input{|\textit{main}|}|
without loading of the package and using |\childdocof|.
If instead all processing is done
with the appropriate \textsf{childdoc} directives,
the argument of \textit{main} of |\childdocmain| can be empty.

An alternative version of the command line processing described
in \secref{sec:commandline} using the detection mechanism reads:
%
\begin{center}
|... -jobname "|\textit{target}|" "|[\textit{flags}]%
[|\def\jobname{|\textit{dest}|}|]|\input{|\textit{main}|}"|
\end{center}

%%%%%%%%%%%%%%%%%%%%%%%%%%%%%%%%%%%%%%%%%%%%%%%%%%%%%%%%%%%%%%%%%%%%%%%%%%%%%%%%
\subsection{Manual Code}
\label{sec:manual}

In case one cannot be certain whether the definitions file |childdoc.def|
is installed on the target \TeX{} distribution
and one prefers not to ship it,
it is conceivable to paste a few relevant commands into the sources.

To that end, drop all statements |\input{childdoc.def}|
and perform the replacements as outlined below.
Instead of |\childdocmain{|\textit{main}|}| add the following code
to the top of the main file:
%
\begin{center}
\begin{tabular}{l}
|\||ifdefined\childdocname\endinput\||fi\newif\ifchilddoc|\\
|\edef\childdocname{\scantokens\expandafter{\jobname\noexpand}}|\\
|\def\childdocmain{|\textit{main}|}\||ifx\childdocmain\childdocname\||else|\\
|\childdoctrue\includeonly{\childdocname}\let\jobname\childdocmain\||fi|\\
\end{tabular}
\end{center}
%
Instead of |\childdocof{|\textit{main}|}| just include the main file
at the top of each child file:
%
\begin{center}
|\input{|\textit{main}|}|
\end{center}
%
A simple redirection |\childdocforward{|\textit{dest}|}| is achieved by:
%
\begin{center}
|\def\jobname{|\textit{dest}|}\input{\jobname}|
\end{center}
%
The redirection with prefix
|\childdocforwardprefix[|\textit{prefix}|]{|\textit{dest}|}|
is accomplished by:
%
\begin{center}
\begin{tabular}{l}
|{\edef\jobname{\scantokens\expandafter{\jobname\noexpand}}|\\
|\def\redirectjob |\textit{prefix}|#1~~~{\gdef\jobname{|\textit{dest}|#1}}|\\
|\expandafter\redirectjob\jobname~~~}\input{\jobname}|
\end{tabular}
\end{center}

In an alternative approach,
child documents can be compiled by a specific command line
without additional code or specific definitions:
%
\begin{center}
|... -jobname "|\textit{target}|" "|[\textit{flags}]%
|\includeonly{|\textit{dest}|}\input{|\textit{main}|}"|
\end{center}
%

%%%%%%%%%%%%%%%%%%%%%%%%%%%%%%%%%%%%%%%%%%%%%%%%%%%%%%%%%%%%%%%%%%%%%%%%%%%%%%%%
%%%%%%%%%%%%%%%%%%%%%%%%%%%%%%%%%%%%%%%%%%%%%%%%%%%%%%%%%%%%%%%%%%%%%%%%%%%%%%%%
\section{Information}

%%%%%%%%%%%%%%%%%%%%%%%%%%%%%%%%%%%%%%%%%%%%%%%%%%%%%%%%%%%%%%%%%%%%%%%%%%%%%%%%
\subsection{Copyright}

Copyright \copyright{} 2017--2018 Niklas Beisert

This work may be distributed and/or modified under the
conditions of the \LaTeX{} Project Public License, either version 1.3
of this license or (at your option) any later version.
The latest version of this license is in
  \url{http://www.latex-project.org/lppl.txt}
and version 1.3 or later is part of all distributions of \LaTeX{}
version 2005/12/01 or later.

This work has the LPPL maintenance status `maintained'.

The Current Maintainer of this work is Niklas Beisert.

This work consists of the files |README.txt|, |childdoc.ins| and |childdoc.dtx|
as well as the derived files |childdoc.def|, |cdocsamp.tex|
with |cdocsch1.tex|, |cdocsch2.tex|, |cdocspt3.tex|, |cdocspt4.tex|,
|cdocsdrf.tex|, |cdocsfn1.tex|, |cdocsfn2.tex|
as well as |childdoc.pdf|.

%%%%%%%%%%%%%%%%%%%%%%%%%%%%%%%%%%%%%%%%%%%%%%%%%%%%%%%%%%%%%%%%%%%%%%%%%%%%%%%%
\subsection{Files and Installation}

The package consists of the files:
%
\begin{center}
\begin{tabular}{ll}
    |README.txt|   & readme file \\
    |childdoc.ins| & installation file \\
    |childdoc.dtx| & source file \\
    |childdoc.def| & definition file \\
    |cdocsamp.tex| & sample main file \\
    |cdocsch1.tex| & sample include file \\
    |cdocsch2.tex| & sample include file \\
    |cdocspt3.tex| & sample part file \\
    |cdocspt4.tex| & sample part file \\
    |cdocsdrf.tex| & sample redirection file \\
    |cdocsfn1.tex| & sample redirection file \\
    |cdocsfn2.tex| & sample redirection file \\
    |childdoc.pdf| & manual
\end{tabular}
\end{center}
%
The distribution consists of the files
|README.txt|, |childdoc.ins| and |childdoc.dtx|.
%
\begin{itemize}
\item
Run (pdf)\LaTeX{} on |childdoc.dtx|
to compile the manual |childdoc.pdf| (this file).
\item
Run \LaTeX{} on |childdoc.ins| to create the definitions file |childdoc.def|
and the sample |cdocsamp.tex| with include files
|cdocsch1.tex|, |cdocsch2.tex|, |cdocspt3.tex|, |cdocspt4.tex|,
|cdocsdrf.tex|, |cdocsfn1.tex|, |cdocsfn2.tex|.
Then copy the file |childdoc.def| to an appropriate directory of your \LaTeX{}
distribution, e.g.\ \textit{texmf-root}|/tex/latex/childdoc|.
\end{itemize}

%%%%%%%%%%%%%%%%%%%%%%%%%%%%%%%%%%%%%%%%%%%%%%%%%%%%%%%%%%%%%%%%%%%%%%%%%%%%%%%%
\subsection{Related CTAN Packages}

There are several other packages which offer a similar functionality:
%
\begin{itemize}
\item
The packages
\href{http://ctan.org/pkg/docmute}{\textsf{docmute}},
\href{http://ctan.org/pkg/includex}{\textsf{includex}} and
\href{http://ctan.org/pkg/standalone}{\textsf{standalone}}
provide commands to include only the document body of
a child file thus allowing both files to be compiled individually.
\item
The packages \href{http://ctan.org/pkg/subdocs}{\textsf{subdocs}}
and \href{http://ctan.org/pkg/subfiles}{\textsf{subfiles}}
provide structures in which the main and child documents can be
encapsulated and allowing them to be compiled individually.
The inclusion mechanism is different from the conventional |\include|.
\item
The package \href{http://ctan.org/pkg/combine}{\textsf{combine}}
is an elaborate solution to combine several documents into one.
\end{itemize}
%
See also the CTAN topic \href{http://ctan.org/topic/subdocs}{\textsf{subdocs}}
for further related packages.
The present package differs from the above solutions in that
a document structure constructed with the conventional |\include| mechanism
just needs two extra commands at the top of every file
such that all constituent files can be compiled individually.

%%%%%%%%%%%%%%%%%%%%%%%%%%%%%%%%%%%%%%%%%%%%%%%%%%%%%%%%%%%%%%%%%%%%%%%%%%%%%%%%
%\subsection{Feature Suggestions}
%
%The following is a list of features which may be useful for future
%versions of this package:
%%
%\begin{itemize}
%\item
%\ldots
%\end{itemize}

%%%%%%%%%%%%%%%%%%%%%%%%%%%%%%%%%%%%%%%%%%%%%%%%%%%%%%%%%%%%%%%%%%%%%%%%%%%%%%%%
\subsection{Revision History}

%%%%%%%%%%%%%%%%%%%%%%%%%%%%%%%%%%%%%%%%
\paragraph{v2.0:} 2018/12/30

\begin{itemize}
\item
immediate forward processing
\item
added |\childdocby| mechanism
\item
manual restructured
\end{itemize}

%%%%%%%%%%%%%%%%%%%%%%%%%%%%%%%%%%%%%%%%
\paragraph{v1.6:} 2018/01/17

\begin{itemize}
\item
application for development of include files
\item
corrections to manual
\end{itemize}

%%%%%%%%%%%%%%%%%%%%%%%%%%%%%%%%%%%%%%%%
\paragraph{v1.5:} 2017/05/21

\begin{itemize}
\item
more complete structuring introduced
\item
|\childdocof| introduced
\item
|\childdoc| renamed to |\childdocmain|
\item
|\childredirect| renamed to |\childdocforward| and |\childdocforwardprefix|
and functionality expanded
\end{itemize}

%%%%%%%%%%%%%%%%%%%%%%%%%%%%%%%%%%%%%%%%
\paragraph{v1.0:} 2017/04/27

\begin{itemize}
\item
manual and install package
\item
first version published on CTAN
\end{itemize}

%%%%%%%%%%%%%%%%%%%%%%%%%%%%%%%%%%%%%%%%
\paragraph{v0.6:} 2017/04/26

\begin{itemize}
\item
redirection mechanism added
\end{itemize}

%%%%%%%%%%%%%%%%%%%%%%%%%%%%%%%%%%%%%%%%
\paragraph{v0.5:} 2017/04/26

\begin{itemize}
\item
functionality in definition file
\end{itemize}


%%%%%%%%%%%%%%%%%%%%%%%%%%%%%%%%%%%%%%%%%%%%%%%%%%%%%%%%%%%%%%%%%%%%%%%%%%%%%%%%
%%%%%%%%%%%%%%%%%%%%%%%%%%%%%%%%%%%%%%%%%%%%%%%%%%%%%%%%%%%%%%%%%%%%%%%%%%%%%%%%
%%%%%%%%%%%%%%%%%%%%%%%%%%%%%%%%%%%%%%%%%%%%%%%%%%%%%%%%%%%%%%%%%%%%%%%%%%%%%%%%
\appendix

\settowidth\MacroIndent{\rmfamily\scriptsize 000\ }

 \DocInput{childdoc.dtx}

\end{document}
%</driver>
% \fi
%
% %%%%%%%%%%%%%%%%%%%%%%%%%%%%%%%%%%%%%%%%%%%%%%%%%%%%%%%%%%%%%%%%%%%%%%%%%%%%%%
% %%%%%%%%%%%%%%%%%%%%%%%%%%%%%%%%%%%%%%%%%%%%%%%%%%%%%%%%%%%%%%%%%%%%%%%%%%%%%%
% \section{Sample}
%\iffalse
%<*samplemain>
%\fi
%
% The following presents a sample document
% with two chapters, two parts, a title page,
% a compile flag as well as three forwarding files to set the flag.
% It consists of eight |.tex| files:
% \begin{center}
% \begin{tabular}{ll}
% |cdocsamp.tex|&main file\\
% |cdocsch1.tex|&include file for chapter 1\\
% |cdocsch2.tex|&include file for chapter 2\\
% |cdocspt3.tex|&include file for part 3\\
% |cdocspt4.tex|&include file for part 4\\
% |cdocsdrf.tex|&forwarding file for main file in draft mode\\
% |cdocsfi1.tex|&forwarding file for final version of chapter 1\\
% |cdocsfi2.tex|&forwarding file for final version of chapter 2\\
% \end{tabular}
% \end{center}
% Each of the eight files can be compiled directly by the \LaTeX{} compiler.
%
% %%%%%%%%%%%%%%%%%%%%%%%%%%%%%%%%%%%%%%
% \paragraph{Main File.}
%
% The main file is called |cdocsamp.tex|.
%
% Load the \textsf{childdoc} definitions and
% declare the filename for the main document:
%    \begin{macrocode}
\input{childdoc.def}
\childdocmain{}
%    \end{macrocode}

% Optional override for |\version| flag:
%    \begin{macrocode}
%%\ifchilddoc\else\providecommand{\version}{draft}\fi
%    \end{macrocode}

% Define the default values for the |\version| flag
% (|final| for the main file and |draft| for childs):
%    \begin{macrocode}
\ifchilddoc
\providecommand{\version}{draft}
\else
\providecommand{\version}{final}
\fi
%    \end{macrocode}

% Load the standard document class:
%    \begin{macrocode}
\documentclass[12pt]{article}
%    \end{macrocode}

% Start the document body:
%    \begin{macrocode}
\begin{document}
%    \end{macrocode}

% Declare a title page.
% Print title, part of document being processed and version flag:
%    \begin{macrocode}
\addtocounter{page}{-1}
\begin{center}
{\LARGE\bfseries{}childdoc example\par}
\vspace{1cm}
\ifchilddoc
\ifchilddocmanual part\else chapter\fi:
`\childdocname' of `\childdocjob'\par
\else
main document: `\childdocjob'\par
\fi
version: \version\par
\end{center}
\newpage
%    \end{macrocode}

% Manually include selected file,
% otherwise process as usual:
%    \begin{macrocode}
\ifchilddocmanual
\section*{part `\childdocname'}
\input{\childdocname}
\else
%    \end{macrocode}

% Include the two chapters:
%    \begin{macrocode}
\include{cdocsch1}
\include{cdocsch2}
%    \end{macrocode}

% Include the two parts unless only chapters should be displayed:
%    \begin{macrocode}
\ifchilddoc\else
\section{part three}
\input{cdocspt3}
\section{part four}
\input{cdocspt4}
\fi
%    \end{macrocode}

% Process as usual until here:
%    \begin{macrocode}
\fi
%    \end{macrocode}

% End of document body:
%    \begin{macrocode}
\end{document}
%    \end{macrocode}
%\iffalse
%</samplemain>
%\fi
%
% %%%%%%%%%%%%%%%%%%%%%%%%%%%%%%%%%%%%%%
% \paragraph{Chapter Include Files.}
%
% The include files are called |cdocsch1.tex| and |cdocsch2.tex|.
%
%\iffalse
%<*samplechap1|samplechap2>
%\fi

% Optional override for |\version| flag:
%    \begin{macrocode}
%%\providecommand{\version}{final}
%    \end{macrocode}

% Include the main document:
%    \begin{macrocode}
\input{childdoc.def}
\childdocof{cdocsamp}
%    \end{macrocode}

%\iffalse
%</samplechap1|samplechap2>
%\fi
%
%\iffalse
%<*samplechap1>
%\fi
% Some text for chapter 1:
%    \begin{macrocode}
\section{one}
some text in chapter one
%    \end{macrocode}

%\iffalse
%</samplechap1>
%\fi
% Some text for chapter 2:
%\iffalse
%<*samplechap2>
%\fi
%    \begin{macrocode}
\section{two}
more text in chapter two
%    \end{macrocode}

%\iffalse
%</samplechap2>
%\fi
%
% %%%%%%%%%%%%%%%%%%%%%%%%%%%%%%%%%%%%%%
% \paragraph{Part Include Files.}
%
% The include files are called |cdocspt3.tex| and |cdocspt4.tex|.
%
%\iffalse
%<*samplepart3|samplepart4>
%\fi

% Optional override for |\version| flag:
%    \begin{macrocode}
%%\providecommand{\version}{final}
%    \end{macrocode}

% Include the main document:
%    \begin{macrocode}
\input{childdoc.def}
\childdocby{cdocsamp}
%    \end{macrocode}

%\iffalse
%</samplepart3|samplepart4>
%\fi
%
%\iffalse
%<*samplepart3>
%\fi
% Some text for part 3:
%    \begin{macrocode}
some text in part three
%    \end{macrocode}

%\iffalse
%</samplepart3>
%\fi
% Some text for part 4:
%\iffalse
%<*samplepart4>
%\fi
%    \begin{macrocode}
more text in part four
%    \end{macrocode}

%\iffalse
%</samplepart4>
%\fi
%
% %%%%%%%%%%%%%%%%%%%%%%%%%%%%%%%%%%%%%%
% \paragraph{Forwarding for a Complete Draft.}
%
% The following forwarding file |cdocsdrf.tex|
% compiles the main document in draft mode:
%\iffalse
%<*sampledraft>
%\fi
%    \begin{macrocode}
\def\version{draft}
\input{childdoc.def}
\childdocforward{cdocsamp}
%    \end{macrocode}

%\iffalse
%</sampledraft>
%\fi
%
% %%%%%%%%%%%%%%%%%%%%%%%%%%%%%%%%%%%%%%
% \paragraph{Forwarding for Final Version of the Chapters.}
%
% The following forwarding files |cdocsfn1.tex| and |cdocsfn2.tex|
% (with identical content)
% compile the final versions of the child documents
% |cdocsch1.tex| and |cdocsch2.tex|, respectively:
%\iffalse
%<*samplefinal>
%\fi
%    \begin{macrocode}
\def\version{final}
\input{childdoc.def}
\childdocforwardprefix[cdocsamp]{cdocsfn}{cdocsch}
%    \end{macrocode}

%\iffalse
%</samplefinal>
%\fi
%
% %%%%%%%%%%%%%%%%%%%%%%%%%%%%%%%%%%%%%%
% \paragraph{Command Line Processing.}
%
% The following three command lines generate the output files
% |cdocscld|, |cdocscl1| and |cdocscl2|
% which should be identical to
% |cdocsdrf|, |cdocsch1| and |cdocsfn2|, respectively:
% \begin{center}
% \begin{tabular}{l}
% |latex -jobname cdocscld \|\\
% |  "\def\version{draft}\input{childdoc.def}\childdocforward{cdocsamp}"|\\
% |latex -jobname cdocscl1 \|\\
% |  "\input{childdoc.def}\childdocforward[cdocsamp]{cdocsch1}"|\\
% |latex -jobname cdocscl2 \|\\
% |  "\def\version{final}\input{childdoc.def}\childdocforward{cdocsch2}"|
% \end{tabular}
% \end{center}
% Note that the trailing backslash on each first line
% merely continues the input to the second line
% (for convenient cut ant paste).
% Furthermore, the command |latex| can be replaced by any
% of its alternative versions such as |pdflatex|.
%
% %%%%%%%%%%%%%%%%%%%%%%%%%%%%%%%%%%%%%%%%%%%%%%%%%%%%%%%%%%%%%%%%%%%%%%%%%%%%%%
% %%%%%%%%%%%%%%%%%%%%%%%%%%%%%%%%%%%%%%%%%%%%%%%%%%%%%%%%%%%%%%%%%%%%%%%%%%%%%%
% \section{Implementation}
%\iffalse
%<*package>
%\fi
%
% This section describes the definitions file |childdoc.def|.

% The definitions cannot be loaded using |\usepackage| or |\RequirePackage|
% which has a mechanism to prevent loading a style file more than once.
% When loading the definitions by means of |\input|
% multiple instances have to be prevented manually:
%\iffalse
%This code needs to be before the `\ProvidesFile' directive
%which is defined at the beginning of this file.
%Therefore it is also placed there and commented out here.
%</package>
%<*discard>
%\fi
%    \begin{macrocode}
\ifdefined\childdocmain\endinput\fi
%    \end{macrocode}
%\iffalse
%</discard>
%<*package>
%\fi
%
% \macro{\ifchilddoc}
% \macro{\ifchilddocmanual}
% The conditional |\ifchilddoc| tells whether a
% child (true) or main (false) document is being compiled.
% The conditional |\ifchilddocmanual| tells whether
% the |\includeonly| mechanism is used (false) or
% the selection of child files must be performed manually (true).
% The definitions initialise to false:
%    \begin{macrocode}
\newif\ifchilddoc
\newif\ifchilddocmanual
%    \end{macrocode}

% \macro{\childdocname}
% \macro{\childdocjob}
% The macro |\childdocname| stores the name of the main document
% to be compiled. The macro |\childdocjob| stores the name of
% the document on which the \LaTeX{} compiler was originally invoked.
% The content of |\jobname| cannot be compared
% to filenames specified in the source due to different catcodes.
% The following code rescans |\jobname|, stores the result
% in |\childdocname| and saves a copy in |\childdocjob|:
%    \begin{macrocode}
\edef\childdocname{\scantokens\expandafter{\jobname\noexpand}}
\let\childdocjob\childdocname
%    \end{macrocode}

% \macro{\childdocdisable}
% The macro |\childdocdisable| prevents the main file
% from being processed more than once.
% At this stage, the main document command |\childdocmain|
% is assumed to be called once again where it should do nothing.
% Any subsequent call to it should prevent
% a secondary processing of the main document
% It overwrites the forwarding commands
% |\childdocof| and |\childdocforward|
% with empty macros to prevent further inclusions of the main document:
%    \begin{macrocode}
\newcommand{\childdocdisable}
{
  \renewcommand{\childdocmain}[1]{\renewcommand{\childdocmain}[1]{\endinput}}
  \renewcommand{\childdocof}[1]{}
  \renewcommand{\childdocby}[2][]{}
  \renewcommand{\childdocforward}[2][]{}
  \renewcommand{\childdocdisable}{}
}
%    \end{macrocode}

% \macro{\childdocmain}
% The macro |\childdocmain| is to be called at the top of the main file
% with nothing or the main filename (without extension) as argument.
% First, it breaks loops.
% If the argument is not empty and does not match |\childdocname|
% (which is set by the first inclusion of |childdoc.def|),
% |\ifchilddoc| is set to true, |\includeonly| is applied to the child file
% and |\jobname| is set to the main file
% (for proper handling of |.aux| files):
%    \begin{macrocode}
\newcommand{\childdocmain}[1]
{
  \childdocdisable\childdocmain{}
  \if?#1?\else
    \begingroup
      \def\childdoctmp{#1}
      \ifx\childdoctmp\childdocname
        \def\childdoctmp{}
      \else
        \def\childdoctmp
        {
          \childdoctrue
          \includeonly{\childdocname}
          \def\childdocjob{#1}
          \def\jobname{#1}
        }
      \fi
      \expandafter
    \endgroup
    \childdoctmp
  \fi
}
%    \end{macrocode}

% \macro{\childdocof}
% The command |\childdocof| redirects
% compilation to the main file |#1|.
%    \begin{macrocode}
\newcommand{\childdocof}[1]
{
  \childdocdisable
  \childdoctrue
  \includeonly{\childdocname}
  \def\jobname{#1}
  \def\childdocjob{#1}
  \input{#1}
}
%    \end{macrocode}

% \macro{\childdocby}
% The command |\childdocby| ....
%    \begin{macrocode}
\newcommand{\childdocby}[2][]
{
  \childdocdisable
  \childdoctrue
  \childdocmanualtrue
  \if?#1?\else
    \def\jobname{#2}
  \fi
  \def\childdocjob{#2}
  \input{#2}
  \endinput
}
%    \end{macrocode}

% \macro{\childdocforward}
% The command |\childdocforward| redirects
% compilation to the main file or
% (if the optional argument is given) a child file.
% Parameters are set as if the main file
% or a child file starting with |\childdocof| was compiled.
% Then compilation is handed over to the main file:
%    \begin{macrocode}
\newcommand{\childdocforward}[2][]
{
  \begingroup
    \if?#1?
      \def\childdoctmp
      {
        \def\childdocname{#2}
        \def\childdocjob{#2}
        \def\jobname{#2}
        \input{#2}
        \endinput
      }
    \else
      \def\childdoctmp
      {
        \childdocdisable
        \def\childdocname{#2}
        \childdoctrue
        \includeonly{#2}
        \def\childdocjob{#1}
        \def\jobname{#1}
        \input{#1}
        \endinput
      }
    \fi
    \expandafter
  \endgroup
  \childdoctmp
}
%    \end{macrocode}

% \macro{\childdocforwardprefix}
% The command |\childdocforwardprefix| redirects
% compilation to the main or a child file by means of a pattern.
% The prefix |#1| in the current filename is replaced by |#2|
% and the suffix of the current filename is kept
% (it is assumed that the filename does not contain the substring `|~~~|'
% which is used as a delimiter).
% Compilation is handed over to the new file by |\childdocforward|:
%    \begin{macrocode}
\newcommand{\childdocforwardprefix}[3][]
{
  \begingroup
    \def\childdocextract #2##1~~~{\def\childdoctmp{\childdocforward[#1]{#3##1}}}
    \expandafter\childdocextract\childdocname~~~
    \expandafter
  \endgroup
  \childdoctmp
}
%    \end{macrocode}

% \macro{\childdoc}
% The deprecated macro |\childdoc| is a legacy version of |\childdocmain|:
%    \begin{macrocode}
\newcommand{\childdoc}{\childdocmain}
%    \end{macrocode}

% \macro{\childdocredirect}
% The deprecated macro |\childdocredirect| is a legacy version
% of |\childdocforward| and |\childdocforwardprefix|:
%    \begin{macrocode}
\newcommand{\childdocredirect}[2][]
{
  \begingroup
    \if?#1?
      \def\childdoctmp{\childdocforward{#2}}
    \else
      \def\childdoctmp{\childdocforwardprefix{#1}{#2}}
    \fi
    \expandafter
  \endgroup
  \childdoctmp
}
%    \end{macrocode}

%\iffalse
%</package>
%\fi
%
\endinput
|\\
|\childdocof{|\textit{main}|}|\\
\end{tabular}
\end{center}
at the top of every child file \textit{child}
which is included by |\include{|\textit{child}|}|
from within the main file
(or at least for those files to be compiled individually).
The argument \textit{main} must be the filename of the main file.

There are a couple of
considerations in setting up the main and child documents:

%%%%%%%%%%%%%%%%%%%%%%%%%%%%%%%%%%%%%%%%
\paragraph{Restrictions.}

Please note the following restrictions:
\begin{itemize}
\item
|\childdocmain| must be called with one argument \textit{main}
to ensure compatibility with earlier version of the package.
It must either be empty (|\childdocmain{}|)
or precisely match the filename of the main file in which it is specified.
See \secref{sec:detection} for further information.
\item
The filename \textit{main} must be specified without the |.tex| extension.
\item
The filename \textit{main} is case sensitive
(even in case-insensitive file systems)
due to internal string comparison.
\item
The argument \textit{main} should be fully expanded, it cannot be a macro.
\item
Subdirectories and special characters should be avoided in filenames.
\item
The command |\childdocmain{|\textit{main}|}| must be followed by a whitespace.
It should not be followed immediately by another command
or by a comment mark `|%|'.
This is because the \TeX{} parser reads the token immediately following
the argument of |\childdocmain| and puts it
at the beginning of every child section;
however, a white\-space is ignored.
\end{itemize}

%%%%%%%%%%%%%%%%%%%%%%%%%%%%%%%%%%%%%%%%
\paragraph{Content of Main File.}

It is advisable to place all content in the child files included by |\include|.
Any output contained in the main file will appear in all child documents
unless suppressed manually;
it cannot be suppressed automatically by the |\includeonly| directive
and thus should normally be avoided.
A method to include some content in the main file
by means of conditional processing is described in \secref{sec:conditional}.

%%%%%%%%%%%%%%%%%%%%%%%%%%%%%%%%%%%%%%%%
\paragraph{Page Numbering.}

When only a part of the document is compiled,
the appropriate numbering of pages
(as well as other status parameters)
is determined from the |.aux| files.
The latter contain information from previous passes.
However this information needs to propagate through
all intermediate child documents.
Therefore the page numbering in child documents may well
be inconsistent until the complete document is compiled at least once.

A useful (if unconventional) way to always ensure a consistent
page numbering is to restart the numbering in each child document
and denote the pages by `\textit{child}|.|\textit{page}'
where \textit{child} represents the chapter/section number of the child file.
This can be achieved by the command
|\numberwithin{page}{|\textit{child}|}|
of the \textsf{amsmath} package
where \textit{child} can be |chapter| or |section|
depending on the chosen structuring.
Alternatively, one can modify the macro |\thepage| appropriately
and reset the counter |page| at the start of each child file.

%%%%%%%%%%%%%%%%%%%%%%%%%%%%%%%%%%%%%%%%%%%%%%%%%%%%%%%%%%%%%%%%%%%%%%%%%%%%%%%%
\subsection{Conditional Processing}
\label{sec:conditional}

The package provides a mechanism to compile different versions
of a document. To customise the versions further some conditional processing
can come in handy to distinguish which version is being compiled.
The package provides two macros to describe the compilation context:

%%%%%%%%%%%%%%%%%%%%%%%%%%%%%%%%%%%%%%%%
\DescribeMacro{\ifchilddoc}
The conditional |\ifchilddoc| distinguishes between the compilation of
child documents and the main document:
%
\begin{center}
|\ifchilddoc |\textit{child-code}| |[|\||else |\textit{main-code}]| \||fi|
\end{center}

%%%%%%%%%%%%%%%%%%%%%%%%%%%%%%%%%%%%%%%%
\DescribeMacro{\childdocname}
\DescribeMacro{\childdocjob}
The macro |\childdocname| contains the filename (without extension)
of the main or child file being processed.
Note that |\childdocjob| will always contain the name of the main file.

%%%%%%%%%%%%%%%%%%%%%%%%%%%%%%%%%%%%%%%%
\paragraph{Title Page.}

Conditional processing can be used to include a title or banner page
in the main document when proper precautions are taken.
Importantly, the code in the main file should ensure that the page counter
(as well as other status parameters which are stored in the |.aux| files)
takes the same value after the conditional processing.
Otherwise the page numbers may take divergent values
depending on which part is compiled.

For example, a title page could be declared by:
%
\begin{center}
\begin{tabular}{l}
|\ifchilddoc\||else|\\
|\addtocounter{page}{-1}|\\
\textit{code for title page}\\
|\newpage|\\
|\||fi|
\end{tabular}
\end{center}
%
A banner page for the child documents can be generated by:
%
\begin{center}
\begin{tabular}{l}
|\ifchilddoc|\\
|\addtocounter{page}{-1}|\\
\textit{code for banner page}\\
|\newpage|\\
|\||fi|
\end{tabular}
\end{center}
%
Here one could write a message such as:
\begin{center}
|This is the part \childdocname{} of \childdocjob{}.|
\end{center}

%%%%%%%%%%%%%%%%%%%%%%%%%%%%%%%%%%%%%%%%%%%%%%%%%%%%%%%%%%%%%%%%%%%%%%%%%%%%%%%%
\subsection{Flags}
\label{sec:flags}

The package makes it easy to generate different versions
of the main or child documents.
To this end compilation flags can be defined
and assigned different default values.
They will be particularly useful in conjunction
with the forwarding mechanism described in \secref{sec:forward}.

For example, it may be useful to have a flag |\version|
which can be set to |draft| or |final|.
The document source will contain some conditional code
depending on the value of |\version|.
Suppose further, the flag should default to |final| for the main file
and to |draft| for child files
which is a natural assignment for editing the document.
This is achieved by placing the following code
in the preamble of the main document
(below the |\childdocmain| directive):
%
\begin{center}
\begin{tabular}{l}
|\ifchilddoc|\\
|\providecommand{\version}{draft}|\\
|\||else|\\
|\providecommand{\version}{final}|\\
|\||fi|
\end{tabular}
\end{center}
%
The definition by |\providecommand| makes sure
that previous definitions are not overwritten.
Further statements |\providecommand{\version}{...}|
can thus be added before the above code to override it.

For the main file, one might add a line
(between |\childdocmain| and the above block)
%
\begin{center}
|%\ifchilddoc\||else\providecommand{\version}{draft}\||fi|
\end{center}
%
which can be uncommented to produce a draft version.
Likewise one can add a line to the very top of a child file
(above the |\childdocof{|\textit{main}|}| directive)
%
\begin{center}
|%\providecommand{\version}{final}|
\end{center}
%
which can be uncommented to produce the final version of this child document.

%%%%%%%%%%%%%%%%%%%%%%%%%%%%%%%%%%%%%%%%%%%%%%%%%%%%%%%%%%%%%%%%%%%%%%%%%%%%%%%%
\subsection{Forwarding}
\label{sec:forward}

Different versions of the main or child documents
using compilation flags as described in \secref{sec:flags}
can be (permanently) stored in different files
for convenient compilation, viewing and distribution.
To this end, the package defines a command
to pass on compilation to a different file:

%%%%%%%%%%%%%%%%%%%%%%%%%%%%%%%%%%%%%%%%
\DescribeMacro{\childdocforward}
The command |\childdocforward| redirects processing to
another source file:
%
\begin{center}
\begin{tabular}{l}
|% \iffalse
%
% childdoc.dtx Copyright (C) 2017-2018 Niklas Beisert
%
% This work may be distributed and/or modified under the
% conditions of the LaTeX Project Public License, either version 1.3
% of this license or (at your option) any later version.
% The latest version of this license is in
%   http://www.latex-project.org/lppl.txt
% and version 1.3 or later is part of all distributions of LaTeX
% version 2005/12/01 or later.
%
% This work has the LPPL maintenance status `maintained'.
%
% The Current Maintainer of this work is Niklas Beisert.
%
% This work consists of the files childdoc.dtx and childdoc.ins
% and the derived files childdoc.def and cdocsamp.tex with
% cdocsch1.tex, cdocsch2.tex, cdocsdrf.tex, cdocsfn1.tex, cdocsfn2.tex.
%
%<package>\ifdefined\childdocmain\endinput\fi
%<package>\ProvidesFile{childdoc.def}[2018/12/30 v2.0 child document driver]
%<samplemain>\ProvidesFile{cdocsamp.tex}[2018/12/30 v2.0 sample for childdoc]
%<*driver>
%\ProvidesFile{childdoc.drv}[2018/12/30 v2.0 childdoc reference manual file]
\PassOptionsToClass{10pt,a4paper}{article}
\documentclass{ltxdoc}

\usepackage[margin=35mm]{geometry}
\usepackage{hyperref}
\usepackage{hyperxmp}
\usepackage[usenames]{color}

\hypersetup{colorlinks=true}
\hypersetup{pdfstartview=FitH}
\hypersetup{pdfpagemode=UseNone}
\hypersetup{pdfsource={}}
\hypersetup{pdflang={en-UK}}
\hypersetup{pdfcopyright={Copyright 2017-2018 Niklas Beisert.
  This work may be distributed and/or modified under the
  conditions of the LaTeX Project Public License, either version 1.3
  of this license or (at your option) any later version.}}
\hypersetup{pdflicenseurl={http://www.latex-project.org/lppl.txt}}
\hypersetup{pdfcontactaddress={ETH Zurich, ITP, HIT K,
  Wolfgang-Pauli-Strasse 27}}
\hypersetup{pdfcontactpostcode={8093}}
\hypersetup{pdfcontactcity={Zurich}}
\hypersetup{pdfcontactcountry={Switzerland}}
\hypersetup{pdfcontactemail={nbeisert@itp.phys.ethz.ch}}
\hypersetup{pdfcontacturl={http://people.phys.ethz.ch/\xmptilde nbeisert/}}

\newcommand{\secref}[1]{\hyperref[#1]{section \ref*{#1}}}

\parskip1ex
\parindent0pt
\let\olditemize\itemize
\def\itemize{\olditemize\parskip0pt}

\begin{document}

\title{The \textsf{childdoc} Package}
\hypersetup{pdftitle={The childdoc Package}}
\author{Niklas Beisert\\[2ex]
  Institut f\"ur Theoretische Physik\\
  Eidgen\"ossische Technische Hochschule Z\"urich\\
  Wolfgang-Pauli-Strasse 27, 8093 Z\"urich, Switzerland\\[1ex]
  \href{mailto:nbeisert@itp.phys.ethz.ch}
  {\texttt{nbeisert@itp.phys.ethz.ch}}}
\hypersetup{pdfauthor={Niklas Beisert}}
\hypersetup{pdfsubject={Manual for the LaTeX2e Package childdoc}}
\date{30 December 2018, \textsf{v2.0}}
\maketitle

\begin{abstract}\noindent
\textsf{childdoc} is a \LaTeXe{} package
that enables the direct compilation
of document sections included by |\include|
to individual files.
\end{abstract}

\begingroup
\parskip0ex
\tableofcontents
\endgroup

%%%%%%%%%%%%%%%%%%%%%%%%%%%%%%%%%%%%%%%%%%%%%%%%%%%%%%%%%%%%%%%%%%%%%%%%%%%%%%%%
%%%%%%%%%%%%%%%%%%%%%%%%%%%%%%%%%%%%%%%%%%%%%%%%%%%%%%%%%%%%%%%%%%%%%%%%%%%%%%%%
\section{Introduction}

\LaTeX{} provides a mechanism to structure a large document (such as a book)
into a main file and several child files (containing the chapters)
using the |\include| command.
This mechanism is beneficial for documents
which span hundreds of pages in order to
make the source file(s) more manageable.
Moreover, compilation can be restricted to
selected child files by means of the |\includeonly| command.
The latter feature can be used to reduce the compilation time while editing
(this was significantly more useful in the earlier days of \LaTeX{})
or to generate a smaller document which is easier to navigate.
Another application of |\includeonly| is to generate
documents consisting of selected parts of the complete document.

However, there are a few drawbacks of the plain |\include| mechanism:
\begin{itemize}
\item
The child files cannot be compiled on their own,
they can only be compiled via the main file.
A naive editing environment
(such as a text editor with an option
to have the current file processed by \LaTeX)
may require one to switch to the main file before compiling;
attempting to compile the child file produces errors.
\item
The main file must be modified (each time)
to adjust the |\includeonly| command
to the present needs. This easily leaves the main file in a messy state.
\item
The generated document will always carry the filename
of the main document. This is inconvenient if
several child files are to be compiled and
to be kept for distribution.
\end{itemize}

The present package provides a simple interface
to make child files individually compilable by \LaTeX{}.
Compiling a child file then has the same effect as compiling
the main file with an |\includeonly| command
to select the appropriate child.
Moreover the generated document will carry the name of the child
rather than the main file.
This resolves all three above issues.

This feature is meant to make the editing of books,
thesis documents and lecture notes somewhat more convenient.
However, the package can also be used efficiently for
composing a series of documents (such as exercise sheets)
which are typically distributed individually.
It then assists the author in generating the individual documents
(potentially in different versions)
as well as a document containing the collected series.
Another application is in developing style files
or other kinds of included material
where compilation of the style file could redirect
to a sample or test file.

%%%%%%%%%%%%%%%%%%%%%%%%%%%%%%%%%%%%%%%%%%%%%%%%%%%%%%%%%%%%%%%%%%%%%%%%%%%%%%%%
%%%%%%%%%%%%%%%%%%%%%%%%%%%%%%%%%%%%%%%%%%%%%%%%%%%%%%%%%%%%%%%%%%%%%%%%%%%%%%%%
\section{Usage}

First of all, the package \textsf{childdoc} is \emph{not} a standard
\LaTeXe{} |.sty| style file! Therefore it needs to be invoked in
a non-standard way.

%%%%%%%%%%%%%%%%%%%%%%%%%%%%%%%%%%%%%%%%%%%%%%%%%%%%%%%%%%%%%%%%%%%%%%%%%%%%%%%%
\subsection{Included Files}
\label{sec:include}

%%%%%%%%%%%%%%%%%%%%%%%%%%%%%%%%%%%%%%%%
\DescribeMacro{\childdocmain}
To use the package, add the commands
\begin{center}
\begin{tabular}{l}
|\input{childdoc.def}|\\
|\childdocmain{}|\\
\end{tabular}
\end{center}
at the very top of the main \LaTeX{} file,
in particular \emph{before} the |\documentclass| statement!
The argument of |\childdocmain| should be left empty
(but it must be present).

%%%%%%%%%%%%%%%%%%%%%%%%%%%%%%%%%%%%%%%%
\DescribeMacro{\childdocof}
Furthermore, add the commands
\begin{center}
\begin{tabular}{l}
|\input{childdoc.def}|\\
|\childdocof{|\textit{main}|}|\\
\end{tabular}
\end{center}
at the top of every child file \textit{child}
which is included by |\include{|\textit{child}|}|
from within the main file
(or at least for those files to be compiled individually).
The argument \textit{main} must be the filename of the main file.

There are a couple of
considerations in setting up the main and child documents:

%%%%%%%%%%%%%%%%%%%%%%%%%%%%%%%%%%%%%%%%
\paragraph{Restrictions.}

Please note the following restrictions:
\begin{itemize}
\item
|\childdocmain| must be called with one argument \textit{main}
to ensure compatibility with earlier version of the package.
It must either be empty (|\childdocmain{}|)
or precisely match the filename of the main file in which it is specified.
See \secref{sec:detection} for further information.
\item
The filename \textit{main} must be specified without the |.tex| extension.
\item
The filename \textit{main} is case sensitive
(even in case-insensitive file systems)
due to internal string comparison.
\item
The argument \textit{main} should be fully expanded, it cannot be a macro.
\item
Subdirectories and special characters should be avoided in filenames.
\item
The command |\childdocmain{|\textit{main}|}| must be followed by a whitespace.
It should not be followed immediately by another command
or by a comment mark `|%|'.
This is because the \TeX{} parser reads the token immediately following
the argument of |\childdocmain| and puts it
at the beginning of every child section;
however, a white\-space is ignored.
\end{itemize}

%%%%%%%%%%%%%%%%%%%%%%%%%%%%%%%%%%%%%%%%
\paragraph{Content of Main File.}

It is advisable to place all content in the child files included by |\include|.
Any output contained in the main file will appear in all child documents
unless suppressed manually;
it cannot be suppressed automatically by the |\includeonly| directive
and thus should normally be avoided.
A method to include some content in the main file
by means of conditional processing is described in \secref{sec:conditional}.

%%%%%%%%%%%%%%%%%%%%%%%%%%%%%%%%%%%%%%%%
\paragraph{Page Numbering.}

When only a part of the document is compiled,
the appropriate numbering of pages
(as well as other status parameters)
is determined from the |.aux| files.
The latter contain information from previous passes.
However this information needs to propagate through
all intermediate child documents.
Therefore the page numbering in child documents may well
be inconsistent until the complete document is compiled at least once.

A useful (if unconventional) way to always ensure a consistent
page numbering is to restart the numbering in each child document
and denote the pages by `\textit{child}|.|\textit{page}'
where \textit{child} represents the chapter/section number of the child file.
This can be achieved by the command
|\numberwithin{page}{|\textit{child}|}|
of the \textsf{amsmath} package
where \textit{child} can be |chapter| or |section|
depending on the chosen structuring.
Alternatively, one can modify the macro |\thepage| appropriately
and reset the counter |page| at the start of each child file.

%%%%%%%%%%%%%%%%%%%%%%%%%%%%%%%%%%%%%%%%%%%%%%%%%%%%%%%%%%%%%%%%%%%%%%%%%%%%%%%%
\subsection{Conditional Processing}
\label{sec:conditional}

The package provides a mechanism to compile different versions
of a document. To customise the versions further some conditional processing
can come in handy to distinguish which version is being compiled.
The package provides two macros to describe the compilation context:

%%%%%%%%%%%%%%%%%%%%%%%%%%%%%%%%%%%%%%%%
\DescribeMacro{\ifchilddoc}
The conditional |\ifchilddoc| distinguishes between the compilation of
child documents and the main document:
%
\begin{center}
|\ifchilddoc |\textit{child-code}| |[|\||else |\textit{main-code}]| \||fi|
\end{center}

%%%%%%%%%%%%%%%%%%%%%%%%%%%%%%%%%%%%%%%%
\DescribeMacro{\childdocname}
\DescribeMacro{\childdocjob}
The macro |\childdocname| contains the filename (without extension)
of the main or child file being processed.
Note that |\childdocjob| will always contain the name of the main file.

%%%%%%%%%%%%%%%%%%%%%%%%%%%%%%%%%%%%%%%%
\paragraph{Title Page.}

Conditional processing can be used to include a title or banner page
in the main document when proper precautions are taken.
Importantly, the code in the main file should ensure that the page counter
(as well as other status parameters which are stored in the |.aux| files)
takes the same value after the conditional processing.
Otherwise the page numbers may take divergent values
depending on which part is compiled.

For example, a title page could be declared by:
%
\begin{center}
\begin{tabular}{l}
|\ifchilddoc\||else|\\
|\addtocounter{page}{-1}|\\
\textit{code for title page}\\
|\newpage|\\
|\||fi|
\end{tabular}
\end{center}
%
A banner page for the child documents can be generated by:
%
\begin{center}
\begin{tabular}{l}
|\ifchilddoc|\\
|\addtocounter{page}{-1}|\\
\textit{code for banner page}\\
|\newpage|\\
|\||fi|
\end{tabular}
\end{center}
%
Here one could write a message such as:
\begin{center}
|This is the part \childdocname{} of \childdocjob{}.|
\end{center}

%%%%%%%%%%%%%%%%%%%%%%%%%%%%%%%%%%%%%%%%%%%%%%%%%%%%%%%%%%%%%%%%%%%%%%%%%%%%%%%%
\subsection{Flags}
\label{sec:flags}

The package makes it easy to generate different versions
of the main or child documents.
To this end compilation flags can be defined
and assigned different default values.
They will be particularly useful in conjunction
with the forwarding mechanism described in \secref{sec:forward}.

For example, it may be useful to have a flag |\version|
which can be set to |draft| or |final|.
The document source will contain some conditional code
depending on the value of |\version|.
Suppose further, the flag should default to |final| for the main file
and to |draft| for child files
which is a natural assignment for editing the document.
This is achieved by placing the following code
in the preamble of the main document
(below the |\childdocmain| directive):
%
\begin{center}
\begin{tabular}{l}
|\ifchilddoc|\\
|\providecommand{\version}{draft}|\\
|\||else|\\
|\providecommand{\version}{final}|\\
|\||fi|
\end{tabular}
\end{center}
%
The definition by |\providecommand| makes sure
that previous definitions are not overwritten.
Further statements |\providecommand{\version}{...}|
can thus be added before the above code to override it.

For the main file, one might add a line
(between |\childdocmain| and the above block)
%
\begin{center}
|%\ifchilddoc\||else\providecommand{\version}{draft}\||fi|
\end{center}
%
which can be uncommented to produce a draft version.
Likewise one can add a line to the very top of a child file
(above the |\childdocof{|\textit{main}|}| directive)
%
\begin{center}
|%\providecommand{\version}{final}|
\end{center}
%
which can be uncommented to produce the final version of this child document.

%%%%%%%%%%%%%%%%%%%%%%%%%%%%%%%%%%%%%%%%%%%%%%%%%%%%%%%%%%%%%%%%%%%%%%%%%%%%%%%%
\subsection{Forwarding}
\label{sec:forward}

Different versions of the main or child documents
using compilation flags as described in \secref{sec:flags}
can be (permanently) stored in different files
for convenient compilation, viewing and distribution.
To this end, the package defines a command
to pass on compilation to a different file:

%%%%%%%%%%%%%%%%%%%%%%%%%%%%%%%%%%%%%%%%
\DescribeMacro{\childdocforward}
The command |\childdocforward| redirects processing to
another source file:
%
\begin{center}
\begin{tabular}{l}
|\input{childdoc.def}|\\
|\childdocforward[|\textit{main}|]{|\textit{dest}|}|\\
\end{tabular}
\end{center}
%
The argument \textit{dest} is the destination file
(without extension).
It should be the main file or one of the child files.
Note that further \textsf{childdoc} directives
such as |\childdocof| and |\childdocforward|
in the indicated file will be processed in this form.
The optional argument \textit{main}
passes on directly to the main file \textit{main}
while pretending to compile the child \textit{dest}.
This form behaves as if \textit{dest}
issues |\childdocof{|\textit{main}|}| right away,
and no further \textsf{childdoc} directives will be processed.

%%%%%%%%%%%%%%%%%%%%%%%%%%%%%%%%%%%%%%%%
\DescribeMacro{\...prefix}
In the alternative form |\childdocforwardprefix|,
%
\begin{center}
\begin{tabular}{l}
|\input{childdoc.def}|\\
|\childdocforwardprefix[|\textit{main}|]{|\textit{prefix}|}{|\textit{dest}|}|
\end{tabular}
\end{center}
%
the destination file is determined by a pattern
depending on the current file:
To make this work, the current file must be called
`{\textit{prefix}\hspace{0.2em}\textit{suffix}}'
with \textit{prefix} matching precisely the argument.
Processing is then passed on to the file
`{\textit{dest}\hspace{0.2em}\textit{suffix}}'.
Surely, the same effect is achieved by
directly specifying the
argument `{\textit{dest}\hspace{0.2em}\textit{suffix}}'
in the first form.
However, that requires to set up a different file
for each child. With the alternative form of the command
all these files can have exactly the same content
which simplifies setting them up and maintaining them.

For example, the following file |draft.tex|
with a compilation flag |\version| as described in \secref{sec:flags}
compiles the main document as a draft:
%
\begin{center}
\begin{tabular}{l}
|\def\version{draft}|\\
|\input{childdoc.def}|\\
|\childdocforward{|\textit{main}|}|
\end{tabular}
\end{center}
%
Likewise, the following files |final|\textit{nn}|.tex|
compile the final version of the child document
|child|\textit{nn}|.tex|:
%
\begin{center}
\begin{tabular}{l}
|\def\version{final}|\\
|\input{childdoc.def}|\\
|\childdocforwardprefix{final}{child}|
\end{tabular}
\end{center}
%

Note that when several versions of a main file and/or of each child file
are to be generated, it may be convenient to set up a |Makefile| or
shell script to automatise the process.

%%%%%%%%%%%%%%%%%%%%%%%%%%%%%%%%%%%%%%%%%%%%%%%%%%%%%%%%%%%%%%%%%%%%%%%%%%%%%%%%
\subsection{Command Line Processing}
\label{sec:commandline}

The effect of redirection files can also be achieved by invoking
the \LaTeX{} compiler with a more elaborate command line.
Most conveniently this should be done as part
of a shell script or a |Makefile|.

When using \textsf{childdoc} in the main file, the following
command lines effectively perform a redirection
(note that depending on the shell being used,
backslashes may have to be doubled: `|\|' $\to$ `|\\|'):
%
\begin{center}
|... -jobname "|\textit{target}|" |\\|"|[\textit{flags}]%
|\input{childdoc.def}\childdocforward[|\textit{main}|]{|\textit{dest}|}"|
\end{center}
%
Here \textit{target} is the name of the output file,
\textit{main} is the name of the main file
and \textit{dest} is the name of the main or child file to be processed
(all filenames without extensions).
The optional argument \textit{main} can be omitted
if \textit{main} matches \textit{dest}.
Optionally, compilation \textit{flags} can be defined via |\def| commands.
This command line makes the \TeX{} engine believe
it is compiling the file \textit{target}
whose content is specified as the latter parameter.
The provided code then forwards the processing to
\textit{main} or \textit{dest} as described in \secref{sec:forward}.

%%%%%%%%%%%%%%%%%%%%%%%%%%%%%%%%%%%%%%%%%%%%%%%%%%%%%%%%%%%%%%%%%%%%%%%%%%%%%%%%
\subsection{Include by Input}
\label{sec:input}

Including child documents by |\include| has some restrictions by design.
Most notably, the content of a child document always occupies
its own set of pages; pages cannot be shared between child documents.
Usually, this behaviour makes perfect sense
because each child document contain an essential part of the document.
However, in some situations it may be desirable to compose
a document from a collection of parts
without having mandatory page breaks between then.
For this case, the package
provides a mechanism to include parts
by |\input| which can also be processed individually.
However, by construction this mechanism
requires manual handling of the content to be output.

%%%%%%%%%%%%%%%%%%%%%%%%%%%%%%%%%%%%%%%%
\DescribeMacro{\ifchilddocmanual}
The main file should be prepared as usual, see \secref{sec:include}.
However, the document body must make a distinction
between processing of an individual part and of the main document, e.g.:
%
\begin{center}
\begin{tabular}{l}
|\ifchilddocmanual|\\
|\input{\childdocname}|\\
|\||else|\\
\textit{document body with }|\input{|\textit{part}|}|\\
|\||fi|
\end{tabular}
\end{center}
%
The conditional |\ifchilddocmanual| is true whenever
a part to be included by |\input| is being compiled,
and the name of the part is stored in |\childdocname|.

%%%%%%%%%%%%%%%%%%%%%%%%%%%%%%%%%%%%%%%%
\DescribeMacro{\childdocby}
Each part to be included by |\input| should start with:
%
\begin{center}
\begin{tabular}{l}
|\input{childdoc.def}|\\
|\childdocby{|\textit{main}|}|\\
\end{tabular}
\end{center}
%
The directive |\childdocby| is similar to |\childdocof|
described in \secref{sec:include},
but the subsequent selection of content must be done manually.
To that end, both |\ifchilddoc| and |\ifchilddocmanual|
will be true upon processing of a part,
and the name of the part is stored in |\childdocname|.
Note that |\jobname| will be set to the filename of the current part
so that each part receives an individual |.aux| file
that does not interfere with the |.aux| file(s) of the main document.
This behaviour can be altered by the alternative form
|\childdocby[*]{|\textit{main}|}| (with a non-empty optional argument)
which uses the |.aux| file of the main document
by setting |\jobname| to \textit{main}.

%%%%%%%%%%%%%%%%%%%%%%%%%%%%%%%%%%%%%%%%%%%%%%%%%%%%%%%%%%%%%%%%%%%%%%%%%%%%%%%%
\subsection{Driver Development}
\label{sec:driver}

The \textsf{childdoc} mechanism can also be use for the development
of definition files such as \LaTeX{} styles or classes.
This case differs from the above setup with multiple parts
included by |\include| in that no |\includeonly| should be invoked.
This can be achieved by starting the include file
(before |\ProvidesPackage|) with:
%
\begin{center}
\begin{tabular}{l}
|\input{childdoc.def}|\\
|\childdocforward{|\textit{main}|}|\\
\end{tabular}
\end{center}
%
or alternatively with:
%
\begin{center}
\begin{tabular}{l}
|\input{childdoc.def}|\\
|\childdocby{|\textit{main}|}|\\
\end{tabular}
\end{center}
%
Both forms have slightly different effects as described above.
The main file is prepared as usual, see \secref{sec:include}.

%%%%%%%%%%%%%%%%%%%%%%%%%%%%%%%%%%%%%%%%%%%%%%%%%%%%%%%%%%%%%%%%%%%%%%%%%%%%%%%%
\subsection{Legacy Detection}
\label{sec:detection}

The directive |\childdocmain| in the main file can detect
whether the complete document or merely a child is to be compiled
even without using the directive |\childdocof|.
This method is deprecated because it is less robust
and there is no compelling reason to use it;
it is merely provided for backward compatibility
and it may be removed in future versions.

If the detection mechanism is to be used,
it is mandatory to correctly specify
the filename of the main file as the argument of |\childdocmain|:
%
\begin{center}
\begin{tabular}{l}
|\input{childdoc.def}|\\
|\childdocmain{|\textit{main}|}|\\
\end{tabular}
\end{center}
%
If |\jobname| does not match the argument \textit{main} of |\childdocmain|,
it is assumed that |\jobname| points to the child file to be compiled.
When using |\childdocmain| with the main file specified as argument,
it suffices to start a child file
with just |\input{|\textit{main}|}|
without loading of the package and using |\childdocof|.
If instead all processing is done
with the appropriate \textsf{childdoc} directives,
the argument of \textit{main} of |\childdocmain| can be empty.

An alternative version of the command line processing described
in \secref{sec:commandline} using the detection mechanism reads:
%
\begin{center}
|... -jobname "|\textit{target}|" "|[\textit{flags}]%
[|\def\jobname{|\textit{dest}|}|]|\input{|\textit{main}|}"|
\end{center}

%%%%%%%%%%%%%%%%%%%%%%%%%%%%%%%%%%%%%%%%%%%%%%%%%%%%%%%%%%%%%%%%%%%%%%%%%%%%%%%%
\subsection{Manual Code}
\label{sec:manual}

In case one cannot be certain whether the definitions file |childdoc.def|
is installed on the target \TeX{} distribution
and one prefers not to ship it,
it is conceivable to paste a few relevant commands into the sources.

To that end, drop all statements |\input{childdoc.def}|
and perform the replacements as outlined below.
Instead of |\childdocmain{|\textit{main}|}| add the following code
to the top of the main file:
%
\begin{center}
\begin{tabular}{l}
|\||ifdefined\childdocname\endinput\||fi\newif\ifchilddoc|\\
|\edef\childdocname{\scantokens\expandafter{\jobname\noexpand}}|\\
|\def\childdocmain{|\textit{main}|}\||ifx\childdocmain\childdocname\||else|\\
|\childdoctrue\includeonly{\childdocname}\let\jobname\childdocmain\||fi|\\
\end{tabular}
\end{center}
%
Instead of |\childdocof{|\textit{main}|}| just include the main file
at the top of each child file:
%
\begin{center}
|\input{|\textit{main}|}|
\end{center}
%
A simple redirection |\childdocforward{|\textit{dest}|}| is achieved by:
%
\begin{center}
|\def\jobname{|\textit{dest}|}\input{\jobname}|
\end{center}
%
The redirection with prefix
|\childdocforwardprefix[|\textit{prefix}|]{|\textit{dest}|}|
is accomplished by:
%
\begin{center}
\begin{tabular}{l}
|{\edef\jobname{\scantokens\expandafter{\jobname\noexpand}}|\\
|\def\redirectjob |\textit{prefix}|#1~~~{\gdef\jobname{|\textit{dest}|#1}}|\\
|\expandafter\redirectjob\jobname~~~}\input{\jobname}|
\end{tabular}
\end{center}

In an alternative approach,
child documents can be compiled by a specific command line
without additional code or specific definitions:
%
\begin{center}
|... -jobname "|\textit{target}|" "|[\textit{flags}]%
|\includeonly{|\textit{dest}|}\input{|\textit{main}|}"|
\end{center}
%

%%%%%%%%%%%%%%%%%%%%%%%%%%%%%%%%%%%%%%%%%%%%%%%%%%%%%%%%%%%%%%%%%%%%%%%%%%%%%%%%
%%%%%%%%%%%%%%%%%%%%%%%%%%%%%%%%%%%%%%%%%%%%%%%%%%%%%%%%%%%%%%%%%%%%%%%%%%%%%%%%
\section{Information}

%%%%%%%%%%%%%%%%%%%%%%%%%%%%%%%%%%%%%%%%%%%%%%%%%%%%%%%%%%%%%%%%%%%%%%%%%%%%%%%%
\subsection{Copyright}

Copyright \copyright{} 2017--2018 Niklas Beisert

This work may be distributed and/or modified under the
conditions of the \LaTeX{} Project Public License, either version 1.3
of this license or (at your option) any later version.
The latest version of this license is in
  \url{http://www.latex-project.org/lppl.txt}
and version 1.3 or later is part of all distributions of \LaTeX{}
version 2005/12/01 or later.

This work has the LPPL maintenance status `maintained'.

The Current Maintainer of this work is Niklas Beisert.

This work consists of the files |README.txt|, |childdoc.ins| and |childdoc.dtx|
as well as the derived files |childdoc.def|, |cdocsamp.tex|
with |cdocsch1.tex|, |cdocsch2.tex|, |cdocspt3.tex|, |cdocspt4.tex|,
|cdocsdrf.tex|, |cdocsfn1.tex|, |cdocsfn2.tex|
as well as |childdoc.pdf|.

%%%%%%%%%%%%%%%%%%%%%%%%%%%%%%%%%%%%%%%%%%%%%%%%%%%%%%%%%%%%%%%%%%%%%%%%%%%%%%%%
\subsection{Files and Installation}

The package consists of the files:
%
\begin{center}
\begin{tabular}{ll}
    |README.txt|   & readme file \\
    |childdoc.ins| & installation file \\
    |childdoc.dtx| & source file \\
    |childdoc.def| & definition file \\
    |cdocsamp.tex| & sample main file \\
    |cdocsch1.tex| & sample include file \\
    |cdocsch2.tex| & sample include file \\
    |cdocspt3.tex| & sample part file \\
    |cdocspt4.tex| & sample part file \\
    |cdocsdrf.tex| & sample redirection file \\
    |cdocsfn1.tex| & sample redirection file \\
    |cdocsfn2.tex| & sample redirection file \\
    |childdoc.pdf| & manual
\end{tabular}
\end{center}
%
The distribution consists of the files
|README.txt|, |childdoc.ins| and |childdoc.dtx|.
%
\begin{itemize}
\item
Run (pdf)\LaTeX{} on |childdoc.dtx|
to compile the manual |childdoc.pdf| (this file).
\item
Run \LaTeX{} on |childdoc.ins| to create the definitions file |childdoc.def|
and the sample |cdocsamp.tex| with include files
|cdocsch1.tex|, |cdocsch2.tex|, |cdocspt3.tex|, |cdocspt4.tex|,
|cdocsdrf.tex|, |cdocsfn1.tex|, |cdocsfn2.tex|.
Then copy the file |childdoc.def| to an appropriate directory of your \LaTeX{}
distribution, e.g.\ \textit{texmf-root}|/tex/latex/childdoc|.
\end{itemize}

%%%%%%%%%%%%%%%%%%%%%%%%%%%%%%%%%%%%%%%%%%%%%%%%%%%%%%%%%%%%%%%%%%%%%%%%%%%%%%%%
\subsection{Related CTAN Packages}

There are several other packages which offer a similar functionality:
%
\begin{itemize}
\item
The packages
\href{http://ctan.org/pkg/docmute}{\textsf{docmute}},
\href{http://ctan.org/pkg/includex}{\textsf{includex}} and
\href{http://ctan.org/pkg/standalone}{\textsf{standalone}}
provide commands to include only the document body of
a child file thus allowing both files to be compiled individually.
\item
The packages \href{http://ctan.org/pkg/subdocs}{\textsf{subdocs}}
and \href{http://ctan.org/pkg/subfiles}{\textsf{subfiles}}
provide structures in which the main and child documents can be
encapsulated and allowing them to be compiled individually.
The inclusion mechanism is different from the conventional |\include|.
\item
The package \href{http://ctan.org/pkg/combine}{\textsf{combine}}
is an elaborate solution to combine several documents into one.
\end{itemize}
%
See also the CTAN topic \href{http://ctan.org/topic/subdocs}{\textsf{subdocs}}
for further related packages.
The present package differs from the above solutions in that
a document structure constructed with the conventional |\include| mechanism
just needs two extra commands at the top of every file
such that all constituent files can be compiled individually.

%%%%%%%%%%%%%%%%%%%%%%%%%%%%%%%%%%%%%%%%%%%%%%%%%%%%%%%%%%%%%%%%%%%%%%%%%%%%%%%%
%\subsection{Feature Suggestions}
%
%The following is a list of features which may be useful for future
%versions of this package:
%%
%\begin{itemize}
%\item
%\ldots
%\end{itemize}

%%%%%%%%%%%%%%%%%%%%%%%%%%%%%%%%%%%%%%%%%%%%%%%%%%%%%%%%%%%%%%%%%%%%%%%%%%%%%%%%
\subsection{Revision History}

%%%%%%%%%%%%%%%%%%%%%%%%%%%%%%%%%%%%%%%%
\paragraph{v2.0:} 2018/12/30

\begin{itemize}
\item
immediate forward processing
\item
added |\childdocby| mechanism
\item
manual restructured
\end{itemize}

%%%%%%%%%%%%%%%%%%%%%%%%%%%%%%%%%%%%%%%%
\paragraph{v1.6:} 2018/01/17

\begin{itemize}
\item
application for development of include files
\item
corrections to manual
\end{itemize}

%%%%%%%%%%%%%%%%%%%%%%%%%%%%%%%%%%%%%%%%
\paragraph{v1.5:} 2017/05/21

\begin{itemize}
\item
more complete structuring introduced
\item
|\childdocof| introduced
\item
|\childdoc| renamed to |\childdocmain|
\item
|\childredirect| renamed to |\childdocforward| and |\childdocforwardprefix|
and functionality expanded
\end{itemize}

%%%%%%%%%%%%%%%%%%%%%%%%%%%%%%%%%%%%%%%%
\paragraph{v1.0:} 2017/04/27

\begin{itemize}
\item
manual and install package
\item
first version published on CTAN
\end{itemize}

%%%%%%%%%%%%%%%%%%%%%%%%%%%%%%%%%%%%%%%%
\paragraph{v0.6:} 2017/04/26

\begin{itemize}
\item
redirection mechanism added
\end{itemize}

%%%%%%%%%%%%%%%%%%%%%%%%%%%%%%%%%%%%%%%%
\paragraph{v0.5:} 2017/04/26

\begin{itemize}
\item
functionality in definition file
\end{itemize}


%%%%%%%%%%%%%%%%%%%%%%%%%%%%%%%%%%%%%%%%%%%%%%%%%%%%%%%%%%%%%%%%%%%%%%%%%%%%%%%%
%%%%%%%%%%%%%%%%%%%%%%%%%%%%%%%%%%%%%%%%%%%%%%%%%%%%%%%%%%%%%%%%%%%%%%%%%%%%%%%%
%%%%%%%%%%%%%%%%%%%%%%%%%%%%%%%%%%%%%%%%%%%%%%%%%%%%%%%%%%%%%%%%%%%%%%%%%%%%%%%%
\appendix

\settowidth\MacroIndent{\rmfamily\scriptsize 000\ }

 \DocInput{childdoc.dtx}

\end{document}
%</driver>
% \fi
%
% %%%%%%%%%%%%%%%%%%%%%%%%%%%%%%%%%%%%%%%%%%%%%%%%%%%%%%%%%%%%%%%%%%%%%%%%%%%%%%
% %%%%%%%%%%%%%%%%%%%%%%%%%%%%%%%%%%%%%%%%%%%%%%%%%%%%%%%%%%%%%%%%%%%%%%%%%%%%%%
% \section{Sample}
%\iffalse
%<*samplemain>
%\fi
%
% The following presents a sample document
% with two chapters, two parts, a title page,
% a compile flag as well as three forwarding files to set the flag.
% It consists of eight |.tex| files:
% \begin{center}
% \begin{tabular}{ll}
% |cdocsamp.tex|&main file\\
% |cdocsch1.tex|&include file for chapter 1\\
% |cdocsch2.tex|&include file for chapter 2\\
% |cdocspt3.tex|&include file for part 3\\
% |cdocspt4.tex|&include file for part 4\\
% |cdocsdrf.tex|&forwarding file for main file in draft mode\\
% |cdocsfi1.tex|&forwarding file for final version of chapter 1\\
% |cdocsfi2.tex|&forwarding file for final version of chapter 2\\
% \end{tabular}
% \end{center}
% Each of the eight files can be compiled directly by the \LaTeX{} compiler.
%
% %%%%%%%%%%%%%%%%%%%%%%%%%%%%%%%%%%%%%%
% \paragraph{Main File.}
%
% The main file is called |cdocsamp.tex|.
%
% Load the \textsf{childdoc} definitions and
% declare the filename for the main document:
%    \begin{macrocode}
\input{childdoc.def}
\childdocmain{}
%    \end{macrocode}

% Optional override for |\version| flag:
%    \begin{macrocode}
%%\ifchilddoc\else\providecommand{\version}{draft}\fi
%    \end{macrocode}

% Define the default values for the |\version| flag
% (|final| for the main file and |draft| for childs):
%    \begin{macrocode}
\ifchilddoc
\providecommand{\version}{draft}
\else
\providecommand{\version}{final}
\fi
%    \end{macrocode}

% Load the standard document class:
%    \begin{macrocode}
\documentclass[12pt]{article}
%    \end{macrocode}

% Start the document body:
%    \begin{macrocode}
\begin{document}
%    \end{macrocode}

% Declare a title page.
% Print title, part of document being processed and version flag:
%    \begin{macrocode}
\addtocounter{page}{-1}
\begin{center}
{\LARGE\bfseries{}childdoc example\par}
\vspace{1cm}
\ifchilddoc
\ifchilddocmanual part\else chapter\fi:
`\childdocname' of `\childdocjob'\par
\else
main document: `\childdocjob'\par
\fi
version: \version\par
\end{center}
\newpage
%    \end{macrocode}

% Manually include selected file,
% otherwise process as usual:
%    \begin{macrocode}
\ifchilddocmanual
\section*{part `\childdocname'}
\input{\childdocname}
\else
%    \end{macrocode}

% Include the two chapters:
%    \begin{macrocode}
\include{cdocsch1}
\include{cdocsch2}
%    \end{macrocode}

% Include the two parts unless only chapters should be displayed:
%    \begin{macrocode}
\ifchilddoc\else
\section{part three}
\input{cdocspt3}
\section{part four}
\input{cdocspt4}
\fi
%    \end{macrocode}

% Process as usual until here:
%    \begin{macrocode}
\fi
%    \end{macrocode}

% End of document body:
%    \begin{macrocode}
\end{document}
%    \end{macrocode}
%\iffalse
%</samplemain>
%\fi
%
% %%%%%%%%%%%%%%%%%%%%%%%%%%%%%%%%%%%%%%
% \paragraph{Chapter Include Files.}
%
% The include files are called |cdocsch1.tex| and |cdocsch2.tex|.
%
%\iffalse
%<*samplechap1|samplechap2>
%\fi

% Optional override for |\version| flag:
%    \begin{macrocode}
%%\providecommand{\version}{final}
%    \end{macrocode}

% Include the main document:
%    \begin{macrocode}
\input{childdoc.def}
\childdocof{cdocsamp}
%    \end{macrocode}

%\iffalse
%</samplechap1|samplechap2>
%\fi
%
%\iffalse
%<*samplechap1>
%\fi
% Some text for chapter 1:
%    \begin{macrocode}
\section{one}
some text in chapter one
%    \end{macrocode}

%\iffalse
%</samplechap1>
%\fi
% Some text for chapter 2:
%\iffalse
%<*samplechap2>
%\fi
%    \begin{macrocode}
\section{two}
more text in chapter two
%    \end{macrocode}

%\iffalse
%</samplechap2>
%\fi
%
% %%%%%%%%%%%%%%%%%%%%%%%%%%%%%%%%%%%%%%
% \paragraph{Part Include Files.}
%
% The include files are called |cdocspt3.tex| and |cdocspt4.tex|.
%
%\iffalse
%<*samplepart3|samplepart4>
%\fi

% Optional override for |\version| flag:
%    \begin{macrocode}
%%\providecommand{\version}{final}
%    \end{macrocode}

% Include the main document:
%    \begin{macrocode}
\input{childdoc.def}
\childdocby{cdocsamp}
%    \end{macrocode}

%\iffalse
%</samplepart3|samplepart4>
%\fi
%
%\iffalse
%<*samplepart3>
%\fi
% Some text for part 3:
%    \begin{macrocode}
some text in part three
%    \end{macrocode}

%\iffalse
%</samplepart3>
%\fi
% Some text for part 4:
%\iffalse
%<*samplepart4>
%\fi
%    \begin{macrocode}
more text in part four
%    \end{macrocode}

%\iffalse
%</samplepart4>
%\fi
%
% %%%%%%%%%%%%%%%%%%%%%%%%%%%%%%%%%%%%%%
% \paragraph{Forwarding for a Complete Draft.}
%
% The following forwarding file |cdocsdrf.tex|
% compiles the main document in draft mode:
%\iffalse
%<*sampledraft>
%\fi
%    \begin{macrocode}
\def\version{draft}
\input{childdoc.def}
\childdocforward{cdocsamp}
%    \end{macrocode}

%\iffalse
%</sampledraft>
%\fi
%
% %%%%%%%%%%%%%%%%%%%%%%%%%%%%%%%%%%%%%%
% \paragraph{Forwarding for Final Version of the Chapters.}
%
% The following forwarding files |cdocsfn1.tex| and |cdocsfn2.tex|
% (with identical content)
% compile the final versions of the child documents
% |cdocsch1.tex| and |cdocsch2.tex|, respectively:
%\iffalse
%<*samplefinal>
%\fi
%    \begin{macrocode}
\def\version{final}
\input{childdoc.def}
\childdocforwardprefix[cdocsamp]{cdocsfn}{cdocsch}
%    \end{macrocode}

%\iffalse
%</samplefinal>
%\fi
%
% %%%%%%%%%%%%%%%%%%%%%%%%%%%%%%%%%%%%%%
% \paragraph{Command Line Processing.}
%
% The following three command lines generate the output files
% |cdocscld|, |cdocscl1| and |cdocscl2|
% which should be identical to
% |cdocsdrf|, |cdocsch1| and |cdocsfn2|, respectively:
% \begin{center}
% \begin{tabular}{l}
% |latex -jobname cdocscld \|\\
% |  "\def\version{draft}\input{childdoc.def}\childdocforward{cdocsamp}"|\\
% |latex -jobname cdocscl1 \|\\
% |  "\input{childdoc.def}\childdocforward[cdocsamp]{cdocsch1}"|\\
% |latex -jobname cdocscl2 \|\\
% |  "\def\version{final}\input{childdoc.def}\childdocforward{cdocsch2}"|
% \end{tabular}
% \end{center}
% Note that the trailing backslash on each first line
% merely continues the input to the second line
% (for convenient cut ant paste).
% Furthermore, the command |latex| can be replaced by any
% of its alternative versions such as |pdflatex|.
%
% %%%%%%%%%%%%%%%%%%%%%%%%%%%%%%%%%%%%%%%%%%%%%%%%%%%%%%%%%%%%%%%%%%%%%%%%%%%%%%
% %%%%%%%%%%%%%%%%%%%%%%%%%%%%%%%%%%%%%%%%%%%%%%%%%%%%%%%%%%%%%%%%%%%%%%%%%%%%%%
% \section{Implementation}
%\iffalse
%<*package>
%\fi
%
% This section describes the definitions file |childdoc.def|.

% The definitions cannot be loaded using |\usepackage| or |\RequirePackage|
% which has a mechanism to prevent loading a style file more than once.
% When loading the definitions by means of |\input|
% multiple instances have to be prevented manually:
%\iffalse
%This code needs to be before the `\ProvidesFile' directive
%which is defined at the beginning of this file.
%Therefore it is also placed there and commented out here.
%</package>
%<*discard>
%\fi
%    \begin{macrocode}
\ifdefined\childdocmain\endinput\fi
%    \end{macrocode}
%\iffalse
%</discard>
%<*package>
%\fi
%
% \macro{\ifchilddoc}
% \macro{\ifchilddocmanual}
% The conditional |\ifchilddoc| tells whether a
% child (true) or main (false) document is being compiled.
% The conditional |\ifchilddocmanual| tells whether
% the |\includeonly| mechanism is used (false) or
% the selection of child files must be performed manually (true).
% The definitions initialise to false:
%    \begin{macrocode}
\newif\ifchilddoc
\newif\ifchilddocmanual
%    \end{macrocode}

% \macro{\childdocname}
% \macro{\childdocjob}
% The macro |\childdocname| stores the name of the main document
% to be compiled. The macro |\childdocjob| stores the name of
% the document on which the \LaTeX{} compiler was originally invoked.
% The content of |\jobname| cannot be compared
% to filenames specified in the source due to different catcodes.
% The following code rescans |\jobname|, stores the result
% in |\childdocname| and saves a copy in |\childdocjob|:
%    \begin{macrocode}
\edef\childdocname{\scantokens\expandafter{\jobname\noexpand}}
\let\childdocjob\childdocname
%    \end{macrocode}

% \macro{\childdocdisable}
% The macro |\childdocdisable| prevents the main file
% from being processed more than once.
% At this stage, the main document command |\childdocmain|
% is assumed to be called once again where it should do nothing.
% Any subsequent call to it should prevent
% a secondary processing of the main document
% It overwrites the forwarding commands
% |\childdocof| and |\childdocforward|
% with empty macros to prevent further inclusions of the main document:
%    \begin{macrocode}
\newcommand{\childdocdisable}
{
  \renewcommand{\childdocmain}[1]{\renewcommand{\childdocmain}[1]{\endinput}}
  \renewcommand{\childdocof}[1]{}
  \renewcommand{\childdocby}[2][]{}
  \renewcommand{\childdocforward}[2][]{}
  \renewcommand{\childdocdisable}{}
}
%    \end{macrocode}

% \macro{\childdocmain}
% The macro |\childdocmain| is to be called at the top of the main file
% with nothing or the main filename (without extension) as argument.
% First, it breaks loops.
% If the argument is not empty and does not match |\childdocname|
% (which is set by the first inclusion of |childdoc.def|),
% |\ifchilddoc| is set to true, |\includeonly| is applied to the child file
% and |\jobname| is set to the main file
% (for proper handling of |.aux| files):
%    \begin{macrocode}
\newcommand{\childdocmain}[1]
{
  \childdocdisable\childdocmain{}
  \if?#1?\else
    \begingroup
      \def\childdoctmp{#1}
      \ifx\childdoctmp\childdocname
        \def\childdoctmp{}
      \else
        \def\childdoctmp
        {
          \childdoctrue
          \includeonly{\childdocname}
          \def\childdocjob{#1}
          \def\jobname{#1}
        }
      \fi
      \expandafter
    \endgroup
    \childdoctmp
  \fi
}
%    \end{macrocode}

% \macro{\childdocof}
% The command |\childdocof| redirects
% compilation to the main file |#1|.
%    \begin{macrocode}
\newcommand{\childdocof}[1]
{
  \childdocdisable
  \childdoctrue
  \includeonly{\childdocname}
  \def\jobname{#1}
  \def\childdocjob{#1}
  \input{#1}
}
%    \end{macrocode}

% \macro{\childdocby}
% The command |\childdocby| ....
%    \begin{macrocode}
\newcommand{\childdocby}[2][]
{
  \childdocdisable
  \childdoctrue
  \childdocmanualtrue
  \if?#1?\else
    \def\jobname{#2}
  \fi
  \def\childdocjob{#2}
  \input{#2}
  \endinput
}
%    \end{macrocode}

% \macro{\childdocforward}
% The command |\childdocforward| redirects
% compilation to the main file or
% (if the optional argument is given) a child file.
% Parameters are set as if the main file
% or a child file starting with |\childdocof| was compiled.
% Then compilation is handed over to the main file:
%    \begin{macrocode}
\newcommand{\childdocforward}[2][]
{
  \begingroup
    \if?#1?
      \def\childdoctmp
      {
        \def\childdocname{#2}
        \def\childdocjob{#2}
        \def\jobname{#2}
        \input{#2}
        \endinput
      }
    \else
      \def\childdoctmp
      {
        \childdocdisable
        \def\childdocname{#2}
        \childdoctrue
        \includeonly{#2}
        \def\childdocjob{#1}
        \def\jobname{#1}
        \input{#1}
        \endinput
      }
    \fi
    \expandafter
  \endgroup
  \childdoctmp
}
%    \end{macrocode}

% \macro{\childdocforwardprefix}
% The command |\childdocforwardprefix| redirects
% compilation to the main or a child file by means of a pattern.
% The prefix |#1| in the current filename is replaced by |#2|
% and the suffix of the current filename is kept
% (it is assumed that the filename does not contain the substring `|~~~|'
% which is used as a delimiter).
% Compilation is handed over to the new file by |\childdocforward|:
%    \begin{macrocode}
\newcommand{\childdocforwardprefix}[3][]
{
  \begingroup
    \def\childdocextract #2##1~~~{\def\childdoctmp{\childdocforward[#1]{#3##1}}}
    \expandafter\childdocextract\childdocname~~~
    \expandafter
  \endgroup
  \childdoctmp
}
%    \end{macrocode}

% \macro{\childdoc}
% The deprecated macro |\childdoc| is a legacy version of |\childdocmain|:
%    \begin{macrocode}
\newcommand{\childdoc}{\childdocmain}
%    \end{macrocode}

% \macro{\childdocredirect}
% The deprecated macro |\childdocredirect| is a legacy version
% of |\childdocforward| and |\childdocforwardprefix|:
%    \begin{macrocode}
\newcommand{\childdocredirect}[2][]
{
  \begingroup
    \if?#1?
      \def\childdoctmp{\childdocforward{#2}}
    \else
      \def\childdoctmp{\childdocforwardprefix{#1}{#2}}
    \fi
    \expandafter
  \endgroup
  \childdoctmp
}
%    \end{macrocode}

%\iffalse
%</package>
%\fi
%
\endinput
|\\
|\childdocforward[|\textit{main}|]{|\textit{dest}|}|\\
\end{tabular}
\end{center}
%
The argument \textit{dest} is the destination file
(without extension).
It should be the main file or one of the child files.
Note that further \textsf{childdoc} directives
such as |\childdocof| and |\childdocforward|
in the indicated file will be processed in this form.
The optional argument \textit{main}
passes on directly to the main file \textit{main}
while pretending to compile the child \textit{dest}.
This form behaves as if \textit{dest}
issues |\childdocof{|\textit{main}|}| right away,
and no further \textsf{childdoc} directives will be processed.

%%%%%%%%%%%%%%%%%%%%%%%%%%%%%%%%%%%%%%%%
\DescribeMacro{\...prefix}
In the alternative form |\childdocforwardprefix|,
%
\begin{center}
\begin{tabular}{l}
|% \iffalse
%
% childdoc.dtx Copyright (C) 2017-2018 Niklas Beisert
%
% This work may be distributed and/or modified under the
% conditions of the LaTeX Project Public License, either version 1.3
% of this license or (at your option) any later version.
% The latest version of this license is in
%   http://www.latex-project.org/lppl.txt
% and version 1.3 or later is part of all distributions of LaTeX
% version 2005/12/01 or later.
%
% This work has the LPPL maintenance status `maintained'.
%
% The Current Maintainer of this work is Niklas Beisert.
%
% This work consists of the files childdoc.dtx and childdoc.ins
% and the derived files childdoc.def and cdocsamp.tex with
% cdocsch1.tex, cdocsch2.tex, cdocsdrf.tex, cdocsfn1.tex, cdocsfn2.tex.
%
%<package>\ifdefined\childdocmain\endinput\fi
%<package>\ProvidesFile{childdoc.def}[2018/12/30 v2.0 child document driver]
%<samplemain>\ProvidesFile{cdocsamp.tex}[2018/12/30 v2.0 sample for childdoc]
%<*driver>
%\ProvidesFile{childdoc.drv}[2018/12/30 v2.0 childdoc reference manual file]
\PassOptionsToClass{10pt,a4paper}{article}
\documentclass{ltxdoc}

\usepackage[margin=35mm]{geometry}
\usepackage{hyperref}
\usepackage{hyperxmp}
\usepackage[usenames]{color}

\hypersetup{colorlinks=true}
\hypersetup{pdfstartview=FitH}
\hypersetup{pdfpagemode=UseNone}
\hypersetup{pdfsource={}}
\hypersetup{pdflang={en-UK}}
\hypersetup{pdfcopyright={Copyright 2017-2018 Niklas Beisert.
  This work may be distributed and/or modified under the
  conditions of the LaTeX Project Public License, either version 1.3
  of this license or (at your option) any later version.}}
\hypersetup{pdflicenseurl={http://www.latex-project.org/lppl.txt}}
\hypersetup{pdfcontactaddress={ETH Zurich, ITP, HIT K,
  Wolfgang-Pauli-Strasse 27}}
\hypersetup{pdfcontactpostcode={8093}}
\hypersetup{pdfcontactcity={Zurich}}
\hypersetup{pdfcontactcountry={Switzerland}}
\hypersetup{pdfcontactemail={nbeisert@itp.phys.ethz.ch}}
\hypersetup{pdfcontacturl={http://people.phys.ethz.ch/\xmptilde nbeisert/}}

\newcommand{\secref}[1]{\hyperref[#1]{section \ref*{#1}}}

\parskip1ex
\parindent0pt
\let\olditemize\itemize
\def\itemize{\olditemize\parskip0pt}

\begin{document}

\title{The \textsf{childdoc} Package}
\hypersetup{pdftitle={The childdoc Package}}
\author{Niklas Beisert\\[2ex]
  Institut f\"ur Theoretische Physik\\
  Eidgen\"ossische Technische Hochschule Z\"urich\\
  Wolfgang-Pauli-Strasse 27, 8093 Z\"urich, Switzerland\\[1ex]
  \href{mailto:nbeisert@itp.phys.ethz.ch}
  {\texttt{nbeisert@itp.phys.ethz.ch}}}
\hypersetup{pdfauthor={Niklas Beisert}}
\hypersetup{pdfsubject={Manual for the LaTeX2e Package childdoc}}
\date{30 December 2018, \textsf{v2.0}}
\maketitle

\begin{abstract}\noindent
\textsf{childdoc} is a \LaTeXe{} package
that enables the direct compilation
of document sections included by |\include|
to individual files.
\end{abstract}

\begingroup
\parskip0ex
\tableofcontents
\endgroup

%%%%%%%%%%%%%%%%%%%%%%%%%%%%%%%%%%%%%%%%%%%%%%%%%%%%%%%%%%%%%%%%%%%%%%%%%%%%%%%%
%%%%%%%%%%%%%%%%%%%%%%%%%%%%%%%%%%%%%%%%%%%%%%%%%%%%%%%%%%%%%%%%%%%%%%%%%%%%%%%%
\section{Introduction}

\LaTeX{} provides a mechanism to structure a large document (such as a book)
into a main file and several child files (containing the chapters)
using the |\include| command.
This mechanism is beneficial for documents
which span hundreds of pages in order to
make the source file(s) more manageable.
Moreover, compilation can be restricted to
selected child files by means of the |\includeonly| command.
The latter feature can be used to reduce the compilation time while editing
(this was significantly more useful in the earlier days of \LaTeX{})
or to generate a smaller document which is easier to navigate.
Another application of |\includeonly| is to generate
documents consisting of selected parts of the complete document.

However, there are a few drawbacks of the plain |\include| mechanism:
\begin{itemize}
\item
The child files cannot be compiled on their own,
they can only be compiled via the main file.
A naive editing environment
(such as a text editor with an option
to have the current file processed by \LaTeX)
may require one to switch to the main file before compiling;
attempting to compile the child file produces errors.
\item
The main file must be modified (each time)
to adjust the |\includeonly| command
to the present needs. This easily leaves the main file in a messy state.
\item
The generated document will always carry the filename
of the main document. This is inconvenient if
several child files are to be compiled and
to be kept for distribution.
\end{itemize}

The present package provides a simple interface
to make child files individually compilable by \LaTeX{}.
Compiling a child file then has the same effect as compiling
the main file with an |\includeonly| command
to select the appropriate child.
Moreover the generated document will carry the name of the child
rather than the main file.
This resolves all three above issues.

This feature is meant to make the editing of books,
thesis documents and lecture notes somewhat more convenient.
However, the package can also be used efficiently for
composing a series of documents (such as exercise sheets)
which are typically distributed individually.
It then assists the author in generating the individual documents
(potentially in different versions)
as well as a document containing the collected series.
Another application is in developing style files
or other kinds of included material
where compilation of the style file could redirect
to a sample or test file.

%%%%%%%%%%%%%%%%%%%%%%%%%%%%%%%%%%%%%%%%%%%%%%%%%%%%%%%%%%%%%%%%%%%%%%%%%%%%%%%%
%%%%%%%%%%%%%%%%%%%%%%%%%%%%%%%%%%%%%%%%%%%%%%%%%%%%%%%%%%%%%%%%%%%%%%%%%%%%%%%%
\section{Usage}

First of all, the package \textsf{childdoc} is \emph{not} a standard
\LaTeXe{} |.sty| style file! Therefore it needs to be invoked in
a non-standard way.

%%%%%%%%%%%%%%%%%%%%%%%%%%%%%%%%%%%%%%%%%%%%%%%%%%%%%%%%%%%%%%%%%%%%%%%%%%%%%%%%
\subsection{Included Files}
\label{sec:include}

%%%%%%%%%%%%%%%%%%%%%%%%%%%%%%%%%%%%%%%%
\DescribeMacro{\childdocmain}
To use the package, add the commands
\begin{center}
\begin{tabular}{l}
|\input{childdoc.def}|\\
|\childdocmain{}|\\
\end{tabular}
\end{center}
at the very top of the main \LaTeX{} file,
in particular \emph{before} the |\documentclass| statement!
The argument of |\childdocmain| should be left empty
(but it must be present).

%%%%%%%%%%%%%%%%%%%%%%%%%%%%%%%%%%%%%%%%
\DescribeMacro{\childdocof}
Furthermore, add the commands
\begin{center}
\begin{tabular}{l}
|\input{childdoc.def}|\\
|\childdocof{|\textit{main}|}|\\
\end{tabular}
\end{center}
at the top of every child file \textit{child}
which is included by |\include{|\textit{child}|}|
from within the main file
(or at least for those files to be compiled individually).
The argument \textit{main} must be the filename of the main file.

There are a couple of
considerations in setting up the main and child documents:

%%%%%%%%%%%%%%%%%%%%%%%%%%%%%%%%%%%%%%%%
\paragraph{Restrictions.}

Please note the following restrictions:
\begin{itemize}
\item
|\childdocmain| must be called with one argument \textit{main}
to ensure compatibility with earlier version of the package.
It must either be empty (|\childdocmain{}|)
or precisely match the filename of the main file in which it is specified.
See \secref{sec:detection} for further information.
\item
The filename \textit{main} must be specified without the |.tex| extension.
\item
The filename \textit{main} is case sensitive
(even in case-insensitive file systems)
due to internal string comparison.
\item
The argument \textit{main} should be fully expanded, it cannot be a macro.
\item
Subdirectories and special characters should be avoided in filenames.
\item
The command |\childdocmain{|\textit{main}|}| must be followed by a whitespace.
It should not be followed immediately by another command
or by a comment mark `|%|'.
This is because the \TeX{} parser reads the token immediately following
the argument of |\childdocmain| and puts it
at the beginning of every child section;
however, a white\-space is ignored.
\end{itemize}

%%%%%%%%%%%%%%%%%%%%%%%%%%%%%%%%%%%%%%%%
\paragraph{Content of Main File.}

It is advisable to place all content in the child files included by |\include|.
Any output contained in the main file will appear in all child documents
unless suppressed manually;
it cannot be suppressed automatically by the |\includeonly| directive
and thus should normally be avoided.
A method to include some content in the main file
by means of conditional processing is described in \secref{sec:conditional}.

%%%%%%%%%%%%%%%%%%%%%%%%%%%%%%%%%%%%%%%%
\paragraph{Page Numbering.}

When only a part of the document is compiled,
the appropriate numbering of pages
(as well as other status parameters)
is determined from the |.aux| files.
The latter contain information from previous passes.
However this information needs to propagate through
all intermediate child documents.
Therefore the page numbering in child documents may well
be inconsistent until the complete document is compiled at least once.

A useful (if unconventional) way to always ensure a consistent
page numbering is to restart the numbering in each child document
and denote the pages by `\textit{child}|.|\textit{page}'
where \textit{child} represents the chapter/section number of the child file.
This can be achieved by the command
|\numberwithin{page}{|\textit{child}|}|
of the \textsf{amsmath} package
where \textit{child} can be |chapter| or |section|
depending on the chosen structuring.
Alternatively, one can modify the macro |\thepage| appropriately
and reset the counter |page| at the start of each child file.

%%%%%%%%%%%%%%%%%%%%%%%%%%%%%%%%%%%%%%%%%%%%%%%%%%%%%%%%%%%%%%%%%%%%%%%%%%%%%%%%
\subsection{Conditional Processing}
\label{sec:conditional}

The package provides a mechanism to compile different versions
of a document. To customise the versions further some conditional processing
can come in handy to distinguish which version is being compiled.
The package provides two macros to describe the compilation context:

%%%%%%%%%%%%%%%%%%%%%%%%%%%%%%%%%%%%%%%%
\DescribeMacro{\ifchilddoc}
The conditional |\ifchilddoc| distinguishes between the compilation of
child documents and the main document:
%
\begin{center}
|\ifchilddoc |\textit{child-code}| |[|\||else |\textit{main-code}]| \||fi|
\end{center}

%%%%%%%%%%%%%%%%%%%%%%%%%%%%%%%%%%%%%%%%
\DescribeMacro{\childdocname}
\DescribeMacro{\childdocjob}
The macro |\childdocname| contains the filename (without extension)
of the main or child file being processed.
Note that |\childdocjob| will always contain the name of the main file.

%%%%%%%%%%%%%%%%%%%%%%%%%%%%%%%%%%%%%%%%
\paragraph{Title Page.}

Conditional processing can be used to include a title or banner page
in the main document when proper precautions are taken.
Importantly, the code in the main file should ensure that the page counter
(as well as other status parameters which are stored in the |.aux| files)
takes the same value after the conditional processing.
Otherwise the page numbers may take divergent values
depending on which part is compiled.

For example, a title page could be declared by:
%
\begin{center}
\begin{tabular}{l}
|\ifchilddoc\||else|\\
|\addtocounter{page}{-1}|\\
\textit{code for title page}\\
|\newpage|\\
|\||fi|
\end{tabular}
\end{center}
%
A banner page for the child documents can be generated by:
%
\begin{center}
\begin{tabular}{l}
|\ifchilddoc|\\
|\addtocounter{page}{-1}|\\
\textit{code for banner page}\\
|\newpage|\\
|\||fi|
\end{tabular}
\end{center}
%
Here one could write a message such as:
\begin{center}
|This is the part \childdocname{} of \childdocjob{}.|
\end{center}

%%%%%%%%%%%%%%%%%%%%%%%%%%%%%%%%%%%%%%%%%%%%%%%%%%%%%%%%%%%%%%%%%%%%%%%%%%%%%%%%
\subsection{Flags}
\label{sec:flags}

The package makes it easy to generate different versions
of the main or child documents.
To this end compilation flags can be defined
and assigned different default values.
They will be particularly useful in conjunction
with the forwarding mechanism described in \secref{sec:forward}.

For example, it may be useful to have a flag |\version|
which can be set to |draft| or |final|.
The document source will contain some conditional code
depending on the value of |\version|.
Suppose further, the flag should default to |final| for the main file
and to |draft| for child files
which is a natural assignment for editing the document.
This is achieved by placing the following code
in the preamble of the main document
(below the |\childdocmain| directive):
%
\begin{center}
\begin{tabular}{l}
|\ifchilddoc|\\
|\providecommand{\version}{draft}|\\
|\||else|\\
|\providecommand{\version}{final}|\\
|\||fi|
\end{tabular}
\end{center}
%
The definition by |\providecommand| makes sure
that previous definitions are not overwritten.
Further statements |\providecommand{\version}{...}|
can thus be added before the above code to override it.

For the main file, one might add a line
(between |\childdocmain| and the above block)
%
\begin{center}
|%\ifchilddoc\||else\providecommand{\version}{draft}\||fi|
\end{center}
%
which can be uncommented to produce a draft version.
Likewise one can add a line to the very top of a child file
(above the |\childdocof{|\textit{main}|}| directive)
%
\begin{center}
|%\providecommand{\version}{final}|
\end{center}
%
which can be uncommented to produce the final version of this child document.

%%%%%%%%%%%%%%%%%%%%%%%%%%%%%%%%%%%%%%%%%%%%%%%%%%%%%%%%%%%%%%%%%%%%%%%%%%%%%%%%
\subsection{Forwarding}
\label{sec:forward}

Different versions of the main or child documents
using compilation flags as described in \secref{sec:flags}
can be (permanently) stored in different files
for convenient compilation, viewing and distribution.
To this end, the package defines a command
to pass on compilation to a different file:

%%%%%%%%%%%%%%%%%%%%%%%%%%%%%%%%%%%%%%%%
\DescribeMacro{\childdocforward}
The command |\childdocforward| redirects processing to
another source file:
%
\begin{center}
\begin{tabular}{l}
|\input{childdoc.def}|\\
|\childdocforward[|\textit{main}|]{|\textit{dest}|}|\\
\end{tabular}
\end{center}
%
The argument \textit{dest} is the destination file
(without extension).
It should be the main file or one of the child files.
Note that further \textsf{childdoc} directives
such as |\childdocof| and |\childdocforward|
in the indicated file will be processed in this form.
The optional argument \textit{main}
passes on directly to the main file \textit{main}
while pretending to compile the child \textit{dest}.
This form behaves as if \textit{dest}
issues |\childdocof{|\textit{main}|}| right away,
and no further \textsf{childdoc} directives will be processed.

%%%%%%%%%%%%%%%%%%%%%%%%%%%%%%%%%%%%%%%%
\DescribeMacro{\...prefix}
In the alternative form |\childdocforwardprefix|,
%
\begin{center}
\begin{tabular}{l}
|\input{childdoc.def}|\\
|\childdocforwardprefix[|\textit{main}|]{|\textit{prefix}|}{|\textit{dest}|}|
\end{tabular}
\end{center}
%
the destination file is determined by a pattern
depending on the current file:
To make this work, the current file must be called
`{\textit{prefix}\hspace{0.2em}\textit{suffix}}'
with \textit{prefix} matching precisely the argument.
Processing is then passed on to the file
`{\textit{dest}\hspace{0.2em}\textit{suffix}}'.
Surely, the same effect is achieved by
directly specifying the
argument `{\textit{dest}\hspace{0.2em}\textit{suffix}}'
in the first form.
However, that requires to set up a different file
for each child. With the alternative form of the command
all these files can have exactly the same content
which simplifies setting them up and maintaining them.

For example, the following file |draft.tex|
with a compilation flag |\version| as described in \secref{sec:flags}
compiles the main document as a draft:
%
\begin{center}
\begin{tabular}{l}
|\def\version{draft}|\\
|\input{childdoc.def}|\\
|\childdocforward{|\textit{main}|}|
\end{tabular}
\end{center}
%
Likewise, the following files |final|\textit{nn}|.tex|
compile the final version of the child document
|child|\textit{nn}|.tex|:
%
\begin{center}
\begin{tabular}{l}
|\def\version{final}|\\
|\input{childdoc.def}|\\
|\childdocforwardprefix{final}{child}|
\end{tabular}
\end{center}
%

Note that when several versions of a main file and/or of each child file
are to be generated, it may be convenient to set up a |Makefile| or
shell script to automatise the process.

%%%%%%%%%%%%%%%%%%%%%%%%%%%%%%%%%%%%%%%%%%%%%%%%%%%%%%%%%%%%%%%%%%%%%%%%%%%%%%%%
\subsection{Command Line Processing}
\label{sec:commandline}

The effect of redirection files can also be achieved by invoking
the \LaTeX{} compiler with a more elaborate command line.
Most conveniently this should be done as part
of a shell script or a |Makefile|.

When using \textsf{childdoc} in the main file, the following
command lines effectively perform a redirection
(note that depending on the shell being used,
backslashes may have to be doubled: `|\|' $\to$ `|\\|'):
%
\begin{center}
|... -jobname "|\textit{target}|" |\\|"|[\textit{flags}]%
|\input{childdoc.def}\childdocforward[|\textit{main}|]{|\textit{dest}|}"|
\end{center}
%
Here \textit{target} is the name of the output file,
\textit{main} is the name of the main file
and \textit{dest} is the name of the main or child file to be processed
(all filenames without extensions).
The optional argument \textit{main} can be omitted
if \textit{main} matches \textit{dest}.
Optionally, compilation \textit{flags} can be defined via |\def| commands.
This command line makes the \TeX{} engine believe
it is compiling the file \textit{target}
whose content is specified as the latter parameter.
The provided code then forwards the processing to
\textit{main} or \textit{dest} as described in \secref{sec:forward}.

%%%%%%%%%%%%%%%%%%%%%%%%%%%%%%%%%%%%%%%%%%%%%%%%%%%%%%%%%%%%%%%%%%%%%%%%%%%%%%%%
\subsection{Include by Input}
\label{sec:input}

Including child documents by |\include| has some restrictions by design.
Most notably, the content of a child document always occupies
its own set of pages; pages cannot be shared between child documents.
Usually, this behaviour makes perfect sense
because each child document contain an essential part of the document.
However, in some situations it may be desirable to compose
a document from a collection of parts
without having mandatory page breaks between then.
For this case, the package
provides a mechanism to include parts
by |\input| which can also be processed individually.
However, by construction this mechanism
requires manual handling of the content to be output.

%%%%%%%%%%%%%%%%%%%%%%%%%%%%%%%%%%%%%%%%
\DescribeMacro{\ifchilddocmanual}
The main file should be prepared as usual, see \secref{sec:include}.
However, the document body must make a distinction
between processing of an individual part and of the main document, e.g.:
%
\begin{center}
\begin{tabular}{l}
|\ifchilddocmanual|\\
|\input{\childdocname}|\\
|\||else|\\
\textit{document body with }|\input{|\textit{part}|}|\\
|\||fi|
\end{tabular}
\end{center}
%
The conditional |\ifchilddocmanual| is true whenever
a part to be included by |\input| is being compiled,
and the name of the part is stored in |\childdocname|.

%%%%%%%%%%%%%%%%%%%%%%%%%%%%%%%%%%%%%%%%
\DescribeMacro{\childdocby}
Each part to be included by |\input| should start with:
%
\begin{center}
\begin{tabular}{l}
|\input{childdoc.def}|\\
|\childdocby{|\textit{main}|}|\\
\end{tabular}
\end{center}
%
The directive |\childdocby| is similar to |\childdocof|
described in \secref{sec:include},
but the subsequent selection of content must be done manually.
To that end, both |\ifchilddoc| and |\ifchilddocmanual|
will be true upon processing of a part,
and the name of the part is stored in |\childdocname|.
Note that |\jobname| will be set to the filename of the current part
so that each part receives an individual |.aux| file
that does not interfere with the |.aux| file(s) of the main document.
This behaviour can be altered by the alternative form
|\childdocby[*]{|\textit{main}|}| (with a non-empty optional argument)
which uses the |.aux| file of the main document
by setting |\jobname| to \textit{main}.

%%%%%%%%%%%%%%%%%%%%%%%%%%%%%%%%%%%%%%%%%%%%%%%%%%%%%%%%%%%%%%%%%%%%%%%%%%%%%%%%
\subsection{Driver Development}
\label{sec:driver}

The \textsf{childdoc} mechanism can also be use for the development
of definition files such as \LaTeX{} styles or classes.
This case differs from the above setup with multiple parts
included by |\include| in that no |\includeonly| should be invoked.
This can be achieved by starting the include file
(before |\ProvidesPackage|) with:
%
\begin{center}
\begin{tabular}{l}
|\input{childdoc.def}|\\
|\childdocforward{|\textit{main}|}|\\
\end{tabular}
\end{center}
%
or alternatively with:
%
\begin{center}
\begin{tabular}{l}
|\input{childdoc.def}|\\
|\childdocby{|\textit{main}|}|\\
\end{tabular}
\end{center}
%
Both forms have slightly different effects as described above.
The main file is prepared as usual, see \secref{sec:include}.

%%%%%%%%%%%%%%%%%%%%%%%%%%%%%%%%%%%%%%%%%%%%%%%%%%%%%%%%%%%%%%%%%%%%%%%%%%%%%%%%
\subsection{Legacy Detection}
\label{sec:detection}

The directive |\childdocmain| in the main file can detect
whether the complete document or merely a child is to be compiled
even without using the directive |\childdocof|.
This method is deprecated because it is less robust
and there is no compelling reason to use it;
it is merely provided for backward compatibility
and it may be removed in future versions.

If the detection mechanism is to be used,
it is mandatory to correctly specify
the filename of the main file as the argument of |\childdocmain|:
%
\begin{center}
\begin{tabular}{l}
|\input{childdoc.def}|\\
|\childdocmain{|\textit{main}|}|\\
\end{tabular}
\end{center}
%
If |\jobname| does not match the argument \textit{main} of |\childdocmain|,
it is assumed that |\jobname| points to the child file to be compiled.
When using |\childdocmain| with the main file specified as argument,
it suffices to start a child file
with just |\input{|\textit{main}|}|
without loading of the package and using |\childdocof|.
If instead all processing is done
with the appropriate \textsf{childdoc} directives,
the argument of \textit{main} of |\childdocmain| can be empty.

An alternative version of the command line processing described
in \secref{sec:commandline} using the detection mechanism reads:
%
\begin{center}
|... -jobname "|\textit{target}|" "|[\textit{flags}]%
[|\def\jobname{|\textit{dest}|}|]|\input{|\textit{main}|}"|
\end{center}

%%%%%%%%%%%%%%%%%%%%%%%%%%%%%%%%%%%%%%%%%%%%%%%%%%%%%%%%%%%%%%%%%%%%%%%%%%%%%%%%
\subsection{Manual Code}
\label{sec:manual}

In case one cannot be certain whether the definitions file |childdoc.def|
is installed on the target \TeX{} distribution
and one prefers not to ship it,
it is conceivable to paste a few relevant commands into the sources.

To that end, drop all statements |\input{childdoc.def}|
and perform the replacements as outlined below.
Instead of |\childdocmain{|\textit{main}|}| add the following code
to the top of the main file:
%
\begin{center}
\begin{tabular}{l}
|\||ifdefined\childdocname\endinput\||fi\newif\ifchilddoc|\\
|\edef\childdocname{\scantokens\expandafter{\jobname\noexpand}}|\\
|\def\childdocmain{|\textit{main}|}\||ifx\childdocmain\childdocname\||else|\\
|\childdoctrue\includeonly{\childdocname}\let\jobname\childdocmain\||fi|\\
\end{tabular}
\end{center}
%
Instead of |\childdocof{|\textit{main}|}| just include the main file
at the top of each child file:
%
\begin{center}
|\input{|\textit{main}|}|
\end{center}
%
A simple redirection |\childdocforward{|\textit{dest}|}| is achieved by:
%
\begin{center}
|\def\jobname{|\textit{dest}|}\input{\jobname}|
\end{center}
%
The redirection with prefix
|\childdocforwardprefix[|\textit{prefix}|]{|\textit{dest}|}|
is accomplished by:
%
\begin{center}
\begin{tabular}{l}
|{\edef\jobname{\scantokens\expandafter{\jobname\noexpand}}|\\
|\def\redirectjob |\textit{prefix}|#1~~~{\gdef\jobname{|\textit{dest}|#1}}|\\
|\expandafter\redirectjob\jobname~~~}\input{\jobname}|
\end{tabular}
\end{center}

In an alternative approach,
child documents can be compiled by a specific command line
without additional code or specific definitions:
%
\begin{center}
|... -jobname "|\textit{target}|" "|[\textit{flags}]%
|\includeonly{|\textit{dest}|}\input{|\textit{main}|}"|
\end{center}
%

%%%%%%%%%%%%%%%%%%%%%%%%%%%%%%%%%%%%%%%%%%%%%%%%%%%%%%%%%%%%%%%%%%%%%%%%%%%%%%%%
%%%%%%%%%%%%%%%%%%%%%%%%%%%%%%%%%%%%%%%%%%%%%%%%%%%%%%%%%%%%%%%%%%%%%%%%%%%%%%%%
\section{Information}

%%%%%%%%%%%%%%%%%%%%%%%%%%%%%%%%%%%%%%%%%%%%%%%%%%%%%%%%%%%%%%%%%%%%%%%%%%%%%%%%
\subsection{Copyright}

Copyright \copyright{} 2017--2018 Niklas Beisert

This work may be distributed and/or modified under the
conditions of the \LaTeX{} Project Public License, either version 1.3
of this license or (at your option) any later version.
The latest version of this license is in
  \url{http://www.latex-project.org/lppl.txt}
and version 1.3 or later is part of all distributions of \LaTeX{}
version 2005/12/01 or later.

This work has the LPPL maintenance status `maintained'.

The Current Maintainer of this work is Niklas Beisert.

This work consists of the files |README.txt|, |childdoc.ins| and |childdoc.dtx|
as well as the derived files |childdoc.def|, |cdocsamp.tex|
with |cdocsch1.tex|, |cdocsch2.tex|, |cdocspt3.tex|, |cdocspt4.tex|,
|cdocsdrf.tex|, |cdocsfn1.tex|, |cdocsfn2.tex|
as well as |childdoc.pdf|.

%%%%%%%%%%%%%%%%%%%%%%%%%%%%%%%%%%%%%%%%%%%%%%%%%%%%%%%%%%%%%%%%%%%%%%%%%%%%%%%%
\subsection{Files and Installation}

The package consists of the files:
%
\begin{center}
\begin{tabular}{ll}
    |README.txt|   & readme file \\
    |childdoc.ins| & installation file \\
    |childdoc.dtx| & source file \\
    |childdoc.def| & definition file \\
    |cdocsamp.tex| & sample main file \\
    |cdocsch1.tex| & sample include file \\
    |cdocsch2.tex| & sample include file \\
    |cdocspt3.tex| & sample part file \\
    |cdocspt4.tex| & sample part file \\
    |cdocsdrf.tex| & sample redirection file \\
    |cdocsfn1.tex| & sample redirection file \\
    |cdocsfn2.tex| & sample redirection file \\
    |childdoc.pdf| & manual
\end{tabular}
\end{center}
%
The distribution consists of the files
|README.txt|, |childdoc.ins| and |childdoc.dtx|.
%
\begin{itemize}
\item
Run (pdf)\LaTeX{} on |childdoc.dtx|
to compile the manual |childdoc.pdf| (this file).
\item
Run \LaTeX{} on |childdoc.ins| to create the definitions file |childdoc.def|
and the sample |cdocsamp.tex| with include files
|cdocsch1.tex|, |cdocsch2.tex|, |cdocspt3.tex|, |cdocspt4.tex|,
|cdocsdrf.tex|, |cdocsfn1.tex|, |cdocsfn2.tex|.
Then copy the file |childdoc.def| to an appropriate directory of your \LaTeX{}
distribution, e.g.\ \textit{texmf-root}|/tex/latex/childdoc|.
\end{itemize}

%%%%%%%%%%%%%%%%%%%%%%%%%%%%%%%%%%%%%%%%%%%%%%%%%%%%%%%%%%%%%%%%%%%%%%%%%%%%%%%%
\subsection{Related CTAN Packages}

There are several other packages which offer a similar functionality:
%
\begin{itemize}
\item
The packages
\href{http://ctan.org/pkg/docmute}{\textsf{docmute}},
\href{http://ctan.org/pkg/includex}{\textsf{includex}} and
\href{http://ctan.org/pkg/standalone}{\textsf{standalone}}
provide commands to include only the document body of
a child file thus allowing both files to be compiled individually.
\item
The packages \href{http://ctan.org/pkg/subdocs}{\textsf{subdocs}}
and \href{http://ctan.org/pkg/subfiles}{\textsf{subfiles}}
provide structures in which the main and child documents can be
encapsulated and allowing them to be compiled individually.
The inclusion mechanism is different from the conventional |\include|.
\item
The package \href{http://ctan.org/pkg/combine}{\textsf{combine}}
is an elaborate solution to combine several documents into one.
\end{itemize}
%
See also the CTAN topic \href{http://ctan.org/topic/subdocs}{\textsf{subdocs}}
for further related packages.
The present package differs from the above solutions in that
a document structure constructed with the conventional |\include| mechanism
just needs two extra commands at the top of every file
such that all constituent files can be compiled individually.

%%%%%%%%%%%%%%%%%%%%%%%%%%%%%%%%%%%%%%%%%%%%%%%%%%%%%%%%%%%%%%%%%%%%%%%%%%%%%%%%
%\subsection{Feature Suggestions}
%
%The following is a list of features which may be useful for future
%versions of this package:
%%
%\begin{itemize}
%\item
%\ldots
%\end{itemize}

%%%%%%%%%%%%%%%%%%%%%%%%%%%%%%%%%%%%%%%%%%%%%%%%%%%%%%%%%%%%%%%%%%%%%%%%%%%%%%%%
\subsection{Revision History}

%%%%%%%%%%%%%%%%%%%%%%%%%%%%%%%%%%%%%%%%
\paragraph{v2.0:} 2018/12/30

\begin{itemize}
\item
immediate forward processing
\item
added |\childdocby| mechanism
\item
manual restructured
\end{itemize}

%%%%%%%%%%%%%%%%%%%%%%%%%%%%%%%%%%%%%%%%
\paragraph{v1.6:} 2018/01/17

\begin{itemize}
\item
application for development of include files
\item
corrections to manual
\end{itemize}

%%%%%%%%%%%%%%%%%%%%%%%%%%%%%%%%%%%%%%%%
\paragraph{v1.5:} 2017/05/21

\begin{itemize}
\item
more complete structuring introduced
\item
|\childdocof| introduced
\item
|\childdoc| renamed to |\childdocmain|
\item
|\childredirect| renamed to |\childdocforward| and |\childdocforwardprefix|
and functionality expanded
\end{itemize}

%%%%%%%%%%%%%%%%%%%%%%%%%%%%%%%%%%%%%%%%
\paragraph{v1.0:} 2017/04/27

\begin{itemize}
\item
manual and install package
\item
first version published on CTAN
\end{itemize}

%%%%%%%%%%%%%%%%%%%%%%%%%%%%%%%%%%%%%%%%
\paragraph{v0.6:} 2017/04/26

\begin{itemize}
\item
redirection mechanism added
\end{itemize}

%%%%%%%%%%%%%%%%%%%%%%%%%%%%%%%%%%%%%%%%
\paragraph{v0.5:} 2017/04/26

\begin{itemize}
\item
functionality in definition file
\end{itemize}


%%%%%%%%%%%%%%%%%%%%%%%%%%%%%%%%%%%%%%%%%%%%%%%%%%%%%%%%%%%%%%%%%%%%%%%%%%%%%%%%
%%%%%%%%%%%%%%%%%%%%%%%%%%%%%%%%%%%%%%%%%%%%%%%%%%%%%%%%%%%%%%%%%%%%%%%%%%%%%%%%
%%%%%%%%%%%%%%%%%%%%%%%%%%%%%%%%%%%%%%%%%%%%%%%%%%%%%%%%%%%%%%%%%%%%%%%%%%%%%%%%
\appendix

\settowidth\MacroIndent{\rmfamily\scriptsize 000\ }

 \DocInput{childdoc.dtx}

\end{document}
%</driver>
% \fi
%
% %%%%%%%%%%%%%%%%%%%%%%%%%%%%%%%%%%%%%%%%%%%%%%%%%%%%%%%%%%%%%%%%%%%%%%%%%%%%%%
% %%%%%%%%%%%%%%%%%%%%%%%%%%%%%%%%%%%%%%%%%%%%%%%%%%%%%%%%%%%%%%%%%%%%%%%%%%%%%%
% \section{Sample}
%\iffalse
%<*samplemain>
%\fi
%
% The following presents a sample document
% with two chapters, two parts, a title page,
% a compile flag as well as three forwarding files to set the flag.
% It consists of eight |.tex| files:
% \begin{center}
% \begin{tabular}{ll}
% |cdocsamp.tex|&main file\\
% |cdocsch1.tex|&include file for chapter 1\\
% |cdocsch2.tex|&include file for chapter 2\\
% |cdocspt3.tex|&include file for part 3\\
% |cdocspt4.tex|&include file for part 4\\
% |cdocsdrf.tex|&forwarding file for main file in draft mode\\
% |cdocsfi1.tex|&forwarding file for final version of chapter 1\\
% |cdocsfi2.tex|&forwarding file for final version of chapter 2\\
% \end{tabular}
% \end{center}
% Each of the eight files can be compiled directly by the \LaTeX{} compiler.
%
% %%%%%%%%%%%%%%%%%%%%%%%%%%%%%%%%%%%%%%
% \paragraph{Main File.}
%
% The main file is called |cdocsamp.tex|.
%
% Load the \textsf{childdoc} definitions and
% declare the filename for the main document:
%    \begin{macrocode}
\input{childdoc.def}
\childdocmain{}
%    \end{macrocode}

% Optional override for |\version| flag:
%    \begin{macrocode}
%%\ifchilddoc\else\providecommand{\version}{draft}\fi
%    \end{macrocode}

% Define the default values for the |\version| flag
% (|final| for the main file and |draft| for childs):
%    \begin{macrocode}
\ifchilddoc
\providecommand{\version}{draft}
\else
\providecommand{\version}{final}
\fi
%    \end{macrocode}

% Load the standard document class:
%    \begin{macrocode}
\documentclass[12pt]{article}
%    \end{macrocode}

% Start the document body:
%    \begin{macrocode}
\begin{document}
%    \end{macrocode}

% Declare a title page.
% Print title, part of document being processed and version flag:
%    \begin{macrocode}
\addtocounter{page}{-1}
\begin{center}
{\LARGE\bfseries{}childdoc example\par}
\vspace{1cm}
\ifchilddoc
\ifchilddocmanual part\else chapter\fi:
`\childdocname' of `\childdocjob'\par
\else
main document: `\childdocjob'\par
\fi
version: \version\par
\end{center}
\newpage
%    \end{macrocode}

% Manually include selected file,
% otherwise process as usual:
%    \begin{macrocode}
\ifchilddocmanual
\section*{part `\childdocname'}
\input{\childdocname}
\else
%    \end{macrocode}

% Include the two chapters:
%    \begin{macrocode}
\include{cdocsch1}
\include{cdocsch2}
%    \end{macrocode}

% Include the two parts unless only chapters should be displayed:
%    \begin{macrocode}
\ifchilddoc\else
\section{part three}
\input{cdocspt3}
\section{part four}
\input{cdocspt4}
\fi
%    \end{macrocode}

% Process as usual until here:
%    \begin{macrocode}
\fi
%    \end{macrocode}

% End of document body:
%    \begin{macrocode}
\end{document}
%    \end{macrocode}
%\iffalse
%</samplemain>
%\fi
%
% %%%%%%%%%%%%%%%%%%%%%%%%%%%%%%%%%%%%%%
% \paragraph{Chapter Include Files.}
%
% The include files are called |cdocsch1.tex| and |cdocsch2.tex|.
%
%\iffalse
%<*samplechap1|samplechap2>
%\fi

% Optional override for |\version| flag:
%    \begin{macrocode}
%%\providecommand{\version}{final}
%    \end{macrocode}

% Include the main document:
%    \begin{macrocode}
\input{childdoc.def}
\childdocof{cdocsamp}
%    \end{macrocode}

%\iffalse
%</samplechap1|samplechap2>
%\fi
%
%\iffalse
%<*samplechap1>
%\fi
% Some text for chapter 1:
%    \begin{macrocode}
\section{one}
some text in chapter one
%    \end{macrocode}

%\iffalse
%</samplechap1>
%\fi
% Some text for chapter 2:
%\iffalse
%<*samplechap2>
%\fi
%    \begin{macrocode}
\section{two}
more text in chapter two
%    \end{macrocode}

%\iffalse
%</samplechap2>
%\fi
%
% %%%%%%%%%%%%%%%%%%%%%%%%%%%%%%%%%%%%%%
% \paragraph{Part Include Files.}
%
% The include files are called |cdocspt3.tex| and |cdocspt4.tex|.
%
%\iffalse
%<*samplepart3|samplepart4>
%\fi

% Optional override for |\version| flag:
%    \begin{macrocode}
%%\providecommand{\version}{final}
%    \end{macrocode}

% Include the main document:
%    \begin{macrocode}
\input{childdoc.def}
\childdocby{cdocsamp}
%    \end{macrocode}

%\iffalse
%</samplepart3|samplepart4>
%\fi
%
%\iffalse
%<*samplepart3>
%\fi
% Some text for part 3:
%    \begin{macrocode}
some text in part three
%    \end{macrocode}

%\iffalse
%</samplepart3>
%\fi
% Some text for part 4:
%\iffalse
%<*samplepart4>
%\fi
%    \begin{macrocode}
more text in part four
%    \end{macrocode}

%\iffalse
%</samplepart4>
%\fi
%
% %%%%%%%%%%%%%%%%%%%%%%%%%%%%%%%%%%%%%%
% \paragraph{Forwarding for a Complete Draft.}
%
% The following forwarding file |cdocsdrf.tex|
% compiles the main document in draft mode:
%\iffalse
%<*sampledraft>
%\fi
%    \begin{macrocode}
\def\version{draft}
\input{childdoc.def}
\childdocforward{cdocsamp}
%    \end{macrocode}

%\iffalse
%</sampledraft>
%\fi
%
% %%%%%%%%%%%%%%%%%%%%%%%%%%%%%%%%%%%%%%
% \paragraph{Forwarding for Final Version of the Chapters.}
%
% The following forwarding files |cdocsfn1.tex| and |cdocsfn2.tex|
% (with identical content)
% compile the final versions of the child documents
% |cdocsch1.tex| and |cdocsch2.tex|, respectively:
%\iffalse
%<*samplefinal>
%\fi
%    \begin{macrocode}
\def\version{final}
\input{childdoc.def}
\childdocforwardprefix[cdocsamp]{cdocsfn}{cdocsch}
%    \end{macrocode}

%\iffalse
%</samplefinal>
%\fi
%
% %%%%%%%%%%%%%%%%%%%%%%%%%%%%%%%%%%%%%%
% \paragraph{Command Line Processing.}
%
% The following three command lines generate the output files
% |cdocscld|, |cdocscl1| and |cdocscl2|
% which should be identical to
% |cdocsdrf|, |cdocsch1| and |cdocsfn2|, respectively:
% \begin{center}
% \begin{tabular}{l}
% |latex -jobname cdocscld \|\\
% |  "\def\version{draft}\input{childdoc.def}\childdocforward{cdocsamp}"|\\
% |latex -jobname cdocscl1 \|\\
% |  "\input{childdoc.def}\childdocforward[cdocsamp]{cdocsch1}"|\\
% |latex -jobname cdocscl2 \|\\
% |  "\def\version{final}\input{childdoc.def}\childdocforward{cdocsch2}"|
% \end{tabular}
% \end{center}
% Note that the trailing backslash on each first line
% merely continues the input to the second line
% (for convenient cut ant paste).
% Furthermore, the command |latex| can be replaced by any
% of its alternative versions such as |pdflatex|.
%
% %%%%%%%%%%%%%%%%%%%%%%%%%%%%%%%%%%%%%%%%%%%%%%%%%%%%%%%%%%%%%%%%%%%%%%%%%%%%%%
% %%%%%%%%%%%%%%%%%%%%%%%%%%%%%%%%%%%%%%%%%%%%%%%%%%%%%%%%%%%%%%%%%%%%%%%%%%%%%%
% \section{Implementation}
%\iffalse
%<*package>
%\fi
%
% This section describes the definitions file |childdoc.def|.

% The definitions cannot be loaded using |\usepackage| or |\RequirePackage|
% which has a mechanism to prevent loading a style file more than once.
% When loading the definitions by means of |\input|
% multiple instances have to be prevented manually:
%\iffalse
%This code needs to be before the `\ProvidesFile' directive
%which is defined at the beginning of this file.
%Therefore it is also placed there and commented out here.
%</package>
%<*discard>
%\fi
%    \begin{macrocode}
\ifdefined\childdocmain\endinput\fi
%    \end{macrocode}
%\iffalse
%</discard>
%<*package>
%\fi
%
% \macro{\ifchilddoc}
% \macro{\ifchilddocmanual}
% The conditional |\ifchilddoc| tells whether a
% child (true) or main (false) document is being compiled.
% The conditional |\ifchilddocmanual| tells whether
% the |\includeonly| mechanism is used (false) or
% the selection of child files must be performed manually (true).
% The definitions initialise to false:
%    \begin{macrocode}
\newif\ifchilddoc
\newif\ifchilddocmanual
%    \end{macrocode}

% \macro{\childdocname}
% \macro{\childdocjob}
% The macro |\childdocname| stores the name of the main document
% to be compiled. The macro |\childdocjob| stores the name of
% the document on which the \LaTeX{} compiler was originally invoked.
% The content of |\jobname| cannot be compared
% to filenames specified in the source due to different catcodes.
% The following code rescans |\jobname|, stores the result
% in |\childdocname| and saves a copy in |\childdocjob|:
%    \begin{macrocode}
\edef\childdocname{\scantokens\expandafter{\jobname\noexpand}}
\let\childdocjob\childdocname
%    \end{macrocode}

% \macro{\childdocdisable}
% The macro |\childdocdisable| prevents the main file
% from being processed more than once.
% At this stage, the main document command |\childdocmain|
% is assumed to be called once again where it should do nothing.
% Any subsequent call to it should prevent
% a secondary processing of the main document
% It overwrites the forwarding commands
% |\childdocof| and |\childdocforward|
% with empty macros to prevent further inclusions of the main document:
%    \begin{macrocode}
\newcommand{\childdocdisable}
{
  \renewcommand{\childdocmain}[1]{\renewcommand{\childdocmain}[1]{\endinput}}
  \renewcommand{\childdocof}[1]{}
  \renewcommand{\childdocby}[2][]{}
  \renewcommand{\childdocforward}[2][]{}
  \renewcommand{\childdocdisable}{}
}
%    \end{macrocode}

% \macro{\childdocmain}
% The macro |\childdocmain| is to be called at the top of the main file
% with nothing or the main filename (without extension) as argument.
% First, it breaks loops.
% If the argument is not empty and does not match |\childdocname|
% (which is set by the first inclusion of |childdoc.def|),
% |\ifchilddoc| is set to true, |\includeonly| is applied to the child file
% and |\jobname| is set to the main file
% (for proper handling of |.aux| files):
%    \begin{macrocode}
\newcommand{\childdocmain}[1]
{
  \childdocdisable\childdocmain{}
  \if?#1?\else
    \begingroup
      \def\childdoctmp{#1}
      \ifx\childdoctmp\childdocname
        \def\childdoctmp{}
      \else
        \def\childdoctmp
        {
          \childdoctrue
          \includeonly{\childdocname}
          \def\childdocjob{#1}
          \def\jobname{#1}
        }
      \fi
      \expandafter
    \endgroup
    \childdoctmp
  \fi
}
%    \end{macrocode}

% \macro{\childdocof}
% The command |\childdocof| redirects
% compilation to the main file |#1|.
%    \begin{macrocode}
\newcommand{\childdocof}[1]
{
  \childdocdisable
  \childdoctrue
  \includeonly{\childdocname}
  \def\jobname{#1}
  \def\childdocjob{#1}
  \input{#1}
}
%    \end{macrocode}

% \macro{\childdocby}
% The command |\childdocby| ....
%    \begin{macrocode}
\newcommand{\childdocby}[2][]
{
  \childdocdisable
  \childdoctrue
  \childdocmanualtrue
  \if?#1?\else
    \def\jobname{#2}
  \fi
  \def\childdocjob{#2}
  \input{#2}
  \endinput
}
%    \end{macrocode}

% \macro{\childdocforward}
% The command |\childdocforward| redirects
% compilation to the main file or
% (if the optional argument is given) a child file.
% Parameters are set as if the main file
% or a child file starting with |\childdocof| was compiled.
% Then compilation is handed over to the main file:
%    \begin{macrocode}
\newcommand{\childdocforward}[2][]
{
  \begingroup
    \if?#1?
      \def\childdoctmp
      {
        \def\childdocname{#2}
        \def\childdocjob{#2}
        \def\jobname{#2}
        \input{#2}
        \endinput
      }
    \else
      \def\childdoctmp
      {
        \childdocdisable
        \def\childdocname{#2}
        \childdoctrue
        \includeonly{#2}
        \def\childdocjob{#1}
        \def\jobname{#1}
        \input{#1}
        \endinput
      }
    \fi
    \expandafter
  \endgroup
  \childdoctmp
}
%    \end{macrocode}

% \macro{\childdocforwardprefix}
% The command |\childdocforwardprefix| redirects
% compilation to the main or a child file by means of a pattern.
% The prefix |#1| in the current filename is replaced by |#2|
% and the suffix of the current filename is kept
% (it is assumed that the filename does not contain the substring `|~~~|'
% which is used as a delimiter).
% Compilation is handed over to the new file by |\childdocforward|:
%    \begin{macrocode}
\newcommand{\childdocforwardprefix}[3][]
{
  \begingroup
    \def\childdocextract #2##1~~~{\def\childdoctmp{\childdocforward[#1]{#3##1}}}
    \expandafter\childdocextract\childdocname~~~
    \expandafter
  \endgroup
  \childdoctmp
}
%    \end{macrocode}

% \macro{\childdoc}
% The deprecated macro |\childdoc| is a legacy version of |\childdocmain|:
%    \begin{macrocode}
\newcommand{\childdoc}{\childdocmain}
%    \end{macrocode}

% \macro{\childdocredirect}
% The deprecated macro |\childdocredirect| is a legacy version
% of |\childdocforward| and |\childdocforwardprefix|:
%    \begin{macrocode}
\newcommand{\childdocredirect}[2][]
{
  \begingroup
    \if?#1?
      \def\childdoctmp{\childdocforward{#2}}
    \else
      \def\childdoctmp{\childdocforwardprefix{#1}{#2}}
    \fi
    \expandafter
  \endgroup
  \childdoctmp
}
%    \end{macrocode}

%\iffalse
%</package>
%\fi
%
\endinput
|\\
|\childdocforwardprefix[|\textit{main}|]{|\textit{prefix}|}{|\textit{dest}|}|
\end{tabular}
\end{center}
%
the destination file is determined by a pattern
depending on the current file:
To make this work, the current file must be called
`{\textit{prefix}\hspace{0.2em}\textit{suffix}}'
with \textit{prefix} matching precisely the argument.
Processing is then passed on to the file
`{\textit{dest}\hspace{0.2em}\textit{suffix}}'.
Surely, the same effect is achieved by
directly specifying the
argument `{\textit{dest}\hspace{0.2em}\textit{suffix}}'
in the first form.
However, that requires to set up a different file
for each child. With the alternative form of the command
all these files can have exactly the same content
which simplifies setting them up and maintaining them.

For example, the following file |draft.tex|
with a compilation flag |\version| as described in \secref{sec:flags}
compiles the main document as a draft:
%
\begin{center}
\begin{tabular}{l}
|\def\version{draft}|\\
|% \iffalse
%
% childdoc.dtx Copyright (C) 2017-2018 Niklas Beisert
%
% This work may be distributed and/or modified under the
% conditions of the LaTeX Project Public License, either version 1.3
% of this license or (at your option) any later version.
% The latest version of this license is in
%   http://www.latex-project.org/lppl.txt
% and version 1.3 or later is part of all distributions of LaTeX
% version 2005/12/01 or later.
%
% This work has the LPPL maintenance status `maintained'.
%
% The Current Maintainer of this work is Niklas Beisert.
%
% This work consists of the files childdoc.dtx and childdoc.ins
% and the derived files childdoc.def and cdocsamp.tex with
% cdocsch1.tex, cdocsch2.tex, cdocsdrf.tex, cdocsfn1.tex, cdocsfn2.tex.
%
%<package>\ifdefined\childdocmain\endinput\fi
%<package>\ProvidesFile{childdoc.def}[2018/12/30 v2.0 child document driver]
%<samplemain>\ProvidesFile{cdocsamp.tex}[2018/12/30 v2.0 sample for childdoc]
%<*driver>
%\ProvidesFile{childdoc.drv}[2018/12/30 v2.0 childdoc reference manual file]
\PassOptionsToClass{10pt,a4paper}{article}
\documentclass{ltxdoc}

\usepackage[margin=35mm]{geometry}
\usepackage{hyperref}
\usepackage{hyperxmp}
\usepackage[usenames]{color}

\hypersetup{colorlinks=true}
\hypersetup{pdfstartview=FitH}
\hypersetup{pdfpagemode=UseNone}
\hypersetup{pdfsource={}}
\hypersetup{pdflang={en-UK}}
\hypersetup{pdfcopyright={Copyright 2017-2018 Niklas Beisert.
  This work may be distributed and/or modified under the
  conditions of the LaTeX Project Public License, either version 1.3
  of this license or (at your option) any later version.}}
\hypersetup{pdflicenseurl={http://www.latex-project.org/lppl.txt}}
\hypersetup{pdfcontactaddress={ETH Zurich, ITP, HIT K,
  Wolfgang-Pauli-Strasse 27}}
\hypersetup{pdfcontactpostcode={8093}}
\hypersetup{pdfcontactcity={Zurich}}
\hypersetup{pdfcontactcountry={Switzerland}}
\hypersetup{pdfcontactemail={nbeisert@itp.phys.ethz.ch}}
\hypersetup{pdfcontacturl={http://people.phys.ethz.ch/\xmptilde nbeisert/}}

\newcommand{\secref}[1]{\hyperref[#1]{section \ref*{#1}}}

\parskip1ex
\parindent0pt
\let\olditemize\itemize
\def\itemize{\olditemize\parskip0pt}

\begin{document}

\title{The \textsf{childdoc} Package}
\hypersetup{pdftitle={The childdoc Package}}
\author{Niklas Beisert\\[2ex]
  Institut f\"ur Theoretische Physik\\
  Eidgen\"ossische Technische Hochschule Z\"urich\\
  Wolfgang-Pauli-Strasse 27, 8093 Z\"urich, Switzerland\\[1ex]
  \href{mailto:nbeisert@itp.phys.ethz.ch}
  {\texttt{nbeisert@itp.phys.ethz.ch}}}
\hypersetup{pdfauthor={Niklas Beisert}}
\hypersetup{pdfsubject={Manual for the LaTeX2e Package childdoc}}
\date{30 December 2018, \textsf{v2.0}}
\maketitle

\begin{abstract}\noindent
\textsf{childdoc} is a \LaTeXe{} package
that enables the direct compilation
of document sections included by |\include|
to individual files.
\end{abstract}

\begingroup
\parskip0ex
\tableofcontents
\endgroup

%%%%%%%%%%%%%%%%%%%%%%%%%%%%%%%%%%%%%%%%%%%%%%%%%%%%%%%%%%%%%%%%%%%%%%%%%%%%%%%%
%%%%%%%%%%%%%%%%%%%%%%%%%%%%%%%%%%%%%%%%%%%%%%%%%%%%%%%%%%%%%%%%%%%%%%%%%%%%%%%%
\section{Introduction}

\LaTeX{} provides a mechanism to structure a large document (such as a book)
into a main file and several child files (containing the chapters)
using the |\include| command.
This mechanism is beneficial for documents
which span hundreds of pages in order to
make the source file(s) more manageable.
Moreover, compilation can be restricted to
selected child files by means of the |\includeonly| command.
The latter feature can be used to reduce the compilation time while editing
(this was significantly more useful in the earlier days of \LaTeX{})
or to generate a smaller document which is easier to navigate.
Another application of |\includeonly| is to generate
documents consisting of selected parts of the complete document.

However, there are a few drawbacks of the plain |\include| mechanism:
\begin{itemize}
\item
The child files cannot be compiled on their own,
they can only be compiled via the main file.
A naive editing environment
(such as a text editor with an option
to have the current file processed by \LaTeX)
may require one to switch to the main file before compiling;
attempting to compile the child file produces errors.
\item
The main file must be modified (each time)
to adjust the |\includeonly| command
to the present needs. This easily leaves the main file in a messy state.
\item
The generated document will always carry the filename
of the main document. This is inconvenient if
several child files are to be compiled and
to be kept for distribution.
\end{itemize}

The present package provides a simple interface
to make child files individually compilable by \LaTeX{}.
Compiling a child file then has the same effect as compiling
the main file with an |\includeonly| command
to select the appropriate child.
Moreover the generated document will carry the name of the child
rather than the main file.
This resolves all three above issues.

This feature is meant to make the editing of books,
thesis documents and lecture notes somewhat more convenient.
However, the package can also be used efficiently for
composing a series of documents (such as exercise sheets)
which are typically distributed individually.
It then assists the author in generating the individual documents
(potentially in different versions)
as well as a document containing the collected series.
Another application is in developing style files
or other kinds of included material
where compilation of the style file could redirect
to a sample or test file.

%%%%%%%%%%%%%%%%%%%%%%%%%%%%%%%%%%%%%%%%%%%%%%%%%%%%%%%%%%%%%%%%%%%%%%%%%%%%%%%%
%%%%%%%%%%%%%%%%%%%%%%%%%%%%%%%%%%%%%%%%%%%%%%%%%%%%%%%%%%%%%%%%%%%%%%%%%%%%%%%%
\section{Usage}

First of all, the package \textsf{childdoc} is \emph{not} a standard
\LaTeXe{} |.sty| style file! Therefore it needs to be invoked in
a non-standard way.

%%%%%%%%%%%%%%%%%%%%%%%%%%%%%%%%%%%%%%%%%%%%%%%%%%%%%%%%%%%%%%%%%%%%%%%%%%%%%%%%
\subsection{Included Files}
\label{sec:include}

%%%%%%%%%%%%%%%%%%%%%%%%%%%%%%%%%%%%%%%%
\DescribeMacro{\childdocmain}
To use the package, add the commands
\begin{center}
\begin{tabular}{l}
|\input{childdoc.def}|\\
|\childdocmain{}|\\
\end{tabular}
\end{center}
at the very top of the main \LaTeX{} file,
in particular \emph{before} the |\documentclass| statement!
The argument of |\childdocmain| should be left empty
(but it must be present).

%%%%%%%%%%%%%%%%%%%%%%%%%%%%%%%%%%%%%%%%
\DescribeMacro{\childdocof}
Furthermore, add the commands
\begin{center}
\begin{tabular}{l}
|\input{childdoc.def}|\\
|\childdocof{|\textit{main}|}|\\
\end{tabular}
\end{center}
at the top of every child file \textit{child}
which is included by |\include{|\textit{child}|}|
from within the main file
(or at least for those files to be compiled individually).
The argument \textit{main} must be the filename of the main file.

There are a couple of
considerations in setting up the main and child documents:

%%%%%%%%%%%%%%%%%%%%%%%%%%%%%%%%%%%%%%%%
\paragraph{Restrictions.}

Please note the following restrictions:
\begin{itemize}
\item
|\childdocmain| must be called with one argument \textit{main}
to ensure compatibility with earlier version of the package.
It must either be empty (|\childdocmain{}|)
or precisely match the filename of the main file in which it is specified.
See \secref{sec:detection} for further information.
\item
The filename \textit{main} must be specified without the |.tex| extension.
\item
The filename \textit{main} is case sensitive
(even in case-insensitive file systems)
due to internal string comparison.
\item
The argument \textit{main} should be fully expanded, it cannot be a macro.
\item
Subdirectories and special characters should be avoided in filenames.
\item
The command |\childdocmain{|\textit{main}|}| must be followed by a whitespace.
It should not be followed immediately by another command
or by a comment mark `|%|'.
This is because the \TeX{} parser reads the token immediately following
the argument of |\childdocmain| and puts it
at the beginning of every child section;
however, a white\-space is ignored.
\end{itemize}

%%%%%%%%%%%%%%%%%%%%%%%%%%%%%%%%%%%%%%%%
\paragraph{Content of Main File.}

It is advisable to place all content in the child files included by |\include|.
Any output contained in the main file will appear in all child documents
unless suppressed manually;
it cannot be suppressed automatically by the |\includeonly| directive
and thus should normally be avoided.
A method to include some content in the main file
by means of conditional processing is described in \secref{sec:conditional}.

%%%%%%%%%%%%%%%%%%%%%%%%%%%%%%%%%%%%%%%%
\paragraph{Page Numbering.}

When only a part of the document is compiled,
the appropriate numbering of pages
(as well as other status parameters)
is determined from the |.aux| files.
The latter contain information from previous passes.
However this information needs to propagate through
all intermediate child documents.
Therefore the page numbering in child documents may well
be inconsistent until the complete document is compiled at least once.

A useful (if unconventional) way to always ensure a consistent
page numbering is to restart the numbering in each child document
and denote the pages by `\textit{child}|.|\textit{page}'
where \textit{child} represents the chapter/section number of the child file.
This can be achieved by the command
|\numberwithin{page}{|\textit{child}|}|
of the \textsf{amsmath} package
where \textit{child} can be |chapter| or |section|
depending on the chosen structuring.
Alternatively, one can modify the macro |\thepage| appropriately
and reset the counter |page| at the start of each child file.

%%%%%%%%%%%%%%%%%%%%%%%%%%%%%%%%%%%%%%%%%%%%%%%%%%%%%%%%%%%%%%%%%%%%%%%%%%%%%%%%
\subsection{Conditional Processing}
\label{sec:conditional}

The package provides a mechanism to compile different versions
of a document. To customise the versions further some conditional processing
can come in handy to distinguish which version is being compiled.
The package provides two macros to describe the compilation context:

%%%%%%%%%%%%%%%%%%%%%%%%%%%%%%%%%%%%%%%%
\DescribeMacro{\ifchilddoc}
The conditional |\ifchilddoc| distinguishes between the compilation of
child documents and the main document:
%
\begin{center}
|\ifchilddoc |\textit{child-code}| |[|\||else |\textit{main-code}]| \||fi|
\end{center}

%%%%%%%%%%%%%%%%%%%%%%%%%%%%%%%%%%%%%%%%
\DescribeMacro{\childdocname}
\DescribeMacro{\childdocjob}
The macro |\childdocname| contains the filename (without extension)
of the main or child file being processed.
Note that |\childdocjob| will always contain the name of the main file.

%%%%%%%%%%%%%%%%%%%%%%%%%%%%%%%%%%%%%%%%
\paragraph{Title Page.}

Conditional processing can be used to include a title or banner page
in the main document when proper precautions are taken.
Importantly, the code in the main file should ensure that the page counter
(as well as other status parameters which are stored in the |.aux| files)
takes the same value after the conditional processing.
Otherwise the page numbers may take divergent values
depending on which part is compiled.

For example, a title page could be declared by:
%
\begin{center}
\begin{tabular}{l}
|\ifchilddoc\||else|\\
|\addtocounter{page}{-1}|\\
\textit{code for title page}\\
|\newpage|\\
|\||fi|
\end{tabular}
\end{center}
%
A banner page for the child documents can be generated by:
%
\begin{center}
\begin{tabular}{l}
|\ifchilddoc|\\
|\addtocounter{page}{-1}|\\
\textit{code for banner page}\\
|\newpage|\\
|\||fi|
\end{tabular}
\end{center}
%
Here one could write a message such as:
\begin{center}
|This is the part \childdocname{} of \childdocjob{}.|
\end{center}

%%%%%%%%%%%%%%%%%%%%%%%%%%%%%%%%%%%%%%%%%%%%%%%%%%%%%%%%%%%%%%%%%%%%%%%%%%%%%%%%
\subsection{Flags}
\label{sec:flags}

The package makes it easy to generate different versions
of the main or child documents.
To this end compilation flags can be defined
and assigned different default values.
They will be particularly useful in conjunction
with the forwarding mechanism described in \secref{sec:forward}.

For example, it may be useful to have a flag |\version|
which can be set to |draft| or |final|.
The document source will contain some conditional code
depending on the value of |\version|.
Suppose further, the flag should default to |final| for the main file
and to |draft| for child files
which is a natural assignment for editing the document.
This is achieved by placing the following code
in the preamble of the main document
(below the |\childdocmain| directive):
%
\begin{center}
\begin{tabular}{l}
|\ifchilddoc|\\
|\providecommand{\version}{draft}|\\
|\||else|\\
|\providecommand{\version}{final}|\\
|\||fi|
\end{tabular}
\end{center}
%
The definition by |\providecommand| makes sure
that previous definitions are not overwritten.
Further statements |\providecommand{\version}{...}|
can thus be added before the above code to override it.

For the main file, one might add a line
(between |\childdocmain| and the above block)
%
\begin{center}
|%\ifchilddoc\||else\providecommand{\version}{draft}\||fi|
\end{center}
%
which can be uncommented to produce a draft version.
Likewise one can add a line to the very top of a child file
(above the |\childdocof{|\textit{main}|}| directive)
%
\begin{center}
|%\providecommand{\version}{final}|
\end{center}
%
which can be uncommented to produce the final version of this child document.

%%%%%%%%%%%%%%%%%%%%%%%%%%%%%%%%%%%%%%%%%%%%%%%%%%%%%%%%%%%%%%%%%%%%%%%%%%%%%%%%
\subsection{Forwarding}
\label{sec:forward}

Different versions of the main or child documents
using compilation flags as described in \secref{sec:flags}
can be (permanently) stored in different files
for convenient compilation, viewing and distribution.
To this end, the package defines a command
to pass on compilation to a different file:

%%%%%%%%%%%%%%%%%%%%%%%%%%%%%%%%%%%%%%%%
\DescribeMacro{\childdocforward}
The command |\childdocforward| redirects processing to
another source file:
%
\begin{center}
\begin{tabular}{l}
|\input{childdoc.def}|\\
|\childdocforward[|\textit{main}|]{|\textit{dest}|}|\\
\end{tabular}
\end{center}
%
The argument \textit{dest} is the destination file
(without extension).
It should be the main file or one of the child files.
Note that further \textsf{childdoc} directives
such as |\childdocof| and |\childdocforward|
in the indicated file will be processed in this form.
The optional argument \textit{main}
passes on directly to the main file \textit{main}
while pretending to compile the child \textit{dest}.
This form behaves as if \textit{dest}
issues |\childdocof{|\textit{main}|}| right away,
and no further \textsf{childdoc} directives will be processed.

%%%%%%%%%%%%%%%%%%%%%%%%%%%%%%%%%%%%%%%%
\DescribeMacro{\...prefix}
In the alternative form |\childdocforwardprefix|,
%
\begin{center}
\begin{tabular}{l}
|\input{childdoc.def}|\\
|\childdocforwardprefix[|\textit{main}|]{|\textit{prefix}|}{|\textit{dest}|}|
\end{tabular}
\end{center}
%
the destination file is determined by a pattern
depending on the current file:
To make this work, the current file must be called
`{\textit{prefix}\hspace{0.2em}\textit{suffix}}'
with \textit{prefix} matching precisely the argument.
Processing is then passed on to the file
`{\textit{dest}\hspace{0.2em}\textit{suffix}}'.
Surely, the same effect is achieved by
directly specifying the
argument `{\textit{dest}\hspace{0.2em}\textit{suffix}}'
in the first form.
However, that requires to set up a different file
for each child. With the alternative form of the command
all these files can have exactly the same content
which simplifies setting them up and maintaining them.

For example, the following file |draft.tex|
with a compilation flag |\version| as described in \secref{sec:flags}
compiles the main document as a draft:
%
\begin{center}
\begin{tabular}{l}
|\def\version{draft}|\\
|\input{childdoc.def}|\\
|\childdocforward{|\textit{main}|}|
\end{tabular}
\end{center}
%
Likewise, the following files |final|\textit{nn}|.tex|
compile the final version of the child document
|child|\textit{nn}|.tex|:
%
\begin{center}
\begin{tabular}{l}
|\def\version{final}|\\
|\input{childdoc.def}|\\
|\childdocforwardprefix{final}{child}|
\end{tabular}
\end{center}
%

Note that when several versions of a main file and/or of each child file
are to be generated, it may be convenient to set up a |Makefile| or
shell script to automatise the process.

%%%%%%%%%%%%%%%%%%%%%%%%%%%%%%%%%%%%%%%%%%%%%%%%%%%%%%%%%%%%%%%%%%%%%%%%%%%%%%%%
\subsection{Command Line Processing}
\label{sec:commandline}

The effect of redirection files can also be achieved by invoking
the \LaTeX{} compiler with a more elaborate command line.
Most conveniently this should be done as part
of a shell script or a |Makefile|.

When using \textsf{childdoc} in the main file, the following
command lines effectively perform a redirection
(note that depending on the shell being used,
backslashes may have to be doubled: `|\|' $\to$ `|\\|'):
%
\begin{center}
|... -jobname "|\textit{target}|" |\\|"|[\textit{flags}]%
|\input{childdoc.def}\childdocforward[|\textit{main}|]{|\textit{dest}|}"|
\end{center}
%
Here \textit{target} is the name of the output file,
\textit{main} is the name of the main file
and \textit{dest} is the name of the main or child file to be processed
(all filenames without extensions).
The optional argument \textit{main} can be omitted
if \textit{main} matches \textit{dest}.
Optionally, compilation \textit{flags} can be defined via |\def| commands.
This command line makes the \TeX{} engine believe
it is compiling the file \textit{target}
whose content is specified as the latter parameter.
The provided code then forwards the processing to
\textit{main} or \textit{dest} as described in \secref{sec:forward}.

%%%%%%%%%%%%%%%%%%%%%%%%%%%%%%%%%%%%%%%%%%%%%%%%%%%%%%%%%%%%%%%%%%%%%%%%%%%%%%%%
\subsection{Include by Input}
\label{sec:input}

Including child documents by |\include| has some restrictions by design.
Most notably, the content of a child document always occupies
its own set of pages; pages cannot be shared between child documents.
Usually, this behaviour makes perfect sense
because each child document contain an essential part of the document.
However, in some situations it may be desirable to compose
a document from a collection of parts
without having mandatory page breaks between then.
For this case, the package
provides a mechanism to include parts
by |\input| which can also be processed individually.
However, by construction this mechanism
requires manual handling of the content to be output.

%%%%%%%%%%%%%%%%%%%%%%%%%%%%%%%%%%%%%%%%
\DescribeMacro{\ifchilddocmanual}
The main file should be prepared as usual, see \secref{sec:include}.
However, the document body must make a distinction
between processing of an individual part and of the main document, e.g.:
%
\begin{center}
\begin{tabular}{l}
|\ifchilddocmanual|\\
|\input{\childdocname}|\\
|\||else|\\
\textit{document body with }|\input{|\textit{part}|}|\\
|\||fi|
\end{tabular}
\end{center}
%
The conditional |\ifchilddocmanual| is true whenever
a part to be included by |\input| is being compiled,
and the name of the part is stored in |\childdocname|.

%%%%%%%%%%%%%%%%%%%%%%%%%%%%%%%%%%%%%%%%
\DescribeMacro{\childdocby}
Each part to be included by |\input| should start with:
%
\begin{center}
\begin{tabular}{l}
|\input{childdoc.def}|\\
|\childdocby{|\textit{main}|}|\\
\end{tabular}
\end{center}
%
The directive |\childdocby| is similar to |\childdocof|
described in \secref{sec:include},
but the subsequent selection of content must be done manually.
To that end, both |\ifchilddoc| and |\ifchilddocmanual|
will be true upon processing of a part,
and the name of the part is stored in |\childdocname|.
Note that |\jobname| will be set to the filename of the current part
so that each part receives an individual |.aux| file
that does not interfere with the |.aux| file(s) of the main document.
This behaviour can be altered by the alternative form
|\childdocby[*]{|\textit{main}|}| (with a non-empty optional argument)
which uses the |.aux| file of the main document
by setting |\jobname| to \textit{main}.

%%%%%%%%%%%%%%%%%%%%%%%%%%%%%%%%%%%%%%%%%%%%%%%%%%%%%%%%%%%%%%%%%%%%%%%%%%%%%%%%
\subsection{Driver Development}
\label{sec:driver}

The \textsf{childdoc} mechanism can also be use for the development
of definition files such as \LaTeX{} styles or classes.
This case differs from the above setup with multiple parts
included by |\include| in that no |\includeonly| should be invoked.
This can be achieved by starting the include file
(before |\ProvidesPackage|) with:
%
\begin{center}
\begin{tabular}{l}
|\input{childdoc.def}|\\
|\childdocforward{|\textit{main}|}|\\
\end{tabular}
\end{center}
%
or alternatively with:
%
\begin{center}
\begin{tabular}{l}
|\input{childdoc.def}|\\
|\childdocby{|\textit{main}|}|\\
\end{tabular}
\end{center}
%
Both forms have slightly different effects as described above.
The main file is prepared as usual, see \secref{sec:include}.

%%%%%%%%%%%%%%%%%%%%%%%%%%%%%%%%%%%%%%%%%%%%%%%%%%%%%%%%%%%%%%%%%%%%%%%%%%%%%%%%
\subsection{Legacy Detection}
\label{sec:detection}

The directive |\childdocmain| in the main file can detect
whether the complete document or merely a child is to be compiled
even without using the directive |\childdocof|.
This method is deprecated because it is less robust
and there is no compelling reason to use it;
it is merely provided for backward compatibility
and it may be removed in future versions.

If the detection mechanism is to be used,
it is mandatory to correctly specify
the filename of the main file as the argument of |\childdocmain|:
%
\begin{center}
\begin{tabular}{l}
|\input{childdoc.def}|\\
|\childdocmain{|\textit{main}|}|\\
\end{tabular}
\end{center}
%
If |\jobname| does not match the argument \textit{main} of |\childdocmain|,
it is assumed that |\jobname| points to the child file to be compiled.
When using |\childdocmain| with the main file specified as argument,
it suffices to start a child file
with just |\input{|\textit{main}|}|
without loading of the package and using |\childdocof|.
If instead all processing is done
with the appropriate \textsf{childdoc} directives,
the argument of \textit{main} of |\childdocmain| can be empty.

An alternative version of the command line processing described
in \secref{sec:commandline} using the detection mechanism reads:
%
\begin{center}
|... -jobname "|\textit{target}|" "|[\textit{flags}]%
[|\def\jobname{|\textit{dest}|}|]|\input{|\textit{main}|}"|
\end{center}

%%%%%%%%%%%%%%%%%%%%%%%%%%%%%%%%%%%%%%%%%%%%%%%%%%%%%%%%%%%%%%%%%%%%%%%%%%%%%%%%
\subsection{Manual Code}
\label{sec:manual}

In case one cannot be certain whether the definitions file |childdoc.def|
is installed on the target \TeX{} distribution
and one prefers not to ship it,
it is conceivable to paste a few relevant commands into the sources.

To that end, drop all statements |\input{childdoc.def}|
and perform the replacements as outlined below.
Instead of |\childdocmain{|\textit{main}|}| add the following code
to the top of the main file:
%
\begin{center}
\begin{tabular}{l}
|\||ifdefined\childdocname\endinput\||fi\newif\ifchilddoc|\\
|\edef\childdocname{\scantokens\expandafter{\jobname\noexpand}}|\\
|\def\childdocmain{|\textit{main}|}\||ifx\childdocmain\childdocname\||else|\\
|\childdoctrue\includeonly{\childdocname}\let\jobname\childdocmain\||fi|\\
\end{tabular}
\end{center}
%
Instead of |\childdocof{|\textit{main}|}| just include the main file
at the top of each child file:
%
\begin{center}
|\input{|\textit{main}|}|
\end{center}
%
A simple redirection |\childdocforward{|\textit{dest}|}| is achieved by:
%
\begin{center}
|\def\jobname{|\textit{dest}|}\input{\jobname}|
\end{center}
%
The redirection with prefix
|\childdocforwardprefix[|\textit{prefix}|]{|\textit{dest}|}|
is accomplished by:
%
\begin{center}
\begin{tabular}{l}
|{\edef\jobname{\scantokens\expandafter{\jobname\noexpand}}|\\
|\def\redirectjob |\textit{prefix}|#1~~~{\gdef\jobname{|\textit{dest}|#1}}|\\
|\expandafter\redirectjob\jobname~~~}\input{\jobname}|
\end{tabular}
\end{center}

In an alternative approach,
child documents can be compiled by a specific command line
without additional code or specific definitions:
%
\begin{center}
|... -jobname "|\textit{target}|" "|[\textit{flags}]%
|\includeonly{|\textit{dest}|}\input{|\textit{main}|}"|
\end{center}
%

%%%%%%%%%%%%%%%%%%%%%%%%%%%%%%%%%%%%%%%%%%%%%%%%%%%%%%%%%%%%%%%%%%%%%%%%%%%%%%%%
%%%%%%%%%%%%%%%%%%%%%%%%%%%%%%%%%%%%%%%%%%%%%%%%%%%%%%%%%%%%%%%%%%%%%%%%%%%%%%%%
\section{Information}

%%%%%%%%%%%%%%%%%%%%%%%%%%%%%%%%%%%%%%%%%%%%%%%%%%%%%%%%%%%%%%%%%%%%%%%%%%%%%%%%
\subsection{Copyright}

Copyright \copyright{} 2017--2018 Niklas Beisert

This work may be distributed and/or modified under the
conditions of the \LaTeX{} Project Public License, either version 1.3
of this license or (at your option) any later version.
The latest version of this license is in
  \url{http://www.latex-project.org/lppl.txt}
and version 1.3 or later is part of all distributions of \LaTeX{}
version 2005/12/01 or later.

This work has the LPPL maintenance status `maintained'.

The Current Maintainer of this work is Niklas Beisert.

This work consists of the files |README.txt|, |childdoc.ins| and |childdoc.dtx|
as well as the derived files |childdoc.def|, |cdocsamp.tex|
with |cdocsch1.tex|, |cdocsch2.tex|, |cdocspt3.tex|, |cdocspt4.tex|,
|cdocsdrf.tex|, |cdocsfn1.tex|, |cdocsfn2.tex|
as well as |childdoc.pdf|.

%%%%%%%%%%%%%%%%%%%%%%%%%%%%%%%%%%%%%%%%%%%%%%%%%%%%%%%%%%%%%%%%%%%%%%%%%%%%%%%%
\subsection{Files and Installation}

The package consists of the files:
%
\begin{center}
\begin{tabular}{ll}
    |README.txt|   & readme file \\
    |childdoc.ins| & installation file \\
    |childdoc.dtx| & source file \\
    |childdoc.def| & definition file \\
    |cdocsamp.tex| & sample main file \\
    |cdocsch1.tex| & sample include file \\
    |cdocsch2.tex| & sample include file \\
    |cdocspt3.tex| & sample part file \\
    |cdocspt4.tex| & sample part file \\
    |cdocsdrf.tex| & sample redirection file \\
    |cdocsfn1.tex| & sample redirection file \\
    |cdocsfn2.tex| & sample redirection file \\
    |childdoc.pdf| & manual
\end{tabular}
\end{center}
%
The distribution consists of the files
|README.txt|, |childdoc.ins| and |childdoc.dtx|.
%
\begin{itemize}
\item
Run (pdf)\LaTeX{} on |childdoc.dtx|
to compile the manual |childdoc.pdf| (this file).
\item
Run \LaTeX{} on |childdoc.ins| to create the definitions file |childdoc.def|
and the sample |cdocsamp.tex| with include files
|cdocsch1.tex|, |cdocsch2.tex|, |cdocspt3.tex|, |cdocspt4.tex|,
|cdocsdrf.tex|, |cdocsfn1.tex|, |cdocsfn2.tex|.
Then copy the file |childdoc.def| to an appropriate directory of your \LaTeX{}
distribution, e.g.\ \textit{texmf-root}|/tex/latex/childdoc|.
\end{itemize}

%%%%%%%%%%%%%%%%%%%%%%%%%%%%%%%%%%%%%%%%%%%%%%%%%%%%%%%%%%%%%%%%%%%%%%%%%%%%%%%%
\subsection{Related CTAN Packages}

There are several other packages which offer a similar functionality:
%
\begin{itemize}
\item
The packages
\href{http://ctan.org/pkg/docmute}{\textsf{docmute}},
\href{http://ctan.org/pkg/includex}{\textsf{includex}} and
\href{http://ctan.org/pkg/standalone}{\textsf{standalone}}
provide commands to include only the document body of
a child file thus allowing both files to be compiled individually.
\item
The packages \href{http://ctan.org/pkg/subdocs}{\textsf{subdocs}}
and \href{http://ctan.org/pkg/subfiles}{\textsf{subfiles}}
provide structures in which the main and child documents can be
encapsulated and allowing them to be compiled individually.
The inclusion mechanism is different from the conventional |\include|.
\item
The package \href{http://ctan.org/pkg/combine}{\textsf{combine}}
is an elaborate solution to combine several documents into one.
\end{itemize}
%
See also the CTAN topic \href{http://ctan.org/topic/subdocs}{\textsf{subdocs}}
for further related packages.
The present package differs from the above solutions in that
a document structure constructed with the conventional |\include| mechanism
just needs two extra commands at the top of every file
such that all constituent files can be compiled individually.

%%%%%%%%%%%%%%%%%%%%%%%%%%%%%%%%%%%%%%%%%%%%%%%%%%%%%%%%%%%%%%%%%%%%%%%%%%%%%%%%
%\subsection{Feature Suggestions}
%
%The following is a list of features which may be useful for future
%versions of this package:
%%
%\begin{itemize}
%\item
%\ldots
%\end{itemize}

%%%%%%%%%%%%%%%%%%%%%%%%%%%%%%%%%%%%%%%%%%%%%%%%%%%%%%%%%%%%%%%%%%%%%%%%%%%%%%%%
\subsection{Revision History}

%%%%%%%%%%%%%%%%%%%%%%%%%%%%%%%%%%%%%%%%
\paragraph{v2.0:} 2018/12/30

\begin{itemize}
\item
immediate forward processing
\item
added |\childdocby| mechanism
\item
manual restructured
\end{itemize}

%%%%%%%%%%%%%%%%%%%%%%%%%%%%%%%%%%%%%%%%
\paragraph{v1.6:} 2018/01/17

\begin{itemize}
\item
application for development of include files
\item
corrections to manual
\end{itemize}

%%%%%%%%%%%%%%%%%%%%%%%%%%%%%%%%%%%%%%%%
\paragraph{v1.5:} 2017/05/21

\begin{itemize}
\item
more complete structuring introduced
\item
|\childdocof| introduced
\item
|\childdoc| renamed to |\childdocmain|
\item
|\childredirect| renamed to |\childdocforward| and |\childdocforwardprefix|
and functionality expanded
\end{itemize}

%%%%%%%%%%%%%%%%%%%%%%%%%%%%%%%%%%%%%%%%
\paragraph{v1.0:} 2017/04/27

\begin{itemize}
\item
manual and install package
\item
first version published on CTAN
\end{itemize}

%%%%%%%%%%%%%%%%%%%%%%%%%%%%%%%%%%%%%%%%
\paragraph{v0.6:} 2017/04/26

\begin{itemize}
\item
redirection mechanism added
\end{itemize}

%%%%%%%%%%%%%%%%%%%%%%%%%%%%%%%%%%%%%%%%
\paragraph{v0.5:} 2017/04/26

\begin{itemize}
\item
functionality in definition file
\end{itemize}


%%%%%%%%%%%%%%%%%%%%%%%%%%%%%%%%%%%%%%%%%%%%%%%%%%%%%%%%%%%%%%%%%%%%%%%%%%%%%%%%
%%%%%%%%%%%%%%%%%%%%%%%%%%%%%%%%%%%%%%%%%%%%%%%%%%%%%%%%%%%%%%%%%%%%%%%%%%%%%%%%
%%%%%%%%%%%%%%%%%%%%%%%%%%%%%%%%%%%%%%%%%%%%%%%%%%%%%%%%%%%%%%%%%%%%%%%%%%%%%%%%
\appendix

\settowidth\MacroIndent{\rmfamily\scriptsize 000\ }

 \DocInput{childdoc.dtx}

\end{document}
%</driver>
% \fi
%
% %%%%%%%%%%%%%%%%%%%%%%%%%%%%%%%%%%%%%%%%%%%%%%%%%%%%%%%%%%%%%%%%%%%%%%%%%%%%%%
% %%%%%%%%%%%%%%%%%%%%%%%%%%%%%%%%%%%%%%%%%%%%%%%%%%%%%%%%%%%%%%%%%%%%%%%%%%%%%%
% \section{Sample}
%\iffalse
%<*samplemain>
%\fi
%
% The following presents a sample document
% with two chapters, two parts, a title page,
% a compile flag as well as three forwarding files to set the flag.
% It consists of eight |.tex| files:
% \begin{center}
% \begin{tabular}{ll}
% |cdocsamp.tex|&main file\\
% |cdocsch1.tex|&include file for chapter 1\\
% |cdocsch2.tex|&include file for chapter 2\\
% |cdocspt3.tex|&include file for part 3\\
% |cdocspt4.tex|&include file for part 4\\
% |cdocsdrf.tex|&forwarding file for main file in draft mode\\
% |cdocsfi1.tex|&forwarding file for final version of chapter 1\\
% |cdocsfi2.tex|&forwarding file for final version of chapter 2\\
% \end{tabular}
% \end{center}
% Each of the eight files can be compiled directly by the \LaTeX{} compiler.
%
% %%%%%%%%%%%%%%%%%%%%%%%%%%%%%%%%%%%%%%
% \paragraph{Main File.}
%
% The main file is called |cdocsamp.tex|.
%
% Load the \textsf{childdoc} definitions and
% declare the filename for the main document:
%    \begin{macrocode}
\input{childdoc.def}
\childdocmain{}
%    \end{macrocode}

% Optional override for |\version| flag:
%    \begin{macrocode}
%%\ifchilddoc\else\providecommand{\version}{draft}\fi
%    \end{macrocode}

% Define the default values for the |\version| flag
% (|final| for the main file and |draft| for childs):
%    \begin{macrocode}
\ifchilddoc
\providecommand{\version}{draft}
\else
\providecommand{\version}{final}
\fi
%    \end{macrocode}

% Load the standard document class:
%    \begin{macrocode}
\documentclass[12pt]{article}
%    \end{macrocode}

% Start the document body:
%    \begin{macrocode}
\begin{document}
%    \end{macrocode}

% Declare a title page.
% Print title, part of document being processed and version flag:
%    \begin{macrocode}
\addtocounter{page}{-1}
\begin{center}
{\LARGE\bfseries{}childdoc example\par}
\vspace{1cm}
\ifchilddoc
\ifchilddocmanual part\else chapter\fi:
`\childdocname' of `\childdocjob'\par
\else
main document: `\childdocjob'\par
\fi
version: \version\par
\end{center}
\newpage
%    \end{macrocode}

% Manually include selected file,
% otherwise process as usual:
%    \begin{macrocode}
\ifchilddocmanual
\section*{part `\childdocname'}
\input{\childdocname}
\else
%    \end{macrocode}

% Include the two chapters:
%    \begin{macrocode}
\include{cdocsch1}
\include{cdocsch2}
%    \end{macrocode}

% Include the two parts unless only chapters should be displayed:
%    \begin{macrocode}
\ifchilddoc\else
\section{part three}
\input{cdocspt3}
\section{part four}
\input{cdocspt4}
\fi
%    \end{macrocode}

% Process as usual until here:
%    \begin{macrocode}
\fi
%    \end{macrocode}

% End of document body:
%    \begin{macrocode}
\end{document}
%    \end{macrocode}
%\iffalse
%</samplemain>
%\fi
%
% %%%%%%%%%%%%%%%%%%%%%%%%%%%%%%%%%%%%%%
% \paragraph{Chapter Include Files.}
%
% The include files are called |cdocsch1.tex| and |cdocsch2.tex|.
%
%\iffalse
%<*samplechap1|samplechap2>
%\fi

% Optional override for |\version| flag:
%    \begin{macrocode}
%%\providecommand{\version}{final}
%    \end{macrocode}

% Include the main document:
%    \begin{macrocode}
\input{childdoc.def}
\childdocof{cdocsamp}
%    \end{macrocode}

%\iffalse
%</samplechap1|samplechap2>
%\fi
%
%\iffalse
%<*samplechap1>
%\fi
% Some text for chapter 1:
%    \begin{macrocode}
\section{one}
some text in chapter one
%    \end{macrocode}

%\iffalse
%</samplechap1>
%\fi
% Some text for chapter 2:
%\iffalse
%<*samplechap2>
%\fi
%    \begin{macrocode}
\section{two}
more text in chapter two
%    \end{macrocode}

%\iffalse
%</samplechap2>
%\fi
%
% %%%%%%%%%%%%%%%%%%%%%%%%%%%%%%%%%%%%%%
% \paragraph{Part Include Files.}
%
% The include files are called |cdocspt3.tex| and |cdocspt4.tex|.
%
%\iffalse
%<*samplepart3|samplepart4>
%\fi

% Optional override for |\version| flag:
%    \begin{macrocode}
%%\providecommand{\version}{final}
%    \end{macrocode}

% Include the main document:
%    \begin{macrocode}
\input{childdoc.def}
\childdocby{cdocsamp}
%    \end{macrocode}

%\iffalse
%</samplepart3|samplepart4>
%\fi
%
%\iffalse
%<*samplepart3>
%\fi
% Some text for part 3:
%    \begin{macrocode}
some text in part three
%    \end{macrocode}

%\iffalse
%</samplepart3>
%\fi
% Some text for part 4:
%\iffalse
%<*samplepart4>
%\fi
%    \begin{macrocode}
more text in part four
%    \end{macrocode}

%\iffalse
%</samplepart4>
%\fi
%
% %%%%%%%%%%%%%%%%%%%%%%%%%%%%%%%%%%%%%%
% \paragraph{Forwarding for a Complete Draft.}
%
% The following forwarding file |cdocsdrf.tex|
% compiles the main document in draft mode:
%\iffalse
%<*sampledraft>
%\fi
%    \begin{macrocode}
\def\version{draft}
\input{childdoc.def}
\childdocforward{cdocsamp}
%    \end{macrocode}

%\iffalse
%</sampledraft>
%\fi
%
% %%%%%%%%%%%%%%%%%%%%%%%%%%%%%%%%%%%%%%
% \paragraph{Forwarding for Final Version of the Chapters.}
%
% The following forwarding files |cdocsfn1.tex| and |cdocsfn2.tex|
% (with identical content)
% compile the final versions of the child documents
% |cdocsch1.tex| and |cdocsch2.tex|, respectively:
%\iffalse
%<*samplefinal>
%\fi
%    \begin{macrocode}
\def\version{final}
\input{childdoc.def}
\childdocforwardprefix[cdocsamp]{cdocsfn}{cdocsch}
%    \end{macrocode}

%\iffalse
%</samplefinal>
%\fi
%
% %%%%%%%%%%%%%%%%%%%%%%%%%%%%%%%%%%%%%%
% \paragraph{Command Line Processing.}
%
% The following three command lines generate the output files
% |cdocscld|, |cdocscl1| and |cdocscl2|
% which should be identical to
% |cdocsdrf|, |cdocsch1| and |cdocsfn2|, respectively:
% \begin{center}
% \begin{tabular}{l}
% |latex -jobname cdocscld \|\\
% |  "\def\version{draft}\input{childdoc.def}\childdocforward{cdocsamp}"|\\
% |latex -jobname cdocscl1 \|\\
% |  "\input{childdoc.def}\childdocforward[cdocsamp]{cdocsch1}"|\\
% |latex -jobname cdocscl2 \|\\
% |  "\def\version{final}\input{childdoc.def}\childdocforward{cdocsch2}"|
% \end{tabular}
% \end{center}
% Note that the trailing backslash on each first line
% merely continues the input to the second line
% (for convenient cut ant paste).
% Furthermore, the command |latex| can be replaced by any
% of its alternative versions such as |pdflatex|.
%
% %%%%%%%%%%%%%%%%%%%%%%%%%%%%%%%%%%%%%%%%%%%%%%%%%%%%%%%%%%%%%%%%%%%%%%%%%%%%%%
% %%%%%%%%%%%%%%%%%%%%%%%%%%%%%%%%%%%%%%%%%%%%%%%%%%%%%%%%%%%%%%%%%%%%%%%%%%%%%%
% \section{Implementation}
%\iffalse
%<*package>
%\fi
%
% This section describes the definitions file |childdoc.def|.

% The definitions cannot be loaded using |\usepackage| or |\RequirePackage|
% which has a mechanism to prevent loading a style file more than once.
% When loading the definitions by means of |\input|
% multiple instances have to be prevented manually:
%\iffalse
%This code needs to be before the `\ProvidesFile' directive
%which is defined at the beginning of this file.
%Therefore it is also placed there and commented out here.
%</package>
%<*discard>
%\fi
%    \begin{macrocode}
\ifdefined\childdocmain\endinput\fi
%    \end{macrocode}
%\iffalse
%</discard>
%<*package>
%\fi
%
% \macro{\ifchilddoc}
% \macro{\ifchilddocmanual}
% The conditional |\ifchilddoc| tells whether a
% child (true) or main (false) document is being compiled.
% The conditional |\ifchilddocmanual| tells whether
% the |\includeonly| mechanism is used (false) or
% the selection of child files must be performed manually (true).
% The definitions initialise to false:
%    \begin{macrocode}
\newif\ifchilddoc
\newif\ifchilddocmanual
%    \end{macrocode}

% \macro{\childdocname}
% \macro{\childdocjob}
% The macro |\childdocname| stores the name of the main document
% to be compiled. The macro |\childdocjob| stores the name of
% the document on which the \LaTeX{} compiler was originally invoked.
% The content of |\jobname| cannot be compared
% to filenames specified in the source due to different catcodes.
% The following code rescans |\jobname|, stores the result
% in |\childdocname| and saves a copy in |\childdocjob|:
%    \begin{macrocode}
\edef\childdocname{\scantokens\expandafter{\jobname\noexpand}}
\let\childdocjob\childdocname
%    \end{macrocode}

% \macro{\childdocdisable}
% The macro |\childdocdisable| prevents the main file
% from being processed more than once.
% At this stage, the main document command |\childdocmain|
% is assumed to be called once again where it should do nothing.
% Any subsequent call to it should prevent
% a secondary processing of the main document
% It overwrites the forwarding commands
% |\childdocof| and |\childdocforward|
% with empty macros to prevent further inclusions of the main document:
%    \begin{macrocode}
\newcommand{\childdocdisable}
{
  \renewcommand{\childdocmain}[1]{\renewcommand{\childdocmain}[1]{\endinput}}
  \renewcommand{\childdocof}[1]{}
  \renewcommand{\childdocby}[2][]{}
  \renewcommand{\childdocforward}[2][]{}
  \renewcommand{\childdocdisable}{}
}
%    \end{macrocode}

% \macro{\childdocmain}
% The macro |\childdocmain| is to be called at the top of the main file
% with nothing or the main filename (without extension) as argument.
% First, it breaks loops.
% If the argument is not empty and does not match |\childdocname|
% (which is set by the first inclusion of |childdoc.def|),
% |\ifchilddoc| is set to true, |\includeonly| is applied to the child file
% and |\jobname| is set to the main file
% (for proper handling of |.aux| files):
%    \begin{macrocode}
\newcommand{\childdocmain}[1]
{
  \childdocdisable\childdocmain{}
  \if?#1?\else
    \begingroup
      \def\childdoctmp{#1}
      \ifx\childdoctmp\childdocname
        \def\childdoctmp{}
      \else
        \def\childdoctmp
        {
          \childdoctrue
          \includeonly{\childdocname}
          \def\childdocjob{#1}
          \def\jobname{#1}
        }
      \fi
      \expandafter
    \endgroup
    \childdoctmp
  \fi
}
%    \end{macrocode}

% \macro{\childdocof}
% The command |\childdocof| redirects
% compilation to the main file |#1|.
%    \begin{macrocode}
\newcommand{\childdocof}[1]
{
  \childdocdisable
  \childdoctrue
  \includeonly{\childdocname}
  \def\jobname{#1}
  \def\childdocjob{#1}
  \input{#1}
}
%    \end{macrocode}

% \macro{\childdocby}
% The command |\childdocby| ....
%    \begin{macrocode}
\newcommand{\childdocby}[2][]
{
  \childdocdisable
  \childdoctrue
  \childdocmanualtrue
  \if?#1?\else
    \def\jobname{#2}
  \fi
  \def\childdocjob{#2}
  \input{#2}
  \endinput
}
%    \end{macrocode}

% \macro{\childdocforward}
% The command |\childdocforward| redirects
% compilation to the main file or
% (if the optional argument is given) a child file.
% Parameters are set as if the main file
% or a child file starting with |\childdocof| was compiled.
% Then compilation is handed over to the main file:
%    \begin{macrocode}
\newcommand{\childdocforward}[2][]
{
  \begingroup
    \if?#1?
      \def\childdoctmp
      {
        \def\childdocname{#2}
        \def\childdocjob{#2}
        \def\jobname{#2}
        \input{#2}
        \endinput
      }
    \else
      \def\childdoctmp
      {
        \childdocdisable
        \def\childdocname{#2}
        \childdoctrue
        \includeonly{#2}
        \def\childdocjob{#1}
        \def\jobname{#1}
        \input{#1}
        \endinput
      }
    \fi
    \expandafter
  \endgroup
  \childdoctmp
}
%    \end{macrocode}

% \macro{\childdocforwardprefix}
% The command |\childdocforwardprefix| redirects
% compilation to the main or a child file by means of a pattern.
% The prefix |#1| in the current filename is replaced by |#2|
% and the suffix of the current filename is kept
% (it is assumed that the filename does not contain the substring `|~~~|'
% which is used as a delimiter).
% Compilation is handed over to the new file by |\childdocforward|:
%    \begin{macrocode}
\newcommand{\childdocforwardprefix}[3][]
{
  \begingroup
    \def\childdocextract #2##1~~~{\def\childdoctmp{\childdocforward[#1]{#3##1}}}
    \expandafter\childdocextract\childdocname~~~
    \expandafter
  \endgroup
  \childdoctmp
}
%    \end{macrocode}

% \macro{\childdoc}
% The deprecated macro |\childdoc| is a legacy version of |\childdocmain|:
%    \begin{macrocode}
\newcommand{\childdoc}{\childdocmain}
%    \end{macrocode}

% \macro{\childdocredirect}
% The deprecated macro |\childdocredirect| is a legacy version
% of |\childdocforward| and |\childdocforwardprefix|:
%    \begin{macrocode}
\newcommand{\childdocredirect}[2][]
{
  \begingroup
    \if?#1?
      \def\childdoctmp{\childdocforward{#2}}
    \else
      \def\childdoctmp{\childdocforwardprefix{#1}{#2}}
    \fi
    \expandafter
  \endgroup
  \childdoctmp
}
%    \end{macrocode}

%\iffalse
%</package>
%\fi
%
\endinput
|\\
|\childdocforward{|\textit{main}|}|
\end{tabular}
\end{center}
%
Likewise, the following files |final|\textit{nn}|.tex|
compile the final version of the child document
|child|\textit{nn}|.tex|:
%
\begin{center}
\begin{tabular}{l}
|\def\version{final}|\\
|% \iffalse
%
% childdoc.dtx Copyright (C) 2017-2018 Niklas Beisert
%
% This work may be distributed and/or modified under the
% conditions of the LaTeX Project Public License, either version 1.3
% of this license or (at your option) any later version.
% The latest version of this license is in
%   http://www.latex-project.org/lppl.txt
% and version 1.3 or later is part of all distributions of LaTeX
% version 2005/12/01 or later.
%
% This work has the LPPL maintenance status `maintained'.
%
% The Current Maintainer of this work is Niklas Beisert.
%
% This work consists of the files childdoc.dtx and childdoc.ins
% and the derived files childdoc.def and cdocsamp.tex with
% cdocsch1.tex, cdocsch2.tex, cdocsdrf.tex, cdocsfn1.tex, cdocsfn2.tex.
%
%<package>\ifdefined\childdocmain\endinput\fi
%<package>\ProvidesFile{childdoc.def}[2018/12/30 v2.0 child document driver]
%<samplemain>\ProvidesFile{cdocsamp.tex}[2018/12/30 v2.0 sample for childdoc]
%<*driver>
%\ProvidesFile{childdoc.drv}[2018/12/30 v2.0 childdoc reference manual file]
\PassOptionsToClass{10pt,a4paper}{article}
\documentclass{ltxdoc}

\usepackage[margin=35mm]{geometry}
\usepackage{hyperref}
\usepackage{hyperxmp}
\usepackage[usenames]{color}

\hypersetup{colorlinks=true}
\hypersetup{pdfstartview=FitH}
\hypersetup{pdfpagemode=UseNone}
\hypersetup{pdfsource={}}
\hypersetup{pdflang={en-UK}}
\hypersetup{pdfcopyright={Copyright 2017-2018 Niklas Beisert.
  This work may be distributed and/or modified under the
  conditions of the LaTeX Project Public License, either version 1.3
  of this license or (at your option) any later version.}}
\hypersetup{pdflicenseurl={http://www.latex-project.org/lppl.txt}}
\hypersetup{pdfcontactaddress={ETH Zurich, ITP, HIT K,
  Wolfgang-Pauli-Strasse 27}}
\hypersetup{pdfcontactpostcode={8093}}
\hypersetup{pdfcontactcity={Zurich}}
\hypersetup{pdfcontactcountry={Switzerland}}
\hypersetup{pdfcontactemail={nbeisert@itp.phys.ethz.ch}}
\hypersetup{pdfcontacturl={http://people.phys.ethz.ch/\xmptilde nbeisert/}}

\newcommand{\secref}[1]{\hyperref[#1]{section \ref*{#1}}}

\parskip1ex
\parindent0pt
\let\olditemize\itemize
\def\itemize{\olditemize\parskip0pt}

\begin{document}

\title{The \textsf{childdoc} Package}
\hypersetup{pdftitle={The childdoc Package}}
\author{Niklas Beisert\\[2ex]
  Institut f\"ur Theoretische Physik\\
  Eidgen\"ossische Technische Hochschule Z\"urich\\
  Wolfgang-Pauli-Strasse 27, 8093 Z\"urich, Switzerland\\[1ex]
  \href{mailto:nbeisert@itp.phys.ethz.ch}
  {\texttt{nbeisert@itp.phys.ethz.ch}}}
\hypersetup{pdfauthor={Niklas Beisert}}
\hypersetup{pdfsubject={Manual for the LaTeX2e Package childdoc}}
\date{30 December 2018, \textsf{v2.0}}
\maketitle

\begin{abstract}\noindent
\textsf{childdoc} is a \LaTeXe{} package
that enables the direct compilation
of document sections included by |\include|
to individual files.
\end{abstract}

\begingroup
\parskip0ex
\tableofcontents
\endgroup

%%%%%%%%%%%%%%%%%%%%%%%%%%%%%%%%%%%%%%%%%%%%%%%%%%%%%%%%%%%%%%%%%%%%%%%%%%%%%%%%
%%%%%%%%%%%%%%%%%%%%%%%%%%%%%%%%%%%%%%%%%%%%%%%%%%%%%%%%%%%%%%%%%%%%%%%%%%%%%%%%
\section{Introduction}

\LaTeX{} provides a mechanism to structure a large document (such as a book)
into a main file and several child files (containing the chapters)
using the |\include| command.
This mechanism is beneficial for documents
which span hundreds of pages in order to
make the source file(s) more manageable.
Moreover, compilation can be restricted to
selected child files by means of the |\includeonly| command.
The latter feature can be used to reduce the compilation time while editing
(this was significantly more useful in the earlier days of \LaTeX{})
or to generate a smaller document which is easier to navigate.
Another application of |\includeonly| is to generate
documents consisting of selected parts of the complete document.

However, there are a few drawbacks of the plain |\include| mechanism:
\begin{itemize}
\item
The child files cannot be compiled on their own,
they can only be compiled via the main file.
A naive editing environment
(such as a text editor with an option
to have the current file processed by \LaTeX)
may require one to switch to the main file before compiling;
attempting to compile the child file produces errors.
\item
The main file must be modified (each time)
to adjust the |\includeonly| command
to the present needs. This easily leaves the main file in a messy state.
\item
The generated document will always carry the filename
of the main document. This is inconvenient if
several child files are to be compiled and
to be kept for distribution.
\end{itemize}

The present package provides a simple interface
to make child files individually compilable by \LaTeX{}.
Compiling a child file then has the same effect as compiling
the main file with an |\includeonly| command
to select the appropriate child.
Moreover the generated document will carry the name of the child
rather than the main file.
This resolves all three above issues.

This feature is meant to make the editing of books,
thesis documents and lecture notes somewhat more convenient.
However, the package can also be used efficiently for
composing a series of documents (such as exercise sheets)
which are typically distributed individually.
It then assists the author in generating the individual documents
(potentially in different versions)
as well as a document containing the collected series.
Another application is in developing style files
or other kinds of included material
where compilation of the style file could redirect
to a sample or test file.

%%%%%%%%%%%%%%%%%%%%%%%%%%%%%%%%%%%%%%%%%%%%%%%%%%%%%%%%%%%%%%%%%%%%%%%%%%%%%%%%
%%%%%%%%%%%%%%%%%%%%%%%%%%%%%%%%%%%%%%%%%%%%%%%%%%%%%%%%%%%%%%%%%%%%%%%%%%%%%%%%
\section{Usage}

First of all, the package \textsf{childdoc} is \emph{not} a standard
\LaTeXe{} |.sty| style file! Therefore it needs to be invoked in
a non-standard way.

%%%%%%%%%%%%%%%%%%%%%%%%%%%%%%%%%%%%%%%%%%%%%%%%%%%%%%%%%%%%%%%%%%%%%%%%%%%%%%%%
\subsection{Included Files}
\label{sec:include}

%%%%%%%%%%%%%%%%%%%%%%%%%%%%%%%%%%%%%%%%
\DescribeMacro{\childdocmain}
To use the package, add the commands
\begin{center}
\begin{tabular}{l}
|\input{childdoc.def}|\\
|\childdocmain{}|\\
\end{tabular}
\end{center}
at the very top of the main \LaTeX{} file,
in particular \emph{before} the |\documentclass| statement!
The argument of |\childdocmain| should be left empty
(but it must be present).

%%%%%%%%%%%%%%%%%%%%%%%%%%%%%%%%%%%%%%%%
\DescribeMacro{\childdocof}
Furthermore, add the commands
\begin{center}
\begin{tabular}{l}
|\input{childdoc.def}|\\
|\childdocof{|\textit{main}|}|\\
\end{tabular}
\end{center}
at the top of every child file \textit{child}
which is included by |\include{|\textit{child}|}|
from within the main file
(or at least for those files to be compiled individually).
The argument \textit{main} must be the filename of the main file.

There are a couple of
considerations in setting up the main and child documents:

%%%%%%%%%%%%%%%%%%%%%%%%%%%%%%%%%%%%%%%%
\paragraph{Restrictions.}

Please note the following restrictions:
\begin{itemize}
\item
|\childdocmain| must be called with one argument \textit{main}
to ensure compatibility with earlier version of the package.
It must either be empty (|\childdocmain{}|)
or precisely match the filename of the main file in which it is specified.
See \secref{sec:detection} for further information.
\item
The filename \textit{main} must be specified without the |.tex| extension.
\item
The filename \textit{main} is case sensitive
(even in case-insensitive file systems)
due to internal string comparison.
\item
The argument \textit{main} should be fully expanded, it cannot be a macro.
\item
Subdirectories and special characters should be avoided in filenames.
\item
The command |\childdocmain{|\textit{main}|}| must be followed by a whitespace.
It should not be followed immediately by another command
or by a comment mark `|%|'.
This is because the \TeX{} parser reads the token immediately following
the argument of |\childdocmain| and puts it
at the beginning of every child section;
however, a white\-space is ignored.
\end{itemize}

%%%%%%%%%%%%%%%%%%%%%%%%%%%%%%%%%%%%%%%%
\paragraph{Content of Main File.}

It is advisable to place all content in the child files included by |\include|.
Any output contained in the main file will appear in all child documents
unless suppressed manually;
it cannot be suppressed automatically by the |\includeonly| directive
and thus should normally be avoided.
A method to include some content in the main file
by means of conditional processing is described in \secref{sec:conditional}.

%%%%%%%%%%%%%%%%%%%%%%%%%%%%%%%%%%%%%%%%
\paragraph{Page Numbering.}

When only a part of the document is compiled,
the appropriate numbering of pages
(as well as other status parameters)
is determined from the |.aux| files.
The latter contain information from previous passes.
However this information needs to propagate through
all intermediate child documents.
Therefore the page numbering in child documents may well
be inconsistent until the complete document is compiled at least once.

A useful (if unconventional) way to always ensure a consistent
page numbering is to restart the numbering in each child document
and denote the pages by `\textit{child}|.|\textit{page}'
where \textit{child} represents the chapter/section number of the child file.
This can be achieved by the command
|\numberwithin{page}{|\textit{child}|}|
of the \textsf{amsmath} package
where \textit{child} can be |chapter| or |section|
depending on the chosen structuring.
Alternatively, one can modify the macro |\thepage| appropriately
and reset the counter |page| at the start of each child file.

%%%%%%%%%%%%%%%%%%%%%%%%%%%%%%%%%%%%%%%%%%%%%%%%%%%%%%%%%%%%%%%%%%%%%%%%%%%%%%%%
\subsection{Conditional Processing}
\label{sec:conditional}

The package provides a mechanism to compile different versions
of a document. To customise the versions further some conditional processing
can come in handy to distinguish which version is being compiled.
The package provides two macros to describe the compilation context:

%%%%%%%%%%%%%%%%%%%%%%%%%%%%%%%%%%%%%%%%
\DescribeMacro{\ifchilddoc}
The conditional |\ifchilddoc| distinguishes between the compilation of
child documents and the main document:
%
\begin{center}
|\ifchilddoc |\textit{child-code}| |[|\||else |\textit{main-code}]| \||fi|
\end{center}

%%%%%%%%%%%%%%%%%%%%%%%%%%%%%%%%%%%%%%%%
\DescribeMacro{\childdocname}
\DescribeMacro{\childdocjob}
The macro |\childdocname| contains the filename (without extension)
of the main or child file being processed.
Note that |\childdocjob| will always contain the name of the main file.

%%%%%%%%%%%%%%%%%%%%%%%%%%%%%%%%%%%%%%%%
\paragraph{Title Page.}

Conditional processing can be used to include a title or banner page
in the main document when proper precautions are taken.
Importantly, the code in the main file should ensure that the page counter
(as well as other status parameters which are stored in the |.aux| files)
takes the same value after the conditional processing.
Otherwise the page numbers may take divergent values
depending on which part is compiled.

For example, a title page could be declared by:
%
\begin{center}
\begin{tabular}{l}
|\ifchilddoc\||else|\\
|\addtocounter{page}{-1}|\\
\textit{code for title page}\\
|\newpage|\\
|\||fi|
\end{tabular}
\end{center}
%
A banner page for the child documents can be generated by:
%
\begin{center}
\begin{tabular}{l}
|\ifchilddoc|\\
|\addtocounter{page}{-1}|\\
\textit{code for banner page}\\
|\newpage|\\
|\||fi|
\end{tabular}
\end{center}
%
Here one could write a message such as:
\begin{center}
|This is the part \childdocname{} of \childdocjob{}.|
\end{center}

%%%%%%%%%%%%%%%%%%%%%%%%%%%%%%%%%%%%%%%%%%%%%%%%%%%%%%%%%%%%%%%%%%%%%%%%%%%%%%%%
\subsection{Flags}
\label{sec:flags}

The package makes it easy to generate different versions
of the main or child documents.
To this end compilation flags can be defined
and assigned different default values.
They will be particularly useful in conjunction
with the forwarding mechanism described in \secref{sec:forward}.

For example, it may be useful to have a flag |\version|
which can be set to |draft| or |final|.
The document source will contain some conditional code
depending on the value of |\version|.
Suppose further, the flag should default to |final| for the main file
and to |draft| for child files
which is a natural assignment for editing the document.
This is achieved by placing the following code
in the preamble of the main document
(below the |\childdocmain| directive):
%
\begin{center}
\begin{tabular}{l}
|\ifchilddoc|\\
|\providecommand{\version}{draft}|\\
|\||else|\\
|\providecommand{\version}{final}|\\
|\||fi|
\end{tabular}
\end{center}
%
The definition by |\providecommand| makes sure
that previous definitions are not overwritten.
Further statements |\providecommand{\version}{...}|
can thus be added before the above code to override it.

For the main file, one might add a line
(between |\childdocmain| and the above block)
%
\begin{center}
|%\ifchilddoc\||else\providecommand{\version}{draft}\||fi|
\end{center}
%
which can be uncommented to produce a draft version.
Likewise one can add a line to the very top of a child file
(above the |\childdocof{|\textit{main}|}| directive)
%
\begin{center}
|%\providecommand{\version}{final}|
\end{center}
%
which can be uncommented to produce the final version of this child document.

%%%%%%%%%%%%%%%%%%%%%%%%%%%%%%%%%%%%%%%%%%%%%%%%%%%%%%%%%%%%%%%%%%%%%%%%%%%%%%%%
\subsection{Forwarding}
\label{sec:forward}

Different versions of the main or child documents
using compilation flags as described in \secref{sec:flags}
can be (permanently) stored in different files
for convenient compilation, viewing and distribution.
To this end, the package defines a command
to pass on compilation to a different file:

%%%%%%%%%%%%%%%%%%%%%%%%%%%%%%%%%%%%%%%%
\DescribeMacro{\childdocforward}
The command |\childdocforward| redirects processing to
another source file:
%
\begin{center}
\begin{tabular}{l}
|\input{childdoc.def}|\\
|\childdocforward[|\textit{main}|]{|\textit{dest}|}|\\
\end{tabular}
\end{center}
%
The argument \textit{dest} is the destination file
(without extension).
It should be the main file or one of the child files.
Note that further \textsf{childdoc} directives
such as |\childdocof| and |\childdocforward|
in the indicated file will be processed in this form.
The optional argument \textit{main}
passes on directly to the main file \textit{main}
while pretending to compile the child \textit{dest}.
This form behaves as if \textit{dest}
issues |\childdocof{|\textit{main}|}| right away,
and no further \textsf{childdoc} directives will be processed.

%%%%%%%%%%%%%%%%%%%%%%%%%%%%%%%%%%%%%%%%
\DescribeMacro{\...prefix}
In the alternative form |\childdocforwardprefix|,
%
\begin{center}
\begin{tabular}{l}
|\input{childdoc.def}|\\
|\childdocforwardprefix[|\textit{main}|]{|\textit{prefix}|}{|\textit{dest}|}|
\end{tabular}
\end{center}
%
the destination file is determined by a pattern
depending on the current file:
To make this work, the current file must be called
`{\textit{prefix}\hspace{0.2em}\textit{suffix}}'
with \textit{prefix} matching precisely the argument.
Processing is then passed on to the file
`{\textit{dest}\hspace{0.2em}\textit{suffix}}'.
Surely, the same effect is achieved by
directly specifying the
argument `{\textit{dest}\hspace{0.2em}\textit{suffix}}'
in the first form.
However, that requires to set up a different file
for each child. With the alternative form of the command
all these files can have exactly the same content
which simplifies setting them up and maintaining them.

For example, the following file |draft.tex|
with a compilation flag |\version| as described in \secref{sec:flags}
compiles the main document as a draft:
%
\begin{center}
\begin{tabular}{l}
|\def\version{draft}|\\
|\input{childdoc.def}|\\
|\childdocforward{|\textit{main}|}|
\end{tabular}
\end{center}
%
Likewise, the following files |final|\textit{nn}|.tex|
compile the final version of the child document
|child|\textit{nn}|.tex|:
%
\begin{center}
\begin{tabular}{l}
|\def\version{final}|\\
|\input{childdoc.def}|\\
|\childdocforwardprefix{final}{child}|
\end{tabular}
\end{center}
%

Note that when several versions of a main file and/or of each child file
are to be generated, it may be convenient to set up a |Makefile| or
shell script to automatise the process.

%%%%%%%%%%%%%%%%%%%%%%%%%%%%%%%%%%%%%%%%%%%%%%%%%%%%%%%%%%%%%%%%%%%%%%%%%%%%%%%%
\subsection{Command Line Processing}
\label{sec:commandline}

The effect of redirection files can also be achieved by invoking
the \LaTeX{} compiler with a more elaborate command line.
Most conveniently this should be done as part
of a shell script or a |Makefile|.

When using \textsf{childdoc} in the main file, the following
command lines effectively perform a redirection
(note that depending on the shell being used,
backslashes may have to be doubled: `|\|' $\to$ `|\\|'):
%
\begin{center}
|... -jobname "|\textit{target}|" |\\|"|[\textit{flags}]%
|\input{childdoc.def}\childdocforward[|\textit{main}|]{|\textit{dest}|}"|
\end{center}
%
Here \textit{target} is the name of the output file,
\textit{main} is the name of the main file
and \textit{dest} is the name of the main or child file to be processed
(all filenames without extensions).
The optional argument \textit{main} can be omitted
if \textit{main} matches \textit{dest}.
Optionally, compilation \textit{flags} can be defined via |\def| commands.
This command line makes the \TeX{} engine believe
it is compiling the file \textit{target}
whose content is specified as the latter parameter.
The provided code then forwards the processing to
\textit{main} or \textit{dest} as described in \secref{sec:forward}.

%%%%%%%%%%%%%%%%%%%%%%%%%%%%%%%%%%%%%%%%%%%%%%%%%%%%%%%%%%%%%%%%%%%%%%%%%%%%%%%%
\subsection{Include by Input}
\label{sec:input}

Including child documents by |\include| has some restrictions by design.
Most notably, the content of a child document always occupies
its own set of pages; pages cannot be shared between child documents.
Usually, this behaviour makes perfect sense
because each child document contain an essential part of the document.
However, in some situations it may be desirable to compose
a document from a collection of parts
without having mandatory page breaks between then.
For this case, the package
provides a mechanism to include parts
by |\input| which can also be processed individually.
However, by construction this mechanism
requires manual handling of the content to be output.

%%%%%%%%%%%%%%%%%%%%%%%%%%%%%%%%%%%%%%%%
\DescribeMacro{\ifchilddocmanual}
The main file should be prepared as usual, see \secref{sec:include}.
However, the document body must make a distinction
between processing of an individual part and of the main document, e.g.:
%
\begin{center}
\begin{tabular}{l}
|\ifchilddocmanual|\\
|\input{\childdocname}|\\
|\||else|\\
\textit{document body with }|\input{|\textit{part}|}|\\
|\||fi|
\end{tabular}
\end{center}
%
The conditional |\ifchilddocmanual| is true whenever
a part to be included by |\input| is being compiled,
and the name of the part is stored in |\childdocname|.

%%%%%%%%%%%%%%%%%%%%%%%%%%%%%%%%%%%%%%%%
\DescribeMacro{\childdocby}
Each part to be included by |\input| should start with:
%
\begin{center}
\begin{tabular}{l}
|\input{childdoc.def}|\\
|\childdocby{|\textit{main}|}|\\
\end{tabular}
\end{center}
%
The directive |\childdocby| is similar to |\childdocof|
described in \secref{sec:include},
but the subsequent selection of content must be done manually.
To that end, both |\ifchilddoc| and |\ifchilddocmanual|
will be true upon processing of a part,
and the name of the part is stored in |\childdocname|.
Note that |\jobname| will be set to the filename of the current part
so that each part receives an individual |.aux| file
that does not interfere with the |.aux| file(s) of the main document.
This behaviour can be altered by the alternative form
|\childdocby[*]{|\textit{main}|}| (with a non-empty optional argument)
which uses the |.aux| file of the main document
by setting |\jobname| to \textit{main}.

%%%%%%%%%%%%%%%%%%%%%%%%%%%%%%%%%%%%%%%%%%%%%%%%%%%%%%%%%%%%%%%%%%%%%%%%%%%%%%%%
\subsection{Driver Development}
\label{sec:driver}

The \textsf{childdoc} mechanism can also be use for the development
of definition files such as \LaTeX{} styles or classes.
This case differs from the above setup with multiple parts
included by |\include| in that no |\includeonly| should be invoked.
This can be achieved by starting the include file
(before |\ProvidesPackage|) with:
%
\begin{center}
\begin{tabular}{l}
|\input{childdoc.def}|\\
|\childdocforward{|\textit{main}|}|\\
\end{tabular}
\end{center}
%
or alternatively with:
%
\begin{center}
\begin{tabular}{l}
|\input{childdoc.def}|\\
|\childdocby{|\textit{main}|}|\\
\end{tabular}
\end{center}
%
Both forms have slightly different effects as described above.
The main file is prepared as usual, see \secref{sec:include}.

%%%%%%%%%%%%%%%%%%%%%%%%%%%%%%%%%%%%%%%%%%%%%%%%%%%%%%%%%%%%%%%%%%%%%%%%%%%%%%%%
\subsection{Legacy Detection}
\label{sec:detection}

The directive |\childdocmain| in the main file can detect
whether the complete document or merely a child is to be compiled
even without using the directive |\childdocof|.
This method is deprecated because it is less robust
and there is no compelling reason to use it;
it is merely provided for backward compatibility
and it may be removed in future versions.

If the detection mechanism is to be used,
it is mandatory to correctly specify
the filename of the main file as the argument of |\childdocmain|:
%
\begin{center}
\begin{tabular}{l}
|\input{childdoc.def}|\\
|\childdocmain{|\textit{main}|}|\\
\end{tabular}
\end{center}
%
If |\jobname| does not match the argument \textit{main} of |\childdocmain|,
it is assumed that |\jobname| points to the child file to be compiled.
When using |\childdocmain| with the main file specified as argument,
it suffices to start a child file
with just |\input{|\textit{main}|}|
without loading of the package and using |\childdocof|.
If instead all processing is done
with the appropriate \textsf{childdoc} directives,
the argument of \textit{main} of |\childdocmain| can be empty.

An alternative version of the command line processing described
in \secref{sec:commandline} using the detection mechanism reads:
%
\begin{center}
|... -jobname "|\textit{target}|" "|[\textit{flags}]%
[|\def\jobname{|\textit{dest}|}|]|\input{|\textit{main}|}"|
\end{center}

%%%%%%%%%%%%%%%%%%%%%%%%%%%%%%%%%%%%%%%%%%%%%%%%%%%%%%%%%%%%%%%%%%%%%%%%%%%%%%%%
\subsection{Manual Code}
\label{sec:manual}

In case one cannot be certain whether the definitions file |childdoc.def|
is installed on the target \TeX{} distribution
and one prefers not to ship it,
it is conceivable to paste a few relevant commands into the sources.

To that end, drop all statements |\input{childdoc.def}|
and perform the replacements as outlined below.
Instead of |\childdocmain{|\textit{main}|}| add the following code
to the top of the main file:
%
\begin{center}
\begin{tabular}{l}
|\||ifdefined\childdocname\endinput\||fi\newif\ifchilddoc|\\
|\edef\childdocname{\scantokens\expandafter{\jobname\noexpand}}|\\
|\def\childdocmain{|\textit{main}|}\||ifx\childdocmain\childdocname\||else|\\
|\childdoctrue\includeonly{\childdocname}\let\jobname\childdocmain\||fi|\\
\end{tabular}
\end{center}
%
Instead of |\childdocof{|\textit{main}|}| just include the main file
at the top of each child file:
%
\begin{center}
|\input{|\textit{main}|}|
\end{center}
%
A simple redirection |\childdocforward{|\textit{dest}|}| is achieved by:
%
\begin{center}
|\def\jobname{|\textit{dest}|}\input{\jobname}|
\end{center}
%
The redirection with prefix
|\childdocforwardprefix[|\textit{prefix}|]{|\textit{dest}|}|
is accomplished by:
%
\begin{center}
\begin{tabular}{l}
|{\edef\jobname{\scantokens\expandafter{\jobname\noexpand}}|\\
|\def\redirectjob |\textit{prefix}|#1~~~{\gdef\jobname{|\textit{dest}|#1}}|\\
|\expandafter\redirectjob\jobname~~~}\input{\jobname}|
\end{tabular}
\end{center}

In an alternative approach,
child documents can be compiled by a specific command line
without additional code or specific definitions:
%
\begin{center}
|... -jobname "|\textit{target}|" "|[\textit{flags}]%
|\includeonly{|\textit{dest}|}\input{|\textit{main}|}"|
\end{center}
%

%%%%%%%%%%%%%%%%%%%%%%%%%%%%%%%%%%%%%%%%%%%%%%%%%%%%%%%%%%%%%%%%%%%%%%%%%%%%%%%%
%%%%%%%%%%%%%%%%%%%%%%%%%%%%%%%%%%%%%%%%%%%%%%%%%%%%%%%%%%%%%%%%%%%%%%%%%%%%%%%%
\section{Information}

%%%%%%%%%%%%%%%%%%%%%%%%%%%%%%%%%%%%%%%%%%%%%%%%%%%%%%%%%%%%%%%%%%%%%%%%%%%%%%%%
\subsection{Copyright}

Copyright \copyright{} 2017--2018 Niklas Beisert

This work may be distributed and/or modified under the
conditions of the \LaTeX{} Project Public License, either version 1.3
of this license or (at your option) any later version.
The latest version of this license is in
  \url{http://www.latex-project.org/lppl.txt}
and version 1.3 or later is part of all distributions of \LaTeX{}
version 2005/12/01 or later.

This work has the LPPL maintenance status `maintained'.

The Current Maintainer of this work is Niklas Beisert.

This work consists of the files |README.txt|, |childdoc.ins| and |childdoc.dtx|
as well as the derived files |childdoc.def|, |cdocsamp.tex|
with |cdocsch1.tex|, |cdocsch2.tex|, |cdocspt3.tex|, |cdocspt4.tex|,
|cdocsdrf.tex|, |cdocsfn1.tex|, |cdocsfn2.tex|
as well as |childdoc.pdf|.

%%%%%%%%%%%%%%%%%%%%%%%%%%%%%%%%%%%%%%%%%%%%%%%%%%%%%%%%%%%%%%%%%%%%%%%%%%%%%%%%
\subsection{Files and Installation}

The package consists of the files:
%
\begin{center}
\begin{tabular}{ll}
    |README.txt|   & readme file \\
    |childdoc.ins| & installation file \\
    |childdoc.dtx| & source file \\
    |childdoc.def| & definition file \\
    |cdocsamp.tex| & sample main file \\
    |cdocsch1.tex| & sample include file \\
    |cdocsch2.tex| & sample include file \\
    |cdocspt3.tex| & sample part file \\
    |cdocspt4.tex| & sample part file \\
    |cdocsdrf.tex| & sample redirection file \\
    |cdocsfn1.tex| & sample redirection file \\
    |cdocsfn2.tex| & sample redirection file \\
    |childdoc.pdf| & manual
\end{tabular}
\end{center}
%
The distribution consists of the files
|README.txt|, |childdoc.ins| and |childdoc.dtx|.
%
\begin{itemize}
\item
Run (pdf)\LaTeX{} on |childdoc.dtx|
to compile the manual |childdoc.pdf| (this file).
\item
Run \LaTeX{} on |childdoc.ins| to create the definitions file |childdoc.def|
and the sample |cdocsamp.tex| with include files
|cdocsch1.tex|, |cdocsch2.tex|, |cdocspt3.tex|, |cdocspt4.tex|,
|cdocsdrf.tex|, |cdocsfn1.tex|, |cdocsfn2.tex|.
Then copy the file |childdoc.def| to an appropriate directory of your \LaTeX{}
distribution, e.g.\ \textit{texmf-root}|/tex/latex/childdoc|.
\end{itemize}

%%%%%%%%%%%%%%%%%%%%%%%%%%%%%%%%%%%%%%%%%%%%%%%%%%%%%%%%%%%%%%%%%%%%%%%%%%%%%%%%
\subsection{Related CTAN Packages}

There are several other packages which offer a similar functionality:
%
\begin{itemize}
\item
The packages
\href{http://ctan.org/pkg/docmute}{\textsf{docmute}},
\href{http://ctan.org/pkg/includex}{\textsf{includex}} and
\href{http://ctan.org/pkg/standalone}{\textsf{standalone}}
provide commands to include only the document body of
a child file thus allowing both files to be compiled individually.
\item
The packages \href{http://ctan.org/pkg/subdocs}{\textsf{subdocs}}
and \href{http://ctan.org/pkg/subfiles}{\textsf{subfiles}}
provide structures in which the main and child documents can be
encapsulated and allowing them to be compiled individually.
The inclusion mechanism is different from the conventional |\include|.
\item
The package \href{http://ctan.org/pkg/combine}{\textsf{combine}}
is an elaborate solution to combine several documents into one.
\end{itemize}
%
See also the CTAN topic \href{http://ctan.org/topic/subdocs}{\textsf{subdocs}}
for further related packages.
The present package differs from the above solutions in that
a document structure constructed with the conventional |\include| mechanism
just needs two extra commands at the top of every file
such that all constituent files can be compiled individually.

%%%%%%%%%%%%%%%%%%%%%%%%%%%%%%%%%%%%%%%%%%%%%%%%%%%%%%%%%%%%%%%%%%%%%%%%%%%%%%%%
%\subsection{Feature Suggestions}
%
%The following is a list of features which may be useful for future
%versions of this package:
%%
%\begin{itemize}
%\item
%\ldots
%\end{itemize}

%%%%%%%%%%%%%%%%%%%%%%%%%%%%%%%%%%%%%%%%%%%%%%%%%%%%%%%%%%%%%%%%%%%%%%%%%%%%%%%%
\subsection{Revision History}

%%%%%%%%%%%%%%%%%%%%%%%%%%%%%%%%%%%%%%%%
\paragraph{v2.0:} 2018/12/30

\begin{itemize}
\item
immediate forward processing
\item
added |\childdocby| mechanism
\item
manual restructured
\end{itemize}

%%%%%%%%%%%%%%%%%%%%%%%%%%%%%%%%%%%%%%%%
\paragraph{v1.6:} 2018/01/17

\begin{itemize}
\item
application for development of include files
\item
corrections to manual
\end{itemize}

%%%%%%%%%%%%%%%%%%%%%%%%%%%%%%%%%%%%%%%%
\paragraph{v1.5:} 2017/05/21

\begin{itemize}
\item
more complete structuring introduced
\item
|\childdocof| introduced
\item
|\childdoc| renamed to |\childdocmain|
\item
|\childredirect| renamed to |\childdocforward| and |\childdocforwardprefix|
and functionality expanded
\end{itemize}

%%%%%%%%%%%%%%%%%%%%%%%%%%%%%%%%%%%%%%%%
\paragraph{v1.0:} 2017/04/27

\begin{itemize}
\item
manual and install package
\item
first version published on CTAN
\end{itemize}

%%%%%%%%%%%%%%%%%%%%%%%%%%%%%%%%%%%%%%%%
\paragraph{v0.6:} 2017/04/26

\begin{itemize}
\item
redirection mechanism added
\end{itemize}

%%%%%%%%%%%%%%%%%%%%%%%%%%%%%%%%%%%%%%%%
\paragraph{v0.5:} 2017/04/26

\begin{itemize}
\item
functionality in definition file
\end{itemize}


%%%%%%%%%%%%%%%%%%%%%%%%%%%%%%%%%%%%%%%%%%%%%%%%%%%%%%%%%%%%%%%%%%%%%%%%%%%%%%%%
%%%%%%%%%%%%%%%%%%%%%%%%%%%%%%%%%%%%%%%%%%%%%%%%%%%%%%%%%%%%%%%%%%%%%%%%%%%%%%%%
%%%%%%%%%%%%%%%%%%%%%%%%%%%%%%%%%%%%%%%%%%%%%%%%%%%%%%%%%%%%%%%%%%%%%%%%%%%%%%%%
\appendix

\settowidth\MacroIndent{\rmfamily\scriptsize 000\ }

 \DocInput{childdoc.dtx}

\end{document}
%</driver>
% \fi
%
% %%%%%%%%%%%%%%%%%%%%%%%%%%%%%%%%%%%%%%%%%%%%%%%%%%%%%%%%%%%%%%%%%%%%%%%%%%%%%%
% %%%%%%%%%%%%%%%%%%%%%%%%%%%%%%%%%%%%%%%%%%%%%%%%%%%%%%%%%%%%%%%%%%%%%%%%%%%%%%
% \section{Sample}
%\iffalse
%<*samplemain>
%\fi
%
% The following presents a sample document
% with two chapters, two parts, a title page,
% a compile flag as well as three forwarding files to set the flag.
% It consists of eight |.tex| files:
% \begin{center}
% \begin{tabular}{ll}
% |cdocsamp.tex|&main file\\
% |cdocsch1.tex|&include file for chapter 1\\
% |cdocsch2.tex|&include file for chapter 2\\
% |cdocspt3.tex|&include file for part 3\\
% |cdocspt4.tex|&include file for part 4\\
% |cdocsdrf.tex|&forwarding file for main file in draft mode\\
% |cdocsfi1.tex|&forwarding file for final version of chapter 1\\
% |cdocsfi2.tex|&forwarding file for final version of chapter 2\\
% \end{tabular}
% \end{center}
% Each of the eight files can be compiled directly by the \LaTeX{} compiler.
%
% %%%%%%%%%%%%%%%%%%%%%%%%%%%%%%%%%%%%%%
% \paragraph{Main File.}
%
% The main file is called |cdocsamp.tex|.
%
% Load the \textsf{childdoc} definitions and
% declare the filename for the main document:
%    \begin{macrocode}
\input{childdoc.def}
\childdocmain{}
%    \end{macrocode}

% Optional override for |\version| flag:
%    \begin{macrocode}
%%\ifchilddoc\else\providecommand{\version}{draft}\fi
%    \end{macrocode}

% Define the default values for the |\version| flag
% (|final| for the main file and |draft| for childs):
%    \begin{macrocode}
\ifchilddoc
\providecommand{\version}{draft}
\else
\providecommand{\version}{final}
\fi
%    \end{macrocode}

% Load the standard document class:
%    \begin{macrocode}
\documentclass[12pt]{article}
%    \end{macrocode}

% Start the document body:
%    \begin{macrocode}
\begin{document}
%    \end{macrocode}

% Declare a title page.
% Print title, part of document being processed and version flag:
%    \begin{macrocode}
\addtocounter{page}{-1}
\begin{center}
{\LARGE\bfseries{}childdoc example\par}
\vspace{1cm}
\ifchilddoc
\ifchilddocmanual part\else chapter\fi:
`\childdocname' of `\childdocjob'\par
\else
main document: `\childdocjob'\par
\fi
version: \version\par
\end{center}
\newpage
%    \end{macrocode}

% Manually include selected file,
% otherwise process as usual:
%    \begin{macrocode}
\ifchilddocmanual
\section*{part `\childdocname'}
\input{\childdocname}
\else
%    \end{macrocode}

% Include the two chapters:
%    \begin{macrocode}
\include{cdocsch1}
\include{cdocsch2}
%    \end{macrocode}

% Include the two parts unless only chapters should be displayed:
%    \begin{macrocode}
\ifchilddoc\else
\section{part three}
\input{cdocspt3}
\section{part four}
\input{cdocspt4}
\fi
%    \end{macrocode}

% Process as usual until here:
%    \begin{macrocode}
\fi
%    \end{macrocode}

% End of document body:
%    \begin{macrocode}
\end{document}
%    \end{macrocode}
%\iffalse
%</samplemain>
%\fi
%
% %%%%%%%%%%%%%%%%%%%%%%%%%%%%%%%%%%%%%%
% \paragraph{Chapter Include Files.}
%
% The include files are called |cdocsch1.tex| and |cdocsch2.tex|.
%
%\iffalse
%<*samplechap1|samplechap2>
%\fi

% Optional override for |\version| flag:
%    \begin{macrocode}
%%\providecommand{\version}{final}
%    \end{macrocode}

% Include the main document:
%    \begin{macrocode}
\input{childdoc.def}
\childdocof{cdocsamp}
%    \end{macrocode}

%\iffalse
%</samplechap1|samplechap2>
%\fi
%
%\iffalse
%<*samplechap1>
%\fi
% Some text for chapter 1:
%    \begin{macrocode}
\section{one}
some text in chapter one
%    \end{macrocode}

%\iffalse
%</samplechap1>
%\fi
% Some text for chapter 2:
%\iffalse
%<*samplechap2>
%\fi
%    \begin{macrocode}
\section{two}
more text in chapter two
%    \end{macrocode}

%\iffalse
%</samplechap2>
%\fi
%
% %%%%%%%%%%%%%%%%%%%%%%%%%%%%%%%%%%%%%%
% \paragraph{Part Include Files.}
%
% The include files are called |cdocspt3.tex| and |cdocspt4.tex|.
%
%\iffalse
%<*samplepart3|samplepart4>
%\fi

% Optional override for |\version| flag:
%    \begin{macrocode}
%%\providecommand{\version}{final}
%    \end{macrocode}

% Include the main document:
%    \begin{macrocode}
\input{childdoc.def}
\childdocby{cdocsamp}
%    \end{macrocode}

%\iffalse
%</samplepart3|samplepart4>
%\fi
%
%\iffalse
%<*samplepart3>
%\fi
% Some text for part 3:
%    \begin{macrocode}
some text in part three
%    \end{macrocode}

%\iffalse
%</samplepart3>
%\fi
% Some text for part 4:
%\iffalse
%<*samplepart4>
%\fi
%    \begin{macrocode}
more text in part four
%    \end{macrocode}

%\iffalse
%</samplepart4>
%\fi
%
% %%%%%%%%%%%%%%%%%%%%%%%%%%%%%%%%%%%%%%
% \paragraph{Forwarding for a Complete Draft.}
%
% The following forwarding file |cdocsdrf.tex|
% compiles the main document in draft mode:
%\iffalse
%<*sampledraft>
%\fi
%    \begin{macrocode}
\def\version{draft}
\input{childdoc.def}
\childdocforward{cdocsamp}
%    \end{macrocode}

%\iffalse
%</sampledraft>
%\fi
%
% %%%%%%%%%%%%%%%%%%%%%%%%%%%%%%%%%%%%%%
% \paragraph{Forwarding for Final Version of the Chapters.}
%
% The following forwarding files |cdocsfn1.tex| and |cdocsfn2.tex|
% (with identical content)
% compile the final versions of the child documents
% |cdocsch1.tex| and |cdocsch2.tex|, respectively:
%\iffalse
%<*samplefinal>
%\fi
%    \begin{macrocode}
\def\version{final}
\input{childdoc.def}
\childdocforwardprefix[cdocsamp]{cdocsfn}{cdocsch}
%    \end{macrocode}

%\iffalse
%</samplefinal>
%\fi
%
% %%%%%%%%%%%%%%%%%%%%%%%%%%%%%%%%%%%%%%
% \paragraph{Command Line Processing.}
%
% The following three command lines generate the output files
% |cdocscld|, |cdocscl1| and |cdocscl2|
% which should be identical to
% |cdocsdrf|, |cdocsch1| and |cdocsfn2|, respectively:
% \begin{center}
% \begin{tabular}{l}
% |latex -jobname cdocscld \|\\
% |  "\def\version{draft}\input{childdoc.def}\childdocforward{cdocsamp}"|\\
% |latex -jobname cdocscl1 \|\\
% |  "\input{childdoc.def}\childdocforward[cdocsamp]{cdocsch1}"|\\
% |latex -jobname cdocscl2 \|\\
% |  "\def\version{final}\input{childdoc.def}\childdocforward{cdocsch2}"|
% \end{tabular}
% \end{center}
% Note that the trailing backslash on each first line
% merely continues the input to the second line
% (for convenient cut ant paste).
% Furthermore, the command |latex| can be replaced by any
% of its alternative versions such as |pdflatex|.
%
% %%%%%%%%%%%%%%%%%%%%%%%%%%%%%%%%%%%%%%%%%%%%%%%%%%%%%%%%%%%%%%%%%%%%%%%%%%%%%%
% %%%%%%%%%%%%%%%%%%%%%%%%%%%%%%%%%%%%%%%%%%%%%%%%%%%%%%%%%%%%%%%%%%%%%%%%%%%%%%
% \section{Implementation}
%\iffalse
%<*package>
%\fi
%
% This section describes the definitions file |childdoc.def|.

% The definitions cannot be loaded using |\usepackage| or |\RequirePackage|
% which has a mechanism to prevent loading a style file more than once.
% When loading the definitions by means of |\input|
% multiple instances have to be prevented manually:
%\iffalse
%This code needs to be before the `\ProvidesFile' directive
%which is defined at the beginning of this file.
%Therefore it is also placed there and commented out here.
%</package>
%<*discard>
%\fi
%    \begin{macrocode}
\ifdefined\childdocmain\endinput\fi
%    \end{macrocode}
%\iffalse
%</discard>
%<*package>
%\fi
%
% \macro{\ifchilddoc}
% \macro{\ifchilddocmanual}
% The conditional |\ifchilddoc| tells whether a
% child (true) or main (false) document is being compiled.
% The conditional |\ifchilddocmanual| tells whether
% the |\includeonly| mechanism is used (false) or
% the selection of child files must be performed manually (true).
% The definitions initialise to false:
%    \begin{macrocode}
\newif\ifchilddoc
\newif\ifchilddocmanual
%    \end{macrocode}

% \macro{\childdocname}
% \macro{\childdocjob}
% The macro |\childdocname| stores the name of the main document
% to be compiled. The macro |\childdocjob| stores the name of
% the document on which the \LaTeX{} compiler was originally invoked.
% The content of |\jobname| cannot be compared
% to filenames specified in the source due to different catcodes.
% The following code rescans |\jobname|, stores the result
% in |\childdocname| and saves a copy in |\childdocjob|:
%    \begin{macrocode}
\edef\childdocname{\scantokens\expandafter{\jobname\noexpand}}
\let\childdocjob\childdocname
%    \end{macrocode}

% \macro{\childdocdisable}
% The macro |\childdocdisable| prevents the main file
% from being processed more than once.
% At this stage, the main document command |\childdocmain|
% is assumed to be called once again where it should do nothing.
% Any subsequent call to it should prevent
% a secondary processing of the main document
% It overwrites the forwarding commands
% |\childdocof| and |\childdocforward|
% with empty macros to prevent further inclusions of the main document:
%    \begin{macrocode}
\newcommand{\childdocdisable}
{
  \renewcommand{\childdocmain}[1]{\renewcommand{\childdocmain}[1]{\endinput}}
  \renewcommand{\childdocof}[1]{}
  \renewcommand{\childdocby}[2][]{}
  \renewcommand{\childdocforward}[2][]{}
  \renewcommand{\childdocdisable}{}
}
%    \end{macrocode}

% \macro{\childdocmain}
% The macro |\childdocmain| is to be called at the top of the main file
% with nothing or the main filename (without extension) as argument.
% First, it breaks loops.
% If the argument is not empty and does not match |\childdocname|
% (which is set by the first inclusion of |childdoc.def|),
% |\ifchilddoc| is set to true, |\includeonly| is applied to the child file
% and |\jobname| is set to the main file
% (for proper handling of |.aux| files):
%    \begin{macrocode}
\newcommand{\childdocmain}[1]
{
  \childdocdisable\childdocmain{}
  \if?#1?\else
    \begingroup
      \def\childdoctmp{#1}
      \ifx\childdoctmp\childdocname
        \def\childdoctmp{}
      \else
        \def\childdoctmp
        {
          \childdoctrue
          \includeonly{\childdocname}
          \def\childdocjob{#1}
          \def\jobname{#1}
        }
      \fi
      \expandafter
    \endgroup
    \childdoctmp
  \fi
}
%    \end{macrocode}

% \macro{\childdocof}
% The command |\childdocof| redirects
% compilation to the main file |#1|.
%    \begin{macrocode}
\newcommand{\childdocof}[1]
{
  \childdocdisable
  \childdoctrue
  \includeonly{\childdocname}
  \def\jobname{#1}
  \def\childdocjob{#1}
  \input{#1}
}
%    \end{macrocode}

% \macro{\childdocby}
% The command |\childdocby| ....
%    \begin{macrocode}
\newcommand{\childdocby}[2][]
{
  \childdocdisable
  \childdoctrue
  \childdocmanualtrue
  \if?#1?\else
    \def\jobname{#2}
  \fi
  \def\childdocjob{#2}
  \input{#2}
  \endinput
}
%    \end{macrocode}

% \macro{\childdocforward}
% The command |\childdocforward| redirects
% compilation to the main file or
% (if the optional argument is given) a child file.
% Parameters are set as if the main file
% or a child file starting with |\childdocof| was compiled.
% Then compilation is handed over to the main file:
%    \begin{macrocode}
\newcommand{\childdocforward}[2][]
{
  \begingroup
    \if?#1?
      \def\childdoctmp
      {
        \def\childdocname{#2}
        \def\childdocjob{#2}
        \def\jobname{#2}
        \input{#2}
        \endinput
      }
    \else
      \def\childdoctmp
      {
        \childdocdisable
        \def\childdocname{#2}
        \childdoctrue
        \includeonly{#2}
        \def\childdocjob{#1}
        \def\jobname{#1}
        \input{#1}
        \endinput
      }
    \fi
    \expandafter
  \endgroup
  \childdoctmp
}
%    \end{macrocode}

% \macro{\childdocforwardprefix}
% The command |\childdocforwardprefix| redirects
% compilation to the main or a child file by means of a pattern.
% The prefix |#1| in the current filename is replaced by |#2|
% and the suffix of the current filename is kept
% (it is assumed that the filename does not contain the substring `|~~~|'
% which is used as a delimiter).
% Compilation is handed over to the new file by |\childdocforward|:
%    \begin{macrocode}
\newcommand{\childdocforwardprefix}[3][]
{
  \begingroup
    \def\childdocextract #2##1~~~{\def\childdoctmp{\childdocforward[#1]{#3##1}}}
    \expandafter\childdocextract\childdocname~~~
    \expandafter
  \endgroup
  \childdoctmp
}
%    \end{macrocode}

% \macro{\childdoc}
% The deprecated macro |\childdoc| is a legacy version of |\childdocmain|:
%    \begin{macrocode}
\newcommand{\childdoc}{\childdocmain}
%    \end{macrocode}

% \macro{\childdocredirect}
% The deprecated macro |\childdocredirect| is a legacy version
% of |\childdocforward| and |\childdocforwardprefix|:
%    \begin{macrocode}
\newcommand{\childdocredirect}[2][]
{
  \begingroup
    \if?#1?
      \def\childdoctmp{\childdocforward{#2}}
    \else
      \def\childdoctmp{\childdocforwardprefix{#1}{#2}}
    \fi
    \expandafter
  \endgroup
  \childdoctmp
}
%    \end{macrocode}

%\iffalse
%</package>
%\fi
%
\endinput
|\\
|\childdocforwardprefix{final}{child}|
\end{tabular}
\end{center}
%

Note that when several versions of a main file and/or of each child file
are to be generated, it may be convenient to set up a |Makefile| or
shell script to automatise the process.

%%%%%%%%%%%%%%%%%%%%%%%%%%%%%%%%%%%%%%%%%%%%%%%%%%%%%%%%%%%%%%%%%%%%%%%%%%%%%%%%
\subsection{Command Line Processing}
\label{sec:commandline}

The effect of redirection files can also be achieved by invoking
the \LaTeX{} compiler with a more elaborate command line.
Most conveniently this should be done as part
of a shell script or a |Makefile|.

When using \textsf{childdoc} in the main file, the following
command lines effectively perform a redirection
(note that depending on the shell being used,
backslashes may have to be doubled: `|\|' $\to$ `|\\|'):
%
\begin{center}
|... -jobname "|\textit{target}|" |\\|"|[\textit{flags}]%
|% \iffalse
%
% childdoc.dtx Copyright (C) 2017-2018 Niklas Beisert
%
% This work may be distributed and/or modified under the
% conditions of the LaTeX Project Public License, either version 1.3
% of this license or (at your option) any later version.
% The latest version of this license is in
%   http://www.latex-project.org/lppl.txt
% and version 1.3 or later is part of all distributions of LaTeX
% version 2005/12/01 or later.
%
% This work has the LPPL maintenance status `maintained'.
%
% The Current Maintainer of this work is Niklas Beisert.
%
% This work consists of the files childdoc.dtx and childdoc.ins
% and the derived files childdoc.def and cdocsamp.tex with
% cdocsch1.tex, cdocsch2.tex, cdocsdrf.tex, cdocsfn1.tex, cdocsfn2.tex.
%
%<package>\ifdefined\childdocmain\endinput\fi
%<package>\ProvidesFile{childdoc.def}[2018/12/30 v2.0 child document driver]
%<samplemain>\ProvidesFile{cdocsamp.tex}[2018/12/30 v2.0 sample for childdoc]
%<*driver>
%\ProvidesFile{childdoc.drv}[2018/12/30 v2.0 childdoc reference manual file]
\PassOptionsToClass{10pt,a4paper}{article}
\documentclass{ltxdoc}

\usepackage[margin=35mm]{geometry}
\usepackage{hyperref}
\usepackage{hyperxmp}
\usepackage[usenames]{color}

\hypersetup{colorlinks=true}
\hypersetup{pdfstartview=FitH}
\hypersetup{pdfpagemode=UseNone}
\hypersetup{pdfsource={}}
\hypersetup{pdflang={en-UK}}
\hypersetup{pdfcopyright={Copyright 2017-2018 Niklas Beisert.
  This work may be distributed and/or modified under the
  conditions of the LaTeX Project Public License, either version 1.3
  of this license or (at your option) any later version.}}
\hypersetup{pdflicenseurl={http://www.latex-project.org/lppl.txt}}
\hypersetup{pdfcontactaddress={ETH Zurich, ITP, HIT K,
  Wolfgang-Pauli-Strasse 27}}
\hypersetup{pdfcontactpostcode={8093}}
\hypersetup{pdfcontactcity={Zurich}}
\hypersetup{pdfcontactcountry={Switzerland}}
\hypersetup{pdfcontactemail={nbeisert@itp.phys.ethz.ch}}
\hypersetup{pdfcontacturl={http://people.phys.ethz.ch/\xmptilde nbeisert/}}

\newcommand{\secref}[1]{\hyperref[#1]{section \ref*{#1}}}

\parskip1ex
\parindent0pt
\let\olditemize\itemize
\def\itemize{\olditemize\parskip0pt}

\begin{document}

\title{The \textsf{childdoc} Package}
\hypersetup{pdftitle={The childdoc Package}}
\author{Niklas Beisert\\[2ex]
  Institut f\"ur Theoretische Physik\\
  Eidgen\"ossische Technische Hochschule Z\"urich\\
  Wolfgang-Pauli-Strasse 27, 8093 Z\"urich, Switzerland\\[1ex]
  \href{mailto:nbeisert@itp.phys.ethz.ch}
  {\texttt{nbeisert@itp.phys.ethz.ch}}}
\hypersetup{pdfauthor={Niklas Beisert}}
\hypersetup{pdfsubject={Manual for the LaTeX2e Package childdoc}}
\date{30 December 2018, \textsf{v2.0}}
\maketitle

\begin{abstract}\noindent
\textsf{childdoc} is a \LaTeXe{} package
that enables the direct compilation
of document sections included by |\include|
to individual files.
\end{abstract}

\begingroup
\parskip0ex
\tableofcontents
\endgroup

%%%%%%%%%%%%%%%%%%%%%%%%%%%%%%%%%%%%%%%%%%%%%%%%%%%%%%%%%%%%%%%%%%%%%%%%%%%%%%%%
%%%%%%%%%%%%%%%%%%%%%%%%%%%%%%%%%%%%%%%%%%%%%%%%%%%%%%%%%%%%%%%%%%%%%%%%%%%%%%%%
\section{Introduction}

\LaTeX{} provides a mechanism to structure a large document (such as a book)
into a main file and several child files (containing the chapters)
using the |\include| command.
This mechanism is beneficial for documents
which span hundreds of pages in order to
make the source file(s) more manageable.
Moreover, compilation can be restricted to
selected child files by means of the |\includeonly| command.
The latter feature can be used to reduce the compilation time while editing
(this was significantly more useful in the earlier days of \LaTeX{})
or to generate a smaller document which is easier to navigate.
Another application of |\includeonly| is to generate
documents consisting of selected parts of the complete document.

However, there are a few drawbacks of the plain |\include| mechanism:
\begin{itemize}
\item
The child files cannot be compiled on their own,
they can only be compiled via the main file.
A naive editing environment
(such as a text editor with an option
to have the current file processed by \LaTeX)
may require one to switch to the main file before compiling;
attempting to compile the child file produces errors.
\item
The main file must be modified (each time)
to adjust the |\includeonly| command
to the present needs. This easily leaves the main file in a messy state.
\item
The generated document will always carry the filename
of the main document. This is inconvenient if
several child files are to be compiled and
to be kept for distribution.
\end{itemize}

The present package provides a simple interface
to make child files individually compilable by \LaTeX{}.
Compiling a child file then has the same effect as compiling
the main file with an |\includeonly| command
to select the appropriate child.
Moreover the generated document will carry the name of the child
rather than the main file.
This resolves all three above issues.

This feature is meant to make the editing of books,
thesis documents and lecture notes somewhat more convenient.
However, the package can also be used efficiently for
composing a series of documents (such as exercise sheets)
which are typically distributed individually.
It then assists the author in generating the individual documents
(potentially in different versions)
as well as a document containing the collected series.
Another application is in developing style files
or other kinds of included material
where compilation of the style file could redirect
to a sample or test file.

%%%%%%%%%%%%%%%%%%%%%%%%%%%%%%%%%%%%%%%%%%%%%%%%%%%%%%%%%%%%%%%%%%%%%%%%%%%%%%%%
%%%%%%%%%%%%%%%%%%%%%%%%%%%%%%%%%%%%%%%%%%%%%%%%%%%%%%%%%%%%%%%%%%%%%%%%%%%%%%%%
\section{Usage}

First of all, the package \textsf{childdoc} is \emph{not} a standard
\LaTeXe{} |.sty| style file! Therefore it needs to be invoked in
a non-standard way.

%%%%%%%%%%%%%%%%%%%%%%%%%%%%%%%%%%%%%%%%%%%%%%%%%%%%%%%%%%%%%%%%%%%%%%%%%%%%%%%%
\subsection{Included Files}
\label{sec:include}

%%%%%%%%%%%%%%%%%%%%%%%%%%%%%%%%%%%%%%%%
\DescribeMacro{\childdocmain}
To use the package, add the commands
\begin{center}
\begin{tabular}{l}
|\input{childdoc.def}|\\
|\childdocmain{}|\\
\end{tabular}
\end{center}
at the very top of the main \LaTeX{} file,
in particular \emph{before} the |\documentclass| statement!
The argument of |\childdocmain| should be left empty
(but it must be present).

%%%%%%%%%%%%%%%%%%%%%%%%%%%%%%%%%%%%%%%%
\DescribeMacro{\childdocof}
Furthermore, add the commands
\begin{center}
\begin{tabular}{l}
|\input{childdoc.def}|\\
|\childdocof{|\textit{main}|}|\\
\end{tabular}
\end{center}
at the top of every child file \textit{child}
which is included by |\include{|\textit{child}|}|
from within the main file
(or at least for those files to be compiled individually).
The argument \textit{main} must be the filename of the main file.

There are a couple of
considerations in setting up the main and child documents:

%%%%%%%%%%%%%%%%%%%%%%%%%%%%%%%%%%%%%%%%
\paragraph{Restrictions.}

Please note the following restrictions:
\begin{itemize}
\item
|\childdocmain| must be called with one argument \textit{main}
to ensure compatibility with earlier version of the package.
It must either be empty (|\childdocmain{}|)
or precisely match the filename of the main file in which it is specified.
See \secref{sec:detection} for further information.
\item
The filename \textit{main} must be specified without the |.tex| extension.
\item
The filename \textit{main} is case sensitive
(even in case-insensitive file systems)
due to internal string comparison.
\item
The argument \textit{main} should be fully expanded, it cannot be a macro.
\item
Subdirectories and special characters should be avoided in filenames.
\item
The command |\childdocmain{|\textit{main}|}| must be followed by a whitespace.
It should not be followed immediately by another command
or by a comment mark `|%|'.
This is because the \TeX{} parser reads the token immediately following
the argument of |\childdocmain| and puts it
at the beginning of every child section;
however, a white\-space is ignored.
\end{itemize}

%%%%%%%%%%%%%%%%%%%%%%%%%%%%%%%%%%%%%%%%
\paragraph{Content of Main File.}

It is advisable to place all content in the child files included by |\include|.
Any output contained in the main file will appear in all child documents
unless suppressed manually;
it cannot be suppressed automatically by the |\includeonly| directive
and thus should normally be avoided.
A method to include some content in the main file
by means of conditional processing is described in \secref{sec:conditional}.

%%%%%%%%%%%%%%%%%%%%%%%%%%%%%%%%%%%%%%%%
\paragraph{Page Numbering.}

When only a part of the document is compiled,
the appropriate numbering of pages
(as well as other status parameters)
is determined from the |.aux| files.
The latter contain information from previous passes.
However this information needs to propagate through
all intermediate child documents.
Therefore the page numbering in child documents may well
be inconsistent until the complete document is compiled at least once.

A useful (if unconventional) way to always ensure a consistent
page numbering is to restart the numbering in each child document
and denote the pages by `\textit{child}|.|\textit{page}'
where \textit{child} represents the chapter/section number of the child file.
This can be achieved by the command
|\numberwithin{page}{|\textit{child}|}|
of the \textsf{amsmath} package
where \textit{child} can be |chapter| or |section|
depending on the chosen structuring.
Alternatively, one can modify the macro |\thepage| appropriately
and reset the counter |page| at the start of each child file.

%%%%%%%%%%%%%%%%%%%%%%%%%%%%%%%%%%%%%%%%%%%%%%%%%%%%%%%%%%%%%%%%%%%%%%%%%%%%%%%%
\subsection{Conditional Processing}
\label{sec:conditional}

The package provides a mechanism to compile different versions
of a document. To customise the versions further some conditional processing
can come in handy to distinguish which version is being compiled.
The package provides two macros to describe the compilation context:

%%%%%%%%%%%%%%%%%%%%%%%%%%%%%%%%%%%%%%%%
\DescribeMacro{\ifchilddoc}
The conditional |\ifchilddoc| distinguishes between the compilation of
child documents and the main document:
%
\begin{center}
|\ifchilddoc |\textit{child-code}| |[|\||else |\textit{main-code}]| \||fi|
\end{center}

%%%%%%%%%%%%%%%%%%%%%%%%%%%%%%%%%%%%%%%%
\DescribeMacro{\childdocname}
\DescribeMacro{\childdocjob}
The macro |\childdocname| contains the filename (without extension)
of the main or child file being processed.
Note that |\childdocjob| will always contain the name of the main file.

%%%%%%%%%%%%%%%%%%%%%%%%%%%%%%%%%%%%%%%%
\paragraph{Title Page.}

Conditional processing can be used to include a title or banner page
in the main document when proper precautions are taken.
Importantly, the code in the main file should ensure that the page counter
(as well as other status parameters which are stored in the |.aux| files)
takes the same value after the conditional processing.
Otherwise the page numbers may take divergent values
depending on which part is compiled.

For example, a title page could be declared by:
%
\begin{center}
\begin{tabular}{l}
|\ifchilddoc\||else|\\
|\addtocounter{page}{-1}|\\
\textit{code for title page}\\
|\newpage|\\
|\||fi|
\end{tabular}
\end{center}
%
A banner page for the child documents can be generated by:
%
\begin{center}
\begin{tabular}{l}
|\ifchilddoc|\\
|\addtocounter{page}{-1}|\\
\textit{code for banner page}\\
|\newpage|\\
|\||fi|
\end{tabular}
\end{center}
%
Here one could write a message such as:
\begin{center}
|This is the part \childdocname{} of \childdocjob{}.|
\end{center}

%%%%%%%%%%%%%%%%%%%%%%%%%%%%%%%%%%%%%%%%%%%%%%%%%%%%%%%%%%%%%%%%%%%%%%%%%%%%%%%%
\subsection{Flags}
\label{sec:flags}

The package makes it easy to generate different versions
of the main or child documents.
To this end compilation flags can be defined
and assigned different default values.
They will be particularly useful in conjunction
with the forwarding mechanism described in \secref{sec:forward}.

For example, it may be useful to have a flag |\version|
which can be set to |draft| or |final|.
The document source will contain some conditional code
depending on the value of |\version|.
Suppose further, the flag should default to |final| for the main file
and to |draft| for child files
which is a natural assignment for editing the document.
This is achieved by placing the following code
in the preamble of the main document
(below the |\childdocmain| directive):
%
\begin{center}
\begin{tabular}{l}
|\ifchilddoc|\\
|\providecommand{\version}{draft}|\\
|\||else|\\
|\providecommand{\version}{final}|\\
|\||fi|
\end{tabular}
\end{center}
%
The definition by |\providecommand| makes sure
that previous definitions are not overwritten.
Further statements |\providecommand{\version}{...}|
can thus be added before the above code to override it.

For the main file, one might add a line
(between |\childdocmain| and the above block)
%
\begin{center}
|%\ifchilddoc\||else\providecommand{\version}{draft}\||fi|
\end{center}
%
which can be uncommented to produce a draft version.
Likewise one can add a line to the very top of a child file
(above the |\childdocof{|\textit{main}|}| directive)
%
\begin{center}
|%\providecommand{\version}{final}|
\end{center}
%
which can be uncommented to produce the final version of this child document.

%%%%%%%%%%%%%%%%%%%%%%%%%%%%%%%%%%%%%%%%%%%%%%%%%%%%%%%%%%%%%%%%%%%%%%%%%%%%%%%%
\subsection{Forwarding}
\label{sec:forward}

Different versions of the main or child documents
using compilation flags as described in \secref{sec:flags}
can be (permanently) stored in different files
for convenient compilation, viewing and distribution.
To this end, the package defines a command
to pass on compilation to a different file:

%%%%%%%%%%%%%%%%%%%%%%%%%%%%%%%%%%%%%%%%
\DescribeMacro{\childdocforward}
The command |\childdocforward| redirects processing to
another source file:
%
\begin{center}
\begin{tabular}{l}
|\input{childdoc.def}|\\
|\childdocforward[|\textit{main}|]{|\textit{dest}|}|\\
\end{tabular}
\end{center}
%
The argument \textit{dest} is the destination file
(without extension).
It should be the main file or one of the child files.
Note that further \textsf{childdoc} directives
such as |\childdocof| and |\childdocforward|
in the indicated file will be processed in this form.
The optional argument \textit{main}
passes on directly to the main file \textit{main}
while pretending to compile the child \textit{dest}.
This form behaves as if \textit{dest}
issues |\childdocof{|\textit{main}|}| right away,
and no further \textsf{childdoc} directives will be processed.

%%%%%%%%%%%%%%%%%%%%%%%%%%%%%%%%%%%%%%%%
\DescribeMacro{\...prefix}
In the alternative form |\childdocforwardprefix|,
%
\begin{center}
\begin{tabular}{l}
|\input{childdoc.def}|\\
|\childdocforwardprefix[|\textit{main}|]{|\textit{prefix}|}{|\textit{dest}|}|
\end{tabular}
\end{center}
%
the destination file is determined by a pattern
depending on the current file:
To make this work, the current file must be called
`{\textit{prefix}\hspace{0.2em}\textit{suffix}}'
with \textit{prefix} matching precisely the argument.
Processing is then passed on to the file
`{\textit{dest}\hspace{0.2em}\textit{suffix}}'.
Surely, the same effect is achieved by
directly specifying the
argument `{\textit{dest}\hspace{0.2em}\textit{suffix}}'
in the first form.
However, that requires to set up a different file
for each child. With the alternative form of the command
all these files can have exactly the same content
which simplifies setting them up and maintaining them.

For example, the following file |draft.tex|
with a compilation flag |\version| as described in \secref{sec:flags}
compiles the main document as a draft:
%
\begin{center}
\begin{tabular}{l}
|\def\version{draft}|\\
|\input{childdoc.def}|\\
|\childdocforward{|\textit{main}|}|
\end{tabular}
\end{center}
%
Likewise, the following files |final|\textit{nn}|.tex|
compile the final version of the child document
|child|\textit{nn}|.tex|:
%
\begin{center}
\begin{tabular}{l}
|\def\version{final}|\\
|\input{childdoc.def}|\\
|\childdocforwardprefix{final}{child}|
\end{tabular}
\end{center}
%

Note that when several versions of a main file and/or of each child file
are to be generated, it may be convenient to set up a |Makefile| or
shell script to automatise the process.

%%%%%%%%%%%%%%%%%%%%%%%%%%%%%%%%%%%%%%%%%%%%%%%%%%%%%%%%%%%%%%%%%%%%%%%%%%%%%%%%
\subsection{Command Line Processing}
\label{sec:commandline}

The effect of redirection files can also be achieved by invoking
the \LaTeX{} compiler with a more elaborate command line.
Most conveniently this should be done as part
of a shell script or a |Makefile|.

When using \textsf{childdoc} in the main file, the following
command lines effectively perform a redirection
(note that depending on the shell being used,
backslashes may have to be doubled: `|\|' $\to$ `|\\|'):
%
\begin{center}
|... -jobname "|\textit{target}|" |\\|"|[\textit{flags}]%
|\input{childdoc.def}\childdocforward[|\textit{main}|]{|\textit{dest}|}"|
\end{center}
%
Here \textit{target} is the name of the output file,
\textit{main} is the name of the main file
and \textit{dest} is the name of the main or child file to be processed
(all filenames without extensions).
The optional argument \textit{main} can be omitted
if \textit{main} matches \textit{dest}.
Optionally, compilation \textit{flags} can be defined via |\def| commands.
This command line makes the \TeX{} engine believe
it is compiling the file \textit{target}
whose content is specified as the latter parameter.
The provided code then forwards the processing to
\textit{main} or \textit{dest} as described in \secref{sec:forward}.

%%%%%%%%%%%%%%%%%%%%%%%%%%%%%%%%%%%%%%%%%%%%%%%%%%%%%%%%%%%%%%%%%%%%%%%%%%%%%%%%
\subsection{Include by Input}
\label{sec:input}

Including child documents by |\include| has some restrictions by design.
Most notably, the content of a child document always occupies
its own set of pages; pages cannot be shared between child documents.
Usually, this behaviour makes perfect sense
because each child document contain an essential part of the document.
However, in some situations it may be desirable to compose
a document from a collection of parts
without having mandatory page breaks between then.
For this case, the package
provides a mechanism to include parts
by |\input| which can also be processed individually.
However, by construction this mechanism
requires manual handling of the content to be output.

%%%%%%%%%%%%%%%%%%%%%%%%%%%%%%%%%%%%%%%%
\DescribeMacro{\ifchilddocmanual}
The main file should be prepared as usual, see \secref{sec:include}.
However, the document body must make a distinction
between processing of an individual part and of the main document, e.g.:
%
\begin{center}
\begin{tabular}{l}
|\ifchilddocmanual|\\
|\input{\childdocname}|\\
|\||else|\\
\textit{document body with }|\input{|\textit{part}|}|\\
|\||fi|
\end{tabular}
\end{center}
%
The conditional |\ifchilddocmanual| is true whenever
a part to be included by |\input| is being compiled,
and the name of the part is stored in |\childdocname|.

%%%%%%%%%%%%%%%%%%%%%%%%%%%%%%%%%%%%%%%%
\DescribeMacro{\childdocby}
Each part to be included by |\input| should start with:
%
\begin{center}
\begin{tabular}{l}
|\input{childdoc.def}|\\
|\childdocby{|\textit{main}|}|\\
\end{tabular}
\end{center}
%
The directive |\childdocby| is similar to |\childdocof|
described in \secref{sec:include},
but the subsequent selection of content must be done manually.
To that end, both |\ifchilddoc| and |\ifchilddocmanual|
will be true upon processing of a part,
and the name of the part is stored in |\childdocname|.
Note that |\jobname| will be set to the filename of the current part
so that each part receives an individual |.aux| file
that does not interfere with the |.aux| file(s) of the main document.
This behaviour can be altered by the alternative form
|\childdocby[*]{|\textit{main}|}| (with a non-empty optional argument)
which uses the |.aux| file of the main document
by setting |\jobname| to \textit{main}.

%%%%%%%%%%%%%%%%%%%%%%%%%%%%%%%%%%%%%%%%%%%%%%%%%%%%%%%%%%%%%%%%%%%%%%%%%%%%%%%%
\subsection{Driver Development}
\label{sec:driver}

The \textsf{childdoc} mechanism can also be use for the development
of definition files such as \LaTeX{} styles or classes.
This case differs from the above setup with multiple parts
included by |\include| in that no |\includeonly| should be invoked.
This can be achieved by starting the include file
(before |\ProvidesPackage|) with:
%
\begin{center}
\begin{tabular}{l}
|\input{childdoc.def}|\\
|\childdocforward{|\textit{main}|}|\\
\end{tabular}
\end{center}
%
or alternatively with:
%
\begin{center}
\begin{tabular}{l}
|\input{childdoc.def}|\\
|\childdocby{|\textit{main}|}|\\
\end{tabular}
\end{center}
%
Both forms have slightly different effects as described above.
The main file is prepared as usual, see \secref{sec:include}.

%%%%%%%%%%%%%%%%%%%%%%%%%%%%%%%%%%%%%%%%%%%%%%%%%%%%%%%%%%%%%%%%%%%%%%%%%%%%%%%%
\subsection{Legacy Detection}
\label{sec:detection}

The directive |\childdocmain| in the main file can detect
whether the complete document or merely a child is to be compiled
even without using the directive |\childdocof|.
This method is deprecated because it is less robust
and there is no compelling reason to use it;
it is merely provided for backward compatibility
and it may be removed in future versions.

If the detection mechanism is to be used,
it is mandatory to correctly specify
the filename of the main file as the argument of |\childdocmain|:
%
\begin{center}
\begin{tabular}{l}
|\input{childdoc.def}|\\
|\childdocmain{|\textit{main}|}|\\
\end{tabular}
\end{center}
%
If |\jobname| does not match the argument \textit{main} of |\childdocmain|,
it is assumed that |\jobname| points to the child file to be compiled.
When using |\childdocmain| with the main file specified as argument,
it suffices to start a child file
with just |\input{|\textit{main}|}|
without loading of the package and using |\childdocof|.
If instead all processing is done
with the appropriate \textsf{childdoc} directives,
the argument of \textit{main} of |\childdocmain| can be empty.

An alternative version of the command line processing described
in \secref{sec:commandline} using the detection mechanism reads:
%
\begin{center}
|... -jobname "|\textit{target}|" "|[\textit{flags}]%
[|\def\jobname{|\textit{dest}|}|]|\input{|\textit{main}|}"|
\end{center}

%%%%%%%%%%%%%%%%%%%%%%%%%%%%%%%%%%%%%%%%%%%%%%%%%%%%%%%%%%%%%%%%%%%%%%%%%%%%%%%%
\subsection{Manual Code}
\label{sec:manual}

In case one cannot be certain whether the definitions file |childdoc.def|
is installed on the target \TeX{} distribution
and one prefers not to ship it,
it is conceivable to paste a few relevant commands into the sources.

To that end, drop all statements |\input{childdoc.def}|
and perform the replacements as outlined below.
Instead of |\childdocmain{|\textit{main}|}| add the following code
to the top of the main file:
%
\begin{center}
\begin{tabular}{l}
|\||ifdefined\childdocname\endinput\||fi\newif\ifchilddoc|\\
|\edef\childdocname{\scantokens\expandafter{\jobname\noexpand}}|\\
|\def\childdocmain{|\textit{main}|}\||ifx\childdocmain\childdocname\||else|\\
|\childdoctrue\includeonly{\childdocname}\let\jobname\childdocmain\||fi|\\
\end{tabular}
\end{center}
%
Instead of |\childdocof{|\textit{main}|}| just include the main file
at the top of each child file:
%
\begin{center}
|\input{|\textit{main}|}|
\end{center}
%
A simple redirection |\childdocforward{|\textit{dest}|}| is achieved by:
%
\begin{center}
|\def\jobname{|\textit{dest}|}\input{\jobname}|
\end{center}
%
The redirection with prefix
|\childdocforwardprefix[|\textit{prefix}|]{|\textit{dest}|}|
is accomplished by:
%
\begin{center}
\begin{tabular}{l}
|{\edef\jobname{\scantokens\expandafter{\jobname\noexpand}}|\\
|\def\redirectjob |\textit{prefix}|#1~~~{\gdef\jobname{|\textit{dest}|#1}}|\\
|\expandafter\redirectjob\jobname~~~}\input{\jobname}|
\end{tabular}
\end{center}

In an alternative approach,
child documents can be compiled by a specific command line
without additional code or specific definitions:
%
\begin{center}
|... -jobname "|\textit{target}|" "|[\textit{flags}]%
|\includeonly{|\textit{dest}|}\input{|\textit{main}|}"|
\end{center}
%

%%%%%%%%%%%%%%%%%%%%%%%%%%%%%%%%%%%%%%%%%%%%%%%%%%%%%%%%%%%%%%%%%%%%%%%%%%%%%%%%
%%%%%%%%%%%%%%%%%%%%%%%%%%%%%%%%%%%%%%%%%%%%%%%%%%%%%%%%%%%%%%%%%%%%%%%%%%%%%%%%
\section{Information}

%%%%%%%%%%%%%%%%%%%%%%%%%%%%%%%%%%%%%%%%%%%%%%%%%%%%%%%%%%%%%%%%%%%%%%%%%%%%%%%%
\subsection{Copyright}

Copyright \copyright{} 2017--2018 Niklas Beisert

This work may be distributed and/or modified under the
conditions of the \LaTeX{} Project Public License, either version 1.3
of this license or (at your option) any later version.
The latest version of this license is in
  \url{http://www.latex-project.org/lppl.txt}
and version 1.3 or later is part of all distributions of \LaTeX{}
version 2005/12/01 or later.

This work has the LPPL maintenance status `maintained'.

The Current Maintainer of this work is Niklas Beisert.

This work consists of the files |README.txt|, |childdoc.ins| and |childdoc.dtx|
as well as the derived files |childdoc.def|, |cdocsamp.tex|
with |cdocsch1.tex|, |cdocsch2.tex|, |cdocspt3.tex|, |cdocspt4.tex|,
|cdocsdrf.tex|, |cdocsfn1.tex|, |cdocsfn2.tex|
as well as |childdoc.pdf|.

%%%%%%%%%%%%%%%%%%%%%%%%%%%%%%%%%%%%%%%%%%%%%%%%%%%%%%%%%%%%%%%%%%%%%%%%%%%%%%%%
\subsection{Files and Installation}

The package consists of the files:
%
\begin{center}
\begin{tabular}{ll}
    |README.txt|   & readme file \\
    |childdoc.ins| & installation file \\
    |childdoc.dtx| & source file \\
    |childdoc.def| & definition file \\
    |cdocsamp.tex| & sample main file \\
    |cdocsch1.tex| & sample include file \\
    |cdocsch2.tex| & sample include file \\
    |cdocspt3.tex| & sample part file \\
    |cdocspt4.tex| & sample part file \\
    |cdocsdrf.tex| & sample redirection file \\
    |cdocsfn1.tex| & sample redirection file \\
    |cdocsfn2.tex| & sample redirection file \\
    |childdoc.pdf| & manual
\end{tabular}
\end{center}
%
The distribution consists of the files
|README.txt|, |childdoc.ins| and |childdoc.dtx|.
%
\begin{itemize}
\item
Run (pdf)\LaTeX{} on |childdoc.dtx|
to compile the manual |childdoc.pdf| (this file).
\item
Run \LaTeX{} on |childdoc.ins| to create the definitions file |childdoc.def|
and the sample |cdocsamp.tex| with include files
|cdocsch1.tex|, |cdocsch2.tex|, |cdocspt3.tex|, |cdocspt4.tex|,
|cdocsdrf.tex|, |cdocsfn1.tex|, |cdocsfn2.tex|.
Then copy the file |childdoc.def| to an appropriate directory of your \LaTeX{}
distribution, e.g.\ \textit{texmf-root}|/tex/latex/childdoc|.
\end{itemize}

%%%%%%%%%%%%%%%%%%%%%%%%%%%%%%%%%%%%%%%%%%%%%%%%%%%%%%%%%%%%%%%%%%%%%%%%%%%%%%%%
\subsection{Related CTAN Packages}

There are several other packages which offer a similar functionality:
%
\begin{itemize}
\item
The packages
\href{http://ctan.org/pkg/docmute}{\textsf{docmute}},
\href{http://ctan.org/pkg/includex}{\textsf{includex}} and
\href{http://ctan.org/pkg/standalone}{\textsf{standalone}}
provide commands to include only the document body of
a child file thus allowing both files to be compiled individually.
\item
The packages \href{http://ctan.org/pkg/subdocs}{\textsf{subdocs}}
and \href{http://ctan.org/pkg/subfiles}{\textsf{subfiles}}
provide structures in which the main and child documents can be
encapsulated and allowing them to be compiled individually.
The inclusion mechanism is different from the conventional |\include|.
\item
The package \href{http://ctan.org/pkg/combine}{\textsf{combine}}
is an elaborate solution to combine several documents into one.
\end{itemize}
%
See also the CTAN topic \href{http://ctan.org/topic/subdocs}{\textsf{subdocs}}
for further related packages.
The present package differs from the above solutions in that
a document structure constructed with the conventional |\include| mechanism
just needs two extra commands at the top of every file
such that all constituent files can be compiled individually.

%%%%%%%%%%%%%%%%%%%%%%%%%%%%%%%%%%%%%%%%%%%%%%%%%%%%%%%%%%%%%%%%%%%%%%%%%%%%%%%%
%\subsection{Feature Suggestions}
%
%The following is a list of features which may be useful for future
%versions of this package:
%%
%\begin{itemize}
%\item
%\ldots
%\end{itemize}

%%%%%%%%%%%%%%%%%%%%%%%%%%%%%%%%%%%%%%%%%%%%%%%%%%%%%%%%%%%%%%%%%%%%%%%%%%%%%%%%
\subsection{Revision History}

%%%%%%%%%%%%%%%%%%%%%%%%%%%%%%%%%%%%%%%%
\paragraph{v2.0:} 2018/12/30

\begin{itemize}
\item
immediate forward processing
\item
added |\childdocby| mechanism
\item
manual restructured
\end{itemize}

%%%%%%%%%%%%%%%%%%%%%%%%%%%%%%%%%%%%%%%%
\paragraph{v1.6:} 2018/01/17

\begin{itemize}
\item
application for development of include files
\item
corrections to manual
\end{itemize}

%%%%%%%%%%%%%%%%%%%%%%%%%%%%%%%%%%%%%%%%
\paragraph{v1.5:} 2017/05/21

\begin{itemize}
\item
more complete structuring introduced
\item
|\childdocof| introduced
\item
|\childdoc| renamed to |\childdocmain|
\item
|\childredirect| renamed to |\childdocforward| and |\childdocforwardprefix|
and functionality expanded
\end{itemize}

%%%%%%%%%%%%%%%%%%%%%%%%%%%%%%%%%%%%%%%%
\paragraph{v1.0:} 2017/04/27

\begin{itemize}
\item
manual and install package
\item
first version published on CTAN
\end{itemize}

%%%%%%%%%%%%%%%%%%%%%%%%%%%%%%%%%%%%%%%%
\paragraph{v0.6:} 2017/04/26

\begin{itemize}
\item
redirection mechanism added
\end{itemize}

%%%%%%%%%%%%%%%%%%%%%%%%%%%%%%%%%%%%%%%%
\paragraph{v0.5:} 2017/04/26

\begin{itemize}
\item
functionality in definition file
\end{itemize}


%%%%%%%%%%%%%%%%%%%%%%%%%%%%%%%%%%%%%%%%%%%%%%%%%%%%%%%%%%%%%%%%%%%%%%%%%%%%%%%%
%%%%%%%%%%%%%%%%%%%%%%%%%%%%%%%%%%%%%%%%%%%%%%%%%%%%%%%%%%%%%%%%%%%%%%%%%%%%%%%%
%%%%%%%%%%%%%%%%%%%%%%%%%%%%%%%%%%%%%%%%%%%%%%%%%%%%%%%%%%%%%%%%%%%%%%%%%%%%%%%%
\appendix

\settowidth\MacroIndent{\rmfamily\scriptsize 000\ }

 \DocInput{childdoc.dtx}

\end{document}
%</driver>
% \fi
%
% %%%%%%%%%%%%%%%%%%%%%%%%%%%%%%%%%%%%%%%%%%%%%%%%%%%%%%%%%%%%%%%%%%%%%%%%%%%%%%
% %%%%%%%%%%%%%%%%%%%%%%%%%%%%%%%%%%%%%%%%%%%%%%%%%%%%%%%%%%%%%%%%%%%%%%%%%%%%%%
% \section{Sample}
%\iffalse
%<*samplemain>
%\fi
%
% The following presents a sample document
% with two chapters, two parts, a title page,
% a compile flag as well as three forwarding files to set the flag.
% It consists of eight |.tex| files:
% \begin{center}
% \begin{tabular}{ll}
% |cdocsamp.tex|&main file\\
% |cdocsch1.tex|&include file for chapter 1\\
% |cdocsch2.tex|&include file for chapter 2\\
% |cdocspt3.tex|&include file for part 3\\
% |cdocspt4.tex|&include file for part 4\\
% |cdocsdrf.tex|&forwarding file for main file in draft mode\\
% |cdocsfi1.tex|&forwarding file for final version of chapter 1\\
% |cdocsfi2.tex|&forwarding file for final version of chapter 2\\
% \end{tabular}
% \end{center}
% Each of the eight files can be compiled directly by the \LaTeX{} compiler.
%
% %%%%%%%%%%%%%%%%%%%%%%%%%%%%%%%%%%%%%%
% \paragraph{Main File.}
%
% The main file is called |cdocsamp.tex|.
%
% Load the \textsf{childdoc} definitions and
% declare the filename for the main document:
%    \begin{macrocode}
\input{childdoc.def}
\childdocmain{}
%    \end{macrocode}

% Optional override for |\version| flag:
%    \begin{macrocode}
%%\ifchilddoc\else\providecommand{\version}{draft}\fi
%    \end{macrocode}

% Define the default values for the |\version| flag
% (|final| for the main file and |draft| for childs):
%    \begin{macrocode}
\ifchilddoc
\providecommand{\version}{draft}
\else
\providecommand{\version}{final}
\fi
%    \end{macrocode}

% Load the standard document class:
%    \begin{macrocode}
\documentclass[12pt]{article}
%    \end{macrocode}

% Start the document body:
%    \begin{macrocode}
\begin{document}
%    \end{macrocode}

% Declare a title page.
% Print title, part of document being processed and version flag:
%    \begin{macrocode}
\addtocounter{page}{-1}
\begin{center}
{\LARGE\bfseries{}childdoc example\par}
\vspace{1cm}
\ifchilddoc
\ifchilddocmanual part\else chapter\fi:
`\childdocname' of `\childdocjob'\par
\else
main document: `\childdocjob'\par
\fi
version: \version\par
\end{center}
\newpage
%    \end{macrocode}

% Manually include selected file,
% otherwise process as usual:
%    \begin{macrocode}
\ifchilddocmanual
\section*{part `\childdocname'}
\input{\childdocname}
\else
%    \end{macrocode}

% Include the two chapters:
%    \begin{macrocode}
\include{cdocsch1}
\include{cdocsch2}
%    \end{macrocode}

% Include the two parts unless only chapters should be displayed:
%    \begin{macrocode}
\ifchilddoc\else
\section{part three}
\input{cdocspt3}
\section{part four}
\input{cdocspt4}
\fi
%    \end{macrocode}

% Process as usual until here:
%    \begin{macrocode}
\fi
%    \end{macrocode}

% End of document body:
%    \begin{macrocode}
\end{document}
%    \end{macrocode}
%\iffalse
%</samplemain>
%\fi
%
% %%%%%%%%%%%%%%%%%%%%%%%%%%%%%%%%%%%%%%
% \paragraph{Chapter Include Files.}
%
% The include files are called |cdocsch1.tex| and |cdocsch2.tex|.
%
%\iffalse
%<*samplechap1|samplechap2>
%\fi

% Optional override for |\version| flag:
%    \begin{macrocode}
%%\providecommand{\version}{final}
%    \end{macrocode}

% Include the main document:
%    \begin{macrocode}
\input{childdoc.def}
\childdocof{cdocsamp}
%    \end{macrocode}

%\iffalse
%</samplechap1|samplechap2>
%\fi
%
%\iffalse
%<*samplechap1>
%\fi
% Some text for chapter 1:
%    \begin{macrocode}
\section{one}
some text in chapter one
%    \end{macrocode}

%\iffalse
%</samplechap1>
%\fi
% Some text for chapter 2:
%\iffalse
%<*samplechap2>
%\fi
%    \begin{macrocode}
\section{two}
more text in chapter two
%    \end{macrocode}

%\iffalse
%</samplechap2>
%\fi
%
% %%%%%%%%%%%%%%%%%%%%%%%%%%%%%%%%%%%%%%
% \paragraph{Part Include Files.}
%
% The include files are called |cdocspt3.tex| and |cdocspt4.tex|.
%
%\iffalse
%<*samplepart3|samplepart4>
%\fi

% Optional override for |\version| flag:
%    \begin{macrocode}
%%\providecommand{\version}{final}
%    \end{macrocode}

% Include the main document:
%    \begin{macrocode}
\input{childdoc.def}
\childdocby{cdocsamp}
%    \end{macrocode}

%\iffalse
%</samplepart3|samplepart4>
%\fi
%
%\iffalse
%<*samplepart3>
%\fi
% Some text for part 3:
%    \begin{macrocode}
some text in part three
%    \end{macrocode}

%\iffalse
%</samplepart3>
%\fi
% Some text for part 4:
%\iffalse
%<*samplepart4>
%\fi
%    \begin{macrocode}
more text in part four
%    \end{macrocode}

%\iffalse
%</samplepart4>
%\fi
%
% %%%%%%%%%%%%%%%%%%%%%%%%%%%%%%%%%%%%%%
% \paragraph{Forwarding for a Complete Draft.}
%
% The following forwarding file |cdocsdrf.tex|
% compiles the main document in draft mode:
%\iffalse
%<*sampledraft>
%\fi
%    \begin{macrocode}
\def\version{draft}
\input{childdoc.def}
\childdocforward{cdocsamp}
%    \end{macrocode}

%\iffalse
%</sampledraft>
%\fi
%
% %%%%%%%%%%%%%%%%%%%%%%%%%%%%%%%%%%%%%%
% \paragraph{Forwarding for Final Version of the Chapters.}
%
% The following forwarding files |cdocsfn1.tex| and |cdocsfn2.tex|
% (with identical content)
% compile the final versions of the child documents
% |cdocsch1.tex| and |cdocsch2.tex|, respectively:
%\iffalse
%<*samplefinal>
%\fi
%    \begin{macrocode}
\def\version{final}
\input{childdoc.def}
\childdocforwardprefix[cdocsamp]{cdocsfn}{cdocsch}
%    \end{macrocode}

%\iffalse
%</samplefinal>
%\fi
%
% %%%%%%%%%%%%%%%%%%%%%%%%%%%%%%%%%%%%%%
% \paragraph{Command Line Processing.}
%
% The following three command lines generate the output files
% |cdocscld|, |cdocscl1| and |cdocscl2|
% which should be identical to
% |cdocsdrf|, |cdocsch1| and |cdocsfn2|, respectively:
% \begin{center}
% \begin{tabular}{l}
% |latex -jobname cdocscld \|\\
% |  "\def\version{draft}\input{childdoc.def}\childdocforward{cdocsamp}"|\\
% |latex -jobname cdocscl1 \|\\
% |  "\input{childdoc.def}\childdocforward[cdocsamp]{cdocsch1}"|\\
% |latex -jobname cdocscl2 \|\\
% |  "\def\version{final}\input{childdoc.def}\childdocforward{cdocsch2}"|
% \end{tabular}
% \end{center}
% Note that the trailing backslash on each first line
% merely continues the input to the second line
% (for convenient cut ant paste).
% Furthermore, the command |latex| can be replaced by any
% of its alternative versions such as |pdflatex|.
%
% %%%%%%%%%%%%%%%%%%%%%%%%%%%%%%%%%%%%%%%%%%%%%%%%%%%%%%%%%%%%%%%%%%%%%%%%%%%%%%
% %%%%%%%%%%%%%%%%%%%%%%%%%%%%%%%%%%%%%%%%%%%%%%%%%%%%%%%%%%%%%%%%%%%%%%%%%%%%%%
% \section{Implementation}
%\iffalse
%<*package>
%\fi
%
% This section describes the definitions file |childdoc.def|.

% The definitions cannot be loaded using |\usepackage| or |\RequirePackage|
% which has a mechanism to prevent loading a style file more than once.
% When loading the definitions by means of |\input|
% multiple instances have to be prevented manually:
%\iffalse
%This code needs to be before the `\ProvidesFile' directive
%which is defined at the beginning of this file.
%Therefore it is also placed there and commented out here.
%</package>
%<*discard>
%\fi
%    \begin{macrocode}
\ifdefined\childdocmain\endinput\fi
%    \end{macrocode}
%\iffalse
%</discard>
%<*package>
%\fi
%
% \macro{\ifchilddoc}
% \macro{\ifchilddocmanual}
% The conditional |\ifchilddoc| tells whether a
% child (true) or main (false) document is being compiled.
% The conditional |\ifchilddocmanual| tells whether
% the |\includeonly| mechanism is used (false) or
% the selection of child files must be performed manually (true).
% The definitions initialise to false:
%    \begin{macrocode}
\newif\ifchilddoc
\newif\ifchilddocmanual
%    \end{macrocode}

% \macro{\childdocname}
% \macro{\childdocjob}
% The macro |\childdocname| stores the name of the main document
% to be compiled. The macro |\childdocjob| stores the name of
% the document on which the \LaTeX{} compiler was originally invoked.
% The content of |\jobname| cannot be compared
% to filenames specified in the source due to different catcodes.
% The following code rescans |\jobname|, stores the result
% in |\childdocname| and saves a copy in |\childdocjob|:
%    \begin{macrocode}
\edef\childdocname{\scantokens\expandafter{\jobname\noexpand}}
\let\childdocjob\childdocname
%    \end{macrocode}

% \macro{\childdocdisable}
% The macro |\childdocdisable| prevents the main file
% from being processed more than once.
% At this stage, the main document command |\childdocmain|
% is assumed to be called once again where it should do nothing.
% Any subsequent call to it should prevent
% a secondary processing of the main document
% It overwrites the forwarding commands
% |\childdocof| and |\childdocforward|
% with empty macros to prevent further inclusions of the main document:
%    \begin{macrocode}
\newcommand{\childdocdisable}
{
  \renewcommand{\childdocmain}[1]{\renewcommand{\childdocmain}[1]{\endinput}}
  \renewcommand{\childdocof}[1]{}
  \renewcommand{\childdocby}[2][]{}
  \renewcommand{\childdocforward}[2][]{}
  \renewcommand{\childdocdisable}{}
}
%    \end{macrocode}

% \macro{\childdocmain}
% The macro |\childdocmain| is to be called at the top of the main file
% with nothing or the main filename (without extension) as argument.
% First, it breaks loops.
% If the argument is not empty and does not match |\childdocname|
% (which is set by the first inclusion of |childdoc.def|),
% |\ifchilddoc| is set to true, |\includeonly| is applied to the child file
% and |\jobname| is set to the main file
% (for proper handling of |.aux| files):
%    \begin{macrocode}
\newcommand{\childdocmain}[1]
{
  \childdocdisable\childdocmain{}
  \if?#1?\else
    \begingroup
      \def\childdoctmp{#1}
      \ifx\childdoctmp\childdocname
        \def\childdoctmp{}
      \else
        \def\childdoctmp
        {
          \childdoctrue
          \includeonly{\childdocname}
          \def\childdocjob{#1}
          \def\jobname{#1}
        }
      \fi
      \expandafter
    \endgroup
    \childdoctmp
  \fi
}
%    \end{macrocode}

% \macro{\childdocof}
% The command |\childdocof| redirects
% compilation to the main file |#1|.
%    \begin{macrocode}
\newcommand{\childdocof}[1]
{
  \childdocdisable
  \childdoctrue
  \includeonly{\childdocname}
  \def\jobname{#1}
  \def\childdocjob{#1}
  \input{#1}
}
%    \end{macrocode}

% \macro{\childdocby}
% The command |\childdocby| ....
%    \begin{macrocode}
\newcommand{\childdocby}[2][]
{
  \childdocdisable
  \childdoctrue
  \childdocmanualtrue
  \if?#1?\else
    \def\jobname{#2}
  \fi
  \def\childdocjob{#2}
  \input{#2}
  \endinput
}
%    \end{macrocode}

% \macro{\childdocforward}
% The command |\childdocforward| redirects
% compilation to the main file or
% (if the optional argument is given) a child file.
% Parameters are set as if the main file
% or a child file starting with |\childdocof| was compiled.
% Then compilation is handed over to the main file:
%    \begin{macrocode}
\newcommand{\childdocforward}[2][]
{
  \begingroup
    \if?#1?
      \def\childdoctmp
      {
        \def\childdocname{#2}
        \def\childdocjob{#2}
        \def\jobname{#2}
        \input{#2}
        \endinput
      }
    \else
      \def\childdoctmp
      {
        \childdocdisable
        \def\childdocname{#2}
        \childdoctrue
        \includeonly{#2}
        \def\childdocjob{#1}
        \def\jobname{#1}
        \input{#1}
        \endinput
      }
    \fi
    \expandafter
  \endgroup
  \childdoctmp
}
%    \end{macrocode}

% \macro{\childdocforwardprefix}
% The command |\childdocforwardprefix| redirects
% compilation to the main or a child file by means of a pattern.
% The prefix |#1| in the current filename is replaced by |#2|
% and the suffix of the current filename is kept
% (it is assumed that the filename does not contain the substring `|~~~|'
% which is used as a delimiter).
% Compilation is handed over to the new file by |\childdocforward|:
%    \begin{macrocode}
\newcommand{\childdocforwardprefix}[3][]
{
  \begingroup
    \def\childdocextract #2##1~~~{\def\childdoctmp{\childdocforward[#1]{#3##1}}}
    \expandafter\childdocextract\childdocname~~~
    \expandafter
  \endgroup
  \childdoctmp
}
%    \end{macrocode}

% \macro{\childdoc}
% The deprecated macro |\childdoc| is a legacy version of |\childdocmain|:
%    \begin{macrocode}
\newcommand{\childdoc}{\childdocmain}
%    \end{macrocode}

% \macro{\childdocredirect}
% The deprecated macro |\childdocredirect| is a legacy version
% of |\childdocforward| and |\childdocforwardprefix|:
%    \begin{macrocode}
\newcommand{\childdocredirect}[2][]
{
  \begingroup
    \if?#1?
      \def\childdoctmp{\childdocforward{#2}}
    \else
      \def\childdoctmp{\childdocforwardprefix{#1}{#2}}
    \fi
    \expandafter
  \endgroup
  \childdoctmp
}
%    \end{macrocode}

%\iffalse
%</package>
%\fi
%
\endinput
\childdocforward[|\textit{main}|]{|\textit{dest}|}"|
\end{center}
%
Here \textit{target} is the name of the output file,
\textit{main} is the name of the main file
and \textit{dest} is the name of the main or child file to be processed
(all filenames without extensions).
The optional argument \textit{main} can be omitted
if \textit{main} matches \textit{dest}.
Optionally, compilation \textit{flags} can be defined via |\def| commands.
This command line makes the \TeX{} engine believe
it is compiling the file \textit{target}
whose content is specified as the latter parameter.
The provided code then forwards the processing to
\textit{main} or \textit{dest} as described in \secref{sec:forward}.

%%%%%%%%%%%%%%%%%%%%%%%%%%%%%%%%%%%%%%%%%%%%%%%%%%%%%%%%%%%%%%%%%%%%%%%%%%%%%%%%
\subsection{Include by Input}
\label{sec:input}

Including child documents by |\include| has some restrictions by design.
Most notably, the content of a child document always occupies
its own set of pages; pages cannot be shared between child documents.
Usually, this behaviour makes perfect sense
because each child document contain an essential part of the document.
However, in some situations it may be desirable to compose
a document from a collection of parts
without having mandatory page breaks between then.
For this case, the package
provides a mechanism to include parts
by |\input| which can also be processed individually.
However, by construction this mechanism
requires manual handling of the content to be output.

%%%%%%%%%%%%%%%%%%%%%%%%%%%%%%%%%%%%%%%%
\DescribeMacro{\ifchilddocmanual}
The main file should be prepared as usual, see \secref{sec:include}.
However, the document body must make a distinction
between processing of an individual part and of the main document, e.g.:
%
\begin{center}
\begin{tabular}{l}
|\ifchilddocmanual|\\
|\input{\childdocname}|\\
|\||else|\\
\textit{document body with }|\input{|\textit{part}|}|\\
|\||fi|
\end{tabular}
\end{center}
%
The conditional |\ifchilddocmanual| is true whenever
a part to be included by |\input| is being compiled,
and the name of the part is stored in |\childdocname|.

%%%%%%%%%%%%%%%%%%%%%%%%%%%%%%%%%%%%%%%%
\DescribeMacro{\childdocby}
Each part to be included by |\input| should start with:
%
\begin{center}
\begin{tabular}{l}
|% \iffalse
%
% childdoc.dtx Copyright (C) 2017-2018 Niklas Beisert
%
% This work may be distributed and/or modified under the
% conditions of the LaTeX Project Public License, either version 1.3
% of this license or (at your option) any later version.
% The latest version of this license is in
%   http://www.latex-project.org/lppl.txt
% and version 1.3 or later is part of all distributions of LaTeX
% version 2005/12/01 or later.
%
% This work has the LPPL maintenance status `maintained'.
%
% The Current Maintainer of this work is Niklas Beisert.
%
% This work consists of the files childdoc.dtx and childdoc.ins
% and the derived files childdoc.def and cdocsamp.tex with
% cdocsch1.tex, cdocsch2.tex, cdocsdrf.tex, cdocsfn1.tex, cdocsfn2.tex.
%
%<package>\ifdefined\childdocmain\endinput\fi
%<package>\ProvidesFile{childdoc.def}[2018/12/30 v2.0 child document driver]
%<samplemain>\ProvidesFile{cdocsamp.tex}[2018/12/30 v2.0 sample for childdoc]
%<*driver>
%\ProvidesFile{childdoc.drv}[2018/12/30 v2.0 childdoc reference manual file]
\PassOptionsToClass{10pt,a4paper}{article}
\documentclass{ltxdoc}

\usepackage[margin=35mm]{geometry}
\usepackage{hyperref}
\usepackage{hyperxmp}
\usepackage[usenames]{color}

\hypersetup{colorlinks=true}
\hypersetup{pdfstartview=FitH}
\hypersetup{pdfpagemode=UseNone}
\hypersetup{pdfsource={}}
\hypersetup{pdflang={en-UK}}
\hypersetup{pdfcopyright={Copyright 2017-2018 Niklas Beisert.
  This work may be distributed and/or modified under the
  conditions of the LaTeX Project Public License, either version 1.3
  of this license or (at your option) any later version.}}
\hypersetup{pdflicenseurl={http://www.latex-project.org/lppl.txt}}
\hypersetup{pdfcontactaddress={ETH Zurich, ITP, HIT K,
  Wolfgang-Pauli-Strasse 27}}
\hypersetup{pdfcontactpostcode={8093}}
\hypersetup{pdfcontactcity={Zurich}}
\hypersetup{pdfcontactcountry={Switzerland}}
\hypersetup{pdfcontactemail={nbeisert@itp.phys.ethz.ch}}
\hypersetup{pdfcontacturl={http://people.phys.ethz.ch/\xmptilde nbeisert/}}

\newcommand{\secref}[1]{\hyperref[#1]{section \ref*{#1}}}

\parskip1ex
\parindent0pt
\let\olditemize\itemize
\def\itemize{\olditemize\parskip0pt}

\begin{document}

\title{The \textsf{childdoc} Package}
\hypersetup{pdftitle={The childdoc Package}}
\author{Niklas Beisert\\[2ex]
  Institut f\"ur Theoretische Physik\\
  Eidgen\"ossische Technische Hochschule Z\"urich\\
  Wolfgang-Pauli-Strasse 27, 8093 Z\"urich, Switzerland\\[1ex]
  \href{mailto:nbeisert@itp.phys.ethz.ch}
  {\texttt{nbeisert@itp.phys.ethz.ch}}}
\hypersetup{pdfauthor={Niklas Beisert}}
\hypersetup{pdfsubject={Manual for the LaTeX2e Package childdoc}}
\date{30 December 2018, \textsf{v2.0}}
\maketitle

\begin{abstract}\noindent
\textsf{childdoc} is a \LaTeXe{} package
that enables the direct compilation
of document sections included by |\include|
to individual files.
\end{abstract}

\begingroup
\parskip0ex
\tableofcontents
\endgroup

%%%%%%%%%%%%%%%%%%%%%%%%%%%%%%%%%%%%%%%%%%%%%%%%%%%%%%%%%%%%%%%%%%%%%%%%%%%%%%%%
%%%%%%%%%%%%%%%%%%%%%%%%%%%%%%%%%%%%%%%%%%%%%%%%%%%%%%%%%%%%%%%%%%%%%%%%%%%%%%%%
\section{Introduction}

\LaTeX{} provides a mechanism to structure a large document (such as a book)
into a main file and several child files (containing the chapters)
using the |\include| command.
This mechanism is beneficial for documents
which span hundreds of pages in order to
make the source file(s) more manageable.
Moreover, compilation can be restricted to
selected child files by means of the |\includeonly| command.
The latter feature can be used to reduce the compilation time while editing
(this was significantly more useful in the earlier days of \LaTeX{})
or to generate a smaller document which is easier to navigate.
Another application of |\includeonly| is to generate
documents consisting of selected parts of the complete document.

However, there are a few drawbacks of the plain |\include| mechanism:
\begin{itemize}
\item
The child files cannot be compiled on their own,
they can only be compiled via the main file.
A naive editing environment
(such as a text editor with an option
to have the current file processed by \LaTeX)
may require one to switch to the main file before compiling;
attempting to compile the child file produces errors.
\item
The main file must be modified (each time)
to adjust the |\includeonly| command
to the present needs. This easily leaves the main file in a messy state.
\item
The generated document will always carry the filename
of the main document. This is inconvenient if
several child files are to be compiled and
to be kept for distribution.
\end{itemize}

The present package provides a simple interface
to make child files individually compilable by \LaTeX{}.
Compiling a child file then has the same effect as compiling
the main file with an |\includeonly| command
to select the appropriate child.
Moreover the generated document will carry the name of the child
rather than the main file.
This resolves all three above issues.

This feature is meant to make the editing of books,
thesis documents and lecture notes somewhat more convenient.
However, the package can also be used efficiently for
composing a series of documents (such as exercise sheets)
which are typically distributed individually.
It then assists the author in generating the individual documents
(potentially in different versions)
as well as a document containing the collected series.
Another application is in developing style files
or other kinds of included material
where compilation of the style file could redirect
to a sample or test file.

%%%%%%%%%%%%%%%%%%%%%%%%%%%%%%%%%%%%%%%%%%%%%%%%%%%%%%%%%%%%%%%%%%%%%%%%%%%%%%%%
%%%%%%%%%%%%%%%%%%%%%%%%%%%%%%%%%%%%%%%%%%%%%%%%%%%%%%%%%%%%%%%%%%%%%%%%%%%%%%%%
\section{Usage}

First of all, the package \textsf{childdoc} is \emph{not} a standard
\LaTeXe{} |.sty| style file! Therefore it needs to be invoked in
a non-standard way.

%%%%%%%%%%%%%%%%%%%%%%%%%%%%%%%%%%%%%%%%%%%%%%%%%%%%%%%%%%%%%%%%%%%%%%%%%%%%%%%%
\subsection{Included Files}
\label{sec:include}

%%%%%%%%%%%%%%%%%%%%%%%%%%%%%%%%%%%%%%%%
\DescribeMacro{\childdocmain}
To use the package, add the commands
\begin{center}
\begin{tabular}{l}
|\input{childdoc.def}|\\
|\childdocmain{}|\\
\end{tabular}
\end{center}
at the very top of the main \LaTeX{} file,
in particular \emph{before} the |\documentclass| statement!
The argument of |\childdocmain| should be left empty
(but it must be present).

%%%%%%%%%%%%%%%%%%%%%%%%%%%%%%%%%%%%%%%%
\DescribeMacro{\childdocof}
Furthermore, add the commands
\begin{center}
\begin{tabular}{l}
|\input{childdoc.def}|\\
|\childdocof{|\textit{main}|}|\\
\end{tabular}
\end{center}
at the top of every child file \textit{child}
which is included by |\include{|\textit{child}|}|
from within the main file
(or at least for those files to be compiled individually).
The argument \textit{main} must be the filename of the main file.

There are a couple of
considerations in setting up the main and child documents:

%%%%%%%%%%%%%%%%%%%%%%%%%%%%%%%%%%%%%%%%
\paragraph{Restrictions.}

Please note the following restrictions:
\begin{itemize}
\item
|\childdocmain| must be called with one argument \textit{main}
to ensure compatibility with earlier version of the package.
It must either be empty (|\childdocmain{}|)
or precisely match the filename of the main file in which it is specified.
See \secref{sec:detection} for further information.
\item
The filename \textit{main} must be specified without the |.tex| extension.
\item
The filename \textit{main} is case sensitive
(even in case-insensitive file systems)
due to internal string comparison.
\item
The argument \textit{main} should be fully expanded, it cannot be a macro.
\item
Subdirectories and special characters should be avoided in filenames.
\item
The command |\childdocmain{|\textit{main}|}| must be followed by a whitespace.
It should not be followed immediately by another command
or by a comment mark `|%|'.
This is because the \TeX{} parser reads the token immediately following
the argument of |\childdocmain| and puts it
at the beginning of every child section;
however, a white\-space is ignored.
\end{itemize}

%%%%%%%%%%%%%%%%%%%%%%%%%%%%%%%%%%%%%%%%
\paragraph{Content of Main File.}

It is advisable to place all content in the child files included by |\include|.
Any output contained in the main file will appear in all child documents
unless suppressed manually;
it cannot be suppressed automatically by the |\includeonly| directive
and thus should normally be avoided.
A method to include some content in the main file
by means of conditional processing is described in \secref{sec:conditional}.

%%%%%%%%%%%%%%%%%%%%%%%%%%%%%%%%%%%%%%%%
\paragraph{Page Numbering.}

When only a part of the document is compiled,
the appropriate numbering of pages
(as well as other status parameters)
is determined from the |.aux| files.
The latter contain information from previous passes.
However this information needs to propagate through
all intermediate child documents.
Therefore the page numbering in child documents may well
be inconsistent until the complete document is compiled at least once.

A useful (if unconventional) way to always ensure a consistent
page numbering is to restart the numbering in each child document
and denote the pages by `\textit{child}|.|\textit{page}'
where \textit{child} represents the chapter/section number of the child file.
This can be achieved by the command
|\numberwithin{page}{|\textit{child}|}|
of the \textsf{amsmath} package
where \textit{child} can be |chapter| or |section|
depending on the chosen structuring.
Alternatively, one can modify the macro |\thepage| appropriately
and reset the counter |page| at the start of each child file.

%%%%%%%%%%%%%%%%%%%%%%%%%%%%%%%%%%%%%%%%%%%%%%%%%%%%%%%%%%%%%%%%%%%%%%%%%%%%%%%%
\subsection{Conditional Processing}
\label{sec:conditional}

The package provides a mechanism to compile different versions
of a document. To customise the versions further some conditional processing
can come in handy to distinguish which version is being compiled.
The package provides two macros to describe the compilation context:

%%%%%%%%%%%%%%%%%%%%%%%%%%%%%%%%%%%%%%%%
\DescribeMacro{\ifchilddoc}
The conditional |\ifchilddoc| distinguishes between the compilation of
child documents and the main document:
%
\begin{center}
|\ifchilddoc |\textit{child-code}| |[|\||else |\textit{main-code}]| \||fi|
\end{center}

%%%%%%%%%%%%%%%%%%%%%%%%%%%%%%%%%%%%%%%%
\DescribeMacro{\childdocname}
\DescribeMacro{\childdocjob}
The macro |\childdocname| contains the filename (without extension)
of the main or child file being processed.
Note that |\childdocjob| will always contain the name of the main file.

%%%%%%%%%%%%%%%%%%%%%%%%%%%%%%%%%%%%%%%%
\paragraph{Title Page.}

Conditional processing can be used to include a title or banner page
in the main document when proper precautions are taken.
Importantly, the code in the main file should ensure that the page counter
(as well as other status parameters which are stored in the |.aux| files)
takes the same value after the conditional processing.
Otherwise the page numbers may take divergent values
depending on which part is compiled.

For example, a title page could be declared by:
%
\begin{center}
\begin{tabular}{l}
|\ifchilddoc\||else|\\
|\addtocounter{page}{-1}|\\
\textit{code for title page}\\
|\newpage|\\
|\||fi|
\end{tabular}
\end{center}
%
A banner page for the child documents can be generated by:
%
\begin{center}
\begin{tabular}{l}
|\ifchilddoc|\\
|\addtocounter{page}{-1}|\\
\textit{code for banner page}\\
|\newpage|\\
|\||fi|
\end{tabular}
\end{center}
%
Here one could write a message such as:
\begin{center}
|This is the part \childdocname{} of \childdocjob{}.|
\end{center}

%%%%%%%%%%%%%%%%%%%%%%%%%%%%%%%%%%%%%%%%%%%%%%%%%%%%%%%%%%%%%%%%%%%%%%%%%%%%%%%%
\subsection{Flags}
\label{sec:flags}

The package makes it easy to generate different versions
of the main or child documents.
To this end compilation flags can be defined
and assigned different default values.
They will be particularly useful in conjunction
with the forwarding mechanism described in \secref{sec:forward}.

For example, it may be useful to have a flag |\version|
which can be set to |draft| or |final|.
The document source will contain some conditional code
depending on the value of |\version|.
Suppose further, the flag should default to |final| for the main file
and to |draft| for child files
which is a natural assignment for editing the document.
This is achieved by placing the following code
in the preamble of the main document
(below the |\childdocmain| directive):
%
\begin{center}
\begin{tabular}{l}
|\ifchilddoc|\\
|\providecommand{\version}{draft}|\\
|\||else|\\
|\providecommand{\version}{final}|\\
|\||fi|
\end{tabular}
\end{center}
%
The definition by |\providecommand| makes sure
that previous definitions are not overwritten.
Further statements |\providecommand{\version}{...}|
can thus be added before the above code to override it.

For the main file, one might add a line
(between |\childdocmain| and the above block)
%
\begin{center}
|%\ifchilddoc\||else\providecommand{\version}{draft}\||fi|
\end{center}
%
which can be uncommented to produce a draft version.
Likewise one can add a line to the very top of a child file
(above the |\childdocof{|\textit{main}|}| directive)
%
\begin{center}
|%\providecommand{\version}{final}|
\end{center}
%
which can be uncommented to produce the final version of this child document.

%%%%%%%%%%%%%%%%%%%%%%%%%%%%%%%%%%%%%%%%%%%%%%%%%%%%%%%%%%%%%%%%%%%%%%%%%%%%%%%%
\subsection{Forwarding}
\label{sec:forward}

Different versions of the main or child documents
using compilation flags as described in \secref{sec:flags}
can be (permanently) stored in different files
for convenient compilation, viewing and distribution.
To this end, the package defines a command
to pass on compilation to a different file:

%%%%%%%%%%%%%%%%%%%%%%%%%%%%%%%%%%%%%%%%
\DescribeMacro{\childdocforward}
The command |\childdocforward| redirects processing to
another source file:
%
\begin{center}
\begin{tabular}{l}
|\input{childdoc.def}|\\
|\childdocforward[|\textit{main}|]{|\textit{dest}|}|\\
\end{tabular}
\end{center}
%
The argument \textit{dest} is the destination file
(without extension).
It should be the main file or one of the child files.
Note that further \textsf{childdoc} directives
such as |\childdocof| and |\childdocforward|
in the indicated file will be processed in this form.
The optional argument \textit{main}
passes on directly to the main file \textit{main}
while pretending to compile the child \textit{dest}.
This form behaves as if \textit{dest}
issues |\childdocof{|\textit{main}|}| right away,
and no further \textsf{childdoc} directives will be processed.

%%%%%%%%%%%%%%%%%%%%%%%%%%%%%%%%%%%%%%%%
\DescribeMacro{\...prefix}
In the alternative form |\childdocforwardprefix|,
%
\begin{center}
\begin{tabular}{l}
|\input{childdoc.def}|\\
|\childdocforwardprefix[|\textit{main}|]{|\textit{prefix}|}{|\textit{dest}|}|
\end{tabular}
\end{center}
%
the destination file is determined by a pattern
depending on the current file:
To make this work, the current file must be called
`{\textit{prefix}\hspace{0.2em}\textit{suffix}}'
with \textit{prefix} matching precisely the argument.
Processing is then passed on to the file
`{\textit{dest}\hspace{0.2em}\textit{suffix}}'.
Surely, the same effect is achieved by
directly specifying the
argument `{\textit{dest}\hspace{0.2em}\textit{suffix}}'
in the first form.
However, that requires to set up a different file
for each child. With the alternative form of the command
all these files can have exactly the same content
which simplifies setting them up and maintaining them.

For example, the following file |draft.tex|
with a compilation flag |\version| as described in \secref{sec:flags}
compiles the main document as a draft:
%
\begin{center}
\begin{tabular}{l}
|\def\version{draft}|\\
|\input{childdoc.def}|\\
|\childdocforward{|\textit{main}|}|
\end{tabular}
\end{center}
%
Likewise, the following files |final|\textit{nn}|.tex|
compile the final version of the child document
|child|\textit{nn}|.tex|:
%
\begin{center}
\begin{tabular}{l}
|\def\version{final}|\\
|\input{childdoc.def}|\\
|\childdocforwardprefix{final}{child}|
\end{tabular}
\end{center}
%

Note that when several versions of a main file and/or of each child file
are to be generated, it may be convenient to set up a |Makefile| or
shell script to automatise the process.

%%%%%%%%%%%%%%%%%%%%%%%%%%%%%%%%%%%%%%%%%%%%%%%%%%%%%%%%%%%%%%%%%%%%%%%%%%%%%%%%
\subsection{Command Line Processing}
\label{sec:commandline}

The effect of redirection files can also be achieved by invoking
the \LaTeX{} compiler with a more elaborate command line.
Most conveniently this should be done as part
of a shell script or a |Makefile|.

When using \textsf{childdoc} in the main file, the following
command lines effectively perform a redirection
(note that depending on the shell being used,
backslashes may have to be doubled: `|\|' $\to$ `|\\|'):
%
\begin{center}
|... -jobname "|\textit{target}|" |\\|"|[\textit{flags}]%
|\input{childdoc.def}\childdocforward[|\textit{main}|]{|\textit{dest}|}"|
\end{center}
%
Here \textit{target} is the name of the output file,
\textit{main} is the name of the main file
and \textit{dest} is the name of the main or child file to be processed
(all filenames without extensions).
The optional argument \textit{main} can be omitted
if \textit{main} matches \textit{dest}.
Optionally, compilation \textit{flags} can be defined via |\def| commands.
This command line makes the \TeX{} engine believe
it is compiling the file \textit{target}
whose content is specified as the latter parameter.
The provided code then forwards the processing to
\textit{main} or \textit{dest} as described in \secref{sec:forward}.

%%%%%%%%%%%%%%%%%%%%%%%%%%%%%%%%%%%%%%%%%%%%%%%%%%%%%%%%%%%%%%%%%%%%%%%%%%%%%%%%
\subsection{Include by Input}
\label{sec:input}

Including child documents by |\include| has some restrictions by design.
Most notably, the content of a child document always occupies
its own set of pages; pages cannot be shared between child documents.
Usually, this behaviour makes perfect sense
because each child document contain an essential part of the document.
However, in some situations it may be desirable to compose
a document from a collection of parts
without having mandatory page breaks between then.
For this case, the package
provides a mechanism to include parts
by |\input| which can also be processed individually.
However, by construction this mechanism
requires manual handling of the content to be output.

%%%%%%%%%%%%%%%%%%%%%%%%%%%%%%%%%%%%%%%%
\DescribeMacro{\ifchilddocmanual}
The main file should be prepared as usual, see \secref{sec:include}.
However, the document body must make a distinction
between processing of an individual part and of the main document, e.g.:
%
\begin{center}
\begin{tabular}{l}
|\ifchilddocmanual|\\
|\input{\childdocname}|\\
|\||else|\\
\textit{document body with }|\input{|\textit{part}|}|\\
|\||fi|
\end{tabular}
\end{center}
%
The conditional |\ifchilddocmanual| is true whenever
a part to be included by |\input| is being compiled,
and the name of the part is stored in |\childdocname|.

%%%%%%%%%%%%%%%%%%%%%%%%%%%%%%%%%%%%%%%%
\DescribeMacro{\childdocby}
Each part to be included by |\input| should start with:
%
\begin{center}
\begin{tabular}{l}
|\input{childdoc.def}|\\
|\childdocby{|\textit{main}|}|\\
\end{tabular}
\end{center}
%
The directive |\childdocby| is similar to |\childdocof|
described in \secref{sec:include},
but the subsequent selection of content must be done manually.
To that end, both |\ifchilddoc| and |\ifchilddocmanual|
will be true upon processing of a part,
and the name of the part is stored in |\childdocname|.
Note that |\jobname| will be set to the filename of the current part
so that each part receives an individual |.aux| file
that does not interfere with the |.aux| file(s) of the main document.
This behaviour can be altered by the alternative form
|\childdocby[*]{|\textit{main}|}| (with a non-empty optional argument)
which uses the |.aux| file of the main document
by setting |\jobname| to \textit{main}.

%%%%%%%%%%%%%%%%%%%%%%%%%%%%%%%%%%%%%%%%%%%%%%%%%%%%%%%%%%%%%%%%%%%%%%%%%%%%%%%%
\subsection{Driver Development}
\label{sec:driver}

The \textsf{childdoc} mechanism can also be use for the development
of definition files such as \LaTeX{} styles or classes.
This case differs from the above setup with multiple parts
included by |\include| in that no |\includeonly| should be invoked.
This can be achieved by starting the include file
(before |\ProvidesPackage|) with:
%
\begin{center}
\begin{tabular}{l}
|\input{childdoc.def}|\\
|\childdocforward{|\textit{main}|}|\\
\end{tabular}
\end{center}
%
or alternatively with:
%
\begin{center}
\begin{tabular}{l}
|\input{childdoc.def}|\\
|\childdocby{|\textit{main}|}|\\
\end{tabular}
\end{center}
%
Both forms have slightly different effects as described above.
The main file is prepared as usual, see \secref{sec:include}.

%%%%%%%%%%%%%%%%%%%%%%%%%%%%%%%%%%%%%%%%%%%%%%%%%%%%%%%%%%%%%%%%%%%%%%%%%%%%%%%%
\subsection{Legacy Detection}
\label{sec:detection}

The directive |\childdocmain| in the main file can detect
whether the complete document or merely a child is to be compiled
even without using the directive |\childdocof|.
This method is deprecated because it is less robust
and there is no compelling reason to use it;
it is merely provided for backward compatibility
and it may be removed in future versions.

If the detection mechanism is to be used,
it is mandatory to correctly specify
the filename of the main file as the argument of |\childdocmain|:
%
\begin{center}
\begin{tabular}{l}
|\input{childdoc.def}|\\
|\childdocmain{|\textit{main}|}|\\
\end{tabular}
\end{center}
%
If |\jobname| does not match the argument \textit{main} of |\childdocmain|,
it is assumed that |\jobname| points to the child file to be compiled.
When using |\childdocmain| with the main file specified as argument,
it suffices to start a child file
with just |\input{|\textit{main}|}|
without loading of the package and using |\childdocof|.
If instead all processing is done
with the appropriate \textsf{childdoc} directives,
the argument of \textit{main} of |\childdocmain| can be empty.

An alternative version of the command line processing described
in \secref{sec:commandline} using the detection mechanism reads:
%
\begin{center}
|... -jobname "|\textit{target}|" "|[\textit{flags}]%
[|\def\jobname{|\textit{dest}|}|]|\input{|\textit{main}|}"|
\end{center}

%%%%%%%%%%%%%%%%%%%%%%%%%%%%%%%%%%%%%%%%%%%%%%%%%%%%%%%%%%%%%%%%%%%%%%%%%%%%%%%%
\subsection{Manual Code}
\label{sec:manual}

In case one cannot be certain whether the definitions file |childdoc.def|
is installed on the target \TeX{} distribution
and one prefers not to ship it,
it is conceivable to paste a few relevant commands into the sources.

To that end, drop all statements |\input{childdoc.def}|
and perform the replacements as outlined below.
Instead of |\childdocmain{|\textit{main}|}| add the following code
to the top of the main file:
%
\begin{center}
\begin{tabular}{l}
|\||ifdefined\childdocname\endinput\||fi\newif\ifchilddoc|\\
|\edef\childdocname{\scantokens\expandafter{\jobname\noexpand}}|\\
|\def\childdocmain{|\textit{main}|}\||ifx\childdocmain\childdocname\||else|\\
|\childdoctrue\includeonly{\childdocname}\let\jobname\childdocmain\||fi|\\
\end{tabular}
\end{center}
%
Instead of |\childdocof{|\textit{main}|}| just include the main file
at the top of each child file:
%
\begin{center}
|\input{|\textit{main}|}|
\end{center}
%
A simple redirection |\childdocforward{|\textit{dest}|}| is achieved by:
%
\begin{center}
|\def\jobname{|\textit{dest}|}\input{\jobname}|
\end{center}
%
The redirection with prefix
|\childdocforwardprefix[|\textit{prefix}|]{|\textit{dest}|}|
is accomplished by:
%
\begin{center}
\begin{tabular}{l}
|{\edef\jobname{\scantokens\expandafter{\jobname\noexpand}}|\\
|\def\redirectjob |\textit{prefix}|#1~~~{\gdef\jobname{|\textit{dest}|#1}}|\\
|\expandafter\redirectjob\jobname~~~}\input{\jobname}|
\end{tabular}
\end{center}

In an alternative approach,
child documents can be compiled by a specific command line
without additional code or specific definitions:
%
\begin{center}
|... -jobname "|\textit{target}|" "|[\textit{flags}]%
|\includeonly{|\textit{dest}|}\input{|\textit{main}|}"|
\end{center}
%

%%%%%%%%%%%%%%%%%%%%%%%%%%%%%%%%%%%%%%%%%%%%%%%%%%%%%%%%%%%%%%%%%%%%%%%%%%%%%%%%
%%%%%%%%%%%%%%%%%%%%%%%%%%%%%%%%%%%%%%%%%%%%%%%%%%%%%%%%%%%%%%%%%%%%%%%%%%%%%%%%
\section{Information}

%%%%%%%%%%%%%%%%%%%%%%%%%%%%%%%%%%%%%%%%%%%%%%%%%%%%%%%%%%%%%%%%%%%%%%%%%%%%%%%%
\subsection{Copyright}

Copyright \copyright{} 2017--2018 Niklas Beisert

This work may be distributed and/or modified under the
conditions of the \LaTeX{} Project Public License, either version 1.3
of this license or (at your option) any later version.
The latest version of this license is in
  \url{http://www.latex-project.org/lppl.txt}
and version 1.3 or later is part of all distributions of \LaTeX{}
version 2005/12/01 or later.

This work has the LPPL maintenance status `maintained'.

The Current Maintainer of this work is Niklas Beisert.

This work consists of the files |README.txt|, |childdoc.ins| and |childdoc.dtx|
as well as the derived files |childdoc.def|, |cdocsamp.tex|
with |cdocsch1.tex|, |cdocsch2.tex|, |cdocspt3.tex|, |cdocspt4.tex|,
|cdocsdrf.tex|, |cdocsfn1.tex|, |cdocsfn2.tex|
as well as |childdoc.pdf|.

%%%%%%%%%%%%%%%%%%%%%%%%%%%%%%%%%%%%%%%%%%%%%%%%%%%%%%%%%%%%%%%%%%%%%%%%%%%%%%%%
\subsection{Files and Installation}

The package consists of the files:
%
\begin{center}
\begin{tabular}{ll}
    |README.txt|   & readme file \\
    |childdoc.ins| & installation file \\
    |childdoc.dtx| & source file \\
    |childdoc.def| & definition file \\
    |cdocsamp.tex| & sample main file \\
    |cdocsch1.tex| & sample include file \\
    |cdocsch2.tex| & sample include file \\
    |cdocspt3.tex| & sample part file \\
    |cdocspt4.tex| & sample part file \\
    |cdocsdrf.tex| & sample redirection file \\
    |cdocsfn1.tex| & sample redirection file \\
    |cdocsfn2.tex| & sample redirection file \\
    |childdoc.pdf| & manual
\end{tabular}
\end{center}
%
The distribution consists of the files
|README.txt|, |childdoc.ins| and |childdoc.dtx|.
%
\begin{itemize}
\item
Run (pdf)\LaTeX{} on |childdoc.dtx|
to compile the manual |childdoc.pdf| (this file).
\item
Run \LaTeX{} on |childdoc.ins| to create the definitions file |childdoc.def|
and the sample |cdocsamp.tex| with include files
|cdocsch1.tex|, |cdocsch2.tex|, |cdocspt3.tex|, |cdocspt4.tex|,
|cdocsdrf.tex|, |cdocsfn1.tex|, |cdocsfn2.tex|.
Then copy the file |childdoc.def| to an appropriate directory of your \LaTeX{}
distribution, e.g.\ \textit{texmf-root}|/tex/latex/childdoc|.
\end{itemize}

%%%%%%%%%%%%%%%%%%%%%%%%%%%%%%%%%%%%%%%%%%%%%%%%%%%%%%%%%%%%%%%%%%%%%%%%%%%%%%%%
\subsection{Related CTAN Packages}

There are several other packages which offer a similar functionality:
%
\begin{itemize}
\item
The packages
\href{http://ctan.org/pkg/docmute}{\textsf{docmute}},
\href{http://ctan.org/pkg/includex}{\textsf{includex}} and
\href{http://ctan.org/pkg/standalone}{\textsf{standalone}}
provide commands to include only the document body of
a child file thus allowing both files to be compiled individually.
\item
The packages \href{http://ctan.org/pkg/subdocs}{\textsf{subdocs}}
and \href{http://ctan.org/pkg/subfiles}{\textsf{subfiles}}
provide structures in which the main and child documents can be
encapsulated and allowing them to be compiled individually.
The inclusion mechanism is different from the conventional |\include|.
\item
The package \href{http://ctan.org/pkg/combine}{\textsf{combine}}
is an elaborate solution to combine several documents into one.
\end{itemize}
%
See also the CTAN topic \href{http://ctan.org/topic/subdocs}{\textsf{subdocs}}
for further related packages.
The present package differs from the above solutions in that
a document structure constructed with the conventional |\include| mechanism
just needs two extra commands at the top of every file
such that all constituent files can be compiled individually.

%%%%%%%%%%%%%%%%%%%%%%%%%%%%%%%%%%%%%%%%%%%%%%%%%%%%%%%%%%%%%%%%%%%%%%%%%%%%%%%%
%\subsection{Feature Suggestions}
%
%The following is a list of features which may be useful for future
%versions of this package:
%%
%\begin{itemize}
%\item
%\ldots
%\end{itemize}

%%%%%%%%%%%%%%%%%%%%%%%%%%%%%%%%%%%%%%%%%%%%%%%%%%%%%%%%%%%%%%%%%%%%%%%%%%%%%%%%
\subsection{Revision History}

%%%%%%%%%%%%%%%%%%%%%%%%%%%%%%%%%%%%%%%%
\paragraph{v2.0:} 2018/12/30

\begin{itemize}
\item
immediate forward processing
\item
added |\childdocby| mechanism
\item
manual restructured
\end{itemize}

%%%%%%%%%%%%%%%%%%%%%%%%%%%%%%%%%%%%%%%%
\paragraph{v1.6:} 2018/01/17

\begin{itemize}
\item
application for development of include files
\item
corrections to manual
\end{itemize}

%%%%%%%%%%%%%%%%%%%%%%%%%%%%%%%%%%%%%%%%
\paragraph{v1.5:} 2017/05/21

\begin{itemize}
\item
more complete structuring introduced
\item
|\childdocof| introduced
\item
|\childdoc| renamed to |\childdocmain|
\item
|\childredirect| renamed to |\childdocforward| and |\childdocforwardprefix|
and functionality expanded
\end{itemize}

%%%%%%%%%%%%%%%%%%%%%%%%%%%%%%%%%%%%%%%%
\paragraph{v1.0:} 2017/04/27

\begin{itemize}
\item
manual and install package
\item
first version published on CTAN
\end{itemize}

%%%%%%%%%%%%%%%%%%%%%%%%%%%%%%%%%%%%%%%%
\paragraph{v0.6:} 2017/04/26

\begin{itemize}
\item
redirection mechanism added
\end{itemize}

%%%%%%%%%%%%%%%%%%%%%%%%%%%%%%%%%%%%%%%%
\paragraph{v0.5:} 2017/04/26

\begin{itemize}
\item
functionality in definition file
\end{itemize}


%%%%%%%%%%%%%%%%%%%%%%%%%%%%%%%%%%%%%%%%%%%%%%%%%%%%%%%%%%%%%%%%%%%%%%%%%%%%%%%%
%%%%%%%%%%%%%%%%%%%%%%%%%%%%%%%%%%%%%%%%%%%%%%%%%%%%%%%%%%%%%%%%%%%%%%%%%%%%%%%%
%%%%%%%%%%%%%%%%%%%%%%%%%%%%%%%%%%%%%%%%%%%%%%%%%%%%%%%%%%%%%%%%%%%%%%%%%%%%%%%%
\appendix

\settowidth\MacroIndent{\rmfamily\scriptsize 000\ }

 \DocInput{childdoc.dtx}

\end{document}
%</driver>
% \fi
%
% %%%%%%%%%%%%%%%%%%%%%%%%%%%%%%%%%%%%%%%%%%%%%%%%%%%%%%%%%%%%%%%%%%%%%%%%%%%%%%
% %%%%%%%%%%%%%%%%%%%%%%%%%%%%%%%%%%%%%%%%%%%%%%%%%%%%%%%%%%%%%%%%%%%%%%%%%%%%%%
% \section{Sample}
%\iffalse
%<*samplemain>
%\fi
%
% The following presents a sample document
% with two chapters, two parts, a title page,
% a compile flag as well as three forwarding files to set the flag.
% It consists of eight |.tex| files:
% \begin{center}
% \begin{tabular}{ll}
% |cdocsamp.tex|&main file\\
% |cdocsch1.tex|&include file for chapter 1\\
% |cdocsch2.tex|&include file for chapter 2\\
% |cdocspt3.tex|&include file for part 3\\
% |cdocspt4.tex|&include file for part 4\\
% |cdocsdrf.tex|&forwarding file for main file in draft mode\\
% |cdocsfi1.tex|&forwarding file for final version of chapter 1\\
% |cdocsfi2.tex|&forwarding file for final version of chapter 2\\
% \end{tabular}
% \end{center}
% Each of the eight files can be compiled directly by the \LaTeX{} compiler.
%
% %%%%%%%%%%%%%%%%%%%%%%%%%%%%%%%%%%%%%%
% \paragraph{Main File.}
%
% The main file is called |cdocsamp.tex|.
%
% Load the \textsf{childdoc} definitions and
% declare the filename for the main document:
%    \begin{macrocode}
\input{childdoc.def}
\childdocmain{}
%    \end{macrocode}

% Optional override for |\version| flag:
%    \begin{macrocode}
%%\ifchilddoc\else\providecommand{\version}{draft}\fi
%    \end{macrocode}

% Define the default values for the |\version| flag
% (|final| for the main file and |draft| for childs):
%    \begin{macrocode}
\ifchilddoc
\providecommand{\version}{draft}
\else
\providecommand{\version}{final}
\fi
%    \end{macrocode}

% Load the standard document class:
%    \begin{macrocode}
\documentclass[12pt]{article}
%    \end{macrocode}

% Start the document body:
%    \begin{macrocode}
\begin{document}
%    \end{macrocode}

% Declare a title page.
% Print title, part of document being processed and version flag:
%    \begin{macrocode}
\addtocounter{page}{-1}
\begin{center}
{\LARGE\bfseries{}childdoc example\par}
\vspace{1cm}
\ifchilddoc
\ifchilddocmanual part\else chapter\fi:
`\childdocname' of `\childdocjob'\par
\else
main document: `\childdocjob'\par
\fi
version: \version\par
\end{center}
\newpage
%    \end{macrocode}

% Manually include selected file,
% otherwise process as usual:
%    \begin{macrocode}
\ifchilddocmanual
\section*{part `\childdocname'}
\input{\childdocname}
\else
%    \end{macrocode}

% Include the two chapters:
%    \begin{macrocode}
\include{cdocsch1}
\include{cdocsch2}
%    \end{macrocode}

% Include the two parts unless only chapters should be displayed:
%    \begin{macrocode}
\ifchilddoc\else
\section{part three}
\input{cdocspt3}
\section{part four}
\input{cdocspt4}
\fi
%    \end{macrocode}

% Process as usual until here:
%    \begin{macrocode}
\fi
%    \end{macrocode}

% End of document body:
%    \begin{macrocode}
\end{document}
%    \end{macrocode}
%\iffalse
%</samplemain>
%\fi
%
% %%%%%%%%%%%%%%%%%%%%%%%%%%%%%%%%%%%%%%
% \paragraph{Chapter Include Files.}
%
% The include files are called |cdocsch1.tex| and |cdocsch2.tex|.
%
%\iffalse
%<*samplechap1|samplechap2>
%\fi

% Optional override for |\version| flag:
%    \begin{macrocode}
%%\providecommand{\version}{final}
%    \end{macrocode}

% Include the main document:
%    \begin{macrocode}
\input{childdoc.def}
\childdocof{cdocsamp}
%    \end{macrocode}

%\iffalse
%</samplechap1|samplechap2>
%\fi
%
%\iffalse
%<*samplechap1>
%\fi
% Some text for chapter 1:
%    \begin{macrocode}
\section{one}
some text in chapter one
%    \end{macrocode}

%\iffalse
%</samplechap1>
%\fi
% Some text for chapter 2:
%\iffalse
%<*samplechap2>
%\fi
%    \begin{macrocode}
\section{two}
more text in chapter two
%    \end{macrocode}

%\iffalse
%</samplechap2>
%\fi
%
% %%%%%%%%%%%%%%%%%%%%%%%%%%%%%%%%%%%%%%
% \paragraph{Part Include Files.}
%
% The include files are called |cdocspt3.tex| and |cdocspt4.tex|.
%
%\iffalse
%<*samplepart3|samplepart4>
%\fi

% Optional override for |\version| flag:
%    \begin{macrocode}
%%\providecommand{\version}{final}
%    \end{macrocode}

% Include the main document:
%    \begin{macrocode}
\input{childdoc.def}
\childdocby{cdocsamp}
%    \end{macrocode}

%\iffalse
%</samplepart3|samplepart4>
%\fi
%
%\iffalse
%<*samplepart3>
%\fi
% Some text for part 3:
%    \begin{macrocode}
some text in part three
%    \end{macrocode}

%\iffalse
%</samplepart3>
%\fi
% Some text for part 4:
%\iffalse
%<*samplepart4>
%\fi
%    \begin{macrocode}
more text in part four
%    \end{macrocode}

%\iffalse
%</samplepart4>
%\fi
%
% %%%%%%%%%%%%%%%%%%%%%%%%%%%%%%%%%%%%%%
% \paragraph{Forwarding for a Complete Draft.}
%
% The following forwarding file |cdocsdrf.tex|
% compiles the main document in draft mode:
%\iffalse
%<*sampledraft>
%\fi
%    \begin{macrocode}
\def\version{draft}
\input{childdoc.def}
\childdocforward{cdocsamp}
%    \end{macrocode}

%\iffalse
%</sampledraft>
%\fi
%
% %%%%%%%%%%%%%%%%%%%%%%%%%%%%%%%%%%%%%%
% \paragraph{Forwarding for Final Version of the Chapters.}
%
% The following forwarding files |cdocsfn1.tex| and |cdocsfn2.tex|
% (with identical content)
% compile the final versions of the child documents
% |cdocsch1.tex| and |cdocsch2.tex|, respectively:
%\iffalse
%<*samplefinal>
%\fi
%    \begin{macrocode}
\def\version{final}
\input{childdoc.def}
\childdocforwardprefix[cdocsamp]{cdocsfn}{cdocsch}
%    \end{macrocode}

%\iffalse
%</samplefinal>
%\fi
%
% %%%%%%%%%%%%%%%%%%%%%%%%%%%%%%%%%%%%%%
% \paragraph{Command Line Processing.}
%
% The following three command lines generate the output files
% |cdocscld|, |cdocscl1| and |cdocscl2|
% which should be identical to
% |cdocsdrf|, |cdocsch1| and |cdocsfn2|, respectively:
% \begin{center}
% \begin{tabular}{l}
% |latex -jobname cdocscld \|\\
% |  "\def\version{draft}\input{childdoc.def}\childdocforward{cdocsamp}"|\\
% |latex -jobname cdocscl1 \|\\
% |  "\input{childdoc.def}\childdocforward[cdocsamp]{cdocsch1}"|\\
% |latex -jobname cdocscl2 \|\\
% |  "\def\version{final}\input{childdoc.def}\childdocforward{cdocsch2}"|
% \end{tabular}
% \end{center}
% Note that the trailing backslash on each first line
% merely continues the input to the second line
% (for convenient cut ant paste).
% Furthermore, the command |latex| can be replaced by any
% of its alternative versions such as |pdflatex|.
%
% %%%%%%%%%%%%%%%%%%%%%%%%%%%%%%%%%%%%%%%%%%%%%%%%%%%%%%%%%%%%%%%%%%%%%%%%%%%%%%
% %%%%%%%%%%%%%%%%%%%%%%%%%%%%%%%%%%%%%%%%%%%%%%%%%%%%%%%%%%%%%%%%%%%%%%%%%%%%%%
% \section{Implementation}
%\iffalse
%<*package>
%\fi
%
% This section describes the definitions file |childdoc.def|.

% The definitions cannot be loaded using |\usepackage| or |\RequirePackage|
% which has a mechanism to prevent loading a style file more than once.
% When loading the definitions by means of |\input|
% multiple instances have to be prevented manually:
%\iffalse
%This code needs to be before the `\ProvidesFile' directive
%which is defined at the beginning of this file.
%Therefore it is also placed there and commented out here.
%</package>
%<*discard>
%\fi
%    \begin{macrocode}
\ifdefined\childdocmain\endinput\fi
%    \end{macrocode}
%\iffalse
%</discard>
%<*package>
%\fi
%
% \macro{\ifchilddoc}
% \macro{\ifchilddocmanual}
% The conditional |\ifchilddoc| tells whether a
% child (true) or main (false) document is being compiled.
% The conditional |\ifchilddocmanual| tells whether
% the |\includeonly| mechanism is used (false) or
% the selection of child files must be performed manually (true).
% The definitions initialise to false:
%    \begin{macrocode}
\newif\ifchilddoc
\newif\ifchilddocmanual
%    \end{macrocode}

% \macro{\childdocname}
% \macro{\childdocjob}
% The macro |\childdocname| stores the name of the main document
% to be compiled. The macro |\childdocjob| stores the name of
% the document on which the \LaTeX{} compiler was originally invoked.
% The content of |\jobname| cannot be compared
% to filenames specified in the source due to different catcodes.
% The following code rescans |\jobname|, stores the result
% in |\childdocname| and saves a copy in |\childdocjob|:
%    \begin{macrocode}
\edef\childdocname{\scantokens\expandafter{\jobname\noexpand}}
\let\childdocjob\childdocname
%    \end{macrocode}

% \macro{\childdocdisable}
% The macro |\childdocdisable| prevents the main file
% from being processed more than once.
% At this stage, the main document command |\childdocmain|
% is assumed to be called once again where it should do nothing.
% Any subsequent call to it should prevent
% a secondary processing of the main document
% It overwrites the forwarding commands
% |\childdocof| and |\childdocforward|
% with empty macros to prevent further inclusions of the main document:
%    \begin{macrocode}
\newcommand{\childdocdisable}
{
  \renewcommand{\childdocmain}[1]{\renewcommand{\childdocmain}[1]{\endinput}}
  \renewcommand{\childdocof}[1]{}
  \renewcommand{\childdocby}[2][]{}
  \renewcommand{\childdocforward}[2][]{}
  \renewcommand{\childdocdisable}{}
}
%    \end{macrocode}

% \macro{\childdocmain}
% The macro |\childdocmain| is to be called at the top of the main file
% with nothing or the main filename (without extension) as argument.
% First, it breaks loops.
% If the argument is not empty and does not match |\childdocname|
% (which is set by the first inclusion of |childdoc.def|),
% |\ifchilddoc| is set to true, |\includeonly| is applied to the child file
% and |\jobname| is set to the main file
% (for proper handling of |.aux| files):
%    \begin{macrocode}
\newcommand{\childdocmain}[1]
{
  \childdocdisable\childdocmain{}
  \if?#1?\else
    \begingroup
      \def\childdoctmp{#1}
      \ifx\childdoctmp\childdocname
        \def\childdoctmp{}
      \else
        \def\childdoctmp
        {
          \childdoctrue
          \includeonly{\childdocname}
          \def\childdocjob{#1}
          \def\jobname{#1}
        }
      \fi
      \expandafter
    \endgroup
    \childdoctmp
  \fi
}
%    \end{macrocode}

% \macro{\childdocof}
% The command |\childdocof| redirects
% compilation to the main file |#1|.
%    \begin{macrocode}
\newcommand{\childdocof}[1]
{
  \childdocdisable
  \childdoctrue
  \includeonly{\childdocname}
  \def\jobname{#1}
  \def\childdocjob{#1}
  \input{#1}
}
%    \end{macrocode}

% \macro{\childdocby}
% The command |\childdocby| ....
%    \begin{macrocode}
\newcommand{\childdocby}[2][]
{
  \childdocdisable
  \childdoctrue
  \childdocmanualtrue
  \if?#1?\else
    \def\jobname{#2}
  \fi
  \def\childdocjob{#2}
  \input{#2}
  \endinput
}
%    \end{macrocode}

% \macro{\childdocforward}
% The command |\childdocforward| redirects
% compilation to the main file or
% (if the optional argument is given) a child file.
% Parameters are set as if the main file
% or a child file starting with |\childdocof| was compiled.
% Then compilation is handed over to the main file:
%    \begin{macrocode}
\newcommand{\childdocforward}[2][]
{
  \begingroup
    \if?#1?
      \def\childdoctmp
      {
        \def\childdocname{#2}
        \def\childdocjob{#2}
        \def\jobname{#2}
        \input{#2}
        \endinput
      }
    \else
      \def\childdoctmp
      {
        \childdocdisable
        \def\childdocname{#2}
        \childdoctrue
        \includeonly{#2}
        \def\childdocjob{#1}
        \def\jobname{#1}
        \input{#1}
        \endinput
      }
    \fi
    \expandafter
  \endgroup
  \childdoctmp
}
%    \end{macrocode}

% \macro{\childdocforwardprefix}
% The command |\childdocforwardprefix| redirects
% compilation to the main or a child file by means of a pattern.
% The prefix |#1| in the current filename is replaced by |#2|
% and the suffix of the current filename is kept
% (it is assumed that the filename does not contain the substring `|~~~|'
% which is used as a delimiter).
% Compilation is handed over to the new file by |\childdocforward|:
%    \begin{macrocode}
\newcommand{\childdocforwardprefix}[3][]
{
  \begingroup
    \def\childdocextract #2##1~~~{\def\childdoctmp{\childdocforward[#1]{#3##1}}}
    \expandafter\childdocextract\childdocname~~~
    \expandafter
  \endgroup
  \childdoctmp
}
%    \end{macrocode}

% \macro{\childdoc}
% The deprecated macro |\childdoc| is a legacy version of |\childdocmain|:
%    \begin{macrocode}
\newcommand{\childdoc}{\childdocmain}
%    \end{macrocode}

% \macro{\childdocredirect}
% The deprecated macro |\childdocredirect| is a legacy version
% of |\childdocforward| and |\childdocforwardprefix|:
%    \begin{macrocode}
\newcommand{\childdocredirect}[2][]
{
  \begingroup
    \if?#1?
      \def\childdoctmp{\childdocforward{#2}}
    \else
      \def\childdoctmp{\childdocforwardprefix{#1}{#2}}
    \fi
    \expandafter
  \endgroup
  \childdoctmp
}
%    \end{macrocode}

%\iffalse
%</package>
%\fi
%
\endinput
|\\
|\childdocby{|\textit{main}|}|\\
\end{tabular}
\end{center}
%
The directive |\childdocby| is similar to |\childdocof|
described in \secref{sec:include},
but the subsequent selection of content must be done manually.
To that end, both |\ifchilddoc| and |\ifchilddocmanual|
will be true upon processing of a part,
and the name of the part is stored in |\childdocname|.
Note that |\jobname| will be set to the filename of the current part
so that each part receives an individual |.aux| file
that does not interfere with the |.aux| file(s) of the main document.
This behaviour can be altered by the alternative form
|\childdocby[*]{|\textit{main}|}| (with a non-empty optional argument)
which uses the |.aux| file of the main document
by setting |\jobname| to \textit{main}.

%%%%%%%%%%%%%%%%%%%%%%%%%%%%%%%%%%%%%%%%%%%%%%%%%%%%%%%%%%%%%%%%%%%%%%%%%%%%%%%%
\subsection{Driver Development}
\label{sec:driver}

The \textsf{childdoc} mechanism can also be use for the development
of definition files such as \LaTeX{} styles or classes.
This case differs from the above setup with multiple parts
included by |\include| in that no |\includeonly| should be invoked.
This can be achieved by starting the include file
(before |\ProvidesPackage|) with:
%
\begin{center}
\begin{tabular}{l}
|% \iffalse
%
% childdoc.dtx Copyright (C) 2017-2018 Niklas Beisert
%
% This work may be distributed and/or modified under the
% conditions of the LaTeX Project Public License, either version 1.3
% of this license or (at your option) any later version.
% The latest version of this license is in
%   http://www.latex-project.org/lppl.txt
% and version 1.3 or later is part of all distributions of LaTeX
% version 2005/12/01 or later.
%
% This work has the LPPL maintenance status `maintained'.
%
% The Current Maintainer of this work is Niklas Beisert.
%
% This work consists of the files childdoc.dtx and childdoc.ins
% and the derived files childdoc.def and cdocsamp.tex with
% cdocsch1.tex, cdocsch2.tex, cdocsdrf.tex, cdocsfn1.tex, cdocsfn2.tex.
%
%<package>\ifdefined\childdocmain\endinput\fi
%<package>\ProvidesFile{childdoc.def}[2018/12/30 v2.0 child document driver]
%<samplemain>\ProvidesFile{cdocsamp.tex}[2018/12/30 v2.0 sample for childdoc]
%<*driver>
%\ProvidesFile{childdoc.drv}[2018/12/30 v2.0 childdoc reference manual file]
\PassOptionsToClass{10pt,a4paper}{article}
\documentclass{ltxdoc}

\usepackage[margin=35mm]{geometry}
\usepackage{hyperref}
\usepackage{hyperxmp}
\usepackage[usenames]{color}

\hypersetup{colorlinks=true}
\hypersetup{pdfstartview=FitH}
\hypersetup{pdfpagemode=UseNone}
\hypersetup{pdfsource={}}
\hypersetup{pdflang={en-UK}}
\hypersetup{pdfcopyright={Copyright 2017-2018 Niklas Beisert.
  This work may be distributed and/or modified under the
  conditions of the LaTeX Project Public License, either version 1.3
  of this license or (at your option) any later version.}}
\hypersetup{pdflicenseurl={http://www.latex-project.org/lppl.txt}}
\hypersetup{pdfcontactaddress={ETH Zurich, ITP, HIT K,
  Wolfgang-Pauli-Strasse 27}}
\hypersetup{pdfcontactpostcode={8093}}
\hypersetup{pdfcontactcity={Zurich}}
\hypersetup{pdfcontactcountry={Switzerland}}
\hypersetup{pdfcontactemail={nbeisert@itp.phys.ethz.ch}}
\hypersetup{pdfcontacturl={http://people.phys.ethz.ch/\xmptilde nbeisert/}}

\newcommand{\secref}[1]{\hyperref[#1]{section \ref*{#1}}}

\parskip1ex
\parindent0pt
\let\olditemize\itemize
\def\itemize{\olditemize\parskip0pt}

\begin{document}

\title{The \textsf{childdoc} Package}
\hypersetup{pdftitle={The childdoc Package}}
\author{Niklas Beisert\\[2ex]
  Institut f\"ur Theoretische Physik\\
  Eidgen\"ossische Technische Hochschule Z\"urich\\
  Wolfgang-Pauli-Strasse 27, 8093 Z\"urich, Switzerland\\[1ex]
  \href{mailto:nbeisert@itp.phys.ethz.ch}
  {\texttt{nbeisert@itp.phys.ethz.ch}}}
\hypersetup{pdfauthor={Niklas Beisert}}
\hypersetup{pdfsubject={Manual for the LaTeX2e Package childdoc}}
\date{30 December 2018, \textsf{v2.0}}
\maketitle

\begin{abstract}\noindent
\textsf{childdoc} is a \LaTeXe{} package
that enables the direct compilation
of document sections included by |\include|
to individual files.
\end{abstract}

\begingroup
\parskip0ex
\tableofcontents
\endgroup

%%%%%%%%%%%%%%%%%%%%%%%%%%%%%%%%%%%%%%%%%%%%%%%%%%%%%%%%%%%%%%%%%%%%%%%%%%%%%%%%
%%%%%%%%%%%%%%%%%%%%%%%%%%%%%%%%%%%%%%%%%%%%%%%%%%%%%%%%%%%%%%%%%%%%%%%%%%%%%%%%
\section{Introduction}

\LaTeX{} provides a mechanism to structure a large document (such as a book)
into a main file and several child files (containing the chapters)
using the |\include| command.
This mechanism is beneficial for documents
which span hundreds of pages in order to
make the source file(s) more manageable.
Moreover, compilation can be restricted to
selected child files by means of the |\includeonly| command.
The latter feature can be used to reduce the compilation time while editing
(this was significantly more useful in the earlier days of \LaTeX{})
or to generate a smaller document which is easier to navigate.
Another application of |\includeonly| is to generate
documents consisting of selected parts of the complete document.

However, there are a few drawbacks of the plain |\include| mechanism:
\begin{itemize}
\item
The child files cannot be compiled on their own,
they can only be compiled via the main file.
A naive editing environment
(such as a text editor with an option
to have the current file processed by \LaTeX)
may require one to switch to the main file before compiling;
attempting to compile the child file produces errors.
\item
The main file must be modified (each time)
to adjust the |\includeonly| command
to the present needs. This easily leaves the main file in a messy state.
\item
The generated document will always carry the filename
of the main document. This is inconvenient if
several child files are to be compiled and
to be kept for distribution.
\end{itemize}

The present package provides a simple interface
to make child files individually compilable by \LaTeX{}.
Compiling a child file then has the same effect as compiling
the main file with an |\includeonly| command
to select the appropriate child.
Moreover the generated document will carry the name of the child
rather than the main file.
This resolves all three above issues.

This feature is meant to make the editing of books,
thesis documents and lecture notes somewhat more convenient.
However, the package can also be used efficiently for
composing a series of documents (such as exercise sheets)
which are typically distributed individually.
It then assists the author in generating the individual documents
(potentially in different versions)
as well as a document containing the collected series.
Another application is in developing style files
or other kinds of included material
where compilation of the style file could redirect
to a sample or test file.

%%%%%%%%%%%%%%%%%%%%%%%%%%%%%%%%%%%%%%%%%%%%%%%%%%%%%%%%%%%%%%%%%%%%%%%%%%%%%%%%
%%%%%%%%%%%%%%%%%%%%%%%%%%%%%%%%%%%%%%%%%%%%%%%%%%%%%%%%%%%%%%%%%%%%%%%%%%%%%%%%
\section{Usage}

First of all, the package \textsf{childdoc} is \emph{not} a standard
\LaTeXe{} |.sty| style file! Therefore it needs to be invoked in
a non-standard way.

%%%%%%%%%%%%%%%%%%%%%%%%%%%%%%%%%%%%%%%%%%%%%%%%%%%%%%%%%%%%%%%%%%%%%%%%%%%%%%%%
\subsection{Included Files}
\label{sec:include}

%%%%%%%%%%%%%%%%%%%%%%%%%%%%%%%%%%%%%%%%
\DescribeMacro{\childdocmain}
To use the package, add the commands
\begin{center}
\begin{tabular}{l}
|\input{childdoc.def}|\\
|\childdocmain{}|\\
\end{tabular}
\end{center}
at the very top of the main \LaTeX{} file,
in particular \emph{before} the |\documentclass| statement!
The argument of |\childdocmain| should be left empty
(but it must be present).

%%%%%%%%%%%%%%%%%%%%%%%%%%%%%%%%%%%%%%%%
\DescribeMacro{\childdocof}
Furthermore, add the commands
\begin{center}
\begin{tabular}{l}
|\input{childdoc.def}|\\
|\childdocof{|\textit{main}|}|\\
\end{tabular}
\end{center}
at the top of every child file \textit{child}
which is included by |\include{|\textit{child}|}|
from within the main file
(or at least for those files to be compiled individually).
The argument \textit{main} must be the filename of the main file.

There are a couple of
considerations in setting up the main and child documents:

%%%%%%%%%%%%%%%%%%%%%%%%%%%%%%%%%%%%%%%%
\paragraph{Restrictions.}

Please note the following restrictions:
\begin{itemize}
\item
|\childdocmain| must be called with one argument \textit{main}
to ensure compatibility with earlier version of the package.
It must either be empty (|\childdocmain{}|)
or precisely match the filename of the main file in which it is specified.
See \secref{sec:detection} for further information.
\item
The filename \textit{main} must be specified without the |.tex| extension.
\item
The filename \textit{main} is case sensitive
(even in case-insensitive file systems)
due to internal string comparison.
\item
The argument \textit{main} should be fully expanded, it cannot be a macro.
\item
Subdirectories and special characters should be avoided in filenames.
\item
The command |\childdocmain{|\textit{main}|}| must be followed by a whitespace.
It should not be followed immediately by another command
or by a comment mark `|%|'.
This is because the \TeX{} parser reads the token immediately following
the argument of |\childdocmain| and puts it
at the beginning of every child section;
however, a white\-space is ignored.
\end{itemize}

%%%%%%%%%%%%%%%%%%%%%%%%%%%%%%%%%%%%%%%%
\paragraph{Content of Main File.}

It is advisable to place all content in the child files included by |\include|.
Any output contained in the main file will appear in all child documents
unless suppressed manually;
it cannot be suppressed automatically by the |\includeonly| directive
and thus should normally be avoided.
A method to include some content in the main file
by means of conditional processing is described in \secref{sec:conditional}.

%%%%%%%%%%%%%%%%%%%%%%%%%%%%%%%%%%%%%%%%
\paragraph{Page Numbering.}

When only a part of the document is compiled,
the appropriate numbering of pages
(as well as other status parameters)
is determined from the |.aux| files.
The latter contain information from previous passes.
However this information needs to propagate through
all intermediate child documents.
Therefore the page numbering in child documents may well
be inconsistent until the complete document is compiled at least once.

A useful (if unconventional) way to always ensure a consistent
page numbering is to restart the numbering in each child document
and denote the pages by `\textit{child}|.|\textit{page}'
where \textit{child} represents the chapter/section number of the child file.
This can be achieved by the command
|\numberwithin{page}{|\textit{child}|}|
of the \textsf{amsmath} package
where \textit{child} can be |chapter| or |section|
depending on the chosen structuring.
Alternatively, one can modify the macro |\thepage| appropriately
and reset the counter |page| at the start of each child file.

%%%%%%%%%%%%%%%%%%%%%%%%%%%%%%%%%%%%%%%%%%%%%%%%%%%%%%%%%%%%%%%%%%%%%%%%%%%%%%%%
\subsection{Conditional Processing}
\label{sec:conditional}

The package provides a mechanism to compile different versions
of a document. To customise the versions further some conditional processing
can come in handy to distinguish which version is being compiled.
The package provides two macros to describe the compilation context:

%%%%%%%%%%%%%%%%%%%%%%%%%%%%%%%%%%%%%%%%
\DescribeMacro{\ifchilddoc}
The conditional |\ifchilddoc| distinguishes between the compilation of
child documents and the main document:
%
\begin{center}
|\ifchilddoc |\textit{child-code}| |[|\||else |\textit{main-code}]| \||fi|
\end{center}

%%%%%%%%%%%%%%%%%%%%%%%%%%%%%%%%%%%%%%%%
\DescribeMacro{\childdocname}
\DescribeMacro{\childdocjob}
The macro |\childdocname| contains the filename (without extension)
of the main or child file being processed.
Note that |\childdocjob| will always contain the name of the main file.

%%%%%%%%%%%%%%%%%%%%%%%%%%%%%%%%%%%%%%%%
\paragraph{Title Page.}

Conditional processing can be used to include a title or banner page
in the main document when proper precautions are taken.
Importantly, the code in the main file should ensure that the page counter
(as well as other status parameters which are stored in the |.aux| files)
takes the same value after the conditional processing.
Otherwise the page numbers may take divergent values
depending on which part is compiled.

For example, a title page could be declared by:
%
\begin{center}
\begin{tabular}{l}
|\ifchilddoc\||else|\\
|\addtocounter{page}{-1}|\\
\textit{code for title page}\\
|\newpage|\\
|\||fi|
\end{tabular}
\end{center}
%
A banner page for the child documents can be generated by:
%
\begin{center}
\begin{tabular}{l}
|\ifchilddoc|\\
|\addtocounter{page}{-1}|\\
\textit{code for banner page}\\
|\newpage|\\
|\||fi|
\end{tabular}
\end{center}
%
Here one could write a message such as:
\begin{center}
|This is the part \childdocname{} of \childdocjob{}.|
\end{center}

%%%%%%%%%%%%%%%%%%%%%%%%%%%%%%%%%%%%%%%%%%%%%%%%%%%%%%%%%%%%%%%%%%%%%%%%%%%%%%%%
\subsection{Flags}
\label{sec:flags}

The package makes it easy to generate different versions
of the main or child documents.
To this end compilation flags can be defined
and assigned different default values.
They will be particularly useful in conjunction
with the forwarding mechanism described in \secref{sec:forward}.

For example, it may be useful to have a flag |\version|
which can be set to |draft| or |final|.
The document source will contain some conditional code
depending on the value of |\version|.
Suppose further, the flag should default to |final| for the main file
and to |draft| for child files
which is a natural assignment for editing the document.
This is achieved by placing the following code
in the preamble of the main document
(below the |\childdocmain| directive):
%
\begin{center}
\begin{tabular}{l}
|\ifchilddoc|\\
|\providecommand{\version}{draft}|\\
|\||else|\\
|\providecommand{\version}{final}|\\
|\||fi|
\end{tabular}
\end{center}
%
The definition by |\providecommand| makes sure
that previous definitions are not overwritten.
Further statements |\providecommand{\version}{...}|
can thus be added before the above code to override it.

For the main file, one might add a line
(between |\childdocmain| and the above block)
%
\begin{center}
|%\ifchilddoc\||else\providecommand{\version}{draft}\||fi|
\end{center}
%
which can be uncommented to produce a draft version.
Likewise one can add a line to the very top of a child file
(above the |\childdocof{|\textit{main}|}| directive)
%
\begin{center}
|%\providecommand{\version}{final}|
\end{center}
%
which can be uncommented to produce the final version of this child document.

%%%%%%%%%%%%%%%%%%%%%%%%%%%%%%%%%%%%%%%%%%%%%%%%%%%%%%%%%%%%%%%%%%%%%%%%%%%%%%%%
\subsection{Forwarding}
\label{sec:forward}

Different versions of the main or child documents
using compilation flags as described in \secref{sec:flags}
can be (permanently) stored in different files
for convenient compilation, viewing and distribution.
To this end, the package defines a command
to pass on compilation to a different file:

%%%%%%%%%%%%%%%%%%%%%%%%%%%%%%%%%%%%%%%%
\DescribeMacro{\childdocforward}
The command |\childdocforward| redirects processing to
another source file:
%
\begin{center}
\begin{tabular}{l}
|\input{childdoc.def}|\\
|\childdocforward[|\textit{main}|]{|\textit{dest}|}|\\
\end{tabular}
\end{center}
%
The argument \textit{dest} is the destination file
(without extension).
It should be the main file or one of the child files.
Note that further \textsf{childdoc} directives
such as |\childdocof| and |\childdocforward|
in the indicated file will be processed in this form.
The optional argument \textit{main}
passes on directly to the main file \textit{main}
while pretending to compile the child \textit{dest}.
This form behaves as if \textit{dest}
issues |\childdocof{|\textit{main}|}| right away,
and no further \textsf{childdoc} directives will be processed.

%%%%%%%%%%%%%%%%%%%%%%%%%%%%%%%%%%%%%%%%
\DescribeMacro{\...prefix}
In the alternative form |\childdocforwardprefix|,
%
\begin{center}
\begin{tabular}{l}
|\input{childdoc.def}|\\
|\childdocforwardprefix[|\textit{main}|]{|\textit{prefix}|}{|\textit{dest}|}|
\end{tabular}
\end{center}
%
the destination file is determined by a pattern
depending on the current file:
To make this work, the current file must be called
`{\textit{prefix}\hspace{0.2em}\textit{suffix}}'
with \textit{prefix} matching precisely the argument.
Processing is then passed on to the file
`{\textit{dest}\hspace{0.2em}\textit{suffix}}'.
Surely, the same effect is achieved by
directly specifying the
argument `{\textit{dest}\hspace{0.2em}\textit{suffix}}'
in the first form.
However, that requires to set up a different file
for each child. With the alternative form of the command
all these files can have exactly the same content
which simplifies setting them up and maintaining them.

For example, the following file |draft.tex|
with a compilation flag |\version| as described in \secref{sec:flags}
compiles the main document as a draft:
%
\begin{center}
\begin{tabular}{l}
|\def\version{draft}|\\
|\input{childdoc.def}|\\
|\childdocforward{|\textit{main}|}|
\end{tabular}
\end{center}
%
Likewise, the following files |final|\textit{nn}|.tex|
compile the final version of the child document
|child|\textit{nn}|.tex|:
%
\begin{center}
\begin{tabular}{l}
|\def\version{final}|\\
|\input{childdoc.def}|\\
|\childdocforwardprefix{final}{child}|
\end{tabular}
\end{center}
%

Note that when several versions of a main file and/or of each child file
are to be generated, it may be convenient to set up a |Makefile| or
shell script to automatise the process.

%%%%%%%%%%%%%%%%%%%%%%%%%%%%%%%%%%%%%%%%%%%%%%%%%%%%%%%%%%%%%%%%%%%%%%%%%%%%%%%%
\subsection{Command Line Processing}
\label{sec:commandline}

The effect of redirection files can also be achieved by invoking
the \LaTeX{} compiler with a more elaborate command line.
Most conveniently this should be done as part
of a shell script or a |Makefile|.

When using \textsf{childdoc} in the main file, the following
command lines effectively perform a redirection
(note that depending on the shell being used,
backslashes may have to be doubled: `|\|' $\to$ `|\\|'):
%
\begin{center}
|... -jobname "|\textit{target}|" |\\|"|[\textit{flags}]%
|\input{childdoc.def}\childdocforward[|\textit{main}|]{|\textit{dest}|}"|
\end{center}
%
Here \textit{target} is the name of the output file,
\textit{main} is the name of the main file
and \textit{dest} is the name of the main or child file to be processed
(all filenames without extensions).
The optional argument \textit{main} can be omitted
if \textit{main} matches \textit{dest}.
Optionally, compilation \textit{flags} can be defined via |\def| commands.
This command line makes the \TeX{} engine believe
it is compiling the file \textit{target}
whose content is specified as the latter parameter.
The provided code then forwards the processing to
\textit{main} or \textit{dest} as described in \secref{sec:forward}.

%%%%%%%%%%%%%%%%%%%%%%%%%%%%%%%%%%%%%%%%%%%%%%%%%%%%%%%%%%%%%%%%%%%%%%%%%%%%%%%%
\subsection{Include by Input}
\label{sec:input}

Including child documents by |\include| has some restrictions by design.
Most notably, the content of a child document always occupies
its own set of pages; pages cannot be shared between child documents.
Usually, this behaviour makes perfect sense
because each child document contain an essential part of the document.
However, in some situations it may be desirable to compose
a document from a collection of parts
without having mandatory page breaks between then.
For this case, the package
provides a mechanism to include parts
by |\input| which can also be processed individually.
However, by construction this mechanism
requires manual handling of the content to be output.

%%%%%%%%%%%%%%%%%%%%%%%%%%%%%%%%%%%%%%%%
\DescribeMacro{\ifchilddocmanual}
The main file should be prepared as usual, see \secref{sec:include}.
However, the document body must make a distinction
between processing of an individual part and of the main document, e.g.:
%
\begin{center}
\begin{tabular}{l}
|\ifchilddocmanual|\\
|\input{\childdocname}|\\
|\||else|\\
\textit{document body with }|\input{|\textit{part}|}|\\
|\||fi|
\end{tabular}
\end{center}
%
The conditional |\ifchilddocmanual| is true whenever
a part to be included by |\input| is being compiled,
and the name of the part is stored in |\childdocname|.

%%%%%%%%%%%%%%%%%%%%%%%%%%%%%%%%%%%%%%%%
\DescribeMacro{\childdocby}
Each part to be included by |\input| should start with:
%
\begin{center}
\begin{tabular}{l}
|\input{childdoc.def}|\\
|\childdocby{|\textit{main}|}|\\
\end{tabular}
\end{center}
%
The directive |\childdocby| is similar to |\childdocof|
described in \secref{sec:include},
but the subsequent selection of content must be done manually.
To that end, both |\ifchilddoc| and |\ifchilddocmanual|
will be true upon processing of a part,
and the name of the part is stored in |\childdocname|.
Note that |\jobname| will be set to the filename of the current part
so that each part receives an individual |.aux| file
that does not interfere with the |.aux| file(s) of the main document.
This behaviour can be altered by the alternative form
|\childdocby[*]{|\textit{main}|}| (with a non-empty optional argument)
which uses the |.aux| file of the main document
by setting |\jobname| to \textit{main}.

%%%%%%%%%%%%%%%%%%%%%%%%%%%%%%%%%%%%%%%%%%%%%%%%%%%%%%%%%%%%%%%%%%%%%%%%%%%%%%%%
\subsection{Driver Development}
\label{sec:driver}

The \textsf{childdoc} mechanism can also be use for the development
of definition files such as \LaTeX{} styles or classes.
This case differs from the above setup with multiple parts
included by |\include| in that no |\includeonly| should be invoked.
This can be achieved by starting the include file
(before |\ProvidesPackage|) with:
%
\begin{center}
\begin{tabular}{l}
|\input{childdoc.def}|\\
|\childdocforward{|\textit{main}|}|\\
\end{tabular}
\end{center}
%
or alternatively with:
%
\begin{center}
\begin{tabular}{l}
|\input{childdoc.def}|\\
|\childdocby{|\textit{main}|}|\\
\end{tabular}
\end{center}
%
Both forms have slightly different effects as described above.
The main file is prepared as usual, see \secref{sec:include}.

%%%%%%%%%%%%%%%%%%%%%%%%%%%%%%%%%%%%%%%%%%%%%%%%%%%%%%%%%%%%%%%%%%%%%%%%%%%%%%%%
\subsection{Legacy Detection}
\label{sec:detection}

The directive |\childdocmain| in the main file can detect
whether the complete document or merely a child is to be compiled
even without using the directive |\childdocof|.
This method is deprecated because it is less robust
and there is no compelling reason to use it;
it is merely provided for backward compatibility
and it may be removed in future versions.

If the detection mechanism is to be used,
it is mandatory to correctly specify
the filename of the main file as the argument of |\childdocmain|:
%
\begin{center}
\begin{tabular}{l}
|\input{childdoc.def}|\\
|\childdocmain{|\textit{main}|}|\\
\end{tabular}
\end{center}
%
If |\jobname| does not match the argument \textit{main} of |\childdocmain|,
it is assumed that |\jobname| points to the child file to be compiled.
When using |\childdocmain| with the main file specified as argument,
it suffices to start a child file
with just |\input{|\textit{main}|}|
without loading of the package and using |\childdocof|.
If instead all processing is done
with the appropriate \textsf{childdoc} directives,
the argument of \textit{main} of |\childdocmain| can be empty.

An alternative version of the command line processing described
in \secref{sec:commandline} using the detection mechanism reads:
%
\begin{center}
|... -jobname "|\textit{target}|" "|[\textit{flags}]%
[|\def\jobname{|\textit{dest}|}|]|\input{|\textit{main}|}"|
\end{center}

%%%%%%%%%%%%%%%%%%%%%%%%%%%%%%%%%%%%%%%%%%%%%%%%%%%%%%%%%%%%%%%%%%%%%%%%%%%%%%%%
\subsection{Manual Code}
\label{sec:manual}

In case one cannot be certain whether the definitions file |childdoc.def|
is installed on the target \TeX{} distribution
and one prefers not to ship it,
it is conceivable to paste a few relevant commands into the sources.

To that end, drop all statements |\input{childdoc.def}|
and perform the replacements as outlined below.
Instead of |\childdocmain{|\textit{main}|}| add the following code
to the top of the main file:
%
\begin{center}
\begin{tabular}{l}
|\||ifdefined\childdocname\endinput\||fi\newif\ifchilddoc|\\
|\edef\childdocname{\scantokens\expandafter{\jobname\noexpand}}|\\
|\def\childdocmain{|\textit{main}|}\||ifx\childdocmain\childdocname\||else|\\
|\childdoctrue\includeonly{\childdocname}\let\jobname\childdocmain\||fi|\\
\end{tabular}
\end{center}
%
Instead of |\childdocof{|\textit{main}|}| just include the main file
at the top of each child file:
%
\begin{center}
|\input{|\textit{main}|}|
\end{center}
%
A simple redirection |\childdocforward{|\textit{dest}|}| is achieved by:
%
\begin{center}
|\def\jobname{|\textit{dest}|}\input{\jobname}|
\end{center}
%
The redirection with prefix
|\childdocforwardprefix[|\textit{prefix}|]{|\textit{dest}|}|
is accomplished by:
%
\begin{center}
\begin{tabular}{l}
|{\edef\jobname{\scantokens\expandafter{\jobname\noexpand}}|\\
|\def\redirectjob |\textit{prefix}|#1~~~{\gdef\jobname{|\textit{dest}|#1}}|\\
|\expandafter\redirectjob\jobname~~~}\input{\jobname}|
\end{tabular}
\end{center}

In an alternative approach,
child documents can be compiled by a specific command line
without additional code or specific definitions:
%
\begin{center}
|... -jobname "|\textit{target}|" "|[\textit{flags}]%
|\includeonly{|\textit{dest}|}\input{|\textit{main}|}"|
\end{center}
%

%%%%%%%%%%%%%%%%%%%%%%%%%%%%%%%%%%%%%%%%%%%%%%%%%%%%%%%%%%%%%%%%%%%%%%%%%%%%%%%%
%%%%%%%%%%%%%%%%%%%%%%%%%%%%%%%%%%%%%%%%%%%%%%%%%%%%%%%%%%%%%%%%%%%%%%%%%%%%%%%%
\section{Information}

%%%%%%%%%%%%%%%%%%%%%%%%%%%%%%%%%%%%%%%%%%%%%%%%%%%%%%%%%%%%%%%%%%%%%%%%%%%%%%%%
\subsection{Copyright}

Copyright \copyright{} 2017--2018 Niklas Beisert

This work may be distributed and/or modified under the
conditions of the \LaTeX{} Project Public License, either version 1.3
of this license or (at your option) any later version.
The latest version of this license is in
  \url{http://www.latex-project.org/lppl.txt}
and version 1.3 or later is part of all distributions of \LaTeX{}
version 2005/12/01 or later.

This work has the LPPL maintenance status `maintained'.

The Current Maintainer of this work is Niklas Beisert.

This work consists of the files |README.txt|, |childdoc.ins| and |childdoc.dtx|
as well as the derived files |childdoc.def|, |cdocsamp.tex|
with |cdocsch1.tex|, |cdocsch2.tex|, |cdocspt3.tex|, |cdocspt4.tex|,
|cdocsdrf.tex|, |cdocsfn1.tex|, |cdocsfn2.tex|
as well as |childdoc.pdf|.

%%%%%%%%%%%%%%%%%%%%%%%%%%%%%%%%%%%%%%%%%%%%%%%%%%%%%%%%%%%%%%%%%%%%%%%%%%%%%%%%
\subsection{Files and Installation}

The package consists of the files:
%
\begin{center}
\begin{tabular}{ll}
    |README.txt|   & readme file \\
    |childdoc.ins| & installation file \\
    |childdoc.dtx| & source file \\
    |childdoc.def| & definition file \\
    |cdocsamp.tex| & sample main file \\
    |cdocsch1.tex| & sample include file \\
    |cdocsch2.tex| & sample include file \\
    |cdocspt3.tex| & sample part file \\
    |cdocspt4.tex| & sample part file \\
    |cdocsdrf.tex| & sample redirection file \\
    |cdocsfn1.tex| & sample redirection file \\
    |cdocsfn2.tex| & sample redirection file \\
    |childdoc.pdf| & manual
\end{tabular}
\end{center}
%
The distribution consists of the files
|README.txt|, |childdoc.ins| and |childdoc.dtx|.
%
\begin{itemize}
\item
Run (pdf)\LaTeX{} on |childdoc.dtx|
to compile the manual |childdoc.pdf| (this file).
\item
Run \LaTeX{} on |childdoc.ins| to create the definitions file |childdoc.def|
and the sample |cdocsamp.tex| with include files
|cdocsch1.tex|, |cdocsch2.tex|, |cdocspt3.tex|, |cdocspt4.tex|,
|cdocsdrf.tex|, |cdocsfn1.tex|, |cdocsfn2.tex|.
Then copy the file |childdoc.def| to an appropriate directory of your \LaTeX{}
distribution, e.g.\ \textit{texmf-root}|/tex/latex/childdoc|.
\end{itemize}

%%%%%%%%%%%%%%%%%%%%%%%%%%%%%%%%%%%%%%%%%%%%%%%%%%%%%%%%%%%%%%%%%%%%%%%%%%%%%%%%
\subsection{Related CTAN Packages}

There are several other packages which offer a similar functionality:
%
\begin{itemize}
\item
The packages
\href{http://ctan.org/pkg/docmute}{\textsf{docmute}},
\href{http://ctan.org/pkg/includex}{\textsf{includex}} and
\href{http://ctan.org/pkg/standalone}{\textsf{standalone}}
provide commands to include only the document body of
a child file thus allowing both files to be compiled individually.
\item
The packages \href{http://ctan.org/pkg/subdocs}{\textsf{subdocs}}
and \href{http://ctan.org/pkg/subfiles}{\textsf{subfiles}}
provide structures in which the main and child documents can be
encapsulated and allowing them to be compiled individually.
The inclusion mechanism is different from the conventional |\include|.
\item
The package \href{http://ctan.org/pkg/combine}{\textsf{combine}}
is an elaborate solution to combine several documents into one.
\end{itemize}
%
See also the CTAN topic \href{http://ctan.org/topic/subdocs}{\textsf{subdocs}}
for further related packages.
The present package differs from the above solutions in that
a document structure constructed with the conventional |\include| mechanism
just needs two extra commands at the top of every file
such that all constituent files can be compiled individually.

%%%%%%%%%%%%%%%%%%%%%%%%%%%%%%%%%%%%%%%%%%%%%%%%%%%%%%%%%%%%%%%%%%%%%%%%%%%%%%%%
%\subsection{Feature Suggestions}
%
%The following is a list of features which may be useful for future
%versions of this package:
%%
%\begin{itemize}
%\item
%\ldots
%\end{itemize}

%%%%%%%%%%%%%%%%%%%%%%%%%%%%%%%%%%%%%%%%%%%%%%%%%%%%%%%%%%%%%%%%%%%%%%%%%%%%%%%%
\subsection{Revision History}

%%%%%%%%%%%%%%%%%%%%%%%%%%%%%%%%%%%%%%%%
\paragraph{v2.0:} 2018/12/30

\begin{itemize}
\item
immediate forward processing
\item
added |\childdocby| mechanism
\item
manual restructured
\end{itemize}

%%%%%%%%%%%%%%%%%%%%%%%%%%%%%%%%%%%%%%%%
\paragraph{v1.6:} 2018/01/17

\begin{itemize}
\item
application for development of include files
\item
corrections to manual
\end{itemize}

%%%%%%%%%%%%%%%%%%%%%%%%%%%%%%%%%%%%%%%%
\paragraph{v1.5:} 2017/05/21

\begin{itemize}
\item
more complete structuring introduced
\item
|\childdocof| introduced
\item
|\childdoc| renamed to |\childdocmain|
\item
|\childredirect| renamed to |\childdocforward| and |\childdocforwardprefix|
and functionality expanded
\end{itemize}

%%%%%%%%%%%%%%%%%%%%%%%%%%%%%%%%%%%%%%%%
\paragraph{v1.0:} 2017/04/27

\begin{itemize}
\item
manual and install package
\item
first version published on CTAN
\end{itemize}

%%%%%%%%%%%%%%%%%%%%%%%%%%%%%%%%%%%%%%%%
\paragraph{v0.6:} 2017/04/26

\begin{itemize}
\item
redirection mechanism added
\end{itemize}

%%%%%%%%%%%%%%%%%%%%%%%%%%%%%%%%%%%%%%%%
\paragraph{v0.5:} 2017/04/26

\begin{itemize}
\item
functionality in definition file
\end{itemize}


%%%%%%%%%%%%%%%%%%%%%%%%%%%%%%%%%%%%%%%%%%%%%%%%%%%%%%%%%%%%%%%%%%%%%%%%%%%%%%%%
%%%%%%%%%%%%%%%%%%%%%%%%%%%%%%%%%%%%%%%%%%%%%%%%%%%%%%%%%%%%%%%%%%%%%%%%%%%%%%%%
%%%%%%%%%%%%%%%%%%%%%%%%%%%%%%%%%%%%%%%%%%%%%%%%%%%%%%%%%%%%%%%%%%%%%%%%%%%%%%%%
\appendix

\settowidth\MacroIndent{\rmfamily\scriptsize 000\ }

 \DocInput{childdoc.dtx}

\end{document}
%</driver>
% \fi
%
% %%%%%%%%%%%%%%%%%%%%%%%%%%%%%%%%%%%%%%%%%%%%%%%%%%%%%%%%%%%%%%%%%%%%%%%%%%%%%%
% %%%%%%%%%%%%%%%%%%%%%%%%%%%%%%%%%%%%%%%%%%%%%%%%%%%%%%%%%%%%%%%%%%%%%%%%%%%%%%
% \section{Sample}
%\iffalse
%<*samplemain>
%\fi
%
% The following presents a sample document
% with two chapters, two parts, a title page,
% a compile flag as well as three forwarding files to set the flag.
% It consists of eight |.tex| files:
% \begin{center}
% \begin{tabular}{ll}
% |cdocsamp.tex|&main file\\
% |cdocsch1.tex|&include file for chapter 1\\
% |cdocsch2.tex|&include file for chapter 2\\
% |cdocspt3.tex|&include file for part 3\\
% |cdocspt4.tex|&include file for part 4\\
% |cdocsdrf.tex|&forwarding file for main file in draft mode\\
% |cdocsfi1.tex|&forwarding file for final version of chapter 1\\
% |cdocsfi2.tex|&forwarding file for final version of chapter 2\\
% \end{tabular}
% \end{center}
% Each of the eight files can be compiled directly by the \LaTeX{} compiler.
%
% %%%%%%%%%%%%%%%%%%%%%%%%%%%%%%%%%%%%%%
% \paragraph{Main File.}
%
% The main file is called |cdocsamp.tex|.
%
% Load the \textsf{childdoc} definitions and
% declare the filename for the main document:
%    \begin{macrocode}
\input{childdoc.def}
\childdocmain{}
%    \end{macrocode}

% Optional override for |\version| flag:
%    \begin{macrocode}
%%\ifchilddoc\else\providecommand{\version}{draft}\fi
%    \end{macrocode}

% Define the default values for the |\version| flag
% (|final| for the main file and |draft| for childs):
%    \begin{macrocode}
\ifchilddoc
\providecommand{\version}{draft}
\else
\providecommand{\version}{final}
\fi
%    \end{macrocode}

% Load the standard document class:
%    \begin{macrocode}
\documentclass[12pt]{article}
%    \end{macrocode}

% Start the document body:
%    \begin{macrocode}
\begin{document}
%    \end{macrocode}

% Declare a title page.
% Print title, part of document being processed and version flag:
%    \begin{macrocode}
\addtocounter{page}{-1}
\begin{center}
{\LARGE\bfseries{}childdoc example\par}
\vspace{1cm}
\ifchilddoc
\ifchilddocmanual part\else chapter\fi:
`\childdocname' of `\childdocjob'\par
\else
main document: `\childdocjob'\par
\fi
version: \version\par
\end{center}
\newpage
%    \end{macrocode}

% Manually include selected file,
% otherwise process as usual:
%    \begin{macrocode}
\ifchilddocmanual
\section*{part `\childdocname'}
\input{\childdocname}
\else
%    \end{macrocode}

% Include the two chapters:
%    \begin{macrocode}
\include{cdocsch1}
\include{cdocsch2}
%    \end{macrocode}

% Include the two parts unless only chapters should be displayed:
%    \begin{macrocode}
\ifchilddoc\else
\section{part three}
\input{cdocspt3}
\section{part four}
\input{cdocspt4}
\fi
%    \end{macrocode}

% Process as usual until here:
%    \begin{macrocode}
\fi
%    \end{macrocode}

% End of document body:
%    \begin{macrocode}
\end{document}
%    \end{macrocode}
%\iffalse
%</samplemain>
%\fi
%
% %%%%%%%%%%%%%%%%%%%%%%%%%%%%%%%%%%%%%%
% \paragraph{Chapter Include Files.}
%
% The include files are called |cdocsch1.tex| and |cdocsch2.tex|.
%
%\iffalse
%<*samplechap1|samplechap2>
%\fi

% Optional override for |\version| flag:
%    \begin{macrocode}
%%\providecommand{\version}{final}
%    \end{macrocode}

% Include the main document:
%    \begin{macrocode}
\input{childdoc.def}
\childdocof{cdocsamp}
%    \end{macrocode}

%\iffalse
%</samplechap1|samplechap2>
%\fi
%
%\iffalse
%<*samplechap1>
%\fi
% Some text for chapter 1:
%    \begin{macrocode}
\section{one}
some text in chapter one
%    \end{macrocode}

%\iffalse
%</samplechap1>
%\fi
% Some text for chapter 2:
%\iffalse
%<*samplechap2>
%\fi
%    \begin{macrocode}
\section{two}
more text in chapter two
%    \end{macrocode}

%\iffalse
%</samplechap2>
%\fi
%
% %%%%%%%%%%%%%%%%%%%%%%%%%%%%%%%%%%%%%%
% \paragraph{Part Include Files.}
%
% The include files are called |cdocspt3.tex| and |cdocspt4.tex|.
%
%\iffalse
%<*samplepart3|samplepart4>
%\fi

% Optional override for |\version| flag:
%    \begin{macrocode}
%%\providecommand{\version}{final}
%    \end{macrocode}

% Include the main document:
%    \begin{macrocode}
\input{childdoc.def}
\childdocby{cdocsamp}
%    \end{macrocode}

%\iffalse
%</samplepart3|samplepart4>
%\fi
%
%\iffalse
%<*samplepart3>
%\fi
% Some text for part 3:
%    \begin{macrocode}
some text in part three
%    \end{macrocode}

%\iffalse
%</samplepart3>
%\fi
% Some text for part 4:
%\iffalse
%<*samplepart4>
%\fi
%    \begin{macrocode}
more text in part four
%    \end{macrocode}

%\iffalse
%</samplepart4>
%\fi
%
% %%%%%%%%%%%%%%%%%%%%%%%%%%%%%%%%%%%%%%
% \paragraph{Forwarding for a Complete Draft.}
%
% The following forwarding file |cdocsdrf.tex|
% compiles the main document in draft mode:
%\iffalse
%<*sampledraft>
%\fi
%    \begin{macrocode}
\def\version{draft}
\input{childdoc.def}
\childdocforward{cdocsamp}
%    \end{macrocode}

%\iffalse
%</sampledraft>
%\fi
%
% %%%%%%%%%%%%%%%%%%%%%%%%%%%%%%%%%%%%%%
% \paragraph{Forwarding for Final Version of the Chapters.}
%
% The following forwarding files |cdocsfn1.tex| and |cdocsfn2.tex|
% (with identical content)
% compile the final versions of the child documents
% |cdocsch1.tex| and |cdocsch2.tex|, respectively:
%\iffalse
%<*samplefinal>
%\fi
%    \begin{macrocode}
\def\version{final}
\input{childdoc.def}
\childdocforwardprefix[cdocsamp]{cdocsfn}{cdocsch}
%    \end{macrocode}

%\iffalse
%</samplefinal>
%\fi
%
% %%%%%%%%%%%%%%%%%%%%%%%%%%%%%%%%%%%%%%
% \paragraph{Command Line Processing.}
%
% The following three command lines generate the output files
% |cdocscld|, |cdocscl1| and |cdocscl2|
% which should be identical to
% |cdocsdrf|, |cdocsch1| and |cdocsfn2|, respectively:
% \begin{center}
% \begin{tabular}{l}
% |latex -jobname cdocscld \|\\
% |  "\def\version{draft}\input{childdoc.def}\childdocforward{cdocsamp}"|\\
% |latex -jobname cdocscl1 \|\\
% |  "\input{childdoc.def}\childdocforward[cdocsamp]{cdocsch1}"|\\
% |latex -jobname cdocscl2 \|\\
% |  "\def\version{final}\input{childdoc.def}\childdocforward{cdocsch2}"|
% \end{tabular}
% \end{center}
% Note that the trailing backslash on each first line
% merely continues the input to the second line
% (for convenient cut ant paste).
% Furthermore, the command |latex| can be replaced by any
% of its alternative versions such as |pdflatex|.
%
% %%%%%%%%%%%%%%%%%%%%%%%%%%%%%%%%%%%%%%%%%%%%%%%%%%%%%%%%%%%%%%%%%%%%%%%%%%%%%%
% %%%%%%%%%%%%%%%%%%%%%%%%%%%%%%%%%%%%%%%%%%%%%%%%%%%%%%%%%%%%%%%%%%%%%%%%%%%%%%
% \section{Implementation}
%\iffalse
%<*package>
%\fi
%
% This section describes the definitions file |childdoc.def|.

% The definitions cannot be loaded using |\usepackage| or |\RequirePackage|
% which has a mechanism to prevent loading a style file more than once.
% When loading the definitions by means of |\input|
% multiple instances have to be prevented manually:
%\iffalse
%This code needs to be before the `\ProvidesFile' directive
%which is defined at the beginning of this file.
%Therefore it is also placed there and commented out here.
%</package>
%<*discard>
%\fi
%    \begin{macrocode}
\ifdefined\childdocmain\endinput\fi
%    \end{macrocode}
%\iffalse
%</discard>
%<*package>
%\fi
%
% \macro{\ifchilddoc}
% \macro{\ifchilddocmanual}
% The conditional |\ifchilddoc| tells whether a
% child (true) or main (false) document is being compiled.
% The conditional |\ifchilddocmanual| tells whether
% the |\includeonly| mechanism is used (false) or
% the selection of child files must be performed manually (true).
% The definitions initialise to false:
%    \begin{macrocode}
\newif\ifchilddoc
\newif\ifchilddocmanual
%    \end{macrocode}

% \macro{\childdocname}
% \macro{\childdocjob}
% The macro |\childdocname| stores the name of the main document
% to be compiled. The macro |\childdocjob| stores the name of
% the document on which the \LaTeX{} compiler was originally invoked.
% The content of |\jobname| cannot be compared
% to filenames specified in the source due to different catcodes.
% The following code rescans |\jobname|, stores the result
% in |\childdocname| and saves a copy in |\childdocjob|:
%    \begin{macrocode}
\edef\childdocname{\scantokens\expandafter{\jobname\noexpand}}
\let\childdocjob\childdocname
%    \end{macrocode}

% \macro{\childdocdisable}
% The macro |\childdocdisable| prevents the main file
% from being processed more than once.
% At this stage, the main document command |\childdocmain|
% is assumed to be called once again where it should do nothing.
% Any subsequent call to it should prevent
% a secondary processing of the main document
% It overwrites the forwarding commands
% |\childdocof| and |\childdocforward|
% with empty macros to prevent further inclusions of the main document:
%    \begin{macrocode}
\newcommand{\childdocdisable}
{
  \renewcommand{\childdocmain}[1]{\renewcommand{\childdocmain}[1]{\endinput}}
  \renewcommand{\childdocof}[1]{}
  \renewcommand{\childdocby}[2][]{}
  \renewcommand{\childdocforward}[2][]{}
  \renewcommand{\childdocdisable}{}
}
%    \end{macrocode}

% \macro{\childdocmain}
% The macro |\childdocmain| is to be called at the top of the main file
% with nothing or the main filename (without extension) as argument.
% First, it breaks loops.
% If the argument is not empty and does not match |\childdocname|
% (which is set by the first inclusion of |childdoc.def|),
% |\ifchilddoc| is set to true, |\includeonly| is applied to the child file
% and |\jobname| is set to the main file
% (for proper handling of |.aux| files):
%    \begin{macrocode}
\newcommand{\childdocmain}[1]
{
  \childdocdisable\childdocmain{}
  \if?#1?\else
    \begingroup
      \def\childdoctmp{#1}
      \ifx\childdoctmp\childdocname
        \def\childdoctmp{}
      \else
        \def\childdoctmp
        {
          \childdoctrue
          \includeonly{\childdocname}
          \def\childdocjob{#1}
          \def\jobname{#1}
        }
      \fi
      \expandafter
    \endgroup
    \childdoctmp
  \fi
}
%    \end{macrocode}

% \macro{\childdocof}
% The command |\childdocof| redirects
% compilation to the main file |#1|.
%    \begin{macrocode}
\newcommand{\childdocof}[1]
{
  \childdocdisable
  \childdoctrue
  \includeonly{\childdocname}
  \def\jobname{#1}
  \def\childdocjob{#1}
  \input{#1}
}
%    \end{macrocode}

% \macro{\childdocby}
% The command |\childdocby| ....
%    \begin{macrocode}
\newcommand{\childdocby}[2][]
{
  \childdocdisable
  \childdoctrue
  \childdocmanualtrue
  \if?#1?\else
    \def\jobname{#2}
  \fi
  \def\childdocjob{#2}
  \input{#2}
  \endinput
}
%    \end{macrocode}

% \macro{\childdocforward}
% The command |\childdocforward| redirects
% compilation to the main file or
% (if the optional argument is given) a child file.
% Parameters are set as if the main file
% or a child file starting with |\childdocof| was compiled.
% Then compilation is handed over to the main file:
%    \begin{macrocode}
\newcommand{\childdocforward}[2][]
{
  \begingroup
    \if?#1?
      \def\childdoctmp
      {
        \def\childdocname{#2}
        \def\childdocjob{#2}
        \def\jobname{#2}
        \input{#2}
        \endinput
      }
    \else
      \def\childdoctmp
      {
        \childdocdisable
        \def\childdocname{#2}
        \childdoctrue
        \includeonly{#2}
        \def\childdocjob{#1}
        \def\jobname{#1}
        \input{#1}
        \endinput
      }
    \fi
    \expandafter
  \endgroup
  \childdoctmp
}
%    \end{macrocode}

% \macro{\childdocforwardprefix}
% The command |\childdocforwardprefix| redirects
% compilation to the main or a child file by means of a pattern.
% The prefix |#1| in the current filename is replaced by |#2|
% and the suffix of the current filename is kept
% (it is assumed that the filename does not contain the substring `|~~~|'
% which is used as a delimiter).
% Compilation is handed over to the new file by |\childdocforward|:
%    \begin{macrocode}
\newcommand{\childdocforwardprefix}[3][]
{
  \begingroup
    \def\childdocextract #2##1~~~{\def\childdoctmp{\childdocforward[#1]{#3##1}}}
    \expandafter\childdocextract\childdocname~~~
    \expandafter
  \endgroup
  \childdoctmp
}
%    \end{macrocode}

% \macro{\childdoc}
% The deprecated macro |\childdoc| is a legacy version of |\childdocmain|:
%    \begin{macrocode}
\newcommand{\childdoc}{\childdocmain}
%    \end{macrocode}

% \macro{\childdocredirect}
% The deprecated macro |\childdocredirect| is a legacy version
% of |\childdocforward| and |\childdocforwardprefix|:
%    \begin{macrocode}
\newcommand{\childdocredirect}[2][]
{
  \begingroup
    \if?#1?
      \def\childdoctmp{\childdocforward{#2}}
    \else
      \def\childdoctmp{\childdocforwardprefix{#1}{#2}}
    \fi
    \expandafter
  \endgroup
  \childdoctmp
}
%    \end{macrocode}

%\iffalse
%</package>
%\fi
%
\endinput
|\\
|\childdocforward{|\textit{main}|}|\\
\end{tabular}
\end{center}
%
or alternatively with:
%
\begin{center}
\begin{tabular}{l}
|% \iffalse
%
% childdoc.dtx Copyright (C) 2017-2018 Niklas Beisert
%
% This work may be distributed and/or modified under the
% conditions of the LaTeX Project Public License, either version 1.3
% of this license or (at your option) any later version.
% The latest version of this license is in
%   http://www.latex-project.org/lppl.txt
% and version 1.3 or later is part of all distributions of LaTeX
% version 2005/12/01 or later.
%
% This work has the LPPL maintenance status `maintained'.
%
% The Current Maintainer of this work is Niklas Beisert.
%
% This work consists of the files childdoc.dtx and childdoc.ins
% and the derived files childdoc.def and cdocsamp.tex with
% cdocsch1.tex, cdocsch2.tex, cdocsdrf.tex, cdocsfn1.tex, cdocsfn2.tex.
%
%<package>\ifdefined\childdocmain\endinput\fi
%<package>\ProvidesFile{childdoc.def}[2018/12/30 v2.0 child document driver]
%<samplemain>\ProvidesFile{cdocsamp.tex}[2018/12/30 v2.0 sample for childdoc]
%<*driver>
%\ProvidesFile{childdoc.drv}[2018/12/30 v2.0 childdoc reference manual file]
\PassOptionsToClass{10pt,a4paper}{article}
\documentclass{ltxdoc}

\usepackage[margin=35mm]{geometry}
\usepackage{hyperref}
\usepackage{hyperxmp}
\usepackage[usenames]{color}

\hypersetup{colorlinks=true}
\hypersetup{pdfstartview=FitH}
\hypersetup{pdfpagemode=UseNone}
\hypersetup{pdfsource={}}
\hypersetup{pdflang={en-UK}}
\hypersetup{pdfcopyright={Copyright 2017-2018 Niklas Beisert.
  This work may be distributed and/or modified under the
  conditions of the LaTeX Project Public License, either version 1.3
  of this license or (at your option) any later version.}}
\hypersetup{pdflicenseurl={http://www.latex-project.org/lppl.txt}}
\hypersetup{pdfcontactaddress={ETH Zurich, ITP, HIT K,
  Wolfgang-Pauli-Strasse 27}}
\hypersetup{pdfcontactpostcode={8093}}
\hypersetup{pdfcontactcity={Zurich}}
\hypersetup{pdfcontactcountry={Switzerland}}
\hypersetup{pdfcontactemail={nbeisert@itp.phys.ethz.ch}}
\hypersetup{pdfcontacturl={http://people.phys.ethz.ch/\xmptilde nbeisert/}}

\newcommand{\secref}[1]{\hyperref[#1]{section \ref*{#1}}}

\parskip1ex
\parindent0pt
\let\olditemize\itemize
\def\itemize{\olditemize\parskip0pt}

\begin{document}

\title{The \textsf{childdoc} Package}
\hypersetup{pdftitle={The childdoc Package}}
\author{Niklas Beisert\\[2ex]
  Institut f\"ur Theoretische Physik\\
  Eidgen\"ossische Technische Hochschule Z\"urich\\
  Wolfgang-Pauli-Strasse 27, 8093 Z\"urich, Switzerland\\[1ex]
  \href{mailto:nbeisert@itp.phys.ethz.ch}
  {\texttt{nbeisert@itp.phys.ethz.ch}}}
\hypersetup{pdfauthor={Niklas Beisert}}
\hypersetup{pdfsubject={Manual for the LaTeX2e Package childdoc}}
\date{30 December 2018, \textsf{v2.0}}
\maketitle

\begin{abstract}\noindent
\textsf{childdoc} is a \LaTeXe{} package
that enables the direct compilation
of document sections included by |\include|
to individual files.
\end{abstract}

\begingroup
\parskip0ex
\tableofcontents
\endgroup

%%%%%%%%%%%%%%%%%%%%%%%%%%%%%%%%%%%%%%%%%%%%%%%%%%%%%%%%%%%%%%%%%%%%%%%%%%%%%%%%
%%%%%%%%%%%%%%%%%%%%%%%%%%%%%%%%%%%%%%%%%%%%%%%%%%%%%%%%%%%%%%%%%%%%%%%%%%%%%%%%
\section{Introduction}

\LaTeX{} provides a mechanism to structure a large document (such as a book)
into a main file and several child files (containing the chapters)
using the |\include| command.
This mechanism is beneficial for documents
which span hundreds of pages in order to
make the source file(s) more manageable.
Moreover, compilation can be restricted to
selected child files by means of the |\includeonly| command.
The latter feature can be used to reduce the compilation time while editing
(this was significantly more useful in the earlier days of \LaTeX{})
or to generate a smaller document which is easier to navigate.
Another application of |\includeonly| is to generate
documents consisting of selected parts of the complete document.

However, there are a few drawbacks of the plain |\include| mechanism:
\begin{itemize}
\item
The child files cannot be compiled on their own,
they can only be compiled via the main file.
A naive editing environment
(such as a text editor with an option
to have the current file processed by \LaTeX)
may require one to switch to the main file before compiling;
attempting to compile the child file produces errors.
\item
The main file must be modified (each time)
to adjust the |\includeonly| command
to the present needs. This easily leaves the main file in a messy state.
\item
The generated document will always carry the filename
of the main document. This is inconvenient if
several child files are to be compiled and
to be kept for distribution.
\end{itemize}

The present package provides a simple interface
to make child files individually compilable by \LaTeX{}.
Compiling a child file then has the same effect as compiling
the main file with an |\includeonly| command
to select the appropriate child.
Moreover the generated document will carry the name of the child
rather than the main file.
This resolves all three above issues.

This feature is meant to make the editing of books,
thesis documents and lecture notes somewhat more convenient.
However, the package can also be used efficiently for
composing a series of documents (such as exercise sheets)
which are typically distributed individually.
It then assists the author in generating the individual documents
(potentially in different versions)
as well as a document containing the collected series.
Another application is in developing style files
or other kinds of included material
where compilation of the style file could redirect
to a sample or test file.

%%%%%%%%%%%%%%%%%%%%%%%%%%%%%%%%%%%%%%%%%%%%%%%%%%%%%%%%%%%%%%%%%%%%%%%%%%%%%%%%
%%%%%%%%%%%%%%%%%%%%%%%%%%%%%%%%%%%%%%%%%%%%%%%%%%%%%%%%%%%%%%%%%%%%%%%%%%%%%%%%
\section{Usage}

First of all, the package \textsf{childdoc} is \emph{not} a standard
\LaTeXe{} |.sty| style file! Therefore it needs to be invoked in
a non-standard way.

%%%%%%%%%%%%%%%%%%%%%%%%%%%%%%%%%%%%%%%%%%%%%%%%%%%%%%%%%%%%%%%%%%%%%%%%%%%%%%%%
\subsection{Included Files}
\label{sec:include}

%%%%%%%%%%%%%%%%%%%%%%%%%%%%%%%%%%%%%%%%
\DescribeMacro{\childdocmain}
To use the package, add the commands
\begin{center}
\begin{tabular}{l}
|\input{childdoc.def}|\\
|\childdocmain{}|\\
\end{tabular}
\end{center}
at the very top of the main \LaTeX{} file,
in particular \emph{before} the |\documentclass| statement!
The argument of |\childdocmain| should be left empty
(but it must be present).

%%%%%%%%%%%%%%%%%%%%%%%%%%%%%%%%%%%%%%%%
\DescribeMacro{\childdocof}
Furthermore, add the commands
\begin{center}
\begin{tabular}{l}
|\input{childdoc.def}|\\
|\childdocof{|\textit{main}|}|\\
\end{tabular}
\end{center}
at the top of every child file \textit{child}
which is included by |\include{|\textit{child}|}|
from within the main file
(or at least for those files to be compiled individually).
The argument \textit{main} must be the filename of the main file.

There are a couple of
considerations in setting up the main and child documents:

%%%%%%%%%%%%%%%%%%%%%%%%%%%%%%%%%%%%%%%%
\paragraph{Restrictions.}

Please note the following restrictions:
\begin{itemize}
\item
|\childdocmain| must be called with one argument \textit{main}
to ensure compatibility with earlier version of the package.
It must either be empty (|\childdocmain{}|)
or precisely match the filename of the main file in which it is specified.
See \secref{sec:detection} for further information.
\item
The filename \textit{main} must be specified without the |.tex| extension.
\item
The filename \textit{main} is case sensitive
(even in case-insensitive file systems)
due to internal string comparison.
\item
The argument \textit{main} should be fully expanded, it cannot be a macro.
\item
Subdirectories and special characters should be avoided in filenames.
\item
The command |\childdocmain{|\textit{main}|}| must be followed by a whitespace.
It should not be followed immediately by another command
or by a comment mark `|%|'.
This is because the \TeX{} parser reads the token immediately following
the argument of |\childdocmain| and puts it
at the beginning of every child section;
however, a white\-space is ignored.
\end{itemize}

%%%%%%%%%%%%%%%%%%%%%%%%%%%%%%%%%%%%%%%%
\paragraph{Content of Main File.}

It is advisable to place all content in the child files included by |\include|.
Any output contained in the main file will appear in all child documents
unless suppressed manually;
it cannot be suppressed automatically by the |\includeonly| directive
and thus should normally be avoided.
A method to include some content in the main file
by means of conditional processing is described in \secref{sec:conditional}.

%%%%%%%%%%%%%%%%%%%%%%%%%%%%%%%%%%%%%%%%
\paragraph{Page Numbering.}

When only a part of the document is compiled,
the appropriate numbering of pages
(as well as other status parameters)
is determined from the |.aux| files.
The latter contain information from previous passes.
However this information needs to propagate through
all intermediate child documents.
Therefore the page numbering in child documents may well
be inconsistent until the complete document is compiled at least once.

A useful (if unconventional) way to always ensure a consistent
page numbering is to restart the numbering in each child document
and denote the pages by `\textit{child}|.|\textit{page}'
where \textit{child} represents the chapter/section number of the child file.
This can be achieved by the command
|\numberwithin{page}{|\textit{child}|}|
of the \textsf{amsmath} package
where \textit{child} can be |chapter| or |section|
depending on the chosen structuring.
Alternatively, one can modify the macro |\thepage| appropriately
and reset the counter |page| at the start of each child file.

%%%%%%%%%%%%%%%%%%%%%%%%%%%%%%%%%%%%%%%%%%%%%%%%%%%%%%%%%%%%%%%%%%%%%%%%%%%%%%%%
\subsection{Conditional Processing}
\label{sec:conditional}

The package provides a mechanism to compile different versions
of a document. To customise the versions further some conditional processing
can come in handy to distinguish which version is being compiled.
The package provides two macros to describe the compilation context:

%%%%%%%%%%%%%%%%%%%%%%%%%%%%%%%%%%%%%%%%
\DescribeMacro{\ifchilddoc}
The conditional |\ifchilddoc| distinguishes between the compilation of
child documents and the main document:
%
\begin{center}
|\ifchilddoc |\textit{child-code}| |[|\||else |\textit{main-code}]| \||fi|
\end{center}

%%%%%%%%%%%%%%%%%%%%%%%%%%%%%%%%%%%%%%%%
\DescribeMacro{\childdocname}
\DescribeMacro{\childdocjob}
The macro |\childdocname| contains the filename (without extension)
of the main or child file being processed.
Note that |\childdocjob| will always contain the name of the main file.

%%%%%%%%%%%%%%%%%%%%%%%%%%%%%%%%%%%%%%%%
\paragraph{Title Page.}

Conditional processing can be used to include a title or banner page
in the main document when proper precautions are taken.
Importantly, the code in the main file should ensure that the page counter
(as well as other status parameters which are stored in the |.aux| files)
takes the same value after the conditional processing.
Otherwise the page numbers may take divergent values
depending on which part is compiled.

For example, a title page could be declared by:
%
\begin{center}
\begin{tabular}{l}
|\ifchilddoc\||else|\\
|\addtocounter{page}{-1}|\\
\textit{code for title page}\\
|\newpage|\\
|\||fi|
\end{tabular}
\end{center}
%
A banner page for the child documents can be generated by:
%
\begin{center}
\begin{tabular}{l}
|\ifchilddoc|\\
|\addtocounter{page}{-1}|\\
\textit{code for banner page}\\
|\newpage|\\
|\||fi|
\end{tabular}
\end{center}
%
Here one could write a message such as:
\begin{center}
|This is the part \childdocname{} of \childdocjob{}.|
\end{center}

%%%%%%%%%%%%%%%%%%%%%%%%%%%%%%%%%%%%%%%%%%%%%%%%%%%%%%%%%%%%%%%%%%%%%%%%%%%%%%%%
\subsection{Flags}
\label{sec:flags}

The package makes it easy to generate different versions
of the main or child documents.
To this end compilation flags can be defined
and assigned different default values.
They will be particularly useful in conjunction
with the forwarding mechanism described in \secref{sec:forward}.

For example, it may be useful to have a flag |\version|
which can be set to |draft| or |final|.
The document source will contain some conditional code
depending on the value of |\version|.
Suppose further, the flag should default to |final| for the main file
and to |draft| for child files
which is a natural assignment for editing the document.
This is achieved by placing the following code
in the preamble of the main document
(below the |\childdocmain| directive):
%
\begin{center}
\begin{tabular}{l}
|\ifchilddoc|\\
|\providecommand{\version}{draft}|\\
|\||else|\\
|\providecommand{\version}{final}|\\
|\||fi|
\end{tabular}
\end{center}
%
The definition by |\providecommand| makes sure
that previous definitions are not overwritten.
Further statements |\providecommand{\version}{...}|
can thus be added before the above code to override it.

For the main file, one might add a line
(between |\childdocmain| and the above block)
%
\begin{center}
|%\ifchilddoc\||else\providecommand{\version}{draft}\||fi|
\end{center}
%
which can be uncommented to produce a draft version.
Likewise one can add a line to the very top of a child file
(above the |\childdocof{|\textit{main}|}| directive)
%
\begin{center}
|%\providecommand{\version}{final}|
\end{center}
%
which can be uncommented to produce the final version of this child document.

%%%%%%%%%%%%%%%%%%%%%%%%%%%%%%%%%%%%%%%%%%%%%%%%%%%%%%%%%%%%%%%%%%%%%%%%%%%%%%%%
\subsection{Forwarding}
\label{sec:forward}

Different versions of the main or child documents
using compilation flags as described in \secref{sec:flags}
can be (permanently) stored in different files
for convenient compilation, viewing and distribution.
To this end, the package defines a command
to pass on compilation to a different file:

%%%%%%%%%%%%%%%%%%%%%%%%%%%%%%%%%%%%%%%%
\DescribeMacro{\childdocforward}
The command |\childdocforward| redirects processing to
another source file:
%
\begin{center}
\begin{tabular}{l}
|\input{childdoc.def}|\\
|\childdocforward[|\textit{main}|]{|\textit{dest}|}|\\
\end{tabular}
\end{center}
%
The argument \textit{dest} is the destination file
(without extension).
It should be the main file or one of the child files.
Note that further \textsf{childdoc} directives
such as |\childdocof| and |\childdocforward|
in the indicated file will be processed in this form.
The optional argument \textit{main}
passes on directly to the main file \textit{main}
while pretending to compile the child \textit{dest}.
This form behaves as if \textit{dest}
issues |\childdocof{|\textit{main}|}| right away,
and no further \textsf{childdoc} directives will be processed.

%%%%%%%%%%%%%%%%%%%%%%%%%%%%%%%%%%%%%%%%
\DescribeMacro{\...prefix}
In the alternative form |\childdocforwardprefix|,
%
\begin{center}
\begin{tabular}{l}
|\input{childdoc.def}|\\
|\childdocforwardprefix[|\textit{main}|]{|\textit{prefix}|}{|\textit{dest}|}|
\end{tabular}
\end{center}
%
the destination file is determined by a pattern
depending on the current file:
To make this work, the current file must be called
`{\textit{prefix}\hspace{0.2em}\textit{suffix}}'
with \textit{prefix} matching precisely the argument.
Processing is then passed on to the file
`{\textit{dest}\hspace{0.2em}\textit{suffix}}'.
Surely, the same effect is achieved by
directly specifying the
argument `{\textit{dest}\hspace{0.2em}\textit{suffix}}'
in the first form.
However, that requires to set up a different file
for each child. With the alternative form of the command
all these files can have exactly the same content
which simplifies setting them up and maintaining them.

For example, the following file |draft.tex|
with a compilation flag |\version| as described in \secref{sec:flags}
compiles the main document as a draft:
%
\begin{center}
\begin{tabular}{l}
|\def\version{draft}|\\
|\input{childdoc.def}|\\
|\childdocforward{|\textit{main}|}|
\end{tabular}
\end{center}
%
Likewise, the following files |final|\textit{nn}|.tex|
compile the final version of the child document
|child|\textit{nn}|.tex|:
%
\begin{center}
\begin{tabular}{l}
|\def\version{final}|\\
|\input{childdoc.def}|\\
|\childdocforwardprefix{final}{child}|
\end{tabular}
\end{center}
%

Note that when several versions of a main file and/or of each child file
are to be generated, it may be convenient to set up a |Makefile| or
shell script to automatise the process.

%%%%%%%%%%%%%%%%%%%%%%%%%%%%%%%%%%%%%%%%%%%%%%%%%%%%%%%%%%%%%%%%%%%%%%%%%%%%%%%%
\subsection{Command Line Processing}
\label{sec:commandline}

The effect of redirection files can also be achieved by invoking
the \LaTeX{} compiler with a more elaborate command line.
Most conveniently this should be done as part
of a shell script or a |Makefile|.

When using \textsf{childdoc} in the main file, the following
command lines effectively perform a redirection
(note that depending on the shell being used,
backslashes may have to be doubled: `|\|' $\to$ `|\\|'):
%
\begin{center}
|... -jobname "|\textit{target}|" |\\|"|[\textit{flags}]%
|\input{childdoc.def}\childdocforward[|\textit{main}|]{|\textit{dest}|}"|
\end{center}
%
Here \textit{target} is the name of the output file,
\textit{main} is the name of the main file
and \textit{dest} is the name of the main or child file to be processed
(all filenames without extensions).
The optional argument \textit{main} can be omitted
if \textit{main} matches \textit{dest}.
Optionally, compilation \textit{flags} can be defined via |\def| commands.
This command line makes the \TeX{} engine believe
it is compiling the file \textit{target}
whose content is specified as the latter parameter.
The provided code then forwards the processing to
\textit{main} or \textit{dest} as described in \secref{sec:forward}.

%%%%%%%%%%%%%%%%%%%%%%%%%%%%%%%%%%%%%%%%%%%%%%%%%%%%%%%%%%%%%%%%%%%%%%%%%%%%%%%%
\subsection{Include by Input}
\label{sec:input}

Including child documents by |\include| has some restrictions by design.
Most notably, the content of a child document always occupies
its own set of pages; pages cannot be shared between child documents.
Usually, this behaviour makes perfect sense
because each child document contain an essential part of the document.
However, in some situations it may be desirable to compose
a document from a collection of parts
without having mandatory page breaks between then.
For this case, the package
provides a mechanism to include parts
by |\input| which can also be processed individually.
However, by construction this mechanism
requires manual handling of the content to be output.

%%%%%%%%%%%%%%%%%%%%%%%%%%%%%%%%%%%%%%%%
\DescribeMacro{\ifchilddocmanual}
The main file should be prepared as usual, see \secref{sec:include}.
However, the document body must make a distinction
between processing of an individual part and of the main document, e.g.:
%
\begin{center}
\begin{tabular}{l}
|\ifchilddocmanual|\\
|\input{\childdocname}|\\
|\||else|\\
\textit{document body with }|\input{|\textit{part}|}|\\
|\||fi|
\end{tabular}
\end{center}
%
The conditional |\ifchilddocmanual| is true whenever
a part to be included by |\input| is being compiled,
and the name of the part is stored in |\childdocname|.

%%%%%%%%%%%%%%%%%%%%%%%%%%%%%%%%%%%%%%%%
\DescribeMacro{\childdocby}
Each part to be included by |\input| should start with:
%
\begin{center}
\begin{tabular}{l}
|\input{childdoc.def}|\\
|\childdocby{|\textit{main}|}|\\
\end{tabular}
\end{center}
%
The directive |\childdocby| is similar to |\childdocof|
described in \secref{sec:include},
but the subsequent selection of content must be done manually.
To that end, both |\ifchilddoc| and |\ifchilddocmanual|
will be true upon processing of a part,
and the name of the part is stored in |\childdocname|.
Note that |\jobname| will be set to the filename of the current part
so that each part receives an individual |.aux| file
that does not interfere with the |.aux| file(s) of the main document.
This behaviour can be altered by the alternative form
|\childdocby[*]{|\textit{main}|}| (with a non-empty optional argument)
which uses the |.aux| file of the main document
by setting |\jobname| to \textit{main}.

%%%%%%%%%%%%%%%%%%%%%%%%%%%%%%%%%%%%%%%%%%%%%%%%%%%%%%%%%%%%%%%%%%%%%%%%%%%%%%%%
\subsection{Driver Development}
\label{sec:driver}

The \textsf{childdoc} mechanism can also be use for the development
of definition files such as \LaTeX{} styles or classes.
This case differs from the above setup with multiple parts
included by |\include| in that no |\includeonly| should be invoked.
This can be achieved by starting the include file
(before |\ProvidesPackage|) with:
%
\begin{center}
\begin{tabular}{l}
|\input{childdoc.def}|\\
|\childdocforward{|\textit{main}|}|\\
\end{tabular}
\end{center}
%
or alternatively with:
%
\begin{center}
\begin{tabular}{l}
|\input{childdoc.def}|\\
|\childdocby{|\textit{main}|}|\\
\end{tabular}
\end{center}
%
Both forms have slightly different effects as described above.
The main file is prepared as usual, see \secref{sec:include}.

%%%%%%%%%%%%%%%%%%%%%%%%%%%%%%%%%%%%%%%%%%%%%%%%%%%%%%%%%%%%%%%%%%%%%%%%%%%%%%%%
\subsection{Legacy Detection}
\label{sec:detection}

The directive |\childdocmain| in the main file can detect
whether the complete document or merely a child is to be compiled
even without using the directive |\childdocof|.
This method is deprecated because it is less robust
and there is no compelling reason to use it;
it is merely provided for backward compatibility
and it may be removed in future versions.

If the detection mechanism is to be used,
it is mandatory to correctly specify
the filename of the main file as the argument of |\childdocmain|:
%
\begin{center}
\begin{tabular}{l}
|\input{childdoc.def}|\\
|\childdocmain{|\textit{main}|}|\\
\end{tabular}
\end{center}
%
If |\jobname| does not match the argument \textit{main} of |\childdocmain|,
it is assumed that |\jobname| points to the child file to be compiled.
When using |\childdocmain| with the main file specified as argument,
it suffices to start a child file
with just |\input{|\textit{main}|}|
without loading of the package and using |\childdocof|.
If instead all processing is done
with the appropriate \textsf{childdoc} directives,
the argument of \textit{main} of |\childdocmain| can be empty.

An alternative version of the command line processing described
in \secref{sec:commandline} using the detection mechanism reads:
%
\begin{center}
|... -jobname "|\textit{target}|" "|[\textit{flags}]%
[|\def\jobname{|\textit{dest}|}|]|\input{|\textit{main}|}"|
\end{center}

%%%%%%%%%%%%%%%%%%%%%%%%%%%%%%%%%%%%%%%%%%%%%%%%%%%%%%%%%%%%%%%%%%%%%%%%%%%%%%%%
\subsection{Manual Code}
\label{sec:manual}

In case one cannot be certain whether the definitions file |childdoc.def|
is installed on the target \TeX{} distribution
and one prefers not to ship it,
it is conceivable to paste a few relevant commands into the sources.

To that end, drop all statements |\input{childdoc.def}|
and perform the replacements as outlined below.
Instead of |\childdocmain{|\textit{main}|}| add the following code
to the top of the main file:
%
\begin{center}
\begin{tabular}{l}
|\||ifdefined\childdocname\endinput\||fi\newif\ifchilddoc|\\
|\edef\childdocname{\scantokens\expandafter{\jobname\noexpand}}|\\
|\def\childdocmain{|\textit{main}|}\||ifx\childdocmain\childdocname\||else|\\
|\childdoctrue\includeonly{\childdocname}\let\jobname\childdocmain\||fi|\\
\end{tabular}
\end{center}
%
Instead of |\childdocof{|\textit{main}|}| just include the main file
at the top of each child file:
%
\begin{center}
|\input{|\textit{main}|}|
\end{center}
%
A simple redirection |\childdocforward{|\textit{dest}|}| is achieved by:
%
\begin{center}
|\def\jobname{|\textit{dest}|}\input{\jobname}|
\end{center}
%
The redirection with prefix
|\childdocforwardprefix[|\textit{prefix}|]{|\textit{dest}|}|
is accomplished by:
%
\begin{center}
\begin{tabular}{l}
|{\edef\jobname{\scantokens\expandafter{\jobname\noexpand}}|\\
|\def\redirectjob |\textit{prefix}|#1~~~{\gdef\jobname{|\textit{dest}|#1}}|\\
|\expandafter\redirectjob\jobname~~~}\input{\jobname}|
\end{tabular}
\end{center}

In an alternative approach,
child documents can be compiled by a specific command line
without additional code or specific definitions:
%
\begin{center}
|... -jobname "|\textit{target}|" "|[\textit{flags}]%
|\includeonly{|\textit{dest}|}\input{|\textit{main}|}"|
\end{center}
%

%%%%%%%%%%%%%%%%%%%%%%%%%%%%%%%%%%%%%%%%%%%%%%%%%%%%%%%%%%%%%%%%%%%%%%%%%%%%%%%%
%%%%%%%%%%%%%%%%%%%%%%%%%%%%%%%%%%%%%%%%%%%%%%%%%%%%%%%%%%%%%%%%%%%%%%%%%%%%%%%%
\section{Information}

%%%%%%%%%%%%%%%%%%%%%%%%%%%%%%%%%%%%%%%%%%%%%%%%%%%%%%%%%%%%%%%%%%%%%%%%%%%%%%%%
\subsection{Copyright}

Copyright \copyright{} 2017--2018 Niklas Beisert

This work may be distributed and/or modified under the
conditions of the \LaTeX{} Project Public License, either version 1.3
of this license or (at your option) any later version.
The latest version of this license is in
  \url{http://www.latex-project.org/lppl.txt}
and version 1.3 or later is part of all distributions of \LaTeX{}
version 2005/12/01 or later.

This work has the LPPL maintenance status `maintained'.

The Current Maintainer of this work is Niklas Beisert.

This work consists of the files |README.txt|, |childdoc.ins| and |childdoc.dtx|
as well as the derived files |childdoc.def|, |cdocsamp.tex|
with |cdocsch1.tex|, |cdocsch2.tex|, |cdocspt3.tex|, |cdocspt4.tex|,
|cdocsdrf.tex|, |cdocsfn1.tex|, |cdocsfn2.tex|
as well as |childdoc.pdf|.

%%%%%%%%%%%%%%%%%%%%%%%%%%%%%%%%%%%%%%%%%%%%%%%%%%%%%%%%%%%%%%%%%%%%%%%%%%%%%%%%
\subsection{Files and Installation}

The package consists of the files:
%
\begin{center}
\begin{tabular}{ll}
    |README.txt|   & readme file \\
    |childdoc.ins| & installation file \\
    |childdoc.dtx| & source file \\
    |childdoc.def| & definition file \\
    |cdocsamp.tex| & sample main file \\
    |cdocsch1.tex| & sample include file \\
    |cdocsch2.tex| & sample include file \\
    |cdocspt3.tex| & sample part file \\
    |cdocspt4.tex| & sample part file \\
    |cdocsdrf.tex| & sample redirection file \\
    |cdocsfn1.tex| & sample redirection file \\
    |cdocsfn2.tex| & sample redirection file \\
    |childdoc.pdf| & manual
\end{tabular}
\end{center}
%
The distribution consists of the files
|README.txt|, |childdoc.ins| and |childdoc.dtx|.
%
\begin{itemize}
\item
Run (pdf)\LaTeX{} on |childdoc.dtx|
to compile the manual |childdoc.pdf| (this file).
\item
Run \LaTeX{} on |childdoc.ins| to create the definitions file |childdoc.def|
and the sample |cdocsamp.tex| with include files
|cdocsch1.tex|, |cdocsch2.tex|, |cdocspt3.tex|, |cdocspt4.tex|,
|cdocsdrf.tex|, |cdocsfn1.tex|, |cdocsfn2.tex|.
Then copy the file |childdoc.def| to an appropriate directory of your \LaTeX{}
distribution, e.g.\ \textit{texmf-root}|/tex/latex/childdoc|.
\end{itemize}

%%%%%%%%%%%%%%%%%%%%%%%%%%%%%%%%%%%%%%%%%%%%%%%%%%%%%%%%%%%%%%%%%%%%%%%%%%%%%%%%
\subsection{Related CTAN Packages}

There are several other packages which offer a similar functionality:
%
\begin{itemize}
\item
The packages
\href{http://ctan.org/pkg/docmute}{\textsf{docmute}},
\href{http://ctan.org/pkg/includex}{\textsf{includex}} and
\href{http://ctan.org/pkg/standalone}{\textsf{standalone}}
provide commands to include only the document body of
a child file thus allowing both files to be compiled individually.
\item
The packages \href{http://ctan.org/pkg/subdocs}{\textsf{subdocs}}
and \href{http://ctan.org/pkg/subfiles}{\textsf{subfiles}}
provide structures in which the main and child documents can be
encapsulated and allowing them to be compiled individually.
The inclusion mechanism is different from the conventional |\include|.
\item
The package \href{http://ctan.org/pkg/combine}{\textsf{combine}}
is an elaborate solution to combine several documents into one.
\end{itemize}
%
See also the CTAN topic \href{http://ctan.org/topic/subdocs}{\textsf{subdocs}}
for further related packages.
The present package differs from the above solutions in that
a document structure constructed with the conventional |\include| mechanism
just needs two extra commands at the top of every file
such that all constituent files can be compiled individually.

%%%%%%%%%%%%%%%%%%%%%%%%%%%%%%%%%%%%%%%%%%%%%%%%%%%%%%%%%%%%%%%%%%%%%%%%%%%%%%%%
%\subsection{Feature Suggestions}
%
%The following is a list of features which may be useful for future
%versions of this package:
%%
%\begin{itemize}
%\item
%\ldots
%\end{itemize}

%%%%%%%%%%%%%%%%%%%%%%%%%%%%%%%%%%%%%%%%%%%%%%%%%%%%%%%%%%%%%%%%%%%%%%%%%%%%%%%%
\subsection{Revision History}

%%%%%%%%%%%%%%%%%%%%%%%%%%%%%%%%%%%%%%%%
\paragraph{v2.0:} 2018/12/30

\begin{itemize}
\item
immediate forward processing
\item
added |\childdocby| mechanism
\item
manual restructured
\end{itemize}

%%%%%%%%%%%%%%%%%%%%%%%%%%%%%%%%%%%%%%%%
\paragraph{v1.6:} 2018/01/17

\begin{itemize}
\item
application for development of include files
\item
corrections to manual
\end{itemize}

%%%%%%%%%%%%%%%%%%%%%%%%%%%%%%%%%%%%%%%%
\paragraph{v1.5:} 2017/05/21

\begin{itemize}
\item
more complete structuring introduced
\item
|\childdocof| introduced
\item
|\childdoc| renamed to |\childdocmain|
\item
|\childredirect| renamed to |\childdocforward| and |\childdocforwardprefix|
and functionality expanded
\end{itemize}

%%%%%%%%%%%%%%%%%%%%%%%%%%%%%%%%%%%%%%%%
\paragraph{v1.0:} 2017/04/27

\begin{itemize}
\item
manual and install package
\item
first version published on CTAN
\end{itemize}

%%%%%%%%%%%%%%%%%%%%%%%%%%%%%%%%%%%%%%%%
\paragraph{v0.6:} 2017/04/26

\begin{itemize}
\item
redirection mechanism added
\end{itemize}

%%%%%%%%%%%%%%%%%%%%%%%%%%%%%%%%%%%%%%%%
\paragraph{v0.5:} 2017/04/26

\begin{itemize}
\item
functionality in definition file
\end{itemize}


%%%%%%%%%%%%%%%%%%%%%%%%%%%%%%%%%%%%%%%%%%%%%%%%%%%%%%%%%%%%%%%%%%%%%%%%%%%%%%%%
%%%%%%%%%%%%%%%%%%%%%%%%%%%%%%%%%%%%%%%%%%%%%%%%%%%%%%%%%%%%%%%%%%%%%%%%%%%%%%%%
%%%%%%%%%%%%%%%%%%%%%%%%%%%%%%%%%%%%%%%%%%%%%%%%%%%%%%%%%%%%%%%%%%%%%%%%%%%%%%%%
\appendix

\settowidth\MacroIndent{\rmfamily\scriptsize 000\ }

 \DocInput{childdoc.dtx}

\end{document}
%</driver>
% \fi
%
% %%%%%%%%%%%%%%%%%%%%%%%%%%%%%%%%%%%%%%%%%%%%%%%%%%%%%%%%%%%%%%%%%%%%%%%%%%%%%%
% %%%%%%%%%%%%%%%%%%%%%%%%%%%%%%%%%%%%%%%%%%%%%%%%%%%%%%%%%%%%%%%%%%%%%%%%%%%%%%
% \section{Sample}
%\iffalse
%<*samplemain>
%\fi
%
% The following presents a sample document
% with two chapters, two parts, a title page,
% a compile flag as well as three forwarding files to set the flag.
% It consists of eight |.tex| files:
% \begin{center}
% \begin{tabular}{ll}
% |cdocsamp.tex|&main file\\
% |cdocsch1.tex|&include file for chapter 1\\
% |cdocsch2.tex|&include file for chapter 2\\
% |cdocspt3.tex|&include file for part 3\\
% |cdocspt4.tex|&include file for part 4\\
% |cdocsdrf.tex|&forwarding file for main file in draft mode\\
% |cdocsfi1.tex|&forwarding file for final version of chapter 1\\
% |cdocsfi2.tex|&forwarding file for final version of chapter 2\\
% \end{tabular}
% \end{center}
% Each of the eight files can be compiled directly by the \LaTeX{} compiler.
%
% %%%%%%%%%%%%%%%%%%%%%%%%%%%%%%%%%%%%%%
% \paragraph{Main File.}
%
% The main file is called |cdocsamp.tex|.
%
% Load the \textsf{childdoc} definitions and
% declare the filename for the main document:
%    \begin{macrocode}
\input{childdoc.def}
\childdocmain{}
%    \end{macrocode}

% Optional override for |\version| flag:
%    \begin{macrocode}
%%\ifchilddoc\else\providecommand{\version}{draft}\fi
%    \end{macrocode}

% Define the default values for the |\version| flag
% (|final| for the main file and |draft| for childs):
%    \begin{macrocode}
\ifchilddoc
\providecommand{\version}{draft}
\else
\providecommand{\version}{final}
\fi
%    \end{macrocode}

% Load the standard document class:
%    \begin{macrocode}
\documentclass[12pt]{article}
%    \end{macrocode}

% Start the document body:
%    \begin{macrocode}
\begin{document}
%    \end{macrocode}

% Declare a title page.
% Print title, part of document being processed and version flag:
%    \begin{macrocode}
\addtocounter{page}{-1}
\begin{center}
{\LARGE\bfseries{}childdoc example\par}
\vspace{1cm}
\ifchilddoc
\ifchilddocmanual part\else chapter\fi:
`\childdocname' of `\childdocjob'\par
\else
main document: `\childdocjob'\par
\fi
version: \version\par
\end{center}
\newpage
%    \end{macrocode}

% Manually include selected file,
% otherwise process as usual:
%    \begin{macrocode}
\ifchilddocmanual
\section*{part `\childdocname'}
\input{\childdocname}
\else
%    \end{macrocode}

% Include the two chapters:
%    \begin{macrocode}
\include{cdocsch1}
\include{cdocsch2}
%    \end{macrocode}

% Include the two parts unless only chapters should be displayed:
%    \begin{macrocode}
\ifchilddoc\else
\section{part three}
\input{cdocspt3}
\section{part four}
\input{cdocspt4}
\fi
%    \end{macrocode}

% Process as usual until here:
%    \begin{macrocode}
\fi
%    \end{macrocode}

% End of document body:
%    \begin{macrocode}
\end{document}
%    \end{macrocode}
%\iffalse
%</samplemain>
%\fi
%
% %%%%%%%%%%%%%%%%%%%%%%%%%%%%%%%%%%%%%%
% \paragraph{Chapter Include Files.}
%
% The include files are called |cdocsch1.tex| and |cdocsch2.tex|.
%
%\iffalse
%<*samplechap1|samplechap2>
%\fi

% Optional override for |\version| flag:
%    \begin{macrocode}
%%\providecommand{\version}{final}
%    \end{macrocode}

% Include the main document:
%    \begin{macrocode}
\input{childdoc.def}
\childdocof{cdocsamp}
%    \end{macrocode}

%\iffalse
%</samplechap1|samplechap2>
%\fi
%
%\iffalse
%<*samplechap1>
%\fi
% Some text for chapter 1:
%    \begin{macrocode}
\section{one}
some text in chapter one
%    \end{macrocode}

%\iffalse
%</samplechap1>
%\fi
% Some text for chapter 2:
%\iffalse
%<*samplechap2>
%\fi
%    \begin{macrocode}
\section{two}
more text in chapter two
%    \end{macrocode}

%\iffalse
%</samplechap2>
%\fi
%
% %%%%%%%%%%%%%%%%%%%%%%%%%%%%%%%%%%%%%%
% \paragraph{Part Include Files.}
%
% The include files are called |cdocspt3.tex| and |cdocspt4.tex|.
%
%\iffalse
%<*samplepart3|samplepart4>
%\fi

% Optional override for |\version| flag:
%    \begin{macrocode}
%%\providecommand{\version}{final}
%    \end{macrocode}

% Include the main document:
%    \begin{macrocode}
\input{childdoc.def}
\childdocby{cdocsamp}
%    \end{macrocode}

%\iffalse
%</samplepart3|samplepart4>
%\fi
%
%\iffalse
%<*samplepart3>
%\fi
% Some text for part 3:
%    \begin{macrocode}
some text in part three
%    \end{macrocode}

%\iffalse
%</samplepart3>
%\fi
% Some text for part 4:
%\iffalse
%<*samplepart4>
%\fi
%    \begin{macrocode}
more text in part four
%    \end{macrocode}

%\iffalse
%</samplepart4>
%\fi
%
% %%%%%%%%%%%%%%%%%%%%%%%%%%%%%%%%%%%%%%
% \paragraph{Forwarding for a Complete Draft.}
%
% The following forwarding file |cdocsdrf.tex|
% compiles the main document in draft mode:
%\iffalse
%<*sampledraft>
%\fi
%    \begin{macrocode}
\def\version{draft}
\input{childdoc.def}
\childdocforward{cdocsamp}
%    \end{macrocode}

%\iffalse
%</sampledraft>
%\fi
%
% %%%%%%%%%%%%%%%%%%%%%%%%%%%%%%%%%%%%%%
% \paragraph{Forwarding for Final Version of the Chapters.}
%
% The following forwarding files |cdocsfn1.tex| and |cdocsfn2.tex|
% (with identical content)
% compile the final versions of the child documents
% |cdocsch1.tex| and |cdocsch2.tex|, respectively:
%\iffalse
%<*samplefinal>
%\fi
%    \begin{macrocode}
\def\version{final}
\input{childdoc.def}
\childdocforwardprefix[cdocsamp]{cdocsfn}{cdocsch}
%    \end{macrocode}

%\iffalse
%</samplefinal>
%\fi
%
% %%%%%%%%%%%%%%%%%%%%%%%%%%%%%%%%%%%%%%
% \paragraph{Command Line Processing.}
%
% The following three command lines generate the output files
% |cdocscld|, |cdocscl1| and |cdocscl2|
% which should be identical to
% |cdocsdrf|, |cdocsch1| and |cdocsfn2|, respectively:
% \begin{center}
% \begin{tabular}{l}
% |latex -jobname cdocscld \|\\
% |  "\def\version{draft}\input{childdoc.def}\childdocforward{cdocsamp}"|\\
% |latex -jobname cdocscl1 \|\\
% |  "\input{childdoc.def}\childdocforward[cdocsamp]{cdocsch1}"|\\
% |latex -jobname cdocscl2 \|\\
% |  "\def\version{final}\input{childdoc.def}\childdocforward{cdocsch2}"|
% \end{tabular}
% \end{center}
% Note that the trailing backslash on each first line
% merely continues the input to the second line
% (for convenient cut ant paste).
% Furthermore, the command |latex| can be replaced by any
% of its alternative versions such as |pdflatex|.
%
% %%%%%%%%%%%%%%%%%%%%%%%%%%%%%%%%%%%%%%%%%%%%%%%%%%%%%%%%%%%%%%%%%%%%%%%%%%%%%%
% %%%%%%%%%%%%%%%%%%%%%%%%%%%%%%%%%%%%%%%%%%%%%%%%%%%%%%%%%%%%%%%%%%%%%%%%%%%%%%
% \section{Implementation}
%\iffalse
%<*package>
%\fi
%
% This section describes the definitions file |childdoc.def|.

% The definitions cannot be loaded using |\usepackage| or |\RequirePackage|
% which has a mechanism to prevent loading a style file more than once.
% When loading the definitions by means of |\input|
% multiple instances have to be prevented manually:
%\iffalse
%This code needs to be before the `\ProvidesFile' directive
%which is defined at the beginning of this file.
%Therefore it is also placed there and commented out here.
%</package>
%<*discard>
%\fi
%    \begin{macrocode}
\ifdefined\childdocmain\endinput\fi
%    \end{macrocode}
%\iffalse
%</discard>
%<*package>
%\fi
%
% \macro{\ifchilddoc}
% \macro{\ifchilddocmanual}
% The conditional |\ifchilddoc| tells whether a
% child (true) or main (false) document is being compiled.
% The conditional |\ifchilddocmanual| tells whether
% the |\includeonly| mechanism is used (false) or
% the selection of child files must be performed manually (true).
% The definitions initialise to false:
%    \begin{macrocode}
\newif\ifchilddoc
\newif\ifchilddocmanual
%    \end{macrocode}

% \macro{\childdocname}
% \macro{\childdocjob}
% The macro |\childdocname| stores the name of the main document
% to be compiled. The macro |\childdocjob| stores the name of
% the document on which the \LaTeX{} compiler was originally invoked.
% The content of |\jobname| cannot be compared
% to filenames specified in the source due to different catcodes.
% The following code rescans |\jobname|, stores the result
% in |\childdocname| and saves a copy in |\childdocjob|:
%    \begin{macrocode}
\edef\childdocname{\scantokens\expandafter{\jobname\noexpand}}
\let\childdocjob\childdocname
%    \end{macrocode}

% \macro{\childdocdisable}
% The macro |\childdocdisable| prevents the main file
% from being processed more than once.
% At this stage, the main document command |\childdocmain|
% is assumed to be called once again where it should do nothing.
% Any subsequent call to it should prevent
% a secondary processing of the main document
% It overwrites the forwarding commands
% |\childdocof| and |\childdocforward|
% with empty macros to prevent further inclusions of the main document:
%    \begin{macrocode}
\newcommand{\childdocdisable}
{
  \renewcommand{\childdocmain}[1]{\renewcommand{\childdocmain}[1]{\endinput}}
  \renewcommand{\childdocof}[1]{}
  \renewcommand{\childdocby}[2][]{}
  \renewcommand{\childdocforward}[2][]{}
  \renewcommand{\childdocdisable}{}
}
%    \end{macrocode}

% \macro{\childdocmain}
% The macro |\childdocmain| is to be called at the top of the main file
% with nothing or the main filename (without extension) as argument.
% First, it breaks loops.
% If the argument is not empty and does not match |\childdocname|
% (which is set by the first inclusion of |childdoc.def|),
% |\ifchilddoc| is set to true, |\includeonly| is applied to the child file
% and |\jobname| is set to the main file
% (for proper handling of |.aux| files):
%    \begin{macrocode}
\newcommand{\childdocmain}[1]
{
  \childdocdisable\childdocmain{}
  \if?#1?\else
    \begingroup
      \def\childdoctmp{#1}
      \ifx\childdoctmp\childdocname
        \def\childdoctmp{}
      \else
        \def\childdoctmp
        {
          \childdoctrue
          \includeonly{\childdocname}
          \def\childdocjob{#1}
          \def\jobname{#1}
        }
      \fi
      \expandafter
    \endgroup
    \childdoctmp
  \fi
}
%    \end{macrocode}

% \macro{\childdocof}
% The command |\childdocof| redirects
% compilation to the main file |#1|.
%    \begin{macrocode}
\newcommand{\childdocof}[1]
{
  \childdocdisable
  \childdoctrue
  \includeonly{\childdocname}
  \def\jobname{#1}
  \def\childdocjob{#1}
  \input{#1}
}
%    \end{macrocode}

% \macro{\childdocby}
% The command |\childdocby| ....
%    \begin{macrocode}
\newcommand{\childdocby}[2][]
{
  \childdocdisable
  \childdoctrue
  \childdocmanualtrue
  \if?#1?\else
    \def\jobname{#2}
  \fi
  \def\childdocjob{#2}
  \input{#2}
  \endinput
}
%    \end{macrocode}

% \macro{\childdocforward}
% The command |\childdocforward| redirects
% compilation to the main file or
% (if the optional argument is given) a child file.
% Parameters are set as if the main file
% or a child file starting with |\childdocof| was compiled.
% Then compilation is handed over to the main file:
%    \begin{macrocode}
\newcommand{\childdocforward}[2][]
{
  \begingroup
    \if?#1?
      \def\childdoctmp
      {
        \def\childdocname{#2}
        \def\childdocjob{#2}
        \def\jobname{#2}
        \input{#2}
        \endinput
      }
    \else
      \def\childdoctmp
      {
        \childdocdisable
        \def\childdocname{#2}
        \childdoctrue
        \includeonly{#2}
        \def\childdocjob{#1}
        \def\jobname{#1}
        \input{#1}
        \endinput
      }
    \fi
    \expandafter
  \endgroup
  \childdoctmp
}
%    \end{macrocode}

% \macro{\childdocforwardprefix}
% The command |\childdocforwardprefix| redirects
% compilation to the main or a child file by means of a pattern.
% The prefix |#1| in the current filename is replaced by |#2|
% and the suffix of the current filename is kept
% (it is assumed that the filename does not contain the substring `|~~~|'
% which is used as a delimiter).
% Compilation is handed over to the new file by |\childdocforward|:
%    \begin{macrocode}
\newcommand{\childdocforwardprefix}[3][]
{
  \begingroup
    \def\childdocextract #2##1~~~{\def\childdoctmp{\childdocforward[#1]{#3##1}}}
    \expandafter\childdocextract\childdocname~~~
    \expandafter
  \endgroup
  \childdoctmp
}
%    \end{macrocode}

% \macro{\childdoc}
% The deprecated macro |\childdoc| is a legacy version of |\childdocmain|:
%    \begin{macrocode}
\newcommand{\childdoc}{\childdocmain}
%    \end{macrocode}

% \macro{\childdocredirect}
% The deprecated macro |\childdocredirect| is a legacy version
% of |\childdocforward| and |\childdocforwardprefix|:
%    \begin{macrocode}
\newcommand{\childdocredirect}[2][]
{
  \begingroup
    \if?#1?
      \def\childdoctmp{\childdocforward{#2}}
    \else
      \def\childdoctmp{\childdocforwardprefix{#1}{#2}}
    \fi
    \expandafter
  \endgroup
  \childdoctmp
}
%    \end{macrocode}

%\iffalse
%</package>
%\fi
%
\endinput
|\\
|\childdocby{|\textit{main}|}|\\
\end{tabular}
\end{center}
%
Both forms have slightly different effects as described above.
The main file is prepared as usual, see \secref{sec:include}.

%%%%%%%%%%%%%%%%%%%%%%%%%%%%%%%%%%%%%%%%%%%%%%%%%%%%%%%%%%%%%%%%%%%%%%%%%%%%%%%%
\subsection{Legacy Detection}
\label{sec:detection}

The directive |\childdocmain| in the main file can detect
whether the complete document or merely a child is to be compiled
even without using the directive |\childdocof|.
This method is deprecated because it is less robust
and there is no compelling reason to use it;
it is merely provided for backward compatibility
and it may be removed in future versions.

If the detection mechanism is to be used,
it is mandatory to correctly specify
the filename of the main file as the argument of |\childdocmain|:
%
\begin{center}
\begin{tabular}{l}
|% \iffalse
%
% childdoc.dtx Copyright (C) 2017-2018 Niklas Beisert
%
% This work may be distributed and/or modified under the
% conditions of the LaTeX Project Public License, either version 1.3
% of this license or (at your option) any later version.
% The latest version of this license is in
%   http://www.latex-project.org/lppl.txt
% and version 1.3 or later is part of all distributions of LaTeX
% version 2005/12/01 or later.
%
% This work has the LPPL maintenance status `maintained'.
%
% The Current Maintainer of this work is Niklas Beisert.
%
% This work consists of the files childdoc.dtx and childdoc.ins
% and the derived files childdoc.def and cdocsamp.tex with
% cdocsch1.tex, cdocsch2.tex, cdocsdrf.tex, cdocsfn1.tex, cdocsfn2.tex.
%
%<package>\ifdefined\childdocmain\endinput\fi
%<package>\ProvidesFile{childdoc.def}[2018/12/30 v2.0 child document driver]
%<samplemain>\ProvidesFile{cdocsamp.tex}[2018/12/30 v2.0 sample for childdoc]
%<*driver>
%\ProvidesFile{childdoc.drv}[2018/12/30 v2.0 childdoc reference manual file]
\PassOptionsToClass{10pt,a4paper}{article}
\documentclass{ltxdoc}

\usepackage[margin=35mm]{geometry}
\usepackage{hyperref}
\usepackage{hyperxmp}
\usepackage[usenames]{color}

\hypersetup{colorlinks=true}
\hypersetup{pdfstartview=FitH}
\hypersetup{pdfpagemode=UseNone}
\hypersetup{pdfsource={}}
\hypersetup{pdflang={en-UK}}
\hypersetup{pdfcopyright={Copyright 2017-2018 Niklas Beisert.
  This work may be distributed and/or modified under the
  conditions of the LaTeX Project Public License, either version 1.3
  of this license or (at your option) any later version.}}
\hypersetup{pdflicenseurl={http://www.latex-project.org/lppl.txt}}
\hypersetup{pdfcontactaddress={ETH Zurich, ITP, HIT K,
  Wolfgang-Pauli-Strasse 27}}
\hypersetup{pdfcontactpostcode={8093}}
\hypersetup{pdfcontactcity={Zurich}}
\hypersetup{pdfcontactcountry={Switzerland}}
\hypersetup{pdfcontactemail={nbeisert@itp.phys.ethz.ch}}
\hypersetup{pdfcontacturl={http://people.phys.ethz.ch/\xmptilde nbeisert/}}

\newcommand{\secref}[1]{\hyperref[#1]{section \ref*{#1}}}

\parskip1ex
\parindent0pt
\let\olditemize\itemize
\def\itemize{\olditemize\parskip0pt}

\begin{document}

\title{The \textsf{childdoc} Package}
\hypersetup{pdftitle={The childdoc Package}}
\author{Niklas Beisert\\[2ex]
  Institut f\"ur Theoretische Physik\\
  Eidgen\"ossische Technische Hochschule Z\"urich\\
  Wolfgang-Pauli-Strasse 27, 8093 Z\"urich, Switzerland\\[1ex]
  \href{mailto:nbeisert@itp.phys.ethz.ch}
  {\texttt{nbeisert@itp.phys.ethz.ch}}}
\hypersetup{pdfauthor={Niklas Beisert}}
\hypersetup{pdfsubject={Manual for the LaTeX2e Package childdoc}}
\date{30 December 2018, \textsf{v2.0}}
\maketitle

\begin{abstract}\noindent
\textsf{childdoc} is a \LaTeXe{} package
that enables the direct compilation
of document sections included by |\include|
to individual files.
\end{abstract}

\begingroup
\parskip0ex
\tableofcontents
\endgroup

%%%%%%%%%%%%%%%%%%%%%%%%%%%%%%%%%%%%%%%%%%%%%%%%%%%%%%%%%%%%%%%%%%%%%%%%%%%%%%%%
%%%%%%%%%%%%%%%%%%%%%%%%%%%%%%%%%%%%%%%%%%%%%%%%%%%%%%%%%%%%%%%%%%%%%%%%%%%%%%%%
\section{Introduction}

\LaTeX{} provides a mechanism to structure a large document (such as a book)
into a main file and several child files (containing the chapters)
using the |\include| command.
This mechanism is beneficial for documents
which span hundreds of pages in order to
make the source file(s) more manageable.
Moreover, compilation can be restricted to
selected child files by means of the |\includeonly| command.
The latter feature can be used to reduce the compilation time while editing
(this was significantly more useful in the earlier days of \LaTeX{})
or to generate a smaller document which is easier to navigate.
Another application of |\includeonly| is to generate
documents consisting of selected parts of the complete document.

However, there are a few drawbacks of the plain |\include| mechanism:
\begin{itemize}
\item
The child files cannot be compiled on their own,
they can only be compiled via the main file.
A naive editing environment
(such as a text editor with an option
to have the current file processed by \LaTeX)
may require one to switch to the main file before compiling;
attempting to compile the child file produces errors.
\item
The main file must be modified (each time)
to adjust the |\includeonly| command
to the present needs. This easily leaves the main file in a messy state.
\item
The generated document will always carry the filename
of the main document. This is inconvenient if
several child files are to be compiled and
to be kept for distribution.
\end{itemize}

The present package provides a simple interface
to make child files individually compilable by \LaTeX{}.
Compiling a child file then has the same effect as compiling
the main file with an |\includeonly| command
to select the appropriate child.
Moreover the generated document will carry the name of the child
rather than the main file.
This resolves all three above issues.

This feature is meant to make the editing of books,
thesis documents and lecture notes somewhat more convenient.
However, the package can also be used efficiently for
composing a series of documents (such as exercise sheets)
which are typically distributed individually.
It then assists the author in generating the individual documents
(potentially in different versions)
as well as a document containing the collected series.
Another application is in developing style files
or other kinds of included material
where compilation of the style file could redirect
to a sample or test file.

%%%%%%%%%%%%%%%%%%%%%%%%%%%%%%%%%%%%%%%%%%%%%%%%%%%%%%%%%%%%%%%%%%%%%%%%%%%%%%%%
%%%%%%%%%%%%%%%%%%%%%%%%%%%%%%%%%%%%%%%%%%%%%%%%%%%%%%%%%%%%%%%%%%%%%%%%%%%%%%%%
\section{Usage}

First of all, the package \textsf{childdoc} is \emph{not} a standard
\LaTeXe{} |.sty| style file! Therefore it needs to be invoked in
a non-standard way.

%%%%%%%%%%%%%%%%%%%%%%%%%%%%%%%%%%%%%%%%%%%%%%%%%%%%%%%%%%%%%%%%%%%%%%%%%%%%%%%%
\subsection{Included Files}
\label{sec:include}

%%%%%%%%%%%%%%%%%%%%%%%%%%%%%%%%%%%%%%%%
\DescribeMacro{\childdocmain}
To use the package, add the commands
\begin{center}
\begin{tabular}{l}
|\input{childdoc.def}|\\
|\childdocmain{}|\\
\end{tabular}
\end{center}
at the very top of the main \LaTeX{} file,
in particular \emph{before} the |\documentclass| statement!
The argument of |\childdocmain| should be left empty
(but it must be present).

%%%%%%%%%%%%%%%%%%%%%%%%%%%%%%%%%%%%%%%%
\DescribeMacro{\childdocof}
Furthermore, add the commands
\begin{center}
\begin{tabular}{l}
|\input{childdoc.def}|\\
|\childdocof{|\textit{main}|}|\\
\end{tabular}
\end{center}
at the top of every child file \textit{child}
which is included by |\include{|\textit{child}|}|
from within the main file
(or at least for those files to be compiled individually).
The argument \textit{main} must be the filename of the main file.

There are a couple of
considerations in setting up the main and child documents:

%%%%%%%%%%%%%%%%%%%%%%%%%%%%%%%%%%%%%%%%
\paragraph{Restrictions.}

Please note the following restrictions:
\begin{itemize}
\item
|\childdocmain| must be called with one argument \textit{main}
to ensure compatibility with earlier version of the package.
It must either be empty (|\childdocmain{}|)
or precisely match the filename of the main file in which it is specified.
See \secref{sec:detection} for further information.
\item
The filename \textit{main} must be specified without the |.tex| extension.
\item
The filename \textit{main} is case sensitive
(even in case-insensitive file systems)
due to internal string comparison.
\item
The argument \textit{main} should be fully expanded, it cannot be a macro.
\item
Subdirectories and special characters should be avoided in filenames.
\item
The command |\childdocmain{|\textit{main}|}| must be followed by a whitespace.
It should not be followed immediately by another command
or by a comment mark `|%|'.
This is because the \TeX{} parser reads the token immediately following
the argument of |\childdocmain| and puts it
at the beginning of every child section;
however, a white\-space is ignored.
\end{itemize}

%%%%%%%%%%%%%%%%%%%%%%%%%%%%%%%%%%%%%%%%
\paragraph{Content of Main File.}

It is advisable to place all content in the child files included by |\include|.
Any output contained in the main file will appear in all child documents
unless suppressed manually;
it cannot be suppressed automatically by the |\includeonly| directive
and thus should normally be avoided.
A method to include some content in the main file
by means of conditional processing is described in \secref{sec:conditional}.

%%%%%%%%%%%%%%%%%%%%%%%%%%%%%%%%%%%%%%%%
\paragraph{Page Numbering.}

When only a part of the document is compiled,
the appropriate numbering of pages
(as well as other status parameters)
is determined from the |.aux| files.
The latter contain information from previous passes.
However this information needs to propagate through
all intermediate child documents.
Therefore the page numbering in child documents may well
be inconsistent until the complete document is compiled at least once.

A useful (if unconventional) way to always ensure a consistent
page numbering is to restart the numbering in each child document
and denote the pages by `\textit{child}|.|\textit{page}'
where \textit{child} represents the chapter/section number of the child file.
This can be achieved by the command
|\numberwithin{page}{|\textit{child}|}|
of the \textsf{amsmath} package
where \textit{child} can be |chapter| or |section|
depending on the chosen structuring.
Alternatively, one can modify the macro |\thepage| appropriately
and reset the counter |page| at the start of each child file.

%%%%%%%%%%%%%%%%%%%%%%%%%%%%%%%%%%%%%%%%%%%%%%%%%%%%%%%%%%%%%%%%%%%%%%%%%%%%%%%%
\subsection{Conditional Processing}
\label{sec:conditional}

The package provides a mechanism to compile different versions
of a document. To customise the versions further some conditional processing
can come in handy to distinguish which version is being compiled.
The package provides two macros to describe the compilation context:

%%%%%%%%%%%%%%%%%%%%%%%%%%%%%%%%%%%%%%%%
\DescribeMacro{\ifchilddoc}
The conditional |\ifchilddoc| distinguishes between the compilation of
child documents and the main document:
%
\begin{center}
|\ifchilddoc |\textit{child-code}| |[|\||else |\textit{main-code}]| \||fi|
\end{center}

%%%%%%%%%%%%%%%%%%%%%%%%%%%%%%%%%%%%%%%%
\DescribeMacro{\childdocname}
\DescribeMacro{\childdocjob}
The macro |\childdocname| contains the filename (without extension)
of the main or child file being processed.
Note that |\childdocjob| will always contain the name of the main file.

%%%%%%%%%%%%%%%%%%%%%%%%%%%%%%%%%%%%%%%%
\paragraph{Title Page.}

Conditional processing can be used to include a title or banner page
in the main document when proper precautions are taken.
Importantly, the code in the main file should ensure that the page counter
(as well as other status parameters which are stored in the |.aux| files)
takes the same value after the conditional processing.
Otherwise the page numbers may take divergent values
depending on which part is compiled.

For example, a title page could be declared by:
%
\begin{center}
\begin{tabular}{l}
|\ifchilddoc\||else|\\
|\addtocounter{page}{-1}|\\
\textit{code for title page}\\
|\newpage|\\
|\||fi|
\end{tabular}
\end{center}
%
A banner page for the child documents can be generated by:
%
\begin{center}
\begin{tabular}{l}
|\ifchilddoc|\\
|\addtocounter{page}{-1}|\\
\textit{code for banner page}\\
|\newpage|\\
|\||fi|
\end{tabular}
\end{center}
%
Here one could write a message such as:
\begin{center}
|This is the part \childdocname{} of \childdocjob{}.|
\end{center}

%%%%%%%%%%%%%%%%%%%%%%%%%%%%%%%%%%%%%%%%%%%%%%%%%%%%%%%%%%%%%%%%%%%%%%%%%%%%%%%%
\subsection{Flags}
\label{sec:flags}

The package makes it easy to generate different versions
of the main or child documents.
To this end compilation flags can be defined
and assigned different default values.
They will be particularly useful in conjunction
with the forwarding mechanism described in \secref{sec:forward}.

For example, it may be useful to have a flag |\version|
which can be set to |draft| or |final|.
The document source will contain some conditional code
depending on the value of |\version|.
Suppose further, the flag should default to |final| for the main file
and to |draft| for child files
which is a natural assignment for editing the document.
This is achieved by placing the following code
in the preamble of the main document
(below the |\childdocmain| directive):
%
\begin{center}
\begin{tabular}{l}
|\ifchilddoc|\\
|\providecommand{\version}{draft}|\\
|\||else|\\
|\providecommand{\version}{final}|\\
|\||fi|
\end{tabular}
\end{center}
%
The definition by |\providecommand| makes sure
that previous definitions are not overwritten.
Further statements |\providecommand{\version}{...}|
can thus be added before the above code to override it.

For the main file, one might add a line
(between |\childdocmain| and the above block)
%
\begin{center}
|%\ifchilddoc\||else\providecommand{\version}{draft}\||fi|
\end{center}
%
which can be uncommented to produce a draft version.
Likewise one can add a line to the very top of a child file
(above the |\childdocof{|\textit{main}|}| directive)
%
\begin{center}
|%\providecommand{\version}{final}|
\end{center}
%
which can be uncommented to produce the final version of this child document.

%%%%%%%%%%%%%%%%%%%%%%%%%%%%%%%%%%%%%%%%%%%%%%%%%%%%%%%%%%%%%%%%%%%%%%%%%%%%%%%%
\subsection{Forwarding}
\label{sec:forward}

Different versions of the main or child documents
using compilation flags as described in \secref{sec:flags}
can be (permanently) stored in different files
for convenient compilation, viewing and distribution.
To this end, the package defines a command
to pass on compilation to a different file:

%%%%%%%%%%%%%%%%%%%%%%%%%%%%%%%%%%%%%%%%
\DescribeMacro{\childdocforward}
The command |\childdocforward| redirects processing to
another source file:
%
\begin{center}
\begin{tabular}{l}
|\input{childdoc.def}|\\
|\childdocforward[|\textit{main}|]{|\textit{dest}|}|\\
\end{tabular}
\end{center}
%
The argument \textit{dest} is the destination file
(without extension).
It should be the main file or one of the child files.
Note that further \textsf{childdoc} directives
such as |\childdocof| and |\childdocforward|
in the indicated file will be processed in this form.
The optional argument \textit{main}
passes on directly to the main file \textit{main}
while pretending to compile the child \textit{dest}.
This form behaves as if \textit{dest}
issues |\childdocof{|\textit{main}|}| right away,
and no further \textsf{childdoc} directives will be processed.

%%%%%%%%%%%%%%%%%%%%%%%%%%%%%%%%%%%%%%%%
\DescribeMacro{\...prefix}
In the alternative form |\childdocforwardprefix|,
%
\begin{center}
\begin{tabular}{l}
|\input{childdoc.def}|\\
|\childdocforwardprefix[|\textit{main}|]{|\textit{prefix}|}{|\textit{dest}|}|
\end{tabular}
\end{center}
%
the destination file is determined by a pattern
depending on the current file:
To make this work, the current file must be called
`{\textit{prefix}\hspace{0.2em}\textit{suffix}}'
with \textit{prefix} matching precisely the argument.
Processing is then passed on to the file
`{\textit{dest}\hspace{0.2em}\textit{suffix}}'.
Surely, the same effect is achieved by
directly specifying the
argument `{\textit{dest}\hspace{0.2em}\textit{suffix}}'
in the first form.
However, that requires to set up a different file
for each child. With the alternative form of the command
all these files can have exactly the same content
which simplifies setting them up and maintaining them.

For example, the following file |draft.tex|
with a compilation flag |\version| as described in \secref{sec:flags}
compiles the main document as a draft:
%
\begin{center}
\begin{tabular}{l}
|\def\version{draft}|\\
|\input{childdoc.def}|\\
|\childdocforward{|\textit{main}|}|
\end{tabular}
\end{center}
%
Likewise, the following files |final|\textit{nn}|.tex|
compile the final version of the child document
|child|\textit{nn}|.tex|:
%
\begin{center}
\begin{tabular}{l}
|\def\version{final}|\\
|\input{childdoc.def}|\\
|\childdocforwardprefix{final}{child}|
\end{tabular}
\end{center}
%

Note that when several versions of a main file and/or of each child file
are to be generated, it may be convenient to set up a |Makefile| or
shell script to automatise the process.

%%%%%%%%%%%%%%%%%%%%%%%%%%%%%%%%%%%%%%%%%%%%%%%%%%%%%%%%%%%%%%%%%%%%%%%%%%%%%%%%
\subsection{Command Line Processing}
\label{sec:commandline}

The effect of redirection files can also be achieved by invoking
the \LaTeX{} compiler with a more elaborate command line.
Most conveniently this should be done as part
of a shell script or a |Makefile|.

When using \textsf{childdoc} in the main file, the following
command lines effectively perform a redirection
(note that depending on the shell being used,
backslashes may have to be doubled: `|\|' $\to$ `|\\|'):
%
\begin{center}
|... -jobname "|\textit{target}|" |\\|"|[\textit{flags}]%
|\input{childdoc.def}\childdocforward[|\textit{main}|]{|\textit{dest}|}"|
\end{center}
%
Here \textit{target} is the name of the output file,
\textit{main} is the name of the main file
and \textit{dest} is the name of the main or child file to be processed
(all filenames without extensions).
The optional argument \textit{main} can be omitted
if \textit{main} matches \textit{dest}.
Optionally, compilation \textit{flags} can be defined via |\def| commands.
This command line makes the \TeX{} engine believe
it is compiling the file \textit{target}
whose content is specified as the latter parameter.
The provided code then forwards the processing to
\textit{main} or \textit{dest} as described in \secref{sec:forward}.

%%%%%%%%%%%%%%%%%%%%%%%%%%%%%%%%%%%%%%%%%%%%%%%%%%%%%%%%%%%%%%%%%%%%%%%%%%%%%%%%
\subsection{Include by Input}
\label{sec:input}

Including child documents by |\include| has some restrictions by design.
Most notably, the content of a child document always occupies
its own set of pages; pages cannot be shared between child documents.
Usually, this behaviour makes perfect sense
because each child document contain an essential part of the document.
However, in some situations it may be desirable to compose
a document from a collection of parts
without having mandatory page breaks between then.
For this case, the package
provides a mechanism to include parts
by |\input| which can also be processed individually.
However, by construction this mechanism
requires manual handling of the content to be output.

%%%%%%%%%%%%%%%%%%%%%%%%%%%%%%%%%%%%%%%%
\DescribeMacro{\ifchilddocmanual}
The main file should be prepared as usual, see \secref{sec:include}.
However, the document body must make a distinction
between processing of an individual part and of the main document, e.g.:
%
\begin{center}
\begin{tabular}{l}
|\ifchilddocmanual|\\
|\input{\childdocname}|\\
|\||else|\\
\textit{document body with }|\input{|\textit{part}|}|\\
|\||fi|
\end{tabular}
\end{center}
%
The conditional |\ifchilddocmanual| is true whenever
a part to be included by |\input| is being compiled,
and the name of the part is stored in |\childdocname|.

%%%%%%%%%%%%%%%%%%%%%%%%%%%%%%%%%%%%%%%%
\DescribeMacro{\childdocby}
Each part to be included by |\input| should start with:
%
\begin{center}
\begin{tabular}{l}
|\input{childdoc.def}|\\
|\childdocby{|\textit{main}|}|\\
\end{tabular}
\end{center}
%
The directive |\childdocby| is similar to |\childdocof|
described in \secref{sec:include},
but the subsequent selection of content must be done manually.
To that end, both |\ifchilddoc| and |\ifchilddocmanual|
will be true upon processing of a part,
and the name of the part is stored in |\childdocname|.
Note that |\jobname| will be set to the filename of the current part
so that each part receives an individual |.aux| file
that does not interfere with the |.aux| file(s) of the main document.
This behaviour can be altered by the alternative form
|\childdocby[*]{|\textit{main}|}| (with a non-empty optional argument)
which uses the |.aux| file of the main document
by setting |\jobname| to \textit{main}.

%%%%%%%%%%%%%%%%%%%%%%%%%%%%%%%%%%%%%%%%%%%%%%%%%%%%%%%%%%%%%%%%%%%%%%%%%%%%%%%%
\subsection{Driver Development}
\label{sec:driver}

The \textsf{childdoc} mechanism can also be use for the development
of definition files such as \LaTeX{} styles or classes.
This case differs from the above setup with multiple parts
included by |\include| in that no |\includeonly| should be invoked.
This can be achieved by starting the include file
(before |\ProvidesPackage|) with:
%
\begin{center}
\begin{tabular}{l}
|\input{childdoc.def}|\\
|\childdocforward{|\textit{main}|}|\\
\end{tabular}
\end{center}
%
or alternatively with:
%
\begin{center}
\begin{tabular}{l}
|\input{childdoc.def}|\\
|\childdocby{|\textit{main}|}|\\
\end{tabular}
\end{center}
%
Both forms have slightly different effects as described above.
The main file is prepared as usual, see \secref{sec:include}.

%%%%%%%%%%%%%%%%%%%%%%%%%%%%%%%%%%%%%%%%%%%%%%%%%%%%%%%%%%%%%%%%%%%%%%%%%%%%%%%%
\subsection{Legacy Detection}
\label{sec:detection}

The directive |\childdocmain| in the main file can detect
whether the complete document or merely a child is to be compiled
even without using the directive |\childdocof|.
This method is deprecated because it is less robust
and there is no compelling reason to use it;
it is merely provided for backward compatibility
and it may be removed in future versions.

If the detection mechanism is to be used,
it is mandatory to correctly specify
the filename of the main file as the argument of |\childdocmain|:
%
\begin{center}
\begin{tabular}{l}
|\input{childdoc.def}|\\
|\childdocmain{|\textit{main}|}|\\
\end{tabular}
\end{center}
%
If |\jobname| does not match the argument \textit{main} of |\childdocmain|,
it is assumed that |\jobname| points to the child file to be compiled.
When using |\childdocmain| with the main file specified as argument,
it suffices to start a child file
with just |\input{|\textit{main}|}|
without loading of the package and using |\childdocof|.
If instead all processing is done
with the appropriate \textsf{childdoc} directives,
the argument of \textit{main} of |\childdocmain| can be empty.

An alternative version of the command line processing described
in \secref{sec:commandline} using the detection mechanism reads:
%
\begin{center}
|... -jobname "|\textit{target}|" "|[\textit{flags}]%
[|\def\jobname{|\textit{dest}|}|]|\input{|\textit{main}|}"|
\end{center}

%%%%%%%%%%%%%%%%%%%%%%%%%%%%%%%%%%%%%%%%%%%%%%%%%%%%%%%%%%%%%%%%%%%%%%%%%%%%%%%%
\subsection{Manual Code}
\label{sec:manual}

In case one cannot be certain whether the definitions file |childdoc.def|
is installed on the target \TeX{} distribution
and one prefers not to ship it,
it is conceivable to paste a few relevant commands into the sources.

To that end, drop all statements |\input{childdoc.def}|
and perform the replacements as outlined below.
Instead of |\childdocmain{|\textit{main}|}| add the following code
to the top of the main file:
%
\begin{center}
\begin{tabular}{l}
|\||ifdefined\childdocname\endinput\||fi\newif\ifchilddoc|\\
|\edef\childdocname{\scantokens\expandafter{\jobname\noexpand}}|\\
|\def\childdocmain{|\textit{main}|}\||ifx\childdocmain\childdocname\||else|\\
|\childdoctrue\includeonly{\childdocname}\let\jobname\childdocmain\||fi|\\
\end{tabular}
\end{center}
%
Instead of |\childdocof{|\textit{main}|}| just include the main file
at the top of each child file:
%
\begin{center}
|\input{|\textit{main}|}|
\end{center}
%
A simple redirection |\childdocforward{|\textit{dest}|}| is achieved by:
%
\begin{center}
|\def\jobname{|\textit{dest}|}\input{\jobname}|
\end{center}
%
The redirection with prefix
|\childdocforwardprefix[|\textit{prefix}|]{|\textit{dest}|}|
is accomplished by:
%
\begin{center}
\begin{tabular}{l}
|{\edef\jobname{\scantokens\expandafter{\jobname\noexpand}}|\\
|\def\redirectjob |\textit{prefix}|#1~~~{\gdef\jobname{|\textit{dest}|#1}}|\\
|\expandafter\redirectjob\jobname~~~}\input{\jobname}|
\end{tabular}
\end{center}

In an alternative approach,
child documents can be compiled by a specific command line
without additional code or specific definitions:
%
\begin{center}
|... -jobname "|\textit{target}|" "|[\textit{flags}]%
|\includeonly{|\textit{dest}|}\input{|\textit{main}|}"|
\end{center}
%

%%%%%%%%%%%%%%%%%%%%%%%%%%%%%%%%%%%%%%%%%%%%%%%%%%%%%%%%%%%%%%%%%%%%%%%%%%%%%%%%
%%%%%%%%%%%%%%%%%%%%%%%%%%%%%%%%%%%%%%%%%%%%%%%%%%%%%%%%%%%%%%%%%%%%%%%%%%%%%%%%
\section{Information}

%%%%%%%%%%%%%%%%%%%%%%%%%%%%%%%%%%%%%%%%%%%%%%%%%%%%%%%%%%%%%%%%%%%%%%%%%%%%%%%%
\subsection{Copyright}

Copyright \copyright{} 2017--2018 Niklas Beisert

This work may be distributed and/or modified under the
conditions of the \LaTeX{} Project Public License, either version 1.3
of this license or (at your option) any later version.
The latest version of this license is in
  \url{http://www.latex-project.org/lppl.txt}
and version 1.3 or later is part of all distributions of \LaTeX{}
version 2005/12/01 or later.

This work has the LPPL maintenance status `maintained'.

The Current Maintainer of this work is Niklas Beisert.

This work consists of the files |README.txt|, |childdoc.ins| and |childdoc.dtx|
as well as the derived files |childdoc.def|, |cdocsamp.tex|
with |cdocsch1.tex|, |cdocsch2.tex|, |cdocspt3.tex|, |cdocspt4.tex|,
|cdocsdrf.tex|, |cdocsfn1.tex|, |cdocsfn2.tex|
as well as |childdoc.pdf|.

%%%%%%%%%%%%%%%%%%%%%%%%%%%%%%%%%%%%%%%%%%%%%%%%%%%%%%%%%%%%%%%%%%%%%%%%%%%%%%%%
\subsection{Files and Installation}

The package consists of the files:
%
\begin{center}
\begin{tabular}{ll}
    |README.txt|   & readme file \\
    |childdoc.ins| & installation file \\
    |childdoc.dtx| & source file \\
    |childdoc.def| & definition file \\
    |cdocsamp.tex| & sample main file \\
    |cdocsch1.tex| & sample include file \\
    |cdocsch2.tex| & sample include file \\
    |cdocspt3.tex| & sample part file \\
    |cdocspt4.tex| & sample part file \\
    |cdocsdrf.tex| & sample redirection file \\
    |cdocsfn1.tex| & sample redirection file \\
    |cdocsfn2.tex| & sample redirection file \\
    |childdoc.pdf| & manual
\end{tabular}
\end{center}
%
The distribution consists of the files
|README.txt|, |childdoc.ins| and |childdoc.dtx|.
%
\begin{itemize}
\item
Run (pdf)\LaTeX{} on |childdoc.dtx|
to compile the manual |childdoc.pdf| (this file).
\item
Run \LaTeX{} on |childdoc.ins| to create the definitions file |childdoc.def|
and the sample |cdocsamp.tex| with include files
|cdocsch1.tex|, |cdocsch2.tex|, |cdocspt3.tex|, |cdocspt4.tex|,
|cdocsdrf.tex|, |cdocsfn1.tex|, |cdocsfn2.tex|.
Then copy the file |childdoc.def| to an appropriate directory of your \LaTeX{}
distribution, e.g.\ \textit{texmf-root}|/tex/latex/childdoc|.
\end{itemize}

%%%%%%%%%%%%%%%%%%%%%%%%%%%%%%%%%%%%%%%%%%%%%%%%%%%%%%%%%%%%%%%%%%%%%%%%%%%%%%%%
\subsection{Related CTAN Packages}

There are several other packages which offer a similar functionality:
%
\begin{itemize}
\item
The packages
\href{http://ctan.org/pkg/docmute}{\textsf{docmute}},
\href{http://ctan.org/pkg/includex}{\textsf{includex}} and
\href{http://ctan.org/pkg/standalone}{\textsf{standalone}}
provide commands to include only the document body of
a child file thus allowing both files to be compiled individually.
\item
The packages \href{http://ctan.org/pkg/subdocs}{\textsf{subdocs}}
and \href{http://ctan.org/pkg/subfiles}{\textsf{subfiles}}
provide structures in which the main and child documents can be
encapsulated and allowing them to be compiled individually.
The inclusion mechanism is different from the conventional |\include|.
\item
The package \href{http://ctan.org/pkg/combine}{\textsf{combine}}
is an elaborate solution to combine several documents into one.
\end{itemize}
%
See also the CTAN topic \href{http://ctan.org/topic/subdocs}{\textsf{subdocs}}
for further related packages.
The present package differs from the above solutions in that
a document structure constructed with the conventional |\include| mechanism
just needs two extra commands at the top of every file
such that all constituent files can be compiled individually.

%%%%%%%%%%%%%%%%%%%%%%%%%%%%%%%%%%%%%%%%%%%%%%%%%%%%%%%%%%%%%%%%%%%%%%%%%%%%%%%%
%\subsection{Feature Suggestions}
%
%The following is a list of features which may be useful for future
%versions of this package:
%%
%\begin{itemize}
%\item
%\ldots
%\end{itemize}

%%%%%%%%%%%%%%%%%%%%%%%%%%%%%%%%%%%%%%%%%%%%%%%%%%%%%%%%%%%%%%%%%%%%%%%%%%%%%%%%
\subsection{Revision History}

%%%%%%%%%%%%%%%%%%%%%%%%%%%%%%%%%%%%%%%%
\paragraph{v2.0:} 2018/12/30

\begin{itemize}
\item
immediate forward processing
\item
added |\childdocby| mechanism
\item
manual restructured
\end{itemize}

%%%%%%%%%%%%%%%%%%%%%%%%%%%%%%%%%%%%%%%%
\paragraph{v1.6:} 2018/01/17

\begin{itemize}
\item
application for development of include files
\item
corrections to manual
\end{itemize}

%%%%%%%%%%%%%%%%%%%%%%%%%%%%%%%%%%%%%%%%
\paragraph{v1.5:} 2017/05/21

\begin{itemize}
\item
more complete structuring introduced
\item
|\childdocof| introduced
\item
|\childdoc| renamed to |\childdocmain|
\item
|\childredirect| renamed to |\childdocforward| and |\childdocforwardprefix|
and functionality expanded
\end{itemize}

%%%%%%%%%%%%%%%%%%%%%%%%%%%%%%%%%%%%%%%%
\paragraph{v1.0:} 2017/04/27

\begin{itemize}
\item
manual and install package
\item
first version published on CTAN
\end{itemize}

%%%%%%%%%%%%%%%%%%%%%%%%%%%%%%%%%%%%%%%%
\paragraph{v0.6:} 2017/04/26

\begin{itemize}
\item
redirection mechanism added
\end{itemize}

%%%%%%%%%%%%%%%%%%%%%%%%%%%%%%%%%%%%%%%%
\paragraph{v0.5:} 2017/04/26

\begin{itemize}
\item
functionality in definition file
\end{itemize}


%%%%%%%%%%%%%%%%%%%%%%%%%%%%%%%%%%%%%%%%%%%%%%%%%%%%%%%%%%%%%%%%%%%%%%%%%%%%%%%%
%%%%%%%%%%%%%%%%%%%%%%%%%%%%%%%%%%%%%%%%%%%%%%%%%%%%%%%%%%%%%%%%%%%%%%%%%%%%%%%%
%%%%%%%%%%%%%%%%%%%%%%%%%%%%%%%%%%%%%%%%%%%%%%%%%%%%%%%%%%%%%%%%%%%%%%%%%%%%%%%%
\appendix

\settowidth\MacroIndent{\rmfamily\scriptsize 000\ }

 \DocInput{childdoc.dtx}

\end{document}
%</driver>
% \fi
%
% %%%%%%%%%%%%%%%%%%%%%%%%%%%%%%%%%%%%%%%%%%%%%%%%%%%%%%%%%%%%%%%%%%%%%%%%%%%%%%
% %%%%%%%%%%%%%%%%%%%%%%%%%%%%%%%%%%%%%%%%%%%%%%%%%%%%%%%%%%%%%%%%%%%%%%%%%%%%%%
% \section{Sample}
%\iffalse
%<*samplemain>
%\fi
%
% The following presents a sample document
% with two chapters, two parts, a title page,
% a compile flag as well as three forwarding files to set the flag.
% It consists of eight |.tex| files:
% \begin{center}
% \begin{tabular}{ll}
% |cdocsamp.tex|&main file\\
% |cdocsch1.tex|&include file for chapter 1\\
% |cdocsch2.tex|&include file for chapter 2\\
% |cdocspt3.tex|&include file for part 3\\
% |cdocspt4.tex|&include file for part 4\\
% |cdocsdrf.tex|&forwarding file for main file in draft mode\\
% |cdocsfi1.tex|&forwarding file for final version of chapter 1\\
% |cdocsfi2.tex|&forwarding file for final version of chapter 2\\
% \end{tabular}
% \end{center}
% Each of the eight files can be compiled directly by the \LaTeX{} compiler.
%
% %%%%%%%%%%%%%%%%%%%%%%%%%%%%%%%%%%%%%%
% \paragraph{Main File.}
%
% The main file is called |cdocsamp.tex|.
%
% Load the \textsf{childdoc} definitions and
% declare the filename for the main document:
%    \begin{macrocode}
\input{childdoc.def}
\childdocmain{}
%    \end{macrocode}

% Optional override for |\version| flag:
%    \begin{macrocode}
%%\ifchilddoc\else\providecommand{\version}{draft}\fi
%    \end{macrocode}

% Define the default values for the |\version| flag
% (|final| for the main file and |draft| for childs):
%    \begin{macrocode}
\ifchilddoc
\providecommand{\version}{draft}
\else
\providecommand{\version}{final}
\fi
%    \end{macrocode}

% Load the standard document class:
%    \begin{macrocode}
\documentclass[12pt]{article}
%    \end{macrocode}

% Start the document body:
%    \begin{macrocode}
\begin{document}
%    \end{macrocode}

% Declare a title page.
% Print title, part of document being processed and version flag:
%    \begin{macrocode}
\addtocounter{page}{-1}
\begin{center}
{\LARGE\bfseries{}childdoc example\par}
\vspace{1cm}
\ifchilddoc
\ifchilddocmanual part\else chapter\fi:
`\childdocname' of `\childdocjob'\par
\else
main document: `\childdocjob'\par
\fi
version: \version\par
\end{center}
\newpage
%    \end{macrocode}

% Manually include selected file,
% otherwise process as usual:
%    \begin{macrocode}
\ifchilddocmanual
\section*{part `\childdocname'}
\input{\childdocname}
\else
%    \end{macrocode}

% Include the two chapters:
%    \begin{macrocode}
\include{cdocsch1}
\include{cdocsch2}
%    \end{macrocode}

% Include the two parts unless only chapters should be displayed:
%    \begin{macrocode}
\ifchilddoc\else
\section{part three}
\input{cdocspt3}
\section{part four}
\input{cdocspt4}
\fi
%    \end{macrocode}

% Process as usual until here:
%    \begin{macrocode}
\fi
%    \end{macrocode}

% End of document body:
%    \begin{macrocode}
\end{document}
%    \end{macrocode}
%\iffalse
%</samplemain>
%\fi
%
% %%%%%%%%%%%%%%%%%%%%%%%%%%%%%%%%%%%%%%
% \paragraph{Chapter Include Files.}
%
% The include files are called |cdocsch1.tex| and |cdocsch2.tex|.
%
%\iffalse
%<*samplechap1|samplechap2>
%\fi

% Optional override for |\version| flag:
%    \begin{macrocode}
%%\providecommand{\version}{final}
%    \end{macrocode}

% Include the main document:
%    \begin{macrocode}
\input{childdoc.def}
\childdocof{cdocsamp}
%    \end{macrocode}

%\iffalse
%</samplechap1|samplechap2>
%\fi
%
%\iffalse
%<*samplechap1>
%\fi
% Some text for chapter 1:
%    \begin{macrocode}
\section{one}
some text in chapter one
%    \end{macrocode}

%\iffalse
%</samplechap1>
%\fi
% Some text for chapter 2:
%\iffalse
%<*samplechap2>
%\fi
%    \begin{macrocode}
\section{two}
more text in chapter two
%    \end{macrocode}

%\iffalse
%</samplechap2>
%\fi
%
% %%%%%%%%%%%%%%%%%%%%%%%%%%%%%%%%%%%%%%
% \paragraph{Part Include Files.}
%
% The include files are called |cdocspt3.tex| and |cdocspt4.tex|.
%
%\iffalse
%<*samplepart3|samplepart4>
%\fi

% Optional override for |\version| flag:
%    \begin{macrocode}
%%\providecommand{\version}{final}
%    \end{macrocode}

% Include the main document:
%    \begin{macrocode}
\input{childdoc.def}
\childdocby{cdocsamp}
%    \end{macrocode}

%\iffalse
%</samplepart3|samplepart4>
%\fi
%
%\iffalse
%<*samplepart3>
%\fi
% Some text for part 3:
%    \begin{macrocode}
some text in part three
%    \end{macrocode}

%\iffalse
%</samplepart3>
%\fi
% Some text for part 4:
%\iffalse
%<*samplepart4>
%\fi
%    \begin{macrocode}
more text in part four
%    \end{macrocode}

%\iffalse
%</samplepart4>
%\fi
%
% %%%%%%%%%%%%%%%%%%%%%%%%%%%%%%%%%%%%%%
% \paragraph{Forwarding for a Complete Draft.}
%
% The following forwarding file |cdocsdrf.tex|
% compiles the main document in draft mode:
%\iffalse
%<*sampledraft>
%\fi
%    \begin{macrocode}
\def\version{draft}
\input{childdoc.def}
\childdocforward{cdocsamp}
%    \end{macrocode}

%\iffalse
%</sampledraft>
%\fi
%
% %%%%%%%%%%%%%%%%%%%%%%%%%%%%%%%%%%%%%%
% \paragraph{Forwarding for Final Version of the Chapters.}
%
% The following forwarding files |cdocsfn1.tex| and |cdocsfn2.tex|
% (with identical content)
% compile the final versions of the child documents
% |cdocsch1.tex| and |cdocsch2.tex|, respectively:
%\iffalse
%<*samplefinal>
%\fi
%    \begin{macrocode}
\def\version{final}
\input{childdoc.def}
\childdocforwardprefix[cdocsamp]{cdocsfn}{cdocsch}
%    \end{macrocode}

%\iffalse
%</samplefinal>
%\fi
%
% %%%%%%%%%%%%%%%%%%%%%%%%%%%%%%%%%%%%%%
% \paragraph{Command Line Processing.}
%
% The following three command lines generate the output files
% |cdocscld|, |cdocscl1| and |cdocscl2|
% which should be identical to
% |cdocsdrf|, |cdocsch1| and |cdocsfn2|, respectively:
% \begin{center}
% \begin{tabular}{l}
% |latex -jobname cdocscld \|\\
% |  "\def\version{draft}\input{childdoc.def}\childdocforward{cdocsamp}"|\\
% |latex -jobname cdocscl1 \|\\
% |  "\input{childdoc.def}\childdocforward[cdocsamp]{cdocsch1}"|\\
% |latex -jobname cdocscl2 \|\\
% |  "\def\version{final}\input{childdoc.def}\childdocforward{cdocsch2}"|
% \end{tabular}
% \end{center}
% Note that the trailing backslash on each first line
% merely continues the input to the second line
% (for convenient cut ant paste).
% Furthermore, the command |latex| can be replaced by any
% of its alternative versions such as |pdflatex|.
%
% %%%%%%%%%%%%%%%%%%%%%%%%%%%%%%%%%%%%%%%%%%%%%%%%%%%%%%%%%%%%%%%%%%%%%%%%%%%%%%
% %%%%%%%%%%%%%%%%%%%%%%%%%%%%%%%%%%%%%%%%%%%%%%%%%%%%%%%%%%%%%%%%%%%%%%%%%%%%%%
% \section{Implementation}
%\iffalse
%<*package>
%\fi
%
% This section describes the definitions file |childdoc.def|.

% The definitions cannot be loaded using |\usepackage| or |\RequirePackage|
% which has a mechanism to prevent loading a style file more than once.
% When loading the definitions by means of |\input|
% multiple instances have to be prevented manually:
%\iffalse
%This code needs to be before the `\ProvidesFile' directive
%which is defined at the beginning of this file.
%Therefore it is also placed there and commented out here.
%</package>
%<*discard>
%\fi
%    \begin{macrocode}
\ifdefined\childdocmain\endinput\fi
%    \end{macrocode}
%\iffalse
%</discard>
%<*package>
%\fi
%
% \macro{\ifchilddoc}
% \macro{\ifchilddocmanual}
% The conditional |\ifchilddoc| tells whether a
% child (true) or main (false) document is being compiled.
% The conditional |\ifchilddocmanual| tells whether
% the |\includeonly| mechanism is used (false) or
% the selection of child files must be performed manually (true).
% The definitions initialise to false:
%    \begin{macrocode}
\newif\ifchilddoc
\newif\ifchilddocmanual
%    \end{macrocode}

% \macro{\childdocname}
% \macro{\childdocjob}
% The macro |\childdocname| stores the name of the main document
% to be compiled. The macro |\childdocjob| stores the name of
% the document on which the \LaTeX{} compiler was originally invoked.
% The content of |\jobname| cannot be compared
% to filenames specified in the source due to different catcodes.
% The following code rescans |\jobname|, stores the result
% in |\childdocname| and saves a copy in |\childdocjob|:
%    \begin{macrocode}
\edef\childdocname{\scantokens\expandafter{\jobname\noexpand}}
\let\childdocjob\childdocname
%    \end{macrocode}

% \macro{\childdocdisable}
% The macro |\childdocdisable| prevents the main file
% from being processed more than once.
% At this stage, the main document command |\childdocmain|
% is assumed to be called once again where it should do nothing.
% Any subsequent call to it should prevent
% a secondary processing of the main document
% It overwrites the forwarding commands
% |\childdocof| and |\childdocforward|
% with empty macros to prevent further inclusions of the main document:
%    \begin{macrocode}
\newcommand{\childdocdisable}
{
  \renewcommand{\childdocmain}[1]{\renewcommand{\childdocmain}[1]{\endinput}}
  \renewcommand{\childdocof}[1]{}
  \renewcommand{\childdocby}[2][]{}
  \renewcommand{\childdocforward}[2][]{}
  \renewcommand{\childdocdisable}{}
}
%    \end{macrocode}

% \macro{\childdocmain}
% The macro |\childdocmain| is to be called at the top of the main file
% with nothing or the main filename (without extension) as argument.
% First, it breaks loops.
% If the argument is not empty and does not match |\childdocname|
% (which is set by the first inclusion of |childdoc.def|),
% |\ifchilddoc| is set to true, |\includeonly| is applied to the child file
% and |\jobname| is set to the main file
% (for proper handling of |.aux| files):
%    \begin{macrocode}
\newcommand{\childdocmain}[1]
{
  \childdocdisable\childdocmain{}
  \if?#1?\else
    \begingroup
      \def\childdoctmp{#1}
      \ifx\childdoctmp\childdocname
        \def\childdoctmp{}
      \else
        \def\childdoctmp
        {
          \childdoctrue
          \includeonly{\childdocname}
          \def\childdocjob{#1}
          \def\jobname{#1}
        }
      \fi
      \expandafter
    \endgroup
    \childdoctmp
  \fi
}
%    \end{macrocode}

% \macro{\childdocof}
% The command |\childdocof| redirects
% compilation to the main file |#1|.
%    \begin{macrocode}
\newcommand{\childdocof}[1]
{
  \childdocdisable
  \childdoctrue
  \includeonly{\childdocname}
  \def\jobname{#1}
  \def\childdocjob{#1}
  \input{#1}
}
%    \end{macrocode}

% \macro{\childdocby}
% The command |\childdocby| ....
%    \begin{macrocode}
\newcommand{\childdocby}[2][]
{
  \childdocdisable
  \childdoctrue
  \childdocmanualtrue
  \if?#1?\else
    \def\jobname{#2}
  \fi
  \def\childdocjob{#2}
  \input{#2}
  \endinput
}
%    \end{macrocode}

% \macro{\childdocforward}
% The command |\childdocforward| redirects
% compilation to the main file or
% (if the optional argument is given) a child file.
% Parameters are set as if the main file
% or a child file starting with |\childdocof| was compiled.
% Then compilation is handed over to the main file:
%    \begin{macrocode}
\newcommand{\childdocforward}[2][]
{
  \begingroup
    \if?#1?
      \def\childdoctmp
      {
        \def\childdocname{#2}
        \def\childdocjob{#2}
        \def\jobname{#2}
        \input{#2}
        \endinput
      }
    \else
      \def\childdoctmp
      {
        \childdocdisable
        \def\childdocname{#2}
        \childdoctrue
        \includeonly{#2}
        \def\childdocjob{#1}
        \def\jobname{#1}
        \input{#1}
        \endinput
      }
    \fi
    \expandafter
  \endgroup
  \childdoctmp
}
%    \end{macrocode}

% \macro{\childdocforwardprefix}
% The command |\childdocforwardprefix| redirects
% compilation to the main or a child file by means of a pattern.
% The prefix |#1| in the current filename is replaced by |#2|
% and the suffix of the current filename is kept
% (it is assumed that the filename does not contain the substring `|~~~|'
% which is used as a delimiter).
% Compilation is handed over to the new file by |\childdocforward|:
%    \begin{macrocode}
\newcommand{\childdocforwardprefix}[3][]
{
  \begingroup
    \def\childdocextract #2##1~~~{\def\childdoctmp{\childdocforward[#1]{#3##1}}}
    \expandafter\childdocextract\childdocname~~~
    \expandafter
  \endgroup
  \childdoctmp
}
%    \end{macrocode}

% \macro{\childdoc}
% The deprecated macro |\childdoc| is a legacy version of |\childdocmain|:
%    \begin{macrocode}
\newcommand{\childdoc}{\childdocmain}
%    \end{macrocode}

% \macro{\childdocredirect}
% The deprecated macro |\childdocredirect| is a legacy version
% of |\childdocforward| and |\childdocforwardprefix|:
%    \begin{macrocode}
\newcommand{\childdocredirect}[2][]
{
  \begingroup
    \if?#1?
      \def\childdoctmp{\childdocforward{#2}}
    \else
      \def\childdoctmp{\childdocforwardprefix{#1}{#2}}
    \fi
    \expandafter
  \endgroup
  \childdoctmp
}
%    \end{macrocode}

%\iffalse
%</package>
%\fi
%
\endinput
|\\
|\childdocmain{|\textit{main}|}|\\
\end{tabular}
\end{center}
%
If |\jobname| does not match the argument \textit{main} of |\childdocmain|,
it is assumed that |\jobname| points to the child file to be compiled.
When using |\childdocmain| with the main file specified as argument,
it suffices to start a child file
with just |\input{|\textit{main}|}|
without loading of the package and using |\childdocof|.
If instead all processing is done
with the appropriate \textsf{childdoc} directives,
the argument of \textit{main} of |\childdocmain| can be empty.

An alternative version of the command line processing described
in \secref{sec:commandline} using the detection mechanism reads:
%
\begin{center}
|... -jobname "|\textit{target}|" "|[\textit{flags}]%
[|\def\jobname{|\textit{dest}|}|]|\input{|\textit{main}|}"|
\end{center}

%%%%%%%%%%%%%%%%%%%%%%%%%%%%%%%%%%%%%%%%%%%%%%%%%%%%%%%%%%%%%%%%%%%%%%%%%%%%%%%%
\subsection{Manual Code}
\label{sec:manual}

In case one cannot be certain whether the definitions file |childdoc.def|
is installed on the target \TeX{} distribution
and one prefers not to ship it,
it is conceivable to paste a few relevant commands into the sources.

To that end, drop all statements |% \iffalse
%
% childdoc.dtx Copyright (C) 2017-2018 Niklas Beisert
%
% This work may be distributed and/or modified under the
% conditions of the LaTeX Project Public License, either version 1.3
% of this license or (at your option) any later version.
% The latest version of this license is in
%   http://www.latex-project.org/lppl.txt
% and version 1.3 or later is part of all distributions of LaTeX
% version 2005/12/01 or later.
%
% This work has the LPPL maintenance status `maintained'.
%
% The Current Maintainer of this work is Niklas Beisert.
%
% This work consists of the files childdoc.dtx and childdoc.ins
% and the derived files childdoc.def and cdocsamp.tex with
% cdocsch1.tex, cdocsch2.tex, cdocsdrf.tex, cdocsfn1.tex, cdocsfn2.tex.
%
%<package>\ifdefined\childdocmain\endinput\fi
%<package>\ProvidesFile{childdoc.def}[2018/12/30 v2.0 child document driver]
%<samplemain>\ProvidesFile{cdocsamp.tex}[2018/12/30 v2.0 sample for childdoc]
%<*driver>
%\ProvidesFile{childdoc.drv}[2018/12/30 v2.0 childdoc reference manual file]
\PassOptionsToClass{10pt,a4paper}{article}
\documentclass{ltxdoc}

\usepackage[margin=35mm]{geometry}
\usepackage{hyperref}
\usepackage{hyperxmp}
\usepackage[usenames]{color}

\hypersetup{colorlinks=true}
\hypersetup{pdfstartview=FitH}
\hypersetup{pdfpagemode=UseNone}
\hypersetup{pdfsource={}}
\hypersetup{pdflang={en-UK}}
\hypersetup{pdfcopyright={Copyright 2017-2018 Niklas Beisert.
  This work may be distributed and/or modified under the
  conditions of the LaTeX Project Public License, either version 1.3
  of this license or (at your option) any later version.}}
\hypersetup{pdflicenseurl={http://www.latex-project.org/lppl.txt}}
\hypersetup{pdfcontactaddress={ETH Zurich, ITP, HIT K,
  Wolfgang-Pauli-Strasse 27}}
\hypersetup{pdfcontactpostcode={8093}}
\hypersetup{pdfcontactcity={Zurich}}
\hypersetup{pdfcontactcountry={Switzerland}}
\hypersetup{pdfcontactemail={nbeisert@itp.phys.ethz.ch}}
\hypersetup{pdfcontacturl={http://people.phys.ethz.ch/\xmptilde nbeisert/}}

\newcommand{\secref}[1]{\hyperref[#1]{section \ref*{#1}}}

\parskip1ex
\parindent0pt
\let\olditemize\itemize
\def\itemize{\olditemize\parskip0pt}

\begin{document}

\title{The \textsf{childdoc} Package}
\hypersetup{pdftitle={The childdoc Package}}
\author{Niklas Beisert\\[2ex]
  Institut f\"ur Theoretische Physik\\
  Eidgen\"ossische Technische Hochschule Z\"urich\\
  Wolfgang-Pauli-Strasse 27, 8093 Z\"urich, Switzerland\\[1ex]
  \href{mailto:nbeisert@itp.phys.ethz.ch}
  {\texttt{nbeisert@itp.phys.ethz.ch}}}
\hypersetup{pdfauthor={Niklas Beisert}}
\hypersetup{pdfsubject={Manual for the LaTeX2e Package childdoc}}
\date{30 December 2018, \textsf{v2.0}}
\maketitle

\begin{abstract}\noindent
\textsf{childdoc} is a \LaTeXe{} package
that enables the direct compilation
of document sections included by |\include|
to individual files.
\end{abstract}

\begingroup
\parskip0ex
\tableofcontents
\endgroup

%%%%%%%%%%%%%%%%%%%%%%%%%%%%%%%%%%%%%%%%%%%%%%%%%%%%%%%%%%%%%%%%%%%%%%%%%%%%%%%%
%%%%%%%%%%%%%%%%%%%%%%%%%%%%%%%%%%%%%%%%%%%%%%%%%%%%%%%%%%%%%%%%%%%%%%%%%%%%%%%%
\section{Introduction}

\LaTeX{} provides a mechanism to structure a large document (such as a book)
into a main file and several child files (containing the chapters)
using the |\include| command.
This mechanism is beneficial for documents
which span hundreds of pages in order to
make the source file(s) more manageable.
Moreover, compilation can be restricted to
selected child files by means of the |\includeonly| command.
The latter feature can be used to reduce the compilation time while editing
(this was significantly more useful in the earlier days of \LaTeX{})
or to generate a smaller document which is easier to navigate.
Another application of |\includeonly| is to generate
documents consisting of selected parts of the complete document.

However, there are a few drawbacks of the plain |\include| mechanism:
\begin{itemize}
\item
The child files cannot be compiled on their own,
they can only be compiled via the main file.
A naive editing environment
(such as a text editor with an option
to have the current file processed by \LaTeX)
may require one to switch to the main file before compiling;
attempting to compile the child file produces errors.
\item
The main file must be modified (each time)
to adjust the |\includeonly| command
to the present needs. This easily leaves the main file in a messy state.
\item
The generated document will always carry the filename
of the main document. This is inconvenient if
several child files are to be compiled and
to be kept for distribution.
\end{itemize}

The present package provides a simple interface
to make child files individually compilable by \LaTeX{}.
Compiling a child file then has the same effect as compiling
the main file with an |\includeonly| command
to select the appropriate child.
Moreover the generated document will carry the name of the child
rather than the main file.
This resolves all three above issues.

This feature is meant to make the editing of books,
thesis documents and lecture notes somewhat more convenient.
However, the package can also be used efficiently for
composing a series of documents (such as exercise sheets)
which are typically distributed individually.
It then assists the author in generating the individual documents
(potentially in different versions)
as well as a document containing the collected series.
Another application is in developing style files
or other kinds of included material
where compilation of the style file could redirect
to a sample or test file.

%%%%%%%%%%%%%%%%%%%%%%%%%%%%%%%%%%%%%%%%%%%%%%%%%%%%%%%%%%%%%%%%%%%%%%%%%%%%%%%%
%%%%%%%%%%%%%%%%%%%%%%%%%%%%%%%%%%%%%%%%%%%%%%%%%%%%%%%%%%%%%%%%%%%%%%%%%%%%%%%%
\section{Usage}

First of all, the package \textsf{childdoc} is \emph{not} a standard
\LaTeXe{} |.sty| style file! Therefore it needs to be invoked in
a non-standard way.

%%%%%%%%%%%%%%%%%%%%%%%%%%%%%%%%%%%%%%%%%%%%%%%%%%%%%%%%%%%%%%%%%%%%%%%%%%%%%%%%
\subsection{Included Files}
\label{sec:include}

%%%%%%%%%%%%%%%%%%%%%%%%%%%%%%%%%%%%%%%%
\DescribeMacro{\childdocmain}
To use the package, add the commands
\begin{center}
\begin{tabular}{l}
|\input{childdoc.def}|\\
|\childdocmain{}|\\
\end{tabular}
\end{center}
at the very top of the main \LaTeX{} file,
in particular \emph{before} the |\documentclass| statement!
The argument of |\childdocmain| should be left empty
(but it must be present).

%%%%%%%%%%%%%%%%%%%%%%%%%%%%%%%%%%%%%%%%
\DescribeMacro{\childdocof}
Furthermore, add the commands
\begin{center}
\begin{tabular}{l}
|\input{childdoc.def}|\\
|\childdocof{|\textit{main}|}|\\
\end{tabular}
\end{center}
at the top of every child file \textit{child}
which is included by |\include{|\textit{child}|}|
from within the main file
(or at least for those files to be compiled individually).
The argument \textit{main} must be the filename of the main file.

There are a couple of
considerations in setting up the main and child documents:

%%%%%%%%%%%%%%%%%%%%%%%%%%%%%%%%%%%%%%%%
\paragraph{Restrictions.}

Please note the following restrictions:
\begin{itemize}
\item
|\childdocmain| must be called with one argument \textit{main}
to ensure compatibility with earlier version of the package.
It must either be empty (|\childdocmain{}|)
or precisely match the filename of the main file in which it is specified.
See \secref{sec:detection} for further information.
\item
The filename \textit{main} must be specified without the |.tex| extension.
\item
The filename \textit{main} is case sensitive
(even in case-insensitive file systems)
due to internal string comparison.
\item
The argument \textit{main} should be fully expanded, it cannot be a macro.
\item
Subdirectories and special characters should be avoided in filenames.
\item
The command |\childdocmain{|\textit{main}|}| must be followed by a whitespace.
It should not be followed immediately by another command
or by a comment mark `|%|'.
This is because the \TeX{} parser reads the token immediately following
the argument of |\childdocmain| and puts it
at the beginning of every child section;
however, a white\-space is ignored.
\end{itemize}

%%%%%%%%%%%%%%%%%%%%%%%%%%%%%%%%%%%%%%%%
\paragraph{Content of Main File.}

It is advisable to place all content in the child files included by |\include|.
Any output contained in the main file will appear in all child documents
unless suppressed manually;
it cannot be suppressed automatically by the |\includeonly| directive
and thus should normally be avoided.
A method to include some content in the main file
by means of conditional processing is described in \secref{sec:conditional}.

%%%%%%%%%%%%%%%%%%%%%%%%%%%%%%%%%%%%%%%%
\paragraph{Page Numbering.}

When only a part of the document is compiled,
the appropriate numbering of pages
(as well as other status parameters)
is determined from the |.aux| files.
The latter contain information from previous passes.
However this information needs to propagate through
all intermediate child documents.
Therefore the page numbering in child documents may well
be inconsistent until the complete document is compiled at least once.

A useful (if unconventional) way to always ensure a consistent
page numbering is to restart the numbering in each child document
and denote the pages by `\textit{child}|.|\textit{page}'
where \textit{child} represents the chapter/section number of the child file.
This can be achieved by the command
|\numberwithin{page}{|\textit{child}|}|
of the \textsf{amsmath} package
where \textit{child} can be |chapter| or |section|
depending on the chosen structuring.
Alternatively, one can modify the macro |\thepage| appropriately
and reset the counter |page| at the start of each child file.

%%%%%%%%%%%%%%%%%%%%%%%%%%%%%%%%%%%%%%%%%%%%%%%%%%%%%%%%%%%%%%%%%%%%%%%%%%%%%%%%
\subsection{Conditional Processing}
\label{sec:conditional}

The package provides a mechanism to compile different versions
of a document. To customise the versions further some conditional processing
can come in handy to distinguish which version is being compiled.
The package provides two macros to describe the compilation context:

%%%%%%%%%%%%%%%%%%%%%%%%%%%%%%%%%%%%%%%%
\DescribeMacro{\ifchilddoc}
The conditional |\ifchilddoc| distinguishes between the compilation of
child documents and the main document:
%
\begin{center}
|\ifchilddoc |\textit{child-code}| |[|\||else |\textit{main-code}]| \||fi|
\end{center}

%%%%%%%%%%%%%%%%%%%%%%%%%%%%%%%%%%%%%%%%
\DescribeMacro{\childdocname}
\DescribeMacro{\childdocjob}
The macro |\childdocname| contains the filename (without extension)
of the main or child file being processed.
Note that |\childdocjob| will always contain the name of the main file.

%%%%%%%%%%%%%%%%%%%%%%%%%%%%%%%%%%%%%%%%
\paragraph{Title Page.}

Conditional processing can be used to include a title or banner page
in the main document when proper precautions are taken.
Importantly, the code in the main file should ensure that the page counter
(as well as other status parameters which are stored in the |.aux| files)
takes the same value after the conditional processing.
Otherwise the page numbers may take divergent values
depending on which part is compiled.

For example, a title page could be declared by:
%
\begin{center}
\begin{tabular}{l}
|\ifchilddoc\||else|\\
|\addtocounter{page}{-1}|\\
\textit{code for title page}\\
|\newpage|\\
|\||fi|
\end{tabular}
\end{center}
%
A banner page for the child documents can be generated by:
%
\begin{center}
\begin{tabular}{l}
|\ifchilddoc|\\
|\addtocounter{page}{-1}|\\
\textit{code for banner page}\\
|\newpage|\\
|\||fi|
\end{tabular}
\end{center}
%
Here one could write a message such as:
\begin{center}
|This is the part \childdocname{} of \childdocjob{}.|
\end{center}

%%%%%%%%%%%%%%%%%%%%%%%%%%%%%%%%%%%%%%%%%%%%%%%%%%%%%%%%%%%%%%%%%%%%%%%%%%%%%%%%
\subsection{Flags}
\label{sec:flags}

The package makes it easy to generate different versions
of the main or child documents.
To this end compilation flags can be defined
and assigned different default values.
They will be particularly useful in conjunction
with the forwarding mechanism described in \secref{sec:forward}.

For example, it may be useful to have a flag |\version|
which can be set to |draft| or |final|.
The document source will contain some conditional code
depending on the value of |\version|.
Suppose further, the flag should default to |final| for the main file
and to |draft| for child files
which is a natural assignment for editing the document.
This is achieved by placing the following code
in the preamble of the main document
(below the |\childdocmain| directive):
%
\begin{center}
\begin{tabular}{l}
|\ifchilddoc|\\
|\providecommand{\version}{draft}|\\
|\||else|\\
|\providecommand{\version}{final}|\\
|\||fi|
\end{tabular}
\end{center}
%
The definition by |\providecommand| makes sure
that previous definitions are not overwritten.
Further statements |\providecommand{\version}{...}|
can thus be added before the above code to override it.

For the main file, one might add a line
(between |\childdocmain| and the above block)
%
\begin{center}
|%\ifchilddoc\||else\providecommand{\version}{draft}\||fi|
\end{center}
%
which can be uncommented to produce a draft version.
Likewise one can add a line to the very top of a child file
(above the |\childdocof{|\textit{main}|}| directive)
%
\begin{center}
|%\providecommand{\version}{final}|
\end{center}
%
which can be uncommented to produce the final version of this child document.

%%%%%%%%%%%%%%%%%%%%%%%%%%%%%%%%%%%%%%%%%%%%%%%%%%%%%%%%%%%%%%%%%%%%%%%%%%%%%%%%
\subsection{Forwarding}
\label{sec:forward}

Different versions of the main or child documents
using compilation flags as described in \secref{sec:flags}
can be (permanently) stored in different files
for convenient compilation, viewing and distribution.
To this end, the package defines a command
to pass on compilation to a different file:

%%%%%%%%%%%%%%%%%%%%%%%%%%%%%%%%%%%%%%%%
\DescribeMacro{\childdocforward}
The command |\childdocforward| redirects processing to
another source file:
%
\begin{center}
\begin{tabular}{l}
|\input{childdoc.def}|\\
|\childdocforward[|\textit{main}|]{|\textit{dest}|}|\\
\end{tabular}
\end{center}
%
The argument \textit{dest} is the destination file
(without extension).
It should be the main file or one of the child files.
Note that further \textsf{childdoc} directives
such as |\childdocof| and |\childdocforward|
in the indicated file will be processed in this form.
The optional argument \textit{main}
passes on directly to the main file \textit{main}
while pretending to compile the child \textit{dest}.
This form behaves as if \textit{dest}
issues |\childdocof{|\textit{main}|}| right away,
and no further \textsf{childdoc} directives will be processed.

%%%%%%%%%%%%%%%%%%%%%%%%%%%%%%%%%%%%%%%%
\DescribeMacro{\...prefix}
In the alternative form |\childdocforwardprefix|,
%
\begin{center}
\begin{tabular}{l}
|\input{childdoc.def}|\\
|\childdocforwardprefix[|\textit{main}|]{|\textit{prefix}|}{|\textit{dest}|}|
\end{tabular}
\end{center}
%
the destination file is determined by a pattern
depending on the current file:
To make this work, the current file must be called
`{\textit{prefix}\hspace{0.2em}\textit{suffix}}'
with \textit{prefix} matching precisely the argument.
Processing is then passed on to the file
`{\textit{dest}\hspace{0.2em}\textit{suffix}}'.
Surely, the same effect is achieved by
directly specifying the
argument `{\textit{dest}\hspace{0.2em}\textit{suffix}}'
in the first form.
However, that requires to set up a different file
for each child. With the alternative form of the command
all these files can have exactly the same content
which simplifies setting them up and maintaining them.

For example, the following file |draft.tex|
with a compilation flag |\version| as described in \secref{sec:flags}
compiles the main document as a draft:
%
\begin{center}
\begin{tabular}{l}
|\def\version{draft}|\\
|\input{childdoc.def}|\\
|\childdocforward{|\textit{main}|}|
\end{tabular}
\end{center}
%
Likewise, the following files |final|\textit{nn}|.tex|
compile the final version of the child document
|child|\textit{nn}|.tex|:
%
\begin{center}
\begin{tabular}{l}
|\def\version{final}|\\
|\input{childdoc.def}|\\
|\childdocforwardprefix{final}{child}|
\end{tabular}
\end{center}
%

Note that when several versions of a main file and/or of each child file
are to be generated, it may be convenient to set up a |Makefile| or
shell script to automatise the process.

%%%%%%%%%%%%%%%%%%%%%%%%%%%%%%%%%%%%%%%%%%%%%%%%%%%%%%%%%%%%%%%%%%%%%%%%%%%%%%%%
\subsection{Command Line Processing}
\label{sec:commandline}

The effect of redirection files can also be achieved by invoking
the \LaTeX{} compiler with a more elaborate command line.
Most conveniently this should be done as part
of a shell script or a |Makefile|.

When using \textsf{childdoc} in the main file, the following
command lines effectively perform a redirection
(note that depending on the shell being used,
backslashes may have to be doubled: `|\|' $\to$ `|\\|'):
%
\begin{center}
|... -jobname "|\textit{target}|" |\\|"|[\textit{flags}]%
|\input{childdoc.def}\childdocforward[|\textit{main}|]{|\textit{dest}|}"|
\end{center}
%
Here \textit{target} is the name of the output file,
\textit{main} is the name of the main file
and \textit{dest} is the name of the main or child file to be processed
(all filenames without extensions).
The optional argument \textit{main} can be omitted
if \textit{main} matches \textit{dest}.
Optionally, compilation \textit{flags} can be defined via |\def| commands.
This command line makes the \TeX{} engine believe
it is compiling the file \textit{target}
whose content is specified as the latter parameter.
The provided code then forwards the processing to
\textit{main} or \textit{dest} as described in \secref{sec:forward}.

%%%%%%%%%%%%%%%%%%%%%%%%%%%%%%%%%%%%%%%%%%%%%%%%%%%%%%%%%%%%%%%%%%%%%%%%%%%%%%%%
\subsection{Include by Input}
\label{sec:input}

Including child documents by |\include| has some restrictions by design.
Most notably, the content of a child document always occupies
its own set of pages; pages cannot be shared between child documents.
Usually, this behaviour makes perfect sense
because each child document contain an essential part of the document.
However, in some situations it may be desirable to compose
a document from a collection of parts
without having mandatory page breaks between then.
For this case, the package
provides a mechanism to include parts
by |\input| which can also be processed individually.
However, by construction this mechanism
requires manual handling of the content to be output.

%%%%%%%%%%%%%%%%%%%%%%%%%%%%%%%%%%%%%%%%
\DescribeMacro{\ifchilddocmanual}
The main file should be prepared as usual, see \secref{sec:include}.
However, the document body must make a distinction
between processing of an individual part and of the main document, e.g.:
%
\begin{center}
\begin{tabular}{l}
|\ifchilddocmanual|\\
|\input{\childdocname}|\\
|\||else|\\
\textit{document body with }|\input{|\textit{part}|}|\\
|\||fi|
\end{tabular}
\end{center}
%
The conditional |\ifchilddocmanual| is true whenever
a part to be included by |\input| is being compiled,
and the name of the part is stored in |\childdocname|.

%%%%%%%%%%%%%%%%%%%%%%%%%%%%%%%%%%%%%%%%
\DescribeMacro{\childdocby}
Each part to be included by |\input| should start with:
%
\begin{center}
\begin{tabular}{l}
|\input{childdoc.def}|\\
|\childdocby{|\textit{main}|}|\\
\end{tabular}
\end{center}
%
The directive |\childdocby| is similar to |\childdocof|
described in \secref{sec:include},
but the subsequent selection of content must be done manually.
To that end, both |\ifchilddoc| and |\ifchilddocmanual|
will be true upon processing of a part,
and the name of the part is stored in |\childdocname|.
Note that |\jobname| will be set to the filename of the current part
so that each part receives an individual |.aux| file
that does not interfere with the |.aux| file(s) of the main document.
This behaviour can be altered by the alternative form
|\childdocby[*]{|\textit{main}|}| (with a non-empty optional argument)
which uses the |.aux| file of the main document
by setting |\jobname| to \textit{main}.

%%%%%%%%%%%%%%%%%%%%%%%%%%%%%%%%%%%%%%%%%%%%%%%%%%%%%%%%%%%%%%%%%%%%%%%%%%%%%%%%
\subsection{Driver Development}
\label{sec:driver}

The \textsf{childdoc} mechanism can also be use for the development
of definition files such as \LaTeX{} styles or classes.
This case differs from the above setup with multiple parts
included by |\include| in that no |\includeonly| should be invoked.
This can be achieved by starting the include file
(before |\ProvidesPackage|) with:
%
\begin{center}
\begin{tabular}{l}
|\input{childdoc.def}|\\
|\childdocforward{|\textit{main}|}|\\
\end{tabular}
\end{center}
%
or alternatively with:
%
\begin{center}
\begin{tabular}{l}
|\input{childdoc.def}|\\
|\childdocby{|\textit{main}|}|\\
\end{tabular}
\end{center}
%
Both forms have slightly different effects as described above.
The main file is prepared as usual, see \secref{sec:include}.

%%%%%%%%%%%%%%%%%%%%%%%%%%%%%%%%%%%%%%%%%%%%%%%%%%%%%%%%%%%%%%%%%%%%%%%%%%%%%%%%
\subsection{Legacy Detection}
\label{sec:detection}

The directive |\childdocmain| in the main file can detect
whether the complete document or merely a child is to be compiled
even without using the directive |\childdocof|.
This method is deprecated because it is less robust
and there is no compelling reason to use it;
it is merely provided for backward compatibility
and it may be removed in future versions.

If the detection mechanism is to be used,
it is mandatory to correctly specify
the filename of the main file as the argument of |\childdocmain|:
%
\begin{center}
\begin{tabular}{l}
|\input{childdoc.def}|\\
|\childdocmain{|\textit{main}|}|\\
\end{tabular}
\end{center}
%
If |\jobname| does not match the argument \textit{main} of |\childdocmain|,
it is assumed that |\jobname| points to the child file to be compiled.
When using |\childdocmain| with the main file specified as argument,
it suffices to start a child file
with just |\input{|\textit{main}|}|
without loading of the package and using |\childdocof|.
If instead all processing is done
with the appropriate \textsf{childdoc} directives,
the argument of \textit{main} of |\childdocmain| can be empty.

An alternative version of the command line processing described
in \secref{sec:commandline} using the detection mechanism reads:
%
\begin{center}
|... -jobname "|\textit{target}|" "|[\textit{flags}]%
[|\def\jobname{|\textit{dest}|}|]|\input{|\textit{main}|}"|
\end{center}

%%%%%%%%%%%%%%%%%%%%%%%%%%%%%%%%%%%%%%%%%%%%%%%%%%%%%%%%%%%%%%%%%%%%%%%%%%%%%%%%
\subsection{Manual Code}
\label{sec:manual}

In case one cannot be certain whether the definitions file |childdoc.def|
is installed on the target \TeX{} distribution
and one prefers not to ship it,
it is conceivable to paste a few relevant commands into the sources.

To that end, drop all statements |\input{childdoc.def}|
and perform the replacements as outlined below.
Instead of |\childdocmain{|\textit{main}|}| add the following code
to the top of the main file:
%
\begin{center}
\begin{tabular}{l}
|\||ifdefined\childdocname\endinput\||fi\newif\ifchilddoc|\\
|\edef\childdocname{\scantokens\expandafter{\jobname\noexpand}}|\\
|\def\childdocmain{|\textit{main}|}\||ifx\childdocmain\childdocname\||else|\\
|\childdoctrue\includeonly{\childdocname}\let\jobname\childdocmain\||fi|\\
\end{tabular}
\end{center}
%
Instead of |\childdocof{|\textit{main}|}| just include the main file
at the top of each child file:
%
\begin{center}
|\input{|\textit{main}|}|
\end{center}
%
A simple redirection |\childdocforward{|\textit{dest}|}| is achieved by:
%
\begin{center}
|\def\jobname{|\textit{dest}|}\input{\jobname}|
\end{center}
%
The redirection with prefix
|\childdocforwardprefix[|\textit{prefix}|]{|\textit{dest}|}|
is accomplished by:
%
\begin{center}
\begin{tabular}{l}
|{\edef\jobname{\scantokens\expandafter{\jobname\noexpand}}|\\
|\def\redirectjob |\textit{prefix}|#1~~~{\gdef\jobname{|\textit{dest}|#1}}|\\
|\expandafter\redirectjob\jobname~~~}\input{\jobname}|
\end{tabular}
\end{center}

In an alternative approach,
child documents can be compiled by a specific command line
without additional code or specific definitions:
%
\begin{center}
|... -jobname "|\textit{target}|" "|[\textit{flags}]%
|\includeonly{|\textit{dest}|}\input{|\textit{main}|}"|
\end{center}
%

%%%%%%%%%%%%%%%%%%%%%%%%%%%%%%%%%%%%%%%%%%%%%%%%%%%%%%%%%%%%%%%%%%%%%%%%%%%%%%%%
%%%%%%%%%%%%%%%%%%%%%%%%%%%%%%%%%%%%%%%%%%%%%%%%%%%%%%%%%%%%%%%%%%%%%%%%%%%%%%%%
\section{Information}

%%%%%%%%%%%%%%%%%%%%%%%%%%%%%%%%%%%%%%%%%%%%%%%%%%%%%%%%%%%%%%%%%%%%%%%%%%%%%%%%
\subsection{Copyright}

Copyright \copyright{} 2017--2018 Niklas Beisert

This work may be distributed and/or modified under the
conditions of the \LaTeX{} Project Public License, either version 1.3
of this license or (at your option) any later version.
The latest version of this license is in
  \url{http://www.latex-project.org/lppl.txt}
and version 1.3 or later is part of all distributions of \LaTeX{}
version 2005/12/01 or later.

This work has the LPPL maintenance status `maintained'.

The Current Maintainer of this work is Niklas Beisert.

This work consists of the files |README.txt|, |childdoc.ins| and |childdoc.dtx|
as well as the derived files |childdoc.def|, |cdocsamp.tex|
with |cdocsch1.tex|, |cdocsch2.tex|, |cdocspt3.tex|, |cdocspt4.tex|,
|cdocsdrf.tex|, |cdocsfn1.tex|, |cdocsfn2.tex|
as well as |childdoc.pdf|.

%%%%%%%%%%%%%%%%%%%%%%%%%%%%%%%%%%%%%%%%%%%%%%%%%%%%%%%%%%%%%%%%%%%%%%%%%%%%%%%%
\subsection{Files and Installation}

The package consists of the files:
%
\begin{center}
\begin{tabular}{ll}
    |README.txt|   & readme file \\
    |childdoc.ins| & installation file \\
    |childdoc.dtx| & source file \\
    |childdoc.def| & definition file \\
    |cdocsamp.tex| & sample main file \\
    |cdocsch1.tex| & sample include file \\
    |cdocsch2.tex| & sample include file \\
    |cdocspt3.tex| & sample part file \\
    |cdocspt4.tex| & sample part file \\
    |cdocsdrf.tex| & sample redirection file \\
    |cdocsfn1.tex| & sample redirection file \\
    |cdocsfn2.tex| & sample redirection file \\
    |childdoc.pdf| & manual
\end{tabular}
\end{center}
%
The distribution consists of the files
|README.txt|, |childdoc.ins| and |childdoc.dtx|.
%
\begin{itemize}
\item
Run (pdf)\LaTeX{} on |childdoc.dtx|
to compile the manual |childdoc.pdf| (this file).
\item
Run \LaTeX{} on |childdoc.ins| to create the definitions file |childdoc.def|
and the sample |cdocsamp.tex| with include files
|cdocsch1.tex|, |cdocsch2.tex|, |cdocspt3.tex|, |cdocspt4.tex|,
|cdocsdrf.tex|, |cdocsfn1.tex|, |cdocsfn2.tex|.
Then copy the file |childdoc.def| to an appropriate directory of your \LaTeX{}
distribution, e.g.\ \textit{texmf-root}|/tex/latex/childdoc|.
\end{itemize}

%%%%%%%%%%%%%%%%%%%%%%%%%%%%%%%%%%%%%%%%%%%%%%%%%%%%%%%%%%%%%%%%%%%%%%%%%%%%%%%%
\subsection{Related CTAN Packages}

There are several other packages which offer a similar functionality:
%
\begin{itemize}
\item
The packages
\href{http://ctan.org/pkg/docmute}{\textsf{docmute}},
\href{http://ctan.org/pkg/includex}{\textsf{includex}} and
\href{http://ctan.org/pkg/standalone}{\textsf{standalone}}
provide commands to include only the document body of
a child file thus allowing both files to be compiled individually.
\item
The packages \href{http://ctan.org/pkg/subdocs}{\textsf{subdocs}}
and \href{http://ctan.org/pkg/subfiles}{\textsf{subfiles}}
provide structures in which the main and child documents can be
encapsulated and allowing them to be compiled individually.
The inclusion mechanism is different from the conventional |\include|.
\item
The package \href{http://ctan.org/pkg/combine}{\textsf{combine}}
is an elaborate solution to combine several documents into one.
\end{itemize}
%
See also the CTAN topic \href{http://ctan.org/topic/subdocs}{\textsf{subdocs}}
for further related packages.
The present package differs from the above solutions in that
a document structure constructed with the conventional |\include| mechanism
just needs two extra commands at the top of every file
such that all constituent files can be compiled individually.

%%%%%%%%%%%%%%%%%%%%%%%%%%%%%%%%%%%%%%%%%%%%%%%%%%%%%%%%%%%%%%%%%%%%%%%%%%%%%%%%
%\subsection{Feature Suggestions}
%
%The following is a list of features which may be useful for future
%versions of this package:
%%
%\begin{itemize}
%\item
%\ldots
%\end{itemize}

%%%%%%%%%%%%%%%%%%%%%%%%%%%%%%%%%%%%%%%%%%%%%%%%%%%%%%%%%%%%%%%%%%%%%%%%%%%%%%%%
\subsection{Revision History}

%%%%%%%%%%%%%%%%%%%%%%%%%%%%%%%%%%%%%%%%
\paragraph{v2.0:} 2018/12/30

\begin{itemize}
\item
immediate forward processing
\item
added |\childdocby| mechanism
\item
manual restructured
\end{itemize}

%%%%%%%%%%%%%%%%%%%%%%%%%%%%%%%%%%%%%%%%
\paragraph{v1.6:} 2018/01/17

\begin{itemize}
\item
application for development of include files
\item
corrections to manual
\end{itemize}

%%%%%%%%%%%%%%%%%%%%%%%%%%%%%%%%%%%%%%%%
\paragraph{v1.5:} 2017/05/21

\begin{itemize}
\item
more complete structuring introduced
\item
|\childdocof| introduced
\item
|\childdoc| renamed to |\childdocmain|
\item
|\childredirect| renamed to |\childdocforward| and |\childdocforwardprefix|
and functionality expanded
\end{itemize}

%%%%%%%%%%%%%%%%%%%%%%%%%%%%%%%%%%%%%%%%
\paragraph{v1.0:} 2017/04/27

\begin{itemize}
\item
manual and install package
\item
first version published on CTAN
\end{itemize}

%%%%%%%%%%%%%%%%%%%%%%%%%%%%%%%%%%%%%%%%
\paragraph{v0.6:} 2017/04/26

\begin{itemize}
\item
redirection mechanism added
\end{itemize}

%%%%%%%%%%%%%%%%%%%%%%%%%%%%%%%%%%%%%%%%
\paragraph{v0.5:} 2017/04/26

\begin{itemize}
\item
functionality in definition file
\end{itemize}


%%%%%%%%%%%%%%%%%%%%%%%%%%%%%%%%%%%%%%%%%%%%%%%%%%%%%%%%%%%%%%%%%%%%%%%%%%%%%%%%
%%%%%%%%%%%%%%%%%%%%%%%%%%%%%%%%%%%%%%%%%%%%%%%%%%%%%%%%%%%%%%%%%%%%%%%%%%%%%%%%
%%%%%%%%%%%%%%%%%%%%%%%%%%%%%%%%%%%%%%%%%%%%%%%%%%%%%%%%%%%%%%%%%%%%%%%%%%%%%%%%
\appendix

\settowidth\MacroIndent{\rmfamily\scriptsize 000\ }

 \DocInput{childdoc.dtx}

\end{document}
%</driver>
% \fi
%
% %%%%%%%%%%%%%%%%%%%%%%%%%%%%%%%%%%%%%%%%%%%%%%%%%%%%%%%%%%%%%%%%%%%%%%%%%%%%%%
% %%%%%%%%%%%%%%%%%%%%%%%%%%%%%%%%%%%%%%%%%%%%%%%%%%%%%%%%%%%%%%%%%%%%%%%%%%%%%%
% \section{Sample}
%\iffalse
%<*samplemain>
%\fi
%
% The following presents a sample document
% with two chapters, two parts, a title page,
% a compile flag as well as three forwarding files to set the flag.
% It consists of eight |.tex| files:
% \begin{center}
% \begin{tabular}{ll}
% |cdocsamp.tex|&main file\\
% |cdocsch1.tex|&include file for chapter 1\\
% |cdocsch2.tex|&include file for chapter 2\\
% |cdocspt3.tex|&include file for part 3\\
% |cdocspt4.tex|&include file for part 4\\
% |cdocsdrf.tex|&forwarding file for main file in draft mode\\
% |cdocsfi1.tex|&forwarding file for final version of chapter 1\\
% |cdocsfi2.tex|&forwarding file for final version of chapter 2\\
% \end{tabular}
% \end{center}
% Each of the eight files can be compiled directly by the \LaTeX{} compiler.
%
% %%%%%%%%%%%%%%%%%%%%%%%%%%%%%%%%%%%%%%
% \paragraph{Main File.}
%
% The main file is called |cdocsamp.tex|.
%
% Load the \textsf{childdoc} definitions and
% declare the filename for the main document:
%    \begin{macrocode}
\input{childdoc.def}
\childdocmain{}
%    \end{macrocode}

% Optional override for |\version| flag:
%    \begin{macrocode}
%%\ifchilddoc\else\providecommand{\version}{draft}\fi
%    \end{macrocode}

% Define the default values for the |\version| flag
% (|final| for the main file and |draft| for childs):
%    \begin{macrocode}
\ifchilddoc
\providecommand{\version}{draft}
\else
\providecommand{\version}{final}
\fi
%    \end{macrocode}

% Load the standard document class:
%    \begin{macrocode}
\documentclass[12pt]{article}
%    \end{macrocode}

% Start the document body:
%    \begin{macrocode}
\begin{document}
%    \end{macrocode}

% Declare a title page.
% Print title, part of document being processed and version flag:
%    \begin{macrocode}
\addtocounter{page}{-1}
\begin{center}
{\LARGE\bfseries{}childdoc example\par}
\vspace{1cm}
\ifchilddoc
\ifchilddocmanual part\else chapter\fi:
`\childdocname' of `\childdocjob'\par
\else
main document: `\childdocjob'\par
\fi
version: \version\par
\end{center}
\newpage
%    \end{macrocode}

% Manually include selected file,
% otherwise process as usual:
%    \begin{macrocode}
\ifchilddocmanual
\section*{part `\childdocname'}
\input{\childdocname}
\else
%    \end{macrocode}

% Include the two chapters:
%    \begin{macrocode}
\include{cdocsch1}
\include{cdocsch2}
%    \end{macrocode}

% Include the two parts unless only chapters should be displayed:
%    \begin{macrocode}
\ifchilddoc\else
\section{part three}
\input{cdocspt3}
\section{part four}
\input{cdocspt4}
\fi
%    \end{macrocode}

% Process as usual until here:
%    \begin{macrocode}
\fi
%    \end{macrocode}

% End of document body:
%    \begin{macrocode}
\end{document}
%    \end{macrocode}
%\iffalse
%</samplemain>
%\fi
%
% %%%%%%%%%%%%%%%%%%%%%%%%%%%%%%%%%%%%%%
% \paragraph{Chapter Include Files.}
%
% The include files are called |cdocsch1.tex| and |cdocsch2.tex|.
%
%\iffalse
%<*samplechap1|samplechap2>
%\fi

% Optional override for |\version| flag:
%    \begin{macrocode}
%%\providecommand{\version}{final}
%    \end{macrocode}

% Include the main document:
%    \begin{macrocode}
\input{childdoc.def}
\childdocof{cdocsamp}
%    \end{macrocode}

%\iffalse
%</samplechap1|samplechap2>
%\fi
%
%\iffalse
%<*samplechap1>
%\fi
% Some text for chapter 1:
%    \begin{macrocode}
\section{one}
some text in chapter one
%    \end{macrocode}

%\iffalse
%</samplechap1>
%\fi
% Some text for chapter 2:
%\iffalse
%<*samplechap2>
%\fi
%    \begin{macrocode}
\section{two}
more text in chapter two
%    \end{macrocode}

%\iffalse
%</samplechap2>
%\fi
%
% %%%%%%%%%%%%%%%%%%%%%%%%%%%%%%%%%%%%%%
% \paragraph{Part Include Files.}
%
% The include files are called |cdocspt3.tex| and |cdocspt4.tex|.
%
%\iffalse
%<*samplepart3|samplepart4>
%\fi

% Optional override for |\version| flag:
%    \begin{macrocode}
%%\providecommand{\version}{final}
%    \end{macrocode}

% Include the main document:
%    \begin{macrocode}
\input{childdoc.def}
\childdocby{cdocsamp}
%    \end{macrocode}

%\iffalse
%</samplepart3|samplepart4>
%\fi
%
%\iffalse
%<*samplepart3>
%\fi
% Some text for part 3:
%    \begin{macrocode}
some text in part three
%    \end{macrocode}

%\iffalse
%</samplepart3>
%\fi
% Some text for part 4:
%\iffalse
%<*samplepart4>
%\fi
%    \begin{macrocode}
more text in part four
%    \end{macrocode}

%\iffalse
%</samplepart4>
%\fi
%
% %%%%%%%%%%%%%%%%%%%%%%%%%%%%%%%%%%%%%%
% \paragraph{Forwarding for a Complete Draft.}
%
% The following forwarding file |cdocsdrf.tex|
% compiles the main document in draft mode:
%\iffalse
%<*sampledraft>
%\fi
%    \begin{macrocode}
\def\version{draft}
\input{childdoc.def}
\childdocforward{cdocsamp}
%    \end{macrocode}

%\iffalse
%</sampledraft>
%\fi
%
% %%%%%%%%%%%%%%%%%%%%%%%%%%%%%%%%%%%%%%
% \paragraph{Forwarding for Final Version of the Chapters.}
%
% The following forwarding files |cdocsfn1.tex| and |cdocsfn2.tex|
% (with identical content)
% compile the final versions of the child documents
% |cdocsch1.tex| and |cdocsch2.tex|, respectively:
%\iffalse
%<*samplefinal>
%\fi
%    \begin{macrocode}
\def\version{final}
\input{childdoc.def}
\childdocforwardprefix[cdocsamp]{cdocsfn}{cdocsch}
%    \end{macrocode}

%\iffalse
%</samplefinal>
%\fi
%
% %%%%%%%%%%%%%%%%%%%%%%%%%%%%%%%%%%%%%%
% \paragraph{Command Line Processing.}
%
% The following three command lines generate the output files
% |cdocscld|, |cdocscl1| and |cdocscl2|
% which should be identical to
% |cdocsdrf|, |cdocsch1| and |cdocsfn2|, respectively:
% \begin{center}
% \begin{tabular}{l}
% |latex -jobname cdocscld \|\\
% |  "\def\version{draft}\input{childdoc.def}\childdocforward{cdocsamp}"|\\
% |latex -jobname cdocscl1 \|\\
% |  "\input{childdoc.def}\childdocforward[cdocsamp]{cdocsch1}"|\\
% |latex -jobname cdocscl2 \|\\
% |  "\def\version{final}\input{childdoc.def}\childdocforward{cdocsch2}"|
% \end{tabular}
% \end{center}
% Note that the trailing backslash on each first line
% merely continues the input to the second line
% (for convenient cut ant paste).
% Furthermore, the command |latex| can be replaced by any
% of its alternative versions such as |pdflatex|.
%
% %%%%%%%%%%%%%%%%%%%%%%%%%%%%%%%%%%%%%%%%%%%%%%%%%%%%%%%%%%%%%%%%%%%%%%%%%%%%%%
% %%%%%%%%%%%%%%%%%%%%%%%%%%%%%%%%%%%%%%%%%%%%%%%%%%%%%%%%%%%%%%%%%%%%%%%%%%%%%%
% \section{Implementation}
%\iffalse
%<*package>
%\fi
%
% This section describes the definitions file |childdoc.def|.

% The definitions cannot be loaded using |\usepackage| or |\RequirePackage|
% which has a mechanism to prevent loading a style file more than once.
% When loading the definitions by means of |\input|
% multiple instances have to be prevented manually:
%\iffalse
%This code needs to be before the `\ProvidesFile' directive
%which is defined at the beginning of this file.
%Therefore it is also placed there and commented out here.
%</package>
%<*discard>
%\fi
%    \begin{macrocode}
\ifdefined\childdocmain\endinput\fi
%    \end{macrocode}
%\iffalse
%</discard>
%<*package>
%\fi
%
% \macro{\ifchilddoc}
% \macro{\ifchilddocmanual}
% The conditional |\ifchilddoc| tells whether a
% child (true) or main (false) document is being compiled.
% The conditional |\ifchilddocmanual| tells whether
% the |\includeonly| mechanism is used (false) or
% the selection of child files must be performed manually (true).
% The definitions initialise to false:
%    \begin{macrocode}
\newif\ifchilddoc
\newif\ifchilddocmanual
%    \end{macrocode}

% \macro{\childdocname}
% \macro{\childdocjob}
% The macro |\childdocname| stores the name of the main document
% to be compiled. The macro |\childdocjob| stores the name of
% the document on which the \LaTeX{} compiler was originally invoked.
% The content of |\jobname| cannot be compared
% to filenames specified in the source due to different catcodes.
% The following code rescans |\jobname|, stores the result
% in |\childdocname| and saves a copy in |\childdocjob|:
%    \begin{macrocode}
\edef\childdocname{\scantokens\expandafter{\jobname\noexpand}}
\let\childdocjob\childdocname
%    \end{macrocode}

% \macro{\childdocdisable}
% The macro |\childdocdisable| prevents the main file
% from being processed more than once.
% At this stage, the main document command |\childdocmain|
% is assumed to be called once again where it should do nothing.
% Any subsequent call to it should prevent
% a secondary processing of the main document
% It overwrites the forwarding commands
% |\childdocof| and |\childdocforward|
% with empty macros to prevent further inclusions of the main document:
%    \begin{macrocode}
\newcommand{\childdocdisable}
{
  \renewcommand{\childdocmain}[1]{\renewcommand{\childdocmain}[1]{\endinput}}
  \renewcommand{\childdocof}[1]{}
  \renewcommand{\childdocby}[2][]{}
  \renewcommand{\childdocforward}[2][]{}
  \renewcommand{\childdocdisable}{}
}
%    \end{macrocode}

% \macro{\childdocmain}
% The macro |\childdocmain| is to be called at the top of the main file
% with nothing or the main filename (without extension) as argument.
% First, it breaks loops.
% If the argument is not empty and does not match |\childdocname|
% (which is set by the first inclusion of |childdoc.def|),
% |\ifchilddoc| is set to true, |\includeonly| is applied to the child file
% and |\jobname| is set to the main file
% (for proper handling of |.aux| files):
%    \begin{macrocode}
\newcommand{\childdocmain}[1]
{
  \childdocdisable\childdocmain{}
  \if?#1?\else
    \begingroup
      \def\childdoctmp{#1}
      \ifx\childdoctmp\childdocname
        \def\childdoctmp{}
      \else
        \def\childdoctmp
        {
          \childdoctrue
          \includeonly{\childdocname}
          \def\childdocjob{#1}
          \def\jobname{#1}
        }
      \fi
      \expandafter
    \endgroup
    \childdoctmp
  \fi
}
%    \end{macrocode}

% \macro{\childdocof}
% The command |\childdocof| redirects
% compilation to the main file |#1|.
%    \begin{macrocode}
\newcommand{\childdocof}[1]
{
  \childdocdisable
  \childdoctrue
  \includeonly{\childdocname}
  \def\jobname{#1}
  \def\childdocjob{#1}
  \input{#1}
}
%    \end{macrocode}

% \macro{\childdocby}
% The command |\childdocby| ....
%    \begin{macrocode}
\newcommand{\childdocby}[2][]
{
  \childdocdisable
  \childdoctrue
  \childdocmanualtrue
  \if?#1?\else
    \def\jobname{#2}
  \fi
  \def\childdocjob{#2}
  \input{#2}
  \endinput
}
%    \end{macrocode}

% \macro{\childdocforward}
% The command |\childdocforward| redirects
% compilation to the main file or
% (if the optional argument is given) a child file.
% Parameters are set as if the main file
% or a child file starting with |\childdocof| was compiled.
% Then compilation is handed over to the main file:
%    \begin{macrocode}
\newcommand{\childdocforward}[2][]
{
  \begingroup
    \if?#1?
      \def\childdoctmp
      {
        \def\childdocname{#2}
        \def\childdocjob{#2}
        \def\jobname{#2}
        \input{#2}
        \endinput
      }
    \else
      \def\childdoctmp
      {
        \childdocdisable
        \def\childdocname{#2}
        \childdoctrue
        \includeonly{#2}
        \def\childdocjob{#1}
        \def\jobname{#1}
        \input{#1}
        \endinput
      }
    \fi
    \expandafter
  \endgroup
  \childdoctmp
}
%    \end{macrocode}

% \macro{\childdocforwardprefix}
% The command |\childdocforwardprefix| redirects
% compilation to the main or a child file by means of a pattern.
% The prefix |#1| in the current filename is replaced by |#2|
% and the suffix of the current filename is kept
% (it is assumed that the filename does not contain the substring `|~~~|'
% which is used as a delimiter).
% Compilation is handed over to the new file by |\childdocforward|:
%    \begin{macrocode}
\newcommand{\childdocforwardprefix}[3][]
{
  \begingroup
    \def\childdocextract #2##1~~~{\def\childdoctmp{\childdocforward[#1]{#3##1}}}
    \expandafter\childdocextract\childdocname~~~
    \expandafter
  \endgroup
  \childdoctmp
}
%    \end{macrocode}

% \macro{\childdoc}
% The deprecated macro |\childdoc| is a legacy version of |\childdocmain|:
%    \begin{macrocode}
\newcommand{\childdoc}{\childdocmain}
%    \end{macrocode}

% \macro{\childdocredirect}
% The deprecated macro |\childdocredirect| is a legacy version
% of |\childdocforward| and |\childdocforwardprefix|:
%    \begin{macrocode}
\newcommand{\childdocredirect}[2][]
{
  \begingroup
    \if?#1?
      \def\childdoctmp{\childdocforward{#2}}
    \else
      \def\childdoctmp{\childdocforwardprefix{#1}{#2}}
    \fi
    \expandafter
  \endgroup
  \childdoctmp
}
%    \end{macrocode}

%\iffalse
%</package>
%\fi
%
\endinput
|
and perform the replacements as outlined below.
Instead of |\childdocmain{|\textit{main}|}| add the following code
to the top of the main file:
%
\begin{center}
\begin{tabular}{l}
|\||ifdefined\childdocname\endinput\||fi\newif\ifchilddoc|\\
|\edef\childdocname{\scantokens\expandafter{\jobname\noexpand}}|\\
|\def\childdocmain{|\textit{main}|}\||ifx\childdocmain\childdocname\||else|\\
|\childdoctrue\includeonly{\childdocname}\let\jobname\childdocmain\||fi|\\
\end{tabular}
\end{center}
%
Instead of |\childdocof{|\textit{main}|}| just include the main file
at the top of each child file:
%
\begin{center}
|\input{|\textit{main}|}|
\end{center}
%
A simple redirection |\childdocforward{|\textit{dest}|}| is achieved by:
%
\begin{center}
|\def\jobname{|\textit{dest}|}\input{\jobname}|
\end{center}
%
The redirection with prefix
|\childdocforwardprefix[|\textit{prefix}|]{|\textit{dest}|}|
is accomplished by:
%
\begin{center}
\begin{tabular}{l}
|{\edef\jobname{\scantokens\expandafter{\jobname\noexpand}}|\\
|\def\redirectjob |\textit{prefix}|#1~~~{\gdef\jobname{|\textit{dest}|#1}}|\\
|\expandafter\redirectjob\jobname~~~}\input{\jobname}|
\end{tabular}
\end{center}

In an alternative approach,
child documents can be compiled by a specific command line
without additional code or specific definitions:
%
\begin{center}
|... -jobname "|\textit{target}|" "|[\textit{flags}]%
|\includeonly{|\textit{dest}|}\input{|\textit{main}|}"|
\end{center}
%

%%%%%%%%%%%%%%%%%%%%%%%%%%%%%%%%%%%%%%%%%%%%%%%%%%%%%%%%%%%%%%%%%%%%%%%%%%%%%%%%
%%%%%%%%%%%%%%%%%%%%%%%%%%%%%%%%%%%%%%%%%%%%%%%%%%%%%%%%%%%%%%%%%%%%%%%%%%%%%%%%
\section{Information}

%%%%%%%%%%%%%%%%%%%%%%%%%%%%%%%%%%%%%%%%%%%%%%%%%%%%%%%%%%%%%%%%%%%%%%%%%%%%%%%%
\subsection{Copyright}

Copyright \copyright{} 2017--2018 Niklas Beisert

This work may be distributed and/or modified under the
conditions of the \LaTeX{} Project Public License, either version 1.3
of this license or (at your option) any later version.
The latest version of this license is in
  \url{http://www.latex-project.org/lppl.txt}
and version 1.3 or later is part of all distributions of \LaTeX{}
version 2005/12/01 or later.

This work has the LPPL maintenance status `maintained'.

The Current Maintainer of this work is Niklas Beisert.

This work consists of the files |README.txt|, |childdoc.ins| and |childdoc.dtx|
as well as the derived files |childdoc.def|, |cdocsamp.tex|
with |cdocsch1.tex|, |cdocsch2.tex|, |cdocspt3.tex|, |cdocspt4.tex|,
|cdocsdrf.tex|, |cdocsfn1.tex|, |cdocsfn2.tex|
as well as |childdoc.pdf|.

%%%%%%%%%%%%%%%%%%%%%%%%%%%%%%%%%%%%%%%%%%%%%%%%%%%%%%%%%%%%%%%%%%%%%%%%%%%%%%%%
\subsection{Files and Installation}

The package consists of the files:
%
\begin{center}
\begin{tabular}{ll}
    |README.txt|   & readme file \\
    |childdoc.ins| & installation file \\
    |childdoc.dtx| & source file \\
    |childdoc.def| & definition file \\
    |cdocsamp.tex| & sample main file \\
    |cdocsch1.tex| & sample include file \\
    |cdocsch2.tex| & sample include file \\
    |cdocspt3.tex| & sample part file \\
    |cdocspt4.tex| & sample part file \\
    |cdocsdrf.tex| & sample redirection file \\
    |cdocsfn1.tex| & sample redirection file \\
    |cdocsfn2.tex| & sample redirection file \\
    |childdoc.pdf| & manual
\end{tabular}
\end{center}
%
The distribution consists of the files
|README.txt|, |childdoc.ins| and |childdoc.dtx|.
%
\begin{itemize}
\item
Run (pdf)\LaTeX{} on |childdoc.dtx|
to compile the manual |childdoc.pdf| (this file).
\item
Run \LaTeX{} on |childdoc.ins| to create the definitions file |childdoc.def|
and the sample |cdocsamp.tex| with include files
|cdocsch1.tex|, |cdocsch2.tex|, |cdocspt3.tex|, |cdocspt4.tex|,
|cdocsdrf.tex|, |cdocsfn1.tex|, |cdocsfn2.tex|.
Then copy the file |childdoc.def| to an appropriate directory of your \LaTeX{}
distribution, e.g.\ \textit{texmf-root}|/tex/latex/childdoc|.
\end{itemize}

%%%%%%%%%%%%%%%%%%%%%%%%%%%%%%%%%%%%%%%%%%%%%%%%%%%%%%%%%%%%%%%%%%%%%%%%%%%%%%%%
\subsection{Related CTAN Packages}

There are several other packages which offer a similar functionality:
%
\begin{itemize}
\item
The packages
\href{http://ctan.org/pkg/docmute}{\textsf{docmute}},
\href{http://ctan.org/pkg/includex}{\textsf{includex}} and
\href{http://ctan.org/pkg/standalone}{\textsf{standalone}}
provide commands to include only the document body of
a child file thus allowing both files to be compiled individually.
\item
The packages \href{http://ctan.org/pkg/subdocs}{\textsf{subdocs}}
and \href{http://ctan.org/pkg/subfiles}{\textsf{subfiles}}
provide structures in which the main and child documents can be
encapsulated and allowing them to be compiled individually.
The inclusion mechanism is different from the conventional |\include|.
\item
The package \href{http://ctan.org/pkg/combine}{\textsf{combine}}
is an elaborate solution to combine several documents into one.
\end{itemize}
%
See also the CTAN topic \href{http://ctan.org/topic/subdocs}{\textsf{subdocs}}
for further related packages.
The present package differs from the above solutions in that
a document structure constructed with the conventional |\include| mechanism
just needs two extra commands at the top of every file
such that all constituent files can be compiled individually.

%%%%%%%%%%%%%%%%%%%%%%%%%%%%%%%%%%%%%%%%%%%%%%%%%%%%%%%%%%%%%%%%%%%%%%%%%%%%%%%%
%\subsection{Feature Suggestions}
%
%The following is a list of features which may be useful for future
%versions of this package:
%%
%\begin{itemize}
%\item
%\ldots
%\end{itemize}

%%%%%%%%%%%%%%%%%%%%%%%%%%%%%%%%%%%%%%%%%%%%%%%%%%%%%%%%%%%%%%%%%%%%%%%%%%%%%%%%
\subsection{Revision History}

%%%%%%%%%%%%%%%%%%%%%%%%%%%%%%%%%%%%%%%%
\paragraph{v2.0:} 2018/12/30

\begin{itemize}
\item
immediate forward processing
\item
added |\childdocby| mechanism
\item
manual restructured
\end{itemize}

%%%%%%%%%%%%%%%%%%%%%%%%%%%%%%%%%%%%%%%%
\paragraph{v1.6:} 2018/01/17

\begin{itemize}
\item
application for development of include files
\item
corrections to manual
\end{itemize}

%%%%%%%%%%%%%%%%%%%%%%%%%%%%%%%%%%%%%%%%
\paragraph{v1.5:} 2017/05/21

\begin{itemize}
\item
more complete structuring introduced
\item
|\childdocof| introduced
\item
|\childdoc| renamed to |\childdocmain|
\item
|\childredirect| renamed to |\childdocforward| and |\childdocforwardprefix|
and functionality expanded
\end{itemize}

%%%%%%%%%%%%%%%%%%%%%%%%%%%%%%%%%%%%%%%%
\paragraph{v1.0:} 2017/04/27

\begin{itemize}
\item
manual and install package
\item
first version published on CTAN
\end{itemize}

%%%%%%%%%%%%%%%%%%%%%%%%%%%%%%%%%%%%%%%%
\paragraph{v0.6:} 2017/04/26

\begin{itemize}
\item
redirection mechanism added
\end{itemize}

%%%%%%%%%%%%%%%%%%%%%%%%%%%%%%%%%%%%%%%%
\paragraph{v0.5:} 2017/04/26

\begin{itemize}
\item
functionality in definition file
\end{itemize}


%%%%%%%%%%%%%%%%%%%%%%%%%%%%%%%%%%%%%%%%%%%%%%%%%%%%%%%%%%%%%%%%%%%%%%%%%%%%%%%%
%%%%%%%%%%%%%%%%%%%%%%%%%%%%%%%%%%%%%%%%%%%%%%%%%%%%%%%%%%%%%%%%%%%%%%%%%%%%%%%%
%%%%%%%%%%%%%%%%%%%%%%%%%%%%%%%%%%%%%%%%%%%%%%%%%%%%%%%%%%%%%%%%%%%%%%%%%%%%%%%%
\appendix

\settowidth\MacroIndent{\rmfamily\scriptsize 000\ }

 \DocInput{childdoc.dtx}

\end{document}
%</driver>
% \fi
%
% %%%%%%%%%%%%%%%%%%%%%%%%%%%%%%%%%%%%%%%%%%%%%%%%%%%%%%%%%%%%%%%%%%%%%%%%%%%%%%
% %%%%%%%%%%%%%%%%%%%%%%%%%%%%%%%%%%%%%%%%%%%%%%%%%%%%%%%%%%%%%%%%%%%%%%%%%%%%%%
% \section{Sample}
%\iffalse
%<*samplemain>
%\fi
%
% The following presents a sample document
% with two chapters, two parts, a title page,
% a compile flag as well as three forwarding files to set the flag.
% It consists of eight |.tex| files:
% \begin{center}
% \begin{tabular}{ll}
% |cdocsamp.tex|&main file\\
% |cdocsch1.tex|&include file for chapter 1\\
% |cdocsch2.tex|&include file for chapter 2\\
% |cdocspt3.tex|&include file for part 3\\
% |cdocspt4.tex|&include file for part 4\\
% |cdocsdrf.tex|&forwarding file for main file in draft mode\\
% |cdocsfi1.tex|&forwarding file for final version of chapter 1\\
% |cdocsfi2.tex|&forwarding file for final version of chapter 2\\
% \end{tabular}
% \end{center}
% Each of the eight files can be compiled directly by the \LaTeX{} compiler.
%
% %%%%%%%%%%%%%%%%%%%%%%%%%%%%%%%%%%%%%%
% \paragraph{Main File.}
%
% The main file is called |cdocsamp.tex|.
%
% Load the \textsf{childdoc} definitions and
% declare the filename for the main document:
%    \begin{macrocode}
% \iffalse
%
% childdoc.dtx Copyright (C) 2017-2018 Niklas Beisert
%
% This work may be distributed and/or modified under the
% conditions of the LaTeX Project Public License, either version 1.3
% of this license or (at your option) any later version.
% The latest version of this license is in
%   http://www.latex-project.org/lppl.txt
% and version 1.3 or later is part of all distributions of LaTeX
% version 2005/12/01 or later.
%
% This work has the LPPL maintenance status `maintained'.
%
% The Current Maintainer of this work is Niklas Beisert.
%
% This work consists of the files childdoc.dtx and childdoc.ins
% and the derived files childdoc.def and cdocsamp.tex with
% cdocsch1.tex, cdocsch2.tex, cdocsdrf.tex, cdocsfn1.tex, cdocsfn2.tex.
%
%<package>\ifdefined\childdocmain\endinput\fi
%<package>\ProvidesFile{childdoc.def}[2018/12/30 v2.0 child document driver]
%<samplemain>\ProvidesFile{cdocsamp.tex}[2018/12/30 v2.0 sample for childdoc]
%<*driver>
%\ProvidesFile{childdoc.drv}[2018/12/30 v2.0 childdoc reference manual file]
\PassOptionsToClass{10pt,a4paper}{article}
\documentclass{ltxdoc}

\usepackage[margin=35mm]{geometry}
\usepackage{hyperref}
\usepackage{hyperxmp}
\usepackage[usenames]{color}

\hypersetup{colorlinks=true}
\hypersetup{pdfstartview=FitH}
\hypersetup{pdfpagemode=UseNone}
\hypersetup{pdfsource={}}
\hypersetup{pdflang={en-UK}}
\hypersetup{pdfcopyright={Copyright 2017-2018 Niklas Beisert.
  This work may be distributed and/or modified under the
  conditions of the LaTeX Project Public License, either version 1.3
  of this license or (at your option) any later version.}}
\hypersetup{pdflicenseurl={http://www.latex-project.org/lppl.txt}}
\hypersetup{pdfcontactaddress={ETH Zurich, ITP, HIT K,
  Wolfgang-Pauli-Strasse 27}}
\hypersetup{pdfcontactpostcode={8093}}
\hypersetup{pdfcontactcity={Zurich}}
\hypersetup{pdfcontactcountry={Switzerland}}
\hypersetup{pdfcontactemail={nbeisert@itp.phys.ethz.ch}}
\hypersetup{pdfcontacturl={http://people.phys.ethz.ch/\xmptilde nbeisert/}}

\newcommand{\secref}[1]{\hyperref[#1]{section \ref*{#1}}}

\parskip1ex
\parindent0pt
\let\olditemize\itemize
\def\itemize{\olditemize\parskip0pt}

\begin{document}

\title{The \textsf{childdoc} Package}
\hypersetup{pdftitle={The childdoc Package}}
\author{Niklas Beisert\\[2ex]
  Institut f\"ur Theoretische Physik\\
  Eidgen\"ossische Technische Hochschule Z\"urich\\
  Wolfgang-Pauli-Strasse 27, 8093 Z\"urich, Switzerland\\[1ex]
  \href{mailto:nbeisert@itp.phys.ethz.ch}
  {\texttt{nbeisert@itp.phys.ethz.ch}}}
\hypersetup{pdfauthor={Niklas Beisert}}
\hypersetup{pdfsubject={Manual for the LaTeX2e Package childdoc}}
\date{30 December 2018, \textsf{v2.0}}
\maketitle

\begin{abstract}\noindent
\textsf{childdoc} is a \LaTeXe{} package
that enables the direct compilation
of document sections included by |\include|
to individual files.
\end{abstract}

\begingroup
\parskip0ex
\tableofcontents
\endgroup

%%%%%%%%%%%%%%%%%%%%%%%%%%%%%%%%%%%%%%%%%%%%%%%%%%%%%%%%%%%%%%%%%%%%%%%%%%%%%%%%
%%%%%%%%%%%%%%%%%%%%%%%%%%%%%%%%%%%%%%%%%%%%%%%%%%%%%%%%%%%%%%%%%%%%%%%%%%%%%%%%
\section{Introduction}

\LaTeX{} provides a mechanism to structure a large document (such as a book)
into a main file and several child files (containing the chapters)
using the |\include| command.
This mechanism is beneficial for documents
which span hundreds of pages in order to
make the source file(s) more manageable.
Moreover, compilation can be restricted to
selected child files by means of the |\includeonly| command.
The latter feature can be used to reduce the compilation time while editing
(this was significantly more useful in the earlier days of \LaTeX{})
or to generate a smaller document which is easier to navigate.
Another application of |\includeonly| is to generate
documents consisting of selected parts of the complete document.

However, there are a few drawbacks of the plain |\include| mechanism:
\begin{itemize}
\item
The child files cannot be compiled on their own,
they can only be compiled via the main file.
A naive editing environment
(such as a text editor with an option
to have the current file processed by \LaTeX)
may require one to switch to the main file before compiling;
attempting to compile the child file produces errors.
\item
The main file must be modified (each time)
to adjust the |\includeonly| command
to the present needs. This easily leaves the main file in a messy state.
\item
The generated document will always carry the filename
of the main document. This is inconvenient if
several child files are to be compiled and
to be kept for distribution.
\end{itemize}

The present package provides a simple interface
to make child files individually compilable by \LaTeX{}.
Compiling a child file then has the same effect as compiling
the main file with an |\includeonly| command
to select the appropriate child.
Moreover the generated document will carry the name of the child
rather than the main file.
This resolves all three above issues.

This feature is meant to make the editing of books,
thesis documents and lecture notes somewhat more convenient.
However, the package can also be used efficiently for
composing a series of documents (such as exercise sheets)
which are typically distributed individually.
It then assists the author in generating the individual documents
(potentially in different versions)
as well as a document containing the collected series.
Another application is in developing style files
or other kinds of included material
where compilation of the style file could redirect
to a sample or test file.

%%%%%%%%%%%%%%%%%%%%%%%%%%%%%%%%%%%%%%%%%%%%%%%%%%%%%%%%%%%%%%%%%%%%%%%%%%%%%%%%
%%%%%%%%%%%%%%%%%%%%%%%%%%%%%%%%%%%%%%%%%%%%%%%%%%%%%%%%%%%%%%%%%%%%%%%%%%%%%%%%
\section{Usage}

First of all, the package \textsf{childdoc} is \emph{not} a standard
\LaTeXe{} |.sty| style file! Therefore it needs to be invoked in
a non-standard way.

%%%%%%%%%%%%%%%%%%%%%%%%%%%%%%%%%%%%%%%%%%%%%%%%%%%%%%%%%%%%%%%%%%%%%%%%%%%%%%%%
\subsection{Included Files}
\label{sec:include}

%%%%%%%%%%%%%%%%%%%%%%%%%%%%%%%%%%%%%%%%
\DescribeMacro{\childdocmain}
To use the package, add the commands
\begin{center}
\begin{tabular}{l}
|\input{childdoc.def}|\\
|\childdocmain{}|\\
\end{tabular}
\end{center}
at the very top of the main \LaTeX{} file,
in particular \emph{before} the |\documentclass| statement!
The argument of |\childdocmain| should be left empty
(but it must be present).

%%%%%%%%%%%%%%%%%%%%%%%%%%%%%%%%%%%%%%%%
\DescribeMacro{\childdocof}
Furthermore, add the commands
\begin{center}
\begin{tabular}{l}
|\input{childdoc.def}|\\
|\childdocof{|\textit{main}|}|\\
\end{tabular}
\end{center}
at the top of every child file \textit{child}
which is included by |\include{|\textit{child}|}|
from within the main file
(or at least for those files to be compiled individually).
The argument \textit{main} must be the filename of the main file.

There are a couple of
considerations in setting up the main and child documents:

%%%%%%%%%%%%%%%%%%%%%%%%%%%%%%%%%%%%%%%%
\paragraph{Restrictions.}

Please note the following restrictions:
\begin{itemize}
\item
|\childdocmain| must be called with one argument \textit{main}
to ensure compatibility with earlier version of the package.
It must either be empty (|\childdocmain{}|)
or precisely match the filename of the main file in which it is specified.
See \secref{sec:detection} for further information.
\item
The filename \textit{main} must be specified without the |.tex| extension.
\item
The filename \textit{main} is case sensitive
(even in case-insensitive file systems)
due to internal string comparison.
\item
The argument \textit{main} should be fully expanded, it cannot be a macro.
\item
Subdirectories and special characters should be avoided in filenames.
\item
The command |\childdocmain{|\textit{main}|}| must be followed by a whitespace.
It should not be followed immediately by another command
or by a comment mark `|%|'.
This is because the \TeX{} parser reads the token immediately following
the argument of |\childdocmain| and puts it
at the beginning of every child section;
however, a white\-space is ignored.
\end{itemize}

%%%%%%%%%%%%%%%%%%%%%%%%%%%%%%%%%%%%%%%%
\paragraph{Content of Main File.}

It is advisable to place all content in the child files included by |\include|.
Any output contained in the main file will appear in all child documents
unless suppressed manually;
it cannot be suppressed automatically by the |\includeonly| directive
and thus should normally be avoided.
A method to include some content in the main file
by means of conditional processing is described in \secref{sec:conditional}.

%%%%%%%%%%%%%%%%%%%%%%%%%%%%%%%%%%%%%%%%
\paragraph{Page Numbering.}

When only a part of the document is compiled,
the appropriate numbering of pages
(as well as other status parameters)
is determined from the |.aux| files.
The latter contain information from previous passes.
However this information needs to propagate through
all intermediate child documents.
Therefore the page numbering in child documents may well
be inconsistent until the complete document is compiled at least once.

A useful (if unconventional) way to always ensure a consistent
page numbering is to restart the numbering in each child document
and denote the pages by `\textit{child}|.|\textit{page}'
where \textit{child} represents the chapter/section number of the child file.
This can be achieved by the command
|\numberwithin{page}{|\textit{child}|}|
of the \textsf{amsmath} package
where \textit{child} can be |chapter| or |section|
depending on the chosen structuring.
Alternatively, one can modify the macro |\thepage| appropriately
and reset the counter |page| at the start of each child file.

%%%%%%%%%%%%%%%%%%%%%%%%%%%%%%%%%%%%%%%%%%%%%%%%%%%%%%%%%%%%%%%%%%%%%%%%%%%%%%%%
\subsection{Conditional Processing}
\label{sec:conditional}

The package provides a mechanism to compile different versions
of a document. To customise the versions further some conditional processing
can come in handy to distinguish which version is being compiled.
The package provides two macros to describe the compilation context:

%%%%%%%%%%%%%%%%%%%%%%%%%%%%%%%%%%%%%%%%
\DescribeMacro{\ifchilddoc}
The conditional |\ifchilddoc| distinguishes between the compilation of
child documents and the main document:
%
\begin{center}
|\ifchilddoc |\textit{child-code}| |[|\||else |\textit{main-code}]| \||fi|
\end{center}

%%%%%%%%%%%%%%%%%%%%%%%%%%%%%%%%%%%%%%%%
\DescribeMacro{\childdocname}
\DescribeMacro{\childdocjob}
The macro |\childdocname| contains the filename (without extension)
of the main or child file being processed.
Note that |\childdocjob| will always contain the name of the main file.

%%%%%%%%%%%%%%%%%%%%%%%%%%%%%%%%%%%%%%%%
\paragraph{Title Page.}

Conditional processing can be used to include a title or banner page
in the main document when proper precautions are taken.
Importantly, the code in the main file should ensure that the page counter
(as well as other status parameters which are stored in the |.aux| files)
takes the same value after the conditional processing.
Otherwise the page numbers may take divergent values
depending on which part is compiled.

For example, a title page could be declared by:
%
\begin{center}
\begin{tabular}{l}
|\ifchilddoc\||else|\\
|\addtocounter{page}{-1}|\\
\textit{code for title page}\\
|\newpage|\\
|\||fi|
\end{tabular}
\end{center}
%
A banner page for the child documents can be generated by:
%
\begin{center}
\begin{tabular}{l}
|\ifchilddoc|\\
|\addtocounter{page}{-1}|\\
\textit{code for banner page}\\
|\newpage|\\
|\||fi|
\end{tabular}
\end{center}
%
Here one could write a message such as:
\begin{center}
|This is the part \childdocname{} of \childdocjob{}.|
\end{center}

%%%%%%%%%%%%%%%%%%%%%%%%%%%%%%%%%%%%%%%%%%%%%%%%%%%%%%%%%%%%%%%%%%%%%%%%%%%%%%%%
\subsection{Flags}
\label{sec:flags}

The package makes it easy to generate different versions
of the main or child documents.
To this end compilation flags can be defined
and assigned different default values.
They will be particularly useful in conjunction
with the forwarding mechanism described in \secref{sec:forward}.

For example, it may be useful to have a flag |\version|
which can be set to |draft| or |final|.
The document source will contain some conditional code
depending on the value of |\version|.
Suppose further, the flag should default to |final| for the main file
and to |draft| for child files
which is a natural assignment for editing the document.
This is achieved by placing the following code
in the preamble of the main document
(below the |\childdocmain| directive):
%
\begin{center}
\begin{tabular}{l}
|\ifchilddoc|\\
|\providecommand{\version}{draft}|\\
|\||else|\\
|\providecommand{\version}{final}|\\
|\||fi|
\end{tabular}
\end{center}
%
The definition by |\providecommand| makes sure
that previous definitions are not overwritten.
Further statements |\providecommand{\version}{...}|
can thus be added before the above code to override it.

For the main file, one might add a line
(between |\childdocmain| and the above block)
%
\begin{center}
|%\ifchilddoc\||else\providecommand{\version}{draft}\||fi|
\end{center}
%
which can be uncommented to produce a draft version.
Likewise one can add a line to the very top of a child file
(above the |\childdocof{|\textit{main}|}| directive)
%
\begin{center}
|%\providecommand{\version}{final}|
\end{center}
%
which can be uncommented to produce the final version of this child document.

%%%%%%%%%%%%%%%%%%%%%%%%%%%%%%%%%%%%%%%%%%%%%%%%%%%%%%%%%%%%%%%%%%%%%%%%%%%%%%%%
\subsection{Forwarding}
\label{sec:forward}

Different versions of the main or child documents
using compilation flags as described in \secref{sec:flags}
can be (permanently) stored in different files
for convenient compilation, viewing and distribution.
To this end, the package defines a command
to pass on compilation to a different file:

%%%%%%%%%%%%%%%%%%%%%%%%%%%%%%%%%%%%%%%%
\DescribeMacro{\childdocforward}
The command |\childdocforward| redirects processing to
another source file:
%
\begin{center}
\begin{tabular}{l}
|\input{childdoc.def}|\\
|\childdocforward[|\textit{main}|]{|\textit{dest}|}|\\
\end{tabular}
\end{center}
%
The argument \textit{dest} is the destination file
(without extension).
It should be the main file or one of the child files.
Note that further \textsf{childdoc} directives
such as |\childdocof| and |\childdocforward|
in the indicated file will be processed in this form.
The optional argument \textit{main}
passes on directly to the main file \textit{main}
while pretending to compile the child \textit{dest}.
This form behaves as if \textit{dest}
issues |\childdocof{|\textit{main}|}| right away,
and no further \textsf{childdoc} directives will be processed.

%%%%%%%%%%%%%%%%%%%%%%%%%%%%%%%%%%%%%%%%
\DescribeMacro{\...prefix}
In the alternative form |\childdocforwardprefix|,
%
\begin{center}
\begin{tabular}{l}
|\input{childdoc.def}|\\
|\childdocforwardprefix[|\textit{main}|]{|\textit{prefix}|}{|\textit{dest}|}|
\end{tabular}
\end{center}
%
the destination file is determined by a pattern
depending on the current file:
To make this work, the current file must be called
`{\textit{prefix}\hspace{0.2em}\textit{suffix}}'
with \textit{prefix} matching precisely the argument.
Processing is then passed on to the file
`{\textit{dest}\hspace{0.2em}\textit{suffix}}'.
Surely, the same effect is achieved by
directly specifying the
argument `{\textit{dest}\hspace{0.2em}\textit{suffix}}'
in the first form.
However, that requires to set up a different file
for each child. With the alternative form of the command
all these files can have exactly the same content
which simplifies setting them up and maintaining them.

For example, the following file |draft.tex|
with a compilation flag |\version| as described in \secref{sec:flags}
compiles the main document as a draft:
%
\begin{center}
\begin{tabular}{l}
|\def\version{draft}|\\
|\input{childdoc.def}|\\
|\childdocforward{|\textit{main}|}|
\end{tabular}
\end{center}
%
Likewise, the following files |final|\textit{nn}|.tex|
compile the final version of the child document
|child|\textit{nn}|.tex|:
%
\begin{center}
\begin{tabular}{l}
|\def\version{final}|\\
|\input{childdoc.def}|\\
|\childdocforwardprefix{final}{child}|
\end{tabular}
\end{center}
%

Note that when several versions of a main file and/or of each child file
are to be generated, it may be convenient to set up a |Makefile| or
shell script to automatise the process.

%%%%%%%%%%%%%%%%%%%%%%%%%%%%%%%%%%%%%%%%%%%%%%%%%%%%%%%%%%%%%%%%%%%%%%%%%%%%%%%%
\subsection{Command Line Processing}
\label{sec:commandline}

The effect of redirection files can also be achieved by invoking
the \LaTeX{} compiler with a more elaborate command line.
Most conveniently this should be done as part
of a shell script or a |Makefile|.

When using \textsf{childdoc} in the main file, the following
command lines effectively perform a redirection
(note that depending on the shell being used,
backslashes may have to be doubled: `|\|' $\to$ `|\\|'):
%
\begin{center}
|... -jobname "|\textit{target}|" |\\|"|[\textit{flags}]%
|\input{childdoc.def}\childdocforward[|\textit{main}|]{|\textit{dest}|}"|
\end{center}
%
Here \textit{target} is the name of the output file,
\textit{main} is the name of the main file
and \textit{dest} is the name of the main or child file to be processed
(all filenames without extensions).
The optional argument \textit{main} can be omitted
if \textit{main} matches \textit{dest}.
Optionally, compilation \textit{flags} can be defined via |\def| commands.
This command line makes the \TeX{} engine believe
it is compiling the file \textit{target}
whose content is specified as the latter parameter.
The provided code then forwards the processing to
\textit{main} or \textit{dest} as described in \secref{sec:forward}.

%%%%%%%%%%%%%%%%%%%%%%%%%%%%%%%%%%%%%%%%%%%%%%%%%%%%%%%%%%%%%%%%%%%%%%%%%%%%%%%%
\subsection{Include by Input}
\label{sec:input}

Including child documents by |\include| has some restrictions by design.
Most notably, the content of a child document always occupies
its own set of pages; pages cannot be shared between child documents.
Usually, this behaviour makes perfect sense
because each child document contain an essential part of the document.
However, in some situations it may be desirable to compose
a document from a collection of parts
without having mandatory page breaks between then.
For this case, the package
provides a mechanism to include parts
by |\input| which can also be processed individually.
However, by construction this mechanism
requires manual handling of the content to be output.

%%%%%%%%%%%%%%%%%%%%%%%%%%%%%%%%%%%%%%%%
\DescribeMacro{\ifchilddocmanual}
The main file should be prepared as usual, see \secref{sec:include}.
However, the document body must make a distinction
between processing of an individual part and of the main document, e.g.:
%
\begin{center}
\begin{tabular}{l}
|\ifchilddocmanual|\\
|\input{\childdocname}|\\
|\||else|\\
\textit{document body with }|\input{|\textit{part}|}|\\
|\||fi|
\end{tabular}
\end{center}
%
The conditional |\ifchilddocmanual| is true whenever
a part to be included by |\input| is being compiled,
and the name of the part is stored in |\childdocname|.

%%%%%%%%%%%%%%%%%%%%%%%%%%%%%%%%%%%%%%%%
\DescribeMacro{\childdocby}
Each part to be included by |\input| should start with:
%
\begin{center}
\begin{tabular}{l}
|\input{childdoc.def}|\\
|\childdocby{|\textit{main}|}|\\
\end{tabular}
\end{center}
%
The directive |\childdocby| is similar to |\childdocof|
described in \secref{sec:include},
but the subsequent selection of content must be done manually.
To that end, both |\ifchilddoc| and |\ifchilddocmanual|
will be true upon processing of a part,
and the name of the part is stored in |\childdocname|.
Note that |\jobname| will be set to the filename of the current part
so that each part receives an individual |.aux| file
that does not interfere with the |.aux| file(s) of the main document.
This behaviour can be altered by the alternative form
|\childdocby[*]{|\textit{main}|}| (with a non-empty optional argument)
which uses the |.aux| file of the main document
by setting |\jobname| to \textit{main}.

%%%%%%%%%%%%%%%%%%%%%%%%%%%%%%%%%%%%%%%%%%%%%%%%%%%%%%%%%%%%%%%%%%%%%%%%%%%%%%%%
\subsection{Driver Development}
\label{sec:driver}

The \textsf{childdoc} mechanism can also be use for the development
of definition files such as \LaTeX{} styles or classes.
This case differs from the above setup with multiple parts
included by |\include| in that no |\includeonly| should be invoked.
This can be achieved by starting the include file
(before |\ProvidesPackage|) with:
%
\begin{center}
\begin{tabular}{l}
|\input{childdoc.def}|\\
|\childdocforward{|\textit{main}|}|\\
\end{tabular}
\end{center}
%
or alternatively with:
%
\begin{center}
\begin{tabular}{l}
|\input{childdoc.def}|\\
|\childdocby{|\textit{main}|}|\\
\end{tabular}
\end{center}
%
Both forms have slightly different effects as described above.
The main file is prepared as usual, see \secref{sec:include}.

%%%%%%%%%%%%%%%%%%%%%%%%%%%%%%%%%%%%%%%%%%%%%%%%%%%%%%%%%%%%%%%%%%%%%%%%%%%%%%%%
\subsection{Legacy Detection}
\label{sec:detection}

The directive |\childdocmain| in the main file can detect
whether the complete document or merely a child is to be compiled
even without using the directive |\childdocof|.
This method is deprecated because it is less robust
and there is no compelling reason to use it;
it is merely provided for backward compatibility
and it may be removed in future versions.

If the detection mechanism is to be used,
it is mandatory to correctly specify
the filename of the main file as the argument of |\childdocmain|:
%
\begin{center}
\begin{tabular}{l}
|\input{childdoc.def}|\\
|\childdocmain{|\textit{main}|}|\\
\end{tabular}
\end{center}
%
If |\jobname| does not match the argument \textit{main} of |\childdocmain|,
it is assumed that |\jobname| points to the child file to be compiled.
When using |\childdocmain| with the main file specified as argument,
it suffices to start a child file
with just |\input{|\textit{main}|}|
without loading of the package and using |\childdocof|.
If instead all processing is done
with the appropriate \textsf{childdoc} directives,
the argument of \textit{main} of |\childdocmain| can be empty.

An alternative version of the command line processing described
in \secref{sec:commandline} using the detection mechanism reads:
%
\begin{center}
|... -jobname "|\textit{target}|" "|[\textit{flags}]%
[|\def\jobname{|\textit{dest}|}|]|\input{|\textit{main}|}"|
\end{center}

%%%%%%%%%%%%%%%%%%%%%%%%%%%%%%%%%%%%%%%%%%%%%%%%%%%%%%%%%%%%%%%%%%%%%%%%%%%%%%%%
\subsection{Manual Code}
\label{sec:manual}

In case one cannot be certain whether the definitions file |childdoc.def|
is installed on the target \TeX{} distribution
and one prefers not to ship it,
it is conceivable to paste a few relevant commands into the sources.

To that end, drop all statements |\input{childdoc.def}|
and perform the replacements as outlined below.
Instead of |\childdocmain{|\textit{main}|}| add the following code
to the top of the main file:
%
\begin{center}
\begin{tabular}{l}
|\||ifdefined\childdocname\endinput\||fi\newif\ifchilddoc|\\
|\edef\childdocname{\scantokens\expandafter{\jobname\noexpand}}|\\
|\def\childdocmain{|\textit{main}|}\||ifx\childdocmain\childdocname\||else|\\
|\childdoctrue\includeonly{\childdocname}\let\jobname\childdocmain\||fi|\\
\end{tabular}
\end{center}
%
Instead of |\childdocof{|\textit{main}|}| just include the main file
at the top of each child file:
%
\begin{center}
|\input{|\textit{main}|}|
\end{center}
%
A simple redirection |\childdocforward{|\textit{dest}|}| is achieved by:
%
\begin{center}
|\def\jobname{|\textit{dest}|}\input{\jobname}|
\end{center}
%
The redirection with prefix
|\childdocforwardprefix[|\textit{prefix}|]{|\textit{dest}|}|
is accomplished by:
%
\begin{center}
\begin{tabular}{l}
|{\edef\jobname{\scantokens\expandafter{\jobname\noexpand}}|\\
|\def\redirectjob |\textit{prefix}|#1~~~{\gdef\jobname{|\textit{dest}|#1}}|\\
|\expandafter\redirectjob\jobname~~~}\input{\jobname}|
\end{tabular}
\end{center}

In an alternative approach,
child documents can be compiled by a specific command line
without additional code or specific definitions:
%
\begin{center}
|... -jobname "|\textit{target}|" "|[\textit{flags}]%
|\includeonly{|\textit{dest}|}\input{|\textit{main}|}"|
\end{center}
%

%%%%%%%%%%%%%%%%%%%%%%%%%%%%%%%%%%%%%%%%%%%%%%%%%%%%%%%%%%%%%%%%%%%%%%%%%%%%%%%%
%%%%%%%%%%%%%%%%%%%%%%%%%%%%%%%%%%%%%%%%%%%%%%%%%%%%%%%%%%%%%%%%%%%%%%%%%%%%%%%%
\section{Information}

%%%%%%%%%%%%%%%%%%%%%%%%%%%%%%%%%%%%%%%%%%%%%%%%%%%%%%%%%%%%%%%%%%%%%%%%%%%%%%%%
\subsection{Copyright}

Copyright \copyright{} 2017--2018 Niklas Beisert

This work may be distributed and/or modified under the
conditions of the \LaTeX{} Project Public License, either version 1.3
of this license or (at your option) any later version.
The latest version of this license is in
  \url{http://www.latex-project.org/lppl.txt}
and version 1.3 or later is part of all distributions of \LaTeX{}
version 2005/12/01 or later.

This work has the LPPL maintenance status `maintained'.

The Current Maintainer of this work is Niklas Beisert.

This work consists of the files |README.txt|, |childdoc.ins| and |childdoc.dtx|
as well as the derived files |childdoc.def|, |cdocsamp.tex|
with |cdocsch1.tex|, |cdocsch2.tex|, |cdocspt3.tex|, |cdocspt4.tex|,
|cdocsdrf.tex|, |cdocsfn1.tex|, |cdocsfn2.tex|
as well as |childdoc.pdf|.

%%%%%%%%%%%%%%%%%%%%%%%%%%%%%%%%%%%%%%%%%%%%%%%%%%%%%%%%%%%%%%%%%%%%%%%%%%%%%%%%
\subsection{Files and Installation}

The package consists of the files:
%
\begin{center}
\begin{tabular}{ll}
    |README.txt|   & readme file \\
    |childdoc.ins| & installation file \\
    |childdoc.dtx| & source file \\
    |childdoc.def| & definition file \\
    |cdocsamp.tex| & sample main file \\
    |cdocsch1.tex| & sample include file \\
    |cdocsch2.tex| & sample include file \\
    |cdocspt3.tex| & sample part file \\
    |cdocspt4.tex| & sample part file \\
    |cdocsdrf.tex| & sample redirection file \\
    |cdocsfn1.tex| & sample redirection file \\
    |cdocsfn2.tex| & sample redirection file \\
    |childdoc.pdf| & manual
\end{tabular}
\end{center}
%
The distribution consists of the files
|README.txt|, |childdoc.ins| and |childdoc.dtx|.
%
\begin{itemize}
\item
Run (pdf)\LaTeX{} on |childdoc.dtx|
to compile the manual |childdoc.pdf| (this file).
\item
Run \LaTeX{} on |childdoc.ins| to create the definitions file |childdoc.def|
and the sample |cdocsamp.tex| with include files
|cdocsch1.tex|, |cdocsch2.tex|, |cdocspt3.tex|, |cdocspt4.tex|,
|cdocsdrf.tex|, |cdocsfn1.tex|, |cdocsfn2.tex|.
Then copy the file |childdoc.def| to an appropriate directory of your \LaTeX{}
distribution, e.g.\ \textit{texmf-root}|/tex/latex/childdoc|.
\end{itemize}

%%%%%%%%%%%%%%%%%%%%%%%%%%%%%%%%%%%%%%%%%%%%%%%%%%%%%%%%%%%%%%%%%%%%%%%%%%%%%%%%
\subsection{Related CTAN Packages}

There are several other packages which offer a similar functionality:
%
\begin{itemize}
\item
The packages
\href{http://ctan.org/pkg/docmute}{\textsf{docmute}},
\href{http://ctan.org/pkg/includex}{\textsf{includex}} and
\href{http://ctan.org/pkg/standalone}{\textsf{standalone}}
provide commands to include only the document body of
a child file thus allowing both files to be compiled individually.
\item
The packages \href{http://ctan.org/pkg/subdocs}{\textsf{subdocs}}
and \href{http://ctan.org/pkg/subfiles}{\textsf{subfiles}}
provide structures in which the main and child documents can be
encapsulated and allowing them to be compiled individually.
The inclusion mechanism is different from the conventional |\include|.
\item
The package \href{http://ctan.org/pkg/combine}{\textsf{combine}}
is an elaborate solution to combine several documents into one.
\end{itemize}
%
See also the CTAN topic \href{http://ctan.org/topic/subdocs}{\textsf{subdocs}}
for further related packages.
The present package differs from the above solutions in that
a document structure constructed with the conventional |\include| mechanism
just needs two extra commands at the top of every file
such that all constituent files can be compiled individually.

%%%%%%%%%%%%%%%%%%%%%%%%%%%%%%%%%%%%%%%%%%%%%%%%%%%%%%%%%%%%%%%%%%%%%%%%%%%%%%%%
%\subsection{Feature Suggestions}
%
%The following is a list of features which may be useful for future
%versions of this package:
%%
%\begin{itemize}
%\item
%\ldots
%\end{itemize}

%%%%%%%%%%%%%%%%%%%%%%%%%%%%%%%%%%%%%%%%%%%%%%%%%%%%%%%%%%%%%%%%%%%%%%%%%%%%%%%%
\subsection{Revision History}

%%%%%%%%%%%%%%%%%%%%%%%%%%%%%%%%%%%%%%%%
\paragraph{v2.0:} 2018/12/30

\begin{itemize}
\item
immediate forward processing
\item
added |\childdocby| mechanism
\item
manual restructured
\end{itemize}

%%%%%%%%%%%%%%%%%%%%%%%%%%%%%%%%%%%%%%%%
\paragraph{v1.6:} 2018/01/17

\begin{itemize}
\item
application for development of include files
\item
corrections to manual
\end{itemize}

%%%%%%%%%%%%%%%%%%%%%%%%%%%%%%%%%%%%%%%%
\paragraph{v1.5:} 2017/05/21

\begin{itemize}
\item
more complete structuring introduced
\item
|\childdocof| introduced
\item
|\childdoc| renamed to |\childdocmain|
\item
|\childredirect| renamed to |\childdocforward| and |\childdocforwardprefix|
and functionality expanded
\end{itemize}

%%%%%%%%%%%%%%%%%%%%%%%%%%%%%%%%%%%%%%%%
\paragraph{v1.0:} 2017/04/27

\begin{itemize}
\item
manual and install package
\item
first version published on CTAN
\end{itemize}

%%%%%%%%%%%%%%%%%%%%%%%%%%%%%%%%%%%%%%%%
\paragraph{v0.6:} 2017/04/26

\begin{itemize}
\item
redirection mechanism added
\end{itemize}

%%%%%%%%%%%%%%%%%%%%%%%%%%%%%%%%%%%%%%%%
\paragraph{v0.5:} 2017/04/26

\begin{itemize}
\item
functionality in definition file
\end{itemize}


%%%%%%%%%%%%%%%%%%%%%%%%%%%%%%%%%%%%%%%%%%%%%%%%%%%%%%%%%%%%%%%%%%%%%%%%%%%%%%%%
%%%%%%%%%%%%%%%%%%%%%%%%%%%%%%%%%%%%%%%%%%%%%%%%%%%%%%%%%%%%%%%%%%%%%%%%%%%%%%%%
%%%%%%%%%%%%%%%%%%%%%%%%%%%%%%%%%%%%%%%%%%%%%%%%%%%%%%%%%%%%%%%%%%%%%%%%%%%%%%%%
\appendix

\settowidth\MacroIndent{\rmfamily\scriptsize 000\ }

 \DocInput{childdoc.dtx}

\end{document}
%</driver>
% \fi
%
% %%%%%%%%%%%%%%%%%%%%%%%%%%%%%%%%%%%%%%%%%%%%%%%%%%%%%%%%%%%%%%%%%%%%%%%%%%%%%%
% %%%%%%%%%%%%%%%%%%%%%%%%%%%%%%%%%%%%%%%%%%%%%%%%%%%%%%%%%%%%%%%%%%%%%%%%%%%%%%
% \section{Sample}
%\iffalse
%<*samplemain>
%\fi
%
% The following presents a sample document
% with two chapters, two parts, a title page,
% a compile flag as well as three forwarding files to set the flag.
% It consists of eight |.tex| files:
% \begin{center}
% \begin{tabular}{ll}
% |cdocsamp.tex|&main file\\
% |cdocsch1.tex|&include file for chapter 1\\
% |cdocsch2.tex|&include file for chapter 2\\
% |cdocspt3.tex|&include file for part 3\\
% |cdocspt4.tex|&include file for part 4\\
% |cdocsdrf.tex|&forwarding file for main file in draft mode\\
% |cdocsfi1.tex|&forwarding file for final version of chapter 1\\
% |cdocsfi2.tex|&forwarding file for final version of chapter 2\\
% \end{tabular}
% \end{center}
% Each of the eight files can be compiled directly by the \LaTeX{} compiler.
%
% %%%%%%%%%%%%%%%%%%%%%%%%%%%%%%%%%%%%%%
% \paragraph{Main File.}
%
% The main file is called |cdocsamp.tex|.
%
% Load the \textsf{childdoc} definitions and
% declare the filename for the main document:
%    \begin{macrocode}
\input{childdoc.def}
\childdocmain{}
%    \end{macrocode}

% Optional override for |\version| flag:
%    \begin{macrocode}
%%\ifchilddoc\else\providecommand{\version}{draft}\fi
%    \end{macrocode}

% Define the default values for the |\version| flag
% (|final| for the main file and |draft| for childs):
%    \begin{macrocode}
\ifchilddoc
\providecommand{\version}{draft}
\else
\providecommand{\version}{final}
\fi
%    \end{macrocode}

% Load the standard document class:
%    \begin{macrocode}
\documentclass[12pt]{article}
%    \end{macrocode}

% Start the document body:
%    \begin{macrocode}
\begin{document}
%    \end{macrocode}

% Declare a title page.
% Print title, part of document being processed and version flag:
%    \begin{macrocode}
\addtocounter{page}{-1}
\begin{center}
{\LARGE\bfseries{}childdoc example\par}
\vspace{1cm}
\ifchilddoc
\ifchilddocmanual part\else chapter\fi:
`\childdocname' of `\childdocjob'\par
\else
main document: `\childdocjob'\par
\fi
version: \version\par
\end{center}
\newpage
%    \end{macrocode}

% Manually include selected file,
% otherwise process as usual:
%    \begin{macrocode}
\ifchilddocmanual
\section*{part `\childdocname'}
\input{\childdocname}
\else
%    \end{macrocode}

% Include the two chapters:
%    \begin{macrocode}
\include{cdocsch1}
\include{cdocsch2}
%    \end{macrocode}

% Include the two parts unless only chapters should be displayed:
%    \begin{macrocode}
\ifchilddoc\else
\section{part three}
\input{cdocspt3}
\section{part four}
\input{cdocspt4}
\fi
%    \end{macrocode}

% Process as usual until here:
%    \begin{macrocode}
\fi
%    \end{macrocode}

% End of document body:
%    \begin{macrocode}
\end{document}
%    \end{macrocode}
%\iffalse
%</samplemain>
%\fi
%
% %%%%%%%%%%%%%%%%%%%%%%%%%%%%%%%%%%%%%%
% \paragraph{Chapter Include Files.}
%
% The include files are called |cdocsch1.tex| and |cdocsch2.tex|.
%
%\iffalse
%<*samplechap1|samplechap2>
%\fi

% Optional override for |\version| flag:
%    \begin{macrocode}
%%\providecommand{\version}{final}
%    \end{macrocode}

% Include the main document:
%    \begin{macrocode}
\input{childdoc.def}
\childdocof{cdocsamp}
%    \end{macrocode}

%\iffalse
%</samplechap1|samplechap2>
%\fi
%
%\iffalse
%<*samplechap1>
%\fi
% Some text for chapter 1:
%    \begin{macrocode}
\section{one}
some text in chapter one
%    \end{macrocode}

%\iffalse
%</samplechap1>
%\fi
% Some text for chapter 2:
%\iffalse
%<*samplechap2>
%\fi
%    \begin{macrocode}
\section{two}
more text in chapter two
%    \end{macrocode}

%\iffalse
%</samplechap2>
%\fi
%
% %%%%%%%%%%%%%%%%%%%%%%%%%%%%%%%%%%%%%%
% \paragraph{Part Include Files.}
%
% The include files are called |cdocspt3.tex| and |cdocspt4.tex|.
%
%\iffalse
%<*samplepart3|samplepart4>
%\fi

% Optional override for |\version| flag:
%    \begin{macrocode}
%%\providecommand{\version}{final}
%    \end{macrocode}

% Include the main document:
%    \begin{macrocode}
\input{childdoc.def}
\childdocby{cdocsamp}
%    \end{macrocode}

%\iffalse
%</samplepart3|samplepart4>
%\fi
%
%\iffalse
%<*samplepart3>
%\fi
% Some text for part 3:
%    \begin{macrocode}
some text in part three
%    \end{macrocode}

%\iffalse
%</samplepart3>
%\fi
% Some text for part 4:
%\iffalse
%<*samplepart4>
%\fi
%    \begin{macrocode}
more text in part four
%    \end{macrocode}

%\iffalse
%</samplepart4>
%\fi
%
% %%%%%%%%%%%%%%%%%%%%%%%%%%%%%%%%%%%%%%
% \paragraph{Forwarding for a Complete Draft.}
%
% The following forwarding file |cdocsdrf.tex|
% compiles the main document in draft mode:
%\iffalse
%<*sampledraft>
%\fi
%    \begin{macrocode}
\def\version{draft}
\input{childdoc.def}
\childdocforward{cdocsamp}
%    \end{macrocode}

%\iffalse
%</sampledraft>
%\fi
%
% %%%%%%%%%%%%%%%%%%%%%%%%%%%%%%%%%%%%%%
% \paragraph{Forwarding for Final Version of the Chapters.}
%
% The following forwarding files |cdocsfn1.tex| and |cdocsfn2.tex|
% (with identical content)
% compile the final versions of the child documents
% |cdocsch1.tex| and |cdocsch2.tex|, respectively:
%\iffalse
%<*samplefinal>
%\fi
%    \begin{macrocode}
\def\version{final}
\input{childdoc.def}
\childdocforwardprefix[cdocsamp]{cdocsfn}{cdocsch}
%    \end{macrocode}

%\iffalse
%</samplefinal>
%\fi
%
% %%%%%%%%%%%%%%%%%%%%%%%%%%%%%%%%%%%%%%
% \paragraph{Command Line Processing.}
%
% The following three command lines generate the output files
% |cdocscld|, |cdocscl1| and |cdocscl2|
% which should be identical to
% |cdocsdrf|, |cdocsch1| and |cdocsfn2|, respectively:
% \begin{center}
% \begin{tabular}{l}
% |latex -jobname cdocscld \|\\
% |  "\def\version{draft}\input{childdoc.def}\childdocforward{cdocsamp}"|\\
% |latex -jobname cdocscl1 \|\\
% |  "\input{childdoc.def}\childdocforward[cdocsamp]{cdocsch1}"|\\
% |latex -jobname cdocscl2 \|\\
% |  "\def\version{final}\input{childdoc.def}\childdocforward{cdocsch2}"|
% \end{tabular}
% \end{center}
% Note that the trailing backslash on each first line
% merely continues the input to the second line
% (for convenient cut ant paste).
% Furthermore, the command |latex| can be replaced by any
% of its alternative versions such as |pdflatex|.
%
% %%%%%%%%%%%%%%%%%%%%%%%%%%%%%%%%%%%%%%%%%%%%%%%%%%%%%%%%%%%%%%%%%%%%%%%%%%%%%%
% %%%%%%%%%%%%%%%%%%%%%%%%%%%%%%%%%%%%%%%%%%%%%%%%%%%%%%%%%%%%%%%%%%%%%%%%%%%%%%
% \section{Implementation}
%\iffalse
%<*package>
%\fi
%
% This section describes the definitions file |childdoc.def|.

% The definitions cannot be loaded using |\usepackage| or |\RequirePackage|
% which has a mechanism to prevent loading a style file more than once.
% When loading the definitions by means of |\input|
% multiple instances have to be prevented manually:
%\iffalse
%This code needs to be before the `\ProvidesFile' directive
%which is defined at the beginning of this file.
%Therefore it is also placed there and commented out here.
%</package>
%<*discard>
%\fi
%    \begin{macrocode}
\ifdefined\childdocmain\endinput\fi
%    \end{macrocode}
%\iffalse
%</discard>
%<*package>
%\fi
%
% \macro{\ifchilddoc}
% \macro{\ifchilddocmanual}
% The conditional |\ifchilddoc| tells whether a
% child (true) or main (false) document is being compiled.
% The conditional |\ifchilddocmanual| tells whether
% the |\includeonly| mechanism is used (false) or
% the selection of child files must be performed manually (true).
% The definitions initialise to false:
%    \begin{macrocode}
\newif\ifchilddoc
\newif\ifchilddocmanual
%    \end{macrocode}

% \macro{\childdocname}
% \macro{\childdocjob}
% The macro |\childdocname| stores the name of the main document
% to be compiled. The macro |\childdocjob| stores the name of
% the document on which the \LaTeX{} compiler was originally invoked.
% The content of |\jobname| cannot be compared
% to filenames specified in the source due to different catcodes.
% The following code rescans |\jobname|, stores the result
% in |\childdocname| and saves a copy in |\childdocjob|:
%    \begin{macrocode}
\edef\childdocname{\scantokens\expandafter{\jobname\noexpand}}
\let\childdocjob\childdocname
%    \end{macrocode}

% \macro{\childdocdisable}
% The macro |\childdocdisable| prevents the main file
% from being processed more than once.
% At this stage, the main document command |\childdocmain|
% is assumed to be called once again where it should do nothing.
% Any subsequent call to it should prevent
% a secondary processing of the main document
% It overwrites the forwarding commands
% |\childdocof| and |\childdocforward|
% with empty macros to prevent further inclusions of the main document:
%    \begin{macrocode}
\newcommand{\childdocdisable}
{
  \renewcommand{\childdocmain}[1]{\renewcommand{\childdocmain}[1]{\endinput}}
  \renewcommand{\childdocof}[1]{}
  \renewcommand{\childdocby}[2][]{}
  \renewcommand{\childdocforward}[2][]{}
  \renewcommand{\childdocdisable}{}
}
%    \end{macrocode}

% \macro{\childdocmain}
% The macro |\childdocmain| is to be called at the top of the main file
% with nothing or the main filename (without extension) as argument.
% First, it breaks loops.
% If the argument is not empty and does not match |\childdocname|
% (which is set by the first inclusion of |childdoc.def|),
% |\ifchilddoc| is set to true, |\includeonly| is applied to the child file
% and |\jobname| is set to the main file
% (for proper handling of |.aux| files):
%    \begin{macrocode}
\newcommand{\childdocmain}[1]
{
  \childdocdisable\childdocmain{}
  \if?#1?\else
    \begingroup
      \def\childdoctmp{#1}
      \ifx\childdoctmp\childdocname
        \def\childdoctmp{}
      \else
        \def\childdoctmp
        {
          \childdoctrue
          \includeonly{\childdocname}
          \def\childdocjob{#1}
          \def\jobname{#1}
        }
      \fi
      \expandafter
    \endgroup
    \childdoctmp
  \fi
}
%    \end{macrocode}

% \macro{\childdocof}
% The command |\childdocof| redirects
% compilation to the main file |#1|.
%    \begin{macrocode}
\newcommand{\childdocof}[1]
{
  \childdocdisable
  \childdoctrue
  \includeonly{\childdocname}
  \def\jobname{#1}
  \def\childdocjob{#1}
  \input{#1}
}
%    \end{macrocode}

% \macro{\childdocby}
% The command |\childdocby| ....
%    \begin{macrocode}
\newcommand{\childdocby}[2][]
{
  \childdocdisable
  \childdoctrue
  \childdocmanualtrue
  \if?#1?\else
    \def\jobname{#2}
  \fi
  \def\childdocjob{#2}
  \input{#2}
  \endinput
}
%    \end{macrocode}

% \macro{\childdocforward}
% The command |\childdocforward| redirects
% compilation to the main file or
% (if the optional argument is given) a child file.
% Parameters are set as if the main file
% or a child file starting with |\childdocof| was compiled.
% Then compilation is handed over to the main file:
%    \begin{macrocode}
\newcommand{\childdocforward}[2][]
{
  \begingroup
    \if?#1?
      \def\childdoctmp
      {
        \def\childdocname{#2}
        \def\childdocjob{#2}
        \def\jobname{#2}
        \input{#2}
        \endinput
      }
    \else
      \def\childdoctmp
      {
        \childdocdisable
        \def\childdocname{#2}
        \childdoctrue
        \includeonly{#2}
        \def\childdocjob{#1}
        \def\jobname{#1}
        \input{#1}
        \endinput
      }
    \fi
    \expandafter
  \endgroup
  \childdoctmp
}
%    \end{macrocode}

% \macro{\childdocforwardprefix}
% The command |\childdocforwardprefix| redirects
% compilation to the main or a child file by means of a pattern.
% The prefix |#1| in the current filename is replaced by |#2|
% and the suffix of the current filename is kept
% (it is assumed that the filename does not contain the substring `|~~~|'
% which is used as a delimiter).
% Compilation is handed over to the new file by |\childdocforward|:
%    \begin{macrocode}
\newcommand{\childdocforwardprefix}[3][]
{
  \begingroup
    \def\childdocextract #2##1~~~{\def\childdoctmp{\childdocforward[#1]{#3##1}}}
    \expandafter\childdocextract\childdocname~~~
    \expandafter
  \endgroup
  \childdoctmp
}
%    \end{macrocode}

% \macro{\childdoc}
% The deprecated macro |\childdoc| is a legacy version of |\childdocmain|:
%    \begin{macrocode}
\newcommand{\childdoc}{\childdocmain}
%    \end{macrocode}

% \macro{\childdocredirect}
% The deprecated macro |\childdocredirect| is a legacy version
% of |\childdocforward| and |\childdocforwardprefix|:
%    \begin{macrocode}
\newcommand{\childdocredirect}[2][]
{
  \begingroup
    \if?#1?
      \def\childdoctmp{\childdocforward{#2}}
    \else
      \def\childdoctmp{\childdocforwardprefix{#1}{#2}}
    \fi
    \expandafter
  \endgroup
  \childdoctmp
}
%    \end{macrocode}

%\iffalse
%</package>
%\fi
%
\endinput

\childdocmain{}
%    \end{macrocode}

% Optional override for |\version| flag:
%    \begin{macrocode}
%%\ifchilddoc\else\providecommand{\version}{draft}\fi
%    \end{macrocode}

% Define the default values for the |\version| flag
% (|final| for the main file and |draft| for childs):
%    \begin{macrocode}
\ifchilddoc
\providecommand{\version}{draft}
\else
\providecommand{\version}{final}
\fi
%    \end{macrocode}

% Load the standard document class:
%    \begin{macrocode}
\documentclass[12pt]{article}
%    \end{macrocode}

% Start the document body:
%    \begin{macrocode}
\begin{document}
%    \end{macrocode}

% Declare a title page.
% Print title, part of document being processed and version flag:
%    \begin{macrocode}
\addtocounter{page}{-1}
\begin{center}
{\LARGE\bfseries{}childdoc example\par}
\vspace{1cm}
\ifchilddoc
\ifchilddocmanual part\else chapter\fi:
`\childdocname' of `\childdocjob'\par
\else
main document: `\childdocjob'\par
\fi
version: \version\par
\end{center}
\newpage
%    \end{macrocode}

% Manually include selected file,
% otherwise process as usual:
%    \begin{macrocode}
\ifchilddocmanual
\section*{part `\childdocname'}
\input{\childdocname}
\else
%    \end{macrocode}

% Include the two chapters:
%    \begin{macrocode}
\include{cdocsch1}
\include{cdocsch2}
%    \end{macrocode}

% Include the two parts unless only chapters should be displayed:
%    \begin{macrocode}
\ifchilddoc\else
\section{part three}
\input{cdocspt3}
\section{part four}
\input{cdocspt4}
\fi
%    \end{macrocode}

% Process as usual until here:
%    \begin{macrocode}
\fi
%    \end{macrocode}

% End of document body:
%    \begin{macrocode}
\end{document}
%    \end{macrocode}
%\iffalse
%</samplemain>
%\fi
%
% %%%%%%%%%%%%%%%%%%%%%%%%%%%%%%%%%%%%%%
% \paragraph{Chapter Include Files.}
%
% The include files are called |cdocsch1.tex| and |cdocsch2.tex|.
%
%\iffalse
%<*samplechap1|samplechap2>
%\fi

% Optional override for |\version| flag:
%    \begin{macrocode}
%%\providecommand{\version}{final}
%    \end{macrocode}

% Include the main document:
%    \begin{macrocode}
% \iffalse
%
% childdoc.dtx Copyright (C) 2017-2018 Niklas Beisert
%
% This work may be distributed and/or modified under the
% conditions of the LaTeX Project Public License, either version 1.3
% of this license or (at your option) any later version.
% The latest version of this license is in
%   http://www.latex-project.org/lppl.txt
% and version 1.3 or later is part of all distributions of LaTeX
% version 2005/12/01 or later.
%
% This work has the LPPL maintenance status `maintained'.
%
% The Current Maintainer of this work is Niklas Beisert.
%
% This work consists of the files childdoc.dtx and childdoc.ins
% and the derived files childdoc.def and cdocsamp.tex with
% cdocsch1.tex, cdocsch2.tex, cdocsdrf.tex, cdocsfn1.tex, cdocsfn2.tex.
%
%<package>\ifdefined\childdocmain\endinput\fi
%<package>\ProvidesFile{childdoc.def}[2018/12/30 v2.0 child document driver]
%<samplemain>\ProvidesFile{cdocsamp.tex}[2018/12/30 v2.0 sample for childdoc]
%<*driver>
%\ProvidesFile{childdoc.drv}[2018/12/30 v2.0 childdoc reference manual file]
\PassOptionsToClass{10pt,a4paper}{article}
\documentclass{ltxdoc}

\usepackage[margin=35mm]{geometry}
\usepackage{hyperref}
\usepackage{hyperxmp}
\usepackage[usenames]{color}

\hypersetup{colorlinks=true}
\hypersetup{pdfstartview=FitH}
\hypersetup{pdfpagemode=UseNone}
\hypersetup{pdfsource={}}
\hypersetup{pdflang={en-UK}}
\hypersetup{pdfcopyright={Copyright 2017-2018 Niklas Beisert.
  This work may be distributed and/or modified under the
  conditions of the LaTeX Project Public License, either version 1.3
  of this license or (at your option) any later version.}}
\hypersetup{pdflicenseurl={http://www.latex-project.org/lppl.txt}}
\hypersetup{pdfcontactaddress={ETH Zurich, ITP, HIT K,
  Wolfgang-Pauli-Strasse 27}}
\hypersetup{pdfcontactpostcode={8093}}
\hypersetup{pdfcontactcity={Zurich}}
\hypersetup{pdfcontactcountry={Switzerland}}
\hypersetup{pdfcontactemail={nbeisert@itp.phys.ethz.ch}}
\hypersetup{pdfcontacturl={http://people.phys.ethz.ch/\xmptilde nbeisert/}}

\newcommand{\secref}[1]{\hyperref[#1]{section \ref*{#1}}}

\parskip1ex
\parindent0pt
\let\olditemize\itemize
\def\itemize{\olditemize\parskip0pt}

\begin{document}

\title{The \textsf{childdoc} Package}
\hypersetup{pdftitle={The childdoc Package}}
\author{Niklas Beisert\\[2ex]
  Institut f\"ur Theoretische Physik\\
  Eidgen\"ossische Technische Hochschule Z\"urich\\
  Wolfgang-Pauli-Strasse 27, 8093 Z\"urich, Switzerland\\[1ex]
  \href{mailto:nbeisert@itp.phys.ethz.ch}
  {\texttt{nbeisert@itp.phys.ethz.ch}}}
\hypersetup{pdfauthor={Niklas Beisert}}
\hypersetup{pdfsubject={Manual for the LaTeX2e Package childdoc}}
\date{30 December 2018, \textsf{v2.0}}
\maketitle

\begin{abstract}\noindent
\textsf{childdoc} is a \LaTeXe{} package
that enables the direct compilation
of document sections included by |\include|
to individual files.
\end{abstract}

\begingroup
\parskip0ex
\tableofcontents
\endgroup

%%%%%%%%%%%%%%%%%%%%%%%%%%%%%%%%%%%%%%%%%%%%%%%%%%%%%%%%%%%%%%%%%%%%%%%%%%%%%%%%
%%%%%%%%%%%%%%%%%%%%%%%%%%%%%%%%%%%%%%%%%%%%%%%%%%%%%%%%%%%%%%%%%%%%%%%%%%%%%%%%
\section{Introduction}

\LaTeX{} provides a mechanism to structure a large document (such as a book)
into a main file and several child files (containing the chapters)
using the |\include| command.
This mechanism is beneficial for documents
which span hundreds of pages in order to
make the source file(s) more manageable.
Moreover, compilation can be restricted to
selected child files by means of the |\includeonly| command.
The latter feature can be used to reduce the compilation time while editing
(this was significantly more useful in the earlier days of \LaTeX{})
or to generate a smaller document which is easier to navigate.
Another application of |\includeonly| is to generate
documents consisting of selected parts of the complete document.

However, there are a few drawbacks of the plain |\include| mechanism:
\begin{itemize}
\item
The child files cannot be compiled on their own,
they can only be compiled via the main file.
A naive editing environment
(such as a text editor with an option
to have the current file processed by \LaTeX)
may require one to switch to the main file before compiling;
attempting to compile the child file produces errors.
\item
The main file must be modified (each time)
to adjust the |\includeonly| command
to the present needs. This easily leaves the main file in a messy state.
\item
The generated document will always carry the filename
of the main document. This is inconvenient if
several child files are to be compiled and
to be kept for distribution.
\end{itemize}

The present package provides a simple interface
to make child files individually compilable by \LaTeX{}.
Compiling a child file then has the same effect as compiling
the main file with an |\includeonly| command
to select the appropriate child.
Moreover the generated document will carry the name of the child
rather than the main file.
This resolves all three above issues.

This feature is meant to make the editing of books,
thesis documents and lecture notes somewhat more convenient.
However, the package can also be used efficiently for
composing a series of documents (such as exercise sheets)
which are typically distributed individually.
It then assists the author in generating the individual documents
(potentially in different versions)
as well as a document containing the collected series.
Another application is in developing style files
or other kinds of included material
where compilation of the style file could redirect
to a sample or test file.

%%%%%%%%%%%%%%%%%%%%%%%%%%%%%%%%%%%%%%%%%%%%%%%%%%%%%%%%%%%%%%%%%%%%%%%%%%%%%%%%
%%%%%%%%%%%%%%%%%%%%%%%%%%%%%%%%%%%%%%%%%%%%%%%%%%%%%%%%%%%%%%%%%%%%%%%%%%%%%%%%
\section{Usage}

First of all, the package \textsf{childdoc} is \emph{not} a standard
\LaTeXe{} |.sty| style file! Therefore it needs to be invoked in
a non-standard way.

%%%%%%%%%%%%%%%%%%%%%%%%%%%%%%%%%%%%%%%%%%%%%%%%%%%%%%%%%%%%%%%%%%%%%%%%%%%%%%%%
\subsection{Included Files}
\label{sec:include}

%%%%%%%%%%%%%%%%%%%%%%%%%%%%%%%%%%%%%%%%
\DescribeMacro{\childdocmain}
To use the package, add the commands
\begin{center}
\begin{tabular}{l}
|\input{childdoc.def}|\\
|\childdocmain{}|\\
\end{tabular}
\end{center}
at the very top of the main \LaTeX{} file,
in particular \emph{before} the |\documentclass| statement!
The argument of |\childdocmain| should be left empty
(but it must be present).

%%%%%%%%%%%%%%%%%%%%%%%%%%%%%%%%%%%%%%%%
\DescribeMacro{\childdocof}
Furthermore, add the commands
\begin{center}
\begin{tabular}{l}
|\input{childdoc.def}|\\
|\childdocof{|\textit{main}|}|\\
\end{tabular}
\end{center}
at the top of every child file \textit{child}
which is included by |\include{|\textit{child}|}|
from within the main file
(or at least for those files to be compiled individually).
The argument \textit{main} must be the filename of the main file.

There are a couple of
considerations in setting up the main and child documents:

%%%%%%%%%%%%%%%%%%%%%%%%%%%%%%%%%%%%%%%%
\paragraph{Restrictions.}

Please note the following restrictions:
\begin{itemize}
\item
|\childdocmain| must be called with one argument \textit{main}
to ensure compatibility with earlier version of the package.
It must either be empty (|\childdocmain{}|)
or precisely match the filename of the main file in which it is specified.
See \secref{sec:detection} for further information.
\item
The filename \textit{main} must be specified without the |.tex| extension.
\item
The filename \textit{main} is case sensitive
(even in case-insensitive file systems)
due to internal string comparison.
\item
The argument \textit{main} should be fully expanded, it cannot be a macro.
\item
Subdirectories and special characters should be avoided in filenames.
\item
The command |\childdocmain{|\textit{main}|}| must be followed by a whitespace.
It should not be followed immediately by another command
or by a comment mark `|%|'.
This is because the \TeX{} parser reads the token immediately following
the argument of |\childdocmain| and puts it
at the beginning of every child section;
however, a white\-space is ignored.
\end{itemize}

%%%%%%%%%%%%%%%%%%%%%%%%%%%%%%%%%%%%%%%%
\paragraph{Content of Main File.}

It is advisable to place all content in the child files included by |\include|.
Any output contained in the main file will appear in all child documents
unless suppressed manually;
it cannot be suppressed automatically by the |\includeonly| directive
and thus should normally be avoided.
A method to include some content in the main file
by means of conditional processing is described in \secref{sec:conditional}.

%%%%%%%%%%%%%%%%%%%%%%%%%%%%%%%%%%%%%%%%
\paragraph{Page Numbering.}

When only a part of the document is compiled,
the appropriate numbering of pages
(as well as other status parameters)
is determined from the |.aux| files.
The latter contain information from previous passes.
However this information needs to propagate through
all intermediate child documents.
Therefore the page numbering in child documents may well
be inconsistent until the complete document is compiled at least once.

A useful (if unconventional) way to always ensure a consistent
page numbering is to restart the numbering in each child document
and denote the pages by `\textit{child}|.|\textit{page}'
where \textit{child} represents the chapter/section number of the child file.
This can be achieved by the command
|\numberwithin{page}{|\textit{child}|}|
of the \textsf{amsmath} package
where \textit{child} can be |chapter| or |section|
depending on the chosen structuring.
Alternatively, one can modify the macro |\thepage| appropriately
and reset the counter |page| at the start of each child file.

%%%%%%%%%%%%%%%%%%%%%%%%%%%%%%%%%%%%%%%%%%%%%%%%%%%%%%%%%%%%%%%%%%%%%%%%%%%%%%%%
\subsection{Conditional Processing}
\label{sec:conditional}

The package provides a mechanism to compile different versions
of a document. To customise the versions further some conditional processing
can come in handy to distinguish which version is being compiled.
The package provides two macros to describe the compilation context:

%%%%%%%%%%%%%%%%%%%%%%%%%%%%%%%%%%%%%%%%
\DescribeMacro{\ifchilddoc}
The conditional |\ifchilddoc| distinguishes between the compilation of
child documents and the main document:
%
\begin{center}
|\ifchilddoc |\textit{child-code}| |[|\||else |\textit{main-code}]| \||fi|
\end{center}

%%%%%%%%%%%%%%%%%%%%%%%%%%%%%%%%%%%%%%%%
\DescribeMacro{\childdocname}
\DescribeMacro{\childdocjob}
The macro |\childdocname| contains the filename (without extension)
of the main or child file being processed.
Note that |\childdocjob| will always contain the name of the main file.

%%%%%%%%%%%%%%%%%%%%%%%%%%%%%%%%%%%%%%%%
\paragraph{Title Page.}

Conditional processing can be used to include a title or banner page
in the main document when proper precautions are taken.
Importantly, the code in the main file should ensure that the page counter
(as well as other status parameters which are stored in the |.aux| files)
takes the same value after the conditional processing.
Otherwise the page numbers may take divergent values
depending on which part is compiled.

For example, a title page could be declared by:
%
\begin{center}
\begin{tabular}{l}
|\ifchilddoc\||else|\\
|\addtocounter{page}{-1}|\\
\textit{code for title page}\\
|\newpage|\\
|\||fi|
\end{tabular}
\end{center}
%
A banner page for the child documents can be generated by:
%
\begin{center}
\begin{tabular}{l}
|\ifchilddoc|\\
|\addtocounter{page}{-1}|\\
\textit{code for banner page}\\
|\newpage|\\
|\||fi|
\end{tabular}
\end{center}
%
Here one could write a message such as:
\begin{center}
|This is the part \childdocname{} of \childdocjob{}.|
\end{center}

%%%%%%%%%%%%%%%%%%%%%%%%%%%%%%%%%%%%%%%%%%%%%%%%%%%%%%%%%%%%%%%%%%%%%%%%%%%%%%%%
\subsection{Flags}
\label{sec:flags}

The package makes it easy to generate different versions
of the main or child documents.
To this end compilation flags can be defined
and assigned different default values.
They will be particularly useful in conjunction
with the forwarding mechanism described in \secref{sec:forward}.

For example, it may be useful to have a flag |\version|
which can be set to |draft| or |final|.
The document source will contain some conditional code
depending on the value of |\version|.
Suppose further, the flag should default to |final| for the main file
and to |draft| for child files
which is a natural assignment for editing the document.
This is achieved by placing the following code
in the preamble of the main document
(below the |\childdocmain| directive):
%
\begin{center}
\begin{tabular}{l}
|\ifchilddoc|\\
|\providecommand{\version}{draft}|\\
|\||else|\\
|\providecommand{\version}{final}|\\
|\||fi|
\end{tabular}
\end{center}
%
The definition by |\providecommand| makes sure
that previous definitions are not overwritten.
Further statements |\providecommand{\version}{...}|
can thus be added before the above code to override it.

For the main file, one might add a line
(between |\childdocmain| and the above block)
%
\begin{center}
|%\ifchilddoc\||else\providecommand{\version}{draft}\||fi|
\end{center}
%
which can be uncommented to produce a draft version.
Likewise one can add a line to the very top of a child file
(above the |\childdocof{|\textit{main}|}| directive)
%
\begin{center}
|%\providecommand{\version}{final}|
\end{center}
%
which can be uncommented to produce the final version of this child document.

%%%%%%%%%%%%%%%%%%%%%%%%%%%%%%%%%%%%%%%%%%%%%%%%%%%%%%%%%%%%%%%%%%%%%%%%%%%%%%%%
\subsection{Forwarding}
\label{sec:forward}

Different versions of the main or child documents
using compilation flags as described in \secref{sec:flags}
can be (permanently) stored in different files
for convenient compilation, viewing and distribution.
To this end, the package defines a command
to pass on compilation to a different file:

%%%%%%%%%%%%%%%%%%%%%%%%%%%%%%%%%%%%%%%%
\DescribeMacro{\childdocforward}
The command |\childdocforward| redirects processing to
another source file:
%
\begin{center}
\begin{tabular}{l}
|\input{childdoc.def}|\\
|\childdocforward[|\textit{main}|]{|\textit{dest}|}|\\
\end{tabular}
\end{center}
%
The argument \textit{dest} is the destination file
(without extension).
It should be the main file or one of the child files.
Note that further \textsf{childdoc} directives
such as |\childdocof| and |\childdocforward|
in the indicated file will be processed in this form.
The optional argument \textit{main}
passes on directly to the main file \textit{main}
while pretending to compile the child \textit{dest}.
This form behaves as if \textit{dest}
issues |\childdocof{|\textit{main}|}| right away,
and no further \textsf{childdoc} directives will be processed.

%%%%%%%%%%%%%%%%%%%%%%%%%%%%%%%%%%%%%%%%
\DescribeMacro{\...prefix}
In the alternative form |\childdocforwardprefix|,
%
\begin{center}
\begin{tabular}{l}
|\input{childdoc.def}|\\
|\childdocforwardprefix[|\textit{main}|]{|\textit{prefix}|}{|\textit{dest}|}|
\end{tabular}
\end{center}
%
the destination file is determined by a pattern
depending on the current file:
To make this work, the current file must be called
`{\textit{prefix}\hspace{0.2em}\textit{suffix}}'
with \textit{prefix} matching precisely the argument.
Processing is then passed on to the file
`{\textit{dest}\hspace{0.2em}\textit{suffix}}'.
Surely, the same effect is achieved by
directly specifying the
argument `{\textit{dest}\hspace{0.2em}\textit{suffix}}'
in the first form.
However, that requires to set up a different file
for each child. With the alternative form of the command
all these files can have exactly the same content
which simplifies setting them up and maintaining them.

For example, the following file |draft.tex|
with a compilation flag |\version| as described in \secref{sec:flags}
compiles the main document as a draft:
%
\begin{center}
\begin{tabular}{l}
|\def\version{draft}|\\
|\input{childdoc.def}|\\
|\childdocforward{|\textit{main}|}|
\end{tabular}
\end{center}
%
Likewise, the following files |final|\textit{nn}|.tex|
compile the final version of the child document
|child|\textit{nn}|.tex|:
%
\begin{center}
\begin{tabular}{l}
|\def\version{final}|\\
|\input{childdoc.def}|\\
|\childdocforwardprefix{final}{child}|
\end{tabular}
\end{center}
%

Note that when several versions of a main file and/or of each child file
are to be generated, it may be convenient to set up a |Makefile| or
shell script to automatise the process.

%%%%%%%%%%%%%%%%%%%%%%%%%%%%%%%%%%%%%%%%%%%%%%%%%%%%%%%%%%%%%%%%%%%%%%%%%%%%%%%%
\subsection{Command Line Processing}
\label{sec:commandline}

The effect of redirection files can also be achieved by invoking
the \LaTeX{} compiler with a more elaborate command line.
Most conveniently this should be done as part
of a shell script or a |Makefile|.

When using \textsf{childdoc} in the main file, the following
command lines effectively perform a redirection
(note that depending on the shell being used,
backslashes may have to be doubled: `|\|' $\to$ `|\\|'):
%
\begin{center}
|... -jobname "|\textit{target}|" |\\|"|[\textit{flags}]%
|\input{childdoc.def}\childdocforward[|\textit{main}|]{|\textit{dest}|}"|
\end{center}
%
Here \textit{target} is the name of the output file,
\textit{main} is the name of the main file
and \textit{dest} is the name of the main or child file to be processed
(all filenames without extensions).
The optional argument \textit{main} can be omitted
if \textit{main} matches \textit{dest}.
Optionally, compilation \textit{flags} can be defined via |\def| commands.
This command line makes the \TeX{} engine believe
it is compiling the file \textit{target}
whose content is specified as the latter parameter.
The provided code then forwards the processing to
\textit{main} or \textit{dest} as described in \secref{sec:forward}.

%%%%%%%%%%%%%%%%%%%%%%%%%%%%%%%%%%%%%%%%%%%%%%%%%%%%%%%%%%%%%%%%%%%%%%%%%%%%%%%%
\subsection{Include by Input}
\label{sec:input}

Including child documents by |\include| has some restrictions by design.
Most notably, the content of a child document always occupies
its own set of pages; pages cannot be shared between child documents.
Usually, this behaviour makes perfect sense
because each child document contain an essential part of the document.
However, in some situations it may be desirable to compose
a document from a collection of parts
without having mandatory page breaks between then.
For this case, the package
provides a mechanism to include parts
by |\input| which can also be processed individually.
However, by construction this mechanism
requires manual handling of the content to be output.

%%%%%%%%%%%%%%%%%%%%%%%%%%%%%%%%%%%%%%%%
\DescribeMacro{\ifchilddocmanual}
The main file should be prepared as usual, see \secref{sec:include}.
However, the document body must make a distinction
between processing of an individual part and of the main document, e.g.:
%
\begin{center}
\begin{tabular}{l}
|\ifchilddocmanual|\\
|\input{\childdocname}|\\
|\||else|\\
\textit{document body with }|\input{|\textit{part}|}|\\
|\||fi|
\end{tabular}
\end{center}
%
The conditional |\ifchilddocmanual| is true whenever
a part to be included by |\input| is being compiled,
and the name of the part is stored in |\childdocname|.

%%%%%%%%%%%%%%%%%%%%%%%%%%%%%%%%%%%%%%%%
\DescribeMacro{\childdocby}
Each part to be included by |\input| should start with:
%
\begin{center}
\begin{tabular}{l}
|\input{childdoc.def}|\\
|\childdocby{|\textit{main}|}|\\
\end{tabular}
\end{center}
%
The directive |\childdocby| is similar to |\childdocof|
described in \secref{sec:include},
but the subsequent selection of content must be done manually.
To that end, both |\ifchilddoc| and |\ifchilddocmanual|
will be true upon processing of a part,
and the name of the part is stored in |\childdocname|.
Note that |\jobname| will be set to the filename of the current part
so that each part receives an individual |.aux| file
that does not interfere with the |.aux| file(s) of the main document.
This behaviour can be altered by the alternative form
|\childdocby[*]{|\textit{main}|}| (with a non-empty optional argument)
which uses the |.aux| file of the main document
by setting |\jobname| to \textit{main}.

%%%%%%%%%%%%%%%%%%%%%%%%%%%%%%%%%%%%%%%%%%%%%%%%%%%%%%%%%%%%%%%%%%%%%%%%%%%%%%%%
\subsection{Driver Development}
\label{sec:driver}

The \textsf{childdoc} mechanism can also be use for the development
of definition files such as \LaTeX{} styles or classes.
This case differs from the above setup with multiple parts
included by |\include| in that no |\includeonly| should be invoked.
This can be achieved by starting the include file
(before |\ProvidesPackage|) with:
%
\begin{center}
\begin{tabular}{l}
|\input{childdoc.def}|\\
|\childdocforward{|\textit{main}|}|\\
\end{tabular}
\end{center}
%
or alternatively with:
%
\begin{center}
\begin{tabular}{l}
|\input{childdoc.def}|\\
|\childdocby{|\textit{main}|}|\\
\end{tabular}
\end{center}
%
Both forms have slightly different effects as described above.
The main file is prepared as usual, see \secref{sec:include}.

%%%%%%%%%%%%%%%%%%%%%%%%%%%%%%%%%%%%%%%%%%%%%%%%%%%%%%%%%%%%%%%%%%%%%%%%%%%%%%%%
\subsection{Legacy Detection}
\label{sec:detection}

The directive |\childdocmain| in the main file can detect
whether the complete document or merely a child is to be compiled
even without using the directive |\childdocof|.
This method is deprecated because it is less robust
and there is no compelling reason to use it;
it is merely provided for backward compatibility
and it may be removed in future versions.

If the detection mechanism is to be used,
it is mandatory to correctly specify
the filename of the main file as the argument of |\childdocmain|:
%
\begin{center}
\begin{tabular}{l}
|\input{childdoc.def}|\\
|\childdocmain{|\textit{main}|}|\\
\end{tabular}
\end{center}
%
If |\jobname| does not match the argument \textit{main} of |\childdocmain|,
it is assumed that |\jobname| points to the child file to be compiled.
When using |\childdocmain| with the main file specified as argument,
it suffices to start a child file
with just |\input{|\textit{main}|}|
without loading of the package and using |\childdocof|.
If instead all processing is done
with the appropriate \textsf{childdoc} directives,
the argument of \textit{main} of |\childdocmain| can be empty.

An alternative version of the command line processing described
in \secref{sec:commandline} using the detection mechanism reads:
%
\begin{center}
|... -jobname "|\textit{target}|" "|[\textit{flags}]%
[|\def\jobname{|\textit{dest}|}|]|\input{|\textit{main}|}"|
\end{center}

%%%%%%%%%%%%%%%%%%%%%%%%%%%%%%%%%%%%%%%%%%%%%%%%%%%%%%%%%%%%%%%%%%%%%%%%%%%%%%%%
\subsection{Manual Code}
\label{sec:manual}

In case one cannot be certain whether the definitions file |childdoc.def|
is installed on the target \TeX{} distribution
and one prefers not to ship it,
it is conceivable to paste a few relevant commands into the sources.

To that end, drop all statements |\input{childdoc.def}|
and perform the replacements as outlined below.
Instead of |\childdocmain{|\textit{main}|}| add the following code
to the top of the main file:
%
\begin{center}
\begin{tabular}{l}
|\||ifdefined\childdocname\endinput\||fi\newif\ifchilddoc|\\
|\edef\childdocname{\scantokens\expandafter{\jobname\noexpand}}|\\
|\def\childdocmain{|\textit{main}|}\||ifx\childdocmain\childdocname\||else|\\
|\childdoctrue\includeonly{\childdocname}\let\jobname\childdocmain\||fi|\\
\end{tabular}
\end{center}
%
Instead of |\childdocof{|\textit{main}|}| just include the main file
at the top of each child file:
%
\begin{center}
|\input{|\textit{main}|}|
\end{center}
%
A simple redirection |\childdocforward{|\textit{dest}|}| is achieved by:
%
\begin{center}
|\def\jobname{|\textit{dest}|}\input{\jobname}|
\end{center}
%
The redirection with prefix
|\childdocforwardprefix[|\textit{prefix}|]{|\textit{dest}|}|
is accomplished by:
%
\begin{center}
\begin{tabular}{l}
|{\edef\jobname{\scantokens\expandafter{\jobname\noexpand}}|\\
|\def\redirectjob |\textit{prefix}|#1~~~{\gdef\jobname{|\textit{dest}|#1}}|\\
|\expandafter\redirectjob\jobname~~~}\input{\jobname}|
\end{tabular}
\end{center}

In an alternative approach,
child documents can be compiled by a specific command line
without additional code or specific definitions:
%
\begin{center}
|... -jobname "|\textit{target}|" "|[\textit{flags}]%
|\includeonly{|\textit{dest}|}\input{|\textit{main}|}"|
\end{center}
%

%%%%%%%%%%%%%%%%%%%%%%%%%%%%%%%%%%%%%%%%%%%%%%%%%%%%%%%%%%%%%%%%%%%%%%%%%%%%%%%%
%%%%%%%%%%%%%%%%%%%%%%%%%%%%%%%%%%%%%%%%%%%%%%%%%%%%%%%%%%%%%%%%%%%%%%%%%%%%%%%%
\section{Information}

%%%%%%%%%%%%%%%%%%%%%%%%%%%%%%%%%%%%%%%%%%%%%%%%%%%%%%%%%%%%%%%%%%%%%%%%%%%%%%%%
\subsection{Copyright}

Copyright \copyright{} 2017--2018 Niklas Beisert

This work may be distributed and/or modified under the
conditions of the \LaTeX{} Project Public License, either version 1.3
of this license or (at your option) any later version.
The latest version of this license is in
  \url{http://www.latex-project.org/lppl.txt}
and version 1.3 or later is part of all distributions of \LaTeX{}
version 2005/12/01 or later.

This work has the LPPL maintenance status `maintained'.

The Current Maintainer of this work is Niklas Beisert.

This work consists of the files |README.txt|, |childdoc.ins| and |childdoc.dtx|
as well as the derived files |childdoc.def|, |cdocsamp.tex|
with |cdocsch1.tex|, |cdocsch2.tex|, |cdocspt3.tex|, |cdocspt4.tex|,
|cdocsdrf.tex|, |cdocsfn1.tex|, |cdocsfn2.tex|
as well as |childdoc.pdf|.

%%%%%%%%%%%%%%%%%%%%%%%%%%%%%%%%%%%%%%%%%%%%%%%%%%%%%%%%%%%%%%%%%%%%%%%%%%%%%%%%
\subsection{Files and Installation}

The package consists of the files:
%
\begin{center}
\begin{tabular}{ll}
    |README.txt|   & readme file \\
    |childdoc.ins| & installation file \\
    |childdoc.dtx| & source file \\
    |childdoc.def| & definition file \\
    |cdocsamp.tex| & sample main file \\
    |cdocsch1.tex| & sample include file \\
    |cdocsch2.tex| & sample include file \\
    |cdocspt3.tex| & sample part file \\
    |cdocspt4.tex| & sample part file \\
    |cdocsdrf.tex| & sample redirection file \\
    |cdocsfn1.tex| & sample redirection file \\
    |cdocsfn2.tex| & sample redirection file \\
    |childdoc.pdf| & manual
\end{tabular}
\end{center}
%
The distribution consists of the files
|README.txt|, |childdoc.ins| and |childdoc.dtx|.
%
\begin{itemize}
\item
Run (pdf)\LaTeX{} on |childdoc.dtx|
to compile the manual |childdoc.pdf| (this file).
\item
Run \LaTeX{} on |childdoc.ins| to create the definitions file |childdoc.def|
and the sample |cdocsamp.tex| with include files
|cdocsch1.tex|, |cdocsch2.tex|, |cdocspt3.tex|, |cdocspt4.tex|,
|cdocsdrf.tex|, |cdocsfn1.tex|, |cdocsfn2.tex|.
Then copy the file |childdoc.def| to an appropriate directory of your \LaTeX{}
distribution, e.g.\ \textit{texmf-root}|/tex/latex/childdoc|.
\end{itemize}

%%%%%%%%%%%%%%%%%%%%%%%%%%%%%%%%%%%%%%%%%%%%%%%%%%%%%%%%%%%%%%%%%%%%%%%%%%%%%%%%
\subsection{Related CTAN Packages}

There are several other packages which offer a similar functionality:
%
\begin{itemize}
\item
The packages
\href{http://ctan.org/pkg/docmute}{\textsf{docmute}},
\href{http://ctan.org/pkg/includex}{\textsf{includex}} and
\href{http://ctan.org/pkg/standalone}{\textsf{standalone}}
provide commands to include only the document body of
a child file thus allowing both files to be compiled individually.
\item
The packages \href{http://ctan.org/pkg/subdocs}{\textsf{subdocs}}
and \href{http://ctan.org/pkg/subfiles}{\textsf{subfiles}}
provide structures in which the main and child documents can be
encapsulated and allowing them to be compiled individually.
The inclusion mechanism is different from the conventional |\include|.
\item
The package \href{http://ctan.org/pkg/combine}{\textsf{combine}}
is an elaborate solution to combine several documents into one.
\end{itemize}
%
See also the CTAN topic \href{http://ctan.org/topic/subdocs}{\textsf{subdocs}}
for further related packages.
The present package differs from the above solutions in that
a document structure constructed with the conventional |\include| mechanism
just needs two extra commands at the top of every file
such that all constituent files can be compiled individually.

%%%%%%%%%%%%%%%%%%%%%%%%%%%%%%%%%%%%%%%%%%%%%%%%%%%%%%%%%%%%%%%%%%%%%%%%%%%%%%%%
%\subsection{Feature Suggestions}
%
%The following is a list of features which may be useful for future
%versions of this package:
%%
%\begin{itemize}
%\item
%\ldots
%\end{itemize}

%%%%%%%%%%%%%%%%%%%%%%%%%%%%%%%%%%%%%%%%%%%%%%%%%%%%%%%%%%%%%%%%%%%%%%%%%%%%%%%%
\subsection{Revision History}

%%%%%%%%%%%%%%%%%%%%%%%%%%%%%%%%%%%%%%%%
\paragraph{v2.0:} 2018/12/30

\begin{itemize}
\item
immediate forward processing
\item
added |\childdocby| mechanism
\item
manual restructured
\end{itemize}

%%%%%%%%%%%%%%%%%%%%%%%%%%%%%%%%%%%%%%%%
\paragraph{v1.6:} 2018/01/17

\begin{itemize}
\item
application for development of include files
\item
corrections to manual
\end{itemize}

%%%%%%%%%%%%%%%%%%%%%%%%%%%%%%%%%%%%%%%%
\paragraph{v1.5:} 2017/05/21

\begin{itemize}
\item
more complete structuring introduced
\item
|\childdocof| introduced
\item
|\childdoc| renamed to |\childdocmain|
\item
|\childredirect| renamed to |\childdocforward| and |\childdocforwardprefix|
and functionality expanded
\end{itemize}

%%%%%%%%%%%%%%%%%%%%%%%%%%%%%%%%%%%%%%%%
\paragraph{v1.0:} 2017/04/27

\begin{itemize}
\item
manual and install package
\item
first version published on CTAN
\end{itemize}

%%%%%%%%%%%%%%%%%%%%%%%%%%%%%%%%%%%%%%%%
\paragraph{v0.6:} 2017/04/26

\begin{itemize}
\item
redirection mechanism added
\end{itemize}

%%%%%%%%%%%%%%%%%%%%%%%%%%%%%%%%%%%%%%%%
\paragraph{v0.5:} 2017/04/26

\begin{itemize}
\item
functionality in definition file
\end{itemize}


%%%%%%%%%%%%%%%%%%%%%%%%%%%%%%%%%%%%%%%%%%%%%%%%%%%%%%%%%%%%%%%%%%%%%%%%%%%%%%%%
%%%%%%%%%%%%%%%%%%%%%%%%%%%%%%%%%%%%%%%%%%%%%%%%%%%%%%%%%%%%%%%%%%%%%%%%%%%%%%%%
%%%%%%%%%%%%%%%%%%%%%%%%%%%%%%%%%%%%%%%%%%%%%%%%%%%%%%%%%%%%%%%%%%%%%%%%%%%%%%%%
\appendix

\settowidth\MacroIndent{\rmfamily\scriptsize 000\ }

 \DocInput{childdoc.dtx}

\end{document}
%</driver>
% \fi
%
% %%%%%%%%%%%%%%%%%%%%%%%%%%%%%%%%%%%%%%%%%%%%%%%%%%%%%%%%%%%%%%%%%%%%%%%%%%%%%%
% %%%%%%%%%%%%%%%%%%%%%%%%%%%%%%%%%%%%%%%%%%%%%%%%%%%%%%%%%%%%%%%%%%%%%%%%%%%%%%
% \section{Sample}
%\iffalse
%<*samplemain>
%\fi
%
% The following presents a sample document
% with two chapters, two parts, a title page,
% a compile flag as well as three forwarding files to set the flag.
% It consists of eight |.tex| files:
% \begin{center}
% \begin{tabular}{ll}
% |cdocsamp.tex|&main file\\
% |cdocsch1.tex|&include file for chapter 1\\
% |cdocsch2.tex|&include file for chapter 2\\
% |cdocspt3.tex|&include file for part 3\\
% |cdocspt4.tex|&include file for part 4\\
% |cdocsdrf.tex|&forwarding file for main file in draft mode\\
% |cdocsfi1.tex|&forwarding file for final version of chapter 1\\
% |cdocsfi2.tex|&forwarding file for final version of chapter 2\\
% \end{tabular}
% \end{center}
% Each of the eight files can be compiled directly by the \LaTeX{} compiler.
%
% %%%%%%%%%%%%%%%%%%%%%%%%%%%%%%%%%%%%%%
% \paragraph{Main File.}
%
% The main file is called |cdocsamp.tex|.
%
% Load the \textsf{childdoc} definitions and
% declare the filename for the main document:
%    \begin{macrocode}
\input{childdoc.def}
\childdocmain{}
%    \end{macrocode}

% Optional override for |\version| flag:
%    \begin{macrocode}
%%\ifchilddoc\else\providecommand{\version}{draft}\fi
%    \end{macrocode}

% Define the default values for the |\version| flag
% (|final| for the main file and |draft| for childs):
%    \begin{macrocode}
\ifchilddoc
\providecommand{\version}{draft}
\else
\providecommand{\version}{final}
\fi
%    \end{macrocode}

% Load the standard document class:
%    \begin{macrocode}
\documentclass[12pt]{article}
%    \end{macrocode}

% Start the document body:
%    \begin{macrocode}
\begin{document}
%    \end{macrocode}

% Declare a title page.
% Print title, part of document being processed and version flag:
%    \begin{macrocode}
\addtocounter{page}{-1}
\begin{center}
{\LARGE\bfseries{}childdoc example\par}
\vspace{1cm}
\ifchilddoc
\ifchilddocmanual part\else chapter\fi:
`\childdocname' of `\childdocjob'\par
\else
main document: `\childdocjob'\par
\fi
version: \version\par
\end{center}
\newpage
%    \end{macrocode}

% Manually include selected file,
% otherwise process as usual:
%    \begin{macrocode}
\ifchilddocmanual
\section*{part `\childdocname'}
\input{\childdocname}
\else
%    \end{macrocode}

% Include the two chapters:
%    \begin{macrocode}
\include{cdocsch1}
\include{cdocsch2}
%    \end{macrocode}

% Include the two parts unless only chapters should be displayed:
%    \begin{macrocode}
\ifchilddoc\else
\section{part three}
\input{cdocspt3}
\section{part four}
\input{cdocspt4}
\fi
%    \end{macrocode}

% Process as usual until here:
%    \begin{macrocode}
\fi
%    \end{macrocode}

% End of document body:
%    \begin{macrocode}
\end{document}
%    \end{macrocode}
%\iffalse
%</samplemain>
%\fi
%
% %%%%%%%%%%%%%%%%%%%%%%%%%%%%%%%%%%%%%%
% \paragraph{Chapter Include Files.}
%
% The include files are called |cdocsch1.tex| and |cdocsch2.tex|.
%
%\iffalse
%<*samplechap1|samplechap2>
%\fi

% Optional override for |\version| flag:
%    \begin{macrocode}
%%\providecommand{\version}{final}
%    \end{macrocode}

% Include the main document:
%    \begin{macrocode}
\input{childdoc.def}
\childdocof{cdocsamp}
%    \end{macrocode}

%\iffalse
%</samplechap1|samplechap2>
%\fi
%
%\iffalse
%<*samplechap1>
%\fi
% Some text for chapter 1:
%    \begin{macrocode}
\section{one}
some text in chapter one
%    \end{macrocode}

%\iffalse
%</samplechap1>
%\fi
% Some text for chapter 2:
%\iffalse
%<*samplechap2>
%\fi
%    \begin{macrocode}
\section{two}
more text in chapter two
%    \end{macrocode}

%\iffalse
%</samplechap2>
%\fi
%
% %%%%%%%%%%%%%%%%%%%%%%%%%%%%%%%%%%%%%%
% \paragraph{Part Include Files.}
%
% The include files are called |cdocspt3.tex| and |cdocspt4.tex|.
%
%\iffalse
%<*samplepart3|samplepart4>
%\fi

% Optional override for |\version| flag:
%    \begin{macrocode}
%%\providecommand{\version}{final}
%    \end{macrocode}

% Include the main document:
%    \begin{macrocode}
\input{childdoc.def}
\childdocby{cdocsamp}
%    \end{macrocode}

%\iffalse
%</samplepart3|samplepart4>
%\fi
%
%\iffalse
%<*samplepart3>
%\fi
% Some text for part 3:
%    \begin{macrocode}
some text in part three
%    \end{macrocode}

%\iffalse
%</samplepart3>
%\fi
% Some text for part 4:
%\iffalse
%<*samplepart4>
%\fi
%    \begin{macrocode}
more text in part four
%    \end{macrocode}

%\iffalse
%</samplepart4>
%\fi
%
% %%%%%%%%%%%%%%%%%%%%%%%%%%%%%%%%%%%%%%
% \paragraph{Forwarding for a Complete Draft.}
%
% The following forwarding file |cdocsdrf.tex|
% compiles the main document in draft mode:
%\iffalse
%<*sampledraft>
%\fi
%    \begin{macrocode}
\def\version{draft}
\input{childdoc.def}
\childdocforward{cdocsamp}
%    \end{macrocode}

%\iffalse
%</sampledraft>
%\fi
%
% %%%%%%%%%%%%%%%%%%%%%%%%%%%%%%%%%%%%%%
% \paragraph{Forwarding for Final Version of the Chapters.}
%
% The following forwarding files |cdocsfn1.tex| and |cdocsfn2.tex|
% (with identical content)
% compile the final versions of the child documents
% |cdocsch1.tex| and |cdocsch2.tex|, respectively:
%\iffalse
%<*samplefinal>
%\fi
%    \begin{macrocode}
\def\version{final}
\input{childdoc.def}
\childdocforwardprefix[cdocsamp]{cdocsfn}{cdocsch}
%    \end{macrocode}

%\iffalse
%</samplefinal>
%\fi
%
% %%%%%%%%%%%%%%%%%%%%%%%%%%%%%%%%%%%%%%
% \paragraph{Command Line Processing.}
%
% The following three command lines generate the output files
% |cdocscld|, |cdocscl1| and |cdocscl2|
% which should be identical to
% |cdocsdrf|, |cdocsch1| and |cdocsfn2|, respectively:
% \begin{center}
% \begin{tabular}{l}
% |latex -jobname cdocscld \|\\
% |  "\def\version{draft}\input{childdoc.def}\childdocforward{cdocsamp}"|\\
% |latex -jobname cdocscl1 \|\\
% |  "\input{childdoc.def}\childdocforward[cdocsamp]{cdocsch1}"|\\
% |latex -jobname cdocscl2 \|\\
% |  "\def\version{final}\input{childdoc.def}\childdocforward{cdocsch2}"|
% \end{tabular}
% \end{center}
% Note that the trailing backslash on each first line
% merely continues the input to the second line
% (for convenient cut ant paste).
% Furthermore, the command |latex| can be replaced by any
% of its alternative versions such as |pdflatex|.
%
% %%%%%%%%%%%%%%%%%%%%%%%%%%%%%%%%%%%%%%%%%%%%%%%%%%%%%%%%%%%%%%%%%%%%%%%%%%%%%%
% %%%%%%%%%%%%%%%%%%%%%%%%%%%%%%%%%%%%%%%%%%%%%%%%%%%%%%%%%%%%%%%%%%%%%%%%%%%%%%
% \section{Implementation}
%\iffalse
%<*package>
%\fi
%
% This section describes the definitions file |childdoc.def|.

% The definitions cannot be loaded using |\usepackage| or |\RequirePackage|
% which has a mechanism to prevent loading a style file more than once.
% When loading the definitions by means of |\input|
% multiple instances have to be prevented manually:
%\iffalse
%This code needs to be before the `\ProvidesFile' directive
%which is defined at the beginning of this file.
%Therefore it is also placed there and commented out here.
%</package>
%<*discard>
%\fi
%    \begin{macrocode}
\ifdefined\childdocmain\endinput\fi
%    \end{macrocode}
%\iffalse
%</discard>
%<*package>
%\fi
%
% \macro{\ifchilddoc}
% \macro{\ifchilddocmanual}
% The conditional |\ifchilddoc| tells whether a
% child (true) or main (false) document is being compiled.
% The conditional |\ifchilddocmanual| tells whether
% the |\includeonly| mechanism is used (false) or
% the selection of child files must be performed manually (true).
% The definitions initialise to false:
%    \begin{macrocode}
\newif\ifchilddoc
\newif\ifchilddocmanual
%    \end{macrocode}

% \macro{\childdocname}
% \macro{\childdocjob}
% The macro |\childdocname| stores the name of the main document
% to be compiled. The macro |\childdocjob| stores the name of
% the document on which the \LaTeX{} compiler was originally invoked.
% The content of |\jobname| cannot be compared
% to filenames specified in the source due to different catcodes.
% The following code rescans |\jobname|, stores the result
% in |\childdocname| and saves a copy in |\childdocjob|:
%    \begin{macrocode}
\edef\childdocname{\scantokens\expandafter{\jobname\noexpand}}
\let\childdocjob\childdocname
%    \end{macrocode}

% \macro{\childdocdisable}
% The macro |\childdocdisable| prevents the main file
% from being processed more than once.
% At this stage, the main document command |\childdocmain|
% is assumed to be called once again where it should do nothing.
% Any subsequent call to it should prevent
% a secondary processing of the main document
% It overwrites the forwarding commands
% |\childdocof| and |\childdocforward|
% with empty macros to prevent further inclusions of the main document:
%    \begin{macrocode}
\newcommand{\childdocdisable}
{
  \renewcommand{\childdocmain}[1]{\renewcommand{\childdocmain}[1]{\endinput}}
  \renewcommand{\childdocof}[1]{}
  \renewcommand{\childdocby}[2][]{}
  \renewcommand{\childdocforward}[2][]{}
  \renewcommand{\childdocdisable}{}
}
%    \end{macrocode}

% \macro{\childdocmain}
% The macro |\childdocmain| is to be called at the top of the main file
% with nothing or the main filename (without extension) as argument.
% First, it breaks loops.
% If the argument is not empty and does not match |\childdocname|
% (which is set by the first inclusion of |childdoc.def|),
% |\ifchilddoc| is set to true, |\includeonly| is applied to the child file
% and |\jobname| is set to the main file
% (for proper handling of |.aux| files):
%    \begin{macrocode}
\newcommand{\childdocmain}[1]
{
  \childdocdisable\childdocmain{}
  \if?#1?\else
    \begingroup
      \def\childdoctmp{#1}
      \ifx\childdoctmp\childdocname
        \def\childdoctmp{}
      \else
        \def\childdoctmp
        {
          \childdoctrue
          \includeonly{\childdocname}
          \def\childdocjob{#1}
          \def\jobname{#1}
        }
      \fi
      \expandafter
    \endgroup
    \childdoctmp
  \fi
}
%    \end{macrocode}

% \macro{\childdocof}
% The command |\childdocof| redirects
% compilation to the main file |#1|.
%    \begin{macrocode}
\newcommand{\childdocof}[1]
{
  \childdocdisable
  \childdoctrue
  \includeonly{\childdocname}
  \def\jobname{#1}
  \def\childdocjob{#1}
  \input{#1}
}
%    \end{macrocode}

% \macro{\childdocby}
% The command |\childdocby| ....
%    \begin{macrocode}
\newcommand{\childdocby}[2][]
{
  \childdocdisable
  \childdoctrue
  \childdocmanualtrue
  \if?#1?\else
    \def\jobname{#2}
  \fi
  \def\childdocjob{#2}
  \input{#2}
  \endinput
}
%    \end{macrocode}

% \macro{\childdocforward}
% The command |\childdocforward| redirects
% compilation to the main file or
% (if the optional argument is given) a child file.
% Parameters are set as if the main file
% or a child file starting with |\childdocof| was compiled.
% Then compilation is handed over to the main file:
%    \begin{macrocode}
\newcommand{\childdocforward}[2][]
{
  \begingroup
    \if?#1?
      \def\childdoctmp
      {
        \def\childdocname{#2}
        \def\childdocjob{#2}
        \def\jobname{#2}
        \input{#2}
        \endinput
      }
    \else
      \def\childdoctmp
      {
        \childdocdisable
        \def\childdocname{#2}
        \childdoctrue
        \includeonly{#2}
        \def\childdocjob{#1}
        \def\jobname{#1}
        \input{#1}
        \endinput
      }
    \fi
    \expandafter
  \endgroup
  \childdoctmp
}
%    \end{macrocode}

% \macro{\childdocforwardprefix}
% The command |\childdocforwardprefix| redirects
% compilation to the main or a child file by means of a pattern.
% The prefix |#1| in the current filename is replaced by |#2|
% and the suffix of the current filename is kept
% (it is assumed that the filename does not contain the substring `|~~~|'
% which is used as a delimiter).
% Compilation is handed over to the new file by |\childdocforward|:
%    \begin{macrocode}
\newcommand{\childdocforwardprefix}[3][]
{
  \begingroup
    \def\childdocextract #2##1~~~{\def\childdoctmp{\childdocforward[#1]{#3##1}}}
    \expandafter\childdocextract\childdocname~~~
    \expandafter
  \endgroup
  \childdoctmp
}
%    \end{macrocode}

% \macro{\childdoc}
% The deprecated macro |\childdoc| is a legacy version of |\childdocmain|:
%    \begin{macrocode}
\newcommand{\childdoc}{\childdocmain}
%    \end{macrocode}

% \macro{\childdocredirect}
% The deprecated macro |\childdocredirect| is a legacy version
% of |\childdocforward| and |\childdocforwardprefix|:
%    \begin{macrocode}
\newcommand{\childdocredirect}[2][]
{
  \begingroup
    \if?#1?
      \def\childdoctmp{\childdocforward{#2}}
    \else
      \def\childdoctmp{\childdocforwardprefix{#1}{#2}}
    \fi
    \expandafter
  \endgroup
  \childdoctmp
}
%    \end{macrocode}

%\iffalse
%</package>
%\fi
%
\endinput

\childdocof{cdocsamp}
%    \end{macrocode}

%\iffalse
%</samplechap1|samplechap2>
%\fi
%
%\iffalse
%<*samplechap1>
%\fi
% Some text for chapter 1:
%    \begin{macrocode}
\section{one}
some text in chapter one
%    \end{macrocode}

%\iffalse
%</samplechap1>
%\fi
% Some text for chapter 2:
%\iffalse
%<*samplechap2>
%\fi
%    \begin{macrocode}
\section{two}
more text in chapter two
%    \end{macrocode}

%\iffalse
%</samplechap2>
%\fi
%
% %%%%%%%%%%%%%%%%%%%%%%%%%%%%%%%%%%%%%%
% \paragraph{Part Include Files.}
%
% The include files are called |cdocspt3.tex| and |cdocspt4.tex|.
%
%\iffalse
%<*samplepart3|samplepart4>
%\fi

% Optional override for |\version| flag:
%    \begin{macrocode}
%%\providecommand{\version}{final}
%    \end{macrocode}

% Include the main document:
%    \begin{macrocode}
% \iffalse
%
% childdoc.dtx Copyright (C) 2017-2018 Niklas Beisert
%
% This work may be distributed and/or modified under the
% conditions of the LaTeX Project Public License, either version 1.3
% of this license or (at your option) any later version.
% The latest version of this license is in
%   http://www.latex-project.org/lppl.txt
% and version 1.3 or later is part of all distributions of LaTeX
% version 2005/12/01 or later.
%
% This work has the LPPL maintenance status `maintained'.
%
% The Current Maintainer of this work is Niklas Beisert.
%
% This work consists of the files childdoc.dtx and childdoc.ins
% and the derived files childdoc.def and cdocsamp.tex with
% cdocsch1.tex, cdocsch2.tex, cdocsdrf.tex, cdocsfn1.tex, cdocsfn2.tex.
%
%<package>\ifdefined\childdocmain\endinput\fi
%<package>\ProvidesFile{childdoc.def}[2018/12/30 v2.0 child document driver]
%<samplemain>\ProvidesFile{cdocsamp.tex}[2018/12/30 v2.0 sample for childdoc]
%<*driver>
%\ProvidesFile{childdoc.drv}[2018/12/30 v2.0 childdoc reference manual file]
\PassOptionsToClass{10pt,a4paper}{article}
\documentclass{ltxdoc}

\usepackage[margin=35mm]{geometry}
\usepackage{hyperref}
\usepackage{hyperxmp}
\usepackage[usenames]{color}

\hypersetup{colorlinks=true}
\hypersetup{pdfstartview=FitH}
\hypersetup{pdfpagemode=UseNone}
\hypersetup{pdfsource={}}
\hypersetup{pdflang={en-UK}}
\hypersetup{pdfcopyright={Copyright 2017-2018 Niklas Beisert.
  This work may be distributed and/or modified under the
  conditions of the LaTeX Project Public License, either version 1.3
  of this license or (at your option) any later version.}}
\hypersetup{pdflicenseurl={http://www.latex-project.org/lppl.txt}}
\hypersetup{pdfcontactaddress={ETH Zurich, ITP, HIT K,
  Wolfgang-Pauli-Strasse 27}}
\hypersetup{pdfcontactpostcode={8093}}
\hypersetup{pdfcontactcity={Zurich}}
\hypersetup{pdfcontactcountry={Switzerland}}
\hypersetup{pdfcontactemail={nbeisert@itp.phys.ethz.ch}}
\hypersetup{pdfcontacturl={http://people.phys.ethz.ch/\xmptilde nbeisert/}}

\newcommand{\secref}[1]{\hyperref[#1]{section \ref*{#1}}}

\parskip1ex
\parindent0pt
\let\olditemize\itemize
\def\itemize{\olditemize\parskip0pt}

\begin{document}

\title{The \textsf{childdoc} Package}
\hypersetup{pdftitle={The childdoc Package}}
\author{Niklas Beisert\\[2ex]
  Institut f\"ur Theoretische Physik\\
  Eidgen\"ossische Technische Hochschule Z\"urich\\
  Wolfgang-Pauli-Strasse 27, 8093 Z\"urich, Switzerland\\[1ex]
  \href{mailto:nbeisert@itp.phys.ethz.ch}
  {\texttt{nbeisert@itp.phys.ethz.ch}}}
\hypersetup{pdfauthor={Niklas Beisert}}
\hypersetup{pdfsubject={Manual for the LaTeX2e Package childdoc}}
\date{30 December 2018, \textsf{v2.0}}
\maketitle

\begin{abstract}\noindent
\textsf{childdoc} is a \LaTeXe{} package
that enables the direct compilation
of document sections included by |\include|
to individual files.
\end{abstract}

\begingroup
\parskip0ex
\tableofcontents
\endgroup

%%%%%%%%%%%%%%%%%%%%%%%%%%%%%%%%%%%%%%%%%%%%%%%%%%%%%%%%%%%%%%%%%%%%%%%%%%%%%%%%
%%%%%%%%%%%%%%%%%%%%%%%%%%%%%%%%%%%%%%%%%%%%%%%%%%%%%%%%%%%%%%%%%%%%%%%%%%%%%%%%
\section{Introduction}

\LaTeX{} provides a mechanism to structure a large document (such as a book)
into a main file and several child files (containing the chapters)
using the |\include| command.
This mechanism is beneficial for documents
which span hundreds of pages in order to
make the source file(s) more manageable.
Moreover, compilation can be restricted to
selected child files by means of the |\includeonly| command.
The latter feature can be used to reduce the compilation time while editing
(this was significantly more useful in the earlier days of \LaTeX{})
or to generate a smaller document which is easier to navigate.
Another application of |\includeonly| is to generate
documents consisting of selected parts of the complete document.

However, there are a few drawbacks of the plain |\include| mechanism:
\begin{itemize}
\item
The child files cannot be compiled on their own,
they can only be compiled via the main file.
A naive editing environment
(such as a text editor with an option
to have the current file processed by \LaTeX)
may require one to switch to the main file before compiling;
attempting to compile the child file produces errors.
\item
The main file must be modified (each time)
to adjust the |\includeonly| command
to the present needs. This easily leaves the main file in a messy state.
\item
The generated document will always carry the filename
of the main document. This is inconvenient if
several child files are to be compiled and
to be kept for distribution.
\end{itemize}

The present package provides a simple interface
to make child files individually compilable by \LaTeX{}.
Compiling a child file then has the same effect as compiling
the main file with an |\includeonly| command
to select the appropriate child.
Moreover the generated document will carry the name of the child
rather than the main file.
This resolves all three above issues.

This feature is meant to make the editing of books,
thesis documents and lecture notes somewhat more convenient.
However, the package can also be used efficiently for
composing a series of documents (such as exercise sheets)
which are typically distributed individually.
It then assists the author in generating the individual documents
(potentially in different versions)
as well as a document containing the collected series.
Another application is in developing style files
or other kinds of included material
where compilation of the style file could redirect
to a sample or test file.

%%%%%%%%%%%%%%%%%%%%%%%%%%%%%%%%%%%%%%%%%%%%%%%%%%%%%%%%%%%%%%%%%%%%%%%%%%%%%%%%
%%%%%%%%%%%%%%%%%%%%%%%%%%%%%%%%%%%%%%%%%%%%%%%%%%%%%%%%%%%%%%%%%%%%%%%%%%%%%%%%
\section{Usage}

First of all, the package \textsf{childdoc} is \emph{not} a standard
\LaTeXe{} |.sty| style file! Therefore it needs to be invoked in
a non-standard way.

%%%%%%%%%%%%%%%%%%%%%%%%%%%%%%%%%%%%%%%%%%%%%%%%%%%%%%%%%%%%%%%%%%%%%%%%%%%%%%%%
\subsection{Included Files}
\label{sec:include}

%%%%%%%%%%%%%%%%%%%%%%%%%%%%%%%%%%%%%%%%
\DescribeMacro{\childdocmain}
To use the package, add the commands
\begin{center}
\begin{tabular}{l}
|\input{childdoc.def}|\\
|\childdocmain{}|\\
\end{tabular}
\end{center}
at the very top of the main \LaTeX{} file,
in particular \emph{before} the |\documentclass| statement!
The argument of |\childdocmain| should be left empty
(but it must be present).

%%%%%%%%%%%%%%%%%%%%%%%%%%%%%%%%%%%%%%%%
\DescribeMacro{\childdocof}
Furthermore, add the commands
\begin{center}
\begin{tabular}{l}
|\input{childdoc.def}|\\
|\childdocof{|\textit{main}|}|\\
\end{tabular}
\end{center}
at the top of every child file \textit{child}
which is included by |\include{|\textit{child}|}|
from within the main file
(or at least for those files to be compiled individually).
The argument \textit{main} must be the filename of the main file.

There are a couple of
considerations in setting up the main and child documents:

%%%%%%%%%%%%%%%%%%%%%%%%%%%%%%%%%%%%%%%%
\paragraph{Restrictions.}

Please note the following restrictions:
\begin{itemize}
\item
|\childdocmain| must be called with one argument \textit{main}
to ensure compatibility with earlier version of the package.
It must either be empty (|\childdocmain{}|)
or precisely match the filename of the main file in which it is specified.
See \secref{sec:detection} for further information.
\item
The filename \textit{main} must be specified without the |.tex| extension.
\item
The filename \textit{main} is case sensitive
(even in case-insensitive file systems)
due to internal string comparison.
\item
The argument \textit{main} should be fully expanded, it cannot be a macro.
\item
Subdirectories and special characters should be avoided in filenames.
\item
The command |\childdocmain{|\textit{main}|}| must be followed by a whitespace.
It should not be followed immediately by another command
or by a comment mark `|%|'.
This is because the \TeX{} parser reads the token immediately following
the argument of |\childdocmain| and puts it
at the beginning of every child section;
however, a white\-space is ignored.
\end{itemize}

%%%%%%%%%%%%%%%%%%%%%%%%%%%%%%%%%%%%%%%%
\paragraph{Content of Main File.}

It is advisable to place all content in the child files included by |\include|.
Any output contained in the main file will appear in all child documents
unless suppressed manually;
it cannot be suppressed automatically by the |\includeonly| directive
and thus should normally be avoided.
A method to include some content in the main file
by means of conditional processing is described in \secref{sec:conditional}.

%%%%%%%%%%%%%%%%%%%%%%%%%%%%%%%%%%%%%%%%
\paragraph{Page Numbering.}

When only a part of the document is compiled,
the appropriate numbering of pages
(as well as other status parameters)
is determined from the |.aux| files.
The latter contain information from previous passes.
However this information needs to propagate through
all intermediate child documents.
Therefore the page numbering in child documents may well
be inconsistent until the complete document is compiled at least once.

A useful (if unconventional) way to always ensure a consistent
page numbering is to restart the numbering in each child document
and denote the pages by `\textit{child}|.|\textit{page}'
where \textit{child} represents the chapter/section number of the child file.
This can be achieved by the command
|\numberwithin{page}{|\textit{child}|}|
of the \textsf{amsmath} package
where \textit{child} can be |chapter| or |section|
depending on the chosen structuring.
Alternatively, one can modify the macro |\thepage| appropriately
and reset the counter |page| at the start of each child file.

%%%%%%%%%%%%%%%%%%%%%%%%%%%%%%%%%%%%%%%%%%%%%%%%%%%%%%%%%%%%%%%%%%%%%%%%%%%%%%%%
\subsection{Conditional Processing}
\label{sec:conditional}

The package provides a mechanism to compile different versions
of a document. To customise the versions further some conditional processing
can come in handy to distinguish which version is being compiled.
The package provides two macros to describe the compilation context:

%%%%%%%%%%%%%%%%%%%%%%%%%%%%%%%%%%%%%%%%
\DescribeMacro{\ifchilddoc}
The conditional |\ifchilddoc| distinguishes between the compilation of
child documents and the main document:
%
\begin{center}
|\ifchilddoc |\textit{child-code}| |[|\||else |\textit{main-code}]| \||fi|
\end{center}

%%%%%%%%%%%%%%%%%%%%%%%%%%%%%%%%%%%%%%%%
\DescribeMacro{\childdocname}
\DescribeMacro{\childdocjob}
The macro |\childdocname| contains the filename (without extension)
of the main or child file being processed.
Note that |\childdocjob| will always contain the name of the main file.

%%%%%%%%%%%%%%%%%%%%%%%%%%%%%%%%%%%%%%%%
\paragraph{Title Page.}

Conditional processing can be used to include a title or banner page
in the main document when proper precautions are taken.
Importantly, the code in the main file should ensure that the page counter
(as well as other status parameters which are stored in the |.aux| files)
takes the same value after the conditional processing.
Otherwise the page numbers may take divergent values
depending on which part is compiled.

For example, a title page could be declared by:
%
\begin{center}
\begin{tabular}{l}
|\ifchilddoc\||else|\\
|\addtocounter{page}{-1}|\\
\textit{code for title page}\\
|\newpage|\\
|\||fi|
\end{tabular}
\end{center}
%
A banner page for the child documents can be generated by:
%
\begin{center}
\begin{tabular}{l}
|\ifchilddoc|\\
|\addtocounter{page}{-1}|\\
\textit{code for banner page}\\
|\newpage|\\
|\||fi|
\end{tabular}
\end{center}
%
Here one could write a message such as:
\begin{center}
|This is the part \childdocname{} of \childdocjob{}.|
\end{center}

%%%%%%%%%%%%%%%%%%%%%%%%%%%%%%%%%%%%%%%%%%%%%%%%%%%%%%%%%%%%%%%%%%%%%%%%%%%%%%%%
\subsection{Flags}
\label{sec:flags}

The package makes it easy to generate different versions
of the main or child documents.
To this end compilation flags can be defined
and assigned different default values.
They will be particularly useful in conjunction
with the forwarding mechanism described in \secref{sec:forward}.

For example, it may be useful to have a flag |\version|
which can be set to |draft| or |final|.
The document source will contain some conditional code
depending on the value of |\version|.
Suppose further, the flag should default to |final| for the main file
and to |draft| for child files
which is a natural assignment for editing the document.
This is achieved by placing the following code
in the preamble of the main document
(below the |\childdocmain| directive):
%
\begin{center}
\begin{tabular}{l}
|\ifchilddoc|\\
|\providecommand{\version}{draft}|\\
|\||else|\\
|\providecommand{\version}{final}|\\
|\||fi|
\end{tabular}
\end{center}
%
The definition by |\providecommand| makes sure
that previous definitions are not overwritten.
Further statements |\providecommand{\version}{...}|
can thus be added before the above code to override it.

For the main file, one might add a line
(between |\childdocmain| and the above block)
%
\begin{center}
|%\ifchilddoc\||else\providecommand{\version}{draft}\||fi|
\end{center}
%
which can be uncommented to produce a draft version.
Likewise one can add a line to the very top of a child file
(above the |\childdocof{|\textit{main}|}| directive)
%
\begin{center}
|%\providecommand{\version}{final}|
\end{center}
%
which can be uncommented to produce the final version of this child document.

%%%%%%%%%%%%%%%%%%%%%%%%%%%%%%%%%%%%%%%%%%%%%%%%%%%%%%%%%%%%%%%%%%%%%%%%%%%%%%%%
\subsection{Forwarding}
\label{sec:forward}

Different versions of the main or child documents
using compilation flags as described in \secref{sec:flags}
can be (permanently) stored in different files
for convenient compilation, viewing and distribution.
To this end, the package defines a command
to pass on compilation to a different file:

%%%%%%%%%%%%%%%%%%%%%%%%%%%%%%%%%%%%%%%%
\DescribeMacro{\childdocforward}
The command |\childdocforward| redirects processing to
another source file:
%
\begin{center}
\begin{tabular}{l}
|\input{childdoc.def}|\\
|\childdocforward[|\textit{main}|]{|\textit{dest}|}|\\
\end{tabular}
\end{center}
%
The argument \textit{dest} is the destination file
(without extension).
It should be the main file or one of the child files.
Note that further \textsf{childdoc} directives
such as |\childdocof| and |\childdocforward|
in the indicated file will be processed in this form.
The optional argument \textit{main}
passes on directly to the main file \textit{main}
while pretending to compile the child \textit{dest}.
This form behaves as if \textit{dest}
issues |\childdocof{|\textit{main}|}| right away,
and no further \textsf{childdoc} directives will be processed.

%%%%%%%%%%%%%%%%%%%%%%%%%%%%%%%%%%%%%%%%
\DescribeMacro{\...prefix}
In the alternative form |\childdocforwardprefix|,
%
\begin{center}
\begin{tabular}{l}
|\input{childdoc.def}|\\
|\childdocforwardprefix[|\textit{main}|]{|\textit{prefix}|}{|\textit{dest}|}|
\end{tabular}
\end{center}
%
the destination file is determined by a pattern
depending on the current file:
To make this work, the current file must be called
`{\textit{prefix}\hspace{0.2em}\textit{suffix}}'
with \textit{prefix} matching precisely the argument.
Processing is then passed on to the file
`{\textit{dest}\hspace{0.2em}\textit{suffix}}'.
Surely, the same effect is achieved by
directly specifying the
argument `{\textit{dest}\hspace{0.2em}\textit{suffix}}'
in the first form.
However, that requires to set up a different file
for each child. With the alternative form of the command
all these files can have exactly the same content
which simplifies setting them up and maintaining them.

For example, the following file |draft.tex|
with a compilation flag |\version| as described in \secref{sec:flags}
compiles the main document as a draft:
%
\begin{center}
\begin{tabular}{l}
|\def\version{draft}|\\
|\input{childdoc.def}|\\
|\childdocforward{|\textit{main}|}|
\end{tabular}
\end{center}
%
Likewise, the following files |final|\textit{nn}|.tex|
compile the final version of the child document
|child|\textit{nn}|.tex|:
%
\begin{center}
\begin{tabular}{l}
|\def\version{final}|\\
|\input{childdoc.def}|\\
|\childdocforwardprefix{final}{child}|
\end{tabular}
\end{center}
%

Note that when several versions of a main file and/or of each child file
are to be generated, it may be convenient to set up a |Makefile| or
shell script to automatise the process.

%%%%%%%%%%%%%%%%%%%%%%%%%%%%%%%%%%%%%%%%%%%%%%%%%%%%%%%%%%%%%%%%%%%%%%%%%%%%%%%%
\subsection{Command Line Processing}
\label{sec:commandline}

The effect of redirection files can also be achieved by invoking
the \LaTeX{} compiler with a more elaborate command line.
Most conveniently this should be done as part
of a shell script or a |Makefile|.

When using \textsf{childdoc} in the main file, the following
command lines effectively perform a redirection
(note that depending on the shell being used,
backslashes may have to be doubled: `|\|' $\to$ `|\\|'):
%
\begin{center}
|... -jobname "|\textit{target}|" |\\|"|[\textit{flags}]%
|\input{childdoc.def}\childdocforward[|\textit{main}|]{|\textit{dest}|}"|
\end{center}
%
Here \textit{target} is the name of the output file,
\textit{main} is the name of the main file
and \textit{dest} is the name of the main or child file to be processed
(all filenames without extensions).
The optional argument \textit{main} can be omitted
if \textit{main} matches \textit{dest}.
Optionally, compilation \textit{flags} can be defined via |\def| commands.
This command line makes the \TeX{} engine believe
it is compiling the file \textit{target}
whose content is specified as the latter parameter.
The provided code then forwards the processing to
\textit{main} or \textit{dest} as described in \secref{sec:forward}.

%%%%%%%%%%%%%%%%%%%%%%%%%%%%%%%%%%%%%%%%%%%%%%%%%%%%%%%%%%%%%%%%%%%%%%%%%%%%%%%%
\subsection{Include by Input}
\label{sec:input}

Including child documents by |\include| has some restrictions by design.
Most notably, the content of a child document always occupies
its own set of pages; pages cannot be shared between child documents.
Usually, this behaviour makes perfect sense
because each child document contain an essential part of the document.
However, in some situations it may be desirable to compose
a document from a collection of parts
without having mandatory page breaks between then.
For this case, the package
provides a mechanism to include parts
by |\input| which can also be processed individually.
However, by construction this mechanism
requires manual handling of the content to be output.

%%%%%%%%%%%%%%%%%%%%%%%%%%%%%%%%%%%%%%%%
\DescribeMacro{\ifchilddocmanual}
The main file should be prepared as usual, see \secref{sec:include}.
However, the document body must make a distinction
between processing of an individual part and of the main document, e.g.:
%
\begin{center}
\begin{tabular}{l}
|\ifchilddocmanual|\\
|\input{\childdocname}|\\
|\||else|\\
\textit{document body with }|\input{|\textit{part}|}|\\
|\||fi|
\end{tabular}
\end{center}
%
The conditional |\ifchilddocmanual| is true whenever
a part to be included by |\input| is being compiled,
and the name of the part is stored in |\childdocname|.

%%%%%%%%%%%%%%%%%%%%%%%%%%%%%%%%%%%%%%%%
\DescribeMacro{\childdocby}
Each part to be included by |\input| should start with:
%
\begin{center}
\begin{tabular}{l}
|\input{childdoc.def}|\\
|\childdocby{|\textit{main}|}|\\
\end{tabular}
\end{center}
%
The directive |\childdocby| is similar to |\childdocof|
described in \secref{sec:include},
but the subsequent selection of content must be done manually.
To that end, both |\ifchilddoc| and |\ifchilddocmanual|
will be true upon processing of a part,
and the name of the part is stored in |\childdocname|.
Note that |\jobname| will be set to the filename of the current part
so that each part receives an individual |.aux| file
that does not interfere with the |.aux| file(s) of the main document.
This behaviour can be altered by the alternative form
|\childdocby[*]{|\textit{main}|}| (with a non-empty optional argument)
which uses the |.aux| file of the main document
by setting |\jobname| to \textit{main}.

%%%%%%%%%%%%%%%%%%%%%%%%%%%%%%%%%%%%%%%%%%%%%%%%%%%%%%%%%%%%%%%%%%%%%%%%%%%%%%%%
\subsection{Driver Development}
\label{sec:driver}

The \textsf{childdoc} mechanism can also be use for the development
of definition files such as \LaTeX{} styles or classes.
This case differs from the above setup with multiple parts
included by |\include| in that no |\includeonly| should be invoked.
This can be achieved by starting the include file
(before |\ProvidesPackage|) with:
%
\begin{center}
\begin{tabular}{l}
|\input{childdoc.def}|\\
|\childdocforward{|\textit{main}|}|\\
\end{tabular}
\end{center}
%
or alternatively with:
%
\begin{center}
\begin{tabular}{l}
|\input{childdoc.def}|\\
|\childdocby{|\textit{main}|}|\\
\end{tabular}
\end{center}
%
Both forms have slightly different effects as described above.
The main file is prepared as usual, see \secref{sec:include}.

%%%%%%%%%%%%%%%%%%%%%%%%%%%%%%%%%%%%%%%%%%%%%%%%%%%%%%%%%%%%%%%%%%%%%%%%%%%%%%%%
\subsection{Legacy Detection}
\label{sec:detection}

The directive |\childdocmain| in the main file can detect
whether the complete document or merely a child is to be compiled
even without using the directive |\childdocof|.
This method is deprecated because it is less robust
and there is no compelling reason to use it;
it is merely provided for backward compatibility
and it may be removed in future versions.

If the detection mechanism is to be used,
it is mandatory to correctly specify
the filename of the main file as the argument of |\childdocmain|:
%
\begin{center}
\begin{tabular}{l}
|\input{childdoc.def}|\\
|\childdocmain{|\textit{main}|}|\\
\end{tabular}
\end{center}
%
If |\jobname| does not match the argument \textit{main} of |\childdocmain|,
it is assumed that |\jobname| points to the child file to be compiled.
When using |\childdocmain| with the main file specified as argument,
it suffices to start a child file
with just |\input{|\textit{main}|}|
without loading of the package and using |\childdocof|.
If instead all processing is done
with the appropriate \textsf{childdoc} directives,
the argument of \textit{main} of |\childdocmain| can be empty.

An alternative version of the command line processing described
in \secref{sec:commandline} using the detection mechanism reads:
%
\begin{center}
|... -jobname "|\textit{target}|" "|[\textit{flags}]%
[|\def\jobname{|\textit{dest}|}|]|\input{|\textit{main}|}"|
\end{center}

%%%%%%%%%%%%%%%%%%%%%%%%%%%%%%%%%%%%%%%%%%%%%%%%%%%%%%%%%%%%%%%%%%%%%%%%%%%%%%%%
\subsection{Manual Code}
\label{sec:manual}

In case one cannot be certain whether the definitions file |childdoc.def|
is installed on the target \TeX{} distribution
and one prefers not to ship it,
it is conceivable to paste a few relevant commands into the sources.

To that end, drop all statements |\input{childdoc.def}|
and perform the replacements as outlined below.
Instead of |\childdocmain{|\textit{main}|}| add the following code
to the top of the main file:
%
\begin{center}
\begin{tabular}{l}
|\||ifdefined\childdocname\endinput\||fi\newif\ifchilddoc|\\
|\edef\childdocname{\scantokens\expandafter{\jobname\noexpand}}|\\
|\def\childdocmain{|\textit{main}|}\||ifx\childdocmain\childdocname\||else|\\
|\childdoctrue\includeonly{\childdocname}\let\jobname\childdocmain\||fi|\\
\end{tabular}
\end{center}
%
Instead of |\childdocof{|\textit{main}|}| just include the main file
at the top of each child file:
%
\begin{center}
|\input{|\textit{main}|}|
\end{center}
%
A simple redirection |\childdocforward{|\textit{dest}|}| is achieved by:
%
\begin{center}
|\def\jobname{|\textit{dest}|}\input{\jobname}|
\end{center}
%
The redirection with prefix
|\childdocforwardprefix[|\textit{prefix}|]{|\textit{dest}|}|
is accomplished by:
%
\begin{center}
\begin{tabular}{l}
|{\edef\jobname{\scantokens\expandafter{\jobname\noexpand}}|\\
|\def\redirectjob |\textit{prefix}|#1~~~{\gdef\jobname{|\textit{dest}|#1}}|\\
|\expandafter\redirectjob\jobname~~~}\input{\jobname}|
\end{tabular}
\end{center}

In an alternative approach,
child documents can be compiled by a specific command line
without additional code or specific definitions:
%
\begin{center}
|... -jobname "|\textit{target}|" "|[\textit{flags}]%
|\includeonly{|\textit{dest}|}\input{|\textit{main}|}"|
\end{center}
%

%%%%%%%%%%%%%%%%%%%%%%%%%%%%%%%%%%%%%%%%%%%%%%%%%%%%%%%%%%%%%%%%%%%%%%%%%%%%%%%%
%%%%%%%%%%%%%%%%%%%%%%%%%%%%%%%%%%%%%%%%%%%%%%%%%%%%%%%%%%%%%%%%%%%%%%%%%%%%%%%%
\section{Information}

%%%%%%%%%%%%%%%%%%%%%%%%%%%%%%%%%%%%%%%%%%%%%%%%%%%%%%%%%%%%%%%%%%%%%%%%%%%%%%%%
\subsection{Copyright}

Copyright \copyright{} 2017--2018 Niklas Beisert

This work may be distributed and/or modified under the
conditions of the \LaTeX{} Project Public License, either version 1.3
of this license or (at your option) any later version.
The latest version of this license is in
  \url{http://www.latex-project.org/lppl.txt}
and version 1.3 or later is part of all distributions of \LaTeX{}
version 2005/12/01 or later.

This work has the LPPL maintenance status `maintained'.

The Current Maintainer of this work is Niklas Beisert.

This work consists of the files |README.txt|, |childdoc.ins| and |childdoc.dtx|
as well as the derived files |childdoc.def|, |cdocsamp.tex|
with |cdocsch1.tex|, |cdocsch2.tex|, |cdocspt3.tex|, |cdocspt4.tex|,
|cdocsdrf.tex|, |cdocsfn1.tex|, |cdocsfn2.tex|
as well as |childdoc.pdf|.

%%%%%%%%%%%%%%%%%%%%%%%%%%%%%%%%%%%%%%%%%%%%%%%%%%%%%%%%%%%%%%%%%%%%%%%%%%%%%%%%
\subsection{Files and Installation}

The package consists of the files:
%
\begin{center}
\begin{tabular}{ll}
    |README.txt|   & readme file \\
    |childdoc.ins| & installation file \\
    |childdoc.dtx| & source file \\
    |childdoc.def| & definition file \\
    |cdocsamp.tex| & sample main file \\
    |cdocsch1.tex| & sample include file \\
    |cdocsch2.tex| & sample include file \\
    |cdocspt3.tex| & sample part file \\
    |cdocspt4.tex| & sample part file \\
    |cdocsdrf.tex| & sample redirection file \\
    |cdocsfn1.tex| & sample redirection file \\
    |cdocsfn2.tex| & sample redirection file \\
    |childdoc.pdf| & manual
\end{tabular}
\end{center}
%
The distribution consists of the files
|README.txt|, |childdoc.ins| and |childdoc.dtx|.
%
\begin{itemize}
\item
Run (pdf)\LaTeX{} on |childdoc.dtx|
to compile the manual |childdoc.pdf| (this file).
\item
Run \LaTeX{} on |childdoc.ins| to create the definitions file |childdoc.def|
and the sample |cdocsamp.tex| with include files
|cdocsch1.tex|, |cdocsch2.tex|, |cdocspt3.tex|, |cdocspt4.tex|,
|cdocsdrf.tex|, |cdocsfn1.tex|, |cdocsfn2.tex|.
Then copy the file |childdoc.def| to an appropriate directory of your \LaTeX{}
distribution, e.g.\ \textit{texmf-root}|/tex/latex/childdoc|.
\end{itemize}

%%%%%%%%%%%%%%%%%%%%%%%%%%%%%%%%%%%%%%%%%%%%%%%%%%%%%%%%%%%%%%%%%%%%%%%%%%%%%%%%
\subsection{Related CTAN Packages}

There are several other packages which offer a similar functionality:
%
\begin{itemize}
\item
The packages
\href{http://ctan.org/pkg/docmute}{\textsf{docmute}},
\href{http://ctan.org/pkg/includex}{\textsf{includex}} and
\href{http://ctan.org/pkg/standalone}{\textsf{standalone}}
provide commands to include only the document body of
a child file thus allowing both files to be compiled individually.
\item
The packages \href{http://ctan.org/pkg/subdocs}{\textsf{subdocs}}
and \href{http://ctan.org/pkg/subfiles}{\textsf{subfiles}}
provide structures in which the main and child documents can be
encapsulated and allowing them to be compiled individually.
The inclusion mechanism is different from the conventional |\include|.
\item
The package \href{http://ctan.org/pkg/combine}{\textsf{combine}}
is an elaborate solution to combine several documents into one.
\end{itemize}
%
See also the CTAN topic \href{http://ctan.org/topic/subdocs}{\textsf{subdocs}}
for further related packages.
The present package differs from the above solutions in that
a document structure constructed with the conventional |\include| mechanism
just needs two extra commands at the top of every file
such that all constituent files can be compiled individually.

%%%%%%%%%%%%%%%%%%%%%%%%%%%%%%%%%%%%%%%%%%%%%%%%%%%%%%%%%%%%%%%%%%%%%%%%%%%%%%%%
%\subsection{Feature Suggestions}
%
%The following is a list of features which may be useful for future
%versions of this package:
%%
%\begin{itemize}
%\item
%\ldots
%\end{itemize}

%%%%%%%%%%%%%%%%%%%%%%%%%%%%%%%%%%%%%%%%%%%%%%%%%%%%%%%%%%%%%%%%%%%%%%%%%%%%%%%%
\subsection{Revision History}

%%%%%%%%%%%%%%%%%%%%%%%%%%%%%%%%%%%%%%%%
\paragraph{v2.0:} 2018/12/30

\begin{itemize}
\item
immediate forward processing
\item
added |\childdocby| mechanism
\item
manual restructured
\end{itemize}

%%%%%%%%%%%%%%%%%%%%%%%%%%%%%%%%%%%%%%%%
\paragraph{v1.6:} 2018/01/17

\begin{itemize}
\item
application for development of include files
\item
corrections to manual
\end{itemize}

%%%%%%%%%%%%%%%%%%%%%%%%%%%%%%%%%%%%%%%%
\paragraph{v1.5:} 2017/05/21

\begin{itemize}
\item
more complete structuring introduced
\item
|\childdocof| introduced
\item
|\childdoc| renamed to |\childdocmain|
\item
|\childredirect| renamed to |\childdocforward| and |\childdocforwardprefix|
and functionality expanded
\end{itemize}

%%%%%%%%%%%%%%%%%%%%%%%%%%%%%%%%%%%%%%%%
\paragraph{v1.0:} 2017/04/27

\begin{itemize}
\item
manual and install package
\item
first version published on CTAN
\end{itemize}

%%%%%%%%%%%%%%%%%%%%%%%%%%%%%%%%%%%%%%%%
\paragraph{v0.6:} 2017/04/26

\begin{itemize}
\item
redirection mechanism added
\end{itemize}

%%%%%%%%%%%%%%%%%%%%%%%%%%%%%%%%%%%%%%%%
\paragraph{v0.5:} 2017/04/26

\begin{itemize}
\item
functionality in definition file
\end{itemize}


%%%%%%%%%%%%%%%%%%%%%%%%%%%%%%%%%%%%%%%%%%%%%%%%%%%%%%%%%%%%%%%%%%%%%%%%%%%%%%%%
%%%%%%%%%%%%%%%%%%%%%%%%%%%%%%%%%%%%%%%%%%%%%%%%%%%%%%%%%%%%%%%%%%%%%%%%%%%%%%%%
%%%%%%%%%%%%%%%%%%%%%%%%%%%%%%%%%%%%%%%%%%%%%%%%%%%%%%%%%%%%%%%%%%%%%%%%%%%%%%%%
\appendix

\settowidth\MacroIndent{\rmfamily\scriptsize 000\ }

 \DocInput{childdoc.dtx}

\end{document}
%</driver>
% \fi
%
% %%%%%%%%%%%%%%%%%%%%%%%%%%%%%%%%%%%%%%%%%%%%%%%%%%%%%%%%%%%%%%%%%%%%%%%%%%%%%%
% %%%%%%%%%%%%%%%%%%%%%%%%%%%%%%%%%%%%%%%%%%%%%%%%%%%%%%%%%%%%%%%%%%%%%%%%%%%%%%
% \section{Sample}
%\iffalse
%<*samplemain>
%\fi
%
% The following presents a sample document
% with two chapters, two parts, a title page,
% a compile flag as well as three forwarding files to set the flag.
% It consists of eight |.tex| files:
% \begin{center}
% \begin{tabular}{ll}
% |cdocsamp.tex|&main file\\
% |cdocsch1.tex|&include file for chapter 1\\
% |cdocsch2.tex|&include file for chapter 2\\
% |cdocspt3.tex|&include file for part 3\\
% |cdocspt4.tex|&include file for part 4\\
% |cdocsdrf.tex|&forwarding file for main file in draft mode\\
% |cdocsfi1.tex|&forwarding file for final version of chapter 1\\
% |cdocsfi2.tex|&forwarding file for final version of chapter 2\\
% \end{tabular}
% \end{center}
% Each of the eight files can be compiled directly by the \LaTeX{} compiler.
%
% %%%%%%%%%%%%%%%%%%%%%%%%%%%%%%%%%%%%%%
% \paragraph{Main File.}
%
% The main file is called |cdocsamp.tex|.
%
% Load the \textsf{childdoc} definitions and
% declare the filename for the main document:
%    \begin{macrocode}
\input{childdoc.def}
\childdocmain{}
%    \end{macrocode}

% Optional override for |\version| flag:
%    \begin{macrocode}
%%\ifchilddoc\else\providecommand{\version}{draft}\fi
%    \end{macrocode}

% Define the default values for the |\version| flag
% (|final| for the main file and |draft| for childs):
%    \begin{macrocode}
\ifchilddoc
\providecommand{\version}{draft}
\else
\providecommand{\version}{final}
\fi
%    \end{macrocode}

% Load the standard document class:
%    \begin{macrocode}
\documentclass[12pt]{article}
%    \end{macrocode}

% Start the document body:
%    \begin{macrocode}
\begin{document}
%    \end{macrocode}

% Declare a title page.
% Print title, part of document being processed and version flag:
%    \begin{macrocode}
\addtocounter{page}{-1}
\begin{center}
{\LARGE\bfseries{}childdoc example\par}
\vspace{1cm}
\ifchilddoc
\ifchilddocmanual part\else chapter\fi:
`\childdocname' of `\childdocjob'\par
\else
main document: `\childdocjob'\par
\fi
version: \version\par
\end{center}
\newpage
%    \end{macrocode}

% Manually include selected file,
% otherwise process as usual:
%    \begin{macrocode}
\ifchilddocmanual
\section*{part `\childdocname'}
\input{\childdocname}
\else
%    \end{macrocode}

% Include the two chapters:
%    \begin{macrocode}
\include{cdocsch1}
\include{cdocsch2}
%    \end{macrocode}

% Include the two parts unless only chapters should be displayed:
%    \begin{macrocode}
\ifchilddoc\else
\section{part three}
\input{cdocspt3}
\section{part four}
\input{cdocspt4}
\fi
%    \end{macrocode}

% Process as usual until here:
%    \begin{macrocode}
\fi
%    \end{macrocode}

% End of document body:
%    \begin{macrocode}
\end{document}
%    \end{macrocode}
%\iffalse
%</samplemain>
%\fi
%
% %%%%%%%%%%%%%%%%%%%%%%%%%%%%%%%%%%%%%%
% \paragraph{Chapter Include Files.}
%
% The include files are called |cdocsch1.tex| and |cdocsch2.tex|.
%
%\iffalse
%<*samplechap1|samplechap2>
%\fi

% Optional override for |\version| flag:
%    \begin{macrocode}
%%\providecommand{\version}{final}
%    \end{macrocode}

% Include the main document:
%    \begin{macrocode}
\input{childdoc.def}
\childdocof{cdocsamp}
%    \end{macrocode}

%\iffalse
%</samplechap1|samplechap2>
%\fi
%
%\iffalse
%<*samplechap1>
%\fi
% Some text for chapter 1:
%    \begin{macrocode}
\section{one}
some text in chapter one
%    \end{macrocode}

%\iffalse
%</samplechap1>
%\fi
% Some text for chapter 2:
%\iffalse
%<*samplechap2>
%\fi
%    \begin{macrocode}
\section{two}
more text in chapter two
%    \end{macrocode}

%\iffalse
%</samplechap2>
%\fi
%
% %%%%%%%%%%%%%%%%%%%%%%%%%%%%%%%%%%%%%%
% \paragraph{Part Include Files.}
%
% The include files are called |cdocspt3.tex| and |cdocspt4.tex|.
%
%\iffalse
%<*samplepart3|samplepart4>
%\fi

% Optional override for |\version| flag:
%    \begin{macrocode}
%%\providecommand{\version}{final}
%    \end{macrocode}

% Include the main document:
%    \begin{macrocode}
\input{childdoc.def}
\childdocby{cdocsamp}
%    \end{macrocode}

%\iffalse
%</samplepart3|samplepart4>
%\fi
%
%\iffalse
%<*samplepart3>
%\fi
% Some text for part 3:
%    \begin{macrocode}
some text in part three
%    \end{macrocode}

%\iffalse
%</samplepart3>
%\fi
% Some text for part 4:
%\iffalse
%<*samplepart4>
%\fi
%    \begin{macrocode}
more text in part four
%    \end{macrocode}

%\iffalse
%</samplepart4>
%\fi
%
% %%%%%%%%%%%%%%%%%%%%%%%%%%%%%%%%%%%%%%
% \paragraph{Forwarding for a Complete Draft.}
%
% The following forwarding file |cdocsdrf.tex|
% compiles the main document in draft mode:
%\iffalse
%<*sampledraft>
%\fi
%    \begin{macrocode}
\def\version{draft}
\input{childdoc.def}
\childdocforward{cdocsamp}
%    \end{macrocode}

%\iffalse
%</sampledraft>
%\fi
%
% %%%%%%%%%%%%%%%%%%%%%%%%%%%%%%%%%%%%%%
% \paragraph{Forwarding for Final Version of the Chapters.}
%
% The following forwarding files |cdocsfn1.tex| and |cdocsfn2.tex|
% (with identical content)
% compile the final versions of the child documents
% |cdocsch1.tex| and |cdocsch2.tex|, respectively:
%\iffalse
%<*samplefinal>
%\fi
%    \begin{macrocode}
\def\version{final}
\input{childdoc.def}
\childdocforwardprefix[cdocsamp]{cdocsfn}{cdocsch}
%    \end{macrocode}

%\iffalse
%</samplefinal>
%\fi
%
% %%%%%%%%%%%%%%%%%%%%%%%%%%%%%%%%%%%%%%
% \paragraph{Command Line Processing.}
%
% The following three command lines generate the output files
% |cdocscld|, |cdocscl1| and |cdocscl2|
% which should be identical to
% |cdocsdrf|, |cdocsch1| and |cdocsfn2|, respectively:
% \begin{center}
% \begin{tabular}{l}
% |latex -jobname cdocscld \|\\
% |  "\def\version{draft}\input{childdoc.def}\childdocforward{cdocsamp}"|\\
% |latex -jobname cdocscl1 \|\\
% |  "\input{childdoc.def}\childdocforward[cdocsamp]{cdocsch1}"|\\
% |latex -jobname cdocscl2 \|\\
% |  "\def\version{final}\input{childdoc.def}\childdocforward{cdocsch2}"|
% \end{tabular}
% \end{center}
% Note that the trailing backslash on each first line
% merely continues the input to the second line
% (for convenient cut ant paste).
% Furthermore, the command |latex| can be replaced by any
% of its alternative versions such as |pdflatex|.
%
% %%%%%%%%%%%%%%%%%%%%%%%%%%%%%%%%%%%%%%%%%%%%%%%%%%%%%%%%%%%%%%%%%%%%%%%%%%%%%%
% %%%%%%%%%%%%%%%%%%%%%%%%%%%%%%%%%%%%%%%%%%%%%%%%%%%%%%%%%%%%%%%%%%%%%%%%%%%%%%
% \section{Implementation}
%\iffalse
%<*package>
%\fi
%
% This section describes the definitions file |childdoc.def|.

% The definitions cannot be loaded using |\usepackage| or |\RequirePackage|
% which has a mechanism to prevent loading a style file more than once.
% When loading the definitions by means of |\input|
% multiple instances have to be prevented manually:
%\iffalse
%This code needs to be before the `\ProvidesFile' directive
%which is defined at the beginning of this file.
%Therefore it is also placed there and commented out here.
%</package>
%<*discard>
%\fi
%    \begin{macrocode}
\ifdefined\childdocmain\endinput\fi
%    \end{macrocode}
%\iffalse
%</discard>
%<*package>
%\fi
%
% \macro{\ifchilddoc}
% \macro{\ifchilddocmanual}
% The conditional |\ifchilddoc| tells whether a
% child (true) or main (false) document is being compiled.
% The conditional |\ifchilddocmanual| tells whether
% the |\includeonly| mechanism is used (false) or
% the selection of child files must be performed manually (true).
% The definitions initialise to false:
%    \begin{macrocode}
\newif\ifchilddoc
\newif\ifchilddocmanual
%    \end{macrocode}

% \macro{\childdocname}
% \macro{\childdocjob}
% The macro |\childdocname| stores the name of the main document
% to be compiled. The macro |\childdocjob| stores the name of
% the document on which the \LaTeX{} compiler was originally invoked.
% The content of |\jobname| cannot be compared
% to filenames specified in the source due to different catcodes.
% The following code rescans |\jobname|, stores the result
% in |\childdocname| and saves a copy in |\childdocjob|:
%    \begin{macrocode}
\edef\childdocname{\scantokens\expandafter{\jobname\noexpand}}
\let\childdocjob\childdocname
%    \end{macrocode}

% \macro{\childdocdisable}
% The macro |\childdocdisable| prevents the main file
% from being processed more than once.
% At this stage, the main document command |\childdocmain|
% is assumed to be called once again where it should do nothing.
% Any subsequent call to it should prevent
% a secondary processing of the main document
% It overwrites the forwarding commands
% |\childdocof| and |\childdocforward|
% with empty macros to prevent further inclusions of the main document:
%    \begin{macrocode}
\newcommand{\childdocdisable}
{
  \renewcommand{\childdocmain}[1]{\renewcommand{\childdocmain}[1]{\endinput}}
  \renewcommand{\childdocof}[1]{}
  \renewcommand{\childdocby}[2][]{}
  \renewcommand{\childdocforward}[2][]{}
  \renewcommand{\childdocdisable}{}
}
%    \end{macrocode}

% \macro{\childdocmain}
% The macro |\childdocmain| is to be called at the top of the main file
% with nothing or the main filename (without extension) as argument.
% First, it breaks loops.
% If the argument is not empty and does not match |\childdocname|
% (which is set by the first inclusion of |childdoc.def|),
% |\ifchilddoc| is set to true, |\includeonly| is applied to the child file
% and |\jobname| is set to the main file
% (for proper handling of |.aux| files):
%    \begin{macrocode}
\newcommand{\childdocmain}[1]
{
  \childdocdisable\childdocmain{}
  \if?#1?\else
    \begingroup
      \def\childdoctmp{#1}
      \ifx\childdoctmp\childdocname
        \def\childdoctmp{}
      \else
        \def\childdoctmp
        {
          \childdoctrue
          \includeonly{\childdocname}
          \def\childdocjob{#1}
          \def\jobname{#1}
        }
      \fi
      \expandafter
    \endgroup
    \childdoctmp
  \fi
}
%    \end{macrocode}

% \macro{\childdocof}
% The command |\childdocof| redirects
% compilation to the main file |#1|.
%    \begin{macrocode}
\newcommand{\childdocof}[1]
{
  \childdocdisable
  \childdoctrue
  \includeonly{\childdocname}
  \def\jobname{#1}
  \def\childdocjob{#1}
  \input{#1}
}
%    \end{macrocode}

% \macro{\childdocby}
% The command |\childdocby| ....
%    \begin{macrocode}
\newcommand{\childdocby}[2][]
{
  \childdocdisable
  \childdoctrue
  \childdocmanualtrue
  \if?#1?\else
    \def\jobname{#2}
  \fi
  \def\childdocjob{#2}
  \input{#2}
  \endinput
}
%    \end{macrocode}

% \macro{\childdocforward}
% The command |\childdocforward| redirects
% compilation to the main file or
% (if the optional argument is given) a child file.
% Parameters are set as if the main file
% or a child file starting with |\childdocof| was compiled.
% Then compilation is handed over to the main file:
%    \begin{macrocode}
\newcommand{\childdocforward}[2][]
{
  \begingroup
    \if?#1?
      \def\childdoctmp
      {
        \def\childdocname{#2}
        \def\childdocjob{#2}
        \def\jobname{#2}
        \input{#2}
        \endinput
      }
    \else
      \def\childdoctmp
      {
        \childdocdisable
        \def\childdocname{#2}
        \childdoctrue
        \includeonly{#2}
        \def\childdocjob{#1}
        \def\jobname{#1}
        \input{#1}
        \endinput
      }
    \fi
    \expandafter
  \endgroup
  \childdoctmp
}
%    \end{macrocode}

% \macro{\childdocforwardprefix}
% The command |\childdocforwardprefix| redirects
% compilation to the main or a child file by means of a pattern.
% The prefix |#1| in the current filename is replaced by |#2|
% and the suffix of the current filename is kept
% (it is assumed that the filename does not contain the substring `|~~~|'
% which is used as a delimiter).
% Compilation is handed over to the new file by |\childdocforward|:
%    \begin{macrocode}
\newcommand{\childdocforwardprefix}[3][]
{
  \begingroup
    \def\childdocextract #2##1~~~{\def\childdoctmp{\childdocforward[#1]{#3##1}}}
    \expandafter\childdocextract\childdocname~~~
    \expandafter
  \endgroup
  \childdoctmp
}
%    \end{macrocode}

% \macro{\childdoc}
% The deprecated macro |\childdoc| is a legacy version of |\childdocmain|:
%    \begin{macrocode}
\newcommand{\childdoc}{\childdocmain}
%    \end{macrocode}

% \macro{\childdocredirect}
% The deprecated macro |\childdocredirect| is a legacy version
% of |\childdocforward| and |\childdocforwardprefix|:
%    \begin{macrocode}
\newcommand{\childdocredirect}[2][]
{
  \begingroup
    \if?#1?
      \def\childdoctmp{\childdocforward{#2}}
    \else
      \def\childdoctmp{\childdocforwardprefix{#1}{#2}}
    \fi
    \expandafter
  \endgroup
  \childdoctmp
}
%    \end{macrocode}

%\iffalse
%</package>
%\fi
%
\endinput

\childdocby{cdocsamp}
%    \end{macrocode}

%\iffalse
%</samplepart3|samplepart4>
%\fi
%
%\iffalse
%<*samplepart3>
%\fi
% Some text for part 3:
%    \begin{macrocode}
some text in part three
%    \end{macrocode}

%\iffalse
%</samplepart3>
%\fi
% Some text for part 4:
%\iffalse
%<*samplepart4>
%\fi
%    \begin{macrocode}
more text in part four
%    \end{macrocode}

%\iffalse
%</samplepart4>
%\fi
%
% %%%%%%%%%%%%%%%%%%%%%%%%%%%%%%%%%%%%%%
% \paragraph{Forwarding for a Complete Draft.}
%
% The following forwarding file |cdocsdrf.tex|
% compiles the main document in draft mode:
%\iffalse
%<*sampledraft>
%\fi
%    \begin{macrocode}
\def\version{draft}
% \iffalse
%
% childdoc.dtx Copyright (C) 2017-2018 Niklas Beisert
%
% This work may be distributed and/or modified under the
% conditions of the LaTeX Project Public License, either version 1.3
% of this license or (at your option) any later version.
% The latest version of this license is in
%   http://www.latex-project.org/lppl.txt
% and version 1.3 or later is part of all distributions of LaTeX
% version 2005/12/01 or later.
%
% This work has the LPPL maintenance status `maintained'.
%
% The Current Maintainer of this work is Niklas Beisert.
%
% This work consists of the files childdoc.dtx and childdoc.ins
% and the derived files childdoc.def and cdocsamp.tex with
% cdocsch1.tex, cdocsch2.tex, cdocsdrf.tex, cdocsfn1.tex, cdocsfn2.tex.
%
%<package>\ifdefined\childdocmain\endinput\fi
%<package>\ProvidesFile{childdoc.def}[2018/12/30 v2.0 child document driver]
%<samplemain>\ProvidesFile{cdocsamp.tex}[2018/12/30 v2.0 sample for childdoc]
%<*driver>
%\ProvidesFile{childdoc.drv}[2018/12/30 v2.0 childdoc reference manual file]
\PassOptionsToClass{10pt,a4paper}{article}
\documentclass{ltxdoc}

\usepackage[margin=35mm]{geometry}
\usepackage{hyperref}
\usepackage{hyperxmp}
\usepackage[usenames]{color}

\hypersetup{colorlinks=true}
\hypersetup{pdfstartview=FitH}
\hypersetup{pdfpagemode=UseNone}
\hypersetup{pdfsource={}}
\hypersetup{pdflang={en-UK}}
\hypersetup{pdfcopyright={Copyright 2017-2018 Niklas Beisert.
  This work may be distributed and/or modified under the
  conditions of the LaTeX Project Public License, either version 1.3
  of this license or (at your option) any later version.}}
\hypersetup{pdflicenseurl={http://www.latex-project.org/lppl.txt}}
\hypersetup{pdfcontactaddress={ETH Zurich, ITP, HIT K,
  Wolfgang-Pauli-Strasse 27}}
\hypersetup{pdfcontactpostcode={8093}}
\hypersetup{pdfcontactcity={Zurich}}
\hypersetup{pdfcontactcountry={Switzerland}}
\hypersetup{pdfcontactemail={nbeisert@itp.phys.ethz.ch}}
\hypersetup{pdfcontacturl={http://people.phys.ethz.ch/\xmptilde nbeisert/}}

\newcommand{\secref}[1]{\hyperref[#1]{section \ref*{#1}}}

\parskip1ex
\parindent0pt
\let\olditemize\itemize
\def\itemize{\olditemize\parskip0pt}

\begin{document}

\title{The \textsf{childdoc} Package}
\hypersetup{pdftitle={The childdoc Package}}
\author{Niklas Beisert\\[2ex]
  Institut f\"ur Theoretische Physik\\
  Eidgen\"ossische Technische Hochschule Z\"urich\\
  Wolfgang-Pauli-Strasse 27, 8093 Z\"urich, Switzerland\\[1ex]
  \href{mailto:nbeisert@itp.phys.ethz.ch}
  {\texttt{nbeisert@itp.phys.ethz.ch}}}
\hypersetup{pdfauthor={Niklas Beisert}}
\hypersetup{pdfsubject={Manual for the LaTeX2e Package childdoc}}
\date{30 December 2018, \textsf{v2.0}}
\maketitle

\begin{abstract}\noindent
\textsf{childdoc} is a \LaTeXe{} package
that enables the direct compilation
of document sections included by |\include|
to individual files.
\end{abstract}

\begingroup
\parskip0ex
\tableofcontents
\endgroup

%%%%%%%%%%%%%%%%%%%%%%%%%%%%%%%%%%%%%%%%%%%%%%%%%%%%%%%%%%%%%%%%%%%%%%%%%%%%%%%%
%%%%%%%%%%%%%%%%%%%%%%%%%%%%%%%%%%%%%%%%%%%%%%%%%%%%%%%%%%%%%%%%%%%%%%%%%%%%%%%%
\section{Introduction}

\LaTeX{} provides a mechanism to structure a large document (such as a book)
into a main file and several child files (containing the chapters)
using the |\include| command.
This mechanism is beneficial for documents
which span hundreds of pages in order to
make the source file(s) more manageable.
Moreover, compilation can be restricted to
selected child files by means of the |\includeonly| command.
The latter feature can be used to reduce the compilation time while editing
(this was significantly more useful in the earlier days of \LaTeX{})
or to generate a smaller document which is easier to navigate.
Another application of |\includeonly| is to generate
documents consisting of selected parts of the complete document.

However, there are a few drawbacks of the plain |\include| mechanism:
\begin{itemize}
\item
The child files cannot be compiled on their own,
they can only be compiled via the main file.
A naive editing environment
(such as a text editor with an option
to have the current file processed by \LaTeX)
may require one to switch to the main file before compiling;
attempting to compile the child file produces errors.
\item
The main file must be modified (each time)
to adjust the |\includeonly| command
to the present needs. This easily leaves the main file in a messy state.
\item
The generated document will always carry the filename
of the main document. This is inconvenient if
several child files are to be compiled and
to be kept for distribution.
\end{itemize}

The present package provides a simple interface
to make child files individually compilable by \LaTeX{}.
Compiling a child file then has the same effect as compiling
the main file with an |\includeonly| command
to select the appropriate child.
Moreover the generated document will carry the name of the child
rather than the main file.
This resolves all three above issues.

This feature is meant to make the editing of books,
thesis documents and lecture notes somewhat more convenient.
However, the package can also be used efficiently for
composing a series of documents (such as exercise sheets)
which are typically distributed individually.
It then assists the author in generating the individual documents
(potentially in different versions)
as well as a document containing the collected series.
Another application is in developing style files
or other kinds of included material
where compilation of the style file could redirect
to a sample or test file.

%%%%%%%%%%%%%%%%%%%%%%%%%%%%%%%%%%%%%%%%%%%%%%%%%%%%%%%%%%%%%%%%%%%%%%%%%%%%%%%%
%%%%%%%%%%%%%%%%%%%%%%%%%%%%%%%%%%%%%%%%%%%%%%%%%%%%%%%%%%%%%%%%%%%%%%%%%%%%%%%%
\section{Usage}

First of all, the package \textsf{childdoc} is \emph{not} a standard
\LaTeXe{} |.sty| style file! Therefore it needs to be invoked in
a non-standard way.

%%%%%%%%%%%%%%%%%%%%%%%%%%%%%%%%%%%%%%%%%%%%%%%%%%%%%%%%%%%%%%%%%%%%%%%%%%%%%%%%
\subsection{Included Files}
\label{sec:include}

%%%%%%%%%%%%%%%%%%%%%%%%%%%%%%%%%%%%%%%%
\DescribeMacro{\childdocmain}
To use the package, add the commands
\begin{center}
\begin{tabular}{l}
|\input{childdoc.def}|\\
|\childdocmain{}|\\
\end{tabular}
\end{center}
at the very top of the main \LaTeX{} file,
in particular \emph{before} the |\documentclass| statement!
The argument of |\childdocmain| should be left empty
(but it must be present).

%%%%%%%%%%%%%%%%%%%%%%%%%%%%%%%%%%%%%%%%
\DescribeMacro{\childdocof}
Furthermore, add the commands
\begin{center}
\begin{tabular}{l}
|\input{childdoc.def}|\\
|\childdocof{|\textit{main}|}|\\
\end{tabular}
\end{center}
at the top of every child file \textit{child}
which is included by |\include{|\textit{child}|}|
from within the main file
(or at least for those files to be compiled individually).
The argument \textit{main} must be the filename of the main file.

There are a couple of
considerations in setting up the main and child documents:

%%%%%%%%%%%%%%%%%%%%%%%%%%%%%%%%%%%%%%%%
\paragraph{Restrictions.}

Please note the following restrictions:
\begin{itemize}
\item
|\childdocmain| must be called with one argument \textit{main}
to ensure compatibility with earlier version of the package.
It must either be empty (|\childdocmain{}|)
or precisely match the filename of the main file in which it is specified.
See \secref{sec:detection} for further information.
\item
The filename \textit{main} must be specified without the |.tex| extension.
\item
The filename \textit{main} is case sensitive
(even in case-insensitive file systems)
due to internal string comparison.
\item
The argument \textit{main} should be fully expanded, it cannot be a macro.
\item
Subdirectories and special characters should be avoided in filenames.
\item
The command |\childdocmain{|\textit{main}|}| must be followed by a whitespace.
It should not be followed immediately by another command
or by a comment mark `|%|'.
This is because the \TeX{} parser reads the token immediately following
the argument of |\childdocmain| and puts it
at the beginning of every child section;
however, a white\-space is ignored.
\end{itemize}

%%%%%%%%%%%%%%%%%%%%%%%%%%%%%%%%%%%%%%%%
\paragraph{Content of Main File.}

It is advisable to place all content in the child files included by |\include|.
Any output contained in the main file will appear in all child documents
unless suppressed manually;
it cannot be suppressed automatically by the |\includeonly| directive
and thus should normally be avoided.
A method to include some content in the main file
by means of conditional processing is described in \secref{sec:conditional}.

%%%%%%%%%%%%%%%%%%%%%%%%%%%%%%%%%%%%%%%%
\paragraph{Page Numbering.}

When only a part of the document is compiled,
the appropriate numbering of pages
(as well as other status parameters)
is determined from the |.aux| files.
The latter contain information from previous passes.
However this information needs to propagate through
all intermediate child documents.
Therefore the page numbering in child documents may well
be inconsistent until the complete document is compiled at least once.

A useful (if unconventional) way to always ensure a consistent
page numbering is to restart the numbering in each child document
and denote the pages by `\textit{child}|.|\textit{page}'
where \textit{child} represents the chapter/section number of the child file.
This can be achieved by the command
|\numberwithin{page}{|\textit{child}|}|
of the \textsf{amsmath} package
where \textit{child} can be |chapter| or |section|
depending on the chosen structuring.
Alternatively, one can modify the macro |\thepage| appropriately
and reset the counter |page| at the start of each child file.

%%%%%%%%%%%%%%%%%%%%%%%%%%%%%%%%%%%%%%%%%%%%%%%%%%%%%%%%%%%%%%%%%%%%%%%%%%%%%%%%
\subsection{Conditional Processing}
\label{sec:conditional}

The package provides a mechanism to compile different versions
of a document. To customise the versions further some conditional processing
can come in handy to distinguish which version is being compiled.
The package provides two macros to describe the compilation context:

%%%%%%%%%%%%%%%%%%%%%%%%%%%%%%%%%%%%%%%%
\DescribeMacro{\ifchilddoc}
The conditional |\ifchilddoc| distinguishes between the compilation of
child documents and the main document:
%
\begin{center}
|\ifchilddoc |\textit{child-code}| |[|\||else |\textit{main-code}]| \||fi|
\end{center}

%%%%%%%%%%%%%%%%%%%%%%%%%%%%%%%%%%%%%%%%
\DescribeMacro{\childdocname}
\DescribeMacro{\childdocjob}
The macro |\childdocname| contains the filename (without extension)
of the main or child file being processed.
Note that |\childdocjob| will always contain the name of the main file.

%%%%%%%%%%%%%%%%%%%%%%%%%%%%%%%%%%%%%%%%
\paragraph{Title Page.}

Conditional processing can be used to include a title or banner page
in the main document when proper precautions are taken.
Importantly, the code in the main file should ensure that the page counter
(as well as other status parameters which are stored in the |.aux| files)
takes the same value after the conditional processing.
Otherwise the page numbers may take divergent values
depending on which part is compiled.

For example, a title page could be declared by:
%
\begin{center}
\begin{tabular}{l}
|\ifchilddoc\||else|\\
|\addtocounter{page}{-1}|\\
\textit{code for title page}\\
|\newpage|\\
|\||fi|
\end{tabular}
\end{center}
%
A banner page for the child documents can be generated by:
%
\begin{center}
\begin{tabular}{l}
|\ifchilddoc|\\
|\addtocounter{page}{-1}|\\
\textit{code for banner page}\\
|\newpage|\\
|\||fi|
\end{tabular}
\end{center}
%
Here one could write a message such as:
\begin{center}
|This is the part \childdocname{} of \childdocjob{}.|
\end{center}

%%%%%%%%%%%%%%%%%%%%%%%%%%%%%%%%%%%%%%%%%%%%%%%%%%%%%%%%%%%%%%%%%%%%%%%%%%%%%%%%
\subsection{Flags}
\label{sec:flags}

The package makes it easy to generate different versions
of the main or child documents.
To this end compilation flags can be defined
and assigned different default values.
They will be particularly useful in conjunction
with the forwarding mechanism described in \secref{sec:forward}.

For example, it may be useful to have a flag |\version|
which can be set to |draft| or |final|.
The document source will contain some conditional code
depending on the value of |\version|.
Suppose further, the flag should default to |final| for the main file
and to |draft| for child files
which is a natural assignment for editing the document.
This is achieved by placing the following code
in the preamble of the main document
(below the |\childdocmain| directive):
%
\begin{center}
\begin{tabular}{l}
|\ifchilddoc|\\
|\providecommand{\version}{draft}|\\
|\||else|\\
|\providecommand{\version}{final}|\\
|\||fi|
\end{tabular}
\end{center}
%
The definition by |\providecommand| makes sure
that previous definitions are not overwritten.
Further statements |\providecommand{\version}{...}|
can thus be added before the above code to override it.

For the main file, one might add a line
(between |\childdocmain| and the above block)
%
\begin{center}
|%\ifchilddoc\||else\providecommand{\version}{draft}\||fi|
\end{center}
%
which can be uncommented to produce a draft version.
Likewise one can add a line to the very top of a child file
(above the |\childdocof{|\textit{main}|}| directive)
%
\begin{center}
|%\providecommand{\version}{final}|
\end{center}
%
which can be uncommented to produce the final version of this child document.

%%%%%%%%%%%%%%%%%%%%%%%%%%%%%%%%%%%%%%%%%%%%%%%%%%%%%%%%%%%%%%%%%%%%%%%%%%%%%%%%
\subsection{Forwarding}
\label{sec:forward}

Different versions of the main or child documents
using compilation flags as described in \secref{sec:flags}
can be (permanently) stored in different files
for convenient compilation, viewing and distribution.
To this end, the package defines a command
to pass on compilation to a different file:

%%%%%%%%%%%%%%%%%%%%%%%%%%%%%%%%%%%%%%%%
\DescribeMacro{\childdocforward}
The command |\childdocforward| redirects processing to
another source file:
%
\begin{center}
\begin{tabular}{l}
|\input{childdoc.def}|\\
|\childdocforward[|\textit{main}|]{|\textit{dest}|}|\\
\end{tabular}
\end{center}
%
The argument \textit{dest} is the destination file
(without extension).
It should be the main file or one of the child files.
Note that further \textsf{childdoc} directives
such as |\childdocof| and |\childdocforward|
in the indicated file will be processed in this form.
The optional argument \textit{main}
passes on directly to the main file \textit{main}
while pretending to compile the child \textit{dest}.
This form behaves as if \textit{dest}
issues |\childdocof{|\textit{main}|}| right away,
and no further \textsf{childdoc} directives will be processed.

%%%%%%%%%%%%%%%%%%%%%%%%%%%%%%%%%%%%%%%%
\DescribeMacro{\...prefix}
In the alternative form |\childdocforwardprefix|,
%
\begin{center}
\begin{tabular}{l}
|\input{childdoc.def}|\\
|\childdocforwardprefix[|\textit{main}|]{|\textit{prefix}|}{|\textit{dest}|}|
\end{tabular}
\end{center}
%
the destination file is determined by a pattern
depending on the current file:
To make this work, the current file must be called
`{\textit{prefix}\hspace{0.2em}\textit{suffix}}'
with \textit{prefix} matching precisely the argument.
Processing is then passed on to the file
`{\textit{dest}\hspace{0.2em}\textit{suffix}}'.
Surely, the same effect is achieved by
directly specifying the
argument `{\textit{dest}\hspace{0.2em}\textit{suffix}}'
in the first form.
However, that requires to set up a different file
for each child. With the alternative form of the command
all these files can have exactly the same content
which simplifies setting them up and maintaining them.

For example, the following file |draft.tex|
with a compilation flag |\version| as described in \secref{sec:flags}
compiles the main document as a draft:
%
\begin{center}
\begin{tabular}{l}
|\def\version{draft}|\\
|\input{childdoc.def}|\\
|\childdocforward{|\textit{main}|}|
\end{tabular}
\end{center}
%
Likewise, the following files |final|\textit{nn}|.tex|
compile the final version of the child document
|child|\textit{nn}|.tex|:
%
\begin{center}
\begin{tabular}{l}
|\def\version{final}|\\
|\input{childdoc.def}|\\
|\childdocforwardprefix{final}{child}|
\end{tabular}
\end{center}
%

Note that when several versions of a main file and/or of each child file
are to be generated, it may be convenient to set up a |Makefile| or
shell script to automatise the process.

%%%%%%%%%%%%%%%%%%%%%%%%%%%%%%%%%%%%%%%%%%%%%%%%%%%%%%%%%%%%%%%%%%%%%%%%%%%%%%%%
\subsection{Command Line Processing}
\label{sec:commandline}

The effect of redirection files can also be achieved by invoking
the \LaTeX{} compiler with a more elaborate command line.
Most conveniently this should be done as part
of a shell script or a |Makefile|.

When using \textsf{childdoc} in the main file, the following
command lines effectively perform a redirection
(note that depending on the shell being used,
backslashes may have to be doubled: `|\|' $\to$ `|\\|'):
%
\begin{center}
|... -jobname "|\textit{target}|" |\\|"|[\textit{flags}]%
|\input{childdoc.def}\childdocforward[|\textit{main}|]{|\textit{dest}|}"|
\end{center}
%
Here \textit{target} is the name of the output file,
\textit{main} is the name of the main file
and \textit{dest} is the name of the main or child file to be processed
(all filenames without extensions).
The optional argument \textit{main} can be omitted
if \textit{main} matches \textit{dest}.
Optionally, compilation \textit{flags} can be defined via |\def| commands.
This command line makes the \TeX{} engine believe
it is compiling the file \textit{target}
whose content is specified as the latter parameter.
The provided code then forwards the processing to
\textit{main} or \textit{dest} as described in \secref{sec:forward}.

%%%%%%%%%%%%%%%%%%%%%%%%%%%%%%%%%%%%%%%%%%%%%%%%%%%%%%%%%%%%%%%%%%%%%%%%%%%%%%%%
\subsection{Include by Input}
\label{sec:input}

Including child documents by |\include| has some restrictions by design.
Most notably, the content of a child document always occupies
its own set of pages; pages cannot be shared between child documents.
Usually, this behaviour makes perfect sense
because each child document contain an essential part of the document.
However, in some situations it may be desirable to compose
a document from a collection of parts
without having mandatory page breaks between then.
For this case, the package
provides a mechanism to include parts
by |\input| which can also be processed individually.
However, by construction this mechanism
requires manual handling of the content to be output.

%%%%%%%%%%%%%%%%%%%%%%%%%%%%%%%%%%%%%%%%
\DescribeMacro{\ifchilddocmanual}
The main file should be prepared as usual, see \secref{sec:include}.
However, the document body must make a distinction
between processing of an individual part and of the main document, e.g.:
%
\begin{center}
\begin{tabular}{l}
|\ifchilddocmanual|\\
|\input{\childdocname}|\\
|\||else|\\
\textit{document body with }|\input{|\textit{part}|}|\\
|\||fi|
\end{tabular}
\end{center}
%
The conditional |\ifchilddocmanual| is true whenever
a part to be included by |\input| is being compiled,
and the name of the part is stored in |\childdocname|.

%%%%%%%%%%%%%%%%%%%%%%%%%%%%%%%%%%%%%%%%
\DescribeMacro{\childdocby}
Each part to be included by |\input| should start with:
%
\begin{center}
\begin{tabular}{l}
|\input{childdoc.def}|\\
|\childdocby{|\textit{main}|}|\\
\end{tabular}
\end{center}
%
The directive |\childdocby| is similar to |\childdocof|
described in \secref{sec:include},
but the subsequent selection of content must be done manually.
To that end, both |\ifchilddoc| and |\ifchilddocmanual|
will be true upon processing of a part,
and the name of the part is stored in |\childdocname|.
Note that |\jobname| will be set to the filename of the current part
so that each part receives an individual |.aux| file
that does not interfere with the |.aux| file(s) of the main document.
This behaviour can be altered by the alternative form
|\childdocby[*]{|\textit{main}|}| (with a non-empty optional argument)
which uses the |.aux| file of the main document
by setting |\jobname| to \textit{main}.

%%%%%%%%%%%%%%%%%%%%%%%%%%%%%%%%%%%%%%%%%%%%%%%%%%%%%%%%%%%%%%%%%%%%%%%%%%%%%%%%
\subsection{Driver Development}
\label{sec:driver}

The \textsf{childdoc} mechanism can also be use for the development
of definition files such as \LaTeX{} styles or classes.
This case differs from the above setup with multiple parts
included by |\include| in that no |\includeonly| should be invoked.
This can be achieved by starting the include file
(before |\ProvidesPackage|) with:
%
\begin{center}
\begin{tabular}{l}
|\input{childdoc.def}|\\
|\childdocforward{|\textit{main}|}|\\
\end{tabular}
\end{center}
%
or alternatively with:
%
\begin{center}
\begin{tabular}{l}
|\input{childdoc.def}|\\
|\childdocby{|\textit{main}|}|\\
\end{tabular}
\end{center}
%
Both forms have slightly different effects as described above.
The main file is prepared as usual, see \secref{sec:include}.

%%%%%%%%%%%%%%%%%%%%%%%%%%%%%%%%%%%%%%%%%%%%%%%%%%%%%%%%%%%%%%%%%%%%%%%%%%%%%%%%
\subsection{Legacy Detection}
\label{sec:detection}

The directive |\childdocmain| in the main file can detect
whether the complete document or merely a child is to be compiled
even without using the directive |\childdocof|.
This method is deprecated because it is less robust
and there is no compelling reason to use it;
it is merely provided for backward compatibility
and it may be removed in future versions.

If the detection mechanism is to be used,
it is mandatory to correctly specify
the filename of the main file as the argument of |\childdocmain|:
%
\begin{center}
\begin{tabular}{l}
|\input{childdoc.def}|\\
|\childdocmain{|\textit{main}|}|\\
\end{tabular}
\end{center}
%
If |\jobname| does not match the argument \textit{main} of |\childdocmain|,
it is assumed that |\jobname| points to the child file to be compiled.
When using |\childdocmain| with the main file specified as argument,
it suffices to start a child file
with just |\input{|\textit{main}|}|
without loading of the package and using |\childdocof|.
If instead all processing is done
with the appropriate \textsf{childdoc} directives,
the argument of \textit{main} of |\childdocmain| can be empty.

An alternative version of the command line processing described
in \secref{sec:commandline} using the detection mechanism reads:
%
\begin{center}
|... -jobname "|\textit{target}|" "|[\textit{flags}]%
[|\def\jobname{|\textit{dest}|}|]|\input{|\textit{main}|}"|
\end{center}

%%%%%%%%%%%%%%%%%%%%%%%%%%%%%%%%%%%%%%%%%%%%%%%%%%%%%%%%%%%%%%%%%%%%%%%%%%%%%%%%
\subsection{Manual Code}
\label{sec:manual}

In case one cannot be certain whether the definitions file |childdoc.def|
is installed on the target \TeX{} distribution
and one prefers not to ship it,
it is conceivable to paste a few relevant commands into the sources.

To that end, drop all statements |\input{childdoc.def}|
and perform the replacements as outlined below.
Instead of |\childdocmain{|\textit{main}|}| add the following code
to the top of the main file:
%
\begin{center}
\begin{tabular}{l}
|\||ifdefined\childdocname\endinput\||fi\newif\ifchilddoc|\\
|\edef\childdocname{\scantokens\expandafter{\jobname\noexpand}}|\\
|\def\childdocmain{|\textit{main}|}\||ifx\childdocmain\childdocname\||else|\\
|\childdoctrue\includeonly{\childdocname}\let\jobname\childdocmain\||fi|\\
\end{tabular}
\end{center}
%
Instead of |\childdocof{|\textit{main}|}| just include the main file
at the top of each child file:
%
\begin{center}
|\input{|\textit{main}|}|
\end{center}
%
A simple redirection |\childdocforward{|\textit{dest}|}| is achieved by:
%
\begin{center}
|\def\jobname{|\textit{dest}|}\input{\jobname}|
\end{center}
%
The redirection with prefix
|\childdocforwardprefix[|\textit{prefix}|]{|\textit{dest}|}|
is accomplished by:
%
\begin{center}
\begin{tabular}{l}
|{\edef\jobname{\scantokens\expandafter{\jobname\noexpand}}|\\
|\def\redirectjob |\textit{prefix}|#1~~~{\gdef\jobname{|\textit{dest}|#1}}|\\
|\expandafter\redirectjob\jobname~~~}\input{\jobname}|
\end{tabular}
\end{center}

In an alternative approach,
child documents can be compiled by a specific command line
without additional code or specific definitions:
%
\begin{center}
|... -jobname "|\textit{target}|" "|[\textit{flags}]%
|\includeonly{|\textit{dest}|}\input{|\textit{main}|}"|
\end{center}
%

%%%%%%%%%%%%%%%%%%%%%%%%%%%%%%%%%%%%%%%%%%%%%%%%%%%%%%%%%%%%%%%%%%%%%%%%%%%%%%%%
%%%%%%%%%%%%%%%%%%%%%%%%%%%%%%%%%%%%%%%%%%%%%%%%%%%%%%%%%%%%%%%%%%%%%%%%%%%%%%%%
\section{Information}

%%%%%%%%%%%%%%%%%%%%%%%%%%%%%%%%%%%%%%%%%%%%%%%%%%%%%%%%%%%%%%%%%%%%%%%%%%%%%%%%
\subsection{Copyright}

Copyright \copyright{} 2017--2018 Niklas Beisert

This work may be distributed and/or modified under the
conditions of the \LaTeX{} Project Public License, either version 1.3
of this license or (at your option) any later version.
The latest version of this license is in
  \url{http://www.latex-project.org/lppl.txt}
and version 1.3 or later is part of all distributions of \LaTeX{}
version 2005/12/01 or later.

This work has the LPPL maintenance status `maintained'.

The Current Maintainer of this work is Niklas Beisert.

This work consists of the files |README.txt|, |childdoc.ins| and |childdoc.dtx|
as well as the derived files |childdoc.def|, |cdocsamp.tex|
with |cdocsch1.tex|, |cdocsch2.tex|, |cdocspt3.tex|, |cdocspt4.tex|,
|cdocsdrf.tex|, |cdocsfn1.tex|, |cdocsfn2.tex|
as well as |childdoc.pdf|.

%%%%%%%%%%%%%%%%%%%%%%%%%%%%%%%%%%%%%%%%%%%%%%%%%%%%%%%%%%%%%%%%%%%%%%%%%%%%%%%%
\subsection{Files and Installation}

The package consists of the files:
%
\begin{center}
\begin{tabular}{ll}
    |README.txt|   & readme file \\
    |childdoc.ins| & installation file \\
    |childdoc.dtx| & source file \\
    |childdoc.def| & definition file \\
    |cdocsamp.tex| & sample main file \\
    |cdocsch1.tex| & sample include file \\
    |cdocsch2.tex| & sample include file \\
    |cdocspt3.tex| & sample part file \\
    |cdocspt4.tex| & sample part file \\
    |cdocsdrf.tex| & sample redirection file \\
    |cdocsfn1.tex| & sample redirection file \\
    |cdocsfn2.tex| & sample redirection file \\
    |childdoc.pdf| & manual
\end{tabular}
\end{center}
%
The distribution consists of the files
|README.txt|, |childdoc.ins| and |childdoc.dtx|.
%
\begin{itemize}
\item
Run (pdf)\LaTeX{} on |childdoc.dtx|
to compile the manual |childdoc.pdf| (this file).
\item
Run \LaTeX{} on |childdoc.ins| to create the definitions file |childdoc.def|
and the sample |cdocsamp.tex| with include files
|cdocsch1.tex|, |cdocsch2.tex|, |cdocspt3.tex|, |cdocspt4.tex|,
|cdocsdrf.tex|, |cdocsfn1.tex|, |cdocsfn2.tex|.
Then copy the file |childdoc.def| to an appropriate directory of your \LaTeX{}
distribution, e.g.\ \textit{texmf-root}|/tex/latex/childdoc|.
\end{itemize}

%%%%%%%%%%%%%%%%%%%%%%%%%%%%%%%%%%%%%%%%%%%%%%%%%%%%%%%%%%%%%%%%%%%%%%%%%%%%%%%%
\subsection{Related CTAN Packages}

There are several other packages which offer a similar functionality:
%
\begin{itemize}
\item
The packages
\href{http://ctan.org/pkg/docmute}{\textsf{docmute}},
\href{http://ctan.org/pkg/includex}{\textsf{includex}} and
\href{http://ctan.org/pkg/standalone}{\textsf{standalone}}
provide commands to include only the document body of
a child file thus allowing both files to be compiled individually.
\item
The packages \href{http://ctan.org/pkg/subdocs}{\textsf{subdocs}}
and \href{http://ctan.org/pkg/subfiles}{\textsf{subfiles}}
provide structures in which the main and child documents can be
encapsulated and allowing them to be compiled individually.
The inclusion mechanism is different from the conventional |\include|.
\item
The package \href{http://ctan.org/pkg/combine}{\textsf{combine}}
is an elaborate solution to combine several documents into one.
\end{itemize}
%
See also the CTAN topic \href{http://ctan.org/topic/subdocs}{\textsf{subdocs}}
for further related packages.
The present package differs from the above solutions in that
a document structure constructed with the conventional |\include| mechanism
just needs two extra commands at the top of every file
such that all constituent files can be compiled individually.

%%%%%%%%%%%%%%%%%%%%%%%%%%%%%%%%%%%%%%%%%%%%%%%%%%%%%%%%%%%%%%%%%%%%%%%%%%%%%%%%
%\subsection{Feature Suggestions}
%
%The following is a list of features which may be useful for future
%versions of this package:
%%
%\begin{itemize}
%\item
%\ldots
%\end{itemize}

%%%%%%%%%%%%%%%%%%%%%%%%%%%%%%%%%%%%%%%%%%%%%%%%%%%%%%%%%%%%%%%%%%%%%%%%%%%%%%%%
\subsection{Revision History}

%%%%%%%%%%%%%%%%%%%%%%%%%%%%%%%%%%%%%%%%
\paragraph{v2.0:} 2018/12/30

\begin{itemize}
\item
immediate forward processing
\item
added |\childdocby| mechanism
\item
manual restructured
\end{itemize}

%%%%%%%%%%%%%%%%%%%%%%%%%%%%%%%%%%%%%%%%
\paragraph{v1.6:} 2018/01/17

\begin{itemize}
\item
application for development of include files
\item
corrections to manual
\end{itemize}

%%%%%%%%%%%%%%%%%%%%%%%%%%%%%%%%%%%%%%%%
\paragraph{v1.5:} 2017/05/21

\begin{itemize}
\item
more complete structuring introduced
\item
|\childdocof| introduced
\item
|\childdoc| renamed to |\childdocmain|
\item
|\childredirect| renamed to |\childdocforward| and |\childdocforwardprefix|
and functionality expanded
\end{itemize}

%%%%%%%%%%%%%%%%%%%%%%%%%%%%%%%%%%%%%%%%
\paragraph{v1.0:} 2017/04/27

\begin{itemize}
\item
manual and install package
\item
first version published on CTAN
\end{itemize}

%%%%%%%%%%%%%%%%%%%%%%%%%%%%%%%%%%%%%%%%
\paragraph{v0.6:} 2017/04/26

\begin{itemize}
\item
redirection mechanism added
\end{itemize}

%%%%%%%%%%%%%%%%%%%%%%%%%%%%%%%%%%%%%%%%
\paragraph{v0.5:} 2017/04/26

\begin{itemize}
\item
functionality in definition file
\end{itemize}


%%%%%%%%%%%%%%%%%%%%%%%%%%%%%%%%%%%%%%%%%%%%%%%%%%%%%%%%%%%%%%%%%%%%%%%%%%%%%%%%
%%%%%%%%%%%%%%%%%%%%%%%%%%%%%%%%%%%%%%%%%%%%%%%%%%%%%%%%%%%%%%%%%%%%%%%%%%%%%%%%
%%%%%%%%%%%%%%%%%%%%%%%%%%%%%%%%%%%%%%%%%%%%%%%%%%%%%%%%%%%%%%%%%%%%%%%%%%%%%%%%
\appendix

\settowidth\MacroIndent{\rmfamily\scriptsize 000\ }

 \DocInput{childdoc.dtx}

\end{document}
%</driver>
% \fi
%
% %%%%%%%%%%%%%%%%%%%%%%%%%%%%%%%%%%%%%%%%%%%%%%%%%%%%%%%%%%%%%%%%%%%%%%%%%%%%%%
% %%%%%%%%%%%%%%%%%%%%%%%%%%%%%%%%%%%%%%%%%%%%%%%%%%%%%%%%%%%%%%%%%%%%%%%%%%%%%%
% \section{Sample}
%\iffalse
%<*samplemain>
%\fi
%
% The following presents a sample document
% with two chapters, two parts, a title page,
% a compile flag as well as three forwarding files to set the flag.
% It consists of eight |.tex| files:
% \begin{center}
% \begin{tabular}{ll}
% |cdocsamp.tex|&main file\\
% |cdocsch1.tex|&include file for chapter 1\\
% |cdocsch2.tex|&include file for chapter 2\\
% |cdocspt3.tex|&include file for part 3\\
% |cdocspt4.tex|&include file for part 4\\
% |cdocsdrf.tex|&forwarding file for main file in draft mode\\
% |cdocsfi1.tex|&forwarding file for final version of chapter 1\\
% |cdocsfi2.tex|&forwarding file for final version of chapter 2\\
% \end{tabular}
% \end{center}
% Each of the eight files can be compiled directly by the \LaTeX{} compiler.
%
% %%%%%%%%%%%%%%%%%%%%%%%%%%%%%%%%%%%%%%
% \paragraph{Main File.}
%
% The main file is called |cdocsamp.tex|.
%
% Load the \textsf{childdoc} definitions and
% declare the filename for the main document:
%    \begin{macrocode}
\input{childdoc.def}
\childdocmain{}
%    \end{macrocode}

% Optional override for |\version| flag:
%    \begin{macrocode}
%%\ifchilddoc\else\providecommand{\version}{draft}\fi
%    \end{macrocode}

% Define the default values for the |\version| flag
% (|final| for the main file and |draft| for childs):
%    \begin{macrocode}
\ifchilddoc
\providecommand{\version}{draft}
\else
\providecommand{\version}{final}
\fi
%    \end{macrocode}

% Load the standard document class:
%    \begin{macrocode}
\documentclass[12pt]{article}
%    \end{macrocode}

% Start the document body:
%    \begin{macrocode}
\begin{document}
%    \end{macrocode}

% Declare a title page.
% Print title, part of document being processed and version flag:
%    \begin{macrocode}
\addtocounter{page}{-1}
\begin{center}
{\LARGE\bfseries{}childdoc example\par}
\vspace{1cm}
\ifchilddoc
\ifchilddocmanual part\else chapter\fi:
`\childdocname' of `\childdocjob'\par
\else
main document: `\childdocjob'\par
\fi
version: \version\par
\end{center}
\newpage
%    \end{macrocode}

% Manually include selected file,
% otherwise process as usual:
%    \begin{macrocode}
\ifchilddocmanual
\section*{part `\childdocname'}
\input{\childdocname}
\else
%    \end{macrocode}

% Include the two chapters:
%    \begin{macrocode}
\include{cdocsch1}
\include{cdocsch2}
%    \end{macrocode}

% Include the two parts unless only chapters should be displayed:
%    \begin{macrocode}
\ifchilddoc\else
\section{part three}
\input{cdocspt3}
\section{part four}
\input{cdocspt4}
\fi
%    \end{macrocode}

% Process as usual until here:
%    \begin{macrocode}
\fi
%    \end{macrocode}

% End of document body:
%    \begin{macrocode}
\end{document}
%    \end{macrocode}
%\iffalse
%</samplemain>
%\fi
%
% %%%%%%%%%%%%%%%%%%%%%%%%%%%%%%%%%%%%%%
% \paragraph{Chapter Include Files.}
%
% The include files are called |cdocsch1.tex| and |cdocsch2.tex|.
%
%\iffalse
%<*samplechap1|samplechap2>
%\fi

% Optional override for |\version| flag:
%    \begin{macrocode}
%%\providecommand{\version}{final}
%    \end{macrocode}

% Include the main document:
%    \begin{macrocode}
\input{childdoc.def}
\childdocof{cdocsamp}
%    \end{macrocode}

%\iffalse
%</samplechap1|samplechap2>
%\fi
%
%\iffalse
%<*samplechap1>
%\fi
% Some text for chapter 1:
%    \begin{macrocode}
\section{one}
some text in chapter one
%    \end{macrocode}

%\iffalse
%</samplechap1>
%\fi
% Some text for chapter 2:
%\iffalse
%<*samplechap2>
%\fi
%    \begin{macrocode}
\section{two}
more text in chapter two
%    \end{macrocode}

%\iffalse
%</samplechap2>
%\fi
%
% %%%%%%%%%%%%%%%%%%%%%%%%%%%%%%%%%%%%%%
% \paragraph{Part Include Files.}
%
% The include files are called |cdocspt3.tex| and |cdocspt4.tex|.
%
%\iffalse
%<*samplepart3|samplepart4>
%\fi

% Optional override for |\version| flag:
%    \begin{macrocode}
%%\providecommand{\version}{final}
%    \end{macrocode}

% Include the main document:
%    \begin{macrocode}
\input{childdoc.def}
\childdocby{cdocsamp}
%    \end{macrocode}

%\iffalse
%</samplepart3|samplepart4>
%\fi
%
%\iffalse
%<*samplepart3>
%\fi
% Some text for part 3:
%    \begin{macrocode}
some text in part three
%    \end{macrocode}

%\iffalse
%</samplepart3>
%\fi
% Some text for part 4:
%\iffalse
%<*samplepart4>
%\fi
%    \begin{macrocode}
more text in part four
%    \end{macrocode}

%\iffalse
%</samplepart4>
%\fi
%
% %%%%%%%%%%%%%%%%%%%%%%%%%%%%%%%%%%%%%%
% \paragraph{Forwarding for a Complete Draft.}
%
% The following forwarding file |cdocsdrf.tex|
% compiles the main document in draft mode:
%\iffalse
%<*sampledraft>
%\fi
%    \begin{macrocode}
\def\version{draft}
\input{childdoc.def}
\childdocforward{cdocsamp}
%    \end{macrocode}

%\iffalse
%</sampledraft>
%\fi
%
% %%%%%%%%%%%%%%%%%%%%%%%%%%%%%%%%%%%%%%
% \paragraph{Forwarding for Final Version of the Chapters.}
%
% The following forwarding files |cdocsfn1.tex| and |cdocsfn2.tex|
% (with identical content)
% compile the final versions of the child documents
% |cdocsch1.tex| and |cdocsch2.tex|, respectively:
%\iffalse
%<*samplefinal>
%\fi
%    \begin{macrocode}
\def\version{final}
\input{childdoc.def}
\childdocforwardprefix[cdocsamp]{cdocsfn}{cdocsch}
%    \end{macrocode}

%\iffalse
%</samplefinal>
%\fi
%
% %%%%%%%%%%%%%%%%%%%%%%%%%%%%%%%%%%%%%%
% \paragraph{Command Line Processing.}
%
% The following three command lines generate the output files
% |cdocscld|, |cdocscl1| and |cdocscl2|
% which should be identical to
% |cdocsdrf|, |cdocsch1| and |cdocsfn2|, respectively:
% \begin{center}
% \begin{tabular}{l}
% |latex -jobname cdocscld \|\\
% |  "\def\version{draft}\input{childdoc.def}\childdocforward{cdocsamp}"|\\
% |latex -jobname cdocscl1 \|\\
% |  "\input{childdoc.def}\childdocforward[cdocsamp]{cdocsch1}"|\\
% |latex -jobname cdocscl2 \|\\
% |  "\def\version{final}\input{childdoc.def}\childdocforward{cdocsch2}"|
% \end{tabular}
% \end{center}
% Note that the trailing backslash on each first line
% merely continues the input to the second line
% (for convenient cut ant paste).
% Furthermore, the command |latex| can be replaced by any
% of its alternative versions such as |pdflatex|.
%
% %%%%%%%%%%%%%%%%%%%%%%%%%%%%%%%%%%%%%%%%%%%%%%%%%%%%%%%%%%%%%%%%%%%%%%%%%%%%%%
% %%%%%%%%%%%%%%%%%%%%%%%%%%%%%%%%%%%%%%%%%%%%%%%%%%%%%%%%%%%%%%%%%%%%%%%%%%%%%%
% \section{Implementation}
%\iffalse
%<*package>
%\fi
%
% This section describes the definitions file |childdoc.def|.

% The definitions cannot be loaded using |\usepackage| or |\RequirePackage|
% which has a mechanism to prevent loading a style file more than once.
% When loading the definitions by means of |\input|
% multiple instances have to be prevented manually:
%\iffalse
%This code needs to be before the `\ProvidesFile' directive
%which is defined at the beginning of this file.
%Therefore it is also placed there and commented out here.
%</package>
%<*discard>
%\fi
%    \begin{macrocode}
\ifdefined\childdocmain\endinput\fi
%    \end{macrocode}
%\iffalse
%</discard>
%<*package>
%\fi
%
% \macro{\ifchilddoc}
% \macro{\ifchilddocmanual}
% The conditional |\ifchilddoc| tells whether a
% child (true) or main (false) document is being compiled.
% The conditional |\ifchilddocmanual| tells whether
% the |\includeonly| mechanism is used (false) or
% the selection of child files must be performed manually (true).
% The definitions initialise to false:
%    \begin{macrocode}
\newif\ifchilddoc
\newif\ifchilddocmanual
%    \end{macrocode}

% \macro{\childdocname}
% \macro{\childdocjob}
% The macro |\childdocname| stores the name of the main document
% to be compiled. The macro |\childdocjob| stores the name of
% the document on which the \LaTeX{} compiler was originally invoked.
% The content of |\jobname| cannot be compared
% to filenames specified in the source due to different catcodes.
% The following code rescans |\jobname|, stores the result
% in |\childdocname| and saves a copy in |\childdocjob|:
%    \begin{macrocode}
\edef\childdocname{\scantokens\expandafter{\jobname\noexpand}}
\let\childdocjob\childdocname
%    \end{macrocode}

% \macro{\childdocdisable}
% The macro |\childdocdisable| prevents the main file
% from being processed more than once.
% At this stage, the main document command |\childdocmain|
% is assumed to be called once again where it should do nothing.
% Any subsequent call to it should prevent
% a secondary processing of the main document
% It overwrites the forwarding commands
% |\childdocof| and |\childdocforward|
% with empty macros to prevent further inclusions of the main document:
%    \begin{macrocode}
\newcommand{\childdocdisable}
{
  \renewcommand{\childdocmain}[1]{\renewcommand{\childdocmain}[1]{\endinput}}
  \renewcommand{\childdocof}[1]{}
  \renewcommand{\childdocby}[2][]{}
  \renewcommand{\childdocforward}[2][]{}
  \renewcommand{\childdocdisable}{}
}
%    \end{macrocode}

% \macro{\childdocmain}
% The macro |\childdocmain| is to be called at the top of the main file
% with nothing or the main filename (without extension) as argument.
% First, it breaks loops.
% If the argument is not empty and does not match |\childdocname|
% (which is set by the first inclusion of |childdoc.def|),
% |\ifchilddoc| is set to true, |\includeonly| is applied to the child file
% and |\jobname| is set to the main file
% (for proper handling of |.aux| files):
%    \begin{macrocode}
\newcommand{\childdocmain}[1]
{
  \childdocdisable\childdocmain{}
  \if?#1?\else
    \begingroup
      \def\childdoctmp{#1}
      \ifx\childdoctmp\childdocname
        \def\childdoctmp{}
      \else
        \def\childdoctmp
        {
          \childdoctrue
          \includeonly{\childdocname}
          \def\childdocjob{#1}
          \def\jobname{#1}
        }
      \fi
      \expandafter
    \endgroup
    \childdoctmp
  \fi
}
%    \end{macrocode}

% \macro{\childdocof}
% The command |\childdocof| redirects
% compilation to the main file |#1|.
%    \begin{macrocode}
\newcommand{\childdocof}[1]
{
  \childdocdisable
  \childdoctrue
  \includeonly{\childdocname}
  \def\jobname{#1}
  \def\childdocjob{#1}
  \input{#1}
}
%    \end{macrocode}

% \macro{\childdocby}
% The command |\childdocby| ....
%    \begin{macrocode}
\newcommand{\childdocby}[2][]
{
  \childdocdisable
  \childdoctrue
  \childdocmanualtrue
  \if?#1?\else
    \def\jobname{#2}
  \fi
  \def\childdocjob{#2}
  \input{#2}
  \endinput
}
%    \end{macrocode}

% \macro{\childdocforward}
% The command |\childdocforward| redirects
% compilation to the main file or
% (if the optional argument is given) a child file.
% Parameters are set as if the main file
% or a child file starting with |\childdocof| was compiled.
% Then compilation is handed over to the main file:
%    \begin{macrocode}
\newcommand{\childdocforward}[2][]
{
  \begingroup
    \if?#1?
      \def\childdoctmp
      {
        \def\childdocname{#2}
        \def\childdocjob{#2}
        \def\jobname{#2}
        \input{#2}
        \endinput
      }
    \else
      \def\childdoctmp
      {
        \childdocdisable
        \def\childdocname{#2}
        \childdoctrue
        \includeonly{#2}
        \def\childdocjob{#1}
        \def\jobname{#1}
        \input{#1}
        \endinput
      }
    \fi
    \expandafter
  \endgroup
  \childdoctmp
}
%    \end{macrocode}

% \macro{\childdocforwardprefix}
% The command |\childdocforwardprefix| redirects
% compilation to the main or a child file by means of a pattern.
% The prefix |#1| in the current filename is replaced by |#2|
% and the suffix of the current filename is kept
% (it is assumed that the filename does not contain the substring `|~~~|'
% which is used as a delimiter).
% Compilation is handed over to the new file by |\childdocforward|:
%    \begin{macrocode}
\newcommand{\childdocforwardprefix}[3][]
{
  \begingroup
    \def\childdocextract #2##1~~~{\def\childdoctmp{\childdocforward[#1]{#3##1}}}
    \expandafter\childdocextract\childdocname~~~
    \expandafter
  \endgroup
  \childdoctmp
}
%    \end{macrocode}

% \macro{\childdoc}
% The deprecated macro |\childdoc| is a legacy version of |\childdocmain|:
%    \begin{macrocode}
\newcommand{\childdoc}{\childdocmain}
%    \end{macrocode}

% \macro{\childdocredirect}
% The deprecated macro |\childdocredirect| is a legacy version
% of |\childdocforward| and |\childdocforwardprefix|:
%    \begin{macrocode}
\newcommand{\childdocredirect}[2][]
{
  \begingroup
    \if?#1?
      \def\childdoctmp{\childdocforward{#2}}
    \else
      \def\childdoctmp{\childdocforwardprefix{#1}{#2}}
    \fi
    \expandafter
  \endgroup
  \childdoctmp
}
%    \end{macrocode}

%\iffalse
%</package>
%\fi
%
\endinput

\childdocforward{cdocsamp}
%    \end{macrocode}

%\iffalse
%</sampledraft>
%\fi
%
% %%%%%%%%%%%%%%%%%%%%%%%%%%%%%%%%%%%%%%
% \paragraph{Forwarding for Final Version of the Chapters.}
%
% The following forwarding files |cdocsfn1.tex| and |cdocsfn2.tex|
% (with identical content)
% compile the final versions of the child documents
% |cdocsch1.tex| and |cdocsch2.tex|, respectively:
%\iffalse
%<*samplefinal>
%\fi
%    \begin{macrocode}
\def\version{final}
% \iffalse
%
% childdoc.dtx Copyright (C) 2017-2018 Niklas Beisert
%
% This work may be distributed and/or modified under the
% conditions of the LaTeX Project Public License, either version 1.3
% of this license or (at your option) any later version.
% The latest version of this license is in
%   http://www.latex-project.org/lppl.txt
% and version 1.3 or later is part of all distributions of LaTeX
% version 2005/12/01 or later.
%
% This work has the LPPL maintenance status `maintained'.
%
% The Current Maintainer of this work is Niklas Beisert.
%
% This work consists of the files childdoc.dtx and childdoc.ins
% and the derived files childdoc.def and cdocsamp.tex with
% cdocsch1.tex, cdocsch2.tex, cdocsdrf.tex, cdocsfn1.tex, cdocsfn2.tex.
%
%<package>\ifdefined\childdocmain\endinput\fi
%<package>\ProvidesFile{childdoc.def}[2018/12/30 v2.0 child document driver]
%<samplemain>\ProvidesFile{cdocsamp.tex}[2018/12/30 v2.0 sample for childdoc]
%<*driver>
%\ProvidesFile{childdoc.drv}[2018/12/30 v2.0 childdoc reference manual file]
\PassOptionsToClass{10pt,a4paper}{article}
\documentclass{ltxdoc}

\usepackage[margin=35mm]{geometry}
\usepackage{hyperref}
\usepackage{hyperxmp}
\usepackage[usenames]{color}

\hypersetup{colorlinks=true}
\hypersetup{pdfstartview=FitH}
\hypersetup{pdfpagemode=UseNone}
\hypersetup{pdfsource={}}
\hypersetup{pdflang={en-UK}}
\hypersetup{pdfcopyright={Copyright 2017-2018 Niklas Beisert.
  This work may be distributed and/or modified under the
  conditions of the LaTeX Project Public License, either version 1.3
  of this license or (at your option) any later version.}}
\hypersetup{pdflicenseurl={http://www.latex-project.org/lppl.txt}}
\hypersetup{pdfcontactaddress={ETH Zurich, ITP, HIT K,
  Wolfgang-Pauli-Strasse 27}}
\hypersetup{pdfcontactpostcode={8093}}
\hypersetup{pdfcontactcity={Zurich}}
\hypersetup{pdfcontactcountry={Switzerland}}
\hypersetup{pdfcontactemail={nbeisert@itp.phys.ethz.ch}}
\hypersetup{pdfcontacturl={http://people.phys.ethz.ch/\xmptilde nbeisert/}}

\newcommand{\secref}[1]{\hyperref[#1]{section \ref*{#1}}}

\parskip1ex
\parindent0pt
\let\olditemize\itemize
\def\itemize{\olditemize\parskip0pt}

\begin{document}

\title{The \textsf{childdoc} Package}
\hypersetup{pdftitle={The childdoc Package}}
\author{Niklas Beisert\\[2ex]
  Institut f\"ur Theoretische Physik\\
  Eidgen\"ossische Technische Hochschule Z\"urich\\
  Wolfgang-Pauli-Strasse 27, 8093 Z\"urich, Switzerland\\[1ex]
  \href{mailto:nbeisert@itp.phys.ethz.ch}
  {\texttt{nbeisert@itp.phys.ethz.ch}}}
\hypersetup{pdfauthor={Niklas Beisert}}
\hypersetup{pdfsubject={Manual for the LaTeX2e Package childdoc}}
\date{30 December 2018, \textsf{v2.0}}
\maketitle

\begin{abstract}\noindent
\textsf{childdoc} is a \LaTeXe{} package
that enables the direct compilation
of document sections included by |\include|
to individual files.
\end{abstract}

\begingroup
\parskip0ex
\tableofcontents
\endgroup

%%%%%%%%%%%%%%%%%%%%%%%%%%%%%%%%%%%%%%%%%%%%%%%%%%%%%%%%%%%%%%%%%%%%%%%%%%%%%%%%
%%%%%%%%%%%%%%%%%%%%%%%%%%%%%%%%%%%%%%%%%%%%%%%%%%%%%%%%%%%%%%%%%%%%%%%%%%%%%%%%
\section{Introduction}

\LaTeX{} provides a mechanism to structure a large document (such as a book)
into a main file and several child files (containing the chapters)
using the |\include| command.
This mechanism is beneficial for documents
which span hundreds of pages in order to
make the source file(s) more manageable.
Moreover, compilation can be restricted to
selected child files by means of the |\includeonly| command.
The latter feature can be used to reduce the compilation time while editing
(this was significantly more useful in the earlier days of \LaTeX{})
or to generate a smaller document which is easier to navigate.
Another application of |\includeonly| is to generate
documents consisting of selected parts of the complete document.

However, there are a few drawbacks of the plain |\include| mechanism:
\begin{itemize}
\item
The child files cannot be compiled on their own,
they can only be compiled via the main file.
A naive editing environment
(such as a text editor with an option
to have the current file processed by \LaTeX)
may require one to switch to the main file before compiling;
attempting to compile the child file produces errors.
\item
The main file must be modified (each time)
to adjust the |\includeonly| command
to the present needs. This easily leaves the main file in a messy state.
\item
The generated document will always carry the filename
of the main document. This is inconvenient if
several child files are to be compiled and
to be kept for distribution.
\end{itemize}

The present package provides a simple interface
to make child files individually compilable by \LaTeX{}.
Compiling a child file then has the same effect as compiling
the main file with an |\includeonly| command
to select the appropriate child.
Moreover the generated document will carry the name of the child
rather than the main file.
This resolves all three above issues.

This feature is meant to make the editing of books,
thesis documents and lecture notes somewhat more convenient.
However, the package can also be used efficiently for
composing a series of documents (such as exercise sheets)
which are typically distributed individually.
It then assists the author in generating the individual documents
(potentially in different versions)
as well as a document containing the collected series.
Another application is in developing style files
or other kinds of included material
where compilation of the style file could redirect
to a sample or test file.

%%%%%%%%%%%%%%%%%%%%%%%%%%%%%%%%%%%%%%%%%%%%%%%%%%%%%%%%%%%%%%%%%%%%%%%%%%%%%%%%
%%%%%%%%%%%%%%%%%%%%%%%%%%%%%%%%%%%%%%%%%%%%%%%%%%%%%%%%%%%%%%%%%%%%%%%%%%%%%%%%
\section{Usage}

First of all, the package \textsf{childdoc} is \emph{not} a standard
\LaTeXe{} |.sty| style file! Therefore it needs to be invoked in
a non-standard way.

%%%%%%%%%%%%%%%%%%%%%%%%%%%%%%%%%%%%%%%%%%%%%%%%%%%%%%%%%%%%%%%%%%%%%%%%%%%%%%%%
\subsection{Included Files}
\label{sec:include}

%%%%%%%%%%%%%%%%%%%%%%%%%%%%%%%%%%%%%%%%
\DescribeMacro{\childdocmain}
To use the package, add the commands
\begin{center}
\begin{tabular}{l}
|\input{childdoc.def}|\\
|\childdocmain{}|\\
\end{tabular}
\end{center}
at the very top of the main \LaTeX{} file,
in particular \emph{before} the |\documentclass| statement!
The argument of |\childdocmain| should be left empty
(but it must be present).

%%%%%%%%%%%%%%%%%%%%%%%%%%%%%%%%%%%%%%%%
\DescribeMacro{\childdocof}
Furthermore, add the commands
\begin{center}
\begin{tabular}{l}
|\input{childdoc.def}|\\
|\childdocof{|\textit{main}|}|\\
\end{tabular}
\end{center}
at the top of every child file \textit{child}
which is included by |\include{|\textit{child}|}|
from within the main file
(or at least for those files to be compiled individually).
The argument \textit{main} must be the filename of the main file.

There are a couple of
considerations in setting up the main and child documents:

%%%%%%%%%%%%%%%%%%%%%%%%%%%%%%%%%%%%%%%%
\paragraph{Restrictions.}

Please note the following restrictions:
\begin{itemize}
\item
|\childdocmain| must be called with one argument \textit{main}
to ensure compatibility with earlier version of the package.
It must either be empty (|\childdocmain{}|)
or precisely match the filename of the main file in which it is specified.
See \secref{sec:detection} for further information.
\item
The filename \textit{main} must be specified without the |.tex| extension.
\item
The filename \textit{main} is case sensitive
(even in case-insensitive file systems)
due to internal string comparison.
\item
The argument \textit{main} should be fully expanded, it cannot be a macro.
\item
Subdirectories and special characters should be avoided in filenames.
\item
The command |\childdocmain{|\textit{main}|}| must be followed by a whitespace.
It should not be followed immediately by another command
or by a comment mark `|%|'.
This is because the \TeX{} parser reads the token immediately following
the argument of |\childdocmain| and puts it
at the beginning of every child section;
however, a white\-space is ignored.
\end{itemize}

%%%%%%%%%%%%%%%%%%%%%%%%%%%%%%%%%%%%%%%%
\paragraph{Content of Main File.}

It is advisable to place all content in the child files included by |\include|.
Any output contained in the main file will appear in all child documents
unless suppressed manually;
it cannot be suppressed automatically by the |\includeonly| directive
and thus should normally be avoided.
A method to include some content in the main file
by means of conditional processing is described in \secref{sec:conditional}.

%%%%%%%%%%%%%%%%%%%%%%%%%%%%%%%%%%%%%%%%
\paragraph{Page Numbering.}

When only a part of the document is compiled,
the appropriate numbering of pages
(as well as other status parameters)
is determined from the |.aux| files.
The latter contain information from previous passes.
However this information needs to propagate through
all intermediate child documents.
Therefore the page numbering in child documents may well
be inconsistent until the complete document is compiled at least once.

A useful (if unconventional) way to always ensure a consistent
page numbering is to restart the numbering in each child document
and denote the pages by `\textit{child}|.|\textit{page}'
where \textit{child} represents the chapter/section number of the child file.
This can be achieved by the command
|\numberwithin{page}{|\textit{child}|}|
of the \textsf{amsmath} package
where \textit{child} can be |chapter| or |section|
depending on the chosen structuring.
Alternatively, one can modify the macro |\thepage| appropriately
and reset the counter |page| at the start of each child file.

%%%%%%%%%%%%%%%%%%%%%%%%%%%%%%%%%%%%%%%%%%%%%%%%%%%%%%%%%%%%%%%%%%%%%%%%%%%%%%%%
\subsection{Conditional Processing}
\label{sec:conditional}

The package provides a mechanism to compile different versions
of a document. To customise the versions further some conditional processing
can come in handy to distinguish which version is being compiled.
The package provides two macros to describe the compilation context:

%%%%%%%%%%%%%%%%%%%%%%%%%%%%%%%%%%%%%%%%
\DescribeMacro{\ifchilddoc}
The conditional |\ifchilddoc| distinguishes between the compilation of
child documents and the main document:
%
\begin{center}
|\ifchilddoc |\textit{child-code}| |[|\||else |\textit{main-code}]| \||fi|
\end{center}

%%%%%%%%%%%%%%%%%%%%%%%%%%%%%%%%%%%%%%%%
\DescribeMacro{\childdocname}
\DescribeMacro{\childdocjob}
The macro |\childdocname| contains the filename (without extension)
of the main or child file being processed.
Note that |\childdocjob| will always contain the name of the main file.

%%%%%%%%%%%%%%%%%%%%%%%%%%%%%%%%%%%%%%%%
\paragraph{Title Page.}

Conditional processing can be used to include a title or banner page
in the main document when proper precautions are taken.
Importantly, the code in the main file should ensure that the page counter
(as well as other status parameters which are stored in the |.aux| files)
takes the same value after the conditional processing.
Otherwise the page numbers may take divergent values
depending on which part is compiled.

For example, a title page could be declared by:
%
\begin{center}
\begin{tabular}{l}
|\ifchilddoc\||else|\\
|\addtocounter{page}{-1}|\\
\textit{code for title page}\\
|\newpage|\\
|\||fi|
\end{tabular}
\end{center}
%
A banner page for the child documents can be generated by:
%
\begin{center}
\begin{tabular}{l}
|\ifchilddoc|\\
|\addtocounter{page}{-1}|\\
\textit{code for banner page}\\
|\newpage|\\
|\||fi|
\end{tabular}
\end{center}
%
Here one could write a message such as:
\begin{center}
|This is the part \childdocname{} of \childdocjob{}.|
\end{center}

%%%%%%%%%%%%%%%%%%%%%%%%%%%%%%%%%%%%%%%%%%%%%%%%%%%%%%%%%%%%%%%%%%%%%%%%%%%%%%%%
\subsection{Flags}
\label{sec:flags}

The package makes it easy to generate different versions
of the main or child documents.
To this end compilation flags can be defined
and assigned different default values.
They will be particularly useful in conjunction
with the forwarding mechanism described in \secref{sec:forward}.

For example, it may be useful to have a flag |\version|
which can be set to |draft| or |final|.
The document source will contain some conditional code
depending on the value of |\version|.
Suppose further, the flag should default to |final| for the main file
and to |draft| for child files
which is a natural assignment for editing the document.
This is achieved by placing the following code
in the preamble of the main document
(below the |\childdocmain| directive):
%
\begin{center}
\begin{tabular}{l}
|\ifchilddoc|\\
|\providecommand{\version}{draft}|\\
|\||else|\\
|\providecommand{\version}{final}|\\
|\||fi|
\end{tabular}
\end{center}
%
The definition by |\providecommand| makes sure
that previous definitions are not overwritten.
Further statements |\providecommand{\version}{...}|
can thus be added before the above code to override it.

For the main file, one might add a line
(between |\childdocmain| and the above block)
%
\begin{center}
|%\ifchilddoc\||else\providecommand{\version}{draft}\||fi|
\end{center}
%
which can be uncommented to produce a draft version.
Likewise one can add a line to the very top of a child file
(above the |\childdocof{|\textit{main}|}| directive)
%
\begin{center}
|%\providecommand{\version}{final}|
\end{center}
%
which can be uncommented to produce the final version of this child document.

%%%%%%%%%%%%%%%%%%%%%%%%%%%%%%%%%%%%%%%%%%%%%%%%%%%%%%%%%%%%%%%%%%%%%%%%%%%%%%%%
\subsection{Forwarding}
\label{sec:forward}

Different versions of the main or child documents
using compilation flags as described in \secref{sec:flags}
can be (permanently) stored in different files
for convenient compilation, viewing and distribution.
To this end, the package defines a command
to pass on compilation to a different file:

%%%%%%%%%%%%%%%%%%%%%%%%%%%%%%%%%%%%%%%%
\DescribeMacro{\childdocforward}
The command |\childdocforward| redirects processing to
another source file:
%
\begin{center}
\begin{tabular}{l}
|\input{childdoc.def}|\\
|\childdocforward[|\textit{main}|]{|\textit{dest}|}|\\
\end{tabular}
\end{center}
%
The argument \textit{dest} is the destination file
(without extension).
It should be the main file or one of the child files.
Note that further \textsf{childdoc} directives
such as |\childdocof| and |\childdocforward|
in the indicated file will be processed in this form.
The optional argument \textit{main}
passes on directly to the main file \textit{main}
while pretending to compile the child \textit{dest}.
This form behaves as if \textit{dest}
issues |\childdocof{|\textit{main}|}| right away,
and no further \textsf{childdoc} directives will be processed.

%%%%%%%%%%%%%%%%%%%%%%%%%%%%%%%%%%%%%%%%
\DescribeMacro{\...prefix}
In the alternative form |\childdocforwardprefix|,
%
\begin{center}
\begin{tabular}{l}
|\input{childdoc.def}|\\
|\childdocforwardprefix[|\textit{main}|]{|\textit{prefix}|}{|\textit{dest}|}|
\end{tabular}
\end{center}
%
the destination file is determined by a pattern
depending on the current file:
To make this work, the current file must be called
`{\textit{prefix}\hspace{0.2em}\textit{suffix}}'
with \textit{prefix} matching precisely the argument.
Processing is then passed on to the file
`{\textit{dest}\hspace{0.2em}\textit{suffix}}'.
Surely, the same effect is achieved by
directly specifying the
argument `{\textit{dest}\hspace{0.2em}\textit{suffix}}'
in the first form.
However, that requires to set up a different file
for each child. With the alternative form of the command
all these files can have exactly the same content
which simplifies setting them up and maintaining them.

For example, the following file |draft.tex|
with a compilation flag |\version| as described in \secref{sec:flags}
compiles the main document as a draft:
%
\begin{center}
\begin{tabular}{l}
|\def\version{draft}|\\
|\input{childdoc.def}|\\
|\childdocforward{|\textit{main}|}|
\end{tabular}
\end{center}
%
Likewise, the following files |final|\textit{nn}|.tex|
compile the final version of the child document
|child|\textit{nn}|.tex|:
%
\begin{center}
\begin{tabular}{l}
|\def\version{final}|\\
|\input{childdoc.def}|\\
|\childdocforwardprefix{final}{child}|
\end{tabular}
\end{center}
%

Note that when several versions of a main file and/or of each child file
are to be generated, it may be convenient to set up a |Makefile| or
shell script to automatise the process.

%%%%%%%%%%%%%%%%%%%%%%%%%%%%%%%%%%%%%%%%%%%%%%%%%%%%%%%%%%%%%%%%%%%%%%%%%%%%%%%%
\subsection{Command Line Processing}
\label{sec:commandline}

The effect of redirection files can also be achieved by invoking
the \LaTeX{} compiler with a more elaborate command line.
Most conveniently this should be done as part
of a shell script or a |Makefile|.

When using \textsf{childdoc} in the main file, the following
command lines effectively perform a redirection
(note that depending on the shell being used,
backslashes may have to be doubled: `|\|' $\to$ `|\\|'):
%
\begin{center}
|... -jobname "|\textit{target}|" |\\|"|[\textit{flags}]%
|\input{childdoc.def}\childdocforward[|\textit{main}|]{|\textit{dest}|}"|
\end{center}
%
Here \textit{target} is the name of the output file,
\textit{main} is the name of the main file
and \textit{dest} is the name of the main or child file to be processed
(all filenames without extensions).
The optional argument \textit{main} can be omitted
if \textit{main} matches \textit{dest}.
Optionally, compilation \textit{flags} can be defined via |\def| commands.
This command line makes the \TeX{} engine believe
it is compiling the file \textit{target}
whose content is specified as the latter parameter.
The provided code then forwards the processing to
\textit{main} or \textit{dest} as described in \secref{sec:forward}.

%%%%%%%%%%%%%%%%%%%%%%%%%%%%%%%%%%%%%%%%%%%%%%%%%%%%%%%%%%%%%%%%%%%%%%%%%%%%%%%%
\subsection{Include by Input}
\label{sec:input}

Including child documents by |\include| has some restrictions by design.
Most notably, the content of a child document always occupies
its own set of pages; pages cannot be shared between child documents.
Usually, this behaviour makes perfect sense
because each child document contain an essential part of the document.
However, in some situations it may be desirable to compose
a document from a collection of parts
without having mandatory page breaks between then.
For this case, the package
provides a mechanism to include parts
by |\input| which can also be processed individually.
However, by construction this mechanism
requires manual handling of the content to be output.

%%%%%%%%%%%%%%%%%%%%%%%%%%%%%%%%%%%%%%%%
\DescribeMacro{\ifchilddocmanual}
The main file should be prepared as usual, see \secref{sec:include}.
However, the document body must make a distinction
between processing of an individual part and of the main document, e.g.:
%
\begin{center}
\begin{tabular}{l}
|\ifchilddocmanual|\\
|\input{\childdocname}|\\
|\||else|\\
\textit{document body with }|\input{|\textit{part}|}|\\
|\||fi|
\end{tabular}
\end{center}
%
The conditional |\ifchilddocmanual| is true whenever
a part to be included by |\input| is being compiled,
and the name of the part is stored in |\childdocname|.

%%%%%%%%%%%%%%%%%%%%%%%%%%%%%%%%%%%%%%%%
\DescribeMacro{\childdocby}
Each part to be included by |\input| should start with:
%
\begin{center}
\begin{tabular}{l}
|\input{childdoc.def}|\\
|\childdocby{|\textit{main}|}|\\
\end{tabular}
\end{center}
%
The directive |\childdocby| is similar to |\childdocof|
described in \secref{sec:include},
but the subsequent selection of content must be done manually.
To that end, both |\ifchilddoc| and |\ifchilddocmanual|
will be true upon processing of a part,
and the name of the part is stored in |\childdocname|.
Note that |\jobname| will be set to the filename of the current part
so that each part receives an individual |.aux| file
that does not interfere with the |.aux| file(s) of the main document.
This behaviour can be altered by the alternative form
|\childdocby[*]{|\textit{main}|}| (with a non-empty optional argument)
which uses the |.aux| file of the main document
by setting |\jobname| to \textit{main}.

%%%%%%%%%%%%%%%%%%%%%%%%%%%%%%%%%%%%%%%%%%%%%%%%%%%%%%%%%%%%%%%%%%%%%%%%%%%%%%%%
\subsection{Driver Development}
\label{sec:driver}

The \textsf{childdoc} mechanism can also be use for the development
of definition files such as \LaTeX{} styles or classes.
This case differs from the above setup with multiple parts
included by |\include| in that no |\includeonly| should be invoked.
This can be achieved by starting the include file
(before |\ProvidesPackage|) with:
%
\begin{center}
\begin{tabular}{l}
|\input{childdoc.def}|\\
|\childdocforward{|\textit{main}|}|\\
\end{tabular}
\end{center}
%
or alternatively with:
%
\begin{center}
\begin{tabular}{l}
|\input{childdoc.def}|\\
|\childdocby{|\textit{main}|}|\\
\end{tabular}
\end{center}
%
Both forms have slightly different effects as described above.
The main file is prepared as usual, see \secref{sec:include}.

%%%%%%%%%%%%%%%%%%%%%%%%%%%%%%%%%%%%%%%%%%%%%%%%%%%%%%%%%%%%%%%%%%%%%%%%%%%%%%%%
\subsection{Legacy Detection}
\label{sec:detection}

The directive |\childdocmain| in the main file can detect
whether the complete document or merely a child is to be compiled
even without using the directive |\childdocof|.
This method is deprecated because it is less robust
and there is no compelling reason to use it;
it is merely provided for backward compatibility
and it may be removed in future versions.

If the detection mechanism is to be used,
it is mandatory to correctly specify
the filename of the main file as the argument of |\childdocmain|:
%
\begin{center}
\begin{tabular}{l}
|\input{childdoc.def}|\\
|\childdocmain{|\textit{main}|}|\\
\end{tabular}
\end{center}
%
If |\jobname| does not match the argument \textit{main} of |\childdocmain|,
it is assumed that |\jobname| points to the child file to be compiled.
When using |\childdocmain| with the main file specified as argument,
it suffices to start a child file
with just |\input{|\textit{main}|}|
without loading of the package and using |\childdocof|.
If instead all processing is done
with the appropriate \textsf{childdoc} directives,
the argument of \textit{main} of |\childdocmain| can be empty.

An alternative version of the command line processing described
in \secref{sec:commandline} using the detection mechanism reads:
%
\begin{center}
|... -jobname "|\textit{target}|" "|[\textit{flags}]%
[|\def\jobname{|\textit{dest}|}|]|\input{|\textit{main}|}"|
\end{center}

%%%%%%%%%%%%%%%%%%%%%%%%%%%%%%%%%%%%%%%%%%%%%%%%%%%%%%%%%%%%%%%%%%%%%%%%%%%%%%%%
\subsection{Manual Code}
\label{sec:manual}

In case one cannot be certain whether the definitions file |childdoc.def|
is installed on the target \TeX{} distribution
and one prefers not to ship it,
it is conceivable to paste a few relevant commands into the sources.

To that end, drop all statements |\input{childdoc.def}|
and perform the replacements as outlined below.
Instead of |\childdocmain{|\textit{main}|}| add the following code
to the top of the main file:
%
\begin{center}
\begin{tabular}{l}
|\||ifdefined\childdocname\endinput\||fi\newif\ifchilddoc|\\
|\edef\childdocname{\scantokens\expandafter{\jobname\noexpand}}|\\
|\def\childdocmain{|\textit{main}|}\||ifx\childdocmain\childdocname\||else|\\
|\childdoctrue\includeonly{\childdocname}\let\jobname\childdocmain\||fi|\\
\end{tabular}
\end{center}
%
Instead of |\childdocof{|\textit{main}|}| just include the main file
at the top of each child file:
%
\begin{center}
|\input{|\textit{main}|}|
\end{center}
%
A simple redirection |\childdocforward{|\textit{dest}|}| is achieved by:
%
\begin{center}
|\def\jobname{|\textit{dest}|}\input{\jobname}|
\end{center}
%
The redirection with prefix
|\childdocforwardprefix[|\textit{prefix}|]{|\textit{dest}|}|
is accomplished by:
%
\begin{center}
\begin{tabular}{l}
|{\edef\jobname{\scantokens\expandafter{\jobname\noexpand}}|\\
|\def\redirectjob |\textit{prefix}|#1~~~{\gdef\jobname{|\textit{dest}|#1}}|\\
|\expandafter\redirectjob\jobname~~~}\input{\jobname}|
\end{tabular}
\end{center}

In an alternative approach,
child documents can be compiled by a specific command line
without additional code or specific definitions:
%
\begin{center}
|... -jobname "|\textit{target}|" "|[\textit{flags}]%
|\includeonly{|\textit{dest}|}\input{|\textit{main}|}"|
\end{center}
%

%%%%%%%%%%%%%%%%%%%%%%%%%%%%%%%%%%%%%%%%%%%%%%%%%%%%%%%%%%%%%%%%%%%%%%%%%%%%%%%%
%%%%%%%%%%%%%%%%%%%%%%%%%%%%%%%%%%%%%%%%%%%%%%%%%%%%%%%%%%%%%%%%%%%%%%%%%%%%%%%%
\section{Information}

%%%%%%%%%%%%%%%%%%%%%%%%%%%%%%%%%%%%%%%%%%%%%%%%%%%%%%%%%%%%%%%%%%%%%%%%%%%%%%%%
\subsection{Copyright}

Copyright \copyright{} 2017--2018 Niklas Beisert

This work may be distributed and/or modified under the
conditions of the \LaTeX{} Project Public License, either version 1.3
of this license or (at your option) any later version.
The latest version of this license is in
  \url{http://www.latex-project.org/lppl.txt}
and version 1.3 or later is part of all distributions of \LaTeX{}
version 2005/12/01 or later.

This work has the LPPL maintenance status `maintained'.

The Current Maintainer of this work is Niklas Beisert.

This work consists of the files |README.txt|, |childdoc.ins| and |childdoc.dtx|
as well as the derived files |childdoc.def|, |cdocsamp.tex|
with |cdocsch1.tex|, |cdocsch2.tex|, |cdocspt3.tex|, |cdocspt4.tex|,
|cdocsdrf.tex|, |cdocsfn1.tex|, |cdocsfn2.tex|
as well as |childdoc.pdf|.

%%%%%%%%%%%%%%%%%%%%%%%%%%%%%%%%%%%%%%%%%%%%%%%%%%%%%%%%%%%%%%%%%%%%%%%%%%%%%%%%
\subsection{Files and Installation}

The package consists of the files:
%
\begin{center}
\begin{tabular}{ll}
    |README.txt|   & readme file \\
    |childdoc.ins| & installation file \\
    |childdoc.dtx| & source file \\
    |childdoc.def| & definition file \\
    |cdocsamp.tex| & sample main file \\
    |cdocsch1.tex| & sample include file \\
    |cdocsch2.tex| & sample include file \\
    |cdocspt3.tex| & sample part file \\
    |cdocspt4.tex| & sample part file \\
    |cdocsdrf.tex| & sample redirection file \\
    |cdocsfn1.tex| & sample redirection file \\
    |cdocsfn2.tex| & sample redirection file \\
    |childdoc.pdf| & manual
\end{tabular}
\end{center}
%
The distribution consists of the files
|README.txt|, |childdoc.ins| and |childdoc.dtx|.
%
\begin{itemize}
\item
Run (pdf)\LaTeX{} on |childdoc.dtx|
to compile the manual |childdoc.pdf| (this file).
\item
Run \LaTeX{} on |childdoc.ins| to create the definitions file |childdoc.def|
and the sample |cdocsamp.tex| with include files
|cdocsch1.tex|, |cdocsch2.tex|, |cdocspt3.tex|, |cdocspt4.tex|,
|cdocsdrf.tex|, |cdocsfn1.tex|, |cdocsfn2.tex|.
Then copy the file |childdoc.def| to an appropriate directory of your \LaTeX{}
distribution, e.g.\ \textit{texmf-root}|/tex/latex/childdoc|.
\end{itemize}

%%%%%%%%%%%%%%%%%%%%%%%%%%%%%%%%%%%%%%%%%%%%%%%%%%%%%%%%%%%%%%%%%%%%%%%%%%%%%%%%
\subsection{Related CTAN Packages}

There are several other packages which offer a similar functionality:
%
\begin{itemize}
\item
The packages
\href{http://ctan.org/pkg/docmute}{\textsf{docmute}},
\href{http://ctan.org/pkg/includex}{\textsf{includex}} and
\href{http://ctan.org/pkg/standalone}{\textsf{standalone}}
provide commands to include only the document body of
a child file thus allowing both files to be compiled individually.
\item
The packages \href{http://ctan.org/pkg/subdocs}{\textsf{subdocs}}
and \href{http://ctan.org/pkg/subfiles}{\textsf{subfiles}}
provide structures in which the main and child documents can be
encapsulated and allowing them to be compiled individually.
The inclusion mechanism is different from the conventional |\include|.
\item
The package \href{http://ctan.org/pkg/combine}{\textsf{combine}}
is an elaborate solution to combine several documents into one.
\end{itemize}
%
See also the CTAN topic \href{http://ctan.org/topic/subdocs}{\textsf{subdocs}}
for further related packages.
The present package differs from the above solutions in that
a document structure constructed with the conventional |\include| mechanism
just needs two extra commands at the top of every file
such that all constituent files can be compiled individually.

%%%%%%%%%%%%%%%%%%%%%%%%%%%%%%%%%%%%%%%%%%%%%%%%%%%%%%%%%%%%%%%%%%%%%%%%%%%%%%%%
%\subsection{Feature Suggestions}
%
%The following is a list of features which may be useful for future
%versions of this package:
%%
%\begin{itemize}
%\item
%\ldots
%\end{itemize}

%%%%%%%%%%%%%%%%%%%%%%%%%%%%%%%%%%%%%%%%%%%%%%%%%%%%%%%%%%%%%%%%%%%%%%%%%%%%%%%%
\subsection{Revision History}

%%%%%%%%%%%%%%%%%%%%%%%%%%%%%%%%%%%%%%%%
\paragraph{v2.0:} 2018/12/30

\begin{itemize}
\item
immediate forward processing
\item
added |\childdocby| mechanism
\item
manual restructured
\end{itemize}

%%%%%%%%%%%%%%%%%%%%%%%%%%%%%%%%%%%%%%%%
\paragraph{v1.6:} 2018/01/17

\begin{itemize}
\item
application for development of include files
\item
corrections to manual
\end{itemize}

%%%%%%%%%%%%%%%%%%%%%%%%%%%%%%%%%%%%%%%%
\paragraph{v1.5:} 2017/05/21

\begin{itemize}
\item
more complete structuring introduced
\item
|\childdocof| introduced
\item
|\childdoc| renamed to |\childdocmain|
\item
|\childredirect| renamed to |\childdocforward| and |\childdocforwardprefix|
and functionality expanded
\end{itemize}

%%%%%%%%%%%%%%%%%%%%%%%%%%%%%%%%%%%%%%%%
\paragraph{v1.0:} 2017/04/27

\begin{itemize}
\item
manual and install package
\item
first version published on CTAN
\end{itemize}

%%%%%%%%%%%%%%%%%%%%%%%%%%%%%%%%%%%%%%%%
\paragraph{v0.6:} 2017/04/26

\begin{itemize}
\item
redirection mechanism added
\end{itemize}

%%%%%%%%%%%%%%%%%%%%%%%%%%%%%%%%%%%%%%%%
\paragraph{v0.5:} 2017/04/26

\begin{itemize}
\item
functionality in definition file
\end{itemize}


%%%%%%%%%%%%%%%%%%%%%%%%%%%%%%%%%%%%%%%%%%%%%%%%%%%%%%%%%%%%%%%%%%%%%%%%%%%%%%%%
%%%%%%%%%%%%%%%%%%%%%%%%%%%%%%%%%%%%%%%%%%%%%%%%%%%%%%%%%%%%%%%%%%%%%%%%%%%%%%%%
%%%%%%%%%%%%%%%%%%%%%%%%%%%%%%%%%%%%%%%%%%%%%%%%%%%%%%%%%%%%%%%%%%%%%%%%%%%%%%%%
\appendix

\settowidth\MacroIndent{\rmfamily\scriptsize 000\ }

 \DocInput{childdoc.dtx}

\end{document}
%</driver>
% \fi
%
% %%%%%%%%%%%%%%%%%%%%%%%%%%%%%%%%%%%%%%%%%%%%%%%%%%%%%%%%%%%%%%%%%%%%%%%%%%%%%%
% %%%%%%%%%%%%%%%%%%%%%%%%%%%%%%%%%%%%%%%%%%%%%%%%%%%%%%%%%%%%%%%%%%%%%%%%%%%%%%
% \section{Sample}
%\iffalse
%<*samplemain>
%\fi
%
% The following presents a sample document
% with two chapters, two parts, a title page,
% a compile flag as well as three forwarding files to set the flag.
% It consists of eight |.tex| files:
% \begin{center}
% \begin{tabular}{ll}
% |cdocsamp.tex|&main file\\
% |cdocsch1.tex|&include file for chapter 1\\
% |cdocsch2.tex|&include file for chapter 2\\
% |cdocspt3.tex|&include file for part 3\\
% |cdocspt4.tex|&include file for part 4\\
% |cdocsdrf.tex|&forwarding file for main file in draft mode\\
% |cdocsfi1.tex|&forwarding file for final version of chapter 1\\
% |cdocsfi2.tex|&forwarding file for final version of chapter 2\\
% \end{tabular}
% \end{center}
% Each of the eight files can be compiled directly by the \LaTeX{} compiler.
%
% %%%%%%%%%%%%%%%%%%%%%%%%%%%%%%%%%%%%%%
% \paragraph{Main File.}
%
% The main file is called |cdocsamp.tex|.
%
% Load the \textsf{childdoc} definitions and
% declare the filename for the main document:
%    \begin{macrocode}
\input{childdoc.def}
\childdocmain{}
%    \end{macrocode}

% Optional override for |\version| flag:
%    \begin{macrocode}
%%\ifchilddoc\else\providecommand{\version}{draft}\fi
%    \end{macrocode}

% Define the default values for the |\version| flag
% (|final| for the main file and |draft| for childs):
%    \begin{macrocode}
\ifchilddoc
\providecommand{\version}{draft}
\else
\providecommand{\version}{final}
\fi
%    \end{macrocode}

% Load the standard document class:
%    \begin{macrocode}
\documentclass[12pt]{article}
%    \end{macrocode}

% Start the document body:
%    \begin{macrocode}
\begin{document}
%    \end{macrocode}

% Declare a title page.
% Print title, part of document being processed and version flag:
%    \begin{macrocode}
\addtocounter{page}{-1}
\begin{center}
{\LARGE\bfseries{}childdoc example\par}
\vspace{1cm}
\ifchilddoc
\ifchilddocmanual part\else chapter\fi:
`\childdocname' of `\childdocjob'\par
\else
main document: `\childdocjob'\par
\fi
version: \version\par
\end{center}
\newpage
%    \end{macrocode}

% Manually include selected file,
% otherwise process as usual:
%    \begin{macrocode}
\ifchilddocmanual
\section*{part `\childdocname'}
\input{\childdocname}
\else
%    \end{macrocode}

% Include the two chapters:
%    \begin{macrocode}
\include{cdocsch1}
\include{cdocsch2}
%    \end{macrocode}

% Include the two parts unless only chapters should be displayed:
%    \begin{macrocode}
\ifchilddoc\else
\section{part three}
\input{cdocspt3}
\section{part four}
\input{cdocspt4}
\fi
%    \end{macrocode}

% Process as usual until here:
%    \begin{macrocode}
\fi
%    \end{macrocode}

% End of document body:
%    \begin{macrocode}
\end{document}
%    \end{macrocode}
%\iffalse
%</samplemain>
%\fi
%
% %%%%%%%%%%%%%%%%%%%%%%%%%%%%%%%%%%%%%%
% \paragraph{Chapter Include Files.}
%
% The include files are called |cdocsch1.tex| and |cdocsch2.tex|.
%
%\iffalse
%<*samplechap1|samplechap2>
%\fi

% Optional override for |\version| flag:
%    \begin{macrocode}
%%\providecommand{\version}{final}
%    \end{macrocode}

% Include the main document:
%    \begin{macrocode}
\input{childdoc.def}
\childdocof{cdocsamp}
%    \end{macrocode}

%\iffalse
%</samplechap1|samplechap2>
%\fi
%
%\iffalse
%<*samplechap1>
%\fi
% Some text for chapter 1:
%    \begin{macrocode}
\section{one}
some text in chapter one
%    \end{macrocode}

%\iffalse
%</samplechap1>
%\fi
% Some text for chapter 2:
%\iffalse
%<*samplechap2>
%\fi
%    \begin{macrocode}
\section{two}
more text in chapter two
%    \end{macrocode}

%\iffalse
%</samplechap2>
%\fi
%
% %%%%%%%%%%%%%%%%%%%%%%%%%%%%%%%%%%%%%%
% \paragraph{Part Include Files.}
%
% The include files are called |cdocspt3.tex| and |cdocspt4.tex|.
%
%\iffalse
%<*samplepart3|samplepart4>
%\fi

% Optional override for |\version| flag:
%    \begin{macrocode}
%%\providecommand{\version}{final}
%    \end{macrocode}

% Include the main document:
%    \begin{macrocode}
\input{childdoc.def}
\childdocby{cdocsamp}
%    \end{macrocode}

%\iffalse
%</samplepart3|samplepart4>
%\fi
%
%\iffalse
%<*samplepart3>
%\fi
% Some text for part 3:
%    \begin{macrocode}
some text in part three
%    \end{macrocode}

%\iffalse
%</samplepart3>
%\fi
% Some text for part 4:
%\iffalse
%<*samplepart4>
%\fi
%    \begin{macrocode}
more text in part four
%    \end{macrocode}

%\iffalse
%</samplepart4>
%\fi
%
% %%%%%%%%%%%%%%%%%%%%%%%%%%%%%%%%%%%%%%
% \paragraph{Forwarding for a Complete Draft.}
%
% The following forwarding file |cdocsdrf.tex|
% compiles the main document in draft mode:
%\iffalse
%<*sampledraft>
%\fi
%    \begin{macrocode}
\def\version{draft}
\input{childdoc.def}
\childdocforward{cdocsamp}
%    \end{macrocode}

%\iffalse
%</sampledraft>
%\fi
%
% %%%%%%%%%%%%%%%%%%%%%%%%%%%%%%%%%%%%%%
% \paragraph{Forwarding for Final Version of the Chapters.}
%
% The following forwarding files |cdocsfn1.tex| and |cdocsfn2.tex|
% (with identical content)
% compile the final versions of the child documents
% |cdocsch1.tex| and |cdocsch2.tex|, respectively:
%\iffalse
%<*samplefinal>
%\fi
%    \begin{macrocode}
\def\version{final}
\input{childdoc.def}
\childdocforwardprefix[cdocsamp]{cdocsfn}{cdocsch}
%    \end{macrocode}

%\iffalse
%</samplefinal>
%\fi
%
% %%%%%%%%%%%%%%%%%%%%%%%%%%%%%%%%%%%%%%
% \paragraph{Command Line Processing.}
%
% The following three command lines generate the output files
% |cdocscld|, |cdocscl1| and |cdocscl2|
% which should be identical to
% |cdocsdrf|, |cdocsch1| and |cdocsfn2|, respectively:
% \begin{center}
% \begin{tabular}{l}
% |latex -jobname cdocscld \|\\
% |  "\def\version{draft}\input{childdoc.def}\childdocforward{cdocsamp}"|\\
% |latex -jobname cdocscl1 \|\\
% |  "\input{childdoc.def}\childdocforward[cdocsamp]{cdocsch1}"|\\
% |latex -jobname cdocscl2 \|\\
% |  "\def\version{final}\input{childdoc.def}\childdocforward{cdocsch2}"|
% \end{tabular}
% \end{center}
% Note that the trailing backslash on each first line
% merely continues the input to the second line
% (for convenient cut ant paste).
% Furthermore, the command |latex| can be replaced by any
% of its alternative versions such as |pdflatex|.
%
% %%%%%%%%%%%%%%%%%%%%%%%%%%%%%%%%%%%%%%%%%%%%%%%%%%%%%%%%%%%%%%%%%%%%%%%%%%%%%%
% %%%%%%%%%%%%%%%%%%%%%%%%%%%%%%%%%%%%%%%%%%%%%%%%%%%%%%%%%%%%%%%%%%%%%%%%%%%%%%
% \section{Implementation}
%\iffalse
%<*package>
%\fi
%
% This section describes the definitions file |childdoc.def|.

% The definitions cannot be loaded using |\usepackage| or |\RequirePackage|
% which has a mechanism to prevent loading a style file more than once.
% When loading the definitions by means of |\input|
% multiple instances have to be prevented manually:
%\iffalse
%This code needs to be before the `\ProvidesFile' directive
%which is defined at the beginning of this file.
%Therefore it is also placed there and commented out here.
%</package>
%<*discard>
%\fi
%    \begin{macrocode}
\ifdefined\childdocmain\endinput\fi
%    \end{macrocode}
%\iffalse
%</discard>
%<*package>
%\fi
%
% \macro{\ifchilddoc}
% \macro{\ifchilddocmanual}
% The conditional |\ifchilddoc| tells whether a
% child (true) or main (false) document is being compiled.
% The conditional |\ifchilddocmanual| tells whether
% the |\includeonly| mechanism is used (false) or
% the selection of child files must be performed manually (true).
% The definitions initialise to false:
%    \begin{macrocode}
\newif\ifchilddoc
\newif\ifchilddocmanual
%    \end{macrocode}

% \macro{\childdocname}
% \macro{\childdocjob}
% The macro |\childdocname| stores the name of the main document
% to be compiled. The macro |\childdocjob| stores the name of
% the document on which the \LaTeX{} compiler was originally invoked.
% The content of |\jobname| cannot be compared
% to filenames specified in the source due to different catcodes.
% The following code rescans |\jobname|, stores the result
% in |\childdocname| and saves a copy in |\childdocjob|:
%    \begin{macrocode}
\edef\childdocname{\scantokens\expandafter{\jobname\noexpand}}
\let\childdocjob\childdocname
%    \end{macrocode}

% \macro{\childdocdisable}
% The macro |\childdocdisable| prevents the main file
% from being processed more than once.
% At this stage, the main document command |\childdocmain|
% is assumed to be called once again where it should do nothing.
% Any subsequent call to it should prevent
% a secondary processing of the main document
% It overwrites the forwarding commands
% |\childdocof| and |\childdocforward|
% with empty macros to prevent further inclusions of the main document:
%    \begin{macrocode}
\newcommand{\childdocdisable}
{
  \renewcommand{\childdocmain}[1]{\renewcommand{\childdocmain}[1]{\endinput}}
  \renewcommand{\childdocof}[1]{}
  \renewcommand{\childdocby}[2][]{}
  \renewcommand{\childdocforward}[2][]{}
  \renewcommand{\childdocdisable}{}
}
%    \end{macrocode}

% \macro{\childdocmain}
% The macro |\childdocmain| is to be called at the top of the main file
% with nothing or the main filename (without extension) as argument.
% First, it breaks loops.
% If the argument is not empty and does not match |\childdocname|
% (which is set by the first inclusion of |childdoc.def|),
% |\ifchilddoc| is set to true, |\includeonly| is applied to the child file
% and |\jobname| is set to the main file
% (for proper handling of |.aux| files):
%    \begin{macrocode}
\newcommand{\childdocmain}[1]
{
  \childdocdisable\childdocmain{}
  \if?#1?\else
    \begingroup
      \def\childdoctmp{#1}
      \ifx\childdoctmp\childdocname
        \def\childdoctmp{}
      \else
        \def\childdoctmp
        {
          \childdoctrue
          \includeonly{\childdocname}
          \def\childdocjob{#1}
          \def\jobname{#1}
        }
      \fi
      \expandafter
    \endgroup
    \childdoctmp
  \fi
}
%    \end{macrocode}

% \macro{\childdocof}
% The command |\childdocof| redirects
% compilation to the main file |#1|.
%    \begin{macrocode}
\newcommand{\childdocof}[1]
{
  \childdocdisable
  \childdoctrue
  \includeonly{\childdocname}
  \def\jobname{#1}
  \def\childdocjob{#1}
  \input{#1}
}
%    \end{macrocode}

% \macro{\childdocby}
% The command |\childdocby| ....
%    \begin{macrocode}
\newcommand{\childdocby}[2][]
{
  \childdocdisable
  \childdoctrue
  \childdocmanualtrue
  \if?#1?\else
    \def\jobname{#2}
  \fi
  \def\childdocjob{#2}
  \input{#2}
  \endinput
}
%    \end{macrocode}

% \macro{\childdocforward}
% The command |\childdocforward| redirects
% compilation to the main file or
% (if the optional argument is given) a child file.
% Parameters are set as if the main file
% or a child file starting with |\childdocof| was compiled.
% Then compilation is handed over to the main file:
%    \begin{macrocode}
\newcommand{\childdocforward}[2][]
{
  \begingroup
    \if?#1?
      \def\childdoctmp
      {
        \def\childdocname{#2}
        \def\childdocjob{#2}
        \def\jobname{#2}
        \input{#2}
        \endinput
      }
    \else
      \def\childdoctmp
      {
        \childdocdisable
        \def\childdocname{#2}
        \childdoctrue
        \includeonly{#2}
        \def\childdocjob{#1}
        \def\jobname{#1}
        \input{#1}
        \endinput
      }
    \fi
    \expandafter
  \endgroup
  \childdoctmp
}
%    \end{macrocode}

% \macro{\childdocforwardprefix}
% The command |\childdocforwardprefix| redirects
% compilation to the main or a child file by means of a pattern.
% The prefix |#1| in the current filename is replaced by |#2|
% and the suffix of the current filename is kept
% (it is assumed that the filename does not contain the substring `|~~~|'
% which is used as a delimiter).
% Compilation is handed over to the new file by |\childdocforward|:
%    \begin{macrocode}
\newcommand{\childdocforwardprefix}[3][]
{
  \begingroup
    \def\childdocextract #2##1~~~{\def\childdoctmp{\childdocforward[#1]{#3##1}}}
    \expandafter\childdocextract\childdocname~~~
    \expandafter
  \endgroup
  \childdoctmp
}
%    \end{macrocode}

% \macro{\childdoc}
% The deprecated macro |\childdoc| is a legacy version of |\childdocmain|:
%    \begin{macrocode}
\newcommand{\childdoc}{\childdocmain}
%    \end{macrocode}

% \macro{\childdocredirect}
% The deprecated macro |\childdocredirect| is a legacy version
% of |\childdocforward| and |\childdocforwardprefix|:
%    \begin{macrocode}
\newcommand{\childdocredirect}[2][]
{
  \begingroup
    \if?#1?
      \def\childdoctmp{\childdocforward{#2}}
    \else
      \def\childdoctmp{\childdocforwardprefix{#1}{#2}}
    \fi
    \expandafter
  \endgroup
  \childdoctmp
}
%    \end{macrocode}

%\iffalse
%</package>
%\fi
%
\endinput

\childdocforwardprefix[cdocsamp]{cdocsfn}{cdocsch}
%    \end{macrocode}

%\iffalse
%</samplefinal>
%\fi
%
% %%%%%%%%%%%%%%%%%%%%%%%%%%%%%%%%%%%%%%
% \paragraph{Command Line Processing.}
%
% The following three command lines generate the output files
% |cdocscld|, |cdocscl1| and |cdocscl2|
% which should be identical to
% |cdocsdrf|, |cdocsch1| and |cdocsfn2|, respectively:
% \begin{center}
% \begin{tabular}{l}
% |latex -jobname cdocscld \|\\
% |  "\def\version{draft}% \iffalse
%
% childdoc.dtx Copyright (C) 2017-2018 Niklas Beisert
%
% This work may be distributed and/or modified under the
% conditions of the LaTeX Project Public License, either version 1.3
% of this license or (at your option) any later version.
% The latest version of this license is in
%   http://www.latex-project.org/lppl.txt
% and version 1.3 or later is part of all distributions of LaTeX
% version 2005/12/01 or later.
%
% This work has the LPPL maintenance status `maintained'.
%
% The Current Maintainer of this work is Niklas Beisert.
%
% This work consists of the files childdoc.dtx and childdoc.ins
% and the derived files childdoc.def and cdocsamp.tex with
% cdocsch1.tex, cdocsch2.tex, cdocsdrf.tex, cdocsfn1.tex, cdocsfn2.tex.
%
%<package>\ifdefined\childdocmain\endinput\fi
%<package>\ProvidesFile{childdoc.def}[2018/12/30 v2.0 child document driver]
%<samplemain>\ProvidesFile{cdocsamp.tex}[2018/12/30 v2.0 sample for childdoc]
%<*driver>
%\ProvidesFile{childdoc.drv}[2018/12/30 v2.0 childdoc reference manual file]
\PassOptionsToClass{10pt,a4paper}{article}
\documentclass{ltxdoc}

\usepackage[margin=35mm]{geometry}
\usepackage{hyperref}
\usepackage{hyperxmp}
\usepackage[usenames]{color}

\hypersetup{colorlinks=true}
\hypersetup{pdfstartview=FitH}
\hypersetup{pdfpagemode=UseNone}
\hypersetup{pdfsource={}}
\hypersetup{pdflang={en-UK}}
\hypersetup{pdfcopyright={Copyright 2017-2018 Niklas Beisert.
  This work may be distributed and/or modified under the
  conditions of the LaTeX Project Public License, either version 1.3
  of this license or (at your option) any later version.}}
\hypersetup{pdflicenseurl={http://www.latex-project.org/lppl.txt}}
\hypersetup{pdfcontactaddress={ETH Zurich, ITP, HIT K,
  Wolfgang-Pauli-Strasse 27}}
\hypersetup{pdfcontactpostcode={8093}}
\hypersetup{pdfcontactcity={Zurich}}
\hypersetup{pdfcontactcountry={Switzerland}}
\hypersetup{pdfcontactemail={nbeisert@itp.phys.ethz.ch}}
\hypersetup{pdfcontacturl={http://people.phys.ethz.ch/\xmptilde nbeisert/}}

\newcommand{\secref}[1]{\hyperref[#1]{section \ref*{#1}}}

\parskip1ex
\parindent0pt
\let\olditemize\itemize
\def\itemize{\olditemize\parskip0pt}

\begin{document}

\title{The \textsf{childdoc} Package}
\hypersetup{pdftitle={The childdoc Package}}
\author{Niklas Beisert\\[2ex]
  Institut f\"ur Theoretische Physik\\
  Eidgen\"ossische Technische Hochschule Z\"urich\\
  Wolfgang-Pauli-Strasse 27, 8093 Z\"urich, Switzerland\\[1ex]
  \href{mailto:nbeisert@itp.phys.ethz.ch}
  {\texttt{nbeisert@itp.phys.ethz.ch}}}
\hypersetup{pdfauthor={Niklas Beisert}}
\hypersetup{pdfsubject={Manual for the LaTeX2e Package childdoc}}
\date{30 December 2018, \textsf{v2.0}}
\maketitle

\begin{abstract}\noindent
\textsf{childdoc} is a \LaTeXe{} package
that enables the direct compilation
of document sections included by |\include|
to individual files.
\end{abstract}

\begingroup
\parskip0ex
\tableofcontents
\endgroup

%%%%%%%%%%%%%%%%%%%%%%%%%%%%%%%%%%%%%%%%%%%%%%%%%%%%%%%%%%%%%%%%%%%%%%%%%%%%%%%%
%%%%%%%%%%%%%%%%%%%%%%%%%%%%%%%%%%%%%%%%%%%%%%%%%%%%%%%%%%%%%%%%%%%%%%%%%%%%%%%%
\section{Introduction}

\LaTeX{} provides a mechanism to structure a large document (such as a book)
into a main file and several child files (containing the chapters)
using the |\include| command.
This mechanism is beneficial for documents
which span hundreds of pages in order to
make the source file(s) more manageable.
Moreover, compilation can be restricted to
selected child files by means of the |\includeonly| command.
The latter feature can be used to reduce the compilation time while editing
(this was significantly more useful in the earlier days of \LaTeX{})
or to generate a smaller document which is easier to navigate.
Another application of |\includeonly| is to generate
documents consisting of selected parts of the complete document.

However, there are a few drawbacks of the plain |\include| mechanism:
\begin{itemize}
\item
The child files cannot be compiled on their own,
they can only be compiled via the main file.
A naive editing environment
(such as a text editor with an option
to have the current file processed by \LaTeX)
may require one to switch to the main file before compiling;
attempting to compile the child file produces errors.
\item
The main file must be modified (each time)
to adjust the |\includeonly| command
to the present needs. This easily leaves the main file in a messy state.
\item
The generated document will always carry the filename
of the main document. This is inconvenient if
several child files are to be compiled and
to be kept for distribution.
\end{itemize}

The present package provides a simple interface
to make child files individually compilable by \LaTeX{}.
Compiling a child file then has the same effect as compiling
the main file with an |\includeonly| command
to select the appropriate child.
Moreover the generated document will carry the name of the child
rather than the main file.
This resolves all three above issues.

This feature is meant to make the editing of books,
thesis documents and lecture notes somewhat more convenient.
However, the package can also be used efficiently for
composing a series of documents (such as exercise sheets)
which are typically distributed individually.
It then assists the author in generating the individual documents
(potentially in different versions)
as well as a document containing the collected series.
Another application is in developing style files
or other kinds of included material
where compilation of the style file could redirect
to a sample or test file.

%%%%%%%%%%%%%%%%%%%%%%%%%%%%%%%%%%%%%%%%%%%%%%%%%%%%%%%%%%%%%%%%%%%%%%%%%%%%%%%%
%%%%%%%%%%%%%%%%%%%%%%%%%%%%%%%%%%%%%%%%%%%%%%%%%%%%%%%%%%%%%%%%%%%%%%%%%%%%%%%%
\section{Usage}

First of all, the package \textsf{childdoc} is \emph{not} a standard
\LaTeXe{} |.sty| style file! Therefore it needs to be invoked in
a non-standard way.

%%%%%%%%%%%%%%%%%%%%%%%%%%%%%%%%%%%%%%%%%%%%%%%%%%%%%%%%%%%%%%%%%%%%%%%%%%%%%%%%
\subsection{Included Files}
\label{sec:include}

%%%%%%%%%%%%%%%%%%%%%%%%%%%%%%%%%%%%%%%%
\DescribeMacro{\childdocmain}
To use the package, add the commands
\begin{center}
\begin{tabular}{l}
|\input{childdoc.def}|\\
|\childdocmain{}|\\
\end{tabular}
\end{center}
at the very top of the main \LaTeX{} file,
in particular \emph{before} the |\documentclass| statement!
The argument of |\childdocmain| should be left empty
(but it must be present).

%%%%%%%%%%%%%%%%%%%%%%%%%%%%%%%%%%%%%%%%
\DescribeMacro{\childdocof}
Furthermore, add the commands
\begin{center}
\begin{tabular}{l}
|\input{childdoc.def}|\\
|\childdocof{|\textit{main}|}|\\
\end{tabular}
\end{center}
at the top of every child file \textit{child}
which is included by |\include{|\textit{child}|}|
from within the main file
(or at least for those files to be compiled individually).
The argument \textit{main} must be the filename of the main file.

There are a couple of
considerations in setting up the main and child documents:

%%%%%%%%%%%%%%%%%%%%%%%%%%%%%%%%%%%%%%%%
\paragraph{Restrictions.}

Please note the following restrictions:
\begin{itemize}
\item
|\childdocmain| must be called with one argument \textit{main}
to ensure compatibility with earlier version of the package.
It must either be empty (|\childdocmain{}|)
or precisely match the filename of the main file in which it is specified.
See \secref{sec:detection} for further information.
\item
The filename \textit{main} must be specified without the |.tex| extension.
\item
The filename \textit{main} is case sensitive
(even in case-insensitive file systems)
due to internal string comparison.
\item
The argument \textit{main} should be fully expanded, it cannot be a macro.
\item
Subdirectories and special characters should be avoided in filenames.
\item
The command |\childdocmain{|\textit{main}|}| must be followed by a whitespace.
It should not be followed immediately by another command
or by a comment mark `|%|'.
This is because the \TeX{} parser reads the token immediately following
the argument of |\childdocmain| and puts it
at the beginning of every child section;
however, a white\-space is ignored.
\end{itemize}

%%%%%%%%%%%%%%%%%%%%%%%%%%%%%%%%%%%%%%%%
\paragraph{Content of Main File.}

It is advisable to place all content in the child files included by |\include|.
Any output contained in the main file will appear in all child documents
unless suppressed manually;
it cannot be suppressed automatically by the |\includeonly| directive
and thus should normally be avoided.
A method to include some content in the main file
by means of conditional processing is described in \secref{sec:conditional}.

%%%%%%%%%%%%%%%%%%%%%%%%%%%%%%%%%%%%%%%%
\paragraph{Page Numbering.}

When only a part of the document is compiled,
the appropriate numbering of pages
(as well as other status parameters)
is determined from the |.aux| files.
The latter contain information from previous passes.
However this information needs to propagate through
all intermediate child documents.
Therefore the page numbering in child documents may well
be inconsistent until the complete document is compiled at least once.

A useful (if unconventional) way to always ensure a consistent
page numbering is to restart the numbering in each child document
and denote the pages by `\textit{child}|.|\textit{page}'
where \textit{child} represents the chapter/section number of the child file.
This can be achieved by the command
|\numberwithin{page}{|\textit{child}|}|
of the \textsf{amsmath} package
where \textit{child} can be |chapter| or |section|
depending on the chosen structuring.
Alternatively, one can modify the macro |\thepage| appropriately
and reset the counter |page| at the start of each child file.

%%%%%%%%%%%%%%%%%%%%%%%%%%%%%%%%%%%%%%%%%%%%%%%%%%%%%%%%%%%%%%%%%%%%%%%%%%%%%%%%
\subsection{Conditional Processing}
\label{sec:conditional}

The package provides a mechanism to compile different versions
of a document. To customise the versions further some conditional processing
can come in handy to distinguish which version is being compiled.
The package provides two macros to describe the compilation context:

%%%%%%%%%%%%%%%%%%%%%%%%%%%%%%%%%%%%%%%%
\DescribeMacro{\ifchilddoc}
The conditional |\ifchilddoc| distinguishes between the compilation of
child documents and the main document:
%
\begin{center}
|\ifchilddoc |\textit{child-code}| |[|\||else |\textit{main-code}]| \||fi|
\end{center}

%%%%%%%%%%%%%%%%%%%%%%%%%%%%%%%%%%%%%%%%
\DescribeMacro{\childdocname}
\DescribeMacro{\childdocjob}
The macro |\childdocname| contains the filename (without extension)
of the main or child file being processed.
Note that |\childdocjob| will always contain the name of the main file.

%%%%%%%%%%%%%%%%%%%%%%%%%%%%%%%%%%%%%%%%
\paragraph{Title Page.}

Conditional processing can be used to include a title or banner page
in the main document when proper precautions are taken.
Importantly, the code in the main file should ensure that the page counter
(as well as other status parameters which are stored in the |.aux| files)
takes the same value after the conditional processing.
Otherwise the page numbers may take divergent values
depending on which part is compiled.

For example, a title page could be declared by:
%
\begin{center}
\begin{tabular}{l}
|\ifchilddoc\||else|\\
|\addtocounter{page}{-1}|\\
\textit{code for title page}\\
|\newpage|\\
|\||fi|
\end{tabular}
\end{center}
%
A banner page for the child documents can be generated by:
%
\begin{center}
\begin{tabular}{l}
|\ifchilddoc|\\
|\addtocounter{page}{-1}|\\
\textit{code for banner page}\\
|\newpage|\\
|\||fi|
\end{tabular}
\end{center}
%
Here one could write a message such as:
\begin{center}
|This is the part \childdocname{} of \childdocjob{}.|
\end{center}

%%%%%%%%%%%%%%%%%%%%%%%%%%%%%%%%%%%%%%%%%%%%%%%%%%%%%%%%%%%%%%%%%%%%%%%%%%%%%%%%
\subsection{Flags}
\label{sec:flags}

The package makes it easy to generate different versions
of the main or child documents.
To this end compilation flags can be defined
and assigned different default values.
They will be particularly useful in conjunction
with the forwarding mechanism described in \secref{sec:forward}.

For example, it may be useful to have a flag |\version|
which can be set to |draft| or |final|.
The document source will contain some conditional code
depending on the value of |\version|.
Suppose further, the flag should default to |final| for the main file
and to |draft| for child files
which is a natural assignment for editing the document.
This is achieved by placing the following code
in the preamble of the main document
(below the |\childdocmain| directive):
%
\begin{center}
\begin{tabular}{l}
|\ifchilddoc|\\
|\providecommand{\version}{draft}|\\
|\||else|\\
|\providecommand{\version}{final}|\\
|\||fi|
\end{tabular}
\end{center}
%
The definition by |\providecommand| makes sure
that previous definitions are not overwritten.
Further statements |\providecommand{\version}{...}|
can thus be added before the above code to override it.

For the main file, one might add a line
(between |\childdocmain| and the above block)
%
\begin{center}
|%\ifchilddoc\||else\providecommand{\version}{draft}\||fi|
\end{center}
%
which can be uncommented to produce a draft version.
Likewise one can add a line to the very top of a child file
(above the |\childdocof{|\textit{main}|}| directive)
%
\begin{center}
|%\providecommand{\version}{final}|
\end{center}
%
which can be uncommented to produce the final version of this child document.

%%%%%%%%%%%%%%%%%%%%%%%%%%%%%%%%%%%%%%%%%%%%%%%%%%%%%%%%%%%%%%%%%%%%%%%%%%%%%%%%
\subsection{Forwarding}
\label{sec:forward}

Different versions of the main or child documents
using compilation flags as described in \secref{sec:flags}
can be (permanently) stored in different files
for convenient compilation, viewing and distribution.
To this end, the package defines a command
to pass on compilation to a different file:

%%%%%%%%%%%%%%%%%%%%%%%%%%%%%%%%%%%%%%%%
\DescribeMacro{\childdocforward}
The command |\childdocforward| redirects processing to
another source file:
%
\begin{center}
\begin{tabular}{l}
|\input{childdoc.def}|\\
|\childdocforward[|\textit{main}|]{|\textit{dest}|}|\\
\end{tabular}
\end{center}
%
The argument \textit{dest} is the destination file
(without extension).
It should be the main file or one of the child files.
Note that further \textsf{childdoc} directives
such as |\childdocof| and |\childdocforward|
in the indicated file will be processed in this form.
The optional argument \textit{main}
passes on directly to the main file \textit{main}
while pretending to compile the child \textit{dest}.
This form behaves as if \textit{dest}
issues |\childdocof{|\textit{main}|}| right away,
and no further \textsf{childdoc} directives will be processed.

%%%%%%%%%%%%%%%%%%%%%%%%%%%%%%%%%%%%%%%%
\DescribeMacro{\...prefix}
In the alternative form |\childdocforwardprefix|,
%
\begin{center}
\begin{tabular}{l}
|\input{childdoc.def}|\\
|\childdocforwardprefix[|\textit{main}|]{|\textit{prefix}|}{|\textit{dest}|}|
\end{tabular}
\end{center}
%
the destination file is determined by a pattern
depending on the current file:
To make this work, the current file must be called
`{\textit{prefix}\hspace{0.2em}\textit{suffix}}'
with \textit{prefix} matching precisely the argument.
Processing is then passed on to the file
`{\textit{dest}\hspace{0.2em}\textit{suffix}}'.
Surely, the same effect is achieved by
directly specifying the
argument `{\textit{dest}\hspace{0.2em}\textit{suffix}}'
in the first form.
However, that requires to set up a different file
for each child. With the alternative form of the command
all these files can have exactly the same content
which simplifies setting them up and maintaining them.

For example, the following file |draft.tex|
with a compilation flag |\version| as described in \secref{sec:flags}
compiles the main document as a draft:
%
\begin{center}
\begin{tabular}{l}
|\def\version{draft}|\\
|\input{childdoc.def}|\\
|\childdocforward{|\textit{main}|}|
\end{tabular}
\end{center}
%
Likewise, the following files |final|\textit{nn}|.tex|
compile the final version of the child document
|child|\textit{nn}|.tex|:
%
\begin{center}
\begin{tabular}{l}
|\def\version{final}|\\
|\input{childdoc.def}|\\
|\childdocforwardprefix{final}{child}|
\end{tabular}
\end{center}
%

Note that when several versions of a main file and/or of each child file
are to be generated, it may be convenient to set up a |Makefile| or
shell script to automatise the process.

%%%%%%%%%%%%%%%%%%%%%%%%%%%%%%%%%%%%%%%%%%%%%%%%%%%%%%%%%%%%%%%%%%%%%%%%%%%%%%%%
\subsection{Command Line Processing}
\label{sec:commandline}

The effect of redirection files can also be achieved by invoking
the \LaTeX{} compiler with a more elaborate command line.
Most conveniently this should be done as part
of a shell script or a |Makefile|.

When using \textsf{childdoc} in the main file, the following
command lines effectively perform a redirection
(note that depending on the shell being used,
backslashes may have to be doubled: `|\|' $\to$ `|\\|'):
%
\begin{center}
|... -jobname "|\textit{target}|" |\\|"|[\textit{flags}]%
|\input{childdoc.def}\childdocforward[|\textit{main}|]{|\textit{dest}|}"|
\end{center}
%
Here \textit{target} is the name of the output file,
\textit{main} is the name of the main file
and \textit{dest} is the name of the main or child file to be processed
(all filenames without extensions).
The optional argument \textit{main} can be omitted
if \textit{main} matches \textit{dest}.
Optionally, compilation \textit{flags} can be defined via |\def| commands.
This command line makes the \TeX{} engine believe
it is compiling the file \textit{target}
whose content is specified as the latter parameter.
The provided code then forwards the processing to
\textit{main} or \textit{dest} as described in \secref{sec:forward}.

%%%%%%%%%%%%%%%%%%%%%%%%%%%%%%%%%%%%%%%%%%%%%%%%%%%%%%%%%%%%%%%%%%%%%%%%%%%%%%%%
\subsection{Include by Input}
\label{sec:input}

Including child documents by |\include| has some restrictions by design.
Most notably, the content of a child document always occupies
its own set of pages; pages cannot be shared between child documents.
Usually, this behaviour makes perfect sense
because each child document contain an essential part of the document.
However, in some situations it may be desirable to compose
a document from a collection of parts
without having mandatory page breaks between then.
For this case, the package
provides a mechanism to include parts
by |\input| which can also be processed individually.
However, by construction this mechanism
requires manual handling of the content to be output.

%%%%%%%%%%%%%%%%%%%%%%%%%%%%%%%%%%%%%%%%
\DescribeMacro{\ifchilddocmanual}
The main file should be prepared as usual, see \secref{sec:include}.
However, the document body must make a distinction
between processing of an individual part and of the main document, e.g.:
%
\begin{center}
\begin{tabular}{l}
|\ifchilddocmanual|\\
|\input{\childdocname}|\\
|\||else|\\
\textit{document body with }|\input{|\textit{part}|}|\\
|\||fi|
\end{tabular}
\end{center}
%
The conditional |\ifchilddocmanual| is true whenever
a part to be included by |\input| is being compiled,
and the name of the part is stored in |\childdocname|.

%%%%%%%%%%%%%%%%%%%%%%%%%%%%%%%%%%%%%%%%
\DescribeMacro{\childdocby}
Each part to be included by |\input| should start with:
%
\begin{center}
\begin{tabular}{l}
|\input{childdoc.def}|\\
|\childdocby{|\textit{main}|}|\\
\end{tabular}
\end{center}
%
The directive |\childdocby| is similar to |\childdocof|
described in \secref{sec:include},
but the subsequent selection of content must be done manually.
To that end, both |\ifchilddoc| and |\ifchilddocmanual|
will be true upon processing of a part,
and the name of the part is stored in |\childdocname|.
Note that |\jobname| will be set to the filename of the current part
so that each part receives an individual |.aux| file
that does not interfere with the |.aux| file(s) of the main document.
This behaviour can be altered by the alternative form
|\childdocby[*]{|\textit{main}|}| (with a non-empty optional argument)
which uses the |.aux| file of the main document
by setting |\jobname| to \textit{main}.

%%%%%%%%%%%%%%%%%%%%%%%%%%%%%%%%%%%%%%%%%%%%%%%%%%%%%%%%%%%%%%%%%%%%%%%%%%%%%%%%
\subsection{Driver Development}
\label{sec:driver}

The \textsf{childdoc} mechanism can also be use for the development
of definition files such as \LaTeX{} styles or classes.
This case differs from the above setup with multiple parts
included by |\include| in that no |\includeonly| should be invoked.
This can be achieved by starting the include file
(before |\ProvidesPackage|) with:
%
\begin{center}
\begin{tabular}{l}
|\input{childdoc.def}|\\
|\childdocforward{|\textit{main}|}|\\
\end{tabular}
\end{center}
%
or alternatively with:
%
\begin{center}
\begin{tabular}{l}
|\input{childdoc.def}|\\
|\childdocby{|\textit{main}|}|\\
\end{tabular}
\end{center}
%
Both forms have slightly different effects as described above.
The main file is prepared as usual, see \secref{sec:include}.

%%%%%%%%%%%%%%%%%%%%%%%%%%%%%%%%%%%%%%%%%%%%%%%%%%%%%%%%%%%%%%%%%%%%%%%%%%%%%%%%
\subsection{Legacy Detection}
\label{sec:detection}

The directive |\childdocmain| in the main file can detect
whether the complete document or merely a child is to be compiled
even without using the directive |\childdocof|.
This method is deprecated because it is less robust
and there is no compelling reason to use it;
it is merely provided for backward compatibility
and it may be removed in future versions.

If the detection mechanism is to be used,
it is mandatory to correctly specify
the filename of the main file as the argument of |\childdocmain|:
%
\begin{center}
\begin{tabular}{l}
|\input{childdoc.def}|\\
|\childdocmain{|\textit{main}|}|\\
\end{tabular}
\end{center}
%
If |\jobname| does not match the argument \textit{main} of |\childdocmain|,
it is assumed that |\jobname| points to the child file to be compiled.
When using |\childdocmain| with the main file specified as argument,
it suffices to start a child file
with just |\input{|\textit{main}|}|
without loading of the package and using |\childdocof|.
If instead all processing is done
with the appropriate \textsf{childdoc} directives,
the argument of \textit{main} of |\childdocmain| can be empty.

An alternative version of the command line processing described
in \secref{sec:commandline} using the detection mechanism reads:
%
\begin{center}
|... -jobname "|\textit{target}|" "|[\textit{flags}]%
[|\def\jobname{|\textit{dest}|}|]|\input{|\textit{main}|}"|
\end{center}

%%%%%%%%%%%%%%%%%%%%%%%%%%%%%%%%%%%%%%%%%%%%%%%%%%%%%%%%%%%%%%%%%%%%%%%%%%%%%%%%
\subsection{Manual Code}
\label{sec:manual}

In case one cannot be certain whether the definitions file |childdoc.def|
is installed on the target \TeX{} distribution
and one prefers not to ship it,
it is conceivable to paste a few relevant commands into the sources.

To that end, drop all statements |\input{childdoc.def}|
and perform the replacements as outlined below.
Instead of |\childdocmain{|\textit{main}|}| add the following code
to the top of the main file:
%
\begin{center}
\begin{tabular}{l}
|\||ifdefined\childdocname\endinput\||fi\newif\ifchilddoc|\\
|\edef\childdocname{\scantokens\expandafter{\jobname\noexpand}}|\\
|\def\childdocmain{|\textit{main}|}\||ifx\childdocmain\childdocname\||else|\\
|\childdoctrue\includeonly{\childdocname}\let\jobname\childdocmain\||fi|\\
\end{tabular}
\end{center}
%
Instead of |\childdocof{|\textit{main}|}| just include the main file
at the top of each child file:
%
\begin{center}
|\input{|\textit{main}|}|
\end{center}
%
A simple redirection |\childdocforward{|\textit{dest}|}| is achieved by:
%
\begin{center}
|\def\jobname{|\textit{dest}|}\input{\jobname}|
\end{center}
%
The redirection with prefix
|\childdocforwardprefix[|\textit{prefix}|]{|\textit{dest}|}|
is accomplished by:
%
\begin{center}
\begin{tabular}{l}
|{\edef\jobname{\scantokens\expandafter{\jobname\noexpand}}|\\
|\def\redirectjob |\textit{prefix}|#1~~~{\gdef\jobname{|\textit{dest}|#1}}|\\
|\expandafter\redirectjob\jobname~~~}\input{\jobname}|
\end{tabular}
\end{center}

In an alternative approach,
child documents can be compiled by a specific command line
without additional code or specific definitions:
%
\begin{center}
|... -jobname "|\textit{target}|" "|[\textit{flags}]%
|\includeonly{|\textit{dest}|}\input{|\textit{main}|}"|
\end{center}
%

%%%%%%%%%%%%%%%%%%%%%%%%%%%%%%%%%%%%%%%%%%%%%%%%%%%%%%%%%%%%%%%%%%%%%%%%%%%%%%%%
%%%%%%%%%%%%%%%%%%%%%%%%%%%%%%%%%%%%%%%%%%%%%%%%%%%%%%%%%%%%%%%%%%%%%%%%%%%%%%%%
\section{Information}

%%%%%%%%%%%%%%%%%%%%%%%%%%%%%%%%%%%%%%%%%%%%%%%%%%%%%%%%%%%%%%%%%%%%%%%%%%%%%%%%
\subsection{Copyright}

Copyright \copyright{} 2017--2018 Niklas Beisert

This work may be distributed and/or modified under the
conditions of the \LaTeX{} Project Public License, either version 1.3
of this license or (at your option) any later version.
The latest version of this license is in
  \url{http://www.latex-project.org/lppl.txt}
and version 1.3 or later is part of all distributions of \LaTeX{}
version 2005/12/01 or later.

This work has the LPPL maintenance status `maintained'.

The Current Maintainer of this work is Niklas Beisert.

This work consists of the files |README.txt|, |childdoc.ins| and |childdoc.dtx|
as well as the derived files |childdoc.def|, |cdocsamp.tex|
with |cdocsch1.tex|, |cdocsch2.tex|, |cdocspt3.tex|, |cdocspt4.tex|,
|cdocsdrf.tex|, |cdocsfn1.tex|, |cdocsfn2.tex|
as well as |childdoc.pdf|.

%%%%%%%%%%%%%%%%%%%%%%%%%%%%%%%%%%%%%%%%%%%%%%%%%%%%%%%%%%%%%%%%%%%%%%%%%%%%%%%%
\subsection{Files and Installation}

The package consists of the files:
%
\begin{center}
\begin{tabular}{ll}
    |README.txt|   & readme file \\
    |childdoc.ins| & installation file \\
    |childdoc.dtx| & source file \\
    |childdoc.def| & definition file \\
    |cdocsamp.tex| & sample main file \\
    |cdocsch1.tex| & sample include file \\
    |cdocsch2.tex| & sample include file \\
    |cdocspt3.tex| & sample part file \\
    |cdocspt4.tex| & sample part file \\
    |cdocsdrf.tex| & sample redirection file \\
    |cdocsfn1.tex| & sample redirection file \\
    |cdocsfn2.tex| & sample redirection file \\
    |childdoc.pdf| & manual
\end{tabular}
\end{center}
%
The distribution consists of the files
|README.txt|, |childdoc.ins| and |childdoc.dtx|.
%
\begin{itemize}
\item
Run (pdf)\LaTeX{} on |childdoc.dtx|
to compile the manual |childdoc.pdf| (this file).
\item
Run \LaTeX{} on |childdoc.ins| to create the definitions file |childdoc.def|
and the sample |cdocsamp.tex| with include files
|cdocsch1.tex|, |cdocsch2.tex|, |cdocspt3.tex|, |cdocspt4.tex|,
|cdocsdrf.tex|, |cdocsfn1.tex|, |cdocsfn2.tex|.
Then copy the file |childdoc.def| to an appropriate directory of your \LaTeX{}
distribution, e.g.\ \textit{texmf-root}|/tex/latex/childdoc|.
\end{itemize}

%%%%%%%%%%%%%%%%%%%%%%%%%%%%%%%%%%%%%%%%%%%%%%%%%%%%%%%%%%%%%%%%%%%%%%%%%%%%%%%%
\subsection{Related CTAN Packages}

There are several other packages which offer a similar functionality:
%
\begin{itemize}
\item
The packages
\href{http://ctan.org/pkg/docmute}{\textsf{docmute}},
\href{http://ctan.org/pkg/includex}{\textsf{includex}} and
\href{http://ctan.org/pkg/standalone}{\textsf{standalone}}
provide commands to include only the document body of
a child file thus allowing both files to be compiled individually.
\item
The packages \href{http://ctan.org/pkg/subdocs}{\textsf{subdocs}}
and \href{http://ctan.org/pkg/subfiles}{\textsf{subfiles}}
provide structures in which the main and child documents can be
encapsulated and allowing them to be compiled individually.
The inclusion mechanism is different from the conventional |\include|.
\item
The package \href{http://ctan.org/pkg/combine}{\textsf{combine}}
is an elaborate solution to combine several documents into one.
\end{itemize}
%
See also the CTAN topic \href{http://ctan.org/topic/subdocs}{\textsf{subdocs}}
for further related packages.
The present package differs from the above solutions in that
a document structure constructed with the conventional |\include| mechanism
just needs two extra commands at the top of every file
such that all constituent files can be compiled individually.

%%%%%%%%%%%%%%%%%%%%%%%%%%%%%%%%%%%%%%%%%%%%%%%%%%%%%%%%%%%%%%%%%%%%%%%%%%%%%%%%
%\subsection{Feature Suggestions}
%
%The following is a list of features which may be useful for future
%versions of this package:
%%
%\begin{itemize}
%\item
%\ldots
%\end{itemize}

%%%%%%%%%%%%%%%%%%%%%%%%%%%%%%%%%%%%%%%%%%%%%%%%%%%%%%%%%%%%%%%%%%%%%%%%%%%%%%%%
\subsection{Revision History}

%%%%%%%%%%%%%%%%%%%%%%%%%%%%%%%%%%%%%%%%
\paragraph{v2.0:} 2018/12/30

\begin{itemize}
\item
immediate forward processing
\item
added |\childdocby| mechanism
\item
manual restructured
\end{itemize}

%%%%%%%%%%%%%%%%%%%%%%%%%%%%%%%%%%%%%%%%
\paragraph{v1.6:} 2018/01/17

\begin{itemize}
\item
application for development of include files
\item
corrections to manual
\end{itemize}

%%%%%%%%%%%%%%%%%%%%%%%%%%%%%%%%%%%%%%%%
\paragraph{v1.5:} 2017/05/21

\begin{itemize}
\item
more complete structuring introduced
\item
|\childdocof| introduced
\item
|\childdoc| renamed to |\childdocmain|
\item
|\childredirect| renamed to |\childdocforward| and |\childdocforwardprefix|
and functionality expanded
\end{itemize}

%%%%%%%%%%%%%%%%%%%%%%%%%%%%%%%%%%%%%%%%
\paragraph{v1.0:} 2017/04/27

\begin{itemize}
\item
manual and install package
\item
first version published on CTAN
\end{itemize}

%%%%%%%%%%%%%%%%%%%%%%%%%%%%%%%%%%%%%%%%
\paragraph{v0.6:} 2017/04/26

\begin{itemize}
\item
redirection mechanism added
\end{itemize}

%%%%%%%%%%%%%%%%%%%%%%%%%%%%%%%%%%%%%%%%
\paragraph{v0.5:} 2017/04/26

\begin{itemize}
\item
functionality in definition file
\end{itemize}


%%%%%%%%%%%%%%%%%%%%%%%%%%%%%%%%%%%%%%%%%%%%%%%%%%%%%%%%%%%%%%%%%%%%%%%%%%%%%%%%
%%%%%%%%%%%%%%%%%%%%%%%%%%%%%%%%%%%%%%%%%%%%%%%%%%%%%%%%%%%%%%%%%%%%%%%%%%%%%%%%
%%%%%%%%%%%%%%%%%%%%%%%%%%%%%%%%%%%%%%%%%%%%%%%%%%%%%%%%%%%%%%%%%%%%%%%%%%%%%%%%
\appendix

\settowidth\MacroIndent{\rmfamily\scriptsize 000\ }

 \DocInput{childdoc.dtx}

\end{document}
%</driver>
% \fi
%
% %%%%%%%%%%%%%%%%%%%%%%%%%%%%%%%%%%%%%%%%%%%%%%%%%%%%%%%%%%%%%%%%%%%%%%%%%%%%%%
% %%%%%%%%%%%%%%%%%%%%%%%%%%%%%%%%%%%%%%%%%%%%%%%%%%%%%%%%%%%%%%%%%%%%%%%%%%%%%%
% \section{Sample}
%\iffalse
%<*samplemain>
%\fi
%
% The following presents a sample document
% with two chapters, two parts, a title page,
% a compile flag as well as three forwarding files to set the flag.
% It consists of eight |.tex| files:
% \begin{center}
% \begin{tabular}{ll}
% |cdocsamp.tex|&main file\\
% |cdocsch1.tex|&include file for chapter 1\\
% |cdocsch2.tex|&include file for chapter 2\\
% |cdocspt3.tex|&include file for part 3\\
% |cdocspt4.tex|&include file for part 4\\
% |cdocsdrf.tex|&forwarding file for main file in draft mode\\
% |cdocsfi1.tex|&forwarding file for final version of chapter 1\\
% |cdocsfi2.tex|&forwarding file for final version of chapter 2\\
% \end{tabular}
% \end{center}
% Each of the eight files can be compiled directly by the \LaTeX{} compiler.
%
% %%%%%%%%%%%%%%%%%%%%%%%%%%%%%%%%%%%%%%
% \paragraph{Main File.}
%
% The main file is called |cdocsamp.tex|.
%
% Load the \textsf{childdoc} definitions and
% declare the filename for the main document:
%    \begin{macrocode}
\input{childdoc.def}
\childdocmain{}
%    \end{macrocode}

% Optional override for |\version| flag:
%    \begin{macrocode}
%%\ifchilddoc\else\providecommand{\version}{draft}\fi
%    \end{macrocode}

% Define the default values for the |\version| flag
% (|final| for the main file and |draft| for childs):
%    \begin{macrocode}
\ifchilddoc
\providecommand{\version}{draft}
\else
\providecommand{\version}{final}
\fi
%    \end{macrocode}

% Load the standard document class:
%    \begin{macrocode}
\documentclass[12pt]{article}
%    \end{macrocode}

% Start the document body:
%    \begin{macrocode}
\begin{document}
%    \end{macrocode}

% Declare a title page.
% Print title, part of document being processed and version flag:
%    \begin{macrocode}
\addtocounter{page}{-1}
\begin{center}
{\LARGE\bfseries{}childdoc example\par}
\vspace{1cm}
\ifchilddoc
\ifchilddocmanual part\else chapter\fi:
`\childdocname' of `\childdocjob'\par
\else
main document: `\childdocjob'\par
\fi
version: \version\par
\end{center}
\newpage
%    \end{macrocode}

% Manually include selected file,
% otherwise process as usual:
%    \begin{macrocode}
\ifchilddocmanual
\section*{part `\childdocname'}
\input{\childdocname}
\else
%    \end{macrocode}

% Include the two chapters:
%    \begin{macrocode}
\include{cdocsch1}
\include{cdocsch2}
%    \end{macrocode}

% Include the two parts unless only chapters should be displayed:
%    \begin{macrocode}
\ifchilddoc\else
\section{part three}
\input{cdocspt3}
\section{part four}
\input{cdocspt4}
\fi
%    \end{macrocode}

% Process as usual until here:
%    \begin{macrocode}
\fi
%    \end{macrocode}

% End of document body:
%    \begin{macrocode}
\end{document}
%    \end{macrocode}
%\iffalse
%</samplemain>
%\fi
%
% %%%%%%%%%%%%%%%%%%%%%%%%%%%%%%%%%%%%%%
% \paragraph{Chapter Include Files.}
%
% The include files are called |cdocsch1.tex| and |cdocsch2.tex|.
%
%\iffalse
%<*samplechap1|samplechap2>
%\fi

% Optional override for |\version| flag:
%    \begin{macrocode}
%%\providecommand{\version}{final}
%    \end{macrocode}

% Include the main document:
%    \begin{macrocode}
\input{childdoc.def}
\childdocof{cdocsamp}
%    \end{macrocode}

%\iffalse
%</samplechap1|samplechap2>
%\fi
%
%\iffalse
%<*samplechap1>
%\fi
% Some text for chapter 1:
%    \begin{macrocode}
\section{one}
some text in chapter one
%    \end{macrocode}

%\iffalse
%</samplechap1>
%\fi
% Some text for chapter 2:
%\iffalse
%<*samplechap2>
%\fi
%    \begin{macrocode}
\section{two}
more text in chapter two
%    \end{macrocode}

%\iffalse
%</samplechap2>
%\fi
%
% %%%%%%%%%%%%%%%%%%%%%%%%%%%%%%%%%%%%%%
% \paragraph{Part Include Files.}
%
% The include files are called |cdocspt3.tex| and |cdocspt4.tex|.
%
%\iffalse
%<*samplepart3|samplepart4>
%\fi

% Optional override for |\version| flag:
%    \begin{macrocode}
%%\providecommand{\version}{final}
%    \end{macrocode}

% Include the main document:
%    \begin{macrocode}
\input{childdoc.def}
\childdocby{cdocsamp}
%    \end{macrocode}

%\iffalse
%</samplepart3|samplepart4>
%\fi
%
%\iffalse
%<*samplepart3>
%\fi
% Some text for part 3:
%    \begin{macrocode}
some text in part three
%    \end{macrocode}

%\iffalse
%</samplepart3>
%\fi
% Some text for part 4:
%\iffalse
%<*samplepart4>
%\fi
%    \begin{macrocode}
more text in part four
%    \end{macrocode}

%\iffalse
%</samplepart4>
%\fi
%
% %%%%%%%%%%%%%%%%%%%%%%%%%%%%%%%%%%%%%%
% \paragraph{Forwarding for a Complete Draft.}
%
% The following forwarding file |cdocsdrf.tex|
% compiles the main document in draft mode:
%\iffalse
%<*sampledraft>
%\fi
%    \begin{macrocode}
\def\version{draft}
\input{childdoc.def}
\childdocforward{cdocsamp}
%    \end{macrocode}

%\iffalse
%</sampledraft>
%\fi
%
% %%%%%%%%%%%%%%%%%%%%%%%%%%%%%%%%%%%%%%
% \paragraph{Forwarding for Final Version of the Chapters.}
%
% The following forwarding files |cdocsfn1.tex| and |cdocsfn2.tex|
% (with identical content)
% compile the final versions of the child documents
% |cdocsch1.tex| and |cdocsch2.tex|, respectively:
%\iffalse
%<*samplefinal>
%\fi
%    \begin{macrocode}
\def\version{final}
\input{childdoc.def}
\childdocforwardprefix[cdocsamp]{cdocsfn}{cdocsch}
%    \end{macrocode}

%\iffalse
%</samplefinal>
%\fi
%
% %%%%%%%%%%%%%%%%%%%%%%%%%%%%%%%%%%%%%%
% \paragraph{Command Line Processing.}
%
% The following three command lines generate the output files
% |cdocscld|, |cdocscl1| and |cdocscl2|
% which should be identical to
% |cdocsdrf|, |cdocsch1| and |cdocsfn2|, respectively:
% \begin{center}
% \begin{tabular}{l}
% |latex -jobname cdocscld \|\\
% |  "\def\version{draft}\input{childdoc.def}\childdocforward{cdocsamp}"|\\
% |latex -jobname cdocscl1 \|\\
% |  "\input{childdoc.def}\childdocforward[cdocsamp]{cdocsch1}"|\\
% |latex -jobname cdocscl2 \|\\
% |  "\def\version{final}\input{childdoc.def}\childdocforward{cdocsch2}"|
% \end{tabular}
% \end{center}
% Note that the trailing backslash on each first line
% merely continues the input to the second line
% (for convenient cut ant paste).
% Furthermore, the command |latex| can be replaced by any
% of its alternative versions such as |pdflatex|.
%
% %%%%%%%%%%%%%%%%%%%%%%%%%%%%%%%%%%%%%%%%%%%%%%%%%%%%%%%%%%%%%%%%%%%%%%%%%%%%%%
% %%%%%%%%%%%%%%%%%%%%%%%%%%%%%%%%%%%%%%%%%%%%%%%%%%%%%%%%%%%%%%%%%%%%%%%%%%%%%%
% \section{Implementation}
%\iffalse
%<*package>
%\fi
%
% This section describes the definitions file |childdoc.def|.

% The definitions cannot be loaded using |\usepackage| or |\RequirePackage|
% which has a mechanism to prevent loading a style file more than once.
% When loading the definitions by means of |\input|
% multiple instances have to be prevented manually:
%\iffalse
%This code needs to be before the `\ProvidesFile' directive
%which is defined at the beginning of this file.
%Therefore it is also placed there and commented out here.
%</package>
%<*discard>
%\fi
%    \begin{macrocode}
\ifdefined\childdocmain\endinput\fi
%    \end{macrocode}
%\iffalse
%</discard>
%<*package>
%\fi
%
% \macro{\ifchilddoc}
% \macro{\ifchilddocmanual}
% The conditional |\ifchilddoc| tells whether a
% child (true) or main (false) document is being compiled.
% The conditional |\ifchilddocmanual| tells whether
% the |\includeonly| mechanism is used (false) or
% the selection of child files must be performed manually (true).
% The definitions initialise to false:
%    \begin{macrocode}
\newif\ifchilddoc
\newif\ifchilddocmanual
%    \end{macrocode}

% \macro{\childdocname}
% \macro{\childdocjob}
% The macro |\childdocname| stores the name of the main document
% to be compiled. The macro |\childdocjob| stores the name of
% the document on which the \LaTeX{} compiler was originally invoked.
% The content of |\jobname| cannot be compared
% to filenames specified in the source due to different catcodes.
% The following code rescans |\jobname|, stores the result
% in |\childdocname| and saves a copy in |\childdocjob|:
%    \begin{macrocode}
\edef\childdocname{\scantokens\expandafter{\jobname\noexpand}}
\let\childdocjob\childdocname
%    \end{macrocode}

% \macro{\childdocdisable}
% The macro |\childdocdisable| prevents the main file
% from being processed more than once.
% At this stage, the main document command |\childdocmain|
% is assumed to be called once again where it should do nothing.
% Any subsequent call to it should prevent
% a secondary processing of the main document
% It overwrites the forwarding commands
% |\childdocof| and |\childdocforward|
% with empty macros to prevent further inclusions of the main document:
%    \begin{macrocode}
\newcommand{\childdocdisable}
{
  \renewcommand{\childdocmain}[1]{\renewcommand{\childdocmain}[1]{\endinput}}
  \renewcommand{\childdocof}[1]{}
  \renewcommand{\childdocby}[2][]{}
  \renewcommand{\childdocforward}[2][]{}
  \renewcommand{\childdocdisable}{}
}
%    \end{macrocode}

% \macro{\childdocmain}
% The macro |\childdocmain| is to be called at the top of the main file
% with nothing or the main filename (without extension) as argument.
% First, it breaks loops.
% If the argument is not empty and does not match |\childdocname|
% (which is set by the first inclusion of |childdoc.def|),
% |\ifchilddoc| is set to true, |\includeonly| is applied to the child file
% and |\jobname| is set to the main file
% (for proper handling of |.aux| files):
%    \begin{macrocode}
\newcommand{\childdocmain}[1]
{
  \childdocdisable\childdocmain{}
  \if?#1?\else
    \begingroup
      \def\childdoctmp{#1}
      \ifx\childdoctmp\childdocname
        \def\childdoctmp{}
      \else
        \def\childdoctmp
        {
          \childdoctrue
          \includeonly{\childdocname}
          \def\childdocjob{#1}
          \def\jobname{#1}
        }
      \fi
      \expandafter
    \endgroup
    \childdoctmp
  \fi
}
%    \end{macrocode}

% \macro{\childdocof}
% The command |\childdocof| redirects
% compilation to the main file |#1|.
%    \begin{macrocode}
\newcommand{\childdocof}[1]
{
  \childdocdisable
  \childdoctrue
  \includeonly{\childdocname}
  \def\jobname{#1}
  \def\childdocjob{#1}
  \input{#1}
}
%    \end{macrocode}

% \macro{\childdocby}
% The command |\childdocby| ....
%    \begin{macrocode}
\newcommand{\childdocby}[2][]
{
  \childdocdisable
  \childdoctrue
  \childdocmanualtrue
  \if?#1?\else
    \def\jobname{#2}
  \fi
  \def\childdocjob{#2}
  \input{#2}
  \endinput
}
%    \end{macrocode}

% \macro{\childdocforward}
% The command |\childdocforward| redirects
% compilation to the main file or
% (if the optional argument is given) a child file.
% Parameters are set as if the main file
% or a child file starting with |\childdocof| was compiled.
% Then compilation is handed over to the main file:
%    \begin{macrocode}
\newcommand{\childdocforward}[2][]
{
  \begingroup
    \if?#1?
      \def\childdoctmp
      {
        \def\childdocname{#2}
        \def\childdocjob{#2}
        \def\jobname{#2}
        \input{#2}
        \endinput
      }
    \else
      \def\childdoctmp
      {
        \childdocdisable
        \def\childdocname{#2}
        \childdoctrue
        \includeonly{#2}
        \def\childdocjob{#1}
        \def\jobname{#1}
        \input{#1}
        \endinput
      }
    \fi
    \expandafter
  \endgroup
  \childdoctmp
}
%    \end{macrocode}

% \macro{\childdocforwardprefix}
% The command |\childdocforwardprefix| redirects
% compilation to the main or a child file by means of a pattern.
% The prefix |#1| in the current filename is replaced by |#2|
% and the suffix of the current filename is kept
% (it is assumed that the filename does not contain the substring `|~~~|'
% which is used as a delimiter).
% Compilation is handed over to the new file by |\childdocforward|:
%    \begin{macrocode}
\newcommand{\childdocforwardprefix}[3][]
{
  \begingroup
    \def\childdocextract #2##1~~~{\def\childdoctmp{\childdocforward[#1]{#3##1}}}
    \expandafter\childdocextract\childdocname~~~
    \expandafter
  \endgroup
  \childdoctmp
}
%    \end{macrocode}

% \macro{\childdoc}
% The deprecated macro |\childdoc| is a legacy version of |\childdocmain|:
%    \begin{macrocode}
\newcommand{\childdoc}{\childdocmain}
%    \end{macrocode}

% \macro{\childdocredirect}
% The deprecated macro |\childdocredirect| is a legacy version
% of |\childdocforward| and |\childdocforwardprefix|:
%    \begin{macrocode}
\newcommand{\childdocredirect}[2][]
{
  \begingroup
    \if?#1?
      \def\childdoctmp{\childdocforward{#2}}
    \else
      \def\childdoctmp{\childdocforwardprefix{#1}{#2}}
    \fi
    \expandafter
  \endgroup
  \childdoctmp
}
%    \end{macrocode}

%\iffalse
%</package>
%\fi
%
\endinput
\childdocforward{cdocsamp}"|\\
% |latex -jobname cdocscl1 \|\\
% |  "% \iffalse
%
% childdoc.dtx Copyright (C) 2017-2018 Niklas Beisert
%
% This work may be distributed and/or modified under the
% conditions of the LaTeX Project Public License, either version 1.3
% of this license or (at your option) any later version.
% The latest version of this license is in
%   http://www.latex-project.org/lppl.txt
% and version 1.3 or later is part of all distributions of LaTeX
% version 2005/12/01 or later.
%
% This work has the LPPL maintenance status `maintained'.
%
% The Current Maintainer of this work is Niklas Beisert.
%
% This work consists of the files childdoc.dtx and childdoc.ins
% and the derived files childdoc.def and cdocsamp.tex with
% cdocsch1.tex, cdocsch2.tex, cdocsdrf.tex, cdocsfn1.tex, cdocsfn2.tex.
%
%<package>\ifdefined\childdocmain\endinput\fi
%<package>\ProvidesFile{childdoc.def}[2018/12/30 v2.0 child document driver]
%<samplemain>\ProvidesFile{cdocsamp.tex}[2018/12/30 v2.0 sample for childdoc]
%<*driver>
%\ProvidesFile{childdoc.drv}[2018/12/30 v2.0 childdoc reference manual file]
\PassOptionsToClass{10pt,a4paper}{article}
\documentclass{ltxdoc}

\usepackage[margin=35mm]{geometry}
\usepackage{hyperref}
\usepackage{hyperxmp}
\usepackage[usenames]{color}

\hypersetup{colorlinks=true}
\hypersetup{pdfstartview=FitH}
\hypersetup{pdfpagemode=UseNone}
\hypersetup{pdfsource={}}
\hypersetup{pdflang={en-UK}}
\hypersetup{pdfcopyright={Copyright 2017-2018 Niklas Beisert.
  This work may be distributed and/or modified under the
  conditions of the LaTeX Project Public License, either version 1.3
  of this license or (at your option) any later version.}}
\hypersetup{pdflicenseurl={http://www.latex-project.org/lppl.txt}}
\hypersetup{pdfcontactaddress={ETH Zurich, ITP, HIT K,
  Wolfgang-Pauli-Strasse 27}}
\hypersetup{pdfcontactpostcode={8093}}
\hypersetup{pdfcontactcity={Zurich}}
\hypersetup{pdfcontactcountry={Switzerland}}
\hypersetup{pdfcontactemail={nbeisert@itp.phys.ethz.ch}}
\hypersetup{pdfcontacturl={http://people.phys.ethz.ch/\xmptilde nbeisert/}}

\newcommand{\secref}[1]{\hyperref[#1]{section \ref*{#1}}}

\parskip1ex
\parindent0pt
\let\olditemize\itemize
\def\itemize{\olditemize\parskip0pt}

\begin{document}

\title{The \textsf{childdoc} Package}
\hypersetup{pdftitle={The childdoc Package}}
\author{Niklas Beisert\\[2ex]
  Institut f\"ur Theoretische Physik\\
  Eidgen\"ossische Technische Hochschule Z\"urich\\
  Wolfgang-Pauli-Strasse 27, 8093 Z\"urich, Switzerland\\[1ex]
  \href{mailto:nbeisert@itp.phys.ethz.ch}
  {\texttt{nbeisert@itp.phys.ethz.ch}}}
\hypersetup{pdfauthor={Niklas Beisert}}
\hypersetup{pdfsubject={Manual for the LaTeX2e Package childdoc}}
\date{30 December 2018, \textsf{v2.0}}
\maketitle

\begin{abstract}\noindent
\textsf{childdoc} is a \LaTeXe{} package
that enables the direct compilation
of document sections included by |\include|
to individual files.
\end{abstract}

\begingroup
\parskip0ex
\tableofcontents
\endgroup

%%%%%%%%%%%%%%%%%%%%%%%%%%%%%%%%%%%%%%%%%%%%%%%%%%%%%%%%%%%%%%%%%%%%%%%%%%%%%%%%
%%%%%%%%%%%%%%%%%%%%%%%%%%%%%%%%%%%%%%%%%%%%%%%%%%%%%%%%%%%%%%%%%%%%%%%%%%%%%%%%
\section{Introduction}

\LaTeX{} provides a mechanism to structure a large document (such as a book)
into a main file and several child files (containing the chapters)
using the |\include| command.
This mechanism is beneficial for documents
which span hundreds of pages in order to
make the source file(s) more manageable.
Moreover, compilation can be restricted to
selected child files by means of the |\includeonly| command.
The latter feature can be used to reduce the compilation time while editing
(this was significantly more useful in the earlier days of \LaTeX{})
or to generate a smaller document which is easier to navigate.
Another application of |\includeonly| is to generate
documents consisting of selected parts of the complete document.

However, there are a few drawbacks of the plain |\include| mechanism:
\begin{itemize}
\item
The child files cannot be compiled on their own,
they can only be compiled via the main file.
A naive editing environment
(such as a text editor with an option
to have the current file processed by \LaTeX)
may require one to switch to the main file before compiling;
attempting to compile the child file produces errors.
\item
The main file must be modified (each time)
to adjust the |\includeonly| command
to the present needs. This easily leaves the main file in a messy state.
\item
The generated document will always carry the filename
of the main document. This is inconvenient if
several child files are to be compiled and
to be kept for distribution.
\end{itemize}

The present package provides a simple interface
to make child files individually compilable by \LaTeX{}.
Compiling a child file then has the same effect as compiling
the main file with an |\includeonly| command
to select the appropriate child.
Moreover the generated document will carry the name of the child
rather than the main file.
This resolves all three above issues.

This feature is meant to make the editing of books,
thesis documents and lecture notes somewhat more convenient.
However, the package can also be used efficiently for
composing a series of documents (such as exercise sheets)
which are typically distributed individually.
It then assists the author in generating the individual documents
(potentially in different versions)
as well as a document containing the collected series.
Another application is in developing style files
or other kinds of included material
where compilation of the style file could redirect
to a sample or test file.

%%%%%%%%%%%%%%%%%%%%%%%%%%%%%%%%%%%%%%%%%%%%%%%%%%%%%%%%%%%%%%%%%%%%%%%%%%%%%%%%
%%%%%%%%%%%%%%%%%%%%%%%%%%%%%%%%%%%%%%%%%%%%%%%%%%%%%%%%%%%%%%%%%%%%%%%%%%%%%%%%
\section{Usage}

First of all, the package \textsf{childdoc} is \emph{not} a standard
\LaTeXe{} |.sty| style file! Therefore it needs to be invoked in
a non-standard way.

%%%%%%%%%%%%%%%%%%%%%%%%%%%%%%%%%%%%%%%%%%%%%%%%%%%%%%%%%%%%%%%%%%%%%%%%%%%%%%%%
\subsection{Included Files}
\label{sec:include}

%%%%%%%%%%%%%%%%%%%%%%%%%%%%%%%%%%%%%%%%
\DescribeMacro{\childdocmain}
To use the package, add the commands
\begin{center}
\begin{tabular}{l}
|\input{childdoc.def}|\\
|\childdocmain{}|\\
\end{tabular}
\end{center}
at the very top of the main \LaTeX{} file,
in particular \emph{before} the |\documentclass| statement!
The argument of |\childdocmain| should be left empty
(but it must be present).

%%%%%%%%%%%%%%%%%%%%%%%%%%%%%%%%%%%%%%%%
\DescribeMacro{\childdocof}
Furthermore, add the commands
\begin{center}
\begin{tabular}{l}
|\input{childdoc.def}|\\
|\childdocof{|\textit{main}|}|\\
\end{tabular}
\end{center}
at the top of every child file \textit{child}
which is included by |\include{|\textit{child}|}|
from within the main file
(or at least for those files to be compiled individually).
The argument \textit{main} must be the filename of the main file.

There are a couple of
considerations in setting up the main and child documents:

%%%%%%%%%%%%%%%%%%%%%%%%%%%%%%%%%%%%%%%%
\paragraph{Restrictions.}

Please note the following restrictions:
\begin{itemize}
\item
|\childdocmain| must be called with one argument \textit{main}
to ensure compatibility with earlier version of the package.
It must either be empty (|\childdocmain{}|)
or precisely match the filename of the main file in which it is specified.
See \secref{sec:detection} for further information.
\item
The filename \textit{main} must be specified without the |.tex| extension.
\item
The filename \textit{main} is case sensitive
(even in case-insensitive file systems)
due to internal string comparison.
\item
The argument \textit{main} should be fully expanded, it cannot be a macro.
\item
Subdirectories and special characters should be avoided in filenames.
\item
The command |\childdocmain{|\textit{main}|}| must be followed by a whitespace.
It should not be followed immediately by another command
or by a comment mark `|%|'.
This is because the \TeX{} parser reads the token immediately following
the argument of |\childdocmain| and puts it
at the beginning of every child section;
however, a white\-space is ignored.
\end{itemize}

%%%%%%%%%%%%%%%%%%%%%%%%%%%%%%%%%%%%%%%%
\paragraph{Content of Main File.}

It is advisable to place all content in the child files included by |\include|.
Any output contained in the main file will appear in all child documents
unless suppressed manually;
it cannot be suppressed automatically by the |\includeonly| directive
and thus should normally be avoided.
A method to include some content in the main file
by means of conditional processing is described in \secref{sec:conditional}.

%%%%%%%%%%%%%%%%%%%%%%%%%%%%%%%%%%%%%%%%
\paragraph{Page Numbering.}

When only a part of the document is compiled,
the appropriate numbering of pages
(as well as other status parameters)
is determined from the |.aux| files.
The latter contain information from previous passes.
However this information needs to propagate through
all intermediate child documents.
Therefore the page numbering in child documents may well
be inconsistent until the complete document is compiled at least once.

A useful (if unconventional) way to always ensure a consistent
page numbering is to restart the numbering in each child document
and denote the pages by `\textit{child}|.|\textit{page}'
where \textit{child} represents the chapter/section number of the child file.
This can be achieved by the command
|\numberwithin{page}{|\textit{child}|}|
of the \textsf{amsmath} package
where \textit{child} can be |chapter| or |section|
depending on the chosen structuring.
Alternatively, one can modify the macro |\thepage| appropriately
and reset the counter |page| at the start of each child file.

%%%%%%%%%%%%%%%%%%%%%%%%%%%%%%%%%%%%%%%%%%%%%%%%%%%%%%%%%%%%%%%%%%%%%%%%%%%%%%%%
\subsection{Conditional Processing}
\label{sec:conditional}

The package provides a mechanism to compile different versions
of a document. To customise the versions further some conditional processing
can come in handy to distinguish which version is being compiled.
The package provides two macros to describe the compilation context:

%%%%%%%%%%%%%%%%%%%%%%%%%%%%%%%%%%%%%%%%
\DescribeMacro{\ifchilddoc}
The conditional |\ifchilddoc| distinguishes between the compilation of
child documents and the main document:
%
\begin{center}
|\ifchilddoc |\textit{child-code}| |[|\||else |\textit{main-code}]| \||fi|
\end{center}

%%%%%%%%%%%%%%%%%%%%%%%%%%%%%%%%%%%%%%%%
\DescribeMacro{\childdocname}
\DescribeMacro{\childdocjob}
The macro |\childdocname| contains the filename (without extension)
of the main or child file being processed.
Note that |\childdocjob| will always contain the name of the main file.

%%%%%%%%%%%%%%%%%%%%%%%%%%%%%%%%%%%%%%%%
\paragraph{Title Page.}

Conditional processing can be used to include a title or banner page
in the main document when proper precautions are taken.
Importantly, the code in the main file should ensure that the page counter
(as well as other status parameters which are stored in the |.aux| files)
takes the same value after the conditional processing.
Otherwise the page numbers may take divergent values
depending on which part is compiled.

For example, a title page could be declared by:
%
\begin{center}
\begin{tabular}{l}
|\ifchilddoc\||else|\\
|\addtocounter{page}{-1}|\\
\textit{code for title page}\\
|\newpage|\\
|\||fi|
\end{tabular}
\end{center}
%
A banner page for the child documents can be generated by:
%
\begin{center}
\begin{tabular}{l}
|\ifchilddoc|\\
|\addtocounter{page}{-1}|\\
\textit{code for banner page}\\
|\newpage|\\
|\||fi|
\end{tabular}
\end{center}
%
Here one could write a message such as:
\begin{center}
|This is the part \childdocname{} of \childdocjob{}.|
\end{center}

%%%%%%%%%%%%%%%%%%%%%%%%%%%%%%%%%%%%%%%%%%%%%%%%%%%%%%%%%%%%%%%%%%%%%%%%%%%%%%%%
\subsection{Flags}
\label{sec:flags}

The package makes it easy to generate different versions
of the main or child documents.
To this end compilation flags can be defined
and assigned different default values.
They will be particularly useful in conjunction
with the forwarding mechanism described in \secref{sec:forward}.

For example, it may be useful to have a flag |\version|
which can be set to |draft| or |final|.
The document source will contain some conditional code
depending on the value of |\version|.
Suppose further, the flag should default to |final| for the main file
and to |draft| for child files
which is a natural assignment for editing the document.
This is achieved by placing the following code
in the preamble of the main document
(below the |\childdocmain| directive):
%
\begin{center}
\begin{tabular}{l}
|\ifchilddoc|\\
|\providecommand{\version}{draft}|\\
|\||else|\\
|\providecommand{\version}{final}|\\
|\||fi|
\end{tabular}
\end{center}
%
The definition by |\providecommand| makes sure
that previous definitions are not overwritten.
Further statements |\providecommand{\version}{...}|
can thus be added before the above code to override it.

For the main file, one might add a line
(between |\childdocmain| and the above block)
%
\begin{center}
|%\ifchilddoc\||else\providecommand{\version}{draft}\||fi|
\end{center}
%
which can be uncommented to produce a draft version.
Likewise one can add a line to the very top of a child file
(above the |\childdocof{|\textit{main}|}| directive)
%
\begin{center}
|%\providecommand{\version}{final}|
\end{center}
%
which can be uncommented to produce the final version of this child document.

%%%%%%%%%%%%%%%%%%%%%%%%%%%%%%%%%%%%%%%%%%%%%%%%%%%%%%%%%%%%%%%%%%%%%%%%%%%%%%%%
\subsection{Forwarding}
\label{sec:forward}

Different versions of the main or child documents
using compilation flags as described in \secref{sec:flags}
can be (permanently) stored in different files
for convenient compilation, viewing and distribution.
To this end, the package defines a command
to pass on compilation to a different file:

%%%%%%%%%%%%%%%%%%%%%%%%%%%%%%%%%%%%%%%%
\DescribeMacro{\childdocforward}
The command |\childdocforward| redirects processing to
another source file:
%
\begin{center}
\begin{tabular}{l}
|\input{childdoc.def}|\\
|\childdocforward[|\textit{main}|]{|\textit{dest}|}|\\
\end{tabular}
\end{center}
%
The argument \textit{dest} is the destination file
(without extension).
It should be the main file or one of the child files.
Note that further \textsf{childdoc} directives
such as |\childdocof| and |\childdocforward|
in the indicated file will be processed in this form.
The optional argument \textit{main}
passes on directly to the main file \textit{main}
while pretending to compile the child \textit{dest}.
This form behaves as if \textit{dest}
issues |\childdocof{|\textit{main}|}| right away,
and no further \textsf{childdoc} directives will be processed.

%%%%%%%%%%%%%%%%%%%%%%%%%%%%%%%%%%%%%%%%
\DescribeMacro{\...prefix}
In the alternative form |\childdocforwardprefix|,
%
\begin{center}
\begin{tabular}{l}
|\input{childdoc.def}|\\
|\childdocforwardprefix[|\textit{main}|]{|\textit{prefix}|}{|\textit{dest}|}|
\end{tabular}
\end{center}
%
the destination file is determined by a pattern
depending on the current file:
To make this work, the current file must be called
`{\textit{prefix}\hspace{0.2em}\textit{suffix}}'
with \textit{prefix} matching precisely the argument.
Processing is then passed on to the file
`{\textit{dest}\hspace{0.2em}\textit{suffix}}'.
Surely, the same effect is achieved by
directly specifying the
argument `{\textit{dest}\hspace{0.2em}\textit{suffix}}'
in the first form.
However, that requires to set up a different file
for each child. With the alternative form of the command
all these files can have exactly the same content
which simplifies setting them up and maintaining them.

For example, the following file |draft.tex|
with a compilation flag |\version| as described in \secref{sec:flags}
compiles the main document as a draft:
%
\begin{center}
\begin{tabular}{l}
|\def\version{draft}|\\
|\input{childdoc.def}|\\
|\childdocforward{|\textit{main}|}|
\end{tabular}
\end{center}
%
Likewise, the following files |final|\textit{nn}|.tex|
compile the final version of the child document
|child|\textit{nn}|.tex|:
%
\begin{center}
\begin{tabular}{l}
|\def\version{final}|\\
|\input{childdoc.def}|\\
|\childdocforwardprefix{final}{child}|
\end{tabular}
\end{center}
%

Note that when several versions of a main file and/or of each child file
are to be generated, it may be convenient to set up a |Makefile| or
shell script to automatise the process.

%%%%%%%%%%%%%%%%%%%%%%%%%%%%%%%%%%%%%%%%%%%%%%%%%%%%%%%%%%%%%%%%%%%%%%%%%%%%%%%%
\subsection{Command Line Processing}
\label{sec:commandline}

The effect of redirection files can also be achieved by invoking
the \LaTeX{} compiler with a more elaborate command line.
Most conveniently this should be done as part
of a shell script or a |Makefile|.

When using \textsf{childdoc} in the main file, the following
command lines effectively perform a redirection
(note that depending on the shell being used,
backslashes may have to be doubled: `|\|' $\to$ `|\\|'):
%
\begin{center}
|... -jobname "|\textit{target}|" |\\|"|[\textit{flags}]%
|\input{childdoc.def}\childdocforward[|\textit{main}|]{|\textit{dest}|}"|
\end{center}
%
Here \textit{target} is the name of the output file,
\textit{main} is the name of the main file
and \textit{dest} is the name of the main or child file to be processed
(all filenames without extensions).
The optional argument \textit{main} can be omitted
if \textit{main} matches \textit{dest}.
Optionally, compilation \textit{flags} can be defined via |\def| commands.
This command line makes the \TeX{} engine believe
it is compiling the file \textit{target}
whose content is specified as the latter parameter.
The provided code then forwards the processing to
\textit{main} or \textit{dest} as described in \secref{sec:forward}.

%%%%%%%%%%%%%%%%%%%%%%%%%%%%%%%%%%%%%%%%%%%%%%%%%%%%%%%%%%%%%%%%%%%%%%%%%%%%%%%%
\subsection{Include by Input}
\label{sec:input}

Including child documents by |\include| has some restrictions by design.
Most notably, the content of a child document always occupies
its own set of pages; pages cannot be shared between child documents.
Usually, this behaviour makes perfect sense
because each child document contain an essential part of the document.
However, in some situations it may be desirable to compose
a document from a collection of parts
without having mandatory page breaks between then.
For this case, the package
provides a mechanism to include parts
by |\input| which can also be processed individually.
However, by construction this mechanism
requires manual handling of the content to be output.

%%%%%%%%%%%%%%%%%%%%%%%%%%%%%%%%%%%%%%%%
\DescribeMacro{\ifchilddocmanual}
The main file should be prepared as usual, see \secref{sec:include}.
However, the document body must make a distinction
between processing of an individual part and of the main document, e.g.:
%
\begin{center}
\begin{tabular}{l}
|\ifchilddocmanual|\\
|\input{\childdocname}|\\
|\||else|\\
\textit{document body with }|\input{|\textit{part}|}|\\
|\||fi|
\end{tabular}
\end{center}
%
The conditional |\ifchilddocmanual| is true whenever
a part to be included by |\input| is being compiled,
and the name of the part is stored in |\childdocname|.

%%%%%%%%%%%%%%%%%%%%%%%%%%%%%%%%%%%%%%%%
\DescribeMacro{\childdocby}
Each part to be included by |\input| should start with:
%
\begin{center}
\begin{tabular}{l}
|\input{childdoc.def}|\\
|\childdocby{|\textit{main}|}|\\
\end{tabular}
\end{center}
%
The directive |\childdocby| is similar to |\childdocof|
described in \secref{sec:include},
but the subsequent selection of content must be done manually.
To that end, both |\ifchilddoc| and |\ifchilddocmanual|
will be true upon processing of a part,
and the name of the part is stored in |\childdocname|.
Note that |\jobname| will be set to the filename of the current part
so that each part receives an individual |.aux| file
that does not interfere with the |.aux| file(s) of the main document.
This behaviour can be altered by the alternative form
|\childdocby[*]{|\textit{main}|}| (with a non-empty optional argument)
which uses the |.aux| file of the main document
by setting |\jobname| to \textit{main}.

%%%%%%%%%%%%%%%%%%%%%%%%%%%%%%%%%%%%%%%%%%%%%%%%%%%%%%%%%%%%%%%%%%%%%%%%%%%%%%%%
\subsection{Driver Development}
\label{sec:driver}

The \textsf{childdoc} mechanism can also be use for the development
of definition files such as \LaTeX{} styles or classes.
This case differs from the above setup with multiple parts
included by |\include| in that no |\includeonly| should be invoked.
This can be achieved by starting the include file
(before |\ProvidesPackage|) with:
%
\begin{center}
\begin{tabular}{l}
|\input{childdoc.def}|\\
|\childdocforward{|\textit{main}|}|\\
\end{tabular}
\end{center}
%
or alternatively with:
%
\begin{center}
\begin{tabular}{l}
|\input{childdoc.def}|\\
|\childdocby{|\textit{main}|}|\\
\end{tabular}
\end{center}
%
Both forms have slightly different effects as described above.
The main file is prepared as usual, see \secref{sec:include}.

%%%%%%%%%%%%%%%%%%%%%%%%%%%%%%%%%%%%%%%%%%%%%%%%%%%%%%%%%%%%%%%%%%%%%%%%%%%%%%%%
\subsection{Legacy Detection}
\label{sec:detection}

The directive |\childdocmain| in the main file can detect
whether the complete document or merely a child is to be compiled
even without using the directive |\childdocof|.
This method is deprecated because it is less robust
and there is no compelling reason to use it;
it is merely provided for backward compatibility
and it may be removed in future versions.

If the detection mechanism is to be used,
it is mandatory to correctly specify
the filename of the main file as the argument of |\childdocmain|:
%
\begin{center}
\begin{tabular}{l}
|\input{childdoc.def}|\\
|\childdocmain{|\textit{main}|}|\\
\end{tabular}
\end{center}
%
If |\jobname| does not match the argument \textit{main} of |\childdocmain|,
it is assumed that |\jobname| points to the child file to be compiled.
When using |\childdocmain| with the main file specified as argument,
it suffices to start a child file
with just |\input{|\textit{main}|}|
without loading of the package and using |\childdocof|.
If instead all processing is done
with the appropriate \textsf{childdoc} directives,
the argument of \textit{main} of |\childdocmain| can be empty.

An alternative version of the command line processing described
in \secref{sec:commandline} using the detection mechanism reads:
%
\begin{center}
|... -jobname "|\textit{target}|" "|[\textit{flags}]%
[|\def\jobname{|\textit{dest}|}|]|\input{|\textit{main}|}"|
\end{center}

%%%%%%%%%%%%%%%%%%%%%%%%%%%%%%%%%%%%%%%%%%%%%%%%%%%%%%%%%%%%%%%%%%%%%%%%%%%%%%%%
\subsection{Manual Code}
\label{sec:manual}

In case one cannot be certain whether the definitions file |childdoc.def|
is installed on the target \TeX{} distribution
and one prefers not to ship it,
it is conceivable to paste a few relevant commands into the sources.

To that end, drop all statements |\input{childdoc.def}|
and perform the replacements as outlined below.
Instead of |\childdocmain{|\textit{main}|}| add the following code
to the top of the main file:
%
\begin{center}
\begin{tabular}{l}
|\||ifdefined\childdocname\endinput\||fi\newif\ifchilddoc|\\
|\edef\childdocname{\scantokens\expandafter{\jobname\noexpand}}|\\
|\def\childdocmain{|\textit{main}|}\||ifx\childdocmain\childdocname\||else|\\
|\childdoctrue\includeonly{\childdocname}\let\jobname\childdocmain\||fi|\\
\end{tabular}
\end{center}
%
Instead of |\childdocof{|\textit{main}|}| just include the main file
at the top of each child file:
%
\begin{center}
|\input{|\textit{main}|}|
\end{center}
%
A simple redirection |\childdocforward{|\textit{dest}|}| is achieved by:
%
\begin{center}
|\def\jobname{|\textit{dest}|}\input{\jobname}|
\end{center}
%
The redirection with prefix
|\childdocforwardprefix[|\textit{prefix}|]{|\textit{dest}|}|
is accomplished by:
%
\begin{center}
\begin{tabular}{l}
|{\edef\jobname{\scantokens\expandafter{\jobname\noexpand}}|\\
|\def\redirectjob |\textit{prefix}|#1~~~{\gdef\jobname{|\textit{dest}|#1}}|\\
|\expandafter\redirectjob\jobname~~~}\input{\jobname}|
\end{tabular}
\end{center}

In an alternative approach,
child documents can be compiled by a specific command line
without additional code or specific definitions:
%
\begin{center}
|... -jobname "|\textit{target}|" "|[\textit{flags}]%
|\includeonly{|\textit{dest}|}\input{|\textit{main}|}"|
\end{center}
%

%%%%%%%%%%%%%%%%%%%%%%%%%%%%%%%%%%%%%%%%%%%%%%%%%%%%%%%%%%%%%%%%%%%%%%%%%%%%%%%%
%%%%%%%%%%%%%%%%%%%%%%%%%%%%%%%%%%%%%%%%%%%%%%%%%%%%%%%%%%%%%%%%%%%%%%%%%%%%%%%%
\section{Information}

%%%%%%%%%%%%%%%%%%%%%%%%%%%%%%%%%%%%%%%%%%%%%%%%%%%%%%%%%%%%%%%%%%%%%%%%%%%%%%%%
\subsection{Copyright}

Copyright \copyright{} 2017--2018 Niklas Beisert

This work may be distributed and/or modified under the
conditions of the \LaTeX{} Project Public License, either version 1.3
of this license or (at your option) any later version.
The latest version of this license is in
  \url{http://www.latex-project.org/lppl.txt}
and version 1.3 or later is part of all distributions of \LaTeX{}
version 2005/12/01 or later.

This work has the LPPL maintenance status `maintained'.

The Current Maintainer of this work is Niklas Beisert.

This work consists of the files |README.txt|, |childdoc.ins| and |childdoc.dtx|
as well as the derived files |childdoc.def|, |cdocsamp.tex|
with |cdocsch1.tex|, |cdocsch2.tex|, |cdocspt3.tex|, |cdocspt4.tex|,
|cdocsdrf.tex|, |cdocsfn1.tex|, |cdocsfn2.tex|
as well as |childdoc.pdf|.

%%%%%%%%%%%%%%%%%%%%%%%%%%%%%%%%%%%%%%%%%%%%%%%%%%%%%%%%%%%%%%%%%%%%%%%%%%%%%%%%
\subsection{Files and Installation}

The package consists of the files:
%
\begin{center}
\begin{tabular}{ll}
    |README.txt|   & readme file \\
    |childdoc.ins| & installation file \\
    |childdoc.dtx| & source file \\
    |childdoc.def| & definition file \\
    |cdocsamp.tex| & sample main file \\
    |cdocsch1.tex| & sample include file \\
    |cdocsch2.tex| & sample include file \\
    |cdocspt3.tex| & sample part file \\
    |cdocspt4.tex| & sample part file \\
    |cdocsdrf.tex| & sample redirection file \\
    |cdocsfn1.tex| & sample redirection file \\
    |cdocsfn2.tex| & sample redirection file \\
    |childdoc.pdf| & manual
\end{tabular}
\end{center}
%
The distribution consists of the files
|README.txt|, |childdoc.ins| and |childdoc.dtx|.
%
\begin{itemize}
\item
Run (pdf)\LaTeX{} on |childdoc.dtx|
to compile the manual |childdoc.pdf| (this file).
\item
Run \LaTeX{} on |childdoc.ins| to create the definitions file |childdoc.def|
and the sample |cdocsamp.tex| with include files
|cdocsch1.tex|, |cdocsch2.tex|, |cdocspt3.tex|, |cdocspt4.tex|,
|cdocsdrf.tex|, |cdocsfn1.tex|, |cdocsfn2.tex|.
Then copy the file |childdoc.def| to an appropriate directory of your \LaTeX{}
distribution, e.g.\ \textit{texmf-root}|/tex/latex/childdoc|.
\end{itemize}

%%%%%%%%%%%%%%%%%%%%%%%%%%%%%%%%%%%%%%%%%%%%%%%%%%%%%%%%%%%%%%%%%%%%%%%%%%%%%%%%
\subsection{Related CTAN Packages}

There are several other packages which offer a similar functionality:
%
\begin{itemize}
\item
The packages
\href{http://ctan.org/pkg/docmute}{\textsf{docmute}},
\href{http://ctan.org/pkg/includex}{\textsf{includex}} and
\href{http://ctan.org/pkg/standalone}{\textsf{standalone}}
provide commands to include only the document body of
a child file thus allowing both files to be compiled individually.
\item
The packages \href{http://ctan.org/pkg/subdocs}{\textsf{subdocs}}
and \href{http://ctan.org/pkg/subfiles}{\textsf{subfiles}}
provide structures in which the main and child documents can be
encapsulated and allowing them to be compiled individually.
The inclusion mechanism is different from the conventional |\include|.
\item
The package \href{http://ctan.org/pkg/combine}{\textsf{combine}}
is an elaborate solution to combine several documents into one.
\end{itemize}
%
See also the CTAN topic \href{http://ctan.org/topic/subdocs}{\textsf{subdocs}}
for further related packages.
The present package differs from the above solutions in that
a document structure constructed with the conventional |\include| mechanism
just needs two extra commands at the top of every file
such that all constituent files can be compiled individually.

%%%%%%%%%%%%%%%%%%%%%%%%%%%%%%%%%%%%%%%%%%%%%%%%%%%%%%%%%%%%%%%%%%%%%%%%%%%%%%%%
%\subsection{Feature Suggestions}
%
%The following is a list of features which may be useful for future
%versions of this package:
%%
%\begin{itemize}
%\item
%\ldots
%\end{itemize}

%%%%%%%%%%%%%%%%%%%%%%%%%%%%%%%%%%%%%%%%%%%%%%%%%%%%%%%%%%%%%%%%%%%%%%%%%%%%%%%%
\subsection{Revision History}

%%%%%%%%%%%%%%%%%%%%%%%%%%%%%%%%%%%%%%%%
\paragraph{v2.0:} 2018/12/30

\begin{itemize}
\item
immediate forward processing
\item
added |\childdocby| mechanism
\item
manual restructured
\end{itemize}

%%%%%%%%%%%%%%%%%%%%%%%%%%%%%%%%%%%%%%%%
\paragraph{v1.6:} 2018/01/17

\begin{itemize}
\item
application for development of include files
\item
corrections to manual
\end{itemize}

%%%%%%%%%%%%%%%%%%%%%%%%%%%%%%%%%%%%%%%%
\paragraph{v1.5:} 2017/05/21

\begin{itemize}
\item
more complete structuring introduced
\item
|\childdocof| introduced
\item
|\childdoc| renamed to |\childdocmain|
\item
|\childredirect| renamed to |\childdocforward| and |\childdocforwardprefix|
and functionality expanded
\end{itemize}

%%%%%%%%%%%%%%%%%%%%%%%%%%%%%%%%%%%%%%%%
\paragraph{v1.0:} 2017/04/27

\begin{itemize}
\item
manual and install package
\item
first version published on CTAN
\end{itemize}

%%%%%%%%%%%%%%%%%%%%%%%%%%%%%%%%%%%%%%%%
\paragraph{v0.6:} 2017/04/26

\begin{itemize}
\item
redirection mechanism added
\end{itemize}

%%%%%%%%%%%%%%%%%%%%%%%%%%%%%%%%%%%%%%%%
\paragraph{v0.5:} 2017/04/26

\begin{itemize}
\item
functionality in definition file
\end{itemize}


%%%%%%%%%%%%%%%%%%%%%%%%%%%%%%%%%%%%%%%%%%%%%%%%%%%%%%%%%%%%%%%%%%%%%%%%%%%%%%%%
%%%%%%%%%%%%%%%%%%%%%%%%%%%%%%%%%%%%%%%%%%%%%%%%%%%%%%%%%%%%%%%%%%%%%%%%%%%%%%%%
%%%%%%%%%%%%%%%%%%%%%%%%%%%%%%%%%%%%%%%%%%%%%%%%%%%%%%%%%%%%%%%%%%%%%%%%%%%%%%%%
\appendix

\settowidth\MacroIndent{\rmfamily\scriptsize 000\ }

 \DocInput{childdoc.dtx}

\end{document}
%</driver>
% \fi
%
% %%%%%%%%%%%%%%%%%%%%%%%%%%%%%%%%%%%%%%%%%%%%%%%%%%%%%%%%%%%%%%%%%%%%%%%%%%%%%%
% %%%%%%%%%%%%%%%%%%%%%%%%%%%%%%%%%%%%%%%%%%%%%%%%%%%%%%%%%%%%%%%%%%%%%%%%%%%%%%
% \section{Sample}
%\iffalse
%<*samplemain>
%\fi
%
% The following presents a sample document
% with two chapters, two parts, a title page,
% a compile flag as well as three forwarding files to set the flag.
% It consists of eight |.tex| files:
% \begin{center}
% \begin{tabular}{ll}
% |cdocsamp.tex|&main file\\
% |cdocsch1.tex|&include file for chapter 1\\
% |cdocsch2.tex|&include file for chapter 2\\
% |cdocspt3.tex|&include file for part 3\\
% |cdocspt4.tex|&include file for part 4\\
% |cdocsdrf.tex|&forwarding file for main file in draft mode\\
% |cdocsfi1.tex|&forwarding file for final version of chapter 1\\
% |cdocsfi2.tex|&forwarding file for final version of chapter 2\\
% \end{tabular}
% \end{center}
% Each of the eight files can be compiled directly by the \LaTeX{} compiler.
%
% %%%%%%%%%%%%%%%%%%%%%%%%%%%%%%%%%%%%%%
% \paragraph{Main File.}
%
% The main file is called |cdocsamp.tex|.
%
% Load the \textsf{childdoc} definitions and
% declare the filename for the main document:
%    \begin{macrocode}
\input{childdoc.def}
\childdocmain{}
%    \end{macrocode}

% Optional override for |\version| flag:
%    \begin{macrocode}
%%\ifchilddoc\else\providecommand{\version}{draft}\fi
%    \end{macrocode}

% Define the default values for the |\version| flag
% (|final| for the main file and |draft| for childs):
%    \begin{macrocode}
\ifchilddoc
\providecommand{\version}{draft}
\else
\providecommand{\version}{final}
\fi
%    \end{macrocode}

% Load the standard document class:
%    \begin{macrocode}
\documentclass[12pt]{article}
%    \end{macrocode}

% Start the document body:
%    \begin{macrocode}
\begin{document}
%    \end{macrocode}

% Declare a title page.
% Print title, part of document being processed and version flag:
%    \begin{macrocode}
\addtocounter{page}{-1}
\begin{center}
{\LARGE\bfseries{}childdoc example\par}
\vspace{1cm}
\ifchilddoc
\ifchilddocmanual part\else chapter\fi:
`\childdocname' of `\childdocjob'\par
\else
main document: `\childdocjob'\par
\fi
version: \version\par
\end{center}
\newpage
%    \end{macrocode}

% Manually include selected file,
% otherwise process as usual:
%    \begin{macrocode}
\ifchilddocmanual
\section*{part `\childdocname'}
\input{\childdocname}
\else
%    \end{macrocode}

% Include the two chapters:
%    \begin{macrocode}
\include{cdocsch1}
\include{cdocsch2}
%    \end{macrocode}

% Include the two parts unless only chapters should be displayed:
%    \begin{macrocode}
\ifchilddoc\else
\section{part three}
\input{cdocspt3}
\section{part four}
\input{cdocspt4}
\fi
%    \end{macrocode}

% Process as usual until here:
%    \begin{macrocode}
\fi
%    \end{macrocode}

% End of document body:
%    \begin{macrocode}
\end{document}
%    \end{macrocode}
%\iffalse
%</samplemain>
%\fi
%
% %%%%%%%%%%%%%%%%%%%%%%%%%%%%%%%%%%%%%%
% \paragraph{Chapter Include Files.}
%
% The include files are called |cdocsch1.tex| and |cdocsch2.tex|.
%
%\iffalse
%<*samplechap1|samplechap2>
%\fi

% Optional override for |\version| flag:
%    \begin{macrocode}
%%\providecommand{\version}{final}
%    \end{macrocode}

% Include the main document:
%    \begin{macrocode}
\input{childdoc.def}
\childdocof{cdocsamp}
%    \end{macrocode}

%\iffalse
%</samplechap1|samplechap2>
%\fi
%
%\iffalse
%<*samplechap1>
%\fi
% Some text for chapter 1:
%    \begin{macrocode}
\section{one}
some text in chapter one
%    \end{macrocode}

%\iffalse
%</samplechap1>
%\fi
% Some text for chapter 2:
%\iffalse
%<*samplechap2>
%\fi
%    \begin{macrocode}
\section{two}
more text in chapter two
%    \end{macrocode}

%\iffalse
%</samplechap2>
%\fi
%
% %%%%%%%%%%%%%%%%%%%%%%%%%%%%%%%%%%%%%%
% \paragraph{Part Include Files.}
%
% The include files are called |cdocspt3.tex| and |cdocspt4.tex|.
%
%\iffalse
%<*samplepart3|samplepart4>
%\fi

% Optional override for |\version| flag:
%    \begin{macrocode}
%%\providecommand{\version}{final}
%    \end{macrocode}

% Include the main document:
%    \begin{macrocode}
\input{childdoc.def}
\childdocby{cdocsamp}
%    \end{macrocode}

%\iffalse
%</samplepart3|samplepart4>
%\fi
%
%\iffalse
%<*samplepart3>
%\fi
% Some text for part 3:
%    \begin{macrocode}
some text in part three
%    \end{macrocode}

%\iffalse
%</samplepart3>
%\fi
% Some text for part 4:
%\iffalse
%<*samplepart4>
%\fi
%    \begin{macrocode}
more text in part four
%    \end{macrocode}

%\iffalse
%</samplepart4>
%\fi
%
% %%%%%%%%%%%%%%%%%%%%%%%%%%%%%%%%%%%%%%
% \paragraph{Forwarding for a Complete Draft.}
%
% The following forwarding file |cdocsdrf.tex|
% compiles the main document in draft mode:
%\iffalse
%<*sampledraft>
%\fi
%    \begin{macrocode}
\def\version{draft}
\input{childdoc.def}
\childdocforward{cdocsamp}
%    \end{macrocode}

%\iffalse
%</sampledraft>
%\fi
%
% %%%%%%%%%%%%%%%%%%%%%%%%%%%%%%%%%%%%%%
% \paragraph{Forwarding for Final Version of the Chapters.}
%
% The following forwarding files |cdocsfn1.tex| and |cdocsfn2.tex|
% (with identical content)
% compile the final versions of the child documents
% |cdocsch1.tex| and |cdocsch2.tex|, respectively:
%\iffalse
%<*samplefinal>
%\fi
%    \begin{macrocode}
\def\version{final}
\input{childdoc.def}
\childdocforwardprefix[cdocsamp]{cdocsfn}{cdocsch}
%    \end{macrocode}

%\iffalse
%</samplefinal>
%\fi
%
% %%%%%%%%%%%%%%%%%%%%%%%%%%%%%%%%%%%%%%
% \paragraph{Command Line Processing.}
%
% The following three command lines generate the output files
% |cdocscld|, |cdocscl1| and |cdocscl2|
% which should be identical to
% |cdocsdrf|, |cdocsch1| and |cdocsfn2|, respectively:
% \begin{center}
% \begin{tabular}{l}
% |latex -jobname cdocscld \|\\
% |  "\def\version{draft}\input{childdoc.def}\childdocforward{cdocsamp}"|\\
% |latex -jobname cdocscl1 \|\\
% |  "\input{childdoc.def}\childdocforward[cdocsamp]{cdocsch1}"|\\
% |latex -jobname cdocscl2 \|\\
% |  "\def\version{final}\input{childdoc.def}\childdocforward{cdocsch2}"|
% \end{tabular}
% \end{center}
% Note that the trailing backslash on each first line
% merely continues the input to the second line
% (for convenient cut ant paste).
% Furthermore, the command |latex| can be replaced by any
% of its alternative versions such as |pdflatex|.
%
% %%%%%%%%%%%%%%%%%%%%%%%%%%%%%%%%%%%%%%%%%%%%%%%%%%%%%%%%%%%%%%%%%%%%%%%%%%%%%%
% %%%%%%%%%%%%%%%%%%%%%%%%%%%%%%%%%%%%%%%%%%%%%%%%%%%%%%%%%%%%%%%%%%%%%%%%%%%%%%
% \section{Implementation}
%\iffalse
%<*package>
%\fi
%
% This section describes the definitions file |childdoc.def|.

% The definitions cannot be loaded using |\usepackage| or |\RequirePackage|
% which has a mechanism to prevent loading a style file more than once.
% When loading the definitions by means of |\input|
% multiple instances have to be prevented manually:
%\iffalse
%This code needs to be before the `\ProvidesFile' directive
%which is defined at the beginning of this file.
%Therefore it is also placed there and commented out here.
%</package>
%<*discard>
%\fi
%    \begin{macrocode}
\ifdefined\childdocmain\endinput\fi
%    \end{macrocode}
%\iffalse
%</discard>
%<*package>
%\fi
%
% \macro{\ifchilddoc}
% \macro{\ifchilddocmanual}
% The conditional |\ifchilddoc| tells whether a
% child (true) or main (false) document is being compiled.
% The conditional |\ifchilddocmanual| tells whether
% the |\includeonly| mechanism is used (false) or
% the selection of child files must be performed manually (true).
% The definitions initialise to false:
%    \begin{macrocode}
\newif\ifchilddoc
\newif\ifchilddocmanual
%    \end{macrocode}

% \macro{\childdocname}
% \macro{\childdocjob}
% The macro |\childdocname| stores the name of the main document
% to be compiled. The macro |\childdocjob| stores the name of
% the document on which the \LaTeX{} compiler was originally invoked.
% The content of |\jobname| cannot be compared
% to filenames specified in the source due to different catcodes.
% The following code rescans |\jobname|, stores the result
% in |\childdocname| and saves a copy in |\childdocjob|:
%    \begin{macrocode}
\edef\childdocname{\scantokens\expandafter{\jobname\noexpand}}
\let\childdocjob\childdocname
%    \end{macrocode}

% \macro{\childdocdisable}
% The macro |\childdocdisable| prevents the main file
% from being processed more than once.
% At this stage, the main document command |\childdocmain|
% is assumed to be called once again where it should do nothing.
% Any subsequent call to it should prevent
% a secondary processing of the main document
% It overwrites the forwarding commands
% |\childdocof| and |\childdocforward|
% with empty macros to prevent further inclusions of the main document:
%    \begin{macrocode}
\newcommand{\childdocdisable}
{
  \renewcommand{\childdocmain}[1]{\renewcommand{\childdocmain}[1]{\endinput}}
  \renewcommand{\childdocof}[1]{}
  \renewcommand{\childdocby}[2][]{}
  \renewcommand{\childdocforward}[2][]{}
  \renewcommand{\childdocdisable}{}
}
%    \end{macrocode}

% \macro{\childdocmain}
% The macro |\childdocmain| is to be called at the top of the main file
% with nothing or the main filename (without extension) as argument.
% First, it breaks loops.
% If the argument is not empty and does not match |\childdocname|
% (which is set by the first inclusion of |childdoc.def|),
% |\ifchilddoc| is set to true, |\includeonly| is applied to the child file
% and |\jobname| is set to the main file
% (for proper handling of |.aux| files):
%    \begin{macrocode}
\newcommand{\childdocmain}[1]
{
  \childdocdisable\childdocmain{}
  \if?#1?\else
    \begingroup
      \def\childdoctmp{#1}
      \ifx\childdoctmp\childdocname
        \def\childdoctmp{}
      \else
        \def\childdoctmp
        {
          \childdoctrue
          \includeonly{\childdocname}
          \def\childdocjob{#1}
          \def\jobname{#1}
        }
      \fi
      \expandafter
    \endgroup
    \childdoctmp
  \fi
}
%    \end{macrocode}

% \macro{\childdocof}
% The command |\childdocof| redirects
% compilation to the main file |#1|.
%    \begin{macrocode}
\newcommand{\childdocof}[1]
{
  \childdocdisable
  \childdoctrue
  \includeonly{\childdocname}
  \def\jobname{#1}
  \def\childdocjob{#1}
  \input{#1}
}
%    \end{macrocode}

% \macro{\childdocby}
% The command |\childdocby| ....
%    \begin{macrocode}
\newcommand{\childdocby}[2][]
{
  \childdocdisable
  \childdoctrue
  \childdocmanualtrue
  \if?#1?\else
    \def\jobname{#2}
  \fi
  \def\childdocjob{#2}
  \input{#2}
  \endinput
}
%    \end{macrocode}

% \macro{\childdocforward}
% The command |\childdocforward| redirects
% compilation to the main file or
% (if the optional argument is given) a child file.
% Parameters are set as if the main file
% or a child file starting with |\childdocof| was compiled.
% Then compilation is handed over to the main file:
%    \begin{macrocode}
\newcommand{\childdocforward}[2][]
{
  \begingroup
    \if?#1?
      \def\childdoctmp
      {
        \def\childdocname{#2}
        \def\childdocjob{#2}
        \def\jobname{#2}
        \input{#2}
        \endinput
      }
    \else
      \def\childdoctmp
      {
        \childdocdisable
        \def\childdocname{#2}
        \childdoctrue
        \includeonly{#2}
        \def\childdocjob{#1}
        \def\jobname{#1}
        \input{#1}
        \endinput
      }
    \fi
    \expandafter
  \endgroup
  \childdoctmp
}
%    \end{macrocode}

% \macro{\childdocforwardprefix}
% The command |\childdocforwardprefix| redirects
% compilation to the main or a child file by means of a pattern.
% The prefix |#1| in the current filename is replaced by |#2|
% and the suffix of the current filename is kept
% (it is assumed that the filename does not contain the substring `|~~~|'
% which is used as a delimiter).
% Compilation is handed over to the new file by |\childdocforward|:
%    \begin{macrocode}
\newcommand{\childdocforwardprefix}[3][]
{
  \begingroup
    \def\childdocextract #2##1~~~{\def\childdoctmp{\childdocforward[#1]{#3##1}}}
    \expandafter\childdocextract\childdocname~~~
    \expandafter
  \endgroup
  \childdoctmp
}
%    \end{macrocode}

% \macro{\childdoc}
% The deprecated macro |\childdoc| is a legacy version of |\childdocmain|:
%    \begin{macrocode}
\newcommand{\childdoc}{\childdocmain}
%    \end{macrocode}

% \macro{\childdocredirect}
% The deprecated macro |\childdocredirect| is a legacy version
% of |\childdocforward| and |\childdocforwardprefix|:
%    \begin{macrocode}
\newcommand{\childdocredirect}[2][]
{
  \begingroup
    \if?#1?
      \def\childdoctmp{\childdocforward{#2}}
    \else
      \def\childdoctmp{\childdocforwardprefix{#1}{#2}}
    \fi
    \expandafter
  \endgroup
  \childdoctmp
}
%    \end{macrocode}

%\iffalse
%</package>
%\fi
%
\endinput
\childdocforward[cdocsamp]{cdocsch1}"|\\
% |latex -jobname cdocscl2 \|\\
% |  "\def\version{final}% \iffalse
%
% childdoc.dtx Copyright (C) 2017-2018 Niklas Beisert
%
% This work may be distributed and/or modified under the
% conditions of the LaTeX Project Public License, either version 1.3
% of this license or (at your option) any later version.
% The latest version of this license is in
%   http://www.latex-project.org/lppl.txt
% and version 1.3 or later is part of all distributions of LaTeX
% version 2005/12/01 or later.
%
% This work has the LPPL maintenance status `maintained'.
%
% The Current Maintainer of this work is Niklas Beisert.
%
% This work consists of the files childdoc.dtx and childdoc.ins
% and the derived files childdoc.def and cdocsamp.tex with
% cdocsch1.tex, cdocsch2.tex, cdocsdrf.tex, cdocsfn1.tex, cdocsfn2.tex.
%
%<package>\ifdefined\childdocmain\endinput\fi
%<package>\ProvidesFile{childdoc.def}[2018/12/30 v2.0 child document driver]
%<samplemain>\ProvidesFile{cdocsamp.tex}[2018/12/30 v2.0 sample for childdoc]
%<*driver>
%\ProvidesFile{childdoc.drv}[2018/12/30 v2.0 childdoc reference manual file]
\PassOptionsToClass{10pt,a4paper}{article}
\documentclass{ltxdoc}

\usepackage[margin=35mm]{geometry}
\usepackage{hyperref}
\usepackage{hyperxmp}
\usepackage[usenames]{color}

\hypersetup{colorlinks=true}
\hypersetup{pdfstartview=FitH}
\hypersetup{pdfpagemode=UseNone}
\hypersetup{pdfsource={}}
\hypersetup{pdflang={en-UK}}
\hypersetup{pdfcopyright={Copyright 2017-2018 Niklas Beisert.
  This work may be distributed and/or modified under the
  conditions of the LaTeX Project Public License, either version 1.3
  of this license or (at your option) any later version.}}
\hypersetup{pdflicenseurl={http://www.latex-project.org/lppl.txt}}
\hypersetup{pdfcontactaddress={ETH Zurich, ITP, HIT K,
  Wolfgang-Pauli-Strasse 27}}
\hypersetup{pdfcontactpostcode={8093}}
\hypersetup{pdfcontactcity={Zurich}}
\hypersetup{pdfcontactcountry={Switzerland}}
\hypersetup{pdfcontactemail={nbeisert@itp.phys.ethz.ch}}
\hypersetup{pdfcontacturl={http://people.phys.ethz.ch/\xmptilde nbeisert/}}

\newcommand{\secref}[1]{\hyperref[#1]{section \ref*{#1}}}

\parskip1ex
\parindent0pt
\let\olditemize\itemize
\def\itemize{\olditemize\parskip0pt}

\begin{document}

\title{The \textsf{childdoc} Package}
\hypersetup{pdftitle={The childdoc Package}}
\author{Niklas Beisert\\[2ex]
  Institut f\"ur Theoretische Physik\\
  Eidgen\"ossische Technische Hochschule Z\"urich\\
  Wolfgang-Pauli-Strasse 27, 8093 Z\"urich, Switzerland\\[1ex]
  \href{mailto:nbeisert@itp.phys.ethz.ch}
  {\texttt{nbeisert@itp.phys.ethz.ch}}}
\hypersetup{pdfauthor={Niklas Beisert}}
\hypersetup{pdfsubject={Manual for the LaTeX2e Package childdoc}}
\date{30 December 2018, \textsf{v2.0}}
\maketitle

\begin{abstract}\noindent
\textsf{childdoc} is a \LaTeXe{} package
that enables the direct compilation
of document sections included by |\include|
to individual files.
\end{abstract}

\begingroup
\parskip0ex
\tableofcontents
\endgroup

%%%%%%%%%%%%%%%%%%%%%%%%%%%%%%%%%%%%%%%%%%%%%%%%%%%%%%%%%%%%%%%%%%%%%%%%%%%%%%%%
%%%%%%%%%%%%%%%%%%%%%%%%%%%%%%%%%%%%%%%%%%%%%%%%%%%%%%%%%%%%%%%%%%%%%%%%%%%%%%%%
\section{Introduction}

\LaTeX{} provides a mechanism to structure a large document (such as a book)
into a main file and several child files (containing the chapters)
using the |\include| command.
This mechanism is beneficial for documents
which span hundreds of pages in order to
make the source file(s) more manageable.
Moreover, compilation can be restricted to
selected child files by means of the |\includeonly| command.
The latter feature can be used to reduce the compilation time while editing
(this was significantly more useful in the earlier days of \LaTeX{})
or to generate a smaller document which is easier to navigate.
Another application of |\includeonly| is to generate
documents consisting of selected parts of the complete document.

However, there are a few drawbacks of the plain |\include| mechanism:
\begin{itemize}
\item
The child files cannot be compiled on their own,
they can only be compiled via the main file.
A naive editing environment
(such as a text editor with an option
to have the current file processed by \LaTeX)
may require one to switch to the main file before compiling;
attempting to compile the child file produces errors.
\item
The main file must be modified (each time)
to adjust the |\includeonly| command
to the present needs. This easily leaves the main file in a messy state.
\item
The generated document will always carry the filename
of the main document. This is inconvenient if
several child files are to be compiled and
to be kept for distribution.
\end{itemize}

The present package provides a simple interface
to make child files individually compilable by \LaTeX{}.
Compiling a child file then has the same effect as compiling
the main file with an |\includeonly| command
to select the appropriate child.
Moreover the generated document will carry the name of the child
rather than the main file.
This resolves all three above issues.

This feature is meant to make the editing of books,
thesis documents and lecture notes somewhat more convenient.
However, the package can also be used efficiently for
composing a series of documents (such as exercise sheets)
which are typically distributed individually.
It then assists the author in generating the individual documents
(potentially in different versions)
as well as a document containing the collected series.
Another application is in developing style files
or other kinds of included material
where compilation of the style file could redirect
to a sample or test file.

%%%%%%%%%%%%%%%%%%%%%%%%%%%%%%%%%%%%%%%%%%%%%%%%%%%%%%%%%%%%%%%%%%%%%%%%%%%%%%%%
%%%%%%%%%%%%%%%%%%%%%%%%%%%%%%%%%%%%%%%%%%%%%%%%%%%%%%%%%%%%%%%%%%%%%%%%%%%%%%%%
\section{Usage}

First of all, the package \textsf{childdoc} is \emph{not} a standard
\LaTeXe{} |.sty| style file! Therefore it needs to be invoked in
a non-standard way.

%%%%%%%%%%%%%%%%%%%%%%%%%%%%%%%%%%%%%%%%%%%%%%%%%%%%%%%%%%%%%%%%%%%%%%%%%%%%%%%%
\subsection{Included Files}
\label{sec:include}

%%%%%%%%%%%%%%%%%%%%%%%%%%%%%%%%%%%%%%%%
\DescribeMacro{\childdocmain}
To use the package, add the commands
\begin{center}
\begin{tabular}{l}
|\input{childdoc.def}|\\
|\childdocmain{}|\\
\end{tabular}
\end{center}
at the very top of the main \LaTeX{} file,
in particular \emph{before} the |\documentclass| statement!
The argument of |\childdocmain| should be left empty
(but it must be present).

%%%%%%%%%%%%%%%%%%%%%%%%%%%%%%%%%%%%%%%%
\DescribeMacro{\childdocof}
Furthermore, add the commands
\begin{center}
\begin{tabular}{l}
|\input{childdoc.def}|\\
|\childdocof{|\textit{main}|}|\\
\end{tabular}
\end{center}
at the top of every child file \textit{child}
which is included by |\include{|\textit{child}|}|
from within the main file
(or at least for those files to be compiled individually).
The argument \textit{main} must be the filename of the main file.

There are a couple of
considerations in setting up the main and child documents:

%%%%%%%%%%%%%%%%%%%%%%%%%%%%%%%%%%%%%%%%
\paragraph{Restrictions.}

Please note the following restrictions:
\begin{itemize}
\item
|\childdocmain| must be called with one argument \textit{main}
to ensure compatibility with earlier version of the package.
It must either be empty (|\childdocmain{}|)
or precisely match the filename of the main file in which it is specified.
See \secref{sec:detection} for further information.
\item
The filename \textit{main} must be specified without the |.tex| extension.
\item
The filename \textit{main} is case sensitive
(even in case-insensitive file systems)
due to internal string comparison.
\item
The argument \textit{main} should be fully expanded, it cannot be a macro.
\item
Subdirectories and special characters should be avoided in filenames.
\item
The command |\childdocmain{|\textit{main}|}| must be followed by a whitespace.
It should not be followed immediately by another command
or by a comment mark `|%|'.
This is because the \TeX{} parser reads the token immediately following
the argument of |\childdocmain| and puts it
at the beginning of every child section;
however, a white\-space is ignored.
\end{itemize}

%%%%%%%%%%%%%%%%%%%%%%%%%%%%%%%%%%%%%%%%
\paragraph{Content of Main File.}

It is advisable to place all content in the child files included by |\include|.
Any output contained in the main file will appear in all child documents
unless suppressed manually;
it cannot be suppressed automatically by the |\includeonly| directive
and thus should normally be avoided.
A method to include some content in the main file
by means of conditional processing is described in \secref{sec:conditional}.

%%%%%%%%%%%%%%%%%%%%%%%%%%%%%%%%%%%%%%%%
\paragraph{Page Numbering.}

When only a part of the document is compiled,
the appropriate numbering of pages
(as well as other status parameters)
is determined from the |.aux| files.
The latter contain information from previous passes.
However this information needs to propagate through
all intermediate child documents.
Therefore the page numbering in child documents may well
be inconsistent until the complete document is compiled at least once.

A useful (if unconventional) way to always ensure a consistent
page numbering is to restart the numbering in each child document
and denote the pages by `\textit{child}|.|\textit{page}'
where \textit{child} represents the chapter/section number of the child file.
This can be achieved by the command
|\numberwithin{page}{|\textit{child}|}|
of the \textsf{amsmath} package
where \textit{child} can be |chapter| or |section|
depending on the chosen structuring.
Alternatively, one can modify the macro |\thepage| appropriately
and reset the counter |page| at the start of each child file.

%%%%%%%%%%%%%%%%%%%%%%%%%%%%%%%%%%%%%%%%%%%%%%%%%%%%%%%%%%%%%%%%%%%%%%%%%%%%%%%%
\subsection{Conditional Processing}
\label{sec:conditional}

The package provides a mechanism to compile different versions
of a document. To customise the versions further some conditional processing
can come in handy to distinguish which version is being compiled.
The package provides two macros to describe the compilation context:

%%%%%%%%%%%%%%%%%%%%%%%%%%%%%%%%%%%%%%%%
\DescribeMacro{\ifchilddoc}
The conditional |\ifchilddoc| distinguishes between the compilation of
child documents and the main document:
%
\begin{center}
|\ifchilddoc |\textit{child-code}| |[|\||else |\textit{main-code}]| \||fi|
\end{center}

%%%%%%%%%%%%%%%%%%%%%%%%%%%%%%%%%%%%%%%%
\DescribeMacro{\childdocname}
\DescribeMacro{\childdocjob}
The macro |\childdocname| contains the filename (without extension)
of the main or child file being processed.
Note that |\childdocjob| will always contain the name of the main file.

%%%%%%%%%%%%%%%%%%%%%%%%%%%%%%%%%%%%%%%%
\paragraph{Title Page.}

Conditional processing can be used to include a title or banner page
in the main document when proper precautions are taken.
Importantly, the code in the main file should ensure that the page counter
(as well as other status parameters which are stored in the |.aux| files)
takes the same value after the conditional processing.
Otherwise the page numbers may take divergent values
depending on which part is compiled.

For example, a title page could be declared by:
%
\begin{center}
\begin{tabular}{l}
|\ifchilddoc\||else|\\
|\addtocounter{page}{-1}|\\
\textit{code for title page}\\
|\newpage|\\
|\||fi|
\end{tabular}
\end{center}
%
A banner page for the child documents can be generated by:
%
\begin{center}
\begin{tabular}{l}
|\ifchilddoc|\\
|\addtocounter{page}{-1}|\\
\textit{code for banner page}\\
|\newpage|\\
|\||fi|
\end{tabular}
\end{center}
%
Here one could write a message such as:
\begin{center}
|This is the part \childdocname{} of \childdocjob{}.|
\end{center}

%%%%%%%%%%%%%%%%%%%%%%%%%%%%%%%%%%%%%%%%%%%%%%%%%%%%%%%%%%%%%%%%%%%%%%%%%%%%%%%%
\subsection{Flags}
\label{sec:flags}

The package makes it easy to generate different versions
of the main or child documents.
To this end compilation flags can be defined
and assigned different default values.
They will be particularly useful in conjunction
with the forwarding mechanism described in \secref{sec:forward}.

For example, it may be useful to have a flag |\version|
which can be set to |draft| or |final|.
The document source will contain some conditional code
depending on the value of |\version|.
Suppose further, the flag should default to |final| for the main file
and to |draft| for child files
which is a natural assignment for editing the document.
This is achieved by placing the following code
in the preamble of the main document
(below the |\childdocmain| directive):
%
\begin{center}
\begin{tabular}{l}
|\ifchilddoc|\\
|\providecommand{\version}{draft}|\\
|\||else|\\
|\providecommand{\version}{final}|\\
|\||fi|
\end{tabular}
\end{center}
%
The definition by |\providecommand| makes sure
that previous definitions are not overwritten.
Further statements |\providecommand{\version}{...}|
can thus be added before the above code to override it.

For the main file, one might add a line
(between |\childdocmain| and the above block)
%
\begin{center}
|%\ifchilddoc\||else\providecommand{\version}{draft}\||fi|
\end{center}
%
which can be uncommented to produce a draft version.
Likewise one can add a line to the very top of a child file
(above the |\childdocof{|\textit{main}|}| directive)
%
\begin{center}
|%\providecommand{\version}{final}|
\end{center}
%
which can be uncommented to produce the final version of this child document.

%%%%%%%%%%%%%%%%%%%%%%%%%%%%%%%%%%%%%%%%%%%%%%%%%%%%%%%%%%%%%%%%%%%%%%%%%%%%%%%%
\subsection{Forwarding}
\label{sec:forward}

Different versions of the main or child documents
using compilation flags as described in \secref{sec:flags}
can be (permanently) stored in different files
for convenient compilation, viewing and distribution.
To this end, the package defines a command
to pass on compilation to a different file:

%%%%%%%%%%%%%%%%%%%%%%%%%%%%%%%%%%%%%%%%
\DescribeMacro{\childdocforward}
The command |\childdocforward| redirects processing to
another source file:
%
\begin{center}
\begin{tabular}{l}
|\input{childdoc.def}|\\
|\childdocforward[|\textit{main}|]{|\textit{dest}|}|\\
\end{tabular}
\end{center}
%
The argument \textit{dest} is the destination file
(without extension).
It should be the main file or one of the child files.
Note that further \textsf{childdoc} directives
such as |\childdocof| and |\childdocforward|
in the indicated file will be processed in this form.
The optional argument \textit{main}
passes on directly to the main file \textit{main}
while pretending to compile the child \textit{dest}.
This form behaves as if \textit{dest}
issues |\childdocof{|\textit{main}|}| right away,
and no further \textsf{childdoc} directives will be processed.

%%%%%%%%%%%%%%%%%%%%%%%%%%%%%%%%%%%%%%%%
\DescribeMacro{\...prefix}
In the alternative form |\childdocforwardprefix|,
%
\begin{center}
\begin{tabular}{l}
|\input{childdoc.def}|\\
|\childdocforwardprefix[|\textit{main}|]{|\textit{prefix}|}{|\textit{dest}|}|
\end{tabular}
\end{center}
%
the destination file is determined by a pattern
depending on the current file:
To make this work, the current file must be called
`{\textit{prefix}\hspace{0.2em}\textit{suffix}}'
with \textit{prefix} matching precisely the argument.
Processing is then passed on to the file
`{\textit{dest}\hspace{0.2em}\textit{suffix}}'.
Surely, the same effect is achieved by
directly specifying the
argument `{\textit{dest}\hspace{0.2em}\textit{suffix}}'
in the first form.
However, that requires to set up a different file
for each child. With the alternative form of the command
all these files can have exactly the same content
which simplifies setting them up and maintaining them.

For example, the following file |draft.tex|
with a compilation flag |\version| as described in \secref{sec:flags}
compiles the main document as a draft:
%
\begin{center}
\begin{tabular}{l}
|\def\version{draft}|\\
|\input{childdoc.def}|\\
|\childdocforward{|\textit{main}|}|
\end{tabular}
\end{center}
%
Likewise, the following files |final|\textit{nn}|.tex|
compile the final version of the child document
|child|\textit{nn}|.tex|:
%
\begin{center}
\begin{tabular}{l}
|\def\version{final}|\\
|\input{childdoc.def}|\\
|\childdocforwardprefix{final}{child}|
\end{tabular}
\end{center}
%

Note that when several versions of a main file and/or of each child file
are to be generated, it may be convenient to set up a |Makefile| or
shell script to automatise the process.

%%%%%%%%%%%%%%%%%%%%%%%%%%%%%%%%%%%%%%%%%%%%%%%%%%%%%%%%%%%%%%%%%%%%%%%%%%%%%%%%
\subsection{Command Line Processing}
\label{sec:commandline}

The effect of redirection files can also be achieved by invoking
the \LaTeX{} compiler with a more elaborate command line.
Most conveniently this should be done as part
of a shell script or a |Makefile|.

When using \textsf{childdoc} in the main file, the following
command lines effectively perform a redirection
(note that depending on the shell being used,
backslashes may have to be doubled: `|\|' $\to$ `|\\|'):
%
\begin{center}
|... -jobname "|\textit{target}|" |\\|"|[\textit{flags}]%
|\input{childdoc.def}\childdocforward[|\textit{main}|]{|\textit{dest}|}"|
\end{center}
%
Here \textit{target} is the name of the output file,
\textit{main} is the name of the main file
and \textit{dest} is the name of the main or child file to be processed
(all filenames without extensions).
The optional argument \textit{main} can be omitted
if \textit{main} matches \textit{dest}.
Optionally, compilation \textit{flags} can be defined via |\def| commands.
This command line makes the \TeX{} engine believe
it is compiling the file \textit{target}
whose content is specified as the latter parameter.
The provided code then forwards the processing to
\textit{main} or \textit{dest} as described in \secref{sec:forward}.

%%%%%%%%%%%%%%%%%%%%%%%%%%%%%%%%%%%%%%%%%%%%%%%%%%%%%%%%%%%%%%%%%%%%%%%%%%%%%%%%
\subsection{Include by Input}
\label{sec:input}

Including child documents by |\include| has some restrictions by design.
Most notably, the content of a child document always occupies
its own set of pages; pages cannot be shared between child documents.
Usually, this behaviour makes perfect sense
because each child document contain an essential part of the document.
However, in some situations it may be desirable to compose
a document from a collection of parts
without having mandatory page breaks between then.
For this case, the package
provides a mechanism to include parts
by |\input| which can also be processed individually.
However, by construction this mechanism
requires manual handling of the content to be output.

%%%%%%%%%%%%%%%%%%%%%%%%%%%%%%%%%%%%%%%%
\DescribeMacro{\ifchilddocmanual}
The main file should be prepared as usual, see \secref{sec:include}.
However, the document body must make a distinction
between processing of an individual part and of the main document, e.g.:
%
\begin{center}
\begin{tabular}{l}
|\ifchilddocmanual|\\
|\input{\childdocname}|\\
|\||else|\\
\textit{document body with }|\input{|\textit{part}|}|\\
|\||fi|
\end{tabular}
\end{center}
%
The conditional |\ifchilddocmanual| is true whenever
a part to be included by |\input| is being compiled,
and the name of the part is stored in |\childdocname|.

%%%%%%%%%%%%%%%%%%%%%%%%%%%%%%%%%%%%%%%%
\DescribeMacro{\childdocby}
Each part to be included by |\input| should start with:
%
\begin{center}
\begin{tabular}{l}
|\input{childdoc.def}|\\
|\childdocby{|\textit{main}|}|\\
\end{tabular}
\end{center}
%
The directive |\childdocby| is similar to |\childdocof|
described in \secref{sec:include},
but the subsequent selection of content must be done manually.
To that end, both |\ifchilddoc| and |\ifchilddocmanual|
will be true upon processing of a part,
and the name of the part is stored in |\childdocname|.
Note that |\jobname| will be set to the filename of the current part
so that each part receives an individual |.aux| file
that does not interfere with the |.aux| file(s) of the main document.
This behaviour can be altered by the alternative form
|\childdocby[*]{|\textit{main}|}| (with a non-empty optional argument)
which uses the |.aux| file of the main document
by setting |\jobname| to \textit{main}.

%%%%%%%%%%%%%%%%%%%%%%%%%%%%%%%%%%%%%%%%%%%%%%%%%%%%%%%%%%%%%%%%%%%%%%%%%%%%%%%%
\subsection{Driver Development}
\label{sec:driver}

The \textsf{childdoc} mechanism can also be use for the development
of definition files such as \LaTeX{} styles or classes.
This case differs from the above setup with multiple parts
included by |\include| in that no |\includeonly| should be invoked.
This can be achieved by starting the include file
(before |\ProvidesPackage|) with:
%
\begin{center}
\begin{tabular}{l}
|\input{childdoc.def}|\\
|\childdocforward{|\textit{main}|}|\\
\end{tabular}
\end{center}
%
or alternatively with:
%
\begin{center}
\begin{tabular}{l}
|\input{childdoc.def}|\\
|\childdocby{|\textit{main}|}|\\
\end{tabular}
\end{center}
%
Both forms have slightly different effects as described above.
The main file is prepared as usual, see \secref{sec:include}.

%%%%%%%%%%%%%%%%%%%%%%%%%%%%%%%%%%%%%%%%%%%%%%%%%%%%%%%%%%%%%%%%%%%%%%%%%%%%%%%%
\subsection{Legacy Detection}
\label{sec:detection}

The directive |\childdocmain| in the main file can detect
whether the complete document or merely a child is to be compiled
even without using the directive |\childdocof|.
This method is deprecated because it is less robust
and there is no compelling reason to use it;
it is merely provided for backward compatibility
and it may be removed in future versions.

If the detection mechanism is to be used,
it is mandatory to correctly specify
the filename of the main file as the argument of |\childdocmain|:
%
\begin{center}
\begin{tabular}{l}
|\input{childdoc.def}|\\
|\childdocmain{|\textit{main}|}|\\
\end{tabular}
\end{center}
%
If |\jobname| does not match the argument \textit{main} of |\childdocmain|,
it is assumed that |\jobname| points to the child file to be compiled.
When using |\childdocmain| with the main file specified as argument,
it suffices to start a child file
with just |\input{|\textit{main}|}|
without loading of the package and using |\childdocof|.
If instead all processing is done
with the appropriate \textsf{childdoc} directives,
the argument of \textit{main} of |\childdocmain| can be empty.

An alternative version of the command line processing described
in \secref{sec:commandline} using the detection mechanism reads:
%
\begin{center}
|... -jobname "|\textit{target}|" "|[\textit{flags}]%
[|\def\jobname{|\textit{dest}|}|]|\input{|\textit{main}|}"|
\end{center}

%%%%%%%%%%%%%%%%%%%%%%%%%%%%%%%%%%%%%%%%%%%%%%%%%%%%%%%%%%%%%%%%%%%%%%%%%%%%%%%%
\subsection{Manual Code}
\label{sec:manual}

In case one cannot be certain whether the definitions file |childdoc.def|
is installed on the target \TeX{} distribution
and one prefers not to ship it,
it is conceivable to paste a few relevant commands into the sources.

To that end, drop all statements |\input{childdoc.def}|
and perform the replacements as outlined below.
Instead of |\childdocmain{|\textit{main}|}| add the following code
to the top of the main file:
%
\begin{center}
\begin{tabular}{l}
|\||ifdefined\childdocname\endinput\||fi\newif\ifchilddoc|\\
|\edef\childdocname{\scantokens\expandafter{\jobname\noexpand}}|\\
|\def\childdocmain{|\textit{main}|}\||ifx\childdocmain\childdocname\||else|\\
|\childdoctrue\includeonly{\childdocname}\let\jobname\childdocmain\||fi|\\
\end{tabular}
\end{center}
%
Instead of |\childdocof{|\textit{main}|}| just include the main file
at the top of each child file:
%
\begin{center}
|\input{|\textit{main}|}|
\end{center}
%
A simple redirection |\childdocforward{|\textit{dest}|}| is achieved by:
%
\begin{center}
|\def\jobname{|\textit{dest}|}\input{\jobname}|
\end{center}
%
The redirection with prefix
|\childdocforwardprefix[|\textit{prefix}|]{|\textit{dest}|}|
is accomplished by:
%
\begin{center}
\begin{tabular}{l}
|{\edef\jobname{\scantokens\expandafter{\jobname\noexpand}}|\\
|\def\redirectjob |\textit{prefix}|#1~~~{\gdef\jobname{|\textit{dest}|#1}}|\\
|\expandafter\redirectjob\jobname~~~}\input{\jobname}|
\end{tabular}
\end{center}

In an alternative approach,
child documents can be compiled by a specific command line
without additional code or specific definitions:
%
\begin{center}
|... -jobname "|\textit{target}|" "|[\textit{flags}]%
|\includeonly{|\textit{dest}|}\input{|\textit{main}|}"|
\end{center}
%

%%%%%%%%%%%%%%%%%%%%%%%%%%%%%%%%%%%%%%%%%%%%%%%%%%%%%%%%%%%%%%%%%%%%%%%%%%%%%%%%
%%%%%%%%%%%%%%%%%%%%%%%%%%%%%%%%%%%%%%%%%%%%%%%%%%%%%%%%%%%%%%%%%%%%%%%%%%%%%%%%
\section{Information}

%%%%%%%%%%%%%%%%%%%%%%%%%%%%%%%%%%%%%%%%%%%%%%%%%%%%%%%%%%%%%%%%%%%%%%%%%%%%%%%%
\subsection{Copyright}

Copyright \copyright{} 2017--2018 Niklas Beisert

This work may be distributed and/or modified under the
conditions of the \LaTeX{} Project Public License, either version 1.3
of this license or (at your option) any later version.
The latest version of this license is in
  \url{http://www.latex-project.org/lppl.txt}
and version 1.3 or later is part of all distributions of \LaTeX{}
version 2005/12/01 or later.

This work has the LPPL maintenance status `maintained'.

The Current Maintainer of this work is Niklas Beisert.

This work consists of the files |README.txt|, |childdoc.ins| and |childdoc.dtx|
as well as the derived files |childdoc.def|, |cdocsamp.tex|
with |cdocsch1.tex|, |cdocsch2.tex|, |cdocspt3.tex|, |cdocspt4.tex|,
|cdocsdrf.tex|, |cdocsfn1.tex|, |cdocsfn2.tex|
as well as |childdoc.pdf|.

%%%%%%%%%%%%%%%%%%%%%%%%%%%%%%%%%%%%%%%%%%%%%%%%%%%%%%%%%%%%%%%%%%%%%%%%%%%%%%%%
\subsection{Files and Installation}

The package consists of the files:
%
\begin{center}
\begin{tabular}{ll}
    |README.txt|   & readme file \\
    |childdoc.ins| & installation file \\
    |childdoc.dtx| & source file \\
    |childdoc.def| & definition file \\
    |cdocsamp.tex| & sample main file \\
    |cdocsch1.tex| & sample include file \\
    |cdocsch2.tex| & sample include file \\
    |cdocspt3.tex| & sample part file \\
    |cdocspt4.tex| & sample part file \\
    |cdocsdrf.tex| & sample redirection file \\
    |cdocsfn1.tex| & sample redirection file \\
    |cdocsfn2.tex| & sample redirection file \\
    |childdoc.pdf| & manual
\end{tabular}
\end{center}
%
The distribution consists of the files
|README.txt|, |childdoc.ins| and |childdoc.dtx|.
%
\begin{itemize}
\item
Run (pdf)\LaTeX{} on |childdoc.dtx|
to compile the manual |childdoc.pdf| (this file).
\item
Run \LaTeX{} on |childdoc.ins| to create the definitions file |childdoc.def|
and the sample |cdocsamp.tex| with include files
|cdocsch1.tex|, |cdocsch2.tex|, |cdocspt3.tex|, |cdocspt4.tex|,
|cdocsdrf.tex|, |cdocsfn1.tex|, |cdocsfn2.tex|.
Then copy the file |childdoc.def| to an appropriate directory of your \LaTeX{}
distribution, e.g.\ \textit{texmf-root}|/tex/latex/childdoc|.
\end{itemize}

%%%%%%%%%%%%%%%%%%%%%%%%%%%%%%%%%%%%%%%%%%%%%%%%%%%%%%%%%%%%%%%%%%%%%%%%%%%%%%%%
\subsection{Related CTAN Packages}

There are several other packages which offer a similar functionality:
%
\begin{itemize}
\item
The packages
\href{http://ctan.org/pkg/docmute}{\textsf{docmute}},
\href{http://ctan.org/pkg/includex}{\textsf{includex}} and
\href{http://ctan.org/pkg/standalone}{\textsf{standalone}}
provide commands to include only the document body of
a child file thus allowing both files to be compiled individually.
\item
The packages \href{http://ctan.org/pkg/subdocs}{\textsf{subdocs}}
and \href{http://ctan.org/pkg/subfiles}{\textsf{subfiles}}
provide structures in which the main and child documents can be
encapsulated and allowing them to be compiled individually.
The inclusion mechanism is different from the conventional |\include|.
\item
The package \href{http://ctan.org/pkg/combine}{\textsf{combine}}
is an elaborate solution to combine several documents into one.
\end{itemize}
%
See also the CTAN topic \href{http://ctan.org/topic/subdocs}{\textsf{subdocs}}
for further related packages.
The present package differs from the above solutions in that
a document structure constructed with the conventional |\include| mechanism
just needs two extra commands at the top of every file
such that all constituent files can be compiled individually.

%%%%%%%%%%%%%%%%%%%%%%%%%%%%%%%%%%%%%%%%%%%%%%%%%%%%%%%%%%%%%%%%%%%%%%%%%%%%%%%%
%\subsection{Feature Suggestions}
%
%The following is a list of features which may be useful for future
%versions of this package:
%%
%\begin{itemize}
%\item
%\ldots
%\end{itemize}

%%%%%%%%%%%%%%%%%%%%%%%%%%%%%%%%%%%%%%%%%%%%%%%%%%%%%%%%%%%%%%%%%%%%%%%%%%%%%%%%
\subsection{Revision History}

%%%%%%%%%%%%%%%%%%%%%%%%%%%%%%%%%%%%%%%%
\paragraph{v2.0:} 2018/12/30

\begin{itemize}
\item
immediate forward processing
\item
added |\childdocby| mechanism
\item
manual restructured
\end{itemize}

%%%%%%%%%%%%%%%%%%%%%%%%%%%%%%%%%%%%%%%%
\paragraph{v1.6:} 2018/01/17

\begin{itemize}
\item
application for development of include files
\item
corrections to manual
\end{itemize}

%%%%%%%%%%%%%%%%%%%%%%%%%%%%%%%%%%%%%%%%
\paragraph{v1.5:} 2017/05/21

\begin{itemize}
\item
more complete structuring introduced
\item
|\childdocof| introduced
\item
|\childdoc| renamed to |\childdocmain|
\item
|\childredirect| renamed to |\childdocforward| and |\childdocforwardprefix|
and functionality expanded
\end{itemize}

%%%%%%%%%%%%%%%%%%%%%%%%%%%%%%%%%%%%%%%%
\paragraph{v1.0:} 2017/04/27

\begin{itemize}
\item
manual and install package
\item
first version published on CTAN
\end{itemize}

%%%%%%%%%%%%%%%%%%%%%%%%%%%%%%%%%%%%%%%%
\paragraph{v0.6:} 2017/04/26

\begin{itemize}
\item
redirection mechanism added
\end{itemize}

%%%%%%%%%%%%%%%%%%%%%%%%%%%%%%%%%%%%%%%%
\paragraph{v0.5:} 2017/04/26

\begin{itemize}
\item
functionality in definition file
\end{itemize}


%%%%%%%%%%%%%%%%%%%%%%%%%%%%%%%%%%%%%%%%%%%%%%%%%%%%%%%%%%%%%%%%%%%%%%%%%%%%%%%%
%%%%%%%%%%%%%%%%%%%%%%%%%%%%%%%%%%%%%%%%%%%%%%%%%%%%%%%%%%%%%%%%%%%%%%%%%%%%%%%%
%%%%%%%%%%%%%%%%%%%%%%%%%%%%%%%%%%%%%%%%%%%%%%%%%%%%%%%%%%%%%%%%%%%%%%%%%%%%%%%%
\appendix

\settowidth\MacroIndent{\rmfamily\scriptsize 000\ }

 \DocInput{childdoc.dtx}

\end{document}
%</driver>
% \fi
%
% %%%%%%%%%%%%%%%%%%%%%%%%%%%%%%%%%%%%%%%%%%%%%%%%%%%%%%%%%%%%%%%%%%%%%%%%%%%%%%
% %%%%%%%%%%%%%%%%%%%%%%%%%%%%%%%%%%%%%%%%%%%%%%%%%%%%%%%%%%%%%%%%%%%%%%%%%%%%%%
% \section{Sample}
%\iffalse
%<*samplemain>
%\fi
%
% The following presents a sample document
% with two chapters, two parts, a title page,
% a compile flag as well as three forwarding files to set the flag.
% It consists of eight |.tex| files:
% \begin{center}
% \begin{tabular}{ll}
% |cdocsamp.tex|&main file\\
% |cdocsch1.tex|&include file for chapter 1\\
% |cdocsch2.tex|&include file for chapter 2\\
% |cdocspt3.tex|&include file for part 3\\
% |cdocspt4.tex|&include file for part 4\\
% |cdocsdrf.tex|&forwarding file for main file in draft mode\\
% |cdocsfi1.tex|&forwarding file for final version of chapter 1\\
% |cdocsfi2.tex|&forwarding file for final version of chapter 2\\
% \end{tabular}
% \end{center}
% Each of the eight files can be compiled directly by the \LaTeX{} compiler.
%
% %%%%%%%%%%%%%%%%%%%%%%%%%%%%%%%%%%%%%%
% \paragraph{Main File.}
%
% The main file is called |cdocsamp.tex|.
%
% Load the \textsf{childdoc} definitions and
% declare the filename for the main document:
%    \begin{macrocode}
\input{childdoc.def}
\childdocmain{}
%    \end{macrocode}

% Optional override for |\version| flag:
%    \begin{macrocode}
%%\ifchilddoc\else\providecommand{\version}{draft}\fi
%    \end{macrocode}

% Define the default values for the |\version| flag
% (|final| for the main file and |draft| for childs):
%    \begin{macrocode}
\ifchilddoc
\providecommand{\version}{draft}
\else
\providecommand{\version}{final}
\fi
%    \end{macrocode}

% Load the standard document class:
%    \begin{macrocode}
\documentclass[12pt]{article}
%    \end{macrocode}

% Start the document body:
%    \begin{macrocode}
\begin{document}
%    \end{macrocode}

% Declare a title page.
% Print title, part of document being processed and version flag:
%    \begin{macrocode}
\addtocounter{page}{-1}
\begin{center}
{\LARGE\bfseries{}childdoc example\par}
\vspace{1cm}
\ifchilddoc
\ifchilddocmanual part\else chapter\fi:
`\childdocname' of `\childdocjob'\par
\else
main document: `\childdocjob'\par
\fi
version: \version\par
\end{center}
\newpage
%    \end{macrocode}

% Manually include selected file,
% otherwise process as usual:
%    \begin{macrocode}
\ifchilddocmanual
\section*{part `\childdocname'}
\input{\childdocname}
\else
%    \end{macrocode}

% Include the two chapters:
%    \begin{macrocode}
\include{cdocsch1}
\include{cdocsch2}
%    \end{macrocode}

% Include the two parts unless only chapters should be displayed:
%    \begin{macrocode}
\ifchilddoc\else
\section{part three}
\input{cdocspt3}
\section{part four}
\input{cdocspt4}
\fi
%    \end{macrocode}

% Process as usual until here:
%    \begin{macrocode}
\fi
%    \end{macrocode}

% End of document body:
%    \begin{macrocode}
\end{document}
%    \end{macrocode}
%\iffalse
%</samplemain>
%\fi
%
% %%%%%%%%%%%%%%%%%%%%%%%%%%%%%%%%%%%%%%
% \paragraph{Chapter Include Files.}
%
% The include files are called |cdocsch1.tex| and |cdocsch2.tex|.
%
%\iffalse
%<*samplechap1|samplechap2>
%\fi

% Optional override for |\version| flag:
%    \begin{macrocode}
%%\providecommand{\version}{final}
%    \end{macrocode}

% Include the main document:
%    \begin{macrocode}
\input{childdoc.def}
\childdocof{cdocsamp}
%    \end{macrocode}

%\iffalse
%</samplechap1|samplechap2>
%\fi
%
%\iffalse
%<*samplechap1>
%\fi
% Some text for chapter 1:
%    \begin{macrocode}
\section{one}
some text in chapter one
%    \end{macrocode}

%\iffalse
%</samplechap1>
%\fi
% Some text for chapter 2:
%\iffalse
%<*samplechap2>
%\fi
%    \begin{macrocode}
\section{two}
more text in chapter two
%    \end{macrocode}

%\iffalse
%</samplechap2>
%\fi
%
% %%%%%%%%%%%%%%%%%%%%%%%%%%%%%%%%%%%%%%
% \paragraph{Part Include Files.}
%
% The include files are called |cdocspt3.tex| and |cdocspt4.tex|.
%
%\iffalse
%<*samplepart3|samplepart4>
%\fi

% Optional override for |\version| flag:
%    \begin{macrocode}
%%\providecommand{\version}{final}
%    \end{macrocode}

% Include the main document:
%    \begin{macrocode}
\input{childdoc.def}
\childdocby{cdocsamp}
%    \end{macrocode}

%\iffalse
%</samplepart3|samplepart4>
%\fi
%
%\iffalse
%<*samplepart3>
%\fi
% Some text for part 3:
%    \begin{macrocode}
some text in part three
%    \end{macrocode}

%\iffalse
%</samplepart3>
%\fi
% Some text for part 4:
%\iffalse
%<*samplepart4>
%\fi
%    \begin{macrocode}
more text in part four
%    \end{macrocode}

%\iffalse
%</samplepart4>
%\fi
%
% %%%%%%%%%%%%%%%%%%%%%%%%%%%%%%%%%%%%%%
% \paragraph{Forwarding for a Complete Draft.}
%
% The following forwarding file |cdocsdrf.tex|
% compiles the main document in draft mode:
%\iffalse
%<*sampledraft>
%\fi
%    \begin{macrocode}
\def\version{draft}
\input{childdoc.def}
\childdocforward{cdocsamp}
%    \end{macrocode}

%\iffalse
%</sampledraft>
%\fi
%
% %%%%%%%%%%%%%%%%%%%%%%%%%%%%%%%%%%%%%%
% \paragraph{Forwarding for Final Version of the Chapters.}
%
% The following forwarding files |cdocsfn1.tex| and |cdocsfn2.tex|
% (with identical content)
% compile the final versions of the child documents
% |cdocsch1.tex| and |cdocsch2.tex|, respectively:
%\iffalse
%<*samplefinal>
%\fi
%    \begin{macrocode}
\def\version{final}
\input{childdoc.def}
\childdocforwardprefix[cdocsamp]{cdocsfn}{cdocsch}
%    \end{macrocode}

%\iffalse
%</samplefinal>
%\fi
%
% %%%%%%%%%%%%%%%%%%%%%%%%%%%%%%%%%%%%%%
% \paragraph{Command Line Processing.}
%
% The following three command lines generate the output files
% |cdocscld|, |cdocscl1| and |cdocscl2|
% which should be identical to
% |cdocsdrf|, |cdocsch1| and |cdocsfn2|, respectively:
% \begin{center}
% \begin{tabular}{l}
% |latex -jobname cdocscld \|\\
% |  "\def\version{draft}\input{childdoc.def}\childdocforward{cdocsamp}"|\\
% |latex -jobname cdocscl1 \|\\
% |  "\input{childdoc.def}\childdocforward[cdocsamp]{cdocsch1}"|\\
% |latex -jobname cdocscl2 \|\\
% |  "\def\version{final}\input{childdoc.def}\childdocforward{cdocsch2}"|
% \end{tabular}
% \end{center}
% Note that the trailing backslash on each first line
% merely continues the input to the second line
% (for convenient cut ant paste).
% Furthermore, the command |latex| can be replaced by any
% of its alternative versions such as |pdflatex|.
%
% %%%%%%%%%%%%%%%%%%%%%%%%%%%%%%%%%%%%%%%%%%%%%%%%%%%%%%%%%%%%%%%%%%%%%%%%%%%%%%
% %%%%%%%%%%%%%%%%%%%%%%%%%%%%%%%%%%%%%%%%%%%%%%%%%%%%%%%%%%%%%%%%%%%%%%%%%%%%%%
% \section{Implementation}
%\iffalse
%<*package>
%\fi
%
% This section describes the definitions file |childdoc.def|.

% The definitions cannot be loaded using |\usepackage| or |\RequirePackage|
% which has a mechanism to prevent loading a style file more than once.
% When loading the definitions by means of |\input|
% multiple instances have to be prevented manually:
%\iffalse
%This code needs to be before the `\ProvidesFile' directive
%which is defined at the beginning of this file.
%Therefore it is also placed there and commented out here.
%</package>
%<*discard>
%\fi
%    \begin{macrocode}
\ifdefined\childdocmain\endinput\fi
%    \end{macrocode}
%\iffalse
%</discard>
%<*package>
%\fi
%
% \macro{\ifchilddoc}
% \macro{\ifchilddocmanual}
% The conditional |\ifchilddoc| tells whether a
% child (true) or main (false) document is being compiled.
% The conditional |\ifchilddocmanual| tells whether
% the |\includeonly| mechanism is used (false) or
% the selection of child files must be performed manually (true).
% The definitions initialise to false:
%    \begin{macrocode}
\newif\ifchilddoc
\newif\ifchilddocmanual
%    \end{macrocode}

% \macro{\childdocname}
% \macro{\childdocjob}
% The macro |\childdocname| stores the name of the main document
% to be compiled. The macro |\childdocjob| stores the name of
% the document on which the \LaTeX{} compiler was originally invoked.
% The content of |\jobname| cannot be compared
% to filenames specified in the source due to different catcodes.
% The following code rescans |\jobname|, stores the result
% in |\childdocname| and saves a copy in |\childdocjob|:
%    \begin{macrocode}
\edef\childdocname{\scantokens\expandafter{\jobname\noexpand}}
\let\childdocjob\childdocname
%    \end{macrocode}

% \macro{\childdocdisable}
% The macro |\childdocdisable| prevents the main file
% from being processed more than once.
% At this stage, the main document command |\childdocmain|
% is assumed to be called once again where it should do nothing.
% Any subsequent call to it should prevent
% a secondary processing of the main document
% It overwrites the forwarding commands
% |\childdocof| and |\childdocforward|
% with empty macros to prevent further inclusions of the main document:
%    \begin{macrocode}
\newcommand{\childdocdisable}
{
  \renewcommand{\childdocmain}[1]{\renewcommand{\childdocmain}[1]{\endinput}}
  \renewcommand{\childdocof}[1]{}
  \renewcommand{\childdocby}[2][]{}
  \renewcommand{\childdocforward}[2][]{}
  \renewcommand{\childdocdisable}{}
}
%    \end{macrocode}

% \macro{\childdocmain}
% The macro |\childdocmain| is to be called at the top of the main file
% with nothing or the main filename (without extension) as argument.
% First, it breaks loops.
% If the argument is not empty and does not match |\childdocname|
% (which is set by the first inclusion of |childdoc.def|),
% |\ifchilddoc| is set to true, |\includeonly| is applied to the child file
% and |\jobname| is set to the main file
% (for proper handling of |.aux| files):
%    \begin{macrocode}
\newcommand{\childdocmain}[1]
{
  \childdocdisable\childdocmain{}
  \if?#1?\else
    \begingroup
      \def\childdoctmp{#1}
      \ifx\childdoctmp\childdocname
        \def\childdoctmp{}
      \else
        \def\childdoctmp
        {
          \childdoctrue
          \includeonly{\childdocname}
          \def\childdocjob{#1}
          \def\jobname{#1}
        }
      \fi
      \expandafter
    \endgroup
    \childdoctmp
  \fi
}
%    \end{macrocode}

% \macro{\childdocof}
% The command |\childdocof| redirects
% compilation to the main file |#1|.
%    \begin{macrocode}
\newcommand{\childdocof}[1]
{
  \childdocdisable
  \childdoctrue
  \includeonly{\childdocname}
  \def\jobname{#1}
  \def\childdocjob{#1}
  \input{#1}
}
%    \end{macrocode}

% \macro{\childdocby}
% The command |\childdocby| ....
%    \begin{macrocode}
\newcommand{\childdocby}[2][]
{
  \childdocdisable
  \childdoctrue
  \childdocmanualtrue
  \if?#1?\else
    \def\jobname{#2}
  \fi
  \def\childdocjob{#2}
  \input{#2}
  \endinput
}
%    \end{macrocode}

% \macro{\childdocforward}
% The command |\childdocforward| redirects
% compilation to the main file or
% (if the optional argument is given) a child file.
% Parameters are set as if the main file
% or a child file starting with |\childdocof| was compiled.
% Then compilation is handed over to the main file:
%    \begin{macrocode}
\newcommand{\childdocforward}[2][]
{
  \begingroup
    \if?#1?
      \def\childdoctmp
      {
        \def\childdocname{#2}
        \def\childdocjob{#2}
        \def\jobname{#2}
        \input{#2}
        \endinput
      }
    \else
      \def\childdoctmp
      {
        \childdocdisable
        \def\childdocname{#2}
        \childdoctrue
        \includeonly{#2}
        \def\childdocjob{#1}
        \def\jobname{#1}
        \input{#1}
        \endinput
      }
    \fi
    \expandafter
  \endgroup
  \childdoctmp
}
%    \end{macrocode}

% \macro{\childdocforwardprefix}
% The command |\childdocforwardprefix| redirects
% compilation to the main or a child file by means of a pattern.
% The prefix |#1| in the current filename is replaced by |#2|
% and the suffix of the current filename is kept
% (it is assumed that the filename does not contain the substring `|~~~|'
% which is used as a delimiter).
% Compilation is handed over to the new file by |\childdocforward|:
%    \begin{macrocode}
\newcommand{\childdocforwardprefix}[3][]
{
  \begingroup
    \def\childdocextract #2##1~~~{\def\childdoctmp{\childdocforward[#1]{#3##1}}}
    \expandafter\childdocextract\childdocname~~~
    \expandafter
  \endgroup
  \childdoctmp
}
%    \end{macrocode}

% \macro{\childdoc}
% The deprecated macro |\childdoc| is a legacy version of |\childdocmain|:
%    \begin{macrocode}
\newcommand{\childdoc}{\childdocmain}
%    \end{macrocode}

% \macro{\childdocredirect}
% The deprecated macro |\childdocredirect| is a legacy version
% of |\childdocforward| and |\childdocforwardprefix|:
%    \begin{macrocode}
\newcommand{\childdocredirect}[2][]
{
  \begingroup
    \if?#1?
      \def\childdoctmp{\childdocforward{#2}}
    \else
      \def\childdoctmp{\childdocforwardprefix{#1}{#2}}
    \fi
    \expandafter
  \endgroup
  \childdoctmp
}
%    \end{macrocode}

%\iffalse
%</package>
%\fi
%
\endinput
\childdocforward{cdocsch2}"|
% \end{tabular}
% \end{center}
% Note that the trailing backslash on each first line
% merely continues the input to the second line
% (for convenient cut ant paste).
% Furthermore, the command |latex| can be replaced by any
% of its alternative versions such as |pdflatex|.
%
% %%%%%%%%%%%%%%%%%%%%%%%%%%%%%%%%%%%%%%%%%%%%%%%%%%%%%%%%%%%%%%%%%%%%%%%%%%%%%%
% %%%%%%%%%%%%%%%%%%%%%%%%%%%%%%%%%%%%%%%%%%%%%%%%%%%%%%%%%%%%%%%%%%%%%%%%%%%%%%
% \section{Implementation}
%\iffalse
%<*package>
%\fi
%
% This section describes the definitions file |childdoc.def|.

% The definitions cannot be loaded using |\usepackage| or |\RequirePackage|
% which has a mechanism to prevent loading a style file more than once.
% When loading the definitions by means of |\input|
% multiple instances have to be prevented manually:
%\iffalse
%This code needs to be before the `\ProvidesFile' directive
%which is defined at the beginning of this file.
%Therefore it is also placed there and commented out here.
%</package>
%<*discard>
%\fi
%    \begin{macrocode}
\ifdefined\childdocmain\endinput\fi
%    \end{macrocode}
%\iffalse
%</discard>
%<*package>
%\fi
%
% \macro{\ifchilddoc}
% \macro{\ifchilddocmanual}
% The conditional |\ifchilddoc| tells whether a
% child (true) or main (false) document is being compiled.
% The conditional |\ifchilddocmanual| tells whether
% the |\includeonly| mechanism is used (false) or
% the selection of child files must be performed manually (true).
% The definitions initialise to false:
%    \begin{macrocode}
\newif\ifchilddoc
\newif\ifchilddocmanual
%    \end{macrocode}

% \macro{\childdocname}
% \macro{\childdocjob}
% The macro |\childdocname| stores the name of the main document
% to be compiled. The macro |\childdocjob| stores the name of
% the document on which the \LaTeX{} compiler was originally invoked.
% The content of |\jobname| cannot be compared
% to filenames specified in the source due to different catcodes.
% The following code rescans |\jobname|, stores the result
% in |\childdocname| and saves a copy in |\childdocjob|:
%    \begin{macrocode}
\edef\childdocname{\scantokens\expandafter{\jobname\noexpand}}
\let\childdocjob\childdocname
%    \end{macrocode}

% \macro{\childdocdisable}
% The macro |\childdocdisable| prevents the main file
% from being processed more than once.
% At this stage, the main document command |\childdocmain|
% is assumed to be called once again where it should do nothing.
% Any subsequent call to it should prevent
% a secondary processing of the main document
% It overwrites the forwarding commands
% |\childdocof| and |\childdocforward|
% with empty macros to prevent further inclusions of the main document:
%    \begin{macrocode}
\newcommand{\childdocdisable}
{
  \renewcommand{\childdocmain}[1]{\renewcommand{\childdocmain}[1]{\endinput}}
  \renewcommand{\childdocof}[1]{}
  \renewcommand{\childdocby}[2][]{}
  \renewcommand{\childdocforward}[2][]{}
  \renewcommand{\childdocdisable}{}
}
%    \end{macrocode}

% \macro{\childdocmain}
% The macro |\childdocmain| is to be called at the top of the main file
% with nothing or the main filename (without extension) as argument.
% First, it breaks loops.
% If the argument is not empty and does not match |\childdocname|
% (which is set by the first inclusion of |childdoc.def|),
% |\ifchilddoc| is set to true, |\includeonly| is applied to the child file
% and |\jobname| is set to the main file
% (for proper handling of |.aux| files):
%    \begin{macrocode}
\newcommand{\childdocmain}[1]
{
  \childdocdisable\childdocmain{}
  \if?#1?\else
    \begingroup
      \def\childdoctmp{#1}
      \ifx\childdoctmp\childdocname
        \def\childdoctmp{}
      \else
        \def\childdoctmp
        {
          \childdoctrue
          \includeonly{\childdocname}
          \def\childdocjob{#1}
          \def\jobname{#1}
        }
      \fi
      \expandafter
    \endgroup
    \childdoctmp
  \fi
}
%    \end{macrocode}

% \macro{\childdocof}
% The command |\childdocof| redirects
% compilation to the main file |#1|.
%    \begin{macrocode}
\newcommand{\childdocof}[1]
{
  \childdocdisable
  \childdoctrue
  \includeonly{\childdocname}
  \def\jobname{#1}
  \def\childdocjob{#1}
  \input{#1}
}
%    \end{macrocode}

% \macro{\childdocby}
% The command |\childdocby| ....
%    \begin{macrocode}
\newcommand{\childdocby}[2][]
{
  \childdocdisable
  \childdoctrue
  \childdocmanualtrue
  \if?#1?\else
    \def\jobname{#2}
  \fi
  \def\childdocjob{#2}
  \input{#2}
  \endinput
}
%    \end{macrocode}

% \macro{\childdocforward}
% The command |\childdocforward| redirects
% compilation to the main file or
% (if the optional argument is given) a child file.
% Parameters are set as if the main file
% or a child file starting with |\childdocof| was compiled.
% Then compilation is handed over to the main file:
%    \begin{macrocode}
\newcommand{\childdocforward}[2][]
{
  \begingroup
    \if?#1?
      \def\childdoctmp
      {
        \def\childdocname{#2}
        \def\childdocjob{#2}
        \def\jobname{#2}
        \input{#2}
        \endinput
      }
    \else
      \def\childdoctmp
      {
        \childdocdisable
        \def\childdocname{#2}
        \childdoctrue
        \includeonly{#2}
        \def\childdocjob{#1}
        \def\jobname{#1}
        \input{#1}
        \endinput
      }
    \fi
    \expandafter
  \endgroup
  \childdoctmp
}
%    \end{macrocode}

% \macro{\childdocforwardprefix}
% The command |\childdocforwardprefix| redirects
% compilation to the main or a child file by means of a pattern.
% The prefix |#1| in the current filename is replaced by |#2|
% and the suffix of the current filename is kept
% (it is assumed that the filename does not contain the substring `|~~~|'
% which is used as a delimiter).
% Compilation is handed over to the new file by |\childdocforward|:
%    \begin{macrocode}
\newcommand{\childdocforwardprefix}[3][]
{
  \begingroup
    \def\childdocextract #2##1~~~{\def\childdoctmp{\childdocforward[#1]{#3##1}}}
    \expandafter\childdocextract\childdocname~~~
    \expandafter
  \endgroup
  \childdoctmp
}
%    \end{macrocode}

% \macro{\childdoc}
% The deprecated macro |\childdoc| is a legacy version of |\childdocmain|:
%    \begin{macrocode}
\newcommand{\childdoc}{\childdocmain}
%    \end{macrocode}

% \macro{\childdocredirect}
% The deprecated macro |\childdocredirect| is a legacy version
% of |\childdocforward| and |\childdocforwardprefix|:
%    \begin{macrocode}
\newcommand{\childdocredirect}[2][]
{
  \begingroup
    \if?#1?
      \def\childdoctmp{\childdocforward{#2}}
    \else
      \def\childdoctmp{\childdocforwardprefix{#1}{#2}}
    \fi
    \expandafter
  \endgroup
  \childdoctmp
}
%    \end{macrocode}

%\iffalse
%</package>
%\fi
%
\endinput
|\\
|\childdocforward{|\textit{main}|}|
\end{tabular}
\end{center}
%
Likewise, the following files |final|\textit{nn}|.tex|
compile the final version of the child document
|child|\textit{nn}|.tex|:
%
\begin{center}
\begin{tabular}{l}
|\def\version{final}|\\
|% \iffalse
%
% childdoc.dtx Copyright (C) 2017-2018 Niklas Beisert
%
% This work may be distributed and/or modified under the
% conditions of the LaTeX Project Public License, either version 1.3
% of this license or (at your option) any later version.
% The latest version of this license is in
%   http://www.latex-project.org/lppl.txt
% and version 1.3 or later is part of all distributions of LaTeX
% version 2005/12/01 or later.
%
% This work has the LPPL maintenance status `maintained'.
%
% The Current Maintainer of this work is Niklas Beisert.
%
% This work consists of the files childdoc.dtx and childdoc.ins
% and the derived files childdoc.def and cdocsamp.tex with
% cdocsch1.tex, cdocsch2.tex, cdocsdrf.tex, cdocsfn1.tex, cdocsfn2.tex.
%
%<package>\ifdefined\childdocmain\endinput\fi
%<package>\ProvidesFile{childdoc.def}[2018/12/30 v2.0 child document driver]
%<samplemain>\ProvidesFile{cdocsamp.tex}[2018/12/30 v2.0 sample for childdoc]
%<*driver>
%\ProvidesFile{childdoc.drv}[2018/12/30 v2.0 childdoc reference manual file]
\PassOptionsToClass{10pt,a4paper}{article}
\documentclass{ltxdoc}

\usepackage[margin=35mm]{geometry}
\usepackage{hyperref}
\usepackage{hyperxmp}
\usepackage[usenames]{color}

\hypersetup{colorlinks=true}
\hypersetup{pdfstartview=FitH}
\hypersetup{pdfpagemode=UseNone}
\hypersetup{pdfsource={}}
\hypersetup{pdflang={en-UK}}
\hypersetup{pdfcopyright={Copyright 2017-2018 Niklas Beisert.
  This work may be distributed and/or modified under the
  conditions of the LaTeX Project Public License, either version 1.3
  of this license or (at your option) any later version.}}
\hypersetup{pdflicenseurl={http://www.latex-project.org/lppl.txt}}
\hypersetup{pdfcontactaddress={ETH Zurich, ITP, HIT K,
  Wolfgang-Pauli-Strasse 27}}
\hypersetup{pdfcontactpostcode={8093}}
\hypersetup{pdfcontactcity={Zurich}}
\hypersetup{pdfcontactcountry={Switzerland}}
\hypersetup{pdfcontactemail={nbeisert@itp.phys.ethz.ch}}
\hypersetup{pdfcontacturl={http://people.phys.ethz.ch/\xmptilde nbeisert/}}

\newcommand{\secref}[1]{\hyperref[#1]{section \ref*{#1}}}

\parskip1ex
\parindent0pt
\let\olditemize\itemize
\def\itemize{\olditemize\parskip0pt}

\begin{document}

\title{The \textsf{childdoc} Package}
\hypersetup{pdftitle={The childdoc Package}}
\author{Niklas Beisert\\[2ex]
  Institut f\"ur Theoretische Physik\\
  Eidgen\"ossische Technische Hochschule Z\"urich\\
  Wolfgang-Pauli-Strasse 27, 8093 Z\"urich, Switzerland\\[1ex]
  \href{mailto:nbeisert@itp.phys.ethz.ch}
  {\texttt{nbeisert@itp.phys.ethz.ch}}}
\hypersetup{pdfauthor={Niklas Beisert}}
\hypersetup{pdfsubject={Manual for the LaTeX2e Package childdoc}}
\date{30 December 2018, \textsf{v2.0}}
\maketitle

\begin{abstract}\noindent
\textsf{childdoc} is a \LaTeXe{} package
that enables the direct compilation
of document sections included by |\include|
to individual files.
\end{abstract}

\begingroup
\parskip0ex
\tableofcontents
\endgroup

%%%%%%%%%%%%%%%%%%%%%%%%%%%%%%%%%%%%%%%%%%%%%%%%%%%%%%%%%%%%%%%%%%%%%%%%%%%%%%%%
%%%%%%%%%%%%%%%%%%%%%%%%%%%%%%%%%%%%%%%%%%%%%%%%%%%%%%%%%%%%%%%%%%%%%%%%%%%%%%%%
\section{Introduction}

\LaTeX{} provides a mechanism to structure a large document (such as a book)
into a main file and several child files (containing the chapters)
using the |\include| command.
This mechanism is beneficial for documents
which span hundreds of pages in order to
make the source file(s) more manageable.
Moreover, compilation can be restricted to
selected child files by means of the |\includeonly| command.
The latter feature can be used to reduce the compilation time while editing
(this was significantly more useful in the earlier days of \LaTeX{})
or to generate a smaller document which is easier to navigate.
Another application of |\includeonly| is to generate
documents consisting of selected parts of the complete document.

However, there are a few drawbacks of the plain |\include| mechanism:
\begin{itemize}
\item
The child files cannot be compiled on their own,
they can only be compiled via the main file.
A naive editing environment
(such as a text editor with an option
to have the current file processed by \LaTeX)
may require one to switch to the main file before compiling;
attempting to compile the child file produces errors.
\item
The main file must be modified (each time)
to adjust the |\includeonly| command
to the present needs. This easily leaves the main file in a messy state.
\item
The generated document will always carry the filename
of the main document. This is inconvenient if
several child files are to be compiled and
to be kept for distribution.
\end{itemize}

The present package provides a simple interface
to make child files individually compilable by \LaTeX{}.
Compiling a child file then has the same effect as compiling
the main file with an |\includeonly| command
to select the appropriate child.
Moreover the generated document will carry the name of the child
rather than the main file.
This resolves all three above issues.

This feature is meant to make the editing of books,
thesis documents and lecture notes somewhat more convenient.
However, the package can also be used efficiently for
composing a series of documents (such as exercise sheets)
which are typically distributed individually.
It then assists the author in generating the individual documents
(potentially in different versions)
as well as a document containing the collected series.
Another application is in developing style files
or other kinds of included material
where compilation of the style file could redirect
to a sample or test file.

%%%%%%%%%%%%%%%%%%%%%%%%%%%%%%%%%%%%%%%%%%%%%%%%%%%%%%%%%%%%%%%%%%%%%%%%%%%%%%%%
%%%%%%%%%%%%%%%%%%%%%%%%%%%%%%%%%%%%%%%%%%%%%%%%%%%%%%%%%%%%%%%%%%%%%%%%%%%%%%%%
\section{Usage}

First of all, the package \textsf{childdoc} is \emph{not} a standard
\LaTeXe{} |.sty| style file! Therefore it needs to be invoked in
a non-standard way.

%%%%%%%%%%%%%%%%%%%%%%%%%%%%%%%%%%%%%%%%%%%%%%%%%%%%%%%%%%%%%%%%%%%%%%%%%%%%%%%%
\subsection{Included Files}
\label{sec:include}

%%%%%%%%%%%%%%%%%%%%%%%%%%%%%%%%%%%%%%%%
\DescribeMacro{\childdocmain}
To use the package, add the commands
\begin{center}
\begin{tabular}{l}
|% \iffalse
%
% childdoc.dtx Copyright (C) 2017-2018 Niklas Beisert
%
% This work may be distributed and/or modified under the
% conditions of the LaTeX Project Public License, either version 1.3
% of this license or (at your option) any later version.
% The latest version of this license is in
%   http://www.latex-project.org/lppl.txt
% and version 1.3 or later is part of all distributions of LaTeX
% version 2005/12/01 or later.
%
% This work has the LPPL maintenance status `maintained'.
%
% The Current Maintainer of this work is Niklas Beisert.
%
% This work consists of the files childdoc.dtx and childdoc.ins
% and the derived files childdoc.def and cdocsamp.tex with
% cdocsch1.tex, cdocsch2.tex, cdocsdrf.tex, cdocsfn1.tex, cdocsfn2.tex.
%
%<package>\ifdefined\childdocmain\endinput\fi
%<package>\ProvidesFile{childdoc.def}[2018/12/30 v2.0 child document driver]
%<samplemain>\ProvidesFile{cdocsamp.tex}[2018/12/30 v2.0 sample for childdoc]
%<*driver>
%\ProvidesFile{childdoc.drv}[2018/12/30 v2.0 childdoc reference manual file]
\PassOptionsToClass{10pt,a4paper}{article}
\documentclass{ltxdoc}

\usepackage[margin=35mm]{geometry}
\usepackage{hyperref}
\usepackage{hyperxmp}
\usepackage[usenames]{color}

\hypersetup{colorlinks=true}
\hypersetup{pdfstartview=FitH}
\hypersetup{pdfpagemode=UseNone}
\hypersetup{pdfsource={}}
\hypersetup{pdflang={en-UK}}
\hypersetup{pdfcopyright={Copyright 2017-2018 Niklas Beisert.
  This work may be distributed and/or modified under the
  conditions of the LaTeX Project Public License, either version 1.3
  of this license or (at your option) any later version.}}
\hypersetup{pdflicenseurl={http://www.latex-project.org/lppl.txt}}
\hypersetup{pdfcontactaddress={ETH Zurich, ITP, HIT K,
  Wolfgang-Pauli-Strasse 27}}
\hypersetup{pdfcontactpostcode={8093}}
\hypersetup{pdfcontactcity={Zurich}}
\hypersetup{pdfcontactcountry={Switzerland}}
\hypersetup{pdfcontactemail={nbeisert@itp.phys.ethz.ch}}
\hypersetup{pdfcontacturl={http://people.phys.ethz.ch/\xmptilde nbeisert/}}

\newcommand{\secref}[1]{\hyperref[#1]{section \ref*{#1}}}

\parskip1ex
\parindent0pt
\let\olditemize\itemize
\def\itemize{\olditemize\parskip0pt}

\begin{document}

\title{The \textsf{childdoc} Package}
\hypersetup{pdftitle={The childdoc Package}}
\author{Niklas Beisert\\[2ex]
  Institut f\"ur Theoretische Physik\\
  Eidgen\"ossische Technische Hochschule Z\"urich\\
  Wolfgang-Pauli-Strasse 27, 8093 Z\"urich, Switzerland\\[1ex]
  \href{mailto:nbeisert@itp.phys.ethz.ch}
  {\texttt{nbeisert@itp.phys.ethz.ch}}}
\hypersetup{pdfauthor={Niklas Beisert}}
\hypersetup{pdfsubject={Manual for the LaTeX2e Package childdoc}}
\date{30 December 2018, \textsf{v2.0}}
\maketitle

\begin{abstract}\noindent
\textsf{childdoc} is a \LaTeXe{} package
that enables the direct compilation
of document sections included by |\include|
to individual files.
\end{abstract}

\begingroup
\parskip0ex
\tableofcontents
\endgroup

%%%%%%%%%%%%%%%%%%%%%%%%%%%%%%%%%%%%%%%%%%%%%%%%%%%%%%%%%%%%%%%%%%%%%%%%%%%%%%%%
%%%%%%%%%%%%%%%%%%%%%%%%%%%%%%%%%%%%%%%%%%%%%%%%%%%%%%%%%%%%%%%%%%%%%%%%%%%%%%%%
\section{Introduction}

\LaTeX{} provides a mechanism to structure a large document (such as a book)
into a main file and several child files (containing the chapters)
using the |\include| command.
This mechanism is beneficial for documents
which span hundreds of pages in order to
make the source file(s) more manageable.
Moreover, compilation can be restricted to
selected child files by means of the |\includeonly| command.
The latter feature can be used to reduce the compilation time while editing
(this was significantly more useful in the earlier days of \LaTeX{})
or to generate a smaller document which is easier to navigate.
Another application of |\includeonly| is to generate
documents consisting of selected parts of the complete document.

However, there are a few drawbacks of the plain |\include| mechanism:
\begin{itemize}
\item
The child files cannot be compiled on their own,
they can only be compiled via the main file.
A naive editing environment
(such as a text editor with an option
to have the current file processed by \LaTeX)
may require one to switch to the main file before compiling;
attempting to compile the child file produces errors.
\item
The main file must be modified (each time)
to adjust the |\includeonly| command
to the present needs. This easily leaves the main file in a messy state.
\item
The generated document will always carry the filename
of the main document. This is inconvenient if
several child files are to be compiled and
to be kept for distribution.
\end{itemize}

The present package provides a simple interface
to make child files individually compilable by \LaTeX{}.
Compiling a child file then has the same effect as compiling
the main file with an |\includeonly| command
to select the appropriate child.
Moreover the generated document will carry the name of the child
rather than the main file.
This resolves all three above issues.

This feature is meant to make the editing of books,
thesis documents and lecture notes somewhat more convenient.
However, the package can also be used efficiently for
composing a series of documents (such as exercise sheets)
which are typically distributed individually.
It then assists the author in generating the individual documents
(potentially in different versions)
as well as a document containing the collected series.
Another application is in developing style files
or other kinds of included material
where compilation of the style file could redirect
to a sample or test file.

%%%%%%%%%%%%%%%%%%%%%%%%%%%%%%%%%%%%%%%%%%%%%%%%%%%%%%%%%%%%%%%%%%%%%%%%%%%%%%%%
%%%%%%%%%%%%%%%%%%%%%%%%%%%%%%%%%%%%%%%%%%%%%%%%%%%%%%%%%%%%%%%%%%%%%%%%%%%%%%%%
\section{Usage}

First of all, the package \textsf{childdoc} is \emph{not} a standard
\LaTeXe{} |.sty| style file! Therefore it needs to be invoked in
a non-standard way.

%%%%%%%%%%%%%%%%%%%%%%%%%%%%%%%%%%%%%%%%%%%%%%%%%%%%%%%%%%%%%%%%%%%%%%%%%%%%%%%%
\subsection{Included Files}
\label{sec:include}

%%%%%%%%%%%%%%%%%%%%%%%%%%%%%%%%%%%%%%%%
\DescribeMacro{\childdocmain}
To use the package, add the commands
\begin{center}
\begin{tabular}{l}
|\input{childdoc.def}|\\
|\childdocmain{}|\\
\end{tabular}
\end{center}
at the very top of the main \LaTeX{} file,
in particular \emph{before} the |\documentclass| statement!
The argument of |\childdocmain| should be left empty
(but it must be present).

%%%%%%%%%%%%%%%%%%%%%%%%%%%%%%%%%%%%%%%%
\DescribeMacro{\childdocof}
Furthermore, add the commands
\begin{center}
\begin{tabular}{l}
|\input{childdoc.def}|\\
|\childdocof{|\textit{main}|}|\\
\end{tabular}
\end{center}
at the top of every child file \textit{child}
which is included by |\include{|\textit{child}|}|
from within the main file
(or at least for those files to be compiled individually).
The argument \textit{main} must be the filename of the main file.

There are a couple of
considerations in setting up the main and child documents:

%%%%%%%%%%%%%%%%%%%%%%%%%%%%%%%%%%%%%%%%
\paragraph{Restrictions.}

Please note the following restrictions:
\begin{itemize}
\item
|\childdocmain| must be called with one argument \textit{main}
to ensure compatibility with earlier version of the package.
It must either be empty (|\childdocmain{}|)
or precisely match the filename of the main file in which it is specified.
See \secref{sec:detection} for further information.
\item
The filename \textit{main} must be specified without the |.tex| extension.
\item
The filename \textit{main} is case sensitive
(even in case-insensitive file systems)
due to internal string comparison.
\item
The argument \textit{main} should be fully expanded, it cannot be a macro.
\item
Subdirectories and special characters should be avoided in filenames.
\item
The command |\childdocmain{|\textit{main}|}| must be followed by a whitespace.
It should not be followed immediately by another command
or by a comment mark `|%|'.
This is because the \TeX{} parser reads the token immediately following
the argument of |\childdocmain| and puts it
at the beginning of every child section;
however, a white\-space is ignored.
\end{itemize}

%%%%%%%%%%%%%%%%%%%%%%%%%%%%%%%%%%%%%%%%
\paragraph{Content of Main File.}

It is advisable to place all content in the child files included by |\include|.
Any output contained in the main file will appear in all child documents
unless suppressed manually;
it cannot be suppressed automatically by the |\includeonly| directive
and thus should normally be avoided.
A method to include some content in the main file
by means of conditional processing is described in \secref{sec:conditional}.

%%%%%%%%%%%%%%%%%%%%%%%%%%%%%%%%%%%%%%%%
\paragraph{Page Numbering.}

When only a part of the document is compiled,
the appropriate numbering of pages
(as well as other status parameters)
is determined from the |.aux| files.
The latter contain information from previous passes.
However this information needs to propagate through
all intermediate child documents.
Therefore the page numbering in child documents may well
be inconsistent until the complete document is compiled at least once.

A useful (if unconventional) way to always ensure a consistent
page numbering is to restart the numbering in each child document
and denote the pages by `\textit{child}|.|\textit{page}'
where \textit{child} represents the chapter/section number of the child file.
This can be achieved by the command
|\numberwithin{page}{|\textit{child}|}|
of the \textsf{amsmath} package
where \textit{child} can be |chapter| or |section|
depending on the chosen structuring.
Alternatively, one can modify the macro |\thepage| appropriately
and reset the counter |page| at the start of each child file.

%%%%%%%%%%%%%%%%%%%%%%%%%%%%%%%%%%%%%%%%%%%%%%%%%%%%%%%%%%%%%%%%%%%%%%%%%%%%%%%%
\subsection{Conditional Processing}
\label{sec:conditional}

The package provides a mechanism to compile different versions
of a document. To customise the versions further some conditional processing
can come in handy to distinguish which version is being compiled.
The package provides two macros to describe the compilation context:

%%%%%%%%%%%%%%%%%%%%%%%%%%%%%%%%%%%%%%%%
\DescribeMacro{\ifchilddoc}
The conditional |\ifchilddoc| distinguishes between the compilation of
child documents and the main document:
%
\begin{center}
|\ifchilddoc |\textit{child-code}| |[|\||else |\textit{main-code}]| \||fi|
\end{center}

%%%%%%%%%%%%%%%%%%%%%%%%%%%%%%%%%%%%%%%%
\DescribeMacro{\childdocname}
\DescribeMacro{\childdocjob}
The macro |\childdocname| contains the filename (without extension)
of the main or child file being processed.
Note that |\childdocjob| will always contain the name of the main file.

%%%%%%%%%%%%%%%%%%%%%%%%%%%%%%%%%%%%%%%%
\paragraph{Title Page.}

Conditional processing can be used to include a title or banner page
in the main document when proper precautions are taken.
Importantly, the code in the main file should ensure that the page counter
(as well as other status parameters which are stored in the |.aux| files)
takes the same value after the conditional processing.
Otherwise the page numbers may take divergent values
depending on which part is compiled.

For example, a title page could be declared by:
%
\begin{center}
\begin{tabular}{l}
|\ifchilddoc\||else|\\
|\addtocounter{page}{-1}|\\
\textit{code for title page}\\
|\newpage|\\
|\||fi|
\end{tabular}
\end{center}
%
A banner page for the child documents can be generated by:
%
\begin{center}
\begin{tabular}{l}
|\ifchilddoc|\\
|\addtocounter{page}{-1}|\\
\textit{code for banner page}\\
|\newpage|\\
|\||fi|
\end{tabular}
\end{center}
%
Here one could write a message such as:
\begin{center}
|This is the part \childdocname{} of \childdocjob{}.|
\end{center}

%%%%%%%%%%%%%%%%%%%%%%%%%%%%%%%%%%%%%%%%%%%%%%%%%%%%%%%%%%%%%%%%%%%%%%%%%%%%%%%%
\subsection{Flags}
\label{sec:flags}

The package makes it easy to generate different versions
of the main or child documents.
To this end compilation flags can be defined
and assigned different default values.
They will be particularly useful in conjunction
with the forwarding mechanism described in \secref{sec:forward}.

For example, it may be useful to have a flag |\version|
which can be set to |draft| or |final|.
The document source will contain some conditional code
depending on the value of |\version|.
Suppose further, the flag should default to |final| for the main file
and to |draft| for child files
which is a natural assignment for editing the document.
This is achieved by placing the following code
in the preamble of the main document
(below the |\childdocmain| directive):
%
\begin{center}
\begin{tabular}{l}
|\ifchilddoc|\\
|\providecommand{\version}{draft}|\\
|\||else|\\
|\providecommand{\version}{final}|\\
|\||fi|
\end{tabular}
\end{center}
%
The definition by |\providecommand| makes sure
that previous definitions are not overwritten.
Further statements |\providecommand{\version}{...}|
can thus be added before the above code to override it.

For the main file, one might add a line
(between |\childdocmain| and the above block)
%
\begin{center}
|%\ifchilddoc\||else\providecommand{\version}{draft}\||fi|
\end{center}
%
which can be uncommented to produce a draft version.
Likewise one can add a line to the very top of a child file
(above the |\childdocof{|\textit{main}|}| directive)
%
\begin{center}
|%\providecommand{\version}{final}|
\end{center}
%
which can be uncommented to produce the final version of this child document.

%%%%%%%%%%%%%%%%%%%%%%%%%%%%%%%%%%%%%%%%%%%%%%%%%%%%%%%%%%%%%%%%%%%%%%%%%%%%%%%%
\subsection{Forwarding}
\label{sec:forward}

Different versions of the main or child documents
using compilation flags as described in \secref{sec:flags}
can be (permanently) stored in different files
for convenient compilation, viewing and distribution.
To this end, the package defines a command
to pass on compilation to a different file:

%%%%%%%%%%%%%%%%%%%%%%%%%%%%%%%%%%%%%%%%
\DescribeMacro{\childdocforward}
The command |\childdocforward| redirects processing to
another source file:
%
\begin{center}
\begin{tabular}{l}
|\input{childdoc.def}|\\
|\childdocforward[|\textit{main}|]{|\textit{dest}|}|\\
\end{tabular}
\end{center}
%
The argument \textit{dest} is the destination file
(without extension).
It should be the main file or one of the child files.
Note that further \textsf{childdoc} directives
such as |\childdocof| and |\childdocforward|
in the indicated file will be processed in this form.
The optional argument \textit{main}
passes on directly to the main file \textit{main}
while pretending to compile the child \textit{dest}.
This form behaves as if \textit{dest}
issues |\childdocof{|\textit{main}|}| right away,
and no further \textsf{childdoc} directives will be processed.

%%%%%%%%%%%%%%%%%%%%%%%%%%%%%%%%%%%%%%%%
\DescribeMacro{\...prefix}
In the alternative form |\childdocforwardprefix|,
%
\begin{center}
\begin{tabular}{l}
|\input{childdoc.def}|\\
|\childdocforwardprefix[|\textit{main}|]{|\textit{prefix}|}{|\textit{dest}|}|
\end{tabular}
\end{center}
%
the destination file is determined by a pattern
depending on the current file:
To make this work, the current file must be called
`{\textit{prefix}\hspace{0.2em}\textit{suffix}}'
with \textit{prefix} matching precisely the argument.
Processing is then passed on to the file
`{\textit{dest}\hspace{0.2em}\textit{suffix}}'.
Surely, the same effect is achieved by
directly specifying the
argument `{\textit{dest}\hspace{0.2em}\textit{suffix}}'
in the first form.
However, that requires to set up a different file
for each child. With the alternative form of the command
all these files can have exactly the same content
which simplifies setting them up and maintaining them.

For example, the following file |draft.tex|
with a compilation flag |\version| as described in \secref{sec:flags}
compiles the main document as a draft:
%
\begin{center}
\begin{tabular}{l}
|\def\version{draft}|\\
|\input{childdoc.def}|\\
|\childdocforward{|\textit{main}|}|
\end{tabular}
\end{center}
%
Likewise, the following files |final|\textit{nn}|.tex|
compile the final version of the child document
|child|\textit{nn}|.tex|:
%
\begin{center}
\begin{tabular}{l}
|\def\version{final}|\\
|\input{childdoc.def}|\\
|\childdocforwardprefix{final}{child}|
\end{tabular}
\end{center}
%

Note that when several versions of a main file and/or of each child file
are to be generated, it may be convenient to set up a |Makefile| or
shell script to automatise the process.

%%%%%%%%%%%%%%%%%%%%%%%%%%%%%%%%%%%%%%%%%%%%%%%%%%%%%%%%%%%%%%%%%%%%%%%%%%%%%%%%
\subsection{Command Line Processing}
\label{sec:commandline}

The effect of redirection files can also be achieved by invoking
the \LaTeX{} compiler with a more elaborate command line.
Most conveniently this should be done as part
of a shell script or a |Makefile|.

When using \textsf{childdoc} in the main file, the following
command lines effectively perform a redirection
(note that depending on the shell being used,
backslashes may have to be doubled: `|\|' $\to$ `|\\|'):
%
\begin{center}
|... -jobname "|\textit{target}|" |\\|"|[\textit{flags}]%
|\input{childdoc.def}\childdocforward[|\textit{main}|]{|\textit{dest}|}"|
\end{center}
%
Here \textit{target} is the name of the output file,
\textit{main} is the name of the main file
and \textit{dest} is the name of the main or child file to be processed
(all filenames without extensions).
The optional argument \textit{main} can be omitted
if \textit{main} matches \textit{dest}.
Optionally, compilation \textit{flags} can be defined via |\def| commands.
This command line makes the \TeX{} engine believe
it is compiling the file \textit{target}
whose content is specified as the latter parameter.
The provided code then forwards the processing to
\textit{main} or \textit{dest} as described in \secref{sec:forward}.

%%%%%%%%%%%%%%%%%%%%%%%%%%%%%%%%%%%%%%%%%%%%%%%%%%%%%%%%%%%%%%%%%%%%%%%%%%%%%%%%
\subsection{Include by Input}
\label{sec:input}

Including child documents by |\include| has some restrictions by design.
Most notably, the content of a child document always occupies
its own set of pages; pages cannot be shared between child documents.
Usually, this behaviour makes perfect sense
because each child document contain an essential part of the document.
However, in some situations it may be desirable to compose
a document from a collection of parts
without having mandatory page breaks between then.
For this case, the package
provides a mechanism to include parts
by |\input| which can also be processed individually.
However, by construction this mechanism
requires manual handling of the content to be output.

%%%%%%%%%%%%%%%%%%%%%%%%%%%%%%%%%%%%%%%%
\DescribeMacro{\ifchilddocmanual}
The main file should be prepared as usual, see \secref{sec:include}.
However, the document body must make a distinction
between processing of an individual part and of the main document, e.g.:
%
\begin{center}
\begin{tabular}{l}
|\ifchilddocmanual|\\
|\input{\childdocname}|\\
|\||else|\\
\textit{document body with }|\input{|\textit{part}|}|\\
|\||fi|
\end{tabular}
\end{center}
%
The conditional |\ifchilddocmanual| is true whenever
a part to be included by |\input| is being compiled,
and the name of the part is stored in |\childdocname|.

%%%%%%%%%%%%%%%%%%%%%%%%%%%%%%%%%%%%%%%%
\DescribeMacro{\childdocby}
Each part to be included by |\input| should start with:
%
\begin{center}
\begin{tabular}{l}
|\input{childdoc.def}|\\
|\childdocby{|\textit{main}|}|\\
\end{tabular}
\end{center}
%
The directive |\childdocby| is similar to |\childdocof|
described in \secref{sec:include},
but the subsequent selection of content must be done manually.
To that end, both |\ifchilddoc| and |\ifchilddocmanual|
will be true upon processing of a part,
and the name of the part is stored in |\childdocname|.
Note that |\jobname| will be set to the filename of the current part
so that each part receives an individual |.aux| file
that does not interfere with the |.aux| file(s) of the main document.
This behaviour can be altered by the alternative form
|\childdocby[*]{|\textit{main}|}| (with a non-empty optional argument)
which uses the |.aux| file of the main document
by setting |\jobname| to \textit{main}.

%%%%%%%%%%%%%%%%%%%%%%%%%%%%%%%%%%%%%%%%%%%%%%%%%%%%%%%%%%%%%%%%%%%%%%%%%%%%%%%%
\subsection{Driver Development}
\label{sec:driver}

The \textsf{childdoc} mechanism can also be use for the development
of definition files such as \LaTeX{} styles or classes.
This case differs from the above setup with multiple parts
included by |\include| in that no |\includeonly| should be invoked.
This can be achieved by starting the include file
(before |\ProvidesPackage|) with:
%
\begin{center}
\begin{tabular}{l}
|\input{childdoc.def}|\\
|\childdocforward{|\textit{main}|}|\\
\end{tabular}
\end{center}
%
or alternatively with:
%
\begin{center}
\begin{tabular}{l}
|\input{childdoc.def}|\\
|\childdocby{|\textit{main}|}|\\
\end{tabular}
\end{center}
%
Both forms have slightly different effects as described above.
The main file is prepared as usual, see \secref{sec:include}.

%%%%%%%%%%%%%%%%%%%%%%%%%%%%%%%%%%%%%%%%%%%%%%%%%%%%%%%%%%%%%%%%%%%%%%%%%%%%%%%%
\subsection{Legacy Detection}
\label{sec:detection}

The directive |\childdocmain| in the main file can detect
whether the complete document or merely a child is to be compiled
even without using the directive |\childdocof|.
This method is deprecated because it is less robust
and there is no compelling reason to use it;
it is merely provided for backward compatibility
and it may be removed in future versions.

If the detection mechanism is to be used,
it is mandatory to correctly specify
the filename of the main file as the argument of |\childdocmain|:
%
\begin{center}
\begin{tabular}{l}
|\input{childdoc.def}|\\
|\childdocmain{|\textit{main}|}|\\
\end{tabular}
\end{center}
%
If |\jobname| does not match the argument \textit{main} of |\childdocmain|,
it is assumed that |\jobname| points to the child file to be compiled.
When using |\childdocmain| with the main file specified as argument,
it suffices to start a child file
with just |\input{|\textit{main}|}|
without loading of the package and using |\childdocof|.
If instead all processing is done
with the appropriate \textsf{childdoc} directives,
the argument of \textit{main} of |\childdocmain| can be empty.

An alternative version of the command line processing described
in \secref{sec:commandline} using the detection mechanism reads:
%
\begin{center}
|... -jobname "|\textit{target}|" "|[\textit{flags}]%
[|\def\jobname{|\textit{dest}|}|]|\input{|\textit{main}|}"|
\end{center}

%%%%%%%%%%%%%%%%%%%%%%%%%%%%%%%%%%%%%%%%%%%%%%%%%%%%%%%%%%%%%%%%%%%%%%%%%%%%%%%%
\subsection{Manual Code}
\label{sec:manual}

In case one cannot be certain whether the definitions file |childdoc.def|
is installed on the target \TeX{} distribution
and one prefers not to ship it,
it is conceivable to paste a few relevant commands into the sources.

To that end, drop all statements |\input{childdoc.def}|
and perform the replacements as outlined below.
Instead of |\childdocmain{|\textit{main}|}| add the following code
to the top of the main file:
%
\begin{center}
\begin{tabular}{l}
|\||ifdefined\childdocname\endinput\||fi\newif\ifchilddoc|\\
|\edef\childdocname{\scantokens\expandafter{\jobname\noexpand}}|\\
|\def\childdocmain{|\textit{main}|}\||ifx\childdocmain\childdocname\||else|\\
|\childdoctrue\includeonly{\childdocname}\let\jobname\childdocmain\||fi|\\
\end{tabular}
\end{center}
%
Instead of |\childdocof{|\textit{main}|}| just include the main file
at the top of each child file:
%
\begin{center}
|\input{|\textit{main}|}|
\end{center}
%
A simple redirection |\childdocforward{|\textit{dest}|}| is achieved by:
%
\begin{center}
|\def\jobname{|\textit{dest}|}\input{\jobname}|
\end{center}
%
The redirection with prefix
|\childdocforwardprefix[|\textit{prefix}|]{|\textit{dest}|}|
is accomplished by:
%
\begin{center}
\begin{tabular}{l}
|{\edef\jobname{\scantokens\expandafter{\jobname\noexpand}}|\\
|\def\redirectjob |\textit{prefix}|#1~~~{\gdef\jobname{|\textit{dest}|#1}}|\\
|\expandafter\redirectjob\jobname~~~}\input{\jobname}|
\end{tabular}
\end{center}

In an alternative approach,
child documents can be compiled by a specific command line
without additional code or specific definitions:
%
\begin{center}
|... -jobname "|\textit{target}|" "|[\textit{flags}]%
|\includeonly{|\textit{dest}|}\input{|\textit{main}|}"|
\end{center}
%

%%%%%%%%%%%%%%%%%%%%%%%%%%%%%%%%%%%%%%%%%%%%%%%%%%%%%%%%%%%%%%%%%%%%%%%%%%%%%%%%
%%%%%%%%%%%%%%%%%%%%%%%%%%%%%%%%%%%%%%%%%%%%%%%%%%%%%%%%%%%%%%%%%%%%%%%%%%%%%%%%
\section{Information}

%%%%%%%%%%%%%%%%%%%%%%%%%%%%%%%%%%%%%%%%%%%%%%%%%%%%%%%%%%%%%%%%%%%%%%%%%%%%%%%%
\subsection{Copyright}

Copyright \copyright{} 2017--2018 Niklas Beisert

This work may be distributed and/or modified under the
conditions of the \LaTeX{} Project Public License, either version 1.3
of this license or (at your option) any later version.
The latest version of this license is in
  \url{http://www.latex-project.org/lppl.txt}
and version 1.3 or later is part of all distributions of \LaTeX{}
version 2005/12/01 or later.

This work has the LPPL maintenance status `maintained'.

The Current Maintainer of this work is Niklas Beisert.

This work consists of the files |README.txt|, |childdoc.ins| and |childdoc.dtx|
as well as the derived files |childdoc.def|, |cdocsamp.tex|
with |cdocsch1.tex|, |cdocsch2.tex|, |cdocspt3.tex|, |cdocspt4.tex|,
|cdocsdrf.tex|, |cdocsfn1.tex|, |cdocsfn2.tex|
as well as |childdoc.pdf|.

%%%%%%%%%%%%%%%%%%%%%%%%%%%%%%%%%%%%%%%%%%%%%%%%%%%%%%%%%%%%%%%%%%%%%%%%%%%%%%%%
\subsection{Files and Installation}

The package consists of the files:
%
\begin{center}
\begin{tabular}{ll}
    |README.txt|   & readme file \\
    |childdoc.ins| & installation file \\
    |childdoc.dtx| & source file \\
    |childdoc.def| & definition file \\
    |cdocsamp.tex| & sample main file \\
    |cdocsch1.tex| & sample include file \\
    |cdocsch2.tex| & sample include file \\
    |cdocspt3.tex| & sample part file \\
    |cdocspt4.tex| & sample part file \\
    |cdocsdrf.tex| & sample redirection file \\
    |cdocsfn1.tex| & sample redirection file \\
    |cdocsfn2.tex| & sample redirection file \\
    |childdoc.pdf| & manual
\end{tabular}
\end{center}
%
The distribution consists of the files
|README.txt|, |childdoc.ins| and |childdoc.dtx|.
%
\begin{itemize}
\item
Run (pdf)\LaTeX{} on |childdoc.dtx|
to compile the manual |childdoc.pdf| (this file).
\item
Run \LaTeX{} on |childdoc.ins| to create the definitions file |childdoc.def|
and the sample |cdocsamp.tex| with include files
|cdocsch1.tex|, |cdocsch2.tex|, |cdocspt3.tex|, |cdocspt4.tex|,
|cdocsdrf.tex|, |cdocsfn1.tex|, |cdocsfn2.tex|.
Then copy the file |childdoc.def| to an appropriate directory of your \LaTeX{}
distribution, e.g.\ \textit{texmf-root}|/tex/latex/childdoc|.
\end{itemize}

%%%%%%%%%%%%%%%%%%%%%%%%%%%%%%%%%%%%%%%%%%%%%%%%%%%%%%%%%%%%%%%%%%%%%%%%%%%%%%%%
\subsection{Related CTAN Packages}

There are several other packages which offer a similar functionality:
%
\begin{itemize}
\item
The packages
\href{http://ctan.org/pkg/docmute}{\textsf{docmute}},
\href{http://ctan.org/pkg/includex}{\textsf{includex}} and
\href{http://ctan.org/pkg/standalone}{\textsf{standalone}}
provide commands to include only the document body of
a child file thus allowing both files to be compiled individually.
\item
The packages \href{http://ctan.org/pkg/subdocs}{\textsf{subdocs}}
and \href{http://ctan.org/pkg/subfiles}{\textsf{subfiles}}
provide structures in which the main and child documents can be
encapsulated and allowing them to be compiled individually.
The inclusion mechanism is different from the conventional |\include|.
\item
The package \href{http://ctan.org/pkg/combine}{\textsf{combine}}
is an elaborate solution to combine several documents into one.
\end{itemize}
%
See also the CTAN topic \href{http://ctan.org/topic/subdocs}{\textsf{subdocs}}
for further related packages.
The present package differs from the above solutions in that
a document structure constructed with the conventional |\include| mechanism
just needs two extra commands at the top of every file
such that all constituent files can be compiled individually.

%%%%%%%%%%%%%%%%%%%%%%%%%%%%%%%%%%%%%%%%%%%%%%%%%%%%%%%%%%%%%%%%%%%%%%%%%%%%%%%%
%\subsection{Feature Suggestions}
%
%The following is a list of features which may be useful for future
%versions of this package:
%%
%\begin{itemize}
%\item
%\ldots
%\end{itemize}

%%%%%%%%%%%%%%%%%%%%%%%%%%%%%%%%%%%%%%%%%%%%%%%%%%%%%%%%%%%%%%%%%%%%%%%%%%%%%%%%
\subsection{Revision History}

%%%%%%%%%%%%%%%%%%%%%%%%%%%%%%%%%%%%%%%%
\paragraph{v2.0:} 2018/12/30

\begin{itemize}
\item
immediate forward processing
\item
added |\childdocby| mechanism
\item
manual restructured
\end{itemize}

%%%%%%%%%%%%%%%%%%%%%%%%%%%%%%%%%%%%%%%%
\paragraph{v1.6:} 2018/01/17

\begin{itemize}
\item
application for development of include files
\item
corrections to manual
\end{itemize}

%%%%%%%%%%%%%%%%%%%%%%%%%%%%%%%%%%%%%%%%
\paragraph{v1.5:} 2017/05/21

\begin{itemize}
\item
more complete structuring introduced
\item
|\childdocof| introduced
\item
|\childdoc| renamed to |\childdocmain|
\item
|\childredirect| renamed to |\childdocforward| and |\childdocforwardprefix|
and functionality expanded
\end{itemize}

%%%%%%%%%%%%%%%%%%%%%%%%%%%%%%%%%%%%%%%%
\paragraph{v1.0:} 2017/04/27

\begin{itemize}
\item
manual and install package
\item
first version published on CTAN
\end{itemize}

%%%%%%%%%%%%%%%%%%%%%%%%%%%%%%%%%%%%%%%%
\paragraph{v0.6:} 2017/04/26

\begin{itemize}
\item
redirection mechanism added
\end{itemize}

%%%%%%%%%%%%%%%%%%%%%%%%%%%%%%%%%%%%%%%%
\paragraph{v0.5:} 2017/04/26

\begin{itemize}
\item
functionality in definition file
\end{itemize}


%%%%%%%%%%%%%%%%%%%%%%%%%%%%%%%%%%%%%%%%%%%%%%%%%%%%%%%%%%%%%%%%%%%%%%%%%%%%%%%%
%%%%%%%%%%%%%%%%%%%%%%%%%%%%%%%%%%%%%%%%%%%%%%%%%%%%%%%%%%%%%%%%%%%%%%%%%%%%%%%%
%%%%%%%%%%%%%%%%%%%%%%%%%%%%%%%%%%%%%%%%%%%%%%%%%%%%%%%%%%%%%%%%%%%%%%%%%%%%%%%%
\appendix

\settowidth\MacroIndent{\rmfamily\scriptsize 000\ }

 \DocInput{childdoc.dtx}

\end{document}
%</driver>
% \fi
%
% %%%%%%%%%%%%%%%%%%%%%%%%%%%%%%%%%%%%%%%%%%%%%%%%%%%%%%%%%%%%%%%%%%%%%%%%%%%%%%
% %%%%%%%%%%%%%%%%%%%%%%%%%%%%%%%%%%%%%%%%%%%%%%%%%%%%%%%%%%%%%%%%%%%%%%%%%%%%%%
% \section{Sample}
%\iffalse
%<*samplemain>
%\fi
%
% The following presents a sample document
% with two chapters, two parts, a title page,
% a compile flag as well as three forwarding files to set the flag.
% It consists of eight |.tex| files:
% \begin{center}
% \begin{tabular}{ll}
% |cdocsamp.tex|&main file\\
% |cdocsch1.tex|&include file for chapter 1\\
% |cdocsch2.tex|&include file for chapter 2\\
% |cdocspt3.tex|&include file for part 3\\
% |cdocspt4.tex|&include file for part 4\\
% |cdocsdrf.tex|&forwarding file for main file in draft mode\\
% |cdocsfi1.tex|&forwarding file for final version of chapter 1\\
% |cdocsfi2.tex|&forwarding file for final version of chapter 2\\
% \end{tabular}
% \end{center}
% Each of the eight files can be compiled directly by the \LaTeX{} compiler.
%
% %%%%%%%%%%%%%%%%%%%%%%%%%%%%%%%%%%%%%%
% \paragraph{Main File.}
%
% The main file is called |cdocsamp.tex|.
%
% Load the \textsf{childdoc} definitions and
% declare the filename for the main document:
%    \begin{macrocode}
\input{childdoc.def}
\childdocmain{}
%    \end{macrocode}

% Optional override for |\version| flag:
%    \begin{macrocode}
%%\ifchilddoc\else\providecommand{\version}{draft}\fi
%    \end{macrocode}

% Define the default values for the |\version| flag
% (|final| for the main file and |draft| for childs):
%    \begin{macrocode}
\ifchilddoc
\providecommand{\version}{draft}
\else
\providecommand{\version}{final}
\fi
%    \end{macrocode}

% Load the standard document class:
%    \begin{macrocode}
\documentclass[12pt]{article}
%    \end{macrocode}

% Start the document body:
%    \begin{macrocode}
\begin{document}
%    \end{macrocode}

% Declare a title page.
% Print title, part of document being processed and version flag:
%    \begin{macrocode}
\addtocounter{page}{-1}
\begin{center}
{\LARGE\bfseries{}childdoc example\par}
\vspace{1cm}
\ifchilddoc
\ifchilddocmanual part\else chapter\fi:
`\childdocname' of `\childdocjob'\par
\else
main document: `\childdocjob'\par
\fi
version: \version\par
\end{center}
\newpage
%    \end{macrocode}

% Manually include selected file,
% otherwise process as usual:
%    \begin{macrocode}
\ifchilddocmanual
\section*{part `\childdocname'}
\input{\childdocname}
\else
%    \end{macrocode}

% Include the two chapters:
%    \begin{macrocode}
\include{cdocsch1}
\include{cdocsch2}
%    \end{macrocode}

% Include the two parts unless only chapters should be displayed:
%    \begin{macrocode}
\ifchilddoc\else
\section{part three}
\input{cdocspt3}
\section{part four}
\input{cdocspt4}
\fi
%    \end{macrocode}

% Process as usual until here:
%    \begin{macrocode}
\fi
%    \end{macrocode}

% End of document body:
%    \begin{macrocode}
\end{document}
%    \end{macrocode}
%\iffalse
%</samplemain>
%\fi
%
% %%%%%%%%%%%%%%%%%%%%%%%%%%%%%%%%%%%%%%
% \paragraph{Chapter Include Files.}
%
% The include files are called |cdocsch1.tex| and |cdocsch2.tex|.
%
%\iffalse
%<*samplechap1|samplechap2>
%\fi

% Optional override for |\version| flag:
%    \begin{macrocode}
%%\providecommand{\version}{final}
%    \end{macrocode}

% Include the main document:
%    \begin{macrocode}
\input{childdoc.def}
\childdocof{cdocsamp}
%    \end{macrocode}

%\iffalse
%</samplechap1|samplechap2>
%\fi
%
%\iffalse
%<*samplechap1>
%\fi
% Some text for chapter 1:
%    \begin{macrocode}
\section{one}
some text in chapter one
%    \end{macrocode}

%\iffalse
%</samplechap1>
%\fi
% Some text for chapter 2:
%\iffalse
%<*samplechap2>
%\fi
%    \begin{macrocode}
\section{two}
more text in chapter two
%    \end{macrocode}

%\iffalse
%</samplechap2>
%\fi
%
% %%%%%%%%%%%%%%%%%%%%%%%%%%%%%%%%%%%%%%
% \paragraph{Part Include Files.}
%
% The include files are called |cdocspt3.tex| and |cdocspt4.tex|.
%
%\iffalse
%<*samplepart3|samplepart4>
%\fi

% Optional override for |\version| flag:
%    \begin{macrocode}
%%\providecommand{\version}{final}
%    \end{macrocode}

% Include the main document:
%    \begin{macrocode}
\input{childdoc.def}
\childdocby{cdocsamp}
%    \end{macrocode}

%\iffalse
%</samplepart3|samplepart4>
%\fi
%
%\iffalse
%<*samplepart3>
%\fi
% Some text for part 3:
%    \begin{macrocode}
some text in part three
%    \end{macrocode}

%\iffalse
%</samplepart3>
%\fi
% Some text for part 4:
%\iffalse
%<*samplepart4>
%\fi
%    \begin{macrocode}
more text in part four
%    \end{macrocode}

%\iffalse
%</samplepart4>
%\fi
%
% %%%%%%%%%%%%%%%%%%%%%%%%%%%%%%%%%%%%%%
% \paragraph{Forwarding for a Complete Draft.}
%
% The following forwarding file |cdocsdrf.tex|
% compiles the main document in draft mode:
%\iffalse
%<*sampledraft>
%\fi
%    \begin{macrocode}
\def\version{draft}
\input{childdoc.def}
\childdocforward{cdocsamp}
%    \end{macrocode}

%\iffalse
%</sampledraft>
%\fi
%
% %%%%%%%%%%%%%%%%%%%%%%%%%%%%%%%%%%%%%%
% \paragraph{Forwarding for Final Version of the Chapters.}
%
% The following forwarding files |cdocsfn1.tex| and |cdocsfn2.tex|
% (with identical content)
% compile the final versions of the child documents
% |cdocsch1.tex| and |cdocsch2.tex|, respectively:
%\iffalse
%<*samplefinal>
%\fi
%    \begin{macrocode}
\def\version{final}
\input{childdoc.def}
\childdocforwardprefix[cdocsamp]{cdocsfn}{cdocsch}
%    \end{macrocode}

%\iffalse
%</samplefinal>
%\fi
%
% %%%%%%%%%%%%%%%%%%%%%%%%%%%%%%%%%%%%%%
% \paragraph{Command Line Processing.}
%
% The following three command lines generate the output files
% |cdocscld|, |cdocscl1| and |cdocscl2|
% which should be identical to
% |cdocsdrf|, |cdocsch1| and |cdocsfn2|, respectively:
% \begin{center}
% \begin{tabular}{l}
% |latex -jobname cdocscld \|\\
% |  "\def\version{draft}\input{childdoc.def}\childdocforward{cdocsamp}"|\\
% |latex -jobname cdocscl1 \|\\
% |  "\input{childdoc.def}\childdocforward[cdocsamp]{cdocsch1}"|\\
% |latex -jobname cdocscl2 \|\\
% |  "\def\version{final}\input{childdoc.def}\childdocforward{cdocsch2}"|
% \end{tabular}
% \end{center}
% Note that the trailing backslash on each first line
% merely continues the input to the second line
% (for convenient cut ant paste).
% Furthermore, the command |latex| can be replaced by any
% of its alternative versions such as |pdflatex|.
%
% %%%%%%%%%%%%%%%%%%%%%%%%%%%%%%%%%%%%%%%%%%%%%%%%%%%%%%%%%%%%%%%%%%%%%%%%%%%%%%
% %%%%%%%%%%%%%%%%%%%%%%%%%%%%%%%%%%%%%%%%%%%%%%%%%%%%%%%%%%%%%%%%%%%%%%%%%%%%%%
% \section{Implementation}
%\iffalse
%<*package>
%\fi
%
% This section describes the definitions file |childdoc.def|.

% The definitions cannot be loaded using |\usepackage| or |\RequirePackage|
% which has a mechanism to prevent loading a style file more than once.
% When loading the definitions by means of |\input|
% multiple instances have to be prevented manually:
%\iffalse
%This code needs to be before the `\ProvidesFile' directive
%which is defined at the beginning of this file.
%Therefore it is also placed there and commented out here.
%</package>
%<*discard>
%\fi
%    \begin{macrocode}
\ifdefined\childdocmain\endinput\fi
%    \end{macrocode}
%\iffalse
%</discard>
%<*package>
%\fi
%
% \macro{\ifchilddoc}
% \macro{\ifchilddocmanual}
% The conditional |\ifchilddoc| tells whether a
% child (true) or main (false) document is being compiled.
% The conditional |\ifchilddocmanual| tells whether
% the |\includeonly| mechanism is used (false) or
% the selection of child files must be performed manually (true).
% The definitions initialise to false:
%    \begin{macrocode}
\newif\ifchilddoc
\newif\ifchilddocmanual
%    \end{macrocode}

% \macro{\childdocname}
% \macro{\childdocjob}
% The macro |\childdocname| stores the name of the main document
% to be compiled. The macro |\childdocjob| stores the name of
% the document on which the \LaTeX{} compiler was originally invoked.
% The content of |\jobname| cannot be compared
% to filenames specified in the source due to different catcodes.
% The following code rescans |\jobname|, stores the result
% in |\childdocname| and saves a copy in |\childdocjob|:
%    \begin{macrocode}
\edef\childdocname{\scantokens\expandafter{\jobname\noexpand}}
\let\childdocjob\childdocname
%    \end{macrocode}

% \macro{\childdocdisable}
% The macro |\childdocdisable| prevents the main file
% from being processed more than once.
% At this stage, the main document command |\childdocmain|
% is assumed to be called once again where it should do nothing.
% Any subsequent call to it should prevent
% a secondary processing of the main document
% It overwrites the forwarding commands
% |\childdocof| and |\childdocforward|
% with empty macros to prevent further inclusions of the main document:
%    \begin{macrocode}
\newcommand{\childdocdisable}
{
  \renewcommand{\childdocmain}[1]{\renewcommand{\childdocmain}[1]{\endinput}}
  \renewcommand{\childdocof}[1]{}
  \renewcommand{\childdocby}[2][]{}
  \renewcommand{\childdocforward}[2][]{}
  \renewcommand{\childdocdisable}{}
}
%    \end{macrocode}

% \macro{\childdocmain}
% The macro |\childdocmain| is to be called at the top of the main file
% with nothing or the main filename (without extension) as argument.
% First, it breaks loops.
% If the argument is not empty and does not match |\childdocname|
% (which is set by the first inclusion of |childdoc.def|),
% |\ifchilddoc| is set to true, |\includeonly| is applied to the child file
% and |\jobname| is set to the main file
% (for proper handling of |.aux| files):
%    \begin{macrocode}
\newcommand{\childdocmain}[1]
{
  \childdocdisable\childdocmain{}
  \if?#1?\else
    \begingroup
      \def\childdoctmp{#1}
      \ifx\childdoctmp\childdocname
        \def\childdoctmp{}
      \else
        \def\childdoctmp
        {
          \childdoctrue
          \includeonly{\childdocname}
          \def\childdocjob{#1}
          \def\jobname{#1}
        }
      \fi
      \expandafter
    \endgroup
    \childdoctmp
  \fi
}
%    \end{macrocode}

% \macro{\childdocof}
% The command |\childdocof| redirects
% compilation to the main file |#1|.
%    \begin{macrocode}
\newcommand{\childdocof}[1]
{
  \childdocdisable
  \childdoctrue
  \includeonly{\childdocname}
  \def\jobname{#1}
  \def\childdocjob{#1}
  \input{#1}
}
%    \end{macrocode}

% \macro{\childdocby}
% The command |\childdocby| ....
%    \begin{macrocode}
\newcommand{\childdocby}[2][]
{
  \childdocdisable
  \childdoctrue
  \childdocmanualtrue
  \if?#1?\else
    \def\jobname{#2}
  \fi
  \def\childdocjob{#2}
  \input{#2}
  \endinput
}
%    \end{macrocode}

% \macro{\childdocforward}
% The command |\childdocforward| redirects
% compilation to the main file or
% (if the optional argument is given) a child file.
% Parameters are set as if the main file
% or a child file starting with |\childdocof| was compiled.
% Then compilation is handed over to the main file:
%    \begin{macrocode}
\newcommand{\childdocforward}[2][]
{
  \begingroup
    \if?#1?
      \def\childdoctmp
      {
        \def\childdocname{#2}
        \def\childdocjob{#2}
        \def\jobname{#2}
        \input{#2}
        \endinput
      }
    \else
      \def\childdoctmp
      {
        \childdocdisable
        \def\childdocname{#2}
        \childdoctrue
        \includeonly{#2}
        \def\childdocjob{#1}
        \def\jobname{#1}
        \input{#1}
        \endinput
      }
    \fi
    \expandafter
  \endgroup
  \childdoctmp
}
%    \end{macrocode}

% \macro{\childdocforwardprefix}
% The command |\childdocforwardprefix| redirects
% compilation to the main or a child file by means of a pattern.
% The prefix |#1| in the current filename is replaced by |#2|
% and the suffix of the current filename is kept
% (it is assumed that the filename does not contain the substring `|~~~|'
% which is used as a delimiter).
% Compilation is handed over to the new file by |\childdocforward|:
%    \begin{macrocode}
\newcommand{\childdocforwardprefix}[3][]
{
  \begingroup
    \def\childdocextract #2##1~~~{\def\childdoctmp{\childdocforward[#1]{#3##1}}}
    \expandafter\childdocextract\childdocname~~~
    \expandafter
  \endgroup
  \childdoctmp
}
%    \end{macrocode}

% \macro{\childdoc}
% The deprecated macro |\childdoc| is a legacy version of |\childdocmain|:
%    \begin{macrocode}
\newcommand{\childdoc}{\childdocmain}
%    \end{macrocode}

% \macro{\childdocredirect}
% The deprecated macro |\childdocredirect| is a legacy version
% of |\childdocforward| and |\childdocforwardprefix|:
%    \begin{macrocode}
\newcommand{\childdocredirect}[2][]
{
  \begingroup
    \if?#1?
      \def\childdoctmp{\childdocforward{#2}}
    \else
      \def\childdoctmp{\childdocforwardprefix{#1}{#2}}
    \fi
    \expandafter
  \endgroup
  \childdoctmp
}
%    \end{macrocode}

%\iffalse
%</package>
%\fi
%
\endinput
|\\
|\childdocmain{}|\\
\end{tabular}
\end{center}
at the very top of the main \LaTeX{} file,
in particular \emph{before} the |\documentclass| statement!
The argument of |\childdocmain| should be left empty
(but it must be present).

%%%%%%%%%%%%%%%%%%%%%%%%%%%%%%%%%%%%%%%%
\DescribeMacro{\childdocof}
Furthermore, add the commands
\begin{center}
\begin{tabular}{l}
|% \iffalse
%
% childdoc.dtx Copyright (C) 2017-2018 Niklas Beisert
%
% This work may be distributed and/or modified under the
% conditions of the LaTeX Project Public License, either version 1.3
% of this license or (at your option) any later version.
% The latest version of this license is in
%   http://www.latex-project.org/lppl.txt
% and version 1.3 or later is part of all distributions of LaTeX
% version 2005/12/01 or later.
%
% This work has the LPPL maintenance status `maintained'.
%
% The Current Maintainer of this work is Niklas Beisert.
%
% This work consists of the files childdoc.dtx and childdoc.ins
% and the derived files childdoc.def and cdocsamp.tex with
% cdocsch1.tex, cdocsch2.tex, cdocsdrf.tex, cdocsfn1.tex, cdocsfn2.tex.
%
%<package>\ifdefined\childdocmain\endinput\fi
%<package>\ProvidesFile{childdoc.def}[2018/12/30 v2.0 child document driver]
%<samplemain>\ProvidesFile{cdocsamp.tex}[2018/12/30 v2.0 sample for childdoc]
%<*driver>
%\ProvidesFile{childdoc.drv}[2018/12/30 v2.0 childdoc reference manual file]
\PassOptionsToClass{10pt,a4paper}{article}
\documentclass{ltxdoc}

\usepackage[margin=35mm]{geometry}
\usepackage{hyperref}
\usepackage{hyperxmp}
\usepackage[usenames]{color}

\hypersetup{colorlinks=true}
\hypersetup{pdfstartview=FitH}
\hypersetup{pdfpagemode=UseNone}
\hypersetup{pdfsource={}}
\hypersetup{pdflang={en-UK}}
\hypersetup{pdfcopyright={Copyright 2017-2018 Niklas Beisert.
  This work may be distributed and/or modified under the
  conditions of the LaTeX Project Public License, either version 1.3
  of this license or (at your option) any later version.}}
\hypersetup{pdflicenseurl={http://www.latex-project.org/lppl.txt}}
\hypersetup{pdfcontactaddress={ETH Zurich, ITP, HIT K,
  Wolfgang-Pauli-Strasse 27}}
\hypersetup{pdfcontactpostcode={8093}}
\hypersetup{pdfcontactcity={Zurich}}
\hypersetup{pdfcontactcountry={Switzerland}}
\hypersetup{pdfcontactemail={nbeisert@itp.phys.ethz.ch}}
\hypersetup{pdfcontacturl={http://people.phys.ethz.ch/\xmptilde nbeisert/}}

\newcommand{\secref}[1]{\hyperref[#1]{section \ref*{#1}}}

\parskip1ex
\parindent0pt
\let\olditemize\itemize
\def\itemize{\olditemize\parskip0pt}

\begin{document}

\title{The \textsf{childdoc} Package}
\hypersetup{pdftitle={The childdoc Package}}
\author{Niklas Beisert\\[2ex]
  Institut f\"ur Theoretische Physik\\
  Eidgen\"ossische Technische Hochschule Z\"urich\\
  Wolfgang-Pauli-Strasse 27, 8093 Z\"urich, Switzerland\\[1ex]
  \href{mailto:nbeisert@itp.phys.ethz.ch}
  {\texttt{nbeisert@itp.phys.ethz.ch}}}
\hypersetup{pdfauthor={Niklas Beisert}}
\hypersetup{pdfsubject={Manual for the LaTeX2e Package childdoc}}
\date{30 December 2018, \textsf{v2.0}}
\maketitle

\begin{abstract}\noindent
\textsf{childdoc} is a \LaTeXe{} package
that enables the direct compilation
of document sections included by |\include|
to individual files.
\end{abstract}

\begingroup
\parskip0ex
\tableofcontents
\endgroup

%%%%%%%%%%%%%%%%%%%%%%%%%%%%%%%%%%%%%%%%%%%%%%%%%%%%%%%%%%%%%%%%%%%%%%%%%%%%%%%%
%%%%%%%%%%%%%%%%%%%%%%%%%%%%%%%%%%%%%%%%%%%%%%%%%%%%%%%%%%%%%%%%%%%%%%%%%%%%%%%%
\section{Introduction}

\LaTeX{} provides a mechanism to structure a large document (such as a book)
into a main file and several child files (containing the chapters)
using the |\include| command.
This mechanism is beneficial for documents
which span hundreds of pages in order to
make the source file(s) more manageable.
Moreover, compilation can be restricted to
selected child files by means of the |\includeonly| command.
The latter feature can be used to reduce the compilation time while editing
(this was significantly more useful in the earlier days of \LaTeX{})
or to generate a smaller document which is easier to navigate.
Another application of |\includeonly| is to generate
documents consisting of selected parts of the complete document.

However, there are a few drawbacks of the plain |\include| mechanism:
\begin{itemize}
\item
The child files cannot be compiled on their own,
they can only be compiled via the main file.
A naive editing environment
(such as a text editor with an option
to have the current file processed by \LaTeX)
may require one to switch to the main file before compiling;
attempting to compile the child file produces errors.
\item
The main file must be modified (each time)
to adjust the |\includeonly| command
to the present needs. This easily leaves the main file in a messy state.
\item
The generated document will always carry the filename
of the main document. This is inconvenient if
several child files are to be compiled and
to be kept for distribution.
\end{itemize}

The present package provides a simple interface
to make child files individually compilable by \LaTeX{}.
Compiling a child file then has the same effect as compiling
the main file with an |\includeonly| command
to select the appropriate child.
Moreover the generated document will carry the name of the child
rather than the main file.
This resolves all three above issues.

This feature is meant to make the editing of books,
thesis documents and lecture notes somewhat more convenient.
However, the package can also be used efficiently for
composing a series of documents (such as exercise sheets)
which are typically distributed individually.
It then assists the author in generating the individual documents
(potentially in different versions)
as well as a document containing the collected series.
Another application is in developing style files
or other kinds of included material
where compilation of the style file could redirect
to a sample or test file.

%%%%%%%%%%%%%%%%%%%%%%%%%%%%%%%%%%%%%%%%%%%%%%%%%%%%%%%%%%%%%%%%%%%%%%%%%%%%%%%%
%%%%%%%%%%%%%%%%%%%%%%%%%%%%%%%%%%%%%%%%%%%%%%%%%%%%%%%%%%%%%%%%%%%%%%%%%%%%%%%%
\section{Usage}

First of all, the package \textsf{childdoc} is \emph{not} a standard
\LaTeXe{} |.sty| style file! Therefore it needs to be invoked in
a non-standard way.

%%%%%%%%%%%%%%%%%%%%%%%%%%%%%%%%%%%%%%%%%%%%%%%%%%%%%%%%%%%%%%%%%%%%%%%%%%%%%%%%
\subsection{Included Files}
\label{sec:include}

%%%%%%%%%%%%%%%%%%%%%%%%%%%%%%%%%%%%%%%%
\DescribeMacro{\childdocmain}
To use the package, add the commands
\begin{center}
\begin{tabular}{l}
|\input{childdoc.def}|\\
|\childdocmain{}|\\
\end{tabular}
\end{center}
at the very top of the main \LaTeX{} file,
in particular \emph{before} the |\documentclass| statement!
The argument of |\childdocmain| should be left empty
(but it must be present).

%%%%%%%%%%%%%%%%%%%%%%%%%%%%%%%%%%%%%%%%
\DescribeMacro{\childdocof}
Furthermore, add the commands
\begin{center}
\begin{tabular}{l}
|\input{childdoc.def}|\\
|\childdocof{|\textit{main}|}|\\
\end{tabular}
\end{center}
at the top of every child file \textit{child}
which is included by |\include{|\textit{child}|}|
from within the main file
(or at least for those files to be compiled individually).
The argument \textit{main} must be the filename of the main file.

There are a couple of
considerations in setting up the main and child documents:

%%%%%%%%%%%%%%%%%%%%%%%%%%%%%%%%%%%%%%%%
\paragraph{Restrictions.}

Please note the following restrictions:
\begin{itemize}
\item
|\childdocmain| must be called with one argument \textit{main}
to ensure compatibility with earlier version of the package.
It must either be empty (|\childdocmain{}|)
or precisely match the filename of the main file in which it is specified.
See \secref{sec:detection} for further information.
\item
The filename \textit{main} must be specified without the |.tex| extension.
\item
The filename \textit{main} is case sensitive
(even in case-insensitive file systems)
due to internal string comparison.
\item
The argument \textit{main} should be fully expanded, it cannot be a macro.
\item
Subdirectories and special characters should be avoided in filenames.
\item
The command |\childdocmain{|\textit{main}|}| must be followed by a whitespace.
It should not be followed immediately by another command
or by a comment mark `|%|'.
This is because the \TeX{} parser reads the token immediately following
the argument of |\childdocmain| and puts it
at the beginning of every child section;
however, a white\-space is ignored.
\end{itemize}

%%%%%%%%%%%%%%%%%%%%%%%%%%%%%%%%%%%%%%%%
\paragraph{Content of Main File.}

It is advisable to place all content in the child files included by |\include|.
Any output contained in the main file will appear in all child documents
unless suppressed manually;
it cannot be suppressed automatically by the |\includeonly| directive
and thus should normally be avoided.
A method to include some content in the main file
by means of conditional processing is described in \secref{sec:conditional}.

%%%%%%%%%%%%%%%%%%%%%%%%%%%%%%%%%%%%%%%%
\paragraph{Page Numbering.}

When only a part of the document is compiled,
the appropriate numbering of pages
(as well as other status parameters)
is determined from the |.aux| files.
The latter contain information from previous passes.
However this information needs to propagate through
all intermediate child documents.
Therefore the page numbering in child documents may well
be inconsistent until the complete document is compiled at least once.

A useful (if unconventional) way to always ensure a consistent
page numbering is to restart the numbering in each child document
and denote the pages by `\textit{child}|.|\textit{page}'
where \textit{child} represents the chapter/section number of the child file.
This can be achieved by the command
|\numberwithin{page}{|\textit{child}|}|
of the \textsf{amsmath} package
where \textit{child} can be |chapter| or |section|
depending on the chosen structuring.
Alternatively, one can modify the macro |\thepage| appropriately
and reset the counter |page| at the start of each child file.

%%%%%%%%%%%%%%%%%%%%%%%%%%%%%%%%%%%%%%%%%%%%%%%%%%%%%%%%%%%%%%%%%%%%%%%%%%%%%%%%
\subsection{Conditional Processing}
\label{sec:conditional}

The package provides a mechanism to compile different versions
of a document. To customise the versions further some conditional processing
can come in handy to distinguish which version is being compiled.
The package provides two macros to describe the compilation context:

%%%%%%%%%%%%%%%%%%%%%%%%%%%%%%%%%%%%%%%%
\DescribeMacro{\ifchilddoc}
The conditional |\ifchilddoc| distinguishes between the compilation of
child documents and the main document:
%
\begin{center}
|\ifchilddoc |\textit{child-code}| |[|\||else |\textit{main-code}]| \||fi|
\end{center}

%%%%%%%%%%%%%%%%%%%%%%%%%%%%%%%%%%%%%%%%
\DescribeMacro{\childdocname}
\DescribeMacro{\childdocjob}
The macro |\childdocname| contains the filename (without extension)
of the main or child file being processed.
Note that |\childdocjob| will always contain the name of the main file.

%%%%%%%%%%%%%%%%%%%%%%%%%%%%%%%%%%%%%%%%
\paragraph{Title Page.}

Conditional processing can be used to include a title or banner page
in the main document when proper precautions are taken.
Importantly, the code in the main file should ensure that the page counter
(as well as other status parameters which are stored in the |.aux| files)
takes the same value after the conditional processing.
Otherwise the page numbers may take divergent values
depending on which part is compiled.

For example, a title page could be declared by:
%
\begin{center}
\begin{tabular}{l}
|\ifchilddoc\||else|\\
|\addtocounter{page}{-1}|\\
\textit{code for title page}\\
|\newpage|\\
|\||fi|
\end{tabular}
\end{center}
%
A banner page for the child documents can be generated by:
%
\begin{center}
\begin{tabular}{l}
|\ifchilddoc|\\
|\addtocounter{page}{-1}|\\
\textit{code for banner page}\\
|\newpage|\\
|\||fi|
\end{tabular}
\end{center}
%
Here one could write a message such as:
\begin{center}
|This is the part \childdocname{} of \childdocjob{}.|
\end{center}

%%%%%%%%%%%%%%%%%%%%%%%%%%%%%%%%%%%%%%%%%%%%%%%%%%%%%%%%%%%%%%%%%%%%%%%%%%%%%%%%
\subsection{Flags}
\label{sec:flags}

The package makes it easy to generate different versions
of the main or child documents.
To this end compilation flags can be defined
and assigned different default values.
They will be particularly useful in conjunction
with the forwarding mechanism described in \secref{sec:forward}.

For example, it may be useful to have a flag |\version|
which can be set to |draft| or |final|.
The document source will contain some conditional code
depending on the value of |\version|.
Suppose further, the flag should default to |final| for the main file
and to |draft| for child files
which is a natural assignment for editing the document.
This is achieved by placing the following code
in the preamble of the main document
(below the |\childdocmain| directive):
%
\begin{center}
\begin{tabular}{l}
|\ifchilddoc|\\
|\providecommand{\version}{draft}|\\
|\||else|\\
|\providecommand{\version}{final}|\\
|\||fi|
\end{tabular}
\end{center}
%
The definition by |\providecommand| makes sure
that previous definitions are not overwritten.
Further statements |\providecommand{\version}{...}|
can thus be added before the above code to override it.

For the main file, one might add a line
(between |\childdocmain| and the above block)
%
\begin{center}
|%\ifchilddoc\||else\providecommand{\version}{draft}\||fi|
\end{center}
%
which can be uncommented to produce a draft version.
Likewise one can add a line to the very top of a child file
(above the |\childdocof{|\textit{main}|}| directive)
%
\begin{center}
|%\providecommand{\version}{final}|
\end{center}
%
which can be uncommented to produce the final version of this child document.

%%%%%%%%%%%%%%%%%%%%%%%%%%%%%%%%%%%%%%%%%%%%%%%%%%%%%%%%%%%%%%%%%%%%%%%%%%%%%%%%
\subsection{Forwarding}
\label{sec:forward}

Different versions of the main or child documents
using compilation flags as described in \secref{sec:flags}
can be (permanently) stored in different files
for convenient compilation, viewing and distribution.
To this end, the package defines a command
to pass on compilation to a different file:

%%%%%%%%%%%%%%%%%%%%%%%%%%%%%%%%%%%%%%%%
\DescribeMacro{\childdocforward}
The command |\childdocforward| redirects processing to
another source file:
%
\begin{center}
\begin{tabular}{l}
|\input{childdoc.def}|\\
|\childdocforward[|\textit{main}|]{|\textit{dest}|}|\\
\end{tabular}
\end{center}
%
The argument \textit{dest} is the destination file
(without extension).
It should be the main file or one of the child files.
Note that further \textsf{childdoc} directives
such as |\childdocof| and |\childdocforward|
in the indicated file will be processed in this form.
The optional argument \textit{main}
passes on directly to the main file \textit{main}
while pretending to compile the child \textit{dest}.
This form behaves as if \textit{dest}
issues |\childdocof{|\textit{main}|}| right away,
and no further \textsf{childdoc} directives will be processed.

%%%%%%%%%%%%%%%%%%%%%%%%%%%%%%%%%%%%%%%%
\DescribeMacro{\...prefix}
In the alternative form |\childdocforwardprefix|,
%
\begin{center}
\begin{tabular}{l}
|\input{childdoc.def}|\\
|\childdocforwardprefix[|\textit{main}|]{|\textit{prefix}|}{|\textit{dest}|}|
\end{tabular}
\end{center}
%
the destination file is determined by a pattern
depending on the current file:
To make this work, the current file must be called
`{\textit{prefix}\hspace{0.2em}\textit{suffix}}'
with \textit{prefix} matching precisely the argument.
Processing is then passed on to the file
`{\textit{dest}\hspace{0.2em}\textit{suffix}}'.
Surely, the same effect is achieved by
directly specifying the
argument `{\textit{dest}\hspace{0.2em}\textit{suffix}}'
in the first form.
However, that requires to set up a different file
for each child. With the alternative form of the command
all these files can have exactly the same content
which simplifies setting them up and maintaining them.

For example, the following file |draft.tex|
with a compilation flag |\version| as described in \secref{sec:flags}
compiles the main document as a draft:
%
\begin{center}
\begin{tabular}{l}
|\def\version{draft}|\\
|\input{childdoc.def}|\\
|\childdocforward{|\textit{main}|}|
\end{tabular}
\end{center}
%
Likewise, the following files |final|\textit{nn}|.tex|
compile the final version of the child document
|child|\textit{nn}|.tex|:
%
\begin{center}
\begin{tabular}{l}
|\def\version{final}|\\
|\input{childdoc.def}|\\
|\childdocforwardprefix{final}{child}|
\end{tabular}
\end{center}
%

Note that when several versions of a main file and/or of each child file
are to be generated, it may be convenient to set up a |Makefile| or
shell script to automatise the process.

%%%%%%%%%%%%%%%%%%%%%%%%%%%%%%%%%%%%%%%%%%%%%%%%%%%%%%%%%%%%%%%%%%%%%%%%%%%%%%%%
\subsection{Command Line Processing}
\label{sec:commandline}

The effect of redirection files can also be achieved by invoking
the \LaTeX{} compiler with a more elaborate command line.
Most conveniently this should be done as part
of a shell script or a |Makefile|.

When using \textsf{childdoc} in the main file, the following
command lines effectively perform a redirection
(note that depending on the shell being used,
backslashes may have to be doubled: `|\|' $\to$ `|\\|'):
%
\begin{center}
|... -jobname "|\textit{target}|" |\\|"|[\textit{flags}]%
|\input{childdoc.def}\childdocforward[|\textit{main}|]{|\textit{dest}|}"|
\end{center}
%
Here \textit{target} is the name of the output file,
\textit{main} is the name of the main file
and \textit{dest} is the name of the main or child file to be processed
(all filenames without extensions).
The optional argument \textit{main} can be omitted
if \textit{main} matches \textit{dest}.
Optionally, compilation \textit{flags} can be defined via |\def| commands.
This command line makes the \TeX{} engine believe
it is compiling the file \textit{target}
whose content is specified as the latter parameter.
The provided code then forwards the processing to
\textit{main} or \textit{dest} as described in \secref{sec:forward}.

%%%%%%%%%%%%%%%%%%%%%%%%%%%%%%%%%%%%%%%%%%%%%%%%%%%%%%%%%%%%%%%%%%%%%%%%%%%%%%%%
\subsection{Include by Input}
\label{sec:input}

Including child documents by |\include| has some restrictions by design.
Most notably, the content of a child document always occupies
its own set of pages; pages cannot be shared between child documents.
Usually, this behaviour makes perfect sense
because each child document contain an essential part of the document.
However, in some situations it may be desirable to compose
a document from a collection of parts
without having mandatory page breaks between then.
For this case, the package
provides a mechanism to include parts
by |\input| which can also be processed individually.
However, by construction this mechanism
requires manual handling of the content to be output.

%%%%%%%%%%%%%%%%%%%%%%%%%%%%%%%%%%%%%%%%
\DescribeMacro{\ifchilddocmanual}
The main file should be prepared as usual, see \secref{sec:include}.
However, the document body must make a distinction
between processing of an individual part and of the main document, e.g.:
%
\begin{center}
\begin{tabular}{l}
|\ifchilddocmanual|\\
|\input{\childdocname}|\\
|\||else|\\
\textit{document body with }|\input{|\textit{part}|}|\\
|\||fi|
\end{tabular}
\end{center}
%
The conditional |\ifchilddocmanual| is true whenever
a part to be included by |\input| is being compiled,
and the name of the part is stored in |\childdocname|.

%%%%%%%%%%%%%%%%%%%%%%%%%%%%%%%%%%%%%%%%
\DescribeMacro{\childdocby}
Each part to be included by |\input| should start with:
%
\begin{center}
\begin{tabular}{l}
|\input{childdoc.def}|\\
|\childdocby{|\textit{main}|}|\\
\end{tabular}
\end{center}
%
The directive |\childdocby| is similar to |\childdocof|
described in \secref{sec:include},
but the subsequent selection of content must be done manually.
To that end, both |\ifchilddoc| and |\ifchilddocmanual|
will be true upon processing of a part,
and the name of the part is stored in |\childdocname|.
Note that |\jobname| will be set to the filename of the current part
so that each part receives an individual |.aux| file
that does not interfere with the |.aux| file(s) of the main document.
This behaviour can be altered by the alternative form
|\childdocby[*]{|\textit{main}|}| (with a non-empty optional argument)
which uses the |.aux| file of the main document
by setting |\jobname| to \textit{main}.

%%%%%%%%%%%%%%%%%%%%%%%%%%%%%%%%%%%%%%%%%%%%%%%%%%%%%%%%%%%%%%%%%%%%%%%%%%%%%%%%
\subsection{Driver Development}
\label{sec:driver}

The \textsf{childdoc} mechanism can also be use for the development
of definition files such as \LaTeX{} styles or classes.
This case differs from the above setup with multiple parts
included by |\include| in that no |\includeonly| should be invoked.
This can be achieved by starting the include file
(before |\ProvidesPackage|) with:
%
\begin{center}
\begin{tabular}{l}
|\input{childdoc.def}|\\
|\childdocforward{|\textit{main}|}|\\
\end{tabular}
\end{center}
%
or alternatively with:
%
\begin{center}
\begin{tabular}{l}
|\input{childdoc.def}|\\
|\childdocby{|\textit{main}|}|\\
\end{tabular}
\end{center}
%
Both forms have slightly different effects as described above.
The main file is prepared as usual, see \secref{sec:include}.

%%%%%%%%%%%%%%%%%%%%%%%%%%%%%%%%%%%%%%%%%%%%%%%%%%%%%%%%%%%%%%%%%%%%%%%%%%%%%%%%
\subsection{Legacy Detection}
\label{sec:detection}

The directive |\childdocmain| in the main file can detect
whether the complete document or merely a child is to be compiled
even without using the directive |\childdocof|.
This method is deprecated because it is less robust
and there is no compelling reason to use it;
it is merely provided for backward compatibility
and it may be removed in future versions.

If the detection mechanism is to be used,
it is mandatory to correctly specify
the filename of the main file as the argument of |\childdocmain|:
%
\begin{center}
\begin{tabular}{l}
|\input{childdoc.def}|\\
|\childdocmain{|\textit{main}|}|\\
\end{tabular}
\end{center}
%
If |\jobname| does not match the argument \textit{main} of |\childdocmain|,
it is assumed that |\jobname| points to the child file to be compiled.
When using |\childdocmain| with the main file specified as argument,
it suffices to start a child file
with just |\input{|\textit{main}|}|
without loading of the package and using |\childdocof|.
If instead all processing is done
with the appropriate \textsf{childdoc} directives,
the argument of \textit{main} of |\childdocmain| can be empty.

An alternative version of the command line processing described
in \secref{sec:commandline} using the detection mechanism reads:
%
\begin{center}
|... -jobname "|\textit{target}|" "|[\textit{flags}]%
[|\def\jobname{|\textit{dest}|}|]|\input{|\textit{main}|}"|
\end{center}

%%%%%%%%%%%%%%%%%%%%%%%%%%%%%%%%%%%%%%%%%%%%%%%%%%%%%%%%%%%%%%%%%%%%%%%%%%%%%%%%
\subsection{Manual Code}
\label{sec:manual}

In case one cannot be certain whether the definitions file |childdoc.def|
is installed on the target \TeX{} distribution
and one prefers not to ship it,
it is conceivable to paste a few relevant commands into the sources.

To that end, drop all statements |\input{childdoc.def}|
and perform the replacements as outlined below.
Instead of |\childdocmain{|\textit{main}|}| add the following code
to the top of the main file:
%
\begin{center}
\begin{tabular}{l}
|\||ifdefined\childdocname\endinput\||fi\newif\ifchilddoc|\\
|\edef\childdocname{\scantokens\expandafter{\jobname\noexpand}}|\\
|\def\childdocmain{|\textit{main}|}\||ifx\childdocmain\childdocname\||else|\\
|\childdoctrue\includeonly{\childdocname}\let\jobname\childdocmain\||fi|\\
\end{tabular}
\end{center}
%
Instead of |\childdocof{|\textit{main}|}| just include the main file
at the top of each child file:
%
\begin{center}
|\input{|\textit{main}|}|
\end{center}
%
A simple redirection |\childdocforward{|\textit{dest}|}| is achieved by:
%
\begin{center}
|\def\jobname{|\textit{dest}|}\input{\jobname}|
\end{center}
%
The redirection with prefix
|\childdocforwardprefix[|\textit{prefix}|]{|\textit{dest}|}|
is accomplished by:
%
\begin{center}
\begin{tabular}{l}
|{\edef\jobname{\scantokens\expandafter{\jobname\noexpand}}|\\
|\def\redirectjob |\textit{prefix}|#1~~~{\gdef\jobname{|\textit{dest}|#1}}|\\
|\expandafter\redirectjob\jobname~~~}\input{\jobname}|
\end{tabular}
\end{center}

In an alternative approach,
child documents can be compiled by a specific command line
without additional code or specific definitions:
%
\begin{center}
|... -jobname "|\textit{target}|" "|[\textit{flags}]%
|\includeonly{|\textit{dest}|}\input{|\textit{main}|}"|
\end{center}
%

%%%%%%%%%%%%%%%%%%%%%%%%%%%%%%%%%%%%%%%%%%%%%%%%%%%%%%%%%%%%%%%%%%%%%%%%%%%%%%%%
%%%%%%%%%%%%%%%%%%%%%%%%%%%%%%%%%%%%%%%%%%%%%%%%%%%%%%%%%%%%%%%%%%%%%%%%%%%%%%%%
\section{Information}

%%%%%%%%%%%%%%%%%%%%%%%%%%%%%%%%%%%%%%%%%%%%%%%%%%%%%%%%%%%%%%%%%%%%%%%%%%%%%%%%
\subsection{Copyright}

Copyright \copyright{} 2017--2018 Niklas Beisert

This work may be distributed and/or modified under the
conditions of the \LaTeX{} Project Public License, either version 1.3
of this license or (at your option) any later version.
The latest version of this license is in
  \url{http://www.latex-project.org/lppl.txt}
and version 1.3 or later is part of all distributions of \LaTeX{}
version 2005/12/01 or later.

This work has the LPPL maintenance status `maintained'.

The Current Maintainer of this work is Niklas Beisert.

This work consists of the files |README.txt|, |childdoc.ins| and |childdoc.dtx|
as well as the derived files |childdoc.def|, |cdocsamp.tex|
with |cdocsch1.tex|, |cdocsch2.tex|, |cdocspt3.tex|, |cdocspt4.tex|,
|cdocsdrf.tex|, |cdocsfn1.tex|, |cdocsfn2.tex|
as well as |childdoc.pdf|.

%%%%%%%%%%%%%%%%%%%%%%%%%%%%%%%%%%%%%%%%%%%%%%%%%%%%%%%%%%%%%%%%%%%%%%%%%%%%%%%%
\subsection{Files and Installation}

The package consists of the files:
%
\begin{center}
\begin{tabular}{ll}
    |README.txt|   & readme file \\
    |childdoc.ins| & installation file \\
    |childdoc.dtx| & source file \\
    |childdoc.def| & definition file \\
    |cdocsamp.tex| & sample main file \\
    |cdocsch1.tex| & sample include file \\
    |cdocsch2.tex| & sample include file \\
    |cdocspt3.tex| & sample part file \\
    |cdocspt4.tex| & sample part file \\
    |cdocsdrf.tex| & sample redirection file \\
    |cdocsfn1.tex| & sample redirection file \\
    |cdocsfn2.tex| & sample redirection file \\
    |childdoc.pdf| & manual
\end{tabular}
\end{center}
%
The distribution consists of the files
|README.txt|, |childdoc.ins| and |childdoc.dtx|.
%
\begin{itemize}
\item
Run (pdf)\LaTeX{} on |childdoc.dtx|
to compile the manual |childdoc.pdf| (this file).
\item
Run \LaTeX{} on |childdoc.ins| to create the definitions file |childdoc.def|
and the sample |cdocsamp.tex| with include files
|cdocsch1.tex|, |cdocsch2.tex|, |cdocspt3.tex|, |cdocspt4.tex|,
|cdocsdrf.tex|, |cdocsfn1.tex|, |cdocsfn2.tex|.
Then copy the file |childdoc.def| to an appropriate directory of your \LaTeX{}
distribution, e.g.\ \textit{texmf-root}|/tex/latex/childdoc|.
\end{itemize}

%%%%%%%%%%%%%%%%%%%%%%%%%%%%%%%%%%%%%%%%%%%%%%%%%%%%%%%%%%%%%%%%%%%%%%%%%%%%%%%%
\subsection{Related CTAN Packages}

There are several other packages which offer a similar functionality:
%
\begin{itemize}
\item
The packages
\href{http://ctan.org/pkg/docmute}{\textsf{docmute}},
\href{http://ctan.org/pkg/includex}{\textsf{includex}} and
\href{http://ctan.org/pkg/standalone}{\textsf{standalone}}
provide commands to include only the document body of
a child file thus allowing both files to be compiled individually.
\item
The packages \href{http://ctan.org/pkg/subdocs}{\textsf{subdocs}}
and \href{http://ctan.org/pkg/subfiles}{\textsf{subfiles}}
provide structures in which the main and child documents can be
encapsulated and allowing them to be compiled individually.
The inclusion mechanism is different from the conventional |\include|.
\item
The package \href{http://ctan.org/pkg/combine}{\textsf{combine}}
is an elaborate solution to combine several documents into one.
\end{itemize}
%
See also the CTAN topic \href{http://ctan.org/topic/subdocs}{\textsf{subdocs}}
for further related packages.
The present package differs from the above solutions in that
a document structure constructed with the conventional |\include| mechanism
just needs two extra commands at the top of every file
such that all constituent files can be compiled individually.

%%%%%%%%%%%%%%%%%%%%%%%%%%%%%%%%%%%%%%%%%%%%%%%%%%%%%%%%%%%%%%%%%%%%%%%%%%%%%%%%
%\subsection{Feature Suggestions}
%
%The following is a list of features which may be useful for future
%versions of this package:
%%
%\begin{itemize}
%\item
%\ldots
%\end{itemize}

%%%%%%%%%%%%%%%%%%%%%%%%%%%%%%%%%%%%%%%%%%%%%%%%%%%%%%%%%%%%%%%%%%%%%%%%%%%%%%%%
\subsection{Revision History}

%%%%%%%%%%%%%%%%%%%%%%%%%%%%%%%%%%%%%%%%
\paragraph{v2.0:} 2018/12/30

\begin{itemize}
\item
immediate forward processing
\item
added |\childdocby| mechanism
\item
manual restructured
\end{itemize}

%%%%%%%%%%%%%%%%%%%%%%%%%%%%%%%%%%%%%%%%
\paragraph{v1.6:} 2018/01/17

\begin{itemize}
\item
application for development of include files
\item
corrections to manual
\end{itemize}

%%%%%%%%%%%%%%%%%%%%%%%%%%%%%%%%%%%%%%%%
\paragraph{v1.5:} 2017/05/21

\begin{itemize}
\item
more complete structuring introduced
\item
|\childdocof| introduced
\item
|\childdoc| renamed to |\childdocmain|
\item
|\childredirect| renamed to |\childdocforward| and |\childdocforwardprefix|
and functionality expanded
\end{itemize}

%%%%%%%%%%%%%%%%%%%%%%%%%%%%%%%%%%%%%%%%
\paragraph{v1.0:} 2017/04/27

\begin{itemize}
\item
manual and install package
\item
first version published on CTAN
\end{itemize}

%%%%%%%%%%%%%%%%%%%%%%%%%%%%%%%%%%%%%%%%
\paragraph{v0.6:} 2017/04/26

\begin{itemize}
\item
redirection mechanism added
\end{itemize}

%%%%%%%%%%%%%%%%%%%%%%%%%%%%%%%%%%%%%%%%
\paragraph{v0.5:} 2017/04/26

\begin{itemize}
\item
functionality in definition file
\end{itemize}


%%%%%%%%%%%%%%%%%%%%%%%%%%%%%%%%%%%%%%%%%%%%%%%%%%%%%%%%%%%%%%%%%%%%%%%%%%%%%%%%
%%%%%%%%%%%%%%%%%%%%%%%%%%%%%%%%%%%%%%%%%%%%%%%%%%%%%%%%%%%%%%%%%%%%%%%%%%%%%%%%
%%%%%%%%%%%%%%%%%%%%%%%%%%%%%%%%%%%%%%%%%%%%%%%%%%%%%%%%%%%%%%%%%%%%%%%%%%%%%%%%
\appendix

\settowidth\MacroIndent{\rmfamily\scriptsize 000\ }

 \DocInput{childdoc.dtx}

\end{document}
%</driver>
% \fi
%
% %%%%%%%%%%%%%%%%%%%%%%%%%%%%%%%%%%%%%%%%%%%%%%%%%%%%%%%%%%%%%%%%%%%%%%%%%%%%%%
% %%%%%%%%%%%%%%%%%%%%%%%%%%%%%%%%%%%%%%%%%%%%%%%%%%%%%%%%%%%%%%%%%%%%%%%%%%%%%%
% \section{Sample}
%\iffalse
%<*samplemain>
%\fi
%
% The following presents a sample document
% with two chapters, two parts, a title page,
% a compile flag as well as three forwarding files to set the flag.
% It consists of eight |.tex| files:
% \begin{center}
% \begin{tabular}{ll}
% |cdocsamp.tex|&main file\\
% |cdocsch1.tex|&include file for chapter 1\\
% |cdocsch2.tex|&include file for chapter 2\\
% |cdocspt3.tex|&include file for part 3\\
% |cdocspt4.tex|&include file for part 4\\
% |cdocsdrf.tex|&forwarding file for main file in draft mode\\
% |cdocsfi1.tex|&forwarding file for final version of chapter 1\\
% |cdocsfi2.tex|&forwarding file for final version of chapter 2\\
% \end{tabular}
% \end{center}
% Each of the eight files can be compiled directly by the \LaTeX{} compiler.
%
% %%%%%%%%%%%%%%%%%%%%%%%%%%%%%%%%%%%%%%
% \paragraph{Main File.}
%
% The main file is called |cdocsamp.tex|.
%
% Load the \textsf{childdoc} definitions and
% declare the filename for the main document:
%    \begin{macrocode}
\input{childdoc.def}
\childdocmain{}
%    \end{macrocode}

% Optional override for |\version| flag:
%    \begin{macrocode}
%%\ifchilddoc\else\providecommand{\version}{draft}\fi
%    \end{macrocode}

% Define the default values for the |\version| flag
% (|final| for the main file and |draft| for childs):
%    \begin{macrocode}
\ifchilddoc
\providecommand{\version}{draft}
\else
\providecommand{\version}{final}
\fi
%    \end{macrocode}

% Load the standard document class:
%    \begin{macrocode}
\documentclass[12pt]{article}
%    \end{macrocode}

% Start the document body:
%    \begin{macrocode}
\begin{document}
%    \end{macrocode}

% Declare a title page.
% Print title, part of document being processed and version flag:
%    \begin{macrocode}
\addtocounter{page}{-1}
\begin{center}
{\LARGE\bfseries{}childdoc example\par}
\vspace{1cm}
\ifchilddoc
\ifchilddocmanual part\else chapter\fi:
`\childdocname' of `\childdocjob'\par
\else
main document: `\childdocjob'\par
\fi
version: \version\par
\end{center}
\newpage
%    \end{macrocode}

% Manually include selected file,
% otherwise process as usual:
%    \begin{macrocode}
\ifchilddocmanual
\section*{part `\childdocname'}
\input{\childdocname}
\else
%    \end{macrocode}

% Include the two chapters:
%    \begin{macrocode}
\include{cdocsch1}
\include{cdocsch2}
%    \end{macrocode}

% Include the two parts unless only chapters should be displayed:
%    \begin{macrocode}
\ifchilddoc\else
\section{part three}
\input{cdocspt3}
\section{part four}
\input{cdocspt4}
\fi
%    \end{macrocode}

% Process as usual until here:
%    \begin{macrocode}
\fi
%    \end{macrocode}

% End of document body:
%    \begin{macrocode}
\end{document}
%    \end{macrocode}
%\iffalse
%</samplemain>
%\fi
%
% %%%%%%%%%%%%%%%%%%%%%%%%%%%%%%%%%%%%%%
% \paragraph{Chapter Include Files.}
%
% The include files are called |cdocsch1.tex| and |cdocsch2.tex|.
%
%\iffalse
%<*samplechap1|samplechap2>
%\fi

% Optional override for |\version| flag:
%    \begin{macrocode}
%%\providecommand{\version}{final}
%    \end{macrocode}

% Include the main document:
%    \begin{macrocode}
\input{childdoc.def}
\childdocof{cdocsamp}
%    \end{macrocode}

%\iffalse
%</samplechap1|samplechap2>
%\fi
%
%\iffalse
%<*samplechap1>
%\fi
% Some text for chapter 1:
%    \begin{macrocode}
\section{one}
some text in chapter one
%    \end{macrocode}

%\iffalse
%</samplechap1>
%\fi
% Some text for chapter 2:
%\iffalse
%<*samplechap2>
%\fi
%    \begin{macrocode}
\section{two}
more text in chapter two
%    \end{macrocode}

%\iffalse
%</samplechap2>
%\fi
%
% %%%%%%%%%%%%%%%%%%%%%%%%%%%%%%%%%%%%%%
% \paragraph{Part Include Files.}
%
% The include files are called |cdocspt3.tex| and |cdocspt4.tex|.
%
%\iffalse
%<*samplepart3|samplepart4>
%\fi

% Optional override for |\version| flag:
%    \begin{macrocode}
%%\providecommand{\version}{final}
%    \end{macrocode}

% Include the main document:
%    \begin{macrocode}
\input{childdoc.def}
\childdocby{cdocsamp}
%    \end{macrocode}

%\iffalse
%</samplepart3|samplepart4>
%\fi
%
%\iffalse
%<*samplepart3>
%\fi
% Some text for part 3:
%    \begin{macrocode}
some text in part three
%    \end{macrocode}

%\iffalse
%</samplepart3>
%\fi
% Some text for part 4:
%\iffalse
%<*samplepart4>
%\fi
%    \begin{macrocode}
more text in part four
%    \end{macrocode}

%\iffalse
%</samplepart4>
%\fi
%
% %%%%%%%%%%%%%%%%%%%%%%%%%%%%%%%%%%%%%%
% \paragraph{Forwarding for a Complete Draft.}
%
% The following forwarding file |cdocsdrf.tex|
% compiles the main document in draft mode:
%\iffalse
%<*sampledraft>
%\fi
%    \begin{macrocode}
\def\version{draft}
\input{childdoc.def}
\childdocforward{cdocsamp}
%    \end{macrocode}

%\iffalse
%</sampledraft>
%\fi
%
% %%%%%%%%%%%%%%%%%%%%%%%%%%%%%%%%%%%%%%
% \paragraph{Forwarding for Final Version of the Chapters.}
%
% The following forwarding files |cdocsfn1.tex| and |cdocsfn2.tex|
% (with identical content)
% compile the final versions of the child documents
% |cdocsch1.tex| and |cdocsch2.tex|, respectively:
%\iffalse
%<*samplefinal>
%\fi
%    \begin{macrocode}
\def\version{final}
\input{childdoc.def}
\childdocforwardprefix[cdocsamp]{cdocsfn}{cdocsch}
%    \end{macrocode}

%\iffalse
%</samplefinal>
%\fi
%
% %%%%%%%%%%%%%%%%%%%%%%%%%%%%%%%%%%%%%%
% \paragraph{Command Line Processing.}
%
% The following three command lines generate the output files
% |cdocscld|, |cdocscl1| and |cdocscl2|
% which should be identical to
% |cdocsdrf|, |cdocsch1| and |cdocsfn2|, respectively:
% \begin{center}
% \begin{tabular}{l}
% |latex -jobname cdocscld \|\\
% |  "\def\version{draft}\input{childdoc.def}\childdocforward{cdocsamp}"|\\
% |latex -jobname cdocscl1 \|\\
% |  "\input{childdoc.def}\childdocforward[cdocsamp]{cdocsch1}"|\\
% |latex -jobname cdocscl2 \|\\
% |  "\def\version{final}\input{childdoc.def}\childdocforward{cdocsch2}"|
% \end{tabular}
% \end{center}
% Note that the trailing backslash on each first line
% merely continues the input to the second line
% (for convenient cut ant paste).
% Furthermore, the command |latex| can be replaced by any
% of its alternative versions such as |pdflatex|.
%
% %%%%%%%%%%%%%%%%%%%%%%%%%%%%%%%%%%%%%%%%%%%%%%%%%%%%%%%%%%%%%%%%%%%%%%%%%%%%%%
% %%%%%%%%%%%%%%%%%%%%%%%%%%%%%%%%%%%%%%%%%%%%%%%%%%%%%%%%%%%%%%%%%%%%%%%%%%%%%%
% \section{Implementation}
%\iffalse
%<*package>
%\fi
%
% This section describes the definitions file |childdoc.def|.

% The definitions cannot be loaded using |\usepackage| or |\RequirePackage|
% which has a mechanism to prevent loading a style file more than once.
% When loading the definitions by means of |\input|
% multiple instances have to be prevented manually:
%\iffalse
%This code needs to be before the `\ProvidesFile' directive
%which is defined at the beginning of this file.
%Therefore it is also placed there and commented out here.
%</package>
%<*discard>
%\fi
%    \begin{macrocode}
\ifdefined\childdocmain\endinput\fi
%    \end{macrocode}
%\iffalse
%</discard>
%<*package>
%\fi
%
% \macro{\ifchilddoc}
% \macro{\ifchilddocmanual}
% The conditional |\ifchilddoc| tells whether a
% child (true) or main (false) document is being compiled.
% The conditional |\ifchilddocmanual| tells whether
% the |\includeonly| mechanism is used (false) or
% the selection of child files must be performed manually (true).
% The definitions initialise to false:
%    \begin{macrocode}
\newif\ifchilddoc
\newif\ifchilddocmanual
%    \end{macrocode}

% \macro{\childdocname}
% \macro{\childdocjob}
% The macro |\childdocname| stores the name of the main document
% to be compiled. The macro |\childdocjob| stores the name of
% the document on which the \LaTeX{} compiler was originally invoked.
% The content of |\jobname| cannot be compared
% to filenames specified in the source due to different catcodes.
% The following code rescans |\jobname|, stores the result
% in |\childdocname| and saves a copy in |\childdocjob|:
%    \begin{macrocode}
\edef\childdocname{\scantokens\expandafter{\jobname\noexpand}}
\let\childdocjob\childdocname
%    \end{macrocode}

% \macro{\childdocdisable}
% The macro |\childdocdisable| prevents the main file
% from being processed more than once.
% At this stage, the main document command |\childdocmain|
% is assumed to be called once again where it should do nothing.
% Any subsequent call to it should prevent
% a secondary processing of the main document
% It overwrites the forwarding commands
% |\childdocof| and |\childdocforward|
% with empty macros to prevent further inclusions of the main document:
%    \begin{macrocode}
\newcommand{\childdocdisable}
{
  \renewcommand{\childdocmain}[1]{\renewcommand{\childdocmain}[1]{\endinput}}
  \renewcommand{\childdocof}[1]{}
  \renewcommand{\childdocby}[2][]{}
  \renewcommand{\childdocforward}[2][]{}
  \renewcommand{\childdocdisable}{}
}
%    \end{macrocode}

% \macro{\childdocmain}
% The macro |\childdocmain| is to be called at the top of the main file
% with nothing or the main filename (without extension) as argument.
% First, it breaks loops.
% If the argument is not empty and does not match |\childdocname|
% (which is set by the first inclusion of |childdoc.def|),
% |\ifchilddoc| is set to true, |\includeonly| is applied to the child file
% and |\jobname| is set to the main file
% (for proper handling of |.aux| files):
%    \begin{macrocode}
\newcommand{\childdocmain}[1]
{
  \childdocdisable\childdocmain{}
  \if?#1?\else
    \begingroup
      \def\childdoctmp{#1}
      \ifx\childdoctmp\childdocname
        \def\childdoctmp{}
      \else
        \def\childdoctmp
        {
          \childdoctrue
          \includeonly{\childdocname}
          \def\childdocjob{#1}
          \def\jobname{#1}
        }
      \fi
      \expandafter
    \endgroup
    \childdoctmp
  \fi
}
%    \end{macrocode}

% \macro{\childdocof}
% The command |\childdocof| redirects
% compilation to the main file |#1|.
%    \begin{macrocode}
\newcommand{\childdocof}[1]
{
  \childdocdisable
  \childdoctrue
  \includeonly{\childdocname}
  \def\jobname{#1}
  \def\childdocjob{#1}
  \input{#1}
}
%    \end{macrocode}

% \macro{\childdocby}
% The command |\childdocby| ....
%    \begin{macrocode}
\newcommand{\childdocby}[2][]
{
  \childdocdisable
  \childdoctrue
  \childdocmanualtrue
  \if?#1?\else
    \def\jobname{#2}
  \fi
  \def\childdocjob{#2}
  \input{#2}
  \endinput
}
%    \end{macrocode}

% \macro{\childdocforward}
% The command |\childdocforward| redirects
% compilation to the main file or
% (if the optional argument is given) a child file.
% Parameters are set as if the main file
% or a child file starting with |\childdocof| was compiled.
% Then compilation is handed over to the main file:
%    \begin{macrocode}
\newcommand{\childdocforward}[2][]
{
  \begingroup
    \if?#1?
      \def\childdoctmp
      {
        \def\childdocname{#2}
        \def\childdocjob{#2}
        \def\jobname{#2}
        \input{#2}
        \endinput
      }
    \else
      \def\childdoctmp
      {
        \childdocdisable
        \def\childdocname{#2}
        \childdoctrue
        \includeonly{#2}
        \def\childdocjob{#1}
        \def\jobname{#1}
        \input{#1}
        \endinput
      }
    \fi
    \expandafter
  \endgroup
  \childdoctmp
}
%    \end{macrocode}

% \macro{\childdocforwardprefix}
% The command |\childdocforwardprefix| redirects
% compilation to the main or a child file by means of a pattern.
% The prefix |#1| in the current filename is replaced by |#2|
% and the suffix of the current filename is kept
% (it is assumed that the filename does not contain the substring `|~~~|'
% which is used as a delimiter).
% Compilation is handed over to the new file by |\childdocforward|:
%    \begin{macrocode}
\newcommand{\childdocforwardprefix}[3][]
{
  \begingroup
    \def\childdocextract #2##1~~~{\def\childdoctmp{\childdocforward[#1]{#3##1}}}
    \expandafter\childdocextract\childdocname~~~
    \expandafter
  \endgroup
  \childdoctmp
}
%    \end{macrocode}

% \macro{\childdoc}
% The deprecated macro |\childdoc| is a legacy version of |\childdocmain|:
%    \begin{macrocode}
\newcommand{\childdoc}{\childdocmain}
%    \end{macrocode}

% \macro{\childdocredirect}
% The deprecated macro |\childdocredirect| is a legacy version
% of |\childdocforward| and |\childdocforwardprefix|:
%    \begin{macrocode}
\newcommand{\childdocredirect}[2][]
{
  \begingroup
    \if?#1?
      \def\childdoctmp{\childdocforward{#2}}
    \else
      \def\childdoctmp{\childdocforwardprefix{#1}{#2}}
    \fi
    \expandafter
  \endgroup
  \childdoctmp
}
%    \end{macrocode}

%\iffalse
%</package>
%\fi
%
\endinput
|\\
|\childdocof{|\textit{main}|}|\\
\end{tabular}
\end{center}
at the top of every child file \textit{child}
which is included by |\include{|\textit{child}|}|
from within the main file
(or at least for those files to be compiled individually).
The argument \textit{main} must be the filename of the main file.

There are a couple of
considerations in setting up the main and child documents:

%%%%%%%%%%%%%%%%%%%%%%%%%%%%%%%%%%%%%%%%
\paragraph{Restrictions.}

Please note the following restrictions:
\begin{itemize}
\item
|\childdocmain| must be called with one argument \textit{main}
to ensure compatibility with earlier version of the package.
It must either be empty (|\childdocmain{}|)
or precisely match the filename of the main file in which it is specified.
See \secref{sec:detection} for further information.
\item
The filename \textit{main} must be specified without the |.tex| extension.
\item
The filename \textit{main} is case sensitive
(even in case-insensitive file systems)
due to internal string comparison.
\item
The argument \textit{main} should be fully expanded, it cannot be a macro.
\item
Subdirectories and special characters should be avoided in filenames.
\item
The command |\childdocmain{|\textit{main}|}| must be followed by a whitespace.
It should not be followed immediately by another command
or by a comment mark `|%|'.
This is because the \TeX{} parser reads the token immediately following
the argument of |\childdocmain| and puts it
at the beginning of every child section;
however, a white\-space is ignored.
\end{itemize}

%%%%%%%%%%%%%%%%%%%%%%%%%%%%%%%%%%%%%%%%
\paragraph{Content of Main File.}

It is advisable to place all content in the child files included by |\include|.
Any output contained in the main file will appear in all child documents
unless suppressed manually;
it cannot be suppressed automatically by the |\includeonly| directive
and thus should normally be avoided.
A method to include some content in the main file
by means of conditional processing is described in \secref{sec:conditional}.

%%%%%%%%%%%%%%%%%%%%%%%%%%%%%%%%%%%%%%%%
\paragraph{Page Numbering.}

When only a part of the document is compiled,
the appropriate numbering of pages
(as well as other status parameters)
is determined from the |.aux| files.
The latter contain information from previous passes.
However this information needs to propagate through
all intermediate child documents.
Therefore the page numbering in child documents may well
be inconsistent until the complete document is compiled at least once.

A useful (if unconventional) way to always ensure a consistent
page numbering is to restart the numbering in each child document
and denote the pages by `\textit{child}|.|\textit{page}'
where \textit{child} represents the chapter/section number of the child file.
This can be achieved by the command
|\numberwithin{page}{|\textit{child}|}|
of the \textsf{amsmath} package
where \textit{child} can be |chapter| or |section|
depending on the chosen structuring.
Alternatively, one can modify the macro |\thepage| appropriately
and reset the counter |page| at the start of each child file.

%%%%%%%%%%%%%%%%%%%%%%%%%%%%%%%%%%%%%%%%%%%%%%%%%%%%%%%%%%%%%%%%%%%%%%%%%%%%%%%%
\subsection{Conditional Processing}
\label{sec:conditional}

The package provides a mechanism to compile different versions
of a document. To customise the versions further some conditional processing
can come in handy to distinguish which version is being compiled.
The package provides two macros to describe the compilation context:

%%%%%%%%%%%%%%%%%%%%%%%%%%%%%%%%%%%%%%%%
\DescribeMacro{\ifchilddoc}
The conditional |\ifchilddoc| distinguishes between the compilation of
child documents and the main document:
%
\begin{center}
|\ifchilddoc |\textit{child-code}| |[|\||else |\textit{main-code}]| \||fi|
\end{center}

%%%%%%%%%%%%%%%%%%%%%%%%%%%%%%%%%%%%%%%%
\DescribeMacro{\childdocname}
\DescribeMacro{\childdocjob}
The macro |\childdocname| contains the filename (without extension)
of the main or child file being processed.
Note that |\childdocjob| will always contain the name of the main file.

%%%%%%%%%%%%%%%%%%%%%%%%%%%%%%%%%%%%%%%%
\paragraph{Title Page.}

Conditional processing can be used to include a title or banner page
in the main document when proper precautions are taken.
Importantly, the code in the main file should ensure that the page counter
(as well as other status parameters which are stored in the |.aux| files)
takes the same value after the conditional processing.
Otherwise the page numbers may take divergent values
depending on which part is compiled.

For example, a title page could be declared by:
%
\begin{center}
\begin{tabular}{l}
|\ifchilddoc\||else|\\
|\addtocounter{page}{-1}|\\
\textit{code for title page}\\
|\newpage|\\
|\||fi|
\end{tabular}
\end{center}
%
A banner page for the child documents can be generated by:
%
\begin{center}
\begin{tabular}{l}
|\ifchilddoc|\\
|\addtocounter{page}{-1}|\\
\textit{code for banner page}\\
|\newpage|\\
|\||fi|
\end{tabular}
\end{center}
%
Here one could write a message such as:
\begin{center}
|This is the part \childdocname{} of \childdocjob{}.|
\end{center}

%%%%%%%%%%%%%%%%%%%%%%%%%%%%%%%%%%%%%%%%%%%%%%%%%%%%%%%%%%%%%%%%%%%%%%%%%%%%%%%%
\subsection{Flags}
\label{sec:flags}

The package makes it easy to generate different versions
of the main or child documents.
To this end compilation flags can be defined
and assigned different default values.
They will be particularly useful in conjunction
with the forwarding mechanism described in \secref{sec:forward}.

For example, it may be useful to have a flag |\version|
which can be set to |draft| or |final|.
The document source will contain some conditional code
depending on the value of |\version|.
Suppose further, the flag should default to |final| for the main file
and to |draft| for child files
which is a natural assignment for editing the document.
This is achieved by placing the following code
in the preamble of the main document
(below the |\childdocmain| directive):
%
\begin{center}
\begin{tabular}{l}
|\ifchilddoc|\\
|\providecommand{\version}{draft}|\\
|\||else|\\
|\providecommand{\version}{final}|\\
|\||fi|
\end{tabular}
\end{center}
%
The definition by |\providecommand| makes sure
that previous definitions are not overwritten.
Further statements |\providecommand{\version}{...}|
can thus be added before the above code to override it.

For the main file, one might add a line
(between |\childdocmain| and the above block)
%
\begin{center}
|%\ifchilddoc\||else\providecommand{\version}{draft}\||fi|
\end{center}
%
which can be uncommented to produce a draft version.
Likewise one can add a line to the very top of a child file
(above the |\childdocof{|\textit{main}|}| directive)
%
\begin{center}
|%\providecommand{\version}{final}|
\end{center}
%
which can be uncommented to produce the final version of this child document.

%%%%%%%%%%%%%%%%%%%%%%%%%%%%%%%%%%%%%%%%%%%%%%%%%%%%%%%%%%%%%%%%%%%%%%%%%%%%%%%%
\subsection{Forwarding}
\label{sec:forward}

Different versions of the main or child documents
using compilation flags as described in \secref{sec:flags}
can be (permanently) stored in different files
for convenient compilation, viewing and distribution.
To this end, the package defines a command
to pass on compilation to a different file:

%%%%%%%%%%%%%%%%%%%%%%%%%%%%%%%%%%%%%%%%
\DescribeMacro{\childdocforward}
The command |\childdocforward| redirects processing to
another source file:
%
\begin{center}
\begin{tabular}{l}
|% \iffalse
%
% childdoc.dtx Copyright (C) 2017-2018 Niklas Beisert
%
% This work may be distributed and/or modified under the
% conditions of the LaTeX Project Public License, either version 1.3
% of this license or (at your option) any later version.
% The latest version of this license is in
%   http://www.latex-project.org/lppl.txt
% and version 1.3 or later is part of all distributions of LaTeX
% version 2005/12/01 or later.
%
% This work has the LPPL maintenance status `maintained'.
%
% The Current Maintainer of this work is Niklas Beisert.
%
% This work consists of the files childdoc.dtx and childdoc.ins
% and the derived files childdoc.def and cdocsamp.tex with
% cdocsch1.tex, cdocsch2.tex, cdocsdrf.tex, cdocsfn1.tex, cdocsfn2.tex.
%
%<package>\ifdefined\childdocmain\endinput\fi
%<package>\ProvidesFile{childdoc.def}[2018/12/30 v2.0 child document driver]
%<samplemain>\ProvidesFile{cdocsamp.tex}[2018/12/30 v2.0 sample for childdoc]
%<*driver>
%\ProvidesFile{childdoc.drv}[2018/12/30 v2.0 childdoc reference manual file]
\PassOptionsToClass{10pt,a4paper}{article}
\documentclass{ltxdoc}

\usepackage[margin=35mm]{geometry}
\usepackage{hyperref}
\usepackage{hyperxmp}
\usepackage[usenames]{color}

\hypersetup{colorlinks=true}
\hypersetup{pdfstartview=FitH}
\hypersetup{pdfpagemode=UseNone}
\hypersetup{pdfsource={}}
\hypersetup{pdflang={en-UK}}
\hypersetup{pdfcopyright={Copyright 2017-2018 Niklas Beisert.
  This work may be distributed and/or modified under the
  conditions of the LaTeX Project Public License, either version 1.3
  of this license or (at your option) any later version.}}
\hypersetup{pdflicenseurl={http://www.latex-project.org/lppl.txt}}
\hypersetup{pdfcontactaddress={ETH Zurich, ITP, HIT K,
  Wolfgang-Pauli-Strasse 27}}
\hypersetup{pdfcontactpostcode={8093}}
\hypersetup{pdfcontactcity={Zurich}}
\hypersetup{pdfcontactcountry={Switzerland}}
\hypersetup{pdfcontactemail={nbeisert@itp.phys.ethz.ch}}
\hypersetup{pdfcontacturl={http://people.phys.ethz.ch/\xmptilde nbeisert/}}

\newcommand{\secref}[1]{\hyperref[#1]{section \ref*{#1}}}

\parskip1ex
\parindent0pt
\let\olditemize\itemize
\def\itemize{\olditemize\parskip0pt}

\begin{document}

\title{The \textsf{childdoc} Package}
\hypersetup{pdftitle={The childdoc Package}}
\author{Niklas Beisert\\[2ex]
  Institut f\"ur Theoretische Physik\\
  Eidgen\"ossische Technische Hochschule Z\"urich\\
  Wolfgang-Pauli-Strasse 27, 8093 Z\"urich, Switzerland\\[1ex]
  \href{mailto:nbeisert@itp.phys.ethz.ch}
  {\texttt{nbeisert@itp.phys.ethz.ch}}}
\hypersetup{pdfauthor={Niklas Beisert}}
\hypersetup{pdfsubject={Manual for the LaTeX2e Package childdoc}}
\date{30 December 2018, \textsf{v2.0}}
\maketitle

\begin{abstract}\noindent
\textsf{childdoc} is a \LaTeXe{} package
that enables the direct compilation
of document sections included by |\include|
to individual files.
\end{abstract}

\begingroup
\parskip0ex
\tableofcontents
\endgroup

%%%%%%%%%%%%%%%%%%%%%%%%%%%%%%%%%%%%%%%%%%%%%%%%%%%%%%%%%%%%%%%%%%%%%%%%%%%%%%%%
%%%%%%%%%%%%%%%%%%%%%%%%%%%%%%%%%%%%%%%%%%%%%%%%%%%%%%%%%%%%%%%%%%%%%%%%%%%%%%%%
\section{Introduction}

\LaTeX{} provides a mechanism to structure a large document (such as a book)
into a main file and several child files (containing the chapters)
using the |\include| command.
This mechanism is beneficial for documents
which span hundreds of pages in order to
make the source file(s) more manageable.
Moreover, compilation can be restricted to
selected child files by means of the |\includeonly| command.
The latter feature can be used to reduce the compilation time while editing
(this was significantly more useful in the earlier days of \LaTeX{})
or to generate a smaller document which is easier to navigate.
Another application of |\includeonly| is to generate
documents consisting of selected parts of the complete document.

However, there are a few drawbacks of the plain |\include| mechanism:
\begin{itemize}
\item
The child files cannot be compiled on their own,
they can only be compiled via the main file.
A naive editing environment
(such as a text editor with an option
to have the current file processed by \LaTeX)
may require one to switch to the main file before compiling;
attempting to compile the child file produces errors.
\item
The main file must be modified (each time)
to adjust the |\includeonly| command
to the present needs. This easily leaves the main file in a messy state.
\item
The generated document will always carry the filename
of the main document. This is inconvenient if
several child files are to be compiled and
to be kept for distribution.
\end{itemize}

The present package provides a simple interface
to make child files individually compilable by \LaTeX{}.
Compiling a child file then has the same effect as compiling
the main file with an |\includeonly| command
to select the appropriate child.
Moreover the generated document will carry the name of the child
rather than the main file.
This resolves all three above issues.

This feature is meant to make the editing of books,
thesis documents and lecture notes somewhat more convenient.
However, the package can also be used efficiently for
composing a series of documents (such as exercise sheets)
which are typically distributed individually.
It then assists the author in generating the individual documents
(potentially in different versions)
as well as a document containing the collected series.
Another application is in developing style files
or other kinds of included material
where compilation of the style file could redirect
to a sample or test file.

%%%%%%%%%%%%%%%%%%%%%%%%%%%%%%%%%%%%%%%%%%%%%%%%%%%%%%%%%%%%%%%%%%%%%%%%%%%%%%%%
%%%%%%%%%%%%%%%%%%%%%%%%%%%%%%%%%%%%%%%%%%%%%%%%%%%%%%%%%%%%%%%%%%%%%%%%%%%%%%%%
\section{Usage}

First of all, the package \textsf{childdoc} is \emph{not} a standard
\LaTeXe{} |.sty| style file! Therefore it needs to be invoked in
a non-standard way.

%%%%%%%%%%%%%%%%%%%%%%%%%%%%%%%%%%%%%%%%%%%%%%%%%%%%%%%%%%%%%%%%%%%%%%%%%%%%%%%%
\subsection{Included Files}
\label{sec:include}

%%%%%%%%%%%%%%%%%%%%%%%%%%%%%%%%%%%%%%%%
\DescribeMacro{\childdocmain}
To use the package, add the commands
\begin{center}
\begin{tabular}{l}
|\input{childdoc.def}|\\
|\childdocmain{}|\\
\end{tabular}
\end{center}
at the very top of the main \LaTeX{} file,
in particular \emph{before} the |\documentclass| statement!
The argument of |\childdocmain| should be left empty
(but it must be present).

%%%%%%%%%%%%%%%%%%%%%%%%%%%%%%%%%%%%%%%%
\DescribeMacro{\childdocof}
Furthermore, add the commands
\begin{center}
\begin{tabular}{l}
|\input{childdoc.def}|\\
|\childdocof{|\textit{main}|}|\\
\end{tabular}
\end{center}
at the top of every child file \textit{child}
which is included by |\include{|\textit{child}|}|
from within the main file
(or at least for those files to be compiled individually).
The argument \textit{main} must be the filename of the main file.

There are a couple of
considerations in setting up the main and child documents:

%%%%%%%%%%%%%%%%%%%%%%%%%%%%%%%%%%%%%%%%
\paragraph{Restrictions.}

Please note the following restrictions:
\begin{itemize}
\item
|\childdocmain| must be called with one argument \textit{main}
to ensure compatibility with earlier version of the package.
It must either be empty (|\childdocmain{}|)
or precisely match the filename of the main file in which it is specified.
See \secref{sec:detection} for further information.
\item
The filename \textit{main} must be specified without the |.tex| extension.
\item
The filename \textit{main} is case sensitive
(even in case-insensitive file systems)
due to internal string comparison.
\item
The argument \textit{main} should be fully expanded, it cannot be a macro.
\item
Subdirectories and special characters should be avoided in filenames.
\item
The command |\childdocmain{|\textit{main}|}| must be followed by a whitespace.
It should not be followed immediately by another command
or by a comment mark `|%|'.
This is because the \TeX{} parser reads the token immediately following
the argument of |\childdocmain| and puts it
at the beginning of every child section;
however, a white\-space is ignored.
\end{itemize}

%%%%%%%%%%%%%%%%%%%%%%%%%%%%%%%%%%%%%%%%
\paragraph{Content of Main File.}

It is advisable to place all content in the child files included by |\include|.
Any output contained in the main file will appear in all child documents
unless suppressed manually;
it cannot be suppressed automatically by the |\includeonly| directive
and thus should normally be avoided.
A method to include some content in the main file
by means of conditional processing is described in \secref{sec:conditional}.

%%%%%%%%%%%%%%%%%%%%%%%%%%%%%%%%%%%%%%%%
\paragraph{Page Numbering.}

When only a part of the document is compiled,
the appropriate numbering of pages
(as well as other status parameters)
is determined from the |.aux| files.
The latter contain information from previous passes.
However this information needs to propagate through
all intermediate child documents.
Therefore the page numbering in child documents may well
be inconsistent until the complete document is compiled at least once.

A useful (if unconventional) way to always ensure a consistent
page numbering is to restart the numbering in each child document
and denote the pages by `\textit{child}|.|\textit{page}'
where \textit{child} represents the chapter/section number of the child file.
This can be achieved by the command
|\numberwithin{page}{|\textit{child}|}|
of the \textsf{amsmath} package
where \textit{child} can be |chapter| or |section|
depending on the chosen structuring.
Alternatively, one can modify the macro |\thepage| appropriately
and reset the counter |page| at the start of each child file.

%%%%%%%%%%%%%%%%%%%%%%%%%%%%%%%%%%%%%%%%%%%%%%%%%%%%%%%%%%%%%%%%%%%%%%%%%%%%%%%%
\subsection{Conditional Processing}
\label{sec:conditional}

The package provides a mechanism to compile different versions
of a document. To customise the versions further some conditional processing
can come in handy to distinguish which version is being compiled.
The package provides two macros to describe the compilation context:

%%%%%%%%%%%%%%%%%%%%%%%%%%%%%%%%%%%%%%%%
\DescribeMacro{\ifchilddoc}
The conditional |\ifchilddoc| distinguishes between the compilation of
child documents and the main document:
%
\begin{center}
|\ifchilddoc |\textit{child-code}| |[|\||else |\textit{main-code}]| \||fi|
\end{center}

%%%%%%%%%%%%%%%%%%%%%%%%%%%%%%%%%%%%%%%%
\DescribeMacro{\childdocname}
\DescribeMacro{\childdocjob}
The macro |\childdocname| contains the filename (without extension)
of the main or child file being processed.
Note that |\childdocjob| will always contain the name of the main file.

%%%%%%%%%%%%%%%%%%%%%%%%%%%%%%%%%%%%%%%%
\paragraph{Title Page.}

Conditional processing can be used to include a title or banner page
in the main document when proper precautions are taken.
Importantly, the code in the main file should ensure that the page counter
(as well as other status parameters which are stored in the |.aux| files)
takes the same value after the conditional processing.
Otherwise the page numbers may take divergent values
depending on which part is compiled.

For example, a title page could be declared by:
%
\begin{center}
\begin{tabular}{l}
|\ifchilddoc\||else|\\
|\addtocounter{page}{-1}|\\
\textit{code for title page}\\
|\newpage|\\
|\||fi|
\end{tabular}
\end{center}
%
A banner page for the child documents can be generated by:
%
\begin{center}
\begin{tabular}{l}
|\ifchilddoc|\\
|\addtocounter{page}{-1}|\\
\textit{code for banner page}\\
|\newpage|\\
|\||fi|
\end{tabular}
\end{center}
%
Here one could write a message such as:
\begin{center}
|This is the part \childdocname{} of \childdocjob{}.|
\end{center}

%%%%%%%%%%%%%%%%%%%%%%%%%%%%%%%%%%%%%%%%%%%%%%%%%%%%%%%%%%%%%%%%%%%%%%%%%%%%%%%%
\subsection{Flags}
\label{sec:flags}

The package makes it easy to generate different versions
of the main or child documents.
To this end compilation flags can be defined
and assigned different default values.
They will be particularly useful in conjunction
with the forwarding mechanism described in \secref{sec:forward}.

For example, it may be useful to have a flag |\version|
which can be set to |draft| or |final|.
The document source will contain some conditional code
depending on the value of |\version|.
Suppose further, the flag should default to |final| for the main file
and to |draft| for child files
which is a natural assignment for editing the document.
This is achieved by placing the following code
in the preamble of the main document
(below the |\childdocmain| directive):
%
\begin{center}
\begin{tabular}{l}
|\ifchilddoc|\\
|\providecommand{\version}{draft}|\\
|\||else|\\
|\providecommand{\version}{final}|\\
|\||fi|
\end{tabular}
\end{center}
%
The definition by |\providecommand| makes sure
that previous definitions are not overwritten.
Further statements |\providecommand{\version}{...}|
can thus be added before the above code to override it.

For the main file, one might add a line
(between |\childdocmain| and the above block)
%
\begin{center}
|%\ifchilddoc\||else\providecommand{\version}{draft}\||fi|
\end{center}
%
which can be uncommented to produce a draft version.
Likewise one can add a line to the very top of a child file
(above the |\childdocof{|\textit{main}|}| directive)
%
\begin{center}
|%\providecommand{\version}{final}|
\end{center}
%
which can be uncommented to produce the final version of this child document.

%%%%%%%%%%%%%%%%%%%%%%%%%%%%%%%%%%%%%%%%%%%%%%%%%%%%%%%%%%%%%%%%%%%%%%%%%%%%%%%%
\subsection{Forwarding}
\label{sec:forward}

Different versions of the main or child documents
using compilation flags as described in \secref{sec:flags}
can be (permanently) stored in different files
for convenient compilation, viewing and distribution.
To this end, the package defines a command
to pass on compilation to a different file:

%%%%%%%%%%%%%%%%%%%%%%%%%%%%%%%%%%%%%%%%
\DescribeMacro{\childdocforward}
The command |\childdocforward| redirects processing to
another source file:
%
\begin{center}
\begin{tabular}{l}
|\input{childdoc.def}|\\
|\childdocforward[|\textit{main}|]{|\textit{dest}|}|\\
\end{tabular}
\end{center}
%
The argument \textit{dest} is the destination file
(without extension).
It should be the main file or one of the child files.
Note that further \textsf{childdoc} directives
such as |\childdocof| and |\childdocforward|
in the indicated file will be processed in this form.
The optional argument \textit{main}
passes on directly to the main file \textit{main}
while pretending to compile the child \textit{dest}.
This form behaves as if \textit{dest}
issues |\childdocof{|\textit{main}|}| right away,
and no further \textsf{childdoc} directives will be processed.

%%%%%%%%%%%%%%%%%%%%%%%%%%%%%%%%%%%%%%%%
\DescribeMacro{\...prefix}
In the alternative form |\childdocforwardprefix|,
%
\begin{center}
\begin{tabular}{l}
|\input{childdoc.def}|\\
|\childdocforwardprefix[|\textit{main}|]{|\textit{prefix}|}{|\textit{dest}|}|
\end{tabular}
\end{center}
%
the destination file is determined by a pattern
depending on the current file:
To make this work, the current file must be called
`{\textit{prefix}\hspace{0.2em}\textit{suffix}}'
with \textit{prefix} matching precisely the argument.
Processing is then passed on to the file
`{\textit{dest}\hspace{0.2em}\textit{suffix}}'.
Surely, the same effect is achieved by
directly specifying the
argument `{\textit{dest}\hspace{0.2em}\textit{suffix}}'
in the first form.
However, that requires to set up a different file
for each child. With the alternative form of the command
all these files can have exactly the same content
which simplifies setting them up and maintaining them.

For example, the following file |draft.tex|
with a compilation flag |\version| as described in \secref{sec:flags}
compiles the main document as a draft:
%
\begin{center}
\begin{tabular}{l}
|\def\version{draft}|\\
|\input{childdoc.def}|\\
|\childdocforward{|\textit{main}|}|
\end{tabular}
\end{center}
%
Likewise, the following files |final|\textit{nn}|.tex|
compile the final version of the child document
|child|\textit{nn}|.tex|:
%
\begin{center}
\begin{tabular}{l}
|\def\version{final}|\\
|\input{childdoc.def}|\\
|\childdocforwardprefix{final}{child}|
\end{tabular}
\end{center}
%

Note that when several versions of a main file and/or of each child file
are to be generated, it may be convenient to set up a |Makefile| or
shell script to automatise the process.

%%%%%%%%%%%%%%%%%%%%%%%%%%%%%%%%%%%%%%%%%%%%%%%%%%%%%%%%%%%%%%%%%%%%%%%%%%%%%%%%
\subsection{Command Line Processing}
\label{sec:commandline}

The effect of redirection files can also be achieved by invoking
the \LaTeX{} compiler with a more elaborate command line.
Most conveniently this should be done as part
of a shell script or a |Makefile|.

When using \textsf{childdoc} in the main file, the following
command lines effectively perform a redirection
(note that depending on the shell being used,
backslashes may have to be doubled: `|\|' $\to$ `|\\|'):
%
\begin{center}
|... -jobname "|\textit{target}|" |\\|"|[\textit{flags}]%
|\input{childdoc.def}\childdocforward[|\textit{main}|]{|\textit{dest}|}"|
\end{center}
%
Here \textit{target} is the name of the output file,
\textit{main} is the name of the main file
and \textit{dest} is the name of the main or child file to be processed
(all filenames without extensions).
The optional argument \textit{main} can be omitted
if \textit{main} matches \textit{dest}.
Optionally, compilation \textit{flags} can be defined via |\def| commands.
This command line makes the \TeX{} engine believe
it is compiling the file \textit{target}
whose content is specified as the latter parameter.
The provided code then forwards the processing to
\textit{main} or \textit{dest} as described in \secref{sec:forward}.

%%%%%%%%%%%%%%%%%%%%%%%%%%%%%%%%%%%%%%%%%%%%%%%%%%%%%%%%%%%%%%%%%%%%%%%%%%%%%%%%
\subsection{Include by Input}
\label{sec:input}

Including child documents by |\include| has some restrictions by design.
Most notably, the content of a child document always occupies
its own set of pages; pages cannot be shared between child documents.
Usually, this behaviour makes perfect sense
because each child document contain an essential part of the document.
However, in some situations it may be desirable to compose
a document from a collection of parts
without having mandatory page breaks between then.
For this case, the package
provides a mechanism to include parts
by |\input| which can also be processed individually.
However, by construction this mechanism
requires manual handling of the content to be output.

%%%%%%%%%%%%%%%%%%%%%%%%%%%%%%%%%%%%%%%%
\DescribeMacro{\ifchilddocmanual}
The main file should be prepared as usual, see \secref{sec:include}.
However, the document body must make a distinction
between processing of an individual part and of the main document, e.g.:
%
\begin{center}
\begin{tabular}{l}
|\ifchilddocmanual|\\
|\input{\childdocname}|\\
|\||else|\\
\textit{document body with }|\input{|\textit{part}|}|\\
|\||fi|
\end{tabular}
\end{center}
%
The conditional |\ifchilddocmanual| is true whenever
a part to be included by |\input| is being compiled,
and the name of the part is stored in |\childdocname|.

%%%%%%%%%%%%%%%%%%%%%%%%%%%%%%%%%%%%%%%%
\DescribeMacro{\childdocby}
Each part to be included by |\input| should start with:
%
\begin{center}
\begin{tabular}{l}
|\input{childdoc.def}|\\
|\childdocby{|\textit{main}|}|\\
\end{tabular}
\end{center}
%
The directive |\childdocby| is similar to |\childdocof|
described in \secref{sec:include},
but the subsequent selection of content must be done manually.
To that end, both |\ifchilddoc| and |\ifchilddocmanual|
will be true upon processing of a part,
and the name of the part is stored in |\childdocname|.
Note that |\jobname| will be set to the filename of the current part
so that each part receives an individual |.aux| file
that does not interfere with the |.aux| file(s) of the main document.
This behaviour can be altered by the alternative form
|\childdocby[*]{|\textit{main}|}| (with a non-empty optional argument)
which uses the |.aux| file of the main document
by setting |\jobname| to \textit{main}.

%%%%%%%%%%%%%%%%%%%%%%%%%%%%%%%%%%%%%%%%%%%%%%%%%%%%%%%%%%%%%%%%%%%%%%%%%%%%%%%%
\subsection{Driver Development}
\label{sec:driver}

The \textsf{childdoc} mechanism can also be use for the development
of definition files such as \LaTeX{} styles or classes.
This case differs from the above setup with multiple parts
included by |\include| in that no |\includeonly| should be invoked.
This can be achieved by starting the include file
(before |\ProvidesPackage|) with:
%
\begin{center}
\begin{tabular}{l}
|\input{childdoc.def}|\\
|\childdocforward{|\textit{main}|}|\\
\end{tabular}
\end{center}
%
or alternatively with:
%
\begin{center}
\begin{tabular}{l}
|\input{childdoc.def}|\\
|\childdocby{|\textit{main}|}|\\
\end{tabular}
\end{center}
%
Both forms have slightly different effects as described above.
The main file is prepared as usual, see \secref{sec:include}.

%%%%%%%%%%%%%%%%%%%%%%%%%%%%%%%%%%%%%%%%%%%%%%%%%%%%%%%%%%%%%%%%%%%%%%%%%%%%%%%%
\subsection{Legacy Detection}
\label{sec:detection}

The directive |\childdocmain| in the main file can detect
whether the complete document or merely a child is to be compiled
even without using the directive |\childdocof|.
This method is deprecated because it is less robust
and there is no compelling reason to use it;
it is merely provided for backward compatibility
and it may be removed in future versions.

If the detection mechanism is to be used,
it is mandatory to correctly specify
the filename of the main file as the argument of |\childdocmain|:
%
\begin{center}
\begin{tabular}{l}
|\input{childdoc.def}|\\
|\childdocmain{|\textit{main}|}|\\
\end{tabular}
\end{center}
%
If |\jobname| does not match the argument \textit{main} of |\childdocmain|,
it is assumed that |\jobname| points to the child file to be compiled.
When using |\childdocmain| with the main file specified as argument,
it suffices to start a child file
with just |\input{|\textit{main}|}|
without loading of the package and using |\childdocof|.
If instead all processing is done
with the appropriate \textsf{childdoc} directives,
the argument of \textit{main} of |\childdocmain| can be empty.

An alternative version of the command line processing described
in \secref{sec:commandline} using the detection mechanism reads:
%
\begin{center}
|... -jobname "|\textit{target}|" "|[\textit{flags}]%
[|\def\jobname{|\textit{dest}|}|]|\input{|\textit{main}|}"|
\end{center}

%%%%%%%%%%%%%%%%%%%%%%%%%%%%%%%%%%%%%%%%%%%%%%%%%%%%%%%%%%%%%%%%%%%%%%%%%%%%%%%%
\subsection{Manual Code}
\label{sec:manual}

In case one cannot be certain whether the definitions file |childdoc.def|
is installed on the target \TeX{} distribution
and one prefers not to ship it,
it is conceivable to paste a few relevant commands into the sources.

To that end, drop all statements |\input{childdoc.def}|
and perform the replacements as outlined below.
Instead of |\childdocmain{|\textit{main}|}| add the following code
to the top of the main file:
%
\begin{center}
\begin{tabular}{l}
|\||ifdefined\childdocname\endinput\||fi\newif\ifchilddoc|\\
|\edef\childdocname{\scantokens\expandafter{\jobname\noexpand}}|\\
|\def\childdocmain{|\textit{main}|}\||ifx\childdocmain\childdocname\||else|\\
|\childdoctrue\includeonly{\childdocname}\let\jobname\childdocmain\||fi|\\
\end{tabular}
\end{center}
%
Instead of |\childdocof{|\textit{main}|}| just include the main file
at the top of each child file:
%
\begin{center}
|\input{|\textit{main}|}|
\end{center}
%
A simple redirection |\childdocforward{|\textit{dest}|}| is achieved by:
%
\begin{center}
|\def\jobname{|\textit{dest}|}\input{\jobname}|
\end{center}
%
The redirection with prefix
|\childdocforwardprefix[|\textit{prefix}|]{|\textit{dest}|}|
is accomplished by:
%
\begin{center}
\begin{tabular}{l}
|{\edef\jobname{\scantokens\expandafter{\jobname\noexpand}}|\\
|\def\redirectjob |\textit{prefix}|#1~~~{\gdef\jobname{|\textit{dest}|#1}}|\\
|\expandafter\redirectjob\jobname~~~}\input{\jobname}|
\end{tabular}
\end{center}

In an alternative approach,
child documents can be compiled by a specific command line
without additional code or specific definitions:
%
\begin{center}
|... -jobname "|\textit{target}|" "|[\textit{flags}]%
|\includeonly{|\textit{dest}|}\input{|\textit{main}|}"|
\end{center}
%

%%%%%%%%%%%%%%%%%%%%%%%%%%%%%%%%%%%%%%%%%%%%%%%%%%%%%%%%%%%%%%%%%%%%%%%%%%%%%%%%
%%%%%%%%%%%%%%%%%%%%%%%%%%%%%%%%%%%%%%%%%%%%%%%%%%%%%%%%%%%%%%%%%%%%%%%%%%%%%%%%
\section{Information}

%%%%%%%%%%%%%%%%%%%%%%%%%%%%%%%%%%%%%%%%%%%%%%%%%%%%%%%%%%%%%%%%%%%%%%%%%%%%%%%%
\subsection{Copyright}

Copyright \copyright{} 2017--2018 Niklas Beisert

This work may be distributed and/or modified under the
conditions of the \LaTeX{} Project Public License, either version 1.3
of this license or (at your option) any later version.
The latest version of this license is in
  \url{http://www.latex-project.org/lppl.txt}
and version 1.3 or later is part of all distributions of \LaTeX{}
version 2005/12/01 or later.

This work has the LPPL maintenance status `maintained'.

The Current Maintainer of this work is Niklas Beisert.

This work consists of the files |README.txt|, |childdoc.ins| and |childdoc.dtx|
as well as the derived files |childdoc.def|, |cdocsamp.tex|
with |cdocsch1.tex|, |cdocsch2.tex|, |cdocspt3.tex|, |cdocspt4.tex|,
|cdocsdrf.tex|, |cdocsfn1.tex|, |cdocsfn2.tex|
as well as |childdoc.pdf|.

%%%%%%%%%%%%%%%%%%%%%%%%%%%%%%%%%%%%%%%%%%%%%%%%%%%%%%%%%%%%%%%%%%%%%%%%%%%%%%%%
\subsection{Files and Installation}

The package consists of the files:
%
\begin{center}
\begin{tabular}{ll}
    |README.txt|   & readme file \\
    |childdoc.ins| & installation file \\
    |childdoc.dtx| & source file \\
    |childdoc.def| & definition file \\
    |cdocsamp.tex| & sample main file \\
    |cdocsch1.tex| & sample include file \\
    |cdocsch2.tex| & sample include file \\
    |cdocspt3.tex| & sample part file \\
    |cdocspt4.tex| & sample part file \\
    |cdocsdrf.tex| & sample redirection file \\
    |cdocsfn1.tex| & sample redirection file \\
    |cdocsfn2.tex| & sample redirection file \\
    |childdoc.pdf| & manual
\end{tabular}
\end{center}
%
The distribution consists of the files
|README.txt|, |childdoc.ins| and |childdoc.dtx|.
%
\begin{itemize}
\item
Run (pdf)\LaTeX{} on |childdoc.dtx|
to compile the manual |childdoc.pdf| (this file).
\item
Run \LaTeX{} on |childdoc.ins| to create the definitions file |childdoc.def|
and the sample |cdocsamp.tex| with include files
|cdocsch1.tex|, |cdocsch2.tex|, |cdocspt3.tex|, |cdocspt4.tex|,
|cdocsdrf.tex|, |cdocsfn1.tex|, |cdocsfn2.tex|.
Then copy the file |childdoc.def| to an appropriate directory of your \LaTeX{}
distribution, e.g.\ \textit{texmf-root}|/tex/latex/childdoc|.
\end{itemize}

%%%%%%%%%%%%%%%%%%%%%%%%%%%%%%%%%%%%%%%%%%%%%%%%%%%%%%%%%%%%%%%%%%%%%%%%%%%%%%%%
\subsection{Related CTAN Packages}

There are several other packages which offer a similar functionality:
%
\begin{itemize}
\item
The packages
\href{http://ctan.org/pkg/docmute}{\textsf{docmute}},
\href{http://ctan.org/pkg/includex}{\textsf{includex}} and
\href{http://ctan.org/pkg/standalone}{\textsf{standalone}}
provide commands to include only the document body of
a child file thus allowing both files to be compiled individually.
\item
The packages \href{http://ctan.org/pkg/subdocs}{\textsf{subdocs}}
and \href{http://ctan.org/pkg/subfiles}{\textsf{subfiles}}
provide structures in which the main and child documents can be
encapsulated and allowing them to be compiled individually.
The inclusion mechanism is different from the conventional |\include|.
\item
The package \href{http://ctan.org/pkg/combine}{\textsf{combine}}
is an elaborate solution to combine several documents into one.
\end{itemize}
%
See also the CTAN topic \href{http://ctan.org/topic/subdocs}{\textsf{subdocs}}
for further related packages.
The present package differs from the above solutions in that
a document structure constructed with the conventional |\include| mechanism
just needs two extra commands at the top of every file
such that all constituent files can be compiled individually.

%%%%%%%%%%%%%%%%%%%%%%%%%%%%%%%%%%%%%%%%%%%%%%%%%%%%%%%%%%%%%%%%%%%%%%%%%%%%%%%%
%\subsection{Feature Suggestions}
%
%The following is a list of features which may be useful for future
%versions of this package:
%%
%\begin{itemize}
%\item
%\ldots
%\end{itemize}

%%%%%%%%%%%%%%%%%%%%%%%%%%%%%%%%%%%%%%%%%%%%%%%%%%%%%%%%%%%%%%%%%%%%%%%%%%%%%%%%
\subsection{Revision History}

%%%%%%%%%%%%%%%%%%%%%%%%%%%%%%%%%%%%%%%%
\paragraph{v2.0:} 2018/12/30

\begin{itemize}
\item
immediate forward processing
\item
added |\childdocby| mechanism
\item
manual restructured
\end{itemize}

%%%%%%%%%%%%%%%%%%%%%%%%%%%%%%%%%%%%%%%%
\paragraph{v1.6:} 2018/01/17

\begin{itemize}
\item
application for development of include files
\item
corrections to manual
\end{itemize}

%%%%%%%%%%%%%%%%%%%%%%%%%%%%%%%%%%%%%%%%
\paragraph{v1.5:} 2017/05/21

\begin{itemize}
\item
more complete structuring introduced
\item
|\childdocof| introduced
\item
|\childdoc| renamed to |\childdocmain|
\item
|\childredirect| renamed to |\childdocforward| and |\childdocforwardprefix|
and functionality expanded
\end{itemize}

%%%%%%%%%%%%%%%%%%%%%%%%%%%%%%%%%%%%%%%%
\paragraph{v1.0:} 2017/04/27

\begin{itemize}
\item
manual and install package
\item
first version published on CTAN
\end{itemize}

%%%%%%%%%%%%%%%%%%%%%%%%%%%%%%%%%%%%%%%%
\paragraph{v0.6:} 2017/04/26

\begin{itemize}
\item
redirection mechanism added
\end{itemize}

%%%%%%%%%%%%%%%%%%%%%%%%%%%%%%%%%%%%%%%%
\paragraph{v0.5:} 2017/04/26

\begin{itemize}
\item
functionality in definition file
\end{itemize}


%%%%%%%%%%%%%%%%%%%%%%%%%%%%%%%%%%%%%%%%%%%%%%%%%%%%%%%%%%%%%%%%%%%%%%%%%%%%%%%%
%%%%%%%%%%%%%%%%%%%%%%%%%%%%%%%%%%%%%%%%%%%%%%%%%%%%%%%%%%%%%%%%%%%%%%%%%%%%%%%%
%%%%%%%%%%%%%%%%%%%%%%%%%%%%%%%%%%%%%%%%%%%%%%%%%%%%%%%%%%%%%%%%%%%%%%%%%%%%%%%%
\appendix

\settowidth\MacroIndent{\rmfamily\scriptsize 000\ }

 \DocInput{childdoc.dtx}

\end{document}
%</driver>
% \fi
%
% %%%%%%%%%%%%%%%%%%%%%%%%%%%%%%%%%%%%%%%%%%%%%%%%%%%%%%%%%%%%%%%%%%%%%%%%%%%%%%
% %%%%%%%%%%%%%%%%%%%%%%%%%%%%%%%%%%%%%%%%%%%%%%%%%%%%%%%%%%%%%%%%%%%%%%%%%%%%%%
% \section{Sample}
%\iffalse
%<*samplemain>
%\fi
%
% The following presents a sample document
% with two chapters, two parts, a title page,
% a compile flag as well as three forwarding files to set the flag.
% It consists of eight |.tex| files:
% \begin{center}
% \begin{tabular}{ll}
% |cdocsamp.tex|&main file\\
% |cdocsch1.tex|&include file for chapter 1\\
% |cdocsch2.tex|&include file for chapter 2\\
% |cdocspt3.tex|&include file for part 3\\
% |cdocspt4.tex|&include file for part 4\\
% |cdocsdrf.tex|&forwarding file for main file in draft mode\\
% |cdocsfi1.tex|&forwarding file for final version of chapter 1\\
% |cdocsfi2.tex|&forwarding file for final version of chapter 2\\
% \end{tabular}
% \end{center}
% Each of the eight files can be compiled directly by the \LaTeX{} compiler.
%
% %%%%%%%%%%%%%%%%%%%%%%%%%%%%%%%%%%%%%%
% \paragraph{Main File.}
%
% The main file is called |cdocsamp.tex|.
%
% Load the \textsf{childdoc} definitions and
% declare the filename for the main document:
%    \begin{macrocode}
\input{childdoc.def}
\childdocmain{}
%    \end{macrocode}

% Optional override for |\version| flag:
%    \begin{macrocode}
%%\ifchilddoc\else\providecommand{\version}{draft}\fi
%    \end{macrocode}

% Define the default values for the |\version| flag
% (|final| for the main file and |draft| for childs):
%    \begin{macrocode}
\ifchilddoc
\providecommand{\version}{draft}
\else
\providecommand{\version}{final}
\fi
%    \end{macrocode}

% Load the standard document class:
%    \begin{macrocode}
\documentclass[12pt]{article}
%    \end{macrocode}

% Start the document body:
%    \begin{macrocode}
\begin{document}
%    \end{macrocode}

% Declare a title page.
% Print title, part of document being processed and version flag:
%    \begin{macrocode}
\addtocounter{page}{-1}
\begin{center}
{\LARGE\bfseries{}childdoc example\par}
\vspace{1cm}
\ifchilddoc
\ifchilddocmanual part\else chapter\fi:
`\childdocname' of `\childdocjob'\par
\else
main document: `\childdocjob'\par
\fi
version: \version\par
\end{center}
\newpage
%    \end{macrocode}

% Manually include selected file,
% otherwise process as usual:
%    \begin{macrocode}
\ifchilddocmanual
\section*{part `\childdocname'}
\input{\childdocname}
\else
%    \end{macrocode}

% Include the two chapters:
%    \begin{macrocode}
\include{cdocsch1}
\include{cdocsch2}
%    \end{macrocode}

% Include the two parts unless only chapters should be displayed:
%    \begin{macrocode}
\ifchilddoc\else
\section{part three}
\input{cdocspt3}
\section{part four}
\input{cdocspt4}
\fi
%    \end{macrocode}

% Process as usual until here:
%    \begin{macrocode}
\fi
%    \end{macrocode}

% End of document body:
%    \begin{macrocode}
\end{document}
%    \end{macrocode}
%\iffalse
%</samplemain>
%\fi
%
% %%%%%%%%%%%%%%%%%%%%%%%%%%%%%%%%%%%%%%
% \paragraph{Chapter Include Files.}
%
% The include files are called |cdocsch1.tex| and |cdocsch2.tex|.
%
%\iffalse
%<*samplechap1|samplechap2>
%\fi

% Optional override for |\version| flag:
%    \begin{macrocode}
%%\providecommand{\version}{final}
%    \end{macrocode}

% Include the main document:
%    \begin{macrocode}
\input{childdoc.def}
\childdocof{cdocsamp}
%    \end{macrocode}

%\iffalse
%</samplechap1|samplechap2>
%\fi
%
%\iffalse
%<*samplechap1>
%\fi
% Some text for chapter 1:
%    \begin{macrocode}
\section{one}
some text in chapter one
%    \end{macrocode}

%\iffalse
%</samplechap1>
%\fi
% Some text for chapter 2:
%\iffalse
%<*samplechap2>
%\fi
%    \begin{macrocode}
\section{two}
more text in chapter two
%    \end{macrocode}

%\iffalse
%</samplechap2>
%\fi
%
% %%%%%%%%%%%%%%%%%%%%%%%%%%%%%%%%%%%%%%
% \paragraph{Part Include Files.}
%
% The include files are called |cdocspt3.tex| and |cdocspt4.tex|.
%
%\iffalse
%<*samplepart3|samplepart4>
%\fi

% Optional override for |\version| flag:
%    \begin{macrocode}
%%\providecommand{\version}{final}
%    \end{macrocode}

% Include the main document:
%    \begin{macrocode}
\input{childdoc.def}
\childdocby{cdocsamp}
%    \end{macrocode}

%\iffalse
%</samplepart3|samplepart4>
%\fi
%
%\iffalse
%<*samplepart3>
%\fi
% Some text for part 3:
%    \begin{macrocode}
some text in part three
%    \end{macrocode}

%\iffalse
%</samplepart3>
%\fi
% Some text for part 4:
%\iffalse
%<*samplepart4>
%\fi
%    \begin{macrocode}
more text in part four
%    \end{macrocode}

%\iffalse
%</samplepart4>
%\fi
%
% %%%%%%%%%%%%%%%%%%%%%%%%%%%%%%%%%%%%%%
% \paragraph{Forwarding for a Complete Draft.}
%
% The following forwarding file |cdocsdrf.tex|
% compiles the main document in draft mode:
%\iffalse
%<*sampledraft>
%\fi
%    \begin{macrocode}
\def\version{draft}
\input{childdoc.def}
\childdocforward{cdocsamp}
%    \end{macrocode}

%\iffalse
%</sampledraft>
%\fi
%
% %%%%%%%%%%%%%%%%%%%%%%%%%%%%%%%%%%%%%%
% \paragraph{Forwarding for Final Version of the Chapters.}
%
% The following forwarding files |cdocsfn1.tex| and |cdocsfn2.tex|
% (with identical content)
% compile the final versions of the child documents
% |cdocsch1.tex| and |cdocsch2.tex|, respectively:
%\iffalse
%<*samplefinal>
%\fi
%    \begin{macrocode}
\def\version{final}
\input{childdoc.def}
\childdocforwardprefix[cdocsamp]{cdocsfn}{cdocsch}
%    \end{macrocode}

%\iffalse
%</samplefinal>
%\fi
%
% %%%%%%%%%%%%%%%%%%%%%%%%%%%%%%%%%%%%%%
% \paragraph{Command Line Processing.}
%
% The following three command lines generate the output files
% |cdocscld|, |cdocscl1| and |cdocscl2|
% which should be identical to
% |cdocsdrf|, |cdocsch1| and |cdocsfn2|, respectively:
% \begin{center}
% \begin{tabular}{l}
% |latex -jobname cdocscld \|\\
% |  "\def\version{draft}\input{childdoc.def}\childdocforward{cdocsamp}"|\\
% |latex -jobname cdocscl1 \|\\
% |  "\input{childdoc.def}\childdocforward[cdocsamp]{cdocsch1}"|\\
% |latex -jobname cdocscl2 \|\\
% |  "\def\version{final}\input{childdoc.def}\childdocforward{cdocsch2}"|
% \end{tabular}
% \end{center}
% Note that the trailing backslash on each first line
% merely continues the input to the second line
% (for convenient cut ant paste).
% Furthermore, the command |latex| can be replaced by any
% of its alternative versions such as |pdflatex|.
%
% %%%%%%%%%%%%%%%%%%%%%%%%%%%%%%%%%%%%%%%%%%%%%%%%%%%%%%%%%%%%%%%%%%%%%%%%%%%%%%
% %%%%%%%%%%%%%%%%%%%%%%%%%%%%%%%%%%%%%%%%%%%%%%%%%%%%%%%%%%%%%%%%%%%%%%%%%%%%%%
% \section{Implementation}
%\iffalse
%<*package>
%\fi
%
% This section describes the definitions file |childdoc.def|.

% The definitions cannot be loaded using |\usepackage| or |\RequirePackage|
% which has a mechanism to prevent loading a style file more than once.
% When loading the definitions by means of |\input|
% multiple instances have to be prevented manually:
%\iffalse
%This code needs to be before the `\ProvidesFile' directive
%which is defined at the beginning of this file.
%Therefore it is also placed there and commented out here.
%</package>
%<*discard>
%\fi
%    \begin{macrocode}
\ifdefined\childdocmain\endinput\fi
%    \end{macrocode}
%\iffalse
%</discard>
%<*package>
%\fi
%
% \macro{\ifchilddoc}
% \macro{\ifchilddocmanual}
% The conditional |\ifchilddoc| tells whether a
% child (true) or main (false) document is being compiled.
% The conditional |\ifchilddocmanual| tells whether
% the |\includeonly| mechanism is used (false) or
% the selection of child files must be performed manually (true).
% The definitions initialise to false:
%    \begin{macrocode}
\newif\ifchilddoc
\newif\ifchilddocmanual
%    \end{macrocode}

% \macro{\childdocname}
% \macro{\childdocjob}
% The macro |\childdocname| stores the name of the main document
% to be compiled. The macro |\childdocjob| stores the name of
% the document on which the \LaTeX{} compiler was originally invoked.
% The content of |\jobname| cannot be compared
% to filenames specified in the source due to different catcodes.
% The following code rescans |\jobname|, stores the result
% in |\childdocname| and saves a copy in |\childdocjob|:
%    \begin{macrocode}
\edef\childdocname{\scantokens\expandafter{\jobname\noexpand}}
\let\childdocjob\childdocname
%    \end{macrocode}

% \macro{\childdocdisable}
% The macro |\childdocdisable| prevents the main file
% from being processed more than once.
% At this stage, the main document command |\childdocmain|
% is assumed to be called once again where it should do nothing.
% Any subsequent call to it should prevent
% a secondary processing of the main document
% It overwrites the forwarding commands
% |\childdocof| and |\childdocforward|
% with empty macros to prevent further inclusions of the main document:
%    \begin{macrocode}
\newcommand{\childdocdisable}
{
  \renewcommand{\childdocmain}[1]{\renewcommand{\childdocmain}[1]{\endinput}}
  \renewcommand{\childdocof}[1]{}
  \renewcommand{\childdocby}[2][]{}
  \renewcommand{\childdocforward}[2][]{}
  \renewcommand{\childdocdisable}{}
}
%    \end{macrocode}

% \macro{\childdocmain}
% The macro |\childdocmain| is to be called at the top of the main file
% with nothing or the main filename (without extension) as argument.
% First, it breaks loops.
% If the argument is not empty and does not match |\childdocname|
% (which is set by the first inclusion of |childdoc.def|),
% |\ifchilddoc| is set to true, |\includeonly| is applied to the child file
% and |\jobname| is set to the main file
% (for proper handling of |.aux| files):
%    \begin{macrocode}
\newcommand{\childdocmain}[1]
{
  \childdocdisable\childdocmain{}
  \if?#1?\else
    \begingroup
      \def\childdoctmp{#1}
      \ifx\childdoctmp\childdocname
        \def\childdoctmp{}
      \else
        \def\childdoctmp
        {
          \childdoctrue
          \includeonly{\childdocname}
          \def\childdocjob{#1}
          \def\jobname{#1}
        }
      \fi
      \expandafter
    \endgroup
    \childdoctmp
  \fi
}
%    \end{macrocode}

% \macro{\childdocof}
% The command |\childdocof| redirects
% compilation to the main file |#1|.
%    \begin{macrocode}
\newcommand{\childdocof}[1]
{
  \childdocdisable
  \childdoctrue
  \includeonly{\childdocname}
  \def\jobname{#1}
  \def\childdocjob{#1}
  \input{#1}
}
%    \end{macrocode}

% \macro{\childdocby}
% The command |\childdocby| ....
%    \begin{macrocode}
\newcommand{\childdocby}[2][]
{
  \childdocdisable
  \childdoctrue
  \childdocmanualtrue
  \if?#1?\else
    \def\jobname{#2}
  \fi
  \def\childdocjob{#2}
  \input{#2}
  \endinput
}
%    \end{macrocode}

% \macro{\childdocforward}
% The command |\childdocforward| redirects
% compilation to the main file or
% (if the optional argument is given) a child file.
% Parameters are set as if the main file
% or a child file starting with |\childdocof| was compiled.
% Then compilation is handed over to the main file:
%    \begin{macrocode}
\newcommand{\childdocforward}[2][]
{
  \begingroup
    \if?#1?
      \def\childdoctmp
      {
        \def\childdocname{#2}
        \def\childdocjob{#2}
        \def\jobname{#2}
        \input{#2}
        \endinput
      }
    \else
      \def\childdoctmp
      {
        \childdocdisable
        \def\childdocname{#2}
        \childdoctrue
        \includeonly{#2}
        \def\childdocjob{#1}
        \def\jobname{#1}
        \input{#1}
        \endinput
      }
    \fi
    \expandafter
  \endgroup
  \childdoctmp
}
%    \end{macrocode}

% \macro{\childdocforwardprefix}
% The command |\childdocforwardprefix| redirects
% compilation to the main or a child file by means of a pattern.
% The prefix |#1| in the current filename is replaced by |#2|
% and the suffix of the current filename is kept
% (it is assumed that the filename does not contain the substring `|~~~|'
% which is used as a delimiter).
% Compilation is handed over to the new file by |\childdocforward|:
%    \begin{macrocode}
\newcommand{\childdocforwardprefix}[3][]
{
  \begingroup
    \def\childdocextract #2##1~~~{\def\childdoctmp{\childdocforward[#1]{#3##1}}}
    \expandafter\childdocextract\childdocname~~~
    \expandafter
  \endgroup
  \childdoctmp
}
%    \end{macrocode}

% \macro{\childdoc}
% The deprecated macro |\childdoc| is a legacy version of |\childdocmain|:
%    \begin{macrocode}
\newcommand{\childdoc}{\childdocmain}
%    \end{macrocode}

% \macro{\childdocredirect}
% The deprecated macro |\childdocredirect| is a legacy version
% of |\childdocforward| and |\childdocforwardprefix|:
%    \begin{macrocode}
\newcommand{\childdocredirect}[2][]
{
  \begingroup
    \if?#1?
      \def\childdoctmp{\childdocforward{#2}}
    \else
      \def\childdoctmp{\childdocforwardprefix{#1}{#2}}
    \fi
    \expandafter
  \endgroup
  \childdoctmp
}
%    \end{macrocode}

%\iffalse
%</package>
%\fi
%
\endinput
|\\
|\childdocforward[|\textit{main}|]{|\textit{dest}|}|\\
\end{tabular}
\end{center}
%
The argument \textit{dest} is the destination file
(without extension).
It should be the main file or one of the child files.
Note that further \textsf{childdoc} directives
such as |\childdocof| and |\childdocforward|
in the indicated file will be processed in this form.
The optional argument \textit{main}
passes on directly to the main file \textit{main}
while pretending to compile the child \textit{dest}.
This form behaves as if \textit{dest}
issues |\childdocof{|\textit{main}|}| right away,
and no further \textsf{childdoc} directives will be processed.

%%%%%%%%%%%%%%%%%%%%%%%%%%%%%%%%%%%%%%%%
\DescribeMacro{\...prefix}
In the alternative form |\childdocforwardprefix|,
%
\begin{center}
\begin{tabular}{l}
|% \iffalse
%
% childdoc.dtx Copyright (C) 2017-2018 Niklas Beisert
%
% This work may be distributed and/or modified under the
% conditions of the LaTeX Project Public License, either version 1.3
% of this license or (at your option) any later version.
% The latest version of this license is in
%   http://www.latex-project.org/lppl.txt
% and version 1.3 or later is part of all distributions of LaTeX
% version 2005/12/01 or later.
%
% This work has the LPPL maintenance status `maintained'.
%
% The Current Maintainer of this work is Niklas Beisert.
%
% This work consists of the files childdoc.dtx and childdoc.ins
% and the derived files childdoc.def and cdocsamp.tex with
% cdocsch1.tex, cdocsch2.tex, cdocsdrf.tex, cdocsfn1.tex, cdocsfn2.tex.
%
%<package>\ifdefined\childdocmain\endinput\fi
%<package>\ProvidesFile{childdoc.def}[2018/12/30 v2.0 child document driver]
%<samplemain>\ProvidesFile{cdocsamp.tex}[2018/12/30 v2.0 sample for childdoc]
%<*driver>
%\ProvidesFile{childdoc.drv}[2018/12/30 v2.0 childdoc reference manual file]
\PassOptionsToClass{10pt,a4paper}{article}
\documentclass{ltxdoc}

\usepackage[margin=35mm]{geometry}
\usepackage{hyperref}
\usepackage{hyperxmp}
\usepackage[usenames]{color}

\hypersetup{colorlinks=true}
\hypersetup{pdfstartview=FitH}
\hypersetup{pdfpagemode=UseNone}
\hypersetup{pdfsource={}}
\hypersetup{pdflang={en-UK}}
\hypersetup{pdfcopyright={Copyright 2017-2018 Niklas Beisert.
  This work may be distributed and/or modified under the
  conditions of the LaTeX Project Public License, either version 1.3
  of this license or (at your option) any later version.}}
\hypersetup{pdflicenseurl={http://www.latex-project.org/lppl.txt}}
\hypersetup{pdfcontactaddress={ETH Zurich, ITP, HIT K,
  Wolfgang-Pauli-Strasse 27}}
\hypersetup{pdfcontactpostcode={8093}}
\hypersetup{pdfcontactcity={Zurich}}
\hypersetup{pdfcontactcountry={Switzerland}}
\hypersetup{pdfcontactemail={nbeisert@itp.phys.ethz.ch}}
\hypersetup{pdfcontacturl={http://people.phys.ethz.ch/\xmptilde nbeisert/}}

\newcommand{\secref}[1]{\hyperref[#1]{section \ref*{#1}}}

\parskip1ex
\parindent0pt
\let\olditemize\itemize
\def\itemize{\olditemize\parskip0pt}

\begin{document}

\title{The \textsf{childdoc} Package}
\hypersetup{pdftitle={The childdoc Package}}
\author{Niklas Beisert\\[2ex]
  Institut f\"ur Theoretische Physik\\
  Eidgen\"ossische Technische Hochschule Z\"urich\\
  Wolfgang-Pauli-Strasse 27, 8093 Z\"urich, Switzerland\\[1ex]
  \href{mailto:nbeisert@itp.phys.ethz.ch}
  {\texttt{nbeisert@itp.phys.ethz.ch}}}
\hypersetup{pdfauthor={Niklas Beisert}}
\hypersetup{pdfsubject={Manual for the LaTeX2e Package childdoc}}
\date{30 December 2018, \textsf{v2.0}}
\maketitle

\begin{abstract}\noindent
\textsf{childdoc} is a \LaTeXe{} package
that enables the direct compilation
of document sections included by |\include|
to individual files.
\end{abstract}

\begingroup
\parskip0ex
\tableofcontents
\endgroup

%%%%%%%%%%%%%%%%%%%%%%%%%%%%%%%%%%%%%%%%%%%%%%%%%%%%%%%%%%%%%%%%%%%%%%%%%%%%%%%%
%%%%%%%%%%%%%%%%%%%%%%%%%%%%%%%%%%%%%%%%%%%%%%%%%%%%%%%%%%%%%%%%%%%%%%%%%%%%%%%%
\section{Introduction}

\LaTeX{} provides a mechanism to structure a large document (such as a book)
into a main file and several child files (containing the chapters)
using the |\include| command.
This mechanism is beneficial for documents
which span hundreds of pages in order to
make the source file(s) more manageable.
Moreover, compilation can be restricted to
selected child files by means of the |\includeonly| command.
The latter feature can be used to reduce the compilation time while editing
(this was significantly more useful in the earlier days of \LaTeX{})
or to generate a smaller document which is easier to navigate.
Another application of |\includeonly| is to generate
documents consisting of selected parts of the complete document.

However, there are a few drawbacks of the plain |\include| mechanism:
\begin{itemize}
\item
The child files cannot be compiled on their own,
they can only be compiled via the main file.
A naive editing environment
(such as a text editor with an option
to have the current file processed by \LaTeX)
may require one to switch to the main file before compiling;
attempting to compile the child file produces errors.
\item
The main file must be modified (each time)
to adjust the |\includeonly| command
to the present needs. This easily leaves the main file in a messy state.
\item
The generated document will always carry the filename
of the main document. This is inconvenient if
several child files are to be compiled and
to be kept for distribution.
\end{itemize}

The present package provides a simple interface
to make child files individually compilable by \LaTeX{}.
Compiling a child file then has the same effect as compiling
the main file with an |\includeonly| command
to select the appropriate child.
Moreover the generated document will carry the name of the child
rather than the main file.
This resolves all three above issues.

This feature is meant to make the editing of books,
thesis documents and lecture notes somewhat more convenient.
However, the package can also be used efficiently for
composing a series of documents (such as exercise sheets)
which are typically distributed individually.
It then assists the author in generating the individual documents
(potentially in different versions)
as well as a document containing the collected series.
Another application is in developing style files
or other kinds of included material
where compilation of the style file could redirect
to a sample or test file.

%%%%%%%%%%%%%%%%%%%%%%%%%%%%%%%%%%%%%%%%%%%%%%%%%%%%%%%%%%%%%%%%%%%%%%%%%%%%%%%%
%%%%%%%%%%%%%%%%%%%%%%%%%%%%%%%%%%%%%%%%%%%%%%%%%%%%%%%%%%%%%%%%%%%%%%%%%%%%%%%%
\section{Usage}

First of all, the package \textsf{childdoc} is \emph{not} a standard
\LaTeXe{} |.sty| style file! Therefore it needs to be invoked in
a non-standard way.

%%%%%%%%%%%%%%%%%%%%%%%%%%%%%%%%%%%%%%%%%%%%%%%%%%%%%%%%%%%%%%%%%%%%%%%%%%%%%%%%
\subsection{Included Files}
\label{sec:include}

%%%%%%%%%%%%%%%%%%%%%%%%%%%%%%%%%%%%%%%%
\DescribeMacro{\childdocmain}
To use the package, add the commands
\begin{center}
\begin{tabular}{l}
|\input{childdoc.def}|\\
|\childdocmain{}|\\
\end{tabular}
\end{center}
at the very top of the main \LaTeX{} file,
in particular \emph{before} the |\documentclass| statement!
The argument of |\childdocmain| should be left empty
(but it must be present).

%%%%%%%%%%%%%%%%%%%%%%%%%%%%%%%%%%%%%%%%
\DescribeMacro{\childdocof}
Furthermore, add the commands
\begin{center}
\begin{tabular}{l}
|\input{childdoc.def}|\\
|\childdocof{|\textit{main}|}|\\
\end{tabular}
\end{center}
at the top of every child file \textit{child}
which is included by |\include{|\textit{child}|}|
from within the main file
(or at least for those files to be compiled individually).
The argument \textit{main} must be the filename of the main file.

There are a couple of
considerations in setting up the main and child documents:

%%%%%%%%%%%%%%%%%%%%%%%%%%%%%%%%%%%%%%%%
\paragraph{Restrictions.}

Please note the following restrictions:
\begin{itemize}
\item
|\childdocmain| must be called with one argument \textit{main}
to ensure compatibility with earlier version of the package.
It must either be empty (|\childdocmain{}|)
or precisely match the filename of the main file in which it is specified.
See \secref{sec:detection} for further information.
\item
The filename \textit{main} must be specified without the |.tex| extension.
\item
The filename \textit{main} is case sensitive
(even in case-insensitive file systems)
due to internal string comparison.
\item
The argument \textit{main} should be fully expanded, it cannot be a macro.
\item
Subdirectories and special characters should be avoided in filenames.
\item
The command |\childdocmain{|\textit{main}|}| must be followed by a whitespace.
It should not be followed immediately by another command
or by a comment mark `|%|'.
This is because the \TeX{} parser reads the token immediately following
the argument of |\childdocmain| and puts it
at the beginning of every child section;
however, a white\-space is ignored.
\end{itemize}

%%%%%%%%%%%%%%%%%%%%%%%%%%%%%%%%%%%%%%%%
\paragraph{Content of Main File.}

It is advisable to place all content in the child files included by |\include|.
Any output contained in the main file will appear in all child documents
unless suppressed manually;
it cannot be suppressed automatically by the |\includeonly| directive
and thus should normally be avoided.
A method to include some content in the main file
by means of conditional processing is described in \secref{sec:conditional}.

%%%%%%%%%%%%%%%%%%%%%%%%%%%%%%%%%%%%%%%%
\paragraph{Page Numbering.}

When only a part of the document is compiled,
the appropriate numbering of pages
(as well as other status parameters)
is determined from the |.aux| files.
The latter contain information from previous passes.
However this information needs to propagate through
all intermediate child documents.
Therefore the page numbering in child documents may well
be inconsistent until the complete document is compiled at least once.

A useful (if unconventional) way to always ensure a consistent
page numbering is to restart the numbering in each child document
and denote the pages by `\textit{child}|.|\textit{page}'
where \textit{child} represents the chapter/section number of the child file.
This can be achieved by the command
|\numberwithin{page}{|\textit{child}|}|
of the \textsf{amsmath} package
where \textit{child} can be |chapter| or |section|
depending on the chosen structuring.
Alternatively, one can modify the macro |\thepage| appropriately
and reset the counter |page| at the start of each child file.

%%%%%%%%%%%%%%%%%%%%%%%%%%%%%%%%%%%%%%%%%%%%%%%%%%%%%%%%%%%%%%%%%%%%%%%%%%%%%%%%
\subsection{Conditional Processing}
\label{sec:conditional}

The package provides a mechanism to compile different versions
of a document. To customise the versions further some conditional processing
can come in handy to distinguish which version is being compiled.
The package provides two macros to describe the compilation context:

%%%%%%%%%%%%%%%%%%%%%%%%%%%%%%%%%%%%%%%%
\DescribeMacro{\ifchilddoc}
The conditional |\ifchilddoc| distinguishes between the compilation of
child documents and the main document:
%
\begin{center}
|\ifchilddoc |\textit{child-code}| |[|\||else |\textit{main-code}]| \||fi|
\end{center}

%%%%%%%%%%%%%%%%%%%%%%%%%%%%%%%%%%%%%%%%
\DescribeMacro{\childdocname}
\DescribeMacro{\childdocjob}
The macro |\childdocname| contains the filename (without extension)
of the main or child file being processed.
Note that |\childdocjob| will always contain the name of the main file.

%%%%%%%%%%%%%%%%%%%%%%%%%%%%%%%%%%%%%%%%
\paragraph{Title Page.}

Conditional processing can be used to include a title or banner page
in the main document when proper precautions are taken.
Importantly, the code in the main file should ensure that the page counter
(as well as other status parameters which are stored in the |.aux| files)
takes the same value after the conditional processing.
Otherwise the page numbers may take divergent values
depending on which part is compiled.

For example, a title page could be declared by:
%
\begin{center}
\begin{tabular}{l}
|\ifchilddoc\||else|\\
|\addtocounter{page}{-1}|\\
\textit{code for title page}\\
|\newpage|\\
|\||fi|
\end{tabular}
\end{center}
%
A banner page for the child documents can be generated by:
%
\begin{center}
\begin{tabular}{l}
|\ifchilddoc|\\
|\addtocounter{page}{-1}|\\
\textit{code for banner page}\\
|\newpage|\\
|\||fi|
\end{tabular}
\end{center}
%
Here one could write a message such as:
\begin{center}
|This is the part \childdocname{} of \childdocjob{}.|
\end{center}

%%%%%%%%%%%%%%%%%%%%%%%%%%%%%%%%%%%%%%%%%%%%%%%%%%%%%%%%%%%%%%%%%%%%%%%%%%%%%%%%
\subsection{Flags}
\label{sec:flags}

The package makes it easy to generate different versions
of the main or child documents.
To this end compilation flags can be defined
and assigned different default values.
They will be particularly useful in conjunction
with the forwarding mechanism described in \secref{sec:forward}.

For example, it may be useful to have a flag |\version|
which can be set to |draft| or |final|.
The document source will contain some conditional code
depending on the value of |\version|.
Suppose further, the flag should default to |final| for the main file
and to |draft| for child files
which is a natural assignment for editing the document.
This is achieved by placing the following code
in the preamble of the main document
(below the |\childdocmain| directive):
%
\begin{center}
\begin{tabular}{l}
|\ifchilddoc|\\
|\providecommand{\version}{draft}|\\
|\||else|\\
|\providecommand{\version}{final}|\\
|\||fi|
\end{tabular}
\end{center}
%
The definition by |\providecommand| makes sure
that previous definitions are not overwritten.
Further statements |\providecommand{\version}{...}|
can thus be added before the above code to override it.

For the main file, one might add a line
(between |\childdocmain| and the above block)
%
\begin{center}
|%\ifchilddoc\||else\providecommand{\version}{draft}\||fi|
\end{center}
%
which can be uncommented to produce a draft version.
Likewise one can add a line to the very top of a child file
(above the |\childdocof{|\textit{main}|}| directive)
%
\begin{center}
|%\providecommand{\version}{final}|
\end{center}
%
which can be uncommented to produce the final version of this child document.

%%%%%%%%%%%%%%%%%%%%%%%%%%%%%%%%%%%%%%%%%%%%%%%%%%%%%%%%%%%%%%%%%%%%%%%%%%%%%%%%
\subsection{Forwarding}
\label{sec:forward}

Different versions of the main or child documents
using compilation flags as described in \secref{sec:flags}
can be (permanently) stored in different files
for convenient compilation, viewing and distribution.
To this end, the package defines a command
to pass on compilation to a different file:

%%%%%%%%%%%%%%%%%%%%%%%%%%%%%%%%%%%%%%%%
\DescribeMacro{\childdocforward}
The command |\childdocforward| redirects processing to
another source file:
%
\begin{center}
\begin{tabular}{l}
|\input{childdoc.def}|\\
|\childdocforward[|\textit{main}|]{|\textit{dest}|}|\\
\end{tabular}
\end{center}
%
The argument \textit{dest} is the destination file
(without extension).
It should be the main file or one of the child files.
Note that further \textsf{childdoc} directives
such as |\childdocof| and |\childdocforward|
in the indicated file will be processed in this form.
The optional argument \textit{main}
passes on directly to the main file \textit{main}
while pretending to compile the child \textit{dest}.
This form behaves as if \textit{dest}
issues |\childdocof{|\textit{main}|}| right away,
and no further \textsf{childdoc} directives will be processed.

%%%%%%%%%%%%%%%%%%%%%%%%%%%%%%%%%%%%%%%%
\DescribeMacro{\...prefix}
In the alternative form |\childdocforwardprefix|,
%
\begin{center}
\begin{tabular}{l}
|\input{childdoc.def}|\\
|\childdocforwardprefix[|\textit{main}|]{|\textit{prefix}|}{|\textit{dest}|}|
\end{tabular}
\end{center}
%
the destination file is determined by a pattern
depending on the current file:
To make this work, the current file must be called
`{\textit{prefix}\hspace{0.2em}\textit{suffix}}'
with \textit{prefix} matching precisely the argument.
Processing is then passed on to the file
`{\textit{dest}\hspace{0.2em}\textit{suffix}}'.
Surely, the same effect is achieved by
directly specifying the
argument `{\textit{dest}\hspace{0.2em}\textit{suffix}}'
in the first form.
However, that requires to set up a different file
for each child. With the alternative form of the command
all these files can have exactly the same content
which simplifies setting them up and maintaining them.

For example, the following file |draft.tex|
with a compilation flag |\version| as described in \secref{sec:flags}
compiles the main document as a draft:
%
\begin{center}
\begin{tabular}{l}
|\def\version{draft}|\\
|\input{childdoc.def}|\\
|\childdocforward{|\textit{main}|}|
\end{tabular}
\end{center}
%
Likewise, the following files |final|\textit{nn}|.tex|
compile the final version of the child document
|child|\textit{nn}|.tex|:
%
\begin{center}
\begin{tabular}{l}
|\def\version{final}|\\
|\input{childdoc.def}|\\
|\childdocforwardprefix{final}{child}|
\end{tabular}
\end{center}
%

Note that when several versions of a main file and/or of each child file
are to be generated, it may be convenient to set up a |Makefile| or
shell script to automatise the process.

%%%%%%%%%%%%%%%%%%%%%%%%%%%%%%%%%%%%%%%%%%%%%%%%%%%%%%%%%%%%%%%%%%%%%%%%%%%%%%%%
\subsection{Command Line Processing}
\label{sec:commandline}

The effect of redirection files can also be achieved by invoking
the \LaTeX{} compiler with a more elaborate command line.
Most conveniently this should be done as part
of a shell script or a |Makefile|.

When using \textsf{childdoc} in the main file, the following
command lines effectively perform a redirection
(note that depending on the shell being used,
backslashes may have to be doubled: `|\|' $\to$ `|\\|'):
%
\begin{center}
|... -jobname "|\textit{target}|" |\\|"|[\textit{flags}]%
|\input{childdoc.def}\childdocforward[|\textit{main}|]{|\textit{dest}|}"|
\end{center}
%
Here \textit{target} is the name of the output file,
\textit{main} is the name of the main file
and \textit{dest} is the name of the main or child file to be processed
(all filenames without extensions).
The optional argument \textit{main} can be omitted
if \textit{main} matches \textit{dest}.
Optionally, compilation \textit{flags} can be defined via |\def| commands.
This command line makes the \TeX{} engine believe
it is compiling the file \textit{target}
whose content is specified as the latter parameter.
The provided code then forwards the processing to
\textit{main} or \textit{dest} as described in \secref{sec:forward}.

%%%%%%%%%%%%%%%%%%%%%%%%%%%%%%%%%%%%%%%%%%%%%%%%%%%%%%%%%%%%%%%%%%%%%%%%%%%%%%%%
\subsection{Include by Input}
\label{sec:input}

Including child documents by |\include| has some restrictions by design.
Most notably, the content of a child document always occupies
its own set of pages; pages cannot be shared between child documents.
Usually, this behaviour makes perfect sense
because each child document contain an essential part of the document.
However, in some situations it may be desirable to compose
a document from a collection of parts
without having mandatory page breaks between then.
For this case, the package
provides a mechanism to include parts
by |\input| which can also be processed individually.
However, by construction this mechanism
requires manual handling of the content to be output.

%%%%%%%%%%%%%%%%%%%%%%%%%%%%%%%%%%%%%%%%
\DescribeMacro{\ifchilddocmanual}
The main file should be prepared as usual, see \secref{sec:include}.
However, the document body must make a distinction
between processing of an individual part and of the main document, e.g.:
%
\begin{center}
\begin{tabular}{l}
|\ifchilddocmanual|\\
|\input{\childdocname}|\\
|\||else|\\
\textit{document body with }|\input{|\textit{part}|}|\\
|\||fi|
\end{tabular}
\end{center}
%
The conditional |\ifchilddocmanual| is true whenever
a part to be included by |\input| is being compiled,
and the name of the part is stored in |\childdocname|.

%%%%%%%%%%%%%%%%%%%%%%%%%%%%%%%%%%%%%%%%
\DescribeMacro{\childdocby}
Each part to be included by |\input| should start with:
%
\begin{center}
\begin{tabular}{l}
|\input{childdoc.def}|\\
|\childdocby{|\textit{main}|}|\\
\end{tabular}
\end{center}
%
The directive |\childdocby| is similar to |\childdocof|
described in \secref{sec:include},
but the subsequent selection of content must be done manually.
To that end, both |\ifchilddoc| and |\ifchilddocmanual|
will be true upon processing of a part,
and the name of the part is stored in |\childdocname|.
Note that |\jobname| will be set to the filename of the current part
so that each part receives an individual |.aux| file
that does not interfere with the |.aux| file(s) of the main document.
This behaviour can be altered by the alternative form
|\childdocby[*]{|\textit{main}|}| (with a non-empty optional argument)
which uses the |.aux| file of the main document
by setting |\jobname| to \textit{main}.

%%%%%%%%%%%%%%%%%%%%%%%%%%%%%%%%%%%%%%%%%%%%%%%%%%%%%%%%%%%%%%%%%%%%%%%%%%%%%%%%
\subsection{Driver Development}
\label{sec:driver}

The \textsf{childdoc} mechanism can also be use for the development
of definition files such as \LaTeX{} styles or classes.
This case differs from the above setup with multiple parts
included by |\include| in that no |\includeonly| should be invoked.
This can be achieved by starting the include file
(before |\ProvidesPackage|) with:
%
\begin{center}
\begin{tabular}{l}
|\input{childdoc.def}|\\
|\childdocforward{|\textit{main}|}|\\
\end{tabular}
\end{center}
%
or alternatively with:
%
\begin{center}
\begin{tabular}{l}
|\input{childdoc.def}|\\
|\childdocby{|\textit{main}|}|\\
\end{tabular}
\end{center}
%
Both forms have slightly different effects as described above.
The main file is prepared as usual, see \secref{sec:include}.

%%%%%%%%%%%%%%%%%%%%%%%%%%%%%%%%%%%%%%%%%%%%%%%%%%%%%%%%%%%%%%%%%%%%%%%%%%%%%%%%
\subsection{Legacy Detection}
\label{sec:detection}

The directive |\childdocmain| in the main file can detect
whether the complete document or merely a child is to be compiled
even without using the directive |\childdocof|.
This method is deprecated because it is less robust
and there is no compelling reason to use it;
it is merely provided for backward compatibility
and it may be removed in future versions.

If the detection mechanism is to be used,
it is mandatory to correctly specify
the filename of the main file as the argument of |\childdocmain|:
%
\begin{center}
\begin{tabular}{l}
|\input{childdoc.def}|\\
|\childdocmain{|\textit{main}|}|\\
\end{tabular}
\end{center}
%
If |\jobname| does not match the argument \textit{main} of |\childdocmain|,
it is assumed that |\jobname| points to the child file to be compiled.
When using |\childdocmain| with the main file specified as argument,
it suffices to start a child file
with just |\input{|\textit{main}|}|
without loading of the package and using |\childdocof|.
If instead all processing is done
with the appropriate \textsf{childdoc} directives,
the argument of \textit{main} of |\childdocmain| can be empty.

An alternative version of the command line processing described
in \secref{sec:commandline} using the detection mechanism reads:
%
\begin{center}
|... -jobname "|\textit{target}|" "|[\textit{flags}]%
[|\def\jobname{|\textit{dest}|}|]|\input{|\textit{main}|}"|
\end{center}

%%%%%%%%%%%%%%%%%%%%%%%%%%%%%%%%%%%%%%%%%%%%%%%%%%%%%%%%%%%%%%%%%%%%%%%%%%%%%%%%
\subsection{Manual Code}
\label{sec:manual}

In case one cannot be certain whether the definitions file |childdoc.def|
is installed on the target \TeX{} distribution
and one prefers not to ship it,
it is conceivable to paste a few relevant commands into the sources.

To that end, drop all statements |\input{childdoc.def}|
and perform the replacements as outlined below.
Instead of |\childdocmain{|\textit{main}|}| add the following code
to the top of the main file:
%
\begin{center}
\begin{tabular}{l}
|\||ifdefined\childdocname\endinput\||fi\newif\ifchilddoc|\\
|\edef\childdocname{\scantokens\expandafter{\jobname\noexpand}}|\\
|\def\childdocmain{|\textit{main}|}\||ifx\childdocmain\childdocname\||else|\\
|\childdoctrue\includeonly{\childdocname}\let\jobname\childdocmain\||fi|\\
\end{tabular}
\end{center}
%
Instead of |\childdocof{|\textit{main}|}| just include the main file
at the top of each child file:
%
\begin{center}
|\input{|\textit{main}|}|
\end{center}
%
A simple redirection |\childdocforward{|\textit{dest}|}| is achieved by:
%
\begin{center}
|\def\jobname{|\textit{dest}|}\input{\jobname}|
\end{center}
%
The redirection with prefix
|\childdocforwardprefix[|\textit{prefix}|]{|\textit{dest}|}|
is accomplished by:
%
\begin{center}
\begin{tabular}{l}
|{\edef\jobname{\scantokens\expandafter{\jobname\noexpand}}|\\
|\def\redirectjob |\textit{prefix}|#1~~~{\gdef\jobname{|\textit{dest}|#1}}|\\
|\expandafter\redirectjob\jobname~~~}\input{\jobname}|
\end{tabular}
\end{center}

In an alternative approach,
child documents can be compiled by a specific command line
without additional code or specific definitions:
%
\begin{center}
|... -jobname "|\textit{target}|" "|[\textit{flags}]%
|\includeonly{|\textit{dest}|}\input{|\textit{main}|}"|
\end{center}
%

%%%%%%%%%%%%%%%%%%%%%%%%%%%%%%%%%%%%%%%%%%%%%%%%%%%%%%%%%%%%%%%%%%%%%%%%%%%%%%%%
%%%%%%%%%%%%%%%%%%%%%%%%%%%%%%%%%%%%%%%%%%%%%%%%%%%%%%%%%%%%%%%%%%%%%%%%%%%%%%%%
\section{Information}

%%%%%%%%%%%%%%%%%%%%%%%%%%%%%%%%%%%%%%%%%%%%%%%%%%%%%%%%%%%%%%%%%%%%%%%%%%%%%%%%
\subsection{Copyright}

Copyright \copyright{} 2017--2018 Niklas Beisert

This work may be distributed and/or modified under the
conditions of the \LaTeX{} Project Public License, either version 1.3
of this license or (at your option) any later version.
The latest version of this license is in
  \url{http://www.latex-project.org/lppl.txt}
and version 1.3 or later is part of all distributions of \LaTeX{}
version 2005/12/01 or later.

This work has the LPPL maintenance status `maintained'.

The Current Maintainer of this work is Niklas Beisert.

This work consists of the files |README.txt|, |childdoc.ins| and |childdoc.dtx|
as well as the derived files |childdoc.def|, |cdocsamp.tex|
with |cdocsch1.tex|, |cdocsch2.tex|, |cdocspt3.tex|, |cdocspt4.tex|,
|cdocsdrf.tex|, |cdocsfn1.tex|, |cdocsfn2.tex|
as well as |childdoc.pdf|.

%%%%%%%%%%%%%%%%%%%%%%%%%%%%%%%%%%%%%%%%%%%%%%%%%%%%%%%%%%%%%%%%%%%%%%%%%%%%%%%%
\subsection{Files and Installation}

The package consists of the files:
%
\begin{center}
\begin{tabular}{ll}
    |README.txt|   & readme file \\
    |childdoc.ins| & installation file \\
    |childdoc.dtx| & source file \\
    |childdoc.def| & definition file \\
    |cdocsamp.tex| & sample main file \\
    |cdocsch1.tex| & sample include file \\
    |cdocsch2.tex| & sample include file \\
    |cdocspt3.tex| & sample part file \\
    |cdocspt4.tex| & sample part file \\
    |cdocsdrf.tex| & sample redirection file \\
    |cdocsfn1.tex| & sample redirection file \\
    |cdocsfn2.tex| & sample redirection file \\
    |childdoc.pdf| & manual
\end{tabular}
\end{center}
%
The distribution consists of the files
|README.txt|, |childdoc.ins| and |childdoc.dtx|.
%
\begin{itemize}
\item
Run (pdf)\LaTeX{} on |childdoc.dtx|
to compile the manual |childdoc.pdf| (this file).
\item
Run \LaTeX{} on |childdoc.ins| to create the definitions file |childdoc.def|
and the sample |cdocsamp.tex| with include files
|cdocsch1.tex|, |cdocsch2.tex|, |cdocspt3.tex|, |cdocspt4.tex|,
|cdocsdrf.tex|, |cdocsfn1.tex|, |cdocsfn2.tex|.
Then copy the file |childdoc.def| to an appropriate directory of your \LaTeX{}
distribution, e.g.\ \textit{texmf-root}|/tex/latex/childdoc|.
\end{itemize}

%%%%%%%%%%%%%%%%%%%%%%%%%%%%%%%%%%%%%%%%%%%%%%%%%%%%%%%%%%%%%%%%%%%%%%%%%%%%%%%%
\subsection{Related CTAN Packages}

There are several other packages which offer a similar functionality:
%
\begin{itemize}
\item
The packages
\href{http://ctan.org/pkg/docmute}{\textsf{docmute}},
\href{http://ctan.org/pkg/includex}{\textsf{includex}} and
\href{http://ctan.org/pkg/standalone}{\textsf{standalone}}
provide commands to include only the document body of
a child file thus allowing both files to be compiled individually.
\item
The packages \href{http://ctan.org/pkg/subdocs}{\textsf{subdocs}}
and \href{http://ctan.org/pkg/subfiles}{\textsf{subfiles}}
provide structures in which the main and child documents can be
encapsulated and allowing them to be compiled individually.
The inclusion mechanism is different from the conventional |\include|.
\item
The package \href{http://ctan.org/pkg/combine}{\textsf{combine}}
is an elaborate solution to combine several documents into one.
\end{itemize}
%
See also the CTAN topic \href{http://ctan.org/topic/subdocs}{\textsf{subdocs}}
for further related packages.
The present package differs from the above solutions in that
a document structure constructed with the conventional |\include| mechanism
just needs two extra commands at the top of every file
such that all constituent files can be compiled individually.

%%%%%%%%%%%%%%%%%%%%%%%%%%%%%%%%%%%%%%%%%%%%%%%%%%%%%%%%%%%%%%%%%%%%%%%%%%%%%%%%
%\subsection{Feature Suggestions}
%
%The following is a list of features which may be useful for future
%versions of this package:
%%
%\begin{itemize}
%\item
%\ldots
%\end{itemize}

%%%%%%%%%%%%%%%%%%%%%%%%%%%%%%%%%%%%%%%%%%%%%%%%%%%%%%%%%%%%%%%%%%%%%%%%%%%%%%%%
\subsection{Revision History}

%%%%%%%%%%%%%%%%%%%%%%%%%%%%%%%%%%%%%%%%
\paragraph{v2.0:} 2018/12/30

\begin{itemize}
\item
immediate forward processing
\item
added |\childdocby| mechanism
\item
manual restructured
\end{itemize}

%%%%%%%%%%%%%%%%%%%%%%%%%%%%%%%%%%%%%%%%
\paragraph{v1.6:} 2018/01/17

\begin{itemize}
\item
application for development of include files
\item
corrections to manual
\end{itemize}

%%%%%%%%%%%%%%%%%%%%%%%%%%%%%%%%%%%%%%%%
\paragraph{v1.5:} 2017/05/21

\begin{itemize}
\item
more complete structuring introduced
\item
|\childdocof| introduced
\item
|\childdoc| renamed to |\childdocmain|
\item
|\childredirect| renamed to |\childdocforward| and |\childdocforwardprefix|
and functionality expanded
\end{itemize}

%%%%%%%%%%%%%%%%%%%%%%%%%%%%%%%%%%%%%%%%
\paragraph{v1.0:} 2017/04/27

\begin{itemize}
\item
manual and install package
\item
first version published on CTAN
\end{itemize}

%%%%%%%%%%%%%%%%%%%%%%%%%%%%%%%%%%%%%%%%
\paragraph{v0.6:} 2017/04/26

\begin{itemize}
\item
redirection mechanism added
\end{itemize}

%%%%%%%%%%%%%%%%%%%%%%%%%%%%%%%%%%%%%%%%
\paragraph{v0.5:} 2017/04/26

\begin{itemize}
\item
functionality in definition file
\end{itemize}


%%%%%%%%%%%%%%%%%%%%%%%%%%%%%%%%%%%%%%%%%%%%%%%%%%%%%%%%%%%%%%%%%%%%%%%%%%%%%%%%
%%%%%%%%%%%%%%%%%%%%%%%%%%%%%%%%%%%%%%%%%%%%%%%%%%%%%%%%%%%%%%%%%%%%%%%%%%%%%%%%
%%%%%%%%%%%%%%%%%%%%%%%%%%%%%%%%%%%%%%%%%%%%%%%%%%%%%%%%%%%%%%%%%%%%%%%%%%%%%%%%
\appendix

\settowidth\MacroIndent{\rmfamily\scriptsize 000\ }

 \DocInput{childdoc.dtx}

\end{document}
%</driver>
% \fi
%
% %%%%%%%%%%%%%%%%%%%%%%%%%%%%%%%%%%%%%%%%%%%%%%%%%%%%%%%%%%%%%%%%%%%%%%%%%%%%%%
% %%%%%%%%%%%%%%%%%%%%%%%%%%%%%%%%%%%%%%%%%%%%%%%%%%%%%%%%%%%%%%%%%%%%%%%%%%%%%%
% \section{Sample}
%\iffalse
%<*samplemain>
%\fi
%
% The following presents a sample document
% with two chapters, two parts, a title page,
% a compile flag as well as three forwarding files to set the flag.
% It consists of eight |.tex| files:
% \begin{center}
% \begin{tabular}{ll}
% |cdocsamp.tex|&main file\\
% |cdocsch1.tex|&include file for chapter 1\\
% |cdocsch2.tex|&include file for chapter 2\\
% |cdocspt3.tex|&include file for part 3\\
% |cdocspt4.tex|&include file for part 4\\
% |cdocsdrf.tex|&forwarding file for main file in draft mode\\
% |cdocsfi1.tex|&forwarding file for final version of chapter 1\\
% |cdocsfi2.tex|&forwarding file for final version of chapter 2\\
% \end{tabular}
% \end{center}
% Each of the eight files can be compiled directly by the \LaTeX{} compiler.
%
% %%%%%%%%%%%%%%%%%%%%%%%%%%%%%%%%%%%%%%
% \paragraph{Main File.}
%
% The main file is called |cdocsamp.tex|.
%
% Load the \textsf{childdoc} definitions and
% declare the filename for the main document:
%    \begin{macrocode}
\input{childdoc.def}
\childdocmain{}
%    \end{macrocode}

% Optional override for |\version| flag:
%    \begin{macrocode}
%%\ifchilddoc\else\providecommand{\version}{draft}\fi
%    \end{macrocode}

% Define the default values for the |\version| flag
% (|final| for the main file and |draft| for childs):
%    \begin{macrocode}
\ifchilddoc
\providecommand{\version}{draft}
\else
\providecommand{\version}{final}
\fi
%    \end{macrocode}

% Load the standard document class:
%    \begin{macrocode}
\documentclass[12pt]{article}
%    \end{macrocode}

% Start the document body:
%    \begin{macrocode}
\begin{document}
%    \end{macrocode}

% Declare a title page.
% Print title, part of document being processed and version flag:
%    \begin{macrocode}
\addtocounter{page}{-1}
\begin{center}
{\LARGE\bfseries{}childdoc example\par}
\vspace{1cm}
\ifchilddoc
\ifchilddocmanual part\else chapter\fi:
`\childdocname' of `\childdocjob'\par
\else
main document: `\childdocjob'\par
\fi
version: \version\par
\end{center}
\newpage
%    \end{macrocode}

% Manually include selected file,
% otherwise process as usual:
%    \begin{macrocode}
\ifchilddocmanual
\section*{part `\childdocname'}
\input{\childdocname}
\else
%    \end{macrocode}

% Include the two chapters:
%    \begin{macrocode}
\include{cdocsch1}
\include{cdocsch2}
%    \end{macrocode}

% Include the two parts unless only chapters should be displayed:
%    \begin{macrocode}
\ifchilddoc\else
\section{part three}
\input{cdocspt3}
\section{part four}
\input{cdocspt4}
\fi
%    \end{macrocode}

% Process as usual until here:
%    \begin{macrocode}
\fi
%    \end{macrocode}

% End of document body:
%    \begin{macrocode}
\end{document}
%    \end{macrocode}
%\iffalse
%</samplemain>
%\fi
%
% %%%%%%%%%%%%%%%%%%%%%%%%%%%%%%%%%%%%%%
% \paragraph{Chapter Include Files.}
%
% The include files are called |cdocsch1.tex| and |cdocsch2.tex|.
%
%\iffalse
%<*samplechap1|samplechap2>
%\fi

% Optional override for |\version| flag:
%    \begin{macrocode}
%%\providecommand{\version}{final}
%    \end{macrocode}

% Include the main document:
%    \begin{macrocode}
\input{childdoc.def}
\childdocof{cdocsamp}
%    \end{macrocode}

%\iffalse
%</samplechap1|samplechap2>
%\fi
%
%\iffalse
%<*samplechap1>
%\fi
% Some text for chapter 1:
%    \begin{macrocode}
\section{one}
some text in chapter one
%    \end{macrocode}

%\iffalse
%</samplechap1>
%\fi
% Some text for chapter 2:
%\iffalse
%<*samplechap2>
%\fi
%    \begin{macrocode}
\section{two}
more text in chapter two
%    \end{macrocode}

%\iffalse
%</samplechap2>
%\fi
%
% %%%%%%%%%%%%%%%%%%%%%%%%%%%%%%%%%%%%%%
% \paragraph{Part Include Files.}
%
% The include files are called |cdocspt3.tex| and |cdocspt4.tex|.
%
%\iffalse
%<*samplepart3|samplepart4>
%\fi

% Optional override for |\version| flag:
%    \begin{macrocode}
%%\providecommand{\version}{final}
%    \end{macrocode}

% Include the main document:
%    \begin{macrocode}
\input{childdoc.def}
\childdocby{cdocsamp}
%    \end{macrocode}

%\iffalse
%</samplepart3|samplepart4>
%\fi
%
%\iffalse
%<*samplepart3>
%\fi
% Some text for part 3:
%    \begin{macrocode}
some text in part three
%    \end{macrocode}

%\iffalse
%</samplepart3>
%\fi
% Some text for part 4:
%\iffalse
%<*samplepart4>
%\fi
%    \begin{macrocode}
more text in part four
%    \end{macrocode}

%\iffalse
%</samplepart4>
%\fi
%
% %%%%%%%%%%%%%%%%%%%%%%%%%%%%%%%%%%%%%%
% \paragraph{Forwarding for a Complete Draft.}
%
% The following forwarding file |cdocsdrf.tex|
% compiles the main document in draft mode:
%\iffalse
%<*sampledraft>
%\fi
%    \begin{macrocode}
\def\version{draft}
\input{childdoc.def}
\childdocforward{cdocsamp}
%    \end{macrocode}

%\iffalse
%</sampledraft>
%\fi
%
% %%%%%%%%%%%%%%%%%%%%%%%%%%%%%%%%%%%%%%
% \paragraph{Forwarding for Final Version of the Chapters.}
%
% The following forwarding files |cdocsfn1.tex| and |cdocsfn2.tex|
% (with identical content)
% compile the final versions of the child documents
% |cdocsch1.tex| and |cdocsch2.tex|, respectively:
%\iffalse
%<*samplefinal>
%\fi
%    \begin{macrocode}
\def\version{final}
\input{childdoc.def}
\childdocforwardprefix[cdocsamp]{cdocsfn}{cdocsch}
%    \end{macrocode}

%\iffalse
%</samplefinal>
%\fi
%
% %%%%%%%%%%%%%%%%%%%%%%%%%%%%%%%%%%%%%%
% \paragraph{Command Line Processing.}
%
% The following three command lines generate the output files
% |cdocscld|, |cdocscl1| and |cdocscl2|
% which should be identical to
% |cdocsdrf|, |cdocsch1| and |cdocsfn2|, respectively:
% \begin{center}
% \begin{tabular}{l}
% |latex -jobname cdocscld \|\\
% |  "\def\version{draft}\input{childdoc.def}\childdocforward{cdocsamp}"|\\
% |latex -jobname cdocscl1 \|\\
% |  "\input{childdoc.def}\childdocforward[cdocsamp]{cdocsch1}"|\\
% |latex -jobname cdocscl2 \|\\
% |  "\def\version{final}\input{childdoc.def}\childdocforward{cdocsch2}"|
% \end{tabular}
% \end{center}
% Note that the trailing backslash on each first line
% merely continues the input to the second line
% (for convenient cut ant paste).
% Furthermore, the command |latex| can be replaced by any
% of its alternative versions such as |pdflatex|.
%
% %%%%%%%%%%%%%%%%%%%%%%%%%%%%%%%%%%%%%%%%%%%%%%%%%%%%%%%%%%%%%%%%%%%%%%%%%%%%%%
% %%%%%%%%%%%%%%%%%%%%%%%%%%%%%%%%%%%%%%%%%%%%%%%%%%%%%%%%%%%%%%%%%%%%%%%%%%%%%%
% \section{Implementation}
%\iffalse
%<*package>
%\fi
%
% This section describes the definitions file |childdoc.def|.

% The definitions cannot be loaded using |\usepackage| or |\RequirePackage|
% which has a mechanism to prevent loading a style file more than once.
% When loading the definitions by means of |\input|
% multiple instances have to be prevented manually:
%\iffalse
%This code needs to be before the `\ProvidesFile' directive
%which is defined at the beginning of this file.
%Therefore it is also placed there and commented out here.
%</package>
%<*discard>
%\fi
%    \begin{macrocode}
\ifdefined\childdocmain\endinput\fi
%    \end{macrocode}
%\iffalse
%</discard>
%<*package>
%\fi
%
% \macro{\ifchilddoc}
% \macro{\ifchilddocmanual}
% The conditional |\ifchilddoc| tells whether a
% child (true) or main (false) document is being compiled.
% The conditional |\ifchilddocmanual| tells whether
% the |\includeonly| mechanism is used (false) or
% the selection of child files must be performed manually (true).
% The definitions initialise to false:
%    \begin{macrocode}
\newif\ifchilddoc
\newif\ifchilddocmanual
%    \end{macrocode}

% \macro{\childdocname}
% \macro{\childdocjob}
% The macro |\childdocname| stores the name of the main document
% to be compiled. The macro |\childdocjob| stores the name of
% the document on which the \LaTeX{} compiler was originally invoked.
% The content of |\jobname| cannot be compared
% to filenames specified in the source due to different catcodes.
% The following code rescans |\jobname|, stores the result
% in |\childdocname| and saves a copy in |\childdocjob|:
%    \begin{macrocode}
\edef\childdocname{\scantokens\expandafter{\jobname\noexpand}}
\let\childdocjob\childdocname
%    \end{macrocode}

% \macro{\childdocdisable}
% The macro |\childdocdisable| prevents the main file
% from being processed more than once.
% At this stage, the main document command |\childdocmain|
% is assumed to be called once again where it should do nothing.
% Any subsequent call to it should prevent
% a secondary processing of the main document
% It overwrites the forwarding commands
% |\childdocof| and |\childdocforward|
% with empty macros to prevent further inclusions of the main document:
%    \begin{macrocode}
\newcommand{\childdocdisable}
{
  \renewcommand{\childdocmain}[1]{\renewcommand{\childdocmain}[1]{\endinput}}
  \renewcommand{\childdocof}[1]{}
  \renewcommand{\childdocby}[2][]{}
  \renewcommand{\childdocforward}[2][]{}
  \renewcommand{\childdocdisable}{}
}
%    \end{macrocode}

% \macro{\childdocmain}
% The macro |\childdocmain| is to be called at the top of the main file
% with nothing or the main filename (without extension) as argument.
% First, it breaks loops.
% If the argument is not empty and does not match |\childdocname|
% (which is set by the first inclusion of |childdoc.def|),
% |\ifchilddoc| is set to true, |\includeonly| is applied to the child file
% and |\jobname| is set to the main file
% (for proper handling of |.aux| files):
%    \begin{macrocode}
\newcommand{\childdocmain}[1]
{
  \childdocdisable\childdocmain{}
  \if?#1?\else
    \begingroup
      \def\childdoctmp{#1}
      \ifx\childdoctmp\childdocname
        \def\childdoctmp{}
      \else
        \def\childdoctmp
        {
          \childdoctrue
          \includeonly{\childdocname}
          \def\childdocjob{#1}
          \def\jobname{#1}
        }
      \fi
      \expandafter
    \endgroup
    \childdoctmp
  \fi
}
%    \end{macrocode}

% \macro{\childdocof}
% The command |\childdocof| redirects
% compilation to the main file |#1|.
%    \begin{macrocode}
\newcommand{\childdocof}[1]
{
  \childdocdisable
  \childdoctrue
  \includeonly{\childdocname}
  \def\jobname{#1}
  \def\childdocjob{#1}
  \input{#1}
}
%    \end{macrocode}

% \macro{\childdocby}
% The command |\childdocby| ....
%    \begin{macrocode}
\newcommand{\childdocby}[2][]
{
  \childdocdisable
  \childdoctrue
  \childdocmanualtrue
  \if?#1?\else
    \def\jobname{#2}
  \fi
  \def\childdocjob{#2}
  \input{#2}
  \endinput
}
%    \end{macrocode}

% \macro{\childdocforward}
% The command |\childdocforward| redirects
% compilation to the main file or
% (if the optional argument is given) a child file.
% Parameters are set as if the main file
% or a child file starting with |\childdocof| was compiled.
% Then compilation is handed over to the main file:
%    \begin{macrocode}
\newcommand{\childdocforward}[2][]
{
  \begingroup
    \if?#1?
      \def\childdoctmp
      {
        \def\childdocname{#2}
        \def\childdocjob{#2}
        \def\jobname{#2}
        \input{#2}
        \endinput
      }
    \else
      \def\childdoctmp
      {
        \childdocdisable
        \def\childdocname{#2}
        \childdoctrue
        \includeonly{#2}
        \def\childdocjob{#1}
        \def\jobname{#1}
        \input{#1}
        \endinput
      }
    \fi
    \expandafter
  \endgroup
  \childdoctmp
}
%    \end{macrocode}

% \macro{\childdocforwardprefix}
% The command |\childdocforwardprefix| redirects
% compilation to the main or a child file by means of a pattern.
% The prefix |#1| in the current filename is replaced by |#2|
% and the suffix of the current filename is kept
% (it is assumed that the filename does not contain the substring `|~~~|'
% which is used as a delimiter).
% Compilation is handed over to the new file by |\childdocforward|:
%    \begin{macrocode}
\newcommand{\childdocforwardprefix}[3][]
{
  \begingroup
    \def\childdocextract #2##1~~~{\def\childdoctmp{\childdocforward[#1]{#3##1}}}
    \expandafter\childdocextract\childdocname~~~
    \expandafter
  \endgroup
  \childdoctmp
}
%    \end{macrocode}

% \macro{\childdoc}
% The deprecated macro |\childdoc| is a legacy version of |\childdocmain|:
%    \begin{macrocode}
\newcommand{\childdoc}{\childdocmain}
%    \end{macrocode}

% \macro{\childdocredirect}
% The deprecated macro |\childdocredirect| is a legacy version
% of |\childdocforward| and |\childdocforwardprefix|:
%    \begin{macrocode}
\newcommand{\childdocredirect}[2][]
{
  \begingroup
    \if?#1?
      \def\childdoctmp{\childdocforward{#2}}
    \else
      \def\childdoctmp{\childdocforwardprefix{#1}{#2}}
    \fi
    \expandafter
  \endgroup
  \childdoctmp
}
%    \end{macrocode}

%\iffalse
%</package>
%\fi
%
\endinput
|\\
|\childdocforwardprefix[|\textit{main}|]{|\textit{prefix}|}{|\textit{dest}|}|
\end{tabular}
\end{center}
%
the destination file is determined by a pattern
depending on the current file:
To make this work, the current file must be called
`{\textit{prefix}\hspace{0.2em}\textit{suffix}}'
with \textit{prefix} matching precisely the argument.
Processing is then passed on to the file
`{\textit{dest}\hspace{0.2em}\textit{suffix}}'.
Surely, the same effect is achieved by
directly specifying the
argument `{\textit{dest}\hspace{0.2em}\textit{suffix}}'
in the first form.
However, that requires to set up a different file
for each child. With the alternative form of the command
all these files can have exactly the same content
which simplifies setting them up and maintaining them.

For example, the following file |draft.tex|
with a compilation flag |\version| as described in \secref{sec:flags}
compiles the main document as a draft:
%
\begin{center}
\begin{tabular}{l}
|\def\version{draft}|\\
|% \iffalse
%
% childdoc.dtx Copyright (C) 2017-2018 Niklas Beisert
%
% This work may be distributed and/or modified under the
% conditions of the LaTeX Project Public License, either version 1.3
% of this license or (at your option) any later version.
% The latest version of this license is in
%   http://www.latex-project.org/lppl.txt
% and version 1.3 or later is part of all distributions of LaTeX
% version 2005/12/01 or later.
%
% This work has the LPPL maintenance status `maintained'.
%
% The Current Maintainer of this work is Niklas Beisert.
%
% This work consists of the files childdoc.dtx and childdoc.ins
% and the derived files childdoc.def and cdocsamp.tex with
% cdocsch1.tex, cdocsch2.tex, cdocsdrf.tex, cdocsfn1.tex, cdocsfn2.tex.
%
%<package>\ifdefined\childdocmain\endinput\fi
%<package>\ProvidesFile{childdoc.def}[2018/12/30 v2.0 child document driver]
%<samplemain>\ProvidesFile{cdocsamp.tex}[2018/12/30 v2.0 sample for childdoc]
%<*driver>
%\ProvidesFile{childdoc.drv}[2018/12/30 v2.0 childdoc reference manual file]
\PassOptionsToClass{10pt,a4paper}{article}
\documentclass{ltxdoc}

\usepackage[margin=35mm]{geometry}
\usepackage{hyperref}
\usepackage{hyperxmp}
\usepackage[usenames]{color}

\hypersetup{colorlinks=true}
\hypersetup{pdfstartview=FitH}
\hypersetup{pdfpagemode=UseNone}
\hypersetup{pdfsource={}}
\hypersetup{pdflang={en-UK}}
\hypersetup{pdfcopyright={Copyright 2017-2018 Niklas Beisert.
  This work may be distributed and/or modified under the
  conditions of the LaTeX Project Public License, either version 1.3
  of this license or (at your option) any later version.}}
\hypersetup{pdflicenseurl={http://www.latex-project.org/lppl.txt}}
\hypersetup{pdfcontactaddress={ETH Zurich, ITP, HIT K,
  Wolfgang-Pauli-Strasse 27}}
\hypersetup{pdfcontactpostcode={8093}}
\hypersetup{pdfcontactcity={Zurich}}
\hypersetup{pdfcontactcountry={Switzerland}}
\hypersetup{pdfcontactemail={nbeisert@itp.phys.ethz.ch}}
\hypersetup{pdfcontacturl={http://people.phys.ethz.ch/\xmptilde nbeisert/}}

\newcommand{\secref}[1]{\hyperref[#1]{section \ref*{#1}}}

\parskip1ex
\parindent0pt
\let\olditemize\itemize
\def\itemize{\olditemize\parskip0pt}

\begin{document}

\title{The \textsf{childdoc} Package}
\hypersetup{pdftitle={The childdoc Package}}
\author{Niklas Beisert\\[2ex]
  Institut f\"ur Theoretische Physik\\
  Eidgen\"ossische Technische Hochschule Z\"urich\\
  Wolfgang-Pauli-Strasse 27, 8093 Z\"urich, Switzerland\\[1ex]
  \href{mailto:nbeisert@itp.phys.ethz.ch}
  {\texttt{nbeisert@itp.phys.ethz.ch}}}
\hypersetup{pdfauthor={Niklas Beisert}}
\hypersetup{pdfsubject={Manual for the LaTeX2e Package childdoc}}
\date{30 December 2018, \textsf{v2.0}}
\maketitle

\begin{abstract}\noindent
\textsf{childdoc} is a \LaTeXe{} package
that enables the direct compilation
of document sections included by |\include|
to individual files.
\end{abstract}

\begingroup
\parskip0ex
\tableofcontents
\endgroup

%%%%%%%%%%%%%%%%%%%%%%%%%%%%%%%%%%%%%%%%%%%%%%%%%%%%%%%%%%%%%%%%%%%%%%%%%%%%%%%%
%%%%%%%%%%%%%%%%%%%%%%%%%%%%%%%%%%%%%%%%%%%%%%%%%%%%%%%%%%%%%%%%%%%%%%%%%%%%%%%%
\section{Introduction}

\LaTeX{} provides a mechanism to structure a large document (such as a book)
into a main file and several child files (containing the chapters)
using the |\include| command.
This mechanism is beneficial for documents
which span hundreds of pages in order to
make the source file(s) more manageable.
Moreover, compilation can be restricted to
selected child files by means of the |\includeonly| command.
The latter feature can be used to reduce the compilation time while editing
(this was significantly more useful in the earlier days of \LaTeX{})
or to generate a smaller document which is easier to navigate.
Another application of |\includeonly| is to generate
documents consisting of selected parts of the complete document.

However, there are a few drawbacks of the plain |\include| mechanism:
\begin{itemize}
\item
The child files cannot be compiled on their own,
they can only be compiled via the main file.
A naive editing environment
(such as a text editor with an option
to have the current file processed by \LaTeX)
may require one to switch to the main file before compiling;
attempting to compile the child file produces errors.
\item
The main file must be modified (each time)
to adjust the |\includeonly| command
to the present needs. This easily leaves the main file in a messy state.
\item
The generated document will always carry the filename
of the main document. This is inconvenient if
several child files are to be compiled and
to be kept for distribution.
\end{itemize}

The present package provides a simple interface
to make child files individually compilable by \LaTeX{}.
Compiling a child file then has the same effect as compiling
the main file with an |\includeonly| command
to select the appropriate child.
Moreover the generated document will carry the name of the child
rather than the main file.
This resolves all three above issues.

This feature is meant to make the editing of books,
thesis documents and lecture notes somewhat more convenient.
However, the package can also be used efficiently for
composing a series of documents (such as exercise sheets)
which are typically distributed individually.
It then assists the author in generating the individual documents
(potentially in different versions)
as well as a document containing the collected series.
Another application is in developing style files
or other kinds of included material
where compilation of the style file could redirect
to a sample or test file.

%%%%%%%%%%%%%%%%%%%%%%%%%%%%%%%%%%%%%%%%%%%%%%%%%%%%%%%%%%%%%%%%%%%%%%%%%%%%%%%%
%%%%%%%%%%%%%%%%%%%%%%%%%%%%%%%%%%%%%%%%%%%%%%%%%%%%%%%%%%%%%%%%%%%%%%%%%%%%%%%%
\section{Usage}

First of all, the package \textsf{childdoc} is \emph{not} a standard
\LaTeXe{} |.sty| style file! Therefore it needs to be invoked in
a non-standard way.

%%%%%%%%%%%%%%%%%%%%%%%%%%%%%%%%%%%%%%%%%%%%%%%%%%%%%%%%%%%%%%%%%%%%%%%%%%%%%%%%
\subsection{Included Files}
\label{sec:include}

%%%%%%%%%%%%%%%%%%%%%%%%%%%%%%%%%%%%%%%%
\DescribeMacro{\childdocmain}
To use the package, add the commands
\begin{center}
\begin{tabular}{l}
|\input{childdoc.def}|\\
|\childdocmain{}|\\
\end{tabular}
\end{center}
at the very top of the main \LaTeX{} file,
in particular \emph{before} the |\documentclass| statement!
The argument of |\childdocmain| should be left empty
(but it must be present).

%%%%%%%%%%%%%%%%%%%%%%%%%%%%%%%%%%%%%%%%
\DescribeMacro{\childdocof}
Furthermore, add the commands
\begin{center}
\begin{tabular}{l}
|\input{childdoc.def}|\\
|\childdocof{|\textit{main}|}|\\
\end{tabular}
\end{center}
at the top of every child file \textit{child}
which is included by |\include{|\textit{child}|}|
from within the main file
(or at least for those files to be compiled individually).
The argument \textit{main} must be the filename of the main file.

There are a couple of
considerations in setting up the main and child documents:

%%%%%%%%%%%%%%%%%%%%%%%%%%%%%%%%%%%%%%%%
\paragraph{Restrictions.}

Please note the following restrictions:
\begin{itemize}
\item
|\childdocmain| must be called with one argument \textit{main}
to ensure compatibility with earlier version of the package.
It must either be empty (|\childdocmain{}|)
or precisely match the filename of the main file in which it is specified.
See \secref{sec:detection} for further information.
\item
The filename \textit{main} must be specified without the |.tex| extension.
\item
The filename \textit{main} is case sensitive
(even in case-insensitive file systems)
due to internal string comparison.
\item
The argument \textit{main} should be fully expanded, it cannot be a macro.
\item
Subdirectories and special characters should be avoided in filenames.
\item
The command |\childdocmain{|\textit{main}|}| must be followed by a whitespace.
It should not be followed immediately by another command
or by a comment mark `|%|'.
This is because the \TeX{} parser reads the token immediately following
the argument of |\childdocmain| and puts it
at the beginning of every child section;
however, a white\-space is ignored.
\end{itemize}

%%%%%%%%%%%%%%%%%%%%%%%%%%%%%%%%%%%%%%%%
\paragraph{Content of Main File.}

It is advisable to place all content in the child files included by |\include|.
Any output contained in the main file will appear in all child documents
unless suppressed manually;
it cannot be suppressed automatically by the |\includeonly| directive
and thus should normally be avoided.
A method to include some content in the main file
by means of conditional processing is described in \secref{sec:conditional}.

%%%%%%%%%%%%%%%%%%%%%%%%%%%%%%%%%%%%%%%%
\paragraph{Page Numbering.}

When only a part of the document is compiled,
the appropriate numbering of pages
(as well as other status parameters)
is determined from the |.aux| files.
The latter contain information from previous passes.
However this information needs to propagate through
all intermediate child documents.
Therefore the page numbering in child documents may well
be inconsistent until the complete document is compiled at least once.

A useful (if unconventional) way to always ensure a consistent
page numbering is to restart the numbering in each child document
and denote the pages by `\textit{child}|.|\textit{page}'
where \textit{child} represents the chapter/section number of the child file.
This can be achieved by the command
|\numberwithin{page}{|\textit{child}|}|
of the \textsf{amsmath} package
where \textit{child} can be |chapter| or |section|
depending on the chosen structuring.
Alternatively, one can modify the macro |\thepage| appropriately
and reset the counter |page| at the start of each child file.

%%%%%%%%%%%%%%%%%%%%%%%%%%%%%%%%%%%%%%%%%%%%%%%%%%%%%%%%%%%%%%%%%%%%%%%%%%%%%%%%
\subsection{Conditional Processing}
\label{sec:conditional}

The package provides a mechanism to compile different versions
of a document. To customise the versions further some conditional processing
can come in handy to distinguish which version is being compiled.
The package provides two macros to describe the compilation context:

%%%%%%%%%%%%%%%%%%%%%%%%%%%%%%%%%%%%%%%%
\DescribeMacro{\ifchilddoc}
The conditional |\ifchilddoc| distinguishes between the compilation of
child documents and the main document:
%
\begin{center}
|\ifchilddoc |\textit{child-code}| |[|\||else |\textit{main-code}]| \||fi|
\end{center}

%%%%%%%%%%%%%%%%%%%%%%%%%%%%%%%%%%%%%%%%
\DescribeMacro{\childdocname}
\DescribeMacro{\childdocjob}
The macro |\childdocname| contains the filename (without extension)
of the main or child file being processed.
Note that |\childdocjob| will always contain the name of the main file.

%%%%%%%%%%%%%%%%%%%%%%%%%%%%%%%%%%%%%%%%
\paragraph{Title Page.}

Conditional processing can be used to include a title or banner page
in the main document when proper precautions are taken.
Importantly, the code in the main file should ensure that the page counter
(as well as other status parameters which are stored in the |.aux| files)
takes the same value after the conditional processing.
Otherwise the page numbers may take divergent values
depending on which part is compiled.

For example, a title page could be declared by:
%
\begin{center}
\begin{tabular}{l}
|\ifchilddoc\||else|\\
|\addtocounter{page}{-1}|\\
\textit{code for title page}\\
|\newpage|\\
|\||fi|
\end{tabular}
\end{center}
%
A banner page for the child documents can be generated by:
%
\begin{center}
\begin{tabular}{l}
|\ifchilddoc|\\
|\addtocounter{page}{-1}|\\
\textit{code for banner page}\\
|\newpage|\\
|\||fi|
\end{tabular}
\end{center}
%
Here one could write a message such as:
\begin{center}
|This is the part \childdocname{} of \childdocjob{}.|
\end{center}

%%%%%%%%%%%%%%%%%%%%%%%%%%%%%%%%%%%%%%%%%%%%%%%%%%%%%%%%%%%%%%%%%%%%%%%%%%%%%%%%
\subsection{Flags}
\label{sec:flags}

The package makes it easy to generate different versions
of the main or child documents.
To this end compilation flags can be defined
and assigned different default values.
They will be particularly useful in conjunction
with the forwarding mechanism described in \secref{sec:forward}.

For example, it may be useful to have a flag |\version|
which can be set to |draft| or |final|.
The document source will contain some conditional code
depending on the value of |\version|.
Suppose further, the flag should default to |final| for the main file
and to |draft| for child files
which is a natural assignment for editing the document.
This is achieved by placing the following code
in the preamble of the main document
(below the |\childdocmain| directive):
%
\begin{center}
\begin{tabular}{l}
|\ifchilddoc|\\
|\providecommand{\version}{draft}|\\
|\||else|\\
|\providecommand{\version}{final}|\\
|\||fi|
\end{tabular}
\end{center}
%
The definition by |\providecommand| makes sure
that previous definitions are not overwritten.
Further statements |\providecommand{\version}{...}|
can thus be added before the above code to override it.

For the main file, one might add a line
(between |\childdocmain| and the above block)
%
\begin{center}
|%\ifchilddoc\||else\providecommand{\version}{draft}\||fi|
\end{center}
%
which can be uncommented to produce a draft version.
Likewise one can add a line to the very top of a child file
(above the |\childdocof{|\textit{main}|}| directive)
%
\begin{center}
|%\providecommand{\version}{final}|
\end{center}
%
which can be uncommented to produce the final version of this child document.

%%%%%%%%%%%%%%%%%%%%%%%%%%%%%%%%%%%%%%%%%%%%%%%%%%%%%%%%%%%%%%%%%%%%%%%%%%%%%%%%
\subsection{Forwarding}
\label{sec:forward}

Different versions of the main or child documents
using compilation flags as described in \secref{sec:flags}
can be (permanently) stored in different files
for convenient compilation, viewing and distribution.
To this end, the package defines a command
to pass on compilation to a different file:

%%%%%%%%%%%%%%%%%%%%%%%%%%%%%%%%%%%%%%%%
\DescribeMacro{\childdocforward}
The command |\childdocforward| redirects processing to
another source file:
%
\begin{center}
\begin{tabular}{l}
|\input{childdoc.def}|\\
|\childdocforward[|\textit{main}|]{|\textit{dest}|}|\\
\end{tabular}
\end{center}
%
The argument \textit{dest} is the destination file
(without extension).
It should be the main file or one of the child files.
Note that further \textsf{childdoc} directives
such as |\childdocof| and |\childdocforward|
in the indicated file will be processed in this form.
The optional argument \textit{main}
passes on directly to the main file \textit{main}
while pretending to compile the child \textit{dest}.
This form behaves as if \textit{dest}
issues |\childdocof{|\textit{main}|}| right away,
and no further \textsf{childdoc} directives will be processed.

%%%%%%%%%%%%%%%%%%%%%%%%%%%%%%%%%%%%%%%%
\DescribeMacro{\...prefix}
In the alternative form |\childdocforwardprefix|,
%
\begin{center}
\begin{tabular}{l}
|\input{childdoc.def}|\\
|\childdocforwardprefix[|\textit{main}|]{|\textit{prefix}|}{|\textit{dest}|}|
\end{tabular}
\end{center}
%
the destination file is determined by a pattern
depending on the current file:
To make this work, the current file must be called
`{\textit{prefix}\hspace{0.2em}\textit{suffix}}'
with \textit{prefix} matching precisely the argument.
Processing is then passed on to the file
`{\textit{dest}\hspace{0.2em}\textit{suffix}}'.
Surely, the same effect is achieved by
directly specifying the
argument `{\textit{dest}\hspace{0.2em}\textit{suffix}}'
in the first form.
However, that requires to set up a different file
for each child. With the alternative form of the command
all these files can have exactly the same content
which simplifies setting them up and maintaining them.

For example, the following file |draft.tex|
with a compilation flag |\version| as described in \secref{sec:flags}
compiles the main document as a draft:
%
\begin{center}
\begin{tabular}{l}
|\def\version{draft}|\\
|\input{childdoc.def}|\\
|\childdocforward{|\textit{main}|}|
\end{tabular}
\end{center}
%
Likewise, the following files |final|\textit{nn}|.tex|
compile the final version of the child document
|child|\textit{nn}|.tex|:
%
\begin{center}
\begin{tabular}{l}
|\def\version{final}|\\
|\input{childdoc.def}|\\
|\childdocforwardprefix{final}{child}|
\end{tabular}
\end{center}
%

Note that when several versions of a main file and/or of each child file
are to be generated, it may be convenient to set up a |Makefile| or
shell script to automatise the process.

%%%%%%%%%%%%%%%%%%%%%%%%%%%%%%%%%%%%%%%%%%%%%%%%%%%%%%%%%%%%%%%%%%%%%%%%%%%%%%%%
\subsection{Command Line Processing}
\label{sec:commandline}

The effect of redirection files can also be achieved by invoking
the \LaTeX{} compiler with a more elaborate command line.
Most conveniently this should be done as part
of a shell script or a |Makefile|.

When using \textsf{childdoc} in the main file, the following
command lines effectively perform a redirection
(note that depending on the shell being used,
backslashes may have to be doubled: `|\|' $\to$ `|\\|'):
%
\begin{center}
|... -jobname "|\textit{target}|" |\\|"|[\textit{flags}]%
|\input{childdoc.def}\childdocforward[|\textit{main}|]{|\textit{dest}|}"|
\end{center}
%
Here \textit{target} is the name of the output file,
\textit{main} is the name of the main file
and \textit{dest} is the name of the main or child file to be processed
(all filenames without extensions).
The optional argument \textit{main} can be omitted
if \textit{main} matches \textit{dest}.
Optionally, compilation \textit{flags} can be defined via |\def| commands.
This command line makes the \TeX{} engine believe
it is compiling the file \textit{target}
whose content is specified as the latter parameter.
The provided code then forwards the processing to
\textit{main} or \textit{dest} as described in \secref{sec:forward}.

%%%%%%%%%%%%%%%%%%%%%%%%%%%%%%%%%%%%%%%%%%%%%%%%%%%%%%%%%%%%%%%%%%%%%%%%%%%%%%%%
\subsection{Include by Input}
\label{sec:input}

Including child documents by |\include| has some restrictions by design.
Most notably, the content of a child document always occupies
its own set of pages; pages cannot be shared between child documents.
Usually, this behaviour makes perfect sense
because each child document contain an essential part of the document.
However, in some situations it may be desirable to compose
a document from a collection of parts
without having mandatory page breaks between then.
For this case, the package
provides a mechanism to include parts
by |\input| which can also be processed individually.
However, by construction this mechanism
requires manual handling of the content to be output.

%%%%%%%%%%%%%%%%%%%%%%%%%%%%%%%%%%%%%%%%
\DescribeMacro{\ifchilddocmanual}
The main file should be prepared as usual, see \secref{sec:include}.
However, the document body must make a distinction
between processing of an individual part and of the main document, e.g.:
%
\begin{center}
\begin{tabular}{l}
|\ifchilddocmanual|\\
|\input{\childdocname}|\\
|\||else|\\
\textit{document body with }|\input{|\textit{part}|}|\\
|\||fi|
\end{tabular}
\end{center}
%
The conditional |\ifchilddocmanual| is true whenever
a part to be included by |\input| is being compiled,
and the name of the part is stored in |\childdocname|.

%%%%%%%%%%%%%%%%%%%%%%%%%%%%%%%%%%%%%%%%
\DescribeMacro{\childdocby}
Each part to be included by |\input| should start with:
%
\begin{center}
\begin{tabular}{l}
|\input{childdoc.def}|\\
|\childdocby{|\textit{main}|}|\\
\end{tabular}
\end{center}
%
The directive |\childdocby| is similar to |\childdocof|
described in \secref{sec:include},
but the subsequent selection of content must be done manually.
To that end, both |\ifchilddoc| and |\ifchilddocmanual|
will be true upon processing of a part,
and the name of the part is stored in |\childdocname|.
Note that |\jobname| will be set to the filename of the current part
so that each part receives an individual |.aux| file
that does not interfere with the |.aux| file(s) of the main document.
This behaviour can be altered by the alternative form
|\childdocby[*]{|\textit{main}|}| (with a non-empty optional argument)
which uses the |.aux| file of the main document
by setting |\jobname| to \textit{main}.

%%%%%%%%%%%%%%%%%%%%%%%%%%%%%%%%%%%%%%%%%%%%%%%%%%%%%%%%%%%%%%%%%%%%%%%%%%%%%%%%
\subsection{Driver Development}
\label{sec:driver}

The \textsf{childdoc} mechanism can also be use for the development
of definition files such as \LaTeX{} styles or classes.
This case differs from the above setup with multiple parts
included by |\include| in that no |\includeonly| should be invoked.
This can be achieved by starting the include file
(before |\ProvidesPackage|) with:
%
\begin{center}
\begin{tabular}{l}
|\input{childdoc.def}|\\
|\childdocforward{|\textit{main}|}|\\
\end{tabular}
\end{center}
%
or alternatively with:
%
\begin{center}
\begin{tabular}{l}
|\input{childdoc.def}|\\
|\childdocby{|\textit{main}|}|\\
\end{tabular}
\end{center}
%
Both forms have slightly different effects as described above.
The main file is prepared as usual, see \secref{sec:include}.

%%%%%%%%%%%%%%%%%%%%%%%%%%%%%%%%%%%%%%%%%%%%%%%%%%%%%%%%%%%%%%%%%%%%%%%%%%%%%%%%
\subsection{Legacy Detection}
\label{sec:detection}

The directive |\childdocmain| in the main file can detect
whether the complete document or merely a child is to be compiled
even without using the directive |\childdocof|.
This method is deprecated because it is less robust
and there is no compelling reason to use it;
it is merely provided for backward compatibility
and it may be removed in future versions.

If the detection mechanism is to be used,
it is mandatory to correctly specify
the filename of the main file as the argument of |\childdocmain|:
%
\begin{center}
\begin{tabular}{l}
|\input{childdoc.def}|\\
|\childdocmain{|\textit{main}|}|\\
\end{tabular}
\end{center}
%
If |\jobname| does not match the argument \textit{main} of |\childdocmain|,
it is assumed that |\jobname| points to the child file to be compiled.
When using |\childdocmain| with the main file specified as argument,
it suffices to start a child file
with just |\input{|\textit{main}|}|
without loading of the package and using |\childdocof|.
If instead all processing is done
with the appropriate \textsf{childdoc} directives,
the argument of \textit{main} of |\childdocmain| can be empty.

An alternative version of the command line processing described
in \secref{sec:commandline} using the detection mechanism reads:
%
\begin{center}
|... -jobname "|\textit{target}|" "|[\textit{flags}]%
[|\def\jobname{|\textit{dest}|}|]|\input{|\textit{main}|}"|
\end{center}

%%%%%%%%%%%%%%%%%%%%%%%%%%%%%%%%%%%%%%%%%%%%%%%%%%%%%%%%%%%%%%%%%%%%%%%%%%%%%%%%
\subsection{Manual Code}
\label{sec:manual}

In case one cannot be certain whether the definitions file |childdoc.def|
is installed on the target \TeX{} distribution
and one prefers not to ship it,
it is conceivable to paste a few relevant commands into the sources.

To that end, drop all statements |\input{childdoc.def}|
and perform the replacements as outlined below.
Instead of |\childdocmain{|\textit{main}|}| add the following code
to the top of the main file:
%
\begin{center}
\begin{tabular}{l}
|\||ifdefined\childdocname\endinput\||fi\newif\ifchilddoc|\\
|\edef\childdocname{\scantokens\expandafter{\jobname\noexpand}}|\\
|\def\childdocmain{|\textit{main}|}\||ifx\childdocmain\childdocname\||else|\\
|\childdoctrue\includeonly{\childdocname}\let\jobname\childdocmain\||fi|\\
\end{tabular}
\end{center}
%
Instead of |\childdocof{|\textit{main}|}| just include the main file
at the top of each child file:
%
\begin{center}
|\input{|\textit{main}|}|
\end{center}
%
A simple redirection |\childdocforward{|\textit{dest}|}| is achieved by:
%
\begin{center}
|\def\jobname{|\textit{dest}|}\input{\jobname}|
\end{center}
%
The redirection with prefix
|\childdocforwardprefix[|\textit{prefix}|]{|\textit{dest}|}|
is accomplished by:
%
\begin{center}
\begin{tabular}{l}
|{\edef\jobname{\scantokens\expandafter{\jobname\noexpand}}|\\
|\def\redirectjob |\textit{prefix}|#1~~~{\gdef\jobname{|\textit{dest}|#1}}|\\
|\expandafter\redirectjob\jobname~~~}\input{\jobname}|
\end{tabular}
\end{center}

In an alternative approach,
child documents can be compiled by a specific command line
without additional code or specific definitions:
%
\begin{center}
|... -jobname "|\textit{target}|" "|[\textit{flags}]%
|\includeonly{|\textit{dest}|}\input{|\textit{main}|}"|
\end{center}
%

%%%%%%%%%%%%%%%%%%%%%%%%%%%%%%%%%%%%%%%%%%%%%%%%%%%%%%%%%%%%%%%%%%%%%%%%%%%%%%%%
%%%%%%%%%%%%%%%%%%%%%%%%%%%%%%%%%%%%%%%%%%%%%%%%%%%%%%%%%%%%%%%%%%%%%%%%%%%%%%%%
\section{Information}

%%%%%%%%%%%%%%%%%%%%%%%%%%%%%%%%%%%%%%%%%%%%%%%%%%%%%%%%%%%%%%%%%%%%%%%%%%%%%%%%
\subsection{Copyright}

Copyright \copyright{} 2017--2018 Niklas Beisert

This work may be distributed and/or modified under the
conditions of the \LaTeX{} Project Public License, either version 1.3
of this license or (at your option) any later version.
The latest version of this license is in
  \url{http://www.latex-project.org/lppl.txt}
and version 1.3 or later is part of all distributions of \LaTeX{}
version 2005/12/01 or later.

This work has the LPPL maintenance status `maintained'.

The Current Maintainer of this work is Niklas Beisert.

This work consists of the files |README.txt|, |childdoc.ins| and |childdoc.dtx|
as well as the derived files |childdoc.def|, |cdocsamp.tex|
with |cdocsch1.tex|, |cdocsch2.tex|, |cdocspt3.tex|, |cdocspt4.tex|,
|cdocsdrf.tex|, |cdocsfn1.tex|, |cdocsfn2.tex|
as well as |childdoc.pdf|.

%%%%%%%%%%%%%%%%%%%%%%%%%%%%%%%%%%%%%%%%%%%%%%%%%%%%%%%%%%%%%%%%%%%%%%%%%%%%%%%%
\subsection{Files and Installation}

The package consists of the files:
%
\begin{center}
\begin{tabular}{ll}
    |README.txt|   & readme file \\
    |childdoc.ins| & installation file \\
    |childdoc.dtx| & source file \\
    |childdoc.def| & definition file \\
    |cdocsamp.tex| & sample main file \\
    |cdocsch1.tex| & sample include file \\
    |cdocsch2.tex| & sample include file \\
    |cdocspt3.tex| & sample part file \\
    |cdocspt4.tex| & sample part file \\
    |cdocsdrf.tex| & sample redirection file \\
    |cdocsfn1.tex| & sample redirection file \\
    |cdocsfn2.tex| & sample redirection file \\
    |childdoc.pdf| & manual
\end{tabular}
\end{center}
%
The distribution consists of the files
|README.txt|, |childdoc.ins| and |childdoc.dtx|.
%
\begin{itemize}
\item
Run (pdf)\LaTeX{} on |childdoc.dtx|
to compile the manual |childdoc.pdf| (this file).
\item
Run \LaTeX{} on |childdoc.ins| to create the definitions file |childdoc.def|
and the sample |cdocsamp.tex| with include files
|cdocsch1.tex|, |cdocsch2.tex|, |cdocspt3.tex|, |cdocspt4.tex|,
|cdocsdrf.tex|, |cdocsfn1.tex|, |cdocsfn2.tex|.
Then copy the file |childdoc.def| to an appropriate directory of your \LaTeX{}
distribution, e.g.\ \textit{texmf-root}|/tex/latex/childdoc|.
\end{itemize}

%%%%%%%%%%%%%%%%%%%%%%%%%%%%%%%%%%%%%%%%%%%%%%%%%%%%%%%%%%%%%%%%%%%%%%%%%%%%%%%%
\subsection{Related CTAN Packages}

There are several other packages which offer a similar functionality:
%
\begin{itemize}
\item
The packages
\href{http://ctan.org/pkg/docmute}{\textsf{docmute}},
\href{http://ctan.org/pkg/includex}{\textsf{includex}} and
\href{http://ctan.org/pkg/standalone}{\textsf{standalone}}
provide commands to include only the document body of
a child file thus allowing both files to be compiled individually.
\item
The packages \href{http://ctan.org/pkg/subdocs}{\textsf{subdocs}}
and \href{http://ctan.org/pkg/subfiles}{\textsf{subfiles}}
provide structures in which the main and child documents can be
encapsulated and allowing them to be compiled individually.
The inclusion mechanism is different from the conventional |\include|.
\item
The package \href{http://ctan.org/pkg/combine}{\textsf{combine}}
is an elaborate solution to combine several documents into one.
\end{itemize}
%
See also the CTAN topic \href{http://ctan.org/topic/subdocs}{\textsf{subdocs}}
for further related packages.
The present package differs from the above solutions in that
a document structure constructed with the conventional |\include| mechanism
just needs two extra commands at the top of every file
such that all constituent files can be compiled individually.

%%%%%%%%%%%%%%%%%%%%%%%%%%%%%%%%%%%%%%%%%%%%%%%%%%%%%%%%%%%%%%%%%%%%%%%%%%%%%%%%
%\subsection{Feature Suggestions}
%
%The following is a list of features which may be useful for future
%versions of this package:
%%
%\begin{itemize}
%\item
%\ldots
%\end{itemize}

%%%%%%%%%%%%%%%%%%%%%%%%%%%%%%%%%%%%%%%%%%%%%%%%%%%%%%%%%%%%%%%%%%%%%%%%%%%%%%%%
\subsection{Revision History}

%%%%%%%%%%%%%%%%%%%%%%%%%%%%%%%%%%%%%%%%
\paragraph{v2.0:} 2018/12/30

\begin{itemize}
\item
immediate forward processing
\item
added |\childdocby| mechanism
\item
manual restructured
\end{itemize}

%%%%%%%%%%%%%%%%%%%%%%%%%%%%%%%%%%%%%%%%
\paragraph{v1.6:} 2018/01/17

\begin{itemize}
\item
application for development of include files
\item
corrections to manual
\end{itemize}

%%%%%%%%%%%%%%%%%%%%%%%%%%%%%%%%%%%%%%%%
\paragraph{v1.5:} 2017/05/21

\begin{itemize}
\item
more complete structuring introduced
\item
|\childdocof| introduced
\item
|\childdoc| renamed to |\childdocmain|
\item
|\childredirect| renamed to |\childdocforward| and |\childdocforwardprefix|
and functionality expanded
\end{itemize}

%%%%%%%%%%%%%%%%%%%%%%%%%%%%%%%%%%%%%%%%
\paragraph{v1.0:} 2017/04/27

\begin{itemize}
\item
manual and install package
\item
first version published on CTAN
\end{itemize}

%%%%%%%%%%%%%%%%%%%%%%%%%%%%%%%%%%%%%%%%
\paragraph{v0.6:} 2017/04/26

\begin{itemize}
\item
redirection mechanism added
\end{itemize}

%%%%%%%%%%%%%%%%%%%%%%%%%%%%%%%%%%%%%%%%
\paragraph{v0.5:} 2017/04/26

\begin{itemize}
\item
functionality in definition file
\end{itemize}


%%%%%%%%%%%%%%%%%%%%%%%%%%%%%%%%%%%%%%%%%%%%%%%%%%%%%%%%%%%%%%%%%%%%%%%%%%%%%%%%
%%%%%%%%%%%%%%%%%%%%%%%%%%%%%%%%%%%%%%%%%%%%%%%%%%%%%%%%%%%%%%%%%%%%%%%%%%%%%%%%
%%%%%%%%%%%%%%%%%%%%%%%%%%%%%%%%%%%%%%%%%%%%%%%%%%%%%%%%%%%%%%%%%%%%%%%%%%%%%%%%
\appendix

\settowidth\MacroIndent{\rmfamily\scriptsize 000\ }

 \DocInput{childdoc.dtx}

\end{document}
%</driver>
% \fi
%
% %%%%%%%%%%%%%%%%%%%%%%%%%%%%%%%%%%%%%%%%%%%%%%%%%%%%%%%%%%%%%%%%%%%%%%%%%%%%%%
% %%%%%%%%%%%%%%%%%%%%%%%%%%%%%%%%%%%%%%%%%%%%%%%%%%%%%%%%%%%%%%%%%%%%%%%%%%%%%%
% \section{Sample}
%\iffalse
%<*samplemain>
%\fi
%
% The following presents a sample document
% with two chapters, two parts, a title page,
% a compile flag as well as three forwarding files to set the flag.
% It consists of eight |.tex| files:
% \begin{center}
% \begin{tabular}{ll}
% |cdocsamp.tex|&main file\\
% |cdocsch1.tex|&include file for chapter 1\\
% |cdocsch2.tex|&include file for chapter 2\\
% |cdocspt3.tex|&include file for part 3\\
% |cdocspt4.tex|&include file for part 4\\
% |cdocsdrf.tex|&forwarding file for main file in draft mode\\
% |cdocsfi1.tex|&forwarding file for final version of chapter 1\\
% |cdocsfi2.tex|&forwarding file for final version of chapter 2\\
% \end{tabular}
% \end{center}
% Each of the eight files can be compiled directly by the \LaTeX{} compiler.
%
% %%%%%%%%%%%%%%%%%%%%%%%%%%%%%%%%%%%%%%
% \paragraph{Main File.}
%
% The main file is called |cdocsamp.tex|.
%
% Load the \textsf{childdoc} definitions and
% declare the filename for the main document:
%    \begin{macrocode}
\input{childdoc.def}
\childdocmain{}
%    \end{macrocode}

% Optional override for |\version| flag:
%    \begin{macrocode}
%%\ifchilddoc\else\providecommand{\version}{draft}\fi
%    \end{macrocode}

% Define the default values for the |\version| flag
% (|final| for the main file and |draft| for childs):
%    \begin{macrocode}
\ifchilddoc
\providecommand{\version}{draft}
\else
\providecommand{\version}{final}
\fi
%    \end{macrocode}

% Load the standard document class:
%    \begin{macrocode}
\documentclass[12pt]{article}
%    \end{macrocode}

% Start the document body:
%    \begin{macrocode}
\begin{document}
%    \end{macrocode}

% Declare a title page.
% Print title, part of document being processed and version flag:
%    \begin{macrocode}
\addtocounter{page}{-1}
\begin{center}
{\LARGE\bfseries{}childdoc example\par}
\vspace{1cm}
\ifchilddoc
\ifchilddocmanual part\else chapter\fi:
`\childdocname' of `\childdocjob'\par
\else
main document: `\childdocjob'\par
\fi
version: \version\par
\end{center}
\newpage
%    \end{macrocode}

% Manually include selected file,
% otherwise process as usual:
%    \begin{macrocode}
\ifchilddocmanual
\section*{part `\childdocname'}
\input{\childdocname}
\else
%    \end{macrocode}

% Include the two chapters:
%    \begin{macrocode}
\include{cdocsch1}
\include{cdocsch2}
%    \end{macrocode}

% Include the two parts unless only chapters should be displayed:
%    \begin{macrocode}
\ifchilddoc\else
\section{part three}
\input{cdocspt3}
\section{part four}
\input{cdocspt4}
\fi
%    \end{macrocode}

% Process as usual until here:
%    \begin{macrocode}
\fi
%    \end{macrocode}

% End of document body:
%    \begin{macrocode}
\end{document}
%    \end{macrocode}
%\iffalse
%</samplemain>
%\fi
%
% %%%%%%%%%%%%%%%%%%%%%%%%%%%%%%%%%%%%%%
% \paragraph{Chapter Include Files.}
%
% The include files are called |cdocsch1.tex| and |cdocsch2.tex|.
%
%\iffalse
%<*samplechap1|samplechap2>
%\fi

% Optional override for |\version| flag:
%    \begin{macrocode}
%%\providecommand{\version}{final}
%    \end{macrocode}

% Include the main document:
%    \begin{macrocode}
\input{childdoc.def}
\childdocof{cdocsamp}
%    \end{macrocode}

%\iffalse
%</samplechap1|samplechap2>
%\fi
%
%\iffalse
%<*samplechap1>
%\fi
% Some text for chapter 1:
%    \begin{macrocode}
\section{one}
some text in chapter one
%    \end{macrocode}

%\iffalse
%</samplechap1>
%\fi
% Some text for chapter 2:
%\iffalse
%<*samplechap2>
%\fi
%    \begin{macrocode}
\section{two}
more text in chapter two
%    \end{macrocode}

%\iffalse
%</samplechap2>
%\fi
%
% %%%%%%%%%%%%%%%%%%%%%%%%%%%%%%%%%%%%%%
% \paragraph{Part Include Files.}
%
% The include files are called |cdocspt3.tex| and |cdocspt4.tex|.
%
%\iffalse
%<*samplepart3|samplepart4>
%\fi

% Optional override for |\version| flag:
%    \begin{macrocode}
%%\providecommand{\version}{final}
%    \end{macrocode}

% Include the main document:
%    \begin{macrocode}
\input{childdoc.def}
\childdocby{cdocsamp}
%    \end{macrocode}

%\iffalse
%</samplepart3|samplepart4>
%\fi
%
%\iffalse
%<*samplepart3>
%\fi
% Some text for part 3:
%    \begin{macrocode}
some text in part three
%    \end{macrocode}

%\iffalse
%</samplepart3>
%\fi
% Some text for part 4:
%\iffalse
%<*samplepart4>
%\fi
%    \begin{macrocode}
more text in part four
%    \end{macrocode}

%\iffalse
%</samplepart4>
%\fi
%
% %%%%%%%%%%%%%%%%%%%%%%%%%%%%%%%%%%%%%%
% \paragraph{Forwarding for a Complete Draft.}
%
% The following forwarding file |cdocsdrf.tex|
% compiles the main document in draft mode:
%\iffalse
%<*sampledraft>
%\fi
%    \begin{macrocode}
\def\version{draft}
\input{childdoc.def}
\childdocforward{cdocsamp}
%    \end{macrocode}

%\iffalse
%</sampledraft>
%\fi
%
% %%%%%%%%%%%%%%%%%%%%%%%%%%%%%%%%%%%%%%
% \paragraph{Forwarding for Final Version of the Chapters.}
%
% The following forwarding files |cdocsfn1.tex| and |cdocsfn2.tex|
% (with identical content)
% compile the final versions of the child documents
% |cdocsch1.tex| and |cdocsch2.tex|, respectively:
%\iffalse
%<*samplefinal>
%\fi
%    \begin{macrocode}
\def\version{final}
\input{childdoc.def}
\childdocforwardprefix[cdocsamp]{cdocsfn}{cdocsch}
%    \end{macrocode}

%\iffalse
%</samplefinal>
%\fi
%
% %%%%%%%%%%%%%%%%%%%%%%%%%%%%%%%%%%%%%%
% \paragraph{Command Line Processing.}
%
% The following three command lines generate the output files
% |cdocscld|, |cdocscl1| and |cdocscl2|
% which should be identical to
% |cdocsdrf|, |cdocsch1| and |cdocsfn2|, respectively:
% \begin{center}
% \begin{tabular}{l}
% |latex -jobname cdocscld \|\\
% |  "\def\version{draft}\input{childdoc.def}\childdocforward{cdocsamp}"|\\
% |latex -jobname cdocscl1 \|\\
% |  "\input{childdoc.def}\childdocforward[cdocsamp]{cdocsch1}"|\\
% |latex -jobname cdocscl2 \|\\
% |  "\def\version{final}\input{childdoc.def}\childdocforward{cdocsch2}"|
% \end{tabular}
% \end{center}
% Note that the trailing backslash on each first line
% merely continues the input to the second line
% (for convenient cut ant paste).
% Furthermore, the command |latex| can be replaced by any
% of its alternative versions such as |pdflatex|.
%
% %%%%%%%%%%%%%%%%%%%%%%%%%%%%%%%%%%%%%%%%%%%%%%%%%%%%%%%%%%%%%%%%%%%%%%%%%%%%%%
% %%%%%%%%%%%%%%%%%%%%%%%%%%%%%%%%%%%%%%%%%%%%%%%%%%%%%%%%%%%%%%%%%%%%%%%%%%%%%%
% \section{Implementation}
%\iffalse
%<*package>
%\fi
%
% This section describes the definitions file |childdoc.def|.

% The definitions cannot be loaded using |\usepackage| or |\RequirePackage|
% which has a mechanism to prevent loading a style file more than once.
% When loading the definitions by means of |\input|
% multiple instances have to be prevented manually:
%\iffalse
%This code needs to be before the `\ProvidesFile' directive
%which is defined at the beginning of this file.
%Therefore it is also placed there and commented out here.
%</package>
%<*discard>
%\fi
%    \begin{macrocode}
\ifdefined\childdocmain\endinput\fi
%    \end{macrocode}
%\iffalse
%</discard>
%<*package>
%\fi
%
% \macro{\ifchilddoc}
% \macro{\ifchilddocmanual}
% The conditional |\ifchilddoc| tells whether a
% child (true) or main (false) document is being compiled.
% The conditional |\ifchilddocmanual| tells whether
% the |\includeonly| mechanism is used (false) or
% the selection of child files must be performed manually (true).
% The definitions initialise to false:
%    \begin{macrocode}
\newif\ifchilddoc
\newif\ifchilddocmanual
%    \end{macrocode}

% \macro{\childdocname}
% \macro{\childdocjob}
% The macro |\childdocname| stores the name of the main document
% to be compiled. The macro |\childdocjob| stores the name of
% the document on which the \LaTeX{} compiler was originally invoked.
% The content of |\jobname| cannot be compared
% to filenames specified in the source due to different catcodes.
% The following code rescans |\jobname|, stores the result
% in |\childdocname| and saves a copy in |\childdocjob|:
%    \begin{macrocode}
\edef\childdocname{\scantokens\expandafter{\jobname\noexpand}}
\let\childdocjob\childdocname
%    \end{macrocode}

% \macro{\childdocdisable}
% The macro |\childdocdisable| prevents the main file
% from being processed more than once.
% At this stage, the main document command |\childdocmain|
% is assumed to be called once again where it should do nothing.
% Any subsequent call to it should prevent
% a secondary processing of the main document
% It overwrites the forwarding commands
% |\childdocof| and |\childdocforward|
% with empty macros to prevent further inclusions of the main document:
%    \begin{macrocode}
\newcommand{\childdocdisable}
{
  \renewcommand{\childdocmain}[1]{\renewcommand{\childdocmain}[1]{\endinput}}
  \renewcommand{\childdocof}[1]{}
  \renewcommand{\childdocby}[2][]{}
  \renewcommand{\childdocforward}[2][]{}
  \renewcommand{\childdocdisable}{}
}
%    \end{macrocode}

% \macro{\childdocmain}
% The macro |\childdocmain| is to be called at the top of the main file
% with nothing or the main filename (without extension) as argument.
% First, it breaks loops.
% If the argument is not empty and does not match |\childdocname|
% (which is set by the first inclusion of |childdoc.def|),
% |\ifchilddoc| is set to true, |\includeonly| is applied to the child file
% and |\jobname| is set to the main file
% (for proper handling of |.aux| files):
%    \begin{macrocode}
\newcommand{\childdocmain}[1]
{
  \childdocdisable\childdocmain{}
  \if?#1?\else
    \begingroup
      \def\childdoctmp{#1}
      \ifx\childdoctmp\childdocname
        \def\childdoctmp{}
      \else
        \def\childdoctmp
        {
          \childdoctrue
          \includeonly{\childdocname}
          \def\childdocjob{#1}
          \def\jobname{#1}
        }
      \fi
      \expandafter
    \endgroup
    \childdoctmp
  \fi
}
%    \end{macrocode}

% \macro{\childdocof}
% The command |\childdocof| redirects
% compilation to the main file |#1|.
%    \begin{macrocode}
\newcommand{\childdocof}[1]
{
  \childdocdisable
  \childdoctrue
  \includeonly{\childdocname}
  \def\jobname{#1}
  \def\childdocjob{#1}
  \input{#1}
}
%    \end{macrocode}

% \macro{\childdocby}
% The command |\childdocby| ....
%    \begin{macrocode}
\newcommand{\childdocby}[2][]
{
  \childdocdisable
  \childdoctrue
  \childdocmanualtrue
  \if?#1?\else
    \def\jobname{#2}
  \fi
  \def\childdocjob{#2}
  \input{#2}
  \endinput
}
%    \end{macrocode}

% \macro{\childdocforward}
% The command |\childdocforward| redirects
% compilation to the main file or
% (if the optional argument is given) a child file.
% Parameters are set as if the main file
% or a child file starting with |\childdocof| was compiled.
% Then compilation is handed over to the main file:
%    \begin{macrocode}
\newcommand{\childdocforward}[2][]
{
  \begingroup
    \if?#1?
      \def\childdoctmp
      {
        \def\childdocname{#2}
        \def\childdocjob{#2}
        \def\jobname{#2}
        \input{#2}
        \endinput
      }
    \else
      \def\childdoctmp
      {
        \childdocdisable
        \def\childdocname{#2}
        \childdoctrue
        \includeonly{#2}
        \def\childdocjob{#1}
        \def\jobname{#1}
        \input{#1}
        \endinput
      }
    \fi
    \expandafter
  \endgroup
  \childdoctmp
}
%    \end{macrocode}

% \macro{\childdocforwardprefix}
% The command |\childdocforwardprefix| redirects
% compilation to the main or a child file by means of a pattern.
% The prefix |#1| in the current filename is replaced by |#2|
% and the suffix of the current filename is kept
% (it is assumed that the filename does not contain the substring `|~~~|'
% which is used as a delimiter).
% Compilation is handed over to the new file by |\childdocforward|:
%    \begin{macrocode}
\newcommand{\childdocforwardprefix}[3][]
{
  \begingroup
    \def\childdocextract #2##1~~~{\def\childdoctmp{\childdocforward[#1]{#3##1}}}
    \expandafter\childdocextract\childdocname~~~
    \expandafter
  \endgroup
  \childdoctmp
}
%    \end{macrocode}

% \macro{\childdoc}
% The deprecated macro |\childdoc| is a legacy version of |\childdocmain|:
%    \begin{macrocode}
\newcommand{\childdoc}{\childdocmain}
%    \end{macrocode}

% \macro{\childdocredirect}
% The deprecated macro |\childdocredirect| is a legacy version
% of |\childdocforward| and |\childdocforwardprefix|:
%    \begin{macrocode}
\newcommand{\childdocredirect}[2][]
{
  \begingroup
    \if?#1?
      \def\childdoctmp{\childdocforward{#2}}
    \else
      \def\childdoctmp{\childdocforwardprefix{#1}{#2}}
    \fi
    \expandafter
  \endgroup
  \childdoctmp
}
%    \end{macrocode}

%\iffalse
%</package>
%\fi
%
\endinput
|\\
|\childdocforward{|\textit{main}|}|
\end{tabular}
\end{center}
%
Likewise, the following files |final|\textit{nn}|.tex|
compile the final version of the child document
|child|\textit{nn}|.tex|:
%
\begin{center}
\begin{tabular}{l}
|\def\version{final}|\\
|% \iffalse
%
% childdoc.dtx Copyright (C) 2017-2018 Niklas Beisert
%
% This work may be distributed and/or modified under the
% conditions of the LaTeX Project Public License, either version 1.3
% of this license or (at your option) any later version.
% The latest version of this license is in
%   http://www.latex-project.org/lppl.txt
% and version 1.3 or later is part of all distributions of LaTeX
% version 2005/12/01 or later.
%
% This work has the LPPL maintenance status `maintained'.
%
% The Current Maintainer of this work is Niklas Beisert.
%
% This work consists of the files childdoc.dtx and childdoc.ins
% and the derived files childdoc.def and cdocsamp.tex with
% cdocsch1.tex, cdocsch2.tex, cdocsdrf.tex, cdocsfn1.tex, cdocsfn2.tex.
%
%<package>\ifdefined\childdocmain\endinput\fi
%<package>\ProvidesFile{childdoc.def}[2018/12/30 v2.0 child document driver]
%<samplemain>\ProvidesFile{cdocsamp.tex}[2018/12/30 v2.0 sample for childdoc]
%<*driver>
%\ProvidesFile{childdoc.drv}[2018/12/30 v2.0 childdoc reference manual file]
\PassOptionsToClass{10pt,a4paper}{article}
\documentclass{ltxdoc}

\usepackage[margin=35mm]{geometry}
\usepackage{hyperref}
\usepackage{hyperxmp}
\usepackage[usenames]{color}

\hypersetup{colorlinks=true}
\hypersetup{pdfstartview=FitH}
\hypersetup{pdfpagemode=UseNone}
\hypersetup{pdfsource={}}
\hypersetup{pdflang={en-UK}}
\hypersetup{pdfcopyright={Copyright 2017-2018 Niklas Beisert.
  This work may be distributed and/or modified under the
  conditions of the LaTeX Project Public License, either version 1.3
  of this license or (at your option) any later version.}}
\hypersetup{pdflicenseurl={http://www.latex-project.org/lppl.txt}}
\hypersetup{pdfcontactaddress={ETH Zurich, ITP, HIT K,
  Wolfgang-Pauli-Strasse 27}}
\hypersetup{pdfcontactpostcode={8093}}
\hypersetup{pdfcontactcity={Zurich}}
\hypersetup{pdfcontactcountry={Switzerland}}
\hypersetup{pdfcontactemail={nbeisert@itp.phys.ethz.ch}}
\hypersetup{pdfcontacturl={http://people.phys.ethz.ch/\xmptilde nbeisert/}}

\newcommand{\secref}[1]{\hyperref[#1]{section \ref*{#1}}}

\parskip1ex
\parindent0pt
\let\olditemize\itemize
\def\itemize{\olditemize\parskip0pt}

\begin{document}

\title{The \textsf{childdoc} Package}
\hypersetup{pdftitle={The childdoc Package}}
\author{Niklas Beisert\\[2ex]
  Institut f\"ur Theoretische Physik\\
  Eidgen\"ossische Technische Hochschule Z\"urich\\
  Wolfgang-Pauli-Strasse 27, 8093 Z\"urich, Switzerland\\[1ex]
  \href{mailto:nbeisert@itp.phys.ethz.ch}
  {\texttt{nbeisert@itp.phys.ethz.ch}}}
\hypersetup{pdfauthor={Niklas Beisert}}
\hypersetup{pdfsubject={Manual for the LaTeX2e Package childdoc}}
\date{30 December 2018, \textsf{v2.0}}
\maketitle

\begin{abstract}\noindent
\textsf{childdoc} is a \LaTeXe{} package
that enables the direct compilation
of document sections included by |\include|
to individual files.
\end{abstract}

\begingroup
\parskip0ex
\tableofcontents
\endgroup

%%%%%%%%%%%%%%%%%%%%%%%%%%%%%%%%%%%%%%%%%%%%%%%%%%%%%%%%%%%%%%%%%%%%%%%%%%%%%%%%
%%%%%%%%%%%%%%%%%%%%%%%%%%%%%%%%%%%%%%%%%%%%%%%%%%%%%%%%%%%%%%%%%%%%%%%%%%%%%%%%
\section{Introduction}

\LaTeX{} provides a mechanism to structure a large document (such as a book)
into a main file and several child files (containing the chapters)
using the |\include| command.
This mechanism is beneficial for documents
which span hundreds of pages in order to
make the source file(s) more manageable.
Moreover, compilation can be restricted to
selected child files by means of the |\includeonly| command.
The latter feature can be used to reduce the compilation time while editing
(this was significantly more useful in the earlier days of \LaTeX{})
or to generate a smaller document which is easier to navigate.
Another application of |\includeonly| is to generate
documents consisting of selected parts of the complete document.

However, there are a few drawbacks of the plain |\include| mechanism:
\begin{itemize}
\item
The child files cannot be compiled on their own,
they can only be compiled via the main file.
A naive editing environment
(such as a text editor with an option
to have the current file processed by \LaTeX)
may require one to switch to the main file before compiling;
attempting to compile the child file produces errors.
\item
The main file must be modified (each time)
to adjust the |\includeonly| command
to the present needs. This easily leaves the main file in a messy state.
\item
The generated document will always carry the filename
of the main document. This is inconvenient if
several child files are to be compiled and
to be kept for distribution.
\end{itemize}

The present package provides a simple interface
to make child files individually compilable by \LaTeX{}.
Compiling a child file then has the same effect as compiling
the main file with an |\includeonly| command
to select the appropriate child.
Moreover the generated document will carry the name of the child
rather than the main file.
This resolves all three above issues.

This feature is meant to make the editing of books,
thesis documents and lecture notes somewhat more convenient.
However, the package can also be used efficiently for
composing a series of documents (such as exercise sheets)
which are typically distributed individually.
It then assists the author in generating the individual documents
(potentially in different versions)
as well as a document containing the collected series.
Another application is in developing style files
or other kinds of included material
where compilation of the style file could redirect
to a sample or test file.

%%%%%%%%%%%%%%%%%%%%%%%%%%%%%%%%%%%%%%%%%%%%%%%%%%%%%%%%%%%%%%%%%%%%%%%%%%%%%%%%
%%%%%%%%%%%%%%%%%%%%%%%%%%%%%%%%%%%%%%%%%%%%%%%%%%%%%%%%%%%%%%%%%%%%%%%%%%%%%%%%
\section{Usage}

First of all, the package \textsf{childdoc} is \emph{not} a standard
\LaTeXe{} |.sty| style file! Therefore it needs to be invoked in
a non-standard way.

%%%%%%%%%%%%%%%%%%%%%%%%%%%%%%%%%%%%%%%%%%%%%%%%%%%%%%%%%%%%%%%%%%%%%%%%%%%%%%%%
\subsection{Included Files}
\label{sec:include}

%%%%%%%%%%%%%%%%%%%%%%%%%%%%%%%%%%%%%%%%
\DescribeMacro{\childdocmain}
To use the package, add the commands
\begin{center}
\begin{tabular}{l}
|\input{childdoc.def}|\\
|\childdocmain{}|\\
\end{tabular}
\end{center}
at the very top of the main \LaTeX{} file,
in particular \emph{before} the |\documentclass| statement!
The argument of |\childdocmain| should be left empty
(but it must be present).

%%%%%%%%%%%%%%%%%%%%%%%%%%%%%%%%%%%%%%%%
\DescribeMacro{\childdocof}
Furthermore, add the commands
\begin{center}
\begin{tabular}{l}
|\input{childdoc.def}|\\
|\childdocof{|\textit{main}|}|\\
\end{tabular}
\end{center}
at the top of every child file \textit{child}
which is included by |\include{|\textit{child}|}|
from within the main file
(or at least for those files to be compiled individually).
The argument \textit{main} must be the filename of the main file.

There are a couple of
considerations in setting up the main and child documents:

%%%%%%%%%%%%%%%%%%%%%%%%%%%%%%%%%%%%%%%%
\paragraph{Restrictions.}

Please note the following restrictions:
\begin{itemize}
\item
|\childdocmain| must be called with one argument \textit{main}
to ensure compatibility with earlier version of the package.
It must either be empty (|\childdocmain{}|)
or precisely match the filename of the main file in which it is specified.
See \secref{sec:detection} for further information.
\item
The filename \textit{main} must be specified without the |.tex| extension.
\item
The filename \textit{main} is case sensitive
(even in case-insensitive file systems)
due to internal string comparison.
\item
The argument \textit{main} should be fully expanded, it cannot be a macro.
\item
Subdirectories and special characters should be avoided in filenames.
\item
The command |\childdocmain{|\textit{main}|}| must be followed by a whitespace.
It should not be followed immediately by another command
or by a comment mark `|%|'.
This is because the \TeX{} parser reads the token immediately following
the argument of |\childdocmain| and puts it
at the beginning of every child section;
however, a white\-space is ignored.
\end{itemize}

%%%%%%%%%%%%%%%%%%%%%%%%%%%%%%%%%%%%%%%%
\paragraph{Content of Main File.}

It is advisable to place all content in the child files included by |\include|.
Any output contained in the main file will appear in all child documents
unless suppressed manually;
it cannot be suppressed automatically by the |\includeonly| directive
and thus should normally be avoided.
A method to include some content in the main file
by means of conditional processing is described in \secref{sec:conditional}.

%%%%%%%%%%%%%%%%%%%%%%%%%%%%%%%%%%%%%%%%
\paragraph{Page Numbering.}

When only a part of the document is compiled,
the appropriate numbering of pages
(as well as other status parameters)
is determined from the |.aux| files.
The latter contain information from previous passes.
However this information needs to propagate through
all intermediate child documents.
Therefore the page numbering in child documents may well
be inconsistent until the complete document is compiled at least once.

A useful (if unconventional) way to always ensure a consistent
page numbering is to restart the numbering in each child document
and denote the pages by `\textit{child}|.|\textit{page}'
where \textit{child} represents the chapter/section number of the child file.
This can be achieved by the command
|\numberwithin{page}{|\textit{child}|}|
of the \textsf{amsmath} package
where \textit{child} can be |chapter| or |section|
depending on the chosen structuring.
Alternatively, one can modify the macro |\thepage| appropriately
and reset the counter |page| at the start of each child file.

%%%%%%%%%%%%%%%%%%%%%%%%%%%%%%%%%%%%%%%%%%%%%%%%%%%%%%%%%%%%%%%%%%%%%%%%%%%%%%%%
\subsection{Conditional Processing}
\label{sec:conditional}

The package provides a mechanism to compile different versions
of a document. To customise the versions further some conditional processing
can come in handy to distinguish which version is being compiled.
The package provides two macros to describe the compilation context:

%%%%%%%%%%%%%%%%%%%%%%%%%%%%%%%%%%%%%%%%
\DescribeMacro{\ifchilddoc}
The conditional |\ifchilddoc| distinguishes between the compilation of
child documents and the main document:
%
\begin{center}
|\ifchilddoc |\textit{child-code}| |[|\||else |\textit{main-code}]| \||fi|
\end{center}

%%%%%%%%%%%%%%%%%%%%%%%%%%%%%%%%%%%%%%%%
\DescribeMacro{\childdocname}
\DescribeMacro{\childdocjob}
The macro |\childdocname| contains the filename (without extension)
of the main or child file being processed.
Note that |\childdocjob| will always contain the name of the main file.

%%%%%%%%%%%%%%%%%%%%%%%%%%%%%%%%%%%%%%%%
\paragraph{Title Page.}

Conditional processing can be used to include a title or banner page
in the main document when proper precautions are taken.
Importantly, the code in the main file should ensure that the page counter
(as well as other status parameters which are stored in the |.aux| files)
takes the same value after the conditional processing.
Otherwise the page numbers may take divergent values
depending on which part is compiled.

For example, a title page could be declared by:
%
\begin{center}
\begin{tabular}{l}
|\ifchilddoc\||else|\\
|\addtocounter{page}{-1}|\\
\textit{code for title page}\\
|\newpage|\\
|\||fi|
\end{tabular}
\end{center}
%
A banner page for the child documents can be generated by:
%
\begin{center}
\begin{tabular}{l}
|\ifchilddoc|\\
|\addtocounter{page}{-1}|\\
\textit{code for banner page}\\
|\newpage|\\
|\||fi|
\end{tabular}
\end{center}
%
Here one could write a message such as:
\begin{center}
|This is the part \childdocname{} of \childdocjob{}.|
\end{center}

%%%%%%%%%%%%%%%%%%%%%%%%%%%%%%%%%%%%%%%%%%%%%%%%%%%%%%%%%%%%%%%%%%%%%%%%%%%%%%%%
\subsection{Flags}
\label{sec:flags}

The package makes it easy to generate different versions
of the main or child documents.
To this end compilation flags can be defined
and assigned different default values.
They will be particularly useful in conjunction
with the forwarding mechanism described in \secref{sec:forward}.

For example, it may be useful to have a flag |\version|
which can be set to |draft| or |final|.
The document source will contain some conditional code
depending on the value of |\version|.
Suppose further, the flag should default to |final| for the main file
and to |draft| for child files
which is a natural assignment for editing the document.
This is achieved by placing the following code
in the preamble of the main document
(below the |\childdocmain| directive):
%
\begin{center}
\begin{tabular}{l}
|\ifchilddoc|\\
|\providecommand{\version}{draft}|\\
|\||else|\\
|\providecommand{\version}{final}|\\
|\||fi|
\end{tabular}
\end{center}
%
The definition by |\providecommand| makes sure
that previous definitions are not overwritten.
Further statements |\providecommand{\version}{...}|
can thus be added before the above code to override it.

For the main file, one might add a line
(between |\childdocmain| and the above block)
%
\begin{center}
|%\ifchilddoc\||else\providecommand{\version}{draft}\||fi|
\end{center}
%
which can be uncommented to produce a draft version.
Likewise one can add a line to the very top of a child file
(above the |\childdocof{|\textit{main}|}| directive)
%
\begin{center}
|%\providecommand{\version}{final}|
\end{center}
%
which can be uncommented to produce the final version of this child document.

%%%%%%%%%%%%%%%%%%%%%%%%%%%%%%%%%%%%%%%%%%%%%%%%%%%%%%%%%%%%%%%%%%%%%%%%%%%%%%%%
\subsection{Forwarding}
\label{sec:forward}

Different versions of the main or child documents
using compilation flags as described in \secref{sec:flags}
can be (permanently) stored in different files
for convenient compilation, viewing and distribution.
To this end, the package defines a command
to pass on compilation to a different file:

%%%%%%%%%%%%%%%%%%%%%%%%%%%%%%%%%%%%%%%%
\DescribeMacro{\childdocforward}
The command |\childdocforward| redirects processing to
another source file:
%
\begin{center}
\begin{tabular}{l}
|\input{childdoc.def}|\\
|\childdocforward[|\textit{main}|]{|\textit{dest}|}|\\
\end{tabular}
\end{center}
%
The argument \textit{dest} is the destination file
(without extension).
It should be the main file or one of the child files.
Note that further \textsf{childdoc} directives
such as |\childdocof| and |\childdocforward|
in the indicated file will be processed in this form.
The optional argument \textit{main}
passes on directly to the main file \textit{main}
while pretending to compile the child \textit{dest}.
This form behaves as if \textit{dest}
issues |\childdocof{|\textit{main}|}| right away,
and no further \textsf{childdoc} directives will be processed.

%%%%%%%%%%%%%%%%%%%%%%%%%%%%%%%%%%%%%%%%
\DescribeMacro{\...prefix}
In the alternative form |\childdocforwardprefix|,
%
\begin{center}
\begin{tabular}{l}
|\input{childdoc.def}|\\
|\childdocforwardprefix[|\textit{main}|]{|\textit{prefix}|}{|\textit{dest}|}|
\end{tabular}
\end{center}
%
the destination file is determined by a pattern
depending on the current file:
To make this work, the current file must be called
`{\textit{prefix}\hspace{0.2em}\textit{suffix}}'
with \textit{prefix} matching precisely the argument.
Processing is then passed on to the file
`{\textit{dest}\hspace{0.2em}\textit{suffix}}'.
Surely, the same effect is achieved by
directly specifying the
argument `{\textit{dest}\hspace{0.2em}\textit{suffix}}'
in the first form.
However, that requires to set up a different file
for each child. With the alternative form of the command
all these files can have exactly the same content
which simplifies setting them up and maintaining them.

For example, the following file |draft.tex|
with a compilation flag |\version| as described in \secref{sec:flags}
compiles the main document as a draft:
%
\begin{center}
\begin{tabular}{l}
|\def\version{draft}|\\
|\input{childdoc.def}|\\
|\childdocforward{|\textit{main}|}|
\end{tabular}
\end{center}
%
Likewise, the following files |final|\textit{nn}|.tex|
compile the final version of the child document
|child|\textit{nn}|.tex|:
%
\begin{center}
\begin{tabular}{l}
|\def\version{final}|\\
|\input{childdoc.def}|\\
|\childdocforwardprefix{final}{child}|
\end{tabular}
\end{center}
%

Note that when several versions of a main file and/or of each child file
are to be generated, it may be convenient to set up a |Makefile| or
shell script to automatise the process.

%%%%%%%%%%%%%%%%%%%%%%%%%%%%%%%%%%%%%%%%%%%%%%%%%%%%%%%%%%%%%%%%%%%%%%%%%%%%%%%%
\subsection{Command Line Processing}
\label{sec:commandline}

The effect of redirection files can also be achieved by invoking
the \LaTeX{} compiler with a more elaborate command line.
Most conveniently this should be done as part
of a shell script or a |Makefile|.

When using \textsf{childdoc} in the main file, the following
command lines effectively perform a redirection
(note that depending on the shell being used,
backslashes may have to be doubled: `|\|' $\to$ `|\\|'):
%
\begin{center}
|... -jobname "|\textit{target}|" |\\|"|[\textit{flags}]%
|\input{childdoc.def}\childdocforward[|\textit{main}|]{|\textit{dest}|}"|
\end{center}
%
Here \textit{target} is the name of the output file,
\textit{main} is the name of the main file
and \textit{dest} is the name of the main or child file to be processed
(all filenames without extensions).
The optional argument \textit{main} can be omitted
if \textit{main} matches \textit{dest}.
Optionally, compilation \textit{flags} can be defined via |\def| commands.
This command line makes the \TeX{} engine believe
it is compiling the file \textit{target}
whose content is specified as the latter parameter.
The provided code then forwards the processing to
\textit{main} or \textit{dest} as described in \secref{sec:forward}.

%%%%%%%%%%%%%%%%%%%%%%%%%%%%%%%%%%%%%%%%%%%%%%%%%%%%%%%%%%%%%%%%%%%%%%%%%%%%%%%%
\subsection{Include by Input}
\label{sec:input}

Including child documents by |\include| has some restrictions by design.
Most notably, the content of a child document always occupies
its own set of pages; pages cannot be shared between child documents.
Usually, this behaviour makes perfect sense
because each child document contain an essential part of the document.
However, in some situations it may be desirable to compose
a document from a collection of parts
without having mandatory page breaks between then.
For this case, the package
provides a mechanism to include parts
by |\input| which can also be processed individually.
However, by construction this mechanism
requires manual handling of the content to be output.

%%%%%%%%%%%%%%%%%%%%%%%%%%%%%%%%%%%%%%%%
\DescribeMacro{\ifchilddocmanual}
The main file should be prepared as usual, see \secref{sec:include}.
However, the document body must make a distinction
between processing of an individual part and of the main document, e.g.:
%
\begin{center}
\begin{tabular}{l}
|\ifchilddocmanual|\\
|\input{\childdocname}|\\
|\||else|\\
\textit{document body with }|\input{|\textit{part}|}|\\
|\||fi|
\end{tabular}
\end{center}
%
The conditional |\ifchilddocmanual| is true whenever
a part to be included by |\input| is being compiled,
and the name of the part is stored in |\childdocname|.

%%%%%%%%%%%%%%%%%%%%%%%%%%%%%%%%%%%%%%%%
\DescribeMacro{\childdocby}
Each part to be included by |\input| should start with:
%
\begin{center}
\begin{tabular}{l}
|\input{childdoc.def}|\\
|\childdocby{|\textit{main}|}|\\
\end{tabular}
\end{center}
%
The directive |\childdocby| is similar to |\childdocof|
described in \secref{sec:include},
but the subsequent selection of content must be done manually.
To that end, both |\ifchilddoc| and |\ifchilddocmanual|
will be true upon processing of a part,
and the name of the part is stored in |\childdocname|.
Note that |\jobname| will be set to the filename of the current part
so that each part receives an individual |.aux| file
that does not interfere with the |.aux| file(s) of the main document.
This behaviour can be altered by the alternative form
|\childdocby[*]{|\textit{main}|}| (with a non-empty optional argument)
which uses the |.aux| file of the main document
by setting |\jobname| to \textit{main}.

%%%%%%%%%%%%%%%%%%%%%%%%%%%%%%%%%%%%%%%%%%%%%%%%%%%%%%%%%%%%%%%%%%%%%%%%%%%%%%%%
\subsection{Driver Development}
\label{sec:driver}

The \textsf{childdoc} mechanism can also be use for the development
of definition files such as \LaTeX{} styles or classes.
This case differs from the above setup with multiple parts
included by |\include| in that no |\includeonly| should be invoked.
This can be achieved by starting the include file
(before |\ProvidesPackage|) with:
%
\begin{center}
\begin{tabular}{l}
|\input{childdoc.def}|\\
|\childdocforward{|\textit{main}|}|\\
\end{tabular}
\end{center}
%
or alternatively with:
%
\begin{center}
\begin{tabular}{l}
|\input{childdoc.def}|\\
|\childdocby{|\textit{main}|}|\\
\end{tabular}
\end{center}
%
Both forms have slightly different effects as described above.
The main file is prepared as usual, see \secref{sec:include}.

%%%%%%%%%%%%%%%%%%%%%%%%%%%%%%%%%%%%%%%%%%%%%%%%%%%%%%%%%%%%%%%%%%%%%%%%%%%%%%%%
\subsection{Legacy Detection}
\label{sec:detection}

The directive |\childdocmain| in the main file can detect
whether the complete document or merely a child is to be compiled
even without using the directive |\childdocof|.
This method is deprecated because it is less robust
and there is no compelling reason to use it;
it is merely provided for backward compatibility
and it may be removed in future versions.

If the detection mechanism is to be used,
it is mandatory to correctly specify
the filename of the main file as the argument of |\childdocmain|:
%
\begin{center}
\begin{tabular}{l}
|\input{childdoc.def}|\\
|\childdocmain{|\textit{main}|}|\\
\end{tabular}
\end{center}
%
If |\jobname| does not match the argument \textit{main} of |\childdocmain|,
it is assumed that |\jobname| points to the child file to be compiled.
When using |\childdocmain| with the main file specified as argument,
it suffices to start a child file
with just |\input{|\textit{main}|}|
without loading of the package and using |\childdocof|.
If instead all processing is done
with the appropriate \textsf{childdoc} directives,
the argument of \textit{main} of |\childdocmain| can be empty.

An alternative version of the command line processing described
in \secref{sec:commandline} using the detection mechanism reads:
%
\begin{center}
|... -jobname "|\textit{target}|" "|[\textit{flags}]%
[|\def\jobname{|\textit{dest}|}|]|\input{|\textit{main}|}"|
\end{center}

%%%%%%%%%%%%%%%%%%%%%%%%%%%%%%%%%%%%%%%%%%%%%%%%%%%%%%%%%%%%%%%%%%%%%%%%%%%%%%%%
\subsection{Manual Code}
\label{sec:manual}

In case one cannot be certain whether the definitions file |childdoc.def|
is installed on the target \TeX{} distribution
and one prefers not to ship it,
it is conceivable to paste a few relevant commands into the sources.

To that end, drop all statements |\input{childdoc.def}|
and perform the replacements as outlined below.
Instead of |\childdocmain{|\textit{main}|}| add the following code
to the top of the main file:
%
\begin{center}
\begin{tabular}{l}
|\||ifdefined\childdocname\endinput\||fi\newif\ifchilddoc|\\
|\edef\childdocname{\scantokens\expandafter{\jobname\noexpand}}|\\
|\def\childdocmain{|\textit{main}|}\||ifx\childdocmain\childdocname\||else|\\
|\childdoctrue\includeonly{\childdocname}\let\jobname\childdocmain\||fi|\\
\end{tabular}
\end{center}
%
Instead of |\childdocof{|\textit{main}|}| just include the main file
at the top of each child file:
%
\begin{center}
|\input{|\textit{main}|}|
\end{center}
%
A simple redirection |\childdocforward{|\textit{dest}|}| is achieved by:
%
\begin{center}
|\def\jobname{|\textit{dest}|}\input{\jobname}|
\end{center}
%
The redirection with prefix
|\childdocforwardprefix[|\textit{prefix}|]{|\textit{dest}|}|
is accomplished by:
%
\begin{center}
\begin{tabular}{l}
|{\edef\jobname{\scantokens\expandafter{\jobname\noexpand}}|\\
|\def\redirectjob |\textit{prefix}|#1~~~{\gdef\jobname{|\textit{dest}|#1}}|\\
|\expandafter\redirectjob\jobname~~~}\input{\jobname}|
\end{tabular}
\end{center}

In an alternative approach,
child documents can be compiled by a specific command line
without additional code or specific definitions:
%
\begin{center}
|... -jobname "|\textit{target}|" "|[\textit{flags}]%
|\includeonly{|\textit{dest}|}\input{|\textit{main}|}"|
\end{center}
%

%%%%%%%%%%%%%%%%%%%%%%%%%%%%%%%%%%%%%%%%%%%%%%%%%%%%%%%%%%%%%%%%%%%%%%%%%%%%%%%%
%%%%%%%%%%%%%%%%%%%%%%%%%%%%%%%%%%%%%%%%%%%%%%%%%%%%%%%%%%%%%%%%%%%%%%%%%%%%%%%%
\section{Information}

%%%%%%%%%%%%%%%%%%%%%%%%%%%%%%%%%%%%%%%%%%%%%%%%%%%%%%%%%%%%%%%%%%%%%%%%%%%%%%%%
\subsection{Copyright}

Copyright \copyright{} 2017--2018 Niklas Beisert

This work may be distributed and/or modified under the
conditions of the \LaTeX{} Project Public License, either version 1.3
of this license or (at your option) any later version.
The latest version of this license is in
  \url{http://www.latex-project.org/lppl.txt}
and version 1.3 or later is part of all distributions of \LaTeX{}
version 2005/12/01 or later.

This work has the LPPL maintenance status `maintained'.

The Current Maintainer of this work is Niklas Beisert.

This work consists of the files |README.txt|, |childdoc.ins| and |childdoc.dtx|
as well as the derived files |childdoc.def|, |cdocsamp.tex|
with |cdocsch1.tex|, |cdocsch2.tex|, |cdocspt3.tex|, |cdocspt4.tex|,
|cdocsdrf.tex|, |cdocsfn1.tex|, |cdocsfn2.tex|
as well as |childdoc.pdf|.

%%%%%%%%%%%%%%%%%%%%%%%%%%%%%%%%%%%%%%%%%%%%%%%%%%%%%%%%%%%%%%%%%%%%%%%%%%%%%%%%
\subsection{Files and Installation}

The package consists of the files:
%
\begin{center}
\begin{tabular}{ll}
    |README.txt|   & readme file \\
    |childdoc.ins| & installation file \\
    |childdoc.dtx| & source file \\
    |childdoc.def| & definition file \\
    |cdocsamp.tex| & sample main file \\
    |cdocsch1.tex| & sample include file \\
    |cdocsch2.tex| & sample include file \\
    |cdocspt3.tex| & sample part file \\
    |cdocspt4.tex| & sample part file \\
    |cdocsdrf.tex| & sample redirection file \\
    |cdocsfn1.tex| & sample redirection file \\
    |cdocsfn2.tex| & sample redirection file \\
    |childdoc.pdf| & manual
\end{tabular}
\end{center}
%
The distribution consists of the files
|README.txt|, |childdoc.ins| and |childdoc.dtx|.
%
\begin{itemize}
\item
Run (pdf)\LaTeX{} on |childdoc.dtx|
to compile the manual |childdoc.pdf| (this file).
\item
Run \LaTeX{} on |childdoc.ins| to create the definitions file |childdoc.def|
and the sample |cdocsamp.tex| with include files
|cdocsch1.tex|, |cdocsch2.tex|, |cdocspt3.tex|, |cdocspt4.tex|,
|cdocsdrf.tex|, |cdocsfn1.tex|, |cdocsfn2.tex|.
Then copy the file |childdoc.def| to an appropriate directory of your \LaTeX{}
distribution, e.g.\ \textit{texmf-root}|/tex/latex/childdoc|.
\end{itemize}

%%%%%%%%%%%%%%%%%%%%%%%%%%%%%%%%%%%%%%%%%%%%%%%%%%%%%%%%%%%%%%%%%%%%%%%%%%%%%%%%
\subsection{Related CTAN Packages}

There are several other packages which offer a similar functionality:
%
\begin{itemize}
\item
The packages
\href{http://ctan.org/pkg/docmute}{\textsf{docmute}},
\href{http://ctan.org/pkg/includex}{\textsf{includex}} and
\href{http://ctan.org/pkg/standalone}{\textsf{standalone}}
provide commands to include only the document body of
a child file thus allowing both files to be compiled individually.
\item
The packages \href{http://ctan.org/pkg/subdocs}{\textsf{subdocs}}
and \href{http://ctan.org/pkg/subfiles}{\textsf{subfiles}}
provide structures in which the main and child documents can be
encapsulated and allowing them to be compiled individually.
The inclusion mechanism is different from the conventional |\include|.
\item
The package \href{http://ctan.org/pkg/combine}{\textsf{combine}}
is an elaborate solution to combine several documents into one.
\end{itemize}
%
See also the CTAN topic \href{http://ctan.org/topic/subdocs}{\textsf{subdocs}}
for further related packages.
The present package differs from the above solutions in that
a document structure constructed with the conventional |\include| mechanism
just needs two extra commands at the top of every file
such that all constituent files can be compiled individually.

%%%%%%%%%%%%%%%%%%%%%%%%%%%%%%%%%%%%%%%%%%%%%%%%%%%%%%%%%%%%%%%%%%%%%%%%%%%%%%%%
%\subsection{Feature Suggestions}
%
%The following is a list of features which may be useful for future
%versions of this package:
%%
%\begin{itemize}
%\item
%\ldots
%\end{itemize}

%%%%%%%%%%%%%%%%%%%%%%%%%%%%%%%%%%%%%%%%%%%%%%%%%%%%%%%%%%%%%%%%%%%%%%%%%%%%%%%%
\subsection{Revision History}

%%%%%%%%%%%%%%%%%%%%%%%%%%%%%%%%%%%%%%%%
\paragraph{v2.0:} 2018/12/30

\begin{itemize}
\item
immediate forward processing
\item
added |\childdocby| mechanism
\item
manual restructured
\end{itemize}

%%%%%%%%%%%%%%%%%%%%%%%%%%%%%%%%%%%%%%%%
\paragraph{v1.6:} 2018/01/17

\begin{itemize}
\item
application for development of include files
\item
corrections to manual
\end{itemize}

%%%%%%%%%%%%%%%%%%%%%%%%%%%%%%%%%%%%%%%%
\paragraph{v1.5:} 2017/05/21

\begin{itemize}
\item
more complete structuring introduced
\item
|\childdocof| introduced
\item
|\childdoc| renamed to |\childdocmain|
\item
|\childredirect| renamed to |\childdocforward| and |\childdocforwardprefix|
and functionality expanded
\end{itemize}

%%%%%%%%%%%%%%%%%%%%%%%%%%%%%%%%%%%%%%%%
\paragraph{v1.0:} 2017/04/27

\begin{itemize}
\item
manual and install package
\item
first version published on CTAN
\end{itemize}

%%%%%%%%%%%%%%%%%%%%%%%%%%%%%%%%%%%%%%%%
\paragraph{v0.6:} 2017/04/26

\begin{itemize}
\item
redirection mechanism added
\end{itemize}

%%%%%%%%%%%%%%%%%%%%%%%%%%%%%%%%%%%%%%%%
\paragraph{v0.5:} 2017/04/26

\begin{itemize}
\item
functionality in definition file
\end{itemize}


%%%%%%%%%%%%%%%%%%%%%%%%%%%%%%%%%%%%%%%%%%%%%%%%%%%%%%%%%%%%%%%%%%%%%%%%%%%%%%%%
%%%%%%%%%%%%%%%%%%%%%%%%%%%%%%%%%%%%%%%%%%%%%%%%%%%%%%%%%%%%%%%%%%%%%%%%%%%%%%%%
%%%%%%%%%%%%%%%%%%%%%%%%%%%%%%%%%%%%%%%%%%%%%%%%%%%%%%%%%%%%%%%%%%%%%%%%%%%%%%%%
\appendix

\settowidth\MacroIndent{\rmfamily\scriptsize 000\ }

 \DocInput{childdoc.dtx}

\end{document}
%</driver>
% \fi
%
% %%%%%%%%%%%%%%%%%%%%%%%%%%%%%%%%%%%%%%%%%%%%%%%%%%%%%%%%%%%%%%%%%%%%%%%%%%%%%%
% %%%%%%%%%%%%%%%%%%%%%%%%%%%%%%%%%%%%%%%%%%%%%%%%%%%%%%%%%%%%%%%%%%%%%%%%%%%%%%
% \section{Sample}
%\iffalse
%<*samplemain>
%\fi
%
% The following presents a sample document
% with two chapters, two parts, a title page,
% a compile flag as well as three forwarding files to set the flag.
% It consists of eight |.tex| files:
% \begin{center}
% \begin{tabular}{ll}
% |cdocsamp.tex|&main file\\
% |cdocsch1.tex|&include file for chapter 1\\
% |cdocsch2.tex|&include file for chapter 2\\
% |cdocspt3.tex|&include file for part 3\\
% |cdocspt4.tex|&include file for part 4\\
% |cdocsdrf.tex|&forwarding file for main file in draft mode\\
% |cdocsfi1.tex|&forwarding file for final version of chapter 1\\
% |cdocsfi2.tex|&forwarding file for final version of chapter 2\\
% \end{tabular}
% \end{center}
% Each of the eight files can be compiled directly by the \LaTeX{} compiler.
%
% %%%%%%%%%%%%%%%%%%%%%%%%%%%%%%%%%%%%%%
% \paragraph{Main File.}
%
% The main file is called |cdocsamp.tex|.
%
% Load the \textsf{childdoc} definitions and
% declare the filename for the main document:
%    \begin{macrocode}
\input{childdoc.def}
\childdocmain{}
%    \end{macrocode}

% Optional override for |\version| flag:
%    \begin{macrocode}
%%\ifchilddoc\else\providecommand{\version}{draft}\fi
%    \end{macrocode}

% Define the default values for the |\version| flag
% (|final| for the main file and |draft| for childs):
%    \begin{macrocode}
\ifchilddoc
\providecommand{\version}{draft}
\else
\providecommand{\version}{final}
\fi
%    \end{macrocode}

% Load the standard document class:
%    \begin{macrocode}
\documentclass[12pt]{article}
%    \end{macrocode}

% Start the document body:
%    \begin{macrocode}
\begin{document}
%    \end{macrocode}

% Declare a title page.
% Print title, part of document being processed and version flag:
%    \begin{macrocode}
\addtocounter{page}{-1}
\begin{center}
{\LARGE\bfseries{}childdoc example\par}
\vspace{1cm}
\ifchilddoc
\ifchilddocmanual part\else chapter\fi:
`\childdocname' of `\childdocjob'\par
\else
main document: `\childdocjob'\par
\fi
version: \version\par
\end{center}
\newpage
%    \end{macrocode}

% Manually include selected file,
% otherwise process as usual:
%    \begin{macrocode}
\ifchilddocmanual
\section*{part `\childdocname'}
\input{\childdocname}
\else
%    \end{macrocode}

% Include the two chapters:
%    \begin{macrocode}
\include{cdocsch1}
\include{cdocsch2}
%    \end{macrocode}

% Include the two parts unless only chapters should be displayed:
%    \begin{macrocode}
\ifchilddoc\else
\section{part three}
\input{cdocspt3}
\section{part four}
\input{cdocspt4}
\fi
%    \end{macrocode}

% Process as usual until here:
%    \begin{macrocode}
\fi
%    \end{macrocode}

% End of document body:
%    \begin{macrocode}
\end{document}
%    \end{macrocode}
%\iffalse
%</samplemain>
%\fi
%
% %%%%%%%%%%%%%%%%%%%%%%%%%%%%%%%%%%%%%%
% \paragraph{Chapter Include Files.}
%
% The include files are called |cdocsch1.tex| and |cdocsch2.tex|.
%
%\iffalse
%<*samplechap1|samplechap2>
%\fi

% Optional override for |\version| flag:
%    \begin{macrocode}
%%\providecommand{\version}{final}
%    \end{macrocode}

% Include the main document:
%    \begin{macrocode}
\input{childdoc.def}
\childdocof{cdocsamp}
%    \end{macrocode}

%\iffalse
%</samplechap1|samplechap2>
%\fi
%
%\iffalse
%<*samplechap1>
%\fi
% Some text for chapter 1:
%    \begin{macrocode}
\section{one}
some text in chapter one
%    \end{macrocode}

%\iffalse
%</samplechap1>
%\fi
% Some text for chapter 2:
%\iffalse
%<*samplechap2>
%\fi
%    \begin{macrocode}
\section{two}
more text in chapter two
%    \end{macrocode}

%\iffalse
%</samplechap2>
%\fi
%
% %%%%%%%%%%%%%%%%%%%%%%%%%%%%%%%%%%%%%%
% \paragraph{Part Include Files.}
%
% The include files are called |cdocspt3.tex| and |cdocspt4.tex|.
%
%\iffalse
%<*samplepart3|samplepart4>
%\fi

% Optional override for |\version| flag:
%    \begin{macrocode}
%%\providecommand{\version}{final}
%    \end{macrocode}

% Include the main document:
%    \begin{macrocode}
\input{childdoc.def}
\childdocby{cdocsamp}
%    \end{macrocode}

%\iffalse
%</samplepart3|samplepart4>
%\fi
%
%\iffalse
%<*samplepart3>
%\fi
% Some text for part 3:
%    \begin{macrocode}
some text in part three
%    \end{macrocode}

%\iffalse
%</samplepart3>
%\fi
% Some text for part 4:
%\iffalse
%<*samplepart4>
%\fi
%    \begin{macrocode}
more text in part four
%    \end{macrocode}

%\iffalse
%</samplepart4>
%\fi
%
% %%%%%%%%%%%%%%%%%%%%%%%%%%%%%%%%%%%%%%
% \paragraph{Forwarding for a Complete Draft.}
%
% The following forwarding file |cdocsdrf.tex|
% compiles the main document in draft mode:
%\iffalse
%<*sampledraft>
%\fi
%    \begin{macrocode}
\def\version{draft}
\input{childdoc.def}
\childdocforward{cdocsamp}
%    \end{macrocode}

%\iffalse
%</sampledraft>
%\fi
%
% %%%%%%%%%%%%%%%%%%%%%%%%%%%%%%%%%%%%%%
% \paragraph{Forwarding for Final Version of the Chapters.}
%
% The following forwarding files |cdocsfn1.tex| and |cdocsfn2.tex|
% (with identical content)
% compile the final versions of the child documents
% |cdocsch1.tex| and |cdocsch2.tex|, respectively:
%\iffalse
%<*samplefinal>
%\fi
%    \begin{macrocode}
\def\version{final}
\input{childdoc.def}
\childdocforwardprefix[cdocsamp]{cdocsfn}{cdocsch}
%    \end{macrocode}

%\iffalse
%</samplefinal>
%\fi
%
% %%%%%%%%%%%%%%%%%%%%%%%%%%%%%%%%%%%%%%
% \paragraph{Command Line Processing.}
%
% The following three command lines generate the output files
% |cdocscld|, |cdocscl1| and |cdocscl2|
% which should be identical to
% |cdocsdrf|, |cdocsch1| and |cdocsfn2|, respectively:
% \begin{center}
% \begin{tabular}{l}
% |latex -jobname cdocscld \|\\
% |  "\def\version{draft}\input{childdoc.def}\childdocforward{cdocsamp}"|\\
% |latex -jobname cdocscl1 \|\\
% |  "\input{childdoc.def}\childdocforward[cdocsamp]{cdocsch1}"|\\
% |latex -jobname cdocscl2 \|\\
% |  "\def\version{final}\input{childdoc.def}\childdocforward{cdocsch2}"|
% \end{tabular}
% \end{center}
% Note that the trailing backslash on each first line
% merely continues the input to the second line
% (for convenient cut ant paste).
% Furthermore, the command |latex| can be replaced by any
% of its alternative versions such as |pdflatex|.
%
% %%%%%%%%%%%%%%%%%%%%%%%%%%%%%%%%%%%%%%%%%%%%%%%%%%%%%%%%%%%%%%%%%%%%%%%%%%%%%%
% %%%%%%%%%%%%%%%%%%%%%%%%%%%%%%%%%%%%%%%%%%%%%%%%%%%%%%%%%%%%%%%%%%%%%%%%%%%%%%
% \section{Implementation}
%\iffalse
%<*package>
%\fi
%
% This section describes the definitions file |childdoc.def|.

% The definitions cannot be loaded using |\usepackage| or |\RequirePackage|
% which has a mechanism to prevent loading a style file more than once.
% When loading the definitions by means of |\input|
% multiple instances have to be prevented manually:
%\iffalse
%This code needs to be before the `\ProvidesFile' directive
%which is defined at the beginning of this file.
%Therefore it is also placed there and commented out here.
%</package>
%<*discard>
%\fi
%    \begin{macrocode}
\ifdefined\childdocmain\endinput\fi
%    \end{macrocode}
%\iffalse
%</discard>
%<*package>
%\fi
%
% \macro{\ifchilddoc}
% \macro{\ifchilddocmanual}
% The conditional |\ifchilddoc| tells whether a
% child (true) or main (false) document is being compiled.
% The conditional |\ifchilddocmanual| tells whether
% the |\includeonly| mechanism is used (false) or
% the selection of child files must be performed manually (true).
% The definitions initialise to false:
%    \begin{macrocode}
\newif\ifchilddoc
\newif\ifchilddocmanual
%    \end{macrocode}

% \macro{\childdocname}
% \macro{\childdocjob}
% The macro |\childdocname| stores the name of the main document
% to be compiled. The macro |\childdocjob| stores the name of
% the document on which the \LaTeX{} compiler was originally invoked.
% The content of |\jobname| cannot be compared
% to filenames specified in the source due to different catcodes.
% The following code rescans |\jobname|, stores the result
% in |\childdocname| and saves a copy in |\childdocjob|:
%    \begin{macrocode}
\edef\childdocname{\scantokens\expandafter{\jobname\noexpand}}
\let\childdocjob\childdocname
%    \end{macrocode}

% \macro{\childdocdisable}
% The macro |\childdocdisable| prevents the main file
% from being processed more than once.
% At this stage, the main document command |\childdocmain|
% is assumed to be called once again where it should do nothing.
% Any subsequent call to it should prevent
% a secondary processing of the main document
% It overwrites the forwarding commands
% |\childdocof| and |\childdocforward|
% with empty macros to prevent further inclusions of the main document:
%    \begin{macrocode}
\newcommand{\childdocdisable}
{
  \renewcommand{\childdocmain}[1]{\renewcommand{\childdocmain}[1]{\endinput}}
  \renewcommand{\childdocof}[1]{}
  \renewcommand{\childdocby}[2][]{}
  \renewcommand{\childdocforward}[2][]{}
  \renewcommand{\childdocdisable}{}
}
%    \end{macrocode}

% \macro{\childdocmain}
% The macro |\childdocmain| is to be called at the top of the main file
% with nothing or the main filename (without extension) as argument.
% First, it breaks loops.
% If the argument is not empty and does not match |\childdocname|
% (which is set by the first inclusion of |childdoc.def|),
% |\ifchilddoc| is set to true, |\includeonly| is applied to the child file
% and |\jobname| is set to the main file
% (for proper handling of |.aux| files):
%    \begin{macrocode}
\newcommand{\childdocmain}[1]
{
  \childdocdisable\childdocmain{}
  \if?#1?\else
    \begingroup
      \def\childdoctmp{#1}
      \ifx\childdoctmp\childdocname
        \def\childdoctmp{}
      \else
        \def\childdoctmp
        {
          \childdoctrue
          \includeonly{\childdocname}
          \def\childdocjob{#1}
          \def\jobname{#1}
        }
      \fi
      \expandafter
    \endgroup
    \childdoctmp
  \fi
}
%    \end{macrocode}

% \macro{\childdocof}
% The command |\childdocof| redirects
% compilation to the main file |#1|.
%    \begin{macrocode}
\newcommand{\childdocof}[1]
{
  \childdocdisable
  \childdoctrue
  \includeonly{\childdocname}
  \def\jobname{#1}
  \def\childdocjob{#1}
  \input{#1}
}
%    \end{macrocode}

% \macro{\childdocby}
% The command |\childdocby| ....
%    \begin{macrocode}
\newcommand{\childdocby}[2][]
{
  \childdocdisable
  \childdoctrue
  \childdocmanualtrue
  \if?#1?\else
    \def\jobname{#2}
  \fi
  \def\childdocjob{#2}
  \input{#2}
  \endinput
}
%    \end{macrocode}

% \macro{\childdocforward}
% The command |\childdocforward| redirects
% compilation to the main file or
% (if the optional argument is given) a child file.
% Parameters are set as if the main file
% or a child file starting with |\childdocof| was compiled.
% Then compilation is handed over to the main file:
%    \begin{macrocode}
\newcommand{\childdocforward}[2][]
{
  \begingroup
    \if?#1?
      \def\childdoctmp
      {
        \def\childdocname{#2}
        \def\childdocjob{#2}
        \def\jobname{#2}
        \input{#2}
        \endinput
      }
    \else
      \def\childdoctmp
      {
        \childdocdisable
        \def\childdocname{#2}
        \childdoctrue
        \includeonly{#2}
        \def\childdocjob{#1}
        \def\jobname{#1}
        \input{#1}
        \endinput
      }
    \fi
    \expandafter
  \endgroup
  \childdoctmp
}
%    \end{macrocode}

% \macro{\childdocforwardprefix}
% The command |\childdocforwardprefix| redirects
% compilation to the main or a child file by means of a pattern.
% The prefix |#1| in the current filename is replaced by |#2|
% and the suffix of the current filename is kept
% (it is assumed that the filename does not contain the substring `|~~~|'
% which is used as a delimiter).
% Compilation is handed over to the new file by |\childdocforward|:
%    \begin{macrocode}
\newcommand{\childdocforwardprefix}[3][]
{
  \begingroup
    \def\childdocextract #2##1~~~{\def\childdoctmp{\childdocforward[#1]{#3##1}}}
    \expandafter\childdocextract\childdocname~~~
    \expandafter
  \endgroup
  \childdoctmp
}
%    \end{macrocode}

% \macro{\childdoc}
% The deprecated macro |\childdoc| is a legacy version of |\childdocmain|:
%    \begin{macrocode}
\newcommand{\childdoc}{\childdocmain}
%    \end{macrocode}

% \macro{\childdocredirect}
% The deprecated macro |\childdocredirect| is a legacy version
% of |\childdocforward| and |\childdocforwardprefix|:
%    \begin{macrocode}
\newcommand{\childdocredirect}[2][]
{
  \begingroup
    \if?#1?
      \def\childdoctmp{\childdocforward{#2}}
    \else
      \def\childdoctmp{\childdocforwardprefix{#1}{#2}}
    \fi
    \expandafter
  \endgroup
  \childdoctmp
}
%    \end{macrocode}

%\iffalse
%</package>
%\fi
%
\endinput
|\\
|\childdocforwardprefix{final}{child}|
\end{tabular}
\end{center}
%

Note that when several versions of a main file and/or of each child file
are to be generated, it may be convenient to set up a |Makefile| or
shell script to automatise the process.

%%%%%%%%%%%%%%%%%%%%%%%%%%%%%%%%%%%%%%%%%%%%%%%%%%%%%%%%%%%%%%%%%%%%%%%%%%%%%%%%
\subsection{Command Line Processing}
\label{sec:commandline}

The effect of redirection files can also be achieved by invoking
the \LaTeX{} compiler with a more elaborate command line.
Most conveniently this should be done as part
of a shell script or a |Makefile|.

When using \textsf{childdoc} in the main file, the following
command lines effectively perform a redirection
(note that depending on the shell being used,
backslashes may have to be doubled: `|\|' $\to$ `|\\|'):
%
\begin{center}
|... -jobname "|\textit{target}|" |\\|"|[\textit{flags}]%
|% \iffalse
%
% childdoc.dtx Copyright (C) 2017-2018 Niklas Beisert
%
% This work may be distributed and/or modified under the
% conditions of the LaTeX Project Public License, either version 1.3
% of this license or (at your option) any later version.
% The latest version of this license is in
%   http://www.latex-project.org/lppl.txt
% and version 1.3 or later is part of all distributions of LaTeX
% version 2005/12/01 or later.
%
% This work has the LPPL maintenance status `maintained'.
%
% The Current Maintainer of this work is Niklas Beisert.
%
% This work consists of the files childdoc.dtx and childdoc.ins
% and the derived files childdoc.def and cdocsamp.tex with
% cdocsch1.tex, cdocsch2.tex, cdocsdrf.tex, cdocsfn1.tex, cdocsfn2.tex.
%
%<package>\ifdefined\childdocmain\endinput\fi
%<package>\ProvidesFile{childdoc.def}[2018/12/30 v2.0 child document driver]
%<samplemain>\ProvidesFile{cdocsamp.tex}[2018/12/30 v2.0 sample for childdoc]
%<*driver>
%\ProvidesFile{childdoc.drv}[2018/12/30 v2.0 childdoc reference manual file]
\PassOptionsToClass{10pt,a4paper}{article}
\documentclass{ltxdoc}

\usepackage[margin=35mm]{geometry}
\usepackage{hyperref}
\usepackage{hyperxmp}
\usepackage[usenames]{color}

\hypersetup{colorlinks=true}
\hypersetup{pdfstartview=FitH}
\hypersetup{pdfpagemode=UseNone}
\hypersetup{pdfsource={}}
\hypersetup{pdflang={en-UK}}
\hypersetup{pdfcopyright={Copyright 2017-2018 Niklas Beisert.
  This work may be distributed and/or modified under the
  conditions of the LaTeX Project Public License, either version 1.3
  of this license or (at your option) any later version.}}
\hypersetup{pdflicenseurl={http://www.latex-project.org/lppl.txt}}
\hypersetup{pdfcontactaddress={ETH Zurich, ITP, HIT K,
  Wolfgang-Pauli-Strasse 27}}
\hypersetup{pdfcontactpostcode={8093}}
\hypersetup{pdfcontactcity={Zurich}}
\hypersetup{pdfcontactcountry={Switzerland}}
\hypersetup{pdfcontactemail={nbeisert@itp.phys.ethz.ch}}
\hypersetup{pdfcontacturl={http://people.phys.ethz.ch/\xmptilde nbeisert/}}

\newcommand{\secref}[1]{\hyperref[#1]{section \ref*{#1}}}

\parskip1ex
\parindent0pt
\let\olditemize\itemize
\def\itemize{\olditemize\parskip0pt}

\begin{document}

\title{The \textsf{childdoc} Package}
\hypersetup{pdftitle={The childdoc Package}}
\author{Niklas Beisert\\[2ex]
  Institut f\"ur Theoretische Physik\\
  Eidgen\"ossische Technische Hochschule Z\"urich\\
  Wolfgang-Pauli-Strasse 27, 8093 Z\"urich, Switzerland\\[1ex]
  \href{mailto:nbeisert@itp.phys.ethz.ch}
  {\texttt{nbeisert@itp.phys.ethz.ch}}}
\hypersetup{pdfauthor={Niklas Beisert}}
\hypersetup{pdfsubject={Manual for the LaTeX2e Package childdoc}}
\date{30 December 2018, \textsf{v2.0}}
\maketitle

\begin{abstract}\noindent
\textsf{childdoc} is a \LaTeXe{} package
that enables the direct compilation
of document sections included by |\include|
to individual files.
\end{abstract}

\begingroup
\parskip0ex
\tableofcontents
\endgroup

%%%%%%%%%%%%%%%%%%%%%%%%%%%%%%%%%%%%%%%%%%%%%%%%%%%%%%%%%%%%%%%%%%%%%%%%%%%%%%%%
%%%%%%%%%%%%%%%%%%%%%%%%%%%%%%%%%%%%%%%%%%%%%%%%%%%%%%%%%%%%%%%%%%%%%%%%%%%%%%%%
\section{Introduction}

\LaTeX{} provides a mechanism to structure a large document (such as a book)
into a main file and several child files (containing the chapters)
using the |\include| command.
This mechanism is beneficial for documents
which span hundreds of pages in order to
make the source file(s) more manageable.
Moreover, compilation can be restricted to
selected child files by means of the |\includeonly| command.
The latter feature can be used to reduce the compilation time while editing
(this was significantly more useful in the earlier days of \LaTeX{})
or to generate a smaller document which is easier to navigate.
Another application of |\includeonly| is to generate
documents consisting of selected parts of the complete document.

However, there are a few drawbacks of the plain |\include| mechanism:
\begin{itemize}
\item
The child files cannot be compiled on their own,
they can only be compiled via the main file.
A naive editing environment
(such as a text editor with an option
to have the current file processed by \LaTeX)
may require one to switch to the main file before compiling;
attempting to compile the child file produces errors.
\item
The main file must be modified (each time)
to adjust the |\includeonly| command
to the present needs. This easily leaves the main file in a messy state.
\item
The generated document will always carry the filename
of the main document. This is inconvenient if
several child files are to be compiled and
to be kept for distribution.
\end{itemize}

The present package provides a simple interface
to make child files individually compilable by \LaTeX{}.
Compiling a child file then has the same effect as compiling
the main file with an |\includeonly| command
to select the appropriate child.
Moreover the generated document will carry the name of the child
rather than the main file.
This resolves all three above issues.

This feature is meant to make the editing of books,
thesis documents and lecture notes somewhat more convenient.
However, the package can also be used efficiently for
composing a series of documents (such as exercise sheets)
which are typically distributed individually.
It then assists the author in generating the individual documents
(potentially in different versions)
as well as a document containing the collected series.
Another application is in developing style files
or other kinds of included material
where compilation of the style file could redirect
to a sample or test file.

%%%%%%%%%%%%%%%%%%%%%%%%%%%%%%%%%%%%%%%%%%%%%%%%%%%%%%%%%%%%%%%%%%%%%%%%%%%%%%%%
%%%%%%%%%%%%%%%%%%%%%%%%%%%%%%%%%%%%%%%%%%%%%%%%%%%%%%%%%%%%%%%%%%%%%%%%%%%%%%%%
\section{Usage}

First of all, the package \textsf{childdoc} is \emph{not} a standard
\LaTeXe{} |.sty| style file! Therefore it needs to be invoked in
a non-standard way.

%%%%%%%%%%%%%%%%%%%%%%%%%%%%%%%%%%%%%%%%%%%%%%%%%%%%%%%%%%%%%%%%%%%%%%%%%%%%%%%%
\subsection{Included Files}
\label{sec:include}

%%%%%%%%%%%%%%%%%%%%%%%%%%%%%%%%%%%%%%%%
\DescribeMacro{\childdocmain}
To use the package, add the commands
\begin{center}
\begin{tabular}{l}
|\input{childdoc.def}|\\
|\childdocmain{}|\\
\end{tabular}
\end{center}
at the very top of the main \LaTeX{} file,
in particular \emph{before} the |\documentclass| statement!
The argument of |\childdocmain| should be left empty
(but it must be present).

%%%%%%%%%%%%%%%%%%%%%%%%%%%%%%%%%%%%%%%%
\DescribeMacro{\childdocof}
Furthermore, add the commands
\begin{center}
\begin{tabular}{l}
|\input{childdoc.def}|\\
|\childdocof{|\textit{main}|}|\\
\end{tabular}
\end{center}
at the top of every child file \textit{child}
which is included by |\include{|\textit{child}|}|
from within the main file
(or at least for those files to be compiled individually).
The argument \textit{main} must be the filename of the main file.

There are a couple of
considerations in setting up the main and child documents:

%%%%%%%%%%%%%%%%%%%%%%%%%%%%%%%%%%%%%%%%
\paragraph{Restrictions.}

Please note the following restrictions:
\begin{itemize}
\item
|\childdocmain| must be called with one argument \textit{main}
to ensure compatibility with earlier version of the package.
It must either be empty (|\childdocmain{}|)
or precisely match the filename of the main file in which it is specified.
See \secref{sec:detection} for further information.
\item
The filename \textit{main} must be specified without the |.tex| extension.
\item
The filename \textit{main} is case sensitive
(even in case-insensitive file systems)
due to internal string comparison.
\item
The argument \textit{main} should be fully expanded, it cannot be a macro.
\item
Subdirectories and special characters should be avoided in filenames.
\item
The command |\childdocmain{|\textit{main}|}| must be followed by a whitespace.
It should not be followed immediately by another command
or by a comment mark `|%|'.
This is because the \TeX{} parser reads the token immediately following
the argument of |\childdocmain| and puts it
at the beginning of every child section;
however, a white\-space is ignored.
\end{itemize}

%%%%%%%%%%%%%%%%%%%%%%%%%%%%%%%%%%%%%%%%
\paragraph{Content of Main File.}

It is advisable to place all content in the child files included by |\include|.
Any output contained in the main file will appear in all child documents
unless suppressed manually;
it cannot be suppressed automatically by the |\includeonly| directive
and thus should normally be avoided.
A method to include some content in the main file
by means of conditional processing is described in \secref{sec:conditional}.

%%%%%%%%%%%%%%%%%%%%%%%%%%%%%%%%%%%%%%%%
\paragraph{Page Numbering.}

When only a part of the document is compiled,
the appropriate numbering of pages
(as well as other status parameters)
is determined from the |.aux| files.
The latter contain information from previous passes.
However this information needs to propagate through
all intermediate child documents.
Therefore the page numbering in child documents may well
be inconsistent until the complete document is compiled at least once.

A useful (if unconventional) way to always ensure a consistent
page numbering is to restart the numbering in each child document
and denote the pages by `\textit{child}|.|\textit{page}'
where \textit{child} represents the chapter/section number of the child file.
This can be achieved by the command
|\numberwithin{page}{|\textit{child}|}|
of the \textsf{amsmath} package
where \textit{child} can be |chapter| or |section|
depending on the chosen structuring.
Alternatively, one can modify the macro |\thepage| appropriately
and reset the counter |page| at the start of each child file.

%%%%%%%%%%%%%%%%%%%%%%%%%%%%%%%%%%%%%%%%%%%%%%%%%%%%%%%%%%%%%%%%%%%%%%%%%%%%%%%%
\subsection{Conditional Processing}
\label{sec:conditional}

The package provides a mechanism to compile different versions
of a document. To customise the versions further some conditional processing
can come in handy to distinguish which version is being compiled.
The package provides two macros to describe the compilation context:

%%%%%%%%%%%%%%%%%%%%%%%%%%%%%%%%%%%%%%%%
\DescribeMacro{\ifchilddoc}
The conditional |\ifchilddoc| distinguishes between the compilation of
child documents and the main document:
%
\begin{center}
|\ifchilddoc |\textit{child-code}| |[|\||else |\textit{main-code}]| \||fi|
\end{center}

%%%%%%%%%%%%%%%%%%%%%%%%%%%%%%%%%%%%%%%%
\DescribeMacro{\childdocname}
\DescribeMacro{\childdocjob}
The macro |\childdocname| contains the filename (without extension)
of the main or child file being processed.
Note that |\childdocjob| will always contain the name of the main file.

%%%%%%%%%%%%%%%%%%%%%%%%%%%%%%%%%%%%%%%%
\paragraph{Title Page.}

Conditional processing can be used to include a title or banner page
in the main document when proper precautions are taken.
Importantly, the code in the main file should ensure that the page counter
(as well as other status parameters which are stored in the |.aux| files)
takes the same value after the conditional processing.
Otherwise the page numbers may take divergent values
depending on which part is compiled.

For example, a title page could be declared by:
%
\begin{center}
\begin{tabular}{l}
|\ifchilddoc\||else|\\
|\addtocounter{page}{-1}|\\
\textit{code for title page}\\
|\newpage|\\
|\||fi|
\end{tabular}
\end{center}
%
A banner page for the child documents can be generated by:
%
\begin{center}
\begin{tabular}{l}
|\ifchilddoc|\\
|\addtocounter{page}{-1}|\\
\textit{code for banner page}\\
|\newpage|\\
|\||fi|
\end{tabular}
\end{center}
%
Here one could write a message such as:
\begin{center}
|This is the part \childdocname{} of \childdocjob{}.|
\end{center}

%%%%%%%%%%%%%%%%%%%%%%%%%%%%%%%%%%%%%%%%%%%%%%%%%%%%%%%%%%%%%%%%%%%%%%%%%%%%%%%%
\subsection{Flags}
\label{sec:flags}

The package makes it easy to generate different versions
of the main or child documents.
To this end compilation flags can be defined
and assigned different default values.
They will be particularly useful in conjunction
with the forwarding mechanism described in \secref{sec:forward}.

For example, it may be useful to have a flag |\version|
which can be set to |draft| or |final|.
The document source will contain some conditional code
depending on the value of |\version|.
Suppose further, the flag should default to |final| for the main file
and to |draft| for child files
which is a natural assignment for editing the document.
This is achieved by placing the following code
in the preamble of the main document
(below the |\childdocmain| directive):
%
\begin{center}
\begin{tabular}{l}
|\ifchilddoc|\\
|\providecommand{\version}{draft}|\\
|\||else|\\
|\providecommand{\version}{final}|\\
|\||fi|
\end{tabular}
\end{center}
%
The definition by |\providecommand| makes sure
that previous definitions are not overwritten.
Further statements |\providecommand{\version}{...}|
can thus be added before the above code to override it.

For the main file, one might add a line
(between |\childdocmain| and the above block)
%
\begin{center}
|%\ifchilddoc\||else\providecommand{\version}{draft}\||fi|
\end{center}
%
which can be uncommented to produce a draft version.
Likewise one can add a line to the very top of a child file
(above the |\childdocof{|\textit{main}|}| directive)
%
\begin{center}
|%\providecommand{\version}{final}|
\end{center}
%
which can be uncommented to produce the final version of this child document.

%%%%%%%%%%%%%%%%%%%%%%%%%%%%%%%%%%%%%%%%%%%%%%%%%%%%%%%%%%%%%%%%%%%%%%%%%%%%%%%%
\subsection{Forwarding}
\label{sec:forward}

Different versions of the main or child documents
using compilation flags as described in \secref{sec:flags}
can be (permanently) stored in different files
for convenient compilation, viewing and distribution.
To this end, the package defines a command
to pass on compilation to a different file:

%%%%%%%%%%%%%%%%%%%%%%%%%%%%%%%%%%%%%%%%
\DescribeMacro{\childdocforward}
The command |\childdocforward| redirects processing to
another source file:
%
\begin{center}
\begin{tabular}{l}
|\input{childdoc.def}|\\
|\childdocforward[|\textit{main}|]{|\textit{dest}|}|\\
\end{tabular}
\end{center}
%
The argument \textit{dest} is the destination file
(without extension).
It should be the main file or one of the child files.
Note that further \textsf{childdoc} directives
such as |\childdocof| and |\childdocforward|
in the indicated file will be processed in this form.
The optional argument \textit{main}
passes on directly to the main file \textit{main}
while pretending to compile the child \textit{dest}.
This form behaves as if \textit{dest}
issues |\childdocof{|\textit{main}|}| right away,
and no further \textsf{childdoc} directives will be processed.

%%%%%%%%%%%%%%%%%%%%%%%%%%%%%%%%%%%%%%%%
\DescribeMacro{\...prefix}
In the alternative form |\childdocforwardprefix|,
%
\begin{center}
\begin{tabular}{l}
|\input{childdoc.def}|\\
|\childdocforwardprefix[|\textit{main}|]{|\textit{prefix}|}{|\textit{dest}|}|
\end{tabular}
\end{center}
%
the destination file is determined by a pattern
depending on the current file:
To make this work, the current file must be called
`{\textit{prefix}\hspace{0.2em}\textit{suffix}}'
with \textit{prefix} matching precisely the argument.
Processing is then passed on to the file
`{\textit{dest}\hspace{0.2em}\textit{suffix}}'.
Surely, the same effect is achieved by
directly specifying the
argument `{\textit{dest}\hspace{0.2em}\textit{suffix}}'
in the first form.
However, that requires to set up a different file
for each child. With the alternative form of the command
all these files can have exactly the same content
which simplifies setting them up and maintaining them.

For example, the following file |draft.tex|
with a compilation flag |\version| as described in \secref{sec:flags}
compiles the main document as a draft:
%
\begin{center}
\begin{tabular}{l}
|\def\version{draft}|\\
|\input{childdoc.def}|\\
|\childdocforward{|\textit{main}|}|
\end{tabular}
\end{center}
%
Likewise, the following files |final|\textit{nn}|.tex|
compile the final version of the child document
|child|\textit{nn}|.tex|:
%
\begin{center}
\begin{tabular}{l}
|\def\version{final}|\\
|\input{childdoc.def}|\\
|\childdocforwardprefix{final}{child}|
\end{tabular}
\end{center}
%

Note that when several versions of a main file and/or of each child file
are to be generated, it may be convenient to set up a |Makefile| or
shell script to automatise the process.

%%%%%%%%%%%%%%%%%%%%%%%%%%%%%%%%%%%%%%%%%%%%%%%%%%%%%%%%%%%%%%%%%%%%%%%%%%%%%%%%
\subsection{Command Line Processing}
\label{sec:commandline}

The effect of redirection files can also be achieved by invoking
the \LaTeX{} compiler with a more elaborate command line.
Most conveniently this should be done as part
of a shell script or a |Makefile|.

When using \textsf{childdoc} in the main file, the following
command lines effectively perform a redirection
(note that depending on the shell being used,
backslashes may have to be doubled: `|\|' $\to$ `|\\|'):
%
\begin{center}
|... -jobname "|\textit{target}|" |\\|"|[\textit{flags}]%
|\input{childdoc.def}\childdocforward[|\textit{main}|]{|\textit{dest}|}"|
\end{center}
%
Here \textit{target} is the name of the output file,
\textit{main} is the name of the main file
and \textit{dest} is the name of the main or child file to be processed
(all filenames without extensions).
The optional argument \textit{main} can be omitted
if \textit{main} matches \textit{dest}.
Optionally, compilation \textit{flags} can be defined via |\def| commands.
This command line makes the \TeX{} engine believe
it is compiling the file \textit{target}
whose content is specified as the latter parameter.
The provided code then forwards the processing to
\textit{main} or \textit{dest} as described in \secref{sec:forward}.

%%%%%%%%%%%%%%%%%%%%%%%%%%%%%%%%%%%%%%%%%%%%%%%%%%%%%%%%%%%%%%%%%%%%%%%%%%%%%%%%
\subsection{Include by Input}
\label{sec:input}

Including child documents by |\include| has some restrictions by design.
Most notably, the content of a child document always occupies
its own set of pages; pages cannot be shared between child documents.
Usually, this behaviour makes perfect sense
because each child document contain an essential part of the document.
However, in some situations it may be desirable to compose
a document from a collection of parts
without having mandatory page breaks between then.
For this case, the package
provides a mechanism to include parts
by |\input| which can also be processed individually.
However, by construction this mechanism
requires manual handling of the content to be output.

%%%%%%%%%%%%%%%%%%%%%%%%%%%%%%%%%%%%%%%%
\DescribeMacro{\ifchilddocmanual}
The main file should be prepared as usual, see \secref{sec:include}.
However, the document body must make a distinction
between processing of an individual part and of the main document, e.g.:
%
\begin{center}
\begin{tabular}{l}
|\ifchilddocmanual|\\
|\input{\childdocname}|\\
|\||else|\\
\textit{document body with }|\input{|\textit{part}|}|\\
|\||fi|
\end{tabular}
\end{center}
%
The conditional |\ifchilddocmanual| is true whenever
a part to be included by |\input| is being compiled,
and the name of the part is stored in |\childdocname|.

%%%%%%%%%%%%%%%%%%%%%%%%%%%%%%%%%%%%%%%%
\DescribeMacro{\childdocby}
Each part to be included by |\input| should start with:
%
\begin{center}
\begin{tabular}{l}
|\input{childdoc.def}|\\
|\childdocby{|\textit{main}|}|\\
\end{tabular}
\end{center}
%
The directive |\childdocby| is similar to |\childdocof|
described in \secref{sec:include},
but the subsequent selection of content must be done manually.
To that end, both |\ifchilddoc| and |\ifchilddocmanual|
will be true upon processing of a part,
and the name of the part is stored in |\childdocname|.
Note that |\jobname| will be set to the filename of the current part
so that each part receives an individual |.aux| file
that does not interfere with the |.aux| file(s) of the main document.
This behaviour can be altered by the alternative form
|\childdocby[*]{|\textit{main}|}| (with a non-empty optional argument)
which uses the |.aux| file of the main document
by setting |\jobname| to \textit{main}.

%%%%%%%%%%%%%%%%%%%%%%%%%%%%%%%%%%%%%%%%%%%%%%%%%%%%%%%%%%%%%%%%%%%%%%%%%%%%%%%%
\subsection{Driver Development}
\label{sec:driver}

The \textsf{childdoc} mechanism can also be use for the development
of definition files such as \LaTeX{} styles or classes.
This case differs from the above setup with multiple parts
included by |\include| in that no |\includeonly| should be invoked.
This can be achieved by starting the include file
(before |\ProvidesPackage|) with:
%
\begin{center}
\begin{tabular}{l}
|\input{childdoc.def}|\\
|\childdocforward{|\textit{main}|}|\\
\end{tabular}
\end{center}
%
or alternatively with:
%
\begin{center}
\begin{tabular}{l}
|\input{childdoc.def}|\\
|\childdocby{|\textit{main}|}|\\
\end{tabular}
\end{center}
%
Both forms have slightly different effects as described above.
The main file is prepared as usual, see \secref{sec:include}.

%%%%%%%%%%%%%%%%%%%%%%%%%%%%%%%%%%%%%%%%%%%%%%%%%%%%%%%%%%%%%%%%%%%%%%%%%%%%%%%%
\subsection{Legacy Detection}
\label{sec:detection}

The directive |\childdocmain| in the main file can detect
whether the complete document or merely a child is to be compiled
even without using the directive |\childdocof|.
This method is deprecated because it is less robust
and there is no compelling reason to use it;
it is merely provided for backward compatibility
and it may be removed in future versions.

If the detection mechanism is to be used,
it is mandatory to correctly specify
the filename of the main file as the argument of |\childdocmain|:
%
\begin{center}
\begin{tabular}{l}
|\input{childdoc.def}|\\
|\childdocmain{|\textit{main}|}|\\
\end{tabular}
\end{center}
%
If |\jobname| does not match the argument \textit{main} of |\childdocmain|,
it is assumed that |\jobname| points to the child file to be compiled.
When using |\childdocmain| with the main file specified as argument,
it suffices to start a child file
with just |\input{|\textit{main}|}|
without loading of the package and using |\childdocof|.
If instead all processing is done
with the appropriate \textsf{childdoc} directives,
the argument of \textit{main} of |\childdocmain| can be empty.

An alternative version of the command line processing described
in \secref{sec:commandline} using the detection mechanism reads:
%
\begin{center}
|... -jobname "|\textit{target}|" "|[\textit{flags}]%
[|\def\jobname{|\textit{dest}|}|]|\input{|\textit{main}|}"|
\end{center}

%%%%%%%%%%%%%%%%%%%%%%%%%%%%%%%%%%%%%%%%%%%%%%%%%%%%%%%%%%%%%%%%%%%%%%%%%%%%%%%%
\subsection{Manual Code}
\label{sec:manual}

In case one cannot be certain whether the definitions file |childdoc.def|
is installed on the target \TeX{} distribution
and one prefers not to ship it,
it is conceivable to paste a few relevant commands into the sources.

To that end, drop all statements |\input{childdoc.def}|
and perform the replacements as outlined below.
Instead of |\childdocmain{|\textit{main}|}| add the following code
to the top of the main file:
%
\begin{center}
\begin{tabular}{l}
|\||ifdefined\childdocname\endinput\||fi\newif\ifchilddoc|\\
|\edef\childdocname{\scantokens\expandafter{\jobname\noexpand}}|\\
|\def\childdocmain{|\textit{main}|}\||ifx\childdocmain\childdocname\||else|\\
|\childdoctrue\includeonly{\childdocname}\let\jobname\childdocmain\||fi|\\
\end{tabular}
\end{center}
%
Instead of |\childdocof{|\textit{main}|}| just include the main file
at the top of each child file:
%
\begin{center}
|\input{|\textit{main}|}|
\end{center}
%
A simple redirection |\childdocforward{|\textit{dest}|}| is achieved by:
%
\begin{center}
|\def\jobname{|\textit{dest}|}\input{\jobname}|
\end{center}
%
The redirection with prefix
|\childdocforwardprefix[|\textit{prefix}|]{|\textit{dest}|}|
is accomplished by:
%
\begin{center}
\begin{tabular}{l}
|{\edef\jobname{\scantokens\expandafter{\jobname\noexpand}}|\\
|\def\redirectjob |\textit{prefix}|#1~~~{\gdef\jobname{|\textit{dest}|#1}}|\\
|\expandafter\redirectjob\jobname~~~}\input{\jobname}|
\end{tabular}
\end{center}

In an alternative approach,
child documents can be compiled by a specific command line
without additional code or specific definitions:
%
\begin{center}
|... -jobname "|\textit{target}|" "|[\textit{flags}]%
|\includeonly{|\textit{dest}|}\input{|\textit{main}|}"|
\end{center}
%

%%%%%%%%%%%%%%%%%%%%%%%%%%%%%%%%%%%%%%%%%%%%%%%%%%%%%%%%%%%%%%%%%%%%%%%%%%%%%%%%
%%%%%%%%%%%%%%%%%%%%%%%%%%%%%%%%%%%%%%%%%%%%%%%%%%%%%%%%%%%%%%%%%%%%%%%%%%%%%%%%
\section{Information}

%%%%%%%%%%%%%%%%%%%%%%%%%%%%%%%%%%%%%%%%%%%%%%%%%%%%%%%%%%%%%%%%%%%%%%%%%%%%%%%%
\subsection{Copyright}

Copyright \copyright{} 2017--2018 Niklas Beisert

This work may be distributed and/or modified under the
conditions of the \LaTeX{} Project Public License, either version 1.3
of this license or (at your option) any later version.
The latest version of this license is in
  \url{http://www.latex-project.org/lppl.txt}
and version 1.3 or later is part of all distributions of \LaTeX{}
version 2005/12/01 or later.

This work has the LPPL maintenance status `maintained'.

The Current Maintainer of this work is Niklas Beisert.

This work consists of the files |README.txt|, |childdoc.ins| and |childdoc.dtx|
as well as the derived files |childdoc.def|, |cdocsamp.tex|
with |cdocsch1.tex|, |cdocsch2.tex|, |cdocspt3.tex|, |cdocspt4.tex|,
|cdocsdrf.tex|, |cdocsfn1.tex|, |cdocsfn2.tex|
as well as |childdoc.pdf|.

%%%%%%%%%%%%%%%%%%%%%%%%%%%%%%%%%%%%%%%%%%%%%%%%%%%%%%%%%%%%%%%%%%%%%%%%%%%%%%%%
\subsection{Files and Installation}

The package consists of the files:
%
\begin{center}
\begin{tabular}{ll}
    |README.txt|   & readme file \\
    |childdoc.ins| & installation file \\
    |childdoc.dtx| & source file \\
    |childdoc.def| & definition file \\
    |cdocsamp.tex| & sample main file \\
    |cdocsch1.tex| & sample include file \\
    |cdocsch2.tex| & sample include file \\
    |cdocspt3.tex| & sample part file \\
    |cdocspt4.tex| & sample part file \\
    |cdocsdrf.tex| & sample redirection file \\
    |cdocsfn1.tex| & sample redirection file \\
    |cdocsfn2.tex| & sample redirection file \\
    |childdoc.pdf| & manual
\end{tabular}
\end{center}
%
The distribution consists of the files
|README.txt|, |childdoc.ins| and |childdoc.dtx|.
%
\begin{itemize}
\item
Run (pdf)\LaTeX{} on |childdoc.dtx|
to compile the manual |childdoc.pdf| (this file).
\item
Run \LaTeX{} on |childdoc.ins| to create the definitions file |childdoc.def|
and the sample |cdocsamp.tex| with include files
|cdocsch1.tex|, |cdocsch2.tex|, |cdocspt3.tex|, |cdocspt4.tex|,
|cdocsdrf.tex|, |cdocsfn1.tex|, |cdocsfn2.tex|.
Then copy the file |childdoc.def| to an appropriate directory of your \LaTeX{}
distribution, e.g.\ \textit{texmf-root}|/tex/latex/childdoc|.
\end{itemize}

%%%%%%%%%%%%%%%%%%%%%%%%%%%%%%%%%%%%%%%%%%%%%%%%%%%%%%%%%%%%%%%%%%%%%%%%%%%%%%%%
\subsection{Related CTAN Packages}

There are several other packages which offer a similar functionality:
%
\begin{itemize}
\item
The packages
\href{http://ctan.org/pkg/docmute}{\textsf{docmute}},
\href{http://ctan.org/pkg/includex}{\textsf{includex}} and
\href{http://ctan.org/pkg/standalone}{\textsf{standalone}}
provide commands to include only the document body of
a child file thus allowing both files to be compiled individually.
\item
The packages \href{http://ctan.org/pkg/subdocs}{\textsf{subdocs}}
and \href{http://ctan.org/pkg/subfiles}{\textsf{subfiles}}
provide structures in which the main and child documents can be
encapsulated and allowing them to be compiled individually.
The inclusion mechanism is different from the conventional |\include|.
\item
The package \href{http://ctan.org/pkg/combine}{\textsf{combine}}
is an elaborate solution to combine several documents into one.
\end{itemize}
%
See also the CTAN topic \href{http://ctan.org/topic/subdocs}{\textsf{subdocs}}
for further related packages.
The present package differs from the above solutions in that
a document structure constructed with the conventional |\include| mechanism
just needs two extra commands at the top of every file
such that all constituent files can be compiled individually.

%%%%%%%%%%%%%%%%%%%%%%%%%%%%%%%%%%%%%%%%%%%%%%%%%%%%%%%%%%%%%%%%%%%%%%%%%%%%%%%%
%\subsection{Feature Suggestions}
%
%The following is a list of features which may be useful for future
%versions of this package:
%%
%\begin{itemize}
%\item
%\ldots
%\end{itemize}

%%%%%%%%%%%%%%%%%%%%%%%%%%%%%%%%%%%%%%%%%%%%%%%%%%%%%%%%%%%%%%%%%%%%%%%%%%%%%%%%
\subsection{Revision History}

%%%%%%%%%%%%%%%%%%%%%%%%%%%%%%%%%%%%%%%%
\paragraph{v2.0:} 2018/12/30

\begin{itemize}
\item
immediate forward processing
\item
added |\childdocby| mechanism
\item
manual restructured
\end{itemize}

%%%%%%%%%%%%%%%%%%%%%%%%%%%%%%%%%%%%%%%%
\paragraph{v1.6:} 2018/01/17

\begin{itemize}
\item
application for development of include files
\item
corrections to manual
\end{itemize}

%%%%%%%%%%%%%%%%%%%%%%%%%%%%%%%%%%%%%%%%
\paragraph{v1.5:} 2017/05/21

\begin{itemize}
\item
more complete structuring introduced
\item
|\childdocof| introduced
\item
|\childdoc| renamed to |\childdocmain|
\item
|\childredirect| renamed to |\childdocforward| and |\childdocforwardprefix|
and functionality expanded
\end{itemize}

%%%%%%%%%%%%%%%%%%%%%%%%%%%%%%%%%%%%%%%%
\paragraph{v1.0:} 2017/04/27

\begin{itemize}
\item
manual and install package
\item
first version published on CTAN
\end{itemize}

%%%%%%%%%%%%%%%%%%%%%%%%%%%%%%%%%%%%%%%%
\paragraph{v0.6:} 2017/04/26

\begin{itemize}
\item
redirection mechanism added
\end{itemize}

%%%%%%%%%%%%%%%%%%%%%%%%%%%%%%%%%%%%%%%%
\paragraph{v0.5:} 2017/04/26

\begin{itemize}
\item
functionality in definition file
\end{itemize}


%%%%%%%%%%%%%%%%%%%%%%%%%%%%%%%%%%%%%%%%%%%%%%%%%%%%%%%%%%%%%%%%%%%%%%%%%%%%%%%%
%%%%%%%%%%%%%%%%%%%%%%%%%%%%%%%%%%%%%%%%%%%%%%%%%%%%%%%%%%%%%%%%%%%%%%%%%%%%%%%%
%%%%%%%%%%%%%%%%%%%%%%%%%%%%%%%%%%%%%%%%%%%%%%%%%%%%%%%%%%%%%%%%%%%%%%%%%%%%%%%%
\appendix

\settowidth\MacroIndent{\rmfamily\scriptsize 000\ }

 \DocInput{childdoc.dtx}

\end{document}
%</driver>
% \fi
%
% %%%%%%%%%%%%%%%%%%%%%%%%%%%%%%%%%%%%%%%%%%%%%%%%%%%%%%%%%%%%%%%%%%%%%%%%%%%%%%
% %%%%%%%%%%%%%%%%%%%%%%%%%%%%%%%%%%%%%%%%%%%%%%%%%%%%%%%%%%%%%%%%%%%%%%%%%%%%%%
% \section{Sample}
%\iffalse
%<*samplemain>
%\fi
%
% The following presents a sample document
% with two chapters, two parts, a title page,
% a compile flag as well as three forwarding files to set the flag.
% It consists of eight |.tex| files:
% \begin{center}
% \begin{tabular}{ll}
% |cdocsamp.tex|&main file\\
% |cdocsch1.tex|&include file for chapter 1\\
% |cdocsch2.tex|&include file for chapter 2\\
% |cdocspt3.tex|&include file for part 3\\
% |cdocspt4.tex|&include file for part 4\\
% |cdocsdrf.tex|&forwarding file for main file in draft mode\\
% |cdocsfi1.tex|&forwarding file for final version of chapter 1\\
% |cdocsfi2.tex|&forwarding file for final version of chapter 2\\
% \end{tabular}
% \end{center}
% Each of the eight files can be compiled directly by the \LaTeX{} compiler.
%
% %%%%%%%%%%%%%%%%%%%%%%%%%%%%%%%%%%%%%%
% \paragraph{Main File.}
%
% The main file is called |cdocsamp.tex|.
%
% Load the \textsf{childdoc} definitions and
% declare the filename for the main document:
%    \begin{macrocode}
\input{childdoc.def}
\childdocmain{}
%    \end{macrocode}

% Optional override for |\version| flag:
%    \begin{macrocode}
%%\ifchilddoc\else\providecommand{\version}{draft}\fi
%    \end{macrocode}

% Define the default values for the |\version| flag
% (|final| for the main file and |draft| for childs):
%    \begin{macrocode}
\ifchilddoc
\providecommand{\version}{draft}
\else
\providecommand{\version}{final}
\fi
%    \end{macrocode}

% Load the standard document class:
%    \begin{macrocode}
\documentclass[12pt]{article}
%    \end{macrocode}

% Start the document body:
%    \begin{macrocode}
\begin{document}
%    \end{macrocode}

% Declare a title page.
% Print title, part of document being processed and version flag:
%    \begin{macrocode}
\addtocounter{page}{-1}
\begin{center}
{\LARGE\bfseries{}childdoc example\par}
\vspace{1cm}
\ifchilddoc
\ifchilddocmanual part\else chapter\fi:
`\childdocname' of `\childdocjob'\par
\else
main document: `\childdocjob'\par
\fi
version: \version\par
\end{center}
\newpage
%    \end{macrocode}

% Manually include selected file,
% otherwise process as usual:
%    \begin{macrocode}
\ifchilddocmanual
\section*{part `\childdocname'}
\input{\childdocname}
\else
%    \end{macrocode}

% Include the two chapters:
%    \begin{macrocode}
\include{cdocsch1}
\include{cdocsch2}
%    \end{macrocode}

% Include the two parts unless only chapters should be displayed:
%    \begin{macrocode}
\ifchilddoc\else
\section{part three}
\input{cdocspt3}
\section{part four}
\input{cdocspt4}
\fi
%    \end{macrocode}

% Process as usual until here:
%    \begin{macrocode}
\fi
%    \end{macrocode}

% End of document body:
%    \begin{macrocode}
\end{document}
%    \end{macrocode}
%\iffalse
%</samplemain>
%\fi
%
% %%%%%%%%%%%%%%%%%%%%%%%%%%%%%%%%%%%%%%
% \paragraph{Chapter Include Files.}
%
% The include files are called |cdocsch1.tex| and |cdocsch2.tex|.
%
%\iffalse
%<*samplechap1|samplechap2>
%\fi

% Optional override for |\version| flag:
%    \begin{macrocode}
%%\providecommand{\version}{final}
%    \end{macrocode}

% Include the main document:
%    \begin{macrocode}
\input{childdoc.def}
\childdocof{cdocsamp}
%    \end{macrocode}

%\iffalse
%</samplechap1|samplechap2>
%\fi
%
%\iffalse
%<*samplechap1>
%\fi
% Some text for chapter 1:
%    \begin{macrocode}
\section{one}
some text in chapter one
%    \end{macrocode}

%\iffalse
%</samplechap1>
%\fi
% Some text for chapter 2:
%\iffalse
%<*samplechap2>
%\fi
%    \begin{macrocode}
\section{two}
more text in chapter two
%    \end{macrocode}

%\iffalse
%</samplechap2>
%\fi
%
% %%%%%%%%%%%%%%%%%%%%%%%%%%%%%%%%%%%%%%
% \paragraph{Part Include Files.}
%
% The include files are called |cdocspt3.tex| and |cdocspt4.tex|.
%
%\iffalse
%<*samplepart3|samplepart4>
%\fi

% Optional override for |\version| flag:
%    \begin{macrocode}
%%\providecommand{\version}{final}
%    \end{macrocode}

% Include the main document:
%    \begin{macrocode}
\input{childdoc.def}
\childdocby{cdocsamp}
%    \end{macrocode}

%\iffalse
%</samplepart3|samplepart4>
%\fi
%
%\iffalse
%<*samplepart3>
%\fi
% Some text for part 3:
%    \begin{macrocode}
some text in part three
%    \end{macrocode}

%\iffalse
%</samplepart3>
%\fi
% Some text for part 4:
%\iffalse
%<*samplepart4>
%\fi
%    \begin{macrocode}
more text in part four
%    \end{macrocode}

%\iffalse
%</samplepart4>
%\fi
%
% %%%%%%%%%%%%%%%%%%%%%%%%%%%%%%%%%%%%%%
% \paragraph{Forwarding for a Complete Draft.}
%
% The following forwarding file |cdocsdrf.tex|
% compiles the main document in draft mode:
%\iffalse
%<*sampledraft>
%\fi
%    \begin{macrocode}
\def\version{draft}
\input{childdoc.def}
\childdocforward{cdocsamp}
%    \end{macrocode}

%\iffalse
%</sampledraft>
%\fi
%
% %%%%%%%%%%%%%%%%%%%%%%%%%%%%%%%%%%%%%%
% \paragraph{Forwarding for Final Version of the Chapters.}
%
% The following forwarding files |cdocsfn1.tex| and |cdocsfn2.tex|
% (with identical content)
% compile the final versions of the child documents
% |cdocsch1.tex| and |cdocsch2.tex|, respectively:
%\iffalse
%<*samplefinal>
%\fi
%    \begin{macrocode}
\def\version{final}
\input{childdoc.def}
\childdocforwardprefix[cdocsamp]{cdocsfn}{cdocsch}
%    \end{macrocode}

%\iffalse
%</samplefinal>
%\fi
%
% %%%%%%%%%%%%%%%%%%%%%%%%%%%%%%%%%%%%%%
% \paragraph{Command Line Processing.}
%
% The following three command lines generate the output files
% |cdocscld|, |cdocscl1| and |cdocscl2|
% which should be identical to
% |cdocsdrf|, |cdocsch1| and |cdocsfn2|, respectively:
% \begin{center}
% \begin{tabular}{l}
% |latex -jobname cdocscld \|\\
% |  "\def\version{draft}\input{childdoc.def}\childdocforward{cdocsamp}"|\\
% |latex -jobname cdocscl1 \|\\
% |  "\input{childdoc.def}\childdocforward[cdocsamp]{cdocsch1}"|\\
% |latex -jobname cdocscl2 \|\\
% |  "\def\version{final}\input{childdoc.def}\childdocforward{cdocsch2}"|
% \end{tabular}
% \end{center}
% Note that the trailing backslash on each first line
% merely continues the input to the second line
% (for convenient cut ant paste).
% Furthermore, the command |latex| can be replaced by any
% of its alternative versions such as |pdflatex|.
%
% %%%%%%%%%%%%%%%%%%%%%%%%%%%%%%%%%%%%%%%%%%%%%%%%%%%%%%%%%%%%%%%%%%%%%%%%%%%%%%
% %%%%%%%%%%%%%%%%%%%%%%%%%%%%%%%%%%%%%%%%%%%%%%%%%%%%%%%%%%%%%%%%%%%%%%%%%%%%%%
% \section{Implementation}
%\iffalse
%<*package>
%\fi
%
% This section describes the definitions file |childdoc.def|.

% The definitions cannot be loaded using |\usepackage| or |\RequirePackage|
% which has a mechanism to prevent loading a style file more than once.
% When loading the definitions by means of |\input|
% multiple instances have to be prevented manually:
%\iffalse
%This code needs to be before the `\ProvidesFile' directive
%which is defined at the beginning of this file.
%Therefore it is also placed there and commented out here.
%</package>
%<*discard>
%\fi
%    \begin{macrocode}
\ifdefined\childdocmain\endinput\fi
%    \end{macrocode}
%\iffalse
%</discard>
%<*package>
%\fi
%
% \macro{\ifchilddoc}
% \macro{\ifchilddocmanual}
% The conditional |\ifchilddoc| tells whether a
% child (true) or main (false) document is being compiled.
% The conditional |\ifchilddocmanual| tells whether
% the |\includeonly| mechanism is used (false) or
% the selection of child files must be performed manually (true).
% The definitions initialise to false:
%    \begin{macrocode}
\newif\ifchilddoc
\newif\ifchilddocmanual
%    \end{macrocode}

% \macro{\childdocname}
% \macro{\childdocjob}
% The macro |\childdocname| stores the name of the main document
% to be compiled. The macro |\childdocjob| stores the name of
% the document on which the \LaTeX{} compiler was originally invoked.
% The content of |\jobname| cannot be compared
% to filenames specified in the source due to different catcodes.
% The following code rescans |\jobname|, stores the result
% in |\childdocname| and saves a copy in |\childdocjob|:
%    \begin{macrocode}
\edef\childdocname{\scantokens\expandafter{\jobname\noexpand}}
\let\childdocjob\childdocname
%    \end{macrocode}

% \macro{\childdocdisable}
% The macro |\childdocdisable| prevents the main file
% from being processed more than once.
% At this stage, the main document command |\childdocmain|
% is assumed to be called once again where it should do nothing.
% Any subsequent call to it should prevent
% a secondary processing of the main document
% It overwrites the forwarding commands
% |\childdocof| and |\childdocforward|
% with empty macros to prevent further inclusions of the main document:
%    \begin{macrocode}
\newcommand{\childdocdisable}
{
  \renewcommand{\childdocmain}[1]{\renewcommand{\childdocmain}[1]{\endinput}}
  \renewcommand{\childdocof}[1]{}
  \renewcommand{\childdocby}[2][]{}
  \renewcommand{\childdocforward}[2][]{}
  \renewcommand{\childdocdisable}{}
}
%    \end{macrocode}

% \macro{\childdocmain}
% The macro |\childdocmain| is to be called at the top of the main file
% with nothing or the main filename (without extension) as argument.
% First, it breaks loops.
% If the argument is not empty and does not match |\childdocname|
% (which is set by the first inclusion of |childdoc.def|),
% |\ifchilddoc| is set to true, |\includeonly| is applied to the child file
% and |\jobname| is set to the main file
% (for proper handling of |.aux| files):
%    \begin{macrocode}
\newcommand{\childdocmain}[1]
{
  \childdocdisable\childdocmain{}
  \if?#1?\else
    \begingroup
      \def\childdoctmp{#1}
      \ifx\childdoctmp\childdocname
        \def\childdoctmp{}
      \else
        \def\childdoctmp
        {
          \childdoctrue
          \includeonly{\childdocname}
          \def\childdocjob{#1}
          \def\jobname{#1}
        }
      \fi
      \expandafter
    \endgroup
    \childdoctmp
  \fi
}
%    \end{macrocode}

% \macro{\childdocof}
% The command |\childdocof| redirects
% compilation to the main file |#1|.
%    \begin{macrocode}
\newcommand{\childdocof}[1]
{
  \childdocdisable
  \childdoctrue
  \includeonly{\childdocname}
  \def\jobname{#1}
  \def\childdocjob{#1}
  \input{#1}
}
%    \end{macrocode}

% \macro{\childdocby}
% The command |\childdocby| ....
%    \begin{macrocode}
\newcommand{\childdocby}[2][]
{
  \childdocdisable
  \childdoctrue
  \childdocmanualtrue
  \if?#1?\else
    \def\jobname{#2}
  \fi
  \def\childdocjob{#2}
  \input{#2}
  \endinput
}
%    \end{macrocode}

% \macro{\childdocforward}
% The command |\childdocforward| redirects
% compilation to the main file or
% (if the optional argument is given) a child file.
% Parameters are set as if the main file
% or a child file starting with |\childdocof| was compiled.
% Then compilation is handed over to the main file:
%    \begin{macrocode}
\newcommand{\childdocforward}[2][]
{
  \begingroup
    \if?#1?
      \def\childdoctmp
      {
        \def\childdocname{#2}
        \def\childdocjob{#2}
        \def\jobname{#2}
        \input{#2}
        \endinput
      }
    \else
      \def\childdoctmp
      {
        \childdocdisable
        \def\childdocname{#2}
        \childdoctrue
        \includeonly{#2}
        \def\childdocjob{#1}
        \def\jobname{#1}
        \input{#1}
        \endinput
      }
    \fi
    \expandafter
  \endgroup
  \childdoctmp
}
%    \end{macrocode}

% \macro{\childdocforwardprefix}
% The command |\childdocforwardprefix| redirects
% compilation to the main or a child file by means of a pattern.
% The prefix |#1| in the current filename is replaced by |#2|
% and the suffix of the current filename is kept
% (it is assumed that the filename does not contain the substring `|~~~|'
% which is used as a delimiter).
% Compilation is handed over to the new file by |\childdocforward|:
%    \begin{macrocode}
\newcommand{\childdocforwardprefix}[3][]
{
  \begingroup
    \def\childdocextract #2##1~~~{\def\childdoctmp{\childdocforward[#1]{#3##1}}}
    \expandafter\childdocextract\childdocname~~~
    \expandafter
  \endgroup
  \childdoctmp
}
%    \end{macrocode}

% \macro{\childdoc}
% The deprecated macro |\childdoc| is a legacy version of |\childdocmain|:
%    \begin{macrocode}
\newcommand{\childdoc}{\childdocmain}
%    \end{macrocode}

% \macro{\childdocredirect}
% The deprecated macro |\childdocredirect| is a legacy version
% of |\childdocforward| and |\childdocforwardprefix|:
%    \begin{macrocode}
\newcommand{\childdocredirect}[2][]
{
  \begingroup
    \if?#1?
      \def\childdoctmp{\childdocforward{#2}}
    \else
      \def\childdoctmp{\childdocforwardprefix{#1}{#2}}
    \fi
    \expandafter
  \endgroup
  \childdoctmp
}
%    \end{macrocode}

%\iffalse
%</package>
%\fi
%
\endinput
\childdocforward[|\textit{main}|]{|\textit{dest}|}"|
\end{center}
%
Here \textit{target} is the name of the output file,
\textit{main} is the name of the main file
and \textit{dest} is the name of the main or child file to be processed
(all filenames without extensions).
The optional argument \textit{main} can be omitted
if \textit{main} matches \textit{dest}.
Optionally, compilation \textit{flags} can be defined via |\def| commands.
This command line makes the \TeX{} engine believe
it is compiling the file \textit{target}
whose content is specified as the latter parameter.
The provided code then forwards the processing to
\textit{main} or \textit{dest} as described in \secref{sec:forward}.

%%%%%%%%%%%%%%%%%%%%%%%%%%%%%%%%%%%%%%%%%%%%%%%%%%%%%%%%%%%%%%%%%%%%%%%%%%%%%%%%
\subsection{Include by Input}
\label{sec:input}

Including child documents by |\include| has some restrictions by design.
Most notably, the content of a child document always occupies
its own set of pages; pages cannot be shared between child documents.
Usually, this behaviour makes perfect sense
because each child document contain an essential part of the document.
However, in some situations it may be desirable to compose
a document from a collection of parts
without having mandatory page breaks between then.
For this case, the package
provides a mechanism to include parts
by |\input| which can also be processed individually.
However, by construction this mechanism
requires manual handling of the content to be output.

%%%%%%%%%%%%%%%%%%%%%%%%%%%%%%%%%%%%%%%%
\DescribeMacro{\ifchilddocmanual}
The main file should be prepared as usual, see \secref{sec:include}.
However, the document body must make a distinction
between processing of an individual part and of the main document, e.g.:
%
\begin{center}
\begin{tabular}{l}
|\ifchilddocmanual|\\
|\input{\childdocname}|\\
|\||else|\\
\textit{document body with }|\input{|\textit{part}|}|\\
|\||fi|
\end{tabular}
\end{center}
%
The conditional |\ifchilddocmanual| is true whenever
a part to be included by |\input| is being compiled,
and the name of the part is stored in |\childdocname|.

%%%%%%%%%%%%%%%%%%%%%%%%%%%%%%%%%%%%%%%%
\DescribeMacro{\childdocby}
Each part to be included by |\input| should start with:
%
\begin{center}
\begin{tabular}{l}
|% \iffalse
%
% childdoc.dtx Copyright (C) 2017-2018 Niklas Beisert
%
% This work may be distributed and/or modified under the
% conditions of the LaTeX Project Public License, either version 1.3
% of this license or (at your option) any later version.
% The latest version of this license is in
%   http://www.latex-project.org/lppl.txt
% and version 1.3 or later is part of all distributions of LaTeX
% version 2005/12/01 or later.
%
% This work has the LPPL maintenance status `maintained'.
%
% The Current Maintainer of this work is Niklas Beisert.
%
% This work consists of the files childdoc.dtx and childdoc.ins
% and the derived files childdoc.def and cdocsamp.tex with
% cdocsch1.tex, cdocsch2.tex, cdocsdrf.tex, cdocsfn1.tex, cdocsfn2.tex.
%
%<package>\ifdefined\childdocmain\endinput\fi
%<package>\ProvidesFile{childdoc.def}[2018/12/30 v2.0 child document driver]
%<samplemain>\ProvidesFile{cdocsamp.tex}[2018/12/30 v2.0 sample for childdoc]
%<*driver>
%\ProvidesFile{childdoc.drv}[2018/12/30 v2.0 childdoc reference manual file]
\PassOptionsToClass{10pt,a4paper}{article}
\documentclass{ltxdoc}

\usepackage[margin=35mm]{geometry}
\usepackage{hyperref}
\usepackage{hyperxmp}
\usepackage[usenames]{color}

\hypersetup{colorlinks=true}
\hypersetup{pdfstartview=FitH}
\hypersetup{pdfpagemode=UseNone}
\hypersetup{pdfsource={}}
\hypersetup{pdflang={en-UK}}
\hypersetup{pdfcopyright={Copyright 2017-2018 Niklas Beisert.
  This work may be distributed and/or modified under the
  conditions of the LaTeX Project Public License, either version 1.3
  of this license or (at your option) any later version.}}
\hypersetup{pdflicenseurl={http://www.latex-project.org/lppl.txt}}
\hypersetup{pdfcontactaddress={ETH Zurich, ITP, HIT K,
  Wolfgang-Pauli-Strasse 27}}
\hypersetup{pdfcontactpostcode={8093}}
\hypersetup{pdfcontactcity={Zurich}}
\hypersetup{pdfcontactcountry={Switzerland}}
\hypersetup{pdfcontactemail={nbeisert@itp.phys.ethz.ch}}
\hypersetup{pdfcontacturl={http://people.phys.ethz.ch/\xmptilde nbeisert/}}

\newcommand{\secref}[1]{\hyperref[#1]{section \ref*{#1}}}

\parskip1ex
\parindent0pt
\let\olditemize\itemize
\def\itemize{\olditemize\parskip0pt}

\begin{document}

\title{The \textsf{childdoc} Package}
\hypersetup{pdftitle={The childdoc Package}}
\author{Niklas Beisert\\[2ex]
  Institut f\"ur Theoretische Physik\\
  Eidgen\"ossische Technische Hochschule Z\"urich\\
  Wolfgang-Pauli-Strasse 27, 8093 Z\"urich, Switzerland\\[1ex]
  \href{mailto:nbeisert@itp.phys.ethz.ch}
  {\texttt{nbeisert@itp.phys.ethz.ch}}}
\hypersetup{pdfauthor={Niklas Beisert}}
\hypersetup{pdfsubject={Manual for the LaTeX2e Package childdoc}}
\date{30 December 2018, \textsf{v2.0}}
\maketitle

\begin{abstract}\noindent
\textsf{childdoc} is a \LaTeXe{} package
that enables the direct compilation
of document sections included by |\include|
to individual files.
\end{abstract}

\begingroup
\parskip0ex
\tableofcontents
\endgroup

%%%%%%%%%%%%%%%%%%%%%%%%%%%%%%%%%%%%%%%%%%%%%%%%%%%%%%%%%%%%%%%%%%%%%%%%%%%%%%%%
%%%%%%%%%%%%%%%%%%%%%%%%%%%%%%%%%%%%%%%%%%%%%%%%%%%%%%%%%%%%%%%%%%%%%%%%%%%%%%%%
\section{Introduction}

\LaTeX{} provides a mechanism to structure a large document (such as a book)
into a main file and several child files (containing the chapters)
using the |\include| command.
This mechanism is beneficial for documents
which span hundreds of pages in order to
make the source file(s) more manageable.
Moreover, compilation can be restricted to
selected child files by means of the |\includeonly| command.
The latter feature can be used to reduce the compilation time while editing
(this was significantly more useful in the earlier days of \LaTeX{})
or to generate a smaller document which is easier to navigate.
Another application of |\includeonly| is to generate
documents consisting of selected parts of the complete document.

However, there are a few drawbacks of the plain |\include| mechanism:
\begin{itemize}
\item
The child files cannot be compiled on their own,
they can only be compiled via the main file.
A naive editing environment
(such as a text editor with an option
to have the current file processed by \LaTeX)
may require one to switch to the main file before compiling;
attempting to compile the child file produces errors.
\item
The main file must be modified (each time)
to adjust the |\includeonly| command
to the present needs. This easily leaves the main file in a messy state.
\item
The generated document will always carry the filename
of the main document. This is inconvenient if
several child files are to be compiled and
to be kept for distribution.
\end{itemize}

The present package provides a simple interface
to make child files individually compilable by \LaTeX{}.
Compiling a child file then has the same effect as compiling
the main file with an |\includeonly| command
to select the appropriate child.
Moreover the generated document will carry the name of the child
rather than the main file.
This resolves all three above issues.

This feature is meant to make the editing of books,
thesis documents and lecture notes somewhat more convenient.
However, the package can also be used efficiently for
composing a series of documents (such as exercise sheets)
which are typically distributed individually.
It then assists the author in generating the individual documents
(potentially in different versions)
as well as a document containing the collected series.
Another application is in developing style files
or other kinds of included material
where compilation of the style file could redirect
to a sample or test file.

%%%%%%%%%%%%%%%%%%%%%%%%%%%%%%%%%%%%%%%%%%%%%%%%%%%%%%%%%%%%%%%%%%%%%%%%%%%%%%%%
%%%%%%%%%%%%%%%%%%%%%%%%%%%%%%%%%%%%%%%%%%%%%%%%%%%%%%%%%%%%%%%%%%%%%%%%%%%%%%%%
\section{Usage}

First of all, the package \textsf{childdoc} is \emph{not} a standard
\LaTeXe{} |.sty| style file! Therefore it needs to be invoked in
a non-standard way.

%%%%%%%%%%%%%%%%%%%%%%%%%%%%%%%%%%%%%%%%%%%%%%%%%%%%%%%%%%%%%%%%%%%%%%%%%%%%%%%%
\subsection{Included Files}
\label{sec:include}

%%%%%%%%%%%%%%%%%%%%%%%%%%%%%%%%%%%%%%%%
\DescribeMacro{\childdocmain}
To use the package, add the commands
\begin{center}
\begin{tabular}{l}
|\input{childdoc.def}|\\
|\childdocmain{}|\\
\end{tabular}
\end{center}
at the very top of the main \LaTeX{} file,
in particular \emph{before} the |\documentclass| statement!
The argument of |\childdocmain| should be left empty
(but it must be present).

%%%%%%%%%%%%%%%%%%%%%%%%%%%%%%%%%%%%%%%%
\DescribeMacro{\childdocof}
Furthermore, add the commands
\begin{center}
\begin{tabular}{l}
|\input{childdoc.def}|\\
|\childdocof{|\textit{main}|}|\\
\end{tabular}
\end{center}
at the top of every child file \textit{child}
which is included by |\include{|\textit{child}|}|
from within the main file
(or at least for those files to be compiled individually).
The argument \textit{main} must be the filename of the main file.

There are a couple of
considerations in setting up the main and child documents:

%%%%%%%%%%%%%%%%%%%%%%%%%%%%%%%%%%%%%%%%
\paragraph{Restrictions.}

Please note the following restrictions:
\begin{itemize}
\item
|\childdocmain| must be called with one argument \textit{main}
to ensure compatibility with earlier version of the package.
It must either be empty (|\childdocmain{}|)
or precisely match the filename of the main file in which it is specified.
See \secref{sec:detection} for further information.
\item
The filename \textit{main} must be specified without the |.tex| extension.
\item
The filename \textit{main} is case sensitive
(even in case-insensitive file systems)
due to internal string comparison.
\item
The argument \textit{main} should be fully expanded, it cannot be a macro.
\item
Subdirectories and special characters should be avoided in filenames.
\item
The command |\childdocmain{|\textit{main}|}| must be followed by a whitespace.
It should not be followed immediately by another command
or by a comment mark `|%|'.
This is because the \TeX{} parser reads the token immediately following
the argument of |\childdocmain| and puts it
at the beginning of every child section;
however, a white\-space is ignored.
\end{itemize}

%%%%%%%%%%%%%%%%%%%%%%%%%%%%%%%%%%%%%%%%
\paragraph{Content of Main File.}

It is advisable to place all content in the child files included by |\include|.
Any output contained in the main file will appear in all child documents
unless suppressed manually;
it cannot be suppressed automatically by the |\includeonly| directive
and thus should normally be avoided.
A method to include some content in the main file
by means of conditional processing is described in \secref{sec:conditional}.

%%%%%%%%%%%%%%%%%%%%%%%%%%%%%%%%%%%%%%%%
\paragraph{Page Numbering.}

When only a part of the document is compiled,
the appropriate numbering of pages
(as well as other status parameters)
is determined from the |.aux| files.
The latter contain information from previous passes.
However this information needs to propagate through
all intermediate child documents.
Therefore the page numbering in child documents may well
be inconsistent until the complete document is compiled at least once.

A useful (if unconventional) way to always ensure a consistent
page numbering is to restart the numbering in each child document
and denote the pages by `\textit{child}|.|\textit{page}'
where \textit{child} represents the chapter/section number of the child file.
This can be achieved by the command
|\numberwithin{page}{|\textit{child}|}|
of the \textsf{amsmath} package
where \textit{child} can be |chapter| or |section|
depending on the chosen structuring.
Alternatively, one can modify the macro |\thepage| appropriately
and reset the counter |page| at the start of each child file.

%%%%%%%%%%%%%%%%%%%%%%%%%%%%%%%%%%%%%%%%%%%%%%%%%%%%%%%%%%%%%%%%%%%%%%%%%%%%%%%%
\subsection{Conditional Processing}
\label{sec:conditional}

The package provides a mechanism to compile different versions
of a document. To customise the versions further some conditional processing
can come in handy to distinguish which version is being compiled.
The package provides two macros to describe the compilation context:

%%%%%%%%%%%%%%%%%%%%%%%%%%%%%%%%%%%%%%%%
\DescribeMacro{\ifchilddoc}
The conditional |\ifchilddoc| distinguishes between the compilation of
child documents and the main document:
%
\begin{center}
|\ifchilddoc |\textit{child-code}| |[|\||else |\textit{main-code}]| \||fi|
\end{center}

%%%%%%%%%%%%%%%%%%%%%%%%%%%%%%%%%%%%%%%%
\DescribeMacro{\childdocname}
\DescribeMacro{\childdocjob}
The macro |\childdocname| contains the filename (without extension)
of the main or child file being processed.
Note that |\childdocjob| will always contain the name of the main file.

%%%%%%%%%%%%%%%%%%%%%%%%%%%%%%%%%%%%%%%%
\paragraph{Title Page.}

Conditional processing can be used to include a title or banner page
in the main document when proper precautions are taken.
Importantly, the code in the main file should ensure that the page counter
(as well as other status parameters which are stored in the |.aux| files)
takes the same value after the conditional processing.
Otherwise the page numbers may take divergent values
depending on which part is compiled.

For example, a title page could be declared by:
%
\begin{center}
\begin{tabular}{l}
|\ifchilddoc\||else|\\
|\addtocounter{page}{-1}|\\
\textit{code for title page}\\
|\newpage|\\
|\||fi|
\end{tabular}
\end{center}
%
A banner page for the child documents can be generated by:
%
\begin{center}
\begin{tabular}{l}
|\ifchilddoc|\\
|\addtocounter{page}{-1}|\\
\textit{code for banner page}\\
|\newpage|\\
|\||fi|
\end{tabular}
\end{center}
%
Here one could write a message such as:
\begin{center}
|This is the part \childdocname{} of \childdocjob{}.|
\end{center}

%%%%%%%%%%%%%%%%%%%%%%%%%%%%%%%%%%%%%%%%%%%%%%%%%%%%%%%%%%%%%%%%%%%%%%%%%%%%%%%%
\subsection{Flags}
\label{sec:flags}

The package makes it easy to generate different versions
of the main or child documents.
To this end compilation flags can be defined
and assigned different default values.
They will be particularly useful in conjunction
with the forwarding mechanism described in \secref{sec:forward}.

For example, it may be useful to have a flag |\version|
which can be set to |draft| or |final|.
The document source will contain some conditional code
depending on the value of |\version|.
Suppose further, the flag should default to |final| for the main file
and to |draft| for child files
which is a natural assignment for editing the document.
This is achieved by placing the following code
in the preamble of the main document
(below the |\childdocmain| directive):
%
\begin{center}
\begin{tabular}{l}
|\ifchilddoc|\\
|\providecommand{\version}{draft}|\\
|\||else|\\
|\providecommand{\version}{final}|\\
|\||fi|
\end{tabular}
\end{center}
%
The definition by |\providecommand| makes sure
that previous definitions are not overwritten.
Further statements |\providecommand{\version}{...}|
can thus be added before the above code to override it.

For the main file, one might add a line
(between |\childdocmain| and the above block)
%
\begin{center}
|%\ifchilddoc\||else\providecommand{\version}{draft}\||fi|
\end{center}
%
which can be uncommented to produce a draft version.
Likewise one can add a line to the very top of a child file
(above the |\childdocof{|\textit{main}|}| directive)
%
\begin{center}
|%\providecommand{\version}{final}|
\end{center}
%
which can be uncommented to produce the final version of this child document.

%%%%%%%%%%%%%%%%%%%%%%%%%%%%%%%%%%%%%%%%%%%%%%%%%%%%%%%%%%%%%%%%%%%%%%%%%%%%%%%%
\subsection{Forwarding}
\label{sec:forward}

Different versions of the main or child documents
using compilation flags as described in \secref{sec:flags}
can be (permanently) stored in different files
for convenient compilation, viewing and distribution.
To this end, the package defines a command
to pass on compilation to a different file:

%%%%%%%%%%%%%%%%%%%%%%%%%%%%%%%%%%%%%%%%
\DescribeMacro{\childdocforward}
The command |\childdocforward| redirects processing to
another source file:
%
\begin{center}
\begin{tabular}{l}
|\input{childdoc.def}|\\
|\childdocforward[|\textit{main}|]{|\textit{dest}|}|\\
\end{tabular}
\end{center}
%
The argument \textit{dest} is the destination file
(without extension).
It should be the main file or one of the child files.
Note that further \textsf{childdoc} directives
such as |\childdocof| and |\childdocforward|
in the indicated file will be processed in this form.
The optional argument \textit{main}
passes on directly to the main file \textit{main}
while pretending to compile the child \textit{dest}.
This form behaves as if \textit{dest}
issues |\childdocof{|\textit{main}|}| right away,
and no further \textsf{childdoc} directives will be processed.

%%%%%%%%%%%%%%%%%%%%%%%%%%%%%%%%%%%%%%%%
\DescribeMacro{\...prefix}
In the alternative form |\childdocforwardprefix|,
%
\begin{center}
\begin{tabular}{l}
|\input{childdoc.def}|\\
|\childdocforwardprefix[|\textit{main}|]{|\textit{prefix}|}{|\textit{dest}|}|
\end{tabular}
\end{center}
%
the destination file is determined by a pattern
depending on the current file:
To make this work, the current file must be called
`{\textit{prefix}\hspace{0.2em}\textit{suffix}}'
with \textit{prefix} matching precisely the argument.
Processing is then passed on to the file
`{\textit{dest}\hspace{0.2em}\textit{suffix}}'.
Surely, the same effect is achieved by
directly specifying the
argument `{\textit{dest}\hspace{0.2em}\textit{suffix}}'
in the first form.
However, that requires to set up a different file
for each child. With the alternative form of the command
all these files can have exactly the same content
which simplifies setting them up and maintaining them.

For example, the following file |draft.tex|
with a compilation flag |\version| as described in \secref{sec:flags}
compiles the main document as a draft:
%
\begin{center}
\begin{tabular}{l}
|\def\version{draft}|\\
|\input{childdoc.def}|\\
|\childdocforward{|\textit{main}|}|
\end{tabular}
\end{center}
%
Likewise, the following files |final|\textit{nn}|.tex|
compile the final version of the child document
|child|\textit{nn}|.tex|:
%
\begin{center}
\begin{tabular}{l}
|\def\version{final}|\\
|\input{childdoc.def}|\\
|\childdocforwardprefix{final}{child}|
\end{tabular}
\end{center}
%

Note that when several versions of a main file and/or of each child file
are to be generated, it may be convenient to set up a |Makefile| or
shell script to automatise the process.

%%%%%%%%%%%%%%%%%%%%%%%%%%%%%%%%%%%%%%%%%%%%%%%%%%%%%%%%%%%%%%%%%%%%%%%%%%%%%%%%
\subsection{Command Line Processing}
\label{sec:commandline}

The effect of redirection files can also be achieved by invoking
the \LaTeX{} compiler with a more elaborate command line.
Most conveniently this should be done as part
of a shell script or a |Makefile|.

When using \textsf{childdoc} in the main file, the following
command lines effectively perform a redirection
(note that depending on the shell being used,
backslashes may have to be doubled: `|\|' $\to$ `|\\|'):
%
\begin{center}
|... -jobname "|\textit{target}|" |\\|"|[\textit{flags}]%
|\input{childdoc.def}\childdocforward[|\textit{main}|]{|\textit{dest}|}"|
\end{center}
%
Here \textit{target} is the name of the output file,
\textit{main} is the name of the main file
and \textit{dest} is the name of the main or child file to be processed
(all filenames without extensions).
The optional argument \textit{main} can be omitted
if \textit{main} matches \textit{dest}.
Optionally, compilation \textit{flags} can be defined via |\def| commands.
This command line makes the \TeX{} engine believe
it is compiling the file \textit{target}
whose content is specified as the latter parameter.
The provided code then forwards the processing to
\textit{main} or \textit{dest} as described in \secref{sec:forward}.

%%%%%%%%%%%%%%%%%%%%%%%%%%%%%%%%%%%%%%%%%%%%%%%%%%%%%%%%%%%%%%%%%%%%%%%%%%%%%%%%
\subsection{Include by Input}
\label{sec:input}

Including child documents by |\include| has some restrictions by design.
Most notably, the content of a child document always occupies
its own set of pages; pages cannot be shared between child documents.
Usually, this behaviour makes perfect sense
because each child document contain an essential part of the document.
However, in some situations it may be desirable to compose
a document from a collection of parts
without having mandatory page breaks between then.
For this case, the package
provides a mechanism to include parts
by |\input| which can also be processed individually.
However, by construction this mechanism
requires manual handling of the content to be output.

%%%%%%%%%%%%%%%%%%%%%%%%%%%%%%%%%%%%%%%%
\DescribeMacro{\ifchilddocmanual}
The main file should be prepared as usual, see \secref{sec:include}.
However, the document body must make a distinction
between processing of an individual part and of the main document, e.g.:
%
\begin{center}
\begin{tabular}{l}
|\ifchilddocmanual|\\
|\input{\childdocname}|\\
|\||else|\\
\textit{document body with }|\input{|\textit{part}|}|\\
|\||fi|
\end{tabular}
\end{center}
%
The conditional |\ifchilddocmanual| is true whenever
a part to be included by |\input| is being compiled,
and the name of the part is stored in |\childdocname|.

%%%%%%%%%%%%%%%%%%%%%%%%%%%%%%%%%%%%%%%%
\DescribeMacro{\childdocby}
Each part to be included by |\input| should start with:
%
\begin{center}
\begin{tabular}{l}
|\input{childdoc.def}|\\
|\childdocby{|\textit{main}|}|\\
\end{tabular}
\end{center}
%
The directive |\childdocby| is similar to |\childdocof|
described in \secref{sec:include},
but the subsequent selection of content must be done manually.
To that end, both |\ifchilddoc| and |\ifchilddocmanual|
will be true upon processing of a part,
and the name of the part is stored in |\childdocname|.
Note that |\jobname| will be set to the filename of the current part
so that each part receives an individual |.aux| file
that does not interfere with the |.aux| file(s) of the main document.
This behaviour can be altered by the alternative form
|\childdocby[*]{|\textit{main}|}| (with a non-empty optional argument)
which uses the |.aux| file of the main document
by setting |\jobname| to \textit{main}.

%%%%%%%%%%%%%%%%%%%%%%%%%%%%%%%%%%%%%%%%%%%%%%%%%%%%%%%%%%%%%%%%%%%%%%%%%%%%%%%%
\subsection{Driver Development}
\label{sec:driver}

The \textsf{childdoc} mechanism can also be use for the development
of definition files such as \LaTeX{} styles or classes.
This case differs from the above setup with multiple parts
included by |\include| in that no |\includeonly| should be invoked.
This can be achieved by starting the include file
(before |\ProvidesPackage|) with:
%
\begin{center}
\begin{tabular}{l}
|\input{childdoc.def}|\\
|\childdocforward{|\textit{main}|}|\\
\end{tabular}
\end{center}
%
or alternatively with:
%
\begin{center}
\begin{tabular}{l}
|\input{childdoc.def}|\\
|\childdocby{|\textit{main}|}|\\
\end{tabular}
\end{center}
%
Both forms have slightly different effects as described above.
The main file is prepared as usual, see \secref{sec:include}.

%%%%%%%%%%%%%%%%%%%%%%%%%%%%%%%%%%%%%%%%%%%%%%%%%%%%%%%%%%%%%%%%%%%%%%%%%%%%%%%%
\subsection{Legacy Detection}
\label{sec:detection}

The directive |\childdocmain| in the main file can detect
whether the complete document or merely a child is to be compiled
even without using the directive |\childdocof|.
This method is deprecated because it is less robust
and there is no compelling reason to use it;
it is merely provided for backward compatibility
and it may be removed in future versions.

If the detection mechanism is to be used,
it is mandatory to correctly specify
the filename of the main file as the argument of |\childdocmain|:
%
\begin{center}
\begin{tabular}{l}
|\input{childdoc.def}|\\
|\childdocmain{|\textit{main}|}|\\
\end{tabular}
\end{center}
%
If |\jobname| does not match the argument \textit{main} of |\childdocmain|,
it is assumed that |\jobname| points to the child file to be compiled.
When using |\childdocmain| with the main file specified as argument,
it suffices to start a child file
with just |\input{|\textit{main}|}|
without loading of the package and using |\childdocof|.
If instead all processing is done
with the appropriate \textsf{childdoc} directives,
the argument of \textit{main} of |\childdocmain| can be empty.

An alternative version of the command line processing described
in \secref{sec:commandline} using the detection mechanism reads:
%
\begin{center}
|... -jobname "|\textit{target}|" "|[\textit{flags}]%
[|\def\jobname{|\textit{dest}|}|]|\input{|\textit{main}|}"|
\end{center}

%%%%%%%%%%%%%%%%%%%%%%%%%%%%%%%%%%%%%%%%%%%%%%%%%%%%%%%%%%%%%%%%%%%%%%%%%%%%%%%%
\subsection{Manual Code}
\label{sec:manual}

In case one cannot be certain whether the definitions file |childdoc.def|
is installed on the target \TeX{} distribution
and one prefers not to ship it,
it is conceivable to paste a few relevant commands into the sources.

To that end, drop all statements |\input{childdoc.def}|
and perform the replacements as outlined below.
Instead of |\childdocmain{|\textit{main}|}| add the following code
to the top of the main file:
%
\begin{center}
\begin{tabular}{l}
|\||ifdefined\childdocname\endinput\||fi\newif\ifchilddoc|\\
|\edef\childdocname{\scantokens\expandafter{\jobname\noexpand}}|\\
|\def\childdocmain{|\textit{main}|}\||ifx\childdocmain\childdocname\||else|\\
|\childdoctrue\includeonly{\childdocname}\let\jobname\childdocmain\||fi|\\
\end{tabular}
\end{center}
%
Instead of |\childdocof{|\textit{main}|}| just include the main file
at the top of each child file:
%
\begin{center}
|\input{|\textit{main}|}|
\end{center}
%
A simple redirection |\childdocforward{|\textit{dest}|}| is achieved by:
%
\begin{center}
|\def\jobname{|\textit{dest}|}\input{\jobname}|
\end{center}
%
The redirection with prefix
|\childdocforwardprefix[|\textit{prefix}|]{|\textit{dest}|}|
is accomplished by:
%
\begin{center}
\begin{tabular}{l}
|{\edef\jobname{\scantokens\expandafter{\jobname\noexpand}}|\\
|\def\redirectjob |\textit{prefix}|#1~~~{\gdef\jobname{|\textit{dest}|#1}}|\\
|\expandafter\redirectjob\jobname~~~}\input{\jobname}|
\end{tabular}
\end{center}

In an alternative approach,
child documents can be compiled by a specific command line
without additional code or specific definitions:
%
\begin{center}
|... -jobname "|\textit{target}|" "|[\textit{flags}]%
|\includeonly{|\textit{dest}|}\input{|\textit{main}|}"|
\end{center}
%

%%%%%%%%%%%%%%%%%%%%%%%%%%%%%%%%%%%%%%%%%%%%%%%%%%%%%%%%%%%%%%%%%%%%%%%%%%%%%%%%
%%%%%%%%%%%%%%%%%%%%%%%%%%%%%%%%%%%%%%%%%%%%%%%%%%%%%%%%%%%%%%%%%%%%%%%%%%%%%%%%
\section{Information}

%%%%%%%%%%%%%%%%%%%%%%%%%%%%%%%%%%%%%%%%%%%%%%%%%%%%%%%%%%%%%%%%%%%%%%%%%%%%%%%%
\subsection{Copyright}

Copyright \copyright{} 2017--2018 Niklas Beisert

This work may be distributed and/or modified under the
conditions of the \LaTeX{} Project Public License, either version 1.3
of this license or (at your option) any later version.
The latest version of this license is in
  \url{http://www.latex-project.org/lppl.txt}
and version 1.3 or later is part of all distributions of \LaTeX{}
version 2005/12/01 or later.

This work has the LPPL maintenance status `maintained'.

The Current Maintainer of this work is Niklas Beisert.

This work consists of the files |README.txt|, |childdoc.ins| and |childdoc.dtx|
as well as the derived files |childdoc.def|, |cdocsamp.tex|
with |cdocsch1.tex|, |cdocsch2.tex|, |cdocspt3.tex|, |cdocspt4.tex|,
|cdocsdrf.tex|, |cdocsfn1.tex|, |cdocsfn2.tex|
as well as |childdoc.pdf|.

%%%%%%%%%%%%%%%%%%%%%%%%%%%%%%%%%%%%%%%%%%%%%%%%%%%%%%%%%%%%%%%%%%%%%%%%%%%%%%%%
\subsection{Files and Installation}

The package consists of the files:
%
\begin{center}
\begin{tabular}{ll}
    |README.txt|   & readme file \\
    |childdoc.ins| & installation file \\
    |childdoc.dtx| & source file \\
    |childdoc.def| & definition file \\
    |cdocsamp.tex| & sample main file \\
    |cdocsch1.tex| & sample include file \\
    |cdocsch2.tex| & sample include file \\
    |cdocspt3.tex| & sample part file \\
    |cdocspt4.tex| & sample part file \\
    |cdocsdrf.tex| & sample redirection file \\
    |cdocsfn1.tex| & sample redirection file \\
    |cdocsfn2.tex| & sample redirection file \\
    |childdoc.pdf| & manual
\end{tabular}
\end{center}
%
The distribution consists of the files
|README.txt|, |childdoc.ins| and |childdoc.dtx|.
%
\begin{itemize}
\item
Run (pdf)\LaTeX{} on |childdoc.dtx|
to compile the manual |childdoc.pdf| (this file).
\item
Run \LaTeX{} on |childdoc.ins| to create the definitions file |childdoc.def|
and the sample |cdocsamp.tex| with include files
|cdocsch1.tex|, |cdocsch2.tex|, |cdocspt3.tex|, |cdocspt4.tex|,
|cdocsdrf.tex|, |cdocsfn1.tex|, |cdocsfn2.tex|.
Then copy the file |childdoc.def| to an appropriate directory of your \LaTeX{}
distribution, e.g.\ \textit{texmf-root}|/tex/latex/childdoc|.
\end{itemize}

%%%%%%%%%%%%%%%%%%%%%%%%%%%%%%%%%%%%%%%%%%%%%%%%%%%%%%%%%%%%%%%%%%%%%%%%%%%%%%%%
\subsection{Related CTAN Packages}

There are several other packages which offer a similar functionality:
%
\begin{itemize}
\item
The packages
\href{http://ctan.org/pkg/docmute}{\textsf{docmute}},
\href{http://ctan.org/pkg/includex}{\textsf{includex}} and
\href{http://ctan.org/pkg/standalone}{\textsf{standalone}}
provide commands to include only the document body of
a child file thus allowing both files to be compiled individually.
\item
The packages \href{http://ctan.org/pkg/subdocs}{\textsf{subdocs}}
and \href{http://ctan.org/pkg/subfiles}{\textsf{subfiles}}
provide structures in which the main and child documents can be
encapsulated and allowing them to be compiled individually.
The inclusion mechanism is different from the conventional |\include|.
\item
The package \href{http://ctan.org/pkg/combine}{\textsf{combine}}
is an elaborate solution to combine several documents into one.
\end{itemize}
%
See also the CTAN topic \href{http://ctan.org/topic/subdocs}{\textsf{subdocs}}
for further related packages.
The present package differs from the above solutions in that
a document structure constructed with the conventional |\include| mechanism
just needs two extra commands at the top of every file
such that all constituent files can be compiled individually.

%%%%%%%%%%%%%%%%%%%%%%%%%%%%%%%%%%%%%%%%%%%%%%%%%%%%%%%%%%%%%%%%%%%%%%%%%%%%%%%%
%\subsection{Feature Suggestions}
%
%The following is a list of features which may be useful for future
%versions of this package:
%%
%\begin{itemize}
%\item
%\ldots
%\end{itemize}

%%%%%%%%%%%%%%%%%%%%%%%%%%%%%%%%%%%%%%%%%%%%%%%%%%%%%%%%%%%%%%%%%%%%%%%%%%%%%%%%
\subsection{Revision History}

%%%%%%%%%%%%%%%%%%%%%%%%%%%%%%%%%%%%%%%%
\paragraph{v2.0:} 2018/12/30

\begin{itemize}
\item
immediate forward processing
\item
added |\childdocby| mechanism
\item
manual restructured
\end{itemize}

%%%%%%%%%%%%%%%%%%%%%%%%%%%%%%%%%%%%%%%%
\paragraph{v1.6:} 2018/01/17

\begin{itemize}
\item
application for development of include files
\item
corrections to manual
\end{itemize}

%%%%%%%%%%%%%%%%%%%%%%%%%%%%%%%%%%%%%%%%
\paragraph{v1.5:} 2017/05/21

\begin{itemize}
\item
more complete structuring introduced
\item
|\childdocof| introduced
\item
|\childdoc| renamed to |\childdocmain|
\item
|\childredirect| renamed to |\childdocforward| and |\childdocforwardprefix|
and functionality expanded
\end{itemize}

%%%%%%%%%%%%%%%%%%%%%%%%%%%%%%%%%%%%%%%%
\paragraph{v1.0:} 2017/04/27

\begin{itemize}
\item
manual and install package
\item
first version published on CTAN
\end{itemize}

%%%%%%%%%%%%%%%%%%%%%%%%%%%%%%%%%%%%%%%%
\paragraph{v0.6:} 2017/04/26

\begin{itemize}
\item
redirection mechanism added
\end{itemize}

%%%%%%%%%%%%%%%%%%%%%%%%%%%%%%%%%%%%%%%%
\paragraph{v0.5:} 2017/04/26

\begin{itemize}
\item
functionality in definition file
\end{itemize}


%%%%%%%%%%%%%%%%%%%%%%%%%%%%%%%%%%%%%%%%%%%%%%%%%%%%%%%%%%%%%%%%%%%%%%%%%%%%%%%%
%%%%%%%%%%%%%%%%%%%%%%%%%%%%%%%%%%%%%%%%%%%%%%%%%%%%%%%%%%%%%%%%%%%%%%%%%%%%%%%%
%%%%%%%%%%%%%%%%%%%%%%%%%%%%%%%%%%%%%%%%%%%%%%%%%%%%%%%%%%%%%%%%%%%%%%%%%%%%%%%%
\appendix

\settowidth\MacroIndent{\rmfamily\scriptsize 000\ }

 \DocInput{childdoc.dtx}

\end{document}
%</driver>
% \fi
%
% %%%%%%%%%%%%%%%%%%%%%%%%%%%%%%%%%%%%%%%%%%%%%%%%%%%%%%%%%%%%%%%%%%%%%%%%%%%%%%
% %%%%%%%%%%%%%%%%%%%%%%%%%%%%%%%%%%%%%%%%%%%%%%%%%%%%%%%%%%%%%%%%%%%%%%%%%%%%%%
% \section{Sample}
%\iffalse
%<*samplemain>
%\fi
%
% The following presents a sample document
% with two chapters, two parts, a title page,
% a compile flag as well as three forwarding files to set the flag.
% It consists of eight |.tex| files:
% \begin{center}
% \begin{tabular}{ll}
% |cdocsamp.tex|&main file\\
% |cdocsch1.tex|&include file for chapter 1\\
% |cdocsch2.tex|&include file for chapter 2\\
% |cdocspt3.tex|&include file for part 3\\
% |cdocspt4.tex|&include file for part 4\\
% |cdocsdrf.tex|&forwarding file for main file in draft mode\\
% |cdocsfi1.tex|&forwarding file for final version of chapter 1\\
% |cdocsfi2.tex|&forwarding file for final version of chapter 2\\
% \end{tabular}
% \end{center}
% Each of the eight files can be compiled directly by the \LaTeX{} compiler.
%
% %%%%%%%%%%%%%%%%%%%%%%%%%%%%%%%%%%%%%%
% \paragraph{Main File.}
%
% The main file is called |cdocsamp.tex|.
%
% Load the \textsf{childdoc} definitions and
% declare the filename for the main document:
%    \begin{macrocode}
\input{childdoc.def}
\childdocmain{}
%    \end{macrocode}

% Optional override for |\version| flag:
%    \begin{macrocode}
%%\ifchilddoc\else\providecommand{\version}{draft}\fi
%    \end{macrocode}

% Define the default values for the |\version| flag
% (|final| for the main file and |draft| for childs):
%    \begin{macrocode}
\ifchilddoc
\providecommand{\version}{draft}
\else
\providecommand{\version}{final}
\fi
%    \end{macrocode}

% Load the standard document class:
%    \begin{macrocode}
\documentclass[12pt]{article}
%    \end{macrocode}

% Start the document body:
%    \begin{macrocode}
\begin{document}
%    \end{macrocode}

% Declare a title page.
% Print title, part of document being processed and version flag:
%    \begin{macrocode}
\addtocounter{page}{-1}
\begin{center}
{\LARGE\bfseries{}childdoc example\par}
\vspace{1cm}
\ifchilddoc
\ifchilddocmanual part\else chapter\fi:
`\childdocname' of `\childdocjob'\par
\else
main document: `\childdocjob'\par
\fi
version: \version\par
\end{center}
\newpage
%    \end{macrocode}

% Manually include selected file,
% otherwise process as usual:
%    \begin{macrocode}
\ifchilddocmanual
\section*{part `\childdocname'}
\input{\childdocname}
\else
%    \end{macrocode}

% Include the two chapters:
%    \begin{macrocode}
\include{cdocsch1}
\include{cdocsch2}
%    \end{macrocode}

% Include the two parts unless only chapters should be displayed:
%    \begin{macrocode}
\ifchilddoc\else
\section{part three}
\input{cdocspt3}
\section{part four}
\input{cdocspt4}
\fi
%    \end{macrocode}

% Process as usual until here:
%    \begin{macrocode}
\fi
%    \end{macrocode}

% End of document body:
%    \begin{macrocode}
\end{document}
%    \end{macrocode}
%\iffalse
%</samplemain>
%\fi
%
% %%%%%%%%%%%%%%%%%%%%%%%%%%%%%%%%%%%%%%
% \paragraph{Chapter Include Files.}
%
% The include files are called |cdocsch1.tex| and |cdocsch2.tex|.
%
%\iffalse
%<*samplechap1|samplechap2>
%\fi

% Optional override for |\version| flag:
%    \begin{macrocode}
%%\providecommand{\version}{final}
%    \end{macrocode}

% Include the main document:
%    \begin{macrocode}
\input{childdoc.def}
\childdocof{cdocsamp}
%    \end{macrocode}

%\iffalse
%</samplechap1|samplechap2>
%\fi
%
%\iffalse
%<*samplechap1>
%\fi
% Some text for chapter 1:
%    \begin{macrocode}
\section{one}
some text in chapter one
%    \end{macrocode}

%\iffalse
%</samplechap1>
%\fi
% Some text for chapter 2:
%\iffalse
%<*samplechap2>
%\fi
%    \begin{macrocode}
\section{two}
more text in chapter two
%    \end{macrocode}

%\iffalse
%</samplechap2>
%\fi
%
% %%%%%%%%%%%%%%%%%%%%%%%%%%%%%%%%%%%%%%
% \paragraph{Part Include Files.}
%
% The include files are called |cdocspt3.tex| and |cdocspt4.tex|.
%
%\iffalse
%<*samplepart3|samplepart4>
%\fi

% Optional override for |\version| flag:
%    \begin{macrocode}
%%\providecommand{\version}{final}
%    \end{macrocode}

% Include the main document:
%    \begin{macrocode}
\input{childdoc.def}
\childdocby{cdocsamp}
%    \end{macrocode}

%\iffalse
%</samplepart3|samplepart4>
%\fi
%
%\iffalse
%<*samplepart3>
%\fi
% Some text for part 3:
%    \begin{macrocode}
some text in part three
%    \end{macrocode}

%\iffalse
%</samplepart3>
%\fi
% Some text for part 4:
%\iffalse
%<*samplepart4>
%\fi
%    \begin{macrocode}
more text in part four
%    \end{macrocode}

%\iffalse
%</samplepart4>
%\fi
%
% %%%%%%%%%%%%%%%%%%%%%%%%%%%%%%%%%%%%%%
% \paragraph{Forwarding for a Complete Draft.}
%
% The following forwarding file |cdocsdrf.tex|
% compiles the main document in draft mode:
%\iffalse
%<*sampledraft>
%\fi
%    \begin{macrocode}
\def\version{draft}
\input{childdoc.def}
\childdocforward{cdocsamp}
%    \end{macrocode}

%\iffalse
%</sampledraft>
%\fi
%
% %%%%%%%%%%%%%%%%%%%%%%%%%%%%%%%%%%%%%%
% \paragraph{Forwarding for Final Version of the Chapters.}
%
% The following forwarding files |cdocsfn1.tex| and |cdocsfn2.tex|
% (with identical content)
% compile the final versions of the child documents
% |cdocsch1.tex| and |cdocsch2.tex|, respectively:
%\iffalse
%<*samplefinal>
%\fi
%    \begin{macrocode}
\def\version{final}
\input{childdoc.def}
\childdocforwardprefix[cdocsamp]{cdocsfn}{cdocsch}
%    \end{macrocode}

%\iffalse
%</samplefinal>
%\fi
%
% %%%%%%%%%%%%%%%%%%%%%%%%%%%%%%%%%%%%%%
% \paragraph{Command Line Processing.}
%
% The following three command lines generate the output files
% |cdocscld|, |cdocscl1| and |cdocscl2|
% which should be identical to
% |cdocsdrf|, |cdocsch1| and |cdocsfn2|, respectively:
% \begin{center}
% \begin{tabular}{l}
% |latex -jobname cdocscld \|\\
% |  "\def\version{draft}\input{childdoc.def}\childdocforward{cdocsamp}"|\\
% |latex -jobname cdocscl1 \|\\
% |  "\input{childdoc.def}\childdocforward[cdocsamp]{cdocsch1}"|\\
% |latex -jobname cdocscl2 \|\\
% |  "\def\version{final}\input{childdoc.def}\childdocforward{cdocsch2}"|
% \end{tabular}
% \end{center}
% Note that the trailing backslash on each first line
% merely continues the input to the second line
% (for convenient cut ant paste).
% Furthermore, the command |latex| can be replaced by any
% of its alternative versions such as |pdflatex|.
%
% %%%%%%%%%%%%%%%%%%%%%%%%%%%%%%%%%%%%%%%%%%%%%%%%%%%%%%%%%%%%%%%%%%%%%%%%%%%%%%
% %%%%%%%%%%%%%%%%%%%%%%%%%%%%%%%%%%%%%%%%%%%%%%%%%%%%%%%%%%%%%%%%%%%%%%%%%%%%%%
% \section{Implementation}
%\iffalse
%<*package>
%\fi
%
% This section describes the definitions file |childdoc.def|.

% The definitions cannot be loaded using |\usepackage| or |\RequirePackage|
% which has a mechanism to prevent loading a style file more than once.
% When loading the definitions by means of |\input|
% multiple instances have to be prevented manually:
%\iffalse
%This code needs to be before the `\ProvidesFile' directive
%which is defined at the beginning of this file.
%Therefore it is also placed there and commented out here.
%</package>
%<*discard>
%\fi
%    \begin{macrocode}
\ifdefined\childdocmain\endinput\fi
%    \end{macrocode}
%\iffalse
%</discard>
%<*package>
%\fi
%
% \macro{\ifchilddoc}
% \macro{\ifchilddocmanual}
% The conditional |\ifchilddoc| tells whether a
% child (true) or main (false) document is being compiled.
% The conditional |\ifchilddocmanual| tells whether
% the |\includeonly| mechanism is used (false) or
% the selection of child files must be performed manually (true).
% The definitions initialise to false:
%    \begin{macrocode}
\newif\ifchilddoc
\newif\ifchilddocmanual
%    \end{macrocode}

% \macro{\childdocname}
% \macro{\childdocjob}
% The macro |\childdocname| stores the name of the main document
% to be compiled. The macro |\childdocjob| stores the name of
% the document on which the \LaTeX{} compiler was originally invoked.
% The content of |\jobname| cannot be compared
% to filenames specified in the source due to different catcodes.
% The following code rescans |\jobname|, stores the result
% in |\childdocname| and saves a copy in |\childdocjob|:
%    \begin{macrocode}
\edef\childdocname{\scantokens\expandafter{\jobname\noexpand}}
\let\childdocjob\childdocname
%    \end{macrocode}

% \macro{\childdocdisable}
% The macro |\childdocdisable| prevents the main file
% from being processed more than once.
% At this stage, the main document command |\childdocmain|
% is assumed to be called once again where it should do nothing.
% Any subsequent call to it should prevent
% a secondary processing of the main document
% It overwrites the forwarding commands
% |\childdocof| and |\childdocforward|
% with empty macros to prevent further inclusions of the main document:
%    \begin{macrocode}
\newcommand{\childdocdisable}
{
  \renewcommand{\childdocmain}[1]{\renewcommand{\childdocmain}[1]{\endinput}}
  \renewcommand{\childdocof}[1]{}
  \renewcommand{\childdocby}[2][]{}
  \renewcommand{\childdocforward}[2][]{}
  \renewcommand{\childdocdisable}{}
}
%    \end{macrocode}

% \macro{\childdocmain}
% The macro |\childdocmain| is to be called at the top of the main file
% with nothing or the main filename (without extension) as argument.
% First, it breaks loops.
% If the argument is not empty and does not match |\childdocname|
% (which is set by the first inclusion of |childdoc.def|),
% |\ifchilddoc| is set to true, |\includeonly| is applied to the child file
% and |\jobname| is set to the main file
% (for proper handling of |.aux| files):
%    \begin{macrocode}
\newcommand{\childdocmain}[1]
{
  \childdocdisable\childdocmain{}
  \if?#1?\else
    \begingroup
      \def\childdoctmp{#1}
      \ifx\childdoctmp\childdocname
        \def\childdoctmp{}
      \else
        \def\childdoctmp
        {
          \childdoctrue
          \includeonly{\childdocname}
          \def\childdocjob{#1}
          \def\jobname{#1}
        }
      \fi
      \expandafter
    \endgroup
    \childdoctmp
  \fi
}
%    \end{macrocode}

% \macro{\childdocof}
% The command |\childdocof| redirects
% compilation to the main file |#1|.
%    \begin{macrocode}
\newcommand{\childdocof}[1]
{
  \childdocdisable
  \childdoctrue
  \includeonly{\childdocname}
  \def\jobname{#1}
  \def\childdocjob{#1}
  \input{#1}
}
%    \end{macrocode}

% \macro{\childdocby}
% The command |\childdocby| ....
%    \begin{macrocode}
\newcommand{\childdocby}[2][]
{
  \childdocdisable
  \childdoctrue
  \childdocmanualtrue
  \if?#1?\else
    \def\jobname{#2}
  \fi
  \def\childdocjob{#2}
  \input{#2}
  \endinput
}
%    \end{macrocode}

% \macro{\childdocforward}
% The command |\childdocforward| redirects
% compilation to the main file or
% (if the optional argument is given) a child file.
% Parameters are set as if the main file
% or a child file starting with |\childdocof| was compiled.
% Then compilation is handed over to the main file:
%    \begin{macrocode}
\newcommand{\childdocforward}[2][]
{
  \begingroup
    \if?#1?
      \def\childdoctmp
      {
        \def\childdocname{#2}
        \def\childdocjob{#2}
        \def\jobname{#2}
        \input{#2}
        \endinput
      }
    \else
      \def\childdoctmp
      {
        \childdocdisable
        \def\childdocname{#2}
        \childdoctrue
        \includeonly{#2}
        \def\childdocjob{#1}
        \def\jobname{#1}
        \input{#1}
        \endinput
      }
    \fi
    \expandafter
  \endgroup
  \childdoctmp
}
%    \end{macrocode}

% \macro{\childdocforwardprefix}
% The command |\childdocforwardprefix| redirects
% compilation to the main or a child file by means of a pattern.
% The prefix |#1| in the current filename is replaced by |#2|
% and the suffix of the current filename is kept
% (it is assumed that the filename does not contain the substring `|~~~|'
% which is used as a delimiter).
% Compilation is handed over to the new file by |\childdocforward|:
%    \begin{macrocode}
\newcommand{\childdocforwardprefix}[3][]
{
  \begingroup
    \def\childdocextract #2##1~~~{\def\childdoctmp{\childdocforward[#1]{#3##1}}}
    \expandafter\childdocextract\childdocname~~~
    \expandafter
  \endgroup
  \childdoctmp
}
%    \end{macrocode}

% \macro{\childdoc}
% The deprecated macro |\childdoc| is a legacy version of |\childdocmain|:
%    \begin{macrocode}
\newcommand{\childdoc}{\childdocmain}
%    \end{macrocode}

% \macro{\childdocredirect}
% The deprecated macro |\childdocredirect| is a legacy version
% of |\childdocforward| and |\childdocforwardprefix|:
%    \begin{macrocode}
\newcommand{\childdocredirect}[2][]
{
  \begingroup
    \if?#1?
      \def\childdoctmp{\childdocforward{#2}}
    \else
      \def\childdoctmp{\childdocforwardprefix{#1}{#2}}
    \fi
    \expandafter
  \endgroup
  \childdoctmp
}
%    \end{macrocode}

%\iffalse
%</package>
%\fi
%
\endinput
|\\
|\childdocby{|\textit{main}|}|\\
\end{tabular}
\end{center}
%
The directive |\childdocby| is similar to |\childdocof|
described in \secref{sec:include},
but the subsequent selection of content must be done manually.
To that end, both |\ifchilddoc| and |\ifchilddocmanual|
will be true upon processing of a part,
and the name of the part is stored in |\childdocname|.
Note that |\jobname| will be set to the filename of the current part
so that each part receives an individual |.aux| file
that does not interfere with the |.aux| file(s) of the main document.
This behaviour can be altered by the alternative form
|\childdocby[*]{|\textit{main}|}| (with a non-empty optional argument)
which uses the |.aux| file of the main document
by setting |\jobname| to \textit{main}.

%%%%%%%%%%%%%%%%%%%%%%%%%%%%%%%%%%%%%%%%%%%%%%%%%%%%%%%%%%%%%%%%%%%%%%%%%%%%%%%%
\subsection{Driver Development}
\label{sec:driver}

The \textsf{childdoc} mechanism can also be use for the development
of definition files such as \LaTeX{} styles or classes.
This case differs from the above setup with multiple parts
included by |\include| in that no |\includeonly| should be invoked.
This can be achieved by starting the include file
(before |\ProvidesPackage|) with:
%
\begin{center}
\begin{tabular}{l}
|% \iffalse
%
% childdoc.dtx Copyright (C) 2017-2018 Niklas Beisert
%
% This work may be distributed and/or modified under the
% conditions of the LaTeX Project Public License, either version 1.3
% of this license or (at your option) any later version.
% The latest version of this license is in
%   http://www.latex-project.org/lppl.txt
% and version 1.3 or later is part of all distributions of LaTeX
% version 2005/12/01 or later.
%
% This work has the LPPL maintenance status `maintained'.
%
% The Current Maintainer of this work is Niklas Beisert.
%
% This work consists of the files childdoc.dtx and childdoc.ins
% and the derived files childdoc.def and cdocsamp.tex with
% cdocsch1.tex, cdocsch2.tex, cdocsdrf.tex, cdocsfn1.tex, cdocsfn2.tex.
%
%<package>\ifdefined\childdocmain\endinput\fi
%<package>\ProvidesFile{childdoc.def}[2018/12/30 v2.0 child document driver]
%<samplemain>\ProvidesFile{cdocsamp.tex}[2018/12/30 v2.0 sample for childdoc]
%<*driver>
%\ProvidesFile{childdoc.drv}[2018/12/30 v2.0 childdoc reference manual file]
\PassOptionsToClass{10pt,a4paper}{article}
\documentclass{ltxdoc}

\usepackage[margin=35mm]{geometry}
\usepackage{hyperref}
\usepackage{hyperxmp}
\usepackage[usenames]{color}

\hypersetup{colorlinks=true}
\hypersetup{pdfstartview=FitH}
\hypersetup{pdfpagemode=UseNone}
\hypersetup{pdfsource={}}
\hypersetup{pdflang={en-UK}}
\hypersetup{pdfcopyright={Copyright 2017-2018 Niklas Beisert.
  This work may be distributed and/or modified under the
  conditions of the LaTeX Project Public License, either version 1.3
  of this license or (at your option) any later version.}}
\hypersetup{pdflicenseurl={http://www.latex-project.org/lppl.txt}}
\hypersetup{pdfcontactaddress={ETH Zurich, ITP, HIT K,
  Wolfgang-Pauli-Strasse 27}}
\hypersetup{pdfcontactpostcode={8093}}
\hypersetup{pdfcontactcity={Zurich}}
\hypersetup{pdfcontactcountry={Switzerland}}
\hypersetup{pdfcontactemail={nbeisert@itp.phys.ethz.ch}}
\hypersetup{pdfcontacturl={http://people.phys.ethz.ch/\xmptilde nbeisert/}}

\newcommand{\secref}[1]{\hyperref[#1]{section \ref*{#1}}}

\parskip1ex
\parindent0pt
\let\olditemize\itemize
\def\itemize{\olditemize\parskip0pt}

\begin{document}

\title{The \textsf{childdoc} Package}
\hypersetup{pdftitle={The childdoc Package}}
\author{Niklas Beisert\\[2ex]
  Institut f\"ur Theoretische Physik\\
  Eidgen\"ossische Technische Hochschule Z\"urich\\
  Wolfgang-Pauli-Strasse 27, 8093 Z\"urich, Switzerland\\[1ex]
  \href{mailto:nbeisert@itp.phys.ethz.ch}
  {\texttt{nbeisert@itp.phys.ethz.ch}}}
\hypersetup{pdfauthor={Niklas Beisert}}
\hypersetup{pdfsubject={Manual for the LaTeX2e Package childdoc}}
\date{30 December 2018, \textsf{v2.0}}
\maketitle

\begin{abstract}\noindent
\textsf{childdoc} is a \LaTeXe{} package
that enables the direct compilation
of document sections included by |\include|
to individual files.
\end{abstract}

\begingroup
\parskip0ex
\tableofcontents
\endgroup

%%%%%%%%%%%%%%%%%%%%%%%%%%%%%%%%%%%%%%%%%%%%%%%%%%%%%%%%%%%%%%%%%%%%%%%%%%%%%%%%
%%%%%%%%%%%%%%%%%%%%%%%%%%%%%%%%%%%%%%%%%%%%%%%%%%%%%%%%%%%%%%%%%%%%%%%%%%%%%%%%
\section{Introduction}

\LaTeX{} provides a mechanism to structure a large document (such as a book)
into a main file and several child files (containing the chapters)
using the |\include| command.
This mechanism is beneficial for documents
which span hundreds of pages in order to
make the source file(s) more manageable.
Moreover, compilation can be restricted to
selected child files by means of the |\includeonly| command.
The latter feature can be used to reduce the compilation time while editing
(this was significantly more useful in the earlier days of \LaTeX{})
or to generate a smaller document which is easier to navigate.
Another application of |\includeonly| is to generate
documents consisting of selected parts of the complete document.

However, there are a few drawbacks of the plain |\include| mechanism:
\begin{itemize}
\item
The child files cannot be compiled on their own,
they can only be compiled via the main file.
A naive editing environment
(such as a text editor with an option
to have the current file processed by \LaTeX)
may require one to switch to the main file before compiling;
attempting to compile the child file produces errors.
\item
The main file must be modified (each time)
to adjust the |\includeonly| command
to the present needs. This easily leaves the main file in a messy state.
\item
The generated document will always carry the filename
of the main document. This is inconvenient if
several child files are to be compiled and
to be kept for distribution.
\end{itemize}

The present package provides a simple interface
to make child files individually compilable by \LaTeX{}.
Compiling a child file then has the same effect as compiling
the main file with an |\includeonly| command
to select the appropriate child.
Moreover the generated document will carry the name of the child
rather than the main file.
This resolves all three above issues.

This feature is meant to make the editing of books,
thesis documents and lecture notes somewhat more convenient.
However, the package can also be used efficiently for
composing a series of documents (such as exercise sheets)
which are typically distributed individually.
It then assists the author in generating the individual documents
(potentially in different versions)
as well as a document containing the collected series.
Another application is in developing style files
or other kinds of included material
where compilation of the style file could redirect
to a sample or test file.

%%%%%%%%%%%%%%%%%%%%%%%%%%%%%%%%%%%%%%%%%%%%%%%%%%%%%%%%%%%%%%%%%%%%%%%%%%%%%%%%
%%%%%%%%%%%%%%%%%%%%%%%%%%%%%%%%%%%%%%%%%%%%%%%%%%%%%%%%%%%%%%%%%%%%%%%%%%%%%%%%
\section{Usage}

First of all, the package \textsf{childdoc} is \emph{not} a standard
\LaTeXe{} |.sty| style file! Therefore it needs to be invoked in
a non-standard way.

%%%%%%%%%%%%%%%%%%%%%%%%%%%%%%%%%%%%%%%%%%%%%%%%%%%%%%%%%%%%%%%%%%%%%%%%%%%%%%%%
\subsection{Included Files}
\label{sec:include}

%%%%%%%%%%%%%%%%%%%%%%%%%%%%%%%%%%%%%%%%
\DescribeMacro{\childdocmain}
To use the package, add the commands
\begin{center}
\begin{tabular}{l}
|\input{childdoc.def}|\\
|\childdocmain{}|\\
\end{tabular}
\end{center}
at the very top of the main \LaTeX{} file,
in particular \emph{before} the |\documentclass| statement!
The argument of |\childdocmain| should be left empty
(but it must be present).

%%%%%%%%%%%%%%%%%%%%%%%%%%%%%%%%%%%%%%%%
\DescribeMacro{\childdocof}
Furthermore, add the commands
\begin{center}
\begin{tabular}{l}
|\input{childdoc.def}|\\
|\childdocof{|\textit{main}|}|\\
\end{tabular}
\end{center}
at the top of every child file \textit{child}
which is included by |\include{|\textit{child}|}|
from within the main file
(or at least for those files to be compiled individually).
The argument \textit{main} must be the filename of the main file.

There are a couple of
considerations in setting up the main and child documents:

%%%%%%%%%%%%%%%%%%%%%%%%%%%%%%%%%%%%%%%%
\paragraph{Restrictions.}

Please note the following restrictions:
\begin{itemize}
\item
|\childdocmain| must be called with one argument \textit{main}
to ensure compatibility with earlier version of the package.
It must either be empty (|\childdocmain{}|)
or precisely match the filename of the main file in which it is specified.
See \secref{sec:detection} for further information.
\item
The filename \textit{main} must be specified without the |.tex| extension.
\item
The filename \textit{main} is case sensitive
(even in case-insensitive file systems)
due to internal string comparison.
\item
The argument \textit{main} should be fully expanded, it cannot be a macro.
\item
Subdirectories and special characters should be avoided in filenames.
\item
The command |\childdocmain{|\textit{main}|}| must be followed by a whitespace.
It should not be followed immediately by another command
or by a comment mark `|%|'.
This is because the \TeX{} parser reads the token immediately following
the argument of |\childdocmain| and puts it
at the beginning of every child section;
however, a white\-space is ignored.
\end{itemize}

%%%%%%%%%%%%%%%%%%%%%%%%%%%%%%%%%%%%%%%%
\paragraph{Content of Main File.}

It is advisable to place all content in the child files included by |\include|.
Any output contained in the main file will appear in all child documents
unless suppressed manually;
it cannot be suppressed automatically by the |\includeonly| directive
and thus should normally be avoided.
A method to include some content in the main file
by means of conditional processing is described in \secref{sec:conditional}.

%%%%%%%%%%%%%%%%%%%%%%%%%%%%%%%%%%%%%%%%
\paragraph{Page Numbering.}

When only a part of the document is compiled,
the appropriate numbering of pages
(as well as other status parameters)
is determined from the |.aux| files.
The latter contain information from previous passes.
However this information needs to propagate through
all intermediate child documents.
Therefore the page numbering in child documents may well
be inconsistent until the complete document is compiled at least once.

A useful (if unconventional) way to always ensure a consistent
page numbering is to restart the numbering in each child document
and denote the pages by `\textit{child}|.|\textit{page}'
where \textit{child} represents the chapter/section number of the child file.
This can be achieved by the command
|\numberwithin{page}{|\textit{child}|}|
of the \textsf{amsmath} package
where \textit{child} can be |chapter| or |section|
depending on the chosen structuring.
Alternatively, one can modify the macro |\thepage| appropriately
and reset the counter |page| at the start of each child file.

%%%%%%%%%%%%%%%%%%%%%%%%%%%%%%%%%%%%%%%%%%%%%%%%%%%%%%%%%%%%%%%%%%%%%%%%%%%%%%%%
\subsection{Conditional Processing}
\label{sec:conditional}

The package provides a mechanism to compile different versions
of a document. To customise the versions further some conditional processing
can come in handy to distinguish which version is being compiled.
The package provides two macros to describe the compilation context:

%%%%%%%%%%%%%%%%%%%%%%%%%%%%%%%%%%%%%%%%
\DescribeMacro{\ifchilddoc}
The conditional |\ifchilddoc| distinguishes between the compilation of
child documents and the main document:
%
\begin{center}
|\ifchilddoc |\textit{child-code}| |[|\||else |\textit{main-code}]| \||fi|
\end{center}

%%%%%%%%%%%%%%%%%%%%%%%%%%%%%%%%%%%%%%%%
\DescribeMacro{\childdocname}
\DescribeMacro{\childdocjob}
The macro |\childdocname| contains the filename (without extension)
of the main or child file being processed.
Note that |\childdocjob| will always contain the name of the main file.

%%%%%%%%%%%%%%%%%%%%%%%%%%%%%%%%%%%%%%%%
\paragraph{Title Page.}

Conditional processing can be used to include a title or banner page
in the main document when proper precautions are taken.
Importantly, the code in the main file should ensure that the page counter
(as well as other status parameters which are stored in the |.aux| files)
takes the same value after the conditional processing.
Otherwise the page numbers may take divergent values
depending on which part is compiled.

For example, a title page could be declared by:
%
\begin{center}
\begin{tabular}{l}
|\ifchilddoc\||else|\\
|\addtocounter{page}{-1}|\\
\textit{code for title page}\\
|\newpage|\\
|\||fi|
\end{tabular}
\end{center}
%
A banner page for the child documents can be generated by:
%
\begin{center}
\begin{tabular}{l}
|\ifchilddoc|\\
|\addtocounter{page}{-1}|\\
\textit{code for banner page}\\
|\newpage|\\
|\||fi|
\end{tabular}
\end{center}
%
Here one could write a message such as:
\begin{center}
|This is the part \childdocname{} of \childdocjob{}.|
\end{center}

%%%%%%%%%%%%%%%%%%%%%%%%%%%%%%%%%%%%%%%%%%%%%%%%%%%%%%%%%%%%%%%%%%%%%%%%%%%%%%%%
\subsection{Flags}
\label{sec:flags}

The package makes it easy to generate different versions
of the main or child documents.
To this end compilation flags can be defined
and assigned different default values.
They will be particularly useful in conjunction
with the forwarding mechanism described in \secref{sec:forward}.

For example, it may be useful to have a flag |\version|
which can be set to |draft| or |final|.
The document source will contain some conditional code
depending on the value of |\version|.
Suppose further, the flag should default to |final| for the main file
and to |draft| for child files
which is a natural assignment for editing the document.
This is achieved by placing the following code
in the preamble of the main document
(below the |\childdocmain| directive):
%
\begin{center}
\begin{tabular}{l}
|\ifchilddoc|\\
|\providecommand{\version}{draft}|\\
|\||else|\\
|\providecommand{\version}{final}|\\
|\||fi|
\end{tabular}
\end{center}
%
The definition by |\providecommand| makes sure
that previous definitions are not overwritten.
Further statements |\providecommand{\version}{...}|
can thus be added before the above code to override it.

For the main file, one might add a line
(between |\childdocmain| and the above block)
%
\begin{center}
|%\ifchilddoc\||else\providecommand{\version}{draft}\||fi|
\end{center}
%
which can be uncommented to produce a draft version.
Likewise one can add a line to the very top of a child file
(above the |\childdocof{|\textit{main}|}| directive)
%
\begin{center}
|%\providecommand{\version}{final}|
\end{center}
%
which can be uncommented to produce the final version of this child document.

%%%%%%%%%%%%%%%%%%%%%%%%%%%%%%%%%%%%%%%%%%%%%%%%%%%%%%%%%%%%%%%%%%%%%%%%%%%%%%%%
\subsection{Forwarding}
\label{sec:forward}

Different versions of the main or child documents
using compilation flags as described in \secref{sec:flags}
can be (permanently) stored in different files
for convenient compilation, viewing and distribution.
To this end, the package defines a command
to pass on compilation to a different file:

%%%%%%%%%%%%%%%%%%%%%%%%%%%%%%%%%%%%%%%%
\DescribeMacro{\childdocforward}
The command |\childdocforward| redirects processing to
another source file:
%
\begin{center}
\begin{tabular}{l}
|\input{childdoc.def}|\\
|\childdocforward[|\textit{main}|]{|\textit{dest}|}|\\
\end{tabular}
\end{center}
%
The argument \textit{dest} is the destination file
(without extension).
It should be the main file or one of the child files.
Note that further \textsf{childdoc} directives
such as |\childdocof| and |\childdocforward|
in the indicated file will be processed in this form.
The optional argument \textit{main}
passes on directly to the main file \textit{main}
while pretending to compile the child \textit{dest}.
This form behaves as if \textit{dest}
issues |\childdocof{|\textit{main}|}| right away,
and no further \textsf{childdoc} directives will be processed.

%%%%%%%%%%%%%%%%%%%%%%%%%%%%%%%%%%%%%%%%
\DescribeMacro{\...prefix}
In the alternative form |\childdocforwardprefix|,
%
\begin{center}
\begin{tabular}{l}
|\input{childdoc.def}|\\
|\childdocforwardprefix[|\textit{main}|]{|\textit{prefix}|}{|\textit{dest}|}|
\end{tabular}
\end{center}
%
the destination file is determined by a pattern
depending on the current file:
To make this work, the current file must be called
`{\textit{prefix}\hspace{0.2em}\textit{suffix}}'
with \textit{prefix} matching precisely the argument.
Processing is then passed on to the file
`{\textit{dest}\hspace{0.2em}\textit{suffix}}'.
Surely, the same effect is achieved by
directly specifying the
argument `{\textit{dest}\hspace{0.2em}\textit{suffix}}'
in the first form.
However, that requires to set up a different file
for each child. With the alternative form of the command
all these files can have exactly the same content
which simplifies setting them up and maintaining them.

For example, the following file |draft.tex|
with a compilation flag |\version| as described in \secref{sec:flags}
compiles the main document as a draft:
%
\begin{center}
\begin{tabular}{l}
|\def\version{draft}|\\
|\input{childdoc.def}|\\
|\childdocforward{|\textit{main}|}|
\end{tabular}
\end{center}
%
Likewise, the following files |final|\textit{nn}|.tex|
compile the final version of the child document
|child|\textit{nn}|.tex|:
%
\begin{center}
\begin{tabular}{l}
|\def\version{final}|\\
|\input{childdoc.def}|\\
|\childdocforwardprefix{final}{child}|
\end{tabular}
\end{center}
%

Note that when several versions of a main file and/or of each child file
are to be generated, it may be convenient to set up a |Makefile| or
shell script to automatise the process.

%%%%%%%%%%%%%%%%%%%%%%%%%%%%%%%%%%%%%%%%%%%%%%%%%%%%%%%%%%%%%%%%%%%%%%%%%%%%%%%%
\subsection{Command Line Processing}
\label{sec:commandline}

The effect of redirection files can also be achieved by invoking
the \LaTeX{} compiler with a more elaborate command line.
Most conveniently this should be done as part
of a shell script or a |Makefile|.

When using \textsf{childdoc} in the main file, the following
command lines effectively perform a redirection
(note that depending on the shell being used,
backslashes may have to be doubled: `|\|' $\to$ `|\\|'):
%
\begin{center}
|... -jobname "|\textit{target}|" |\\|"|[\textit{flags}]%
|\input{childdoc.def}\childdocforward[|\textit{main}|]{|\textit{dest}|}"|
\end{center}
%
Here \textit{target} is the name of the output file,
\textit{main} is the name of the main file
and \textit{dest} is the name of the main or child file to be processed
(all filenames without extensions).
The optional argument \textit{main} can be omitted
if \textit{main} matches \textit{dest}.
Optionally, compilation \textit{flags} can be defined via |\def| commands.
This command line makes the \TeX{} engine believe
it is compiling the file \textit{target}
whose content is specified as the latter parameter.
The provided code then forwards the processing to
\textit{main} or \textit{dest} as described in \secref{sec:forward}.

%%%%%%%%%%%%%%%%%%%%%%%%%%%%%%%%%%%%%%%%%%%%%%%%%%%%%%%%%%%%%%%%%%%%%%%%%%%%%%%%
\subsection{Include by Input}
\label{sec:input}

Including child documents by |\include| has some restrictions by design.
Most notably, the content of a child document always occupies
its own set of pages; pages cannot be shared between child documents.
Usually, this behaviour makes perfect sense
because each child document contain an essential part of the document.
However, in some situations it may be desirable to compose
a document from a collection of parts
without having mandatory page breaks between then.
For this case, the package
provides a mechanism to include parts
by |\input| which can also be processed individually.
However, by construction this mechanism
requires manual handling of the content to be output.

%%%%%%%%%%%%%%%%%%%%%%%%%%%%%%%%%%%%%%%%
\DescribeMacro{\ifchilddocmanual}
The main file should be prepared as usual, see \secref{sec:include}.
However, the document body must make a distinction
between processing of an individual part and of the main document, e.g.:
%
\begin{center}
\begin{tabular}{l}
|\ifchilddocmanual|\\
|\input{\childdocname}|\\
|\||else|\\
\textit{document body with }|\input{|\textit{part}|}|\\
|\||fi|
\end{tabular}
\end{center}
%
The conditional |\ifchilddocmanual| is true whenever
a part to be included by |\input| is being compiled,
and the name of the part is stored in |\childdocname|.

%%%%%%%%%%%%%%%%%%%%%%%%%%%%%%%%%%%%%%%%
\DescribeMacro{\childdocby}
Each part to be included by |\input| should start with:
%
\begin{center}
\begin{tabular}{l}
|\input{childdoc.def}|\\
|\childdocby{|\textit{main}|}|\\
\end{tabular}
\end{center}
%
The directive |\childdocby| is similar to |\childdocof|
described in \secref{sec:include},
but the subsequent selection of content must be done manually.
To that end, both |\ifchilddoc| and |\ifchilddocmanual|
will be true upon processing of a part,
and the name of the part is stored in |\childdocname|.
Note that |\jobname| will be set to the filename of the current part
so that each part receives an individual |.aux| file
that does not interfere with the |.aux| file(s) of the main document.
This behaviour can be altered by the alternative form
|\childdocby[*]{|\textit{main}|}| (with a non-empty optional argument)
which uses the |.aux| file of the main document
by setting |\jobname| to \textit{main}.

%%%%%%%%%%%%%%%%%%%%%%%%%%%%%%%%%%%%%%%%%%%%%%%%%%%%%%%%%%%%%%%%%%%%%%%%%%%%%%%%
\subsection{Driver Development}
\label{sec:driver}

The \textsf{childdoc} mechanism can also be use for the development
of definition files such as \LaTeX{} styles or classes.
This case differs from the above setup with multiple parts
included by |\include| in that no |\includeonly| should be invoked.
This can be achieved by starting the include file
(before |\ProvidesPackage|) with:
%
\begin{center}
\begin{tabular}{l}
|\input{childdoc.def}|\\
|\childdocforward{|\textit{main}|}|\\
\end{tabular}
\end{center}
%
or alternatively with:
%
\begin{center}
\begin{tabular}{l}
|\input{childdoc.def}|\\
|\childdocby{|\textit{main}|}|\\
\end{tabular}
\end{center}
%
Both forms have slightly different effects as described above.
The main file is prepared as usual, see \secref{sec:include}.

%%%%%%%%%%%%%%%%%%%%%%%%%%%%%%%%%%%%%%%%%%%%%%%%%%%%%%%%%%%%%%%%%%%%%%%%%%%%%%%%
\subsection{Legacy Detection}
\label{sec:detection}

The directive |\childdocmain| in the main file can detect
whether the complete document or merely a child is to be compiled
even without using the directive |\childdocof|.
This method is deprecated because it is less robust
and there is no compelling reason to use it;
it is merely provided for backward compatibility
and it may be removed in future versions.

If the detection mechanism is to be used,
it is mandatory to correctly specify
the filename of the main file as the argument of |\childdocmain|:
%
\begin{center}
\begin{tabular}{l}
|\input{childdoc.def}|\\
|\childdocmain{|\textit{main}|}|\\
\end{tabular}
\end{center}
%
If |\jobname| does not match the argument \textit{main} of |\childdocmain|,
it is assumed that |\jobname| points to the child file to be compiled.
When using |\childdocmain| with the main file specified as argument,
it suffices to start a child file
with just |\input{|\textit{main}|}|
without loading of the package and using |\childdocof|.
If instead all processing is done
with the appropriate \textsf{childdoc} directives,
the argument of \textit{main} of |\childdocmain| can be empty.

An alternative version of the command line processing described
in \secref{sec:commandline} using the detection mechanism reads:
%
\begin{center}
|... -jobname "|\textit{target}|" "|[\textit{flags}]%
[|\def\jobname{|\textit{dest}|}|]|\input{|\textit{main}|}"|
\end{center}

%%%%%%%%%%%%%%%%%%%%%%%%%%%%%%%%%%%%%%%%%%%%%%%%%%%%%%%%%%%%%%%%%%%%%%%%%%%%%%%%
\subsection{Manual Code}
\label{sec:manual}

In case one cannot be certain whether the definitions file |childdoc.def|
is installed on the target \TeX{} distribution
and one prefers not to ship it,
it is conceivable to paste a few relevant commands into the sources.

To that end, drop all statements |\input{childdoc.def}|
and perform the replacements as outlined below.
Instead of |\childdocmain{|\textit{main}|}| add the following code
to the top of the main file:
%
\begin{center}
\begin{tabular}{l}
|\||ifdefined\childdocname\endinput\||fi\newif\ifchilddoc|\\
|\edef\childdocname{\scantokens\expandafter{\jobname\noexpand}}|\\
|\def\childdocmain{|\textit{main}|}\||ifx\childdocmain\childdocname\||else|\\
|\childdoctrue\includeonly{\childdocname}\let\jobname\childdocmain\||fi|\\
\end{tabular}
\end{center}
%
Instead of |\childdocof{|\textit{main}|}| just include the main file
at the top of each child file:
%
\begin{center}
|\input{|\textit{main}|}|
\end{center}
%
A simple redirection |\childdocforward{|\textit{dest}|}| is achieved by:
%
\begin{center}
|\def\jobname{|\textit{dest}|}\input{\jobname}|
\end{center}
%
The redirection with prefix
|\childdocforwardprefix[|\textit{prefix}|]{|\textit{dest}|}|
is accomplished by:
%
\begin{center}
\begin{tabular}{l}
|{\edef\jobname{\scantokens\expandafter{\jobname\noexpand}}|\\
|\def\redirectjob |\textit{prefix}|#1~~~{\gdef\jobname{|\textit{dest}|#1}}|\\
|\expandafter\redirectjob\jobname~~~}\input{\jobname}|
\end{tabular}
\end{center}

In an alternative approach,
child documents can be compiled by a specific command line
without additional code or specific definitions:
%
\begin{center}
|... -jobname "|\textit{target}|" "|[\textit{flags}]%
|\includeonly{|\textit{dest}|}\input{|\textit{main}|}"|
\end{center}
%

%%%%%%%%%%%%%%%%%%%%%%%%%%%%%%%%%%%%%%%%%%%%%%%%%%%%%%%%%%%%%%%%%%%%%%%%%%%%%%%%
%%%%%%%%%%%%%%%%%%%%%%%%%%%%%%%%%%%%%%%%%%%%%%%%%%%%%%%%%%%%%%%%%%%%%%%%%%%%%%%%
\section{Information}

%%%%%%%%%%%%%%%%%%%%%%%%%%%%%%%%%%%%%%%%%%%%%%%%%%%%%%%%%%%%%%%%%%%%%%%%%%%%%%%%
\subsection{Copyright}

Copyright \copyright{} 2017--2018 Niklas Beisert

This work may be distributed and/or modified under the
conditions of the \LaTeX{} Project Public License, either version 1.3
of this license or (at your option) any later version.
The latest version of this license is in
  \url{http://www.latex-project.org/lppl.txt}
and version 1.3 or later is part of all distributions of \LaTeX{}
version 2005/12/01 or later.

This work has the LPPL maintenance status `maintained'.

The Current Maintainer of this work is Niklas Beisert.

This work consists of the files |README.txt|, |childdoc.ins| and |childdoc.dtx|
as well as the derived files |childdoc.def|, |cdocsamp.tex|
with |cdocsch1.tex|, |cdocsch2.tex|, |cdocspt3.tex|, |cdocspt4.tex|,
|cdocsdrf.tex|, |cdocsfn1.tex|, |cdocsfn2.tex|
as well as |childdoc.pdf|.

%%%%%%%%%%%%%%%%%%%%%%%%%%%%%%%%%%%%%%%%%%%%%%%%%%%%%%%%%%%%%%%%%%%%%%%%%%%%%%%%
\subsection{Files and Installation}

The package consists of the files:
%
\begin{center}
\begin{tabular}{ll}
    |README.txt|   & readme file \\
    |childdoc.ins| & installation file \\
    |childdoc.dtx| & source file \\
    |childdoc.def| & definition file \\
    |cdocsamp.tex| & sample main file \\
    |cdocsch1.tex| & sample include file \\
    |cdocsch2.tex| & sample include file \\
    |cdocspt3.tex| & sample part file \\
    |cdocspt4.tex| & sample part file \\
    |cdocsdrf.tex| & sample redirection file \\
    |cdocsfn1.tex| & sample redirection file \\
    |cdocsfn2.tex| & sample redirection file \\
    |childdoc.pdf| & manual
\end{tabular}
\end{center}
%
The distribution consists of the files
|README.txt|, |childdoc.ins| and |childdoc.dtx|.
%
\begin{itemize}
\item
Run (pdf)\LaTeX{} on |childdoc.dtx|
to compile the manual |childdoc.pdf| (this file).
\item
Run \LaTeX{} on |childdoc.ins| to create the definitions file |childdoc.def|
and the sample |cdocsamp.tex| with include files
|cdocsch1.tex|, |cdocsch2.tex|, |cdocspt3.tex|, |cdocspt4.tex|,
|cdocsdrf.tex|, |cdocsfn1.tex|, |cdocsfn2.tex|.
Then copy the file |childdoc.def| to an appropriate directory of your \LaTeX{}
distribution, e.g.\ \textit{texmf-root}|/tex/latex/childdoc|.
\end{itemize}

%%%%%%%%%%%%%%%%%%%%%%%%%%%%%%%%%%%%%%%%%%%%%%%%%%%%%%%%%%%%%%%%%%%%%%%%%%%%%%%%
\subsection{Related CTAN Packages}

There are several other packages which offer a similar functionality:
%
\begin{itemize}
\item
The packages
\href{http://ctan.org/pkg/docmute}{\textsf{docmute}},
\href{http://ctan.org/pkg/includex}{\textsf{includex}} and
\href{http://ctan.org/pkg/standalone}{\textsf{standalone}}
provide commands to include only the document body of
a child file thus allowing both files to be compiled individually.
\item
The packages \href{http://ctan.org/pkg/subdocs}{\textsf{subdocs}}
and \href{http://ctan.org/pkg/subfiles}{\textsf{subfiles}}
provide structures in which the main and child documents can be
encapsulated and allowing them to be compiled individually.
The inclusion mechanism is different from the conventional |\include|.
\item
The package \href{http://ctan.org/pkg/combine}{\textsf{combine}}
is an elaborate solution to combine several documents into one.
\end{itemize}
%
See also the CTAN topic \href{http://ctan.org/topic/subdocs}{\textsf{subdocs}}
for further related packages.
The present package differs from the above solutions in that
a document structure constructed with the conventional |\include| mechanism
just needs two extra commands at the top of every file
such that all constituent files can be compiled individually.

%%%%%%%%%%%%%%%%%%%%%%%%%%%%%%%%%%%%%%%%%%%%%%%%%%%%%%%%%%%%%%%%%%%%%%%%%%%%%%%%
%\subsection{Feature Suggestions}
%
%The following is a list of features which may be useful for future
%versions of this package:
%%
%\begin{itemize}
%\item
%\ldots
%\end{itemize}

%%%%%%%%%%%%%%%%%%%%%%%%%%%%%%%%%%%%%%%%%%%%%%%%%%%%%%%%%%%%%%%%%%%%%%%%%%%%%%%%
\subsection{Revision History}

%%%%%%%%%%%%%%%%%%%%%%%%%%%%%%%%%%%%%%%%
\paragraph{v2.0:} 2018/12/30

\begin{itemize}
\item
immediate forward processing
\item
added |\childdocby| mechanism
\item
manual restructured
\end{itemize}

%%%%%%%%%%%%%%%%%%%%%%%%%%%%%%%%%%%%%%%%
\paragraph{v1.6:} 2018/01/17

\begin{itemize}
\item
application for development of include files
\item
corrections to manual
\end{itemize}

%%%%%%%%%%%%%%%%%%%%%%%%%%%%%%%%%%%%%%%%
\paragraph{v1.5:} 2017/05/21

\begin{itemize}
\item
more complete structuring introduced
\item
|\childdocof| introduced
\item
|\childdoc| renamed to |\childdocmain|
\item
|\childredirect| renamed to |\childdocforward| and |\childdocforwardprefix|
and functionality expanded
\end{itemize}

%%%%%%%%%%%%%%%%%%%%%%%%%%%%%%%%%%%%%%%%
\paragraph{v1.0:} 2017/04/27

\begin{itemize}
\item
manual and install package
\item
first version published on CTAN
\end{itemize}

%%%%%%%%%%%%%%%%%%%%%%%%%%%%%%%%%%%%%%%%
\paragraph{v0.6:} 2017/04/26

\begin{itemize}
\item
redirection mechanism added
\end{itemize}

%%%%%%%%%%%%%%%%%%%%%%%%%%%%%%%%%%%%%%%%
\paragraph{v0.5:} 2017/04/26

\begin{itemize}
\item
functionality in definition file
\end{itemize}


%%%%%%%%%%%%%%%%%%%%%%%%%%%%%%%%%%%%%%%%%%%%%%%%%%%%%%%%%%%%%%%%%%%%%%%%%%%%%%%%
%%%%%%%%%%%%%%%%%%%%%%%%%%%%%%%%%%%%%%%%%%%%%%%%%%%%%%%%%%%%%%%%%%%%%%%%%%%%%%%%
%%%%%%%%%%%%%%%%%%%%%%%%%%%%%%%%%%%%%%%%%%%%%%%%%%%%%%%%%%%%%%%%%%%%%%%%%%%%%%%%
\appendix

\settowidth\MacroIndent{\rmfamily\scriptsize 000\ }

 \DocInput{childdoc.dtx}

\end{document}
%</driver>
% \fi
%
% %%%%%%%%%%%%%%%%%%%%%%%%%%%%%%%%%%%%%%%%%%%%%%%%%%%%%%%%%%%%%%%%%%%%%%%%%%%%%%
% %%%%%%%%%%%%%%%%%%%%%%%%%%%%%%%%%%%%%%%%%%%%%%%%%%%%%%%%%%%%%%%%%%%%%%%%%%%%%%
% \section{Sample}
%\iffalse
%<*samplemain>
%\fi
%
% The following presents a sample document
% with two chapters, two parts, a title page,
% a compile flag as well as three forwarding files to set the flag.
% It consists of eight |.tex| files:
% \begin{center}
% \begin{tabular}{ll}
% |cdocsamp.tex|&main file\\
% |cdocsch1.tex|&include file for chapter 1\\
% |cdocsch2.tex|&include file for chapter 2\\
% |cdocspt3.tex|&include file for part 3\\
% |cdocspt4.tex|&include file for part 4\\
% |cdocsdrf.tex|&forwarding file for main file in draft mode\\
% |cdocsfi1.tex|&forwarding file for final version of chapter 1\\
% |cdocsfi2.tex|&forwarding file for final version of chapter 2\\
% \end{tabular}
% \end{center}
% Each of the eight files can be compiled directly by the \LaTeX{} compiler.
%
% %%%%%%%%%%%%%%%%%%%%%%%%%%%%%%%%%%%%%%
% \paragraph{Main File.}
%
% The main file is called |cdocsamp.tex|.
%
% Load the \textsf{childdoc} definitions and
% declare the filename for the main document:
%    \begin{macrocode}
\input{childdoc.def}
\childdocmain{}
%    \end{macrocode}

% Optional override for |\version| flag:
%    \begin{macrocode}
%%\ifchilddoc\else\providecommand{\version}{draft}\fi
%    \end{macrocode}

% Define the default values for the |\version| flag
% (|final| for the main file and |draft| for childs):
%    \begin{macrocode}
\ifchilddoc
\providecommand{\version}{draft}
\else
\providecommand{\version}{final}
\fi
%    \end{macrocode}

% Load the standard document class:
%    \begin{macrocode}
\documentclass[12pt]{article}
%    \end{macrocode}

% Start the document body:
%    \begin{macrocode}
\begin{document}
%    \end{macrocode}

% Declare a title page.
% Print title, part of document being processed and version flag:
%    \begin{macrocode}
\addtocounter{page}{-1}
\begin{center}
{\LARGE\bfseries{}childdoc example\par}
\vspace{1cm}
\ifchilddoc
\ifchilddocmanual part\else chapter\fi:
`\childdocname' of `\childdocjob'\par
\else
main document: `\childdocjob'\par
\fi
version: \version\par
\end{center}
\newpage
%    \end{macrocode}

% Manually include selected file,
% otherwise process as usual:
%    \begin{macrocode}
\ifchilddocmanual
\section*{part `\childdocname'}
\input{\childdocname}
\else
%    \end{macrocode}

% Include the two chapters:
%    \begin{macrocode}
\include{cdocsch1}
\include{cdocsch2}
%    \end{macrocode}

% Include the two parts unless only chapters should be displayed:
%    \begin{macrocode}
\ifchilddoc\else
\section{part three}
\input{cdocspt3}
\section{part four}
\input{cdocspt4}
\fi
%    \end{macrocode}

% Process as usual until here:
%    \begin{macrocode}
\fi
%    \end{macrocode}

% End of document body:
%    \begin{macrocode}
\end{document}
%    \end{macrocode}
%\iffalse
%</samplemain>
%\fi
%
% %%%%%%%%%%%%%%%%%%%%%%%%%%%%%%%%%%%%%%
% \paragraph{Chapter Include Files.}
%
% The include files are called |cdocsch1.tex| and |cdocsch2.tex|.
%
%\iffalse
%<*samplechap1|samplechap2>
%\fi

% Optional override for |\version| flag:
%    \begin{macrocode}
%%\providecommand{\version}{final}
%    \end{macrocode}

% Include the main document:
%    \begin{macrocode}
\input{childdoc.def}
\childdocof{cdocsamp}
%    \end{macrocode}

%\iffalse
%</samplechap1|samplechap2>
%\fi
%
%\iffalse
%<*samplechap1>
%\fi
% Some text for chapter 1:
%    \begin{macrocode}
\section{one}
some text in chapter one
%    \end{macrocode}

%\iffalse
%</samplechap1>
%\fi
% Some text for chapter 2:
%\iffalse
%<*samplechap2>
%\fi
%    \begin{macrocode}
\section{two}
more text in chapter two
%    \end{macrocode}

%\iffalse
%</samplechap2>
%\fi
%
% %%%%%%%%%%%%%%%%%%%%%%%%%%%%%%%%%%%%%%
% \paragraph{Part Include Files.}
%
% The include files are called |cdocspt3.tex| and |cdocspt4.tex|.
%
%\iffalse
%<*samplepart3|samplepart4>
%\fi

% Optional override for |\version| flag:
%    \begin{macrocode}
%%\providecommand{\version}{final}
%    \end{macrocode}

% Include the main document:
%    \begin{macrocode}
\input{childdoc.def}
\childdocby{cdocsamp}
%    \end{macrocode}

%\iffalse
%</samplepart3|samplepart4>
%\fi
%
%\iffalse
%<*samplepart3>
%\fi
% Some text for part 3:
%    \begin{macrocode}
some text in part three
%    \end{macrocode}

%\iffalse
%</samplepart3>
%\fi
% Some text for part 4:
%\iffalse
%<*samplepart4>
%\fi
%    \begin{macrocode}
more text in part four
%    \end{macrocode}

%\iffalse
%</samplepart4>
%\fi
%
% %%%%%%%%%%%%%%%%%%%%%%%%%%%%%%%%%%%%%%
% \paragraph{Forwarding for a Complete Draft.}
%
% The following forwarding file |cdocsdrf.tex|
% compiles the main document in draft mode:
%\iffalse
%<*sampledraft>
%\fi
%    \begin{macrocode}
\def\version{draft}
\input{childdoc.def}
\childdocforward{cdocsamp}
%    \end{macrocode}

%\iffalse
%</sampledraft>
%\fi
%
% %%%%%%%%%%%%%%%%%%%%%%%%%%%%%%%%%%%%%%
% \paragraph{Forwarding for Final Version of the Chapters.}
%
% The following forwarding files |cdocsfn1.tex| and |cdocsfn2.tex|
% (with identical content)
% compile the final versions of the child documents
% |cdocsch1.tex| and |cdocsch2.tex|, respectively:
%\iffalse
%<*samplefinal>
%\fi
%    \begin{macrocode}
\def\version{final}
\input{childdoc.def}
\childdocforwardprefix[cdocsamp]{cdocsfn}{cdocsch}
%    \end{macrocode}

%\iffalse
%</samplefinal>
%\fi
%
% %%%%%%%%%%%%%%%%%%%%%%%%%%%%%%%%%%%%%%
% \paragraph{Command Line Processing.}
%
% The following three command lines generate the output files
% |cdocscld|, |cdocscl1| and |cdocscl2|
% which should be identical to
% |cdocsdrf|, |cdocsch1| and |cdocsfn2|, respectively:
% \begin{center}
% \begin{tabular}{l}
% |latex -jobname cdocscld \|\\
% |  "\def\version{draft}\input{childdoc.def}\childdocforward{cdocsamp}"|\\
% |latex -jobname cdocscl1 \|\\
% |  "\input{childdoc.def}\childdocforward[cdocsamp]{cdocsch1}"|\\
% |latex -jobname cdocscl2 \|\\
% |  "\def\version{final}\input{childdoc.def}\childdocforward{cdocsch2}"|
% \end{tabular}
% \end{center}
% Note that the trailing backslash on each first line
% merely continues the input to the second line
% (for convenient cut ant paste).
% Furthermore, the command |latex| can be replaced by any
% of its alternative versions such as |pdflatex|.
%
% %%%%%%%%%%%%%%%%%%%%%%%%%%%%%%%%%%%%%%%%%%%%%%%%%%%%%%%%%%%%%%%%%%%%%%%%%%%%%%
% %%%%%%%%%%%%%%%%%%%%%%%%%%%%%%%%%%%%%%%%%%%%%%%%%%%%%%%%%%%%%%%%%%%%%%%%%%%%%%
% \section{Implementation}
%\iffalse
%<*package>
%\fi
%
% This section describes the definitions file |childdoc.def|.

% The definitions cannot be loaded using |\usepackage| or |\RequirePackage|
% which has a mechanism to prevent loading a style file more than once.
% When loading the definitions by means of |\input|
% multiple instances have to be prevented manually:
%\iffalse
%This code needs to be before the `\ProvidesFile' directive
%which is defined at the beginning of this file.
%Therefore it is also placed there and commented out here.
%</package>
%<*discard>
%\fi
%    \begin{macrocode}
\ifdefined\childdocmain\endinput\fi
%    \end{macrocode}
%\iffalse
%</discard>
%<*package>
%\fi
%
% \macro{\ifchilddoc}
% \macro{\ifchilddocmanual}
% The conditional |\ifchilddoc| tells whether a
% child (true) or main (false) document is being compiled.
% The conditional |\ifchilddocmanual| tells whether
% the |\includeonly| mechanism is used (false) or
% the selection of child files must be performed manually (true).
% The definitions initialise to false:
%    \begin{macrocode}
\newif\ifchilddoc
\newif\ifchilddocmanual
%    \end{macrocode}

% \macro{\childdocname}
% \macro{\childdocjob}
% The macro |\childdocname| stores the name of the main document
% to be compiled. The macro |\childdocjob| stores the name of
% the document on which the \LaTeX{} compiler was originally invoked.
% The content of |\jobname| cannot be compared
% to filenames specified in the source due to different catcodes.
% The following code rescans |\jobname|, stores the result
% in |\childdocname| and saves a copy in |\childdocjob|:
%    \begin{macrocode}
\edef\childdocname{\scantokens\expandafter{\jobname\noexpand}}
\let\childdocjob\childdocname
%    \end{macrocode}

% \macro{\childdocdisable}
% The macro |\childdocdisable| prevents the main file
% from being processed more than once.
% At this stage, the main document command |\childdocmain|
% is assumed to be called once again where it should do nothing.
% Any subsequent call to it should prevent
% a secondary processing of the main document
% It overwrites the forwarding commands
% |\childdocof| and |\childdocforward|
% with empty macros to prevent further inclusions of the main document:
%    \begin{macrocode}
\newcommand{\childdocdisable}
{
  \renewcommand{\childdocmain}[1]{\renewcommand{\childdocmain}[1]{\endinput}}
  \renewcommand{\childdocof}[1]{}
  \renewcommand{\childdocby}[2][]{}
  \renewcommand{\childdocforward}[2][]{}
  \renewcommand{\childdocdisable}{}
}
%    \end{macrocode}

% \macro{\childdocmain}
% The macro |\childdocmain| is to be called at the top of the main file
% with nothing or the main filename (without extension) as argument.
% First, it breaks loops.
% If the argument is not empty and does not match |\childdocname|
% (which is set by the first inclusion of |childdoc.def|),
% |\ifchilddoc| is set to true, |\includeonly| is applied to the child file
% and |\jobname| is set to the main file
% (for proper handling of |.aux| files):
%    \begin{macrocode}
\newcommand{\childdocmain}[1]
{
  \childdocdisable\childdocmain{}
  \if?#1?\else
    \begingroup
      \def\childdoctmp{#1}
      \ifx\childdoctmp\childdocname
        \def\childdoctmp{}
      \else
        \def\childdoctmp
        {
          \childdoctrue
          \includeonly{\childdocname}
          \def\childdocjob{#1}
          \def\jobname{#1}
        }
      \fi
      \expandafter
    \endgroup
    \childdoctmp
  \fi
}
%    \end{macrocode}

% \macro{\childdocof}
% The command |\childdocof| redirects
% compilation to the main file |#1|.
%    \begin{macrocode}
\newcommand{\childdocof}[1]
{
  \childdocdisable
  \childdoctrue
  \includeonly{\childdocname}
  \def\jobname{#1}
  \def\childdocjob{#1}
  \input{#1}
}
%    \end{macrocode}

% \macro{\childdocby}
% The command |\childdocby| ....
%    \begin{macrocode}
\newcommand{\childdocby}[2][]
{
  \childdocdisable
  \childdoctrue
  \childdocmanualtrue
  \if?#1?\else
    \def\jobname{#2}
  \fi
  \def\childdocjob{#2}
  \input{#2}
  \endinput
}
%    \end{macrocode}

% \macro{\childdocforward}
% The command |\childdocforward| redirects
% compilation to the main file or
% (if the optional argument is given) a child file.
% Parameters are set as if the main file
% or a child file starting with |\childdocof| was compiled.
% Then compilation is handed over to the main file:
%    \begin{macrocode}
\newcommand{\childdocforward}[2][]
{
  \begingroup
    \if?#1?
      \def\childdoctmp
      {
        \def\childdocname{#2}
        \def\childdocjob{#2}
        \def\jobname{#2}
        \input{#2}
        \endinput
      }
    \else
      \def\childdoctmp
      {
        \childdocdisable
        \def\childdocname{#2}
        \childdoctrue
        \includeonly{#2}
        \def\childdocjob{#1}
        \def\jobname{#1}
        \input{#1}
        \endinput
      }
    \fi
    \expandafter
  \endgroup
  \childdoctmp
}
%    \end{macrocode}

% \macro{\childdocforwardprefix}
% The command |\childdocforwardprefix| redirects
% compilation to the main or a child file by means of a pattern.
% The prefix |#1| in the current filename is replaced by |#2|
% and the suffix of the current filename is kept
% (it is assumed that the filename does not contain the substring `|~~~|'
% which is used as a delimiter).
% Compilation is handed over to the new file by |\childdocforward|:
%    \begin{macrocode}
\newcommand{\childdocforwardprefix}[3][]
{
  \begingroup
    \def\childdocextract #2##1~~~{\def\childdoctmp{\childdocforward[#1]{#3##1}}}
    \expandafter\childdocextract\childdocname~~~
    \expandafter
  \endgroup
  \childdoctmp
}
%    \end{macrocode}

% \macro{\childdoc}
% The deprecated macro |\childdoc| is a legacy version of |\childdocmain|:
%    \begin{macrocode}
\newcommand{\childdoc}{\childdocmain}
%    \end{macrocode}

% \macro{\childdocredirect}
% The deprecated macro |\childdocredirect| is a legacy version
% of |\childdocforward| and |\childdocforwardprefix|:
%    \begin{macrocode}
\newcommand{\childdocredirect}[2][]
{
  \begingroup
    \if?#1?
      \def\childdoctmp{\childdocforward{#2}}
    \else
      \def\childdoctmp{\childdocforwardprefix{#1}{#2}}
    \fi
    \expandafter
  \endgroup
  \childdoctmp
}
%    \end{macrocode}

%\iffalse
%</package>
%\fi
%
\endinput
|\\
|\childdocforward{|\textit{main}|}|\\
\end{tabular}
\end{center}
%
or alternatively with:
%
\begin{center}
\begin{tabular}{l}
|% \iffalse
%
% childdoc.dtx Copyright (C) 2017-2018 Niklas Beisert
%
% This work may be distributed and/or modified under the
% conditions of the LaTeX Project Public License, either version 1.3
% of this license or (at your option) any later version.
% The latest version of this license is in
%   http://www.latex-project.org/lppl.txt
% and version 1.3 or later is part of all distributions of LaTeX
% version 2005/12/01 or later.
%
% This work has the LPPL maintenance status `maintained'.
%
% The Current Maintainer of this work is Niklas Beisert.
%
% This work consists of the files childdoc.dtx and childdoc.ins
% and the derived files childdoc.def and cdocsamp.tex with
% cdocsch1.tex, cdocsch2.tex, cdocsdrf.tex, cdocsfn1.tex, cdocsfn2.tex.
%
%<package>\ifdefined\childdocmain\endinput\fi
%<package>\ProvidesFile{childdoc.def}[2018/12/30 v2.0 child document driver]
%<samplemain>\ProvidesFile{cdocsamp.tex}[2018/12/30 v2.0 sample for childdoc]
%<*driver>
%\ProvidesFile{childdoc.drv}[2018/12/30 v2.0 childdoc reference manual file]
\PassOptionsToClass{10pt,a4paper}{article}
\documentclass{ltxdoc}

\usepackage[margin=35mm]{geometry}
\usepackage{hyperref}
\usepackage{hyperxmp}
\usepackage[usenames]{color}

\hypersetup{colorlinks=true}
\hypersetup{pdfstartview=FitH}
\hypersetup{pdfpagemode=UseNone}
\hypersetup{pdfsource={}}
\hypersetup{pdflang={en-UK}}
\hypersetup{pdfcopyright={Copyright 2017-2018 Niklas Beisert.
  This work may be distributed and/or modified under the
  conditions of the LaTeX Project Public License, either version 1.3
  of this license or (at your option) any later version.}}
\hypersetup{pdflicenseurl={http://www.latex-project.org/lppl.txt}}
\hypersetup{pdfcontactaddress={ETH Zurich, ITP, HIT K,
  Wolfgang-Pauli-Strasse 27}}
\hypersetup{pdfcontactpostcode={8093}}
\hypersetup{pdfcontactcity={Zurich}}
\hypersetup{pdfcontactcountry={Switzerland}}
\hypersetup{pdfcontactemail={nbeisert@itp.phys.ethz.ch}}
\hypersetup{pdfcontacturl={http://people.phys.ethz.ch/\xmptilde nbeisert/}}

\newcommand{\secref}[1]{\hyperref[#1]{section \ref*{#1}}}

\parskip1ex
\parindent0pt
\let\olditemize\itemize
\def\itemize{\olditemize\parskip0pt}

\begin{document}

\title{The \textsf{childdoc} Package}
\hypersetup{pdftitle={The childdoc Package}}
\author{Niklas Beisert\\[2ex]
  Institut f\"ur Theoretische Physik\\
  Eidgen\"ossische Technische Hochschule Z\"urich\\
  Wolfgang-Pauli-Strasse 27, 8093 Z\"urich, Switzerland\\[1ex]
  \href{mailto:nbeisert@itp.phys.ethz.ch}
  {\texttt{nbeisert@itp.phys.ethz.ch}}}
\hypersetup{pdfauthor={Niklas Beisert}}
\hypersetup{pdfsubject={Manual for the LaTeX2e Package childdoc}}
\date{30 December 2018, \textsf{v2.0}}
\maketitle

\begin{abstract}\noindent
\textsf{childdoc} is a \LaTeXe{} package
that enables the direct compilation
of document sections included by |\include|
to individual files.
\end{abstract}

\begingroup
\parskip0ex
\tableofcontents
\endgroup

%%%%%%%%%%%%%%%%%%%%%%%%%%%%%%%%%%%%%%%%%%%%%%%%%%%%%%%%%%%%%%%%%%%%%%%%%%%%%%%%
%%%%%%%%%%%%%%%%%%%%%%%%%%%%%%%%%%%%%%%%%%%%%%%%%%%%%%%%%%%%%%%%%%%%%%%%%%%%%%%%
\section{Introduction}

\LaTeX{} provides a mechanism to structure a large document (such as a book)
into a main file and several child files (containing the chapters)
using the |\include| command.
This mechanism is beneficial for documents
which span hundreds of pages in order to
make the source file(s) more manageable.
Moreover, compilation can be restricted to
selected child files by means of the |\includeonly| command.
The latter feature can be used to reduce the compilation time while editing
(this was significantly more useful in the earlier days of \LaTeX{})
or to generate a smaller document which is easier to navigate.
Another application of |\includeonly| is to generate
documents consisting of selected parts of the complete document.

However, there are a few drawbacks of the plain |\include| mechanism:
\begin{itemize}
\item
The child files cannot be compiled on their own,
they can only be compiled via the main file.
A naive editing environment
(such as a text editor with an option
to have the current file processed by \LaTeX)
may require one to switch to the main file before compiling;
attempting to compile the child file produces errors.
\item
The main file must be modified (each time)
to adjust the |\includeonly| command
to the present needs. This easily leaves the main file in a messy state.
\item
The generated document will always carry the filename
of the main document. This is inconvenient if
several child files are to be compiled and
to be kept for distribution.
\end{itemize}

The present package provides a simple interface
to make child files individually compilable by \LaTeX{}.
Compiling a child file then has the same effect as compiling
the main file with an |\includeonly| command
to select the appropriate child.
Moreover the generated document will carry the name of the child
rather than the main file.
This resolves all three above issues.

This feature is meant to make the editing of books,
thesis documents and lecture notes somewhat more convenient.
However, the package can also be used efficiently for
composing a series of documents (such as exercise sheets)
which are typically distributed individually.
It then assists the author in generating the individual documents
(potentially in different versions)
as well as a document containing the collected series.
Another application is in developing style files
or other kinds of included material
where compilation of the style file could redirect
to a sample or test file.

%%%%%%%%%%%%%%%%%%%%%%%%%%%%%%%%%%%%%%%%%%%%%%%%%%%%%%%%%%%%%%%%%%%%%%%%%%%%%%%%
%%%%%%%%%%%%%%%%%%%%%%%%%%%%%%%%%%%%%%%%%%%%%%%%%%%%%%%%%%%%%%%%%%%%%%%%%%%%%%%%
\section{Usage}

First of all, the package \textsf{childdoc} is \emph{not} a standard
\LaTeXe{} |.sty| style file! Therefore it needs to be invoked in
a non-standard way.

%%%%%%%%%%%%%%%%%%%%%%%%%%%%%%%%%%%%%%%%%%%%%%%%%%%%%%%%%%%%%%%%%%%%%%%%%%%%%%%%
\subsection{Included Files}
\label{sec:include}

%%%%%%%%%%%%%%%%%%%%%%%%%%%%%%%%%%%%%%%%
\DescribeMacro{\childdocmain}
To use the package, add the commands
\begin{center}
\begin{tabular}{l}
|\input{childdoc.def}|\\
|\childdocmain{}|\\
\end{tabular}
\end{center}
at the very top of the main \LaTeX{} file,
in particular \emph{before} the |\documentclass| statement!
The argument of |\childdocmain| should be left empty
(but it must be present).

%%%%%%%%%%%%%%%%%%%%%%%%%%%%%%%%%%%%%%%%
\DescribeMacro{\childdocof}
Furthermore, add the commands
\begin{center}
\begin{tabular}{l}
|\input{childdoc.def}|\\
|\childdocof{|\textit{main}|}|\\
\end{tabular}
\end{center}
at the top of every child file \textit{child}
which is included by |\include{|\textit{child}|}|
from within the main file
(or at least for those files to be compiled individually).
The argument \textit{main} must be the filename of the main file.

There are a couple of
considerations in setting up the main and child documents:

%%%%%%%%%%%%%%%%%%%%%%%%%%%%%%%%%%%%%%%%
\paragraph{Restrictions.}

Please note the following restrictions:
\begin{itemize}
\item
|\childdocmain| must be called with one argument \textit{main}
to ensure compatibility with earlier version of the package.
It must either be empty (|\childdocmain{}|)
or precisely match the filename of the main file in which it is specified.
See \secref{sec:detection} for further information.
\item
The filename \textit{main} must be specified without the |.tex| extension.
\item
The filename \textit{main} is case sensitive
(even in case-insensitive file systems)
due to internal string comparison.
\item
The argument \textit{main} should be fully expanded, it cannot be a macro.
\item
Subdirectories and special characters should be avoided in filenames.
\item
The command |\childdocmain{|\textit{main}|}| must be followed by a whitespace.
It should not be followed immediately by another command
or by a comment mark `|%|'.
This is because the \TeX{} parser reads the token immediately following
the argument of |\childdocmain| and puts it
at the beginning of every child section;
however, a white\-space is ignored.
\end{itemize}

%%%%%%%%%%%%%%%%%%%%%%%%%%%%%%%%%%%%%%%%
\paragraph{Content of Main File.}

It is advisable to place all content in the child files included by |\include|.
Any output contained in the main file will appear in all child documents
unless suppressed manually;
it cannot be suppressed automatically by the |\includeonly| directive
and thus should normally be avoided.
A method to include some content in the main file
by means of conditional processing is described in \secref{sec:conditional}.

%%%%%%%%%%%%%%%%%%%%%%%%%%%%%%%%%%%%%%%%
\paragraph{Page Numbering.}

When only a part of the document is compiled,
the appropriate numbering of pages
(as well as other status parameters)
is determined from the |.aux| files.
The latter contain information from previous passes.
However this information needs to propagate through
all intermediate child documents.
Therefore the page numbering in child documents may well
be inconsistent until the complete document is compiled at least once.

A useful (if unconventional) way to always ensure a consistent
page numbering is to restart the numbering in each child document
and denote the pages by `\textit{child}|.|\textit{page}'
where \textit{child} represents the chapter/section number of the child file.
This can be achieved by the command
|\numberwithin{page}{|\textit{child}|}|
of the \textsf{amsmath} package
where \textit{child} can be |chapter| or |section|
depending on the chosen structuring.
Alternatively, one can modify the macro |\thepage| appropriately
and reset the counter |page| at the start of each child file.

%%%%%%%%%%%%%%%%%%%%%%%%%%%%%%%%%%%%%%%%%%%%%%%%%%%%%%%%%%%%%%%%%%%%%%%%%%%%%%%%
\subsection{Conditional Processing}
\label{sec:conditional}

The package provides a mechanism to compile different versions
of a document. To customise the versions further some conditional processing
can come in handy to distinguish which version is being compiled.
The package provides two macros to describe the compilation context:

%%%%%%%%%%%%%%%%%%%%%%%%%%%%%%%%%%%%%%%%
\DescribeMacro{\ifchilddoc}
The conditional |\ifchilddoc| distinguishes between the compilation of
child documents and the main document:
%
\begin{center}
|\ifchilddoc |\textit{child-code}| |[|\||else |\textit{main-code}]| \||fi|
\end{center}

%%%%%%%%%%%%%%%%%%%%%%%%%%%%%%%%%%%%%%%%
\DescribeMacro{\childdocname}
\DescribeMacro{\childdocjob}
The macro |\childdocname| contains the filename (without extension)
of the main or child file being processed.
Note that |\childdocjob| will always contain the name of the main file.

%%%%%%%%%%%%%%%%%%%%%%%%%%%%%%%%%%%%%%%%
\paragraph{Title Page.}

Conditional processing can be used to include a title or banner page
in the main document when proper precautions are taken.
Importantly, the code in the main file should ensure that the page counter
(as well as other status parameters which are stored in the |.aux| files)
takes the same value after the conditional processing.
Otherwise the page numbers may take divergent values
depending on which part is compiled.

For example, a title page could be declared by:
%
\begin{center}
\begin{tabular}{l}
|\ifchilddoc\||else|\\
|\addtocounter{page}{-1}|\\
\textit{code for title page}\\
|\newpage|\\
|\||fi|
\end{tabular}
\end{center}
%
A banner page for the child documents can be generated by:
%
\begin{center}
\begin{tabular}{l}
|\ifchilddoc|\\
|\addtocounter{page}{-1}|\\
\textit{code for banner page}\\
|\newpage|\\
|\||fi|
\end{tabular}
\end{center}
%
Here one could write a message such as:
\begin{center}
|This is the part \childdocname{} of \childdocjob{}.|
\end{center}

%%%%%%%%%%%%%%%%%%%%%%%%%%%%%%%%%%%%%%%%%%%%%%%%%%%%%%%%%%%%%%%%%%%%%%%%%%%%%%%%
\subsection{Flags}
\label{sec:flags}

The package makes it easy to generate different versions
of the main or child documents.
To this end compilation flags can be defined
and assigned different default values.
They will be particularly useful in conjunction
with the forwarding mechanism described in \secref{sec:forward}.

For example, it may be useful to have a flag |\version|
which can be set to |draft| or |final|.
The document source will contain some conditional code
depending on the value of |\version|.
Suppose further, the flag should default to |final| for the main file
and to |draft| for child files
which is a natural assignment for editing the document.
This is achieved by placing the following code
in the preamble of the main document
(below the |\childdocmain| directive):
%
\begin{center}
\begin{tabular}{l}
|\ifchilddoc|\\
|\providecommand{\version}{draft}|\\
|\||else|\\
|\providecommand{\version}{final}|\\
|\||fi|
\end{tabular}
\end{center}
%
The definition by |\providecommand| makes sure
that previous definitions are not overwritten.
Further statements |\providecommand{\version}{...}|
can thus be added before the above code to override it.

For the main file, one might add a line
(between |\childdocmain| and the above block)
%
\begin{center}
|%\ifchilddoc\||else\providecommand{\version}{draft}\||fi|
\end{center}
%
which can be uncommented to produce a draft version.
Likewise one can add a line to the very top of a child file
(above the |\childdocof{|\textit{main}|}| directive)
%
\begin{center}
|%\providecommand{\version}{final}|
\end{center}
%
which can be uncommented to produce the final version of this child document.

%%%%%%%%%%%%%%%%%%%%%%%%%%%%%%%%%%%%%%%%%%%%%%%%%%%%%%%%%%%%%%%%%%%%%%%%%%%%%%%%
\subsection{Forwarding}
\label{sec:forward}

Different versions of the main or child documents
using compilation flags as described in \secref{sec:flags}
can be (permanently) stored in different files
for convenient compilation, viewing and distribution.
To this end, the package defines a command
to pass on compilation to a different file:

%%%%%%%%%%%%%%%%%%%%%%%%%%%%%%%%%%%%%%%%
\DescribeMacro{\childdocforward}
The command |\childdocforward| redirects processing to
another source file:
%
\begin{center}
\begin{tabular}{l}
|\input{childdoc.def}|\\
|\childdocforward[|\textit{main}|]{|\textit{dest}|}|\\
\end{tabular}
\end{center}
%
The argument \textit{dest} is the destination file
(without extension).
It should be the main file or one of the child files.
Note that further \textsf{childdoc} directives
such as |\childdocof| and |\childdocforward|
in the indicated file will be processed in this form.
The optional argument \textit{main}
passes on directly to the main file \textit{main}
while pretending to compile the child \textit{dest}.
This form behaves as if \textit{dest}
issues |\childdocof{|\textit{main}|}| right away,
and no further \textsf{childdoc} directives will be processed.

%%%%%%%%%%%%%%%%%%%%%%%%%%%%%%%%%%%%%%%%
\DescribeMacro{\...prefix}
In the alternative form |\childdocforwardprefix|,
%
\begin{center}
\begin{tabular}{l}
|\input{childdoc.def}|\\
|\childdocforwardprefix[|\textit{main}|]{|\textit{prefix}|}{|\textit{dest}|}|
\end{tabular}
\end{center}
%
the destination file is determined by a pattern
depending on the current file:
To make this work, the current file must be called
`{\textit{prefix}\hspace{0.2em}\textit{suffix}}'
with \textit{prefix} matching precisely the argument.
Processing is then passed on to the file
`{\textit{dest}\hspace{0.2em}\textit{suffix}}'.
Surely, the same effect is achieved by
directly specifying the
argument `{\textit{dest}\hspace{0.2em}\textit{suffix}}'
in the first form.
However, that requires to set up a different file
for each child. With the alternative form of the command
all these files can have exactly the same content
which simplifies setting them up and maintaining them.

For example, the following file |draft.tex|
with a compilation flag |\version| as described in \secref{sec:flags}
compiles the main document as a draft:
%
\begin{center}
\begin{tabular}{l}
|\def\version{draft}|\\
|\input{childdoc.def}|\\
|\childdocforward{|\textit{main}|}|
\end{tabular}
\end{center}
%
Likewise, the following files |final|\textit{nn}|.tex|
compile the final version of the child document
|child|\textit{nn}|.tex|:
%
\begin{center}
\begin{tabular}{l}
|\def\version{final}|\\
|\input{childdoc.def}|\\
|\childdocforwardprefix{final}{child}|
\end{tabular}
\end{center}
%

Note that when several versions of a main file and/or of each child file
are to be generated, it may be convenient to set up a |Makefile| or
shell script to automatise the process.

%%%%%%%%%%%%%%%%%%%%%%%%%%%%%%%%%%%%%%%%%%%%%%%%%%%%%%%%%%%%%%%%%%%%%%%%%%%%%%%%
\subsection{Command Line Processing}
\label{sec:commandline}

The effect of redirection files can also be achieved by invoking
the \LaTeX{} compiler with a more elaborate command line.
Most conveniently this should be done as part
of a shell script or a |Makefile|.

When using \textsf{childdoc} in the main file, the following
command lines effectively perform a redirection
(note that depending on the shell being used,
backslashes may have to be doubled: `|\|' $\to$ `|\\|'):
%
\begin{center}
|... -jobname "|\textit{target}|" |\\|"|[\textit{flags}]%
|\input{childdoc.def}\childdocforward[|\textit{main}|]{|\textit{dest}|}"|
\end{center}
%
Here \textit{target} is the name of the output file,
\textit{main} is the name of the main file
and \textit{dest} is the name of the main or child file to be processed
(all filenames without extensions).
The optional argument \textit{main} can be omitted
if \textit{main} matches \textit{dest}.
Optionally, compilation \textit{flags} can be defined via |\def| commands.
This command line makes the \TeX{} engine believe
it is compiling the file \textit{target}
whose content is specified as the latter parameter.
The provided code then forwards the processing to
\textit{main} or \textit{dest} as described in \secref{sec:forward}.

%%%%%%%%%%%%%%%%%%%%%%%%%%%%%%%%%%%%%%%%%%%%%%%%%%%%%%%%%%%%%%%%%%%%%%%%%%%%%%%%
\subsection{Include by Input}
\label{sec:input}

Including child documents by |\include| has some restrictions by design.
Most notably, the content of a child document always occupies
its own set of pages; pages cannot be shared between child documents.
Usually, this behaviour makes perfect sense
because each child document contain an essential part of the document.
However, in some situations it may be desirable to compose
a document from a collection of parts
without having mandatory page breaks between then.
For this case, the package
provides a mechanism to include parts
by |\input| which can also be processed individually.
However, by construction this mechanism
requires manual handling of the content to be output.

%%%%%%%%%%%%%%%%%%%%%%%%%%%%%%%%%%%%%%%%
\DescribeMacro{\ifchilddocmanual}
The main file should be prepared as usual, see \secref{sec:include}.
However, the document body must make a distinction
between processing of an individual part and of the main document, e.g.:
%
\begin{center}
\begin{tabular}{l}
|\ifchilddocmanual|\\
|\input{\childdocname}|\\
|\||else|\\
\textit{document body with }|\input{|\textit{part}|}|\\
|\||fi|
\end{tabular}
\end{center}
%
The conditional |\ifchilddocmanual| is true whenever
a part to be included by |\input| is being compiled,
and the name of the part is stored in |\childdocname|.

%%%%%%%%%%%%%%%%%%%%%%%%%%%%%%%%%%%%%%%%
\DescribeMacro{\childdocby}
Each part to be included by |\input| should start with:
%
\begin{center}
\begin{tabular}{l}
|\input{childdoc.def}|\\
|\childdocby{|\textit{main}|}|\\
\end{tabular}
\end{center}
%
The directive |\childdocby| is similar to |\childdocof|
described in \secref{sec:include},
but the subsequent selection of content must be done manually.
To that end, both |\ifchilddoc| and |\ifchilddocmanual|
will be true upon processing of a part,
and the name of the part is stored in |\childdocname|.
Note that |\jobname| will be set to the filename of the current part
so that each part receives an individual |.aux| file
that does not interfere with the |.aux| file(s) of the main document.
This behaviour can be altered by the alternative form
|\childdocby[*]{|\textit{main}|}| (with a non-empty optional argument)
which uses the |.aux| file of the main document
by setting |\jobname| to \textit{main}.

%%%%%%%%%%%%%%%%%%%%%%%%%%%%%%%%%%%%%%%%%%%%%%%%%%%%%%%%%%%%%%%%%%%%%%%%%%%%%%%%
\subsection{Driver Development}
\label{sec:driver}

The \textsf{childdoc} mechanism can also be use for the development
of definition files such as \LaTeX{} styles or classes.
This case differs from the above setup with multiple parts
included by |\include| in that no |\includeonly| should be invoked.
This can be achieved by starting the include file
(before |\ProvidesPackage|) with:
%
\begin{center}
\begin{tabular}{l}
|\input{childdoc.def}|\\
|\childdocforward{|\textit{main}|}|\\
\end{tabular}
\end{center}
%
or alternatively with:
%
\begin{center}
\begin{tabular}{l}
|\input{childdoc.def}|\\
|\childdocby{|\textit{main}|}|\\
\end{tabular}
\end{center}
%
Both forms have slightly different effects as described above.
The main file is prepared as usual, see \secref{sec:include}.

%%%%%%%%%%%%%%%%%%%%%%%%%%%%%%%%%%%%%%%%%%%%%%%%%%%%%%%%%%%%%%%%%%%%%%%%%%%%%%%%
\subsection{Legacy Detection}
\label{sec:detection}

The directive |\childdocmain| in the main file can detect
whether the complete document or merely a child is to be compiled
even without using the directive |\childdocof|.
This method is deprecated because it is less robust
and there is no compelling reason to use it;
it is merely provided for backward compatibility
and it may be removed in future versions.

If the detection mechanism is to be used,
it is mandatory to correctly specify
the filename of the main file as the argument of |\childdocmain|:
%
\begin{center}
\begin{tabular}{l}
|\input{childdoc.def}|\\
|\childdocmain{|\textit{main}|}|\\
\end{tabular}
\end{center}
%
If |\jobname| does not match the argument \textit{main} of |\childdocmain|,
it is assumed that |\jobname| points to the child file to be compiled.
When using |\childdocmain| with the main file specified as argument,
it suffices to start a child file
with just |\input{|\textit{main}|}|
without loading of the package and using |\childdocof|.
If instead all processing is done
with the appropriate \textsf{childdoc} directives,
the argument of \textit{main} of |\childdocmain| can be empty.

An alternative version of the command line processing described
in \secref{sec:commandline} using the detection mechanism reads:
%
\begin{center}
|... -jobname "|\textit{target}|" "|[\textit{flags}]%
[|\def\jobname{|\textit{dest}|}|]|\input{|\textit{main}|}"|
\end{center}

%%%%%%%%%%%%%%%%%%%%%%%%%%%%%%%%%%%%%%%%%%%%%%%%%%%%%%%%%%%%%%%%%%%%%%%%%%%%%%%%
\subsection{Manual Code}
\label{sec:manual}

In case one cannot be certain whether the definitions file |childdoc.def|
is installed on the target \TeX{} distribution
and one prefers not to ship it,
it is conceivable to paste a few relevant commands into the sources.

To that end, drop all statements |\input{childdoc.def}|
and perform the replacements as outlined below.
Instead of |\childdocmain{|\textit{main}|}| add the following code
to the top of the main file:
%
\begin{center}
\begin{tabular}{l}
|\||ifdefined\childdocname\endinput\||fi\newif\ifchilddoc|\\
|\edef\childdocname{\scantokens\expandafter{\jobname\noexpand}}|\\
|\def\childdocmain{|\textit{main}|}\||ifx\childdocmain\childdocname\||else|\\
|\childdoctrue\includeonly{\childdocname}\let\jobname\childdocmain\||fi|\\
\end{tabular}
\end{center}
%
Instead of |\childdocof{|\textit{main}|}| just include the main file
at the top of each child file:
%
\begin{center}
|\input{|\textit{main}|}|
\end{center}
%
A simple redirection |\childdocforward{|\textit{dest}|}| is achieved by:
%
\begin{center}
|\def\jobname{|\textit{dest}|}\input{\jobname}|
\end{center}
%
The redirection with prefix
|\childdocforwardprefix[|\textit{prefix}|]{|\textit{dest}|}|
is accomplished by:
%
\begin{center}
\begin{tabular}{l}
|{\edef\jobname{\scantokens\expandafter{\jobname\noexpand}}|\\
|\def\redirectjob |\textit{prefix}|#1~~~{\gdef\jobname{|\textit{dest}|#1}}|\\
|\expandafter\redirectjob\jobname~~~}\input{\jobname}|
\end{tabular}
\end{center}

In an alternative approach,
child documents can be compiled by a specific command line
without additional code or specific definitions:
%
\begin{center}
|... -jobname "|\textit{target}|" "|[\textit{flags}]%
|\includeonly{|\textit{dest}|}\input{|\textit{main}|}"|
\end{center}
%

%%%%%%%%%%%%%%%%%%%%%%%%%%%%%%%%%%%%%%%%%%%%%%%%%%%%%%%%%%%%%%%%%%%%%%%%%%%%%%%%
%%%%%%%%%%%%%%%%%%%%%%%%%%%%%%%%%%%%%%%%%%%%%%%%%%%%%%%%%%%%%%%%%%%%%%%%%%%%%%%%
\section{Information}

%%%%%%%%%%%%%%%%%%%%%%%%%%%%%%%%%%%%%%%%%%%%%%%%%%%%%%%%%%%%%%%%%%%%%%%%%%%%%%%%
\subsection{Copyright}

Copyright \copyright{} 2017--2018 Niklas Beisert

This work may be distributed and/or modified under the
conditions of the \LaTeX{} Project Public License, either version 1.3
of this license or (at your option) any later version.
The latest version of this license is in
  \url{http://www.latex-project.org/lppl.txt}
and version 1.3 or later is part of all distributions of \LaTeX{}
version 2005/12/01 or later.

This work has the LPPL maintenance status `maintained'.

The Current Maintainer of this work is Niklas Beisert.

This work consists of the files |README.txt|, |childdoc.ins| and |childdoc.dtx|
as well as the derived files |childdoc.def|, |cdocsamp.tex|
with |cdocsch1.tex|, |cdocsch2.tex|, |cdocspt3.tex|, |cdocspt4.tex|,
|cdocsdrf.tex|, |cdocsfn1.tex|, |cdocsfn2.tex|
as well as |childdoc.pdf|.

%%%%%%%%%%%%%%%%%%%%%%%%%%%%%%%%%%%%%%%%%%%%%%%%%%%%%%%%%%%%%%%%%%%%%%%%%%%%%%%%
\subsection{Files and Installation}

The package consists of the files:
%
\begin{center}
\begin{tabular}{ll}
    |README.txt|   & readme file \\
    |childdoc.ins| & installation file \\
    |childdoc.dtx| & source file \\
    |childdoc.def| & definition file \\
    |cdocsamp.tex| & sample main file \\
    |cdocsch1.tex| & sample include file \\
    |cdocsch2.tex| & sample include file \\
    |cdocspt3.tex| & sample part file \\
    |cdocspt4.tex| & sample part file \\
    |cdocsdrf.tex| & sample redirection file \\
    |cdocsfn1.tex| & sample redirection file \\
    |cdocsfn2.tex| & sample redirection file \\
    |childdoc.pdf| & manual
\end{tabular}
\end{center}
%
The distribution consists of the files
|README.txt|, |childdoc.ins| and |childdoc.dtx|.
%
\begin{itemize}
\item
Run (pdf)\LaTeX{} on |childdoc.dtx|
to compile the manual |childdoc.pdf| (this file).
\item
Run \LaTeX{} on |childdoc.ins| to create the definitions file |childdoc.def|
and the sample |cdocsamp.tex| with include files
|cdocsch1.tex|, |cdocsch2.tex|, |cdocspt3.tex|, |cdocspt4.tex|,
|cdocsdrf.tex|, |cdocsfn1.tex|, |cdocsfn2.tex|.
Then copy the file |childdoc.def| to an appropriate directory of your \LaTeX{}
distribution, e.g.\ \textit{texmf-root}|/tex/latex/childdoc|.
\end{itemize}

%%%%%%%%%%%%%%%%%%%%%%%%%%%%%%%%%%%%%%%%%%%%%%%%%%%%%%%%%%%%%%%%%%%%%%%%%%%%%%%%
\subsection{Related CTAN Packages}

There are several other packages which offer a similar functionality:
%
\begin{itemize}
\item
The packages
\href{http://ctan.org/pkg/docmute}{\textsf{docmute}},
\href{http://ctan.org/pkg/includex}{\textsf{includex}} and
\href{http://ctan.org/pkg/standalone}{\textsf{standalone}}
provide commands to include only the document body of
a child file thus allowing both files to be compiled individually.
\item
The packages \href{http://ctan.org/pkg/subdocs}{\textsf{subdocs}}
and \href{http://ctan.org/pkg/subfiles}{\textsf{subfiles}}
provide structures in which the main and child documents can be
encapsulated and allowing them to be compiled individually.
The inclusion mechanism is different from the conventional |\include|.
\item
The package \href{http://ctan.org/pkg/combine}{\textsf{combine}}
is an elaborate solution to combine several documents into one.
\end{itemize}
%
See also the CTAN topic \href{http://ctan.org/topic/subdocs}{\textsf{subdocs}}
for further related packages.
The present package differs from the above solutions in that
a document structure constructed with the conventional |\include| mechanism
just needs two extra commands at the top of every file
such that all constituent files can be compiled individually.

%%%%%%%%%%%%%%%%%%%%%%%%%%%%%%%%%%%%%%%%%%%%%%%%%%%%%%%%%%%%%%%%%%%%%%%%%%%%%%%%
%\subsection{Feature Suggestions}
%
%The following is a list of features which may be useful for future
%versions of this package:
%%
%\begin{itemize}
%\item
%\ldots
%\end{itemize}

%%%%%%%%%%%%%%%%%%%%%%%%%%%%%%%%%%%%%%%%%%%%%%%%%%%%%%%%%%%%%%%%%%%%%%%%%%%%%%%%
\subsection{Revision History}

%%%%%%%%%%%%%%%%%%%%%%%%%%%%%%%%%%%%%%%%
\paragraph{v2.0:} 2018/12/30

\begin{itemize}
\item
immediate forward processing
\item
added |\childdocby| mechanism
\item
manual restructured
\end{itemize}

%%%%%%%%%%%%%%%%%%%%%%%%%%%%%%%%%%%%%%%%
\paragraph{v1.6:} 2018/01/17

\begin{itemize}
\item
application for development of include files
\item
corrections to manual
\end{itemize}

%%%%%%%%%%%%%%%%%%%%%%%%%%%%%%%%%%%%%%%%
\paragraph{v1.5:} 2017/05/21

\begin{itemize}
\item
more complete structuring introduced
\item
|\childdocof| introduced
\item
|\childdoc| renamed to |\childdocmain|
\item
|\childredirect| renamed to |\childdocforward| and |\childdocforwardprefix|
and functionality expanded
\end{itemize}

%%%%%%%%%%%%%%%%%%%%%%%%%%%%%%%%%%%%%%%%
\paragraph{v1.0:} 2017/04/27

\begin{itemize}
\item
manual and install package
\item
first version published on CTAN
\end{itemize}

%%%%%%%%%%%%%%%%%%%%%%%%%%%%%%%%%%%%%%%%
\paragraph{v0.6:} 2017/04/26

\begin{itemize}
\item
redirection mechanism added
\end{itemize}

%%%%%%%%%%%%%%%%%%%%%%%%%%%%%%%%%%%%%%%%
\paragraph{v0.5:} 2017/04/26

\begin{itemize}
\item
functionality in definition file
\end{itemize}


%%%%%%%%%%%%%%%%%%%%%%%%%%%%%%%%%%%%%%%%%%%%%%%%%%%%%%%%%%%%%%%%%%%%%%%%%%%%%%%%
%%%%%%%%%%%%%%%%%%%%%%%%%%%%%%%%%%%%%%%%%%%%%%%%%%%%%%%%%%%%%%%%%%%%%%%%%%%%%%%%
%%%%%%%%%%%%%%%%%%%%%%%%%%%%%%%%%%%%%%%%%%%%%%%%%%%%%%%%%%%%%%%%%%%%%%%%%%%%%%%%
\appendix

\settowidth\MacroIndent{\rmfamily\scriptsize 000\ }

 \DocInput{childdoc.dtx}

\end{document}
%</driver>
% \fi
%
% %%%%%%%%%%%%%%%%%%%%%%%%%%%%%%%%%%%%%%%%%%%%%%%%%%%%%%%%%%%%%%%%%%%%%%%%%%%%%%
% %%%%%%%%%%%%%%%%%%%%%%%%%%%%%%%%%%%%%%%%%%%%%%%%%%%%%%%%%%%%%%%%%%%%%%%%%%%%%%
% \section{Sample}
%\iffalse
%<*samplemain>
%\fi
%
% The following presents a sample document
% with two chapters, two parts, a title page,
% a compile flag as well as three forwarding files to set the flag.
% It consists of eight |.tex| files:
% \begin{center}
% \begin{tabular}{ll}
% |cdocsamp.tex|&main file\\
% |cdocsch1.tex|&include file for chapter 1\\
% |cdocsch2.tex|&include file for chapter 2\\
% |cdocspt3.tex|&include file for part 3\\
% |cdocspt4.tex|&include file for part 4\\
% |cdocsdrf.tex|&forwarding file for main file in draft mode\\
% |cdocsfi1.tex|&forwarding file for final version of chapter 1\\
% |cdocsfi2.tex|&forwarding file for final version of chapter 2\\
% \end{tabular}
% \end{center}
% Each of the eight files can be compiled directly by the \LaTeX{} compiler.
%
% %%%%%%%%%%%%%%%%%%%%%%%%%%%%%%%%%%%%%%
% \paragraph{Main File.}
%
% The main file is called |cdocsamp.tex|.
%
% Load the \textsf{childdoc} definitions and
% declare the filename for the main document:
%    \begin{macrocode}
\input{childdoc.def}
\childdocmain{}
%    \end{macrocode}

% Optional override for |\version| flag:
%    \begin{macrocode}
%%\ifchilddoc\else\providecommand{\version}{draft}\fi
%    \end{macrocode}

% Define the default values for the |\version| flag
% (|final| for the main file and |draft| for childs):
%    \begin{macrocode}
\ifchilddoc
\providecommand{\version}{draft}
\else
\providecommand{\version}{final}
\fi
%    \end{macrocode}

% Load the standard document class:
%    \begin{macrocode}
\documentclass[12pt]{article}
%    \end{macrocode}

% Start the document body:
%    \begin{macrocode}
\begin{document}
%    \end{macrocode}

% Declare a title page.
% Print title, part of document being processed and version flag:
%    \begin{macrocode}
\addtocounter{page}{-1}
\begin{center}
{\LARGE\bfseries{}childdoc example\par}
\vspace{1cm}
\ifchilddoc
\ifchilddocmanual part\else chapter\fi:
`\childdocname' of `\childdocjob'\par
\else
main document: `\childdocjob'\par
\fi
version: \version\par
\end{center}
\newpage
%    \end{macrocode}

% Manually include selected file,
% otherwise process as usual:
%    \begin{macrocode}
\ifchilddocmanual
\section*{part `\childdocname'}
\input{\childdocname}
\else
%    \end{macrocode}

% Include the two chapters:
%    \begin{macrocode}
\include{cdocsch1}
\include{cdocsch2}
%    \end{macrocode}

% Include the two parts unless only chapters should be displayed:
%    \begin{macrocode}
\ifchilddoc\else
\section{part three}
\input{cdocspt3}
\section{part four}
\input{cdocspt4}
\fi
%    \end{macrocode}

% Process as usual until here:
%    \begin{macrocode}
\fi
%    \end{macrocode}

% End of document body:
%    \begin{macrocode}
\end{document}
%    \end{macrocode}
%\iffalse
%</samplemain>
%\fi
%
% %%%%%%%%%%%%%%%%%%%%%%%%%%%%%%%%%%%%%%
% \paragraph{Chapter Include Files.}
%
% The include files are called |cdocsch1.tex| and |cdocsch2.tex|.
%
%\iffalse
%<*samplechap1|samplechap2>
%\fi

% Optional override for |\version| flag:
%    \begin{macrocode}
%%\providecommand{\version}{final}
%    \end{macrocode}

% Include the main document:
%    \begin{macrocode}
\input{childdoc.def}
\childdocof{cdocsamp}
%    \end{macrocode}

%\iffalse
%</samplechap1|samplechap2>
%\fi
%
%\iffalse
%<*samplechap1>
%\fi
% Some text for chapter 1:
%    \begin{macrocode}
\section{one}
some text in chapter one
%    \end{macrocode}

%\iffalse
%</samplechap1>
%\fi
% Some text for chapter 2:
%\iffalse
%<*samplechap2>
%\fi
%    \begin{macrocode}
\section{two}
more text in chapter two
%    \end{macrocode}

%\iffalse
%</samplechap2>
%\fi
%
% %%%%%%%%%%%%%%%%%%%%%%%%%%%%%%%%%%%%%%
% \paragraph{Part Include Files.}
%
% The include files are called |cdocspt3.tex| and |cdocspt4.tex|.
%
%\iffalse
%<*samplepart3|samplepart4>
%\fi

% Optional override for |\version| flag:
%    \begin{macrocode}
%%\providecommand{\version}{final}
%    \end{macrocode}

% Include the main document:
%    \begin{macrocode}
\input{childdoc.def}
\childdocby{cdocsamp}
%    \end{macrocode}

%\iffalse
%</samplepart3|samplepart4>
%\fi
%
%\iffalse
%<*samplepart3>
%\fi
% Some text for part 3:
%    \begin{macrocode}
some text in part three
%    \end{macrocode}

%\iffalse
%</samplepart3>
%\fi
% Some text for part 4:
%\iffalse
%<*samplepart4>
%\fi
%    \begin{macrocode}
more text in part four
%    \end{macrocode}

%\iffalse
%</samplepart4>
%\fi
%
% %%%%%%%%%%%%%%%%%%%%%%%%%%%%%%%%%%%%%%
% \paragraph{Forwarding for a Complete Draft.}
%
% The following forwarding file |cdocsdrf.tex|
% compiles the main document in draft mode:
%\iffalse
%<*sampledraft>
%\fi
%    \begin{macrocode}
\def\version{draft}
\input{childdoc.def}
\childdocforward{cdocsamp}
%    \end{macrocode}

%\iffalse
%</sampledraft>
%\fi
%
% %%%%%%%%%%%%%%%%%%%%%%%%%%%%%%%%%%%%%%
% \paragraph{Forwarding for Final Version of the Chapters.}
%
% The following forwarding files |cdocsfn1.tex| and |cdocsfn2.tex|
% (with identical content)
% compile the final versions of the child documents
% |cdocsch1.tex| and |cdocsch2.tex|, respectively:
%\iffalse
%<*samplefinal>
%\fi
%    \begin{macrocode}
\def\version{final}
\input{childdoc.def}
\childdocforwardprefix[cdocsamp]{cdocsfn}{cdocsch}
%    \end{macrocode}

%\iffalse
%</samplefinal>
%\fi
%
% %%%%%%%%%%%%%%%%%%%%%%%%%%%%%%%%%%%%%%
% \paragraph{Command Line Processing.}
%
% The following three command lines generate the output files
% |cdocscld|, |cdocscl1| and |cdocscl2|
% which should be identical to
% |cdocsdrf|, |cdocsch1| and |cdocsfn2|, respectively:
% \begin{center}
% \begin{tabular}{l}
% |latex -jobname cdocscld \|\\
% |  "\def\version{draft}\input{childdoc.def}\childdocforward{cdocsamp}"|\\
% |latex -jobname cdocscl1 \|\\
% |  "\input{childdoc.def}\childdocforward[cdocsamp]{cdocsch1}"|\\
% |latex -jobname cdocscl2 \|\\
% |  "\def\version{final}\input{childdoc.def}\childdocforward{cdocsch2}"|
% \end{tabular}
% \end{center}
% Note that the trailing backslash on each first line
% merely continues the input to the second line
% (for convenient cut ant paste).
% Furthermore, the command |latex| can be replaced by any
% of its alternative versions such as |pdflatex|.
%
% %%%%%%%%%%%%%%%%%%%%%%%%%%%%%%%%%%%%%%%%%%%%%%%%%%%%%%%%%%%%%%%%%%%%%%%%%%%%%%
% %%%%%%%%%%%%%%%%%%%%%%%%%%%%%%%%%%%%%%%%%%%%%%%%%%%%%%%%%%%%%%%%%%%%%%%%%%%%%%
% \section{Implementation}
%\iffalse
%<*package>
%\fi
%
% This section describes the definitions file |childdoc.def|.

% The definitions cannot be loaded using |\usepackage| or |\RequirePackage|
% which has a mechanism to prevent loading a style file more than once.
% When loading the definitions by means of |\input|
% multiple instances have to be prevented manually:
%\iffalse
%This code needs to be before the `\ProvidesFile' directive
%which is defined at the beginning of this file.
%Therefore it is also placed there and commented out here.
%</package>
%<*discard>
%\fi
%    \begin{macrocode}
\ifdefined\childdocmain\endinput\fi
%    \end{macrocode}
%\iffalse
%</discard>
%<*package>
%\fi
%
% \macro{\ifchilddoc}
% \macro{\ifchilddocmanual}
% The conditional |\ifchilddoc| tells whether a
% child (true) or main (false) document is being compiled.
% The conditional |\ifchilddocmanual| tells whether
% the |\includeonly| mechanism is used (false) or
% the selection of child files must be performed manually (true).
% The definitions initialise to false:
%    \begin{macrocode}
\newif\ifchilddoc
\newif\ifchilddocmanual
%    \end{macrocode}

% \macro{\childdocname}
% \macro{\childdocjob}
% The macro |\childdocname| stores the name of the main document
% to be compiled. The macro |\childdocjob| stores the name of
% the document on which the \LaTeX{} compiler was originally invoked.
% The content of |\jobname| cannot be compared
% to filenames specified in the source due to different catcodes.
% The following code rescans |\jobname|, stores the result
% in |\childdocname| and saves a copy in |\childdocjob|:
%    \begin{macrocode}
\edef\childdocname{\scantokens\expandafter{\jobname\noexpand}}
\let\childdocjob\childdocname
%    \end{macrocode}

% \macro{\childdocdisable}
% The macro |\childdocdisable| prevents the main file
% from being processed more than once.
% At this stage, the main document command |\childdocmain|
% is assumed to be called once again where it should do nothing.
% Any subsequent call to it should prevent
% a secondary processing of the main document
% It overwrites the forwarding commands
% |\childdocof| and |\childdocforward|
% with empty macros to prevent further inclusions of the main document:
%    \begin{macrocode}
\newcommand{\childdocdisable}
{
  \renewcommand{\childdocmain}[1]{\renewcommand{\childdocmain}[1]{\endinput}}
  \renewcommand{\childdocof}[1]{}
  \renewcommand{\childdocby}[2][]{}
  \renewcommand{\childdocforward}[2][]{}
  \renewcommand{\childdocdisable}{}
}
%    \end{macrocode}

% \macro{\childdocmain}
% The macro |\childdocmain| is to be called at the top of the main file
% with nothing or the main filename (without extension) as argument.
% First, it breaks loops.
% If the argument is not empty and does not match |\childdocname|
% (which is set by the first inclusion of |childdoc.def|),
% |\ifchilddoc| is set to true, |\includeonly| is applied to the child file
% and |\jobname| is set to the main file
% (for proper handling of |.aux| files):
%    \begin{macrocode}
\newcommand{\childdocmain}[1]
{
  \childdocdisable\childdocmain{}
  \if?#1?\else
    \begingroup
      \def\childdoctmp{#1}
      \ifx\childdoctmp\childdocname
        \def\childdoctmp{}
      \else
        \def\childdoctmp
        {
          \childdoctrue
          \includeonly{\childdocname}
          \def\childdocjob{#1}
          \def\jobname{#1}
        }
      \fi
      \expandafter
    \endgroup
    \childdoctmp
  \fi
}
%    \end{macrocode}

% \macro{\childdocof}
% The command |\childdocof| redirects
% compilation to the main file |#1|.
%    \begin{macrocode}
\newcommand{\childdocof}[1]
{
  \childdocdisable
  \childdoctrue
  \includeonly{\childdocname}
  \def\jobname{#1}
  \def\childdocjob{#1}
  \input{#1}
}
%    \end{macrocode}

% \macro{\childdocby}
% The command |\childdocby| ....
%    \begin{macrocode}
\newcommand{\childdocby}[2][]
{
  \childdocdisable
  \childdoctrue
  \childdocmanualtrue
  \if?#1?\else
    \def\jobname{#2}
  \fi
  \def\childdocjob{#2}
  \input{#2}
  \endinput
}
%    \end{macrocode}

% \macro{\childdocforward}
% The command |\childdocforward| redirects
% compilation to the main file or
% (if the optional argument is given) a child file.
% Parameters are set as if the main file
% or a child file starting with |\childdocof| was compiled.
% Then compilation is handed over to the main file:
%    \begin{macrocode}
\newcommand{\childdocforward}[2][]
{
  \begingroup
    \if?#1?
      \def\childdoctmp
      {
        \def\childdocname{#2}
        \def\childdocjob{#2}
        \def\jobname{#2}
        \input{#2}
        \endinput
      }
    \else
      \def\childdoctmp
      {
        \childdocdisable
        \def\childdocname{#2}
        \childdoctrue
        \includeonly{#2}
        \def\childdocjob{#1}
        \def\jobname{#1}
        \input{#1}
        \endinput
      }
    \fi
    \expandafter
  \endgroup
  \childdoctmp
}
%    \end{macrocode}

% \macro{\childdocforwardprefix}
% The command |\childdocforwardprefix| redirects
% compilation to the main or a child file by means of a pattern.
% The prefix |#1| in the current filename is replaced by |#2|
% and the suffix of the current filename is kept
% (it is assumed that the filename does not contain the substring `|~~~|'
% which is used as a delimiter).
% Compilation is handed over to the new file by |\childdocforward|:
%    \begin{macrocode}
\newcommand{\childdocforwardprefix}[3][]
{
  \begingroup
    \def\childdocextract #2##1~~~{\def\childdoctmp{\childdocforward[#1]{#3##1}}}
    \expandafter\childdocextract\childdocname~~~
    \expandafter
  \endgroup
  \childdoctmp
}
%    \end{macrocode}

% \macro{\childdoc}
% The deprecated macro |\childdoc| is a legacy version of |\childdocmain|:
%    \begin{macrocode}
\newcommand{\childdoc}{\childdocmain}
%    \end{macrocode}

% \macro{\childdocredirect}
% The deprecated macro |\childdocredirect| is a legacy version
% of |\childdocforward| and |\childdocforwardprefix|:
%    \begin{macrocode}
\newcommand{\childdocredirect}[2][]
{
  \begingroup
    \if?#1?
      \def\childdoctmp{\childdocforward{#2}}
    \else
      \def\childdoctmp{\childdocforwardprefix{#1}{#2}}
    \fi
    \expandafter
  \endgroup
  \childdoctmp
}
%    \end{macrocode}

%\iffalse
%</package>
%\fi
%
\endinput
|\\
|\childdocby{|\textit{main}|}|\\
\end{tabular}
\end{center}
%
Both forms have slightly different effects as described above.
The main file is prepared as usual, see \secref{sec:include}.

%%%%%%%%%%%%%%%%%%%%%%%%%%%%%%%%%%%%%%%%%%%%%%%%%%%%%%%%%%%%%%%%%%%%%%%%%%%%%%%%
\subsection{Legacy Detection}
\label{sec:detection}

The directive |\childdocmain| in the main file can detect
whether the complete document or merely a child is to be compiled
even without using the directive |\childdocof|.
This method is deprecated because it is less robust
and there is no compelling reason to use it;
it is merely provided for backward compatibility
and it may be removed in future versions.

If the detection mechanism is to be used,
it is mandatory to correctly specify
the filename of the main file as the argument of |\childdocmain|:
%
\begin{center}
\begin{tabular}{l}
|% \iffalse
%
% childdoc.dtx Copyright (C) 2017-2018 Niklas Beisert
%
% This work may be distributed and/or modified under the
% conditions of the LaTeX Project Public License, either version 1.3
% of this license or (at your option) any later version.
% The latest version of this license is in
%   http://www.latex-project.org/lppl.txt
% and version 1.3 or later is part of all distributions of LaTeX
% version 2005/12/01 or later.
%
% This work has the LPPL maintenance status `maintained'.
%
% The Current Maintainer of this work is Niklas Beisert.
%
% This work consists of the files childdoc.dtx and childdoc.ins
% and the derived files childdoc.def and cdocsamp.tex with
% cdocsch1.tex, cdocsch2.tex, cdocsdrf.tex, cdocsfn1.tex, cdocsfn2.tex.
%
%<package>\ifdefined\childdocmain\endinput\fi
%<package>\ProvidesFile{childdoc.def}[2018/12/30 v2.0 child document driver]
%<samplemain>\ProvidesFile{cdocsamp.tex}[2018/12/30 v2.0 sample for childdoc]
%<*driver>
%\ProvidesFile{childdoc.drv}[2018/12/30 v2.0 childdoc reference manual file]
\PassOptionsToClass{10pt,a4paper}{article}
\documentclass{ltxdoc}

\usepackage[margin=35mm]{geometry}
\usepackage{hyperref}
\usepackage{hyperxmp}
\usepackage[usenames]{color}

\hypersetup{colorlinks=true}
\hypersetup{pdfstartview=FitH}
\hypersetup{pdfpagemode=UseNone}
\hypersetup{pdfsource={}}
\hypersetup{pdflang={en-UK}}
\hypersetup{pdfcopyright={Copyright 2017-2018 Niklas Beisert.
  This work may be distributed and/or modified under the
  conditions of the LaTeX Project Public License, either version 1.3
  of this license or (at your option) any later version.}}
\hypersetup{pdflicenseurl={http://www.latex-project.org/lppl.txt}}
\hypersetup{pdfcontactaddress={ETH Zurich, ITP, HIT K,
  Wolfgang-Pauli-Strasse 27}}
\hypersetup{pdfcontactpostcode={8093}}
\hypersetup{pdfcontactcity={Zurich}}
\hypersetup{pdfcontactcountry={Switzerland}}
\hypersetup{pdfcontactemail={nbeisert@itp.phys.ethz.ch}}
\hypersetup{pdfcontacturl={http://people.phys.ethz.ch/\xmptilde nbeisert/}}

\newcommand{\secref}[1]{\hyperref[#1]{section \ref*{#1}}}

\parskip1ex
\parindent0pt
\let\olditemize\itemize
\def\itemize{\olditemize\parskip0pt}

\begin{document}

\title{The \textsf{childdoc} Package}
\hypersetup{pdftitle={The childdoc Package}}
\author{Niklas Beisert\\[2ex]
  Institut f\"ur Theoretische Physik\\
  Eidgen\"ossische Technische Hochschule Z\"urich\\
  Wolfgang-Pauli-Strasse 27, 8093 Z\"urich, Switzerland\\[1ex]
  \href{mailto:nbeisert@itp.phys.ethz.ch}
  {\texttt{nbeisert@itp.phys.ethz.ch}}}
\hypersetup{pdfauthor={Niklas Beisert}}
\hypersetup{pdfsubject={Manual for the LaTeX2e Package childdoc}}
\date{30 December 2018, \textsf{v2.0}}
\maketitle

\begin{abstract}\noindent
\textsf{childdoc} is a \LaTeXe{} package
that enables the direct compilation
of document sections included by |\include|
to individual files.
\end{abstract}

\begingroup
\parskip0ex
\tableofcontents
\endgroup

%%%%%%%%%%%%%%%%%%%%%%%%%%%%%%%%%%%%%%%%%%%%%%%%%%%%%%%%%%%%%%%%%%%%%%%%%%%%%%%%
%%%%%%%%%%%%%%%%%%%%%%%%%%%%%%%%%%%%%%%%%%%%%%%%%%%%%%%%%%%%%%%%%%%%%%%%%%%%%%%%
\section{Introduction}

\LaTeX{} provides a mechanism to structure a large document (such as a book)
into a main file and several child files (containing the chapters)
using the |\include| command.
This mechanism is beneficial for documents
which span hundreds of pages in order to
make the source file(s) more manageable.
Moreover, compilation can be restricted to
selected child files by means of the |\includeonly| command.
The latter feature can be used to reduce the compilation time while editing
(this was significantly more useful in the earlier days of \LaTeX{})
or to generate a smaller document which is easier to navigate.
Another application of |\includeonly| is to generate
documents consisting of selected parts of the complete document.

However, there are a few drawbacks of the plain |\include| mechanism:
\begin{itemize}
\item
The child files cannot be compiled on their own,
they can only be compiled via the main file.
A naive editing environment
(such as a text editor with an option
to have the current file processed by \LaTeX)
may require one to switch to the main file before compiling;
attempting to compile the child file produces errors.
\item
The main file must be modified (each time)
to adjust the |\includeonly| command
to the present needs. This easily leaves the main file in a messy state.
\item
The generated document will always carry the filename
of the main document. This is inconvenient if
several child files are to be compiled and
to be kept for distribution.
\end{itemize}

The present package provides a simple interface
to make child files individually compilable by \LaTeX{}.
Compiling a child file then has the same effect as compiling
the main file with an |\includeonly| command
to select the appropriate child.
Moreover the generated document will carry the name of the child
rather than the main file.
This resolves all three above issues.

This feature is meant to make the editing of books,
thesis documents and lecture notes somewhat more convenient.
However, the package can also be used efficiently for
composing a series of documents (such as exercise sheets)
which are typically distributed individually.
It then assists the author in generating the individual documents
(potentially in different versions)
as well as a document containing the collected series.
Another application is in developing style files
or other kinds of included material
where compilation of the style file could redirect
to a sample or test file.

%%%%%%%%%%%%%%%%%%%%%%%%%%%%%%%%%%%%%%%%%%%%%%%%%%%%%%%%%%%%%%%%%%%%%%%%%%%%%%%%
%%%%%%%%%%%%%%%%%%%%%%%%%%%%%%%%%%%%%%%%%%%%%%%%%%%%%%%%%%%%%%%%%%%%%%%%%%%%%%%%
\section{Usage}

First of all, the package \textsf{childdoc} is \emph{not} a standard
\LaTeXe{} |.sty| style file! Therefore it needs to be invoked in
a non-standard way.

%%%%%%%%%%%%%%%%%%%%%%%%%%%%%%%%%%%%%%%%%%%%%%%%%%%%%%%%%%%%%%%%%%%%%%%%%%%%%%%%
\subsection{Included Files}
\label{sec:include}

%%%%%%%%%%%%%%%%%%%%%%%%%%%%%%%%%%%%%%%%
\DescribeMacro{\childdocmain}
To use the package, add the commands
\begin{center}
\begin{tabular}{l}
|\input{childdoc.def}|\\
|\childdocmain{}|\\
\end{tabular}
\end{center}
at the very top of the main \LaTeX{} file,
in particular \emph{before} the |\documentclass| statement!
The argument of |\childdocmain| should be left empty
(but it must be present).

%%%%%%%%%%%%%%%%%%%%%%%%%%%%%%%%%%%%%%%%
\DescribeMacro{\childdocof}
Furthermore, add the commands
\begin{center}
\begin{tabular}{l}
|\input{childdoc.def}|\\
|\childdocof{|\textit{main}|}|\\
\end{tabular}
\end{center}
at the top of every child file \textit{child}
which is included by |\include{|\textit{child}|}|
from within the main file
(or at least for those files to be compiled individually).
The argument \textit{main} must be the filename of the main file.

There are a couple of
considerations in setting up the main and child documents:

%%%%%%%%%%%%%%%%%%%%%%%%%%%%%%%%%%%%%%%%
\paragraph{Restrictions.}

Please note the following restrictions:
\begin{itemize}
\item
|\childdocmain| must be called with one argument \textit{main}
to ensure compatibility with earlier version of the package.
It must either be empty (|\childdocmain{}|)
or precisely match the filename of the main file in which it is specified.
See \secref{sec:detection} for further information.
\item
The filename \textit{main} must be specified without the |.tex| extension.
\item
The filename \textit{main} is case sensitive
(even in case-insensitive file systems)
due to internal string comparison.
\item
The argument \textit{main} should be fully expanded, it cannot be a macro.
\item
Subdirectories and special characters should be avoided in filenames.
\item
The command |\childdocmain{|\textit{main}|}| must be followed by a whitespace.
It should not be followed immediately by another command
or by a comment mark `|%|'.
This is because the \TeX{} parser reads the token immediately following
the argument of |\childdocmain| and puts it
at the beginning of every child section;
however, a white\-space is ignored.
\end{itemize}

%%%%%%%%%%%%%%%%%%%%%%%%%%%%%%%%%%%%%%%%
\paragraph{Content of Main File.}

It is advisable to place all content in the child files included by |\include|.
Any output contained in the main file will appear in all child documents
unless suppressed manually;
it cannot be suppressed automatically by the |\includeonly| directive
and thus should normally be avoided.
A method to include some content in the main file
by means of conditional processing is described in \secref{sec:conditional}.

%%%%%%%%%%%%%%%%%%%%%%%%%%%%%%%%%%%%%%%%
\paragraph{Page Numbering.}

When only a part of the document is compiled,
the appropriate numbering of pages
(as well as other status parameters)
is determined from the |.aux| files.
The latter contain information from previous passes.
However this information needs to propagate through
all intermediate child documents.
Therefore the page numbering in child documents may well
be inconsistent until the complete document is compiled at least once.

A useful (if unconventional) way to always ensure a consistent
page numbering is to restart the numbering in each child document
and denote the pages by `\textit{child}|.|\textit{page}'
where \textit{child} represents the chapter/section number of the child file.
This can be achieved by the command
|\numberwithin{page}{|\textit{child}|}|
of the \textsf{amsmath} package
where \textit{child} can be |chapter| or |section|
depending on the chosen structuring.
Alternatively, one can modify the macro |\thepage| appropriately
and reset the counter |page| at the start of each child file.

%%%%%%%%%%%%%%%%%%%%%%%%%%%%%%%%%%%%%%%%%%%%%%%%%%%%%%%%%%%%%%%%%%%%%%%%%%%%%%%%
\subsection{Conditional Processing}
\label{sec:conditional}

The package provides a mechanism to compile different versions
of a document. To customise the versions further some conditional processing
can come in handy to distinguish which version is being compiled.
The package provides two macros to describe the compilation context:

%%%%%%%%%%%%%%%%%%%%%%%%%%%%%%%%%%%%%%%%
\DescribeMacro{\ifchilddoc}
The conditional |\ifchilddoc| distinguishes between the compilation of
child documents and the main document:
%
\begin{center}
|\ifchilddoc |\textit{child-code}| |[|\||else |\textit{main-code}]| \||fi|
\end{center}

%%%%%%%%%%%%%%%%%%%%%%%%%%%%%%%%%%%%%%%%
\DescribeMacro{\childdocname}
\DescribeMacro{\childdocjob}
The macro |\childdocname| contains the filename (without extension)
of the main or child file being processed.
Note that |\childdocjob| will always contain the name of the main file.

%%%%%%%%%%%%%%%%%%%%%%%%%%%%%%%%%%%%%%%%
\paragraph{Title Page.}

Conditional processing can be used to include a title or banner page
in the main document when proper precautions are taken.
Importantly, the code in the main file should ensure that the page counter
(as well as other status parameters which are stored in the |.aux| files)
takes the same value after the conditional processing.
Otherwise the page numbers may take divergent values
depending on which part is compiled.

For example, a title page could be declared by:
%
\begin{center}
\begin{tabular}{l}
|\ifchilddoc\||else|\\
|\addtocounter{page}{-1}|\\
\textit{code for title page}\\
|\newpage|\\
|\||fi|
\end{tabular}
\end{center}
%
A banner page for the child documents can be generated by:
%
\begin{center}
\begin{tabular}{l}
|\ifchilddoc|\\
|\addtocounter{page}{-1}|\\
\textit{code for banner page}\\
|\newpage|\\
|\||fi|
\end{tabular}
\end{center}
%
Here one could write a message such as:
\begin{center}
|This is the part \childdocname{} of \childdocjob{}.|
\end{center}

%%%%%%%%%%%%%%%%%%%%%%%%%%%%%%%%%%%%%%%%%%%%%%%%%%%%%%%%%%%%%%%%%%%%%%%%%%%%%%%%
\subsection{Flags}
\label{sec:flags}

The package makes it easy to generate different versions
of the main or child documents.
To this end compilation flags can be defined
and assigned different default values.
They will be particularly useful in conjunction
with the forwarding mechanism described in \secref{sec:forward}.

For example, it may be useful to have a flag |\version|
which can be set to |draft| or |final|.
The document source will contain some conditional code
depending on the value of |\version|.
Suppose further, the flag should default to |final| for the main file
and to |draft| for child files
which is a natural assignment for editing the document.
This is achieved by placing the following code
in the preamble of the main document
(below the |\childdocmain| directive):
%
\begin{center}
\begin{tabular}{l}
|\ifchilddoc|\\
|\providecommand{\version}{draft}|\\
|\||else|\\
|\providecommand{\version}{final}|\\
|\||fi|
\end{tabular}
\end{center}
%
The definition by |\providecommand| makes sure
that previous definitions are not overwritten.
Further statements |\providecommand{\version}{...}|
can thus be added before the above code to override it.

For the main file, one might add a line
(between |\childdocmain| and the above block)
%
\begin{center}
|%\ifchilddoc\||else\providecommand{\version}{draft}\||fi|
\end{center}
%
which can be uncommented to produce a draft version.
Likewise one can add a line to the very top of a child file
(above the |\childdocof{|\textit{main}|}| directive)
%
\begin{center}
|%\providecommand{\version}{final}|
\end{center}
%
which can be uncommented to produce the final version of this child document.

%%%%%%%%%%%%%%%%%%%%%%%%%%%%%%%%%%%%%%%%%%%%%%%%%%%%%%%%%%%%%%%%%%%%%%%%%%%%%%%%
\subsection{Forwarding}
\label{sec:forward}

Different versions of the main or child documents
using compilation flags as described in \secref{sec:flags}
can be (permanently) stored in different files
for convenient compilation, viewing and distribution.
To this end, the package defines a command
to pass on compilation to a different file:

%%%%%%%%%%%%%%%%%%%%%%%%%%%%%%%%%%%%%%%%
\DescribeMacro{\childdocforward}
The command |\childdocforward| redirects processing to
another source file:
%
\begin{center}
\begin{tabular}{l}
|\input{childdoc.def}|\\
|\childdocforward[|\textit{main}|]{|\textit{dest}|}|\\
\end{tabular}
\end{center}
%
The argument \textit{dest} is the destination file
(without extension).
It should be the main file or one of the child files.
Note that further \textsf{childdoc} directives
such as |\childdocof| and |\childdocforward|
in the indicated file will be processed in this form.
The optional argument \textit{main}
passes on directly to the main file \textit{main}
while pretending to compile the child \textit{dest}.
This form behaves as if \textit{dest}
issues |\childdocof{|\textit{main}|}| right away,
and no further \textsf{childdoc} directives will be processed.

%%%%%%%%%%%%%%%%%%%%%%%%%%%%%%%%%%%%%%%%
\DescribeMacro{\...prefix}
In the alternative form |\childdocforwardprefix|,
%
\begin{center}
\begin{tabular}{l}
|\input{childdoc.def}|\\
|\childdocforwardprefix[|\textit{main}|]{|\textit{prefix}|}{|\textit{dest}|}|
\end{tabular}
\end{center}
%
the destination file is determined by a pattern
depending on the current file:
To make this work, the current file must be called
`{\textit{prefix}\hspace{0.2em}\textit{suffix}}'
with \textit{prefix} matching precisely the argument.
Processing is then passed on to the file
`{\textit{dest}\hspace{0.2em}\textit{suffix}}'.
Surely, the same effect is achieved by
directly specifying the
argument `{\textit{dest}\hspace{0.2em}\textit{suffix}}'
in the first form.
However, that requires to set up a different file
for each child. With the alternative form of the command
all these files can have exactly the same content
which simplifies setting them up and maintaining them.

For example, the following file |draft.tex|
with a compilation flag |\version| as described in \secref{sec:flags}
compiles the main document as a draft:
%
\begin{center}
\begin{tabular}{l}
|\def\version{draft}|\\
|\input{childdoc.def}|\\
|\childdocforward{|\textit{main}|}|
\end{tabular}
\end{center}
%
Likewise, the following files |final|\textit{nn}|.tex|
compile the final version of the child document
|child|\textit{nn}|.tex|:
%
\begin{center}
\begin{tabular}{l}
|\def\version{final}|\\
|\input{childdoc.def}|\\
|\childdocforwardprefix{final}{child}|
\end{tabular}
\end{center}
%

Note that when several versions of a main file and/or of each child file
are to be generated, it may be convenient to set up a |Makefile| or
shell script to automatise the process.

%%%%%%%%%%%%%%%%%%%%%%%%%%%%%%%%%%%%%%%%%%%%%%%%%%%%%%%%%%%%%%%%%%%%%%%%%%%%%%%%
\subsection{Command Line Processing}
\label{sec:commandline}

The effect of redirection files can also be achieved by invoking
the \LaTeX{} compiler with a more elaborate command line.
Most conveniently this should be done as part
of a shell script or a |Makefile|.

When using \textsf{childdoc} in the main file, the following
command lines effectively perform a redirection
(note that depending on the shell being used,
backslashes may have to be doubled: `|\|' $\to$ `|\\|'):
%
\begin{center}
|... -jobname "|\textit{target}|" |\\|"|[\textit{flags}]%
|\input{childdoc.def}\childdocforward[|\textit{main}|]{|\textit{dest}|}"|
\end{center}
%
Here \textit{target} is the name of the output file,
\textit{main} is the name of the main file
and \textit{dest} is the name of the main or child file to be processed
(all filenames without extensions).
The optional argument \textit{main} can be omitted
if \textit{main} matches \textit{dest}.
Optionally, compilation \textit{flags} can be defined via |\def| commands.
This command line makes the \TeX{} engine believe
it is compiling the file \textit{target}
whose content is specified as the latter parameter.
The provided code then forwards the processing to
\textit{main} or \textit{dest} as described in \secref{sec:forward}.

%%%%%%%%%%%%%%%%%%%%%%%%%%%%%%%%%%%%%%%%%%%%%%%%%%%%%%%%%%%%%%%%%%%%%%%%%%%%%%%%
\subsection{Include by Input}
\label{sec:input}

Including child documents by |\include| has some restrictions by design.
Most notably, the content of a child document always occupies
its own set of pages; pages cannot be shared between child documents.
Usually, this behaviour makes perfect sense
because each child document contain an essential part of the document.
However, in some situations it may be desirable to compose
a document from a collection of parts
without having mandatory page breaks between then.
For this case, the package
provides a mechanism to include parts
by |\input| which can also be processed individually.
However, by construction this mechanism
requires manual handling of the content to be output.

%%%%%%%%%%%%%%%%%%%%%%%%%%%%%%%%%%%%%%%%
\DescribeMacro{\ifchilddocmanual}
The main file should be prepared as usual, see \secref{sec:include}.
However, the document body must make a distinction
between processing of an individual part and of the main document, e.g.:
%
\begin{center}
\begin{tabular}{l}
|\ifchilddocmanual|\\
|\input{\childdocname}|\\
|\||else|\\
\textit{document body with }|\input{|\textit{part}|}|\\
|\||fi|
\end{tabular}
\end{center}
%
The conditional |\ifchilddocmanual| is true whenever
a part to be included by |\input| is being compiled,
and the name of the part is stored in |\childdocname|.

%%%%%%%%%%%%%%%%%%%%%%%%%%%%%%%%%%%%%%%%
\DescribeMacro{\childdocby}
Each part to be included by |\input| should start with:
%
\begin{center}
\begin{tabular}{l}
|\input{childdoc.def}|\\
|\childdocby{|\textit{main}|}|\\
\end{tabular}
\end{center}
%
The directive |\childdocby| is similar to |\childdocof|
described in \secref{sec:include},
but the subsequent selection of content must be done manually.
To that end, both |\ifchilddoc| and |\ifchilddocmanual|
will be true upon processing of a part,
and the name of the part is stored in |\childdocname|.
Note that |\jobname| will be set to the filename of the current part
so that each part receives an individual |.aux| file
that does not interfere with the |.aux| file(s) of the main document.
This behaviour can be altered by the alternative form
|\childdocby[*]{|\textit{main}|}| (with a non-empty optional argument)
which uses the |.aux| file of the main document
by setting |\jobname| to \textit{main}.

%%%%%%%%%%%%%%%%%%%%%%%%%%%%%%%%%%%%%%%%%%%%%%%%%%%%%%%%%%%%%%%%%%%%%%%%%%%%%%%%
\subsection{Driver Development}
\label{sec:driver}

The \textsf{childdoc} mechanism can also be use for the development
of definition files such as \LaTeX{} styles or classes.
This case differs from the above setup with multiple parts
included by |\include| in that no |\includeonly| should be invoked.
This can be achieved by starting the include file
(before |\ProvidesPackage|) with:
%
\begin{center}
\begin{tabular}{l}
|\input{childdoc.def}|\\
|\childdocforward{|\textit{main}|}|\\
\end{tabular}
\end{center}
%
or alternatively with:
%
\begin{center}
\begin{tabular}{l}
|\input{childdoc.def}|\\
|\childdocby{|\textit{main}|}|\\
\end{tabular}
\end{center}
%
Both forms have slightly different effects as described above.
The main file is prepared as usual, see \secref{sec:include}.

%%%%%%%%%%%%%%%%%%%%%%%%%%%%%%%%%%%%%%%%%%%%%%%%%%%%%%%%%%%%%%%%%%%%%%%%%%%%%%%%
\subsection{Legacy Detection}
\label{sec:detection}

The directive |\childdocmain| in the main file can detect
whether the complete document or merely a child is to be compiled
even without using the directive |\childdocof|.
This method is deprecated because it is less robust
and there is no compelling reason to use it;
it is merely provided for backward compatibility
and it may be removed in future versions.

If the detection mechanism is to be used,
it is mandatory to correctly specify
the filename of the main file as the argument of |\childdocmain|:
%
\begin{center}
\begin{tabular}{l}
|\input{childdoc.def}|\\
|\childdocmain{|\textit{main}|}|\\
\end{tabular}
\end{center}
%
If |\jobname| does not match the argument \textit{main} of |\childdocmain|,
it is assumed that |\jobname| points to the child file to be compiled.
When using |\childdocmain| with the main file specified as argument,
it suffices to start a child file
with just |\input{|\textit{main}|}|
without loading of the package and using |\childdocof|.
If instead all processing is done
with the appropriate \textsf{childdoc} directives,
the argument of \textit{main} of |\childdocmain| can be empty.

An alternative version of the command line processing described
in \secref{sec:commandline} using the detection mechanism reads:
%
\begin{center}
|... -jobname "|\textit{target}|" "|[\textit{flags}]%
[|\def\jobname{|\textit{dest}|}|]|\input{|\textit{main}|}"|
\end{center}

%%%%%%%%%%%%%%%%%%%%%%%%%%%%%%%%%%%%%%%%%%%%%%%%%%%%%%%%%%%%%%%%%%%%%%%%%%%%%%%%
\subsection{Manual Code}
\label{sec:manual}

In case one cannot be certain whether the definitions file |childdoc.def|
is installed on the target \TeX{} distribution
and one prefers not to ship it,
it is conceivable to paste a few relevant commands into the sources.

To that end, drop all statements |\input{childdoc.def}|
and perform the replacements as outlined below.
Instead of |\childdocmain{|\textit{main}|}| add the following code
to the top of the main file:
%
\begin{center}
\begin{tabular}{l}
|\||ifdefined\childdocname\endinput\||fi\newif\ifchilddoc|\\
|\edef\childdocname{\scantokens\expandafter{\jobname\noexpand}}|\\
|\def\childdocmain{|\textit{main}|}\||ifx\childdocmain\childdocname\||else|\\
|\childdoctrue\includeonly{\childdocname}\let\jobname\childdocmain\||fi|\\
\end{tabular}
\end{center}
%
Instead of |\childdocof{|\textit{main}|}| just include the main file
at the top of each child file:
%
\begin{center}
|\input{|\textit{main}|}|
\end{center}
%
A simple redirection |\childdocforward{|\textit{dest}|}| is achieved by:
%
\begin{center}
|\def\jobname{|\textit{dest}|}\input{\jobname}|
\end{center}
%
The redirection with prefix
|\childdocforwardprefix[|\textit{prefix}|]{|\textit{dest}|}|
is accomplished by:
%
\begin{center}
\begin{tabular}{l}
|{\edef\jobname{\scantokens\expandafter{\jobname\noexpand}}|\\
|\def\redirectjob |\textit{prefix}|#1~~~{\gdef\jobname{|\textit{dest}|#1}}|\\
|\expandafter\redirectjob\jobname~~~}\input{\jobname}|
\end{tabular}
\end{center}

In an alternative approach,
child documents can be compiled by a specific command line
without additional code or specific definitions:
%
\begin{center}
|... -jobname "|\textit{target}|" "|[\textit{flags}]%
|\includeonly{|\textit{dest}|}\input{|\textit{main}|}"|
\end{center}
%

%%%%%%%%%%%%%%%%%%%%%%%%%%%%%%%%%%%%%%%%%%%%%%%%%%%%%%%%%%%%%%%%%%%%%%%%%%%%%%%%
%%%%%%%%%%%%%%%%%%%%%%%%%%%%%%%%%%%%%%%%%%%%%%%%%%%%%%%%%%%%%%%%%%%%%%%%%%%%%%%%
\section{Information}

%%%%%%%%%%%%%%%%%%%%%%%%%%%%%%%%%%%%%%%%%%%%%%%%%%%%%%%%%%%%%%%%%%%%%%%%%%%%%%%%
\subsection{Copyright}

Copyright \copyright{} 2017--2018 Niklas Beisert

This work may be distributed and/or modified under the
conditions of the \LaTeX{} Project Public License, either version 1.3
of this license or (at your option) any later version.
The latest version of this license is in
  \url{http://www.latex-project.org/lppl.txt}
and version 1.3 or later is part of all distributions of \LaTeX{}
version 2005/12/01 or later.

This work has the LPPL maintenance status `maintained'.

The Current Maintainer of this work is Niklas Beisert.

This work consists of the files |README.txt|, |childdoc.ins| and |childdoc.dtx|
as well as the derived files |childdoc.def|, |cdocsamp.tex|
with |cdocsch1.tex|, |cdocsch2.tex|, |cdocspt3.tex|, |cdocspt4.tex|,
|cdocsdrf.tex|, |cdocsfn1.tex|, |cdocsfn2.tex|
as well as |childdoc.pdf|.

%%%%%%%%%%%%%%%%%%%%%%%%%%%%%%%%%%%%%%%%%%%%%%%%%%%%%%%%%%%%%%%%%%%%%%%%%%%%%%%%
\subsection{Files and Installation}

The package consists of the files:
%
\begin{center}
\begin{tabular}{ll}
    |README.txt|   & readme file \\
    |childdoc.ins| & installation file \\
    |childdoc.dtx| & source file \\
    |childdoc.def| & definition file \\
    |cdocsamp.tex| & sample main file \\
    |cdocsch1.tex| & sample include file \\
    |cdocsch2.tex| & sample include file \\
    |cdocspt3.tex| & sample part file \\
    |cdocspt4.tex| & sample part file \\
    |cdocsdrf.tex| & sample redirection file \\
    |cdocsfn1.tex| & sample redirection file \\
    |cdocsfn2.tex| & sample redirection file \\
    |childdoc.pdf| & manual
\end{tabular}
\end{center}
%
The distribution consists of the files
|README.txt|, |childdoc.ins| and |childdoc.dtx|.
%
\begin{itemize}
\item
Run (pdf)\LaTeX{} on |childdoc.dtx|
to compile the manual |childdoc.pdf| (this file).
\item
Run \LaTeX{} on |childdoc.ins| to create the definitions file |childdoc.def|
and the sample |cdocsamp.tex| with include files
|cdocsch1.tex|, |cdocsch2.tex|, |cdocspt3.tex|, |cdocspt4.tex|,
|cdocsdrf.tex|, |cdocsfn1.tex|, |cdocsfn2.tex|.
Then copy the file |childdoc.def| to an appropriate directory of your \LaTeX{}
distribution, e.g.\ \textit{texmf-root}|/tex/latex/childdoc|.
\end{itemize}

%%%%%%%%%%%%%%%%%%%%%%%%%%%%%%%%%%%%%%%%%%%%%%%%%%%%%%%%%%%%%%%%%%%%%%%%%%%%%%%%
\subsection{Related CTAN Packages}

There are several other packages which offer a similar functionality:
%
\begin{itemize}
\item
The packages
\href{http://ctan.org/pkg/docmute}{\textsf{docmute}},
\href{http://ctan.org/pkg/includex}{\textsf{includex}} and
\href{http://ctan.org/pkg/standalone}{\textsf{standalone}}
provide commands to include only the document body of
a child file thus allowing both files to be compiled individually.
\item
The packages \href{http://ctan.org/pkg/subdocs}{\textsf{subdocs}}
and \href{http://ctan.org/pkg/subfiles}{\textsf{subfiles}}
provide structures in which the main and child documents can be
encapsulated and allowing them to be compiled individually.
The inclusion mechanism is different from the conventional |\include|.
\item
The package \href{http://ctan.org/pkg/combine}{\textsf{combine}}
is an elaborate solution to combine several documents into one.
\end{itemize}
%
See also the CTAN topic \href{http://ctan.org/topic/subdocs}{\textsf{subdocs}}
for further related packages.
The present package differs from the above solutions in that
a document structure constructed with the conventional |\include| mechanism
just needs two extra commands at the top of every file
such that all constituent files can be compiled individually.

%%%%%%%%%%%%%%%%%%%%%%%%%%%%%%%%%%%%%%%%%%%%%%%%%%%%%%%%%%%%%%%%%%%%%%%%%%%%%%%%
%\subsection{Feature Suggestions}
%
%The following is a list of features which may be useful for future
%versions of this package:
%%
%\begin{itemize}
%\item
%\ldots
%\end{itemize}

%%%%%%%%%%%%%%%%%%%%%%%%%%%%%%%%%%%%%%%%%%%%%%%%%%%%%%%%%%%%%%%%%%%%%%%%%%%%%%%%
\subsection{Revision History}

%%%%%%%%%%%%%%%%%%%%%%%%%%%%%%%%%%%%%%%%
\paragraph{v2.0:} 2018/12/30

\begin{itemize}
\item
immediate forward processing
\item
added |\childdocby| mechanism
\item
manual restructured
\end{itemize}

%%%%%%%%%%%%%%%%%%%%%%%%%%%%%%%%%%%%%%%%
\paragraph{v1.6:} 2018/01/17

\begin{itemize}
\item
application for development of include files
\item
corrections to manual
\end{itemize}

%%%%%%%%%%%%%%%%%%%%%%%%%%%%%%%%%%%%%%%%
\paragraph{v1.5:} 2017/05/21

\begin{itemize}
\item
more complete structuring introduced
\item
|\childdocof| introduced
\item
|\childdoc| renamed to |\childdocmain|
\item
|\childredirect| renamed to |\childdocforward| and |\childdocforwardprefix|
and functionality expanded
\end{itemize}

%%%%%%%%%%%%%%%%%%%%%%%%%%%%%%%%%%%%%%%%
\paragraph{v1.0:} 2017/04/27

\begin{itemize}
\item
manual and install package
\item
first version published on CTAN
\end{itemize}

%%%%%%%%%%%%%%%%%%%%%%%%%%%%%%%%%%%%%%%%
\paragraph{v0.6:} 2017/04/26

\begin{itemize}
\item
redirection mechanism added
\end{itemize}

%%%%%%%%%%%%%%%%%%%%%%%%%%%%%%%%%%%%%%%%
\paragraph{v0.5:} 2017/04/26

\begin{itemize}
\item
functionality in definition file
\end{itemize}


%%%%%%%%%%%%%%%%%%%%%%%%%%%%%%%%%%%%%%%%%%%%%%%%%%%%%%%%%%%%%%%%%%%%%%%%%%%%%%%%
%%%%%%%%%%%%%%%%%%%%%%%%%%%%%%%%%%%%%%%%%%%%%%%%%%%%%%%%%%%%%%%%%%%%%%%%%%%%%%%%
%%%%%%%%%%%%%%%%%%%%%%%%%%%%%%%%%%%%%%%%%%%%%%%%%%%%%%%%%%%%%%%%%%%%%%%%%%%%%%%%
\appendix

\settowidth\MacroIndent{\rmfamily\scriptsize 000\ }

 \DocInput{childdoc.dtx}

\end{document}
%</driver>
% \fi
%
% %%%%%%%%%%%%%%%%%%%%%%%%%%%%%%%%%%%%%%%%%%%%%%%%%%%%%%%%%%%%%%%%%%%%%%%%%%%%%%
% %%%%%%%%%%%%%%%%%%%%%%%%%%%%%%%%%%%%%%%%%%%%%%%%%%%%%%%%%%%%%%%%%%%%%%%%%%%%%%
% \section{Sample}
%\iffalse
%<*samplemain>
%\fi
%
% The following presents a sample document
% with two chapters, two parts, a title page,
% a compile flag as well as three forwarding files to set the flag.
% It consists of eight |.tex| files:
% \begin{center}
% \begin{tabular}{ll}
% |cdocsamp.tex|&main file\\
% |cdocsch1.tex|&include file for chapter 1\\
% |cdocsch2.tex|&include file for chapter 2\\
% |cdocspt3.tex|&include file for part 3\\
% |cdocspt4.tex|&include file for part 4\\
% |cdocsdrf.tex|&forwarding file for main file in draft mode\\
% |cdocsfi1.tex|&forwarding file for final version of chapter 1\\
% |cdocsfi2.tex|&forwarding file for final version of chapter 2\\
% \end{tabular}
% \end{center}
% Each of the eight files can be compiled directly by the \LaTeX{} compiler.
%
% %%%%%%%%%%%%%%%%%%%%%%%%%%%%%%%%%%%%%%
% \paragraph{Main File.}
%
% The main file is called |cdocsamp.tex|.
%
% Load the \textsf{childdoc} definitions and
% declare the filename for the main document:
%    \begin{macrocode}
\input{childdoc.def}
\childdocmain{}
%    \end{macrocode}

% Optional override for |\version| flag:
%    \begin{macrocode}
%%\ifchilddoc\else\providecommand{\version}{draft}\fi
%    \end{macrocode}

% Define the default values for the |\version| flag
% (|final| for the main file and |draft| for childs):
%    \begin{macrocode}
\ifchilddoc
\providecommand{\version}{draft}
\else
\providecommand{\version}{final}
\fi
%    \end{macrocode}

% Load the standard document class:
%    \begin{macrocode}
\documentclass[12pt]{article}
%    \end{macrocode}

% Start the document body:
%    \begin{macrocode}
\begin{document}
%    \end{macrocode}

% Declare a title page.
% Print title, part of document being processed and version flag:
%    \begin{macrocode}
\addtocounter{page}{-1}
\begin{center}
{\LARGE\bfseries{}childdoc example\par}
\vspace{1cm}
\ifchilddoc
\ifchilddocmanual part\else chapter\fi:
`\childdocname' of `\childdocjob'\par
\else
main document: `\childdocjob'\par
\fi
version: \version\par
\end{center}
\newpage
%    \end{macrocode}

% Manually include selected file,
% otherwise process as usual:
%    \begin{macrocode}
\ifchilddocmanual
\section*{part `\childdocname'}
\input{\childdocname}
\else
%    \end{macrocode}

% Include the two chapters:
%    \begin{macrocode}
\include{cdocsch1}
\include{cdocsch2}
%    \end{macrocode}

% Include the two parts unless only chapters should be displayed:
%    \begin{macrocode}
\ifchilddoc\else
\section{part three}
\input{cdocspt3}
\section{part four}
\input{cdocspt4}
\fi
%    \end{macrocode}

% Process as usual until here:
%    \begin{macrocode}
\fi
%    \end{macrocode}

% End of document body:
%    \begin{macrocode}
\end{document}
%    \end{macrocode}
%\iffalse
%</samplemain>
%\fi
%
% %%%%%%%%%%%%%%%%%%%%%%%%%%%%%%%%%%%%%%
% \paragraph{Chapter Include Files.}
%
% The include files are called |cdocsch1.tex| and |cdocsch2.tex|.
%
%\iffalse
%<*samplechap1|samplechap2>
%\fi

% Optional override for |\version| flag:
%    \begin{macrocode}
%%\providecommand{\version}{final}
%    \end{macrocode}

% Include the main document:
%    \begin{macrocode}
\input{childdoc.def}
\childdocof{cdocsamp}
%    \end{macrocode}

%\iffalse
%</samplechap1|samplechap2>
%\fi
%
%\iffalse
%<*samplechap1>
%\fi
% Some text for chapter 1:
%    \begin{macrocode}
\section{one}
some text in chapter one
%    \end{macrocode}

%\iffalse
%</samplechap1>
%\fi
% Some text for chapter 2:
%\iffalse
%<*samplechap2>
%\fi
%    \begin{macrocode}
\section{two}
more text in chapter two
%    \end{macrocode}

%\iffalse
%</samplechap2>
%\fi
%
% %%%%%%%%%%%%%%%%%%%%%%%%%%%%%%%%%%%%%%
% \paragraph{Part Include Files.}
%
% The include files are called |cdocspt3.tex| and |cdocspt4.tex|.
%
%\iffalse
%<*samplepart3|samplepart4>
%\fi

% Optional override for |\version| flag:
%    \begin{macrocode}
%%\providecommand{\version}{final}
%    \end{macrocode}

% Include the main document:
%    \begin{macrocode}
\input{childdoc.def}
\childdocby{cdocsamp}
%    \end{macrocode}

%\iffalse
%</samplepart3|samplepart4>
%\fi
%
%\iffalse
%<*samplepart3>
%\fi
% Some text for part 3:
%    \begin{macrocode}
some text in part three
%    \end{macrocode}

%\iffalse
%</samplepart3>
%\fi
% Some text for part 4:
%\iffalse
%<*samplepart4>
%\fi
%    \begin{macrocode}
more text in part four
%    \end{macrocode}

%\iffalse
%</samplepart4>
%\fi
%
% %%%%%%%%%%%%%%%%%%%%%%%%%%%%%%%%%%%%%%
% \paragraph{Forwarding for a Complete Draft.}
%
% The following forwarding file |cdocsdrf.tex|
% compiles the main document in draft mode:
%\iffalse
%<*sampledraft>
%\fi
%    \begin{macrocode}
\def\version{draft}
\input{childdoc.def}
\childdocforward{cdocsamp}
%    \end{macrocode}

%\iffalse
%</sampledraft>
%\fi
%
% %%%%%%%%%%%%%%%%%%%%%%%%%%%%%%%%%%%%%%
% \paragraph{Forwarding for Final Version of the Chapters.}
%
% The following forwarding files |cdocsfn1.tex| and |cdocsfn2.tex|
% (with identical content)
% compile the final versions of the child documents
% |cdocsch1.tex| and |cdocsch2.tex|, respectively:
%\iffalse
%<*samplefinal>
%\fi
%    \begin{macrocode}
\def\version{final}
\input{childdoc.def}
\childdocforwardprefix[cdocsamp]{cdocsfn}{cdocsch}
%    \end{macrocode}

%\iffalse
%</samplefinal>
%\fi
%
% %%%%%%%%%%%%%%%%%%%%%%%%%%%%%%%%%%%%%%
% \paragraph{Command Line Processing.}
%
% The following three command lines generate the output files
% |cdocscld|, |cdocscl1| and |cdocscl2|
% which should be identical to
% |cdocsdrf|, |cdocsch1| and |cdocsfn2|, respectively:
% \begin{center}
% \begin{tabular}{l}
% |latex -jobname cdocscld \|\\
% |  "\def\version{draft}\input{childdoc.def}\childdocforward{cdocsamp}"|\\
% |latex -jobname cdocscl1 \|\\
% |  "\input{childdoc.def}\childdocforward[cdocsamp]{cdocsch1}"|\\
% |latex -jobname cdocscl2 \|\\
% |  "\def\version{final}\input{childdoc.def}\childdocforward{cdocsch2}"|
% \end{tabular}
% \end{center}
% Note that the trailing backslash on each first line
% merely continues the input to the second line
% (for convenient cut ant paste).
% Furthermore, the command |latex| can be replaced by any
% of its alternative versions such as |pdflatex|.
%
% %%%%%%%%%%%%%%%%%%%%%%%%%%%%%%%%%%%%%%%%%%%%%%%%%%%%%%%%%%%%%%%%%%%%%%%%%%%%%%
% %%%%%%%%%%%%%%%%%%%%%%%%%%%%%%%%%%%%%%%%%%%%%%%%%%%%%%%%%%%%%%%%%%%%%%%%%%%%%%
% \section{Implementation}
%\iffalse
%<*package>
%\fi
%
% This section describes the definitions file |childdoc.def|.

% The definitions cannot be loaded using |\usepackage| or |\RequirePackage|
% which has a mechanism to prevent loading a style file more than once.
% When loading the definitions by means of |\input|
% multiple instances have to be prevented manually:
%\iffalse
%This code needs to be before the `\ProvidesFile' directive
%which is defined at the beginning of this file.
%Therefore it is also placed there and commented out here.
%</package>
%<*discard>
%\fi
%    \begin{macrocode}
\ifdefined\childdocmain\endinput\fi
%    \end{macrocode}
%\iffalse
%</discard>
%<*package>
%\fi
%
% \macro{\ifchilddoc}
% \macro{\ifchilddocmanual}
% The conditional |\ifchilddoc| tells whether a
% child (true) or main (false) document is being compiled.
% The conditional |\ifchilddocmanual| tells whether
% the |\includeonly| mechanism is used (false) or
% the selection of child files must be performed manually (true).
% The definitions initialise to false:
%    \begin{macrocode}
\newif\ifchilddoc
\newif\ifchilddocmanual
%    \end{macrocode}

% \macro{\childdocname}
% \macro{\childdocjob}
% The macro |\childdocname| stores the name of the main document
% to be compiled. The macro |\childdocjob| stores the name of
% the document on which the \LaTeX{} compiler was originally invoked.
% The content of |\jobname| cannot be compared
% to filenames specified in the source due to different catcodes.
% The following code rescans |\jobname|, stores the result
% in |\childdocname| and saves a copy in |\childdocjob|:
%    \begin{macrocode}
\edef\childdocname{\scantokens\expandafter{\jobname\noexpand}}
\let\childdocjob\childdocname
%    \end{macrocode}

% \macro{\childdocdisable}
% The macro |\childdocdisable| prevents the main file
% from being processed more than once.
% At this stage, the main document command |\childdocmain|
% is assumed to be called once again where it should do nothing.
% Any subsequent call to it should prevent
% a secondary processing of the main document
% It overwrites the forwarding commands
% |\childdocof| and |\childdocforward|
% with empty macros to prevent further inclusions of the main document:
%    \begin{macrocode}
\newcommand{\childdocdisable}
{
  \renewcommand{\childdocmain}[1]{\renewcommand{\childdocmain}[1]{\endinput}}
  \renewcommand{\childdocof}[1]{}
  \renewcommand{\childdocby}[2][]{}
  \renewcommand{\childdocforward}[2][]{}
  \renewcommand{\childdocdisable}{}
}
%    \end{macrocode}

% \macro{\childdocmain}
% The macro |\childdocmain| is to be called at the top of the main file
% with nothing or the main filename (without extension) as argument.
% First, it breaks loops.
% If the argument is not empty and does not match |\childdocname|
% (which is set by the first inclusion of |childdoc.def|),
% |\ifchilddoc| is set to true, |\includeonly| is applied to the child file
% and |\jobname| is set to the main file
% (for proper handling of |.aux| files):
%    \begin{macrocode}
\newcommand{\childdocmain}[1]
{
  \childdocdisable\childdocmain{}
  \if?#1?\else
    \begingroup
      \def\childdoctmp{#1}
      \ifx\childdoctmp\childdocname
        \def\childdoctmp{}
      \else
        \def\childdoctmp
        {
          \childdoctrue
          \includeonly{\childdocname}
          \def\childdocjob{#1}
          \def\jobname{#1}
        }
      \fi
      \expandafter
    \endgroup
    \childdoctmp
  \fi
}
%    \end{macrocode}

% \macro{\childdocof}
% The command |\childdocof| redirects
% compilation to the main file |#1|.
%    \begin{macrocode}
\newcommand{\childdocof}[1]
{
  \childdocdisable
  \childdoctrue
  \includeonly{\childdocname}
  \def\jobname{#1}
  \def\childdocjob{#1}
  \input{#1}
}
%    \end{macrocode}

% \macro{\childdocby}
% The command |\childdocby| ....
%    \begin{macrocode}
\newcommand{\childdocby}[2][]
{
  \childdocdisable
  \childdoctrue
  \childdocmanualtrue
  \if?#1?\else
    \def\jobname{#2}
  \fi
  \def\childdocjob{#2}
  \input{#2}
  \endinput
}
%    \end{macrocode}

% \macro{\childdocforward}
% The command |\childdocforward| redirects
% compilation to the main file or
% (if the optional argument is given) a child file.
% Parameters are set as if the main file
% or a child file starting with |\childdocof| was compiled.
% Then compilation is handed over to the main file:
%    \begin{macrocode}
\newcommand{\childdocforward}[2][]
{
  \begingroup
    \if?#1?
      \def\childdoctmp
      {
        \def\childdocname{#2}
        \def\childdocjob{#2}
        \def\jobname{#2}
        \input{#2}
        \endinput
      }
    \else
      \def\childdoctmp
      {
        \childdocdisable
        \def\childdocname{#2}
        \childdoctrue
        \includeonly{#2}
        \def\childdocjob{#1}
        \def\jobname{#1}
        \input{#1}
        \endinput
      }
    \fi
    \expandafter
  \endgroup
  \childdoctmp
}
%    \end{macrocode}

% \macro{\childdocforwardprefix}
% The command |\childdocforwardprefix| redirects
% compilation to the main or a child file by means of a pattern.
% The prefix |#1| in the current filename is replaced by |#2|
% and the suffix of the current filename is kept
% (it is assumed that the filename does not contain the substring `|~~~|'
% which is used as a delimiter).
% Compilation is handed over to the new file by |\childdocforward|:
%    \begin{macrocode}
\newcommand{\childdocforwardprefix}[3][]
{
  \begingroup
    \def\childdocextract #2##1~~~{\def\childdoctmp{\childdocforward[#1]{#3##1}}}
    \expandafter\childdocextract\childdocname~~~
    \expandafter
  \endgroup
  \childdoctmp
}
%    \end{macrocode}

% \macro{\childdoc}
% The deprecated macro |\childdoc| is a legacy version of |\childdocmain|:
%    \begin{macrocode}
\newcommand{\childdoc}{\childdocmain}
%    \end{macrocode}

% \macro{\childdocredirect}
% The deprecated macro |\childdocredirect| is a legacy version
% of |\childdocforward| and |\childdocforwardprefix|:
%    \begin{macrocode}
\newcommand{\childdocredirect}[2][]
{
  \begingroup
    \if?#1?
      \def\childdoctmp{\childdocforward{#2}}
    \else
      \def\childdoctmp{\childdocforwardprefix{#1}{#2}}
    \fi
    \expandafter
  \endgroup
  \childdoctmp
}
%    \end{macrocode}

%\iffalse
%</package>
%\fi
%
\endinput
|\\
|\childdocmain{|\textit{main}|}|\\
\end{tabular}
\end{center}
%
If |\jobname| does not match the argument \textit{main} of |\childdocmain|,
it is assumed that |\jobname| points to the child file to be compiled.
When using |\childdocmain| with the main file specified as argument,
it suffices to start a child file
with just |\input{|\textit{main}|}|
without loading of the package and using |\childdocof|.
If instead all processing is done
with the appropriate \textsf{childdoc} directives,
the argument of \textit{main} of |\childdocmain| can be empty.

An alternative version of the command line processing described
in \secref{sec:commandline} using the detection mechanism reads:
%
\begin{center}
|... -jobname "|\textit{target}|" "|[\textit{flags}]%
[|\def\jobname{|\textit{dest}|}|]|\input{|\textit{main}|}"|
\end{center}

%%%%%%%%%%%%%%%%%%%%%%%%%%%%%%%%%%%%%%%%%%%%%%%%%%%%%%%%%%%%%%%%%%%%%%%%%%%%%%%%
\subsection{Manual Code}
\label{sec:manual}

In case one cannot be certain whether the definitions file |childdoc.def|
is installed on the target \TeX{} distribution
and one prefers not to ship it,
it is conceivable to paste a few relevant commands into the sources.

To that end, drop all statements |% \iffalse
%
% childdoc.dtx Copyright (C) 2017-2018 Niklas Beisert
%
% This work may be distributed and/or modified under the
% conditions of the LaTeX Project Public License, either version 1.3
% of this license or (at your option) any later version.
% The latest version of this license is in
%   http://www.latex-project.org/lppl.txt
% and version 1.3 or later is part of all distributions of LaTeX
% version 2005/12/01 or later.
%
% This work has the LPPL maintenance status `maintained'.
%
% The Current Maintainer of this work is Niklas Beisert.
%
% This work consists of the files childdoc.dtx and childdoc.ins
% and the derived files childdoc.def and cdocsamp.tex with
% cdocsch1.tex, cdocsch2.tex, cdocsdrf.tex, cdocsfn1.tex, cdocsfn2.tex.
%
%<package>\ifdefined\childdocmain\endinput\fi
%<package>\ProvidesFile{childdoc.def}[2018/12/30 v2.0 child document driver]
%<samplemain>\ProvidesFile{cdocsamp.tex}[2018/12/30 v2.0 sample for childdoc]
%<*driver>
%\ProvidesFile{childdoc.drv}[2018/12/30 v2.0 childdoc reference manual file]
\PassOptionsToClass{10pt,a4paper}{article}
\documentclass{ltxdoc}

\usepackage[margin=35mm]{geometry}
\usepackage{hyperref}
\usepackage{hyperxmp}
\usepackage[usenames]{color}

\hypersetup{colorlinks=true}
\hypersetup{pdfstartview=FitH}
\hypersetup{pdfpagemode=UseNone}
\hypersetup{pdfsource={}}
\hypersetup{pdflang={en-UK}}
\hypersetup{pdfcopyright={Copyright 2017-2018 Niklas Beisert.
  This work may be distributed and/or modified under the
  conditions of the LaTeX Project Public License, either version 1.3
  of this license or (at your option) any later version.}}
\hypersetup{pdflicenseurl={http://www.latex-project.org/lppl.txt}}
\hypersetup{pdfcontactaddress={ETH Zurich, ITP, HIT K,
  Wolfgang-Pauli-Strasse 27}}
\hypersetup{pdfcontactpostcode={8093}}
\hypersetup{pdfcontactcity={Zurich}}
\hypersetup{pdfcontactcountry={Switzerland}}
\hypersetup{pdfcontactemail={nbeisert@itp.phys.ethz.ch}}
\hypersetup{pdfcontacturl={http://people.phys.ethz.ch/\xmptilde nbeisert/}}

\newcommand{\secref}[1]{\hyperref[#1]{section \ref*{#1}}}

\parskip1ex
\parindent0pt
\let\olditemize\itemize
\def\itemize{\olditemize\parskip0pt}

\begin{document}

\title{The \textsf{childdoc} Package}
\hypersetup{pdftitle={The childdoc Package}}
\author{Niklas Beisert\\[2ex]
  Institut f\"ur Theoretische Physik\\
  Eidgen\"ossische Technische Hochschule Z\"urich\\
  Wolfgang-Pauli-Strasse 27, 8093 Z\"urich, Switzerland\\[1ex]
  \href{mailto:nbeisert@itp.phys.ethz.ch}
  {\texttt{nbeisert@itp.phys.ethz.ch}}}
\hypersetup{pdfauthor={Niklas Beisert}}
\hypersetup{pdfsubject={Manual for the LaTeX2e Package childdoc}}
\date{30 December 2018, \textsf{v2.0}}
\maketitle

\begin{abstract}\noindent
\textsf{childdoc} is a \LaTeXe{} package
that enables the direct compilation
of document sections included by |\include|
to individual files.
\end{abstract}

\begingroup
\parskip0ex
\tableofcontents
\endgroup

%%%%%%%%%%%%%%%%%%%%%%%%%%%%%%%%%%%%%%%%%%%%%%%%%%%%%%%%%%%%%%%%%%%%%%%%%%%%%%%%
%%%%%%%%%%%%%%%%%%%%%%%%%%%%%%%%%%%%%%%%%%%%%%%%%%%%%%%%%%%%%%%%%%%%%%%%%%%%%%%%
\section{Introduction}

\LaTeX{} provides a mechanism to structure a large document (such as a book)
into a main file and several child files (containing the chapters)
using the |\include| command.
This mechanism is beneficial for documents
which span hundreds of pages in order to
make the source file(s) more manageable.
Moreover, compilation can be restricted to
selected child files by means of the |\includeonly| command.
The latter feature can be used to reduce the compilation time while editing
(this was significantly more useful in the earlier days of \LaTeX{})
or to generate a smaller document which is easier to navigate.
Another application of |\includeonly| is to generate
documents consisting of selected parts of the complete document.

However, there are a few drawbacks of the plain |\include| mechanism:
\begin{itemize}
\item
The child files cannot be compiled on their own,
they can only be compiled via the main file.
A naive editing environment
(such as a text editor with an option
to have the current file processed by \LaTeX)
may require one to switch to the main file before compiling;
attempting to compile the child file produces errors.
\item
The main file must be modified (each time)
to adjust the |\includeonly| command
to the present needs. This easily leaves the main file in a messy state.
\item
The generated document will always carry the filename
of the main document. This is inconvenient if
several child files are to be compiled and
to be kept for distribution.
\end{itemize}

The present package provides a simple interface
to make child files individually compilable by \LaTeX{}.
Compiling a child file then has the same effect as compiling
the main file with an |\includeonly| command
to select the appropriate child.
Moreover the generated document will carry the name of the child
rather than the main file.
This resolves all three above issues.

This feature is meant to make the editing of books,
thesis documents and lecture notes somewhat more convenient.
However, the package can also be used efficiently for
composing a series of documents (such as exercise sheets)
which are typically distributed individually.
It then assists the author in generating the individual documents
(potentially in different versions)
as well as a document containing the collected series.
Another application is in developing style files
or other kinds of included material
where compilation of the style file could redirect
to a sample or test file.

%%%%%%%%%%%%%%%%%%%%%%%%%%%%%%%%%%%%%%%%%%%%%%%%%%%%%%%%%%%%%%%%%%%%%%%%%%%%%%%%
%%%%%%%%%%%%%%%%%%%%%%%%%%%%%%%%%%%%%%%%%%%%%%%%%%%%%%%%%%%%%%%%%%%%%%%%%%%%%%%%
\section{Usage}

First of all, the package \textsf{childdoc} is \emph{not} a standard
\LaTeXe{} |.sty| style file! Therefore it needs to be invoked in
a non-standard way.

%%%%%%%%%%%%%%%%%%%%%%%%%%%%%%%%%%%%%%%%%%%%%%%%%%%%%%%%%%%%%%%%%%%%%%%%%%%%%%%%
\subsection{Included Files}
\label{sec:include}

%%%%%%%%%%%%%%%%%%%%%%%%%%%%%%%%%%%%%%%%
\DescribeMacro{\childdocmain}
To use the package, add the commands
\begin{center}
\begin{tabular}{l}
|\input{childdoc.def}|\\
|\childdocmain{}|\\
\end{tabular}
\end{center}
at the very top of the main \LaTeX{} file,
in particular \emph{before} the |\documentclass| statement!
The argument of |\childdocmain| should be left empty
(but it must be present).

%%%%%%%%%%%%%%%%%%%%%%%%%%%%%%%%%%%%%%%%
\DescribeMacro{\childdocof}
Furthermore, add the commands
\begin{center}
\begin{tabular}{l}
|\input{childdoc.def}|\\
|\childdocof{|\textit{main}|}|\\
\end{tabular}
\end{center}
at the top of every child file \textit{child}
which is included by |\include{|\textit{child}|}|
from within the main file
(or at least for those files to be compiled individually).
The argument \textit{main} must be the filename of the main file.

There are a couple of
considerations in setting up the main and child documents:

%%%%%%%%%%%%%%%%%%%%%%%%%%%%%%%%%%%%%%%%
\paragraph{Restrictions.}

Please note the following restrictions:
\begin{itemize}
\item
|\childdocmain| must be called with one argument \textit{main}
to ensure compatibility with earlier version of the package.
It must either be empty (|\childdocmain{}|)
or precisely match the filename of the main file in which it is specified.
See \secref{sec:detection} for further information.
\item
The filename \textit{main} must be specified without the |.tex| extension.
\item
The filename \textit{main} is case sensitive
(even in case-insensitive file systems)
due to internal string comparison.
\item
The argument \textit{main} should be fully expanded, it cannot be a macro.
\item
Subdirectories and special characters should be avoided in filenames.
\item
The command |\childdocmain{|\textit{main}|}| must be followed by a whitespace.
It should not be followed immediately by another command
or by a comment mark `|%|'.
This is because the \TeX{} parser reads the token immediately following
the argument of |\childdocmain| and puts it
at the beginning of every child section;
however, a white\-space is ignored.
\end{itemize}

%%%%%%%%%%%%%%%%%%%%%%%%%%%%%%%%%%%%%%%%
\paragraph{Content of Main File.}

It is advisable to place all content in the child files included by |\include|.
Any output contained in the main file will appear in all child documents
unless suppressed manually;
it cannot be suppressed automatically by the |\includeonly| directive
and thus should normally be avoided.
A method to include some content in the main file
by means of conditional processing is described in \secref{sec:conditional}.

%%%%%%%%%%%%%%%%%%%%%%%%%%%%%%%%%%%%%%%%
\paragraph{Page Numbering.}

When only a part of the document is compiled,
the appropriate numbering of pages
(as well as other status parameters)
is determined from the |.aux| files.
The latter contain information from previous passes.
However this information needs to propagate through
all intermediate child documents.
Therefore the page numbering in child documents may well
be inconsistent until the complete document is compiled at least once.

A useful (if unconventional) way to always ensure a consistent
page numbering is to restart the numbering in each child document
and denote the pages by `\textit{child}|.|\textit{page}'
where \textit{child} represents the chapter/section number of the child file.
This can be achieved by the command
|\numberwithin{page}{|\textit{child}|}|
of the \textsf{amsmath} package
where \textit{child} can be |chapter| or |section|
depending on the chosen structuring.
Alternatively, one can modify the macro |\thepage| appropriately
and reset the counter |page| at the start of each child file.

%%%%%%%%%%%%%%%%%%%%%%%%%%%%%%%%%%%%%%%%%%%%%%%%%%%%%%%%%%%%%%%%%%%%%%%%%%%%%%%%
\subsection{Conditional Processing}
\label{sec:conditional}

The package provides a mechanism to compile different versions
of a document. To customise the versions further some conditional processing
can come in handy to distinguish which version is being compiled.
The package provides two macros to describe the compilation context:

%%%%%%%%%%%%%%%%%%%%%%%%%%%%%%%%%%%%%%%%
\DescribeMacro{\ifchilddoc}
The conditional |\ifchilddoc| distinguishes between the compilation of
child documents and the main document:
%
\begin{center}
|\ifchilddoc |\textit{child-code}| |[|\||else |\textit{main-code}]| \||fi|
\end{center}

%%%%%%%%%%%%%%%%%%%%%%%%%%%%%%%%%%%%%%%%
\DescribeMacro{\childdocname}
\DescribeMacro{\childdocjob}
The macro |\childdocname| contains the filename (without extension)
of the main or child file being processed.
Note that |\childdocjob| will always contain the name of the main file.

%%%%%%%%%%%%%%%%%%%%%%%%%%%%%%%%%%%%%%%%
\paragraph{Title Page.}

Conditional processing can be used to include a title or banner page
in the main document when proper precautions are taken.
Importantly, the code in the main file should ensure that the page counter
(as well as other status parameters which are stored in the |.aux| files)
takes the same value after the conditional processing.
Otherwise the page numbers may take divergent values
depending on which part is compiled.

For example, a title page could be declared by:
%
\begin{center}
\begin{tabular}{l}
|\ifchilddoc\||else|\\
|\addtocounter{page}{-1}|\\
\textit{code for title page}\\
|\newpage|\\
|\||fi|
\end{tabular}
\end{center}
%
A banner page for the child documents can be generated by:
%
\begin{center}
\begin{tabular}{l}
|\ifchilddoc|\\
|\addtocounter{page}{-1}|\\
\textit{code for banner page}\\
|\newpage|\\
|\||fi|
\end{tabular}
\end{center}
%
Here one could write a message such as:
\begin{center}
|This is the part \childdocname{} of \childdocjob{}.|
\end{center}

%%%%%%%%%%%%%%%%%%%%%%%%%%%%%%%%%%%%%%%%%%%%%%%%%%%%%%%%%%%%%%%%%%%%%%%%%%%%%%%%
\subsection{Flags}
\label{sec:flags}

The package makes it easy to generate different versions
of the main or child documents.
To this end compilation flags can be defined
and assigned different default values.
They will be particularly useful in conjunction
with the forwarding mechanism described in \secref{sec:forward}.

For example, it may be useful to have a flag |\version|
which can be set to |draft| or |final|.
The document source will contain some conditional code
depending on the value of |\version|.
Suppose further, the flag should default to |final| for the main file
and to |draft| for child files
which is a natural assignment for editing the document.
This is achieved by placing the following code
in the preamble of the main document
(below the |\childdocmain| directive):
%
\begin{center}
\begin{tabular}{l}
|\ifchilddoc|\\
|\providecommand{\version}{draft}|\\
|\||else|\\
|\providecommand{\version}{final}|\\
|\||fi|
\end{tabular}
\end{center}
%
The definition by |\providecommand| makes sure
that previous definitions are not overwritten.
Further statements |\providecommand{\version}{...}|
can thus be added before the above code to override it.

For the main file, one might add a line
(between |\childdocmain| and the above block)
%
\begin{center}
|%\ifchilddoc\||else\providecommand{\version}{draft}\||fi|
\end{center}
%
which can be uncommented to produce a draft version.
Likewise one can add a line to the very top of a child file
(above the |\childdocof{|\textit{main}|}| directive)
%
\begin{center}
|%\providecommand{\version}{final}|
\end{center}
%
which can be uncommented to produce the final version of this child document.

%%%%%%%%%%%%%%%%%%%%%%%%%%%%%%%%%%%%%%%%%%%%%%%%%%%%%%%%%%%%%%%%%%%%%%%%%%%%%%%%
\subsection{Forwarding}
\label{sec:forward}

Different versions of the main or child documents
using compilation flags as described in \secref{sec:flags}
can be (permanently) stored in different files
for convenient compilation, viewing and distribution.
To this end, the package defines a command
to pass on compilation to a different file:

%%%%%%%%%%%%%%%%%%%%%%%%%%%%%%%%%%%%%%%%
\DescribeMacro{\childdocforward}
The command |\childdocforward| redirects processing to
another source file:
%
\begin{center}
\begin{tabular}{l}
|\input{childdoc.def}|\\
|\childdocforward[|\textit{main}|]{|\textit{dest}|}|\\
\end{tabular}
\end{center}
%
The argument \textit{dest} is the destination file
(without extension).
It should be the main file or one of the child files.
Note that further \textsf{childdoc} directives
such as |\childdocof| and |\childdocforward|
in the indicated file will be processed in this form.
The optional argument \textit{main}
passes on directly to the main file \textit{main}
while pretending to compile the child \textit{dest}.
This form behaves as if \textit{dest}
issues |\childdocof{|\textit{main}|}| right away,
and no further \textsf{childdoc} directives will be processed.

%%%%%%%%%%%%%%%%%%%%%%%%%%%%%%%%%%%%%%%%
\DescribeMacro{\...prefix}
In the alternative form |\childdocforwardprefix|,
%
\begin{center}
\begin{tabular}{l}
|\input{childdoc.def}|\\
|\childdocforwardprefix[|\textit{main}|]{|\textit{prefix}|}{|\textit{dest}|}|
\end{tabular}
\end{center}
%
the destination file is determined by a pattern
depending on the current file:
To make this work, the current file must be called
`{\textit{prefix}\hspace{0.2em}\textit{suffix}}'
with \textit{prefix} matching precisely the argument.
Processing is then passed on to the file
`{\textit{dest}\hspace{0.2em}\textit{suffix}}'.
Surely, the same effect is achieved by
directly specifying the
argument `{\textit{dest}\hspace{0.2em}\textit{suffix}}'
in the first form.
However, that requires to set up a different file
for each child. With the alternative form of the command
all these files can have exactly the same content
which simplifies setting them up and maintaining them.

For example, the following file |draft.tex|
with a compilation flag |\version| as described in \secref{sec:flags}
compiles the main document as a draft:
%
\begin{center}
\begin{tabular}{l}
|\def\version{draft}|\\
|\input{childdoc.def}|\\
|\childdocforward{|\textit{main}|}|
\end{tabular}
\end{center}
%
Likewise, the following files |final|\textit{nn}|.tex|
compile the final version of the child document
|child|\textit{nn}|.tex|:
%
\begin{center}
\begin{tabular}{l}
|\def\version{final}|\\
|\input{childdoc.def}|\\
|\childdocforwardprefix{final}{child}|
\end{tabular}
\end{center}
%

Note that when several versions of a main file and/or of each child file
are to be generated, it may be convenient to set up a |Makefile| or
shell script to automatise the process.

%%%%%%%%%%%%%%%%%%%%%%%%%%%%%%%%%%%%%%%%%%%%%%%%%%%%%%%%%%%%%%%%%%%%%%%%%%%%%%%%
\subsection{Command Line Processing}
\label{sec:commandline}

The effect of redirection files can also be achieved by invoking
the \LaTeX{} compiler with a more elaborate command line.
Most conveniently this should be done as part
of a shell script or a |Makefile|.

When using \textsf{childdoc} in the main file, the following
command lines effectively perform a redirection
(note that depending on the shell being used,
backslashes may have to be doubled: `|\|' $\to$ `|\\|'):
%
\begin{center}
|... -jobname "|\textit{target}|" |\\|"|[\textit{flags}]%
|\input{childdoc.def}\childdocforward[|\textit{main}|]{|\textit{dest}|}"|
\end{center}
%
Here \textit{target} is the name of the output file,
\textit{main} is the name of the main file
and \textit{dest} is the name of the main or child file to be processed
(all filenames without extensions).
The optional argument \textit{main} can be omitted
if \textit{main} matches \textit{dest}.
Optionally, compilation \textit{flags} can be defined via |\def| commands.
This command line makes the \TeX{} engine believe
it is compiling the file \textit{target}
whose content is specified as the latter parameter.
The provided code then forwards the processing to
\textit{main} or \textit{dest} as described in \secref{sec:forward}.

%%%%%%%%%%%%%%%%%%%%%%%%%%%%%%%%%%%%%%%%%%%%%%%%%%%%%%%%%%%%%%%%%%%%%%%%%%%%%%%%
\subsection{Include by Input}
\label{sec:input}

Including child documents by |\include| has some restrictions by design.
Most notably, the content of a child document always occupies
its own set of pages; pages cannot be shared between child documents.
Usually, this behaviour makes perfect sense
because each child document contain an essential part of the document.
However, in some situations it may be desirable to compose
a document from a collection of parts
without having mandatory page breaks between then.
For this case, the package
provides a mechanism to include parts
by |\input| which can also be processed individually.
However, by construction this mechanism
requires manual handling of the content to be output.

%%%%%%%%%%%%%%%%%%%%%%%%%%%%%%%%%%%%%%%%
\DescribeMacro{\ifchilddocmanual}
The main file should be prepared as usual, see \secref{sec:include}.
However, the document body must make a distinction
between processing of an individual part and of the main document, e.g.:
%
\begin{center}
\begin{tabular}{l}
|\ifchilddocmanual|\\
|\input{\childdocname}|\\
|\||else|\\
\textit{document body with }|\input{|\textit{part}|}|\\
|\||fi|
\end{tabular}
\end{center}
%
The conditional |\ifchilddocmanual| is true whenever
a part to be included by |\input| is being compiled,
and the name of the part is stored in |\childdocname|.

%%%%%%%%%%%%%%%%%%%%%%%%%%%%%%%%%%%%%%%%
\DescribeMacro{\childdocby}
Each part to be included by |\input| should start with:
%
\begin{center}
\begin{tabular}{l}
|\input{childdoc.def}|\\
|\childdocby{|\textit{main}|}|\\
\end{tabular}
\end{center}
%
The directive |\childdocby| is similar to |\childdocof|
described in \secref{sec:include},
but the subsequent selection of content must be done manually.
To that end, both |\ifchilddoc| and |\ifchilddocmanual|
will be true upon processing of a part,
and the name of the part is stored in |\childdocname|.
Note that |\jobname| will be set to the filename of the current part
so that each part receives an individual |.aux| file
that does not interfere with the |.aux| file(s) of the main document.
This behaviour can be altered by the alternative form
|\childdocby[*]{|\textit{main}|}| (with a non-empty optional argument)
which uses the |.aux| file of the main document
by setting |\jobname| to \textit{main}.

%%%%%%%%%%%%%%%%%%%%%%%%%%%%%%%%%%%%%%%%%%%%%%%%%%%%%%%%%%%%%%%%%%%%%%%%%%%%%%%%
\subsection{Driver Development}
\label{sec:driver}

The \textsf{childdoc} mechanism can also be use for the development
of definition files such as \LaTeX{} styles or classes.
This case differs from the above setup with multiple parts
included by |\include| in that no |\includeonly| should be invoked.
This can be achieved by starting the include file
(before |\ProvidesPackage|) with:
%
\begin{center}
\begin{tabular}{l}
|\input{childdoc.def}|\\
|\childdocforward{|\textit{main}|}|\\
\end{tabular}
\end{center}
%
or alternatively with:
%
\begin{center}
\begin{tabular}{l}
|\input{childdoc.def}|\\
|\childdocby{|\textit{main}|}|\\
\end{tabular}
\end{center}
%
Both forms have slightly different effects as described above.
The main file is prepared as usual, see \secref{sec:include}.

%%%%%%%%%%%%%%%%%%%%%%%%%%%%%%%%%%%%%%%%%%%%%%%%%%%%%%%%%%%%%%%%%%%%%%%%%%%%%%%%
\subsection{Legacy Detection}
\label{sec:detection}

The directive |\childdocmain| in the main file can detect
whether the complete document or merely a child is to be compiled
even without using the directive |\childdocof|.
This method is deprecated because it is less robust
and there is no compelling reason to use it;
it is merely provided for backward compatibility
and it may be removed in future versions.

If the detection mechanism is to be used,
it is mandatory to correctly specify
the filename of the main file as the argument of |\childdocmain|:
%
\begin{center}
\begin{tabular}{l}
|\input{childdoc.def}|\\
|\childdocmain{|\textit{main}|}|\\
\end{tabular}
\end{center}
%
If |\jobname| does not match the argument \textit{main} of |\childdocmain|,
it is assumed that |\jobname| points to the child file to be compiled.
When using |\childdocmain| with the main file specified as argument,
it suffices to start a child file
with just |\input{|\textit{main}|}|
without loading of the package and using |\childdocof|.
If instead all processing is done
with the appropriate \textsf{childdoc} directives,
the argument of \textit{main} of |\childdocmain| can be empty.

An alternative version of the command line processing described
in \secref{sec:commandline} using the detection mechanism reads:
%
\begin{center}
|... -jobname "|\textit{target}|" "|[\textit{flags}]%
[|\def\jobname{|\textit{dest}|}|]|\input{|\textit{main}|}"|
\end{center}

%%%%%%%%%%%%%%%%%%%%%%%%%%%%%%%%%%%%%%%%%%%%%%%%%%%%%%%%%%%%%%%%%%%%%%%%%%%%%%%%
\subsection{Manual Code}
\label{sec:manual}

In case one cannot be certain whether the definitions file |childdoc.def|
is installed on the target \TeX{} distribution
and one prefers not to ship it,
it is conceivable to paste a few relevant commands into the sources.

To that end, drop all statements |\input{childdoc.def}|
and perform the replacements as outlined below.
Instead of |\childdocmain{|\textit{main}|}| add the following code
to the top of the main file:
%
\begin{center}
\begin{tabular}{l}
|\||ifdefined\childdocname\endinput\||fi\newif\ifchilddoc|\\
|\edef\childdocname{\scantokens\expandafter{\jobname\noexpand}}|\\
|\def\childdocmain{|\textit{main}|}\||ifx\childdocmain\childdocname\||else|\\
|\childdoctrue\includeonly{\childdocname}\let\jobname\childdocmain\||fi|\\
\end{tabular}
\end{center}
%
Instead of |\childdocof{|\textit{main}|}| just include the main file
at the top of each child file:
%
\begin{center}
|\input{|\textit{main}|}|
\end{center}
%
A simple redirection |\childdocforward{|\textit{dest}|}| is achieved by:
%
\begin{center}
|\def\jobname{|\textit{dest}|}\input{\jobname}|
\end{center}
%
The redirection with prefix
|\childdocforwardprefix[|\textit{prefix}|]{|\textit{dest}|}|
is accomplished by:
%
\begin{center}
\begin{tabular}{l}
|{\edef\jobname{\scantokens\expandafter{\jobname\noexpand}}|\\
|\def\redirectjob |\textit{prefix}|#1~~~{\gdef\jobname{|\textit{dest}|#1}}|\\
|\expandafter\redirectjob\jobname~~~}\input{\jobname}|
\end{tabular}
\end{center}

In an alternative approach,
child documents can be compiled by a specific command line
without additional code or specific definitions:
%
\begin{center}
|... -jobname "|\textit{target}|" "|[\textit{flags}]%
|\includeonly{|\textit{dest}|}\input{|\textit{main}|}"|
\end{center}
%

%%%%%%%%%%%%%%%%%%%%%%%%%%%%%%%%%%%%%%%%%%%%%%%%%%%%%%%%%%%%%%%%%%%%%%%%%%%%%%%%
%%%%%%%%%%%%%%%%%%%%%%%%%%%%%%%%%%%%%%%%%%%%%%%%%%%%%%%%%%%%%%%%%%%%%%%%%%%%%%%%
\section{Information}

%%%%%%%%%%%%%%%%%%%%%%%%%%%%%%%%%%%%%%%%%%%%%%%%%%%%%%%%%%%%%%%%%%%%%%%%%%%%%%%%
\subsection{Copyright}

Copyright \copyright{} 2017--2018 Niklas Beisert

This work may be distributed and/or modified under the
conditions of the \LaTeX{} Project Public License, either version 1.3
of this license or (at your option) any later version.
The latest version of this license is in
  \url{http://www.latex-project.org/lppl.txt}
and version 1.3 or later is part of all distributions of \LaTeX{}
version 2005/12/01 or later.

This work has the LPPL maintenance status `maintained'.

The Current Maintainer of this work is Niklas Beisert.

This work consists of the files |README.txt|, |childdoc.ins| and |childdoc.dtx|
as well as the derived files |childdoc.def|, |cdocsamp.tex|
with |cdocsch1.tex|, |cdocsch2.tex|, |cdocspt3.tex|, |cdocspt4.tex|,
|cdocsdrf.tex|, |cdocsfn1.tex|, |cdocsfn2.tex|
as well as |childdoc.pdf|.

%%%%%%%%%%%%%%%%%%%%%%%%%%%%%%%%%%%%%%%%%%%%%%%%%%%%%%%%%%%%%%%%%%%%%%%%%%%%%%%%
\subsection{Files and Installation}

The package consists of the files:
%
\begin{center}
\begin{tabular}{ll}
    |README.txt|   & readme file \\
    |childdoc.ins| & installation file \\
    |childdoc.dtx| & source file \\
    |childdoc.def| & definition file \\
    |cdocsamp.tex| & sample main file \\
    |cdocsch1.tex| & sample include file \\
    |cdocsch2.tex| & sample include file \\
    |cdocspt3.tex| & sample part file \\
    |cdocspt4.tex| & sample part file \\
    |cdocsdrf.tex| & sample redirection file \\
    |cdocsfn1.tex| & sample redirection file \\
    |cdocsfn2.tex| & sample redirection file \\
    |childdoc.pdf| & manual
\end{tabular}
\end{center}
%
The distribution consists of the files
|README.txt|, |childdoc.ins| and |childdoc.dtx|.
%
\begin{itemize}
\item
Run (pdf)\LaTeX{} on |childdoc.dtx|
to compile the manual |childdoc.pdf| (this file).
\item
Run \LaTeX{} on |childdoc.ins| to create the definitions file |childdoc.def|
and the sample |cdocsamp.tex| with include files
|cdocsch1.tex|, |cdocsch2.tex|, |cdocspt3.tex|, |cdocspt4.tex|,
|cdocsdrf.tex|, |cdocsfn1.tex|, |cdocsfn2.tex|.
Then copy the file |childdoc.def| to an appropriate directory of your \LaTeX{}
distribution, e.g.\ \textit{texmf-root}|/tex/latex/childdoc|.
\end{itemize}

%%%%%%%%%%%%%%%%%%%%%%%%%%%%%%%%%%%%%%%%%%%%%%%%%%%%%%%%%%%%%%%%%%%%%%%%%%%%%%%%
\subsection{Related CTAN Packages}

There are several other packages which offer a similar functionality:
%
\begin{itemize}
\item
The packages
\href{http://ctan.org/pkg/docmute}{\textsf{docmute}},
\href{http://ctan.org/pkg/includex}{\textsf{includex}} and
\href{http://ctan.org/pkg/standalone}{\textsf{standalone}}
provide commands to include only the document body of
a child file thus allowing both files to be compiled individually.
\item
The packages \href{http://ctan.org/pkg/subdocs}{\textsf{subdocs}}
and \href{http://ctan.org/pkg/subfiles}{\textsf{subfiles}}
provide structures in which the main and child documents can be
encapsulated and allowing them to be compiled individually.
The inclusion mechanism is different from the conventional |\include|.
\item
The package \href{http://ctan.org/pkg/combine}{\textsf{combine}}
is an elaborate solution to combine several documents into one.
\end{itemize}
%
See also the CTAN topic \href{http://ctan.org/topic/subdocs}{\textsf{subdocs}}
for further related packages.
The present package differs from the above solutions in that
a document structure constructed with the conventional |\include| mechanism
just needs two extra commands at the top of every file
such that all constituent files can be compiled individually.

%%%%%%%%%%%%%%%%%%%%%%%%%%%%%%%%%%%%%%%%%%%%%%%%%%%%%%%%%%%%%%%%%%%%%%%%%%%%%%%%
%\subsection{Feature Suggestions}
%
%The following is a list of features which may be useful for future
%versions of this package:
%%
%\begin{itemize}
%\item
%\ldots
%\end{itemize}

%%%%%%%%%%%%%%%%%%%%%%%%%%%%%%%%%%%%%%%%%%%%%%%%%%%%%%%%%%%%%%%%%%%%%%%%%%%%%%%%
\subsection{Revision History}

%%%%%%%%%%%%%%%%%%%%%%%%%%%%%%%%%%%%%%%%
\paragraph{v2.0:} 2018/12/30

\begin{itemize}
\item
immediate forward processing
\item
added |\childdocby| mechanism
\item
manual restructured
\end{itemize}

%%%%%%%%%%%%%%%%%%%%%%%%%%%%%%%%%%%%%%%%
\paragraph{v1.6:} 2018/01/17

\begin{itemize}
\item
application for development of include files
\item
corrections to manual
\end{itemize}

%%%%%%%%%%%%%%%%%%%%%%%%%%%%%%%%%%%%%%%%
\paragraph{v1.5:} 2017/05/21

\begin{itemize}
\item
more complete structuring introduced
\item
|\childdocof| introduced
\item
|\childdoc| renamed to |\childdocmain|
\item
|\childredirect| renamed to |\childdocforward| and |\childdocforwardprefix|
and functionality expanded
\end{itemize}

%%%%%%%%%%%%%%%%%%%%%%%%%%%%%%%%%%%%%%%%
\paragraph{v1.0:} 2017/04/27

\begin{itemize}
\item
manual and install package
\item
first version published on CTAN
\end{itemize}

%%%%%%%%%%%%%%%%%%%%%%%%%%%%%%%%%%%%%%%%
\paragraph{v0.6:} 2017/04/26

\begin{itemize}
\item
redirection mechanism added
\end{itemize}

%%%%%%%%%%%%%%%%%%%%%%%%%%%%%%%%%%%%%%%%
\paragraph{v0.5:} 2017/04/26

\begin{itemize}
\item
functionality in definition file
\end{itemize}


%%%%%%%%%%%%%%%%%%%%%%%%%%%%%%%%%%%%%%%%%%%%%%%%%%%%%%%%%%%%%%%%%%%%%%%%%%%%%%%%
%%%%%%%%%%%%%%%%%%%%%%%%%%%%%%%%%%%%%%%%%%%%%%%%%%%%%%%%%%%%%%%%%%%%%%%%%%%%%%%%
%%%%%%%%%%%%%%%%%%%%%%%%%%%%%%%%%%%%%%%%%%%%%%%%%%%%%%%%%%%%%%%%%%%%%%%%%%%%%%%%
\appendix

\settowidth\MacroIndent{\rmfamily\scriptsize 000\ }

 \DocInput{childdoc.dtx}

\end{document}
%</driver>
% \fi
%
% %%%%%%%%%%%%%%%%%%%%%%%%%%%%%%%%%%%%%%%%%%%%%%%%%%%%%%%%%%%%%%%%%%%%%%%%%%%%%%
% %%%%%%%%%%%%%%%%%%%%%%%%%%%%%%%%%%%%%%%%%%%%%%%%%%%%%%%%%%%%%%%%%%%%%%%%%%%%%%
% \section{Sample}
%\iffalse
%<*samplemain>
%\fi
%
% The following presents a sample document
% with two chapters, two parts, a title page,
% a compile flag as well as three forwarding files to set the flag.
% It consists of eight |.tex| files:
% \begin{center}
% \begin{tabular}{ll}
% |cdocsamp.tex|&main file\\
% |cdocsch1.tex|&include file for chapter 1\\
% |cdocsch2.tex|&include file for chapter 2\\
% |cdocspt3.tex|&include file for part 3\\
% |cdocspt4.tex|&include file for part 4\\
% |cdocsdrf.tex|&forwarding file for main file in draft mode\\
% |cdocsfi1.tex|&forwarding file for final version of chapter 1\\
% |cdocsfi2.tex|&forwarding file for final version of chapter 2\\
% \end{tabular}
% \end{center}
% Each of the eight files can be compiled directly by the \LaTeX{} compiler.
%
% %%%%%%%%%%%%%%%%%%%%%%%%%%%%%%%%%%%%%%
% \paragraph{Main File.}
%
% The main file is called |cdocsamp.tex|.
%
% Load the \textsf{childdoc} definitions and
% declare the filename for the main document:
%    \begin{macrocode}
\input{childdoc.def}
\childdocmain{}
%    \end{macrocode}

% Optional override for |\version| flag:
%    \begin{macrocode}
%%\ifchilddoc\else\providecommand{\version}{draft}\fi
%    \end{macrocode}

% Define the default values for the |\version| flag
% (|final| for the main file and |draft| for childs):
%    \begin{macrocode}
\ifchilddoc
\providecommand{\version}{draft}
\else
\providecommand{\version}{final}
\fi
%    \end{macrocode}

% Load the standard document class:
%    \begin{macrocode}
\documentclass[12pt]{article}
%    \end{macrocode}

% Start the document body:
%    \begin{macrocode}
\begin{document}
%    \end{macrocode}

% Declare a title page.
% Print title, part of document being processed and version flag:
%    \begin{macrocode}
\addtocounter{page}{-1}
\begin{center}
{\LARGE\bfseries{}childdoc example\par}
\vspace{1cm}
\ifchilddoc
\ifchilddocmanual part\else chapter\fi:
`\childdocname' of `\childdocjob'\par
\else
main document: `\childdocjob'\par
\fi
version: \version\par
\end{center}
\newpage
%    \end{macrocode}

% Manually include selected file,
% otherwise process as usual:
%    \begin{macrocode}
\ifchilddocmanual
\section*{part `\childdocname'}
\input{\childdocname}
\else
%    \end{macrocode}

% Include the two chapters:
%    \begin{macrocode}
\include{cdocsch1}
\include{cdocsch2}
%    \end{macrocode}

% Include the two parts unless only chapters should be displayed:
%    \begin{macrocode}
\ifchilddoc\else
\section{part three}
\input{cdocspt3}
\section{part four}
\input{cdocspt4}
\fi
%    \end{macrocode}

% Process as usual until here:
%    \begin{macrocode}
\fi
%    \end{macrocode}

% End of document body:
%    \begin{macrocode}
\end{document}
%    \end{macrocode}
%\iffalse
%</samplemain>
%\fi
%
% %%%%%%%%%%%%%%%%%%%%%%%%%%%%%%%%%%%%%%
% \paragraph{Chapter Include Files.}
%
% The include files are called |cdocsch1.tex| and |cdocsch2.tex|.
%
%\iffalse
%<*samplechap1|samplechap2>
%\fi

% Optional override for |\version| flag:
%    \begin{macrocode}
%%\providecommand{\version}{final}
%    \end{macrocode}

% Include the main document:
%    \begin{macrocode}
\input{childdoc.def}
\childdocof{cdocsamp}
%    \end{macrocode}

%\iffalse
%</samplechap1|samplechap2>
%\fi
%
%\iffalse
%<*samplechap1>
%\fi
% Some text for chapter 1:
%    \begin{macrocode}
\section{one}
some text in chapter one
%    \end{macrocode}

%\iffalse
%</samplechap1>
%\fi
% Some text for chapter 2:
%\iffalse
%<*samplechap2>
%\fi
%    \begin{macrocode}
\section{two}
more text in chapter two
%    \end{macrocode}

%\iffalse
%</samplechap2>
%\fi
%
% %%%%%%%%%%%%%%%%%%%%%%%%%%%%%%%%%%%%%%
% \paragraph{Part Include Files.}
%
% The include files are called |cdocspt3.tex| and |cdocspt4.tex|.
%
%\iffalse
%<*samplepart3|samplepart4>
%\fi

% Optional override for |\version| flag:
%    \begin{macrocode}
%%\providecommand{\version}{final}
%    \end{macrocode}

% Include the main document:
%    \begin{macrocode}
\input{childdoc.def}
\childdocby{cdocsamp}
%    \end{macrocode}

%\iffalse
%</samplepart3|samplepart4>
%\fi
%
%\iffalse
%<*samplepart3>
%\fi
% Some text for part 3:
%    \begin{macrocode}
some text in part three
%    \end{macrocode}

%\iffalse
%</samplepart3>
%\fi
% Some text for part 4:
%\iffalse
%<*samplepart4>
%\fi
%    \begin{macrocode}
more text in part four
%    \end{macrocode}

%\iffalse
%</samplepart4>
%\fi
%
% %%%%%%%%%%%%%%%%%%%%%%%%%%%%%%%%%%%%%%
% \paragraph{Forwarding for a Complete Draft.}
%
% The following forwarding file |cdocsdrf.tex|
% compiles the main document in draft mode:
%\iffalse
%<*sampledraft>
%\fi
%    \begin{macrocode}
\def\version{draft}
\input{childdoc.def}
\childdocforward{cdocsamp}
%    \end{macrocode}

%\iffalse
%</sampledraft>
%\fi
%
% %%%%%%%%%%%%%%%%%%%%%%%%%%%%%%%%%%%%%%
% \paragraph{Forwarding for Final Version of the Chapters.}
%
% The following forwarding files |cdocsfn1.tex| and |cdocsfn2.tex|
% (with identical content)
% compile the final versions of the child documents
% |cdocsch1.tex| and |cdocsch2.tex|, respectively:
%\iffalse
%<*samplefinal>
%\fi
%    \begin{macrocode}
\def\version{final}
\input{childdoc.def}
\childdocforwardprefix[cdocsamp]{cdocsfn}{cdocsch}
%    \end{macrocode}

%\iffalse
%</samplefinal>
%\fi
%
% %%%%%%%%%%%%%%%%%%%%%%%%%%%%%%%%%%%%%%
% \paragraph{Command Line Processing.}
%
% The following three command lines generate the output files
% |cdocscld|, |cdocscl1| and |cdocscl2|
% which should be identical to
% |cdocsdrf|, |cdocsch1| and |cdocsfn2|, respectively:
% \begin{center}
% \begin{tabular}{l}
% |latex -jobname cdocscld \|\\
% |  "\def\version{draft}\input{childdoc.def}\childdocforward{cdocsamp}"|\\
% |latex -jobname cdocscl1 \|\\
% |  "\input{childdoc.def}\childdocforward[cdocsamp]{cdocsch1}"|\\
% |latex -jobname cdocscl2 \|\\
% |  "\def\version{final}\input{childdoc.def}\childdocforward{cdocsch2}"|
% \end{tabular}
% \end{center}
% Note that the trailing backslash on each first line
% merely continues the input to the second line
% (for convenient cut ant paste).
% Furthermore, the command |latex| can be replaced by any
% of its alternative versions such as |pdflatex|.
%
% %%%%%%%%%%%%%%%%%%%%%%%%%%%%%%%%%%%%%%%%%%%%%%%%%%%%%%%%%%%%%%%%%%%%%%%%%%%%%%
% %%%%%%%%%%%%%%%%%%%%%%%%%%%%%%%%%%%%%%%%%%%%%%%%%%%%%%%%%%%%%%%%%%%%%%%%%%%%%%
% \section{Implementation}
%\iffalse
%<*package>
%\fi
%
% This section describes the definitions file |childdoc.def|.

% The definitions cannot be loaded using |\usepackage| or |\RequirePackage|
% which has a mechanism to prevent loading a style file more than once.
% When loading the definitions by means of |\input|
% multiple instances have to be prevented manually:
%\iffalse
%This code needs to be before the `\ProvidesFile' directive
%which is defined at the beginning of this file.
%Therefore it is also placed there and commented out here.
%</package>
%<*discard>
%\fi
%    \begin{macrocode}
\ifdefined\childdocmain\endinput\fi
%    \end{macrocode}
%\iffalse
%</discard>
%<*package>
%\fi
%
% \macro{\ifchilddoc}
% \macro{\ifchilddocmanual}
% The conditional |\ifchilddoc| tells whether a
% child (true) or main (false) document is being compiled.
% The conditional |\ifchilddocmanual| tells whether
% the |\includeonly| mechanism is used (false) or
% the selection of child files must be performed manually (true).
% The definitions initialise to false:
%    \begin{macrocode}
\newif\ifchilddoc
\newif\ifchilddocmanual
%    \end{macrocode}

% \macro{\childdocname}
% \macro{\childdocjob}
% The macro |\childdocname| stores the name of the main document
% to be compiled. The macro |\childdocjob| stores the name of
% the document on which the \LaTeX{} compiler was originally invoked.
% The content of |\jobname| cannot be compared
% to filenames specified in the source due to different catcodes.
% The following code rescans |\jobname|, stores the result
% in |\childdocname| and saves a copy in |\childdocjob|:
%    \begin{macrocode}
\edef\childdocname{\scantokens\expandafter{\jobname\noexpand}}
\let\childdocjob\childdocname
%    \end{macrocode}

% \macro{\childdocdisable}
% The macro |\childdocdisable| prevents the main file
% from being processed more than once.
% At this stage, the main document command |\childdocmain|
% is assumed to be called once again where it should do nothing.
% Any subsequent call to it should prevent
% a secondary processing of the main document
% It overwrites the forwarding commands
% |\childdocof| and |\childdocforward|
% with empty macros to prevent further inclusions of the main document:
%    \begin{macrocode}
\newcommand{\childdocdisable}
{
  \renewcommand{\childdocmain}[1]{\renewcommand{\childdocmain}[1]{\endinput}}
  \renewcommand{\childdocof}[1]{}
  \renewcommand{\childdocby}[2][]{}
  \renewcommand{\childdocforward}[2][]{}
  \renewcommand{\childdocdisable}{}
}
%    \end{macrocode}

% \macro{\childdocmain}
% The macro |\childdocmain| is to be called at the top of the main file
% with nothing or the main filename (without extension) as argument.
% First, it breaks loops.
% If the argument is not empty and does not match |\childdocname|
% (which is set by the first inclusion of |childdoc.def|),
% |\ifchilddoc| is set to true, |\includeonly| is applied to the child file
% and |\jobname| is set to the main file
% (for proper handling of |.aux| files):
%    \begin{macrocode}
\newcommand{\childdocmain}[1]
{
  \childdocdisable\childdocmain{}
  \if?#1?\else
    \begingroup
      \def\childdoctmp{#1}
      \ifx\childdoctmp\childdocname
        \def\childdoctmp{}
      \else
        \def\childdoctmp
        {
          \childdoctrue
          \includeonly{\childdocname}
          \def\childdocjob{#1}
          \def\jobname{#1}
        }
      \fi
      \expandafter
    \endgroup
    \childdoctmp
  \fi
}
%    \end{macrocode}

% \macro{\childdocof}
% The command |\childdocof| redirects
% compilation to the main file |#1|.
%    \begin{macrocode}
\newcommand{\childdocof}[1]
{
  \childdocdisable
  \childdoctrue
  \includeonly{\childdocname}
  \def\jobname{#1}
  \def\childdocjob{#1}
  \input{#1}
}
%    \end{macrocode}

% \macro{\childdocby}
% The command |\childdocby| ....
%    \begin{macrocode}
\newcommand{\childdocby}[2][]
{
  \childdocdisable
  \childdoctrue
  \childdocmanualtrue
  \if?#1?\else
    \def\jobname{#2}
  \fi
  \def\childdocjob{#2}
  \input{#2}
  \endinput
}
%    \end{macrocode}

% \macro{\childdocforward}
% The command |\childdocforward| redirects
% compilation to the main file or
% (if the optional argument is given) a child file.
% Parameters are set as if the main file
% or a child file starting with |\childdocof| was compiled.
% Then compilation is handed over to the main file:
%    \begin{macrocode}
\newcommand{\childdocforward}[2][]
{
  \begingroup
    \if?#1?
      \def\childdoctmp
      {
        \def\childdocname{#2}
        \def\childdocjob{#2}
        \def\jobname{#2}
        \input{#2}
        \endinput
      }
    \else
      \def\childdoctmp
      {
        \childdocdisable
        \def\childdocname{#2}
        \childdoctrue
        \includeonly{#2}
        \def\childdocjob{#1}
        \def\jobname{#1}
        \input{#1}
        \endinput
      }
    \fi
    \expandafter
  \endgroup
  \childdoctmp
}
%    \end{macrocode}

% \macro{\childdocforwardprefix}
% The command |\childdocforwardprefix| redirects
% compilation to the main or a child file by means of a pattern.
% The prefix |#1| in the current filename is replaced by |#2|
% and the suffix of the current filename is kept
% (it is assumed that the filename does not contain the substring `|~~~|'
% which is used as a delimiter).
% Compilation is handed over to the new file by |\childdocforward|:
%    \begin{macrocode}
\newcommand{\childdocforwardprefix}[3][]
{
  \begingroup
    \def\childdocextract #2##1~~~{\def\childdoctmp{\childdocforward[#1]{#3##1}}}
    \expandafter\childdocextract\childdocname~~~
    \expandafter
  \endgroup
  \childdoctmp
}
%    \end{macrocode}

% \macro{\childdoc}
% The deprecated macro |\childdoc| is a legacy version of |\childdocmain|:
%    \begin{macrocode}
\newcommand{\childdoc}{\childdocmain}
%    \end{macrocode}

% \macro{\childdocredirect}
% The deprecated macro |\childdocredirect| is a legacy version
% of |\childdocforward| and |\childdocforwardprefix|:
%    \begin{macrocode}
\newcommand{\childdocredirect}[2][]
{
  \begingroup
    \if?#1?
      \def\childdoctmp{\childdocforward{#2}}
    \else
      \def\childdoctmp{\childdocforwardprefix{#1}{#2}}
    \fi
    \expandafter
  \endgroup
  \childdoctmp
}
%    \end{macrocode}

%\iffalse
%</package>
%\fi
%
\endinput
|
and perform the replacements as outlined below.
Instead of |\childdocmain{|\textit{main}|}| add the following code
to the top of the main file:
%
\begin{center}
\begin{tabular}{l}
|\||ifdefined\childdocname\endinput\||fi\newif\ifchilddoc|\\
|\edef\childdocname{\scantokens\expandafter{\jobname\noexpand}}|\\
|\def\childdocmain{|\textit{main}|}\||ifx\childdocmain\childdocname\||else|\\
|\childdoctrue\includeonly{\childdocname}\let\jobname\childdocmain\||fi|\\
\end{tabular}
\end{center}
%
Instead of |\childdocof{|\textit{main}|}| just include the main file
at the top of each child file:
%
\begin{center}
|\input{|\textit{main}|}|
\end{center}
%
A simple redirection |\childdocforward{|\textit{dest}|}| is achieved by:
%
\begin{center}
|\def\jobname{|\textit{dest}|}\input{\jobname}|
\end{center}
%
The redirection with prefix
|\childdocforwardprefix[|\textit{prefix}|]{|\textit{dest}|}|
is accomplished by:
%
\begin{center}
\begin{tabular}{l}
|{\edef\jobname{\scantokens\expandafter{\jobname\noexpand}}|\\
|\def\redirectjob |\textit{prefix}|#1~~~{\gdef\jobname{|\textit{dest}|#1}}|\\
|\expandafter\redirectjob\jobname~~~}\input{\jobname}|
\end{tabular}
\end{center}

In an alternative approach,
child documents can be compiled by a specific command line
without additional code or specific definitions:
%
\begin{center}
|... -jobname "|\textit{target}|" "|[\textit{flags}]%
|\includeonly{|\textit{dest}|}\input{|\textit{main}|}"|
\end{center}
%

%%%%%%%%%%%%%%%%%%%%%%%%%%%%%%%%%%%%%%%%%%%%%%%%%%%%%%%%%%%%%%%%%%%%%%%%%%%%%%%%
%%%%%%%%%%%%%%%%%%%%%%%%%%%%%%%%%%%%%%%%%%%%%%%%%%%%%%%%%%%%%%%%%%%%%%%%%%%%%%%%
\section{Information}

%%%%%%%%%%%%%%%%%%%%%%%%%%%%%%%%%%%%%%%%%%%%%%%%%%%%%%%%%%%%%%%%%%%%%%%%%%%%%%%%
\subsection{Copyright}

Copyright \copyright{} 2017--2018 Niklas Beisert

This work may be distributed and/or modified under the
conditions of the \LaTeX{} Project Public License, either version 1.3
of this license or (at your option) any later version.
The latest version of this license is in
  \url{http://www.latex-project.org/lppl.txt}
and version 1.3 or later is part of all distributions of \LaTeX{}
version 2005/12/01 or later.

This work has the LPPL maintenance status `maintained'.

The Current Maintainer of this work is Niklas Beisert.

This work consists of the files |README.txt|, |childdoc.ins| and |childdoc.dtx|
as well as the derived files |childdoc.def|, |cdocsamp.tex|
with |cdocsch1.tex|, |cdocsch2.tex|, |cdocspt3.tex|, |cdocspt4.tex|,
|cdocsdrf.tex|, |cdocsfn1.tex|, |cdocsfn2.tex|
as well as |childdoc.pdf|.

%%%%%%%%%%%%%%%%%%%%%%%%%%%%%%%%%%%%%%%%%%%%%%%%%%%%%%%%%%%%%%%%%%%%%%%%%%%%%%%%
\subsection{Files and Installation}

The package consists of the files:
%
\begin{center}
\begin{tabular}{ll}
    |README.txt|   & readme file \\
    |childdoc.ins| & installation file \\
    |childdoc.dtx| & source file \\
    |childdoc.def| & definition file \\
    |cdocsamp.tex| & sample main file \\
    |cdocsch1.tex| & sample include file \\
    |cdocsch2.tex| & sample include file \\
    |cdocspt3.tex| & sample part file \\
    |cdocspt4.tex| & sample part file \\
    |cdocsdrf.tex| & sample redirection file \\
    |cdocsfn1.tex| & sample redirection file \\
    |cdocsfn2.tex| & sample redirection file \\
    |childdoc.pdf| & manual
\end{tabular}
\end{center}
%
The distribution consists of the files
|README.txt|, |childdoc.ins| and |childdoc.dtx|.
%
\begin{itemize}
\item
Run (pdf)\LaTeX{} on |childdoc.dtx|
to compile the manual |childdoc.pdf| (this file).
\item
Run \LaTeX{} on |childdoc.ins| to create the definitions file |childdoc.def|
and the sample |cdocsamp.tex| with include files
|cdocsch1.tex|, |cdocsch2.tex|, |cdocspt3.tex|, |cdocspt4.tex|,
|cdocsdrf.tex|, |cdocsfn1.tex|, |cdocsfn2.tex|.
Then copy the file |childdoc.def| to an appropriate directory of your \LaTeX{}
distribution, e.g.\ \textit{texmf-root}|/tex/latex/childdoc|.
\end{itemize}

%%%%%%%%%%%%%%%%%%%%%%%%%%%%%%%%%%%%%%%%%%%%%%%%%%%%%%%%%%%%%%%%%%%%%%%%%%%%%%%%
\subsection{Related CTAN Packages}

There are several other packages which offer a similar functionality:
%
\begin{itemize}
\item
The packages
\href{http://ctan.org/pkg/docmute}{\textsf{docmute}},
\href{http://ctan.org/pkg/includex}{\textsf{includex}} and
\href{http://ctan.org/pkg/standalone}{\textsf{standalone}}
provide commands to include only the document body of
a child file thus allowing both files to be compiled individually.
\item
The packages \href{http://ctan.org/pkg/subdocs}{\textsf{subdocs}}
and \href{http://ctan.org/pkg/subfiles}{\textsf{subfiles}}
provide structures in which the main and child documents can be
encapsulated and allowing them to be compiled individually.
The inclusion mechanism is different from the conventional |\include|.
\item
The package \href{http://ctan.org/pkg/combine}{\textsf{combine}}
is an elaborate solution to combine several documents into one.
\end{itemize}
%
See also the CTAN topic \href{http://ctan.org/topic/subdocs}{\textsf{subdocs}}
for further related packages.
The present package differs from the above solutions in that
a document structure constructed with the conventional |\include| mechanism
just needs two extra commands at the top of every file
such that all constituent files can be compiled individually.

%%%%%%%%%%%%%%%%%%%%%%%%%%%%%%%%%%%%%%%%%%%%%%%%%%%%%%%%%%%%%%%%%%%%%%%%%%%%%%%%
%\subsection{Feature Suggestions}
%
%The following is a list of features which may be useful for future
%versions of this package:
%%
%\begin{itemize}
%\item
%\ldots
%\end{itemize}

%%%%%%%%%%%%%%%%%%%%%%%%%%%%%%%%%%%%%%%%%%%%%%%%%%%%%%%%%%%%%%%%%%%%%%%%%%%%%%%%
\subsection{Revision History}

%%%%%%%%%%%%%%%%%%%%%%%%%%%%%%%%%%%%%%%%
\paragraph{v2.0:} 2018/12/30

\begin{itemize}
\item
immediate forward processing
\item
added |\childdocby| mechanism
\item
manual restructured
\end{itemize}

%%%%%%%%%%%%%%%%%%%%%%%%%%%%%%%%%%%%%%%%
\paragraph{v1.6:} 2018/01/17

\begin{itemize}
\item
application for development of include files
\item
corrections to manual
\end{itemize}

%%%%%%%%%%%%%%%%%%%%%%%%%%%%%%%%%%%%%%%%
\paragraph{v1.5:} 2017/05/21

\begin{itemize}
\item
more complete structuring introduced
\item
|\childdocof| introduced
\item
|\childdoc| renamed to |\childdocmain|
\item
|\childredirect| renamed to |\childdocforward| and |\childdocforwardprefix|
and functionality expanded
\end{itemize}

%%%%%%%%%%%%%%%%%%%%%%%%%%%%%%%%%%%%%%%%
\paragraph{v1.0:} 2017/04/27

\begin{itemize}
\item
manual and install package
\item
first version published on CTAN
\end{itemize}

%%%%%%%%%%%%%%%%%%%%%%%%%%%%%%%%%%%%%%%%
\paragraph{v0.6:} 2017/04/26

\begin{itemize}
\item
redirection mechanism added
\end{itemize}

%%%%%%%%%%%%%%%%%%%%%%%%%%%%%%%%%%%%%%%%
\paragraph{v0.5:} 2017/04/26

\begin{itemize}
\item
functionality in definition file
\end{itemize}


%%%%%%%%%%%%%%%%%%%%%%%%%%%%%%%%%%%%%%%%%%%%%%%%%%%%%%%%%%%%%%%%%%%%%%%%%%%%%%%%
%%%%%%%%%%%%%%%%%%%%%%%%%%%%%%%%%%%%%%%%%%%%%%%%%%%%%%%%%%%%%%%%%%%%%%%%%%%%%%%%
%%%%%%%%%%%%%%%%%%%%%%%%%%%%%%%%%%%%%%%%%%%%%%%%%%%%%%%%%%%%%%%%%%%%%%%%%%%%%%%%
\appendix

\settowidth\MacroIndent{\rmfamily\scriptsize 000\ }

 \DocInput{childdoc.dtx}

\end{document}
%</driver>
% \fi
%
% %%%%%%%%%%%%%%%%%%%%%%%%%%%%%%%%%%%%%%%%%%%%%%%%%%%%%%%%%%%%%%%%%%%%%%%%%%%%%%
% %%%%%%%%%%%%%%%%%%%%%%%%%%%%%%%%%%%%%%%%%%%%%%%%%%%%%%%%%%%%%%%%%%%%%%%%%%%%%%
% \section{Sample}
%\iffalse
%<*samplemain>
%\fi
%
% The following presents a sample document
% with two chapters, two parts, a title page,
% a compile flag as well as three forwarding files to set the flag.
% It consists of eight |.tex| files:
% \begin{center}
% \begin{tabular}{ll}
% |cdocsamp.tex|&main file\\
% |cdocsch1.tex|&include file for chapter 1\\
% |cdocsch2.tex|&include file for chapter 2\\
% |cdocspt3.tex|&include file for part 3\\
% |cdocspt4.tex|&include file for part 4\\
% |cdocsdrf.tex|&forwarding file for main file in draft mode\\
% |cdocsfi1.tex|&forwarding file for final version of chapter 1\\
% |cdocsfi2.tex|&forwarding file for final version of chapter 2\\
% \end{tabular}
% \end{center}
% Each of the eight files can be compiled directly by the \LaTeX{} compiler.
%
% %%%%%%%%%%%%%%%%%%%%%%%%%%%%%%%%%%%%%%
% \paragraph{Main File.}
%
% The main file is called |cdocsamp.tex|.
%
% Load the \textsf{childdoc} definitions and
% declare the filename for the main document:
%    \begin{macrocode}
% \iffalse
%
% childdoc.dtx Copyright (C) 2017-2018 Niklas Beisert
%
% This work may be distributed and/or modified under the
% conditions of the LaTeX Project Public License, either version 1.3
% of this license or (at your option) any later version.
% The latest version of this license is in
%   http://www.latex-project.org/lppl.txt
% and version 1.3 or later is part of all distributions of LaTeX
% version 2005/12/01 or later.
%
% This work has the LPPL maintenance status `maintained'.
%
% The Current Maintainer of this work is Niklas Beisert.
%
% This work consists of the files childdoc.dtx and childdoc.ins
% and the derived files childdoc.def and cdocsamp.tex with
% cdocsch1.tex, cdocsch2.tex, cdocsdrf.tex, cdocsfn1.tex, cdocsfn2.tex.
%
%<package>\ifdefined\childdocmain\endinput\fi
%<package>\ProvidesFile{childdoc.def}[2018/12/30 v2.0 child document driver]
%<samplemain>\ProvidesFile{cdocsamp.tex}[2018/12/30 v2.0 sample for childdoc]
%<*driver>
%\ProvidesFile{childdoc.drv}[2018/12/30 v2.0 childdoc reference manual file]
\PassOptionsToClass{10pt,a4paper}{article}
\documentclass{ltxdoc}

\usepackage[margin=35mm]{geometry}
\usepackage{hyperref}
\usepackage{hyperxmp}
\usepackage[usenames]{color}

\hypersetup{colorlinks=true}
\hypersetup{pdfstartview=FitH}
\hypersetup{pdfpagemode=UseNone}
\hypersetup{pdfsource={}}
\hypersetup{pdflang={en-UK}}
\hypersetup{pdfcopyright={Copyright 2017-2018 Niklas Beisert.
  This work may be distributed and/or modified under the
  conditions of the LaTeX Project Public License, either version 1.3
  of this license or (at your option) any later version.}}
\hypersetup{pdflicenseurl={http://www.latex-project.org/lppl.txt}}
\hypersetup{pdfcontactaddress={ETH Zurich, ITP, HIT K,
  Wolfgang-Pauli-Strasse 27}}
\hypersetup{pdfcontactpostcode={8093}}
\hypersetup{pdfcontactcity={Zurich}}
\hypersetup{pdfcontactcountry={Switzerland}}
\hypersetup{pdfcontactemail={nbeisert@itp.phys.ethz.ch}}
\hypersetup{pdfcontacturl={http://people.phys.ethz.ch/\xmptilde nbeisert/}}

\newcommand{\secref}[1]{\hyperref[#1]{section \ref*{#1}}}

\parskip1ex
\parindent0pt
\let\olditemize\itemize
\def\itemize{\olditemize\parskip0pt}

\begin{document}

\title{The \textsf{childdoc} Package}
\hypersetup{pdftitle={The childdoc Package}}
\author{Niklas Beisert\\[2ex]
  Institut f\"ur Theoretische Physik\\
  Eidgen\"ossische Technische Hochschule Z\"urich\\
  Wolfgang-Pauli-Strasse 27, 8093 Z\"urich, Switzerland\\[1ex]
  \href{mailto:nbeisert@itp.phys.ethz.ch}
  {\texttt{nbeisert@itp.phys.ethz.ch}}}
\hypersetup{pdfauthor={Niklas Beisert}}
\hypersetup{pdfsubject={Manual for the LaTeX2e Package childdoc}}
\date{30 December 2018, \textsf{v2.0}}
\maketitle

\begin{abstract}\noindent
\textsf{childdoc} is a \LaTeXe{} package
that enables the direct compilation
of document sections included by |\include|
to individual files.
\end{abstract}

\begingroup
\parskip0ex
\tableofcontents
\endgroup

%%%%%%%%%%%%%%%%%%%%%%%%%%%%%%%%%%%%%%%%%%%%%%%%%%%%%%%%%%%%%%%%%%%%%%%%%%%%%%%%
%%%%%%%%%%%%%%%%%%%%%%%%%%%%%%%%%%%%%%%%%%%%%%%%%%%%%%%%%%%%%%%%%%%%%%%%%%%%%%%%
\section{Introduction}

\LaTeX{} provides a mechanism to structure a large document (such as a book)
into a main file and several child files (containing the chapters)
using the |\include| command.
This mechanism is beneficial for documents
which span hundreds of pages in order to
make the source file(s) more manageable.
Moreover, compilation can be restricted to
selected child files by means of the |\includeonly| command.
The latter feature can be used to reduce the compilation time while editing
(this was significantly more useful in the earlier days of \LaTeX{})
or to generate a smaller document which is easier to navigate.
Another application of |\includeonly| is to generate
documents consisting of selected parts of the complete document.

However, there are a few drawbacks of the plain |\include| mechanism:
\begin{itemize}
\item
The child files cannot be compiled on their own,
they can only be compiled via the main file.
A naive editing environment
(such as a text editor with an option
to have the current file processed by \LaTeX)
may require one to switch to the main file before compiling;
attempting to compile the child file produces errors.
\item
The main file must be modified (each time)
to adjust the |\includeonly| command
to the present needs. This easily leaves the main file in a messy state.
\item
The generated document will always carry the filename
of the main document. This is inconvenient if
several child files are to be compiled and
to be kept for distribution.
\end{itemize}

The present package provides a simple interface
to make child files individually compilable by \LaTeX{}.
Compiling a child file then has the same effect as compiling
the main file with an |\includeonly| command
to select the appropriate child.
Moreover the generated document will carry the name of the child
rather than the main file.
This resolves all three above issues.

This feature is meant to make the editing of books,
thesis documents and lecture notes somewhat more convenient.
However, the package can also be used efficiently for
composing a series of documents (such as exercise sheets)
which are typically distributed individually.
It then assists the author in generating the individual documents
(potentially in different versions)
as well as a document containing the collected series.
Another application is in developing style files
or other kinds of included material
where compilation of the style file could redirect
to a sample or test file.

%%%%%%%%%%%%%%%%%%%%%%%%%%%%%%%%%%%%%%%%%%%%%%%%%%%%%%%%%%%%%%%%%%%%%%%%%%%%%%%%
%%%%%%%%%%%%%%%%%%%%%%%%%%%%%%%%%%%%%%%%%%%%%%%%%%%%%%%%%%%%%%%%%%%%%%%%%%%%%%%%
\section{Usage}

First of all, the package \textsf{childdoc} is \emph{not} a standard
\LaTeXe{} |.sty| style file! Therefore it needs to be invoked in
a non-standard way.

%%%%%%%%%%%%%%%%%%%%%%%%%%%%%%%%%%%%%%%%%%%%%%%%%%%%%%%%%%%%%%%%%%%%%%%%%%%%%%%%
\subsection{Included Files}
\label{sec:include}

%%%%%%%%%%%%%%%%%%%%%%%%%%%%%%%%%%%%%%%%
\DescribeMacro{\childdocmain}
To use the package, add the commands
\begin{center}
\begin{tabular}{l}
|\input{childdoc.def}|\\
|\childdocmain{}|\\
\end{tabular}
\end{center}
at the very top of the main \LaTeX{} file,
in particular \emph{before} the |\documentclass| statement!
The argument of |\childdocmain| should be left empty
(but it must be present).

%%%%%%%%%%%%%%%%%%%%%%%%%%%%%%%%%%%%%%%%
\DescribeMacro{\childdocof}
Furthermore, add the commands
\begin{center}
\begin{tabular}{l}
|\input{childdoc.def}|\\
|\childdocof{|\textit{main}|}|\\
\end{tabular}
\end{center}
at the top of every child file \textit{child}
which is included by |\include{|\textit{child}|}|
from within the main file
(or at least for those files to be compiled individually).
The argument \textit{main} must be the filename of the main file.

There are a couple of
considerations in setting up the main and child documents:

%%%%%%%%%%%%%%%%%%%%%%%%%%%%%%%%%%%%%%%%
\paragraph{Restrictions.}

Please note the following restrictions:
\begin{itemize}
\item
|\childdocmain| must be called with one argument \textit{main}
to ensure compatibility with earlier version of the package.
It must either be empty (|\childdocmain{}|)
or precisely match the filename of the main file in which it is specified.
See \secref{sec:detection} for further information.
\item
The filename \textit{main} must be specified without the |.tex| extension.
\item
The filename \textit{main} is case sensitive
(even in case-insensitive file systems)
due to internal string comparison.
\item
The argument \textit{main} should be fully expanded, it cannot be a macro.
\item
Subdirectories and special characters should be avoided in filenames.
\item
The command |\childdocmain{|\textit{main}|}| must be followed by a whitespace.
It should not be followed immediately by another command
or by a comment mark `|%|'.
This is because the \TeX{} parser reads the token immediately following
the argument of |\childdocmain| and puts it
at the beginning of every child section;
however, a white\-space is ignored.
\end{itemize}

%%%%%%%%%%%%%%%%%%%%%%%%%%%%%%%%%%%%%%%%
\paragraph{Content of Main File.}

It is advisable to place all content in the child files included by |\include|.
Any output contained in the main file will appear in all child documents
unless suppressed manually;
it cannot be suppressed automatically by the |\includeonly| directive
and thus should normally be avoided.
A method to include some content in the main file
by means of conditional processing is described in \secref{sec:conditional}.

%%%%%%%%%%%%%%%%%%%%%%%%%%%%%%%%%%%%%%%%
\paragraph{Page Numbering.}

When only a part of the document is compiled,
the appropriate numbering of pages
(as well as other status parameters)
is determined from the |.aux| files.
The latter contain information from previous passes.
However this information needs to propagate through
all intermediate child documents.
Therefore the page numbering in child documents may well
be inconsistent until the complete document is compiled at least once.

A useful (if unconventional) way to always ensure a consistent
page numbering is to restart the numbering in each child document
and denote the pages by `\textit{child}|.|\textit{page}'
where \textit{child} represents the chapter/section number of the child file.
This can be achieved by the command
|\numberwithin{page}{|\textit{child}|}|
of the \textsf{amsmath} package
where \textit{child} can be |chapter| or |section|
depending on the chosen structuring.
Alternatively, one can modify the macro |\thepage| appropriately
and reset the counter |page| at the start of each child file.

%%%%%%%%%%%%%%%%%%%%%%%%%%%%%%%%%%%%%%%%%%%%%%%%%%%%%%%%%%%%%%%%%%%%%%%%%%%%%%%%
\subsection{Conditional Processing}
\label{sec:conditional}

The package provides a mechanism to compile different versions
of a document. To customise the versions further some conditional processing
can come in handy to distinguish which version is being compiled.
The package provides two macros to describe the compilation context:

%%%%%%%%%%%%%%%%%%%%%%%%%%%%%%%%%%%%%%%%
\DescribeMacro{\ifchilddoc}
The conditional |\ifchilddoc| distinguishes between the compilation of
child documents and the main document:
%
\begin{center}
|\ifchilddoc |\textit{child-code}| |[|\||else |\textit{main-code}]| \||fi|
\end{center}

%%%%%%%%%%%%%%%%%%%%%%%%%%%%%%%%%%%%%%%%
\DescribeMacro{\childdocname}
\DescribeMacro{\childdocjob}
The macro |\childdocname| contains the filename (without extension)
of the main or child file being processed.
Note that |\childdocjob| will always contain the name of the main file.

%%%%%%%%%%%%%%%%%%%%%%%%%%%%%%%%%%%%%%%%
\paragraph{Title Page.}

Conditional processing can be used to include a title or banner page
in the main document when proper precautions are taken.
Importantly, the code in the main file should ensure that the page counter
(as well as other status parameters which are stored in the |.aux| files)
takes the same value after the conditional processing.
Otherwise the page numbers may take divergent values
depending on which part is compiled.

For example, a title page could be declared by:
%
\begin{center}
\begin{tabular}{l}
|\ifchilddoc\||else|\\
|\addtocounter{page}{-1}|\\
\textit{code for title page}\\
|\newpage|\\
|\||fi|
\end{tabular}
\end{center}
%
A banner page for the child documents can be generated by:
%
\begin{center}
\begin{tabular}{l}
|\ifchilddoc|\\
|\addtocounter{page}{-1}|\\
\textit{code for banner page}\\
|\newpage|\\
|\||fi|
\end{tabular}
\end{center}
%
Here one could write a message such as:
\begin{center}
|This is the part \childdocname{} of \childdocjob{}.|
\end{center}

%%%%%%%%%%%%%%%%%%%%%%%%%%%%%%%%%%%%%%%%%%%%%%%%%%%%%%%%%%%%%%%%%%%%%%%%%%%%%%%%
\subsection{Flags}
\label{sec:flags}

The package makes it easy to generate different versions
of the main or child documents.
To this end compilation flags can be defined
and assigned different default values.
They will be particularly useful in conjunction
with the forwarding mechanism described in \secref{sec:forward}.

For example, it may be useful to have a flag |\version|
which can be set to |draft| or |final|.
The document source will contain some conditional code
depending on the value of |\version|.
Suppose further, the flag should default to |final| for the main file
and to |draft| for child files
which is a natural assignment for editing the document.
This is achieved by placing the following code
in the preamble of the main document
(below the |\childdocmain| directive):
%
\begin{center}
\begin{tabular}{l}
|\ifchilddoc|\\
|\providecommand{\version}{draft}|\\
|\||else|\\
|\providecommand{\version}{final}|\\
|\||fi|
\end{tabular}
\end{center}
%
The definition by |\providecommand| makes sure
that previous definitions are not overwritten.
Further statements |\providecommand{\version}{...}|
can thus be added before the above code to override it.

For the main file, one might add a line
(between |\childdocmain| and the above block)
%
\begin{center}
|%\ifchilddoc\||else\providecommand{\version}{draft}\||fi|
\end{center}
%
which can be uncommented to produce a draft version.
Likewise one can add a line to the very top of a child file
(above the |\childdocof{|\textit{main}|}| directive)
%
\begin{center}
|%\providecommand{\version}{final}|
\end{center}
%
which can be uncommented to produce the final version of this child document.

%%%%%%%%%%%%%%%%%%%%%%%%%%%%%%%%%%%%%%%%%%%%%%%%%%%%%%%%%%%%%%%%%%%%%%%%%%%%%%%%
\subsection{Forwarding}
\label{sec:forward}

Different versions of the main or child documents
using compilation flags as described in \secref{sec:flags}
can be (permanently) stored in different files
for convenient compilation, viewing and distribution.
To this end, the package defines a command
to pass on compilation to a different file:

%%%%%%%%%%%%%%%%%%%%%%%%%%%%%%%%%%%%%%%%
\DescribeMacro{\childdocforward}
The command |\childdocforward| redirects processing to
another source file:
%
\begin{center}
\begin{tabular}{l}
|\input{childdoc.def}|\\
|\childdocforward[|\textit{main}|]{|\textit{dest}|}|\\
\end{tabular}
\end{center}
%
The argument \textit{dest} is the destination file
(without extension).
It should be the main file or one of the child files.
Note that further \textsf{childdoc} directives
such as |\childdocof| and |\childdocforward|
in the indicated file will be processed in this form.
The optional argument \textit{main}
passes on directly to the main file \textit{main}
while pretending to compile the child \textit{dest}.
This form behaves as if \textit{dest}
issues |\childdocof{|\textit{main}|}| right away,
and no further \textsf{childdoc} directives will be processed.

%%%%%%%%%%%%%%%%%%%%%%%%%%%%%%%%%%%%%%%%
\DescribeMacro{\...prefix}
In the alternative form |\childdocforwardprefix|,
%
\begin{center}
\begin{tabular}{l}
|\input{childdoc.def}|\\
|\childdocforwardprefix[|\textit{main}|]{|\textit{prefix}|}{|\textit{dest}|}|
\end{tabular}
\end{center}
%
the destination file is determined by a pattern
depending on the current file:
To make this work, the current file must be called
`{\textit{prefix}\hspace{0.2em}\textit{suffix}}'
with \textit{prefix} matching precisely the argument.
Processing is then passed on to the file
`{\textit{dest}\hspace{0.2em}\textit{suffix}}'.
Surely, the same effect is achieved by
directly specifying the
argument `{\textit{dest}\hspace{0.2em}\textit{suffix}}'
in the first form.
However, that requires to set up a different file
for each child. With the alternative form of the command
all these files can have exactly the same content
which simplifies setting them up and maintaining them.

For example, the following file |draft.tex|
with a compilation flag |\version| as described in \secref{sec:flags}
compiles the main document as a draft:
%
\begin{center}
\begin{tabular}{l}
|\def\version{draft}|\\
|\input{childdoc.def}|\\
|\childdocforward{|\textit{main}|}|
\end{tabular}
\end{center}
%
Likewise, the following files |final|\textit{nn}|.tex|
compile the final version of the child document
|child|\textit{nn}|.tex|:
%
\begin{center}
\begin{tabular}{l}
|\def\version{final}|\\
|\input{childdoc.def}|\\
|\childdocforwardprefix{final}{child}|
\end{tabular}
\end{center}
%

Note that when several versions of a main file and/or of each child file
are to be generated, it may be convenient to set up a |Makefile| or
shell script to automatise the process.

%%%%%%%%%%%%%%%%%%%%%%%%%%%%%%%%%%%%%%%%%%%%%%%%%%%%%%%%%%%%%%%%%%%%%%%%%%%%%%%%
\subsection{Command Line Processing}
\label{sec:commandline}

The effect of redirection files can also be achieved by invoking
the \LaTeX{} compiler with a more elaborate command line.
Most conveniently this should be done as part
of a shell script or a |Makefile|.

When using \textsf{childdoc} in the main file, the following
command lines effectively perform a redirection
(note that depending on the shell being used,
backslashes may have to be doubled: `|\|' $\to$ `|\\|'):
%
\begin{center}
|... -jobname "|\textit{target}|" |\\|"|[\textit{flags}]%
|\input{childdoc.def}\childdocforward[|\textit{main}|]{|\textit{dest}|}"|
\end{center}
%
Here \textit{target} is the name of the output file,
\textit{main} is the name of the main file
and \textit{dest} is the name of the main or child file to be processed
(all filenames without extensions).
The optional argument \textit{main} can be omitted
if \textit{main} matches \textit{dest}.
Optionally, compilation \textit{flags} can be defined via |\def| commands.
This command line makes the \TeX{} engine believe
it is compiling the file \textit{target}
whose content is specified as the latter parameter.
The provided code then forwards the processing to
\textit{main} or \textit{dest} as described in \secref{sec:forward}.

%%%%%%%%%%%%%%%%%%%%%%%%%%%%%%%%%%%%%%%%%%%%%%%%%%%%%%%%%%%%%%%%%%%%%%%%%%%%%%%%
\subsection{Include by Input}
\label{sec:input}

Including child documents by |\include| has some restrictions by design.
Most notably, the content of a child document always occupies
its own set of pages; pages cannot be shared between child documents.
Usually, this behaviour makes perfect sense
because each child document contain an essential part of the document.
However, in some situations it may be desirable to compose
a document from a collection of parts
without having mandatory page breaks between then.
For this case, the package
provides a mechanism to include parts
by |\input| which can also be processed individually.
However, by construction this mechanism
requires manual handling of the content to be output.

%%%%%%%%%%%%%%%%%%%%%%%%%%%%%%%%%%%%%%%%
\DescribeMacro{\ifchilddocmanual}
The main file should be prepared as usual, see \secref{sec:include}.
However, the document body must make a distinction
between processing of an individual part and of the main document, e.g.:
%
\begin{center}
\begin{tabular}{l}
|\ifchilddocmanual|\\
|\input{\childdocname}|\\
|\||else|\\
\textit{document body with }|\input{|\textit{part}|}|\\
|\||fi|
\end{tabular}
\end{center}
%
The conditional |\ifchilddocmanual| is true whenever
a part to be included by |\input| is being compiled,
and the name of the part is stored in |\childdocname|.

%%%%%%%%%%%%%%%%%%%%%%%%%%%%%%%%%%%%%%%%
\DescribeMacro{\childdocby}
Each part to be included by |\input| should start with:
%
\begin{center}
\begin{tabular}{l}
|\input{childdoc.def}|\\
|\childdocby{|\textit{main}|}|\\
\end{tabular}
\end{center}
%
The directive |\childdocby| is similar to |\childdocof|
described in \secref{sec:include},
but the subsequent selection of content must be done manually.
To that end, both |\ifchilddoc| and |\ifchilddocmanual|
will be true upon processing of a part,
and the name of the part is stored in |\childdocname|.
Note that |\jobname| will be set to the filename of the current part
so that each part receives an individual |.aux| file
that does not interfere with the |.aux| file(s) of the main document.
This behaviour can be altered by the alternative form
|\childdocby[*]{|\textit{main}|}| (with a non-empty optional argument)
which uses the |.aux| file of the main document
by setting |\jobname| to \textit{main}.

%%%%%%%%%%%%%%%%%%%%%%%%%%%%%%%%%%%%%%%%%%%%%%%%%%%%%%%%%%%%%%%%%%%%%%%%%%%%%%%%
\subsection{Driver Development}
\label{sec:driver}

The \textsf{childdoc} mechanism can also be use for the development
of definition files such as \LaTeX{} styles or classes.
This case differs from the above setup with multiple parts
included by |\include| in that no |\includeonly| should be invoked.
This can be achieved by starting the include file
(before |\ProvidesPackage|) with:
%
\begin{center}
\begin{tabular}{l}
|\input{childdoc.def}|\\
|\childdocforward{|\textit{main}|}|\\
\end{tabular}
\end{center}
%
or alternatively with:
%
\begin{center}
\begin{tabular}{l}
|\input{childdoc.def}|\\
|\childdocby{|\textit{main}|}|\\
\end{tabular}
\end{center}
%
Both forms have slightly different effects as described above.
The main file is prepared as usual, see \secref{sec:include}.

%%%%%%%%%%%%%%%%%%%%%%%%%%%%%%%%%%%%%%%%%%%%%%%%%%%%%%%%%%%%%%%%%%%%%%%%%%%%%%%%
\subsection{Legacy Detection}
\label{sec:detection}

The directive |\childdocmain| in the main file can detect
whether the complete document or merely a child is to be compiled
even without using the directive |\childdocof|.
This method is deprecated because it is less robust
and there is no compelling reason to use it;
it is merely provided for backward compatibility
and it may be removed in future versions.

If the detection mechanism is to be used,
it is mandatory to correctly specify
the filename of the main file as the argument of |\childdocmain|:
%
\begin{center}
\begin{tabular}{l}
|\input{childdoc.def}|\\
|\childdocmain{|\textit{main}|}|\\
\end{tabular}
\end{center}
%
If |\jobname| does not match the argument \textit{main} of |\childdocmain|,
it is assumed that |\jobname| points to the child file to be compiled.
When using |\childdocmain| with the main file specified as argument,
it suffices to start a child file
with just |\input{|\textit{main}|}|
without loading of the package and using |\childdocof|.
If instead all processing is done
with the appropriate \textsf{childdoc} directives,
the argument of \textit{main} of |\childdocmain| can be empty.

An alternative version of the command line processing described
in \secref{sec:commandline} using the detection mechanism reads:
%
\begin{center}
|... -jobname "|\textit{target}|" "|[\textit{flags}]%
[|\def\jobname{|\textit{dest}|}|]|\input{|\textit{main}|}"|
\end{center}

%%%%%%%%%%%%%%%%%%%%%%%%%%%%%%%%%%%%%%%%%%%%%%%%%%%%%%%%%%%%%%%%%%%%%%%%%%%%%%%%
\subsection{Manual Code}
\label{sec:manual}

In case one cannot be certain whether the definitions file |childdoc.def|
is installed on the target \TeX{} distribution
and one prefers not to ship it,
it is conceivable to paste a few relevant commands into the sources.

To that end, drop all statements |\input{childdoc.def}|
and perform the replacements as outlined below.
Instead of |\childdocmain{|\textit{main}|}| add the following code
to the top of the main file:
%
\begin{center}
\begin{tabular}{l}
|\||ifdefined\childdocname\endinput\||fi\newif\ifchilddoc|\\
|\edef\childdocname{\scantokens\expandafter{\jobname\noexpand}}|\\
|\def\childdocmain{|\textit{main}|}\||ifx\childdocmain\childdocname\||else|\\
|\childdoctrue\includeonly{\childdocname}\let\jobname\childdocmain\||fi|\\
\end{tabular}
\end{center}
%
Instead of |\childdocof{|\textit{main}|}| just include the main file
at the top of each child file:
%
\begin{center}
|\input{|\textit{main}|}|
\end{center}
%
A simple redirection |\childdocforward{|\textit{dest}|}| is achieved by:
%
\begin{center}
|\def\jobname{|\textit{dest}|}\input{\jobname}|
\end{center}
%
The redirection with prefix
|\childdocforwardprefix[|\textit{prefix}|]{|\textit{dest}|}|
is accomplished by:
%
\begin{center}
\begin{tabular}{l}
|{\edef\jobname{\scantokens\expandafter{\jobname\noexpand}}|\\
|\def\redirectjob |\textit{prefix}|#1~~~{\gdef\jobname{|\textit{dest}|#1}}|\\
|\expandafter\redirectjob\jobname~~~}\input{\jobname}|
\end{tabular}
\end{center}

In an alternative approach,
child documents can be compiled by a specific command line
without additional code or specific definitions:
%
\begin{center}
|... -jobname "|\textit{target}|" "|[\textit{flags}]%
|\includeonly{|\textit{dest}|}\input{|\textit{main}|}"|
\end{center}
%

%%%%%%%%%%%%%%%%%%%%%%%%%%%%%%%%%%%%%%%%%%%%%%%%%%%%%%%%%%%%%%%%%%%%%%%%%%%%%%%%
%%%%%%%%%%%%%%%%%%%%%%%%%%%%%%%%%%%%%%%%%%%%%%%%%%%%%%%%%%%%%%%%%%%%%%%%%%%%%%%%
\section{Information}

%%%%%%%%%%%%%%%%%%%%%%%%%%%%%%%%%%%%%%%%%%%%%%%%%%%%%%%%%%%%%%%%%%%%%%%%%%%%%%%%
\subsection{Copyright}

Copyright \copyright{} 2017--2018 Niklas Beisert

This work may be distributed and/or modified under the
conditions of the \LaTeX{} Project Public License, either version 1.3
of this license or (at your option) any later version.
The latest version of this license is in
  \url{http://www.latex-project.org/lppl.txt}
and version 1.3 or later is part of all distributions of \LaTeX{}
version 2005/12/01 or later.

This work has the LPPL maintenance status `maintained'.

The Current Maintainer of this work is Niklas Beisert.

This work consists of the files |README.txt|, |childdoc.ins| and |childdoc.dtx|
as well as the derived files |childdoc.def|, |cdocsamp.tex|
with |cdocsch1.tex|, |cdocsch2.tex|, |cdocspt3.tex|, |cdocspt4.tex|,
|cdocsdrf.tex|, |cdocsfn1.tex|, |cdocsfn2.tex|
as well as |childdoc.pdf|.

%%%%%%%%%%%%%%%%%%%%%%%%%%%%%%%%%%%%%%%%%%%%%%%%%%%%%%%%%%%%%%%%%%%%%%%%%%%%%%%%
\subsection{Files and Installation}

The package consists of the files:
%
\begin{center}
\begin{tabular}{ll}
    |README.txt|   & readme file \\
    |childdoc.ins| & installation file \\
    |childdoc.dtx| & source file \\
    |childdoc.def| & definition file \\
    |cdocsamp.tex| & sample main file \\
    |cdocsch1.tex| & sample include file \\
    |cdocsch2.tex| & sample include file \\
    |cdocspt3.tex| & sample part file \\
    |cdocspt4.tex| & sample part file \\
    |cdocsdrf.tex| & sample redirection file \\
    |cdocsfn1.tex| & sample redirection file \\
    |cdocsfn2.tex| & sample redirection file \\
    |childdoc.pdf| & manual
\end{tabular}
\end{center}
%
The distribution consists of the files
|README.txt|, |childdoc.ins| and |childdoc.dtx|.
%
\begin{itemize}
\item
Run (pdf)\LaTeX{} on |childdoc.dtx|
to compile the manual |childdoc.pdf| (this file).
\item
Run \LaTeX{} on |childdoc.ins| to create the definitions file |childdoc.def|
and the sample |cdocsamp.tex| with include files
|cdocsch1.tex|, |cdocsch2.tex|, |cdocspt3.tex|, |cdocspt4.tex|,
|cdocsdrf.tex|, |cdocsfn1.tex|, |cdocsfn2.tex|.
Then copy the file |childdoc.def| to an appropriate directory of your \LaTeX{}
distribution, e.g.\ \textit{texmf-root}|/tex/latex/childdoc|.
\end{itemize}

%%%%%%%%%%%%%%%%%%%%%%%%%%%%%%%%%%%%%%%%%%%%%%%%%%%%%%%%%%%%%%%%%%%%%%%%%%%%%%%%
\subsection{Related CTAN Packages}

There are several other packages which offer a similar functionality:
%
\begin{itemize}
\item
The packages
\href{http://ctan.org/pkg/docmute}{\textsf{docmute}},
\href{http://ctan.org/pkg/includex}{\textsf{includex}} and
\href{http://ctan.org/pkg/standalone}{\textsf{standalone}}
provide commands to include only the document body of
a child file thus allowing both files to be compiled individually.
\item
The packages \href{http://ctan.org/pkg/subdocs}{\textsf{subdocs}}
and \href{http://ctan.org/pkg/subfiles}{\textsf{subfiles}}
provide structures in which the main and child documents can be
encapsulated and allowing them to be compiled individually.
The inclusion mechanism is different from the conventional |\include|.
\item
The package \href{http://ctan.org/pkg/combine}{\textsf{combine}}
is an elaborate solution to combine several documents into one.
\end{itemize}
%
See also the CTAN topic \href{http://ctan.org/topic/subdocs}{\textsf{subdocs}}
for further related packages.
The present package differs from the above solutions in that
a document structure constructed with the conventional |\include| mechanism
just needs two extra commands at the top of every file
such that all constituent files can be compiled individually.

%%%%%%%%%%%%%%%%%%%%%%%%%%%%%%%%%%%%%%%%%%%%%%%%%%%%%%%%%%%%%%%%%%%%%%%%%%%%%%%%
%\subsection{Feature Suggestions}
%
%The following is a list of features which may be useful for future
%versions of this package:
%%
%\begin{itemize}
%\item
%\ldots
%\end{itemize}

%%%%%%%%%%%%%%%%%%%%%%%%%%%%%%%%%%%%%%%%%%%%%%%%%%%%%%%%%%%%%%%%%%%%%%%%%%%%%%%%
\subsection{Revision History}

%%%%%%%%%%%%%%%%%%%%%%%%%%%%%%%%%%%%%%%%
\paragraph{v2.0:} 2018/12/30

\begin{itemize}
\item
immediate forward processing
\item
added |\childdocby| mechanism
\item
manual restructured
\end{itemize}

%%%%%%%%%%%%%%%%%%%%%%%%%%%%%%%%%%%%%%%%
\paragraph{v1.6:} 2018/01/17

\begin{itemize}
\item
application for development of include files
\item
corrections to manual
\end{itemize}

%%%%%%%%%%%%%%%%%%%%%%%%%%%%%%%%%%%%%%%%
\paragraph{v1.5:} 2017/05/21

\begin{itemize}
\item
more complete structuring introduced
\item
|\childdocof| introduced
\item
|\childdoc| renamed to |\childdocmain|
\item
|\childredirect| renamed to |\childdocforward| and |\childdocforwardprefix|
and functionality expanded
\end{itemize}

%%%%%%%%%%%%%%%%%%%%%%%%%%%%%%%%%%%%%%%%
\paragraph{v1.0:} 2017/04/27

\begin{itemize}
\item
manual and install package
\item
first version published on CTAN
\end{itemize}

%%%%%%%%%%%%%%%%%%%%%%%%%%%%%%%%%%%%%%%%
\paragraph{v0.6:} 2017/04/26

\begin{itemize}
\item
redirection mechanism added
\end{itemize}

%%%%%%%%%%%%%%%%%%%%%%%%%%%%%%%%%%%%%%%%
\paragraph{v0.5:} 2017/04/26

\begin{itemize}
\item
functionality in definition file
\end{itemize}


%%%%%%%%%%%%%%%%%%%%%%%%%%%%%%%%%%%%%%%%%%%%%%%%%%%%%%%%%%%%%%%%%%%%%%%%%%%%%%%%
%%%%%%%%%%%%%%%%%%%%%%%%%%%%%%%%%%%%%%%%%%%%%%%%%%%%%%%%%%%%%%%%%%%%%%%%%%%%%%%%
%%%%%%%%%%%%%%%%%%%%%%%%%%%%%%%%%%%%%%%%%%%%%%%%%%%%%%%%%%%%%%%%%%%%%%%%%%%%%%%%
\appendix

\settowidth\MacroIndent{\rmfamily\scriptsize 000\ }

 \DocInput{childdoc.dtx}

\end{document}
%</driver>
% \fi
%
% %%%%%%%%%%%%%%%%%%%%%%%%%%%%%%%%%%%%%%%%%%%%%%%%%%%%%%%%%%%%%%%%%%%%%%%%%%%%%%
% %%%%%%%%%%%%%%%%%%%%%%%%%%%%%%%%%%%%%%%%%%%%%%%%%%%%%%%%%%%%%%%%%%%%%%%%%%%%%%
% \section{Sample}
%\iffalse
%<*samplemain>
%\fi
%
% The following presents a sample document
% with two chapters, two parts, a title page,
% a compile flag as well as three forwarding files to set the flag.
% It consists of eight |.tex| files:
% \begin{center}
% \begin{tabular}{ll}
% |cdocsamp.tex|&main file\\
% |cdocsch1.tex|&include file for chapter 1\\
% |cdocsch2.tex|&include file for chapter 2\\
% |cdocspt3.tex|&include file for part 3\\
% |cdocspt4.tex|&include file for part 4\\
% |cdocsdrf.tex|&forwarding file for main file in draft mode\\
% |cdocsfi1.tex|&forwarding file for final version of chapter 1\\
% |cdocsfi2.tex|&forwarding file for final version of chapter 2\\
% \end{tabular}
% \end{center}
% Each of the eight files can be compiled directly by the \LaTeX{} compiler.
%
% %%%%%%%%%%%%%%%%%%%%%%%%%%%%%%%%%%%%%%
% \paragraph{Main File.}
%
% The main file is called |cdocsamp.tex|.
%
% Load the \textsf{childdoc} definitions and
% declare the filename for the main document:
%    \begin{macrocode}
\input{childdoc.def}
\childdocmain{}
%    \end{macrocode}

% Optional override for |\version| flag:
%    \begin{macrocode}
%%\ifchilddoc\else\providecommand{\version}{draft}\fi
%    \end{macrocode}

% Define the default values for the |\version| flag
% (|final| for the main file and |draft| for childs):
%    \begin{macrocode}
\ifchilddoc
\providecommand{\version}{draft}
\else
\providecommand{\version}{final}
\fi
%    \end{macrocode}

% Load the standard document class:
%    \begin{macrocode}
\documentclass[12pt]{article}
%    \end{macrocode}

% Start the document body:
%    \begin{macrocode}
\begin{document}
%    \end{macrocode}

% Declare a title page.
% Print title, part of document being processed and version flag:
%    \begin{macrocode}
\addtocounter{page}{-1}
\begin{center}
{\LARGE\bfseries{}childdoc example\par}
\vspace{1cm}
\ifchilddoc
\ifchilddocmanual part\else chapter\fi:
`\childdocname' of `\childdocjob'\par
\else
main document: `\childdocjob'\par
\fi
version: \version\par
\end{center}
\newpage
%    \end{macrocode}

% Manually include selected file,
% otherwise process as usual:
%    \begin{macrocode}
\ifchilddocmanual
\section*{part `\childdocname'}
\input{\childdocname}
\else
%    \end{macrocode}

% Include the two chapters:
%    \begin{macrocode}
\include{cdocsch1}
\include{cdocsch2}
%    \end{macrocode}

% Include the two parts unless only chapters should be displayed:
%    \begin{macrocode}
\ifchilddoc\else
\section{part three}
\input{cdocspt3}
\section{part four}
\input{cdocspt4}
\fi
%    \end{macrocode}

% Process as usual until here:
%    \begin{macrocode}
\fi
%    \end{macrocode}

% End of document body:
%    \begin{macrocode}
\end{document}
%    \end{macrocode}
%\iffalse
%</samplemain>
%\fi
%
% %%%%%%%%%%%%%%%%%%%%%%%%%%%%%%%%%%%%%%
% \paragraph{Chapter Include Files.}
%
% The include files are called |cdocsch1.tex| and |cdocsch2.tex|.
%
%\iffalse
%<*samplechap1|samplechap2>
%\fi

% Optional override for |\version| flag:
%    \begin{macrocode}
%%\providecommand{\version}{final}
%    \end{macrocode}

% Include the main document:
%    \begin{macrocode}
\input{childdoc.def}
\childdocof{cdocsamp}
%    \end{macrocode}

%\iffalse
%</samplechap1|samplechap2>
%\fi
%
%\iffalse
%<*samplechap1>
%\fi
% Some text for chapter 1:
%    \begin{macrocode}
\section{one}
some text in chapter one
%    \end{macrocode}

%\iffalse
%</samplechap1>
%\fi
% Some text for chapter 2:
%\iffalse
%<*samplechap2>
%\fi
%    \begin{macrocode}
\section{two}
more text in chapter two
%    \end{macrocode}

%\iffalse
%</samplechap2>
%\fi
%
% %%%%%%%%%%%%%%%%%%%%%%%%%%%%%%%%%%%%%%
% \paragraph{Part Include Files.}
%
% The include files are called |cdocspt3.tex| and |cdocspt4.tex|.
%
%\iffalse
%<*samplepart3|samplepart4>
%\fi

% Optional override for |\version| flag:
%    \begin{macrocode}
%%\providecommand{\version}{final}
%    \end{macrocode}

% Include the main document:
%    \begin{macrocode}
\input{childdoc.def}
\childdocby{cdocsamp}
%    \end{macrocode}

%\iffalse
%</samplepart3|samplepart4>
%\fi
%
%\iffalse
%<*samplepart3>
%\fi
% Some text for part 3:
%    \begin{macrocode}
some text in part three
%    \end{macrocode}

%\iffalse
%</samplepart3>
%\fi
% Some text for part 4:
%\iffalse
%<*samplepart4>
%\fi
%    \begin{macrocode}
more text in part four
%    \end{macrocode}

%\iffalse
%</samplepart4>
%\fi
%
% %%%%%%%%%%%%%%%%%%%%%%%%%%%%%%%%%%%%%%
% \paragraph{Forwarding for a Complete Draft.}
%
% The following forwarding file |cdocsdrf.tex|
% compiles the main document in draft mode:
%\iffalse
%<*sampledraft>
%\fi
%    \begin{macrocode}
\def\version{draft}
\input{childdoc.def}
\childdocforward{cdocsamp}
%    \end{macrocode}

%\iffalse
%</sampledraft>
%\fi
%
% %%%%%%%%%%%%%%%%%%%%%%%%%%%%%%%%%%%%%%
% \paragraph{Forwarding for Final Version of the Chapters.}
%
% The following forwarding files |cdocsfn1.tex| and |cdocsfn2.tex|
% (with identical content)
% compile the final versions of the child documents
% |cdocsch1.tex| and |cdocsch2.tex|, respectively:
%\iffalse
%<*samplefinal>
%\fi
%    \begin{macrocode}
\def\version{final}
\input{childdoc.def}
\childdocforwardprefix[cdocsamp]{cdocsfn}{cdocsch}
%    \end{macrocode}

%\iffalse
%</samplefinal>
%\fi
%
% %%%%%%%%%%%%%%%%%%%%%%%%%%%%%%%%%%%%%%
% \paragraph{Command Line Processing.}
%
% The following three command lines generate the output files
% |cdocscld|, |cdocscl1| and |cdocscl2|
% which should be identical to
% |cdocsdrf|, |cdocsch1| and |cdocsfn2|, respectively:
% \begin{center}
% \begin{tabular}{l}
% |latex -jobname cdocscld \|\\
% |  "\def\version{draft}\input{childdoc.def}\childdocforward{cdocsamp}"|\\
% |latex -jobname cdocscl1 \|\\
% |  "\input{childdoc.def}\childdocforward[cdocsamp]{cdocsch1}"|\\
% |latex -jobname cdocscl2 \|\\
% |  "\def\version{final}\input{childdoc.def}\childdocforward{cdocsch2}"|
% \end{tabular}
% \end{center}
% Note that the trailing backslash on each first line
% merely continues the input to the second line
% (for convenient cut ant paste).
% Furthermore, the command |latex| can be replaced by any
% of its alternative versions such as |pdflatex|.
%
% %%%%%%%%%%%%%%%%%%%%%%%%%%%%%%%%%%%%%%%%%%%%%%%%%%%%%%%%%%%%%%%%%%%%%%%%%%%%%%
% %%%%%%%%%%%%%%%%%%%%%%%%%%%%%%%%%%%%%%%%%%%%%%%%%%%%%%%%%%%%%%%%%%%%%%%%%%%%%%
% \section{Implementation}
%\iffalse
%<*package>
%\fi
%
% This section describes the definitions file |childdoc.def|.

% The definitions cannot be loaded using |\usepackage| or |\RequirePackage|
% which has a mechanism to prevent loading a style file more than once.
% When loading the definitions by means of |\input|
% multiple instances have to be prevented manually:
%\iffalse
%This code needs to be before the `\ProvidesFile' directive
%which is defined at the beginning of this file.
%Therefore it is also placed there and commented out here.
%</package>
%<*discard>
%\fi
%    \begin{macrocode}
\ifdefined\childdocmain\endinput\fi
%    \end{macrocode}
%\iffalse
%</discard>
%<*package>
%\fi
%
% \macro{\ifchilddoc}
% \macro{\ifchilddocmanual}
% The conditional |\ifchilddoc| tells whether a
% child (true) or main (false) document is being compiled.
% The conditional |\ifchilddocmanual| tells whether
% the |\includeonly| mechanism is used (false) or
% the selection of child files must be performed manually (true).
% The definitions initialise to false:
%    \begin{macrocode}
\newif\ifchilddoc
\newif\ifchilddocmanual
%    \end{macrocode}

% \macro{\childdocname}
% \macro{\childdocjob}
% The macro |\childdocname| stores the name of the main document
% to be compiled. The macro |\childdocjob| stores the name of
% the document on which the \LaTeX{} compiler was originally invoked.
% The content of |\jobname| cannot be compared
% to filenames specified in the source due to different catcodes.
% The following code rescans |\jobname|, stores the result
% in |\childdocname| and saves a copy in |\childdocjob|:
%    \begin{macrocode}
\edef\childdocname{\scantokens\expandafter{\jobname\noexpand}}
\let\childdocjob\childdocname
%    \end{macrocode}

% \macro{\childdocdisable}
% The macro |\childdocdisable| prevents the main file
% from being processed more than once.
% At this stage, the main document command |\childdocmain|
% is assumed to be called once again where it should do nothing.
% Any subsequent call to it should prevent
% a secondary processing of the main document
% It overwrites the forwarding commands
% |\childdocof| and |\childdocforward|
% with empty macros to prevent further inclusions of the main document:
%    \begin{macrocode}
\newcommand{\childdocdisable}
{
  \renewcommand{\childdocmain}[1]{\renewcommand{\childdocmain}[1]{\endinput}}
  \renewcommand{\childdocof}[1]{}
  \renewcommand{\childdocby}[2][]{}
  \renewcommand{\childdocforward}[2][]{}
  \renewcommand{\childdocdisable}{}
}
%    \end{macrocode}

% \macro{\childdocmain}
% The macro |\childdocmain| is to be called at the top of the main file
% with nothing or the main filename (without extension) as argument.
% First, it breaks loops.
% If the argument is not empty and does not match |\childdocname|
% (which is set by the first inclusion of |childdoc.def|),
% |\ifchilddoc| is set to true, |\includeonly| is applied to the child file
% and |\jobname| is set to the main file
% (for proper handling of |.aux| files):
%    \begin{macrocode}
\newcommand{\childdocmain}[1]
{
  \childdocdisable\childdocmain{}
  \if?#1?\else
    \begingroup
      \def\childdoctmp{#1}
      \ifx\childdoctmp\childdocname
        \def\childdoctmp{}
      \else
        \def\childdoctmp
        {
          \childdoctrue
          \includeonly{\childdocname}
          \def\childdocjob{#1}
          \def\jobname{#1}
        }
      \fi
      \expandafter
    \endgroup
    \childdoctmp
  \fi
}
%    \end{macrocode}

% \macro{\childdocof}
% The command |\childdocof| redirects
% compilation to the main file |#1|.
%    \begin{macrocode}
\newcommand{\childdocof}[1]
{
  \childdocdisable
  \childdoctrue
  \includeonly{\childdocname}
  \def\jobname{#1}
  \def\childdocjob{#1}
  \input{#1}
}
%    \end{macrocode}

% \macro{\childdocby}
% The command |\childdocby| ....
%    \begin{macrocode}
\newcommand{\childdocby}[2][]
{
  \childdocdisable
  \childdoctrue
  \childdocmanualtrue
  \if?#1?\else
    \def\jobname{#2}
  \fi
  \def\childdocjob{#2}
  \input{#2}
  \endinput
}
%    \end{macrocode}

% \macro{\childdocforward}
% The command |\childdocforward| redirects
% compilation to the main file or
% (if the optional argument is given) a child file.
% Parameters are set as if the main file
% or a child file starting with |\childdocof| was compiled.
% Then compilation is handed over to the main file:
%    \begin{macrocode}
\newcommand{\childdocforward}[2][]
{
  \begingroup
    \if?#1?
      \def\childdoctmp
      {
        \def\childdocname{#2}
        \def\childdocjob{#2}
        \def\jobname{#2}
        \input{#2}
        \endinput
      }
    \else
      \def\childdoctmp
      {
        \childdocdisable
        \def\childdocname{#2}
        \childdoctrue
        \includeonly{#2}
        \def\childdocjob{#1}
        \def\jobname{#1}
        \input{#1}
        \endinput
      }
    \fi
    \expandafter
  \endgroup
  \childdoctmp
}
%    \end{macrocode}

% \macro{\childdocforwardprefix}
% The command |\childdocforwardprefix| redirects
% compilation to the main or a child file by means of a pattern.
% The prefix |#1| in the current filename is replaced by |#2|
% and the suffix of the current filename is kept
% (it is assumed that the filename does not contain the substring `|~~~|'
% which is used as a delimiter).
% Compilation is handed over to the new file by |\childdocforward|:
%    \begin{macrocode}
\newcommand{\childdocforwardprefix}[3][]
{
  \begingroup
    \def\childdocextract #2##1~~~{\def\childdoctmp{\childdocforward[#1]{#3##1}}}
    \expandafter\childdocextract\childdocname~~~
    \expandafter
  \endgroup
  \childdoctmp
}
%    \end{macrocode}

% \macro{\childdoc}
% The deprecated macro |\childdoc| is a legacy version of |\childdocmain|:
%    \begin{macrocode}
\newcommand{\childdoc}{\childdocmain}
%    \end{macrocode}

% \macro{\childdocredirect}
% The deprecated macro |\childdocredirect| is a legacy version
% of |\childdocforward| and |\childdocforwardprefix|:
%    \begin{macrocode}
\newcommand{\childdocredirect}[2][]
{
  \begingroup
    \if?#1?
      \def\childdoctmp{\childdocforward{#2}}
    \else
      \def\childdoctmp{\childdocforwardprefix{#1}{#2}}
    \fi
    \expandafter
  \endgroup
  \childdoctmp
}
%    \end{macrocode}

%\iffalse
%</package>
%\fi
%
\endinput

\childdocmain{}
%    \end{macrocode}

% Optional override for |\version| flag:
%    \begin{macrocode}
%%\ifchilddoc\else\providecommand{\version}{draft}\fi
%    \end{macrocode}

% Define the default values for the |\version| flag
% (|final| for the main file and |draft| for childs):
%    \begin{macrocode}
\ifchilddoc
\providecommand{\version}{draft}
\else
\providecommand{\version}{final}
\fi
%    \end{macrocode}

% Load the standard document class:
%    \begin{macrocode}
\documentclass[12pt]{article}
%    \end{macrocode}

% Start the document body:
%    \begin{macrocode}
\begin{document}
%    \end{macrocode}

% Declare a title page.
% Print title, part of document being processed and version flag:
%    \begin{macrocode}
\addtocounter{page}{-1}
\begin{center}
{\LARGE\bfseries{}childdoc example\par}
\vspace{1cm}
\ifchilddoc
\ifchilddocmanual part\else chapter\fi:
`\childdocname' of `\childdocjob'\par
\else
main document: `\childdocjob'\par
\fi
version: \version\par
\end{center}
\newpage
%    \end{macrocode}

% Manually include selected file,
% otherwise process as usual:
%    \begin{macrocode}
\ifchilddocmanual
\section*{part `\childdocname'}
\input{\childdocname}
\else
%    \end{macrocode}

% Include the two chapters:
%    \begin{macrocode}
\include{cdocsch1}
\include{cdocsch2}
%    \end{macrocode}

% Include the two parts unless only chapters should be displayed:
%    \begin{macrocode}
\ifchilddoc\else
\section{part three}
\input{cdocspt3}
\section{part four}
\input{cdocspt4}
\fi
%    \end{macrocode}

% Process as usual until here:
%    \begin{macrocode}
\fi
%    \end{macrocode}

% End of document body:
%    \begin{macrocode}
\end{document}
%    \end{macrocode}
%\iffalse
%</samplemain>
%\fi
%
% %%%%%%%%%%%%%%%%%%%%%%%%%%%%%%%%%%%%%%
% \paragraph{Chapter Include Files.}
%
% The include files are called |cdocsch1.tex| and |cdocsch2.tex|.
%
%\iffalse
%<*samplechap1|samplechap2>
%\fi

% Optional override for |\version| flag:
%    \begin{macrocode}
%%\providecommand{\version}{final}
%    \end{macrocode}

% Include the main document:
%    \begin{macrocode}
% \iffalse
%
% childdoc.dtx Copyright (C) 2017-2018 Niklas Beisert
%
% This work may be distributed and/or modified under the
% conditions of the LaTeX Project Public License, either version 1.3
% of this license or (at your option) any later version.
% The latest version of this license is in
%   http://www.latex-project.org/lppl.txt
% and version 1.3 or later is part of all distributions of LaTeX
% version 2005/12/01 or later.
%
% This work has the LPPL maintenance status `maintained'.
%
% The Current Maintainer of this work is Niklas Beisert.
%
% This work consists of the files childdoc.dtx and childdoc.ins
% and the derived files childdoc.def and cdocsamp.tex with
% cdocsch1.tex, cdocsch2.tex, cdocsdrf.tex, cdocsfn1.tex, cdocsfn2.tex.
%
%<package>\ifdefined\childdocmain\endinput\fi
%<package>\ProvidesFile{childdoc.def}[2018/12/30 v2.0 child document driver]
%<samplemain>\ProvidesFile{cdocsamp.tex}[2018/12/30 v2.0 sample for childdoc]
%<*driver>
%\ProvidesFile{childdoc.drv}[2018/12/30 v2.0 childdoc reference manual file]
\PassOptionsToClass{10pt,a4paper}{article}
\documentclass{ltxdoc}

\usepackage[margin=35mm]{geometry}
\usepackage{hyperref}
\usepackage{hyperxmp}
\usepackage[usenames]{color}

\hypersetup{colorlinks=true}
\hypersetup{pdfstartview=FitH}
\hypersetup{pdfpagemode=UseNone}
\hypersetup{pdfsource={}}
\hypersetup{pdflang={en-UK}}
\hypersetup{pdfcopyright={Copyright 2017-2018 Niklas Beisert.
  This work may be distributed and/or modified under the
  conditions of the LaTeX Project Public License, either version 1.3
  of this license or (at your option) any later version.}}
\hypersetup{pdflicenseurl={http://www.latex-project.org/lppl.txt}}
\hypersetup{pdfcontactaddress={ETH Zurich, ITP, HIT K,
  Wolfgang-Pauli-Strasse 27}}
\hypersetup{pdfcontactpostcode={8093}}
\hypersetup{pdfcontactcity={Zurich}}
\hypersetup{pdfcontactcountry={Switzerland}}
\hypersetup{pdfcontactemail={nbeisert@itp.phys.ethz.ch}}
\hypersetup{pdfcontacturl={http://people.phys.ethz.ch/\xmptilde nbeisert/}}

\newcommand{\secref}[1]{\hyperref[#1]{section \ref*{#1}}}

\parskip1ex
\parindent0pt
\let\olditemize\itemize
\def\itemize{\olditemize\parskip0pt}

\begin{document}

\title{The \textsf{childdoc} Package}
\hypersetup{pdftitle={The childdoc Package}}
\author{Niklas Beisert\\[2ex]
  Institut f\"ur Theoretische Physik\\
  Eidgen\"ossische Technische Hochschule Z\"urich\\
  Wolfgang-Pauli-Strasse 27, 8093 Z\"urich, Switzerland\\[1ex]
  \href{mailto:nbeisert@itp.phys.ethz.ch}
  {\texttt{nbeisert@itp.phys.ethz.ch}}}
\hypersetup{pdfauthor={Niklas Beisert}}
\hypersetup{pdfsubject={Manual for the LaTeX2e Package childdoc}}
\date{30 December 2018, \textsf{v2.0}}
\maketitle

\begin{abstract}\noindent
\textsf{childdoc} is a \LaTeXe{} package
that enables the direct compilation
of document sections included by |\include|
to individual files.
\end{abstract}

\begingroup
\parskip0ex
\tableofcontents
\endgroup

%%%%%%%%%%%%%%%%%%%%%%%%%%%%%%%%%%%%%%%%%%%%%%%%%%%%%%%%%%%%%%%%%%%%%%%%%%%%%%%%
%%%%%%%%%%%%%%%%%%%%%%%%%%%%%%%%%%%%%%%%%%%%%%%%%%%%%%%%%%%%%%%%%%%%%%%%%%%%%%%%
\section{Introduction}

\LaTeX{} provides a mechanism to structure a large document (such as a book)
into a main file and several child files (containing the chapters)
using the |\include| command.
This mechanism is beneficial for documents
which span hundreds of pages in order to
make the source file(s) more manageable.
Moreover, compilation can be restricted to
selected child files by means of the |\includeonly| command.
The latter feature can be used to reduce the compilation time while editing
(this was significantly more useful in the earlier days of \LaTeX{})
or to generate a smaller document which is easier to navigate.
Another application of |\includeonly| is to generate
documents consisting of selected parts of the complete document.

However, there are a few drawbacks of the plain |\include| mechanism:
\begin{itemize}
\item
The child files cannot be compiled on their own,
they can only be compiled via the main file.
A naive editing environment
(such as a text editor with an option
to have the current file processed by \LaTeX)
may require one to switch to the main file before compiling;
attempting to compile the child file produces errors.
\item
The main file must be modified (each time)
to adjust the |\includeonly| command
to the present needs. This easily leaves the main file in a messy state.
\item
The generated document will always carry the filename
of the main document. This is inconvenient if
several child files are to be compiled and
to be kept for distribution.
\end{itemize}

The present package provides a simple interface
to make child files individually compilable by \LaTeX{}.
Compiling a child file then has the same effect as compiling
the main file with an |\includeonly| command
to select the appropriate child.
Moreover the generated document will carry the name of the child
rather than the main file.
This resolves all three above issues.

This feature is meant to make the editing of books,
thesis documents and lecture notes somewhat more convenient.
However, the package can also be used efficiently for
composing a series of documents (such as exercise sheets)
which are typically distributed individually.
It then assists the author in generating the individual documents
(potentially in different versions)
as well as a document containing the collected series.
Another application is in developing style files
or other kinds of included material
where compilation of the style file could redirect
to a sample or test file.

%%%%%%%%%%%%%%%%%%%%%%%%%%%%%%%%%%%%%%%%%%%%%%%%%%%%%%%%%%%%%%%%%%%%%%%%%%%%%%%%
%%%%%%%%%%%%%%%%%%%%%%%%%%%%%%%%%%%%%%%%%%%%%%%%%%%%%%%%%%%%%%%%%%%%%%%%%%%%%%%%
\section{Usage}

First of all, the package \textsf{childdoc} is \emph{not} a standard
\LaTeXe{} |.sty| style file! Therefore it needs to be invoked in
a non-standard way.

%%%%%%%%%%%%%%%%%%%%%%%%%%%%%%%%%%%%%%%%%%%%%%%%%%%%%%%%%%%%%%%%%%%%%%%%%%%%%%%%
\subsection{Included Files}
\label{sec:include}

%%%%%%%%%%%%%%%%%%%%%%%%%%%%%%%%%%%%%%%%
\DescribeMacro{\childdocmain}
To use the package, add the commands
\begin{center}
\begin{tabular}{l}
|\input{childdoc.def}|\\
|\childdocmain{}|\\
\end{tabular}
\end{center}
at the very top of the main \LaTeX{} file,
in particular \emph{before} the |\documentclass| statement!
The argument of |\childdocmain| should be left empty
(but it must be present).

%%%%%%%%%%%%%%%%%%%%%%%%%%%%%%%%%%%%%%%%
\DescribeMacro{\childdocof}
Furthermore, add the commands
\begin{center}
\begin{tabular}{l}
|\input{childdoc.def}|\\
|\childdocof{|\textit{main}|}|\\
\end{tabular}
\end{center}
at the top of every child file \textit{child}
which is included by |\include{|\textit{child}|}|
from within the main file
(or at least for those files to be compiled individually).
The argument \textit{main} must be the filename of the main file.

There are a couple of
considerations in setting up the main and child documents:

%%%%%%%%%%%%%%%%%%%%%%%%%%%%%%%%%%%%%%%%
\paragraph{Restrictions.}

Please note the following restrictions:
\begin{itemize}
\item
|\childdocmain| must be called with one argument \textit{main}
to ensure compatibility with earlier version of the package.
It must either be empty (|\childdocmain{}|)
or precisely match the filename of the main file in which it is specified.
See \secref{sec:detection} for further information.
\item
The filename \textit{main} must be specified without the |.tex| extension.
\item
The filename \textit{main} is case sensitive
(even in case-insensitive file systems)
due to internal string comparison.
\item
The argument \textit{main} should be fully expanded, it cannot be a macro.
\item
Subdirectories and special characters should be avoided in filenames.
\item
The command |\childdocmain{|\textit{main}|}| must be followed by a whitespace.
It should not be followed immediately by another command
or by a comment mark `|%|'.
This is because the \TeX{} parser reads the token immediately following
the argument of |\childdocmain| and puts it
at the beginning of every child section;
however, a white\-space is ignored.
\end{itemize}

%%%%%%%%%%%%%%%%%%%%%%%%%%%%%%%%%%%%%%%%
\paragraph{Content of Main File.}

It is advisable to place all content in the child files included by |\include|.
Any output contained in the main file will appear in all child documents
unless suppressed manually;
it cannot be suppressed automatically by the |\includeonly| directive
and thus should normally be avoided.
A method to include some content in the main file
by means of conditional processing is described in \secref{sec:conditional}.

%%%%%%%%%%%%%%%%%%%%%%%%%%%%%%%%%%%%%%%%
\paragraph{Page Numbering.}

When only a part of the document is compiled,
the appropriate numbering of pages
(as well as other status parameters)
is determined from the |.aux| files.
The latter contain information from previous passes.
However this information needs to propagate through
all intermediate child documents.
Therefore the page numbering in child documents may well
be inconsistent until the complete document is compiled at least once.

A useful (if unconventional) way to always ensure a consistent
page numbering is to restart the numbering in each child document
and denote the pages by `\textit{child}|.|\textit{page}'
where \textit{child} represents the chapter/section number of the child file.
This can be achieved by the command
|\numberwithin{page}{|\textit{child}|}|
of the \textsf{amsmath} package
where \textit{child} can be |chapter| or |section|
depending on the chosen structuring.
Alternatively, one can modify the macro |\thepage| appropriately
and reset the counter |page| at the start of each child file.

%%%%%%%%%%%%%%%%%%%%%%%%%%%%%%%%%%%%%%%%%%%%%%%%%%%%%%%%%%%%%%%%%%%%%%%%%%%%%%%%
\subsection{Conditional Processing}
\label{sec:conditional}

The package provides a mechanism to compile different versions
of a document. To customise the versions further some conditional processing
can come in handy to distinguish which version is being compiled.
The package provides two macros to describe the compilation context:

%%%%%%%%%%%%%%%%%%%%%%%%%%%%%%%%%%%%%%%%
\DescribeMacro{\ifchilddoc}
The conditional |\ifchilddoc| distinguishes between the compilation of
child documents and the main document:
%
\begin{center}
|\ifchilddoc |\textit{child-code}| |[|\||else |\textit{main-code}]| \||fi|
\end{center}

%%%%%%%%%%%%%%%%%%%%%%%%%%%%%%%%%%%%%%%%
\DescribeMacro{\childdocname}
\DescribeMacro{\childdocjob}
The macro |\childdocname| contains the filename (without extension)
of the main or child file being processed.
Note that |\childdocjob| will always contain the name of the main file.

%%%%%%%%%%%%%%%%%%%%%%%%%%%%%%%%%%%%%%%%
\paragraph{Title Page.}

Conditional processing can be used to include a title or banner page
in the main document when proper precautions are taken.
Importantly, the code in the main file should ensure that the page counter
(as well as other status parameters which are stored in the |.aux| files)
takes the same value after the conditional processing.
Otherwise the page numbers may take divergent values
depending on which part is compiled.

For example, a title page could be declared by:
%
\begin{center}
\begin{tabular}{l}
|\ifchilddoc\||else|\\
|\addtocounter{page}{-1}|\\
\textit{code for title page}\\
|\newpage|\\
|\||fi|
\end{tabular}
\end{center}
%
A banner page for the child documents can be generated by:
%
\begin{center}
\begin{tabular}{l}
|\ifchilddoc|\\
|\addtocounter{page}{-1}|\\
\textit{code for banner page}\\
|\newpage|\\
|\||fi|
\end{tabular}
\end{center}
%
Here one could write a message such as:
\begin{center}
|This is the part \childdocname{} of \childdocjob{}.|
\end{center}

%%%%%%%%%%%%%%%%%%%%%%%%%%%%%%%%%%%%%%%%%%%%%%%%%%%%%%%%%%%%%%%%%%%%%%%%%%%%%%%%
\subsection{Flags}
\label{sec:flags}

The package makes it easy to generate different versions
of the main or child documents.
To this end compilation flags can be defined
and assigned different default values.
They will be particularly useful in conjunction
with the forwarding mechanism described in \secref{sec:forward}.

For example, it may be useful to have a flag |\version|
which can be set to |draft| or |final|.
The document source will contain some conditional code
depending on the value of |\version|.
Suppose further, the flag should default to |final| for the main file
and to |draft| for child files
which is a natural assignment for editing the document.
This is achieved by placing the following code
in the preamble of the main document
(below the |\childdocmain| directive):
%
\begin{center}
\begin{tabular}{l}
|\ifchilddoc|\\
|\providecommand{\version}{draft}|\\
|\||else|\\
|\providecommand{\version}{final}|\\
|\||fi|
\end{tabular}
\end{center}
%
The definition by |\providecommand| makes sure
that previous definitions are not overwritten.
Further statements |\providecommand{\version}{...}|
can thus be added before the above code to override it.

For the main file, one might add a line
(between |\childdocmain| and the above block)
%
\begin{center}
|%\ifchilddoc\||else\providecommand{\version}{draft}\||fi|
\end{center}
%
which can be uncommented to produce a draft version.
Likewise one can add a line to the very top of a child file
(above the |\childdocof{|\textit{main}|}| directive)
%
\begin{center}
|%\providecommand{\version}{final}|
\end{center}
%
which can be uncommented to produce the final version of this child document.

%%%%%%%%%%%%%%%%%%%%%%%%%%%%%%%%%%%%%%%%%%%%%%%%%%%%%%%%%%%%%%%%%%%%%%%%%%%%%%%%
\subsection{Forwarding}
\label{sec:forward}

Different versions of the main or child documents
using compilation flags as described in \secref{sec:flags}
can be (permanently) stored in different files
for convenient compilation, viewing and distribution.
To this end, the package defines a command
to pass on compilation to a different file:

%%%%%%%%%%%%%%%%%%%%%%%%%%%%%%%%%%%%%%%%
\DescribeMacro{\childdocforward}
The command |\childdocforward| redirects processing to
another source file:
%
\begin{center}
\begin{tabular}{l}
|\input{childdoc.def}|\\
|\childdocforward[|\textit{main}|]{|\textit{dest}|}|\\
\end{tabular}
\end{center}
%
The argument \textit{dest} is the destination file
(without extension).
It should be the main file or one of the child files.
Note that further \textsf{childdoc} directives
such as |\childdocof| and |\childdocforward|
in the indicated file will be processed in this form.
The optional argument \textit{main}
passes on directly to the main file \textit{main}
while pretending to compile the child \textit{dest}.
This form behaves as if \textit{dest}
issues |\childdocof{|\textit{main}|}| right away,
and no further \textsf{childdoc} directives will be processed.

%%%%%%%%%%%%%%%%%%%%%%%%%%%%%%%%%%%%%%%%
\DescribeMacro{\...prefix}
In the alternative form |\childdocforwardprefix|,
%
\begin{center}
\begin{tabular}{l}
|\input{childdoc.def}|\\
|\childdocforwardprefix[|\textit{main}|]{|\textit{prefix}|}{|\textit{dest}|}|
\end{tabular}
\end{center}
%
the destination file is determined by a pattern
depending on the current file:
To make this work, the current file must be called
`{\textit{prefix}\hspace{0.2em}\textit{suffix}}'
with \textit{prefix} matching precisely the argument.
Processing is then passed on to the file
`{\textit{dest}\hspace{0.2em}\textit{suffix}}'.
Surely, the same effect is achieved by
directly specifying the
argument `{\textit{dest}\hspace{0.2em}\textit{suffix}}'
in the first form.
However, that requires to set up a different file
for each child. With the alternative form of the command
all these files can have exactly the same content
which simplifies setting them up and maintaining them.

For example, the following file |draft.tex|
with a compilation flag |\version| as described in \secref{sec:flags}
compiles the main document as a draft:
%
\begin{center}
\begin{tabular}{l}
|\def\version{draft}|\\
|\input{childdoc.def}|\\
|\childdocforward{|\textit{main}|}|
\end{tabular}
\end{center}
%
Likewise, the following files |final|\textit{nn}|.tex|
compile the final version of the child document
|child|\textit{nn}|.tex|:
%
\begin{center}
\begin{tabular}{l}
|\def\version{final}|\\
|\input{childdoc.def}|\\
|\childdocforwardprefix{final}{child}|
\end{tabular}
\end{center}
%

Note that when several versions of a main file and/or of each child file
are to be generated, it may be convenient to set up a |Makefile| or
shell script to automatise the process.

%%%%%%%%%%%%%%%%%%%%%%%%%%%%%%%%%%%%%%%%%%%%%%%%%%%%%%%%%%%%%%%%%%%%%%%%%%%%%%%%
\subsection{Command Line Processing}
\label{sec:commandline}

The effect of redirection files can also be achieved by invoking
the \LaTeX{} compiler with a more elaborate command line.
Most conveniently this should be done as part
of a shell script or a |Makefile|.

When using \textsf{childdoc} in the main file, the following
command lines effectively perform a redirection
(note that depending on the shell being used,
backslashes may have to be doubled: `|\|' $\to$ `|\\|'):
%
\begin{center}
|... -jobname "|\textit{target}|" |\\|"|[\textit{flags}]%
|\input{childdoc.def}\childdocforward[|\textit{main}|]{|\textit{dest}|}"|
\end{center}
%
Here \textit{target} is the name of the output file,
\textit{main} is the name of the main file
and \textit{dest} is the name of the main or child file to be processed
(all filenames without extensions).
The optional argument \textit{main} can be omitted
if \textit{main} matches \textit{dest}.
Optionally, compilation \textit{flags} can be defined via |\def| commands.
This command line makes the \TeX{} engine believe
it is compiling the file \textit{target}
whose content is specified as the latter parameter.
The provided code then forwards the processing to
\textit{main} or \textit{dest} as described in \secref{sec:forward}.

%%%%%%%%%%%%%%%%%%%%%%%%%%%%%%%%%%%%%%%%%%%%%%%%%%%%%%%%%%%%%%%%%%%%%%%%%%%%%%%%
\subsection{Include by Input}
\label{sec:input}

Including child documents by |\include| has some restrictions by design.
Most notably, the content of a child document always occupies
its own set of pages; pages cannot be shared between child documents.
Usually, this behaviour makes perfect sense
because each child document contain an essential part of the document.
However, in some situations it may be desirable to compose
a document from a collection of parts
without having mandatory page breaks between then.
For this case, the package
provides a mechanism to include parts
by |\input| which can also be processed individually.
However, by construction this mechanism
requires manual handling of the content to be output.

%%%%%%%%%%%%%%%%%%%%%%%%%%%%%%%%%%%%%%%%
\DescribeMacro{\ifchilddocmanual}
The main file should be prepared as usual, see \secref{sec:include}.
However, the document body must make a distinction
between processing of an individual part and of the main document, e.g.:
%
\begin{center}
\begin{tabular}{l}
|\ifchilddocmanual|\\
|\input{\childdocname}|\\
|\||else|\\
\textit{document body with }|\input{|\textit{part}|}|\\
|\||fi|
\end{tabular}
\end{center}
%
The conditional |\ifchilddocmanual| is true whenever
a part to be included by |\input| is being compiled,
and the name of the part is stored in |\childdocname|.

%%%%%%%%%%%%%%%%%%%%%%%%%%%%%%%%%%%%%%%%
\DescribeMacro{\childdocby}
Each part to be included by |\input| should start with:
%
\begin{center}
\begin{tabular}{l}
|\input{childdoc.def}|\\
|\childdocby{|\textit{main}|}|\\
\end{tabular}
\end{center}
%
The directive |\childdocby| is similar to |\childdocof|
described in \secref{sec:include},
but the subsequent selection of content must be done manually.
To that end, both |\ifchilddoc| and |\ifchilddocmanual|
will be true upon processing of a part,
and the name of the part is stored in |\childdocname|.
Note that |\jobname| will be set to the filename of the current part
so that each part receives an individual |.aux| file
that does not interfere with the |.aux| file(s) of the main document.
This behaviour can be altered by the alternative form
|\childdocby[*]{|\textit{main}|}| (with a non-empty optional argument)
which uses the |.aux| file of the main document
by setting |\jobname| to \textit{main}.

%%%%%%%%%%%%%%%%%%%%%%%%%%%%%%%%%%%%%%%%%%%%%%%%%%%%%%%%%%%%%%%%%%%%%%%%%%%%%%%%
\subsection{Driver Development}
\label{sec:driver}

The \textsf{childdoc} mechanism can also be use for the development
of definition files such as \LaTeX{} styles or classes.
This case differs from the above setup with multiple parts
included by |\include| in that no |\includeonly| should be invoked.
This can be achieved by starting the include file
(before |\ProvidesPackage|) with:
%
\begin{center}
\begin{tabular}{l}
|\input{childdoc.def}|\\
|\childdocforward{|\textit{main}|}|\\
\end{tabular}
\end{center}
%
or alternatively with:
%
\begin{center}
\begin{tabular}{l}
|\input{childdoc.def}|\\
|\childdocby{|\textit{main}|}|\\
\end{tabular}
\end{center}
%
Both forms have slightly different effects as described above.
The main file is prepared as usual, see \secref{sec:include}.

%%%%%%%%%%%%%%%%%%%%%%%%%%%%%%%%%%%%%%%%%%%%%%%%%%%%%%%%%%%%%%%%%%%%%%%%%%%%%%%%
\subsection{Legacy Detection}
\label{sec:detection}

The directive |\childdocmain| in the main file can detect
whether the complete document or merely a child is to be compiled
even without using the directive |\childdocof|.
This method is deprecated because it is less robust
and there is no compelling reason to use it;
it is merely provided for backward compatibility
and it may be removed in future versions.

If the detection mechanism is to be used,
it is mandatory to correctly specify
the filename of the main file as the argument of |\childdocmain|:
%
\begin{center}
\begin{tabular}{l}
|\input{childdoc.def}|\\
|\childdocmain{|\textit{main}|}|\\
\end{tabular}
\end{center}
%
If |\jobname| does not match the argument \textit{main} of |\childdocmain|,
it is assumed that |\jobname| points to the child file to be compiled.
When using |\childdocmain| with the main file specified as argument,
it suffices to start a child file
with just |\input{|\textit{main}|}|
without loading of the package and using |\childdocof|.
If instead all processing is done
with the appropriate \textsf{childdoc} directives,
the argument of \textit{main} of |\childdocmain| can be empty.

An alternative version of the command line processing described
in \secref{sec:commandline} using the detection mechanism reads:
%
\begin{center}
|... -jobname "|\textit{target}|" "|[\textit{flags}]%
[|\def\jobname{|\textit{dest}|}|]|\input{|\textit{main}|}"|
\end{center}

%%%%%%%%%%%%%%%%%%%%%%%%%%%%%%%%%%%%%%%%%%%%%%%%%%%%%%%%%%%%%%%%%%%%%%%%%%%%%%%%
\subsection{Manual Code}
\label{sec:manual}

In case one cannot be certain whether the definitions file |childdoc.def|
is installed on the target \TeX{} distribution
and one prefers not to ship it,
it is conceivable to paste a few relevant commands into the sources.

To that end, drop all statements |\input{childdoc.def}|
and perform the replacements as outlined below.
Instead of |\childdocmain{|\textit{main}|}| add the following code
to the top of the main file:
%
\begin{center}
\begin{tabular}{l}
|\||ifdefined\childdocname\endinput\||fi\newif\ifchilddoc|\\
|\edef\childdocname{\scantokens\expandafter{\jobname\noexpand}}|\\
|\def\childdocmain{|\textit{main}|}\||ifx\childdocmain\childdocname\||else|\\
|\childdoctrue\includeonly{\childdocname}\let\jobname\childdocmain\||fi|\\
\end{tabular}
\end{center}
%
Instead of |\childdocof{|\textit{main}|}| just include the main file
at the top of each child file:
%
\begin{center}
|\input{|\textit{main}|}|
\end{center}
%
A simple redirection |\childdocforward{|\textit{dest}|}| is achieved by:
%
\begin{center}
|\def\jobname{|\textit{dest}|}\input{\jobname}|
\end{center}
%
The redirection with prefix
|\childdocforwardprefix[|\textit{prefix}|]{|\textit{dest}|}|
is accomplished by:
%
\begin{center}
\begin{tabular}{l}
|{\edef\jobname{\scantokens\expandafter{\jobname\noexpand}}|\\
|\def\redirectjob |\textit{prefix}|#1~~~{\gdef\jobname{|\textit{dest}|#1}}|\\
|\expandafter\redirectjob\jobname~~~}\input{\jobname}|
\end{tabular}
\end{center}

In an alternative approach,
child documents can be compiled by a specific command line
without additional code or specific definitions:
%
\begin{center}
|... -jobname "|\textit{target}|" "|[\textit{flags}]%
|\includeonly{|\textit{dest}|}\input{|\textit{main}|}"|
\end{center}
%

%%%%%%%%%%%%%%%%%%%%%%%%%%%%%%%%%%%%%%%%%%%%%%%%%%%%%%%%%%%%%%%%%%%%%%%%%%%%%%%%
%%%%%%%%%%%%%%%%%%%%%%%%%%%%%%%%%%%%%%%%%%%%%%%%%%%%%%%%%%%%%%%%%%%%%%%%%%%%%%%%
\section{Information}

%%%%%%%%%%%%%%%%%%%%%%%%%%%%%%%%%%%%%%%%%%%%%%%%%%%%%%%%%%%%%%%%%%%%%%%%%%%%%%%%
\subsection{Copyright}

Copyright \copyright{} 2017--2018 Niklas Beisert

This work may be distributed and/or modified under the
conditions of the \LaTeX{} Project Public License, either version 1.3
of this license or (at your option) any later version.
The latest version of this license is in
  \url{http://www.latex-project.org/lppl.txt}
and version 1.3 or later is part of all distributions of \LaTeX{}
version 2005/12/01 or later.

This work has the LPPL maintenance status `maintained'.

The Current Maintainer of this work is Niklas Beisert.

This work consists of the files |README.txt|, |childdoc.ins| and |childdoc.dtx|
as well as the derived files |childdoc.def|, |cdocsamp.tex|
with |cdocsch1.tex|, |cdocsch2.tex|, |cdocspt3.tex|, |cdocspt4.tex|,
|cdocsdrf.tex|, |cdocsfn1.tex|, |cdocsfn2.tex|
as well as |childdoc.pdf|.

%%%%%%%%%%%%%%%%%%%%%%%%%%%%%%%%%%%%%%%%%%%%%%%%%%%%%%%%%%%%%%%%%%%%%%%%%%%%%%%%
\subsection{Files and Installation}

The package consists of the files:
%
\begin{center}
\begin{tabular}{ll}
    |README.txt|   & readme file \\
    |childdoc.ins| & installation file \\
    |childdoc.dtx| & source file \\
    |childdoc.def| & definition file \\
    |cdocsamp.tex| & sample main file \\
    |cdocsch1.tex| & sample include file \\
    |cdocsch2.tex| & sample include file \\
    |cdocspt3.tex| & sample part file \\
    |cdocspt4.tex| & sample part file \\
    |cdocsdrf.tex| & sample redirection file \\
    |cdocsfn1.tex| & sample redirection file \\
    |cdocsfn2.tex| & sample redirection file \\
    |childdoc.pdf| & manual
\end{tabular}
\end{center}
%
The distribution consists of the files
|README.txt|, |childdoc.ins| and |childdoc.dtx|.
%
\begin{itemize}
\item
Run (pdf)\LaTeX{} on |childdoc.dtx|
to compile the manual |childdoc.pdf| (this file).
\item
Run \LaTeX{} on |childdoc.ins| to create the definitions file |childdoc.def|
and the sample |cdocsamp.tex| with include files
|cdocsch1.tex|, |cdocsch2.tex|, |cdocspt3.tex|, |cdocspt4.tex|,
|cdocsdrf.tex|, |cdocsfn1.tex|, |cdocsfn2.tex|.
Then copy the file |childdoc.def| to an appropriate directory of your \LaTeX{}
distribution, e.g.\ \textit{texmf-root}|/tex/latex/childdoc|.
\end{itemize}

%%%%%%%%%%%%%%%%%%%%%%%%%%%%%%%%%%%%%%%%%%%%%%%%%%%%%%%%%%%%%%%%%%%%%%%%%%%%%%%%
\subsection{Related CTAN Packages}

There are several other packages which offer a similar functionality:
%
\begin{itemize}
\item
The packages
\href{http://ctan.org/pkg/docmute}{\textsf{docmute}},
\href{http://ctan.org/pkg/includex}{\textsf{includex}} and
\href{http://ctan.org/pkg/standalone}{\textsf{standalone}}
provide commands to include only the document body of
a child file thus allowing both files to be compiled individually.
\item
The packages \href{http://ctan.org/pkg/subdocs}{\textsf{subdocs}}
and \href{http://ctan.org/pkg/subfiles}{\textsf{subfiles}}
provide structures in which the main and child documents can be
encapsulated and allowing them to be compiled individually.
The inclusion mechanism is different from the conventional |\include|.
\item
The package \href{http://ctan.org/pkg/combine}{\textsf{combine}}
is an elaborate solution to combine several documents into one.
\end{itemize}
%
See also the CTAN topic \href{http://ctan.org/topic/subdocs}{\textsf{subdocs}}
for further related packages.
The present package differs from the above solutions in that
a document structure constructed with the conventional |\include| mechanism
just needs two extra commands at the top of every file
such that all constituent files can be compiled individually.

%%%%%%%%%%%%%%%%%%%%%%%%%%%%%%%%%%%%%%%%%%%%%%%%%%%%%%%%%%%%%%%%%%%%%%%%%%%%%%%%
%\subsection{Feature Suggestions}
%
%The following is a list of features which may be useful for future
%versions of this package:
%%
%\begin{itemize}
%\item
%\ldots
%\end{itemize}

%%%%%%%%%%%%%%%%%%%%%%%%%%%%%%%%%%%%%%%%%%%%%%%%%%%%%%%%%%%%%%%%%%%%%%%%%%%%%%%%
\subsection{Revision History}

%%%%%%%%%%%%%%%%%%%%%%%%%%%%%%%%%%%%%%%%
\paragraph{v2.0:} 2018/12/30

\begin{itemize}
\item
immediate forward processing
\item
added |\childdocby| mechanism
\item
manual restructured
\end{itemize}

%%%%%%%%%%%%%%%%%%%%%%%%%%%%%%%%%%%%%%%%
\paragraph{v1.6:} 2018/01/17

\begin{itemize}
\item
application for development of include files
\item
corrections to manual
\end{itemize}

%%%%%%%%%%%%%%%%%%%%%%%%%%%%%%%%%%%%%%%%
\paragraph{v1.5:} 2017/05/21

\begin{itemize}
\item
more complete structuring introduced
\item
|\childdocof| introduced
\item
|\childdoc| renamed to |\childdocmain|
\item
|\childredirect| renamed to |\childdocforward| and |\childdocforwardprefix|
and functionality expanded
\end{itemize}

%%%%%%%%%%%%%%%%%%%%%%%%%%%%%%%%%%%%%%%%
\paragraph{v1.0:} 2017/04/27

\begin{itemize}
\item
manual and install package
\item
first version published on CTAN
\end{itemize}

%%%%%%%%%%%%%%%%%%%%%%%%%%%%%%%%%%%%%%%%
\paragraph{v0.6:} 2017/04/26

\begin{itemize}
\item
redirection mechanism added
\end{itemize}

%%%%%%%%%%%%%%%%%%%%%%%%%%%%%%%%%%%%%%%%
\paragraph{v0.5:} 2017/04/26

\begin{itemize}
\item
functionality in definition file
\end{itemize}


%%%%%%%%%%%%%%%%%%%%%%%%%%%%%%%%%%%%%%%%%%%%%%%%%%%%%%%%%%%%%%%%%%%%%%%%%%%%%%%%
%%%%%%%%%%%%%%%%%%%%%%%%%%%%%%%%%%%%%%%%%%%%%%%%%%%%%%%%%%%%%%%%%%%%%%%%%%%%%%%%
%%%%%%%%%%%%%%%%%%%%%%%%%%%%%%%%%%%%%%%%%%%%%%%%%%%%%%%%%%%%%%%%%%%%%%%%%%%%%%%%
\appendix

\settowidth\MacroIndent{\rmfamily\scriptsize 000\ }

 \DocInput{childdoc.dtx}

\end{document}
%</driver>
% \fi
%
% %%%%%%%%%%%%%%%%%%%%%%%%%%%%%%%%%%%%%%%%%%%%%%%%%%%%%%%%%%%%%%%%%%%%%%%%%%%%%%
% %%%%%%%%%%%%%%%%%%%%%%%%%%%%%%%%%%%%%%%%%%%%%%%%%%%%%%%%%%%%%%%%%%%%%%%%%%%%%%
% \section{Sample}
%\iffalse
%<*samplemain>
%\fi
%
% The following presents a sample document
% with two chapters, two parts, a title page,
% a compile flag as well as three forwarding files to set the flag.
% It consists of eight |.tex| files:
% \begin{center}
% \begin{tabular}{ll}
% |cdocsamp.tex|&main file\\
% |cdocsch1.tex|&include file for chapter 1\\
% |cdocsch2.tex|&include file for chapter 2\\
% |cdocspt3.tex|&include file for part 3\\
% |cdocspt4.tex|&include file for part 4\\
% |cdocsdrf.tex|&forwarding file for main file in draft mode\\
% |cdocsfi1.tex|&forwarding file for final version of chapter 1\\
% |cdocsfi2.tex|&forwarding file for final version of chapter 2\\
% \end{tabular}
% \end{center}
% Each of the eight files can be compiled directly by the \LaTeX{} compiler.
%
% %%%%%%%%%%%%%%%%%%%%%%%%%%%%%%%%%%%%%%
% \paragraph{Main File.}
%
% The main file is called |cdocsamp.tex|.
%
% Load the \textsf{childdoc} definitions and
% declare the filename for the main document:
%    \begin{macrocode}
\input{childdoc.def}
\childdocmain{}
%    \end{macrocode}

% Optional override for |\version| flag:
%    \begin{macrocode}
%%\ifchilddoc\else\providecommand{\version}{draft}\fi
%    \end{macrocode}

% Define the default values for the |\version| flag
% (|final| for the main file and |draft| for childs):
%    \begin{macrocode}
\ifchilddoc
\providecommand{\version}{draft}
\else
\providecommand{\version}{final}
\fi
%    \end{macrocode}

% Load the standard document class:
%    \begin{macrocode}
\documentclass[12pt]{article}
%    \end{macrocode}

% Start the document body:
%    \begin{macrocode}
\begin{document}
%    \end{macrocode}

% Declare a title page.
% Print title, part of document being processed and version flag:
%    \begin{macrocode}
\addtocounter{page}{-1}
\begin{center}
{\LARGE\bfseries{}childdoc example\par}
\vspace{1cm}
\ifchilddoc
\ifchilddocmanual part\else chapter\fi:
`\childdocname' of `\childdocjob'\par
\else
main document: `\childdocjob'\par
\fi
version: \version\par
\end{center}
\newpage
%    \end{macrocode}

% Manually include selected file,
% otherwise process as usual:
%    \begin{macrocode}
\ifchilddocmanual
\section*{part `\childdocname'}
\input{\childdocname}
\else
%    \end{macrocode}

% Include the two chapters:
%    \begin{macrocode}
\include{cdocsch1}
\include{cdocsch2}
%    \end{macrocode}

% Include the two parts unless only chapters should be displayed:
%    \begin{macrocode}
\ifchilddoc\else
\section{part three}
\input{cdocspt3}
\section{part four}
\input{cdocspt4}
\fi
%    \end{macrocode}

% Process as usual until here:
%    \begin{macrocode}
\fi
%    \end{macrocode}

% End of document body:
%    \begin{macrocode}
\end{document}
%    \end{macrocode}
%\iffalse
%</samplemain>
%\fi
%
% %%%%%%%%%%%%%%%%%%%%%%%%%%%%%%%%%%%%%%
% \paragraph{Chapter Include Files.}
%
% The include files are called |cdocsch1.tex| and |cdocsch2.tex|.
%
%\iffalse
%<*samplechap1|samplechap2>
%\fi

% Optional override for |\version| flag:
%    \begin{macrocode}
%%\providecommand{\version}{final}
%    \end{macrocode}

% Include the main document:
%    \begin{macrocode}
\input{childdoc.def}
\childdocof{cdocsamp}
%    \end{macrocode}

%\iffalse
%</samplechap1|samplechap2>
%\fi
%
%\iffalse
%<*samplechap1>
%\fi
% Some text for chapter 1:
%    \begin{macrocode}
\section{one}
some text in chapter one
%    \end{macrocode}

%\iffalse
%</samplechap1>
%\fi
% Some text for chapter 2:
%\iffalse
%<*samplechap2>
%\fi
%    \begin{macrocode}
\section{two}
more text in chapter two
%    \end{macrocode}

%\iffalse
%</samplechap2>
%\fi
%
% %%%%%%%%%%%%%%%%%%%%%%%%%%%%%%%%%%%%%%
% \paragraph{Part Include Files.}
%
% The include files are called |cdocspt3.tex| and |cdocspt4.tex|.
%
%\iffalse
%<*samplepart3|samplepart4>
%\fi

% Optional override for |\version| flag:
%    \begin{macrocode}
%%\providecommand{\version}{final}
%    \end{macrocode}

% Include the main document:
%    \begin{macrocode}
\input{childdoc.def}
\childdocby{cdocsamp}
%    \end{macrocode}

%\iffalse
%</samplepart3|samplepart4>
%\fi
%
%\iffalse
%<*samplepart3>
%\fi
% Some text for part 3:
%    \begin{macrocode}
some text in part three
%    \end{macrocode}

%\iffalse
%</samplepart3>
%\fi
% Some text for part 4:
%\iffalse
%<*samplepart4>
%\fi
%    \begin{macrocode}
more text in part four
%    \end{macrocode}

%\iffalse
%</samplepart4>
%\fi
%
% %%%%%%%%%%%%%%%%%%%%%%%%%%%%%%%%%%%%%%
% \paragraph{Forwarding for a Complete Draft.}
%
% The following forwarding file |cdocsdrf.tex|
% compiles the main document in draft mode:
%\iffalse
%<*sampledraft>
%\fi
%    \begin{macrocode}
\def\version{draft}
\input{childdoc.def}
\childdocforward{cdocsamp}
%    \end{macrocode}

%\iffalse
%</sampledraft>
%\fi
%
% %%%%%%%%%%%%%%%%%%%%%%%%%%%%%%%%%%%%%%
% \paragraph{Forwarding for Final Version of the Chapters.}
%
% The following forwarding files |cdocsfn1.tex| and |cdocsfn2.tex|
% (with identical content)
% compile the final versions of the child documents
% |cdocsch1.tex| and |cdocsch2.tex|, respectively:
%\iffalse
%<*samplefinal>
%\fi
%    \begin{macrocode}
\def\version{final}
\input{childdoc.def}
\childdocforwardprefix[cdocsamp]{cdocsfn}{cdocsch}
%    \end{macrocode}

%\iffalse
%</samplefinal>
%\fi
%
% %%%%%%%%%%%%%%%%%%%%%%%%%%%%%%%%%%%%%%
% \paragraph{Command Line Processing.}
%
% The following three command lines generate the output files
% |cdocscld|, |cdocscl1| and |cdocscl2|
% which should be identical to
% |cdocsdrf|, |cdocsch1| and |cdocsfn2|, respectively:
% \begin{center}
% \begin{tabular}{l}
% |latex -jobname cdocscld \|\\
% |  "\def\version{draft}\input{childdoc.def}\childdocforward{cdocsamp}"|\\
% |latex -jobname cdocscl1 \|\\
% |  "\input{childdoc.def}\childdocforward[cdocsamp]{cdocsch1}"|\\
% |latex -jobname cdocscl2 \|\\
% |  "\def\version{final}\input{childdoc.def}\childdocforward{cdocsch2}"|
% \end{tabular}
% \end{center}
% Note that the trailing backslash on each first line
% merely continues the input to the second line
% (for convenient cut ant paste).
% Furthermore, the command |latex| can be replaced by any
% of its alternative versions such as |pdflatex|.
%
% %%%%%%%%%%%%%%%%%%%%%%%%%%%%%%%%%%%%%%%%%%%%%%%%%%%%%%%%%%%%%%%%%%%%%%%%%%%%%%
% %%%%%%%%%%%%%%%%%%%%%%%%%%%%%%%%%%%%%%%%%%%%%%%%%%%%%%%%%%%%%%%%%%%%%%%%%%%%%%
% \section{Implementation}
%\iffalse
%<*package>
%\fi
%
% This section describes the definitions file |childdoc.def|.

% The definitions cannot be loaded using |\usepackage| or |\RequirePackage|
% which has a mechanism to prevent loading a style file more than once.
% When loading the definitions by means of |\input|
% multiple instances have to be prevented manually:
%\iffalse
%This code needs to be before the `\ProvidesFile' directive
%which is defined at the beginning of this file.
%Therefore it is also placed there and commented out here.
%</package>
%<*discard>
%\fi
%    \begin{macrocode}
\ifdefined\childdocmain\endinput\fi
%    \end{macrocode}
%\iffalse
%</discard>
%<*package>
%\fi
%
% \macro{\ifchilddoc}
% \macro{\ifchilddocmanual}
% The conditional |\ifchilddoc| tells whether a
% child (true) or main (false) document is being compiled.
% The conditional |\ifchilddocmanual| tells whether
% the |\includeonly| mechanism is used (false) or
% the selection of child files must be performed manually (true).
% The definitions initialise to false:
%    \begin{macrocode}
\newif\ifchilddoc
\newif\ifchilddocmanual
%    \end{macrocode}

% \macro{\childdocname}
% \macro{\childdocjob}
% The macro |\childdocname| stores the name of the main document
% to be compiled. The macro |\childdocjob| stores the name of
% the document on which the \LaTeX{} compiler was originally invoked.
% The content of |\jobname| cannot be compared
% to filenames specified in the source due to different catcodes.
% The following code rescans |\jobname|, stores the result
% in |\childdocname| and saves a copy in |\childdocjob|:
%    \begin{macrocode}
\edef\childdocname{\scantokens\expandafter{\jobname\noexpand}}
\let\childdocjob\childdocname
%    \end{macrocode}

% \macro{\childdocdisable}
% The macro |\childdocdisable| prevents the main file
% from being processed more than once.
% At this stage, the main document command |\childdocmain|
% is assumed to be called once again where it should do nothing.
% Any subsequent call to it should prevent
% a secondary processing of the main document
% It overwrites the forwarding commands
% |\childdocof| and |\childdocforward|
% with empty macros to prevent further inclusions of the main document:
%    \begin{macrocode}
\newcommand{\childdocdisable}
{
  \renewcommand{\childdocmain}[1]{\renewcommand{\childdocmain}[1]{\endinput}}
  \renewcommand{\childdocof}[1]{}
  \renewcommand{\childdocby}[2][]{}
  \renewcommand{\childdocforward}[2][]{}
  \renewcommand{\childdocdisable}{}
}
%    \end{macrocode}

% \macro{\childdocmain}
% The macro |\childdocmain| is to be called at the top of the main file
% with nothing or the main filename (without extension) as argument.
% First, it breaks loops.
% If the argument is not empty and does not match |\childdocname|
% (which is set by the first inclusion of |childdoc.def|),
% |\ifchilddoc| is set to true, |\includeonly| is applied to the child file
% and |\jobname| is set to the main file
% (for proper handling of |.aux| files):
%    \begin{macrocode}
\newcommand{\childdocmain}[1]
{
  \childdocdisable\childdocmain{}
  \if?#1?\else
    \begingroup
      \def\childdoctmp{#1}
      \ifx\childdoctmp\childdocname
        \def\childdoctmp{}
      \else
        \def\childdoctmp
        {
          \childdoctrue
          \includeonly{\childdocname}
          \def\childdocjob{#1}
          \def\jobname{#1}
        }
      \fi
      \expandafter
    \endgroup
    \childdoctmp
  \fi
}
%    \end{macrocode}

% \macro{\childdocof}
% The command |\childdocof| redirects
% compilation to the main file |#1|.
%    \begin{macrocode}
\newcommand{\childdocof}[1]
{
  \childdocdisable
  \childdoctrue
  \includeonly{\childdocname}
  \def\jobname{#1}
  \def\childdocjob{#1}
  \input{#1}
}
%    \end{macrocode}

% \macro{\childdocby}
% The command |\childdocby| ....
%    \begin{macrocode}
\newcommand{\childdocby}[2][]
{
  \childdocdisable
  \childdoctrue
  \childdocmanualtrue
  \if?#1?\else
    \def\jobname{#2}
  \fi
  \def\childdocjob{#2}
  \input{#2}
  \endinput
}
%    \end{macrocode}

% \macro{\childdocforward}
% The command |\childdocforward| redirects
% compilation to the main file or
% (if the optional argument is given) a child file.
% Parameters are set as if the main file
% or a child file starting with |\childdocof| was compiled.
% Then compilation is handed over to the main file:
%    \begin{macrocode}
\newcommand{\childdocforward}[2][]
{
  \begingroup
    \if?#1?
      \def\childdoctmp
      {
        \def\childdocname{#2}
        \def\childdocjob{#2}
        \def\jobname{#2}
        \input{#2}
        \endinput
      }
    \else
      \def\childdoctmp
      {
        \childdocdisable
        \def\childdocname{#2}
        \childdoctrue
        \includeonly{#2}
        \def\childdocjob{#1}
        \def\jobname{#1}
        \input{#1}
        \endinput
      }
    \fi
    \expandafter
  \endgroup
  \childdoctmp
}
%    \end{macrocode}

% \macro{\childdocforwardprefix}
% The command |\childdocforwardprefix| redirects
% compilation to the main or a child file by means of a pattern.
% The prefix |#1| in the current filename is replaced by |#2|
% and the suffix of the current filename is kept
% (it is assumed that the filename does not contain the substring `|~~~|'
% which is used as a delimiter).
% Compilation is handed over to the new file by |\childdocforward|:
%    \begin{macrocode}
\newcommand{\childdocforwardprefix}[3][]
{
  \begingroup
    \def\childdocextract #2##1~~~{\def\childdoctmp{\childdocforward[#1]{#3##1}}}
    \expandafter\childdocextract\childdocname~~~
    \expandafter
  \endgroup
  \childdoctmp
}
%    \end{macrocode}

% \macro{\childdoc}
% The deprecated macro |\childdoc| is a legacy version of |\childdocmain|:
%    \begin{macrocode}
\newcommand{\childdoc}{\childdocmain}
%    \end{macrocode}

% \macro{\childdocredirect}
% The deprecated macro |\childdocredirect| is a legacy version
% of |\childdocforward| and |\childdocforwardprefix|:
%    \begin{macrocode}
\newcommand{\childdocredirect}[2][]
{
  \begingroup
    \if?#1?
      \def\childdoctmp{\childdocforward{#2}}
    \else
      \def\childdoctmp{\childdocforwardprefix{#1}{#2}}
    \fi
    \expandafter
  \endgroup
  \childdoctmp
}
%    \end{macrocode}

%\iffalse
%</package>
%\fi
%
\endinput

\childdocof{cdocsamp}
%    \end{macrocode}

%\iffalse
%</samplechap1|samplechap2>
%\fi
%
%\iffalse
%<*samplechap1>
%\fi
% Some text for chapter 1:
%    \begin{macrocode}
\section{one}
some text in chapter one
%    \end{macrocode}

%\iffalse
%</samplechap1>
%\fi
% Some text for chapter 2:
%\iffalse
%<*samplechap2>
%\fi
%    \begin{macrocode}
\section{two}
more text in chapter two
%    \end{macrocode}

%\iffalse
%</samplechap2>
%\fi
%
% %%%%%%%%%%%%%%%%%%%%%%%%%%%%%%%%%%%%%%
% \paragraph{Part Include Files.}
%
% The include files are called |cdocspt3.tex| and |cdocspt4.tex|.
%
%\iffalse
%<*samplepart3|samplepart4>
%\fi

% Optional override for |\version| flag:
%    \begin{macrocode}
%%\providecommand{\version}{final}
%    \end{macrocode}

% Include the main document:
%    \begin{macrocode}
% \iffalse
%
% childdoc.dtx Copyright (C) 2017-2018 Niklas Beisert
%
% This work may be distributed and/or modified under the
% conditions of the LaTeX Project Public License, either version 1.3
% of this license or (at your option) any later version.
% The latest version of this license is in
%   http://www.latex-project.org/lppl.txt
% and version 1.3 or later is part of all distributions of LaTeX
% version 2005/12/01 or later.
%
% This work has the LPPL maintenance status `maintained'.
%
% The Current Maintainer of this work is Niklas Beisert.
%
% This work consists of the files childdoc.dtx and childdoc.ins
% and the derived files childdoc.def and cdocsamp.tex with
% cdocsch1.tex, cdocsch2.tex, cdocsdrf.tex, cdocsfn1.tex, cdocsfn2.tex.
%
%<package>\ifdefined\childdocmain\endinput\fi
%<package>\ProvidesFile{childdoc.def}[2018/12/30 v2.0 child document driver]
%<samplemain>\ProvidesFile{cdocsamp.tex}[2018/12/30 v2.0 sample for childdoc]
%<*driver>
%\ProvidesFile{childdoc.drv}[2018/12/30 v2.0 childdoc reference manual file]
\PassOptionsToClass{10pt,a4paper}{article}
\documentclass{ltxdoc}

\usepackage[margin=35mm]{geometry}
\usepackage{hyperref}
\usepackage{hyperxmp}
\usepackage[usenames]{color}

\hypersetup{colorlinks=true}
\hypersetup{pdfstartview=FitH}
\hypersetup{pdfpagemode=UseNone}
\hypersetup{pdfsource={}}
\hypersetup{pdflang={en-UK}}
\hypersetup{pdfcopyright={Copyright 2017-2018 Niklas Beisert.
  This work may be distributed and/or modified under the
  conditions of the LaTeX Project Public License, either version 1.3
  of this license or (at your option) any later version.}}
\hypersetup{pdflicenseurl={http://www.latex-project.org/lppl.txt}}
\hypersetup{pdfcontactaddress={ETH Zurich, ITP, HIT K,
  Wolfgang-Pauli-Strasse 27}}
\hypersetup{pdfcontactpostcode={8093}}
\hypersetup{pdfcontactcity={Zurich}}
\hypersetup{pdfcontactcountry={Switzerland}}
\hypersetup{pdfcontactemail={nbeisert@itp.phys.ethz.ch}}
\hypersetup{pdfcontacturl={http://people.phys.ethz.ch/\xmptilde nbeisert/}}

\newcommand{\secref}[1]{\hyperref[#1]{section \ref*{#1}}}

\parskip1ex
\parindent0pt
\let\olditemize\itemize
\def\itemize{\olditemize\parskip0pt}

\begin{document}

\title{The \textsf{childdoc} Package}
\hypersetup{pdftitle={The childdoc Package}}
\author{Niklas Beisert\\[2ex]
  Institut f\"ur Theoretische Physik\\
  Eidgen\"ossische Technische Hochschule Z\"urich\\
  Wolfgang-Pauli-Strasse 27, 8093 Z\"urich, Switzerland\\[1ex]
  \href{mailto:nbeisert@itp.phys.ethz.ch}
  {\texttt{nbeisert@itp.phys.ethz.ch}}}
\hypersetup{pdfauthor={Niklas Beisert}}
\hypersetup{pdfsubject={Manual for the LaTeX2e Package childdoc}}
\date{30 December 2018, \textsf{v2.0}}
\maketitle

\begin{abstract}\noindent
\textsf{childdoc} is a \LaTeXe{} package
that enables the direct compilation
of document sections included by |\include|
to individual files.
\end{abstract}

\begingroup
\parskip0ex
\tableofcontents
\endgroup

%%%%%%%%%%%%%%%%%%%%%%%%%%%%%%%%%%%%%%%%%%%%%%%%%%%%%%%%%%%%%%%%%%%%%%%%%%%%%%%%
%%%%%%%%%%%%%%%%%%%%%%%%%%%%%%%%%%%%%%%%%%%%%%%%%%%%%%%%%%%%%%%%%%%%%%%%%%%%%%%%
\section{Introduction}

\LaTeX{} provides a mechanism to structure a large document (such as a book)
into a main file and several child files (containing the chapters)
using the |\include| command.
This mechanism is beneficial for documents
which span hundreds of pages in order to
make the source file(s) more manageable.
Moreover, compilation can be restricted to
selected child files by means of the |\includeonly| command.
The latter feature can be used to reduce the compilation time while editing
(this was significantly more useful in the earlier days of \LaTeX{})
or to generate a smaller document which is easier to navigate.
Another application of |\includeonly| is to generate
documents consisting of selected parts of the complete document.

However, there are a few drawbacks of the plain |\include| mechanism:
\begin{itemize}
\item
The child files cannot be compiled on their own,
they can only be compiled via the main file.
A naive editing environment
(such as a text editor with an option
to have the current file processed by \LaTeX)
may require one to switch to the main file before compiling;
attempting to compile the child file produces errors.
\item
The main file must be modified (each time)
to adjust the |\includeonly| command
to the present needs. This easily leaves the main file in a messy state.
\item
The generated document will always carry the filename
of the main document. This is inconvenient if
several child files are to be compiled and
to be kept for distribution.
\end{itemize}

The present package provides a simple interface
to make child files individually compilable by \LaTeX{}.
Compiling a child file then has the same effect as compiling
the main file with an |\includeonly| command
to select the appropriate child.
Moreover the generated document will carry the name of the child
rather than the main file.
This resolves all three above issues.

This feature is meant to make the editing of books,
thesis documents and lecture notes somewhat more convenient.
However, the package can also be used efficiently for
composing a series of documents (such as exercise sheets)
which are typically distributed individually.
It then assists the author in generating the individual documents
(potentially in different versions)
as well as a document containing the collected series.
Another application is in developing style files
or other kinds of included material
where compilation of the style file could redirect
to a sample or test file.

%%%%%%%%%%%%%%%%%%%%%%%%%%%%%%%%%%%%%%%%%%%%%%%%%%%%%%%%%%%%%%%%%%%%%%%%%%%%%%%%
%%%%%%%%%%%%%%%%%%%%%%%%%%%%%%%%%%%%%%%%%%%%%%%%%%%%%%%%%%%%%%%%%%%%%%%%%%%%%%%%
\section{Usage}

First of all, the package \textsf{childdoc} is \emph{not} a standard
\LaTeXe{} |.sty| style file! Therefore it needs to be invoked in
a non-standard way.

%%%%%%%%%%%%%%%%%%%%%%%%%%%%%%%%%%%%%%%%%%%%%%%%%%%%%%%%%%%%%%%%%%%%%%%%%%%%%%%%
\subsection{Included Files}
\label{sec:include}

%%%%%%%%%%%%%%%%%%%%%%%%%%%%%%%%%%%%%%%%
\DescribeMacro{\childdocmain}
To use the package, add the commands
\begin{center}
\begin{tabular}{l}
|\input{childdoc.def}|\\
|\childdocmain{}|\\
\end{tabular}
\end{center}
at the very top of the main \LaTeX{} file,
in particular \emph{before} the |\documentclass| statement!
The argument of |\childdocmain| should be left empty
(but it must be present).

%%%%%%%%%%%%%%%%%%%%%%%%%%%%%%%%%%%%%%%%
\DescribeMacro{\childdocof}
Furthermore, add the commands
\begin{center}
\begin{tabular}{l}
|\input{childdoc.def}|\\
|\childdocof{|\textit{main}|}|\\
\end{tabular}
\end{center}
at the top of every child file \textit{child}
which is included by |\include{|\textit{child}|}|
from within the main file
(or at least for those files to be compiled individually).
The argument \textit{main} must be the filename of the main file.

There are a couple of
considerations in setting up the main and child documents:

%%%%%%%%%%%%%%%%%%%%%%%%%%%%%%%%%%%%%%%%
\paragraph{Restrictions.}

Please note the following restrictions:
\begin{itemize}
\item
|\childdocmain| must be called with one argument \textit{main}
to ensure compatibility with earlier version of the package.
It must either be empty (|\childdocmain{}|)
or precisely match the filename of the main file in which it is specified.
See \secref{sec:detection} for further information.
\item
The filename \textit{main} must be specified without the |.tex| extension.
\item
The filename \textit{main} is case sensitive
(even in case-insensitive file systems)
due to internal string comparison.
\item
The argument \textit{main} should be fully expanded, it cannot be a macro.
\item
Subdirectories and special characters should be avoided in filenames.
\item
The command |\childdocmain{|\textit{main}|}| must be followed by a whitespace.
It should not be followed immediately by another command
or by a comment mark `|%|'.
This is because the \TeX{} parser reads the token immediately following
the argument of |\childdocmain| and puts it
at the beginning of every child section;
however, a white\-space is ignored.
\end{itemize}

%%%%%%%%%%%%%%%%%%%%%%%%%%%%%%%%%%%%%%%%
\paragraph{Content of Main File.}

It is advisable to place all content in the child files included by |\include|.
Any output contained in the main file will appear in all child documents
unless suppressed manually;
it cannot be suppressed automatically by the |\includeonly| directive
and thus should normally be avoided.
A method to include some content in the main file
by means of conditional processing is described in \secref{sec:conditional}.

%%%%%%%%%%%%%%%%%%%%%%%%%%%%%%%%%%%%%%%%
\paragraph{Page Numbering.}

When only a part of the document is compiled,
the appropriate numbering of pages
(as well as other status parameters)
is determined from the |.aux| files.
The latter contain information from previous passes.
However this information needs to propagate through
all intermediate child documents.
Therefore the page numbering in child documents may well
be inconsistent until the complete document is compiled at least once.

A useful (if unconventional) way to always ensure a consistent
page numbering is to restart the numbering in each child document
and denote the pages by `\textit{child}|.|\textit{page}'
where \textit{child} represents the chapter/section number of the child file.
This can be achieved by the command
|\numberwithin{page}{|\textit{child}|}|
of the \textsf{amsmath} package
where \textit{child} can be |chapter| or |section|
depending on the chosen structuring.
Alternatively, one can modify the macro |\thepage| appropriately
and reset the counter |page| at the start of each child file.

%%%%%%%%%%%%%%%%%%%%%%%%%%%%%%%%%%%%%%%%%%%%%%%%%%%%%%%%%%%%%%%%%%%%%%%%%%%%%%%%
\subsection{Conditional Processing}
\label{sec:conditional}

The package provides a mechanism to compile different versions
of a document. To customise the versions further some conditional processing
can come in handy to distinguish which version is being compiled.
The package provides two macros to describe the compilation context:

%%%%%%%%%%%%%%%%%%%%%%%%%%%%%%%%%%%%%%%%
\DescribeMacro{\ifchilddoc}
The conditional |\ifchilddoc| distinguishes between the compilation of
child documents and the main document:
%
\begin{center}
|\ifchilddoc |\textit{child-code}| |[|\||else |\textit{main-code}]| \||fi|
\end{center}

%%%%%%%%%%%%%%%%%%%%%%%%%%%%%%%%%%%%%%%%
\DescribeMacro{\childdocname}
\DescribeMacro{\childdocjob}
The macro |\childdocname| contains the filename (without extension)
of the main or child file being processed.
Note that |\childdocjob| will always contain the name of the main file.

%%%%%%%%%%%%%%%%%%%%%%%%%%%%%%%%%%%%%%%%
\paragraph{Title Page.}

Conditional processing can be used to include a title or banner page
in the main document when proper precautions are taken.
Importantly, the code in the main file should ensure that the page counter
(as well as other status parameters which are stored in the |.aux| files)
takes the same value after the conditional processing.
Otherwise the page numbers may take divergent values
depending on which part is compiled.

For example, a title page could be declared by:
%
\begin{center}
\begin{tabular}{l}
|\ifchilddoc\||else|\\
|\addtocounter{page}{-1}|\\
\textit{code for title page}\\
|\newpage|\\
|\||fi|
\end{tabular}
\end{center}
%
A banner page for the child documents can be generated by:
%
\begin{center}
\begin{tabular}{l}
|\ifchilddoc|\\
|\addtocounter{page}{-1}|\\
\textit{code for banner page}\\
|\newpage|\\
|\||fi|
\end{tabular}
\end{center}
%
Here one could write a message such as:
\begin{center}
|This is the part \childdocname{} of \childdocjob{}.|
\end{center}

%%%%%%%%%%%%%%%%%%%%%%%%%%%%%%%%%%%%%%%%%%%%%%%%%%%%%%%%%%%%%%%%%%%%%%%%%%%%%%%%
\subsection{Flags}
\label{sec:flags}

The package makes it easy to generate different versions
of the main or child documents.
To this end compilation flags can be defined
and assigned different default values.
They will be particularly useful in conjunction
with the forwarding mechanism described in \secref{sec:forward}.

For example, it may be useful to have a flag |\version|
which can be set to |draft| or |final|.
The document source will contain some conditional code
depending on the value of |\version|.
Suppose further, the flag should default to |final| for the main file
and to |draft| for child files
which is a natural assignment for editing the document.
This is achieved by placing the following code
in the preamble of the main document
(below the |\childdocmain| directive):
%
\begin{center}
\begin{tabular}{l}
|\ifchilddoc|\\
|\providecommand{\version}{draft}|\\
|\||else|\\
|\providecommand{\version}{final}|\\
|\||fi|
\end{tabular}
\end{center}
%
The definition by |\providecommand| makes sure
that previous definitions are not overwritten.
Further statements |\providecommand{\version}{...}|
can thus be added before the above code to override it.

For the main file, one might add a line
(between |\childdocmain| and the above block)
%
\begin{center}
|%\ifchilddoc\||else\providecommand{\version}{draft}\||fi|
\end{center}
%
which can be uncommented to produce a draft version.
Likewise one can add a line to the very top of a child file
(above the |\childdocof{|\textit{main}|}| directive)
%
\begin{center}
|%\providecommand{\version}{final}|
\end{center}
%
which can be uncommented to produce the final version of this child document.

%%%%%%%%%%%%%%%%%%%%%%%%%%%%%%%%%%%%%%%%%%%%%%%%%%%%%%%%%%%%%%%%%%%%%%%%%%%%%%%%
\subsection{Forwarding}
\label{sec:forward}

Different versions of the main or child documents
using compilation flags as described in \secref{sec:flags}
can be (permanently) stored in different files
for convenient compilation, viewing and distribution.
To this end, the package defines a command
to pass on compilation to a different file:

%%%%%%%%%%%%%%%%%%%%%%%%%%%%%%%%%%%%%%%%
\DescribeMacro{\childdocforward}
The command |\childdocforward| redirects processing to
another source file:
%
\begin{center}
\begin{tabular}{l}
|\input{childdoc.def}|\\
|\childdocforward[|\textit{main}|]{|\textit{dest}|}|\\
\end{tabular}
\end{center}
%
The argument \textit{dest} is the destination file
(without extension).
It should be the main file or one of the child files.
Note that further \textsf{childdoc} directives
such as |\childdocof| and |\childdocforward|
in the indicated file will be processed in this form.
The optional argument \textit{main}
passes on directly to the main file \textit{main}
while pretending to compile the child \textit{dest}.
This form behaves as if \textit{dest}
issues |\childdocof{|\textit{main}|}| right away,
and no further \textsf{childdoc} directives will be processed.

%%%%%%%%%%%%%%%%%%%%%%%%%%%%%%%%%%%%%%%%
\DescribeMacro{\...prefix}
In the alternative form |\childdocforwardprefix|,
%
\begin{center}
\begin{tabular}{l}
|\input{childdoc.def}|\\
|\childdocforwardprefix[|\textit{main}|]{|\textit{prefix}|}{|\textit{dest}|}|
\end{tabular}
\end{center}
%
the destination file is determined by a pattern
depending on the current file:
To make this work, the current file must be called
`{\textit{prefix}\hspace{0.2em}\textit{suffix}}'
with \textit{prefix} matching precisely the argument.
Processing is then passed on to the file
`{\textit{dest}\hspace{0.2em}\textit{suffix}}'.
Surely, the same effect is achieved by
directly specifying the
argument `{\textit{dest}\hspace{0.2em}\textit{suffix}}'
in the first form.
However, that requires to set up a different file
for each child. With the alternative form of the command
all these files can have exactly the same content
which simplifies setting them up and maintaining them.

For example, the following file |draft.tex|
with a compilation flag |\version| as described in \secref{sec:flags}
compiles the main document as a draft:
%
\begin{center}
\begin{tabular}{l}
|\def\version{draft}|\\
|\input{childdoc.def}|\\
|\childdocforward{|\textit{main}|}|
\end{tabular}
\end{center}
%
Likewise, the following files |final|\textit{nn}|.tex|
compile the final version of the child document
|child|\textit{nn}|.tex|:
%
\begin{center}
\begin{tabular}{l}
|\def\version{final}|\\
|\input{childdoc.def}|\\
|\childdocforwardprefix{final}{child}|
\end{tabular}
\end{center}
%

Note that when several versions of a main file and/or of each child file
are to be generated, it may be convenient to set up a |Makefile| or
shell script to automatise the process.

%%%%%%%%%%%%%%%%%%%%%%%%%%%%%%%%%%%%%%%%%%%%%%%%%%%%%%%%%%%%%%%%%%%%%%%%%%%%%%%%
\subsection{Command Line Processing}
\label{sec:commandline}

The effect of redirection files can also be achieved by invoking
the \LaTeX{} compiler with a more elaborate command line.
Most conveniently this should be done as part
of a shell script or a |Makefile|.

When using \textsf{childdoc} in the main file, the following
command lines effectively perform a redirection
(note that depending on the shell being used,
backslashes may have to be doubled: `|\|' $\to$ `|\\|'):
%
\begin{center}
|... -jobname "|\textit{target}|" |\\|"|[\textit{flags}]%
|\input{childdoc.def}\childdocforward[|\textit{main}|]{|\textit{dest}|}"|
\end{center}
%
Here \textit{target} is the name of the output file,
\textit{main} is the name of the main file
and \textit{dest} is the name of the main or child file to be processed
(all filenames without extensions).
The optional argument \textit{main} can be omitted
if \textit{main} matches \textit{dest}.
Optionally, compilation \textit{flags} can be defined via |\def| commands.
This command line makes the \TeX{} engine believe
it is compiling the file \textit{target}
whose content is specified as the latter parameter.
The provided code then forwards the processing to
\textit{main} or \textit{dest} as described in \secref{sec:forward}.

%%%%%%%%%%%%%%%%%%%%%%%%%%%%%%%%%%%%%%%%%%%%%%%%%%%%%%%%%%%%%%%%%%%%%%%%%%%%%%%%
\subsection{Include by Input}
\label{sec:input}

Including child documents by |\include| has some restrictions by design.
Most notably, the content of a child document always occupies
its own set of pages; pages cannot be shared between child documents.
Usually, this behaviour makes perfect sense
because each child document contain an essential part of the document.
However, in some situations it may be desirable to compose
a document from a collection of parts
without having mandatory page breaks between then.
For this case, the package
provides a mechanism to include parts
by |\input| which can also be processed individually.
However, by construction this mechanism
requires manual handling of the content to be output.

%%%%%%%%%%%%%%%%%%%%%%%%%%%%%%%%%%%%%%%%
\DescribeMacro{\ifchilddocmanual}
The main file should be prepared as usual, see \secref{sec:include}.
However, the document body must make a distinction
between processing of an individual part and of the main document, e.g.:
%
\begin{center}
\begin{tabular}{l}
|\ifchilddocmanual|\\
|\input{\childdocname}|\\
|\||else|\\
\textit{document body with }|\input{|\textit{part}|}|\\
|\||fi|
\end{tabular}
\end{center}
%
The conditional |\ifchilddocmanual| is true whenever
a part to be included by |\input| is being compiled,
and the name of the part is stored in |\childdocname|.

%%%%%%%%%%%%%%%%%%%%%%%%%%%%%%%%%%%%%%%%
\DescribeMacro{\childdocby}
Each part to be included by |\input| should start with:
%
\begin{center}
\begin{tabular}{l}
|\input{childdoc.def}|\\
|\childdocby{|\textit{main}|}|\\
\end{tabular}
\end{center}
%
The directive |\childdocby| is similar to |\childdocof|
described in \secref{sec:include},
but the subsequent selection of content must be done manually.
To that end, both |\ifchilddoc| and |\ifchilddocmanual|
will be true upon processing of a part,
and the name of the part is stored in |\childdocname|.
Note that |\jobname| will be set to the filename of the current part
so that each part receives an individual |.aux| file
that does not interfere with the |.aux| file(s) of the main document.
This behaviour can be altered by the alternative form
|\childdocby[*]{|\textit{main}|}| (with a non-empty optional argument)
which uses the |.aux| file of the main document
by setting |\jobname| to \textit{main}.

%%%%%%%%%%%%%%%%%%%%%%%%%%%%%%%%%%%%%%%%%%%%%%%%%%%%%%%%%%%%%%%%%%%%%%%%%%%%%%%%
\subsection{Driver Development}
\label{sec:driver}

The \textsf{childdoc} mechanism can also be use for the development
of definition files such as \LaTeX{} styles or classes.
This case differs from the above setup with multiple parts
included by |\include| in that no |\includeonly| should be invoked.
This can be achieved by starting the include file
(before |\ProvidesPackage|) with:
%
\begin{center}
\begin{tabular}{l}
|\input{childdoc.def}|\\
|\childdocforward{|\textit{main}|}|\\
\end{tabular}
\end{center}
%
or alternatively with:
%
\begin{center}
\begin{tabular}{l}
|\input{childdoc.def}|\\
|\childdocby{|\textit{main}|}|\\
\end{tabular}
\end{center}
%
Both forms have slightly different effects as described above.
The main file is prepared as usual, see \secref{sec:include}.

%%%%%%%%%%%%%%%%%%%%%%%%%%%%%%%%%%%%%%%%%%%%%%%%%%%%%%%%%%%%%%%%%%%%%%%%%%%%%%%%
\subsection{Legacy Detection}
\label{sec:detection}

The directive |\childdocmain| in the main file can detect
whether the complete document or merely a child is to be compiled
even without using the directive |\childdocof|.
This method is deprecated because it is less robust
and there is no compelling reason to use it;
it is merely provided for backward compatibility
and it may be removed in future versions.

If the detection mechanism is to be used,
it is mandatory to correctly specify
the filename of the main file as the argument of |\childdocmain|:
%
\begin{center}
\begin{tabular}{l}
|\input{childdoc.def}|\\
|\childdocmain{|\textit{main}|}|\\
\end{tabular}
\end{center}
%
If |\jobname| does not match the argument \textit{main} of |\childdocmain|,
it is assumed that |\jobname| points to the child file to be compiled.
When using |\childdocmain| with the main file specified as argument,
it suffices to start a child file
with just |\input{|\textit{main}|}|
without loading of the package and using |\childdocof|.
If instead all processing is done
with the appropriate \textsf{childdoc} directives,
the argument of \textit{main} of |\childdocmain| can be empty.

An alternative version of the command line processing described
in \secref{sec:commandline} using the detection mechanism reads:
%
\begin{center}
|... -jobname "|\textit{target}|" "|[\textit{flags}]%
[|\def\jobname{|\textit{dest}|}|]|\input{|\textit{main}|}"|
\end{center}

%%%%%%%%%%%%%%%%%%%%%%%%%%%%%%%%%%%%%%%%%%%%%%%%%%%%%%%%%%%%%%%%%%%%%%%%%%%%%%%%
\subsection{Manual Code}
\label{sec:manual}

In case one cannot be certain whether the definitions file |childdoc.def|
is installed on the target \TeX{} distribution
and one prefers not to ship it,
it is conceivable to paste a few relevant commands into the sources.

To that end, drop all statements |\input{childdoc.def}|
and perform the replacements as outlined below.
Instead of |\childdocmain{|\textit{main}|}| add the following code
to the top of the main file:
%
\begin{center}
\begin{tabular}{l}
|\||ifdefined\childdocname\endinput\||fi\newif\ifchilddoc|\\
|\edef\childdocname{\scantokens\expandafter{\jobname\noexpand}}|\\
|\def\childdocmain{|\textit{main}|}\||ifx\childdocmain\childdocname\||else|\\
|\childdoctrue\includeonly{\childdocname}\let\jobname\childdocmain\||fi|\\
\end{tabular}
\end{center}
%
Instead of |\childdocof{|\textit{main}|}| just include the main file
at the top of each child file:
%
\begin{center}
|\input{|\textit{main}|}|
\end{center}
%
A simple redirection |\childdocforward{|\textit{dest}|}| is achieved by:
%
\begin{center}
|\def\jobname{|\textit{dest}|}\input{\jobname}|
\end{center}
%
The redirection with prefix
|\childdocforwardprefix[|\textit{prefix}|]{|\textit{dest}|}|
is accomplished by:
%
\begin{center}
\begin{tabular}{l}
|{\edef\jobname{\scantokens\expandafter{\jobname\noexpand}}|\\
|\def\redirectjob |\textit{prefix}|#1~~~{\gdef\jobname{|\textit{dest}|#1}}|\\
|\expandafter\redirectjob\jobname~~~}\input{\jobname}|
\end{tabular}
\end{center}

In an alternative approach,
child documents can be compiled by a specific command line
without additional code or specific definitions:
%
\begin{center}
|... -jobname "|\textit{target}|" "|[\textit{flags}]%
|\includeonly{|\textit{dest}|}\input{|\textit{main}|}"|
\end{center}
%

%%%%%%%%%%%%%%%%%%%%%%%%%%%%%%%%%%%%%%%%%%%%%%%%%%%%%%%%%%%%%%%%%%%%%%%%%%%%%%%%
%%%%%%%%%%%%%%%%%%%%%%%%%%%%%%%%%%%%%%%%%%%%%%%%%%%%%%%%%%%%%%%%%%%%%%%%%%%%%%%%
\section{Information}

%%%%%%%%%%%%%%%%%%%%%%%%%%%%%%%%%%%%%%%%%%%%%%%%%%%%%%%%%%%%%%%%%%%%%%%%%%%%%%%%
\subsection{Copyright}

Copyright \copyright{} 2017--2018 Niklas Beisert

This work may be distributed and/or modified under the
conditions of the \LaTeX{} Project Public License, either version 1.3
of this license or (at your option) any later version.
The latest version of this license is in
  \url{http://www.latex-project.org/lppl.txt}
and version 1.3 or later is part of all distributions of \LaTeX{}
version 2005/12/01 or later.

This work has the LPPL maintenance status `maintained'.

The Current Maintainer of this work is Niklas Beisert.

This work consists of the files |README.txt|, |childdoc.ins| and |childdoc.dtx|
as well as the derived files |childdoc.def|, |cdocsamp.tex|
with |cdocsch1.tex|, |cdocsch2.tex|, |cdocspt3.tex|, |cdocspt4.tex|,
|cdocsdrf.tex|, |cdocsfn1.tex|, |cdocsfn2.tex|
as well as |childdoc.pdf|.

%%%%%%%%%%%%%%%%%%%%%%%%%%%%%%%%%%%%%%%%%%%%%%%%%%%%%%%%%%%%%%%%%%%%%%%%%%%%%%%%
\subsection{Files and Installation}

The package consists of the files:
%
\begin{center}
\begin{tabular}{ll}
    |README.txt|   & readme file \\
    |childdoc.ins| & installation file \\
    |childdoc.dtx| & source file \\
    |childdoc.def| & definition file \\
    |cdocsamp.tex| & sample main file \\
    |cdocsch1.tex| & sample include file \\
    |cdocsch2.tex| & sample include file \\
    |cdocspt3.tex| & sample part file \\
    |cdocspt4.tex| & sample part file \\
    |cdocsdrf.tex| & sample redirection file \\
    |cdocsfn1.tex| & sample redirection file \\
    |cdocsfn2.tex| & sample redirection file \\
    |childdoc.pdf| & manual
\end{tabular}
\end{center}
%
The distribution consists of the files
|README.txt|, |childdoc.ins| and |childdoc.dtx|.
%
\begin{itemize}
\item
Run (pdf)\LaTeX{} on |childdoc.dtx|
to compile the manual |childdoc.pdf| (this file).
\item
Run \LaTeX{} on |childdoc.ins| to create the definitions file |childdoc.def|
and the sample |cdocsamp.tex| with include files
|cdocsch1.tex|, |cdocsch2.tex|, |cdocspt3.tex|, |cdocspt4.tex|,
|cdocsdrf.tex|, |cdocsfn1.tex|, |cdocsfn2.tex|.
Then copy the file |childdoc.def| to an appropriate directory of your \LaTeX{}
distribution, e.g.\ \textit{texmf-root}|/tex/latex/childdoc|.
\end{itemize}

%%%%%%%%%%%%%%%%%%%%%%%%%%%%%%%%%%%%%%%%%%%%%%%%%%%%%%%%%%%%%%%%%%%%%%%%%%%%%%%%
\subsection{Related CTAN Packages}

There are several other packages which offer a similar functionality:
%
\begin{itemize}
\item
The packages
\href{http://ctan.org/pkg/docmute}{\textsf{docmute}},
\href{http://ctan.org/pkg/includex}{\textsf{includex}} and
\href{http://ctan.org/pkg/standalone}{\textsf{standalone}}
provide commands to include only the document body of
a child file thus allowing both files to be compiled individually.
\item
The packages \href{http://ctan.org/pkg/subdocs}{\textsf{subdocs}}
and \href{http://ctan.org/pkg/subfiles}{\textsf{subfiles}}
provide structures in which the main and child documents can be
encapsulated and allowing them to be compiled individually.
The inclusion mechanism is different from the conventional |\include|.
\item
The package \href{http://ctan.org/pkg/combine}{\textsf{combine}}
is an elaborate solution to combine several documents into one.
\end{itemize}
%
See also the CTAN topic \href{http://ctan.org/topic/subdocs}{\textsf{subdocs}}
for further related packages.
The present package differs from the above solutions in that
a document structure constructed with the conventional |\include| mechanism
just needs two extra commands at the top of every file
such that all constituent files can be compiled individually.

%%%%%%%%%%%%%%%%%%%%%%%%%%%%%%%%%%%%%%%%%%%%%%%%%%%%%%%%%%%%%%%%%%%%%%%%%%%%%%%%
%\subsection{Feature Suggestions}
%
%The following is a list of features which may be useful for future
%versions of this package:
%%
%\begin{itemize}
%\item
%\ldots
%\end{itemize}

%%%%%%%%%%%%%%%%%%%%%%%%%%%%%%%%%%%%%%%%%%%%%%%%%%%%%%%%%%%%%%%%%%%%%%%%%%%%%%%%
\subsection{Revision History}

%%%%%%%%%%%%%%%%%%%%%%%%%%%%%%%%%%%%%%%%
\paragraph{v2.0:} 2018/12/30

\begin{itemize}
\item
immediate forward processing
\item
added |\childdocby| mechanism
\item
manual restructured
\end{itemize}

%%%%%%%%%%%%%%%%%%%%%%%%%%%%%%%%%%%%%%%%
\paragraph{v1.6:} 2018/01/17

\begin{itemize}
\item
application for development of include files
\item
corrections to manual
\end{itemize}

%%%%%%%%%%%%%%%%%%%%%%%%%%%%%%%%%%%%%%%%
\paragraph{v1.5:} 2017/05/21

\begin{itemize}
\item
more complete structuring introduced
\item
|\childdocof| introduced
\item
|\childdoc| renamed to |\childdocmain|
\item
|\childredirect| renamed to |\childdocforward| and |\childdocforwardprefix|
and functionality expanded
\end{itemize}

%%%%%%%%%%%%%%%%%%%%%%%%%%%%%%%%%%%%%%%%
\paragraph{v1.0:} 2017/04/27

\begin{itemize}
\item
manual and install package
\item
first version published on CTAN
\end{itemize}

%%%%%%%%%%%%%%%%%%%%%%%%%%%%%%%%%%%%%%%%
\paragraph{v0.6:} 2017/04/26

\begin{itemize}
\item
redirection mechanism added
\end{itemize}

%%%%%%%%%%%%%%%%%%%%%%%%%%%%%%%%%%%%%%%%
\paragraph{v0.5:} 2017/04/26

\begin{itemize}
\item
functionality in definition file
\end{itemize}


%%%%%%%%%%%%%%%%%%%%%%%%%%%%%%%%%%%%%%%%%%%%%%%%%%%%%%%%%%%%%%%%%%%%%%%%%%%%%%%%
%%%%%%%%%%%%%%%%%%%%%%%%%%%%%%%%%%%%%%%%%%%%%%%%%%%%%%%%%%%%%%%%%%%%%%%%%%%%%%%%
%%%%%%%%%%%%%%%%%%%%%%%%%%%%%%%%%%%%%%%%%%%%%%%%%%%%%%%%%%%%%%%%%%%%%%%%%%%%%%%%
\appendix

\settowidth\MacroIndent{\rmfamily\scriptsize 000\ }

 \DocInput{childdoc.dtx}

\end{document}
%</driver>
% \fi
%
% %%%%%%%%%%%%%%%%%%%%%%%%%%%%%%%%%%%%%%%%%%%%%%%%%%%%%%%%%%%%%%%%%%%%%%%%%%%%%%
% %%%%%%%%%%%%%%%%%%%%%%%%%%%%%%%%%%%%%%%%%%%%%%%%%%%%%%%%%%%%%%%%%%%%%%%%%%%%%%
% \section{Sample}
%\iffalse
%<*samplemain>
%\fi
%
% The following presents a sample document
% with two chapters, two parts, a title page,
% a compile flag as well as three forwarding files to set the flag.
% It consists of eight |.tex| files:
% \begin{center}
% \begin{tabular}{ll}
% |cdocsamp.tex|&main file\\
% |cdocsch1.tex|&include file for chapter 1\\
% |cdocsch2.tex|&include file for chapter 2\\
% |cdocspt3.tex|&include file for part 3\\
% |cdocspt4.tex|&include file for part 4\\
% |cdocsdrf.tex|&forwarding file for main file in draft mode\\
% |cdocsfi1.tex|&forwarding file for final version of chapter 1\\
% |cdocsfi2.tex|&forwarding file for final version of chapter 2\\
% \end{tabular}
% \end{center}
% Each of the eight files can be compiled directly by the \LaTeX{} compiler.
%
% %%%%%%%%%%%%%%%%%%%%%%%%%%%%%%%%%%%%%%
% \paragraph{Main File.}
%
% The main file is called |cdocsamp.tex|.
%
% Load the \textsf{childdoc} definitions and
% declare the filename for the main document:
%    \begin{macrocode}
\input{childdoc.def}
\childdocmain{}
%    \end{macrocode}

% Optional override for |\version| flag:
%    \begin{macrocode}
%%\ifchilddoc\else\providecommand{\version}{draft}\fi
%    \end{macrocode}

% Define the default values for the |\version| flag
% (|final| for the main file and |draft| for childs):
%    \begin{macrocode}
\ifchilddoc
\providecommand{\version}{draft}
\else
\providecommand{\version}{final}
\fi
%    \end{macrocode}

% Load the standard document class:
%    \begin{macrocode}
\documentclass[12pt]{article}
%    \end{macrocode}

% Start the document body:
%    \begin{macrocode}
\begin{document}
%    \end{macrocode}

% Declare a title page.
% Print title, part of document being processed and version flag:
%    \begin{macrocode}
\addtocounter{page}{-1}
\begin{center}
{\LARGE\bfseries{}childdoc example\par}
\vspace{1cm}
\ifchilddoc
\ifchilddocmanual part\else chapter\fi:
`\childdocname' of `\childdocjob'\par
\else
main document: `\childdocjob'\par
\fi
version: \version\par
\end{center}
\newpage
%    \end{macrocode}

% Manually include selected file,
% otherwise process as usual:
%    \begin{macrocode}
\ifchilddocmanual
\section*{part `\childdocname'}
\input{\childdocname}
\else
%    \end{macrocode}

% Include the two chapters:
%    \begin{macrocode}
\include{cdocsch1}
\include{cdocsch2}
%    \end{macrocode}

% Include the two parts unless only chapters should be displayed:
%    \begin{macrocode}
\ifchilddoc\else
\section{part three}
\input{cdocspt3}
\section{part four}
\input{cdocspt4}
\fi
%    \end{macrocode}

% Process as usual until here:
%    \begin{macrocode}
\fi
%    \end{macrocode}

% End of document body:
%    \begin{macrocode}
\end{document}
%    \end{macrocode}
%\iffalse
%</samplemain>
%\fi
%
% %%%%%%%%%%%%%%%%%%%%%%%%%%%%%%%%%%%%%%
% \paragraph{Chapter Include Files.}
%
% The include files are called |cdocsch1.tex| and |cdocsch2.tex|.
%
%\iffalse
%<*samplechap1|samplechap2>
%\fi

% Optional override for |\version| flag:
%    \begin{macrocode}
%%\providecommand{\version}{final}
%    \end{macrocode}

% Include the main document:
%    \begin{macrocode}
\input{childdoc.def}
\childdocof{cdocsamp}
%    \end{macrocode}

%\iffalse
%</samplechap1|samplechap2>
%\fi
%
%\iffalse
%<*samplechap1>
%\fi
% Some text for chapter 1:
%    \begin{macrocode}
\section{one}
some text in chapter one
%    \end{macrocode}

%\iffalse
%</samplechap1>
%\fi
% Some text for chapter 2:
%\iffalse
%<*samplechap2>
%\fi
%    \begin{macrocode}
\section{two}
more text in chapter two
%    \end{macrocode}

%\iffalse
%</samplechap2>
%\fi
%
% %%%%%%%%%%%%%%%%%%%%%%%%%%%%%%%%%%%%%%
% \paragraph{Part Include Files.}
%
% The include files are called |cdocspt3.tex| and |cdocspt4.tex|.
%
%\iffalse
%<*samplepart3|samplepart4>
%\fi

% Optional override for |\version| flag:
%    \begin{macrocode}
%%\providecommand{\version}{final}
%    \end{macrocode}

% Include the main document:
%    \begin{macrocode}
\input{childdoc.def}
\childdocby{cdocsamp}
%    \end{macrocode}

%\iffalse
%</samplepart3|samplepart4>
%\fi
%
%\iffalse
%<*samplepart3>
%\fi
% Some text for part 3:
%    \begin{macrocode}
some text in part three
%    \end{macrocode}

%\iffalse
%</samplepart3>
%\fi
% Some text for part 4:
%\iffalse
%<*samplepart4>
%\fi
%    \begin{macrocode}
more text in part four
%    \end{macrocode}

%\iffalse
%</samplepart4>
%\fi
%
% %%%%%%%%%%%%%%%%%%%%%%%%%%%%%%%%%%%%%%
% \paragraph{Forwarding for a Complete Draft.}
%
% The following forwarding file |cdocsdrf.tex|
% compiles the main document in draft mode:
%\iffalse
%<*sampledraft>
%\fi
%    \begin{macrocode}
\def\version{draft}
\input{childdoc.def}
\childdocforward{cdocsamp}
%    \end{macrocode}

%\iffalse
%</sampledraft>
%\fi
%
% %%%%%%%%%%%%%%%%%%%%%%%%%%%%%%%%%%%%%%
% \paragraph{Forwarding for Final Version of the Chapters.}
%
% The following forwarding files |cdocsfn1.tex| and |cdocsfn2.tex|
% (with identical content)
% compile the final versions of the child documents
% |cdocsch1.tex| and |cdocsch2.tex|, respectively:
%\iffalse
%<*samplefinal>
%\fi
%    \begin{macrocode}
\def\version{final}
\input{childdoc.def}
\childdocforwardprefix[cdocsamp]{cdocsfn}{cdocsch}
%    \end{macrocode}

%\iffalse
%</samplefinal>
%\fi
%
% %%%%%%%%%%%%%%%%%%%%%%%%%%%%%%%%%%%%%%
% \paragraph{Command Line Processing.}
%
% The following three command lines generate the output files
% |cdocscld|, |cdocscl1| and |cdocscl2|
% which should be identical to
% |cdocsdrf|, |cdocsch1| and |cdocsfn2|, respectively:
% \begin{center}
% \begin{tabular}{l}
% |latex -jobname cdocscld \|\\
% |  "\def\version{draft}\input{childdoc.def}\childdocforward{cdocsamp}"|\\
% |latex -jobname cdocscl1 \|\\
% |  "\input{childdoc.def}\childdocforward[cdocsamp]{cdocsch1}"|\\
% |latex -jobname cdocscl2 \|\\
% |  "\def\version{final}\input{childdoc.def}\childdocforward{cdocsch2}"|
% \end{tabular}
% \end{center}
% Note that the trailing backslash on each first line
% merely continues the input to the second line
% (for convenient cut ant paste).
% Furthermore, the command |latex| can be replaced by any
% of its alternative versions such as |pdflatex|.
%
% %%%%%%%%%%%%%%%%%%%%%%%%%%%%%%%%%%%%%%%%%%%%%%%%%%%%%%%%%%%%%%%%%%%%%%%%%%%%%%
% %%%%%%%%%%%%%%%%%%%%%%%%%%%%%%%%%%%%%%%%%%%%%%%%%%%%%%%%%%%%%%%%%%%%%%%%%%%%%%
% \section{Implementation}
%\iffalse
%<*package>
%\fi
%
% This section describes the definitions file |childdoc.def|.

% The definitions cannot be loaded using |\usepackage| or |\RequirePackage|
% which has a mechanism to prevent loading a style file more than once.
% When loading the definitions by means of |\input|
% multiple instances have to be prevented manually:
%\iffalse
%This code needs to be before the `\ProvidesFile' directive
%which is defined at the beginning of this file.
%Therefore it is also placed there and commented out here.
%</package>
%<*discard>
%\fi
%    \begin{macrocode}
\ifdefined\childdocmain\endinput\fi
%    \end{macrocode}
%\iffalse
%</discard>
%<*package>
%\fi
%
% \macro{\ifchilddoc}
% \macro{\ifchilddocmanual}
% The conditional |\ifchilddoc| tells whether a
% child (true) or main (false) document is being compiled.
% The conditional |\ifchilddocmanual| tells whether
% the |\includeonly| mechanism is used (false) or
% the selection of child files must be performed manually (true).
% The definitions initialise to false:
%    \begin{macrocode}
\newif\ifchilddoc
\newif\ifchilddocmanual
%    \end{macrocode}

% \macro{\childdocname}
% \macro{\childdocjob}
% The macro |\childdocname| stores the name of the main document
% to be compiled. The macro |\childdocjob| stores the name of
% the document on which the \LaTeX{} compiler was originally invoked.
% The content of |\jobname| cannot be compared
% to filenames specified in the source due to different catcodes.
% The following code rescans |\jobname|, stores the result
% in |\childdocname| and saves a copy in |\childdocjob|:
%    \begin{macrocode}
\edef\childdocname{\scantokens\expandafter{\jobname\noexpand}}
\let\childdocjob\childdocname
%    \end{macrocode}

% \macro{\childdocdisable}
% The macro |\childdocdisable| prevents the main file
% from being processed more than once.
% At this stage, the main document command |\childdocmain|
% is assumed to be called once again where it should do nothing.
% Any subsequent call to it should prevent
% a secondary processing of the main document
% It overwrites the forwarding commands
% |\childdocof| and |\childdocforward|
% with empty macros to prevent further inclusions of the main document:
%    \begin{macrocode}
\newcommand{\childdocdisable}
{
  \renewcommand{\childdocmain}[1]{\renewcommand{\childdocmain}[1]{\endinput}}
  \renewcommand{\childdocof}[1]{}
  \renewcommand{\childdocby}[2][]{}
  \renewcommand{\childdocforward}[2][]{}
  \renewcommand{\childdocdisable}{}
}
%    \end{macrocode}

% \macro{\childdocmain}
% The macro |\childdocmain| is to be called at the top of the main file
% with nothing or the main filename (without extension) as argument.
% First, it breaks loops.
% If the argument is not empty and does not match |\childdocname|
% (which is set by the first inclusion of |childdoc.def|),
% |\ifchilddoc| is set to true, |\includeonly| is applied to the child file
% and |\jobname| is set to the main file
% (for proper handling of |.aux| files):
%    \begin{macrocode}
\newcommand{\childdocmain}[1]
{
  \childdocdisable\childdocmain{}
  \if?#1?\else
    \begingroup
      \def\childdoctmp{#1}
      \ifx\childdoctmp\childdocname
        \def\childdoctmp{}
      \else
        \def\childdoctmp
        {
          \childdoctrue
          \includeonly{\childdocname}
          \def\childdocjob{#1}
          \def\jobname{#1}
        }
      \fi
      \expandafter
    \endgroup
    \childdoctmp
  \fi
}
%    \end{macrocode}

% \macro{\childdocof}
% The command |\childdocof| redirects
% compilation to the main file |#1|.
%    \begin{macrocode}
\newcommand{\childdocof}[1]
{
  \childdocdisable
  \childdoctrue
  \includeonly{\childdocname}
  \def\jobname{#1}
  \def\childdocjob{#1}
  \input{#1}
}
%    \end{macrocode}

% \macro{\childdocby}
% The command |\childdocby| ....
%    \begin{macrocode}
\newcommand{\childdocby}[2][]
{
  \childdocdisable
  \childdoctrue
  \childdocmanualtrue
  \if?#1?\else
    \def\jobname{#2}
  \fi
  \def\childdocjob{#2}
  \input{#2}
  \endinput
}
%    \end{macrocode}

% \macro{\childdocforward}
% The command |\childdocforward| redirects
% compilation to the main file or
% (if the optional argument is given) a child file.
% Parameters are set as if the main file
% or a child file starting with |\childdocof| was compiled.
% Then compilation is handed over to the main file:
%    \begin{macrocode}
\newcommand{\childdocforward}[2][]
{
  \begingroup
    \if?#1?
      \def\childdoctmp
      {
        \def\childdocname{#2}
        \def\childdocjob{#2}
        \def\jobname{#2}
        \input{#2}
        \endinput
      }
    \else
      \def\childdoctmp
      {
        \childdocdisable
        \def\childdocname{#2}
        \childdoctrue
        \includeonly{#2}
        \def\childdocjob{#1}
        \def\jobname{#1}
        \input{#1}
        \endinput
      }
    \fi
    \expandafter
  \endgroup
  \childdoctmp
}
%    \end{macrocode}

% \macro{\childdocforwardprefix}
% The command |\childdocforwardprefix| redirects
% compilation to the main or a child file by means of a pattern.
% The prefix |#1| in the current filename is replaced by |#2|
% and the suffix of the current filename is kept
% (it is assumed that the filename does not contain the substring `|~~~|'
% which is used as a delimiter).
% Compilation is handed over to the new file by |\childdocforward|:
%    \begin{macrocode}
\newcommand{\childdocforwardprefix}[3][]
{
  \begingroup
    \def\childdocextract #2##1~~~{\def\childdoctmp{\childdocforward[#1]{#3##1}}}
    \expandafter\childdocextract\childdocname~~~
    \expandafter
  \endgroup
  \childdoctmp
}
%    \end{macrocode}

% \macro{\childdoc}
% The deprecated macro |\childdoc| is a legacy version of |\childdocmain|:
%    \begin{macrocode}
\newcommand{\childdoc}{\childdocmain}
%    \end{macrocode}

% \macro{\childdocredirect}
% The deprecated macro |\childdocredirect| is a legacy version
% of |\childdocforward| and |\childdocforwardprefix|:
%    \begin{macrocode}
\newcommand{\childdocredirect}[2][]
{
  \begingroup
    \if?#1?
      \def\childdoctmp{\childdocforward{#2}}
    \else
      \def\childdoctmp{\childdocforwardprefix{#1}{#2}}
    \fi
    \expandafter
  \endgroup
  \childdoctmp
}
%    \end{macrocode}

%\iffalse
%</package>
%\fi
%
\endinput

\childdocby{cdocsamp}
%    \end{macrocode}

%\iffalse
%</samplepart3|samplepart4>
%\fi
%
%\iffalse
%<*samplepart3>
%\fi
% Some text for part 3:
%    \begin{macrocode}
some text in part three
%    \end{macrocode}

%\iffalse
%</samplepart3>
%\fi
% Some text for part 4:
%\iffalse
%<*samplepart4>
%\fi
%    \begin{macrocode}
more text in part four
%    \end{macrocode}

%\iffalse
%</samplepart4>
%\fi
%
% %%%%%%%%%%%%%%%%%%%%%%%%%%%%%%%%%%%%%%
% \paragraph{Forwarding for a Complete Draft.}
%
% The following forwarding file |cdocsdrf.tex|
% compiles the main document in draft mode:
%\iffalse
%<*sampledraft>
%\fi
%    \begin{macrocode}
\def\version{draft}
% \iffalse
%
% childdoc.dtx Copyright (C) 2017-2018 Niklas Beisert
%
% This work may be distributed and/or modified under the
% conditions of the LaTeX Project Public License, either version 1.3
% of this license or (at your option) any later version.
% The latest version of this license is in
%   http://www.latex-project.org/lppl.txt
% and version 1.3 or later is part of all distributions of LaTeX
% version 2005/12/01 or later.
%
% This work has the LPPL maintenance status `maintained'.
%
% The Current Maintainer of this work is Niklas Beisert.
%
% This work consists of the files childdoc.dtx and childdoc.ins
% and the derived files childdoc.def and cdocsamp.tex with
% cdocsch1.tex, cdocsch2.tex, cdocsdrf.tex, cdocsfn1.tex, cdocsfn2.tex.
%
%<package>\ifdefined\childdocmain\endinput\fi
%<package>\ProvidesFile{childdoc.def}[2018/12/30 v2.0 child document driver]
%<samplemain>\ProvidesFile{cdocsamp.tex}[2018/12/30 v2.0 sample for childdoc]
%<*driver>
%\ProvidesFile{childdoc.drv}[2018/12/30 v2.0 childdoc reference manual file]
\PassOptionsToClass{10pt,a4paper}{article}
\documentclass{ltxdoc}

\usepackage[margin=35mm]{geometry}
\usepackage{hyperref}
\usepackage{hyperxmp}
\usepackage[usenames]{color}

\hypersetup{colorlinks=true}
\hypersetup{pdfstartview=FitH}
\hypersetup{pdfpagemode=UseNone}
\hypersetup{pdfsource={}}
\hypersetup{pdflang={en-UK}}
\hypersetup{pdfcopyright={Copyright 2017-2018 Niklas Beisert.
  This work may be distributed and/or modified under the
  conditions of the LaTeX Project Public License, either version 1.3
  of this license or (at your option) any later version.}}
\hypersetup{pdflicenseurl={http://www.latex-project.org/lppl.txt}}
\hypersetup{pdfcontactaddress={ETH Zurich, ITP, HIT K,
  Wolfgang-Pauli-Strasse 27}}
\hypersetup{pdfcontactpostcode={8093}}
\hypersetup{pdfcontactcity={Zurich}}
\hypersetup{pdfcontactcountry={Switzerland}}
\hypersetup{pdfcontactemail={nbeisert@itp.phys.ethz.ch}}
\hypersetup{pdfcontacturl={http://people.phys.ethz.ch/\xmptilde nbeisert/}}

\newcommand{\secref}[1]{\hyperref[#1]{section \ref*{#1}}}

\parskip1ex
\parindent0pt
\let\olditemize\itemize
\def\itemize{\olditemize\parskip0pt}

\begin{document}

\title{The \textsf{childdoc} Package}
\hypersetup{pdftitle={The childdoc Package}}
\author{Niklas Beisert\\[2ex]
  Institut f\"ur Theoretische Physik\\
  Eidgen\"ossische Technische Hochschule Z\"urich\\
  Wolfgang-Pauli-Strasse 27, 8093 Z\"urich, Switzerland\\[1ex]
  \href{mailto:nbeisert@itp.phys.ethz.ch}
  {\texttt{nbeisert@itp.phys.ethz.ch}}}
\hypersetup{pdfauthor={Niklas Beisert}}
\hypersetup{pdfsubject={Manual for the LaTeX2e Package childdoc}}
\date{30 December 2018, \textsf{v2.0}}
\maketitle

\begin{abstract}\noindent
\textsf{childdoc} is a \LaTeXe{} package
that enables the direct compilation
of document sections included by |\include|
to individual files.
\end{abstract}

\begingroup
\parskip0ex
\tableofcontents
\endgroup

%%%%%%%%%%%%%%%%%%%%%%%%%%%%%%%%%%%%%%%%%%%%%%%%%%%%%%%%%%%%%%%%%%%%%%%%%%%%%%%%
%%%%%%%%%%%%%%%%%%%%%%%%%%%%%%%%%%%%%%%%%%%%%%%%%%%%%%%%%%%%%%%%%%%%%%%%%%%%%%%%
\section{Introduction}

\LaTeX{} provides a mechanism to structure a large document (such as a book)
into a main file and several child files (containing the chapters)
using the |\include| command.
This mechanism is beneficial for documents
which span hundreds of pages in order to
make the source file(s) more manageable.
Moreover, compilation can be restricted to
selected child files by means of the |\includeonly| command.
The latter feature can be used to reduce the compilation time while editing
(this was significantly more useful in the earlier days of \LaTeX{})
or to generate a smaller document which is easier to navigate.
Another application of |\includeonly| is to generate
documents consisting of selected parts of the complete document.

However, there are a few drawbacks of the plain |\include| mechanism:
\begin{itemize}
\item
The child files cannot be compiled on their own,
they can only be compiled via the main file.
A naive editing environment
(such as a text editor with an option
to have the current file processed by \LaTeX)
may require one to switch to the main file before compiling;
attempting to compile the child file produces errors.
\item
The main file must be modified (each time)
to adjust the |\includeonly| command
to the present needs. This easily leaves the main file in a messy state.
\item
The generated document will always carry the filename
of the main document. This is inconvenient if
several child files are to be compiled and
to be kept for distribution.
\end{itemize}

The present package provides a simple interface
to make child files individually compilable by \LaTeX{}.
Compiling a child file then has the same effect as compiling
the main file with an |\includeonly| command
to select the appropriate child.
Moreover the generated document will carry the name of the child
rather than the main file.
This resolves all three above issues.

This feature is meant to make the editing of books,
thesis documents and lecture notes somewhat more convenient.
However, the package can also be used efficiently for
composing a series of documents (such as exercise sheets)
which are typically distributed individually.
It then assists the author in generating the individual documents
(potentially in different versions)
as well as a document containing the collected series.
Another application is in developing style files
or other kinds of included material
where compilation of the style file could redirect
to a sample or test file.

%%%%%%%%%%%%%%%%%%%%%%%%%%%%%%%%%%%%%%%%%%%%%%%%%%%%%%%%%%%%%%%%%%%%%%%%%%%%%%%%
%%%%%%%%%%%%%%%%%%%%%%%%%%%%%%%%%%%%%%%%%%%%%%%%%%%%%%%%%%%%%%%%%%%%%%%%%%%%%%%%
\section{Usage}

First of all, the package \textsf{childdoc} is \emph{not} a standard
\LaTeXe{} |.sty| style file! Therefore it needs to be invoked in
a non-standard way.

%%%%%%%%%%%%%%%%%%%%%%%%%%%%%%%%%%%%%%%%%%%%%%%%%%%%%%%%%%%%%%%%%%%%%%%%%%%%%%%%
\subsection{Included Files}
\label{sec:include}

%%%%%%%%%%%%%%%%%%%%%%%%%%%%%%%%%%%%%%%%
\DescribeMacro{\childdocmain}
To use the package, add the commands
\begin{center}
\begin{tabular}{l}
|\input{childdoc.def}|\\
|\childdocmain{}|\\
\end{tabular}
\end{center}
at the very top of the main \LaTeX{} file,
in particular \emph{before} the |\documentclass| statement!
The argument of |\childdocmain| should be left empty
(but it must be present).

%%%%%%%%%%%%%%%%%%%%%%%%%%%%%%%%%%%%%%%%
\DescribeMacro{\childdocof}
Furthermore, add the commands
\begin{center}
\begin{tabular}{l}
|\input{childdoc.def}|\\
|\childdocof{|\textit{main}|}|\\
\end{tabular}
\end{center}
at the top of every child file \textit{child}
which is included by |\include{|\textit{child}|}|
from within the main file
(or at least for those files to be compiled individually).
The argument \textit{main} must be the filename of the main file.

There are a couple of
considerations in setting up the main and child documents:

%%%%%%%%%%%%%%%%%%%%%%%%%%%%%%%%%%%%%%%%
\paragraph{Restrictions.}

Please note the following restrictions:
\begin{itemize}
\item
|\childdocmain| must be called with one argument \textit{main}
to ensure compatibility with earlier version of the package.
It must either be empty (|\childdocmain{}|)
or precisely match the filename of the main file in which it is specified.
See \secref{sec:detection} for further information.
\item
The filename \textit{main} must be specified without the |.tex| extension.
\item
The filename \textit{main} is case sensitive
(even in case-insensitive file systems)
due to internal string comparison.
\item
The argument \textit{main} should be fully expanded, it cannot be a macro.
\item
Subdirectories and special characters should be avoided in filenames.
\item
The command |\childdocmain{|\textit{main}|}| must be followed by a whitespace.
It should not be followed immediately by another command
or by a comment mark `|%|'.
This is because the \TeX{} parser reads the token immediately following
the argument of |\childdocmain| and puts it
at the beginning of every child section;
however, a white\-space is ignored.
\end{itemize}

%%%%%%%%%%%%%%%%%%%%%%%%%%%%%%%%%%%%%%%%
\paragraph{Content of Main File.}

It is advisable to place all content in the child files included by |\include|.
Any output contained in the main file will appear in all child documents
unless suppressed manually;
it cannot be suppressed automatically by the |\includeonly| directive
and thus should normally be avoided.
A method to include some content in the main file
by means of conditional processing is described in \secref{sec:conditional}.

%%%%%%%%%%%%%%%%%%%%%%%%%%%%%%%%%%%%%%%%
\paragraph{Page Numbering.}

When only a part of the document is compiled,
the appropriate numbering of pages
(as well as other status parameters)
is determined from the |.aux| files.
The latter contain information from previous passes.
However this information needs to propagate through
all intermediate child documents.
Therefore the page numbering in child documents may well
be inconsistent until the complete document is compiled at least once.

A useful (if unconventional) way to always ensure a consistent
page numbering is to restart the numbering in each child document
and denote the pages by `\textit{child}|.|\textit{page}'
where \textit{child} represents the chapter/section number of the child file.
This can be achieved by the command
|\numberwithin{page}{|\textit{child}|}|
of the \textsf{amsmath} package
where \textit{child} can be |chapter| or |section|
depending on the chosen structuring.
Alternatively, one can modify the macro |\thepage| appropriately
and reset the counter |page| at the start of each child file.

%%%%%%%%%%%%%%%%%%%%%%%%%%%%%%%%%%%%%%%%%%%%%%%%%%%%%%%%%%%%%%%%%%%%%%%%%%%%%%%%
\subsection{Conditional Processing}
\label{sec:conditional}

The package provides a mechanism to compile different versions
of a document. To customise the versions further some conditional processing
can come in handy to distinguish which version is being compiled.
The package provides two macros to describe the compilation context:

%%%%%%%%%%%%%%%%%%%%%%%%%%%%%%%%%%%%%%%%
\DescribeMacro{\ifchilddoc}
The conditional |\ifchilddoc| distinguishes between the compilation of
child documents and the main document:
%
\begin{center}
|\ifchilddoc |\textit{child-code}| |[|\||else |\textit{main-code}]| \||fi|
\end{center}

%%%%%%%%%%%%%%%%%%%%%%%%%%%%%%%%%%%%%%%%
\DescribeMacro{\childdocname}
\DescribeMacro{\childdocjob}
The macro |\childdocname| contains the filename (without extension)
of the main or child file being processed.
Note that |\childdocjob| will always contain the name of the main file.

%%%%%%%%%%%%%%%%%%%%%%%%%%%%%%%%%%%%%%%%
\paragraph{Title Page.}

Conditional processing can be used to include a title or banner page
in the main document when proper precautions are taken.
Importantly, the code in the main file should ensure that the page counter
(as well as other status parameters which are stored in the |.aux| files)
takes the same value after the conditional processing.
Otherwise the page numbers may take divergent values
depending on which part is compiled.

For example, a title page could be declared by:
%
\begin{center}
\begin{tabular}{l}
|\ifchilddoc\||else|\\
|\addtocounter{page}{-1}|\\
\textit{code for title page}\\
|\newpage|\\
|\||fi|
\end{tabular}
\end{center}
%
A banner page for the child documents can be generated by:
%
\begin{center}
\begin{tabular}{l}
|\ifchilddoc|\\
|\addtocounter{page}{-1}|\\
\textit{code for banner page}\\
|\newpage|\\
|\||fi|
\end{tabular}
\end{center}
%
Here one could write a message such as:
\begin{center}
|This is the part \childdocname{} of \childdocjob{}.|
\end{center}

%%%%%%%%%%%%%%%%%%%%%%%%%%%%%%%%%%%%%%%%%%%%%%%%%%%%%%%%%%%%%%%%%%%%%%%%%%%%%%%%
\subsection{Flags}
\label{sec:flags}

The package makes it easy to generate different versions
of the main or child documents.
To this end compilation flags can be defined
and assigned different default values.
They will be particularly useful in conjunction
with the forwarding mechanism described in \secref{sec:forward}.

For example, it may be useful to have a flag |\version|
which can be set to |draft| or |final|.
The document source will contain some conditional code
depending on the value of |\version|.
Suppose further, the flag should default to |final| for the main file
and to |draft| for child files
which is a natural assignment for editing the document.
This is achieved by placing the following code
in the preamble of the main document
(below the |\childdocmain| directive):
%
\begin{center}
\begin{tabular}{l}
|\ifchilddoc|\\
|\providecommand{\version}{draft}|\\
|\||else|\\
|\providecommand{\version}{final}|\\
|\||fi|
\end{tabular}
\end{center}
%
The definition by |\providecommand| makes sure
that previous definitions are not overwritten.
Further statements |\providecommand{\version}{...}|
can thus be added before the above code to override it.

For the main file, one might add a line
(between |\childdocmain| and the above block)
%
\begin{center}
|%\ifchilddoc\||else\providecommand{\version}{draft}\||fi|
\end{center}
%
which can be uncommented to produce a draft version.
Likewise one can add a line to the very top of a child file
(above the |\childdocof{|\textit{main}|}| directive)
%
\begin{center}
|%\providecommand{\version}{final}|
\end{center}
%
which can be uncommented to produce the final version of this child document.

%%%%%%%%%%%%%%%%%%%%%%%%%%%%%%%%%%%%%%%%%%%%%%%%%%%%%%%%%%%%%%%%%%%%%%%%%%%%%%%%
\subsection{Forwarding}
\label{sec:forward}

Different versions of the main or child documents
using compilation flags as described in \secref{sec:flags}
can be (permanently) stored in different files
for convenient compilation, viewing and distribution.
To this end, the package defines a command
to pass on compilation to a different file:

%%%%%%%%%%%%%%%%%%%%%%%%%%%%%%%%%%%%%%%%
\DescribeMacro{\childdocforward}
The command |\childdocforward| redirects processing to
another source file:
%
\begin{center}
\begin{tabular}{l}
|\input{childdoc.def}|\\
|\childdocforward[|\textit{main}|]{|\textit{dest}|}|\\
\end{tabular}
\end{center}
%
The argument \textit{dest} is the destination file
(without extension).
It should be the main file or one of the child files.
Note that further \textsf{childdoc} directives
such as |\childdocof| and |\childdocforward|
in the indicated file will be processed in this form.
The optional argument \textit{main}
passes on directly to the main file \textit{main}
while pretending to compile the child \textit{dest}.
This form behaves as if \textit{dest}
issues |\childdocof{|\textit{main}|}| right away,
and no further \textsf{childdoc} directives will be processed.

%%%%%%%%%%%%%%%%%%%%%%%%%%%%%%%%%%%%%%%%
\DescribeMacro{\...prefix}
In the alternative form |\childdocforwardprefix|,
%
\begin{center}
\begin{tabular}{l}
|\input{childdoc.def}|\\
|\childdocforwardprefix[|\textit{main}|]{|\textit{prefix}|}{|\textit{dest}|}|
\end{tabular}
\end{center}
%
the destination file is determined by a pattern
depending on the current file:
To make this work, the current file must be called
`{\textit{prefix}\hspace{0.2em}\textit{suffix}}'
with \textit{prefix} matching precisely the argument.
Processing is then passed on to the file
`{\textit{dest}\hspace{0.2em}\textit{suffix}}'.
Surely, the same effect is achieved by
directly specifying the
argument `{\textit{dest}\hspace{0.2em}\textit{suffix}}'
in the first form.
However, that requires to set up a different file
for each child. With the alternative form of the command
all these files can have exactly the same content
which simplifies setting them up and maintaining them.

For example, the following file |draft.tex|
with a compilation flag |\version| as described in \secref{sec:flags}
compiles the main document as a draft:
%
\begin{center}
\begin{tabular}{l}
|\def\version{draft}|\\
|\input{childdoc.def}|\\
|\childdocforward{|\textit{main}|}|
\end{tabular}
\end{center}
%
Likewise, the following files |final|\textit{nn}|.tex|
compile the final version of the child document
|child|\textit{nn}|.tex|:
%
\begin{center}
\begin{tabular}{l}
|\def\version{final}|\\
|\input{childdoc.def}|\\
|\childdocforwardprefix{final}{child}|
\end{tabular}
\end{center}
%

Note that when several versions of a main file and/or of each child file
are to be generated, it may be convenient to set up a |Makefile| or
shell script to automatise the process.

%%%%%%%%%%%%%%%%%%%%%%%%%%%%%%%%%%%%%%%%%%%%%%%%%%%%%%%%%%%%%%%%%%%%%%%%%%%%%%%%
\subsection{Command Line Processing}
\label{sec:commandline}

The effect of redirection files can also be achieved by invoking
the \LaTeX{} compiler with a more elaborate command line.
Most conveniently this should be done as part
of a shell script or a |Makefile|.

When using \textsf{childdoc} in the main file, the following
command lines effectively perform a redirection
(note that depending on the shell being used,
backslashes may have to be doubled: `|\|' $\to$ `|\\|'):
%
\begin{center}
|... -jobname "|\textit{target}|" |\\|"|[\textit{flags}]%
|\input{childdoc.def}\childdocforward[|\textit{main}|]{|\textit{dest}|}"|
\end{center}
%
Here \textit{target} is the name of the output file,
\textit{main} is the name of the main file
and \textit{dest} is the name of the main or child file to be processed
(all filenames without extensions).
The optional argument \textit{main} can be omitted
if \textit{main} matches \textit{dest}.
Optionally, compilation \textit{flags} can be defined via |\def| commands.
This command line makes the \TeX{} engine believe
it is compiling the file \textit{target}
whose content is specified as the latter parameter.
The provided code then forwards the processing to
\textit{main} or \textit{dest} as described in \secref{sec:forward}.

%%%%%%%%%%%%%%%%%%%%%%%%%%%%%%%%%%%%%%%%%%%%%%%%%%%%%%%%%%%%%%%%%%%%%%%%%%%%%%%%
\subsection{Include by Input}
\label{sec:input}

Including child documents by |\include| has some restrictions by design.
Most notably, the content of a child document always occupies
its own set of pages; pages cannot be shared between child documents.
Usually, this behaviour makes perfect sense
because each child document contain an essential part of the document.
However, in some situations it may be desirable to compose
a document from a collection of parts
without having mandatory page breaks between then.
For this case, the package
provides a mechanism to include parts
by |\input| which can also be processed individually.
However, by construction this mechanism
requires manual handling of the content to be output.

%%%%%%%%%%%%%%%%%%%%%%%%%%%%%%%%%%%%%%%%
\DescribeMacro{\ifchilddocmanual}
The main file should be prepared as usual, see \secref{sec:include}.
However, the document body must make a distinction
between processing of an individual part and of the main document, e.g.:
%
\begin{center}
\begin{tabular}{l}
|\ifchilddocmanual|\\
|\input{\childdocname}|\\
|\||else|\\
\textit{document body with }|\input{|\textit{part}|}|\\
|\||fi|
\end{tabular}
\end{center}
%
The conditional |\ifchilddocmanual| is true whenever
a part to be included by |\input| is being compiled,
and the name of the part is stored in |\childdocname|.

%%%%%%%%%%%%%%%%%%%%%%%%%%%%%%%%%%%%%%%%
\DescribeMacro{\childdocby}
Each part to be included by |\input| should start with:
%
\begin{center}
\begin{tabular}{l}
|\input{childdoc.def}|\\
|\childdocby{|\textit{main}|}|\\
\end{tabular}
\end{center}
%
The directive |\childdocby| is similar to |\childdocof|
described in \secref{sec:include},
but the subsequent selection of content must be done manually.
To that end, both |\ifchilddoc| and |\ifchilddocmanual|
will be true upon processing of a part,
and the name of the part is stored in |\childdocname|.
Note that |\jobname| will be set to the filename of the current part
so that each part receives an individual |.aux| file
that does not interfere with the |.aux| file(s) of the main document.
This behaviour can be altered by the alternative form
|\childdocby[*]{|\textit{main}|}| (with a non-empty optional argument)
which uses the |.aux| file of the main document
by setting |\jobname| to \textit{main}.

%%%%%%%%%%%%%%%%%%%%%%%%%%%%%%%%%%%%%%%%%%%%%%%%%%%%%%%%%%%%%%%%%%%%%%%%%%%%%%%%
\subsection{Driver Development}
\label{sec:driver}

The \textsf{childdoc} mechanism can also be use for the development
of definition files such as \LaTeX{} styles or classes.
This case differs from the above setup with multiple parts
included by |\include| in that no |\includeonly| should be invoked.
This can be achieved by starting the include file
(before |\ProvidesPackage|) with:
%
\begin{center}
\begin{tabular}{l}
|\input{childdoc.def}|\\
|\childdocforward{|\textit{main}|}|\\
\end{tabular}
\end{center}
%
or alternatively with:
%
\begin{center}
\begin{tabular}{l}
|\input{childdoc.def}|\\
|\childdocby{|\textit{main}|}|\\
\end{tabular}
\end{center}
%
Both forms have slightly different effects as described above.
The main file is prepared as usual, see \secref{sec:include}.

%%%%%%%%%%%%%%%%%%%%%%%%%%%%%%%%%%%%%%%%%%%%%%%%%%%%%%%%%%%%%%%%%%%%%%%%%%%%%%%%
\subsection{Legacy Detection}
\label{sec:detection}

The directive |\childdocmain| in the main file can detect
whether the complete document or merely a child is to be compiled
even without using the directive |\childdocof|.
This method is deprecated because it is less robust
and there is no compelling reason to use it;
it is merely provided for backward compatibility
and it may be removed in future versions.

If the detection mechanism is to be used,
it is mandatory to correctly specify
the filename of the main file as the argument of |\childdocmain|:
%
\begin{center}
\begin{tabular}{l}
|\input{childdoc.def}|\\
|\childdocmain{|\textit{main}|}|\\
\end{tabular}
\end{center}
%
If |\jobname| does not match the argument \textit{main} of |\childdocmain|,
it is assumed that |\jobname| points to the child file to be compiled.
When using |\childdocmain| with the main file specified as argument,
it suffices to start a child file
with just |\input{|\textit{main}|}|
without loading of the package and using |\childdocof|.
If instead all processing is done
with the appropriate \textsf{childdoc} directives,
the argument of \textit{main} of |\childdocmain| can be empty.

An alternative version of the command line processing described
in \secref{sec:commandline} using the detection mechanism reads:
%
\begin{center}
|... -jobname "|\textit{target}|" "|[\textit{flags}]%
[|\def\jobname{|\textit{dest}|}|]|\input{|\textit{main}|}"|
\end{center}

%%%%%%%%%%%%%%%%%%%%%%%%%%%%%%%%%%%%%%%%%%%%%%%%%%%%%%%%%%%%%%%%%%%%%%%%%%%%%%%%
\subsection{Manual Code}
\label{sec:manual}

In case one cannot be certain whether the definitions file |childdoc.def|
is installed on the target \TeX{} distribution
and one prefers not to ship it,
it is conceivable to paste a few relevant commands into the sources.

To that end, drop all statements |\input{childdoc.def}|
and perform the replacements as outlined below.
Instead of |\childdocmain{|\textit{main}|}| add the following code
to the top of the main file:
%
\begin{center}
\begin{tabular}{l}
|\||ifdefined\childdocname\endinput\||fi\newif\ifchilddoc|\\
|\edef\childdocname{\scantokens\expandafter{\jobname\noexpand}}|\\
|\def\childdocmain{|\textit{main}|}\||ifx\childdocmain\childdocname\||else|\\
|\childdoctrue\includeonly{\childdocname}\let\jobname\childdocmain\||fi|\\
\end{tabular}
\end{center}
%
Instead of |\childdocof{|\textit{main}|}| just include the main file
at the top of each child file:
%
\begin{center}
|\input{|\textit{main}|}|
\end{center}
%
A simple redirection |\childdocforward{|\textit{dest}|}| is achieved by:
%
\begin{center}
|\def\jobname{|\textit{dest}|}\input{\jobname}|
\end{center}
%
The redirection with prefix
|\childdocforwardprefix[|\textit{prefix}|]{|\textit{dest}|}|
is accomplished by:
%
\begin{center}
\begin{tabular}{l}
|{\edef\jobname{\scantokens\expandafter{\jobname\noexpand}}|\\
|\def\redirectjob |\textit{prefix}|#1~~~{\gdef\jobname{|\textit{dest}|#1}}|\\
|\expandafter\redirectjob\jobname~~~}\input{\jobname}|
\end{tabular}
\end{center}

In an alternative approach,
child documents can be compiled by a specific command line
without additional code or specific definitions:
%
\begin{center}
|... -jobname "|\textit{target}|" "|[\textit{flags}]%
|\includeonly{|\textit{dest}|}\input{|\textit{main}|}"|
\end{center}
%

%%%%%%%%%%%%%%%%%%%%%%%%%%%%%%%%%%%%%%%%%%%%%%%%%%%%%%%%%%%%%%%%%%%%%%%%%%%%%%%%
%%%%%%%%%%%%%%%%%%%%%%%%%%%%%%%%%%%%%%%%%%%%%%%%%%%%%%%%%%%%%%%%%%%%%%%%%%%%%%%%
\section{Information}

%%%%%%%%%%%%%%%%%%%%%%%%%%%%%%%%%%%%%%%%%%%%%%%%%%%%%%%%%%%%%%%%%%%%%%%%%%%%%%%%
\subsection{Copyright}

Copyright \copyright{} 2017--2018 Niklas Beisert

This work may be distributed and/or modified under the
conditions of the \LaTeX{} Project Public License, either version 1.3
of this license or (at your option) any later version.
The latest version of this license is in
  \url{http://www.latex-project.org/lppl.txt}
and version 1.3 or later is part of all distributions of \LaTeX{}
version 2005/12/01 or later.

This work has the LPPL maintenance status `maintained'.

The Current Maintainer of this work is Niklas Beisert.

This work consists of the files |README.txt|, |childdoc.ins| and |childdoc.dtx|
as well as the derived files |childdoc.def|, |cdocsamp.tex|
with |cdocsch1.tex|, |cdocsch2.tex|, |cdocspt3.tex|, |cdocspt4.tex|,
|cdocsdrf.tex|, |cdocsfn1.tex|, |cdocsfn2.tex|
as well as |childdoc.pdf|.

%%%%%%%%%%%%%%%%%%%%%%%%%%%%%%%%%%%%%%%%%%%%%%%%%%%%%%%%%%%%%%%%%%%%%%%%%%%%%%%%
\subsection{Files and Installation}

The package consists of the files:
%
\begin{center}
\begin{tabular}{ll}
    |README.txt|   & readme file \\
    |childdoc.ins| & installation file \\
    |childdoc.dtx| & source file \\
    |childdoc.def| & definition file \\
    |cdocsamp.tex| & sample main file \\
    |cdocsch1.tex| & sample include file \\
    |cdocsch2.tex| & sample include file \\
    |cdocspt3.tex| & sample part file \\
    |cdocspt4.tex| & sample part file \\
    |cdocsdrf.tex| & sample redirection file \\
    |cdocsfn1.tex| & sample redirection file \\
    |cdocsfn2.tex| & sample redirection file \\
    |childdoc.pdf| & manual
\end{tabular}
\end{center}
%
The distribution consists of the files
|README.txt|, |childdoc.ins| and |childdoc.dtx|.
%
\begin{itemize}
\item
Run (pdf)\LaTeX{} on |childdoc.dtx|
to compile the manual |childdoc.pdf| (this file).
\item
Run \LaTeX{} on |childdoc.ins| to create the definitions file |childdoc.def|
and the sample |cdocsamp.tex| with include files
|cdocsch1.tex|, |cdocsch2.tex|, |cdocspt3.tex|, |cdocspt4.tex|,
|cdocsdrf.tex|, |cdocsfn1.tex|, |cdocsfn2.tex|.
Then copy the file |childdoc.def| to an appropriate directory of your \LaTeX{}
distribution, e.g.\ \textit{texmf-root}|/tex/latex/childdoc|.
\end{itemize}

%%%%%%%%%%%%%%%%%%%%%%%%%%%%%%%%%%%%%%%%%%%%%%%%%%%%%%%%%%%%%%%%%%%%%%%%%%%%%%%%
\subsection{Related CTAN Packages}

There are several other packages which offer a similar functionality:
%
\begin{itemize}
\item
The packages
\href{http://ctan.org/pkg/docmute}{\textsf{docmute}},
\href{http://ctan.org/pkg/includex}{\textsf{includex}} and
\href{http://ctan.org/pkg/standalone}{\textsf{standalone}}
provide commands to include only the document body of
a child file thus allowing both files to be compiled individually.
\item
The packages \href{http://ctan.org/pkg/subdocs}{\textsf{subdocs}}
and \href{http://ctan.org/pkg/subfiles}{\textsf{subfiles}}
provide structures in which the main and child documents can be
encapsulated and allowing them to be compiled individually.
The inclusion mechanism is different from the conventional |\include|.
\item
The package \href{http://ctan.org/pkg/combine}{\textsf{combine}}
is an elaborate solution to combine several documents into one.
\end{itemize}
%
See also the CTAN topic \href{http://ctan.org/topic/subdocs}{\textsf{subdocs}}
for further related packages.
The present package differs from the above solutions in that
a document structure constructed with the conventional |\include| mechanism
just needs two extra commands at the top of every file
such that all constituent files can be compiled individually.

%%%%%%%%%%%%%%%%%%%%%%%%%%%%%%%%%%%%%%%%%%%%%%%%%%%%%%%%%%%%%%%%%%%%%%%%%%%%%%%%
%\subsection{Feature Suggestions}
%
%The following is a list of features which may be useful for future
%versions of this package:
%%
%\begin{itemize}
%\item
%\ldots
%\end{itemize}

%%%%%%%%%%%%%%%%%%%%%%%%%%%%%%%%%%%%%%%%%%%%%%%%%%%%%%%%%%%%%%%%%%%%%%%%%%%%%%%%
\subsection{Revision History}

%%%%%%%%%%%%%%%%%%%%%%%%%%%%%%%%%%%%%%%%
\paragraph{v2.0:} 2018/12/30

\begin{itemize}
\item
immediate forward processing
\item
added |\childdocby| mechanism
\item
manual restructured
\end{itemize}

%%%%%%%%%%%%%%%%%%%%%%%%%%%%%%%%%%%%%%%%
\paragraph{v1.6:} 2018/01/17

\begin{itemize}
\item
application for development of include files
\item
corrections to manual
\end{itemize}

%%%%%%%%%%%%%%%%%%%%%%%%%%%%%%%%%%%%%%%%
\paragraph{v1.5:} 2017/05/21

\begin{itemize}
\item
more complete structuring introduced
\item
|\childdocof| introduced
\item
|\childdoc| renamed to |\childdocmain|
\item
|\childredirect| renamed to |\childdocforward| and |\childdocforwardprefix|
and functionality expanded
\end{itemize}

%%%%%%%%%%%%%%%%%%%%%%%%%%%%%%%%%%%%%%%%
\paragraph{v1.0:} 2017/04/27

\begin{itemize}
\item
manual and install package
\item
first version published on CTAN
\end{itemize}

%%%%%%%%%%%%%%%%%%%%%%%%%%%%%%%%%%%%%%%%
\paragraph{v0.6:} 2017/04/26

\begin{itemize}
\item
redirection mechanism added
\end{itemize}

%%%%%%%%%%%%%%%%%%%%%%%%%%%%%%%%%%%%%%%%
\paragraph{v0.5:} 2017/04/26

\begin{itemize}
\item
functionality in definition file
\end{itemize}


%%%%%%%%%%%%%%%%%%%%%%%%%%%%%%%%%%%%%%%%%%%%%%%%%%%%%%%%%%%%%%%%%%%%%%%%%%%%%%%%
%%%%%%%%%%%%%%%%%%%%%%%%%%%%%%%%%%%%%%%%%%%%%%%%%%%%%%%%%%%%%%%%%%%%%%%%%%%%%%%%
%%%%%%%%%%%%%%%%%%%%%%%%%%%%%%%%%%%%%%%%%%%%%%%%%%%%%%%%%%%%%%%%%%%%%%%%%%%%%%%%
\appendix

\settowidth\MacroIndent{\rmfamily\scriptsize 000\ }

 \DocInput{childdoc.dtx}

\end{document}
%</driver>
% \fi
%
% %%%%%%%%%%%%%%%%%%%%%%%%%%%%%%%%%%%%%%%%%%%%%%%%%%%%%%%%%%%%%%%%%%%%%%%%%%%%%%
% %%%%%%%%%%%%%%%%%%%%%%%%%%%%%%%%%%%%%%%%%%%%%%%%%%%%%%%%%%%%%%%%%%%%%%%%%%%%%%
% \section{Sample}
%\iffalse
%<*samplemain>
%\fi
%
% The following presents a sample document
% with two chapters, two parts, a title page,
% a compile flag as well as three forwarding files to set the flag.
% It consists of eight |.tex| files:
% \begin{center}
% \begin{tabular}{ll}
% |cdocsamp.tex|&main file\\
% |cdocsch1.tex|&include file for chapter 1\\
% |cdocsch2.tex|&include file for chapter 2\\
% |cdocspt3.tex|&include file for part 3\\
% |cdocspt4.tex|&include file for part 4\\
% |cdocsdrf.tex|&forwarding file for main file in draft mode\\
% |cdocsfi1.tex|&forwarding file for final version of chapter 1\\
% |cdocsfi2.tex|&forwarding file for final version of chapter 2\\
% \end{tabular}
% \end{center}
% Each of the eight files can be compiled directly by the \LaTeX{} compiler.
%
% %%%%%%%%%%%%%%%%%%%%%%%%%%%%%%%%%%%%%%
% \paragraph{Main File.}
%
% The main file is called |cdocsamp.tex|.
%
% Load the \textsf{childdoc} definitions and
% declare the filename for the main document:
%    \begin{macrocode}
\input{childdoc.def}
\childdocmain{}
%    \end{macrocode}

% Optional override for |\version| flag:
%    \begin{macrocode}
%%\ifchilddoc\else\providecommand{\version}{draft}\fi
%    \end{macrocode}

% Define the default values for the |\version| flag
% (|final| for the main file and |draft| for childs):
%    \begin{macrocode}
\ifchilddoc
\providecommand{\version}{draft}
\else
\providecommand{\version}{final}
\fi
%    \end{macrocode}

% Load the standard document class:
%    \begin{macrocode}
\documentclass[12pt]{article}
%    \end{macrocode}

% Start the document body:
%    \begin{macrocode}
\begin{document}
%    \end{macrocode}

% Declare a title page.
% Print title, part of document being processed and version flag:
%    \begin{macrocode}
\addtocounter{page}{-1}
\begin{center}
{\LARGE\bfseries{}childdoc example\par}
\vspace{1cm}
\ifchilddoc
\ifchilddocmanual part\else chapter\fi:
`\childdocname' of `\childdocjob'\par
\else
main document: `\childdocjob'\par
\fi
version: \version\par
\end{center}
\newpage
%    \end{macrocode}

% Manually include selected file,
% otherwise process as usual:
%    \begin{macrocode}
\ifchilddocmanual
\section*{part `\childdocname'}
\input{\childdocname}
\else
%    \end{macrocode}

% Include the two chapters:
%    \begin{macrocode}
\include{cdocsch1}
\include{cdocsch2}
%    \end{macrocode}

% Include the two parts unless only chapters should be displayed:
%    \begin{macrocode}
\ifchilddoc\else
\section{part three}
\input{cdocspt3}
\section{part four}
\input{cdocspt4}
\fi
%    \end{macrocode}

% Process as usual until here:
%    \begin{macrocode}
\fi
%    \end{macrocode}

% End of document body:
%    \begin{macrocode}
\end{document}
%    \end{macrocode}
%\iffalse
%</samplemain>
%\fi
%
% %%%%%%%%%%%%%%%%%%%%%%%%%%%%%%%%%%%%%%
% \paragraph{Chapter Include Files.}
%
% The include files are called |cdocsch1.tex| and |cdocsch2.tex|.
%
%\iffalse
%<*samplechap1|samplechap2>
%\fi

% Optional override for |\version| flag:
%    \begin{macrocode}
%%\providecommand{\version}{final}
%    \end{macrocode}

% Include the main document:
%    \begin{macrocode}
\input{childdoc.def}
\childdocof{cdocsamp}
%    \end{macrocode}

%\iffalse
%</samplechap1|samplechap2>
%\fi
%
%\iffalse
%<*samplechap1>
%\fi
% Some text for chapter 1:
%    \begin{macrocode}
\section{one}
some text in chapter one
%    \end{macrocode}

%\iffalse
%</samplechap1>
%\fi
% Some text for chapter 2:
%\iffalse
%<*samplechap2>
%\fi
%    \begin{macrocode}
\section{two}
more text in chapter two
%    \end{macrocode}

%\iffalse
%</samplechap2>
%\fi
%
% %%%%%%%%%%%%%%%%%%%%%%%%%%%%%%%%%%%%%%
% \paragraph{Part Include Files.}
%
% The include files are called |cdocspt3.tex| and |cdocspt4.tex|.
%
%\iffalse
%<*samplepart3|samplepart4>
%\fi

% Optional override for |\version| flag:
%    \begin{macrocode}
%%\providecommand{\version}{final}
%    \end{macrocode}

% Include the main document:
%    \begin{macrocode}
\input{childdoc.def}
\childdocby{cdocsamp}
%    \end{macrocode}

%\iffalse
%</samplepart3|samplepart4>
%\fi
%
%\iffalse
%<*samplepart3>
%\fi
% Some text for part 3:
%    \begin{macrocode}
some text in part three
%    \end{macrocode}

%\iffalse
%</samplepart3>
%\fi
% Some text for part 4:
%\iffalse
%<*samplepart4>
%\fi
%    \begin{macrocode}
more text in part four
%    \end{macrocode}

%\iffalse
%</samplepart4>
%\fi
%
% %%%%%%%%%%%%%%%%%%%%%%%%%%%%%%%%%%%%%%
% \paragraph{Forwarding for a Complete Draft.}
%
% The following forwarding file |cdocsdrf.tex|
% compiles the main document in draft mode:
%\iffalse
%<*sampledraft>
%\fi
%    \begin{macrocode}
\def\version{draft}
\input{childdoc.def}
\childdocforward{cdocsamp}
%    \end{macrocode}

%\iffalse
%</sampledraft>
%\fi
%
% %%%%%%%%%%%%%%%%%%%%%%%%%%%%%%%%%%%%%%
% \paragraph{Forwarding for Final Version of the Chapters.}
%
% The following forwarding files |cdocsfn1.tex| and |cdocsfn2.tex|
% (with identical content)
% compile the final versions of the child documents
% |cdocsch1.tex| and |cdocsch2.tex|, respectively:
%\iffalse
%<*samplefinal>
%\fi
%    \begin{macrocode}
\def\version{final}
\input{childdoc.def}
\childdocforwardprefix[cdocsamp]{cdocsfn}{cdocsch}
%    \end{macrocode}

%\iffalse
%</samplefinal>
%\fi
%
% %%%%%%%%%%%%%%%%%%%%%%%%%%%%%%%%%%%%%%
% \paragraph{Command Line Processing.}
%
% The following three command lines generate the output files
% |cdocscld|, |cdocscl1| and |cdocscl2|
% which should be identical to
% |cdocsdrf|, |cdocsch1| and |cdocsfn2|, respectively:
% \begin{center}
% \begin{tabular}{l}
% |latex -jobname cdocscld \|\\
% |  "\def\version{draft}\input{childdoc.def}\childdocforward{cdocsamp}"|\\
% |latex -jobname cdocscl1 \|\\
% |  "\input{childdoc.def}\childdocforward[cdocsamp]{cdocsch1}"|\\
% |latex -jobname cdocscl2 \|\\
% |  "\def\version{final}\input{childdoc.def}\childdocforward{cdocsch2}"|
% \end{tabular}
% \end{center}
% Note that the trailing backslash on each first line
% merely continues the input to the second line
% (for convenient cut ant paste).
% Furthermore, the command |latex| can be replaced by any
% of its alternative versions such as |pdflatex|.
%
% %%%%%%%%%%%%%%%%%%%%%%%%%%%%%%%%%%%%%%%%%%%%%%%%%%%%%%%%%%%%%%%%%%%%%%%%%%%%%%
% %%%%%%%%%%%%%%%%%%%%%%%%%%%%%%%%%%%%%%%%%%%%%%%%%%%%%%%%%%%%%%%%%%%%%%%%%%%%%%
% \section{Implementation}
%\iffalse
%<*package>
%\fi
%
% This section describes the definitions file |childdoc.def|.

% The definitions cannot be loaded using |\usepackage| or |\RequirePackage|
% which has a mechanism to prevent loading a style file more than once.
% When loading the definitions by means of |\input|
% multiple instances have to be prevented manually:
%\iffalse
%This code needs to be before the `\ProvidesFile' directive
%which is defined at the beginning of this file.
%Therefore it is also placed there and commented out here.
%</package>
%<*discard>
%\fi
%    \begin{macrocode}
\ifdefined\childdocmain\endinput\fi
%    \end{macrocode}
%\iffalse
%</discard>
%<*package>
%\fi
%
% \macro{\ifchilddoc}
% \macro{\ifchilddocmanual}
% The conditional |\ifchilddoc| tells whether a
% child (true) or main (false) document is being compiled.
% The conditional |\ifchilddocmanual| tells whether
% the |\includeonly| mechanism is used (false) or
% the selection of child files must be performed manually (true).
% The definitions initialise to false:
%    \begin{macrocode}
\newif\ifchilddoc
\newif\ifchilddocmanual
%    \end{macrocode}

% \macro{\childdocname}
% \macro{\childdocjob}
% The macro |\childdocname| stores the name of the main document
% to be compiled. The macro |\childdocjob| stores the name of
% the document on which the \LaTeX{} compiler was originally invoked.
% The content of |\jobname| cannot be compared
% to filenames specified in the source due to different catcodes.
% The following code rescans |\jobname|, stores the result
% in |\childdocname| and saves a copy in |\childdocjob|:
%    \begin{macrocode}
\edef\childdocname{\scantokens\expandafter{\jobname\noexpand}}
\let\childdocjob\childdocname
%    \end{macrocode}

% \macro{\childdocdisable}
% The macro |\childdocdisable| prevents the main file
% from being processed more than once.
% At this stage, the main document command |\childdocmain|
% is assumed to be called once again where it should do nothing.
% Any subsequent call to it should prevent
% a secondary processing of the main document
% It overwrites the forwarding commands
% |\childdocof| and |\childdocforward|
% with empty macros to prevent further inclusions of the main document:
%    \begin{macrocode}
\newcommand{\childdocdisable}
{
  \renewcommand{\childdocmain}[1]{\renewcommand{\childdocmain}[1]{\endinput}}
  \renewcommand{\childdocof}[1]{}
  \renewcommand{\childdocby}[2][]{}
  \renewcommand{\childdocforward}[2][]{}
  \renewcommand{\childdocdisable}{}
}
%    \end{macrocode}

% \macro{\childdocmain}
% The macro |\childdocmain| is to be called at the top of the main file
% with nothing or the main filename (without extension) as argument.
% First, it breaks loops.
% If the argument is not empty and does not match |\childdocname|
% (which is set by the first inclusion of |childdoc.def|),
% |\ifchilddoc| is set to true, |\includeonly| is applied to the child file
% and |\jobname| is set to the main file
% (for proper handling of |.aux| files):
%    \begin{macrocode}
\newcommand{\childdocmain}[1]
{
  \childdocdisable\childdocmain{}
  \if?#1?\else
    \begingroup
      \def\childdoctmp{#1}
      \ifx\childdoctmp\childdocname
        \def\childdoctmp{}
      \else
        \def\childdoctmp
        {
          \childdoctrue
          \includeonly{\childdocname}
          \def\childdocjob{#1}
          \def\jobname{#1}
        }
      \fi
      \expandafter
    \endgroup
    \childdoctmp
  \fi
}
%    \end{macrocode}

% \macro{\childdocof}
% The command |\childdocof| redirects
% compilation to the main file |#1|.
%    \begin{macrocode}
\newcommand{\childdocof}[1]
{
  \childdocdisable
  \childdoctrue
  \includeonly{\childdocname}
  \def\jobname{#1}
  \def\childdocjob{#1}
  \input{#1}
}
%    \end{macrocode}

% \macro{\childdocby}
% The command |\childdocby| ....
%    \begin{macrocode}
\newcommand{\childdocby}[2][]
{
  \childdocdisable
  \childdoctrue
  \childdocmanualtrue
  \if?#1?\else
    \def\jobname{#2}
  \fi
  \def\childdocjob{#2}
  \input{#2}
  \endinput
}
%    \end{macrocode}

% \macro{\childdocforward}
% The command |\childdocforward| redirects
% compilation to the main file or
% (if the optional argument is given) a child file.
% Parameters are set as if the main file
% or a child file starting with |\childdocof| was compiled.
% Then compilation is handed over to the main file:
%    \begin{macrocode}
\newcommand{\childdocforward}[2][]
{
  \begingroup
    \if?#1?
      \def\childdoctmp
      {
        \def\childdocname{#2}
        \def\childdocjob{#2}
        \def\jobname{#2}
        \input{#2}
        \endinput
      }
    \else
      \def\childdoctmp
      {
        \childdocdisable
        \def\childdocname{#2}
        \childdoctrue
        \includeonly{#2}
        \def\childdocjob{#1}
        \def\jobname{#1}
        \input{#1}
        \endinput
      }
    \fi
    \expandafter
  \endgroup
  \childdoctmp
}
%    \end{macrocode}

% \macro{\childdocforwardprefix}
% The command |\childdocforwardprefix| redirects
% compilation to the main or a child file by means of a pattern.
% The prefix |#1| in the current filename is replaced by |#2|
% and the suffix of the current filename is kept
% (it is assumed that the filename does not contain the substring `|~~~|'
% which is used as a delimiter).
% Compilation is handed over to the new file by |\childdocforward|:
%    \begin{macrocode}
\newcommand{\childdocforwardprefix}[3][]
{
  \begingroup
    \def\childdocextract #2##1~~~{\def\childdoctmp{\childdocforward[#1]{#3##1}}}
    \expandafter\childdocextract\childdocname~~~
    \expandafter
  \endgroup
  \childdoctmp
}
%    \end{macrocode}

% \macro{\childdoc}
% The deprecated macro |\childdoc| is a legacy version of |\childdocmain|:
%    \begin{macrocode}
\newcommand{\childdoc}{\childdocmain}
%    \end{macrocode}

% \macro{\childdocredirect}
% The deprecated macro |\childdocredirect| is a legacy version
% of |\childdocforward| and |\childdocforwardprefix|:
%    \begin{macrocode}
\newcommand{\childdocredirect}[2][]
{
  \begingroup
    \if?#1?
      \def\childdoctmp{\childdocforward{#2}}
    \else
      \def\childdoctmp{\childdocforwardprefix{#1}{#2}}
    \fi
    \expandafter
  \endgroup
  \childdoctmp
}
%    \end{macrocode}

%\iffalse
%</package>
%\fi
%
\endinput

\childdocforward{cdocsamp}
%    \end{macrocode}

%\iffalse
%</sampledraft>
%\fi
%
% %%%%%%%%%%%%%%%%%%%%%%%%%%%%%%%%%%%%%%
% \paragraph{Forwarding for Final Version of the Chapters.}
%
% The following forwarding files |cdocsfn1.tex| and |cdocsfn2.tex|
% (with identical content)
% compile the final versions of the child documents
% |cdocsch1.tex| and |cdocsch2.tex|, respectively:
%\iffalse
%<*samplefinal>
%\fi
%    \begin{macrocode}
\def\version{final}
% \iffalse
%
% childdoc.dtx Copyright (C) 2017-2018 Niklas Beisert
%
% This work may be distributed and/or modified under the
% conditions of the LaTeX Project Public License, either version 1.3
% of this license or (at your option) any later version.
% The latest version of this license is in
%   http://www.latex-project.org/lppl.txt
% and version 1.3 or later is part of all distributions of LaTeX
% version 2005/12/01 or later.
%
% This work has the LPPL maintenance status `maintained'.
%
% The Current Maintainer of this work is Niklas Beisert.
%
% This work consists of the files childdoc.dtx and childdoc.ins
% and the derived files childdoc.def and cdocsamp.tex with
% cdocsch1.tex, cdocsch2.tex, cdocsdrf.tex, cdocsfn1.tex, cdocsfn2.tex.
%
%<package>\ifdefined\childdocmain\endinput\fi
%<package>\ProvidesFile{childdoc.def}[2018/12/30 v2.0 child document driver]
%<samplemain>\ProvidesFile{cdocsamp.tex}[2018/12/30 v2.0 sample for childdoc]
%<*driver>
%\ProvidesFile{childdoc.drv}[2018/12/30 v2.0 childdoc reference manual file]
\PassOptionsToClass{10pt,a4paper}{article}
\documentclass{ltxdoc}

\usepackage[margin=35mm]{geometry}
\usepackage{hyperref}
\usepackage{hyperxmp}
\usepackage[usenames]{color}

\hypersetup{colorlinks=true}
\hypersetup{pdfstartview=FitH}
\hypersetup{pdfpagemode=UseNone}
\hypersetup{pdfsource={}}
\hypersetup{pdflang={en-UK}}
\hypersetup{pdfcopyright={Copyright 2017-2018 Niklas Beisert.
  This work may be distributed and/or modified under the
  conditions of the LaTeX Project Public License, either version 1.3
  of this license or (at your option) any later version.}}
\hypersetup{pdflicenseurl={http://www.latex-project.org/lppl.txt}}
\hypersetup{pdfcontactaddress={ETH Zurich, ITP, HIT K,
  Wolfgang-Pauli-Strasse 27}}
\hypersetup{pdfcontactpostcode={8093}}
\hypersetup{pdfcontactcity={Zurich}}
\hypersetup{pdfcontactcountry={Switzerland}}
\hypersetup{pdfcontactemail={nbeisert@itp.phys.ethz.ch}}
\hypersetup{pdfcontacturl={http://people.phys.ethz.ch/\xmptilde nbeisert/}}

\newcommand{\secref}[1]{\hyperref[#1]{section \ref*{#1}}}

\parskip1ex
\parindent0pt
\let\olditemize\itemize
\def\itemize{\olditemize\parskip0pt}

\begin{document}

\title{The \textsf{childdoc} Package}
\hypersetup{pdftitle={The childdoc Package}}
\author{Niklas Beisert\\[2ex]
  Institut f\"ur Theoretische Physik\\
  Eidgen\"ossische Technische Hochschule Z\"urich\\
  Wolfgang-Pauli-Strasse 27, 8093 Z\"urich, Switzerland\\[1ex]
  \href{mailto:nbeisert@itp.phys.ethz.ch}
  {\texttt{nbeisert@itp.phys.ethz.ch}}}
\hypersetup{pdfauthor={Niklas Beisert}}
\hypersetup{pdfsubject={Manual for the LaTeX2e Package childdoc}}
\date{30 December 2018, \textsf{v2.0}}
\maketitle

\begin{abstract}\noindent
\textsf{childdoc} is a \LaTeXe{} package
that enables the direct compilation
of document sections included by |\include|
to individual files.
\end{abstract}

\begingroup
\parskip0ex
\tableofcontents
\endgroup

%%%%%%%%%%%%%%%%%%%%%%%%%%%%%%%%%%%%%%%%%%%%%%%%%%%%%%%%%%%%%%%%%%%%%%%%%%%%%%%%
%%%%%%%%%%%%%%%%%%%%%%%%%%%%%%%%%%%%%%%%%%%%%%%%%%%%%%%%%%%%%%%%%%%%%%%%%%%%%%%%
\section{Introduction}

\LaTeX{} provides a mechanism to structure a large document (such as a book)
into a main file and several child files (containing the chapters)
using the |\include| command.
This mechanism is beneficial for documents
which span hundreds of pages in order to
make the source file(s) more manageable.
Moreover, compilation can be restricted to
selected child files by means of the |\includeonly| command.
The latter feature can be used to reduce the compilation time while editing
(this was significantly more useful in the earlier days of \LaTeX{})
or to generate a smaller document which is easier to navigate.
Another application of |\includeonly| is to generate
documents consisting of selected parts of the complete document.

However, there are a few drawbacks of the plain |\include| mechanism:
\begin{itemize}
\item
The child files cannot be compiled on their own,
they can only be compiled via the main file.
A naive editing environment
(such as a text editor with an option
to have the current file processed by \LaTeX)
may require one to switch to the main file before compiling;
attempting to compile the child file produces errors.
\item
The main file must be modified (each time)
to adjust the |\includeonly| command
to the present needs. This easily leaves the main file in a messy state.
\item
The generated document will always carry the filename
of the main document. This is inconvenient if
several child files are to be compiled and
to be kept for distribution.
\end{itemize}

The present package provides a simple interface
to make child files individually compilable by \LaTeX{}.
Compiling a child file then has the same effect as compiling
the main file with an |\includeonly| command
to select the appropriate child.
Moreover the generated document will carry the name of the child
rather than the main file.
This resolves all three above issues.

This feature is meant to make the editing of books,
thesis documents and lecture notes somewhat more convenient.
However, the package can also be used efficiently for
composing a series of documents (such as exercise sheets)
which are typically distributed individually.
It then assists the author in generating the individual documents
(potentially in different versions)
as well as a document containing the collected series.
Another application is in developing style files
or other kinds of included material
where compilation of the style file could redirect
to a sample or test file.

%%%%%%%%%%%%%%%%%%%%%%%%%%%%%%%%%%%%%%%%%%%%%%%%%%%%%%%%%%%%%%%%%%%%%%%%%%%%%%%%
%%%%%%%%%%%%%%%%%%%%%%%%%%%%%%%%%%%%%%%%%%%%%%%%%%%%%%%%%%%%%%%%%%%%%%%%%%%%%%%%
\section{Usage}

First of all, the package \textsf{childdoc} is \emph{not} a standard
\LaTeXe{} |.sty| style file! Therefore it needs to be invoked in
a non-standard way.

%%%%%%%%%%%%%%%%%%%%%%%%%%%%%%%%%%%%%%%%%%%%%%%%%%%%%%%%%%%%%%%%%%%%%%%%%%%%%%%%
\subsection{Included Files}
\label{sec:include}

%%%%%%%%%%%%%%%%%%%%%%%%%%%%%%%%%%%%%%%%
\DescribeMacro{\childdocmain}
To use the package, add the commands
\begin{center}
\begin{tabular}{l}
|\input{childdoc.def}|\\
|\childdocmain{}|\\
\end{tabular}
\end{center}
at the very top of the main \LaTeX{} file,
in particular \emph{before} the |\documentclass| statement!
The argument of |\childdocmain| should be left empty
(but it must be present).

%%%%%%%%%%%%%%%%%%%%%%%%%%%%%%%%%%%%%%%%
\DescribeMacro{\childdocof}
Furthermore, add the commands
\begin{center}
\begin{tabular}{l}
|\input{childdoc.def}|\\
|\childdocof{|\textit{main}|}|\\
\end{tabular}
\end{center}
at the top of every child file \textit{child}
which is included by |\include{|\textit{child}|}|
from within the main file
(or at least for those files to be compiled individually).
The argument \textit{main} must be the filename of the main file.

There are a couple of
considerations in setting up the main and child documents:

%%%%%%%%%%%%%%%%%%%%%%%%%%%%%%%%%%%%%%%%
\paragraph{Restrictions.}

Please note the following restrictions:
\begin{itemize}
\item
|\childdocmain| must be called with one argument \textit{main}
to ensure compatibility with earlier version of the package.
It must either be empty (|\childdocmain{}|)
or precisely match the filename of the main file in which it is specified.
See \secref{sec:detection} for further information.
\item
The filename \textit{main} must be specified without the |.tex| extension.
\item
The filename \textit{main} is case sensitive
(even in case-insensitive file systems)
due to internal string comparison.
\item
The argument \textit{main} should be fully expanded, it cannot be a macro.
\item
Subdirectories and special characters should be avoided in filenames.
\item
The command |\childdocmain{|\textit{main}|}| must be followed by a whitespace.
It should not be followed immediately by another command
or by a comment mark `|%|'.
This is because the \TeX{} parser reads the token immediately following
the argument of |\childdocmain| and puts it
at the beginning of every child section;
however, a white\-space is ignored.
\end{itemize}

%%%%%%%%%%%%%%%%%%%%%%%%%%%%%%%%%%%%%%%%
\paragraph{Content of Main File.}

It is advisable to place all content in the child files included by |\include|.
Any output contained in the main file will appear in all child documents
unless suppressed manually;
it cannot be suppressed automatically by the |\includeonly| directive
and thus should normally be avoided.
A method to include some content in the main file
by means of conditional processing is described in \secref{sec:conditional}.

%%%%%%%%%%%%%%%%%%%%%%%%%%%%%%%%%%%%%%%%
\paragraph{Page Numbering.}

When only a part of the document is compiled,
the appropriate numbering of pages
(as well as other status parameters)
is determined from the |.aux| files.
The latter contain information from previous passes.
However this information needs to propagate through
all intermediate child documents.
Therefore the page numbering in child documents may well
be inconsistent until the complete document is compiled at least once.

A useful (if unconventional) way to always ensure a consistent
page numbering is to restart the numbering in each child document
and denote the pages by `\textit{child}|.|\textit{page}'
where \textit{child} represents the chapter/section number of the child file.
This can be achieved by the command
|\numberwithin{page}{|\textit{child}|}|
of the \textsf{amsmath} package
where \textit{child} can be |chapter| or |section|
depending on the chosen structuring.
Alternatively, one can modify the macro |\thepage| appropriately
and reset the counter |page| at the start of each child file.

%%%%%%%%%%%%%%%%%%%%%%%%%%%%%%%%%%%%%%%%%%%%%%%%%%%%%%%%%%%%%%%%%%%%%%%%%%%%%%%%
\subsection{Conditional Processing}
\label{sec:conditional}

The package provides a mechanism to compile different versions
of a document. To customise the versions further some conditional processing
can come in handy to distinguish which version is being compiled.
The package provides two macros to describe the compilation context:

%%%%%%%%%%%%%%%%%%%%%%%%%%%%%%%%%%%%%%%%
\DescribeMacro{\ifchilddoc}
The conditional |\ifchilddoc| distinguishes between the compilation of
child documents and the main document:
%
\begin{center}
|\ifchilddoc |\textit{child-code}| |[|\||else |\textit{main-code}]| \||fi|
\end{center}

%%%%%%%%%%%%%%%%%%%%%%%%%%%%%%%%%%%%%%%%
\DescribeMacro{\childdocname}
\DescribeMacro{\childdocjob}
The macro |\childdocname| contains the filename (without extension)
of the main or child file being processed.
Note that |\childdocjob| will always contain the name of the main file.

%%%%%%%%%%%%%%%%%%%%%%%%%%%%%%%%%%%%%%%%
\paragraph{Title Page.}

Conditional processing can be used to include a title or banner page
in the main document when proper precautions are taken.
Importantly, the code in the main file should ensure that the page counter
(as well as other status parameters which are stored in the |.aux| files)
takes the same value after the conditional processing.
Otherwise the page numbers may take divergent values
depending on which part is compiled.

For example, a title page could be declared by:
%
\begin{center}
\begin{tabular}{l}
|\ifchilddoc\||else|\\
|\addtocounter{page}{-1}|\\
\textit{code for title page}\\
|\newpage|\\
|\||fi|
\end{tabular}
\end{center}
%
A banner page for the child documents can be generated by:
%
\begin{center}
\begin{tabular}{l}
|\ifchilddoc|\\
|\addtocounter{page}{-1}|\\
\textit{code for banner page}\\
|\newpage|\\
|\||fi|
\end{tabular}
\end{center}
%
Here one could write a message such as:
\begin{center}
|This is the part \childdocname{} of \childdocjob{}.|
\end{center}

%%%%%%%%%%%%%%%%%%%%%%%%%%%%%%%%%%%%%%%%%%%%%%%%%%%%%%%%%%%%%%%%%%%%%%%%%%%%%%%%
\subsection{Flags}
\label{sec:flags}

The package makes it easy to generate different versions
of the main or child documents.
To this end compilation flags can be defined
and assigned different default values.
They will be particularly useful in conjunction
with the forwarding mechanism described in \secref{sec:forward}.

For example, it may be useful to have a flag |\version|
which can be set to |draft| or |final|.
The document source will contain some conditional code
depending on the value of |\version|.
Suppose further, the flag should default to |final| for the main file
and to |draft| for child files
which is a natural assignment for editing the document.
This is achieved by placing the following code
in the preamble of the main document
(below the |\childdocmain| directive):
%
\begin{center}
\begin{tabular}{l}
|\ifchilddoc|\\
|\providecommand{\version}{draft}|\\
|\||else|\\
|\providecommand{\version}{final}|\\
|\||fi|
\end{tabular}
\end{center}
%
The definition by |\providecommand| makes sure
that previous definitions are not overwritten.
Further statements |\providecommand{\version}{...}|
can thus be added before the above code to override it.

For the main file, one might add a line
(between |\childdocmain| and the above block)
%
\begin{center}
|%\ifchilddoc\||else\providecommand{\version}{draft}\||fi|
\end{center}
%
which can be uncommented to produce a draft version.
Likewise one can add a line to the very top of a child file
(above the |\childdocof{|\textit{main}|}| directive)
%
\begin{center}
|%\providecommand{\version}{final}|
\end{center}
%
which can be uncommented to produce the final version of this child document.

%%%%%%%%%%%%%%%%%%%%%%%%%%%%%%%%%%%%%%%%%%%%%%%%%%%%%%%%%%%%%%%%%%%%%%%%%%%%%%%%
\subsection{Forwarding}
\label{sec:forward}

Different versions of the main or child documents
using compilation flags as described in \secref{sec:flags}
can be (permanently) stored in different files
for convenient compilation, viewing and distribution.
To this end, the package defines a command
to pass on compilation to a different file:

%%%%%%%%%%%%%%%%%%%%%%%%%%%%%%%%%%%%%%%%
\DescribeMacro{\childdocforward}
The command |\childdocforward| redirects processing to
another source file:
%
\begin{center}
\begin{tabular}{l}
|\input{childdoc.def}|\\
|\childdocforward[|\textit{main}|]{|\textit{dest}|}|\\
\end{tabular}
\end{center}
%
The argument \textit{dest} is the destination file
(without extension).
It should be the main file or one of the child files.
Note that further \textsf{childdoc} directives
such as |\childdocof| and |\childdocforward|
in the indicated file will be processed in this form.
The optional argument \textit{main}
passes on directly to the main file \textit{main}
while pretending to compile the child \textit{dest}.
This form behaves as if \textit{dest}
issues |\childdocof{|\textit{main}|}| right away,
and no further \textsf{childdoc} directives will be processed.

%%%%%%%%%%%%%%%%%%%%%%%%%%%%%%%%%%%%%%%%
\DescribeMacro{\...prefix}
In the alternative form |\childdocforwardprefix|,
%
\begin{center}
\begin{tabular}{l}
|\input{childdoc.def}|\\
|\childdocforwardprefix[|\textit{main}|]{|\textit{prefix}|}{|\textit{dest}|}|
\end{tabular}
\end{center}
%
the destination file is determined by a pattern
depending on the current file:
To make this work, the current file must be called
`{\textit{prefix}\hspace{0.2em}\textit{suffix}}'
with \textit{prefix} matching precisely the argument.
Processing is then passed on to the file
`{\textit{dest}\hspace{0.2em}\textit{suffix}}'.
Surely, the same effect is achieved by
directly specifying the
argument `{\textit{dest}\hspace{0.2em}\textit{suffix}}'
in the first form.
However, that requires to set up a different file
for each child. With the alternative form of the command
all these files can have exactly the same content
which simplifies setting them up and maintaining them.

For example, the following file |draft.tex|
with a compilation flag |\version| as described in \secref{sec:flags}
compiles the main document as a draft:
%
\begin{center}
\begin{tabular}{l}
|\def\version{draft}|\\
|\input{childdoc.def}|\\
|\childdocforward{|\textit{main}|}|
\end{tabular}
\end{center}
%
Likewise, the following files |final|\textit{nn}|.tex|
compile the final version of the child document
|child|\textit{nn}|.tex|:
%
\begin{center}
\begin{tabular}{l}
|\def\version{final}|\\
|\input{childdoc.def}|\\
|\childdocforwardprefix{final}{child}|
\end{tabular}
\end{center}
%

Note that when several versions of a main file and/or of each child file
are to be generated, it may be convenient to set up a |Makefile| or
shell script to automatise the process.

%%%%%%%%%%%%%%%%%%%%%%%%%%%%%%%%%%%%%%%%%%%%%%%%%%%%%%%%%%%%%%%%%%%%%%%%%%%%%%%%
\subsection{Command Line Processing}
\label{sec:commandline}

The effect of redirection files can also be achieved by invoking
the \LaTeX{} compiler with a more elaborate command line.
Most conveniently this should be done as part
of a shell script or a |Makefile|.

When using \textsf{childdoc} in the main file, the following
command lines effectively perform a redirection
(note that depending on the shell being used,
backslashes may have to be doubled: `|\|' $\to$ `|\\|'):
%
\begin{center}
|... -jobname "|\textit{target}|" |\\|"|[\textit{flags}]%
|\input{childdoc.def}\childdocforward[|\textit{main}|]{|\textit{dest}|}"|
\end{center}
%
Here \textit{target} is the name of the output file,
\textit{main} is the name of the main file
and \textit{dest} is the name of the main or child file to be processed
(all filenames without extensions).
The optional argument \textit{main} can be omitted
if \textit{main} matches \textit{dest}.
Optionally, compilation \textit{flags} can be defined via |\def| commands.
This command line makes the \TeX{} engine believe
it is compiling the file \textit{target}
whose content is specified as the latter parameter.
The provided code then forwards the processing to
\textit{main} or \textit{dest} as described in \secref{sec:forward}.

%%%%%%%%%%%%%%%%%%%%%%%%%%%%%%%%%%%%%%%%%%%%%%%%%%%%%%%%%%%%%%%%%%%%%%%%%%%%%%%%
\subsection{Include by Input}
\label{sec:input}

Including child documents by |\include| has some restrictions by design.
Most notably, the content of a child document always occupies
its own set of pages; pages cannot be shared between child documents.
Usually, this behaviour makes perfect sense
because each child document contain an essential part of the document.
However, in some situations it may be desirable to compose
a document from a collection of parts
without having mandatory page breaks between then.
For this case, the package
provides a mechanism to include parts
by |\input| which can also be processed individually.
However, by construction this mechanism
requires manual handling of the content to be output.

%%%%%%%%%%%%%%%%%%%%%%%%%%%%%%%%%%%%%%%%
\DescribeMacro{\ifchilddocmanual}
The main file should be prepared as usual, see \secref{sec:include}.
However, the document body must make a distinction
between processing of an individual part and of the main document, e.g.:
%
\begin{center}
\begin{tabular}{l}
|\ifchilddocmanual|\\
|\input{\childdocname}|\\
|\||else|\\
\textit{document body with }|\input{|\textit{part}|}|\\
|\||fi|
\end{tabular}
\end{center}
%
The conditional |\ifchilddocmanual| is true whenever
a part to be included by |\input| is being compiled,
and the name of the part is stored in |\childdocname|.

%%%%%%%%%%%%%%%%%%%%%%%%%%%%%%%%%%%%%%%%
\DescribeMacro{\childdocby}
Each part to be included by |\input| should start with:
%
\begin{center}
\begin{tabular}{l}
|\input{childdoc.def}|\\
|\childdocby{|\textit{main}|}|\\
\end{tabular}
\end{center}
%
The directive |\childdocby| is similar to |\childdocof|
described in \secref{sec:include},
but the subsequent selection of content must be done manually.
To that end, both |\ifchilddoc| and |\ifchilddocmanual|
will be true upon processing of a part,
and the name of the part is stored in |\childdocname|.
Note that |\jobname| will be set to the filename of the current part
so that each part receives an individual |.aux| file
that does not interfere with the |.aux| file(s) of the main document.
This behaviour can be altered by the alternative form
|\childdocby[*]{|\textit{main}|}| (with a non-empty optional argument)
which uses the |.aux| file of the main document
by setting |\jobname| to \textit{main}.

%%%%%%%%%%%%%%%%%%%%%%%%%%%%%%%%%%%%%%%%%%%%%%%%%%%%%%%%%%%%%%%%%%%%%%%%%%%%%%%%
\subsection{Driver Development}
\label{sec:driver}

The \textsf{childdoc} mechanism can also be use for the development
of definition files such as \LaTeX{} styles or classes.
This case differs from the above setup with multiple parts
included by |\include| in that no |\includeonly| should be invoked.
This can be achieved by starting the include file
(before |\ProvidesPackage|) with:
%
\begin{center}
\begin{tabular}{l}
|\input{childdoc.def}|\\
|\childdocforward{|\textit{main}|}|\\
\end{tabular}
\end{center}
%
or alternatively with:
%
\begin{center}
\begin{tabular}{l}
|\input{childdoc.def}|\\
|\childdocby{|\textit{main}|}|\\
\end{tabular}
\end{center}
%
Both forms have slightly different effects as described above.
The main file is prepared as usual, see \secref{sec:include}.

%%%%%%%%%%%%%%%%%%%%%%%%%%%%%%%%%%%%%%%%%%%%%%%%%%%%%%%%%%%%%%%%%%%%%%%%%%%%%%%%
\subsection{Legacy Detection}
\label{sec:detection}

The directive |\childdocmain| in the main file can detect
whether the complete document or merely a child is to be compiled
even without using the directive |\childdocof|.
This method is deprecated because it is less robust
and there is no compelling reason to use it;
it is merely provided for backward compatibility
and it may be removed in future versions.

If the detection mechanism is to be used,
it is mandatory to correctly specify
the filename of the main file as the argument of |\childdocmain|:
%
\begin{center}
\begin{tabular}{l}
|\input{childdoc.def}|\\
|\childdocmain{|\textit{main}|}|\\
\end{tabular}
\end{center}
%
If |\jobname| does not match the argument \textit{main} of |\childdocmain|,
it is assumed that |\jobname| points to the child file to be compiled.
When using |\childdocmain| with the main file specified as argument,
it suffices to start a child file
with just |\input{|\textit{main}|}|
without loading of the package and using |\childdocof|.
If instead all processing is done
with the appropriate \textsf{childdoc} directives,
the argument of \textit{main} of |\childdocmain| can be empty.

An alternative version of the command line processing described
in \secref{sec:commandline} using the detection mechanism reads:
%
\begin{center}
|... -jobname "|\textit{target}|" "|[\textit{flags}]%
[|\def\jobname{|\textit{dest}|}|]|\input{|\textit{main}|}"|
\end{center}

%%%%%%%%%%%%%%%%%%%%%%%%%%%%%%%%%%%%%%%%%%%%%%%%%%%%%%%%%%%%%%%%%%%%%%%%%%%%%%%%
\subsection{Manual Code}
\label{sec:manual}

In case one cannot be certain whether the definitions file |childdoc.def|
is installed on the target \TeX{} distribution
and one prefers not to ship it,
it is conceivable to paste a few relevant commands into the sources.

To that end, drop all statements |\input{childdoc.def}|
and perform the replacements as outlined below.
Instead of |\childdocmain{|\textit{main}|}| add the following code
to the top of the main file:
%
\begin{center}
\begin{tabular}{l}
|\||ifdefined\childdocname\endinput\||fi\newif\ifchilddoc|\\
|\edef\childdocname{\scantokens\expandafter{\jobname\noexpand}}|\\
|\def\childdocmain{|\textit{main}|}\||ifx\childdocmain\childdocname\||else|\\
|\childdoctrue\includeonly{\childdocname}\let\jobname\childdocmain\||fi|\\
\end{tabular}
\end{center}
%
Instead of |\childdocof{|\textit{main}|}| just include the main file
at the top of each child file:
%
\begin{center}
|\input{|\textit{main}|}|
\end{center}
%
A simple redirection |\childdocforward{|\textit{dest}|}| is achieved by:
%
\begin{center}
|\def\jobname{|\textit{dest}|}\input{\jobname}|
\end{center}
%
The redirection with prefix
|\childdocforwardprefix[|\textit{prefix}|]{|\textit{dest}|}|
is accomplished by:
%
\begin{center}
\begin{tabular}{l}
|{\edef\jobname{\scantokens\expandafter{\jobname\noexpand}}|\\
|\def\redirectjob |\textit{prefix}|#1~~~{\gdef\jobname{|\textit{dest}|#1}}|\\
|\expandafter\redirectjob\jobname~~~}\input{\jobname}|
\end{tabular}
\end{center}

In an alternative approach,
child documents can be compiled by a specific command line
without additional code or specific definitions:
%
\begin{center}
|... -jobname "|\textit{target}|" "|[\textit{flags}]%
|\includeonly{|\textit{dest}|}\input{|\textit{main}|}"|
\end{center}
%

%%%%%%%%%%%%%%%%%%%%%%%%%%%%%%%%%%%%%%%%%%%%%%%%%%%%%%%%%%%%%%%%%%%%%%%%%%%%%%%%
%%%%%%%%%%%%%%%%%%%%%%%%%%%%%%%%%%%%%%%%%%%%%%%%%%%%%%%%%%%%%%%%%%%%%%%%%%%%%%%%
\section{Information}

%%%%%%%%%%%%%%%%%%%%%%%%%%%%%%%%%%%%%%%%%%%%%%%%%%%%%%%%%%%%%%%%%%%%%%%%%%%%%%%%
\subsection{Copyright}

Copyright \copyright{} 2017--2018 Niklas Beisert

This work may be distributed and/or modified under the
conditions of the \LaTeX{} Project Public License, either version 1.3
of this license or (at your option) any later version.
The latest version of this license is in
  \url{http://www.latex-project.org/lppl.txt}
and version 1.3 or later is part of all distributions of \LaTeX{}
version 2005/12/01 or later.

This work has the LPPL maintenance status `maintained'.

The Current Maintainer of this work is Niklas Beisert.

This work consists of the files |README.txt|, |childdoc.ins| and |childdoc.dtx|
as well as the derived files |childdoc.def|, |cdocsamp.tex|
with |cdocsch1.tex|, |cdocsch2.tex|, |cdocspt3.tex|, |cdocspt4.tex|,
|cdocsdrf.tex|, |cdocsfn1.tex|, |cdocsfn2.tex|
as well as |childdoc.pdf|.

%%%%%%%%%%%%%%%%%%%%%%%%%%%%%%%%%%%%%%%%%%%%%%%%%%%%%%%%%%%%%%%%%%%%%%%%%%%%%%%%
\subsection{Files and Installation}

The package consists of the files:
%
\begin{center}
\begin{tabular}{ll}
    |README.txt|   & readme file \\
    |childdoc.ins| & installation file \\
    |childdoc.dtx| & source file \\
    |childdoc.def| & definition file \\
    |cdocsamp.tex| & sample main file \\
    |cdocsch1.tex| & sample include file \\
    |cdocsch2.tex| & sample include file \\
    |cdocspt3.tex| & sample part file \\
    |cdocspt4.tex| & sample part file \\
    |cdocsdrf.tex| & sample redirection file \\
    |cdocsfn1.tex| & sample redirection file \\
    |cdocsfn2.tex| & sample redirection file \\
    |childdoc.pdf| & manual
\end{tabular}
\end{center}
%
The distribution consists of the files
|README.txt|, |childdoc.ins| and |childdoc.dtx|.
%
\begin{itemize}
\item
Run (pdf)\LaTeX{} on |childdoc.dtx|
to compile the manual |childdoc.pdf| (this file).
\item
Run \LaTeX{} on |childdoc.ins| to create the definitions file |childdoc.def|
and the sample |cdocsamp.tex| with include files
|cdocsch1.tex|, |cdocsch2.tex|, |cdocspt3.tex|, |cdocspt4.tex|,
|cdocsdrf.tex|, |cdocsfn1.tex|, |cdocsfn2.tex|.
Then copy the file |childdoc.def| to an appropriate directory of your \LaTeX{}
distribution, e.g.\ \textit{texmf-root}|/tex/latex/childdoc|.
\end{itemize}

%%%%%%%%%%%%%%%%%%%%%%%%%%%%%%%%%%%%%%%%%%%%%%%%%%%%%%%%%%%%%%%%%%%%%%%%%%%%%%%%
\subsection{Related CTAN Packages}

There are several other packages which offer a similar functionality:
%
\begin{itemize}
\item
The packages
\href{http://ctan.org/pkg/docmute}{\textsf{docmute}},
\href{http://ctan.org/pkg/includex}{\textsf{includex}} and
\href{http://ctan.org/pkg/standalone}{\textsf{standalone}}
provide commands to include only the document body of
a child file thus allowing both files to be compiled individually.
\item
The packages \href{http://ctan.org/pkg/subdocs}{\textsf{subdocs}}
and \href{http://ctan.org/pkg/subfiles}{\textsf{subfiles}}
provide structures in which the main and child documents can be
encapsulated and allowing them to be compiled individually.
The inclusion mechanism is different from the conventional |\include|.
\item
The package \href{http://ctan.org/pkg/combine}{\textsf{combine}}
is an elaborate solution to combine several documents into one.
\end{itemize}
%
See also the CTAN topic \href{http://ctan.org/topic/subdocs}{\textsf{subdocs}}
for further related packages.
The present package differs from the above solutions in that
a document structure constructed with the conventional |\include| mechanism
just needs two extra commands at the top of every file
such that all constituent files can be compiled individually.

%%%%%%%%%%%%%%%%%%%%%%%%%%%%%%%%%%%%%%%%%%%%%%%%%%%%%%%%%%%%%%%%%%%%%%%%%%%%%%%%
%\subsection{Feature Suggestions}
%
%The following is a list of features which may be useful for future
%versions of this package:
%%
%\begin{itemize}
%\item
%\ldots
%\end{itemize}

%%%%%%%%%%%%%%%%%%%%%%%%%%%%%%%%%%%%%%%%%%%%%%%%%%%%%%%%%%%%%%%%%%%%%%%%%%%%%%%%
\subsection{Revision History}

%%%%%%%%%%%%%%%%%%%%%%%%%%%%%%%%%%%%%%%%
\paragraph{v2.0:} 2018/12/30

\begin{itemize}
\item
immediate forward processing
\item
added |\childdocby| mechanism
\item
manual restructured
\end{itemize}

%%%%%%%%%%%%%%%%%%%%%%%%%%%%%%%%%%%%%%%%
\paragraph{v1.6:} 2018/01/17

\begin{itemize}
\item
application for development of include files
\item
corrections to manual
\end{itemize}

%%%%%%%%%%%%%%%%%%%%%%%%%%%%%%%%%%%%%%%%
\paragraph{v1.5:} 2017/05/21

\begin{itemize}
\item
more complete structuring introduced
\item
|\childdocof| introduced
\item
|\childdoc| renamed to |\childdocmain|
\item
|\childredirect| renamed to |\childdocforward| and |\childdocforwardprefix|
and functionality expanded
\end{itemize}

%%%%%%%%%%%%%%%%%%%%%%%%%%%%%%%%%%%%%%%%
\paragraph{v1.0:} 2017/04/27

\begin{itemize}
\item
manual and install package
\item
first version published on CTAN
\end{itemize}

%%%%%%%%%%%%%%%%%%%%%%%%%%%%%%%%%%%%%%%%
\paragraph{v0.6:} 2017/04/26

\begin{itemize}
\item
redirection mechanism added
\end{itemize}

%%%%%%%%%%%%%%%%%%%%%%%%%%%%%%%%%%%%%%%%
\paragraph{v0.5:} 2017/04/26

\begin{itemize}
\item
functionality in definition file
\end{itemize}


%%%%%%%%%%%%%%%%%%%%%%%%%%%%%%%%%%%%%%%%%%%%%%%%%%%%%%%%%%%%%%%%%%%%%%%%%%%%%%%%
%%%%%%%%%%%%%%%%%%%%%%%%%%%%%%%%%%%%%%%%%%%%%%%%%%%%%%%%%%%%%%%%%%%%%%%%%%%%%%%%
%%%%%%%%%%%%%%%%%%%%%%%%%%%%%%%%%%%%%%%%%%%%%%%%%%%%%%%%%%%%%%%%%%%%%%%%%%%%%%%%
\appendix

\settowidth\MacroIndent{\rmfamily\scriptsize 000\ }

 \DocInput{childdoc.dtx}

\end{document}
%</driver>
% \fi
%
% %%%%%%%%%%%%%%%%%%%%%%%%%%%%%%%%%%%%%%%%%%%%%%%%%%%%%%%%%%%%%%%%%%%%%%%%%%%%%%
% %%%%%%%%%%%%%%%%%%%%%%%%%%%%%%%%%%%%%%%%%%%%%%%%%%%%%%%%%%%%%%%%%%%%%%%%%%%%%%
% \section{Sample}
%\iffalse
%<*samplemain>
%\fi
%
% The following presents a sample document
% with two chapters, two parts, a title page,
% a compile flag as well as three forwarding files to set the flag.
% It consists of eight |.tex| files:
% \begin{center}
% \begin{tabular}{ll}
% |cdocsamp.tex|&main file\\
% |cdocsch1.tex|&include file for chapter 1\\
% |cdocsch2.tex|&include file for chapter 2\\
% |cdocspt3.tex|&include file for part 3\\
% |cdocspt4.tex|&include file for part 4\\
% |cdocsdrf.tex|&forwarding file for main file in draft mode\\
% |cdocsfi1.tex|&forwarding file for final version of chapter 1\\
% |cdocsfi2.tex|&forwarding file for final version of chapter 2\\
% \end{tabular}
% \end{center}
% Each of the eight files can be compiled directly by the \LaTeX{} compiler.
%
% %%%%%%%%%%%%%%%%%%%%%%%%%%%%%%%%%%%%%%
% \paragraph{Main File.}
%
% The main file is called |cdocsamp.tex|.
%
% Load the \textsf{childdoc} definitions and
% declare the filename for the main document:
%    \begin{macrocode}
\input{childdoc.def}
\childdocmain{}
%    \end{macrocode}

% Optional override for |\version| flag:
%    \begin{macrocode}
%%\ifchilddoc\else\providecommand{\version}{draft}\fi
%    \end{macrocode}

% Define the default values for the |\version| flag
% (|final| for the main file and |draft| for childs):
%    \begin{macrocode}
\ifchilddoc
\providecommand{\version}{draft}
\else
\providecommand{\version}{final}
\fi
%    \end{macrocode}

% Load the standard document class:
%    \begin{macrocode}
\documentclass[12pt]{article}
%    \end{macrocode}

% Start the document body:
%    \begin{macrocode}
\begin{document}
%    \end{macrocode}

% Declare a title page.
% Print title, part of document being processed and version flag:
%    \begin{macrocode}
\addtocounter{page}{-1}
\begin{center}
{\LARGE\bfseries{}childdoc example\par}
\vspace{1cm}
\ifchilddoc
\ifchilddocmanual part\else chapter\fi:
`\childdocname' of `\childdocjob'\par
\else
main document: `\childdocjob'\par
\fi
version: \version\par
\end{center}
\newpage
%    \end{macrocode}

% Manually include selected file,
% otherwise process as usual:
%    \begin{macrocode}
\ifchilddocmanual
\section*{part `\childdocname'}
\input{\childdocname}
\else
%    \end{macrocode}

% Include the two chapters:
%    \begin{macrocode}
\include{cdocsch1}
\include{cdocsch2}
%    \end{macrocode}

% Include the two parts unless only chapters should be displayed:
%    \begin{macrocode}
\ifchilddoc\else
\section{part three}
\input{cdocspt3}
\section{part four}
\input{cdocspt4}
\fi
%    \end{macrocode}

% Process as usual until here:
%    \begin{macrocode}
\fi
%    \end{macrocode}

% End of document body:
%    \begin{macrocode}
\end{document}
%    \end{macrocode}
%\iffalse
%</samplemain>
%\fi
%
% %%%%%%%%%%%%%%%%%%%%%%%%%%%%%%%%%%%%%%
% \paragraph{Chapter Include Files.}
%
% The include files are called |cdocsch1.tex| and |cdocsch2.tex|.
%
%\iffalse
%<*samplechap1|samplechap2>
%\fi

% Optional override for |\version| flag:
%    \begin{macrocode}
%%\providecommand{\version}{final}
%    \end{macrocode}

% Include the main document:
%    \begin{macrocode}
\input{childdoc.def}
\childdocof{cdocsamp}
%    \end{macrocode}

%\iffalse
%</samplechap1|samplechap2>
%\fi
%
%\iffalse
%<*samplechap1>
%\fi
% Some text for chapter 1:
%    \begin{macrocode}
\section{one}
some text in chapter one
%    \end{macrocode}

%\iffalse
%</samplechap1>
%\fi
% Some text for chapter 2:
%\iffalse
%<*samplechap2>
%\fi
%    \begin{macrocode}
\section{two}
more text in chapter two
%    \end{macrocode}

%\iffalse
%</samplechap2>
%\fi
%
% %%%%%%%%%%%%%%%%%%%%%%%%%%%%%%%%%%%%%%
% \paragraph{Part Include Files.}
%
% The include files are called |cdocspt3.tex| and |cdocspt4.tex|.
%
%\iffalse
%<*samplepart3|samplepart4>
%\fi

% Optional override for |\version| flag:
%    \begin{macrocode}
%%\providecommand{\version}{final}
%    \end{macrocode}

% Include the main document:
%    \begin{macrocode}
\input{childdoc.def}
\childdocby{cdocsamp}
%    \end{macrocode}

%\iffalse
%</samplepart3|samplepart4>
%\fi
%
%\iffalse
%<*samplepart3>
%\fi
% Some text for part 3:
%    \begin{macrocode}
some text in part three
%    \end{macrocode}

%\iffalse
%</samplepart3>
%\fi
% Some text for part 4:
%\iffalse
%<*samplepart4>
%\fi
%    \begin{macrocode}
more text in part four
%    \end{macrocode}

%\iffalse
%</samplepart4>
%\fi
%
% %%%%%%%%%%%%%%%%%%%%%%%%%%%%%%%%%%%%%%
% \paragraph{Forwarding for a Complete Draft.}
%
% The following forwarding file |cdocsdrf.tex|
% compiles the main document in draft mode:
%\iffalse
%<*sampledraft>
%\fi
%    \begin{macrocode}
\def\version{draft}
\input{childdoc.def}
\childdocforward{cdocsamp}
%    \end{macrocode}

%\iffalse
%</sampledraft>
%\fi
%
% %%%%%%%%%%%%%%%%%%%%%%%%%%%%%%%%%%%%%%
% \paragraph{Forwarding for Final Version of the Chapters.}
%
% The following forwarding files |cdocsfn1.tex| and |cdocsfn2.tex|
% (with identical content)
% compile the final versions of the child documents
% |cdocsch1.tex| and |cdocsch2.tex|, respectively:
%\iffalse
%<*samplefinal>
%\fi
%    \begin{macrocode}
\def\version{final}
\input{childdoc.def}
\childdocforwardprefix[cdocsamp]{cdocsfn}{cdocsch}
%    \end{macrocode}

%\iffalse
%</samplefinal>
%\fi
%
% %%%%%%%%%%%%%%%%%%%%%%%%%%%%%%%%%%%%%%
% \paragraph{Command Line Processing.}
%
% The following three command lines generate the output files
% |cdocscld|, |cdocscl1| and |cdocscl2|
% which should be identical to
% |cdocsdrf|, |cdocsch1| and |cdocsfn2|, respectively:
% \begin{center}
% \begin{tabular}{l}
% |latex -jobname cdocscld \|\\
% |  "\def\version{draft}\input{childdoc.def}\childdocforward{cdocsamp}"|\\
% |latex -jobname cdocscl1 \|\\
% |  "\input{childdoc.def}\childdocforward[cdocsamp]{cdocsch1}"|\\
% |latex -jobname cdocscl2 \|\\
% |  "\def\version{final}\input{childdoc.def}\childdocforward{cdocsch2}"|
% \end{tabular}
% \end{center}
% Note that the trailing backslash on each first line
% merely continues the input to the second line
% (for convenient cut ant paste).
% Furthermore, the command |latex| can be replaced by any
% of its alternative versions such as |pdflatex|.
%
% %%%%%%%%%%%%%%%%%%%%%%%%%%%%%%%%%%%%%%%%%%%%%%%%%%%%%%%%%%%%%%%%%%%%%%%%%%%%%%
% %%%%%%%%%%%%%%%%%%%%%%%%%%%%%%%%%%%%%%%%%%%%%%%%%%%%%%%%%%%%%%%%%%%%%%%%%%%%%%
% \section{Implementation}
%\iffalse
%<*package>
%\fi
%
% This section describes the definitions file |childdoc.def|.

% The definitions cannot be loaded using |\usepackage| or |\RequirePackage|
% which has a mechanism to prevent loading a style file more than once.
% When loading the definitions by means of |\input|
% multiple instances have to be prevented manually:
%\iffalse
%This code needs to be before the `\ProvidesFile' directive
%which is defined at the beginning of this file.
%Therefore it is also placed there and commented out here.
%</package>
%<*discard>
%\fi
%    \begin{macrocode}
\ifdefined\childdocmain\endinput\fi
%    \end{macrocode}
%\iffalse
%</discard>
%<*package>
%\fi
%
% \macro{\ifchilddoc}
% \macro{\ifchilddocmanual}
% The conditional |\ifchilddoc| tells whether a
% child (true) or main (false) document is being compiled.
% The conditional |\ifchilddocmanual| tells whether
% the |\includeonly| mechanism is used (false) or
% the selection of child files must be performed manually (true).
% The definitions initialise to false:
%    \begin{macrocode}
\newif\ifchilddoc
\newif\ifchilddocmanual
%    \end{macrocode}

% \macro{\childdocname}
% \macro{\childdocjob}
% The macro |\childdocname| stores the name of the main document
% to be compiled. The macro |\childdocjob| stores the name of
% the document on which the \LaTeX{} compiler was originally invoked.
% The content of |\jobname| cannot be compared
% to filenames specified in the source due to different catcodes.
% The following code rescans |\jobname|, stores the result
% in |\childdocname| and saves a copy in |\childdocjob|:
%    \begin{macrocode}
\edef\childdocname{\scantokens\expandafter{\jobname\noexpand}}
\let\childdocjob\childdocname
%    \end{macrocode}

% \macro{\childdocdisable}
% The macro |\childdocdisable| prevents the main file
% from being processed more than once.
% At this stage, the main document command |\childdocmain|
% is assumed to be called once again where it should do nothing.
% Any subsequent call to it should prevent
% a secondary processing of the main document
% It overwrites the forwarding commands
% |\childdocof| and |\childdocforward|
% with empty macros to prevent further inclusions of the main document:
%    \begin{macrocode}
\newcommand{\childdocdisable}
{
  \renewcommand{\childdocmain}[1]{\renewcommand{\childdocmain}[1]{\endinput}}
  \renewcommand{\childdocof}[1]{}
  \renewcommand{\childdocby}[2][]{}
  \renewcommand{\childdocforward}[2][]{}
  \renewcommand{\childdocdisable}{}
}
%    \end{macrocode}

% \macro{\childdocmain}
% The macro |\childdocmain| is to be called at the top of the main file
% with nothing or the main filename (without extension) as argument.
% First, it breaks loops.
% If the argument is not empty and does not match |\childdocname|
% (which is set by the first inclusion of |childdoc.def|),
% |\ifchilddoc| is set to true, |\includeonly| is applied to the child file
% and |\jobname| is set to the main file
% (for proper handling of |.aux| files):
%    \begin{macrocode}
\newcommand{\childdocmain}[1]
{
  \childdocdisable\childdocmain{}
  \if?#1?\else
    \begingroup
      \def\childdoctmp{#1}
      \ifx\childdoctmp\childdocname
        \def\childdoctmp{}
      \else
        \def\childdoctmp
        {
          \childdoctrue
          \includeonly{\childdocname}
          \def\childdocjob{#1}
          \def\jobname{#1}
        }
      \fi
      \expandafter
    \endgroup
    \childdoctmp
  \fi
}
%    \end{macrocode}

% \macro{\childdocof}
% The command |\childdocof| redirects
% compilation to the main file |#1|.
%    \begin{macrocode}
\newcommand{\childdocof}[1]
{
  \childdocdisable
  \childdoctrue
  \includeonly{\childdocname}
  \def\jobname{#1}
  \def\childdocjob{#1}
  \input{#1}
}
%    \end{macrocode}

% \macro{\childdocby}
% The command |\childdocby| ....
%    \begin{macrocode}
\newcommand{\childdocby}[2][]
{
  \childdocdisable
  \childdoctrue
  \childdocmanualtrue
  \if?#1?\else
    \def\jobname{#2}
  \fi
  \def\childdocjob{#2}
  \input{#2}
  \endinput
}
%    \end{macrocode}

% \macro{\childdocforward}
% The command |\childdocforward| redirects
% compilation to the main file or
% (if the optional argument is given) a child file.
% Parameters are set as if the main file
% or a child file starting with |\childdocof| was compiled.
% Then compilation is handed over to the main file:
%    \begin{macrocode}
\newcommand{\childdocforward}[2][]
{
  \begingroup
    \if?#1?
      \def\childdoctmp
      {
        \def\childdocname{#2}
        \def\childdocjob{#2}
        \def\jobname{#2}
        \input{#2}
        \endinput
      }
    \else
      \def\childdoctmp
      {
        \childdocdisable
        \def\childdocname{#2}
        \childdoctrue
        \includeonly{#2}
        \def\childdocjob{#1}
        \def\jobname{#1}
        \input{#1}
        \endinput
      }
    \fi
    \expandafter
  \endgroup
  \childdoctmp
}
%    \end{macrocode}

% \macro{\childdocforwardprefix}
% The command |\childdocforwardprefix| redirects
% compilation to the main or a child file by means of a pattern.
% The prefix |#1| in the current filename is replaced by |#2|
% and the suffix of the current filename is kept
% (it is assumed that the filename does not contain the substring `|~~~|'
% which is used as a delimiter).
% Compilation is handed over to the new file by |\childdocforward|:
%    \begin{macrocode}
\newcommand{\childdocforwardprefix}[3][]
{
  \begingroup
    \def\childdocextract #2##1~~~{\def\childdoctmp{\childdocforward[#1]{#3##1}}}
    \expandafter\childdocextract\childdocname~~~
    \expandafter
  \endgroup
  \childdoctmp
}
%    \end{macrocode}

% \macro{\childdoc}
% The deprecated macro |\childdoc| is a legacy version of |\childdocmain|:
%    \begin{macrocode}
\newcommand{\childdoc}{\childdocmain}
%    \end{macrocode}

% \macro{\childdocredirect}
% The deprecated macro |\childdocredirect| is a legacy version
% of |\childdocforward| and |\childdocforwardprefix|:
%    \begin{macrocode}
\newcommand{\childdocredirect}[2][]
{
  \begingroup
    \if?#1?
      \def\childdoctmp{\childdocforward{#2}}
    \else
      \def\childdoctmp{\childdocforwardprefix{#1}{#2}}
    \fi
    \expandafter
  \endgroup
  \childdoctmp
}
%    \end{macrocode}

%\iffalse
%</package>
%\fi
%
\endinput

\childdocforwardprefix[cdocsamp]{cdocsfn}{cdocsch}
%    \end{macrocode}

%\iffalse
%</samplefinal>
%\fi
%
% %%%%%%%%%%%%%%%%%%%%%%%%%%%%%%%%%%%%%%
% \paragraph{Command Line Processing.}
%
% The following three command lines generate the output files
% |cdocscld|, |cdocscl1| and |cdocscl2|
% which should be identical to
% |cdocsdrf|, |cdocsch1| and |cdocsfn2|, respectively:
% \begin{center}
% \begin{tabular}{l}
% |latex -jobname cdocscld \|\\
% |  "\def\version{draft}% \iffalse
%
% childdoc.dtx Copyright (C) 2017-2018 Niklas Beisert
%
% This work may be distributed and/or modified under the
% conditions of the LaTeX Project Public License, either version 1.3
% of this license or (at your option) any later version.
% The latest version of this license is in
%   http://www.latex-project.org/lppl.txt
% and version 1.3 or later is part of all distributions of LaTeX
% version 2005/12/01 or later.
%
% This work has the LPPL maintenance status `maintained'.
%
% The Current Maintainer of this work is Niklas Beisert.
%
% This work consists of the files childdoc.dtx and childdoc.ins
% and the derived files childdoc.def and cdocsamp.tex with
% cdocsch1.tex, cdocsch2.tex, cdocsdrf.tex, cdocsfn1.tex, cdocsfn2.tex.
%
%<package>\ifdefined\childdocmain\endinput\fi
%<package>\ProvidesFile{childdoc.def}[2018/12/30 v2.0 child document driver]
%<samplemain>\ProvidesFile{cdocsamp.tex}[2018/12/30 v2.0 sample for childdoc]
%<*driver>
%\ProvidesFile{childdoc.drv}[2018/12/30 v2.0 childdoc reference manual file]
\PassOptionsToClass{10pt,a4paper}{article}
\documentclass{ltxdoc}

\usepackage[margin=35mm]{geometry}
\usepackage{hyperref}
\usepackage{hyperxmp}
\usepackage[usenames]{color}

\hypersetup{colorlinks=true}
\hypersetup{pdfstartview=FitH}
\hypersetup{pdfpagemode=UseNone}
\hypersetup{pdfsource={}}
\hypersetup{pdflang={en-UK}}
\hypersetup{pdfcopyright={Copyright 2017-2018 Niklas Beisert.
  This work may be distributed and/or modified under the
  conditions of the LaTeX Project Public License, either version 1.3
  of this license or (at your option) any later version.}}
\hypersetup{pdflicenseurl={http://www.latex-project.org/lppl.txt}}
\hypersetup{pdfcontactaddress={ETH Zurich, ITP, HIT K,
  Wolfgang-Pauli-Strasse 27}}
\hypersetup{pdfcontactpostcode={8093}}
\hypersetup{pdfcontactcity={Zurich}}
\hypersetup{pdfcontactcountry={Switzerland}}
\hypersetup{pdfcontactemail={nbeisert@itp.phys.ethz.ch}}
\hypersetup{pdfcontacturl={http://people.phys.ethz.ch/\xmptilde nbeisert/}}

\newcommand{\secref}[1]{\hyperref[#1]{section \ref*{#1}}}

\parskip1ex
\parindent0pt
\let\olditemize\itemize
\def\itemize{\olditemize\parskip0pt}

\begin{document}

\title{The \textsf{childdoc} Package}
\hypersetup{pdftitle={The childdoc Package}}
\author{Niklas Beisert\\[2ex]
  Institut f\"ur Theoretische Physik\\
  Eidgen\"ossische Technische Hochschule Z\"urich\\
  Wolfgang-Pauli-Strasse 27, 8093 Z\"urich, Switzerland\\[1ex]
  \href{mailto:nbeisert@itp.phys.ethz.ch}
  {\texttt{nbeisert@itp.phys.ethz.ch}}}
\hypersetup{pdfauthor={Niklas Beisert}}
\hypersetup{pdfsubject={Manual for the LaTeX2e Package childdoc}}
\date{30 December 2018, \textsf{v2.0}}
\maketitle

\begin{abstract}\noindent
\textsf{childdoc} is a \LaTeXe{} package
that enables the direct compilation
of document sections included by |\include|
to individual files.
\end{abstract}

\begingroup
\parskip0ex
\tableofcontents
\endgroup

%%%%%%%%%%%%%%%%%%%%%%%%%%%%%%%%%%%%%%%%%%%%%%%%%%%%%%%%%%%%%%%%%%%%%%%%%%%%%%%%
%%%%%%%%%%%%%%%%%%%%%%%%%%%%%%%%%%%%%%%%%%%%%%%%%%%%%%%%%%%%%%%%%%%%%%%%%%%%%%%%
\section{Introduction}

\LaTeX{} provides a mechanism to structure a large document (such as a book)
into a main file and several child files (containing the chapters)
using the |\include| command.
This mechanism is beneficial for documents
which span hundreds of pages in order to
make the source file(s) more manageable.
Moreover, compilation can be restricted to
selected child files by means of the |\includeonly| command.
The latter feature can be used to reduce the compilation time while editing
(this was significantly more useful in the earlier days of \LaTeX{})
or to generate a smaller document which is easier to navigate.
Another application of |\includeonly| is to generate
documents consisting of selected parts of the complete document.

However, there are a few drawbacks of the plain |\include| mechanism:
\begin{itemize}
\item
The child files cannot be compiled on their own,
they can only be compiled via the main file.
A naive editing environment
(such as a text editor with an option
to have the current file processed by \LaTeX)
may require one to switch to the main file before compiling;
attempting to compile the child file produces errors.
\item
The main file must be modified (each time)
to adjust the |\includeonly| command
to the present needs. This easily leaves the main file in a messy state.
\item
The generated document will always carry the filename
of the main document. This is inconvenient if
several child files are to be compiled and
to be kept for distribution.
\end{itemize}

The present package provides a simple interface
to make child files individually compilable by \LaTeX{}.
Compiling a child file then has the same effect as compiling
the main file with an |\includeonly| command
to select the appropriate child.
Moreover the generated document will carry the name of the child
rather than the main file.
This resolves all three above issues.

This feature is meant to make the editing of books,
thesis documents and lecture notes somewhat more convenient.
However, the package can also be used efficiently for
composing a series of documents (such as exercise sheets)
which are typically distributed individually.
It then assists the author in generating the individual documents
(potentially in different versions)
as well as a document containing the collected series.
Another application is in developing style files
or other kinds of included material
where compilation of the style file could redirect
to a sample or test file.

%%%%%%%%%%%%%%%%%%%%%%%%%%%%%%%%%%%%%%%%%%%%%%%%%%%%%%%%%%%%%%%%%%%%%%%%%%%%%%%%
%%%%%%%%%%%%%%%%%%%%%%%%%%%%%%%%%%%%%%%%%%%%%%%%%%%%%%%%%%%%%%%%%%%%%%%%%%%%%%%%
\section{Usage}

First of all, the package \textsf{childdoc} is \emph{not} a standard
\LaTeXe{} |.sty| style file! Therefore it needs to be invoked in
a non-standard way.

%%%%%%%%%%%%%%%%%%%%%%%%%%%%%%%%%%%%%%%%%%%%%%%%%%%%%%%%%%%%%%%%%%%%%%%%%%%%%%%%
\subsection{Included Files}
\label{sec:include}

%%%%%%%%%%%%%%%%%%%%%%%%%%%%%%%%%%%%%%%%
\DescribeMacro{\childdocmain}
To use the package, add the commands
\begin{center}
\begin{tabular}{l}
|\input{childdoc.def}|\\
|\childdocmain{}|\\
\end{tabular}
\end{center}
at the very top of the main \LaTeX{} file,
in particular \emph{before} the |\documentclass| statement!
The argument of |\childdocmain| should be left empty
(but it must be present).

%%%%%%%%%%%%%%%%%%%%%%%%%%%%%%%%%%%%%%%%
\DescribeMacro{\childdocof}
Furthermore, add the commands
\begin{center}
\begin{tabular}{l}
|\input{childdoc.def}|\\
|\childdocof{|\textit{main}|}|\\
\end{tabular}
\end{center}
at the top of every child file \textit{child}
which is included by |\include{|\textit{child}|}|
from within the main file
(or at least for those files to be compiled individually).
The argument \textit{main} must be the filename of the main file.

There are a couple of
considerations in setting up the main and child documents:

%%%%%%%%%%%%%%%%%%%%%%%%%%%%%%%%%%%%%%%%
\paragraph{Restrictions.}

Please note the following restrictions:
\begin{itemize}
\item
|\childdocmain| must be called with one argument \textit{main}
to ensure compatibility with earlier version of the package.
It must either be empty (|\childdocmain{}|)
or precisely match the filename of the main file in which it is specified.
See \secref{sec:detection} for further information.
\item
The filename \textit{main} must be specified without the |.tex| extension.
\item
The filename \textit{main} is case sensitive
(even in case-insensitive file systems)
due to internal string comparison.
\item
The argument \textit{main} should be fully expanded, it cannot be a macro.
\item
Subdirectories and special characters should be avoided in filenames.
\item
The command |\childdocmain{|\textit{main}|}| must be followed by a whitespace.
It should not be followed immediately by another command
or by a comment mark `|%|'.
This is because the \TeX{} parser reads the token immediately following
the argument of |\childdocmain| and puts it
at the beginning of every child section;
however, a white\-space is ignored.
\end{itemize}

%%%%%%%%%%%%%%%%%%%%%%%%%%%%%%%%%%%%%%%%
\paragraph{Content of Main File.}

It is advisable to place all content in the child files included by |\include|.
Any output contained in the main file will appear in all child documents
unless suppressed manually;
it cannot be suppressed automatically by the |\includeonly| directive
and thus should normally be avoided.
A method to include some content in the main file
by means of conditional processing is described in \secref{sec:conditional}.

%%%%%%%%%%%%%%%%%%%%%%%%%%%%%%%%%%%%%%%%
\paragraph{Page Numbering.}

When only a part of the document is compiled,
the appropriate numbering of pages
(as well as other status parameters)
is determined from the |.aux| files.
The latter contain information from previous passes.
However this information needs to propagate through
all intermediate child documents.
Therefore the page numbering in child documents may well
be inconsistent until the complete document is compiled at least once.

A useful (if unconventional) way to always ensure a consistent
page numbering is to restart the numbering in each child document
and denote the pages by `\textit{child}|.|\textit{page}'
where \textit{child} represents the chapter/section number of the child file.
This can be achieved by the command
|\numberwithin{page}{|\textit{child}|}|
of the \textsf{amsmath} package
where \textit{child} can be |chapter| or |section|
depending on the chosen structuring.
Alternatively, one can modify the macro |\thepage| appropriately
and reset the counter |page| at the start of each child file.

%%%%%%%%%%%%%%%%%%%%%%%%%%%%%%%%%%%%%%%%%%%%%%%%%%%%%%%%%%%%%%%%%%%%%%%%%%%%%%%%
\subsection{Conditional Processing}
\label{sec:conditional}

The package provides a mechanism to compile different versions
of a document. To customise the versions further some conditional processing
can come in handy to distinguish which version is being compiled.
The package provides two macros to describe the compilation context:

%%%%%%%%%%%%%%%%%%%%%%%%%%%%%%%%%%%%%%%%
\DescribeMacro{\ifchilddoc}
The conditional |\ifchilddoc| distinguishes between the compilation of
child documents and the main document:
%
\begin{center}
|\ifchilddoc |\textit{child-code}| |[|\||else |\textit{main-code}]| \||fi|
\end{center}

%%%%%%%%%%%%%%%%%%%%%%%%%%%%%%%%%%%%%%%%
\DescribeMacro{\childdocname}
\DescribeMacro{\childdocjob}
The macro |\childdocname| contains the filename (without extension)
of the main or child file being processed.
Note that |\childdocjob| will always contain the name of the main file.

%%%%%%%%%%%%%%%%%%%%%%%%%%%%%%%%%%%%%%%%
\paragraph{Title Page.}

Conditional processing can be used to include a title or banner page
in the main document when proper precautions are taken.
Importantly, the code in the main file should ensure that the page counter
(as well as other status parameters which are stored in the |.aux| files)
takes the same value after the conditional processing.
Otherwise the page numbers may take divergent values
depending on which part is compiled.

For example, a title page could be declared by:
%
\begin{center}
\begin{tabular}{l}
|\ifchilddoc\||else|\\
|\addtocounter{page}{-1}|\\
\textit{code for title page}\\
|\newpage|\\
|\||fi|
\end{tabular}
\end{center}
%
A banner page for the child documents can be generated by:
%
\begin{center}
\begin{tabular}{l}
|\ifchilddoc|\\
|\addtocounter{page}{-1}|\\
\textit{code for banner page}\\
|\newpage|\\
|\||fi|
\end{tabular}
\end{center}
%
Here one could write a message such as:
\begin{center}
|This is the part \childdocname{} of \childdocjob{}.|
\end{center}

%%%%%%%%%%%%%%%%%%%%%%%%%%%%%%%%%%%%%%%%%%%%%%%%%%%%%%%%%%%%%%%%%%%%%%%%%%%%%%%%
\subsection{Flags}
\label{sec:flags}

The package makes it easy to generate different versions
of the main or child documents.
To this end compilation flags can be defined
and assigned different default values.
They will be particularly useful in conjunction
with the forwarding mechanism described in \secref{sec:forward}.

For example, it may be useful to have a flag |\version|
which can be set to |draft| or |final|.
The document source will contain some conditional code
depending on the value of |\version|.
Suppose further, the flag should default to |final| for the main file
and to |draft| for child files
which is a natural assignment for editing the document.
This is achieved by placing the following code
in the preamble of the main document
(below the |\childdocmain| directive):
%
\begin{center}
\begin{tabular}{l}
|\ifchilddoc|\\
|\providecommand{\version}{draft}|\\
|\||else|\\
|\providecommand{\version}{final}|\\
|\||fi|
\end{tabular}
\end{center}
%
The definition by |\providecommand| makes sure
that previous definitions are not overwritten.
Further statements |\providecommand{\version}{...}|
can thus be added before the above code to override it.

For the main file, one might add a line
(between |\childdocmain| and the above block)
%
\begin{center}
|%\ifchilddoc\||else\providecommand{\version}{draft}\||fi|
\end{center}
%
which can be uncommented to produce a draft version.
Likewise one can add a line to the very top of a child file
(above the |\childdocof{|\textit{main}|}| directive)
%
\begin{center}
|%\providecommand{\version}{final}|
\end{center}
%
which can be uncommented to produce the final version of this child document.

%%%%%%%%%%%%%%%%%%%%%%%%%%%%%%%%%%%%%%%%%%%%%%%%%%%%%%%%%%%%%%%%%%%%%%%%%%%%%%%%
\subsection{Forwarding}
\label{sec:forward}

Different versions of the main or child documents
using compilation flags as described in \secref{sec:flags}
can be (permanently) stored in different files
for convenient compilation, viewing and distribution.
To this end, the package defines a command
to pass on compilation to a different file:

%%%%%%%%%%%%%%%%%%%%%%%%%%%%%%%%%%%%%%%%
\DescribeMacro{\childdocforward}
The command |\childdocforward| redirects processing to
another source file:
%
\begin{center}
\begin{tabular}{l}
|\input{childdoc.def}|\\
|\childdocforward[|\textit{main}|]{|\textit{dest}|}|\\
\end{tabular}
\end{center}
%
The argument \textit{dest} is the destination file
(without extension).
It should be the main file or one of the child files.
Note that further \textsf{childdoc} directives
such as |\childdocof| and |\childdocforward|
in the indicated file will be processed in this form.
The optional argument \textit{main}
passes on directly to the main file \textit{main}
while pretending to compile the child \textit{dest}.
This form behaves as if \textit{dest}
issues |\childdocof{|\textit{main}|}| right away,
and no further \textsf{childdoc} directives will be processed.

%%%%%%%%%%%%%%%%%%%%%%%%%%%%%%%%%%%%%%%%
\DescribeMacro{\...prefix}
In the alternative form |\childdocforwardprefix|,
%
\begin{center}
\begin{tabular}{l}
|\input{childdoc.def}|\\
|\childdocforwardprefix[|\textit{main}|]{|\textit{prefix}|}{|\textit{dest}|}|
\end{tabular}
\end{center}
%
the destination file is determined by a pattern
depending on the current file:
To make this work, the current file must be called
`{\textit{prefix}\hspace{0.2em}\textit{suffix}}'
with \textit{prefix} matching precisely the argument.
Processing is then passed on to the file
`{\textit{dest}\hspace{0.2em}\textit{suffix}}'.
Surely, the same effect is achieved by
directly specifying the
argument `{\textit{dest}\hspace{0.2em}\textit{suffix}}'
in the first form.
However, that requires to set up a different file
for each child. With the alternative form of the command
all these files can have exactly the same content
which simplifies setting them up and maintaining them.

For example, the following file |draft.tex|
with a compilation flag |\version| as described in \secref{sec:flags}
compiles the main document as a draft:
%
\begin{center}
\begin{tabular}{l}
|\def\version{draft}|\\
|\input{childdoc.def}|\\
|\childdocforward{|\textit{main}|}|
\end{tabular}
\end{center}
%
Likewise, the following files |final|\textit{nn}|.tex|
compile the final version of the child document
|child|\textit{nn}|.tex|:
%
\begin{center}
\begin{tabular}{l}
|\def\version{final}|\\
|\input{childdoc.def}|\\
|\childdocforwardprefix{final}{child}|
\end{tabular}
\end{center}
%

Note that when several versions of a main file and/or of each child file
are to be generated, it may be convenient to set up a |Makefile| or
shell script to automatise the process.

%%%%%%%%%%%%%%%%%%%%%%%%%%%%%%%%%%%%%%%%%%%%%%%%%%%%%%%%%%%%%%%%%%%%%%%%%%%%%%%%
\subsection{Command Line Processing}
\label{sec:commandline}

The effect of redirection files can also be achieved by invoking
the \LaTeX{} compiler with a more elaborate command line.
Most conveniently this should be done as part
of a shell script or a |Makefile|.

When using \textsf{childdoc} in the main file, the following
command lines effectively perform a redirection
(note that depending on the shell being used,
backslashes may have to be doubled: `|\|' $\to$ `|\\|'):
%
\begin{center}
|... -jobname "|\textit{target}|" |\\|"|[\textit{flags}]%
|\input{childdoc.def}\childdocforward[|\textit{main}|]{|\textit{dest}|}"|
\end{center}
%
Here \textit{target} is the name of the output file,
\textit{main} is the name of the main file
and \textit{dest} is the name of the main or child file to be processed
(all filenames without extensions).
The optional argument \textit{main} can be omitted
if \textit{main} matches \textit{dest}.
Optionally, compilation \textit{flags} can be defined via |\def| commands.
This command line makes the \TeX{} engine believe
it is compiling the file \textit{target}
whose content is specified as the latter parameter.
The provided code then forwards the processing to
\textit{main} or \textit{dest} as described in \secref{sec:forward}.

%%%%%%%%%%%%%%%%%%%%%%%%%%%%%%%%%%%%%%%%%%%%%%%%%%%%%%%%%%%%%%%%%%%%%%%%%%%%%%%%
\subsection{Include by Input}
\label{sec:input}

Including child documents by |\include| has some restrictions by design.
Most notably, the content of a child document always occupies
its own set of pages; pages cannot be shared between child documents.
Usually, this behaviour makes perfect sense
because each child document contain an essential part of the document.
However, in some situations it may be desirable to compose
a document from a collection of parts
without having mandatory page breaks between then.
For this case, the package
provides a mechanism to include parts
by |\input| which can also be processed individually.
However, by construction this mechanism
requires manual handling of the content to be output.

%%%%%%%%%%%%%%%%%%%%%%%%%%%%%%%%%%%%%%%%
\DescribeMacro{\ifchilddocmanual}
The main file should be prepared as usual, see \secref{sec:include}.
However, the document body must make a distinction
between processing of an individual part and of the main document, e.g.:
%
\begin{center}
\begin{tabular}{l}
|\ifchilddocmanual|\\
|\input{\childdocname}|\\
|\||else|\\
\textit{document body with }|\input{|\textit{part}|}|\\
|\||fi|
\end{tabular}
\end{center}
%
The conditional |\ifchilddocmanual| is true whenever
a part to be included by |\input| is being compiled,
and the name of the part is stored in |\childdocname|.

%%%%%%%%%%%%%%%%%%%%%%%%%%%%%%%%%%%%%%%%
\DescribeMacro{\childdocby}
Each part to be included by |\input| should start with:
%
\begin{center}
\begin{tabular}{l}
|\input{childdoc.def}|\\
|\childdocby{|\textit{main}|}|\\
\end{tabular}
\end{center}
%
The directive |\childdocby| is similar to |\childdocof|
described in \secref{sec:include},
but the subsequent selection of content must be done manually.
To that end, both |\ifchilddoc| and |\ifchilddocmanual|
will be true upon processing of a part,
and the name of the part is stored in |\childdocname|.
Note that |\jobname| will be set to the filename of the current part
so that each part receives an individual |.aux| file
that does not interfere with the |.aux| file(s) of the main document.
This behaviour can be altered by the alternative form
|\childdocby[*]{|\textit{main}|}| (with a non-empty optional argument)
which uses the |.aux| file of the main document
by setting |\jobname| to \textit{main}.

%%%%%%%%%%%%%%%%%%%%%%%%%%%%%%%%%%%%%%%%%%%%%%%%%%%%%%%%%%%%%%%%%%%%%%%%%%%%%%%%
\subsection{Driver Development}
\label{sec:driver}

The \textsf{childdoc} mechanism can also be use for the development
of definition files such as \LaTeX{} styles or classes.
This case differs from the above setup with multiple parts
included by |\include| in that no |\includeonly| should be invoked.
This can be achieved by starting the include file
(before |\ProvidesPackage|) with:
%
\begin{center}
\begin{tabular}{l}
|\input{childdoc.def}|\\
|\childdocforward{|\textit{main}|}|\\
\end{tabular}
\end{center}
%
or alternatively with:
%
\begin{center}
\begin{tabular}{l}
|\input{childdoc.def}|\\
|\childdocby{|\textit{main}|}|\\
\end{tabular}
\end{center}
%
Both forms have slightly different effects as described above.
The main file is prepared as usual, see \secref{sec:include}.

%%%%%%%%%%%%%%%%%%%%%%%%%%%%%%%%%%%%%%%%%%%%%%%%%%%%%%%%%%%%%%%%%%%%%%%%%%%%%%%%
\subsection{Legacy Detection}
\label{sec:detection}

The directive |\childdocmain| in the main file can detect
whether the complete document or merely a child is to be compiled
even without using the directive |\childdocof|.
This method is deprecated because it is less robust
and there is no compelling reason to use it;
it is merely provided for backward compatibility
and it may be removed in future versions.

If the detection mechanism is to be used,
it is mandatory to correctly specify
the filename of the main file as the argument of |\childdocmain|:
%
\begin{center}
\begin{tabular}{l}
|\input{childdoc.def}|\\
|\childdocmain{|\textit{main}|}|\\
\end{tabular}
\end{center}
%
If |\jobname| does not match the argument \textit{main} of |\childdocmain|,
it is assumed that |\jobname| points to the child file to be compiled.
When using |\childdocmain| with the main file specified as argument,
it suffices to start a child file
with just |\input{|\textit{main}|}|
without loading of the package and using |\childdocof|.
If instead all processing is done
with the appropriate \textsf{childdoc} directives,
the argument of \textit{main} of |\childdocmain| can be empty.

An alternative version of the command line processing described
in \secref{sec:commandline} using the detection mechanism reads:
%
\begin{center}
|... -jobname "|\textit{target}|" "|[\textit{flags}]%
[|\def\jobname{|\textit{dest}|}|]|\input{|\textit{main}|}"|
\end{center}

%%%%%%%%%%%%%%%%%%%%%%%%%%%%%%%%%%%%%%%%%%%%%%%%%%%%%%%%%%%%%%%%%%%%%%%%%%%%%%%%
\subsection{Manual Code}
\label{sec:manual}

In case one cannot be certain whether the definitions file |childdoc.def|
is installed on the target \TeX{} distribution
and one prefers not to ship it,
it is conceivable to paste a few relevant commands into the sources.

To that end, drop all statements |\input{childdoc.def}|
and perform the replacements as outlined below.
Instead of |\childdocmain{|\textit{main}|}| add the following code
to the top of the main file:
%
\begin{center}
\begin{tabular}{l}
|\||ifdefined\childdocname\endinput\||fi\newif\ifchilddoc|\\
|\edef\childdocname{\scantokens\expandafter{\jobname\noexpand}}|\\
|\def\childdocmain{|\textit{main}|}\||ifx\childdocmain\childdocname\||else|\\
|\childdoctrue\includeonly{\childdocname}\let\jobname\childdocmain\||fi|\\
\end{tabular}
\end{center}
%
Instead of |\childdocof{|\textit{main}|}| just include the main file
at the top of each child file:
%
\begin{center}
|\input{|\textit{main}|}|
\end{center}
%
A simple redirection |\childdocforward{|\textit{dest}|}| is achieved by:
%
\begin{center}
|\def\jobname{|\textit{dest}|}\input{\jobname}|
\end{center}
%
The redirection with prefix
|\childdocforwardprefix[|\textit{prefix}|]{|\textit{dest}|}|
is accomplished by:
%
\begin{center}
\begin{tabular}{l}
|{\edef\jobname{\scantokens\expandafter{\jobname\noexpand}}|\\
|\def\redirectjob |\textit{prefix}|#1~~~{\gdef\jobname{|\textit{dest}|#1}}|\\
|\expandafter\redirectjob\jobname~~~}\input{\jobname}|
\end{tabular}
\end{center}

In an alternative approach,
child documents can be compiled by a specific command line
without additional code or specific definitions:
%
\begin{center}
|... -jobname "|\textit{target}|" "|[\textit{flags}]%
|\includeonly{|\textit{dest}|}\input{|\textit{main}|}"|
\end{center}
%

%%%%%%%%%%%%%%%%%%%%%%%%%%%%%%%%%%%%%%%%%%%%%%%%%%%%%%%%%%%%%%%%%%%%%%%%%%%%%%%%
%%%%%%%%%%%%%%%%%%%%%%%%%%%%%%%%%%%%%%%%%%%%%%%%%%%%%%%%%%%%%%%%%%%%%%%%%%%%%%%%
\section{Information}

%%%%%%%%%%%%%%%%%%%%%%%%%%%%%%%%%%%%%%%%%%%%%%%%%%%%%%%%%%%%%%%%%%%%%%%%%%%%%%%%
\subsection{Copyright}

Copyright \copyright{} 2017--2018 Niklas Beisert

This work may be distributed and/or modified under the
conditions of the \LaTeX{} Project Public License, either version 1.3
of this license or (at your option) any later version.
The latest version of this license is in
  \url{http://www.latex-project.org/lppl.txt}
and version 1.3 or later is part of all distributions of \LaTeX{}
version 2005/12/01 or later.

This work has the LPPL maintenance status `maintained'.

The Current Maintainer of this work is Niklas Beisert.

This work consists of the files |README.txt|, |childdoc.ins| and |childdoc.dtx|
as well as the derived files |childdoc.def|, |cdocsamp.tex|
with |cdocsch1.tex|, |cdocsch2.tex|, |cdocspt3.tex|, |cdocspt4.tex|,
|cdocsdrf.tex|, |cdocsfn1.tex|, |cdocsfn2.tex|
as well as |childdoc.pdf|.

%%%%%%%%%%%%%%%%%%%%%%%%%%%%%%%%%%%%%%%%%%%%%%%%%%%%%%%%%%%%%%%%%%%%%%%%%%%%%%%%
\subsection{Files and Installation}

The package consists of the files:
%
\begin{center}
\begin{tabular}{ll}
    |README.txt|   & readme file \\
    |childdoc.ins| & installation file \\
    |childdoc.dtx| & source file \\
    |childdoc.def| & definition file \\
    |cdocsamp.tex| & sample main file \\
    |cdocsch1.tex| & sample include file \\
    |cdocsch2.tex| & sample include file \\
    |cdocspt3.tex| & sample part file \\
    |cdocspt4.tex| & sample part file \\
    |cdocsdrf.tex| & sample redirection file \\
    |cdocsfn1.tex| & sample redirection file \\
    |cdocsfn2.tex| & sample redirection file \\
    |childdoc.pdf| & manual
\end{tabular}
\end{center}
%
The distribution consists of the files
|README.txt|, |childdoc.ins| and |childdoc.dtx|.
%
\begin{itemize}
\item
Run (pdf)\LaTeX{} on |childdoc.dtx|
to compile the manual |childdoc.pdf| (this file).
\item
Run \LaTeX{} on |childdoc.ins| to create the definitions file |childdoc.def|
and the sample |cdocsamp.tex| with include files
|cdocsch1.tex|, |cdocsch2.tex|, |cdocspt3.tex|, |cdocspt4.tex|,
|cdocsdrf.tex|, |cdocsfn1.tex|, |cdocsfn2.tex|.
Then copy the file |childdoc.def| to an appropriate directory of your \LaTeX{}
distribution, e.g.\ \textit{texmf-root}|/tex/latex/childdoc|.
\end{itemize}

%%%%%%%%%%%%%%%%%%%%%%%%%%%%%%%%%%%%%%%%%%%%%%%%%%%%%%%%%%%%%%%%%%%%%%%%%%%%%%%%
\subsection{Related CTAN Packages}

There are several other packages which offer a similar functionality:
%
\begin{itemize}
\item
The packages
\href{http://ctan.org/pkg/docmute}{\textsf{docmute}},
\href{http://ctan.org/pkg/includex}{\textsf{includex}} and
\href{http://ctan.org/pkg/standalone}{\textsf{standalone}}
provide commands to include only the document body of
a child file thus allowing both files to be compiled individually.
\item
The packages \href{http://ctan.org/pkg/subdocs}{\textsf{subdocs}}
and \href{http://ctan.org/pkg/subfiles}{\textsf{subfiles}}
provide structures in which the main and child documents can be
encapsulated and allowing them to be compiled individually.
The inclusion mechanism is different from the conventional |\include|.
\item
The package \href{http://ctan.org/pkg/combine}{\textsf{combine}}
is an elaborate solution to combine several documents into one.
\end{itemize}
%
See also the CTAN topic \href{http://ctan.org/topic/subdocs}{\textsf{subdocs}}
for further related packages.
The present package differs from the above solutions in that
a document structure constructed with the conventional |\include| mechanism
just needs two extra commands at the top of every file
such that all constituent files can be compiled individually.

%%%%%%%%%%%%%%%%%%%%%%%%%%%%%%%%%%%%%%%%%%%%%%%%%%%%%%%%%%%%%%%%%%%%%%%%%%%%%%%%
%\subsection{Feature Suggestions}
%
%The following is a list of features which may be useful for future
%versions of this package:
%%
%\begin{itemize}
%\item
%\ldots
%\end{itemize}

%%%%%%%%%%%%%%%%%%%%%%%%%%%%%%%%%%%%%%%%%%%%%%%%%%%%%%%%%%%%%%%%%%%%%%%%%%%%%%%%
\subsection{Revision History}

%%%%%%%%%%%%%%%%%%%%%%%%%%%%%%%%%%%%%%%%
\paragraph{v2.0:} 2018/12/30

\begin{itemize}
\item
immediate forward processing
\item
added |\childdocby| mechanism
\item
manual restructured
\end{itemize}

%%%%%%%%%%%%%%%%%%%%%%%%%%%%%%%%%%%%%%%%
\paragraph{v1.6:} 2018/01/17

\begin{itemize}
\item
application for development of include files
\item
corrections to manual
\end{itemize}

%%%%%%%%%%%%%%%%%%%%%%%%%%%%%%%%%%%%%%%%
\paragraph{v1.5:} 2017/05/21

\begin{itemize}
\item
more complete structuring introduced
\item
|\childdocof| introduced
\item
|\childdoc| renamed to |\childdocmain|
\item
|\childredirect| renamed to |\childdocforward| and |\childdocforwardprefix|
and functionality expanded
\end{itemize}

%%%%%%%%%%%%%%%%%%%%%%%%%%%%%%%%%%%%%%%%
\paragraph{v1.0:} 2017/04/27

\begin{itemize}
\item
manual and install package
\item
first version published on CTAN
\end{itemize}

%%%%%%%%%%%%%%%%%%%%%%%%%%%%%%%%%%%%%%%%
\paragraph{v0.6:} 2017/04/26

\begin{itemize}
\item
redirection mechanism added
\end{itemize}

%%%%%%%%%%%%%%%%%%%%%%%%%%%%%%%%%%%%%%%%
\paragraph{v0.5:} 2017/04/26

\begin{itemize}
\item
functionality in definition file
\end{itemize}


%%%%%%%%%%%%%%%%%%%%%%%%%%%%%%%%%%%%%%%%%%%%%%%%%%%%%%%%%%%%%%%%%%%%%%%%%%%%%%%%
%%%%%%%%%%%%%%%%%%%%%%%%%%%%%%%%%%%%%%%%%%%%%%%%%%%%%%%%%%%%%%%%%%%%%%%%%%%%%%%%
%%%%%%%%%%%%%%%%%%%%%%%%%%%%%%%%%%%%%%%%%%%%%%%%%%%%%%%%%%%%%%%%%%%%%%%%%%%%%%%%
\appendix

\settowidth\MacroIndent{\rmfamily\scriptsize 000\ }

 \DocInput{childdoc.dtx}

\end{document}
%</driver>
% \fi
%
% %%%%%%%%%%%%%%%%%%%%%%%%%%%%%%%%%%%%%%%%%%%%%%%%%%%%%%%%%%%%%%%%%%%%%%%%%%%%%%
% %%%%%%%%%%%%%%%%%%%%%%%%%%%%%%%%%%%%%%%%%%%%%%%%%%%%%%%%%%%%%%%%%%%%%%%%%%%%%%
% \section{Sample}
%\iffalse
%<*samplemain>
%\fi
%
% The following presents a sample document
% with two chapters, two parts, a title page,
% a compile flag as well as three forwarding files to set the flag.
% It consists of eight |.tex| files:
% \begin{center}
% \begin{tabular}{ll}
% |cdocsamp.tex|&main file\\
% |cdocsch1.tex|&include file for chapter 1\\
% |cdocsch2.tex|&include file for chapter 2\\
% |cdocspt3.tex|&include file for part 3\\
% |cdocspt4.tex|&include file for part 4\\
% |cdocsdrf.tex|&forwarding file for main file in draft mode\\
% |cdocsfi1.tex|&forwarding file for final version of chapter 1\\
% |cdocsfi2.tex|&forwarding file for final version of chapter 2\\
% \end{tabular}
% \end{center}
% Each of the eight files can be compiled directly by the \LaTeX{} compiler.
%
% %%%%%%%%%%%%%%%%%%%%%%%%%%%%%%%%%%%%%%
% \paragraph{Main File.}
%
% The main file is called |cdocsamp.tex|.
%
% Load the \textsf{childdoc} definitions and
% declare the filename for the main document:
%    \begin{macrocode}
\input{childdoc.def}
\childdocmain{}
%    \end{macrocode}

% Optional override for |\version| flag:
%    \begin{macrocode}
%%\ifchilddoc\else\providecommand{\version}{draft}\fi
%    \end{macrocode}

% Define the default values for the |\version| flag
% (|final| for the main file and |draft| for childs):
%    \begin{macrocode}
\ifchilddoc
\providecommand{\version}{draft}
\else
\providecommand{\version}{final}
\fi
%    \end{macrocode}

% Load the standard document class:
%    \begin{macrocode}
\documentclass[12pt]{article}
%    \end{macrocode}

% Start the document body:
%    \begin{macrocode}
\begin{document}
%    \end{macrocode}

% Declare a title page.
% Print title, part of document being processed and version flag:
%    \begin{macrocode}
\addtocounter{page}{-1}
\begin{center}
{\LARGE\bfseries{}childdoc example\par}
\vspace{1cm}
\ifchilddoc
\ifchilddocmanual part\else chapter\fi:
`\childdocname' of `\childdocjob'\par
\else
main document: `\childdocjob'\par
\fi
version: \version\par
\end{center}
\newpage
%    \end{macrocode}

% Manually include selected file,
% otherwise process as usual:
%    \begin{macrocode}
\ifchilddocmanual
\section*{part `\childdocname'}
\input{\childdocname}
\else
%    \end{macrocode}

% Include the two chapters:
%    \begin{macrocode}
\include{cdocsch1}
\include{cdocsch2}
%    \end{macrocode}

% Include the two parts unless only chapters should be displayed:
%    \begin{macrocode}
\ifchilddoc\else
\section{part three}
\input{cdocspt3}
\section{part four}
\input{cdocspt4}
\fi
%    \end{macrocode}

% Process as usual until here:
%    \begin{macrocode}
\fi
%    \end{macrocode}

% End of document body:
%    \begin{macrocode}
\end{document}
%    \end{macrocode}
%\iffalse
%</samplemain>
%\fi
%
% %%%%%%%%%%%%%%%%%%%%%%%%%%%%%%%%%%%%%%
% \paragraph{Chapter Include Files.}
%
% The include files are called |cdocsch1.tex| and |cdocsch2.tex|.
%
%\iffalse
%<*samplechap1|samplechap2>
%\fi

% Optional override for |\version| flag:
%    \begin{macrocode}
%%\providecommand{\version}{final}
%    \end{macrocode}

% Include the main document:
%    \begin{macrocode}
\input{childdoc.def}
\childdocof{cdocsamp}
%    \end{macrocode}

%\iffalse
%</samplechap1|samplechap2>
%\fi
%
%\iffalse
%<*samplechap1>
%\fi
% Some text for chapter 1:
%    \begin{macrocode}
\section{one}
some text in chapter one
%    \end{macrocode}

%\iffalse
%</samplechap1>
%\fi
% Some text for chapter 2:
%\iffalse
%<*samplechap2>
%\fi
%    \begin{macrocode}
\section{two}
more text in chapter two
%    \end{macrocode}

%\iffalse
%</samplechap2>
%\fi
%
% %%%%%%%%%%%%%%%%%%%%%%%%%%%%%%%%%%%%%%
% \paragraph{Part Include Files.}
%
% The include files are called |cdocspt3.tex| and |cdocspt4.tex|.
%
%\iffalse
%<*samplepart3|samplepart4>
%\fi

% Optional override for |\version| flag:
%    \begin{macrocode}
%%\providecommand{\version}{final}
%    \end{macrocode}

% Include the main document:
%    \begin{macrocode}
\input{childdoc.def}
\childdocby{cdocsamp}
%    \end{macrocode}

%\iffalse
%</samplepart3|samplepart4>
%\fi
%
%\iffalse
%<*samplepart3>
%\fi
% Some text for part 3:
%    \begin{macrocode}
some text in part three
%    \end{macrocode}

%\iffalse
%</samplepart3>
%\fi
% Some text for part 4:
%\iffalse
%<*samplepart4>
%\fi
%    \begin{macrocode}
more text in part four
%    \end{macrocode}

%\iffalse
%</samplepart4>
%\fi
%
% %%%%%%%%%%%%%%%%%%%%%%%%%%%%%%%%%%%%%%
% \paragraph{Forwarding for a Complete Draft.}
%
% The following forwarding file |cdocsdrf.tex|
% compiles the main document in draft mode:
%\iffalse
%<*sampledraft>
%\fi
%    \begin{macrocode}
\def\version{draft}
\input{childdoc.def}
\childdocforward{cdocsamp}
%    \end{macrocode}

%\iffalse
%</sampledraft>
%\fi
%
% %%%%%%%%%%%%%%%%%%%%%%%%%%%%%%%%%%%%%%
% \paragraph{Forwarding for Final Version of the Chapters.}
%
% The following forwarding files |cdocsfn1.tex| and |cdocsfn2.tex|
% (with identical content)
% compile the final versions of the child documents
% |cdocsch1.tex| and |cdocsch2.tex|, respectively:
%\iffalse
%<*samplefinal>
%\fi
%    \begin{macrocode}
\def\version{final}
\input{childdoc.def}
\childdocforwardprefix[cdocsamp]{cdocsfn}{cdocsch}
%    \end{macrocode}

%\iffalse
%</samplefinal>
%\fi
%
% %%%%%%%%%%%%%%%%%%%%%%%%%%%%%%%%%%%%%%
% \paragraph{Command Line Processing.}
%
% The following three command lines generate the output files
% |cdocscld|, |cdocscl1| and |cdocscl2|
% which should be identical to
% |cdocsdrf|, |cdocsch1| and |cdocsfn2|, respectively:
% \begin{center}
% \begin{tabular}{l}
% |latex -jobname cdocscld \|\\
% |  "\def\version{draft}\input{childdoc.def}\childdocforward{cdocsamp}"|\\
% |latex -jobname cdocscl1 \|\\
% |  "\input{childdoc.def}\childdocforward[cdocsamp]{cdocsch1}"|\\
% |latex -jobname cdocscl2 \|\\
% |  "\def\version{final}\input{childdoc.def}\childdocforward{cdocsch2}"|
% \end{tabular}
% \end{center}
% Note that the trailing backslash on each first line
% merely continues the input to the second line
% (for convenient cut ant paste).
% Furthermore, the command |latex| can be replaced by any
% of its alternative versions such as |pdflatex|.
%
% %%%%%%%%%%%%%%%%%%%%%%%%%%%%%%%%%%%%%%%%%%%%%%%%%%%%%%%%%%%%%%%%%%%%%%%%%%%%%%
% %%%%%%%%%%%%%%%%%%%%%%%%%%%%%%%%%%%%%%%%%%%%%%%%%%%%%%%%%%%%%%%%%%%%%%%%%%%%%%
% \section{Implementation}
%\iffalse
%<*package>
%\fi
%
% This section describes the definitions file |childdoc.def|.

% The definitions cannot be loaded using |\usepackage| or |\RequirePackage|
% which has a mechanism to prevent loading a style file more than once.
% When loading the definitions by means of |\input|
% multiple instances have to be prevented manually:
%\iffalse
%This code needs to be before the `\ProvidesFile' directive
%which is defined at the beginning of this file.
%Therefore it is also placed there and commented out here.
%</package>
%<*discard>
%\fi
%    \begin{macrocode}
\ifdefined\childdocmain\endinput\fi
%    \end{macrocode}
%\iffalse
%</discard>
%<*package>
%\fi
%
% \macro{\ifchilddoc}
% \macro{\ifchilddocmanual}
% The conditional |\ifchilddoc| tells whether a
% child (true) or main (false) document is being compiled.
% The conditional |\ifchilddocmanual| tells whether
% the |\includeonly| mechanism is used (false) or
% the selection of child files must be performed manually (true).
% The definitions initialise to false:
%    \begin{macrocode}
\newif\ifchilddoc
\newif\ifchilddocmanual
%    \end{macrocode}

% \macro{\childdocname}
% \macro{\childdocjob}
% The macro |\childdocname| stores the name of the main document
% to be compiled. The macro |\childdocjob| stores the name of
% the document on which the \LaTeX{} compiler was originally invoked.
% The content of |\jobname| cannot be compared
% to filenames specified in the source due to different catcodes.
% The following code rescans |\jobname|, stores the result
% in |\childdocname| and saves a copy in |\childdocjob|:
%    \begin{macrocode}
\edef\childdocname{\scantokens\expandafter{\jobname\noexpand}}
\let\childdocjob\childdocname
%    \end{macrocode}

% \macro{\childdocdisable}
% The macro |\childdocdisable| prevents the main file
% from being processed more than once.
% At this stage, the main document command |\childdocmain|
% is assumed to be called once again where it should do nothing.
% Any subsequent call to it should prevent
% a secondary processing of the main document
% It overwrites the forwarding commands
% |\childdocof| and |\childdocforward|
% with empty macros to prevent further inclusions of the main document:
%    \begin{macrocode}
\newcommand{\childdocdisable}
{
  \renewcommand{\childdocmain}[1]{\renewcommand{\childdocmain}[1]{\endinput}}
  \renewcommand{\childdocof}[1]{}
  \renewcommand{\childdocby}[2][]{}
  \renewcommand{\childdocforward}[2][]{}
  \renewcommand{\childdocdisable}{}
}
%    \end{macrocode}

% \macro{\childdocmain}
% The macro |\childdocmain| is to be called at the top of the main file
% with nothing or the main filename (without extension) as argument.
% First, it breaks loops.
% If the argument is not empty and does not match |\childdocname|
% (which is set by the first inclusion of |childdoc.def|),
% |\ifchilddoc| is set to true, |\includeonly| is applied to the child file
% and |\jobname| is set to the main file
% (for proper handling of |.aux| files):
%    \begin{macrocode}
\newcommand{\childdocmain}[1]
{
  \childdocdisable\childdocmain{}
  \if?#1?\else
    \begingroup
      \def\childdoctmp{#1}
      \ifx\childdoctmp\childdocname
        \def\childdoctmp{}
      \else
        \def\childdoctmp
        {
          \childdoctrue
          \includeonly{\childdocname}
          \def\childdocjob{#1}
          \def\jobname{#1}
        }
      \fi
      \expandafter
    \endgroup
    \childdoctmp
  \fi
}
%    \end{macrocode}

% \macro{\childdocof}
% The command |\childdocof| redirects
% compilation to the main file |#1|.
%    \begin{macrocode}
\newcommand{\childdocof}[1]
{
  \childdocdisable
  \childdoctrue
  \includeonly{\childdocname}
  \def\jobname{#1}
  \def\childdocjob{#1}
  \input{#1}
}
%    \end{macrocode}

% \macro{\childdocby}
% The command |\childdocby| ....
%    \begin{macrocode}
\newcommand{\childdocby}[2][]
{
  \childdocdisable
  \childdoctrue
  \childdocmanualtrue
  \if?#1?\else
    \def\jobname{#2}
  \fi
  \def\childdocjob{#2}
  \input{#2}
  \endinput
}
%    \end{macrocode}

% \macro{\childdocforward}
% The command |\childdocforward| redirects
% compilation to the main file or
% (if the optional argument is given) a child file.
% Parameters are set as if the main file
% or a child file starting with |\childdocof| was compiled.
% Then compilation is handed over to the main file:
%    \begin{macrocode}
\newcommand{\childdocforward}[2][]
{
  \begingroup
    \if?#1?
      \def\childdoctmp
      {
        \def\childdocname{#2}
        \def\childdocjob{#2}
        \def\jobname{#2}
        \input{#2}
        \endinput
      }
    \else
      \def\childdoctmp
      {
        \childdocdisable
        \def\childdocname{#2}
        \childdoctrue
        \includeonly{#2}
        \def\childdocjob{#1}
        \def\jobname{#1}
        \input{#1}
        \endinput
      }
    \fi
    \expandafter
  \endgroup
  \childdoctmp
}
%    \end{macrocode}

% \macro{\childdocforwardprefix}
% The command |\childdocforwardprefix| redirects
% compilation to the main or a child file by means of a pattern.
% The prefix |#1| in the current filename is replaced by |#2|
% and the suffix of the current filename is kept
% (it is assumed that the filename does not contain the substring `|~~~|'
% which is used as a delimiter).
% Compilation is handed over to the new file by |\childdocforward|:
%    \begin{macrocode}
\newcommand{\childdocforwardprefix}[3][]
{
  \begingroup
    \def\childdocextract #2##1~~~{\def\childdoctmp{\childdocforward[#1]{#3##1}}}
    \expandafter\childdocextract\childdocname~~~
    \expandafter
  \endgroup
  \childdoctmp
}
%    \end{macrocode}

% \macro{\childdoc}
% The deprecated macro |\childdoc| is a legacy version of |\childdocmain|:
%    \begin{macrocode}
\newcommand{\childdoc}{\childdocmain}
%    \end{macrocode}

% \macro{\childdocredirect}
% The deprecated macro |\childdocredirect| is a legacy version
% of |\childdocforward| and |\childdocforwardprefix|:
%    \begin{macrocode}
\newcommand{\childdocredirect}[2][]
{
  \begingroup
    \if?#1?
      \def\childdoctmp{\childdocforward{#2}}
    \else
      \def\childdoctmp{\childdocforwardprefix{#1}{#2}}
    \fi
    \expandafter
  \endgroup
  \childdoctmp
}
%    \end{macrocode}

%\iffalse
%</package>
%\fi
%
\endinput
\childdocforward{cdocsamp}"|\\
% |latex -jobname cdocscl1 \|\\
% |  "% \iffalse
%
% childdoc.dtx Copyright (C) 2017-2018 Niklas Beisert
%
% This work may be distributed and/or modified under the
% conditions of the LaTeX Project Public License, either version 1.3
% of this license or (at your option) any later version.
% The latest version of this license is in
%   http://www.latex-project.org/lppl.txt
% and version 1.3 or later is part of all distributions of LaTeX
% version 2005/12/01 or later.
%
% This work has the LPPL maintenance status `maintained'.
%
% The Current Maintainer of this work is Niklas Beisert.
%
% This work consists of the files childdoc.dtx and childdoc.ins
% and the derived files childdoc.def and cdocsamp.tex with
% cdocsch1.tex, cdocsch2.tex, cdocsdrf.tex, cdocsfn1.tex, cdocsfn2.tex.
%
%<package>\ifdefined\childdocmain\endinput\fi
%<package>\ProvidesFile{childdoc.def}[2018/12/30 v2.0 child document driver]
%<samplemain>\ProvidesFile{cdocsamp.tex}[2018/12/30 v2.0 sample for childdoc]
%<*driver>
%\ProvidesFile{childdoc.drv}[2018/12/30 v2.0 childdoc reference manual file]
\PassOptionsToClass{10pt,a4paper}{article}
\documentclass{ltxdoc}

\usepackage[margin=35mm]{geometry}
\usepackage{hyperref}
\usepackage{hyperxmp}
\usepackage[usenames]{color}

\hypersetup{colorlinks=true}
\hypersetup{pdfstartview=FitH}
\hypersetup{pdfpagemode=UseNone}
\hypersetup{pdfsource={}}
\hypersetup{pdflang={en-UK}}
\hypersetup{pdfcopyright={Copyright 2017-2018 Niklas Beisert.
  This work may be distributed and/or modified under the
  conditions of the LaTeX Project Public License, either version 1.3
  of this license or (at your option) any later version.}}
\hypersetup{pdflicenseurl={http://www.latex-project.org/lppl.txt}}
\hypersetup{pdfcontactaddress={ETH Zurich, ITP, HIT K,
  Wolfgang-Pauli-Strasse 27}}
\hypersetup{pdfcontactpostcode={8093}}
\hypersetup{pdfcontactcity={Zurich}}
\hypersetup{pdfcontactcountry={Switzerland}}
\hypersetup{pdfcontactemail={nbeisert@itp.phys.ethz.ch}}
\hypersetup{pdfcontacturl={http://people.phys.ethz.ch/\xmptilde nbeisert/}}

\newcommand{\secref}[1]{\hyperref[#1]{section \ref*{#1}}}

\parskip1ex
\parindent0pt
\let\olditemize\itemize
\def\itemize{\olditemize\parskip0pt}

\begin{document}

\title{The \textsf{childdoc} Package}
\hypersetup{pdftitle={The childdoc Package}}
\author{Niklas Beisert\\[2ex]
  Institut f\"ur Theoretische Physik\\
  Eidgen\"ossische Technische Hochschule Z\"urich\\
  Wolfgang-Pauli-Strasse 27, 8093 Z\"urich, Switzerland\\[1ex]
  \href{mailto:nbeisert@itp.phys.ethz.ch}
  {\texttt{nbeisert@itp.phys.ethz.ch}}}
\hypersetup{pdfauthor={Niklas Beisert}}
\hypersetup{pdfsubject={Manual for the LaTeX2e Package childdoc}}
\date{30 December 2018, \textsf{v2.0}}
\maketitle

\begin{abstract}\noindent
\textsf{childdoc} is a \LaTeXe{} package
that enables the direct compilation
of document sections included by |\include|
to individual files.
\end{abstract}

\begingroup
\parskip0ex
\tableofcontents
\endgroup

%%%%%%%%%%%%%%%%%%%%%%%%%%%%%%%%%%%%%%%%%%%%%%%%%%%%%%%%%%%%%%%%%%%%%%%%%%%%%%%%
%%%%%%%%%%%%%%%%%%%%%%%%%%%%%%%%%%%%%%%%%%%%%%%%%%%%%%%%%%%%%%%%%%%%%%%%%%%%%%%%
\section{Introduction}

\LaTeX{} provides a mechanism to structure a large document (such as a book)
into a main file and several child files (containing the chapters)
using the |\include| command.
This mechanism is beneficial for documents
which span hundreds of pages in order to
make the source file(s) more manageable.
Moreover, compilation can be restricted to
selected child files by means of the |\includeonly| command.
The latter feature can be used to reduce the compilation time while editing
(this was significantly more useful in the earlier days of \LaTeX{})
or to generate a smaller document which is easier to navigate.
Another application of |\includeonly| is to generate
documents consisting of selected parts of the complete document.

However, there are a few drawbacks of the plain |\include| mechanism:
\begin{itemize}
\item
The child files cannot be compiled on their own,
they can only be compiled via the main file.
A naive editing environment
(such as a text editor with an option
to have the current file processed by \LaTeX)
may require one to switch to the main file before compiling;
attempting to compile the child file produces errors.
\item
The main file must be modified (each time)
to adjust the |\includeonly| command
to the present needs. This easily leaves the main file in a messy state.
\item
The generated document will always carry the filename
of the main document. This is inconvenient if
several child files are to be compiled and
to be kept for distribution.
\end{itemize}

The present package provides a simple interface
to make child files individually compilable by \LaTeX{}.
Compiling a child file then has the same effect as compiling
the main file with an |\includeonly| command
to select the appropriate child.
Moreover the generated document will carry the name of the child
rather than the main file.
This resolves all three above issues.

This feature is meant to make the editing of books,
thesis documents and lecture notes somewhat more convenient.
However, the package can also be used efficiently for
composing a series of documents (such as exercise sheets)
which are typically distributed individually.
It then assists the author in generating the individual documents
(potentially in different versions)
as well as a document containing the collected series.
Another application is in developing style files
or other kinds of included material
where compilation of the style file could redirect
to a sample or test file.

%%%%%%%%%%%%%%%%%%%%%%%%%%%%%%%%%%%%%%%%%%%%%%%%%%%%%%%%%%%%%%%%%%%%%%%%%%%%%%%%
%%%%%%%%%%%%%%%%%%%%%%%%%%%%%%%%%%%%%%%%%%%%%%%%%%%%%%%%%%%%%%%%%%%%%%%%%%%%%%%%
\section{Usage}

First of all, the package \textsf{childdoc} is \emph{not} a standard
\LaTeXe{} |.sty| style file! Therefore it needs to be invoked in
a non-standard way.

%%%%%%%%%%%%%%%%%%%%%%%%%%%%%%%%%%%%%%%%%%%%%%%%%%%%%%%%%%%%%%%%%%%%%%%%%%%%%%%%
\subsection{Included Files}
\label{sec:include}

%%%%%%%%%%%%%%%%%%%%%%%%%%%%%%%%%%%%%%%%
\DescribeMacro{\childdocmain}
To use the package, add the commands
\begin{center}
\begin{tabular}{l}
|\input{childdoc.def}|\\
|\childdocmain{}|\\
\end{tabular}
\end{center}
at the very top of the main \LaTeX{} file,
in particular \emph{before} the |\documentclass| statement!
The argument of |\childdocmain| should be left empty
(but it must be present).

%%%%%%%%%%%%%%%%%%%%%%%%%%%%%%%%%%%%%%%%
\DescribeMacro{\childdocof}
Furthermore, add the commands
\begin{center}
\begin{tabular}{l}
|\input{childdoc.def}|\\
|\childdocof{|\textit{main}|}|\\
\end{tabular}
\end{center}
at the top of every child file \textit{child}
which is included by |\include{|\textit{child}|}|
from within the main file
(or at least for those files to be compiled individually).
The argument \textit{main} must be the filename of the main file.

There are a couple of
considerations in setting up the main and child documents:

%%%%%%%%%%%%%%%%%%%%%%%%%%%%%%%%%%%%%%%%
\paragraph{Restrictions.}

Please note the following restrictions:
\begin{itemize}
\item
|\childdocmain| must be called with one argument \textit{main}
to ensure compatibility with earlier version of the package.
It must either be empty (|\childdocmain{}|)
or precisely match the filename of the main file in which it is specified.
See \secref{sec:detection} for further information.
\item
The filename \textit{main} must be specified without the |.tex| extension.
\item
The filename \textit{main} is case sensitive
(even in case-insensitive file systems)
due to internal string comparison.
\item
The argument \textit{main} should be fully expanded, it cannot be a macro.
\item
Subdirectories and special characters should be avoided in filenames.
\item
The command |\childdocmain{|\textit{main}|}| must be followed by a whitespace.
It should not be followed immediately by another command
or by a comment mark `|%|'.
This is because the \TeX{} parser reads the token immediately following
the argument of |\childdocmain| and puts it
at the beginning of every child section;
however, a white\-space is ignored.
\end{itemize}

%%%%%%%%%%%%%%%%%%%%%%%%%%%%%%%%%%%%%%%%
\paragraph{Content of Main File.}

It is advisable to place all content in the child files included by |\include|.
Any output contained in the main file will appear in all child documents
unless suppressed manually;
it cannot be suppressed automatically by the |\includeonly| directive
and thus should normally be avoided.
A method to include some content in the main file
by means of conditional processing is described in \secref{sec:conditional}.

%%%%%%%%%%%%%%%%%%%%%%%%%%%%%%%%%%%%%%%%
\paragraph{Page Numbering.}

When only a part of the document is compiled,
the appropriate numbering of pages
(as well as other status parameters)
is determined from the |.aux| files.
The latter contain information from previous passes.
However this information needs to propagate through
all intermediate child documents.
Therefore the page numbering in child documents may well
be inconsistent until the complete document is compiled at least once.

A useful (if unconventional) way to always ensure a consistent
page numbering is to restart the numbering in each child document
and denote the pages by `\textit{child}|.|\textit{page}'
where \textit{child} represents the chapter/section number of the child file.
This can be achieved by the command
|\numberwithin{page}{|\textit{child}|}|
of the \textsf{amsmath} package
where \textit{child} can be |chapter| or |section|
depending on the chosen structuring.
Alternatively, one can modify the macro |\thepage| appropriately
and reset the counter |page| at the start of each child file.

%%%%%%%%%%%%%%%%%%%%%%%%%%%%%%%%%%%%%%%%%%%%%%%%%%%%%%%%%%%%%%%%%%%%%%%%%%%%%%%%
\subsection{Conditional Processing}
\label{sec:conditional}

The package provides a mechanism to compile different versions
of a document. To customise the versions further some conditional processing
can come in handy to distinguish which version is being compiled.
The package provides two macros to describe the compilation context:

%%%%%%%%%%%%%%%%%%%%%%%%%%%%%%%%%%%%%%%%
\DescribeMacro{\ifchilddoc}
The conditional |\ifchilddoc| distinguishes between the compilation of
child documents and the main document:
%
\begin{center}
|\ifchilddoc |\textit{child-code}| |[|\||else |\textit{main-code}]| \||fi|
\end{center}

%%%%%%%%%%%%%%%%%%%%%%%%%%%%%%%%%%%%%%%%
\DescribeMacro{\childdocname}
\DescribeMacro{\childdocjob}
The macro |\childdocname| contains the filename (without extension)
of the main or child file being processed.
Note that |\childdocjob| will always contain the name of the main file.

%%%%%%%%%%%%%%%%%%%%%%%%%%%%%%%%%%%%%%%%
\paragraph{Title Page.}

Conditional processing can be used to include a title or banner page
in the main document when proper precautions are taken.
Importantly, the code in the main file should ensure that the page counter
(as well as other status parameters which are stored in the |.aux| files)
takes the same value after the conditional processing.
Otherwise the page numbers may take divergent values
depending on which part is compiled.

For example, a title page could be declared by:
%
\begin{center}
\begin{tabular}{l}
|\ifchilddoc\||else|\\
|\addtocounter{page}{-1}|\\
\textit{code for title page}\\
|\newpage|\\
|\||fi|
\end{tabular}
\end{center}
%
A banner page for the child documents can be generated by:
%
\begin{center}
\begin{tabular}{l}
|\ifchilddoc|\\
|\addtocounter{page}{-1}|\\
\textit{code for banner page}\\
|\newpage|\\
|\||fi|
\end{tabular}
\end{center}
%
Here one could write a message such as:
\begin{center}
|This is the part \childdocname{} of \childdocjob{}.|
\end{center}

%%%%%%%%%%%%%%%%%%%%%%%%%%%%%%%%%%%%%%%%%%%%%%%%%%%%%%%%%%%%%%%%%%%%%%%%%%%%%%%%
\subsection{Flags}
\label{sec:flags}

The package makes it easy to generate different versions
of the main or child documents.
To this end compilation flags can be defined
and assigned different default values.
They will be particularly useful in conjunction
with the forwarding mechanism described in \secref{sec:forward}.

For example, it may be useful to have a flag |\version|
which can be set to |draft| or |final|.
The document source will contain some conditional code
depending on the value of |\version|.
Suppose further, the flag should default to |final| for the main file
and to |draft| for child files
which is a natural assignment for editing the document.
This is achieved by placing the following code
in the preamble of the main document
(below the |\childdocmain| directive):
%
\begin{center}
\begin{tabular}{l}
|\ifchilddoc|\\
|\providecommand{\version}{draft}|\\
|\||else|\\
|\providecommand{\version}{final}|\\
|\||fi|
\end{tabular}
\end{center}
%
The definition by |\providecommand| makes sure
that previous definitions are not overwritten.
Further statements |\providecommand{\version}{...}|
can thus be added before the above code to override it.

For the main file, one might add a line
(between |\childdocmain| and the above block)
%
\begin{center}
|%\ifchilddoc\||else\providecommand{\version}{draft}\||fi|
\end{center}
%
which can be uncommented to produce a draft version.
Likewise one can add a line to the very top of a child file
(above the |\childdocof{|\textit{main}|}| directive)
%
\begin{center}
|%\providecommand{\version}{final}|
\end{center}
%
which can be uncommented to produce the final version of this child document.

%%%%%%%%%%%%%%%%%%%%%%%%%%%%%%%%%%%%%%%%%%%%%%%%%%%%%%%%%%%%%%%%%%%%%%%%%%%%%%%%
\subsection{Forwarding}
\label{sec:forward}

Different versions of the main or child documents
using compilation flags as described in \secref{sec:flags}
can be (permanently) stored in different files
for convenient compilation, viewing and distribution.
To this end, the package defines a command
to pass on compilation to a different file:

%%%%%%%%%%%%%%%%%%%%%%%%%%%%%%%%%%%%%%%%
\DescribeMacro{\childdocforward}
The command |\childdocforward| redirects processing to
another source file:
%
\begin{center}
\begin{tabular}{l}
|\input{childdoc.def}|\\
|\childdocforward[|\textit{main}|]{|\textit{dest}|}|\\
\end{tabular}
\end{center}
%
The argument \textit{dest} is the destination file
(without extension).
It should be the main file or one of the child files.
Note that further \textsf{childdoc} directives
such as |\childdocof| and |\childdocforward|
in the indicated file will be processed in this form.
The optional argument \textit{main}
passes on directly to the main file \textit{main}
while pretending to compile the child \textit{dest}.
This form behaves as if \textit{dest}
issues |\childdocof{|\textit{main}|}| right away,
and no further \textsf{childdoc} directives will be processed.

%%%%%%%%%%%%%%%%%%%%%%%%%%%%%%%%%%%%%%%%
\DescribeMacro{\...prefix}
In the alternative form |\childdocforwardprefix|,
%
\begin{center}
\begin{tabular}{l}
|\input{childdoc.def}|\\
|\childdocforwardprefix[|\textit{main}|]{|\textit{prefix}|}{|\textit{dest}|}|
\end{tabular}
\end{center}
%
the destination file is determined by a pattern
depending on the current file:
To make this work, the current file must be called
`{\textit{prefix}\hspace{0.2em}\textit{suffix}}'
with \textit{prefix} matching precisely the argument.
Processing is then passed on to the file
`{\textit{dest}\hspace{0.2em}\textit{suffix}}'.
Surely, the same effect is achieved by
directly specifying the
argument `{\textit{dest}\hspace{0.2em}\textit{suffix}}'
in the first form.
However, that requires to set up a different file
for each child. With the alternative form of the command
all these files can have exactly the same content
which simplifies setting them up and maintaining them.

For example, the following file |draft.tex|
with a compilation flag |\version| as described in \secref{sec:flags}
compiles the main document as a draft:
%
\begin{center}
\begin{tabular}{l}
|\def\version{draft}|\\
|\input{childdoc.def}|\\
|\childdocforward{|\textit{main}|}|
\end{tabular}
\end{center}
%
Likewise, the following files |final|\textit{nn}|.tex|
compile the final version of the child document
|child|\textit{nn}|.tex|:
%
\begin{center}
\begin{tabular}{l}
|\def\version{final}|\\
|\input{childdoc.def}|\\
|\childdocforwardprefix{final}{child}|
\end{tabular}
\end{center}
%

Note that when several versions of a main file and/or of each child file
are to be generated, it may be convenient to set up a |Makefile| or
shell script to automatise the process.

%%%%%%%%%%%%%%%%%%%%%%%%%%%%%%%%%%%%%%%%%%%%%%%%%%%%%%%%%%%%%%%%%%%%%%%%%%%%%%%%
\subsection{Command Line Processing}
\label{sec:commandline}

The effect of redirection files can also be achieved by invoking
the \LaTeX{} compiler with a more elaborate command line.
Most conveniently this should be done as part
of a shell script or a |Makefile|.

When using \textsf{childdoc} in the main file, the following
command lines effectively perform a redirection
(note that depending on the shell being used,
backslashes may have to be doubled: `|\|' $\to$ `|\\|'):
%
\begin{center}
|... -jobname "|\textit{target}|" |\\|"|[\textit{flags}]%
|\input{childdoc.def}\childdocforward[|\textit{main}|]{|\textit{dest}|}"|
\end{center}
%
Here \textit{target} is the name of the output file,
\textit{main} is the name of the main file
and \textit{dest} is the name of the main or child file to be processed
(all filenames without extensions).
The optional argument \textit{main} can be omitted
if \textit{main} matches \textit{dest}.
Optionally, compilation \textit{flags} can be defined via |\def| commands.
This command line makes the \TeX{} engine believe
it is compiling the file \textit{target}
whose content is specified as the latter parameter.
The provided code then forwards the processing to
\textit{main} or \textit{dest} as described in \secref{sec:forward}.

%%%%%%%%%%%%%%%%%%%%%%%%%%%%%%%%%%%%%%%%%%%%%%%%%%%%%%%%%%%%%%%%%%%%%%%%%%%%%%%%
\subsection{Include by Input}
\label{sec:input}

Including child documents by |\include| has some restrictions by design.
Most notably, the content of a child document always occupies
its own set of pages; pages cannot be shared between child documents.
Usually, this behaviour makes perfect sense
because each child document contain an essential part of the document.
However, in some situations it may be desirable to compose
a document from a collection of parts
without having mandatory page breaks between then.
For this case, the package
provides a mechanism to include parts
by |\input| which can also be processed individually.
However, by construction this mechanism
requires manual handling of the content to be output.

%%%%%%%%%%%%%%%%%%%%%%%%%%%%%%%%%%%%%%%%
\DescribeMacro{\ifchilddocmanual}
The main file should be prepared as usual, see \secref{sec:include}.
However, the document body must make a distinction
between processing of an individual part and of the main document, e.g.:
%
\begin{center}
\begin{tabular}{l}
|\ifchilddocmanual|\\
|\input{\childdocname}|\\
|\||else|\\
\textit{document body with }|\input{|\textit{part}|}|\\
|\||fi|
\end{tabular}
\end{center}
%
The conditional |\ifchilddocmanual| is true whenever
a part to be included by |\input| is being compiled,
and the name of the part is stored in |\childdocname|.

%%%%%%%%%%%%%%%%%%%%%%%%%%%%%%%%%%%%%%%%
\DescribeMacro{\childdocby}
Each part to be included by |\input| should start with:
%
\begin{center}
\begin{tabular}{l}
|\input{childdoc.def}|\\
|\childdocby{|\textit{main}|}|\\
\end{tabular}
\end{center}
%
The directive |\childdocby| is similar to |\childdocof|
described in \secref{sec:include},
but the subsequent selection of content must be done manually.
To that end, both |\ifchilddoc| and |\ifchilddocmanual|
will be true upon processing of a part,
and the name of the part is stored in |\childdocname|.
Note that |\jobname| will be set to the filename of the current part
so that each part receives an individual |.aux| file
that does not interfere with the |.aux| file(s) of the main document.
This behaviour can be altered by the alternative form
|\childdocby[*]{|\textit{main}|}| (with a non-empty optional argument)
which uses the |.aux| file of the main document
by setting |\jobname| to \textit{main}.

%%%%%%%%%%%%%%%%%%%%%%%%%%%%%%%%%%%%%%%%%%%%%%%%%%%%%%%%%%%%%%%%%%%%%%%%%%%%%%%%
\subsection{Driver Development}
\label{sec:driver}

The \textsf{childdoc} mechanism can also be use for the development
of definition files such as \LaTeX{} styles or classes.
This case differs from the above setup with multiple parts
included by |\include| in that no |\includeonly| should be invoked.
This can be achieved by starting the include file
(before |\ProvidesPackage|) with:
%
\begin{center}
\begin{tabular}{l}
|\input{childdoc.def}|\\
|\childdocforward{|\textit{main}|}|\\
\end{tabular}
\end{center}
%
or alternatively with:
%
\begin{center}
\begin{tabular}{l}
|\input{childdoc.def}|\\
|\childdocby{|\textit{main}|}|\\
\end{tabular}
\end{center}
%
Both forms have slightly different effects as described above.
The main file is prepared as usual, see \secref{sec:include}.

%%%%%%%%%%%%%%%%%%%%%%%%%%%%%%%%%%%%%%%%%%%%%%%%%%%%%%%%%%%%%%%%%%%%%%%%%%%%%%%%
\subsection{Legacy Detection}
\label{sec:detection}

The directive |\childdocmain| in the main file can detect
whether the complete document or merely a child is to be compiled
even without using the directive |\childdocof|.
This method is deprecated because it is less robust
and there is no compelling reason to use it;
it is merely provided for backward compatibility
and it may be removed in future versions.

If the detection mechanism is to be used,
it is mandatory to correctly specify
the filename of the main file as the argument of |\childdocmain|:
%
\begin{center}
\begin{tabular}{l}
|\input{childdoc.def}|\\
|\childdocmain{|\textit{main}|}|\\
\end{tabular}
\end{center}
%
If |\jobname| does not match the argument \textit{main} of |\childdocmain|,
it is assumed that |\jobname| points to the child file to be compiled.
When using |\childdocmain| with the main file specified as argument,
it suffices to start a child file
with just |\input{|\textit{main}|}|
without loading of the package and using |\childdocof|.
If instead all processing is done
with the appropriate \textsf{childdoc} directives,
the argument of \textit{main} of |\childdocmain| can be empty.

An alternative version of the command line processing described
in \secref{sec:commandline} using the detection mechanism reads:
%
\begin{center}
|... -jobname "|\textit{target}|" "|[\textit{flags}]%
[|\def\jobname{|\textit{dest}|}|]|\input{|\textit{main}|}"|
\end{center}

%%%%%%%%%%%%%%%%%%%%%%%%%%%%%%%%%%%%%%%%%%%%%%%%%%%%%%%%%%%%%%%%%%%%%%%%%%%%%%%%
\subsection{Manual Code}
\label{sec:manual}

In case one cannot be certain whether the definitions file |childdoc.def|
is installed on the target \TeX{} distribution
and one prefers not to ship it,
it is conceivable to paste a few relevant commands into the sources.

To that end, drop all statements |\input{childdoc.def}|
and perform the replacements as outlined below.
Instead of |\childdocmain{|\textit{main}|}| add the following code
to the top of the main file:
%
\begin{center}
\begin{tabular}{l}
|\||ifdefined\childdocname\endinput\||fi\newif\ifchilddoc|\\
|\edef\childdocname{\scantokens\expandafter{\jobname\noexpand}}|\\
|\def\childdocmain{|\textit{main}|}\||ifx\childdocmain\childdocname\||else|\\
|\childdoctrue\includeonly{\childdocname}\let\jobname\childdocmain\||fi|\\
\end{tabular}
\end{center}
%
Instead of |\childdocof{|\textit{main}|}| just include the main file
at the top of each child file:
%
\begin{center}
|\input{|\textit{main}|}|
\end{center}
%
A simple redirection |\childdocforward{|\textit{dest}|}| is achieved by:
%
\begin{center}
|\def\jobname{|\textit{dest}|}\input{\jobname}|
\end{center}
%
The redirection with prefix
|\childdocforwardprefix[|\textit{prefix}|]{|\textit{dest}|}|
is accomplished by:
%
\begin{center}
\begin{tabular}{l}
|{\edef\jobname{\scantokens\expandafter{\jobname\noexpand}}|\\
|\def\redirectjob |\textit{prefix}|#1~~~{\gdef\jobname{|\textit{dest}|#1}}|\\
|\expandafter\redirectjob\jobname~~~}\input{\jobname}|
\end{tabular}
\end{center}

In an alternative approach,
child documents can be compiled by a specific command line
without additional code or specific definitions:
%
\begin{center}
|... -jobname "|\textit{target}|" "|[\textit{flags}]%
|\includeonly{|\textit{dest}|}\input{|\textit{main}|}"|
\end{center}
%

%%%%%%%%%%%%%%%%%%%%%%%%%%%%%%%%%%%%%%%%%%%%%%%%%%%%%%%%%%%%%%%%%%%%%%%%%%%%%%%%
%%%%%%%%%%%%%%%%%%%%%%%%%%%%%%%%%%%%%%%%%%%%%%%%%%%%%%%%%%%%%%%%%%%%%%%%%%%%%%%%
\section{Information}

%%%%%%%%%%%%%%%%%%%%%%%%%%%%%%%%%%%%%%%%%%%%%%%%%%%%%%%%%%%%%%%%%%%%%%%%%%%%%%%%
\subsection{Copyright}

Copyright \copyright{} 2017--2018 Niklas Beisert

This work may be distributed and/or modified under the
conditions of the \LaTeX{} Project Public License, either version 1.3
of this license or (at your option) any later version.
The latest version of this license is in
  \url{http://www.latex-project.org/lppl.txt}
and version 1.3 or later is part of all distributions of \LaTeX{}
version 2005/12/01 or later.

This work has the LPPL maintenance status `maintained'.

The Current Maintainer of this work is Niklas Beisert.

This work consists of the files |README.txt|, |childdoc.ins| and |childdoc.dtx|
as well as the derived files |childdoc.def|, |cdocsamp.tex|
with |cdocsch1.tex|, |cdocsch2.tex|, |cdocspt3.tex|, |cdocspt4.tex|,
|cdocsdrf.tex|, |cdocsfn1.tex|, |cdocsfn2.tex|
as well as |childdoc.pdf|.

%%%%%%%%%%%%%%%%%%%%%%%%%%%%%%%%%%%%%%%%%%%%%%%%%%%%%%%%%%%%%%%%%%%%%%%%%%%%%%%%
\subsection{Files and Installation}

The package consists of the files:
%
\begin{center}
\begin{tabular}{ll}
    |README.txt|   & readme file \\
    |childdoc.ins| & installation file \\
    |childdoc.dtx| & source file \\
    |childdoc.def| & definition file \\
    |cdocsamp.tex| & sample main file \\
    |cdocsch1.tex| & sample include file \\
    |cdocsch2.tex| & sample include file \\
    |cdocspt3.tex| & sample part file \\
    |cdocspt4.tex| & sample part file \\
    |cdocsdrf.tex| & sample redirection file \\
    |cdocsfn1.tex| & sample redirection file \\
    |cdocsfn2.tex| & sample redirection file \\
    |childdoc.pdf| & manual
\end{tabular}
\end{center}
%
The distribution consists of the files
|README.txt|, |childdoc.ins| and |childdoc.dtx|.
%
\begin{itemize}
\item
Run (pdf)\LaTeX{} on |childdoc.dtx|
to compile the manual |childdoc.pdf| (this file).
\item
Run \LaTeX{} on |childdoc.ins| to create the definitions file |childdoc.def|
and the sample |cdocsamp.tex| with include files
|cdocsch1.tex|, |cdocsch2.tex|, |cdocspt3.tex|, |cdocspt4.tex|,
|cdocsdrf.tex|, |cdocsfn1.tex|, |cdocsfn2.tex|.
Then copy the file |childdoc.def| to an appropriate directory of your \LaTeX{}
distribution, e.g.\ \textit{texmf-root}|/tex/latex/childdoc|.
\end{itemize}

%%%%%%%%%%%%%%%%%%%%%%%%%%%%%%%%%%%%%%%%%%%%%%%%%%%%%%%%%%%%%%%%%%%%%%%%%%%%%%%%
\subsection{Related CTAN Packages}

There are several other packages which offer a similar functionality:
%
\begin{itemize}
\item
The packages
\href{http://ctan.org/pkg/docmute}{\textsf{docmute}},
\href{http://ctan.org/pkg/includex}{\textsf{includex}} and
\href{http://ctan.org/pkg/standalone}{\textsf{standalone}}
provide commands to include only the document body of
a child file thus allowing both files to be compiled individually.
\item
The packages \href{http://ctan.org/pkg/subdocs}{\textsf{subdocs}}
and \href{http://ctan.org/pkg/subfiles}{\textsf{subfiles}}
provide structures in which the main and child documents can be
encapsulated and allowing them to be compiled individually.
The inclusion mechanism is different from the conventional |\include|.
\item
The package \href{http://ctan.org/pkg/combine}{\textsf{combine}}
is an elaborate solution to combine several documents into one.
\end{itemize}
%
See also the CTAN topic \href{http://ctan.org/topic/subdocs}{\textsf{subdocs}}
for further related packages.
The present package differs from the above solutions in that
a document structure constructed with the conventional |\include| mechanism
just needs two extra commands at the top of every file
such that all constituent files can be compiled individually.

%%%%%%%%%%%%%%%%%%%%%%%%%%%%%%%%%%%%%%%%%%%%%%%%%%%%%%%%%%%%%%%%%%%%%%%%%%%%%%%%
%\subsection{Feature Suggestions}
%
%The following is a list of features which may be useful for future
%versions of this package:
%%
%\begin{itemize}
%\item
%\ldots
%\end{itemize}

%%%%%%%%%%%%%%%%%%%%%%%%%%%%%%%%%%%%%%%%%%%%%%%%%%%%%%%%%%%%%%%%%%%%%%%%%%%%%%%%
\subsection{Revision History}

%%%%%%%%%%%%%%%%%%%%%%%%%%%%%%%%%%%%%%%%
\paragraph{v2.0:} 2018/12/30

\begin{itemize}
\item
immediate forward processing
\item
added |\childdocby| mechanism
\item
manual restructured
\end{itemize}

%%%%%%%%%%%%%%%%%%%%%%%%%%%%%%%%%%%%%%%%
\paragraph{v1.6:} 2018/01/17

\begin{itemize}
\item
application for development of include files
\item
corrections to manual
\end{itemize}

%%%%%%%%%%%%%%%%%%%%%%%%%%%%%%%%%%%%%%%%
\paragraph{v1.5:} 2017/05/21

\begin{itemize}
\item
more complete structuring introduced
\item
|\childdocof| introduced
\item
|\childdoc| renamed to |\childdocmain|
\item
|\childredirect| renamed to |\childdocforward| and |\childdocforwardprefix|
and functionality expanded
\end{itemize}

%%%%%%%%%%%%%%%%%%%%%%%%%%%%%%%%%%%%%%%%
\paragraph{v1.0:} 2017/04/27

\begin{itemize}
\item
manual and install package
\item
first version published on CTAN
\end{itemize}

%%%%%%%%%%%%%%%%%%%%%%%%%%%%%%%%%%%%%%%%
\paragraph{v0.6:} 2017/04/26

\begin{itemize}
\item
redirection mechanism added
\end{itemize}

%%%%%%%%%%%%%%%%%%%%%%%%%%%%%%%%%%%%%%%%
\paragraph{v0.5:} 2017/04/26

\begin{itemize}
\item
functionality in definition file
\end{itemize}


%%%%%%%%%%%%%%%%%%%%%%%%%%%%%%%%%%%%%%%%%%%%%%%%%%%%%%%%%%%%%%%%%%%%%%%%%%%%%%%%
%%%%%%%%%%%%%%%%%%%%%%%%%%%%%%%%%%%%%%%%%%%%%%%%%%%%%%%%%%%%%%%%%%%%%%%%%%%%%%%%
%%%%%%%%%%%%%%%%%%%%%%%%%%%%%%%%%%%%%%%%%%%%%%%%%%%%%%%%%%%%%%%%%%%%%%%%%%%%%%%%
\appendix

\settowidth\MacroIndent{\rmfamily\scriptsize 000\ }

 \DocInput{childdoc.dtx}

\end{document}
%</driver>
% \fi
%
% %%%%%%%%%%%%%%%%%%%%%%%%%%%%%%%%%%%%%%%%%%%%%%%%%%%%%%%%%%%%%%%%%%%%%%%%%%%%%%
% %%%%%%%%%%%%%%%%%%%%%%%%%%%%%%%%%%%%%%%%%%%%%%%%%%%%%%%%%%%%%%%%%%%%%%%%%%%%%%
% \section{Sample}
%\iffalse
%<*samplemain>
%\fi
%
% The following presents a sample document
% with two chapters, two parts, a title page,
% a compile flag as well as three forwarding files to set the flag.
% It consists of eight |.tex| files:
% \begin{center}
% \begin{tabular}{ll}
% |cdocsamp.tex|&main file\\
% |cdocsch1.tex|&include file for chapter 1\\
% |cdocsch2.tex|&include file for chapter 2\\
% |cdocspt3.tex|&include file for part 3\\
% |cdocspt4.tex|&include file for part 4\\
% |cdocsdrf.tex|&forwarding file for main file in draft mode\\
% |cdocsfi1.tex|&forwarding file for final version of chapter 1\\
% |cdocsfi2.tex|&forwarding file for final version of chapter 2\\
% \end{tabular}
% \end{center}
% Each of the eight files can be compiled directly by the \LaTeX{} compiler.
%
% %%%%%%%%%%%%%%%%%%%%%%%%%%%%%%%%%%%%%%
% \paragraph{Main File.}
%
% The main file is called |cdocsamp.tex|.
%
% Load the \textsf{childdoc} definitions and
% declare the filename for the main document:
%    \begin{macrocode}
\input{childdoc.def}
\childdocmain{}
%    \end{macrocode}

% Optional override for |\version| flag:
%    \begin{macrocode}
%%\ifchilddoc\else\providecommand{\version}{draft}\fi
%    \end{macrocode}

% Define the default values for the |\version| flag
% (|final| for the main file and |draft| for childs):
%    \begin{macrocode}
\ifchilddoc
\providecommand{\version}{draft}
\else
\providecommand{\version}{final}
\fi
%    \end{macrocode}

% Load the standard document class:
%    \begin{macrocode}
\documentclass[12pt]{article}
%    \end{macrocode}

% Start the document body:
%    \begin{macrocode}
\begin{document}
%    \end{macrocode}

% Declare a title page.
% Print title, part of document being processed and version flag:
%    \begin{macrocode}
\addtocounter{page}{-1}
\begin{center}
{\LARGE\bfseries{}childdoc example\par}
\vspace{1cm}
\ifchilddoc
\ifchilddocmanual part\else chapter\fi:
`\childdocname' of `\childdocjob'\par
\else
main document: `\childdocjob'\par
\fi
version: \version\par
\end{center}
\newpage
%    \end{macrocode}

% Manually include selected file,
% otherwise process as usual:
%    \begin{macrocode}
\ifchilddocmanual
\section*{part `\childdocname'}
\input{\childdocname}
\else
%    \end{macrocode}

% Include the two chapters:
%    \begin{macrocode}
\include{cdocsch1}
\include{cdocsch2}
%    \end{macrocode}

% Include the two parts unless only chapters should be displayed:
%    \begin{macrocode}
\ifchilddoc\else
\section{part three}
\input{cdocspt3}
\section{part four}
\input{cdocspt4}
\fi
%    \end{macrocode}

% Process as usual until here:
%    \begin{macrocode}
\fi
%    \end{macrocode}

% End of document body:
%    \begin{macrocode}
\end{document}
%    \end{macrocode}
%\iffalse
%</samplemain>
%\fi
%
% %%%%%%%%%%%%%%%%%%%%%%%%%%%%%%%%%%%%%%
% \paragraph{Chapter Include Files.}
%
% The include files are called |cdocsch1.tex| and |cdocsch2.tex|.
%
%\iffalse
%<*samplechap1|samplechap2>
%\fi

% Optional override for |\version| flag:
%    \begin{macrocode}
%%\providecommand{\version}{final}
%    \end{macrocode}

% Include the main document:
%    \begin{macrocode}
\input{childdoc.def}
\childdocof{cdocsamp}
%    \end{macrocode}

%\iffalse
%</samplechap1|samplechap2>
%\fi
%
%\iffalse
%<*samplechap1>
%\fi
% Some text for chapter 1:
%    \begin{macrocode}
\section{one}
some text in chapter one
%    \end{macrocode}

%\iffalse
%</samplechap1>
%\fi
% Some text for chapter 2:
%\iffalse
%<*samplechap2>
%\fi
%    \begin{macrocode}
\section{two}
more text in chapter two
%    \end{macrocode}

%\iffalse
%</samplechap2>
%\fi
%
% %%%%%%%%%%%%%%%%%%%%%%%%%%%%%%%%%%%%%%
% \paragraph{Part Include Files.}
%
% The include files are called |cdocspt3.tex| and |cdocspt4.tex|.
%
%\iffalse
%<*samplepart3|samplepart4>
%\fi

% Optional override for |\version| flag:
%    \begin{macrocode}
%%\providecommand{\version}{final}
%    \end{macrocode}

% Include the main document:
%    \begin{macrocode}
\input{childdoc.def}
\childdocby{cdocsamp}
%    \end{macrocode}

%\iffalse
%</samplepart3|samplepart4>
%\fi
%
%\iffalse
%<*samplepart3>
%\fi
% Some text for part 3:
%    \begin{macrocode}
some text in part three
%    \end{macrocode}

%\iffalse
%</samplepart3>
%\fi
% Some text for part 4:
%\iffalse
%<*samplepart4>
%\fi
%    \begin{macrocode}
more text in part four
%    \end{macrocode}

%\iffalse
%</samplepart4>
%\fi
%
% %%%%%%%%%%%%%%%%%%%%%%%%%%%%%%%%%%%%%%
% \paragraph{Forwarding for a Complete Draft.}
%
% The following forwarding file |cdocsdrf.tex|
% compiles the main document in draft mode:
%\iffalse
%<*sampledraft>
%\fi
%    \begin{macrocode}
\def\version{draft}
\input{childdoc.def}
\childdocforward{cdocsamp}
%    \end{macrocode}

%\iffalse
%</sampledraft>
%\fi
%
% %%%%%%%%%%%%%%%%%%%%%%%%%%%%%%%%%%%%%%
% \paragraph{Forwarding for Final Version of the Chapters.}
%
% The following forwarding files |cdocsfn1.tex| and |cdocsfn2.tex|
% (with identical content)
% compile the final versions of the child documents
% |cdocsch1.tex| and |cdocsch2.tex|, respectively:
%\iffalse
%<*samplefinal>
%\fi
%    \begin{macrocode}
\def\version{final}
\input{childdoc.def}
\childdocforwardprefix[cdocsamp]{cdocsfn}{cdocsch}
%    \end{macrocode}

%\iffalse
%</samplefinal>
%\fi
%
% %%%%%%%%%%%%%%%%%%%%%%%%%%%%%%%%%%%%%%
% \paragraph{Command Line Processing.}
%
% The following three command lines generate the output files
% |cdocscld|, |cdocscl1| and |cdocscl2|
% which should be identical to
% |cdocsdrf|, |cdocsch1| and |cdocsfn2|, respectively:
% \begin{center}
% \begin{tabular}{l}
% |latex -jobname cdocscld \|\\
% |  "\def\version{draft}\input{childdoc.def}\childdocforward{cdocsamp}"|\\
% |latex -jobname cdocscl1 \|\\
% |  "\input{childdoc.def}\childdocforward[cdocsamp]{cdocsch1}"|\\
% |latex -jobname cdocscl2 \|\\
% |  "\def\version{final}\input{childdoc.def}\childdocforward{cdocsch2}"|
% \end{tabular}
% \end{center}
% Note that the trailing backslash on each first line
% merely continues the input to the second line
% (for convenient cut ant paste).
% Furthermore, the command |latex| can be replaced by any
% of its alternative versions such as |pdflatex|.
%
% %%%%%%%%%%%%%%%%%%%%%%%%%%%%%%%%%%%%%%%%%%%%%%%%%%%%%%%%%%%%%%%%%%%%%%%%%%%%%%
% %%%%%%%%%%%%%%%%%%%%%%%%%%%%%%%%%%%%%%%%%%%%%%%%%%%%%%%%%%%%%%%%%%%%%%%%%%%%%%
% \section{Implementation}
%\iffalse
%<*package>
%\fi
%
% This section describes the definitions file |childdoc.def|.

% The definitions cannot be loaded using |\usepackage| or |\RequirePackage|
% which has a mechanism to prevent loading a style file more than once.
% When loading the definitions by means of |\input|
% multiple instances have to be prevented manually:
%\iffalse
%This code needs to be before the `\ProvidesFile' directive
%which is defined at the beginning of this file.
%Therefore it is also placed there and commented out here.
%</package>
%<*discard>
%\fi
%    \begin{macrocode}
\ifdefined\childdocmain\endinput\fi
%    \end{macrocode}
%\iffalse
%</discard>
%<*package>
%\fi
%
% \macro{\ifchilddoc}
% \macro{\ifchilddocmanual}
% The conditional |\ifchilddoc| tells whether a
% child (true) or main (false) document is being compiled.
% The conditional |\ifchilddocmanual| tells whether
% the |\includeonly| mechanism is used (false) or
% the selection of child files must be performed manually (true).
% The definitions initialise to false:
%    \begin{macrocode}
\newif\ifchilddoc
\newif\ifchilddocmanual
%    \end{macrocode}

% \macro{\childdocname}
% \macro{\childdocjob}
% The macro |\childdocname| stores the name of the main document
% to be compiled. The macro |\childdocjob| stores the name of
% the document on which the \LaTeX{} compiler was originally invoked.
% The content of |\jobname| cannot be compared
% to filenames specified in the source due to different catcodes.
% The following code rescans |\jobname|, stores the result
% in |\childdocname| and saves a copy in |\childdocjob|:
%    \begin{macrocode}
\edef\childdocname{\scantokens\expandafter{\jobname\noexpand}}
\let\childdocjob\childdocname
%    \end{macrocode}

% \macro{\childdocdisable}
% The macro |\childdocdisable| prevents the main file
% from being processed more than once.
% At this stage, the main document command |\childdocmain|
% is assumed to be called once again where it should do nothing.
% Any subsequent call to it should prevent
% a secondary processing of the main document
% It overwrites the forwarding commands
% |\childdocof| and |\childdocforward|
% with empty macros to prevent further inclusions of the main document:
%    \begin{macrocode}
\newcommand{\childdocdisable}
{
  \renewcommand{\childdocmain}[1]{\renewcommand{\childdocmain}[1]{\endinput}}
  \renewcommand{\childdocof}[1]{}
  \renewcommand{\childdocby}[2][]{}
  \renewcommand{\childdocforward}[2][]{}
  \renewcommand{\childdocdisable}{}
}
%    \end{macrocode}

% \macro{\childdocmain}
% The macro |\childdocmain| is to be called at the top of the main file
% with nothing or the main filename (without extension) as argument.
% First, it breaks loops.
% If the argument is not empty and does not match |\childdocname|
% (which is set by the first inclusion of |childdoc.def|),
% |\ifchilddoc| is set to true, |\includeonly| is applied to the child file
% and |\jobname| is set to the main file
% (for proper handling of |.aux| files):
%    \begin{macrocode}
\newcommand{\childdocmain}[1]
{
  \childdocdisable\childdocmain{}
  \if?#1?\else
    \begingroup
      \def\childdoctmp{#1}
      \ifx\childdoctmp\childdocname
        \def\childdoctmp{}
      \else
        \def\childdoctmp
        {
          \childdoctrue
          \includeonly{\childdocname}
          \def\childdocjob{#1}
          \def\jobname{#1}
        }
      \fi
      \expandafter
    \endgroup
    \childdoctmp
  \fi
}
%    \end{macrocode}

% \macro{\childdocof}
% The command |\childdocof| redirects
% compilation to the main file |#1|.
%    \begin{macrocode}
\newcommand{\childdocof}[1]
{
  \childdocdisable
  \childdoctrue
  \includeonly{\childdocname}
  \def\jobname{#1}
  \def\childdocjob{#1}
  \input{#1}
}
%    \end{macrocode}

% \macro{\childdocby}
% The command |\childdocby| ....
%    \begin{macrocode}
\newcommand{\childdocby}[2][]
{
  \childdocdisable
  \childdoctrue
  \childdocmanualtrue
  \if?#1?\else
    \def\jobname{#2}
  \fi
  \def\childdocjob{#2}
  \input{#2}
  \endinput
}
%    \end{macrocode}

% \macro{\childdocforward}
% The command |\childdocforward| redirects
% compilation to the main file or
% (if the optional argument is given) a child file.
% Parameters are set as if the main file
% or a child file starting with |\childdocof| was compiled.
% Then compilation is handed over to the main file:
%    \begin{macrocode}
\newcommand{\childdocforward}[2][]
{
  \begingroup
    \if?#1?
      \def\childdoctmp
      {
        \def\childdocname{#2}
        \def\childdocjob{#2}
        \def\jobname{#2}
        \input{#2}
        \endinput
      }
    \else
      \def\childdoctmp
      {
        \childdocdisable
        \def\childdocname{#2}
        \childdoctrue
        \includeonly{#2}
        \def\childdocjob{#1}
        \def\jobname{#1}
        \input{#1}
        \endinput
      }
    \fi
    \expandafter
  \endgroup
  \childdoctmp
}
%    \end{macrocode}

% \macro{\childdocforwardprefix}
% The command |\childdocforwardprefix| redirects
% compilation to the main or a child file by means of a pattern.
% The prefix |#1| in the current filename is replaced by |#2|
% and the suffix of the current filename is kept
% (it is assumed that the filename does not contain the substring `|~~~|'
% which is used as a delimiter).
% Compilation is handed over to the new file by |\childdocforward|:
%    \begin{macrocode}
\newcommand{\childdocforwardprefix}[3][]
{
  \begingroup
    \def\childdocextract #2##1~~~{\def\childdoctmp{\childdocforward[#1]{#3##1}}}
    \expandafter\childdocextract\childdocname~~~
    \expandafter
  \endgroup
  \childdoctmp
}
%    \end{macrocode}

% \macro{\childdoc}
% The deprecated macro |\childdoc| is a legacy version of |\childdocmain|:
%    \begin{macrocode}
\newcommand{\childdoc}{\childdocmain}
%    \end{macrocode}

% \macro{\childdocredirect}
% The deprecated macro |\childdocredirect| is a legacy version
% of |\childdocforward| and |\childdocforwardprefix|:
%    \begin{macrocode}
\newcommand{\childdocredirect}[2][]
{
  \begingroup
    \if?#1?
      \def\childdoctmp{\childdocforward{#2}}
    \else
      \def\childdoctmp{\childdocforwardprefix{#1}{#2}}
    \fi
    \expandafter
  \endgroup
  \childdoctmp
}
%    \end{macrocode}

%\iffalse
%</package>
%\fi
%
\endinput
\childdocforward[cdocsamp]{cdocsch1}"|\\
% |latex -jobname cdocscl2 \|\\
% |  "\def\version{final}% \iffalse
%
% childdoc.dtx Copyright (C) 2017-2018 Niklas Beisert
%
% This work may be distributed and/or modified under the
% conditions of the LaTeX Project Public License, either version 1.3
% of this license or (at your option) any later version.
% The latest version of this license is in
%   http://www.latex-project.org/lppl.txt
% and version 1.3 or later is part of all distributions of LaTeX
% version 2005/12/01 or later.
%
% This work has the LPPL maintenance status `maintained'.
%
% The Current Maintainer of this work is Niklas Beisert.
%
% This work consists of the files childdoc.dtx and childdoc.ins
% and the derived files childdoc.def and cdocsamp.tex with
% cdocsch1.tex, cdocsch2.tex, cdocsdrf.tex, cdocsfn1.tex, cdocsfn2.tex.
%
%<package>\ifdefined\childdocmain\endinput\fi
%<package>\ProvidesFile{childdoc.def}[2018/12/30 v2.0 child document driver]
%<samplemain>\ProvidesFile{cdocsamp.tex}[2018/12/30 v2.0 sample for childdoc]
%<*driver>
%\ProvidesFile{childdoc.drv}[2018/12/30 v2.0 childdoc reference manual file]
\PassOptionsToClass{10pt,a4paper}{article}
\documentclass{ltxdoc}

\usepackage[margin=35mm]{geometry}
\usepackage{hyperref}
\usepackage{hyperxmp}
\usepackage[usenames]{color}

\hypersetup{colorlinks=true}
\hypersetup{pdfstartview=FitH}
\hypersetup{pdfpagemode=UseNone}
\hypersetup{pdfsource={}}
\hypersetup{pdflang={en-UK}}
\hypersetup{pdfcopyright={Copyright 2017-2018 Niklas Beisert.
  This work may be distributed and/or modified under the
  conditions of the LaTeX Project Public License, either version 1.3
  of this license or (at your option) any later version.}}
\hypersetup{pdflicenseurl={http://www.latex-project.org/lppl.txt}}
\hypersetup{pdfcontactaddress={ETH Zurich, ITP, HIT K,
  Wolfgang-Pauli-Strasse 27}}
\hypersetup{pdfcontactpostcode={8093}}
\hypersetup{pdfcontactcity={Zurich}}
\hypersetup{pdfcontactcountry={Switzerland}}
\hypersetup{pdfcontactemail={nbeisert@itp.phys.ethz.ch}}
\hypersetup{pdfcontacturl={http://people.phys.ethz.ch/\xmptilde nbeisert/}}

\newcommand{\secref}[1]{\hyperref[#1]{section \ref*{#1}}}

\parskip1ex
\parindent0pt
\let\olditemize\itemize
\def\itemize{\olditemize\parskip0pt}

\begin{document}

\title{The \textsf{childdoc} Package}
\hypersetup{pdftitle={The childdoc Package}}
\author{Niklas Beisert\\[2ex]
  Institut f\"ur Theoretische Physik\\
  Eidgen\"ossische Technische Hochschule Z\"urich\\
  Wolfgang-Pauli-Strasse 27, 8093 Z\"urich, Switzerland\\[1ex]
  \href{mailto:nbeisert@itp.phys.ethz.ch}
  {\texttt{nbeisert@itp.phys.ethz.ch}}}
\hypersetup{pdfauthor={Niklas Beisert}}
\hypersetup{pdfsubject={Manual for the LaTeX2e Package childdoc}}
\date{30 December 2018, \textsf{v2.0}}
\maketitle

\begin{abstract}\noindent
\textsf{childdoc} is a \LaTeXe{} package
that enables the direct compilation
of document sections included by |\include|
to individual files.
\end{abstract}

\begingroup
\parskip0ex
\tableofcontents
\endgroup

%%%%%%%%%%%%%%%%%%%%%%%%%%%%%%%%%%%%%%%%%%%%%%%%%%%%%%%%%%%%%%%%%%%%%%%%%%%%%%%%
%%%%%%%%%%%%%%%%%%%%%%%%%%%%%%%%%%%%%%%%%%%%%%%%%%%%%%%%%%%%%%%%%%%%%%%%%%%%%%%%
\section{Introduction}

\LaTeX{} provides a mechanism to structure a large document (such as a book)
into a main file and several child files (containing the chapters)
using the |\include| command.
This mechanism is beneficial for documents
which span hundreds of pages in order to
make the source file(s) more manageable.
Moreover, compilation can be restricted to
selected child files by means of the |\includeonly| command.
The latter feature can be used to reduce the compilation time while editing
(this was significantly more useful in the earlier days of \LaTeX{})
or to generate a smaller document which is easier to navigate.
Another application of |\includeonly| is to generate
documents consisting of selected parts of the complete document.

However, there are a few drawbacks of the plain |\include| mechanism:
\begin{itemize}
\item
The child files cannot be compiled on their own,
they can only be compiled via the main file.
A naive editing environment
(such as a text editor with an option
to have the current file processed by \LaTeX)
may require one to switch to the main file before compiling;
attempting to compile the child file produces errors.
\item
The main file must be modified (each time)
to adjust the |\includeonly| command
to the present needs. This easily leaves the main file in a messy state.
\item
The generated document will always carry the filename
of the main document. This is inconvenient if
several child files are to be compiled and
to be kept for distribution.
\end{itemize}

The present package provides a simple interface
to make child files individually compilable by \LaTeX{}.
Compiling a child file then has the same effect as compiling
the main file with an |\includeonly| command
to select the appropriate child.
Moreover the generated document will carry the name of the child
rather than the main file.
This resolves all three above issues.

This feature is meant to make the editing of books,
thesis documents and lecture notes somewhat more convenient.
However, the package can also be used efficiently for
composing a series of documents (such as exercise sheets)
which are typically distributed individually.
It then assists the author in generating the individual documents
(potentially in different versions)
as well as a document containing the collected series.
Another application is in developing style files
or other kinds of included material
where compilation of the style file could redirect
to a sample or test file.

%%%%%%%%%%%%%%%%%%%%%%%%%%%%%%%%%%%%%%%%%%%%%%%%%%%%%%%%%%%%%%%%%%%%%%%%%%%%%%%%
%%%%%%%%%%%%%%%%%%%%%%%%%%%%%%%%%%%%%%%%%%%%%%%%%%%%%%%%%%%%%%%%%%%%%%%%%%%%%%%%
\section{Usage}

First of all, the package \textsf{childdoc} is \emph{not} a standard
\LaTeXe{} |.sty| style file! Therefore it needs to be invoked in
a non-standard way.

%%%%%%%%%%%%%%%%%%%%%%%%%%%%%%%%%%%%%%%%%%%%%%%%%%%%%%%%%%%%%%%%%%%%%%%%%%%%%%%%
\subsection{Included Files}
\label{sec:include}

%%%%%%%%%%%%%%%%%%%%%%%%%%%%%%%%%%%%%%%%
\DescribeMacro{\childdocmain}
To use the package, add the commands
\begin{center}
\begin{tabular}{l}
|\input{childdoc.def}|\\
|\childdocmain{}|\\
\end{tabular}
\end{center}
at the very top of the main \LaTeX{} file,
in particular \emph{before} the |\documentclass| statement!
The argument of |\childdocmain| should be left empty
(but it must be present).

%%%%%%%%%%%%%%%%%%%%%%%%%%%%%%%%%%%%%%%%
\DescribeMacro{\childdocof}
Furthermore, add the commands
\begin{center}
\begin{tabular}{l}
|\input{childdoc.def}|\\
|\childdocof{|\textit{main}|}|\\
\end{tabular}
\end{center}
at the top of every child file \textit{child}
which is included by |\include{|\textit{child}|}|
from within the main file
(or at least for those files to be compiled individually).
The argument \textit{main} must be the filename of the main file.

There are a couple of
considerations in setting up the main and child documents:

%%%%%%%%%%%%%%%%%%%%%%%%%%%%%%%%%%%%%%%%
\paragraph{Restrictions.}

Please note the following restrictions:
\begin{itemize}
\item
|\childdocmain| must be called with one argument \textit{main}
to ensure compatibility with earlier version of the package.
It must either be empty (|\childdocmain{}|)
or precisely match the filename of the main file in which it is specified.
See \secref{sec:detection} for further information.
\item
The filename \textit{main} must be specified without the |.tex| extension.
\item
The filename \textit{main} is case sensitive
(even in case-insensitive file systems)
due to internal string comparison.
\item
The argument \textit{main} should be fully expanded, it cannot be a macro.
\item
Subdirectories and special characters should be avoided in filenames.
\item
The command |\childdocmain{|\textit{main}|}| must be followed by a whitespace.
It should not be followed immediately by another command
or by a comment mark `|%|'.
This is because the \TeX{} parser reads the token immediately following
the argument of |\childdocmain| and puts it
at the beginning of every child section;
however, a white\-space is ignored.
\end{itemize}

%%%%%%%%%%%%%%%%%%%%%%%%%%%%%%%%%%%%%%%%
\paragraph{Content of Main File.}

It is advisable to place all content in the child files included by |\include|.
Any output contained in the main file will appear in all child documents
unless suppressed manually;
it cannot be suppressed automatically by the |\includeonly| directive
and thus should normally be avoided.
A method to include some content in the main file
by means of conditional processing is described in \secref{sec:conditional}.

%%%%%%%%%%%%%%%%%%%%%%%%%%%%%%%%%%%%%%%%
\paragraph{Page Numbering.}

When only a part of the document is compiled,
the appropriate numbering of pages
(as well as other status parameters)
is determined from the |.aux| files.
The latter contain information from previous passes.
However this information needs to propagate through
all intermediate child documents.
Therefore the page numbering in child documents may well
be inconsistent until the complete document is compiled at least once.

A useful (if unconventional) way to always ensure a consistent
page numbering is to restart the numbering in each child document
and denote the pages by `\textit{child}|.|\textit{page}'
where \textit{child} represents the chapter/section number of the child file.
This can be achieved by the command
|\numberwithin{page}{|\textit{child}|}|
of the \textsf{amsmath} package
where \textit{child} can be |chapter| or |section|
depending on the chosen structuring.
Alternatively, one can modify the macro |\thepage| appropriately
and reset the counter |page| at the start of each child file.

%%%%%%%%%%%%%%%%%%%%%%%%%%%%%%%%%%%%%%%%%%%%%%%%%%%%%%%%%%%%%%%%%%%%%%%%%%%%%%%%
\subsection{Conditional Processing}
\label{sec:conditional}

The package provides a mechanism to compile different versions
of a document. To customise the versions further some conditional processing
can come in handy to distinguish which version is being compiled.
The package provides two macros to describe the compilation context:

%%%%%%%%%%%%%%%%%%%%%%%%%%%%%%%%%%%%%%%%
\DescribeMacro{\ifchilddoc}
The conditional |\ifchilddoc| distinguishes between the compilation of
child documents and the main document:
%
\begin{center}
|\ifchilddoc |\textit{child-code}| |[|\||else |\textit{main-code}]| \||fi|
\end{center}

%%%%%%%%%%%%%%%%%%%%%%%%%%%%%%%%%%%%%%%%
\DescribeMacro{\childdocname}
\DescribeMacro{\childdocjob}
The macro |\childdocname| contains the filename (without extension)
of the main or child file being processed.
Note that |\childdocjob| will always contain the name of the main file.

%%%%%%%%%%%%%%%%%%%%%%%%%%%%%%%%%%%%%%%%
\paragraph{Title Page.}

Conditional processing can be used to include a title or banner page
in the main document when proper precautions are taken.
Importantly, the code in the main file should ensure that the page counter
(as well as other status parameters which are stored in the |.aux| files)
takes the same value after the conditional processing.
Otherwise the page numbers may take divergent values
depending on which part is compiled.

For example, a title page could be declared by:
%
\begin{center}
\begin{tabular}{l}
|\ifchilddoc\||else|\\
|\addtocounter{page}{-1}|\\
\textit{code for title page}\\
|\newpage|\\
|\||fi|
\end{tabular}
\end{center}
%
A banner page for the child documents can be generated by:
%
\begin{center}
\begin{tabular}{l}
|\ifchilddoc|\\
|\addtocounter{page}{-1}|\\
\textit{code for banner page}\\
|\newpage|\\
|\||fi|
\end{tabular}
\end{center}
%
Here one could write a message such as:
\begin{center}
|This is the part \childdocname{} of \childdocjob{}.|
\end{center}

%%%%%%%%%%%%%%%%%%%%%%%%%%%%%%%%%%%%%%%%%%%%%%%%%%%%%%%%%%%%%%%%%%%%%%%%%%%%%%%%
\subsection{Flags}
\label{sec:flags}

The package makes it easy to generate different versions
of the main or child documents.
To this end compilation flags can be defined
and assigned different default values.
They will be particularly useful in conjunction
with the forwarding mechanism described in \secref{sec:forward}.

For example, it may be useful to have a flag |\version|
which can be set to |draft| or |final|.
The document source will contain some conditional code
depending on the value of |\version|.
Suppose further, the flag should default to |final| for the main file
and to |draft| for child files
which is a natural assignment for editing the document.
This is achieved by placing the following code
in the preamble of the main document
(below the |\childdocmain| directive):
%
\begin{center}
\begin{tabular}{l}
|\ifchilddoc|\\
|\providecommand{\version}{draft}|\\
|\||else|\\
|\providecommand{\version}{final}|\\
|\||fi|
\end{tabular}
\end{center}
%
The definition by |\providecommand| makes sure
that previous definitions are not overwritten.
Further statements |\providecommand{\version}{...}|
can thus be added before the above code to override it.

For the main file, one might add a line
(between |\childdocmain| and the above block)
%
\begin{center}
|%\ifchilddoc\||else\providecommand{\version}{draft}\||fi|
\end{center}
%
which can be uncommented to produce a draft version.
Likewise one can add a line to the very top of a child file
(above the |\childdocof{|\textit{main}|}| directive)
%
\begin{center}
|%\providecommand{\version}{final}|
\end{center}
%
which can be uncommented to produce the final version of this child document.

%%%%%%%%%%%%%%%%%%%%%%%%%%%%%%%%%%%%%%%%%%%%%%%%%%%%%%%%%%%%%%%%%%%%%%%%%%%%%%%%
\subsection{Forwarding}
\label{sec:forward}

Different versions of the main or child documents
using compilation flags as described in \secref{sec:flags}
can be (permanently) stored in different files
for convenient compilation, viewing and distribution.
To this end, the package defines a command
to pass on compilation to a different file:

%%%%%%%%%%%%%%%%%%%%%%%%%%%%%%%%%%%%%%%%
\DescribeMacro{\childdocforward}
The command |\childdocforward| redirects processing to
another source file:
%
\begin{center}
\begin{tabular}{l}
|\input{childdoc.def}|\\
|\childdocforward[|\textit{main}|]{|\textit{dest}|}|\\
\end{tabular}
\end{center}
%
The argument \textit{dest} is the destination file
(without extension).
It should be the main file or one of the child files.
Note that further \textsf{childdoc} directives
such as |\childdocof| and |\childdocforward|
in the indicated file will be processed in this form.
The optional argument \textit{main}
passes on directly to the main file \textit{main}
while pretending to compile the child \textit{dest}.
This form behaves as if \textit{dest}
issues |\childdocof{|\textit{main}|}| right away,
and no further \textsf{childdoc} directives will be processed.

%%%%%%%%%%%%%%%%%%%%%%%%%%%%%%%%%%%%%%%%
\DescribeMacro{\...prefix}
In the alternative form |\childdocforwardprefix|,
%
\begin{center}
\begin{tabular}{l}
|\input{childdoc.def}|\\
|\childdocforwardprefix[|\textit{main}|]{|\textit{prefix}|}{|\textit{dest}|}|
\end{tabular}
\end{center}
%
the destination file is determined by a pattern
depending on the current file:
To make this work, the current file must be called
`{\textit{prefix}\hspace{0.2em}\textit{suffix}}'
with \textit{prefix} matching precisely the argument.
Processing is then passed on to the file
`{\textit{dest}\hspace{0.2em}\textit{suffix}}'.
Surely, the same effect is achieved by
directly specifying the
argument `{\textit{dest}\hspace{0.2em}\textit{suffix}}'
in the first form.
However, that requires to set up a different file
for each child. With the alternative form of the command
all these files can have exactly the same content
which simplifies setting them up and maintaining them.

For example, the following file |draft.tex|
with a compilation flag |\version| as described in \secref{sec:flags}
compiles the main document as a draft:
%
\begin{center}
\begin{tabular}{l}
|\def\version{draft}|\\
|\input{childdoc.def}|\\
|\childdocforward{|\textit{main}|}|
\end{tabular}
\end{center}
%
Likewise, the following files |final|\textit{nn}|.tex|
compile the final version of the child document
|child|\textit{nn}|.tex|:
%
\begin{center}
\begin{tabular}{l}
|\def\version{final}|\\
|\input{childdoc.def}|\\
|\childdocforwardprefix{final}{child}|
\end{tabular}
\end{center}
%

Note that when several versions of a main file and/or of each child file
are to be generated, it may be convenient to set up a |Makefile| or
shell script to automatise the process.

%%%%%%%%%%%%%%%%%%%%%%%%%%%%%%%%%%%%%%%%%%%%%%%%%%%%%%%%%%%%%%%%%%%%%%%%%%%%%%%%
\subsection{Command Line Processing}
\label{sec:commandline}

The effect of redirection files can also be achieved by invoking
the \LaTeX{} compiler with a more elaborate command line.
Most conveniently this should be done as part
of a shell script or a |Makefile|.

When using \textsf{childdoc} in the main file, the following
command lines effectively perform a redirection
(note that depending on the shell being used,
backslashes may have to be doubled: `|\|' $\to$ `|\\|'):
%
\begin{center}
|... -jobname "|\textit{target}|" |\\|"|[\textit{flags}]%
|\input{childdoc.def}\childdocforward[|\textit{main}|]{|\textit{dest}|}"|
\end{center}
%
Here \textit{target} is the name of the output file,
\textit{main} is the name of the main file
and \textit{dest} is the name of the main or child file to be processed
(all filenames without extensions).
The optional argument \textit{main} can be omitted
if \textit{main} matches \textit{dest}.
Optionally, compilation \textit{flags} can be defined via |\def| commands.
This command line makes the \TeX{} engine believe
it is compiling the file \textit{target}
whose content is specified as the latter parameter.
The provided code then forwards the processing to
\textit{main} or \textit{dest} as described in \secref{sec:forward}.

%%%%%%%%%%%%%%%%%%%%%%%%%%%%%%%%%%%%%%%%%%%%%%%%%%%%%%%%%%%%%%%%%%%%%%%%%%%%%%%%
\subsection{Include by Input}
\label{sec:input}

Including child documents by |\include| has some restrictions by design.
Most notably, the content of a child document always occupies
its own set of pages; pages cannot be shared between child documents.
Usually, this behaviour makes perfect sense
because each child document contain an essential part of the document.
However, in some situations it may be desirable to compose
a document from a collection of parts
without having mandatory page breaks between then.
For this case, the package
provides a mechanism to include parts
by |\input| which can also be processed individually.
However, by construction this mechanism
requires manual handling of the content to be output.

%%%%%%%%%%%%%%%%%%%%%%%%%%%%%%%%%%%%%%%%
\DescribeMacro{\ifchilddocmanual}
The main file should be prepared as usual, see \secref{sec:include}.
However, the document body must make a distinction
between processing of an individual part and of the main document, e.g.:
%
\begin{center}
\begin{tabular}{l}
|\ifchilddocmanual|\\
|\input{\childdocname}|\\
|\||else|\\
\textit{document body with }|\input{|\textit{part}|}|\\
|\||fi|
\end{tabular}
\end{center}
%
The conditional |\ifchilddocmanual| is true whenever
a part to be included by |\input| is being compiled,
and the name of the part is stored in |\childdocname|.

%%%%%%%%%%%%%%%%%%%%%%%%%%%%%%%%%%%%%%%%
\DescribeMacro{\childdocby}
Each part to be included by |\input| should start with:
%
\begin{center}
\begin{tabular}{l}
|\input{childdoc.def}|\\
|\childdocby{|\textit{main}|}|\\
\end{tabular}
\end{center}
%
The directive |\childdocby| is similar to |\childdocof|
described in \secref{sec:include},
but the subsequent selection of content must be done manually.
To that end, both |\ifchilddoc| and |\ifchilddocmanual|
will be true upon processing of a part,
and the name of the part is stored in |\childdocname|.
Note that |\jobname| will be set to the filename of the current part
so that each part receives an individual |.aux| file
that does not interfere with the |.aux| file(s) of the main document.
This behaviour can be altered by the alternative form
|\childdocby[*]{|\textit{main}|}| (with a non-empty optional argument)
which uses the |.aux| file of the main document
by setting |\jobname| to \textit{main}.

%%%%%%%%%%%%%%%%%%%%%%%%%%%%%%%%%%%%%%%%%%%%%%%%%%%%%%%%%%%%%%%%%%%%%%%%%%%%%%%%
\subsection{Driver Development}
\label{sec:driver}

The \textsf{childdoc} mechanism can also be use for the development
of definition files such as \LaTeX{} styles or classes.
This case differs from the above setup with multiple parts
included by |\include| in that no |\includeonly| should be invoked.
This can be achieved by starting the include file
(before |\ProvidesPackage|) with:
%
\begin{center}
\begin{tabular}{l}
|\input{childdoc.def}|\\
|\childdocforward{|\textit{main}|}|\\
\end{tabular}
\end{center}
%
or alternatively with:
%
\begin{center}
\begin{tabular}{l}
|\input{childdoc.def}|\\
|\childdocby{|\textit{main}|}|\\
\end{tabular}
\end{center}
%
Both forms have slightly different effects as described above.
The main file is prepared as usual, see \secref{sec:include}.

%%%%%%%%%%%%%%%%%%%%%%%%%%%%%%%%%%%%%%%%%%%%%%%%%%%%%%%%%%%%%%%%%%%%%%%%%%%%%%%%
\subsection{Legacy Detection}
\label{sec:detection}

The directive |\childdocmain| in the main file can detect
whether the complete document or merely a child is to be compiled
even without using the directive |\childdocof|.
This method is deprecated because it is less robust
and there is no compelling reason to use it;
it is merely provided for backward compatibility
and it may be removed in future versions.

If the detection mechanism is to be used,
it is mandatory to correctly specify
the filename of the main file as the argument of |\childdocmain|:
%
\begin{center}
\begin{tabular}{l}
|\input{childdoc.def}|\\
|\childdocmain{|\textit{main}|}|\\
\end{tabular}
\end{center}
%
If |\jobname| does not match the argument \textit{main} of |\childdocmain|,
it is assumed that |\jobname| points to the child file to be compiled.
When using |\childdocmain| with the main file specified as argument,
it suffices to start a child file
with just |\input{|\textit{main}|}|
without loading of the package and using |\childdocof|.
If instead all processing is done
with the appropriate \textsf{childdoc} directives,
the argument of \textit{main} of |\childdocmain| can be empty.

An alternative version of the command line processing described
in \secref{sec:commandline} using the detection mechanism reads:
%
\begin{center}
|... -jobname "|\textit{target}|" "|[\textit{flags}]%
[|\def\jobname{|\textit{dest}|}|]|\input{|\textit{main}|}"|
\end{center}

%%%%%%%%%%%%%%%%%%%%%%%%%%%%%%%%%%%%%%%%%%%%%%%%%%%%%%%%%%%%%%%%%%%%%%%%%%%%%%%%
\subsection{Manual Code}
\label{sec:manual}

In case one cannot be certain whether the definitions file |childdoc.def|
is installed on the target \TeX{} distribution
and one prefers not to ship it,
it is conceivable to paste a few relevant commands into the sources.

To that end, drop all statements |\input{childdoc.def}|
and perform the replacements as outlined below.
Instead of |\childdocmain{|\textit{main}|}| add the following code
to the top of the main file:
%
\begin{center}
\begin{tabular}{l}
|\||ifdefined\childdocname\endinput\||fi\newif\ifchilddoc|\\
|\edef\childdocname{\scantokens\expandafter{\jobname\noexpand}}|\\
|\def\childdocmain{|\textit{main}|}\||ifx\childdocmain\childdocname\||else|\\
|\childdoctrue\includeonly{\childdocname}\let\jobname\childdocmain\||fi|\\
\end{tabular}
\end{center}
%
Instead of |\childdocof{|\textit{main}|}| just include the main file
at the top of each child file:
%
\begin{center}
|\input{|\textit{main}|}|
\end{center}
%
A simple redirection |\childdocforward{|\textit{dest}|}| is achieved by:
%
\begin{center}
|\def\jobname{|\textit{dest}|}\input{\jobname}|
\end{center}
%
The redirection with prefix
|\childdocforwardprefix[|\textit{prefix}|]{|\textit{dest}|}|
is accomplished by:
%
\begin{center}
\begin{tabular}{l}
|{\edef\jobname{\scantokens\expandafter{\jobname\noexpand}}|\\
|\def\redirectjob |\textit{prefix}|#1~~~{\gdef\jobname{|\textit{dest}|#1}}|\\
|\expandafter\redirectjob\jobname~~~}\input{\jobname}|
\end{tabular}
\end{center}

In an alternative approach,
child documents can be compiled by a specific command line
without additional code or specific definitions:
%
\begin{center}
|... -jobname "|\textit{target}|" "|[\textit{flags}]%
|\includeonly{|\textit{dest}|}\input{|\textit{main}|}"|
\end{center}
%

%%%%%%%%%%%%%%%%%%%%%%%%%%%%%%%%%%%%%%%%%%%%%%%%%%%%%%%%%%%%%%%%%%%%%%%%%%%%%%%%
%%%%%%%%%%%%%%%%%%%%%%%%%%%%%%%%%%%%%%%%%%%%%%%%%%%%%%%%%%%%%%%%%%%%%%%%%%%%%%%%
\section{Information}

%%%%%%%%%%%%%%%%%%%%%%%%%%%%%%%%%%%%%%%%%%%%%%%%%%%%%%%%%%%%%%%%%%%%%%%%%%%%%%%%
\subsection{Copyright}

Copyright \copyright{} 2017--2018 Niklas Beisert

This work may be distributed and/or modified under the
conditions of the \LaTeX{} Project Public License, either version 1.3
of this license or (at your option) any later version.
The latest version of this license is in
  \url{http://www.latex-project.org/lppl.txt}
and version 1.3 or later is part of all distributions of \LaTeX{}
version 2005/12/01 or later.

This work has the LPPL maintenance status `maintained'.

The Current Maintainer of this work is Niklas Beisert.

This work consists of the files |README.txt|, |childdoc.ins| and |childdoc.dtx|
as well as the derived files |childdoc.def|, |cdocsamp.tex|
with |cdocsch1.tex|, |cdocsch2.tex|, |cdocspt3.tex|, |cdocspt4.tex|,
|cdocsdrf.tex|, |cdocsfn1.tex|, |cdocsfn2.tex|
as well as |childdoc.pdf|.

%%%%%%%%%%%%%%%%%%%%%%%%%%%%%%%%%%%%%%%%%%%%%%%%%%%%%%%%%%%%%%%%%%%%%%%%%%%%%%%%
\subsection{Files and Installation}

The package consists of the files:
%
\begin{center}
\begin{tabular}{ll}
    |README.txt|   & readme file \\
    |childdoc.ins| & installation file \\
    |childdoc.dtx| & source file \\
    |childdoc.def| & definition file \\
    |cdocsamp.tex| & sample main file \\
    |cdocsch1.tex| & sample include file \\
    |cdocsch2.tex| & sample include file \\
    |cdocspt3.tex| & sample part file \\
    |cdocspt4.tex| & sample part file \\
    |cdocsdrf.tex| & sample redirection file \\
    |cdocsfn1.tex| & sample redirection file \\
    |cdocsfn2.tex| & sample redirection file \\
    |childdoc.pdf| & manual
\end{tabular}
\end{center}
%
The distribution consists of the files
|README.txt|, |childdoc.ins| and |childdoc.dtx|.
%
\begin{itemize}
\item
Run (pdf)\LaTeX{} on |childdoc.dtx|
to compile the manual |childdoc.pdf| (this file).
\item
Run \LaTeX{} on |childdoc.ins| to create the definitions file |childdoc.def|
and the sample |cdocsamp.tex| with include files
|cdocsch1.tex|, |cdocsch2.tex|, |cdocspt3.tex|, |cdocspt4.tex|,
|cdocsdrf.tex|, |cdocsfn1.tex|, |cdocsfn2.tex|.
Then copy the file |childdoc.def| to an appropriate directory of your \LaTeX{}
distribution, e.g.\ \textit{texmf-root}|/tex/latex/childdoc|.
\end{itemize}

%%%%%%%%%%%%%%%%%%%%%%%%%%%%%%%%%%%%%%%%%%%%%%%%%%%%%%%%%%%%%%%%%%%%%%%%%%%%%%%%
\subsection{Related CTAN Packages}

There are several other packages which offer a similar functionality:
%
\begin{itemize}
\item
The packages
\href{http://ctan.org/pkg/docmute}{\textsf{docmute}},
\href{http://ctan.org/pkg/includex}{\textsf{includex}} and
\href{http://ctan.org/pkg/standalone}{\textsf{standalone}}
provide commands to include only the document body of
a child file thus allowing both files to be compiled individually.
\item
The packages \href{http://ctan.org/pkg/subdocs}{\textsf{subdocs}}
and \href{http://ctan.org/pkg/subfiles}{\textsf{subfiles}}
provide structures in which the main and child documents can be
encapsulated and allowing them to be compiled individually.
The inclusion mechanism is different from the conventional |\include|.
\item
The package \href{http://ctan.org/pkg/combine}{\textsf{combine}}
is an elaborate solution to combine several documents into one.
\end{itemize}
%
See also the CTAN topic \href{http://ctan.org/topic/subdocs}{\textsf{subdocs}}
for further related packages.
The present package differs from the above solutions in that
a document structure constructed with the conventional |\include| mechanism
just needs two extra commands at the top of every file
such that all constituent files can be compiled individually.

%%%%%%%%%%%%%%%%%%%%%%%%%%%%%%%%%%%%%%%%%%%%%%%%%%%%%%%%%%%%%%%%%%%%%%%%%%%%%%%%
%\subsection{Feature Suggestions}
%
%The following is a list of features which may be useful for future
%versions of this package:
%%
%\begin{itemize}
%\item
%\ldots
%\end{itemize}

%%%%%%%%%%%%%%%%%%%%%%%%%%%%%%%%%%%%%%%%%%%%%%%%%%%%%%%%%%%%%%%%%%%%%%%%%%%%%%%%
\subsection{Revision History}

%%%%%%%%%%%%%%%%%%%%%%%%%%%%%%%%%%%%%%%%
\paragraph{v2.0:} 2018/12/30

\begin{itemize}
\item
immediate forward processing
\item
added |\childdocby| mechanism
\item
manual restructured
\end{itemize}

%%%%%%%%%%%%%%%%%%%%%%%%%%%%%%%%%%%%%%%%
\paragraph{v1.6:} 2018/01/17

\begin{itemize}
\item
application for development of include files
\item
corrections to manual
\end{itemize}

%%%%%%%%%%%%%%%%%%%%%%%%%%%%%%%%%%%%%%%%
\paragraph{v1.5:} 2017/05/21

\begin{itemize}
\item
more complete structuring introduced
\item
|\childdocof| introduced
\item
|\childdoc| renamed to |\childdocmain|
\item
|\childredirect| renamed to |\childdocforward| and |\childdocforwardprefix|
and functionality expanded
\end{itemize}

%%%%%%%%%%%%%%%%%%%%%%%%%%%%%%%%%%%%%%%%
\paragraph{v1.0:} 2017/04/27

\begin{itemize}
\item
manual and install package
\item
first version published on CTAN
\end{itemize}

%%%%%%%%%%%%%%%%%%%%%%%%%%%%%%%%%%%%%%%%
\paragraph{v0.6:} 2017/04/26

\begin{itemize}
\item
redirection mechanism added
\end{itemize}

%%%%%%%%%%%%%%%%%%%%%%%%%%%%%%%%%%%%%%%%
\paragraph{v0.5:} 2017/04/26

\begin{itemize}
\item
functionality in definition file
\end{itemize}


%%%%%%%%%%%%%%%%%%%%%%%%%%%%%%%%%%%%%%%%%%%%%%%%%%%%%%%%%%%%%%%%%%%%%%%%%%%%%%%%
%%%%%%%%%%%%%%%%%%%%%%%%%%%%%%%%%%%%%%%%%%%%%%%%%%%%%%%%%%%%%%%%%%%%%%%%%%%%%%%%
%%%%%%%%%%%%%%%%%%%%%%%%%%%%%%%%%%%%%%%%%%%%%%%%%%%%%%%%%%%%%%%%%%%%%%%%%%%%%%%%
\appendix

\settowidth\MacroIndent{\rmfamily\scriptsize 000\ }

 \DocInput{childdoc.dtx}

\end{document}
%</driver>
% \fi
%
% %%%%%%%%%%%%%%%%%%%%%%%%%%%%%%%%%%%%%%%%%%%%%%%%%%%%%%%%%%%%%%%%%%%%%%%%%%%%%%
% %%%%%%%%%%%%%%%%%%%%%%%%%%%%%%%%%%%%%%%%%%%%%%%%%%%%%%%%%%%%%%%%%%%%%%%%%%%%%%
% \section{Sample}
%\iffalse
%<*samplemain>
%\fi
%
% The following presents a sample document
% with two chapters, two parts, a title page,
% a compile flag as well as three forwarding files to set the flag.
% It consists of eight |.tex| files:
% \begin{center}
% \begin{tabular}{ll}
% |cdocsamp.tex|&main file\\
% |cdocsch1.tex|&include file for chapter 1\\
% |cdocsch2.tex|&include file for chapter 2\\
% |cdocspt3.tex|&include file for part 3\\
% |cdocspt4.tex|&include file for part 4\\
% |cdocsdrf.tex|&forwarding file for main file in draft mode\\
% |cdocsfi1.tex|&forwarding file for final version of chapter 1\\
% |cdocsfi2.tex|&forwarding file for final version of chapter 2\\
% \end{tabular}
% \end{center}
% Each of the eight files can be compiled directly by the \LaTeX{} compiler.
%
% %%%%%%%%%%%%%%%%%%%%%%%%%%%%%%%%%%%%%%
% \paragraph{Main File.}
%
% The main file is called |cdocsamp.tex|.
%
% Load the \textsf{childdoc} definitions and
% declare the filename for the main document:
%    \begin{macrocode}
\input{childdoc.def}
\childdocmain{}
%    \end{macrocode}

% Optional override for |\version| flag:
%    \begin{macrocode}
%%\ifchilddoc\else\providecommand{\version}{draft}\fi
%    \end{macrocode}

% Define the default values for the |\version| flag
% (|final| for the main file and |draft| for childs):
%    \begin{macrocode}
\ifchilddoc
\providecommand{\version}{draft}
\else
\providecommand{\version}{final}
\fi
%    \end{macrocode}

% Load the standard document class:
%    \begin{macrocode}
\documentclass[12pt]{article}
%    \end{macrocode}

% Start the document body:
%    \begin{macrocode}
\begin{document}
%    \end{macrocode}

% Declare a title page.
% Print title, part of document being processed and version flag:
%    \begin{macrocode}
\addtocounter{page}{-1}
\begin{center}
{\LARGE\bfseries{}childdoc example\par}
\vspace{1cm}
\ifchilddoc
\ifchilddocmanual part\else chapter\fi:
`\childdocname' of `\childdocjob'\par
\else
main document: `\childdocjob'\par
\fi
version: \version\par
\end{center}
\newpage
%    \end{macrocode}

% Manually include selected file,
% otherwise process as usual:
%    \begin{macrocode}
\ifchilddocmanual
\section*{part `\childdocname'}
\input{\childdocname}
\else
%    \end{macrocode}

% Include the two chapters:
%    \begin{macrocode}
\include{cdocsch1}
\include{cdocsch2}
%    \end{macrocode}

% Include the two parts unless only chapters should be displayed:
%    \begin{macrocode}
\ifchilddoc\else
\section{part three}
\input{cdocspt3}
\section{part four}
\input{cdocspt4}
\fi
%    \end{macrocode}

% Process as usual until here:
%    \begin{macrocode}
\fi
%    \end{macrocode}

% End of document body:
%    \begin{macrocode}
\end{document}
%    \end{macrocode}
%\iffalse
%</samplemain>
%\fi
%
% %%%%%%%%%%%%%%%%%%%%%%%%%%%%%%%%%%%%%%
% \paragraph{Chapter Include Files.}
%
% The include files are called |cdocsch1.tex| and |cdocsch2.tex|.
%
%\iffalse
%<*samplechap1|samplechap2>
%\fi

% Optional override for |\version| flag:
%    \begin{macrocode}
%%\providecommand{\version}{final}
%    \end{macrocode}

% Include the main document:
%    \begin{macrocode}
\input{childdoc.def}
\childdocof{cdocsamp}
%    \end{macrocode}

%\iffalse
%</samplechap1|samplechap2>
%\fi
%
%\iffalse
%<*samplechap1>
%\fi
% Some text for chapter 1:
%    \begin{macrocode}
\section{one}
some text in chapter one
%    \end{macrocode}

%\iffalse
%</samplechap1>
%\fi
% Some text for chapter 2:
%\iffalse
%<*samplechap2>
%\fi
%    \begin{macrocode}
\section{two}
more text in chapter two
%    \end{macrocode}

%\iffalse
%</samplechap2>
%\fi
%
% %%%%%%%%%%%%%%%%%%%%%%%%%%%%%%%%%%%%%%
% \paragraph{Part Include Files.}
%
% The include files are called |cdocspt3.tex| and |cdocspt4.tex|.
%
%\iffalse
%<*samplepart3|samplepart4>
%\fi

% Optional override for |\version| flag:
%    \begin{macrocode}
%%\providecommand{\version}{final}
%    \end{macrocode}

% Include the main document:
%    \begin{macrocode}
\input{childdoc.def}
\childdocby{cdocsamp}
%    \end{macrocode}

%\iffalse
%</samplepart3|samplepart4>
%\fi
%
%\iffalse
%<*samplepart3>
%\fi
% Some text for part 3:
%    \begin{macrocode}
some text in part three
%    \end{macrocode}

%\iffalse
%</samplepart3>
%\fi
% Some text for part 4:
%\iffalse
%<*samplepart4>
%\fi
%    \begin{macrocode}
more text in part four
%    \end{macrocode}

%\iffalse
%</samplepart4>
%\fi
%
% %%%%%%%%%%%%%%%%%%%%%%%%%%%%%%%%%%%%%%
% \paragraph{Forwarding for a Complete Draft.}
%
% The following forwarding file |cdocsdrf.tex|
% compiles the main document in draft mode:
%\iffalse
%<*sampledraft>
%\fi
%    \begin{macrocode}
\def\version{draft}
\input{childdoc.def}
\childdocforward{cdocsamp}
%    \end{macrocode}

%\iffalse
%</sampledraft>
%\fi
%
% %%%%%%%%%%%%%%%%%%%%%%%%%%%%%%%%%%%%%%
% \paragraph{Forwarding for Final Version of the Chapters.}
%
% The following forwarding files |cdocsfn1.tex| and |cdocsfn2.tex|
% (with identical content)
% compile the final versions of the child documents
% |cdocsch1.tex| and |cdocsch2.tex|, respectively:
%\iffalse
%<*samplefinal>
%\fi
%    \begin{macrocode}
\def\version{final}
\input{childdoc.def}
\childdocforwardprefix[cdocsamp]{cdocsfn}{cdocsch}
%    \end{macrocode}

%\iffalse
%</samplefinal>
%\fi
%
% %%%%%%%%%%%%%%%%%%%%%%%%%%%%%%%%%%%%%%
% \paragraph{Command Line Processing.}
%
% The following three command lines generate the output files
% |cdocscld|, |cdocscl1| and |cdocscl2|
% which should be identical to
% |cdocsdrf|, |cdocsch1| and |cdocsfn2|, respectively:
% \begin{center}
% \begin{tabular}{l}
% |latex -jobname cdocscld \|\\
% |  "\def\version{draft}\input{childdoc.def}\childdocforward{cdocsamp}"|\\
% |latex -jobname cdocscl1 \|\\
% |  "\input{childdoc.def}\childdocforward[cdocsamp]{cdocsch1}"|\\
% |latex -jobname cdocscl2 \|\\
% |  "\def\version{final}\input{childdoc.def}\childdocforward{cdocsch2}"|
% \end{tabular}
% \end{center}
% Note that the trailing backslash on each first line
% merely continues the input to the second line
% (for convenient cut ant paste).
% Furthermore, the command |latex| can be replaced by any
% of its alternative versions such as |pdflatex|.
%
% %%%%%%%%%%%%%%%%%%%%%%%%%%%%%%%%%%%%%%%%%%%%%%%%%%%%%%%%%%%%%%%%%%%%%%%%%%%%%%
% %%%%%%%%%%%%%%%%%%%%%%%%%%%%%%%%%%%%%%%%%%%%%%%%%%%%%%%%%%%%%%%%%%%%%%%%%%%%%%
% \section{Implementation}
%\iffalse
%<*package>
%\fi
%
% This section describes the definitions file |childdoc.def|.

% The definitions cannot be loaded using |\usepackage| or |\RequirePackage|
% which has a mechanism to prevent loading a style file more than once.
% When loading the definitions by means of |\input|
% multiple instances have to be prevented manually:
%\iffalse
%This code needs to be before the `\ProvidesFile' directive
%which is defined at the beginning of this file.
%Therefore it is also placed there and commented out here.
%</package>
%<*discard>
%\fi
%    \begin{macrocode}
\ifdefined\childdocmain\endinput\fi
%    \end{macrocode}
%\iffalse
%</discard>
%<*package>
%\fi
%
% \macro{\ifchilddoc}
% \macro{\ifchilddocmanual}
% The conditional |\ifchilddoc| tells whether a
% child (true) or main (false) document is being compiled.
% The conditional |\ifchilddocmanual| tells whether
% the |\includeonly| mechanism is used (false) or
% the selection of child files must be performed manually (true).
% The definitions initialise to false:
%    \begin{macrocode}
\newif\ifchilddoc
\newif\ifchilddocmanual
%    \end{macrocode}

% \macro{\childdocname}
% \macro{\childdocjob}
% The macro |\childdocname| stores the name of the main document
% to be compiled. The macro |\childdocjob| stores the name of
% the document on which the \LaTeX{} compiler was originally invoked.
% The content of |\jobname| cannot be compared
% to filenames specified in the source due to different catcodes.
% The following code rescans |\jobname|, stores the result
% in |\childdocname| and saves a copy in |\childdocjob|:
%    \begin{macrocode}
\edef\childdocname{\scantokens\expandafter{\jobname\noexpand}}
\let\childdocjob\childdocname
%    \end{macrocode}

% \macro{\childdocdisable}
% The macro |\childdocdisable| prevents the main file
% from being processed more than once.
% At this stage, the main document command |\childdocmain|
% is assumed to be called once again where it should do nothing.
% Any subsequent call to it should prevent
% a secondary processing of the main document
% It overwrites the forwarding commands
% |\childdocof| and |\childdocforward|
% with empty macros to prevent further inclusions of the main document:
%    \begin{macrocode}
\newcommand{\childdocdisable}
{
  \renewcommand{\childdocmain}[1]{\renewcommand{\childdocmain}[1]{\endinput}}
  \renewcommand{\childdocof}[1]{}
  \renewcommand{\childdocby}[2][]{}
  \renewcommand{\childdocforward}[2][]{}
  \renewcommand{\childdocdisable}{}
}
%    \end{macrocode}

% \macro{\childdocmain}
% The macro |\childdocmain| is to be called at the top of the main file
% with nothing or the main filename (without extension) as argument.
% First, it breaks loops.
% If the argument is not empty and does not match |\childdocname|
% (which is set by the first inclusion of |childdoc.def|),
% |\ifchilddoc| is set to true, |\includeonly| is applied to the child file
% and |\jobname| is set to the main file
% (for proper handling of |.aux| files):
%    \begin{macrocode}
\newcommand{\childdocmain}[1]
{
  \childdocdisable\childdocmain{}
  \if?#1?\else
    \begingroup
      \def\childdoctmp{#1}
      \ifx\childdoctmp\childdocname
        \def\childdoctmp{}
      \else
        \def\childdoctmp
        {
          \childdoctrue
          \includeonly{\childdocname}
          \def\childdocjob{#1}
          \def\jobname{#1}
        }
      \fi
      \expandafter
    \endgroup
    \childdoctmp
  \fi
}
%    \end{macrocode}

% \macro{\childdocof}
% The command |\childdocof| redirects
% compilation to the main file |#1|.
%    \begin{macrocode}
\newcommand{\childdocof}[1]
{
  \childdocdisable
  \childdoctrue
  \includeonly{\childdocname}
  \def\jobname{#1}
  \def\childdocjob{#1}
  \input{#1}
}
%    \end{macrocode}

% \macro{\childdocby}
% The command |\childdocby| ....
%    \begin{macrocode}
\newcommand{\childdocby}[2][]
{
  \childdocdisable
  \childdoctrue
  \childdocmanualtrue
  \if?#1?\else
    \def\jobname{#2}
  \fi
  \def\childdocjob{#2}
  \input{#2}
  \endinput
}
%    \end{macrocode}

% \macro{\childdocforward}
% The command |\childdocforward| redirects
% compilation to the main file or
% (if the optional argument is given) a child file.
% Parameters are set as if the main file
% or a child file starting with |\childdocof| was compiled.
% Then compilation is handed over to the main file:
%    \begin{macrocode}
\newcommand{\childdocforward}[2][]
{
  \begingroup
    \if?#1?
      \def\childdoctmp
      {
        \def\childdocname{#2}
        \def\childdocjob{#2}
        \def\jobname{#2}
        \input{#2}
        \endinput
      }
    \else
      \def\childdoctmp
      {
        \childdocdisable
        \def\childdocname{#2}
        \childdoctrue
        \includeonly{#2}
        \def\childdocjob{#1}
        \def\jobname{#1}
        \input{#1}
        \endinput
      }
    \fi
    \expandafter
  \endgroup
  \childdoctmp
}
%    \end{macrocode}

% \macro{\childdocforwardprefix}
% The command |\childdocforwardprefix| redirects
% compilation to the main or a child file by means of a pattern.
% The prefix |#1| in the current filename is replaced by |#2|
% and the suffix of the current filename is kept
% (it is assumed that the filename does not contain the substring `|~~~|'
% which is used as a delimiter).
% Compilation is handed over to the new file by |\childdocforward|:
%    \begin{macrocode}
\newcommand{\childdocforwardprefix}[3][]
{
  \begingroup
    \def\childdocextract #2##1~~~{\def\childdoctmp{\childdocforward[#1]{#3##1}}}
    \expandafter\childdocextract\childdocname~~~
    \expandafter
  \endgroup
  \childdoctmp
}
%    \end{macrocode}

% \macro{\childdoc}
% The deprecated macro |\childdoc| is a legacy version of |\childdocmain|:
%    \begin{macrocode}
\newcommand{\childdoc}{\childdocmain}
%    \end{macrocode}

% \macro{\childdocredirect}
% The deprecated macro |\childdocredirect| is a legacy version
% of |\childdocforward| and |\childdocforwardprefix|:
%    \begin{macrocode}
\newcommand{\childdocredirect}[2][]
{
  \begingroup
    \if?#1?
      \def\childdoctmp{\childdocforward{#2}}
    \else
      \def\childdoctmp{\childdocforwardprefix{#1}{#2}}
    \fi
    \expandafter
  \endgroup
  \childdoctmp
}
%    \end{macrocode}

%\iffalse
%</package>
%\fi
%
\endinput
\childdocforward{cdocsch2}"|
% \end{tabular}
% \end{center}
% Note that the trailing backslash on each first line
% merely continues the input to the second line
% (for convenient cut ant paste).
% Furthermore, the command |latex| can be replaced by any
% of its alternative versions such as |pdflatex|.
%
% %%%%%%%%%%%%%%%%%%%%%%%%%%%%%%%%%%%%%%%%%%%%%%%%%%%%%%%%%%%%%%%%%%%%%%%%%%%%%%
% %%%%%%%%%%%%%%%%%%%%%%%%%%%%%%%%%%%%%%%%%%%%%%%%%%%%%%%%%%%%%%%%%%%%%%%%%%%%%%
% \section{Implementation}
%\iffalse
%<*package>
%\fi
%
% This section describes the definitions file |childdoc.def|.

% The definitions cannot be loaded using |\usepackage| or |\RequirePackage|
% which has a mechanism to prevent loading a style file more than once.
% When loading the definitions by means of |\input|
% multiple instances have to be prevented manually:
%\iffalse
%This code needs to be before the `\ProvidesFile' directive
%which is defined at the beginning of this file.
%Therefore it is also placed there and commented out here.
%</package>
%<*discard>
%\fi
%    \begin{macrocode}
\ifdefined\childdocmain\endinput\fi
%    \end{macrocode}
%\iffalse
%</discard>
%<*package>
%\fi
%
% \macro{\ifchilddoc}
% \macro{\ifchilddocmanual}
% The conditional |\ifchilddoc| tells whether a
% child (true) or main (false) document is being compiled.
% The conditional |\ifchilddocmanual| tells whether
% the |\includeonly| mechanism is used (false) or
% the selection of child files must be performed manually (true).
% The definitions initialise to false:
%    \begin{macrocode}
\newif\ifchilddoc
\newif\ifchilddocmanual
%    \end{macrocode}

% \macro{\childdocname}
% \macro{\childdocjob}
% The macro |\childdocname| stores the name of the main document
% to be compiled. The macro |\childdocjob| stores the name of
% the document on which the \LaTeX{} compiler was originally invoked.
% The content of |\jobname| cannot be compared
% to filenames specified in the source due to different catcodes.
% The following code rescans |\jobname|, stores the result
% in |\childdocname| and saves a copy in |\childdocjob|:
%    \begin{macrocode}
\edef\childdocname{\scantokens\expandafter{\jobname\noexpand}}
\let\childdocjob\childdocname
%    \end{macrocode}

% \macro{\childdocdisable}
% The macro |\childdocdisable| prevents the main file
% from being processed more than once.
% At this stage, the main document command |\childdocmain|
% is assumed to be called once again where it should do nothing.
% Any subsequent call to it should prevent
% a secondary processing of the main document
% It overwrites the forwarding commands
% |\childdocof| and |\childdocforward|
% with empty macros to prevent further inclusions of the main document:
%    \begin{macrocode}
\newcommand{\childdocdisable}
{
  \renewcommand{\childdocmain}[1]{\renewcommand{\childdocmain}[1]{\endinput}}
  \renewcommand{\childdocof}[1]{}
  \renewcommand{\childdocby}[2][]{}
  \renewcommand{\childdocforward}[2][]{}
  \renewcommand{\childdocdisable}{}
}
%    \end{macrocode}

% \macro{\childdocmain}
% The macro |\childdocmain| is to be called at the top of the main file
% with nothing or the main filename (without extension) as argument.
% First, it breaks loops.
% If the argument is not empty and does not match |\childdocname|
% (which is set by the first inclusion of |childdoc.def|),
% |\ifchilddoc| is set to true, |\includeonly| is applied to the child file
% and |\jobname| is set to the main file
% (for proper handling of |.aux| files):
%    \begin{macrocode}
\newcommand{\childdocmain}[1]
{
  \childdocdisable\childdocmain{}
  \if?#1?\else
    \begingroup
      \def\childdoctmp{#1}
      \ifx\childdoctmp\childdocname
        \def\childdoctmp{}
      \else
        \def\childdoctmp
        {
          \childdoctrue
          \includeonly{\childdocname}
          \def\childdocjob{#1}
          \def\jobname{#1}
        }
      \fi
      \expandafter
    \endgroup
    \childdoctmp
  \fi
}
%    \end{macrocode}

% \macro{\childdocof}
% The command |\childdocof| redirects
% compilation to the main file |#1|.
%    \begin{macrocode}
\newcommand{\childdocof}[1]
{
  \childdocdisable
  \childdoctrue
  \includeonly{\childdocname}
  \def\jobname{#1}
  \def\childdocjob{#1}
  \input{#1}
}
%    \end{macrocode}

% \macro{\childdocby}
% The command |\childdocby| ....
%    \begin{macrocode}
\newcommand{\childdocby}[2][]
{
  \childdocdisable
  \childdoctrue
  \childdocmanualtrue
  \if?#1?\else
    \def\jobname{#2}
  \fi
  \def\childdocjob{#2}
  \input{#2}
  \endinput
}
%    \end{macrocode}

% \macro{\childdocforward}
% The command |\childdocforward| redirects
% compilation to the main file or
% (if the optional argument is given) a child file.
% Parameters are set as if the main file
% or a child file starting with |\childdocof| was compiled.
% Then compilation is handed over to the main file:
%    \begin{macrocode}
\newcommand{\childdocforward}[2][]
{
  \begingroup
    \if?#1?
      \def\childdoctmp
      {
        \def\childdocname{#2}
        \def\childdocjob{#2}
        \def\jobname{#2}
        \input{#2}
        \endinput
      }
    \else
      \def\childdoctmp
      {
        \childdocdisable
        \def\childdocname{#2}
        \childdoctrue
        \includeonly{#2}
        \def\childdocjob{#1}
        \def\jobname{#1}
        \input{#1}
        \endinput
      }
    \fi
    \expandafter
  \endgroup
  \childdoctmp
}
%    \end{macrocode}

% \macro{\childdocforwardprefix}
% The command |\childdocforwardprefix| redirects
% compilation to the main or a child file by means of a pattern.
% The prefix |#1| in the current filename is replaced by |#2|
% and the suffix of the current filename is kept
% (it is assumed that the filename does not contain the substring `|~~~|'
% which is used as a delimiter).
% Compilation is handed over to the new file by |\childdocforward|:
%    \begin{macrocode}
\newcommand{\childdocforwardprefix}[3][]
{
  \begingroup
    \def\childdocextract #2##1~~~{\def\childdoctmp{\childdocforward[#1]{#3##1}}}
    \expandafter\childdocextract\childdocname~~~
    \expandafter
  \endgroup
  \childdoctmp
}
%    \end{macrocode}

% \macro{\childdoc}
% The deprecated macro |\childdoc| is a legacy version of |\childdocmain|:
%    \begin{macrocode}
\newcommand{\childdoc}{\childdocmain}
%    \end{macrocode}

% \macro{\childdocredirect}
% The deprecated macro |\childdocredirect| is a legacy version
% of |\childdocforward| and |\childdocforwardprefix|:
%    \begin{macrocode}
\newcommand{\childdocredirect}[2][]
{
  \begingroup
    \if?#1?
      \def\childdoctmp{\childdocforward{#2}}
    \else
      \def\childdoctmp{\childdocforwardprefix{#1}{#2}}
    \fi
    \expandafter
  \endgroup
  \childdoctmp
}
%    \end{macrocode}

%\iffalse
%</package>
%\fi
%
\endinput
|\\
|\childdocforwardprefix{final}{child}|
\end{tabular}
\end{center}
%

Note that when several versions of a main file and/or of each child file
are to be generated, it may be convenient to set up a |Makefile| or
shell script to automatise the process.

%%%%%%%%%%%%%%%%%%%%%%%%%%%%%%%%%%%%%%%%%%%%%%%%%%%%%%%%%%%%%%%%%%%%%%%%%%%%%%%%
\subsection{Command Line Processing}
\label{sec:commandline}

The effect of redirection files can also be achieved by invoking
the \LaTeX{} compiler with a more elaborate command line.
Most conveniently this should be done as part
of a shell script or a |Makefile|.

When using \textsf{childdoc} in the main file, the following
command lines effectively perform a redirection
(note that depending on the shell being used,
backslashes may have to be doubled: `|\|' $\to$ `|\\|'):
%
\begin{center}
|... -jobname "|\textit{target}|" |\\|"|[\textit{flags}]%
|% \iffalse
%
% childdoc.dtx Copyright (C) 2017-2018 Niklas Beisert
%
% This work may be distributed and/or modified under the
% conditions of the LaTeX Project Public License, either version 1.3
% of this license or (at your option) any later version.
% The latest version of this license is in
%   http://www.latex-project.org/lppl.txt
% and version 1.3 or later is part of all distributions of LaTeX
% version 2005/12/01 or later.
%
% This work has the LPPL maintenance status `maintained'.
%
% The Current Maintainer of this work is Niklas Beisert.
%
% This work consists of the files childdoc.dtx and childdoc.ins
% and the derived files childdoc.def and cdocsamp.tex with
% cdocsch1.tex, cdocsch2.tex, cdocsdrf.tex, cdocsfn1.tex, cdocsfn2.tex.
%
%<package>\ifdefined\childdocmain\endinput\fi
%<package>\ProvidesFile{childdoc.def}[2018/12/30 v2.0 child document driver]
%<samplemain>\ProvidesFile{cdocsamp.tex}[2018/12/30 v2.0 sample for childdoc]
%<*driver>
%\ProvidesFile{childdoc.drv}[2018/12/30 v2.0 childdoc reference manual file]
\PassOptionsToClass{10pt,a4paper}{article}
\documentclass{ltxdoc}

\usepackage[margin=35mm]{geometry}
\usepackage{hyperref}
\usepackage{hyperxmp}
\usepackage[usenames]{color}

\hypersetup{colorlinks=true}
\hypersetup{pdfstartview=FitH}
\hypersetup{pdfpagemode=UseNone}
\hypersetup{pdfsource={}}
\hypersetup{pdflang={en-UK}}
\hypersetup{pdfcopyright={Copyright 2017-2018 Niklas Beisert.
  This work may be distributed and/or modified under the
  conditions of the LaTeX Project Public License, either version 1.3
  of this license or (at your option) any later version.}}
\hypersetup{pdflicenseurl={http://www.latex-project.org/lppl.txt}}
\hypersetup{pdfcontactaddress={ETH Zurich, ITP, HIT K,
  Wolfgang-Pauli-Strasse 27}}
\hypersetup{pdfcontactpostcode={8093}}
\hypersetup{pdfcontactcity={Zurich}}
\hypersetup{pdfcontactcountry={Switzerland}}
\hypersetup{pdfcontactemail={nbeisert@itp.phys.ethz.ch}}
\hypersetup{pdfcontacturl={http://people.phys.ethz.ch/\xmptilde nbeisert/}}

\newcommand{\secref}[1]{\hyperref[#1]{section \ref*{#1}}}

\parskip1ex
\parindent0pt
\let\olditemize\itemize
\def\itemize{\olditemize\parskip0pt}

\begin{document}

\title{The \textsf{childdoc} Package}
\hypersetup{pdftitle={The childdoc Package}}
\author{Niklas Beisert\\[2ex]
  Institut f\"ur Theoretische Physik\\
  Eidgen\"ossische Technische Hochschule Z\"urich\\
  Wolfgang-Pauli-Strasse 27, 8093 Z\"urich, Switzerland\\[1ex]
  \href{mailto:nbeisert@itp.phys.ethz.ch}
  {\texttt{nbeisert@itp.phys.ethz.ch}}}
\hypersetup{pdfauthor={Niklas Beisert}}
\hypersetup{pdfsubject={Manual for the LaTeX2e Package childdoc}}
\date{30 December 2018, \textsf{v2.0}}
\maketitle

\begin{abstract}\noindent
\textsf{childdoc} is a \LaTeXe{} package
that enables the direct compilation
of document sections included by |\include|
to individual files.
\end{abstract}

\begingroup
\parskip0ex
\tableofcontents
\endgroup

%%%%%%%%%%%%%%%%%%%%%%%%%%%%%%%%%%%%%%%%%%%%%%%%%%%%%%%%%%%%%%%%%%%%%%%%%%%%%%%%
%%%%%%%%%%%%%%%%%%%%%%%%%%%%%%%%%%%%%%%%%%%%%%%%%%%%%%%%%%%%%%%%%%%%%%%%%%%%%%%%
\section{Introduction}

\LaTeX{} provides a mechanism to structure a large document (such as a book)
into a main file and several child files (containing the chapters)
using the |\include| command.
This mechanism is beneficial for documents
which span hundreds of pages in order to
make the source file(s) more manageable.
Moreover, compilation can be restricted to
selected child files by means of the |\includeonly| command.
The latter feature can be used to reduce the compilation time while editing
(this was significantly more useful in the earlier days of \LaTeX{})
or to generate a smaller document which is easier to navigate.
Another application of |\includeonly| is to generate
documents consisting of selected parts of the complete document.

However, there are a few drawbacks of the plain |\include| mechanism:
\begin{itemize}
\item
The child files cannot be compiled on their own,
they can only be compiled via the main file.
A naive editing environment
(such as a text editor with an option
to have the current file processed by \LaTeX)
may require one to switch to the main file before compiling;
attempting to compile the child file produces errors.
\item
The main file must be modified (each time)
to adjust the |\includeonly| command
to the present needs. This easily leaves the main file in a messy state.
\item
The generated document will always carry the filename
of the main document. This is inconvenient if
several child files are to be compiled and
to be kept for distribution.
\end{itemize}

The present package provides a simple interface
to make child files individually compilable by \LaTeX{}.
Compiling a child file then has the same effect as compiling
the main file with an |\includeonly| command
to select the appropriate child.
Moreover the generated document will carry the name of the child
rather than the main file.
This resolves all three above issues.

This feature is meant to make the editing of books,
thesis documents and lecture notes somewhat more convenient.
However, the package can also be used efficiently for
composing a series of documents (such as exercise sheets)
which are typically distributed individually.
It then assists the author in generating the individual documents
(potentially in different versions)
as well as a document containing the collected series.
Another application is in developing style files
or other kinds of included material
where compilation of the style file could redirect
to a sample or test file.

%%%%%%%%%%%%%%%%%%%%%%%%%%%%%%%%%%%%%%%%%%%%%%%%%%%%%%%%%%%%%%%%%%%%%%%%%%%%%%%%
%%%%%%%%%%%%%%%%%%%%%%%%%%%%%%%%%%%%%%%%%%%%%%%%%%%%%%%%%%%%%%%%%%%%%%%%%%%%%%%%
\section{Usage}

First of all, the package \textsf{childdoc} is \emph{not} a standard
\LaTeXe{} |.sty| style file! Therefore it needs to be invoked in
a non-standard way.

%%%%%%%%%%%%%%%%%%%%%%%%%%%%%%%%%%%%%%%%%%%%%%%%%%%%%%%%%%%%%%%%%%%%%%%%%%%%%%%%
\subsection{Included Files}
\label{sec:include}

%%%%%%%%%%%%%%%%%%%%%%%%%%%%%%%%%%%%%%%%
\DescribeMacro{\childdocmain}
To use the package, add the commands
\begin{center}
\begin{tabular}{l}
|% \iffalse
%
% childdoc.dtx Copyright (C) 2017-2018 Niklas Beisert
%
% This work may be distributed and/or modified under the
% conditions of the LaTeX Project Public License, either version 1.3
% of this license or (at your option) any later version.
% The latest version of this license is in
%   http://www.latex-project.org/lppl.txt
% and version 1.3 or later is part of all distributions of LaTeX
% version 2005/12/01 or later.
%
% This work has the LPPL maintenance status `maintained'.
%
% The Current Maintainer of this work is Niklas Beisert.
%
% This work consists of the files childdoc.dtx and childdoc.ins
% and the derived files childdoc.def and cdocsamp.tex with
% cdocsch1.tex, cdocsch2.tex, cdocsdrf.tex, cdocsfn1.tex, cdocsfn2.tex.
%
%<package>\ifdefined\childdocmain\endinput\fi
%<package>\ProvidesFile{childdoc.def}[2018/12/30 v2.0 child document driver]
%<samplemain>\ProvidesFile{cdocsamp.tex}[2018/12/30 v2.0 sample for childdoc]
%<*driver>
%\ProvidesFile{childdoc.drv}[2018/12/30 v2.0 childdoc reference manual file]
\PassOptionsToClass{10pt,a4paper}{article}
\documentclass{ltxdoc}

\usepackage[margin=35mm]{geometry}
\usepackage{hyperref}
\usepackage{hyperxmp}
\usepackage[usenames]{color}

\hypersetup{colorlinks=true}
\hypersetup{pdfstartview=FitH}
\hypersetup{pdfpagemode=UseNone}
\hypersetup{pdfsource={}}
\hypersetup{pdflang={en-UK}}
\hypersetup{pdfcopyright={Copyright 2017-2018 Niklas Beisert.
  This work may be distributed and/or modified under the
  conditions of the LaTeX Project Public License, either version 1.3
  of this license or (at your option) any later version.}}
\hypersetup{pdflicenseurl={http://www.latex-project.org/lppl.txt}}
\hypersetup{pdfcontactaddress={ETH Zurich, ITP, HIT K,
  Wolfgang-Pauli-Strasse 27}}
\hypersetup{pdfcontactpostcode={8093}}
\hypersetup{pdfcontactcity={Zurich}}
\hypersetup{pdfcontactcountry={Switzerland}}
\hypersetup{pdfcontactemail={nbeisert@itp.phys.ethz.ch}}
\hypersetup{pdfcontacturl={http://people.phys.ethz.ch/\xmptilde nbeisert/}}

\newcommand{\secref}[1]{\hyperref[#1]{section \ref*{#1}}}

\parskip1ex
\parindent0pt
\let\olditemize\itemize
\def\itemize{\olditemize\parskip0pt}

\begin{document}

\title{The \textsf{childdoc} Package}
\hypersetup{pdftitle={The childdoc Package}}
\author{Niklas Beisert\\[2ex]
  Institut f\"ur Theoretische Physik\\
  Eidgen\"ossische Technische Hochschule Z\"urich\\
  Wolfgang-Pauli-Strasse 27, 8093 Z\"urich, Switzerland\\[1ex]
  \href{mailto:nbeisert@itp.phys.ethz.ch}
  {\texttt{nbeisert@itp.phys.ethz.ch}}}
\hypersetup{pdfauthor={Niklas Beisert}}
\hypersetup{pdfsubject={Manual for the LaTeX2e Package childdoc}}
\date{30 December 2018, \textsf{v2.0}}
\maketitle

\begin{abstract}\noindent
\textsf{childdoc} is a \LaTeXe{} package
that enables the direct compilation
of document sections included by |\include|
to individual files.
\end{abstract}

\begingroup
\parskip0ex
\tableofcontents
\endgroup

%%%%%%%%%%%%%%%%%%%%%%%%%%%%%%%%%%%%%%%%%%%%%%%%%%%%%%%%%%%%%%%%%%%%%%%%%%%%%%%%
%%%%%%%%%%%%%%%%%%%%%%%%%%%%%%%%%%%%%%%%%%%%%%%%%%%%%%%%%%%%%%%%%%%%%%%%%%%%%%%%
\section{Introduction}

\LaTeX{} provides a mechanism to structure a large document (such as a book)
into a main file and several child files (containing the chapters)
using the |\include| command.
This mechanism is beneficial for documents
which span hundreds of pages in order to
make the source file(s) more manageable.
Moreover, compilation can be restricted to
selected child files by means of the |\includeonly| command.
The latter feature can be used to reduce the compilation time while editing
(this was significantly more useful in the earlier days of \LaTeX{})
or to generate a smaller document which is easier to navigate.
Another application of |\includeonly| is to generate
documents consisting of selected parts of the complete document.

However, there are a few drawbacks of the plain |\include| mechanism:
\begin{itemize}
\item
The child files cannot be compiled on their own,
they can only be compiled via the main file.
A naive editing environment
(such as a text editor with an option
to have the current file processed by \LaTeX)
may require one to switch to the main file before compiling;
attempting to compile the child file produces errors.
\item
The main file must be modified (each time)
to adjust the |\includeonly| command
to the present needs. This easily leaves the main file in a messy state.
\item
The generated document will always carry the filename
of the main document. This is inconvenient if
several child files are to be compiled and
to be kept for distribution.
\end{itemize}

The present package provides a simple interface
to make child files individually compilable by \LaTeX{}.
Compiling a child file then has the same effect as compiling
the main file with an |\includeonly| command
to select the appropriate child.
Moreover the generated document will carry the name of the child
rather than the main file.
This resolves all three above issues.

This feature is meant to make the editing of books,
thesis documents and lecture notes somewhat more convenient.
However, the package can also be used efficiently for
composing a series of documents (such as exercise sheets)
which are typically distributed individually.
It then assists the author in generating the individual documents
(potentially in different versions)
as well as a document containing the collected series.
Another application is in developing style files
or other kinds of included material
where compilation of the style file could redirect
to a sample or test file.

%%%%%%%%%%%%%%%%%%%%%%%%%%%%%%%%%%%%%%%%%%%%%%%%%%%%%%%%%%%%%%%%%%%%%%%%%%%%%%%%
%%%%%%%%%%%%%%%%%%%%%%%%%%%%%%%%%%%%%%%%%%%%%%%%%%%%%%%%%%%%%%%%%%%%%%%%%%%%%%%%
\section{Usage}

First of all, the package \textsf{childdoc} is \emph{not} a standard
\LaTeXe{} |.sty| style file! Therefore it needs to be invoked in
a non-standard way.

%%%%%%%%%%%%%%%%%%%%%%%%%%%%%%%%%%%%%%%%%%%%%%%%%%%%%%%%%%%%%%%%%%%%%%%%%%%%%%%%
\subsection{Included Files}
\label{sec:include}

%%%%%%%%%%%%%%%%%%%%%%%%%%%%%%%%%%%%%%%%
\DescribeMacro{\childdocmain}
To use the package, add the commands
\begin{center}
\begin{tabular}{l}
|\input{childdoc.def}|\\
|\childdocmain{}|\\
\end{tabular}
\end{center}
at the very top of the main \LaTeX{} file,
in particular \emph{before} the |\documentclass| statement!
The argument of |\childdocmain| should be left empty
(but it must be present).

%%%%%%%%%%%%%%%%%%%%%%%%%%%%%%%%%%%%%%%%
\DescribeMacro{\childdocof}
Furthermore, add the commands
\begin{center}
\begin{tabular}{l}
|\input{childdoc.def}|\\
|\childdocof{|\textit{main}|}|\\
\end{tabular}
\end{center}
at the top of every child file \textit{child}
which is included by |\include{|\textit{child}|}|
from within the main file
(or at least for those files to be compiled individually).
The argument \textit{main} must be the filename of the main file.

There are a couple of
considerations in setting up the main and child documents:

%%%%%%%%%%%%%%%%%%%%%%%%%%%%%%%%%%%%%%%%
\paragraph{Restrictions.}

Please note the following restrictions:
\begin{itemize}
\item
|\childdocmain| must be called with one argument \textit{main}
to ensure compatibility with earlier version of the package.
It must either be empty (|\childdocmain{}|)
or precisely match the filename of the main file in which it is specified.
See \secref{sec:detection} for further information.
\item
The filename \textit{main} must be specified without the |.tex| extension.
\item
The filename \textit{main} is case sensitive
(even in case-insensitive file systems)
due to internal string comparison.
\item
The argument \textit{main} should be fully expanded, it cannot be a macro.
\item
Subdirectories and special characters should be avoided in filenames.
\item
The command |\childdocmain{|\textit{main}|}| must be followed by a whitespace.
It should not be followed immediately by another command
or by a comment mark `|%|'.
This is because the \TeX{} parser reads the token immediately following
the argument of |\childdocmain| and puts it
at the beginning of every child section;
however, a white\-space is ignored.
\end{itemize}

%%%%%%%%%%%%%%%%%%%%%%%%%%%%%%%%%%%%%%%%
\paragraph{Content of Main File.}

It is advisable to place all content in the child files included by |\include|.
Any output contained in the main file will appear in all child documents
unless suppressed manually;
it cannot be suppressed automatically by the |\includeonly| directive
and thus should normally be avoided.
A method to include some content in the main file
by means of conditional processing is described in \secref{sec:conditional}.

%%%%%%%%%%%%%%%%%%%%%%%%%%%%%%%%%%%%%%%%
\paragraph{Page Numbering.}

When only a part of the document is compiled,
the appropriate numbering of pages
(as well as other status parameters)
is determined from the |.aux| files.
The latter contain information from previous passes.
However this information needs to propagate through
all intermediate child documents.
Therefore the page numbering in child documents may well
be inconsistent until the complete document is compiled at least once.

A useful (if unconventional) way to always ensure a consistent
page numbering is to restart the numbering in each child document
and denote the pages by `\textit{child}|.|\textit{page}'
where \textit{child} represents the chapter/section number of the child file.
This can be achieved by the command
|\numberwithin{page}{|\textit{child}|}|
of the \textsf{amsmath} package
where \textit{child} can be |chapter| or |section|
depending on the chosen structuring.
Alternatively, one can modify the macro |\thepage| appropriately
and reset the counter |page| at the start of each child file.

%%%%%%%%%%%%%%%%%%%%%%%%%%%%%%%%%%%%%%%%%%%%%%%%%%%%%%%%%%%%%%%%%%%%%%%%%%%%%%%%
\subsection{Conditional Processing}
\label{sec:conditional}

The package provides a mechanism to compile different versions
of a document. To customise the versions further some conditional processing
can come in handy to distinguish which version is being compiled.
The package provides two macros to describe the compilation context:

%%%%%%%%%%%%%%%%%%%%%%%%%%%%%%%%%%%%%%%%
\DescribeMacro{\ifchilddoc}
The conditional |\ifchilddoc| distinguishes between the compilation of
child documents and the main document:
%
\begin{center}
|\ifchilddoc |\textit{child-code}| |[|\||else |\textit{main-code}]| \||fi|
\end{center}

%%%%%%%%%%%%%%%%%%%%%%%%%%%%%%%%%%%%%%%%
\DescribeMacro{\childdocname}
\DescribeMacro{\childdocjob}
The macro |\childdocname| contains the filename (without extension)
of the main or child file being processed.
Note that |\childdocjob| will always contain the name of the main file.

%%%%%%%%%%%%%%%%%%%%%%%%%%%%%%%%%%%%%%%%
\paragraph{Title Page.}

Conditional processing can be used to include a title or banner page
in the main document when proper precautions are taken.
Importantly, the code in the main file should ensure that the page counter
(as well as other status parameters which are stored in the |.aux| files)
takes the same value after the conditional processing.
Otherwise the page numbers may take divergent values
depending on which part is compiled.

For example, a title page could be declared by:
%
\begin{center}
\begin{tabular}{l}
|\ifchilddoc\||else|\\
|\addtocounter{page}{-1}|\\
\textit{code for title page}\\
|\newpage|\\
|\||fi|
\end{tabular}
\end{center}
%
A banner page for the child documents can be generated by:
%
\begin{center}
\begin{tabular}{l}
|\ifchilddoc|\\
|\addtocounter{page}{-1}|\\
\textit{code for banner page}\\
|\newpage|\\
|\||fi|
\end{tabular}
\end{center}
%
Here one could write a message such as:
\begin{center}
|This is the part \childdocname{} of \childdocjob{}.|
\end{center}

%%%%%%%%%%%%%%%%%%%%%%%%%%%%%%%%%%%%%%%%%%%%%%%%%%%%%%%%%%%%%%%%%%%%%%%%%%%%%%%%
\subsection{Flags}
\label{sec:flags}

The package makes it easy to generate different versions
of the main or child documents.
To this end compilation flags can be defined
and assigned different default values.
They will be particularly useful in conjunction
with the forwarding mechanism described in \secref{sec:forward}.

For example, it may be useful to have a flag |\version|
which can be set to |draft| or |final|.
The document source will contain some conditional code
depending on the value of |\version|.
Suppose further, the flag should default to |final| for the main file
and to |draft| for child files
which is a natural assignment for editing the document.
This is achieved by placing the following code
in the preamble of the main document
(below the |\childdocmain| directive):
%
\begin{center}
\begin{tabular}{l}
|\ifchilddoc|\\
|\providecommand{\version}{draft}|\\
|\||else|\\
|\providecommand{\version}{final}|\\
|\||fi|
\end{tabular}
\end{center}
%
The definition by |\providecommand| makes sure
that previous definitions are not overwritten.
Further statements |\providecommand{\version}{...}|
can thus be added before the above code to override it.

For the main file, one might add a line
(between |\childdocmain| and the above block)
%
\begin{center}
|%\ifchilddoc\||else\providecommand{\version}{draft}\||fi|
\end{center}
%
which can be uncommented to produce a draft version.
Likewise one can add a line to the very top of a child file
(above the |\childdocof{|\textit{main}|}| directive)
%
\begin{center}
|%\providecommand{\version}{final}|
\end{center}
%
which can be uncommented to produce the final version of this child document.

%%%%%%%%%%%%%%%%%%%%%%%%%%%%%%%%%%%%%%%%%%%%%%%%%%%%%%%%%%%%%%%%%%%%%%%%%%%%%%%%
\subsection{Forwarding}
\label{sec:forward}

Different versions of the main or child documents
using compilation flags as described in \secref{sec:flags}
can be (permanently) stored in different files
for convenient compilation, viewing and distribution.
To this end, the package defines a command
to pass on compilation to a different file:

%%%%%%%%%%%%%%%%%%%%%%%%%%%%%%%%%%%%%%%%
\DescribeMacro{\childdocforward}
The command |\childdocforward| redirects processing to
another source file:
%
\begin{center}
\begin{tabular}{l}
|\input{childdoc.def}|\\
|\childdocforward[|\textit{main}|]{|\textit{dest}|}|\\
\end{tabular}
\end{center}
%
The argument \textit{dest} is the destination file
(without extension).
It should be the main file or one of the child files.
Note that further \textsf{childdoc} directives
such as |\childdocof| and |\childdocforward|
in the indicated file will be processed in this form.
The optional argument \textit{main}
passes on directly to the main file \textit{main}
while pretending to compile the child \textit{dest}.
This form behaves as if \textit{dest}
issues |\childdocof{|\textit{main}|}| right away,
and no further \textsf{childdoc} directives will be processed.

%%%%%%%%%%%%%%%%%%%%%%%%%%%%%%%%%%%%%%%%
\DescribeMacro{\...prefix}
In the alternative form |\childdocforwardprefix|,
%
\begin{center}
\begin{tabular}{l}
|\input{childdoc.def}|\\
|\childdocforwardprefix[|\textit{main}|]{|\textit{prefix}|}{|\textit{dest}|}|
\end{tabular}
\end{center}
%
the destination file is determined by a pattern
depending on the current file:
To make this work, the current file must be called
`{\textit{prefix}\hspace{0.2em}\textit{suffix}}'
with \textit{prefix} matching precisely the argument.
Processing is then passed on to the file
`{\textit{dest}\hspace{0.2em}\textit{suffix}}'.
Surely, the same effect is achieved by
directly specifying the
argument `{\textit{dest}\hspace{0.2em}\textit{suffix}}'
in the first form.
However, that requires to set up a different file
for each child. With the alternative form of the command
all these files can have exactly the same content
which simplifies setting them up and maintaining them.

For example, the following file |draft.tex|
with a compilation flag |\version| as described in \secref{sec:flags}
compiles the main document as a draft:
%
\begin{center}
\begin{tabular}{l}
|\def\version{draft}|\\
|\input{childdoc.def}|\\
|\childdocforward{|\textit{main}|}|
\end{tabular}
\end{center}
%
Likewise, the following files |final|\textit{nn}|.tex|
compile the final version of the child document
|child|\textit{nn}|.tex|:
%
\begin{center}
\begin{tabular}{l}
|\def\version{final}|\\
|\input{childdoc.def}|\\
|\childdocforwardprefix{final}{child}|
\end{tabular}
\end{center}
%

Note that when several versions of a main file and/or of each child file
are to be generated, it may be convenient to set up a |Makefile| or
shell script to automatise the process.

%%%%%%%%%%%%%%%%%%%%%%%%%%%%%%%%%%%%%%%%%%%%%%%%%%%%%%%%%%%%%%%%%%%%%%%%%%%%%%%%
\subsection{Command Line Processing}
\label{sec:commandline}

The effect of redirection files can also be achieved by invoking
the \LaTeX{} compiler with a more elaborate command line.
Most conveniently this should be done as part
of a shell script or a |Makefile|.

When using \textsf{childdoc} in the main file, the following
command lines effectively perform a redirection
(note that depending on the shell being used,
backslashes may have to be doubled: `|\|' $\to$ `|\\|'):
%
\begin{center}
|... -jobname "|\textit{target}|" |\\|"|[\textit{flags}]%
|\input{childdoc.def}\childdocforward[|\textit{main}|]{|\textit{dest}|}"|
\end{center}
%
Here \textit{target} is the name of the output file,
\textit{main} is the name of the main file
and \textit{dest} is the name of the main or child file to be processed
(all filenames without extensions).
The optional argument \textit{main} can be omitted
if \textit{main} matches \textit{dest}.
Optionally, compilation \textit{flags} can be defined via |\def| commands.
This command line makes the \TeX{} engine believe
it is compiling the file \textit{target}
whose content is specified as the latter parameter.
The provided code then forwards the processing to
\textit{main} or \textit{dest} as described in \secref{sec:forward}.

%%%%%%%%%%%%%%%%%%%%%%%%%%%%%%%%%%%%%%%%%%%%%%%%%%%%%%%%%%%%%%%%%%%%%%%%%%%%%%%%
\subsection{Include by Input}
\label{sec:input}

Including child documents by |\include| has some restrictions by design.
Most notably, the content of a child document always occupies
its own set of pages; pages cannot be shared between child documents.
Usually, this behaviour makes perfect sense
because each child document contain an essential part of the document.
However, in some situations it may be desirable to compose
a document from a collection of parts
without having mandatory page breaks between then.
For this case, the package
provides a mechanism to include parts
by |\input| which can also be processed individually.
However, by construction this mechanism
requires manual handling of the content to be output.

%%%%%%%%%%%%%%%%%%%%%%%%%%%%%%%%%%%%%%%%
\DescribeMacro{\ifchilddocmanual}
The main file should be prepared as usual, see \secref{sec:include}.
However, the document body must make a distinction
between processing of an individual part and of the main document, e.g.:
%
\begin{center}
\begin{tabular}{l}
|\ifchilddocmanual|\\
|\input{\childdocname}|\\
|\||else|\\
\textit{document body with }|\input{|\textit{part}|}|\\
|\||fi|
\end{tabular}
\end{center}
%
The conditional |\ifchilddocmanual| is true whenever
a part to be included by |\input| is being compiled,
and the name of the part is stored in |\childdocname|.

%%%%%%%%%%%%%%%%%%%%%%%%%%%%%%%%%%%%%%%%
\DescribeMacro{\childdocby}
Each part to be included by |\input| should start with:
%
\begin{center}
\begin{tabular}{l}
|\input{childdoc.def}|\\
|\childdocby{|\textit{main}|}|\\
\end{tabular}
\end{center}
%
The directive |\childdocby| is similar to |\childdocof|
described in \secref{sec:include},
but the subsequent selection of content must be done manually.
To that end, both |\ifchilddoc| and |\ifchilddocmanual|
will be true upon processing of a part,
and the name of the part is stored in |\childdocname|.
Note that |\jobname| will be set to the filename of the current part
so that each part receives an individual |.aux| file
that does not interfere with the |.aux| file(s) of the main document.
This behaviour can be altered by the alternative form
|\childdocby[*]{|\textit{main}|}| (with a non-empty optional argument)
which uses the |.aux| file of the main document
by setting |\jobname| to \textit{main}.

%%%%%%%%%%%%%%%%%%%%%%%%%%%%%%%%%%%%%%%%%%%%%%%%%%%%%%%%%%%%%%%%%%%%%%%%%%%%%%%%
\subsection{Driver Development}
\label{sec:driver}

The \textsf{childdoc} mechanism can also be use for the development
of definition files such as \LaTeX{} styles or classes.
This case differs from the above setup with multiple parts
included by |\include| in that no |\includeonly| should be invoked.
This can be achieved by starting the include file
(before |\ProvidesPackage|) with:
%
\begin{center}
\begin{tabular}{l}
|\input{childdoc.def}|\\
|\childdocforward{|\textit{main}|}|\\
\end{tabular}
\end{center}
%
or alternatively with:
%
\begin{center}
\begin{tabular}{l}
|\input{childdoc.def}|\\
|\childdocby{|\textit{main}|}|\\
\end{tabular}
\end{center}
%
Both forms have slightly different effects as described above.
The main file is prepared as usual, see \secref{sec:include}.

%%%%%%%%%%%%%%%%%%%%%%%%%%%%%%%%%%%%%%%%%%%%%%%%%%%%%%%%%%%%%%%%%%%%%%%%%%%%%%%%
\subsection{Legacy Detection}
\label{sec:detection}

The directive |\childdocmain| in the main file can detect
whether the complete document or merely a child is to be compiled
even without using the directive |\childdocof|.
This method is deprecated because it is less robust
and there is no compelling reason to use it;
it is merely provided for backward compatibility
and it may be removed in future versions.

If the detection mechanism is to be used,
it is mandatory to correctly specify
the filename of the main file as the argument of |\childdocmain|:
%
\begin{center}
\begin{tabular}{l}
|\input{childdoc.def}|\\
|\childdocmain{|\textit{main}|}|\\
\end{tabular}
\end{center}
%
If |\jobname| does not match the argument \textit{main} of |\childdocmain|,
it is assumed that |\jobname| points to the child file to be compiled.
When using |\childdocmain| with the main file specified as argument,
it suffices to start a child file
with just |\input{|\textit{main}|}|
without loading of the package and using |\childdocof|.
If instead all processing is done
with the appropriate \textsf{childdoc} directives,
the argument of \textit{main} of |\childdocmain| can be empty.

An alternative version of the command line processing described
in \secref{sec:commandline} using the detection mechanism reads:
%
\begin{center}
|... -jobname "|\textit{target}|" "|[\textit{flags}]%
[|\def\jobname{|\textit{dest}|}|]|\input{|\textit{main}|}"|
\end{center}

%%%%%%%%%%%%%%%%%%%%%%%%%%%%%%%%%%%%%%%%%%%%%%%%%%%%%%%%%%%%%%%%%%%%%%%%%%%%%%%%
\subsection{Manual Code}
\label{sec:manual}

In case one cannot be certain whether the definitions file |childdoc.def|
is installed on the target \TeX{} distribution
and one prefers not to ship it,
it is conceivable to paste a few relevant commands into the sources.

To that end, drop all statements |\input{childdoc.def}|
and perform the replacements as outlined below.
Instead of |\childdocmain{|\textit{main}|}| add the following code
to the top of the main file:
%
\begin{center}
\begin{tabular}{l}
|\||ifdefined\childdocname\endinput\||fi\newif\ifchilddoc|\\
|\edef\childdocname{\scantokens\expandafter{\jobname\noexpand}}|\\
|\def\childdocmain{|\textit{main}|}\||ifx\childdocmain\childdocname\||else|\\
|\childdoctrue\includeonly{\childdocname}\let\jobname\childdocmain\||fi|\\
\end{tabular}
\end{center}
%
Instead of |\childdocof{|\textit{main}|}| just include the main file
at the top of each child file:
%
\begin{center}
|\input{|\textit{main}|}|
\end{center}
%
A simple redirection |\childdocforward{|\textit{dest}|}| is achieved by:
%
\begin{center}
|\def\jobname{|\textit{dest}|}\input{\jobname}|
\end{center}
%
The redirection with prefix
|\childdocforwardprefix[|\textit{prefix}|]{|\textit{dest}|}|
is accomplished by:
%
\begin{center}
\begin{tabular}{l}
|{\edef\jobname{\scantokens\expandafter{\jobname\noexpand}}|\\
|\def\redirectjob |\textit{prefix}|#1~~~{\gdef\jobname{|\textit{dest}|#1}}|\\
|\expandafter\redirectjob\jobname~~~}\input{\jobname}|
\end{tabular}
\end{center}

In an alternative approach,
child documents can be compiled by a specific command line
without additional code or specific definitions:
%
\begin{center}
|... -jobname "|\textit{target}|" "|[\textit{flags}]%
|\includeonly{|\textit{dest}|}\input{|\textit{main}|}"|
\end{center}
%

%%%%%%%%%%%%%%%%%%%%%%%%%%%%%%%%%%%%%%%%%%%%%%%%%%%%%%%%%%%%%%%%%%%%%%%%%%%%%%%%
%%%%%%%%%%%%%%%%%%%%%%%%%%%%%%%%%%%%%%%%%%%%%%%%%%%%%%%%%%%%%%%%%%%%%%%%%%%%%%%%
\section{Information}

%%%%%%%%%%%%%%%%%%%%%%%%%%%%%%%%%%%%%%%%%%%%%%%%%%%%%%%%%%%%%%%%%%%%%%%%%%%%%%%%
\subsection{Copyright}

Copyright \copyright{} 2017--2018 Niklas Beisert

This work may be distributed and/or modified under the
conditions of the \LaTeX{} Project Public License, either version 1.3
of this license or (at your option) any later version.
The latest version of this license is in
  \url{http://www.latex-project.org/lppl.txt}
and version 1.3 or later is part of all distributions of \LaTeX{}
version 2005/12/01 or later.

This work has the LPPL maintenance status `maintained'.

The Current Maintainer of this work is Niklas Beisert.

This work consists of the files |README.txt|, |childdoc.ins| and |childdoc.dtx|
as well as the derived files |childdoc.def|, |cdocsamp.tex|
with |cdocsch1.tex|, |cdocsch2.tex|, |cdocspt3.tex|, |cdocspt4.tex|,
|cdocsdrf.tex|, |cdocsfn1.tex|, |cdocsfn2.tex|
as well as |childdoc.pdf|.

%%%%%%%%%%%%%%%%%%%%%%%%%%%%%%%%%%%%%%%%%%%%%%%%%%%%%%%%%%%%%%%%%%%%%%%%%%%%%%%%
\subsection{Files and Installation}

The package consists of the files:
%
\begin{center}
\begin{tabular}{ll}
    |README.txt|   & readme file \\
    |childdoc.ins| & installation file \\
    |childdoc.dtx| & source file \\
    |childdoc.def| & definition file \\
    |cdocsamp.tex| & sample main file \\
    |cdocsch1.tex| & sample include file \\
    |cdocsch2.tex| & sample include file \\
    |cdocspt3.tex| & sample part file \\
    |cdocspt4.tex| & sample part file \\
    |cdocsdrf.tex| & sample redirection file \\
    |cdocsfn1.tex| & sample redirection file \\
    |cdocsfn2.tex| & sample redirection file \\
    |childdoc.pdf| & manual
\end{tabular}
\end{center}
%
The distribution consists of the files
|README.txt|, |childdoc.ins| and |childdoc.dtx|.
%
\begin{itemize}
\item
Run (pdf)\LaTeX{} on |childdoc.dtx|
to compile the manual |childdoc.pdf| (this file).
\item
Run \LaTeX{} on |childdoc.ins| to create the definitions file |childdoc.def|
and the sample |cdocsamp.tex| with include files
|cdocsch1.tex|, |cdocsch2.tex|, |cdocspt3.tex|, |cdocspt4.tex|,
|cdocsdrf.tex|, |cdocsfn1.tex|, |cdocsfn2.tex|.
Then copy the file |childdoc.def| to an appropriate directory of your \LaTeX{}
distribution, e.g.\ \textit{texmf-root}|/tex/latex/childdoc|.
\end{itemize}

%%%%%%%%%%%%%%%%%%%%%%%%%%%%%%%%%%%%%%%%%%%%%%%%%%%%%%%%%%%%%%%%%%%%%%%%%%%%%%%%
\subsection{Related CTAN Packages}

There are several other packages which offer a similar functionality:
%
\begin{itemize}
\item
The packages
\href{http://ctan.org/pkg/docmute}{\textsf{docmute}},
\href{http://ctan.org/pkg/includex}{\textsf{includex}} and
\href{http://ctan.org/pkg/standalone}{\textsf{standalone}}
provide commands to include only the document body of
a child file thus allowing both files to be compiled individually.
\item
The packages \href{http://ctan.org/pkg/subdocs}{\textsf{subdocs}}
and \href{http://ctan.org/pkg/subfiles}{\textsf{subfiles}}
provide structures in which the main and child documents can be
encapsulated and allowing them to be compiled individually.
The inclusion mechanism is different from the conventional |\include|.
\item
The package \href{http://ctan.org/pkg/combine}{\textsf{combine}}
is an elaborate solution to combine several documents into one.
\end{itemize}
%
See also the CTAN topic \href{http://ctan.org/topic/subdocs}{\textsf{subdocs}}
for further related packages.
The present package differs from the above solutions in that
a document structure constructed with the conventional |\include| mechanism
just needs two extra commands at the top of every file
such that all constituent files can be compiled individually.

%%%%%%%%%%%%%%%%%%%%%%%%%%%%%%%%%%%%%%%%%%%%%%%%%%%%%%%%%%%%%%%%%%%%%%%%%%%%%%%%
%\subsection{Feature Suggestions}
%
%The following is a list of features which may be useful for future
%versions of this package:
%%
%\begin{itemize}
%\item
%\ldots
%\end{itemize}

%%%%%%%%%%%%%%%%%%%%%%%%%%%%%%%%%%%%%%%%%%%%%%%%%%%%%%%%%%%%%%%%%%%%%%%%%%%%%%%%
\subsection{Revision History}

%%%%%%%%%%%%%%%%%%%%%%%%%%%%%%%%%%%%%%%%
\paragraph{v2.0:} 2018/12/30

\begin{itemize}
\item
immediate forward processing
\item
added |\childdocby| mechanism
\item
manual restructured
\end{itemize}

%%%%%%%%%%%%%%%%%%%%%%%%%%%%%%%%%%%%%%%%
\paragraph{v1.6:} 2018/01/17

\begin{itemize}
\item
application for development of include files
\item
corrections to manual
\end{itemize}

%%%%%%%%%%%%%%%%%%%%%%%%%%%%%%%%%%%%%%%%
\paragraph{v1.5:} 2017/05/21

\begin{itemize}
\item
more complete structuring introduced
\item
|\childdocof| introduced
\item
|\childdoc| renamed to |\childdocmain|
\item
|\childredirect| renamed to |\childdocforward| and |\childdocforwardprefix|
and functionality expanded
\end{itemize}

%%%%%%%%%%%%%%%%%%%%%%%%%%%%%%%%%%%%%%%%
\paragraph{v1.0:} 2017/04/27

\begin{itemize}
\item
manual and install package
\item
first version published on CTAN
\end{itemize}

%%%%%%%%%%%%%%%%%%%%%%%%%%%%%%%%%%%%%%%%
\paragraph{v0.6:} 2017/04/26

\begin{itemize}
\item
redirection mechanism added
\end{itemize}

%%%%%%%%%%%%%%%%%%%%%%%%%%%%%%%%%%%%%%%%
\paragraph{v0.5:} 2017/04/26

\begin{itemize}
\item
functionality in definition file
\end{itemize}


%%%%%%%%%%%%%%%%%%%%%%%%%%%%%%%%%%%%%%%%%%%%%%%%%%%%%%%%%%%%%%%%%%%%%%%%%%%%%%%%
%%%%%%%%%%%%%%%%%%%%%%%%%%%%%%%%%%%%%%%%%%%%%%%%%%%%%%%%%%%%%%%%%%%%%%%%%%%%%%%%
%%%%%%%%%%%%%%%%%%%%%%%%%%%%%%%%%%%%%%%%%%%%%%%%%%%%%%%%%%%%%%%%%%%%%%%%%%%%%%%%
\appendix

\settowidth\MacroIndent{\rmfamily\scriptsize 000\ }

 \DocInput{childdoc.dtx}

\end{document}
%</driver>
% \fi
%
% %%%%%%%%%%%%%%%%%%%%%%%%%%%%%%%%%%%%%%%%%%%%%%%%%%%%%%%%%%%%%%%%%%%%%%%%%%%%%%
% %%%%%%%%%%%%%%%%%%%%%%%%%%%%%%%%%%%%%%%%%%%%%%%%%%%%%%%%%%%%%%%%%%%%%%%%%%%%%%
% \section{Sample}
%\iffalse
%<*samplemain>
%\fi
%
% The following presents a sample document
% with two chapters, two parts, a title page,
% a compile flag as well as three forwarding files to set the flag.
% It consists of eight |.tex| files:
% \begin{center}
% \begin{tabular}{ll}
% |cdocsamp.tex|&main file\\
% |cdocsch1.tex|&include file for chapter 1\\
% |cdocsch2.tex|&include file for chapter 2\\
% |cdocspt3.tex|&include file for part 3\\
% |cdocspt4.tex|&include file for part 4\\
% |cdocsdrf.tex|&forwarding file for main file in draft mode\\
% |cdocsfi1.tex|&forwarding file for final version of chapter 1\\
% |cdocsfi2.tex|&forwarding file for final version of chapter 2\\
% \end{tabular}
% \end{center}
% Each of the eight files can be compiled directly by the \LaTeX{} compiler.
%
% %%%%%%%%%%%%%%%%%%%%%%%%%%%%%%%%%%%%%%
% \paragraph{Main File.}
%
% The main file is called |cdocsamp.tex|.
%
% Load the \textsf{childdoc} definitions and
% declare the filename for the main document:
%    \begin{macrocode}
\input{childdoc.def}
\childdocmain{}
%    \end{macrocode}

% Optional override for |\version| flag:
%    \begin{macrocode}
%%\ifchilddoc\else\providecommand{\version}{draft}\fi
%    \end{macrocode}

% Define the default values for the |\version| flag
% (|final| for the main file and |draft| for childs):
%    \begin{macrocode}
\ifchilddoc
\providecommand{\version}{draft}
\else
\providecommand{\version}{final}
\fi
%    \end{macrocode}

% Load the standard document class:
%    \begin{macrocode}
\documentclass[12pt]{article}
%    \end{macrocode}

% Start the document body:
%    \begin{macrocode}
\begin{document}
%    \end{macrocode}

% Declare a title page.
% Print title, part of document being processed and version flag:
%    \begin{macrocode}
\addtocounter{page}{-1}
\begin{center}
{\LARGE\bfseries{}childdoc example\par}
\vspace{1cm}
\ifchilddoc
\ifchilddocmanual part\else chapter\fi:
`\childdocname' of `\childdocjob'\par
\else
main document: `\childdocjob'\par
\fi
version: \version\par
\end{center}
\newpage
%    \end{macrocode}

% Manually include selected file,
% otherwise process as usual:
%    \begin{macrocode}
\ifchilddocmanual
\section*{part `\childdocname'}
\input{\childdocname}
\else
%    \end{macrocode}

% Include the two chapters:
%    \begin{macrocode}
\include{cdocsch1}
\include{cdocsch2}
%    \end{macrocode}

% Include the two parts unless only chapters should be displayed:
%    \begin{macrocode}
\ifchilddoc\else
\section{part three}
\input{cdocspt3}
\section{part four}
\input{cdocspt4}
\fi
%    \end{macrocode}

% Process as usual until here:
%    \begin{macrocode}
\fi
%    \end{macrocode}

% End of document body:
%    \begin{macrocode}
\end{document}
%    \end{macrocode}
%\iffalse
%</samplemain>
%\fi
%
% %%%%%%%%%%%%%%%%%%%%%%%%%%%%%%%%%%%%%%
% \paragraph{Chapter Include Files.}
%
% The include files are called |cdocsch1.tex| and |cdocsch2.tex|.
%
%\iffalse
%<*samplechap1|samplechap2>
%\fi

% Optional override for |\version| flag:
%    \begin{macrocode}
%%\providecommand{\version}{final}
%    \end{macrocode}

% Include the main document:
%    \begin{macrocode}
\input{childdoc.def}
\childdocof{cdocsamp}
%    \end{macrocode}

%\iffalse
%</samplechap1|samplechap2>
%\fi
%
%\iffalse
%<*samplechap1>
%\fi
% Some text for chapter 1:
%    \begin{macrocode}
\section{one}
some text in chapter one
%    \end{macrocode}

%\iffalse
%</samplechap1>
%\fi
% Some text for chapter 2:
%\iffalse
%<*samplechap2>
%\fi
%    \begin{macrocode}
\section{two}
more text in chapter two
%    \end{macrocode}

%\iffalse
%</samplechap2>
%\fi
%
% %%%%%%%%%%%%%%%%%%%%%%%%%%%%%%%%%%%%%%
% \paragraph{Part Include Files.}
%
% The include files are called |cdocspt3.tex| and |cdocspt4.tex|.
%
%\iffalse
%<*samplepart3|samplepart4>
%\fi

% Optional override for |\version| flag:
%    \begin{macrocode}
%%\providecommand{\version}{final}
%    \end{macrocode}

% Include the main document:
%    \begin{macrocode}
\input{childdoc.def}
\childdocby{cdocsamp}
%    \end{macrocode}

%\iffalse
%</samplepart3|samplepart4>
%\fi
%
%\iffalse
%<*samplepart3>
%\fi
% Some text for part 3:
%    \begin{macrocode}
some text in part three
%    \end{macrocode}

%\iffalse
%</samplepart3>
%\fi
% Some text for part 4:
%\iffalse
%<*samplepart4>
%\fi
%    \begin{macrocode}
more text in part four
%    \end{macrocode}

%\iffalse
%</samplepart4>
%\fi
%
% %%%%%%%%%%%%%%%%%%%%%%%%%%%%%%%%%%%%%%
% \paragraph{Forwarding for a Complete Draft.}
%
% The following forwarding file |cdocsdrf.tex|
% compiles the main document in draft mode:
%\iffalse
%<*sampledraft>
%\fi
%    \begin{macrocode}
\def\version{draft}
\input{childdoc.def}
\childdocforward{cdocsamp}
%    \end{macrocode}

%\iffalse
%</sampledraft>
%\fi
%
% %%%%%%%%%%%%%%%%%%%%%%%%%%%%%%%%%%%%%%
% \paragraph{Forwarding for Final Version of the Chapters.}
%
% The following forwarding files |cdocsfn1.tex| and |cdocsfn2.tex|
% (with identical content)
% compile the final versions of the child documents
% |cdocsch1.tex| and |cdocsch2.tex|, respectively:
%\iffalse
%<*samplefinal>
%\fi
%    \begin{macrocode}
\def\version{final}
\input{childdoc.def}
\childdocforwardprefix[cdocsamp]{cdocsfn}{cdocsch}
%    \end{macrocode}

%\iffalse
%</samplefinal>
%\fi
%
% %%%%%%%%%%%%%%%%%%%%%%%%%%%%%%%%%%%%%%
% \paragraph{Command Line Processing.}
%
% The following three command lines generate the output files
% |cdocscld|, |cdocscl1| and |cdocscl2|
% which should be identical to
% |cdocsdrf|, |cdocsch1| and |cdocsfn2|, respectively:
% \begin{center}
% \begin{tabular}{l}
% |latex -jobname cdocscld \|\\
% |  "\def\version{draft}\input{childdoc.def}\childdocforward{cdocsamp}"|\\
% |latex -jobname cdocscl1 \|\\
% |  "\input{childdoc.def}\childdocforward[cdocsamp]{cdocsch1}"|\\
% |latex -jobname cdocscl2 \|\\
% |  "\def\version{final}\input{childdoc.def}\childdocforward{cdocsch2}"|
% \end{tabular}
% \end{center}
% Note that the trailing backslash on each first line
% merely continues the input to the second line
% (for convenient cut ant paste).
% Furthermore, the command |latex| can be replaced by any
% of its alternative versions such as |pdflatex|.
%
% %%%%%%%%%%%%%%%%%%%%%%%%%%%%%%%%%%%%%%%%%%%%%%%%%%%%%%%%%%%%%%%%%%%%%%%%%%%%%%
% %%%%%%%%%%%%%%%%%%%%%%%%%%%%%%%%%%%%%%%%%%%%%%%%%%%%%%%%%%%%%%%%%%%%%%%%%%%%%%
% \section{Implementation}
%\iffalse
%<*package>
%\fi
%
% This section describes the definitions file |childdoc.def|.

% The definitions cannot be loaded using |\usepackage| or |\RequirePackage|
% which has a mechanism to prevent loading a style file more than once.
% When loading the definitions by means of |\input|
% multiple instances have to be prevented manually:
%\iffalse
%This code needs to be before the `\ProvidesFile' directive
%which is defined at the beginning of this file.
%Therefore it is also placed there and commented out here.
%</package>
%<*discard>
%\fi
%    \begin{macrocode}
\ifdefined\childdocmain\endinput\fi
%    \end{macrocode}
%\iffalse
%</discard>
%<*package>
%\fi
%
% \macro{\ifchilddoc}
% \macro{\ifchilddocmanual}
% The conditional |\ifchilddoc| tells whether a
% child (true) or main (false) document is being compiled.
% The conditional |\ifchilddocmanual| tells whether
% the |\includeonly| mechanism is used (false) or
% the selection of child files must be performed manually (true).
% The definitions initialise to false:
%    \begin{macrocode}
\newif\ifchilddoc
\newif\ifchilddocmanual
%    \end{macrocode}

% \macro{\childdocname}
% \macro{\childdocjob}
% The macro |\childdocname| stores the name of the main document
% to be compiled. The macro |\childdocjob| stores the name of
% the document on which the \LaTeX{} compiler was originally invoked.
% The content of |\jobname| cannot be compared
% to filenames specified in the source due to different catcodes.
% The following code rescans |\jobname|, stores the result
% in |\childdocname| and saves a copy in |\childdocjob|:
%    \begin{macrocode}
\edef\childdocname{\scantokens\expandafter{\jobname\noexpand}}
\let\childdocjob\childdocname
%    \end{macrocode}

% \macro{\childdocdisable}
% The macro |\childdocdisable| prevents the main file
% from being processed more than once.
% At this stage, the main document command |\childdocmain|
% is assumed to be called once again where it should do nothing.
% Any subsequent call to it should prevent
% a secondary processing of the main document
% It overwrites the forwarding commands
% |\childdocof| and |\childdocforward|
% with empty macros to prevent further inclusions of the main document:
%    \begin{macrocode}
\newcommand{\childdocdisable}
{
  \renewcommand{\childdocmain}[1]{\renewcommand{\childdocmain}[1]{\endinput}}
  \renewcommand{\childdocof}[1]{}
  \renewcommand{\childdocby}[2][]{}
  \renewcommand{\childdocforward}[2][]{}
  \renewcommand{\childdocdisable}{}
}
%    \end{macrocode}

% \macro{\childdocmain}
% The macro |\childdocmain| is to be called at the top of the main file
% with nothing or the main filename (without extension) as argument.
% First, it breaks loops.
% If the argument is not empty and does not match |\childdocname|
% (which is set by the first inclusion of |childdoc.def|),
% |\ifchilddoc| is set to true, |\includeonly| is applied to the child file
% and |\jobname| is set to the main file
% (for proper handling of |.aux| files):
%    \begin{macrocode}
\newcommand{\childdocmain}[1]
{
  \childdocdisable\childdocmain{}
  \if?#1?\else
    \begingroup
      \def\childdoctmp{#1}
      \ifx\childdoctmp\childdocname
        \def\childdoctmp{}
      \else
        \def\childdoctmp
        {
          \childdoctrue
          \includeonly{\childdocname}
          \def\childdocjob{#1}
          \def\jobname{#1}
        }
      \fi
      \expandafter
    \endgroup
    \childdoctmp
  \fi
}
%    \end{macrocode}

% \macro{\childdocof}
% The command |\childdocof| redirects
% compilation to the main file |#1|.
%    \begin{macrocode}
\newcommand{\childdocof}[1]
{
  \childdocdisable
  \childdoctrue
  \includeonly{\childdocname}
  \def\jobname{#1}
  \def\childdocjob{#1}
  \input{#1}
}
%    \end{macrocode}

% \macro{\childdocby}
% The command |\childdocby| ....
%    \begin{macrocode}
\newcommand{\childdocby}[2][]
{
  \childdocdisable
  \childdoctrue
  \childdocmanualtrue
  \if?#1?\else
    \def\jobname{#2}
  \fi
  \def\childdocjob{#2}
  \input{#2}
  \endinput
}
%    \end{macrocode}

% \macro{\childdocforward}
% The command |\childdocforward| redirects
% compilation to the main file or
% (if the optional argument is given) a child file.
% Parameters are set as if the main file
% or a child file starting with |\childdocof| was compiled.
% Then compilation is handed over to the main file:
%    \begin{macrocode}
\newcommand{\childdocforward}[2][]
{
  \begingroup
    \if?#1?
      \def\childdoctmp
      {
        \def\childdocname{#2}
        \def\childdocjob{#2}
        \def\jobname{#2}
        \input{#2}
        \endinput
      }
    \else
      \def\childdoctmp
      {
        \childdocdisable
        \def\childdocname{#2}
        \childdoctrue
        \includeonly{#2}
        \def\childdocjob{#1}
        \def\jobname{#1}
        \input{#1}
        \endinput
      }
    \fi
    \expandafter
  \endgroup
  \childdoctmp
}
%    \end{macrocode}

% \macro{\childdocforwardprefix}
% The command |\childdocforwardprefix| redirects
% compilation to the main or a child file by means of a pattern.
% The prefix |#1| in the current filename is replaced by |#2|
% and the suffix of the current filename is kept
% (it is assumed that the filename does not contain the substring `|~~~|'
% which is used as a delimiter).
% Compilation is handed over to the new file by |\childdocforward|:
%    \begin{macrocode}
\newcommand{\childdocforwardprefix}[3][]
{
  \begingroup
    \def\childdocextract #2##1~~~{\def\childdoctmp{\childdocforward[#1]{#3##1}}}
    \expandafter\childdocextract\childdocname~~~
    \expandafter
  \endgroup
  \childdoctmp
}
%    \end{macrocode}

% \macro{\childdoc}
% The deprecated macro |\childdoc| is a legacy version of |\childdocmain|:
%    \begin{macrocode}
\newcommand{\childdoc}{\childdocmain}
%    \end{macrocode}

% \macro{\childdocredirect}
% The deprecated macro |\childdocredirect| is a legacy version
% of |\childdocforward| and |\childdocforwardprefix|:
%    \begin{macrocode}
\newcommand{\childdocredirect}[2][]
{
  \begingroup
    \if?#1?
      \def\childdoctmp{\childdocforward{#2}}
    \else
      \def\childdoctmp{\childdocforwardprefix{#1}{#2}}
    \fi
    \expandafter
  \endgroup
  \childdoctmp
}
%    \end{macrocode}

%\iffalse
%</package>
%\fi
%
\endinput
|\\
|\childdocmain{}|\\
\end{tabular}
\end{center}
at the very top of the main \LaTeX{} file,
in particular \emph{before} the |\documentclass| statement!
The argument of |\childdocmain| should be left empty
(but it must be present).

%%%%%%%%%%%%%%%%%%%%%%%%%%%%%%%%%%%%%%%%
\DescribeMacro{\childdocof}
Furthermore, add the commands
\begin{center}
\begin{tabular}{l}
|% \iffalse
%
% childdoc.dtx Copyright (C) 2017-2018 Niklas Beisert
%
% This work may be distributed and/or modified under the
% conditions of the LaTeX Project Public License, either version 1.3
% of this license or (at your option) any later version.
% The latest version of this license is in
%   http://www.latex-project.org/lppl.txt
% and version 1.3 or later is part of all distributions of LaTeX
% version 2005/12/01 or later.
%
% This work has the LPPL maintenance status `maintained'.
%
% The Current Maintainer of this work is Niklas Beisert.
%
% This work consists of the files childdoc.dtx and childdoc.ins
% and the derived files childdoc.def and cdocsamp.tex with
% cdocsch1.tex, cdocsch2.tex, cdocsdrf.tex, cdocsfn1.tex, cdocsfn2.tex.
%
%<package>\ifdefined\childdocmain\endinput\fi
%<package>\ProvidesFile{childdoc.def}[2018/12/30 v2.0 child document driver]
%<samplemain>\ProvidesFile{cdocsamp.tex}[2018/12/30 v2.0 sample for childdoc]
%<*driver>
%\ProvidesFile{childdoc.drv}[2018/12/30 v2.0 childdoc reference manual file]
\PassOptionsToClass{10pt,a4paper}{article}
\documentclass{ltxdoc}

\usepackage[margin=35mm]{geometry}
\usepackage{hyperref}
\usepackage{hyperxmp}
\usepackage[usenames]{color}

\hypersetup{colorlinks=true}
\hypersetup{pdfstartview=FitH}
\hypersetup{pdfpagemode=UseNone}
\hypersetup{pdfsource={}}
\hypersetup{pdflang={en-UK}}
\hypersetup{pdfcopyright={Copyright 2017-2018 Niklas Beisert.
  This work may be distributed and/or modified under the
  conditions of the LaTeX Project Public License, either version 1.3
  of this license or (at your option) any later version.}}
\hypersetup{pdflicenseurl={http://www.latex-project.org/lppl.txt}}
\hypersetup{pdfcontactaddress={ETH Zurich, ITP, HIT K,
  Wolfgang-Pauli-Strasse 27}}
\hypersetup{pdfcontactpostcode={8093}}
\hypersetup{pdfcontactcity={Zurich}}
\hypersetup{pdfcontactcountry={Switzerland}}
\hypersetup{pdfcontactemail={nbeisert@itp.phys.ethz.ch}}
\hypersetup{pdfcontacturl={http://people.phys.ethz.ch/\xmptilde nbeisert/}}

\newcommand{\secref}[1]{\hyperref[#1]{section \ref*{#1}}}

\parskip1ex
\parindent0pt
\let\olditemize\itemize
\def\itemize{\olditemize\parskip0pt}

\begin{document}

\title{The \textsf{childdoc} Package}
\hypersetup{pdftitle={The childdoc Package}}
\author{Niklas Beisert\\[2ex]
  Institut f\"ur Theoretische Physik\\
  Eidgen\"ossische Technische Hochschule Z\"urich\\
  Wolfgang-Pauli-Strasse 27, 8093 Z\"urich, Switzerland\\[1ex]
  \href{mailto:nbeisert@itp.phys.ethz.ch}
  {\texttt{nbeisert@itp.phys.ethz.ch}}}
\hypersetup{pdfauthor={Niklas Beisert}}
\hypersetup{pdfsubject={Manual for the LaTeX2e Package childdoc}}
\date{30 December 2018, \textsf{v2.0}}
\maketitle

\begin{abstract}\noindent
\textsf{childdoc} is a \LaTeXe{} package
that enables the direct compilation
of document sections included by |\include|
to individual files.
\end{abstract}

\begingroup
\parskip0ex
\tableofcontents
\endgroup

%%%%%%%%%%%%%%%%%%%%%%%%%%%%%%%%%%%%%%%%%%%%%%%%%%%%%%%%%%%%%%%%%%%%%%%%%%%%%%%%
%%%%%%%%%%%%%%%%%%%%%%%%%%%%%%%%%%%%%%%%%%%%%%%%%%%%%%%%%%%%%%%%%%%%%%%%%%%%%%%%
\section{Introduction}

\LaTeX{} provides a mechanism to structure a large document (such as a book)
into a main file and several child files (containing the chapters)
using the |\include| command.
This mechanism is beneficial for documents
which span hundreds of pages in order to
make the source file(s) more manageable.
Moreover, compilation can be restricted to
selected child files by means of the |\includeonly| command.
The latter feature can be used to reduce the compilation time while editing
(this was significantly more useful in the earlier days of \LaTeX{})
or to generate a smaller document which is easier to navigate.
Another application of |\includeonly| is to generate
documents consisting of selected parts of the complete document.

However, there are a few drawbacks of the plain |\include| mechanism:
\begin{itemize}
\item
The child files cannot be compiled on their own,
they can only be compiled via the main file.
A naive editing environment
(such as a text editor with an option
to have the current file processed by \LaTeX)
may require one to switch to the main file before compiling;
attempting to compile the child file produces errors.
\item
The main file must be modified (each time)
to adjust the |\includeonly| command
to the present needs. This easily leaves the main file in a messy state.
\item
The generated document will always carry the filename
of the main document. This is inconvenient if
several child files are to be compiled and
to be kept for distribution.
\end{itemize}

The present package provides a simple interface
to make child files individually compilable by \LaTeX{}.
Compiling a child file then has the same effect as compiling
the main file with an |\includeonly| command
to select the appropriate child.
Moreover the generated document will carry the name of the child
rather than the main file.
This resolves all three above issues.

This feature is meant to make the editing of books,
thesis documents and lecture notes somewhat more convenient.
However, the package can also be used efficiently for
composing a series of documents (such as exercise sheets)
which are typically distributed individually.
It then assists the author in generating the individual documents
(potentially in different versions)
as well as a document containing the collected series.
Another application is in developing style files
or other kinds of included material
where compilation of the style file could redirect
to a sample or test file.

%%%%%%%%%%%%%%%%%%%%%%%%%%%%%%%%%%%%%%%%%%%%%%%%%%%%%%%%%%%%%%%%%%%%%%%%%%%%%%%%
%%%%%%%%%%%%%%%%%%%%%%%%%%%%%%%%%%%%%%%%%%%%%%%%%%%%%%%%%%%%%%%%%%%%%%%%%%%%%%%%
\section{Usage}

First of all, the package \textsf{childdoc} is \emph{not} a standard
\LaTeXe{} |.sty| style file! Therefore it needs to be invoked in
a non-standard way.

%%%%%%%%%%%%%%%%%%%%%%%%%%%%%%%%%%%%%%%%%%%%%%%%%%%%%%%%%%%%%%%%%%%%%%%%%%%%%%%%
\subsection{Included Files}
\label{sec:include}

%%%%%%%%%%%%%%%%%%%%%%%%%%%%%%%%%%%%%%%%
\DescribeMacro{\childdocmain}
To use the package, add the commands
\begin{center}
\begin{tabular}{l}
|\input{childdoc.def}|\\
|\childdocmain{}|\\
\end{tabular}
\end{center}
at the very top of the main \LaTeX{} file,
in particular \emph{before} the |\documentclass| statement!
The argument of |\childdocmain| should be left empty
(but it must be present).

%%%%%%%%%%%%%%%%%%%%%%%%%%%%%%%%%%%%%%%%
\DescribeMacro{\childdocof}
Furthermore, add the commands
\begin{center}
\begin{tabular}{l}
|\input{childdoc.def}|\\
|\childdocof{|\textit{main}|}|\\
\end{tabular}
\end{center}
at the top of every child file \textit{child}
which is included by |\include{|\textit{child}|}|
from within the main file
(or at least for those files to be compiled individually).
The argument \textit{main} must be the filename of the main file.

There are a couple of
considerations in setting up the main and child documents:

%%%%%%%%%%%%%%%%%%%%%%%%%%%%%%%%%%%%%%%%
\paragraph{Restrictions.}

Please note the following restrictions:
\begin{itemize}
\item
|\childdocmain| must be called with one argument \textit{main}
to ensure compatibility with earlier version of the package.
It must either be empty (|\childdocmain{}|)
or precisely match the filename of the main file in which it is specified.
See \secref{sec:detection} for further information.
\item
The filename \textit{main} must be specified without the |.tex| extension.
\item
The filename \textit{main} is case sensitive
(even in case-insensitive file systems)
due to internal string comparison.
\item
The argument \textit{main} should be fully expanded, it cannot be a macro.
\item
Subdirectories and special characters should be avoided in filenames.
\item
The command |\childdocmain{|\textit{main}|}| must be followed by a whitespace.
It should not be followed immediately by another command
or by a comment mark `|%|'.
This is because the \TeX{} parser reads the token immediately following
the argument of |\childdocmain| and puts it
at the beginning of every child section;
however, a white\-space is ignored.
\end{itemize}

%%%%%%%%%%%%%%%%%%%%%%%%%%%%%%%%%%%%%%%%
\paragraph{Content of Main File.}

It is advisable to place all content in the child files included by |\include|.
Any output contained in the main file will appear in all child documents
unless suppressed manually;
it cannot be suppressed automatically by the |\includeonly| directive
and thus should normally be avoided.
A method to include some content in the main file
by means of conditional processing is described in \secref{sec:conditional}.

%%%%%%%%%%%%%%%%%%%%%%%%%%%%%%%%%%%%%%%%
\paragraph{Page Numbering.}

When only a part of the document is compiled,
the appropriate numbering of pages
(as well as other status parameters)
is determined from the |.aux| files.
The latter contain information from previous passes.
However this information needs to propagate through
all intermediate child documents.
Therefore the page numbering in child documents may well
be inconsistent until the complete document is compiled at least once.

A useful (if unconventional) way to always ensure a consistent
page numbering is to restart the numbering in each child document
and denote the pages by `\textit{child}|.|\textit{page}'
where \textit{child} represents the chapter/section number of the child file.
This can be achieved by the command
|\numberwithin{page}{|\textit{child}|}|
of the \textsf{amsmath} package
where \textit{child} can be |chapter| or |section|
depending on the chosen structuring.
Alternatively, one can modify the macro |\thepage| appropriately
and reset the counter |page| at the start of each child file.

%%%%%%%%%%%%%%%%%%%%%%%%%%%%%%%%%%%%%%%%%%%%%%%%%%%%%%%%%%%%%%%%%%%%%%%%%%%%%%%%
\subsection{Conditional Processing}
\label{sec:conditional}

The package provides a mechanism to compile different versions
of a document. To customise the versions further some conditional processing
can come in handy to distinguish which version is being compiled.
The package provides two macros to describe the compilation context:

%%%%%%%%%%%%%%%%%%%%%%%%%%%%%%%%%%%%%%%%
\DescribeMacro{\ifchilddoc}
The conditional |\ifchilddoc| distinguishes between the compilation of
child documents and the main document:
%
\begin{center}
|\ifchilddoc |\textit{child-code}| |[|\||else |\textit{main-code}]| \||fi|
\end{center}

%%%%%%%%%%%%%%%%%%%%%%%%%%%%%%%%%%%%%%%%
\DescribeMacro{\childdocname}
\DescribeMacro{\childdocjob}
The macro |\childdocname| contains the filename (without extension)
of the main or child file being processed.
Note that |\childdocjob| will always contain the name of the main file.

%%%%%%%%%%%%%%%%%%%%%%%%%%%%%%%%%%%%%%%%
\paragraph{Title Page.}

Conditional processing can be used to include a title or banner page
in the main document when proper precautions are taken.
Importantly, the code in the main file should ensure that the page counter
(as well as other status parameters which are stored in the |.aux| files)
takes the same value after the conditional processing.
Otherwise the page numbers may take divergent values
depending on which part is compiled.

For example, a title page could be declared by:
%
\begin{center}
\begin{tabular}{l}
|\ifchilddoc\||else|\\
|\addtocounter{page}{-1}|\\
\textit{code for title page}\\
|\newpage|\\
|\||fi|
\end{tabular}
\end{center}
%
A banner page for the child documents can be generated by:
%
\begin{center}
\begin{tabular}{l}
|\ifchilddoc|\\
|\addtocounter{page}{-1}|\\
\textit{code for banner page}\\
|\newpage|\\
|\||fi|
\end{tabular}
\end{center}
%
Here one could write a message such as:
\begin{center}
|This is the part \childdocname{} of \childdocjob{}.|
\end{center}

%%%%%%%%%%%%%%%%%%%%%%%%%%%%%%%%%%%%%%%%%%%%%%%%%%%%%%%%%%%%%%%%%%%%%%%%%%%%%%%%
\subsection{Flags}
\label{sec:flags}

The package makes it easy to generate different versions
of the main or child documents.
To this end compilation flags can be defined
and assigned different default values.
They will be particularly useful in conjunction
with the forwarding mechanism described in \secref{sec:forward}.

For example, it may be useful to have a flag |\version|
which can be set to |draft| or |final|.
The document source will contain some conditional code
depending on the value of |\version|.
Suppose further, the flag should default to |final| for the main file
and to |draft| for child files
which is a natural assignment for editing the document.
This is achieved by placing the following code
in the preamble of the main document
(below the |\childdocmain| directive):
%
\begin{center}
\begin{tabular}{l}
|\ifchilddoc|\\
|\providecommand{\version}{draft}|\\
|\||else|\\
|\providecommand{\version}{final}|\\
|\||fi|
\end{tabular}
\end{center}
%
The definition by |\providecommand| makes sure
that previous definitions are not overwritten.
Further statements |\providecommand{\version}{...}|
can thus be added before the above code to override it.

For the main file, one might add a line
(between |\childdocmain| and the above block)
%
\begin{center}
|%\ifchilddoc\||else\providecommand{\version}{draft}\||fi|
\end{center}
%
which can be uncommented to produce a draft version.
Likewise one can add a line to the very top of a child file
(above the |\childdocof{|\textit{main}|}| directive)
%
\begin{center}
|%\providecommand{\version}{final}|
\end{center}
%
which can be uncommented to produce the final version of this child document.

%%%%%%%%%%%%%%%%%%%%%%%%%%%%%%%%%%%%%%%%%%%%%%%%%%%%%%%%%%%%%%%%%%%%%%%%%%%%%%%%
\subsection{Forwarding}
\label{sec:forward}

Different versions of the main or child documents
using compilation flags as described in \secref{sec:flags}
can be (permanently) stored in different files
for convenient compilation, viewing and distribution.
To this end, the package defines a command
to pass on compilation to a different file:

%%%%%%%%%%%%%%%%%%%%%%%%%%%%%%%%%%%%%%%%
\DescribeMacro{\childdocforward}
The command |\childdocforward| redirects processing to
another source file:
%
\begin{center}
\begin{tabular}{l}
|\input{childdoc.def}|\\
|\childdocforward[|\textit{main}|]{|\textit{dest}|}|\\
\end{tabular}
\end{center}
%
The argument \textit{dest} is the destination file
(without extension).
It should be the main file or one of the child files.
Note that further \textsf{childdoc} directives
such as |\childdocof| and |\childdocforward|
in the indicated file will be processed in this form.
The optional argument \textit{main}
passes on directly to the main file \textit{main}
while pretending to compile the child \textit{dest}.
This form behaves as if \textit{dest}
issues |\childdocof{|\textit{main}|}| right away,
and no further \textsf{childdoc} directives will be processed.

%%%%%%%%%%%%%%%%%%%%%%%%%%%%%%%%%%%%%%%%
\DescribeMacro{\...prefix}
In the alternative form |\childdocforwardprefix|,
%
\begin{center}
\begin{tabular}{l}
|\input{childdoc.def}|\\
|\childdocforwardprefix[|\textit{main}|]{|\textit{prefix}|}{|\textit{dest}|}|
\end{tabular}
\end{center}
%
the destination file is determined by a pattern
depending on the current file:
To make this work, the current file must be called
`{\textit{prefix}\hspace{0.2em}\textit{suffix}}'
with \textit{prefix} matching precisely the argument.
Processing is then passed on to the file
`{\textit{dest}\hspace{0.2em}\textit{suffix}}'.
Surely, the same effect is achieved by
directly specifying the
argument `{\textit{dest}\hspace{0.2em}\textit{suffix}}'
in the first form.
However, that requires to set up a different file
for each child. With the alternative form of the command
all these files can have exactly the same content
which simplifies setting them up and maintaining them.

For example, the following file |draft.tex|
with a compilation flag |\version| as described in \secref{sec:flags}
compiles the main document as a draft:
%
\begin{center}
\begin{tabular}{l}
|\def\version{draft}|\\
|\input{childdoc.def}|\\
|\childdocforward{|\textit{main}|}|
\end{tabular}
\end{center}
%
Likewise, the following files |final|\textit{nn}|.tex|
compile the final version of the child document
|child|\textit{nn}|.tex|:
%
\begin{center}
\begin{tabular}{l}
|\def\version{final}|\\
|\input{childdoc.def}|\\
|\childdocforwardprefix{final}{child}|
\end{tabular}
\end{center}
%

Note that when several versions of a main file and/or of each child file
are to be generated, it may be convenient to set up a |Makefile| or
shell script to automatise the process.

%%%%%%%%%%%%%%%%%%%%%%%%%%%%%%%%%%%%%%%%%%%%%%%%%%%%%%%%%%%%%%%%%%%%%%%%%%%%%%%%
\subsection{Command Line Processing}
\label{sec:commandline}

The effect of redirection files can also be achieved by invoking
the \LaTeX{} compiler with a more elaborate command line.
Most conveniently this should be done as part
of a shell script or a |Makefile|.

When using \textsf{childdoc} in the main file, the following
command lines effectively perform a redirection
(note that depending on the shell being used,
backslashes may have to be doubled: `|\|' $\to$ `|\\|'):
%
\begin{center}
|... -jobname "|\textit{target}|" |\\|"|[\textit{flags}]%
|\input{childdoc.def}\childdocforward[|\textit{main}|]{|\textit{dest}|}"|
\end{center}
%
Here \textit{target} is the name of the output file,
\textit{main} is the name of the main file
and \textit{dest} is the name of the main or child file to be processed
(all filenames without extensions).
The optional argument \textit{main} can be omitted
if \textit{main} matches \textit{dest}.
Optionally, compilation \textit{flags} can be defined via |\def| commands.
This command line makes the \TeX{} engine believe
it is compiling the file \textit{target}
whose content is specified as the latter parameter.
The provided code then forwards the processing to
\textit{main} or \textit{dest} as described in \secref{sec:forward}.

%%%%%%%%%%%%%%%%%%%%%%%%%%%%%%%%%%%%%%%%%%%%%%%%%%%%%%%%%%%%%%%%%%%%%%%%%%%%%%%%
\subsection{Include by Input}
\label{sec:input}

Including child documents by |\include| has some restrictions by design.
Most notably, the content of a child document always occupies
its own set of pages; pages cannot be shared between child documents.
Usually, this behaviour makes perfect sense
because each child document contain an essential part of the document.
However, in some situations it may be desirable to compose
a document from a collection of parts
without having mandatory page breaks between then.
For this case, the package
provides a mechanism to include parts
by |\input| which can also be processed individually.
However, by construction this mechanism
requires manual handling of the content to be output.

%%%%%%%%%%%%%%%%%%%%%%%%%%%%%%%%%%%%%%%%
\DescribeMacro{\ifchilddocmanual}
The main file should be prepared as usual, see \secref{sec:include}.
However, the document body must make a distinction
between processing of an individual part and of the main document, e.g.:
%
\begin{center}
\begin{tabular}{l}
|\ifchilddocmanual|\\
|\input{\childdocname}|\\
|\||else|\\
\textit{document body with }|\input{|\textit{part}|}|\\
|\||fi|
\end{tabular}
\end{center}
%
The conditional |\ifchilddocmanual| is true whenever
a part to be included by |\input| is being compiled,
and the name of the part is stored in |\childdocname|.

%%%%%%%%%%%%%%%%%%%%%%%%%%%%%%%%%%%%%%%%
\DescribeMacro{\childdocby}
Each part to be included by |\input| should start with:
%
\begin{center}
\begin{tabular}{l}
|\input{childdoc.def}|\\
|\childdocby{|\textit{main}|}|\\
\end{tabular}
\end{center}
%
The directive |\childdocby| is similar to |\childdocof|
described in \secref{sec:include},
but the subsequent selection of content must be done manually.
To that end, both |\ifchilddoc| and |\ifchilddocmanual|
will be true upon processing of a part,
and the name of the part is stored in |\childdocname|.
Note that |\jobname| will be set to the filename of the current part
so that each part receives an individual |.aux| file
that does not interfere with the |.aux| file(s) of the main document.
This behaviour can be altered by the alternative form
|\childdocby[*]{|\textit{main}|}| (with a non-empty optional argument)
which uses the |.aux| file of the main document
by setting |\jobname| to \textit{main}.

%%%%%%%%%%%%%%%%%%%%%%%%%%%%%%%%%%%%%%%%%%%%%%%%%%%%%%%%%%%%%%%%%%%%%%%%%%%%%%%%
\subsection{Driver Development}
\label{sec:driver}

The \textsf{childdoc} mechanism can also be use for the development
of definition files such as \LaTeX{} styles or classes.
This case differs from the above setup with multiple parts
included by |\include| in that no |\includeonly| should be invoked.
This can be achieved by starting the include file
(before |\ProvidesPackage|) with:
%
\begin{center}
\begin{tabular}{l}
|\input{childdoc.def}|\\
|\childdocforward{|\textit{main}|}|\\
\end{tabular}
\end{center}
%
or alternatively with:
%
\begin{center}
\begin{tabular}{l}
|\input{childdoc.def}|\\
|\childdocby{|\textit{main}|}|\\
\end{tabular}
\end{center}
%
Both forms have slightly different effects as described above.
The main file is prepared as usual, see \secref{sec:include}.

%%%%%%%%%%%%%%%%%%%%%%%%%%%%%%%%%%%%%%%%%%%%%%%%%%%%%%%%%%%%%%%%%%%%%%%%%%%%%%%%
\subsection{Legacy Detection}
\label{sec:detection}

The directive |\childdocmain| in the main file can detect
whether the complete document or merely a child is to be compiled
even without using the directive |\childdocof|.
This method is deprecated because it is less robust
and there is no compelling reason to use it;
it is merely provided for backward compatibility
and it may be removed in future versions.

If the detection mechanism is to be used,
it is mandatory to correctly specify
the filename of the main file as the argument of |\childdocmain|:
%
\begin{center}
\begin{tabular}{l}
|\input{childdoc.def}|\\
|\childdocmain{|\textit{main}|}|\\
\end{tabular}
\end{center}
%
If |\jobname| does not match the argument \textit{main} of |\childdocmain|,
it is assumed that |\jobname| points to the child file to be compiled.
When using |\childdocmain| with the main file specified as argument,
it suffices to start a child file
with just |\input{|\textit{main}|}|
without loading of the package and using |\childdocof|.
If instead all processing is done
with the appropriate \textsf{childdoc} directives,
the argument of \textit{main} of |\childdocmain| can be empty.

An alternative version of the command line processing described
in \secref{sec:commandline} using the detection mechanism reads:
%
\begin{center}
|... -jobname "|\textit{target}|" "|[\textit{flags}]%
[|\def\jobname{|\textit{dest}|}|]|\input{|\textit{main}|}"|
\end{center}

%%%%%%%%%%%%%%%%%%%%%%%%%%%%%%%%%%%%%%%%%%%%%%%%%%%%%%%%%%%%%%%%%%%%%%%%%%%%%%%%
\subsection{Manual Code}
\label{sec:manual}

In case one cannot be certain whether the definitions file |childdoc.def|
is installed on the target \TeX{} distribution
and one prefers not to ship it,
it is conceivable to paste a few relevant commands into the sources.

To that end, drop all statements |\input{childdoc.def}|
and perform the replacements as outlined below.
Instead of |\childdocmain{|\textit{main}|}| add the following code
to the top of the main file:
%
\begin{center}
\begin{tabular}{l}
|\||ifdefined\childdocname\endinput\||fi\newif\ifchilddoc|\\
|\edef\childdocname{\scantokens\expandafter{\jobname\noexpand}}|\\
|\def\childdocmain{|\textit{main}|}\||ifx\childdocmain\childdocname\||else|\\
|\childdoctrue\includeonly{\childdocname}\let\jobname\childdocmain\||fi|\\
\end{tabular}
\end{center}
%
Instead of |\childdocof{|\textit{main}|}| just include the main file
at the top of each child file:
%
\begin{center}
|\input{|\textit{main}|}|
\end{center}
%
A simple redirection |\childdocforward{|\textit{dest}|}| is achieved by:
%
\begin{center}
|\def\jobname{|\textit{dest}|}\input{\jobname}|
\end{center}
%
The redirection with prefix
|\childdocforwardprefix[|\textit{prefix}|]{|\textit{dest}|}|
is accomplished by:
%
\begin{center}
\begin{tabular}{l}
|{\edef\jobname{\scantokens\expandafter{\jobname\noexpand}}|\\
|\def\redirectjob |\textit{prefix}|#1~~~{\gdef\jobname{|\textit{dest}|#1}}|\\
|\expandafter\redirectjob\jobname~~~}\input{\jobname}|
\end{tabular}
\end{center}

In an alternative approach,
child documents can be compiled by a specific command line
without additional code or specific definitions:
%
\begin{center}
|... -jobname "|\textit{target}|" "|[\textit{flags}]%
|\includeonly{|\textit{dest}|}\input{|\textit{main}|}"|
\end{center}
%

%%%%%%%%%%%%%%%%%%%%%%%%%%%%%%%%%%%%%%%%%%%%%%%%%%%%%%%%%%%%%%%%%%%%%%%%%%%%%%%%
%%%%%%%%%%%%%%%%%%%%%%%%%%%%%%%%%%%%%%%%%%%%%%%%%%%%%%%%%%%%%%%%%%%%%%%%%%%%%%%%
\section{Information}

%%%%%%%%%%%%%%%%%%%%%%%%%%%%%%%%%%%%%%%%%%%%%%%%%%%%%%%%%%%%%%%%%%%%%%%%%%%%%%%%
\subsection{Copyright}

Copyright \copyright{} 2017--2018 Niklas Beisert

This work may be distributed and/or modified under the
conditions of the \LaTeX{} Project Public License, either version 1.3
of this license or (at your option) any later version.
The latest version of this license is in
  \url{http://www.latex-project.org/lppl.txt}
and version 1.3 or later is part of all distributions of \LaTeX{}
version 2005/12/01 or later.

This work has the LPPL maintenance status `maintained'.

The Current Maintainer of this work is Niklas Beisert.

This work consists of the files |README.txt|, |childdoc.ins| and |childdoc.dtx|
as well as the derived files |childdoc.def|, |cdocsamp.tex|
with |cdocsch1.tex|, |cdocsch2.tex|, |cdocspt3.tex|, |cdocspt4.tex|,
|cdocsdrf.tex|, |cdocsfn1.tex|, |cdocsfn2.tex|
as well as |childdoc.pdf|.

%%%%%%%%%%%%%%%%%%%%%%%%%%%%%%%%%%%%%%%%%%%%%%%%%%%%%%%%%%%%%%%%%%%%%%%%%%%%%%%%
\subsection{Files and Installation}

The package consists of the files:
%
\begin{center}
\begin{tabular}{ll}
    |README.txt|   & readme file \\
    |childdoc.ins| & installation file \\
    |childdoc.dtx| & source file \\
    |childdoc.def| & definition file \\
    |cdocsamp.tex| & sample main file \\
    |cdocsch1.tex| & sample include file \\
    |cdocsch2.tex| & sample include file \\
    |cdocspt3.tex| & sample part file \\
    |cdocspt4.tex| & sample part file \\
    |cdocsdrf.tex| & sample redirection file \\
    |cdocsfn1.tex| & sample redirection file \\
    |cdocsfn2.tex| & sample redirection file \\
    |childdoc.pdf| & manual
\end{tabular}
\end{center}
%
The distribution consists of the files
|README.txt|, |childdoc.ins| and |childdoc.dtx|.
%
\begin{itemize}
\item
Run (pdf)\LaTeX{} on |childdoc.dtx|
to compile the manual |childdoc.pdf| (this file).
\item
Run \LaTeX{} on |childdoc.ins| to create the definitions file |childdoc.def|
and the sample |cdocsamp.tex| with include files
|cdocsch1.tex|, |cdocsch2.tex|, |cdocspt3.tex|, |cdocspt4.tex|,
|cdocsdrf.tex|, |cdocsfn1.tex|, |cdocsfn2.tex|.
Then copy the file |childdoc.def| to an appropriate directory of your \LaTeX{}
distribution, e.g.\ \textit{texmf-root}|/tex/latex/childdoc|.
\end{itemize}

%%%%%%%%%%%%%%%%%%%%%%%%%%%%%%%%%%%%%%%%%%%%%%%%%%%%%%%%%%%%%%%%%%%%%%%%%%%%%%%%
\subsection{Related CTAN Packages}

There are several other packages which offer a similar functionality:
%
\begin{itemize}
\item
The packages
\href{http://ctan.org/pkg/docmute}{\textsf{docmute}},
\href{http://ctan.org/pkg/includex}{\textsf{includex}} and
\href{http://ctan.org/pkg/standalone}{\textsf{standalone}}
provide commands to include only the document body of
a child file thus allowing both files to be compiled individually.
\item
The packages \href{http://ctan.org/pkg/subdocs}{\textsf{subdocs}}
and \href{http://ctan.org/pkg/subfiles}{\textsf{subfiles}}
provide structures in which the main and child documents can be
encapsulated and allowing them to be compiled individually.
The inclusion mechanism is different from the conventional |\include|.
\item
The package \href{http://ctan.org/pkg/combine}{\textsf{combine}}
is an elaborate solution to combine several documents into one.
\end{itemize}
%
See also the CTAN topic \href{http://ctan.org/topic/subdocs}{\textsf{subdocs}}
for further related packages.
The present package differs from the above solutions in that
a document structure constructed with the conventional |\include| mechanism
just needs two extra commands at the top of every file
such that all constituent files can be compiled individually.

%%%%%%%%%%%%%%%%%%%%%%%%%%%%%%%%%%%%%%%%%%%%%%%%%%%%%%%%%%%%%%%%%%%%%%%%%%%%%%%%
%\subsection{Feature Suggestions}
%
%The following is a list of features which may be useful for future
%versions of this package:
%%
%\begin{itemize}
%\item
%\ldots
%\end{itemize}

%%%%%%%%%%%%%%%%%%%%%%%%%%%%%%%%%%%%%%%%%%%%%%%%%%%%%%%%%%%%%%%%%%%%%%%%%%%%%%%%
\subsection{Revision History}

%%%%%%%%%%%%%%%%%%%%%%%%%%%%%%%%%%%%%%%%
\paragraph{v2.0:} 2018/12/30

\begin{itemize}
\item
immediate forward processing
\item
added |\childdocby| mechanism
\item
manual restructured
\end{itemize}

%%%%%%%%%%%%%%%%%%%%%%%%%%%%%%%%%%%%%%%%
\paragraph{v1.6:} 2018/01/17

\begin{itemize}
\item
application for development of include files
\item
corrections to manual
\end{itemize}

%%%%%%%%%%%%%%%%%%%%%%%%%%%%%%%%%%%%%%%%
\paragraph{v1.5:} 2017/05/21

\begin{itemize}
\item
more complete structuring introduced
\item
|\childdocof| introduced
\item
|\childdoc| renamed to |\childdocmain|
\item
|\childredirect| renamed to |\childdocforward| and |\childdocforwardprefix|
and functionality expanded
\end{itemize}

%%%%%%%%%%%%%%%%%%%%%%%%%%%%%%%%%%%%%%%%
\paragraph{v1.0:} 2017/04/27

\begin{itemize}
\item
manual and install package
\item
first version published on CTAN
\end{itemize}

%%%%%%%%%%%%%%%%%%%%%%%%%%%%%%%%%%%%%%%%
\paragraph{v0.6:} 2017/04/26

\begin{itemize}
\item
redirection mechanism added
\end{itemize}

%%%%%%%%%%%%%%%%%%%%%%%%%%%%%%%%%%%%%%%%
\paragraph{v0.5:} 2017/04/26

\begin{itemize}
\item
functionality in definition file
\end{itemize}


%%%%%%%%%%%%%%%%%%%%%%%%%%%%%%%%%%%%%%%%%%%%%%%%%%%%%%%%%%%%%%%%%%%%%%%%%%%%%%%%
%%%%%%%%%%%%%%%%%%%%%%%%%%%%%%%%%%%%%%%%%%%%%%%%%%%%%%%%%%%%%%%%%%%%%%%%%%%%%%%%
%%%%%%%%%%%%%%%%%%%%%%%%%%%%%%%%%%%%%%%%%%%%%%%%%%%%%%%%%%%%%%%%%%%%%%%%%%%%%%%%
\appendix

\settowidth\MacroIndent{\rmfamily\scriptsize 000\ }

 \DocInput{childdoc.dtx}

\end{document}
%</driver>
% \fi
%
% %%%%%%%%%%%%%%%%%%%%%%%%%%%%%%%%%%%%%%%%%%%%%%%%%%%%%%%%%%%%%%%%%%%%%%%%%%%%%%
% %%%%%%%%%%%%%%%%%%%%%%%%%%%%%%%%%%%%%%%%%%%%%%%%%%%%%%%%%%%%%%%%%%%%%%%%%%%%%%
% \section{Sample}
%\iffalse
%<*samplemain>
%\fi
%
% The following presents a sample document
% with two chapters, two parts, a title page,
% a compile flag as well as three forwarding files to set the flag.
% It consists of eight |.tex| files:
% \begin{center}
% \begin{tabular}{ll}
% |cdocsamp.tex|&main file\\
% |cdocsch1.tex|&include file for chapter 1\\
% |cdocsch2.tex|&include file for chapter 2\\
% |cdocspt3.tex|&include file for part 3\\
% |cdocspt4.tex|&include file for part 4\\
% |cdocsdrf.tex|&forwarding file for main file in draft mode\\
% |cdocsfi1.tex|&forwarding file for final version of chapter 1\\
% |cdocsfi2.tex|&forwarding file for final version of chapter 2\\
% \end{tabular}
% \end{center}
% Each of the eight files can be compiled directly by the \LaTeX{} compiler.
%
% %%%%%%%%%%%%%%%%%%%%%%%%%%%%%%%%%%%%%%
% \paragraph{Main File.}
%
% The main file is called |cdocsamp.tex|.
%
% Load the \textsf{childdoc} definitions and
% declare the filename for the main document:
%    \begin{macrocode}
\input{childdoc.def}
\childdocmain{}
%    \end{macrocode}

% Optional override for |\version| flag:
%    \begin{macrocode}
%%\ifchilddoc\else\providecommand{\version}{draft}\fi
%    \end{macrocode}

% Define the default values for the |\version| flag
% (|final| for the main file and |draft| for childs):
%    \begin{macrocode}
\ifchilddoc
\providecommand{\version}{draft}
\else
\providecommand{\version}{final}
\fi
%    \end{macrocode}

% Load the standard document class:
%    \begin{macrocode}
\documentclass[12pt]{article}
%    \end{macrocode}

% Start the document body:
%    \begin{macrocode}
\begin{document}
%    \end{macrocode}

% Declare a title page.
% Print title, part of document being processed and version flag:
%    \begin{macrocode}
\addtocounter{page}{-1}
\begin{center}
{\LARGE\bfseries{}childdoc example\par}
\vspace{1cm}
\ifchilddoc
\ifchilddocmanual part\else chapter\fi:
`\childdocname' of `\childdocjob'\par
\else
main document: `\childdocjob'\par
\fi
version: \version\par
\end{center}
\newpage
%    \end{macrocode}

% Manually include selected file,
% otherwise process as usual:
%    \begin{macrocode}
\ifchilddocmanual
\section*{part `\childdocname'}
\input{\childdocname}
\else
%    \end{macrocode}

% Include the two chapters:
%    \begin{macrocode}
\include{cdocsch1}
\include{cdocsch2}
%    \end{macrocode}

% Include the two parts unless only chapters should be displayed:
%    \begin{macrocode}
\ifchilddoc\else
\section{part three}
\input{cdocspt3}
\section{part four}
\input{cdocspt4}
\fi
%    \end{macrocode}

% Process as usual until here:
%    \begin{macrocode}
\fi
%    \end{macrocode}

% End of document body:
%    \begin{macrocode}
\end{document}
%    \end{macrocode}
%\iffalse
%</samplemain>
%\fi
%
% %%%%%%%%%%%%%%%%%%%%%%%%%%%%%%%%%%%%%%
% \paragraph{Chapter Include Files.}
%
% The include files are called |cdocsch1.tex| and |cdocsch2.tex|.
%
%\iffalse
%<*samplechap1|samplechap2>
%\fi

% Optional override for |\version| flag:
%    \begin{macrocode}
%%\providecommand{\version}{final}
%    \end{macrocode}

% Include the main document:
%    \begin{macrocode}
\input{childdoc.def}
\childdocof{cdocsamp}
%    \end{macrocode}

%\iffalse
%</samplechap1|samplechap2>
%\fi
%
%\iffalse
%<*samplechap1>
%\fi
% Some text for chapter 1:
%    \begin{macrocode}
\section{one}
some text in chapter one
%    \end{macrocode}

%\iffalse
%</samplechap1>
%\fi
% Some text for chapter 2:
%\iffalse
%<*samplechap2>
%\fi
%    \begin{macrocode}
\section{two}
more text in chapter two
%    \end{macrocode}

%\iffalse
%</samplechap2>
%\fi
%
% %%%%%%%%%%%%%%%%%%%%%%%%%%%%%%%%%%%%%%
% \paragraph{Part Include Files.}
%
% The include files are called |cdocspt3.tex| and |cdocspt4.tex|.
%
%\iffalse
%<*samplepart3|samplepart4>
%\fi

% Optional override for |\version| flag:
%    \begin{macrocode}
%%\providecommand{\version}{final}
%    \end{macrocode}

% Include the main document:
%    \begin{macrocode}
\input{childdoc.def}
\childdocby{cdocsamp}
%    \end{macrocode}

%\iffalse
%</samplepart3|samplepart4>
%\fi
%
%\iffalse
%<*samplepart3>
%\fi
% Some text for part 3:
%    \begin{macrocode}
some text in part three
%    \end{macrocode}

%\iffalse
%</samplepart3>
%\fi
% Some text for part 4:
%\iffalse
%<*samplepart4>
%\fi
%    \begin{macrocode}
more text in part four
%    \end{macrocode}

%\iffalse
%</samplepart4>
%\fi
%
% %%%%%%%%%%%%%%%%%%%%%%%%%%%%%%%%%%%%%%
% \paragraph{Forwarding for a Complete Draft.}
%
% The following forwarding file |cdocsdrf.tex|
% compiles the main document in draft mode:
%\iffalse
%<*sampledraft>
%\fi
%    \begin{macrocode}
\def\version{draft}
\input{childdoc.def}
\childdocforward{cdocsamp}
%    \end{macrocode}

%\iffalse
%</sampledraft>
%\fi
%
% %%%%%%%%%%%%%%%%%%%%%%%%%%%%%%%%%%%%%%
% \paragraph{Forwarding for Final Version of the Chapters.}
%
% The following forwarding files |cdocsfn1.tex| and |cdocsfn2.tex|
% (with identical content)
% compile the final versions of the child documents
% |cdocsch1.tex| and |cdocsch2.tex|, respectively:
%\iffalse
%<*samplefinal>
%\fi
%    \begin{macrocode}
\def\version{final}
\input{childdoc.def}
\childdocforwardprefix[cdocsamp]{cdocsfn}{cdocsch}
%    \end{macrocode}

%\iffalse
%</samplefinal>
%\fi
%
% %%%%%%%%%%%%%%%%%%%%%%%%%%%%%%%%%%%%%%
% \paragraph{Command Line Processing.}
%
% The following three command lines generate the output files
% |cdocscld|, |cdocscl1| and |cdocscl2|
% which should be identical to
% |cdocsdrf|, |cdocsch1| and |cdocsfn2|, respectively:
% \begin{center}
% \begin{tabular}{l}
% |latex -jobname cdocscld \|\\
% |  "\def\version{draft}\input{childdoc.def}\childdocforward{cdocsamp}"|\\
% |latex -jobname cdocscl1 \|\\
% |  "\input{childdoc.def}\childdocforward[cdocsamp]{cdocsch1}"|\\
% |latex -jobname cdocscl2 \|\\
% |  "\def\version{final}\input{childdoc.def}\childdocforward{cdocsch2}"|
% \end{tabular}
% \end{center}
% Note that the trailing backslash on each first line
% merely continues the input to the second line
% (for convenient cut ant paste).
% Furthermore, the command |latex| can be replaced by any
% of its alternative versions such as |pdflatex|.
%
% %%%%%%%%%%%%%%%%%%%%%%%%%%%%%%%%%%%%%%%%%%%%%%%%%%%%%%%%%%%%%%%%%%%%%%%%%%%%%%
% %%%%%%%%%%%%%%%%%%%%%%%%%%%%%%%%%%%%%%%%%%%%%%%%%%%%%%%%%%%%%%%%%%%%%%%%%%%%%%
% \section{Implementation}
%\iffalse
%<*package>
%\fi
%
% This section describes the definitions file |childdoc.def|.

% The definitions cannot be loaded using |\usepackage| or |\RequirePackage|
% which has a mechanism to prevent loading a style file more than once.
% When loading the definitions by means of |\input|
% multiple instances have to be prevented manually:
%\iffalse
%This code needs to be before the `\ProvidesFile' directive
%which is defined at the beginning of this file.
%Therefore it is also placed there and commented out here.
%</package>
%<*discard>
%\fi
%    \begin{macrocode}
\ifdefined\childdocmain\endinput\fi
%    \end{macrocode}
%\iffalse
%</discard>
%<*package>
%\fi
%
% \macro{\ifchilddoc}
% \macro{\ifchilddocmanual}
% The conditional |\ifchilddoc| tells whether a
% child (true) or main (false) document is being compiled.
% The conditional |\ifchilddocmanual| tells whether
% the |\includeonly| mechanism is used (false) or
% the selection of child files must be performed manually (true).
% The definitions initialise to false:
%    \begin{macrocode}
\newif\ifchilddoc
\newif\ifchilddocmanual
%    \end{macrocode}

% \macro{\childdocname}
% \macro{\childdocjob}
% The macro |\childdocname| stores the name of the main document
% to be compiled. The macro |\childdocjob| stores the name of
% the document on which the \LaTeX{} compiler was originally invoked.
% The content of |\jobname| cannot be compared
% to filenames specified in the source due to different catcodes.
% The following code rescans |\jobname|, stores the result
% in |\childdocname| and saves a copy in |\childdocjob|:
%    \begin{macrocode}
\edef\childdocname{\scantokens\expandafter{\jobname\noexpand}}
\let\childdocjob\childdocname
%    \end{macrocode}

% \macro{\childdocdisable}
% The macro |\childdocdisable| prevents the main file
% from being processed more than once.
% At this stage, the main document command |\childdocmain|
% is assumed to be called once again where it should do nothing.
% Any subsequent call to it should prevent
% a secondary processing of the main document
% It overwrites the forwarding commands
% |\childdocof| and |\childdocforward|
% with empty macros to prevent further inclusions of the main document:
%    \begin{macrocode}
\newcommand{\childdocdisable}
{
  \renewcommand{\childdocmain}[1]{\renewcommand{\childdocmain}[1]{\endinput}}
  \renewcommand{\childdocof}[1]{}
  \renewcommand{\childdocby}[2][]{}
  \renewcommand{\childdocforward}[2][]{}
  \renewcommand{\childdocdisable}{}
}
%    \end{macrocode}

% \macro{\childdocmain}
% The macro |\childdocmain| is to be called at the top of the main file
% with nothing or the main filename (without extension) as argument.
% First, it breaks loops.
% If the argument is not empty and does not match |\childdocname|
% (which is set by the first inclusion of |childdoc.def|),
% |\ifchilddoc| is set to true, |\includeonly| is applied to the child file
% and |\jobname| is set to the main file
% (for proper handling of |.aux| files):
%    \begin{macrocode}
\newcommand{\childdocmain}[1]
{
  \childdocdisable\childdocmain{}
  \if?#1?\else
    \begingroup
      \def\childdoctmp{#1}
      \ifx\childdoctmp\childdocname
        \def\childdoctmp{}
      \else
        \def\childdoctmp
        {
          \childdoctrue
          \includeonly{\childdocname}
          \def\childdocjob{#1}
          \def\jobname{#1}
        }
      \fi
      \expandafter
    \endgroup
    \childdoctmp
  \fi
}
%    \end{macrocode}

% \macro{\childdocof}
% The command |\childdocof| redirects
% compilation to the main file |#1|.
%    \begin{macrocode}
\newcommand{\childdocof}[1]
{
  \childdocdisable
  \childdoctrue
  \includeonly{\childdocname}
  \def\jobname{#1}
  \def\childdocjob{#1}
  \input{#1}
}
%    \end{macrocode}

% \macro{\childdocby}
% The command |\childdocby| ....
%    \begin{macrocode}
\newcommand{\childdocby}[2][]
{
  \childdocdisable
  \childdoctrue
  \childdocmanualtrue
  \if?#1?\else
    \def\jobname{#2}
  \fi
  \def\childdocjob{#2}
  \input{#2}
  \endinput
}
%    \end{macrocode}

% \macro{\childdocforward}
% The command |\childdocforward| redirects
% compilation to the main file or
% (if the optional argument is given) a child file.
% Parameters are set as if the main file
% or a child file starting with |\childdocof| was compiled.
% Then compilation is handed over to the main file:
%    \begin{macrocode}
\newcommand{\childdocforward}[2][]
{
  \begingroup
    \if?#1?
      \def\childdoctmp
      {
        \def\childdocname{#2}
        \def\childdocjob{#2}
        \def\jobname{#2}
        \input{#2}
        \endinput
      }
    \else
      \def\childdoctmp
      {
        \childdocdisable
        \def\childdocname{#2}
        \childdoctrue
        \includeonly{#2}
        \def\childdocjob{#1}
        \def\jobname{#1}
        \input{#1}
        \endinput
      }
    \fi
    \expandafter
  \endgroup
  \childdoctmp
}
%    \end{macrocode}

% \macro{\childdocforwardprefix}
% The command |\childdocforwardprefix| redirects
% compilation to the main or a child file by means of a pattern.
% The prefix |#1| in the current filename is replaced by |#2|
% and the suffix of the current filename is kept
% (it is assumed that the filename does not contain the substring `|~~~|'
% which is used as a delimiter).
% Compilation is handed over to the new file by |\childdocforward|:
%    \begin{macrocode}
\newcommand{\childdocforwardprefix}[3][]
{
  \begingroup
    \def\childdocextract #2##1~~~{\def\childdoctmp{\childdocforward[#1]{#3##1}}}
    \expandafter\childdocextract\childdocname~~~
    \expandafter
  \endgroup
  \childdoctmp
}
%    \end{macrocode}

% \macro{\childdoc}
% The deprecated macro |\childdoc| is a legacy version of |\childdocmain|:
%    \begin{macrocode}
\newcommand{\childdoc}{\childdocmain}
%    \end{macrocode}

% \macro{\childdocredirect}
% The deprecated macro |\childdocredirect| is a legacy version
% of |\childdocforward| and |\childdocforwardprefix|:
%    \begin{macrocode}
\newcommand{\childdocredirect}[2][]
{
  \begingroup
    \if?#1?
      \def\childdoctmp{\childdocforward{#2}}
    \else
      \def\childdoctmp{\childdocforwardprefix{#1}{#2}}
    \fi
    \expandafter
  \endgroup
  \childdoctmp
}
%    \end{macrocode}

%\iffalse
%</package>
%\fi
%
\endinput
|\\
|\childdocof{|\textit{main}|}|\\
\end{tabular}
\end{center}
at the top of every child file \textit{child}
which is included by |\include{|\textit{child}|}|
from within the main file
(or at least for those files to be compiled individually).
The argument \textit{main} must be the filename of the main file.

There are a couple of
considerations in setting up the main and child documents:

%%%%%%%%%%%%%%%%%%%%%%%%%%%%%%%%%%%%%%%%
\paragraph{Restrictions.}

Please note the following restrictions:
\begin{itemize}
\item
|\childdocmain| must be called with one argument \textit{main}
to ensure compatibility with earlier version of the package.
It must either be empty (|\childdocmain{}|)
or precisely match the filename of the main file in which it is specified.
See \secref{sec:detection} for further information.
\item
The filename \textit{main} must be specified without the |.tex| extension.
\item
The filename \textit{main} is case sensitive
(even in case-insensitive file systems)
due to internal string comparison.
\item
The argument \textit{main} should be fully expanded, it cannot be a macro.
\item
Subdirectories and special characters should be avoided in filenames.
\item
The command |\childdocmain{|\textit{main}|}| must be followed by a whitespace.
It should not be followed immediately by another command
or by a comment mark `|%|'.
This is because the \TeX{} parser reads the token immediately following
the argument of |\childdocmain| and puts it
at the beginning of every child section;
however, a white\-space is ignored.
\end{itemize}

%%%%%%%%%%%%%%%%%%%%%%%%%%%%%%%%%%%%%%%%
\paragraph{Content of Main File.}

It is advisable to place all content in the child files included by |\include|.
Any output contained in the main file will appear in all child documents
unless suppressed manually;
it cannot be suppressed automatically by the |\includeonly| directive
and thus should normally be avoided.
A method to include some content in the main file
by means of conditional processing is described in \secref{sec:conditional}.

%%%%%%%%%%%%%%%%%%%%%%%%%%%%%%%%%%%%%%%%
\paragraph{Page Numbering.}

When only a part of the document is compiled,
the appropriate numbering of pages
(as well as other status parameters)
is determined from the |.aux| files.
The latter contain information from previous passes.
However this information needs to propagate through
all intermediate child documents.
Therefore the page numbering in child documents may well
be inconsistent until the complete document is compiled at least once.

A useful (if unconventional) way to always ensure a consistent
page numbering is to restart the numbering in each child document
and denote the pages by `\textit{child}|.|\textit{page}'
where \textit{child} represents the chapter/section number of the child file.
This can be achieved by the command
|\numberwithin{page}{|\textit{child}|}|
of the \textsf{amsmath} package
where \textit{child} can be |chapter| or |section|
depending on the chosen structuring.
Alternatively, one can modify the macro |\thepage| appropriately
and reset the counter |page| at the start of each child file.

%%%%%%%%%%%%%%%%%%%%%%%%%%%%%%%%%%%%%%%%%%%%%%%%%%%%%%%%%%%%%%%%%%%%%%%%%%%%%%%%
\subsection{Conditional Processing}
\label{sec:conditional}

The package provides a mechanism to compile different versions
of a document. To customise the versions further some conditional processing
can come in handy to distinguish which version is being compiled.
The package provides two macros to describe the compilation context:

%%%%%%%%%%%%%%%%%%%%%%%%%%%%%%%%%%%%%%%%
\DescribeMacro{\ifchilddoc}
The conditional |\ifchilddoc| distinguishes between the compilation of
child documents and the main document:
%
\begin{center}
|\ifchilddoc |\textit{child-code}| |[|\||else |\textit{main-code}]| \||fi|
\end{center}

%%%%%%%%%%%%%%%%%%%%%%%%%%%%%%%%%%%%%%%%
\DescribeMacro{\childdocname}
\DescribeMacro{\childdocjob}
The macro |\childdocname| contains the filename (without extension)
of the main or child file being processed.
Note that |\childdocjob| will always contain the name of the main file.

%%%%%%%%%%%%%%%%%%%%%%%%%%%%%%%%%%%%%%%%
\paragraph{Title Page.}

Conditional processing can be used to include a title or banner page
in the main document when proper precautions are taken.
Importantly, the code in the main file should ensure that the page counter
(as well as other status parameters which are stored in the |.aux| files)
takes the same value after the conditional processing.
Otherwise the page numbers may take divergent values
depending on which part is compiled.

For example, a title page could be declared by:
%
\begin{center}
\begin{tabular}{l}
|\ifchilddoc\||else|\\
|\addtocounter{page}{-1}|\\
\textit{code for title page}\\
|\newpage|\\
|\||fi|
\end{tabular}
\end{center}
%
A banner page for the child documents can be generated by:
%
\begin{center}
\begin{tabular}{l}
|\ifchilddoc|\\
|\addtocounter{page}{-1}|\\
\textit{code for banner page}\\
|\newpage|\\
|\||fi|
\end{tabular}
\end{center}
%
Here one could write a message such as:
\begin{center}
|This is the part \childdocname{} of \childdocjob{}.|
\end{center}

%%%%%%%%%%%%%%%%%%%%%%%%%%%%%%%%%%%%%%%%%%%%%%%%%%%%%%%%%%%%%%%%%%%%%%%%%%%%%%%%
\subsection{Flags}
\label{sec:flags}

The package makes it easy to generate different versions
of the main or child documents.
To this end compilation flags can be defined
and assigned different default values.
They will be particularly useful in conjunction
with the forwarding mechanism described in \secref{sec:forward}.

For example, it may be useful to have a flag |\version|
which can be set to |draft| or |final|.
The document source will contain some conditional code
depending on the value of |\version|.
Suppose further, the flag should default to |final| for the main file
and to |draft| for child files
which is a natural assignment for editing the document.
This is achieved by placing the following code
in the preamble of the main document
(below the |\childdocmain| directive):
%
\begin{center}
\begin{tabular}{l}
|\ifchilddoc|\\
|\providecommand{\version}{draft}|\\
|\||else|\\
|\providecommand{\version}{final}|\\
|\||fi|
\end{tabular}
\end{center}
%
The definition by |\providecommand| makes sure
that previous definitions are not overwritten.
Further statements |\providecommand{\version}{...}|
can thus be added before the above code to override it.

For the main file, one might add a line
(between |\childdocmain| and the above block)
%
\begin{center}
|%\ifchilddoc\||else\providecommand{\version}{draft}\||fi|
\end{center}
%
which can be uncommented to produce a draft version.
Likewise one can add a line to the very top of a child file
(above the |\childdocof{|\textit{main}|}| directive)
%
\begin{center}
|%\providecommand{\version}{final}|
\end{center}
%
which can be uncommented to produce the final version of this child document.

%%%%%%%%%%%%%%%%%%%%%%%%%%%%%%%%%%%%%%%%%%%%%%%%%%%%%%%%%%%%%%%%%%%%%%%%%%%%%%%%
\subsection{Forwarding}
\label{sec:forward}

Different versions of the main or child documents
using compilation flags as described in \secref{sec:flags}
can be (permanently) stored in different files
for convenient compilation, viewing and distribution.
To this end, the package defines a command
to pass on compilation to a different file:

%%%%%%%%%%%%%%%%%%%%%%%%%%%%%%%%%%%%%%%%
\DescribeMacro{\childdocforward}
The command |\childdocforward| redirects processing to
another source file:
%
\begin{center}
\begin{tabular}{l}
|% \iffalse
%
% childdoc.dtx Copyright (C) 2017-2018 Niklas Beisert
%
% This work may be distributed and/or modified under the
% conditions of the LaTeX Project Public License, either version 1.3
% of this license or (at your option) any later version.
% The latest version of this license is in
%   http://www.latex-project.org/lppl.txt
% and version 1.3 or later is part of all distributions of LaTeX
% version 2005/12/01 or later.
%
% This work has the LPPL maintenance status `maintained'.
%
% The Current Maintainer of this work is Niklas Beisert.
%
% This work consists of the files childdoc.dtx and childdoc.ins
% and the derived files childdoc.def and cdocsamp.tex with
% cdocsch1.tex, cdocsch2.tex, cdocsdrf.tex, cdocsfn1.tex, cdocsfn2.tex.
%
%<package>\ifdefined\childdocmain\endinput\fi
%<package>\ProvidesFile{childdoc.def}[2018/12/30 v2.0 child document driver]
%<samplemain>\ProvidesFile{cdocsamp.tex}[2018/12/30 v2.0 sample for childdoc]
%<*driver>
%\ProvidesFile{childdoc.drv}[2018/12/30 v2.0 childdoc reference manual file]
\PassOptionsToClass{10pt,a4paper}{article}
\documentclass{ltxdoc}

\usepackage[margin=35mm]{geometry}
\usepackage{hyperref}
\usepackage{hyperxmp}
\usepackage[usenames]{color}

\hypersetup{colorlinks=true}
\hypersetup{pdfstartview=FitH}
\hypersetup{pdfpagemode=UseNone}
\hypersetup{pdfsource={}}
\hypersetup{pdflang={en-UK}}
\hypersetup{pdfcopyright={Copyright 2017-2018 Niklas Beisert.
  This work may be distributed and/or modified under the
  conditions of the LaTeX Project Public License, either version 1.3
  of this license or (at your option) any later version.}}
\hypersetup{pdflicenseurl={http://www.latex-project.org/lppl.txt}}
\hypersetup{pdfcontactaddress={ETH Zurich, ITP, HIT K,
  Wolfgang-Pauli-Strasse 27}}
\hypersetup{pdfcontactpostcode={8093}}
\hypersetup{pdfcontactcity={Zurich}}
\hypersetup{pdfcontactcountry={Switzerland}}
\hypersetup{pdfcontactemail={nbeisert@itp.phys.ethz.ch}}
\hypersetup{pdfcontacturl={http://people.phys.ethz.ch/\xmptilde nbeisert/}}

\newcommand{\secref}[1]{\hyperref[#1]{section \ref*{#1}}}

\parskip1ex
\parindent0pt
\let\olditemize\itemize
\def\itemize{\olditemize\parskip0pt}

\begin{document}

\title{The \textsf{childdoc} Package}
\hypersetup{pdftitle={The childdoc Package}}
\author{Niklas Beisert\\[2ex]
  Institut f\"ur Theoretische Physik\\
  Eidgen\"ossische Technische Hochschule Z\"urich\\
  Wolfgang-Pauli-Strasse 27, 8093 Z\"urich, Switzerland\\[1ex]
  \href{mailto:nbeisert@itp.phys.ethz.ch}
  {\texttt{nbeisert@itp.phys.ethz.ch}}}
\hypersetup{pdfauthor={Niklas Beisert}}
\hypersetup{pdfsubject={Manual for the LaTeX2e Package childdoc}}
\date{30 December 2018, \textsf{v2.0}}
\maketitle

\begin{abstract}\noindent
\textsf{childdoc} is a \LaTeXe{} package
that enables the direct compilation
of document sections included by |\include|
to individual files.
\end{abstract}

\begingroup
\parskip0ex
\tableofcontents
\endgroup

%%%%%%%%%%%%%%%%%%%%%%%%%%%%%%%%%%%%%%%%%%%%%%%%%%%%%%%%%%%%%%%%%%%%%%%%%%%%%%%%
%%%%%%%%%%%%%%%%%%%%%%%%%%%%%%%%%%%%%%%%%%%%%%%%%%%%%%%%%%%%%%%%%%%%%%%%%%%%%%%%
\section{Introduction}

\LaTeX{} provides a mechanism to structure a large document (such as a book)
into a main file and several child files (containing the chapters)
using the |\include| command.
This mechanism is beneficial for documents
which span hundreds of pages in order to
make the source file(s) more manageable.
Moreover, compilation can be restricted to
selected child files by means of the |\includeonly| command.
The latter feature can be used to reduce the compilation time while editing
(this was significantly more useful in the earlier days of \LaTeX{})
or to generate a smaller document which is easier to navigate.
Another application of |\includeonly| is to generate
documents consisting of selected parts of the complete document.

However, there are a few drawbacks of the plain |\include| mechanism:
\begin{itemize}
\item
The child files cannot be compiled on their own,
they can only be compiled via the main file.
A naive editing environment
(such as a text editor with an option
to have the current file processed by \LaTeX)
may require one to switch to the main file before compiling;
attempting to compile the child file produces errors.
\item
The main file must be modified (each time)
to adjust the |\includeonly| command
to the present needs. This easily leaves the main file in a messy state.
\item
The generated document will always carry the filename
of the main document. This is inconvenient if
several child files are to be compiled and
to be kept for distribution.
\end{itemize}

The present package provides a simple interface
to make child files individually compilable by \LaTeX{}.
Compiling a child file then has the same effect as compiling
the main file with an |\includeonly| command
to select the appropriate child.
Moreover the generated document will carry the name of the child
rather than the main file.
This resolves all three above issues.

This feature is meant to make the editing of books,
thesis documents and lecture notes somewhat more convenient.
However, the package can also be used efficiently for
composing a series of documents (such as exercise sheets)
which are typically distributed individually.
It then assists the author in generating the individual documents
(potentially in different versions)
as well as a document containing the collected series.
Another application is in developing style files
or other kinds of included material
where compilation of the style file could redirect
to a sample or test file.

%%%%%%%%%%%%%%%%%%%%%%%%%%%%%%%%%%%%%%%%%%%%%%%%%%%%%%%%%%%%%%%%%%%%%%%%%%%%%%%%
%%%%%%%%%%%%%%%%%%%%%%%%%%%%%%%%%%%%%%%%%%%%%%%%%%%%%%%%%%%%%%%%%%%%%%%%%%%%%%%%
\section{Usage}

First of all, the package \textsf{childdoc} is \emph{not} a standard
\LaTeXe{} |.sty| style file! Therefore it needs to be invoked in
a non-standard way.

%%%%%%%%%%%%%%%%%%%%%%%%%%%%%%%%%%%%%%%%%%%%%%%%%%%%%%%%%%%%%%%%%%%%%%%%%%%%%%%%
\subsection{Included Files}
\label{sec:include}

%%%%%%%%%%%%%%%%%%%%%%%%%%%%%%%%%%%%%%%%
\DescribeMacro{\childdocmain}
To use the package, add the commands
\begin{center}
\begin{tabular}{l}
|\input{childdoc.def}|\\
|\childdocmain{}|\\
\end{tabular}
\end{center}
at the very top of the main \LaTeX{} file,
in particular \emph{before} the |\documentclass| statement!
The argument of |\childdocmain| should be left empty
(but it must be present).

%%%%%%%%%%%%%%%%%%%%%%%%%%%%%%%%%%%%%%%%
\DescribeMacro{\childdocof}
Furthermore, add the commands
\begin{center}
\begin{tabular}{l}
|\input{childdoc.def}|\\
|\childdocof{|\textit{main}|}|\\
\end{tabular}
\end{center}
at the top of every child file \textit{child}
which is included by |\include{|\textit{child}|}|
from within the main file
(or at least for those files to be compiled individually).
The argument \textit{main} must be the filename of the main file.

There are a couple of
considerations in setting up the main and child documents:

%%%%%%%%%%%%%%%%%%%%%%%%%%%%%%%%%%%%%%%%
\paragraph{Restrictions.}

Please note the following restrictions:
\begin{itemize}
\item
|\childdocmain| must be called with one argument \textit{main}
to ensure compatibility with earlier version of the package.
It must either be empty (|\childdocmain{}|)
or precisely match the filename of the main file in which it is specified.
See \secref{sec:detection} for further information.
\item
The filename \textit{main} must be specified without the |.tex| extension.
\item
The filename \textit{main} is case sensitive
(even in case-insensitive file systems)
due to internal string comparison.
\item
The argument \textit{main} should be fully expanded, it cannot be a macro.
\item
Subdirectories and special characters should be avoided in filenames.
\item
The command |\childdocmain{|\textit{main}|}| must be followed by a whitespace.
It should not be followed immediately by another command
or by a comment mark `|%|'.
This is because the \TeX{} parser reads the token immediately following
the argument of |\childdocmain| and puts it
at the beginning of every child section;
however, a white\-space is ignored.
\end{itemize}

%%%%%%%%%%%%%%%%%%%%%%%%%%%%%%%%%%%%%%%%
\paragraph{Content of Main File.}

It is advisable to place all content in the child files included by |\include|.
Any output contained in the main file will appear in all child documents
unless suppressed manually;
it cannot be suppressed automatically by the |\includeonly| directive
and thus should normally be avoided.
A method to include some content in the main file
by means of conditional processing is described in \secref{sec:conditional}.

%%%%%%%%%%%%%%%%%%%%%%%%%%%%%%%%%%%%%%%%
\paragraph{Page Numbering.}

When only a part of the document is compiled,
the appropriate numbering of pages
(as well as other status parameters)
is determined from the |.aux| files.
The latter contain information from previous passes.
However this information needs to propagate through
all intermediate child documents.
Therefore the page numbering in child documents may well
be inconsistent until the complete document is compiled at least once.

A useful (if unconventional) way to always ensure a consistent
page numbering is to restart the numbering in each child document
and denote the pages by `\textit{child}|.|\textit{page}'
where \textit{child} represents the chapter/section number of the child file.
This can be achieved by the command
|\numberwithin{page}{|\textit{child}|}|
of the \textsf{amsmath} package
where \textit{child} can be |chapter| or |section|
depending on the chosen structuring.
Alternatively, one can modify the macro |\thepage| appropriately
and reset the counter |page| at the start of each child file.

%%%%%%%%%%%%%%%%%%%%%%%%%%%%%%%%%%%%%%%%%%%%%%%%%%%%%%%%%%%%%%%%%%%%%%%%%%%%%%%%
\subsection{Conditional Processing}
\label{sec:conditional}

The package provides a mechanism to compile different versions
of a document. To customise the versions further some conditional processing
can come in handy to distinguish which version is being compiled.
The package provides two macros to describe the compilation context:

%%%%%%%%%%%%%%%%%%%%%%%%%%%%%%%%%%%%%%%%
\DescribeMacro{\ifchilddoc}
The conditional |\ifchilddoc| distinguishes between the compilation of
child documents and the main document:
%
\begin{center}
|\ifchilddoc |\textit{child-code}| |[|\||else |\textit{main-code}]| \||fi|
\end{center}

%%%%%%%%%%%%%%%%%%%%%%%%%%%%%%%%%%%%%%%%
\DescribeMacro{\childdocname}
\DescribeMacro{\childdocjob}
The macro |\childdocname| contains the filename (without extension)
of the main or child file being processed.
Note that |\childdocjob| will always contain the name of the main file.

%%%%%%%%%%%%%%%%%%%%%%%%%%%%%%%%%%%%%%%%
\paragraph{Title Page.}

Conditional processing can be used to include a title or banner page
in the main document when proper precautions are taken.
Importantly, the code in the main file should ensure that the page counter
(as well as other status parameters which are stored in the |.aux| files)
takes the same value after the conditional processing.
Otherwise the page numbers may take divergent values
depending on which part is compiled.

For example, a title page could be declared by:
%
\begin{center}
\begin{tabular}{l}
|\ifchilddoc\||else|\\
|\addtocounter{page}{-1}|\\
\textit{code for title page}\\
|\newpage|\\
|\||fi|
\end{tabular}
\end{center}
%
A banner page for the child documents can be generated by:
%
\begin{center}
\begin{tabular}{l}
|\ifchilddoc|\\
|\addtocounter{page}{-1}|\\
\textit{code for banner page}\\
|\newpage|\\
|\||fi|
\end{tabular}
\end{center}
%
Here one could write a message such as:
\begin{center}
|This is the part \childdocname{} of \childdocjob{}.|
\end{center}

%%%%%%%%%%%%%%%%%%%%%%%%%%%%%%%%%%%%%%%%%%%%%%%%%%%%%%%%%%%%%%%%%%%%%%%%%%%%%%%%
\subsection{Flags}
\label{sec:flags}

The package makes it easy to generate different versions
of the main or child documents.
To this end compilation flags can be defined
and assigned different default values.
They will be particularly useful in conjunction
with the forwarding mechanism described in \secref{sec:forward}.

For example, it may be useful to have a flag |\version|
which can be set to |draft| or |final|.
The document source will contain some conditional code
depending on the value of |\version|.
Suppose further, the flag should default to |final| for the main file
and to |draft| for child files
which is a natural assignment for editing the document.
This is achieved by placing the following code
in the preamble of the main document
(below the |\childdocmain| directive):
%
\begin{center}
\begin{tabular}{l}
|\ifchilddoc|\\
|\providecommand{\version}{draft}|\\
|\||else|\\
|\providecommand{\version}{final}|\\
|\||fi|
\end{tabular}
\end{center}
%
The definition by |\providecommand| makes sure
that previous definitions are not overwritten.
Further statements |\providecommand{\version}{...}|
can thus be added before the above code to override it.

For the main file, one might add a line
(between |\childdocmain| and the above block)
%
\begin{center}
|%\ifchilddoc\||else\providecommand{\version}{draft}\||fi|
\end{center}
%
which can be uncommented to produce a draft version.
Likewise one can add a line to the very top of a child file
(above the |\childdocof{|\textit{main}|}| directive)
%
\begin{center}
|%\providecommand{\version}{final}|
\end{center}
%
which can be uncommented to produce the final version of this child document.

%%%%%%%%%%%%%%%%%%%%%%%%%%%%%%%%%%%%%%%%%%%%%%%%%%%%%%%%%%%%%%%%%%%%%%%%%%%%%%%%
\subsection{Forwarding}
\label{sec:forward}

Different versions of the main or child documents
using compilation flags as described in \secref{sec:flags}
can be (permanently) stored in different files
for convenient compilation, viewing and distribution.
To this end, the package defines a command
to pass on compilation to a different file:

%%%%%%%%%%%%%%%%%%%%%%%%%%%%%%%%%%%%%%%%
\DescribeMacro{\childdocforward}
The command |\childdocforward| redirects processing to
another source file:
%
\begin{center}
\begin{tabular}{l}
|\input{childdoc.def}|\\
|\childdocforward[|\textit{main}|]{|\textit{dest}|}|\\
\end{tabular}
\end{center}
%
The argument \textit{dest} is the destination file
(without extension).
It should be the main file or one of the child files.
Note that further \textsf{childdoc} directives
such as |\childdocof| and |\childdocforward|
in the indicated file will be processed in this form.
The optional argument \textit{main}
passes on directly to the main file \textit{main}
while pretending to compile the child \textit{dest}.
This form behaves as if \textit{dest}
issues |\childdocof{|\textit{main}|}| right away,
and no further \textsf{childdoc} directives will be processed.

%%%%%%%%%%%%%%%%%%%%%%%%%%%%%%%%%%%%%%%%
\DescribeMacro{\...prefix}
In the alternative form |\childdocforwardprefix|,
%
\begin{center}
\begin{tabular}{l}
|\input{childdoc.def}|\\
|\childdocforwardprefix[|\textit{main}|]{|\textit{prefix}|}{|\textit{dest}|}|
\end{tabular}
\end{center}
%
the destination file is determined by a pattern
depending on the current file:
To make this work, the current file must be called
`{\textit{prefix}\hspace{0.2em}\textit{suffix}}'
with \textit{prefix} matching precisely the argument.
Processing is then passed on to the file
`{\textit{dest}\hspace{0.2em}\textit{suffix}}'.
Surely, the same effect is achieved by
directly specifying the
argument `{\textit{dest}\hspace{0.2em}\textit{suffix}}'
in the first form.
However, that requires to set up a different file
for each child. With the alternative form of the command
all these files can have exactly the same content
which simplifies setting them up and maintaining them.

For example, the following file |draft.tex|
with a compilation flag |\version| as described in \secref{sec:flags}
compiles the main document as a draft:
%
\begin{center}
\begin{tabular}{l}
|\def\version{draft}|\\
|\input{childdoc.def}|\\
|\childdocforward{|\textit{main}|}|
\end{tabular}
\end{center}
%
Likewise, the following files |final|\textit{nn}|.tex|
compile the final version of the child document
|child|\textit{nn}|.tex|:
%
\begin{center}
\begin{tabular}{l}
|\def\version{final}|\\
|\input{childdoc.def}|\\
|\childdocforwardprefix{final}{child}|
\end{tabular}
\end{center}
%

Note that when several versions of a main file and/or of each child file
are to be generated, it may be convenient to set up a |Makefile| or
shell script to automatise the process.

%%%%%%%%%%%%%%%%%%%%%%%%%%%%%%%%%%%%%%%%%%%%%%%%%%%%%%%%%%%%%%%%%%%%%%%%%%%%%%%%
\subsection{Command Line Processing}
\label{sec:commandline}

The effect of redirection files can also be achieved by invoking
the \LaTeX{} compiler with a more elaborate command line.
Most conveniently this should be done as part
of a shell script or a |Makefile|.

When using \textsf{childdoc} in the main file, the following
command lines effectively perform a redirection
(note that depending on the shell being used,
backslashes may have to be doubled: `|\|' $\to$ `|\\|'):
%
\begin{center}
|... -jobname "|\textit{target}|" |\\|"|[\textit{flags}]%
|\input{childdoc.def}\childdocforward[|\textit{main}|]{|\textit{dest}|}"|
\end{center}
%
Here \textit{target} is the name of the output file,
\textit{main} is the name of the main file
and \textit{dest} is the name of the main or child file to be processed
(all filenames without extensions).
The optional argument \textit{main} can be omitted
if \textit{main} matches \textit{dest}.
Optionally, compilation \textit{flags} can be defined via |\def| commands.
This command line makes the \TeX{} engine believe
it is compiling the file \textit{target}
whose content is specified as the latter parameter.
The provided code then forwards the processing to
\textit{main} or \textit{dest} as described in \secref{sec:forward}.

%%%%%%%%%%%%%%%%%%%%%%%%%%%%%%%%%%%%%%%%%%%%%%%%%%%%%%%%%%%%%%%%%%%%%%%%%%%%%%%%
\subsection{Include by Input}
\label{sec:input}

Including child documents by |\include| has some restrictions by design.
Most notably, the content of a child document always occupies
its own set of pages; pages cannot be shared between child documents.
Usually, this behaviour makes perfect sense
because each child document contain an essential part of the document.
However, in some situations it may be desirable to compose
a document from a collection of parts
without having mandatory page breaks between then.
For this case, the package
provides a mechanism to include parts
by |\input| which can also be processed individually.
However, by construction this mechanism
requires manual handling of the content to be output.

%%%%%%%%%%%%%%%%%%%%%%%%%%%%%%%%%%%%%%%%
\DescribeMacro{\ifchilddocmanual}
The main file should be prepared as usual, see \secref{sec:include}.
However, the document body must make a distinction
between processing of an individual part and of the main document, e.g.:
%
\begin{center}
\begin{tabular}{l}
|\ifchilddocmanual|\\
|\input{\childdocname}|\\
|\||else|\\
\textit{document body with }|\input{|\textit{part}|}|\\
|\||fi|
\end{tabular}
\end{center}
%
The conditional |\ifchilddocmanual| is true whenever
a part to be included by |\input| is being compiled,
and the name of the part is stored in |\childdocname|.

%%%%%%%%%%%%%%%%%%%%%%%%%%%%%%%%%%%%%%%%
\DescribeMacro{\childdocby}
Each part to be included by |\input| should start with:
%
\begin{center}
\begin{tabular}{l}
|\input{childdoc.def}|\\
|\childdocby{|\textit{main}|}|\\
\end{tabular}
\end{center}
%
The directive |\childdocby| is similar to |\childdocof|
described in \secref{sec:include},
but the subsequent selection of content must be done manually.
To that end, both |\ifchilddoc| and |\ifchilddocmanual|
will be true upon processing of a part,
and the name of the part is stored in |\childdocname|.
Note that |\jobname| will be set to the filename of the current part
so that each part receives an individual |.aux| file
that does not interfere with the |.aux| file(s) of the main document.
This behaviour can be altered by the alternative form
|\childdocby[*]{|\textit{main}|}| (with a non-empty optional argument)
which uses the |.aux| file of the main document
by setting |\jobname| to \textit{main}.

%%%%%%%%%%%%%%%%%%%%%%%%%%%%%%%%%%%%%%%%%%%%%%%%%%%%%%%%%%%%%%%%%%%%%%%%%%%%%%%%
\subsection{Driver Development}
\label{sec:driver}

The \textsf{childdoc} mechanism can also be use for the development
of definition files such as \LaTeX{} styles or classes.
This case differs from the above setup with multiple parts
included by |\include| in that no |\includeonly| should be invoked.
This can be achieved by starting the include file
(before |\ProvidesPackage|) with:
%
\begin{center}
\begin{tabular}{l}
|\input{childdoc.def}|\\
|\childdocforward{|\textit{main}|}|\\
\end{tabular}
\end{center}
%
or alternatively with:
%
\begin{center}
\begin{tabular}{l}
|\input{childdoc.def}|\\
|\childdocby{|\textit{main}|}|\\
\end{tabular}
\end{center}
%
Both forms have slightly different effects as described above.
The main file is prepared as usual, see \secref{sec:include}.

%%%%%%%%%%%%%%%%%%%%%%%%%%%%%%%%%%%%%%%%%%%%%%%%%%%%%%%%%%%%%%%%%%%%%%%%%%%%%%%%
\subsection{Legacy Detection}
\label{sec:detection}

The directive |\childdocmain| in the main file can detect
whether the complete document or merely a child is to be compiled
even without using the directive |\childdocof|.
This method is deprecated because it is less robust
and there is no compelling reason to use it;
it is merely provided for backward compatibility
and it may be removed in future versions.

If the detection mechanism is to be used,
it is mandatory to correctly specify
the filename of the main file as the argument of |\childdocmain|:
%
\begin{center}
\begin{tabular}{l}
|\input{childdoc.def}|\\
|\childdocmain{|\textit{main}|}|\\
\end{tabular}
\end{center}
%
If |\jobname| does not match the argument \textit{main} of |\childdocmain|,
it is assumed that |\jobname| points to the child file to be compiled.
When using |\childdocmain| with the main file specified as argument,
it suffices to start a child file
with just |\input{|\textit{main}|}|
without loading of the package and using |\childdocof|.
If instead all processing is done
with the appropriate \textsf{childdoc} directives,
the argument of \textit{main} of |\childdocmain| can be empty.

An alternative version of the command line processing described
in \secref{sec:commandline} using the detection mechanism reads:
%
\begin{center}
|... -jobname "|\textit{target}|" "|[\textit{flags}]%
[|\def\jobname{|\textit{dest}|}|]|\input{|\textit{main}|}"|
\end{center}

%%%%%%%%%%%%%%%%%%%%%%%%%%%%%%%%%%%%%%%%%%%%%%%%%%%%%%%%%%%%%%%%%%%%%%%%%%%%%%%%
\subsection{Manual Code}
\label{sec:manual}

In case one cannot be certain whether the definitions file |childdoc.def|
is installed on the target \TeX{} distribution
and one prefers not to ship it,
it is conceivable to paste a few relevant commands into the sources.

To that end, drop all statements |\input{childdoc.def}|
and perform the replacements as outlined below.
Instead of |\childdocmain{|\textit{main}|}| add the following code
to the top of the main file:
%
\begin{center}
\begin{tabular}{l}
|\||ifdefined\childdocname\endinput\||fi\newif\ifchilddoc|\\
|\edef\childdocname{\scantokens\expandafter{\jobname\noexpand}}|\\
|\def\childdocmain{|\textit{main}|}\||ifx\childdocmain\childdocname\||else|\\
|\childdoctrue\includeonly{\childdocname}\let\jobname\childdocmain\||fi|\\
\end{tabular}
\end{center}
%
Instead of |\childdocof{|\textit{main}|}| just include the main file
at the top of each child file:
%
\begin{center}
|\input{|\textit{main}|}|
\end{center}
%
A simple redirection |\childdocforward{|\textit{dest}|}| is achieved by:
%
\begin{center}
|\def\jobname{|\textit{dest}|}\input{\jobname}|
\end{center}
%
The redirection with prefix
|\childdocforwardprefix[|\textit{prefix}|]{|\textit{dest}|}|
is accomplished by:
%
\begin{center}
\begin{tabular}{l}
|{\edef\jobname{\scantokens\expandafter{\jobname\noexpand}}|\\
|\def\redirectjob |\textit{prefix}|#1~~~{\gdef\jobname{|\textit{dest}|#1}}|\\
|\expandafter\redirectjob\jobname~~~}\input{\jobname}|
\end{tabular}
\end{center}

In an alternative approach,
child documents can be compiled by a specific command line
without additional code or specific definitions:
%
\begin{center}
|... -jobname "|\textit{target}|" "|[\textit{flags}]%
|\includeonly{|\textit{dest}|}\input{|\textit{main}|}"|
\end{center}
%

%%%%%%%%%%%%%%%%%%%%%%%%%%%%%%%%%%%%%%%%%%%%%%%%%%%%%%%%%%%%%%%%%%%%%%%%%%%%%%%%
%%%%%%%%%%%%%%%%%%%%%%%%%%%%%%%%%%%%%%%%%%%%%%%%%%%%%%%%%%%%%%%%%%%%%%%%%%%%%%%%
\section{Information}

%%%%%%%%%%%%%%%%%%%%%%%%%%%%%%%%%%%%%%%%%%%%%%%%%%%%%%%%%%%%%%%%%%%%%%%%%%%%%%%%
\subsection{Copyright}

Copyright \copyright{} 2017--2018 Niklas Beisert

This work may be distributed and/or modified under the
conditions of the \LaTeX{} Project Public License, either version 1.3
of this license or (at your option) any later version.
The latest version of this license is in
  \url{http://www.latex-project.org/lppl.txt}
and version 1.3 or later is part of all distributions of \LaTeX{}
version 2005/12/01 or later.

This work has the LPPL maintenance status `maintained'.

The Current Maintainer of this work is Niklas Beisert.

This work consists of the files |README.txt|, |childdoc.ins| and |childdoc.dtx|
as well as the derived files |childdoc.def|, |cdocsamp.tex|
with |cdocsch1.tex|, |cdocsch2.tex|, |cdocspt3.tex|, |cdocspt4.tex|,
|cdocsdrf.tex|, |cdocsfn1.tex|, |cdocsfn2.tex|
as well as |childdoc.pdf|.

%%%%%%%%%%%%%%%%%%%%%%%%%%%%%%%%%%%%%%%%%%%%%%%%%%%%%%%%%%%%%%%%%%%%%%%%%%%%%%%%
\subsection{Files and Installation}

The package consists of the files:
%
\begin{center}
\begin{tabular}{ll}
    |README.txt|   & readme file \\
    |childdoc.ins| & installation file \\
    |childdoc.dtx| & source file \\
    |childdoc.def| & definition file \\
    |cdocsamp.tex| & sample main file \\
    |cdocsch1.tex| & sample include file \\
    |cdocsch2.tex| & sample include file \\
    |cdocspt3.tex| & sample part file \\
    |cdocspt4.tex| & sample part file \\
    |cdocsdrf.tex| & sample redirection file \\
    |cdocsfn1.tex| & sample redirection file \\
    |cdocsfn2.tex| & sample redirection file \\
    |childdoc.pdf| & manual
\end{tabular}
\end{center}
%
The distribution consists of the files
|README.txt|, |childdoc.ins| and |childdoc.dtx|.
%
\begin{itemize}
\item
Run (pdf)\LaTeX{} on |childdoc.dtx|
to compile the manual |childdoc.pdf| (this file).
\item
Run \LaTeX{} on |childdoc.ins| to create the definitions file |childdoc.def|
and the sample |cdocsamp.tex| with include files
|cdocsch1.tex|, |cdocsch2.tex|, |cdocspt3.tex|, |cdocspt4.tex|,
|cdocsdrf.tex|, |cdocsfn1.tex|, |cdocsfn2.tex|.
Then copy the file |childdoc.def| to an appropriate directory of your \LaTeX{}
distribution, e.g.\ \textit{texmf-root}|/tex/latex/childdoc|.
\end{itemize}

%%%%%%%%%%%%%%%%%%%%%%%%%%%%%%%%%%%%%%%%%%%%%%%%%%%%%%%%%%%%%%%%%%%%%%%%%%%%%%%%
\subsection{Related CTAN Packages}

There are several other packages which offer a similar functionality:
%
\begin{itemize}
\item
The packages
\href{http://ctan.org/pkg/docmute}{\textsf{docmute}},
\href{http://ctan.org/pkg/includex}{\textsf{includex}} and
\href{http://ctan.org/pkg/standalone}{\textsf{standalone}}
provide commands to include only the document body of
a child file thus allowing both files to be compiled individually.
\item
The packages \href{http://ctan.org/pkg/subdocs}{\textsf{subdocs}}
and \href{http://ctan.org/pkg/subfiles}{\textsf{subfiles}}
provide structures in which the main and child documents can be
encapsulated and allowing them to be compiled individually.
The inclusion mechanism is different from the conventional |\include|.
\item
The package \href{http://ctan.org/pkg/combine}{\textsf{combine}}
is an elaborate solution to combine several documents into one.
\end{itemize}
%
See also the CTAN topic \href{http://ctan.org/topic/subdocs}{\textsf{subdocs}}
for further related packages.
The present package differs from the above solutions in that
a document structure constructed with the conventional |\include| mechanism
just needs two extra commands at the top of every file
such that all constituent files can be compiled individually.

%%%%%%%%%%%%%%%%%%%%%%%%%%%%%%%%%%%%%%%%%%%%%%%%%%%%%%%%%%%%%%%%%%%%%%%%%%%%%%%%
%\subsection{Feature Suggestions}
%
%The following is a list of features which may be useful for future
%versions of this package:
%%
%\begin{itemize}
%\item
%\ldots
%\end{itemize}

%%%%%%%%%%%%%%%%%%%%%%%%%%%%%%%%%%%%%%%%%%%%%%%%%%%%%%%%%%%%%%%%%%%%%%%%%%%%%%%%
\subsection{Revision History}

%%%%%%%%%%%%%%%%%%%%%%%%%%%%%%%%%%%%%%%%
\paragraph{v2.0:} 2018/12/30

\begin{itemize}
\item
immediate forward processing
\item
added |\childdocby| mechanism
\item
manual restructured
\end{itemize}

%%%%%%%%%%%%%%%%%%%%%%%%%%%%%%%%%%%%%%%%
\paragraph{v1.6:} 2018/01/17

\begin{itemize}
\item
application for development of include files
\item
corrections to manual
\end{itemize}

%%%%%%%%%%%%%%%%%%%%%%%%%%%%%%%%%%%%%%%%
\paragraph{v1.5:} 2017/05/21

\begin{itemize}
\item
more complete structuring introduced
\item
|\childdocof| introduced
\item
|\childdoc| renamed to |\childdocmain|
\item
|\childredirect| renamed to |\childdocforward| and |\childdocforwardprefix|
and functionality expanded
\end{itemize}

%%%%%%%%%%%%%%%%%%%%%%%%%%%%%%%%%%%%%%%%
\paragraph{v1.0:} 2017/04/27

\begin{itemize}
\item
manual and install package
\item
first version published on CTAN
\end{itemize}

%%%%%%%%%%%%%%%%%%%%%%%%%%%%%%%%%%%%%%%%
\paragraph{v0.6:} 2017/04/26

\begin{itemize}
\item
redirection mechanism added
\end{itemize}

%%%%%%%%%%%%%%%%%%%%%%%%%%%%%%%%%%%%%%%%
\paragraph{v0.5:} 2017/04/26

\begin{itemize}
\item
functionality in definition file
\end{itemize}


%%%%%%%%%%%%%%%%%%%%%%%%%%%%%%%%%%%%%%%%%%%%%%%%%%%%%%%%%%%%%%%%%%%%%%%%%%%%%%%%
%%%%%%%%%%%%%%%%%%%%%%%%%%%%%%%%%%%%%%%%%%%%%%%%%%%%%%%%%%%%%%%%%%%%%%%%%%%%%%%%
%%%%%%%%%%%%%%%%%%%%%%%%%%%%%%%%%%%%%%%%%%%%%%%%%%%%%%%%%%%%%%%%%%%%%%%%%%%%%%%%
\appendix

\settowidth\MacroIndent{\rmfamily\scriptsize 000\ }

 \DocInput{childdoc.dtx}

\end{document}
%</driver>
% \fi
%
% %%%%%%%%%%%%%%%%%%%%%%%%%%%%%%%%%%%%%%%%%%%%%%%%%%%%%%%%%%%%%%%%%%%%%%%%%%%%%%
% %%%%%%%%%%%%%%%%%%%%%%%%%%%%%%%%%%%%%%%%%%%%%%%%%%%%%%%%%%%%%%%%%%%%%%%%%%%%%%
% \section{Sample}
%\iffalse
%<*samplemain>
%\fi
%
% The following presents a sample document
% with two chapters, two parts, a title page,
% a compile flag as well as three forwarding files to set the flag.
% It consists of eight |.tex| files:
% \begin{center}
% \begin{tabular}{ll}
% |cdocsamp.tex|&main file\\
% |cdocsch1.tex|&include file for chapter 1\\
% |cdocsch2.tex|&include file for chapter 2\\
% |cdocspt3.tex|&include file for part 3\\
% |cdocspt4.tex|&include file for part 4\\
% |cdocsdrf.tex|&forwarding file for main file in draft mode\\
% |cdocsfi1.tex|&forwarding file for final version of chapter 1\\
% |cdocsfi2.tex|&forwarding file for final version of chapter 2\\
% \end{tabular}
% \end{center}
% Each of the eight files can be compiled directly by the \LaTeX{} compiler.
%
% %%%%%%%%%%%%%%%%%%%%%%%%%%%%%%%%%%%%%%
% \paragraph{Main File.}
%
% The main file is called |cdocsamp.tex|.
%
% Load the \textsf{childdoc} definitions and
% declare the filename for the main document:
%    \begin{macrocode}
\input{childdoc.def}
\childdocmain{}
%    \end{macrocode}

% Optional override for |\version| flag:
%    \begin{macrocode}
%%\ifchilddoc\else\providecommand{\version}{draft}\fi
%    \end{macrocode}

% Define the default values for the |\version| flag
% (|final| for the main file and |draft| for childs):
%    \begin{macrocode}
\ifchilddoc
\providecommand{\version}{draft}
\else
\providecommand{\version}{final}
\fi
%    \end{macrocode}

% Load the standard document class:
%    \begin{macrocode}
\documentclass[12pt]{article}
%    \end{macrocode}

% Start the document body:
%    \begin{macrocode}
\begin{document}
%    \end{macrocode}

% Declare a title page.
% Print title, part of document being processed and version flag:
%    \begin{macrocode}
\addtocounter{page}{-1}
\begin{center}
{\LARGE\bfseries{}childdoc example\par}
\vspace{1cm}
\ifchilddoc
\ifchilddocmanual part\else chapter\fi:
`\childdocname' of `\childdocjob'\par
\else
main document: `\childdocjob'\par
\fi
version: \version\par
\end{center}
\newpage
%    \end{macrocode}

% Manually include selected file,
% otherwise process as usual:
%    \begin{macrocode}
\ifchilddocmanual
\section*{part `\childdocname'}
\input{\childdocname}
\else
%    \end{macrocode}

% Include the two chapters:
%    \begin{macrocode}
\include{cdocsch1}
\include{cdocsch2}
%    \end{macrocode}

% Include the two parts unless only chapters should be displayed:
%    \begin{macrocode}
\ifchilddoc\else
\section{part three}
\input{cdocspt3}
\section{part four}
\input{cdocspt4}
\fi
%    \end{macrocode}

% Process as usual until here:
%    \begin{macrocode}
\fi
%    \end{macrocode}

% End of document body:
%    \begin{macrocode}
\end{document}
%    \end{macrocode}
%\iffalse
%</samplemain>
%\fi
%
% %%%%%%%%%%%%%%%%%%%%%%%%%%%%%%%%%%%%%%
% \paragraph{Chapter Include Files.}
%
% The include files are called |cdocsch1.tex| and |cdocsch2.tex|.
%
%\iffalse
%<*samplechap1|samplechap2>
%\fi

% Optional override for |\version| flag:
%    \begin{macrocode}
%%\providecommand{\version}{final}
%    \end{macrocode}

% Include the main document:
%    \begin{macrocode}
\input{childdoc.def}
\childdocof{cdocsamp}
%    \end{macrocode}

%\iffalse
%</samplechap1|samplechap2>
%\fi
%
%\iffalse
%<*samplechap1>
%\fi
% Some text for chapter 1:
%    \begin{macrocode}
\section{one}
some text in chapter one
%    \end{macrocode}

%\iffalse
%</samplechap1>
%\fi
% Some text for chapter 2:
%\iffalse
%<*samplechap2>
%\fi
%    \begin{macrocode}
\section{two}
more text in chapter two
%    \end{macrocode}

%\iffalse
%</samplechap2>
%\fi
%
% %%%%%%%%%%%%%%%%%%%%%%%%%%%%%%%%%%%%%%
% \paragraph{Part Include Files.}
%
% The include files are called |cdocspt3.tex| and |cdocspt4.tex|.
%
%\iffalse
%<*samplepart3|samplepart4>
%\fi

% Optional override for |\version| flag:
%    \begin{macrocode}
%%\providecommand{\version}{final}
%    \end{macrocode}

% Include the main document:
%    \begin{macrocode}
\input{childdoc.def}
\childdocby{cdocsamp}
%    \end{macrocode}

%\iffalse
%</samplepart3|samplepart4>
%\fi
%
%\iffalse
%<*samplepart3>
%\fi
% Some text for part 3:
%    \begin{macrocode}
some text in part three
%    \end{macrocode}

%\iffalse
%</samplepart3>
%\fi
% Some text for part 4:
%\iffalse
%<*samplepart4>
%\fi
%    \begin{macrocode}
more text in part four
%    \end{macrocode}

%\iffalse
%</samplepart4>
%\fi
%
% %%%%%%%%%%%%%%%%%%%%%%%%%%%%%%%%%%%%%%
% \paragraph{Forwarding for a Complete Draft.}
%
% The following forwarding file |cdocsdrf.tex|
% compiles the main document in draft mode:
%\iffalse
%<*sampledraft>
%\fi
%    \begin{macrocode}
\def\version{draft}
\input{childdoc.def}
\childdocforward{cdocsamp}
%    \end{macrocode}

%\iffalse
%</sampledraft>
%\fi
%
% %%%%%%%%%%%%%%%%%%%%%%%%%%%%%%%%%%%%%%
% \paragraph{Forwarding for Final Version of the Chapters.}
%
% The following forwarding files |cdocsfn1.tex| and |cdocsfn2.tex|
% (with identical content)
% compile the final versions of the child documents
% |cdocsch1.tex| and |cdocsch2.tex|, respectively:
%\iffalse
%<*samplefinal>
%\fi
%    \begin{macrocode}
\def\version{final}
\input{childdoc.def}
\childdocforwardprefix[cdocsamp]{cdocsfn}{cdocsch}
%    \end{macrocode}

%\iffalse
%</samplefinal>
%\fi
%
% %%%%%%%%%%%%%%%%%%%%%%%%%%%%%%%%%%%%%%
% \paragraph{Command Line Processing.}
%
% The following three command lines generate the output files
% |cdocscld|, |cdocscl1| and |cdocscl2|
% which should be identical to
% |cdocsdrf|, |cdocsch1| and |cdocsfn2|, respectively:
% \begin{center}
% \begin{tabular}{l}
% |latex -jobname cdocscld \|\\
% |  "\def\version{draft}\input{childdoc.def}\childdocforward{cdocsamp}"|\\
% |latex -jobname cdocscl1 \|\\
% |  "\input{childdoc.def}\childdocforward[cdocsamp]{cdocsch1}"|\\
% |latex -jobname cdocscl2 \|\\
% |  "\def\version{final}\input{childdoc.def}\childdocforward{cdocsch2}"|
% \end{tabular}
% \end{center}
% Note that the trailing backslash on each first line
% merely continues the input to the second line
% (for convenient cut ant paste).
% Furthermore, the command |latex| can be replaced by any
% of its alternative versions such as |pdflatex|.
%
% %%%%%%%%%%%%%%%%%%%%%%%%%%%%%%%%%%%%%%%%%%%%%%%%%%%%%%%%%%%%%%%%%%%%%%%%%%%%%%
% %%%%%%%%%%%%%%%%%%%%%%%%%%%%%%%%%%%%%%%%%%%%%%%%%%%%%%%%%%%%%%%%%%%%%%%%%%%%%%
% \section{Implementation}
%\iffalse
%<*package>
%\fi
%
% This section describes the definitions file |childdoc.def|.

% The definitions cannot be loaded using |\usepackage| or |\RequirePackage|
% which has a mechanism to prevent loading a style file more than once.
% When loading the definitions by means of |\input|
% multiple instances have to be prevented manually:
%\iffalse
%This code needs to be before the `\ProvidesFile' directive
%which is defined at the beginning of this file.
%Therefore it is also placed there and commented out here.
%</package>
%<*discard>
%\fi
%    \begin{macrocode}
\ifdefined\childdocmain\endinput\fi
%    \end{macrocode}
%\iffalse
%</discard>
%<*package>
%\fi
%
% \macro{\ifchilddoc}
% \macro{\ifchilddocmanual}
% The conditional |\ifchilddoc| tells whether a
% child (true) or main (false) document is being compiled.
% The conditional |\ifchilddocmanual| tells whether
% the |\includeonly| mechanism is used (false) or
% the selection of child files must be performed manually (true).
% The definitions initialise to false:
%    \begin{macrocode}
\newif\ifchilddoc
\newif\ifchilddocmanual
%    \end{macrocode}

% \macro{\childdocname}
% \macro{\childdocjob}
% The macro |\childdocname| stores the name of the main document
% to be compiled. The macro |\childdocjob| stores the name of
% the document on which the \LaTeX{} compiler was originally invoked.
% The content of |\jobname| cannot be compared
% to filenames specified in the source due to different catcodes.
% The following code rescans |\jobname|, stores the result
% in |\childdocname| and saves a copy in |\childdocjob|:
%    \begin{macrocode}
\edef\childdocname{\scantokens\expandafter{\jobname\noexpand}}
\let\childdocjob\childdocname
%    \end{macrocode}

% \macro{\childdocdisable}
% The macro |\childdocdisable| prevents the main file
% from being processed more than once.
% At this stage, the main document command |\childdocmain|
% is assumed to be called once again where it should do nothing.
% Any subsequent call to it should prevent
% a secondary processing of the main document
% It overwrites the forwarding commands
% |\childdocof| and |\childdocforward|
% with empty macros to prevent further inclusions of the main document:
%    \begin{macrocode}
\newcommand{\childdocdisable}
{
  \renewcommand{\childdocmain}[1]{\renewcommand{\childdocmain}[1]{\endinput}}
  \renewcommand{\childdocof}[1]{}
  \renewcommand{\childdocby}[2][]{}
  \renewcommand{\childdocforward}[2][]{}
  \renewcommand{\childdocdisable}{}
}
%    \end{macrocode}

% \macro{\childdocmain}
% The macro |\childdocmain| is to be called at the top of the main file
% with nothing or the main filename (without extension) as argument.
% First, it breaks loops.
% If the argument is not empty and does not match |\childdocname|
% (which is set by the first inclusion of |childdoc.def|),
% |\ifchilddoc| is set to true, |\includeonly| is applied to the child file
% and |\jobname| is set to the main file
% (for proper handling of |.aux| files):
%    \begin{macrocode}
\newcommand{\childdocmain}[1]
{
  \childdocdisable\childdocmain{}
  \if?#1?\else
    \begingroup
      \def\childdoctmp{#1}
      \ifx\childdoctmp\childdocname
        \def\childdoctmp{}
      \else
        \def\childdoctmp
        {
          \childdoctrue
          \includeonly{\childdocname}
          \def\childdocjob{#1}
          \def\jobname{#1}
        }
      \fi
      \expandafter
    \endgroup
    \childdoctmp
  \fi
}
%    \end{macrocode}

% \macro{\childdocof}
% The command |\childdocof| redirects
% compilation to the main file |#1|.
%    \begin{macrocode}
\newcommand{\childdocof}[1]
{
  \childdocdisable
  \childdoctrue
  \includeonly{\childdocname}
  \def\jobname{#1}
  \def\childdocjob{#1}
  \input{#1}
}
%    \end{macrocode}

% \macro{\childdocby}
% The command |\childdocby| ....
%    \begin{macrocode}
\newcommand{\childdocby}[2][]
{
  \childdocdisable
  \childdoctrue
  \childdocmanualtrue
  \if?#1?\else
    \def\jobname{#2}
  \fi
  \def\childdocjob{#2}
  \input{#2}
  \endinput
}
%    \end{macrocode}

% \macro{\childdocforward}
% The command |\childdocforward| redirects
% compilation to the main file or
% (if the optional argument is given) a child file.
% Parameters are set as if the main file
% or a child file starting with |\childdocof| was compiled.
% Then compilation is handed over to the main file:
%    \begin{macrocode}
\newcommand{\childdocforward}[2][]
{
  \begingroup
    \if?#1?
      \def\childdoctmp
      {
        \def\childdocname{#2}
        \def\childdocjob{#2}
        \def\jobname{#2}
        \input{#2}
        \endinput
      }
    \else
      \def\childdoctmp
      {
        \childdocdisable
        \def\childdocname{#2}
        \childdoctrue
        \includeonly{#2}
        \def\childdocjob{#1}
        \def\jobname{#1}
        \input{#1}
        \endinput
      }
    \fi
    \expandafter
  \endgroup
  \childdoctmp
}
%    \end{macrocode}

% \macro{\childdocforwardprefix}
% The command |\childdocforwardprefix| redirects
% compilation to the main or a child file by means of a pattern.
% The prefix |#1| in the current filename is replaced by |#2|
% and the suffix of the current filename is kept
% (it is assumed that the filename does not contain the substring `|~~~|'
% which is used as a delimiter).
% Compilation is handed over to the new file by |\childdocforward|:
%    \begin{macrocode}
\newcommand{\childdocforwardprefix}[3][]
{
  \begingroup
    \def\childdocextract #2##1~~~{\def\childdoctmp{\childdocforward[#1]{#3##1}}}
    \expandafter\childdocextract\childdocname~~~
    \expandafter
  \endgroup
  \childdoctmp
}
%    \end{macrocode}

% \macro{\childdoc}
% The deprecated macro |\childdoc| is a legacy version of |\childdocmain|:
%    \begin{macrocode}
\newcommand{\childdoc}{\childdocmain}
%    \end{macrocode}

% \macro{\childdocredirect}
% The deprecated macro |\childdocredirect| is a legacy version
% of |\childdocforward| and |\childdocforwardprefix|:
%    \begin{macrocode}
\newcommand{\childdocredirect}[2][]
{
  \begingroup
    \if?#1?
      \def\childdoctmp{\childdocforward{#2}}
    \else
      \def\childdoctmp{\childdocforwardprefix{#1}{#2}}
    \fi
    \expandafter
  \endgroup
  \childdoctmp
}
%    \end{macrocode}

%\iffalse
%</package>
%\fi
%
\endinput
|\\
|\childdocforward[|\textit{main}|]{|\textit{dest}|}|\\
\end{tabular}
\end{center}
%
The argument \textit{dest} is the destination file
(without extension).
It should be the main file or one of the child files.
Note that further \textsf{childdoc} directives
such as |\childdocof| and |\childdocforward|
in the indicated file will be processed in this form.
The optional argument \textit{main}
passes on directly to the main file \textit{main}
while pretending to compile the child \textit{dest}.
This form behaves as if \textit{dest}
issues |\childdocof{|\textit{main}|}| right away,
and no further \textsf{childdoc} directives will be processed.

%%%%%%%%%%%%%%%%%%%%%%%%%%%%%%%%%%%%%%%%
\DescribeMacro{\...prefix}
In the alternative form |\childdocforwardprefix|,
%
\begin{center}
\begin{tabular}{l}
|% \iffalse
%
% childdoc.dtx Copyright (C) 2017-2018 Niklas Beisert
%
% This work may be distributed and/or modified under the
% conditions of the LaTeX Project Public License, either version 1.3
% of this license or (at your option) any later version.
% The latest version of this license is in
%   http://www.latex-project.org/lppl.txt
% and version 1.3 or later is part of all distributions of LaTeX
% version 2005/12/01 or later.
%
% This work has the LPPL maintenance status `maintained'.
%
% The Current Maintainer of this work is Niklas Beisert.
%
% This work consists of the files childdoc.dtx and childdoc.ins
% and the derived files childdoc.def and cdocsamp.tex with
% cdocsch1.tex, cdocsch2.tex, cdocsdrf.tex, cdocsfn1.tex, cdocsfn2.tex.
%
%<package>\ifdefined\childdocmain\endinput\fi
%<package>\ProvidesFile{childdoc.def}[2018/12/30 v2.0 child document driver]
%<samplemain>\ProvidesFile{cdocsamp.tex}[2018/12/30 v2.0 sample for childdoc]
%<*driver>
%\ProvidesFile{childdoc.drv}[2018/12/30 v2.0 childdoc reference manual file]
\PassOptionsToClass{10pt,a4paper}{article}
\documentclass{ltxdoc}

\usepackage[margin=35mm]{geometry}
\usepackage{hyperref}
\usepackage{hyperxmp}
\usepackage[usenames]{color}

\hypersetup{colorlinks=true}
\hypersetup{pdfstartview=FitH}
\hypersetup{pdfpagemode=UseNone}
\hypersetup{pdfsource={}}
\hypersetup{pdflang={en-UK}}
\hypersetup{pdfcopyright={Copyright 2017-2018 Niklas Beisert.
  This work may be distributed and/or modified under the
  conditions of the LaTeX Project Public License, either version 1.3
  of this license or (at your option) any later version.}}
\hypersetup{pdflicenseurl={http://www.latex-project.org/lppl.txt}}
\hypersetup{pdfcontactaddress={ETH Zurich, ITP, HIT K,
  Wolfgang-Pauli-Strasse 27}}
\hypersetup{pdfcontactpostcode={8093}}
\hypersetup{pdfcontactcity={Zurich}}
\hypersetup{pdfcontactcountry={Switzerland}}
\hypersetup{pdfcontactemail={nbeisert@itp.phys.ethz.ch}}
\hypersetup{pdfcontacturl={http://people.phys.ethz.ch/\xmptilde nbeisert/}}

\newcommand{\secref}[1]{\hyperref[#1]{section \ref*{#1}}}

\parskip1ex
\parindent0pt
\let\olditemize\itemize
\def\itemize{\olditemize\parskip0pt}

\begin{document}

\title{The \textsf{childdoc} Package}
\hypersetup{pdftitle={The childdoc Package}}
\author{Niklas Beisert\\[2ex]
  Institut f\"ur Theoretische Physik\\
  Eidgen\"ossische Technische Hochschule Z\"urich\\
  Wolfgang-Pauli-Strasse 27, 8093 Z\"urich, Switzerland\\[1ex]
  \href{mailto:nbeisert@itp.phys.ethz.ch}
  {\texttt{nbeisert@itp.phys.ethz.ch}}}
\hypersetup{pdfauthor={Niklas Beisert}}
\hypersetup{pdfsubject={Manual for the LaTeX2e Package childdoc}}
\date{30 December 2018, \textsf{v2.0}}
\maketitle

\begin{abstract}\noindent
\textsf{childdoc} is a \LaTeXe{} package
that enables the direct compilation
of document sections included by |\include|
to individual files.
\end{abstract}

\begingroup
\parskip0ex
\tableofcontents
\endgroup

%%%%%%%%%%%%%%%%%%%%%%%%%%%%%%%%%%%%%%%%%%%%%%%%%%%%%%%%%%%%%%%%%%%%%%%%%%%%%%%%
%%%%%%%%%%%%%%%%%%%%%%%%%%%%%%%%%%%%%%%%%%%%%%%%%%%%%%%%%%%%%%%%%%%%%%%%%%%%%%%%
\section{Introduction}

\LaTeX{} provides a mechanism to structure a large document (such as a book)
into a main file and several child files (containing the chapters)
using the |\include| command.
This mechanism is beneficial for documents
which span hundreds of pages in order to
make the source file(s) more manageable.
Moreover, compilation can be restricted to
selected child files by means of the |\includeonly| command.
The latter feature can be used to reduce the compilation time while editing
(this was significantly more useful in the earlier days of \LaTeX{})
or to generate a smaller document which is easier to navigate.
Another application of |\includeonly| is to generate
documents consisting of selected parts of the complete document.

However, there are a few drawbacks of the plain |\include| mechanism:
\begin{itemize}
\item
The child files cannot be compiled on their own,
they can only be compiled via the main file.
A naive editing environment
(such as a text editor with an option
to have the current file processed by \LaTeX)
may require one to switch to the main file before compiling;
attempting to compile the child file produces errors.
\item
The main file must be modified (each time)
to adjust the |\includeonly| command
to the present needs. This easily leaves the main file in a messy state.
\item
The generated document will always carry the filename
of the main document. This is inconvenient if
several child files are to be compiled and
to be kept for distribution.
\end{itemize}

The present package provides a simple interface
to make child files individually compilable by \LaTeX{}.
Compiling a child file then has the same effect as compiling
the main file with an |\includeonly| command
to select the appropriate child.
Moreover the generated document will carry the name of the child
rather than the main file.
This resolves all three above issues.

This feature is meant to make the editing of books,
thesis documents and lecture notes somewhat more convenient.
However, the package can also be used efficiently for
composing a series of documents (such as exercise sheets)
which are typically distributed individually.
It then assists the author in generating the individual documents
(potentially in different versions)
as well as a document containing the collected series.
Another application is in developing style files
or other kinds of included material
where compilation of the style file could redirect
to a sample or test file.

%%%%%%%%%%%%%%%%%%%%%%%%%%%%%%%%%%%%%%%%%%%%%%%%%%%%%%%%%%%%%%%%%%%%%%%%%%%%%%%%
%%%%%%%%%%%%%%%%%%%%%%%%%%%%%%%%%%%%%%%%%%%%%%%%%%%%%%%%%%%%%%%%%%%%%%%%%%%%%%%%
\section{Usage}

First of all, the package \textsf{childdoc} is \emph{not} a standard
\LaTeXe{} |.sty| style file! Therefore it needs to be invoked in
a non-standard way.

%%%%%%%%%%%%%%%%%%%%%%%%%%%%%%%%%%%%%%%%%%%%%%%%%%%%%%%%%%%%%%%%%%%%%%%%%%%%%%%%
\subsection{Included Files}
\label{sec:include}

%%%%%%%%%%%%%%%%%%%%%%%%%%%%%%%%%%%%%%%%
\DescribeMacro{\childdocmain}
To use the package, add the commands
\begin{center}
\begin{tabular}{l}
|\input{childdoc.def}|\\
|\childdocmain{}|\\
\end{tabular}
\end{center}
at the very top of the main \LaTeX{} file,
in particular \emph{before} the |\documentclass| statement!
The argument of |\childdocmain| should be left empty
(but it must be present).

%%%%%%%%%%%%%%%%%%%%%%%%%%%%%%%%%%%%%%%%
\DescribeMacro{\childdocof}
Furthermore, add the commands
\begin{center}
\begin{tabular}{l}
|\input{childdoc.def}|\\
|\childdocof{|\textit{main}|}|\\
\end{tabular}
\end{center}
at the top of every child file \textit{child}
which is included by |\include{|\textit{child}|}|
from within the main file
(or at least for those files to be compiled individually).
The argument \textit{main} must be the filename of the main file.

There are a couple of
considerations in setting up the main and child documents:

%%%%%%%%%%%%%%%%%%%%%%%%%%%%%%%%%%%%%%%%
\paragraph{Restrictions.}

Please note the following restrictions:
\begin{itemize}
\item
|\childdocmain| must be called with one argument \textit{main}
to ensure compatibility with earlier version of the package.
It must either be empty (|\childdocmain{}|)
or precisely match the filename of the main file in which it is specified.
See \secref{sec:detection} for further information.
\item
The filename \textit{main} must be specified without the |.tex| extension.
\item
The filename \textit{main} is case sensitive
(even in case-insensitive file systems)
due to internal string comparison.
\item
The argument \textit{main} should be fully expanded, it cannot be a macro.
\item
Subdirectories and special characters should be avoided in filenames.
\item
The command |\childdocmain{|\textit{main}|}| must be followed by a whitespace.
It should not be followed immediately by another command
or by a comment mark `|%|'.
This is because the \TeX{} parser reads the token immediately following
the argument of |\childdocmain| and puts it
at the beginning of every child section;
however, a white\-space is ignored.
\end{itemize}

%%%%%%%%%%%%%%%%%%%%%%%%%%%%%%%%%%%%%%%%
\paragraph{Content of Main File.}

It is advisable to place all content in the child files included by |\include|.
Any output contained in the main file will appear in all child documents
unless suppressed manually;
it cannot be suppressed automatically by the |\includeonly| directive
and thus should normally be avoided.
A method to include some content in the main file
by means of conditional processing is described in \secref{sec:conditional}.

%%%%%%%%%%%%%%%%%%%%%%%%%%%%%%%%%%%%%%%%
\paragraph{Page Numbering.}

When only a part of the document is compiled,
the appropriate numbering of pages
(as well as other status parameters)
is determined from the |.aux| files.
The latter contain information from previous passes.
However this information needs to propagate through
all intermediate child documents.
Therefore the page numbering in child documents may well
be inconsistent until the complete document is compiled at least once.

A useful (if unconventional) way to always ensure a consistent
page numbering is to restart the numbering in each child document
and denote the pages by `\textit{child}|.|\textit{page}'
where \textit{child} represents the chapter/section number of the child file.
This can be achieved by the command
|\numberwithin{page}{|\textit{child}|}|
of the \textsf{amsmath} package
where \textit{child} can be |chapter| or |section|
depending on the chosen structuring.
Alternatively, one can modify the macro |\thepage| appropriately
and reset the counter |page| at the start of each child file.

%%%%%%%%%%%%%%%%%%%%%%%%%%%%%%%%%%%%%%%%%%%%%%%%%%%%%%%%%%%%%%%%%%%%%%%%%%%%%%%%
\subsection{Conditional Processing}
\label{sec:conditional}

The package provides a mechanism to compile different versions
of a document. To customise the versions further some conditional processing
can come in handy to distinguish which version is being compiled.
The package provides two macros to describe the compilation context:

%%%%%%%%%%%%%%%%%%%%%%%%%%%%%%%%%%%%%%%%
\DescribeMacro{\ifchilddoc}
The conditional |\ifchilddoc| distinguishes between the compilation of
child documents and the main document:
%
\begin{center}
|\ifchilddoc |\textit{child-code}| |[|\||else |\textit{main-code}]| \||fi|
\end{center}

%%%%%%%%%%%%%%%%%%%%%%%%%%%%%%%%%%%%%%%%
\DescribeMacro{\childdocname}
\DescribeMacro{\childdocjob}
The macro |\childdocname| contains the filename (without extension)
of the main or child file being processed.
Note that |\childdocjob| will always contain the name of the main file.

%%%%%%%%%%%%%%%%%%%%%%%%%%%%%%%%%%%%%%%%
\paragraph{Title Page.}

Conditional processing can be used to include a title or banner page
in the main document when proper precautions are taken.
Importantly, the code in the main file should ensure that the page counter
(as well as other status parameters which are stored in the |.aux| files)
takes the same value after the conditional processing.
Otherwise the page numbers may take divergent values
depending on which part is compiled.

For example, a title page could be declared by:
%
\begin{center}
\begin{tabular}{l}
|\ifchilddoc\||else|\\
|\addtocounter{page}{-1}|\\
\textit{code for title page}\\
|\newpage|\\
|\||fi|
\end{tabular}
\end{center}
%
A banner page for the child documents can be generated by:
%
\begin{center}
\begin{tabular}{l}
|\ifchilddoc|\\
|\addtocounter{page}{-1}|\\
\textit{code for banner page}\\
|\newpage|\\
|\||fi|
\end{tabular}
\end{center}
%
Here one could write a message such as:
\begin{center}
|This is the part \childdocname{} of \childdocjob{}.|
\end{center}

%%%%%%%%%%%%%%%%%%%%%%%%%%%%%%%%%%%%%%%%%%%%%%%%%%%%%%%%%%%%%%%%%%%%%%%%%%%%%%%%
\subsection{Flags}
\label{sec:flags}

The package makes it easy to generate different versions
of the main or child documents.
To this end compilation flags can be defined
and assigned different default values.
They will be particularly useful in conjunction
with the forwarding mechanism described in \secref{sec:forward}.

For example, it may be useful to have a flag |\version|
which can be set to |draft| or |final|.
The document source will contain some conditional code
depending on the value of |\version|.
Suppose further, the flag should default to |final| for the main file
and to |draft| for child files
which is a natural assignment for editing the document.
This is achieved by placing the following code
in the preamble of the main document
(below the |\childdocmain| directive):
%
\begin{center}
\begin{tabular}{l}
|\ifchilddoc|\\
|\providecommand{\version}{draft}|\\
|\||else|\\
|\providecommand{\version}{final}|\\
|\||fi|
\end{tabular}
\end{center}
%
The definition by |\providecommand| makes sure
that previous definitions are not overwritten.
Further statements |\providecommand{\version}{...}|
can thus be added before the above code to override it.

For the main file, one might add a line
(between |\childdocmain| and the above block)
%
\begin{center}
|%\ifchilddoc\||else\providecommand{\version}{draft}\||fi|
\end{center}
%
which can be uncommented to produce a draft version.
Likewise one can add a line to the very top of a child file
(above the |\childdocof{|\textit{main}|}| directive)
%
\begin{center}
|%\providecommand{\version}{final}|
\end{center}
%
which can be uncommented to produce the final version of this child document.

%%%%%%%%%%%%%%%%%%%%%%%%%%%%%%%%%%%%%%%%%%%%%%%%%%%%%%%%%%%%%%%%%%%%%%%%%%%%%%%%
\subsection{Forwarding}
\label{sec:forward}

Different versions of the main or child documents
using compilation flags as described in \secref{sec:flags}
can be (permanently) stored in different files
for convenient compilation, viewing and distribution.
To this end, the package defines a command
to pass on compilation to a different file:

%%%%%%%%%%%%%%%%%%%%%%%%%%%%%%%%%%%%%%%%
\DescribeMacro{\childdocforward}
The command |\childdocforward| redirects processing to
another source file:
%
\begin{center}
\begin{tabular}{l}
|\input{childdoc.def}|\\
|\childdocforward[|\textit{main}|]{|\textit{dest}|}|\\
\end{tabular}
\end{center}
%
The argument \textit{dest} is the destination file
(without extension).
It should be the main file or one of the child files.
Note that further \textsf{childdoc} directives
such as |\childdocof| and |\childdocforward|
in the indicated file will be processed in this form.
The optional argument \textit{main}
passes on directly to the main file \textit{main}
while pretending to compile the child \textit{dest}.
This form behaves as if \textit{dest}
issues |\childdocof{|\textit{main}|}| right away,
and no further \textsf{childdoc} directives will be processed.

%%%%%%%%%%%%%%%%%%%%%%%%%%%%%%%%%%%%%%%%
\DescribeMacro{\...prefix}
In the alternative form |\childdocforwardprefix|,
%
\begin{center}
\begin{tabular}{l}
|\input{childdoc.def}|\\
|\childdocforwardprefix[|\textit{main}|]{|\textit{prefix}|}{|\textit{dest}|}|
\end{tabular}
\end{center}
%
the destination file is determined by a pattern
depending on the current file:
To make this work, the current file must be called
`{\textit{prefix}\hspace{0.2em}\textit{suffix}}'
with \textit{prefix} matching precisely the argument.
Processing is then passed on to the file
`{\textit{dest}\hspace{0.2em}\textit{suffix}}'.
Surely, the same effect is achieved by
directly specifying the
argument `{\textit{dest}\hspace{0.2em}\textit{suffix}}'
in the first form.
However, that requires to set up a different file
for each child. With the alternative form of the command
all these files can have exactly the same content
which simplifies setting them up and maintaining them.

For example, the following file |draft.tex|
with a compilation flag |\version| as described in \secref{sec:flags}
compiles the main document as a draft:
%
\begin{center}
\begin{tabular}{l}
|\def\version{draft}|\\
|\input{childdoc.def}|\\
|\childdocforward{|\textit{main}|}|
\end{tabular}
\end{center}
%
Likewise, the following files |final|\textit{nn}|.tex|
compile the final version of the child document
|child|\textit{nn}|.tex|:
%
\begin{center}
\begin{tabular}{l}
|\def\version{final}|\\
|\input{childdoc.def}|\\
|\childdocforwardprefix{final}{child}|
\end{tabular}
\end{center}
%

Note that when several versions of a main file and/or of each child file
are to be generated, it may be convenient to set up a |Makefile| or
shell script to automatise the process.

%%%%%%%%%%%%%%%%%%%%%%%%%%%%%%%%%%%%%%%%%%%%%%%%%%%%%%%%%%%%%%%%%%%%%%%%%%%%%%%%
\subsection{Command Line Processing}
\label{sec:commandline}

The effect of redirection files can also be achieved by invoking
the \LaTeX{} compiler with a more elaborate command line.
Most conveniently this should be done as part
of a shell script or a |Makefile|.

When using \textsf{childdoc} in the main file, the following
command lines effectively perform a redirection
(note that depending on the shell being used,
backslashes may have to be doubled: `|\|' $\to$ `|\\|'):
%
\begin{center}
|... -jobname "|\textit{target}|" |\\|"|[\textit{flags}]%
|\input{childdoc.def}\childdocforward[|\textit{main}|]{|\textit{dest}|}"|
\end{center}
%
Here \textit{target} is the name of the output file,
\textit{main} is the name of the main file
and \textit{dest} is the name of the main or child file to be processed
(all filenames without extensions).
The optional argument \textit{main} can be omitted
if \textit{main} matches \textit{dest}.
Optionally, compilation \textit{flags} can be defined via |\def| commands.
This command line makes the \TeX{} engine believe
it is compiling the file \textit{target}
whose content is specified as the latter parameter.
The provided code then forwards the processing to
\textit{main} or \textit{dest} as described in \secref{sec:forward}.

%%%%%%%%%%%%%%%%%%%%%%%%%%%%%%%%%%%%%%%%%%%%%%%%%%%%%%%%%%%%%%%%%%%%%%%%%%%%%%%%
\subsection{Include by Input}
\label{sec:input}

Including child documents by |\include| has some restrictions by design.
Most notably, the content of a child document always occupies
its own set of pages; pages cannot be shared between child documents.
Usually, this behaviour makes perfect sense
because each child document contain an essential part of the document.
However, in some situations it may be desirable to compose
a document from a collection of parts
without having mandatory page breaks between then.
For this case, the package
provides a mechanism to include parts
by |\input| which can also be processed individually.
However, by construction this mechanism
requires manual handling of the content to be output.

%%%%%%%%%%%%%%%%%%%%%%%%%%%%%%%%%%%%%%%%
\DescribeMacro{\ifchilddocmanual}
The main file should be prepared as usual, see \secref{sec:include}.
However, the document body must make a distinction
between processing of an individual part and of the main document, e.g.:
%
\begin{center}
\begin{tabular}{l}
|\ifchilddocmanual|\\
|\input{\childdocname}|\\
|\||else|\\
\textit{document body with }|\input{|\textit{part}|}|\\
|\||fi|
\end{tabular}
\end{center}
%
The conditional |\ifchilddocmanual| is true whenever
a part to be included by |\input| is being compiled,
and the name of the part is stored in |\childdocname|.

%%%%%%%%%%%%%%%%%%%%%%%%%%%%%%%%%%%%%%%%
\DescribeMacro{\childdocby}
Each part to be included by |\input| should start with:
%
\begin{center}
\begin{tabular}{l}
|\input{childdoc.def}|\\
|\childdocby{|\textit{main}|}|\\
\end{tabular}
\end{center}
%
The directive |\childdocby| is similar to |\childdocof|
described in \secref{sec:include},
but the subsequent selection of content must be done manually.
To that end, both |\ifchilddoc| and |\ifchilddocmanual|
will be true upon processing of a part,
and the name of the part is stored in |\childdocname|.
Note that |\jobname| will be set to the filename of the current part
so that each part receives an individual |.aux| file
that does not interfere with the |.aux| file(s) of the main document.
This behaviour can be altered by the alternative form
|\childdocby[*]{|\textit{main}|}| (with a non-empty optional argument)
which uses the |.aux| file of the main document
by setting |\jobname| to \textit{main}.

%%%%%%%%%%%%%%%%%%%%%%%%%%%%%%%%%%%%%%%%%%%%%%%%%%%%%%%%%%%%%%%%%%%%%%%%%%%%%%%%
\subsection{Driver Development}
\label{sec:driver}

The \textsf{childdoc} mechanism can also be use for the development
of definition files such as \LaTeX{} styles or classes.
This case differs from the above setup with multiple parts
included by |\include| in that no |\includeonly| should be invoked.
This can be achieved by starting the include file
(before |\ProvidesPackage|) with:
%
\begin{center}
\begin{tabular}{l}
|\input{childdoc.def}|\\
|\childdocforward{|\textit{main}|}|\\
\end{tabular}
\end{center}
%
or alternatively with:
%
\begin{center}
\begin{tabular}{l}
|\input{childdoc.def}|\\
|\childdocby{|\textit{main}|}|\\
\end{tabular}
\end{center}
%
Both forms have slightly different effects as described above.
The main file is prepared as usual, see \secref{sec:include}.

%%%%%%%%%%%%%%%%%%%%%%%%%%%%%%%%%%%%%%%%%%%%%%%%%%%%%%%%%%%%%%%%%%%%%%%%%%%%%%%%
\subsection{Legacy Detection}
\label{sec:detection}

The directive |\childdocmain| in the main file can detect
whether the complete document or merely a child is to be compiled
even without using the directive |\childdocof|.
This method is deprecated because it is less robust
and there is no compelling reason to use it;
it is merely provided for backward compatibility
and it may be removed in future versions.

If the detection mechanism is to be used,
it is mandatory to correctly specify
the filename of the main file as the argument of |\childdocmain|:
%
\begin{center}
\begin{tabular}{l}
|\input{childdoc.def}|\\
|\childdocmain{|\textit{main}|}|\\
\end{tabular}
\end{center}
%
If |\jobname| does not match the argument \textit{main} of |\childdocmain|,
it is assumed that |\jobname| points to the child file to be compiled.
When using |\childdocmain| with the main file specified as argument,
it suffices to start a child file
with just |\input{|\textit{main}|}|
without loading of the package and using |\childdocof|.
If instead all processing is done
with the appropriate \textsf{childdoc} directives,
the argument of \textit{main} of |\childdocmain| can be empty.

An alternative version of the command line processing described
in \secref{sec:commandline} using the detection mechanism reads:
%
\begin{center}
|... -jobname "|\textit{target}|" "|[\textit{flags}]%
[|\def\jobname{|\textit{dest}|}|]|\input{|\textit{main}|}"|
\end{center}

%%%%%%%%%%%%%%%%%%%%%%%%%%%%%%%%%%%%%%%%%%%%%%%%%%%%%%%%%%%%%%%%%%%%%%%%%%%%%%%%
\subsection{Manual Code}
\label{sec:manual}

In case one cannot be certain whether the definitions file |childdoc.def|
is installed on the target \TeX{} distribution
and one prefers not to ship it,
it is conceivable to paste a few relevant commands into the sources.

To that end, drop all statements |\input{childdoc.def}|
and perform the replacements as outlined below.
Instead of |\childdocmain{|\textit{main}|}| add the following code
to the top of the main file:
%
\begin{center}
\begin{tabular}{l}
|\||ifdefined\childdocname\endinput\||fi\newif\ifchilddoc|\\
|\edef\childdocname{\scantokens\expandafter{\jobname\noexpand}}|\\
|\def\childdocmain{|\textit{main}|}\||ifx\childdocmain\childdocname\||else|\\
|\childdoctrue\includeonly{\childdocname}\let\jobname\childdocmain\||fi|\\
\end{tabular}
\end{center}
%
Instead of |\childdocof{|\textit{main}|}| just include the main file
at the top of each child file:
%
\begin{center}
|\input{|\textit{main}|}|
\end{center}
%
A simple redirection |\childdocforward{|\textit{dest}|}| is achieved by:
%
\begin{center}
|\def\jobname{|\textit{dest}|}\input{\jobname}|
\end{center}
%
The redirection with prefix
|\childdocforwardprefix[|\textit{prefix}|]{|\textit{dest}|}|
is accomplished by:
%
\begin{center}
\begin{tabular}{l}
|{\edef\jobname{\scantokens\expandafter{\jobname\noexpand}}|\\
|\def\redirectjob |\textit{prefix}|#1~~~{\gdef\jobname{|\textit{dest}|#1}}|\\
|\expandafter\redirectjob\jobname~~~}\input{\jobname}|
\end{tabular}
\end{center}

In an alternative approach,
child documents can be compiled by a specific command line
without additional code or specific definitions:
%
\begin{center}
|... -jobname "|\textit{target}|" "|[\textit{flags}]%
|\includeonly{|\textit{dest}|}\input{|\textit{main}|}"|
\end{center}
%

%%%%%%%%%%%%%%%%%%%%%%%%%%%%%%%%%%%%%%%%%%%%%%%%%%%%%%%%%%%%%%%%%%%%%%%%%%%%%%%%
%%%%%%%%%%%%%%%%%%%%%%%%%%%%%%%%%%%%%%%%%%%%%%%%%%%%%%%%%%%%%%%%%%%%%%%%%%%%%%%%
\section{Information}

%%%%%%%%%%%%%%%%%%%%%%%%%%%%%%%%%%%%%%%%%%%%%%%%%%%%%%%%%%%%%%%%%%%%%%%%%%%%%%%%
\subsection{Copyright}

Copyright \copyright{} 2017--2018 Niklas Beisert

This work may be distributed and/or modified under the
conditions of the \LaTeX{} Project Public License, either version 1.3
of this license or (at your option) any later version.
The latest version of this license is in
  \url{http://www.latex-project.org/lppl.txt}
and version 1.3 or later is part of all distributions of \LaTeX{}
version 2005/12/01 or later.

This work has the LPPL maintenance status `maintained'.

The Current Maintainer of this work is Niklas Beisert.

This work consists of the files |README.txt|, |childdoc.ins| and |childdoc.dtx|
as well as the derived files |childdoc.def|, |cdocsamp.tex|
with |cdocsch1.tex|, |cdocsch2.tex|, |cdocspt3.tex|, |cdocspt4.tex|,
|cdocsdrf.tex|, |cdocsfn1.tex|, |cdocsfn2.tex|
as well as |childdoc.pdf|.

%%%%%%%%%%%%%%%%%%%%%%%%%%%%%%%%%%%%%%%%%%%%%%%%%%%%%%%%%%%%%%%%%%%%%%%%%%%%%%%%
\subsection{Files and Installation}

The package consists of the files:
%
\begin{center}
\begin{tabular}{ll}
    |README.txt|   & readme file \\
    |childdoc.ins| & installation file \\
    |childdoc.dtx| & source file \\
    |childdoc.def| & definition file \\
    |cdocsamp.tex| & sample main file \\
    |cdocsch1.tex| & sample include file \\
    |cdocsch2.tex| & sample include file \\
    |cdocspt3.tex| & sample part file \\
    |cdocspt4.tex| & sample part file \\
    |cdocsdrf.tex| & sample redirection file \\
    |cdocsfn1.tex| & sample redirection file \\
    |cdocsfn2.tex| & sample redirection file \\
    |childdoc.pdf| & manual
\end{tabular}
\end{center}
%
The distribution consists of the files
|README.txt|, |childdoc.ins| and |childdoc.dtx|.
%
\begin{itemize}
\item
Run (pdf)\LaTeX{} on |childdoc.dtx|
to compile the manual |childdoc.pdf| (this file).
\item
Run \LaTeX{} on |childdoc.ins| to create the definitions file |childdoc.def|
and the sample |cdocsamp.tex| with include files
|cdocsch1.tex|, |cdocsch2.tex|, |cdocspt3.tex|, |cdocspt4.tex|,
|cdocsdrf.tex|, |cdocsfn1.tex|, |cdocsfn2.tex|.
Then copy the file |childdoc.def| to an appropriate directory of your \LaTeX{}
distribution, e.g.\ \textit{texmf-root}|/tex/latex/childdoc|.
\end{itemize}

%%%%%%%%%%%%%%%%%%%%%%%%%%%%%%%%%%%%%%%%%%%%%%%%%%%%%%%%%%%%%%%%%%%%%%%%%%%%%%%%
\subsection{Related CTAN Packages}

There are several other packages which offer a similar functionality:
%
\begin{itemize}
\item
The packages
\href{http://ctan.org/pkg/docmute}{\textsf{docmute}},
\href{http://ctan.org/pkg/includex}{\textsf{includex}} and
\href{http://ctan.org/pkg/standalone}{\textsf{standalone}}
provide commands to include only the document body of
a child file thus allowing both files to be compiled individually.
\item
The packages \href{http://ctan.org/pkg/subdocs}{\textsf{subdocs}}
and \href{http://ctan.org/pkg/subfiles}{\textsf{subfiles}}
provide structures in which the main and child documents can be
encapsulated and allowing them to be compiled individually.
The inclusion mechanism is different from the conventional |\include|.
\item
The package \href{http://ctan.org/pkg/combine}{\textsf{combine}}
is an elaborate solution to combine several documents into one.
\end{itemize}
%
See also the CTAN topic \href{http://ctan.org/topic/subdocs}{\textsf{subdocs}}
for further related packages.
The present package differs from the above solutions in that
a document structure constructed with the conventional |\include| mechanism
just needs two extra commands at the top of every file
such that all constituent files can be compiled individually.

%%%%%%%%%%%%%%%%%%%%%%%%%%%%%%%%%%%%%%%%%%%%%%%%%%%%%%%%%%%%%%%%%%%%%%%%%%%%%%%%
%\subsection{Feature Suggestions}
%
%The following is a list of features which may be useful for future
%versions of this package:
%%
%\begin{itemize}
%\item
%\ldots
%\end{itemize}

%%%%%%%%%%%%%%%%%%%%%%%%%%%%%%%%%%%%%%%%%%%%%%%%%%%%%%%%%%%%%%%%%%%%%%%%%%%%%%%%
\subsection{Revision History}

%%%%%%%%%%%%%%%%%%%%%%%%%%%%%%%%%%%%%%%%
\paragraph{v2.0:} 2018/12/30

\begin{itemize}
\item
immediate forward processing
\item
added |\childdocby| mechanism
\item
manual restructured
\end{itemize}

%%%%%%%%%%%%%%%%%%%%%%%%%%%%%%%%%%%%%%%%
\paragraph{v1.6:} 2018/01/17

\begin{itemize}
\item
application for development of include files
\item
corrections to manual
\end{itemize}

%%%%%%%%%%%%%%%%%%%%%%%%%%%%%%%%%%%%%%%%
\paragraph{v1.5:} 2017/05/21

\begin{itemize}
\item
more complete structuring introduced
\item
|\childdocof| introduced
\item
|\childdoc| renamed to |\childdocmain|
\item
|\childredirect| renamed to |\childdocforward| and |\childdocforwardprefix|
and functionality expanded
\end{itemize}

%%%%%%%%%%%%%%%%%%%%%%%%%%%%%%%%%%%%%%%%
\paragraph{v1.0:} 2017/04/27

\begin{itemize}
\item
manual and install package
\item
first version published on CTAN
\end{itemize}

%%%%%%%%%%%%%%%%%%%%%%%%%%%%%%%%%%%%%%%%
\paragraph{v0.6:} 2017/04/26

\begin{itemize}
\item
redirection mechanism added
\end{itemize}

%%%%%%%%%%%%%%%%%%%%%%%%%%%%%%%%%%%%%%%%
\paragraph{v0.5:} 2017/04/26

\begin{itemize}
\item
functionality in definition file
\end{itemize}


%%%%%%%%%%%%%%%%%%%%%%%%%%%%%%%%%%%%%%%%%%%%%%%%%%%%%%%%%%%%%%%%%%%%%%%%%%%%%%%%
%%%%%%%%%%%%%%%%%%%%%%%%%%%%%%%%%%%%%%%%%%%%%%%%%%%%%%%%%%%%%%%%%%%%%%%%%%%%%%%%
%%%%%%%%%%%%%%%%%%%%%%%%%%%%%%%%%%%%%%%%%%%%%%%%%%%%%%%%%%%%%%%%%%%%%%%%%%%%%%%%
\appendix

\settowidth\MacroIndent{\rmfamily\scriptsize 000\ }

 \DocInput{childdoc.dtx}

\end{document}
%</driver>
% \fi
%
% %%%%%%%%%%%%%%%%%%%%%%%%%%%%%%%%%%%%%%%%%%%%%%%%%%%%%%%%%%%%%%%%%%%%%%%%%%%%%%
% %%%%%%%%%%%%%%%%%%%%%%%%%%%%%%%%%%%%%%%%%%%%%%%%%%%%%%%%%%%%%%%%%%%%%%%%%%%%%%
% \section{Sample}
%\iffalse
%<*samplemain>
%\fi
%
% The following presents a sample document
% with two chapters, two parts, a title page,
% a compile flag as well as three forwarding files to set the flag.
% It consists of eight |.tex| files:
% \begin{center}
% \begin{tabular}{ll}
% |cdocsamp.tex|&main file\\
% |cdocsch1.tex|&include file for chapter 1\\
% |cdocsch2.tex|&include file for chapter 2\\
% |cdocspt3.tex|&include file for part 3\\
% |cdocspt4.tex|&include file for part 4\\
% |cdocsdrf.tex|&forwarding file for main file in draft mode\\
% |cdocsfi1.tex|&forwarding file for final version of chapter 1\\
% |cdocsfi2.tex|&forwarding file for final version of chapter 2\\
% \end{tabular}
% \end{center}
% Each of the eight files can be compiled directly by the \LaTeX{} compiler.
%
% %%%%%%%%%%%%%%%%%%%%%%%%%%%%%%%%%%%%%%
% \paragraph{Main File.}
%
% The main file is called |cdocsamp.tex|.
%
% Load the \textsf{childdoc} definitions and
% declare the filename for the main document:
%    \begin{macrocode}
\input{childdoc.def}
\childdocmain{}
%    \end{macrocode}

% Optional override for |\version| flag:
%    \begin{macrocode}
%%\ifchilddoc\else\providecommand{\version}{draft}\fi
%    \end{macrocode}

% Define the default values for the |\version| flag
% (|final| for the main file and |draft| for childs):
%    \begin{macrocode}
\ifchilddoc
\providecommand{\version}{draft}
\else
\providecommand{\version}{final}
\fi
%    \end{macrocode}

% Load the standard document class:
%    \begin{macrocode}
\documentclass[12pt]{article}
%    \end{macrocode}

% Start the document body:
%    \begin{macrocode}
\begin{document}
%    \end{macrocode}

% Declare a title page.
% Print title, part of document being processed and version flag:
%    \begin{macrocode}
\addtocounter{page}{-1}
\begin{center}
{\LARGE\bfseries{}childdoc example\par}
\vspace{1cm}
\ifchilddoc
\ifchilddocmanual part\else chapter\fi:
`\childdocname' of `\childdocjob'\par
\else
main document: `\childdocjob'\par
\fi
version: \version\par
\end{center}
\newpage
%    \end{macrocode}

% Manually include selected file,
% otherwise process as usual:
%    \begin{macrocode}
\ifchilddocmanual
\section*{part `\childdocname'}
\input{\childdocname}
\else
%    \end{macrocode}

% Include the two chapters:
%    \begin{macrocode}
\include{cdocsch1}
\include{cdocsch2}
%    \end{macrocode}

% Include the two parts unless only chapters should be displayed:
%    \begin{macrocode}
\ifchilddoc\else
\section{part three}
\input{cdocspt3}
\section{part four}
\input{cdocspt4}
\fi
%    \end{macrocode}

% Process as usual until here:
%    \begin{macrocode}
\fi
%    \end{macrocode}

% End of document body:
%    \begin{macrocode}
\end{document}
%    \end{macrocode}
%\iffalse
%</samplemain>
%\fi
%
% %%%%%%%%%%%%%%%%%%%%%%%%%%%%%%%%%%%%%%
% \paragraph{Chapter Include Files.}
%
% The include files are called |cdocsch1.tex| and |cdocsch2.tex|.
%
%\iffalse
%<*samplechap1|samplechap2>
%\fi

% Optional override for |\version| flag:
%    \begin{macrocode}
%%\providecommand{\version}{final}
%    \end{macrocode}

% Include the main document:
%    \begin{macrocode}
\input{childdoc.def}
\childdocof{cdocsamp}
%    \end{macrocode}

%\iffalse
%</samplechap1|samplechap2>
%\fi
%
%\iffalse
%<*samplechap1>
%\fi
% Some text for chapter 1:
%    \begin{macrocode}
\section{one}
some text in chapter one
%    \end{macrocode}

%\iffalse
%</samplechap1>
%\fi
% Some text for chapter 2:
%\iffalse
%<*samplechap2>
%\fi
%    \begin{macrocode}
\section{two}
more text in chapter two
%    \end{macrocode}

%\iffalse
%</samplechap2>
%\fi
%
% %%%%%%%%%%%%%%%%%%%%%%%%%%%%%%%%%%%%%%
% \paragraph{Part Include Files.}
%
% The include files are called |cdocspt3.tex| and |cdocspt4.tex|.
%
%\iffalse
%<*samplepart3|samplepart4>
%\fi

% Optional override for |\version| flag:
%    \begin{macrocode}
%%\providecommand{\version}{final}
%    \end{macrocode}

% Include the main document:
%    \begin{macrocode}
\input{childdoc.def}
\childdocby{cdocsamp}
%    \end{macrocode}

%\iffalse
%</samplepart3|samplepart4>
%\fi
%
%\iffalse
%<*samplepart3>
%\fi
% Some text for part 3:
%    \begin{macrocode}
some text in part three
%    \end{macrocode}

%\iffalse
%</samplepart3>
%\fi
% Some text for part 4:
%\iffalse
%<*samplepart4>
%\fi
%    \begin{macrocode}
more text in part four
%    \end{macrocode}

%\iffalse
%</samplepart4>
%\fi
%
% %%%%%%%%%%%%%%%%%%%%%%%%%%%%%%%%%%%%%%
% \paragraph{Forwarding for a Complete Draft.}
%
% The following forwarding file |cdocsdrf.tex|
% compiles the main document in draft mode:
%\iffalse
%<*sampledraft>
%\fi
%    \begin{macrocode}
\def\version{draft}
\input{childdoc.def}
\childdocforward{cdocsamp}
%    \end{macrocode}

%\iffalse
%</sampledraft>
%\fi
%
% %%%%%%%%%%%%%%%%%%%%%%%%%%%%%%%%%%%%%%
% \paragraph{Forwarding for Final Version of the Chapters.}
%
% The following forwarding files |cdocsfn1.tex| and |cdocsfn2.tex|
% (with identical content)
% compile the final versions of the child documents
% |cdocsch1.tex| and |cdocsch2.tex|, respectively:
%\iffalse
%<*samplefinal>
%\fi
%    \begin{macrocode}
\def\version{final}
\input{childdoc.def}
\childdocforwardprefix[cdocsamp]{cdocsfn}{cdocsch}
%    \end{macrocode}

%\iffalse
%</samplefinal>
%\fi
%
% %%%%%%%%%%%%%%%%%%%%%%%%%%%%%%%%%%%%%%
% \paragraph{Command Line Processing.}
%
% The following three command lines generate the output files
% |cdocscld|, |cdocscl1| and |cdocscl2|
% which should be identical to
% |cdocsdrf|, |cdocsch1| and |cdocsfn2|, respectively:
% \begin{center}
% \begin{tabular}{l}
% |latex -jobname cdocscld \|\\
% |  "\def\version{draft}\input{childdoc.def}\childdocforward{cdocsamp}"|\\
% |latex -jobname cdocscl1 \|\\
% |  "\input{childdoc.def}\childdocforward[cdocsamp]{cdocsch1}"|\\
% |latex -jobname cdocscl2 \|\\
% |  "\def\version{final}\input{childdoc.def}\childdocforward{cdocsch2}"|
% \end{tabular}
% \end{center}
% Note that the trailing backslash on each first line
% merely continues the input to the second line
% (for convenient cut ant paste).
% Furthermore, the command |latex| can be replaced by any
% of its alternative versions such as |pdflatex|.
%
% %%%%%%%%%%%%%%%%%%%%%%%%%%%%%%%%%%%%%%%%%%%%%%%%%%%%%%%%%%%%%%%%%%%%%%%%%%%%%%
% %%%%%%%%%%%%%%%%%%%%%%%%%%%%%%%%%%%%%%%%%%%%%%%%%%%%%%%%%%%%%%%%%%%%%%%%%%%%%%
% \section{Implementation}
%\iffalse
%<*package>
%\fi
%
% This section describes the definitions file |childdoc.def|.

% The definitions cannot be loaded using |\usepackage| or |\RequirePackage|
% which has a mechanism to prevent loading a style file more than once.
% When loading the definitions by means of |\input|
% multiple instances have to be prevented manually:
%\iffalse
%This code needs to be before the `\ProvidesFile' directive
%which is defined at the beginning of this file.
%Therefore it is also placed there and commented out here.
%</package>
%<*discard>
%\fi
%    \begin{macrocode}
\ifdefined\childdocmain\endinput\fi
%    \end{macrocode}
%\iffalse
%</discard>
%<*package>
%\fi
%
% \macro{\ifchilddoc}
% \macro{\ifchilddocmanual}
% The conditional |\ifchilddoc| tells whether a
% child (true) or main (false) document is being compiled.
% The conditional |\ifchilddocmanual| tells whether
% the |\includeonly| mechanism is used (false) or
% the selection of child files must be performed manually (true).
% The definitions initialise to false:
%    \begin{macrocode}
\newif\ifchilddoc
\newif\ifchilddocmanual
%    \end{macrocode}

% \macro{\childdocname}
% \macro{\childdocjob}
% The macro |\childdocname| stores the name of the main document
% to be compiled. The macro |\childdocjob| stores the name of
% the document on which the \LaTeX{} compiler was originally invoked.
% The content of |\jobname| cannot be compared
% to filenames specified in the source due to different catcodes.
% The following code rescans |\jobname|, stores the result
% in |\childdocname| and saves a copy in |\childdocjob|:
%    \begin{macrocode}
\edef\childdocname{\scantokens\expandafter{\jobname\noexpand}}
\let\childdocjob\childdocname
%    \end{macrocode}

% \macro{\childdocdisable}
% The macro |\childdocdisable| prevents the main file
% from being processed more than once.
% At this stage, the main document command |\childdocmain|
% is assumed to be called once again where it should do nothing.
% Any subsequent call to it should prevent
% a secondary processing of the main document
% It overwrites the forwarding commands
% |\childdocof| and |\childdocforward|
% with empty macros to prevent further inclusions of the main document:
%    \begin{macrocode}
\newcommand{\childdocdisable}
{
  \renewcommand{\childdocmain}[1]{\renewcommand{\childdocmain}[1]{\endinput}}
  \renewcommand{\childdocof}[1]{}
  \renewcommand{\childdocby}[2][]{}
  \renewcommand{\childdocforward}[2][]{}
  \renewcommand{\childdocdisable}{}
}
%    \end{macrocode}

% \macro{\childdocmain}
% The macro |\childdocmain| is to be called at the top of the main file
% with nothing or the main filename (without extension) as argument.
% First, it breaks loops.
% If the argument is not empty and does not match |\childdocname|
% (which is set by the first inclusion of |childdoc.def|),
% |\ifchilddoc| is set to true, |\includeonly| is applied to the child file
% and |\jobname| is set to the main file
% (for proper handling of |.aux| files):
%    \begin{macrocode}
\newcommand{\childdocmain}[1]
{
  \childdocdisable\childdocmain{}
  \if?#1?\else
    \begingroup
      \def\childdoctmp{#1}
      \ifx\childdoctmp\childdocname
        \def\childdoctmp{}
      \else
        \def\childdoctmp
        {
          \childdoctrue
          \includeonly{\childdocname}
          \def\childdocjob{#1}
          \def\jobname{#1}
        }
      \fi
      \expandafter
    \endgroup
    \childdoctmp
  \fi
}
%    \end{macrocode}

% \macro{\childdocof}
% The command |\childdocof| redirects
% compilation to the main file |#1|.
%    \begin{macrocode}
\newcommand{\childdocof}[1]
{
  \childdocdisable
  \childdoctrue
  \includeonly{\childdocname}
  \def\jobname{#1}
  \def\childdocjob{#1}
  \input{#1}
}
%    \end{macrocode}

% \macro{\childdocby}
% The command |\childdocby| ....
%    \begin{macrocode}
\newcommand{\childdocby}[2][]
{
  \childdocdisable
  \childdoctrue
  \childdocmanualtrue
  \if?#1?\else
    \def\jobname{#2}
  \fi
  \def\childdocjob{#2}
  \input{#2}
  \endinput
}
%    \end{macrocode}

% \macro{\childdocforward}
% The command |\childdocforward| redirects
% compilation to the main file or
% (if the optional argument is given) a child file.
% Parameters are set as if the main file
% or a child file starting with |\childdocof| was compiled.
% Then compilation is handed over to the main file:
%    \begin{macrocode}
\newcommand{\childdocforward}[2][]
{
  \begingroup
    \if?#1?
      \def\childdoctmp
      {
        \def\childdocname{#2}
        \def\childdocjob{#2}
        \def\jobname{#2}
        \input{#2}
        \endinput
      }
    \else
      \def\childdoctmp
      {
        \childdocdisable
        \def\childdocname{#2}
        \childdoctrue
        \includeonly{#2}
        \def\childdocjob{#1}
        \def\jobname{#1}
        \input{#1}
        \endinput
      }
    \fi
    \expandafter
  \endgroup
  \childdoctmp
}
%    \end{macrocode}

% \macro{\childdocforwardprefix}
% The command |\childdocforwardprefix| redirects
% compilation to the main or a child file by means of a pattern.
% The prefix |#1| in the current filename is replaced by |#2|
% and the suffix of the current filename is kept
% (it is assumed that the filename does not contain the substring `|~~~|'
% which is used as a delimiter).
% Compilation is handed over to the new file by |\childdocforward|:
%    \begin{macrocode}
\newcommand{\childdocforwardprefix}[3][]
{
  \begingroup
    \def\childdocextract #2##1~~~{\def\childdoctmp{\childdocforward[#1]{#3##1}}}
    \expandafter\childdocextract\childdocname~~~
    \expandafter
  \endgroup
  \childdoctmp
}
%    \end{macrocode}

% \macro{\childdoc}
% The deprecated macro |\childdoc| is a legacy version of |\childdocmain|:
%    \begin{macrocode}
\newcommand{\childdoc}{\childdocmain}
%    \end{macrocode}

% \macro{\childdocredirect}
% The deprecated macro |\childdocredirect| is a legacy version
% of |\childdocforward| and |\childdocforwardprefix|:
%    \begin{macrocode}
\newcommand{\childdocredirect}[2][]
{
  \begingroup
    \if?#1?
      \def\childdoctmp{\childdocforward{#2}}
    \else
      \def\childdoctmp{\childdocforwardprefix{#1}{#2}}
    \fi
    \expandafter
  \endgroup
  \childdoctmp
}
%    \end{macrocode}

%\iffalse
%</package>
%\fi
%
\endinput
|\\
|\childdocforwardprefix[|\textit{main}|]{|\textit{prefix}|}{|\textit{dest}|}|
\end{tabular}
\end{center}
%
the destination file is determined by a pattern
depending on the current file:
To make this work, the current file must be called
`{\textit{prefix}\hspace{0.2em}\textit{suffix}}'
with \textit{prefix} matching precisely the argument.
Processing is then passed on to the file
`{\textit{dest}\hspace{0.2em}\textit{suffix}}'.
Surely, the same effect is achieved by
directly specifying the
argument `{\textit{dest}\hspace{0.2em}\textit{suffix}}'
in the first form.
However, that requires to set up a different file
for each child. With the alternative form of the command
all these files can have exactly the same content
which simplifies setting them up and maintaining them.

For example, the following file |draft.tex|
with a compilation flag |\version| as described in \secref{sec:flags}
compiles the main document as a draft:
%
\begin{center}
\begin{tabular}{l}
|\def\version{draft}|\\
|% \iffalse
%
% childdoc.dtx Copyright (C) 2017-2018 Niklas Beisert
%
% This work may be distributed and/or modified under the
% conditions of the LaTeX Project Public License, either version 1.3
% of this license or (at your option) any later version.
% The latest version of this license is in
%   http://www.latex-project.org/lppl.txt
% and version 1.3 or later is part of all distributions of LaTeX
% version 2005/12/01 or later.
%
% This work has the LPPL maintenance status `maintained'.
%
% The Current Maintainer of this work is Niklas Beisert.
%
% This work consists of the files childdoc.dtx and childdoc.ins
% and the derived files childdoc.def and cdocsamp.tex with
% cdocsch1.tex, cdocsch2.tex, cdocsdrf.tex, cdocsfn1.tex, cdocsfn2.tex.
%
%<package>\ifdefined\childdocmain\endinput\fi
%<package>\ProvidesFile{childdoc.def}[2018/12/30 v2.0 child document driver]
%<samplemain>\ProvidesFile{cdocsamp.tex}[2018/12/30 v2.0 sample for childdoc]
%<*driver>
%\ProvidesFile{childdoc.drv}[2018/12/30 v2.0 childdoc reference manual file]
\PassOptionsToClass{10pt,a4paper}{article}
\documentclass{ltxdoc}

\usepackage[margin=35mm]{geometry}
\usepackage{hyperref}
\usepackage{hyperxmp}
\usepackage[usenames]{color}

\hypersetup{colorlinks=true}
\hypersetup{pdfstartview=FitH}
\hypersetup{pdfpagemode=UseNone}
\hypersetup{pdfsource={}}
\hypersetup{pdflang={en-UK}}
\hypersetup{pdfcopyright={Copyright 2017-2018 Niklas Beisert.
  This work may be distributed and/or modified under the
  conditions of the LaTeX Project Public License, either version 1.3
  of this license or (at your option) any later version.}}
\hypersetup{pdflicenseurl={http://www.latex-project.org/lppl.txt}}
\hypersetup{pdfcontactaddress={ETH Zurich, ITP, HIT K,
  Wolfgang-Pauli-Strasse 27}}
\hypersetup{pdfcontactpostcode={8093}}
\hypersetup{pdfcontactcity={Zurich}}
\hypersetup{pdfcontactcountry={Switzerland}}
\hypersetup{pdfcontactemail={nbeisert@itp.phys.ethz.ch}}
\hypersetup{pdfcontacturl={http://people.phys.ethz.ch/\xmptilde nbeisert/}}

\newcommand{\secref}[1]{\hyperref[#1]{section \ref*{#1}}}

\parskip1ex
\parindent0pt
\let\olditemize\itemize
\def\itemize{\olditemize\parskip0pt}

\begin{document}

\title{The \textsf{childdoc} Package}
\hypersetup{pdftitle={The childdoc Package}}
\author{Niklas Beisert\\[2ex]
  Institut f\"ur Theoretische Physik\\
  Eidgen\"ossische Technische Hochschule Z\"urich\\
  Wolfgang-Pauli-Strasse 27, 8093 Z\"urich, Switzerland\\[1ex]
  \href{mailto:nbeisert@itp.phys.ethz.ch}
  {\texttt{nbeisert@itp.phys.ethz.ch}}}
\hypersetup{pdfauthor={Niklas Beisert}}
\hypersetup{pdfsubject={Manual for the LaTeX2e Package childdoc}}
\date{30 December 2018, \textsf{v2.0}}
\maketitle

\begin{abstract}\noindent
\textsf{childdoc} is a \LaTeXe{} package
that enables the direct compilation
of document sections included by |\include|
to individual files.
\end{abstract}

\begingroup
\parskip0ex
\tableofcontents
\endgroup

%%%%%%%%%%%%%%%%%%%%%%%%%%%%%%%%%%%%%%%%%%%%%%%%%%%%%%%%%%%%%%%%%%%%%%%%%%%%%%%%
%%%%%%%%%%%%%%%%%%%%%%%%%%%%%%%%%%%%%%%%%%%%%%%%%%%%%%%%%%%%%%%%%%%%%%%%%%%%%%%%
\section{Introduction}

\LaTeX{} provides a mechanism to structure a large document (such as a book)
into a main file and several child files (containing the chapters)
using the |\include| command.
This mechanism is beneficial for documents
which span hundreds of pages in order to
make the source file(s) more manageable.
Moreover, compilation can be restricted to
selected child files by means of the |\includeonly| command.
The latter feature can be used to reduce the compilation time while editing
(this was significantly more useful in the earlier days of \LaTeX{})
or to generate a smaller document which is easier to navigate.
Another application of |\includeonly| is to generate
documents consisting of selected parts of the complete document.

However, there are a few drawbacks of the plain |\include| mechanism:
\begin{itemize}
\item
The child files cannot be compiled on their own,
they can only be compiled via the main file.
A naive editing environment
(such as a text editor with an option
to have the current file processed by \LaTeX)
may require one to switch to the main file before compiling;
attempting to compile the child file produces errors.
\item
The main file must be modified (each time)
to adjust the |\includeonly| command
to the present needs. This easily leaves the main file in a messy state.
\item
The generated document will always carry the filename
of the main document. This is inconvenient if
several child files are to be compiled and
to be kept for distribution.
\end{itemize}

The present package provides a simple interface
to make child files individually compilable by \LaTeX{}.
Compiling a child file then has the same effect as compiling
the main file with an |\includeonly| command
to select the appropriate child.
Moreover the generated document will carry the name of the child
rather than the main file.
This resolves all three above issues.

This feature is meant to make the editing of books,
thesis documents and lecture notes somewhat more convenient.
However, the package can also be used efficiently for
composing a series of documents (such as exercise sheets)
which are typically distributed individually.
It then assists the author in generating the individual documents
(potentially in different versions)
as well as a document containing the collected series.
Another application is in developing style files
or other kinds of included material
where compilation of the style file could redirect
to a sample or test file.

%%%%%%%%%%%%%%%%%%%%%%%%%%%%%%%%%%%%%%%%%%%%%%%%%%%%%%%%%%%%%%%%%%%%%%%%%%%%%%%%
%%%%%%%%%%%%%%%%%%%%%%%%%%%%%%%%%%%%%%%%%%%%%%%%%%%%%%%%%%%%%%%%%%%%%%%%%%%%%%%%
\section{Usage}

First of all, the package \textsf{childdoc} is \emph{not} a standard
\LaTeXe{} |.sty| style file! Therefore it needs to be invoked in
a non-standard way.

%%%%%%%%%%%%%%%%%%%%%%%%%%%%%%%%%%%%%%%%%%%%%%%%%%%%%%%%%%%%%%%%%%%%%%%%%%%%%%%%
\subsection{Included Files}
\label{sec:include}

%%%%%%%%%%%%%%%%%%%%%%%%%%%%%%%%%%%%%%%%
\DescribeMacro{\childdocmain}
To use the package, add the commands
\begin{center}
\begin{tabular}{l}
|\input{childdoc.def}|\\
|\childdocmain{}|\\
\end{tabular}
\end{center}
at the very top of the main \LaTeX{} file,
in particular \emph{before} the |\documentclass| statement!
The argument of |\childdocmain| should be left empty
(but it must be present).

%%%%%%%%%%%%%%%%%%%%%%%%%%%%%%%%%%%%%%%%
\DescribeMacro{\childdocof}
Furthermore, add the commands
\begin{center}
\begin{tabular}{l}
|\input{childdoc.def}|\\
|\childdocof{|\textit{main}|}|\\
\end{tabular}
\end{center}
at the top of every child file \textit{child}
which is included by |\include{|\textit{child}|}|
from within the main file
(or at least for those files to be compiled individually).
The argument \textit{main} must be the filename of the main file.

There are a couple of
considerations in setting up the main and child documents:

%%%%%%%%%%%%%%%%%%%%%%%%%%%%%%%%%%%%%%%%
\paragraph{Restrictions.}

Please note the following restrictions:
\begin{itemize}
\item
|\childdocmain| must be called with one argument \textit{main}
to ensure compatibility with earlier version of the package.
It must either be empty (|\childdocmain{}|)
or precisely match the filename of the main file in which it is specified.
See \secref{sec:detection} for further information.
\item
The filename \textit{main} must be specified without the |.tex| extension.
\item
The filename \textit{main} is case sensitive
(even in case-insensitive file systems)
due to internal string comparison.
\item
The argument \textit{main} should be fully expanded, it cannot be a macro.
\item
Subdirectories and special characters should be avoided in filenames.
\item
The command |\childdocmain{|\textit{main}|}| must be followed by a whitespace.
It should not be followed immediately by another command
or by a comment mark `|%|'.
This is because the \TeX{} parser reads the token immediately following
the argument of |\childdocmain| and puts it
at the beginning of every child section;
however, a white\-space is ignored.
\end{itemize}

%%%%%%%%%%%%%%%%%%%%%%%%%%%%%%%%%%%%%%%%
\paragraph{Content of Main File.}

It is advisable to place all content in the child files included by |\include|.
Any output contained in the main file will appear in all child documents
unless suppressed manually;
it cannot be suppressed automatically by the |\includeonly| directive
and thus should normally be avoided.
A method to include some content in the main file
by means of conditional processing is described in \secref{sec:conditional}.

%%%%%%%%%%%%%%%%%%%%%%%%%%%%%%%%%%%%%%%%
\paragraph{Page Numbering.}

When only a part of the document is compiled,
the appropriate numbering of pages
(as well as other status parameters)
is determined from the |.aux| files.
The latter contain information from previous passes.
However this information needs to propagate through
all intermediate child documents.
Therefore the page numbering in child documents may well
be inconsistent until the complete document is compiled at least once.

A useful (if unconventional) way to always ensure a consistent
page numbering is to restart the numbering in each child document
and denote the pages by `\textit{child}|.|\textit{page}'
where \textit{child} represents the chapter/section number of the child file.
This can be achieved by the command
|\numberwithin{page}{|\textit{child}|}|
of the \textsf{amsmath} package
where \textit{child} can be |chapter| or |section|
depending on the chosen structuring.
Alternatively, one can modify the macro |\thepage| appropriately
and reset the counter |page| at the start of each child file.

%%%%%%%%%%%%%%%%%%%%%%%%%%%%%%%%%%%%%%%%%%%%%%%%%%%%%%%%%%%%%%%%%%%%%%%%%%%%%%%%
\subsection{Conditional Processing}
\label{sec:conditional}

The package provides a mechanism to compile different versions
of a document. To customise the versions further some conditional processing
can come in handy to distinguish which version is being compiled.
The package provides two macros to describe the compilation context:

%%%%%%%%%%%%%%%%%%%%%%%%%%%%%%%%%%%%%%%%
\DescribeMacro{\ifchilddoc}
The conditional |\ifchilddoc| distinguishes between the compilation of
child documents and the main document:
%
\begin{center}
|\ifchilddoc |\textit{child-code}| |[|\||else |\textit{main-code}]| \||fi|
\end{center}

%%%%%%%%%%%%%%%%%%%%%%%%%%%%%%%%%%%%%%%%
\DescribeMacro{\childdocname}
\DescribeMacro{\childdocjob}
The macro |\childdocname| contains the filename (without extension)
of the main or child file being processed.
Note that |\childdocjob| will always contain the name of the main file.

%%%%%%%%%%%%%%%%%%%%%%%%%%%%%%%%%%%%%%%%
\paragraph{Title Page.}

Conditional processing can be used to include a title or banner page
in the main document when proper precautions are taken.
Importantly, the code in the main file should ensure that the page counter
(as well as other status parameters which are stored in the |.aux| files)
takes the same value after the conditional processing.
Otherwise the page numbers may take divergent values
depending on which part is compiled.

For example, a title page could be declared by:
%
\begin{center}
\begin{tabular}{l}
|\ifchilddoc\||else|\\
|\addtocounter{page}{-1}|\\
\textit{code for title page}\\
|\newpage|\\
|\||fi|
\end{tabular}
\end{center}
%
A banner page for the child documents can be generated by:
%
\begin{center}
\begin{tabular}{l}
|\ifchilddoc|\\
|\addtocounter{page}{-1}|\\
\textit{code for banner page}\\
|\newpage|\\
|\||fi|
\end{tabular}
\end{center}
%
Here one could write a message such as:
\begin{center}
|This is the part \childdocname{} of \childdocjob{}.|
\end{center}

%%%%%%%%%%%%%%%%%%%%%%%%%%%%%%%%%%%%%%%%%%%%%%%%%%%%%%%%%%%%%%%%%%%%%%%%%%%%%%%%
\subsection{Flags}
\label{sec:flags}

The package makes it easy to generate different versions
of the main or child documents.
To this end compilation flags can be defined
and assigned different default values.
They will be particularly useful in conjunction
with the forwarding mechanism described in \secref{sec:forward}.

For example, it may be useful to have a flag |\version|
which can be set to |draft| or |final|.
The document source will contain some conditional code
depending on the value of |\version|.
Suppose further, the flag should default to |final| for the main file
and to |draft| for child files
which is a natural assignment for editing the document.
This is achieved by placing the following code
in the preamble of the main document
(below the |\childdocmain| directive):
%
\begin{center}
\begin{tabular}{l}
|\ifchilddoc|\\
|\providecommand{\version}{draft}|\\
|\||else|\\
|\providecommand{\version}{final}|\\
|\||fi|
\end{tabular}
\end{center}
%
The definition by |\providecommand| makes sure
that previous definitions are not overwritten.
Further statements |\providecommand{\version}{...}|
can thus be added before the above code to override it.

For the main file, one might add a line
(between |\childdocmain| and the above block)
%
\begin{center}
|%\ifchilddoc\||else\providecommand{\version}{draft}\||fi|
\end{center}
%
which can be uncommented to produce a draft version.
Likewise one can add a line to the very top of a child file
(above the |\childdocof{|\textit{main}|}| directive)
%
\begin{center}
|%\providecommand{\version}{final}|
\end{center}
%
which can be uncommented to produce the final version of this child document.

%%%%%%%%%%%%%%%%%%%%%%%%%%%%%%%%%%%%%%%%%%%%%%%%%%%%%%%%%%%%%%%%%%%%%%%%%%%%%%%%
\subsection{Forwarding}
\label{sec:forward}

Different versions of the main or child documents
using compilation flags as described in \secref{sec:flags}
can be (permanently) stored in different files
for convenient compilation, viewing and distribution.
To this end, the package defines a command
to pass on compilation to a different file:

%%%%%%%%%%%%%%%%%%%%%%%%%%%%%%%%%%%%%%%%
\DescribeMacro{\childdocforward}
The command |\childdocforward| redirects processing to
another source file:
%
\begin{center}
\begin{tabular}{l}
|\input{childdoc.def}|\\
|\childdocforward[|\textit{main}|]{|\textit{dest}|}|\\
\end{tabular}
\end{center}
%
The argument \textit{dest} is the destination file
(without extension).
It should be the main file or one of the child files.
Note that further \textsf{childdoc} directives
such as |\childdocof| and |\childdocforward|
in the indicated file will be processed in this form.
The optional argument \textit{main}
passes on directly to the main file \textit{main}
while pretending to compile the child \textit{dest}.
This form behaves as if \textit{dest}
issues |\childdocof{|\textit{main}|}| right away,
and no further \textsf{childdoc} directives will be processed.

%%%%%%%%%%%%%%%%%%%%%%%%%%%%%%%%%%%%%%%%
\DescribeMacro{\...prefix}
In the alternative form |\childdocforwardprefix|,
%
\begin{center}
\begin{tabular}{l}
|\input{childdoc.def}|\\
|\childdocforwardprefix[|\textit{main}|]{|\textit{prefix}|}{|\textit{dest}|}|
\end{tabular}
\end{center}
%
the destination file is determined by a pattern
depending on the current file:
To make this work, the current file must be called
`{\textit{prefix}\hspace{0.2em}\textit{suffix}}'
with \textit{prefix} matching precisely the argument.
Processing is then passed on to the file
`{\textit{dest}\hspace{0.2em}\textit{suffix}}'.
Surely, the same effect is achieved by
directly specifying the
argument `{\textit{dest}\hspace{0.2em}\textit{suffix}}'
in the first form.
However, that requires to set up a different file
for each child. With the alternative form of the command
all these files can have exactly the same content
which simplifies setting them up and maintaining them.

For example, the following file |draft.tex|
with a compilation flag |\version| as described in \secref{sec:flags}
compiles the main document as a draft:
%
\begin{center}
\begin{tabular}{l}
|\def\version{draft}|\\
|\input{childdoc.def}|\\
|\childdocforward{|\textit{main}|}|
\end{tabular}
\end{center}
%
Likewise, the following files |final|\textit{nn}|.tex|
compile the final version of the child document
|child|\textit{nn}|.tex|:
%
\begin{center}
\begin{tabular}{l}
|\def\version{final}|\\
|\input{childdoc.def}|\\
|\childdocforwardprefix{final}{child}|
\end{tabular}
\end{center}
%

Note that when several versions of a main file and/or of each child file
are to be generated, it may be convenient to set up a |Makefile| or
shell script to automatise the process.

%%%%%%%%%%%%%%%%%%%%%%%%%%%%%%%%%%%%%%%%%%%%%%%%%%%%%%%%%%%%%%%%%%%%%%%%%%%%%%%%
\subsection{Command Line Processing}
\label{sec:commandline}

The effect of redirection files can also be achieved by invoking
the \LaTeX{} compiler with a more elaborate command line.
Most conveniently this should be done as part
of a shell script or a |Makefile|.

When using \textsf{childdoc} in the main file, the following
command lines effectively perform a redirection
(note that depending on the shell being used,
backslashes may have to be doubled: `|\|' $\to$ `|\\|'):
%
\begin{center}
|... -jobname "|\textit{target}|" |\\|"|[\textit{flags}]%
|\input{childdoc.def}\childdocforward[|\textit{main}|]{|\textit{dest}|}"|
\end{center}
%
Here \textit{target} is the name of the output file,
\textit{main} is the name of the main file
and \textit{dest} is the name of the main or child file to be processed
(all filenames without extensions).
The optional argument \textit{main} can be omitted
if \textit{main} matches \textit{dest}.
Optionally, compilation \textit{flags} can be defined via |\def| commands.
This command line makes the \TeX{} engine believe
it is compiling the file \textit{target}
whose content is specified as the latter parameter.
The provided code then forwards the processing to
\textit{main} or \textit{dest} as described in \secref{sec:forward}.

%%%%%%%%%%%%%%%%%%%%%%%%%%%%%%%%%%%%%%%%%%%%%%%%%%%%%%%%%%%%%%%%%%%%%%%%%%%%%%%%
\subsection{Include by Input}
\label{sec:input}

Including child documents by |\include| has some restrictions by design.
Most notably, the content of a child document always occupies
its own set of pages; pages cannot be shared between child documents.
Usually, this behaviour makes perfect sense
because each child document contain an essential part of the document.
However, in some situations it may be desirable to compose
a document from a collection of parts
without having mandatory page breaks between then.
For this case, the package
provides a mechanism to include parts
by |\input| which can also be processed individually.
However, by construction this mechanism
requires manual handling of the content to be output.

%%%%%%%%%%%%%%%%%%%%%%%%%%%%%%%%%%%%%%%%
\DescribeMacro{\ifchilddocmanual}
The main file should be prepared as usual, see \secref{sec:include}.
However, the document body must make a distinction
between processing of an individual part and of the main document, e.g.:
%
\begin{center}
\begin{tabular}{l}
|\ifchilddocmanual|\\
|\input{\childdocname}|\\
|\||else|\\
\textit{document body with }|\input{|\textit{part}|}|\\
|\||fi|
\end{tabular}
\end{center}
%
The conditional |\ifchilddocmanual| is true whenever
a part to be included by |\input| is being compiled,
and the name of the part is stored in |\childdocname|.

%%%%%%%%%%%%%%%%%%%%%%%%%%%%%%%%%%%%%%%%
\DescribeMacro{\childdocby}
Each part to be included by |\input| should start with:
%
\begin{center}
\begin{tabular}{l}
|\input{childdoc.def}|\\
|\childdocby{|\textit{main}|}|\\
\end{tabular}
\end{center}
%
The directive |\childdocby| is similar to |\childdocof|
described in \secref{sec:include},
but the subsequent selection of content must be done manually.
To that end, both |\ifchilddoc| and |\ifchilddocmanual|
will be true upon processing of a part,
and the name of the part is stored in |\childdocname|.
Note that |\jobname| will be set to the filename of the current part
so that each part receives an individual |.aux| file
that does not interfere with the |.aux| file(s) of the main document.
This behaviour can be altered by the alternative form
|\childdocby[*]{|\textit{main}|}| (with a non-empty optional argument)
which uses the |.aux| file of the main document
by setting |\jobname| to \textit{main}.

%%%%%%%%%%%%%%%%%%%%%%%%%%%%%%%%%%%%%%%%%%%%%%%%%%%%%%%%%%%%%%%%%%%%%%%%%%%%%%%%
\subsection{Driver Development}
\label{sec:driver}

The \textsf{childdoc} mechanism can also be use for the development
of definition files such as \LaTeX{} styles or classes.
This case differs from the above setup with multiple parts
included by |\include| in that no |\includeonly| should be invoked.
This can be achieved by starting the include file
(before |\ProvidesPackage|) with:
%
\begin{center}
\begin{tabular}{l}
|\input{childdoc.def}|\\
|\childdocforward{|\textit{main}|}|\\
\end{tabular}
\end{center}
%
or alternatively with:
%
\begin{center}
\begin{tabular}{l}
|\input{childdoc.def}|\\
|\childdocby{|\textit{main}|}|\\
\end{tabular}
\end{center}
%
Both forms have slightly different effects as described above.
The main file is prepared as usual, see \secref{sec:include}.

%%%%%%%%%%%%%%%%%%%%%%%%%%%%%%%%%%%%%%%%%%%%%%%%%%%%%%%%%%%%%%%%%%%%%%%%%%%%%%%%
\subsection{Legacy Detection}
\label{sec:detection}

The directive |\childdocmain| in the main file can detect
whether the complete document or merely a child is to be compiled
even without using the directive |\childdocof|.
This method is deprecated because it is less robust
and there is no compelling reason to use it;
it is merely provided for backward compatibility
and it may be removed in future versions.

If the detection mechanism is to be used,
it is mandatory to correctly specify
the filename of the main file as the argument of |\childdocmain|:
%
\begin{center}
\begin{tabular}{l}
|\input{childdoc.def}|\\
|\childdocmain{|\textit{main}|}|\\
\end{tabular}
\end{center}
%
If |\jobname| does not match the argument \textit{main} of |\childdocmain|,
it is assumed that |\jobname| points to the child file to be compiled.
When using |\childdocmain| with the main file specified as argument,
it suffices to start a child file
with just |\input{|\textit{main}|}|
without loading of the package and using |\childdocof|.
If instead all processing is done
with the appropriate \textsf{childdoc} directives,
the argument of \textit{main} of |\childdocmain| can be empty.

An alternative version of the command line processing described
in \secref{sec:commandline} using the detection mechanism reads:
%
\begin{center}
|... -jobname "|\textit{target}|" "|[\textit{flags}]%
[|\def\jobname{|\textit{dest}|}|]|\input{|\textit{main}|}"|
\end{center}

%%%%%%%%%%%%%%%%%%%%%%%%%%%%%%%%%%%%%%%%%%%%%%%%%%%%%%%%%%%%%%%%%%%%%%%%%%%%%%%%
\subsection{Manual Code}
\label{sec:manual}

In case one cannot be certain whether the definitions file |childdoc.def|
is installed on the target \TeX{} distribution
and one prefers not to ship it,
it is conceivable to paste a few relevant commands into the sources.

To that end, drop all statements |\input{childdoc.def}|
and perform the replacements as outlined below.
Instead of |\childdocmain{|\textit{main}|}| add the following code
to the top of the main file:
%
\begin{center}
\begin{tabular}{l}
|\||ifdefined\childdocname\endinput\||fi\newif\ifchilddoc|\\
|\edef\childdocname{\scantokens\expandafter{\jobname\noexpand}}|\\
|\def\childdocmain{|\textit{main}|}\||ifx\childdocmain\childdocname\||else|\\
|\childdoctrue\includeonly{\childdocname}\let\jobname\childdocmain\||fi|\\
\end{tabular}
\end{center}
%
Instead of |\childdocof{|\textit{main}|}| just include the main file
at the top of each child file:
%
\begin{center}
|\input{|\textit{main}|}|
\end{center}
%
A simple redirection |\childdocforward{|\textit{dest}|}| is achieved by:
%
\begin{center}
|\def\jobname{|\textit{dest}|}\input{\jobname}|
\end{center}
%
The redirection with prefix
|\childdocforwardprefix[|\textit{prefix}|]{|\textit{dest}|}|
is accomplished by:
%
\begin{center}
\begin{tabular}{l}
|{\edef\jobname{\scantokens\expandafter{\jobname\noexpand}}|\\
|\def\redirectjob |\textit{prefix}|#1~~~{\gdef\jobname{|\textit{dest}|#1}}|\\
|\expandafter\redirectjob\jobname~~~}\input{\jobname}|
\end{tabular}
\end{center}

In an alternative approach,
child documents can be compiled by a specific command line
without additional code or specific definitions:
%
\begin{center}
|... -jobname "|\textit{target}|" "|[\textit{flags}]%
|\includeonly{|\textit{dest}|}\input{|\textit{main}|}"|
\end{center}
%

%%%%%%%%%%%%%%%%%%%%%%%%%%%%%%%%%%%%%%%%%%%%%%%%%%%%%%%%%%%%%%%%%%%%%%%%%%%%%%%%
%%%%%%%%%%%%%%%%%%%%%%%%%%%%%%%%%%%%%%%%%%%%%%%%%%%%%%%%%%%%%%%%%%%%%%%%%%%%%%%%
\section{Information}

%%%%%%%%%%%%%%%%%%%%%%%%%%%%%%%%%%%%%%%%%%%%%%%%%%%%%%%%%%%%%%%%%%%%%%%%%%%%%%%%
\subsection{Copyright}

Copyright \copyright{} 2017--2018 Niklas Beisert

This work may be distributed and/or modified under the
conditions of the \LaTeX{} Project Public License, either version 1.3
of this license or (at your option) any later version.
The latest version of this license is in
  \url{http://www.latex-project.org/lppl.txt}
and version 1.3 or later is part of all distributions of \LaTeX{}
version 2005/12/01 or later.

This work has the LPPL maintenance status `maintained'.

The Current Maintainer of this work is Niklas Beisert.

This work consists of the files |README.txt|, |childdoc.ins| and |childdoc.dtx|
as well as the derived files |childdoc.def|, |cdocsamp.tex|
with |cdocsch1.tex|, |cdocsch2.tex|, |cdocspt3.tex|, |cdocspt4.tex|,
|cdocsdrf.tex|, |cdocsfn1.tex|, |cdocsfn2.tex|
as well as |childdoc.pdf|.

%%%%%%%%%%%%%%%%%%%%%%%%%%%%%%%%%%%%%%%%%%%%%%%%%%%%%%%%%%%%%%%%%%%%%%%%%%%%%%%%
\subsection{Files and Installation}

The package consists of the files:
%
\begin{center}
\begin{tabular}{ll}
    |README.txt|   & readme file \\
    |childdoc.ins| & installation file \\
    |childdoc.dtx| & source file \\
    |childdoc.def| & definition file \\
    |cdocsamp.tex| & sample main file \\
    |cdocsch1.tex| & sample include file \\
    |cdocsch2.tex| & sample include file \\
    |cdocspt3.tex| & sample part file \\
    |cdocspt4.tex| & sample part file \\
    |cdocsdrf.tex| & sample redirection file \\
    |cdocsfn1.tex| & sample redirection file \\
    |cdocsfn2.tex| & sample redirection file \\
    |childdoc.pdf| & manual
\end{tabular}
\end{center}
%
The distribution consists of the files
|README.txt|, |childdoc.ins| and |childdoc.dtx|.
%
\begin{itemize}
\item
Run (pdf)\LaTeX{} on |childdoc.dtx|
to compile the manual |childdoc.pdf| (this file).
\item
Run \LaTeX{} on |childdoc.ins| to create the definitions file |childdoc.def|
and the sample |cdocsamp.tex| with include files
|cdocsch1.tex|, |cdocsch2.tex|, |cdocspt3.tex|, |cdocspt4.tex|,
|cdocsdrf.tex|, |cdocsfn1.tex|, |cdocsfn2.tex|.
Then copy the file |childdoc.def| to an appropriate directory of your \LaTeX{}
distribution, e.g.\ \textit{texmf-root}|/tex/latex/childdoc|.
\end{itemize}

%%%%%%%%%%%%%%%%%%%%%%%%%%%%%%%%%%%%%%%%%%%%%%%%%%%%%%%%%%%%%%%%%%%%%%%%%%%%%%%%
\subsection{Related CTAN Packages}

There are several other packages which offer a similar functionality:
%
\begin{itemize}
\item
The packages
\href{http://ctan.org/pkg/docmute}{\textsf{docmute}},
\href{http://ctan.org/pkg/includex}{\textsf{includex}} and
\href{http://ctan.org/pkg/standalone}{\textsf{standalone}}
provide commands to include only the document body of
a child file thus allowing both files to be compiled individually.
\item
The packages \href{http://ctan.org/pkg/subdocs}{\textsf{subdocs}}
and \href{http://ctan.org/pkg/subfiles}{\textsf{subfiles}}
provide structures in which the main and child documents can be
encapsulated and allowing them to be compiled individually.
The inclusion mechanism is different from the conventional |\include|.
\item
The package \href{http://ctan.org/pkg/combine}{\textsf{combine}}
is an elaborate solution to combine several documents into one.
\end{itemize}
%
See also the CTAN topic \href{http://ctan.org/topic/subdocs}{\textsf{subdocs}}
for further related packages.
The present package differs from the above solutions in that
a document structure constructed with the conventional |\include| mechanism
just needs two extra commands at the top of every file
such that all constituent files can be compiled individually.

%%%%%%%%%%%%%%%%%%%%%%%%%%%%%%%%%%%%%%%%%%%%%%%%%%%%%%%%%%%%%%%%%%%%%%%%%%%%%%%%
%\subsection{Feature Suggestions}
%
%The following is a list of features which may be useful for future
%versions of this package:
%%
%\begin{itemize}
%\item
%\ldots
%\end{itemize}

%%%%%%%%%%%%%%%%%%%%%%%%%%%%%%%%%%%%%%%%%%%%%%%%%%%%%%%%%%%%%%%%%%%%%%%%%%%%%%%%
\subsection{Revision History}

%%%%%%%%%%%%%%%%%%%%%%%%%%%%%%%%%%%%%%%%
\paragraph{v2.0:} 2018/12/30

\begin{itemize}
\item
immediate forward processing
\item
added |\childdocby| mechanism
\item
manual restructured
\end{itemize}

%%%%%%%%%%%%%%%%%%%%%%%%%%%%%%%%%%%%%%%%
\paragraph{v1.6:} 2018/01/17

\begin{itemize}
\item
application for development of include files
\item
corrections to manual
\end{itemize}

%%%%%%%%%%%%%%%%%%%%%%%%%%%%%%%%%%%%%%%%
\paragraph{v1.5:} 2017/05/21

\begin{itemize}
\item
more complete structuring introduced
\item
|\childdocof| introduced
\item
|\childdoc| renamed to |\childdocmain|
\item
|\childredirect| renamed to |\childdocforward| and |\childdocforwardprefix|
and functionality expanded
\end{itemize}

%%%%%%%%%%%%%%%%%%%%%%%%%%%%%%%%%%%%%%%%
\paragraph{v1.0:} 2017/04/27

\begin{itemize}
\item
manual and install package
\item
first version published on CTAN
\end{itemize}

%%%%%%%%%%%%%%%%%%%%%%%%%%%%%%%%%%%%%%%%
\paragraph{v0.6:} 2017/04/26

\begin{itemize}
\item
redirection mechanism added
\end{itemize}

%%%%%%%%%%%%%%%%%%%%%%%%%%%%%%%%%%%%%%%%
\paragraph{v0.5:} 2017/04/26

\begin{itemize}
\item
functionality in definition file
\end{itemize}


%%%%%%%%%%%%%%%%%%%%%%%%%%%%%%%%%%%%%%%%%%%%%%%%%%%%%%%%%%%%%%%%%%%%%%%%%%%%%%%%
%%%%%%%%%%%%%%%%%%%%%%%%%%%%%%%%%%%%%%%%%%%%%%%%%%%%%%%%%%%%%%%%%%%%%%%%%%%%%%%%
%%%%%%%%%%%%%%%%%%%%%%%%%%%%%%%%%%%%%%%%%%%%%%%%%%%%%%%%%%%%%%%%%%%%%%%%%%%%%%%%
\appendix

\settowidth\MacroIndent{\rmfamily\scriptsize 000\ }

 \DocInput{childdoc.dtx}

\end{document}
%</driver>
% \fi
%
% %%%%%%%%%%%%%%%%%%%%%%%%%%%%%%%%%%%%%%%%%%%%%%%%%%%%%%%%%%%%%%%%%%%%%%%%%%%%%%
% %%%%%%%%%%%%%%%%%%%%%%%%%%%%%%%%%%%%%%%%%%%%%%%%%%%%%%%%%%%%%%%%%%%%%%%%%%%%%%
% \section{Sample}
%\iffalse
%<*samplemain>
%\fi
%
% The following presents a sample document
% with two chapters, two parts, a title page,
% a compile flag as well as three forwarding files to set the flag.
% It consists of eight |.tex| files:
% \begin{center}
% \begin{tabular}{ll}
% |cdocsamp.tex|&main file\\
% |cdocsch1.tex|&include file for chapter 1\\
% |cdocsch2.tex|&include file for chapter 2\\
% |cdocspt3.tex|&include file for part 3\\
% |cdocspt4.tex|&include file for part 4\\
% |cdocsdrf.tex|&forwarding file for main file in draft mode\\
% |cdocsfi1.tex|&forwarding file for final version of chapter 1\\
% |cdocsfi2.tex|&forwarding file for final version of chapter 2\\
% \end{tabular}
% \end{center}
% Each of the eight files can be compiled directly by the \LaTeX{} compiler.
%
% %%%%%%%%%%%%%%%%%%%%%%%%%%%%%%%%%%%%%%
% \paragraph{Main File.}
%
% The main file is called |cdocsamp.tex|.
%
% Load the \textsf{childdoc} definitions and
% declare the filename for the main document:
%    \begin{macrocode}
\input{childdoc.def}
\childdocmain{}
%    \end{macrocode}

% Optional override for |\version| flag:
%    \begin{macrocode}
%%\ifchilddoc\else\providecommand{\version}{draft}\fi
%    \end{macrocode}

% Define the default values for the |\version| flag
% (|final| for the main file and |draft| for childs):
%    \begin{macrocode}
\ifchilddoc
\providecommand{\version}{draft}
\else
\providecommand{\version}{final}
\fi
%    \end{macrocode}

% Load the standard document class:
%    \begin{macrocode}
\documentclass[12pt]{article}
%    \end{macrocode}

% Start the document body:
%    \begin{macrocode}
\begin{document}
%    \end{macrocode}

% Declare a title page.
% Print title, part of document being processed and version flag:
%    \begin{macrocode}
\addtocounter{page}{-1}
\begin{center}
{\LARGE\bfseries{}childdoc example\par}
\vspace{1cm}
\ifchilddoc
\ifchilddocmanual part\else chapter\fi:
`\childdocname' of `\childdocjob'\par
\else
main document: `\childdocjob'\par
\fi
version: \version\par
\end{center}
\newpage
%    \end{macrocode}

% Manually include selected file,
% otherwise process as usual:
%    \begin{macrocode}
\ifchilddocmanual
\section*{part `\childdocname'}
\input{\childdocname}
\else
%    \end{macrocode}

% Include the two chapters:
%    \begin{macrocode}
\include{cdocsch1}
\include{cdocsch2}
%    \end{macrocode}

% Include the two parts unless only chapters should be displayed:
%    \begin{macrocode}
\ifchilddoc\else
\section{part three}
\input{cdocspt3}
\section{part four}
\input{cdocspt4}
\fi
%    \end{macrocode}

% Process as usual until here:
%    \begin{macrocode}
\fi
%    \end{macrocode}

% End of document body:
%    \begin{macrocode}
\end{document}
%    \end{macrocode}
%\iffalse
%</samplemain>
%\fi
%
% %%%%%%%%%%%%%%%%%%%%%%%%%%%%%%%%%%%%%%
% \paragraph{Chapter Include Files.}
%
% The include files are called |cdocsch1.tex| and |cdocsch2.tex|.
%
%\iffalse
%<*samplechap1|samplechap2>
%\fi

% Optional override for |\version| flag:
%    \begin{macrocode}
%%\providecommand{\version}{final}
%    \end{macrocode}

% Include the main document:
%    \begin{macrocode}
\input{childdoc.def}
\childdocof{cdocsamp}
%    \end{macrocode}

%\iffalse
%</samplechap1|samplechap2>
%\fi
%
%\iffalse
%<*samplechap1>
%\fi
% Some text for chapter 1:
%    \begin{macrocode}
\section{one}
some text in chapter one
%    \end{macrocode}

%\iffalse
%</samplechap1>
%\fi
% Some text for chapter 2:
%\iffalse
%<*samplechap2>
%\fi
%    \begin{macrocode}
\section{two}
more text in chapter two
%    \end{macrocode}

%\iffalse
%</samplechap2>
%\fi
%
% %%%%%%%%%%%%%%%%%%%%%%%%%%%%%%%%%%%%%%
% \paragraph{Part Include Files.}
%
% The include files are called |cdocspt3.tex| and |cdocspt4.tex|.
%
%\iffalse
%<*samplepart3|samplepart4>
%\fi

% Optional override for |\version| flag:
%    \begin{macrocode}
%%\providecommand{\version}{final}
%    \end{macrocode}

% Include the main document:
%    \begin{macrocode}
\input{childdoc.def}
\childdocby{cdocsamp}
%    \end{macrocode}

%\iffalse
%</samplepart3|samplepart4>
%\fi
%
%\iffalse
%<*samplepart3>
%\fi
% Some text for part 3:
%    \begin{macrocode}
some text in part three
%    \end{macrocode}

%\iffalse
%</samplepart3>
%\fi
% Some text for part 4:
%\iffalse
%<*samplepart4>
%\fi
%    \begin{macrocode}
more text in part four
%    \end{macrocode}

%\iffalse
%</samplepart4>
%\fi
%
% %%%%%%%%%%%%%%%%%%%%%%%%%%%%%%%%%%%%%%
% \paragraph{Forwarding for a Complete Draft.}
%
% The following forwarding file |cdocsdrf.tex|
% compiles the main document in draft mode:
%\iffalse
%<*sampledraft>
%\fi
%    \begin{macrocode}
\def\version{draft}
\input{childdoc.def}
\childdocforward{cdocsamp}
%    \end{macrocode}

%\iffalse
%</sampledraft>
%\fi
%
% %%%%%%%%%%%%%%%%%%%%%%%%%%%%%%%%%%%%%%
% \paragraph{Forwarding for Final Version of the Chapters.}
%
% The following forwarding files |cdocsfn1.tex| and |cdocsfn2.tex|
% (with identical content)
% compile the final versions of the child documents
% |cdocsch1.tex| and |cdocsch2.tex|, respectively:
%\iffalse
%<*samplefinal>
%\fi
%    \begin{macrocode}
\def\version{final}
\input{childdoc.def}
\childdocforwardprefix[cdocsamp]{cdocsfn}{cdocsch}
%    \end{macrocode}

%\iffalse
%</samplefinal>
%\fi
%
% %%%%%%%%%%%%%%%%%%%%%%%%%%%%%%%%%%%%%%
% \paragraph{Command Line Processing.}
%
% The following three command lines generate the output files
% |cdocscld|, |cdocscl1| and |cdocscl2|
% which should be identical to
% |cdocsdrf|, |cdocsch1| and |cdocsfn2|, respectively:
% \begin{center}
% \begin{tabular}{l}
% |latex -jobname cdocscld \|\\
% |  "\def\version{draft}\input{childdoc.def}\childdocforward{cdocsamp}"|\\
% |latex -jobname cdocscl1 \|\\
% |  "\input{childdoc.def}\childdocforward[cdocsamp]{cdocsch1}"|\\
% |latex -jobname cdocscl2 \|\\
% |  "\def\version{final}\input{childdoc.def}\childdocforward{cdocsch2}"|
% \end{tabular}
% \end{center}
% Note that the trailing backslash on each first line
% merely continues the input to the second line
% (for convenient cut ant paste).
% Furthermore, the command |latex| can be replaced by any
% of its alternative versions such as |pdflatex|.
%
% %%%%%%%%%%%%%%%%%%%%%%%%%%%%%%%%%%%%%%%%%%%%%%%%%%%%%%%%%%%%%%%%%%%%%%%%%%%%%%
% %%%%%%%%%%%%%%%%%%%%%%%%%%%%%%%%%%%%%%%%%%%%%%%%%%%%%%%%%%%%%%%%%%%%%%%%%%%%%%
% \section{Implementation}
%\iffalse
%<*package>
%\fi
%
% This section describes the definitions file |childdoc.def|.

% The definitions cannot be loaded using |\usepackage| or |\RequirePackage|
% which has a mechanism to prevent loading a style file more than once.
% When loading the definitions by means of |\input|
% multiple instances have to be prevented manually:
%\iffalse
%This code needs to be before the `\ProvidesFile' directive
%which is defined at the beginning of this file.
%Therefore it is also placed there and commented out here.
%</package>
%<*discard>
%\fi
%    \begin{macrocode}
\ifdefined\childdocmain\endinput\fi
%    \end{macrocode}
%\iffalse
%</discard>
%<*package>
%\fi
%
% \macro{\ifchilddoc}
% \macro{\ifchilddocmanual}
% The conditional |\ifchilddoc| tells whether a
% child (true) or main (false) document is being compiled.
% The conditional |\ifchilddocmanual| tells whether
% the |\includeonly| mechanism is used (false) or
% the selection of child files must be performed manually (true).
% The definitions initialise to false:
%    \begin{macrocode}
\newif\ifchilddoc
\newif\ifchilddocmanual
%    \end{macrocode}

% \macro{\childdocname}
% \macro{\childdocjob}
% The macro |\childdocname| stores the name of the main document
% to be compiled. The macro |\childdocjob| stores the name of
% the document on which the \LaTeX{} compiler was originally invoked.
% The content of |\jobname| cannot be compared
% to filenames specified in the source due to different catcodes.
% The following code rescans |\jobname|, stores the result
% in |\childdocname| and saves a copy in |\childdocjob|:
%    \begin{macrocode}
\edef\childdocname{\scantokens\expandafter{\jobname\noexpand}}
\let\childdocjob\childdocname
%    \end{macrocode}

% \macro{\childdocdisable}
% The macro |\childdocdisable| prevents the main file
% from being processed more than once.
% At this stage, the main document command |\childdocmain|
% is assumed to be called once again where it should do nothing.
% Any subsequent call to it should prevent
% a secondary processing of the main document
% It overwrites the forwarding commands
% |\childdocof| and |\childdocforward|
% with empty macros to prevent further inclusions of the main document:
%    \begin{macrocode}
\newcommand{\childdocdisable}
{
  \renewcommand{\childdocmain}[1]{\renewcommand{\childdocmain}[1]{\endinput}}
  \renewcommand{\childdocof}[1]{}
  \renewcommand{\childdocby}[2][]{}
  \renewcommand{\childdocforward}[2][]{}
  \renewcommand{\childdocdisable}{}
}
%    \end{macrocode}

% \macro{\childdocmain}
% The macro |\childdocmain| is to be called at the top of the main file
% with nothing or the main filename (without extension) as argument.
% First, it breaks loops.
% If the argument is not empty and does not match |\childdocname|
% (which is set by the first inclusion of |childdoc.def|),
% |\ifchilddoc| is set to true, |\includeonly| is applied to the child file
% and |\jobname| is set to the main file
% (for proper handling of |.aux| files):
%    \begin{macrocode}
\newcommand{\childdocmain}[1]
{
  \childdocdisable\childdocmain{}
  \if?#1?\else
    \begingroup
      \def\childdoctmp{#1}
      \ifx\childdoctmp\childdocname
        \def\childdoctmp{}
      \else
        \def\childdoctmp
        {
          \childdoctrue
          \includeonly{\childdocname}
          \def\childdocjob{#1}
          \def\jobname{#1}
        }
      \fi
      \expandafter
    \endgroup
    \childdoctmp
  \fi
}
%    \end{macrocode}

% \macro{\childdocof}
% The command |\childdocof| redirects
% compilation to the main file |#1|.
%    \begin{macrocode}
\newcommand{\childdocof}[1]
{
  \childdocdisable
  \childdoctrue
  \includeonly{\childdocname}
  \def\jobname{#1}
  \def\childdocjob{#1}
  \input{#1}
}
%    \end{macrocode}

% \macro{\childdocby}
% The command |\childdocby| ....
%    \begin{macrocode}
\newcommand{\childdocby}[2][]
{
  \childdocdisable
  \childdoctrue
  \childdocmanualtrue
  \if?#1?\else
    \def\jobname{#2}
  \fi
  \def\childdocjob{#2}
  \input{#2}
  \endinput
}
%    \end{macrocode}

% \macro{\childdocforward}
% The command |\childdocforward| redirects
% compilation to the main file or
% (if the optional argument is given) a child file.
% Parameters are set as if the main file
% or a child file starting with |\childdocof| was compiled.
% Then compilation is handed over to the main file:
%    \begin{macrocode}
\newcommand{\childdocforward}[2][]
{
  \begingroup
    \if?#1?
      \def\childdoctmp
      {
        \def\childdocname{#2}
        \def\childdocjob{#2}
        \def\jobname{#2}
        \input{#2}
        \endinput
      }
    \else
      \def\childdoctmp
      {
        \childdocdisable
        \def\childdocname{#2}
        \childdoctrue
        \includeonly{#2}
        \def\childdocjob{#1}
        \def\jobname{#1}
        \input{#1}
        \endinput
      }
    \fi
    \expandafter
  \endgroup
  \childdoctmp
}
%    \end{macrocode}

% \macro{\childdocforwardprefix}
% The command |\childdocforwardprefix| redirects
% compilation to the main or a child file by means of a pattern.
% The prefix |#1| in the current filename is replaced by |#2|
% and the suffix of the current filename is kept
% (it is assumed that the filename does not contain the substring `|~~~|'
% which is used as a delimiter).
% Compilation is handed over to the new file by |\childdocforward|:
%    \begin{macrocode}
\newcommand{\childdocforwardprefix}[3][]
{
  \begingroup
    \def\childdocextract #2##1~~~{\def\childdoctmp{\childdocforward[#1]{#3##1}}}
    \expandafter\childdocextract\childdocname~~~
    \expandafter
  \endgroup
  \childdoctmp
}
%    \end{macrocode}

% \macro{\childdoc}
% The deprecated macro |\childdoc| is a legacy version of |\childdocmain|:
%    \begin{macrocode}
\newcommand{\childdoc}{\childdocmain}
%    \end{macrocode}

% \macro{\childdocredirect}
% The deprecated macro |\childdocredirect| is a legacy version
% of |\childdocforward| and |\childdocforwardprefix|:
%    \begin{macrocode}
\newcommand{\childdocredirect}[2][]
{
  \begingroup
    \if?#1?
      \def\childdoctmp{\childdocforward{#2}}
    \else
      \def\childdoctmp{\childdocforwardprefix{#1}{#2}}
    \fi
    \expandafter
  \endgroup
  \childdoctmp
}
%    \end{macrocode}

%\iffalse
%</package>
%\fi
%
\endinput
|\\
|\childdocforward{|\textit{main}|}|
\end{tabular}
\end{center}
%
Likewise, the following files |final|\textit{nn}|.tex|
compile the final version of the child document
|child|\textit{nn}|.tex|:
%
\begin{center}
\begin{tabular}{l}
|\def\version{final}|\\
|% \iffalse
%
% childdoc.dtx Copyright (C) 2017-2018 Niklas Beisert
%
% This work may be distributed and/or modified under the
% conditions of the LaTeX Project Public License, either version 1.3
% of this license or (at your option) any later version.
% The latest version of this license is in
%   http://www.latex-project.org/lppl.txt
% and version 1.3 or later is part of all distributions of LaTeX
% version 2005/12/01 or later.
%
% This work has the LPPL maintenance status `maintained'.
%
% The Current Maintainer of this work is Niklas Beisert.
%
% This work consists of the files childdoc.dtx and childdoc.ins
% and the derived files childdoc.def and cdocsamp.tex with
% cdocsch1.tex, cdocsch2.tex, cdocsdrf.tex, cdocsfn1.tex, cdocsfn2.tex.
%
%<package>\ifdefined\childdocmain\endinput\fi
%<package>\ProvidesFile{childdoc.def}[2018/12/30 v2.0 child document driver]
%<samplemain>\ProvidesFile{cdocsamp.tex}[2018/12/30 v2.0 sample for childdoc]
%<*driver>
%\ProvidesFile{childdoc.drv}[2018/12/30 v2.0 childdoc reference manual file]
\PassOptionsToClass{10pt,a4paper}{article}
\documentclass{ltxdoc}

\usepackage[margin=35mm]{geometry}
\usepackage{hyperref}
\usepackage{hyperxmp}
\usepackage[usenames]{color}

\hypersetup{colorlinks=true}
\hypersetup{pdfstartview=FitH}
\hypersetup{pdfpagemode=UseNone}
\hypersetup{pdfsource={}}
\hypersetup{pdflang={en-UK}}
\hypersetup{pdfcopyright={Copyright 2017-2018 Niklas Beisert.
  This work may be distributed and/or modified under the
  conditions of the LaTeX Project Public License, either version 1.3
  of this license or (at your option) any later version.}}
\hypersetup{pdflicenseurl={http://www.latex-project.org/lppl.txt}}
\hypersetup{pdfcontactaddress={ETH Zurich, ITP, HIT K,
  Wolfgang-Pauli-Strasse 27}}
\hypersetup{pdfcontactpostcode={8093}}
\hypersetup{pdfcontactcity={Zurich}}
\hypersetup{pdfcontactcountry={Switzerland}}
\hypersetup{pdfcontactemail={nbeisert@itp.phys.ethz.ch}}
\hypersetup{pdfcontacturl={http://people.phys.ethz.ch/\xmptilde nbeisert/}}

\newcommand{\secref}[1]{\hyperref[#1]{section \ref*{#1}}}

\parskip1ex
\parindent0pt
\let\olditemize\itemize
\def\itemize{\olditemize\parskip0pt}

\begin{document}

\title{The \textsf{childdoc} Package}
\hypersetup{pdftitle={The childdoc Package}}
\author{Niklas Beisert\\[2ex]
  Institut f\"ur Theoretische Physik\\
  Eidgen\"ossische Technische Hochschule Z\"urich\\
  Wolfgang-Pauli-Strasse 27, 8093 Z\"urich, Switzerland\\[1ex]
  \href{mailto:nbeisert@itp.phys.ethz.ch}
  {\texttt{nbeisert@itp.phys.ethz.ch}}}
\hypersetup{pdfauthor={Niklas Beisert}}
\hypersetup{pdfsubject={Manual for the LaTeX2e Package childdoc}}
\date{30 December 2018, \textsf{v2.0}}
\maketitle

\begin{abstract}\noindent
\textsf{childdoc} is a \LaTeXe{} package
that enables the direct compilation
of document sections included by |\include|
to individual files.
\end{abstract}

\begingroup
\parskip0ex
\tableofcontents
\endgroup

%%%%%%%%%%%%%%%%%%%%%%%%%%%%%%%%%%%%%%%%%%%%%%%%%%%%%%%%%%%%%%%%%%%%%%%%%%%%%%%%
%%%%%%%%%%%%%%%%%%%%%%%%%%%%%%%%%%%%%%%%%%%%%%%%%%%%%%%%%%%%%%%%%%%%%%%%%%%%%%%%
\section{Introduction}

\LaTeX{} provides a mechanism to structure a large document (such as a book)
into a main file and several child files (containing the chapters)
using the |\include| command.
This mechanism is beneficial for documents
which span hundreds of pages in order to
make the source file(s) more manageable.
Moreover, compilation can be restricted to
selected child files by means of the |\includeonly| command.
The latter feature can be used to reduce the compilation time while editing
(this was significantly more useful in the earlier days of \LaTeX{})
or to generate a smaller document which is easier to navigate.
Another application of |\includeonly| is to generate
documents consisting of selected parts of the complete document.

However, there are a few drawbacks of the plain |\include| mechanism:
\begin{itemize}
\item
The child files cannot be compiled on their own,
they can only be compiled via the main file.
A naive editing environment
(such as a text editor with an option
to have the current file processed by \LaTeX)
may require one to switch to the main file before compiling;
attempting to compile the child file produces errors.
\item
The main file must be modified (each time)
to adjust the |\includeonly| command
to the present needs. This easily leaves the main file in a messy state.
\item
The generated document will always carry the filename
of the main document. This is inconvenient if
several child files are to be compiled and
to be kept for distribution.
\end{itemize}

The present package provides a simple interface
to make child files individually compilable by \LaTeX{}.
Compiling a child file then has the same effect as compiling
the main file with an |\includeonly| command
to select the appropriate child.
Moreover the generated document will carry the name of the child
rather than the main file.
This resolves all three above issues.

This feature is meant to make the editing of books,
thesis documents and lecture notes somewhat more convenient.
However, the package can also be used efficiently for
composing a series of documents (such as exercise sheets)
which are typically distributed individually.
It then assists the author in generating the individual documents
(potentially in different versions)
as well as a document containing the collected series.
Another application is in developing style files
or other kinds of included material
where compilation of the style file could redirect
to a sample or test file.

%%%%%%%%%%%%%%%%%%%%%%%%%%%%%%%%%%%%%%%%%%%%%%%%%%%%%%%%%%%%%%%%%%%%%%%%%%%%%%%%
%%%%%%%%%%%%%%%%%%%%%%%%%%%%%%%%%%%%%%%%%%%%%%%%%%%%%%%%%%%%%%%%%%%%%%%%%%%%%%%%
\section{Usage}

First of all, the package \textsf{childdoc} is \emph{not} a standard
\LaTeXe{} |.sty| style file! Therefore it needs to be invoked in
a non-standard way.

%%%%%%%%%%%%%%%%%%%%%%%%%%%%%%%%%%%%%%%%%%%%%%%%%%%%%%%%%%%%%%%%%%%%%%%%%%%%%%%%
\subsection{Included Files}
\label{sec:include}

%%%%%%%%%%%%%%%%%%%%%%%%%%%%%%%%%%%%%%%%
\DescribeMacro{\childdocmain}
To use the package, add the commands
\begin{center}
\begin{tabular}{l}
|\input{childdoc.def}|\\
|\childdocmain{}|\\
\end{tabular}
\end{center}
at the very top of the main \LaTeX{} file,
in particular \emph{before} the |\documentclass| statement!
The argument of |\childdocmain| should be left empty
(but it must be present).

%%%%%%%%%%%%%%%%%%%%%%%%%%%%%%%%%%%%%%%%
\DescribeMacro{\childdocof}
Furthermore, add the commands
\begin{center}
\begin{tabular}{l}
|\input{childdoc.def}|\\
|\childdocof{|\textit{main}|}|\\
\end{tabular}
\end{center}
at the top of every child file \textit{child}
which is included by |\include{|\textit{child}|}|
from within the main file
(or at least for those files to be compiled individually).
The argument \textit{main} must be the filename of the main file.

There are a couple of
considerations in setting up the main and child documents:

%%%%%%%%%%%%%%%%%%%%%%%%%%%%%%%%%%%%%%%%
\paragraph{Restrictions.}

Please note the following restrictions:
\begin{itemize}
\item
|\childdocmain| must be called with one argument \textit{main}
to ensure compatibility with earlier version of the package.
It must either be empty (|\childdocmain{}|)
or precisely match the filename of the main file in which it is specified.
See \secref{sec:detection} for further information.
\item
The filename \textit{main} must be specified without the |.tex| extension.
\item
The filename \textit{main} is case sensitive
(even in case-insensitive file systems)
due to internal string comparison.
\item
The argument \textit{main} should be fully expanded, it cannot be a macro.
\item
Subdirectories and special characters should be avoided in filenames.
\item
The command |\childdocmain{|\textit{main}|}| must be followed by a whitespace.
It should not be followed immediately by another command
or by a comment mark `|%|'.
This is because the \TeX{} parser reads the token immediately following
the argument of |\childdocmain| and puts it
at the beginning of every child section;
however, a white\-space is ignored.
\end{itemize}

%%%%%%%%%%%%%%%%%%%%%%%%%%%%%%%%%%%%%%%%
\paragraph{Content of Main File.}

It is advisable to place all content in the child files included by |\include|.
Any output contained in the main file will appear in all child documents
unless suppressed manually;
it cannot be suppressed automatically by the |\includeonly| directive
and thus should normally be avoided.
A method to include some content in the main file
by means of conditional processing is described in \secref{sec:conditional}.

%%%%%%%%%%%%%%%%%%%%%%%%%%%%%%%%%%%%%%%%
\paragraph{Page Numbering.}

When only a part of the document is compiled,
the appropriate numbering of pages
(as well as other status parameters)
is determined from the |.aux| files.
The latter contain information from previous passes.
However this information needs to propagate through
all intermediate child documents.
Therefore the page numbering in child documents may well
be inconsistent until the complete document is compiled at least once.

A useful (if unconventional) way to always ensure a consistent
page numbering is to restart the numbering in each child document
and denote the pages by `\textit{child}|.|\textit{page}'
where \textit{child} represents the chapter/section number of the child file.
This can be achieved by the command
|\numberwithin{page}{|\textit{child}|}|
of the \textsf{amsmath} package
where \textit{child} can be |chapter| or |section|
depending on the chosen structuring.
Alternatively, one can modify the macro |\thepage| appropriately
and reset the counter |page| at the start of each child file.

%%%%%%%%%%%%%%%%%%%%%%%%%%%%%%%%%%%%%%%%%%%%%%%%%%%%%%%%%%%%%%%%%%%%%%%%%%%%%%%%
\subsection{Conditional Processing}
\label{sec:conditional}

The package provides a mechanism to compile different versions
of a document. To customise the versions further some conditional processing
can come in handy to distinguish which version is being compiled.
The package provides two macros to describe the compilation context:

%%%%%%%%%%%%%%%%%%%%%%%%%%%%%%%%%%%%%%%%
\DescribeMacro{\ifchilddoc}
The conditional |\ifchilddoc| distinguishes between the compilation of
child documents and the main document:
%
\begin{center}
|\ifchilddoc |\textit{child-code}| |[|\||else |\textit{main-code}]| \||fi|
\end{center}

%%%%%%%%%%%%%%%%%%%%%%%%%%%%%%%%%%%%%%%%
\DescribeMacro{\childdocname}
\DescribeMacro{\childdocjob}
The macro |\childdocname| contains the filename (without extension)
of the main or child file being processed.
Note that |\childdocjob| will always contain the name of the main file.

%%%%%%%%%%%%%%%%%%%%%%%%%%%%%%%%%%%%%%%%
\paragraph{Title Page.}

Conditional processing can be used to include a title or banner page
in the main document when proper precautions are taken.
Importantly, the code in the main file should ensure that the page counter
(as well as other status parameters which are stored in the |.aux| files)
takes the same value after the conditional processing.
Otherwise the page numbers may take divergent values
depending on which part is compiled.

For example, a title page could be declared by:
%
\begin{center}
\begin{tabular}{l}
|\ifchilddoc\||else|\\
|\addtocounter{page}{-1}|\\
\textit{code for title page}\\
|\newpage|\\
|\||fi|
\end{tabular}
\end{center}
%
A banner page for the child documents can be generated by:
%
\begin{center}
\begin{tabular}{l}
|\ifchilddoc|\\
|\addtocounter{page}{-1}|\\
\textit{code for banner page}\\
|\newpage|\\
|\||fi|
\end{tabular}
\end{center}
%
Here one could write a message such as:
\begin{center}
|This is the part \childdocname{} of \childdocjob{}.|
\end{center}

%%%%%%%%%%%%%%%%%%%%%%%%%%%%%%%%%%%%%%%%%%%%%%%%%%%%%%%%%%%%%%%%%%%%%%%%%%%%%%%%
\subsection{Flags}
\label{sec:flags}

The package makes it easy to generate different versions
of the main or child documents.
To this end compilation flags can be defined
and assigned different default values.
They will be particularly useful in conjunction
with the forwarding mechanism described in \secref{sec:forward}.

For example, it may be useful to have a flag |\version|
which can be set to |draft| or |final|.
The document source will contain some conditional code
depending on the value of |\version|.
Suppose further, the flag should default to |final| for the main file
and to |draft| for child files
which is a natural assignment for editing the document.
This is achieved by placing the following code
in the preamble of the main document
(below the |\childdocmain| directive):
%
\begin{center}
\begin{tabular}{l}
|\ifchilddoc|\\
|\providecommand{\version}{draft}|\\
|\||else|\\
|\providecommand{\version}{final}|\\
|\||fi|
\end{tabular}
\end{center}
%
The definition by |\providecommand| makes sure
that previous definitions are not overwritten.
Further statements |\providecommand{\version}{...}|
can thus be added before the above code to override it.

For the main file, one might add a line
(between |\childdocmain| and the above block)
%
\begin{center}
|%\ifchilddoc\||else\providecommand{\version}{draft}\||fi|
\end{center}
%
which can be uncommented to produce a draft version.
Likewise one can add a line to the very top of a child file
(above the |\childdocof{|\textit{main}|}| directive)
%
\begin{center}
|%\providecommand{\version}{final}|
\end{center}
%
which can be uncommented to produce the final version of this child document.

%%%%%%%%%%%%%%%%%%%%%%%%%%%%%%%%%%%%%%%%%%%%%%%%%%%%%%%%%%%%%%%%%%%%%%%%%%%%%%%%
\subsection{Forwarding}
\label{sec:forward}

Different versions of the main or child documents
using compilation flags as described in \secref{sec:flags}
can be (permanently) stored in different files
for convenient compilation, viewing and distribution.
To this end, the package defines a command
to pass on compilation to a different file:

%%%%%%%%%%%%%%%%%%%%%%%%%%%%%%%%%%%%%%%%
\DescribeMacro{\childdocforward}
The command |\childdocforward| redirects processing to
another source file:
%
\begin{center}
\begin{tabular}{l}
|\input{childdoc.def}|\\
|\childdocforward[|\textit{main}|]{|\textit{dest}|}|\\
\end{tabular}
\end{center}
%
The argument \textit{dest} is the destination file
(without extension).
It should be the main file or one of the child files.
Note that further \textsf{childdoc} directives
such as |\childdocof| and |\childdocforward|
in the indicated file will be processed in this form.
The optional argument \textit{main}
passes on directly to the main file \textit{main}
while pretending to compile the child \textit{dest}.
This form behaves as if \textit{dest}
issues |\childdocof{|\textit{main}|}| right away,
and no further \textsf{childdoc} directives will be processed.

%%%%%%%%%%%%%%%%%%%%%%%%%%%%%%%%%%%%%%%%
\DescribeMacro{\...prefix}
In the alternative form |\childdocforwardprefix|,
%
\begin{center}
\begin{tabular}{l}
|\input{childdoc.def}|\\
|\childdocforwardprefix[|\textit{main}|]{|\textit{prefix}|}{|\textit{dest}|}|
\end{tabular}
\end{center}
%
the destination file is determined by a pattern
depending on the current file:
To make this work, the current file must be called
`{\textit{prefix}\hspace{0.2em}\textit{suffix}}'
with \textit{prefix} matching precisely the argument.
Processing is then passed on to the file
`{\textit{dest}\hspace{0.2em}\textit{suffix}}'.
Surely, the same effect is achieved by
directly specifying the
argument `{\textit{dest}\hspace{0.2em}\textit{suffix}}'
in the first form.
However, that requires to set up a different file
for each child. With the alternative form of the command
all these files can have exactly the same content
which simplifies setting them up and maintaining them.

For example, the following file |draft.tex|
with a compilation flag |\version| as described in \secref{sec:flags}
compiles the main document as a draft:
%
\begin{center}
\begin{tabular}{l}
|\def\version{draft}|\\
|\input{childdoc.def}|\\
|\childdocforward{|\textit{main}|}|
\end{tabular}
\end{center}
%
Likewise, the following files |final|\textit{nn}|.tex|
compile the final version of the child document
|child|\textit{nn}|.tex|:
%
\begin{center}
\begin{tabular}{l}
|\def\version{final}|\\
|\input{childdoc.def}|\\
|\childdocforwardprefix{final}{child}|
\end{tabular}
\end{center}
%

Note that when several versions of a main file and/or of each child file
are to be generated, it may be convenient to set up a |Makefile| or
shell script to automatise the process.

%%%%%%%%%%%%%%%%%%%%%%%%%%%%%%%%%%%%%%%%%%%%%%%%%%%%%%%%%%%%%%%%%%%%%%%%%%%%%%%%
\subsection{Command Line Processing}
\label{sec:commandline}

The effect of redirection files can also be achieved by invoking
the \LaTeX{} compiler with a more elaborate command line.
Most conveniently this should be done as part
of a shell script or a |Makefile|.

When using \textsf{childdoc} in the main file, the following
command lines effectively perform a redirection
(note that depending on the shell being used,
backslashes may have to be doubled: `|\|' $\to$ `|\\|'):
%
\begin{center}
|... -jobname "|\textit{target}|" |\\|"|[\textit{flags}]%
|\input{childdoc.def}\childdocforward[|\textit{main}|]{|\textit{dest}|}"|
\end{center}
%
Here \textit{target} is the name of the output file,
\textit{main} is the name of the main file
and \textit{dest} is the name of the main or child file to be processed
(all filenames without extensions).
The optional argument \textit{main} can be omitted
if \textit{main} matches \textit{dest}.
Optionally, compilation \textit{flags} can be defined via |\def| commands.
This command line makes the \TeX{} engine believe
it is compiling the file \textit{target}
whose content is specified as the latter parameter.
The provided code then forwards the processing to
\textit{main} or \textit{dest} as described in \secref{sec:forward}.

%%%%%%%%%%%%%%%%%%%%%%%%%%%%%%%%%%%%%%%%%%%%%%%%%%%%%%%%%%%%%%%%%%%%%%%%%%%%%%%%
\subsection{Include by Input}
\label{sec:input}

Including child documents by |\include| has some restrictions by design.
Most notably, the content of a child document always occupies
its own set of pages; pages cannot be shared between child documents.
Usually, this behaviour makes perfect sense
because each child document contain an essential part of the document.
However, in some situations it may be desirable to compose
a document from a collection of parts
without having mandatory page breaks between then.
For this case, the package
provides a mechanism to include parts
by |\input| which can also be processed individually.
However, by construction this mechanism
requires manual handling of the content to be output.

%%%%%%%%%%%%%%%%%%%%%%%%%%%%%%%%%%%%%%%%
\DescribeMacro{\ifchilddocmanual}
The main file should be prepared as usual, see \secref{sec:include}.
However, the document body must make a distinction
between processing of an individual part and of the main document, e.g.:
%
\begin{center}
\begin{tabular}{l}
|\ifchilddocmanual|\\
|\input{\childdocname}|\\
|\||else|\\
\textit{document body with }|\input{|\textit{part}|}|\\
|\||fi|
\end{tabular}
\end{center}
%
The conditional |\ifchilddocmanual| is true whenever
a part to be included by |\input| is being compiled,
and the name of the part is stored in |\childdocname|.

%%%%%%%%%%%%%%%%%%%%%%%%%%%%%%%%%%%%%%%%
\DescribeMacro{\childdocby}
Each part to be included by |\input| should start with:
%
\begin{center}
\begin{tabular}{l}
|\input{childdoc.def}|\\
|\childdocby{|\textit{main}|}|\\
\end{tabular}
\end{center}
%
The directive |\childdocby| is similar to |\childdocof|
described in \secref{sec:include},
but the subsequent selection of content must be done manually.
To that end, both |\ifchilddoc| and |\ifchilddocmanual|
will be true upon processing of a part,
and the name of the part is stored in |\childdocname|.
Note that |\jobname| will be set to the filename of the current part
so that each part receives an individual |.aux| file
that does not interfere with the |.aux| file(s) of the main document.
This behaviour can be altered by the alternative form
|\childdocby[*]{|\textit{main}|}| (with a non-empty optional argument)
which uses the |.aux| file of the main document
by setting |\jobname| to \textit{main}.

%%%%%%%%%%%%%%%%%%%%%%%%%%%%%%%%%%%%%%%%%%%%%%%%%%%%%%%%%%%%%%%%%%%%%%%%%%%%%%%%
\subsection{Driver Development}
\label{sec:driver}

The \textsf{childdoc} mechanism can also be use for the development
of definition files such as \LaTeX{} styles or classes.
This case differs from the above setup with multiple parts
included by |\include| in that no |\includeonly| should be invoked.
This can be achieved by starting the include file
(before |\ProvidesPackage|) with:
%
\begin{center}
\begin{tabular}{l}
|\input{childdoc.def}|\\
|\childdocforward{|\textit{main}|}|\\
\end{tabular}
\end{center}
%
or alternatively with:
%
\begin{center}
\begin{tabular}{l}
|\input{childdoc.def}|\\
|\childdocby{|\textit{main}|}|\\
\end{tabular}
\end{center}
%
Both forms have slightly different effects as described above.
The main file is prepared as usual, see \secref{sec:include}.

%%%%%%%%%%%%%%%%%%%%%%%%%%%%%%%%%%%%%%%%%%%%%%%%%%%%%%%%%%%%%%%%%%%%%%%%%%%%%%%%
\subsection{Legacy Detection}
\label{sec:detection}

The directive |\childdocmain| in the main file can detect
whether the complete document or merely a child is to be compiled
even without using the directive |\childdocof|.
This method is deprecated because it is less robust
and there is no compelling reason to use it;
it is merely provided for backward compatibility
and it may be removed in future versions.

If the detection mechanism is to be used,
it is mandatory to correctly specify
the filename of the main file as the argument of |\childdocmain|:
%
\begin{center}
\begin{tabular}{l}
|\input{childdoc.def}|\\
|\childdocmain{|\textit{main}|}|\\
\end{tabular}
\end{center}
%
If |\jobname| does not match the argument \textit{main} of |\childdocmain|,
it is assumed that |\jobname| points to the child file to be compiled.
When using |\childdocmain| with the main file specified as argument,
it suffices to start a child file
with just |\input{|\textit{main}|}|
without loading of the package and using |\childdocof|.
If instead all processing is done
with the appropriate \textsf{childdoc} directives,
the argument of \textit{main} of |\childdocmain| can be empty.

An alternative version of the command line processing described
in \secref{sec:commandline} using the detection mechanism reads:
%
\begin{center}
|... -jobname "|\textit{target}|" "|[\textit{flags}]%
[|\def\jobname{|\textit{dest}|}|]|\input{|\textit{main}|}"|
\end{center}

%%%%%%%%%%%%%%%%%%%%%%%%%%%%%%%%%%%%%%%%%%%%%%%%%%%%%%%%%%%%%%%%%%%%%%%%%%%%%%%%
\subsection{Manual Code}
\label{sec:manual}

In case one cannot be certain whether the definitions file |childdoc.def|
is installed on the target \TeX{} distribution
and one prefers not to ship it,
it is conceivable to paste a few relevant commands into the sources.

To that end, drop all statements |\input{childdoc.def}|
and perform the replacements as outlined below.
Instead of |\childdocmain{|\textit{main}|}| add the following code
to the top of the main file:
%
\begin{center}
\begin{tabular}{l}
|\||ifdefined\childdocname\endinput\||fi\newif\ifchilddoc|\\
|\edef\childdocname{\scantokens\expandafter{\jobname\noexpand}}|\\
|\def\childdocmain{|\textit{main}|}\||ifx\childdocmain\childdocname\||else|\\
|\childdoctrue\includeonly{\childdocname}\let\jobname\childdocmain\||fi|\\
\end{tabular}
\end{center}
%
Instead of |\childdocof{|\textit{main}|}| just include the main file
at the top of each child file:
%
\begin{center}
|\input{|\textit{main}|}|
\end{center}
%
A simple redirection |\childdocforward{|\textit{dest}|}| is achieved by:
%
\begin{center}
|\def\jobname{|\textit{dest}|}\input{\jobname}|
\end{center}
%
The redirection with prefix
|\childdocforwardprefix[|\textit{prefix}|]{|\textit{dest}|}|
is accomplished by:
%
\begin{center}
\begin{tabular}{l}
|{\edef\jobname{\scantokens\expandafter{\jobname\noexpand}}|\\
|\def\redirectjob |\textit{prefix}|#1~~~{\gdef\jobname{|\textit{dest}|#1}}|\\
|\expandafter\redirectjob\jobname~~~}\input{\jobname}|
\end{tabular}
\end{center}

In an alternative approach,
child documents can be compiled by a specific command line
without additional code or specific definitions:
%
\begin{center}
|... -jobname "|\textit{target}|" "|[\textit{flags}]%
|\includeonly{|\textit{dest}|}\input{|\textit{main}|}"|
\end{center}
%

%%%%%%%%%%%%%%%%%%%%%%%%%%%%%%%%%%%%%%%%%%%%%%%%%%%%%%%%%%%%%%%%%%%%%%%%%%%%%%%%
%%%%%%%%%%%%%%%%%%%%%%%%%%%%%%%%%%%%%%%%%%%%%%%%%%%%%%%%%%%%%%%%%%%%%%%%%%%%%%%%
\section{Information}

%%%%%%%%%%%%%%%%%%%%%%%%%%%%%%%%%%%%%%%%%%%%%%%%%%%%%%%%%%%%%%%%%%%%%%%%%%%%%%%%
\subsection{Copyright}

Copyright \copyright{} 2017--2018 Niklas Beisert

This work may be distributed and/or modified under the
conditions of the \LaTeX{} Project Public License, either version 1.3
of this license or (at your option) any later version.
The latest version of this license is in
  \url{http://www.latex-project.org/lppl.txt}
and version 1.3 or later is part of all distributions of \LaTeX{}
version 2005/12/01 or later.

This work has the LPPL maintenance status `maintained'.

The Current Maintainer of this work is Niklas Beisert.

This work consists of the files |README.txt|, |childdoc.ins| and |childdoc.dtx|
as well as the derived files |childdoc.def|, |cdocsamp.tex|
with |cdocsch1.tex|, |cdocsch2.tex|, |cdocspt3.tex|, |cdocspt4.tex|,
|cdocsdrf.tex|, |cdocsfn1.tex|, |cdocsfn2.tex|
as well as |childdoc.pdf|.

%%%%%%%%%%%%%%%%%%%%%%%%%%%%%%%%%%%%%%%%%%%%%%%%%%%%%%%%%%%%%%%%%%%%%%%%%%%%%%%%
\subsection{Files and Installation}

The package consists of the files:
%
\begin{center}
\begin{tabular}{ll}
    |README.txt|   & readme file \\
    |childdoc.ins| & installation file \\
    |childdoc.dtx| & source file \\
    |childdoc.def| & definition file \\
    |cdocsamp.tex| & sample main file \\
    |cdocsch1.tex| & sample include file \\
    |cdocsch2.tex| & sample include file \\
    |cdocspt3.tex| & sample part file \\
    |cdocspt4.tex| & sample part file \\
    |cdocsdrf.tex| & sample redirection file \\
    |cdocsfn1.tex| & sample redirection file \\
    |cdocsfn2.tex| & sample redirection file \\
    |childdoc.pdf| & manual
\end{tabular}
\end{center}
%
The distribution consists of the files
|README.txt|, |childdoc.ins| and |childdoc.dtx|.
%
\begin{itemize}
\item
Run (pdf)\LaTeX{} on |childdoc.dtx|
to compile the manual |childdoc.pdf| (this file).
\item
Run \LaTeX{} on |childdoc.ins| to create the definitions file |childdoc.def|
and the sample |cdocsamp.tex| with include files
|cdocsch1.tex|, |cdocsch2.tex|, |cdocspt3.tex|, |cdocspt4.tex|,
|cdocsdrf.tex|, |cdocsfn1.tex|, |cdocsfn2.tex|.
Then copy the file |childdoc.def| to an appropriate directory of your \LaTeX{}
distribution, e.g.\ \textit{texmf-root}|/tex/latex/childdoc|.
\end{itemize}

%%%%%%%%%%%%%%%%%%%%%%%%%%%%%%%%%%%%%%%%%%%%%%%%%%%%%%%%%%%%%%%%%%%%%%%%%%%%%%%%
\subsection{Related CTAN Packages}

There are several other packages which offer a similar functionality:
%
\begin{itemize}
\item
The packages
\href{http://ctan.org/pkg/docmute}{\textsf{docmute}},
\href{http://ctan.org/pkg/includex}{\textsf{includex}} and
\href{http://ctan.org/pkg/standalone}{\textsf{standalone}}
provide commands to include only the document body of
a child file thus allowing both files to be compiled individually.
\item
The packages \href{http://ctan.org/pkg/subdocs}{\textsf{subdocs}}
and \href{http://ctan.org/pkg/subfiles}{\textsf{subfiles}}
provide structures in which the main and child documents can be
encapsulated and allowing them to be compiled individually.
The inclusion mechanism is different from the conventional |\include|.
\item
The package \href{http://ctan.org/pkg/combine}{\textsf{combine}}
is an elaborate solution to combine several documents into one.
\end{itemize}
%
See also the CTAN topic \href{http://ctan.org/topic/subdocs}{\textsf{subdocs}}
for further related packages.
The present package differs from the above solutions in that
a document structure constructed with the conventional |\include| mechanism
just needs two extra commands at the top of every file
such that all constituent files can be compiled individually.

%%%%%%%%%%%%%%%%%%%%%%%%%%%%%%%%%%%%%%%%%%%%%%%%%%%%%%%%%%%%%%%%%%%%%%%%%%%%%%%%
%\subsection{Feature Suggestions}
%
%The following is a list of features which may be useful for future
%versions of this package:
%%
%\begin{itemize}
%\item
%\ldots
%\end{itemize}

%%%%%%%%%%%%%%%%%%%%%%%%%%%%%%%%%%%%%%%%%%%%%%%%%%%%%%%%%%%%%%%%%%%%%%%%%%%%%%%%
\subsection{Revision History}

%%%%%%%%%%%%%%%%%%%%%%%%%%%%%%%%%%%%%%%%
\paragraph{v2.0:} 2018/12/30

\begin{itemize}
\item
immediate forward processing
\item
added |\childdocby| mechanism
\item
manual restructured
\end{itemize}

%%%%%%%%%%%%%%%%%%%%%%%%%%%%%%%%%%%%%%%%
\paragraph{v1.6:} 2018/01/17

\begin{itemize}
\item
application for development of include files
\item
corrections to manual
\end{itemize}

%%%%%%%%%%%%%%%%%%%%%%%%%%%%%%%%%%%%%%%%
\paragraph{v1.5:} 2017/05/21

\begin{itemize}
\item
more complete structuring introduced
\item
|\childdocof| introduced
\item
|\childdoc| renamed to |\childdocmain|
\item
|\childredirect| renamed to |\childdocforward| and |\childdocforwardprefix|
and functionality expanded
\end{itemize}

%%%%%%%%%%%%%%%%%%%%%%%%%%%%%%%%%%%%%%%%
\paragraph{v1.0:} 2017/04/27

\begin{itemize}
\item
manual and install package
\item
first version published on CTAN
\end{itemize}

%%%%%%%%%%%%%%%%%%%%%%%%%%%%%%%%%%%%%%%%
\paragraph{v0.6:} 2017/04/26

\begin{itemize}
\item
redirection mechanism added
\end{itemize}

%%%%%%%%%%%%%%%%%%%%%%%%%%%%%%%%%%%%%%%%
\paragraph{v0.5:} 2017/04/26

\begin{itemize}
\item
functionality in definition file
\end{itemize}


%%%%%%%%%%%%%%%%%%%%%%%%%%%%%%%%%%%%%%%%%%%%%%%%%%%%%%%%%%%%%%%%%%%%%%%%%%%%%%%%
%%%%%%%%%%%%%%%%%%%%%%%%%%%%%%%%%%%%%%%%%%%%%%%%%%%%%%%%%%%%%%%%%%%%%%%%%%%%%%%%
%%%%%%%%%%%%%%%%%%%%%%%%%%%%%%%%%%%%%%%%%%%%%%%%%%%%%%%%%%%%%%%%%%%%%%%%%%%%%%%%
\appendix

\settowidth\MacroIndent{\rmfamily\scriptsize 000\ }

 \DocInput{childdoc.dtx}

\end{document}
%</driver>
% \fi
%
% %%%%%%%%%%%%%%%%%%%%%%%%%%%%%%%%%%%%%%%%%%%%%%%%%%%%%%%%%%%%%%%%%%%%%%%%%%%%%%
% %%%%%%%%%%%%%%%%%%%%%%%%%%%%%%%%%%%%%%%%%%%%%%%%%%%%%%%%%%%%%%%%%%%%%%%%%%%%%%
% \section{Sample}
%\iffalse
%<*samplemain>
%\fi
%
% The following presents a sample document
% with two chapters, two parts, a title page,
% a compile flag as well as three forwarding files to set the flag.
% It consists of eight |.tex| files:
% \begin{center}
% \begin{tabular}{ll}
% |cdocsamp.tex|&main file\\
% |cdocsch1.tex|&include file for chapter 1\\
% |cdocsch2.tex|&include file for chapter 2\\
% |cdocspt3.tex|&include file for part 3\\
% |cdocspt4.tex|&include file for part 4\\
% |cdocsdrf.tex|&forwarding file for main file in draft mode\\
% |cdocsfi1.tex|&forwarding file for final version of chapter 1\\
% |cdocsfi2.tex|&forwarding file for final version of chapter 2\\
% \end{tabular}
% \end{center}
% Each of the eight files can be compiled directly by the \LaTeX{} compiler.
%
% %%%%%%%%%%%%%%%%%%%%%%%%%%%%%%%%%%%%%%
% \paragraph{Main File.}
%
% The main file is called |cdocsamp.tex|.
%
% Load the \textsf{childdoc} definitions and
% declare the filename for the main document:
%    \begin{macrocode}
\input{childdoc.def}
\childdocmain{}
%    \end{macrocode}

% Optional override for |\version| flag:
%    \begin{macrocode}
%%\ifchilddoc\else\providecommand{\version}{draft}\fi
%    \end{macrocode}

% Define the default values for the |\version| flag
% (|final| for the main file and |draft| for childs):
%    \begin{macrocode}
\ifchilddoc
\providecommand{\version}{draft}
\else
\providecommand{\version}{final}
\fi
%    \end{macrocode}

% Load the standard document class:
%    \begin{macrocode}
\documentclass[12pt]{article}
%    \end{macrocode}

% Start the document body:
%    \begin{macrocode}
\begin{document}
%    \end{macrocode}

% Declare a title page.
% Print title, part of document being processed and version flag:
%    \begin{macrocode}
\addtocounter{page}{-1}
\begin{center}
{\LARGE\bfseries{}childdoc example\par}
\vspace{1cm}
\ifchilddoc
\ifchilddocmanual part\else chapter\fi:
`\childdocname' of `\childdocjob'\par
\else
main document: `\childdocjob'\par
\fi
version: \version\par
\end{center}
\newpage
%    \end{macrocode}

% Manually include selected file,
% otherwise process as usual:
%    \begin{macrocode}
\ifchilddocmanual
\section*{part `\childdocname'}
\input{\childdocname}
\else
%    \end{macrocode}

% Include the two chapters:
%    \begin{macrocode}
\include{cdocsch1}
\include{cdocsch2}
%    \end{macrocode}

% Include the two parts unless only chapters should be displayed:
%    \begin{macrocode}
\ifchilddoc\else
\section{part three}
\input{cdocspt3}
\section{part four}
\input{cdocspt4}
\fi
%    \end{macrocode}

% Process as usual until here:
%    \begin{macrocode}
\fi
%    \end{macrocode}

% End of document body:
%    \begin{macrocode}
\end{document}
%    \end{macrocode}
%\iffalse
%</samplemain>
%\fi
%
% %%%%%%%%%%%%%%%%%%%%%%%%%%%%%%%%%%%%%%
% \paragraph{Chapter Include Files.}
%
% The include files are called |cdocsch1.tex| and |cdocsch2.tex|.
%
%\iffalse
%<*samplechap1|samplechap2>
%\fi

% Optional override for |\version| flag:
%    \begin{macrocode}
%%\providecommand{\version}{final}
%    \end{macrocode}

% Include the main document:
%    \begin{macrocode}
\input{childdoc.def}
\childdocof{cdocsamp}
%    \end{macrocode}

%\iffalse
%</samplechap1|samplechap2>
%\fi
%
%\iffalse
%<*samplechap1>
%\fi
% Some text for chapter 1:
%    \begin{macrocode}
\section{one}
some text in chapter one
%    \end{macrocode}

%\iffalse
%</samplechap1>
%\fi
% Some text for chapter 2:
%\iffalse
%<*samplechap2>
%\fi
%    \begin{macrocode}
\section{two}
more text in chapter two
%    \end{macrocode}

%\iffalse
%</samplechap2>
%\fi
%
% %%%%%%%%%%%%%%%%%%%%%%%%%%%%%%%%%%%%%%
% \paragraph{Part Include Files.}
%
% The include files are called |cdocspt3.tex| and |cdocspt4.tex|.
%
%\iffalse
%<*samplepart3|samplepart4>
%\fi

% Optional override for |\version| flag:
%    \begin{macrocode}
%%\providecommand{\version}{final}
%    \end{macrocode}

% Include the main document:
%    \begin{macrocode}
\input{childdoc.def}
\childdocby{cdocsamp}
%    \end{macrocode}

%\iffalse
%</samplepart3|samplepart4>
%\fi
%
%\iffalse
%<*samplepart3>
%\fi
% Some text for part 3:
%    \begin{macrocode}
some text in part three
%    \end{macrocode}

%\iffalse
%</samplepart3>
%\fi
% Some text for part 4:
%\iffalse
%<*samplepart4>
%\fi
%    \begin{macrocode}
more text in part four
%    \end{macrocode}

%\iffalse
%</samplepart4>
%\fi
%
% %%%%%%%%%%%%%%%%%%%%%%%%%%%%%%%%%%%%%%
% \paragraph{Forwarding for a Complete Draft.}
%
% The following forwarding file |cdocsdrf.tex|
% compiles the main document in draft mode:
%\iffalse
%<*sampledraft>
%\fi
%    \begin{macrocode}
\def\version{draft}
\input{childdoc.def}
\childdocforward{cdocsamp}
%    \end{macrocode}

%\iffalse
%</sampledraft>
%\fi
%
% %%%%%%%%%%%%%%%%%%%%%%%%%%%%%%%%%%%%%%
% \paragraph{Forwarding for Final Version of the Chapters.}
%
% The following forwarding files |cdocsfn1.tex| and |cdocsfn2.tex|
% (with identical content)
% compile the final versions of the child documents
% |cdocsch1.tex| and |cdocsch2.tex|, respectively:
%\iffalse
%<*samplefinal>
%\fi
%    \begin{macrocode}
\def\version{final}
\input{childdoc.def}
\childdocforwardprefix[cdocsamp]{cdocsfn}{cdocsch}
%    \end{macrocode}

%\iffalse
%</samplefinal>
%\fi
%
% %%%%%%%%%%%%%%%%%%%%%%%%%%%%%%%%%%%%%%
% \paragraph{Command Line Processing.}
%
% The following three command lines generate the output files
% |cdocscld|, |cdocscl1| and |cdocscl2|
% which should be identical to
% |cdocsdrf|, |cdocsch1| and |cdocsfn2|, respectively:
% \begin{center}
% \begin{tabular}{l}
% |latex -jobname cdocscld \|\\
% |  "\def\version{draft}\input{childdoc.def}\childdocforward{cdocsamp}"|\\
% |latex -jobname cdocscl1 \|\\
% |  "\input{childdoc.def}\childdocforward[cdocsamp]{cdocsch1}"|\\
% |latex -jobname cdocscl2 \|\\
% |  "\def\version{final}\input{childdoc.def}\childdocforward{cdocsch2}"|
% \end{tabular}
% \end{center}
% Note that the trailing backslash on each first line
% merely continues the input to the second line
% (for convenient cut ant paste).
% Furthermore, the command |latex| can be replaced by any
% of its alternative versions such as |pdflatex|.
%
% %%%%%%%%%%%%%%%%%%%%%%%%%%%%%%%%%%%%%%%%%%%%%%%%%%%%%%%%%%%%%%%%%%%%%%%%%%%%%%
% %%%%%%%%%%%%%%%%%%%%%%%%%%%%%%%%%%%%%%%%%%%%%%%%%%%%%%%%%%%%%%%%%%%%%%%%%%%%%%
% \section{Implementation}
%\iffalse
%<*package>
%\fi
%
% This section describes the definitions file |childdoc.def|.

% The definitions cannot be loaded using |\usepackage| or |\RequirePackage|
% which has a mechanism to prevent loading a style file more than once.
% When loading the definitions by means of |\input|
% multiple instances have to be prevented manually:
%\iffalse
%This code needs to be before the `\ProvidesFile' directive
%which is defined at the beginning of this file.
%Therefore it is also placed there and commented out here.
%</package>
%<*discard>
%\fi
%    \begin{macrocode}
\ifdefined\childdocmain\endinput\fi
%    \end{macrocode}
%\iffalse
%</discard>
%<*package>
%\fi
%
% \macro{\ifchilddoc}
% \macro{\ifchilddocmanual}
% The conditional |\ifchilddoc| tells whether a
% child (true) or main (false) document is being compiled.
% The conditional |\ifchilddocmanual| tells whether
% the |\includeonly| mechanism is used (false) or
% the selection of child files must be performed manually (true).
% The definitions initialise to false:
%    \begin{macrocode}
\newif\ifchilddoc
\newif\ifchilddocmanual
%    \end{macrocode}

% \macro{\childdocname}
% \macro{\childdocjob}
% The macro |\childdocname| stores the name of the main document
% to be compiled. The macro |\childdocjob| stores the name of
% the document on which the \LaTeX{} compiler was originally invoked.
% The content of |\jobname| cannot be compared
% to filenames specified in the source due to different catcodes.
% The following code rescans |\jobname|, stores the result
% in |\childdocname| and saves a copy in |\childdocjob|:
%    \begin{macrocode}
\edef\childdocname{\scantokens\expandafter{\jobname\noexpand}}
\let\childdocjob\childdocname
%    \end{macrocode}

% \macro{\childdocdisable}
% The macro |\childdocdisable| prevents the main file
% from being processed more than once.
% At this stage, the main document command |\childdocmain|
% is assumed to be called once again where it should do nothing.
% Any subsequent call to it should prevent
% a secondary processing of the main document
% It overwrites the forwarding commands
% |\childdocof| and |\childdocforward|
% with empty macros to prevent further inclusions of the main document:
%    \begin{macrocode}
\newcommand{\childdocdisable}
{
  \renewcommand{\childdocmain}[1]{\renewcommand{\childdocmain}[1]{\endinput}}
  \renewcommand{\childdocof}[1]{}
  \renewcommand{\childdocby}[2][]{}
  \renewcommand{\childdocforward}[2][]{}
  \renewcommand{\childdocdisable}{}
}
%    \end{macrocode}

% \macro{\childdocmain}
% The macro |\childdocmain| is to be called at the top of the main file
% with nothing or the main filename (without extension) as argument.
% First, it breaks loops.
% If the argument is not empty and does not match |\childdocname|
% (which is set by the first inclusion of |childdoc.def|),
% |\ifchilddoc| is set to true, |\includeonly| is applied to the child file
% and |\jobname| is set to the main file
% (for proper handling of |.aux| files):
%    \begin{macrocode}
\newcommand{\childdocmain}[1]
{
  \childdocdisable\childdocmain{}
  \if?#1?\else
    \begingroup
      \def\childdoctmp{#1}
      \ifx\childdoctmp\childdocname
        \def\childdoctmp{}
      \else
        \def\childdoctmp
        {
          \childdoctrue
          \includeonly{\childdocname}
          \def\childdocjob{#1}
          \def\jobname{#1}
        }
      \fi
      \expandafter
    \endgroup
    \childdoctmp
  \fi
}
%    \end{macrocode}

% \macro{\childdocof}
% The command |\childdocof| redirects
% compilation to the main file |#1|.
%    \begin{macrocode}
\newcommand{\childdocof}[1]
{
  \childdocdisable
  \childdoctrue
  \includeonly{\childdocname}
  \def\jobname{#1}
  \def\childdocjob{#1}
  \input{#1}
}
%    \end{macrocode}

% \macro{\childdocby}
% The command |\childdocby| ....
%    \begin{macrocode}
\newcommand{\childdocby}[2][]
{
  \childdocdisable
  \childdoctrue
  \childdocmanualtrue
  \if?#1?\else
    \def\jobname{#2}
  \fi
  \def\childdocjob{#2}
  \input{#2}
  \endinput
}
%    \end{macrocode}

% \macro{\childdocforward}
% The command |\childdocforward| redirects
% compilation to the main file or
% (if the optional argument is given) a child file.
% Parameters are set as if the main file
% or a child file starting with |\childdocof| was compiled.
% Then compilation is handed over to the main file:
%    \begin{macrocode}
\newcommand{\childdocforward}[2][]
{
  \begingroup
    \if?#1?
      \def\childdoctmp
      {
        \def\childdocname{#2}
        \def\childdocjob{#2}
        \def\jobname{#2}
        \input{#2}
        \endinput
      }
    \else
      \def\childdoctmp
      {
        \childdocdisable
        \def\childdocname{#2}
        \childdoctrue
        \includeonly{#2}
        \def\childdocjob{#1}
        \def\jobname{#1}
        \input{#1}
        \endinput
      }
    \fi
    \expandafter
  \endgroup
  \childdoctmp
}
%    \end{macrocode}

% \macro{\childdocforwardprefix}
% The command |\childdocforwardprefix| redirects
% compilation to the main or a child file by means of a pattern.
% The prefix |#1| in the current filename is replaced by |#2|
% and the suffix of the current filename is kept
% (it is assumed that the filename does not contain the substring `|~~~|'
% which is used as a delimiter).
% Compilation is handed over to the new file by |\childdocforward|:
%    \begin{macrocode}
\newcommand{\childdocforwardprefix}[3][]
{
  \begingroup
    \def\childdocextract #2##1~~~{\def\childdoctmp{\childdocforward[#1]{#3##1}}}
    \expandafter\childdocextract\childdocname~~~
    \expandafter
  \endgroup
  \childdoctmp
}
%    \end{macrocode}

% \macro{\childdoc}
% The deprecated macro |\childdoc| is a legacy version of |\childdocmain|:
%    \begin{macrocode}
\newcommand{\childdoc}{\childdocmain}
%    \end{macrocode}

% \macro{\childdocredirect}
% The deprecated macro |\childdocredirect| is a legacy version
% of |\childdocforward| and |\childdocforwardprefix|:
%    \begin{macrocode}
\newcommand{\childdocredirect}[2][]
{
  \begingroup
    \if?#1?
      \def\childdoctmp{\childdocforward{#2}}
    \else
      \def\childdoctmp{\childdocforwardprefix{#1}{#2}}
    \fi
    \expandafter
  \endgroup
  \childdoctmp
}
%    \end{macrocode}

%\iffalse
%</package>
%\fi
%
\endinput
|\\
|\childdocforwardprefix{final}{child}|
\end{tabular}
\end{center}
%

Note that when several versions of a main file and/or of each child file
are to be generated, it may be convenient to set up a |Makefile| or
shell script to automatise the process.

%%%%%%%%%%%%%%%%%%%%%%%%%%%%%%%%%%%%%%%%%%%%%%%%%%%%%%%%%%%%%%%%%%%%%%%%%%%%%%%%
\subsection{Command Line Processing}
\label{sec:commandline}

The effect of redirection files can also be achieved by invoking
the \LaTeX{} compiler with a more elaborate command line.
Most conveniently this should be done as part
of a shell script or a |Makefile|.

When using \textsf{childdoc} in the main file, the following
command lines effectively perform a redirection
(note that depending on the shell being used,
backslashes may have to be doubled: `|\|' $\to$ `|\\|'):
%
\begin{center}
|... -jobname "|\textit{target}|" |\\|"|[\textit{flags}]%
|% \iffalse
%
% childdoc.dtx Copyright (C) 2017-2018 Niklas Beisert
%
% This work may be distributed and/or modified under the
% conditions of the LaTeX Project Public License, either version 1.3
% of this license or (at your option) any later version.
% The latest version of this license is in
%   http://www.latex-project.org/lppl.txt
% and version 1.3 or later is part of all distributions of LaTeX
% version 2005/12/01 or later.
%
% This work has the LPPL maintenance status `maintained'.
%
% The Current Maintainer of this work is Niklas Beisert.
%
% This work consists of the files childdoc.dtx and childdoc.ins
% and the derived files childdoc.def and cdocsamp.tex with
% cdocsch1.tex, cdocsch2.tex, cdocsdrf.tex, cdocsfn1.tex, cdocsfn2.tex.
%
%<package>\ifdefined\childdocmain\endinput\fi
%<package>\ProvidesFile{childdoc.def}[2018/12/30 v2.0 child document driver]
%<samplemain>\ProvidesFile{cdocsamp.tex}[2018/12/30 v2.0 sample for childdoc]
%<*driver>
%\ProvidesFile{childdoc.drv}[2018/12/30 v2.0 childdoc reference manual file]
\PassOptionsToClass{10pt,a4paper}{article}
\documentclass{ltxdoc}

\usepackage[margin=35mm]{geometry}
\usepackage{hyperref}
\usepackage{hyperxmp}
\usepackage[usenames]{color}

\hypersetup{colorlinks=true}
\hypersetup{pdfstartview=FitH}
\hypersetup{pdfpagemode=UseNone}
\hypersetup{pdfsource={}}
\hypersetup{pdflang={en-UK}}
\hypersetup{pdfcopyright={Copyright 2017-2018 Niklas Beisert.
  This work may be distributed and/or modified under the
  conditions of the LaTeX Project Public License, either version 1.3
  of this license or (at your option) any later version.}}
\hypersetup{pdflicenseurl={http://www.latex-project.org/lppl.txt}}
\hypersetup{pdfcontactaddress={ETH Zurich, ITP, HIT K,
  Wolfgang-Pauli-Strasse 27}}
\hypersetup{pdfcontactpostcode={8093}}
\hypersetup{pdfcontactcity={Zurich}}
\hypersetup{pdfcontactcountry={Switzerland}}
\hypersetup{pdfcontactemail={nbeisert@itp.phys.ethz.ch}}
\hypersetup{pdfcontacturl={http://people.phys.ethz.ch/\xmptilde nbeisert/}}

\newcommand{\secref}[1]{\hyperref[#1]{section \ref*{#1}}}

\parskip1ex
\parindent0pt
\let\olditemize\itemize
\def\itemize{\olditemize\parskip0pt}

\begin{document}

\title{The \textsf{childdoc} Package}
\hypersetup{pdftitle={The childdoc Package}}
\author{Niklas Beisert\\[2ex]
  Institut f\"ur Theoretische Physik\\
  Eidgen\"ossische Technische Hochschule Z\"urich\\
  Wolfgang-Pauli-Strasse 27, 8093 Z\"urich, Switzerland\\[1ex]
  \href{mailto:nbeisert@itp.phys.ethz.ch}
  {\texttt{nbeisert@itp.phys.ethz.ch}}}
\hypersetup{pdfauthor={Niklas Beisert}}
\hypersetup{pdfsubject={Manual for the LaTeX2e Package childdoc}}
\date{30 December 2018, \textsf{v2.0}}
\maketitle

\begin{abstract}\noindent
\textsf{childdoc} is a \LaTeXe{} package
that enables the direct compilation
of document sections included by |\include|
to individual files.
\end{abstract}

\begingroup
\parskip0ex
\tableofcontents
\endgroup

%%%%%%%%%%%%%%%%%%%%%%%%%%%%%%%%%%%%%%%%%%%%%%%%%%%%%%%%%%%%%%%%%%%%%%%%%%%%%%%%
%%%%%%%%%%%%%%%%%%%%%%%%%%%%%%%%%%%%%%%%%%%%%%%%%%%%%%%%%%%%%%%%%%%%%%%%%%%%%%%%
\section{Introduction}

\LaTeX{} provides a mechanism to structure a large document (such as a book)
into a main file and several child files (containing the chapters)
using the |\include| command.
This mechanism is beneficial for documents
which span hundreds of pages in order to
make the source file(s) more manageable.
Moreover, compilation can be restricted to
selected child files by means of the |\includeonly| command.
The latter feature can be used to reduce the compilation time while editing
(this was significantly more useful in the earlier days of \LaTeX{})
or to generate a smaller document which is easier to navigate.
Another application of |\includeonly| is to generate
documents consisting of selected parts of the complete document.

However, there are a few drawbacks of the plain |\include| mechanism:
\begin{itemize}
\item
The child files cannot be compiled on their own,
they can only be compiled via the main file.
A naive editing environment
(such as a text editor with an option
to have the current file processed by \LaTeX)
may require one to switch to the main file before compiling;
attempting to compile the child file produces errors.
\item
The main file must be modified (each time)
to adjust the |\includeonly| command
to the present needs. This easily leaves the main file in a messy state.
\item
The generated document will always carry the filename
of the main document. This is inconvenient if
several child files are to be compiled and
to be kept for distribution.
\end{itemize}

The present package provides a simple interface
to make child files individually compilable by \LaTeX{}.
Compiling a child file then has the same effect as compiling
the main file with an |\includeonly| command
to select the appropriate child.
Moreover the generated document will carry the name of the child
rather than the main file.
This resolves all three above issues.

This feature is meant to make the editing of books,
thesis documents and lecture notes somewhat more convenient.
However, the package can also be used efficiently for
composing a series of documents (such as exercise sheets)
which are typically distributed individually.
It then assists the author in generating the individual documents
(potentially in different versions)
as well as a document containing the collected series.
Another application is in developing style files
or other kinds of included material
where compilation of the style file could redirect
to a sample or test file.

%%%%%%%%%%%%%%%%%%%%%%%%%%%%%%%%%%%%%%%%%%%%%%%%%%%%%%%%%%%%%%%%%%%%%%%%%%%%%%%%
%%%%%%%%%%%%%%%%%%%%%%%%%%%%%%%%%%%%%%%%%%%%%%%%%%%%%%%%%%%%%%%%%%%%%%%%%%%%%%%%
\section{Usage}

First of all, the package \textsf{childdoc} is \emph{not} a standard
\LaTeXe{} |.sty| style file! Therefore it needs to be invoked in
a non-standard way.

%%%%%%%%%%%%%%%%%%%%%%%%%%%%%%%%%%%%%%%%%%%%%%%%%%%%%%%%%%%%%%%%%%%%%%%%%%%%%%%%
\subsection{Included Files}
\label{sec:include}

%%%%%%%%%%%%%%%%%%%%%%%%%%%%%%%%%%%%%%%%
\DescribeMacro{\childdocmain}
To use the package, add the commands
\begin{center}
\begin{tabular}{l}
|\input{childdoc.def}|\\
|\childdocmain{}|\\
\end{tabular}
\end{center}
at the very top of the main \LaTeX{} file,
in particular \emph{before} the |\documentclass| statement!
The argument of |\childdocmain| should be left empty
(but it must be present).

%%%%%%%%%%%%%%%%%%%%%%%%%%%%%%%%%%%%%%%%
\DescribeMacro{\childdocof}
Furthermore, add the commands
\begin{center}
\begin{tabular}{l}
|\input{childdoc.def}|\\
|\childdocof{|\textit{main}|}|\\
\end{tabular}
\end{center}
at the top of every child file \textit{child}
which is included by |\include{|\textit{child}|}|
from within the main file
(or at least for those files to be compiled individually).
The argument \textit{main} must be the filename of the main file.

There are a couple of
considerations in setting up the main and child documents:

%%%%%%%%%%%%%%%%%%%%%%%%%%%%%%%%%%%%%%%%
\paragraph{Restrictions.}

Please note the following restrictions:
\begin{itemize}
\item
|\childdocmain| must be called with one argument \textit{main}
to ensure compatibility with earlier version of the package.
It must either be empty (|\childdocmain{}|)
or precisely match the filename of the main file in which it is specified.
See \secref{sec:detection} for further information.
\item
The filename \textit{main} must be specified without the |.tex| extension.
\item
The filename \textit{main} is case sensitive
(even in case-insensitive file systems)
due to internal string comparison.
\item
The argument \textit{main} should be fully expanded, it cannot be a macro.
\item
Subdirectories and special characters should be avoided in filenames.
\item
The command |\childdocmain{|\textit{main}|}| must be followed by a whitespace.
It should not be followed immediately by another command
or by a comment mark `|%|'.
This is because the \TeX{} parser reads the token immediately following
the argument of |\childdocmain| and puts it
at the beginning of every child section;
however, a white\-space is ignored.
\end{itemize}

%%%%%%%%%%%%%%%%%%%%%%%%%%%%%%%%%%%%%%%%
\paragraph{Content of Main File.}

It is advisable to place all content in the child files included by |\include|.
Any output contained in the main file will appear in all child documents
unless suppressed manually;
it cannot be suppressed automatically by the |\includeonly| directive
and thus should normally be avoided.
A method to include some content in the main file
by means of conditional processing is described in \secref{sec:conditional}.

%%%%%%%%%%%%%%%%%%%%%%%%%%%%%%%%%%%%%%%%
\paragraph{Page Numbering.}

When only a part of the document is compiled,
the appropriate numbering of pages
(as well as other status parameters)
is determined from the |.aux| files.
The latter contain information from previous passes.
However this information needs to propagate through
all intermediate child documents.
Therefore the page numbering in child documents may well
be inconsistent until the complete document is compiled at least once.

A useful (if unconventional) way to always ensure a consistent
page numbering is to restart the numbering in each child document
and denote the pages by `\textit{child}|.|\textit{page}'
where \textit{child} represents the chapter/section number of the child file.
This can be achieved by the command
|\numberwithin{page}{|\textit{child}|}|
of the \textsf{amsmath} package
where \textit{child} can be |chapter| or |section|
depending on the chosen structuring.
Alternatively, one can modify the macro |\thepage| appropriately
and reset the counter |page| at the start of each child file.

%%%%%%%%%%%%%%%%%%%%%%%%%%%%%%%%%%%%%%%%%%%%%%%%%%%%%%%%%%%%%%%%%%%%%%%%%%%%%%%%
\subsection{Conditional Processing}
\label{sec:conditional}

The package provides a mechanism to compile different versions
of a document. To customise the versions further some conditional processing
can come in handy to distinguish which version is being compiled.
The package provides two macros to describe the compilation context:

%%%%%%%%%%%%%%%%%%%%%%%%%%%%%%%%%%%%%%%%
\DescribeMacro{\ifchilddoc}
The conditional |\ifchilddoc| distinguishes between the compilation of
child documents and the main document:
%
\begin{center}
|\ifchilddoc |\textit{child-code}| |[|\||else |\textit{main-code}]| \||fi|
\end{center}

%%%%%%%%%%%%%%%%%%%%%%%%%%%%%%%%%%%%%%%%
\DescribeMacro{\childdocname}
\DescribeMacro{\childdocjob}
The macro |\childdocname| contains the filename (without extension)
of the main or child file being processed.
Note that |\childdocjob| will always contain the name of the main file.

%%%%%%%%%%%%%%%%%%%%%%%%%%%%%%%%%%%%%%%%
\paragraph{Title Page.}

Conditional processing can be used to include a title or banner page
in the main document when proper precautions are taken.
Importantly, the code in the main file should ensure that the page counter
(as well as other status parameters which are stored in the |.aux| files)
takes the same value after the conditional processing.
Otherwise the page numbers may take divergent values
depending on which part is compiled.

For example, a title page could be declared by:
%
\begin{center}
\begin{tabular}{l}
|\ifchilddoc\||else|\\
|\addtocounter{page}{-1}|\\
\textit{code for title page}\\
|\newpage|\\
|\||fi|
\end{tabular}
\end{center}
%
A banner page for the child documents can be generated by:
%
\begin{center}
\begin{tabular}{l}
|\ifchilddoc|\\
|\addtocounter{page}{-1}|\\
\textit{code for banner page}\\
|\newpage|\\
|\||fi|
\end{tabular}
\end{center}
%
Here one could write a message such as:
\begin{center}
|This is the part \childdocname{} of \childdocjob{}.|
\end{center}

%%%%%%%%%%%%%%%%%%%%%%%%%%%%%%%%%%%%%%%%%%%%%%%%%%%%%%%%%%%%%%%%%%%%%%%%%%%%%%%%
\subsection{Flags}
\label{sec:flags}

The package makes it easy to generate different versions
of the main or child documents.
To this end compilation flags can be defined
and assigned different default values.
They will be particularly useful in conjunction
with the forwarding mechanism described in \secref{sec:forward}.

For example, it may be useful to have a flag |\version|
which can be set to |draft| or |final|.
The document source will contain some conditional code
depending on the value of |\version|.
Suppose further, the flag should default to |final| for the main file
and to |draft| for child files
which is a natural assignment for editing the document.
This is achieved by placing the following code
in the preamble of the main document
(below the |\childdocmain| directive):
%
\begin{center}
\begin{tabular}{l}
|\ifchilddoc|\\
|\providecommand{\version}{draft}|\\
|\||else|\\
|\providecommand{\version}{final}|\\
|\||fi|
\end{tabular}
\end{center}
%
The definition by |\providecommand| makes sure
that previous definitions are not overwritten.
Further statements |\providecommand{\version}{...}|
can thus be added before the above code to override it.

For the main file, one might add a line
(between |\childdocmain| and the above block)
%
\begin{center}
|%\ifchilddoc\||else\providecommand{\version}{draft}\||fi|
\end{center}
%
which can be uncommented to produce a draft version.
Likewise one can add a line to the very top of a child file
(above the |\childdocof{|\textit{main}|}| directive)
%
\begin{center}
|%\providecommand{\version}{final}|
\end{center}
%
which can be uncommented to produce the final version of this child document.

%%%%%%%%%%%%%%%%%%%%%%%%%%%%%%%%%%%%%%%%%%%%%%%%%%%%%%%%%%%%%%%%%%%%%%%%%%%%%%%%
\subsection{Forwarding}
\label{sec:forward}

Different versions of the main or child documents
using compilation flags as described in \secref{sec:flags}
can be (permanently) stored in different files
for convenient compilation, viewing and distribution.
To this end, the package defines a command
to pass on compilation to a different file:

%%%%%%%%%%%%%%%%%%%%%%%%%%%%%%%%%%%%%%%%
\DescribeMacro{\childdocforward}
The command |\childdocforward| redirects processing to
another source file:
%
\begin{center}
\begin{tabular}{l}
|\input{childdoc.def}|\\
|\childdocforward[|\textit{main}|]{|\textit{dest}|}|\\
\end{tabular}
\end{center}
%
The argument \textit{dest} is the destination file
(without extension).
It should be the main file or one of the child files.
Note that further \textsf{childdoc} directives
such as |\childdocof| and |\childdocforward|
in the indicated file will be processed in this form.
The optional argument \textit{main}
passes on directly to the main file \textit{main}
while pretending to compile the child \textit{dest}.
This form behaves as if \textit{dest}
issues |\childdocof{|\textit{main}|}| right away,
and no further \textsf{childdoc} directives will be processed.

%%%%%%%%%%%%%%%%%%%%%%%%%%%%%%%%%%%%%%%%
\DescribeMacro{\...prefix}
In the alternative form |\childdocforwardprefix|,
%
\begin{center}
\begin{tabular}{l}
|\input{childdoc.def}|\\
|\childdocforwardprefix[|\textit{main}|]{|\textit{prefix}|}{|\textit{dest}|}|
\end{tabular}
\end{center}
%
the destination file is determined by a pattern
depending on the current file:
To make this work, the current file must be called
`{\textit{prefix}\hspace{0.2em}\textit{suffix}}'
with \textit{prefix} matching precisely the argument.
Processing is then passed on to the file
`{\textit{dest}\hspace{0.2em}\textit{suffix}}'.
Surely, the same effect is achieved by
directly specifying the
argument `{\textit{dest}\hspace{0.2em}\textit{suffix}}'
in the first form.
However, that requires to set up a different file
for each child. With the alternative form of the command
all these files can have exactly the same content
which simplifies setting them up and maintaining them.

For example, the following file |draft.tex|
with a compilation flag |\version| as described in \secref{sec:flags}
compiles the main document as a draft:
%
\begin{center}
\begin{tabular}{l}
|\def\version{draft}|\\
|\input{childdoc.def}|\\
|\childdocforward{|\textit{main}|}|
\end{tabular}
\end{center}
%
Likewise, the following files |final|\textit{nn}|.tex|
compile the final version of the child document
|child|\textit{nn}|.tex|:
%
\begin{center}
\begin{tabular}{l}
|\def\version{final}|\\
|\input{childdoc.def}|\\
|\childdocforwardprefix{final}{child}|
\end{tabular}
\end{center}
%

Note that when several versions of a main file and/or of each child file
are to be generated, it may be convenient to set up a |Makefile| or
shell script to automatise the process.

%%%%%%%%%%%%%%%%%%%%%%%%%%%%%%%%%%%%%%%%%%%%%%%%%%%%%%%%%%%%%%%%%%%%%%%%%%%%%%%%
\subsection{Command Line Processing}
\label{sec:commandline}

The effect of redirection files can also be achieved by invoking
the \LaTeX{} compiler with a more elaborate command line.
Most conveniently this should be done as part
of a shell script or a |Makefile|.

When using \textsf{childdoc} in the main file, the following
command lines effectively perform a redirection
(note that depending on the shell being used,
backslashes may have to be doubled: `|\|' $\to$ `|\\|'):
%
\begin{center}
|... -jobname "|\textit{target}|" |\\|"|[\textit{flags}]%
|\input{childdoc.def}\childdocforward[|\textit{main}|]{|\textit{dest}|}"|
\end{center}
%
Here \textit{target} is the name of the output file,
\textit{main} is the name of the main file
and \textit{dest} is the name of the main or child file to be processed
(all filenames without extensions).
The optional argument \textit{main} can be omitted
if \textit{main} matches \textit{dest}.
Optionally, compilation \textit{flags} can be defined via |\def| commands.
This command line makes the \TeX{} engine believe
it is compiling the file \textit{target}
whose content is specified as the latter parameter.
The provided code then forwards the processing to
\textit{main} or \textit{dest} as described in \secref{sec:forward}.

%%%%%%%%%%%%%%%%%%%%%%%%%%%%%%%%%%%%%%%%%%%%%%%%%%%%%%%%%%%%%%%%%%%%%%%%%%%%%%%%
\subsection{Include by Input}
\label{sec:input}

Including child documents by |\include| has some restrictions by design.
Most notably, the content of a child document always occupies
its own set of pages; pages cannot be shared between child documents.
Usually, this behaviour makes perfect sense
because each child document contain an essential part of the document.
However, in some situations it may be desirable to compose
a document from a collection of parts
without having mandatory page breaks between then.
For this case, the package
provides a mechanism to include parts
by |\input| which can also be processed individually.
However, by construction this mechanism
requires manual handling of the content to be output.

%%%%%%%%%%%%%%%%%%%%%%%%%%%%%%%%%%%%%%%%
\DescribeMacro{\ifchilddocmanual}
The main file should be prepared as usual, see \secref{sec:include}.
However, the document body must make a distinction
between processing of an individual part and of the main document, e.g.:
%
\begin{center}
\begin{tabular}{l}
|\ifchilddocmanual|\\
|\input{\childdocname}|\\
|\||else|\\
\textit{document body with }|\input{|\textit{part}|}|\\
|\||fi|
\end{tabular}
\end{center}
%
The conditional |\ifchilddocmanual| is true whenever
a part to be included by |\input| is being compiled,
and the name of the part is stored in |\childdocname|.

%%%%%%%%%%%%%%%%%%%%%%%%%%%%%%%%%%%%%%%%
\DescribeMacro{\childdocby}
Each part to be included by |\input| should start with:
%
\begin{center}
\begin{tabular}{l}
|\input{childdoc.def}|\\
|\childdocby{|\textit{main}|}|\\
\end{tabular}
\end{center}
%
The directive |\childdocby| is similar to |\childdocof|
described in \secref{sec:include},
but the subsequent selection of content must be done manually.
To that end, both |\ifchilddoc| and |\ifchilddocmanual|
will be true upon processing of a part,
and the name of the part is stored in |\childdocname|.
Note that |\jobname| will be set to the filename of the current part
so that each part receives an individual |.aux| file
that does not interfere with the |.aux| file(s) of the main document.
This behaviour can be altered by the alternative form
|\childdocby[*]{|\textit{main}|}| (with a non-empty optional argument)
which uses the |.aux| file of the main document
by setting |\jobname| to \textit{main}.

%%%%%%%%%%%%%%%%%%%%%%%%%%%%%%%%%%%%%%%%%%%%%%%%%%%%%%%%%%%%%%%%%%%%%%%%%%%%%%%%
\subsection{Driver Development}
\label{sec:driver}

The \textsf{childdoc} mechanism can also be use for the development
of definition files such as \LaTeX{} styles or classes.
This case differs from the above setup with multiple parts
included by |\include| in that no |\includeonly| should be invoked.
This can be achieved by starting the include file
(before |\ProvidesPackage|) with:
%
\begin{center}
\begin{tabular}{l}
|\input{childdoc.def}|\\
|\childdocforward{|\textit{main}|}|\\
\end{tabular}
\end{center}
%
or alternatively with:
%
\begin{center}
\begin{tabular}{l}
|\input{childdoc.def}|\\
|\childdocby{|\textit{main}|}|\\
\end{tabular}
\end{center}
%
Both forms have slightly different effects as described above.
The main file is prepared as usual, see \secref{sec:include}.

%%%%%%%%%%%%%%%%%%%%%%%%%%%%%%%%%%%%%%%%%%%%%%%%%%%%%%%%%%%%%%%%%%%%%%%%%%%%%%%%
\subsection{Legacy Detection}
\label{sec:detection}

The directive |\childdocmain| in the main file can detect
whether the complete document or merely a child is to be compiled
even without using the directive |\childdocof|.
This method is deprecated because it is less robust
and there is no compelling reason to use it;
it is merely provided for backward compatibility
and it may be removed in future versions.

If the detection mechanism is to be used,
it is mandatory to correctly specify
the filename of the main file as the argument of |\childdocmain|:
%
\begin{center}
\begin{tabular}{l}
|\input{childdoc.def}|\\
|\childdocmain{|\textit{main}|}|\\
\end{tabular}
\end{center}
%
If |\jobname| does not match the argument \textit{main} of |\childdocmain|,
it is assumed that |\jobname| points to the child file to be compiled.
When using |\childdocmain| with the main file specified as argument,
it suffices to start a child file
with just |\input{|\textit{main}|}|
without loading of the package and using |\childdocof|.
If instead all processing is done
with the appropriate \textsf{childdoc} directives,
the argument of \textit{main} of |\childdocmain| can be empty.

An alternative version of the command line processing described
in \secref{sec:commandline} using the detection mechanism reads:
%
\begin{center}
|... -jobname "|\textit{target}|" "|[\textit{flags}]%
[|\def\jobname{|\textit{dest}|}|]|\input{|\textit{main}|}"|
\end{center}

%%%%%%%%%%%%%%%%%%%%%%%%%%%%%%%%%%%%%%%%%%%%%%%%%%%%%%%%%%%%%%%%%%%%%%%%%%%%%%%%
\subsection{Manual Code}
\label{sec:manual}

In case one cannot be certain whether the definitions file |childdoc.def|
is installed on the target \TeX{} distribution
and one prefers not to ship it,
it is conceivable to paste a few relevant commands into the sources.

To that end, drop all statements |\input{childdoc.def}|
and perform the replacements as outlined below.
Instead of |\childdocmain{|\textit{main}|}| add the following code
to the top of the main file:
%
\begin{center}
\begin{tabular}{l}
|\||ifdefined\childdocname\endinput\||fi\newif\ifchilddoc|\\
|\edef\childdocname{\scantokens\expandafter{\jobname\noexpand}}|\\
|\def\childdocmain{|\textit{main}|}\||ifx\childdocmain\childdocname\||else|\\
|\childdoctrue\includeonly{\childdocname}\let\jobname\childdocmain\||fi|\\
\end{tabular}
\end{center}
%
Instead of |\childdocof{|\textit{main}|}| just include the main file
at the top of each child file:
%
\begin{center}
|\input{|\textit{main}|}|
\end{center}
%
A simple redirection |\childdocforward{|\textit{dest}|}| is achieved by:
%
\begin{center}
|\def\jobname{|\textit{dest}|}\input{\jobname}|
\end{center}
%
The redirection with prefix
|\childdocforwardprefix[|\textit{prefix}|]{|\textit{dest}|}|
is accomplished by:
%
\begin{center}
\begin{tabular}{l}
|{\edef\jobname{\scantokens\expandafter{\jobname\noexpand}}|\\
|\def\redirectjob |\textit{prefix}|#1~~~{\gdef\jobname{|\textit{dest}|#1}}|\\
|\expandafter\redirectjob\jobname~~~}\input{\jobname}|
\end{tabular}
\end{center}

In an alternative approach,
child documents can be compiled by a specific command line
without additional code or specific definitions:
%
\begin{center}
|... -jobname "|\textit{target}|" "|[\textit{flags}]%
|\includeonly{|\textit{dest}|}\input{|\textit{main}|}"|
\end{center}
%

%%%%%%%%%%%%%%%%%%%%%%%%%%%%%%%%%%%%%%%%%%%%%%%%%%%%%%%%%%%%%%%%%%%%%%%%%%%%%%%%
%%%%%%%%%%%%%%%%%%%%%%%%%%%%%%%%%%%%%%%%%%%%%%%%%%%%%%%%%%%%%%%%%%%%%%%%%%%%%%%%
\section{Information}

%%%%%%%%%%%%%%%%%%%%%%%%%%%%%%%%%%%%%%%%%%%%%%%%%%%%%%%%%%%%%%%%%%%%%%%%%%%%%%%%
\subsection{Copyright}

Copyright \copyright{} 2017--2018 Niklas Beisert

This work may be distributed and/or modified under the
conditions of the \LaTeX{} Project Public License, either version 1.3
of this license or (at your option) any later version.
The latest version of this license is in
  \url{http://www.latex-project.org/lppl.txt}
and version 1.3 or later is part of all distributions of \LaTeX{}
version 2005/12/01 or later.

This work has the LPPL maintenance status `maintained'.

The Current Maintainer of this work is Niklas Beisert.

This work consists of the files |README.txt|, |childdoc.ins| and |childdoc.dtx|
as well as the derived files |childdoc.def|, |cdocsamp.tex|
with |cdocsch1.tex|, |cdocsch2.tex|, |cdocspt3.tex|, |cdocspt4.tex|,
|cdocsdrf.tex|, |cdocsfn1.tex|, |cdocsfn2.tex|
as well as |childdoc.pdf|.

%%%%%%%%%%%%%%%%%%%%%%%%%%%%%%%%%%%%%%%%%%%%%%%%%%%%%%%%%%%%%%%%%%%%%%%%%%%%%%%%
\subsection{Files and Installation}

The package consists of the files:
%
\begin{center}
\begin{tabular}{ll}
    |README.txt|   & readme file \\
    |childdoc.ins| & installation file \\
    |childdoc.dtx| & source file \\
    |childdoc.def| & definition file \\
    |cdocsamp.tex| & sample main file \\
    |cdocsch1.tex| & sample include file \\
    |cdocsch2.tex| & sample include file \\
    |cdocspt3.tex| & sample part file \\
    |cdocspt4.tex| & sample part file \\
    |cdocsdrf.tex| & sample redirection file \\
    |cdocsfn1.tex| & sample redirection file \\
    |cdocsfn2.tex| & sample redirection file \\
    |childdoc.pdf| & manual
\end{tabular}
\end{center}
%
The distribution consists of the files
|README.txt|, |childdoc.ins| and |childdoc.dtx|.
%
\begin{itemize}
\item
Run (pdf)\LaTeX{} on |childdoc.dtx|
to compile the manual |childdoc.pdf| (this file).
\item
Run \LaTeX{} on |childdoc.ins| to create the definitions file |childdoc.def|
and the sample |cdocsamp.tex| with include files
|cdocsch1.tex|, |cdocsch2.tex|, |cdocspt3.tex|, |cdocspt4.tex|,
|cdocsdrf.tex|, |cdocsfn1.tex|, |cdocsfn2.tex|.
Then copy the file |childdoc.def| to an appropriate directory of your \LaTeX{}
distribution, e.g.\ \textit{texmf-root}|/tex/latex/childdoc|.
\end{itemize}

%%%%%%%%%%%%%%%%%%%%%%%%%%%%%%%%%%%%%%%%%%%%%%%%%%%%%%%%%%%%%%%%%%%%%%%%%%%%%%%%
\subsection{Related CTAN Packages}

There are several other packages which offer a similar functionality:
%
\begin{itemize}
\item
The packages
\href{http://ctan.org/pkg/docmute}{\textsf{docmute}},
\href{http://ctan.org/pkg/includex}{\textsf{includex}} and
\href{http://ctan.org/pkg/standalone}{\textsf{standalone}}
provide commands to include only the document body of
a child file thus allowing both files to be compiled individually.
\item
The packages \href{http://ctan.org/pkg/subdocs}{\textsf{subdocs}}
and \href{http://ctan.org/pkg/subfiles}{\textsf{subfiles}}
provide structures in which the main and child documents can be
encapsulated and allowing them to be compiled individually.
The inclusion mechanism is different from the conventional |\include|.
\item
The package \href{http://ctan.org/pkg/combine}{\textsf{combine}}
is an elaborate solution to combine several documents into one.
\end{itemize}
%
See also the CTAN topic \href{http://ctan.org/topic/subdocs}{\textsf{subdocs}}
for further related packages.
The present package differs from the above solutions in that
a document structure constructed with the conventional |\include| mechanism
just needs two extra commands at the top of every file
such that all constituent files can be compiled individually.

%%%%%%%%%%%%%%%%%%%%%%%%%%%%%%%%%%%%%%%%%%%%%%%%%%%%%%%%%%%%%%%%%%%%%%%%%%%%%%%%
%\subsection{Feature Suggestions}
%
%The following is a list of features which may be useful for future
%versions of this package:
%%
%\begin{itemize}
%\item
%\ldots
%\end{itemize}

%%%%%%%%%%%%%%%%%%%%%%%%%%%%%%%%%%%%%%%%%%%%%%%%%%%%%%%%%%%%%%%%%%%%%%%%%%%%%%%%
\subsection{Revision History}

%%%%%%%%%%%%%%%%%%%%%%%%%%%%%%%%%%%%%%%%
\paragraph{v2.0:} 2018/12/30

\begin{itemize}
\item
immediate forward processing
\item
added |\childdocby| mechanism
\item
manual restructured
\end{itemize}

%%%%%%%%%%%%%%%%%%%%%%%%%%%%%%%%%%%%%%%%
\paragraph{v1.6:} 2018/01/17

\begin{itemize}
\item
application for development of include files
\item
corrections to manual
\end{itemize}

%%%%%%%%%%%%%%%%%%%%%%%%%%%%%%%%%%%%%%%%
\paragraph{v1.5:} 2017/05/21

\begin{itemize}
\item
more complete structuring introduced
\item
|\childdocof| introduced
\item
|\childdoc| renamed to |\childdocmain|
\item
|\childredirect| renamed to |\childdocforward| and |\childdocforwardprefix|
and functionality expanded
\end{itemize}

%%%%%%%%%%%%%%%%%%%%%%%%%%%%%%%%%%%%%%%%
\paragraph{v1.0:} 2017/04/27

\begin{itemize}
\item
manual and install package
\item
first version published on CTAN
\end{itemize}

%%%%%%%%%%%%%%%%%%%%%%%%%%%%%%%%%%%%%%%%
\paragraph{v0.6:} 2017/04/26

\begin{itemize}
\item
redirection mechanism added
\end{itemize}

%%%%%%%%%%%%%%%%%%%%%%%%%%%%%%%%%%%%%%%%
\paragraph{v0.5:} 2017/04/26

\begin{itemize}
\item
functionality in definition file
\end{itemize}


%%%%%%%%%%%%%%%%%%%%%%%%%%%%%%%%%%%%%%%%%%%%%%%%%%%%%%%%%%%%%%%%%%%%%%%%%%%%%%%%
%%%%%%%%%%%%%%%%%%%%%%%%%%%%%%%%%%%%%%%%%%%%%%%%%%%%%%%%%%%%%%%%%%%%%%%%%%%%%%%%
%%%%%%%%%%%%%%%%%%%%%%%%%%%%%%%%%%%%%%%%%%%%%%%%%%%%%%%%%%%%%%%%%%%%%%%%%%%%%%%%
\appendix

\settowidth\MacroIndent{\rmfamily\scriptsize 000\ }

 \DocInput{childdoc.dtx}

\end{document}
%</driver>
% \fi
%
% %%%%%%%%%%%%%%%%%%%%%%%%%%%%%%%%%%%%%%%%%%%%%%%%%%%%%%%%%%%%%%%%%%%%%%%%%%%%%%
% %%%%%%%%%%%%%%%%%%%%%%%%%%%%%%%%%%%%%%%%%%%%%%%%%%%%%%%%%%%%%%%%%%%%%%%%%%%%%%
% \section{Sample}
%\iffalse
%<*samplemain>
%\fi
%
% The following presents a sample document
% with two chapters, two parts, a title page,
% a compile flag as well as three forwarding files to set the flag.
% It consists of eight |.tex| files:
% \begin{center}
% \begin{tabular}{ll}
% |cdocsamp.tex|&main file\\
% |cdocsch1.tex|&include file for chapter 1\\
% |cdocsch2.tex|&include file for chapter 2\\
% |cdocspt3.tex|&include file for part 3\\
% |cdocspt4.tex|&include file for part 4\\
% |cdocsdrf.tex|&forwarding file for main file in draft mode\\
% |cdocsfi1.tex|&forwarding file for final version of chapter 1\\
% |cdocsfi2.tex|&forwarding file for final version of chapter 2\\
% \end{tabular}
% \end{center}
% Each of the eight files can be compiled directly by the \LaTeX{} compiler.
%
% %%%%%%%%%%%%%%%%%%%%%%%%%%%%%%%%%%%%%%
% \paragraph{Main File.}
%
% The main file is called |cdocsamp.tex|.
%
% Load the \textsf{childdoc} definitions and
% declare the filename for the main document:
%    \begin{macrocode}
\input{childdoc.def}
\childdocmain{}
%    \end{macrocode}

% Optional override for |\version| flag:
%    \begin{macrocode}
%%\ifchilddoc\else\providecommand{\version}{draft}\fi
%    \end{macrocode}

% Define the default values for the |\version| flag
% (|final| for the main file and |draft| for childs):
%    \begin{macrocode}
\ifchilddoc
\providecommand{\version}{draft}
\else
\providecommand{\version}{final}
\fi
%    \end{macrocode}

% Load the standard document class:
%    \begin{macrocode}
\documentclass[12pt]{article}
%    \end{macrocode}

% Start the document body:
%    \begin{macrocode}
\begin{document}
%    \end{macrocode}

% Declare a title page.
% Print title, part of document being processed and version flag:
%    \begin{macrocode}
\addtocounter{page}{-1}
\begin{center}
{\LARGE\bfseries{}childdoc example\par}
\vspace{1cm}
\ifchilddoc
\ifchilddocmanual part\else chapter\fi:
`\childdocname' of `\childdocjob'\par
\else
main document: `\childdocjob'\par
\fi
version: \version\par
\end{center}
\newpage
%    \end{macrocode}

% Manually include selected file,
% otherwise process as usual:
%    \begin{macrocode}
\ifchilddocmanual
\section*{part `\childdocname'}
\input{\childdocname}
\else
%    \end{macrocode}

% Include the two chapters:
%    \begin{macrocode}
\include{cdocsch1}
\include{cdocsch2}
%    \end{macrocode}

% Include the two parts unless only chapters should be displayed:
%    \begin{macrocode}
\ifchilddoc\else
\section{part three}
\input{cdocspt3}
\section{part four}
\input{cdocspt4}
\fi
%    \end{macrocode}

% Process as usual until here:
%    \begin{macrocode}
\fi
%    \end{macrocode}

% End of document body:
%    \begin{macrocode}
\end{document}
%    \end{macrocode}
%\iffalse
%</samplemain>
%\fi
%
% %%%%%%%%%%%%%%%%%%%%%%%%%%%%%%%%%%%%%%
% \paragraph{Chapter Include Files.}
%
% The include files are called |cdocsch1.tex| and |cdocsch2.tex|.
%
%\iffalse
%<*samplechap1|samplechap2>
%\fi

% Optional override for |\version| flag:
%    \begin{macrocode}
%%\providecommand{\version}{final}
%    \end{macrocode}

% Include the main document:
%    \begin{macrocode}
\input{childdoc.def}
\childdocof{cdocsamp}
%    \end{macrocode}

%\iffalse
%</samplechap1|samplechap2>
%\fi
%
%\iffalse
%<*samplechap1>
%\fi
% Some text for chapter 1:
%    \begin{macrocode}
\section{one}
some text in chapter one
%    \end{macrocode}

%\iffalse
%</samplechap1>
%\fi
% Some text for chapter 2:
%\iffalse
%<*samplechap2>
%\fi
%    \begin{macrocode}
\section{two}
more text in chapter two
%    \end{macrocode}

%\iffalse
%</samplechap2>
%\fi
%
% %%%%%%%%%%%%%%%%%%%%%%%%%%%%%%%%%%%%%%
% \paragraph{Part Include Files.}
%
% The include files are called |cdocspt3.tex| and |cdocspt4.tex|.
%
%\iffalse
%<*samplepart3|samplepart4>
%\fi

% Optional override for |\version| flag:
%    \begin{macrocode}
%%\providecommand{\version}{final}
%    \end{macrocode}

% Include the main document:
%    \begin{macrocode}
\input{childdoc.def}
\childdocby{cdocsamp}
%    \end{macrocode}

%\iffalse
%</samplepart3|samplepart4>
%\fi
%
%\iffalse
%<*samplepart3>
%\fi
% Some text for part 3:
%    \begin{macrocode}
some text in part three
%    \end{macrocode}

%\iffalse
%</samplepart3>
%\fi
% Some text for part 4:
%\iffalse
%<*samplepart4>
%\fi
%    \begin{macrocode}
more text in part four
%    \end{macrocode}

%\iffalse
%</samplepart4>
%\fi
%
% %%%%%%%%%%%%%%%%%%%%%%%%%%%%%%%%%%%%%%
% \paragraph{Forwarding for a Complete Draft.}
%
% The following forwarding file |cdocsdrf.tex|
% compiles the main document in draft mode:
%\iffalse
%<*sampledraft>
%\fi
%    \begin{macrocode}
\def\version{draft}
\input{childdoc.def}
\childdocforward{cdocsamp}
%    \end{macrocode}

%\iffalse
%</sampledraft>
%\fi
%
% %%%%%%%%%%%%%%%%%%%%%%%%%%%%%%%%%%%%%%
% \paragraph{Forwarding for Final Version of the Chapters.}
%
% The following forwarding files |cdocsfn1.tex| and |cdocsfn2.tex|
% (with identical content)
% compile the final versions of the child documents
% |cdocsch1.tex| and |cdocsch2.tex|, respectively:
%\iffalse
%<*samplefinal>
%\fi
%    \begin{macrocode}
\def\version{final}
\input{childdoc.def}
\childdocforwardprefix[cdocsamp]{cdocsfn}{cdocsch}
%    \end{macrocode}

%\iffalse
%</samplefinal>
%\fi
%
% %%%%%%%%%%%%%%%%%%%%%%%%%%%%%%%%%%%%%%
% \paragraph{Command Line Processing.}
%
% The following three command lines generate the output files
% |cdocscld|, |cdocscl1| and |cdocscl2|
% which should be identical to
% |cdocsdrf|, |cdocsch1| and |cdocsfn2|, respectively:
% \begin{center}
% \begin{tabular}{l}
% |latex -jobname cdocscld \|\\
% |  "\def\version{draft}\input{childdoc.def}\childdocforward{cdocsamp}"|\\
% |latex -jobname cdocscl1 \|\\
% |  "\input{childdoc.def}\childdocforward[cdocsamp]{cdocsch1}"|\\
% |latex -jobname cdocscl2 \|\\
% |  "\def\version{final}\input{childdoc.def}\childdocforward{cdocsch2}"|
% \end{tabular}
% \end{center}
% Note that the trailing backslash on each first line
% merely continues the input to the second line
% (for convenient cut ant paste).
% Furthermore, the command |latex| can be replaced by any
% of its alternative versions such as |pdflatex|.
%
% %%%%%%%%%%%%%%%%%%%%%%%%%%%%%%%%%%%%%%%%%%%%%%%%%%%%%%%%%%%%%%%%%%%%%%%%%%%%%%
% %%%%%%%%%%%%%%%%%%%%%%%%%%%%%%%%%%%%%%%%%%%%%%%%%%%%%%%%%%%%%%%%%%%%%%%%%%%%%%
% \section{Implementation}
%\iffalse
%<*package>
%\fi
%
% This section describes the definitions file |childdoc.def|.

% The definitions cannot be loaded using |\usepackage| or |\RequirePackage|
% which has a mechanism to prevent loading a style file more than once.
% When loading the definitions by means of |\input|
% multiple instances have to be prevented manually:
%\iffalse
%This code needs to be before the `\ProvidesFile' directive
%which is defined at the beginning of this file.
%Therefore it is also placed there and commented out here.
%</package>
%<*discard>
%\fi
%    \begin{macrocode}
\ifdefined\childdocmain\endinput\fi
%    \end{macrocode}
%\iffalse
%</discard>
%<*package>
%\fi
%
% \macro{\ifchilddoc}
% \macro{\ifchilddocmanual}
% The conditional |\ifchilddoc| tells whether a
% child (true) or main (false) document is being compiled.
% The conditional |\ifchilddocmanual| tells whether
% the |\includeonly| mechanism is used (false) or
% the selection of child files must be performed manually (true).
% The definitions initialise to false:
%    \begin{macrocode}
\newif\ifchilddoc
\newif\ifchilddocmanual
%    \end{macrocode}

% \macro{\childdocname}
% \macro{\childdocjob}
% The macro |\childdocname| stores the name of the main document
% to be compiled. The macro |\childdocjob| stores the name of
% the document on which the \LaTeX{} compiler was originally invoked.
% The content of |\jobname| cannot be compared
% to filenames specified in the source due to different catcodes.
% The following code rescans |\jobname|, stores the result
% in |\childdocname| and saves a copy in |\childdocjob|:
%    \begin{macrocode}
\edef\childdocname{\scantokens\expandafter{\jobname\noexpand}}
\let\childdocjob\childdocname
%    \end{macrocode}

% \macro{\childdocdisable}
% The macro |\childdocdisable| prevents the main file
% from being processed more than once.
% At this stage, the main document command |\childdocmain|
% is assumed to be called once again where it should do nothing.
% Any subsequent call to it should prevent
% a secondary processing of the main document
% It overwrites the forwarding commands
% |\childdocof| and |\childdocforward|
% with empty macros to prevent further inclusions of the main document:
%    \begin{macrocode}
\newcommand{\childdocdisable}
{
  \renewcommand{\childdocmain}[1]{\renewcommand{\childdocmain}[1]{\endinput}}
  \renewcommand{\childdocof}[1]{}
  \renewcommand{\childdocby}[2][]{}
  \renewcommand{\childdocforward}[2][]{}
  \renewcommand{\childdocdisable}{}
}
%    \end{macrocode}

% \macro{\childdocmain}
% The macro |\childdocmain| is to be called at the top of the main file
% with nothing or the main filename (without extension) as argument.
% First, it breaks loops.
% If the argument is not empty and does not match |\childdocname|
% (which is set by the first inclusion of |childdoc.def|),
% |\ifchilddoc| is set to true, |\includeonly| is applied to the child file
% and |\jobname| is set to the main file
% (for proper handling of |.aux| files):
%    \begin{macrocode}
\newcommand{\childdocmain}[1]
{
  \childdocdisable\childdocmain{}
  \if?#1?\else
    \begingroup
      \def\childdoctmp{#1}
      \ifx\childdoctmp\childdocname
        \def\childdoctmp{}
      \else
        \def\childdoctmp
        {
          \childdoctrue
          \includeonly{\childdocname}
          \def\childdocjob{#1}
          \def\jobname{#1}
        }
      \fi
      \expandafter
    \endgroup
    \childdoctmp
  \fi
}
%    \end{macrocode}

% \macro{\childdocof}
% The command |\childdocof| redirects
% compilation to the main file |#1|.
%    \begin{macrocode}
\newcommand{\childdocof}[1]
{
  \childdocdisable
  \childdoctrue
  \includeonly{\childdocname}
  \def\jobname{#1}
  \def\childdocjob{#1}
  \input{#1}
}
%    \end{macrocode}

% \macro{\childdocby}
% The command |\childdocby| ....
%    \begin{macrocode}
\newcommand{\childdocby}[2][]
{
  \childdocdisable
  \childdoctrue
  \childdocmanualtrue
  \if?#1?\else
    \def\jobname{#2}
  \fi
  \def\childdocjob{#2}
  \input{#2}
  \endinput
}
%    \end{macrocode}

% \macro{\childdocforward}
% The command |\childdocforward| redirects
% compilation to the main file or
% (if the optional argument is given) a child file.
% Parameters are set as if the main file
% or a child file starting with |\childdocof| was compiled.
% Then compilation is handed over to the main file:
%    \begin{macrocode}
\newcommand{\childdocforward}[2][]
{
  \begingroup
    \if?#1?
      \def\childdoctmp
      {
        \def\childdocname{#2}
        \def\childdocjob{#2}
        \def\jobname{#2}
        \input{#2}
        \endinput
      }
    \else
      \def\childdoctmp
      {
        \childdocdisable
        \def\childdocname{#2}
        \childdoctrue
        \includeonly{#2}
        \def\childdocjob{#1}
        \def\jobname{#1}
        \input{#1}
        \endinput
      }
    \fi
    \expandafter
  \endgroup
  \childdoctmp
}
%    \end{macrocode}

% \macro{\childdocforwardprefix}
% The command |\childdocforwardprefix| redirects
% compilation to the main or a child file by means of a pattern.
% The prefix |#1| in the current filename is replaced by |#2|
% and the suffix of the current filename is kept
% (it is assumed that the filename does not contain the substring `|~~~|'
% which is used as a delimiter).
% Compilation is handed over to the new file by |\childdocforward|:
%    \begin{macrocode}
\newcommand{\childdocforwardprefix}[3][]
{
  \begingroup
    \def\childdocextract #2##1~~~{\def\childdoctmp{\childdocforward[#1]{#3##1}}}
    \expandafter\childdocextract\childdocname~~~
    \expandafter
  \endgroup
  \childdoctmp
}
%    \end{macrocode}

% \macro{\childdoc}
% The deprecated macro |\childdoc| is a legacy version of |\childdocmain|:
%    \begin{macrocode}
\newcommand{\childdoc}{\childdocmain}
%    \end{macrocode}

% \macro{\childdocredirect}
% The deprecated macro |\childdocredirect| is a legacy version
% of |\childdocforward| and |\childdocforwardprefix|:
%    \begin{macrocode}
\newcommand{\childdocredirect}[2][]
{
  \begingroup
    \if?#1?
      \def\childdoctmp{\childdocforward{#2}}
    \else
      \def\childdoctmp{\childdocforwardprefix{#1}{#2}}
    \fi
    \expandafter
  \endgroup
  \childdoctmp
}
%    \end{macrocode}

%\iffalse
%</package>
%\fi
%
\endinput
\childdocforward[|\textit{main}|]{|\textit{dest}|}"|
\end{center}
%
Here \textit{target} is the name of the output file,
\textit{main} is the name of the main file
and \textit{dest} is the name of the main or child file to be processed
(all filenames without extensions).
The optional argument \textit{main} can be omitted
if \textit{main} matches \textit{dest}.
Optionally, compilation \textit{flags} can be defined via |\def| commands.
This command line makes the \TeX{} engine believe
it is compiling the file \textit{target}
whose content is specified as the latter parameter.
The provided code then forwards the processing to
\textit{main} or \textit{dest} as described in \secref{sec:forward}.

%%%%%%%%%%%%%%%%%%%%%%%%%%%%%%%%%%%%%%%%%%%%%%%%%%%%%%%%%%%%%%%%%%%%%%%%%%%%%%%%
\subsection{Include by Input}
\label{sec:input}

Including child documents by |\include| has some restrictions by design.
Most notably, the content of a child document always occupies
its own set of pages; pages cannot be shared between child documents.
Usually, this behaviour makes perfect sense
because each child document contain an essential part of the document.
However, in some situations it may be desirable to compose
a document from a collection of parts
without having mandatory page breaks between then.
For this case, the package
provides a mechanism to include parts
by |\input| which can also be processed individually.
However, by construction this mechanism
requires manual handling of the content to be output.

%%%%%%%%%%%%%%%%%%%%%%%%%%%%%%%%%%%%%%%%
\DescribeMacro{\ifchilddocmanual}
The main file should be prepared as usual, see \secref{sec:include}.
However, the document body must make a distinction
between processing of an individual part and of the main document, e.g.:
%
\begin{center}
\begin{tabular}{l}
|\ifchilddocmanual|\\
|\input{\childdocname}|\\
|\||else|\\
\textit{document body with }|\input{|\textit{part}|}|\\
|\||fi|
\end{tabular}
\end{center}
%
The conditional |\ifchilddocmanual| is true whenever
a part to be included by |\input| is being compiled,
and the name of the part is stored in |\childdocname|.

%%%%%%%%%%%%%%%%%%%%%%%%%%%%%%%%%%%%%%%%
\DescribeMacro{\childdocby}
Each part to be included by |\input| should start with:
%
\begin{center}
\begin{tabular}{l}
|% \iffalse
%
% childdoc.dtx Copyright (C) 2017-2018 Niklas Beisert
%
% This work may be distributed and/or modified under the
% conditions of the LaTeX Project Public License, either version 1.3
% of this license or (at your option) any later version.
% The latest version of this license is in
%   http://www.latex-project.org/lppl.txt
% and version 1.3 or later is part of all distributions of LaTeX
% version 2005/12/01 or later.
%
% This work has the LPPL maintenance status `maintained'.
%
% The Current Maintainer of this work is Niklas Beisert.
%
% This work consists of the files childdoc.dtx and childdoc.ins
% and the derived files childdoc.def and cdocsamp.tex with
% cdocsch1.tex, cdocsch2.tex, cdocsdrf.tex, cdocsfn1.tex, cdocsfn2.tex.
%
%<package>\ifdefined\childdocmain\endinput\fi
%<package>\ProvidesFile{childdoc.def}[2018/12/30 v2.0 child document driver]
%<samplemain>\ProvidesFile{cdocsamp.tex}[2018/12/30 v2.0 sample for childdoc]
%<*driver>
%\ProvidesFile{childdoc.drv}[2018/12/30 v2.0 childdoc reference manual file]
\PassOptionsToClass{10pt,a4paper}{article}
\documentclass{ltxdoc}

\usepackage[margin=35mm]{geometry}
\usepackage{hyperref}
\usepackage{hyperxmp}
\usepackage[usenames]{color}

\hypersetup{colorlinks=true}
\hypersetup{pdfstartview=FitH}
\hypersetup{pdfpagemode=UseNone}
\hypersetup{pdfsource={}}
\hypersetup{pdflang={en-UK}}
\hypersetup{pdfcopyright={Copyright 2017-2018 Niklas Beisert.
  This work may be distributed and/or modified under the
  conditions of the LaTeX Project Public License, either version 1.3
  of this license or (at your option) any later version.}}
\hypersetup{pdflicenseurl={http://www.latex-project.org/lppl.txt}}
\hypersetup{pdfcontactaddress={ETH Zurich, ITP, HIT K,
  Wolfgang-Pauli-Strasse 27}}
\hypersetup{pdfcontactpostcode={8093}}
\hypersetup{pdfcontactcity={Zurich}}
\hypersetup{pdfcontactcountry={Switzerland}}
\hypersetup{pdfcontactemail={nbeisert@itp.phys.ethz.ch}}
\hypersetup{pdfcontacturl={http://people.phys.ethz.ch/\xmptilde nbeisert/}}

\newcommand{\secref}[1]{\hyperref[#1]{section \ref*{#1}}}

\parskip1ex
\parindent0pt
\let\olditemize\itemize
\def\itemize{\olditemize\parskip0pt}

\begin{document}

\title{The \textsf{childdoc} Package}
\hypersetup{pdftitle={The childdoc Package}}
\author{Niklas Beisert\\[2ex]
  Institut f\"ur Theoretische Physik\\
  Eidgen\"ossische Technische Hochschule Z\"urich\\
  Wolfgang-Pauli-Strasse 27, 8093 Z\"urich, Switzerland\\[1ex]
  \href{mailto:nbeisert@itp.phys.ethz.ch}
  {\texttt{nbeisert@itp.phys.ethz.ch}}}
\hypersetup{pdfauthor={Niklas Beisert}}
\hypersetup{pdfsubject={Manual for the LaTeX2e Package childdoc}}
\date{30 December 2018, \textsf{v2.0}}
\maketitle

\begin{abstract}\noindent
\textsf{childdoc} is a \LaTeXe{} package
that enables the direct compilation
of document sections included by |\include|
to individual files.
\end{abstract}

\begingroup
\parskip0ex
\tableofcontents
\endgroup

%%%%%%%%%%%%%%%%%%%%%%%%%%%%%%%%%%%%%%%%%%%%%%%%%%%%%%%%%%%%%%%%%%%%%%%%%%%%%%%%
%%%%%%%%%%%%%%%%%%%%%%%%%%%%%%%%%%%%%%%%%%%%%%%%%%%%%%%%%%%%%%%%%%%%%%%%%%%%%%%%
\section{Introduction}

\LaTeX{} provides a mechanism to structure a large document (such as a book)
into a main file and several child files (containing the chapters)
using the |\include| command.
This mechanism is beneficial for documents
which span hundreds of pages in order to
make the source file(s) more manageable.
Moreover, compilation can be restricted to
selected child files by means of the |\includeonly| command.
The latter feature can be used to reduce the compilation time while editing
(this was significantly more useful in the earlier days of \LaTeX{})
or to generate a smaller document which is easier to navigate.
Another application of |\includeonly| is to generate
documents consisting of selected parts of the complete document.

However, there are a few drawbacks of the plain |\include| mechanism:
\begin{itemize}
\item
The child files cannot be compiled on their own,
they can only be compiled via the main file.
A naive editing environment
(such as a text editor with an option
to have the current file processed by \LaTeX)
may require one to switch to the main file before compiling;
attempting to compile the child file produces errors.
\item
The main file must be modified (each time)
to adjust the |\includeonly| command
to the present needs. This easily leaves the main file in a messy state.
\item
The generated document will always carry the filename
of the main document. This is inconvenient if
several child files are to be compiled and
to be kept for distribution.
\end{itemize}

The present package provides a simple interface
to make child files individually compilable by \LaTeX{}.
Compiling a child file then has the same effect as compiling
the main file with an |\includeonly| command
to select the appropriate child.
Moreover the generated document will carry the name of the child
rather than the main file.
This resolves all three above issues.

This feature is meant to make the editing of books,
thesis documents and lecture notes somewhat more convenient.
However, the package can also be used efficiently for
composing a series of documents (such as exercise sheets)
which are typically distributed individually.
It then assists the author in generating the individual documents
(potentially in different versions)
as well as a document containing the collected series.
Another application is in developing style files
or other kinds of included material
where compilation of the style file could redirect
to a sample or test file.

%%%%%%%%%%%%%%%%%%%%%%%%%%%%%%%%%%%%%%%%%%%%%%%%%%%%%%%%%%%%%%%%%%%%%%%%%%%%%%%%
%%%%%%%%%%%%%%%%%%%%%%%%%%%%%%%%%%%%%%%%%%%%%%%%%%%%%%%%%%%%%%%%%%%%%%%%%%%%%%%%
\section{Usage}

First of all, the package \textsf{childdoc} is \emph{not} a standard
\LaTeXe{} |.sty| style file! Therefore it needs to be invoked in
a non-standard way.

%%%%%%%%%%%%%%%%%%%%%%%%%%%%%%%%%%%%%%%%%%%%%%%%%%%%%%%%%%%%%%%%%%%%%%%%%%%%%%%%
\subsection{Included Files}
\label{sec:include}

%%%%%%%%%%%%%%%%%%%%%%%%%%%%%%%%%%%%%%%%
\DescribeMacro{\childdocmain}
To use the package, add the commands
\begin{center}
\begin{tabular}{l}
|\input{childdoc.def}|\\
|\childdocmain{}|\\
\end{tabular}
\end{center}
at the very top of the main \LaTeX{} file,
in particular \emph{before} the |\documentclass| statement!
The argument of |\childdocmain| should be left empty
(but it must be present).

%%%%%%%%%%%%%%%%%%%%%%%%%%%%%%%%%%%%%%%%
\DescribeMacro{\childdocof}
Furthermore, add the commands
\begin{center}
\begin{tabular}{l}
|\input{childdoc.def}|\\
|\childdocof{|\textit{main}|}|\\
\end{tabular}
\end{center}
at the top of every child file \textit{child}
which is included by |\include{|\textit{child}|}|
from within the main file
(or at least for those files to be compiled individually).
The argument \textit{main} must be the filename of the main file.

There are a couple of
considerations in setting up the main and child documents:

%%%%%%%%%%%%%%%%%%%%%%%%%%%%%%%%%%%%%%%%
\paragraph{Restrictions.}

Please note the following restrictions:
\begin{itemize}
\item
|\childdocmain| must be called with one argument \textit{main}
to ensure compatibility with earlier version of the package.
It must either be empty (|\childdocmain{}|)
or precisely match the filename of the main file in which it is specified.
See \secref{sec:detection} for further information.
\item
The filename \textit{main} must be specified without the |.tex| extension.
\item
The filename \textit{main} is case sensitive
(even in case-insensitive file systems)
due to internal string comparison.
\item
The argument \textit{main} should be fully expanded, it cannot be a macro.
\item
Subdirectories and special characters should be avoided in filenames.
\item
The command |\childdocmain{|\textit{main}|}| must be followed by a whitespace.
It should not be followed immediately by another command
or by a comment mark `|%|'.
This is because the \TeX{} parser reads the token immediately following
the argument of |\childdocmain| and puts it
at the beginning of every child section;
however, a white\-space is ignored.
\end{itemize}

%%%%%%%%%%%%%%%%%%%%%%%%%%%%%%%%%%%%%%%%
\paragraph{Content of Main File.}

It is advisable to place all content in the child files included by |\include|.
Any output contained in the main file will appear in all child documents
unless suppressed manually;
it cannot be suppressed automatically by the |\includeonly| directive
and thus should normally be avoided.
A method to include some content in the main file
by means of conditional processing is described in \secref{sec:conditional}.

%%%%%%%%%%%%%%%%%%%%%%%%%%%%%%%%%%%%%%%%
\paragraph{Page Numbering.}

When only a part of the document is compiled,
the appropriate numbering of pages
(as well as other status parameters)
is determined from the |.aux| files.
The latter contain information from previous passes.
However this information needs to propagate through
all intermediate child documents.
Therefore the page numbering in child documents may well
be inconsistent until the complete document is compiled at least once.

A useful (if unconventional) way to always ensure a consistent
page numbering is to restart the numbering in each child document
and denote the pages by `\textit{child}|.|\textit{page}'
where \textit{child} represents the chapter/section number of the child file.
This can be achieved by the command
|\numberwithin{page}{|\textit{child}|}|
of the \textsf{amsmath} package
where \textit{child} can be |chapter| or |section|
depending on the chosen structuring.
Alternatively, one can modify the macro |\thepage| appropriately
and reset the counter |page| at the start of each child file.

%%%%%%%%%%%%%%%%%%%%%%%%%%%%%%%%%%%%%%%%%%%%%%%%%%%%%%%%%%%%%%%%%%%%%%%%%%%%%%%%
\subsection{Conditional Processing}
\label{sec:conditional}

The package provides a mechanism to compile different versions
of a document. To customise the versions further some conditional processing
can come in handy to distinguish which version is being compiled.
The package provides two macros to describe the compilation context:

%%%%%%%%%%%%%%%%%%%%%%%%%%%%%%%%%%%%%%%%
\DescribeMacro{\ifchilddoc}
The conditional |\ifchilddoc| distinguishes between the compilation of
child documents and the main document:
%
\begin{center}
|\ifchilddoc |\textit{child-code}| |[|\||else |\textit{main-code}]| \||fi|
\end{center}

%%%%%%%%%%%%%%%%%%%%%%%%%%%%%%%%%%%%%%%%
\DescribeMacro{\childdocname}
\DescribeMacro{\childdocjob}
The macro |\childdocname| contains the filename (without extension)
of the main or child file being processed.
Note that |\childdocjob| will always contain the name of the main file.

%%%%%%%%%%%%%%%%%%%%%%%%%%%%%%%%%%%%%%%%
\paragraph{Title Page.}

Conditional processing can be used to include a title or banner page
in the main document when proper precautions are taken.
Importantly, the code in the main file should ensure that the page counter
(as well as other status parameters which are stored in the |.aux| files)
takes the same value after the conditional processing.
Otherwise the page numbers may take divergent values
depending on which part is compiled.

For example, a title page could be declared by:
%
\begin{center}
\begin{tabular}{l}
|\ifchilddoc\||else|\\
|\addtocounter{page}{-1}|\\
\textit{code for title page}\\
|\newpage|\\
|\||fi|
\end{tabular}
\end{center}
%
A banner page for the child documents can be generated by:
%
\begin{center}
\begin{tabular}{l}
|\ifchilddoc|\\
|\addtocounter{page}{-1}|\\
\textit{code for banner page}\\
|\newpage|\\
|\||fi|
\end{tabular}
\end{center}
%
Here one could write a message such as:
\begin{center}
|This is the part \childdocname{} of \childdocjob{}.|
\end{center}

%%%%%%%%%%%%%%%%%%%%%%%%%%%%%%%%%%%%%%%%%%%%%%%%%%%%%%%%%%%%%%%%%%%%%%%%%%%%%%%%
\subsection{Flags}
\label{sec:flags}

The package makes it easy to generate different versions
of the main or child documents.
To this end compilation flags can be defined
and assigned different default values.
They will be particularly useful in conjunction
with the forwarding mechanism described in \secref{sec:forward}.

For example, it may be useful to have a flag |\version|
which can be set to |draft| or |final|.
The document source will contain some conditional code
depending on the value of |\version|.
Suppose further, the flag should default to |final| for the main file
and to |draft| for child files
which is a natural assignment for editing the document.
This is achieved by placing the following code
in the preamble of the main document
(below the |\childdocmain| directive):
%
\begin{center}
\begin{tabular}{l}
|\ifchilddoc|\\
|\providecommand{\version}{draft}|\\
|\||else|\\
|\providecommand{\version}{final}|\\
|\||fi|
\end{tabular}
\end{center}
%
The definition by |\providecommand| makes sure
that previous definitions are not overwritten.
Further statements |\providecommand{\version}{...}|
can thus be added before the above code to override it.

For the main file, one might add a line
(between |\childdocmain| and the above block)
%
\begin{center}
|%\ifchilddoc\||else\providecommand{\version}{draft}\||fi|
\end{center}
%
which can be uncommented to produce a draft version.
Likewise one can add a line to the very top of a child file
(above the |\childdocof{|\textit{main}|}| directive)
%
\begin{center}
|%\providecommand{\version}{final}|
\end{center}
%
which can be uncommented to produce the final version of this child document.

%%%%%%%%%%%%%%%%%%%%%%%%%%%%%%%%%%%%%%%%%%%%%%%%%%%%%%%%%%%%%%%%%%%%%%%%%%%%%%%%
\subsection{Forwarding}
\label{sec:forward}

Different versions of the main or child documents
using compilation flags as described in \secref{sec:flags}
can be (permanently) stored in different files
for convenient compilation, viewing and distribution.
To this end, the package defines a command
to pass on compilation to a different file:

%%%%%%%%%%%%%%%%%%%%%%%%%%%%%%%%%%%%%%%%
\DescribeMacro{\childdocforward}
The command |\childdocforward| redirects processing to
another source file:
%
\begin{center}
\begin{tabular}{l}
|\input{childdoc.def}|\\
|\childdocforward[|\textit{main}|]{|\textit{dest}|}|\\
\end{tabular}
\end{center}
%
The argument \textit{dest} is the destination file
(without extension).
It should be the main file or one of the child files.
Note that further \textsf{childdoc} directives
such as |\childdocof| and |\childdocforward|
in the indicated file will be processed in this form.
The optional argument \textit{main}
passes on directly to the main file \textit{main}
while pretending to compile the child \textit{dest}.
This form behaves as if \textit{dest}
issues |\childdocof{|\textit{main}|}| right away,
and no further \textsf{childdoc} directives will be processed.

%%%%%%%%%%%%%%%%%%%%%%%%%%%%%%%%%%%%%%%%
\DescribeMacro{\...prefix}
In the alternative form |\childdocforwardprefix|,
%
\begin{center}
\begin{tabular}{l}
|\input{childdoc.def}|\\
|\childdocforwardprefix[|\textit{main}|]{|\textit{prefix}|}{|\textit{dest}|}|
\end{tabular}
\end{center}
%
the destination file is determined by a pattern
depending on the current file:
To make this work, the current file must be called
`{\textit{prefix}\hspace{0.2em}\textit{suffix}}'
with \textit{prefix} matching precisely the argument.
Processing is then passed on to the file
`{\textit{dest}\hspace{0.2em}\textit{suffix}}'.
Surely, the same effect is achieved by
directly specifying the
argument `{\textit{dest}\hspace{0.2em}\textit{suffix}}'
in the first form.
However, that requires to set up a different file
for each child. With the alternative form of the command
all these files can have exactly the same content
which simplifies setting them up and maintaining them.

For example, the following file |draft.tex|
with a compilation flag |\version| as described in \secref{sec:flags}
compiles the main document as a draft:
%
\begin{center}
\begin{tabular}{l}
|\def\version{draft}|\\
|\input{childdoc.def}|\\
|\childdocforward{|\textit{main}|}|
\end{tabular}
\end{center}
%
Likewise, the following files |final|\textit{nn}|.tex|
compile the final version of the child document
|child|\textit{nn}|.tex|:
%
\begin{center}
\begin{tabular}{l}
|\def\version{final}|\\
|\input{childdoc.def}|\\
|\childdocforwardprefix{final}{child}|
\end{tabular}
\end{center}
%

Note that when several versions of a main file and/or of each child file
are to be generated, it may be convenient to set up a |Makefile| or
shell script to automatise the process.

%%%%%%%%%%%%%%%%%%%%%%%%%%%%%%%%%%%%%%%%%%%%%%%%%%%%%%%%%%%%%%%%%%%%%%%%%%%%%%%%
\subsection{Command Line Processing}
\label{sec:commandline}

The effect of redirection files can also be achieved by invoking
the \LaTeX{} compiler with a more elaborate command line.
Most conveniently this should be done as part
of a shell script or a |Makefile|.

When using \textsf{childdoc} in the main file, the following
command lines effectively perform a redirection
(note that depending on the shell being used,
backslashes may have to be doubled: `|\|' $\to$ `|\\|'):
%
\begin{center}
|... -jobname "|\textit{target}|" |\\|"|[\textit{flags}]%
|\input{childdoc.def}\childdocforward[|\textit{main}|]{|\textit{dest}|}"|
\end{center}
%
Here \textit{target} is the name of the output file,
\textit{main} is the name of the main file
and \textit{dest} is the name of the main or child file to be processed
(all filenames without extensions).
The optional argument \textit{main} can be omitted
if \textit{main} matches \textit{dest}.
Optionally, compilation \textit{flags} can be defined via |\def| commands.
This command line makes the \TeX{} engine believe
it is compiling the file \textit{target}
whose content is specified as the latter parameter.
The provided code then forwards the processing to
\textit{main} or \textit{dest} as described in \secref{sec:forward}.

%%%%%%%%%%%%%%%%%%%%%%%%%%%%%%%%%%%%%%%%%%%%%%%%%%%%%%%%%%%%%%%%%%%%%%%%%%%%%%%%
\subsection{Include by Input}
\label{sec:input}

Including child documents by |\include| has some restrictions by design.
Most notably, the content of a child document always occupies
its own set of pages; pages cannot be shared between child documents.
Usually, this behaviour makes perfect sense
because each child document contain an essential part of the document.
However, in some situations it may be desirable to compose
a document from a collection of parts
without having mandatory page breaks between then.
For this case, the package
provides a mechanism to include parts
by |\input| which can also be processed individually.
However, by construction this mechanism
requires manual handling of the content to be output.

%%%%%%%%%%%%%%%%%%%%%%%%%%%%%%%%%%%%%%%%
\DescribeMacro{\ifchilddocmanual}
The main file should be prepared as usual, see \secref{sec:include}.
However, the document body must make a distinction
between processing of an individual part and of the main document, e.g.:
%
\begin{center}
\begin{tabular}{l}
|\ifchilddocmanual|\\
|\input{\childdocname}|\\
|\||else|\\
\textit{document body with }|\input{|\textit{part}|}|\\
|\||fi|
\end{tabular}
\end{center}
%
The conditional |\ifchilddocmanual| is true whenever
a part to be included by |\input| is being compiled,
and the name of the part is stored in |\childdocname|.

%%%%%%%%%%%%%%%%%%%%%%%%%%%%%%%%%%%%%%%%
\DescribeMacro{\childdocby}
Each part to be included by |\input| should start with:
%
\begin{center}
\begin{tabular}{l}
|\input{childdoc.def}|\\
|\childdocby{|\textit{main}|}|\\
\end{tabular}
\end{center}
%
The directive |\childdocby| is similar to |\childdocof|
described in \secref{sec:include},
but the subsequent selection of content must be done manually.
To that end, both |\ifchilddoc| and |\ifchilddocmanual|
will be true upon processing of a part,
and the name of the part is stored in |\childdocname|.
Note that |\jobname| will be set to the filename of the current part
so that each part receives an individual |.aux| file
that does not interfere with the |.aux| file(s) of the main document.
This behaviour can be altered by the alternative form
|\childdocby[*]{|\textit{main}|}| (with a non-empty optional argument)
which uses the |.aux| file of the main document
by setting |\jobname| to \textit{main}.

%%%%%%%%%%%%%%%%%%%%%%%%%%%%%%%%%%%%%%%%%%%%%%%%%%%%%%%%%%%%%%%%%%%%%%%%%%%%%%%%
\subsection{Driver Development}
\label{sec:driver}

The \textsf{childdoc} mechanism can also be use for the development
of definition files such as \LaTeX{} styles or classes.
This case differs from the above setup with multiple parts
included by |\include| in that no |\includeonly| should be invoked.
This can be achieved by starting the include file
(before |\ProvidesPackage|) with:
%
\begin{center}
\begin{tabular}{l}
|\input{childdoc.def}|\\
|\childdocforward{|\textit{main}|}|\\
\end{tabular}
\end{center}
%
or alternatively with:
%
\begin{center}
\begin{tabular}{l}
|\input{childdoc.def}|\\
|\childdocby{|\textit{main}|}|\\
\end{tabular}
\end{center}
%
Both forms have slightly different effects as described above.
The main file is prepared as usual, see \secref{sec:include}.

%%%%%%%%%%%%%%%%%%%%%%%%%%%%%%%%%%%%%%%%%%%%%%%%%%%%%%%%%%%%%%%%%%%%%%%%%%%%%%%%
\subsection{Legacy Detection}
\label{sec:detection}

The directive |\childdocmain| in the main file can detect
whether the complete document or merely a child is to be compiled
even without using the directive |\childdocof|.
This method is deprecated because it is less robust
and there is no compelling reason to use it;
it is merely provided for backward compatibility
and it may be removed in future versions.

If the detection mechanism is to be used,
it is mandatory to correctly specify
the filename of the main file as the argument of |\childdocmain|:
%
\begin{center}
\begin{tabular}{l}
|\input{childdoc.def}|\\
|\childdocmain{|\textit{main}|}|\\
\end{tabular}
\end{center}
%
If |\jobname| does not match the argument \textit{main} of |\childdocmain|,
it is assumed that |\jobname| points to the child file to be compiled.
When using |\childdocmain| with the main file specified as argument,
it suffices to start a child file
with just |\input{|\textit{main}|}|
without loading of the package and using |\childdocof|.
If instead all processing is done
with the appropriate \textsf{childdoc} directives,
the argument of \textit{main} of |\childdocmain| can be empty.

An alternative version of the command line processing described
in \secref{sec:commandline} using the detection mechanism reads:
%
\begin{center}
|... -jobname "|\textit{target}|" "|[\textit{flags}]%
[|\def\jobname{|\textit{dest}|}|]|\input{|\textit{main}|}"|
\end{center}

%%%%%%%%%%%%%%%%%%%%%%%%%%%%%%%%%%%%%%%%%%%%%%%%%%%%%%%%%%%%%%%%%%%%%%%%%%%%%%%%
\subsection{Manual Code}
\label{sec:manual}

In case one cannot be certain whether the definitions file |childdoc.def|
is installed on the target \TeX{} distribution
and one prefers not to ship it,
it is conceivable to paste a few relevant commands into the sources.

To that end, drop all statements |\input{childdoc.def}|
and perform the replacements as outlined below.
Instead of |\childdocmain{|\textit{main}|}| add the following code
to the top of the main file:
%
\begin{center}
\begin{tabular}{l}
|\||ifdefined\childdocname\endinput\||fi\newif\ifchilddoc|\\
|\edef\childdocname{\scantokens\expandafter{\jobname\noexpand}}|\\
|\def\childdocmain{|\textit{main}|}\||ifx\childdocmain\childdocname\||else|\\
|\childdoctrue\includeonly{\childdocname}\let\jobname\childdocmain\||fi|\\
\end{tabular}
\end{center}
%
Instead of |\childdocof{|\textit{main}|}| just include the main file
at the top of each child file:
%
\begin{center}
|\input{|\textit{main}|}|
\end{center}
%
A simple redirection |\childdocforward{|\textit{dest}|}| is achieved by:
%
\begin{center}
|\def\jobname{|\textit{dest}|}\input{\jobname}|
\end{center}
%
The redirection with prefix
|\childdocforwardprefix[|\textit{prefix}|]{|\textit{dest}|}|
is accomplished by:
%
\begin{center}
\begin{tabular}{l}
|{\edef\jobname{\scantokens\expandafter{\jobname\noexpand}}|\\
|\def\redirectjob |\textit{prefix}|#1~~~{\gdef\jobname{|\textit{dest}|#1}}|\\
|\expandafter\redirectjob\jobname~~~}\input{\jobname}|
\end{tabular}
\end{center}

In an alternative approach,
child documents can be compiled by a specific command line
without additional code or specific definitions:
%
\begin{center}
|... -jobname "|\textit{target}|" "|[\textit{flags}]%
|\includeonly{|\textit{dest}|}\input{|\textit{main}|}"|
\end{center}
%

%%%%%%%%%%%%%%%%%%%%%%%%%%%%%%%%%%%%%%%%%%%%%%%%%%%%%%%%%%%%%%%%%%%%%%%%%%%%%%%%
%%%%%%%%%%%%%%%%%%%%%%%%%%%%%%%%%%%%%%%%%%%%%%%%%%%%%%%%%%%%%%%%%%%%%%%%%%%%%%%%
\section{Information}

%%%%%%%%%%%%%%%%%%%%%%%%%%%%%%%%%%%%%%%%%%%%%%%%%%%%%%%%%%%%%%%%%%%%%%%%%%%%%%%%
\subsection{Copyright}

Copyright \copyright{} 2017--2018 Niklas Beisert

This work may be distributed and/or modified under the
conditions of the \LaTeX{} Project Public License, either version 1.3
of this license or (at your option) any later version.
The latest version of this license is in
  \url{http://www.latex-project.org/lppl.txt}
and version 1.3 or later is part of all distributions of \LaTeX{}
version 2005/12/01 or later.

This work has the LPPL maintenance status `maintained'.

The Current Maintainer of this work is Niklas Beisert.

This work consists of the files |README.txt|, |childdoc.ins| and |childdoc.dtx|
as well as the derived files |childdoc.def|, |cdocsamp.tex|
with |cdocsch1.tex|, |cdocsch2.tex|, |cdocspt3.tex|, |cdocspt4.tex|,
|cdocsdrf.tex|, |cdocsfn1.tex|, |cdocsfn2.tex|
as well as |childdoc.pdf|.

%%%%%%%%%%%%%%%%%%%%%%%%%%%%%%%%%%%%%%%%%%%%%%%%%%%%%%%%%%%%%%%%%%%%%%%%%%%%%%%%
\subsection{Files and Installation}

The package consists of the files:
%
\begin{center}
\begin{tabular}{ll}
    |README.txt|   & readme file \\
    |childdoc.ins| & installation file \\
    |childdoc.dtx| & source file \\
    |childdoc.def| & definition file \\
    |cdocsamp.tex| & sample main file \\
    |cdocsch1.tex| & sample include file \\
    |cdocsch2.tex| & sample include file \\
    |cdocspt3.tex| & sample part file \\
    |cdocspt4.tex| & sample part file \\
    |cdocsdrf.tex| & sample redirection file \\
    |cdocsfn1.tex| & sample redirection file \\
    |cdocsfn2.tex| & sample redirection file \\
    |childdoc.pdf| & manual
\end{tabular}
\end{center}
%
The distribution consists of the files
|README.txt|, |childdoc.ins| and |childdoc.dtx|.
%
\begin{itemize}
\item
Run (pdf)\LaTeX{} on |childdoc.dtx|
to compile the manual |childdoc.pdf| (this file).
\item
Run \LaTeX{} on |childdoc.ins| to create the definitions file |childdoc.def|
and the sample |cdocsamp.tex| with include files
|cdocsch1.tex|, |cdocsch2.tex|, |cdocspt3.tex|, |cdocspt4.tex|,
|cdocsdrf.tex|, |cdocsfn1.tex|, |cdocsfn2.tex|.
Then copy the file |childdoc.def| to an appropriate directory of your \LaTeX{}
distribution, e.g.\ \textit{texmf-root}|/tex/latex/childdoc|.
\end{itemize}

%%%%%%%%%%%%%%%%%%%%%%%%%%%%%%%%%%%%%%%%%%%%%%%%%%%%%%%%%%%%%%%%%%%%%%%%%%%%%%%%
\subsection{Related CTAN Packages}

There are several other packages which offer a similar functionality:
%
\begin{itemize}
\item
The packages
\href{http://ctan.org/pkg/docmute}{\textsf{docmute}},
\href{http://ctan.org/pkg/includex}{\textsf{includex}} and
\href{http://ctan.org/pkg/standalone}{\textsf{standalone}}
provide commands to include only the document body of
a child file thus allowing both files to be compiled individually.
\item
The packages \href{http://ctan.org/pkg/subdocs}{\textsf{subdocs}}
and \href{http://ctan.org/pkg/subfiles}{\textsf{subfiles}}
provide structures in which the main and child documents can be
encapsulated and allowing them to be compiled individually.
The inclusion mechanism is different from the conventional |\include|.
\item
The package \href{http://ctan.org/pkg/combine}{\textsf{combine}}
is an elaborate solution to combine several documents into one.
\end{itemize}
%
See also the CTAN topic \href{http://ctan.org/topic/subdocs}{\textsf{subdocs}}
for further related packages.
The present package differs from the above solutions in that
a document structure constructed with the conventional |\include| mechanism
just needs two extra commands at the top of every file
such that all constituent files can be compiled individually.

%%%%%%%%%%%%%%%%%%%%%%%%%%%%%%%%%%%%%%%%%%%%%%%%%%%%%%%%%%%%%%%%%%%%%%%%%%%%%%%%
%\subsection{Feature Suggestions}
%
%The following is a list of features which may be useful for future
%versions of this package:
%%
%\begin{itemize}
%\item
%\ldots
%\end{itemize}

%%%%%%%%%%%%%%%%%%%%%%%%%%%%%%%%%%%%%%%%%%%%%%%%%%%%%%%%%%%%%%%%%%%%%%%%%%%%%%%%
\subsection{Revision History}

%%%%%%%%%%%%%%%%%%%%%%%%%%%%%%%%%%%%%%%%
\paragraph{v2.0:} 2018/12/30

\begin{itemize}
\item
immediate forward processing
\item
added |\childdocby| mechanism
\item
manual restructured
\end{itemize}

%%%%%%%%%%%%%%%%%%%%%%%%%%%%%%%%%%%%%%%%
\paragraph{v1.6:} 2018/01/17

\begin{itemize}
\item
application for development of include files
\item
corrections to manual
\end{itemize}

%%%%%%%%%%%%%%%%%%%%%%%%%%%%%%%%%%%%%%%%
\paragraph{v1.5:} 2017/05/21

\begin{itemize}
\item
more complete structuring introduced
\item
|\childdocof| introduced
\item
|\childdoc| renamed to |\childdocmain|
\item
|\childredirect| renamed to |\childdocforward| and |\childdocforwardprefix|
and functionality expanded
\end{itemize}

%%%%%%%%%%%%%%%%%%%%%%%%%%%%%%%%%%%%%%%%
\paragraph{v1.0:} 2017/04/27

\begin{itemize}
\item
manual and install package
\item
first version published on CTAN
\end{itemize}

%%%%%%%%%%%%%%%%%%%%%%%%%%%%%%%%%%%%%%%%
\paragraph{v0.6:} 2017/04/26

\begin{itemize}
\item
redirection mechanism added
\end{itemize}

%%%%%%%%%%%%%%%%%%%%%%%%%%%%%%%%%%%%%%%%
\paragraph{v0.5:} 2017/04/26

\begin{itemize}
\item
functionality in definition file
\end{itemize}


%%%%%%%%%%%%%%%%%%%%%%%%%%%%%%%%%%%%%%%%%%%%%%%%%%%%%%%%%%%%%%%%%%%%%%%%%%%%%%%%
%%%%%%%%%%%%%%%%%%%%%%%%%%%%%%%%%%%%%%%%%%%%%%%%%%%%%%%%%%%%%%%%%%%%%%%%%%%%%%%%
%%%%%%%%%%%%%%%%%%%%%%%%%%%%%%%%%%%%%%%%%%%%%%%%%%%%%%%%%%%%%%%%%%%%%%%%%%%%%%%%
\appendix

\settowidth\MacroIndent{\rmfamily\scriptsize 000\ }

 \DocInput{childdoc.dtx}

\end{document}
%</driver>
% \fi
%
% %%%%%%%%%%%%%%%%%%%%%%%%%%%%%%%%%%%%%%%%%%%%%%%%%%%%%%%%%%%%%%%%%%%%%%%%%%%%%%
% %%%%%%%%%%%%%%%%%%%%%%%%%%%%%%%%%%%%%%%%%%%%%%%%%%%%%%%%%%%%%%%%%%%%%%%%%%%%%%
% \section{Sample}
%\iffalse
%<*samplemain>
%\fi
%
% The following presents a sample document
% with two chapters, two parts, a title page,
% a compile flag as well as three forwarding files to set the flag.
% It consists of eight |.tex| files:
% \begin{center}
% \begin{tabular}{ll}
% |cdocsamp.tex|&main file\\
% |cdocsch1.tex|&include file for chapter 1\\
% |cdocsch2.tex|&include file for chapter 2\\
% |cdocspt3.tex|&include file for part 3\\
% |cdocspt4.tex|&include file for part 4\\
% |cdocsdrf.tex|&forwarding file for main file in draft mode\\
% |cdocsfi1.tex|&forwarding file for final version of chapter 1\\
% |cdocsfi2.tex|&forwarding file for final version of chapter 2\\
% \end{tabular}
% \end{center}
% Each of the eight files can be compiled directly by the \LaTeX{} compiler.
%
% %%%%%%%%%%%%%%%%%%%%%%%%%%%%%%%%%%%%%%
% \paragraph{Main File.}
%
% The main file is called |cdocsamp.tex|.
%
% Load the \textsf{childdoc} definitions and
% declare the filename for the main document:
%    \begin{macrocode}
\input{childdoc.def}
\childdocmain{}
%    \end{macrocode}

% Optional override for |\version| flag:
%    \begin{macrocode}
%%\ifchilddoc\else\providecommand{\version}{draft}\fi
%    \end{macrocode}

% Define the default values for the |\version| flag
% (|final| for the main file and |draft| for childs):
%    \begin{macrocode}
\ifchilddoc
\providecommand{\version}{draft}
\else
\providecommand{\version}{final}
\fi
%    \end{macrocode}

% Load the standard document class:
%    \begin{macrocode}
\documentclass[12pt]{article}
%    \end{macrocode}

% Start the document body:
%    \begin{macrocode}
\begin{document}
%    \end{macrocode}

% Declare a title page.
% Print title, part of document being processed and version flag:
%    \begin{macrocode}
\addtocounter{page}{-1}
\begin{center}
{\LARGE\bfseries{}childdoc example\par}
\vspace{1cm}
\ifchilddoc
\ifchilddocmanual part\else chapter\fi:
`\childdocname' of `\childdocjob'\par
\else
main document: `\childdocjob'\par
\fi
version: \version\par
\end{center}
\newpage
%    \end{macrocode}

% Manually include selected file,
% otherwise process as usual:
%    \begin{macrocode}
\ifchilddocmanual
\section*{part `\childdocname'}
\input{\childdocname}
\else
%    \end{macrocode}

% Include the two chapters:
%    \begin{macrocode}
\include{cdocsch1}
\include{cdocsch2}
%    \end{macrocode}

% Include the two parts unless only chapters should be displayed:
%    \begin{macrocode}
\ifchilddoc\else
\section{part three}
\input{cdocspt3}
\section{part four}
\input{cdocspt4}
\fi
%    \end{macrocode}

% Process as usual until here:
%    \begin{macrocode}
\fi
%    \end{macrocode}

% End of document body:
%    \begin{macrocode}
\end{document}
%    \end{macrocode}
%\iffalse
%</samplemain>
%\fi
%
% %%%%%%%%%%%%%%%%%%%%%%%%%%%%%%%%%%%%%%
% \paragraph{Chapter Include Files.}
%
% The include files are called |cdocsch1.tex| and |cdocsch2.tex|.
%
%\iffalse
%<*samplechap1|samplechap2>
%\fi

% Optional override for |\version| flag:
%    \begin{macrocode}
%%\providecommand{\version}{final}
%    \end{macrocode}

% Include the main document:
%    \begin{macrocode}
\input{childdoc.def}
\childdocof{cdocsamp}
%    \end{macrocode}

%\iffalse
%</samplechap1|samplechap2>
%\fi
%
%\iffalse
%<*samplechap1>
%\fi
% Some text for chapter 1:
%    \begin{macrocode}
\section{one}
some text in chapter one
%    \end{macrocode}

%\iffalse
%</samplechap1>
%\fi
% Some text for chapter 2:
%\iffalse
%<*samplechap2>
%\fi
%    \begin{macrocode}
\section{two}
more text in chapter two
%    \end{macrocode}

%\iffalse
%</samplechap2>
%\fi
%
% %%%%%%%%%%%%%%%%%%%%%%%%%%%%%%%%%%%%%%
% \paragraph{Part Include Files.}
%
% The include files are called |cdocspt3.tex| and |cdocspt4.tex|.
%
%\iffalse
%<*samplepart3|samplepart4>
%\fi

% Optional override for |\version| flag:
%    \begin{macrocode}
%%\providecommand{\version}{final}
%    \end{macrocode}

% Include the main document:
%    \begin{macrocode}
\input{childdoc.def}
\childdocby{cdocsamp}
%    \end{macrocode}

%\iffalse
%</samplepart3|samplepart4>
%\fi
%
%\iffalse
%<*samplepart3>
%\fi
% Some text for part 3:
%    \begin{macrocode}
some text in part three
%    \end{macrocode}

%\iffalse
%</samplepart3>
%\fi
% Some text for part 4:
%\iffalse
%<*samplepart4>
%\fi
%    \begin{macrocode}
more text in part four
%    \end{macrocode}

%\iffalse
%</samplepart4>
%\fi
%
% %%%%%%%%%%%%%%%%%%%%%%%%%%%%%%%%%%%%%%
% \paragraph{Forwarding for a Complete Draft.}
%
% The following forwarding file |cdocsdrf.tex|
% compiles the main document in draft mode:
%\iffalse
%<*sampledraft>
%\fi
%    \begin{macrocode}
\def\version{draft}
\input{childdoc.def}
\childdocforward{cdocsamp}
%    \end{macrocode}

%\iffalse
%</sampledraft>
%\fi
%
% %%%%%%%%%%%%%%%%%%%%%%%%%%%%%%%%%%%%%%
% \paragraph{Forwarding for Final Version of the Chapters.}
%
% The following forwarding files |cdocsfn1.tex| and |cdocsfn2.tex|
% (with identical content)
% compile the final versions of the child documents
% |cdocsch1.tex| and |cdocsch2.tex|, respectively:
%\iffalse
%<*samplefinal>
%\fi
%    \begin{macrocode}
\def\version{final}
\input{childdoc.def}
\childdocforwardprefix[cdocsamp]{cdocsfn}{cdocsch}
%    \end{macrocode}

%\iffalse
%</samplefinal>
%\fi
%
% %%%%%%%%%%%%%%%%%%%%%%%%%%%%%%%%%%%%%%
% \paragraph{Command Line Processing.}
%
% The following three command lines generate the output files
% |cdocscld|, |cdocscl1| and |cdocscl2|
% which should be identical to
% |cdocsdrf|, |cdocsch1| and |cdocsfn2|, respectively:
% \begin{center}
% \begin{tabular}{l}
% |latex -jobname cdocscld \|\\
% |  "\def\version{draft}\input{childdoc.def}\childdocforward{cdocsamp}"|\\
% |latex -jobname cdocscl1 \|\\
% |  "\input{childdoc.def}\childdocforward[cdocsamp]{cdocsch1}"|\\
% |latex -jobname cdocscl2 \|\\
% |  "\def\version{final}\input{childdoc.def}\childdocforward{cdocsch2}"|
% \end{tabular}
% \end{center}
% Note that the trailing backslash on each first line
% merely continues the input to the second line
% (for convenient cut ant paste).
% Furthermore, the command |latex| can be replaced by any
% of its alternative versions such as |pdflatex|.
%
% %%%%%%%%%%%%%%%%%%%%%%%%%%%%%%%%%%%%%%%%%%%%%%%%%%%%%%%%%%%%%%%%%%%%%%%%%%%%%%
% %%%%%%%%%%%%%%%%%%%%%%%%%%%%%%%%%%%%%%%%%%%%%%%%%%%%%%%%%%%%%%%%%%%%%%%%%%%%%%
% \section{Implementation}
%\iffalse
%<*package>
%\fi
%
% This section describes the definitions file |childdoc.def|.

% The definitions cannot be loaded using |\usepackage| or |\RequirePackage|
% which has a mechanism to prevent loading a style file more than once.
% When loading the definitions by means of |\input|
% multiple instances have to be prevented manually:
%\iffalse
%This code needs to be before the `\ProvidesFile' directive
%which is defined at the beginning of this file.
%Therefore it is also placed there and commented out here.
%</package>
%<*discard>
%\fi
%    \begin{macrocode}
\ifdefined\childdocmain\endinput\fi
%    \end{macrocode}
%\iffalse
%</discard>
%<*package>
%\fi
%
% \macro{\ifchilddoc}
% \macro{\ifchilddocmanual}
% The conditional |\ifchilddoc| tells whether a
% child (true) or main (false) document is being compiled.
% The conditional |\ifchilddocmanual| tells whether
% the |\includeonly| mechanism is used (false) or
% the selection of child files must be performed manually (true).
% The definitions initialise to false:
%    \begin{macrocode}
\newif\ifchilddoc
\newif\ifchilddocmanual
%    \end{macrocode}

% \macro{\childdocname}
% \macro{\childdocjob}
% The macro |\childdocname| stores the name of the main document
% to be compiled. The macro |\childdocjob| stores the name of
% the document on which the \LaTeX{} compiler was originally invoked.
% The content of |\jobname| cannot be compared
% to filenames specified in the source due to different catcodes.
% The following code rescans |\jobname|, stores the result
% in |\childdocname| and saves a copy in |\childdocjob|:
%    \begin{macrocode}
\edef\childdocname{\scantokens\expandafter{\jobname\noexpand}}
\let\childdocjob\childdocname
%    \end{macrocode}

% \macro{\childdocdisable}
% The macro |\childdocdisable| prevents the main file
% from being processed more than once.
% At this stage, the main document command |\childdocmain|
% is assumed to be called once again where it should do nothing.
% Any subsequent call to it should prevent
% a secondary processing of the main document
% It overwrites the forwarding commands
% |\childdocof| and |\childdocforward|
% with empty macros to prevent further inclusions of the main document:
%    \begin{macrocode}
\newcommand{\childdocdisable}
{
  \renewcommand{\childdocmain}[1]{\renewcommand{\childdocmain}[1]{\endinput}}
  \renewcommand{\childdocof}[1]{}
  \renewcommand{\childdocby}[2][]{}
  \renewcommand{\childdocforward}[2][]{}
  \renewcommand{\childdocdisable}{}
}
%    \end{macrocode}

% \macro{\childdocmain}
% The macro |\childdocmain| is to be called at the top of the main file
% with nothing or the main filename (without extension) as argument.
% First, it breaks loops.
% If the argument is not empty and does not match |\childdocname|
% (which is set by the first inclusion of |childdoc.def|),
% |\ifchilddoc| is set to true, |\includeonly| is applied to the child file
% and |\jobname| is set to the main file
% (for proper handling of |.aux| files):
%    \begin{macrocode}
\newcommand{\childdocmain}[1]
{
  \childdocdisable\childdocmain{}
  \if?#1?\else
    \begingroup
      \def\childdoctmp{#1}
      \ifx\childdoctmp\childdocname
        \def\childdoctmp{}
      \else
        \def\childdoctmp
        {
          \childdoctrue
          \includeonly{\childdocname}
          \def\childdocjob{#1}
          \def\jobname{#1}
        }
      \fi
      \expandafter
    \endgroup
    \childdoctmp
  \fi
}
%    \end{macrocode}

% \macro{\childdocof}
% The command |\childdocof| redirects
% compilation to the main file |#1|.
%    \begin{macrocode}
\newcommand{\childdocof}[1]
{
  \childdocdisable
  \childdoctrue
  \includeonly{\childdocname}
  \def\jobname{#1}
  \def\childdocjob{#1}
  \input{#1}
}
%    \end{macrocode}

% \macro{\childdocby}
% The command |\childdocby| ....
%    \begin{macrocode}
\newcommand{\childdocby}[2][]
{
  \childdocdisable
  \childdoctrue
  \childdocmanualtrue
  \if?#1?\else
    \def\jobname{#2}
  \fi
  \def\childdocjob{#2}
  \input{#2}
  \endinput
}
%    \end{macrocode}

% \macro{\childdocforward}
% The command |\childdocforward| redirects
% compilation to the main file or
% (if the optional argument is given) a child file.
% Parameters are set as if the main file
% or a child file starting with |\childdocof| was compiled.
% Then compilation is handed over to the main file:
%    \begin{macrocode}
\newcommand{\childdocforward}[2][]
{
  \begingroup
    \if?#1?
      \def\childdoctmp
      {
        \def\childdocname{#2}
        \def\childdocjob{#2}
        \def\jobname{#2}
        \input{#2}
        \endinput
      }
    \else
      \def\childdoctmp
      {
        \childdocdisable
        \def\childdocname{#2}
        \childdoctrue
        \includeonly{#2}
        \def\childdocjob{#1}
        \def\jobname{#1}
        \input{#1}
        \endinput
      }
    \fi
    \expandafter
  \endgroup
  \childdoctmp
}
%    \end{macrocode}

% \macro{\childdocforwardprefix}
% The command |\childdocforwardprefix| redirects
% compilation to the main or a child file by means of a pattern.
% The prefix |#1| in the current filename is replaced by |#2|
% and the suffix of the current filename is kept
% (it is assumed that the filename does not contain the substring `|~~~|'
% which is used as a delimiter).
% Compilation is handed over to the new file by |\childdocforward|:
%    \begin{macrocode}
\newcommand{\childdocforwardprefix}[3][]
{
  \begingroup
    \def\childdocextract #2##1~~~{\def\childdoctmp{\childdocforward[#1]{#3##1}}}
    \expandafter\childdocextract\childdocname~~~
    \expandafter
  \endgroup
  \childdoctmp
}
%    \end{macrocode}

% \macro{\childdoc}
% The deprecated macro |\childdoc| is a legacy version of |\childdocmain|:
%    \begin{macrocode}
\newcommand{\childdoc}{\childdocmain}
%    \end{macrocode}

% \macro{\childdocredirect}
% The deprecated macro |\childdocredirect| is a legacy version
% of |\childdocforward| and |\childdocforwardprefix|:
%    \begin{macrocode}
\newcommand{\childdocredirect}[2][]
{
  \begingroup
    \if?#1?
      \def\childdoctmp{\childdocforward{#2}}
    \else
      \def\childdoctmp{\childdocforwardprefix{#1}{#2}}
    \fi
    \expandafter
  \endgroup
  \childdoctmp
}
%    \end{macrocode}

%\iffalse
%</package>
%\fi
%
\endinput
|\\
|\childdocby{|\textit{main}|}|\\
\end{tabular}
\end{center}
%
The directive |\childdocby| is similar to |\childdocof|
described in \secref{sec:include},
but the subsequent selection of content must be done manually.
To that end, both |\ifchilddoc| and |\ifchilddocmanual|
will be true upon processing of a part,
and the name of the part is stored in |\childdocname|.
Note that |\jobname| will be set to the filename of the current part
so that each part receives an individual |.aux| file
that does not interfere with the |.aux| file(s) of the main document.
This behaviour can be altered by the alternative form
|\childdocby[*]{|\textit{main}|}| (with a non-empty optional argument)
which uses the |.aux| file of the main document
by setting |\jobname| to \textit{main}.

%%%%%%%%%%%%%%%%%%%%%%%%%%%%%%%%%%%%%%%%%%%%%%%%%%%%%%%%%%%%%%%%%%%%%%%%%%%%%%%%
\subsection{Driver Development}
\label{sec:driver}

The \textsf{childdoc} mechanism can also be use for the development
of definition files such as \LaTeX{} styles or classes.
This case differs from the above setup with multiple parts
included by |\include| in that no |\includeonly| should be invoked.
This can be achieved by starting the include file
(before |\ProvidesPackage|) with:
%
\begin{center}
\begin{tabular}{l}
|% \iffalse
%
% childdoc.dtx Copyright (C) 2017-2018 Niklas Beisert
%
% This work may be distributed and/or modified under the
% conditions of the LaTeX Project Public License, either version 1.3
% of this license or (at your option) any later version.
% The latest version of this license is in
%   http://www.latex-project.org/lppl.txt
% and version 1.3 or later is part of all distributions of LaTeX
% version 2005/12/01 or later.
%
% This work has the LPPL maintenance status `maintained'.
%
% The Current Maintainer of this work is Niklas Beisert.
%
% This work consists of the files childdoc.dtx and childdoc.ins
% and the derived files childdoc.def and cdocsamp.tex with
% cdocsch1.tex, cdocsch2.tex, cdocsdrf.tex, cdocsfn1.tex, cdocsfn2.tex.
%
%<package>\ifdefined\childdocmain\endinput\fi
%<package>\ProvidesFile{childdoc.def}[2018/12/30 v2.0 child document driver]
%<samplemain>\ProvidesFile{cdocsamp.tex}[2018/12/30 v2.0 sample for childdoc]
%<*driver>
%\ProvidesFile{childdoc.drv}[2018/12/30 v2.0 childdoc reference manual file]
\PassOptionsToClass{10pt,a4paper}{article}
\documentclass{ltxdoc}

\usepackage[margin=35mm]{geometry}
\usepackage{hyperref}
\usepackage{hyperxmp}
\usepackage[usenames]{color}

\hypersetup{colorlinks=true}
\hypersetup{pdfstartview=FitH}
\hypersetup{pdfpagemode=UseNone}
\hypersetup{pdfsource={}}
\hypersetup{pdflang={en-UK}}
\hypersetup{pdfcopyright={Copyright 2017-2018 Niklas Beisert.
  This work may be distributed and/or modified under the
  conditions of the LaTeX Project Public License, either version 1.3
  of this license or (at your option) any later version.}}
\hypersetup{pdflicenseurl={http://www.latex-project.org/lppl.txt}}
\hypersetup{pdfcontactaddress={ETH Zurich, ITP, HIT K,
  Wolfgang-Pauli-Strasse 27}}
\hypersetup{pdfcontactpostcode={8093}}
\hypersetup{pdfcontactcity={Zurich}}
\hypersetup{pdfcontactcountry={Switzerland}}
\hypersetup{pdfcontactemail={nbeisert@itp.phys.ethz.ch}}
\hypersetup{pdfcontacturl={http://people.phys.ethz.ch/\xmptilde nbeisert/}}

\newcommand{\secref}[1]{\hyperref[#1]{section \ref*{#1}}}

\parskip1ex
\parindent0pt
\let\olditemize\itemize
\def\itemize{\olditemize\parskip0pt}

\begin{document}

\title{The \textsf{childdoc} Package}
\hypersetup{pdftitle={The childdoc Package}}
\author{Niklas Beisert\\[2ex]
  Institut f\"ur Theoretische Physik\\
  Eidgen\"ossische Technische Hochschule Z\"urich\\
  Wolfgang-Pauli-Strasse 27, 8093 Z\"urich, Switzerland\\[1ex]
  \href{mailto:nbeisert@itp.phys.ethz.ch}
  {\texttt{nbeisert@itp.phys.ethz.ch}}}
\hypersetup{pdfauthor={Niklas Beisert}}
\hypersetup{pdfsubject={Manual for the LaTeX2e Package childdoc}}
\date{30 December 2018, \textsf{v2.0}}
\maketitle

\begin{abstract}\noindent
\textsf{childdoc} is a \LaTeXe{} package
that enables the direct compilation
of document sections included by |\include|
to individual files.
\end{abstract}

\begingroup
\parskip0ex
\tableofcontents
\endgroup

%%%%%%%%%%%%%%%%%%%%%%%%%%%%%%%%%%%%%%%%%%%%%%%%%%%%%%%%%%%%%%%%%%%%%%%%%%%%%%%%
%%%%%%%%%%%%%%%%%%%%%%%%%%%%%%%%%%%%%%%%%%%%%%%%%%%%%%%%%%%%%%%%%%%%%%%%%%%%%%%%
\section{Introduction}

\LaTeX{} provides a mechanism to structure a large document (such as a book)
into a main file and several child files (containing the chapters)
using the |\include| command.
This mechanism is beneficial for documents
which span hundreds of pages in order to
make the source file(s) more manageable.
Moreover, compilation can be restricted to
selected child files by means of the |\includeonly| command.
The latter feature can be used to reduce the compilation time while editing
(this was significantly more useful in the earlier days of \LaTeX{})
or to generate a smaller document which is easier to navigate.
Another application of |\includeonly| is to generate
documents consisting of selected parts of the complete document.

However, there are a few drawbacks of the plain |\include| mechanism:
\begin{itemize}
\item
The child files cannot be compiled on their own,
they can only be compiled via the main file.
A naive editing environment
(such as a text editor with an option
to have the current file processed by \LaTeX)
may require one to switch to the main file before compiling;
attempting to compile the child file produces errors.
\item
The main file must be modified (each time)
to adjust the |\includeonly| command
to the present needs. This easily leaves the main file in a messy state.
\item
The generated document will always carry the filename
of the main document. This is inconvenient if
several child files are to be compiled and
to be kept for distribution.
\end{itemize}

The present package provides a simple interface
to make child files individually compilable by \LaTeX{}.
Compiling a child file then has the same effect as compiling
the main file with an |\includeonly| command
to select the appropriate child.
Moreover the generated document will carry the name of the child
rather than the main file.
This resolves all three above issues.

This feature is meant to make the editing of books,
thesis documents and lecture notes somewhat more convenient.
However, the package can also be used efficiently for
composing a series of documents (such as exercise sheets)
which are typically distributed individually.
It then assists the author in generating the individual documents
(potentially in different versions)
as well as a document containing the collected series.
Another application is in developing style files
or other kinds of included material
where compilation of the style file could redirect
to a sample or test file.

%%%%%%%%%%%%%%%%%%%%%%%%%%%%%%%%%%%%%%%%%%%%%%%%%%%%%%%%%%%%%%%%%%%%%%%%%%%%%%%%
%%%%%%%%%%%%%%%%%%%%%%%%%%%%%%%%%%%%%%%%%%%%%%%%%%%%%%%%%%%%%%%%%%%%%%%%%%%%%%%%
\section{Usage}

First of all, the package \textsf{childdoc} is \emph{not} a standard
\LaTeXe{} |.sty| style file! Therefore it needs to be invoked in
a non-standard way.

%%%%%%%%%%%%%%%%%%%%%%%%%%%%%%%%%%%%%%%%%%%%%%%%%%%%%%%%%%%%%%%%%%%%%%%%%%%%%%%%
\subsection{Included Files}
\label{sec:include}

%%%%%%%%%%%%%%%%%%%%%%%%%%%%%%%%%%%%%%%%
\DescribeMacro{\childdocmain}
To use the package, add the commands
\begin{center}
\begin{tabular}{l}
|\input{childdoc.def}|\\
|\childdocmain{}|\\
\end{tabular}
\end{center}
at the very top of the main \LaTeX{} file,
in particular \emph{before} the |\documentclass| statement!
The argument of |\childdocmain| should be left empty
(but it must be present).

%%%%%%%%%%%%%%%%%%%%%%%%%%%%%%%%%%%%%%%%
\DescribeMacro{\childdocof}
Furthermore, add the commands
\begin{center}
\begin{tabular}{l}
|\input{childdoc.def}|\\
|\childdocof{|\textit{main}|}|\\
\end{tabular}
\end{center}
at the top of every child file \textit{child}
which is included by |\include{|\textit{child}|}|
from within the main file
(or at least for those files to be compiled individually).
The argument \textit{main} must be the filename of the main file.

There are a couple of
considerations in setting up the main and child documents:

%%%%%%%%%%%%%%%%%%%%%%%%%%%%%%%%%%%%%%%%
\paragraph{Restrictions.}

Please note the following restrictions:
\begin{itemize}
\item
|\childdocmain| must be called with one argument \textit{main}
to ensure compatibility with earlier version of the package.
It must either be empty (|\childdocmain{}|)
or precisely match the filename of the main file in which it is specified.
See \secref{sec:detection} for further information.
\item
The filename \textit{main} must be specified without the |.tex| extension.
\item
The filename \textit{main} is case sensitive
(even in case-insensitive file systems)
due to internal string comparison.
\item
The argument \textit{main} should be fully expanded, it cannot be a macro.
\item
Subdirectories and special characters should be avoided in filenames.
\item
The command |\childdocmain{|\textit{main}|}| must be followed by a whitespace.
It should not be followed immediately by another command
or by a comment mark `|%|'.
This is because the \TeX{} parser reads the token immediately following
the argument of |\childdocmain| and puts it
at the beginning of every child section;
however, a white\-space is ignored.
\end{itemize}

%%%%%%%%%%%%%%%%%%%%%%%%%%%%%%%%%%%%%%%%
\paragraph{Content of Main File.}

It is advisable to place all content in the child files included by |\include|.
Any output contained in the main file will appear in all child documents
unless suppressed manually;
it cannot be suppressed automatically by the |\includeonly| directive
and thus should normally be avoided.
A method to include some content in the main file
by means of conditional processing is described in \secref{sec:conditional}.

%%%%%%%%%%%%%%%%%%%%%%%%%%%%%%%%%%%%%%%%
\paragraph{Page Numbering.}

When only a part of the document is compiled,
the appropriate numbering of pages
(as well as other status parameters)
is determined from the |.aux| files.
The latter contain information from previous passes.
However this information needs to propagate through
all intermediate child documents.
Therefore the page numbering in child documents may well
be inconsistent until the complete document is compiled at least once.

A useful (if unconventional) way to always ensure a consistent
page numbering is to restart the numbering in each child document
and denote the pages by `\textit{child}|.|\textit{page}'
where \textit{child} represents the chapter/section number of the child file.
This can be achieved by the command
|\numberwithin{page}{|\textit{child}|}|
of the \textsf{amsmath} package
where \textit{child} can be |chapter| or |section|
depending on the chosen structuring.
Alternatively, one can modify the macro |\thepage| appropriately
and reset the counter |page| at the start of each child file.

%%%%%%%%%%%%%%%%%%%%%%%%%%%%%%%%%%%%%%%%%%%%%%%%%%%%%%%%%%%%%%%%%%%%%%%%%%%%%%%%
\subsection{Conditional Processing}
\label{sec:conditional}

The package provides a mechanism to compile different versions
of a document. To customise the versions further some conditional processing
can come in handy to distinguish which version is being compiled.
The package provides two macros to describe the compilation context:

%%%%%%%%%%%%%%%%%%%%%%%%%%%%%%%%%%%%%%%%
\DescribeMacro{\ifchilddoc}
The conditional |\ifchilddoc| distinguishes between the compilation of
child documents and the main document:
%
\begin{center}
|\ifchilddoc |\textit{child-code}| |[|\||else |\textit{main-code}]| \||fi|
\end{center}

%%%%%%%%%%%%%%%%%%%%%%%%%%%%%%%%%%%%%%%%
\DescribeMacro{\childdocname}
\DescribeMacro{\childdocjob}
The macro |\childdocname| contains the filename (without extension)
of the main or child file being processed.
Note that |\childdocjob| will always contain the name of the main file.

%%%%%%%%%%%%%%%%%%%%%%%%%%%%%%%%%%%%%%%%
\paragraph{Title Page.}

Conditional processing can be used to include a title or banner page
in the main document when proper precautions are taken.
Importantly, the code in the main file should ensure that the page counter
(as well as other status parameters which are stored in the |.aux| files)
takes the same value after the conditional processing.
Otherwise the page numbers may take divergent values
depending on which part is compiled.

For example, a title page could be declared by:
%
\begin{center}
\begin{tabular}{l}
|\ifchilddoc\||else|\\
|\addtocounter{page}{-1}|\\
\textit{code for title page}\\
|\newpage|\\
|\||fi|
\end{tabular}
\end{center}
%
A banner page for the child documents can be generated by:
%
\begin{center}
\begin{tabular}{l}
|\ifchilddoc|\\
|\addtocounter{page}{-1}|\\
\textit{code for banner page}\\
|\newpage|\\
|\||fi|
\end{tabular}
\end{center}
%
Here one could write a message such as:
\begin{center}
|This is the part \childdocname{} of \childdocjob{}.|
\end{center}

%%%%%%%%%%%%%%%%%%%%%%%%%%%%%%%%%%%%%%%%%%%%%%%%%%%%%%%%%%%%%%%%%%%%%%%%%%%%%%%%
\subsection{Flags}
\label{sec:flags}

The package makes it easy to generate different versions
of the main or child documents.
To this end compilation flags can be defined
and assigned different default values.
They will be particularly useful in conjunction
with the forwarding mechanism described in \secref{sec:forward}.

For example, it may be useful to have a flag |\version|
which can be set to |draft| or |final|.
The document source will contain some conditional code
depending on the value of |\version|.
Suppose further, the flag should default to |final| for the main file
and to |draft| for child files
which is a natural assignment for editing the document.
This is achieved by placing the following code
in the preamble of the main document
(below the |\childdocmain| directive):
%
\begin{center}
\begin{tabular}{l}
|\ifchilddoc|\\
|\providecommand{\version}{draft}|\\
|\||else|\\
|\providecommand{\version}{final}|\\
|\||fi|
\end{tabular}
\end{center}
%
The definition by |\providecommand| makes sure
that previous definitions are not overwritten.
Further statements |\providecommand{\version}{...}|
can thus be added before the above code to override it.

For the main file, one might add a line
(between |\childdocmain| and the above block)
%
\begin{center}
|%\ifchilddoc\||else\providecommand{\version}{draft}\||fi|
\end{center}
%
which can be uncommented to produce a draft version.
Likewise one can add a line to the very top of a child file
(above the |\childdocof{|\textit{main}|}| directive)
%
\begin{center}
|%\providecommand{\version}{final}|
\end{center}
%
which can be uncommented to produce the final version of this child document.

%%%%%%%%%%%%%%%%%%%%%%%%%%%%%%%%%%%%%%%%%%%%%%%%%%%%%%%%%%%%%%%%%%%%%%%%%%%%%%%%
\subsection{Forwarding}
\label{sec:forward}

Different versions of the main or child documents
using compilation flags as described in \secref{sec:flags}
can be (permanently) stored in different files
for convenient compilation, viewing and distribution.
To this end, the package defines a command
to pass on compilation to a different file:

%%%%%%%%%%%%%%%%%%%%%%%%%%%%%%%%%%%%%%%%
\DescribeMacro{\childdocforward}
The command |\childdocforward| redirects processing to
another source file:
%
\begin{center}
\begin{tabular}{l}
|\input{childdoc.def}|\\
|\childdocforward[|\textit{main}|]{|\textit{dest}|}|\\
\end{tabular}
\end{center}
%
The argument \textit{dest} is the destination file
(without extension).
It should be the main file or one of the child files.
Note that further \textsf{childdoc} directives
such as |\childdocof| and |\childdocforward|
in the indicated file will be processed in this form.
The optional argument \textit{main}
passes on directly to the main file \textit{main}
while pretending to compile the child \textit{dest}.
This form behaves as if \textit{dest}
issues |\childdocof{|\textit{main}|}| right away,
and no further \textsf{childdoc} directives will be processed.

%%%%%%%%%%%%%%%%%%%%%%%%%%%%%%%%%%%%%%%%
\DescribeMacro{\...prefix}
In the alternative form |\childdocforwardprefix|,
%
\begin{center}
\begin{tabular}{l}
|\input{childdoc.def}|\\
|\childdocforwardprefix[|\textit{main}|]{|\textit{prefix}|}{|\textit{dest}|}|
\end{tabular}
\end{center}
%
the destination file is determined by a pattern
depending on the current file:
To make this work, the current file must be called
`{\textit{prefix}\hspace{0.2em}\textit{suffix}}'
with \textit{prefix} matching precisely the argument.
Processing is then passed on to the file
`{\textit{dest}\hspace{0.2em}\textit{suffix}}'.
Surely, the same effect is achieved by
directly specifying the
argument `{\textit{dest}\hspace{0.2em}\textit{suffix}}'
in the first form.
However, that requires to set up a different file
for each child. With the alternative form of the command
all these files can have exactly the same content
which simplifies setting them up and maintaining them.

For example, the following file |draft.tex|
with a compilation flag |\version| as described in \secref{sec:flags}
compiles the main document as a draft:
%
\begin{center}
\begin{tabular}{l}
|\def\version{draft}|\\
|\input{childdoc.def}|\\
|\childdocforward{|\textit{main}|}|
\end{tabular}
\end{center}
%
Likewise, the following files |final|\textit{nn}|.tex|
compile the final version of the child document
|child|\textit{nn}|.tex|:
%
\begin{center}
\begin{tabular}{l}
|\def\version{final}|\\
|\input{childdoc.def}|\\
|\childdocforwardprefix{final}{child}|
\end{tabular}
\end{center}
%

Note that when several versions of a main file and/or of each child file
are to be generated, it may be convenient to set up a |Makefile| or
shell script to automatise the process.

%%%%%%%%%%%%%%%%%%%%%%%%%%%%%%%%%%%%%%%%%%%%%%%%%%%%%%%%%%%%%%%%%%%%%%%%%%%%%%%%
\subsection{Command Line Processing}
\label{sec:commandline}

The effect of redirection files can also be achieved by invoking
the \LaTeX{} compiler with a more elaborate command line.
Most conveniently this should be done as part
of a shell script or a |Makefile|.

When using \textsf{childdoc} in the main file, the following
command lines effectively perform a redirection
(note that depending on the shell being used,
backslashes may have to be doubled: `|\|' $\to$ `|\\|'):
%
\begin{center}
|... -jobname "|\textit{target}|" |\\|"|[\textit{flags}]%
|\input{childdoc.def}\childdocforward[|\textit{main}|]{|\textit{dest}|}"|
\end{center}
%
Here \textit{target} is the name of the output file,
\textit{main} is the name of the main file
and \textit{dest} is the name of the main or child file to be processed
(all filenames without extensions).
The optional argument \textit{main} can be omitted
if \textit{main} matches \textit{dest}.
Optionally, compilation \textit{flags} can be defined via |\def| commands.
This command line makes the \TeX{} engine believe
it is compiling the file \textit{target}
whose content is specified as the latter parameter.
The provided code then forwards the processing to
\textit{main} or \textit{dest} as described in \secref{sec:forward}.

%%%%%%%%%%%%%%%%%%%%%%%%%%%%%%%%%%%%%%%%%%%%%%%%%%%%%%%%%%%%%%%%%%%%%%%%%%%%%%%%
\subsection{Include by Input}
\label{sec:input}

Including child documents by |\include| has some restrictions by design.
Most notably, the content of a child document always occupies
its own set of pages; pages cannot be shared between child documents.
Usually, this behaviour makes perfect sense
because each child document contain an essential part of the document.
However, in some situations it may be desirable to compose
a document from a collection of parts
without having mandatory page breaks between then.
For this case, the package
provides a mechanism to include parts
by |\input| which can also be processed individually.
However, by construction this mechanism
requires manual handling of the content to be output.

%%%%%%%%%%%%%%%%%%%%%%%%%%%%%%%%%%%%%%%%
\DescribeMacro{\ifchilddocmanual}
The main file should be prepared as usual, see \secref{sec:include}.
However, the document body must make a distinction
between processing of an individual part and of the main document, e.g.:
%
\begin{center}
\begin{tabular}{l}
|\ifchilddocmanual|\\
|\input{\childdocname}|\\
|\||else|\\
\textit{document body with }|\input{|\textit{part}|}|\\
|\||fi|
\end{tabular}
\end{center}
%
The conditional |\ifchilddocmanual| is true whenever
a part to be included by |\input| is being compiled,
and the name of the part is stored in |\childdocname|.

%%%%%%%%%%%%%%%%%%%%%%%%%%%%%%%%%%%%%%%%
\DescribeMacro{\childdocby}
Each part to be included by |\input| should start with:
%
\begin{center}
\begin{tabular}{l}
|\input{childdoc.def}|\\
|\childdocby{|\textit{main}|}|\\
\end{tabular}
\end{center}
%
The directive |\childdocby| is similar to |\childdocof|
described in \secref{sec:include},
but the subsequent selection of content must be done manually.
To that end, both |\ifchilddoc| and |\ifchilddocmanual|
will be true upon processing of a part,
and the name of the part is stored in |\childdocname|.
Note that |\jobname| will be set to the filename of the current part
so that each part receives an individual |.aux| file
that does not interfere with the |.aux| file(s) of the main document.
This behaviour can be altered by the alternative form
|\childdocby[*]{|\textit{main}|}| (with a non-empty optional argument)
which uses the |.aux| file of the main document
by setting |\jobname| to \textit{main}.

%%%%%%%%%%%%%%%%%%%%%%%%%%%%%%%%%%%%%%%%%%%%%%%%%%%%%%%%%%%%%%%%%%%%%%%%%%%%%%%%
\subsection{Driver Development}
\label{sec:driver}

The \textsf{childdoc} mechanism can also be use for the development
of definition files such as \LaTeX{} styles or classes.
This case differs from the above setup with multiple parts
included by |\include| in that no |\includeonly| should be invoked.
This can be achieved by starting the include file
(before |\ProvidesPackage|) with:
%
\begin{center}
\begin{tabular}{l}
|\input{childdoc.def}|\\
|\childdocforward{|\textit{main}|}|\\
\end{tabular}
\end{center}
%
or alternatively with:
%
\begin{center}
\begin{tabular}{l}
|\input{childdoc.def}|\\
|\childdocby{|\textit{main}|}|\\
\end{tabular}
\end{center}
%
Both forms have slightly different effects as described above.
The main file is prepared as usual, see \secref{sec:include}.

%%%%%%%%%%%%%%%%%%%%%%%%%%%%%%%%%%%%%%%%%%%%%%%%%%%%%%%%%%%%%%%%%%%%%%%%%%%%%%%%
\subsection{Legacy Detection}
\label{sec:detection}

The directive |\childdocmain| in the main file can detect
whether the complete document or merely a child is to be compiled
even without using the directive |\childdocof|.
This method is deprecated because it is less robust
and there is no compelling reason to use it;
it is merely provided for backward compatibility
and it may be removed in future versions.

If the detection mechanism is to be used,
it is mandatory to correctly specify
the filename of the main file as the argument of |\childdocmain|:
%
\begin{center}
\begin{tabular}{l}
|\input{childdoc.def}|\\
|\childdocmain{|\textit{main}|}|\\
\end{tabular}
\end{center}
%
If |\jobname| does not match the argument \textit{main} of |\childdocmain|,
it is assumed that |\jobname| points to the child file to be compiled.
When using |\childdocmain| with the main file specified as argument,
it suffices to start a child file
with just |\input{|\textit{main}|}|
without loading of the package and using |\childdocof|.
If instead all processing is done
with the appropriate \textsf{childdoc} directives,
the argument of \textit{main} of |\childdocmain| can be empty.

An alternative version of the command line processing described
in \secref{sec:commandline} using the detection mechanism reads:
%
\begin{center}
|... -jobname "|\textit{target}|" "|[\textit{flags}]%
[|\def\jobname{|\textit{dest}|}|]|\input{|\textit{main}|}"|
\end{center}

%%%%%%%%%%%%%%%%%%%%%%%%%%%%%%%%%%%%%%%%%%%%%%%%%%%%%%%%%%%%%%%%%%%%%%%%%%%%%%%%
\subsection{Manual Code}
\label{sec:manual}

In case one cannot be certain whether the definitions file |childdoc.def|
is installed on the target \TeX{} distribution
and one prefers not to ship it,
it is conceivable to paste a few relevant commands into the sources.

To that end, drop all statements |\input{childdoc.def}|
and perform the replacements as outlined below.
Instead of |\childdocmain{|\textit{main}|}| add the following code
to the top of the main file:
%
\begin{center}
\begin{tabular}{l}
|\||ifdefined\childdocname\endinput\||fi\newif\ifchilddoc|\\
|\edef\childdocname{\scantokens\expandafter{\jobname\noexpand}}|\\
|\def\childdocmain{|\textit{main}|}\||ifx\childdocmain\childdocname\||else|\\
|\childdoctrue\includeonly{\childdocname}\let\jobname\childdocmain\||fi|\\
\end{tabular}
\end{center}
%
Instead of |\childdocof{|\textit{main}|}| just include the main file
at the top of each child file:
%
\begin{center}
|\input{|\textit{main}|}|
\end{center}
%
A simple redirection |\childdocforward{|\textit{dest}|}| is achieved by:
%
\begin{center}
|\def\jobname{|\textit{dest}|}\input{\jobname}|
\end{center}
%
The redirection with prefix
|\childdocforwardprefix[|\textit{prefix}|]{|\textit{dest}|}|
is accomplished by:
%
\begin{center}
\begin{tabular}{l}
|{\edef\jobname{\scantokens\expandafter{\jobname\noexpand}}|\\
|\def\redirectjob |\textit{prefix}|#1~~~{\gdef\jobname{|\textit{dest}|#1}}|\\
|\expandafter\redirectjob\jobname~~~}\input{\jobname}|
\end{tabular}
\end{center}

In an alternative approach,
child documents can be compiled by a specific command line
without additional code or specific definitions:
%
\begin{center}
|... -jobname "|\textit{target}|" "|[\textit{flags}]%
|\includeonly{|\textit{dest}|}\input{|\textit{main}|}"|
\end{center}
%

%%%%%%%%%%%%%%%%%%%%%%%%%%%%%%%%%%%%%%%%%%%%%%%%%%%%%%%%%%%%%%%%%%%%%%%%%%%%%%%%
%%%%%%%%%%%%%%%%%%%%%%%%%%%%%%%%%%%%%%%%%%%%%%%%%%%%%%%%%%%%%%%%%%%%%%%%%%%%%%%%
\section{Information}

%%%%%%%%%%%%%%%%%%%%%%%%%%%%%%%%%%%%%%%%%%%%%%%%%%%%%%%%%%%%%%%%%%%%%%%%%%%%%%%%
\subsection{Copyright}

Copyright \copyright{} 2017--2018 Niklas Beisert

This work may be distributed and/or modified under the
conditions of the \LaTeX{} Project Public License, either version 1.3
of this license or (at your option) any later version.
The latest version of this license is in
  \url{http://www.latex-project.org/lppl.txt}
and version 1.3 or later is part of all distributions of \LaTeX{}
version 2005/12/01 or later.

This work has the LPPL maintenance status `maintained'.

The Current Maintainer of this work is Niklas Beisert.

This work consists of the files |README.txt|, |childdoc.ins| and |childdoc.dtx|
as well as the derived files |childdoc.def|, |cdocsamp.tex|
with |cdocsch1.tex|, |cdocsch2.tex|, |cdocspt3.tex|, |cdocspt4.tex|,
|cdocsdrf.tex|, |cdocsfn1.tex|, |cdocsfn2.tex|
as well as |childdoc.pdf|.

%%%%%%%%%%%%%%%%%%%%%%%%%%%%%%%%%%%%%%%%%%%%%%%%%%%%%%%%%%%%%%%%%%%%%%%%%%%%%%%%
\subsection{Files and Installation}

The package consists of the files:
%
\begin{center}
\begin{tabular}{ll}
    |README.txt|   & readme file \\
    |childdoc.ins| & installation file \\
    |childdoc.dtx| & source file \\
    |childdoc.def| & definition file \\
    |cdocsamp.tex| & sample main file \\
    |cdocsch1.tex| & sample include file \\
    |cdocsch2.tex| & sample include file \\
    |cdocspt3.tex| & sample part file \\
    |cdocspt4.tex| & sample part file \\
    |cdocsdrf.tex| & sample redirection file \\
    |cdocsfn1.tex| & sample redirection file \\
    |cdocsfn2.tex| & sample redirection file \\
    |childdoc.pdf| & manual
\end{tabular}
\end{center}
%
The distribution consists of the files
|README.txt|, |childdoc.ins| and |childdoc.dtx|.
%
\begin{itemize}
\item
Run (pdf)\LaTeX{} on |childdoc.dtx|
to compile the manual |childdoc.pdf| (this file).
\item
Run \LaTeX{} on |childdoc.ins| to create the definitions file |childdoc.def|
and the sample |cdocsamp.tex| with include files
|cdocsch1.tex|, |cdocsch2.tex|, |cdocspt3.tex|, |cdocspt4.tex|,
|cdocsdrf.tex|, |cdocsfn1.tex|, |cdocsfn2.tex|.
Then copy the file |childdoc.def| to an appropriate directory of your \LaTeX{}
distribution, e.g.\ \textit{texmf-root}|/tex/latex/childdoc|.
\end{itemize}

%%%%%%%%%%%%%%%%%%%%%%%%%%%%%%%%%%%%%%%%%%%%%%%%%%%%%%%%%%%%%%%%%%%%%%%%%%%%%%%%
\subsection{Related CTAN Packages}

There are several other packages which offer a similar functionality:
%
\begin{itemize}
\item
The packages
\href{http://ctan.org/pkg/docmute}{\textsf{docmute}},
\href{http://ctan.org/pkg/includex}{\textsf{includex}} and
\href{http://ctan.org/pkg/standalone}{\textsf{standalone}}
provide commands to include only the document body of
a child file thus allowing both files to be compiled individually.
\item
The packages \href{http://ctan.org/pkg/subdocs}{\textsf{subdocs}}
and \href{http://ctan.org/pkg/subfiles}{\textsf{subfiles}}
provide structures in which the main and child documents can be
encapsulated and allowing them to be compiled individually.
The inclusion mechanism is different from the conventional |\include|.
\item
The package \href{http://ctan.org/pkg/combine}{\textsf{combine}}
is an elaborate solution to combine several documents into one.
\end{itemize}
%
See also the CTAN topic \href{http://ctan.org/topic/subdocs}{\textsf{subdocs}}
for further related packages.
The present package differs from the above solutions in that
a document structure constructed with the conventional |\include| mechanism
just needs two extra commands at the top of every file
such that all constituent files can be compiled individually.

%%%%%%%%%%%%%%%%%%%%%%%%%%%%%%%%%%%%%%%%%%%%%%%%%%%%%%%%%%%%%%%%%%%%%%%%%%%%%%%%
%\subsection{Feature Suggestions}
%
%The following is a list of features which may be useful for future
%versions of this package:
%%
%\begin{itemize}
%\item
%\ldots
%\end{itemize}

%%%%%%%%%%%%%%%%%%%%%%%%%%%%%%%%%%%%%%%%%%%%%%%%%%%%%%%%%%%%%%%%%%%%%%%%%%%%%%%%
\subsection{Revision History}

%%%%%%%%%%%%%%%%%%%%%%%%%%%%%%%%%%%%%%%%
\paragraph{v2.0:} 2018/12/30

\begin{itemize}
\item
immediate forward processing
\item
added |\childdocby| mechanism
\item
manual restructured
\end{itemize}

%%%%%%%%%%%%%%%%%%%%%%%%%%%%%%%%%%%%%%%%
\paragraph{v1.6:} 2018/01/17

\begin{itemize}
\item
application for development of include files
\item
corrections to manual
\end{itemize}

%%%%%%%%%%%%%%%%%%%%%%%%%%%%%%%%%%%%%%%%
\paragraph{v1.5:} 2017/05/21

\begin{itemize}
\item
more complete structuring introduced
\item
|\childdocof| introduced
\item
|\childdoc| renamed to |\childdocmain|
\item
|\childredirect| renamed to |\childdocforward| and |\childdocforwardprefix|
and functionality expanded
\end{itemize}

%%%%%%%%%%%%%%%%%%%%%%%%%%%%%%%%%%%%%%%%
\paragraph{v1.0:} 2017/04/27

\begin{itemize}
\item
manual and install package
\item
first version published on CTAN
\end{itemize}

%%%%%%%%%%%%%%%%%%%%%%%%%%%%%%%%%%%%%%%%
\paragraph{v0.6:} 2017/04/26

\begin{itemize}
\item
redirection mechanism added
\end{itemize}

%%%%%%%%%%%%%%%%%%%%%%%%%%%%%%%%%%%%%%%%
\paragraph{v0.5:} 2017/04/26

\begin{itemize}
\item
functionality in definition file
\end{itemize}


%%%%%%%%%%%%%%%%%%%%%%%%%%%%%%%%%%%%%%%%%%%%%%%%%%%%%%%%%%%%%%%%%%%%%%%%%%%%%%%%
%%%%%%%%%%%%%%%%%%%%%%%%%%%%%%%%%%%%%%%%%%%%%%%%%%%%%%%%%%%%%%%%%%%%%%%%%%%%%%%%
%%%%%%%%%%%%%%%%%%%%%%%%%%%%%%%%%%%%%%%%%%%%%%%%%%%%%%%%%%%%%%%%%%%%%%%%%%%%%%%%
\appendix

\settowidth\MacroIndent{\rmfamily\scriptsize 000\ }

 \DocInput{childdoc.dtx}

\end{document}
%</driver>
% \fi
%
% %%%%%%%%%%%%%%%%%%%%%%%%%%%%%%%%%%%%%%%%%%%%%%%%%%%%%%%%%%%%%%%%%%%%%%%%%%%%%%
% %%%%%%%%%%%%%%%%%%%%%%%%%%%%%%%%%%%%%%%%%%%%%%%%%%%%%%%%%%%%%%%%%%%%%%%%%%%%%%
% \section{Sample}
%\iffalse
%<*samplemain>
%\fi
%
% The following presents a sample document
% with two chapters, two parts, a title page,
% a compile flag as well as three forwarding files to set the flag.
% It consists of eight |.tex| files:
% \begin{center}
% \begin{tabular}{ll}
% |cdocsamp.tex|&main file\\
% |cdocsch1.tex|&include file for chapter 1\\
% |cdocsch2.tex|&include file for chapter 2\\
% |cdocspt3.tex|&include file for part 3\\
% |cdocspt4.tex|&include file for part 4\\
% |cdocsdrf.tex|&forwarding file for main file in draft mode\\
% |cdocsfi1.tex|&forwarding file for final version of chapter 1\\
% |cdocsfi2.tex|&forwarding file for final version of chapter 2\\
% \end{tabular}
% \end{center}
% Each of the eight files can be compiled directly by the \LaTeX{} compiler.
%
% %%%%%%%%%%%%%%%%%%%%%%%%%%%%%%%%%%%%%%
% \paragraph{Main File.}
%
% The main file is called |cdocsamp.tex|.
%
% Load the \textsf{childdoc} definitions and
% declare the filename for the main document:
%    \begin{macrocode}
\input{childdoc.def}
\childdocmain{}
%    \end{macrocode}

% Optional override for |\version| flag:
%    \begin{macrocode}
%%\ifchilddoc\else\providecommand{\version}{draft}\fi
%    \end{macrocode}

% Define the default values for the |\version| flag
% (|final| for the main file and |draft| for childs):
%    \begin{macrocode}
\ifchilddoc
\providecommand{\version}{draft}
\else
\providecommand{\version}{final}
\fi
%    \end{macrocode}

% Load the standard document class:
%    \begin{macrocode}
\documentclass[12pt]{article}
%    \end{macrocode}

% Start the document body:
%    \begin{macrocode}
\begin{document}
%    \end{macrocode}

% Declare a title page.
% Print title, part of document being processed and version flag:
%    \begin{macrocode}
\addtocounter{page}{-1}
\begin{center}
{\LARGE\bfseries{}childdoc example\par}
\vspace{1cm}
\ifchilddoc
\ifchilddocmanual part\else chapter\fi:
`\childdocname' of `\childdocjob'\par
\else
main document: `\childdocjob'\par
\fi
version: \version\par
\end{center}
\newpage
%    \end{macrocode}

% Manually include selected file,
% otherwise process as usual:
%    \begin{macrocode}
\ifchilddocmanual
\section*{part `\childdocname'}
\input{\childdocname}
\else
%    \end{macrocode}

% Include the two chapters:
%    \begin{macrocode}
\include{cdocsch1}
\include{cdocsch2}
%    \end{macrocode}

% Include the two parts unless only chapters should be displayed:
%    \begin{macrocode}
\ifchilddoc\else
\section{part three}
\input{cdocspt3}
\section{part four}
\input{cdocspt4}
\fi
%    \end{macrocode}

% Process as usual until here:
%    \begin{macrocode}
\fi
%    \end{macrocode}

% End of document body:
%    \begin{macrocode}
\end{document}
%    \end{macrocode}
%\iffalse
%</samplemain>
%\fi
%
% %%%%%%%%%%%%%%%%%%%%%%%%%%%%%%%%%%%%%%
% \paragraph{Chapter Include Files.}
%
% The include files are called |cdocsch1.tex| and |cdocsch2.tex|.
%
%\iffalse
%<*samplechap1|samplechap2>
%\fi

% Optional override for |\version| flag:
%    \begin{macrocode}
%%\providecommand{\version}{final}
%    \end{macrocode}

% Include the main document:
%    \begin{macrocode}
\input{childdoc.def}
\childdocof{cdocsamp}
%    \end{macrocode}

%\iffalse
%</samplechap1|samplechap2>
%\fi
%
%\iffalse
%<*samplechap1>
%\fi
% Some text for chapter 1:
%    \begin{macrocode}
\section{one}
some text in chapter one
%    \end{macrocode}

%\iffalse
%</samplechap1>
%\fi
% Some text for chapter 2:
%\iffalse
%<*samplechap2>
%\fi
%    \begin{macrocode}
\section{two}
more text in chapter two
%    \end{macrocode}

%\iffalse
%</samplechap2>
%\fi
%
% %%%%%%%%%%%%%%%%%%%%%%%%%%%%%%%%%%%%%%
% \paragraph{Part Include Files.}
%
% The include files are called |cdocspt3.tex| and |cdocspt4.tex|.
%
%\iffalse
%<*samplepart3|samplepart4>
%\fi

% Optional override for |\version| flag:
%    \begin{macrocode}
%%\providecommand{\version}{final}
%    \end{macrocode}

% Include the main document:
%    \begin{macrocode}
\input{childdoc.def}
\childdocby{cdocsamp}
%    \end{macrocode}

%\iffalse
%</samplepart3|samplepart4>
%\fi
%
%\iffalse
%<*samplepart3>
%\fi
% Some text for part 3:
%    \begin{macrocode}
some text in part three
%    \end{macrocode}

%\iffalse
%</samplepart3>
%\fi
% Some text for part 4:
%\iffalse
%<*samplepart4>
%\fi
%    \begin{macrocode}
more text in part four
%    \end{macrocode}

%\iffalse
%</samplepart4>
%\fi
%
% %%%%%%%%%%%%%%%%%%%%%%%%%%%%%%%%%%%%%%
% \paragraph{Forwarding for a Complete Draft.}
%
% The following forwarding file |cdocsdrf.tex|
% compiles the main document in draft mode:
%\iffalse
%<*sampledraft>
%\fi
%    \begin{macrocode}
\def\version{draft}
\input{childdoc.def}
\childdocforward{cdocsamp}
%    \end{macrocode}

%\iffalse
%</sampledraft>
%\fi
%
% %%%%%%%%%%%%%%%%%%%%%%%%%%%%%%%%%%%%%%
% \paragraph{Forwarding for Final Version of the Chapters.}
%
% The following forwarding files |cdocsfn1.tex| and |cdocsfn2.tex|
% (with identical content)
% compile the final versions of the child documents
% |cdocsch1.tex| and |cdocsch2.tex|, respectively:
%\iffalse
%<*samplefinal>
%\fi
%    \begin{macrocode}
\def\version{final}
\input{childdoc.def}
\childdocforwardprefix[cdocsamp]{cdocsfn}{cdocsch}
%    \end{macrocode}

%\iffalse
%</samplefinal>
%\fi
%
% %%%%%%%%%%%%%%%%%%%%%%%%%%%%%%%%%%%%%%
% \paragraph{Command Line Processing.}
%
% The following three command lines generate the output files
% |cdocscld|, |cdocscl1| and |cdocscl2|
% which should be identical to
% |cdocsdrf|, |cdocsch1| and |cdocsfn2|, respectively:
% \begin{center}
% \begin{tabular}{l}
% |latex -jobname cdocscld \|\\
% |  "\def\version{draft}\input{childdoc.def}\childdocforward{cdocsamp}"|\\
% |latex -jobname cdocscl1 \|\\
% |  "\input{childdoc.def}\childdocforward[cdocsamp]{cdocsch1}"|\\
% |latex -jobname cdocscl2 \|\\
% |  "\def\version{final}\input{childdoc.def}\childdocforward{cdocsch2}"|
% \end{tabular}
% \end{center}
% Note that the trailing backslash on each first line
% merely continues the input to the second line
% (for convenient cut ant paste).
% Furthermore, the command |latex| can be replaced by any
% of its alternative versions such as |pdflatex|.
%
% %%%%%%%%%%%%%%%%%%%%%%%%%%%%%%%%%%%%%%%%%%%%%%%%%%%%%%%%%%%%%%%%%%%%%%%%%%%%%%
% %%%%%%%%%%%%%%%%%%%%%%%%%%%%%%%%%%%%%%%%%%%%%%%%%%%%%%%%%%%%%%%%%%%%%%%%%%%%%%
% \section{Implementation}
%\iffalse
%<*package>
%\fi
%
% This section describes the definitions file |childdoc.def|.

% The definitions cannot be loaded using |\usepackage| or |\RequirePackage|
% which has a mechanism to prevent loading a style file more than once.
% When loading the definitions by means of |\input|
% multiple instances have to be prevented manually:
%\iffalse
%This code needs to be before the `\ProvidesFile' directive
%which is defined at the beginning of this file.
%Therefore it is also placed there and commented out here.
%</package>
%<*discard>
%\fi
%    \begin{macrocode}
\ifdefined\childdocmain\endinput\fi
%    \end{macrocode}
%\iffalse
%</discard>
%<*package>
%\fi
%
% \macro{\ifchilddoc}
% \macro{\ifchilddocmanual}
% The conditional |\ifchilddoc| tells whether a
% child (true) or main (false) document is being compiled.
% The conditional |\ifchilddocmanual| tells whether
% the |\includeonly| mechanism is used (false) or
% the selection of child files must be performed manually (true).
% The definitions initialise to false:
%    \begin{macrocode}
\newif\ifchilddoc
\newif\ifchilddocmanual
%    \end{macrocode}

% \macro{\childdocname}
% \macro{\childdocjob}
% The macro |\childdocname| stores the name of the main document
% to be compiled. The macro |\childdocjob| stores the name of
% the document on which the \LaTeX{} compiler was originally invoked.
% The content of |\jobname| cannot be compared
% to filenames specified in the source due to different catcodes.
% The following code rescans |\jobname|, stores the result
% in |\childdocname| and saves a copy in |\childdocjob|:
%    \begin{macrocode}
\edef\childdocname{\scantokens\expandafter{\jobname\noexpand}}
\let\childdocjob\childdocname
%    \end{macrocode}

% \macro{\childdocdisable}
% The macro |\childdocdisable| prevents the main file
% from being processed more than once.
% At this stage, the main document command |\childdocmain|
% is assumed to be called once again where it should do nothing.
% Any subsequent call to it should prevent
% a secondary processing of the main document
% It overwrites the forwarding commands
% |\childdocof| and |\childdocforward|
% with empty macros to prevent further inclusions of the main document:
%    \begin{macrocode}
\newcommand{\childdocdisable}
{
  \renewcommand{\childdocmain}[1]{\renewcommand{\childdocmain}[1]{\endinput}}
  \renewcommand{\childdocof}[1]{}
  \renewcommand{\childdocby}[2][]{}
  \renewcommand{\childdocforward}[2][]{}
  \renewcommand{\childdocdisable}{}
}
%    \end{macrocode}

% \macro{\childdocmain}
% The macro |\childdocmain| is to be called at the top of the main file
% with nothing or the main filename (without extension) as argument.
% First, it breaks loops.
% If the argument is not empty and does not match |\childdocname|
% (which is set by the first inclusion of |childdoc.def|),
% |\ifchilddoc| is set to true, |\includeonly| is applied to the child file
% and |\jobname| is set to the main file
% (for proper handling of |.aux| files):
%    \begin{macrocode}
\newcommand{\childdocmain}[1]
{
  \childdocdisable\childdocmain{}
  \if?#1?\else
    \begingroup
      \def\childdoctmp{#1}
      \ifx\childdoctmp\childdocname
        \def\childdoctmp{}
      \else
        \def\childdoctmp
        {
          \childdoctrue
          \includeonly{\childdocname}
          \def\childdocjob{#1}
          \def\jobname{#1}
        }
      \fi
      \expandafter
    \endgroup
    \childdoctmp
  \fi
}
%    \end{macrocode}

% \macro{\childdocof}
% The command |\childdocof| redirects
% compilation to the main file |#1|.
%    \begin{macrocode}
\newcommand{\childdocof}[1]
{
  \childdocdisable
  \childdoctrue
  \includeonly{\childdocname}
  \def\jobname{#1}
  \def\childdocjob{#1}
  \input{#1}
}
%    \end{macrocode}

% \macro{\childdocby}
% The command |\childdocby| ....
%    \begin{macrocode}
\newcommand{\childdocby}[2][]
{
  \childdocdisable
  \childdoctrue
  \childdocmanualtrue
  \if?#1?\else
    \def\jobname{#2}
  \fi
  \def\childdocjob{#2}
  \input{#2}
  \endinput
}
%    \end{macrocode}

% \macro{\childdocforward}
% The command |\childdocforward| redirects
% compilation to the main file or
% (if the optional argument is given) a child file.
% Parameters are set as if the main file
% or a child file starting with |\childdocof| was compiled.
% Then compilation is handed over to the main file:
%    \begin{macrocode}
\newcommand{\childdocforward}[2][]
{
  \begingroup
    \if?#1?
      \def\childdoctmp
      {
        \def\childdocname{#2}
        \def\childdocjob{#2}
        \def\jobname{#2}
        \input{#2}
        \endinput
      }
    \else
      \def\childdoctmp
      {
        \childdocdisable
        \def\childdocname{#2}
        \childdoctrue
        \includeonly{#2}
        \def\childdocjob{#1}
        \def\jobname{#1}
        \input{#1}
        \endinput
      }
    \fi
    \expandafter
  \endgroup
  \childdoctmp
}
%    \end{macrocode}

% \macro{\childdocforwardprefix}
% The command |\childdocforwardprefix| redirects
% compilation to the main or a child file by means of a pattern.
% The prefix |#1| in the current filename is replaced by |#2|
% and the suffix of the current filename is kept
% (it is assumed that the filename does not contain the substring `|~~~|'
% which is used as a delimiter).
% Compilation is handed over to the new file by |\childdocforward|:
%    \begin{macrocode}
\newcommand{\childdocforwardprefix}[3][]
{
  \begingroup
    \def\childdocextract #2##1~~~{\def\childdoctmp{\childdocforward[#1]{#3##1}}}
    \expandafter\childdocextract\childdocname~~~
    \expandafter
  \endgroup
  \childdoctmp
}
%    \end{macrocode}

% \macro{\childdoc}
% The deprecated macro |\childdoc| is a legacy version of |\childdocmain|:
%    \begin{macrocode}
\newcommand{\childdoc}{\childdocmain}
%    \end{macrocode}

% \macro{\childdocredirect}
% The deprecated macro |\childdocredirect| is a legacy version
% of |\childdocforward| and |\childdocforwardprefix|:
%    \begin{macrocode}
\newcommand{\childdocredirect}[2][]
{
  \begingroup
    \if?#1?
      \def\childdoctmp{\childdocforward{#2}}
    \else
      \def\childdoctmp{\childdocforwardprefix{#1}{#2}}
    \fi
    \expandafter
  \endgroup
  \childdoctmp
}
%    \end{macrocode}

%\iffalse
%</package>
%\fi
%
\endinput
|\\
|\childdocforward{|\textit{main}|}|\\
\end{tabular}
\end{center}
%
or alternatively with:
%
\begin{center}
\begin{tabular}{l}
|% \iffalse
%
% childdoc.dtx Copyright (C) 2017-2018 Niklas Beisert
%
% This work may be distributed and/or modified under the
% conditions of the LaTeX Project Public License, either version 1.3
% of this license or (at your option) any later version.
% The latest version of this license is in
%   http://www.latex-project.org/lppl.txt
% and version 1.3 or later is part of all distributions of LaTeX
% version 2005/12/01 or later.
%
% This work has the LPPL maintenance status `maintained'.
%
% The Current Maintainer of this work is Niklas Beisert.
%
% This work consists of the files childdoc.dtx and childdoc.ins
% and the derived files childdoc.def and cdocsamp.tex with
% cdocsch1.tex, cdocsch2.tex, cdocsdrf.tex, cdocsfn1.tex, cdocsfn2.tex.
%
%<package>\ifdefined\childdocmain\endinput\fi
%<package>\ProvidesFile{childdoc.def}[2018/12/30 v2.0 child document driver]
%<samplemain>\ProvidesFile{cdocsamp.tex}[2018/12/30 v2.0 sample for childdoc]
%<*driver>
%\ProvidesFile{childdoc.drv}[2018/12/30 v2.0 childdoc reference manual file]
\PassOptionsToClass{10pt,a4paper}{article}
\documentclass{ltxdoc}

\usepackage[margin=35mm]{geometry}
\usepackage{hyperref}
\usepackage{hyperxmp}
\usepackage[usenames]{color}

\hypersetup{colorlinks=true}
\hypersetup{pdfstartview=FitH}
\hypersetup{pdfpagemode=UseNone}
\hypersetup{pdfsource={}}
\hypersetup{pdflang={en-UK}}
\hypersetup{pdfcopyright={Copyright 2017-2018 Niklas Beisert.
  This work may be distributed and/or modified under the
  conditions of the LaTeX Project Public License, either version 1.3
  of this license or (at your option) any later version.}}
\hypersetup{pdflicenseurl={http://www.latex-project.org/lppl.txt}}
\hypersetup{pdfcontactaddress={ETH Zurich, ITP, HIT K,
  Wolfgang-Pauli-Strasse 27}}
\hypersetup{pdfcontactpostcode={8093}}
\hypersetup{pdfcontactcity={Zurich}}
\hypersetup{pdfcontactcountry={Switzerland}}
\hypersetup{pdfcontactemail={nbeisert@itp.phys.ethz.ch}}
\hypersetup{pdfcontacturl={http://people.phys.ethz.ch/\xmptilde nbeisert/}}

\newcommand{\secref}[1]{\hyperref[#1]{section \ref*{#1}}}

\parskip1ex
\parindent0pt
\let\olditemize\itemize
\def\itemize{\olditemize\parskip0pt}

\begin{document}

\title{The \textsf{childdoc} Package}
\hypersetup{pdftitle={The childdoc Package}}
\author{Niklas Beisert\\[2ex]
  Institut f\"ur Theoretische Physik\\
  Eidgen\"ossische Technische Hochschule Z\"urich\\
  Wolfgang-Pauli-Strasse 27, 8093 Z\"urich, Switzerland\\[1ex]
  \href{mailto:nbeisert@itp.phys.ethz.ch}
  {\texttt{nbeisert@itp.phys.ethz.ch}}}
\hypersetup{pdfauthor={Niklas Beisert}}
\hypersetup{pdfsubject={Manual for the LaTeX2e Package childdoc}}
\date{30 December 2018, \textsf{v2.0}}
\maketitle

\begin{abstract}\noindent
\textsf{childdoc} is a \LaTeXe{} package
that enables the direct compilation
of document sections included by |\include|
to individual files.
\end{abstract}

\begingroup
\parskip0ex
\tableofcontents
\endgroup

%%%%%%%%%%%%%%%%%%%%%%%%%%%%%%%%%%%%%%%%%%%%%%%%%%%%%%%%%%%%%%%%%%%%%%%%%%%%%%%%
%%%%%%%%%%%%%%%%%%%%%%%%%%%%%%%%%%%%%%%%%%%%%%%%%%%%%%%%%%%%%%%%%%%%%%%%%%%%%%%%
\section{Introduction}

\LaTeX{} provides a mechanism to structure a large document (such as a book)
into a main file and several child files (containing the chapters)
using the |\include| command.
This mechanism is beneficial for documents
which span hundreds of pages in order to
make the source file(s) more manageable.
Moreover, compilation can be restricted to
selected child files by means of the |\includeonly| command.
The latter feature can be used to reduce the compilation time while editing
(this was significantly more useful in the earlier days of \LaTeX{})
or to generate a smaller document which is easier to navigate.
Another application of |\includeonly| is to generate
documents consisting of selected parts of the complete document.

However, there are a few drawbacks of the plain |\include| mechanism:
\begin{itemize}
\item
The child files cannot be compiled on their own,
they can only be compiled via the main file.
A naive editing environment
(such as a text editor with an option
to have the current file processed by \LaTeX)
may require one to switch to the main file before compiling;
attempting to compile the child file produces errors.
\item
The main file must be modified (each time)
to adjust the |\includeonly| command
to the present needs. This easily leaves the main file in a messy state.
\item
The generated document will always carry the filename
of the main document. This is inconvenient if
several child files are to be compiled and
to be kept for distribution.
\end{itemize}

The present package provides a simple interface
to make child files individually compilable by \LaTeX{}.
Compiling a child file then has the same effect as compiling
the main file with an |\includeonly| command
to select the appropriate child.
Moreover the generated document will carry the name of the child
rather than the main file.
This resolves all three above issues.

This feature is meant to make the editing of books,
thesis documents and lecture notes somewhat more convenient.
However, the package can also be used efficiently for
composing a series of documents (such as exercise sheets)
which are typically distributed individually.
It then assists the author in generating the individual documents
(potentially in different versions)
as well as a document containing the collected series.
Another application is in developing style files
or other kinds of included material
where compilation of the style file could redirect
to a sample or test file.

%%%%%%%%%%%%%%%%%%%%%%%%%%%%%%%%%%%%%%%%%%%%%%%%%%%%%%%%%%%%%%%%%%%%%%%%%%%%%%%%
%%%%%%%%%%%%%%%%%%%%%%%%%%%%%%%%%%%%%%%%%%%%%%%%%%%%%%%%%%%%%%%%%%%%%%%%%%%%%%%%
\section{Usage}

First of all, the package \textsf{childdoc} is \emph{not} a standard
\LaTeXe{} |.sty| style file! Therefore it needs to be invoked in
a non-standard way.

%%%%%%%%%%%%%%%%%%%%%%%%%%%%%%%%%%%%%%%%%%%%%%%%%%%%%%%%%%%%%%%%%%%%%%%%%%%%%%%%
\subsection{Included Files}
\label{sec:include}

%%%%%%%%%%%%%%%%%%%%%%%%%%%%%%%%%%%%%%%%
\DescribeMacro{\childdocmain}
To use the package, add the commands
\begin{center}
\begin{tabular}{l}
|\input{childdoc.def}|\\
|\childdocmain{}|\\
\end{tabular}
\end{center}
at the very top of the main \LaTeX{} file,
in particular \emph{before} the |\documentclass| statement!
The argument of |\childdocmain| should be left empty
(but it must be present).

%%%%%%%%%%%%%%%%%%%%%%%%%%%%%%%%%%%%%%%%
\DescribeMacro{\childdocof}
Furthermore, add the commands
\begin{center}
\begin{tabular}{l}
|\input{childdoc.def}|\\
|\childdocof{|\textit{main}|}|\\
\end{tabular}
\end{center}
at the top of every child file \textit{child}
which is included by |\include{|\textit{child}|}|
from within the main file
(or at least for those files to be compiled individually).
The argument \textit{main} must be the filename of the main file.

There are a couple of
considerations in setting up the main and child documents:

%%%%%%%%%%%%%%%%%%%%%%%%%%%%%%%%%%%%%%%%
\paragraph{Restrictions.}

Please note the following restrictions:
\begin{itemize}
\item
|\childdocmain| must be called with one argument \textit{main}
to ensure compatibility with earlier version of the package.
It must either be empty (|\childdocmain{}|)
or precisely match the filename of the main file in which it is specified.
See \secref{sec:detection} for further information.
\item
The filename \textit{main} must be specified without the |.tex| extension.
\item
The filename \textit{main} is case sensitive
(even in case-insensitive file systems)
due to internal string comparison.
\item
The argument \textit{main} should be fully expanded, it cannot be a macro.
\item
Subdirectories and special characters should be avoided in filenames.
\item
The command |\childdocmain{|\textit{main}|}| must be followed by a whitespace.
It should not be followed immediately by another command
or by a comment mark `|%|'.
This is because the \TeX{} parser reads the token immediately following
the argument of |\childdocmain| and puts it
at the beginning of every child section;
however, a white\-space is ignored.
\end{itemize}

%%%%%%%%%%%%%%%%%%%%%%%%%%%%%%%%%%%%%%%%
\paragraph{Content of Main File.}

It is advisable to place all content in the child files included by |\include|.
Any output contained in the main file will appear in all child documents
unless suppressed manually;
it cannot be suppressed automatically by the |\includeonly| directive
and thus should normally be avoided.
A method to include some content in the main file
by means of conditional processing is described in \secref{sec:conditional}.

%%%%%%%%%%%%%%%%%%%%%%%%%%%%%%%%%%%%%%%%
\paragraph{Page Numbering.}

When only a part of the document is compiled,
the appropriate numbering of pages
(as well as other status parameters)
is determined from the |.aux| files.
The latter contain information from previous passes.
However this information needs to propagate through
all intermediate child documents.
Therefore the page numbering in child documents may well
be inconsistent until the complete document is compiled at least once.

A useful (if unconventional) way to always ensure a consistent
page numbering is to restart the numbering in each child document
and denote the pages by `\textit{child}|.|\textit{page}'
where \textit{child} represents the chapter/section number of the child file.
This can be achieved by the command
|\numberwithin{page}{|\textit{child}|}|
of the \textsf{amsmath} package
where \textit{child} can be |chapter| or |section|
depending on the chosen structuring.
Alternatively, one can modify the macro |\thepage| appropriately
and reset the counter |page| at the start of each child file.

%%%%%%%%%%%%%%%%%%%%%%%%%%%%%%%%%%%%%%%%%%%%%%%%%%%%%%%%%%%%%%%%%%%%%%%%%%%%%%%%
\subsection{Conditional Processing}
\label{sec:conditional}

The package provides a mechanism to compile different versions
of a document. To customise the versions further some conditional processing
can come in handy to distinguish which version is being compiled.
The package provides two macros to describe the compilation context:

%%%%%%%%%%%%%%%%%%%%%%%%%%%%%%%%%%%%%%%%
\DescribeMacro{\ifchilddoc}
The conditional |\ifchilddoc| distinguishes between the compilation of
child documents and the main document:
%
\begin{center}
|\ifchilddoc |\textit{child-code}| |[|\||else |\textit{main-code}]| \||fi|
\end{center}

%%%%%%%%%%%%%%%%%%%%%%%%%%%%%%%%%%%%%%%%
\DescribeMacro{\childdocname}
\DescribeMacro{\childdocjob}
The macro |\childdocname| contains the filename (without extension)
of the main or child file being processed.
Note that |\childdocjob| will always contain the name of the main file.

%%%%%%%%%%%%%%%%%%%%%%%%%%%%%%%%%%%%%%%%
\paragraph{Title Page.}

Conditional processing can be used to include a title or banner page
in the main document when proper precautions are taken.
Importantly, the code in the main file should ensure that the page counter
(as well as other status parameters which are stored in the |.aux| files)
takes the same value after the conditional processing.
Otherwise the page numbers may take divergent values
depending on which part is compiled.

For example, a title page could be declared by:
%
\begin{center}
\begin{tabular}{l}
|\ifchilddoc\||else|\\
|\addtocounter{page}{-1}|\\
\textit{code for title page}\\
|\newpage|\\
|\||fi|
\end{tabular}
\end{center}
%
A banner page for the child documents can be generated by:
%
\begin{center}
\begin{tabular}{l}
|\ifchilddoc|\\
|\addtocounter{page}{-1}|\\
\textit{code for banner page}\\
|\newpage|\\
|\||fi|
\end{tabular}
\end{center}
%
Here one could write a message such as:
\begin{center}
|This is the part \childdocname{} of \childdocjob{}.|
\end{center}

%%%%%%%%%%%%%%%%%%%%%%%%%%%%%%%%%%%%%%%%%%%%%%%%%%%%%%%%%%%%%%%%%%%%%%%%%%%%%%%%
\subsection{Flags}
\label{sec:flags}

The package makes it easy to generate different versions
of the main or child documents.
To this end compilation flags can be defined
and assigned different default values.
They will be particularly useful in conjunction
with the forwarding mechanism described in \secref{sec:forward}.

For example, it may be useful to have a flag |\version|
which can be set to |draft| or |final|.
The document source will contain some conditional code
depending on the value of |\version|.
Suppose further, the flag should default to |final| for the main file
and to |draft| for child files
which is a natural assignment for editing the document.
This is achieved by placing the following code
in the preamble of the main document
(below the |\childdocmain| directive):
%
\begin{center}
\begin{tabular}{l}
|\ifchilddoc|\\
|\providecommand{\version}{draft}|\\
|\||else|\\
|\providecommand{\version}{final}|\\
|\||fi|
\end{tabular}
\end{center}
%
The definition by |\providecommand| makes sure
that previous definitions are not overwritten.
Further statements |\providecommand{\version}{...}|
can thus be added before the above code to override it.

For the main file, one might add a line
(between |\childdocmain| and the above block)
%
\begin{center}
|%\ifchilddoc\||else\providecommand{\version}{draft}\||fi|
\end{center}
%
which can be uncommented to produce a draft version.
Likewise one can add a line to the very top of a child file
(above the |\childdocof{|\textit{main}|}| directive)
%
\begin{center}
|%\providecommand{\version}{final}|
\end{center}
%
which can be uncommented to produce the final version of this child document.

%%%%%%%%%%%%%%%%%%%%%%%%%%%%%%%%%%%%%%%%%%%%%%%%%%%%%%%%%%%%%%%%%%%%%%%%%%%%%%%%
\subsection{Forwarding}
\label{sec:forward}

Different versions of the main or child documents
using compilation flags as described in \secref{sec:flags}
can be (permanently) stored in different files
for convenient compilation, viewing and distribution.
To this end, the package defines a command
to pass on compilation to a different file:

%%%%%%%%%%%%%%%%%%%%%%%%%%%%%%%%%%%%%%%%
\DescribeMacro{\childdocforward}
The command |\childdocforward| redirects processing to
another source file:
%
\begin{center}
\begin{tabular}{l}
|\input{childdoc.def}|\\
|\childdocforward[|\textit{main}|]{|\textit{dest}|}|\\
\end{tabular}
\end{center}
%
The argument \textit{dest} is the destination file
(without extension).
It should be the main file or one of the child files.
Note that further \textsf{childdoc} directives
such as |\childdocof| and |\childdocforward|
in the indicated file will be processed in this form.
The optional argument \textit{main}
passes on directly to the main file \textit{main}
while pretending to compile the child \textit{dest}.
This form behaves as if \textit{dest}
issues |\childdocof{|\textit{main}|}| right away,
and no further \textsf{childdoc} directives will be processed.

%%%%%%%%%%%%%%%%%%%%%%%%%%%%%%%%%%%%%%%%
\DescribeMacro{\...prefix}
In the alternative form |\childdocforwardprefix|,
%
\begin{center}
\begin{tabular}{l}
|\input{childdoc.def}|\\
|\childdocforwardprefix[|\textit{main}|]{|\textit{prefix}|}{|\textit{dest}|}|
\end{tabular}
\end{center}
%
the destination file is determined by a pattern
depending on the current file:
To make this work, the current file must be called
`{\textit{prefix}\hspace{0.2em}\textit{suffix}}'
with \textit{prefix} matching precisely the argument.
Processing is then passed on to the file
`{\textit{dest}\hspace{0.2em}\textit{suffix}}'.
Surely, the same effect is achieved by
directly specifying the
argument `{\textit{dest}\hspace{0.2em}\textit{suffix}}'
in the first form.
However, that requires to set up a different file
for each child. With the alternative form of the command
all these files can have exactly the same content
which simplifies setting them up and maintaining them.

For example, the following file |draft.tex|
with a compilation flag |\version| as described in \secref{sec:flags}
compiles the main document as a draft:
%
\begin{center}
\begin{tabular}{l}
|\def\version{draft}|\\
|\input{childdoc.def}|\\
|\childdocforward{|\textit{main}|}|
\end{tabular}
\end{center}
%
Likewise, the following files |final|\textit{nn}|.tex|
compile the final version of the child document
|child|\textit{nn}|.tex|:
%
\begin{center}
\begin{tabular}{l}
|\def\version{final}|\\
|\input{childdoc.def}|\\
|\childdocforwardprefix{final}{child}|
\end{tabular}
\end{center}
%

Note that when several versions of a main file and/or of each child file
are to be generated, it may be convenient to set up a |Makefile| or
shell script to automatise the process.

%%%%%%%%%%%%%%%%%%%%%%%%%%%%%%%%%%%%%%%%%%%%%%%%%%%%%%%%%%%%%%%%%%%%%%%%%%%%%%%%
\subsection{Command Line Processing}
\label{sec:commandline}

The effect of redirection files can also be achieved by invoking
the \LaTeX{} compiler with a more elaborate command line.
Most conveniently this should be done as part
of a shell script or a |Makefile|.

When using \textsf{childdoc} in the main file, the following
command lines effectively perform a redirection
(note that depending on the shell being used,
backslashes may have to be doubled: `|\|' $\to$ `|\\|'):
%
\begin{center}
|... -jobname "|\textit{target}|" |\\|"|[\textit{flags}]%
|\input{childdoc.def}\childdocforward[|\textit{main}|]{|\textit{dest}|}"|
\end{center}
%
Here \textit{target} is the name of the output file,
\textit{main} is the name of the main file
and \textit{dest} is the name of the main or child file to be processed
(all filenames without extensions).
The optional argument \textit{main} can be omitted
if \textit{main} matches \textit{dest}.
Optionally, compilation \textit{flags} can be defined via |\def| commands.
This command line makes the \TeX{} engine believe
it is compiling the file \textit{target}
whose content is specified as the latter parameter.
The provided code then forwards the processing to
\textit{main} or \textit{dest} as described in \secref{sec:forward}.

%%%%%%%%%%%%%%%%%%%%%%%%%%%%%%%%%%%%%%%%%%%%%%%%%%%%%%%%%%%%%%%%%%%%%%%%%%%%%%%%
\subsection{Include by Input}
\label{sec:input}

Including child documents by |\include| has some restrictions by design.
Most notably, the content of a child document always occupies
its own set of pages; pages cannot be shared between child documents.
Usually, this behaviour makes perfect sense
because each child document contain an essential part of the document.
However, in some situations it may be desirable to compose
a document from a collection of parts
without having mandatory page breaks between then.
For this case, the package
provides a mechanism to include parts
by |\input| which can also be processed individually.
However, by construction this mechanism
requires manual handling of the content to be output.

%%%%%%%%%%%%%%%%%%%%%%%%%%%%%%%%%%%%%%%%
\DescribeMacro{\ifchilddocmanual}
The main file should be prepared as usual, see \secref{sec:include}.
However, the document body must make a distinction
between processing of an individual part and of the main document, e.g.:
%
\begin{center}
\begin{tabular}{l}
|\ifchilddocmanual|\\
|\input{\childdocname}|\\
|\||else|\\
\textit{document body with }|\input{|\textit{part}|}|\\
|\||fi|
\end{tabular}
\end{center}
%
The conditional |\ifchilddocmanual| is true whenever
a part to be included by |\input| is being compiled,
and the name of the part is stored in |\childdocname|.

%%%%%%%%%%%%%%%%%%%%%%%%%%%%%%%%%%%%%%%%
\DescribeMacro{\childdocby}
Each part to be included by |\input| should start with:
%
\begin{center}
\begin{tabular}{l}
|\input{childdoc.def}|\\
|\childdocby{|\textit{main}|}|\\
\end{tabular}
\end{center}
%
The directive |\childdocby| is similar to |\childdocof|
described in \secref{sec:include},
but the subsequent selection of content must be done manually.
To that end, both |\ifchilddoc| and |\ifchilddocmanual|
will be true upon processing of a part,
and the name of the part is stored in |\childdocname|.
Note that |\jobname| will be set to the filename of the current part
so that each part receives an individual |.aux| file
that does not interfere with the |.aux| file(s) of the main document.
This behaviour can be altered by the alternative form
|\childdocby[*]{|\textit{main}|}| (with a non-empty optional argument)
which uses the |.aux| file of the main document
by setting |\jobname| to \textit{main}.

%%%%%%%%%%%%%%%%%%%%%%%%%%%%%%%%%%%%%%%%%%%%%%%%%%%%%%%%%%%%%%%%%%%%%%%%%%%%%%%%
\subsection{Driver Development}
\label{sec:driver}

The \textsf{childdoc} mechanism can also be use for the development
of definition files such as \LaTeX{} styles or classes.
This case differs from the above setup with multiple parts
included by |\include| in that no |\includeonly| should be invoked.
This can be achieved by starting the include file
(before |\ProvidesPackage|) with:
%
\begin{center}
\begin{tabular}{l}
|\input{childdoc.def}|\\
|\childdocforward{|\textit{main}|}|\\
\end{tabular}
\end{center}
%
or alternatively with:
%
\begin{center}
\begin{tabular}{l}
|\input{childdoc.def}|\\
|\childdocby{|\textit{main}|}|\\
\end{tabular}
\end{center}
%
Both forms have slightly different effects as described above.
The main file is prepared as usual, see \secref{sec:include}.

%%%%%%%%%%%%%%%%%%%%%%%%%%%%%%%%%%%%%%%%%%%%%%%%%%%%%%%%%%%%%%%%%%%%%%%%%%%%%%%%
\subsection{Legacy Detection}
\label{sec:detection}

The directive |\childdocmain| in the main file can detect
whether the complete document or merely a child is to be compiled
even without using the directive |\childdocof|.
This method is deprecated because it is less robust
and there is no compelling reason to use it;
it is merely provided for backward compatibility
and it may be removed in future versions.

If the detection mechanism is to be used,
it is mandatory to correctly specify
the filename of the main file as the argument of |\childdocmain|:
%
\begin{center}
\begin{tabular}{l}
|\input{childdoc.def}|\\
|\childdocmain{|\textit{main}|}|\\
\end{tabular}
\end{center}
%
If |\jobname| does not match the argument \textit{main} of |\childdocmain|,
it is assumed that |\jobname| points to the child file to be compiled.
When using |\childdocmain| with the main file specified as argument,
it suffices to start a child file
with just |\input{|\textit{main}|}|
without loading of the package and using |\childdocof|.
If instead all processing is done
with the appropriate \textsf{childdoc} directives,
the argument of \textit{main} of |\childdocmain| can be empty.

An alternative version of the command line processing described
in \secref{sec:commandline} using the detection mechanism reads:
%
\begin{center}
|... -jobname "|\textit{target}|" "|[\textit{flags}]%
[|\def\jobname{|\textit{dest}|}|]|\input{|\textit{main}|}"|
\end{center}

%%%%%%%%%%%%%%%%%%%%%%%%%%%%%%%%%%%%%%%%%%%%%%%%%%%%%%%%%%%%%%%%%%%%%%%%%%%%%%%%
\subsection{Manual Code}
\label{sec:manual}

In case one cannot be certain whether the definitions file |childdoc.def|
is installed on the target \TeX{} distribution
and one prefers not to ship it,
it is conceivable to paste a few relevant commands into the sources.

To that end, drop all statements |\input{childdoc.def}|
and perform the replacements as outlined below.
Instead of |\childdocmain{|\textit{main}|}| add the following code
to the top of the main file:
%
\begin{center}
\begin{tabular}{l}
|\||ifdefined\childdocname\endinput\||fi\newif\ifchilddoc|\\
|\edef\childdocname{\scantokens\expandafter{\jobname\noexpand}}|\\
|\def\childdocmain{|\textit{main}|}\||ifx\childdocmain\childdocname\||else|\\
|\childdoctrue\includeonly{\childdocname}\let\jobname\childdocmain\||fi|\\
\end{tabular}
\end{center}
%
Instead of |\childdocof{|\textit{main}|}| just include the main file
at the top of each child file:
%
\begin{center}
|\input{|\textit{main}|}|
\end{center}
%
A simple redirection |\childdocforward{|\textit{dest}|}| is achieved by:
%
\begin{center}
|\def\jobname{|\textit{dest}|}\input{\jobname}|
\end{center}
%
The redirection with prefix
|\childdocforwardprefix[|\textit{prefix}|]{|\textit{dest}|}|
is accomplished by:
%
\begin{center}
\begin{tabular}{l}
|{\edef\jobname{\scantokens\expandafter{\jobname\noexpand}}|\\
|\def\redirectjob |\textit{prefix}|#1~~~{\gdef\jobname{|\textit{dest}|#1}}|\\
|\expandafter\redirectjob\jobname~~~}\input{\jobname}|
\end{tabular}
\end{center}

In an alternative approach,
child documents can be compiled by a specific command line
without additional code or specific definitions:
%
\begin{center}
|... -jobname "|\textit{target}|" "|[\textit{flags}]%
|\includeonly{|\textit{dest}|}\input{|\textit{main}|}"|
\end{center}
%

%%%%%%%%%%%%%%%%%%%%%%%%%%%%%%%%%%%%%%%%%%%%%%%%%%%%%%%%%%%%%%%%%%%%%%%%%%%%%%%%
%%%%%%%%%%%%%%%%%%%%%%%%%%%%%%%%%%%%%%%%%%%%%%%%%%%%%%%%%%%%%%%%%%%%%%%%%%%%%%%%
\section{Information}

%%%%%%%%%%%%%%%%%%%%%%%%%%%%%%%%%%%%%%%%%%%%%%%%%%%%%%%%%%%%%%%%%%%%%%%%%%%%%%%%
\subsection{Copyright}

Copyright \copyright{} 2017--2018 Niklas Beisert

This work may be distributed and/or modified under the
conditions of the \LaTeX{} Project Public License, either version 1.3
of this license or (at your option) any later version.
The latest version of this license is in
  \url{http://www.latex-project.org/lppl.txt}
and version 1.3 or later is part of all distributions of \LaTeX{}
version 2005/12/01 or later.

This work has the LPPL maintenance status `maintained'.

The Current Maintainer of this work is Niklas Beisert.

This work consists of the files |README.txt|, |childdoc.ins| and |childdoc.dtx|
as well as the derived files |childdoc.def|, |cdocsamp.tex|
with |cdocsch1.tex|, |cdocsch2.tex|, |cdocspt3.tex|, |cdocspt4.tex|,
|cdocsdrf.tex|, |cdocsfn1.tex|, |cdocsfn2.tex|
as well as |childdoc.pdf|.

%%%%%%%%%%%%%%%%%%%%%%%%%%%%%%%%%%%%%%%%%%%%%%%%%%%%%%%%%%%%%%%%%%%%%%%%%%%%%%%%
\subsection{Files and Installation}

The package consists of the files:
%
\begin{center}
\begin{tabular}{ll}
    |README.txt|   & readme file \\
    |childdoc.ins| & installation file \\
    |childdoc.dtx| & source file \\
    |childdoc.def| & definition file \\
    |cdocsamp.tex| & sample main file \\
    |cdocsch1.tex| & sample include file \\
    |cdocsch2.tex| & sample include file \\
    |cdocspt3.tex| & sample part file \\
    |cdocspt4.tex| & sample part file \\
    |cdocsdrf.tex| & sample redirection file \\
    |cdocsfn1.tex| & sample redirection file \\
    |cdocsfn2.tex| & sample redirection file \\
    |childdoc.pdf| & manual
\end{tabular}
\end{center}
%
The distribution consists of the files
|README.txt|, |childdoc.ins| and |childdoc.dtx|.
%
\begin{itemize}
\item
Run (pdf)\LaTeX{} on |childdoc.dtx|
to compile the manual |childdoc.pdf| (this file).
\item
Run \LaTeX{} on |childdoc.ins| to create the definitions file |childdoc.def|
and the sample |cdocsamp.tex| with include files
|cdocsch1.tex|, |cdocsch2.tex|, |cdocspt3.tex|, |cdocspt4.tex|,
|cdocsdrf.tex|, |cdocsfn1.tex|, |cdocsfn2.tex|.
Then copy the file |childdoc.def| to an appropriate directory of your \LaTeX{}
distribution, e.g.\ \textit{texmf-root}|/tex/latex/childdoc|.
\end{itemize}

%%%%%%%%%%%%%%%%%%%%%%%%%%%%%%%%%%%%%%%%%%%%%%%%%%%%%%%%%%%%%%%%%%%%%%%%%%%%%%%%
\subsection{Related CTAN Packages}

There are several other packages which offer a similar functionality:
%
\begin{itemize}
\item
The packages
\href{http://ctan.org/pkg/docmute}{\textsf{docmute}},
\href{http://ctan.org/pkg/includex}{\textsf{includex}} and
\href{http://ctan.org/pkg/standalone}{\textsf{standalone}}
provide commands to include only the document body of
a child file thus allowing both files to be compiled individually.
\item
The packages \href{http://ctan.org/pkg/subdocs}{\textsf{subdocs}}
and \href{http://ctan.org/pkg/subfiles}{\textsf{subfiles}}
provide structures in which the main and child documents can be
encapsulated and allowing them to be compiled individually.
The inclusion mechanism is different from the conventional |\include|.
\item
The package \href{http://ctan.org/pkg/combine}{\textsf{combine}}
is an elaborate solution to combine several documents into one.
\end{itemize}
%
See also the CTAN topic \href{http://ctan.org/topic/subdocs}{\textsf{subdocs}}
for further related packages.
The present package differs from the above solutions in that
a document structure constructed with the conventional |\include| mechanism
just needs two extra commands at the top of every file
such that all constituent files can be compiled individually.

%%%%%%%%%%%%%%%%%%%%%%%%%%%%%%%%%%%%%%%%%%%%%%%%%%%%%%%%%%%%%%%%%%%%%%%%%%%%%%%%
%\subsection{Feature Suggestions}
%
%The following is a list of features which may be useful for future
%versions of this package:
%%
%\begin{itemize}
%\item
%\ldots
%\end{itemize}

%%%%%%%%%%%%%%%%%%%%%%%%%%%%%%%%%%%%%%%%%%%%%%%%%%%%%%%%%%%%%%%%%%%%%%%%%%%%%%%%
\subsection{Revision History}

%%%%%%%%%%%%%%%%%%%%%%%%%%%%%%%%%%%%%%%%
\paragraph{v2.0:} 2018/12/30

\begin{itemize}
\item
immediate forward processing
\item
added |\childdocby| mechanism
\item
manual restructured
\end{itemize}

%%%%%%%%%%%%%%%%%%%%%%%%%%%%%%%%%%%%%%%%
\paragraph{v1.6:} 2018/01/17

\begin{itemize}
\item
application for development of include files
\item
corrections to manual
\end{itemize}

%%%%%%%%%%%%%%%%%%%%%%%%%%%%%%%%%%%%%%%%
\paragraph{v1.5:} 2017/05/21

\begin{itemize}
\item
more complete structuring introduced
\item
|\childdocof| introduced
\item
|\childdoc| renamed to |\childdocmain|
\item
|\childredirect| renamed to |\childdocforward| and |\childdocforwardprefix|
and functionality expanded
\end{itemize}

%%%%%%%%%%%%%%%%%%%%%%%%%%%%%%%%%%%%%%%%
\paragraph{v1.0:} 2017/04/27

\begin{itemize}
\item
manual and install package
\item
first version published on CTAN
\end{itemize}

%%%%%%%%%%%%%%%%%%%%%%%%%%%%%%%%%%%%%%%%
\paragraph{v0.6:} 2017/04/26

\begin{itemize}
\item
redirection mechanism added
\end{itemize}

%%%%%%%%%%%%%%%%%%%%%%%%%%%%%%%%%%%%%%%%
\paragraph{v0.5:} 2017/04/26

\begin{itemize}
\item
functionality in definition file
\end{itemize}


%%%%%%%%%%%%%%%%%%%%%%%%%%%%%%%%%%%%%%%%%%%%%%%%%%%%%%%%%%%%%%%%%%%%%%%%%%%%%%%%
%%%%%%%%%%%%%%%%%%%%%%%%%%%%%%%%%%%%%%%%%%%%%%%%%%%%%%%%%%%%%%%%%%%%%%%%%%%%%%%%
%%%%%%%%%%%%%%%%%%%%%%%%%%%%%%%%%%%%%%%%%%%%%%%%%%%%%%%%%%%%%%%%%%%%%%%%%%%%%%%%
\appendix

\settowidth\MacroIndent{\rmfamily\scriptsize 000\ }

 \DocInput{childdoc.dtx}

\end{document}
%</driver>
% \fi
%
% %%%%%%%%%%%%%%%%%%%%%%%%%%%%%%%%%%%%%%%%%%%%%%%%%%%%%%%%%%%%%%%%%%%%%%%%%%%%%%
% %%%%%%%%%%%%%%%%%%%%%%%%%%%%%%%%%%%%%%%%%%%%%%%%%%%%%%%%%%%%%%%%%%%%%%%%%%%%%%
% \section{Sample}
%\iffalse
%<*samplemain>
%\fi
%
% The following presents a sample document
% with two chapters, two parts, a title page,
% a compile flag as well as three forwarding files to set the flag.
% It consists of eight |.tex| files:
% \begin{center}
% \begin{tabular}{ll}
% |cdocsamp.tex|&main file\\
% |cdocsch1.tex|&include file for chapter 1\\
% |cdocsch2.tex|&include file for chapter 2\\
% |cdocspt3.tex|&include file for part 3\\
% |cdocspt4.tex|&include file for part 4\\
% |cdocsdrf.tex|&forwarding file for main file in draft mode\\
% |cdocsfi1.tex|&forwarding file for final version of chapter 1\\
% |cdocsfi2.tex|&forwarding file for final version of chapter 2\\
% \end{tabular}
% \end{center}
% Each of the eight files can be compiled directly by the \LaTeX{} compiler.
%
% %%%%%%%%%%%%%%%%%%%%%%%%%%%%%%%%%%%%%%
% \paragraph{Main File.}
%
% The main file is called |cdocsamp.tex|.
%
% Load the \textsf{childdoc} definitions and
% declare the filename for the main document:
%    \begin{macrocode}
\input{childdoc.def}
\childdocmain{}
%    \end{macrocode}

% Optional override for |\version| flag:
%    \begin{macrocode}
%%\ifchilddoc\else\providecommand{\version}{draft}\fi
%    \end{macrocode}

% Define the default values for the |\version| flag
% (|final| for the main file and |draft| for childs):
%    \begin{macrocode}
\ifchilddoc
\providecommand{\version}{draft}
\else
\providecommand{\version}{final}
\fi
%    \end{macrocode}

% Load the standard document class:
%    \begin{macrocode}
\documentclass[12pt]{article}
%    \end{macrocode}

% Start the document body:
%    \begin{macrocode}
\begin{document}
%    \end{macrocode}

% Declare a title page.
% Print title, part of document being processed and version flag:
%    \begin{macrocode}
\addtocounter{page}{-1}
\begin{center}
{\LARGE\bfseries{}childdoc example\par}
\vspace{1cm}
\ifchilddoc
\ifchilddocmanual part\else chapter\fi:
`\childdocname' of `\childdocjob'\par
\else
main document: `\childdocjob'\par
\fi
version: \version\par
\end{center}
\newpage
%    \end{macrocode}

% Manually include selected file,
% otherwise process as usual:
%    \begin{macrocode}
\ifchilddocmanual
\section*{part `\childdocname'}
\input{\childdocname}
\else
%    \end{macrocode}

% Include the two chapters:
%    \begin{macrocode}
\include{cdocsch1}
\include{cdocsch2}
%    \end{macrocode}

% Include the two parts unless only chapters should be displayed:
%    \begin{macrocode}
\ifchilddoc\else
\section{part three}
\input{cdocspt3}
\section{part four}
\input{cdocspt4}
\fi
%    \end{macrocode}

% Process as usual until here:
%    \begin{macrocode}
\fi
%    \end{macrocode}

% End of document body:
%    \begin{macrocode}
\end{document}
%    \end{macrocode}
%\iffalse
%</samplemain>
%\fi
%
% %%%%%%%%%%%%%%%%%%%%%%%%%%%%%%%%%%%%%%
% \paragraph{Chapter Include Files.}
%
% The include files are called |cdocsch1.tex| and |cdocsch2.tex|.
%
%\iffalse
%<*samplechap1|samplechap2>
%\fi

% Optional override for |\version| flag:
%    \begin{macrocode}
%%\providecommand{\version}{final}
%    \end{macrocode}

% Include the main document:
%    \begin{macrocode}
\input{childdoc.def}
\childdocof{cdocsamp}
%    \end{macrocode}

%\iffalse
%</samplechap1|samplechap2>
%\fi
%
%\iffalse
%<*samplechap1>
%\fi
% Some text for chapter 1:
%    \begin{macrocode}
\section{one}
some text in chapter one
%    \end{macrocode}

%\iffalse
%</samplechap1>
%\fi
% Some text for chapter 2:
%\iffalse
%<*samplechap2>
%\fi
%    \begin{macrocode}
\section{two}
more text in chapter two
%    \end{macrocode}

%\iffalse
%</samplechap2>
%\fi
%
% %%%%%%%%%%%%%%%%%%%%%%%%%%%%%%%%%%%%%%
% \paragraph{Part Include Files.}
%
% The include files are called |cdocspt3.tex| and |cdocspt4.tex|.
%
%\iffalse
%<*samplepart3|samplepart4>
%\fi

% Optional override for |\version| flag:
%    \begin{macrocode}
%%\providecommand{\version}{final}
%    \end{macrocode}

% Include the main document:
%    \begin{macrocode}
\input{childdoc.def}
\childdocby{cdocsamp}
%    \end{macrocode}

%\iffalse
%</samplepart3|samplepart4>
%\fi
%
%\iffalse
%<*samplepart3>
%\fi
% Some text for part 3:
%    \begin{macrocode}
some text in part three
%    \end{macrocode}

%\iffalse
%</samplepart3>
%\fi
% Some text for part 4:
%\iffalse
%<*samplepart4>
%\fi
%    \begin{macrocode}
more text in part four
%    \end{macrocode}

%\iffalse
%</samplepart4>
%\fi
%
% %%%%%%%%%%%%%%%%%%%%%%%%%%%%%%%%%%%%%%
% \paragraph{Forwarding for a Complete Draft.}
%
% The following forwarding file |cdocsdrf.tex|
% compiles the main document in draft mode:
%\iffalse
%<*sampledraft>
%\fi
%    \begin{macrocode}
\def\version{draft}
\input{childdoc.def}
\childdocforward{cdocsamp}
%    \end{macrocode}

%\iffalse
%</sampledraft>
%\fi
%
% %%%%%%%%%%%%%%%%%%%%%%%%%%%%%%%%%%%%%%
% \paragraph{Forwarding for Final Version of the Chapters.}
%
% The following forwarding files |cdocsfn1.tex| and |cdocsfn2.tex|
% (with identical content)
% compile the final versions of the child documents
% |cdocsch1.tex| and |cdocsch2.tex|, respectively:
%\iffalse
%<*samplefinal>
%\fi
%    \begin{macrocode}
\def\version{final}
\input{childdoc.def}
\childdocforwardprefix[cdocsamp]{cdocsfn}{cdocsch}
%    \end{macrocode}

%\iffalse
%</samplefinal>
%\fi
%
% %%%%%%%%%%%%%%%%%%%%%%%%%%%%%%%%%%%%%%
% \paragraph{Command Line Processing.}
%
% The following three command lines generate the output files
% |cdocscld|, |cdocscl1| and |cdocscl2|
% which should be identical to
% |cdocsdrf|, |cdocsch1| and |cdocsfn2|, respectively:
% \begin{center}
% \begin{tabular}{l}
% |latex -jobname cdocscld \|\\
% |  "\def\version{draft}\input{childdoc.def}\childdocforward{cdocsamp}"|\\
% |latex -jobname cdocscl1 \|\\
% |  "\input{childdoc.def}\childdocforward[cdocsamp]{cdocsch1}"|\\
% |latex -jobname cdocscl2 \|\\
% |  "\def\version{final}\input{childdoc.def}\childdocforward{cdocsch2}"|
% \end{tabular}
% \end{center}
% Note that the trailing backslash on each first line
% merely continues the input to the second line
% (for convenient cut ant paste).
% Furthermore, the command |latex| can be replaced by any
% of its alternative versions such as |pdflatex|.
%
% %%%%%%%%%%%%%%%%%%%%%%%%%%%%%%%%%%%%%%%%%%%%%%%%%%%%%%%%%%%%%%%%%%%%%%%%%%%%%%
% %%%%%%%%%%%%%%%%%%%%%%%%%%%%%%%%%%%%%%%%%%%%%%%%%%%%%%%%%%%%%%%%%%%%%%%%%%%%%%
% \section{Implementation}
%\iffalse
%<*package>
%\fi
%
% This section describes the definitions file |childdoc.def|.

% The definitions cannot be loaded using |\usepackage| or |\RequirePackage|
% which has a mechanism to prevent loading a style file more than once.
% When loading the definitions by means of |\input|
% multiple instances have to be prevented manually:
%\iffalse
%This code needs to be before the `\ProvidesFile' directive
%which is defined at the beginning of this file.
%Therefore it is also placed there and commented out here.
%</package>
%<*discard>
%\fi
%    \begin{macrocode}
\ifdefined\childdocmain\endinput\fi
%    \end{macrocode}
%\iffalse
%</discard>
%<*package>
%\fi
%
% \macro{\ifchilddoc}
% \macro{\ifchilddocmanual}
% The conditional |\ifchilddoc| tells whether a
% child (true) or main (false) document is being compiled.
% The conditional |\ifchilddocmanual| tells whether
% the |\includeonly| mechanism is used (false) or
% the selection of child files must be performed manually (true).
% The definitions initialise to false:
%    \begin{macrocode}
\newif\ifchilddoc
\newif\ifchilddocmanual
%    \end{macrocode}

% \macro{\childdocname}
% \macro{\childdocjob}
% The macro |\childdocname| stores the name of the main document
% to be compiled. The macro |\childdocjob| stores the name of
% the document on which the \LaTeX{} compiler was originally invoked.
% The content of |\jobname| cannot be compared
% to filenames specified in the source due to different catcodes.
% The following code rescans |\jobname|, stores the result
% in |\childdocname| and saves a copy in |\childdocjob|:
%    \begin{macrocode}
\edef\childdocname{\scantokens\expandafter{\jobname\noexpand}}
\let\childdocjob\childdocname
%    \end{macrocode}

% \macro{\childdocdisable}
% The macro |\childdocdisable| prevents the main file
% from being processed more than once.
% At this stage, the main document command |\childdocmain|
% is assumed to be called once again where it should do nothing.
% Any subsequent call to it should prevent
% a secondary processing of the main document
% It overwrites the forwarding commands
% |\childdocof| and |\childdocforward|
% with empty macros to prevent further inclusions of the main document:
%    \begin{macrocode}
\newcommand{\childdocdisable}
{
  \renewcommand{\childdocmain}[1]{\renewcommand{\childdocmain}[1]{\endinput}}
  \renewcommand{\childdocof}[1]{}
  \renewcommand{\childdocby}[2][]{}
  \renewcommand{\childdocforward}[2][]{}
  \renewcommand{\childdocdisable}{}
}
%    \end{macrocode}

% \macro{\childdocmain}
% The macro |\childdocmain| is to be called at the top of the main file
% with nothing or the main filename (without extension) as argument.
% First, it breaks loops.
% If the argument is not empty and does not match |\childdocname|
% (which is set by the first inclusion of |childdoc.def|),
% |\ifchilddoc| is set to true, |\includeonly| is applied to the child file
% and |\jobname| is set to the main file
% (for proper handling of |.aux| files):
%    \begin{macrocode}
\newcommand{\childdocmain}[1]
{
  \childdocdisable\childdocmain{}
  \if?#1?\else
    \begingroup
      \def\childdoctmp{#1}
      \ifx\childdoctmp\childdocname
        \def\childdoctmp{}
      \else
        \def\childdoctmp
        {
          \childdoctrue
          \includeonly{\childdocname}
          \def\childdocjob{#1}
          \def\jobname{#1}
        }
      \fi
      \expandafter
    \endgroup
    \childdoctmp
  \fi
}
%    \end{macrocode}

% \macro{\childdocof}
% The command |\childdocof| redirects
% compilation to the main file |#1|.
%    \begin{macrocode}
\newcommand{\childdocof}[1]
{
  \childdocdisable
  \childdoctrue
  \includeonly{\childdocname}
  \def\jobname{#1}
  \def\childdocjob{#1}
  \input{#1}
}
%    \end{macrocode}

% \macro{\childdocby}
% The command |\childdocby| ....
%    \begin{macrocode}
\newcommand{\childdocby}[2][]
{
  \childdocdisable
  \childdoctrue
  \childdocmanualtrue
  \if?#1?\else
    \def\jobname{#2}
  \fi
  \def\childdocjob{#2}
  \input{#2}
  \endinput
}
%    \end{macrocode}

% \macro{\childdocforward}
% The command |\childdocforward| redirects
% compilation to the main file or
% (if the optional argument is given) a child file.
% Parameters are set as if the main file
% or a child file starting with |\childdocof| was compiled.
% Then compilation is handed over to the main file:
%    \begin{macrocode}
\newcommand{\childdocforward}[2][]
{
  \begingroup
    \if?#1?
      \def\childdoctmp
      {
        \def\childdocname{#2}
        \def\childdocjob{#2}
        \def\jobname{#2}
        \input{#2}
        \endinput
      }
    \else
      \def\childdoctmp
      {
        \childdocdisable
        \def\childdocname{#2}
        \childdoctrue
        \includeonly{#2}
        \def\childdocjob{#1}
        \def\jobname{#1}
        \input{#1}
        \endinput
      }
    \fi
    \expandafter
  \endgroup
  \childdoctmp
}
%    \end{macrocode}

% \macro{\childdocforwardprefix}
% The command |\childdocforwardprefix| redirects
% compilation to the main or a child file by means of a pattern.
% The prefix |#1| in the current filename is replaced by |#2|
% and the suffix of the current filename is kept
% (it is assumed that the filename does not contain the substring `|~~~|'
% which is used as a delimiter).
% Compilation is handed over to the new file by |\childdocforward|:
%    \begin{macrocode}
\newcommand{\childdocforwardprefix}[3][]
{
  \begingroup
    \def\childdocextract #2##1~~~{\def\childdoctmp{\childdocforward[#1]{#3##1}}}
    \expandafter\childdocextract\childdocname~~~
    \expandafter
  \endgroup
  \childdoctmp
}
%    \end{macrocode}

% \macro{\childdoc}
% The deprecated macro |\childdoc| is a legacy version of |\childdocmain|:
%    \begin{macrocode}
\newcommand{\childdoc}{\childdocmain}
%    \end{macrocode}

% \macro{\childdocredirect}
% The deprecated macro |\childdocredirect| is a legacy version
% of |\childdocforward| and |\childdocforwardprefix|:
%    \begin{macrocode}
\newcommand{\childdocredirect}[2][]
{
  \begingroup
    \if?#1?
      \def\childdoctmp{\childdocforward{#2}}
    \else
      \def\childdoctmp{\childdocforwardprefix{#1}{#2}}
    \fi
    \expandafter
  \endgroup
  \childdoctmp
}
%    \end{macrocode}

%\iffalse
%</package>
%\fi
%
\endinput
|\\
|\childdocby{|\textit{main}|}|\\
\end{tabular}
\end{center}
%
Both forms have slightly different effects as described above.
The main file is prepared as usual, see \secref{sec:include}.

%%%%%%%%%%%%%%%%%%%%%%%%%%%%%%%%%%%%%%%%%%%%%%%%%%%%%%%%%%%%%%%%%%%%%%%%%%%%%%%%
\subsection{Legacy Detection}
\label{sec:detection}

The directive |\childdocmain| in the main file can detect
whether the complete document or merely a child is to be compiled
even without using the directive |\childdocof|.
This method is deprecated because it is less robust
and there is no compelling reason to use it;
it is merely provided for backward compatibility
and it may be removed in future versions.

If the detection mechanism is to be used,
it is mandatory to correctly specify
the filename of the main file as the argument of |\childdocmain|:
%
\begin{center}
\begin{tabular}{l}
|% \iffalse
%
% childdoc.dtx Copyright (C) 2017-2018 Niklas Beisert
%
% This work may be distributed and/or modified under the
% conditions of the LaTeX Project Public License, either version 1.3
% of this license or (at your option) any later version.
% The latest version of this license is in
%   http://www.latex-project.org/lppl.txt
% and version 1.3 or later is part of all distributions of LaTeX
% version 2005/12/01 or later.
%
% This work has the LPPL maintenance status `maintained'.
%
% The Current Maintainer of this work is Niklas Beisert.
%
% This work consists of the files childdoc.dtx and childdoc.ins
% and the derived files childdoc.def and cdocsamp.tex with
% cdocsch1.tex, cdocsch2.tex, cdocsdrf.tex, cdocsfn1.tex, cdocsfn2.tex.
%
%<package>\ifdefined\childdocmain\endinput\fi
%<package>\ProvidesFile{childdoc.def}[2018/12/30 v2.0 child document driver]
%<samplemain>\ProvidesFile{cdocsamp.tex}[2018/12/30 v2.0 sample for childdoc]
%<*driver>
%\ProvidesFile{childdoc.drv}[2018/12/30 v2.0 childdoc reference manual file]
\PassOptionsToClass{10pt,a4paper}{article}
\documentclass{ltxdoc}

\usepackage[margin=35mm]{geometry}
\usepackage{hyperref}
\usepackage{hyperxmp}
\usepackage[usenames]{color}

\hypersetup{colorlinks=true}
\hypersetup{pdfstartview=FitH}
\hypersetup{pdfpagemode=UseNone}
\hypersetup{pdfsource={}}
\hypersetup{pdflang={en-UK}}
\hypersetup{pdfcopyright={Copyright 2017-2018 Niklas Beisert.
  This work may be distributed and/or modified under the
  conditions of the LaTeX Project Public License, either version 1.3
  of this license or (at your option) any later version.}}
\hypersetup{pdflicenseurl={http://www.latex-project.org/lppl.txt}}
\hypersetup{pdfcontactaddress={ETH Zurich, ITP, HIT K,
  Wolfgang-Pauli-Strasse 27}}
\hypersetup{pdfcontactpostcode={8093}}
\hypersetup{pdfcontactcity={Zurich}}
\hypersetup{pdfcontactcountry={Switzerland}}
\hypersetup{pdfcontactemail={nbeisert@itp.phys.ethz.ch}}
\hypersetup{pdfcontacturl={http://people.phys.ethz.ch/\xmptilde nbeisert/}}

\newcommand{\secref}[1]{\hyperref[#1]{section \ref*{#1}}}

\parskip1ex
\parindent0pt
\let\olditemize\itemize
\def\itemize{\olditemize\parskip0pt}

\begin{document}

\title{The \textsf{childdoc} Package}
\hypersetup{pdftitle={The childdoc Package}}
\author{Niklas Beisert\\[2ex]
  Institut f\"ur Theoretische Physik\\
  Eidgen\"ossische Technische Hochschule Z\"urich\\
  Wolfgang-Pauli-Strasse 27, 8093 Z\"urich, Switzerland\\[1ex]
  \href{mailto:nbeisert@itp.phys.ethz.ch}
  {\texttt{nbeisert@itp.phys.ethz.ch}}}
\hypersetup{pdfauthor={Niklas Beisert}}
\hypersetup{pdfsubject={Manual for the LaTeX2e Package childdoc}}
\date{30 December 2018, \textsf{v2.0}}
\maketitle

\begin{abstract}\noindent
\textsf{childdoc} is a \LaTeXe{} package
that enables the direct compilation
of document sections included by |\include|
to individual files.
\end{abstract}

\begingroup
\parskip0ex
\tableofcontents
\endgroup

%%%%%%%%%%%%%%%%%%%%%%%%%%%%%%%%%%%%%%%%%%%%%%%%%%%%%%%%%%%%%%%%%%%%%%%%%%%%%%%%
%%%%%%%%%%%%%%%%%%%%%%%%%%%%%%%%%%%%%%%%%%%%%%%%%%%%%%%%%%%%%%%%%%%%%%%%%%%%%%%%
\section{Introduction}

\LaTeX{} provides a mechanism to structure a large document (such as a book)
into a main file and several child files (containing the chapters)
using the |\include| command.
This mechanism is beneficial for documents
which span hundreds of pages in order to
make the source file(s) more manageable.
Moreover, compilation can be restricted to
selected child files by means of the |\includeonly| command.
The latter feature can be used to reduce the compilation time while editing
(this was significantly more useful in the earlier days of \LaTeX{})
or to generate a smaller document which is easier to navigate.
Another application of |\includeonly| is to generate
documents consisting of selected parts of the complete document.

However, there are a few drawbacks of the plain |\include| mechanism:
\begin{itemize}
\item
The child files cannot be compiled on their own,
they can only be compiled via the main file.
A naive editing environment
(such as a text editor with an option
to have the current file processed by \LaTeX)
may require one to switch to the main file before compiling;
attempting to compile the child file produces errors.
\item
The main file must be modified (each time)
to adjust the |\includeonly| command
to the present needs. This easily leaves the main file in a messy state.
\item
The generated document will always carry the filename
of the main document. This is inconvenient if
several child files are to be compiled and
to be kept for distribution.
\end{itemize}

The present package provides a simple interface
to make child files individually compilable by \LaTeX{}.
Compiling a child file then has the same effect as compiling
the main file with an |\includeonly| command
to select the appropriate child.
Moreover the generated document will carry the name of the child
rather than the main file.
This resolves all three above issues.

This feature is meant to make the editing of books,
thesis documents and lecture notes somewhat more convenient.
However, the package can also be used efficiently for
composing a series of documents (such as exercise sheets)
which are typically distributed individually.
It then assists the author in generating the individual documents
(potentially in different versions)
as well as a document containing the collected series.
Another application is in developing style files
or other kinds of included material
where compilation of the style file could redirect
to a sample or test file.

%%%%%%%%%%%%%%%%%%%%%%%%%%%%%%%%%%%%%%%%%%%%%%%%%%%%%%%%%%%%%%%%%%%%%%%%%%%%%%%%
%%%%%%%%%%%%%%%%%%%%%%%%%%%%%%%%%%%%%%%%%%%%%%%%%%%%%%%%%%%%%%%%%%%%%%%%%%%%%%%%
\section{Usage}

First of all, the package \textsf{childdoc} is \emph{not} a standard
\LaTeXe{} |.sty| style file! Therefore it needs to be invoked in
a non-standard way.

%%%%%%%%%%%%%%%%%%%%%%%%%%%%%%%%%%%%%%%%%%%%%%%%%%%%%%%%%%%%%%%%%%%%%%%%%%%%%%%%
\subsection{Included Files}
\label{sec:include}

%%%%%%%%%%%%%%%%%%%%%%%%%%%%%%%%%%%%%%%%
\DescribeMacro{\childdocmain}
To use the package, add the commands
\begin{center}
\begin{tabular}{l}
|\input{childdoc.def}|\\
|\childdocmain{}|\\
\end{tabular}
\end{center}
at the very top of the main \LaTeX{} file,
in particular \emph{before} the |\documentclass| statement!
The argument of |\childdocmain| should be left empty
(but it must be present).

%%%%%%%%%%%%%%%%%%%%%%%%%%%%%%%%%%%%%%%%
\DescribeMacro{\childdocof}
Furthermore, add the commands
\begin{center}
\begin{tabular}{l}
|\input{childdoc.def}|\\
|\childdocof{|\textit{main}|}|\\
\end{tabular}
\end{center}
at the top of every child file \textit{child}
which is included by |\include{|\textit{child}|}|
from within the main file
(or at least for those files to be compiled individually).
The argument \textit{main} must be the filename of the main file.

There are a couple of
considerations in setting up the main and child documents:

%%%%%%%%%%%%%%%%%%%%%%%%%%%%%%%%%%%%%%%%
\paragraph{Restrictions.}

Please note the following restrictions:
\begin{itemize}
\item
|\childdocmain| must be called with one argument \textit{main}
to ensure compatibility with earlier version of the package.
It must either be empty (|\childdocmain{}|)
or precisely match the filename of the main file in which it is specified.
See \secref{sec:detection} for further information.
\item
The filename \textit{main} must be specified without the |.tex| extension.
\item
The filename \textit{main} is case sensitive
(even in case-insensitive file systems)
due to internal string comparison.
\item
The argument \textit{main} should be fully expanded, it cannot be a macro.
\item
Subdirectories and special characters should be avoided in filenames.
\item
The command |\childdocmain{|\textit{main}|}| must be followed by a whitespace.
It should not be followed immediately by another command
or by a comment mark `|%|'.
This is because the \TeX{} parser reads the token immediately following
the argument of |\childdocmain| and puts it
at the beginning of every child section;
however, a white\-space is ignored.
\end{itemize}

%%%%%%%%%%%%%%%%%%%%%%%%%%%%%%%%%%%%%%%%
\paragraph{Content of Main File.}

It is advisable to place all content in the child files included by |\include|.
Any output contained in the main file will appear in all child documents
unless suppressed manually;
it cannot be suppressed automatically by the |\includeonly| directive
and thus should normally be avoided.
A method to include some content in the main file
by means of conditional processing is described in \secref{sec:conditional}.

%%%%%%%%%%%%%%%%%%%%%%%%%%%%%%%%%%%%%%%%
\paragraph{Page Numbering.}

When only a part of the document is compiled,
the appropriate numbering of pages
(as well as other status parameters)
is determined from the |.aux| files.
The latter contain information from previous passes.
However this information needs to propagate through
all intermediate child documents.
Therefore the page numbering in child documents may well
be inconsistent until the complete document is compiled at least once.

A useful (if unconventional) way to always ensure a consistent
page numbering is to restart the numbering in each child document
and denote the pages by `\textit{child}|.|\textit{page}'
where \textit{child} represents the chapter/section number of the child file.
This can be achieved by the command
|\numberwithin{page}{|\textit{child}|}|
of the \textsf{amsmath} package
where \textit{child} can be |chapter| or |section|
depending on the chosen structuring.
Alternatively, one can modify the macro |\thepage| appropriately
and reset the counter |page| at the start of each child file.

%%%%%%%%%%%%%%%%%%%%%%%%%%%%%%%%%%%%%%%%%%%%%%%%%%%%%%%%%%%%%%%%%%%%%%%%%%%%%%%%
\subsection{Conditional Processing}
\label{sec:conditional}

The package provides a mechanism to compile different versions
of a document. To customise the versions further some conditional processing
can come in handy to distinguish which version is being compiled.
The package provides two macros to describe the compilation context:

%%%%%%%%%%%%%%%%%%%%%%%%%%%%%%%%%%%%%%%%
\DescribeMacro{\ifchilddoc}
The conditional |\ifchilddoc| distinguishes between the compilation of
child documents and the main document:
%
\begin{center}
|\ifchilddoc |\textit{child-code}| |[|\||else |\textit{main-code}]| \||fi|
\end{center}

%%%%%%%%%%%%%%%%%%%%%%%%%%%%%%%%%%%%%%%%
\DescribeMacro{\childdocname}
\DescribeMacro{\childdocjob}
The macro |\childdocname| contains the filename (without extension)
of the main or child file being processed.
Note that |\childdocjob| will always contain the name of the main file.

%%%%%%%%%%%%%%%%%%%%%%%%%%%%%%%%%%%%%%%%
\paragraph{Title Page.}

Conditional processing can be used to include a title or banner page
in the main document when proper precautions are taken.
Importantly, the code in the main file should ensure that the page counter
(as well as other status parameters which are stored in the |.aux| files)
takes the same value after the conditional processing.
Otherwise the page numbers may take divergent values
depending on which part is compiled.

For example, a title page could be declared by:
%
\begin{center}
\begin{tabular}{l}
|\ifchilddoc\||else|\\
|\addtocounter{page}{-1}|\\
\textit{code for title page}\\
|\newpage|\\
|\||fi|
\end{tabular}
\end{center}
%
A banner page for the child documents can be generated by:
%
\begin{center}
\begin{tabular}{l}
|\ifchilddoc|\\
|\addtocounter{page}{-1}|\\
\textit{code for banner page}\\
|\newpage|\\
|\||fi|
\end{tabular}
\end{center}
%
Here one could write a message such as:
\begin{center}
|This is the part \childdocname{} of \childdocjob{}.|
\end{center}

%%%%%%%%%%%%%%%%%%%%%%%%%%%%%%%%%%%%%%%%%%%%%%%%%%%%%%%%%%%%%%%%%%%%%%%%%%%%%%%%
\subsection{Flags}
\label{sec:flags}

The package makes it easy to generate different versions
of the main or child documents.
To this end compilation flags can be defined
and assigned different default values.
They will be particularly useful in conjunction
with the forwarding mechanism described in \secref{sec:forward}.

For example, it may be useful to have a flag |\version|
which can be set to |draft| or |final|.
The document source will contain some conditional code
depending on the value of |\version|.
Suppose further, the flag should default to |final| for the main file
and to |draft| for child files
which is a natural assignment for editing the document.
This is achieved by placing the following code
in the preamble of the main document
(below the |\childdocmain| directive):
%
\begin{center}
\begin{tabular}{l}
|\ifchilddoc|\\
|\providecommand{\version}{draft}|\\
|\||else|\\
|\providecommand{\version}{final}|\\
|\||fi|
\end{tabular}
\end{center}
%
The definition by |\providecommand| makes sure
that previous definitions are not overwritten.
Further statements |\providecommand{\version}{...}|
can thus be added before the above code to override it.

For the main file, one might add a line
(between |\childdocmain| and the above block)
%
\begin{center}
|%\ifchilddoc\||else\providecommand{\version}{draft}\||fi|
\end{center}
%
which can be uncommented to produce a draft version.
Likewise one can add a line to the very top of a child file
(above the |\childdocof{|\textit{main}|}| directive)
%
\begin{center}
|%\providecommand{\version}{final}|
\end{center}
%
which can be uncommented to produce the final version of this child document.

%%%%%%%%%%%%%%%%%%%%%%%%%%%%%%%%%%%%%%%%%%%%%%%%%%%%%%%%%%%%%%%%%%%%%%%%%%%%%%%%
\subsection{Forwarding}
\label{sec:forward}

Different versions of the main or child documents
using compilation flags as described in \secref{sec:flags}
can be (permanently) stored in different files
for convenient compilation, viewing and distribution.
To this end, the package defines a command
to pass on compilation to a different file:

%%%%%%%%%%%%%%%%%%%%%%%%%%%%%%%%%%%%%%%%
\DescribeMacro{\childdocforward}
The command |\childdocforward| redirects processing to
another source file:
%
\begin{center}
\begin{tabular}{l}
|\input{childdoc.def}|\\
|\childdocforward[|\textit{main}|]{|\textit{dest}|}|\\
\end{tabular}
\end{center}
%
The argument \textit{dest} is the destination file
(without extension).
It should be the main file or one of the child files.
Note that further \textsf{childdoc} directives
such as |\childdocof| and |\childdocforward|
in the indicated file will be processed in this form.
The optional argument \textit{main}
passes on directly to the main file \textit{main}
while pretending to compile the child \textit{dest}.
This form behaves as if \textit{dest}
issues |\childdocof{|\textit{main}|}| right away,
and no further \textsf{childdoc} directives will be processed.

%%%%%%%%%%%%%%%%%%%%%%%%%%%%%%%%%%%%%%%%
\DescribeMacro{\...prefix}
In the alternative form |\childdocforwardprefix|,
%
\begin{center}
\begin{tabular}{l}
|\input{childdoc.def}|\\
|\childdocforwardprefix[|\textit{main}|]{|\textit{prefix}|}{|\textit{dest}|}|
\end{tabular}
\end{center}
%
the destination file is determined by a pattern
depending on the current file:
To make this work, the current file must be called
`{\textit{prefix}\hspace{0.2em}\textit{suffix}}'
with \textit{prefix} matching precisely the argument.
Processing is then passed on to the file
`{\textit{dest}\hspace{0.2em}\textit{suffix}}'.
Surely, the same effect is achieved by
directly specifying the
argument `{\textit{dest}\hspace{0.2em}\textit{suffix}}'
in the first form.
However, that requires to set up a different file
for each child. With the alternative form of the command
all these files can have exactly the same content
which simplifies setting them up and maintaining them.

For example, the following file |draft.tex|
with a compilation flag |\version| as described in \secref{sec:flags}
compiles the main document as a draft:
%
\begin{center}
\begin{tabular}{l}
|\def\version{draft}|\\
|\input{childdoc.def}|\\
|\childdocforward{|\textit{main}|}|
\end{tabular}
\end{center}
%
Likewise, the following files |final|\textit{nn}|.tex|
compile the final version of the child document
|child|\textit{nn}|.tex|:
%
\begin{center}
\begin{tabular}{l}
|\def\version{final}|\\
|\input{childdoc.def}|\\
|\childdocforwardprefix{final}{child}|
\end{tabular}
\end{center}
%

Note that when several versions of a main file and/or of each child file
are to be generated, it may be convenient to set up a |Makefile| or
shell script to automatise the process.

%%%%%%%%%%%%%%%%%%%%%%%%%%%%%%%%%%%%%%%%%%%%%%%%%%%%%%%%%%%%%%%%%%%%%%%%%%%%%%%%
\subsection{Command Line Processing}
\label{sec:commandline}

The effect of redirection files can also be achieved by invoking
the \LaTeX{} compiler with a more elaborate command line.
Most conveniently this should be done as part
of a shell script or a |Makefile|.

When using \textsf{childdoc} in the main file, the following
command lines effectively perform a redirection
(note that depending on the shell being used,
backslashes may have to be doubled: `|\|' $\to$ `|\\|'):
%
\begin{center}
|... -jobname "|\textit{target}|" |\\|"|[\textit{flags}]%
|\input{childdoc.def}\childdocforward[|\textit{main}|]{|\textit{dest}|}"|
\end{center}
%
Here \textit{target} is the name of the output file,
\textit{main} is the name of the main file
and \textit{dest} is the name of the main or child file to be processed
(all filenames without extensions).
The optional argument \textit{main} can be omitted
if \textit{main} matches \textit{dest}.
Optionally, compilation \textit{flags} can be defined via |\def| commands.
This command line makes the \TeX{} engine believe
it is compiling the file \textit{target}
whose content is specified as the latter parameter.
The provided code then forwards the processing to
\textit{main} or \textit{dest} as described in \secref{sec:forward}.

%%%%%%%%%%%%%%%%%%%%%%%%%%%%%%%%%%%%%%%%%%%%%%%%%%%%%%%%%%%%%%%%%%%%%%%%%%%%%%%%
\subsection{Include by Input}
\label{sec:input}

Including child documents by |\include| has some restrictions by design.
Most notably, the content of a child document always occupies
its own set of pages; pages cannot be shared between child documents.
Usually, this behaviour makes perfect sense
because each child document contain an essential part of the document.
However, in some situations it may be desirable to compose
a document from a collection of parts
without having mandatory page breaks between then.
For this case, the package
provides a mechanism to include parts
by |\input| which can also be processed individually.
However, by construction this mechanism
requires manual handling of the content to be output.

%%%%%%%%%%%%%%%%%%%%%%%%%%%%%%%%%%%%%%%%
\DescribeMacro{\ifchilddocmanual}
The main file should be prepared as usual, see \secref{sec:include}.
However, the document body must make a distinction
between processing of an individual part and of the main document, e.g.:
%
\begin{center}
\begin{tabular}{l}
|\ifchilddocmanual|\\
|\input{\childdocname}|\\
|\||else|\\
\textit{document body with }|\input{|\textit{part}|}|\\
|\||fi|
\end{tabular}
\end{center}
%
The conditional |\ifchilddocmanual| is true whenever
a part to be included by |\input| is being compiled,
and the name of the part is stored in |\childdocname|.

%%%%%%%%%%%%%%%%%%%%%%%%%%%%%%%%%%%%%%%%
\DescribeMacro{\childdocby}
Each part to be included by |\input| should start with:
%
\begin{center}
\begin{tabular}{l}
|\input{childdoc.def}|\\
|\childdocby{|\textit{main}|}|\\
\end{tabular}
\end{center}
%
The directive |\childdocby| is similar to |\childdocof|
described in \secref{sec:include},
but the subsequent selection of content must be done manually.
To that end, both |\ifchilddoc| and |\ifchilddocmanual|
will be true upon processing of a part,
and the name of the part is stored in |\childdocname|.
Note that |\jobname| will be set to the filename of the current part
so that each part receives an individual |.aux| file
that does not interfere with the |.aux| file(s) of the main document.
This behaviour can be altered by the alternative form
|\childdocby[*]{|\textit{main}|}| (with a non-empty optional argument)
which uses the |.aux| file of the main document
by setting |\jobname| to \textit{main}.

%%%%%%%%%%%%%%%%%%%%%%%%%%%%%%%%%%%%%%%%%%%%%%%%%%%%%%%%%%%%%%%%%%%%%%%%%%%%%%%%
\subsection{Driver Development}
\label{sec:driver}

The \textsf{childdoc} mechanism can also be use for the development
of definition files such as \LaTeX{} styles or classes.
This case differs from the above setup with multiple parts
included by |\include| in that no |\includeonly| should be invoked.
This can be achieved by starting the include file
(before |\ProvidesPackage|) with:
%
\begin{center}
\begin{tabular}{l}
|\input{childdoc.def}|\\
|\childdocforward{|\textit{main}|}|\\
\end{tabular}
\end{center}
%
or alternatively with:
%
\begin{center}
\begin{tabular}{l}
|\input{childdoc.def}|\\
|\childdocby{|\textit{main}|}|\\
\end{tabular}
\end{center}
%
Both forms have slightly different effects as described above.
The main file is prepared as usual, see \secref{sec:include}.

%%%%%%%%%%%%%%%%%%%%%%%%%%%%%%%%%%%%%%%%%%%%%%%%%%%%%%%%%%%%%%%%%%%%%%%%%%%%%%%%
\subsection{Legacy Detection}
\label{sec:detection}

The directive |\childdocmain| in the main file can detect
whether the complete document or merely a child is to be compiled
even without using the directive |\childdocof|.
This method is deprecated because it is less robust
and there is no compelling reason to use it;
it is merely provided for backward compatibility
and it may be removed in future versions.

If the detection mechanism is to be used,
it is mandatory to correctly specify
the filename of the main file as the argument of |\childdocmain|:
%
\begin{center}
\begin{tabular}{l}
|\input{childdoc.def}|\\
|\childdocmain{|\textit{main}|}|\\
\end{tabular}
\end{center}
%
If |\jobname| does not match the argument \textit{main} of |\childdocmain|,
it is assumed that |\jobname| points to the child file to be compiled.
When using |\childdocmain| with the main file specified as argument,
it suffices to start a child file
with just |\input{|\textit{main}|}|
without loading of the package and using |\childdocof|.
If instead all processing is done
with the appropriate \textsf{childdoc} directives,
the argument of \textit{main} of |\childdocmain| can be empty.

An alternative version of the command line processing described
in \secref{sec:commandline} using the detection mechanism reads:
%
\begin{center}
|... -jobname "|\textit{target}|" "|[\textit{flags}]%
[|\def\jobname{|\textit{dest}|}|]|\input{|\textit{main}|}"|
\end{center}

%%%%%%%%%%%%%%%%%%%%%%%%%%%%%%%%%%%%%%%%%%%%%%%%%%%%%%%%%%%%%%%%%%%%%%%%%%%%%%%%
\subsection{Manual Code}
\label{sec:manual}

In case one cannot be certain whether the definitions file |childdoc.def|
is installed on the target \TeX{} distribution
and one prefers not to ship it,
it is conceivable to paste a few relevant commands into the sources.

To that end, drop all statements |\input{childdoc.def}|
and perform the replacements as outlined below.
Instead of |\childdocmain{|\textit{main}|}| add the following code
to the top of the main file:
%
\begin{center}
\begin{tabular}{l}
|\||ifdefined\childdocname\endinput\||fi\newif\ifchilddoc|\\
|\edef\childdocname{\scantokens\expandafter{\jobname\noexpand}}|\\
|\def\childdocmain{|\textit{main}|}\||ifx\childdocmain\childdocname\||else|\\
|\childdoctrue\includeonly{\childdocname}\let\jobname\childdocmain\||fi|\\
\end{tabular}
\end{center}
%
Instead of |\childdocof{|\textit{main}|}| just include the main file
at the top of each child file:
%
\begin{center}
|\input{|\textit{main}|}|
\end{center}
%
A simple redirection |\childdocforward{|\textit{dest}|}| is achieved by:
%
\begin{center}
|\def\jobname{|\textit{dest}|}\input{\jobname}|
\end{center}
%
The redirection with prefix
|\childdocforwardprefix[|\textit{prefix}|]{|\textit{dest}|}|
is accomplished by:
%
\begin{center}
\begin{tabular}{l}
|{\edef\jobname{\scantokens\expandafter{\jobname\noexpand}}|\\
|\def\redirectjob |\textit{prefix}|#1~~~{\gdef\jobname{|\textit{dest}|#1}}|\\
|\expandafter\redirectjob\jobname~~~}\input{\jobname}|
\end{tabular}
\end{center}

In an alternative approach,
child documents can be compiled by a specific command line
without additional code or specific definitions:
%
\begin{center}
|... -jobname "|\textit{target}|" "|[\textit{flags}]%
|\includeonly{|\textit{dest}|}\input{|\textit{main}|}"|
\end{center}
%

%%%%%%%%%%%%%%%%%%%%%%%%%%%%%%%%%%%%%%%%%%%%%%%%%%%%%%%%%%%%%%%%%%%%%%%%%%%%%%%%
%%%%%%%%%%%%%%%%%%%%%%%%%%%%%%%%%%%%%%%%%%%%%%%%%%%%%%%%%%%%%%%%%%%%%%%%%%%%%%%%
\section{Information}

%%%%%%%%%%%%%%%%%%%%%%%%%%%%%%%%%%%%%%%%%%%%%%%%%%%%%%%%%%%%%%%%%%%%%%%%%%%%%%%%
\subsection{Copyright}

Copyright \copyright{} 2017--2018 Niklas Beisert

This work may be distributed and/or modified under the
conditions of the \LaTeX{} Project Public License, either version 1.3
of this license or (at your option) any later version.
The latest version of this license is in
  \url{http://www.latex-project.org/lppl.txt}
and version 1.3 or later is part of all distributions of \LaTeX{}
version 2005/12/01 or later.

This work has the LPPL maintenance status `maintained'.

The Current Maintainer of this work is Niklas Beisert.

This work consists of the files |README.txt|, |childdoc.ins| and |childdoc.dtx|
as well as the derived files |childdoc.def|, |cdocsamp.tex|
with |cdocsch1.tex|, |cdocsch2.tex|, |cdocspt3.tex|, |cdocspt4.tex|,
|cdocsdrf.tex|, |cdocsfn1.tex|, |cdocsfn2.tex|
as well as |childdoc.pdf|.

%%%%%%%%%%%%%%%%%%%%%%%%%%%%%%%%%%%%%%%%%%%%%%%%%%%%%%%%%%%%%%%%%%%%%%%%%%%%%%%%
\subsection{Files and Installation}

The package consists of the files:
%
\begin{center}
\begin{tabular}{ll}
    |README.txt|   & readme file \\
    |childdoc.ins| & installation file \\
    |childdoc.dtx| & source file \\
    |childdoc.def| & definition file \\
    |cdocsamp.tex| & sample main file \\
    |cdocsch1.tex| & sample include file \\
    |cdocsch2.tex| & sample include file \\
    |cdocspt3.tex| & sample part file \\
    |cdocspt4.tex| & sample part file \\
    |cdocsdrf.tex| & sample redirection file \\
    |cdocsfn1.tex| & sample redirection file \\
    |cdocsfn2.tex| & sample redirection file \\
    |childdoc.pdf| & manual
\end{tabular}
\end{center}
%
The distribution consists of the files
|README.txt|, |childdoc.ins| and |childdoc.dtx|.
%
\begin{itemize}
\item
Run (pdf)\LaTeX{} on |childdoc.dtx|
to compile the manual |childdoc.pdf| (this file).
\item
Run \LaTeX{} on |childdoc.ins| to create the definitions file |childdoc.def|
and the sample |cdocsamp.tex| with include files
|cdocsch1.tex|, |cdocsch2.tex|, |cdocspt3.tex|, |cdocspt4.tex|,
|cdocsdrf.tex|, |cdocsfn1.tex|, |cdocsfn2.tex|.
Then copy the file |childdoc.def| to an appropriate directory of your \LaTeX{}
distribution, e.g.\ \textit{texmf-root}|/tex/latex/childdoc|.
\end{itemize}

%%%%%%%%%%%%%%%%%%%%%%%%%%%%%%%%%%%%%%%%%%%%%%%%%%%%%%%%%%%%%%%%%%%%%%%%%%%%%%%%
\subsection{Related CTAN Packages}

There are several other packages which offer a similar functionality:
%
\begin{itemize}
\item
The packages
\href{http://ctan.org/pkg/docmute}{\textsf{docmute}},
\href{http://ctan.org/pkg/includex}{\textsf{includex}} and
\href{http://ctan.org/pkg/standalone}{\textsf{standalone}}
provide commands to include only the document body of
a child file thus allowing both files to be compiled individually.
\item
The packages \href{http://ctan.org/pkg/subdocs}{\textsf{subdocs}}
and \href{http://ctan.org/pkg/subfiles}{\textsf{subfiles}}
provide structures in which the main and child documents can be
encapsulated and allowing them to be compiled individually.
The inclusion mechanism is different from the conventional |\include|.
\item
The package \href{http://ctan.org/pkg/combine}{\textsf{combine}}
is an elaborate solution to combine several documents into one.
\end{itemize}
%
See also the CTAN topic \href{http://ctan.org/topic/subdocs}{\textsf{subdocs}}
for further related packages.
The present package differs from the above solutions in that
a document structure constructed with the conventional |\include| mechanism
just needs two extra commands at the top of every file
such that all constituent files can be compiled individually.

%%%%%%%%%%%%%%%%%%%%%%%%%%%%%%%%%%%%%%%%%%%%%%%%%%%%%%%%%%%%%%%%%%%%%%%%%%%%%%%%
%\subsection{Feature Suggestions}
%
%The following is a list of features which may be useful for future
%versions of this package:
%%
%\begin{itemize}
%\item
%\ldots
%\end{itemize}

%%%%%%%%%%%%%%%%%%%%%%%%%%%%%%%%%%%%%%%%%%%%%%%%%%%%%%%%%%%%%%%%%%%%%%%%%%%%%%%%
\subsection{Revision History}

%%%%%%%%%%%%%%%%%%%%%%%%%%%%%%%%%%%%%%%%
\paragraph{v2.0:} 2018/12/30

\begin{itemize}
\item
immediate forward processing
\item
added |\childdocby| mechanism
\item
manual restructured
\end{itemize}

%%%%%%%%%%%%%%%%%%%%%%%%%%%%%%%%%%%%%%%%
\paragraph{v1.6:} 2018/01/17

\begin{itemize}
\item
application for development of include files
\item
corrections to manual
\end{itemize}

%%%%%%%%%%%%%%%%%%%%%%%%%%%%%%%%%%%%%%%%
\paragraph{v1.5:} 2017/05/21

\begin{itemize}
\item
more complete structuring introduced
\item
|\childdocof| introduced
\item
|\childdoc| renamed to |\childdocmain|
\item
|\childredirect| renamed to |\childdocforward| and |\childdocforwardprefix|
and functionality expanded
\end{itemize}

%%%%%%%%%%%%%%%%%%%%%%%%%%%%%%%%%%%%%%%%
\paragraph{v1.0:} 2017/04/27

\begin{itemize}
\item
manual and install package
\item
first version published on CTAN
\end{itemize}

%%%%%%%%%%%%%%%%%%%%%%%%%%%%%%%%%%%%%%%%
\paragraph{v0.6:} 2017/04/26

\begin{itemize}
\item
redirection mechanism added
\end{itemize}

%%%%%%%%%%%%%%%%%%%%%%%%%%%%%%%%%%%%%%%%
\paragraph{v0.5:} 2017/04/26

\begin{itemize}
\item
functionality in definition file
\end{itemize}


%%%%%%%%%%%%%%%%%%%%%%%%%%%%%%%%%%%%%%%%%%%%%%%%%%%%%%%%%%%%%%%%%%%%%%%%%%%%%%%%
%%%%%%%%%%%%%%%%%%%%%%%%%%%%%%%%%%%%%%%%%%%%%%%%%%%%%%%%%%%%%%%%%%%%%%%%%%%%%%%%
%%%%%%%%%%%%%%%%%%%%%%%%%%%%%%%%%%%%%%%%%%%%%%%%%%%%%%%%%%%%%%%%%%%%%%%%%%%%%%%%
\appendix

\settowidth\MacroIndent{\rmfamily\scriptsize 000\ }

 \DocInput{childdoc.dtx}

\end{document}
%</driver>
% \fi
%
% %%%%%%%%%%%%%%%%%%%%%%%%%%%%%%%%%%%%%%%%%%%%%%%%%%%%%%%%%%%%%%%%%%%%%%%%%%%%%%
% %%%%%%%%%%%%%%%%%%%%%%%%%%%%%%%%%%%%%%%%%%%%%%%%%%%%%%%%%%%%%%%%%%%%%%%%%%%%%%
% \section{Sample}
%\iffalse
%<*samplemain>
%\fi
%
% The following presents a sample document
% with two chapters, two parts, a title page,
% a compile flag as well as three forwarding files to set the flag.
% It consists of eight |.tex| files:
% \begin{center}
% \begin{tabular}{ll}
% |cdocsamp.tex|&main file\\
% |cdocsch1.tex|&include file for chapter 1\\
% |cdocsch2.tex|&include file for chapter 2\\
% |cdocspt3.tex|&include file for part 3\\
% |cdocspt4.tex|&include file for part 4\\
% |cdocsdrf.tex|&forwarding file for main file in draft mode\\
% |cdocsfi1.tex|&forwarding file for final version of chapter 1\\
% |cdocsfi2.tex|&forwarding file for final version of chapter 2\\
% \end{tabular}
% \end{center}
% Each of the eight files can be compiled directly by the \LaTeX{} compiler.
%
% %%%%%%%%%%%%%%%%%%%%%%%%%%%%%%%%%%%%%%
% \paragraph{Main File.}
%
% The main file is called |cdocsamp.tex|.
%
% Load the \textsf{childdoc} definitions and
% declare the filename for the main document:
%    \begin{macrocode}
\input{childdoc.def}
\childdocmain{}
%    \end{macrocode}

% Optional override for |\version| flag:
%    \begin{macrocode}
%%\ifchilddoc\else\providecommand{\version}{draft}\fi
%    \end{macrocode}

% Define the default values for the |\version| flag
% (|final| for the main file and |draft| for childs):
%    \begin{macrocode}
\ifchilddoc
\providecommand{\version}{draft}
\else
\providecommand{\version}{final}
\fi
%    \end{macrocode}

% Load the standard document class:
%    \begin{macrocode}
\documentclass[12pt]{article}
%    \end{macrocode}

% Start the document body:
%    \begin{macrocode}
\begin{document}
%    \end{macrocode}

% Declare a title page.
% Print title, part of document being processed and version flag:
%    \begin{macrocode}
\addtocounter{page}{-1}
\begin{center}
{\LARGE\bfseries{}childdoc example\par}
\vspace{1cm}
\ifchilddoc
\ifchilddocmanual part\else chapter\fi:
`\childdocname' of `\childdocjob'\par
\else
main document: `\childdocjob'\par
\fi
version: \version\par
\end{center}
\newpage
%    \end{macrocode}

% Manually include selected file,
% otherwise process as usual:
%    \begin{macrocode}
\ifchilddocmanual
\section*{part `\childdocname'}
\input{\childdocname}
\else
%    \end{macrocode}

% Include the two chapters:
%    \begin{macrocode}
\include{cdocsch1}
\include{cdocsch2}
%    \end{macrocode}

% Include the two parts unless only chapters should be displayed:
%    \begin{macrocode}
\ifchilddoc\else
\section{part three}
\input{cdocspt3}
\section{part four}
\input{cdocspt4}
\fi
%    \end{macrocode}

% Process as usual until here:
%    \begin{macrocode}
\fi
%    \end{macrocode}

% End of document body:
%    \begin{macrocode}
\end{document}
%    \end{macrocode}
%\iffalse
%</samplemain>
%\fi
%
% %%%%%%%%%%%%%%%%%%%%%%%%%%%%%%%%%%%%%%
% \paragraph{Chapter Include Files.}
%
% The include files are called |cdocsch1.tex| and |cdocsch2.tex|.
%
%\iffalse
%<*samplechap1|samplechap2>
%\fi

% Optional override for |\version| flag:
%    \begin{macrocode}
%%\providecommand{\version}{final}
%    \end{macrocode}

% Include the main document:
%    \begin{macrocode}
\input{childdoc.def}
\childdocof{cdocsamp}
%    \end{macrocode}

%\iffalse
%</samplechap1|samplechap2>
%\fi
%
%\iffalse
%<*samplechap1>
%\fi
% Some text for chapter 1:
%    \begin{macrocode}
\section{one}
some text in chapter one
%    \end{macrocode}

%\iffalse
%</samplechap1>
%\fi
% Some text for chapter 2:
%\iffalse
%<*samplechap2>
%\fi
%    \begin{macrocode}
\section{two}
more text in chapter two
%    \end{macrocode}

%\iffalse
%</samplechap2>
%\fi
%
% %%%%%%%%%%%%%%%%%%%%%%%%%%%%%%%%%%%%%%
% \paragraph{Part Include Files.}
%
% The include files are called |cdocspt3.tex| and |cdocspt4.tex|.
%
%\iffalse
%<*samplepart3|samplepart4>
%\fi

% Optional override for |\version| flag:
%    \begin{macrocode}
%%\providecommand{\version}{final}
%    \end{macrocode}

% Include the main document:
%    \begin{macrocode}
\input{childdoc.def}
\childdocby{cdocsamp}
%    \end{macrocode}

%\iffalse
%</samplepart3|samplepart4>
%\fi
%
%\iffalse
%<*samplepart3>
%\fi
% Some text for part 3:
%    \begin{macrocode}
some text in part three
%    \end{macrocode}

%\iffalse
%</samplepart3>
%\fi
% Some text for part 4:
%\iffalse
%<*samplepart4>
%\fi
%    \begin{macrocode}
more text in part four
%    \end{macrocode}

%\iffalse
%</samplepart4>
%\fi
%
% %%%%%%%%%%%%%%%%%%%%%%%%%%%%%%%%%%%%%%
% \paragraph{Forwarding for a Complete Draft.}
%
% The following forwarding file |cdocsdrf.tex|
% compiles the main document in draft mode:
%\iffalse
%<*sampledraft>
%\fi
%    \begin{macrocode}
\def\version{draft}
\input{childdoc.def}
\childdocforward{cdocsamp}
%    \end{macrocode}

%\iffalse
%</sampledraft>
%\fi
%
% %%%%%%%%%%%%%%%%%%%%%%%%%%%%%%%%%%%%%%
% \paragraph{Forwarding for Final Version of the Chapters.}
%
% The following forwarding files |cdocsfn1.tex| and |cdocsfn2.tex|
% (with identical content)
% compile the final versions of the child documents
% |cdocsch1.tex| and |cdocsch2.tex|, respectively:
%\iffalse
%<*samplefinal>
%\fi
%    \begin{macrocode}
\def\version{final}
\input{childdoc.def}
\childdocforwardprefix[cdocsamp]{cdocsfn}{cdocsch}
%    \end{macrocode}

%\iffalse
%</samplefinal>
%\fi
%
% %%%%%%%%%%%%%%%%%%%%%%%%%%%%%%%%%%%%%%
% \paragraph{Command Line Processing.}
%
% The following three command lines generate the output files
% |cdocscld|, |cdocscl1| and |cdocscl2|
% which should be identical to
% |cdocsdrf|, |cdocsch1| and |cdocsfn2|, respectively:
% \begin{center}
% \begin{tabular}{l}
% |latex -jobname cdocscld \|\\
% |  "\def\version{draft}\input{childdoc.def}\childdocforward{cdocsamp}"|\\
% |latex -jobname cdocscl1 \|\\
% |  "\input{childdoc.def}\childdocforward[cdocsamp]{cdocsch1}"|\\
% |latex -jobname cdocscl2 \|\\
% |  "\def\version{final}\input{childdoc.def}\childdocforward{cdocsch2}"|
% \end{tabular}
% \end{center}
% Note that the trailing backslash on each first line
% merely continues the input to the second line
% (for convenient cut ant paste).
% Furthermore, the command |latex| can be replaced by any
% of its alternative versions such as |pdflatex|.
%
% %%%%%%%%%%%%%%%%%%%%%%%%%%%%%%%%%%%%%%%%%%%%%%%%%%%%%%%%%%%%%%%%%%%%%%%%%%%%%%
% %%%%%%%%%%%%%%%%%%%%%%%%%%%%%%%%%%%%%%%%%%%%%%%%%%%%%%%%%%%%%%%%%%%%%%%%%%%%%%
% \section{Implementation}
%\iffalse
%<*package>
%\fi
%
% This section describes the definitions file |childdoc.def|.

% The definitions cannot be loaded using |\usepackage| or |\RequirePackage|
% which has a mechanism to prevent loading a style file more than once.
% When loading the definitions by means of |\input|
% multiple instances have to be prevented manually:
%\iffalse
%This code needs to be before the `\ProvidesFile' directive
%which is defined at the beginning of this file.
%Therefore it is also placed there and commented out here.
%</package>
%<*discard>
%\fi
%    \begin{macrocode}
\ifdefined\childdocmain\endinput\fi
%    \end{macrocode}
%\iffalse
%</discard>
%<*package>
%\fi
%
% \macro{\ifchilddoc}
% \macro{\ifchilddocmanual}
% The conditional |\ifchilddoc| tells whether a
% child (true) or main (false) document is being compiled.
% The conditional |\ifchilddocmanual| tells whether
% the |\includeonly| mechanism is used (false) or
% the selection of child files must be performed manually (true).
% The definitions initialise to false:
%    \begin{macrocode}
\newif\ifchilddoc
\newif\ifchilddocmanual
%    \end{macrocode}

% \macro{\childdocname}
% \macro{\childdocjob}
% The macro |\childdocname| stores the name of the main document
% to be compiled. The macro |\childdocjob| stores the name of
% the document on which the \LaTeX{} compiler was originally invoked.
% The content of |\jobname| cannot be compared
% to filenames specified in the source due to different catcodes.
% The following code rescans |\jobname|, stores the result
% in |\childdocname| and saves a copy in |\childdocjob|:
%    \begin{macrocode}
\edef\childdocname{\scantokens\expandafter{\jobname\noexpand}}
\let\childdocjob\childdocname
%    \end{macrocode}

% \macro{\childdocdisable}
% The macro |\childdocdisable| prevents the main file
% from being processed more than once.
% At this stage, the main document command |\childdocmain|
% is assumed to be called once again where it should do nothing.
% Any subsequent call to it should prevent
% a secondary processing of the main document
% It overwrites the forwarding commands
% |\childdocof| and |\childdocforward|
% with empty macros to prevent further inclusions of the main document:
%    \begin{macrocode}
\newcommand{\childdocdisable}
{
  \renewcommand{\childdocmain}[1]{\renewcommand{\childdocmain}[1]{\endinput}}
  \renewcommand{\childdocof}[1]{}
  \renewcommand{\childdocby}[2][]{}
  \renewcommand{\childdocforward}[2][]{}
  \renewcommand{\childdocdisable}{}
}
%    \end{macrocode}

% \macro{\childdocmain}
% The macro |\childdocmain| is to be called at the top of the main file
% with nothing or the main filename (without extension) as argument.
% First, it breaks loops.
% If the argument is not empty and does not match |\childdocname|
% (which is set by the first inclusion of |childdoc.def|),
% |\ifchilddoc| is set to true, |\includeonly| is applied to the child file
% and |\jobname| is set to the main file
% (for proper handling of |.aux| files):
%    \begin{macrocode}
\newcommand{\childdocmain}[1]
{
  \childdocdisable\childdocmain{}
  \if?#1?\else
    \begingroup
      \def\childdoctmp{#1}
      \ifx\childdoctmp\childdocname
        \def\childdoctmp{}
      \else
        \def\childdoctmp
        {
          \childdoctrue
          \includeonly{\childdocname}
          \def\childdocjob{#1}
          \def\jobname{#1}
        }
      \fi
      \expandafter
    \endgroup
    \childdoctmp
  \fi
}
%    \end{macrocode}

% \macro{\childdocof}
% The command |\childdocof| redirects
% compilation to the main file |#1|.
%    \begin{macrocode}
\newcommand{\childdocof}[1]
{
  \childdocdisable
  \childdoctrue
  \includeonly{\childdocname}
  \def\jobname{#1}
  \def\childdocjob{#1}
  \input{#1}
}
%    \end{macrocode}

% \macro{\childdocby}
% The command |\childdocby| ....
%    \begin{macrocode}
\newcommand{\childdocby}[2][]
{
  \childdocdisable
  \childdoctrue
  \childdocmanualtrue
  \if?#1?\else
    \def\jobname{#2}
  \fi
  \def\childdocjob{#2}
  \input{#2}
  \endinput
}
%    \end{macrocode}

% \macro{\childdocforward}
% The command |\childdocforward| redirects
% compilation to the main file or
% (if the optional argument is given) a child file.
% Parameters are set as if the main file
% or a child file starting with |\childdocof| was compiled.
% Then compilation is handed over to the main file:
%    \begin{macrocode}
\newcommand{\childdocforward}[2][]
{
  \begingroup
    \if?#1?
      \def\childdoctmp
      {
        \def\childdocname{#2}
        \def\childdocjob{#2}
        \def\jobname{#2}
        \input{#2}
        \endinput
      }
    \else
      \def\childdoctmp
      {
        \childdocdisable
        \def\childdocname{#2}
        \childdoctrue
        \includeonly{#2}
        \def\childdocjob{#1}
        \def\jobname{#1}
        \input{#1}
        \endinput
      }
    \fi
    \expandafter
  \endgroup
  \childdoctmp
}
%    \end{macrocode}

% \macro{\childdocforwardprefix}
% The command |\childdocforwardprefix| redirects
% compilation to the main or a child file by means of a pattern.
% The prefix |#1| in the current filename is replaced by |#2|
% and the suffix of the current filename is kept
% (it is assumed that the filename does not contain the substring `|~~~|'
% which is used as a delimiter).
% Compilation is handed over to the new file by |\childdocforward|:
%    \begin{macrocode}
\newcommand{\childdocforwardprefix}[3][]
{
  \begingroup
    \def\childdocextract #2##1~~~{\def\childdoctmp{\childdocforward[#1]{#3##1}}}
    \expandafter\childdocextract\childdocname~~~
    \expandafter
  \endgroup
  \childdoctmp
}
%    \end{macrocode}

% \macro{\childdoc}
% The deprecated macro |\childdoc| is a legacy version of |\childdocmain|:
%    \begin{macrocode}
\newcommand{\childdoc}{\childdocmain}
%    \end{macrocode}

% \macro{\childdocredirect}
% The deprecated macro |\childdocredirect| is a legacy version
% of |\childdocforward| and |\childdocforwardprefix|:
%    \begin{macrocode}
\newcommand{\childdocredirect}[2][]
{
  \begingroup
    \if?#1?
      \def\childdoctmp{\childdocforward{#2}}
    \else
      \def\childdoctmp{\childdocforwardprefix{#1}{#2}}
    \fi
    \expandafter
  \endgroup
  \childdoctmp
}
%    \end{macrocode}

%\iffalse
%</package>
%\fi
%
\endinput
|\\
|\childdocmain{|\textit{main}|}|\\
\end{tabular}
\end{center}
%
If |\jobname| does not match the argument \textit{main} of |\childdocmain|,
it is assumed that |\jobname| points to the child file to be compiled.
When using |\childdocmain| with the main file specified as argument,
it suffices to start a child file
with just |\input{|\textit{main}|}|
without loading of the package and using |\childdocof|.
If instead all processing is done
with the appropriate \textsf{childdoc} directives,
the argument of \textit{main} of |\childdocmain| can be empty.

An alternative version of the command line processing described
in \secref{sec:commandline} using the detection mechanism reads:
%
\begin{center}
|... -jobname "|\textit{target}|" "|[\textit{flags}]%
[|\def\jobname{|\textit{dest}|}|]|\input{|\textit{main}|}"|
\end{center}

%%%%%%%%%%%%%%%%%%%%%%%%%%%%%%%%%%%%%%%%%%%%%%%%%%%%%%%%%%%%%%%%%%%%%%%%%%%%%%%%
\subsection{Manual Code}
\label{sec:manual}

In case one cannot be certain whether the definitions file |childdoc.def|
is installed on the target \TeX{} distribution
and one prefers not to ship it,
it is conceivable to paste a few relevant commands into the sources.

To that end, drop all statements |% \iffalse
%
% childdoc.dtx Copyright (C) 2017-2018 Niklas Beisert
%
% This work may be distributed and/or modified under the
% conditions of the LaTeX Project Public License, either version 1.3
% of this license or (at your option) any later version.
% The latest version of this license is in
%   http://www.latex-project.org/lppl.txt
% and version 1.3 or later is part of all distributions of LaTeX
% version 2005/12/01 or later.
%
% This work has the LPPL maintenance status `maintained'.
%
% The Current Maintainer of this work is Niklas Beisert.
%
% This work consists of the files childdoc.dtx and childdoc.ins
% and the derived files childdoc.def and cdocsamp.tex with
% cdocsch1.tex, cdocsch2.tex, cdocsdrf.tex, cdocsfn1.tex, cdocsfn2.tex.
%
%<package>\ifdefined\childdocmain\endinput\fi
%<package>\ProvidesFile{childdoc.def}[2018/12/30 v2.0 child document driver]
%<samplemain>\ProvidesFile{cdocsamp.tex}[2018/12/30 v2.0 sample for childdoc]
%<*driver>
%\ProvidesFile{childdoc.drv}[2018/12/30 v2.0 childdoc reference manual file]
\PassOptionsToClass{10pt,a4paper}{article}
\documentclass{ltxdoc}

\usepackage[margin=35mm]{geometry}
\usepackage{hyperref}
\usepackage{hyperxmp}
\usepackage[usenames]{color}

\hypersetup{colorlinks=true}
\hypersetup{pdfstartview=FitH}
\hypersetup{pdfpagemode=UseNone}
\hypersetup{pdfsource={}}
\hypersetup{pdflang={en-UK}}
\hypersetup{pdfcopyright={Copyright 2017-2018 Niklas Beisert.
  This work may be distributed and/or modified under the
  conditions of the LaTeX Project Public License, either version 1.3
  of this license or (at your option) any later version.}}
\hypersetup{pdflicenseurl={http://www.latex-project.org/lppl.txt}}
\hypersetup{pdfcontactaddress={ETH Zurich, ITP, HIT K,
  Wolfgang-Pauli-Strasse 27}}
\hypersetup{pdfcontactpostcode={8093}}
\hypersetup{pdfcontactcity={Zurich}}
\hypersetup{pdfcontactcountry={Switzerland}}
\hypersetup{pdfcontactemail={nbeisert@itp.phys.ethz.ch}}
\hypersetup{pdfcontacturl={http://people.phys.ethz.ch/\xmptilde nbeisert/}}

\newcommand{\secref}[1]{\hyperref[#1]{section \ref*{#1}}}

\parskip1ex
\parindent0pt
\let\olditemize\itemize
\def\itemize{\olditemize\parskip0pt}

\begin{document}

\title{The \textsf{childdoc} Package}
\hypersetup{pdftitle={The childdoc Package}}
\author{Niklas Beisert\\[2ex]
  Institut f\"ur Theoretische Physik\\
  Eidgen\"ossische Technische Hochschule Z\"urich\\
  Wolfgang-Pauli-Strasse 27, 8093 Z\"urich, Switzerland\\[1ex]
  \href{mailto:nbeisert@itp.phys.ethz.ch}
  {\texttt{nbeisert@itp.phys.ethz.ch}}}
\hypersetup{pdfauthor={Niklas Beisert}}
\hypersetup{pdfsubject={Manual for the LaTeX2e Package childdoc}}
\date{30 December 2018, \textsf{v2.0}}
\maketitle

\begin{abstract}\noindent
\textsf{childdoc} is a \LaTeXe{} package
that enables the direct compilation
of document sections included by |\include|
to individual files.
\end{abstract}

\begingroup
\parskip0ex
\tableofcontents
\endgroup

%%%%%%%%%%%%%%%%%%%%%%%%%%%%%%%%%%%%%%%%%%%%%%%%%%%%%%%%%%%%%%%%%%%%%%%%%%%%%%%%
%%%%%%%%%%%%%%%%%%%%%%%%%%%%%%%%%%%%%%%%%%%%%%%%%%%%%%%%%%%%%%%%%%%%%%%%%%%%%%%%
\section{Introduction}

\LaTeX{} provides a mechanism to structure a large document (such as a book)
into a main file and several child files (containing the chapters)
using the |\include| command.
This mechanism is beneficial for documents
which span hundreds of pages in order to
make the source file(s) more manageable.
Moreover, compilation can be restricted to
selected child files by means of the |\includeonly| command.
The latter feature can be used to reduce the compilation time while editing
(this was significantly more useful in the earlier days of \LaTeX{})
or to generate a smaller document which is easier to navigate.
Another application of |\includeonly| is to generate
documents consisting of selected parts of the complete document.

However, there are a few drawbacks of the plain |\include| mechanism:
\begin{itemize}
\item
The child files cannot be compiled on their own,
they can only be compiled via the main file.
A naive editing environment
(such as a text editor with an option
to have the current file processed by \LaTeX)
may require one to switch to the main file before compiling;
attempting to compile the child file produces errors.
\item
The main file must be modified (each time)
to adjust the |\includeonly| command
to the present needs. This easily leaves the main file in a messy state.
\item
The generated document will always carry the filename
of the main document. This is inconvenient if
several child files are to be compiled and
to be kept for distribution.
\end{itemize}

The present package provides a simple interface
to make child files individually compilable by \LaTeX{}.
Compiling a child file then has the same effect as compiling
the main file with an |\includeonly| command
to select the appropriate child.
Moreover the generated document will carry the name of the child
rather than the main file.
This resolves all three above issues.

This feature is meant to make the editing of books,
thesis documents and lecture notes somewhat more convenient.
However, the package can also be used efficiently for
composing a series of documents (such as exercise sheets)
which are typically distributed individually.
It then assists the author in generating the individual documents
(potentially in different versions)
as well as a document containing the collected series.
Another application is in developing style files
or other kinds of included material
where compilation of the style file could redirect
to a sample or test file.

%%%%%%%%%%%%%%%%%%%%%%%%%%%%%%%%%%%%%%%%%%%%%%%%%%%%%%%%%%%%%%%%%%%%%%%%%%%%%%%%
%%%%%%%%%%%%%%%%%%%%%%%%%%%%%%%%%%%%%%%%%%%%%%%%%%%%%%%%%%%%%%%%%%%%%%%%%%%%%%%%
\section{Usage}

First of all, the package \textsf{childdoc} is \emph{not} a standard
\LaTeXe{} |.sty| style file! Therefore it needs to be invoked in
a non-standard way.

%%%%%%%%%%%%%%%%%%%%%%%%%%%%%%%%%%%%%%%%%%%%%%%%%%%%%%%%%%%%%%%%%%%%%%%%%%%%%%%%
\subsection{Included Files}
\label{sec:include}

%%%%%%%%%%%%%%%%%%%%%%%%%%%%%%%%%%%%%%%%
\DescribeMacro{\childdocmain}
To use the package, add the commands
\begin{center}
\begin{tabular}{l}
|\input{childdoc.def}|\\
|\childdocmain{}|\\
\end{tabular}
\end{center}
at the very top of the main \LaTeX{} file,
in particular \emph{before} the |\documentclass| statement!
The argument of |\childdocmain| should be left empty
(but it must be present).

%%%%%%%%%%%%%%%%%%%%%%%%%%%%%%%%%%%%%%%%
\DescribeMacro{\childdocof}
Furthermore, add the commands
\begin{center}
\begin{tabular}{l}
|\input{childdoc.def}|\\
|\childdocof{|\textit{main}|}|\\
\end{tabular}
\end{center}
at the top of every child file \textit{child}
which is included by |\include{|\textit{child}|}|
from within the main file
(or at least for those files to be compiled individually).
The argument \textit{main} must be the filename of the main file.

There are a couple of
considerations in setting up the main and child documents:

%%%%%%%%%%%%%%%%%%%%%%%%%%%%%%%%%%%%%%%%
\paragraph{Restrictions.}

Please note the following restrictions:
\begin{itemize}
\item
|\childdocmain| must be called with one argument \textit{main}
to ensure compatibility with earlier version of the package.
It must either be empty (|\childdocmain{}|)
or precisely match the filename of the main file in which it is specified.
See \secref{sec:detection} for further information.
\item
The filename \textit{main} must be specified without the |.tex| extension.
\item
The filename \textit{main} is case sensitive
(even in case-insensitive file systems)
due to internal string comparison.
\item
The argument \textit{main} should be fully expanded, it cannot be a macro.
\item
Subdirectories and special characters should be avoided in filenames.
\item
The command |\childdocmain{|\textit{main}|}| must be followed by a whitespace.
It should not be followed immediately by another command
or by a comment mark `|%|'.
This is because the \TeX{} parser reads the token immediately following
the argument of |\childdocmain| and puts it
at the beginning of every child section;
however, a white\-space is ignored.
\end{itemize}

%%%%%%%%%%%%%%%%%%%%%%%%%%%%%%%%%%%%%%%%
\paragraph{Content of Main File.}

It is advisable to place all content in the child files included by |\include|.
Any output contained in the main file will appear in all child documents
unless suppressed manually;
it cannot be suppressed automatically by the |\includeonly| directive
and thus should normally be avoided.
A method to include some content in the main file
by means of conditional processing is described in \secref{sec:conditional}.

%%%%%%%%%%%%%%%%%%%%%%%%%%%%%%%%%%%%%%%%
\paragraph{Page Numbering.}

When only a part of the document is compiled,
the appropriate numbering of pages
(as well as other status parameters)
is determined from the |.aux| files.
The latter contain information from previous passes.
However this information needs to propagate through
all intermediate child documents.
Therefore the page numbering in child documents may well
be inconsistent until the complete document is compiled at least once.

A useful (if unconventional) way to always ensure a consistent
page numbering is to restart the numbering in each child document
and denote the pages by `\textit{child}|.|\textit{page}'
where \textit{child} represents the chapter/section number of the child file.
This can be achieved by the command
|\numberwithin{page}{|\textit{child}|}|
of the \textsf{amsmath} package
where \textit{child} can be |chapter| or |section|
depending on the chosen structuring.
Alternatively, one can modify the macro |\thepage| appropriately
and reset the counter |page| at the start of each child file.

%%%%%%%%%%%%%%%%%%%%%%%%%%%%%%%%%%%%%%%%%%%%%%%%%%%%%%%%%%%%%%%%%%%%%%%%%%%%%%%%
\subsection{Conditional Processing}
\label{sec:conditional}

The package provides a mechanism to compile different versions
of a document. To customise the versions further some conditional processing
can come in handy to distinguish which version is being compiled.
The package provides two macros to describe the compilation context:

%%%%%%%%%%%%%%%%%%%%%%%%%%%%%%%%%%%%%%%%
\DescribeMacro{\ifchilddoc}
The conditional |\ifchilddoc| distinguishes between the compilation of
child documents and the main document:
%
\begin{center}
|\ifchilddoc |\textit{child-code}| |[|\||else |\textit{main-code}]| \||fi|
\end{center}

%%%%%%%%%%%%%%%%%%%%%%%%%%%%%%%%%%%%%%%%
\DescribeMacro{\childdocname}
\DescribeMacro{\childdocjob}
The macro |\childdocname| contains the filename (without extension)
of the main or child file being processed.
Note that |\childdocjob| will always contain the name of the main file.

%%%%%%%%%%%%%%%%%%%%%%%%%%%%%%%%%%%%%%%%
\paragraph{Title Page.}

Conditional processing can be used to include a title or banner page
in the main document when proper precautions are taken.
Importantly, the code in the main file should ensure that the page counter
(as well as other status parameters which are stored in the |.aux| files)
takes the same value after the conditional processing.
Otherwise the page numbers may take divergent values
depending on which part is compiled.

For example, a title page could be declared by:
%
\begin{center}
\begin{tabular}{l}
|\ifchilddoc\||else|\\
|\addtocounter{page}{-1}|\\
\textit{code for title page}\\
|\newpage|\\
|\||fi|
\end{tabular}
\end{center}
%
A banner page for the child documents can be generated by:
%
\begin{center}
\begin{tabular}{l}
|\ifchilddoc|\\
|\addtocounter{page}{-1}|\\
\textit{code for banner page}\\
|\newpage|\\
|\||fi|
\end{tabular}
\end{center}
%
Here one could write a message such as:
\begin{center}
|This is the part \childdocname{} of \childdocjob{}.|
\end{center}

%%%%%%%%%%%%%%%%%%%%%%%%%%%%%%%%%%%%%%%%%%%%%%%%%%%%%%%%%%%%%%%%%%%%%%%%%%%%%%%%
\subsection{Flags}
\label{sec:flags}

The package makes it easy to generate different versions
of the main or child documents.
To this end compilation flags can be defined
and assigned different default values.
They will be particularly useful in conjunction
with the forwarding mechanism described in \secref{sec:forward}.

For example, it may be useful to have a flag |\version|
which can be set to |draft| or |final|.
The document source will contain some conditional code
depending on the value of |\version|.
Suppose further, the flag should default to |final| for the main file
and to |draft| for child files
which is a natural assignment for editing the document.
This is achieved by placing the following code
in the preamble of the main document
(below the |\childdocmain| directive):
%
\begin{center}
\begin{tabular}{l}
|\ifchilddoc|\\
|\providecommand{\version}{draft}|\\
|\||else|\\
|\providecommand{\version}{final}|\\
|\||fi|
\end{tabular}
\end{center}
%
The definition by |\providecommand| makes sure
that previous definitions are not overwritten.
Further statements |\providecommand{\version}{...}|
can thus be added before the above code to override it.

For the main file, one might add a line
(between |\childdocmain| and the above block)
%
\begin{center}
|%\ifchilddoc\||else\providecommand{\version}{draft}\||fi|
\end{center}
%
which can be uncommented to produce a draft version.
Likewise one can add a line to the very top of a child file
(above the |\childdocof{|\textit{main}|}| directive)
%
\begin{center}
|%\providecommand{\version}{final}|
\end{center}
%
which can be uncommented to produce the final version of this child document.

%%%%%%%%%%%%%%%%%%%%%%%%%%%%%%%%%%%%%%%%%%%%%%%%%%%%%%%%%%%%%%%%%%%%%%%%%%%%%%%%
\subsection{Forwarding}
\label{sec:forward}

Different versions of the main or child documents
using compilation flags as described in \secref{sec:flags}
can be (permanently) stored in different files
for convenient compilation, viewing and distribution.
To this end, the package defines a command
to pass on compilation to a different file:

%%%%%%%%%%%%%%%%%%%%%%%%%%%%%%%%%%%%%%%%
\DescribeMacro{\childdocforward}
The command |\childdocforward| redirects processing to
another source file:
%
\begin{center}
\begin{tabular}{l}
|\input{childdoc.def}|\\
|\childdocforward[|\textit{main}|]{|\textit{dest}|}|\\
\end{tabular}
\end{center}
%
The argument \textit{dest} is the destination file
(without extension).
It should be the main file or one of the child files.
Note that further \textsf{childdoc} directives
such as |\childdocof| and |\childdocforward|
in the indicated file will be processed in this form.
The optional argument \textit{main}
passes on directly to the main file \textit{main}
while pretending to compile the child \textit{dest}.
This form behaves as if \textit{dest}
issues |\childdocof{|\textit{main}|}| right away,
and no further \textsf{childdoc} directives will be processed.

%%%%%%%%%%%%%%%%%%%%%%%%%%%%%%%%%%%%%%%%
\DescribeMacro{\...prefix}
In the alternative form |\childdocforwardprefix|,
%
\begin{center}
\begin{tabular}{l}
|\input{childdoc.def}|\\
|\childdocforwardprefix[|\textit{main}|]{|\textit{prefix}|}{|\textit{dest}|}|
\end{tabular}
\end{center}
%
the destination file is determined by a pattern
depending on the current file:
To make this work, the current file must be called
`{\textit{prefix}\hspace{0.2em}\textit{suffix}}'
with \textit{prefix} matching precisely the argument.
Processing is then passed on to the file
`{\textit{dest}\hspace{0.2em}\textit{suffix}}'.
Surely, the same effect is achieved by
directly specifying the
argument `{\textit{dest}\hspace{0.2em}\textit{suffix}}'
in the first form.
However, that requires to set up a different file
for each child. With the alternative form of the command
all these files can have exactly the same content
which simplifies setting them up and maintaining them.

For example, the following file |draft.tex|
with a compilation flag |\version| as described in \secref{sec:flags}
compiles the main document as a draft:
%
\begin{center}
\begin{tabular}{l}
|\def\version{draft}|\\
|\input{childdoc.def}|\\
|\childdocforward{|\textit{main}|}|
\end{tabular}
\end{center}
%
Likewise, the following files |final|\textit{nn}|.tex|
compile the final version of the child document
|child|\textit{nn}|.tex|:
%
\begin{center}
\begin{tabular}{l}
|\def\version{final}|\\
|\input{childdoc.def}|\\
|\childdocforwardprefix{final}{child}|
\end{tabular}
\end{center}
%

Note that when several versions of a main file and/or of each child file
are to be generated, it may be convenient to set up a |Makefile| or
shell script to automatise the process.

%%%%%%%%%%%%%%%%%%%%%%%%%%%%%%%%%%%%%%%%%%%%%%%%%%%%%%%%%%%%%%%%%%%%%%%%%%%%%%%%
\subsection{Command Line Processing}
\label{sec:commandline}

The effect of redirection files can also be achieved by invoking
the \LaTeX{} compiler with a more elaborate command line.
Most conveniently this should be done as part
of a shell script or a |Makefile|.

When using \textsf{childdoc} in the main file, the following
command lines effectively perform a redirection
(note that depending on the shell being used,
backslashes may have to be doubled: `|\|' $\to$ `|\\|'):
%
\begin{center}
|... -jobname "|\textit{target}|" |\\|"|[\textit{flags}]%
|\input{childdoc.def}\childdocforward[|\textit{main}|]{|\textit{dest}|}"|
\end{center}
%
Here \textit{target} is the name of the output file,
\textit{main} is the name of the main file
and \textit{dest} is the name of the main or child file to be processed
(all filenames without extensions).
The optional argument \textit{main} can be omitted
if \textit{main} matches \textit{dest}.
Optionally, compilation \textit{flags} can be defined via |\def| commands.
This command line makes the \TeX{} engine believe
it is compiling the file \textit{target}
whose content is specified as the latter parameter.
The provided code then forwards the processing to
\textit{main} or \textit{dest} as described in \secref{sec:forward}.

%%%%%%%%%%%%%%%%%%%%%%%%%%%%%%%%%%%%%%%%%%%%%%%%%%%%%%%%%%%%%%%%%%%%%%%%%%%%%%%%
\subsection{Include by Input}
\label{sec:input}

Including child documents by |\include| has some restrictions by design.
Most notably, the content of a child document always occupies
its own set of pages; pages cannot be shared between child documents.
Usually, this behaviour makes perfect sense
because each child document contain an essential part of the document.
However, in some situations it may be desirable to compose
a document from a collection of parts
without having mandatory page breaks between then.
For this case, the package
provides a mechanism to include parts
by |\input| which can also be processed individually.
However, by construction this mechanism
requires manual handling of the content to be output.

%%%%%%%%%%%%%%%%%%%%%%%%%%%%%%%%%%%%%%%%
\DescribeMacro{\ifchilddocmanual}
The main file should be prepared as usual, see \secref{sec:include}.
However, the document body must make a distinction
between processing of an individual part and of the main document, e.g.:
%
\begin{center}
\begin{tabular}{l}
|\ifchilddocmanual|\\
|\input{\childdocname}|\\
|\||else|\\
\textit{document body with }|\input{|\textit{part}|}|\\
|\||fi|
\end{tabular}
\end{center}
%
The conditional |\ifchilddocmanual| is true whenever
a part to be included by |\input| is being compiled,
and the name of the part is stored in |\childdocname|.

%%%%%%%%%%%%%%%%%%%%%%%%%%%%%%%%%%%%%%%%
\DescribeMacro{\childdocby}
Each part to be included by |\input| should start with:
%
\begin{center}
\begin{tabular}{l}
|\input{childdoc.def}|\\
|\childdocby{|\textit{main}|}|\\
\end{tabular}
\end{center}
%
The directive |\childdocby| is similar to |\childdocof|
described in \secref{sec:include},
but the subsequent selection of content must be done manually.
To that end, both |\ifchilddoc| and |\ifchilddocmanual|
will be true upon processing of a part,
and the name of the part is stored in |\childdocname|.
Note that |\jobname| will be set to the filename of the current part
so that each part receives an individual |.aux| file
that does not interfere with the |.aux| file(s) of the main document.
This behaviour can be altered by the alternative form
|\childdocby[*]{|\textit{main}|}| (with a non-empty optional argument)
which uses the |.aux| file of the main document
by setting |\jobname| to \textit{main}.

%%%%%%%%%%%%%%%%%%%%%%%%%%%%%%%%%%%%%%%%%%%%%%%%%%%%%%%%%%%%%%%%%%%%%%%%%%%%%%%%
\subsection{Driver Development}
\label{sec:driver}

The \textsf{childdoc} mechanism can also be use for the development
of definition files such as \LaTeX{} styles or classes.
This case differs from the above setup with multiple parts
included by |\include| in that no |\includeonly| should be invoked.
This can be achieved by starting the include file
(before |\ProvidesPackage|) with:
%
\begin{center}
\begin{tabular}{l}
|\input{childdoc.def}|\\
|\childdocforward{|\textit{main}|}|\\
\end{tabular}
\end{center}
%
or alternatively with:
%
\begin{center}
\begin{tabular}{l}
|\input{childdoc.def}|\\
|\childdocby{|\textit{main}|}|\\
\end{tabular}
\end{center}
%
Both forms have slightly different effects as described above.
The main file is prepared as usual, see \secref{sec:include}.

%%%%%%%%%%%%%%%%%%%%%%%%%%%%%%%%%%%%%%%%%%%%%%%%%%%%%%%%%%%%%%%%%%%%%%%%%%%%%%%%
\subsection{Legacy Detection}
\label{sec:detection}

The directive |\childdocmain| in the main file can detect
whether the complete document or merely a child is to be compiled
even without using the directive |\childdocof|.
This method is deprecated because it is less robust
and there is no compelling reason to use it;
it is merely provided for backward compatibility
and it may be removed in future versions.

If the detection mechanism is to be used,
it is mandatory to correctly specify
the filename of the main file as the argument of |\childdocmain|:
%
\begin{center}
\begin{tabular}{l}
|\input{childdoc.def}|\\
|\childdocmain{|\textit{main}|}|\\
\end{tabular}
\end{center}
%
If |\jobname| does not match the argument \textit{main} of |\childdocmain|,
it is assumed that |\jobname| points to the child file to be compiled.
When using |\childdocmain| with the main file specified as argument,
it suffices to start a child file
with just |\input{|\textit{main}|}|
without loading of the package and using |\childdocof|.
If instead all processing is done
with the appropriate \textsf{childdoc} directives,
the argument of \textit{main} of |\childdocmain| can be empty.

An alternative version of the command line processing described
in \secref{sec:commandline} using the detection mechanism reads:
%
\begin{center}
|... -jobname "|\textit{target}|" "|[\textit{flags}]%
[|\def\jobname{|\textit{dest}|}|]|\input{|\textit{main}|}"|
\end{center}

%%%%%%%%%%%%%%%%%%%%%%%%%%%%%%%%%%%%%%%%%%%%%%%%%%%%%%%%%%%%%%%%%%%%%%%%%%%%%%%%
\subsection{Manual Code}
\label{sec:manual}

In case one cannot be certain whether the definitions file |childdoc.def|
is installed on the target \TeX{} distribution
and one prefers not to ship it,
it is conceivable to paste a few relevant commands into the sources.

To that end, drop all statements |\input{childdoc.def}|
and perform the replacements as outlined below.
Instead of |\childdocmain{|\textit{main}|}| add the following code
to the top of the main file:
%
\begin{center}
\begin{tabular}{l}
|\||ifdefined\childdocname\endinput\||fi\newif\ifchilddoc|\\
|\edef\childdocname{\scantokens\expandafter{\jobname\noexpand}}|\\
|\def\childdocmain{|\textit{main}|}\||ifx\childdocmain\childdocname\||else|\\
|\childdoctrue\includeonly{\childdocname}\let\jobname\childdocmain\||fi|\\
\end{tabular}
\end{center}
%
Instead of |\childdocof{|\textit{main}|}| just include the main file
at the top of each child file:
%
\begin{center}
|\input{|\textit{main}|}|
\end{center}
%
A simple redirection |\childdocforward{|\textit{dest}|}| is achieved by:
%
\begin{center}
|\def\jobname{|\textit{dest}|}\input{\jobname}|
\end{center}
%
The redirection with prefix
|\childdocforwardprefix[|\textit{prefix}|]{|\textit{dest}|}|
is accomplished by:
%
\begin{center}
\begin{tabular}{l}
|{\edef\jobname{\scantokens\expandafter{\jobname\noexpand}}|\\
|\def\redirectjob |\textit{prefix}|#1~~~{\gdef\jobname{|\textit{dest}|#1}}|\\
|\expandafter\redirectjob\jobname~~~}\input{\jobname}|
\end{tabular}
\end{center}

In an alternative approach,
child documents can be compiled by a specific command line
without additional code or specific definitions:
%
\begin{center}
|... -jobname "|\textit{target}|" "|[\textit{flags}]%
|\includeonly{|\textit{dest}|}\input{|\textit{main}|}"|
\end{center}
%

%%%%%%%%%%%%%%%%%%%%%%%%%%%%%%%%%%%%%%%%%%%%%%%%%%%%%%%%%%%%%%%%%%%%%%%%%%%%%%%%
%%%%%%%%%%%%%%%%%%%%%%%%%%%%%%%%%%%%%%%%%%%%%%%%%%%%%%%%%%%%%%%%%%%%%%%%%%%%%%%%
\section{Information}

%%%%%%%%%%%%%%%%%%%%%%%%%%%%%%%%%%%%%%%%%%%%%%%%%%%%%%%%%%%%%%%%%%%%%%%%%%%%%%%%
\subsection{Copyright}

Copyright \copyright{} 2017--2018 Niklas Beisert

This work may be distributed and/or modified under the
conditions of the \LaTeX{} Project Public License, either version 1.3
of this license or (at your option) any later version.
The latest version of this license is in
  \url{http://www.latex-project.org/lppl.txt}
and version 1.3 or later is part of all distributions of \LaTeX{}
version 2005/12/01 or later.

This work has the LPPL maintenance status `maintained'.

The Current Maintainer of this work is Niklas Beisert.

This work consists of the files |README.txt|, |childdoc.ins| and |childdoc.dtx|
as well as the derived files |childdoc.def|, |cdocsamp.tex|
with |cdocsch1.tex|, |cdocsch2.tex|, |cdocspt3.tex|, |cdocspt4.tex|,
|cdocsdrf.tex|, |cdocsfn1.tex|, |cdocsfn2.tex|
as well as |childdoc.pdf|.

%%%%%%%%%%%%%%%%%%%%%%%%%%%%%%%%%%%%%%%%%%%%%%%%%%%%%%%%%%%%%%%%%%%%%%%%%%%%%%%%
\subsection{Files and Installation}

The package consists of the files:
%
\begin{center}
\begin{tabular}{ll}
    |README.txt|   & readme file \\
    |childdoc.ins| & installation file \\
    |childdoc.dtx| & source file \\
    |childdoc.def| & definition file \\
    |cdocsamp.tex| & sample main file \\
    |cdocsch1.tex| & sample include file \\
    |cdocsch2.tex| & sample include file \\
    |cdocspt3.tex| & sample part file \\
    |cdocspt4.tex| & sample part file \\
    |cdocsdrf.tex| & sample redirection file \\
    |cdocsfn1.tex| & sample redirection file \\
    |cdocsfn2.tex| & sample redirection file \\
    |childdoc.pdf| & manual
\end{tabular}
\end{center}
%
The distribution consists of the files
|README.txt|, |childdoc.ins| and |childdoc.dtx|.
%
\begin{itemize}
\item
Run (pdf)\LaTeX{} on |childdoc.dtx|
to compile the manual |childdoc.pdf| (this file).
\item
Run \LaTeX{} on |childdoc.ins| to create the definitions file |childdoc.def|
and the sample |cdocsamp.tex| with include files
|cdocsch1.tex|, |cdocsch2.tex|, |cdocspt3.tex|, |cdocspt4.tex|,
|cdocsdrf.tex|, |cdocsfn1.tex|, |cdocsfn2.tex|.
Then copy the file |childdoc.def| to an appropriate directory of your \LaTeX{}
distribution, e.g.\ \textit{texmf-root}|/tex/latex/childdoc|.
\end{itemize}

%%%%%%%%%%%%%%%%%%%%%%%%%%%%%%%%%%%%%%%%%%%%%%%%%%%%%%%%%%%%%%%%%%%%%%%%%%%%%%%%
\subsection{Related CTAN Packages}

There are several other packages which offer a similar functionality:
%
\begin{itemize}
\item
The packages
\href{http://ctan.org/pkg/docmute}{\textsf{docmute}},
\href{http://ctan.org/pkg/includex}{\textsf{includex}} and
\href{http://ctan.org/pkg/standalone}{\textsf{standalone}}
provide commands to include only the document body of
a child file thus allowing both files to be compiled individually.
\item
The packages \href{http://ctan.org/pkg/subdocs}{\textsf{subdocs}}
and \href{http://ctan.org/pkg/subfiles}{\textsf{subfiles}}
provide structures in which the main and child documents can be
encapsulated and allowing them to be compiled individually.
The inclusion mechanism is different from the conventional |\include|.
\item
The package \href{http://ctan.org/pkg/combine}{\textsf{combine}}
is an elaborate solution to combine several documents into one.
\end{itemize}
%
See also the CTAN topic \href{http://ctan.org/topic/subdocs}{\textsf{subdocs}}
for further related packages.
The present package differs from the above solutions in that
a document structure constructed with the conventional |\include| mechanism
just needs two extra commands at the top of every file
such that all constituent files can be compiled individually.

%%%%%%%%%%%%%%%%%%%%%%%%%%%%%%%%%%%%%%%%%%%%%%%%%%%%%%%%%%%%%%%%%%%%%%%%%%%%%%%%
%\subsection{Feature Suggestions}
%
%The following is a list of features which may be useful for future
%versions of this package:
%%
%\begin{itemize}
%\item
%\ldots
%\end{itemize}

%%%%%%%%%%%%%%%%%%%%%%%%%%%%%%%%%%%%%%%%%%%%%%%%%%%%%%%%%%%%%%%%%%%%%%%%%%%%%%%%
\subsection{Revision History}

%%%%%%%%%%%%%%%%%%%%%%%%%%%%%%%%%%%%%%%%
\paragraph{v2.0:} 2018/12/30

\begin{itemize}
\item
immediate forward processing
\item
added |\childdocby| mechanism
\item
manual restructured
\end{itemize}

%%%%%%%%%%%%%%%%%%%%%%%%%%%%%%%%%%%%%%%%
\paragraph{v1.6:} 2018/01/17

\begin{itemize}
\item
application for development of include files
\item
corrections to manual
\end{itemize}

%%%%%%%%%%%%%%%%%%%%%%%%%%%%%%%%%%%%%%%%
\paragraph{v1.5:} 2017/05/21

\begin{itemize}
\item
more complete structuring introduced
\item
|\childdocof| introduced
\item
|\childdoc| renamed to |\childdocmain|
\item
|\childredirect| renamed to |\childdocforward| and |\childdocforwardprefix|
and functionality expanded
\end{itemize}

%%%%%%%%%%%%%%%%%%%%%%%%%%%%%%%%%%%%%%%%
\paragraph{v1.0:} 2017/04/27

\begin{itemize}
\item
manual and install package
\item
first version published on CTAN
\end{itemize}

%%%%%%%%%%%%%%%%%%%%%%%%%%%%%%%%%%%%%%%%
\paragraph{v0.6:} 2017/04/26

\begin{itemize}
\item
redirection mechanism added
\end{itemize}

%%%%%%%%%%%%%%%%%%%%%%%%%%%%%%%%%%%%%%%%
\paragraph{v0.5:} 2017/04/26

\begin{itemize}
\item
functionality in definition file
\end{itemize}


%%%%%%%%%%%%%%%%%%%%%%%%%%%%%%%%%%%%%%%%%%%%%%%%%%%%%%%%%%%%%%%%%%%%%%%%%%%%%%%%
%%%%%%%%%%%%%%%%%%%%%%%%%%%%%%%%%%%%%%%%%%%%%%%%%%%%%%%%%%%%%%%%%%%%%%%%%%%%%%%%
%%%%%%%%%%%%%%%%%%%%%%%%%%%%%%%%%%%%%%%%%%%%%%%%%%%%%%%%%%%%%%%%%%%%%%%%%%%%%%%%
\appendix

\settowidth\MacroIndent{\rmfamily\scriptsize 000\ }

 \DocInput{childdoc.dtx}

\end{document}
%</driver>
% \fi
%
% %%%%%%%%%%%%%%%%%%%%%%%%%%%%%%%%%%%%%%%%%%%%%%%%%%%%%%%%%%%%%%%%%%%%%%%%%%%%%%
% %%%%%%%%%%%%%%%%%%%%%%%%%%%%%%%%%%%%%%%%%%%%%%%%%%%%%%%%%%%%%%%%%%%%%%%%%%%%%%
% \section{Sample}
%\iffalse
%<*samplemain>
%\fi
%
% The following presents a sample document
% with two chapters, two parts, a title page,
% a compile flag as well as three forwarding files to set the flag.
% It consists of eight |.tex| files:
% \begin{center}
% \begin{tabular}{ll}
% |cdocsamp.tex|&main file\\
% |cdocsch1.tex|&include file for chapter 1\\
% |cdocsch2.tex|&include file for chapter 2\\
% |cdocspt3.tex|&include file for part 3\\
% |cdocspt4.tex|&include file for part 4\\
% |cdocsdrf.tex|&forwarding file for main file in draft mode\\
% |cdocsfi1.tex|&forwarding file for final version of chapter 1\\
% |cdocsfi2.tex|&forwarding file for final version of chapter 2\\
% \end{tabular}
% \end{center}
% Each of the eight files can be compiled directly by the \LaTeX{} compiler.
%
% %%%%%%%%%%%%%%%%%%%%%%%%%%%%%%%%%%%%%%
% \paragraph{Main File.}
%
% The main file is called |cdocsamp.tex|.
%
% Load the \textsf{childdoc} definitions and
% declare the filename for the main document:
%    \begin{macrocode}
\input{childdoc.def}
\childdocmain{}
%    \end{macrocode}

% Optional override for |\version| flag:
%    \begin{macrocode}
%%\ifchilddoc\else\providecommand{\version}{draft}\fi
%    \end{macrocode}

% Define the default values for the |\version| flag
% (|final| for the main file and |draft| for childs):
%    \begin{macrocode}
\ifchilddoc
\providecommand{\version}{draft}
\else
\providecommand{\version}{final}
\fi
%    \end{macrocode}

% Load the standard document class:
%    \begin{macrocode}
\documentclass[12pt]{article}
%    \end{macrocode}

% Start the document body:
%    \begin{macrocode}
\begin{document}
%    \end{macrocode}

% Declare a title page.
% Print title, part of document being processed and version flag:
%    \begin{macrocode}
\addtocounter{page}{-1}
\begin{center}
{\LARGE\bfseries{}childdoc example\par}
\vspace{1cm}
\ifchilddoc
\ifchilddocmanual part\else chapter\fi:
`\childdocname' of `\childdocjob'\par
\else
main document: `\childdocjob'\par
\fi
version: \version\par
\end{center}
\newpage
%    \end{macrocode}

% Manually include selected file,
% otherwise process as usual:
%    \begin{macrocode}
\ifchilddocmanual
\section*{part `\childdocname'}
\input{\childdocname}
\else
%    \end{macrocode}

% Include the two chapters:
%    \begin{macrocode}
\include{cdocsch1}
\include{cdocsch2}
%    \end{macrocode}

% Include the two parts unless only chapters should be displayed:
%    \begin{macrocode}
\ifchilddoc\else
\section{part three}
\input{cdocspt3}
\section{part four}
\input{cdocspt4}
\fi
%    \end{macrocode}

% Process as usual until here:
%    \begin{macrocode}
\fi
%    \end{macrocode}

% End of document body:
%    \begin{macrocode}
\end{document}
%    \end{macrocode}
%\iffalse
%</samplemain>
%\fi
%
% %%%%%%%%%%%%%%%%%%%%%%%%%%%%%%%%%%%%%%
% \paragraph{Chapter Include Files.}
%
% The include files are called |cdocsch1.tex| and |cdocsch2.tex|.
%
%\iffalse
%<*samplechap1|samplechap2>
%\fi

% Optional override for |\version| flag:
%    \begin{macrocode}
%%\providecommand{\version}{final}
%    \end{macrocode}

% Include the main document:
%    \begin{macrocode}
\input{childdoc.def}
\childdocof{cdocsamp}
%    \end{macrocode}

%\iffalse
%</samplechap1|samplechap2>
%\fi
%
%\iffalse
%<*samplechap1>
%\fi
% Some text for chapter 1:
%    \begin{macrocode}
\section{one}
some text in chapter one
%    \end{macrocode}

%\iffalse
%</samplechap1>
%\fi
% Some text for chapter 2:
%\iffalse
%<*samplechap2>
%\fi
%    \begin{macrocode}
\section{two}
more text in chapter two
%    \end{macrocode}

%\iffalse
%</samplechap2>
%\fi
%
% %%%%%%%%%%%%%%%%%%%%%%%%%%%%%%%%%%%%%%
% \paragraph{Part Include Files.}
%
% The include files are called |cdocspt3.tex| and |cdocspt4.tex|.
%
%\iffalse
%<*samplepart3|samplepart4>
%\fi

% Optional override for |\version| flag:
%    \begin{macrocode}
%%\providecommand{\version}{final}
%    \end{macrocode}

% Include the main document:
%    \begin{macrocode}
\input{childdoc.def}
\childdocby{cdocsamp}
%    \end{macrocode}

%\iffalse
%</samplepart3|samplepart4>
%\fi
%
%\iffalse
%<*samplepart3>
%\fi
% Some text for part 3:
%    \begin{macrocode}
some text in part three
%    \end{macrocode}

%\iffalse
%</samplepart3>
%\fi
% Some text for part 4:
%\iffalse
%<*samplepart4>
%\fi
%    \begin{macrocode}
more text in part four
%    \end{macrocode}

%\iffalse
%</samplepart4>
%\fi
%
% %%%%%%%%%%%%%%%%%%%%%%%%%%%%%%%%%%%%%%
% \paragraph{Forwarding for a Complete Draft.}
%
% The following forwarding file |cdocsdrf.tex|
% compiles the main document in draft mode:
%\iffalse
%<*sampledraft>
%\fi
%    \begin{macrocode}
\def\version{draft}
\input{childdoc.def}
\childdocforward{cdocsamp}
%    \end{macrocode}

%\iffalse
%</sampledraft>
%\fi
%
% %%%%%%%%%%%%%%%%%%%%%%%%%%%%%%%%%%%%%%
% \paragraph{Forwarding for Final Version of the Chapters.}
%
% The following forwarding files |cdocsfn1.tex| and |cdocsfn2.tex|
% (with identical content)
% compile the final versions of the child documents
% |cdocsch1.tex| and |cdocsch2.tex|, respectively:
%\iffalse
%<*samplefinal>
%\fi
%    \begin{macrocode}
\def\version{final}
\input{childdoc.def}
\childdocforwardprefix[cdocsamp]{cdocsfn}{cdocsch}
%    \end{macrocode}

%\iffalse
%</samplefinal>
%\fi
%
% %%%%%%%%%%%%%%%%%%%%%%%%%%%%%%%%%%%%%%
% \paragraph{Command Line Processing.}
%
% The following three command lines generate the output files
% |cdocscld|, |cdocscl1| and |cdocscl2|
% which should be identical to
% |cdocsdrf|, |cdocsch1| and |cdocsfn2|, respectively:
% \begin{center}
% \begin{tabular}{l}
% |latex -jobname cdocscld \|\\
% |  "\def\version{draft}\input{childdoc.def}\childdocforward{cdocsamp}"|\\
% |latex -jobname cdocscl1 \|\\
% |  "\input{childdoc.def}\childdocforward[cdocsamp]{cdocsch1}"|\\
% |latex -jobname cdocscl2 \|\\
% |  "\def\version{final}\input{childdoc.def}\childdocforward{cdocsch2}"|
% \end{tabular}
% \end{center}
% Note that the trailing backslash on each first line
% merely continues the input to the second line
% (for convenient cut ant paste).
% Furthermore, the command |latex| can be replaced by any
% of its alternative versions such as |pdflatex|.
%
% %%%%%%%%%%%%%%%%%%%%%%%%%%%%%%%%%%%%%%%%%%%%%%%%%%%%%%%%%%%%%%%%%%%%%%%%%%%%%%
% %%%%%%%%%%%%%%%%%%%%%%%%%%%%%%%%%%%%%%%%%%%%%%%%%%%%%%%%%%%%%%%%%%%%%%%%%%%%%%
% \section{Implementation}
%\iffalse
%<*package>
%\fi
%
% This section describes the definitions file |childdoc.def|.

% The definitions cannot be loaded using |\usepackage| or |\RequirePackage|
% which has a mechanism to prevent loading a style file more than once.
% When loading the definitions by means of |\input|
% multiple instances have to be prevented manually:
%\iffalse
%This code needs to be before the `\ProvidesFile' directive
%which is defined at the beginning of this file.
%Therefore it is also placed there and commented out here.
%</package>
%<*discard>
%\fi
%    \begin{macrocode}
\ifdefined\childdocmain\endinput\fi
%    \end{macrocode}
%\iffalse
%</discard>
%<*package>
%\fi
%
% \macro{\ifchilddoc}
% \macro{\ifchilddocmanual}
% The conditional |\ifchilddoc| tells whether a
% child (true) or main (false) document is being compiled.
% The conditional |\ifchilddocmanual| tells whether
% the |\includeonly| mechanism is used (false) or
% the selection of child files must be performed manually (true).
% The definitions initialise to false:
%    \begin{macrocode}
\newif\ifchilddoc
\newif\ifchilddocmanual
%    \end{macrocode}

% \macro{\childdocname}
% \macro{\childdocjob}
% The macro |\childdocname| stores the name of the main document
% to be compiled. The macro |\childdocjob| stores the name of
% the document on which the \LaTeX{} compiler was originally invoked.
% The content of |\jobname| cannot be compared
% to filenames specified in the source due to different catcodes.
% The following code rescans |\jobname|, stores the result
% in |\childdocname| and saves a copy in |\childdocjob|:
%    \begin{macrocode}
\edef\childdocname{\scantokens\expandafter{\jobname\noexpand}}
\let\childdocjob\childdocname
%    \end{macrocode}

% \macro{\childdocdisable}
% The macro |\childdocdisable| prevents the main file
% from being processed more than once.
% At this stage, the main document command |\childdocmain|
% is assumed to be called once again where it should do nothing.
% Any subsequent call to it should prevent
% a secondary processing of the main document
% It overwrites the forwarding commands
% |\childdocof| and |\childdocforward|
% with empty macros to prevent further inclusions of the main document:
%    \begin{macrocode}
\newcommand{\childdocdisable}
{
  \renewcommand{\childdocmain}[1]{\renewcommand{\childdocmain}[1]{\endinput}}
  \renewcommand{\childdocof}[1]{}
  \renewcommand{\childdocby}[2][]{}
  \renewcommand{\childdocforward}[2][]{}
  \renewcommand{\childdocdisable}{}
}
%    \end{macrocode}

% \macro{\childdocmain}
% The macro |\childdocmain| is to be called at the top of the main file
% with nothing or the main filename (without extension) as argument.
% First, it breaks loops.
% If the argument is not empty and does not match |\childdocname|
% (which is set by the first inclusion of |childdoc.def|),
% |\ifchilddoc| is set to true, |\includeonly| is applied to the child file
% and |\jobname| is set to the main file
% (for proper handling of |.aux| files):
%    \begin{macrocode}
\newcommand{\childdocmain}[1]
{
  \childdocdisable\childdocmain{}
  \if?#1?\else
    \begingroup
      \def\childdoctmp{#1}
      \ifx\childdoctmp\childdocname
        \def\childdoctmp{}
      \else
        \def\childdoctmp
        {
          \childdoctrue
          \includeonly{\childdocname}
          \def\childdocjob{#1}
          \def\jobname{#1}
        }
      \fi
      \expandafter
    \endgroup
    \childdoctmp
  \fi
}
%    \end{macrocode}

% \macro{\childdocof}
% The command |\childdocof| redirects
% compilation to the main file |#1|.
%    \begin{macrocode}
\newcommand{\childdocof}[1]
{
  \childdocdisable
  \childdoctrue
  \includeonly{\childdocname}
  \def\jobname{#1}
  \def\childdocjob{#1}
  \input{#1}
}
%    \end{macrocode}

% \macro{\childdocby}
% The command |\childdocby| ....
%    \begin{macrocode}
\newcommand{\childdocby}[2][]
{
  \childdocdisable
  \childdoctrue
  \childdocmanualtrue
  \if?#1?\else
    \def\jobname{#2}
  \fi
  \def\childdocjob{#2}
  \input{#2}
  \endinput
}
%    \end{macrocode}

% \macro{\childdocforward}
% The command |\childdocforward| redirects
% compilation to the main file or
% (if the optional argument is given) a child file.
% Parameters are set as if the main file
% or a child file starting with |\childdocof| was compiled.
% Then compilation is handed over to the main file:
%    \begin{macrocode}
\newcommand{\childdocforward}[2][]
{
  \begingroup
    \if?#1?
      \def\childdoctmp
      {
        \def\childdocname{#2}
        \def\childdocjob{#2}
        \def\jobname{#2}
        \input{#2}
        \endinput
      }
    \else
      \def\childdoctmp
      {
        \childdocdisable
        \def\childdocname{#2}
        \childdoctrue
        \includeonly{#2}
        \def\childdocjob{#1}
        \def\jobname{#1}
        \input{#1}
        \endinput
      }
    \fi
    \expandafter
  \endgroup
  \childdoctmp
}
%    \end{macrocode}

% \macro{\childdocforwardprefix}
% The command |\childdocforwardprefix| redirects
% compilation to the main or a child file by means of a pattern.
% The prefix |#1| in the current filename is replaced by |#2|
% and the suffix of the current filename is kept
% (it is assumed that the filename does not contain the substring `|~~~|'
% which is used as a delimiter).
% Compilation is handed over to the new file by |\childdocforward|:
%    \begin{macrocode}
\newcommand{\childdocforwardprefix}[3][]
{
  \begingroup
    \def\childdocextract #2##1~~~{\def\childdoctmp{\childdocforward[#1]{#3##1}}}
    \expandafter\childdocextract\childdocname~~~
    \expandafter
  \endgroup
  \childdoctmp
}
%    \end{macrocode}

% \macro{\childdoc}
% The deprecated macro |\childdoc| is a legacy version of |\childdocmain|:
%    \begin{macrocode}
\newcommand{\childdoc}{\childdocmain}
%    \end{macrocode}

% \macro{\childdocredirect}
% The deprecated macro |\childdocredirect| is a legacy version
% of |\childdocforward| and |\childdocforwardprefix|:
%    \begin{macrocode}
\newcommand{\childdocredirect}[2][]
{
  \begingroup
    \if?#1?
      \def\childdoctmp{\childdocforward{#2}}
    \else
      \def\childdoctmp{\childdocforwardprefix{#1}{#2}}
    \fi
    \expandafter
  \endgroup
  \childdoctmp
}
%    \end{macrocode}

%\iffalse
%</package>
%\fi
%
\endinput
|
and perform the replacements as outlined below.
Instead of |\childdocmain{|\textit{main}|}| add the following code
to the top of the main file:
%
\begin{center}
\begin{tabular}{l}
|\||ifdefined\childdocname\endinput\||fi\newif\ifchilddoc|\\
|\edef\childdocname{\scantokens\expandafter{\jobname\noexpand}}|\\
|\def\childdocmain{|\textit{main}|}\||ifx\childdocmain\childdocname\||else|\\
|\childdoctrue\includeonly{\childdocname}\let\jobname\childdocmain\||fi|\\
\end{tabular}
\end{center}
%
Instead of |\childdocof{|\textit{main}|}| just include the main file
at the top of each child file:
%
\begin{center}
|\input{|\textit{main}|}|
\end{center}
%
A simple redirection |\childdocforward{|\textit{dest}|}| is achieved by:
%
\begin{center}
|\def\jobname{|\textit{dest}|}\input{\jobname}|
\end{center}
%
The redirection with prefix
|\childdocforwardprefix[|\textit{prefix}|]{|\textit{dest}|}|
is accomplished by:
%
\begin{center}
\begin{tabular}{l}
|{\edef\jobname{\scantokens\expandafter{\jobname\noexpand}}|\\
|\def\redirectjob |\textit{prefix}|#1~~~{\gdef\jobname{|\textit{dest}|#1}}|\\
|\expandafter\redirectjob\jobname~~~}\input{\jobname}|
\end{tabular}
\end{center}

In an alternative approach,
child documents can be compiled by a specific command line
without additional code or specific definitions:
%
\begin{center}
|... -jobname "|\textit{target}|" "|[\textit{flags}]%
|\includeonly{|\textit{dest}|}\input{|\textit{main}|}"|
\end{center}
%

%%%%%%%%%%%%%%%%%%%%%%%%%%%%%%%%%%%%%%%%%%%%%%%%%%%%%%%%%%%%%%%%%%%%%%%%%%%%%%%%
%%%%%%%%%%%%%%%%%%%%%%%%%%%%%%%%%%%%%%%%%%%%%%%%%%%%%%%%%%%%%%%%%%%%%%%%%%%%%%%%
\section{Information}

%%%%%%%%%%%%%%%%%%%%%%%%%%%%%%%%%%%%%%%%%%%%%%%%%%%%%%%%%%%%%%%%%%%%%%%%%%%%%%%%
\subsection{Copyright}

Copyright \copyright{} 2017--2018 Niklas Beisert

This work may be distributed and/or modified under the
conditions of the \LaTeX{} Project Public License, either version 1.3
of this license or (at your option) any later version.
The latest version of this license is in
  \url{http://www.latex-project.org/lppl.txt}
and version 1.3 or later is part of all distributions of \LaTeX{}
version 2005/12/01 or later.

This work has the LPPL maintenance status `maintained'.

The Current Maintainer of this work is Niklas Beisert.

This work consists of the files |README.txt|, |childdoc.ins| and |childdoc.dtx|
as well as the derived files |childdoc.def|, |cdocsamp.tex|
with |cdocsch1.tex|, |cdocsch2.tex|, |cdocspt3.tex|, |cdocspt4.tex|,
|cdocsdrf.tex|, |cdocsfn1.tex|, |cdocsfn2.tex|
as well as |childdoc.pdf|.

%%%%%%%%%%%%%%%%%%%%%%%%%%%%%%%%%%%%%%%%%%%%%%%%%%%%%%%%%%%%%%%%%%%%%%%%%%%%%%%%
\subsection{Files and Installation}

The package consists of the files:
%
\begin{center}
\begin{tabular}{ll}
    |README.txt|   & readme file \\
    |childdoc.ins| & installation file \\
    |childdoc.dtx| & source file \\
    |childdoc.def| & definition file \\
    |cdocsamp.tex| & sample main file \\
    |cdocsch1.tex| & sample include file \\
    |cdocsch2.tex| & sample include file \\
    |cdocspt3.tex| & sample part file \\
    |cdocspt4.tex| & sample part file \\
    |cdocsdrf.tex| & sample redirection file \\
    |cdocsfn1.tex| & sample redirection file \\
    |cdocsfn2.tex| & sample redirection file \\
    |childdoc.pdf| & manual
\end{tabular}
\end{center}
%
The distribution consists of the files
|README.txt|, |childdoc.ins| and |childdoc.dtx|.
%
\begin{itemize}
\item
Run (pdf)\LaTeX{} on |childdoc.dtx|
to compile the manual |childdoc.pdf| (this file).
\item
Run \LaTeX{} on |childdoc.ins| to create the definitions file |childdoc.def|
and the sample |cdocsamp.tex| with include files
|cdocsch1.tex|, |cdocsch2.tex|, |cdocspt3.tex|, |cdocspt4.tex|,
|cdocsdrf.tex|, |cdocsfn1.tex|, |cdocsfn2.tex|.
Then copy the file |childdoc.def| to an appropriate directory of your \LaTeX{}
distribution, e.g.\ \textit{texmf-root}|/tex/latex/childdoc|.
\end{itemize}

%%%%%%%%%%%%%%%%%%%%%%%%%%%%%%%%%%%%%%%%%%%%%%%%%%%%%%%%%%%%%%%%%%%%%%%%%%%%%%%%
\subsection{Related CTAN Packages}

There are several other packages which offer a similar functionality:
%
\begin{itemize}
\item
The packages
\href{http://ctan.org/pkg/docmute}{\textsf{docmute}},
\href{http://ctan.org/pkg/includex}{\textsf{includex}} and
\href{http://ctan.org/pkg/standalone}{\textsf{standalone}}
provide commands to include only the document body of
a child file thus allowing both files to be compiled individually.
\item
The packages \href{http://ctan.org/pkg/subdocs}{\textsf{subdocs}}
and \href{http://ctan.org/pkg/subfiles}{\textsf{subfiles}}
provide structures in which the main and child documents can be
encapsulated and allowing them to be compiled individually.
The inclusion mechanism is different from the conventional |\include|.
\item
The package \href{http://ctan.org/pkg/combine}{\textsf{combine}}
is an elaborate solution to combine several documents into one.
\end{itemize}
%
See also the CTAN topic \href{http://ctan.org/topic/subdocs}{\textsf{subdocs}}
for further related packages.
The present package differs from the above solutions in that
a document structure constructed with the conventional |\include| mechanism
just needs two extra commands at the top of every file
such that all constituent files can be compiled individually.

%%%%%%%%%%%%%%%%%%%%%%%%%%%%%%%%%%%%%%%%%%%%%%%%%%%%%%%%%%%%%%%%%%%%%%%%%%%%%%%%
%\subsection{Feature Suggestions}
%
%The following is a list of features which may be useful for future
%versions of this package:
%%
%\begin{itemize}
%\item
%\ldots
%\end{itemize}

%%%%%%%%%%%%%%%%%%%%%%%%%%%%%%%%%%%%%%%%%%%%%%%%%%%%%%%%%%%%%%%%%%%%%%%%%%%%%%%%
\subsection{Revision History}

%%%%%%%%%%%%%%%%%%%%%%%%%%%%%%%%%%%%%%%%
\paragraph{v2.0:} 2018/12/30

\begin{itemize}
\item
immediate forward processing
\item
added |\childdocby| mechanism
\item
manual restructured
\end{itemize}

%%%%%%%%%%%%%%%%%%%%%%%%%%%%%%%%%%%%%%%%
\paragraph{v1.6:} 2018/01/17

\begin{itemize}
\item
application for development of include files
\item
corrections to manual
\end{itemize}

%%%%%%%%%%%%%%%%%%%%%%%%%%%%%%%%%%%%%%%%
\paragraph{v1.5:} 2017/05/21

\begin{itemize}
\item
more complete structuring introduced
\item
|\childdocof| introduced
\item
|\childdoc| renamed to |\childdocmain|
\item
|\childredirect| renamed to |\childdocforward| and |\childdocforwardprefix|
and functionality expanded
\end{itemize}

%%%%%%%%%%%%%%%%%%%%%%%%%%%%%%%%%%%%%%%%
\paragraph{v1.0:} 2017/04/27

\begin{itemize}
\item
manual and install package
\item
first version published on CTAN
\end{itemize}

%%%%%%%%%%%%%%%%%%%%%%%%%%%%%%%%%%%%%%%%
\paragraph{v0.6:} 2017/04/26

\begin{itemize}
\item
redirection mechanism added
\end{itemize}

%%%%%%%%%%%%%%%%%%%%%%%%%%%%%%%%%%%%%%%%
\paragraph{v0.5:} 2017/04/26

\begin{itemize}
\item
functionality in definition file
\end{itemize}


%%%%%%%%%%%%%%%%%%%%%%%%%%%%%%%%%%%%%%%%%%%%%%%%%%%%%%%%%%%%%%%%%%%%%%%%%%%%%%%%
%%%%%%%%%%%%%%%%%%%%%%%%%%%%%%%%%%%%%%%%%%%%%%%%%%%%%%%%%%%%%%%%%%%%%%%%%%%%%%%%
%%%%%%%%%%%%%%%%%%%%%%%%%%%%%%%%%%%%%%%%%%%%%%%%%%%%%%%%%%%%%%%%%%%%%%%%%%%%%%%%
\appendix

\settowidth\MacroIndent{\rmfamily\scriptsize 000\ }

 \DocInput{childdoc.dtx}

\end{document}
%</driver>
% \fi
%
% %%%%%%%%%%%%%%%%%%%%%%%%%%%%%%%%%%%%%%%%%%%%%%%%%%%%%%%%%%%%%%%%%%%%%%%%%%%%%%
% %%%%%%%%%%%%%%%%%%%%%%%%%%%%%%%%%%%%%%%%%%%%%%%%%%%%%%%%%%%%%%%%%%%%%%%%%%%%%%
% \section{Sample}
%\iffalse
%<*samplemain>
%\fi
%
% The following presents a sample document
% with two chapters, two parts, a title page,
% a compile flag as well as three forwarding files to set the flag.
% It consists of eight |.tex| files:
% \begin{center}
% \begin{tabular}{ll}
% |cdocsamp.tex|&main file\\
% |cdocsch1.tex|&include file for chapter 1\\
% |cdocsch2.tex|&include file for chapter 2\\
% |cdocspt3.tex|&include file for part 3\\
% |cdocspt4.tex|&include file for part 4\\
% |cdocsdrf.tex|&forwarding file for main file in draft mode\\
% |cdocsfi1.tex|&forwarding file for final version of chapter 1\\
% |cdocsfi2.tex|&forwarding file for final version of chapter 2\\
% \end{tabular}
% \end{center}
% Each of the eight files can be compiled directly by the \LaTeX{} compiler.
%
% %%%%%%%%%%%%%%%%%%%%%%%%%%%%%%%%%%%%%%
% \paragraph{Main File.}
%
% The main file is called |cdocsamp.tex|.
%
% Load the \textsf{childdoc} definitions and
% declare the filename for the main document:
%    \begin{macrocode}
% \iffalse
%
% childdoc.dtx Copyright (C) 2017-2018 Niklas Beisert
%
% This work may be distributed and/or modified under the
% conditions of the LaTeX Project Public License, either version 1.3
% of this license or (at your option) any later version.
% The latest version of this license is in
%   http://www.latex-project.org/lppl.txt
% and version 1.3 or later is part of all distributions of LaTeX
% version 2005/12/01 or later.
%
% This work has the LPPL maintenance status `maintained'.
%
% The Current Maintainer of this work is Niklas Beisert.
%
% This work consists of the files childdoc.dtx and childdoc.ins
% and the derived files childdoc.def and cdocsamp.tex with
% cdocsch1.tex, cdocsch2.tex, cdocsdrf.tex, cdocsfn1.tex, cdocsfn2.tex.
%
%<package>\ifdefined\childdocmain\endinput\fi
%<package>\ProvidesFile{childdoc.def}[2018/12/30 v2.0 child document driver]
%<samplemain>\ProvidesFile{cdocsamp.tex}[2018/12/30 v2.0 sample for childdoc]
%<*driver>
%\ProvidesFile{childdoc.drv}[2018/12/30 v2.0 childdoc reference manual file]
\PassOptionsToClass{10pt,a4paper}{article}
\documentclass{ltxdoc}

\usepackage[margin=35mm]{geometry}
\usepackage{hyperref}
\usepackage{hyperxmp}
\usepackage[usenames]{color}

\hypersetup{colorlinks=true}
\hypersetup{pdfstartview=FitH}
\hypersetup{pdfpagemode=UseNone}
\hypersetup{pdfsource={}}
\hypersetup{pdflang={en-UK}}
\hypersetup{pdfcopyright={Copyright 2017-2018 Niklas Beisert.
  This work may be distributed and/or modified under the
  conditions of the LaTeX Project Public License, either version 1.3
  of this license or (at your option) any later version.}}
\hypersetup{pdflicenseurl={http://www.latex-project.org/lppl.txt}}
\hypersetup{pdfcontactaddress={ETH Zurich, ITP, HIT K,
  Wolfgang-Pauli-Strasse 27}}
\hypersetup{pdfcontactpostcode={8093}}
\hypersetup{pdfcontactcity={Zurich}}
\hypersetup{pdfcontactcountry={Switzerland}}
\hypersetup{pdfcontactemail={nbeisert@itp.phys.ethz.ch}}
\hypersetup{pdfcontacturl={http://people.phys.ethz.ch/\xmptilde nbeisert/}}

\newcommand{\secref}[1]{\hyperref[#1]{section \ref*{#1}}}

\parskip1ex
\parindent0pt
\let\olditemize\itemize
\def\itemize{\olditemize\parskip0pt}

\begin{document}

\title{The \textsf{childdoc} Package}
\hypersetup{pdftitle={The childdoc Package}}
\author{Niklas Beisert\\[2ex]
  Institut f\"ur Theoretische Physik\\
  Eidgen\"ossische Technische Hochschule Z\"urich\\
  Wolfgang-Pauli-Strasse 27, 8093 Z\"urich, Switzerland\\[1ex]
  \href{mailto:nbeisert@itp.phys.ethz.ch}
  {\texttt{nbeisert@itp.phys.ethz.ch}}}
\hypersetup{pdfauthor={Niklas Beisert}}
\hypersetup{pdfsubject={Manual for the LaTeX2e Package childdoc}}
\date{30 December 2018, \textsf{v2.0}}
\maketitle

\begin{abstract}\noindent
\textsf{childdoc} is a \LaTeXe{} package
that enables the direct compilation
of document sections included by |\include|
to individual files.
\end{abstract}

\begingroup
\parskip0ex
\tableofcontents
\endgroup

%%%%%%%%%%%%%%%%%%%%%%%%%%%%%%%%%%%%%%%%%%%%%%%%%%%%%%%%%%%%%%%%%%%%%%%%%%%%%%%%
%%%%%%%%%%%%%%%%%%%%%%%%%%%%%%%%%%%%%%%%%%%%%%%%%%%%%%%%%%%%%%%%%%%%%%%%%%%%%%%%
\section{Introduction}

\LaTeX{} provides a mechanism to structure a large document (such as a book)
into a main file and several child files (containing the chapters)
using the |\include| command.
This mechanism is beneficial for documents
which span hundreds of pages in order to
make the source file(s) more manageable.
Moreover, compilation can be restricted to
selected child files by means of the |\includeonly| command.
The latter feature can be used to reduce the compilation time while editing
(this was significantly more useful in the earlier days of \LaTeX{})
or to generate a smaller document which is easier to navigate.
Another application of |\includeonly| is to generate
documents consisting of selected parts of the complete document.

However, there are a few drawbacks of the plain |\include| mechanism:
\begin{itemize}
\item
The child files cannot be compiled on their own,
they can only be compiled via the main file.
A naive editing environment
(such as a text editor with an option
to have the current file processed by \LaTeX)
may require one to switch to the main file before compiling;
attempting to compile the child file produces errors.
\item
The main file must be modified (each time)
to adjust the |\includeonly| command
to the present needs. This easily leaves the main file in a messy state.
\item
The generated document will always carry the filename
of the main document. This is inconvenient if
several child files are to be compiled and
to be kept for distribution.
\end{itemize}

The present package provides a simple interface
to make child files individually compilable by \LaTeX{}.
Compiling a child file then has the same effect as compiling
the main file with an |\includeonly| command
to select the appropriate child.
Moreover the generated document will carry the name of the child
rather than the main file.
This resolves all three above issues.

This feature is meant to make the editing of books,
thesis documents and lecture notes somewhat more convenient.
However, the package can also be used efficiently for
composing a series of documents (such as exercise sheets)
which are typically distributed individually.
It then assists the author in generating the individual documents
(potentially in different versions)
as well as a document containing the collected series.
Another application is in developing style files
or other kinds of included material
where compilation of the style file could redirect
to a sample or test file.

%%%%%%%%%%%%%%%%%%%%%%%%%%%%%%%%%%%%%%%%%%%%%%%%%%%%%%%%%%%%%%%%%%%%%%%%%%%%%%%%
%%%%%%%%%%%%%%%%%%%%%%%%%%%%%%%%%%%%%%%%%%%%%%%%%%%%%%%%%%%%%%%%%%%%%%%%%%%%%%%%
\section{Usage}

First of all, the package \textsf{childdoc} is \emph{not} a standard
\LaTeXe{} |.sty| style file! Therefore it needs to be invoked in
a non-standard way.

%%%%%%%%%%%%%%%%%%%%%%%%%%%%%%%%%%%%%%%%%%%%%%%%%%%%%%%%%%%%%%%%%%%%%%%%%%%%%%%%
\subsection{Included Files}
\label{sec:include}

%%%%%%%%%%%%%%%%%%%%%%%%%%%%%%%%%%%%%%%%
\DescribeMacro{\childdocmain}
To use the package, add the commands
\begin{center}
\begin{tabular}{l}
|\input{childdoc.def}|\\
|\childdocmain{}|\\
\end{tabular}
\end{center}
at the very top of the main \LaTeX{} file,
in particular \emph{before} the |\documentclass| statement!
The argument of |\childdocmain| should be left empty
(but it must be present).

%%%%%%%%%%%%%%%%%%%%%%%%%%%%%%%%%%%%%%%%
\DescribeMacro{\childdocof}
Furthermore, add the commands
\begin{center}
\begin{tabular}{l}
|\input{childdoc.def}|\\
|\childdocof{|\textit{main}|}|\\
\end{tabular}
\end{center}
at the top of every child file \textit{child}
which is included by |\include{|\textit{child}|}|
from within the main file
(or at least for those files to be compiled individually).
The argument \textit{main} must be the filename of the main file.

There are a couple of
considerations in setting up the main and child documents:

%%%%%%%%%%%%%%%%%%%%%%%%%%%%%%%%%%%%%%%%
\paragraph{Restrictions.}

Please note the following restrictions:
\begin{itemize}
\item
|\childdocmain| must be called with one argument \textit{main}
to ensure compatibility with earlier version of the package.
It must either be empty (|\childdocmain{}|)
or precisely match the filename of the main file in which it is specified.
See \secref{sec:detection} for further information.
\item
The filename \textit{main} must be specified without the |.tex| extension.
\item
The filename \textit{main} is case sensitive
(even in case-insensitive file systems)
due to internal string comparison.
\item
The argument \textit{main} should be fully expanded, it cannot be a macro.
\item
Subdirectories and special characters should be avoided in filenames.
\item
The command |\childdocmain{|\textit{main}|}| must be followed by a whitespace.
It should not be followed immediately by another command
or by a comment mark `|%|'.
This is because the \TeX{} parser reads the token immediately following
the argument of |\childdocmain| and puts it
at the beginning of every child section;
however, a white\-space is ignored.
\end{itemize}

%%%%%%%%%%%%%%%%%%%%%%%%%%%%%%%%%%%%%%%%
\paragraph{Content of Main File.}

It is advisable to place all content in the child files included by |\include|.
Any output contained in the main file will appear in all child documents
unless suppressed manually;
it cannot be suppressed automatically by the |\includeonly| directive
and thus should normally be avoided.
A method to include some content in the main file
by means of conditional processing is described in \secref{sec:conditional}.

%%%%%%%%%%%%%%%%%%%%%%%%%%%%%%%%%%%%%%%%
\paragraph{Page Numbering.}

When only a part of the document is compiled,
the appropriate numbering of pages
(as well as other status parameters)
is determined from the |.aux| files.
The latter contain information from previous passes.
However this information needs to propagate through
all intermediate child documents.
Therefore the page numbering in child documents may well
be inconsistent until the complete document is compiled at least once.

A useful (if unconventional) way to always ensure a consistent
page numbering is to restart the numbering in each child document
and denote the pages by `\textit{child}|.|\textit{page}'
where \textit{child} represents the chapter/section number of the child file.
This can be achieved by the command
|\numberwithin{page}{|\textit{child}|}|
of the \textsf{amsmath} package
where \textit{child} can be |chapter| or |section|
depending on the chosen structuring.
Alternatively, one can modify the macro |\thepage| appropriately
and reset the counter |page| at the start of each child file.

%%%%%%%%%%%%%%%%%%%%%%%%%%%%%%%%%%%%%%%%%%%%%%%%%%%%%%%%%%%%%%%%%%%%%%%%%%%%%%%%
\subsection{Conditional Processing}
\label{sec:conditional}

The package provides a mechanism to compile different versions
of a document. To customise the versions further some conditional processing
can come in handy to distinguish which version is being compiled.
The package provides two macros to describe the compilation context:

%%%%%%%%%%%%%%%%%%%%%%%%%%%%%%%%%%%%%%%%
\DescribeMacro{\ifchilddoc}
The conditional |\ifchilddoc| distinguishes between the compilation of
child documents and the main document:
%
\begin{center}
|\ifchilddoc |\textit{child-code}| |[|\||else |\textit{main-code}]| \||fi|
\end{center}

%%%%%%%%%%%%%%%%%%%%%%%%%%%%%%%%%%%%%%%%
\DescribeMacro{\childdocname}
\DescribeMacro{\childdocjob}
The macro |\childdocname| contains the filename (without extension)
of the main or child file being processed.
Note that |\childdocjob| will always contain the name of the main file.

%%%%%%%%%%%%%%%%%%%%%%%%%%%%%%%%%%%%%%%%
\paragraph{Title Page.}

Conditional processing can be used to include a title or banner page
in the main document when proper precautions are taken.
Importantly, the code in the main file should ensure that the page counter
(as well as other status parameters which are stored in the |.aux| files)
takes the same value after the conditional processing.
Otherwise the page numbers may take divergent values
depending on which part is compiled.

For example, a title page could be declared by:
%
\begin{center}
\begin{tabular}{l}
|\ifchilddoc\||else|\\
|\addtocounter{page}{-1}|\\
\textit{code for title page}\\
|\newpage|\\
|\||fi|
\end{tabular}
\end{center}
%
A banner page for the child documents can be generated by:
%
\begin{center}
\begin{tabular}{l}
|\ifchilddoc|\\
|\addtocounter{page}{-1}|\\
\textit{code for banner page}\\
|\newpage|\\
|\||fi|
\end{tabular}
\end{center}
%
Here one could write a message such as:
\begin{center}
|This is the part \childdocname{} of \childdocjob{}.|
\end{center}

%%%%%%%%%%%%%%%%%%%%%%%%%%%%%%%%%%%%%%%%%%%%%%%%%%%%%%%%%%%%%%%%%%%%%%%%%%%%%%%%
\subsection{Flags}
\label{sec:flags}

The package makes it easy to generate different versions
of the main or child documents.
To this end compilation flags can be defined
and assigned different default values.
They will be particularly useful in conjunction
with the forwarding mechanism described in \secref{sec:forward}.

For example, it may be useful to have a flag |\version|
which can be set to |draft| or |final|.
The document source will contain some conditional code
depending on the value of |\version|.
Suppose further, the flag should default to |final| for the main file
and to |draft| for child files
which is a natural assignment for editing the document.
This is achieved by placing the following code
in the preamble of the main document
(below the |\childdocmain| directive):
%
\begin{center}
\begin{tabular}{l}
|\ifchilddoc|\\
|\providecommand{\version}{draft}|\\
|\||else|\\
|\providecommand{\version}{final}|\\
|\||fi|
\end{tabular}
\end{center}
%
The definition by |\providecommand| makes sure
that previous definitions are not overwritten.
Further statements |\providecommand{\version}{...}|
can thus be added before the above code to override it.

For the main file, one might add a line
(between |\childdocmain| and the above block)
%
\begin{center}
|%\ifchilddoc\||else\providecommand{\version}{draft}\||fi|
\end{center}
%
which can be uncommented to produce a draft version.
Likewise one can add a line to the very top of a child file
(above the |\childdocof{|\textit{main}|}| directive)
%
\begin{center}
|%\providecommand{\version}{final}|
\end{center}
%
which can be uncommented to produce the final version of this child document.

%%%%%%%%%%%%%%%%%%%%%%%%%%%%%%%%%%%%%%%%%%%%%%%%%%%%%%%%%%%%%%%%%%%%%%%%%%%%%%%%
\subsection{Forwarding}
\label{sec:forward}

Different versions of the main or child documents
using compilation flags as described in \secref{sec:flags}
can be (permanently) stored in different files
for convenient compilation, viewing and distribution.
To this end, the package defines a command
to pass on compilation to a different file:

%%%%%%%%%%%%%%%%%%%%%%%%%%%%%%%%%%%%%%%%
\DescribeMacro{\childdocforward}
The command |\childdocforward| redirects processing to
another source file:
%
\begin{center}
\begin{tabular}{l}
|\input{childdoc.def}|\\
|\childdocforward[|\textit{main}|]{|\textit{dest}|}|\\
\end{tabular}
\end{center}
%
The argument \textit{dest} is the destination file
(without extension).
It should be the main file or one of the child files.
Note that further \textsf{childdoc} directives
such as |\childdocof| and |\childdocforward|
in the indicated file will be processed in this form.
The optional argument \textit{main}
passes on directly to the main file \textit{main}
while pretending to compile the child \textit{dest}.
This form behaves as if \textit{dest}
issues |\childdocof{|\textit{main}|}| right away,
and no further \textsf{childdoc} directives will be processed.

%%%%%%%%%%%%%%%%%%%%%%%%%%%%%%%%%%%%%%%%
\DescribeMacro{\...prefix}
In the alternative form |\childdocforwardprefix|,
%
\begin{center}
\begin{tabular}{l}
|\input{childdoc.def}|\\
|\childdocforwardprefix[|\textit{main}|]{|\textit{prefix}|}{|\textit{dest}|}|
\end{tabular}
\end{center}
%
the destination file is determined by a pattern
depending on the current file:
To make this work, the current file must be called
`{\textit{prefix}\hspace{0.2em}\textit{suffix}}'
with \textit{prefix} matching precisely the argument.
Processing is then passed on to the file
`{\textit{dest}\hspace{0.2em}\textit{suffix}}'.
Surely, the same effect is achieved by
directly specifying the
argument `{\textit{dest}\hspace{0.2em}\textit{suffix}}'
in the first form.
However, that requires to set up a different file
for each child. With the alternative form of the command
all these files can have exactly the same content
which simplifies setting them up and maintaining them.

For example, the following file |draft.tex|
with a compilation flag |\version| as described in \secref{sec:flags}
compiles the main document as a draft:
%
\begin{center}
\begin{tabular}{l}
|\def\version{draft}|\\
|\input{childdoc.def}|\\
|\childdocforward{|\textit{main}|}|
\end{tabular}
\end{center}
%
Likewise, the following files |final|\textit{nn}|.tex|
compile the final version of the child document
|child|\textit{nn}|.tex|:
%
\begin{center}
\begin{tabular}{l}
|\def\version{final}|\\
|\input{childdoc.def}|\\
|\childdocforwardprefix{final}{child}|
\end{tabular}
\end{center}
%

Note that when several versions of a main file and/or of each child file
are to be generated, it may be convenient to set up a |Makefile| or
shell script to automatise the process.

%%%%%%%%%%%%%%%%%%%%%%%%%%%%%%%%%%%%%%%%%%%%%%%%%%%%%%%%%%%%%%%%%%%%%%%%%%%%%%%%
\subsection{Command Line Processing}
\label{sec:commandline}

The effect of redirection files can also be achieved by invoking
the \LaTeX{} compiler with a more elaborate command line.
Most conveniently this should be done as part
of a shell script or a |Makefile|.

When using \textsf{childdoc} in the main file, the following
command lines effectively perform a redirection
(note that depending on the shell being used,
backslashes may have to be doubled: `|\|' $\to$ `|\\|'):
%
\begin{center}
|... -jobname "|\textit{target}|" |\\|"|[\textit{flags}]%
|\input{childdoc.def}\childdocforward[|\textit{main}|]{|\textit{dest}|}"|
\end{center}
%
Here \textit{target} is the name of the output file,
\textit{main} is the name of the main file
and \textit{dest} is the name of the main or child file to be processed
(all filenames without extensions).
The optional argument \textit{main} can be omitted
if \textit{main} matches \textit{dest}.
Optionally, compilation \textit{flags} can be defined via |\def| commands.
This command line makes the \TeX{} engine believe
it is compiling the file \textit{target}
whose content is specified as the latter parameter.
The provided code then forwards the processing to
\textit{main} or \textit{dest} as described in \secref{sec:forward}.

%%%%%%%%%%%%%%%%%%%%%%%%%%%%%%%%%%%%%%%%%%%%%%%%%%%%%%%%%%%%%%%%%%%%%%%%%%%%%%%%
\subsection{Include by Input}
\label{sec:input}

Including child documents by |\include| has some restrictions by design.
Most notably, the content of a child document always occupies
its own set of pages; pages cannot be shared between child documents.
Usually, this behaviour makes perfect sense
because each child document contain an essential part of the document.
However, in some situations it may be desirable to compose
a document from a collection of parts
without having mandatory page breaks between then.
For this case, the package
provides a mechanism to include parts
by |\input| which can also be processed individually.
However, by construction this mechanism
requires manual handling of the content to be output.

%%%%%%%%%%%%%%%%%%%%%%%%%%%%%%%%%%%%%%%%
\DescribeMacro{\ifchilddocmanual}
The main file should be prepared as usual, see \secref{sec:include}.
However, the document body must make a distinction
between processing of an individual part and of the main document, e.g.:
%
\begin{center}
\begin{tabular}{l}
|\ifchilddocmanual|\\
|\input{\childdocname}|\\
|\||else|\\
\textit{document body with }|\input{|\textit{part}|}|\\
|\||fi|
\end{tabular}
\end{center}
%
The conditional |\ifchilddocmanual| is true whenever
a part to be included by |\input| is being compiled,
and the name of the part is stored in |\childdocname|.

%%%%%%%%%%%%%%%%%%%%%%%%%%%%%%%%%%%%%%%%
\DescribeMacro{\childdocby}
Each part to be included by |\input| should start with:
%
\begin{center}
\begin{tabular}{l}
|\input{childdoc.def}|\\
|\childdocby{|\textit{main}|}|\\
\end{tabular}
\end{center}
%
The directive |\childdocby| is similar to |\childdocof|
described in \secref{sec:include},
but the subsequent selection of content must be done manually.
To that end, both |\ifchilddoc| and |\ifchilddocmanual|
will be true upon processing of a part,
and the name of the part is stored in |\childdocname|.
Note that |\jobname| will be set to the filename of the current part
so that each part receives an individual |.aux| file
that does not interfere with the |.aux| file(s) of the main document.
This behaviour can be altered by the alternative form
|\childdocby[*]{|\textit{main}|}| (with a non-empty optional argument)
which uses the |.aux| file of the main document
by setting |\jobname| to \textit{main}.

%%%%%%%%%%%%%%%%%%%%%%%%%%%%%%%%%%%%%%%%%%%%%%%%%%%%%%%%%%%%%%%%%%%%%%%%%%%%%%%%
\subsection{Driver Development}
\label{sec:driver}

The \textsf{childdoc} mechanism can also be use for the development
of definition files such as \LaTeX{} styles or classes.
This case differs from the above setup with multiple parts
included by |\include| in that no |\includeonly| should be invoked.
This can be achieved by starting the include file
(before |\ProvidesPackage|) with:
%
\begin{center}
\begin{tabular}{l}
|\input{childdoc.def}|\\
|\childdocforward{|\textit{main}|}|\\
\end{tabular}
\end{center}
%
or alternatively with:
%
\begin{center}
\begin{tabular}{l}
|\input{childdoc.def}|\\
|\childdocby{|\textit{main}|}|\\
\end{tabular}
\end{center}
%
Both forms have slightly different effects as described above.
The main file is prepared as usual, see \secref{sec:include}.

%%%%%%%%%%%%%%%%%%%%%%%%%%%%%%%%%%%%%%%%%%%%%%%%%%%%%%%%%%%%%%%%%%%%%%%%%%%%%%%%
\subsection{Legacy Detection}
\label{sec:detection}

The directive |\childdocmain| in the main file can detect
whether the complete document or merely a child is to be compiled
even without using the directive |\childdocof|.
This method is deprecated because it is less robust
and there is no compelling reason to use it;
it is merely provided for backward compatibility
and it may be removed in future versions.

If the detection mechanism is to be used,
it is mandatory to correctly specify
the filename of the main file as the argument of |\childdocmain|:
%
\begin{center}
\begin{tabular}{l}
|\input{childdoc.def}|\\
|\childdocmain{|\textit{main}|}|\\
\end{tabular}
\end{center}
%
If |\jobname| does not match the argument \textit{main} of |\childdocmain|,
it is assumed that |\jobname| points to the child file to be compiled.
When using |\childdocmain| with the main file specified as argument,
it suffices to start a child file
with just |\input{|\textit{main}|}|
without loading of the package and using |\childdocof|.
If instead all processing is done
with the appropriate \textsf{childdoc} directives,
the argument of \textit{main} of |\childdocmain| can be empty.

An alternative version of the command line processing described
in \secref{sec:commandline} using the detection mechanism reads:
%
\begin{center}
|... -jobname "|\textit{target}|" "|[\textit{flags}]%
[|\def\jobname{|\textit{dest}|}|]|\input{|\textit{main}|}"|
\end{center}

%%%%%%%%%%%%%%%%%%%%%%%%%%%%%%%%%%%%%%%%%%%%%%%%%%%%%%%%%%%%%%%%%%%%%%%%%%%%%%%%
\subsection{Manual Code}
\label{sec:manual}

In case one cannot be certain whether the definitions file |childdoc.def|
is installed on the target \TeX{} distribution
and one prefers not to ship it,
it is conceivable to paste a few relevant commands into the sources.

To that end, drop all statements |\input{childdoc.def}|
and perform the replacements as outlined below.
Instead of |\childdocmain{|\textit{main}|}| add the following code
to the top of the main file:
%
\begin{center}
\begin{tabular}{l}
|\||ifdefined\childdocname\endinput\||fi\newif\ifchilddoc|\\
|\edef\childdocname{\scantokens\expandafter{\jobname\noexpand}}|\\
|\def\childdocmain{|\textit{main}|}\||ifx\childdocmain\childdocname\||else|\\
|\childdoctrue\includeonly{\childdocname}\let\jobname\childdocmain\||fi|\\
\end{tabular}
\end{center}
%
Instead of |\childdocof{|\textit{main}|}| just include the main file
at the top of each child file:
%
\begin{center}
|\input{|\textit{main}|}|
\end{center}
%
A simple redirection |\childdocforward{|\textit{dest}|}| is achieved by:
%
\begin{center}
|\def\jobname{|\textit{dest}|}\input{\jobname}|
\end{center}
%
The redirection with prefix
|\childdocforwardprefix[|\textit{prefix}|]{|\textit{dest}|}|
is accomplished by:
%
\begin{center}
\begin{tabular}{l}
|{\edef\jobname{\scantokens\expandafter{\jobname\noexpand}}|\\
|\def\redirectjob |\textit{prefix}|#1~~~{\gdef\jobname{|\textit{dest}|#1}}|\\
|\expandafter\redirectjob\jobname~~~}\input{\jobname}|
\end{tabular}
\end{center}

In an alternative approach,
child documents can be compiled by a specific command line
without additional code or specific definitions:
%
\begin{center}
|... -jobname "|\textit{target}|" "|[\textit{flags}]%
|\includeonly{|\textit{dest}|}\input{|\textit{main}|}"|
\end{center}
%

%%%%%%%%%%%%%%%%%%%%%%%%%%%%%%%%%%%%%%%%%%%%%%%%%%%%%%%%%%%%%%%%%%%%%%%%%%%%%%%%
%%%%%%%%%%%%%%%%%%%%%%%%%%%%%%%%%%%%%%%%%%%%%%%%%%%%%%%%%%%%%%%%%%%%%%%%%%%%%%%%
\section{Information}

%%%%%%%%%%%%%%%%%%%%%%%%%%%%%%%%%%%%%%%%%%%%%%%%%%%%%%%%%%%%%%%%%%%%%%%%%%%%%%%%
\subsection{Copyright}

Copyright \copyright{} 2017--2018 Niklas Beisert

This work may be distributed and/or modified under the
conditions of the \LaTeX{} Project Public License, either version 1.3
of this license or (at your option) any later version.
The latest version of this license is in
  \url{http://www.latex-project.org/lppl.txt}
and version 1.3 or later is part of all distributions of \LaTeX{}
version 2005/12/01 or later.

This work has the LPPL maintenance status `maintained'.

The Current Maintainer of this work is Niklas Beisert.

This work consists of the files |README.txt|, |childdoc.ins| and |childdoc.dtx|
as well as the derived files |childdoc.def|, |cdocsamp.tex|
with |cdocsch1.tex|, |cdocsch2.tex|, |cdocspt3.tex|, |cdocspt4.tex|,
|cdocsdrf.tex|, |cdocsfn1.tex|, |cdocsfn2.tex|
as well as |childdoc.pdf|.

%%%%%%%%%%%%%%%%%%%%%%%%%%%%%%%%%%%%%%%%%%%%%%%%%%%%%%%%%%%%%%%%%%%%%%%%%%%%%%%%
\subsection{Files and Installation}

The package consists of the files:
%
\begin{center}
\begin{tabular}{ll}
    |README.txt|   & readme file \\
    |childdoc.ins| & installation file \\
    |childdoc.dtx| & source file \\
    |childdoc.def| & definition file \\
    |cdocsamp.tex| & sample main file \\
    |cdocsch1.tex| & sample include file \\
    |cdocsch2.tex| & sample include file \\
    |cdocspt3.tex| & sample part file \\
    |cdocspt4.tex| & sample part file \\
    |cdocsdrf.tex| & sample redirection file \\
    |cdocsfn1.tex| & sample redirection file \\
    |cdocsfn2.tex| & sample redirection file \\
    |childdoc.pdf| & manual
\end{tabular}
\end{center}
%
The distribution consists of the files
|README.txt|, |childdoc.ins| and |childdoc.dtx|.
%
\begin{itemize}
\item
Run (pdf)\LaTeX{} on |childdoc.dtx|
to compile the manual |childdoc.pdf| (this file).
\item
Run \LaTeX{} on |childdoc.ins| to create the definitions file |childdoc.def|
and the sample |cdocsamp.tex| with include files
|cdocsch1.tex|, |cdocsch2.tex|, |cdocspt3.tex|, |cdocspt4.tex|,
|cdocsdrf.tex|, |cdocsfn1.tex|, |cdocsfn2.tex|.
Then copy the file |childdoc.def| to an appropriate directory of your \LaTeX{}
distribution, e.g.\ \textit{texmf-root}|/tex/latex/childdoc|.
\end{itemize}

%%%%%%%%%%%%%%%%%%%%%%%%%%%%%%%%%%%%%%%%%%%%%%%%%%%%%%%%%%%%%%%%%%%%%%%%%%%%%%%%
\subsection{Related CTAN Packages}

There are several other packages which offer a similar functionality:
%
\begin{itemize}
\item
The packages
\href{http://ctan.org/pkg/docmute}{\textsf{docmute}},
\href{http://ctan.org/pkg/includex}{\textsf{includex}} and
\href{http://ctan.org/pkg/standalone}{\textsf{standalone}}
provide commands to include only the document body of
a child file thus allowing both files to be compiled individually.
\item
The packages \href{http://ctan.org/pkg/subdocs}{\textsf{subdocs}}
and \href{http://ctan.org/pkg/subfiles}{\textsf{subfiles}}
provide structures in which the main and child documents can be
encapsulated and allowing them to be compiled individually.
The inclusion mechanism is different from the conventional |\include|.
\item
The package \href{http://ctan.org/pkg/combine}{\textsf{combine}}
is an elaborate solution to combine several documents into one.
\end{itemize}
%
See also the CTAN topic \href{http://ctan.org/topic/subdocs}{\textsf{subdocs}}
for further related packages.
The present package differs from the above solutions in that
a document structure constructed with the conventional |\include| mechanism
just needs two extra commands at the top of every file
such that all constituent files can be compiled individually.

%%%%%%%%%%%%%%%%%%%%%%%%%%%%%%%%%%%%%%%%%%%%%%%%%%%%%%%%%%%%%%%%%%%%%%%%%%%%%%%%
%\subsection{Feature Suggestions}
%
%The following is a list of features which may be useful for future
%versions of this package:
%%
%\begin{itemize}
%\item
%\ldots
%\end{itemize}

%%%%%%%%%%%%%%%%%%%%%%%%%%%%%%%%%%%%%%%%%%%%%%%%%%%%%%%%%%%%%%%%%%%%%%%%%%%%%%%%
\subsection{Revision History}

%%%%%%%%%%%%%%%%%%%%%%%%%%%%%%%%%%%%%%%%
\paragraph{v2.0:} 2018/12/30

\begin{itemize}
\item
immediate forward processing
\item
added |\childdocby| mechanism
\item
manual restructured
\end{itemize}

%%%%%%%%%%%%%%%%%%%%%%%%%%%%%%%%%%%%%%%%
\paragraph{v1.6:} 2018/01/17

\begin{itemize}
\item
application for development of include files
\item
corrections to manual
\end{itemize}

%%%%%%%%%%%%%%%%%%%%%%%%%%%%%%%%%%%%%%%%
\paragraph{v1.5:} 2017/05/21

\begin{itemize}
\item
more complete structuring introduced
\item
|\childdocof| introduced
\item
|\childdoc| renamed to |\childdocmain|
\item
|\childredirect| renamed to |\childdocforward| and |\childdocforwardprefix|
and functionality expanded
\end{itemize}

%%%%%%%%%%%%%%%%%%%%%%%%%%%%%%%%%%%%%%%%
\paragraph{v1.0:} 2017/04/27

\begin{itemize}
\item
manual and install package
\item
first version published on CTAN
\end{itemize}

%%%%%%%%%%%%%%%%%%%%%%%%%%%%%%%%%%%%%%%%
\paragraph{v0.6:} 2017/04/26

\begin{itemize}
\item
redirection mechanism added
\end{itemize}

%%%%%%%%%%%%%%%%%%%%%%%%%%%%%%%%%%%%%%%%
\paragraph{v0.5:} 2017/04/26

\begin{itemize}
\item
functionality in definition file
\end{itemize}


%%%%%%%%%%%%%%%%%%%%%%%%%%%%%%%%%%%%%%%%%%%%%%%%%%%%%%%%%%%%%%%%%%%%%%%%%%%%%%%%
%%%%%%%%%%%%%%%%%%%%%%%%%%%%%%%%%%%%%%%%%%%%%%%%%%%%%%%%%%%%%%%%%%%%%%%%%%%%%%%%
%%%%%%%%%%%%%%%%%%%%%%%%%%%%%%%%%%%%%%%%%%%%%%%%%%%%%%%%%%%%%%%%%%%%%%%%%%%%%%%%
\appendix

\settowidth\MacroIndent{\rmfamily\scriptsize 000\ }

 \DocInput{childdoc.dtx}

\end{document}
%</driver>
% \fi
%
% %%%%%%%%%%%%%%%%%%%%%%%%%%%%%%%%%%%%%%%%%%%%%%%%%%%%%%%%%%%%%%%%%%%%%%%%%%%%%%
% %%%%%%%%%%%%%%%%%%%%%%%%%%%%%%%%%%%%%%%%%%%%%%%%%%%%%%%%%%%%%%%%%%%%%%%%%%%%%%
% \section{Sample}
%\iffalse
%<*samplemain>
%\fi
%
% The following presents a sample document
% with two chapters, two parts, a title page,
% a compile flag as well as three forwarding files to set the flag.
% It consists of eight |.tex| files:
% \begin{center}
% \begin{tabular}{ll}
% |cdocsamp.tex|&main file\\
% |cdocsch1.tex|&include file for chapter 1\\
% |cdocsch2.tex|&include file for chapter 2\\
% |cdocspt3.tex|&include file for part 3\\
% |cdocspt4.tex|&include file for part 4\\
% |cdocsdrf.tex|&forwarding file for main file in draft mode\\
% |cdocsfi1.tex|&forwarding file for final version of chapter 1\\
% |cdocsfi2.tex|&forwarding file for final version of chapter 2\\
% \end{tabular}
% \end{center}
% Each of the eight files can be compiled directly by the \LaTeX{} compiler.
%
% %%%%%%%%%%%%%%%%%%%%%%%%%%%%%%%%%%%%%%
% \paragraph{Main File.}
%
% The main file is called |cdocsamp.tex|.
%
% Load the \textsf{childdoc} definitions and
% declare the filename for the main document:
%    \begin{macrocode}
\input{childdoc.def}
\childdocmain{}
%    \end{macrocode}

% Optional override for |\version| flag:
%    \begin{macrocode}
%%\ifchilddoc\else\providecommand{\version}{draft}\fi
%    \end{macrocode}

% Define the default values for the |\version| flag
% (|final| for the main file and |draft| for childs):
%    \begin{macrocode}
\ifchilddoc
\providecommand{\version}{draft}
\else
\providecommand{\version}{final}
\fi
%    \end{macrocode}

% Load the standard document class:
%    \begin{macrocode}
\documentclass[12pt]{article}
%    \end{macrocode}

% Start the document body:
%    \begin{macrocode}
\begin{document}
%    \end{macrocode}

% Declare a title page.
% Print title, part of document being processed and version flag:
%    \begin{macrocode}
\addtocounter{page}{-1}
\begin{center}
{\LARGE\bfseries{}childdoc example\par}
\vspace{1cm}
\ifchilddoc
\ifchilddocmanual part\else chapter\fi:
`\childdocname' of `\childdocjob'\par
\else
main document: `\childdocjob'\par
\fi
version: \version\par
\end{center}
\newpage
%    \end{macrocode}

% Manually include selected file,
% otherwise process as usual:
%    \begin{macrocode}
\ifchilddocmanual
\section*{part `\childdocname'}
\input{\childdocname}
\else
%    \end{macrocode}

% Include the two chapters:
%    \begin{macrocode}
\include{cdocsch1}
\include{cdocsch2}
%    \end{macrocode}

% Include the two parts unless only chapters should be displayed:
%    \begin{macrocode}
\ifchilddoc\else
\section{part three}
\input{cdocspt3}
\section{part four}
\input{cdocspt4}
\fi
%    \end{macrocode}

% Process as usual until here:
%    \begin{macrocode}
\fi
%    \end{macrocode}

% End of document body:
%    \begin{macrocode}
\end{document}
%    \end{macrocode}
%\iffalse
%</samplemain>
%\fi
%
% %%%%%%%%%%%%%%%%%%%%%%%%%%%%%%%%%%%%%%
% \paragraph{Chapter Include Files.}
%
% The include files are called |cdocsch1.tex| and |cdocsch2.tex|.
%
%\iffalse
%<*samplechap1|samplechap2>
%\fi

% Optional override for |\version| flag:
%    \begin{macrocode}
%%\providecommand{\version}{final}
%    \end{macrocode}

% Include the main document:
%    \begin{macrocode}
\input{childdoc.def}
\childdocof{cdocsamp}
%    \end{macrocode}

%\iffalse
%</samplechap1|samplechap2>
%\fi
%
%\iffalse
%<*samplechap1>
%\fi
% Some text for chapter 1:
%    \begin{macrocode}
\section{one}
some text in chapter one
%    \end{macrocode}

%\iffalse
%</samplechap1>
%\fi
% Some text for chapter 2:
%\iffalse
%<*samplechap2>
%\fi
%    \begin{macrocode}
\section{two}
more text in chapter two
%    \end{macrocode}

%\iffalse
%</samplechap2>
%\fi
%
% %%%%%%%%%%%%%%%%%%%%%%%%%%%%%%%%%%%%%%
% \paragraph{Part Include Files.}
%
% The include files are called |cdocspt3.tex| and |cdocspt4.tex|.
%
%\iffalse
%<*samplepart3|samplepart4>
%\fi

% Optional override for |\version| flag:
%    \begin{macrocode}
%%\providecommand{\version}{final}
%    \end{macrocode}

% Include the main document:
%    \begin{macrocode}
\input{childdoc.def}
\childdocby{cdocsamp}
%    \end{macrocode}

%\iffalse
%</samplepart3|samplepart4>
%\fi
%
%\iffalse
%<*samplepart3>
%\fi
% Some text for part 3:
%    \begin{macrocode}
some text in part three
%    \end{macrocode}

%\iffalse
%</samplepart3>
%\fi
% Some text for part 4:
%\iffalse
%<*samplepart4>
%\fi
%    \begin{macrocode}
more text in part four
%    \end{macrocode}

%\iffalse
%</samplepart4>
%\fi
%
% %%%%%%%%%%%%%%%%%%%%%%%%%%%%%%%%%%%%%%
% \paragraph{Forwarding for a Complete Draft.}
%
% The following forwarding file |cdocsdrf.tex|
% compiles the main document in draft mode:
%\iffalse
%<*sampledraft>
%\fi
%    \begin{macrocode}
\def\version{draft}
\input{childdoc.def}
\childdocforward{cdocsamp}
%    \end{macrocode}

%\iffalse
%</sampledraft>
%\fi
%
% %%%%%%%%%%%%%%%%%%%%%%%%%%%%%%%%%%%%%%
% \paragraph{Forwarding for Final Version of the Chapters.}
%
% The following forwarding files |cdocsfn1.tex| and |cdocsfn2.tex|
% (with identical content)
% compile the final versions of the child documents
% |cdocsch1.tex| and |cdocsch2.tex|, respectively:
%\iffalse
%<*samplefinal>
%\fi
%    \begin{macrocode}
\def\version{final}
\input{childdoc.def}
\childdocforwardprefix[cdocsamp]{cdocsfn}{cdocsch}
%    \end{macrocode}

%\iffalse
%</samplefinal>
%\fi
%
% %%%%%%%%%%%%%%%%%%%%%%%%%%%%%%%%%%%%%%
% \paragraph{Command Line Processing.}
%
% The following three command lines generate the output files
% |cdocscld|, |cdocscl1| and |cdocscl2|
% which should be identical to
% |cdocsdrf|, |cdocsch1| and |cdocsfn2|, respectively:
% \begin{center}
% \begin{tabular}{l}
% |latex -jobname cdocscld \|\\
% |  "\def\version{draft}\input{childdoc.def}\childdocforward{cdocsamp}"|\\
% |latex -jobname cdocscl1 \|\\
% |  "\input{childdoc.def}\childdocforward[cdocsamp]{cdocsch1}"|\\
% |latex -jobname cdocscl2 \|\\
% |  "\def\version{final}\input{childdoc.def}\childdocforward{cdocsch2}"|
% \end{tabular}
% \end{center}
% Note that the trailing backslash on each first line
% merely continues the input to the second line
% (for convenient cut ant paste).
% Furthermore, the command |latex| can be replaced by any
% of its alternative versions such as |pdflatex|.
%
% %%%%%%%%%%%%%%%%%%%%%%%%%%%%%%%%%%%%%%%%%%%%%%%%%%%%%%%%%%%%%%%%%%%%%%%%%%%%%%
% %%%%%%%%%%%%%%%%%%%%%%%%%%%%%%%%%%%%%%%%%%%%%%%%%%%%%%%%%%%%%%%%%%%%%%%%%%%%%%
% \section{Implementation}
%\iffalse
%<*package>
%\fi
%
% This section describes the definitions file |childdoc.def|.

% The definitions cannot be loaded using |\usepackage| or |\RequirePackage|
% which has a mechanism to prevent loading a style file more than once.
% When loading the definitions by means of |\input|
% multiple instances have to be prevented manually:
%\iffalse
%This code needs to be before the `\ProvidesFile' directive
%which is defined at the beginning of this file.
%Therefore it is also placed there and commented out here.
%</package>
%<*discard>
%\fi
%    \begin{macrocode}
\ifdefined\childdocmain\endinput\fi
%    \end{macrocode}
%\iffalse
%</discard>
%<*package>
%\fi
%
% \macro{\ifchilddoc}
% \macro{\ifchilddocmanual}
% The conditional |\ifchilddoc| tells whether a
% child (true) or main (false) document is being compiled.
% The conditional |\ifchilddocmanual| tells whether
% the |\includeonly| mechanism is used (false) or
% the selection of child files must be performed manually (true).
% The definitions initialise to false:
%    \begin{macrocode}
\newif\ifchilddoc
\newif\ifchilddocmanual
%    \end{macrocode}

% \macro{\childdocname}
% \macro{\childdocjob}
% The macro |\childdocname| stores the name of the main document
% to be compiled. The macro |\childdocjob| stores the name of
% the document on which the \LaTeX{} compiler was originally invoked.
% The content of |\jobname| cannot be compared
% to filenames specified in the source due to different catcodes.
% The following code rescans |\jobname|, stores the result
% in |\childdocname| and saves a copy in |\childdocjob|:
%    \begin{macrocode}
\edef\childdocname{\scantokens\expandafter{\jobname\noexpand}}
\let\childdocjob\childdocname
%    \end{macrocode}

% \macro{\childdocdisable}
% The macro |\childdocdisable| prevents the main file
% from being processed more than once.
% At this stage, the main document command |\childdocmain|
% is assumed to be called once again where it should do nothing.
% Any subsequent call to it should prevent
% a secondary processing of the main document
% It overwrites the forwarding commands
% |\childdocof| and |\childdocforward|
% with empty macros to prevent further inclusions of the main document:
%    \begin{macrocode}
\newcommand{\childdocdisable}
{
  \renewcommand{\childdocmain}[1]{\renewcommand{\childdocmain}[1]{\endinput}}
  \renewcommand{\childdocof}[1]{}
  \renewcommand{\childdocby}[2][]{}
  \renewcommand{\childdocforward}[2][]{}
  \renewcommand{\childdocdisable}{}
}
%    \end{macrocode}

% \macro{\childdocmain}
% The macro |\childdocmain| is to be called at the top of the main file
% with nothing or the main filename (without extension) as argument.
% First, it breaks loops.
% If the argument is not empty and does not match |\childdocname|
% (which is set by the first inclusion of |childdoc.def|),
% |\ifchilddoc| is set to true, |\includeonly| is applied to the child file
% and |\jobname| is set to the main file
% (for proper handling of |.aux| files):
%    \begin{macrocode}
\newcommand{\childdocmain}[1]
{
  \childdocdisable\childdocmain{}
  \if?#1?\else
    \begingroup
      \def\childdoctmp{#1}
      \ifx\childdoctmp\childdocname
        \def\childdoctmp{}
      \else
        \def\childdoctmp
        {
          \childdoctrue
          \includeonly{\childdocname}
          \def\childdocjob{#1}
          \def\jobname{#1}
        }
      \fi
      \expandafter
    \endgroup
    \childdoctmp
  \fi
}
%    \end{macrocode}

% \macro{\childdocof}
% The command |\childdocof| redirects
% compilation to the main file |#1|.
%    \begin{macrocode}
\newcommand{\childdocof}[1]
{
  \childdocdisable
  \childdoctrue
  \includeonly{\childdocname}
  \def\jobname{#1}
  \def\childdocjob{#1}
  \input{#1}
}
%    \end{macrocode}

% \macro{\childdocby}
% The command |\childdocby| ....
%    \begin{macrocode}
\newcommand{\childdocby}[2][]
{
  \childdocdisable
  \childdoctrue
  \childdocmanualtrue
  \if?#1?\else
    \def\jobname{#2}
  \fi
  \def\childdocjob{#2}
  \input{#2}
  \endinput
}
%    \end{macrocode}

% \macro{\childdocforward}
% The command |\childdocforward| redirects
% compilation to the main file or
% (if the optional argument is given) a child file.
% Parameters are set as if the main file
% or a child file starting with |\childdocof| was compiled.
% Then compilation is handed over to the main file:
%    \begin{macrocode}
\newcommand{\childdocforward}[2][]
{
  \begingroup
    \if?#1?
      \def\childdoctmp
      {
        \def\childdocname{#2}
        \def\childdocjob{#2}
        \def\jobname{#2}
        \input{#2}
        \endinput
      }
    \else
      \def\childdoctmp
      {
        \childdocdisable
        \def\childdocname{#2}
        \childdoctrue
        \includeonly{#2}
        \def\childdocjob{#1}
        \def\jobname{#1}
        \input{#1}
        \endinput
      }
    \fi
    \expandafter
  \endgroup
  \childdoctmp
}
%    \end{macrocode}

% \macro{\childdocforwardprefix}
% The command |\childdocforwardprefix| redirects
% compilation to the main or a child file by means of a pattern.
% The prefix |#1| in the current filename is replaced by |#2|
% and the suffix of the current filename is kept
% (it is assumed that the filename does not contain the substring `|~~~|'
% which is used as a delimiter).
% Compilation is handed over to the new file by |\childdocforward|:
%    \begin{macrocode}
\newcommand{\childdocforwardprefix}[3][]
{
  \begingroup
    \def\childdocextract #2##1~~~{\def\childdoctmp{\childdocforward[#1]{#3##1}}}
    \expandafter\childdocextract\childdocname~~~
    \expandafter
  \endgroup
  \childdoctmp
}
%    \end{macrocode}

% \macro{\childdoc}
% The deprecated macro |\childdoc| is a legacy version of |\childdocmain|:
%    \begin{macrocode}
\newcommand{\childdoc}{\childdocmain}
%    \end{macrocode}

% \macro{\childdocredirect}
% The deprecated macro |\childdocredirect| is a legacy version
% of |\childdocforward| and |\childdocforwardprefix|:
%    \begin{macrocode}
\newcommand{\childdocredirect}[2][]
{
  \begingroup
    \if?#1?
      \def\childdoctmp{\childdocforward{#2}}
    \else
      \def\childdoctmp{\childdocforwardprefix{#1}{#2}}
    \fi
    \expandafter
  \endgroup
  \childdoctmp
}
%    \end{macrocode}

%\iffalse
%</package>
%\fi
%
\endinput

\childdocmain{}
%    \end{macrocode}

% Optional override for |\version| flag:
%    \begin{macrocode}
%%\ifchilddoc\else\providecommand{\version}{draft}\fi
%    \end{macrocode}

% Define the default values for the |\version| flag
% (|final| for the main file and |draft| for childs):
%    \begin{macrocode}
\ifchilddoc
\providecommand{\version}{draft}
\else
\providecommand{\version}{final}
\fi
%    \end{macrocode}

% Load the standard document class:
%    \begin{macrocode}
\documentclass[12pt]{article}
%    \end{macrocode}

% Start the document body:
%    \begin{macrocode}
\begin{document}
%    \end{macrocode}

% Declare a title page.
% Print title, part of document being processed and version flag:
%    \begin{macrocode}
\addtocounter{page}{-1}
\begin{center}
{\LARGE\bfseries{}childdoc example\par}
\vspace{1cm}
\ifchilddoc
\ifchilddocmanual part\else chapter\fi:
`\childdocname' of `\childdocjob'\par
\else
main document: `\childdocjob'\par
\fi
version: \version\par
\end{center}
\newpage
%    \end{macrocode}

% Manually include selected file,
% otherwise process as usual:
%    \begin{macrocode}
\ifchilddocmanual
\section*{part `\childdocname'}
\input{\childdocname}
\else
%    \end{macrocode}

% Include the two chapters:
%    \begin{macrocode}
\include{cdocsch1}
\include{cdocsch2}
%    \end{macrocode}

% Include the two parts unless only chapters should be displayed:
%    \begin{macrocode}
\ifchilddoc\else
\section{part three}
\input{cdocspt3}
\section{part four}
\input{cdocspt4}
\fi
%    \end{macrocode}

% Process as usual until here:
%    \begin{macrocode}
\fi
%    \end{macrocode}

% End of document body:
%    \begin{macrocode}
\end{document}
%    \end{macrocode}
%\iffalse
%</samplemain>
%\fi
%
% %%%%%%%%%%%%%%%%%%%%%%%%%%%%%%%%%%%%%%
% \paragraph{Chapter Include Files.}
%
% The include files are called |cdocsch1.tex| and |cdocsch2.tex|.
%
%\iffalse
%<*samplechap1|samplechap2>
%\fi

% Optional override for |\version| flag:
%    \begin{macrocode}
%%\providecommand{\version}{final}
%    \end{macrocode}

% Include the main document:
%    \begin{macrocode}
% \iffalse
%
% childdoc.dtx Copyright (C) 2017-2018 Niklas Beisert
%
% This work may be distributed and/or modified under the
% conditions of the LaTeX Project Public License, either version 1.3
% of this license or (at your option) any later version.
% The latest version of this license is in
%   http://www.latex-project.org/lppl.txt
% and version 1.3 or later is part of all distributions of LaTeX
% version 2005/12/01 or later.
%
% This work has the LPPL maintenance status `maintained'.
%
% The Current Maintainer of this work is Niklas Beisert.
%
% This work consists of the files childdoc.dtx and childdoc.ins
% and the derived files childdoc.def and cdocsamp.tex with
% cdocsch1.tex, cdocsch2.tex, cdocsdrf.tex, cdocsfn1.tex, cdocsfn2.tex.
%
%<package>\ifdefined\childdocmain\endinput\fi
%<package>\ProvidesFile{childdoc.def}[2018/12/30 v2.0 child document driver]
%<samplemain>\ProvidesFile{cdocsamp.tex}[2018/12/30 v2.0 sample for childdoc]
%<*driver>
%\ProvidesFile{childdoc.drv}[2018/12/30 v2.0 childdoc reference manual file]
\PassOptionsToClass{10pt,a4paper}{article}
\documentclass{ltxdoc}

\usepackage[margin=35mm]{geometry}
\usepackage{hyperref}
\usepackage{hyperxmp}
\usepackage[usenames]{color}

\hypersetup{colorlinks=true}
\hypersetup{pdfstartview=FitH}
\hypersetup{pdfpagemode=UseNone}
\hypersetup{pdfsource={}}
\hypersetup{pdflang={en-UK}}
\hypersetup{pdfcopyright={Copyright 2017-2018 Niklas Beisert.
  This work may be distributed and/or modified under the
  conditions of the LaTeX Project Public License, either version 1.3
  of this license or (at your option) any later version.}}
\hypersetup{pdflicenseurl={http://www.latex-project.org/lppl.txt}}
\hypersetup{pdfcontactaddress={ETH Zurich, ITP, HIT K,
  Wolfgang-Pauli-Strasse 27}}
\hypersetup{pdfcontactpostcode={8093}}
\hypersetup{pdfcontactcity={Zurich}}
\hypersetup{pdfcontactcountry={Switzerland}}
\hypersetup{pdfcontactemail={nbeisert@itp.phys.ethz.ch}}
\hypersetup{pdfcontacturl={http://people.phys.ethz.ch/\xmptilde nbeisert/}}

\newcommand{\secref}[1]{\hyperref[#1]{section \ref*{#1}}}

\parskip1ex
\parindent0pt
\let\olditemize\itemize
\def\itemize{\olditemize\parskip0pt}

\begin{document}

\title{The \textsf{childdoc} Package}
\hypersetup{pdftitle={The childdoc Package}}
\author{Niklas Beisert\\[2ex]
  Institut f\"ur Theoretische Physik\\
  Eidgen\"ossische Technische Hochschule Z\"urich\\
  Wolfgang-Pauli-Strasse 27, 8093 Z\"urich, Switzerland\\[1ex]
  \href{mailto:nbeisert@itp.phys.ethz.ch}
  {\texttt{nbeisert@itp.phys.ethz.ch}}}
\hypersetup{pdfauthor={Niklas Beisert}}
\hypersetup{pdfsubject={Manual for the LaTeX2e Package childdoc}}
\date{30 December 2018, \textsf{v2.0}}
\maketitle

\begin{abstract}\noindent
\textsf{childdoc} is a \LaTeXe{} package
that enables the direct compilation
of document sections included by |\include|
to individual files.
\end{abstract}

\begingroup
\parskip0ex
\tableofcontents
\endgroup

%%%%%%%%%%%%%%%%%%%%%%%%%%%%%%%%%%%%%%%%%%%%%%%%%%%%%%%%%%%%%%%%%%%%%%%%%%%%%%%%
%%%%%%%%%%%%%%%%%%%%%%%%%%%%%%%%%%%%%%%%%%%%%%%%%%%%%%%%%%%%%%%%%%%%%%%%%%%%%%%%
\section{Introduction}

\LaTeX{} provides a mechanism to structure a large document (such as a book)
into a main file and several child files (containing the chapters)
using the |\include| command.
This mechanism is beneficial for documents
which span hundreds of pages in order to
make the source file(s) more manageable.
Moreover, compilation can be restricted to
selected child files by means of the |\includeonly| command.
The latter feature can be used to reduce the compilation time while editing
(this was significantly more useful in the earlier days of \LaTeX{})
or to generate a smaller document which is easier to navigate.
Another application of |\includeonly| is to generate
documents consisting of selected parts of the complete document.

However, there are a few drawbacks of the plain |\include| mechanism:
\begin{itemize}
\item
The child files cannot be compiled on their own,
they can only be compiled via the main file.
A naive editing environment
(such as a text editor with an option
to have the current file processed by \LaTeX)
may require one to switch to the main file before compiling;
attempting to compile the child file produces errors.
\item
The main file must be modified (each time)
to adjust the |\includeonly| command
to the present needs. This easily leaves the main file in a messy state.
\item
The generated document will always carry the filename
of the main document. This is inconvenient if
several child files are to be compiled and
to be kept for distribution.
\end{itemize}

The present package provides a simple interface
to make child files individually compilable by \LaTeX{}.
Compiling a child file then has the same effect as compiling
the main file with an |\includeonly| command
to select the appropriate child.
Moreover the generated document will carry the name of the child
rather than the main file.
This resolves all three above issues.

This feature is meant to make the editing of books,
thesis documents and lecture notes somewhat more convenient.
However, the package can also be used efficiently for
composing a series of documents (such as exercise sheets)
which are typically distributed individually.
It then assists the author in generating the individual documents
(potentially in different versions)
as well as a document containing the collected series.
Another application is in developing style files
or other kinds of included material
where compilation of the style file could redirect
to a sample or test file.

%%%%%%%%%%%%%%%%%%%%%%%%%%%%%%%%%%%%%%%%%%%%%%%%%%%%%%%%%%%%%%%%%%%%%%%%%%%%%%%%
%%%%%%%%%%%%%%%%%%%%%%%%%%%%%%%%%%%%%%%%%%%%%%%%%%%%%%%%%%%%%%%%%%%%%%%%%%%%%%%%
\section{Usage}

First of all, the package \textsf{childdoc} is \emph{not} a standard
\LaTeXe{} |.sty| style file! Therefore it needs to be invoked in
a non-standard way.

%%%%%%%%%%%%%%%%%%%%%%%%%%%%%%%%%%%%%%%%%%%%%%%%%%%%%%%%%%%%%%%%%%%%%%%%%%%%%%%%
\subsection{Included Files}
\label{sec:include}

%%%%%%%%%%%%%%%%%%%%%%%%%%%%%%%%%%%%%%%%
\DescribeMacro{\childdocmain}
To use the package, add the commands
\begin{center}
\begin{tabular}{l}
|\input{childdoc.def}|\\
|\childdocmain{}|\\
\end{tabular}
\end{center}
at the very top of the main \LaTeX{} file,
in particular \emph{before} the |\documentclass| statement!
The argument of |\childdocmain| should be left empty
(but it must be present).

%%%%%%%%%%%%%%%%%%%%%%%%%%%%%%%%%%%%%%%%
\DescribeMacro{\childdocof}
Furthermore, add the commands
\begin{center}
\begin{tabular}{l}
|\input{childdoc.def}|\\
|\childdocof{|\textit{main}|}|\\
\end{tabular}
\end{center}
at the top of every child file \textit{child}
which is included by |\include{|\textit{child}|}|
from within the main file
(or at least for those files to be compiled individually).
The argument \textit{main} must be the filename of the main file.

There are a couple of
considerations in setting up the main and child documents:

%%%%%%%%%%%%%%%%%%%%%%%%%%%%%%%%%%%%%%%%
\paragraph{Restrictions.}

Please note the following restrictions:
\begin{itemize}
\item
|\childdocmain| must be called with one argument \textit{main}
to ensure compatibility with earlier version of the package.
It must either be empty (|\childdocmain{}|)
or precisely match the filename of the main file in which it is specified.
See \secref{sec:detection} for further information.
\item
The filename \textit{main} must be specified without the |.tex| extension.
\item
The filename \textit{main} is case sensitive
(even in case-insensitive file systems)
due to internal string comparison.
\item
The argument \textit{main} should be fully expanded, it cannot be a macro.
\item
Subdirectories and special characters should be avoided in filenames.
\item
The command |\childdocmain{|\textit{main}|}| must be followed by a whitespace.
It should not be followed immediately by another command
or by a comment mark `|%|'.
This is because the \TeX{} parser reads the token immediately following
the argument of |\childdocmain| and puts it
at the beginning of every child section;
however, a white\-space is ignored.
\end{itemize}

%%%%%%%%%%%%%%%%%%%%%%%%%%%%%%%%%%%%%%%%
\paragraph{Content of Main File.}

It is advisable to place all content in the child files included by |\include|.
Any output contained in the main file will appear in all child documents
unless suppressed manually;
it cannot be suppressed automatically by the |\includeonly| directive
and thus should normally be avoided.
A method to include some content in the main file
by means of conditional processing is described in \secref{sec:conditional}.

%%%%%%%%%%%%%%%%%%%%%%%%%%%%%%%%%%%%%%%%
\paragraph{Page Numbering.}

When only a part of the document is compiled,
the appropriate numbering of pages
(as well as other status parameters)
is determined from the |.aux| files.
The latter contain information from previous passes.
However this information needs to propagate through
all intermediate child documents.
Therefore the page numbering in child documents may well
be inconsistent until the complete document is compiled at least once.

A useful (if unconventional) way to always ensure a consistent
page numbering is to restart the numbering in each child document
and denote the pages by `\textit{child}|.|\textit{page}'
where \textit{child} represents the chapter/section number of the child file.
This can be achieved by the command
|\numberwithin{page}{|\textit{child}|}|
of the \textsf{amsmath} package
where \textit{child} can be |chapter| or |section|
depending on the chosen structuring.
Alternatively, one can modify the macro |\thepage| appropriately
and reset the counter |page| at the start of each child file.

%%%%%%%%%%%%%%%%%%%%%%%%%%%%%%%%%%%%%%%%%%%%%%%%%%%%%%%%%%%%%%%%%%%%%%%%%%%%%%%%
\subsection{Conditional Processing}
\label{sec:conditional}

The package provides a mechanism to compile different versions
of a document. To customise the versions further some conditional processing
can come in handy to distinguish which version is being compiled.
The package provides two macros to describe the compilation context:

%%%%%%%%%%%%%%%%%%%%%%%%%%%%%%%%%%%%%%%%
\DescribeMacro{\ifchilddoc}
The conditional |\ifchilddoc| distinguishes between the compilation of
child documents and the main document:
%
\begin{center}
|\ifchilddoc |\textit{child-code}| |[|\||else |\textit{main-code}]| \||fi|
\end{center}

%%%%%%%%%%%%%%%%%%%%%%%%%%%%%%%%%%%%%%%%
\DescribeMacro{\childdocname}
\DescribeMacro{\childdocjob}
The macro |\childdocname| contains the filename (without extension)
of the main or child file being processed.
Note that |\childdocjob| will always contain the name of the main file.

%%%%%%%%%%%%%%%%%%%%%%%%%%%%%%%%%%%%%%%%
\paragraph{Title Page.}

Conditional processing can be used to include a title or banner page
in the main document when proper precautions are taken.
Importantly, the code in the main file should ensure that the page counter
(as well as other status parameters which are stored in the |.aux| files)
takes the same value after the conditional processing.
Otherwise the page numbers may take divergent values
depending on which part is compiled.

For example, a title page could be declared by:
%
\begin{center}
\begin{tabular}{l}
|\ifchilddoc\||else|\\
|\addtocounter{page}{-1}|\\
\textit{code for title page}\\
|\newpage|\\
|\||fi|
\end{tabular}
\end{center}
%
A banner page for the child documents can be generated by:
%
\begin{center}
\begin{tabular}{l}
|\ifchilddoc|\\
|\addtocounter{page}{-1}|\\
\textit{code for banner page}\\
|\newpage|\\
|\||fi|
\end{tabular}
\end{center}
%
Here one could write a message such as:
\begin{center}
|This is the part \childdocname{} of \childdocjob{}.|
\end{center}

%%%%%%%%%%%%%%%%%%%%%%%%%%%%%%%%%%%%%%%%%%%%%%%%%%%%%%%%%%%%%%%%%%%%%%%%%%%%%%%%
\subsection{Flags}
\label{sec:flags}

The package makes it easy to generate different versions
of the main or child documents.
To this end compilation flags can be defined
and assigned different default values.
They will be particularly useful in conjunction
with the forwarding mechanism described in \secref{sec:forward}.

For example, it may be useful to have a flag |\version|
which can be set to |draft| or |final|.
The document source will contain some conditional code
depending on the value of |\version|.
Suppose further, the flag should default to |final| for the main file
and to |draft| for child files
which is a natural assignment for editing the document.
This is achieved by placing the following code
in the preamble of the main document
(below the |\childdocmain| directive):
%
\begin{center}
\begin{tabular}{l}
|\ifchilddoc|\\
|\providecommand{\version}{draft}|\\
|\||else|\\
|\providecommand{\version}{final}|\\
|\||fi|
\end{tabular}
\end{center}
%
The definition by |\providecommand| makes sure
that previous definitions are not overwritten.
Further statements |\providecommand{\version}{...}|
can thus be added before the above code to override it.

For the main file, one might add a line
(between |\childdocmain| and the above block)
%
\begin{center}
|%\ifchilddoc\||else\providecommand{\version}{draft}\||fi|
\end{center}
%
which can be uncommented to produce a draft version.
Likewise one can add a line to the very top of a child file
(above the |\childdocof{|\textit{main}|}| directive)
%
\begin{center}
|%\providecommand{\version}{final}|
\end{center}
%
which can be uncommented to produce the final version of this child document.

%%%%%%%%%%%%%%%%%%%%%%%%%%%%%%%%%%%%%%%%%%%%%%%%%%%%%%%%%%%%%%%%%%%%%%%%%%%%%%%%
\subsection{Forwarding}
\label{sec:forward}

Different versions of the main or child documents
using compilation flags as described in \secref{sec:flags}
can be (permanently) stored in different files
for convenient compilation, viewing and distribution.
To this end, the package defines a command
to pass on compilation to a different file:

%%%%%%%%%%%%%%%%%%%%%%%%%%%%%%%%%%%%%%%%
\DescribeMacro{\childdocforward}
The command |\childdocforward| redirects processing to
another source file:
%
\begin{center}
\begin{tabular}{l}
|\input{childdoc.def}|\\
|\childdocforward[|\textit{main}|]{|\textit{dest}|}|\\
\end{tabular}
\end{center}
%
The argument \textit{dest} is the destination file
(without extension).
It should be the main file or one of the child files.
Note that further \textsf{childdoc} directives
such as |\childdocof| and |\childdocforward|
in the indicated file will be processed in this form.
The optional argument \textit{main}
passes on directly to the main file \textit{main}
while pretending to compile the child \textit{dest}.
This form behaves as if \textit{dest}
issues |\childdocof{|\textit{main}|}| right away,
and no further \textsf{childdoc} directives will be processed.

%%%%%%%%%%%%%%%%%%%%%%%%%%%%%%%%%%%%%%%%
\DescribeMacro{\...prefix}
In the alternative form |\childdocforwardprefix|,
%
\begin{center}
\begin{tabular}{l}
|\input{childdoc.def}|\\
|\childdocforwardprefix[|\textit{main}|]{|\textit{prefix}|}{|\textit{dest}|}|
\end{tabular}
\end{center}
%
the destination file is determined by a pattern
depending on the current file:
To make this work, the current file must be called
`{\textit{prefix}\hspace{0.2em}\textit{suffix}}'
with \textit{prefix} matching precisely the argument.
Processing is then passed on to the file
`{\textit{dest}\hspace{0.2em}\textit{suffix}}'.
Surely, the same effect is achieved by
directly specifying the
argument `{\textit{dest}\hspace{0.2em}\textit{suffix}}'
in the first form.
However, that requires to set up a different file
for each child. With the alternative form of the command
all these files can have exactly the same content
which simplifies setting them up and maintaining them.

For example, the following file |draft.tex|
with a compilation flag |\version| as described in \secref{sec:flags}
compiles the main document as a draft:
%
\begin{center}
\begin{tabular}{l}
|\def\version{draft}|\\
|\input{childdoc.def}|\\
|\childdocforward{|\textit{main}|}|
\end{tabular}
\end{center}
%
Likewise, the following files |final|\textit{nn}|.tex|
compile the final version of the child document
|child|\textit{nn}|.tex|:
%
\begin{center}
\begin{tabular}{l}
|\def\version{final}|\\
|\input{childdoc.def}|\\
|\childdocforwardprefix{final}{child}|
\end{tabular}
\end{center}
%

Note that when several versions of a main file and/or of each child file
are to be generated, it may be convenient to set up a |Makefile| or
shell script to automatise the process.

%%%%%%%%%%%%%%%%%%%%%%%%%%%%%%%%%%%%%%%%%%%%%%%%%%%%%%%%%%%%%%%%%%%%%%%%%%%%%%%%
\subsection{Command Line Processing}
\label{sec:commandline}

The effect of redirection files can also be achieved by invoking
the \LaTeX{} compiler with a more elaborate command line.
Most conveniently this should be done as part
of a shell script or a |Makefile|.

When using \textsf{childdoc} in the main file, the following
command lines effectively perform a redirection
(note that depending on the shell being used,
backslashes may have to be doubled: `|\|' $\to$ `|\\|'):
%
\begin{center}
|... -jobname "|\textit{target}|" |\\|"|[\textit{flags}]%
|\input{childdoc.def}\childdocforward[|\textit{main}|]{|\textit{dest}|}"|
\end{center}
%
Here \textit{target} is the name of the output file,
\textit{main} is the name of the main file
and \textit{dest} is the name of the main or child file to be processed
(all filenames without extensions).
The optional argument \textit{main} can be omitted
if \textit{main} matches \textit{dest}.
Optionally, compilation \textit{flags} can be defined via |\def| commands.
This command line makes the \TeX{} engine believe
it is compiling the file \textit{target}
whose content is specified as the latter parameter.
The provided code then forwards the processing to
\textit{main} or \textit{dest} as described in \secref{sec:forward}.

%%%%%%%%%%%%%%%%%%%%%%%%%%%%%%%%%%%%%%%%%%%%%%%%%%%%%%%%%%%%%%%%%%%%%%%%%%%%%%%%
\subsection{Include by Input}
\label{sec:input}

Including child documents by |\include| has some restrictions by design.
Most notably, the content of a child document always occupies
its own set of pages; pages cannot be shared between child documents.
Usually, this behaviour makes perfect sense
because each child document contain an essential part of the document.
However, in some situations it may be desirable to compose
a document from a collection of parts
without having mandatory page breaks between then.
For this case, the package
provides a mechanism to include parts
by |\input| which can also be processed individually.
However, by construction this mechanism
requires manual handling of the content to be output.

%%%%%%%%%%%%%%%%%%%%%%%%%%%%%%%%%%%%%%%%
\DescribeMacro{\ifchilddocmanual}
The main file should be prepared as usual, see \secref{sec:include}.
However, the document body must make a distinction
between processing of an individual part and of the main document, e.g.:
%
\begin{center}
\begin{tabular}{l}
|\ifchilddocmanual|\\
|\input{\childdocname}|\\
|\||else|\\
\textit{document body with }|\input{|\textit{part}|}|\\
|\||fi|
\end{tabular}
\end{center}
%
The conditional |\ifchilddocmanual| is true whenever
a part to be included by |\input| is being compiled,
and the name of the part is stored in |\childdocname|.

%%%%%%%%%%%%%%%%%%%%%%%%%%%%%%%%%%%%%%%%
\DescribeMacro{\childdocby}
Each part to be included by |\input| should start with:
%
\begin{center}
\begin{tabular}{l}
|\input{childdoc.def}|\\
|\childdocby{|\textit{main}|}|\\
\end{tabular}
\end{center}
%
The directive |\childdocby| is similar to |\childdocof|
described in \secref{sec:include},
but the subsequent selection of content must be done manually.
To that end, both |\ifchilddoc| and |\ifchilddocmanual|
will be true upon processing of a part,
and the name of the part is stored in |\childdocname|.
Note that |\jobname| will be set to the filename of the current part
so that each part receives an individual |.aux| file
that does not interfere with the |.aux| file(s) of the main document.
This behaviour can be altered by the alternative form
|\childdocby[*]{|\textit{main}|}| (with a non-empty optional argument)
which uses the |.aux| file of the main document
by setting |\jobname| to \textit{main}.

%%%%%%%%%%%%%%%%%%%%%%%%%%%%%%%%%%%%%%%%%%%%%%%%%%%%%%%%%%%%%%%%%%%%%%%%%%%%%%%%
\subsection{Driver Development}
\label{sec:driver}

The \textsf{childdoc} mechanism can also be use for the development
of definition files such as \LaTeX{} styles or classes.
This case differs from the above setup with multiple parts
included by |\include| in that no |\includeonly| should be invoked.
This can be achieved by starting the include file
(before |\ProvidesPackage|) with:
%
\begin{center}
\begin{tabular}{l}
|\input{childdoc.def}|\\
|\childdocforward{|\textit{main}|}|\\
\end{tabular}
\end{center}
%
or alternatively with:
%
\begin{center}
\begin{tabular}{l}
|\input{childdoc.def}|\\
|\childdocby{|\textit{main}|}|\\
\end{tabular}
\end{center}
%
Both forms have slightly different effects as described above.
The main file is prepared as usual, see \secref{sec:include}.

%%%%%%%%%%%%%%%%%%%%%%%%%%%%%%%%%%%%%%%%%%%%%%%%%%%%%%%%%%%%%%%%%%%%%%%%%%%%%%%%
\subsection{Legacy Detection}
\label{sec:detection}

The directive |\childdocmain| in the main file can detect
whether the complete document or merely a child is to be compiled
even without using the directive |\childdocof|.
This method is deprecated because it is less robust
and there is no compelling reason to use it;
it is merely provided for backward compatibility
and it may be removed in future versions.

If the detection mechanism is to be used,
it is mandatory to correctly specify
the filename of the main file as the argument of |\childdocmain|:
%
\begin{center}
\begin{tabular}{l}
|\input{childdoc.def}|\\
|\childdocmain{|\textit{main}|}|\\
\end{tabular}
\end{center}
%
If |\jobname| does not match the argument \textit{main} of |\childdocmain|,
it is assumed that |\jobname| points to the child file to be compiled.
When using |\childdocmain| with the main file specified as argument,
it suffices to start a child file
with just |\input{|\textit{main}|}|
without loading of the package and using |\childdocof|.
If instead all processing is done
with the appropriate \textsf{childdoc} directives,
the argument of \textit{main} of |\childdocmain| can be empty.

An alternative version of the command line processing described
in \secref{sec:commandline} using the detection mechanism reads:
%
\begin{center}
|... -jobname "|\textit{target}|" "|[\textit{flags}]%
[|\def\jobname{|\textit{dest}|}|]|\input{|\textit{main}|}"|
\end{center}

%%%%%%%%%%%%%%%%%%%%%%%%%%%%%%%%%%%%%%%%%%%%%%%%%%%%%%%%%%%%%%%%%%%%%%%%%%%%%%%%
\subsection{Manual Code}
\label{sec:manual}

In case one cannot be certain whether the definitions file |childdoc.def|
is installed on the target \TeX{} distribution
and one prefers not to ship it,
it is conceivable to paste a few relevant commands into the sources.

To that end, drop all statements |\input{childdoc.def}|
and perform the replacements as outlined below.
Instead of |\childdocmain{|\textit{main}|}| add the following code
to the top of the main file:
%
\begin{center}
\begin{tabular}{l}
|\||ifdefined\childdocname\endinput\||fi\newif\ifchilddoc|\\
|\edef\childdocname{\scantokens\expandafter{\jobname\noexpand}}|\\
|\def\childdocmain{|\textit{main}|}\||ifx\childdocmain\childdocname\||else|\\
|\childdoctrue\includeonly{\childdocname}\let\jobname\childdocmain\||fi|\\
\end{tabular}
\end{center}
%
Instead of |\childdocof{|\textit{main}|}| just include the main file
at the top of each child file:
%
\begin{center}
|\input{|\textit{main}|}|
\end{center}
%
A simple redirection |\childdocforward{|\textit{dest}|}| is achieved by:
%
\begin{center}
|\def\jobname{|\textit{dest}|}\input{\jobname}|
\end{center}
%
The redirection with prefix
|\childdocforwardprefix[|\textit{prefix}|]{|\textit{dest}|}|
is accomplished by:
%
\begin{center}
\begin{tabular}{l}
|{\edef\jobname{\scantokens\expandafter{\jobname\noexpand}}|\\
|\def\redirectjob |\textit{prefix}|#1~~~{\gdef\jobname{|\textit{dest}|#1}}|\\
|\expandafter\redirectjob\jobname~~~}\input{\jobname}|
\end{tabular}
\end{center}

In an alternative approach,
child documents can be compiled by a specific command line
without additional code or specific definitions:
%
\begin{center}
|... -jobname "|\textit{target}|" "|[\textit{flags}]%
|\includeonly{|\textit{dest}|}\input{|\textit{main}|}"|
\end{center}
%

%%%%%%%%%%%%%%%%%%%%%%%%%%%%%%%%%%%%%%%%%%%%%%%%%%%%%%%%%%%%%%%%%%%%%%%%%%%%%%%%
%%%%%%%%%%%%%%%%%%%%%%%%%%%%%%%%%%%%%%%%%%%%%%%%%%%%%%%%%%%%%%%%%%%%%%%%%%%%%%%%
\section{Information}

%%%%%%%%%%%%%%%%%%%%%%%%%%%%%%%%%%%%%%%%%%%%%%%%%%%%%%%%%%%%%%%%%%%%%%%%%%%%%%%%
\subsection{Copyright}

Copyright \copyright{} 2017--2018 Niklas Beisert

This work may be distributed and/or modified under the
conditions of the \LaTeX{} Project Public License, either version 1.3
of this license or (at your option) any later version.
The latest version of this license is in
  \url{http://www.latex-project.org/lppl.txt}
and version 1.3 or later is part of all distributions of \LaTeX{}
version 2005/12/01 or later.

This work has the LPPL maintenance status `maintained'.

The Current Maintainer of this work is Niklas Beisert.

This work consists of the files |README.txt|, |childdoc.ins| and |childdoc.dtx|
as well as the derived files |childdoc.def|, |cdocsamp.tex|
with |cdocsch1.tex|, |cdocsch2.tex|, |cdocspt3.tex|, |cdocspt4.tex|,
|cdocsdrf.tex|, |cdocsfn1.tex|, |cdocsfn2.tex|
as well as |childdoc.pdf|.

%%%%%%%%%%%%%%%%%%%%%%%%%%%%%%%%%%%%%%%%%%%%%%%%%%%%%%%%%%%%%%%%%%%%%%%%%%%%%%%%
\subsection{Files and Installation}

The package consists of the files:
%
\begin{center}
\begin{tabular}{ll}
    |README.txt|   & readme file \\
    |childdoc.ins| & installation file \\
    |childdoc.dtx| & source file \\
    |childdoc.def| & definition file \\
    |cdocsamp.tex| & sample main file \\
    |cdocsch1.tex| & sample include file \\
    |cdocsch2.tex| & sample include file \\
    |cdocspt3.tex| & sample part file \\
    |cdocspt4.tex| & sample part file \\
    |cdocsdrf.tex| & sample redirection file \\
    |cdocsfn1.tex| & sample redirection file \\
    |cdocsfn2.tex| & sample redirection file \\
    |childdoc.pdf| & manual
\end{tabular}
\end{center}
%
The distribution consists of the files
|README.txt|, |childdoc.ins| and |childdoc.dtx|.
%
\begin{itemize}
\item
Run (pdf)\LaTeX{} on |childdoc.dtx|
to compile the manual |childdoc.pdf| (this file).
\item
Run \LaTeX{} on |childdoc.ins| to create the definitions file |childdoc.def|
and the sample |cdocsamp.tex| with include files
|cdocsch1.tex|, |cdocsch2.tex|, |cdocspt3.tex|, |cdocspt4.tex|,
|cdocsdrf.tex|, |cdocsfn1.tex|, |cdocsfn2.tex|.
Then copy the file |childdoc.def| to an appropriate directory of your \LaTeX{}
distribution, e.g.\ \textit{texmf-root}|/tex/latex/childdoc|.
\end{itemize}

%%%%%%%%%%%%%%%%%%%%%%%%%%%%%%%%%%%%%%%%%%%%%%%%%%%%%%%%%%%%%%%%%%%%%%%%%%%%%%%%
\subsection{Related CTAN Packages}

There are several other packages which offer a similar functionality:
%
\begin{itemize}
\item
The packages
\href{http://ctan.org/pkg/docmute}{\textsf{docmute}},
\href{http://ctan.org/pkg/includex}{\textsf{includex}} and
\href{http://ctan.org/pkg/standalone}{\textsf{standalone}}
provide commands to include only the document body of
a child file thus allowing both files to be compiled individually.
\item
The packages \href{http://ctan.org/pkg/subdocs}{\textsf{subdocs}}
and \href{http://ctan.org/pkg/subfiles}{\textsf{subfiles}}
provide structures in which the main and child documents can be
encapsulated and allowing them to be compiled individually.
The inclusion mechanism is different from the conventional |\include|.
\item
The package \href{http://ctan.org/pkg/combine}{\textsf{combine}}
is an elaborate solution to combine several documents into one.
\end{itemize}
%
See also the CTAN topic \href{http://ctan.org/topic/subdocs}{\textsf{subdocs}}
for further related packages.
The present package differs from the above solutions in that
a document structure constructed with the conventional |\include| mechanism
just needs two extra commands at the top of every file
such that all constituent files can be compiled individually.

%%%%%%%%%%%%%%%%%%%%%%%%%%%%%%%%%%%%%%%%%%%%%%%%%%%%%%%%%%%%%%%%%%%%%%%%%%%%%%%%
%\subsection{Feature Suggestions}
%
%The following is a list of features which may be useful for future
%versions of this package:
%%
%\begin{itemize}
%\item
%\ldots
%\end{itemize}

%%%%%%%%%%%%%%%%%%%%%%%%%%%%%%%%%%%%%%%%%%%%%%%%%%%%%%%%%%%%%%%%%%%%%%%%%%%%%%%%
\subsection{Revision History}

%%%%%%%%%%%%%%%%%%%%%%%%%%%%%%%%%%%%%%%%
\paragraph{v2.0:} 2018/12/30

\begin{itemize}
\item
immediate forward processing
\item
added |\childdocby| mechanism
\item
manual restructured
\end{itemize}

%%%%%%%%%%%%%%%%%%%%%%%%%%%%%%%%%%%%%%%%
\paragraph{v1.6:} 2018/01/17

\begin{itemize}
\item
application for development of include files
\item
corrections to manual
\end{itemize}

%%%%%%%%%%%%%%%%%%%%%%%%%%%%%%%%%%%%%%%%
\paragraph{v1.5:} 2017/05/21

\begin{itemize}
\item
more complete structuring introduced
\item
|\childdocof| introduced
\item
|\childdoc| renamed to |\childdocmain|
\item
|\childredirect| renamed to |\childdocforward| and |\childdocforwardprefix|
and functionality expanded
\end{itemize}

%%%%%%%%%%%%%%%%%%%%%%%%%%%%%%%%%%%%%%%%
\paragraph{v1.0:} 2017/04/27

\begin{itemize}
\item
manual and install package
\item
first version published on CTAN
\end{itemize}

%%%%%%%%%%%%%%%%%%%%%%%%%%%%%%%%%%%%%%%%
\paragraph{v0.6:} 2017/04/26

\begin{itemize}
\item
redirection mechanism added
\end{itemize}

%%%%%%%%%%%%%%%%%%%%%%%%%%%%%%%%%%%%%%%%
\paragraph{v0.5:} 2017/04/26

\begin{itemize}
\item
functionality in definition file
\end{itemize}


%%%%%%%%%%%%%%%%%%%%%%%%%%%%%%%%%%%%%%%%%%%%%%%%%%%%%%%%%%%%%%%%%%%%%%%%%%%%%%%%
%%%%%%%%%%%%%%%%%%%%%%%%%%%%%%%%%%%%%%%%%%%%%%%%%%%%%%%%%%%%%%%%%%%%%%%%%%%%%%%%
%%%%%%%%%%%%%%%%%%%%%%%%%%%%%%%%%%%%%%%%%%%%%%%%%%%%%%%%%%%%%%%%%%%%%%%%%%%%%%%%
\appendix

\settowidth\MacroIndent{\rmfamily\scriptsize 000\ }

 \DocInput{childdoc.dtx}

\end{document}
%</driver>
% \fi
%
% %%%%%%%%%%%%%%%%%%%%%%%%%%%%%%%%%%%%%%%%%%%%%%%%%%%%%%%%%%%%%%%%%%%%%%%%%%%%%%
% %%%%%%%%%%%%%%%%%%%%%%%%%%%%%%%%%%%%%%%%%%%%%%%%%%%%%%%%%%%%%%%%%%%%%%%%%%%%%%
% \section{Sample}
%\iffalse
%<*samplemain>
%\fi
%
% The following presents a sample document
% with two chapters, two parts, a title page,
% a compile flag as well as three forwarding files to set the flag.
% It consists of eight |.tex| files:
% \begin{center}
% \begin{tabular}{ll}
% |cdocsamp.tex|&main file\\
% |cdocsch1.tex|&include file for chapter 1\\
% |cdocsch2.tex|&include file for chapter 2\\
% |cdocspt3.tex|&include file for part 3\\
% |cdocspt4.tex|&include file for part 4\\
% |cdocsdrf.tex|&forwarding file for main file in draft mode\\
% |cdocsfi1.tex|&forwarding file for final version of chapter 1\\
% |cdocsfi2.tex|&forwarding file for final version of chapter 2\\
% \end{tabular}
% \end{center}
% Each of the eight files can be compiled directly by the \LaTeX{} compiler.
%
% %%%%%%%%%%%%%%%%%%%%%%%%%%%%%%%%%%%%%%
% \paragraph{Main File.}
%
% The main file is called |cdocsamp.tex|.
%
% Load the \textsf{childdoc} definitions and
% declare the filename for the main document:
%    \begin{macrocode}
\input{childdoc.def}
\childdocmain{}
%    \end{macrocode}

% Optional override for |\version| flag:
%    \begin{macrocode}
%%\ifchilddoc\else\providecommand{\version}{draft}\fi
%    \end{macrocode}

% Define the default values for the |\version| flag
% (|final| for the main file and |draft| for childs):
%    \begin{macrocode}
\ifchilddoc
\providecommand{\version}{draft}
\else
\providecommand{\version}{final}
\fi
%    \end{macrocode}

% Load the standard document class:
%    \begin{macrocode}
\documentclass[12pt]{article}
%    \end{macrocode}

% Start the document body:
%    \begin{macrocode}
\begin{document}
%    \end{macrocode}

% Declare a title page.
% Print title, part of document being processed and version flag:
%    \begin{macrocode}
\addtocounter{page}{-1}
\begin{center}
{\LARGE\bfseries{}childdoc example\par}
\vspace{1cm}
\ifchilddoc
\ifchilddocmanual part\else chapter\fi:
`\childdocname' of `\childdocjob'\par
\else
main document: `\childdocjob'\par
\fi
version: \version\par
\end{center}
\newpage
%    \end{macrocode}

% Manually include selected file,
% otherwise process as usual:
%    \begin{macrocode}
\ifchilddocmanual
\section*{part `\childdocname'}
\input{\childdocname}
\else
%    \end{macrocode}

% Include the two chapters:
%    \begin{macrocode}
\include{cdocsch1}
\include{cdocsch2}
%    \end{macrocode}

% Include the two parts unless only chapters should be displayed:
%    \begin{macrocode}
\ifchilddoc\else
\section{part three}
\input{cdocspt3}
\section{part four}
\input{cdocspt4}
\fi
%    \end{macrocode}

% Process as usual until here:
%    \begin{macrocode}
\fi
%    \end{macrocode}

% End of document body:
%    \begin{macrocode}
\end{document}
%    \end{macrocode}
%\iffalse
%</samplemain>
%\fi
%
% %%%%%%%%%%%%%%%%%%%%%%%%%%%%%%%%%%%%%%
% \paragraph{Chapter Include Files.}
%
% The include files are called |cdocsch1.tex| and |cdocsch2.tex|.
%
%\iffalse
%<*samplechap1|samplechap2>
%\fi

% Optional override for |\version| flag:
%    \begin{macrocode}
%%\providecommand{\version}{final}
%    \end{macrocode}

% Include the main document:
%    \begin{macrocode}
\input{childdoc.def}
\childdocof{cdocsamp}
%    \end{macrocode}

%\iffalse
%</samplechap1|samplechap2>
%\fi
%
%\iffalse
%<*samplechap1>
%\fi
% Some text for chapter 1:
%    \begin{macrocode}
\section{one}
some text in chapter one
%    \end{macrocode}

%\iffalse
%</samplechap1>
%\fi
% Some text for chapter 2:
%\iffalse
%<*samplechap2>
%\fi
%    \begin{macrocode}
\section{two}
more text in chapter two
%    \end{macrocode}

%\iffalse
%</samplechap2>
%\fi
%
% %%%%%%%%%%%%%%%%%%%%%%%%%%%%%%%%%%%%%%
% \paragraph{Part Include Files.}
%
% The include files are called |cdocspt3.tex| and |cdocspt4.tex|.
%
%\iffalse
%<*samplepart3|samplepart4>
%\fi

% Optional override for |\version| flag:
%    \begin{macrocode}
%%\providecommand{\version}{final}
%    \end{macrocode}

% Include the main document:
%    \begin{macrocode}
\input{childdoc.def}
\childdocby{cdocsamp}
%    \end{macrocode}

%\iffalse
%</samplepart3|samplepart4>
%\fi
%
%\iffalse
%<*samplepart3>
%\fi
% Some text for part 3:
%    \begin{macrocode}
some text in part three
%    \end{macrocode}

%\iffalse
%</samplepart3>
%\fi
% Some text for part 4:
%\iffalse
%<*samplepart4>
%\fi
%    \begin{macrocode}
more text in part four
%    \end{macrocode}

%\iffalse
%</samplepart4>
%\fi
%
% %%%%%%%%%%%%%%%%%%%%%%%%%%%%%%%%%%%%%%
% \paragraph{Forwarding for a Complete Draft.}
%
% The following forwarding file |cdocsdrf.tex|
% compiles the main document in draft mode:
%\iffalse
%<*sampledraft>
%\fi
%    \begin{macrocode}
\def\version{draft}
\input{childdoc.def}
\childdocforward{cdocsamp}
%    \end{macrocode}

%\iffalse
%</sampledraft>
%\fi
%
% %%%%%%%%%%%%%%%%%%%%%%%%%%%%%%%%%%%%%%
% \paragraph{Forwarding for Final Version of the Chapters.}
%
% The following forwarding files |cdocsfn1.tex| and |cdocsfn2.tex|
% (with identical content)
% compile the final versions of the child documents
% |cdocsch1.tex| and |cdocsch2.tex|, respectively:
%\iffalse
%<*samplefinal>
%\fi
%    \begin{macrocode}
\def\version{final}
\input{childdoc.def}
\childdocforwardprefix[cdocsamp]{cdocsfn}{cdocsch}
%    \end{macrocode}

%\iffalse
%</samplefinal>
%\fi
%
% %%%%%%%%%%%%%%%%%%%%%%%%%%%%%%%%%%%%%%
% \paragraph{Command Line Processing.}
%
% The following three command lines generate the output files
% |cdocscld|, |cdocscl1| and |cdocscl2|
% which should be identical to
% |cdocsdrf|, |cdocsch1| and |cdocsfn2|, respectively:
% \begin{center}
% \begin{tabular}{l}
% |latex -jobname cdocscld \|\\
% |  "\def\version{draft}\input{childdoc.def}\childdocforward{cdocsamp}"|\\
% |latex -jobname cdocscl1 \|\\
% |  "\input{childdoc.def}\childdocforward[cdocsamp]{cdocsch1}"|\\
% |latex -jobname cdocscl2 \|\\
% |  "\def\version{final}\input{childdoc.def}\childdocforward{cdocsch2}"|
% \end{tabular}
% \end{center}
% Note that the trailing backslash on each first line
% merely continues the input to the second line
% (for convenient cut ant paste).
% Furthermore, the command |latex| can be replaced by any
% of its alternative versions such as |pdflatex|.
%
% %%%%%%%%%%%%%%%%%%%%%%%%%%%%%%%%%%%%%%%%%%%%%%%%%%%%%%%%%%%%%%%%%%%%%%%%%%%%%%
% %%%%%%%%%%%%%%%%%%%%%%%%%%%%%%%%%%%%%%%%%%%%%%%%%%%%%%%%%%%%%%%%%%%%%%%%%%%%%%
% \section{Implementation}
%\iffalse
%<*package>
%\fi
%
% This section describes the definitions file |childdoc.def|.

% The definitions cannot be loaded using |\usepackage| or |\RequirePackage|
% which has a mechanism to prevent loading a style file more than once.
% When loading the definitions by means of |\input|
% multiple instances have to be prevented manually:
%\iffalse
%This code needs to be before the `\ProvidesFile' directive
%which is defined at the beginning of this file.
%Therefore it is also placed there and commented out here.
%</package>
%<*discard>
%\fi
%    \begin{macrocode}
\ifdefined\childdocmain\endinput\fi
%    \end{macrocode}
%\iffalse
%</discard>
%<*package>
%\fi
%
% \macro{\ifchilddoc}
% \macro{\ifchilddocmanual}
% The conditional |\ifchilddoc| tells whether a
% child (true) or main (false) document is being compiled.
% The conditional |\ifchilddocmanual| tells whether
% the |\includeonly| mechanism is used (false) or
% the selection of child files must be performed manually (true).
% The definitions initialise to false:
%    \begin{macrocode}
\newif\ifchilddoc
\newif\ifchilddocmanual
%    \end{macrocode}

% \macro{\childdocname}
% \macro{\childdocjob}
% The macro |\childdocname| stores the name of the main document
% to be compiled. The macro |\childdocjob| stores the name of
% the document on which the \LaTeX{} compiler was originally invoked.
% The content of |\jobname| cannot be compared
% to filenames specified in the source due to different catcodes.
% The following code rescans |\jobname|, stores the result
% in |\childdocname| and saves a copy in |\childdocjob|:
%    \begin{macrocode}
\edef\childdocname{\scantokens\expandafter{\jobname\noexpand}}
\let\childdocjob\childdocname
%    \end{macrocode}

% \macro{\childdocdisable}
% The macro |\childdocdisable| prevents the main file
% from being processed more than once.
% At this stage, the main document command |\childdocmain|
% is assumed to be called once again where it should do nothing.
% Any subsequent call to it should prevent
% a secondary processing of the main document
% It overwrites the forwarding commands
% |\childdocof| and |\childdocforward|
% with empty macros to prevent further inclusions of the main document:
%    \begin{macrocode}
\newcommand{\childdocdisable}
{
  \renewcommand{\childdocmain}[1]{\renewcommand{\childdocmain}[1]{\endinput}}
  \renewcommand{\childdocof}[1]{}
  \renewcommand{\childdocby}[2][]{}
  \renewcommand{\childdocforward}[2][]{}
  \renewcommand{\childdocdisable}{}
}
%    \end{macrocode}

% \macro{\childdocmain}
% The macro |\childdocmain| is to be called at the top of the main file
% with nothing or the main filename (without extension) as argument.
% First, it breaks loops.
% If the argument is not empty and does not match |\childdocname|
% (which is set by the first inclusion of |childdoc.def|),
% |\ifchilddoc| is set to true, |\includeonly| is applied to the child file
% and |\jobname| is set to the main file
% (for proper handling of |.aux| files):
%    \begin{macrocode}
\newcommand{\childdocmain}[1]
{
  \childdocdisable\childdocmain{}
  \if?#1?\else
    \begingroup
      \def\childdoctmp{#1}
      \ifx\childdoctmp\childdocname
        \def\childdoctmp{}
      \else
        \def\childdoctmp
        {
          \childdoctrue
          \includeonly{\childdocname}
          \def\childdocjob{#1}
          \def\jobname{#1}
        }
      \fi
      \expandafter
    \endgroup
    \childdoctmp
  \fi
}
%    \end{macrocode}

% \macro{\childdocof}
% The command |\childdocof| redirects
% compilation to the main file |#1|.
%    \begin{macrocode}
\newcommand{\childdocof}[1]
{
  \childdocdisable
  \childdoctrue
  \includeonly{\childdocname}
  \def\jobname{#1}
  \def\childdocjob{#1}
  \input{#1}
}
%    \end{macrocode}

% \macro{\childdocby}
% The command |\childdocby| ....
%    \begin{macrocode}
\newcommand{\childdocby}[2][]
{
  \childdocdisable
  \childdoctrue
  \childdocmanualtrue
  \if?#1?\else
    \def\jobname{#2}
  \fi
  \def\childdocjob{#2}
  \input{#2}
  \endinput
}
%    \end{macrocode}

% \macro{\childdocforward}
% The command |\childdocforward| redirects
% compilation to the main file or
% (if the optional argument is given) a child file.
% Parameters are set as if the main file
% or a child file starting with |\childdocof| was compiled.
% Then compilation is handed over to the main file:
%    \begin{macrocode}
\newcommand{\childdocforward}[2][]
{
  \begingroup
    \if?#1?
      \def\childdoctmp
      {
        \def\childdocname{#2}
        \def\childdocjob{#2}
        \def\jobname{#2}
        \input{#2}
        \endinput
      }
    \else
      \def\childdoctmp
      {
        \childdocdisable
        \def\childdocname{#2}
        \childdoctrue
        \includeonly{#2}
        \def\childdocjob{#1}
        \def\jobname{#1}
        \input{#1}
        \endinput
      }
    \fi
    \expandafter
  \endgroup
  \childdoctmp
}
%    \end{macrocode}

% \macro{\childdocforwardprefix}
% The command |\childdocforwardprefix| redirects
% compilation to the main or a child file by means of a pattern.
% The prefix |#1| in the current filename is replaced by |#2|
% and the suffix of the current filename is kept
% (it is assumed that the filename does not contain the substring `|~~~|'
% which is used as a delimiter).
% Compilation is handed over to the new file by |\childdocforward|:
%    \begin{macrocode}
\newcommand{\childdocforwardprefix}[3][]
{
  \begingroup
    \def\childdocextract #2##1~~~{\def\childdoctmp{\childdocforward[#1]{#3##1}}}
    \expandafter\childdocextract\childdocname~~~
    \expandafter
  \endgroup
  \childdoctmp
}
%    \end{macrocode}

% \macro{\childdoc}
% The deprecated macro |\childdoc| is a legacy version of |\childdocmain|:
%    \begin{macrocode}
\newcommand{\childdoc}{\childdocmain}
%    \end{macrocode}

% \macro{\childdocredirect}
% The deprecated macro |\childdocredirect| is a legacy version
% of |\childdocforward| and |\childdocforwardprefix|:
%    \begin{macrocode}
\newcommand{\childdocredirect}[2][]
{
  \begingroup
    \if?#1?
      \def\childdoctmp{\childdocforward{#2}}
    \else
      \def\childdoctmp{\childdocforwardprefix{#1}{#2}}
    \fi
    \expandafter
  \endgroup
  \childdoctmp
}
%    \end{macrocode}

%\iffalse
%</package>
%\fi
%
\endinput

\childdocof{cdocsamp}
%    \end{macrocode}

%\iffalse
%</samplechap1|samplechap2>
%\fi
%
%\iffalse
%<*samplechap1>
%\fi
% Some text for chapter 1:
%    \begin{macrocode}
\section{one}
some text in chapter one
%    \end{macrocode}

%\iffalse
%</samplechap1>
%\fi
% Some text for chapter 2:
%\iffalse
%<*samplechap2>
%\fi
%    \begin{macrocode}
\section{two}
more text in chapter two
%    \end{macrocode}

%\iffalse
%</samplechap2>
%\fi
%
% %%%%%%%%%%%%%%%%%%%%%%%%%%%%%%%%%%%%%%
% \paragraph{Part Include Files.}
%
% The include files are called |cdocspt3.tex| and |cdocspt4.tex|.
%
%\iffalse
%<*samplepart3|samplepart4>
%\fi

% Optional override for |\version| flag:
%    \begin{macrocode}
%%\providecommand{\version}{final}
%    \end{macrocode}

% Include the main document:
%    \begin{macrocode}
% \iffalse
%
% childdoc.dtx Copyright (C) 2017-2018 Niklas Beisert
%
% This work may be distributed and/or modified under the
% conditions of the LaTeX Project Public License, either version 1.3
% of this license or (at your option) any later version.
% The latest version of this license is in
%   http://www.latex-project.org/lppl.txt
% and version 1.3 or later is part of all distributions of LaTeX
% version 2005/12/01 or later.
%
% This work has the LPPL maintenance status `maintained'.
%
% The Current Maintainer of this work is Niklas Beisert.
%
% This work consists of the files childdoc.dtx and childdoc.ins
% and the derived files childdoc.def and cdocsamp.tex with
% cdocsch1.tex, cdocsch2.tex, cdocsdrf.tex, cdocsfn1.tex, cdocsfn2.tex.
%
%<package>\ifdefined\childdocmain\endinput\fi
%<package>\ProvidesFile{childdoc.def}[2018/12/30 v2.0 child document driver]
%<samplemain>\ProvidesFile{cdocsamp.tex}[2018/12/30 v2.0 sample for childdoc]
%<*driver>
%\ProvidesFile{childdoc.drv}[2018/12/30 v2.0 childdoc reference manual file]
\PassOptionsToClass{10pt,a4paper}{article}
\documentclass{ltxdoc}

\usepackage[margin=35mm]{geometry}
\usepackage{hyperref}
\usepackage{hyperxmp}
\usepackage[usenames]{color}

\hypersetup{colorlinks=true}
\hypersetup{pdfstartview=FitH}
\hypersetup{pdfpagemode=UseNone}
\hypersetup{pdfsource={}}
\hypersetup{pdflang={en-UK}}
\hypersetup{pdfcopyright={Copyright 2017-2018 Niklas Beisert.
  This work may be distributed and/or modified under the
  conditions of the LaTeX Project Public License, either version 1.3
  of this license or (at your option) any later version.}}
\hypersetup{pdflicenseurl={http://www.latex-project.org/lppl.txt}}
\hypersetup{pdfcontactaddress={ETH Zurich, ITP, HIT K,
  Wolfgang-Pauli-Strasse 27}}
\hypersetup{pdfcontactpostcode={8093}}
\hypersetup{pdfcontactcity={Zurich}}
\hypersetup{pdfcontactcountry={Switzerland}}
\hypersetup{pdfcontactemail={nbeisert@itp.phys.ethz.ch}}
\hypersetup{pdfcontacturl={http://people.phys.ethz.ch/\xmptilde nbeisert/}}

\newcommand{\secref}[1]{\hyperref[#1]{section \ref*{#1}}}

\parskip1ex
\parindent0pt
\let\olditemize\itemize
\def\itemize{\olditemize\parskip0pt}

\begin{document}

\title{The \textsf{childdoc} Package}
\hypersetup{pdftitle={The childdoc Package}}
\author{Niklas Beisert\\[2ex]
  Institut f\"ur Theoretische Physik\\
  Eidgen\"ossische Technische Hochschule Z\"urich\\
  Wolfgang-Pauli-Strasse 27, 8093 Z\"urich, Switzerland\\[1ex]
  \href{mailto:nbeisert@itp.phys.ethz.ch}
  {\texttt{nbeisert@itp.phys.ethz.ch}}}
\hypersetup{pdfauthor={Niklas Beisert}}
\hypersetup{pdfsubject={Manual for the LaTeX2e Package childdoc}}
\date{30 December 2018, \textsf{v2.0}}
\maketitle

\begin{abstract}\noindent
\textsf{childdoc} is a \LaTeXe{} package
that enables the direct compilation
of document sections included by |\include|
to individual files.
\end{abstract}

\begingroup
\parskip0ex
\tableofcontents
\endgroup

%%%%%%%%%%%%%%%%%%%%%%%%%%%%%%%%%%%%%%%%%%%%%%%%%%%%%%%%%%%%%%%%%%%%%%%%%%%%%%%%
%%%%%%%%%%%%%%%%%%%%%%%%%%%%%%%%%%%%%%%%%%%%%%%%%%%%%%%%%%%%%%%%%%%%%%%%%%%%%%%%
\section{Introduction}

\LaTeX{} provides a mechanism to structure a large document (such as a book)
into a main file and several child files (containing the chapters)
using the |\include| command.
This mechanism is beneficial for documents
which span hundreds of pages in order to
make the source file(s) more manageable.
Moreover, compilation can be restricted to
selected child files by means of the |\includeonly| command.
The latter feature can be used to reduce the compilation time while editing
(this was significantly more useful in the earlier days of \LaTeX{})
or to generate a smaller document which is easier to navigate.
Another application of |\includeonly| is to generate
documents consisting of selected parts of the complete document.

However, there are a few drawbacks of the plain |\include| mechanism:
\begin{itemize}
\item
The child files cannot be compiled on their own,
they can only be compiled via the main file.
A naive editing environment
(such as a text editor with an option
to have the current file processed by \LaTeX)
may require one to switch to the main file before compiling;
attempting to compile the child file produces errors.
\item
The main file must be modified (each time)
to adjust the |\includeonly| command
to the present needs. This easily leaves the main file in a messy state.
\item
The generated document will always carry the filename
of the main document. This is inconvenient if
several child files are to be compiled and
to be kept for distribution.
\end{itemize}

The present package provides a simple interface
to make child files individually compilable by \LaTeX{}.
Compiling a child file then has the same effect as compiling
the main file with an |\includeonly| command
to select the appropriate child.
Moreover the generated document will carry the name of the child
rather than the main file.
This resolves all three above issues.

This feature is meant to make the editing of books,
thesis documents and lecture notes somewhat more convenient.
However, the package can also be used efficiently for
composing a series of documents (such as exercise sheets)
which are typically distributed individually.
It then assists the author in generating the individual documents
(potentially in different versions)
as well as a document containing the collected series.
Another application is in developing style files
or other kinds of included material
where compilation of the style file could redirect
to a sample or test file.

%%%%%%%%%%%%%%%%%%%%%%%%%%%%%%%%%%%%%%%%%%%%%%%%%%%%%%%%%%%%%%%%%%%%%%%%%%%%%%%%
%%%%%%%%%%%%%%%%%%%%%%%%%%%%%%%%%%%%%%%%%%%%%%%%%%%%%%%%%%%%%%%%%%%%%%%%%%%%%%%%
\section{Usage}

First of all, the package \textsf{childdoc} is \emph{not} a standard
\LaTeXe{} |.sty| style file! Therefore it needs to be invoked in
a non-standard way.

%%%%%%%%%%%%%%%%%%%%%%%%%%%%%%%%%%%%%%%%%%%%%%%%%%%%%%%%%%%%%%%%%%%%%%%%%%%%%%%%
\subsection{Included Files}
\label{sec:include}

%%%%%%%%%%%%%%%%%%%%%%%%%%%%%%%%%%%%%%%%
\DescribeMacro{\childdocmain}
To use the package, add the commands
\begin{center}
\begin{tabular}{l}
|\input{childdoc.def}|\\
|\childdocmain{}|\\
\end{tabular}
\end{center}
at the very top of the main \LaTeX{} file,
in particular \emph{before} the |\documentclass| statement!
The argument of |\childdocmain| should be left empty
(but it must be present).

%%%%%%%%%%%%%%%%%%%%%%%%%%%%%%%%%%%%%%%%
\DescribeMacro{\childdocof}
Furthermore, add the commands
\begin{center}
\begin{tabular}{l}
|\input{childdoc.def}|\\
|\childdocof{|\textit{main}|}|\\
\end{tabular}
\end{center}
at the top of every child file \textit{child}
which is included by |\include{|\textit{child}|}|
from within the main file
(or at least for those files to be compiled individually).
The argument \textit{main} must be the filename of the main file.

There are a couple of
considerations in setting up the main and child documents:

%%%%%%%%%%%%%%%%%%%%%%%%%%%%%%%%%%%%%%%%
\paragraph{Restrictions.}

Please note the following restrictions:
\begin{itemize}
\item
|\childdocmain| must be called with one argument \textit{main}
to ensure compatibility with earlier version of the package.
It must either be empty (|\childdocmain{}|)
or precisely match the filename of the main file in which it is specified.
See \secref{sec:detection} for further information.
\item
The filename \textit{main} must be specified without the |.tex| extension.
\item
The filename \textit{main} is case sensitive
(even in case-insensitive file systems)
due to internal string comparison.
\item
The argument \textit{main} should be fully expanded, it cannot be a macro.
\item
Subdirectories and special characters should be avoided in filenames.
\item
The command |\childdocmain{|\textit{main}|}| must be followed by a whitespace.
It should not be followed immediately by another command
or by a comment mark `|%|'.
This is because the \TeX{} parser reads the token immediately following
the argument of |\childdocmain| and puts it
at the beginning of every child section;
however, a white\-space is ignored.
\end{itemize}

%%%%%%%%%%%%%%%%%%%%%%%%%%%%%%%%%%%%%%%%
\paragraph{Content of Main File.}

It is advisable to place all content in the child files included by |\include|.
Any output contained in the main file will appear in all child documents
unless suppressed manually;
it cannot be suppressed automatically by the |\includeonly| directive
and thus should normally be avoided.
A method to include some content in the main file
by means of conditional processing is described in \secref{sec:conditional}.

%%%%%%%%%%%%%%%%%%%%%%%%%%%%%%%%%%%%%%%%
\paragraph{Page Numbering.}

When only a part of the document is compiled,
the appropriate numbering of pages
(as well as other status parameters)
is determined from the |.aux| files.
The latter contain information from previous passes.
However this information needs to propagate through
all intermediate child documents.
Therefore the page numbering in child documents may well
be inconsistent until the complete document is compiled at least once.

A useful (if unconventional) way to always ensure a consistent
page numbering is to restart the numbering in each child document
and denote the pages by `\textit{child}|.|\textit{page}'
where \textit{child} represents the chapter/section number of the child file.
This can be achieved by the command
|\numberwithin{page}{|\textit{child}|}|
of the \textsf{amsmath} package
where \textit{child} can be |chapter| or |section|
depending on the chosen structuring.
Alternatively, one can modify the macro |\thepage| appropriately
and reset the counter |page| at the start of each child file.

%%%%%%%%%%%%%%%%%%%%%%%%%%%%%%%%%%%%%%%%%%%%%%%%%%%%%%%%%%%%%%%%%%%%%%%%%%%%%%%%
\subsection{Conditional Processing}
\label{sec:conditional}

The package provides a mechanism to compile different versions
of a document. To customise the versions further some conditional processing
can come in handy to distinguish which version is being compiled.
The package provides two macros to describe the compilation context:

%%%%%%%%%%%%%%%%%%%%%%%%%%%%%%%%%%%%%%%%
\DescribeMacro{\ifchilddoc}
The conditional |\ifchilddoc| distinguishes between the compilation of
child documents and the main document:
%
\begin{center}
|\ifchilddoc |\textit{child-code}| |[|\||else |\textit{main-code}]| \||fi|
\end{center}

%%%%%%%%%%%%%%%%%%%%%%%%%%%%%%%%%%%%%%%%
\DescribeMacro{\childdocname}
\DescribeMacro{\childdocjob}
The macro |\childdocname| contains the filename (without extension)
of the main or child file being processed.
Note that |\childdocjob| will always contain the name of the main file.

%%%%%%%%%%%%%%%%%%%%%%%%%%%%%%%%%%%%%%%%
\paragraph{Title Page.}

Conditional processing can be used to include a title or banner page
in the main document when proper precautions are taken.
Importantly, the code in the main file should ensure that the page counter
(as well as other status parameters which are stored in the |.aux| files)
takes the same value after the conditional processing.
Otherwise the page numbers may take divergent values
depending on which part is compiled.

For example, a title page could be declared by:
%
\begin{center}
\begin{tabular}{l}
|\ifchilddoc\||else|\\
|\addtocounter{page}{-1}|\\
\textit{code for title page}\\
|\newpage|\\
|\||fi|
\end{tabular}
\end{center}
%
A banner page for the child documents can be generated by:
%
\begin{center}
\begin{tabular}{l}
|\ifchilddoc|\\
|\addtocounter{page}{-1}|\\
\textit{code for banner page}\\
|\newpage|\\
|\||fi|
\end{tabular}
\end{center}
%
Here one could write a message such as:
\begin{center}
|This is the part \childdocname{} of \childdocjob{}.|
\end{center}

%%%%%%%%%%%%%%%%%%%%%%%%%%%%%%%%%%%%%%%%%%%%%%%%%%%%%%%%%%%%%%%%%%%%%%%%%%%%%%%%
\subsection{Flags}
\label{sec:flags}

The package makes it easy to generate different versions
of the main or child documents.
To this end compilation flags can be defined
and assigned different default values.
They will be particularly useful in conjunction
with the forwarding mechanism described in \secref{sec:forward}.

For example, it may be useful to have a flag |\version|
which can be set to |draft| or |final|.
The document source will contain some conditional code
depending on the value of |\version|.
Suppose further, the flag should default to |final| for the main file
and to |draft| for child files
which is a natural assignment for editing the document.
This is achieved by placing the following code
in the preamble of the main document
(below the |\childdocmain| directive):
%
\begin{center}
\begin{tabular}{l}
|\ifchilddoc|\\
|\providecommand{\version}{draft}|\\
|\||else|\\
|\providecommand{\version}{final}|\\
|\||fi|
\end{tabular}
\end{center}
%
The definition by |\providecommand| makes sure
that previous definitions are not overwritten.
Further statements |\providecommand{\version}{...}|
can thus be added before the above code to override it.

For the main file, one might add a line
(between |\childdocmain| and the above block)
%
\begin{center}
|%\ifchilddoc\||else\providecommand{\version}{draft}\||fi|
\end{center}
%
which can be uncommented to produce a draft version.
Likewise one can add a line to the very top of a child file
(above the |\childdocof{|\textit{main}|}| directive)
%
\begin{center}
|%\providecommand{\version}{final}|
\end{center}
%
which can be uncommented to produce the final version of this child document.

%%%%%%%%%%%%%%%%%%%%%%%%%%%%%%%%%%%%%%%%%%%%%%%%%%%%%%%%%%%%%%%%%%%%%%%%%%%%%%%%
\subsection{Forwarding}
\label{sec:forward}

Different versions of the main or child documents
using compilation flags as described in \secref{sec:flags}
can be (permanently) stored in different files
for convenient compilation, viewing and distribution.
To this end, the package defines a command
to pass on compilation to a different file:

%%%%%%%%%%%%%%%%%%%%%%%%%%%%%%%%%%%%%%%%
\DescribeMacro{\childdocforward}
The command |\childdocforward| redirects processing to
another source file:
%
\begin{center}
\begin{tabular}{l}
|\input{childdoc.def}|\\
|\childdocforward[|\textit{main}|]{|\textit{dest}|}|\\
\end{tabular}
\end{center}
%
The argument \textit{dest} is the destination file
(without extension).
It should be the main file or one of the child files.
Note that further \textsf{childdoc} directives
such as |\childdocof| and |\childdocforward|
in the indicated file will be processed in this form.
The optional argument \textit{main}
passes on directly to the main file \textit{main}
while pretending to compile the child \textit{dest}.
This form behaves as if \textit{dest}
issues |\childdocof{|\textit{main}|}| right away,
and no further \textsf{childdoc} directives will be processed.

%%%%%%%%%%%%%%%%%%%%%%%%%%%%%%%%%%%%%%%%
\DescribeMacro{\...prefix}
In the alternative form |\childdocforwardprefix|,
%
\begin{center}
\begin{tabular}{l}
|\input{childdoc.def}|\\
|\childdocforwardprefix[|\textit{main}|]{|\textit{prefix}|}{|\textit{dest}|}|
\end{tabular}
\end{center}
%
the destination file is determined by a pattern
depending on the current file:
To make this work, the current file must be called
`{\textit{prefix}\hspace{0.2em}\textit{suffix}}'
with \textit{prefix} matching precisely the argument.
Processing is then passed on to the file
`{\textit{dest}\hspace{0.2em}\textit{suffix}}'.
Surely, the same effect is achieved by
directly specifying the
argument `{\textit{dest}\hspace{0.2em}\textit{suffix}}'
in the first form.
However, that requires to set up a different file
for each child. With the alternative form of the command
all these files can have exactly the same content
which simplifies setting them up and maintaining them.

For example, the following file |draft.tex|
with a compilation flag |\version| as described in \secref{sec:flags}
compiles the main document as a draft:
%
\begin{center}
\begin{tabular}{l}
|\def\version{draft}|\\
|\input{childdoc.def}|\\
|\childdocforward{|\textit{main}|}|
\end{tabular}
\end{center}
%
Likewise, the following files |final|\textit{nn}|.tex|
compile the final version of the child document
|child|\textit{nn}|.tex|:
%
\begin{center}
\begin{tabular}{l}
|\def\version{final}|\\
|\input{childdoc.def}|\\
|\childdocforwardprefix{final}{child}|
\end{tabular}
\end{center}
%

Note that when several versions of a main file and/or of each child file
are to be generated, it may be convenient to set up a |Makefile| or
shell script to automatise the process.

%%%%%%%%%%%%%%%%%%%%%%%%%%%%%%%%%%%%%%%%%%%%%%%%%%%%%%%%%%%%%%%%%%%%%%%%%%%%%%%%
\subsection{Command Line Processing}
\label{sec:commandline}

The effect of redirection files can also be achieved by invoking
the \LaTeX{} compiler with a more elaborate command line.
Most conveniently this should be done as part
of a shell script or a |Makefile|.

When using \textsf{childdoc} in the main file, the following
command lines effectively perform a redirection
(note that depending on the shell being used,
backslashes may have to be doubled: `|\|' $\to$ `|\\|'):
%
\begin{center}
|... -jobname "|\textit{target}|" |\\|"|[\textit{flags}]%
|\input{childdoc.def}\childdocforward[|\textit{main}|]{|\textit{dest}|}"|
\end{center}
%
Here \textit{target} is the name of the output file,
\textit{main} is the name of the main file
and \textit{dest} is the name of the main or child file to be processed
(all filenames without extensions).
The optional argument \textit{main} can be omitted
if \textit{main} matches \textit{dest}.
Optionally, compilation \textit{flags} can be defined via |\def| commands.
This command line makes the \TeX{} engine believe
it is compiling the file \textit{target}
whose content is specified as the latter parameter.
The provided code then forwards the processing to
\textit{main} or \textit{dest} as described in \secref{sec:forward}.

%%%%%%%%%%%%%%%%%%%%%%%%%%%%%%%%%%%%%%%%%%%%%%%%%%%%%%%%%%%%%%%%%%%%%%%%%%%%%%%%
\subsection{Include by Input}
\label{sec:input}

Including child documents by |\include| has some restrictions by design.
Most notably, the content of a child document always occupies
its own set of pages; pages cannot be shared between child documents.
Usually, this behaviour makes perfect sense
because each child document contain an essential part of the document.
However, in some situations it may be desirable to compose
a document from a collection of parts
without having mandatory page breaks between then.
For this case, the package
provides a mechanism to include parts
by |\input| which can also be processed individually.
However, by construction this mechanism
requires manual handling of the content to be output.

%%%%%%%%%%%%%%%%%%%%%%%%%%%%%%%%%%%%%%%%
\DescribeMacro{\ifchilddocmanual}
The main file should be prepared as usual, see \secref{sec:include}.
However, the document body must make a distinction
between processing of an individual part and of the main document, e.g.:
%
\begin{center}
\begin{tabular}{l}
|\ifchilddocmanual|\\
|\input{\childdocname}|\\
|\||else|\\
\textit{document body with }|\input{|\textit{part}|}|\\
|\||fi|
\end{tabular}
\end{center}
%
The conditional |\ifchilddocmanual| is true whenever
a part to be included by |\input| is being compiled,
and the name of the part is stored in |\childdocname|.

%%%%%%%%%%%%%%%%%%%%%%%%%%%%%%%%%%%%%%%%
\DescribeMacro{\childdocby}
Each part to be included by |\input| should start with:
%
\begin{center}
\begin{tabular}{l}
|\input{childdoc.def}|\\
|\childdocby{|\textit{main}|}|\\
\end{tabular}
\end{center}
%
The directive |\childdocby| is similar to |\childdocof|
described in \secref{sec:include},
but the subsequent selection of content must be done manually.
To that end, both |\ifchilddoc| and |\ifchilddocmanual|
will be true upon processing of a part,
and the name of the part is stored in |\childdocname|.
Note that |\jobname| will be set to the filename of the current part
so that each part receives an individual |.aux| file
that does not interfere with the |.aux| file(s) of the main document.
This behaviour can be altered by the alternative form
|\childdocby[*]{|\textit{main}|}| (with a non-empty optional argument)
which uses the |.aux| file of the main document
by setting |\jobname| to \textit{main}.

%%%%%%%%%%%%%%%%%%%%%%%%%%%%%%%%%%%%%%%%%%%%%%%%%%%%%%%%%%%%%%%%%%%%%%%%%%%%%%%%
\subsection{Driver Development}
\label{sec:driver}

The \textsf{childdoc} mechanism can also be use for the development
of definition files such as \LaTeX{} styles or classes.
This case differs from the above setup with multiple parts
included by |\include| in that no |\includeonly| should be invoked.
This can be achieved by starting the include file
(before |\ProvidesPackage|) with:
%
\begin{center}
\begin{tabular}{l}
|\input{childdoc.def}|\\
|\childdocforward{|\textit{main}|}|\\
\end{tabular}
\end{center}
%
or alternatively with:
%
\begin{center}
\begin{tabular}{l}
|\input{childdoc.def}|\\
|\childdocby{|\textit{main}|}|\\
\end{tabular}
\end{center}
%
Both forms have slightly different effects as described above.
The main file is prepared as usual, see \secref{sec:include}.

%%%%%%%%%%%%%%%%%%%%%%%%%%%%%%%%%%%%%%%%%%%%%%%%%%%%%%%%%%%%%%%%%%%%%%%%%%%%%%%%
\subsection{Legacy Detection}
\label{sec:detection}

The directive |\childdocmain| in the main file can detect
whether the complete document or merely a child is to be compiled
even without using the directive |\childdocof|.
This method is deprecated because it is less robust
and there is no compelling reason to use it;
it is merely provided for backward compatibility
and it may be removed in future versions.

If the detection mechanism is to be used,
it is mandatory to correctly specify
the filename of the main file as the argument of |\childdocmain|:
%
\begin{center}
\begin{tabular}{l}
|\input{childdoc.def}|\\
|\childdocmain{|\textit{main}|}|\\
\end{tabular}
\end{center}
%
If |\jobname| does not match the argument \textit{main} of |\childdocmain|,
it is assumed that |\jobname| points to the child file to be compiled.
When using |\childdocmain| with the main file specified as argument,
it suffices to start a child file
with just |\input{|\textit{main}|}|
without loading of the package and using |\childdocof|.
If instead all processing is done
with the appropriate \textsf{childdoc} directives,
the argument of \textit{main} of |\childdocmain| can be empty.

An alternative version of the command line processing described
in \secref{sec:commandline} using the detection mechanism reads:
%
\begin{center}
|... -jobname "|\textit{target}|" "|[\textit{flags}]%
[|\def\jobname{|\textit{dest}|}|]|\input{|\textit{main}|}"|
\end{center}

%%%%%%%%%%%%%%%%%%%%%%%%%%%%%%%%%%%%%%%%%%%%%%%%%%%%%%%%%%%%%%%%%%%%%%%%%%%%%%%%
\subsection{Manual Code}
\label{sec:manual}

In case one cannot be certain whether the definitions file |childdoc.def|
is installed on the target \TeX{} distribution
and one prefers not to ship it,
it is conceivable to paste a few relevant commands into the sources.

To that end, drop all statements |\input{childdoc.def}|
and perform the replacements as outlined below.
Instead of |\childdocmain{|\textit{main}|}| add the following code
to the top of the main file:
%
\begin{center}
\begin{tabular}{l}
|\||ifdefined\childdocname\endinput\||fi\newif\ifchilddoc|\\
|\edef\childdocname{\scantokens\expandafter{\jobname\noexpand}}|\\
|\def\childdocmain{|\textit{main}|}\||ifx\childdocmain\childdocname\||else|\\
|\childdoctrue\includeonly{\childdocname}\let\jobname\childdocmain\||fi|\\
\end{tabular}
\end{center}
%
Instead of |\childdocof{|\textit{main}|}| just include the main file
at the top of each child file:
%
\begin{center}
|\input{|\textit{main}|}|
\end{center}
%
A simple redirection |\childdocforward{|\textit{dest}|}| is achieved by:
%
\begin{center}
|\def\jobname{|\textit{dest}|}\input{\jobname}|
\end{center}
%
The redirection with prefix
|\childdocforwardprefix[|\textit{prefix}|]{|\textit{dest}|}|
is accomplished by:
%
\begin{center}
\begin{tabular}{l}
|{\edef\jobname{\scantokens\expandafter{\jobname\noexpand}}|\\
|\def\redirectjob |\textit{prefix}|#1~~~{\gdef\jobname{|\textit{dest}|#1}}|\\
|\expandafter\redirectjob\jobname~~~}\input{\jobname}|
\end{tabular}
\end{center}

In an alternative approach,
child documents can be compiled by a specific command line
without additional code or specific definitions:
%
\begin{center}
|... -jobname "|\textit{target}|" "|[\textit{flags}]%
|\includeonly{|\textit{dest}|}\input{|\textit{main}|}"|
\end{center}
%

%%%%%%%%%%%%%%%%%%%%%%%%%%%%%%%%%%%%%%%%%%%%%%%%%%%%%%%%%%%%%%%%%%%%%%%%%%%%%%%%
%%%%%%%%%%%%%%%%%%%%%%%%%%%%%%%%%%%%%%%%%%%%%%%%%%%%%%%%%%%%%%%%%%%%%%%%%%%%%%%%
\section{Information}

%%%%%%%%%%%%%%%%%%%%%%%%%%%%%%%%%%%%%%%%%%%%%%%%%%%%%%%%%%%%%%%%%%%%%%%%%%%%%%%%
\subsection{Copyright}

Copyright \copyright{} 2017--2018 Niklas Beisert

This work may be distributed and/or modified under the
conditions of the \LaTeX{} Project Public License, either version 1.3
of this license or (at your option) any later version.
The latest version of this license is in
  \url{http://www.latex-project.org/lppl.txt}
and version 1.3 or later is part of all distributions of \LaTeX{}
version 2005/12/01 or later.

This work has the LPPL maintenance status `maintained'.

The Current Maintainer of this work is Niklas Beisert.

This work consists of the files |README.txt|, |childdoc.ins| and |childdoc.dtx|
as well as the derived files |childdoc.def|, |cdocsamp.tex|
with |cdocsch1.tex|, |cdocsch2.tex|, |cdocspt3.tex|, |cdocspt4.tex|,
|cdocsdrf.tex|, |cdocsfn1.tex|, |cdocsfn2.tex|
as well as |childdoc.pdf|.

%%%%%%%%%%%%%%%%%%%%%%%%%%%%%%%%%%%%%%%%%%%%%%%%%%%%%%%%%%%%%%%%%%%%%%%%%%%%%%%%
\subsection{Files and Installation}

The package consists of the files:
%
\begin{center}
\begin{tabular}{ll}
    |README.txt|   & readme file \\
    |childdoc.ins| & installation file \\
    |childdoc.dtx| & source file \\
    |childdoc.def| & definition file \\
    |cdocsamp.tex| & sample main file \\
    |cdocsch1.tex| & sample include file \\
    |cdocsch2.tex| & sample include file \\
    |cdocspt3.tex| & sample part file \\
    |cdocspt4.tex| & sample part file \\
    |cdocsdrf.tex| & sample redirection file \\
    |cdocsfn1.tex| & sample redirection file \\
    |cdocsfn2.tex| & sample redirection file \\
    |childdoc.pdf| & manual
\end{tabular}
\end{center}
%
The distribution consists of the files
|README.txt|, |childdoc.ins| and |childdoc.dtx|.
%
\begin{itemize}
\item
Run (pdf)\LaTeX{} on |childdoc.dtx|
to compile the manual |childdoc.pdf| (this file).
\item
Run \LaTeX{} on |childdoc.ins| to create the definitions file |childdoc.def|
and the sample |cdocsamp.tex| with include files
|cdocsch1.tex|, |cdocsch2.tex|, |cdocspt3.tex|, |cdocspt4.tex|,
|cdocsdrf.tex|, |cdocsfn1.tex|, |cdocsfn2.tex|.
Then copy the file |childdoc.def| to an appropriate directory of your \LaTeX{}
distribution, e.g.\ \textit{texmf-root}|/tex/latex/childdoc|.
\end{itemize}

%%%%%%%%%%%%%%%%%%%%%%%%%%%%%%%%%%%%%%%%%%%%%%%%%%%%%%%%%%%%%%%%%%%%%%%%%%%%%%%%
\subsection{Related CTAN Packages}

There are several other packages which offer a similar functionality:
%
\begin{itemize}
\item
The packages
\href{http://ctan.org/pkg/docmute}{\textsf{docmute}},
\href{http://ctan.org/pkg/includex}{\textsf{includex}} and
\href{http://ctan.org/pkg/standalone}{\textsf{standalone}}
provide commands to include only the document body of
a child file thus allowing both files to be compiled individually.
\item
The packages \href{http://ctan.org/pkg/subdocs}{\textsf{subdocs}}
and \href{http://ctan.org/pkg/subfiles}{\textsf{subfiles}}
provide structures in which the main and child documents can be
encapsulated and allowing them to be compiled individually.
The inclusion mechanism is different from the conventional |\include|.
\item
The package \href{http://ctan.org/pkg/combine}{\textsf{combine}}
is an elaborate solution to combine several documents into one.
\end{itemize}
%
See also the CTAN topic \href{http://ctan.org/topic/subdocs}{\textsf{subdocs}}
for further related packages.
The present package differs from the above solutions in that
a document structure constructed with the conventional |\include| mechanism
just needs two extra commands at the top of every file
such that all constituent files can be compiled individually.

%%%%%%%%%%%%%%%%%%%%%%%%%%%%%%%%%%%%%%%%%%%%%%%%%%%%%%%%%%%%%%%%%%%%%%%%%%%%%%%%
%\subsection{Feature Suggestions}
%
%The following is a list of features which may be useful for future
%versions of this package:
%%
%\begin{itemize}
%\item
%\ldots
%\end{itemize}

%%%%%%%%%%%%%%%%%%%%%%%%%%%%%%%%%%%%%%%%%%%%%%%%%%%%%%%%%%%%%%%%%%%%%%%%%%%%%%%%
\subsection{Revision History}

%%%%%%%%%%%%%%%%%%%%%%%%%%%%%%%%%%%%%%%%
\paragraph{v2.0:} 2018/12/30

\begin{itemize}
\item
immediate forward processing
\item
added |\childdocby| mechanism
\item
manual restructured
\end{itemize}

%%%%%%%%%%%%%%%%%%%%%%%%%%%%%%%%%%%%%%%%
\paragraph{v1.6:} 2018/01/17

\begin{itemize}
\item
application for development of include files
\item
corrections to manual
\end{itemize}

%%%%%%%%%%%%%%%%%%%%%%%%%%%%%%%%%%%%%%%%
\paragraph{v1.5:} 2017/05/21

\begin{itemize}
\item
more complete structuring introduced
\item
|\childdocof| introduced
\item
|\childdoc| renamed to |\childdocmain|
\item
|\childredirect| renamed to |\childdocforward| and |\childdocforwardprefix|
and functionality expanded
\end{itemize}

%%%%%%%%%%%%%%%%%%%%%%%%%%%%%%%%%%%%%%%%
\paragraph{v1.0:} 2017/04/27

\begin{itemize}
\item
manual and install package
\item
first version published on CTAN
\end{itemize}

%%%%%%%%%%%%%%%%%%%%%%%%%%%%%%%%%%%%%%%%
\paragraph{v0.6:} 2017/04/26

\begin{itemize}
\item
redirection mechanism added
\end{itemize}

%%%%%%%%%%%%%%%%%%%%%%%%%%%%%%%%%%%%%%%%
\paragraph{v0.5:} 2017/04/26

\begin{itemize}
\item
functionality in definition file
\end{itemize}


%%%%%%%%%%%%%%%%%%%%%%%%%%%%%%%%%%%%%%%%%%%%%%%%%%%%%%%%%%%%%%%%%%%%%%%%%%%%%%%%
%%%%%%%%%%%%%%%%%%%%%%%%%%%%%%%%%%%%%%%%%%%%%%%%%%%%%%%%%%%%%%%%%%%%%%%%%%%%%%%%
%%%%%%%%%%%%%%%%%%%%%%%%%%%%%%%%%%%%%%%%%%%%%%%%%%%%%%%%%%%%%%%%%%%%%%%%%%%%%%%%
\appendix

\settowidth\MacroIndent{\rmfamily\scriptsize 000\ }

 \DocInput{childdoc.dtx}

\end{document}
%</driver>
% \fi
%
% %%%%%%%%%%%%%%%%%%%%%%%%%%%%%%%%%%%%%%%%%%%%%%%%%%%%%%%%%%%%%%%%%%%%%%%%%%%%%%
% %%%%%%%%%%%%%%%%%%%%%%%%%%%%%%%%%%%%%%%%%%%%%%%%%%%%%%%%%%%%%%%%%%%%%%%%%%%%%%
% \section{Sample}
%\iffalse
%<*samplemain>
%\fi
%
% The following presents a sample document
% with two chapters, two parts, a title page,
% a compile flag as well as three forwarding files to set the flag.
% It consists of eight |.tex| files:
% \begin{center}
% \begin{tabular}{ll}
% |cdocsamp.tex|&main file\\
% |cdocsch1.tex|&include file for chapter 1\\
% |cdocsch2.tex|&include file for chapter 2\\
% |cdocspt3.tex|&include file for part 3\\
% |cdocspt4.tex|&include file for part 4\\
% |cdocsdrf.tex|&forwarding file for main file in draft mode\\
% |cdocsfi1.tex|&forwarding file for final version of chapter 1\\
% |cdocsfi2.tex|&forwarding file for final version of chapter 2\\
% \end{tabular}
% \end{center}
% Each of the eight files can be compiled directly by the \LaTeX{} compiler.
%
% %%%%%%%%%%%%%%%%%%%%%%%%%%%%%%%%%%%%%%
% \paragraph{Main File.}
%
% The main file is called |cdocsamp.tex|.
%
% Load the \textsf{childdoc} definitions and
% declare the filename for the main document:
%    \begin{macrocode}
\input{childdoc.def}
\childdocmain{}
%    \end{macrocode}

% Optional override for |\version| flag:
%    \begin{macrocode}
%%\ifchilddoc\else\providecommand{\version}{draft}\fi
%    \end{macrocode}

% Define the default values for the |\version| flag
% (|final| for the main file and |draft| for childs):
%    \begin{macrocode}
\ifchilddoc
\providecommand{\version}{draft}
\else
\providecommand{\version}{final}
\fi
%    \end{macrocode}

% Load the standard document class:
%    \begin{macrocode}
\documentclass[12pt]{article}
%    \end{macrocode}

% Start the document body:
%    \begin{macrocode}
\begin{document}
%    \end{macrocode}

% Declare a title page.
% Print title, part of document being processed and version flag:
%    \begin{macrocode}
\addtocounter{page}{-1}
\begin{center}
{\LARGE\bfseries{}childdoc example\par}
\vspace{1cm}
\ifchilddoc
\ifchilddocmanual part\else chapter\fi:
`\childdocname' of `\childdocjob'\par
\else
main document: `\childdocjob'\par
\fi
version: \version\par
\end{center}
\newpage
%    \end{macrocode}

% Manually include selected file,
% otherwise process as usual:
%    \begin{macrocode}
\ifchilddocmanual
\section*{part `\childdocname'}
\input{\childdocname}
\else
%    \end{macrocode}

% Include the two chapters:
%    \begin{macrocode}
\include{cdocsch1}
\include{cdocsch2}
%    \end{macrocode}

% Include the two parts unless only chapters should be displayed:
%    \begin{macrocode}
\ifchilddoc\else
\section{part three}
\input{cdocspt3}
\section{part four}
\input{cdocspt4}
\fi
%    \end{macrocode}

% Process as usual until here:
%    \begin{macrocode}
\fi
%    \end{macrocode}

% End of document body:
%    \begin{macrocode}
\end{document}
%    \end{macrocode}
%\iffalse
%</samplemain>
%\fi
%
% %%%%%%%%%%%%%%%%%%%%%%%%%%%%%%%%%%%%%%
% \paragraph{Chapter Include Files.}
%
% The include files are called |cdocsch1.tex| and |cdocsch2.tex|.
%
%\iffalse
%<*samplechap1|samplechap2>
%\fi

% Optional override for |\version| flag:
%    \begin{macrocode}
%%\providecommand{\version}{final}
%    \end{macrocode}

% Include the main document:
%    \begin{macrocode}
\input{childdoc.def}
\childdocof{cdocsamp}
%    \end{macrocode}

%\iffalse
%</samplechap1|samplechap2>
%\fi
%
%\iffalse
%<*samplechap1>
%\fi
% Some text for chapter 1:
%    \begin{macrocode}
\section{one}
some text in chapter one
%    \end{macrocode}

%\iffalse
%</samplechap1>
%\fi
% Some text for chapter 2:
%\iffalse
%<*samplechap2>
%\fi
%    \begin{macrocode}
\section{two}
more text in chapter two
%    \end{macrocode}

%\iffalse
%</samplechap2>
%\fi
%
% %%%%%%%%%%%%%%%%%%%%%%%%%%%%%%%%%%%%%%
% \paragraph{Part Include Files.}
%
% The include files are called |cdocspt3.tex| and |cdocspt4.tex|.
%
%\iffalse
%<*samplepart3|samplepart4>
%\fi

% Optional override for |\version| flag:
%    \begin{macrocode}
%%\providecommand{\version}{final}
%    \end{macrocode}

% Include the main document:
%    \begin{macrocode}
\input{childdoc.def}
\childdocby{cdocsamp}
%    \end{macrocode}

%\iffalse
%</samplepart3|samplepart4>
%\fi
%
%\iffalse
%<*samplepart3>
%\fi
% Some text for part 3:
%    \begin{macrocode}
some text in part three
%    \end{macrocode}

%\iffalse
%</samplepart3>
%\fi
% Some text for part 4:
%\iffalse
%<*samplepart4>
%\fi
%    \begin{macrocode}
more text in part four
%    \end{macrocode}

%\iffalse
%</samplepart4>
%\fi
%
% %%%%%%%%%%%%%%%%%%%%%%%%%%%%%%%%%%%%%%
% \paragraph{Forwarding for a Complete Draft.}
%
% The following forwarding file |cdocsdrf.tex|
% compiles the main document in draft mode:
%\iffalse
%<*sampledraft>
%\fi
%    \begin{macrocode}
\def\version{draft}
\input{childdoc.def}
\childdocforward{cdocsamp}
%    \end{macrocode}

%\iffalse
%</sampledraft>
%\fi
%
% %%%%%%%%%%%%%%%%%%%%%%%%%%%%%%%%%%%%%%
% \paragraph{Forwarding for Final Version of the Chapters.}
%
% The following forwarding files |cdocsfn1.tex| and |cdocsfn2.tex|
% (with identical content)
% compile the final versions of the child documents
% |cdocsch1.tex| and |cdocsch2.tex|, respectively:
%\iffalse
%<*samplefinal>
%\fi
%    \begin{macrocode}
\def\version{final}
\input{childdoc.def}
\childdocforwardprefix[cdocsamp]{cdocsfn}{cdocsch}
%    \end{macrocode}

%\iffalse
%</samplefinal>
%\fi
%
% %%%%%%%%%%%%%%%%%%%%%%%%%%%%%%%%%%%%%%
% \paragraph{Command Line Processing.}
%
% The following three command lines generate the output files
% |cdocscld|, |cdocscl1| and |cdocscl2|
% which should be identical to
% |cdocsdrf|, |cdocsch1| and |cdocsfn2|, respectively:
% \begin{center}
% \begin{tabular}{l}
% |latex -jobname cdocscld \|\\
% |  "\def\version{draft}\input{childdoc.def}\childdocforward{cdocsamp}"|\\
% |latex -jobname cdocscl1 \|\\
% |  "\input{childdoc.def}\childdocforward[cdocsamp]{cdocsch1}"|\\
% |latex -jobname cdocscl2 \|\\
% |  "\def\version{final}\input{childdoc.def}\childdocforward{cdocsch2}"|
% \end{tabular}
% \end{center}
% Note that the trailing backslash on each first line
% merely continues the input to the second line
% (for convenient cut ant paste).
% Furthermore, the command |latex| can be replaced by any
% of its alternative versions such as |pdflatex|.
%
% %%%%%%%%%%%%%%%%%%%%%%%%%%%%%%%%%%%%%%%%%%%%%%%%%%%%%%%%%%%%%%%%%%%%%%%%%%%%%%
% %%%%%%%%%%%%%%%%%%%%%%%%%%%%%%%%%%%%%%%%%%%%%%%%%%%%%%%%%%%%%%%%%%%%%%%%%%%%%%
% \section{Implementation}
%\iffalse
%<*package>
%\fi
%
% This section describes the definitions file |childdoc.def|.

% The definitions cannot be loaded using |\usepackage| or |\RequirePackage|
% which has a mechanism to prevent loading a style file more than once.
% When loading the definitions by means of |\input|
% multiple instances have to be prevented manually:
%\iffalse
%This code needs to be before the `\ProvidesFile' directive
%which is defined at the beginning of this file.
%Therefore it is also placed there and commented out here.
%</package>
%<*discard>
%\fi
%    \begin{macrocode}
\ifdefined\childdocmain\endinput\fi
%    \end{macrocode}
%\iffalse
%</discard>
%<*package>
%\fi
%
% \macro{\ifchilddoc}
% \macro{\ifchilddocmanual}
% The conditional |\ifchilddoc| tells whether a
% child (true) or main (false) document is being compiled.
% The conditional |\ifchilddocmanual| tells whether
% the |\includeonly| mechanism is used (false) or
% the selection of child files must be performed manually (true).
% The definitions initialise to false:
%    \begin{macrocode}
\newif\ifchilddoc
\newif\ifchilddocmanual
%    \end{macrocode}

% \macro{\childdocname}
% \macro{\childdocjob}
% The macro |\childdocname| stores the name of the main document
% to be compiled. The macro |\childdocjob| stores the name of
% the document on which the \LaTeX{} compiler was originally invoked.
% The content of |\jobname| cannot be compared
% to filenames specified in the source due to different catcodes.
% The following code rescans |\jobname|, stores the result
% in |\childdocname| and saves a copy in |\childdocjob|:
%    \begin{macrocode}
\edef\childdocname{\scantokens\expandafter{\jobname\noexpand}}
\let\childdocjob\childdocname
%    \end{macrocode}

% \macro{\childdocdisable}
% The macro |\childdocdisable| prevents the main file
% from being processed more than once.
% At this stage, the main document command |\childdocmain|
% is assumed to be called once again where it should do nothing.
% Any subsequent call to it should prevent
% a secondary processing of the main document
% It overwrites the forwarding commands
% |\childdocof| and |\childdocforward|
% with empty macros to prevent further inclusions of the main document:
%    \begin{macrocode}
\newcommand{\childdocdisable}
{
  \renewcommand{\childdocmain}[1]{\renewcommand{\childdocmain}[1]{\endinput}}
  \renewcommand{\childdocof}[1]{}
  \renewcommand{\childdocby}[2][]{}
  \renewcommand{\childdocforward}[2][]{}
  \renewcommand{\childdocdisable}{}
}
%    \end{macrocode}

% \macro{\childdocmain}
% The macro |\childdocmain| is to be called at the top of the main file
% with nothing or the main filename (without extension) as argument.
% First, it breaks loops.
% If the argument is not empty and does not match |\childdocname|
% (which is set by the first inclusion of |childdoc.def|),
% |\ifchilddoc| is set to true, |\includeonly| is applied to the child file
% and |\jobname| is set to the main file
% (for proper handling of |.aux| files):
%    \begin{macrocode}
\newcommand{\childdocmain}[1]
{
  \childdocdisable\childdocmain{}
  \if?#1?\else
    \begingroup
      \def\childdoctmp{#1}
      \ifx\childdoctmp\childdocname
        \def\childdoctmp{}
      \else
        \def\childdoctmp
        {
          \childdoctrue
          \includeonly{\childdocname}
          \def\childdocjob{#1}
          \def\jobname{#1}
        }
      \fi
      \expandafter
    \endgroup
    \childdoctmp
  \fi
}
%    \end{macrocode}

% \macro{\childdocof}
% The command |\childdocof| redirects
% compilation to the main file |#1|.
%    \begin{macrocode}
\newcommand{\childdocof}[1]
{
  \childdocdisable
  \childdoctrue
  \includeonly{\childdocname}
  \def\jobname{#1}
  \def\childdocjob{#1}
  \input{#1}
}
%    \end{macrocode}

% \macro{\childdocby}
% The command |\childdocby| ....
%    \begin{macrocode}
\newcommand{\childdocby}[2][]
{
  \childdocdisable
  \childdoctrue
  \childdocmanualtrue
  \if?#1?\else
    \def\jobname{#2}
  \fi
  \def\childdocjob{#2}
  \input{#2}
  \endinput
}
%    \end{macrocode}

% \macro{\childdocforward}
% The command |\childdocforward| redirects
% compilation to the main file or
% (if the optional argument is given) a child file.
% Parameters are set as if the main file
% or a child file starting with |\childdocof| was compiled.
% Then compilation is handed over to the main file:
%    \begin{macrocode}
\newcommand{\childdocforward}[2][]
{
  \begingroup
    \if?#1?
      \def\childdoctmp
      {
        \def\childdocname{#2}
        \def\childdocjob{#2}
        \def\jobname{#2}
        \input{#2}
        \endinput
      }
    \else
      \def\childdoctmp
      {
        \childdocdisable
        \def\childdocname{#2}
        \childdoctrue
        \includeonly{#2}
        \def\childdocjob{#1}
        \def\jobname{#1}
        \input{#1}
        \endinput
      }
    \fi
    \expandafter
  \endgroup
  \childdoctmp
}
%    \end{macrocode}

% \macro{\childdocforwardprefix}
% The command |\childdocforwardprefix| redirects
% compilation to the main or a child file by means of a pattern.
% The prefix |#1| in the current filename is replaced by |#2|
% and the suffix of the current filename is kept
% (it is assumed that the filename does not contain the substring `|~~~|'
% which is used as a delimiter).
% Compilation is handed over to the new file by |\childdocforward|:
%    \begin{macrocode}
\newcommand{\childdocforwardprefix}[3][]
{
  \begingroup
    \def\childdocextract #2##1~~~{\def\childdoctmp{\childdocforward[#1]{#3##1}}}
    \expandafter\childdocextract\childdocname~~~
    \expandafter
  \endgroup
  \childdoctmp
}
%    \end{macrocode}

% \macro{\childdoc}
% The deprecated macro |\childdoc| is a legacy version of |\childdocmain|:
%    \begin{macrocode}
\newcommand{\childdoc}{\childdocmain}
%    \end{macrocode}

% \macro{\childdocredirect}
% The deprecated macro |\childdocredirect| is a legacy version
% of |\childdocforward| and |\childdocforwardprefix|:
%    \begin{macrocode}
\newcommand{\childdocredirect}[2][]
{
  \begingroup
    \if?#1?
      \def\childdoctmp{\childdocforward{#2}}
    \else
      \def\childdoctmp{\childdocforwardprefix{#1}{#2}}
    \fi
    \expandafter
  \endgroup
  \childdoctmp
}
%    \end{macrocode}

%\iffalse
%</package>
%\fi
%
\endinput

\childdocby{cdocsamp}
%    \end{macrocode}

%\iffalse
%</samplepart3|samplepart4>
%\fi
%
%\iffalse
%<*samplepart3>
%\fi
% Some text for part 3:
%    \begin{macrocode}
some text in part three
%    \end{macrocode}

%\iffalse
%</samplepart3>
%\fi
% Some text for part 4:
%\iffalse
%<*samplepart4>
%\fi
%    \begin{macrocode}
more text in part four
%    \end{macrocode}

%\iffalse
%</samplepart4>
%\fi
%
% %%%%%%%%%%%%%%%%%%%%%%%%%%%%%%%%%%%%%%
% \paragraph{Forwarding for a Complete Draft.}
%
% The following forwarding file |cdocsdrf.tex|
% compiles the main document in draft mode:
%\iffalse
%<*sampledraft>
%\fi
%    \begin{macrocode}
\def\version{draft}
% \iffalse
%
% childdoc.dtx Copyright (C) 2017-2018 Niklas Beisert
%
% This work may be distributed and/or modified under the
% conditions of the LaTeX Project Public License, either version 1.3
% of this license or (at your option) any later version.
% The latest version of this license is in
%   http://www.latex-project.org/lppl.txt
% and version 1.3 or later is part of all distributions of LaTeX
% version 2005/12/01 or later.
%
% This work has the LPPL maintenance status `maintained'.
%
% The Current Maintainer of this work is Niklas Beisert.
%
% This work consists of the files childdoc.dtx and childdoc.ins
% and the derived files childdoc.def and cdocsamp.tex with
% cdocsch1.tex, cdocsch2.tex, cdocsdrf.tex, cdocsfn1.tex, cdocsfn2.tex.
%
%<package>\ifdefined\childdocmain\endinput\fi
%<package>\ProvidesFile{childdoc.def}[2018/12/30 v2.0 child document driver]
%<samplemain>\ProvidesFile{cdocsamp.tex}[2018/12/30 v2.0 sample for childdoc]
%<*driver>
%\ProvidesFile{childdoc.drv}[2018/12/30 v2.0 childdoc reference manual file]
\PassOptionsToClass{10pt,a4paper}{article}
\documentclass{ltxdoc}

\usepackage[margin=35mm]{geometry}
\usepackage{hyperref}
\usepackage{hyperxmp}
\usepackage[usenames]{color}

\hypersetup{colorlinks=true}
\hypersetup{pdfstartview=FitH}
\hypersetup{pdfpagemode=UseNone}
\hypersetup{pdfsource={}}
\hypersetup{pdflang={en-UK}}
\hypersetup{pdfcopyright={Copyright 2017-2018 Niklas Beisert.
  This work may be distributed and/or modified under the
  conditions of the LaTeX Project Public License, either version 1.3
  of this license or (at your option) any later version.}}
\hypersetup{pdflicenseurl={http://www.latex-project.org/lppl.txt}}
\hypersetup{pdfcontactaddress={ETH Zurich, ITP, HIT K,
  Wolfgang-Pauli-Strasse 27}}
\hypersetup{pdfcontactpostcode={8093}}
\hypersetup{pdfcontactcity={Zurich}}
\hypersetup{pdfcontactcountry={Switzerland}}
\hypersetup{pdfcontactemail={nbeisert@itp.phys.ethz.ch}}
\hypersetup{pdfcontacturl={http://people.phys.ethz.ch/\xmptilde nbeisert/}}

\newcommand{\secref}[1]{\hyperref[#1]{section \ref*{#1}}}

\parskip1ex
\parindent0pt
\let\olditemize\itemize
\def\itemize{\olditemize\parskip0pt}

\begin{document}

\title{The \textsf{childdoc} Package}
\hypersetup{pdftitle={The childdoc Package}}
\author{Niklas Beisert\\[2ex]
  Institut f\"ur Theoretische Physik\\
  Eidgen\"ossische Technische Hochschule Z\"urich\\
  Wolfgang-Pauli-Strasse 27, 8093 Z\"urich, Switzerland\\[1ex]
  \href{mailto:nbeisert@itp.phys.ethz.ch}
  {\texttt{nbeisert@itp.phys.ethz.ch}}}
\hypersetup{pdfauthor={Niklas Beisert}}
\hypersetup{pdfsubject={Manual for the LaTeX2e Package childdoc}}
\date{30 December 2018, \textsf{v2.0}}
\maketitle

\begin{abstract}\noindent
\textsf{childdoc} is a \LaTeXe{} package
that enables the direct compilation
of document sections included by |\include|
to individual files.
\end{abstract}

\begingroup
\parskip0ex
\tableofcontents
\endgroup

%%%%%%%%%%%%%%%%%%%%%%%%%%%%%%%%%%%%%%%%%%%%%%%%%%%%%%%%%%%%%%%%%%%%%%%%%%%%%%%%
%%%%%%%%%%%%%%%%%%%%%%%%%%%%%%%%%%%%%%%%%%%%%%%%%%%%%%%%%%%%%%%%%%%%%%%%%%%%%%%%
\section{Introduction}

\LaTeX{} provides a mechanism to structure a large document (such as a book)
into a main file and several child files (containing the chapters)
using the |\include| command.
This mechanism is beneficial for documents
which span hundreds of pages in order to
make the source file(s) more manageable.
Moreover, compilation can be restricted to
selected child files by means of the |\includeonly| command.
The latter feature can be used to reduce the compilation time while editing
(this was significantly more useful in the earlier days of \LaTeX{})
or to generate a smaller document which is easier to navigate.
Another application of |\includeonly| is to generate
documents consisting of selected parts of the complete document.

However, there are a few drawbacks of the plain |\include| mechanism:
\begin{itemize}
\item
The child files cannot be compiled on their own,
they can only be compiled via the main file.
A naive editing environment
(such as a text editor with an option
to have the current file processed by \LaTeX)
may require one to switch to the main file before compiling;
attempting to compile the child file produces errors.
\item
The main file must be modified (each time)
to adjust the |\includeonly| command
to the present needs. This easily leaves the main file in a messy state.
\item
The generated document will always carry the filename
of the main document. This is inconvenient if
several child files are to be compiled and
to be kept for distribution.
\end{itemize}

The present package provides a simple interface
to make child files individually compilable by \LaTeX{}.
Compiling a child file then has the same effect as compiling
the main file with an |\includeonly| command
to select the appropriate child.
Moreover the generated document will carry the name of the child
rather than the main file.
This resolves all three above issues.

This feature is meant to make the editing of books,
thesis documents and lecture notes somewhat more convenient.
However, the package can also be used efficiently for
composing a series of documents (such as exercise sheets)
which are typically distributed individually.
It then assists the author in generating the individual documents
(potentially in different versions)
as well as a document containing the collected series.
Another application is in developing style files
or other kinds of included material
where compilation of the style file could redirect
to a sample or test file.

%%%%%%%%%%%%%%%%%%%%%%%%%%%%%%%%%%%%%%%%%%%%%%%%%%%%%%%%%%%%%%%%%%%%%%%%%%%%%%%%
%%%%%%%%%%%%%%%%%%%%%%%%%%%%%%%%%%%%%%%%%%%%%%%%%%%%%%%%%%%%%%%%%%%%%%%%%%%%%%%%
\section{Usage}

First of all, the package \textsf{childdoc} is \emph{not} a standard
\LaTeXe{} |.sty| style file! Therefore it needs to be invoked in
a non-standard way.

%%%%%%%%%%%%%%%%%%%%%%%%%%%%%%%%%%%%%%%%%%%%%%%%%%%%%%%%%%%%%%%%%%%%%%%%%%%%%%%%
\subsection{Included Files}
\label{sec:include}

%%%%%%%%%%%%%%%%%%%%%%%%%%%%%%%%%%%%%%%%
\DescribeMacro{\childdocmain}
To use the package, add the commands
\begin{center}
\begin{tabular}{l}
|\input{childdoc.def}|\\
|\childdocmain{}|\\
\end{tabular}
\end{center}
at the very top of the main \LaTeX{} file,
in particular \emph{before} the |\documentclass| statement!
The argument of |\childdocmain| should be left empty
(but it must be present).

%%%%%%%%%%%%%%%%%%%%%%%%%%%%%%%%%%%%%%%%
\DescribeMacro{\childdocof}
Furthermore, add the commands
\begin{center}
\begin{tabular}{l}
|\input{childdoc.def}|\\
|\childdocof{|\textit{main}|}|\\
\end{tabular}
\end{center}
at the top of every child file \textit{child}
which is included by |\include{|\textit{child}|}|
from within the main file
(or at least for those files to be compiled individually).
The argument \textit{main} must be the filename of the main file.

There are a couple of
considerations in setting up the main and child documents:

%%%%%%%%%%%%%%%%%%%%%%%%%%%%%%%%%%%%%%%%
\paragraph{Restrictions.}

Please note the following restrictions:
\begin{itemize}
\item
|\childdocmain| must be called with one argument \textit{main}
to ensure compatibility with earlier version of the package.
It must either be empty (|\childdocmain{}|)
or precisely match the filename of the main file in which it is specified.
See \secref{sec:detection} for further information.
\item
The filename \textit{main} must be specified without the |.tex| extension.
\item
The filename \textit{main} is case sensitive
(even in case-insensitive file systems)
due to internal string comparison.
\item
The argument \textit{main} should be fully expanded, it cannot be a macro.
\item
Subdirectories and special characters should be avoided in filenames.
\item
The command |\childdocmain{|\textit{main}|}| must be followed by a whitespace.
It should not be followed immediately by another command
or by a comment mark `|%|'.
This is because the \TeX{} parser reads the token immediately following
the argument of |\childdocmain| and puts it
at the beginning of every child section;
however, a white\-space is ignored.
\end{itemize}

%%%%%%%%%%%%%%%%%%%%%%%%%%%%%%%%%%%%%%%%
\paragraph{Content of Main File.}

It is advisable to place all content in the child files included by |\include|.
Any output contained in the main file will appear in all child documents
unless suppressed manually;
it cannot be suppressed automatically by the |\includeonly| directive
and thus should normally be avoided.
A method to include some content in the main file
by means of conditional processing is described in \secref{sec:conditional}.

%%%%%%%%%%%%%%%%%%%%%%%%%%%%%%%%%%%%%%%%
\paragraph{Page Numbering.}

When only a part of the document is compiled,
the appropriate numbering of pages
(as well as other status parameters)
is determined from the |.aux| files.
The latter contain information from previous passes.
However this information needs to propagate through
all intermediate child documents.
Therefore the page numbering in child documents may well
be inconsistent until the complete document is compiled at least once.

A useful (if unconventional) way to always ensure a consistent
page numbering is to restart the numbering in each child document
and denote the pages by `\textit{child}|.|\textit{page}'
where \textit{child} represents the chapter/section number of the child file.
This can be achieved by the command
|\numberwithin{page}{|\textit{child}|}|
of the \textsf{amsmath} package
where \textit{child} can be |chapter| or |section|
depending on the chosen structuring.
Alternatively, one can modify the macro |\thepage| appropriately
and reset the counter |page| at the start of each child file.

%%%%%%%%%%%%%%%%%%%%%%%%%%%%%%%%%%%%%%%%%%%%%%%%%%%%%%%%%%%%%%%%%%%%%%%%%%%%%%%%
\subsection{Conditional Processing}
\label{sec:conditional}

The package provides a mechanism to compile different versions
of a document. To customise the versions further some conditional processing
can come in handy to distinguish which version is being compiled.
The package provides two macros to describe the compilation context:

%%%%%%%%%%%%%%%%%%%%%%%%%%%%%%%%%%%%%%%%
\DescribeMacro{\ifchilddoc}
The conditional |\ifchilddoc| distinguishes between the compilation of
child documents and the main document:
%
\begin{center}
|\ifchilddoc |\textit{child-code}| |[|\||else |\textit{main-code}]| \||fi|
\end{center}

%%%%%%%%%%%%%%%%%%%%%%%%%%%%%%%%%%%%%%%%
\DescribeMacro{\childdocname}
\DescribeMacro{\childdocjob}
The macro |\childdocname| contains the filename (without extension)
of the main or child file being processed.
Note that |\childdocjob| will always contain the name of the main file.

%%%%%%%%%%%%%%%%%%%%%%%%%%%%%%%%%%%%%%%%
\paragraph{Title Page.}

Conditional processing can be used to include a title or banner page
in the main document when proper precautions are taken.
Importantly, the code in the main file should ensure that the page counter
(as well as other status parameters which are stored in the |.aux| files)
takes the same value after the conditional processing.
Otherwise the page numbers may take divergent values
depending on which part is compiled.

For example, a title page could be declared by:
%
\begin{center}
\begin{tabular}{l}
|\ifchilddoc\||else|\\
|\addtocounter{page}{-1}|\\
\textit{code for title page}\\
|\newpage|\\
|\||fi|
\end{tabular}
\end{center}
%
A banner page for the child documents can be generated by:
%
\begin{center}
\begin{tabular}{l}
|\ifchilddoc|\\
|\addtocounter{page}{-1}|\\
\textit{code for banner page}\\
|\newpage|\\
|\||fi|
\end{tabular}
\end{center}
%
Here one could write a message such as:
\begin{center}
|This is the part \childdocname{} of \childdocjob{}.|
\end{center}

%%%%%%%%%%%%%%%%%%%%%%%%%%%%%%%%%%%%%%%%%%%%%%%%%%%%%%%%%%%%%%%%%%%%%%%%%%%%%%%%
\subsection{Flags}
\label{sec:flags}

The package makes it easy to generate different versions
of the main or child documents.
To this end compilation flags can be defined
and assigned different default values.
They will be particularly useful in conjunction
with the forwarding mechanism described in \secref{sec:forward}.

For example, it may be useful to have a flag |\version|
which can be set to |draft| or |final|.
The document source will contain some conditional code
depending on the value of |\version|.
Suppose further, the flag should default to |final| for the main file
and to |draft| for child files
which is a natural assignment for editing the document.
This is achieved by placing the following code
in the preamble of the main document
(below the |\childdocmain| directive):
%
\begin{center}
\begin{tabular}{l}
|\ifchilddoc|\\
|\providecommand{\version}{draft}|\\
|\||else|\\
|\providecommand{\version}{final}|\\
|\||fi|
\end{tabular}
\end{center}
%
The definition by |\providecommand| makes sure
that previous definitions are not overwritten.
Further statements |\providecommand{\version}{...}|
can thus be added before the above code to override it.

For the main file, one might add a line
(between |\childdocmain| and the above block)
%
\begin{center}
|%\ifchilddoc\||else\providecommand{\version}{draft}\||fi|
\end{center}
%
which can be uncommented to produce a draft version.
Likewise one can add a line to the very top of a child file
(above the |\childdocof{|\textit{main}|}| directive)
%
\begin{center}
|%\providecommand{\version}{final}|
\end{center}
%
which can be uncommented to produce the final version of this child document.

%%%%%%%%%%%%%%%%%%%%%%%%%%%%%%%%%%%%%%%%%%%%%%%%%%%%%%%%%%%%%%%%%%%%%%%%%%%%%%%%
\subsection{Forwarding}
\label{sec:forward}

Different versions of the main or child documents
using compilation flags as described in \secref{sec:flags}
can be (permanently) stored in different files
for convenient compilation, viewing and distribution.
To this end, the package defines a command
to pass on compilation to a different file:

%%%%%%%%%%%%%%%%%%%%%%%%%%%%%%%%%%%%%%%%
\DescribeMacro{\childdocforward}
The command |\childdocforward| redirects processing to
another source file:
%
\begin{center}
\begin{tabular}{l}
|\input{childdoc.def}|\\
|\childdocforward[|\textit{main}|]{|\textit{dest}|}|\\
\end{tabular}
\end{center}
%
The argument \textit{dest} is the destination file
(without extension).
It should be the main file or one of the child files.
Note that further \textsf{childdoc} directives
such as |\childdocof| and |\childdocforward|
in the indicated file will be processed in this form.
The optional argument \textit{main}
passes on directly to the main file \textit{main}
while pretending to compile the child \textit{dest}.
This form behaves as if \textit{dest}
issues |\childdocof{|\textit{main}|}| right away,
and no further \textsf{childdoc} directives will be processed.

%%%%%%%%%%%%%%%%%%%%%%%%%%%%%%%%%%%%%%%%
\DescribeMacro{\...prefix}
In the alternative form |\childdocforwardprefix|,
%
\begin{center}
\begin{tabular}{l}
|\input{childdoc.def}|\\
|\childdocforwardprefix[|\textit{main}|]{|\textit{prefix}|}{|\textit{dest}|}|
\end{tabular}
\end{center}
%
the destination file is determined by a pattern
depending on the current file:
To make this work, the current file must be called
`{\textit{prefix}\hspace{0.2em}\textit{suffix}}'
with \textit{prefix} matching precisely the argument.
Processing is then passed on to the file
`{\textit{dest}\hspace{0.2em}\textit{suffix}}'.
Surely, the same effect is achieved by
directly specifying the
argument `{\textit{dest}\hspace{0.2em}\textit{suffix}}'
in the first form.
However, that requires to set up a different file
for each child. With the alternative form of the command
all these files can have exactly the same content
which simplifies setting them up and maintaining them.

For example, the following file |draft.tex|
with a compilation flag |\version| as described in \secref{sec:flags}
compiles the main document as a draft:
%
\begin{center}
\begin{tabular}{l}
|\def\version{draft}|\\
|\input{childdoc.def}|\\
|\childdocforward{|\textit{main}|}|
\end{tabular}
\end{center}
%
Likewise, the following files |final|\textit{nn}|.tex|
compile the final version of the child document
|child|\textit{nn}|.tex|:
%
\begin{center}
\begin{tabular}{l}
|\def\version{final}|\\
|\input{childdoc.def}|\\
|\childdocforwardprefix{final}{child}|
\end{tabular}
\end{center}
%

Note that when several versions of a main file and/or of each child file
are to be generated, it may be convenient to set up a |Makefile| or
shell script to automatise the process.

%%%%%%%%%%%%%%%%%%%%%%%%%%%%%%%%%%%%%%%%%%%%%%%%%%%%%%%%%%%%%%%%%%%%%%%%%%%%%%%%
\subsection{Command Line Processing}
\label{sec:commandline}

The effect of redirection files can also be achieved by invoking
the \LaTeX{} compiler with a more elaborate command line.
Most conveniently this should be done as part
of a shell script or a |Makefile|.

When using \textsf{childdoc} in the main file, the following
command lines effectively perform a redirection
(note that depending on the shell being used,
backslashes may have to be doubled: `|\|' $\to$ `|\\|'):
%
\begin{center}
|... -jobname "|\textit{target}|" |\\|"|[\textit{flags}]%
|\input{childdoc.def}\childdocforward[|\textit{main}|]{|\textit{dest}|}"|
\end{center}
%
Here \textit{target} is the name of the output file,
\textit{main} is the name of the main file
and \textit{dest} is the name of the main or child file to be processed
(all filenames without extensions).
The optional argument \textit{main} can be omitted
if \textit{main} matches \textit{dest}.
Optionally, compilation \textit{flags} can be defined via |\def| commands.
This command line makes the \TeX{} engine believe
it is compiling the file \textit{target}
whose content is specified as the latter parameter.
The provided code then forwards the processing to
\textit{main} or \textit{dest} as described in \secref{sec:forward}.

%%%%%%%%%%%%%%%%%%%%%%%%%%%%%%%%%%%%%%%%%%%%%%%%%%%%%%%%%%%%%%%%%%%%%%%%%%%%%%%%
\subsection{Include by Input}
\label{sec:input}

Including child documents by |\include| has some restrictions by design.
Most notably, the content of a child document always occupies
its own set of pages; pages cannot be shared between child documents.
Usually, this behaviour makes perfect sense
because each child document contain an essential part of the document.
However, in some situations it may be desirable to compose
a document from a collection of parts
without having mandatory page breaks between then.
For this case, the package
provides a mechanism to include parts
by |\input| which can also be processed individually.
However, by construction this mechanism
requires manual handling of the content to be output.

%%%%%%%%%%%%%%%%%%%%%%%%%%%%%%%%%%%%%%%%
\DescribeMacro{\ifchilddocmanual}
The main file should be prepared as usual, see \secref{sec:include}.
However, the document body must make a distinction
between processing of an individual part and of the main document, e.g.:
%
\begin{center}
\begin{tabular}{l}
|\ifchilddocmanual|\\
|\input{\childdocname}|\\
|\||else|\\
\textit{document body with }|\input{|\textit{part}|}|\\
|\||fi|
\end{tabular}
\end{center}
%
The conditional |\ifchilddocmanual| is true whenever
a part to be included by |\input| is being compiled,
and the name of the part is stored in |\childdocname|.

%%%%%%%%%%%%%%%%%%%%%%%%%%%%%%%%%%%%%%%%
\DescribeMacro{\childdocby}
Each part to be included by |\input| should start with:
%
\begin{center}
\begin{tabular}{l}
|\input{childdoc.def}|\\
|\childdocby{|\textit{main}|}|\\
\end{tabular}
\end{center}
%
The directive |\childdocby| is similar to |\childdocof|
described in \secref{sec:include},
but the subsequent selection of content must be done manually.
To that end, both |\ifchilddoc| and |\ifchilddocmanual|
will be true upon processing of a part,
and the name of the part is stored in |\childdocname|.
Note that |\jobname| will be set to the filename of the current part
so that each part receives an individual |.aux| file
that does not interfere with the |.aux| file(s) of the main document.
This behaviour can be altered by the alternative form
|\childdocby[*]{|\textit{main}|}| (with a non-empty optional argument)
which uses the |.aux| file of the main document
by setting |\jobname| to \textit{main}.

%%%%%%%%%%%%%%%%%%%%%%%%%%%%%%%%%%%%%%%%%%%%%%%%%%%%%%%%%%%%%%%%%%%%%%%%%%%%%%%%
\subsection{Driver Development}
\label{sec:driver}

The \textsf{childdoc} mechanism can also be use for the development
of definition files such as \LaTeX{} styles or classes.
This case differs from the above setup with multiple parts
included by |\include| in that no |\includeonly| should be invoked.
This can be achieved by starting the include file
(before |\ProvidesPackage|) with:
%
\begin{center}
\begin{tabular}{l}
|\input{childdoc.def}|\\
|\childdocforward{|\textit{main}|}|\\
\end{tabular}
\end{center}
%
or alternatively with:
%
\begin{center}
\begin{tabular}{l}
|\input{childdoc.def}|\\
|\childdocby{|\textit{main}|}|\\
\end{tabular}
\end{center}
%
Both forms have slightly different effects as described above.
The main file is prepared as usual, see \secref{sec:include}.

%%%%%%%%%%%%%%%%%%%%%%%%%%%%%%%%%%%%%%%%%%%%%%%%%%%%%%%%%%%%%%%%%%%%%%%%%%%%%%%%
\subsection{Legacy Detection}
\label{sec:detection}

The directive |\childdocmain| in the main file can detect
whether the complete document or merely a child is to be compiled
even without using the directive |\childdocof|.
This method is deprecated because it is less robust
and there is no compelling reason to use it;
it is merely provided for backward compatibility
and it may be removed in future versions.

If the detection mechanism is to be used,
it is mandatory to correctly specify
the filename of the main file as the argument of |\childdocmain|:
%
\begin{center}
\begin{tabular}{l}
|\input{childdoc.def}|\\
|\childdocmain{|\textit{main}|}|\\
\end{tabular}
\end{center}
%
If |\jobname| does not match the argument \textit{main} of |\childdocmain|,
it is assumed that |\jobname| points to the child file to be compiled.
When using |\childdocmain| with the main file specified as argument,
it suffices to start a child file
with just |\input{|\textit{main}|}|
without loading of the package and using |\childdocof|.
If instead all processing is done
with the appropriate \textsf{childdoc} directives,
the argument of \textit{main} of |\childdocmain| can be empty.

An alternative version of the command line processing described
in \secref{sec:commandline} using the detection mechanism reads:
%
\begin{center}
|... -jobname "|\textit{target}|" "|[\textit{flags}]%
[|\def\jobname{|\textit{dest}|}|]|\input{|\textit{main}|}"|
\end{center}

%%%%%%%%%%%%%%%%%%%%%%%%%%%%%%%%%%%%%%%%%%%%%%%%%%%%%%%%%%%%%%%%%%%%%%%%%%%%%%%%
\subsection{Manual Code}
\label{sec:manual}

In case one cannot be certain whether the definitions file |childdoc.def|
is installed on the target \TeX{} distribution
and one prefers not to ship it,
it is conceivable to paste a few relevant commands into the sources.

To that end, drop all statements |\input{childdoc.def}|
and perform the replacements as outlined below.
Instead of |\childdocmain{|\textit{main}|}| add the following code
to the top of the main file:
%
\begin{center}
\begin{tabular}{l}
|\||ifdefined\childdocname\endinput\||fi\newif\ifchilddoc|\\
|\edef\childdocname{\scantokens\expandafter{\jobname\noexpand}}|\\
|\def\childdocmain{|\textit{main}|}\||ifx\childdocmain\childdocname\||else|\\
|\childdoctrue\includeonly{\childdocname}\let\jobname\childdocmain\||fi|\\
\end{tabular}
\end{center}
%
Instead of |\childdocof{|\textit{main}|}| just include the main file
at the top of each child file:
%
\begin{center}
|\input{|\textit{main}|}|
\end{center}
%
A simple redirection |\childdocforward{|\textit{dest}|}| is achieved by:
%
\begin{center}
|\def\jobname{|\textit{dest}|}\input{\jobname}|
\end{center}
%
The redirection with prefix
|\childdocforwardprefix[|\textit{prefix}|]{|\textit{dest}|}|
is accomplished by:
%
\begin{center}
\begin{tabular}{l}
|{\edef\jobname{\scantokens\expandafter{\jobname\noexpand}}|\\
|\def\redirectjob |\textit{prefix}|#1~~~{\gdef\jobname{|\textit{dest}|#1}}|\\
|\expandafter\redirectjob\jobname~~~}\input{\jobname}|
\end{tabular}
\end{center}

In an alternative approach,
child documents can be compiled by a specific command line
without additional code or specific definitions:
%
\begin{center}
|... -jobname "|\textit{target}|" "|[\textit{flags}]%
|\includeonly{|\textit{dest}|}\input{|\textit{main}|}"|
\end{center}
%

%%%%%%%%%%%%%%%%%%%%%%%%%%%%%%%%%%%%%%%%%%%%%%%%%%%%%%%%%%%%%%%%%%%%%%%%%%%%%%%%
%%%%%%%%%%%%%%%%%%%%%%%%%%%%%%%%%%%%%%%%%%%%%%%%%%%%%%%%%%%%%%%%%%%%%%%%%%%%%%%%
\section{Information}

%%%%%%%%%%%%%%%%%%%%%%%%%%%%%%%%%%%%%%%%%%%%%%%%%%%%%%%%%%%%%%%%%%%%%%%%%%%%%%%%
\subsection{Copyright}

Copyright \copyright{} 2017--2018 Niklas Beisert

This work may be distributed and/or modified under the
conditions of the \LaTeX{} Project Public License, either version 1.3
of this license or (at your option) any later version.
The latest version of this license is in
  \url{http://www.latex-project.org/lppl.txt}
and version 1.3 or later is part of all distributions of \LaTeX{}
version 2005/12/01 or later.

This work has the LPPL maintenance status `maintained'.

The Current Maintainer of this work is Niklas Beisert.

This work consists of the files |README.txt|, |childdoc.ins| and |childdoc.dtx|
as well as the derived files |childdoc.def|, |cdocsamp.tex|
with |cdocsch1.tex|, |cdocsch2.tex|, |cdocspt3.tex|, |cdocspt4.tex|,
|cdocsdrf.tex|, |cdocsfn1.tex|, |cdocsfn2.tex|
as well as |childdoc.pdf|.

%%%%%%%%%%%%%%%%%%%%%%%%%%%%%%%%%%%%%%%%%%%%%%%%%%%%%%%%%%%%%%%%%%%%%%%%%%%%%%%%
\subsection{Files and Installation}

The package consists of the files:
%
\begin{center}
\begin{tabular}{ll}
    |README.txt|   & readme file \\
    |childdoc.ins| & installation file \\
    |childdoc.dtx| & source file \\
    |childdoc.def| & definition file \\
    |cdocsamp.tex| & sample main file \\
    |cdocsch1.tex| & sample include file \\
    |cdocsch2.tex| & sample include file \\
    |cdocspt3.tex| & sample part file \\
    |cdocspt4.tex| & sample part file \\
    |cdocsdrf.tex| & sample redirection file \\
    |cdocsfn1.tex| & sample redirection file \\
    |cdocsfn2.tex| & sample redirection file \\
    |childdoc.pdf| & manual
\end{tabular}
\end{center}
%
The distribution consists of the files
|README.txt|, |childdoc.ins| and |childdoc.dtx|.
%
\begin{itemize}
\item
Run (pdf)\LaTeX{} on |childdoc.dtx|
to compile the manual |childdoc.pdf| (this file).
\item
Run \LaTeX{} on |childdoc.ins| to create the definitions file |childdoc.def|
and the sample |cdocsamp.tex| with include files
|cdocsch1.tex|, |cdocsch2.tex|, |cdocspt3.tex|, |cdocspt4.tex|,
|cdocsdrf.tex|, |cdocsfn1.tex|, |cdocsfn2.tex|.
Then copy the file |childdoc.def| to an appropriate directory of your \LaTeX{}
distribution, e.g.\ \textit{texmf-root}|/tex/latex/childdoc|.
\end{itemize}

%%%%%%%%%%%%%%%%%%%%%%%%%%%%%%%%%%%%%%%%%%%%%%%%%%%%%%%%%%%%%%%%%%%%%%%%%%%%%%%%
\subsection{Related CTAN Packages}

There are several other packages which offer a similar functionality:
%
\begin{itemize}
\item
The packages
\href{http://ctan.org/pkg/docmute}{\textsf{docmute}},
\href{http://ctan.org/pkg/includex}{\textsf{includex}} and
\href{http://ctan.org/pkg/standalone}{\textsf{standalone}}
provide commands to include only the document body of
a child file thus allowing both files to be compiled individually.
\item
The packages \href{http://ctan.org/pkg/subdocs}{\textsf{subdocs}}
and \href{http://ctan.org/pkg/subfiles}{\textsf{subfiles}}
provide structures in which the main and child documents can be
encapsulated and allowing them to be compiled individually.
The inclusion mechanism is different from the conventional |\include|.
\item
The package \href{http://ctan.org/pkg/combine}{\textsf{combine}}
is an elaborate solution to combine several documents into one.
\end{itemize}
%
See also the CTAN topic \href{http://ctan.org/topic/subdocs}{\textsf{subdocs}}
for further related packages.
The present package differs from the above solutions in that
a document structure constructed with the conventional |\include| mechanism
just needs two extra commands at the top of every file
such that all constituent files can be compiled individually.

%%%%%%%%%%%%%%%%%%%%%%%%%%%%%%%%%%%%%%%%%%%%%%%%%%%%%%%%%%%%%%%%%%%%%%%%%%%%%%%%
%\subsection{Feature Suggestions}
%
%The following is a list of features which may be useful for future
%versions of this package:
%%
%\begin{itemize}
%\item
%\ldots
%\end{itemize}

%%%%%%%%%%%%%%%%%%%%%%%%%%%%%%%%%%%%%%%%%%%%%%%%%%%%%%%%%%%%%%%%%%%%%%%%%%%%%%%%
\subsection{Revision History}

%%%%%%%%%%%%%%%%%%%%%%%%%%%%%%%%%%%%%%%%
\paragraph{v2.0:} 2018/12/30

\begin{itemize}
\item
immediate forward processing
\item
added |\childdocby| mechanism
\item
manual restructured
\end{itemize}

%%%%%%%%%%%%%%%%%%%%%%%%%%%%%%%%%%%%%%%%
\paragraph{v1.6:} 2018/01/17

\begin{itemize}
\item
application for development of include files
\item
corrections to manual
\end{itemize}

%%%%%%%%%%%%%%%%%%%%%%%%%%%%%%%%%%%%%%%%
\paragraph{v1.5:} 2017/05/21

\begin{itemize}
\item
more complete structuring introduced
\item
|\childdocof| introduced
\item
|\childdoc| renamed to |\childdocmain|
\item
|\childredirect| renamed to |\childdocforward| and |\childdocforwardprefix|
and functionality expanded
\end{itemize}

%%%%%%%%%%%%%%%%%%%%%%%%%%%%%%%%%%%%%%%%
\paragraph{v1.0:} 2017/04/27

\begin{itemize}
\item
manual and install package
\item
first version published on CTAN
\end{itemize}

%%%%%%%%%%%%%%%%%%%%%%%%%%%%%%%%%%%%%%%%
\paragraph{v0.6:} 2017/04/26

\begin{itemize}
\item
redirection mechanism added
\end{itemize}

%%%%%%%%%%%%%%%%%%%%%%%%%%%%%%%%%%%%%%%%
\paragraph{v0.5:} 2017/04/26

\begin{itemize}
\item
functionality in definition file
\end{itemize}


%%%%%%%%%%%%%%%%%%%%%%%%%%%%%%%%%%%%%%%%%%%%%%%%%%%%%%%%%%%%%%%%%%%%%%%%%%%%%%%%
%%%%%%%%%%%%%%%%%%%%%%%%%%%%%%%%%%%%%%%%%%%%%%%%%%%%%%%%%%%%%%%%%%%%%%%%%%%%%%%%
%%%%%%%%%%%%%%%%%%%%%%%%%%%%%%%%%%%%%%%%%%%%%%%%%%%%%%%%%%%%%%%%%%%%%%%%%%%%%%%%
\appendix

\settowidth\MacroIndent{\rmfamily\scriptsize 000\ }

 \DocInput{childdoc.dtx}

\end{document}
%</driver>
% \fi
%
% %%%%%%%%%%%%%%%%%%%%%%%%%%%%%%%%%%%%%%%%%%%%%%%%%%%%%%%%%%%%%%%%%%%%%%%%%%%%%%
% %%%%%%%%%%%%%%%%%%%%%%%%%%%%%%%%%%%%%%%%%%%%%%%%%%%%%%%%%%%%%%%%%%%%%%%%%%%%%%
% \section{Sample}
%\iffalse
%<*samplemain>
%\fi
%
% The following presents a sample document
% with two chapters, two parts, a title page,
% a compile flag as well as three forwarding files to set the flag.
% It consists of eight |.tex| files:
% \begin{center}
% \begin{tabular}{ll}
% |cdocsamp.tex|&main file\\
% |cdocsch1.tex|&include file for chapter 1\\
% |cdocsch2.tex|&include file for chapter 2\\
% |cdocspt3.tex|&include file for part 3\\
% |cdocspt4.tex|&include file for part 4\\
% |cdocsdrf.tex|&forwarding file for main file in draft mode\\
% |cdocsfi1.tex|&forwarding file for final version of chapter 1\\
% |cdocsfi2.tex|&forwarding file for final version of chapter 2\\
% \end{tabular}
% \end{center}
% Each of the eight files can be compiled directly by the \LaTeX{} compiler.
%
% %%%%%%%%%%%%%%%%%%%%%%%%%%%%%%%%%%%%%%
% \paragraph{Main File.}
%
% The main file is called |cdocsamp.tex|.
%
% Load the \textsf{childdoc} definitions and
% declare the filename for the main document:
%    \begin{macrocode}
\input{childdoc.def}
\childdocmain{}
%    \end{macrocode}

% Optional override for |\version| flag:
%    \begin{macrocode}
%%\ifchilddoc\else\providecommand{\version}{draft}\fi
%    \end{macrocode}

% Define the default values for the |\version| flag
% (|final| for the main file and |draft| for childs):
%    \begin{macrocode}
\ifchilddoc
\providecommand{\version}{draft}
\else
\providecommand{\version}{final}
\fi
%    \end{macrocode}

% Load the standard document class:
%    \begin{macrocode}
\documentclass[12pt]{article}
%    \end{macrocode}

% Start the document body:
%    \begin{macrocode}
\begin{document}
%    \end{macrocode}

% Declare a title page.
% Print title, part of document being processed and version flag:
%    \begin{macrocode}
\addtocounter{page}{-1}
\begin{center}
{\LARGE\bfseries{}childdoc example\par}
\vspace{1cm}
\ifchilddoc
\ifchilddocmanual part\else chapter\fi:
`\childdocname' of `\childdocjob'\par
\else
main document: `\childdocjob'\par
\fi
version: \version\par
\end{center}
\newpage
%    \end{macrocode}

% Manually include selected file,
% otherwise process as usual:
%    \begin{macrocode}
\ifchilddocmanual
\section*{part `\childdocname'}
\input{\childdocname}
\else
%    \end{macrocode}

% Include the two chapters:
%    \begin{macrocode}
\include{cdocsch1}
\include{cdocsch2}
%    \end{macrocode}

% Include the two parts unless only chapters should be displayed:
%    \begin{macrocode}
\ifchilddoc\else
\section{part three}
\input{cdocspt3}
\section{part four}
\input{cdocspt4}
\fi
%    \end{macrocode}

% Process as usual until here:
%    \begin{macrocode}
\fi
%    \end{macrocode}

% End of document body:
%    \begin{macrocode}
\end{document}
%    \end{macrocode}
%\iffalse
%</samplemain>
%\fi
%
% %%%%%%%%%%%%%%%%%%%%%%%%%%%%%%%%%%%%%%
% \paragraph{Chapter Include Files.}
%
% The include files are called |cdocsch1.tex| and |cdocsch2.tex|.
%
%\iffalse
%<*samplechap1|samplechap2>
%\fi

% Optional override for |\version| flag:
%    \begin{macrocode}
%%\providecommand{\version}{final}
%    \end{macrocode}

% Include the main document:
%    \begin{macrocode}
\input{childdoc.def}
\childdocof{cdocsamp}
%    \end{macrocode}

%\iffalse
%</samplechap1|samplechap2>
%\fi
%
%\iffalse
%<*samplechap1>
%\fi
% Some text for chapter 1:
%    \begin{macrocode}
\section{one}
some text in chapter one
%    \end{macrocode}

%\iffalse
%</samplechap1>
%\fi
% Some text for chapter 2:
%\iffalse
%<*samplechap2>
%\fi
%    \begin{macrocode}
\section{two}
more text in chapter two
%    \end{macrocode}

%\iffalse
%</samplechap2>
%\fi
%
% %%%%%%%%%%%%%%%%%%%%%%%%%%%%%%%%%%%%%%
% \paragraph{Part Include Files.}
%
% The include files are called |cdocspt3.tex| and |cdocspt4.tex|.
%
%\iffalse
%<*samplepart3|samplepart4>
%\fi

% Optional override for |\version| flag:
%    \begin{macrocode}
%%\providecommand{\version}{final}
%    \end{macrocode}

% Include the main document:
%    \begin{macrocode}
\input{childdoc.def}
\childdocby{cdocsamp}
%    \end{macrocode}

%\iffalse
%</samplepart3|samplepart4>
%\fi
%
%\iffalse
%<*samplepart3>
%\fi
% Some text for part 3:
%    \begin{macrocode}
some text in part three
%    \end{macrocode}

%\iffalse
%</samplepart3>
%\fi
% Some text for part 4:
%\iffalse
%<*samplepart4>
%\fi
%    \begin{macrocode}
more text in part four
%    \end{macrocode}

%\iffalse
%</samplepart4>
%\fi
%
% %%%%%%%%%%%%%%%%%%%%%%%%%%%%%%%%%%%%%%
% \paragraph{Forwarding for a Complete Draft.}
%
% The following forwarding file |cdocsdrf.tex|
% compiles the main document in draft mode:
%\iffalse
%<*sampledraft>
%\fi
%    \begin{macrocode}
\def\version{draft}
\input{childdoc.def}
\childdocforward{cdocsamp}
%    \end{macrocode}

%\iffalse
%</sampledraft>
%\fi
%
% %%%%%%%%%%%%%%%%%%%%%%%%%%%%%%%%%%%%%%
% \paragraph{Forwarding for Final Version of the Chapters.}
%
% The following forwarding files |cdocsfn1.tex| and |cdocsfn2.tex|
% (with identical content)
% compile the final versions of the child documents
% |cdocsch1.tex| and |cdocsch2.tex|, respectively:
%\iffalse
%<*samplefinal>
%\fi
%    \begin{macrocode}
\def\version{final}
\input{childdoc.def}
\childdocforwardprefix[cdocsamp]{cdocsfn}{cdocsch}
%    \end{macrocode}

%\iffalse
%</samplefinal>
%\fi
%
% %%%%%%%%%%%%%%%%%%%%%%%%%%%%%%%%%%%%%%
% \paragraph{Command Line Processing.}
%
% The following three command lines generate the output files
% |cdocscld|, |cdocscl1| and |cdocscl2|
% which should be identical to
% |cdocsdrf|, |cdocsch1| and |cdocsfn2|, respectively:
% \begin{center}
% \begin{tabular}{l}
% |latex -jobname cdocscld \|\\
% |  "\def\version{draft}\input{childdoc.def}\childdocforward{cdocsamp}"|\\
% |latex -jobname cdocscl1 \|\\
% |  "\input{childdoc.def}\childdocforward[cdocsamp]{cdocsch1}"|\\
% |latex -jobname cdocscl2 \|\\
% |  "\def\version{final}\input{childdoc.def}\childdocforward{cdocsch2}"|
% \end{tabular}
% \end{center}
% Note that the trailing backslash on each first line
% merely continues the input to the second line
% (for convenient cut ant paste).
% Furthermore, the command |latex| can be replaced by any
% of its alternative versions such as |pdflatex|.
%
% %%%%%%%%%%%%%%%%%%%%%%%%%%%%%%%%%%%%%%%%%%%%%%%%%%%%%%%%%%%%%%%%%%%%%%%%%%%%%%
% %%%%%%%%%%%%%%%%%%%%%%%%%%%%%%%%%%%%%%%%%%%%%%%%%%%%%%%%%%%%%%%%%%%%%%%%%%%%%%
% \section{Implementation}
%\iffalse
%<*package>
%\fi
%
% This section describes the definitions file |childdoc.def|.

% The definitions cannot be loaded using |\usepackage| or |\RequirePackage|
% which has a mechanism to prevent loading a style file more than once.
% When loading the definitions by means of |\input|
% multiple instances have to be prevented manually:
%\iffalse
%This code needs to be before the `\ProvidesFile' directive
%which is defined at the beginning of this file.
%Therefore it is also placed there and commented out here.
%</package>
%<*discard>
%\fi
%    \begin{macrocode}
\ifdefined\childdocmain\endinput\fi
%    \end{macrocode}
%\iffalse
%</discard>
%<*package>
%\fi
%
% \macro{\ifchilddoc}
% \macro{\ifchilddocmanual}
% The conditional |\ifchilddoc| tells whether a
% child (true) or main (false) document is being compiled.
% The conditional |\ifchilddocmanual| tells whether
% the |\includeonly| mechanism is used (false) or
% the selection of child files must be performed manually (true).
% The definitions initialise to false:
%    \begin{macrocode}
\newif\ifchilddoc
\newif\ifchilddocmanual
%    \end{macrocode}

% \macro{\childdocname}
% \macro{\childdocjob}
% The macro |\childdocname| stores the name of the main document
% to be compiled. The macro |\childdocjob| stores the name of
% the document on which the \LaTeX{} compiler was originally invoked.
% The content of |\jobname| cannot be compared
% to filenames specified in the source due to different catcodes.
% The following code rescans |\jobname|, stores the result
% in |\childdocname| and saves a copy in |\childdocjob|:
%    \begin{macrocode}
\edef\childdocname{\scantokens\expandafter{\jobname\noexpand}}
\let\childdocjob\childdocname
%    \end{macrocode}

% \macro{\childdocdisable}
% The macro |\childdocdisable| prevents the main file
% from being processed more than once.
% At this stage, the main document command |\childdocmain|
% is assumed to be called once again where it should do nothing.
% Any subsequent call to it should prevent
% a secondary processing of the main document
% It overwrites the forwarding commands
% |\childdocof| and |\childdocforward|
% with empty macros to prevent further inclusions of the main document:
%    \begin{macrocode}
\newcommand{\childdocdisable}
{
  \renewcommand{\childdocmain}[1]{\renewcommand{\childdocmain}[1]{\endinput}}
  \renewcommand{\childdocof}[1]{}
  \renewcommand{\childdocby}[2][]{}
  \renewcommand{\childdocforward}[2][]{}
  \renewcommand{\childdocdisable}{}
}
%    \end{macrocode}

% \macro{\childdocmain}
% The macro |\childdocmain| is to be called at the top of the main file
% with nothing or the main filename (without extension) as argument.
% First, it breaks loops.
% If the argument is not empty and does not match |\childdocname|
% (which is set by the first inclusion of |childdoc.def|),
% |\ifchilddoc| is set to true, |\includeonly| is applied to the child file
% and |\jobname| is set to the main file
% (for proper handling of |.aux| files):
%    \begin{macrocode}
\newcommand{\childdocmain}[1]
{
  \childdocdisable\childdocmain{}
  \if?#1?\else
    \begingroup
      \def\childdoctmp{#1}
      \ifx\childdoctmp\childdocname
        \def\childdoctmp{}
      \else
        \def\childdoctmp
        {
          \childdoctrue
          \includeonly{\childdocname}
          \def\childdocjob{#1}
          \def\jobname{#1}
        }
      \fi
      \expandafter
    \endgroup
    \childdoctmp
  \fi
}
%    \end{macrocode}

% \macro{\childdocof}
% The command |\childdocof| redirects
% compilation to the main file |#1|.
%    \begin{macrocode}
\newcommand{\childdocof}[1]
{
  \childdocdisable
  \childdoctrue
  \includeonly{\childdocname}
  \def\jobname{#1}
  \def\childdocjob{#1}
  \input{#1}
}
%    \end{macrocode}

% \macro{\childdocby}
% The command |\childdocby| ....
%    \begin{macrocode}
\newcommand{\childdocby}[2][]
{
  \childdocdisable
  \childdoctrue
  \childdocmanualtrue
  \if?#1?\else
    \def\jobname{#2}
  \fi
  \def\childdocjob{#2}
  \input{#2}
  \endinput
}
%    \end{macrocode}

% \macro{\childdocforward}
% The command |\childdocforward| redirects
% compilation to the main file or
% (if the optional argument is given) a child file.
% Parameters are set as if the main file
% or a child file starting with |\childdocof| was compiled.
% Then compilation is handed over to the main file:
%    \begin{macrocode}
\newcommand{\childdocforward}[2][]
{
  \begingroup
    \if?#1?
      \def\childdoctmp
      {
        \def\childdocname{#2}
        \def\childdocjob{#2}
        \def\jobname{#2}
        \input{#2}
        \endinput
      }
    \else
      \def\childdoctmp
      {
        \childdocdisable
        \def\childdocname{#2}
        \childdoctrue
        \includeonly{#2}
        \def\childdocjob{#1}
        \def\jobname{#1}
        \input{#1}
        \endinput
      }
    \fi
    \expandafter
  \endgroup
  \childdoctmp
}
%    \end{macrocode}

% \macro{\childdocforwardprefix}
% The command |\childdocforwardprefix| redirects
% compilation to the main or a child file by means of a pattern.
% The prefix |#1| in the current filename is replaced by |#2|
% and the suffix of the current filename is kept
% (it is assumed that the filename does not contain the substring `|~~~|'
% which is used as a delimiter).
% Compilation is handed over to the new file by |\childdocforward|:
%    \begin{macrocode}
\newcommand{\childdocforwardprefix}[3][]
{
  \begingroup
    \def\childdocextract #2##1~~~{\def\childdoctmp{\childdocforward[#1]{#3##1}}}
    \expandafter\childdocextract\childdocname~~~
    \expandafter
  \endgroup
  \childdoctmp
}
%    \end{macrocode}

% \macro{\childdoc}
% The deprecated macro |\childdoc| is a legacy version of |\childdocmain|:
%    \begin{macrocode}
\newcommand{\childdoc}{\childdocmain}
%    \end{macrocode}

% \macro{\childdocredirect}
% The deprecated macro |\childdocredirect| is a legacy version
% of |\childdocforward| and |\childdocforwardprefix|:
%    \begin{macrocode}
\newcommand{\childdocredirect}[2][]
{
  \begingroup
    \if?#1?
      \def\childdoctmp{\childdocforward{#2}}
    \else
      \def\childdoctmp{\childdocforwardprefix{#1}{#2}}
    \fi
    \expandafter
  \endgroup
  \childdoctmp
}
%    \end{macrocode}

%\iffalse
%</package>
%\fi
%
\endinput

\childdocforward{cdocsamp}
%    \end{macrocode}

%\iffalse
%</sampledraft>
%\fi
%
% %%%%%%%%%%%%%%%%%%%%%%%%%%%%%%%%%%%%%%
% \paragraph{Forwarding for Final Version of the Chapters.}
%
% The following forwarding files |cdocsfn1.tex| and |cdocsfn2.tex|
% (with identical content)
% compile the final versions of the child documents
% |cdocsch1.tex| and |cdocsch2.tex|, respectively:
%\iffalse
%<*samplefinal>
%\fi
%    \begin{macrocode}
\def\version{final}
% \iffalse
%
% childdoc.dtx Copyright (C) 2017-2018 Niklas Beisert
%
% This work may be distributed and/or modified under the
% conditions of the LaTeX Project Public License, either version 1.3
% of this license or (at your option) any later version.
% The latest version of this license is in
%   http://www.latex-project.org/lppl.txt
% and version 1.3 or later is part of all distributions of LaTeX
% version 2005/12/01 or later.
%
% This work has the LPPL maintenance status `maintained'.
%
% The Current Maintainer of this work is Niklas Beisert.
%
% This work consists of the files childdoc.dtx and childdoc.ins
% and the derived files childdoc.def and cdocsamp.tex with
% cdocsch1.tex, cdocsch2.tex, cdocsdrf.tex, cdocsfn1.tex, cdocsfn2.tex.
%
%<package>\ifdefined\childdocmain\endinput\fi
%<package>\ProvidesFile{childdoc.def}[2018/12/30 v2.0 child document driver]
%<samplemain>\ProvidesFile{cdocsamp.tex}[2018/12/30 v2.0 sample for childdoc]
%<*driver>
%\ProvidesFile{childdoc.drv}[2018/12/30 v2.0 childdoc reference manual file]
\PassOptionsToClass{10pt,a4paper}{article}
\documentclass{ltxdoc}

\usepackage[margin=35mm]{geometry}
\usepackage{hyperref}
\usepackage{hyperxmp}
\usepackage[usenames]{color}

\hypersetup{colorlinks=true}
\hypersetup{pdfstartview=FitH}
\hypersetup{pdfpagemode=UseNone}
\hypersetup{pdfsource={}}
\hypersetup{pdflang={en-UK}}
\hypersetup{pdfcopyright={Copyright 2017-2018 Niklas Beisert.
  This work may be distributed and/or modified under the
  conditions of the LaTeX Project Public License, either version 1.3
  of this license or (at your option) any later version.}}
\hypersetup{pdflicenseurl={http://www.latex-project.org/lppl.txt}}
\hypersetup{pdfcontactaddress={ETH Zurich, ITP, HIT K,
  Wolfgang-Pauli-Strasse 27}}
\hypersetup{pdfcontactpostcode={8093}}
\hypersetup{pdfcontactcity={Zurich}}
\hypersetup{pdfcontactcountry={Switzerland}}
\hypersetup{pdfcontactemail={nbeisert@itp.phys.ethz.ch}}
\hypersetup{pdfcontacturl={http://people.phys.ethz.ch/\xmptilde nbeisert/}}

\newcommand{\secref}[1]{\hyperref[#1]{section \ref*{#1}}}

\parskip1ex
\parindent0pt
\let\olditemize\itemize
\def\itemize{\olditemize\parskip0pt}

\begin{document}

\title{The \textsf{childdoc} Package}
\hypersetup{pdftitle={The childdoc Package}}
\author{Niklas Beisert\\[2ex]
  Institut f\"ur Theoretische Physik\\
  Eidgen\"ossische Technische Hochschule Z\"urich\\
  Wolfgang-Pauli-Strasse 27, 8093 Z\"urich, Switzerland\\[1ex]
  \href{mailto:nbeisert@itp.phys.ethz.ch}
  {\texttt{nbeisert@itp.phys.ethz.ch}}}
\hypersetup{pdfauthor={Niklas Beisert}}
\hypersetup{pdfsubject={Manual for the LaTeX2e Package childdoc}}
\date{30 December 2018, \textsf{v2.0}}
\maketitle

\begin{abstract}\noindent
\textsf{childdoc} is a \LaTeXe{} package
that enables the direct compilation
of document sections included by |\include|
to individual files.
\end{abstract}

\begingroup
\parskip0ex
\tableofcontents
\endgroup

%%%%%%%%%%%%%%%%%%%%%%%%%%%%%%%%%%%%%%%%%%%%%%%%%%%%%%%%%%%%%%%%%%%%%%%%%%%%%%%%
%%%%%%%%%%%%%%%%%%%%%%%%%%%%%%%%%%%%%%%%%%%%%%%%%%%%%%%%%%%%%%%%%%%%%%%%%%%%%%%%
\section{Introduction}

\LaTeX{} provides a mechanism to structure a large document (such as a book)
into a main file and several child files (containing the chapters)
using the |\include| command.
This mechanism is beneficial for documents
which span hundreds of pages in order to
make the source file(s) more manageable.
Moreover, compilation can be restricted to
selected child files by means of the |\includeonly| command.
The latter feature can be used to reduce the compilation time while editing
(this was significantly more useful in the earlier days of \LaTeX{})
or to generate a smaller document which is easier to navigate.
Another application of |\includeonly| is to generate
documents consisting of selected parts of the complete document.

However, there are a few drawbacks of the plain |\include| mechanism:
\begin{itemize}
\item
The child files cannot be compiled on their own,
they can only be compiled via the main file.
A naive editing environment
(such as a text editor with an option
to have the current file processed by \LaTeX)
may require one to switch to the main file before compiling;
attempting to compile the child file produces errors.
\item
The main file must be modified (each time)
to adjust the |\includeonly| command
to the present needs. This easily leaves the main file in a messy state.
\item
The generated document will always carry the filename
of the main document. This is inconvenient if
several child files are to be compiled and
to be kept for distribution.
\end{itemize}

The present package provides a simple interface
to make child files individually compilable by \LaTeX{}.
Compiling a child file then has the same effect as compiling
the main file with an |\includeonly| command
to select the appropriate child.
Moreover the generated document will carry the name of the child
rather than the main file.
This resolves all three above issues.

This feature is meant to make the editing of books,
thesis documents and lecture notes somewhat more convenient.
However, the package can also be used efficiently for
composing a series of documents (such as exercise sheets)
which are typically distributed individually.
It then assists the author in generating the individual documents
(potentially in different versions)
as well as a document containing the collected series.
Another application is in developing style files
or other kinds of included material
where compilation of the style file could redirect
to a sample or test file.

%%%%%%%%%%%%%%%%%%%%%%%%%%%%%%%%%%%%%%%%%%%%%%%%%%%%%%%%%%%%%%%%%%%%%%%%%%%%%%%%
%%%%%%%%%%%%%%%%%%%%%%%%%%%%%%%%%%%%%%%%%%%%%%%%%%%%%%%%%%%%%%%%%%%%%%%%%%%%%%%%
\section{Usage}

First of all, the package \textsf{childdoc} is \emph{not} a standard
\LaTeXe{} |.sty| style file! Therefore it needs to be invoked in
a non-standard way.

%%%%%%%%%%%%%%%%%%%%%%%%%%%%%%%%%%%%%%%%%%%%%%%%%%%%%%%%%%%%%%%%%%%%%%%%%%%%%%%%
\subsection{Included Files}
\label{sec:include}

%%%%%%%%%%%%%%%%%%%%%%%%%%%%%%%%%%%%%%%%
\DescribeMacro{\childdocmain}
To use the package, add the commands
\begin{center}
\begin{tabular}{l}
|\input{childdoc.def}|\\
|\childdocmain{}|\\
\end{tabular}
\end{center}
at the very top of the main \LaTeX{} file,
in particular \emph{before} the |\documentclass| statement!
The argument of |\childdocmain| should be left empty
(but it must be present).

%%%%%%%%%%%%%%%%%%%%%%%%%%%%%%%%%%%%%%%%
\DescribeMacro{\childdocof}
Furthermore, add the commands
\begin{center}
\begin{tabular}{l}
|\input{childdoc.def}|\\
|\childdocof{|\textit{main}|}|\\
\end{tabular}
\end{center}
at the top of every child file \textit{child}
which is included by |\include{|\textit{child}|}|
from within the main file
(or at least for those files to be compiled individually).
The argument \textit{main} must be the filename of the main file.

There are a couple of
considerations in setting up the main and child documents:

%%%%%%%%%%%%%%%%%%%%%%%%%%%%%%%%%%%%%%%%
\paragraph{Restrictions.}

Please note the following restrictions:
\begin{itemize}
\item
|\childdocmain| must be called with one argument \textit{main}
to ensure compatibility with earlier version of the package.
It must either be empty (|\childdocmain{}|)
or precisely match the filename of the main file in which it is specified.
See \secref{sec:detection} for further information.
\item
The filename \textit{main} must be specified without the |.tex| extension.
\item
The filename \textit{main} is case sensitive
(even in case-insensitive file systems)
due to internal string comparison.
\item
The argument \textit{main} should be fully expanded, it cannot be a macro.
\item
Subdirectories and special characters should be avoided in filenames.
\item
The command |\childdocmain{|\textit{main}|}| must be followed by a whitespace.
It should not be followed immediately by another command
or by a comment mark `|%|'.
This is because the \TeX{} parser reads the token immediately following
the argument of |\childdocmain| and puts it
at the beginning of every child section;
however, a white\-space is ignored.
\end{itemize}

%%%%%%%%%%%%%%%%%%%%%%%%%%%%%%%%%%%%%%%%
\paragraph{Content of Main File.}

It is advisable to place all content in the child files included by |\include|.
Any output contained in the main file will appear in all child documents
unless suppressed manually;
it cannot be suppressed automatically by the |\includeonly| directive
and thus should normally be avoided.
A method to include some content in the main file
by means of conditional processing is described in \secref{sec:conditional}.

%%%%%%%%%%%%%%%%%%%%%%%%%%%%%%%%%%%%%%%%
\paragraph{Page Numbering.}

When only a part of the document is compiled,
the appropriate numbering of pages
(as well as other status parameters)
is determined from the |.aux| files.
The latter contain information from previous passes.
However this information needs to propagate through
all intermediate child documents.
Therefore the page numbering in child documents may well
be inconsistent until the complete document is compiled at least once.

A useful (if unconventional) way to always ensure a consistent
page numbering is to restart the numbering in each child document
and denote the pages by `\textit{child}|.|\textit{page}'
where \textit{child} represents the chapter/section number of the child file.
This can be achieved by the command
|\numberwithin{page}{|\textit{child}|}|
of the \textsf{amsmath} package
where \textit{child} can be |chapter| or |section|
depending on the chosen structuring.
Alternatively, one can modify the macro |\thepage| appropriately
and reset the counter |page| at the start of each child file.

%%%%%%%%%%%%%%%%%%%%%%%%%%%%%%%%%%%%%%%%%%%%%%%%%%%%%%%%%%%%%%%%%%%%%%%%%%%%%%%%
\subsection{Conditional Processing}
\label{sec:conditional}

The package provides a mechanism to compile different versions
of a document. To customise the versions further some conditional processing
can come in handy to distinguish which version is being compiled.
The package provides two macros to describe the compilation context:

%%%%%%%%%%%%%%%%%%%%%%%%%%%%%%%%%%%%%%%%
\DescribeMacro{\ifchilddoc}
The conditional |\ifchilddoc| distinguishes between the compilation of
child documents and the main document:
%
\begin{center}
|\ifchilddoc |\textit{child-code}| |[|\||else |\textit{main-code}]| \||fi|
\end{center}

%%%%%%%%%%%%%%%%%%%%%%%%%%%%%%%%%%%%%%%%
\DescribeMacro{\childdocname}
\DescribeMacro{\childdocjob}
The macro |\childdocname| contains the filename (without extension)
of the main or child file being processed.
Note that |\childdocjob| will always contain the name of the main file.

%%%%%%%%%%%%%%%%%%%%%%%%%%%%%%%%%%%%%%%%
\paragraph{Title Page.}

Conditional processing can be used to include a title or banner page
in the main document when proper precautions are taken.
Importantly, the code in the main file should ensure that the page counter
(as well as other status parameters which are stored in the |.aux| files)
takes the same value after the conditional processing.
Otherwise the page numbers may take divergent values
depending on which part is compiled.

For example, a title page could be declared by:
%
\begin{center}
\begin{tabular}{l}
|\ifchilddoc\||else|\\
|\addtocounter{page}{-1}|\\
\textit{code for title page}\\
|\newpage|\\
|\||fi|
\end{tabular}
\end{center}
%
A banner page for the child documents can be generated by:
%
\begin{center}
\begin{tabular}{l}
|\ifchilddoc|\\
|\addtocounter{page}{-1}|\\
\textit{code for banner page}\\
|\newpage|\\
|\||fi|
\end{tabular}
\end{center}
%
Here one could write a message such as:
\begin{center}
|This is the part \childdocname{} of \childdocjob{}.|
\end{center}

%%%%%%%%%%%%%%%%%%%%%%%%%%%%%%%%%%%%%%%%%%%%%%%%%%%%%%%%%%%%%%%%%%%%%%%%%%%%%%%%
\subsection{Flags}
\label{sec:flags}

The package makes it easy to generate different versions
of the main or child documents.
To this end compilation flags can be defined
and assigned different default values.
They will be particularly useful in conjunction
with the forwarding mechanism described in \secref{sec:forward}.

For example, it may be useful to have a flag |\version|
which can be set to |draft| or |final|.
The document source will contain some conditional code
depending on the value of |\version|.
Suppose further, the flag should default to |final| for the main file
and to |draft| for child files
which is a natural assignment for editing the document.
This is achieved by placing the following code
in the preamble of the main document
(below the |\childdocmain| directive):
%
\begin{center}
\begin{tabular}{l}
|\ifchilddoc|\\
|\providecommand{\version}{draft}|\\
|\||else|\\
|\providecommand{\version}{final}|\\
|\||fi|
\end{tabular}
\end{center}
%
The definition by |\providecommand| makes sure
that previous definitions are not overwritten.
Further statements |\providecommand{\version}{...}|
can thus be added before the above code to override it.

For the main file, one might add a line
(between |\childdocmain| and the above block)
%
\begin{center}
|%\ifchilddoc\||else\providecommand{\version}{draft}\||fi|
\end{center}
%
which can be uncommented to produce a draft version.
Likewise one can add a line to the very top of a child file
(above the |\childdocof{|\textit{main}|}| directive)
%
\begin{center}
|%\providecommand{\version}{final}|
\end{center}
%
which can be uncommented to produce the final version of this child document.

%%%%%%%%%%%%%%%%%%%%%%%%%%%%%%%%%%%%%%%%%%%%%%%%%%%%%%%%%%%%%%%%%%%%%%%%%%%%%%%%
\subsection{Forwarding}
\label{sec:forward}

Different versions of the main or child documents
using compilation flags as described in \secref{sec:flags}
can be (permanently) stored in different files
for convenient compilation, viewing and distribution.
To this end, the package defines a command
to pass on compilation to a different file:

%%%%%%%%%%%%%%%%%%%%%%%%%%%%%%%%%%%%%%%%
\DescribeMacro{\childdocforward}
The command |\childdocforward| redirects processing to
another source file:
%
\begin{center}
\begin{tabular}{l}
|\input{childdoc.def}|\\
|\childdocforward[|\textit{main}|]{|\textit{dest}|}|\\
\end{tabular}
\end{center}
%
The argument \textit{dest} is the destination file
(without extension).
It should be the main file or one of the child files.
Note that further \textsf{childdoc} directives
such as |\childdocof| and |\childdocforward|
in the indicated file will be processed in this form.
The optional argument \textit{main}
passes on directly to the main file \textit{main}
while pretending to compile the child \textit{dest}.
This form behaves as if \textit{dest}
issues |\childdocof{|\textit{main}|}| right away,
and no further \textsf{childdoc} directives will be processed.

%%%%%%%%%%%%%%%%%%%%%%%%%%%%%%%%%%%%%%%%
\DescribeMacro{\...prefix}
In the alternative form |\childdocforwardprefix|,
%
\begin{center}
\begin{tabular}{l}
|\input{childdoc.def}|\\
|\childdocforwardprefix[|\textit{main}|]{|\textit{prefix}|}{|\textit{dest}|}|
\end{tabular}
\end{center}
%
the destination file is determined by a pattern
depending on the current file:
To make this work, the current file must be called
`{\textit{prefix}\hspace{0.2em}\textit{suffix}}'
with \textit{prefix} matching precisely the argument.
Processing is then passed on to the file
`{\textit{dest}\hspace{0.2em}\textit{suffix}}'.
Surely, the same effect is achieved by
directly specifying the
argument `{\textit{dest}\hspace{0.2em}\textit{suffix}}'
in the first form.
However, that requires to set up a different file
for each child. With the alternative form of the command
all these files can have exactly the same content
which simplifies setting them up and maintaining them.

For example, the following file |draft.tex|
with a compilation flag |\version| as described in \secref{sec:flags}
compiles the main document as a draft:
%
\begin{center}
\begin{tabular}{l}
|\def\version{draft}|\\
|\input{childdoc.def}|\\
|\childdocforward{|\textit{main}|}|
\end{tabular}
\end{center}
%
Likewise, the following files |final|\textit{nn}|.tex|
compile the final version of the child document
|child|\textit{nn}|.tex|:
%
\begin{center}
\begin{tabular}{l}
|\def\version{final}|\\
|\input{childdoc.def}|\\
|\childdocforwardprefix{final}{child}|
\end{tabular}
\end{center}
%

Note that when several versions of a main file and/or of each child file
are to be generated, it may be convenient to set up a |Makefile| or
shell script to automatise the process.

%%%%%%%%%%%%%%%%%%%%%%%%%%%%%%%%%%%%%%%%%%%%%%%%%%%%%%%%%%%%%%%%%%%%%%%%%%%%%%%%
\subsection{Command Line Processing}
\label{sec:commandline}

The effect of redirection files can also be achieved by invoking
the \LaTeX{} compiler with a more elaborate command line.
Most conveniently this should be done as part
of a shell script or a |Makefile|.

When using \textsf{childdoc} in the main file, the following
command lines effectively perform a redirection
(note that depending on the shell being used,
backslashes may have to be doubled: `|\|' $\to$ `|\\|'):
%
\begin{center}
|... -jobname "|\textit{target}|" |\\|"|[\textit{flags}]%
|\input{childdoc.def}\childdocforward[|\textit{main}|]{|\textit{dest}|}"|
\end{center}
%
Here \textit{target} is the name of the output file,
\textit{main} is the name of the main file
and \textit{dest} is the name of the main or child file to be processed
(all filenames without extensions).
The optional argument \textit{main} can be omitted
if \textit{main} matches \textit{dest}.
Optionally, compilation \textit{flags} can be defined via |\def| commands.
This command line makes the \TeX{} engine believe
it is compiling the file \textit{target}
whose content is specified as the latter parameter.
The provided code then forwards the processing to
\textit{main} or \textit{dest} as described in \secref{sec:forward}.

%%%%%%%%%%%%%%%%%%%%%%%%%%%%%%%%%%%%%%%%%%%%%%%%%%%%%%%%%%%%%%%%%%%%%%%%%%%%%%%%
\subsection{Include by Input}
\label{sec:input}

Including child documents by |\include| has some restrictions by design.
Most notably, the content of a child document always occupies
its own set of pages; pages cannot be shared between child documents.
Usually, this behaviour makes perfect sense
because each child document contain an essential part of the document.
However, in some situations it may be desirable to compose
a document from a collection of parts
without having mandatory page breaks between then.
For this case, the package
provides a mechanism to include parts
by |\input| which can also be processed individually.
However, by construction this mechanism
requires manual handling of the content to be output.

%%%%%%%%%%%%%%%%%%%%%%%%%%%%%%%%%%%%%%%%
\DescribeMacro{\ifchilddocmanual}
The main file should be prepared as usual, see \secref{sec:include}.
However, the document body must make a distinction
between processing of an individual part and of the main document, e.g.:
%
\begin{center}
\begin{tabular}{l}
|\ifchilddocmanual|\\
|\input{\childdocname}|\\
|\||else|\\
\textit{document body with }|\input{|\textit{part}|}|\\
|\||fi|
\end{tabular}
\end{center}
%
The conditional |\ifchilddocmanual| is true whenever
a part to be included by |\input| is being compiled,
and the name of the part is stored in |\childdocname|.

%%%%%%%%%%%%%%%%%%%%%%%%%%%%%%%%%%%%%%%%
\DescribeMacro{\childdocby}
Each part to be included by |\input| should start with:
%
\begin{center}
\begin{tabular}{l}
|\input{childdoc.def}|\\
|\childdocby{|\textit{main}|}|\\
\end{tabular}
\end{center}
%
The directive |\childdocby| is similar to |\childdocof|
described in \secref{sec:include},
but the subsequent selection of content must be done manually.
To that end, both |\ifchilddoc| and |\ifchilddocmanual|
will be true upon processing of a part,
and the name of the part is stored in |\childdocname|.
Note that |\jobname| will be set to the filename of the current part
so that each part receives an individual |.aux| file
that does not interfere with the |.aux| file(s) of the main document.
This behaviour can be altered by the alternative form
|\childdocby[*]{|\textit{main}|}| (with a non-empty optional argument)
which uses the |.aux| file of the main document
by setting |\jobname| to \textit{main}.

%%%%%%%%%%%%%%%%%%%%%%%%%%%%%%%%%%%%%%%%%%%%%%%%%%%%%%%%%%%%%%%%%%%%%%%%%%%%%%%%
\subsection{Driver Development}
\label{sec:driver}

The \textsf{childdoc} mechanism can also be use for the development
of definition files such as \LaTeX{} styles or classes.
This case differs from the above setup with multiple parts
included by |\include| in that no |\includeonly| should be invoked.
This can be achieved by starting the include file
(before |\ProvidesPackage|) with:
%
\begin{center}
\begin{tabular}{l}
|\input{childdoc.def}|\\
|\childdocforward{|\textit{main}|}|\\
\end{tabular}
\end{center}
%
or alternatively with:
%
\begin{center}
\begin{tabular}{l}
|\input{childdoc.def}|\\
|\childdocby{|\textit{main}|}|\\
\end{tabular}
\end{center}
%
Both forms have slightly different effects as described above.
The main file is prepared as usual, see \secref{sec:include}.

%%%%%%%%%%%%%%%%%%%%%%%%%%%%%%%%%%%%%%%%%%%%%%%%%%%%%%%%%%%%%%%%%%%%%%%%%%%%%%%%
\subsection{Legacy Detection}
\label{sec:detection}

The directive |\childdocmain| in the main file can detect
whether the complete document or merely a child is to be compiled
even without using the directive |\childdocof|.
This method is deprecated because it is less robust
and there is no compelling reason to use it;
it is merely provided for backward compatibility
and it may be removed in future versions.

If the detection mechanism is to be used,
it is mandatory to correctly specify
the filename of the main file as the argument of |\childdocmain|:
%
\begin{center}
\begin{tabular}{l}
|\input{childdoc.def}|\\
|\childdocmain{|\textit{main}|}|\\
\end{tabular}
\end{center}
%
If |\jobname| does not match the argument \textit{main} of |\childdocmain|,
it is assumed that |\jobname| points to the child file to be compiled.
When using |\childdocmain| with the main file specified as argument,
it suffices to start a child file
with just |\input{|\textit{main}|}|
without loading of the package and using |\childdocof|.
If instead all processing is done
with the appropriate \textsf{childdoc} directives,
the argument of \textit{main} of |\childdocmain| can be empty.

An alternative version of the command line processing described
in \secref{sec:commandline} using the detection mechanism reads:
%
\begin{center}
|... -jobname "|\textit{target}|" "|[\textit{flags}]%
[|\def\jobname{|\textit{dest}|}|]|\input{|\textit{main}|}"|
\end{center}

%%%%%%%%%%%%%%%%%%%%%%%%%%%%%%%%%%%%%%%%%%%%%%%%%%%%%%%%%%%%%%%%%%%%%%%%%%%%%%%%
\subsection{Manual Code}
\label{sec:manual}

In case one cannot be certain whether the definitions file |childdoc.def|
is installed on the target \TeX{} distribution
and one prefers not to ship it,
it is conceivable to paste a few relevant commands into the sources.

To that end, drop all statements |\input{childdoc.def}|
and perform the replacements as outlined below.
Instead of |\childdocmain{|\textit{main}|}| add the following code
to the top of the main file:
%
\begin{center}
\begin{tabular}{l}
|\||ifdefined\childdocname\endinput\||fi\newif\ifchilddoc|\\
|\edef\childdocname{\scantokens\expandafter{\jobname\noexpand}}|\\
|\def\childdocmain{|\textit{main}|}\||ifx\childdocmain\childdocname\||else|\\
|\childdoctrue\includeonly{\childdocname}\let\jobname\childdocmain\||fi|\\
\end{tabular}
\end{center}
%
Instead of |\childdocof{|\textit{main}|}| just include the main file
at the top of each child file:
%
\begin{center}
|\input{|\textit{main}|}|
\end{center}
%
A simple redirection |\childdocforward{|\textit{dest}|}| is achieved by:
%
\begin{center}
|\def\jobname{|\textit{dest}|}\input{\jobname}|
\end{center}
%
The redirection with prefix
|\childdocforwardprefix[|\textit{prefix}|]{|\textit{dest}|}|
is accomplished by:
%
\begin{center}
\begin{tabular}{l}
|{\edef\jobname{\scantokens\expandafter{\jobname\noexpand}}|\\
|\def\redirectjob |\textit{prefix}|#1~~~{\gdef\jobname{|\textit{dest}|#1}}|\\
|\expandafter\redirectjob\jobname~~~}\input{\jobname}|
\end{tabular}
\end{center}

In an alternative approach,
child documents can be compiled by a specific command line
without additional code or specific definitions:
%
\begin{center}
|... -jobname "|\textit{target}|" "|[\textit{flags}]%
|\includeonly{|\textit{dest}|}\input{|\textit{main}|}"|
\end{center}
%

%%%%%%%%%%%%%%%%%%%%%%%%%%%%%%%%%%%%%%%%%%%%%%%%%%%%%%%%%%%%%%%%%%%%%%%%%%%%%%%%
%%%%%%%%%%%%%%%%%%%%%%%%%%%%%%%%%%%%%%%%%%%%%%%%%%%%%%%%%%%%%%%%%%%%%%%%%%%%%%%%
\section{Information}

%%%%%%%%%%%%%%%%%%%%%%%%%%%%%%%%%%%%%%%%%%%%%%%%%%%%%%%%%%%%%%%%%%%%%%%%%%%%%%%%
\subsection{Copyright}

Copyright \copyright{} 2017--2018 Niklas Beisert

This work may be distributed and/or modified under the
conditions of the \LaTeX{} Project Public License, either version 1.3
of this license or (at your option) any later version.
The latest version of this license is in
  \url{http://www.latex-project.org/lppl.txt}
and version 1.3 or later is part of all distributions of \LaTeX{}
version 2005/12/01 or later.

This work has the LPPL maintenance status `maintained'.

The Current Maintainer of this work is Niklas Beisert.

This work consists of the files |README.txt|, |childdoc.ins| and |childdoc.dtx|
as well as the derived files |childdoc.def|, |cdocsamp.tex|
with |cdocsch1.tex|, |cdocsch2.tex|, |cdocspt3.tex|, |cdocspt4.tex|,
|cdocsdrf.tex|, |cdocsfn1.tex|, |cdocsfn2.tex|
as well as |childdoc.pdf|.

%%%%%%%%%%%%%%%%%%%%%%%%%%%%%%%%%%%%%%%%%%%%%%%%%%%%%%%%%%%%%%%%%%%%%%%%%%%%%%%%
\subsection{Files and Installation}

The package consists of the files:
%
\begin{center}
\begin{tabular}{ll}
    |README.txt|   & readme file \\
    |childdoc.ins| & installation file \\
    |childdoc.dtx| & source file \\
    |childdoc.def| & definition file \\
    |cdocsamp.tex| & sample main file \\
    |cdocsch1.tex| & sample include file \\
    |cdocsch2.tex| & sample include file \\
    |cdocspt3.tex| & sample part file \\
    |cdocspt4.tex| & sample part file \\
    |cdocsdrf.tex| & sample redirection file \\
    |cdocsfn1.tex| & sample redirection file \\
    |cdocsfn2.tex| & sample redirection file \\
    |childdoc.pdf| & manual
\end{tabular}
\end{center}
%
The distribution consists of the files
|README.txt|, |childdoc.ins| and |childdoc.dtx|.
%
\begin{itemize}
\item
Run (pdf)\LaTeX{} on |childdoc.dtx|
to compile the manual |childdoc.pdf| (this file).
\item
Run \LaTeX{} on |childdoc.ins| to create the definitions file |childdoc.def|
and the sample |cdocsamp.tex| with include files
|cdocsch1.tex|, |cdocsch2.tex|, |cdocspt3.tex|, |cdocspt4.tex|,
|cdocsdrf.tex|, |cdocsfn1.tex|, |cdocsfn2.tex|.
Then copy the file |childdoc.def| to an appropriate directory of your \LaTeX{}
distribution, e.g.\ \textit{texmf-root}|/tex/latex/childdoc|.
\end{itemize}

%%%%%%%%%%%%%%%%%%%%%%%%%%%%%%%%%%%%%%%%%%%%%%%%%%%%%%%%%%%%%%%%%%%%%%%%%%%%%%%%
\subsection{Related CTAN Packages}

There are several other packages which offer a similar functionality:
%
\begin{itemize}
\item
The packages
\href{http://ctan.org/pkg/docmute}{\textsf{docmute}},
\href{http://ctan.org/pkg/includex}{\textsf{includex}} and
\href{http://ctan.org/pkg/standalone}{\textsf{standalone}}
provide commands to include only the document body of
a child file thus allowing both files to be compiled individually.
\item
The packages \href{http://ctan.org/pkg/subdocs}{\textsf{subdocs}}
and \href{http://ctan.org/pkg/subfiles}{\textsf{subfiles}}
provide structures in which the main and child documents can be
encapsulated and allowing them to be compiled individually.
The inclusion mechanism is different from the conventional |\include|.
\item
The package \href{http://ctan.org/pkg/combine}{\textsf{combine}}
is an elaborate solution to combine several documents into one.
\end{itemize}
%
See also the CTAN topic \href{http://ctan.org/topic/subdocs}{\textsf{subdocs}}
for further related packages.
The present package differs from the above solutions in that
a document structure constructed with the conventional |\include| mechanism
just needs two extra commands at the top of every file
such that all constituent files can be compiled individually.

%%%%%%%%%%%%%%%%%%%%%%%%%%%%%%%%%%%%%%%%%%%%%%%%%%%%%%%%%%%%%%%%%%%%%%%%%%%%%%%%
%\subsection{Feature Suggestions}
%
%The following is a list of features which may be useful for future
%versions of this package:
%%
%\begin{itemize}
%\item
%\ldots
%\end{itemize}

%%%%%%%%%%%%%%%%%%%%%%%%%%%%%%%%%%%%%%%%%%%%%%%%%%%%%%%%%%%%%%%%%%%%%%%%%%%%%%%%
\subsection{Revision History}

%%%%%%%%%%%%%%%%%%%%%%%%%%%%%%%%%%%%%%%%
\paragraph{v2.0:} 2018/12/30

\begin{itemize}
\item
immediate forward processing
\item
added |\childdocby| mechanism
\item
manual restructured
\end{itemize}

%%%%%%%%%%%%%%%%%%%%%%%%%%%%%%%%%%%%%%%%
\paragraph{v1.6:} 2018/01/17

\begin{itemize}
\item
application for development of include files
\item
corrections to manual
\end{itemize}

%%%%%%%%%%%%%%%%%%%%%%%%%%%%%%%%%%%%%%%%
\paragraph{v1.5:} 2017/05/21

\begin{itemize}
\item
more complete structuring introduced
\item
|\childdocof| introduced
\item
|\childdoc| renamed to |\childdocmain|
\item
|\childredirect| renamed to |\childdocforward| and |\childdocforwardprefix|
and functionality expanded
\end{itemize}

%%%%%%%%%%%%%%%%%%%%%%%%%%%%%%%%%%%%%%%%
\paragraph{v1.0:} 2017/04/27

\begin{itemize}
\item
manual and install package
\item
first version published on CTAN
\end{itemize}

%%%%%%%%%%%%%%%%%%%%%%%%%%%%%%%%%%%%%%%%
\paragraph{v0.6:} 2017/04/26

\begin{itemize}
\item
redirection mechanism added
\end{itemize}

%%%%%%%%%%%%%%%%%%%%%%%%%%%%%%%%%%%%%%%%
\paragraph{v0.5:} 2017/04/26

\begin{itemize}
\item
functionality in definition file
\end{itemize}


%%%%%%%%%%%%%%%%%%%%%%%%%%%%%%%%%%%%%%%%%%%%%%%%%%%%%%%%%%%%%%%%%%%%%%%%%%%%%%%%
%%%%%%%%%%%%%%%%%%%%%%%%%%%%%%%%%%%%%%%%%%%%%%%%%%%%%%%%%%%%%%%%%%%%%%%%%%%%%%%%
%%%%%%%%%%%%%%%%%%%%%%%%%%%%%%%%%%%%%%%%%%%%%%%%%%%%%%%%%%%%%%%%%%%%%%%%%%%%%%%%
\appendix

\settowidth\MacroIndent{\rmfamily\scriptsize 000\ }

 \DocInput{childdoc.dtx}

\end{document}
%</driver>
% \fi
%
% %%%%%%%%%%%%%%%%%%%%%%%%%%%%%%%%%%%%%%%%%%%%%%%%%%%%%%%%%%%%%%%%%%%%%%%%%%%%%%
% %%%%%%%%%%%%%%%%%%%%%%%%%%%%%%%%%%%%%%%%%%%%%%%%%%%%%%%%%%%%%%%%%%%%%%%%%%%%%%
% \section{Sample}
%\iffalse
%<*samplemain>
%\fi
%
% The following presents a sample document
% with two chapters, two parts, a title page,
% a compile flag as well as three forwarding files to set the flag.
% It consists of eight |.tex| files:
% \begin{center}
% \begin{tabular}{ll}
% |cdocsamp.tex|&main file\\
% |cdocsch1.tex|&include file for chapter 1\\
% |cdocsch2.tex|&include file for chapter 2\\
% |cdocspt3.tex|&include file for part 3\\
% |cdocspt4.tex|&include file for part 4\\
% |cdocsdrf.tex|&forwarding file for main file in draft mode\\
% |cdocsfi1.tex|&forwarding file for final version of chapter 1\\
% |cdocsfi2.tex|&forwarding file for final version of chapter 2\\
% \end{tabular}
% \end{center}
% Each of the eight files can be compiled directly by the \LaTeX{} compiler.
%
% %%%%%%%%%%%%%%%%%%%%%%%%%%%%%%%%%%%%%%
% \paragraph{Main File.}
%
% The main file is called |cdocsamp.tex|.
%
% Load the \textsf{childdoc} definitions and
% declare the filename for the main document:
%    \begin{macrocode}
\input{childdoc.def}
\childdocmain{}
%    \end{macrocode}

% Optional override for |\version| flag:
%    \begin{macrocode}
%%\ifchilddoc\else\providecommand{\version}{draft}\fi
%    \end{macrocode}

% Define the default values for the |\version| flag
% (|final| for the main file and |draft| for childs):
%    \begin{macrocode}
\ifchilddoc
\providecommand{\version}{draft}
\else
\providecommand{\version}{final}
\fi
%    \end{macrocode}

% Load the standard document class:
%    \begin{macrocode}
\documentclass[12pt]{article}
%    \end{macrocode}

% Start the document body:
%    \begin{macrocode}
\begin{document}
%    \end{macrocode}

% Declare a title page.
% Print title, part of document being processed and version flag:
%    \begin{macrocode}
\addtocounter{page}{-1}
\begin{center}
{\LARGE\bfseries{}childdoc example\par}
\vspace{1cm}
\ifchilddoc
\ifchilddocmanual part\else chapter\fi:
`\childdocname' of `\childdocjob'\par
\else
main document: `\childdocjob'\par
\fi
version: \version\par
\end{center}
\newpage
%    \end{macrocode}

% Manually include selected file,
% otherwise process as usual:
%    \begin{macrocode}
\ifchilddocmanual
\section*{part `\childdocname'}
\input{\childdocname}
\else
%    \end{macrocode}

% Include the two chapters:
%    \begin{macrocode}
\include{cdocsch1}
\include{cdocsch2}
%    \end{macrocode}

% Include the two parts unless only chapters should be displayed:
%    \begin{macrocode}
\ifchilddoc\else
\section{part three}
\input{cdocspt3}
\section{part four}
\input{cdocspt4}
\fi
%    \end{macrocode}

% Process as usual until here:
%    \begin{macrocode}
\fi
%    \end{macrocode}

% End of document body:
%    \begin{macrocode}
\end{document}
%    \end{macrocode}
%\iffalse
%</samplemain>
%\fi
%
% %%%%%%%%%%%%%%%%%%%%%%%%%%%%%%%%%%%%%%
% \paragraph{Chapter Include Files.}
%
% The include files are called |cdocsch1.tex| and |cdocsch2.tex|.
%
%\iffalse
%<*samplechap1|samplechap2>
%\fi

% Optional override for |\version| flag:
%    \begin{macrocode}
%%\providecommand{\version}{final}
%    \end{macrocode}

% Include the main document:
%    \begin{macrocode}
\input{childdoc.def}
\childdocof{cdocsamp}
%    \end{macrocode}

%\iffalse
%</samplechap1|samplechap2>
%\fi
%
%\iffalse
%<*samplechap1>
%\fi
% Some text for chapter 1:
%    \begin{macrocode}
\section{one}
some text in chapter one
%    \end{macrocode}

%\iffalse
%</samplechap1>
%\fi
% Some text for chapter 2:
%\iffalse
%<*samplechap2>
%\fi
%    \begin{macrocode}
\section{two}
more text in chapter two
%    \end{macrocode}

%\iffalse
%</samplechap2>
%\fi
%
% %%%%%%%%%%%%%%%%%%%%%%%%%%%%%%%%%%%%%%
% \paragraph{Part Include Files.}
%
% The include files are called |cdocspt3.tex| and |cdocspt4.tex|.
%
%\iffalse
%<*samplepart3|samplepart4>
%\fi

% Optional override for |\version| flag:
%    \begin{macrocode}
%%\providecommand{\version}{final}
%    \end{macrocode}

% Include the main document:
%    \begin{macrocode}
\input{childdoc.def}
\childdocby{cdocsamp}
%    \end{macrocode}

%\iffalse
%</samplepart3|samplepart4>
%\fi
%
%\iffalse
%<*samplepart3>
%\fi
% Some text for part 3:
%    \begin{macrocode}
some text in part three
%    \end{macrocode}

%\iffalse
%</samplepart3>
%\fi
% Some text for part 4:
%\iffalse
%<*samplepart4>
%\fi
%    \begin{macrocode}
more text in part four
%    \end{macrocode}

%\iffalse
%</samplepart4>
%\fi
%
% %%%%%%%%%%%%%%%%%%%%%%%%%%%%%%%%%%%%%%
% \paragraph{Forwarding for a Complete Draft.}
%
% The following forwarding file |cdocsdrf.tex|
% compiles the main document in draft mode:
%\iffalse
%<*sampledraft>
%\fi
%    \begin{macrocode}
\def\version{draft}
\input{childdoc.def}
\childdocforward{cdocsamp}
%    \end{macrocode}

%\iffalse
%</sampledraft>
%\fi
%
% %%%%%%%%%%%%%%%%%%%%%%%%%%%%%%%%%%%%%%
% \paragraph{Forwarding for Final Version of the Chapters.}
%
% The following forwarding files |cdocsfn1.tex| and |cdocsfn2.tex|
% (with identical content)
% compile the final versions of the child documents
% |cdocsch1.tex| and |cdocsch2.tex|, respectively:
%\iffalse
%<*samplefinal>
%\fi
%    \begin{macrocode}
\def\version{final}
\input{childdoc.def}
\childdocforwardprefix[cdocsamp]{cdocsfn}{cdocsch}
%    \end{macrocode}

%\iffalse
%</samplefinal>
%\fi
%
% %%%%%%%%%%%%%%%%%%%%%%%%%%%%%%%%%%%%%%
% \paragraph{Command Line Processing.}
%
% The following three command lines generate the output files
% |cdocscld|, |cdocscl1| and |cdocscl2|
% which should be identical to
% |cdocsdrf|, |cdocsch1| and |cdocsfn2|, respectively:
% \begin{center}
% \begin{tabular}{l}
% |latex -jobname cdocscld \|\\
% |  "\def\version{draft}\input{childdoc.def}\childdocforward{cdocsamp}"|\\
% |latex -jobname cdocscl1 \|\\
% |  "\input{childdoc.def}\childdocforward[cdocsamp]{cdocsch1}"|\\
% |latex -jobname cdocscl2 \|\\
% |  "\def\version{final}\input{childdoc.def}\childdocforward{cdocsch2}"|
% \end{tabular}
% \end{center}
% Note that the trailing backslash on each first line
% merely continues the input to the second line
% (for convenient cut ant paste).
% Furthermore, the command |latex| can be replaced by any
% of its alternative versions such as |pdflatex|.
%
% %%%%%%%%%%%%%%%%%%%%%%%%%%%%%%%%%%%%%%%%%%%%%%%%%%%%%%%%%%%%%%%%%%%%%%%%%%%%%%
% %%%%%%%%%%%%%%%%%%%%%%%%%%%%%%%%%%%%%%%%%%%%%%%%%%%%%%%%%%%%%%%%%%%%%%%%%%%%%%
% \section{Implementation}
%\iffalse
%<*package>
%\fi
%
% This section describes the definitions file |childdoc.def|.

% The definitions cannot be loaded using |\usepackage| or |\RequirePackage|
% which has a mechanism to prevent loading a style file more than once.
% When loading the definitions by means of |\input|
% multiple instances have to be prevented manually:
%\iffalse
%This code needs to be before the `\ProvidesFile' directive
%which is defined at the beginning of this file.
%Therefore it is also placed there and commented out here.
%</package>
%<*discard>
%\fi
%    \begin{macrocode}
\ifdefined\childdocmain\endinput\fi
%    \end{macrocode}
%\iffalse
%</discard>
%<*package>
%\fi
%
% \macro{\ifchilddoc}
% \macro{\ifchilddocmanual}
% The conditional |\ifchilddoc| tells whether a
% child (true) or main (false) document is being compiled.
% The conditional |\ifchilddocmanual| tells whether
% the |\includeonly| mechanism is used (false) or
% the selection of child files must be performed manually (true).
% The definitions initialise to false:
%    \begin{macrocode}
\newif\ifchilddoc
\newif\ifchilddocmanual
%    \end{macrocode}

% \macro{\childdocname}
% \macro{\childdocjob}
% The macro |\childdocname| stores the name of the main document
% to be compiled. The macro |\childdocjob| stores the name of
% the document on which the \LaTeX{} compiler was originally invoked.
% The content of |\jobname| cannot be compared
% to filenames specified in the source due to different catcodes.
% The following code rescans |\jobname|, stores the result
% in |\childdocname| and saves a copy in |\childdocjob|:
%    \begin{macrocode}
\edef\childdocname{\scantokens\expandafter{\jobname\noexpand}}
\let\childdocjob\childdocname
%    \end{macrocode}

% \macro{\childdocdisable}
% The macro |\childdocdisable| prevents the main file
% from being processed more than once.
% At this stage, the main document command |\childdocmain|
% is assumed to be called once again where it should do nothing.
% Any subsequent call to it should prevent
% a secondary processing of the main document
% It overwrites the forwarding commands
% |\childdocof| and |\childdocforward|
% with empty macros to prevent further inclusions of the main document:
%    \begin{macrocode}
\newcommand{\childdocdisable}
{
  \renewcommand{\childdocmain}[1]{\renewcommand{\childdocmain}[1]{\endinput}}
  \renewcommand{\childdocof}[1]{}
  \renewcommand{\childdocby}[2][]{}
  \renewcommand{\childdocforward}[2][]{}
  \renewcommand{\childdocdisable}{}
}
%    \end{macrocode}

% \macro{\childdocmain}
% The macro |\childdocmain| is to be called at the top of the main file
% with nothing or the main filename (without extension) as argument.
% First, it breaks loops.
% If the argument is not empty and does not match |\childdocname|
% (which is set by the first inclusion of |childdoc.def|),
% |\ifchilddoc| is set to true, |\includeonly| is applied to the child file
% and |\jobname| is set to the main file
% (for proper handling of |.aux| files):
%    \begin{macrocode}
\newcommand{\childdocmain}[1]
{
  \childdocdisable\childdocmain{}
  \if?#1?\else
    \begingroup
      \def\childdoctmp{#1}
      \ifx\childdoctmp\childdocname
        \def\childdoctmp{}
      \else
        \def\childdoctmp
        {
          \childdoctrue
          \includeonly{\childdocname}
          \def\childdocjob{#1}
          \def\jobname{#1}
        }
      \fi
      \expandafter
    \endgroup
    \childdoctmp
  \fi
}
%    \end{macrocode}

% \macro{\childdocof}
% The command |\childdocof| redirects
% compilation to the main file |#1|.
%    \begin{macrocode}
\newcommand{\childdocof}[1]
{
  \childdocdisable
  \childdoctrue
  \includeonly{\childdocname}
  \def\jobname{#1}
  \def\childdocjob{#1}
  \input{#1}
}
%    \end{macrocode}

% \macro{\childdocby}
% The command |\childdocby| ....
%    \begin{macrocode}
\newcommand{\childdocby}[2][]
{
  \childdocdisable
  \childdoctrue
  \childdocmanualtrue
  \if?#1?\else
    \def\jobname{#2}
  \fi
  \def\childdocjob{#2}
  \input{#2}
  \endinput
}
%    \end{macrocode}

% \macro{\childdocforward}
% The command |\childdocforward| redirects
% compilation to the main file or
% (if the optional argument is given) a child file.
% Parameters are set as if the main file
% or a child file starting with |\childdocof| was compiled.
% Then compilation is handed over to the main file:
%    \begin{macrocode}
\newcommand{\childdocforward}[2][]
{
  \begingroup
    \if?#1?
      \def\childdoctmp
      {
        \def\childdocname{#2}
        \def\childdocjob{#2}
        \def\jobname{#2}
        \input{#2}
        \endinput
      }
    \else
      \def\childdoctmp
      {
        \childdocdisable
        \def\childdocname{#2}
        \childdoctrue
        \includeonly{#2}
        \def\childdocjob{#1}
        \def\jobname{#1}
        \input{#1}
        \endinput
      }
    \fi
    \expandafter
  \endgroup
  \childdoctmp
}
%    \end{macrocode}

% \macro{\childdocforwardprefix}
% The command |\childdocforwardprefix| redirects
% compilation to the main or a child file by means of a pattern.
% The prefix |#1| in the current filename is replaced by |#2|
% and the suffix of the current filename is kept
% (it is assumed that the filename does not contain the substring `|~~~|'
% which is used as a delimiter).
% Compilation is handed over to the new file by |\childdocforward|:
%    \begin{macrocode}
\newcommand{\childdocforwardprefix}[3][]
{
  \begingroup
    \def\childdocextract #2##1~~~{\def\childdoctmp{\childdocforward[#1]{#3##1}}}
    \expandafter\childdocextract\childdocname~~~
    \expandafter
  \endgroup
  \childdoctmp
}
%    \end{macrocode}

% \macro{\childdoc}
% The deprecated macro |\childdoc| is a legacy version of |\childdocmain|:
%    \begin{macrocode}
\newcommand{\childdoc}{\childdocmain}
%    \end{macrocode}

% \macro{\childdocredirect}
% The deprecated macro |\childdocredirect| is a legacy version
% of |\childdocforward| and |\childdocforwardprefix|:
%    \begin{macrocode}
\newcommand{\childdocredirect}[2][]
{
  \begingroup
    \if?#1?
      \def\childdoctmp{\childdocforward{#2}}
    \else
      \def\childdoctmp{\childdocforwardprefix{#1}{#2}}
    \fi
    \expandafter
  \endgroup
  \childdoctmp
}
%    \end{macrocode}

%\iffalse
%</package>
%\fi
%
\endinput

\childdocforwardprefix[cdocsamp]{cdocsfn}{cdocsch}
%    \end{macrocode}

%\iffalse
%</samplefinal>
%\fi
%
% %%%%%%%%%%%%%%%%%%%%%%%%%%%%%%%%%%%%%%
% \paragraph{Command Line Processing.}
%
% The following three command lines generate the output files
% |cdocscld|, |cdocscl1| and |cdocscl2|
% which should be identical to
% |cdocsdrf|, |cdocsch1| and |cdocsfn2|, respectively:
% \begin{center}
% \begin{tabular}{l}
% |latex -jobname cdocscld \|\\
% |  "\def\version{draft}% \iffalse
%
% childdoc.dtx Copyright (C) 2017-2018 Niklas Beisert
%
% This work may be distributed and/or modified under the
% conditions of the LaTeX Project Public License, either version 1.3
% of this license or (at your option) any later version.
% The latest version of this license is in
%   http://www.latex-project.org/lppl.txt
% and version 1.3 or later is part of all distributions of LaTeX
% version 2005/12/01 or later.
%
% This work has the LPPL maintenance status `maintained'.
%
% The Current Maintainer of this work is Niklas Beisert.
%
% This work consists of the files childdoc.dtx and childdoc.ins
% and the derived files childdoc.def and cdocsamp.tex with
% cdocsch1.tex, cdocsch2.tex, cdocsdrf.tex, cdocsfn1.tex, cdocsfn2.tex.
%
%<package>\ifdefined\childdocmain\endinput\fi
%<package>\ProvidesFile{childdoc.def}[2018/12/30 v2.0 child document driver]
%<samplemain>\ProvidesFile{cdocsamp.tex}[2018/12/30 v2.0 sample for childdoc]
%<*driver>
%\ProvidesFile{childdoc.drv}[2018/12/30 v2.0 childdoc reference manual file]
\PassOptionsToClass{10pt,a4paper}{article}
\documentclass{ltxdoc}

\usepackage[margin=35mm]{geometry}
\usepackage{hyperref}
\usepackage{hyperxmp}
\usepackage[usenames]{color}

\hypersetup{colorlinks=true}
\hypersetup{pdfstartview=FitH}
\hypersetup{pdfpagemode=UseNone}
\hypersetup{pdfsource={}}
\hypersetup{pdflang={en-UK}}
\hypersetup{pdfcopyright={Copyright 2017-2018 Niklas Beisert.
  This work may be distributed and/or modified under the
  conditions of the LaTeX Project Public License, either version 1.3
  of this license or (at your option) any later version.}}
\hypersetup{pdflicenseurl={http://www.latex-project.org/lppl.txt}}
\hypersetup{pdfcontactaddress={ETH Zurich, ITP, HIT K,
  Wolfgang-Pauli-Strasse 27}}
\hypersetup{pdfcontactpostcode={8093}}
\hypersetup{pdfcontactcity={Zurich}}
\hypersetup{pdfcontactcountry={Switzerland}}
\hypersetup{pdfcontactemail={nbeisert@itp.phys.ethz.ch}}
\hypersetup{pdfcontacturl={http://people.phys.ethz.ch/\xmptilde nbeisert/}}

\newcommand{\secref}[1]{\hyperref[#1]{section \ref*{#1}}}

\parskip1ex
\parindent0pt
\let\olditemize\itemize
\def\itemize{\olditemize\parskip0pt}

\begin{document}

\title{The \textsf{childdoc} Package}
\hypersetup{pdftitle={The childdoc Package}}
\author{Niklas Beisert\\[2ex]
  Institut f\"ur Theoretische Physik\\
  Eidgen\"ossische Technische Hochschule Z\"urich\\
  Wolfgang-Pauli-Strasse 27, 8093 Z\"urich, Switzerland\\[1ex]
  \href{mailto:nbeisert@itp.phys.ethz.ch}
  {\texttt{nbeisert@itp.phys.ethz.ch}}}
\hypersetup{pdfauthor={Niklas Beisert}}
\hypersetup{pdfsubject={Manual for the LaTeX2e Package childdoc}}
\date{30 December 2018, \textsf{v2.0}}
\maketitle

\begin{abstract}\noindent
\textsf{childdoc} is a \LaTeXe{} package
that enables the direct compilation
of document sections included by |\include|
to individual files.
\end{abstract}

\begingroup
\parskip0ex
\tableofcontents
\endgroup

%%%%%%%%%%%%%%%%%%%%%%%%%%%%%%%%%%%%%%%%%%%%%%%%%%%%%%%%%%%%%%%%%%%%%%%%%%%%%%%%
%%%%%%%%%%%%%%%%%%%%%%%%%%%%%%%%%%%%%%%%%%%%%%%%%%%%%%%%%%%%%%%%%%%%%%%%%%%%%%%%
\section{Introduction}

\LaTeX{} provides a mechanism to structure a large document (such as a book)
into a main file and several child files (containing the chapters)
using the |\include| command.
This mechanism is beneficial for documents
which span hundreds of pages in order to
make the source file(s) more manageable.
Moreover, compilation can be restricted to
selected child files by means of the |\includeonly| command.
The latter feature can be used to reduce the compilation time while editing
(this was significantly more useful in the earlier days of \LaTeX{})
or to generate a smaller document which is easier to navigate.
Another application of |\includeonly| is to generate
documents consisting of selected parts of the complete document.

However, there are a few drawbacks of the plain |\include| mechanism:
\begin{itemize}
\item
The child files cannot be compiled on their own,
they can only be compiled via the main file.
A naive editing environment
(such as a text editor with an option
to have the current file processed by \LaTeX)
may require one to switch to the main file before compiling;
attempting to compile the child file produces errors.
\item
The main file must be modified (each time)
to adjust the |\includeonly| command
to the present needs. This easily leaves the main file in a messy state.
\item
The generated document will always carry the filename
of the main document. This is inconvenient if
several child files are to be compiled and
to be kept for distribution.
\end{itemize}

The present package provides a simple interface
to make child files individually compilable by \LaTeX{}.
Compiling a child file then has the same effect as compiling
the main file with an |\includeonly| command
to select the appropriate child.
Moreover the generated document will carry the name of the child
rather than the main file.
This resolves all three above issues.

This feature is meant to make the editing of books,
thesis documents and lecture notes somewhat more convenient.
However, the package can also be used efficiently for
composing a series of documents (such as exercise sheets)
which are typically distributed individually.
It then assists the author in generating the individual documents
(potentially in different versions)
as well as a document containing the collected series.
Another application is in developing style files
or other kinds of included material
where compilation of the style file could redirect
to a sample or test file.

%%%%%%%%%%%%%%%%%%%%%%%%%%%%%%%%%%%%%%%%%%%%%%%%%%%%%%%%%%%%%%%%%%%%%%%%%%%%%%%%
%%%%%%%%%%%%%%%%%%%%%%%%%%%%%%%%%%%%%%%%%%%%%%%%%%%%%%%%%%%%%%%%%%%%%%%%%%%%%%%%
\section{Usage}

First of all, the package \textsf{childdoc} is \emph{not} a standard
\LaTeXe{} |.sty| style file! Therefore it needs to be invoked in
a non-standard way.

%%%%%%%%%%%%%%%%%%%%%%%%%%%%%%%%%%%%%%%%%%%%%%%%%%%%%%%%%%%%%%%%%%%%%%%%%%%%%%%%
\subsection{Included Files}
\label{sec:include}

%%%%%%%%%%%%%%%%%%%%%%%%%%%%%%%%%%%%%%%%
\DescribeMacro{\childdocmain}
To use the package, add the commands
\begin{center}
\begin{tabular}{l}
|\input{childdoc.def}|\\
|\childdocmain{}|\\
\end{tabular}
\end{center}
at the very top of the main \LaTeX{} file,
in particular \emph{before} the |\documentclass| statement!
The argument of |\childdocmain| should be left empty
(but it must be present).

%%%%%%%%%%%%%%%%%%%%%%%%%%%%%%%%%%%%%%%%
\DescribeMacro{\childdocof}
Furthermore, add the commands
\begin{center}
\begin{tabular}{l}
|\input{childdoc.def}|\\
|\childdocof{|\textit{main}|}|\\
\end{tabular}
\end{center}
at the top of every child file \textit{child}
which is included by |\include{|\textit{child}|}|
from within the main file
(or at least for those files to be compiled individually).
The argument \textit{main} must be the filename of the main file.

There are a couple of
considerations in setting up the main and child documents:

%%%%%%%%%%%%%%%%%%%%%%%%%%%%%%%%%%%%%%%%
\paragraph{Restrictions.}

Please note the following restrictions:
\begin{itemize}
\item
|\childdocmain| must be called with one argument \textit{main}
to ensure compatibility with earlier version of the package.
It must either be empty (|\childdocmain{}|)
or precisely match the filename of the main file in which it is specified.
See \secref{sec:detection} for further information.
\item
The filename \textit{main} must be specified without the |.tex| extension.
\item
The filename \textit{main} is case sensitive
(even in case-insensitive file systems)
due to internal string comparison.
\item
The argument \textit{main} should be fully expanded, it cannot be a macro.
\item
Subdirectories and special characters should be avoided in filenames.
\item
The command |\childdocmain{|\textit{main}|}| must be followed by a whitespace.
It should not be followed immediately by another command
or by a comment mark `|%|'.
This is because the \TeX{} parser reads the token immediately following
the argument of |\childdocmain| and puts it
at the beginning of every child section;
however, a white\-space is ignored.
\end{itemize}

%%%%%%%%%%%%%%%%%%%%%%%%%%%%%%%%%%%%%%%%
\paragraph{Content of Main File.}

It is advisable to place all content in the child files included by |\include|.
Any output contained in the main file will appear in all child documents
unless suppressed manually;
it cannot be suppressed automatically by the |\includeonly| directive
and thus should normally be avoided.
A method to include some content in the main file
by means of conditional processing is described in \secref{sec:conditional}.

%%%%%%%%%%%%%%%%%%%%%%%%%%%%%%%%%%%%%%%%
\paragraph{Page Numbering.}

When only a part of the document is compiled,
the appropriate numbering of pages
(as well as other status parameters)
is determined from the |.aux| files.
The latter contain information from previous passes.
However this information needs to propagate through
all intermediate child documents.
Therefore the page numbering in child documents may well
be inconsistent until the complete document is compiled at least once.

A useful (if unconventional) way to always ensure a consistent
page numbering is to restart the numbering in each child document
and denote the pages by `\textit{child}|.|\textit{page}'
where \textit{child} represents the chapter/section number of the child file.
This can be achieved by the command
|\numberwithin{page}{|\textit{child}|}|
of the \textsf{amsmath} package
where \textit{child} can be |chapter| or |section|
depending on the chosen structuring.
Alternatively, one can modify the macro |\thepage| appropriately
and reset the counter |page| at the start of each child file.

%%%%%%%%%%%%%%%%%%%%%%%%%%%%%%%%%%%%%%%%%%%%%%%%%%%%%%%%%%%%%%%%%%%%%%%%%%%%%%%%
\subsection{Conditional Processing}
\label{sec:conditional}

The package provides a mechanism to compile different versions
of a document. To customise the versions further some conditional processing
can come in handy to distinguish which version is being compiled.
The package provides two macros to describe the compilation context:

%%%%%%%%%%%%%%%%%%%%%%%%%%%%%%%%%%%%%%%%
\DescribeMacro{\ifchilddoc}
The conditional |\ifchilddoc| distinguishes between the compilation of
child documents and the main document:
%
\begin{center}
|\ifchilddoc |\textit{child-code}| |[|\||else |\textit{main-code}]| \||fi|
\end{center}

%%%%%%%%%%%%%%%%%%%%%%%%%%%%%%%%%%%%%%%%
\DescribeMacro{\childdocname}
\DescribeMacro{\childdocjob}
The macro |\childdocname| contains the filename (without extension)
of the main or child file being processed.
Note that |\childdocjob| will always contain the name of the main file.

%%%%%%%%%%%%%%%%%%%%%%%%%%%%%%%%%%%%%%%%
\paragraph{Title Page.}

Conditional processing can be used to include a title or banner page
in the main document when proper precautions are taken.
Importantly, the code in the main file should ensure that the page counter
(as well as other status parameters which are stored in the |.aux| files)
takes the same value after the conditional processing.
Otherwise the page numbers may take divergent values
depending on which part is compiled.

For example, a title page could be declared by:
%
\begin{center}
\begin{tabular}{l}
|\ifchilddoc\||else|\\
|\addtocounter{page}{-1}|\\
\textit{code for title page}\\
|\newpage|\\
|\||fi|
\end{tabular}
\end{center}
%
A banner page for the child documents can be generated by:
%
\begin{center}
\begin{tabular}{l}
|\ifchilddoc|\\
|\addtocounter{page}{-1}|\\
\textit{code for banner page}\\
|\newpage|\\
|\||fi|
\end{tabular}
\end{center}
%
Here one could write a message such as:
\begin{center}
|This is the part \childdocname{} of \childdocjob{}.|
\end{center}

%%%%%%%%%%%%%%%%%%%%%%%%%%%%%%%%%%%%%%%%%%%%%%%%%%%%%%%%%%%%%%%%%%%%%%%%%%%%%%%%
\subsection{Flags}
\label{sec:flags}

The package makes it easy to generate different versions
of the main or child documents.
To this end compilation flags can be defined
and assigned different default values.
They will be particularly useful in conjunction
with the forwarding mechanism described in \secref{sec:forward}.

For example, it may be useful to have a flag |\version|
which can be set to |draft| or |final|.
The document source will contain some conditional code
depending on the value of |\version|.
Suppose further, the flag should default to |final| for the main file
and to |draft| for child files
which is a natural assignment for editing the document.
This is achieved by placing the following code
in the preamble of the main document
(below the |\childdocmain| directive):
%
\begin{center}
\begin{tabular}{l}
|\ifchilddoc|\\
|\providecommand{\version}{draft}|\\
|\||else|\\
|\providecommand{\version}{final}|\\
|\||fi|
\end{tabular}
\end{center}
%
The definition by |\providecommand| makes sure
that previous definitions are not overwritten.
Further statements |\providecommand{\version}{...}|
can thus be added before the above code to override it.

For the main file, one might add a line
(between |\childdocmain| and the above block)
%
\begin{center}
|%\ifchilddoc\||else\providecommand{\version}{draft}\||fi|
\end{center}
%
which can be uncommented to produce a draft version.
Likewise one can add a line to the very top of a child file
(above the |\childdocof{|\textit{main}|}| directive)
%
\begin{center}
|%\providecommand{\version}{final}|
\end{center}
%
which can be uncommented to produce the final version of this child document.

%%%%%%%%%%%%%%%%%%%%%%%%%%%%%%%%%%%%%%%%%%%%%%%%%%%%%%%%%%%%%%%%%%%%%%%%%%%%%%%%
\subsection{Forwarding}
\label{sec:forward}

Different versions of the main or child documents
using compilation flags as described in \secref{sec:flags}
can be (permanently) stored in different files
for convenient compilation, viewing and distribution.
To this end, the package defines a command
to pass on compilation to a different file:

%%%%%%%%%%%%%%%%%%%%%%%%%%%%%%%%%%%%%%%%
\DescribeMacro{\childdocforward}
The command |\childdocforward| redirects processing to
another source file:
%
\begin{center}
\begin{tabular}{l}
|\input{childdoc.def}|\\
|\childdocforward[|\textit{main}|]{|\textit{dest}|}|\\
\end{tabular}
\end{center}
%
The argument \textit{dest} is the destination file
(without extension).
It should be the main file or one of the child files.
Note that further \textsf{childdoc} directives
such as |\childdocof| and |\childdocforward|
in the indicated file will be processed in this form.
The optional argument \textit{main}
passes on directly to the main file \textit{main}
while pretending to compile the child \textit{dest}.
This form behaves as if \textit{dest}
issues |\childdocof{|\textit{main}|}| right away,
and no further \textsf{childdoc} directives will be processed.

%%%%%%%%%%%%%%%%%%%%%%%%%%%%%%%%%%%%%%%%
\DescribeMacro{\...prefix}
In the alternative form |\childdocforwardprefix|,
%
\begin{center}
\begin{tabular}{l}
|\input{childdoc.def}|\\
|\childdocforwardprefix[|\textit{main}|]{|\textit{prefix}|}{|\textit{dest}|}|
\end{tabular}
\end{center}
%
the destination file is determined by a pattern
depending on the current file:
To make this work, the current file must be called
`{\textit{prefix}\hspace{0.2em}\textit{suffix}}'
with \textit{prefix} matching precisely the argument.
Processing is then passed on to the file
`{\textit{dest}\hspace{0.2em}\textit{suffix}}'.
Surely, the same effect is achieved by
directly specifying the
argument `{\textit{dest}\hspace{0.2em}\textit{suffix}}'
in the first form.
However, that requires to set up a different file
for each child. With the alternative form of the command
all these files can have exactly the same content
which simplifies setting them up and maintaining them.

For example, the following file |draft.tex|
with a compilation flag |\version| as described in \secref{sec:flags}
compiles the main document as a draft:
%
\begin{center}
\begin{tabular}{l}
|\def\version{draft}|\\
|\input{childdoc.def}|\\
|\childdocforward{|\textit{main}|}|
\end{tabular}
\end{center}
%
Likewise, the following files |final|\textit{nn}|.tex|
compile the final version of the child document
|child|\textit{nn}|.tex|:
%
\begin{center}
\begin{tabular}{l}
|\def\version{final}|\\
|\input{childdoc.def}|\\
|\childdocforwardprefix{final}{child}|
\end{tabular}
\end{center}
%

Note that when several versions of a main file and/or of each child file
are to be generated, it may be convenient to set up a |Makefile| or
shell script to automatise the process.

%%%%%%%%%%%%%%%%%%%%%%%%%%%%%%%%%%%%%%%%%%%%%%%%%%%%%%%%%%%%%%%%%%%%%%%%%%%%%%%%
\subsection{Command Line Processing}
\label{sec:commandline}

The effect of redirection files can also be achieved by invoking
the \LaTeX{} compiler with a more elaborate command line.
Most conveniently this should be done as part
of a shell script or a |Makefile|.

When using \textsf{childdoc} in the main file, the following
command lines effectively perform a redirection
(note that depending on the shell being used,
backslashes may have to be doubled: `|\|' $\to$ `|\\|'):
%
\begin{center}
|... -jobname "|\textit{target}|" |\\|"|[\textit{flags}]%
|\input{childdoc.def}\childdocforward[|\textit{main}|]{|\textit{dest}|}"|
\end{center}
%
Here \textit{target} is the name of the output file,
\textit{main} is the name of the main file
and \textit{dest} is the name of the main or child file to be processed
(all filenames without extensions).
The optional argument \textit{main} can be omitted
if \textit{main} matches \textit{dest}.
Optionally, compilation \textit{flags} can be defined via |\def| commands.
This command line makes the \TeX{} engine believe
it is compiling the file \textit{target}
whose content is specified as the latter parameter.
The provided code then forwards the processing to
\textit{main} or \textit{dest} as described in \secref{sec:forward}.

%%%%%%%%%%%%%%%%%%%%%%%%%%%%%%%%%%%%%%%%%%%%%%%%%%%%%%%%%%%%%%%%%%%%%%%%%%%%%%%%
\subsection{Include by Input}
\label{sec:input}

Including child documents by |\include| has some restrictions by design.
Most notably, the content of a child document always occupies
its own set of pages; pages cannot be shared between child documents.
Usually, this behaviour makes perfect sense
because each child document contain an essential part of the document.
However, in some situations it may be desirable to compose
a document from a collection of parts
without having mandatory page breaks between then.
For this case, the package
provides a mechanism to include parts
by |\input| which can also be processed individually.
However, by construction this mechanism
requires manual handling of the content to be output.

%%%%%%%%%%%%%%%%%%%%%%%%%%%%%%%%%%%%%%%%
\DescribeMacro{\ifchilddocmanual}
The main file should be prepared as usual, see \secref{sec:include}.
However, the document body must make a distinction
between processing of an individual part and of the main document, e.g.:
%
\begin{center}
\begin{tabular}{l}
|\ifchilddocmanual|\\
|\input{\childdocname}|\\
|\||else|\\
\textit{document body with }|\input{|\textit{part}|}|\\
|\||fi|
\end{tabular}
\end{center}
%
The conditional |\ifchilddocmanual| is true whenever
a part to be included by |\input| is being compiled,
and the name of the part is stored in |\childdocname|.

%%%%%%%%%%%%%%%%%%%%%%%%%%%%%%%%%%%%%%%%
\DescribeMacro{\childdocby}
Each part to be included by |\input| should start with:
%
\begin{center}
\begin{tabular}{l}
|\input{childdoc.def}|\\
|\childdocby{|\textit{main}|}|\\
\end{tabular}
\end{center}
%
The directive |\childdocby| is similar to |\childdocof|
described in \secref{sec:include},
but the subsequent selection of content must be done manually.
To that end, both |\ifchilddoc| and |\ifchilddocmanual|
will be true upon processing of a part,
and the name of the part is stored in |\childdocname|.
Note that |\jobname| will be set to the filename of the current part
so that each part receives an individual |.aux| file
that does not interfere with the |.aux| file(s) of the main document.
This behaviour can be altered by the alternative form
|\childdocby[*]{|\textit{main}|}| (with a non-empty optional argument)
which uses the |.aux| file of the main document
by setting |\jobname| to \textit{main}.

%%%%%%%%%%%%%%%%%%%%%%%%%%%%%%%%%%%%%%%%%%%%%%%%%%%%%%%%%%%%%%%%%%%%%%%%%%%%%%%%
\subsection{Driver Development}
\label{sec:driver}

The \textsf{childdoc} mechanism can also be use for the development
of definition files such as \LaTeX{} styles or classes.
This case differs from the above setup with multiple parts
included by |\include| in that no |\includeonly| should be invoked.
This can be achieved by starting the include file
(before |\ProvidesPackage|) with:
%
\begin{center}
\begin{tabular}{l}
|\input{childdoc.def}|\\
|\childdocforward{|\textit{main}|}|\\
\end{tabular}
\end{center}
%
or alternatively with:
%
\begin{center}
\begin{tabular}{l}
|\input{childdoc.def}|\\
|\childdocby{|\textit{main}|}|\\
\end{tabular}
\end{center}
%
Both forms have slightly different effects as described above.
The main file is prepared as usual, see \secref{sec:include}.

%%%%%%%%%%%%%%%%%%%%%%%%%%%%%%%%%%%%%%%%%%%%%%%%%%%%%%%%%%%%%%%%%%%%%%%%%%%%%%%%
\subsection{Legacy Detection}
\label{sec:detection}

The directive |\childdocmain| in the main file can detect
whether the complete document or merely a child is to be compiled
even without using the directive |\childdocof|.
This method is deprecated because it is less robust
and there is no compelling reason to use it;
it is merely provided for backward compatibility
and it may be removed in future versions.

If the detection mechanism is to be used,
it is mandatory to correctly specify
the filename of the main file as the argument of |\childdocmain|:
%
\begin{center}
\begin{tabular}{l}
|\input{childdoc.def}|\\
|\childdocmain{|\textit{main}|}|\\
\end{tabular}
\end{center}
%
If |\jobname| does not match the argument \textit{main} of |\childdocmain|,
it is assumed that |\jobname| points to the child file to be compiled.
When using |\childdocmain| with the main file specified as argument,
it suffices to start a child file
with just |\input{|\textit{main}|}|
without loading of the package and using |\childdocof|.
If instead all processing is done
with the appropriate \textsf{childdoc} directives,
the argument of \textit{main} of |\childdocmain| can be empty.

An alternative version of the command line processing described
in \secref{sec:commandline} using the detection mechanism reads:
%
\begin{center}
|... -jobname "|\textit{target}|" "|[\textit{flags}]%
[|\def\jobname{|\textit{dest}|}|]|\input{|\textit{main}|}"|
\end{center}

%%%%%%%%%%%%%%%%%%%%%%%%%%%%%%%%%%%%%%%%%%%%%%%%%%%%%%%%%%%%%%%%%%%%%%%%%%%%%%%%
\subsection{Manual Code}
\label{sec:manual}

In case one cannot be certain whether the definitions file |childdoc.def|
is installed on the target \TeX{} distribution
and one prefers not to ship it,
it is conceivable to paste a few relevant commands into the sources.

To that end, drop all statements |\input{childdoc.def}|
and perform the replacements as outlined below.
Instead of |\childdocmain{|\textit{main}|}| add the following code
to the top of the main file:
%
\begin{center}
\begin{tabular}{l}
|\||ifdefined\childdocname\endinput\||fi\newif\ifchilddoc|\\
|\edef\childdocname{\scantokens\expandafter{\jobname\noexpand}}|\\
|\def\childdocmain{|\textit{main}|}\||ifx\childdocmain\childdocname\||else|\\
|\childdoctrue\includeonly{\childdocname}\let\jobname\childdocmain\||fi|\\
\end{tabular}
\end{center}
%
Instead of |\childdocof{|\textit{main}|}| just include the main file
at the top of each child file:
%
\begin{center}
|\input{|\textit{main}|}|
\end{center}
%
A simple redirection |\childdocforward{|\textit{dest}|}| is achieved by:
%
\begin{center}
|\def\jobname{|\textit{dest}|}\input{\jobname}|
\end{center}
%
The redirection with prefix
|\childdocforwardprefix[|\textit{prefix}|]{|\textit{dest}|}|
is accomplished by:
%
\begin{center}
\begin{tabular}{l}
|{\edef\jobname{\scantokens\expandafter{\jobname\noexpand}}|\\
|\def\redirectjob |\textit{prefix}|#1~~~{\gdef\jobname{|\textit{dest}|#1}}|\\
|\expandafter\redirectjob\jobname~~~}\input{\jobname}|
\end{tabular}
\end{center}

In an alternative approach,
child documents can be compiled by a specific command line
without additional code or specific definitions:
%
\begin{center}
|... -jobname "|\textit{target}|" "|[\textit{flags}]%
|\includeonly{|\textit{dest}|}\input{|\textit{main}|}"|
\end{center}
%

%%%%%%%%%%%%%%%%%%%%%%%%%%%%%%%%%%%%%%%%%%%%%%%%%%%%%%%%%%%%%%%%%%%%%%%%%%%%%%%%
%%%%%%%%%%%%%%%%%%%%%%%%%%%%%%%%%%%%%%%%%%%%%%%%%%%%%%%%%%%%%%%%%%%%%%%%%%%%%%%%
\section{Information}

%%%%%%%%%%%%%%%%%%%%%%%%%%%%%%%%%%%%%%%%%%%%%%%%%%%%%%%%%%%%%%%%%%%%%%%%%%%%%%%%
\subsection{Copyright}

Copyright \copyright{} 2017--2018 Niklas Beisert

This work may be distributed and/or modified under the
conditions of the \LaTeX{} Project Public License, either version 1.3
of this license or (at your option) any later version.
The latest version of this license is in
  \url{http://www.latex-project.org/lppl.txt}
and version 1.3 or later is part of all distributions of \LaTeX{}
version 2005/12/01 or later.

This work has the LPPL maintenance status `maintained'.

The Current Maintainer of this work is Niklas Beisert.

This work consists of the files |README.txt|, |childdoc.ins| and |childdoc.dtx|
as well as the derived files |childdoc.def|, |cdocsamp.tex|
with |cdocsch1.tex|, |cdocsch2.tex|, |cdocspt3.tex|, |cdocspt4.tex|,
|cdocsdrf.tex|, |cdocsfn1.tex|, |cdocsfn2.tex|
as well as |childdoc.pdf|.

%%%%%%%%%%%%%%%%%%%%%%%%%%%%%%%%%%%%%%%%%%%%%%%%%%%%%%%%%%%%%%%%%%%%%%%%%%%%%%%%
\subsection{Files and Installation}

The package consists of the files:
%
\begin{center}
\begin{tabular}{ll}
    |README.txt|   & readme file \\
    |childdoc.ins| & installation file \\
    |childdoc.dtx| & source file \\
    |childdoc.def| & definition file \\
    |cdocsamp.tex| & sample main file \\
    |cdocsch1.tex| & sample include file \\
    |cdocsch2.tex| & sample include file \\
    |cdocspt3.tex| & sample part file \\
    |cdocspt4.tex| & sample part file \\
    |cdocsdrf.tex| & sample redirection file \\
    |cdocsfn1.tex| & sample redirection file \\
    |cdocsfn2.tex| & sample redirection file \\
    |childdoc.pdf| & manual
\end{tabular}
\end{center}
%
The distribution consists of the files
|README.txt|, |childdoc.ins| and |childdoc.dtx|.
%
\begin{itemize}
\item
Run (pdf)\LaTeX{} on |childdoc.dtx|
to compile the manual |childdoc.pdf| (this file).
\item
Run \LaTeX{} on |childdoc.ins| to create the definitions file |childdoc.def|
and the sample |cdocsamp.tex| with include files
|cdocsch1.tex|, |cdocsch2.tex|, |cdocspt3.tex|, |cdocspt4.tex|,
|cdocsdrf.tex|, |cdocsfn1.tex|, |cdocsfn2.tex|.
Then copy the file |childdoc.def| to an appropriate directory of your \LaTeX{}
distribution, e.g.\ \textit{texmf-root}|/tex/latex/childdoc|.
\end{itemize}

%%%%%%%%%%%%%%%%%%%%%%%%%%%%%%%%%%%%%%%%%%%%%%%%%%%%%%%%%%%%%%%%%%%%%%%%%%%%%%%%
\subsection{Related CTAN Packages}

There are several other packages which offer a similar functionality:
%
\begin{itemize}
\item
The packages
\href{http://ctan.org/pkg/docmute}{\textsf{docmute}},
\href{http://ctan.org/pkg/includex}{\textsf{includex}} and
\href{http://ctan.org/pkg/standalone}{\textsf{standalone}}
provide commands to include only the document body of
a child file thus allowing both files to be compiled individually.
\item
The packages \href{http://ctan.org/pkg/subdocs}{\textsf{subdocs}}
and \href{http://ctan.org/pkg/subfiles}{\textsf{subfiles}}
provide structures in which the main and child documents can be
encapsulated and allowing them to be compiled individually.
The inclusion mechanism is different from the conventional |\include|.
\item
The package \href{http://ctan.org/pkg/combine}{\textsf{combine}}
is an elaborate solution to combine several documents into one.
\end{itemize}
%
See also the CTAN topic \href{http://ctan.org/topic/subdocs}{\textsf{subdocs}}
for further related packages.
The present package differs from the above solutions in that
a document structure constructed with the conventional |\include| mechanism
just needs two extra commands at the top of every file
such that all constituent files can be compiled individually.

%%%%%%%%%%%%%%%%%%%%%%%%%%%%%%%%%%%%%%%%%%%%%%%%%%%%%%%%%%%%%%%%%%%%%%%%%%%%%%%%
%\subsection{Feature Suggestions}
%
%The following is a list of features which may be useful for future
%versions of this package:
%%
%\begin{itemize}
%\item
%\ldots
%\end{itemize}

%%%%%%%%%%%%%%%%%%%%%%%%%%%%%%%%%%%%%%%%%%%%%%%%%%%%%%%%%%%%%%%%%%%%%%%%%%%%%%%%
\subsection{Revision History}

%%%%%%%%%%%%%%%%%%%%%%%%%%%%%%%%%%%%%%%%
\paragraph{v2.0:} 2018/12/30

\begin{itemize}
\item
immediate forward processing
\item
added |\childdocby| mechanism
\item
manual restructured
\end{itemize}

%%%%%%%%%%%%%%%%%%%%%%%%%%%%%%%%%%%%%%%%
\paragraph{v1.6:} 2018/01/17

\begin{itemize}
\item
application for development of include files
\item
corrections to manual
\end{itemize}

%%%%%%%%%%%%%%%%%%%%%%%%%%%%%%%%%%%%%%%%
\paragraph{v1.5:} 2017/05/21

\begin{itemize}
\item
more complete structuring introduced
\item
|\childdocof| introduced
\item
|\childdoc| renamed to |\childdocmain|
\item
|\childredirect| renamed to |\childdocforward| and |\childdocforwardprefix|
and functionality expanded
\end{itemize}

%%%%%%%%%%%%%%%%%%%%%%%%%%%%%%%%%%%%%%%%
\paragraph{v1.0:} 2017/04/27

\begin{itemize}
\item
manual and install package
\item
first version published on CTAN
\end{itemize}

%%%%%%%%%%%%%%%%%%%%%%%%%%%%%%%%%%%%%%%%
\paragraph{v0.6:} 2017/04/26

\begin{itemize}
\item
redirection mechanism added
\end{itemize}

%%%%%%%%%%%%%%%%%%%%%%%%%%%%%%%%%%%%%%%%
\paragraph{v0.5:} 2017/04/26

\begin{itemize}
\item
functionality in definition file
\end{itemize}


%%%%%%%%%%%%%%%%%%%%%%%%%%%%%%%%%%%%%%%%%%%%%%%%%%%%%%%%%%%%%%%%%%%%%%%%%%%%%%%%
%%%%%%%%%%%%%%%%%%%%%%%%%%%%%%%%%%%%%%%%%%%%%%%%%%%%%%%%%%%%%%%%%%%%%%%%%%%%%%%%
%%%%%%%%%%%%%%%%%%%%%%%%%%%%%%%%%%%%%%%%%%%%%%%%%%%%%%%%%%%%%%%%%%%%%%%%%%%%%%%%
\appendix

\settowidth\MacroIndent{\rmfamily\scriptsize 000\ }

 \DocInput{childdoc.dtx}

\end{document}
%</driver>
% \fi
%
% %%%%%%%%%%%%%%%%%%%%%%%%%%%%%%%%%%%%%%%%%%%%%%%%%%%%%%%%%%%%%%%%%%%%%%%%%%%%%%
% %%%%%%%%%%%%%%%%%%%%%%%%%%%%%%%%%%%%%%%%%%%%%%%%%%%%%%%%%%%%%%%%%%%%%%%%%%%%%%
% \section{Sample}
%\iffalse
%<*samplemain>
%\fi
%
% The following presents a sample document
% with two chapters, two parts, a title page,
% a compile flag as well as three forwarding files to set the flag.
% It consists of eight |.tex| files:
% \begin{center}
% \begin{tabular}{ll}
% |cdocsamp.tex|&main file\\
% |cdocsch1.tex|&include file for chapter 1\\
% |cdocsch2.tex|&include file for chapter 2\\
% |cdocspt3.tex|&include file for part 3\\
% |cdocspt4.tex|&include file for part 4\\
% |cdocsdrf.tex|&forwarding file for main file in draft mode\\
% |cdocsfi1.tex|&forwarding file for final version of chapter 1\\
% |cdocsfi2.tex|&forwarding file for final version of chapter 2\\
% \end{tabular}
% \end{center}
% Each of the eight files can be compiled directly by the \LaTeX{} compiler.
%
% %%%%%%%%%%%%%%%%%%%%%%%%%%%%%%%%%%%%%%
% \paragraph{Main File.}
%
% The main file is called |cdocsamp.tex|.
%
% Load the \textsf{childdoc} definitions and
% declare the filename for the main document:
%    \begin{macrocode}
\input{childdoc.def}
\childdocmain{}
%    \end{macrocode}

% Optional override for |\version| flag:
%    \begin{macrocode}
%%\ifchilddoc\else\providecommand{\version}{draft}\fi
%    \end{macrocode}

% Define the default values for the |\version| flag
% (|final| for the main file and |draft| for childs):
%    \begin{macrocode}
\ifchilddoc
\providecommand{\version}{draft}
\else
\providecommand{\version}{final}
\fi
%    \end{macrocode}

% Load the standard document class:
%    \begin{macrocode}
\documentclass[12pt]{article}
%    \end{macrocode}

% Start the document body:
%    \begin{macrocode}
\begin{document}
%    \end{macrocode}

% Declare a title page.
% Print title, part of document being processed and version flag:
%    \begin{macrocode}
\addtocounter{page}{-1}
\begin{center}
{\LARGE\bfseries{}childdoc example\par}
\vspace{1cm}
\ifchilddoc
\ifchilddocmanual part\else chapter\fi:
`\childdocname' of `\childdocjob'\par
\else
main document: `\childdocjob'\par
\fi
version: \version\par
\end{center}
\newpage
%    \end{macrocode}

% Manually include selected file,
% otherwise process as usual:
%    \begin{macrocode}
\ifchilddocmanual
\section*{part `\childdocname'}
\input{\childdocname}
\else
%    \end{macrocode}

% Include the two chapters:
%    \begin{macrocode}
\include{cdocsch1}
\include{cdocsch2}
%    \end{macrocode}

% Include the two parts unless only chapters should be displayed:
%    \begin{macrocode}
\ifchilddoc\else
\section{part three}
\input{cdocspt3}
\section{part four}
\input{cdocspt4}
\fi
%    \end{macrocode}

% Process as usual until here:
%    \begin{macrocode}
\fi
%    \end{macrocode}

% End of document body:
%    \begin{macrocode}
\end{document}
%    \end{macrocode}
%\iffalse
%</samplemain>
%\fi
%
% %%%%%%%%%%%%%%%%%%%%%%%%%%%%%%%%%%%%%%
% \paragraph{Chapter Include Files.}
%
% The include files are called |cdocsch1.tex| and |cdocsch2.tex|.
%
%\iffalse
%<*samplechap1|samplechap2>
%\fi

% Optional override for |\version| flag:
%    \begin{macrocode}
%%\providecommand{\version}{final}
%    \end{macrocode}

% Include the main document:
%    \begin{macrocode}
\input{childdoc.def}
\childdocof{cdocsamp}
%    \end{macrocode}

%\iffalse
%</samplechap1|samplechap2>
%\fi
%
%\iffalse
%<*samplechap1>
%\fi
% Some text for chapter 1:
%    \begin{macrocode}
\section{one}
some text in chapter one
%    \end{macrocode}

%\iffalse
%</samplechap1>
%\fi
% Some text for chapter 2:
%\iffalse
%<*samplechap2>
%\fi
%    \begin{macrocode}
\section{two}
more text in chapter two
%    \end{macrocode}

%\iffalse
%</samplechap2>
%\fi
%
% %%%%%%%%%%%%%%%%%%%%%%%%%%%%%%%%%%%%%%
% \paragraph{Part Include Files.}
%
% The include files are called |cdocspt3.tex| and |cdocspt4.tex|.
%
%\iffalse
%<*samplepart3|samplepart4>
%\fi

% Optional override for |\version| flag:
%    \begin{macrocode}
%%\providecommand{\version}{final}
%    \end{macrocode}

% Include the main document:
%    \begin{macrocode}
\input{childdoc.def}
\childdocby{cdocsamp}
%    \end{macrocode}

%\iffalse
%</samplepart3|samplepart4>
%\fi
%
%\iffalse
%<*samplepart3>
%\fi
% Some text for part 3:
%    \begin{macrocode}
some text in part three
%    \end{macrocode}

%\iffalse
%</samplepart3>
%\fi
% Some text for part 4:
%\iffalse
%<*samplepart4>
%\fi
%    \begin{macrocode}
more text in part four
%    \end{macrocode}

%\iffalse
%</samplepart4>
%\fi
%
% %%%%%%%%%%%%%%%%%%%%%%%%%%%%%%%%%%%%%%
% \paragraph{Forwarding for a Complete Draft.}
%
% The following forwarding file |cdocsdrf.tex|
% compiles the main document in draft mode:
%\iffalse
%<*sampledraft>
%\fi
%    \begin{macrocode}
\def\version{draft}
\input{childdoc.def}
\childdocforward{cdocsamp}
%    \end{macrocode}

%\iffalse
%</sampledraft>
%\fi
%
% %%%%%%%%%%%%%%%%%%%%%%%%%%%%%%%%%%%%%%
% \paragraph{Forwarding for Final Version of the Chapters.}
%
% The following forwarding files |cdocsfn1.tex| and |cdocsfn2.tex|
% (with identical content)
% compile the final versions of the child documents
% |cdocsch1.tex| and |cdocsch2.tex|, respectively:
%\iffalse
%<*samplefinal>
%\fi
%    \begin{macrocode}
\def\version{final}
\input{childdoc.def}
\childdocforwardprefix[cdocsamp]{cdocsfn}{cdocsch}
%    \end{macrocode}

%\iffalse
%</samplefinal>
%\fi
%
% %%%%%%%%%%%%%%%%%%%%%%%%%%%%%%%%%%%%%%
% \paragraph{Command Line Processing.}
%
% The following three command lines generate the output files
% |cdocscld|, |cdocscl1| and |cdocscl2|
% which should be identical to
% |cdocsdrf|, |cdocsch1| and |cdocsfn2|, respectively:
% \begin{center}
% \begin{tabular}{l}
% |latex -jobname cdocscld \|\\
% |  "\def\version{draft}\input{childdoc.def}\childdocforward{cdocsamp}"|\\
% |latex -jobname cdocscl1 \|\\
% |  "\input{childdoc.def}\childdocforward[cdocsamp]{cdocsch1}"|\\
% |latex -jobname cdocscl2 \|\\
% |  "\def\version{final}\input{childdoc.def}\childdocforward{cdocsch2}"|
% \end{tabular}
% \end{center}
% Note that the trailing backslash on each first line
% merely continues the input to the second line
% (for convenient cut ant paste).
% Furthermore, the command |latex| can be replaced by any
% of its alternative versions such as |pdflatex|.
%
% %%%%%%%%%%%%%%%%%%%%%%%%%%%%%%%%%%%%%%%%%%%%%%%%%%%%%%%%%%%%%%%%%%%%%%%%%%%%%%
% %%%%%%%%%%%%%%%%%%%%%%%%%%%%%%%%%%%%%%%%%%%%%%%%%%%%%%%%%%%%%%%%%%%%%%%%%%%%%%
% \section{Implementation}
%\iffalse
%<*package>
%\fi
%
% This section describes the definitions file |childdoc.def|.

% The definitions cannot be loaded using |\usepackage| or |\RequirePackage|
% which has a mechanism to prevent loading a style file more than once.
% When loading the definitions by means of |\input|
% multiple instances have to be prevented manually:
%\iffalse
%This code needs to be before the `\ProvidesFile' directive
%which is defined at the beginning of this file.
%Therefore it is also placed there and commented out here.
%</package>
%<*discard>
%\fi
%    \begin{macrocode}
\ifdefined\childdocmain\endinput\fi
%    \end{macrocode}
%\iffalse
%</discard>
%<*package>
%\fi
%
% \macro{\ifchilddoc}
% \macro{\ifchilddocmanual}
% The conditional |\ifchilddoc| tells whether a
% child (true) or main (false) document is being compiled.
% The conditional |\ifchilddocmanual| tells whether
% the |\includeonly| mechanism is used (false) or
% the selection of child files must be performed manually (true).
% The definitions initialise to false:
%    \begin{macrocode}
\newif\ifchilddoc
\newif\ifchilddocmanual
%    \end{macrocode}

% \macro{\childdocname}
% \macro{\childdocjob}
% The macro |\childdocname| stores the name of the main document
% to be compiled. The macro |\childdocjob| stores the name of
% the document on which the \LaTeX{} compiler was originally invoked.
% The content of |\jobname| cannot be compared
% to filenames specified in the source due to different catcodes.
% The following code rescans |\jobname|, stores the result
% in |\childdocname| and saves a copy in |\childdocjob|:
%    \begin{macrocode}
\edef\childdocname{\scantokens\expandafter{\jobname\noexpand}}
\let\childdocjob\childdocname
%    \end{macrocode}

% \macro{\childdocdisable}
% The macro |\childdocdisable| prevents the main file
% from being processed more than once.
% At this stage, the main document command |\childdocmain|
% is assumed to be called once again where it should do nothing.
% Any subsequent call to it should prevent
% a secondary processing of the main document
% It overwrites the forwarding commands
% |\childdocof| and |\childdocforward|
% with empty macros to prevent further inclusions of the main document:
%    \begin{macrocode}
\newcommand{\childdocdisable}
{
  \renewcommand{\childdocmain}[1]{\renewcommand{\childdocmain}[1]{\endinput}}
  \renewcommand{\childdocof}[1]{}
  \renewcommand{\childdocby}[2][]{}
  \renewcommand{\childdocforward}[2][]{}
  \renewcommand{\childdocdisable}{}
}
%    \end{macrocode}

% \macro{\childdocmain}
% The macro |\childdocmain| is to be called at the top of the main file
% with nothing or the main filename (without extension) as argument.
% First, it breaks loops.
% If the argument is not empty and does not match |\childdocname|
% (which is set by the first inclusion of |childdoc.def|),
% |\ifchilddoc| is set to true, |\includeonly| is applied to the child file
% and |\jobname| is set to the main file
% (for proper handling of |.aux| files):
%    \begin{macrocode}
\newcommand{\childdocmain}[1]
{
  \childdocdisable\childdocmain{}
  \if?#1?\else
    \begingroup
      \def\childdoctmp{#1}
      \ifx\childdoctmp\childdocname
        \def\childdoctmp{}
      \else
        \def\childdoctmp
        {
          \childdoctrue
          \includeonly{\childdocname}
          \def\childdocjob{#1}
          \def\jobname{#1}
        }
      \fi
      \expandafter
    \endgroup
    \childdoctmp
  \fi
}
%    \end{macrocode}

% \macro{\childdocof}
% The command |\childdocof| redirects
% compilation to the main file |#1|.
%    \begin{macrocode}
\newcommand{\childdocof}[1]
{
  \childdocdisable
  \childdoctrue
  \includeonly{\childdocname}
  \def\jobname{#1}
  \def\childdocjob{#1}
  \input{#1}
}
%    \end{macrocode}

% \macro{\childdocby}
% The command |\childdocby| ....
%    \begin{macrocode}
\newcommand{\childdocby}[2][]
{
  \childdocdisable
  \childdoctrue
  \childdocmanualtrue
  \if?#1?\else
    \def\jobname{#2}
  \fi
  \def\childdocjob{#2}
  \input{#2}
  \endinput
}
%    \end{macrocode}

% \macro{\childdocforward}
% The command |\childdocforward| redirects
% compilation to the main file or
% (if the optional argument is given) a child file.
% Parameters are set as if the main file
% or a child file starting with |\childdocof| was compiled.
% Then compilation is handed over to the main file:
%    \begin{macrocode}
\newcommand{\childdocforward}[2][]
{
  \begingroup
    \if?#1?
      \def\childdoctmp
      {
        \def\childdocname{#2}
        \def\childdocjob{#2}
        \def\jobname{#2}
        \input{#2}
        \endinput
      }
    \else
      \def\childdoctmp
      {
        \childdocdisable
        \def\childdocname{#2}
        \childdoctrue
        \includeonly{#2}
        \def\childdocjob{#1}
        \def\jobname{#1}
        \input{#1}
        \endinput
      }
    \fi
    \expandafter
  \endgroup
  \childdoctmp
}
%    \end{macrocode}

% \macro{\childdocforwardprefix}
% The command |\childdocforwardprefix| redirects
% compilation to the main or a child file by means of a pattern.
% The prefix |#1| in the current filename is replaced by |#2|
% and the suffix of the current filename is kept
% (it is assumed that the filename does not contain the substring `|~~~|'
% which is used as a delimiter).
% Compilation is handed over to the new file by |\childdocforward|:
%    \begin{macrocode}
\newcommand{\childdocforwardprefix}[3][]
{
  \begingroup
    \def\childdocextract #2##1~~~{\def\childdoctmp{\childdocforward[#1]{#3##1}}}
    \expandafter\childdocextract\childdocname~~~
    \expandafter
  \endgroup
  \childdoctmp
}
%    \end{macrocode}

% \macro{\childdoc}
% The deprecated macro |\childdoc| is a legacy version of |\childdocmain|:
%    \begin{macrocode}
\newcommand{\childdoc}{\childdocmain}
%    \end{macrocode}

% \macro{\childdocredirect}
% The deprecated macro |\childdocredirect| is a legacy version
% of |\childdocforward| and |\childdocforwardprefix|:
%    \begin{macrocode}
\newcommand{\childdocredirect}[2][]
{
  \begingroup
    \if?#1?
      \def\childdoctmp{\childdocforward{#2}}
    \else
      \def\childdoctmp{\childdocforwardprefix{#1}{#2}}
    \fi
    \expandafter
  \endgroup
  \childdoctmp
}
%    \end{macrocode}

%\iffalse
%</package>
%\fi
%
\endinput
\childdocforward{cdocsamp}"|\\
% |latex -jobname cdocscl1 \|\\
% |  "% \iffalse
%
% childdoc.dtx Copyright (C) 2017-2018 Niklas Beisert
%
% This work may be distributed and/or modified under the
% conditions of the LaTeX Project Public License, either version 1.3
% of this license or (at your option) any later version.
% The latest version of this license is in
%   http://www.latex-project.org/lppl.txt
% and version 1.3 or later is part of all distributions of LaTeX
% version 2005/12/01 or later.
%
% This work has the LPPL maintenance status `maintained'.
%
% The Current Maintainer of this work is Niklas Beisert.
%
% This work consists of the files childdoc.dtx and childdoc.ins
% and the derived files childdoc.def and cdocsamp.tex with
% cdocsch1.tex, cdocsch2.tex, cdocsdrf.tex, cdocsfn1.tex, cdocsfn2.tex.
%
%<package>\ifdefined\childdocmain\endinput\fi
%<package>\ProvidesFile{childdoc.def}[2018/12/30 v2.0 child document driver]
%<samplemain>\ProvidesFile{cdocsamp.tex}[2018/12/30 v2.0 sample for childdoc]
%<*driver>
%\ProvidesFile{childdoc.drv}[2018/12/30 v2.0 childdoc reference manual file]
\PassOptionsToClass{10pt,a4paper}{article}
\documentclass{ltxdoc}

\usepackage[margin=35mm]{geometry}
\usepackage{hyperref}
\usepackage{hyperxmp}
\usepackage[usenames]{color}

\hypersetup{colorlinks=true}
\hypersetup{pdfstartview=FitH}
\hypersetup{pdfpagemode=UseNone}
\hypersetup{pdfsource={}}
\hypersetup{pdflang={en-UK}}
\hypersetup{pdfcopyright={Copyright 2017-2018 Niklas Beisert.
  This work may be distributed and/or modified under the
  conditions of the LaTeX Project Public License, either version 1.3
  of this license or (at your option) any later version.}}
\hypersetup{pdflicenseurl={http://www.latex-project.org/lppl.txt}}
\hypersetup{pdfcontactaddress={ETH Zurich, ITP, HIT K,
  Wolfgang-Pauli-Strasse 27}}
\hypersetup{pdfcontactpostcode={8093}}
\hypersetup{pdfcontactcity={Zurich}}
\hypersetup{pdfcontactcountry={Switzerland}}
\hypersetup{pdfcontactemail={nbeisert@itp.phys.ethz.ch}}
\hypersetup{pdfcontacturl={http://people.phys.ethz.ch/\xmptilde nbeisert/}}

\newcommand{\secref}[1]{\hyperref[#1]{section \ref*{#1}}}

\parskip1ex
\parindent0pt
\let\olditemize\itemize
\def\itemize{\olditemize\parskip0pt}

\begin{document}

\title{The \textsf{childdoc} Package}
\hypersetup{pdftitle={The childdoc Package}}
\author{Niklas Beisert\\[2ex]
  Institut f\"ur Theoretische Physik\\
  Eidgen\"ossische Technische Hochschule Z\"urich\\
  Wolfgang-Pauli-Strasse 27, 8093 Z\"urich, Switzerland\\[1ex]
  \href{mailto:nbeisert@itp.phys.ethz.ch}
  {\texttt{nbeisert@itp.phys.ethz.ch}}}
\hypersetup{pdfauthor={Niklas Beisert}}
\hypersetup{pdfsubject={Manual for the LaTeX2e Package childdoc}}
\date{30 December 2018, \textsf{v2.0}}
\maketitle

\begin{abstract}\noindent
\textsf{childdoc} is a \LaTeXe{} package
that enables the direct compilation
of document sections included by |\include|
to individual files.
\end{abstract}

\begingroup
\parskip0ex
\tableofcontents
\endgroup

%%%%%%%%%%%%%%%%%%%%%%%%%%%%%%%%%%%%%%%%%%%%%%%%%%%%%%%%%%%%%%%%%%%%%%%%%%%%%%%%
%%%%%%%%%%%%%%%%%%%%%%%%%%%%%%%%%%%%%%%%%%%%%%%%%%%%%%%%%%%%%%%%%%%%%%%%%%%%%%%%
\section{Introduction}

\LaTeX{} provides a mechanism to structure a large document (such as a book)
into a main file and several child files (containing the chapters)
using the |\include| command.
This mechanism is beneficial for documents
which span hundreds of pages in order to
make the source file(s) more manageable.
Moreover, compilation can be restricted to
selected child files by means of the |\includeonly| command.
The latter feature can be used to reduce the compilation time while editing
(this was significantly more useful in the earlier days of \LaTeX{})
or to generate a smaller document which is easier to navigate.
Another application of |\includeonly| is to generate
documents consisting of selected parts of the complete document.

However, there are a few drawbacks of the plain |\include| mechanism:
\begin{itemize}
\item
The child files cannot be compiled on their own,
they can only be compiled via the main file.
A naive editing environment
(such as a text editor with an option
to have the current file processed by \LaTeX)
may require one to switch to the main file before compiling;
attempting to compile the child file produces errors.
\item
The main file must be modified (each time)
to adjust the |\includeonly| command
to the present needs. This easily leaves the main file in a messy state.
\item
The generated document will always carry the filename
of the main document. This is inconvenient if
several child files are to be compiled and
to be kept for distribution.
\end{itemize}

The present package provides a simple interface
to make child files individually compilable by \LaTeX{}.
Compiling a child file then has the same effect as compiling
the main file with an |\includeonly| command
to select the appropriate child.
Moreover the generated document will carry the name of the child
rather than the main file.
This resolves all three above issues.

This feature is meant to make the editing of books,
thesis documents and lecture notes somewhat more convenient.
However, the package can also be used efficiently for
composing a series of documents (such as exercise sheets)
which are typically distributed individually.
It then assists the author in generating the individual documents
(potentially in different versions)
as well as a document containing the collected series.
Another application is in developing style files
or other kinds of included material
where compilation of the style file could redirect
to a sample or test file.

%%%%%%%%%%%%%%%%%%%%%%%%%%%%%%%%%%%%%%%%%%%%%%%%%%%%%%%%%%%%%%%%%%%%%%%%%%%%%%%%
%%%%%%%%%%%%%%%%%%%%%%%%%%%%%%%%%%%%%%%%%%%%%%%%%%%%%%%%%%%%%%%%%%%%%%%%%%%%%%%%
\section{Usage}

First of all, the package \textsf{childdoc} is \emph{not} a standard
\LaTeXe{} |.sty| style file! Therefore it needs to be invoked in
a non-standard way.

%%%%%%%%%%%%%%%%%%%%%%%%%%%%%%%%%%%%%%%%%%%%%%%%%%%%%%%%%%%%%%%%%%%%%%%%%%%%%%%%
\subsection{Included Files}
\label{sec:include}

%%%%%%%%%%%%%%%%%%%%%%%%%%%%%%%%%%%%%%%%
\DescribeMacro{\childdocmain}
To use the package, add the commands
\begin{center}
\begin{tabular}{l}
|\input{childdoc.def}|\\
|\childdocmain{}|\\
\end{tabular}
\end{center}
at the very top of the main \LaTeX{} file,
in particular \emph{before} the |\documentclass| statement!
The argument of |\childdocmain| should be left empty
(but it must be present).

%%%%%%%%%%%%%%%%%%%%%%%%%%%%%%%%%%%%%%%%
\DescribeMacro{\childdocof}
Furthermore, add the commands
\begin{center}
\begin{tabular}{l}
|\input{childdoc.def}|\\
|\childdocof{|\textit{main}|}|\\
\end{tabular}
\end{center}
at the top of every child file \textit{child}
which is included by |\include{|\textit{child}|}|
from within the main file
(or at least for those files to be compiled individually).
The argument \textit{main} must be the filename of the main file.

There are a couple of
considerations in setting up the main and child documents:

%%%%%%%%%%%%%%%%%%%%%%%%%%%%%%%%%%%%%%%%
\paragraph{Restrictions.}

Please note the following restrictions:
\begin{itemize}
\item
|\childdocmain| must be called with one argument \textit{main}
to ensure compatibility with earlier version of the package.
It must either be empty (|\childdocmain{}|)
or precisely match the filename of the main file in which it is specified.
See \secref{sec:detection} for further information.
\item
The filename \textit{main} must be specified without the |.tex| extension.
\item
The filename \textit{main} is case sensitive
(even in case-insensitive file systems)
due to internal string comparison.
\item
The argument \textit{main} should be fully expanded, it cannot be a macro.
\item
Subdirectories and special characters should be avoided in filenames.
\item
The command |\childdocmain{|\textit{main}|}| must be followed by a whitespace.
It should not be followed immediately by another command
or by a comment mark `|%|'.
This is because the \TeX{} parser reads the token immediately following
the argument of |\childdocmain| and puts it
at the beginning of every child section;
however, a white\-space is ignored.
\end{itemize}

%%%%%%%%%%%%%%%%%%%%%%%%%%%%%%%%%%%%%%%%
\paragraph{Content of Main File.}

It is advisable to place all content in the child files included by |\include|.
Any output contained in the main file will appear in all child documents
unless suppressed manually;
it cannot be suppressed automatically by the |\includeonly| directive
and thus should normally be avoided.
A method to include some content in the main file
by means of conditional processing is described in \secref{sec:conditional}.

%%%%%%%%%%%%%%%%%%%%%%%%%%%%%%%%%%%%%%%%
\paragraph{Page Numbering.}

When only a part of the document is compiled,
the appropriate numbering of pages
(as well as other status parameters)
is determined from the |.aux| files.
The latter contain information from previous passes.
However this information needs to propagate through
all intermediate child documents.
Therefore the page numbering in child documents may well
be inconsistent until the complete document is compiled at least once.

A useful (if unconventional) way to always ensure a consistent
page numbering is to restart the numbering in each child document
and denote the pages by `\textit{child}|.|\textit{page}'
where \textit{child} represents the chapter/section number of the child file.
This can be achieved by the command
|\numberwithin{page}{|\textit{child}|}|
of the \textsf{amsmath} package
where \textit{child} can be |chapter| or |section|
depending on the chosen structuring.
Alternatively, one can modify the macro |\thepage| appropriately
and reset the counter |page| at the start of each child file.

%%%%%%%%%%%%%%%%%%%%%%%%%%%%%%%%%%%%%%%%%%%%%%%%%%%%%%%%%%%%%%%%%%%%%%%%%%%%%%%%
\subsection{Conditional Processing}
\label{sec:conditional}

The package provides a mechanism to compile different versions
of a document. To customise the versions further some conditional processing
can come in handy to distinguish which version is being compiled.
The package provides two macros to describe the compilation context:

%%%%%%%%%%%%%%%%%%%%%%%%%%%%%%%%%%%%%%%%
\DescribeMacro{\ifchilddoc}
The conditional |\ifchilddoc| distinguishes between the compilation of
child documents and the main document:
%
\begin{center}
|\ifchilddoc |\textit{child-code}| |[|\||else |\textit{main-code}]| \||fi|
\end{center}

%%%%%%%%%%%%%%%%%%%%%%%%%%%%%%%%%%%%%%%%
\DescribeMacro{\childdocname}
\DescribeMacro{\childdocjob}
The macro |\childdocname| contains the filename (without extension)
of the main or child file being processed.
Note that |\childdocjob| will always contain the name of the main file.

%%%%%%%%%%%%%%%%%%%%%%%%%%%%%%%%%%%%%%%%
\paragraph{Title Page.}

Conditional processing can be used to include a title or banner page
in the main document when proper precautions are taken.
Importantly, the code in the main file should ensure that the page counter
(as well as other status parameters which are stored in the |.aux| files)
takes the same value after the conditional processing.
Otherwise the page numbers may take divergent values
depending on which part is compiled.

For example, a title page could be declared by:
%
\begin{center}
\begin{tabular}{l}
|\ifchilddoc\||else|\\
|\addtocounter{page}{-1}|\\
\textit{code for title page}\\
|\newpage|\\
|\||fi|
\end{tabular}
\end{center}
%
A banner page for the child documents can be generated by:
%
\begin{center}
\begin{tabular}{l}
|\ifchilddoc|\\
|\addtocounter{page}{-1}|\\
\textit{code for banner page}\\
|\newpage|\\
|\||fi|
\end{tabular}
\end{center}
%
Here one could write a message such as:
\begin{center}
|This is the part \childdocname{} of \childdocjob{}.|
\end{center}

%%%%%%%%%%%%%%%%%%%%%%%%%%%%%%%%%%%%%%%%%%%%%%%%%%%%%%%%%%%%%%%%%%%%%%%%%%%%%%%%
\subsection{Flags}
\label{sec:flags}

The package makes it easy to generate different versions
of the main or child documents.
To this end compilation flags can be defined
and assigned different default values.
They will be particularly useful in conjunction
with the forwarding mechanism described in \secref{sec:forward}.

For example, it may be useful to have a flag |\version|
which can be set to |draft| or |final|.
The document source will contain some conditional code
depending on the value of |\version|.
Suppose further, the flag should default to |final| for the main file
and to |draft| for child files
which is a natural assignment for editing the document.
This is achieved by placing the following code
in the preamble of the main document
(below the |\childdocmain| directive):
%
\begin{center}
\begin{tabular}{l}
|\ifchilddoc|\\
|\providecommand{\version}{draft}|\\
|\||else|\\
|\providecommand{\version}{final}|\\
|\||fi|
\end{tabular}
\end{center}
%
The definition by |\providecommand| makes sure
that previous definitions are not overwritten.
Further statements |\providecommand{\version}{...}|
can thus be added before the above code to override it.

For the main file, one might add a line
(between |\childdocmain| and the above block)
%
\begin{center}
|%\ifchilddoc\||else\providecommand{\version}{draft}\||fi|
\end{center}
%
which can be uncommented to produce a draft version.
Likewise one can add a line to the very top of a child file
(above the |\childdocof{|\textit{main}|}| directive)
%
\begin{center}
|%\providecommand{\version}{final}|
\end{center}
%
which can be uncommented to produce the final version of this child document.

%%%%%%%%%%%%%%%%%%%%%%%%%%%%%%%%%%%%%%%%%%%%%%%%%%%%%%%%%%%%%%%%%%%%%%%%%%%%%%%%
\subsection{Forwarding}
\label{sec:forward}

Different versions of the main or child documents
using compilation flags as described in \secref{sec:flags}
can be (permanently) stored in different files
for convenient compilation, viewing and distribution.
To this end, the package defines a command
to pass on compilation to a different file:

%%%%%%%%%%%%%%%%%%%%%%%%%%%%%%%%%%%%%%%%
\DescribeMacro{\childdocforward}
The command |\childdocforward| redirects processing to
another source file:
%
\begin{center}
\begin{tabular}{l}
|\input{childdoc.def}|\\
|\childdocforward[|\textit{main}|]{|\textit{dest}|}|\\
\end{tabular}
\end{center}
%
The argument \textit{dest} is the destination file
(without extension).
It should be the main file or one of the child files.
Note that further \textsf{childdoc} directives
such as |\childdocof| and |\childdocforward|
in the indicated file will be processed in this form.
The optional argument \textit{main}
passes on directly to the main file \textit{main}
while pretending to compile the child \textit{dest}.
This form behaves as if \textit{dest}
issues |\childdocof{|\textit{main}|}| right away,
and no further \textsf{childdoc} directives will be processed.

%%%%%%%%%%%%%%%%%%%%%%%%%%%%%%%%%%%%%%%%
\DescribeMacro{\...prefix}
In the alternative form |\childdocforwardprefix|,
%
\begin{center}
\begin{tabular}{l}
|\input{childdoc.def}|\\
|\childdocforwardprefix[|\textit{main}|]{|\textit{prefix}|}{|\textit{dest}|}|
\end{tabular}
\end{center}
%
the destination file is determined by a pattern
depending on the current file:
To make this work, the current file must be called
`{\textit{prefix}\hspace{0.2em}\textit{suffix}}'
with \textit{prefix} matching precisely the argument.
Processing is then passed on to the file
`{\textit{dest}\hspace{0.2em}\textit{suffix}}'.
Surely, the same effect is achieved by
directly specifying the
argument `{\textit{dest}\hspace{0.2em}\textit{suffix}}'
in the first form.
However, that requires to set up a different file
for each child. With the alternative form of the command
all these files can have exactly the same content
which simplifies setting them up and maintaining them.

For example, the following file |draft.tex|
with a compilation flag |\version| as described in \secref{sec:flags}
compiles the main document as a draft:
%
\begin{center}
\begin{tabular}{l}
|\def\version{draft}|\\
|\input{childdoc.def}|\\
|\childdocforward{|\textit{main}|}|
\end{tabular}
\end{center}
%
Likewise, the following files |final|\textit{nn}|.tex|
compile the final version of the child document
|child|\textit{nn}|.tex|:
%
\begin{center}
\begin{tabular}{l}
|\def\version{final}|\\
|\input{childdoc.def}|\\
|\childdocforwardprefix{final}{child}|
\end{tabular}
\end{center}
%

Note that when several versions of a main file and/or of each child file
are to be generated, it may be convenient to set up a |Makefile| or
shell script to automatise the process.

%%%%%%%%%%%%%%%%%%%%%%%%%%%%%%%%%%%%%%%%%%%%%%%%%%%%%%%%%%%%%%%%%%%%%%%%%%%%%%%%
\subsection{Command Line Processing}
\label{sec:commandline}

The effect of redirection files can also be achieved by invoking
the \LaTeX{} compiler with a more elaborate command line.
Most conveniently this should be done as part
of a shell script or a |Makefile|.

When using \textsf{childdoc} in the main file, the following
command lines effectively perform a redirection
(note that depending on the shell being used,
backslashes may have to be doubled: `|\|' $\to$ `|\\|'):
%
\begin{center}
|... -jobname "|\textit{target}|" |\\|"|[\textit{flags}]%
|\input{childdoc.def}\childdocforward[|\textit{main}|]{|\textit{dest}|}"|
\end{center}
%
Here \textit{target} is the name of the output file,
\textit{main} is the name of the main file
and \textit{dest} is the name of the main or child file to be processed
(all filenames without extensions).
The optional argument \textit{main} can be omitted
if \textit{main} matches \textit{dest}.
Optionally, compilation \textit{flags} can be defined via |\def| commands.
This command line makes the \TeX{} engine believe
it is compiling the file \textit{target}
whose content is specified as the latter parameter.
The provided code then forwards the processing to
\textit{main} or \textit{dest} as described in \secref{sec:forward}.

%%%%%%%%%%%%%%%%%%%%%%%%%%%%%%%%%%%%%%%%%%%%%%%%%%%%%%%%%%%%%%%%%%%%%%%%%%%%%%%%
\subsection{Include by Input}
\label{sec:input}

Including child documents by |\include| has some restrictions by design.
Most notably, the content of a child document always occupies
its own set of pages; pages cannot be shared between child documents.
Usually, this behaviour makes perfect sense
because each child document contain an essential part of the document.
However, in some situations it may be desirable to compose
a document from a collection of parts
without having mandatory page breaks between then.
For this case, the package
provides a mechanism to include parts
by |\input| which can also be processed individually.
However, by construction this mechanism
requires manual handling of the content to be output.

%%%%%%%%%%%%%%%%%%%%%%%%%%%%%%%%%%%%%%%%
\DescribeMacro{\ifchilddocmanual}
The main file should be prepared as usual, see \secref{sec:include}.
However, the document body must make a distinction
between processing of an individual part and of the main document, e.g.:
%
\begin{center}
\begin{tabular}{l}
|\ifchilddocmanual|\\
|\input{\childdocname}|\\
|\||else|\\
\textit{document body with }|\input{|\textit{part}|}|\\
|\||fi|
\end{tabular}
\end{center}
%
The conditional |\ifchilddocmanual| is true whenever
a part to be included by |\input| is being compiled,
and the name of the part is stored in |\childdocname|.

%%%%%%%%%%%%%%%%%%%%%%%%%%%%%%%%%%%%%%%%
\DescribeMacro{\childdocby}
Each part to be included by |\input| should start with:
%
\begin{center}
\begin{tabular}{l}
|\input{childdoc.def}|\\
|\childdocby{|\textit{main}|}|\\
\end{tabular}
\end{center}
%
The directive |\childdocby| is similar to |\childdocof|
described in \secref{sec:include},
but the subsequent selection of content must be done manually.
To that end, both |\ifchilddoc| and |\ifchilddocmanual|
will be true upon processing of a part,
and the name of the part is stored in |\childdocname|.
Note that |\jobname| will be set to the filename of the current part
so that each part receives an individual |.aux| file
that does not interfere with the |.aux| file(s) of the main document.
This behaviour can be altered by the alternative form
|\childdocby[*]{|\textit{main}|}| (with a non-empty optional argument)
which uses the |.aux| file of the main document
by setting |\jobname| to \textit{main}.

%%%%%%%%%%%%%%%%%%%%%%%%%%%%%%%%%%%%%%%%%%%%%%%%%%%%%%%%%%%%%%%%%%%%%%%%%%%%%%%%
\subsection{Driver Development}
\label{sec:driver}

The \textsf{childdoc} mechanism can also be use for the development
of definition files such as \LaTeX{} styles or classes.
This case differs from the above setup with multiple parts
included by |\include| in that no |\includeonly| should be invoked.
This can be achieved by starting the include file
(before |\ProvidesPackage|) with:
%
\begin{center}
\begin{tabular}{l}
|\input{childdoc.def}|\\
|\childdocforward{|\textit{main}|}|\\
\end{tabular}
\end{center}
%
or alternatively with:
%
\begin{center}
\begin{tabular}{l}
|\input{childdoc.def}|\\
|\childdocby{|\textit{main}|}|\\
\end{tabular}
\end{center}
%
Both forms have slightly different effects as described above.
The main file is prepared as usual, see \secref{sec:include}.

%%%%%%%%%%%%%%%%%%%%%%%%%%%%%%%%%%%%%%%%%%%%%%%%%%%%%%%%%%%%%%%%%%%%%%%%%%%%%%%%
\subsection{Legacy Detection}
\label{sec:detection}

The directive |\childdocmain| in the main file can detect
whether the complete document or merely a child is to be compiled
even without using the directive |\childdocof|.
This method is deprecated because it is less robust
and there is no compelling reason to use it;
it is merely provided for backward compatibility
and it may be removed in future versions.

If the detection mechanism is to be used,
it is mandatory to correctly specify
the filename of the main file as the argument of |\childdocmain|:
%
\begin{center}
\begin{tabular}{l}
|\input{childdoc.def}|\\
|\childdocmain{|\textit{main}|}|\\
\end{tabular}
\end{center}
%
If |\jobname| does not match the argument \textit{main} of |\childdocmain|,
it is assumed that |\jobname| points to the child file to be compiled.
When using |\childdocmain| with the main file specified as argument,
it suffices to start a child file
with just |\input{|\textit{main}|}|
without loading of the package and using |\childdocof|.
If instead all processing is done
with the appropriate \textsf{childdoc} directives,
the argument of \textit{main} of |\childdocmain| can be empty.

An alternative version of the command line processing described
in \secref{sec:commandline} using the detection mechanism reads:
%
\begin{center}
|... -jobname "|\textit{target}|" "|[\textit{flags}]%
[|\def\jobname{|\textit{dest}|}|]|\input{|\textit{main}|}"|
\end{center}

%%%%%%%%%%%%%%%%%%%%%%%%%%%%%%%%%%%%%%%%%%%%%%%%%%%%%%%%%%%%%%%%%%%%%%%%%%%%%%%%
\subsection{Manual Code}
\label{sec:manual}

In case one cannot be certain whether the definitions file |childdoc.def|
is installed on the target \TeX{} distribution
and one prefers not to ship it,
it is conceivable to paste a few relevant commands into the sources.

To that end, drop all statements |\input{childdoc.def}|
and perform the replacements as outlined below.
Instead of |\childdocmain{|\textit{main}|}| add the following code
to the top of the main file:
%
\begin{center}
\begin{tabular}{l}
|\||ifdefined\childdocname\endinput\||fi\newif\ifchilddoc|\\
|\edef\childdocname{\scantokens\expandafter{\jobname\noexpand}}|\\
|\def\childdocmain{|\textit{main}|}\||ifx\childdocmain\childdocname\||else|\\
|\childdoctrue\includeonly{\childdocname}\let\jobname\childdocmain\||fi|\\
\end{tabular}
\end{center}
%
Instead of |\childdocof{|\textit{main}|}| just include the main file
at the top of each child file:
%
\begin{center}
|\input{|\textit{main}|}|
\end{center}
%
A simple redirection |\childdocforward{|\textit{dest}|}| is achieved by:
%
\begin{center}
|\def\jobname{|\textit{dest}|}\input{\jobname}|
\end{center}
%
The redirection with prefix
|\childdocforwardprefix[|\textit{prefix}|]{|\textit{dest}|}|
is accomplished by:
%
\begin{center}
\begin{tabular}{l}
|{\edef\jobname{\scantokens\expandafter{\jobname\noexpand}}|\\
|\def\redirectjob |\textit{prefix}|#1~~~{\gdef\jobname{|\textit{dest}|#1}}|\\
|\expandafter\redirectjob\jobname~~~}\input{\jobname}|
\end{tabular}
\end{center}

In an alternative approach,
child documents can be compiled by a specific command line
without additional code or specific definitions:
%
\begin{center}
|... -jobname "|\textit{target}|" "|[\textit{flags}]%
|\includeonly{|\textit{dest}|}\input{|\textit{main}|}"|
\end{center}
%

%%%%%%%%%%%%%%%%%%%%%%%%%%%%%%%%%%%%%%%%%%%%%%%%%%%%%%%%%%%%%%%%%%%%%%%%%%%%%%%%
%%%%%%%%%%%%%%%%%%%%%%%%%%%%%%%%%%%%%%%%%%%%%%%%%%%%%%%%%%%%%%%%%%%%%%%%%%%%%%%%
\section{Information}

%%%%%%%%%%%%%%%%%%%%%%%%%%%%%%%%%%%%%%%%%%%%%%%%%%%%%%%%%%%%%%%%%%%%%%%%%%%%%%%%
\subsection{Copyright}

Copyright \copyright{} 2017--2018 Niklas Beisert

This work may be distributed and/or modified under the
conditions of the \LaTeX{} Project Public License, either version 1.3
of this license or (at your option) any later version.
The latest version of this license is in
  \url{http://www.latex-project.org/lppl.txt}
and version 1.3 or later is part of all distributions of \LaTeX{}
version 2005/12/01 or later.

This work has the LPPL maintenance status `maintained'.

The Current Maintainer of this work is Niklas Beisert.

This work consists of the files |README.txt|, |childdoc.ins| and |childdoc.dtx|
as well as the derived files |childdoc.def|, |cdocsamp.tex|
with |cdocsch1.tex|, |cdocsch2.tex|, |cdocspt3.tex|, |cdocspt4.tex|,
|cdocsdrf.tex|, |cdocsfn1.tex|, |cdocsfn2.tex|
as well as |childdoc.pdf|.

%%%%%%%%%%%%%%%%%%%%%%%%%%%%%%%%%%%%%%%%%%%%%%%%%%%%%%%%%%%%%%%%%%%%%%%%%%%%%%%%
\subsection{Files and Installation}

The package consists of the files:
%
\begin{center}
\begin{tabular}{ll}
    |README.txt|   & readme file \\
    |childdoc.ins| & installation file \\
    |childdoc.dtx| & source file \\
    |childdoc.def| & definition file \\
    |cdocsamp.tex| & sample main file \\
    |cdocsch1.tex| & sample include file \\
    |cdocsch2.tex| & sample include file \\
    |cdocspt3.tex| & sample part file \\
    |cdocspt4.tex| & sample part file \\
    |cdocsdrf.tex| & sample redirection file \\
    |cdocsfn1.tex| & sample redirection file \\
    |cdocsfn2.tex| & sample redirection file \\
    |childdoc.pdf| & manual
\end{tabular}
\end{center}
%
The distribution consists of the files
|README.txt|, |childdoc.ins| and |childdoc.dtx|.
%
\begin{itemize}
\item
Run (pdf)\LaTeX{} on |childdoc.dtx|
to compile the manual |childdoc.pdf| (this file).
\item
Run \LaTeX{} on |childdoc.ins| to create the definitions file |childdoc.def|
and the sample |cdocsamp.tex| with include files
|cdocsch1.tex|, |cdocsch2.tex|, |cdocspt3.tex|, |cdocspt4.tex|,
|cdocsdrf.tex|, |cdocsfn1.tex|, |cdocsfn2.tex|.
Then copy the file |childdoc.def| to an appropriate directory of your \LaTeX{}
distribution, e.g.\ \textit{texmf-root}|/tex/latex/childdoc|.
\end{itemize}

%%%%%%%%%%%%%%%%%%%%%%%%%%%%%%%%%%%%%%%%%%%%%%%%%%%%%%%%%%%%%%%%%%%%%%%%%%%%%%%%
\subsection{Related CTAN Packages}

There are several other packages which offer a similar functionality:
%
\begin{itemize}
\item
The packages
\href{http://ctan.org/pkg/docmute}{\textsf{docmute}},
\href{http://ctan.org/pkg/includex}{\textsf{includex}} and
\href{http://ctan.org/pkg/standalone}{\textsf{standalone}}
provide commands to include only the document body of
a child file thus allowing both files to be compiled individually.
\item
The packages \href{http://ctan.org/pkg/subdocs}{\textsf{subdocs}}
and \href{http://ctan.org/pkg/subfiles}{\textsf{subfiles}}
provide structures in which the main and child documents can be
encapsulated and allowing them to be compiled individually.
The inclusion mechanism is different from the conventional |\include|.
\item
The package \href{http://ctan.org/pkg/combine}{\textsf{combine}}
is an elaborate solution to combine several documents into one.
\end{itemize}
%
See also the CTAN topic \href{http://ctan.org/topic/subdocs}{\textsf{subdocs}}
for further related packages.
The present package differs from the above solutions in that
a document structure constructed with the conventional |\include| mechanism
just needs two extra commands at the top of every file
such that all constituent files can be compiled individually.

%%%%%%%%%%%%%%%%%%%%%%%%%%%%%%%%%%%%%%%%%%%%%%%%%%%%%%%%%%%%%%%%%%%%%%%%%%%%%%%%
%\subsection{Feature Suggestions}
%
%The following is a list of features which may be useful for future
%versions of this package:
%%
%\begin{itemize}
%\item
%\ldots
%\end{itemize}

%%%%%%%%%%%%%%%%%%%%%%%%%%%%%%%%%%%%%%%%%%%%%%%%%%%%%%%%%%%%%%%%%%%%%%%%%%%%%%%%
\subsection{Revision History}

%%%%%%%%%%%%%%%%%%%%%%%%%%%%%%%%%%%%%%%%
\paragraph{v2.0:} 2018/12/30

\begin{itemize}
\item
immediate forward processing
\item
added |\childdocby| mechanism
\item
manual restructured
\end{itemize}

%%%%%%%%%%%%%%%%%%%%%%%%%%%%%%%%%%%%%%%%
\paragraph{v1.6:} 2018/01/17

\begin{itemize}
\item
application for development of include files
\item
corrections to manual
\end{itemize}

%%%%%%%%%%%%%%%%%%%%%%%%%%%%%%%%%%%%%%%%
\paragraph{v1.5:} 2017/05/21

\begin{itemize}
\item
more complete structuring introduced
\item
|\childdocof| introduced
\item
|\childdoc| renamed to |\childdocmain|
\item
|\childredirect| renamed to |\childdocforward| and |\childdocforwardprefix|
and functionality expanded
\end{itemize}

%%%%%%%%%%%%%%%%%%%%%%%%%%%%%%%%%%%%%%%%
\paragraph{v1.0:} 2017/04/27

\begin{itemize}
\item
manual and install package
\item
first version published on CTAN
\end{itemize}

%%%%%%%%%%%%%%%%%%%%%%%%%%%%%%%%%%%%%%%%
\paragraph{v0.6:} 2017/04/26

\begin{itemize}
\item
redirection mechanism added
\end{itemize}

%%%%%%%%%%%%%%%%%%%%%%%%%%%%%%%%%%%%%%%%
\paragraph{v0.5:} 2017/04/26

\begin{itemize}
\item
functionality in definition file
\end{itemize}


%%%%%%%%%%%%%%%%%%%%%%%%%%%%%%%%%%%%%%%%%%%%%%%%%%%%%%%%%%%%%%%%%%%%%%%%%%%%%%%%
%%%%%%%%%%%%%%%%%%%%%%%%%%%%%%%%%%%%%%%%%%%%%%%%%%%%%%%%%%%%%%%%%%%%%%%%%%%%%%%%
%%%%%%%%%%%%%%%%%%%%%%%%%%%%%%%%%%%%%%%%%%%%%%%%%%%%%%%%%%%%%%%%%%%%%%%%%%%%%%%%
\appendix

\settowidth\MacroIndent{\rmfamily\scriptsize 000\ }

 \DocInput{childdoc.dtx}

\end{document}
%</driver>
% \fi
%
% %%%%%%%%%%%%%%%%%%%%%%%%%%%%%%%%%%%%%%%%%%%%%%%%%%%%%%%%%%%%%%%%%%%%%%%%%%%%%%
% %%%%%%%%%%%%%%%%%%%%%%%%%%%%%%%%%%%%%%%%%%%%%%%%%%%%%%%%%%%%%%%%%%%%%%%%%%%%%%
% \section{Sample}
%\iffalse
%<*samplemain>
%\fi
%
% The following presents a sample document
% with two chapters, two parts, a title page,
% a compile flag as well as three forwarding files to set the flag.
% It consists of eight |.tex| files:
% \begin{center}
% \begin{tabular}{ll}
% |cdocsamp.tex|&main file\\
% |cdocsch1.tex|&include file for chapter 1\\
% |cdocsch2.tex|&include file for chapter 2\\
% |cdocspt3.tex|&include file for part 3\\
% |cdocspt4.tex|&include file for part 4\\
% |cdocsdrf.tex|&forwarding file for main file in draft mode\\
% |cdocsfi1.tex|&forwarding file for final version of chapter 1\\
% |cdocsfi2.tex|&forwarding file for final version of chapter 2\\
% \end{tabular}
% \end{center}
% Each of the eight files can be compiled directly by the \LaTeX{} compiler.
%
% %%%%%%%%%%%%%%%%%%%%%%%%%%%%%%%%%%%%%%
% \paragraph{Main File.}
%
% The main file is called |cdocsamp.tex|.
%
% Load the \textsf{childdoc} definitions and
% declare the filename for the main document:
%    \begin{macrocode}
\input{childdoc.def}
\childdocmain{}
%    \end{macrocode}

% Optional override for |\version| flag:
%    \begin{macrocode}
%%\ifchilddoc\else\providecommand{\version}{draft}\fi
%    \end{macrocode}

% Define the default values for the |\version| flag
% (|final| for the main file and |draft| for childs):
%    \begin{macrocode}
\ifchilddoc
\providecommand{\version}{draft}
\else
\providecommand{\version}{final}
\fi
%    \end{macrocode}

% Load the standard document class:
%    \begin{macrocode}
\documentclass[12pt]{article}
%    \end{macrocode}

% Start the document body:
%    \begin{macrocode}
\begin{document}
%    \end{macrocode}

% Declare a title page.
% Print title, part of document being processed and version flag:
%    \begin{macrocode}
\addtocounter{page}{-1}
\begin{center}
{\LARGE\bfseries{}childdoc example\par}
\vspace{1cm}
\ifchilddoc
\ifchilddocmanual part\else chapter\fi:
`\childdocname' of `\childdocjob'\par
\else
main document: `\childdocjob'\par
\fi
version: \version\par
\end{center}
\newpage
%    \end{macrocode}

% Manually include selected file,
% otherwise process as usual:
%    \begin{macrocode}
\ifchilddocmanual
\section*{part `\childdocname'}
\input{\childdocname}
\else
%    \end{macrocode}

% Include the two chapters:
%    \begin{macrocode}
\include{cdocsch1}
\include{cdocsch2}
%    \end{macrocode}

% Include the two parts unless only chapters should be displayed:
%    \begin{macrocode}
\ifchilddoc\else
\section{part three}
\input{cdocspt3}
\section{part four}
\input{cdocspt4}
\fi
%    \end{macrocode}

% Process as usual until here:
%    \begin{macrocode}
\fi
%    \end{macrocode}

% End of document body:
%    \begin{macrocode}
\end{document}
%    \end{macrocode}
%\iffalse
%</samplemain>
%\fi
%
% %%%%%%%%%%%%%%%%%%%%%%%%%%%%%%%%%%%%%%
% \paragraph{Chapter Include Files.}
%
% The include files are called |cdocsch1.tex| and |cdocsch2.tex|.
%
%\iffalse
%<*samplechap1|samplechap2>
%\fi

% Optional override for |\version| flag:
%    \begin{macrocode}
%%\providecommand{\version}{final}
%    \end{macrocode}

% Include the main document:
%    \begin{macrocode}
\input{childdoc.def}
\childdocof{cdocsamp}
%    \end{macrocode}

%\iffalse
%</samplechap1|samplechap2>
%\fi
%
%\iffalse
%<*samplechap1>
%\fi
% Some text for chapter 1:
%    \begin{macrocode}
\section{one}
some text in chapter one
%    \end{macrocode}

%\iffalse
%</samplechap1>
%\fi
% Some text for chapter 2:
%\iffalse
%<*samplechap2>
%\fi
%    \begin{macrocode}
\section{two}
more text in chapter two
%    \end{macrocode}

%\iffalse
%</samplechap2>
%\fi
%
% %%%%%%%%%%%%%%%%%%%%%%%%%%%%%%%%%%%%%%
% \paragraph{Part Include Files.}
%
% The include files are called |cdocspt3.tex| and |cdocspt4.tex|.
%
%\iffalse
%<*samplepart3|samplepart4>
%\fi

% Optional override for |\version| flag:
%    \begin{macrocode}
%%\providecommand{\version}{final}
%    \end{macrocode}

% Include the main document:
%    \begin{macrocode}
\input{childdoc.def}
\childdocby{cdocsamp}
%    \end{macrocode}

%\iffalse
%</samplepart3|samplepart4>
%\fi
%
%\iffalse
%<*samplepart3>
%\fi
% Some text for part 3:
%    \begin{macrocode}
some text in part three
%    \end{macrocode}

%\iffalse
%</samplepart3>
%\fi
% Some text for part 4:
%\iffalse
%<*samplepart4>
%\fi
%    \begin{macrocode}
more text in part four
%    \end{macrocode}

%\iffalse
%</samplepart4>
%\fi
%
% %%%%%%%%%%%%%%%%%%%%%%%%%%%%%%%%%%%%%%
% \paragraph{Forwarding for a Complete Draft.}
%
% The following forwarding file |cdocsdrf.tex|
% compiles the main document in draft mode:
%\iffalse
%<*sampledraft>
%\fi
%    \begin{macrocode}
\def\version{draft}
\input{childdoc.def}
\childdocforward{cdocsamp}
%    \end{macrocode}

%\iffalse
%</sampledraft>
%\fi
%
% %%%%%%%%%%%%%%%%%%%%%%%%%%%%%%%%%%%%%%
% \paragraph{Forwarding for Final Version of the Chapters.}
%
% The following forwarding files |cdocsfn1.tex| and |cdocsfn2.tex|
% (with identical content)
% compile the final versions of the child documents
% |cdocsch1.tex| and |cdocsch2.tex|, respectively:
%\iffalse
%<*samplefinal>
%\fi
%    \begin{macrocode}
\def\version{final}
\input{childdoc.def}
\childdocforwardprefix[cdocsamp]{cdocsfn}{cdocsch}
%    \end{macrocode}

%\iffalse
%</samplefinal>
%\fi
%
% %%%%%%%%%%%%%%%%%%%%%%%%%%%%%%%%%%%%%%
% \paragraph{Command Line Processing.}
%
% The following three command lines generate the output files
% |cdocscld|, |cdocscl1| and |cdocscl2|
% which should be identical to
% |cdocsdrf|, |cdocsch1| and |cdocsfn2|, respectively:
% \begin{center}
% \begin{tabular}{l}
% |latex -jobname cdocscld \|\\
% |  "\def\version{draft}\input{childdoc.def}\childdocforward{cdocsamp}"|\\
% |latex -jobname cdocscl1 \|\\
% |  "\input{childdoc.def}\childdocforward[cdocsamp]{cdocsch1}"|\\
% |latex -jobname cdocscl2 \|\\
% |  "\def\version{final}\input{childdoc.def}\childdocforward{cdocsch2}"|
% \end{tabular}
% \end{center}
% Note that the trailing backslash on each first line
% merely continues the input to the second line
% (for convenient cut ant paste).
% Furthermore, the command |latex| can be replaced by any
% of its alternative versions such as |pdflatex|.
%
% %%%%%%%%%%%%%%%%%%%%%%%%%%%%%%%%%%%%%%%%%%%%%%%%%%%%%%%%%%%%%%%%%%%%%%%%%%%%%%
% %%%%%%%%%%%%%%%%%%%%%%%%%%%%%%%%%%%%%%%%%%%%%%%%%%%%%%%%%%%%%%%%%%%%%%%%%%%%%%
% \section{Implementation}
%\iffalse
%<*package>
%\fi
%
% This section describes the definitions file |childdoc.def|.

% The definitions cannot be loaded using |\usepackage| or |\RequirePackage|
% which has a mechanism to prevent loading a style file more than once.
% When loading the definitions by means of |\input|
% multiple instances have to be prevented manually:
%\iffalse
%This code needs to be before the `\ProvidesFile' directive
%which is defined at the beginning of this file.
%Therefore it is also placed there and commented out here.
%</package>
%<*discard>
%\fi
%    \begin{macrocode}
\ifdefined\childdocmain\endinput\fi
%    \end{macrocode}
%\iffalse
%</discard>
%<*package>
%\fi
%
% \macro{\ifchilddoc}
% \macro{\ifchilddocmanual}
% The conditional |\ifchilddoc| tells whether a
% child (true) or main (false) document is being compiled.
% The conditional |\ifchilddocmanual| tells whether
% the |\includeonly| mechanism is used (false) or
% the selection of child files must be performed manually (true).
% The definitions initialise to false:
%    \begin{macrocode}
\newif\ifchilddoc
\newif\ifchilddocmanual
%    \end{macrocode}

% \macro{\childdocname}
% \macro{\childdocjob}
% The macro |\childdocname| stores the name of the main document
% to be compiled. The macro |\childdocjob| stores the name of
% the document on which the \LaTeX{} compiler was originally invoked.
% The content of |\jobname| cannot be compared
% to filenames specified in the source due to different catcodes.
% The following code rescans |\jobname|, stores the result
% in |\childdocname| and saves a copy in |\childdocjob|:
%    \begin{macrocode}
\edef\childdocname{\scantokens\expandafter{\jobname\noexpand}}
\let\childdocjob\childdocname
%    \end{macrocode}

% \macro{\childdocdisable}
% The macro |\childdocdisable| prevents the main file
% from being processed more than once.
% At this stage, the main document command |\childdocmain|
% is assumed to be called once again where it should do nothing.
% Any subsequent call to it should prevent
% a secondary processing of the main document
% It overwrites the forwarding commands
% |\childdocof| and |\childdocforward|
% with empty macros to prevent further inclusions of the main document:
%    \begin{macrocode}
\newcommand{\childdocdisable}
{
  \renewcommand{\childdocmain}[1]{\renewcommand{\childdocmain}[1]{\endinput}}
  \renewcommand{\childdocof}[1]{}
  \renewcommand{\childdocby}[2][]{}
  \renewcommand{\childdocforward}[2][]{}
  \renewcommand{\childdocdisable}{}
}
%    \end{macrocode}

% \macro{\childdocmain}
% The macro |\childdocmain| is to be called at the top of the main file
% with nothing or the main filename (without extension) as argument.
% First, it breaks loops.
% If the argument is not empty and does not match |\childdocname|
% (which is set by the first inclusion of |childdoc.def|),
% |\ifchilddoc| is set to true, |\includeonly| is applied to the child file
% and |\jobname| is set to the main file
% (for proper handling of |.aux| files):
%    \begin{macrocode}
\newcommand{\childdocmain}[1]
{
  \childdocdisable\childdocmain{}
  \if?#1?\else
    \begingroup
      \def\childdoctmp{#1}
      \ifx\childdoctmp\childdocname
        \def\childdoctmp{}
      \else
        \def\childdoctmp
        {
          \childdoctrue
          \includeonly{\childdocname}
          \def\childdocjob{#1}
          \def\jobname{#1}
        }
      \fi
      \expandafter
    \endgroup
    \childdoctmp
  \fi
}
%    \end{macrocode}

% \macro{\childdocof}
% The command |\childdocof| redirects
% compilation to the main file |#1|.
%    \begin{macrocode}
\newcommand{\childdocof}[1]
{
  \childdocdisable
  \childdoctrue
  \includeonly{\childdocname}
  \def\jobname{#1}
  \def\childdocjob{#1}
  \input{#1}
}
%    \end{macrocode}

% \macro{\childdocby}
% The command |\childdocby| ....
%    \begin{macrocode}
\newcommand{\childdocby}[2][]
{
  \childdocdisable
  \childdoctrue
  \childdocmanualtrue
  \if?#1?\else
    \def\jobname{#2}
  \fi
  \def\childdocjob{#2}
  \input{#2}
  \endinput
}
%    \end{macrocode}

% \macro{\childdocforward}
% The command |\childdocforward| redirects
% compilation to the main file or
% (if the optional argument is given) a child file.
% Parameters are set as if the main file
% or a child file starting with |\childdocof| was compiled.
% Then compilation is handed over to the main file:
%    \begin{macrocode}
\newcommand{\childdocforward}[2][]
{
  \begingroup
    \if?#1?
      \def\childdoctmp
      {
        \def\childdocname{#2}
        \def\childdocjob{#2}
        \def\jobname{#2}
        \input{#2}
        \endinput
      }
    \else
      \def\childdoctmp
      {
        \childdocdisable
        \def\childdocname{#2}
        \childdoctrue
        \includeonly{#2}
        \def\childdocjob{#1}
        \def\jobname{#1}
        \input{#1}
        \endinput
      }
    \fi
    \expandafter
  \endgroup
  \childdoctmp
}
%    \end{macrocode}

% \macro{\childdocforwardprefix}
% The command |\childdocforwardprefix| redirects
% compilation to the main or a child file by means of a pattern.
% The prefix |#1| in the current filename is replaced by |#2|
% and the suffix of the current filename is kept
% (it is assumed that the filename does not contain the substring `|~~~|'
% which is used as a delimiter).
% Compilation is handed over to the new file by |\childdocforward|:
%    \begin{macrocode}
\newcommand{\childdocforwardprefix}[3][]
{
  \begingroup
    \def\childdocextract #2##1~~~{\def\childdoctmp{\childdocforward[#1]{#3##1}}}
    \expandafter\childdocextract\childdocname~~~
    \expandafter
  \endgroup
  \childdoctmp
}
%    \end{macrocode}

% \macro{\childdoc}
% The deprecated macro |\childdoc| is a legacy version of |\childdocmain|:
%    \begin{macrocode}
\newcommand{\childdoc}{\childdocmain}
%    \end{macrocode}

% \macro{\childdocredirect}
% The deprecated macro |\childdocredirect| is a legacy version
% of |\childdocforward| and |\childdocforwardprefix|:
%    \begin{macrocode}
\newcommand{\childdocredirect}[2][]
{
  \begingroup
    \if?#1?
      \def\childdoctmp{\childdocforward{#2}}
    \else
      \def\childdoctmp{\childdocforwardprefix{#1}{#2}}
    \fi
    \expandafter
  \endgroup
  \childdoctmp
}
%    \end{macrocode}

%\iffalse
%</package>
%\fi
%
\endinput
\childdocforward[cdocsamp]{cdocsch1}"|\\
% |latex -jobname cdocscl2 \|\\
% |  "\def\version{final}% \iffalse
%
% childdoc.dtx Copyright (C) 2017-2018 Niklas Beisert
%
% This work may be distributed and/or modified under the
% conditions of the LaTeX Project Public License, either version 1.3
% of this license or (at your option) any later version.
% The latest version of this license is in
%   http://www.latex-project.org/lppl.txt
% and version 1.3 or later is part of all distributions of LaTeX
% version 2005/12/01 or later.
%
% This work has the LPPL maintenance status `maintained'.
%
% The Current Maintainer of this work is Niklas Beisert.
%
% This work consists of the files childdoc.dtx and childdoc.ins
% and the derived files childdoc.def and cdocsamp.tex with
% cdocsch1.tex, cdocsch2.tex, cdocsdrf.tex, cdocsfn1.tex, cdocsfn2.tex.
%
%<package>\ifdefined\childdocmain\endinput\fi
%<package>\ProvidesFile{childdoc.def}[2018/12/30 v2.0 child document driver]
%<samplemain>\ProvidesFile{cdocsamp.tex}[2018/12/30 v2.0 sample for childdoc]
%<*driver>
%\ProvidesFile{childdoc.drv}[2018/12/30 v2.0 childdoc reference manual file]
\PassOptionsToClass{10pt,a4paper}{article}
\documentclass{ltxdoc}

\usepackage[margin=35mm]{geometry}
\usepackage{hyperref}
\usepackage{hyperxmp}
\usepackage[usenames]{color}

\hypersetup{colorlinks=true}
\hypersetup{pdfstartview=FitH}
\hypersetup{pdfpagemode=UseNone}
\hypersetup{pdfsource={}}
\hypersetup{pdflang={en-UK}}
\hypersetup{pdfcopyright={Copyright 2017-2018 Niklas Beisert.
  This work may be distributed and/or modified under the
  conditions of the LaTeX Project Public License, either version 1.3
  of this license or (at your option) any later version.}}
\hypersetup{pdflicenseurl={http://www.latex-project.org/lppl.txt}}
\hypersetup{pdfcontactaddress={ETH Zurich, ITP, HIT K,
  Wolfgang-Pauli-Strasse 27}}
\hypersetup{pdfcontactpostcode={8093}}
\hypersetup{pdfcontactcity={Zurich}}
\hypersetup{pdfcontactcountry={Switzerland}}
\hypersetup{pdfcontactemail={nbeisert@itp.phys.ethz.ch}}
\hypersetup{pdfcontacturl={http://people.phys.ethz.ch/\xmptilde nbeisert/}}

\newcommand{\secref}[1]{\hyperref[#1]{section \ref*{#1}}}

\parskip1ex
\parindent0pt
\let\olditemize\itemize
\def\itemize{\olditemize\parskip0pt}

\begin{document}

\title{The \textsf{childdoc} Package}
\hypersetup{pdftitle={The childdoc Package}}
\author{Niklas Beisert\\[2ex]
  Institut f\"ur Theoretische Physik\\
  Eidgen\"ossische Technische Hochschule Z\"urich\\
  Wolfgang-Pauli-Strasse 27, 8093 Z\"urich, Switzerland\\[1ex]
  \href{mailto:nbeisert@itp.phys.ethz.ch}
  {\texttt{nbeisert@itp.phys.ethz.ch}}}
\hypersetup{pdfauthor={Niklas Beisert}}
\hypersetup{pdfsubject={Manual for the LaTeX2e Package childdoc}}
\date{30 December 2018, \textsf{v2.0}}
\maketitle

\begin{abstract}\noindent
\textsf{childdoc} is a \LaTeXe{} package
that enables the direct compilation
of document sections included by |\include|
to individual files.
\end{abstract}

\begingroup
\parskip0ex
\tableofcontents
\endgroup

%%%%%%%%%%%%%%%%%%%%%%%%%%%%%%%%%%%%%%%%%%%%%%%%%%%%%%%%%%%%%%%%%%%%%%%%%%%%%%%%
%%%%%%%%%%%%%%%%%%%%%%%%%%%%%%%%%%%%%%%%%%%%%%%%%%%%%%%%%%%%%%%%%%%%%%%%%%%%%%%%
\section{Introduction}

\LaTeX{} provides a mechanism to structure a large document (such as a book)
into a main file and several child files (containing the chapters)
using the |\include| command.
This mechanism is beneficial for documents
which span hundreds of pages in order to
make the source file(s) more manageable.
Moreover, compilation can be restricted to
selected child files by means of the |\includeonly| command.
The latter feature can be used to reduce the compilation time while editing
(this was significantly more useful in the earlier days of \LaTeX{})
or to generate a smaller document which is easier to navigate.
Another application of |\includeonly| is to generate
documents consisting of selected parts of the complete document.

However, there are a few drawbacks of the plain |\include| mechanism:
\begin{itemize}
\item
The child files cannot be compiled on their own,
they can only be compiled via the main file.
A naive editing environment
(such as a text editor with an option
to have the current file processed by \LaTeX)
may require one to switch to the main file before compiling;
attempting to compile the child file produces errors.
\item
The main file must be modified (each time)
to adjust the |\includeonly| command
to the present needs. This easily leaves the main file in a messy state.
\item
The generated document will always carry the filename
of the main document. This is inconvenient if
several child files are to be compiled and
to be kept for distribution.
\end{itemize}

The present package provides a simple interface
to make child files individually compilable by \LaTeX{}.
Compiling a child file then has the same effect as compiling
the main file with an |\includeonly| command
to select the appropriate child.
Moreover the generated document will carry the name of the child
rather than the main file.
This resolves all three above issues.

This feature is meant to make the editing of books,
thesis documents and lecture notes somewhat more convenient.
However, the package can also be used efficiently for
composing a series of documents (such as exercise sheets)
which are typically distributed individually.
It then assists the author in generating the individual documents
(potentially in different versions)
as well as a document containing the collected series.
Another application is in developing style files
or other kinds of included material
where compilation of the style file could redirect
to a sample or test file.

%%%%%%%%%%%%%%%%%%%%%%%%%%%%%%%%%%%%%%%%%%%%%%%%%%%%%%%%%%%%%%%%%%%%%%%%%%%%%%%%
%%%%%%%%%%%%%%%%%%%%%%%%%%%%%%%%%%%%%%%%%%%%%%%%%%%%%%%%%%%%%%%%%%%%%%%%%%%%%%%%
\section{Usage}

First of all, the package \textsf{childdoc} is \emph{not} a standard
\LaTeXe{} |.sty| style file! Therefore it needs to be invoked in
a non-standard way.

%%%%%%%%%%%%%%%%%%%%%%%%%%%%%%%%%%%%%%%%%%%%%%%%%%%%%%%%%%%%%%%%%%%%%%%%%%%%%%%%
\subsection{Included Files}
\label{sec:include}

%%%%%%%%%%%%%%%%%%%%%%%%%%%%%%%%%%%%%%%%
\DescribeMacro{\childdocmain}
To use the package, add the commands
\begin{center}
\begin{tabular}{l}
|\input{childdoc.def}|\\
|\childdocmain{}|\\
\end{tabular}
\end{center}
at the very top of the main \LaTeX{} file,
in particular \emph{before} the |\documentclass| statement!
The argument of |\childdocmain| should be left empty
(but it must be present).

%%%%%%%%%%%%%%%%%%%%%%%%%%%%%%%%%%%%%%%%
\DescribeMacro{\childdocof}
Furthermore, add the commands
\begin{center}
\begin{tabular}{l}
|\input{childdoc.def}|\\
|\childdocof{|\textit{main}|}|\\
\end{tabular}
\end{center}
at the top of every child file \textit{child}
which is included by |\include{|\textit{child}|}|
from within the main file
(or at least for those files to be compiled individually).
The argument \textit{main} must be the filename of the main file.

There are a couple of
considerations in setting up the main and child documents:

%%%%%%%%%%%%%%%%%%%%%%%%%%%%%%%%%%%%%%%%
\paragraph{Restrictions.}

Please note the following restrictions:
\begin{itemize}
\item
|\childdocmain| must be called with one argument \textit{main}
to ensure compatibility with earlier version of the package.
It must either be empty (|\childdocmain{}|)
or precisely match the filename of the main file in which it is specified.
See \secref{sec:detection} for further information.
\item
The filename \textit{main} must be specified without the |.tex| extension.
\item
The filename \textit{main} is case sensitive
(even in case-insensitive file systems)
due to internal string comparison.
\item
The argument \textit{main} should be fully expanded, it cannot be a macro.
\item
Subdirectories and special characters should be avoided in filenames.
\item
The command |\childdocmain{|\textit{main}|}| must be followed by a whitespace.
It should not be followed immediately by another command
or by a comment mark `|%|'.
This is because the \TeX{} parser reads the token immediately following
the argument of |\childdocmain| and puts it
at the beginning of every child section;
however, a white\-space is ignored.
\end{itemize}

%%%%%%%%%%%%%%%%%%%%%%%%%%%%%%%%%%%%%%%%
\paragraph{Content of Main File.}

It is advisable to place all content in the child files included by |\include|.
Any output contained in the main file will appear in all child documents
unless suppressed manually;
it cannot be suppressed automatically by the |\includeonly| directive
and thus should normally be avoided.
A method to include some content in the main file
by means of conditional processing is described in \secref{sec:conditional}.

%%%%%%%%%%%%%%%%%%%%%%%%%%%%%%%%%%%%%%%%
\paragraph{Page Numbering.}

When only a part of the document is compiled,
the appropriate numbering of pages
(as well as other status parameters)
is determined from the |.aux| files.
The latter contain information from previous passes.
However this information needs to propagate through
all intermediate child documents.
Therefore the page numbering in child documents may well
be inconsistent until the complete document is compiled at least once.

A useful (if unconventional) way to always ensure a consistent
page numbering is to restart the numbering in each child document
and denote the pages by `\textit{child}|.|\textit{page}'
where \textit{child} represents the chapter/section number of the child file.
This can be achieved by the command
|\numberwithin{page}{|\textit{child}|}|
of the \textsf{amsmath} package
where \textit{child} can be |chapter| or |section|
depending on the chosen structuring.
Alternatively, one can modify the macro |\thepage| appropriately
and reset the counter |page| at the start of each child file.

%%%%%%%%%%%%%%%%%%%%%%%%%%%%%%%%%%%%%%%%%%%%%%%%%%%%%%%%%%%%%%%%%%%%%%%%%%%%%%%%
\subsection{Conditional Processing}
\label{sec:conditional}

The package provides a mechanism to compile different versions
of a document. To customise the versions further some conditional processing
can come in handy to distinguish which version is being compiled.
The package provides two macros to describe the compilation context:

%%%%%%%%%%%%%%%%%%%%%%%%%%%%%%%%%%%%%%%%
\DescribeMacro{\ifchilddoc}
The conditional |\ifchilddoc| distinguishes between the compilation of
child documents and the main document:
%
\begin{center}
|\ifchilddoc |\textit{child-code}| |[|\||else |\textit{main-code}]| \||fi|
\end{center}

%%%%%%%%%%%%%%%%%%%%%%%%%%%%%%%%%%%%%%%%
\DescribeMacro{\childdocname}
\DescribeMacro{\childdocjob}
The macro |\childdocname| contains the filename (without extension)
of the main or child file being processed.
Note that |\childdocjob| will always contain the name of the main file.

%%%%%%%%%%%%%%%%%%%%%%%%%%%%%%%%%%%%%%%%
\paragraph{Title Page.}

Conditional processing can be used to include a title or banner page
in the main document when proper precautions are taken.
Importantly, the code in the main file should ensure that the page counter
(as well as other status parameters which are stored in the |.aux| files)
takes the same value after the conditional processing.
Otherwise the page numbers may take divergent values
depending on which part is compiled.

For example, a title page could be declared by:
%
\begin{center}
\begin{tabular}{l}
|\ifchilddoc\||else|\\
|\addtocounter{page}{-1}|\\
\textit{code for title page}\\
|\newpage|\\
|\||fi|
\end{tabular}
\end{center}
%
A banner page for the child documents can be generated by:
%
\begin{center}
\begin{tabular}{l}
|\ifchilddoc|\\
|\addtocounter{page}{-1}|\\
\textit{code for banner page}\\
|\newpage|\\
|\||fi|
\end{tabular}
\end{center}
%
Here one could write a message such as:
\begin{center}
|This is the part \childdocname{} of \childdocjob{}.|
\end{center}

%%%%%%%%%%%%%%%%%%%%%%%%%%%%%%%%%%%%%%%%%%%%%%%%%%%%%%%%%%%%%%%%%%%%%%%%%%%%%%%%
\subsection{Flags}
\label{sec:flags}

The package makes it easy to generate different versions
of the main or child documents.
To this end compilation flags can be defined
and assigned different default values.
They will be particularly useful in conjunction
with the forwarding mechanism described in \secref{sec:forward}.

For example, it may be useful to have a flag |\version|
which can be set to |draft| or |final|.
The document source will contain some conditional code
depending on the value of |\version|.
Suppose further, the flag should default to |final| for the main file
and to |draft| for child files
which is a natural assignment for editing the document.
This is achieved by placing the following code
in the preamble of the main document
(below the |\childdocmain| directive):
%
\begin{center}
\begin{tabular}{l}
|\ifchilddoc|\\
|\providecommand{\version}{draft}|\\
|\||else|\\
|\providecommand{\version}{final}|\\
|\||fi|
\end{tabular}
\end{center}
%
The definition by |\providecommand| makes sure
that previous definitions are not overwritten.
Further statements |\providecommand{\version}{...}|
can thus be added before the above code to override it.

For the main file, one might add a line
(between |\childdocmain| and the above block)
%
\begin{center}
|%\ifchilddoc\||else\providecommand{\version}{draft}\||fi|
\end{center}
%
which can be uncommented to produce a draft version.
Likewise one can add a line to the very top of a child file
(above the |\childdocof{|\textit{main}|}| directive)
%
\begin{center}
|%\providecommand{\version}{final}|
\end{center}
%
which can be uncommented to produce the final version of this child document.

%%%%%%%%%%%%%%%%%%%%%%%%%%%%%%%%%%%%%%%%%%%%%%%%%%%%%%%%%%%%%%%%%%%%%%%%%%%%%%%%
\subsection{Forwarding}
\label{sec:forward}

Different versions of the main or child documents
using compilation flags as described in \secref{sec:flags}
can be (permanently) stored in different files
for convenient compilation, viewing and distribution.
To this end, the package defines a command
to pass on compilation to a different file:

%%%%%%%%%%%%%%%%%%%%%%%%%%%%%%%%%%%%%%%%
\DescribeMacro{\childdocforward}
The command |\childdocforward| redirects processing to
another source file:
%
\begin{center}
\begin{tabular}{l}
|\input{childdoc.def}|\\
|\childdocforward[|\textit{main}|]{|\textit{dest}|}|\\
\end{tabular}
\end{center}
%
The argument \textit{dest} is the destination file
(without extension).
It should be the main file or one of the child files.
Note that further \textsf{childdoc} directives
such as |\childdocof| and |\childdocforward|
in the indicated file will be processed in this form.
The optional argument \textit{main}
passes on directly to the main file \textit{main}
while pretending to compile the child \textit{dest}.
This form behaves as if \textit{dest}
issues |\childdocof{|\textit{main}|}| right away,
and no further \textsf{childdoc} directives will be processed.

%%%%%%%%%%%%%%%%%%%%%%%%%%%%%%%%%%%%%%%%
\DescribeMacro{\...prefix}
In the alternative form |\childdocforwardprefix|,
%
\begin{center}
\begin{tabular}{l}
|\input{childdoc.def}|\\
|\childdocforwardprefix[|\textit{main}|]{|\textit{prefix}|}{|\textit{dest}|}|
\end{tabular}
\end{center}
%
the destination file is determined by a pattern
depending on the current file:
To make this work, the current file must be called
`{\textit{prefix}\hspace{0.2em}\textit{suffix}}'
with \textit{prefix} matching precisely the argument.
Processing is then passed on to the file
`{\textit{dest}\hspace{0.2em}\textit{suffix}}'.
Surely, the same effect is achieved by
directly specifying the
argument `{\textit{dest}\hspace{0.2em}\textit{suffix}}'
in the first form.
However, that requires to set up a different file
for each child. With the alternative form of the command
all these files can have exactly the same content
which simplifies setting them up and maintaining them.

For example, the following file |draft.tex|
with a compilation flag |\version| as described in \secref{sec:flags}
compiles the main document as a draft:
%
\begin{center}
\begin{tabular}{l}
|\def\version{draft}|\\
|\input{childdoc.def}|\\
|\childdocforward{|\textit{main}|}|
\end{tabular}
\end{center}
%
Likewise, the following files |final|\textit{nn}|.tex|
compile the final version of the child document
|child|\textit{nn}|.tex|:
%
\begin{center}
\begin{tabular}{l}
|\def\version{final}|\\
|\input{childdoc.def}|\\
|\childdocforwardprefix{final}{child}|
\end{tabular}
\end{center}
%

Note that when several versions of a main file and/or of each child file
are to be generated, it may be convenient to set up a |Makefile| or
shell script to automatise the process.

%%%%%%%%%%%%%%%%%%%%%%%%%%%%%%%%%%%%%%%%%%%%%%%%%%%%%%%%%%%%%%%%%%%%%%%%%%%%%%%%
\subsection{Command Line Processing}
\label{sec:commandline}

The effect of redirection files can also be achieved by invoking
the \LaTeX{} compiler with a more elaborate command line.
Most conveniently this should be done as part
of a shell script or a |Makefile|.

When using \textsf{childdoc} in the main file, the following
command lines effectively perform a redirection
(note that depending on the shell being used,
backslashes may have to be doubled: `|\|' $\to$ `|\\|'):
%
\begin{center}
|... -jobname "|\textit{target}|" |\\|"|[\textit{flags}]%
|\input{childdoc.def}\childdocforward[|\textit{main}|]{|\textit{dest}|}"|
\end{center}
%
Here \textit{target} is the name of the output file,
\textit{main} is the name of the main file
and \textit{dest} is the name of the main or child file to be processed
(all filenames without extensions).
The optional argument \textit{main} can be omitted
if \textit{main} matches \textit{dest}.
Optionally, compilation \textit{flags} can be defined via |\def| commands.
This command line makes the \TeX{} engine believe
it is compiling the file \textit{target}
whose content is specified as the latter parameter.
The provided code then forwards the processing to
\textit{main} or \textit{dest} as described in \secref{sec:forward}.

%%%%%%%%%%%%%%%%%%%%%%%%%%%%%%%%%%%%%%%%%%%%%%%%%%%%%%%%%%%%%%%%%%%%%%%%%%%%%%%%
\subsection{Include by Input}
\label{sec:input}

Including child documents by |\include| has some restrictions by design.
Most notably, the content of a child document always occupies
its own set of pages; pages cannot be shared between child documents.
Usually, this behaviour makes perfect sense
because each child document contain an essential part of the document.
However, in some situations it may be desirable to compose
a document from a collection of parts
without having mandatory page breaks between then.
For this case, the package
provides a mechanism to include parts
by |\input| which can also be processed individually.
However, by construction this mechanism
requires manual handling of the content to be output.

%%%%%%%%%%%%%%%%%%%%%%%%%%%%%%%%%%%%%%%%
\DescribeMacro{\ifchilddocmanual}
The main file should be prepared as usual, see \secref{sec:include}.
However, the document body must make a distinction
between processing of an individual part and of the main document, e.g.:
%
\begin{center}
\begin{tabular}{l}
|\ifchilddocmanual|\\
|\input{\childdocname}|\\
|\||else|\\
\textit{document body with }|\input{|\textit{part}|}|\\
|\||fi|
\end{tabular}
\end{center}
%
The conditional |\ifchilddocmanual| is true whenever
a part to be included by |\input| is being compiled,
and the name of the part is stored in |\childdocname|.

%%%%%%%%%%%%%%%%%%%%%%%%%%%%%%%%%%%%%%%%
\DescribeMacro{\childdocby}
Each part to be included by |\input| should start with:
%
\begin{center}
\begin{tabular}{l}
|\input{childdoc.def}|\\
|\childdocby{|\textit{main}|}|\\
\end{tabular}
\end{center}
%
The directive |\childdocby| is similar to |\childdocof|
described in \secref{sec:include},
but the subsequent selection of content must be done manually.
To that end, both |\ifchilddoc| and |\ifchilddocmanual|
will be true upon processing of a part,
and the name of the part is stored in |\childdocname|.
Note that |\jobname| will be set to the filename of the current part
so that each part receives an individual |.aux| file
that does not interfere with the |.aux| file(s) of the main document.
This behaviour can be altered by the alternative form
|\childdocby[*]{|\textit{main}|}| (with a non-empty optional argument)
which uses the |.aux| file of the main document
by setting |\jobname| to \textit{main}.

%%%%%%%%%%%%%%%%%%%%%%%%%%%%%%%%%%%%%%%%%%%%%%%%%%%%%%%%%%%%%%%%%%%%%%%%%%%%%%%%
\subsection{Driver Development}
\label{sec:driver}

The \textsf{childdoc} mechanism can also be use for the development
of definition files such as \LaTeX{} styles or classes.
This case differs from the above setup with multiple parts
included by |\include| in that no |\includeonly| should be invoked.
This can be achieved by starting the include file
(before |\ProvidesPackage|) with:
%
\begin{center}
\begin{tabular}{l}
|\input{childdoc.def}|\\
|\childdocforward{|\textit{main}|}|\\
\end{tabular}
\end{center}
%
or alternatively with:
%
\begin{center}
\begin{tabular}{l}
|\input{childdoc.def}|\\
|\childdocby{|\textit{main}|}|\\
\end{tabular}
\end{center}
%
Both forms have slightly different effects as described above.
The main file is prepared as usual, see \secref{sec:include}.

%%%%%%%%%%%%%%%%%%%%%%%%%%%%%%%%%%%%%%%%%%%%%%%%%%%%%%%%%%%%%%%%%%%%%%%%%%%%%%%%
\subsection{Legacy Detection}
\label{sec:detection}

The directive |\childdocmain| in the main file can detect
whether the complete document or merely a child is to be compiled
even without using the directive |\childdocof|.
This method is deprecated because it is less robust
and there is no compelling reason to use it;
it is merely provided for backward compatibility
and it may be removed in future versions.

If the detection mechanism is to be used,
it is mandatory to correctly specify
the filename of the main file as the argument of |\childdocmain|:
%
\begin{center}
\begin{tabular}{l}
|\input{childdoc.def}|\\
|\childdocmain{|\textit{main}|}|\\
\end{tabular}
\end{center}
%
If |\jobname| does not match the argument \textit{main} of |\childdocmain|,
it is assumed that |\jobname| points to the child file to be compiled.
When using |\childdocmain| with the main file specified as argument,
it suffices to start a child file
with just |\input{|\textit{main}|}|
without loading of the package and using |\childdocof|.
If instead all processing is done
with the appropriate \textsf{childdoc} directives,
the argument of \textit{main} of |\childdocmain| can be empty.

An alternative version of the command line processing described
in \secref{sec:commandline} using the detection mechanism reads:
%
\begin{center}
|... -jobname "|\textit{target}|" "|[\textit{flags}]%
[|\def\jobname{|\textit{dest}|}|]|\input{|\textit{main}|}"|
\end{center}

%%%%%%%%%%%%%%%%%%%%%%%%%%%%%%%%%%%%%%%%%%%%%%%%%%%%%%%%%%%%%%%%%%%%%%%%%%%%%%%%
\subsection{Manual Code}
\label{sec:manual}

In case one cannot be certain whether the definitions file |childdoc.def|
is installed on the target \TeX{} distribution
and one prefers not to ship it,
it is conceivable to paste a few relevant commands into the sources.

To that end, drop all statements |\input{childdoc.def}|
and perform the replacements as outlined below.
Instead of |\childdocmain{|\textit{main}|}| add the following code
to the top of the main file:
%
\begin{center}
\begin{tabular}{l}
|\||ifdefined\childdocname\endinput\||fi\newif\ifchilddoc|\\
|\edef\childdocname{\scantokens\expandafter{\jobname\noexpand}}|\\
|\def\childdocmain{|\textit{main}|}\||ifx\childdocmain\childdocname\||else|\\
|\childdoctrue\includeonly{\childdocname}\let\jobname\childdocmain\||fi|\\
\end{tabular}
\end{center}
%
Instead of |\childdocof{|\textit{main}|}| just include the main file
at the top of each child file:
%
\begin{center}
|\input{|\textit{main}|}|
\end{center}
%
A simple redirection |\childdocforward{|\textit{dest}|}| is achieved by:
%
\begin{center}
|\def\jobname{|\textit{dest}|}\input{\jobname}|
\end{center}
%
The redirection with prefix
|\childdocforwardprefix[|\textit{prefix}|]{|\textit{dest}|}|
is accomplished by:
%
\begin{center}
\begin{tabular}{l}
|{\edef\jobname{\scantokens\expandafter{\jobname\noexpand}}|\\
|\def\redirectjob |\textit{prefix}|#1~~~{\gdef\jobname{|\textit{dest}|#1}}|\\
|\expandafter\redirectjob\jobname~~~}\input{\jobname}|
\end{tabular}
\end{center}

In an alternative approach,
child documents can be compiled by a specific command line
without additional code or specific definitions:
%
\begin{center}
|... -jobname "|\textit{target}|" "|[\textit{flags}]%
|\includeonly{|\textit{dest}|}\input{|\textit{main}|}"|
\end{center}
%

%%%%%%%%%%%%%%%%%%%%%%%%%%%%%%%%%%%%%%%%%%%%%%%%%%%%%%%%%%%%%%%%%%%%%%%%%%%%%%%%
%%%%%%%%%%%%%%%%%%%%%%%%%%%%%%%%%%%%%%%%%%%%%%%%%%%%%%%%%%%%%%%%%%%%%%%%%%%%%%%%
\section{Information}

%%%%%%%%%%%%%%%%%%%%%%%%%%%%%%%%%%%%%%%%%%%%%%%%%%%%%%%%%%%%%%%%%%%%%%%%%%%%%%%%
\subsection{Copyright}

Copyright \copyright{} 2017--2018 Niklas Beisert

This work may be distributed and/or modified under the
conditions of the \LaTeX{} Project Public License, either version 1.3
of this license or (at your option) any later version.
The latest version of this license is in
  \url{http://www.latex-project.org/lppl.txt}
and version 1.3 or later is part of all distributions of \LaTeX{}
version 2005/12/01 or later.

This work has the LPPL maintenance status `maintained'.

The Current Maintainer of this work is Niklas Beisert.

This work consists of the files |README.txt|, |childdoc.ins| and |childdoc.dtx|
as well as the derived files |childdoc.def|, |cdocsamp.tex|
with |cdocsch1.tex|, |cdocsch2.tex|, |cdocspt3.tex|, |cdocspt4.tex|,
|cdocsdrf.tex|, |cdocsfn1.tex|, |cdocsfn2.tex|
as well as |childdoc.pdf|.

%%%%%%%%%%%%%%%%%%%%%%%%%%%%%%%%%%%%%%%%%%%%%%%%%%%%%%%%%%%%%%%%%%%%%%%%%%%%%%%%
\subsection{Files and Installation}

The package consists of the files:
%
\begin{center}
\begin{tabular}{ll}
    |README.txt|   & readme file \\
    |childdoc.ins| & installation file \\
    |childdoc.dtx| & source file \\
    |childdoc.def| & definition file \\
    |cdocsamp.tex| & sample main file \\
    |cdocsch1.tex| & sample include file \\
    |cdocsch2.tex| & sample include file \\
    |cdocspt3.tex| & sample part file \\
    |cdocspt4.tex| & sample part file \\
    |cdocsdrf.tex| & sample redirection file \\
    |cdocsfn1.tex| & sample redirection file \\
    |cdocsfn2.tex| & sample redirection file \\
    |childdoc.pdf| & manual
\end{tabular}
\end{center}
%
The distribution consists of the files
|README.txt|, |childdoc.ins| and |childdoc.dtx|.
%
\begin{itemize}
\item
Run (pdf)\LaTeX{} on |childdoc.dtx|
to compile the manual |childdoc.pdf| (this file).
\item
Run \LaTeX{} on |childdoc.ins| to create the definitions file |childdoc.def|
and the sample |cdocsamp.tex| with include files
|cdocsch1.tex|, |cdocsch2.tex|, |cdocspt3.tex|, |cdocspt4.tex|,
|cdocsdrf.tex|, |cdocsfn1.tex|, |cdocsfn2.tex|.
Then copy the file |childdoc.def| to an appropriate directory of your \LaTeX{}
distribution, e.g.\ \textit{texmf-root}|/tex/latex/childdoc|.
\end{itemize}

%%%%%%%%%%%%%%%%%%%%%%%%%%%%%%%%%%%%%%%%%%%%%%%%%%%%%%%%%%%%%%%%%%%%%%%%%%%%%%%%
\subsection{Related CTAN Packages}

There are several other packages which offer a similar functionality:
%
\begin{itemize}
\item
The packages
\href{http://ctan.org/pkg/docmute}{\textsf{docmute}},
\href{http://ctan.org/pkg/includex}{\textsf{includex}} and
\href{http://ctan.org/pkg/standalone}{\textsf{standalone}}
provide commands to include only the document body of
a child file thus allowing both files to be compiled individually.
\item
The packages \href{http://ctan.org/pkg/subdocs}{\textsf{subdocs}}
and \href{http://ctan.org/pkg/subfiles}{\textsf{subfiles}}
provide structures in which the main and child documents can be
encapsulated and allowing them to be compiled individually.
The inclusion mechanism is different from the conventional |\include|.
\item
The package \href{http://ctan.org/pkg/combine}{\textsf{combine}}
is an elaborate solution to combine several documents into one.
\end{itemize}
%
See also the CTAN topic \href{http://ctan.org/topic/subdocs}{\textsf{subdocs}}
for further related packages.
The present package differs from the above solutions in that
a document structure constructed with the conventional |\include| mechanism
just needs two extra commands at the top of every file
such that all constituent files can be compiled individually.

%%%%%%%%%%%%%%%%%%%%%%%%%%%%%%%%%%%%%%%%%%%%%%%%%%%%%%%%%%%%%%%%%%%%%%%%%%%%%%%%
%\subsection{Feature Suggestions}
%
%The following is a list of features which may be useful for future
%versions of this package:
%%
%\begin{itemize}
%\item
%\ldots
%\end{itemize}

%%%%%%%%%%%%%%%%%%%%%%%%%%%%%%%%%%%%%%%%%%%%%%%%%%%%%%%%%%%%%%%%%%%%%%%%%%%%%%%%
\subsection{Revision History}

%%%%%%%%%%%%%%%%%%%%%%%%%%%%%%%%%%%%%%%%
\paragraph{v2.0:} 2018/12/30

\begin{itemize}
\item
immediate forward processing
\item
added |\childdocby| mechanism
\item
manual restructured
\end{itemize}

%%%%%%%%%%%%%%%%%%%%%%%%%%%%%%%%%%%%%%%%
\paragraph{v1.6:} 2018/01/17

\begin{itemize}
\item
application for development of include files
\item
corrections to manual
\end{itemize}

%%%%%%%%%%%%%%%%%%%%%%%%%%%%%%%%%%%%%%%%
\paragraph{v1.5:} 2017/05/21

\begin{itemize}
\item
more complete structuring introduced
\item
|\childdocof| introduced
\item
|\childdoc| renamed to |\childdocmain|
\item
|\childredirect| renamed to |\childdocforward| and |\childdocforwardprefix|
and functionality expanded
\end{itemize}

%%%%%%%%%%%%%%%%%%%%%%%%%%%%%%%%%%%%%%%%
\paragraph{v1.0:} 2017/04/27

\begin{itemize}
\item
manual and install package
\item
first version published on CTAN
\end{itemize}

%%%%%%%%%%%%%%%%%%%%%%%%%%%%%%%%%%%%%%%%
\paragraph{v0.6:} 2017/04/26

\begin{itemize}
\item
redirection mechanism added
\end{itemize}

%%%%%%%%%%%%%%%%%%%%%%%%%%%%%%%%%%%%%%%%
\paragraph{v0.5:} 2017/04/26

\begin{itemize}
\item
functionality in definition file
\end{itemize}


%%%%%%%%%%%%%%%%%%%%%%%%%%%%%%%%%%%%%%%%%%%%%%%%%%%%%%%%%%%%%%%%%%%%%%%%%%%%%%%%
%%%%%%%%%%%%%%%%%%%%%%%%%%%%%%%%%%%%%%%%%%%%%%%%%%%%%%%%%%%%%%%%%%%%%%%%%%%%%%%%
%%%%%%%%%%%%%%%%%%%%%%%%%%%%%%%%%%%%%%%%%%%%%%%%%%%%%%%%%%%%%%%%%%%%%%%%%%%%%%%%
\appendix

\settowidth\MacroIndent{\rmfamily\scriptsize 000\ }

 \DocInput{childdoc.dtx}

\end{document}
%</driver>
% \fi
%
% %%%%%%%%%%%%%%%%%%%%%%%%%%%%%%%%%%%%%%%%%%%%%%%%%%%%%%%%%%%%%%%%%%%%%%%%%%%%%%
% %%%%%%%%%%%%%%%%%%%%%%%%%%%%%%%%%%%%%%%%%%%%%%%%%%%%%%%%%%%%%%%%%%%%%%%%%%%%%%
% \section{Sample}
%\iffalse
%<*samplemain>
%\fi
%
% The following presents a sample document
% with two chapters, two parts, a title page,
% a compile flag as well as three forwarding files to set the flag.
% It consists of eight |.tex| files:
% \begin{center}
% \begin{tabular}{ll}
% |cdocsamp.tex|&main file\\
% |cdocsch1.tex|&include file for chapter 1\\
% |cdocsch2.tex|&include file for chapter 2\\
% |cdocspt3.tex|&include file for part 3\\
% |cdocspt4.tex|&include file for part 4\\
% |cdocsdrf.tex|&forwarding file for main file in draft mode\\
% |cdocsfi1.tex|&forwarding file for final version of chapter 1\\
% |cdocsfi2.tex|&forwarding file for final version of chapter 2\\
% \end{tabular}
% \end{center}
% Each of the eight files can be compiled directly by the \LaTeX{} compiler.
%
% %%%%%%%%%%%%%%%%%%%%%%%%%%%%%%%%%%%%%%
% \paragraph{Main File.}
%
% The main file is called |cdocsamp.tex|.
%
% Load the \textsf{childdoc} definitions and
% declare the filename for the main document:
%    \begin{macrocode}
\input{childdoc.def}
\childdocmain{}
%    \end{macrocode}

% Optional override for |\version| flag:
%    \begin{macrocode}
%%\ifchilddoc\else\providecommand{\version}{draft}\fi
%    \end{macrocode}

% Define the default values for the |\version| flag
% (|final| for the main file and |draft| for childs):
%    \begin{macrocode}
\ifchilddoc
\providecommand{\version}{draft}
\else
\providecommand{\version}{final}
\fi
%    \end{macrocode}

% Load the standard document class:
%    \begin{macrocode}
\documentclass[12pt]{article}
%    \end{macrocode}

% Start the document body:
%    \begin{macrocode}
\begin{document}
%    \end{macrocode}

% Declare a title page.
% Print title, part of document being processed and version flag:
%    \begin{macrocode}
\addtocounter{page}{-1}
\begin{center}
{\LARGE\bfseries{}childdoc example\par}
\vspace{1cm}
\ifchilddoc
\ifchilddocmanual part\else chapter\fi:
`\childdocname' of `\childdocjob'\par
\else
main document: `\childdocjob'\par
\fi
version: \version\par
\end{center}
\newpage
%    \end{macrocode}

% Manually include selected file,
% otherwise process as usual:
%    \begin{macrocode}
\ifchilddocmanual
\section*{part `\childdocname'}
\input{\childdocname}
\else
%    \end{macrocode}

% Include the two chapters:
%    \begin{macrocode}
\include{cdocsch1}
\include{cdocsch2}
%    \end{macrocode}

% Include the two parts unless only chapters should be displayed:
%    \begin{macrocode}
\ifchilddoc\else
\section{part three}
\input{cdocspt3}
\section{part four}
\input{cdocspt4}
\fi
%    \end{macrocode}

% Process as usual until here:
%    \begin{macrocode}
\fi
%    \end{macrocode}

% End of document body:
%    \begin{macrocode}
\end{document}
%    \end{macrocode}
%\iffalse
%</samplemain>
%\fi
%
% %%%%%%%%%%%%%%%%%%%%%%%%%%%%%%%%%%%%%%
% \paragraph{Chapter Include Files.}
%
% The include files are called |cdocsch1.tex| and |cdocsch2.tex|.
%
%\iffalse
%<*samplechap1|samplechap2>
%\fi

% Optional override for |\version| flag:
%    \begin{macrocode}
%%\providecommand{\version}{final}
%    \end{macrocode}

% Include the main document:
%    \begin{macrocode}
\input{childdoc.def}
\childdocof{cdocsamp}
%    \end{macrocode}

%\iffalse
%</samplechap1|samplechap2>
%\fi
%
%\iffalse
%<*samplechap1>
%\fi
% Some text for chapter 1:
%    \begin{macrocode}
\section{one}
some text in chapter one
%    \end{macrocode}

%\iffalse
%</samplechap1>
%\fi
% Some text for chapter 2:
%\iffalse
%<*samplechap2>
%\fi
%    \begin{macrocode}
\section{two}
more text in chapter two
%    \end{macrocode}

%\iffalse
%</samplechap2>
%\fi
%
% %%%%%%%%%%%%%%%%%%%%%%%%%%%%%%%%%%%%%%
% \paragraph{Part Include Files.}
%
% The include files are called |cdocspt3.tex| and |cdocspt4.tex|.
%
%\iffalse
%<*samplepart3|samplepart4>
%\fi

% Optional override for |\version| flag:
%    \begin{macrocode}
%%\providecommand{\version}{final}
%    \end{macrocode}

% Include the main document:
%    \begin{macrocode}
\input{childdoc.def}
\childdocby{cdocsamp}
%    \end{macrocode}

%\iffalse
%</samplepart3|samplepart4>
%\fi
%
%\iffalse
%<*samplepart3>
%\fi
% Some text for part 3:
%    \begin{macrocode}
some text in part three
%    \end{macrocode}

%\iffalse
%</samplepart3>
%\fi
% Some text for part 4:
%\iffalse
%<*samplepart4>
%\fi
%    \begin{macrocode}
more text in part four
%    \end{macrocode}

%\iffalse
%</samplepart4>
%\fi
%
% %%%%%%%%%%%%%%%%%%%%%%%%%%%%%%%%%%%%%%
% \paragraph{Forwarding for a Complete Draft.}
%
% The following forwarding file |cdocsdrf.tex|
% compiles the main document in draft mode:
%\iffalse
%<*sampledraft>
%\fi
%    \begin{macrocode}
\def\version{draft}
\input{childdoc.def}
\childdocforward{cdocsamp}
%    \end{macrocode}

%\iffalse
%</sampledraft>
%\fi
%
% %%%%%%%%%%%%%%%%%%%%%%%%%%%%%%%%%%%%%%
% \paragraph{Forwarding for Final Version of the Chapters.}
%
% The following forwarding files |cdocsfn1.tex| and |cdocsfn2.tex|
% (with identical content)
% compile the final versions of the child documents
% |cdocsch1.tex| and |cdocsch2.tex|, respectively:
%\iffalse
%<*samplefinal>
%\fi
%    \begin{macrocode}
\def\version{final}
\input{childdoc.def}
\childdocforwardprefix[cdocsamp]{cdocsfn}{cdocsch}
%    \end{macrocode}

%\iffalse
%</samplefinal>
%\fi
%
% %%%%%%%%%%%%%%%%%%%%%%%%%%%%%%%%%%%%%%
% \paragraph{Command Line Processing.}
%
% The following three command lines generate the output files
% |cdocscld|, |cdocscl1| and |cdocscl2|
% which should be identical to
% |cdocsdrf|, |cdocsch1| and |cdocsfn2|, respectively:
% \begin{center}
% \begin{tabular}{l}
% |latex -jobname cdocscld \|\\
% |  "\def\version{draft}\input{childdoc.def}\childdocforward{cdocsamp}"|\\
% |latex -jobname cdocscl1 \|\\
% |  "\input{childdoc.def}\childdocforward[cdocsamp]{cdocsch1}"|\\
% |latex -jobname cdocscl2 \|\\
% |  "\def\version{final}\input{childdoc.def}\childdocforward{cdocsch2}"|
% \end{tabular}
% \end{center}
% Note that the trailing backslash on each first line
% merely continues the input to the second line
% (for convenient cut ant paste).
% Furthermore, the command |latex| can be replaced by any
% of its alternative versions such as |pdflatex|.
%
% %%%%%%%%%%%%%%%%%%%%%%%%%%%%%%%%%%%%%%%%%%%%%%%%%%%%%%%%%%%%%%%%%%%%%%%%%%%%%%
% %%%%%%%%%%%%%%%%%%%%%%%%%%%%%%%%%%%%%%%%%%%%%%%%%%%%%%%%%%%%%%%%%%%%%%%%%%%%%%
% \section{Implementation}
%\iffalse
%<*package>
%\fi
%
% This section describes the definitions file |childdoc.def|.

% The definitions cannot be loaded using |\usepackage| or |\RequirePackage|
% which has a mechanism to prevent loading a style file more than once.
% When loading the definitions by means of |\input|
% multiple instances have to be prevented manually:
%\iffalse
%This code needs to be before the `\ProvidesFile' directive
%which is defined at the beginning of this file.
%Therefore it is also placed there and commented out here.
%</package>
%<*discard>
%\fi
%    \begin{macrocode}
\ifdefined\childdocmain\endinput\fi
%    \end{macrocode}
%\iffalse
%</discard>
%<*package>
%\fi
%
% \macro{\ifchilddoc}
% \macro{\ifchilddocmanual}
% The conditional |\ifchilddoc| tells whether a
% child (true) or main (false) document is being compiled.
% The conditional |\ifchilddocmanual| tells whether
% the |\includeonly| mechanism is used (false) or
% the selection of child files must be performed manually (true).
% The definitions initialise to false:
%    \begin{macrocode}
\newif\ifchilddoc
\newif\ifchilddocmanual
%    \end{macrocode}

% \macro{\childdocname}
% \macro{\childdocjob}
% The macro |\childdocname| stores the name of the main document
% to be compiled. The macro |\childdocjob| stores the name of
% the document on which the \LaTeX{} compiler was originally invoked.
% The content of |\jobname| cannot be compared
% to filenames specified in the source due to different catcodes.
% The following code rescans |\jobname|, stores the result
% in |\childdocname| and saves a copy in |\childdocjob|:
%    \begin{macrocode}
\edef\childdocname{\scantokens\expandafter{\jobname\noexpand}}
\let\childdocjob\childdocname
%    \end{macrocode}

% \macro{\childdocdisable}
% The macro |\childdocdisable| prevents the main file
% from being processed more than once.
% At this stage, the main document command |\childdocmain|
% is assumed to be called once again where it should do nothing.
% Any subsequent call to it should prevent
% a secondary processing of the main document
% It overwrites the forwarding commands
% |\childdocof| and |\childdocforward|
% with empty macros to prevent further inclusions of the main document:
%    \begin{macrocode}
\newcommand{\childdocdisable}
{
  \renewcommand{\childdocmain}[1]{\renewcommand{\childdocmain}[1]{\endinput}}
  \renewcommand{\childdocof}[1]{}
  \renewcommand{\childdocby}[2][]{}
  \renewcommand{\childdocforward}[2][]{}
  \renewcommand{\childdocdisable}{}
}
%    \end{macrocode}

% \macro{\childdocmain}
% The macro |\childdocmain| is to be called at the top of the main file
% with nothing or the main filename (without extension) as argument.
% First, it breaks loops.
% If the argument is not empty and does not match |\childdocname|
% (which is set by the first inclusion of |childdoc.def|),
% |\ifchilddoc| is set to true, |\includeonly| is applied to the child file
% and |\jobname| is set to the main file
% (for proper handling of |.aux| files):
%    \begin{macrocode}
\newcommand{\childdocmain}[1]
{
  \childdocdisable\childdocmain{}
  \if?#1?\else
    \begingroup
      \def\childdoctmp{#1}
      \ifx\childdoctmp\childdocname
        \def\childdoctmp{}
      \else
        \def\childdoctmp
        {
          \childdoctrue
          \includeonly{\childdocname}
          \def\childdocjob{#1}
          \def\jobname{#1}
        }
      \fi
      \expandafter
    \endgroup
    \childdoctmp
  \fi
}
%    \end{macrocode}

% \macro{\childdocof}
% The command |\childdocof| redirects
% compilation to the main file |#1|.
%    \begin{macrocode}
\newcommand{\childdocof}[1]
{
  \childdocdisable
  \childdoctrue
  \includeonly{\childdocname}
  \def\jobname{#1}
  \def\childdocjob{#1}
  \input{#1}
}
%    \end{macrocode}

% \macro{\childdocby}
% The command |\childdocby| ....
%    \begin{macrocode}
\newcommand{\childdocby}[2][]
{
  \childdocdisable
  \childdoctrue
  \childdocmanualtrue
  \if?#1?\else
    \def\jobname{#2}
  \fi
  \def\childdocjob{#2}
  \input{#2}
  \endinput
}
%    \end{macrocode}

% \macro{\childdocforward}
% The command |\childdocforward| redirects
% compilation to the main file or
% (if the optional argument is given) a child file.
% Parameters are set as if the main file
% or a child file starting with |\childdocof| was compiled.
% Then compilation is handed over to the main file:
%    \begin{macrocode}
\newcommand{\childdocforward}[2][]
{
  \begingroup
    \if?#1?
      \def\childdoctmp
      {
        \def\childdocname{#2}
        \def\childdocjob{#2}
        \def\jobname{#2}
        \input{#2}
        \endinput
      }
    \else
      \def\childdoctmp
      {
        \childdocdisable
        \def\childdocname{#2}
        \childdoctrue
        \includeonly{#2}
        \def\childdocjob{#1}
        \def\jobname{#1}
        \input{#1}
        \endinput
      }
    \fi
    \expandafter
  \endgroup
  \childdoctmp
}
%    \end{macrocode}

% \macro{\childdocforwardprefix}
% The command |\childdocforwardprefix| redirects
% compilation to the main or a child file by means of a pattern.
% The prefix |#1| in the current filename is replaced by |#2|
% and the suffix of the current filename is kept
% (it is assumed that the filename does not contain the substring `|~~~|'
% which is used as a delimiter).
% Compilation is handed over to the new file by |\childdocforward|:
%    \begin{macrocode}
\newcommand{\childdocforwardprefix}[3][]
{
  \begingroup
    \def\childdocextract #2##1~~~{\def\childdoctmp{\childdocforward[#1]{#3##1}}}
    \expandafter\childdocextract\childdocname~~~
    \expandafter
  \endgroup
  \childdoctmp
}
%    \end{macrocode}

% \macro{\childdoc}
% The deprecated macro |\childdoc| is a legacy version of |\childdocmain|:
%    \begin{macrocode}
\newcommand{\childdoc}{\childdocmain}
%    \end{macrocode}

% \macro{\childdocredirect}
% The deprecated macro |\childdocredirect| is a legacy version
% of |\childdocforward| and |\childdocforwardprefix|:
%    \begin{macrocode}
\newcommand{\childdocredirect}[2][]
{
  \begingroup
    \if?#1?
      \def\childdoctmp{\childdocforward{#2}}
    \else
      \def\childdoctmp{\childdocforwardprefix{#1}{#2}}
    \fi
    \expandafter
  \endgroup
  \childdoctmp
}
%    \end{macrocode}

%\iffalse
%</package>
%\fi
%
\endinput
\childdocforward{cdocsch2}"|
% \end{tabular}
% \end{center}
% Note that the trailing backslash on each first line
% merely continues the input to the second line
% (for convenient cut ant paste).
% Furthermore, the command |latex| can be replaced by any
% of its alternative versions such as |pdflatex|.
%
% %%%%%%%%%%%%%%%%%%%%%%%%%%%%%%%%%%%%%%%%%%%%%%%%%%%%%%%%%%%%%%%%%%%%%%%%%%%%%%
% %%%%%%%%%%%%%%%%%%%%%%%%%%%%%%%%%%%%%%%%%%%%%%%%%%%%%%%%%%%%%%%%%%%%%%%%%%%%%%
% \section{Implementation}
%\iffalse
%<*package>
%\fi
%
% This section describes the definitions file |childdoc.def|.

% The definitions cannot be loaded using |\usepackage| or |\RequirePackage|
% which has a mechanism to prevent loading a style file more than once.
% When loading the definitions by means of |\input|
% multiple instances have to be prevented manually:
%\iffalse
%This code needs to be before the `\ProvidesFile' directive
%which is defined at the beginning of this file.
%Therefore it is also placed there and commented out here.
%</package>
%<*discard>
%\fi
%    \begin{macrocode}
\ifdefined\childdocmain\endinput\fi
%    \end{macrocode}
%\iffalse
%</discard>
%<*package>
%\fi
%
% \macro{\ifchilddoc}
% \macro{\ifchilddocmanual}
% The conditional |\ifchilddoc| tells whether a
% child (true) or main (false) document is being compiled.
% The conditional |\ifchilddocmanual| tells whether
% the |\includeonly| mechanism is used (false) or
% the selection of child files must be performed manually (true).
% The definitions initialise to false:
%    \begin{macrocode}
\newif\ifchilddoc
\newif\ifchilddocmanual
%    \end{macrocode}

% \macro{\childdocname}
% \macro{\childdocjob}
% The macro |\childdocname| stores the name of the main document
% to be compiled. The macro |\childdocjob| stores the name of
% the document on which the \LaTeX{} compiler was originally invoked.
% The content of |\jobname| cannot be compared
% to filenames specified in the source due to different catcodes.
% The following code rescans |\jobname|, stores the result
% in |\childdocname| and saves a copy in |\childdocjob|:
%    \begin{macrocode}
\edef\childdocname{\scantokens\expandafter{\jobname\noexpand}}
\let\childdocjob\childdocname
%    \end{macrocode}

% \macro{\childdocdisable}
% The macro |\childdocdisable| prevents the main file
% from being processed more than once.
% At this stage, the main document command |\childdocmain|
% is assumed to be called once again where it should do nothing.
% Any subsequent call to it should prevent
% a secondary processing of the main document
% It overwrites the forwarding commands
% |\childdocof| and |\childdocforward|
% with empty macros to prevent further inclusions of the main document:
%    \begin{macrocode}
\newcommand{\childdocdisable}
{
  \renewcommand{\childdocmain}[1]{\renewcommand{\childdocmain}[1]{\endinput}}
  \renewcommand{\childdocof}[1]{}
  \renewcommand{\childdocby}[2][]{}
  \renewcommand{\childdocforward}[2][]{}
  \renewcommand{\childdocdisable}{}
}
%    \end{macrocode}

% \macro{\childdocmain}
% The macro |\childdocmain| is to be called at the top of the main file
% with nothing or the main filename (without extension) as argument.
% First, it breaks loops.
% If the argument is not empty and does not match |\childdocname|
% (which is set by the first inclusion of |childdoc.def|),
% |\ifchilddoc| is set to true, |\includeonly| is applied to the child file
% and |\jobname| is set to the main file
% (for proper handling of |.aux| files):
%    \begin{macrocode}
\newcommand{\childdocmain}[1]
{
  \childdocdisable\childdocmain{}
  \if?#1?\else
    \begingroup
      \def\childdoctmp{#1}
      \ifx\childdoctmp\childdocname
        \def\childdoctmp{}
      \else
        \def\childdoctmp
        {
          \childdoctrue
          \includeonly{\childdocname}
          \def\childdocjob{#1}
          \def\jobname{#1}
        }
      \fi
      \expandafter
    \endgroup
    \childdoctmp
  \fi
}
%    \end{macrocode}

% \macro{\childdocof}
% The command |\childdocof| redirects
% compilation to the main file |#1|.
%    \begin{macrocode}
\newcommand{\childdocof}[1]
{
  \childdocdisable
  \childdoctrue
  \includeonly{\childdocname}
  \def\jobname{#1}
  \def\childdocjob{#1}
  \input{#1}
}
%    \end{macrocode}

% \macro{\childdocby}
% The command |\childdocby| ....
%    \begin{macrocode}
\newcommand{\childdocby}[2][]
{
  \childdocdisable
  \childdoctrue
  \childdocmanualtrue
  \if?#1?\else
    \def\jobname{#2}
  \fi
  \def\childdocjob{#2}
  \input{#2}
  \endinput
}
%    \end{macrocode}

% \macro{\childdocforward}
% The command |\childdocforward| redirects
% compilation to the main file or
% (if the optional argument is given) a child file.
% Parameters are set as if the main file
% or a child file starting with |\childdocof| was compiled.
% Then compilation is handed over to the main file:
%    \begin{macrocode}
\newcommand{\childdocforward}[2][]
{
  \begingroup
    \if?#1?
      \def\childdoctmp
      {
        \def\childdocname{#2}
        \def\childdocjob{#2}
        \def\jobname{#2}
        \input{#2}
        \endinput
      }
    \else
      \def\childdoctmp
      {
        \childdocdisable
        \def\childdocname{#2}
        \childdoctrue
        \includeonly{#2}
        \def\childdocjob{#1}
        \def\jobname{#1}
        \input{#1}
        \endinput
      }
    \fi
    \expandafter
  \endgroup
  \childdoctmp
}
%    \end{macrocode}

% \macro{\childdocforwardprefix}
% The command |\childdocforwardprefix| redirects
% compilation to the main or a child file by means of a pattern.
% The prefix |#1| in the current filename is replaced by |#2|
% and the suffix of the current filename is kept
% (it is assumed that the filename does not contain the substring `|~~~|'
% which is used as a delimiter).
% Compilation is handed over to the new file by |\childdocforward|:
%    \begin{macrocode}
\newcommand{\childdocforwardprefix}[3][]
{
  \begingroup
    \def\childdocextract #2##1~~~{\def\childdoctmp{\childdocforward[#1]{#3##1}}}
    \expandafter\childdocextract\childdocname~~~
    \expandafter
  \endgroup
  \childdoctmp
}
%    \end{macrocode}

% \macro{\childdoc}
% The deprecated macro |\childdoc| is a legacy version of |\childdocmain|:
%    \begin{macrocode}
\newcommand{\childdoc}{\childdocmain}
%    \end{macrocode}

% \macro{\childdocredirect}
% The deprecated macro |\childdocredirect| is a legacy version
% of |\childdocforward| and |\childdocforwardprefix|:
%    \begin{macrocode}
\newcommand{\childdocredirect}[2][]
{
  \begingroup
    \if?#1?
      \def\childdoctmp{\childdocforward{#2}}
    \else
      \def\childdoctmp{\childdocforwardprefix{#1}{#2}}
    \fi
    \expandafter
  \endgroup
  \childdoctmp
}
%    \end{macrocode}

%\iffalse
%</package>
%\fi
%
\endinput
\childdocforward[|\textit{main}|]{|\textit{dest}|}"|
\end{center}
%
Here \textit{target} is the name of the output file,
\textit{main} is the name of the main file
and \textit{dest} is the name of the main or child file to be processed
(all filenames without extensions).
The optional argument \textit{main} can be omitted
if \textit{main} matches \textit{dest}.
Optionally, compilation \textit{flags} can be defined via |\def| commands.
This command line makes the \TeX{} engine believe
it is compiling the file \textit{target}
whose content is specified as the latter parameter.
The provided code then forwards the processing to
\textit{main} or \textit{dest} as described in \secref{sec:forward}.

%%%%%%%%%%%%%%%%%%%%%%%%%%%%%%%%%%%%%%%%%%%%%%%%%%%%%%%%%%%%%%%%%%%%%%%%%%%%%%%%
\subsection{Include by Input}
\label{sec:input}

Including child documents by |\include| has some restrictions by design.
Most notably, the content of a child document always occupies
its own set of pages; pages cannot be shared between child documents.
Usually, this behaviour makes perfect sense
because each child document contain an essential part of the document.
However, in some situations it may be desirable to compose
a document from a collection of parts
without having mandatory page breaks between then.
For this case, the package
provides a mechanism to include parts
by |\input| which can also be processed individually.
However, by construction this mechanism
requires manual handling of the content to be output.

%%%%%%%%%%%%%%%%%%%%%%%%%%%%%%%%%%%%%%%%
\DescribeMacro{\ifchilddocmanual}
The main file should be prepared as usual, see \secref{sec:include}.
However, the document body must make a distinction
between processing of an individual part and of the main document, e.g.:
%
\begin{center}
\begin{tabular}{l}
|\ifchilddocmanual|\\
|\input{\childdocname}|\\
|\||else|\\
\textit{document body with }|\input{|\textit{part}|}|\\
|\||fi|
\end{tabular}
\end{center}
%
The conditional |\ifchilddocmanual| is true whenever
a part to be included by |\input| is being compiled,
and the name of the part is stored in |\childdocname|.

%%%%%%%%%%%%%%%%%%%%%%%%%%%%%%%%%%%%%%%%
\DescribeMacro{\childdocby}
Each part to be included by |\input| should start with:
%
\begin{center}
\begin{tabular}{l}
|% \iffalse
%
% childdoc.dtx Copyright (C) 2017-2018 Niklas Beisert
%
% This work may be distributed and/or modified under the
% conditions of the LaTeX Project Public License, either version 1.3
% of this license or (at your option) any later version.
% The latest version of this license is in
%   http://www.latex-project.org/lppl.txt
% and version 1.3 or later is part of all distributions of LaTeX
% version 2005/12/01 or later.
%
% This work has the LPPL maintenance status `maintained'.
%
% The Current Maintainer of this work is Niklas Beisert.
%
% This work consists of the files childdoc.dtx and childdoc.ins
% and the derived files childdoc.def and cdocsamp.tex with
% cdocsch1.tex, cdocsch2.tex, cdocsdrf.tex, cdocsfn1.tex, cdocsfn2.tex.
%
%<package>\ifdefined\childdocmain\endinput\fi
%<package>\ProvidesFile{childdoc.def}[2018/12/30 v2.0 child document driver]
%<samplemain>\ProvidesFile{cdocsamp.tex}[2018/12/30 v2.0 sample for childdoc]
%<*driver>
%\ProvidesFile{childdoc.drv}[2018/12/30 v2.0 childdoc reference manual file]
\PassOptionsToClass{10pt,a4paper}{article}
\documentclass{ltxdoc}

\usepackage[margin=35mm]{geometry}
\usepackage{hyperref}
\usepackage{hyperxmp}
\usepackage[usenames]{color}

\hypersetup{colorlinks=true}
\hypersetup{pdfstartview=FitH}
\hypersetup{pdfpagemode=UseNone}
\hypersetup{pdfsource={}}
\hypersetup{pdflang={en-UK}}
\hypersetup{pdfcopyright={Copyright 2017-2018 Niklas Beisert.
  This work may be distributed and/or modified under the
  conditions of the LaTeX Project Public License, either version 1.3
  of this license or (at your option) any later version.}}
\hypersetup{pdflicenseurl={http://www.latex-project.org/lppl.txt}}
\hypersetup{pdfcontactaddress={ETH Zurich, ITP, HIT K,
  Wolfgang-Pauli-Strasse 27}}
\hypersetup{pdfcontactpostcode={8093}}
\hypersetup{pdfcontactcity={Zurich}}
\hypersetup{pdfcontactcountry={Switzerland}}
\hypersetup{pdfcontactemail={nbeisert@itp.phys.ethz.ch}}
\hypersetup{pdfcontacturl={http://people.phys.ethz.ch/\xmptilde nbeisert/}}

\newcommand{\secref}[1]{\hyperref[#1]{section \ref*{#1}}}

\parskip1ex
\parindent0pt
\let\olditemize\itemize
\def\itemize{\olditemize\parskip0pt}

\begin{document}

\title{The \textsf{childdoc} Package}
\hypersetup{pdftitle={The childdoc Package}}
\author{Niklas Beisert\\[2ex]
  Institut f\"ur Theoretische Physik\\
  Eidgen\"ossische Technische Hochschule Z\"urich\\
  Wolfgang-Pauli-Strasse 27, 8093 Z\"urich, Switzerland\\[1ex]
  \href{mailto:nbeisert@itp.phys.ethz.ch}
  {\texttt{nbeisert@itp.phys.ethz.ch}}}
\hypersetup{pdfauthor={Niklas Beisert}}
\hypersetup{pdfsubject={Manual for the LaTeX2e Package childdoc}}
\date{30 December 2018, \textsf{v2.0}}
\maketitle

\begin{abstract}\noindent
\textsf{childdoc} is a \LaTeXe{} package
that enables the direct compilation
of document sections included by |\include|
to individual files.
\end{abstract}

\begingroup
\parskip0ex
\tableofcontents
\endgroup

%%%%%%%%%%%%%%%%%%%%%%%%%%%%%%%%%%%%%%%%%%%%%%%%%%%%%%%%%%%%%%%%%%%%%%%%%%%%%%%%
%%%%%%%%%%%%%%%%%%%%%%%%%%%%%%%%%%%%%%%%%%%%%%%%%%%%%%%%%%%%%%%%%%%%%%%%%%%%%%%%
\section{Introduction}

\LaTeX{} provides a mechanism to structure a large document (such as a book)
into a main file and several child files (containing the chapters)
using the |\include| command.
This mechanism is beneficial for documents
which span hundreds of pages in order to
make the source file(s) more manageable.
Moreover, compilation can be restricted to
selected child files by means of the |\includeonly| command.
The latter feature can be used to reduce the compilation time while editing
(this was significantly more useful in the earlier days of \LaTeX{})
or to generate a smaller document which is easier to navigate.
Another application of |\includeonly| is to generate
documents consisting of selected parts of the complete document.

However, there are a few drawbacks of the plain |\include| mechanism:
\begin{itemize}
\item
The child files cannot be compiled on their own,
they can only be compiled via the main file.
A naive editing environment
(such as a text editor with an option
to have the current file processed by \LaTeX)
may require one to switch to the main file before compiling;
attempting to compile the child file produces errors.
\item
The main file must be modified (each time)
to adjust the |\includeonly| command
to the present needs. This easily leaves the main file in a messy state.
\item
The generated document will always carry the filename
of the main document. This is inconvenient if
several child files are to be compiled and
to be kept for distribution.
\end{itemize}

The present package provides a simple interface
to make child files individually compilable by \LaTeX{}.
Compiling a child file then has the same effect as compiling
the main file with an |\includeonly| command
to select the appropriate child.
Moreover the generated document will carry the name of the child
rather than the main file.
This resolves all three above issues.

This feature is meant to make the editing of books,
thesis documents and lecture notes somewhat more convenient.
However, the package can also be used efficiently for
composing a series of documents (such as exercise sheets)
which are typically distributed individually.
It then assists the author in generating the individual documents
(potentially in different versions)
as well as a document containing the collected series.
Another application is in developing style files
or other kinds of included material
where compilation of the style file could redirect
to a sample or test file.

%%%%%%%%%%%%%%%%%%%%%%%%%%%%%%%%%%%%%%%%%%%%%%%%%%%%%%%%%%%%%%%%%%%%%%%%%%%%%%%%
%%%%%%%%%%%%%%%%%%%%%%%%%%%%%%%%%%%%%%%%%%%%%%%%%%%%%%%%%%%%%%%%%%%%%%%%%%%%%%%%
\section{Usage}

First of all, the package \textsf{childdoc} is \emph{not} a standard
\LaTeXe{} |.sty| style file! Therefore it needs to be invoked in
a non-standard way.

%%%%%%%%%%%%%%%%%%%%%%%%%%%%%%%%%%%%%%%%%%%%%%%%%%%%%%%%%%%%%%%%%%%%%%%%%%%%%%%%
\subsection{Included Files}
\label{sec:include}

%%%%%%%%%%%%%%%%%%%%%%%%%%%%%%%%%%%%%%%%
\DescribeMacro{\childdocmain}
To use the package, add the commands
\begin{center}
\begin{tabular}{l}
|% \iffalse
%
% childdoc.dtx Copyright (C) 2017-2018 Niklas Beisert
%
% This work may be distributed and/or modified under the
% conditions of the LaTeX Project Public License, either version 1.3
% of this license or (at your option) any later version.
% The latest version of this license is in
%   http://www.latex-project.org/lppl.txt
% and version 1.3 or later is part of all distributions of LaTeX
% version 2005/12/01 or later.
%
% This work has the LPPL maintenance status `maintained'.
%
% The Current Maintainer of this work is Niklas Beisert.
%
% This work consists of the files childdoc.dtx and childdoc.ins
% and the derived files childdoc.def and cdocsamp.tex with
% cdocsch1.tex, cdocsch2.tex, cdocsdrf.tex, cdocsfn1.tex, cdocsfn2.tex.
%
%<package>\ifdefined\childdocmain\endinput\fi
%<package>\ProvidesFile{childdoc.def}[2018/12/30 v2.0 child document driver]
%<samplemain>\ProvidesFile{cdocsamp.tex}[2018/12/30 v2.0 sample for childdoc]
%<*driver>
%\ProvidesFile{childdoc.drv}[2018/12/30 v2.0 childdoc reference manual file]
\PassOptionsToClass{10pt,a4paper}{article}
\documentclass{ltxdoc}

\usepackage[margin=35mm]{geometry}
\usepackage{hyperref}
\usepackage{hyperxmp}
\usepackage[usenames]{color}

\hypersetup{colorlinks=true}
\hypersetup{pdfstartview=FitH}
\hypersetup{pdfpagemode=UseNone}
\hypersetup{pdfsource={}}
\hypersetup{pdflang={en-UK}}
\hypersetup{pdfcopyright={Copyright 2017-2018 Niklas Beisert.
  This work may be distributed and/or modified under the
  conditions of the LaTeX Project Public License, either version 1.3
  of this license or (at your option) any later version.}}
\hypersetup{pdflicenseurl={http://www.latex-project.org/lppl.txt}}
\hypersetup{pdfcontactaddress={ETH Zurich, ITP, HIT K,
  Wolfgang-Pauli-Strasse 27}}
\hypersetup{pdfcontactpostcode={8093}}
\hypersetup{pdfcontactcity={Zurich}}
\hypersetup{pdfcontactcountry={Switzerland}}
\hypersetup{pdfcontactemail={nbeisert@itp.phys.ethz.ch}}
\hypersetup{pdfcontacturl={http://people.phys.ethz.ch/\xmptilde nbeisert/}}

\newcommand{\secref}[1]{\hyperref[#1]{section \ref*{#1}}}

\parskip1ex
\parindent0pt
\let\olditemize\itemize
\def\itemize{\olditemize\parskip0pt}

\begin{document}

\title{The \textsf{childdoc} Package}
\hypersetup{pdftitle={The childdoc Package}}
\author{Niklas Beisert\\[2ex]
  Institut f\"ur Theoretische Physik\\
  Eidgen\"ossische Technische Hochschule Z\"urich\\
  Wolfgang-Pauli-Strasse 27, 8093 Z\"urich, Switzerland\\[1ex]
  \href{mailto:nbeisert@itp.phys.ethz.ch}
  {\texttt{nbeisert@itp.phys.ethz.ch}}}
\hypersetup{pdfauthor={Niklas Beisert}}
\hypersetup{pdfsubject={Manual for the LaTeX2e Package childdoc}}
\date{30 December 2018, \textsf{v2.0}}
\maketitle

\begin{abstract}\noindent
\textsf{childdoc} is a \LaTeXe{} package
that enables the direct compilation
of document sections included by |\include|
to individual files.
\end{abstract}

\begingroup
\parskip0ex
\tableofcontents
\endgroup

%%%%%%%%%%%%%%%%%%%%%%%%%%%%%%%%%%%%%%%%%%%%%%%%%%%%%%%%%%%%%%%%%%%%%%%%%%%%%%%%
%%%%%%%%%%%%%%%%%%%%%%%%%%%%%%%%%%%%%%%%%%%%%%%%%%%%%%%%%%%%%%%%%%%%%%%%%%%%%%%%
\section{Introduction}

\LaTeX{} provides a mechanism to structure a large document (such as a book)
into a main file and several child files (containing the chapters)
using the |\include| command.
This mechanism is beneficial for documents
which span hundreds of pages in order to
make the source file(s) more manageable.
Moreover, compilation can be restricted to
selected child files by means of the |\includeonly| command.
The latter feature can be used to reduce the compilation time while editing
(this was significantly more useful in the earlier days of \LaTeX{})
or to generate a smaller document which is easier to navigate.
Another application of |\includeonly| is to generate
documents consisting of selected parts of the complete document.

However, there are a few drawbacks of the plain |\include| mechanism:
\begin{itemize}
\item
The child files cannot be compiled on their own,
they can only be compiled via the main file.
A naive editing environment
(such as a text editor with an option
to have the current file processed by \LaTeX)
may require one to switch to the main file before compiling;
attempting to compile the child file produces errors.
\item
The main file must be modified (each time)
to adjust the |\includeonly| command
to the present needs. This easily leaves the main file in a messy state.
\item
The generated document will always carry the filename
of the main document. This is inconvenient if
several child files are to be compiled and
to be kept for distribution.
\end{itemize}

The present package provides a simple interface
to make child files individually compilable by \LaTeX{}.
Compiling a child file then has the same effect as compiling
the main file with an |\includeonly| command
to select the appropriate child.
Moreover the generated document will carry the name of the child
rather than the main file.
This resolves all three above issues.

This feature is meant to make the editing of books,
thesis documents and lecture notes somewhat more convenient.
However, the package can also be used efficiently for
composing a series of documents (such as exercise sheets)
which are typically distributed individually.
It then assists the author in generating the individual documents
(potentially in different versions)
as well as a document containing the collected series.
Another application is in developing style files
or other kinds of included material
where compilation of the style file could redirect
to a sample or test file.

%%%%%%%%%%%%%%%%%%%%%%%%%%%%%%%%%%%%%%%%%%%%%%%%%%%%%%%%%%%%%%%%%%%%%%%%%%%%%%%%
%%%%%%%%%%%%%%%%%%%%%%%%%%%%%%%%%%%%%%%%%%%%%%%%%%%%%%%%%%%%%%%%%%%%%%%%%%%%%%%%
\section{Usage}

First of all, the package \textsf{childdoc} is \emph{not} a standard
\LaTeXe{} |.sty| style file! Therefore it needs to be invoked in
a non-standard way.

%%%%%%%%%%%%%%%%%%%%%%%%%%%%%%%%%%%%%%%%%%%%%%%%%%%%%%%%%%%%%%%%%%%%%%%%%%%%%%%%
\subsection{Included Files}
\label{sec:include}

%%%%%%%%%%%%%%%%%%%%%%%%%%%%%%%%%%%%%%%%
\DescribeMacro{\childdocmain}
To use the package, add the commands
\begin{center}
\begin{tabular}{l}
|\input{childdoc.def}|\\
|\childdocmain{}|\\
\end{tabular}
\end{center}
at the very top of the main \LaTeX{} file,
in particular \emph{before} the |\documentclass| statement!
The argument of |\childdocmain| should be left empty
(but it must be present).

%%%%%%%%%%%%%%%%%%%%%%%%%%%%%%%%%%%%%%%%
\DescribeMacro{\childdocof}
Furthermore, add the commands
\begin{center}
\begin{tabular}{l}
|\input{childdoc.def}|\\
|\childdocof{|\textit{main}|}|\\
\end{tabular}
\end{center}
at the top of every child file \textit{child}
which is included by |\include{|\textit{child}|}|
from within the main file
(or at least for those files to be compiled individually).
The argument \textit{main} must be the filename of the main file.

There are a couple of
considerations in setting up the main and child documents:

%%%%%%%%%%%%%%%%%%%%%%%%%%%%%%%%%%%%%%%%
\paragraph{Restrictions.}

Please note the following restrictions:
\begin{itemize}
\item
|\childdocmain| must be called with one argument \textit{main}
to ensure compatibility with earlier version of the package.
It must either be empty (|\childdocmain{}|)
or precisely match the filename of the main file in which it is specified.
See \secref{sec:detection} for further information.
\item
The filename \textit{main} must be specified without the |.tex| extension.
\item
The filename \textit{main} is case sensitive
(even in case-insensitive file systems)
due to internal string comparison.
\item
The argument \textit{main} should be fully expanded, it cannot be a macro.
\item
Subdirectories and special characters should be avoided in filenames.
\item
The command |\childdocmain{|\textit{main}|}| must be followed by a whitespace.
It should not be followed immediately by another command
or by a comment mark `|%|'.
This is because the \TeX{} parser reads the token immediately following
the argument of |\childdocmain| and puts it
at the beginning of every child section;
however, a white\-space is ignored.
\end{itemize}

%%%%%%%%%%%%%%%%%%%%%%%%%%%%%%%%%%%%%%%%
\paragraph{Content of Main File.}

It is advisable to place all content in the child files included by |\include|.
Any output contained in the main file will appear in all child documents
unless suppressed manually;
it cannot be suppressed automatically by the |\includeonly| directive
and thus should normally be avoided.
A method to include some content in the main file
by means of conditional processing is described in \secref{sec:conditional}.

%%%%%%%%%%%%%%%%%%%%%%%%%%%%%%%%%%%%%%%%
\paragraph{Page Numbering.}

When only a part of the document is compiled,
the appropriate numbering of pages
(as well as other status parameters)
is determined from the |.aux| files.
The latter contain information from previous passes.
However this information needs to propagate through
all intermediate child documents.
Therefore the page numbering in child documents may well
be inconsistent until the complete document is compiled at least once.

A useful (if unconventional) way to always ensure a consistent
page numbering is to restart the numbering in each child document
and denote the pages by `\textit{child}|.|\textit{page}'
where \textit{child} represents the chapter/section number of the child file.
This can be achieved by the command
|\numberwithin{page}{|\textit{child}|}|
of the \textsf{amsmath} package
where \textit{child} can be |chapter| or |section|
depending on the chosen structuring.
Alternatively, one can modify the macro |\thepage| appropriately
and reset the counter |page| at the start of each child file.

%%%%%%%%%%%%%%%%%%%%%%%%%%%%%%%%%%%%%%%%%%%%%%%%%%%%%%%%%%%%%%%%%%%%%%%%%%%%%%%%
\subsection{Conditional Processing}
\label{sec:conditional}

The package provides a mechanism to compile different versions
of a document. To customise the versions further some conditional processing
can come in handy to distinguish which version is being compiled.
The package provides two macros to describe the compilation context:

%%%%%%%%%%%%%%%%%%%%%%%%%%%%%%%%%%%%%%%%
\DescribeMacro{\ifchilddoc}
The conditional |\ifchilddoc| distinguishes between the compilation of
child documents and the main document:
%
\begin{center}
|\ifchilddoc |\textit{child-code}| |[|\||else |\textit{main-code}]| \||fi|
\end{center}

%%%%%%%%%%%%%%%%%%%%%%%%%%%%%%%%%%%%%%%%
\DescribeMacro{\childdocname}
\DescribeMacro{\childdocjob}
The macro |\childdocname| contains the filename (without extension)
of the main or child file being processed.
Note that |\childdocjob| will always contain the name of the main file.

%%%%%%%%%%%%%%%%%%%%%%%%%%%%%%%%%%%%%%%%
\paragraph{Title Page.}

Conditional processing can be used to include a title or banner page
in the main document when proper precautions are taken.
Importantly, the code in the main file should ensure that the page counter
(as well as other status parameters which are stored in the |.aux| files)
takes the same value after the conditional processing.
Otherwise the page numbers may take divergent values
depending on which part is compiled.

For example, a title page could be declared by:
%
\begin{center}
\begin{tabular}{l}
|\ifchilddoc\||else|\\
|\addtocounter{page}{-1}|\\
\textit{code for title page}\\
|\newpage|\\
|\||fi|
\end{tabular}
\end{center}
%
A banner page for the child documents can be generated by:
%
\begin{center}
\begin{tabular}{l}
|\ifchilddoc|\\
|\addtocounter{page}{-1}|\\
\textit{code for banner page}\\
|\newpage|\\
|\||fi|
\end{tabular}
\end{center}
%
Here one could write a message such as:
\begin{center}
|This is the part \childdocname{} of \childdocjob{}.|
\end{center}

%%%%%%%%%%%%%%%%%%%%%%%%%%%%%%%%%%%%%%%%%%%%%%%%%%%%%%%%%%%%%%%%%%%%%%%%%%%%%%%%
\subsection{Flags}
\label{sec:flags}

The package makes it easy to generate different versions
of the main or child documents.
To this end compilation flags can be defined
and assigned different default values.
They will be particularly useful in conjunction
with the forwarding mechanism described in \secref{sec:forward}.

For example, it may be useful to have a flag |\version|
which can be set to |draft| or |final|.
The document source will contain some conditional code
depending on the value of |\version|.
Suppose further, the flag should default to |final| for the main file
and to |draft| for child files
which is a natural assignment for editing the document.
This is achieved by placing the following code
in the preamble of the main document
(below the |\childdocmain| directive):
%
\begin{center}
\begin{tabular}{l}
|\ifchilddoc|\\
|\providecommand{\version}{draft}|\\
|\||else|\\
|\providecommand{\version}{final}|\\
|\||fi|
\end{tabular}
\end{center}
%
The definition by |\providecommand| makes sure
that previous definitions are not overwritten.
Further statements |\providecommand{\version}{...}|
can thus be added before the above code to override it.

For the main file, one might add a line
(between |\childdocmain| and the above block)
%
\begin{center}
|%\ifchilddoc\||else\providecommand{\version}{draft}\||fi|
\end{center}
%
which can be uncommented to produce a draft version.
Likewise one can add a line to the very top of a child file
(above the |\childdocof{|\textit{main}|}| directive)
%
\begin{center}
|%\providecommand{\version}{final}|
\end{center}
%
which can be uncommented to produce the final version of this child document.

%%%%%%%%%%%%%%%%%%%%%%%%%%%%%%%%%%%%%%%%%%%%%%%%%%%%%%%%%%%%%%%%%%%%%%%%%%%%%%%%
\subsection{Forwarding}
\label{sec:forward}

Different versions of the main or child documents
using compilation flags as described in \secref{sec:flags}
can be (permanently) stored in different files
for convenient compilation, viewing and distribution.
To this end, the package defines a command
to pass on compilation to a different file:

%%%%%%%%%%%%%%%%%%%%%%%%%%%%%%%%%%%%%%%%
\DescribeMacro{\childdocforward}
The command |\childdocforward| redirects processing to
another source file:
%
\begin{center}
\begin{tabular}{l}
|\input{childdoc.def}|\\
|\childdocforward[|\textit{main}|]{|\textit{dest}|}|\\
\end{tabular}
\end{center}
%
The argument \textit{dest} is the destination file
(without extension).
It should be the main file or one of the child files.
Note that further \textsf{childdoc} directives
such as |\childdocof| and |\childdocforward|
in the indicated file will be processed in this form.
The optional argument \textit{main}
passes on directly to the main file \textit{main}
while pretending to compile the child \textit{dest}.
This form behaves as if \textit{dest}
issues |\childdocof{|\textit{main}|}| right away,
and no further \textsf{childdoc} directives will be processed.

%%%%%%%%%%%%%%%%%%%%%%%%%%%%%%%%%%%%%%%%
\DescribeMacro{\...prefix}
In the alternative form |\childdocforwardprefix|,
%
\begin{center}
\begin{tabular}{l}
|\input{childdoc.def}|\\
|\childdocforwardprefix[|\textit{main}|]{|\textit{prefix}|}{|\textit{dest}|}|
\end{tabular}
\end{center}
%
the destination file is determined by a pattern
depending on the current file:
To make this work, the current file must be called
`{\textit{prefix}\hspace{0.2em}\textit{suffix}}'
with \textit{prefix} matching precisely the argument.
Processing is then passed on to the file
`{\textit{dest}\hspace{0.2em}\textit{suffix}}'.
Surely, the same effect is achieved by
directly specifying the
argument `{\textit{dest}\hspace{0.2em}\textit{suffix}}'
in the first form.
However, that requires to set up a different file
for each child. With the alternative form of the command
all these files can have exactly the same content
which simplifies setting them up and maintaining them.

For example, the following file |draft.tex|
with a compilation flag |\version| as described in \secref{sec:flags}
compiles the main document as a draft:
%
\begin{center}
\begin{tabular}{l}
|\def\version{draft}|\\
|\input{childdoc.def}|\\
|\childdocforward{|\textit{main}|}|
\end{tabular}
\end{center}
%
Likewise, the following files |final|\textit{nn}|.tex|
compile the final version of the child document
|child|\textit{nn}|.tex|:
%
\begin{center}
\begin{tabular}{l}
|\def\version{final}|\\
|\input{childdoc.def}|\\
|\childdocforwardprefix{final}{child}|
\end{tabular}
\end{center}
%

Note that when several versions of a main file and/or of each child file
are to be generated, it may be convenient to set up a |Makefile| or
shell script to automatise the process.

%%%%%%%%%%%%%%%%%%%%%%%%%%%%%%%%%%%%%%%%%%%%%%%%%%%%%%%%%%%%%%%%%%%%%%%%%%%%%%%%
\subsection{Command Line Processing}
\label{sec:commandline}

The effect of redirection files can also be achieved by invoking
the \LaTeX{} compiler with a more elaborate command line.
Most conveniently this should be done as part
of a shell script or a |Makefile|.

When using \textsf{childdoc} in the main file, the following
command lines effectively perform a redirection
(note that depending on the shell being used,
backslashes may have to be doubled: `|\|' $\to$ `|\\|'):
%
\begin{center}
|... -jobname "|\textit{target}|" |\\|"|[\textit{flags}]%
|\input{childdoc.def}\childdocforward[|\textit{main}|]{|\textit{dest}|}"|
\end{center}
%
Here \textit{target} is the name of the output file,
\textit{main} is the name of the main file
and \textit{dest} is the name of the main or child file to be processed
(all filenames without extensions).
The optional argument \textit{main} can be omitted
if \textit{main} matches \textit{dest}.
Optionally, compilation \textit{flags} can be defined via |\def| commands.
This command line makes the \TeX{} engine believe
it is compiling the file \textit{target}
whose content is specified as the latter parameter.
The provided code then forwards the processing to
\textit{main} or \textit{dest} as described in \secref{sec:forward}.

%%%%%%%%%%%%%%%%%%%%%%%%%%%%%%%%%%%%%%%%%%%%%%%%%%%%%%%%%%%%%%%%%%%%%%%%%%%%%%%%
\subsection{Include by Input}
\label{sec:input}

Including child documents by |\include| has some restrictions by design.
Most notably, the content of a child document always occupies
its own set of pages; pages cannot be shared between child documents.
Usually, this behaviour makes perfect sense
because each child document contain an essential part of the document.
However, in some situations it may be desirable to compose
a document from a collection of parts
without having mandatory page breaks between then.
For this case, the package
provides a mechanism to include parts
by |\input| which can also be processed individually.
However, by construction this mechanism
requires manual handling of the content to be output.

%%%%%%%%%%%%%%%%%%%%%%%%%%%%%%%%%%%%%%%%
\DescribeMacro{\ifchilddocmanual}
The main file should be prepared as usual, see \secref{sec:include}.
However, the document body must make a distinction
between processing of an individual part and of the main document, e.g.:
%
\begin{center}
\begin{tabular}{l}
|\ifchilddocmanual|\\
|\input{\childdocname}|\\
|\||else|\\
\textit{document body with }|\input{|\textit{part}|}|\\
|\||fi|
\end{tabular}
\end{center}
%
The conditional |\ifchilddocmanual| is true whenever
a part to be included by |\input| is being compiled,
and the name of the part is stored in |\childdocname|.

%%%%%%%%%%%%%%%%%%%%%%%%%%%%%%%%%%%%%%%%
\DescribeMacro{\childdocby}
Each part to be included by |\input| should start with:
%
\begin{center}
\begin{tabular}{l}
|\input{childdoc.def}|\\
|\childdocby{|\textit{main}|}|\\
\end{tabular}
\end{center}
%
The directive |\childdocby| is similar to |\childdocof|
described in \secref{sec:include},
but the subsequent selection of content must be done manually.
To that end, both |\ifchilddoc| and |\ifchilddocmanual|
will be true upon processing of a part,
and the name of the part is stored in |\childdocname|.
Note that |\jobname| will be set to the filename of the current part
so that each part receives an individual |.aux| file
that does not interfere with the |.aux| file(s) of the main document.
This behaviour can be altered by the alternative form
|\childdocby[*]{|\textit{main}|}| (with a non-empty optional argument)
which uses the |.aux| file of the main document
by setting |\jobname| to \textit{main}.

%%%%%%%%%%%%%%%%%%%%%%%%%%%%%%%%%%%%%%%%%%%%%%%%%%%%%%%%%%%%%%%%%%%%%%%%%%%%%%%%
\subsection{Driver Development}
\label{sec:driver}

The \textsf{childdoc} mechanism can also be use for the development
of definition files such as \LaTeX{} styles or classes.
This case differs from the above setup with multiple parts
included by |\include| in that no |\includeonly| should be invoked.
This can be achieved by starting the include file
(before |\ProvidesPackage|) with:
%
\begin{center}
\begin{tabular}{l}
|\input{childdoc.def}|\\
|\childdocforward{|\textit{main}|}|\\
\end{tabular}
\end{center}
%
or alternatively with:
%
\begin{center}
\begin{tabular}{l}
|\input{childdoc.def}|\\
|\childdocby{|\textit{main}|}|\\
\end{tabular}
\end{center}
%
Both forms have slightly different effects as described above.
The main file is prepared as usual, see \secref{sec:include}.

%%%%%%%%%%%%%%%%%%%%%%%%%%%%%%%%%%%%%%%%%%%%%%%%%%%%%%%%%%%%%%%%%%%%%%%%%%%%%%%%
\subsection{Legacy Detection}
\label{sec:detection}

The directive |\childdocmain| in the main file can detect
whether the complete document or merely a child is to be compiled
even without using the directive |\childdocof|.
This method is deprecated because it is less robust
and there is no compelling reason to use it;
it is merely provided for backward compatibility
and it may be removed in future versions.

If the detection mechanism is to be used,
it is mandatory to correctly specify
the filename of the main file as the argument of |\childdocmain|:
%
\begin{center}
\begin{tabular}{l}
|\input{childdoc.def}|\\
|\childdocmain{|\textit{main}|}|\\
\end{tabular}
\end{center}
%
If |\jobname| does not match the argument \textit{main} of |\childdocmain|,
it is assumed that |\jobname| points to the child file to be compiled.
When using |\childdocmain| with the main file specified as argument,
it suffices to start a child file
with just |\input{|\textit{main}|}|
without loading of the package and using |\childdocof|.
If instead all processing is done
with the appropriate \textsf{childdoc} directives,
the argument of \textit{main} of |\childdocmain| can be empty.

An alternative version of the command line processing described
in \secref{sec:commandline} using the detection mechanism reads:
%
\begin{center}
|... -jobname "|\textit{target}|" "|[\textit{flags}]%
[|\def\jobname{|\textit{dest}|}|]|\input{|\textit{main}|}"|
\end{center}

%%%%%%%%%%%%%%%%%%%%%%%%%%%%%%%%%%%%%%%%%%%%%%%%%%%%%%%%%%%%%%%%%%%%%%%%%%%%%%%%
\subsection{Manual Code}
\label{sec:manual}

In case one cannot be certain whether the definitions file |childdoc.def|
is installed on the target \TeX{} distribution
and one prefers not to ship it,
it is conceivable to paste a few relevant commands into the sources.

To that end, drop all statements |\input{childdoc.def}|
and perform the replacements as outlined below.
Instead of |\childdocmain{|\textit{main}|}| add the following code
to the top of the main file:
%
\begin{center}
\begin{tabular}{l}
|\||ifdefined\childdocname\endinput\||fi\newif\ifchilddoc|\\
|\edef\childdocname{\scantokens\expandafter{\jobname\noexpand}}|\\
|\def\childdocmain{|\textit{main}|}\||ifx\childdocmain\childdocname\||else|\\
|\childdoctrue\includeonly{\childdocname}\let\jobname\childdocmain\||fi|\\
\end{tabular}
\end{center}
%
Instead of |\childdocof{|\textit{main}|}| just include the main file
at the top of each child file:
%
\begin{center}
|\input{|\textit{main}|}|
\end{center}
%
A simple redirection |\childdocforward{|\textit{dest}|}| is achieved by:
%
\begin{center}
|\def\jobname{|\textit{dest}|}\input{\jobname}|
\end{center}
%
The redirection with prefix
|\childdocforwardprefix[|\textit{prefix}|]{|\textit{dest}|}|
is accomplished by:
%
\begin{center}
\begin{tabular}{l}
|{\edef\jobname{\scantokens\expandafter{\jobname\noexpand}}|\\
|\def\redirectjob |\textit{prefix}|#1~~~{\gdef\jobname{|\textit{dest}|#1}}|\\
|\expandafter\redirectjob\jobname~~~}\input{\jobname}|
\end{tabular}
\end{center}

In an alternative approach,
child documents can be compiled by a specific command line
without additional code or specific definitions:
%
\begin{center}
|... -jobname "|\textit{target}|" "|[\textit{flags}]%
|\includeonly{|\textit{dest}|}\input{|\textit{main}|}"|
\end{center}
%

%%%%%%%%%%%%%%%%%%%%%%%%%%%%%%%%%%%%%%%%%%%%%%%%%%%%%%%%%%%%%%%%%%%%%%%%%%%%%%%%
%%%%%%%%%%%%%%%%%%%%%%%%%%%%%%%%%%%%%%%%%%%%%%%%%%%%%%%%%%%%%%%%%%%%%%%%%%%%%%%%
\section{Information}

%%%%%%%%%%%%%%%%%%%%%%%%%%%%%%%%%%%%%%%%%%%%%%%%%%%%%%%%%%%%%%%%%%%%%%%%%%%%%%%%
\subsection{Copyright}

Copyright \copyright{} 2017--2018 Niklas Beisert

This work may be distributed and/or modified under the
conditions of the \LaTeX{} Project Public License, either version 1.3
of this license or (at your option) any later version.
The latest version of this license is in
  \url{http://www.latex-project.org/lppl.txt}
and version 1.3 or later is part of all distributions of \LaTeX{}
version 2005/12/01 or later.

This work has the LPPL maintenance status `maintained'.

The Current Maintainer of this work is Niklas Beisert.

This work consists of the files |README.txt|, |childdoc.ins| and |childdoc.dtx|
as well as the derived files |childdoc.def|, |cdocsamp.tex|
with |cdocsch1.tex|, |cdocsch2.tex|, |cdocspt3.tex|, |cdocspt4.tex|,
|cdocsdrf.tex|, |cdocsfn1.tex|, |cdocsfn2.tex|
as well as |childdoc.pdf|.

%%%%%%%%%%%%%%%%%%%%%%%%%%%%%%%%%%%%%%%%%%%%%%%%%%%%%%%%%%%%%%%%%%%%%%%%%%%%%%%%
\subsection{Files and Installation}

The package consists of the files:
%
\begin{center}
\begin{tabular}{ll}
    |README.txt|   & readme file \\
    |childdoc.ins| & installation file \\
    |childdoc.dtx| & source file \\
    |childdoc.def| & definition file \\
    |cdocsamp.tex| & sample main file \\
    |cdocsch1.tex| & sample include file \\
    |cdocsch2.tex| & sample include file \\
    |cdocspt3.tex| & sample part file \\
    |cdocspt4.tex| & sample part file \\
    |cdocsdrf.tex| & sample redirection file \\
    |cdocsfn1.tex| & sample redirection file \\
    |cdocsfn2.tex| & sample redirection file \\
    |childdoc.pdf| & manual
\end{tabular}
\end{center}
%
The distribution consists of the files
|README.txt|, |childdoc.ins| and |childdoc.dtx|.
%
\begin{itemize}
\item
Run (pdf)\LaTeX{} on |childdoc.dtx|
to compile the manual |childdoc.pdf| (this file).
\item
Run \LaTeX{} on |childdoc.ins| to create the definitions file |childdoc.def|
and the sample |cdocsamp.tex| with include files
|cdocsch1.tex|, |cdocsch2.tex|, |cdocspt3.tex|, |cdocspt4.tex|,
|cdocsdrf.tex|, |cdocsfn1.tex|, |cdocsfn2.tex|.
Then copy the file |childdoc.def| to an appropriate directory of your \LaTeX{}
distribution, e.g.\ \textit{texmf-root}|/tex/latex/childdoc|.
\end{itemize}

%%%%%%%%%%%%%%%%%%%%%%%%%%%%%%%%%%%%%%%%%%%%%%%%%%%%%%%%%%%%%%%%%%%%%%%%%%%%%%%%
\subsection{Related CTAN Packages}

There are several other packages which offer a similar functionality:
%
\begin{itemize}
\item
The packages
\href{http://ctan.org/pkg/docmute}{\textsf{docmute}},
\href{http://ctan.org/pkg/includex}{\textsf{includex}} and
\href{http://ctan.org/pkg/standalone}{\textsf{standalone}}
provide commands to include only the document body of
a child file thus allowing both files to be compiled individually.
\item
The packages \href{http://ctan.org/pkg/subdocs}{\textsf{subdocs}}
and \href{http://ctan.org/pkg/subfiles}{\textsf{subfiles}}
provide structures in which the main and child documents can be
encapsulated and allowing them to be compiled individually.
The inclusion mechanism is different from the conventional |\include|.
\item
The package \href{http://ctan.org/pkg/combine}{\textsf{combine}}
is an elaborate solution to combine several documents into one.
\end{itemize}
%
See also the CTAN topic \href{http://ctan.org/topic/subdocs}{\textsf{subdocs}}
for further related packages.
The present package differs from the above solutions in that
a document structure constructed with the conventional |\include| mechanism
just needs two extra commands at the top of every file
such that all constituent files can be compiled individually.

%%%%%%%%%%%%%%%%%%%%%%%%%%%%%%%%%%%%%%%%%%%%%%%%%%%%%%%%%%%%%%%%%%%%%%%%%%%%%%%%
%\subsection{Feature Suggestions}
%
%The following is a list of features which may be useful for future
%versions of this package:
%%
%\begin{itemize}
%\item
%\ldots
%\end{itemize}

%%%%%%%%%%%%%%%%%%%%%%%%%%%%%%%%%%%%%%%%%%%%%%%%%%%%%%%%%%%%%%%%%%%%%%%%%%%%%%%%
\subsection{Revision History}

%%%%%%%%%%%%%%%%%%%%%%%%%%%%%%%%%%%%%%%%
\paragraph{v2.0:} 2018/12/30

\begin{itemize}
\item
immediate forward processing
\item
added |\childdocby| mechanism
\item
manual restructured
\end{itemize}

%%%%%%%%%%%%%%%%%%%%%%%%%%%%%%%%%%%%%%%%
\paragraph{v1.6:} 2018/01/17

\begin{itemize}
\item
application for development of include files
\item
corrections to manual
\end{itemize}

%%%%%%%%%%%%%%%%%%%%%%%%%%%%%%%%%%%%%%%%
\paragraph{v1.5:} 2017/05/21

\begin{itemize}
\item
more complete structuring introduced
\item
|\childdocof| introduced
\item
|\childdoc| renamed to |\childdocmain|
\item
|\childredirect| renamed to |\childdocforward| and |\childdocforwardprefix|
and functionality expanded
\end{itemize}

%%%%%%%%%%%%%%%%%%%%%%%%%%%%%%%%%%%%%%%%
\paragraph{v1.0:} 2017/04/27

\begin{itemize}
\item
manual and install package
\item
first version published on CTAN
\end{itemize}

%%%%%%%%%%%%%%%%%%%%%%%%%%%%%%%%%%%%%%%%
\paragraph{v0.6:} 2017/04/26

\begin{itemize}
\item
redirection mechanism added
\end{itemize}

%%%%%%%%%%%%%%%%%%%%%%%%%%%%%%%%%%%%%%%%
\paragraph{v0.5:} 2017/04/26

\begin{itemize}
\item
functionality in definition file
\end{itemize}


%%%%%%%%%%%%%%%%%%%%%%%%%%%%%%%%%%%%%%%%%%%%%%%%%%%%%%%%%%%%%%%%%%%%%%%%%%%%%%%%
%%%%%%%%%%%%%%%%%%%%%%%%%%%%%%%%%%%%%%%%%%%%%%%%%%%%%%%%%%%%%%%%%%%%%%%%%%%%%%%%
%%%%%%%%%%%%%%%%%%%%%%%%%%%%%%%%%%%%%%%%%%%%%%%%%%%%%%%%%%%%%%%%%%%%%%%%%%%%%%%%
\appendix

\settowidth\MacroIndent{\rmfamily\scriptsize 000\ }

 \DocInput{childdoc.dtx}

\end{document}
%</driver>
% \fi
%
% %%%%%%%%%%%%%%%%%%%%%%%%%%%%%%%%%%%%%%%%%%%%%%%%%%%%%%%%%%%%%%%%%%%%%%%%%%%%%%
% %%%%%%%%%%%%%%%%%%%%%%%%%%%%%%%%%%%%%%%%%%%%%%%%%%%%%%%%%%%%%%%%%%%%%%%%%%%%%%
% \section{Sample}
%\iffalse
%<*samplemain>
%\fi
%
% The following presents a sample document
% with two chapters, two parts, a title page,
% a compile flag as well as three forwarding files to set the flag.
% It consists of eight |.tex| files:
% \begin{center}
% \begin{tabular}{ll}
% |cdocsamp.tex|&main file\\
% |cdocsch1.tex|&include file for chapter 1\\
% |cdocsch2.tex|&include file for chapter 2\\
% |cdocspt3.tex|&include file for part 3\\
% |cdocspt4.tex|&include file for part 4\\
% |cdocsdrf.tex|&forwarding file for main file in draft mode\\
% |cdocsfi1.tex|&forwarding file for final version of chapter 1\\
% |cdocsfi2.tex|&forwarding file for final version of chapter 2\\
% \end{tabular}
% \end{center}
% Each of the eight files can be compiled directly by the \LaTeX{} compiler.
%
% %%%%%%%%%%%%%%%%%%%%%%%%%%%%%%%%%%%%%%
% \paragraph{Main File.}
%
% The main file is called |cdocsamp.tex|.
%
% Load the \textsf{childdoc} definitions and
% declare the filename for the main document:
%    \begin{macrocode}
\input{childdoc.def}
\childdocmain{}
%    \end{macrocode}

% Optional override for |\version| flag:
%    \begin{macrocode}
%%\ifchilddoc\else\providecommand{\version}{draft}\fi
%    \end{macrocode}

% Define the default values for the |\version| flag
% (|final| for the main file and |draft| for childs):
%    \begin{macrocode}
\ifchilddoc
\providecommand{\version}{draft}
\else
\providecommand{\version}{final}
\fi
%    \end{macrocode}

% Load the standard document class:
%    \begin{macrocode}
\documentclass[12pt]{article}
%    \end{macrocode}

% Start the document body:
%    \begin{macrocode}
\begin{document}
%    \end{macrocode}

% Declare a title page.
% Print title, part of document being processed and version flag:
%    \begin{macrocode}
\addtocounter{page}{-1}
\begin{center}
{\LARGE\bfseries{}childdoc example\par}
\vspace{1cm}
\ifchilddoc
\ifchilddocmanual part\else chapter\fi:
`\childdocname' of `\childdocjob'\par
\else
main document: `\childdocjob'\par
\fi
version: \version\par
\end{center}
\newpage
%    \end{macrocode}

% Manually include selected file,
% otherwise process as usual:
%    \begin{macrocode}
\ifchilddocmanual
\section*{part `\childdocname'}
\input{\childdocname}
\else
%    \end{macrocode}

% Include the two chapters:
%    \begin{macrocode}
\include{cdocsch1}
\include{cdocsch2}
%    \end{macrocode}

% Include the two parts unless only chapters should be displayed:
%    \begin{macrocode}
\ifchilddoc\else
\section{part three}
\input{cdocspt3}
\section{part four}
\input{cdocspt4}
\fi
%    \end{macrocode}

% Process as usual until here:
%    \begin{macrocode}
\fi
%    \end{macrocode}

% End of document body:
%    \begin{macrocode}
\end{document}
%    \end{macrocode}
%\iffalse
%</samplemain>
%\fi
%
% %%%%%%%%%%%%%%%%%%%%%%%%%%%%%%%%%%%%%%
% \paragraph{Chapter Include Files.}
%
% The include files are called |cdocsch1.tex| and |cdocsch2.tex|.
%
%\iffalse
%<*samplechap1|samplechap2>
%\fi

% Optional override for |\version| flag:
%    \begin{macrocode}
%%\providecommand{\version}{final}
%    \end{macrocode}

% Include the main document:
%    \begin{macrocode}
\input{childdoc.def}
\childdocof{cdocsamp}
%    \end{macrocode}

%\iffalse
%</samplechap1|samplechap2>
%\fi
%
%\iffalse
%<*samplechap1>
%\fi
% Some text for chapter 1:
%    \begin{macrocode}
\section{one}
some text in chapter one
%    \end{macrocode}

%\iffalse
%</samplechap1>
%\fi
% Some text for chapter 2:
%\iffalse
%<*samplechap2>
%\fi
%    \begin{macrocode}
\section{two}
more text in chapter two
%    \end{macrocode}

%\iffalse
%</samplechap2>
%\fi
%
% %%%%%%%%%%%%%%%%%%%%%%%%%%%%%%%%%%%%%%
% \paragraph{Part Include Files.}
%
% The include files are called |cdocspt3.tex| and |cdocspt4.tex|.
%
%\iffalse
%<*samplepart3|samplepart4>
%\fi

% Optional override for |\version| flag:
%    \begin{macrocode}
%%\providecommand{\version}{final}
%    \end{macrocode}

% Include the main document:
%    \begin{macrocode}
\input{childdoc.def}
\childdocby{cdocsamp}
%    \end{macrocode}

%\iffalse
%</samplepart3|samplepart4>
%\fi
%
%\iffalse
%<*samplepart3>
%\fi
% Some text for part 3:
%    \begin{macrocode}
some text in part three
%    \end{macrocode}

%\iffalse
%</samplepart3>
%\fi
% Some text for part 4:
%\iffalse
%<*samplepart4>
%\fi
%    \begin{macrocode}
more text in part four
%    \end{macrocode}

%\iffalse
%</samplepart4>
%\fi
%
% %%%%%%%%%%%%%%%%%%%%%%%%%%%%%%%%%%%%%%
% \paragraph{Forwarding for a Complete Draft.}
%
% The following forwarding file |cdocsdrf.tex|
% compiles the main document in draft mode:
%\iffalse
%<*sampledraft>
%\fi
%    \begin{macrocode}
\def\version{draft}
\input{childdoc.def}
\childdocforward{cdocsamp}
%    \end{macrocode}

%\iffalse
%</sampledraft>
%\fi
%
% %%%%%%%%%%%%%%%%%%%%%%%%%%%%%%%%%%%%%%
% \paragraph{Forwarding for Final Version of the Chapters.}
%
% The following forwarding files |cdocsfn1.tex| and |cdocsfn2.tex|
% (with identical content)
% compile the final versions of the child documents
% |cdocsch1.tex| and |cdocsch2.tex|, respectively:
%\iffalse
%<*samplefinal>
%\fi
%    \begin{macrocode}
\def\version{final}
\input{childdoc.def}
\childdocforwardprefix[cdocsamp]{cdocsfn}{cdocsch}
%    \end{macrocode}

%\iffalse
%</samplefinal>
%\fi
%
% %%%%%%%%%%%%%%%%%%%%%%%%%%%%%%%%%%%%%%
% \paragraph{Command Line Processing.}
%
% The following three command lines generate the output files
% |cdocscld|, |cdocscl1| and |cdocscl2|
% which should be identical to
% |cdocsdrf|, |cdocsch1| and |cdocsfn2|, respectively:
% \begin{center}
% \begin{tabular}{l}
% |latex -jobname cdocscld \|\\
% |  "\def\version{draft}\input{childdoc.def}\childdocforward{cdocsamp}"|\\
% |latex -jobname cdocscl1 \|\\
% |  "\input{childdoc.def}\childdocforward[cdocsamp]{cdocsch1}"|\\
% |latex -jobname cdocscl2 \|\\
% |  "\def\version{final}\input{childdoc.def}\childdocforward{cdocsch2}"|
% \end{tabular}
% \end{center}
% Note that the trailing backslash on each first line
% merely continues the input to the second line
% (for convenient cut ant paste).
% Furthermore, the command |latex| can be replaced by any
% of its alternative versions such as |pdflatex|.
%
% %%%%%%%%%%%%%%%%%%%%%%%%%%%%%%%%%%%%%%%%%%%%%%%%%%%%%%%%%%%%%%%%%%%%%%%%%%%%%%
% %%%%%%%%%%%%%%%%%%%%%%%%%%%%%%%%%%%%%%%%%%%%%%%%%%%%%%%%%%%%%%%%%%%%%%%%%%%%%%
% \section{Implementation}
%\iffalse
%<*package>
%\fi
%
% This section describes the definitions file |childdoc.def|.

% The definitions cannot be loaded using |\usepackage| or |\RequirePackage|
% which has a mechanism to prevent loading a style file more than once.
% When loading the definitions by means of |\input|
% multiple instances have to be prevented manually:
%\iffalse
%This code needs to be before the `\ProvidesFile' directive
%which is defined at the beginning of this file.
%Therefore it is also placed there and commented out here.
%</package>
%<*discard>
%\fi
%    \begin{macrocode}
\ifdefined\childdocmain\endinput\fi
%    \end{macrocode}
%\iffalse
%</discard>
%<*package>
%\fi
%
% \macro{\ifchilddoc}
% \macro{\ifchilddocmanual}
% The conditional |\ifchilddoc| tells whether a
% child (true) or main (false) document is being compiled.
% The conditional |\ifchilddocmanual| tells whether
% the |\includeonly| mechanism is used (false) or
% the selection of child files must be performed manually (true).
% The definitions initialise to false:
%    \begin{macrocode}
\newif\ifchilddoc
\newif\ifchilddocmanual
%    \end{macrocode}

% \macro{\childdocname}
% \macro{\childdocjob}
% The macro |\childdocname| stores the name of the main document
% to be compiled. The macro |\childdocjob| stores the name of
% the document on which the \LaTeX{} compiler was originally invoked.
% The content of |\jobname| cannot be compared
% to filenames specified in the source due to different catcodes.
% The following code rescans |\jobname|, stores the result
% in |\childdocname| and saves a copy in |\childdocjob|:
%    \begin{macrocode}
\edef\childdocname{\scantokens\expandafter{\jobname\noexpand}}
\let\childdocjob\childdocname
%    \end{macrocode}

% \macro{\childdocdisable}
% The macro |\childdocdisable| prevents the main file
% from being processed more than once.
% At this stage, the main document command |\childdocmain|
% is assumed to be called once again where it should do nothing.
% Any subsequent call to it should prevent
% a secondary processing of the main document
% It overwrites the forwarding commands
% |\childdocof| and |\childdocforward|
% with empty macros to prevent further inclusions of the main document:
%    \begin{macrocode}
\newcommand{\childdocdisable}
{
  \renewcommand{\childdocmain}[1]{\renewcommand{\childdocmain}[1]{\endinput}}
  \renewcommand{\childdocof}[1]{}
  \renewcommand{\childdocby}[2][]{}
  \renewcommand{\childdocforward}[2][]{}
  \renewcommand{\childdocdisable}{}
}
%    \end{macrocode}

% \macro{\childdocmain}
% The macro |\childdocmain| is to be called at the top of the main file
% with nothing or the main filename (without extension) as argument.
% First, it breaks loops.
% If the argument is not empty and does not match |\childdocname|
% (which is set by the first inclusion of |childdoc.def|),
% |\ifchilddoc| is set to true, |\includeonly| is applied to the child file
% and |\jobname| is set to the main file
% (for proper handling of |.aux| files):
%    \begin{macrocode}
\newcommand{\childdocmain}[1]
{
  \childdocdisable\childdocmain{}
  \if?#1?\else
    \begingroup
      \def\childdoctmp{#1}
      \ifx\childdoctmp\childdocname
        \def\childdoctmp{}
      \else
        \def\childdoctmp
        {
          \childdoctrue
          \includeonly{\childdocname}
          \def\childdocjob{#1}
          \def\jobname{#1}
        }
      \fi
      \expandafter
    \endgroup
    \childdoctmp
  \fi
}
%    \end{macrocode}

% \macro{\childdocof}
% The command |\childdocof| redirects
% compilation to the main file |#1|.
%    \begin{macrocode}
\newcommand{\childdocof}[1]
{
  \childdocdisable
  \childdoctrue
  \includeonly{\childdocname}
  \def\jobname{#1}
  \def\childdocjob{#1}
  \input{#1}
}
%    \end{macrocode}

% \macro{\childdocby}
% The command |\childdocby| ....
%    \begin{macrocode}
\newcommand{\childdocby}[2][]
{
  \childdocdisable
  \childdoctrue
  \childdocmanualtrue
  \if?#1?\else
    \def\jobname{#2}
  \fi
  \def\childdocjob{#2}
  \input{#2}
  \endinput
}
%    \end{macrocode}

% \macro{\childdocforward}
% The command |\childdocforward| redirects
% compilation to the main file or
% (if the optional argument is given) a child file.
% Parameters are set as if the main file
% or a child file starting with |\childdocof| was compiled.
% Then compilation is handed over to the main file:
%    \begin{macrocode}
\newcommand{\childdocforward}[2][]
{
  \begingroup
    \if?#1?
      \def\childdoctmp
      {
        \def\childdocname{#2}
        \def\childdocjob{#2}
        \def\jobname{#2}
        \input{#2}
        \endinput
      }
    \else
      \def\childdoctmp
      {
        \childdocdisable
        \def\childdocname{#2}
        \childdoctrue
        \includeonly{#2}
        \def\childdocjob{#1}
        \def\jobname{#1}
        \input{#1}
        \endinput
      }
    \fi
    \expandafter
  \endgroup
  \childdoctmp
}
%    \end{macrocode}

% \macro{\childdocforwardprefix}
% The command |\childdocforwardprefix| redirects
% compilation to the main or a child file by means of a pattern.
% The prefix |#1| in the current filename is replaced by |#2|
% and the suffix of the current filename is kept
% (it is assumed that the filename does not contain the substring `|~~~|'
% which is used as a delimiter).
% Compilation is handed over to the new file by |\childdocforward|:
%    \begin{macrocode}
\newcommand{\childdocforwardprefix}[3][]
{
  \begingroup
    \def\childdocextract #2##1~~~{\def\childdoctmp{\childdocforward[#1]{#3##1}}}
    \expandafter\childdocextract\childdocname~~~
    \expandafter
  \endgroup
  \childdoctmp
}
%    \end{macrocode}

% \macro{\childdoc}
% The deprecated macro |\childdoc| is a legacy version of |\childdocmain|:
%    \begin{macrocode}
\newcommand{\childdoc}{\childdocmain}
%    \end{macrocode}

% \macro{\childdocredirect}
% The deprecated macro |\childdocredirect| is a legacy version
% of |\childdocforward| and |\childdocforwardprefix|:
%    \begin{macrocode}
\newcommand{\childdocredirect}[2][]
{
  \begingroup
    \if?#1?
      \def\childdoctmp{\childdocforward{#2}}
    \else
      \def\childdoctmp{\childdocforwardprefix{#1}{#2}}
    \fi
    \expandafter
  \endgroup
  \childdoctmp
}
%    \end{macrocode}

%\iffalse
%</package>
%\fi
%
\endinput
|\\
|\childdocmain{}|\\
\end{tabular}
\end{center}
at the very top of the main \LaTeX{} file,
in particular \emph{before} the |\documentclass| statement!
The argument of |\childdocmain| should be left empty
(but it must be present).

%%%%%%%%%%%%%%%%%%%%%%%%%%%%%%%%%%%%%%%%
\DescribeMacro{\childdocof}
Furthermore, add the commands
\begin{center}
\begin{tabular}{l}
|% \iffalse
%
% childdoc.dtx Copyright (C) 2017-2018 Niklas Beisert
%
% This work may be distributed and/or modified under the
% conditions of the LaTeX Project Public License, either version 1.3
% of this license or (at your option) any later version.
% The latest version of this license is in
%   http://www.latex-project.org/lppl.txt
% and version 1.3 or later is part of all distributions of LaTeX
% version 2005/12/01 or later.
%
% This work has the LPPL maintenance status `maintained'.
%
% The Current Maintainer of this work is Niklas Beisert.
%
% This work consists of the files childdoc.dtx and childdoc.ins
% and the derived files childdoc.def and cdocsamp.tex with
% cdocsch1.tex, cdocsch2.tex, cdocsdrf.tex, cdocsfn1.tex, cdocsfn2.tex.
%
%<package>\ifdefined\childdocmain\endinput\fi
%<package>\ProvidesFile{childdoc.def}[2018/12/30 v2.0 child document driver]
%<samplemain>\ProvidesFile{cdocsamp.tex}[2018/12/30 v2.0 sample for childdoc]
%<*driver>
%\ProvidesFile{childdoc.drv}[2018/12/30 v2.0 childdoc reference manual file]
\PassOptionsToClass{10pt,a4paper}{article}
\documentclass{ltxdoc}

\usepackage[margin=35mm]{geometry}
\usepackage{hyperref}
\usepackage{hyperxmp}
\usepackage[usenames]{color}

\hypersetup{colorlinks=true}
\hypersetup{pdfstartview=FitH}
\hypersetup{pdfpagemode=UseNone}
\hypersetup{pdfsource={}}
\hypersetup{pdflang={en-UK}}
\hypersetup{pdfcopyright={Copyright 2017-2018 Niklas Beisert.
  This work may be distributed and/or modified under the
  conditions of the LaTeX Project Public License, either version 1.3
  of this license or (at your option) any later version.}}
\hypersetup{pdflicenseurl={http://www.latex-project.org/lppl.txt}}
\hypersetup{pdfcontactaddress={ETH Zurich, ITP, HIT K,
  Wolfgang-Pauli-Strasse 27}}
\hypersetup{pdfcontactpostcode={8093}}
\hypersetup{pdfcontactcity={Zurich}}
\hypersetup{pdfcontactcountry={Switzerland}}
\hypersetup{pdfcontactemail={nbeisert@itp.phys.ethz.ch}}
\hypersetup{pdfcontacturl={http://people.phys.ethz.ch/\xmptilde nbeisert/}}

\newcommand{\secref}[1]{\hyperref[#1]{section \ref*{#1}}}

\parskip1ex
\parindent0pt
\let\olditemize\itemize
\def\itemize{\olditemize\parskip0pt}

\begin{document}

\title{The \textsf{childdoc} Package}
\hypersetup{pdftitle={The childdoc Package}}
\author{Niklas Beisert\\[2ex]
  Institut f\"ur Theoretische Physik\\
  Eidgen\"ossische Technische Hochschule Z\"urich\\
  Wolfgang-Pauli-Strasse 27, 8093 Z\"urich, Switzerland\\[1ex]
  \href{mailto:nbeisert@itp.phys.ethz.ch}
  {\texttt{nbeisert@itp.phys.ethz.ch}}}
\hypersetup{pdfauthor={Niklas Beisert}}
\hypersetup{pdfsubject={Manual for the LaTeX2e Package childdoc}}
\date{30 December 2018, \textsf{v2.0}}
\maketitle

\begin{abstract}\noindent
\textsf{childdoc} is a \LaTeXe{} package
that enables the direct compilation
of document sections included by |\include|
to individual files.
\end{abstract}

\begingroup
\parskip0ex
\tableofcontents
\endgroup

%%%%%%%%%%%%%%%%%%%%%%%%%%%%%%%%%%%%%%%%%%%%%%%%%%%%%%%%%%%%%%%%%%%%%%%%%%%%%%%%
%%%%%%%%%%%%%%%%%%%%%%%%%%%%%%%%%%%%%%%%%%%%%%%%%%%%%%%%%%%%%%%%%%%%%%%%%%%%%%%%
\section{Introduction}

\LaTeX{} provides a mechanism to structure a large document (such as a book)
into a main file and several child files (containing the chapters)
using the |\include| command.
This mechanism is beneficial for documents
which span hundreds of pages in order to
make the source file(s) more manageable.
Moreover, compilation can be restricted to
selected child files by means of the |\includeonly| command.
The latter feature can be used to reduce the compilation time while editing
(this was significantly more useful in the earlier days of \LaTeX{})
or to generate a smaller document which is easier to navigate.
Another application of |\includeonly| is to generate
documents consisting of selected parts of the complete document.

However, there are a few drawbacks of the plain |\include| mechanism:
\begin{itemize}
\item
The child files cannot be compiled on their own,
they can only be compiled via the main file.
A naive editing environment
(such as a text editor with an option
to have the current file processed by \LaTeX)
may require one to switch to the main file before compiling;
attempting to compile the child file produces errors.
\item
The main file must be modified (each time)
to adjust the |\includeonly| command
to the present needs. This easily leaves the main file in a messy state.
\item
The generated document will always carry the filename
of the main document. This is inconvenient if
several child files are to be compiled and
to be kept for distribution.
\end{itemize}

The present package provides a simple interface
to make child files individually compilable by \LaTeX{}.
Compiling a child file then has the same effect as compiling
the main file with an |\includeonly| command
to select the appropriate child.
Moreover the generated document will carry the name of the child
rather than the main file.
This resolves all three above issues.

This feature is meant to make the editing of books,
thesis documents and lecture notes somewhat more convenient.
However, the package can also be used efficiently for
composing a series of documents (such as exercise sheets)
which are typically distributed individually.
It then assists the author in generating the individual documents
(potentially in different versions)
as well as a document containing the collected series.
Another application is in developing style files
or other kinds of included material
where compilation of the style file could redirect
to a sample or test file.

%%%%%%%%%%%%%%%%%%%%%%%%%%%%%%%%%%%%%%%%%%%%%%%%%%%%%%%%%%%%%%%%%%%%%%%%%%%%%%%%
%%%%%%%%%%%%%%%%%%%%%%%%%%%%%%%%%%%%%%%%%%%%%%%%%%%%%%%%%%%%%%%%%%%%%%%%%%%%%%%%
\section{Usage}

First of all, the package \textsf{childdoc} is \emph{not} a standard
\LaTeXe{} |.sty| style file! Therefore it needs to be invoked in
a non-standard way.

%%%%%%%%%%%%%%%%%%%%%%%%%%%%%%%%%%%%%%%%%%%%%%%%%%%%%%%%%%%%%%%%%%%%%%%%%%%%%%%%
\subsection{Included Files}
\label{sec:include}

%%%%%%%%%%%%%%%%%%%%%%%%%%%%%%%%%%%%%%%%
\DescribeMacro{\childdocmain}
To use the package, add the commands
\begin{center}
\begin{tabular}{l}
|\input{childdoc.def}|\\
|\childdocmain{}|\\
\end{tabular}
\end{center}
at the very top of the main \LaTeX{} file,
in particular \emph{before} the |\documentclass| statement!
The argument of |\childdocmain| should be left empty
(but it must be present).

%%%%%%%%%%%%%%%%%%%%%%%%%%%%%%%%%%%%%%%%
\DescribeMacro{\childdocof}
Furthermore, add the commands
\begin{center}
\begin{tabular}{l}
|\input{childdoc.def}|\\
|\childdocof{|\textit{main}|}|\\
\end{tabular}
\end{center}
at the top of every child file \textit{child}
which is included by |\include{|\textit{child}|}|
from within the main file
(or at least for those files to be compiled individually).
The argument \textit{main} must be the filename of the main file.

There are a couple of
considerations in setting up the main and child documents:

%%%%%%%%%%%%%%%%%%%%%%%%%%%%%%%%%%%%%%%%
\paragraph{Restrictions.}

Please note the following restrictions:
\begin{itemize}
\item
|\childdocmain| must be called with one argument \textit{main}
to ensure compatibility with earlier version of the package.
It must either be empty (|\childdocmain{}|)
or precisely match the filename of the main file in which it is specified.
See \secref{sec:detection} for further information.
\item
The filename \textit{main} must be specified without the |.tex| extension.
\item
The filename \textit{main} is case sensitive
(even in case-insensitive file systems)
due to internal string comparison.
\item
The argument \textit{main} should be fully expanded, it cannot be a macro.
\item
Subdirectories and special characters should be avoided in filenames.
\item
The command |\childdocmain{|\textit{main}|}| must be followed by a whitespace.
It should not be followed immediately by another command
or by a comment mark `|%|'.
This is because the \TeX{} parser reads the token immediately following
the argument of |\childdocmain| and puts it
at the beginning of every child section;
however, a white\-space is ignored.
\end{itemize}

%%%%%%%%%%%%%%%%%%%%%%%%%%%%%%%%%%%%%%%%
\paragraph{Content of Main File.}

It is advisable to place all content in the child files included by |\include|.
Any output contained in the main file will appear in all child documents
unless suppressed manually;
it cannot be suppressed automatically by the |\includeonly| directive
and thus should normally be avoided.
A method to include some content in the main file
by means of conditional processing is described in \secref{sec:conditional}.

%%%%%%%%%%%%%%%%%%%%%%%%%%%%%%%%%%%%%%%%
\paragraph{Page Numbering.}

When only a part of the document is compiled,
the appropriate numbering of pages
(as well as other status parameters)
is determined from the |.aux| files.
The latter contain information from previous passes.
However this information needs to propagate through
all intermediate child documents.
Therefore the page numbering in child documents may well
be inconsistent until the complete document is compiled at least once.

A useful (if unconventional) way to always ensure a consistent
page numbering is to restart the numbering in each child document
and denote the pages by `\textit{child}|.|\textit{page}'
where \textit{child} represents the chapter/section number of the child file.
This can be achieved by the command
|\numberwithin{page}{|\textit{child}|}|
of the \textsf{amsmath} package
where \textit{child} can be |chapter| or |section|
depending on the chosen structuring.
Alternatively, one can modify the macro |\thepage| appropriately
and reset the counter |page| at the start of each child file.

%%%%%%%%%%%%%%%%%%%%%%%%%%%%%%%%%%%%%%%%%%%%%%%%%%%%%%%%%%%%%%%%%%%%%%%%%%%%%%%%
\subsection{Conditional Processing}
\label{sec:conditional}

The package provides a mechanism to compile different versions
of a document. To customise the versions further some conditional processing
can come in handy to distinguish which version is being compiled.
The package provides two macros to describe the compilation context:

%%%%%%%%%%%%%%%%%%%%%%%%%%%%%%%%%%%%%%%%
\DescribeMacro{\ifchilddoc}
The conditional |\ifchilddoc| distinguishes between the compilation of
child documents and the main document:
%
\begin{center}
|\ifchilddoc |\textit{child-code}| |[|\||else |\textit{main-code}]| \||fi|
\end{center}

%%%%%%%%%%%%%%%%%%%%%%%%%%%%%%%%%%%%%%%%
\DescribeMacro{\childdocname}
\DescribeMacro{\childdocjob}
The macro |\childdocname| contains the filename (without extension)
of the main or child file being processed.
Note that |\childdocjob| will always contain the name of the main file.

%%%%%%%%%%%%%%%%%%%%%%%%%%%%%%%%%%%%%%%%
\paragraph{Title Page.}

Conditional processing can be used to include a title or banner page
in the main document when proper precautions are taken.
Importantly, the code in the main file should ensure that the page counter
(as well as other status parameters which are stored in the |.aux| files)
takes the same value after the conditional processing.
Otherwise the page numbers may take divergent values
depending on which part is compiled.

For example, a title page could be declared by:
%
\begin{center}
\begin{tabular}{l}
|\ifchilddoc\||else|\\
|\addtocounter{page}{-1}|\\
\textit{code for title page}\\
|\newpage|\\
|\||fi|
\end{tabular}
\end{center}
%
A banner page for the child documents can be generated by:
%
\begin{center}
\begin{tabular}{l}
|\ifchilddoc|\\
|\addtocounter{page}{-1}|\\
\textit{code for banner page}\\
|\newpage|\\
|\||fi|
\end{tabular}
\end{center}
%
Here one could write a message such as:
\begin{center}
|This is the part \childdocname{} of \childdocjob{}.|
\end{center}

%%%%%%%%%%%%%%%%%%%%%%%%%%%%%%%%%%%%%%%%%%%%%%%%%%%%%%%%%%%%%%%%%%%%%%%%%%%%%%%%
\subsection{Flags}
\label{sec:flags}

The package makes it easy to generate different versions
of the main or child documents.
To this end compilation flags can be defined
and assigned different default values.
They will be particularly useful in conjunction
with the forwarding mechanism described in \secref{sec:forward}.

For example, it may be useful to have a flag |\version|
which can be set to |draft| or |final|.
The document source will contain some conditional code
depending on the value of |\version|.
Suppose further, the flag should default to |final| for the main file
and to |draft| for child files
which is a natural assignment for editing the document.
This is achieved by placing the following code
in the preamble of the main document
(below the |\childdocmain| directive):
%
\begin{center}
\begin{tabular}{l}
|\ifchilddoc|\\
|\providecommand{\version}{draft}|\\
|\||else|\\
|\providecommand{\version}{final}|\\
|\||fi|
\end{tabular}
\end{center}
%
The definition by |\providecommand| makes sure
that previous definitions are not overwritten.
Further statements |\providecommand{\version}{...}|
can thus be added before the above code to override it.

For the main file, one might add a line
(between |\childdocmain| and the above block)
%
\begin{center}
|%\ifchilddoc\||else\providecommand{\version}{draft}\||fi|
\end{center}
%
which can be uncommented to produce a draft version.
Likewise one can add a line to the very top of a child file
(above the |\childdocof{|\textit{main}|}| directive)
%
\begin{center}
|%\providecommand{\version}{final}|
\end{center}
%
which can be uncommented to produce the final version of this child document.

%%%%%%%%%%%%%%%%%%%%%%%%%%%%%%%%%%%%%%%%%%%%%%%%%%%%%%%%%%%%%%%%%%%%%%%%%%%%%%%%
\subsection{Forwarding}
\label{sec:forward}

Different versions of the main or child documents
using compilation flags as described in \secref{sec:flags}
can be (permanently) stored in different files
for convenient compilation, viewing and distribution.
To this end, the package defines a command
to pass on compilation to a different file:

%%%%%%%%%%%%%%%%%%%%%%%%%%%%%%%%%%%%%%%%
\DescribeMacro{\childdocforward}
The command |\childdocforward| redirects processing to
another source file:
%
\begin{center}
\begin{tabular}{l}
|\input{childdoc.def}|\\
|\childdocforward[|\textit{main}|]{|\textit{dest}|}|\\
\end{tabular}
\end{center}
%
The argument \textit{dest} is the destination file
(without extension).
It should be the main file or one of the child files.
Note that further \textsf{childdoc} directives
such as |\childdocof| and |\childdocforward|
in the indicated file will be processed in this form.
The optional argument \textit{main}
passes on directly to the main file \textit{main}
while pretending to compile the child \textit{dest}.
This form behaves as if \textit{dest}
issues |\childdocof{|\textit{main}|}| right away,
and no further \textsf{childdoc} directives will be processed.

%%%%%%%%%%%%%%%%%%%%%%%%%%%%%%%%%%%%%%%%
\DescribeMacro{\...prefix}
In the alternative form |\childdocforwardprefix|,
%
\begin{center}
\begin{tabular}{l}
|\input{childdoc.def}|\\
|\childdocforwardprefix[|\textit{main}|]{|\textit{prefix}|}{|\textit{dest}|}|
\end{tabular}
\end{center}
%
the destination file is determined by a pattern
depending on the current file:
To make this work, the current file must be called
`{\textit{prefix}\hspace{0.2em}\textit{suffix}}'
with \textit{prefix} matching precisely the argument.
Processing is then passed on to the file
`{\textit{dest}\hspace{0.2em}\textit{suffix}}'.
Surely, the same effect is achieved by
directly specifying the
argument `{\textit{dest}\hspace{0.2em}\textit{suffix}}'
in the first form.
However, that requires to set up a different file
for each child. With the alternative form of the command
all these files can have exactly the same content
which simplifies setting them up and maintaining them.

For example, the following file |draft.tex|
with a compilation flag |\version| as described in \secref{sec:flags}
compiles the main document as a draft:
%
\begin{center}
\begin{tabular}{l}
|\def\version{draft}|\\
|\input{childdoc.def}|\\
|\childdocforward{|\textit{main}|}|
\end{tabular}
\end{center}
%
Likewise, the following files |final|\textit{nn}|.tex|
compile the final version of the child document
|child|\textit{nn}|.tex|:
%
\begin{center}
\begin{tabular}{l}
|\def\version{final}|\\
|\input{childdoc.def}|\\
|\childdocforwardprefix{final}{child}|
\end{tabular}
\end{center}
%

Note that when several versions of a main file and/or of each child file
are to be generated, it may be convenient to set up a |Makefile| or
shell script to automatise the process.

%%%%%%%%%%%%%%%%%%%%%%%%%%%%%%%%%%%%%%%%%%%%%%%%%%%%%%%%%%%%%%%%%%%%%%%%%%%%%%%%
\subsection{Command Line Processing}
\label{sec:commandline}

The effect of redirection files can also be achieved by invoking
the \LaTeX{} compiler with a more elaborate command line.
Most conveniently this should be done as part
of a shell script or a |Makefile|.

When using \textsf{childdoc} in the main file, the following
command lines effectively perform a redirection
(note that depending on the shell being used,
backslashes may have to be doubled: `|\|' $\to$ `|\\|'):
%
\begin{center}
|... -jobname "|\textit{target}|" |\\|"|[\textit{flags}]%
|\input{childdoc.def}\childdocforward[|\textit{main}|]{|\textit{dest}|}"|
\end{center}
%
Here \textit{target} is the name of the output file,
\textit{main} is the name of the main file
and \textit{dest} is the name of the main or child file to be processed
(all filenames without extensions).
The optional argument \textit{main} can be omitted
if \textit{main} matches \textit{dest}.
Optionally, compilation \textit{flags} can be defined via |\def| commands.
This command line makes the \TeX{} engine believe
it is compiling the file \textit{target}
whose content is specified as the latter parameter.
The provided code then forwards the processing to
\textit{main} or \textit{dest} as described in \secref{sec:forward}.

%%%%%%%%%%%%%%%%%%%%%%%%%%%%%%%%%%%%%%%%%%%%%%%%%%%%%%%%%%%%%%%%%%%%%%%%%%%%%%%%
\subsection{Include by Input}
\label{sec:input}

Including child documents by |\include| has some restrictions by design.
Most notably, the content of a child document always occupies
its own set of pages; pages cannot be shared between child documents.
Usually, this behaviour makes perfect sense
because each child document contain an essential part of the document.
However, in some situations it may be desirable to compose
a document from a collection of parts
without having mandatory page breaks between then.
For this case, the package
provides a mechanism to include parts
by |\input| which can also be processed individually.
However, by construction this mechanism
requires manual handling of the content to be output.

%%%%%%%%%%%%%%%%%%%%%%%%%%%%%%%%%%%%%%%%
\DescribeMacro{\ifchilddocmanual}
The main file should be prepared as usual, see \secref{sec:include}.
However, the document body must make a distinction
between processing of an individual part and of the main document, e.g.:
%
\begin{center}
\begin{tabular}{l}
|\ifchilddocmanual|\\
|\input{\childdocname}|\\
|\||else|\\
\textit{document body with }|\input{|\textit{part}|}|\\
|\||fi|
\end{tabular}
\end{center}
%
The conditional |\ifchilddocmanual| is true whenever
a part to be included by |\input| is being compiled,
and the name of the part is stored in |\childdocname|.

%%%%%%%%%%%%%%%%%%%%%%%%%%%%%%%%%%%%%%%%
\DescribeMacro{\childdocby}
Each part to be included by |\input| should start with:
%
\begin{center}
\begin{tabular}{l}
|\input{childdoc.def}|\\
|\childdocby{|\textit{main}|}|\\
\end{tabular}
\end{center}
%
The directive |\childdocby| is similar to |\childdocof|
described in \secref{sec:include},
but the subsequent selection of content must be done manually.
To that end, both |\ifchilddoc| and |\ifchilddocmanual|
will be true upon processing of a part,
and the name of the part is stored in |\childdocname|.
Note that |\jobname| will be set to the filename of the current part
so that each part receives an individual |.aux| file
that does not interfere with the |.aux| file(s) of the main document.
This behaviour can be altered by the alternative form
|\childdocby[*]{|\textit{main}|}| (with a non-empty optional argument)
which uses the |.aux| file of the main document
by setting |\jobname| to \textit{main}.

%%%%%%%%%%%%%%%%%%%%%%%%%%%%%%%%%%%%%%%%%%%%%%%%%%%%%%%%%%%%%%%%%%%%%%%%%%%%%%%%
\subsection{Driver Development}
\label{sec:driver}

The \textsf{childdoc} mechanism can also be use for the development
of definition files such as \LaTeX{} styles or classes.
This case differs from the above setup with multiple parts
included by |\include| in that no |\includeonly| should be invoked.
This can be achieved by starting the include file
(before |\ProvidesPackage|) with:
%
\begin{center}
\begin{tabular}{l}
|\input{childdoc.def}|\\
|\childdocforward{|\textit{main}|}|\\
\end{tabular}
\end{center}
%
or alternatively with:
%
\begin{center}
\begin{tabular}{l}
|\input{childdoc.def}|\\
|\childdocby{|\textit{main}|}|\\
\end{tabular}
\end{center}
%
Both forms have slightly different effects as described above.
The main file is prepared as usual, see \secref{sec:include}.

%%%%%%%%%%%%%%%%%%%%%%%%%%%%%%%%%%%%%%%%%%%%%%%%%%%%%%%%%%%%%%%%%%%%%%%%%%%%%%%%
\subsection{Legacy Detection}
\label{sec:detection}

The directive |\childdocmain| in the main file can detect
whether the complete document or merely a child is to be compiled
even without using the directive |\childdocof|.
This method is deprecated because it is less robust
and there is no compelling reason to use it;
it is merely provided for backward compatibility
and it may be removed in future versions.

If the detection mechanism is to be used,
it is mandatory to correctly specify
the filename of the main file as the argument of |\childdocmain|:
%
\begin{center}
\begin{tabular}{l}
|\input{childdoc.def}|\\
|\childdocmain{|\textit{main}|}|\\
\end{tabular}
\end{center}
%
If |\jobname| does not match the argument \textit{main} of |\childdocmain|,
it is assumed that |\jobname| points to the child file to be compiled.
When using |\childdocmain| with the main file specified as argument,
it suffices to start a child file
with just |\input{|\textit{main}|}|
without loading of the package and using |\childdocof|.
If instead all processing is done
with the appropriate \textsf{childdoc} directives,
the argument of \textit{main} of |\childdocmain| can be empty.

An alternative version of the command line processing described
in \secref{sec:commandline} using the detection mechanism reads:
%
\begin{center}
|... -jobname "|\textit{target}|" "|[\textit{flags}]%
[|\def\jobname{|\textit{dest}|}|]|\input{|\textit{main}|}"|
\end{center}

%%%%%%%%%%%%%%%%%%%%%%%%%%%%%%%%%%%%%%%%%%%%%%%%%%%%%%%%%%%%%%%%%%%%%%%%%%%%%%%%
\subsection{Manual Code}
\label{sec:manual}

In case one cannot be certain whether the definitions file |childdoc.def|
is installed on the target \TeX{} distribution
and one prefers not to ship it,
it is conceivable to paste a few relevant commands into the sources.

To that end, drop all statements |\input{childdoc.def}|
and perform the replacements as outlined below.
Instead of |\childdocmain{|\textit{main}|}| add the following code
to the top of the main file:
%
\begin{center}
\begin{tabular}{l}
|\||ifdefined\childdocname\endinput\||fi\newif\ifchilddoc|\\
|\edef\childdocname{\scantokens\expandafter{\jobname\noexpand}}|\\
|\def\childdocmain{|\textit{main}|}\||ifx\childdocmain\childdocname\||else|\\
|\childdoctrue\includeonly{\childdocname}\let\jobname\childdocmain\||fi|\\
\end{tabular}
\end{center}
%
Instead of |\childdocof{|\textit{main}|}| just include the main file
at the top of each child file:
%
\begin{center}
|\input{|\textit{main}|}|
\end{center}
%
A simple redirection |\childdocforward{|\textit{dest}|}| is achieved by:
%
\begin{center}
|\def\jobname{|\textit{dest}|}\input{\jobname}|
\end{center}
%
The redirection with prefix
|\childdocforwardprefix[|\textit{prefix}|]{|\textit{dest}|}|
is accomplished by:
%
\begin{center}
\begin{tabular}{l}
|{\edef\jobname{\scantokens\expandafter{\jobname\noexpand}}|\\
|\def\redirectjob |\textit{prefix}|#1~~~{\gdef\jobname{|\textit{dest}|#1}}|\\
|\expandafter\redirectjob\jobname~~~}\input{\jobname}|
\end{tabular}
\end{center}

In an alternative approach,
child documents can be compiled by a specific command line
without additional code or specific definitions:
%
\begin{center}
|... -jobname "|\textit{target}|" "|[\textit{flags}]%
|\includeonly{|\textit{dest}|}\input{|\textit{main}|}"|
\end{center}
%

%%%%%%%%%%%%%%%%%%%%%%%%%%%%%%%%%%%%%%%%%%%%%%%%%%%%%%%%%%%%%%%%%%%%%%%%%%%%%%%%
%%%%%%%%%%%%%%%%%%%%%%%%%%%%%%%%%%%%%%%%%%%%%%%%%%%%%%%%%%%%%%%%%%%%%%%%%%%%%%%%
\section{Information}

%%%%%%%%%%%%%%%%%%%%%%%%%%%%%%%%%%%%%%%%%%%%%%%%%%%%%%%%%%%%%%%%%%%%%%%%%%%%%%%%
\subsection{Copyright}

Copyright \copyright{} 2017--2018 Niklas Beisert

This work may be distributed and/or modified under the
conditions of the \LaTeX{} Project Public License, either version 1.3
of this license or (at your option) any later version.
The latest version of this license is in
  \url{http://www.latex-project.org/lppl.txt}
and version 1.3 or later is part of all distributions of \LaTeX{}
version 2005/12/01 or later.

This work has the LPPL maintenance status `maintained'.

The Current Maintainer of this work is Niklas Beisert.

This work consists of the files |README.txt|, |childdoc.ins| and |childdoc.dtx|
as well as the derived files |childdoc.def|, |cdocsamp.tex|
with |cdocsch1.tex|, |cdocsch2.tex|, |cdocspt3.tex|, |cdocspt4.tex|,
|cdocsdrf.tex|, |cdocsfn1.tex|, |cdocsfn2.tex|
as well as |childdoc.pdf|.

%%%%%%%%%%%%%%%%%%%%%%%%%%%%%%%%%%%%%%%%%%%%%%%%%%%%%%%%%%%%%%%%%%%%%%%%%%%%%%%%
\subsection{Files and Installation}

The package consists of the files:
%
\begin{center}
\begin{tabular}{ll}
    |README.txt|   & readme file \\
    |childdoc.ins| & installation file \\
    |childdoc.dtx| & source file \\
    |childdoc.def| & definition file \\
    |cdocsamp.tex| & sample main file \\
    |cdocsch1.tex| & sample include file \\
    |cdocsch2.tex| & sample include file \\
    |cdocspt3.tex| & sample part file \\
    |cdocspt4.tex| & sample part file \\
    |cdocsdrf.tex| & sample redirection file \\
    |cdocsfn1.tex| & sample redirection file \\
    |cdocsfn2.tex| & sample redirection file \\
    |childdoc.pdf| & manual
\end{tabular}
\end{center}
%
The distribution consists of the files
|README.txt|, |childdoc.ins| and |childdoc.dtx|.
%
\begin{itemize}
\item
Run (pdf)\LaTeX{} on |childdoc.dtx|
to compile the manual |childdoc.pdf| (this file).
\item
Run \LaTeX{} on |childdoc.ins| to create the definitions file |childdoc.def|
and the sample |cdocsamp.tex| with include files
|cdocsch1.tex|, |cdocsch2.tex|, |cdocspt3.tex|, |cdocspt4.tex|,
|cdocsdrf.tex|, |cdocsfn1.tex|, |cdocsfn2.tex|.
Then copy the file |childdoc.def| to an appropriate directory of your \LaTeX{}
distribution, e.g.\ \textit{texmf-root}|/tex/latex/childdoc|.
\end{itemize}

%%%%%%%%%%%%%%%%%%%%%%%%%%%%%%%%%%%%%%%%%%%%%%%%%%%%%%%%%%%%%%%%%%%%%%%%%%%%%%%%
\subsection{Related CTAN Packages}

There are several other packages which offer a similar functionality:
%
\begin{itemize}
\item
The packages
\href{http://ctan.org/pkg/docmute}{\textsf{docmute}},
\href{http://ctan.org/pkg/includex}{\textsf{includex}} and
\href{http://ctan.org/pkg/standalone}{\textsf{standalone}}
provide commands to include only the document body of
a child file thus allowing both files to be compiled individually.
\item
The packages \href{http://ctan.org/pkg/subdocs}{\textsf{subdocs}}
and \href{http://ctan.org/pkg/subfiles}{\textsf{subfiles}}
provide structures in which the main and child documents can be
encapsulated and allowing them to be compiled individually.
The inclusion mechanism is different from the conventional |\include|.
\item
The package \href{http://ctan.org/pkg/combine}{\textsf{combine}}
is an elaborate solution to combine several documents into one.
\end{itemize}
%
See also the CTAN topic \href{http://ctan.org/topic/subdocs}{\textsf{subdocs}}
for further related packages.
The present package differs from the above solutions in that
a document structure constructed with the conventional |\include| mechanism
just needs two extra commands at the top of every file
such that all constituent files can be compiled individually.

%%%%%%%%%%%%%%%%%%%%%%%%%%%%%%%%%%%%%%%%%%%%%%%%%%%%%%%%%%%%%%%%%%%%%%%%%%%%%%%%
%\subsection{Feature Suggestions}
%
%The following is a list of features which may be useful for future
%versions of this package:
%%
%\begin{itemize}
%\item
%\ldots
%\end{itemize}

%%%%%%%%%%%%%%%%%%%%%%%%%%%%%%%%%%%%%%%%%%%%%%%%%%%%%%%%%%%%%%%%%%%%%%%%%%%%%%%%
\subsection{Revision History}

%%%%%%%%%%%%%%%%%%%%%%%%%%%%%%%%%%%%%%%%
\paragraph{v2.0:} 2018/12/30

\begin{itemize}
\item
immediate forward processing
\item
added |\childdocby| mechanism
\item
manual restructured
\end{itemize}

%%%%%%%%%%%%%%%%%%%%%%%%%%%%%%%%%%%%%%%%
\paragraph{v1.6:} 2018/01/17

\begin{itemize}
\item
application for development of include files
\item
corrections to manual
\end{itemize}

%%%%%%%%%%%%%%%%%%%%%%%%%%%%%%%%%%%%%%%%
\paragraph{v1.5:} 2017/05/21

\begin{itemize}
\item
more complete structuring introduced
\item
|\childdocof| introduced
\item
|\childdoc| renamed to |\childdocmain|
\item
|\childredirect| renamed to |\childdocforward| and |\childdocforwardprefix|
and functionality expanded
\end{itemize}

%%%%%%%%%%%%%%%%%%%%%%%%%%%%%%%%%%%%%%%%
\paragraph{v1.0:} 2017/04/27

\begin{itemize}
\item
manual and install package
\item
first version published on CTAN
\end{itemize}

%%%%%%%%%%%%%%%%%%%%%%%%%%%%%%%%%%%%%%%%
\paragraph{v0.6:} 2017/04/26

\begin{itemize}
\item
redirection mechanism added
\end{itemize}

%%%%%%%%%%%%%%%%%%%%%%%%%%%%%%%%%%%%%%%%
\paragraph{v0.5:} 2017/04/26

\begin{itemize}
\item
functionality in definition file
\end{itemize}


%%%%%%%%%%%%%%%%%%%%%%%%%%%%%%%%%%%%%%%%%%%%%%%%%%%%%%%%%%%%%%%%%%%%%%%%%%%%%%%%
%%%%%%%%%%%%%%%%%%%%%%%%%%%%%%%%%%%%%%%%%%%%%%%%%%%%%%%%%%%%%%%%%%%%%%%%%%%%%%%%
%%%%%%%%%%%%%%%%%%%%%%%%%%%%%%%%%%%%%%%%%%%%%%%%%%%%%%%%%%%%%%%%%%%%%%%%%%%%%%%%
\appendix

\settowidth\MacroIndent{\rmfamily\scriptsize 000\ }

 \DocInput{childdoc.dtx}

\end{document}
%</driver>
% \fi
%
% %%%%%%%%%%%%%%%%%%%%%%%%%%%%%%%%%%%%%%%%%%%%%%%%%%%%%%%%%%%%%%%%%%%%%%%%%%%%%%
% %%%%%%%%%%%%%%%%%%%%%%%%%%%%%%%%%%%%%%%%%%%%%%%%%%%%%%%%%%%%%%%%%%%%%%%%%%%%%%
% \section{Sample}
%\iffalse
%<*samplemain>
%\fi
%
% The following presents a sample document
% with two chapters, two parts, a title page,
% a compile flag as well as three forwarding files to set the flag.
% It consists of eight |.tex| files:
% \begin{center}
% \begin{tabular}{ll}
% |cdocsamp.tex|&main file\\
% |cdocsch1.tex|&include file for chapter 1\\
% |cdocsch2.tex|&include file for chapter 2\\
% |cdocspt3.tex|&include file for part 3\\
% |cdocspt4.tex|&include file for part 4\\
% |cdocsdrf.tex|&forwarding file for main file in draft mode\\
% |cdocsfi1.tex|&forwarding file for final version of chapter 1\\
% |cdocsfi2.tex|&forwarding file for final version of chapter 2\\
% \end{tabular}
% \end{center}
% Each of the eight files can be compiled directly by the \LaTeX{} compiler.
%
% %%%%%%%%%%%%%%%%%%%%%%%%%%%%%%%%%%%%%%
% \paragraph{Main File.}
%
% The main file is called |cdocsamp.tex|.
%
% Load the \textsf{childdoc} definitions and
% declare the filename for the main document:
%    \begin{macrocode}
\input{childdoc.def}
\childdocmain{}
%    \end{macrocode}

% Optional override for |\version| flag:
%    \begin{macrocode}
%%\ifchilddoc\else\providecommand{\version}{draft}\fi
%    \end{macrocode}

% Define the default values for the |\version| flag
% (|final| for the main file and |draft| for childs):
%    \begin{macrocode}
\ifchilddoc
\providecommand{\version}{draft}
\else
\providecommand{\version}{final}
\fi
%    \end{macrocode}

% Load the standard document class:
%    \begin{macrocode}
\documentclass[12pt]{article}
%    \end{macrocode}

% Start the document body:
%    \begin{macrocode}
\begin{document}
%    \end{macrocode}

% Declare a title page.
% Print title, part of document being processed and version flag:
%    \begin{macrocode}
\addtocounter{page}{-1}
\begin{center}
{\LARGE\bfseries{}childdoc example\par}
\vspace{1cm}
\ifchilddoc
\ifchilddocmanual part\else chapter\fi:
`\childdocname' of `\childdocjob'\par
\else
main document: `\childdocjob'\par
\fi
version: \version\par
\end{center}
\newpage
%    \end{macrocode}

% Manually include selected file,
% otherwise process as usual:
%    \begin{macrocode}
\ifchilddocmanual
\section*{part `\childdocname'}
\input{\childdocname}
\else
%    \end{macrocode}

% Include the two chapters:
%    \begin{macrocode}
\include{cdocsch1}
\include{cdocsch2}
%    \end{macrocode}

% Include the two parts unless only chapters should be displayed:
%    \begin{macrocode}
\ifchilddoc\else
\section{part three}
\input{cdocspt3}
\section{part four}
\input{cdocspt4}
\fi
%    \end{macrocode}

% Process as usual until here:
%    \begin{macrocode}
\fi
%    \end{macrocode}

% End of document body:
%    \begin{macrocode}
\end{document}
%    \end{macrocode}
%\iffalse
%</samplemain>
%\fi
%
% %%%%%%%%%%%%%%%%%%%%%%%%%%%%%%%%%%%%%%
% \paragraph{Chapter Include Files.}
%
% The include files are called |cdocsch1.tex| and |cdocsch2.tex|.
%
%\iffalse
%<*samplechap1|samplechap2>
%\fi

% Optional override for |\version| flag:
%    \begin{macrocode}
%%\providecommand{\version}{final}
%    \end{macrocode}

% Include the main document:
%    \begin{macrocode}
\input{childdoc.def}
\childdocof{cdocsamp}
%    \end{macrocode}

%\iffalse
%</samplechap1|samplechap2>
%\fi
%
%\iffalse
%<*samplechap1>
%\fi
% Some text for chapter 1:
%    \begin{macrocode}
\section{one}
some text in chapter one
%    \end{macrocode}

%\iffalse
%</samplechap1>
%\fi
% Some text for chapter 2:
%\iffalse
%<*samplechap2>
%\fi
%    \begin{macrocode}
\section{two}
more text in chapter two
%    \end{macrocode}

%\iffalse
%</samplechap2>
%\fi
%
% %%%%%%%%%%%%%%%%%%%%%%%%%%%%%%%%%%%%%%
% \paragraph{Part Include Files.}
%
% The include files are called |cdocspt3.tex| and |cdocspt4.tex|.
%
%\iffalse
%<*samplepart3|samplepart4>
%\fi

% Optional override for |\version| flag:
%    \begin{macrocode}
%%\providecommand{\version}{final}
%    \end{macrocode}

% Include the main document:
%    \begin{macrocode}
\input{childdoc.def}
\childdocby{cdocsamp}
%    \end{macrocode}

%\iffalse
%</samplepart3|samplepart4>
%\fi
%
%\iffalse
%<*samplepart3>
%\fi
% Some text for part 3:
%    \begin{macrocode}
some text in part three
%    \end{macrocode}

%\iffalse
%</samplepart3>
%\fi
% Some text for part 4:
%\iffalse
%<*samplepart4>
%\fi
%    \begin{macrocode}
more text in part four
%    \end{macrocode}

%\iffalse
%</samplepart4>
%\fi
%
% %%%%%%%%%%%%%%%%%%%%%%%%%%%%%%%%%%%%%%
% \paragraph{Forwarding for a Complete Draft.}
%
% The following forwarding file |cdocsdrf.tex|
% compiles the main document in draft mode:
%\iffalse
%<*sampledraft>
%\fi
%    \begin{macrocode}
\def\version{draft}
\input{childdoc.def}
\childdocforward{cdocsamp}
%    \end{macrocode}

%\iffalse
%</sampledraft>
%\fi
%
% %%%%%%%%%%%%%%%%%%%%%%%%%%%%%%%%%%%%%%
% \paragraph{Forwarding for Final Version of the Chapters.}
%
% The following forwarding files |cdocsfn1.tex| and |cdocsfn2.tex|
% (with identical content)
% compile the final versions of the child documents
% |cdocsch1.tex| and |cdocsch2.tex|, respectively:
%\iffalse
%<*samplefinal>
%\fi
%    \begin{macrocode}
\def\version{final}
\input{childdoc.def}
\childdocforwardprefix[cdocsamp]{cdocsfn}{cdocsch}
%    \end{macrocode}

%\iffalse
%</samplefinal>
%\fi
%
% %%%%%%%%%%%%%%%%%%%%%%%%%%%%%%%%%%%%%%
% \paragraph{Command Line Processing.}
%
% The following three command lines generate the output files
% |cdocscld|, |cdocscl1| and |cdocscl2|
% which should be identical to
% |cdocsdrf|, |cdocsch1| and |cdocsfn2|, respectively:
% \begin{center}
% \begin{tabular}{l}
% |latex -jobname cdocscld \|\\
% |  "\def\version{draft}\input{childdoc.def}\childdocforward{cdocsamp}"|\\
% |latex -jobname cdocscl1 \|\\
% |  "\input{childdoc.def}\childdocforward[cdocsamp]{cdocsch1}"|\\
% |latex -jobname cdocscl2 \|\\
% |  "\def\version{final}\input{childdoc.def}\childdocforward{cdocsch2}"|
% \end{tabular}
% \end{center}
% Note that the trailing backslash on each first line
% merely continues the input to the second line
% (for convenient cut ant paste).
% Furthermore, the command |latex| can be replaced by any
% of its alternative versions such as |pdflatex|.
%
% %%%%%%%%%%%%%%%%%%%%%%%%%%%%%%%%%%%%%%%%%%%%%%%%%%%%%%%%%%%%%%%%%%%%%%%%%%%%%%
% %%%%%%%%%%%%%%%%%%%%%%%%%%%%%%%%%%%%%%%%%%%%%%%%%%%%%%%%%%%%%%%%%%%%%%%%%%%%%%
% \section{Implementation}
%\iffalse
%<*package>
%\fi
%
% This section describes the definitions file |childdoc.def|.

% The definitions cannot be loaded using |\usepackage| or |\RequirePackage|
% which has a mechanism to prevent loading a style file more than once.
% When loading the definitions by means of |\input|
% multiple instances have to be prevented manually:
%\iffalse
%This code needs to be before the `\ProvidesFile' directive
%which is defined at the beginning of this file.
%Therefore it is also placed there and commented out here.
%</package>
%<*discard>
%\fi
%    \begin{macrocode}
\ifdefined\childdocmain\endinput\fi
%    \end{macrocode}
%\iffalse
%</discard>
%<*package>
%\fi
%
% \macro{\ifchilddoc}
% \macro{\ifchilddocmanual}
% The conditional |\ifchilddoc| tells whether a
% child (true) or main (false) document is being compiled.
% The conditional |\ifchilddocmanual| tells whether
% the |\includeonly| mechanism is used (false) or
% the selection of child files must be performed manually (true).
% The definitions initialise to false:
%    \begin{macrocode}
\newif\ifchilddoc
\newif\ifchilddocmanual
%    \end{macrocode}

% \macro{\childdocname}
% \macro{\childdocjob}
% The macro |\childdocname| stores the name of the main document
% to be compiled. The macro |\childdocjob| stores the name of
% the document on which the \LaTeX{} compiler was originally invoked.
% The content of |\jobname| cannot be compared
% to filenames specified in the source due to different catcodes.
% The following code rescans |\jobname|, stores the result
% in |\childdocname| and saves a copy in |\childdocjob|:
%    \begin{macrocode}
\edef\childdocname{\scantokens\expandafter{\jobname\noexpand}}
\let\childdocjob\childdocname
%    \end{macrocode}

% \macro{\childdocdisable}
% The macro |\childdocdisable| prevents the main file
% from being processed more than once.
% At this stage, the main document command |\childdocmain|
% is assumed to be called once again where it should do nothing.
% Any subsequent call to it should prevent
% a secondary processing of the main document
% It overwrites the forwarding commands
% |\childdocof| and |\childdocforward|
% with empty macros to prevent further inclusions of the main document:
%    \begin{macrocode}
\newcommand{\childdocdisable}
{
  \renewcommand{\childdocmain}[1]{\renewcommand{\childdocmain}[1]{\endinput}}
  \renewcommand{\childdocof}[1]{}
  \renewcommand{\childdocby}[2][]{}
  \renewcommand{\childdocforward}[2][]{}
  \renewcommand{\childdocdisable}{}
}
%    \end{macrocode}

% \macro{\childdocmain}
% The macro |\childdocmain| is to be called at the top of the main file
% with nothing or the main filename (without extension) as argument.
% First, it breaks loops.
% If the argument is not empty and does not match |\childdocname|
% (which is set by the first inclusion of |childdoc.def|),
% |\ifchilddoc| is set to true, |\includeonly| is applied to the child file
% and |\jobname| is set to the main file
% (for proper handling of |.aux| files):
%    \begin{macrocode}
\newcommand{\childdocmain}[1]
{
  \childdocdisable\childdocmain{}
  \if?#1?\else
    \begingroup
      \def\childdoctmp{#1}
      \ifx\childdoctmp\childdocname
        \def\childdoctmp{}
      \else
        \def\childdoctmp
        {
          \childdoctrue
          \includeonly{\childdocname}
          \def\childdocjob{#1}
          \def\jobname{#1}
        }
      \fi
      \expandafter
    \endgroup
    \childdoctmp
  \fi
}
%    \end{macrocode}

% \macro{\childdocof}
% The command |\childdocof| redirects
% compilation to the main file |#1|.
%    \begin{macrocode}
\newcommand{\childdocof}[1]
{
  \childdocdisable
  \childdoctrue
  \includeonly{\childdocname}
  \def\jobname{#1}
  \def\childdocjob{#1}
  \input{#1}
}
%    \end{macrocode}

% \macro{\childdocby}
% The command |\childdocby| ....
%    \begin{macrocode}
\newcommand{\childdocby}[2][]
{
  \childdocdisable
  \childdoctrue
  \childdocmanualtrue
  \if?#1?\else
    \def\jobname{#2}
  \fi
  \def\childdocjob{#2}
  \input{#2}
  \endinput
}
%    \end{macrocode}

% \macro{\childdocforward}
% The command |\childdocforward| redirects
% compilation to the main file or
% (if the optional argument is given) a child file.
% Parameters are set as if the main file
% or a child file starting with |\childdocof| was compiled.
% Then compilation is handed over to the main file:
%    \begin{macrocode}
\newcommand{\childdocforward}[2][]
{
  \begingroup
    \if?#1?
      \def\childdoctmp
      {
        \def\childdocname{#2}
        \def\childdocjob{#2}
        \def\jobname{#2}
        \input{#2}
        \endinput
      }
    \else
      \def\childdoctmp
      {
        \childdocdisable
        \def\childdocname{#2}
        \childdoctrue
        \includeonly{#2}
        \def\childdocjob{#1}
        \def\jobname{#1}
        \input{#1}
        \endinput
      }
    \fi
    \expandafter
  \endgroup
  \childdoctmp
}
%    \end{macrocode}

% \macro{\childdocforwardprefix}
% The command |\childdocforwardprefix| redirects
% compilation to the main or a child file by means of a pattern.
% The prefix |#1| in the current filename is replaced by |#2|
% and the suffix of the current filename is kept
% (it is assumed that the filename does not contain the substring `|~~~|'
% which is used as a delimiter).
% Compilation is handed over to the new file by |\childdocforward|:
%    \begin{macrocode}
\newcommand{\childdocforwardprefix}[3][]
{
  \begingroup
    \def\childdocextract #2##1~~~{\def\childdoctmp{\childdocforward[#1]{#3##1}}}
    \expandafter\childdocextract\childdocname~~~
    \expandafter
  \endgroup
  \childdoctmp
}
%    \end{macrocode}

% \macro{\childdoc}
% The deprecated macro |\childdoc| is a legacy version of |\childdocmain|:
%    \begin{macrocode}
\newcommand{\childdoc}{\childdocmain}
%    \end{macrocode}

% \macro{\childdocredirect}
% The deprecated macro |\childdocredirect| is a legacy version
% of |\childdocforward| and |\childdocforwardprefix|:
%    \begin{macrocode}
\newcommand{\childdocredirect}[2][]
{
  \begingroup
    \if?#1?
      \def\childdoctmp{\childdocforward{#2}}
    \else
      \def\childdoctmp{\childdocforwardprefix{#1}{#2}}
    \fi
    \expandafter
  \endgroup
  \childdoctmp
}
%    \end{macrocode}

%\iffalse
%</package>
%\fi
%
\endinput
|\\
|\childdocof{|\textit{main}|}|\\
\end{tabular}
\end{center}
at the top of every child file \textit{child}
which is included by |\include{|\textit{child}|}|
from within the main file
(or at least for those files to be compiled individually).
The argument \textit{main} must be the filename of the main file.

There are a couple of
considerations in setting up the main and child documents:

%%%%%%%%%%%%%%%%%%%%%%%%%%%%%%%%%%%%%%%%
\paragraph{Restrictions.}

Please note the following restrictions:
\begin{itemize}
\item
|\childdocmain| must be called with one argument \textit{main}
to ensure compatibility with earlier version of the package.
It must either be empty (|\childdocmain{}|)
or precisely match the filename of the main file in which it is specified.
See \secref{sec:detection} for further information.
\item
The filename \textit{main} must be specified without the |.tex| extension.
\item
The filename \textit{main} is case sensitive
(even in case-insensitive file systems)
due to internal string comparison.
\item
The argument \textit{main} should be fully expanded, it cannot be a macro.
\item
Subdirectories and special characters should be avoided in filenames.
\item
The command |\childdocmain{|\textit{main}|}| must be followed by a whitespace.
It should not be followed immediately by another command
or by a comment mark `|%|'.
This is because the \TeX{} parser reads the token immediately following
the argument of |\childdocmain| and puts it
at the beginning of every child section;
however, a white\-space is ignored.
\end{itemize}

%%%%%%%%%%%%%%%%%%%%%%%%%%%%%%%%%%%%%%%%
\paragraph{Content of Main File.}

It is advisable to place all content in the child files included by |\include|.
Any output contained in the main file will appear in all child documents
unless suppressed manually;
it cannot be suppressed automatically by the |\includeonly| directive
and thus should normally be avoided.
A method to include some content in the main file
by means of conditional processing is described in \secref{sec:conditional}.

%%%%%%%%%%%%%%%%%%%%%%%%%%%%%%%%%%%%%%%%
\paragraph{Page Numbering.}

When only a part of the document is compiled,
the appropriate numbering of pages
(as well as other status parameters)
is determined from the |.aux| files.
The latter contain information from previous passes.
However this information needs to propagate through
all intermediate child documents.
Therefore the page numbering in child documents may well
be inconsistent until the complete document is compiled at least once.

A useful (if unconventional) way to always ensure a consistent
page numbering is to restart the numbering in each child document
and denote the pages by `\textit{child}|.|\textit{page}'
where \textit{child} represents the chapter/section number of the child file.
This can be achieved by the command
|\numberwithin{page}{|\textit{child}|}|
of the \textsf{amsmath} package
where \textit{child} can be |chapter| or |section|
depending on the chosen structuring.
Alternatively, one can modify the macro |\thepage| appropriately
and reset the counter |page| at the start of each child file.

%%%%%%%%%%%%%%%%%%%%%%%%%%%%%%%%%%%%%%%%%%%%%%%%%%%%%%%%%%%%%%%%%%%%%%%%%%%%%%%%
\subsection{Conditional Processing}
\label{sec:conditional}

The package provides a mechanism to compile different versions
of a document. To customise the versions further some conditional processing
can come in handy to distinguish which version is being compiled.
The package provides two macros to describe the compilation context:

%%%%%%%%%%%%%%%%%%%%%%%%%%%%%%%%%%%%%%%%
\DescribeMacro{\ifchilddoc}
The conditional |\ifchilddoc| distinguishes between the compilation of
child documents and the main document:
%
\begin{center}
|\ifchilddoc |\textit{child-code}| |[|\||else |\textit{main-code}]| \||fi|
\end{center}

%%%%%%%%%%%%%%%%%%%%%%%%%%%%%%%%%%%%%%%%
\DescribeMacro{\childdocname}
\DescribeMacro{\childdocjob}
The macro |\childdocname| contains the filename (without extension)
of the main or child file being processed.
Note that |\childdocjob| will always contain the name of the main file.

%%%%%%%%%%%%%%%%%%%%%%%%%%%%%%%%%%%%%%%%
\paragraph{Title Page.}

Conditional processing can be used to include a title or banner page
in the main document when proper precautions are taken.
Importantly, the code in the main file should ensure that the page counter
(as well as other status parameters which are stored in the |.aux| files)
takes the same value after the conditional processing.
Otherwise the page numbers may take divergent values
depending on which part is compiled.

For example, a title page could be declared by:
%
\begin{center}
\begin{tabular}{l}
|\ifchilddoc\||else|\\
|\addtocounter{page}{-1}|\\
\textit{code for title page}\\
|\newpage|\\
|\||fi|
\end{tabular}
\end{center}
%
A banner page for the child documents can be generated by:
%
\begin{center}
\begin{tabular}{l}
|\ifchilddoc|\\
|\addtocounter{page}{-1}|\\
\textit{code for banner page}\\
|\newpage|\\
|\||fi|
\end{tabular}
\end{center}
%
Here one could write a message such as:
\begin{center}
|This is the part \childdocname{} of \childdocjob{}.|
\end{center}

%%%%%%%%%%%%%%%%%%%%%%%%%%%%%%%%%%%%%%%%%%%%%%%%%%%%%%%%%%%%%%%%%%%%%%%%%%%%%%%%
\subsection{Flags}
\label{sec:flags}

The package makes it easy to generate different versions
of the main or child documents.
To this end compilation flags can be defined
and assigned different default values.
They will be particularly useful in conjunction
with the forwarding mechanism described in \secref{sec:forward}.

For example, it may be useful to have a flag |\version|
which can be set to |draft| or |final|.
The document source will contain some conditional code
depending on the value of |\version|.
Suppose further, the flag should default to |final| for the main file
and to |draft| for child files
which is a natural assignment for editing the document.
This is achieved by placing the following code
in the preamble of the main document
(below the |\childdocmain| directive):
%
\begin{center}
\begin{tabular}{l}
|\ifchilddoc|\\
|\providecommand{\version}{draft}|\\
|\||else|\\
|\providecommand{\version}{final}|\\
|\||fi|
\end{tabular}
\end{center}
%
The definition by |\providecommand| makes sure
that previous definitions are not overwritten.
Further statements |\providecommand{\version}{...}|
can thus be added before the above code to override it.

For the main file, one might add a line
(between |\childdocmain| and the above block)
%
\begin{center}
|%\ifchilddoc\||else\providecommand{\version}{draft}\||fi|
\end{center}
%
which can be uncommented to produce a draft version.
Likewise one can add a line to the very top of a child file
(above the |\childdocof{|\textit{main}|}| directive)
%
\begin{center}
|%\providecommand{\version}{final}|
\end{center}
%
which can be uncommented to produce the final version of this child document.

%%%%%%%%%%%%%%%%%%%%%%%%%%%%%%%%%%%%%%%%%%%%%%%%%%%%%%%%%%%%%%%%%%%%%%%%%%%%%%%%
\subsection{Forwarding}
\label{sec:forward}

Different versions of the main or child documents
using compilation flags as described in \secref{sec:flags}
can be (permanently) stored in different files
for convenient compilation, viewing and distribution.
To this end, the package defines a command
to pass on compilation to a different file:

%%%%%%%%%%%%%%%%%%%%%%%%%%%%%%%%%%%%%%%%
\DescribeMacro{\childdocforward}
The command |\childdocforward| redirects processing to
another source file:
%
\begin{center}
\begin{tabular}{l}
|% \iffalse
%
% childdoc.dtx Copyright (C) 2017-2018 Niklas Beisert
%
% This work may be distributed and/or modified under the
% conditions of the LaTeX Project Public License, either version 1.3
% of this license or (at your option) any later version.
% The latest version of this license is in
%   http://www.latex-project.org/lppl.txt
% and version 1.3 or later is part of all distributions of LaTeX
% version 2005/12/01 or later.
%
% This work has the LPPL maintenance status `maintained'.
%
% The Current Maintainer of this work is Niklas Beisert.
%
% This work consists of the files childdoc.dtx and childdoc.ins
% and the derived files childdoc.def and cdocsamp.tex with
% cdocsch1.tex, cdocsch2.tex, cdocsdrf.tex, cdocsfn1.tex, cdocsfn2.tex.
%
%<package>\ifdefined\childdocmain\endinput\fi
%<package>\ProvidesFile{childdoc.def}[2018/12/30 v2.0 child document driver]
%<samplemain>\ProvidesFile{cdocsamp.tex}[2018/12/30 v2.0 sample for childdoc]
%<*driver>
%\ProvidesFile{childdoc.drv}[2018/12/30 v2.0 childdoc reference manual file]
\PassOptionsToClass{10pt,a4paper}{article}
\documentclass{ltxdoc}

\usepackage[margin=35mm]{geometry}
\usepackage{hyperref}
\usepackage{hyperxmp}
\usepackage[usenames]{color}

\hypersetup{colorlinks=true}
\hypersetup{pdfstartview=FitH}
\hypersetup{pdfpagemode=UseNone}
\hypersetup{pdfsource={}}
\hypersetup{pdflang={en-UK}}
\hypersetup{pdfcopyright={Copyright 2017-2018 Niklas Beisert.
  This work may be distributed and/or modified under the
  conditions of the LaTeX Project Public License, either version 1.3
  of this license or (at your option) any later version.}}
\hypersetup{pdflicenseurl={http://www.latex-project.org/lppl.txt}}
\hypersetup{pdfcontactaddress={ETH Zurich, ITP, HIT K,
  Wolfgang-Pauli-Strasse 27}}
\hypersetup{pdfcontactpostcode={8093}}
\hypersetup{pdfcontactcity={Zurich}}
\hypersetup{pdfcontactcountry={Switzerland}}
\hypersetup{pdfcontactemail={nbeisert@itp.phys.ethz.ch}}
\hypersetup{pdfcontacturl={http://people.phys.ethz.ch/\xmptilde nbeisert/}}

\newcommand{\secref}[1]{\hyperref[#1]{section \ref*{#1}}}

\parskip1ex
\parindent0pt
\let\olditemize\itemize
\def\itemize{\olditemize\parskip0pt}

\begin{document}

\title{The \textsf{childdoc} Package}
\hypersetup{pdftitle={The childdoc Package}}
\author{Niklas Beisert\\[2ex]
  Institut f\"ur Theoretische Physik\\
  Eidgen\"ossische Technische Hochschule Z\"urich\\
  Wolfgang-Pauli-Strasse 27, 8093 Z\"urich, Switzerland\\[1ex]
  \href{mailto:nbeisert@itp.phys.ethz.ch}
  {\texttt{nbeisert@itp.phys.ethz.ch}}}
\hypersetup{pdfauthor={Niklas Beisert}}
\hypersetup{pdfsubject={Manual for the LaTeX2e Package childdoc}}
\date{30 December 2018, \textsf{v2.0}}
\maketitle

\begin{abstract}\noindent
\textsf{childdoc} is a \LaTeXe{} package
that enables the direct compilation
of document sections included by |\include|
to individual files.
\end{abstract}

\begingroup
\parskip0ex
\tableofcontents
\endgroup

%%%%%%%%%%%%%%%%%%%%%%%%%%%%%%%%%%%%%%%%%%%%%%%%%%%%%%%%%%%%%%%%%%%%%%%%%%%%%%%%
%%%%%%%%%%%%%%%%%%%%%%%%%%%%%%%%%%%%%%%%%%%%%%%%%%%%%%%%%%%%%%%%%%%%%%%%%%%%%%%%
\section{Introduction}

\LaTeX{} provides a mechanism to structure a large document (such as a book)
into a main file and several child files (containing the chapters)
using the |\include| command.
This mechanism is beneficial for documents
which span hundreds of pages in order to
make the source file(s) more manageable.
Moreover, compilation can be restricted to
selected child files by means of the |\includeonly| command.
The latter feature can be used to reduce the compilation time while editing
(this was significantly more useful in the earlier days of \LaTeX{})
or to generate a smaller document which is easier to navigate.
Another application of |\includeonly| is to generate
documents consisting of selected parts of the complete document.

However, there are a few drawbacks of the plain |\include| mechanism:
\begin{itemize}
\item
The child files cannot be compiled on their own,
they can only be compiled via the main file.
A naive editing environment
(such as a text editor with an option
to have the current file processed by \LaTeX)
may require one to switch to the main file before compiling;
attempting to compile the child file produces errors.
\item
The main file must be modified (each time)
to adjust the |\includeonly| command
to the present needs. This easily leaves the main file in a messy state.
\item
The generated document will always carry the filename
of the main document. This is inconvenient if
several child files are to be compiled and
to be kept for distribution.
\end{itemize}

The present package provides a simple interface
to make child files individually compilable by \LaTeX{}.
Compiling a child file then has the same effect as compiling
the main file with an |\includeonly| command
to select the appropriate child.
Moreover the generated document will carry the name of the child
rather than the main file.
This resolves all three above issues.

This feature is meant to make the editing of books,
thesis documents and lecture notes somewhat more convenient.
However, the package can also be used efficiently for
composing a series of documents (such as exercise sheets)
which are typically distributed individually.
It then assists the author in generating the individual documents
(potentially in different versions)
as well as a document containing the collected series.
Another application is in developing style files
or other kinds of included material
where compilation of the style file could redirect
to a sample or test file.

%%%%%%%%%%%%%%%%%%%%%%%%%%%%%%%%%%%%%%%%%%%%%%%%%%%%%%%%%%%%%%%%%%%%%%%%%%%%%%%%
%%%%%%%%%%%%%%%%%%%%%%%%%%%%%%%%%%%%%%%%%%%%%%%%%%%%%%%%%%%%%%%%%%%%%%%%%%%%%%%%
\section{Usage}

First of all, the package \textsf{childdoc} is \emph{not} a standard
\LaTeXe{} |.sty| style file! Therefore it needs to be invoked in
a non-standard way.

%%%%%%%%%%%%%%%%%%%%%%%%%%%%%%%%%%%%%%%%%%%%%%%%%%%%%%%%%%%%%%%%%%%%%%%%%%%%%%%%
\subsection{Included Files}
\label{sec:include}

%%%%%%%%%%%%%%%%%%%%%%%%%%%%%%%%%%%%%%%%
\DescribeMacro{\childdocmain}
To use the package, add the commands
\begin{center}
\begin{tabular}{l}
|\input{childdoc.def}|\\
|\childdocmain{}|\\
\end{tabular}
\end{center}
at the very top of the main \LaTeX{} file,
in particular \emph{before} the |\documentclass| statement!
The argument of |\childdocmain| should be left empty
(but it must be present).

%%%%%%%%%%%%%%%%%%%%%%%%%%%%%%%%%%%%%%%%
\DescribeMacro{\childdocof}
Furthermore, add the commands
\begin{center}
\begin{tabular}{l}
|\input{childdoc.def}|\\
|\childdocof{|\textit{main}|}|\\
\end{tabular}
\end{center}
at the top of every child file \textit{child}
which is included by |\include{|\textit{child}|}|
from within the main file
(or at least for those files to be compiled individually).
The argument \textit{main} must be the filename of the main file.

There are a couple of
considerations in setting up the main and child documents:

%%%%%%%%%%%%%%%%%%%%%%%%%%%%%%%%%%%%%%%%
\paragraph{Restrictions.}

Please note the following restrictions:
\begin{itemize}
\item
|\childdocmain| must be called with one argument \textit{main}
to ensure compatibility with earlier version of the package.
It must either be empty (|\childdocmain{}|)
or precisely match the filename of the main file in which it is specified.
See \secref{sec:detection} for further information.
\item
The filename \textit{main} must be specified without the |.tex| extension.
\item
The filename \textit{main} is case sensitive
(even in case-insensitive file systems)
due to internal string comparison.
\item
The argument \textit{main} should be fully expanded, it cannot be a macro.
\item
Subdirectories and special characters should be avoided in filenames.
\item
The command |\childdocmain{|\textit{main}|}| must be followed by a whitespace.
It should not be followed immediately by another command
or by a comment mark `|%|'.
This is because the \TeX{} parser reads the token immediately following
the argument of |\childdocmain| and puts it
at the beginning of every child section;
however, a white\-space is ignored.
\end{itemize}

%%%%%%%%%%%%%%%%%%%%%%%%%%%%%%%%%%%%%%%%
\paragraph{Content of Main File.}

It is advisable to place all content in the child files included by |\include|.
Any output contained in the main file will appear in all child documents
unless suppressed manually;
it cannot be suppressed automatically by the |\includeonly| directive
and thus should normally be avoided.
A method to include some content in the main file
by means of conditional processing is described in \secref{sec:conditional}.

%%%%%%%%%%%%%%%%%%%%%%%%%%%%%%%%%%%%%%%%
\paragraph{Page Numbering.}

When only a part of the document is compiled,
the appropriate numbering of pages
(as well as other status parameters)
is determined from the |.aux| files.
The latter contain information from previous passes.
However this information needs to propagate through
all intermediate child documents.
Therefore the page numbering in child documents may well
be inconsistent until the complete document is compiled at least once.

A useful (if unconventional) way to always ensure a consistent
page numbering is to restart the numbering in each child document
and denote the pages by `\textit{child}|.|\textit{page}'
where \textit{child} represents the chapter/section number of the child file.
This can be achieved by the command
|\numberwithin{page}{|\textit{child}|}|
of the \textsf{amsmath} package
where \textit{child} can be |chapter| or |section|
depending on the chosen structuring.
Alternatively, one can modify the macro |\thepage| appropriately
and reset the counter |page| at the start of each child file.

%%%%%%%%%%%%%%%%%%%%%%%%%%%%%%%%%%%%%%%%%%%%%%%%%%%%%%%%%%%%%%%%%%%%%%%%%%%%%%%%
\subsection{Conditional Processing}
\label{sec:conditional}

The package provides a mechanism to compile different versions
of a document. To customise the versions further some conditional processing
can come in handy to distinguish which version is being compiled.
The package provides two macros to describe the compilation context:

%%%%%%%%%%%%%%%%%%%%%%%%%%%%%%%%%%%%%%%%
\DescribeMacro{\ifchilddoc}
The conditional |\ifchilddoc| distinguishes between the compilation of
child documents and the main document:
%
\begin{center}
|\ifchilddoc |\textit{child-code}| |[|\||else |\textit{main-code}]| \||fi|
\end{center}

%%%%%%%%%%%%%%%%%%%%%%%%%%%%%%%%%%%%%%%%
\DescribeMacro{\childdocname}
\DescribeMacro{\childdocjob}
The macro |\childdocname| contains the filename (without extension)
of the main or child file being processed.
Note that |\childdocjob| will always contain the name of the main file.

%%%%%%%%%%%%%%%%%%%%%%%%%%%%%%%%%%%%%%%%
\paragraph{Title Page.}

Conditional processing can be used to include a title or banner page
in the main document when proper precautions are taken.
Importantly, the code in the main file should ensure that the page counter
(as well as other status parameters which are stored in the |.aux| files)
takes the same value after the conditional processing.
Otherwise the page numbers may take divergent values
depending on which part is compiled.

For example, a title page could be declared by:
%
\begin{center}
\begin{tabular}{l}
|\ifchilddoc\||else|\\
|\addtocounter{page}{-1}|\\
\textit{code for title page}\\
|\newpage|\\
|\||fi|
\end{tabular}
\end{center}
%
A banner page for the child documents can be generated by:
%
\begin{center}
\begin{tabular}{l}
|\ifchilddoc|\\
|\addtocounter{page}{-1}|\\
\textit{code for banner page}\\
|\newpage|\\
|\||fi|
\end{tabular}
\end{center}
%
Here one could write a message such as:
\begin{center}
|This is the part \childdocname{} of \childdocjob{}.|
\end{center}

%%%%%%%%%%%%%%%%%%%%%%%%%%%%%%%%%%%%%%%%%%%%%%%%%%%%%%%%%%%%%%%%%%%%%%%%%%%%%%%%
\subsection{Flags}
\label{sec:flags}

The package makes it easy to generate different versions
of the main or child documents.
To this end compilation flags can be defined
and assigned different default values.
They will be particularly useful in conjunction
with the forwarding mechanism described in \secref{sec:forward}.

For example, it may be useful to have a flag |\version|
which can be set to |draft| or |final|.
The document source will contain some conditional code
depending on the value of |\version|.
Suppose further, the flag should default to |final| for the main file
and to |draft| for child files
which is a natural assignment for editing the document.
This is achieved by placing the following code
in the preamble of the main document
(below the |\childdocmain| directive):
%
\begin{center}
\begin{tabular}{l}
|\ifchilddoc|\\
|\providecommand{\version}{draft}|\\
|\||else|\\
|\providecommand{\version}{final}|\\
|\||fi|
\end{tabular}
\end{center}
%
The definition by |\providecommand| makes sure
that previous definitions are not overwritten.
Further statements |\providecommand{\version}{...}|
can thus be added before the above code to override it.

For the main file, one might add a line
(between |\childdocmain| and the above block)
%
\begin{center}
|%\ifchilddoc\||else\providecommand{\version}{draft}\||fi|
\end{center}
%
which can be uncommented to produce a draft version.
Likewise one can add a line to the very top of a child file
(above the |\childdocof{|\textit{main}|}| directive)
%
\begin{center}
|%\providecommand{\version}{final}|
\end{center}
%
which can be uncommented to produce the final version of this child document.

%%%%%%%%%%%%%%%%%%%%%%%%%%%%%%%%%%%%%%%%%%%%%%%%%%%%%%%%%%%%%%%%%%%%%%%%%%%%%%%%
\subsection{Forwarding}
\label{sec:forward}

Different versions of the main or child documents
using compilation flags as described in \secref{sec:flags}
can be (permanently) stored in different files
for convenient compilation, viewing and distribution.
To this end, the package defines a command
to pass on compilation to a different file:

%%%%%%%%%%%%%%%%%%%%%%%%%%%%%%%%%%%%%%%%
\DescribeMacro{\childdocforward}
The command |\childdocforward| redirects processing to
another source file:
%
\begin{center}
\begin{tabular}{l}
|\input{childdoc.def}|\\
|\childdocforward[|\textit{main}|]{|\textit{dest}|}|\\
\end{tabular}
\end{center}
%
The argument \textit{dest} is the destination file
(without extension).
It should be the main file or one of the child files.
Note that further \textsf{childdoc} directives
such as |\childdocof| and |\childdocforward|
in the indicated file will be processed in this form.
The optional argument \textit{main}
passes on directly to the main file \textit{main}
while pretending to compile the child \textit{dest}.
This form behaves as if \textit{dest}
issues |\childdocof{|\textit{main}|}| right away,
and no further \textsf{childdoc} directives will be processed.

%%%%%%%%%%%%%%%%%%%%%%%%%%%%%%%%%%%%%%%%
\DescribeMacro{\...prefix}
In the alternative form |\childdocforwardprefix|,
%
\begin{center}
\begin{tabular}{l}
|\input{childdoc.def}|\\
|\childdocforwardprefix[|\textit{main}|]{|\textit{prefix}|}{|\textit{dest}|}|
\end{tabular}
\end{center}
%
the destination file is determined by a pattern
depending on the current file:
To make this work, the current file must be called
`{\textit{prefix}\hspace{0.2em}\textit{suffix}}'
with \textit{prefix} matching precisely the argument.
Processing is then passed on to the file
`{\textit{dest}\hspace{0.2em}\textit{suffix}}'.
Surely, the same effect is achieved by
directly specifying the
argument `{\textit{dest}\hspace{0.2em}\textit{suffix}}'
in the first form.
However, that requires to set up a different file
for each child. With the alternative form of the command
all these files can have exactly the same content
which simplifies setting them up and maintaining them.

For example, the following file |draft.tex|
with a compilation flag |\version| as described in \secref{sec:flags}
compiles the main document as a draft:
%
\begin{center}
\begin{tabular}{l}
|\def\version{draft}|\\
|\input{childdoc.def}|\\
|\childdocforward{|\textit{main}|}|
\end{tabular}
\end{center}
%
Likewise, the following files |final|\textit{nn}|.tex|
compile the final version of the child document
|child|\textit{nn}|.tex|:
%
\begin{center}
\begin{tabular}{l}
|\def\version{final}|\\
|\input{childdoc.def}|\\
|\childdocforwardprefix{final}{child}|
\end{tabular}
\end{center}
%

Note that when several versions of a main file and/or of each child file
are to be generated, it may be convenient to set up a |Makefile| or
shell script to automatise the process.

%%%%%%%%%%%%%%%%%%%%%%%%%%%%%%%%%%%%%%%%%%%%%%%%%%%%%%%%%%%%%%%%%%%%%%%%%%%%%%%%
\subsection{Command Line Processing}
\label{sec:commandline}

The effect of redirection files can also be achieved by invoking
the \LaTeX{} compiler with a more elaborate command line.
Most conveniently this should be done as part
of a shell script or a |Makefile|.

When using \textsf{childdoc} in the main file, the following
command lines effectively perform a redirection
(note that depending on the shell being used,
backslashes may have to be doubled: `|\|' $\to$ `|\\|'):
%
\begin{center}
|... -jobname "|\textit{target}|" |\\|"|[\textit{flags}]%
|\input{childdoc.def}\childdocforward[|\textit{main}|]{|\textit{dest}|}"|
\end{center}
%
Here \textit{target} is the name of the output file,
\textit{main} is the name of the main file
and \textit{dest} is the name of the main or child file to be processed
(all filenames without extensions).
The optional argument \textit{main} can be omitted
if \textit{main} matches \textit{dest}.
Optionally, compilation \textit{flags} can be defined via |\def| commands.
This command line makes the \TeX{} engine believe
it is compiling the file \textit{target}
whose content is specified as the latter parameter.
The provided code then forwards the processing to
\textit{main} or \textit{dest} as described in \secref{sec:forward}.

%%%%%%%%%%%%%%%%%%%%%%%%%%%%%%%%%%%%%%%%%%%%%%%%%%%%%%%%%%%%%%%%%%%%%%%%%%%%%%%%
\subsection{Include by Input}
\label{sec:input}

Including child documents by |\include| has some restrictions by design.
Most notably, the content of a child document always occupies
its own set of pages; pages cannot be shared between child documents.
Usually, this behaviour makes perfect sense
because each child document contain an essential part of the document.
However, in some situations it may be desirable to compose
a document from a collection of parts
without having mandatory page breaks between then.
For this case, the package
provides a mechanism to include parts
by |\input| which can also be processed individually.
However, by construction this mechanism
requires manual handling of the content to be output.

%%%%%%%%%%%%%%%%%%%%%%%%%%%%%%%%%%%%%%%%
\DescribeMacro{\ifchilddocmanual}
The main file should be prepared as usual, see \secref{sec:include}.
However, the document body must make a distinction
between processing of an individual part and of the main document, e.g.:
%
\begin{center}
\begin{tabular}{l}
|\ifchilddocmanual|\\
|\input{\childdocname}|\\
|\||else|\\
\textit{document body with }|\input{|\textit{part}|}|\\
|\||fi|
\end{tabular}
\end{center}
%
The conditional |\ifchilddocmanual| is true whenever
a part to be included by |\input| is being compiled,
and the name of the part is stored in |\childdocname|.

%%%%%%%%%%%%%%%%%%%%%%%%%%%%%%%%%%%%%%%%
\DescribeMacro{\childdocby}
Each part to be included by |\input| should start with:
%
\begin{center}
\begin{tabular}{l}
|\input{childdoc.def}|\\
|\childdocby{|\textit{main}|}|\\
\end{tabular}
\end{center}
%
The directive |\childdocby| is similar to |\childdocof|
described in \secref{sec:include},
but the subsequent selection of content must be done manually.
To that end, both |\ifchilddoc| and |\ifchilddocmanual|
will be true upon processing of a part,
and the name of the part is stored in |\childdocname|.
Note that |\jobname| will be set to the filename of the current part
so that each part receives an individual |.aux| file
that does not interfere with the |.aux| file(s) of the main document.
This behaviour can be altered by the alternative form
|\childdocby[*]{|\textit{main}|}| (with a non-empty optional argument)
which uses the |.aux| file of the main document
by setting |\jobname| to \textit{main}.

%%%%%%%%%%%%%%%%%%%%%%%%%%%%%%%%%%%%%%%%%%%%%%%%%%%%%%%%%%%%%%%%%%%%%%%%%%%%%%%%
\subsection{Driver Development}
\label{sec:driver}

The \textsf{childdoc} mechanism can also be use for the development
of definition files such as \LaTeX{} styles or classes.
This case differs from the above setup with multiple parts
included by |\include| in that no |\includeonly| should be invoked.
This can be achieved by starting the include file
(before |\ProvidesPackage|) with:
%
\begin{center}
\begin{tabular}{l}
|\input{childdoc.def}|\\
|\childdocforward{|\textit{main}|}|\\
\end{tabular}
\end{center}
%
or alternatively with:
%
\begin{center}
\begin{tabular}{l}
|\input{childdoc.def}|\\
|\childdocby{|\textit{main}|}|\\
\end{tabular}
\end{center}
%
Both forms have slightly different effects as described above.
The main file is prepared as usual, see \secref{sec:include}.

%%%%%%%%%%%%%%%%%%%%%%%%%%%%%%%%%%%%%%%%%%%%%%%%%%%%%%%%%%%%%%%%%%%%%%%%%%%%%%%%
\subsection{Legacy Detection}
\label{sec:detection}

The directive |\childdocmain| in the main file can detect
whether the complete document or merely a child is to be compiled
even without using the directive |\childdocof|.
This method is deprecated because it is less robust
and there is no compelling reason to use it;
it is merely provided for backward compatibility
and it may be removed in future versions.

If the detection mechanism is to be used,
it is mandatory to correctly specify
the filename of the main file as the argument of |\childdocmain|:
%
\begin{center}
\begin{tabular}{l}
|\input{childdoc.def}|\\
|\childdocmain{|\textit{main}|}|\\
\end{tabular}
\end{center}
%
If |\jobname| does not match the argument \textit{main} of |\childdocmain|,
it is assumed that |\jobname| points to the child file to be compiled.
When using |\childdocmain| with the main file specified as argument,
it suffices to start a child file
with just |\input{|\textit{main}|}|
without loading of the package and using |\childdocof|.
If instead all processing is done
with the appropriate \textsf{childdoc} directives,
the argument of \textit{main} of |\childdocmain| can be empty.

An alternative version of the command line processing described
in \secref{sec:commandline} using the detection mechanism reads:
%
\begin{center}
|... -jobname "|\textit{target}|" "|[\textit{flags}]%
[|\def\jobname{|\textit{dest}|}|]|\input{|\textit{main}|}"|
\end{center}

%%%%%%%%%%%%%%%%%%%%%%%%%%%%%%%%%%%%%%%%%%%%%%%%%%%%%%%%%%%%%%%%%%%%%%%%%%%%%%%%
\subsection{Manual Code}
\label{sec:manual}

In case one cannot be certain whether the definitions file |childdoc.def|
is installed on the target \TeX{} distribution
and one prefers not to ship it,
it is conceivable to paste a few relevant commands into the sources.

To that end, drop all statements |\input{childdoc.def}|
and perform the replacements as outlined below.
Instead of |\childdocmain{|\textit{main}|}| add the following code
to the top of the main file:
%
\begin{center}
\begin{tabular}{l}
|\||ifdefined\childdocname\endinput\||fi\newif\ifchilddoc|\\
|\edef\childdocname{\scantokens\expandafter{\jobname\noexpand}}|\\
|\def\childdocmain{|\textit{main}|}\||ifx\childdocmain\childdocname\||else|\\
|\childdoctrue\includeonly{\childdocname}\let\jobname\childdocmain\||fi|\\
\end{tabular}
\end{center}
%
Instead of |\childdocof{|\textit{main}|}| just include the main file
at the top of each child file:
%
\begin{center}
|\input{|\textit{main}|}|
\end{center}
%
A simple redirection |\childdocforward{|\textit{dest}|}| is achieved by:
%
\begin{center}
|\def\jobname{|\textit{dest}|}\input{\jobname}|
\end{center}
%
The redirection with prefix
|\childdocforwardprefix[|\textit{prefix}|]{|\textit{dest}|}|
is accomplished by:
%
\begin{center}
\begin{tabular}{l}
|{\edef\jobname{\scantokens\expandafter{\jobname\noexpand}}|\\
|\def\redirectjob |\textit{prefix}|#1~~~{\gdef\jobname{|\textit{dest}|#1}}|\\
|\expandafter\redirectjob\jobname~~~}\input{\jobname}|
\end{tabular}
\end{center}

In an alternative approach,
child documents can be compiled by a specific command line
without additional code or specific definitions:
%
\begin{center}
|... -jobname "|\textit{target}|" "|[\textit{flags}]%
|\includeonly{|\textit{dest}|}\input{|\textit{main}|}"|
\end{center}
%

%%%%%%%%%%%%%%%%%%%%%%%%%%%%%%%%%%%%%%%%%%%%%%%%%%%%%%%%%%%%%%%%%%%%%%%%%%%%%%%%
%%%%%%%%%%%%%%%%%%%%%%%%%%%%%%%%%%%%%%%%%%%%%%%%%%%%%%%%%%%%%%%%%%%%%%%%%%%%%%%%
\section{Information}

%%%%%%%%%%%%%%%%%%%%%%%%%%%%%%%%%%%%%%%%%%%%%%%%%%%%%%%%%%%%%%%%%%%%%%%%%%%%%%%%
\subsection{Copyright}

Copyright \copyright{} 2017--2018 Niklas Beisert

This work may be distributed and/or modified under the
conditions of the \LaTeX{} Project Public License, either version 1.3
of this license or (at your option) any later version.
The latest version of this license is in
  \url{http://www.latex-project.org/lppl.txt}
and version 1.3 or later is part of all distributions of \LaTeX{}
version 2005/12/01 or later.

This work has the LPPL maintenance status `maintained'.

The Current Maintainer of this work is Niklas Beisert.

This work consists of the files |README.txt|, |childdoc.ins| and |childdoc.dtx|
as well as the derived files |childdoc.def|, |cdocsamp.tex|
with |cdocsch1.tex|, |cdocsch2.tex|, |cdocspt3.tex|, |cdocspt4.tex|,
|cdocsdrf.tex|, |cdocsfn1.tex|, |cdocsfn2.tex|
as well as |childdoc.pdf|.

%%%%%%%%%%%%%%%%%%%%%%%%%%%%%%%%%%%%%%%%%%%%%%%%%%%%%%%%%%%%%%%%%%%%%%%%%%%%%%%%
\subsection{Files and Installation}

The package consists of the files:
%
\begin{center}
\begin{tabular}{ll}
    |README.txt|   & readme file \\
    |childdoc.ins| & installation file \\
    |childdoc.dtx| & source file \\
    |childdoc.def| & definition file \\
    |cdocsamp.tex| & sample main file \\
    |cdocsch1.tex| & sample include file \\
    |cdocsch2.tex| & sample include file \\
    |cdocspt3.tex| & sample part file \\
    |cdocspt4.tex| & sample part file \\
    |cdocsdrf.tex| & sample redirection file \\
    |cdocsfn1.tex| & sample redirection file \\
    |cdocsfn2.tex| & sample redirection file \\
    |childdoc.pdf| & manual
\end{tabular}
\end{center}
%
The distribution consists of the files
|README.txt|, |childdoc.ins| and |childdoc.dtx|.
%
\begin{itemize}
\item
Run (pdf)\LaTeX{} on |childdoc.dtx|
to compile the manual |childdoc.pdf| (this file).
\item
Run \LaTeX{} on |childdoc.ins| to create the definitions file |childdoc.def|
and the sample |cdocsamp.tex| with include files
|cdocsch1.tex|, |cdocsch2.tex|, |cdocspt3.tex|, |cdocspt4.tex|,
|cdocsdrf.tex|, |cdocsfn1.tex|, |cdocsfn2.tex|.
Then copy the file |childdoc.def| to an appropriate directory of your \LaTeX{}
distribution, e.g.\ \textit{texmf-root}|/tex/latex/childdoc|.
\end{itemize}

%%%%%%%%%%%%%%%%%%%%%%%%%%%%%%%%%%%%%%%%%%%%%%%%%%%%%%%%%%%%%%%%%%%%%%%%%%%%%%%%
\subsection{Related CTAN Packages}

There are several other packages which offer a similar functionality:
%
\begin{itemize}
\item
The packages
\href{http://ctan.org/pkg/docmute}{\textsf{docmute}},
\href{http://ctan.org/pkg/includex}{\textsf{includex}} and
\href{http://ctan.org/pkg/standalone}{\textsf{standalone}}
provide commands to include only the document body of
a child file thus allowing both files to be compiled individually.
\item
The packages \href{http://ctan.org/pkg/subdocs}{\textsf{subdocs}}
and \href{http://ctan.org/pkg/subfiles}{\textsf{subfiles}}
provide structures in which the main and child documents can be
encapsulated and allowing them to be compiled individually.
The inclusion mechanism is different from the conventional |\include|.
\item
The package \href{http://ctan.org/pkg/combine}{\textsf{combine}}
is an elaborate solution to combine several documents into one.
\end{itemize}
%
See also the CTAN topic \href{http://ctan.org/topic/subdocs}{\textsf{subdocs}}
for further related packages.
The present package differs from the above solutions in that
a document structure constructed with the conventional |\include| mechanism
just needs two extra commands at the top of every file
such that all constituent files can be compiled individually.

%%%%%%%%%%%%%%%%%%%%%%%%%%%%%%%%%%%%%%%%%%%%%%%%%%%%%%%%%%%%%%%%%%%%%%%%%%%%%%%%
%\subsection{Feature Suggestions}
%
%The following is a list of features which may be useful for future
%versions of this package:
%%
%\begin{itemize}
%\item
%\ldots
%\end{itemize}

%%%%%%%%%%%%%%%%%%%%%%%%%%%%%%%%%%%%%%%%%%%%%%%%%%%%%%%%%%%%%%%%%%%%%%%%%%%%%%%%
\subsection{Revision History}

%%%%%%%%%%%%%%%%%%%%%%%%%%%%%%%%%%%%%%%%
\paragraph{v2.0:} 2018/12/30

\begin{itemize}
\item
immediate forward processing
\item
added |\childdocby| mechanism
\item
manual restructured
\end{itemize}

%%%%%%%%%%%%%%%%%%%%%%%%%%%%%%%%%%%%%%%%
\paragraph{v1.6:} 2018/01/17

\begin{itemize}
\item
application for development of include files
\item
corrections to manual
\end{itemize}

%%%%%%%%%%%%%%%%%%%%%%%%%%%%%%%%%%%%%%%%
\paragraph{v1.5:} 2017/05/21

\begin{itemize}
\item
more complete structuring introduced
\item
|\childdocof| introduced
\item
|\childdoc| renamed to |\childdocmain|
\item
|\childredirect| renamed to |\childdocforward| and |\childdocforwardprefix|
and functionality expanded
\end{itemize}

%%%%%%%%%%%%%%%%%%%%%%%%%%%%%%%%%%%%%%%%
\paragraph{v1.0:} 2017/04/27

\begin{itemize}
\item
manual and install package
\item
first version published on CTAN
\end{itemize}

%%%%%%%%%%%%%%%%%%%%%%%%%%%%%%%%%%%%%%%%
\paragraph{v0.6:} 2017/04/26

\begin{itemize}
\item
redirection mechanism added
\end{itemize}

%%%%%%%%%%%%%%%%%%%%%%%%%%%%%%%%%%%%%%%%
\paragraph{v0.5:} 2017/04/26

\begin{itemize}
\item
functionality in definition file
\end{itemize}


%%%%%%%%%%%%%%%%%%%%%%%%%%%%%%%%%%%%%%%%%%%%%%%%%%%%%%%%%%%%%%%%%%%%%%%%%%%%%%%%
%%%%%%%%%%%%%%%%%%%%%%%%%%%%%%%%%%%%%%%%%%%%%%%%%%%%%%%%%%%%%%%%%%%%%%%%%%%%%%%%
%%%%%%%%%%%%%%%%%%%%%%%%%%%%%%%%%%%%%%%%%%%%%%%%%%%%%%%%%%%%%%%%%%%%%%%%%%%%%%%%
\appendix

\settowidth\MacroIndent{\rmfamily\scriptsize 000\ }

 \DocInput{childdoc.dtx}

\end{document}
%</driver>
% \fi
%
% %%%%%%%%%%%%%%%%%%%%%%%%%%%%%%%%%%%%%%%%%%%%%%%%%%%%%%%%%%%%%%%%%%%%%%%%%%%%%%
% %%%%%%%%%%%%%%%%%%%%%%%%%%%%%%%%%%%%%%%%%%%%%%%%%%%%%%%%%%%%%%%%%%%%%%%%%%%%%%
% \section{Sample}
%\iffalse
%<*samplemain>
%\fi
%
% The following presents a sample document
% with two chapters, two parts, a title page,
% a compile flag as well as three forwarding files to set the flag.
% It consists of eight |.tex| files:
% \begin{center}
% \begin{tabular}{ll}
% |cdocsamp.tex|&main file\\
% |cdocsch1.tex|&include file for chapter 1\\
% |cdocsch2.tex|&include file for chapter 2\\
% |cdocspt3.tex|&include file for part 3\\
% |cdocspt4.tex|&include file for part 4\\
% |cdocsdrf.tex|&forwarding file for main file in draft mode\\
% |cdocsfi1.tex|&forwarding file for final version of chapter 1\\
% |cdocsfi2.tex|&forwarding file for final version of chapter 2\\
% \end{tabular}
% \end{center}
% Each of the eight files can be compiled directly by the \LaTeX{} compiler.
%
% %%%%%%%%%%%%%%%%%%%%%%%%%%%%%%%%%%%%%%
% \paragraph{Main File.}
%
% The main file is called |cdocsamp.tex|.
%
% Load the \textsf{childdoc} definitions and
% declare the filename for the main document:
%    \begin{macrocode}
\input{childdoc.def}
\childdocmain{}
%    \end{macrocode}

% Optional override for |\version| flag:
%    \begin{macrocode}
%%\ifchilddoc\else\providecommand{\version}{draft}\fi
%    \end{macrocode}

% Define the default values for the |\version| flag
% (|final| for the main file and |draft| for childs):
%    \begin{macrocode}
\ifchilddoc
\providecommand{\version}{draft}
\else
\providecommand{\version}{final}
\fi
%    \end{macrocode}

% Load the standard document class:
%    \begin{macrocode}
\documentclass[12pt]{article}
%    \end{macrocode}

% Start the document body:
%    \begin{macrocode}
\begin{document}
%    \end{macrocode}

% Declare a title page.
% Print title, part of document being processed and version flag:
%    \begin{macrocode}
\addtocounter{page}{-1}
\begin{center}
{\LARGE\bfseries{}childdoc example\par}
\vspace{1cm}
\ifchilddoc
\ifchilddocmanual part\else chapter\fi:
`\childdocname' of `\childdocjob'\par
\else
main document: `\childdocjob'\par
\fi
version: \version\par
\end{center}
\newpage
%    \end{macrocode}

% Manually include selected file,
% otherwise process as usual:
%    \begin{macrocode}
\ifchilddocmanual
\section*{part `\childdocname'}
\input{\childdocname}
\else
%    \end{macrocode}

% Include the two chapters:
%    \begin{macrocode}
\include{cdocsch1}
\include{cdocsch2}
%    \end{macrocode}

% Include the two parts unless only chapters should be displayed:
%    \begin{macrocode}
\ifchilddoc\else
\section{part three}
\input{cdocspt3}
\section{part four}
\input{cdocspt4}
\fi
%    \end{macrocode}

% Process as usual until here:
%    \begin{macrocode}
\fi
%    \end{macrocode}

% End of document body:
%    \begin{macrocode}
\end{document}
%    \end{macrocode}
%\iffalse
%</samplemain>
%\fi
%
% %%%%%%%%%%%%%%%%%%%%%%%%%%%%%%%%%%%%%%
% \paragraph{Chapter Include Files.}
%
% The include files are called |cdocsch1.tex| and |cdocsch2.tex|.
%
%\iffalse
%<*samplechap1|samplechap2>
%\fi

% Optional override for |\version| flag:
%    \begin{macrocode}
%%\providecommand{\version}{final}
%    \end{macrocode}

% Include the main document:
%    \begin{macrocode}
\input{childdoc.def}
\childdocof{cdocsamp}
%    \end{macrocode}

%\iffalse
%</samplechap1|samplechap2>
%\fi
%
%\iffalse
%<*samplechap1>
%\fi
% Some text for chapter 1:
%    \begin{macrocode}
\section{one}
some text in chapter one
%    \end{macrocode}

%\iffalse
%</samplechap1>
%\fi
% Some text for chapter 2:
%\iffalse
%<*samplechap2>
%\fi
%    \begin{macrocode}
\section{two}
more text in chapter two
%    \end{macrocode}

%\iffalse
%</samplechap2>
%\fi
%
% %%%%%%%%%%%%%%%%%%%%%%%%%%%%%%%%%%%%%%
% \paragraph{Part Include Files.}
%
% The include files are called |cdocspt3.tex| and |cdocspt4.tex|.
%
%\iffalse
%<*samplepart3|samplepart4>
%\fi

% Optional override for |\version| flag:
%    \begin{macrocode}
%%\providecommand{\version}{final}
%    \end{macrocode}

% Include the main document:
%    \begin{macrocode}
\input{childdoc.def}
\childdocby{cdocsamp}
%    \end{macrocode}

%\iffalse
%</samplepart3|samplepart4>
%\fi
%
%\iffalse
%<*samplepart3>
%\fi
% Some text for part 3:
%    \begin{macrocode}
some text in part three
%    \end{macrocode}

%\iffalse
%</samplepart3>
%\fi
% Some text for part 4:
%\iffalse
%<*samplepart4>
%\fi
%    \begin{macrocode}
more text in part four
%    \end{macrocode}

%\iffalse
%</samplepart4>
%\fi
%
% %%%%%%%%%%%%%%%%%%%%%%%%%%%%%%%%%%%%%%
% \paragraph{Forwarding for a Complete Draft.}
%
% The following forwarding file |cdocsdrf.tex|
% compiles the main document in draft mode:
%\iffalse
%<*sampledraft>
%\fi
%    \begin{macrocode}
\def\version{draft}
\input{childdoc.def}
\childdocforward{cdocsamp}
%    \end{macrocode}

%\iffalse
%</sampledraft>
%\fi
%
% %%%%%%%%%%%%%%%%%%%%%%%%%%%%%%%%%%%%%%
% \paragraph{Forwarding for Final Version of the Chapters.}
%
% The following forwarding files |cdocsfn1.tex| and |cdocsfn2.tex|
% (with identical content)
% compile the final versions of the child documents
% |cdocsch1.tex| and |cdocsch2.tex|, respectively:
%\iffalse
%<*samplefinal>
%\fi
%    \begin{macrocode}
\def\version{final}
\input{childdoc.def}
\childdocforwardprefix[cdocsamp]{cdocsfn}{cdocsch}
%    \end{macrocode}

%\iffalse
%</samplefinal>
%\fi
%
% %%%%%%%%%%%%%%%%%%%%%%%%%%%%%%%%%%%%%%
% \paragraph{Command Line Processing.}
%
% The following three command lines generate the output files
% |cdocscld|, |cdocscl1| and |cdocscl2|
% which should be identical to
% |cdocsdrf|, |cdocsch1| and |cdocsfn2|, respectively:
% \begin{center}
% \begin{tabular}{l}
% |latex -jobname cdocscld \|\\
% |  "\def\version{draft}\input{childdoc.def}\childdocforward{cdocsamp}"|\\
% |latex -jobname cdocscl1 \|\\
% |  "\input{childdoc.def}\childdocforward[cdocsamp]{cdocsch1}"|\\
% |latex -jobname cdocscl2 \|\\
% |  "\def\version{final}\input{childdoc.def}\childdocforward{cdocsch2}"|
% \end{tabular}
% \end{center}
% Note that the trailing backslash on each first line
% merely continues the input to the second line
% (for convenient cut ant paste).
% Furthermore, the command |latex| can be replaced by any
% of its alternative versions such as |pdflatex|.
%
% %%%%%%%%%%%%%%%%%%%%%%%%%%%%%%%%%%%%%%%%%%%%%%%%%%%%%%%%%%%%%%%%%%%%%%%%%%%%%%
% %%%%%%%%%%%%%%%%%%%%%%%%%%%%%%%%%%%%%%%%%%%%%%%%%%%%%%%%%%%%%%%%%%%%%%%%%%%%%%
% \section{Implementation}
%\iffalse
%<*package>
%\fi
%
% This section describes the definitions file |childdoc.def|.

% The definitions cannot be loaded using |\usepackage| or |\RequirePackage|
% which has a mechanism to prevent loading a style file more than once.
% When loading the definitions by means of |\input|
% multiple instances have to be prevented manually:
%\iffalse
%This code needs to be before the `\ProvidesFile' directive
%which is defined at the beginning of this file.
%Therefore it is also placed there and commented out here.
%</package>
%<*discard>
%\fi
%    \begin{macrocode}
\ifdefined\childdocmain\endinput\fi
%    \end{macrocode}
%\iffalse
%</discard>
%<*package>
%\fi
%
% \macro{\ifchilddoc}
% \macro{\ifchilddocmanual}
% The conditional |\ifchilddoc| tells whether a
% child (true) or main (false) document is being compiled.
% The conditional |\ifchilddocmanual| tells whether
% the |\includeonly| mechanism is used (false) or
% the selection of child files must be performed manually (true).
% The definitions initialise to false:
%    \begin{macrocode}
\newif\ifchilddoc
\newif\ifchilddocmanual
%    \end{macrocode}

% \macro{\childdocname}
% \macro{\childdocjob}
% The macro |\childdocname| stores the name of the main document
% to be compiled. The macro |\childdocjob| stores the name of
% the document on which the \LaTeX{} compiler was originally invoked.
% The content of |\jobname| cannot be compared
% to filenames specified in the source due to different catcodes.
% The following code rescans |\jobname|, stores the result
% in |\childdocname| and saves a copy in |\childdocjob|:
%    \begin{macrocode}
\edef\childdocname{\scantokens\expandafter{\jobname\noexpand}}
\let\childdocjob\childdocname
%    \end{macrocode}

% \macro{\childdocdisable}
% The macro |\childdocdisable| prevents the main file
% from being processed more than once.
% At this stage, the main document command |\childdocmain|
% is assumed to be called once again where it should do nothing.
% Any subsequent call to it should prevent
% a secondary processing of the main document
% It overwrites the forwarding commands
% |\childdocof| and |\childdocforward|
% with empty macros to prevent further inclusions of the main document:
%    \begin{macrocode}
\newcommand{\childdocdisable}
{
  \renewcommand{\childdocmain}[1]{\renewcommand{\childdocmain}[1]{\endinput}}
  \renewcommand{\childdocof}[1]{}
  \renewcommand{\childdocby}[2][]{}
  \renewcommand{\childdocforward}[2][]{}
  \renewcommand{\childdocdisable}{}
}
%    \end{macrocode}

% \macro{\childdocmain}
% The macro |\childdocmain| is to be called at the top of the main file
% with nothing or the main filename (without extension) as argument.
% First, it breaks loops.
% If the argument is not empty and does not match |\childdocname|
% (which is set by the first inclusion of |childdoc.def|),
% |\ifchilddoc| is set to true, |\includeonly| is applied to the child file
% and |\jobname| is set to the main file
% (for proper handling of |.aux| files):
%    \begin{macrocode}
\newcommand{\childdocmain}[1]
{
  \childdocdisable\childdocmain{}
  \if?#1?\else
    \begingroup
      \def\childdoctmp{#1}
      \ifx\childdoctmp\childdocname
        \def\childdoctmp{}
      \else
        \def\childdoctmp
        {
          \childdoctrue
          \includeonly{\childdocname}
          \def\childdocjob{#1}
          \def\jobname{#1}
        }
      \fi
      \expandafter
    \endgroup
    \childdoctmp
  \fi
}
%    \end{macrocode}

% \macro{\childdocof}
% The command |\childdocof| redirects
% compilation to the main file |#1|.
%    \begin{macrocode}
\newcommand{\childdocof}[1]
{
  \childdocdisable
  \childdoctrue
  \includeonly{\childdocname}
  \def\jobname{#1}
  \def\childdocjob{#1}
  \input{#1}
}
%    \end{macrocode}

% \macro{\childdocby}
% The command |\childdocby| ....
%    \begin{macrocode}
\newcommand{\childdocby}[2][]
{
  \childdocdisable
  \childdoctrue
  \childdocmanualtrue
  \if?#1?\else
    \def\jobname{#2}
  \fi
  \def\childdocjob{#2}
  \input{#2}
  \endinput
}
%    \end{macrocode}

% \macro{\childdocforward}
% The command |\childdocforward| redirects
% compilation to the main file or
% (if the optional argument is given) a child file.
% Parameters are set as if the main file
% or a child file starting with |\childdocof| was compiled.
% Then compilation is handed over to the main file:
%    \begin{macrocode}
\newcommand{\childdocforward}[2][]
{
  \begingroup
    \if?#1?
      \def\childdoctmp
      {
        \def\childdocname{#2}
        \def\childdocjob{#2}
        \def\jobname{#2}
        \input{#2}
        \endinput
      }
    \else
      \def\childdoctmp
      {
        \childdocdisable
        \def\childdocname{#2}
        \childdoctrue
        \includeonly{#2}
        \def\childdocjob{#1}
        \def\jobname{#1}
        \input{#1}
        \endinput
      }
    \fi
    \expandafter
  \endgroup
  \childdoctmp
}
%    \end{macrocode}

% \macro{\childdocforwardprefix}
% The command |\childdocforwardprefix| redirects
% compilation to the main or a child file by means of a pattern.
% The prefix |#1| in the current filename is replaced by |#2|
% and the suffix of the current filename is kept
% (it is assumed that the filename does not contain the substring `|~~~|'
% which is used as a delimiter).
% Compilation is handed over to the new file by |\childdocforward|:
%    \begin{macrocode}
\newcommand{\childdocforwardprefix}[3][]
{
  \begingroup
    \def\childdocextract #2##1~~~{\def\childdoctmp{\childdocforward[#1]{#3##1}}}
    \expandafter\childdocextract\childdocname~~~
    \expandafter
  \endgroup
  \childdoctmp
}
%    \end{macrocode}

% \macro{\childdoc}
% The deprecated macro |\childdoc| is a legacy version of |\childdocmain|:
%    \begin{macrocode}
\newcommand{\childdoc}{\childdocmain}
%    \end{macrocode}

% \macro{\childdocredirect}
% The deprecated macro |\childdocredirect| is a legacy version
% of |\childdocforward| and |\childdocforwardprefix|:
%    \begin{macrocode}
\newcommand{\childdocredirect}[2][]
{
  \begingroup
    \if?#1?
      \def\childdoctmp{\childdocforward{#2}}
    \else
      \def\childdoctmp{\childdocforwardprefix{#1}{#2}}
    \fi
    \expandafter
  \endgroup
  \childdoctmp
}
%    \end{macrocode}

%\iffalse
%</package>
%\fi
%
\endinput
|\\
|\childdocforward[|\textit{main}|]{|\textit{dest}|}|\\
\end{tabular}
\end{center}
%
The argument \textit{dest} is the destination file
(without extension).
It should be the main file or one of the child files.
Note that further \textsf{childdoc} directives
such as |\childdocof| and |\childdocforward|
in the indicated file will be processed in this form.
The optional argument \textit{main}
passes on directly to the main file \textit{main}
while pretending to compile the child \textit{dest}.
This form behaves as if \textit{dest}
issues |\childdocof{|\textit{main}|}| right away,
and no further \textsf{childdoc} directives will be processed.

%%%%%%%%%%%%%%%%%%%%%%%%%%%%%%%%%%%%%%%%
\DescribeMacro{\...prefix}
In the alternative form |\childdocforwardprefix|,
%
\begin{center}
\begin{tabular}{l}
|% \iffalse
%
% childdoc.dtx Copyright (C) 2017-2018 Niklas Beisert
%
% This work may be distributed and/or modified under the
% conditions of the LaTeX Project Public License, either version 1.3
% of this license or (at your option) any later version.
% The latest version of this license is in
%   http://www.latex-project.org/lppl.txt
% and version 1.3 or later is part of all distributions of LaTeX
% version 2005/12/01 or later.
%
% This work has the LPPL maintenance status `maintained'.
%
% The Current Maintainer of this work is Niklas Beisert.
%
% This work consists of the files childdoc.dtx and childdoc.ins
% and the derived files childdoc.def and cdocsamp.tex with
% cdocsch1.tex, cdocsch2.tex, cdocsdrf.tex, cdocsfn1.tex, cdocsfn2.tex.
%
%<package>\ifdefined\childdocmain\endinput\fi
%<package>\ProvidesFile{childdoc.def}[2018/12/30 v2.0 child document driver]
%<samplemain>\ProvidesFile{cdocsamp.tex}[2018/12/30 v2.0 sample for childdoc]
%<*driver>
%\ProvidesFile{childdoc.drv}[2018/12/30 v2.0 childdoc reference manual file]
\PassOptionsToClass{10pt,a4paper}{article}
\documentclass{ltxdoc}

\usepackage[margin=35mm]{geometry}
\usepackage{hyperref}
\usepackage{hyperxmp}
\usepackage[usenames]{color}

\hypersetup{colorlinks=true}
\hypersetup{pdfstartview=FitH}
\hypersetup{pdfpagemode=UseNone}
\hypersetup{pdfsource={}}
\hypersetup{pdflang={en-UK}}
\hypersetup{pdfcopyright={Copyright 2017-2018 Niklas Beisert.
  This work may be distributed and/or modified under the
  conditions of the LaTeX Project Public License, either version 1.3
  of this license or (at your option) any later version.}}
\hypersetup{pdflicenseurl={http://www.latex-project.org/lppl.txt}}
\hypersetup{pdfcontactaddress={ETH Zurich, ITP, HIT K,
  Wolfgang-Pauli-Strasse 27}}
\hypersetup{pdfcontactpostcode={8093}}
\hypersetup{pdfcontactcity={Zurich}}
\hypersetup{pdfcontactcountry={Switzerland}}
\hypersetup{pdfcontactemail={nbeisert@itp.phys.ethz.ch}}
\hypersetup{pdfcontacturl={http://people.phys.ethz.ch/\xmptilde nbeisert/}}

\newcommand{\secref}[1]{\hyperref[#1]{section \ref*{#1}}}

\parskip1ex
\parindent0pt
\let\olditemize\itemize
\def\itemize{\olditemize\parskip0pt}

\begin{document}

\title{The \textsf{childdoc} Package}
\hypersetup{pdftitle={The childdoc Package}}
\author{Niklas Beisert\\[2ex]
  Institut f\"ur Theoretische Physik\\
  Eidgen\"ossische Technische Hochschule Z\"urich\\
  Wolfgang-Pauli-Strasse 27, 8093 Z\"urich, Switzerland\\[1ex]
  \href{mailto:nbeisert@itp.phys.ethz.ch}
  {\texttt{nbeisert@itp.phys.ethz.ch}}}
\hypersetup{pdfauthor={Niklas Beisert}}
\hypersetup{pdfsubject={Manual for the LaTeX2e Package childdoc}}
\date{30 December 2018, \textsf{v2.0}}
\maketitle

\begin{abstract}\noindent
\textsf{childdoc} is a \LaTeXe{} package
that enables the direct compilation
of document sections included by |\include|
to individual files.
\end{abstract}

\begingroup
\parskip0ex
\tableofcontents
\endgroup

%%%%%%%%%%%%%%%%%%%%%%%%%%%%%%%%%%%%%%%%%%%%%%%%%%%%%%%%%%%%%%%%%%%%%%%%%%%%%%%%
%%%%%%%%%%%%%%%%%%%%%%%%%%%%%%%%%%%%%%%%%%%%%%%%%%%%%%%%%%%%%%%%%%%%%%%%%%%%%%%%
\section{Introduction}

\LaTeX{} provides a mechanism to structure a large document (such as a book)
into a main file and several child files (containing the chapters)
using the |\include| command.
This mechanism is beneficial for documents
which span hundreds of pages in order to
make the source file(s) more manageable.
Moreover, compilation can be restricted to
selected child files by means of the |\includeonly| command.
The latter feature can be used to reduce the compilation time while editing
(this was significantly more useful in the earlier days of \LaTeX{})
or to generate a smaller document which is easier to navigate.
Another application of |\includeonly| is to generate
documents consisting of selected parts of the complete document.

However, there are a few drawbacks of the plain |\include| mechanism:
\begin{itemize}
\item
The child files cannot be compiled on their own,
they can only be compiled via the main file.
A naive editing environment
(such as a text editor with an option
to have the current file processed by \LaTeX)
may require one to switch to the main file before compiling;
attempting to compile the child file produces errors.
\item
The main file must be modified (each time)
to adjust the |\includeonly| command
to the present needs. This easily leaves the main file in a messy state.
\item
The generated document will always carry the filename
of the main document. This is inconvenient if
several child files are to be compiled and
to be kept for distribution.
\end{itemize}

The present package provides a simple interface
to make child files individually compilable by \LaTeX{}.
Compiling a child file then has the same effect as compiling
the main file with an |\includeonly| command
to select the appropriate child.
Moreover the generated document will carry the name of the child
rather than the main file.
This resolves all three above issues.

This feature is meant to make the editing of books,
thesis documents and lecture notes somewhat more convenient.
However, the package can also be used efficiently for
composing a series of documents (such as exercise sheets)
which are typically distributed individually.
It then assists the author in generating the individual documents
(potentially in different versions)
as well as a document containing the collected series.
Another application is in developing style files
or other kinds of included material
where compilation of the style file could redirect
to a sample or test file.

%%%%%%%%%%%%%%%%%%%%%%%%%%%%%%%%%%%%%%%%%%%%%%%%%%%%%%%%%%%%%%%%%%%%%%%%%%%%%%%%
%%%%%%%%%%%%%%%%%%%%%%%%%%%%%%%%%%%%%%%%%%%%%%%%%%%%%%%%%%%%%%%%%%%%%%%%%%%%%%%%
\section{Usage}

First of all, the package \textsf{childdoc} is \emph{not} a standard
\LaTeXe{} |.sty| style file! Therefore it needs to be invoked in
a non-standard way.

%%%%%%%%%%%%%%%%%%%%%%%%%%%%%%%%%%%%%%%%%%%%%%%%%%%%%%%%%%%%%%%%%%%%%%%%%%%%%%%%
\subsection{Included Files}
\label{sec:include}

%%%%%%%%%%%%%%%%%%%%%%%%%%%%%%%%%%%%%%%%
\DescribeMacro{\childdocmain}
To use the package, add the commands
\begin{center}
\begin{tabular}{l}
|\input{childdoc.def}|\\
|\childdocmain{}|\\
\end{tabular}
\end{center}
at the very top of the main \LaTeX{} file,
in particular \emph{before} the |\documentclass| statement!
The argument of |\childdocmain| should be left empty
(but it must be present).

%%%%%%%%%%%%%%%%%%%%%%%%%%%%%%%%%%%%%%%%
\DescribeMacro{\childdocof}
Furthermore, add the commands
\begin{center}
\begin{tabular}{l}
|\input{childdoc.def}|\\
|\childdocof{|\textit{main}|}|\\
\end{tabular}
\end{center}
at the top of every child file \textit{child}
which is included by |\include{|\textit{child}|}|
from within the main file
(or at least for those files to be compiled individually).
The argument \textit{main} must be the filename of the main file.

There are a couple of
considerations in setting up the main and child documents:

%%%%%%%%%%%%%%%%%%%%%%%%%%%%%%%%%%%%%%%%
\paragraph{Restrictions.}

Please note the following restrictions:
\begin{itemize}
\item
|\childdocmain| must be called with one argument \textit{main}
to ensure compatibility with earlier version of the package.
It must either be empty (|\childdocmain{}|)
or precisely match the filename of the main file in which it is specified.
See \secref{sec:detection} for further information.
\item
The filename \textit{main} must be specified without the |.tex| extension.
\item
The filename \textit{main} is case sensitive
(even in case-insensitive file systems)
due to internal string comparison.
\item
The argument \textit{main} should be fully expanded, it cannot be a macro.
\item
Subdirectories and special characters should be avoided in filenames.
\item
The command |\childdocmain{|\textit{main}|}| must be followed by a whitespace.
It should not be followed immediately by another command
or by a comment mark `|%|'.
This is because the \TeX{} parser reads the token immediately following
the argument of |\childdocmain| and puts it
at the beginning of every child section;
however, a white\-space is ignored.
\end{itemize}

%%%%%%%%%%%%%%%%%%%%%%%%%%%%%%%%%%%%%%%%
\paragraph{Content of Main File.}

It is advisable to place all content in the child files included by |\include|.
Any output contained in the main file will appear in all child documents
unless suppressed manually;
it cannot be suppressed automatically by the |\includeonly| directive
and thus should normally be avoided.
A method to include some content in the main file
by means of conditional processing is described in \secref{sec:conditional}.

%%%%%%%%%%%%%%%%%%%%%%%%%%%%%%%%%%%%%%%%
\paragraph{Page Numbering.}

When only a part of the document is compiled,
the appropriate numbering of pages
(as well as other status parameters)
is determined from the |.aux| files.
The latter contain information from previous passes.
However this information needs to propagate through
all intermediate child documents.
Therefore the page numbering in child documents may well
be inconsistent until the complete document is compiled at least once.

A useful (if unconventional) way to always ensure a consistent
page numbering is to restart the numbering in each child document
and denote the pages by `\textit{child}|.|\textit{page}'
where \textit{child} represents the chapter/section number of the child file.
This can be achieved by the command
|\numberwithin{page}{|\textit{child}|}|
of the \textsf{amsmath} package
where \textit{child} can be |chapter| or |section|
depending on the chosen structuring.
Alternatively, one can modify the macro |\thepage| appropriately
and reset the counter |page| at the start of each child file.

%%%%%%%%%%%%%%%%%%%%%%%%%%%%%%%%%%%%%%%%%%%%%%%%%%%%%%%%%%%%%%%%%%%%%%%%%%%%%%%%
\subsection{Conditional Processing}
\label{sec:conditional}

The package provides a mechanism to compile different versions
of a document. To customise the versions further some conditional processing
can come in handy to distinguish which version is being compiled.
The package provides two macros to describe the compilation context:

%%%%%%%%%%%%%%%%%%%%%%%%%%%%%%%%%%%%%%%%
\DescribeMacro{\ifchilddoc}
The conditional |\ifchilddoc| distinguishes between the compilation of
child documents and the main document:
%
\begin{center}
|\ifchilddoc |\textit{child-code}| |[|\||else |\textit{main-code}]| \||fi|
\end{center}

%%%%%%%%%%%%%%%%%%%%%%%%%%%%%%%%%%%%%%%%
\DescribeMacro{\childdocname}
\DescribeMacro{\childdocjob}
The macro |\childdocname| contains the filename (without extension)
of the main or child file being processed.
Note that |\childdocjob| will always contain the name of the main file.

%%%%%%%%%%%%%%%%%%%%%%%%%%%%%%%%%%%%%%%%
\paragraph{Title Page.}

Conditional processing can be used to include a title or banner page
in the main document when proper precautions are taken.
Importantly, the code in the main file should ensure that the page counter
(as well as other status parameters which are stored in the |.aux| files)
takes the same value after the conditional processing.
Otherwise the page numbers may take divergent values
depending on which part is compiled.

For example, a title page could be declared by:
%
\begin{center}
\begin{tabular}{l}
|\ifchilddoc\||else|\\
|\addtocounter{page}{-1}|\\
\textit{code for title page}\\
|\newpage|\\
|\||fi|
\end{tabular}
\end{center}
%
A banner page for the child documents can be generated by:
%
\begin{center}
\begin{tabular}{l}
|\ifchilddoc|\\
|\addtocounter{page}{-1}|\\
\textit{code for banner page}\\
|\newpage|\\
|\||fi|
\end{tabular}
\end{center}
%
Here one could write a message such as:
\begin{center}
|This is the part \childdocname{} of \childdocjob{}.|
\end{center}

%%%%%%%%%%%%%%%%%%%%%%%%%%%%%%%%%%%%%%%%%%%%%%%%%%%%%%%%%%%%%%%%%%%%%%%%%%%%%%%%
\subsection{Flags}
\label{sec:flags}

The package makes it easy to generate different versions
of the main or child documents.
To this end compilation flags can be defined
and assigned different default values.
They will be particularly useful in conjunction
with the forwarding mechanism described in \secref{sec:forward}.

For example, it may be useful to have a flag |\version|
which can be set to |draft| or |final|.
The document source will contain some conditional code
depending on the value of |\version|.
Suppose further, the flag should default to |final| for the main file
and to |draft| for child files
which is a natural assignment for editing the document.
This is achieved by placing the following code
in the preamble of the main document
(below the |\childdocmain| directive):
%
\begin{center}
\begin{tabular}{l}
|\ifchilddoc|\\
|\providecommand{\version}{draft}|\\
|\||else|\\
|\providecommand{\version}{final}|\\
|\||fi|
\end{tabular}
\end{center}
%
The definition by |\providecommand| makes sure
that previous definitions are not overwritten.
Further statements |\providecommand{\version}{...}|
can thus be added before the above code to override it.

For the main file, one might add a line
(between |\childdocmain| and the above block)
%
\begin{center}
|%\ifchilddoc\||else\providecommand{\version}{draft}\||fi|
\end{center}
%
which can be uncommented to produce a draft version.
Likewise one can add a line to the very top of a child file
(above the |\childdocof{|\textit{main}|}| directive)
%
\begin{center}
|%\providecommand{\version}{final}|
\end{center}
%
which can be uncommented to produce the final version of this child document.

%%%%%%%%%%%%%%%%%%%%%%%%%%%%%%%%%%%%%%%%%%%%%%%%%%%%%%%%%%%%%%%%%%%%%%%%%%%%%%%%
\subsection{Forwarding}
\label{sec:forward}

Different versions of the main or child documents
using compilation flags as described in \secref{sec:flags}
can be (permanently) stored in different files
for convenient compilation, viewing and distribution.
To this end, the package defines a command
to pass on compilation to a different file:

%%%%%%%%%%%%%%%%%%%%%%%%%%%%%%%%%%%%%%%%
\DescribeMacro{\childdocforward}
The command |\childdocforward| redirects processing to
another source file:
%
\begin{center}
\begin{tabular}{l}
|\input{childdoc.def}|\\
|\childdocforward[|\textit{main}|]{|\textit{dest}|}|\\
\end{tabular}
\end{center}
%
The argument \textit{dest} is the destination file
(without extension).
It should be the main file or one of the child files.
Note that further \textsf{childdoc} directives
such as |\childdocof| and |\childdocforward|
in the indicated file will be processed in this form.
The optional argument \textit{main}
passes on directly to the main file \textit{main}
while pretending to compile the child \textit{dest}.
This form behaves as if \textit{dest}
issues |\childdocof{|\textit{main}|}| right away,
and no further \textsf{childdoc} directives will be processed.

%%%%%%%%%%%%%%%%%%%%%%%%%%%%%%%%%%%%%%%%
\DescribeMacro{\...prefix}
In the alternative form |\childdocforwardprefix|,
%
\begin{center}
\begin{tabular}{l}
|\input{childdoc.def}|\\
|\childdocforwardprefix[|\textit{main}|]{|\textit{prefix}|}{|\textit{dest}|}|
\end{tabular}
\end{center}
%
the destination file is determined by a pattern
depending on the current file:
To make this work, the current file must be called
`{\textit{prefix}\hspace{0.2em}\textit{suffix}}'
with \textit{prefix} matching precisely the argument.
Processing is then passed on to the file
`{\textit{dest}\hspace{0.2em}\textit{suffix}}'.
Surely, the same effect is achieved by
directly specifying the
argument `{\textit{dest}\hspace{0.2em}\textit{suffix}}'
in the first form.
However, that requires to set up a different file
for each child. With the alternative form of the command
all these files can have exactly the same content
which simplifies setting them up and maintaining them.

For example, the following file |draft.tex|
with a compilation flag |\version| as described in \secref{sec:flags}
compiles the main document as a draft:
%
\begin{center}
\begin{tabular}{l}
|\def\version{draft}|\\
|\input{childdoc.def}|\\
|\childdocforward{|\textit{main}|}|
\end{tabular}
\end{center}
%
Likewise, the following files |final|\textit{nn}|.tex|
compile the final version of the child document
|child|\textit{nn}|.tex|:
%
\begin{center}
\begin{tabular}{l}
|\def\version{final}|\\
|\input{childdoc.def}|\\
|\childdocforwardprefix{final}{child}|
\end{tabular}
\end{center}
%

Note that when several versions of a main file and/or of each child file
are to be generated, it may be convenient to set up a |Makefile| or
shell script to automatise the process.

%%%%%%%%%%%%%%%%%%%%%%%%%%%%%%%%%%%%%%%%%%%%%%%%%%%%%%%%%%%%%%%%%%%%%%%%%%%%%%%%
\subsection{Command Line Processing}
\label{sec:commandline}

The effect of redirection files can also be achieved by invoking
the \LaTeX{} compiler with a more elaborate command line.
Most conveniently this should be done as part
of a shell script or a |Makefile|.

When using \textsf{childdoc} in the main file, the following
command lines effectively perform a redirection
(note that depending on the shell being used,
backslashes may have to be doubled: `|\|' $\to$ `|\\|'):
%
\begin{center}
|... -jobname "|\textit{target}|" |\\|"|[\textit{flags}]%
|\input{childdoc.def}\childdocforward[|\textit{main}|]{|\textit{dest}|}"|
\end{center}
%
Here \textit{target} is the name of the output file,
\textit{main} is the name of the main file
and \textit{dest} is the name of the main or child file to be processed
(all filenames without extensions).
The optional argument \textit{main} can be omitted
if \textit{main} matches \textit{dest}.
Optionally, compilation \textit{flags} can be defined via |\def| commands.
This command line makes the \TeX{} engine believe
it is compiling the file \textit{target}
whose content is specified as the latter parameter.
The provided code then forwards the processing to
\textit{main} or \textit{dest} as described in \secref{sec:forward}.

%%%%%%%%%%%%%%%%%%%%%%%%%%%%%%%%%%%%%%%%%%%%%%%%%%%%%%%%%%%%%%%%%%%%%%%%%%%%%%%%
\subsection{Include by Input}
\label{sec:input}

Including child documents by |\include| has some restrictions by design.
Most notably, the content of a child document always occupies
its own set of pages; pages cannot be shared between child documents.
Usually, this behaviour makes perfect sense
because each child document contain an essential part of the document.
However, in some situations it may be desirable to compose
a document from a collection of parts
without having mandatory page breaks between then.
For this case, the package
provides a mechanism to include parts
by |\input| which can also be processed individually.
However, by construction this mechanism
requires manual handling of the content to be output.

%%%%%%%%%%%%%%%%%%%%%%%%%%%%%%%%%%%%%%%%
\DescribeMacro{\ifchilddocmanual}
The main file should be prepared as usual, see \secref{sec:include}.
However, the document body must make a distinction
between processing of an individual part and of the main document, e.g.:
%
\begin{center}
\begin{tabular}{l}
|\ifchilddocmanual|\\
|\input{\childdocname}|\\
|\||else|\\
\textit{document body with }|\input{|\textit{part}|}|\\
|\||fi|
\end{tabular}
\end{center}
%
The conditional |\ifchilddocmanual| is true whenever
a part to be included by |\input| is being compiled,
and the name of the part is stored in |\childdocname|.

%%%%%%%%%%%%%%%%%%%%%%%%%%%%%%%%%%%%%%%%
\DescribeMacro{\childdocby}
Each part to be included by |\input| should start with:
%
\begin{center}
\begin{tabular}{l}
|\input{childdoc.def}|\\
|\childdocby{|\textit{main}|}|\\
\end{tabular}
\end{center}
%
The directive |\childdocby| is similar to |\childdocof|
described in \secref{sec:include},
but the subsequent selection of content must be done manually.
To that end, both |\ifchilddoc| and |\ifchilddocmanual|
will be true upon processing of a part,
and the name of the part is stored in |\childdocname|.
Note that |\jobname| will be set to the filename of the current part
so that each part receives an individual |.aux| file
that does not interfere with the |.aux| file(s) of the main document.
This behaviour can be altered by the alternative form
|\childdocby[*]{|\textit{main}|}| (with a non-empty optional argument)
which uses the |.aux| file of the main document
by setting |\jobname| to \textit{main}.

%%%%%%%%%%%%%%%%%%%%%%%%%%%%%%%%%%%%%%%%%%%%%%%%%%%%%%%%%%%%%%%%%%%%%%%%%%%%%%%%
\subsection{Driver Development}
\label{sec:driver}

The \textsf{childdoc} mechanism can also be use for the development
of definition files such as \LaTeX{} styles or classes.
This case differs from the above setup with multiple parts
included by |\include| in that no |\includeonly| should be invoked.
This can be achieved by starting the include file
(before |\ProvidesPackage|) with:
%
\begin{center}
\begin{tabular}{l}
|\input{childdoc.def}|\\
|\childdocforward{|\textit{main}|}|\\
\end{tabular}
\end{center}
%
or alternatively with:
%
\begin{center}
\begin{tabular}{l}
|\input{childdoc.def}|\\
|\childdocby{|\textit{main}|}|\\
\end{tabular}
\end{center}
%
Both forms have slightly different effects as described above.
The main file is prepared as usual, see \secref{sec:include}.

%%%%%%%%%%%%%%%%%%%%%%%%%%%%%%%%%%%%%%%%%%%%%%%%%%%%%%%%%%%%%%%%%%%%%%%%%%%%%%%%
\subsection{Legacy Detection}
\label{sec:detection}

The directive |\childdocmain| in the main file can detect
whether the complete document or merely a child is to be compiled
even without using the directive |\childdocof|.
This method is deprecated because it is less robust
and there is no compelling reason to use it;
it is merely provided for backward compatibility
and it may be removed in future versions.

If the detection mechanism is to be used,
it is mandatory to correctly specify
the filename of the main file as the argument of |\childdocmain|:
%
\begin{center}
\begin{tabular}{l}
|\input{childdoc.def}|\\
|\childdocmain{|\textit{main}|}|\\
\end{tabular}
\end{center}
%
If |\jobname| does not match the argument \textit{main} of |\childdocmain|,
it is assumed that |\jobname| points to the child file to be compiled.
When using |\childdocmain| with the main file specified as argument,
it suffices to start a child file
with just |\input{|\textit{main}|}|
without loading of the package and using |\childdocof|.
If instead all processing is done
with the appropriate \textsf{childdoc} directives,
the argument of \textit{main} of |\childdocmain| can be empty.

An alternative version of the command line processing described
in \secref{sec:commandline} using the detection mechanism reads:
%
\begin{center}
|... -jobname "|\textit{target}|" "|[\textit{flags}]%
[|\def\jobname{|\textit{dest}|}|]|\input{|\textit{main}|}"|
\end{center}

%%%%%%%%%%%%%%%%%%%%%%%%%%%%%%%%%%%%%%%%%%%%%%%%%%%%%%%%%%%%%%%%%%%%%%%%%%%%%%%%
\subsection{Manual Code}
\label{sec:manual}

In case one cannot be certain whether the definitions file |childdoc.def|
is installed on the target \TeX{} distribution
and one prefers not to ship it,
it is conceivable to paste a few relevant commands into the sources.

To that end, drop all statements |\input{childdoc.def}|
and perform the replacements as outlined below.
Instead of |\childdocmain{|\textit{main}|}| add the following code
to the top of the main file:
%
\begin{center}
\begin{tabular}{l}
|\||ifdefined\childdocname\endinput\||fi\newif\ifchilddoc|\\
|\edef\childdocname{\scantokens\expandafter{\jobname\noexpand}}|\\
|\def\childdocmain{|\textit{main}|}\||ifx\childdocmain\childdocname\||else|\\
|\childdoctrue\includeonly{\childdocname}\let\jobname\childdocmain\||fi|\\
\end{tabular}
\end{center}
%
Instead of |\childdocof{|\textit{main}|}| just include the main file
at the top of each child file:
%
\begin{center}
|\input{|\textit{main}|}|
\end{center}
%
A simple redirection |\childdocforward{|\textit{dest}|}| is achieved by:
%
\begin{center}
|\def\jobname{|\textit{dest}|}\input{\jobname}|
\end{center}
%
The redirection with prefix
|\childdocforwardprefix[|\textit{prefix}|]{|\textit{dest}|}|
is accomplished by:
%
\begin{center}
\begin{tabular}{l}
|{\edef\jobname{\scantokens\expandafter{\jobname\noexpand}}|\\
|\def\redirectjob |\textit{prefix}|#1~~~{\gdef\jobname{|\textit{dest}|#1}}|\\
|\expandafter\redirectjob\jobname~~~}\input{\jobname}|
\end{tabular}
\end{center}

In an alternative approach,
child documents can be compiled by a specific command line
without additional code or specific definitions:
%
\begin{center}
|... -jobname "|\textit{target}|" "|[\textit{flags}]%
|\includeonly{|\textit{dest}|}\input{|\textit{main}|}"|
\end{center}
%

%%%%%%%%%%%%%%%%%%%%%%%%%%%%%%%%%%%%%%%%%%%%%%%%%%%%%%%%%%%%%%%%%%%%%%%%%%%%%%%%
%%%%%%%%%%%%%%%%%%%%%%%%%%%%%%%%%%%%%%%%%%%%%%%%%%%%%%%%%%%%%%%%%%%%%%%%%%%%%%%%
\section{Information}

%%%%%%%%%%%%%%%%%%%%%%%%%%%%%%%%%%%%%%%%%%%%%%%%%%%%%%%%%%%%%%%%%%%%%%%%%%%%%%%%
\subsection{Copyright}

Copyright \copyright{} 2017--2018 Niklas Beisert

This work may be distributed and/or modified under the
conditions of the \LaTeX{} Project Public License, either version 1.3
of this license or (at your option) any later version.
The latest version of this license is in
  \url{http://www.latex-project.org/lppl.txt}
and version 1.3 or later is part of all distributions of \LaTeX{}
version 2005/12/01 or later.

This work has the LPPL maintenance status `maintained'.

The Current Maintainer of this work is Niklas Beisert.

This work consists of the files |README.txt|, |childdoc.ins| and |childdoc.dtx|
as well as the derived files |childdoc.def|, |cdocsamp.tex|
with |cdocsch1.tex|, |cdocsch2.tex|, |cdocspt3.tex|, |cdocspt4.tex|,
|cdocsdrf.tex|, |cdocsfn1.tex|, |cdocsfn2.tex|
as well as |childdoc.pdf|.

%%%%%%%%%%%%%%%%%%%%%%%%%%%%%%%%%%%%%%%%%%%%%%%%%%%%%%%%%%%%%%%%%%%%%%%%%%%%%%%%
\subsection{Files and Installation}

The package consists of the files:
%
\begin{center}
\begin{tabular}{ll}
    |README.txt|   & readme file \\
    |childdoc.ins| & installation file \\
    |childdoc.dtx| & source file \\
    |childdoc.def| & definition file \\
    |cdocsamp.tex| & sample main file \\
    |cdocsch1.tex| & sample include file \\
    |cdocsch2.tex| & sample include file \\
    |cdocspt3.tex| & sample part file \\
    |cdocspt4.tex| & sample part file \\
    |cdocsdrf.tex| & sample redirection file \\
    |cdocsfn1.tex| & sample redirection file \\
    |cdocsfn2.tex| & sample redirection file \\
    |childdoc.pdf| & manual
\end{tabular}
\end{center}
%
The distribution consists of the files
|README.txt|, |childdoc.ins| and |childdoc.dtx|.
%
\begin{itemize}
\item
Run (pdf)\LaTeX{} on |childdoc.dtx|
to compile the manual |childdoc.pdf| (this file).
\item
Run \LaTeX{} on |childdoc.ins| to create the definitions file |childdoc.def|
and the sample |cdocsamp.tex| with include files
|cdocsch1.tex|, |cdocsch2.tex|, |cdocspt3.tex|, |cdocspt4.tex|,
|cdocsdrf.tex|, |cdocsfn1.tex|, |cdocsfn2.tex|.
Then copy the file |childdoc.def| to an appropriate directory of your \LaTeX{}
distribution, e.g.\ \textit{texmf-root}|/tex/latex/childdoc|.
\end{itemize}

%%%%%%%%%%%%%%%%%%%%%%%%%%%%%%%%%%%%%%%%%%%%%%%%%%%%%%%%%%%%%%%%%%%%%%%%%%%%%%%%
\subsection{Related CTAN Packages}

There are several other packages which offer a similar functionality:
%
\begin{itemize}
\item
The packages
\href{http://ctan.org/pkg/docmute}{\textsf{docmute}},
\href{http://ctan.org/pkg/includex}{\textsf{includex}} and
\href{http://ctan.org/pkg/standalone}{\textsf{standalone}}
provide commands to include only the document body of
a child file thus allowing both files to be compiled individually.
\item
The packages \href{http://ctan.org/pkg/subdocs}{\textsf{subdocs}}
and \href{http://ctan.org/pkg/subfiles}{\textsf{subfiles}}
provide structures in which the main and child documents can be
encapsulated and allowing them to be compiled individually.
The inclusion mechanism is different from the conventional |\include|.
\item
The package \href{http://ctan.org/pkg/combine}{\textsf{combine}}
is an elaborate solution to combine several documents into one.
\end{itemize}
%
See also the CTAN topic \href{http://ctan.org/topic/subdocs}{\textsf{subdocs}}
for further related packages.
The present package differs from the above solutions in that
a document structure constructed with the conventional |\include| mechanism
just needs two extra commands at the top of every file
such that all constituent files can be compiled individually.

%%%%%%%%%%%%%%%%%%%%%%%%%%%%%%%%%%%%%%%%%%%%%%%%%%%%%%%%%%%%%%%%%%%%%%%%%%%%%%%%
%\subsection{Feature Suggestions}
%
%The following is a list of features which may be useful for future
%versions of this package:
%%
%\begin{itemize}
%\item
%\ldots
%\end{itemize}

%%%%%%%%%%%%%%%%%%%%%%%%%%%%%%%%%%%%%%%%%%%%%%%%%%%%%%%%%%%%%%%%%%%%%%%%%%%%%%%%
\subsection{Revision History}

%%%%%%%%%%%%%%%%%%%%%%%%%%%%%%%%%%%%%%%%
\paragraph{v2.0:} 2018/12/30

\begin{itemize}
\item
immediate forward processing
\item
added |\childdocby| mechanism
\item
manual restructured
\end{itemize}

%%%%%%%%%%%%%%%%%%%%%%%%%%%%%%%%%%%%%%%%
\paragraph{v1.6:} 2018/01/17

\begin{itemize}
\item
application for development of include files
\item
corrections to manual
\end{itemize}

%%%%%%%%%%%%%%%%%%%%%%%%%%%%%%%%%%%%%%%%
\paragraph{v1.5:} 2017/05/21

\begin{itemize}
\item
more complete structuring introduced
\item
|\childdocof| introduced
\item
|\childdoc| renamed to |\childdocmain|
\item
|\childredirect| renamed to |\childdocforward| and |\childdocforwardprefix|
and functionality expanded
\end{itemize}

%%%%%%%%%%%%%%%%%%%%%%%%%%%%%%%%%%%%%%%%
\paragraph{v1.0:} 2017/04/27

\begin{itemize}
\item
manual and install package
\item
first version published on CTAN
\end{itemize}

%%%%%%%%%%%%%%%%%%%%%%%%%%%%%%%%%%%%%%%%
\paragraph{v0.6:} 2017/04/26

\begin{itemize}
\item
redirection mechanism added
\end{itemize}

%%%%%%%%%%%%%%%%%%%%%%%%%%%%%%%%%%%%%%%%
\paragraph{v0.5:} 2017/04/26

\begin{itemize}
\item
functionality in definition file
\end{itemize}


%%%%%%%%%%%%%%%%%%%%%%%%%%%%%%%%%%%%%%%%%%%%%%%%%%%%%%%%%%%%%%%%%%%%%%%%%%%%%%%%
%%%%%%%%%%%%%%%%%%%%%%%%%%%%%%%%%%%%%%%%%%%%%%%%%%%%%%%%%%%%%%%%%%%%%%%%%%%%%%%%
%%%%%%%%%%%%%%%%%%%%%%%%%%%%%%%%%%%%%%%%%%%%%%%%%%%%%%%%%%%%%%%%%%%%%%%%%%%%%%%%
\appendix

\settowidth\MacroIndent{\rmfamily\scriptsize 000\ }

 \DocInput{childdoc.dtx}

\end{document}
%</driver>
% \fi
%
% %%%%%%%%%%%%%%%%%%%%%%%%%%%%%%%%%%%%%%%%%%%%%%%%%%%%%%%%%%%%%%%%%%%%%%%%%%%%%%
% %%%%%%%%%%%%%%%%%%%%%%%%%%%%%%%%%%%%%%%%%%%%%%%%%%%%%%%%%%%%%%%%%%%%%%%%%%%%%%
% \section{Sample}
%\iffalse
%<*samplemain>
%\fi
%
% The following presents a sample document
% with two chapters, two parts, a title page,
% a compile flag as well as three forwarding files to set the flag.
% It consists of eight |.tex| files:
% \begin{center}
% \begin{tabular}{ll}
% |cdocsamp.tex|&main file\\
% |cdocsch1.tex|&include file for chapter 1\\
% |cdocsch2.tex|&include file for chapter 2\\
% |cdocspt3.tex|&include file for part 3\\
% |cdocspt4.tex|&include file for part 4\\
% |cdocsdrf.tex|&forwarding file for main file in draft mode\\
% |cdocsfi1.tex|&forwarding file for final version of chapter 1\\
% |cdocsfi2.tex|&forwarding file for final version of chapter 2\\
% \end{tabular}
% \end{center}
% Each of the eight files can be compiled directly by the \LaTeX{} compiler.
%
% %%%%%%%%%%%%%%%%%%%%%%%%%%%%%%%%%%%%%%
% \paragraph{Main File.}
%
% The main file is called |cdocsamp.tex|.
%
% Load the \textsf{childdoc} definitions and
% declare the filename for the main document:
%    \begin{macrocode}
\input{childdoc.def}
\childdocmain{}
%    \end{macrocode}

% Optional override for |\version| flag:
%    \begin{macrocode}
%%\ifchilddoc\else\providecommand{\version}{draft}\fi
%    \end{macrocode}

% Define the default values for the |\version| flag
% (|final| for the main file and |draft| for childs):
%    \begin{macrocode}
\ifchilddoc
\providecommand{\version}{draft}
\else
\providecommand{\version}{final}
\fi
%    \end{macrocode}

% Load the standard document class:
%    \begin{macrocode}
\documentclass[12pt]{article}
%    \end{macrocode}

% Start the document body:
%    \begin{macrocode}
\begin{document}
%    \end{macrocode}

% Declare a title page.
% Print title, part of document being processed and version flag:
%    \begin{macrocode}
\addtocounter{page}{-1}
\begin{center}
{\LARGE\bfseries{}childdoc example\par}
\vspace{1cm}
\ifchilddoc
\ifchilddocmanual part\else chapter\fi:
`\childdocname' of `\childdocjob'\par
\else
main document: `\childdocjob'\par
\fi
version: \version\par
\end{center}
\newpage
%    \end{macrocode}

% Manually include selected file,
% otherwise process as usual:
%    \begin{macrocode}
\ifchilddocmanual
\section*{part `\childdocname'}
\input{\childdocname}
\else
%    \end{macrocode}

% Include the two chapters:
%    \begin{macrocode}
\include{cdocsch1}
\include{cdocsch2}
%    \end{macrocode}

% Include the two parts unless only chapters should be displayed:
%    \begin{macrocode}
\ifchilddoc\else
\section{part three}
\input{cdocspt3}
\section{part four}
\input{cdocspt4}
\fi
%    \end{macrocode}

% Process as usual until here:
%    \begin{macrocode}
\fi
%    \end{macrocode}

% End of document body:
%    \begin{macrocode}
\end{document}
%    \end{macrocode}
%\iffalse
%</samplemain>
%\fi
%
% %%%%%%%%%%%%%%%%%%%%%%%%%%%%%%%%%%%%%%
% \paragraph{Chapter Include Files.}
%
% The include files are called |cdocsch1.tex| and |cdocsch2.tex|.
%
%\iffalse
%<*samplechap1|samplechap2>
%\fi

% Optional override for |\version| flag:
%    \begin{macrocode}
%%\providecommand{\version}{final}
%    \end{macrocode}

% Include the main document:
%    \begin{macrocode}
\input{childdoc.def}
\childdocof{cdocsamp}
%    \end{macrocode}

%\iffalse
%</samplechap1|samplechap2>
%\fi
%
%\iffalse
%<*samplechap1>
%\fi
% Some text for chapter 1:
%    \begin{macrocode}
\section{one}
some text in chapter one
%    \end{macrocode}

%\iffalse
%</samplechap1>
%\fi
% Some text for chapter 2:
%\iffalse
%<*samplechap2>
%\fi
%    \begin{macrocode}
\section{two}
more text in chapter two
%    \end{macrocode}

%\iffalse
%</samplechap2>
%\fi
%
% %%%%%%%%%%%%%%%%%%%%%%%%%%%%%%%%%%%%%%
% \paragraph{Part Include Files.}
%
% The include files are called |cdocspt3.tex| and |cdocspt4.tex|.
%
%\iffalse
%<*samplepart3|samplepart4>
%\fi

% Optional override for |\version| flag:
%    \begin{macrocode}
%%\providecommand{\version}{final}
%    \end{macrocode}

% Include the main document:
%    \begin{macrocode}
\input{childdoc.def}
\childdocby{cdocsamp}
%    \end{macrocode}

%\iffalse
%</samplepart3|samplepart4>
%\fi
%
%\iffalse
%<*samplepart3>
%\fi
% Some text for part 3:
%    \begin{macrocode}
some text in part three
%    \end{macrocode}

%\iffalse
%</samplepart3>
%\fi
% Some text for part 4:
%\iffalse
%<*samplepart4>
%\fi
%    \begin{macrocode}
more text in part four
%    \end{macrocode}

%\iffalse
%</samplepart4>
%\fi
%
% %%%%%%%%%%%%%%%%%%%%%%%%%%%%%%%%%%%%%%
% \paragraph{Forwarding for a Complete Draft.}
%
% The following forwarding file |cdocsdrf.tex|
% compiles the main document in draft mode:
%\iffalse
%<*sampledraft>
%\fi
%    \begin{macrocode}
\def\version{draft}
\input{childdoc.def}
\childdocforward{cdocsamp}
%    \end{macrocode}

%\iffalse
%</sampledraft>
%\fi
%
% %%%%%%%%%%%%%%%%%%%%%%%%%%%%%%%%%%%%%%
% \paragraph{Forwarding for Final Version of the Chapters.}
%
% The following forwarding files |cdocsfn1.tex| and |cdocsfn2.tex|
% (with identical content)
% compile the final versions of the child documents
% |cdocsch1.tex| and |cdocsch2.tex|, respectively:
%\iffalse
%<*samplefinal>
%\fi
%    \begin{macrocode}
\def\version{final}
\input{childdoc.def}
\childdocforwardprefix[cdocsamp]{cdocsfn}{cdocsch}
%    \end{macrocode}

%\iffalse
%</samplefinal>
%\fi
%
% %%%%%%%%%%%%%%%%%%%%%%%%%%%%%%%%%%%%%%
% \paragraph{Command Line Processing.}
%
% The following three command lines generate the output files
% |cdocscld|, |cdocscl1| and |cdocscl2|
% which should be identical to
% |cdocsdrf|, |cdocsch1| and |cdocsfn2|, respectively:
% \begin{center}
% \begin{tabular}{l}
% |latex -jobname cdocscld \|\\
% |  "\def\version{draft}\input{childdoc.def}\childdocforward{cdocsamp}"|\\
% |latex -jobname cdocscl1 \|\\
% |  "\input{childdoc.def}\childdocforward[cdocsamp]{cdocsch1}"|\\
% |latex -jobname cdocscl2 \|\\
% |  "\def\version{final}\input{childdoc.def}\childdocforward{cdocsch2}"|
% \end{tabular}
% \end{center}
% Note that the trailing backslash on each first line
% merely continues the input to the second line
% (for convenient cut ant paste).
% Furthermore, the command |latex| can be replaced by any
% of its alternative versions such as |pdflatex|.
%
% %%%%%%%%%%%%%%%%%%%%%%%%%%%%%%%%%%%%%%%%%%%%%%%%%%%%%%%%%%%%%%%%%%%%%%%%%%%%%%
% %%%%%%%%%%%%%%%%%%%%%%%%%%%%%%%%%%%%%%%%%%%%%%%%%%%%%%%%%%%%%%%%%%%%%%%%%%%%%%
% \section{Implementation}
%\iffalse
%<*package>
%\fi
%
% This section describes the definitions file |childdoc.def|.

% The definitions cannot be loaded using |\usepackage| or |\RequirePackage|
% which has a mechanism to prevent loading a style file more than once.
% When loading the definitions by means of |\input|
% multiple instances have to be prevented manually:
%\iffalse
%This code needs to be before the `\ProvidesFile' directive
%which is defined at the beginning of this file.
%Therefore it is also placed there and commented out here.
%</package>
%<*discard>
%\fi
%    \begin{macrocode}
\ifdefined\childdocmain\endinput\fi
%    \end{macrocode}
%\iffalse
%</discard>
%<*package>
%\fi
%
% \macro{\ifchilddoc}
% \macro{\ifchilddocmanual}
% The conditional |\ifchilddoc| tells whether a
% child (true) or main (false) document is being compiled.
% The conditional |\ifchilddocmanual| tells whether
% the |\includeonly| mechanism is used (false) or
% the selection of child files must be performed manually (true).
% The definitions initialise to false:
%    \begin{macrocode}
\newif\ifchilddoc
\newif\ifchilddocmanual
%    \end{macrocode}

% \macro{\childdocname}
% \macro{\childdocjob}
% The macro |\childdocname| stores the name of the main document
% to be compiled. The macro |\childdocjob| stores the name of
% the document on which the \LaTeX{} compiler was originally invoked.
% The content of |\jobname| cannot be compared
% to filenames specified in the source due to different catcodes.
% The following code rescans |\jobname|, stores the result
% in |\childdocname| and saves a copy in |\childdocjob|:
%    \begin{macrocode}
\edef\childdocname{\scantokens\expandafter{\jobname\noexpand}}
\let\childdocjob\childdocname
%    \end{macrocode}

% \macro{\childdocdisable}
% The macro |\childdocdisable| prevents the main file
% from being processed more than once.
% At this stage, the main document command |\childdocmain|
% is assumed to be called once again where it should do nothing.
% Any subsequent call to it should prevent
% a secondary processing of the main document
% It overwrites the forwarding commands
% |\childdocof| and |\childdocforward|
% with empty macros to prevent further inclusions of the main document:
%    \begin{macrocode}
\newcommand{\childdocdisable}
{
  \renewcommand{\childdocmain}[1]{\renewcommand{\childdocmain}[1]{\endinput}}
  \renewcommand{\childdocof}[1]{}
  \renewcommand{\childdocby}[2][]{}
  \renewcommand{\childdocforward}[2][]{}
  \renewcommand{\childdocdisable}{}
}
%    \end{macrocode}

% \macro{\childdocmain}
% The macro |\childdocmain| is to be called at the top of the main file
% with nothing or the main filename (without extension) as argument.
% First, it breaks loops.
% If the argument is not empty and does not match |\childdocname|
% (which is set by the first inclusion of |childdoc.def|),
% |\ifchilddoc| is set to true, |\includeonly| is applied to the child file
% and |\jobname| is set to the main file
% (for proper handling of |.aux| files):
%    \begin{macrocode}
\newcommand{\childdocmain}[1]
{
  \childdocdisable\childdocmain{}
  \if?#1?\else
    \begingroup
      \def\childdoctmp{#1}
      \ifx\childdoctmp\childdocname
        \def\childdoctmp{}
      \else
        \def\childdoctmp
        {
          \childdoctrue
          \includeonly{\childdocname}
          \def\childdocjob{#1}
          \def\jobname{#1}
        }
      \fi
      \expandafter
    \endgroup
    \childdoctmp
  \fi
}
%    \end{macrocode}

% \macro{\childdocof}
% The command |\childdocof| redirects
% compilation to the main file |#1|.
%    \begin{macrocode}
\newcommand{\childdocof}[1]
{
  \childdocdisable
  \childdoctrue
  \includeonly{\childdocname}
  \def\jobname{#1}
  \def\childdocjob{#1}
  \input{#1}
}
%    \end{macrocode}

% \macro{\childdocby}
% The command |\childdocby| ....
%    \begin{macrocode}
\newcommand{\childdocby}[2][]
{
  \childdocdisable
  \childdoctrue
  \childdocmanualtrue
  \if?#1?\else
    \def\jobname{#2}
  \fi
  \def\childdocjob{#2}
  \input{#2}
  \endinput
}
%    \end{macrocode}

% \macro{\childdocforward}
% The command |\childdocforward| redirects
% compilation to the main file or
% (if the optional argument is given) a child file.
% Parameters are set as if the main file
% or a child file starting with |\childdocof| was compiled.
% Then compilation is handed over to the main file:
%    \begin{macrocode}
\newcommand{\childdocforward}[2][]
{
  \begingroup
    \if?#1?
      \def\childdoctmp
      {
        \def\childdocname{#2}
        \def\childdocjob{#2}
        \def\jobname{#2}
        \input{#2}
        \endinput
      }
    \else
      \def\childdoctmp
      {
        \childdocdisable
        \def\childdocname{#2}
        \childdoctrue
        \includeonly{#2}
        \def\childdocjob{#1}
        \def\jobname{#1}
        \input{#1}
        \endinput
      }
    \fi
    \expandafter
  \endgroup
  \childdoctmp
}
%    \end{macrocode}

% \macro{\childdocforwardprefix}
% The command |\childdocforwardprefix| redirects
% compilation to the main or a child file by means of a pattern.
% The prefix |#1| in the current filename is replaced by |#2|
% and the suffix of the current filename is kept
% (it is assumed that the filename does not contain the substring `|~~~|'
% which is used as a delimiter).
% Compilation is handed over to the new file by |\childdocforward|:
%    \begin{macrocode}
\newcommand{\childdocforwardprefix}[3][]
{
  \begingroup
    \def\childdocextract #2##1~~~{\def\childdoctmp{\childdocforward[#1]{#3##1}}}
    \expandafter\childdocextract\childdocname~~~
    \expandafter
  \endgroup
  \childdoctmp
}
%    \end{macrocode}

% \macro{\childdoc}
% The deprecated macro |\childdoc| is a legacy version of |\childdocmain|:
%    \begin{macrocode}
\newcommand{\childdoc}{\childdocmain}
%    \end{macrocode}

% \macro{\childdocredirect}
% The deprecated macro |\childdocredirect| is a legacy version
% of |\childdocforward| and |\childdocforwardprefix|:
%    \begin{macrocode}
\newcommand{\childdocredirect}[2][]
{
  \begingroup
    \if?#1?
      \def\childdoctmp{\childdocforward{#2}}
    \else
      \def\childdoctmp{\childdocforwardprefix{#1}{#2}}
    \fi
    \expandafter
  \endgroup
  \childdoctmp
}
%    \end{macrocode}

%\iffalse
%</package>
%\fi
%
\endinput
|\\
|\childdocforwardprefix[|\textit{main}|]{|\textit{prefix}|}{|\textit{dest}|}|
\end{tabular}
\end{center}
%
the destination file is determined by a pattern
depending on the current file:
To make this work, the current file must be called
`{\textit{prefix}\hspace{0.2em}\textit{suffix}}'
with \textit{prefix} matching precisely the argument.
Processing is then passed on to the file
`{\textit{dest}\hspace{0.2em}\textit{suffix}}'.
Surely, the same effect is achieved by
directly specifying the
argument `{\textit{dest}\hspace{0.2em}\textit{suffix}}'
in the first form.
However, that requires to set up a different file
for each child. With the alternative form of the command
all these files can have exactly the same content
which simplifies setting them up and maintaining them.

For example, the following file |draft.tex|
with a compilation flag |\version| as described in \secref{sec:flags}
compiles the main document as a draft:
%
\begin{center}
\begin{tabular}{l}
|\def\version{draft}|\\
|% \iffalse
%
% childdoc.dtx Copyright (C) 2017-2018 Niklas Beisert
%
% This work may be distributed and/or modified under the
% conditions of the LaTeX Project Public License, either version 1.3
% of this license or (at your option) any later version.
% The latest version of this license is in
%   http://www.latex-project.org/lppl.txt
% and version 1.3 or later is part of all distributions of LaTeX
% version 2005/12/01 or later.
%
% This work has the LPPL maintenance status `maintained'.
%
% The Current Maintainer of this work is Niklas Beisert.
%
% This work consists of the files childdoc.dtx and childdoc.ins
% and the derived files childdoc.def and cdocsamp.tex with
% cdocsch1.tex, cdocsch2.tex, cdocsdrf.tex, cdocsfn1.tex, cdocsfn2.tex.
%
%<package>\ifdefined\childdocmain\endinput\fi
%<package>\ProvidesFile{childdoc.def}[2018/12/30 v2.0 child document driver]
%<samplemain>\ProvidesFile{cdocsamp.tex}[2018/12/30 v2.0 sample for childdoc]
%<*driver>
%\ProvidesFile{childdoc.drv}[2018/12/30 v2.0 childdoc reference manual file]
\PassOptionsToClass{10pt,a4paper}{article}
\documentclass{ltxdoc}

\usepackage[margin=35mm]{geometry}
\usepackage{hyperref}
\usepackage{hyperxmp}
\usepackage[usenames]{color}

\hypersetup{colorlinks=true}
\hypersetup{pdfstartview=FitH}
\hypersetup{pdfpagemode=UseNone}
\hypersetup{pdfsource={}}
\hypersetup{pdflang={en-UK}}
\hypersetup{pdfcopyright={Copyright 2017-2018 Niklas Beisert.
  This work may be distributed and/or modified under the
  conditions of the LaTeX Project Public License, either version 1.3
  of this license or (at your option) any later version.}}
\hypersetup{pdflicenseurl={http://www.latex-project.org/lppl.txt}}
\hypersetup{pdfcontactaddress={ETH Zurich, ITP, HIT K,
  Wolfgang-Pauli-Strasse 27}}
\hypersetup{pdfcontactpostcode={8093}}
\hypersetup{pdfcontactcity={Zurich}}
\hypersetup{pdfcontactcountry={Switzerland}}
\hypersetup{pdfcontactemail={nbeisert@itp.phys.ethz.ch}}
\hypersetup{pdfcontacturl={http://people.phys.ethz.ch/\xmptilde nbeisert/}}

\newcommand{\secref}[1]{\hyperref[#1]{section \ref*{#1}}}

\parskip1ex
\parindent0pt
\let\olditemize\itemize
\def\itemize{\olditemize\parskip0pt}

\begin{document}

\title{The \textsf{childdoc} Package}
\hypersetup{pdftitle={The childdoc Package}}
\author{Niklas Beisert\\[2ex]
  Institut f\"ur Theoretische Physik\\
  Eidgen\"ossische Technische Hochschule Z\"urich\\
  Wolfgang-Pauli-Strasse 27, 8093 Z\"urich, Switzerland\\[1ex]
  \href{mailto:nbeisert@itp.phys.ethz.ch}
  {\texttt{nbeisert@itp.phys.ethz.ch}}}
\hypersetup{pdfauthor={Niklas Beisert}}
\hypersetup{pdfsubject={Manual for the LaTeX2e Package childdoc}}
\date{30 December 2018, \textsf{v2.0}}
\maketitle

\begin{abstract}\noindent
\textsf{childdoc} is a \LaTeXe{} package
that enables the direct compilation
of document sections included by |\include|
to individual files.
\end{abstract}

\begingroup
\parskip0ex
\tableofcontents
\endgroup

%%%%%%%%%%%%%%%%%%%%%%%%%%%%%%%%%%%%%%%%%%%%%%%%%%%%%%%%%%%%%%%%%%%%%%%%%%%%%%%%
%%%%%%%%%%%%%%%%%%%%%%%%%%%%%%%%%%%%%%%%%%%%%%%%%%%%%%%%%%%%%%%%%%%%%%%%%%%%%%%%
\section{Introduction}

\LaTeX{} provides a mechanism to structure a large document (such as a book)
into a main file and several child files (containing the chapters)
using the |\include| command.
This mechanism is beneficial for documents
which span hundreds of pages in order to
make the source file(s) more manageable.
Moreover, compilation can be restricted to
selected child files by means of the |\includeonly| command.
The latter feature can be used to reduce the compilation time while editing
(this was significantly more useful in the earlier days of \LaTeX{})
or to generate a smaller document which is easier to navigate.
Another application of |\includeonly| is to generate
documents consisting of selected parts of the complete document.

However, there are a few drawbacks of the plain |\include| mechanism:
\begin{itemize}
\item
The child files cannot be compiled on their own,
they can only be compiled via the main file.
A naive editing environment
(such as a text editor with an option
to have the current file processed by \LaTeX)
may require one to switch to the main file before compiling;
attempting to compile the child file produces errors.
\item
The main file must be modified (each time)
to adjust the |\includeonly| command
to the present needs. This easily leaves the main file in a messy state.
\item
The generated document will always carry the filename
of the main document. This is inconvenient if
several child files are to be compiled and
to be kept for distribution.
\end{itemize}

The present package provides a simple interface
to make child files individually compilable by \LaTeX{}.
Compiling a child file then has the same effect as compiling
the main file with an |\includeonly| command
to select the appropriate child.
Moreover the generated document will carry the name of the child
rather than the main file.
This resolves all three above issues.

This feature is meant to make the editing of books,
thesis documents and lecture notes somewhat more convenient.
However, the package can also be used efficiently for
composing a series of documents (such as exercise sheets)
which are typically distributed individually.
It then assists the author in generating the individual documents
(potentially in different versions)
as well as a document containing the collected series.
Another application is in developing style files
or other kinds of included material
where compilation of the style file could redirect
to a sample or test file.

%%%%%%%%%%%%%%%%%%%%%%%%%%%%%%%%%%%%%%%%%%%%%%%%%%%%%%%%%%%%%%%%%%%%%%%%%%%%%%%%
%%%%%%%%%%%%%%%%%%%%%%%%%%%%%%%%%%%%%%%%%%%%%%%%%%%%%%%%%%%%%%%%%%%%%%%%%%%%%%%%
\section{Usage}

First of all, the package \textsf{childdoc} is \emph{not} a standard
\LaTeXe{} |.sty| style file! Therefore it needs to be invoked in
a non-standard way.

%%%%%%%%%%%%%%%%%%%%%%%%%%%%%%%%%%%%%%%%%%%%%%%%%%%%%%%%%%%%%%%%%%%%%%%%%%%%%%%%
\subsection{Included Files}
\label{sec:include}

%%%%%%%%%%%%%%%%%%%%%%%%%%%%%%%%%%%%%%%%
\DescribeMacro{\childdocmain}
To use the package, add the commands
\begin{center}
\begin{tabular}{l}
|\input{childdoc.def}|\\
|\childdocmain{}|\\
\end{tabular}
\end{center}
at the very top of the main \LaTeX{} file,
in particular \emph{before} the |\documentclass| statement!
The argument of |\childdocmain| should be left empty
(but it must be present).

%%%%%%%%%%%%%%%%%%%%%%%%%%%%%%%%%%%%%%%%
\DescribeMacro{\childdocof}
Furthermore, add the commands
\begin{center}
\begin{tabular}{l}
|\input{childdoc.def}|\\
|\childdocof{|\textit{main}|}|\\
\end{tabular}
\end{center}
at the top of every child file \textit{child}
which is included by |\include{|\textit{child}|}|
from within the main file
(or at least for those files to be compiled individually).
The argument \textit{main} must be the filename of the main file.

There are a couple of
considerations in setting up the main and child documents:

%%%%%%%%%%%%%%%%%%%%%%%%%%%%%%%%%%%%%%%%
\paragraph{Restrictions.}

Please note the following restrictions:
\begin{itemize}
\item
|\childdocmain| must be called with one argument \textit{main}
to ensure compatibility with earlier version of the package.
It must either be empty (|\childdocmain{}|)
or precisely match the filename of the main file in which it is specified.
See \secref{sec:detection} for further information.
\item
The filename \textit{main} must be specified without the |.tex| extension.
\item
The filename \textit{main} is case sensitive
(even in case-insensitive file systems)
due to internal string comparison.
\item
The argument \textit{main} should be fully expanded, it cannot be a macro.
\item
Subdirectories and special characters should be avoided in filenames.
\item
The command |\childdocmain{|\textit{main}|}| must be followed by a whitespace.
It should not be followed immediately by another command
or by a comment mark `|%|'.
This is because the \TeX{} parser reads the token immediately following
the argument of |\childdocmain| and puts it
at the beginning of every child section;
however, a white\-space is ignored.
\end{itemize}

%%%%%%%%%%%%%%%%%%%%%%%%%%%%%%%%%%%%%%%%
\paragraph{Content of Main File.}

It is advisable to place all content in the child files included by |\include|.
Any output contained in the main file will appear in all child documents
unless suppressed manually;
it cannot be suppressed automatically by the |\includeonly| directive
and thus should normally be avoided.
A method to include some content in the main file
by means of conditional processing is described in \secref{sec:conditional}.

%%%%%%%%%%%%%%%%%%%%%%%%%%%%%%%%%%%%%%%%
\paragraph{Page Numbering.}

When only a part of the document is compiled,
the appropriate numbering of pages
(as well as other status parameters)
is determined from the |.aux| files.
The latter contain information from previous passes.
However this information needs to propagate through
all intermediate child documents.
Therefore the page numbering in child documents may well
be inconsistent until the complete document is compiled at least once.

A useful (if unconventional) way to always ensure a consistent
page numbering is to restart the numbering in each child document
and denote the pages by `\textit{child}|.|\textit{page}'
where \textit{child} represents the chapter/section number of the child file.
This can be achieved by the command
|\numberwithin{page}{|\textit{child}|}|
of the \textsf{amsmath} package
where \textit{child} can be |chapter| or |section|
depending on the chosen structuring.
Alternatively, one can modify the macro |\thepage| appropriately
and reset the counter |page| at the start of each child file.

%%%%%%%%%%%%%%%%%%%%%%%%%%%%%%%%%%%%%%%%%%%%%%%%%%%%%%%%%%%%%%%%%%%%%%%%%%%%%%%%
\subsection{Conditional Processing}
\label{sec:conditional}

The package provides a mechanism to compile different versions
of a document. To customise the versions further some conditional processing
can come in handy to distinguish which version is being compiled.
The package provides two macros to describe the compilation context:

%%%%%%%%%%%%%%%%%%%%%%%%%%%%%%%%%%%%%%%%
\DescribeMacro{\ifchilddoc}
The conditional |\ifchilddoc| distinguishes between the compilation of
child documents and the main document:
%
\begin{center}
|\ifchilddoc |\textit{child-code}| |[|\||else |\textit{main-code}]| \||fi|
\end{center}

%%%%%%%%%%%%%%%%%%%%%%%%%%%%%%%%%%%%%%%%
\DescribeMacro{\childdocname}
\DescribeMacro{\childdocjob}
The macro |\childdocname| contains the filename (without extension)
of the main or child file being processed.
Note that |\childdocjob| will always contain the name of the main file.

%%%%%%%%%%%%%%%%%%%%%%%%%%%%%%%%%%%%%%%%
\paragraph{Title Page.}

Conditional processing can be used to include a title or banner page
in the main document when proper precautions are taken.
Importantly, the code in the main file should ensure that the page counter
(as well as other status parameters which are stored in the |.aux| files)
takes the same value after the conditional processing.
Otherwise the page numbers may take divergent values
depending on which part is compiled.

For example, a title page could be declared by:
%
\begin{center}
\begin{tabular}{l}
|\ifchilddoc\||else|\\
|\addtocounter{page}{-1}|\\
\textit{code for title page}\\
|\newpage|\\
|\||fi|
\end{tabular}
\end{center}
%
A banner page for the child documents can be generated by:
%
\begin{center}
\begin{tabular}{l}
|\ifchilddoc|\\
|\addtocounter{page}{-1}|\\
\textit{code for banner page}\\
|\newpage|\\
|\||fi|
\end{tabular}
\end{center}
%
Here one could write a message such as:
\begin{center}
|This is the part \childdocname{} of \childdocjob{}.|
\end{center}

%%%%%%%%%%%%%%%%%%%%%%%%%%%%%%%%%%%%%%%%%%%%%%%%%%%%%%%%%%%%%%%%%%%%%%%%%%%%%%%%
\subsection{Flags}
\label{sec:flags}

The package makes it easy to generate different versions
of the main or child documents.
To this end compilation flags can be defined
and assigned different default values.
They will be particularly useful in conjunction
with the forwarding mechanism described in \secref{sec:forward}.

For example, it may be useful to have a flag |\version|
which can be set to |draft| or |final|.
The document source will contain some conditional code
depending on the value of |\version|.
Suppose further, the flag should default to |final| for the main file
and to |draft| for child files
which is a natural assignment for editing the document.
This is achieved by placing the following code
in the preamble of the main document
(below the |\childdocmain| directive):
%
\begin{center}
\begin{tabular}{l}
|\ifchilddoc|\\
|\providecommand{\version}{draft}|\\
|\||else|\\
|\providecommand{\version}{final}|\\
|\||fi|
\end{tabular}
\end{center}
%
The definition by |\providecommand| makes sure
that previous definitions are not overwritten.
Further statements |\providecommand{\version}{...}|
can thus be added before the above code to override it.

For the main file, one might add a line
(between |\childdocmain| and the above block)
%
\begin{center}
|%\ifchilddoc\||else\providecommand{\version}{draft}\||fi|
\end{center}
%
which can be uncommented to produce a draft version.
Likewise one can add a line to the very top of a child file
(above the |\childdocof{|\textit{main}|}| directive)
%
\begin{center}
|%\providecommand{\version}{final}|
\end{center}
%
which can be uncommented to produce the final version of this child document.

%%%%%%%%%%%%%%%%%%%%%%%%%%%%%%%%%%%%%%%%%%%%%%%%%%%%%%%%%%%%%%%%%%%%%%%%%%%%%%%%
\subsection{Forwarding}
\label{sec:forward}

Different versions of the main or child documents
using compilation flags as described in \secref{sec:flags}
can be (permanently) stored in different files
for convenient compilation, viewing and distribution.
To this end, the package defines a command
to pass on compilation to a different file:

%%%%%%%%%%%%%%%%%%%%%%%%%%%%%%%%%%%%%%%%
\DescribeMacro{\childdocforward}
The command |\childdocforward| redirects processing to
another source file:
%
\begin{center}
\begin{tabular}{l}
|\input{childdoc.def}|\\
|\childdocforward[|\textit{main}|]{|\textit{dest}|}|\\
\end{tabular}
\end{center}
%
The argument \textit{dest} is the destination file
(without extension).
It should be the main file or one of the child files.
Note that further \textsf{childdoc} directives
such as |\childdocof| and |\childdocforward|
in the indicated file will be processed in this form.
The optional argument \textit{main}
passes on directly to the main file \textit{main}
while pretending to compile the child \textit{dest}.
This form behaves as if \textit{dest}
issues |\childdocof{|\textit{main}|}| right away,
and no further \textsf{childdoc} directives will be processed.

%%%%%%%%%%%%%%%%%%%%%%%%%%%%%%%%%%%%%%%%
\DescribeMacro{\...prefix}
In the alternative form |\childdocforwardprefix|,
%
\begin{center}
\begin{tabular}{l}
|\input{childdoc.def}|\\
|\childdocforwardprefix[|\textit{main}|]{|\textit{prefix}|}{|\textit{dest}|}|
\end{tabular}
\end{center}
%
the destination file is determined by a pattern
depending on the current file:
To make this work, the current file must be called
`{\textit{prefix}\hspace{0.2em}\textit{suffix}}'
with \textit{prefix} matching precisely the argument.
Processing is then passed on to the file
`{\textit{dest}\hspace{0.2em}\textit{suffix}}'.
Surely, the same effect is achieved by
directly specifying the
argument `{\textit{dest}\hspace{0.2em}\textit{suffix}}'
in the first form.
However, that requires to set up a different file
for each child. With the alternative form of the command
all these files can have exactly the same content
which simplifies setting them up and maintaining them.

For example, the following file |draft.tex|
with a compilation flag |\version| as described in \secref{sec:flags}
compiles the main document as a draft:
%
\begin{center}
\begin{tabular}{l}
|\def\version{draft}|\\
|\input{childdoc.def}|\\
|\childdocforward{|\textit{main}|}|
\end{tabular}
\end{center}
%
Likewise, the following files |final|\textit{nn}|.tex|
compile the final version of the child document
|child|\textit{nn}|.tex|:
%
\begin{center}
\begin{tabular}{l}
|\def\version{final}|\\
|\input{childdoc.def}|\\
|\childdocforwardprefix{final}{child}|
\end{tabular}
\end{center}
%

Note that when several versions of a main file and/or of each child file
are to be generated, it may be convenient to set up a |Makefile| or
shell script to automatise the process.

%%%%%%%%%%%%%%%%%%%%%%%%%%%%%%%%%%%%%%%%%%%%%%%%%%%%%%%%%%%%%%%%%%%%%%%%%%%%%%%%
\subsection{Command Line Processing}
\label{sec:commandline}

The effect of redirection files can also be achieved by invoking
the \LaTeX{} compiler with a more elaborate command line.
Most conveniently this should be done as part
of a shell script or a |Makefile|.

When using \textsf{childdoc} in the main file, the following
command lines effectively perform a redirection
(note that depending on the shell being used,
backslashes may have to be doubled: `|\|' $\to$ `|\\|'):
%
\begin{center}
|... -jobname "|\textit{target}|" |\\|"|[\textit{flags}]%
|\input{childdoc.def}\childdocforward[|\textit{main}|]{|\textit{dest}|}"|
\end{center}
%
Here \textit{target} is the name of the output file,
\textit{main} is the name of the main file
and \textit{dest} is the name of the main or child file to be processed
(all filenames without extensions).
The optional argument \textit{main} can be omitted
if \textit{main} matches \textit{dest}.
Optionally, compilation \textit{flags} can be defined via |\def| commands.
This command line makes the \TeX{} engine believe
it is compiling the file \textit{target}
whose content is specified as the latter parameter.
The provided code then forwards the processing to
\textit{main} or \textit{dest} as described in \secref{sec:forward}.

%%%%%%%%%%%%%%%%%%%%%%%%%%%%%%%%%%%%%%%%%%%%%%%%%%%%%%%%%%%%%%%%%%%%%%%%%%%%%%%%
\subsection{Include by Input}
\label{sec:input}

Including child documents by |\include| has some restrictions by design.
Most notably, the content of a child document always occupies
its own set of pages; pages cannot be shared between child documents.
Usually, this behaviour makes perfect sense
because each child document contain an essential part of the document.
However, in some situations it may be desirable to compose
a document from a collection of parts
without having mandatory page breaks between then.
For this case, the package
provides a mechanism to include parts
by |\input| which can also be processed individually.
However, by construction this mechanism
requires manual handling of the content to be output.

%%%%%%%%%%%%%%%%%%%%%%%%%%%%%%%%%%%%%%%%
\DescribeMacro{\ifchilddocmanual}
The main file should be prepared as usual, see \secref{sec:include}.
However, the document body must make a distinction
between processing of an individual part and of the main document, e.g.:
%
\begin{center}
\begin{tabular}{l}
|\ifchilddocmanual|\\
|\input{\childdocname}|\\
|\||else|\\
\textit{document body with }|\input{|\textit{part}|}|\\
|\||fi|
\end{tabular}
\end{center}
%
The conditional |\ifchilddocmanual| is true whenever
a part to be included by |\input| is being compiled,
and the name of the part is stored in |\childdocname|.

%%%%%%%%%%%%%%%%%%%%%%%%%%%%%%%%%%%%%%%%
\DescribeMacro{\childdocby}
Each part to be included by |\input| should start with:
%
\begin{center}
\begin{tabular}{l}
|\input{childdoc.def}|\\
|\childdocby{|\textit{main}|}|\\
\end{tabular}
\end{center}
%
The directive |\childdocby| is similar to |\childdocof|
described in \secref{sec:include},
but the subsequent selection of content must be done manually.
To that end, both |\ifchilddoc| and |\ifchilddocmanual|
will be true upon processing of a part,
and the name of the part is stored in |\childdocname|.
Note that |\jobname| will be set to the filename of the current part
so that each part receives an individual |.aux| file
that does not interfere with the |.aux| file(s) of the main document.
This behaviour can be altered by the alternative form
|\childdocby[*]{|\textit{main}|}| (with a non-empty optional argument)
which uses the |.aux| file of the main document
by setting |\jobname| to \textit{main}.

%%%%%%%%%%%%%%%%%%%%%%%%%%%%%%%%%%%%%%%%%%%%%%%%%%%%%%%%%%%%%%%%%%%%%%%%%%%%%%%%
\subsection{Driver Development}
\label{sec:driver}

The \textsf{childdoc} mechanism can also be use for the development
of definition files such as \LaTeX{} styles or classes.
This case differs from the above setup with multiple parts
included by |\include| in that no |\includeonly| should be invoked.
This can be achieved by starting the include file
(before |\ProvidesPackage|) with:
%
\begin{center}
\begin{tabular}{l}
|\input{childdoc.def}|\\
|\childdocforward{|\textit{main}|}|\\
\end{tabular}
\end{center}
%
or alternatively with:
%
\begin{center}
\begin{tabular}{l}
|\input{childdoc.def}|\\
|\childdocby{|\textit{main}|}|\\
\end{tabular}
\end{center}
%
Both forms have slightly different effects as described above.
The main file is prepared as usual, see \secref{sec:include}.

%%%%%%%%%%%%%%%%%%%%%%%%%%%%%%%%%%%%%%%%%%%%%%%%%%%%%%%%%%%%%%%%%%%%%%%%%%%%%%%%
\subsection{Legacy Detection}
\label{sec:detection}

The directive |\childdocmain| in the main file can detect
whether the complete document or merely a child is to be compiled
even without using the directive |\childdocof|.
This method is deprecated because it is less robust
and there is no compelling reason to use it;
it is merely provided for backward compatibility
and it may be removed in future versions.

If the detection mechanism is to be used,
it is mandatory to correctly specify
the filename of the main file as the argument of |\childdocmain|:
%
\begin{center}
\begin{tabular}{l}
|\input{childdoc.def}|\\
|\childdocmain{|\textit{main}|}|\\
\end{tabular}
\end{center}
%
If |\jobname| does not match the argument \textit{main} of |\childdocmain|,
it is assumed that |\jobname| points to the child file to be compiled.
When using |\childdocmain| with the main file specified as argument,
it suffices to start a child file
with just |\input{|\textit{main}|}|
without loading of the package and using |\childdocof|.
If instead all processing is done
with the appropriate \textsf{childdoc} directives,
the argument of \textit{main} of |\childdocmain| can be empty.

An alternative version of the command line processing described
in \secref{sec:commandline} using the detection mechanism reads:
%
\begin{center}
|... -jobname "|\textit{target}|" "|[\textit{flags}]%
[|\def\jobname{|\textit{dest}|}|]|\input{|\textit{main}|}"|
\end{center}

%%%%%%%%%%%%%%%%%%%%%%%%%%%%%%%%%%%%%%%%%%%%%%%%%%%%%%%%%%%%%%%%%%%%%%%%%%%%%%%%
\subsection{Manual Code}
\label{sec:manual}

In case one cannot be certain whether the definitions file |childdoc.def|
is installed on the target \TeX{} distribution
and one prefers not to ship it,
it is conceivable to paste a few relevant commands into the sources.

To that end, drop all statements |\input{childdoc.def}|
and perform the replacements as outlined below.
Instead of |\childdocmain{|\textit{main}|}| add the following code
to the top of the main file:
%
\begin{center}
\begin{tabular}{l}
|\||ifdefined\childdocname\endinput\||fi\newif\ifchilddoc|\\
|\edef\childdocname{\scantokens\expandafter{\jobname\noexpand}}|\\
|\def\childdocmain{|\textit{main}|}\||ifx\childdocmain\childdocname\||else|\\
|\childdoctrue\includeonly{\childdocname}\let\jobname\childdocmain\||fi|\\
\end{tabular}
\end{center}
%
Instead of |\childdocof{|\textit{main}|}| just include the main file
at the top of each child file:
%
\begin{center}
|\input{|\textit{main}|}|
\end{center}
%
A simple redirection |\childdocforward{|\textit{dest}|}| is achieved by:
%
\begin{center}
|\def\jobname{|\textit{dest}|}\input{\jobname}|
\end{center}
%
The redirection with prefix
|\childdocforwardprefix[|\textit{prefix}|]{|\textit{dest}|}|
is accomplished by:
%
\begin{center}
\begin{tabular}{l}
|{\edef\jobname{\scantokens\expandafter{\jobname\noexpand}}|\\
|\def\redirectjob |\textit{prefix}|#1~~~{\gdef\jobname{|\textit{dest}|#1}}|\\
|\expandafter\redirectjob\jobname~~~}\input{\jobname}|
\end{tabular}
\end{center}

In an alternative approach,
child documents can be compiled by a specific command line
without additional code or specific definitions:
%
\begin{center}
|... -jobname "|\textit{target}|" "|[\textit{flags}]%
|\includeonly{|\textit{dest}|}\input{|\textit{main}|}"|
\end{center}
%

%%%%%%%%%%%%%%%%%%%%%%%%%%%%%%%%%%%%%%%%%%%%%%%%%%%%%%%%%%%%%%%%%%%%%%%%%%%%%%%%
%%%%%%%%%%%%%%%%%%%%%%%%%%%%%%%%%%%%%%%%%%%%%%%%%%%%%%%%%%%%%%%%%%%%%%%%%%%%%%%%
\section{Information}

%%%%%%%%%%%%%%%%%%%%%%%%%%%%%%%%%%%%%%%%%%%%%%%%%%%%%%%%%%%%%%%%%%%%%%%%%%%%%%%%
\subsection{Copyright}

Copyright \copyright{} 2017--2018 Niklas Beisert

This work may be distributed and/or modified under the
conditions of the \LaTeX{} Project Public License, either version 1.3
of this license or (at your option) any later version.
The latest version of this license is in
  \url{http://www.latex-project.org/lppl.txt}
and version 1.3 or later is part of all distributions of \LaTeX{}
version 2005/12/01 or later.

This work has the LPPL maintenance status `maintained'.

The Current Maintainer of this work is Niklas Beisert.

This work consists of the files |README.txt|, |childdoc.ins| and |childdoc.dtx|
as well as the derived files |childdoc.def|, |cdocsamp.tex|
with |cdocsch1.tex|, |cdocsch2.tex|, |cdocspt3.tex|, |cdocspt4.tex|,
|cdocsdrf.tex|, |cdocsfn1.tex|, |cdocsfn2.tex|
as well as |childdoc.pdf|.

%%%%%%%%%%%%%%%%%%%%%%%%%%%%%%%%%%%%%%%%%%%%%%%%%%%%%%%%%%%%%%%%%%%%%%%%%%%%%%%%
\subsection{Files and Installation}

The package consists of the files:
%
\begin{center}
\begin{tabular}{ll}
    |README.txt|   & readme file \\
    |childdoc.ins| & installation file \\
    |childdoc.dtx| & source file \\
    |childdoc.def| & definition file \\
    |cdocsamp.tex| & sample main file \\
    |cdocsch1.tex| & sample include file \\
    |cdocsch2.tex| & sample include file \\
    |cdocspt3.tex| & sample part file \\
    |cdocspt4.tex| & sample part file \\
    |cdocsdrf.tex| & sample redirection file \\
    |cdocsfn1.tex| & sample redirection file \\
    |cdocsfn2.tex| & sample redirection file \\
    |childdoc.pdf| & manual
\end{tabular}
\end{center}
%
The distribution consists of the files
|README.txt|, |childdoc.ins| and |childdoc.dtx|.
%
\begin{itemize}
\item
Run (pdf)\LaTeX{} on |childdoc.dtx|
to compile the manual |childdoc.pdf| (this file).
\item
Run \LaTeX{} on |childdoc.ins| to create the definitions file |childdoc.def|
and the sample |cdocsamp.tex| with include files
|cdocsch1.tex|, |cdocsch2.tex|, |cdocspt3.tex|, |cdocspt4.tex|,
|cdocsdrf.tex|, |cdocsfn1.tex|, |cdocsfn2.tex|.
Then copy the file |childdoc.def| to an appropriate directory of your \LaTeX{}
distribution, e.g.\ \textit{texmf-root}|/tex/latex/childdoc|.
\end{itemize}

%%%%%%%%%%%%%%%%%%%%%%%%%%%%%%%%%%%%%%%%%%%%%%%%%%%%%%%%%%%%%%%%%%%%%%%%%%%%%%%%
\subsection{Related CTAN Packages}

There are several other packages which offer a similar functionality:
%
\begin{itemize}
\item
The packages
\href{http://ctan.org/pkg/docmute}{\textsf{docmute}},
\href{http://ctan.org/pkg/includex}{\textsf{includex}} and
\href{http://ctan.org/pkg/standalone}{\textsf{standalone}}
provide commands to include only the document body of
a child file thus allowing both files to be compiled individually.
\item
The packages \href{http://ctan.org/pkg/subdocs}{\textsf{subdocs}}
and \href{http://ctan.org/pkg/subfiles}{\textsf{subfiles}}
provide structures in which the main and child documents can be
encapsulated and allowing them to be compiled individually.
The inclusion mechanism is different from the conventional |\include|.
\item
The package \href{http://ctan.org/pkg/combine}{\textsf{combine}}
is an elaborate solution to combine several documents into one.
\end{itemize}
%
See also the CTAN topic \href{http://ctan.org/topic/subdocs}{\textsf{subdocs}}
for further related packages.
The present package differs from the above solutions in that
a document structure constructed with the conventional |\include| mechanism
just needs two extra commands at the top of every file
such that all constituent files can be compiled individually.

%%%%%%%%%%%%%%%%%%%%%%%%%%%%%%%%%%%%%%%%%%%%%%%%%%%%%%%%%%%%%%%%%%%%%%%%%%%%%%%%
%\subsection{Feature Suggestions}
%
%The following is a list of features which may be useful for future
%versions of this package:
%%
%\begin{itemize}
%\item
%\ldots
%\end{itemize}

%%%%%%%%%%%%%%%%%%%%%%%%%%%%%%%%%%%%%%%%%%%%%%%%%%%%%%%%%%%%%%%%%%%%%%%%%%%%%%%%
\subsection{Revision History}

%%%%%%%%%%%%%%%%%%%%%%%%%%%%%%%%%%%%%%%%
\paragraph{v2.0:} 2018/12/30

\begin{itemize}
\item
immediate forward processing
\item
added |\childdocby| mechanism
\item
manual restructured
\end{itemize}

%%%%%%%%%%%%%%%%%%%%%%%%%%%%%%%%%%%%%%%%
\paragraph{v1.6:} 2018/01/17

\begin{itemize}
\item
application for development of include files
\item
corrections to manual
\end{itemize}

%%%%%%%%%%%%%%%%%%%%%%%%%%%%%%%%%%%%%%%%
\paragraph{v1.5:} 2017/05/21

\begin{itemize}
\item
more complete structuring introduced
\item
|\childdocof| introduced
\item
|\childdoc| renamed to |\childdocmain|
\item
|\childredirect| renamed to |\childdocforward| and |\childdocforwardprefix|
and functionality expanded
\end{itemize}

%%%%%%%%%%%%%%%%%%%%%%%%%%%%%%%%%%%%%%%%
\paragraph{v1.0:} 2017/04/27

\begin{itemize}
\item
manual and install package
\item
first version published on CTAN
\end{itemize}

%%%%%%%%%%%%%%%%%%%%%%%%%%%%%%%%%%%%%%%%
\paragraph{v0.6:} 2017/04/26

\begin{itemize}
\item
redirection mechanism added
\end{itemize}

%%%%%%%%%%%%%%%%%%%%%%%%%%%%%%%%%%%%%%%%
\paragraph{v0.5:} 2017/04/26

\begin{itemize}
\item
functionality in definition file
\end{itemize}


%%%%%%%%%%%%%%%%%%%%%%%%%%%%%%%%%%%%%%%%%%%%%%%%%%%%%%%%%%%%%%%%%%%%%%%%%%%%%%%%
%%%%%%%%%%%%%%%%%%%%%%%%%%%%%%%%%%%%%%%%%%%%%%%%%%%%%%%%%%%%%%%%%%%%%%%%%%%%%%%%
%%%%%%%%%%%%%%%%%%%%%%%%%%%%%%%%%%%%%%%%%%%%%%%%%%%%%%%%%%%%%%%%%%%%%%%%%%%%%%%%
\appendix

\settowidth\MacroIndent{\rmfamily\scriptsize 000\ }

 \DocInput{childdoc.dtx}

\end{document}
%</driver>
% \fi
%
% %%%%%%%%%%%%%%%%%%%%%%%%%%%%%%%%%%%%%%%%%%%%%%%%%%%%%%%%%%%%%%%%%%%%%%%%%%%%%%
% %%%%%%%%%%%%%%%%%%%%%%%%%%%%%%%%%%%%%%%%%%%%%%%%%%%%%%%%%%%%%%%%%%%%%%%%%%%%%%
% \section{Sample}
%\iffalse
%<*samplemain>
%\fi
%
% The following presents a sample document
% with two chapters, two parts, a title page,
% a compile flag as well as three forwarding files to set the flag.
% It consists of eight |.tex| files:
% \begin{center}
% \begin{tabular}{ll}
% |cdocsamp.tex|&main file\\
% |cdocsch1.tex|&include file for chapter 1\\
% |cdocsch2.tex|&include file for chapter 2\\
% |cdocspt3.tex|&include file for part 3\\
% |cdocspt4.tex|&include file for part 4\\
% |cdocsdrf.tex|&forwarding file for main file in draft mode\\
% |cdocsfi1.tex|&forwarding file for final version of chapter 1\\
% |cdocsfi2.tex|&forwarding file for final version of chapter 2\\
% \end{tabular}
% \end{center}
% Each of the eight files can be compiled directly by the \LaTeX{} compiler.
%
% %%%%%%%%%%%%%%%%%%%%%%%%%%%%%%%%%%%%%%
% \paragraph{Main File.}
%
% The main file is called |cdocsamp.tex|.
%
% Load the \textsf{childdoc} definitions and
% declare the filename for the main document:
%    \begin{macrocode}
\input{childdoc.def}
\childdocmain{}
%    \end{macrocode}

% Optional override for |\version| flag:
%    \begin{macrocode}
%%\ifchilddoc\else\providecommand{\version}{draft}\fi
%    \end{macrocode}

% Define the default values for the |\version| flag
% (|final| for the main file and |draft| for childs):
%    \begin{macrocode}
\ifchilddoc
\providecommand{\version}{draft}
\else
\providecommand{\version}{final}
\fi
%    \end{macrocode}

% Load the standard document class:
%    \begin{macrocode}
\documentclass[12pt]{article}
%    \end{macrocode}

% Start the document body:
%    \begin{macrocode}
\begin{document}
%    \end{macrocode}

% Declare a title page.
% Print title, part of document being processed and version flag:
%    \begin{macrocode}
\addtocounter{page}{-1}
\begin{center}
{\LARGE\bfseries{}childdoc example\par}
\vspace{1cm}
\ifchilddoc
\ifchilddocmanual part\else chapter\fi:
`\childdocname' of `\childdocjob'\par
\else
main document: `\childdocjob'\par
\fi
version: \version\par
\end{center}
\newpage
%    \end{macrocode}

% Manually include selected file,
% otherwise process as usual:
%    \begin{macrocode}
\ifchilddocmanual
\section*{part `\childdocname'}
\input{\childdocname}
\else
%    \end{macrocode}

% Include the two chapters:
%    \begin{macrocode}
\include{cdocsch1}
\include{cdocsch2}
%    \end{macrocode}

% Include the two parts unless only chapters should be displayed:
%    \begin{macrocode}
\ifchilddoc\else
\section{part three}
\input{cdocspt3}
\section{part four}
\input{cdocspt4}
\fi
%    \end{macrocode}

% Process as usual until here:
%    \begin{macrocode}
\fi
%    \end{macrocode}

% End of document body:
%    \begin{macrocode}
\end{document}
%    \end{macrocode}
%\iffalse
%</samplemain>
%\fi
%
% %%%%%%%%%%%%%%%%%%%%%%%%%%%%%%%%%%%%%%
% \paragraph{Chapter Include Files.}
%
% The include files are called |cdocsch1.tex| and |cdocsch2.tex|.
%
%\iffalse
%<*samplechap1|samplechap2>
%\fi

% Optional override for |\version| flag:
%    \begin{macrocode}
%%\providecommand{\version}{final}
%    \end{macrocode}

% Include the main document:
%    \begin{macrocode}
\input{childdoc.def}
\childdocof{cdocsamp}
%    \end{macrocode}

%\iffalse
%</samplechap1|samplechap2>
%\fi
%
%\iffalse
%<*samplechap1>
%\fi
% Some text for chapter 1:
%    \begin{macrocode}
\section{one}
some text in chapter one
%    \end{macrocode}

%\iffalse
%</samplechap1>
%\fi
% Some text for chapter 2:
%\iffalse
%<*samplechap2>
%\fi
%    \begin{macrocode}
\section{two}
more text in chapter two
%    \end{macrocode}

%\iffalse
%</samplechap2>
%\fi
%
% %%%%%%%%%%%%%%%%%%%%%%%%%%%%%%%%%%%%%%
% \paragraph{Part Include Files.}
%
% The include files are called |cdocspt3.tex| and |cdocspt4.tex|.
%
%\iffalse
%<*samplepart3|samplepart4>
%\fi

% Optional override for |\version| flag:
%    \begin{macrocode}
%%\providecommand{\version}{final}
%    \end{macrocode}

% Include the main document:
%    \begin{macrocode}
\input{childdoc.def}
\childdocby{cdocsamp}
%    \end{macrocode}

%\iffalse
%</samplepart3|samplepart4>
%\fi
%
%\iffalse
%<*samplepart3>
%\fi
% Some text for part 3:
%    \begin{macrocode}
some text in part three
%    \end{macrocode}

%\iffalse
%</samplepart3>
%\fi
% Some text for part 4:
%\iffalse
%<*samplepart4>
%\fi
%    \begin{macrocode}
more text in part four
%    \end{macrocode}

%\iffalse
%</samplepart4>
%\fi
%
% %%%%%%%%%%%%%%%%%%%%%%%%%%%%%%%%%%%%%%
% \paragraph{Forwarding for a Complete Draft.}
%
% The following forwarding file |cdocsdrf.tex|
% compiles the main document in draft mode:
%\iffalse
%<*sampledraft>
%\fi
%    \begin{macrocode}
\def\version{draft}
\input{childdoc.def}
\childdocforward{cdocsamp}
%    \end{macrocode}

%\iffalse
%</sampledraft>
%\fi
%
% %%%%%%%%%%%%%%%%%%%%%%%%%%%%%%%%%%%%%%
% \paragraph{Forwarding for Final Version of the Chapters.}
%
% The following forwarding files |cdocsfn1.tex| and |cdocsfn2.tex|
% (with identical content)
% compile the final versions of the child documents
% |cdocsch1.tex| and |cdocsch2.tex|, respectively:
%\iffalse
%<*samplefinal>
%\fi
%    \begin{macrocode}
\def\version{final}
\input{childdoc.def}
\childdocforwardprefix[cdocsamp]{cdocsfn}{cdocsch}
%    \end{macrocode}

%\iffalse
%</samplefinal>
%\fi
%
% %%%%%%%%%%%%%%%%%%%%%%%%%%%%%%%%%%%%%%
% \paragraph{Command Line Processing.}
%
% The following three command lines generate the output files
% |cdocscld|, |cdocscl1| and |cdocscl2|
% which should be identical to
% |cdocsdrf|, |cdocsch1| and |cdocsfn2|, respectively:
% \begin{center}
% \begin{tabular}{l}
% |latex -jobname cdocscld \|\\
% |  "\def\version{draft}\input{childdoc.def}\childdocforward{cdocsamp}"|\\
% |latex -jobname cdocscl1 \|\\
% |  "\input{childdoc.def}\childdocforward[cdocsamp]{cdocsch1}"|\\
% |latex -jobname cdocscl2 \|\\
% |  "\def\version{final}\input{childdoc.def}\childdocforward{cdocsch2}"|
% \end{tabular}
% \end{center}
% Note that the trailing backslash on each first line
% merely continues the input to the second line
% (for convenient cut ant paste).
% Furthermore, the command |latex| can be replaced by any
% of its alternative versions such as |pdflatex|.
%
% %%%%%%%%%%%%%%%%%%%%%%%%%%%%%%%%%%%%%%%%%%%%%%%%%%%%%%%%%%%%%%%%%%%%%%%%%%%%%%
% %%%%%%%%%%%%%%%%%%%%%%%%%%%%%%%%%%%%%%%%%%%%%%%%%%%%%%%%%%%%%%%%%%%%%%%%%%%%%%
% \section{Implementation}
%\iffalse
%<*package>
%\fi
%
% This section describes the definitions file |childdoc.def|.

% The definitions cannot be loaded using |\usepackage| or |\RequirePackage|
% which has a mechanism to prevent loading a style file more than once.
% When loading the definitions by means of |\input|
% multiple instances have to be prevented manually:
%\iffalse
%This code needs to be before the `\ProvidesFile' directive
%which is defined at the beginning of this file.
%Therefore it is also placed there and commented out here.
%</package>
%<*discard>
%\fi
%    \begin{macrocode}
\ifdefined\childdocmain\endinput\fi
%    \end{macrocode}
%\iffalse
%</discard>
%<*package>
%\fi
%
% \macro{\ifchilddoc}
% \macro{\ifchilddocmanual}
% The conditional |\ifchilddoc| tells whether a
% child (true) or main (false) document is being compiled.
% The conditional |\ifchilddocmanual| tells whether
% the |\includeonly| mechanism is used (false) or
% the selection of child files must be performed manually (true).
% The definitions initialise to false:
%    \begin{macrocode}
\newif\ifchilddoc
\newif\ifchilddocmanual
%    \end{macrocode}

% \macro{\childdocname}
% \macro{\childdocjob}
% The macro |\childdocname| stores the name of the main document
% to be compiled. The macro |\childdocjob| stores the name of
% the document on which the \LaTeX{} compiler was originally invoked.
% The content of |\jobname| cannot be compared
% to filenames specified in the source due to different catcodes.
% The following code rescans |\jobname|, stores the result
% in |\childdocname| and saves a copy in |\childdocjob|:
%    \begin{macrocode}
\edef\childdocname{\scantokens\expandafter{\jobname\noexpand}}
\let\childdocjob\childdocname
%    \end{macrocode}

% \macro{\childdocdisable}
% The macro |\childdocdisable| prevents the main file
% from being processed more than once.
% At this stage, the main document command |\childdocmain|
% is assumed to be called once again where it should do nothing.
% Any subsequent call to it should prevent
% a secondary processing of the main document
% It overwrites the forwarding commands
% |\childdocof| and |\childdocforward|
% with empty macros to prevent further inclusions of the main document:
%    \begin{macrocode}
\newcommand{\childdocdisable}
{
  \renewcommand{\childdocmain}[1]{\renewcommand{\childdocmain}[1]{\endinput}}
  \renewcommand{\childdocof}[1]{}
  \renewcommand{\childdocby}[2][]{}
  \renewcommand{\childdocforward}[2][]{}
  \renewcommand{\childdocdisable}{}
}
%    \end{macrocode}

% \macro{\childdocmain}
% The macro |\childdocmain| is to be called at the top of the main file
% with nothing or the main filename (without extension) as argument.
% First, it breaks loops.
% If the argument is not empty and does not match |\childdocname|
% (which is set by the first inclusion of |childdoc.def|),
% |\ifchilddoc| is set to true, |\includeonly| is applied to the child file
% and |\jobname| is set to the main file
% (for proper handling of |.aux| files):
%    \begin{macrocode}
\newcommand{\childdocmain}[1]
{
  \childdocdisable\childdocmain{}
  \if?#1?\else
    \begingroup
      \def\childdoctmp{#1}
      \ifx\childdoctmp\childdocname
        \def\childdoctmp{}
      \else
        \def\childdoctmp
        {
          \childdoctrue
          \includeonly{\childdocname}
          \def\childdocjob{#1}
          \def\jobname{#1}
        }
      \fi
      \expandafter
    \endgroup
    \childdoctmp
  \fi
}
%    \end{macrocode}

% \macro{\childdocof}
% The command |\childdocof| redirects
% compilation to the main file |#1|.
%    \begin{macrocode}
\newcommand{\childdocof}[1]
{
  \childdocdisable
  \childdoctrue
  \includeonly{\childdocname}
  \def\jobname{#1}
  \def\childdocjob{#1}
  \input{#1}
}
%    \end{macrocode}

% \macro{\childdocby}
% The command |\childdocby| ....
%    \begin{macrocode}
\newcommand{\childdocby}[2][]
{
  \childdocdisable
  \childdoctrue
  \childdocmanualtrue
  \if?#1?\else
    \def\jobname{#2}
  \fi
  \def\childdocjob{#2}
  \input{#2}
  \endinput
}
%    \end{macrocode}

% \macro{\childdocforward}
% The command |\childdocforward| redirects
% compilation to the main file or
% (if the optional argument is given) a child file.
% Parameters are set as if the main file
% or a child file starting with |\childdocof| was compiled.
% Then compilation is handed over to the main file:
%    \begin{macrocode}
\newcommand{\childdocforward}[2][]
{
  \begingroup
    \if?#1?
      \def\childdoctmp
      {
        \def\childdocname{#2}
        \def\childdocjob{#2}
        \def\jobname{#2}
        \input{#2}
        \endinput
      }
    \else
      \def\childdoctmp
      {
        \childdocdisable
        \def\childdocname{#2}
        \childdoctrue
        \includeonly{#2}
        \def\childdocjob{#1}
        \def\jobname{#1}
        \input{#1}
        \endinput
      }
    \fi
    \expandafter
  \endgroup
  \childdoctmp
}
%    \end{macrocode}

% \macro{\childdocforwardprefix}
% The command |\childdocforwardprefix| redirects
% compilation to the main or a child file by means of a pattern.
% The prefix |#1| in the current filename is replaced by |#2|
% and the suffix of the current filename is kept
% (it is assumed that the filename does not contain the substring `|~~~|'
% which is used as a delimiter).
% Compilation is handed over to the new file by |\childdocforward|:
%    \begin{macrocode}
\newcommand{\childdocforwardprefix}[3][]
{
  \begingroup
    \def\childdocextract #2##1~~~{\def\childdoctmp{\childdocforward[#1]{#3##1}}}
    \expandafter\childdocextract\childdocname~~~
    \expandafter
  \endgroup
  \childdoctmp
}
%    \end{macrocode}

% \macro{\childdoc}
% The deprecated macro |\childdoc| is a legacy version of |\childdocmain|:
%    \begin{macrocode}
\newcommand{\childdoc}{\childdocmain}
%    \end{macrocode}

% \macro{\childdocredirect}
% The deprecated macro |\childdocredirect| is a legacy version
% of |\childdocforward| and |\childdocforwardprefix|:
%    \begin{macrocode}
\newcommand{\childdocredirect}[2][]
{
  \begingroup
    \if?#1?
      \def\childdoctmp{\childdocforward{#2}}
    \else
      \def\childdoctmp{\childdocforwardprefix{#1}{#2}}
    \fi
    \expandafter
  \endgroup
  \childdoctmp
}
%    \end{macrocode}

%\iffalse
%</package>
%\fi
%
\endinput
|\\
|\childdocforward{|\textit{main}|}|
\end{tabular}
\end{center}
%
Likewise, the following files |final|\textit{nn}|.tex|
compile the final version of the child document
|child|\textit{nn}|.tex|:
%
\begin{center}
\begin{tabular}{l}
|\def\version{final}|\\
|% \iffalse
%
% childdoc.dtx Copyright (C) 2017-2018 Niklas Beisert
%
% This work may be distributed and/or modified under the
% conditions of the LaTeX Project Public License, either version 1.3
% of this license or (at your option) any later version.
% The latest version of this license is in
%   http://www.latex-project.org/lppl.txt
% and version 1.3 or later is part of all distributions of LaTeX
% version 2005/12/01 or later.
%
% This work has the LPPL maintenance status `maintained'.
%
% The Current Maintainer of this work is Niklas Beisert.
%
% This work consists of the files childdoc.dtx and childdoc.ins
% and the derived files childdoc.def and cdocsamp.tex with
% cdocsch1.tex, cdocsch2.tex, cdocsdrf.tex, cdocsfn1.tex, cdocsfn2.tex.
%
%<package>\ifdefined\childdocmain\endinput\fi
%<package>\ProvidesFile{childdoc.def}[2018/12/30 v2.0 child document driver]
%<samplemain>\ProvidesFile{cdocsamp.tex}[2018/12/30 v2.0 sample for childdoc]
%<*driver>
%\ProvidesFile{childdoc.drv}[2018/12/30 v2.0 childdoc reference manual file]
\PassOptionsToClass{10pt,a4paper}{article}
\documentclass{ltxdoc}

\usepackage[margin=35mm]{geometry}
\usepackage{hyperref}
\usepackage{hyperxmp}
\usepackage[usenames]{color}

\hypersetup{colorlinks=true}
\hypersetup{pdfstartview=FitH}
\hypersetup{pdfpagemode=UseNone}
\hypersetup{pdfsource={}}
\hypersetup{pdflang={en-UK}}
\hypersetup{pdfcopyright={Copyright 2017-2018 Niklas Beisert.
  This work may be distributed and/or modified under the
  conditions of the LaTeX Project Public License, either version 1.3
  of this license or (at your option) any later version.}}
\hypersetup{pdflicenseurl={http://www.latex-project.org/lppl.txt}}
\hypersetup{pdfcontactaddress={ETH Zurich, ITP, HIT K,
  Wolfgang-Pauli-Strasse 27}}
\hypersetup{pdfcontactpostcode={8093}}
\hypersetup{pdfcontactcity={Zurich}}
\hypersetup{pdfcontactcountry={Switzerland}}
\hypersetup{pdfcontactemail={nbeisert@itp.phys.ethz.ch}}
\hypersetup{pdfcontacturl={http://people.phys.ethz.ch/\xmptilde nbeisert/}}

\newcommand{\secref}[1]{\hyperref[#1]{section \ref*{#1}}}

\parskip1ex
\parindent0pt
\let\olditemize\itemize
\def\itemize{\olditemize\parskip0pt}

\begin{document}

\title{The \textsf{childdoc} Package}
\hypersetup{pdftitle={The childdoc Package}}
\author{Niklas Beisert\\[2ex]
  Institut f\"ur Theoretische Physik\\
  Eidgen\"ossische Technische Hochschule Z\"urich\\
  Wolfgang-Pauli-Strasse 27, 8093 Z\"urich, Switzerland\\[1ex]
  \href{mailto:nbeisert@itp.phys.ethz.ch}
  {\texttt{nbeisert@itp.phys.ethz.ch}}}
\hypersetup{pdfauthor={Niklas Beisert}}
\hypersetup{pdfsubject={Manual for the LaTeX2e Package childdoc}}
\date{30 December 2018, \textsf{v2.0}}
\maketitle

\begin{abstract}\noindent
\textsf{childdoc} is a \LaTeXe{} package
that enables the direct compilation
of document sections included by |\include|
to individual files.
\end{abstract}

\begingroup
\parskip0ex
\tableofcontents
\endgroup

%%%%%%%%%%%%%%%%%%%%%%%%%%%%%%%%%%%%%%%%%%%%%%%%%%%%%%%%%%%%%%%%%%%%%%%%%%%%%%%%
%%%%%%%%%%%%%%%%%%%%%%%%%%%%%%%%%%%%%%%%%%%%%%%%%%%%%%%%%%%%%%%%%%%%%%%%%%%%%%%%
\section{Introduction}

\LaTeX{} provides a mechanism to structure a large document (such as a book)
into a main file and several child files (containing the chapters)
using the |\include| command.
This mechanism is beneficial for documents
which span hundreds of pages in order to
make the source file(s) more manageable.
Moreover, compilation can be restricted to
selected child files by means of the |\includeonly| command.
The latter feature can be used to reduce the compilation time while editing
(this was significantly more useful in the earlier days of \LaTeX{})
or to generate a smaller document which is easier to navigate.
Another application of |\includeonly| is to generate
documents consisting of selected parts of the complete document.

However, there are a few drawbacks of the plain |\include| mechanism:
\begin{itemize}
\item
The child files cannot be compiled on their own,
they can only be compiled via the main file.
A naive editing environment
(such as a text editor with an option
to have the current file processed by \LaTeX)
may require one to switch to the main file before compiling;
attempting to compile the child file produces errors.
\item
The main file must be modified (each time)
to adjust the |\includeonly| command
to the present needs. This easily leaves the main file in a messy state.
\item
The generated document will always carry the filename
of the main document. This is inconvenient if
several child files are to be compiled and
to be kept for distribution.
\end{itemize}

The present package provides a simple interface
to make child files individually compilable by \LaTeX{}.
Compiling a child file then has the same effect as compiling
the main file with an |\includeonly| command
to select the appropriate child.
Moreover the generated document will carry the name of the child
rather than the main file.
This resolves all three above issues.

This feature is meant to make the editing of books,
thesis documents and lecture notes somewhat more convenient.
However, the package can also be used efficiently for
composing a series of documents (such as exercise sheets)
which are typically distributed individually.
It then assists the author in generating the individual documents
(potentially in different versions)
as well as a document containing the collected series.
Another application is in developing style files
or other kinds of included material
where compilation of the style file could redirect
to a sample or test file.

%%%%%%%%%%%%%%%%%%%%%%%%%%%%%%%%%%%%%%%%%%%%%%%%%%%%%%%%%%%%%%%%%%%%%%%%%%%%%%%%
%%%%%%%%%%%%%%%%%%%%%%%%%%%%%%%%%%%%%%%%%%%%%%%%%%%%%%%%%%%%%%%%%%%%%%%%%%%%%%%%
\section{Usage}

First of all, the package \textsf{childdoc} is \emph{not} a standard
\LaTeXe{} |.sty| style file! Therefore it needs to be invoked in
a non-standard way.

%%%%%%%%%%%%%%%%%%%%%%%%%%%%%%%%%%%%%%%%%%%%%%%%%%%%%%%%%%%%%%%%%%%%%%%%%%%%%%%%
\subsection{Included Files}
\label{sec:include}

%%%%%%%%%%%%%%%%%%%%%%%%%%%%%%%%%%%%%%%%
\DescribeMacro{\childdocmain}
To use the package, add the commands
\begin{center}
\begin{tabular}{l}
|\input{childdoc.def}|\\
|\childdocmain{}|\\
\end{tabular}
\end{center}
at the very top of the main \LaTeX{} file,
in particular \emph{before} the |\documentclass| statement!
The argument of |\childdocmain| should be left empty
(but it must be present).

%%%%%%%%%%%%%%%%%%%%%%%%%%%%%%%%%%%%%%%%
\DescribeMacro{\childdocof}
Furthermore, add the commands
\begin{center}
\begin{tabular}{l}
|\input{childdoc.def}|\\
|\childdocof{|\textit{main}|}|\\
\end{tabular}
\end{center}
at the top of every child file \textit{child}
which is included by |\include{|\textit{child}|}|
from within the main file
(or at least for those files to be compiled individually).
The argument \textit{main} must be the filename of the main file.

There are a couple of
considerations in setting up the main and child documents:

%%%%%%%%%%%%%%%%%%%%%%%%%%%%%%%%%%%%%%%%
\paragraph{Restrictions.}

Please note the following restrictions:
\begin{itemize}
\item
|\childdocmain| must be called with one argument \textit{main}
to ensure compatibility with earlier version of the package.
It must either be empty (|\childdocmain{}|)
or precisely match the filename of the main file in which it is specified.
See \secref{sec:detection} for further information.
\item
The filename \textit{main} must be specified without the |.tex| extension.
\item
The filename \textit{main} is case sensitive
(even in case-insensitive file systems)
due to internal string comparison.
\item
The argument \textit{main} should be fully expanded, it cannot be a macro.
\item
Subdirectories and special characters should be avoided in filenames.
\item
The command |\childdocmain{|\textit{main}|}| must be followed by a whitespace.
It should not be followed immediately by another command
or by a comment mark `|%|'.
This is because the \TeX{} parser reads the token immediately following
the argument of |\childdocmain| and puts it
at the beginning of every child section;
however, a white\-space is ignored.
\end{itemize}

%%%%%%%%%%%%%%%%%%%%%%%%%%%%%%%%%%%%%%%%
\paragraph{Content of Main File.}

It is advisable to place all content in the child files included by |\include|.
Any output contained in the main file will appear in all child documents
unless suppressed manually;
it cannot be suppressed automatically by the |\includeonly| directive
and thus should normally be avoided.
A method to include some content in the main file
by means of conditional processing is described in \secref{sec:conditional}.

%%%%%%%%%%%%%%%%%%%%%%%%%%%%%%%%%%%%%%%%
\paragraph{Page Numbering.}

When only a part of the document is compiled,
the appropriate numbering of pages
(as well as other status parameters)
is determined from the |.aux| files.
The latter contain information from previous passes.
However this information needs to propagate through
all intermediate child documents.
Therefore the page numbering in child documents may well
be inconsistent until the complete document is compiled at least once.

A useful (if unconventional) way to always ensure a consistent
page numbering is to restart the numbering in each child document
and denote the pages by `\textit{child}|.|\textit{page}'
where \textit{child} represents the chapter/section number of the child file.
This can be achieved by the command
|\numberwithin{page}{|\textit{child}|}|
of the \textsf{amsmath} package
where \textit{child} can be |chapter| or |section|
depending on the chosen structuring.
Alternatively, one can modify the macro |\thepage| appropriately
and reset the counter |page| at the start of each child file.

%%%%%%%%%%%%%%%%%%%%%%%%%%%%%%%%%%%%%%%%%%%%%%%%%%%%%%%%%%%%%%%%%%%%%%%%%%%%%%%%
\subsection{Conditional Processing}
\label{sec:conditional}

The package provides a mechanism to compile different versions
of a document. To customise the versions further some conditional processing
can come in handy to distinguish which version is being compiled.
The package provides two macros to describe the compilation context:

%%%%%%%%%%%%%%%%%%%%%%%%%%%%%%%%%%%%%%%%
\DescribeMacro{\ifchilddoc}
The conditional |\ifchilddoc| distinguishes between the compilation of
child documents and the main document:
%
\begin{center}
|\ifchilddoc |\textit{child-code}| |[|\||else |\textit{main-code}]| \||fi|
\end{center}

%%%%%%%%%%%%%%%%%%%%%%%%%%%%%%%%%%%%%%%%
\DescribeMacro{\childdocname}
\DescribeMacro{\childdocjob}
The macro |\childdocname| contains the filename (without extension)
of the main or child file being processed.
Note that |\childdocjob| will always contain the name of the main file.

%%%%%%%%%%%%%%%%%%%%%%%%%%%%%%%%%%%%%%%%
\paragraph{Title Page.}

Conditional processing can be used to include a title or banner page
in the main document when proper precautions are taken.
Importantly, the code in the main file should ensure that the page counter
(as well as other status parameters which are stored in the |.aux| files)
takes the same value after the conditional processing.
Otherwise the page numbers may take divergent values
depending on which part is compiled.

For example, a title page could be declared by:
%
\begin{center}
\begin{tabular}{l}
|\ifchilddoc\||else|\\
|\addtocounter{page}{-1}|\\
\textit{code for title page}\\
|\newpage|\\
|\||fi|
\end{tabular}
\end{center}
%
A banner page for the child documents can be generated by:
%
\begin{center}
\begin{tabular}{l}
|\ifchilddoc|\\
|\addtocounter{page}{-1}|\\
\textit{code for banner page}\\
|\newpage|\\
|\||fi|
\end{tabular}
\end{center}
%
Here one could write a message such as:
\begin{center}
|This is the part \childdocname{} of \childdocjob{}.|
\end{center}

%%%%%%%%%%%%%%%%%%%%%%%%%%%%%%%%%%%%%%%%%%%%%%%%%%%%%%%%%%%%%%%%%%%%%%%%%%%%%%%%
\subsection{Flags}
\label{sec:flags}

The package makes it easy to generate different versions
of the main or child documents.
To this end compilation flags can be defined
and assigned different default values.
They will be particularly useful in conjunction
with the forwarding mechanism described in \secref{sec:forward}.

For example, it may be useful to have a flag |\version|
which can be set to |draft| or |final|.
The document source will contain some conditional code
depending on the value of |\version|.
Suppose further, the flag should default to |final| for the main file
and to |draft| for child files
which is a natural assignment for editing the document.
This is achieved by placing the following code
in the preamble of the main document
(below the |\childdocmain| directive):
%
\begin{center}
\begin{tabular}{l}
|\ifchilddoc|\\
|\providecommand{\version}{draft}|\\
|\||else|\\
|\providecommand{\version}{final}|\\
|\||fi|
\end{tabular}
\end{center}
%
The definition by |\providecommand| makes sure
that previous definitions are not overwritten.
Further statements |\providecommand{\version}{...}|
can thus be added before the above code to override it.

For the main file, one might add a line
(between |\childdocmain| and the above block)
%
\begin{center}
|%\ifchilddoc\||else\providecommand{\version}{draft}\||fi|
\end{center}
%
which can be uncommented to produce a draft version.
Likewise one can add a line to the very top of a child file
(above the |\childdocof{|\textit{main}|}| directive)
%
\begin{center}
|%\providecommand{\version}{final}|
\end{center}
%
which can be uncommented to produce the final version of this child document.

%%%%%%%%%%%%%%%%%%%%%%%%%%%%%%%%%%%%%%%%%%%%%%%%%%%%%%%%%%%%%%%%%%%%%%%%%%%%%%%%
\subsection{Forwarding}
\label{sec:forward}

Different versions of the main or child documents
using compilation flags as described in \secref{sec:flags}
can be (permanently) stored in different files
for convenient compilation, viewing and distribution.
To this end, the package defines a command
to pass on compilation to a different file:

%%%%%%%%%%%%%%%%%%%%%%%%%%%%%%%%%%%%%%%%
\DescribeMacro{\childdocforward}
The command |\childdocforward| redirects processing to
another source file:
%
\begin{center}
\begin{tabular}{l}
|\input{childdoc.def}|\\
|\childdocforward[|\textit{main}|]{|\textit{dest}|}|\\
\end{tabular}
\end{center}
%
The argument \textit{dest} is the destination file
(without extension).
It should be the main file or one of the child files.
Note that further \textsf{childdoc} directives
such as |\childdocof| and |\childdocforward|
in the indicated file will be processed in this form.
The optional argument \textit{main}
passes on directly to the main file \textit{main}
while pretending to compile the child \textit{dest}.
This form behaves as if \textit{dest}
issues |\childdocof{|\textit{main}|}| right away,
and no further \textsf{childdoc} directives will be processed.

%%%%%%%%%%%%%%%%%%%%%%%%%%%%%%%%%%%%%%%%
\DescribeMacro{\...prefix}
In the alternative form |\childdocforwardprefix|,
%
\begin{center}
\begin{tabular}{l}
|\input{childdoc.def}|\\
|\childdocforwardprefix[|\textit{main}|]{|\textit{prefix}|}{|\textit{dest}|}|
\end{tabular}
\end{center}
%
the destination file is determined by a pattern
depending on the current file:
To make this work, the current file must be called
`{\textit{prefix}\hspace{0.2em}\textit{suffix}}'
with \textit{prefix} matching precisely the argument.
Processing is then passed on to the file
`{\textit{dest}\hspace{0.2em}\textit{suffix}}'.
Surely, the same effect is achieved by
directly specifying the
argument `{\textit{dest}\hspace{0.2em}\textit{suffix}}'
in the first form.
However, that requires to set up a different file
for each child. With the alternative form of the command
all these files can have exactly the same content
which simplifies setting them up and maintaining them.

For example, the following file |draft.tex|
with a compilation flag |\version| as described in \secref{sec:flags}
compiles the main document as a draft:
%
\begin{center}
\begin{tabular}{l}
|\def\version{draft}|\\
|\input{childdoc.def}|\\
|\childdocforward{|\textit{main}|}|
\end{tabular}
\end{center}
%
Likewise, the following files |final|\textit{nn}|.tex|
compile the final version of the child document
|child|\textit{nn}|.tex|:
%
\begin{center}
\begin{tabular}{l}
|\def\version{final}|\\
|\input{childdoc.def}|\\
|\childdocforwardprefix{final}{child}|
\end{tabular}
\end{center}
%

Note that when several versions of a main file and/or of each child file
are to be generated, it may be convenient to set up a |Makefile| or
shell script to automatise the process.

%%%%%%%%%%%%%%%%%%%%%%%%%%%%%%%%%%%%%%%%%%%%%%%%%%%%%%%%%%%%%%%%%%%%%%%%%%%%%%%%
\subsection{Command Line Processing}
\label{sec:commandline}

The effect of redirection files can also be achieved by invoking
the \LaTeX{} compiler with a more elaborate command line.
Most conveniently this should be done as part
of a shell script or a |Makefile|.

When using \textsf{childdoc} in the main file, the following
command lines effectively perform a redirection
(note that depending on the shell being used,
backslashes may have to be doubled: `|\|' $\to$ `|\\|'):
%
\begin{center}
|... -jobname "|\textit{target}|" |\\|"|[\textit{flags}]%
|\input{childdoc.def}\childdocforward[|\textit{main}|]{|\textit{dest}|}"|
\end{center}
%
Here \textit{target} is the name of the output file,
\textit{main} is the name of the main file
and \textit{dest} is the name of the main or child file to be processed
(all filenames without extensions).
The optional argument \textit{main} can be omitted
if \textit{main} matches \textit{dest}.
Optionally, compilation \textit{flags} can be defined via |\def| commands.
This command line makes the \TeX{} engine believe
it is compiling the file \textit{target}
whose content is specified as the latter parameter.
The provided code then forwards the processing to
\textit{main} or \textit{dest} as described in \secref{sec:forward}.

%%%%%%%%%%%%%%%%%%%%%%%%%%%%%%%%%%%%%%%%%%%%%%%%%%%%%%%%%%%%%%%%%%%%%%%%%%%%%%%%
\subsection{Include by Input}
\label{sec:input}

Including child documents by |\include| has some restrictions by design.
Most notably, the content of a child document always occupies
its own set of pages; pages cannot be shared between child documents.
Usually, this behaviour makes perfect sense
because each child document contain an essential part of the document.
However, in some situations it may be desirable to compose
a document from a collection of parts
without having mandatory page breaks between then.
For this case, the package
provides a mechanism to include parts
by |\input| which can also be processed individually.
However, by construction this mechanism
requires manual handling of the content to be output.

%%%%%%%%%%%%%%%%%%%%%%%%%%%%%%%%%%%%%%%%
\DescribeMacro{\ifchilddocmanual}
The main file should be prepared as usual, see \secref{sec:include}.
However, the document body must make a distinction
between processing of an individual part and of the main document, e.g.:
%
\begin{center}
\begin{tabular}{l}
|\ifchilddocmanual|\\
|\input{\childdocname}|\\
|\||else|\\
\textit{document body with }|\input{|\textit{part}|}|\\
|\||fi|
\end{tabular}
\end{center}
%
The conditional |\ifchilddocmanual| is true whenever
a part to be included by |\input| is being compiled,
and the name of the part is stored in |\childdocname|.

%%%%%%%%%%%%%%%%%%%%%%%%%%%%%%%%%%%%%%%%
\DescribeMacro{\childdocby}
Each part to be included by |\input| should start with:
%
\begin{center}
\begin{tabular}{l}
|\input{childdoc.def}|\\
|\childdocby{|\textit{main}|}|\\
\end{tabular}
\end{center}
%
The directive |\childdocby| is similar to |\childdocof|
described in \secref{sec:include},
but the subsequent selection of content must be done manually.
To that end, both |\ifchilddoc| and |\ifchilddocmanual|
will be true upon processing of a part,
and the name of the part is stored in |\childdocname|.
Note that |\jobname| will be set to the filename of the current part
so that each part receives an individual |.aux| file
that does not interfere with the |.aux| file(s) of the main document.
This behaviour can be altered by the alternative form
|\childdocby[*]{|\textit{main}|}| (with a non-empty optional argument)
which uses the |.aux| file of the main document
by setting |\jobname| to \textit{main}.

%%%%%%%%%%%%%%%%%%%%%%%%%%%%%%%%%%%%%%%%%%%%%%%%%%%%%%%%%%%%%%%%%%%%%%%%%%%%%%%%
\subsection{Driver Development}
\label{sec:driver}

The \textsf{childdoc} mechanism can also be use for the development
of definition files such as \LaTeX{} styles or classes.
This case differs from the above setup with multiple parts
included by |\include| in that no |\includeonly| should be invoked.
This can be achieved by starting the include file
(before |\ProvidesPackage|) with:
%
\begin{center}
\begin{tabular}{l}
|\input{childdoc.def}|\\
|\childdocforward{|\textit{main}|}|\\
\end{tabular}
\end{center}
%
or alternatively with:
%
\begin{center}
\begin{tabular}{l}
|\input{childdoc.def}|\\
|\childdocby{|\textit{main}|}|\\
\end{tabular}
\end{center}
%
Both forms have slightly different effects as described above.
The main file is prepared as usual, see \secref{sec:include}.

%%%%%%%%%%%%%%%%%%%%%%%%%%%%%%%%%%%%%%%%%%%%%%%%%%%%%%%%%%%%%%%%%%%%%%%%%%%%%%%%
\subsection{Legacy Detection}
\label{sec:detection}

The directive |\childdocmain| in the main file can detect
whether the complete document or merely a child is to be compiled
even without using the directive |\childdocof|.
This method is deprecated because it is less robust
and there is no compelling reason to use it;
it is merely provided for backward compatibility
and it may be removed in future versions.

If the detection mechanism is to be used,
it is mandatory to correctly specify
the filename of the main file as the argument of |\childdocmain|:
%
\begin{center}
\begin{tabular}{l}
|\input{childdoc.def}|\\
|\childdocmain{|\textit{main}|}|\\
\end{tabular}
\end{center}
%
If |\jobname| does not match the argument \textit{main} of |\childdocmain|,
it is assumed that |\jobname| points to the child file to be compiled.
When using |\childdocmain| with the main file specified as argument,
it suffices to start a child file
with just |\input{|\textit{main}|}|
without loading of the package and using |\childdocof|.
If instead all processing is done
with the appropriate \textsf{childdoc} directives,
the argument of \textit{main} of |\childdocmain| can be empty.

An alternative version of the command line processing described
in \secref{sec:commandline} using the detection mechanism reads:
%
\begin{center}
|... -jobname "|\textit{target}|" "|[\textit{flags}]%
[|\def\jobname{|\textit{dest}|}|]|\input{|\textit{main}|}"|
\end{center}

%%%%%%%%%%%%%%%%%%%%%%%%%%%%%%%%%%%%%%%%%%%%%%%%%%%%%%%%%%%%%%%%%%%%%%%%%%%%%%%%
\subsection{Manual Code}
\label{sec:manual}

In case one cannot be certain whether the definitions file |childdoc.def|
is installed on the target \TeX{} distribution
and one prefers not to ship it,
it is conceivable to paste a few relevant commands into the sources.

To that end, drop all statements |\input{childdoc.def}|
and perform the replacements as outlined below.
Instead of |\childdocmain{|\textit{main}|}| add the following code
to the top of the main file:
%
\begin{center}
\begin{tabular}{l}
|\||ifdefined\childdocname\endinput\||fi\newif\ifchilddoc|\\
|\edef\childdocname{\scantokens\expandafter{\jobname\noexpand}}|\\
|\def\childdocmain{|\textit{main}|}\||ifx\childdocmain\childdocname\||else|\\
|\childdoctrue\includeonly{\childdocname}\let\jobname\childdocmain\||fi|\\
\end{tabular}
\end{center}
%
Instead of |\childdocof{|\textit{main}|}| just include the main file
at the top of each child file:
%
\begin{center}
|\input{|\textit{main}|}|
\end{center}
%
A simple redirection |\childdocforward{|\textit{dest}|}| is achieved by:
%
\begin{center}
|\def\jobname{|\textit{dest}|}\input{\jobname}|
\end{center}
%
The redirection with prefix
|\childdocforwardprefix[|\textit{prefix}|]{|\textit{dest}|}|
is accomplished by:
%
\begin{center}
\begin{tabular}{l}
|{\edef\jobname{\scantokens\expandafter{\jobname\noexpand}}|\\
|\def\redirectjob |\textit{prefix}|#1~~~{\gdef\jobname{|\textit{dest}|#1}}|\\
|\expandafter\redirectjob\jobname~~~}\input{\jobname}|
\end{tabular}
\end{center}

In an alternative approach,
child documents can be compiled by a specific command line
without additional code or specific definitions:
%
\begin{center}
|... -jobname "|\textit{target}|" "|[\textit{flags}]%
|\includeonly{|\textit{dest}|}\input{|\textit{main}|}"|
\end{center}
%

%%%%%%%%%%%%%%%%%%%%%%%%%%%%%%%%%%%%%%%%%%%%%%%%%%%%%%%%%%%%%%%%%%%%%%%%%%%%%%%%
%%%%%%%%%%%%%%%%%%%%%%%%%%%%%%%%%%%%%%%%%%%%%%%%%%%%%%%%%%%%%%%%%%%%%%%%%%%%%%%%
\section{Information}

%%%%%%%%%%%%%%%%%%%%%%%%%%%%%%%%%%%%%%%%%%%%%%%%%%%%%%%%%%%%%%%%%%%%%%%%%%%%%%%%
\subsection{Copyright}

Copyright \copyright{} 2017--2018 Niklas Beisert

This work may be distributed and/or modified under the
conditions of the \LaTeX{} Project Public License, either version 1.3
of this license or (at your option) any later version.
The latest version of this license is in
  \url{http://www.latex-project.org/lppl.txt}
and version 1.3 or later is part of all distributions of \LaTeX{}
version 2005/12/01 or later.

This work has the LPPL maintenance status `maintained'.

The Current Maintainer of this work is Niklas Beisert.

This work consists of the files |README.txt|, |childdoc.ins| and |childdoc.dtx|
as well as the derived files |childdoc.def|, |cdocsamp.tex|
with |cdocsch1.tex|, |cdocsch2.tex|, |cdocspt3.tex|, |cdocspt4.tex|,
|cdocsdrf.tex|, |cdocsfn1.tex|, |cdocsfn2.tex|
as well as |childdoc.pdf|.

%%%%%%%%%%%%%%%%%%%%%%%%%%%%%%%%%%%%%%%%%%%%%%%%%%%%%%%%%%%%%%%%%%%%%%%%%%%%%%%%
\subsection{Files and Installation}

The package consists of the files:
%
\begin{center}
\begin{tabular}{ll}
    |README.txt|   & readme file \\
    |childdoc.ins| & installation file \\
    |childdoc.dtx| & source file \\
    |childdoc.def| & definition file \\
    |cdocsamp.tex| & sample main file \\
    |cdocsch1.tex| & sample include file \\
    |cdocsch2.tex| & sample include file \\
    |cdocspt3.tex| & sample part file \\
    |cdocspt4.tex| & sample part file \\
    |cdocsdrf.tex| & sample redirection file \\
    |cdocsfn1.tex| & sample redirection file \\
    |cdocsfn2.tex| & sample redirection file \\
    |childdoc.pdf| & manual
\end{tabular}
\end{center}
%
The distribution consists of the files
|README.txt|, |childdoc.ins| and |childdoc.dtx|.
%
\begin{itemize}
\item
Run (pdf)\LaTeX{} on |childdoc.dtx|
to compile the manual |childdoc.pdf| (this file).
\item
Run \LaTeX{} on |childdoc.ins| to create the definitions file |childdoc.def|
and the sample |cdocsamp.tex| with include files
|cdocsch1.tex|, |cdocsch2.tex|, |cdocspt3.tex|, |cdocspt4.tex|,
|cdocsdrf.tex|, |cdocsfn1.tex|, |cdocsfn2.tex|.
Then copy the file |childdoc.def| to an appropriate directory of your \LaTeX{}
distribution, e.g.\ \textit{texmf-root}|/tex/latex/childdoc|.
\end{itemize}

%%%%%%%%%%%%%%%%%%%%%%%%%%%%%%%%%%%%%%%%%%%%%%%%%%%%%%%%%%%%%%%%%%%%%%%%%%%%%%%%
\subsection{Related CTAN Packages}

There are several other packages which offer a similar functionality:
%
\begin{itemize}
\item
The packages
\href{http://ctan.org/pkg/docmute}{\textsf{docmute}},
\href{http://ctan.org/pkg/includex}{\textsf{includex}} and
\href{http://ctan.org/pkg/standalone}{\textsf{standalone}}
provide commands to include only the document body of
a child file thus allowing both files to be compiled individually.
\item
The packages \href{http://ctan.org/pkg/subdocs}{\textsf{subdocs}}
and \href{http://ctan.org/pkg/subfiles}{\textsf{subfiles}}
provide structures in which the main and child documents can be
encapsulated and allowing them to be compiled individually.
The inclusion mechanism is different from the conventional |\include|.
\item
The package \href{http://ctan.org/pkg/combine}{\textsf{combine}}
is an elaborate solution to combine several documents into one.
\end{itemize}
%
See also the CTAN topic \href{http://ctan.org/topic/subdocs}{\textsf{subdocs}}
for further related packages.
The present package differs from the above solutions in that
a document structure constructed with the conventional |\include| mechanism
just needs two extra commands at the top of every file
such that all constituent files can be compiled individually.

%%%%%%%%%%%%%%%%%%%%%%%%%%%%%%%%%%%%%%%%%%%%%%%%%%%%%%%%%%%%%%%%%%%%%%%%%%%%%%%%
%\subsection{Feature Suggestions}
%
%The following is a list of features which may be useful for future
%versions of this package:
%%
%\begin{itemize}
%\item
%\ldots
%\end{itemize}

%%%%%%%%%%%%%%%%%%%%%%%%%%%%%%%%%%%%%%%%%%%%%%%%%%%%%%%%%%%%%%%%%%%%%%%%%%%%%%%%
\subsection{Revision History}

%%%%%%%%%%%%%%%%%%%%%%%%%%%%%%%%%%%%%%%%
\paragraph{v2.0:} 2018/12/30

\begin{itemize}
\item
immediate forward processing
\item
added |\childdocby| mechanism
\item
manual restructured
\end{itemize}

%%%%%%%%%%%%%%%%%%%%%%%%%%%%%%%%%%%%%%%%
\paragraph{v1.6:} 2018/01/17

\begin{itemize}
\item
application for development of include files
\item
corrections to manual
\end{itemize}

%%%%%%%%%%%%%%%%%%%%%%%%%%%%%%%%%%%%%%%%
\paragraph{v1.5:} 2017/05/21

\begin{itemize}
\item
more complete structuring introduced
\item
|\childdocof| introduced
\item
|\childdoc| renamed to |\childdocmain|
\item
|\childredirect| renamed to |\childdocforward| and |\childdocforwardprefix|
and functionality expanded
\end{itemize}

%%%%%%%%%%%%%%%%%%%%%%%%%%%%%%%%%%%%%%%%
\paragraph{v1.0:} 2017/04/27

\begin{itemize}
\item
manual and install package
\item
first version published on CTAN
\end{itemize}

%%%%%%%%%%%%%%%%%%%%%%%%%%%%%%%%%%%%%%%%
\paragraph{v0.6:} 2017/04/26

\begin{itemize}
\item
redirection mechanism added
\end{itemize}

%%%%%%%%%%%%%%%%%%%%%%%%%%%%%%%%%%%%%%%%
\paragraph{v0.5:} 2017/04/26

\begin{itemize}
\item
functionality in definition file
\end{itemize}


%%%%%%%%%%%%%%%%%%%%%%%%%%%%%%%%%%%%%%%%%%%%%%%%%%%%%%%%%%%%%%%%%%%%%%%%%%%%%%%%
%%%%%%%%%%%%%%%%%%%%%%%%%%%%%%%%%%%%%%%%%%%%%%%%%%%%%%%%%%%%%%%%%%%%%%%%%%%%%%%%
%%%%%%%%%%%%%%%%%%%%%%%%%%%%%%%%%%%%%%%%%%%%%%%%%%%%%%%%%%%%%%%%%%%%%%%%%%%%%%%%
\appendix

\settowidth\MacroIndent{\rmfamily\scriptsize 000\ }

 \DocInput{childdoc.dtx}

\end{document}
%</driver>
% \fi
%
% %%%%%%%%%%%%%%%%%%%%%%%%%%%%%%%%%%%%%%%%%%%%%%%%%%%%%%%%%%%%%%%%%%%%%%%%%%%%%%
% %%%%%%%%%%%%%%%%%%%%%%%%%%%%%%%%%%%%%%%%%%%%%%%%%%%%%%%%%%%%%%%%%%%%%%%%%%%%%%
% \section{Sample}
%\iffalse
%<*samplemain>
%\fi
%
% The following presents a sample document
% with two chapters, two parts, a title page,
% a compile flag as well as three forwarding files to set the flag.
% It consists of eight |.tex| files:
% \begin{center}
% \begin{tabular}{ll}
% |cdocsamp.tex|&main file\\
% |cdocsch1.tex|&include file for chapter 1\\
% |cdocsch2.tex|&include file for chapter 2\\
% |cdocspt3.tex|&include file for part 3\\
% |cdocspt4.tex|&include file for part 4\\
% |cdocsdrf.tex|&forwarding file for main file in draft mode\\
% |cdocsfi1.tex|&forwarding file for final version of chapter 1\\
% |cdocsfi2.tex|&forwarding file for final version of chapter 2\\
% \end{tabular}
% \end{center}
% Each of the eight files can be compiled directly by the \LaTeX{} compiler.
%
% %%%%%%%%%%%%%%%%%%%%%%%%%%%%%%%%%%%%%%
% \paragraph{Main File.}
%
% The main file is called |cdocsamp.tex|.
%
% Load the \textsf{childdoc} definitions and
% declare the filename for the main document:
%    \begin{macrocode}
\input{childdoc.def}
\childdocmain{}
%    \end{macrocode}

% Optional override for |\version| flag:
%    \begin{macrocode}
%%\ifchilddoc\else\providecommand{\version}{draft}\fi
%    \end{macrocode}

% Define the default values for the |\version| flag
% (|final| for the main file and |draft| for childs):
%    \begin{macrocode}
\ifchilddoc
\providecommand{\version}{draft}
\else
\providecommand{\version}{final}
\fi
%    \end{macrocode}

% Load the standard document class:
%    \begin{macrocode}
\documentclass[12pt]{article}
%    \end{macrocode}

% Start the document body:
%    \begin{macrocode}
\begin{document}
%    \end{macrocode}

% Declare a title page.
% Print title, part of document being processed and version flag:
%    \begin{macrocode}
\addtocounter{page}{-1}
\begin{center}
{\LARGE\bfseries{}childdoc example\par}
\vspace{1cm}
\ifchilddoc
\ifchilddocmanual part\else chapter\fi:
`\childdocname' of `\childdocjob'\par
\else
main document: `\childdocjob'\par
\fi
version: \version\par
\end{center}
\newpage
%    \end{macrocode}

% Manually include selected file,
% otherwise process as usual:
%    \begin{macrocode}
\ifchilddocmanual
\section*{part `\childdocname'}
\input{\childdocname}
\else
%    \end{macrocode}

% Include the two chapters:
%    \begin{macrocode}
\include{cdocsch1}
\include{cdocsch2}
%    \end{macrocode}

% Include the two parts unless only chapters should be displayed:
%    \begin{macrocode}
\ifchilddoc\else
\section{part three}
\input{cdocspt3}
\section{part four}
\input{cdocspt4}
\fi
%    \end{macrocode}

% Process as usual until here:
%    \begin{macrocode}
\fi
%    \end{macrocode}

% End of document body:
%    \begin{macrocode}
\end{document}
%    \end{macrocode}
%\iffalse
%</samplemain>
%\fi
%
% %%%%%%%%%%%%%%%%%%%%%%%%%%%%%%%%%%%%%%
% \paragraph{Chapter Include Files.}
%
% The include files are called |cdocsch1.tex| and |cdocsch2.tex|.
%
%\iffalse
%<*samplechap1|samplechap2>
%\fi

% Optional override for |\version| flag:
%    \begin{macrocode}
%%\providecommand{\version}{final}
%    \end{macrocode}

% Include the main document:
%    \begin{macrocode}
\input{childdoc.def}
\childdocof{cdocsamp}
%    \end{macrocode}

%\iffalse
%</samplechap1|samplechap2>
%\fi
%
%\iffalse
%<*samplechap1>
%\fi
% Some text for chapter 1:
%    \begin{macrocode}
\section{one}
some text in chapter one
%    \end{macrocode}

%\iffalse
%</samplechap1>
%\fi
% Some text for chapter 2:
%\iffalse
%<*samplechap2>
%\fi
%    \begin{macrocode}
\section{two}
more text in chapter two
%    \end{macrocode}

%\iffalse
%</samplechap2>
%\fi
%
% %%%%%%%%%%%%%%%%%%%%%%%%%%%%%%%%%%%%%%
% \paragraph{Part Include Files.}
%
% The include files are called |cdocspt3.tex| and |cdocspt4.tex|.
%
%\iffalse
%<*samplepart3|samplepart4>
%\fi

% Optional override for |\version| flag:
%    \begin{macrocode}
%%\providecommand{\version}{final}
%    \end{macrocode}

% Include the main document:
%    \begin{macrocode}
\input{childdoc.def}
\childdocby{cdocsamp}
%    \end{macrocode}

%\iffalse
%</samplepart3|samplepart4>
%\fi
%
%\iffalse
%<*samplepart3>
%\fi
% Some text for part 3:
%    \begin{macrocode}
some text in part three
%    \end{macrocode}

%\iffalse
%</samplepart3>
%\fi
% Some text for part 4:
%\iffalse
%<*samplepart4>
%\fi
%    \begin{macrocode}
more text in part four
%    \end{macrocode}

%\iffalse
%</samplepart4>
%\fi
%
% %%%%%%%%%%%%%%%%%%%%%%%%%%%%%%%%%%%%%%
% \paragraph{Forwarding for a Complete Draft.}
%
% The following forwarding file |cdocsdrf.tex|
% compiles the main document in draft mode:
%\iffalse
%<*sampledraft>
%\fi
%    \begin{macrocode}
\def\version{draft}
\input{childdoc.def}
\childdocforward{cdocsamp}
%    \end{macrocode}

%\iffalse
%</sampledraft>
%\fi
%
% %%%%%%%%%%%%%%%%%%%%%%%%%%%%%%%%%%%%%%
% \paragraph{Forwarding for Final Version of the Chapters.}
%
% The following forwarding files |cdocsfn1.tex| and |cdocsfn2.tex|
% (with identical content)
% compile the final versions of the child documents
% |cdocsch1.tex| and |cdocsch2.tex|, respectively:
%\iffalse
%<*samplefinal>
%\fi
%    \begin{macrocode}
\def\version{final}
\input{childdoc.def}
\childdocforwardprefix[cdocsamp]{cdocsfn}{cdocsch}
%    \end{macrocode}

%\iffalse
%</samplefinal>
%\fi
%
% %%%%%%%%%%%%%%%%%%%%%%%%%%%%%%%%%%%%%%
% \paragraph{Command Line Processing.}
%
% The following three command lines generate the output files
% |cdocscld|, |cdocscl1| and |cdocscl2|
% which should be identical to
% |cdocsdrf|, |cdocsch1| and |cdocsfn2|, respectively:
% \begin{center}
% \begin{tabular}{l}
% |latex -jobname cdocscld \|\\
% |  "\def\version{draft}\input{childdoc.def}\childdocforward{cdocsamp}"|\\
% |latex -jobname cdocscl1 \|\\
% |  "\input{childdoc.def}\childdocforward[cdocsamp]{cdocsch1}"|\\
% |latex -jobname cdocscl2 \|\\
% |  "\def\version{final}\input{childdoc.def}\childdocforward{cdocsch2}"|
% \end{tabular}
% \end{center}
% Note that the trailing backslash on each first line
% merely continues the input to the second line
% (for convenient cut ant paste).
% Furthermore, the command |latex| can be replaced by any
% of its alternative versions such as |pdflatex|.
%
% %%%%%%%%%%%%%%%%%%%%%%%%%%%%%%%%%%%%%%%%%%%%%%%%%%%%%%%%%%%%%%%%%%%%%%%%%%%%%%
% %%%%%%%%%%%%%%%%%%%%%%%%%%%%%%%%%%%%%%%%%%%%%%%%%%%%%%%%%%%%%%%%%%%%%%%%%%%%%%
% \section{Implementation}
%\iffalse
%<*package>
%\fi
%
% This section describes the definitions file |childdoc.def|.

% The definitions cannot be loaded using |\usepackage| or |\RequirePackage|
% which has a mechanism to prevent loading a style file more than once.
% When loading the definitions by means of |\input|
% multiple instances have to be prevented manually:
%\iffalse
%This code needs to be before the `\ProvidesFile' directive
%which is defined at the beginning of this file.
%Therefore it is also placed there and commented out here.
%</package>
%<*discard>
%\fi
%    \begin{macrocode}
\ifdefined\childdocmain\endinput\fi
%    \end{macrocode}
%\iffalse
%</discard>
%<*package>
%\fi
%
% \macro{\ifchilddoc}
% \macro{\ifchilddocmanual}
% The conditional |\ifchilddoc| tells whether a
% child (true) or main (false) document is being compiled.
% The conditional |\ifchilddocmanual| tells whether
% the |\includeonly| mechanism is used (false) or
% the selection of child files must be performed manually (true).
% The definitions initialise to false:
%    \begin{macrocode}
\newif\ifchilddoc
\newif\ifchilddocmanual
%    \end{macrocode}

% \macro{\childdocname}
% \macro{\childdocjob}
% The macro |\childdocname| stores the name of the main document
% to be compiled. The macro |\childdocjob| stores the name of
% the document on which the \LaTeX{} compiler was originally invoked.
% The content of |\jobname| cannot be compared
% to filenames specified in the source due to different catcodes.
% The following code rescans |\jobname|, stores the result
% in |\childdocname| and saves a copy in |\childdocjob|:
%    \begin{macrocode}
\edef\childdocname{\scantokens\expandafter{\jobname\noexpand}}
\let\childdocjob\childdocname
%    \end{macrocode}

% \macro{\childdocdisable}
% The macro |\childdocdisable| prevents the main file
% from being processed more than once.
% At this stage, the main document command |\childdocmain|
% is assumed to be called once again where it should do nothing.
% Any subsequent call to it should prevent
% a secondary processing of the main document
% It overwrites the forwarding commands
% |\childdocof| and |\childdocforward|
% with empty macros to prevent further inclusions of the main document:
%    \begin{macrocode}
\newcommand{\childdocdisable}
{
  \renewcommand{\childdocmain}[1]{\renewcommand{\childdocmain}[1]{\endinput}}
  \renewcommand{\childdocof}[1]{}
  \renewcommand{\childdocby}[2][]{}
  \renewcommand{\childdocforward}[2][]{}
  \renewcommand{\childdocdisable}{}
}
%    \end{macrocode}

% \macro{\childdocmain}
% The macro |\childdocmain| is to be called at the top of the main file
% with nothing or the main filename (without extension) as argument.
% First, it breaks loops.
% If the argument is not empty and does not match |\childdocname|
% (which is set by the first inclusion of |childdoc.def|),
% |\ifchilddoc| is set to true, |\includeonly| is applied to the child file
% and |\jobname| is set to the main file
% (for proper handling of |.aux| files):
%    \begin{macrocode}
\newcommand{\childdocmain}[1]
{
  \childdocdisable\childdocmain{}
  \if?#1?\else
    \begingroup
      \def\childdoctmp{#1}
      \ifx\childdoctmp\childdocname
        \def\childdoctmp{}
      \else
        \def\childdoctmp
        {
          \childdoctrue
          \includeonly{\childdocname}
          \def\childdocjob{#1}
          \def\jobname{#1}
        }
      \fi
      \expandafter
    \endgroup
    \childdoctmp
  \fi
}
%    \end{macrocode}

% \macro{\childdocof}
% The command |\childdocof| redirects
% compilation to the main file |#1|.
%    \begin{macrocode}
\newcommand{\childdocof}[1]
{
  \childdocdisable
  \childdoctrue
  \includeonly{\childdocname}
  \def\jobname{#1}
  \def\childdocjob{#1}
  \input{#1}
}
%    \end{macrocode}

% \macro{\childdocby}
% The command |\childdocby| ....
%    \begin{macrocode}
\newcommand{\childdocby}[2][]
{
  \childdocdisable
  \childdoctrue
  \childdocmanualtrue
  \if?#1?\else
    \def\jobname{#2}
  \fi
  \def\childdocjob{#2}
  \input{#2}
  \endinput
}
%    \end{macrocode}

% \macro{\childdocforward}
% The command |\childdocforward| redirects
% compilation to the main file or
% (if the optional argument is given) a child file.
% Parameters are set as if the main file
% or a child file starting with |\childdocof| was compiled.
% Then compilation is handed over to the main file:
%    \begin{macrocode}
\newcommand{\childdocforward}[2][]
{
  \begingroup
    \if?#1?
      \def\childdoctmp
      {
        \def\childdocname{#2}
        \def\childdocjob{#2}
        \def\jobname{#2}
        \input{#2}
        \endinput
      }
    \else
      \def\childdoctmp
      {
        \childdocdisable
        \def\childdocname{#2}
        \childdoctrue
        \includeonly{#2}
        \def\childdocjob{#1}
        \def\jobname{#1}
        \input{#1}
        \endinput
      }
    \fi
    \expandafter
  \endgroup
  \childdoctmp
}
%    \end{macrocode}

% \macro{\childdocforwardprefix}
% The command |\childdocforwardprefix| redirects
% compilation to the main or a child file by means of a pattern.
% The prefix |#1| in the current filename is replaced by |#2|
% and the suffix of the current filename is kept
% (it is assumed that the filename does not contain the substring `|~~~|'
% which is used as a delimiter).
% Compilation is handed over to the new file by |\childdocforward|:
%    \begin{macrocode}
\newcommand{\childdocforwardprefix}[3][]
{
  \begingroup
    \def\childdocextract #2##1~~~{\def\childdoctmp{\childdocforward[#1]{#3##1}}}
    \expandafter\childdocextract\childdocname~~~
    \expandafter
  \endgroup
  \childdoctmp
}
%    \end{macrocode}

% \macro{\childdoc}
% The deprecated macro |\childdoc| is a legacy version of |\childdocmain|:
%    \begin{macrocode}
\newcommand{\childdoc}{\childdocmain}
%    \end{macrocode}

% \macro{\childdocredirect}
% The deprecated macro |\childdocredirect| is a legacy version
% of |\childdocforward| and |\childdocforwardprefix|:
%    \begin{macrocode}
\newcommand{\childdocredirect}[2][]
{
  \begingroup
    \if?#1?
      \def\childdoctmp{\childdocforward{#2}}
    \else
      \def\childdoctmp{\childdocforwardprefix{#1}{#2}}
    \fi
    \expandafter
  \endgroup
  \childdoctmp
}
%    \end{macrocode}

%\iffalse
%</package>
%\fi
%
\endinput
|\\
|\childdocforwardprefix{final}{child}|
\end{tabular}
\end{center}
%

Note that when several versions of a main file and/or of each child file
are to be generated, it may be convenient to set up a |Makefile| or
shell script to automatise the process.

%%%%%%%%%%%%%%%%%%%%%%%%%%%%%%%%%%%%%%%%%%%%%%%%%%%%%%%%%%%%%%%%%%%%%%%%%%%%%%%%
\subsection{Command Line Processing}
\label{sec:commandline}

The effect of redirection files can also be achieved by invoking
the \LaTeX{} compiler with a more elaborate command line.
Most conveniently this should be done as part
of a shell script or a |Makefile|.

When using \textsf{childdoc} in the main file, the following
command lines effectively perform a redirection
(note that depending on the shell being used,
backslashes may have to be doubled: `|\|' $\to$ `|\\|'):
%
\begin{center}
|... -jobname "|\textit{target}|" |\\|"|[\textit{flags}]%
|% \iffalse
%
% childdoc.dtx Copyright (C) 2017-2018 Niklas Beisert
%
% This work may be distributed and/or modified under the
% conditions of the LaTeX Project Public License, either version 1.3
% of this license or (at your option) any later version.
% The latest version of this license is in
%   http://www.latex-project.org/lppl.txt
% and version 1.3 or later is part of all distributions of LaTeX
% version 2005/12/01 or later.
%
% This work has the LPPL maintenance status `maintained'.
%
% The Current Maintainer of this work is Niklas Beisert.
%
% This work consists of the files childdoc.dtx and childdoc.ins
% and the derived files childdoc.def and cdocsamp.tex with
% cdocsch1.tex, cdocsch2.tex, cdocsdrf.tex, cdocsfn1.tex, cdocsfn2.tex.
%
%<package>\ifdefined\childdocmain\endinput\fi
%<package>\ProvidesFile{childdoc.def}[2018/12/30 v2.0 child document driver]
%<samplemain>\ProvidesFile{cdocsamp.tex}[2018/12/30 v2.0 sample for childdoc]
%<*driver>
%\ProvidesFile{childdoc.drv}[2018/12/30 v2.0 childdoc reference manual file]
\PassOptionsToClass{10pt,a4paper}{article}
\documentclass{ltxdoc}

\usepackage[margin=35mm]{geometry}
\usepackage{hyperref}
\usepackage{hyperxmp}
\usepackage[usenames]{color}

\hypersetup{colorlinks=true}
\hypersetup{pdfstartview=FitH}
\hypersetup{pdfpagemode=UseNone}
\hypersetup{pdfsource={}}
\hypersetup{pdflang={en-UK}}
\hypersetup{pdfcopyright={Copyright 2017-2018 Niklas Beisert.
  This work may be distributed and/or modified under the
  conditions of the LaTeX Project Public License, either version 1.3
  of this license or (at your option) any later version.}}
\hypersetup{pdflicenseurl={http://www.latex-project.org/lppl.txt}}
\hypersetup{pdfcontactaddress={ETH Zurich, ITP, HIT K,
  Wolfgang-Pauli-Strasse 27}}
\hypersetup{pdfcontactpostcode={8093}}
\hypersetup{pdfcontactcity={Zurich}}
\hypersetup{pdfcontactcountry={Switzerland}}
\hypersetup{pdfcontactemail={nbeisert@itp.phys.ethz.ch}}
\hypersetup{pdfcontacturl={http://people.phys.ethz.ch/\xmptilde nbeisert/}}

\newcommand{\secref}[1]{\hyperref[#1]{section \ref*{#1}}}

\parskip1ex
\parindent0pt
\let\olditemize\itemize
\def\itemize{\olditemize\parskip0pt}

\begin{document}

\title{The \textsf{childdoc} Package}
\hypersetup{pdftitle={The childdoc Package}}
\author{Niklas Beisert\\[2ex]
  Institut f\"ur Theoretische Physik\\
  Eidgen\"ossische Technische Hochschule Z\"urich\\
  Wolfgang-Pauli-Strasse 27, 8093 Z\"urich, Switzerland\\[1ex]
  \href{mailto:nbeisert@itp.phys.ethz.ch}
  {\texttt{nbeisert@itp.phys.ethz.ch}}}
\hypersetup{pdfauthor={Niklas Beisert}}
\hypersetup{pdfsubject={Manual for the LaTeX2e Package childdoc}}
\date{30 December 2018, \textsf{v2.0}}
\maketitle

\begin{abstract}\noindent
\textsf{childdoc} is a \LaTeXe{} package
that enables the direct compilation
of document sections included by |\include|
to individual files.
\end{abstract}

\begingroup
\parskip0ex
\tableofcontents
\endgroup

%%%%%%%%%%%%%%%%%%%%%%%%%%%%%%%%%%%%%%%%%%%%%%%%%%%%%%%%%%%%%%%%%%%%%%%%%%%%%%%%
%%%%%%%%%%%%%%%%%%%%%%%%%%%%%%%%%%%%%%%%%%%%%%%%%%%%%%%%%%%%%%%%%%%%%%%%%%%%%%%%
\section{Introduction}

\LaTeX{} provides a mechanism to structure a large document (such as a book)
into a main file and several child files (containing the chapters)
using the |\include| command.
This mechanism is beneficial for documents
which span hundreds of pages in order to
make the source file(s) more manageable.
Moreover, compilation can be restricted to
selected child files by means of the |\includeonly| command.
The latter feature can be used to reduce the compilation time while editing
(this was significantly more useful in the earlier days of \LaTeX{})
or to generate a smaller document which is easier to navigate.
Another application of |\includeonly| is to generate
documents consisting of selected parts of the complete document.

However, there are a few drawbacks of the plain |\include| mechanism:
\begin{itemize}
\item
The child files cannot be compiled on their own,
they can only be compiled via the main file.
A naive editing environment
(such as a text editor with an option
to have the current file processed by \LaTeX)
may require one to switch to the main file before compiling;
attempting to compile the child file produces errors.
\item
The main file must be modified (each time)
to adjust the |\includeonly| command
to the present needs. This easily leaves the main file in a messy state.
\item
The generated document will always carry the filename
of the main document. This is inconvenient if
several child files are to be compiled and
to be kept for distribution.
\end{itemize}

The present package provides a simple interface
to make child files individually compilable by \LaTeX{}.
Compiling a child file then has the same effect as compiling
the main file with an |\includeonly| command
to select the appropriate child.
Moreover the generated document will carry the name of the child
rather than the main file.
This resolves all three above issues.

This feature is meant to make the editing of books,
thesis documents and lecture notes somewhat more convenient.
However, the package can also be used efficiently for
composing a series of documents (such as exercise sheets)
which are typically distributed individually.
It then assists the author in generating the individual documents
(potentially in different versions)
as well as a document containing the collected series.
Another application is in developing style files
or other kinds of included material
where compilation of the style file could redirect
to a sample or test file.

%%%%%%%%%%%%%%%%%%%%%%%%%%%%%%%%%%%%%%%%%%%%%%%%%%%%%%%%%%%%%%%%%%%%%%%%%%%%%%%%
%%%%%%%%%%%%%%%%%%%%%%%%%%%%%%%%%%%%%%%%%%%%%%%%%%%%%%%%%%%%%%%%%%%%%%%%%%%%%%%%
\section{Usage}

First of all, the package \textsf{childdoc} is \emph{not} a standard
\LaTeXe{} |.sty| style file! Therefore it needs to be invoked in
a non-standard way.

%%%%%%%%%%%%%%%%%%%%%%%%%%%%%%%%%%%%%%%%%%%%%%%%%%%%%%%%%%%%%%%%%%%%%%%%%%%%%%%%
\subsection{Included Files}
\label{sec:include}

%%%%%%%%%%%%%%%%%%%%%%%%%%%%%%%%%%%%%%%%
\DescribeMacro{\childdocmain}
To use the package, add the commands
\begin{center}
\begin{tabular}{l}
|\input{childdoc.def}|\\
|\childdocmain{}|\\
\end{tabular}
\end{center}
at the very top of the main \LaTeX{} file,
in particular \emph{before} the |\documentclass| statement!
The argument of |\childdocmain| should be left empty
(but it must be present).

%%%%%%%%%%%%%%%%%%%%%%%%%%%%%%%%%%%%%%%%
\DescribeMacro{\childdocof}
Furthermore, add the commands
\begin{center}
\begin{tabular}{l}
|\input{childdoc.def}|\\
|\childdocof{|\textit{main}|}|\\
\end{tabular}
\end{center}
at the top of every child file \textit{child}
which is included by |\include{|\textit{child}|}|
from within the main file
(or at least for those files to be compiled individually).
The argument \textit{main} must be the filename of the main file.

There are a couple of
considerations in setting up the main and child documents:

%%%%%%%%%%%%%%%%%%%%%%%%%%%%%%%%%%%%%%%%
\paragraph{Restrictions.}

Please note the following restrictions:
\begin{itemize}
\item
|\childdocmain| must be called with one argument \textit{main}
to ensure compatibility with earlier version of the package.
It must either be empty (|\childdocmain{}|)
or precisely match the filename of the main file in which it is specified.
See \secref{sec:detection} for further information.
\item
The filename \textit{main} must be specified without the |.tex| extension.
\item
The filename \textit{main} is case sensitive
(even in case-insensitive file systems)
due to internal string comparison.
\item
The argument \textit{main} should be fully expanded, it cannot be a macro.
\item
Subdirectories and special characters should be avoided in filenames.
\item
The command |\childdocmain{|\textit{main}|}| must be followed by a whitespace.
It should not be followed immediately by another command
or by a comment mark `|%|'.
This is because the \TeX{} parser reads the token immediately following
the argument of |\childdocmain| and puts it
at the beginning of every child section;
however, a white\-space is ignored.
\end{itemize}

%%%%%%%%%%%%%%%%%%%%%%%%%%%%%%%%%%%%%%%%
\paragraph{Content of Main File.}

It is advisable to place all content in the child files included by |\include|.
Any output contained in the main file will appear in all child documents
unless suppressed manually;
it cannot be suppressed automatically by the |\includeonly| directive
and thus should normally be avoided.
A method to include some content in the main file
by means of conditional processing is described in \secref{sec:conditional}.

%%%%%%%%%%%%%%%%%%%%%%%%%%%%%%%%%%%%%%%%
\paragraph{Page Numbering.}

When only a part of the document is compiled,
the appropriate numbering of pages
(as well as other status parameters)
is determined from the |.aux| files.
The latter contain information from previous passes.
However this information needs to propagate through
all intermediate child documents.
Therefore the page numbering in child documents may well
be inconsistent until the complete document is compiled at least once.

A useful (if unconventional) way to always ensure a consistent
page numbering is to restart the numbering in each child document
and denote the pages by `\textit{child}|.|\textit{page}'
where \textit{child} represents the chapter/section number of the child file.
This can be achieved by the command
|\numberwithin{page}{|\textit{child}|}|
of the \textsf{amsmath} package
where \textit{child} can be |chapter| or |section|
depending on the chosen structuring.
Alternatively, one can modify the macro |\thepage| appropriately
and reset the counter |page| at the start of each child file.

%%%%%%%%%%%%%%%%%%%%%%%%%%%%%%%%%%%%%%%%%%%%%%%%%%%%%%%%%%%%%%%%%%%%%%%%%%%%%%%%
\subsection{Conditional Processing}
\label{sec:conditional}

The package provides a mechanism to compile different versions
of a document. To customise the versions further some conditional processing
can come in handy to distinguish which version is being compiled.
The package provides two macros to describe the compilation context:

%%%%%%%%%%%%%%%%%%%%%%%%%%%%%%%%%%%%%%%%
\DescribeMacro{\ifchilddoc}
The conditional |\ifchilddoc| distinguishes between the compilation of
child documents and the main document:
%
\begin{center}
|\ifchilddoc |\textit{child-code}| |[|\||else |\textit{main-code}]| \||fi|
\end{center}

%%%%%%%%%%%%%%%%%%%%%%%%%%%%%%%%%%%%%%%%
\DescribeMacro{\childdocname}
\DescribeMacro{\childdocjob}
The macro |\childdocname| contains the filename (without extension)
of the main or child file being processed.
Note that |\childdocjob| will always contain the name of the main file.

%%%%%%%%%%%%%%%%%%%%%%%%%%%%%%%%%%%%%%%%
\paragraph{Title Page.}

Conditional processing can be used to include a title or banner page
in the main document when proper precautions are taken.
Importantly, the code in the main file should ensure that the page counter
(as well as other status parameters which are stored in the |.aux| files)
takes the same value after the conditional processing.
Otherwise the page numbers may take divergent values
depending on which part is compiled.

For example, a title page could be declared by:
%
\begin{center}
\begin{tabular}{l}
|\ifchilddoc\||else|\\
|\addtocounter{page}{-1}|\\
\textit{code for title page}\\
|\newpage|\\
|\||fi|
\end{tabular}
\end{center}
%
A banner page for the child documents can be generated by:
%
\begin{center}
\begin{tabular}{l}
|\ifchilddoc|\\
|\addtocounter{page}{-1}|\\
\textit{code for banner page}\\
|\newpage|\\
|\||fi|
\end{tabular}
\end{center}
%
Here one could write a message such as:
\begin{center}
|This is the part \childdocname{} of \childdocjob{}.|
\end{center}

%%%%%%%%%%%%%%%%%%%%%%%%%%%%%%%%%%%%%%%%%%%%%%%%%%%%%%%%%%%%%%%%%%%%%%%%%%%%%%%%
\subsection{Flags}
\label{sec:flags}

The package makes it easy to generate different versions
of the main or child documents.
To this end compilation flags can be defined
and assigned different default values.
They will be particularly useful in conjunction
with the forwarding mechanism described in \secref{sec:forward}.

For example, it may be useful to have a flag |\version|
which can be set to |draft| or |final|.
The document source will contain some conditional code
depending on the value of |\version|.
Suppose further, the flag should default to |final| for the main file
and to |draft| for child files
which is a natural assignment for editing the document.
This is achieved by placing the following code
in the preamble of the main document
(below the |\childdocmain| directive):
%
\begin{center}
\begin{tabular}{l}
|\ifchilddoc|\\
|\providecommand{\version}{draft}|\\
|\||else|\\
|\providecommand{\version}{final}|\\
|\||fi|
\end{tabular}
\end{center}
%
The definition by |\providecommand| makes sure
that previous definitions are not overwritten.
Further statements |\providecommand{\version}{...}|
can thus be added before the above code to override it.

For the main file, one might add a line
(between |\childdocmain| and the above block)
%
\begin{center}
|%\ifchilddoc\||else\providecommand{\version}{draft}\||fi|
\end{center}
%
which can be uncommented to produce a draft version.
Likewise one can add a line to the very top of a child file
(above the |\childdocof{|\textit{main}|}| directive)
%
\begin{center}
|%\providecommand{\version}{final}|
\end{center}
%
which can be uncommented to produce the final version of this child document.

%%%%%%%%%%%%%%%%%%%%%%%%%%%%%%%%%%%%%%%%%%%%%%%%%%%%%%%%%%%%%%%%%%%%%%%%%%%%%%%%
\subsection{Forwarding}
\label{sec:forward}

Different versions of the main or child documents
using compilation flags as described in \secref{sec:flags}
can be (permanently) stored in different files
for convenient compilation, viewing and distribution.
To this end, the package defines a command
to pass on compilation to a different file:

%%%%%%%%%%%%%%%%%%%%%%%%%%%%%%%%%%%%%%%%
\DescribeMacro{\childdocforward}
The command |\childdocforward| redirects processing to
another source file:
%
\begin{center}
\begin{tabular}{l}
|\input{childdoc.def}|\\
|\childdocforward[|\textit{main}|]{|\textit{dest}|}|\\
\end{tabular}
\end{center}
%
The argument \textit{dest} is the destination file
(without extension).
It should be the main file or one of the child files.
Note that further \textsf{childdoc} directives
such as |\childdocof| and |\childdocforward|
in the indicated file will be processed in this form.
The optional argument \textit{main}
passes on directly to the main file \textit{main}
while pretending to compile the child \textit{dest}.
This form behaves as if \textit{dest}
issues |\childdocof{|\textit{main}|}| right away,
and no further \textsf{childdoc} directives will be processed.

%%%%%%%%%%%%%%%%%%%%%%%%%%%%%%%%%%%%%%%%
\DescribeMacro{\...prefix}
In the alternative form |\childdocforwardprefix|,
%
\begin{center}
\begin{tabular}{l}
|\input{childdoc.def}|\\
|\childdocforwardprefix[|\textit{main}|]{|\textit{prefix}|}{|\textit{dest}|}|
\end{tabular}
\end{center}
%
the destination file is determined by a pattern
depending on the current file:
To make this work, the current file must be called
`{\textit{prefix}\hspace{0.2em}\textit{suffix}}'
with \textit{prefix} matching precisely the argument.
Processing is then passed on to the file
`{\textit{dest}\hspace{0.2em}\textit{suffix}}'.
Surely, the same effect is achieved by
directly specifying the
argument `{\textit{dest}\hspace{0.2em}\textit{suffix}}'
in the first form.
However, that requires to set up a different file
for each child. With the alternative form of the command
all these files can have exactly the same content
which simplifies setting them up and maintaining them.

For example, the following file |draft.tex|
with a compilation flag |\version| as described in \secref{sec:flags}
compiles the main document as a draft:
%
\begin{center}
\begin{tabular}{l}
|\def\version{draft}|\\
|\input{childdoc.def}|\\
|\childdocforward{|\textit{main}|}|
\end{tabular}
\end{center}
%
Likewise, the following files |final|\textit{nn}|.tex|
compile the final version of the child document
|child|\textit{nn}|.tex|:
%
\begin{center}
\begin{tabular}{l}
|\def\version{final}|\\
|\input{childdoc.def}|\\
|\childdocforwardprefix{final}{child}|
\end{tabular}
\end{center}
%

Note that when several versions of a main file and/or of each child file
are to be generated, it may be convenient to set up a |Makefile| or
shell script to automatise the process.

%%%%%%%%%%%%%%%%%%%%%%%%%%%%%%%%%%%%%%%%%%%%%%%%%%%%%%%%%%%%%%%%%%%%%%%%%%%%%%%%
\subsection{Command Line Processing}
\label{sec:commandline}

The effect of redirection files can also be achieved by invoking
the \LaTeX{} compiler with a more elaborate command line.
Most conveniently this should be done as part
of a shell script or a |Makefile|.

When using \textsf{childdoc} in the main file, the following
command lines effectively perform a redirection
(note that depending on the shell being used,
backslashes may have to be doubled: `|\|' $\to$ `|\\|'):
%
\begin{center}
|... -jobname "|\textit{target}|" |\\|"|[\textit{flags}]%
|\input{childdoc.def}\childdocforward[|\textit{main}|]{|\textit{dest}|}"|
\end{center}
%
Here \textit{target} is the name of the output file,
\textit{main} is the name of the main file
and \textit{dest} is the name of the main or child file to be processed
(all filenames without extensions).
The optional argument \textit{main} can be omitted
if \textit{main} matches \textit{dest}.
Optionally, compilation \textit{flags} can be defined via |\def| commands.
This command line makes the \TeX{} engine believe
it is compiling the file \textit{target}
whose content is specified as the latter parameter.
The provided code then forwards the processing to
\textit{main} or \textit{dest} as described in \secref{sec:forward}.

%%%%%%%%%%%%%%%%%%%%%%%%%%%%%%%%%%%%%%%%%%%%%%%%%%%%%%%%%%%%%%%%%%%%%%%%%%%%%%%%
\subsection{Include by Input}
\label{sec:input}

Including child documents by |\include| has some restrictions by design.
Most notably, the content of a child document always occupies
its own set of pages; pages cannot be shared between child documents.
Usually, this behaviour makes perfect sense
because each child document contain an essential part of the document.
However, in some situations it may be desirable to compose
a document from a collection of parts
without having mandatory page breaks between then.
For this case, the package
provides a mechanism to include parts
by |\input| which can also be processed individually.
However, by construction this mechanism
requires manual handling of the content to be output.

%%%%%%%%%%%%%%%%%%%%%%%%%%%%%%%%%%%%%%%%
\DescribeMacro{\ifchilddocmanual}
The main file should be prepared as usual, see \secref{sec:include}.
However, the document body must make a distinction
between processing of an individual part and of the main document, e.g.:
%
\begin{center}
\begin{tabular}{l}
|\ifchilddocmanual|\\
|\input{\childdocname}|\\
|\||else|\\
\textit{document body with }|\input{|\textit{part}|}|\\
|\||fi|
\end{tabular}
\end{center}
%
The conditional |\ifchilddocmanual| is true whenever
a part to be included by |\input| is being compiled,
and the name of the part is stored in |\childdocname|.

%%%%%%%%%%%%%%%%%%%%%%%%%%%%%%%%%%%%%%%%
\DescribeMacro{\childdocby}
Each part to be included by |\input| should start with:
%
\begin{center}
\begin{tabular}{l}
|\input{childdoc.def}|\\
|\childdocby{|\textit{main}|}|\\
\end{tabular}
\end{center}
%
The directive |\childdocby| is similar to |\childdocof|
described in \secref{sec:include},
but the subsequent selection of content must be done manually.
To that end, both |\ifchilddoc| and |\ifchilddocmanual|
will be true upon processing of a part,
and the name of the part is stored in |\childdocname|.
Note that |\jobname| will be set to the filename of the current part
so that each part receives an individual |.aux| file
that does not interfere with the |.aux| file(s) of the main document.
This behaviour can be altered by the alternative form
|\childdocby[*]{|\textit{main}|}| (with a non-empty optional argument)
which uses the |.aux| file of the main document
by setting |\jobname| to \textit{main}.

%%%%%%%%%%%%%%%%%%%%%%%%%%%%%%%%%%%%%%%%%%%%%%%%%%%%%%%%%%%%%%%%%%%%%%%%%%%%%%%%
\subsection{Driver Development}
\label{sec:driver}

The \textsf{childdoc} mechanism can also be use for the development
of definition files such as \LaTeX{} styles or classes.
This case differs from the above setup with multiple parts
included by |\include| in that no |\includeonly| should be invoked.
This can be achieved by starting the include file
(before |\ProvidesPackage|) with:
%
\begin{center}
\begin{tabular}{l}
|\input{childdoc.def}|\\
|\childdocforward{|\textit{main}|}|\\
\end{tabular}
\end{center}
%
or alternatively with:
%
\begin{center}
\begin{tabular}{l}
|\input{childdoc.def}|\\
|\childdocby{|\textit{main}|}|\\
\end{tabular}
\end{center}
%
Both forms have slightly different effects as described above.
The main file is prepared as usual, see \secref{sec:include}.

%%%%%%%%%%%%%%%%%%%%%%%%%%%%%%%%%%%%%%%%%%%%%%%%%%%%%%%%%%%%%%%%%%%%%%%%%%%%%%%%
\subsection{Legacy Detection}
\label{sec:detection}

The directive |\childdocmain| in the main file can detect
whether the complete document or merely a child is to be compiled
even without using the directive |\childdocof|.
This method is deprecated because it is less robust
and there is no compelling reason to use it;
it is merely provided for backward compatibility
and it may be removed in future versions.

If the detection mechanism is to be used,
it is mandatory to correctly specify
the filename of the main file as the argument of |\childdocmain|:
%
\begin{center}
\begin{tabular}{l}
|\input{childdoc.def}|\\
|\childdocmain{|\textit{main}|}|\\
\end{tabular}
\end{center}
%
If |\jobname| does not match the argument \textit{main} of |\childdocmain|,
it is assumed that |\jobname| points to the child file to be compiled.
When using |\childdocmain| with the main file specified as argument,
it suffices to start a child file
with just |\input{|\textit{main}|}|
without loading of the package and using |\childdocof|.
If instead all processing is done
with the appropriate \textsf{childdoc} directives,
the argument of \textit{main} of |\childdocmain| can be empty.

An alternative version of the command line processing described
in \secref{sec:commandline} using the detection mechanism reads:
%
\begin{center}
|... -jobname "|\textit{target}|" "|[\textit{flags}]%
[|\def\jobname{|\textit{dest}|}|]|\input{|\textit{main}|}"|
\end{center}

%%%%%%%%%%%%%%%%%%%%%%%%%%%%%%%%%%%%%%%%%%%%%%%%%%%%%%%%%%%%%%%%%%%%%%%%%%%%%%%%
\subsection{Manual Code}
\label{sec:manual}

In case one cannot be certain whether the definitions file |childdoc.def|
is installed on the target \TeX{} distribution
and one prefers not to ship it,
it is conceivable to paste a few relevant commands into the sources.

To that end, drop all statements |\input{childdoc.def}|
and perform the replacements as outlined below.
Instead of |\childdocmain{|\textit{main}|}| add the following code
to the top of the main file:
%
\begin{center}
\begin{tabular}{l}
|\||ifdefined\childdocname\endinput\||fi\newif\ifchilddoc|\\
|\edef\childdocname{\scantokens\expandafter{\jobname\noexpand}}|\\
|\def\childdocmain{|\textit{main}|}\||ifx\childdocmain\childdocname\||else|\\
|\childdoctrue\includeonly{\childdocname}\let\jobname\childdocmain\||fi|\\
\end{tabular}
\end{center}
%
Instead of |\childdocof{|\textit{main}|}| just include the main file
at the top of each child file:
%
\begin{center}
|\input{|\textit{main}|}|
\end{center}
%
A simple redirection |\childdocforward{|\textit{dest}|}| is achieved by:
%
\begin{center}
|\def\jobname{|\textit{dest}|}\input{\jobname}|
\end{center}
%
The redirection with prefix
|\childdocforwardprefix[|\textit{prefix}|]{|\textit{dest}|}|
is accomplished by:
%
\begin{center}
\begin{tabular}{l}
|{\edef\jobname{\scantokens\expandafter{\jobname\noexpand}}|\\
|\def\redirectjob |\textit{prefix}|#1~~~{\gdef\jobname{|\textit{dest}|#1}}|\\
|\expandafter\redirectjob\jobname~~~}\input{\jobname}|
\end{tabular}
\end{center}

In an alternative approach,
child documents can be compiled by a specific command line
without additional code or specific definitions:
%
\begin{center}
|... -jobname "|\textit{target}|" "|[\textit{flags}]%
|\includeonly{|\textit{dest}|}\input{|\textit{main}|}"|
\end{center}
%

%%%%%%%%%%%%%%%%%%%%%%%%%%%%%%%%%%%%%%%%%%%%%%%%%%%%%%%%%%%%%%%%%%%%%%%%%%%%%%%%
%%%%%%%%%%%%%%%%%%%%%%%%%%%%%%%%%%%%%%%%%%%%%%%%%%%%%%%%%%%%%%%%%%%%%%%%%%%%%%%%
\section{Information}

%%%%%%%%%%%%%%%%%%%%%%%%%%%%%%%%%%%%%%%%%%%%%%%%%%%%%%%%%%%%%%%%%%%%%%%%%%%%%%%%
\subsection{Copyright}

Copyright \copyright{} 2017--2018 Niklas Beisert

This work may be distributed and/or modified under the
conditions of the \LaTeX{} Project Public License, either version 1.3
of this license or (at your option) any later version.
The latest version of this license is in
  \url{http://www.latex-project.org/lppl.txt}
and version 1.3 or later is part of all distributions of \LaTeX{}
version 2005/12/01 or later.

This work has the LPPL maintenance status `maintained'.

The Current Maintainer of this work is Niklas Beisert.

This work consists of the files |README.txt|, |childdoc.ins| and |childdoc.dtx|
as well as the derived files |childdoc.def|, |cdocsamp.tex|
with |cdocsch1.tex|, |cdocsch2.tex|, |cdocspt3.tex|, |cdocspt4.tex|,
|cdocsdrf.tex|, |cdocsfn1.tex|, |cdocsfn2.tex|
as well as |childdoc.pdf|.

%%%%%%%%%%%%%%%%%%%%%%%%%%%%%%%%%%%%%%%%%%%%%%%%%%%%%%%%%%%%%%%%%%%%%%%%%%%%%%%%
\subsection{Files and Installation}

The package consists of the files:
%
\begin{center}
\begin{tabular}{ll}
    |README.txt|   & readme file \\
    |childdoc.ins| & installation file \\
    |childdoc.dtx| & source file \\
    |childdoc.def| & definition file \\
    |cdocsamp.tex| & sample main file \\
    |cdocsch1.tex| & sample include file \\
    |cdocsch2.tex| & sample include file \\
    |cdocspt3.tex| & sample part file \\
    |cdocspt4.tex| & sample part file \\
    |cdocsdrf.tex| & sample redirection file \\
    |cdocsfn1.tex| & sample redirection file \\
    |cdocsfn2.tex| & sample redirection file \\
    |childdoc.pdf| & manual
\end{tabular}
\end{center}
%
The distribution consists of the files
|README.txt|, |childdoc.ins| and |childdoc.dtx|.
%
\begin{itemize}
\item
Run (pdf)\LaTeX{} on |childdoc.dtx|
to compile the manual |childdoc.pdf| (this file).
\item
Run \LaTeX{} on |childdoc.ins| to create the definitions file |childdoc.def|
and the sample |cdocsamp.tex| with include files
|cdocsch1.tex|, |cdocsch2.tex|, |cdocspt3.tex|, |cdocspt4.tex|,
|cdocsdrf.tex|, |cdocsfn1.tex|, |cdocsfn2.tex|.
Then copy the file |childdoc.def| to an appropriate directory of your \LaTeX{}
distribution, e.g.\ \textit{texmf-root}|/tex/latex/childdoc|.
\end{itemize}

%%%%%%%%%%%%%%%%%%%%%%%%%%%%%%%%%%%%%%%%%%%%%%%%%%%%%%%%%%%%%%%%%%%%%%%%%%%%%%%%
\subsection{Related CTAN Packages}

There are several other packages which offer a similar functionality:
%
\begin{itemize}
\item
The packages
\href{http://ctan.org/pkg/docmute}{\textsf{docmute}},
\href{http://ctan.org/pkg/includex}{\textsf{includex}} and
\href{http://ctan.org/pkg/standalone}{\textsf{standalone}}
provide commands to include only the document body of
a child file thus allowing both files to be compiled individually.
\item
The packages \href{http://ctan.org/pkg/subdocs}{\textsf{subdocs}}
and \href{http://ctan.org/pkg/subfiles}{\textsf{subfiles}}
provide structures in which the main and child documents can be
encapsulated and allowing them to be compiled individually.
The inclusion mechanism is different from the conventional |\include|.
\item
The package \href{http://ctan.org/pkg/combine}{\textsf{combine}}
is an elaborate solution to combine several documents into one.
\end{itemize}
%
See also the CTAN topic \href{http://ctan.org/topic/subdocs}{\textsf{subdocs}}
for further related packages.
The present package differs from the above solutions in that
a document structure constructed with the conventional |\include| mechanism
just needs two extra commands at the top of every file
such that all constituent files can be compiled individually.

%%%%%%%%%%%%%%%%%%%%%%%%%%%%%%%%%%%%%%%%%%%%%%%%%%%%%%%%%%%%%%%%%%%%%%%%%%%%%%%%
%\subsection{Feature Suggestions}
%
%The following is a list of features which may be useful for future
%versions of this package:
%%
%\begin{itemize}
%\item
%\ldots
%\end{itemize}

%%%%%%%%%%%%%%%%%%%%%%%%%%%%%%%%%%%%%%%%%%%%%%%%%%%%%%%%%%%%%%%%%%%%%%%%%%%%%%%%
\subsection{Revision History}

%%%%%%%%%%%%%%%%%%%%%%%%%%%%%%%%%%%%%%%%
\paragraph{v2.0:} 2018/12/30

\begin{itemize}
\item
immediate forward processing
\item
added |\childdocby| mechanism
\item
manual restructured
\end{itemize}

%%%%%%%%%%%%%%%%%%%%%%%%%%%%%%%%%%%%%%%%
\paragraph{v1.6:} 2018/01/17

\begin{itemize}
\item
application for development of include files
\item
corrections to manual
\end{itemize}

%%%%%%%%%%%%%%%%%%%%%%%%%%%%%%%%%%%%%%%%
\paragraph{v1.5:} 2017/05/21

\begin{itemize}
\item
more complete structuring introduced
\item
|\childdocof| introduced
\item
|\childdoc| renamed to |\childdocmain|
\item
|\childredirect| renamed to |\childdocforward| and |\childdocforwardprefix|
and functionality expanded
\end{itemize}

%%%%%%%%%%%%%%%%%%%%%%%%%%%%%%%%%%%%%%%%
\paragraph{v1.0:} 2017/04/27

\begin{itemize}
\item
manual and install package
\item
first version published on CTAN
\end{itemize}

%%%%%%%%%%%%%%%%%%%%%%%%%%%%%%%%%%%%%%%%
\paragraph{v0.6:} 2017/04/26

\begin{itemize}
\item
redirection mechanism added
\end{itemize}

%%%%%%%%%%%%%%%%%%%%%%%%%%%%%%%%%%%%%%%%
\paragraph{v0.5:} 2017/04/26

\begin{itemize}
\item
functionality in definition file
\end{itemize}


%%%%%%%%%%%%%%%%%%%%%%%%%%%%%%%%%%%%%%%%%%%%%%%%%%%%%%%%%%%%%%%%%%%%%%%%%%%%%%%%
%%%%%%%%%%%%%%%%%%%%%%%%%%%%%%%%%%%%%%%%%%%%%%%%%%%%%%%%%%%%%%%%%%%%%%%%%%%%%%%%
%%%%%%%%%%%%%%%%%%%%%%%%%%%%%%%%%%%%%%%%%%%%%%%%%%%%%%%%%%%%%%%%%%%%%%%%%%%%%%%%
\appendix

\settowidth\MacroIndent{\rmfamily\scriptsize 000\ }

 \DocInput{childdoc.dtx}

\end{document}
%</driver>
% \fi
%
% %%%%%%%%%%%%%%%%%%%%%%%%%%%%%%%%%%%%%%%%%%%%%%%%%%%%%%%%%%%%%%%%%%%%%%%%%%%%%%
% %%%%%%%%%%%%%%%%%%%%%%%%%%%%%%%%%%%%%%%%%%%%%%%%%%%%%%%%%%%%%%%%%%%%%%%%%%%%%%
% \section{Sample}
%\iffalse
%<*samplemain>
%\fi
%
% The following presents a sample document
% with two chapters, two parts, a title page,
% a compile flag as well as three forwarding files to set the flag.
% It consists of eight |.tex| files:
% \begin{center}
% \begin{tabular}{ll}
% |cdocsamp.tex|&main file\\
% |cdocsch1.tex|&include file for chapter 1\\
% |cdocsch2.tex|&include file for chapter 2\\
% |cdocspt3.tex|&include file for part 3\\
% |cdocspt4.tex|&include file for part 4\\
% |cdocsdrf.tex|&forwarding file for main file in draft mode\\
% |cdocsfi1.tex|&forwarding file for final version of chapter 1\\
% |cdocsfi2.tex|&forwarding file for final version of chapter 2\\
% \end{tabular}
% \end{center}
% Each of the eight files can be compiled directly by the \LaTeX{} compiler.
%
% %%%%%%%%%%%%%%%%%%%%%%%%%%%%%%%%%%%%%%
% \paragraph{Main File.}
%
% The main file is called |cdocsamp.tex|.
%
% Load the \textsf{childdoc} definitions and
% declare the filename for the main document:
%    \begin{macrocode}
\input{childdoc.def}
\childdocmain{}
%    \end{macrocode}

% Optional override for |\version| flag:
%    \begin{macrocode}
%%\ifchilddoc\else\providecommand{\version}{draft}\fi
%    \end{macrocode}

% Define the default values for the |\version| flag
% (|final| for the main file and |draft| for childs):
%    \begin{macrocode}
\ifchilddoc
\providecommand{\version}{draft}
\else
\providecommand{\version}{final}
\fi
%    \end{macrocode}

% Load the standard document class:
%    \begin{macrocode}
\documentclass[12pt]{article}
%    \end{macrocode}

% Start the document body:
%    \begin{macrocode}
\begin{document}
%    \end{macrocode}

% Declare a title page.
% Print title, part of document being processed and version flag:
%    \begin{macrocode}
\addtocounter{page}{-1}
\begin{center}
{\LARGE\bfseries{}childdoc example\par}
\vspace{1cm}
\ifchilddoc
\ifchilddocmanual part\else chapter\fi:
`\childdocname' of `\childdocjob'\par
\else
main document: `\childdocjob'\par
\fi
version: \version\par
\end{center}
\newpage
%    \end{macrocode}

% Manually include selected file,
% otherwise process as usual:
%    \begin{macrocode}
\ifchilddocmanual
\section*{part `\childdocname'}
\input{\childdocname}
\else
%    \end{macrocode}

% Include the two chapters:
%    \begin{macrocode}
\include{cdocsch1}
\include{cdocsch2}
%    \end{macrocode}

% Include the two parts unless only chapters should be displayed:
%    \begin{macrocode}
\ifchilddoc\else
\section{part three}
\input{cdocspt3}
\section{part four}
\input{cdocspt4}
\fi
%    \end{macrocode}

% Process as usual until here:
%    \begin{macrocode}
\fi
%    \end{macrocode}

% End of document body:
%    \begin{macrocode}
\end{document}
%    \end{macrocode}
%\iffalse
%</samplemain>
%\fi
%
% %%%%%%%%%%%%%%%%%%%%%%%%%%%%%%%%%%%%%%
% \paragraph{Chapter Include Files.}
%
% The include files are called |cdocsch1.tex| and |cdocsch2.tex|.
%
%\iffalse
%<*samplechap1|samplechap2>
%\fi

% Optional override for |\version| flag:
%    \begin{macrocode}
%%\providecommand{\version}{final}
%    \end{macrocode}

% Include the main document:
%    \begin{macrocode}
\input{childdoc.def}
\childdocof{cdocsamp}
%    \end{macrocode}

%\iffalse
%</samplechap1|samplechap2>
%\fi
%
%\iffalse
%<*samplechap1>
%\fi
% Some text for chapter 1:
%    \begin{macrocode}
\section{one}
some text in chapter one
%    \end{macrocode}

%\iffalse
%</samplechap1>
%\fi
% Some text for chapter 2:
%\iffalse
%<*samplechap2>
%\fi
%    \begin{macrocode}
\section{two}
more text in chapter two
%    \end{macrocode}

%\iffalse
%</samplechap2>
%\fi
%
% %%%%%%%%%%%%%%%%%%%%%%%%%%%%%%%%%%%%%%
% \paragraph{Part Include Files.}
%
% The include files are called |cdocspt3.tex| and |cdocspt4.tex|.
%
%\iffalse
%<*samplepart3|samplepart4>
%\fi

% Optional override for |\version| flag:
%    \begin{macrocode}
%%\providecommand{\version}{final}
%    \end{macrocode}

% Include the main document:
%    \begin{macrocode}
\input{childdoc.def}
\childdocby{cdocsamp}
%    \end{macrocode}

%\iffalse
%</samplepart3|samplepart4>
%\fi
%
%\iffalse
%<*samplepart3>
%\fi
% Some text for part 3:
%    \begin{macrocode}
some text in part three
%    \end{macrocode}

%\iffalse
%</samplepart3>
%\fi
% Some text for part 4:
%\iffalse
%<*samplepart4>
%\fi
%    \begin{macrocode}
more text in part four
%    \end{macrocode}

%\iffalse
%</samplepart4>
%\fi
%
% %%%%%%%%%%%%%%%%%%%%%%%%%%%%%%%%%%%%%%
% \paragraph{Forwarding for a Complete Draft.}
%
% The following forwarding file |cdocsdrf.tex|
% compiles the main document in draft mode:
%\iffalse
%<*sampledraft>
%\fi
%    \begin{macrocode}
\def\version{draft}
\input{childdoc.def}
\childdocforward{cdocsamp}
%    \end{macrocode}

%\iffalse
%</sampledraft>
%\fi
%
% %%%%%%%%%%%%%%%%%%%%%%%%%%%%%%%%%%%%%%
% \paragraph{Forwarding for Final Version of the Chapters.}
%
% The following forwarding files |cdocsfn1.tex| and |cdocsfn2.tex|
% (with identical content)
% compile the final versions of the child documents
% |cdocsch1.tex| and |cdocsch2.tex|, respectively:
%\iffalse
%<*samplefinal>
%\fi
%    \begin{macrocode}
\def\version{final}
\input{childdoc.def}
\childdocforwardprefix[cdocsamp]{cdocsfn}{cdocsch}
%    \end{macrocode}

%\iffalse
%</samplefinal>
%\fi
%
% %%%%%%%%%%%%%%%%%%%%%%%%%%%%%%%%%%%%%%
% \paragraph{Command Line Processing.}
%
% The following three command lines generate the output files
% |cdocscld|, |cdocscl1| and |cdocscl2|
% which should be identical to
% |cdocsdrf|, |cdocsch1| and |cdocsfn2|, respectively:
% \begin{center}
% \begin{tabular}{l}
% |latex -jobname cdocscld \|\\
% |  "\def\version{draft}\input{childdoc.def}\childdocforward{cdocsamp}"|\\
% |latex -jobname cdocscl1 \|\\
% |  "\input{childdoc.def}\childdocforward[cdocsamp]{cdocsch1}"|\\
% |latex -jobname cdocscl2 \|\\
% |  "\def\version{final}\input{childdoc.def}\childdocforward{cdocsch2}"|
% \end{tabular}
% \end{center}
% Note that the trailing backslash on each first line
% merely continues the input to the second line
% (for convenient cut ant paste).
% Furthermore, the command |latex| can be replaced by any
% of its alternative versions such as |pdflatex|.
%
% %%%%%%%%%%%%%%%%%%%%%%%%%%%%%%%%%%%%%%%%%%%%%%%%%%%%%%%%%%%%%%%%%%%%%%%%%%%%%%
% %%%%%%%%%%%%%%%%%%%%%%%%%%%%%%%%%%%%%%%%%%%%%%%%%%%%%%%%%%%%%%%%%%%%%%%%%%%%%%
% \section{Implementation}
%\iffalse
%<*package>
%\fi
%
% This section describes the definitions file |childdoc.def|.

% The definitions cannot be loaded using |\usepackage| or |\RequirePackage|
% which has a mechanism to prevent loading a style file more than once.
% When loading the definitions by means of |\input|
% multiple instances have to be prevented manually:
%\iffalse
%This code needs to be before the `\ProvidesFile' directive
%which is defined at the beginning of this file.
%Therefore it is also placed there and commented out here.
%</package>
%<*discard>
%\fi
%    \begin{macrocode}
\ifdefined\childdocmain\endinput\fi
%    \end{macrocode}
%\iffalse
%</discard>
%<*package>
%\fi
%
% \macro{\ifchilddoc}
% \macro{\ifchilddocmanual}
% The conditional |\ifchilddoc| tells whether a
% child (true) or main (false) document is being compiled.
% The conditional |\ifchilddocmanual| tells whether
% the |\includeonly| mechanism is used (false) or
% the selection of child files must be performed manually (true).
% The definitions initialise to false:
%    \begin{macrocode}
\newif\ifchilddoc
\newif\ifchilddocmanual
%    \end{macrocode}

% \macro{\childdocname}
% \macro{\childdocjob}
% The macro |\childdocname| stores the name of the main document
% to be compiled. The macro |\childdocjob| stores the name of
% the document on which the \LaTeX{} compiler was originally invoked.
% The content of |\jobname| cannot be compared
% to filenames specified in the source due to different catcodes.
% The following code rescans |\jobname|, stores the result
% in |\childdocname| and saves a copy in |\childdocjob|:
%    \begin{macrocode}
\edef\childdocname{\scantokens\expandafter{\jobname\noexpand}}
\let\childdocjob\childdocname
%    \end{macrocode}

% \macro{\childdocdisable}
% The macro |\childdocdisable| prevents the main file
% from being processed more than once.
% At this stage, the main document command |\childdocmain|
% is assumed to be called once again where it should do nothing.
% Any subsequent call to it should prevent
% a secondary processing of the main document
% It overwrites the forwarding commands
% |\childdocof| and |\childdocforward|
% with empty macros to prevent further inclusions of the main document:
%    \begin{macrocode}
\newcommand{\childdocdisable}
{
  \renewcommand{\childdocmain}[1]{\renewcommand{\childdocmain}[1]{\endinput}}
  \renewcommand{\childdocof}[1]{}
  \renewcommand{\childdocby}[2][]{}
  \renewcommand{\childdocforward}[2][]{}
  \renewcommand{\childdocdisable}{}
}
%    \end{macrocode}

% \macro{\childdocmain}
% The macro |\childdocmain| is to be called at the top of the main file
% with nothing or the main filename (without extension) as argument.
% First, it breaks loops.
% If the argument is not empty and does not match |\childdocname|
% (which is set by the first inclusion of |childdoc.def|),
% |\ifchilddoc| is set to true, |\includeonly| is applied to the child file
% and |\jobname| is set to the main file
% (for proper handling of |.aux| files):
%    \begin{macrocode}
\newcommand{\childdocmain}[1]
{
  \childdocdisable\childdocmain{}
  \if?#1?\else
    \begingroup
      \def\childdoctmp{#1}
      \ifx\childdoctmp\childdocname
        \def\childdoctmp{}
      \else
        \def\childdoctmp
        {
          \childdoctrue
          \includeonly{\childdocname}
          \def\childdocjob{#1}
          \def\jobname{#1}
        }
      \fi
      \expandafter
    \endgroup
    \childdoctmp
  \fi
}
%    \end{macrocode}

% \macro{\childdocof}
% The command |\childdocof| redirects
% compilation to the main file |#1|.
%    \begin{macrocode}
\newcommand{\childdocof}[1]
{
  \childdocdisable
  \childdoctrue
  \includeonly{\childdocname}
  \def\jobname{#1}
  \def\childdocjob{#1}
  \input{#1}
}
%    \end{macrocode}

% \macro{\childdocby}
% The command |\childdocby| ....
%    \begin{macrocode}
\newcommand{\childdocby}[2][]
{
  \childdocdisable
  \childdoctrue
  \childdocmanualtrue
  \if?#1?\else
    \def\jobname{#2}
  \fi
  \def\childdocjob{#2}
  \input{#2}
  \endinput
}
%    \end{macrocode}

% \macro{\childdocforward}
% The command |\childdocforward| redirects
% compilation to the main file or
% (if the optional argument is given) a child file.
% Parameters are set as if the main file
% or a child file starting with |\childdocof| was compiled.
% Then compilation is handed over to the main file:
%    \begin{macrocode}
\newcommand{\childdocforward}[2][]
{
  \begingroup
    \if?#1?
      \def\childdoctmp
      {
        \def\childdocname{#2}
        \def\childdocjob{#2}
        \def\jobname{#2}
        \input{#2}
        \endinput
      }
    \else
      \def\childdoctmp
      {
        \childdocdisable
        \def\childdocname{#2}
        \childdoctrue
        \includeonly{#2}
        \def\childdocjob{#1}
        \def\jobname{#1}
        \input{#1}
        \endinput
      }
    \fi
    \expandafter
  \endgroup
  \childdoctmp
}
%    \end{macrocode}

% \macro{\childdocforwardprefix}
% The command |\childdocforwardprefix| redirects
% compilation to the main or a child file by means of a pattern.
% The prefix |#1| in the current filename is replaced by |#2|
% and the suffix of the current filename is kept
% (it is assumed that the filename does not contain the substring `|~~~|'
% which is used as a delimiter).
% Compilation is handed over to the new file by |\childdocforward|:
%    \begin{macrocode}
\newcommand{\childdocforwardprefix}[3][]
{
  \begingroup
    \def\childdocextract #2##1~~~{\def\childdoctmp{\childdocforward[#1]{#3##1}}}
    \expandafter\childdocextract\childdocname~~~
    \expandafter
  \endgroup
  \childdoctmp
}
%    \end{macrocode}

% \macro{\childdoc}
% The deprecated macro |\childdoc| is a legacy version of |\childdocmain|:
%    \begin{macrocode}
\newcommand{\childdoc}{\childdocmain}
%    \end{macrocode}

% \macro{\childdocredirect}
% The deprecated macro |\childdocredirect| is a legacy version
% of |\childdocforward| and |\childdocforwardprefix|:
%    \begin{macrocode}
\newcommand{\childdocredirect}[2][]
{
  \begingroup
    \if?#1?
      \def\childdoctmp{\childdocforward{#2}}
    \else
      \def\childdoctmp{\childdocforwardprefix{#1}{#2}}
    \fi
    \expandafter
  \endgroup
  \childdoctmp
}
%    \end{macrocode}

%\iffalse
%</package>
%\fi
%
\endinput
\childdocforward[|\textit{main}|]{|\textit{dest}|}"|
\end{center}
%
Here \textit{target} is the name of the output file,
\textit{main} is the name of the main file
and \textit{dest} is the name of the main or child file to be processed
(all filenames without extensions).
The optional argument \textit{main} can be omitted
if \textit{main} matches \textit{dest}.
Optionally, compilation \textit{flags} can be defined via |\def| commands.
This command line makes the \TeX{} engine believe
it is compiling the file \textit{target}
whose content is specified as the latter parameter.
The provided code then forwards the processing to
\textit{main} or \textit{dest} as described in \secref{sec:forward}.

%%%%%%%%%%%%%%%%%%%%%%%%%%%%%%%%%%%%%%%%%%%%%%%%%%%%%%%%%%%%%%%%%%%%%%%%%%%%%%%%
\subsection{Include by Input}
\label{sec:input}

Including child documents by |\include| has some restrictions by design.
Most notably, the content of a child document always occupies
its own set of pages; pages cannot be shared between child documents.
Usually, this behaviour makes perfect sense
because each child document contain an essential part of the document.
However, in some situations it may be desirable to compose
a document from a collection of parts
without having mandatory page breaks between then.
For this case, the package
provides a mechanism to include parts
by |\input| which can also be processed individually.
However, by construction this mechanism
requires manual handling of the content to be output.

%%%%%%%%%%%%%%%%%%%%%%%%%%%%%%%%%%%%%%%%
\DescribeMacro{\ifchilddocmanual}
The main file should be prepared as usual, see \secref{sec:include}.
However, the document body must make a distinction
between processing of an individual part and of the main document, e.g.:
%
\begin{center}
\begin{tabular}{l}
|\ifchilddocmanual|\\
|\input{\childdocname}|\\
|\||else|\\
\textit{document body with }|\input{|\textit{part}|}|\\
|\||fi|
\end{tabular}
\end{center}
%
The conditional |\ifchilddocmanual| is true whenever
a part to be included by |\input| is being compiled,
and the name of the part is stored in |\childdocname|.

%%%%%%%%%%%%%%%%%%%%%%%%%%%%%%%%%%%%%%%%
\DescribeMacro{\childdocby}
Each part to be included by |\input| should start with:
%
\begin{center}
\begin{tabular}{l}
|% \iffalse
%
% childdoc.dtx Copyright (C) 2017-2018 Niklas Beisert
%
% This work may be distributed and/or modified under the
% conditions of the LaTeX Project Public License, either version 1.3
% of this license or (at your option) any later version.
% The latest version of this license is in
%   http://www.latex-project.org/lppl.txt
% and version 1.3 or later is part of all distributions of LaTeX
% version 2005/12/01 or later.
%
% This work has the LPPL maintenance status `maintained'.
%
% The Current Maintainer of this work is Niklas Beisert.
%
% This work consists of the files childdoc.dtx and childdoc.ins
% and the derived files childdoc.def and cdocsamp.tex with
% cdocsch1.tex, cdocsch2.tex, cdocsdrf.tex, cdocsfn1.tex, cdocsfn2.tex.
%
%<package>\ifdefined\childdocmain\endinput\fi
%<package>\ProvidesFile{childdoc.def}[2018/12/30 v2.0 child document driver]
%<samplemain>\ProvidesFile{cdocsamp.tex}[2018/12/30 v2.0 sample for childdoc]
%<*driver>
%\ProvidesFile{childdoc.drv}[2018/12/30 v2.0 childdoc reference manual file]
\PassOptionsToClass{10pt,a4paper}{article}
\documentclass{ltxdoc}

\usepackage[margin=35mm]{geometry}
\usepackage{hyperref}
\usepackage{hyperxmp}
\usepackage[usenames]{color}

\hypersetup{colorlinks=true}
\hypersetup{pdfstartview=FitH}
\hypersetup{pdfpagemode=UseNone}
\hypersetup{pdfsource={}}
\hypersetup{pdflang={en-UK}}
\hypersetup{pdfcopyright={Copyright 2017-2018 Niklas Beisert.
  This work may be distributed and/or modified under the
  conditions of the LaTeX Project Public License, either version 1.3
  of this license or (at your option) any later version.}}
\hypersetup{pdflicenseurl={http://www.latex-project.org/lppl.txt}}
\hypersetup{pdfcontactaddress={ETH Zurich, ITP, HIT K,
  Wolfgang-Pauli-Strasse 27}}
\hypersetup{pdfcontactpostcode={8093}}
\hypersetup{pdfcontactcity={Zurich}}
\hypersetup{pdfcontactcountry={Switzerland}}
\hypersetup{pdfcontactemail={nbeisert@itp.phys.ethz.ch}}
\hypersetup{pdfcontacturl={http://people.phys.ethz.ch/\xmptilde nbeisert/}}

\newcommand{\secref}[1]{\hyperref[#1]{section \ref*{#1}}}

\parskip1ex
\parindent0pt
\let\olditemize\itemize
\def\itemize{\olditemize\parskip0pt}

\begin{document}

\title{The \textsf{childdoc} Package}
\hypersetup{pdftitle={The childdoc Package}}
\author{Niklas Beisert\\[2ex]
  Institut f\"ur Theoretische Physik\\
  Eidgen\"ossische Technische Hochschule Z\"urich\\
  Wolfgang-Pauli-Strasse 27, 8093 Z\"urich, Switzerland\\[1ex]
  \href{mailto:nbeisert@itp.phys.ethz.ch}
  {\texttt{nbeisert@itp.phys.ethz.ch}}}
\hypersetup{pdfauthor={Niklas Beisert}}
\hypersetup{pdfsubject={Manual for the LaTeX2e Package childdoc}}
\date{30 December 2018, \textsf{v2.0}}
\maketitle

\begin{abstract}\noindent
\textsf{childdoc} is a \LaTeXe{} package
that enables the direct compilation
of document sections included by |\include|
to individual files.
\end{abstract}

\begingroup
\parskip0ex
\tableofcontents
\endgroup

%%%%%%%%%%%%%%%%%%%%%%%%%%%%%%%%%%%%%%%%%%%%%%%%%%%%%%%%%%%%%%%%%%%%%%%%%%%%%%%%
%%%%%%%%%%%%%%%%%%%%%%%%%%%%%%%%%%%%%%%%%%%%%%%%%%%%%%%%%%%%%%%%%%%%%%%%%%%%%%%%
\section{Introduction}

\LaTeX{} provides a mechanism to structure a large document (such as a book)
into a main file and several child files (containing the chapters)
using the |\include| command.
This mechanism is beneficial for documents
which span hundreds of pages in order to
make the source file(s) more manageable.
Moreover, compilation can be restricted to
selected child files by means of the |\includeonly| command.
The latter feature can be used to reduce the compilation time while editing
(this was significantly more useful in the earlier days of \LaTeX{})
or to generate a smaller document which is easier to navigate.
Another application of |\includeonly| is to generate
documents consisting of selected parts of the complete document.

However, there are a few drawbacks of the plain |\include| mechanism:
\begin{itemize}
\item
The child files cannot be compiled on their own,
they can only be compiled via the main file.
A naive editing environment
(such as a text editor with an option
to have the current file processed by \LaTeX)
may require one to switch to the main file before compiling;
attempting to compile the child file produces errors.
\item
The main file must be modified (each time)
to adjust the |\includeonly| command
to the present needs. This easily leaves the main file in a messy state.
\item
The generated document will always carry the filename
of the main document. This is inconvenient if
several child files are to be compiled and
to be kept for distribution.
\end{itemize}

The present package provides a simple interface
to make child files individually compilable by \LaTeX{}.
Compiling a child file then has the same effect as compiling
the main file with an |\includeonly| command
to select the appropriate child.
Moreover the generated document will carry the name of the child
rather than the main file.
This resolves all three above issues.

This feature is meant to make the editing of books,
thesis documents and lecture notes somewhat more convenient.
However, the package can also be used efficiently for
composing a series of documents (such as exercise sheets)
which are typically distributed individually.
It then assists the author in generating the individual documents
(potentially in different versions)
as well as a document containing the collected series.
Another application is in developing style files
or other kinds of included material
where compilation of the style file could redirect
to a sample or test file.

%%%%%%%%%%%%%%%%%%%%%%%%%%%%%%%%%%%%%%%%%%%%%%%%%%%%%%%%%%%%%%%%%%%%%%%%%%%%%%%%
%%%%%%%%%%%%%%%%%%%%%%%%%%%%%%%%%%%%%%%%%%%%%%%%%%%%%%%%%%%%%%%%%%%%%%%%%%%%%%%%
\section{Usage}

First of all, the package \textsf{childdoc} is \emph{not} a standard
\LaTeXe{} |.sty| style file! Therefore it needs to be invoked in
a non-standard way.

%%%%%%%%%%%%%%%%%%%%%%%%%%%%%%%%%%%%%%%%%%%%%%%%%%%%%%%%%%%%%%%%%%%%%%%%%%%%%%%%
\subsection{Included Files}
\label{sec:include}

%%%%%%%%%%%%%%%%%%%%%%%%%%%%%%%%%%%%%%%%
\DescribeMacro{\childdocmain}
To use the package, add the commands
\begin{center}
\begin{tabular}{l}
|\input{childdoc.def}|\\
|\childdocmain{}|\\
\end{tabular}
\end{center}
at the very top of the main \LaTeX{} file,
in particular \emph{before} the |\documentclass| statement!
The argument of |\childdocmain| should be left empty
(but it must be present).

%%%%%%%%%%%%%%%%%%%%%%%%%%%%%%%%%%%%%%%%
\DescribeMacro{\childdocof}
Furthermore, add the commands
\begin{center}
\begin{tabular}{l}
|\input{childdoc.def}|\\
|\childdocof{|\textit{main}|}|\\
\end{tabular}
\end{center}
at the top of every child file \textit{child}
which is included by |\include{|\textit{child}|}|
from within the main file
(or at least for those files to be compiled individually).
The argument \textit{main} must be the filename of the main file.

There are a couple of
considerations in setting up the main and child documents:

%%%%%%%%%%%%%%%%%%%%%%%%%%%%%%%%%%%%%%%%
\paragraph{Restrictions.}

Please note the following restrictions:
\begin{itemize}
\item
|\childdocmain| must be called with one argument \textit{main}
to ensure compatibility with earlier version of the package.
It must either be empty (|\childdocmain{}|)
or precisely match the filename of the main file in which it is specified.
See \secref{sec:detection} for further information.
\item
The filename \textit{main} must be specified without the |.tex| extension.
\item
The filename \textit{main} is case sensitive
(even in case-insensitive file systems)
due to internal string comparison.
\item
The argument \textit{main} should be fully expanded, it cannot be a macro.
\item
Subdirectories and special characters should be avoided in filenames.
\item
The command |\childdocmain{|\textit{main}|}| must be followed by a whitespace.
It should not be followed immediately by another command
or by a comment mark `|%|'.
This is because the \TeX{} parser reads the token immediately following
the argument of |\childdocmain| and puts it
at the beginning of every child section;
however, a white\-space is ignored.
\end{itemize}

%%%%%%%%%%%%%%%%%%%%%%%%%%%%%%%%%%%%%%%%
\paragraph{Content of Main File.}

It is advisable to place all content in the child files included by |\include|.
Any output contained in the main file will appear in all child documents
unless suppressed manually;
it cannot be suppressed automatically by the |\includeonly| directive
and thus should normally be avoided.
A method to include some content in the main file
by means of conditional processing is described in \secref{sec:conditional}.

%%%%%%%%%%%%%%%%%%%%%%%%%%%%%%%%%%%%%%%%
\paragraph{Page Numbering.}

When only a part of the document is compiled,
the appropriate numbering of pages
(as well as other status parameters)
is determined from the |.aux| files.
The latter contain information from previous passes.
However this information needs to propagate through
all intermediate child documents.
Therefore the page numbering in child documents may well
be inconsistent until the complete document is compiled at least once.

A useful (if unconventional) way to always ensure a consistent
page numbering is to restart the numbering in each child document
and denote the pages by `\textit{child}|.|\textit{page}'
where \textit{child} represents the chapter/section number of the child file.
This can be achieved by the command
|\numberwithin{page}{|\textit{child}|}|
of the \textsf{amsmath} package
where \textit{child} can be |chapter| or |section|
depending on the chosen structuring.
Alternatively, one can modify the macro |\thepage| appropriately
and reset the counter |page| at the start of each child file.

%%%%%%%%%%%%%%%%%%%%%%%%%%%%%%%%%%%%%%%%%%%%%%%%%%%%%%%%%%%%%%%%%%%%%%%%%%%%%%%%
\subsection{Conditional Processing}
\label{sec:conditional}

The package provides a mechanism to compile different versions
of a document. To customise the versions further some conditional processing
can come in handy to distinguish which version is being compiled.
The package provides two macros to describe the compilation context:

%%%%%%%%%%%%%%%%%%%%%%%%%%%%%%%%%%%%%%%%
\DescribeMacro{\ifchilddoc}
The conditional |\ifchilddoc| distinguishes between the compilation of
child documents and the main document:
%
\begin{center}
|\ifchilddoc |\textit{child-code}| |[|\||else |\textit{main-code}]| \||fi|
\end{center}

%%%%%%%%%%%%%%%%%%%%%%%%%%%%%%%%%%%%%%%%
\DescribeMacro{\childdocname}
\DescribeMacro{\childdocjob}
The macro |\childdocname| contains the filename (without extension)
of the main or child file being processed.
Note that |\childdocjob| will always contain the name of the main file.

%%%%%%%%%%%%%%%%%%%%%%%%%%%%%%%%%%%%%%%%
\paragraph{Title Page.}

Conditional processing can be used to include a title or banner page
in the main document when proper precautions are taken.
Importantly, the code in the main file should ensure that the page counter
(as well as other status parameters which are stored in the |.aux| files)
takes the same value after the conditional processing.
Otherwise the page numbers may take divergent values
depending on which part is compiled.

For example, a title page could be declared by:
%
\begin{center}
\begin{tabular}{l}
|\ifchilddoc\||else|\\
|\addtocounter{page}{-1}|\\
\textit{code for title page}\\
|\newpage|\\
|\||fi|
\end{tabular}
\end{center}
%
A banner page for the child documents can be generated by:
%
\begin{center}
\begin{tabular}{l}
|\ifchilddoc|\\
|\addtocounter{page}{-1}|\\
\textit{code for banner page}\\
|\newpage|\\
|\||fi|
\end{tabular}
\end{center}
%
Here one could write a message such as:
\begin{center}
|This is the part \childdocname{} of \childdocjob{}.|
\end{center}

%%%%%%%%%%%%%%%%%%%%%%%%%%%%%%%%%%%%%%%%%%%%%%%%%%%%%%%%%%%%%%%%%%%%%%%%%%%%%%%%
\subsection{Flags}
\label{sec:flags}

The package makes it easy to generate different versions
of the main or child documents.
To this end compilation flags can be defined
and assigned different default values.
They will be particularly useful in conjunction
with the forwarding mechanism described in \secref{sec:forward}.

For example, it may be useful to have a flag |\version|
which can be set to |draft| or |final|.
The document source will contain some conditional code
depending on the value of |\version|.
Suppose further, the flag should default to |final| for the main file
and to |draft| for child files
which is a natural assignment for editing the document.
This is achieved by placing the following code
in the preamble of the main document
(below the |\childdocmain| directive):
%
\begin{center}
\begin{tabular}{l}
|\ifchilddoc|\\
|\providecommand{\version}{draft}|\\
|\||else|\\
|\providecommand{\version}{final}|\\
|\||fi|
\end{tabular}
\end{center}
%
The definition by |\providecommand| makes sure
that previous definitions are not overwritten.
Further statements |\providecommand{\version}{...}|
can thus be added before the above code to override it.

For the main file, one might add a line
(between |\childdocmain| and the above block)
%
\begin{center}
|%\ifchilddoc\||else\providecommand{\version}{draft}\||fi|
\end{center}
%
which can be uncommented to produce a draft version.
Likewise one can add a line to the very top of a child file
(above the |\childdocof{|\textit{main}|}| directive)
%
\begin{center}
|%\providecommand{\version}{final}|
\end{center}
%
which can be uncommented to produce the final version of this child document.

%%%%%%%%%%%%%%%%%%%%%%%%%%%%%%%%%%%%%%%%%%%%%%%%%%%%%%%%%%%%%%%%%%%%%%%%%%%%%%%%
\subsection{Forwarding}
\label{sec:forward}

Different versions of the main or child documents
using compilation flags as described in \secref{sec:flags}
can be (permanently) stored in different files
for convenient compilation, viewing and distribution.
To this end, the package defines a command
to pass on compilation to a different file:

%%%%%%%%%%%%%%%%%%%%%%%%%%%%%%%%%%%%%%%%
\DescribeMacro{\childdocforward}
The command |\childdocforward| redirects processing to
another source file:
%
\begin{center}
\begin{tabular}{l}
|\input{childdoc.def}|\\
|\childdocforward[|\textit{main}|]{|\textit{dest}|}|\\
\end{tabular}
\end{center}
%
The argument \textit{dest} is the destination file
(without extension).
It should be the main file or one of the child files.
Note that further \textsf{childdoc} directives
such as |\childdocof| and |\childdocforward|
in the indicated file will be processed in this form.
The optional argument \textit{main}
passes on directly to the main file \textit{main}
while pretending to compile the child \textit{dest}.
This form behaves as if \textit{dest}
issues |\childdocof{|\textit{main}|}| right away,
and no further \textsf{childdoc} directives will be processed.

%%%%%%%%%%%%%%%%%%%%%%%%%%%%%%%%%%%%%%%%
\DescribeMacro{\...prefix}
In the alternative form |\childdocforwardprefix|,
%
\begin{center}
\begin{tabular}{l}
|\input{childdoc.def}|\\
|\childdocforwardprefix[|\textit{main}|]{|\textit{prefix}|}{|\textit{dest}|}|
\end{tabular}
\end{center}
%
the destination file is determined by a pattern
depending on the current file:
To make this work, the current file must be called
`{\textit{prefix}\hspace{0.2em}\textit{suffix}}'
with \textit{prefix} matching precisely the argument.
Processing is then passed on to the file
`{\textit{dest}\hspace{0.2em}\textit{suffix}}'.
Surely, the same effect is achieved by
directly specifying the
argument `{\textit{dest}\hspace{0.2em}\textit{suffix}}'
in the first form.
However, that requires to set up a different file
for each child. With the alternative form of the command
all these files can have exactly the same content
which simplifies setting them up and maintaining them.

For example, the following file |draft.tex|
with a compilation flag |\version| as described in \secref{sec:flags}
compiles the main document as a draft:
%
\begin{center}
\begin{tabular}{l}
|\def\version{draft}|\\
|\input{childdoc.def}|\\
|\childdocforward{|\textit{main}|}|
\end{tabular}
\end{center}
%
Likewise, the following files |final|\textit{nn}|.tex|
compile the final version of the child document
|child|\textit{nn}|.tex|:
%
\begin{center}
\begin{tabular}{l}
|\def\version{final}|\\
|\input{childdoc.def}|\\
|\childdocforwardprefix{final}{child}|
\end{tabular}
\end{center}
%

Note that when several versions of a main file and/or of each child file
are to be generated, it may be convenient to set up a |Makefile| or
shell script to automatise the process.

%%%%%%%%%%%%%%%%%%%%%%%%%%%%%%%%%%%%%%%%%%%%%%%%%%%%%%%%%%%%%%%%%%%%%%%%%%%%%%%%
\subsection{Command Line Processing}
\label{sec:commandline}

The effect of redirection files can also be achieved by invoking
the \LaTeX{} compiler with a more elaborate command line.
Most conveniently this should be done as part
of a shell script or a |Makefile|.

When using \textsf{childdoc} in the main file, the following
command lines effectively perform a redirection
(note that depending on the shell being used,
backslashes may have to be doubled: `|\|' $\to$ `|\\|'):
%
\begin{center}
|... -jobname "|\textit{target}|" |\\|"|[\textit{flags}]%
|\input{childdoc.def}\childdocforward[|\textit{main}|]{|\textit{dest}|}"|
\end{center}
%
Here \textit{target} is the name of the output file,
\textit{main} is the name of the main file
and \textit{dest} is the name of the main or child file to be processed
(all filenames without extensions).
The optional argument \textit{main} can be omitted
if \textit{main} matches \textit{dest}.
Optionally, compilation \textit{flags} can be defined via |\def| commands.
This command line makes the \TeX{} engine believe
it is compiling the file \textit{target}
whose content is specified as the latter parameter.
The provided code then forwards the processing to
\textit{main} or \textit{dest} as described in \secref{sec:forward}.

%%%%%%%%%%%%%%%%%%%%%%%%%%%%%%%%%%%%%%%%%%%%%%%%%%%%%%%%%%%%%%%%%%%%%%%%%%%%%%%%
\subsection{Include by Input}
\label{sec:input}

Including child documents by |\include| has some restrictions by design.
Most notably, the content of a child document always occupies
its own set of pages; pages cannot be shared between child documents.
Usually, this behaviour makes perfect sense
because each child document contain an essential part of the document.
However, in some situations it may be desirable to compose
a document from a collection of parts
without having mandatory page breaks between then.
For this case, the package
provides a mechanism to include parts
by |\input| which can also be processed individually.
However, by construction this mechanism
requires manual handling of the content to be output.

%%%%%%%%%%%%%%%%%%%%%%%%%%%%%%%%%%%%%%%%
\DescribeMacro{\ifchilddocmanual}
The main file should be prepared as usual, see \secref{sec:include}.
However, the document body must make a distinction
between processing of an individual part and of the main document, e.g.:
%
\begin{center}
\begin{tabular}{l}
|\ifchilddocmanual|\\
|\input{\childdocname}|\\
|\||else|\\
\textit{document body with }|\input{|\textit{part}|}|\\
|\||fi|
\end{tabular}
\end{center}
%
The conditional |\ifchilddocmanual| is true whenever
a part to be included by |\input| is being compiled,
and the name of the part is stored in |\childdocname|.

%%%%%%%%%%%%%%%%%%%%%%%%%%%%%%%%%%%%%%%%
\DescribeMacro{\childdocby}
Each part to be included by |\input| should start with:
%
\begin{center}
\begin{tabular}{l}
|\input{childdoc.def}|\\
|\childdocby{|\textit{main}|}|\\
\end{tabular}
\end{center}
%
The directive |\childdocby| is similar to |\childdocof|
described in \secref{sec:include},
but the subsequent selection of content must be done manually.
To that end, both |\ifchilddoc| and |\ifchilddocmanual|
will be true upon processing of a part,
and the name of the part is stored in |\childdocname|.
Note that |\jobname| will be set to the filename of the current part
so that each part receives an individual |.aux| file
that does not interfere with the |.aux| file(s) of the main document.
This behaviour can be altered by the alternative form
|\childdocby[*]{|\textit{main}|}| (with a non-empty optional argument)
which uses the |.aux| file of the main document
by setting |\jobname| to \textit{main}.

%%%%%%%%%%%%%%%%%%%%%%%%%%%%%%%%%%%%%%%%%%%%%%%%%%%%%%%%%%%%%%%%%%%%%%%%%%%%%%%%
\subsection{Driver Development}
\label{sec:driver}

The \textsf{childdoc} mechanism can also be use for the development
of definition files such as \LaTeX{} styles or classes.
This case differs from the above setup with multiple parts
included by |\include| in that no |\includeonly| should be invoked.
This can be achieved by starting the include file
(before |\ProvidesPackage|) with:
%
\begin{center}
\begin{tabular}{l}
|\input{childdoc.def}|\\
|\childdocforward{|\textit{main}|}|\\
\end{tabular}
\end{center}
%
or alternatively with:
%
\begin{center}
\begin{tabular}{l}
|\input{childdoc.def}|\\
|\childdocby{|\textit{main}|}|\\
\end{tabular}
\end{center}
%
Both forms have slightly different effects as described above.
The main file is prepared as usual, see \secref{sec:include}.

%%%%%%%%%%%%%%%%%%%%%%%%%%%%%%%%%%%%%%%%%%%%%%%%%%%%%%%%%%%%%%%%%%%%%%%%%%%%%%%%
\subsection{Legacy Detection}
\label{sec:detection}

The directive |\childdocmain| in the main file can detect
whether the complete document or merely a child is to be compiled
even without using the directive |\childdocof|.
This method is deprecated because it is less robust
and there is no compelling reason to use it;
it is merely provided for backward compatibility
and it may be removed in future versions.

If the detection mechanism is to be used,
it is mandatory to correctly specify
the filename of the main file as the argument of |\childdocmain|:
%
\begin{center}
\begin{tabular}{l}
|\input{childdoc.def}|\\
|\childdocmain{|\textit{main}|}|\\
\end{tabular}
\end{center}
%
If |\jobname| does not match the argument \textit{main} of |\childdocmain|,
it is assumed that |\jobname| points to the child file to be compiled.
When using |\childdocmain| with the main file specified as argument,
it suffices to start a child file
with just |\input{|\textit{main}|}|
without loading of the package and using |\childdocof|.
If instead all processing is done
with the appropriate \textsf{childdoc} directives,
the argument of \textit{main} of |\childdocmain| can be empty.

An alternative version of the command line processing described
in \secref{sec:commandline} using the detection mechanism reads:
%
\begin{center}
|... -jobname "|\textit{target}|" "|[\textit{flags}]%
[|\def\jobname{|\textit{dest}|}|]|\input{|\textit{main}|}"|
\end{center}

%%%%%%%%%%%%%%%%%%%%%%%%%%%%%%%%%%%%%%%%%%%%%%%%%%%%%%%%%%%%%%%%%%%%%%%%%%%%%%%%
\subsection{Manual Code}
\label{sec:manual}

In case one cannot be certain whether the definitions file |childdoc.def|
is installed on the target \TeX{} distribution
and one prefers not to ship it,
it is conceivable to paste a few relevant commands into the sources.

To that end, drop all statements |\input{childdoc.def}|
and perform the replacements as outlined below.
Instead of |\childdocmain{|\textit{main}|}| add the following code
to the top of the main file:
%
\begin{center}
\begin{tabular}{l}
|\||ifdefined\childdocname\endinput\||fi\newif\ifchilddoc|\\
|\edef\childdocname{\scantokens\expandafter{\jobname\noexpand}}|\\
|\def\childdocmain{|\textit{main}|}\||ifx\childdocmain\childdocname\||else|\\
|\childdoctrue\includeonly{\childdocname}\let\jobname\childdocmain\||fi|\\
\end{tabular}
\end{center}
%
Instead of |\childdocof{|\textit{main}|}| just include the main file
at the top of each child file:
%
\begin{center}
|\input{|\textit{main}|}|
\end{center}
%
A simple redirection |\childdocforward{|\textit{dest}|}| is achieved by:
%
\begin{center}
|\def\jobname{|\textit{dest}|}\input{\jobname}|
\end{center}
%
The redirection with prefix
|\childdocforwardprefix[|\textit{prefix}|]{|\textit{dest}|}|
is accomplished by:
%
\begin{center}
\begin{tabular}{l}
|{\edef\jobname{\scantokens\expandafter{\jobname\noexpand}}|\\
|\def\redirectjob |\textit{prefix}|#1~~~{\gdef\jobname{|\textit{dest}|#1}}|\\
|\expandafter\redirectjob\jobname~~~}\input{\jobname}|
\end{tabular}
\end{center}

In an alternative approach,
child documents can be compiled by a specific command line
without additional code or specific definitions:
%
\begin{center}
|... -jobname "|\textit{target}|" "|[\textit{flags}]%
|\includeonly{|\textit{dest}|}\input{|\textit{main}|}"|
\end{center}
%

%%%%%%%%%%%%%%%%%%%%%%%%%%%%%%%%%%%%%%%%%%%%%%%%%%%%%%%%%%%%%%%%%%%%%%%%%%%%%%%%
%%%%%%%%%%%%%%%%%%%%%%%%%%%%%%%%%%%%%%%%%%%%%%%%%%%%%%%%%%%%%%%%%%%%%%%%%%%%%%%%
\section{Information}

%%%%%%%%%%%%%%%%%%%%%%%%%%%%%%%%%%%%%%%%%%%%%%%%%%%%%%%%%%%%%%%%%%%%%%%%%%%%%%%%
\subsection{Copyright}

Copyright \copyright{} 2017--2018 Niklas Beisert

This work may be distributed and/or modified under the
conditions of the \LaTeX{} Project Public License, either version 1.3
of this license or (at your option) any later version.
The latest version of this license is in
  \url{http://www.latex-project.org/lppl.txt}
and version 1.3 or later is part of all distributions of \LaTeX{}
version 2005/12/01 or later.

This work has the LPPL maintenance status `maintained'.

The Current Maintainer of this work is Niklas Beisert.

This work consists of the files |README.txt|, |childdoc.ins| and |childdoc.dtx|
as well as the derived files |childdoc.def|, |cdocsamp.tex|
with |cdocsch1.tex|, |cdocsch2.tex|, |cdocspt3.tex|, |cdocspt4.tex|,
|cdocsdrf.tex|, |cdocsfn1.tex|, |cdocsfn2.tex|
as well as |childdoc.pdf|.

%%%%%%%%%%%%%%%%%%%%%%%%%%%%%%%%%%%%%%%%%%%%%%%%%%%%%%%%%%%%%%%%%%%%%%%%%%%%%%%%
\subsection{Files and Installation}

The package consists of the files:
%
\begin{center}
\begin{tabular}{ll}
    |README.txt|   & readme file \\
    |childdoc.ins| & installation file \\
    |childdoc.dtx| & source file \\
    |childdoc.def| & definition file \\
    |cdocsamp.tex| & sample main file \\
    |cdocsch1.tex| & sample include file \\
    |cdocsch2.tex| & sample include file \\
    |cdocspt3.tex| & sample part file \\
    |cdocspt4.tex| & sample part file \\
    |cdocsdrf.tex| & sample redirection file \\
    |cdocsfn1.tex| & sample redirection file \\
    |cdocsfn2.tex| & sample redirection file \\
    |childdoc.pdf| & manual
\end{tabular}
\end{center}
%
The distribution consists of the files
|README.txt|, |childdoc.ins| and |childdoc.dtx|.
%
\begin{itemize}
\item
Run (pdf)\LaTeX{} on |childdoc.dtx|
to compile the manual |childdoc.pdf| (this file).
\item
Run \LaTeX{} on |childdoc.ins| to create the definitions file |childdoc.def|
and the sample |cdocsamp.tex| with include files
|cdocsch1.tex|, |cdocsch2.tex|, |cdocspt3.tex|, |cdocspt4.tex|,
|cdocsdrf.tex|, |cdocsfn1.tex|, |cdocsfn2.tex|.
Then copy the file |childdoc.def| to an appropriate directory of your \LaTeX{}
distribution, e.g.\ \textit{texmf-root}|/tex/latex/childdoc|.
\end{itemize}

%%%%%%%%%%%%%%%%%%%%%%%%%%%%%%%%%%%%%%%%%%%%%%%%%%%%%%%%%%%%%%%%%%%%%%%%%%%%%%%%
\subsection{Related CTAN Packages}

There are several other packages which offer a similar functionality:
%
\begin{itemize}
\item
The packages
\href{http://ctan.org/pkg/docmute}{\textsf{docmute}},
\href{http://ctan.org/pkg/includex}{\textsf{includex}} and
\href{http://ctan.org/pkg/standalone}{\textsf{standalone}}
provide commands to include only the document body of
a child file thus allowing both files to be compiled individually.
\item
The packages \href{http://ctan.org/pkg/subdocs}{\textsf{subdocs}}
and \href{http://ctan.org/pkg/subfiles}{\textsf{subfiles}}
provide structures in which the main and child documents can be
encapsulated and allowing them to be compiled individually.
The inclusion mechanism is different from the conventional |\include|.
\item
The package \href{http://ctan.org/pkg/combine}{\textsf{combine}}
is an elaborate solution to combine several documents into one.
\end{itemize}
%
See also the CTAN topic \href{http://ctan.org/topic/subdocs}{\textsf{subdocs}}
for further related packages.
The present package differs from the above solutions in that
a document structure constructed with the conventional |\include| mechanism
just needs two extra commands at the top of every file
such that all constituent files can be compiled individually.

%%%%%%%%%%%%%%%%%%%%%%%%%%%%%%%%%%%%%%%%%%%%%%%%%%%%%%%%%%%%%%%%%%%%%%%%%%%%%%%%
%\subsection{Feature Suggestions}
%
%The following is a list of features which may be useful for future
%versions of this package:
%%
%\begin{itemize}
%\item
%\ldots
%\end{itemize}

%%%%%%%%%%%%%%%%%%%%%%%%%%%%%%%%%%%%%%%%%%%%%%%%%%%%%%%%%%%%%%%%%%%%%%%%%%%%%%%%
\subsection{Revision History}

%%%%%%%%%%%%%%%%%%%%%%%%%%%%%%%%%%%%%%%%
\paragraph{v2.0:} 2018/12/30

\begin{itemize}
\item
immediate forward processing
\item
added |\childdocby| mechanism
\item
manual restructured
\end{itemize}

%%%%%%%%%%%%%%%%%%%%%%%%%%%%%%%%%%%%%%%%
\paragraph{v1.6:} 2018/01/17

\begin{itemize}
\item
application for development of include files
\item
corrections to manual
\end{itemize}

%%%%%%%%%%%%%%%%%%%%%%%%%%%%%%%%%%%%%%%%
\paragraph{v1.5:} 2017/05/21

\begin{itemize}
\item
more complete structuring introduced
\item
|\childdocof| introduced
\item
|\childdoc| renamed to |\childdocmain|
\item
|\childredirect| renamed to |\childdocforward| and |\childdocforwardprefix|
and functionality expanded
\end{itemize}

%%%%%%%%%%%%%%%%%%%%%%%%%%%%%%%%%%%%%%%%
\paragraph{v1.0:} 2017/04/27

\begin{itemize}
\item
manual and install package
\item
first version published on CTAN
\end{itemize}

%%%%%%%%%%%%%%%%%%%%%%%%%%%%%%%%%%%%%%%%
\paragraph{v0.6:} 2017/04/26

\begin{itemize}
\item
redirection mechanism added
\end{itemize}

%%%%%%%%%%%%%%%%%%%%%%%%%%%%%%%%%%%%%%%%
\paragraph{v0.5:} 2017/04/26

\begin{itemize}
\item
functionality in definition file
\end{itemize}


%%%%%%%%%%%%%%%%%%%%%%%%%%%%%%%%%%%%%%%%%%%%%%%%%%%%%%%%%%%%%%%%%%%%%%%%%%%%%%%%
%%%%%%%%%%%%%%%%%%%%%%%%%%%%%%%%%%%%%%%%%%%%%%%%%%%%%%%%%%%%%%%%%%%%%%%%%%%%%%%%
%%%%%%%%%%%%%%%%%%%%%%%%%%%%%%%%%%%%%%%%%%%%%%%%%%%%%%%%%%%%%%%%%%%%%%%%%%%%%%%%
\appendix

\settowidth\MacroIndent{\rmfamily\scriptsize 000\ }

 \DocInput{childdoc.dtx}

\end{document}
%</driver>
% \fi
%
% %%%%%%%%%%%%%%%%%%%%%%%%%%%%%%%%%%%%%%%%%%%%%%%%%%%%%%%%%%%%%%%%%%%%%%%%%%%%%%
% %%%%%%%%%%%%%%%%%%%%%%%%%%%%%%%%%%%%%%%%%%%%%%%%%%%%%%%%%%%%%%%%%%%%%%%%%%%%%%
% \section{Sample}
%\iffalse
%<*samplemain>
%\fi
%
% The following presents a sample document
% with two chapters, two parts, a title page,
% a compile flag as well as three forwarding files to set the flag.
% It consists of eight |.tex| files:
% \begin{center}
% \begin{tabular}{ll}
% |cdocsamp.tex|&main file\\
% |cdocsch1.tex|&include file for chapter 1\\
% |cdocsch2.tex|&include file for chapter 2\\
% |cdocspt3.tex|&include file for part 3\\
% |cdocspt4.tex|&include file for part 4\\
% |cdocsdrf.tex|&forwarding file for main file in draft mode\\
% |cdocsfi1.tex|&forwarding file for final version of chapter 1\\
% |cdocsfi2.tex|&forwarding file for final version of chapter 2\\
% \end{tabular}
% \end{center}
% Each of the eight files can be compiled directly by the \LaTeX{} compiler.
%
% %%%%%%%%%%%%%%%%%%%%%%%%%%%%%%%%%%%%%%
% \paragraph{Main File.}
%
% The main file is called |cdocsamp.tex|.
%
% Load the \textsf{childdoc} definitions and
% declare the filename for the main document:
%    \begin{macrocode}
\input{childdoc.def}
\childdocmain{}
%    \end{macrocode}

% Optional override for |\version| flag:
%    \begin{macrocode}
%%\ifchilddoc\else\providecommand{\version}{draft}\fi
%    \end{macrocode}

% Define the default values for the |\version| flag
% (|final| for the main file and |draft| for childs):
%    \begin{macrocode}
\ifchilddoc
\providecommand{\version}{draft}
\else
\providecommand{\version}{final}
\fi
%    \end{macrocode}

% Load the standard document class:
%    \begin{macrocode}
\documentclass[12pt]{article}
%    \end{macrocode}

% Start the document body:
%    \begin{macrocode}
\begin{document}
%    \end{macrocode}

% Declare a title page.
% Print title, part of document being processed and version flag:
%    \begin{macrocode}
\addtocounter{page}{-1}
\begin{center}
{\LARGE\bfseries{}childdoc example\par}
\vspace{1cm}
\ifchilddoc
\ifchilddocmanual part\else chapter\fi:
`\childdocname' of `\childdocjob'\par
\else
main document: `\childdocjob'\par
\fi
version: \version\par
\end{center}
\newpage
%    \end{macrocode}

% Manually include selected file,
% otherwise process as usual:
%    \begin{macrocode}
\ifchilddocmanual
\section*{part `\childdocname'}
\input{\childdocname}
\else
%    \end{macrocode}

% Include the two chapters:
%    \begin{macrocode}
\include{cdocsch1}
\include{cdocsch2}
%    \end{macrocode}

% Include the two parts unless only chapters should be displayed:
%    \begin{macrocode}
\ifchilddoc\else
\section{part three}
\input{cdocspt3}
\section{part four}
\input{cdocspt4}
\fi
%    \end{macrocode}

% Process as usual until here:
%    \begin{macrocode}
\fi
%    \end{macrocode}

% End of document body:
%    \begin{macrocode}
\end{document}
%    \end{macrocode}
%\iffalse
%</samplemain>
%\fi
%
% %%%%%%%%%%%%%%%%%%%%%%%%%%%%%%%%%%%%%%
% \paragraph{Chapter Include Files.}
%
% The include files are called |cdocsch1.tex| and |cdocsch2.tex|.
%
%\iffalse
%<*samplechap1|samplechap2>
%\fi

% Optional override for |\version| flag:
%    \begin{macrocode}
%%\providecommand{\version}{final}
%    \end{macrocode}

% Include the main document:
%    \begin{macrocode}
\input{childdoc.def}
\childdocof{cdocsamp}
%    \end{macrocode}

%\iffalse
%</samplechap1|samplechap2>
%\fi
%
%\iffalse
%<*samplechap1>
%\fi
% Some text for chapter 1:
%    \begin{macrocode}
\section{one}
some text in chapter one
%    \end{macrocode}

%\iffalse
%</samplechap1>
%\fi
% Some text for chapter 2:
%\iffalse
%<*samplechap2>
%\fi
%    \begin{macrocode}
\section{two}
more text in chapter two
%    \end{macrocode}

%\iffalse
%</samplechap2>
%\fi
%
% %%%%%%%%%%%%%%%%%%%%%%%%%%%%%%%%%%%%%%
% \paragraph{Part Include Files.}
%
% The include files are called |cdocspt3.tex| and |cdocspt4.tex|.
%
%\iffalse
%<*samplepart3|samplepart4>
%\fi

% Optional override for |\version| flag:
%    \begin{macrocode}
%%\providecommand{\version}{final}
%    \end{macrocode}

% Include the main document:
%    \begin{macrocode}
\input{childdoc.def}
\childdocby{cdocsamp}
%    \end{macrocode}

%\iffalse
%</samplepart3|samplepart4>
%\fi
%
%\iffalse
%<*samplepart3>
%\fi
% Some text for part 3:
%    \begin{macrocode}
some text in part three
%    \end{macrocode}

%\iffalse
%</samplepart3>
%\fi
% Some text for part 4:
%\iffalse
%<*samplepart4>
%\fi
%    \begin{macrocode}
more text in part four
%    \end{macrocode}

%\iffalse
%</samplepart4>
%\fi
%
% %%%%%%%%%%%%%%%%%%%%%%%%%%%%%%%%%%%%%%
% \paragraph{Forwarding for a Complete Draft.}
%
% The following forwarding file |cdocsdrf.tex|
% compiles the main document in draft mode:
%\iffalse
%<*sampledraft>
%\fi
%    \begin{macrocode}
\def\version{draft}
\input{childdoc.def}
\childdocforward{cdocsamp}
%    \end{macrocode}

%\iffalse
%</sampledraft>
%\fi
%
% %%%%%%%%%%%%%%%%%%%%%%%%%%%%%%%%%%%%%%
% \paragraph{Forwarding for Final Version of the Chapters.}
%
% The following forwarding files |cdocsfn1.tex| and |cdocsfn2.tex|
% (with identical content)
% compile the final versions of the child documents
% |cdocsch1.tex| and |cdocsch2.tex|, respectively:
%\iffalse
%<*samplefinal>
%\fi
%    \begin{macrocode}
\def\version{final}
\input{childdoc.def}
\childdocforwardprefix[cdocsamp]{cdocsfn}{cdocsch}
%    \end{macrocode}

%\iffalse
%</samplefinal>
%\fi
%
% %%%%%%%%%%%%%%%%%%%%%%%%%%%%%%%%%%%%%%
% \paragraph{Command Line Processing.}
%
% The following three command lines generate the output files
% |cdocscld|, |cdocscl1| and |cdocscl2|
% which should be identical to
% |cdocsdrf|, |cdocsch1| and |cdocsfn2|, respectively:
% \begin{center}
% \begin{tabular}{l}
% |latex -jobname cdocscld \|\\
% |  "\def\version{draft}\input{childdoc.def}\childdocforward{cdocsamp}"|\\
% |latex -jobname cdocscl1 \|\\
% |  "\input{childdoc.def}\childdocforward[cdocsamp]{cdocsch1}"|\\
% |latex -jobname cdocscl2 \|\\
% |  "\def\version{final}\input{childdoc.def}\childdocforward{cdocsch2}"|
% \end{tabular}
% \end{center}
% Note that the trailing backslash on each first line
% merely continues the input to the second line
% (for convenient cut ant paste).
% Furthermore, the command |latex| can be replaced by any
% of its alternative versions such as |pdflatex|.
%
% %%%%%%%%%%%%%%%%%%%%%%%%%%%%%%%%%%%%%%%%%%%%%%%%%%%%%%%%%%%%%%%%%%%%%%%%%%%%%%
% %%%%%%%%%%%%%%%%%%%%%%%%%%%%%%%%%%%%%%%%%%%%%%%%%%%%%%%%%%%%%%%%%%%%%%%%%%%%%%
% \section{Implementation}
%\iffalse
%<*package>
%\fi
%
% This section describes the definitions file |childdoc.def|.

% The definitions cannot be loaded using |\usepackage| or |\RequirePackage|
% which has a mechanism to prevent loading a style file more than once.
% When loading the definitions by means of |\input|
% multiple instances have to be prevented manually:
%\iffalse
%This code needs to be before the `\ProvidesFile' directive
%which is defined at the beginning of this file.
%Therefore it is also placed there and commented out here.
%</package>
%<*discard>
%\fi
%    \begin{macrocode}
\ifdefined\childdocmain\endinput\fi
%    \end{macrocode}
%\iffalse
%</discard>
%<*package>
%\fi
%
% \macro{\ifchilddoc}
% \macro{\ifchilddocmanual}
% The conditional |\ifchilddoc| tells whether a
% child (true) or main (false) document is being compiled.
% The conditional |\ifchilddocmanual| tells whether
% the |\includeonly| mechanism is used (false) or
% the selection of child files must be performed manually (true).
% The definitions initialise to false:
%    \begin{macrocode}
\newif\ifchilddoc
\newif\ifchilddocmanual
%    \end{macrocode}

% \macro{\childdocname}
% \macro{\childdocjob}
% The macro |\childdocname| stores the name of the main document
% to be compiled. The macro |\childdocjob| stores the name of
% the document on which the \LaTeX{} compiler was originally invoked.
% The content of |\jobname| cannot be compared
% to filenames specified in the source due to different catcodes.
% The following code rescans |\jobname|, stores the result
% in |\childdocname| and saves a copy in |\childdocjob|:
%    \begin{macrocode}
\edef\childdocname{\scantokens\expandafter{\jobname\noexpand}}
\let\childdocjob\childdocname
%    \end{macrocode}

% \macro{\childdocdisable}
% The macro |\childdocdisable| prevents the main file
% from being processed more than once.
% At this stage, the main document command |\childdocmain|
% is assumed to be called once again where it should do nothing.
% Any subsequent call to it should prevent
% a secondary processing of the main document
% It overwrites the forwarding commands
% |\childdocof| and |\childdocforward|
% with empty macros to prevent further inclusions of the main document:
%    \begin{macrocode}
\newcommand{\childdocdisable}
{
  \renewcommand{\childdocmain}[1]{\renewcommand{\childdocmain}[1]{\endinput}}
  \renewcommand{\childdocof}[1]{}
  \renewcommand{\childdocby}[2][]{}
  \renewcommand{\childdocforward}[2][]{}
  \renewcommand{\childdocdisable}{}
}
%    \end{macrocode}

% \macro{\childdocmain}
% The macro |\childdocmain| is to be called at the top of the main file
% with nothing or the main filename (without extension) as argument.
% First, it breaks loops.
% If the argument is not empty and does not match |\childdocname|
% (which is set by the first inclusion of |childdoc.def|),
% |\ifchilddoc| is set to true, |\includeonly| is applied to the child file
% and |\jobname| is set to the main file
% (for proper handling of |.aux| files):
%    \begin{macrocode}
\newcommand{\childdocmain}[1]
{
  \childdocdisable\childdocmain{}
  \if?#1?\else
    \begingroup
      \def\childdoctmp{#1}
      \ifx\childdoctmp\childdocname
        \def\childdoctmp{}
      \else
        \def\childdoctmp
        {
          \childdoctrue
          \includeonly{\childdocname}
          \def\childdocjob{#1}
          \def\jobname{#1}
        }
      \fi
      \expandafter
    \endgroup
    \childdoctmp
  \fi
}
%    \end{macrocode}

% \macro{\childdocof}
% The command |\childdocof| redirects
% compilation to the main file |#1|.
%    \begin{macrocode}
\newcommand{\childdocof}[1]
{
  \childdocdisable
  \childdoctrue
  \includeonly{\childdocname}
  \def\jobname{#1}
  \def\childdocjob{#1}
  \input{#1}
}
%    \end{macrocode}

% \macro{\childdocby}
% The command |\childdocby| ....
%    \begin{macrocode}
\newcommand{\childdocby}[2][]
{
  \childdocdisable
  \childdoctrue
  \childdocmanualtrue
  \if?#1?\else
    \def\jobname{#2}
  \fi
  \def\childdocjob{#2}
  \input{#2}
  \endinput
}
%    \end{macrocode}

% \macro{\childdocforward}
% The command |\childdocforward| redirects
% compilation to the main file or
% (if the optional argument is given) a child file.
% Parameters are set as if the main file
% or a child file starting with |\childdocof| was compiled.
% Then compilation is handed over to the main file:
%    \begin{macrocode}
\newcommand{\childdocforward}[2][]
{
  \begingroup
    \if?#1?
      \def\childdoctmp
      {
        \def\childdocname{#2}
        \def\childdocjob{#2}
        \def\jobname{#2}
        \input{#2}
        \endinput
      }
    \else
      \def\childdoctmp
      {
        \childdocdisable
        \def\childdocname{#2}
        \childdoctrue
        \includeonly{#2}
        \def\childdocjob{#1}
        \def\jobname{#1}
        \input{#1}
        \endinput
      }
    \fi
    \expandafter
  \endgroup
  \childdoctmp
}
%    \end{macrocode}

% \macro{\childdocforwardprefix}
% The command |\childdocforwardprefix| redirects
% compilation to the main or a child file by means of a pattern.
% The prefix |#1| in the current filename is replaced by |#2|
% and the suffix of the current filename is kept
% (it is assumed that the filename does not contain the substring `|~~~|'
% which is used as a delimiter).
% Compilation is handed over to the new file by |\childdocforward|:
%    \begin{macrocode}
\newcommand{\childdocforwardprefix}[3][]
{
  \begingroup
    \def\childdocextract #2##1~~~{\def\childdoctmp{\childdocforward[#1]{#3##1}}}
    \expandafter\childdocextract\childdocname~~~
    \expandafter
  \endgroup
  \childdoctmp
}
%    \end{macrocode}

% \macro{\childdoc}
% The deprecated macro |\childdoc| is a legacy version of |\childdocmain|:
%    \begin{macrocode}
\newcommand{\childdoc}{\childdocmain}
%    \end{macrocode}

% \macro{\childdocredirect}
% The deprecated macro |\childdocredirect| is a legacy version
% of |\childdocforward| and |\childdocforwardprefix|:
%    \begin{macrocode}
\newcommand{\childdocredirect}[2][]
{
  \begingroup
    \if?#1?
      \def\childdoctmp{\childdocforward{#2}}
    \else
      \def\childdoctmp{\childdocforwardprefix{#1}{#2}}
    \fi
    \expandafter
  \endgroup
  \childdoctmp
}
%    \end{macrocode}

%\iffalse
%</package>
%\fi
%
\endinput
|\\
|\childdocby{|\textit{main}|}|\\
\end{tabular}
\end{center}
%
The directive |\childdocby| is similar to |\childdocof|
described in \secref{sec:include},
but the subsequent selection of content must be done manually.
To that end, both |\ifchilddoc| and |\ifchilddocmanual|
will be true upon processing of a part,
and the name of the part is stored in |\childdocname|.
Note that |\jobname| will be set to the filename of the current part
so that each part receives an individual |.aux| file
that does not interfere with the |.aux| file(s) of the main document.
This behaviour can be altered by the alternative form
|\childdocby[*]{|\textit{main}|}| (with a non-empty optional argument)
which uses the |.aux| file of the main document
by setting |\jobname| to \textit{main}.

%%%%%%%%%%%%%%%%%%%%%%%%%%%%%%%%%%%%%%%%%%%%%%%%%%%%%%%%%%%%%%%%%%%%%%%%%%%%%%%%
\subsection{Driver Development}
\label{sec:driver}

The \textsf{childdoc} mechanism can also be use for the development
of definition files such as \LaTeX{} styles or classes.
This case differs from the above setup with multiple parts
included by |\include| in that no |\includeonly| should be invoked.
This can be achieved by starting the include file
(before |\ProvidesPackage|) with:
%
\begin{center}
\begin{tabular}{l}
|% \iffalse
%
% childdoc.dtx Copyright (C) 2017-2018 Niklas Beisert
%
% This work may be distributed and/or modified under the
% conditions of the LaTeX Project Public License, either version 1.3
% of this license or (at your option) any later version.
% The latest version of this license is in
%   http://www.latex-project.org/lppl.txt
% and version 1.3 or later is part of all distributions of LaTeX
% version 2005/12/01 or later.
%
% This work has the LPPL maintenance status `maintained'.
%
% The Current Maintainer of this work is Niklas Beisert.
%
% This work consists of the files childdoc.dtx and childdoc.ins
% and the derived files childdoc.def and cdocsamp.tex with
% cdocsch1.tex, cdocsch2.tex, cdocsdrf.tex, cdocsfn1.tex, cdocsfn2.tex.
%
%<package>\ifdefined\childdocmain\endinput\fi
%<package>\ProvidesFile{childdoc.def}[2018/12/30 v2.0 child document driver]
%<samplemain>\ProvidesFile{cdocsamp.tex}[2018/12/30 v2.0 sample for childdoc]
%<*driver>
%\ProvidesFile{childdoc.drv}[2018/12/30 v2.0 childdoc reference manual file]
\PassOptionsToClass{10pt,a4paper}{article}
\documentclass{ltxdoc}

\usepackage[margin=35mm]{geometry}
\usepackage{hyperref}
\usepackage{hyperxmp}
\usepackage[usenames]{color}

\hypersetup{colorlinks=true}
\hypersetup{pdfstartview=FitH}
\hypersetup{pdfpagemode=UseNone}
\hypersetup{pdfsource={}}
\hypersetup{pdflang={en-UK}}
\hypersetup{pdfcopyright={Copyright 2017-2018 Niklas Beisert.
  This work may be distributed and/or modified under the
  conditions of the LaTeX Project Public License, either version 1.3
  of this license or (at your option) any later version.}}
\hypersetup{pdflicenseurl={http://www.latex-project.org/lppl.txt}}
\hypersetup{pdfcontactaddress={ETH Zurich, ITP, HIT K,
  Wolfgang-Pauli-Strasse 27}}
\hypersetup{pdfcontactpostcode={8093}}
\hypersetup{pdfcontactcity={Zurich}}
\hypersetup{pdfcontactcountry={Switzerland}}
\hypersetup{pdfcontactemail={nbeisert@itp.phys.ethz.ch}}
\hypersetup{pdfcontacturl={http://people.phys.ethz.ch/\xmptilde nbeisert/}}

\newcommand{\secref}[1]{\hyperref[#1]{section \ref*{#1}}}

\parskip1ex
\parindent0pt
\let\olditemize\itemize
\def\itemize{\olditemize\parskip0pt}

\begin{document}

\title{The \textsf{childdoc} Package}
\hypersetup{pdftitle={The childdoc Package}}
\author{Niklas Beisert\\[2ex]
  Institut f\"ur Theoretische Physik\\
  Eidgen\"ossische Technische Hochschule Z\"urich\\
  Wolfgang-Pauli-Strasse 27, 8093 Z\"urich, Switzerland\\[1ex]
  \href{mailto:nbeisert@itp.phys.ethz.ch}
  {\texttt{nbeisert@itp.phys.ethz.ch}}}
\hypersetup{pdfauthor={Niklas Beisert}}
\hypersetup{pdfsubject={Manual for the LaTeX2e Package childdoc}}
\date{30 December 2018, \textsf{v2.0}}
\maketitle

\begin{abstract}\noindent
\textsf{childdoc} is a \LaTeXe{} package
that enables the direct compilation
of document sections included by |\include|
to individual files.
\end{abstract}

\begingroup
\parskip0ex
\tableofcontents
\endgroup

%%%%%%%%%%%%%%%%%%%%%%%%%%%%%%%%%%%%%%%%%%%%%%%%%%%%%%%%%%%%%%%%%%%%%%%%%%%%%%%%
%%%%%%%%%%%%%%%%%%%%%%%%%%%%%%%%%%%%%%%%%%%%%%%%%%%%%%%%%%%%%%%%%%%%%%%%%%%%%%%%
\section{Introduction}

\LaTeX{} provides a mechanism to structure a large document (such as a book)
into a main file and several child files (containing the chapters)
using the |\include| command.
This mechanism is beneficial for documents
which span hundreds of pages in order to
make the source file(s) more manageable.
Moreover, compilation can be restricted to
selected child files by means of the |\includeonly| command.
The latter feature can be used to reduce the compilation time while editing
(this was significantly more useful in the earlier days of \LaTeX{})
or to generate a smaller document which is easier to navigate.
Another application of |\includeonly| is to generate
documents consisting of selected parts of the complete document.

However, there are a few drawbacks of the plain |\include| mechanism:
\begin{itemize}
\item
The child files cannot be compiled on their own,
they can only be compiled via the main file.
A naive editing environment
(such as a text editor with an option
to have the current file processed by \LaTeX)
may require one to switch to the main file before compiling;
attempting to compile the child file produces errors.
\item
The main file must be modified (each time)
to adjust the |\includeonly| command
to the present needs. This easily leaves the main file in a messy state.
\item
The generated document will always carry the filename
of the main document. This is inconvenient if
several child files are to be compiled and
to be kept for distribution.
\end{itemize}

The present package provides a simple interface
to make child files individually compilable by \LaTeX{}.
Compiling a child file then has the same effect as compiling
the main file with an |\includeonly| command
to select the appropriate child.
Moreover the generated document will carry the name of the child
rather than the main file.
This resolves all three above issues.

This feature is meant to make the editing of books,
thesis documents and lecture notes somewhat more convenient.
However, the package can also be used efficiently for
composing a series of documents (such as exercise sheets)
which are typically distributed individually.
It then assists the author in generating the individual documents
(potentially in different versions)
as well as a document containing the collected series.
Another application is in developing style files
or other kinds of included material
where compilation of the style file could redirect
to a sample or test file.

%%%%%%%%%%%%%%%%%%%%%%%%%%%%%%%%%%%%%%%%%%%%%%%%%%%%%%%%%%%%%%%%%%%%%%%%%%%%%%%%
%%%%%%%%%%%%%%%%%%%%%%%%%%%%%%%%%%%%%%%%%%%%%%%%%%%%%%%%%%%%%%%%%%%%%%%%%%%%%%%%
\section{Usage}

First of all, the package \textsf{childdoc} is \emph{not} a standard
\LaTeXe{} |.sty| style file! Therefore it needs to be invoked in
a non-standard way.

%%%%%%%%%%%%%%%%%%%%%%%%%%%%%%%%%%%%%%%%%%%%%%%%%%%%%%%%%%%%%%%%%%%%%%%%%%%%%%%%
\subsection{Included Files}
\label{sec:include}

%%%%%%%%%%%%%%%%%%%%%%%%%%%%%%%%%%%%%%%%
\DescribeMacro{\childdocmain}
To use the package, add the commands
\begin{center}
\begin{tabular}{l}
|\input{childdoc.def}|\\
|\childdocmain{}|\\
\end{tabular}
\end{center}
at the very top of the main \LaTeX{} file,
in particular \emph{before} the |\documentclass| statement!
The argument of |\childdocmain| should be left empty
(but it must be present).

%%%%%%%%%%%%%%%%%%%%%%%%%%%%%%%%%%%%%%%%
\DescribeMacro{\childdocof}
Furthermore, add the commands
\begin{center}
\begin{tabular}{l}
|\input{childdoc.def}|\\
|\childdocof{|\textit{main}|}|\\
\end{tabular}
\end{center}
at the top of every child file \textit{child}
which is included by |\include{|\textit{child}|}|
from within the main file
(or at least for those files to be compiled individually).
The argument \textit{main} must be the filename of the main file.

There are a couple of
considerations in setting up the main and child documents:

%%%%%%%%%%%%%%%%%%%%%%%%%%%%%%%%%%%%%%%%
\paragraph{Restrictions.}

Please note the following restrictions:
\begin{itemize}
\item
|\childdocmain| must be called with one argument \textit{main}
to ensure compatibility with earlier version of the package.
It must either be empty (|\childdocmain{}|)
or precisely match the filename of the main file in which it is specified.
See \secref{sec:detection} for further information.
\item
The filename \textit{main} must be specified without the |.tex| extension.
\item
The filename \textit{main} is case sensitive
(even in case-insensitive file systems)
due to internal string comparison.
\item
The argument \textit{main} should be fully expanded, it cannot be a macro.
\item
Subdirectories and special characters should be avoided in filenames.
\item
The command |\childdocmain{|\textit{main}|}| must be followed by a whitespace.
It should not be followed immediately by another command
or by a comment mark `|%|'.
This is because the \TeX{} parser reads the token immediately following
the argument of |\childdocmain| and puts it
at the beginning of every child section;
however, a white\-space is ignored.
\end{itemize}

%%%%%%%%%%%%%%%%%%%%%%%%%%%%%%%%%%%%%%%%
\paragraph{Content of Main File.}

It is advisable to place all content in the child files included by |\include|.
Any output contained in the main file will appear in all child documents
unless suppressed manually;
it cannot be suppressed automatically by the |\includeonly| directive
and thus should normally be avoided.
A method to include some content in the main file
by means of conditional processing is described in \secref{sec:conditional}.

%%%%%%%%%%%%%%%%%%%%%%%%%%%%%%%%%%%%%%%%
\paragraph{Page Numbering.}

When only a part of the document is compiled,
the appropriate numbering of pages
(as well as other status parameters)
is determined from the |.aux| files.
The latter contain information from previous passes.
However this information needs to propagate through
all intermediate child documents.
Therefore the page numbering in child documents may well
be inconsistent until the complete document is compiled at least once.

A useful (if unconventional) way to always ensure a consistent
page numbering is to restart the numbering in each child document
and denote the pages by `\textit{child}|.|\textit{page}'
where \textit{child} represents the chapter/section number of the child file.
This can be achieved by the command
|\numberwithin{page}{|\textit{child}|}|
of the \textsf{amsmath} package
where \textit{child} can be |chapter| or |section|
depending on the chosen structuring.
Alternatively, one can modify the macro |\thepage| appropriately
and reset the counter |page| at the start of each child file.

%%%%%%%%%%%%%%%%%%%%%%%%%%%%%%%%%%%%%%%%%%%%%%%%%%%%%%%%%%%%%%%%%%%%%%%%%%%%%%%%
\subsection{Conditional Processing}
\label{sec:conditional}

The package provides a mechanism to compile different versions
of a document. To customise the versions further some conditional processing
can come in handy to distinguish which version is being compiled.
The package provides two macros to describe the compilation context:

%%%%%%%%%%%%%%%%%%%%%%%%%%%%%%%%%%%%%%%%
\DescribeMacro{\ifchilddoc}
The conditional |\ifchilddoc| distinguishes between the compilation of
child documents and the main document:
%
\begin{center}
|\ifchilddoc |\textit{child-code}| |[|\||else |\textit{main-code}]| \||fi|
\end{center}

%%%%%%%%%%%%%%%%%%%%%%%%%%%%%%%%%%%%%%%%
\DescribeMacro{\childdocname}
\DescribeMacro{\childdocjob}
The macro |\childdocname| contains the filename (without extension)
of the main or child file being processed.
Note that |\childdocjob| will always contain the name of the main file.

%%%%%%%%%%%%%%%%%%%%%%%%%%%%%%%%%%%%%%%%
\paragraph{Title Page.}

Conditional processing can be used to include a title or banner page
in the main document when proper precautions are taken.
Importantly, the code in the main file should ensure that the page counter
(as well as other status parameters which are stored in the |.aux| files)
takes the same value after the conditional processing.
Otherwise the page numbers may take divergent values
depending on which part is compiled.

For example, a title page could be declared by:
%
\begin{center}
\begin{tabular}{l}
|\ifchilddoc\||else|\\
|\addtocounter{page}{-1}|\\
\textit{code for title page}\\
|\newpage|\\
|\||fi|
\end{tabular}
\end{center}
%
A banner page for the child documents can be generated by:
%
\begin{center}
\begin{tabular}{l}
|\ifchilddoc|\\
|\addtocounter{page}{-1}|\\
\textit{code for banner page}\\
|\newpage|\\
|\||fi|
\end{tabular}
\end{center}
%
Here one could write a message such as:
\begin{center}
|This is the part \childdocname{} of \childdocjob{}.|
\end{center}

%%%%%%%%%%%%%%%%%%%%%%%%%%%%%%%%%%%%%%%%%%%%%%%%%%%%%%%%%%%%%%%%%%%%%%%%%%%%%%%%
\subsection{Flags}
\label{sec:flags}

The package makes it easy to generate different versions
of the main or child documents.
To this end compilation flags can be defined
and assigned different default values.
They will be particularly useful in conjunction
with the forwarding mechanism described in \secref{sec:forward}.

For example, it may be useful to have a flag |\version|
which can be set to |draft| or |final|.
The document source will contain some conditional code
depending on the value of |\version|.
Suppose further, the flag should default to |final| for the main file
and to |draft| for child files
which is a natural assignment for editing the document.
This is achieved by placing the following code
in the preamble of the main document
(below the |\childdocmain| directive):
%
\begin{center}
\begin{tabular}{l}
|\ifchilddoc|\\
|\providecommand{\version}{draft}|\\
|\||else|\\
|\providecommand{\version}{final}|\\
|\||fi|
\end{tabular}
\end{center}
%
The definition by |\providecommand| makes sure
that previous definitions are not overwritten.
Further statements |\providecommand{\version}{...}|
can thus be added before the above code to override it.

For the main file, one might add a line
(between |\childdocmain| and the above block)
%
\begin{center}
|%\ifchilddoc\||else\providecommand{\version}{draft}\||fi|
\end{center}
%
which can be uncommented to produce a draft version.
Likewise one can add a line to the very top of a child file
(above the |\childdocof{|\textit{main}|}| directive)
%
\begin{center}
|%\providecommand{\version}{final}|
\end{center}
%
which can be uncommented to produce the final version of this child document.

%%%%%%%%%%%%%%%%%%%%%%%%%%%%%%%%%%%%%%%%%%%%%%%%%%%%%%%%%%%%%%%%%%%%%%%%%%%%%%%%
\subsection{Forwarding}
\label{sec:forward}

Different versions of the main or child documents
using compilation flags as described in \secref{sec:flags}
can be (permanently) stored in different files
for convenient compilation, viewing and distribution.
To this end, the package defines a command
to pass on compilation to a different file:

%%%%%%%%%%%%%%%%%%%%%%%%%%%%%%%%%%%%%%%%
\DescribeMacro{\childdocforward}
The command |\childdocforward| redirects processing to
another source file:
%
\begin{center}
\begin{tabular}{l}
|\input{childdoc.def}|\\
|\childdocforward[|\textit{main}|]{|\textit{dest}|}|\\
\end{tabular}
\end{center}
%
The argument \textit{dest} is the destination file
(without extension).
It should be the main file or one of the child files.
Note that further \textsf{childdoc} directives
such as |\childdocof| and |\childdocforward|
in the indicated file will be processed in this form.
The optional argument \textit{main}
passes on directly to the main file \textit{main}
while pretending to compile the child \textit{dest}.
This form behaves as if \textit{dest}
issues |\childdocof{|\textit{main}|}| right away,
and no further \textsf{childdoc} directives will be processed.

%%%%%%%%%%%%%%%%%%%%%%%%%%%%%%%%%%%%%%%%
\DescribeMacro{\...prefix}
In the alternative form |\childdocforwardprefix|,
%
\begin{center}
\begin{tabular}{l}
|\input{childdoc.def}|\\
|\childdocforwardprefix[|\textit{main}|]{|\textit{prefix}|}{|\textit{dest}|}|
\end{tabular}
\end{center}
%
the destination file is determined by a pattern
depending on the current file:
To make this work, the current file must be called
`{\textit{prefix}\hspace{0.2em}\textit{suffix}}'
with \textit{prefix} matching precisely the argument.
Processing is then passed on to the file
`{\textit{dest}\hspace{0.2em}\textit{suffix}}'.
Surely, the same effect is achieved by
directly specifying the
argument `{\textit{dest}\hspace{0.2em}\textit{suffix}}'
in the first form.
However, that requires to set up a different file
for each child. With the alternative form of the command
all these files can have exactly the same content
which simplifies setting them up and maintaining them.

For example, the following file |draft.tex|
with a compilation flag |\version| as described in \secref{sec:flags}
compiles the main document as a draft:
%
\begin{center}
\begin{tabular}{l}
|\def\version{draft}|\\
|\input{childdoc.def}|\\
|\childdocforward{|\textit{main}|}|
\end{tabular}
\end{center}
%
Likewise, the following files |final|\textit{nn}|.tex|
compile the final version of the child document
|child|\textit{nn}|.tex|:
%
\begin{center}
\begin{tabular}{l}
|\def\version{final}|\\
|\input{childdoc.def}|\\
|\childdocforwardprefix{final}{child}|
\end{tabular}
\end{center}
%

Note that when several versions of a main file and/or of each child file
are to be generated, it may be convenient to set up a |Makefile| or
shell script to automatise the process.

%%%%%%%%%%%%%%%%%%%%%%%%%%%%%%%%%%%%%%%%%%%%%%%%%%%%%%%%%%%%%%%%%%%%%%%%%%%%%%%%
\subsection{Command Line Processing}
\label{sec:commandline}

The effect of redirection files can also be achieved by invoking
the \LaTeX{} compiler with a more elaborate command line.
Most conveniently this should be done as part
of a shell script or a |Makefile|.

When using \textsf{childdoc} in the main file, the following
command lines effectively perform a redirection
(note that depending on the shell being used,
backslashes may have to be doubled: `|\|' $\to$ `|\\|'):
%
\begin{center}
|... -jobname "|\textit{target}|" |\\|"|[\textit{flags}]%
|\input{childdoc.def}\childdocforward[|\textit{main}|]{|\textit{dest}|}"|
\end{center}
%
Here \textit{target} is the name of the output file,
\textit{main} is the name of the main file
and \textit{dest} is the name of the main or child file to be processed
(all filenames without extensions).
The optional argument \textit{main} can be omitted
if \textit{main} matches \textit{dest}.
Optionally, compilation \textit{flags} can be defined via |\def| commands.
This command line makes the \TeX{} engine believe
it is compiling the file \textit{target}
whose content is specified as the latter parameter.
The provided code then forwards the processing to
\textit{main} or \textit{dest} as described in \secref{sec:forward}.

%%%%%%%%%%%%%%%%%%%%%%%%%%%%%%%%%%%%%%%%%%%%%%%%%%%%%%%%%%%%%%%%%%%%%%%%%%%%%%%%
\subsection{Include by Input}
\label{sec:input}

Including child documents by |\include| has some restrictions by design.
Most notably, the content of a child document always occupies
its own set of pages; pages cannot be shared between child documents.
Usually, this behaviour makes perfect sense
because each child document contain an essential part of the document.
However, in some situations it may be desirable to compose
a document from a collection of parts
without having mandatory page breaks between then.
For this case, the package
provides a mechanism to include parts
by |\input| which can also be processed individually.
However, by construction this mechanism
requires manual handling of the content to be output.

%%%%%%%%%%%%%%%%%%%%%%%%%%%%%%%%%%%%%%%%
\DescribeMacro{\ifchilddocmanual}
The main file should be prepared as usual, see \secref{sec:include}.
However, the document body must make a distinction
between processing of an individual part and of the main document, e.g.:
%
\begin{center}
\begin{tabular}{l}
|\ifchilddocmanual|\\
|\input{\childdocname}|\\
|\||else|\\
\textit{document body with }|\input{|\textit{part}|}|\\
|\||fi|
\end{tabular}
\end{center}
%
The conditional |\ifchilddocmanual| is true whenever
a part to be included by |\input| is being compiled,
and the name of the part is stored in |\childdocname|.

%%%%%%%%%%%%%%%%%%%%%%%%%%%%%%%%%%%%%%%%
\DescribeMacro{\childdocby}
Each part to be included by |\input| should start with:
%
\begin{center}
\begin{tabular}{l}
|\input{childdoc.def}|\\
|\childdocby{|\textit{main}|}|\\
\end{tabular}
\end{center}
%
The directive |\childdocby| is similar to |\childdocof|
described in \secref{sec:include},
but the subsequent selection of content must be done manually.
To that end, both |\ifchilddoc| and |\ifchilddocmanual|
will be true upon processing of a part,
and the name of the part is stored in |\childdocname|.
Note that |\jobname| will be set to the filename of the current part
so that each part receives an individual |.aux| file
that does not interfere with the |.aux| file(s) of the main document.
This behaviour can be altered by the alternative form
|\childdocby[*]{|\textit{main}|}| (with a non-empty optional argument)
which uses the |.aux| file of the main document
by setting |\jobname| to \textit{main}.

%%%%%%%%%%%%%%%%%%%%%%%%%%%%%%%%%%%%%%%%%%%%%%%%%%%%%%%%%%%%%%%%%%%%%%%%%%%%%%%%
\subsection{Driver Development}
\label{sec:driver}

The \textsf{childdoc} mechanism can also be use for the development
of definition files such as \LaTeX{} styles or classes.
This case differs from the above setup with multiple parts
included by |\include| in that no |\includeonly| should be invoked.
This can be achieved by starting the include file
(before |\ProvidesPackage|) with:
%
\begin{center}
\begin{tabular}{l}
|\input{childdoc.def}|\\
|\childdocforward{|\textit{main}|}|\\
\end{tabular}
\end{center}
%
or alternatively with:
%
\begin{center}
\begin{tabular}{l}
|\input{childdoc.def}|\\
|\childdocby{|\textit{main}|}|\\
\end{tabular}
\end{center}
%
Both forms have slightly different effects as described above.
The main file is prepared as usual, see \secref{sec:include}.

%%%%%%%%%%%%%%%%%%%%%%%%%%%%%%%%%%%%%%%%%%%%%%%%%%%%%%%%%%%%%%%%%%%%%%%%%%%%%%%%
\subsection{Legacy Detection}
\label{sec:detection}

The directive |\childdocmain| in the main file can detect
whether the complete document or merely a child is to be compiled
even without using the directive |\childdocof|.
This method is deprecated because it is less robust
and there is no compelling reason to use it;
it is merely provided for backward compatibility
and it may be removed in future versions.

If the detection mechanism is to be used,
it is mandatory to correctly specify
the filename of the main file as the argument of |\childdocmain|:
%
\begin{center}
\begin{tabular}{l}
|\input{childdoc.def}|\\
|\childdocmain{|\textit{main}|}|\\
\end{tabular}
\end{center}
%
If |\jobname| does not match the argument \textit{main} of |\childdocmain|,
it is assumed that |\jobname| points to the child file to be compiled.
When using |\childdocmain| with the main file specified as argument,
it suffices to start a child file
with just |\input{|\textit{main}|}|
without loading of the package and using |\childdocof|.
If instead all processing is done
with the appropriate \textsf{childdoc} directives,
the argument of \textit{main} of |\childdocmain| can be empty.

An alternative version of the command line processing described
in \secref{sec:commandline} using the detection mechanism reads:
%
\begin{center}
|... -jobname "|\textit{target}|" "|[\textit{flags}]%
[|\def\jobname{|\textit{dest}|}|]|\input{|\textit{main}|}"|
\end{center}

%%%%%%%%%%%%%%%%%%%%%%%%%%%%%%%%%%%%%%%%%%%%%%%%%%%%%%%%%%%%%%%%%%%%%%%%%%%%%%%%
\subsection{Manual Code}
\label{sec:manual}

In case one cannot be certain whether the definitions file |childdoc.def|
is installed on the target \TeX{} distribution
and one prefers not to ship it,
it is conceivable to paste a few relevant commands into the sources.

To that end, drop all statements |\input{childdoc.def}|
and perform the replacements as outlined below.
Instead of |\childdocmain{|\textit{main}|}| add the following code
to the top of the main file:
%
\begin{center}
\begin{tabular}{l}
|\||ifdefined\childdocname\endinput\||fi\newif\ifchilddoc|\\
|\edef\childdocname{\scantokens\expandafter{\jobname\noexpand}}|\\
|\def\childdocmain{|\textit{main}|}\||ifx\childdocmain\childdocname\||else|\\
|\childdoctrue\includeonly{\childdocname}\let\jobname\childdocmain\||fi|\\
\end{tabular}
\end{center}
%
Instead of |\childdocof{|\textit{main}|}| just include the main file
at the top of each child file:
%
\begin{center}
|\input{|\textit{main}|}|
\end{center}
%
A simple redirection |\childdocforward{|\textit{dest}|}| is achieved by:
%
\begin{center}
|\def\jobname{|\textit{dest}|}\input{\jobname}|
\end{center}
%
The redirection with prefix
|\childdocforwardprefix[|\textit{prefix}|]{|\textit{dest}|}|
is accomplished by:
%
\begin{center}
\begin{tabular}{l}
|{\edef\jobname{\scantokens\expandafter{\jobname\noexpand}}|\\
|\def\redirectjob |\textit{prefix}|#1~~~{\gdef\jobname{|\textit{dest}|#1}}|\\
|\expandafter\redirectjob\jobname~~~}\input{\jobname}|
\end{tabular}
\end{center}

In an alternative approach,
child documents can be compiled by a specific command line
without additional code or specific definitions:
%
\begin{center}
|... -jobname "|\textit{target}|" "|[\textit{flags}]%
|\includeonly{|\textit{dest}|}\input{|\textit{main}|}"|
\end{center}
%

%%%%%%%%%%%%%%%%%%%%%%%%%%%%%%%%%%%%%%%%%%%%%%%%%%%%%%%%%%%%%%%%%%%%%%%%%%%%%%%%
%%%%%%%%%%%%%%%%%%%%%%%%%%%%%%%%%%%%%%%%%%%%%%%%%%%%%%%%%%%%%%%%%%%%%%%%%%%%%%%%
\section{Information}

%%%%%%%%%%%%%%%%%%%%%%%%%%%%%%%%%%%%%%%%%%%%%%%%%%%%%%%%%%%%%%%%%%%%%%%%%%%%%%%%
\subsection{Copyright}

Copyright \copyright{} 2017--2018 Niklas Beisert

This work may be distributed and/or modified under the
conditions of the \LaTeX{} Project Public License, either version 1.3
of this license or (at your option) any later version.
The latest version of this license is in
  \url{http://www.latex-project.org/lppl.txt}
and version 1.3 or later is part of all distributions of \LaTeX{}
version 2005/12/01 or later.

This work has the LPPL maintenance status `maintained'.

The Current Maintainer of this work is Niklas Beisert.

This work consists of the files |README.txt|, |childdoc.ins| and |childdoc.dtx|
as well as the derived files |childdoc.def|, |cdocsamp.tex|
with |cdocsch1.tex|, |cdocsch2.tex|, |cdocspt3.tex|, |cdocspt4.tex|,
|cdocsdrf.tex|, |cdocsfn1.tex|, |cdocsfn2.tex|
as well as |childdoc.pdf|.

%%%%%%%%%%%%%%%%%%%%%%%%%%%%%%%%%%%%%%%%%%%%%%%%%%%%%%%%%%%%%%%%%%%%%%%%%%%%%%%%
\subsection{Files and Installation}

The package consists of the files:
%
\begin{center}
\begin{tabular}{ll}
    |README.txt|   & readme file \\
    |childdoc.ins| & installation file \\
    |childdoc.dtx| & source file \\
    |childdoc.def| & definition file \\
    |cdocsamp.tex| & sample main file \\
    |cdocsch1.tex| & sample include file \\
    |cdocsch2.tex| & sample include file \\
    |cdocspt3.tex| & sample part file \\
    |cdocspt4.tex| & sample part file \\
    |cdocsdrf.tex| & sample redirection file \\
    |cdocsfn1.tex| & sample redirection file \\
    |cdocsfn2.tex| & sample redirection file \\
    |childdoc.pdf| & manual
\end{tabular}
\end{center}
%
The distribution consists of the files
|README.txt|, |childdoc.ins| and |childdoc.dtx|.
%
\begin{itemize}
\item
Run (pdf)\LaTeX{} on |childdoc.dtx|
to compile the manual |childdoc.pdf| (this file).
\item
Run \LaTeX{} on |childdoc.ins| to create the definitions file |childdoc.def|
and the sample |cdocsamp.tex| with include files
|cdocsch1.tex|, |cdocsch2.tex|, |cdocspt3.tex|, |cdocspt4.tex|,
|cdocsdrf.tex|, |cdocsfn1.tex|, |cdocsfn2.tex|.
Then copy the file |childdoc.def| to an appropriate directory of your \LaTeX{}
distribution, e.g.\ \textit{texmf-root}|/tex/latex/childdoc|.
\end{itemize}

%%%%%%%%%%%%%%%%%%%%%%%%%%%%%%%%%%%%%%%%%%%%%%%%%%%%%%%%%%%%%%%%%%%%%%%%%%%%%%%%
\subsection{Related CTAN Packages}

There are several other packages which offer a similar functionality:
%
\begin{itemize}
\item
The packages
\href{http://ctan.org/pkg/docmute}{\textsf{docmute}},
\href{http://ctan.org/pkg/includex}{\textsf{includex}} and
\href{http://ctan.org/pkg/standalone}{\textsf{standalone}}
provide commands to include only the document body of
a child file thus allowing both files to be compiled individually.
\item
The packages \href{http://ctan.org/pkg/subdocs}{\textsf{subdocs}}
and \href{http://ctan.org/pkg/subfiles}{\textsf{subfiles}}
provide structures in which the main and child documents can be
encapsulated and allowing them to be compiled individually.
The inclusion mechanism is different from the conventional |\include|.
\item
The package \href{http://ctan.org/pkg/combine}{\textsf{combine}}
is an elaborate solution to combine several documents into one.
\end{itemize}
%
See also the CTAN topic \href{http://ctan.org/topic/subdocs}{\textsf{subdocs}}
for further related packages.
The present package differs from the above solutions in that
a document structure constructed with the conventional |\include| mechanism
just needs two extra commands at the top of every file
such that all constituent files can be compiled individually.

%%%%%%%%%%%%%%%%%%%%%%%%%%%%%%%%%%%%%%%%%%%%%%%%%%%%%%%%%%%%%%%%%%%%%%%%%%%%%%%%
%\subsection{Feature Suggestions}
%
%The following is a list of features which may be useful for future
%versions of this package:
%%
%\begin{itemize}
%\item
%\ldots
%\end{itemize}

%%%%%%%%%%%%%%%%%%%%%%%%%%%%%%%%%%%%%%%%%%%%%%%%%%%%%%%%%%%%%%%%%%%%%%%%%%%%%%%%
\subsection{Revision History}

%%%%%%%%%%%%%%%%%%%%%%%%%%%%%%%%%%%%%%%%
\paragraph{v2.0:} 2018/12/30

\begin{itemize}
\item
immediate forward processing
\item
added |\childdocby| mechanism
\item
manual restructured
\end{itemize}

%%%%%%%%%%%%%%%%%%%%%%%%%%%%%%%%%%%%%%%%
\paragraph{v1.6:} 2018/01/17

\begin{itemize}
\item
application for development of include files
\item
corrections to manual
\end{itemize}

%%%%%%%%%%%%%%%%%%%%%%%%%%%%%%%%%%%%%%%%
\paragraph{v1.5:} 2017/05/21

\begin{itemize}
\item
more complete structuring introduced
\item
|\childdocof| introduced
\item
|\childdoc| renamed to |\childdocmain|
\item
|\childredirect| renamed to |\childdocforward| and |\childdocforwardprefix|
and functionality expanded
\end{itemize}

%%%%%%%%%%%%%%%%%%%%%%%%%%%%%%%%%%%%%%%%
\paragraph{v1.0:} 2017/04/27

\begin{itemize}
\item
manual and install package
\item
first version published on CTAN
\end{itemize}

%%%%%%%%%%%%%%%%%%%%%%%%%%%%%%%%%%%%%%%%
\paragraph{v0.6:} 2017/04/26

\begin{itemize}
\item
redirection mechanism added
\end{itemize}

%%%%%%%%%%%%%%%%%%%%%%%%%%%%%%%%%%%%%%%%
\paragraph{v0.5:} 2017/04/26

\begin{itemize}
\item
functionality in definition file
\end{itemize}


%%%%%%%%%%%%%%%%%%%%%%%%%%%%%%%%%%%%%%%%%%%%%%%%%%%%%%%%%%%%%%%%%%%%%%%%%%%%%%%%
%%%%%%%%%%%%%%%%%%%%%%%%%%%%%%%%%%%%%%%%%%%%%%%%%%%%%%%%%%%%%%%%%%%%%%%%%%%%%%%%
%%%%%%%%%%%%%%%%%%%%%%%%%%%%%%%%%%%%%%%%%%%%%%%%%%%%%%%%%%%%%%%%%%%%%%%%%%%%%%%%
\appendix

\settowidth\MacroIndent{\rmfamily\scriptsize 000\ }

 \DocInput{childdoc.dtx}

\end{document}
%</driver>
% \fi
%
% %%%%%%%%%%%%%%%%%%%%%%%%%%%%%%%%%%%%%%%%%%%%%%%%%%%%%%%%%%%%%%%%%%%%%%%%%%%%%%
% %%%%%%%%%%%%%%%%%%%%%%%%%%%%%%%%%%%%%%%%%%%%%%%%%%%%%%%%%%%%%%%%%%%%%%%%%%%%%%
% \section{Sample}
%\iffalse
%<*samplemain>
%\fi
%
% The following presents a sample document
% with two chapters, two parts, a title page,
% a compile flag as well as three forwarding files to set the flag.
% It consists of eight |.tex| files:
% \begin{center}
% \begin{tabular}{ll}
% |cdocsamp.tex|&main file\\
% |cdocsch1.tex|&include file for chapter 1\\
% |cdocsch2.tex|&include file for chapter 2\\
% |cdocspt3.tex|&include file for part 3\\
% |cdocspt4.tex|&include file for part 4\\
% |cdocsdrf.tex|&forwarding file for main file in draft mode\\
% |cdocsfi1.tex|&forwarding file for final version of chapter 1\\
% |cdocsfi2.tex|&forwarding file for final version of chapter 2\\
% \end{tabular}
% \end{center}
% Each of the eight files can be compiled directly by the \LaTeX{} compiler.
%
% %%%%%%%%%%%%%%%%%%%%%%%%%%%%%%%%%%%%%%
% \paragraph{Main File.}
%
% The main file is called |cdocsamp.tex|.
%
% Load the \textsf{childdoc} definitions and
% declare the filename for the main document:
%    \begin{macrocode}
\input{childdoc.def}
\childdocmain{}
%    \end{macrocode}

% Optional override for |\version| flag:
%    \begin{macrocode}
%%\ifchilddoc\else\providecommand{\version}{draft}\fi
%    \end{macrocode}

% Define the default values for the |\version| flag
% (|final| for the main file and |draft| for childs):
%    \begin{macrocode}
\ifchilddoc
\providecommand{\version}{draft}
\else
\providecommand{\version}{final}
\fi
%    \end{macrocode}

% Load the standard document class:
%    \begin{macrocode}
\documentclass[12pt]{article}
%    \end{macrocode}

% Start the document body:
%    \begin{macrocode}
\begin{document}
%    \end{macrocode}

% Declare a title page.
% Print title, part of document being processed and version flag:
%    \begin{macrocode}
\addtocounter{page}{-1}
\begin{center}
{\LARGE\bfseries{}childdoc example\par}
\vspace{1cm}
\ifchilddoc
\ifchilddocmanual part\else chapter\fi:
`\childdocname' of `\childdocjob'\par
\else
main document: `\childdocjob'\par
\fi
version: \version\par
\end{center}
\newpage
%    \end{macrocode}

% Manually include selected file,
% otherwise process as usual:
%    \begin{macrocode}
\ifchilddocmanual
\section*{part `\childdocname'}
\input{\childdocname}
\else
%    \end{macrocode}

% Include the two chapters:
%    \begin{macrocode}
\include{cdocsch1}
\include{cdocsch2}
%    \end{macrocode}

% Include the two parts unless only chapters should be displayed:
%    \begin{macrocode}
\ifchilddoc\else
\section{part three}
\input{cdocspt3}
\section{part four}
\input{cdocspt4}
\fi
%    \end{macrocode}

% Process as usual until here:
%    \begin{macrocode}
\fi
%    \end{macrocode}

% End of document body:
%    \begin{macrocode}
\end{document}
%    \end{macrocode}
%\iffalse
%</samplemain>
%\fi
%
% %%%%%%%%%%%%%%%%%%%%%%%%%%%%%%%%%%%%%%
% \paragraph{Chapter Include Files.}
%
% The include files are called |cdocsch1.tex| and |cdocsch2.tex|.
%
%\iffalse
%<*samplechap1|samplechap2>
%\fi

% Optional override for |\version| flag:
%    \begin{macrocode}
%%\providecommand{\version}{final}
%    \end{macrocode}

% Include the main document:
%    \begin{macrocode}
\input{childdoc.def}
\childdocof{cdocsamp}
%    \end{macrocode}

%\iffalse
%</samplechap1|samplechap2>
%\fi
%
%\iffalse
%<*samplechap1>
%\fi
% Some text for chapter 1:
%    \begin{macrocode}
\section{one}
some text in chapter one
%    \end{macrocode}

%\iffalse
%</samplechap1>
%\fi
% Some text for chapter 2:
%\iffalse
%<*samplechap2>
%\fi
%    \begin{macrocode}
\section{two}
more text in chapter two
%    \end{macrocode}

%\iffalse
%</samplechap2>
%\fi
%
% %%%%%%%%%%%%%%%%%%%%%%%%%%%%%%%%%%%%%%
% \paragraph{Part Include Files.}
%
% The include files are called |cdocspt3.tex| and |cdocspt4.tex|.
%
%\iffalse
%<*samplepart3|samplepart4>
%\fi

% Optional override for |\version| flag:
%    \begin{macrocode}
%%\providecommand{\version}{final}
%    \end{macrocode}

% Include the main document:
%    \begin{macrocode}
\input{childdoc.def}
\childdocby{cdocsamp}
%    \end{macrocode}

%\iffalse
%</samplepart3|samplepart4>
%\fi
%
%\iffalse
%<*samplepart3>
%\fi
% Some text for part 3:
%    \begin{macrocode}
some text in part three
%    \end{macrocode}

%\iffalse
%</samplepart3>
%\fi
% Some text for part 4:
%\iffalse
%<*samplepart4>
%\fi
%    \begin{macrocode}
more text in part four
%    \end{macrocode}

%\iffalse
%</samplepart4>
%\fi
%
% %%%%%%%%%%%%%%%%%%%%%%%%%%%%%%%%%%%%%%
% \paragraph{Forwarding for a Complete Draft.}
%
% The following forwarding file |cdocsdrf.tex|
% compiles the main document in draft mode:
%\iffalse
%<*sampledraft>
%\fi
%    \begin{macrocode}
\def\version{draft}
\input{childdoc.def}
\childdocforward{cdocsamp}
%    \end{macrocode}

%\iffalse
%</sampledraft>
%\fi
%
% %%%%%%%%%%%%%%%%%%%%%%%%%%%%%%%%%%%%%%
% \paragraph{Forwarding for Final Version of the Chapters.}
%
% The following forwarding files |cdocsfn1.tex| and |cdocsfn2.tex|
% (with identical content)
% compile the final versions of the child documents
% |cdocsch1.tex| and |cdocsch2.tex|, respectively:
%\iffalse
%<*samplefinal>
%\fi
%    \begin{macrocode}
\def\version{final}
\input{childdoc.def}
\childdocforwardprefix[cdocsamp]{cdocsfn}{cdocsch}
%    \end{macrocode}

%\iffalse
%</samplefinal>
%\fi
%
% %%%%%%%%%%%%%%%%%%%%%%%%%%%%%%%%%%%%%%
% \paragraph{Command Line Processing.}
%
% The following three command lines generate the output files
% |cdocscld|, |cdocscl1| and |cdocscl2|
% which should be identical to
% |cdocsdrf|, |cdocsch1| and |cdocsfn2|, respectively:
% \begin{center}
% \begin{tabular}{l}
% |latex -jobname cdocscld \|\\
% |  "\def\version{draft}\input{childdoc.def}\childdocforward{cdocsamp}"|\\
% |latex -jobname cdocscl1 \|\\
% |  "\input{childdoc.def}\childdocforward[cdocsamp]{cdocsch1}"|\\
% |latex -jobname cdocscl2 \|\\
% |  "\def\version{final}\input{childdoc.def}\childdocforward{cdocsch2}"|
% \end{tabular}
% \end{center}
% Note that the trailing backslash on each first line
% merely continues the input to the second line
% (for convenient cut ant paste).
% Furthermore, the command |latex| can be replaced by any
% of its alternative versions such as |pdflatex|.
%
% %%%%%%%%%%%%%%%%%%%%%%%%%%%%%%%%%%%%%%%%%%%%%%%%%%%%%%%%%%%%%%%%%%%%%%%%%%%%%%
% %%%%%%%%%%%%%%%%%%%%%%%%%%%%%%%%%%%%%%%%%%%%%%%%%%%%%%%%%%%%%%%%%%%%%%%%%%%%%%
% \section{Implementation}
%\iffalse
%<*package>
%\fi
%
% This section describes the definitions file |childdoc.def|.

% The definitions cannot be loaded using |\usepackage| or |\RequirePackage|
% which has a mechanism to prevent loading a style file more than once.
% When loading the definitions by means of |\input|
% multiple instances have to be prevented manually:
%\iffalse
%This code needs to be before the `\ProvidesFile' directive
%which is defined at the beginning of this file.
%Therefore it is also placed there and commented out here.
%</package>
%<*discard>
%\fi
%    \begin{macrocode}
\ifdefined\childdocmain\endinput\fi
%    \end{macrocode}
%\iffalse
%</discard>
%<*package>
%\fi
%
% \macro{\ifchilddoc}
% \macro{\ifchilddocmanual}
% The conditional |\ifchilddoc| tells whether a
% child (true) or main (false) document is being compiled.
% The conditional |\ifchilddocmanual| tells whether
% the |\includeonly| mechanism is used (false) or
% the selection of child files must be performed manually (true).
% The definitions initialise to false:
%    \begin{macrocode}
\newif\ifchilddoc
\newif\ifchilddocmanual
%    \end{macrocode}

% \macro{\childdocname}
% \macro{\childdocjob}
% The macro |\childdocname| stores the name of the main document
% to be compiled. The macro |\childdocjob| stores the name of
% the document on which the \LaTeX{} compiler was originally invoked.
% The content of |\jobname| cannot be compared
% to filenames specified in the source due to different catcodes.
% The following code rescans |\jobname|, stores the result
% in |\childdocname| and saves a copy in |\childdocjob|:
%    \begin{macrocode}
\edef\childdocname{\scantokens\expandafter{\jobname\noexpand}}
\let\childdocjob\childdocname
%    \end{macrocode}

% \macro{\childdocdisable}
% The macro |\childdocdisable| prevents the main file
% from being processed more than once.
% At this stage, the main document command |\childdocmain|
% is assumed to be called once again where it should do nothing.
% Any subsequent call to it should prevent
% a secondary processing of the main document
% It overwrites the forwarding commands
% |\childdocof| and |\childdocforward|
% with empty macros to prevent further inclusions of the main document:
%    \begin{macrocode}
\newcommand{\childdocdisable}
{
  \renewcommand{\childdocmain}[1]{\renewcommand{\childdocmain}[1]{\endinput}}
  \renewcommand{\childdocof}[1]{}
  \renewcommand{\childdocby}[2][]{}
  \renewcommand{\childdocforward}[2][]{}
  \renewcommand{\childdocdisable}{}
}
%    \end{macrocode}

% \macro{\childdocmain}
% The macro |\childdocmain| is to be called at the top of the main file
% with nothing or the main filename (without extension) as argument.
% First, it breaks loops.
% If the argument is not empty and does not match |\childdocname|
% (which is set by the first inclusion of |childdoc.def|),
% |\ifchilddoc| is set to true, |\includeonly| is applied to the child file
% and |\jobname| is set to the main file
% (for proper handling of |.aux| files):
%    \begin{macrocode}
\newcommand{\childdocmain}[1]
{
  \childdocdisable\childdocmain{}
  \if?#1?\else
    \begingroup
      \def\childdoctmp{#1}
      \ifx\childdoctmp\childdocname
        \def\childdoctmp{}
      \else
        \def\childdoctmp
        {
          \childdoctrue
          \includeonly{\childdocname}
          \def\childdocjob{#1}
          \def\jobname{#1}
        }
      \fi
      \expandafter
    \endgroup
    \childdoctmp
  \fi
}
%    \end{macrocode}

% \macro{\childdocof}
% The command |\childdocof| redirects
% compilation to the main file |#1|.
%    \begin{macrocode}
\newcommand{\childdocof}[1]
{
  \childdocdisable
  \childdoctrue
  \includeonly{\childdocname}
  \def\jobname{#1}
  \def\childdocjob{#1}
  \input{#1}
}
%    \end{macrocode}

% \macro{\childdocby}
% The command |\childdocby| ....
%    \begin{macrocode}
\newcommand{\childdocby}[2][]
{
  \childdocdisable
  \childdoctrue
  \childdocmanualtrue
  \if?#1?\else
    \def\jobname{#2}
  \fi
  \def\childdocjob{#2}
  \input{#2}
  \endinput
}
%    \end{macrocode}

% \macro{\childdocforward}
% The command |\childdocforward| redirects
% compilation to the main file or
% (if the optional argument is given) a child file.
% Parameters are set as if the main file
% or a child file starting with |\childdocof| was compiled.
% Then compilation is handed over to the main file:
%    \begin{macrocode}
\newcommand{\childdocforward}[2][]
{
  \begingroup
    \if?#1?
      \def\childdoctmp
      {
        \def\childdocname{#2}
        \def\childdocjob{#2}
        \def\jobname{#2}
        \input{#2}
        \endinput
      }
    \else
      \def\childdoctmp
      {
        \childdocdisable
        \def\childdocname{#2}
        \childdoctrue
        \includeonly{#2}
        \def\childdocjob{#1}
        \def\jobname{#1}
        \input{#1}
        \endinput
      }
    \fi
    \expandafter
  \endgroup
  \childdoctmp
}
%    \end{macrocode}

% \macro{\childdocforwardprefix}
% The command |\childdocforwardprefix| redirects
% compilation to the main or a child file by means of a pattern.
% The prefix |#1| in the current filename is replaced by |#2|
% and the suffix of the current filename is kept
% (it is assumed that the filename does not contain the substring `|~~~|'
% which is used as a delimiter).
% Compilation is handed over to the new file by |\childdocforward|:
%    \begin{macrocode}
\newcommand{\childdocforwardprefix}[3][]
{
  \begingroup
    \def\childdocextract #2##1~~~{\def\childdoctmp{\childdocforward[#1]{#3##1}}}
    \expandafter\childdocextract\childdocname~~~
    \expandafter
  \endgroup
  \childdoctmp
}
%    \end{macrocode}

% \macro{\childdoc}
% The deprecated macro |\childdoc| is a legacy version of |\childdocmain|:
%    \begin{macrocode}
\newcommand{\childdoc}{\childdocmain}
%    \end{macrocode}

% \macro{\childdocredirect}
% The deprecated macro |\childdocredirect| is a legacy version
% of |\childdocforward| and |\childdocforwardprefix|:
%    \begin{macrocode}
\newcommand{\childdocredirect}[2][]
{
  \begingroup
    \if?#1?
      \def\childdoctmp{\childdocforward{#2}}
    \else
      \def\childdoctmp{\childdocforwardprefix{#1}{#2}}
    \fi
    \expandafter
  \endgroup
  \childdoctmp
}
%    \end{macrocode}

%\iffalse
%</package>
%\fi
%
\endinput
|\\
|\childdocforward{|\textit{main}|}|\\
\end{tabular}
\end{center}
%
or alternatively with:
%
\begin{center}
\begin{tabular}{l}
|% \iffalse
%
% childdoc.dtx Copyright (C) 2017-2018 Niklas Beisert
%
% This work may be distributed and/or modified under the
% conditions of the LaTeX Project Public License, either version 1.3
% of this license or (at your option) any later version.
% The latest version of this license is in
%   http://www.latex-project.org/lppl.txt
% and version 1.3 or later is part of all distributions of LaTeX
% version 2005/12/01 or later.
%
% This work has the LPPL maintenance status `maintained'.
%
% The Current Maintainer of this work is Niklas Beisert.
%
% This work consists of the files childdoc.dtx and childdoc.ins
% and the derived files childdoc.def and cdocsamp.tex with
% cdocsch1.tex, cdocsch2.tex, cdocsdrf.tex, cdocsfn1.tex, cdocsfn2.tex.
%
%<package>\ifdefined\childdocmain\endinput\fi
%<package>\ProvidesFile{childdoc.def}[2018/12/30 v2.0 child document driver]
%<samplemain>\ProvidesFile{cdocsamp.tex}[2018/12/30 v2.0 sample for childdoc]
%<*driver>
%\ProvidesFile{childdoc.drv}[2018/12/30 v2.0 childdoc reference manual file]
\PassOptionsToClass{10pt,a4paper}{article}
\documentclass{ltxdoc}

\usepackage[margin=35mm]{geometry}
\usepackage{hyperref}
\usepackage{hyperxmp}
\usepackage[usenames]{color}

\hypersetup{colorlinks=true}
\hypersetup{pdfstartview=FitH}
\hypersetup{pdfpagemode=UseNone}
\hypersetup{pdfsource={}}
\hypersetup{pdflang={en-UK}}
\hypersetup{pdfcopyright={Copyright 2017-2018 Niklas Beisert.
  This work may be distributed and/or modified under the
  conditions of the LaTeX Project Public License, either version 1.3
  of this license or (at your option) any later version.}}
\hypersetup{pdflicenseurl={http://www.latex-project.org/lppl.txt}}
\hypersetup{pdfcontactaddress={ETH Zurich, ITP, HIT K,
  Wolfgang-Pauli-Strasse 27}}
\hypersetup{pdfcontactpostcode={8093}}
\hypersetup{pdfcontactcity={Zurich}}
\hypersetup{pdfcontactcountry={Switzerland}}
\hypersetup{pdfcontactemail={nbeisert@itp.phys.ethz.ch}}
\hypersetup{pdfcontacturl={http://people.phys.ethz.ch/\xmptilde nbeisert/}}

\newcommand{\secref}[1]{\hyperref[#1]{section \ref*{#1}}}

\parskip1ex
\parindent0pt
\let\olditemize\itemize
\def\itemize{\olditemize\parskip0pt}

\begin{document}

\title{The \textsf{childdoc} Package}
\hypersetup{pdftitle={The childdoc Package}}
\author{Niklas Beisert\\[2ex]
  Institut f\"ur Theoretische Physik\\
  Eidgen\"ossische Technische Hochschule Z\"urich\\
  Wolfgang-Pauli-Strasse 27, 8093 Z\"urich, Switzerland\\[1ex]
  \href{mailto:nbeisert@itp.phys.ethz.ch}
  {\texttt{nbeisert@itp.phys.ethz.ch}}}
\hypersetup{pdfauthor={Niklas Beisert}}
\hypersetup{pdfsubject={Manual for the LaTeX2e Package childdoc}}
\date{30 December 2018, \textsf{v2.0}}
\maketitle

\begin{abstract}\noindent
\textsf{childdoc} is a \LaTeXe{} package
that enables the direct compilation
of document sections included by |\include|
to individual files.
\end{abstract}

\begingroup
\parskip0ex
\tableofcontents
\endgroup

%%%%%%%%%%%%%%%%%%%%%%%%%%%%%%%%%%%%%%%%%%%%%%%%%%%%%%%%%%%%%%%%%%%%%%%%%%%%%%%%
%%%%%%%%%%%%%%%%%%%%%%%%%%%%%%%%%%%%%%%%%%%%%%%%%%%%%%%%%%%%%%%%%%%%%%%%%%%%%%%%
\section{Introduction}

\LaTeX{} provides a mechanism to structure a large document (such as a book)
into a main file and several child files (containing the chapters)
using the |\include| command.
This mechanism is beneficial for documents
which span hundreds of pages in order to
make the source file(s) more manageable.
Moreover, compilation can be restricted to
selected child files by means of the |\includeonly| command.
The latter feature can be used to reduce the compilation time while editing
(this was significantly more useful in the earlier days of \LaTeX{})
or to generate a smaller document which is easier to navigate.
Another application of |\includeonly| is to generate
documents consisting of selected parts of the complete document.

However, there are a few drawbacks of the plain |\include| mechanism:
\begin{itemize}
\item
The child files cannot be compiled on their own,
they can only be compiled via the main file.
A naive editing environment
(such as a text editor with an option
to have the current file processed by \LaTeX)
may require one to switch to the main file before compiling;
attempting to compile the child file produces errors.
\item
The main file must be modified (each time)
to adjust the |\includeonly| command
to the present needs. This easily leaves the main file in a messy state.
\item
The generated document will always carry the filename
of the main document. This is inconvenient if
several child files are to be compiled and
to be kept for distribution.
\end{itemize}

The present package provides a simple interface
to make child files individually compilable by \LaTeX{}.
Compiling a child file then has the same effect as compiling
the main file with an |\includeonly| command
to select the appropriate child.
Moreover the generated document will carry the name of the child
rather than the main file.
This resolves all three above issues.

This feature is meant to make the editing of books,
thesis documents and lecture notes somewhat more convenient.
However, the package can also be used efficiently for
composing a series of documents (such as exercise sheets)
which are typically distributed individually.
It then assists the author in generating the individual documents
(potentially in different versions)
as well as a document containing the collected series.
Another application is in developing style files
or other kinds of included material
where compilation of the style file could redirect
to a sample or test file.

%%%%%%%%%%%%%%%%%%%%%%%%%%%%%%%%%%%%%%%%%%%%%%%%%%%%%%%%%%%%%%%%%%%%%%%%%%%%%%%%
%%%%%%%%%%%%%%%%%%%%%%%%%%%%%%%%%%%%%%%%%%%%%%%%%%%%%%%%%%%%%%%%%%%%%%%%%%%%%%%%
\section{Usage}

First of all, the package \textsf{childdoc} is \emph{not} a standard
\LaTeXe{} |.sty| style file! Therefore it needs to be invoked in
a non-standard way.

%%%%%%%%%%%%%%%%%%%%%%%%%%%%%%%%%%%%%%%%%%%%%%%%%%%%%%%%%%%%%%%%%%%%%%%%%%%%%%%%
\subsection{Included Files}
\label{sec:include}

%%%%%%%%%%%%%%%%%%%%%%%%%%%%%%%%%%%%%%%%
\DescribeMacro{\childdocmain}
To use the package, add the commands
\begin{center}
\begin{tabular}{l}
|\input{childdoc.def}|\\
|\childdocmain{}|\\
\end{tabular}
\end{center}
at the very top of the main \LaTeX{} file,
in particular \emph{before} the |\documentclass| statement!
The argument of |\childdocmain| should be left empty
(but it must be present).

%%%%%%%%%%%%%%%%%%%%%%%%%%%%%%%%%%%%%%%%
\DescribeMacro{\childdocof}
Furthermore, add the commands
\begin{center}
\begin{tabular}{l}
|\input{childdoc.def}|\\
|\childdocof{|\textit{main}|}|\\
\end{tabular}
\end{center}
at the top of every child file \textit{child}
which is included by |\include{|\textit{child}|}|
from within the main file
(or at least for those files to be compiled individually).
The argument \textit{main} must be the filename of the main file.

There are a couple of
considerations in setting up the main and child documents:

%%%%%%%%%%%%%%%%%%%%%%%%%%%%%%%%%%%%%%%%
\paragraph{Restrictions.}

Please note the following restrictions:
\begin{itemize}
\item
|\childdocmain| must be called with one argument \textit{main}
to ensure compatibility with earlier version of the package.
It must either be empty (|\childdocmain{}|)
or precisely match the filename of the main file in which it is specified.
See \secref{sec:detection} for further information.
\item
The filename \textit{main} must be specified without the |.tex| extension.
\item
The filename \textit{main} is case sensitive
(even in case-insensitive file systems)
due to internal string comparison.
\item
The argument \textit{main} should be fully expanded, it cannot be a macro.
\item
Subdirectories and special characters should be avoided in filenames.
\item
The command |\childdocmain{|\textit{main}|}| must be followed by a whitespace.
It should not be followed immediately by another command
or by a comment mark `|%|'.
This is because the \TeX{} parser reads the token immediately following
the argument of |\childdocmain| and puts it
at the beginning of every child section;
however, a white\-space is ignored.
\end{itemize}

%%%%%%%%%%%%%%%%%%%%%%%%%%%%%%%%%%%%%%%%
\paragraph{Content of Main File.}

It is advisable to place all content in the child files included by |\include|.
Any output contained in the main file will appear in all child documents
unless suppressed manually;
it cannot be suppressed automatically by the |\includeonly| directive
and thus should normally be avoided.
A method to include some content in the main file
by means of conditional processing is described in \secref{sec:conditional}.

%%%%%%%%%%%%%%%%%%%%%%%%%%%%%%%%%%%%%%%%
\paragraph{Page Numbering.}

When only a part of the document is compiled,
the appropriate numbering of pages
(as well as other status parameters)
is determined from the |.aux| files.
The latter contain information from previous passes.
However this information needs to propagate through
all intermediate child documents.
Therefore the page numbering in child documents may well
be inconsistent until the complete document is compiled at least once.

A useful (if unconventional) way to always ensure a consistent
page numbering is to restart the numbering in each child document
and denote the pages by `\textit{child}|.|\textit{page}'
where \textit{child} represents the chapter/section number of the child file.
This can be achieved by the command
|\numberwithin{page}{|\textit{child}|}|
of the \textsf{amsmath} package
where \textit{child} can be |chapter| or |section|
depending on the chosen structuring.
Alternatively, one can modify the macro |\thepage| appropriately
and reset the counter |page| at the start of each child file.

%%%%%%%%%%%%%%%%%%%%%%%%%%%%%%%%%%%%%%%%%%%%%%%%%%%%%%%%%%%%%%%%%%%%%%%%%%%%%%%%
\subsection{Conditional Processing}
\label{sec:conditional}

The package provides a mechanism to compile different versions
of a document. To customise the versions further some conditional processing
can come in handy to distinguish which version is being compiled.
The package provides two macros to describe the compilation context:

%%%%%%%%%%%%%%%%%%%%%%%%%%%%%%%%%%%%%%%%
\DescribeMacro{\ifchilddoc}
The conditional |\ifchilddoc| distinguishes between the compilation of
child documents and the main document:
%
\begin{center}
|\ifchilddoc |\textit{child-code}| |[|\||else |\textit{main-code}]| \||fi|
\end{center}

%%%%%%%%%%%%%%%%%%%%%%%%%%%%%%%%%%%%%%%%
\DescribeMacro{\childdocname}
\DescribeMacro{\childdocjob}
The macro |\childdocname| contains the filename (without extension)
of the main or child file being processed.
Note that |\childdocjob| will always contain the name of the main file.

%%%%%%%%%%%%%%%%%%%%%%%%%%%%%%%%%%%%%%%%
\paragraph{Title Page.}

Conditional processing can be used to include a title or banner page
in the main document when proper precautions are taken.
Importantly, the code in the main file should ensure that the page counter
(as well as other status parameters which are stored in the |.aux| files)
takes the same value after the conditional processing.
Otherwise the page numbers may take divergent values
depending on which part is compiled.

For example, a title page could be declared by:
%
\begin{center}
\begin{tabular}{l}
|\ifchilddoc\||else|\\
|\addtocounter{page}{-1}|\\
\textit{code for title page}\\
|\newpage|\\
|\||fi|
\end{tabular}
\end{center}
%
A banner page for the child documents can be generated by:
%
\begin{center}
\begin{tabular}{l}
|\ifchilddoc|\\
|\addtocounter{page}{-1}|\\
\textit{code for banner page}\\
|\newpage|\\
|\||fi|
\end{tabular}
\end{center}
%
Here one could write a message such as:
\begin{center}
|This is the part \childdocname{} of \childdocjob{}.|
\end{center}

%%%%%%%%%%%%%%%%%%%%%%%%%%%%%%%%%%%%%%%%%%%%%%%%%%%%%%%%%%%%%%%%%%%%%%%%%%%%%%%%
\subsection{Flags}
\label{sec:flags}

The package makes it easy to generate different versions
of the main or child documents.
To this end compilation flags can be defined
and assigned different default values.
They will be particularly useful in conjunction
with the forwarding mechanism described in \secref{sec:forward}.

For example, it may be useful to have a flag |\version|
which can be set to |draft| or |final|.
The document source will contain some conditional code
depending on the value of |\version|.
Suppose further, the flag should default to |final| for the main file
and to |draft| for child files
which is a natural assignment for editing the document.
This is achieved by placing the following code
in the preamble of the main document
(below the |\childdocmain| directive):
%
\begin{center}
\begin{tabular}{l}
|\ifchilddoc|\\
|\providecommand{\version}{draft}|\\
|\||else|\\
|\providecommand{\version}{final}|\\
|\||fi|
\end{tabular}
\end{center}
%
The definition by |\providecommand| makes sure
that previous definitions are not overwritten.
Further statements |\providecommand{\version}{...}|
can thus be added before the above code to override it.

For the main file, one might add a line
(between |\childdocmain| and the above block)
%
\begin{center}
|%\ifchilddoc\||else\providecommand{\version}{draft}\||fi|
\end{center}
%
which can be uncommented to produce a draft version.
Likewise one can add a line to the very top of a child file
(above the |\childdocof{|\textit{main}|}| directive)
%
\begin{center}
|%\providecommand{\version}{final}|
\end{center}
%
which can be uncommented to produce the final version of this child document.

%%%%%%%%%%%%%%%%%%%%%%%%%%%%%%%%%%%%%%%%%%%%%%%%%%%%%%%%%%%%%%%%%%%%%%%%%%%%%%%%
\subsection{Forwarding}
\label{sec:forward}

Different versions of the main or child documents
using compilation flags as described in \secref{sec:flags}
can be (permanently) stored in different files
for convenient compilation, viewing and distribution.
To this end, the package defines a command
to pass on compilation to a different file:

%%%%%%%%%%%%%%%%%%%%%%%%%%%%%%%%%%%%%%%%
\DescribeMacro{\childdocforward}
The command |\childdocforward| redirects processing to
another source file:
%
\begin{center}
\begin{tabular}{l}
|\input{childdoc.def}|\\
|\childdocforward[|\textit{main}|]{|\textit{dest}|}|\\
\end{tabular}
\end{center}
%
The argument \textit{dest} is the destination file
(without extension).
It should be the main file or one of the child files.
Note that further \textsf{childdoc} directives
such as |\childdocof| and |\childdocforward|
in the indicated file will be processed in this form.
The optional argument \textit{main}
passes on directly to the main file \textit{main}
while pretending to compile the child \textit{dest}.
This form behaves as if \textit{dest}
issues |\childdocof{|\textit{main}|}| right away,
and no further \textsf{childdoc} directives will be processed.

%%%%%%%%%%%%%%%%%%%%%%%%%%%%%%%%%%%%%%%%
\DescribeMacro{\...prefix}
In the alternative form |\childdocforwardprefix|,
%
\begin{center}
\begin{tabular}{l}
|\input{childdoc.def}|\\
|\childdocforwardprefix[|\textit{main}|]{|\textit{prefix}|}{|\textit{dest}|}|
\end{tabular}
\end{center}
%
the destination file is determined by a pattern
depending on the current file:
To make this work, the current file must be called
`{\textit{prefix}\hspace{0.2em}\textit{suffix}}'
with \textit{prefix} matching precisely the argument.
Processing is then passed on to the file
`{\textit{dest}\hspace{0.2em}\textit{suffix}}'.
Surely, the same effect is achieved by
directly specifying the
argument `{\textit{dest}\hspace{0.2em}\textit{suffix}}'
in the first form.
However, that requires to set up a different file
for each child. With the alternative form of the command
all these files can have exactly the same content
which simplifies setting them up and maintaining them.

For example, the following file |draft.tex|
with a compilation flag |\version| as described in \secref{sec:flags}
compiles the main document as a draft:
%
\begin{center}
\begin{tabular}{l}
|\def\version{draft}|\\
|\input{childdoc.def}|\\
|\childdocforward{|\textit{main}|}|
\end{tabular}
\end{center}
%
Likewise, the following files |final|\textit{nn}|.tex|
compile the final version of the child document
|child|\textit{nn}|.tex|:
%
\begin{center}
\begin{tabular}{l}
|\def\version{final}|\\
|\input{childdoc.def}|\\
|\childdocforwardprefix{final}{child}|
\end{tabular}
\end{center}
%

Note that when several versions of a main file and/or of each child file
are to be generated, it may be convenient to set up a |Makefile| or
shell script to automatise the process.

%%%%%%%%%%%%%%%%%%%%%%%%%%%%%%%%%%%%%%%%%%%%%%%%%%%%%%%%%%%%%%%%%%%%%%%%%%%%%%%%
\subsection{Command Line Processing}
\label{sec:commandline}

The effect of redirection files can also be achieved by invoking
the \LaTeX{} compiler with a more elaborate command line.
Most conveniently this should be done as part
of a shell script or a |Makefile|.

When using \textsf{childdoc} in the main file, the following
command lines effectively perform a redirection
(note that depending on the shell being used,
backslashes may have to be doubled: `|\|' $\to$ `|\\|'):
%
\begin{center}
|... -jobname "|\textit{target}|" |\\|"|[\textit{flags}]%
|\input{childdoc.def}\childdocforward[|\textit{main}|]{|\textit{dest}|}"|
\end{center}
%
Here \textit{target} is the name of the output file,
\textit{main} is the name of the main file
and \textit{dest} is the name of the main or child file to be processed
(all filenames without extensions).
The optional argument \textit{main} can be omitted
if \textit{main} matches \textit{dest}.
Optionally, compilation \textit{flags} can be defined via |\def| commands.
This command line makes the \TeX{} engine believe
it is compiling the file \textit{target}
whose content is specified as the latter parameter.
The provided code then forwards the processing to
\textit{main} or \textit{dest} as described in \secref{sec:forward}.

%%%%%%%%%%%%%%%%%%%%%%%%%%%%%%%%%%%%%%%%%%%%%%%%%%%%%%%%%%%%%%%%%%%%%%%%%%%%%%%%
\subsection{Include by Input}
\label{sec:input}

Including child documents by |\include| has some restrictions by design.
Most notably, the content of a child document always occupies
its own set of pages; pages cannot be shared between child documents.
Usually, this behaviour makes perfect sense
because each child document contain an essential part of the document.
However, in some situations it may be desirable to compose
a document from a collection of parts
without having mandatory page breaks between then.
For this case, the package
provides a mechanism to include parts
by |\input| which can also be processed individually.
However, by construction this mechanism
requires manual handling of the content to be output.

%%%%%%%%%%%%%%%%%%%%%%%%%%%%%%%%%%%%%%%%
\DescribeMacro{\ifchilddocmanual}
The main file should be prepared as usual, see \secref{sec:include}.
However, the document body must make a distinction
between processing of an individual part and of the main document, e.g.:
%
\begin{center}
\begin{tabular}{l}
|\ifchilddocmanual|\\
|\input{\childdocname}|\\
|\||else|\\
\textit{document body with }|\input{|\textit{part}|}|\\
|\||fi|
\end{tabular}
\end{center}
%
The conditional |\ifchilddocmanual| is true whenever
a part to be included by |\input| is being compiled,
and the name of the part is stored in |\childdocname|.

%%%%%%%%%%%%%%%%%%%%%%%%%%%%%%%%%%%%%%%%
\DescribeMacro{\childdocby}
Each part to be included by |\input| should start with:
%
\begin{center}
\begin{tabular}{l}
|\input{childdoc.def}|\\
|\childdocby{|\textit{main}|}|\\
\end{tabular}
\end{center}
%
The directive |\childdocby| is similar to |\childdocof|
described in \secref{sec:include},
but the subsequent selection of content must be done manually.
To that end, both |\ifchilddoc| and |\ifchilddocmanual|
will be true upon processing of a part,
and the name of the part is stored in |\childdocname|.
Note that |\jobname| will be set to the filename of the current part
so that each part receives an individual |.aux| file
that does not interfere with the |.aux| file(s) of the main document.
This behaviour can be altered by the alternative form
|\childdocby[*]{|\textit{main}|}| (with a non-empty optional argument)
which uses the |.aux| file of the main document
by setting |\jobname| to \textit{main}.

%%%%%%%%%%%%%%%%%%%%%%%%%%%%%%%%%%%%%%%%%%%%%%%%%%%%%%%%%%%%%%%%%%%%%%%%%%%%%%%%
\subsection{Driver Development}
\label{sec:driver}

The \textsf{childdoc} mechanism can also be use for the development
of definition files such as \LaTeX{} styles or classes.
This case differs from the above setup with multiple parts
included by |\include| in that no |\includeonly| should be invoked.
This can be achieved by starting the include file
(before |\ProvidesPackage|) with:
%
\begin{center}
\begin{tabular}{l}
|\input{childdoc.def}|\\
|\childdocforward{|\textit{main}|}|\\
\end{tabular}
\end{center}
%
or alternatively with:
%
\begin{center}
\begin{tabular}{l}
|\input{childdoc.def}|\\
|\childdocby{|\textit{main}|}|\\
\end{tabular}
\end{center}
%
Both forms have slightly different effects as described above.
The main file is prepared as usual, see \secref{sec:include}.

%%%%%%%%%%%%%%%%%%%%%%%%%%%%%%%%%%%%%%%%%%%%%%%%%%%%%%%%%%%%%%%%%%%%%%%%%%%%%%%%
\subsection{Legacy Detection}
\label{sec:detection}

The directive |\childdocmain| in the main file can detect
whether the complete document or merely a child is to be compiled
even without using the directive |\childdocof|.
This method is deprecated because it is less robust
and there is no compelling reason to use it;
it is merely provided for backward compatibility
and it may be removed in future versions.

If the detection mechanism is to be used,
it is mandatory to correctly specify
the filename of the main file as the argument of |\childdocmain|:
%
\begin{center}
\begin{tabular}{l}
|\input{childdoc.def}|\\
|\childdocmain{|\textit{main}|}|\\
\end{tabular}
\end{center}
%
If |\jobname| does not match the argument \textit{main} of |\childdocmain|,
it is assumed that |\jobname| points to the child file to be compiled.
When using |\childdocmain| with the main file specified as argument,
it suffices to start a child file
with just |\input{|\textit{main}|}|
without loading of the package and using |\childdocof|.
If instead all processing is done
with the appropriate \textsf{childdoc} directives,
the argument of \textit{main} of |\childdocmain| can be empty.

An alternative version of the command line processing described
in \secref{sec:commandline} using the detection mechanism reads:
%
\begin{center}
|... -jobname "|\textit{target}|" "|[\textit{flags}]%
[|\def\jobname{|\textit{dest}|}|]|\input{|\textit{main}|}"|
\end{center}

%%%%%%%%%%%%%%%%%%%%%%%%%%%%%%%%%%%%%%%%%%%%%%%%%%%%%%%%%%%%%%%%%%%%%%%%%%%%%%%%
\subsection{Manual Code}
\label{sec:manual}

In case one cannot be certain whether the definitions file |childdoc.def|
is installed on the target \TeX{} distribution
and one prefers not to ship it,
it is conceivable to paste a few relevant commands into the sources.

To that end, drop all statements |\input{childdoc.def}|
and perform the replacements as outlined below.
Instead of |\childdocmain{|\textit{main}|}| add the following code
to the top of the main file:
%
\begin{center}
\begin{tabular}{l}
|\||ifdefined\childdocname\endinput\||fi\newif\ifchilddoc|\\
|\edef\childdocname{\scantokens\expandafter{\jobname\noexpand}}|\\
|\def\childdocmain{|\textit{main}|}\||ifx\childdocmain\childdocname\||else|\\
|\childdoctrue\includeonly{\childdocname}\let\jobname\childdocmain\||fi|\\
\end{tabular}
\end{center}
%
Instead of |\childdocof{|\textit{main}|}| just include the main file
at the top of each child file:
%
\begin{center}
|\input{|\textit{main}|}|
\end{center}
%
A simple redirection |\childdocforward{|\textit{dest}|}| is achieved by:
%
\begin{center}
|\def\jobname{|\textit{dest}|}\input{\jobname}|
\end{center}
%
The redirection with prefix
|\childdocforwardprefix[|\textit{prefix}|]{|\textit{dest}|}|
is accomplished by:
%
\begin{center}
\begin{tabular}{l}
|{\edef\jobname{\scantokens\expandafter{\jobname\noexpand}}|\\
|\def\redirectjob |\textit{prefix}|#1~~~{\gdef\jobname{|\textit{dest}|#1}}|\\
|\expandafter\redirectjob\jobname~~~}\input{\jobname}|
\end{tabular}
\end{center}

In an alternative approach,
child documents can be compiled by a specific command line
without additional code or specific definitions:
%
\begin{center}
|... -jobname "|\textit{target}|" "|[\textit{flags}]%
|\includeonly{|\textit{dest}|}\input{|\textit{main}|}"|
\end{center}
%

%%%%%%%%%%%%%%%%%%%%%%%%%%%%%%%%%%%%%%%%%%%%%%%%%%%%%%%%%%%%%%%%%%%%%%%%%%%%%%%%
%%%%%%%%%%%%%%%%%%%%%%%%%%%%%%%%%%%%%%%%%%%%%%%%%%%%%%%%%%%%%%%%%%%%%%%%%%%%%%%%
\section{Information}

%%%%%%%%%%%%%%%%%%%%%%%%%%%%%%%%%%%%%%%%%%%%%%%%%%%%%%%%%%%%%%%%%%%%%%%%%%%%%%%%
\subsection{Copyright}

Copyright \copyright{} 2017--2018 Niklas Beisert

This work may be distributed and/or modified under the
conditions of the \LaTeX{} Project Public License, either version 1.3
of this license or (at your option) any later version.
The latest version of this license is in
  \url{http://www.latex-project.org/lppl.txt}
and version 1.3 or later is part of all distributions of \LaTeX{}
version 2005/12/01 or later.

This work has the LPPL maintenance status `maintained'.

The Current Maintainer of this work is Niklas Beisert.

This work consists of the files |README.txt|, |childdoc.ins| and |childdoc.dtx|
as well as the derived files |childdoc.def|, |cdocsamp.tex|
with |cdocsch1.tex|, |cdocsch2.tex|, |cdocspt3.tex|, |cdocspt4.tex|,
|cdocsdrf.tex|, |cdocsfn1.tex|, |cdocsfn2.tex|
as well as |childdoc.pdf|.

%%%%%%%%%%%%%%%%%%%%%%%%%%%%%%%%%%%%%%%%%%%%%%%%%%%%%%%%%%%%%%%%%%%%%%%%%%%%%%%%
\subsection{Files and Installation}

The package consists of the files:
%
\begin{center}
\begin{tabular}{ll}
    |README.txt|   & readme file \\
    |childdoc.ins| & installation file \\
    |childdoc.dtx| & source file \\
    |childdoc.def| & definition file \\
    |cdocsamp.tex| & sample main file \\
    |cdocsch1.tex| & sample include file \\
    |cdocsch2.tex| & sample include file \\
    |cdocspt3.tex| & sample part file \\
    |cdocspt4.tex| & sample part file \\
    |cdocsdrf.tex| & sample redirection file \\
    |cdocsfn1.tex| & sample redirection file \\
    |cdocsfn2.tex| & sample redirection file \\
    |childdoc.pdf| & manual
\end{tabular}
\end{center}
%
The distribution consists of the files
|README.txt|, |childdoc.ins| and |childdoc.dtx|.
%
\begin{itemize}
\item
Run (pdf)\LaTeX{} on |childdoc.dtx|
to compile the manual |childdoc.pdf| (this file).
\item
Run \LaTeX{} on |childdoc.ins| to create the definitions file |childdoc.def|
and the sample |cdocsamp.tex| with include files
|cdocsch1.tex|, |cdocsch2.tex|, |cdocspt3.tex|, |cdocspt4.tex|,
|cdocsdrf.tex|, |cdocsfn1.tex|, |cdocsfn2.tex|.
Then copy the file |childdoc.def| to an appropriate directory of your \LaTeX{}
distribution, e.g.\ \textit{texmf-root}|/tex/latex/childdoc|.
\end{itemize}

%%%%%%%%%%%%%%%%%%%%%%%%%%%%%%%%%%%%%%%%%%%%%%%%%%%%%%%%%%%%%%%%%%%%%%%%%%%%%%%%
\subsection{Related CTAN Packages}

There are several other packages which offer a similar functionality:
%
\begin{itemize}
\item
The packages
\href{http://ctan.org/pkg/docmute}{\textsf{docmute}},
\href{http://ctan.org/pkg/includex}{\textsf{includex}} and
\href{http://ctan.org/pkg/standalone}{\textsf{standalone}}
provide commands to include only the document body of
a child file thus allowing both files to be compiled individually.
\item
The packages \href{http://ctan.org/pkg/subdocs}{\textsf{subdocs}}
and \href{http://ctan.org/pkg/subfiles}{\textsf{subfiles}}
provide structures in which the main and child documents can be
encapsulated and allowing them to be compiled individually.
The inclusion mechanism is different from the conventional |\include|.
\item
The package \href{http://ctan.org/pkg/combine}{\textsf{combine}}
is an elaborate solution to combine several documents into one.
\end{itemize}
%
See also the CTAN topic \href{http://ctan.org/topic/subdocs}{\textsf{subdocs}}
for further related packages.
The present package differs from the above solutions in that
a document structure constructed with the conventional |\include| mechanism
just needs two extra commands at the top of every file
such that all constituent files can be compiled individually.

%%%%%%%%%%%%%%%%%%%%%%%%%%%%%%%%%%%%%%%%%%%%%%%%%%%%%%%%%%%%%%%%%%%%%%%%%%%%%%%%
%\subsection{Feature Suggestions}
%
%The following is a list of features which may be useful for future
%versions of this package:
%%
%\begin{itemize}
%\item
%\ldots
%\end{itemize}

%%%%%%%%%%%%%%%%%%%%%%%%%%%%%%%%%%%%%%%%%%%%%%%%%%%%%%%%%%%%%%%%%%%%%%%%%%%%%%%%
\subsection{Revision History}

%%%%%%%%%%%%%%%%%%%%%%%%%%%%%%%%%%%%%%%%
\paragraph{v2.0:} 2018/12/30

\begin{itemize}
\item
immediate forward processing
\item
added |\childdocby| mechanism
\item
manual restructured
\end{itemize}

%%%%%%%%%%%%%%%%%%%%%%%%%%%%%%%%%%%%%%%%
\paragraph{v1.6:} 2018/01/17

\begin{itemize}
\item
application for development of include files
\item
corrections to manual
\end{itemize}

%%%%%%%%%%%%%%%%%%%%%%%%%%%%%%%%%%%%%%%%
\paragraph{v1.5:} 2017/05/21

\begin{itemize}
\item
more complete structuring introduced
\item
|\childdocof| introduced
\item
|\childdoc| renamed to |\childdocmain|
\item
|\childredirect| renamed to |\childdocforward| and |\childdocforwardprefix|
and functionality expanded
\end{itemize}

%%%%%%%%%%%%%%%%%%%%%%%%%%%%%%%%%%%%%%%%
\paragraph{v1.0:} 2017/04/27

\begin{itemize}
\item
manual and install package
\item
first version published on CTAN
\end{itemize}

%%%%%%%%%%%%%%%%%%%%%%%%%%%%%%%%%%%%%%%%
\paragraph{v0.6:} 2017/04/26

\begin{itemize}
\item
redirection mechanism added
\end{itemize}

%%%%%%%%%%%%%%%%%%%%%%%%%%%%%%%%%%%%%%%%
\paragraph{v0.5:} 2017/04/26

\begin{itemize}
\item
functionality in definition file
\end{itemize}


%%%%%%%%%%%%%%%%%%%%%%%%%%%%%%%%%%%%%%%%%%%%%%%%%%%%%%%%%%%%%%%%%%%%%%%%%%%%%%%%
%%%%%%%%%%%%%%%%%%%%%%%%%%%%%%%%%%%%%%%%%%%%%%%%%%%%%%%%%%%%%%%%%%%%%%%%%%%%%%%%
%%%%%%%%%%%%%%%%%%%%%%%%%%%%%%%%%%%%%%%%%%%%%%%%%%%%%%%%%%%%%%%%%%%%%%%%%%%%%%%%
\appendix

\settowidth\MacroIndent{\rmfamily\scriptsize 000\ }

 \DocInput{childdoc.dtx}

\end{document}
%</driver>
% \fi
%
% %%%%%%%%%%%%%%%%%%%%%%%%%%%%%%%%%%%%%%%%%%%%%%%%%%%%%%%%%%%%%%%%%%%%%%%%%%%%%%
% %%%%%%%%%%%%%%%%%%%%%%%%%%%%%%%%%%%%%%%%%%%%%%%%%%%%%%%%%%%%%%%%%%%%%%%%%%%%%%
% \section{Sample}
%\iffalse
%<*samplemain>
%\fi
%
% The following presents a sample document
% with two chapters, two parts, a title page,
% a compile flag as well as three forwarding files to set the flag.
% It consists of eight |.tex| files:
% \begin{center}
% \begin{tabular}{ll}
% |cdocsamp.tex|&main file\\
% |cdocsch1.tex|&include file for chapter 1\\
% |cdocsch2.tex|&include file for chapter 2\\
% |cdocspt3.tex|&include file for part 3\\
% |cdocspt4.tex|&include file for part 4\\
% |cdocsdrf.tex|&forwarding file for main file in draft mode\\
% |cdocsfi1.tex|&forwarding file for final version of chapter 1\\
% |cdocsfi2.tex|&forwarding file for final version of chapter 2\\
% \end{tabular}
% \end{center}
% Each of the eight files can be compiled directly by the \LaTeX{} compiler.
%
% %%%%%%%%%%%%%%%%%%%%%%%%%%%%%%%%%%%%%%
% \paragraph{Main File.}
%
% The main file is called |cdocsamp.tex|.
%
% Load the \textsf{childdoc} definitions and
% declare the filename for the main document:
%    \begin{macrocode}
\input{childdoc.def}
\childdocmain{}
%    \end{macrocode}

% Optional override for |\version| flag:
%    \begin{macrocode}
%%\ifchilddoc\else\providecommand{\version}{draft}\fi
%    \end{macrocode}

% Define the default values for the |\version| flag
% (|final| for the main file and |draft| for childs):
%    \begin{macrocode}
\ifchilddoc
\providecommand{\version}{draft}
\else
\providecommand{\version}{final}
\fi
%    \end{macrocode}

% Load the standard document class:
%    \begin{macrocode}
\documentclass[12pt]{article}
%    \end{macrocode}

% Start the document body:
%    \begin{macrocode}
\begin{document}
%    \end{macrocode}

% Declare a title page.
% Print title, part of document being processed and version flag:
%    \begin{macrocode}
\addtocounter{page}{-1}
\begin{center}
{\LARGE\bfseries{}childdoc example\par}
\vspace{1cm}
\ifchilddoc
\ifchilddocmanual part\else chapter\fi:
`\childdocname' of `\childdocjob'\par
\else
main document: `\childdocjob'\par
\fi
version: \version\par
\end{center}
\newpage
%    \end{macrocode}

% Manually include selected file,
% otherwise process as usual:
%    \begin{macrocode}
\ifchilddocmanual
\section*{part `\childdocname'}
\input{\childdocname}
\else
%    \end{macrocode}

% Include the two chapters:
%    \begin{macrocode}
\include{cdocsch1}
\include{cdocsch2}
%    \end{macrocode}

% Include the two parts unless only chapters should be displayed:
%    \begin{macrocode}
\ifchilddoc\else
\section{part three}
\input{cdocspt3}
\section{part four}
\input{cdocspt4}
\fi
%    \end{macrocode}

% Process as usual until here:
%    \begin{macrocode}
\fi
%    \end{macrocode}

% End of document body:
%    \begin{macrocode}
\end{document}
%    \end{macrocode}
%\iffalse
%</samplemain>
%\fi
%
% %%%%%%%%%%%%%%%%%%%%%%%%%%%%%%%%%%%%%%
% \paragraph{Chapter Include Files.}
%
% The include files are called |cdocsch1.tex| and |cdocsch2.tex|.
%
%\iffalse
%<*samplechap1|samplechap2>
%\fi

% Optional override for |\version| flag:
%    \begin{macrocode}
%%\providecommand{\version}{final}
%    \end{macrocode}

% Include the main document:
%    \begin{macrocode}
\input{childdoc.def}
\childdocof{cdocsamp}
%    \end{macrocode}

%\iffalse
%</samplechap1|samplechap2>
%\fi
%
%\iffalse
%<*samplechap1>
%\fi
% Some text for chapter 1:
%    \begin{macrocode}
\section{one}
some text in chapter one
%    \end{macrocode}

%\iffalse
%</samplechap1>
%\fi
% Some text for chapter 2:
%\iffalse
%<*samplechap2>
%\fi
%    \begin{macrocode}
\section{two}
more text in chapter two
%    \end{macrocode}

%\iffalse
%</samplechap2>
%\fi
%
% %%%%%%%%%%%%%%%%%%%%%%%%%%%%%%%%%%%%%%
% \paragraph{Part Include Files.}
%
% The include files are called |cdocspt3.tex| and |cdocspt4.tex|.
%
%\iffalse
%<*samplepart3|samplepart4>
%\fi

% Optional override for |\version| flag:
%    \begin{macrocode}
%%\providecommand{\version}{final}
%    \end{macrocode}

% Include the main document:
%    \begin{macrocode}
\input{childdoc.def}
\childdocby{cdocsamp}
%    \end{macrocode}

%\iffalse
%</samplepart3|samplepart4>
%\fi
%
%\iffalse
%<*samplepart3>
%\fi
% Some text for part 3:
%    \begin{macrocode}
some text in part three
%    \end{macrocode}

%\iffalse
%</samplepart3>
%\fi
% Some text for part 4:
%\iffalse
%<*samplepart4>
%\fi
%    \begin{macrocode}
more text in part four
%    \end{macrocode}

%\iffalse
%</samplepart4>
%\fi
%
% %%%%%%%%%%%%%%%%%%%%%%%%%%%%%%%%%%%%%%
% \paragraph{Forwarding for a Complete Draft.}
%
% The following forwarding file |cdocsdrf.tex|
% compiles the main document in draft mode:
%\iffalse
%<*sampledraft>
%\fi
%    \begin{macrocode}
\def\version{draft}
\input{childdoc.def}
\childdocforward{cdocsamp}
%    \end{macrocode}

%\iffalse
%</sampledraft>
%\fi
%
% %%%%%%%%%%%%%%%%%%%%%%%%%%%%%%%%%%%%%%
% \paragraph{Forwarding for Final Version of the Chapters.}
%
% The following forwarding files |cdocsfn1.tex| and |cdocsfn2.tex|
% (with identical content)
% compile the final versions of the child documents
% |cdocsch1.tex| and |cdocsch2.tex|, respectively:
%\iffalse
%<*samplefinal>
%\fi
%    \begin{macrocode}
\def\version{final}
\input{childdoc.def}
\childdocforwardprefix[cdocsamp]{cdocsfn}{cdocsch}
%    \end{macrocode}

%\iffalse
%</samplefinal>
%\fi
%
% %%%%%%%%%%%%%%%%%%%%%%%%%%%%%%%%%%%%%%
% \paragraph{Command Line Processing.}
%
% The following three command lines generate the output files
% |cdocscld|, |cdocscl1| and |cdocscl2|
% which should be identical to
% |cdocsdrf|, |cdocsch1| and |cdocsfn2|, respectively:
% \begin{center}
% \begin{tabular}{l}
% |latex -jobname cdocscld \|\\
% |  "\def\version{draft}\input{childdoc.def}\childdocforward{cdocsamp}"|\\
% |latex -jobname cdocscl1 \|\\
% |  "\input{childdoc.def}\childdocforward[cdocsamp]{cdocsch1}"|\\
% |latex -jobname cdocscl2 \|\\
% |  "\def\version{final}\input{childdoc.def}\childdocforward{cdocsch2}"|
% \end{tabular}
% \end{center}
% Note that the trailing backslash on each first line
% merely continues the input to the second line
% (for convenient cut ant paste).
% Furthermore, the command |latex| can be replaced by any
% of its alternative versions such as |pdflatex|.
%
% %%%%%%%%%%%%%%%%%%%%%%%%%%%%%%%%%%%%%%%%%%%%%%%%%%%%%%%%%%%%%%%%%%%%%%%%%%%%%%
% %%%%%%%%%%%%%%%%%%%%%%%%%%%%%%%%%%%%%%%%%%%%%%%%%%%%%%%%%%%%%%%%%%%%%%%%%%%%%%
% \section{Implementation}
%\iffalse
%<*package>
%\fi
%
% This section describes the definitions file |childdoc.def|.

% The definitions cannot be loaded using |\usepackage| or |\RequirePackage|
% which has a mechanism to prevent loading a style file more than once.
% When loading the definitions by means of |\input|
% multiple instances have to be prevented manually:
%\iffalse
%This code needs to be before the `\ProvidesFile' directive
%which is defined at the beginning of this file.
%Therefore it is also placed there and commented out here.
%</package>
%<*discard>
%\fi
%    \begin{macrocode}
\ifdefined\childdocmain\endinput\fi
%    \end{macrocode}
%\iffalse
%</discard>
%<*package>
%\fi
%
% \macro{\ifchilddoc}
% \macro{\ifchilddocmanual}
% The conditional |\ifchilddoc| tells whether a
% child (true) or main (false) document is being compiled.
% The conditional |\ifchilddocmanual| tells whether
% the |\includeonly| mechanism is used (false) or
% the selection of child files must be performed manually (true).
% The definitions initialise to false:
%    \begin{macrocode}
\newif\ifchilddoc
\newif\ifchilddocmanual
%    \end{macrocode}

% \macro{\childdocname}
% \macro{\childdocjob}
% The macro |\childdocname| stores the name of the main document
% to be compiled. The macro |\childdocjob| stores the name of
% the document on which the \LaTeX{} compiler was originally invoked.
% The content of |\jobname| cannot be compared
% to filenames specified in the source due to different catcodes.
% The following code rescans |\jobname|, stores the result
% in |\childdocname| and saves a copy in |\childdocjob|:
%    \begin{macrocode}
\edef\childdocname{\scantokens\expandafter{\jobname\noexpand}}
\let\childdocjob\childdocname
%    \end{macrocode}

% \macro{\childdocdisable}
% The macro |\childdocdisable| prevents the main file
% from being processed more than once.
% At this stage, the main document command |\childdocmain|
% is assumed to be called once again where it should do nothing.
% Any subsequent call to it should prevent
% a secondary processing of the main document
% It overwrites the forwarding commands
% |\childdocof| and |\childdocforward|
% with empty macros to prevent further inclusions of the main document:
%    \begin{macrocode}
\newcommand{\childdocdisable}
{
  \renewcommand{\childdocmain}[1]{\renewcommand{\childdocmain}[1]{\endinput}}
  \renewcommand{\childdocof}[1]{}
  \renewcommand{\childdocby}[2][]{}
  \renewcommand{\childdocforward}[2][]{}
  \renewcommand{\childdocdisable}{}
}
%    \end{macrocode}

% \macro{\childdocmain}
% The macro |\childdocmain| is to be called at the top of the main file
% with nothing or the main filename (without extension) as argument.
% First, it breaks loops.
% If the argument is not empty and does not match |\childdocname|
% (which is set by the first inclusion of |childdoc.def|),
% |\ifchilddoc| is set to true, |\includeonly| is applied to the child file
% and |\jobname| is set to the main file
% (for proper handling of |.aux| files):
%    \begin{macrocode}
\newcommand{\childdocmain}[1]
{
  \childdocdisable\childdocmain{}
  \if?#1?\else
    \begingroup
      \def\childdoctmp{#1}
      \ifx\childdoctmp\childdocname
        \def\childdoctmp{}
      \else
        \def\childdoctmp
        {
          \childdoctrue
          \includeonly{\childdocname}
          \def\childdocjob{#1}
          \def\jobname{#1}
        }
      \fi
      \expandafter
    \endgroup
    \childdoctmp
  \fi
}
%    \end{macrocode}

% \macro{\childdocof}
% The command |\childdocof| redirects
% compilation to the main file |#1|.
%    \begin{macrocode}
\newcommand{\childdocof}[1]
{
  \childdocdisable
  \childdoctrue
  \includeonly{\childdocname}
  \def\jobname{#1}
  \def\childdocjob{#1}
  \input{#1}
}
%    \end{macrocode}

% \macro{\childdocby}
% The command |\childdocby| ....
%    \begin{macrocode}
\newcommand{\childdocby}[2][]
{
  \childdocdisable
  \childdoctrue
  \childdocmanualtrue
  \if?#1?\else
    \def\jobname{#2}
  \fi
  \def\childdocjob{#2}
  \input{#2}
  \endinput
}
%    \end{macrocode}

% \macro{\childdocforward}
% The command |\childdocforward| redirects
% compilation to the main file or
% (if the optional argument is given) a child file.
% Parameters are set as if the main file
% or a child file starting with |\childdocof| was compiled.
% Then compilation is handed over to the main file:
%    \begin{macrocode}
\newcommand{\childdocforward}[2][]
{
  \begingroup
    \if?#1?
      \def\childdoctmp
      {
        \def\childdocname{#2}
        \def\childdocjob{#2}
        \def\jobname{#2}
        \input{#2}
        \endinput
      }
    \else
      \def\childdoctmp
      {
        \childdocdisable
        \def\childdocname{#2}
        \childdoctrue
        \includeonly{#2}
        \def\childdocjob{#1}
        \def\jobname{#1}
        \input{#1}
        \endinput
      }
    \fi
    \expandafter
  \endgroup
  \childdoctmp
}
%    \end{macrocode}

% \macro{\childdocforwardprefix}
% The command |\childdocforwardprefix| redirects
% compilation to the main or a child file by means of a pattern.
% The prefix |#1| in the current filename is replaced by |#2|
% and the suffix of the current filename is kept
% (it is assumed that the filename does not contain the substring `|~~~|'
% which is used as a delimiter).
% Compilation is handed over to the new file by |\childdocforward|:
%    \begin{macrocode}
\newcommand{\childdocforwardprefix}[3][]
{
  \begingroup
    \def\childdocextract #2##1~~~{\def\childdoctmp{\childdocforward[#1]{#3##1}}}
    \expandafter\childdocextract\childdocname~~~
    \expandafter
  \endgroup
  \childdoctmp
}
%    \end{macrocode}

% \macro{\childdoc}
% The deprecated macro |\childdoc| is a legacy version of |\childdocmain|:
%    \begin{macrocode}
\newcommand{\childdoc}{\childdocmain}
%    \end{macrocode}

% \macro{\childdocredirect}
% The deprecated macro |\childdocredirect| is a legacy version
% of |\childdocforward| and |\childdocforwardprefix|:
%    \begin{macrocode}
\newcommand{\childdocredirect}[2][]
{
  \begingroup
    \if?#1?
      \def\childdoctmp{\childdocforward{#2}}
    \else
      \def\childdoctmp{\childdocforwardprefix{#1}{#2}}
    \fi
    \expandafter
  \endgroup
  \childdoctmp
}
%    \end{macrocode}

%\iffalse
%</package>
%\fi
%
\endinput
|\\
|\childdocby{|\textit{main}|}|\\
\end{tabular}
\end{center}
%
Both forms have slightly different effects as described above.
The main file is prepared as usual, see \secref{sec:include}.

%%%%%%%%%%%%%%%%%%%%%%%%%%%%%%%%%%%%%%%%%%%%%%%%%%%%%%%%%%%%%%%%%%%%%%%%%%%%%%%%
\subsection{Legacy Detection}
\label{sec:detection}

The directive |\childdocmain| in the main file can detect
whether the complete document or merely a child is to be compiled
even without using the directive |\childdocof|.
This method is deprecated because it is less robust
and there is no compelling reason to use it;
it is merely provided for backward compatibility
and it may be removed in future versions.

If the detection mechanism is to be used,
it is mandatory to correctly specify
the filename of the main file as the argument of |\childdocmain|:
%
\begin{center}
\begin{tabular}{l}
|% \iffalse
%
% childdoc.dtx Copyright (C) 2017-2018 Niklas Beisert
%
% This work may be distributed and/or modified under the
% conditions of the LaTeX Project Public License, either version 1.3
% of this license or (at your option) any later version.
% The latest version of this license is in
%   http://www.latex-project.org/lppl.txt
% and version 1.3 or later is part of all distributions of LaTeX
% version 2005/12/01 or later.
%
% This work has the LPPL maintenance status `maintained'.
%
% The Current Maintainer of this work is Niklas Beisert.
%
% This work consists of the files childdoc.dtx and childdoc.ins
% and the derived files childdoc.def and cdocsamp.tex with
% cdocsch1.tex, cdocsch2.tex, cdocsdrf.tex, cdocsfn1.tex, cdocsfn2.tex.
%
%<package>\ifdefined\childdocmain\endinput\fi
%<package>\ProvidesFile{childdoc.def}[2018/12/30 v2.0 child document driver]
%<samplemain>\ProvidesFile{cdocsamp.tex}[2018/12/30 v2.0 sample for childdoc]
%<*driver>
%\ProvidesFile{childdoc.drv}[2018/12/30 v2.0 childdoc reference manual file]
\PassOptionsToClass{10pt,a4paper}{article}
\documentclass{ltxdoc}

\usepackage[margin=35mm]{geometry}
\usepackage{hyperref}
\usepackage{hyperxmp}
\usepackage[usenames]{color}

\hypersetup{colorlinks=true}
\hypersetup{pdfstartview=FitH}
\hypersetup{pdfpagemode=UseNone}
\hypersetup{pdfsource={}}
\hypersetup{pdflang={en-UK}}
\hypersetup{pdfcopyright={Copyright 2017-2018 Niklas Beisert.
  This work may be distributed and/or modified under the
  conditions of the LaTeX Project Public License, either version 1.3
  of this license or (at your option) any later version.}}
\hypersetup{pdflicenseurl={http://www.latex-project.org/lppl.txt}}
\hypersetup{pdfcontactaddress={ETH Zurich, ITP, HIT K,
  Wolfgang-Pauli-Strasse 27}}
\hypersetup{pdfcontactpostcode={8093}}
\hypersetup{pdfcontactcity={Zurich}}
\hypersetup{pdfcontactcountry={Switzerland}}
\hypersetup{pdfcontactemail={nbeisert@itp.phys.ethz.ch}}
\hypersetup{pdfcontacturl={http://people.phys.ethz.ch/\xmptilde nbeisert/}}

\newcommand{\secref}[1]{\hyperref[#1]{section \ref*{#1}}}

\parskip1ex
\parindent0pt
\let\olditemize\itemize
\def\itemize{\olditemize\parskip0pt}

\begin{document}

\title{The \textsf{childdoc} Package}
\hypersetup{pdftitle={The childdoc Package}}
\author{Niklas Beisert\\[2ex]
  Institut f\"ur Theoretische Physik\\
  Eidgen\"ossische Technische Hochschule Z\"urich\\
  Wolfgang-Pauli-Strasse 27, 8093 Z\"urich, Switzerland\\[1ex]
  \href{mailto:nbeisert@itp.phys.ethz.ch}
  {\texttt{nbeisert@itp.phys.ethz.ch}}}
\hypersetup{pdfauthor={Niklas Beisert}}
\hypersetup{pdfsubject={Manual for the LaTeX2e Package childdoc}}
\date{30 December 2018, \textsf{v2.0}}
\maketitle

\begin{abstract}\noindent
\textsf{childdoc} is a \LaTeXe{} package
that enables the direct compilation
of document sections included by |\include|
to individual files.
\end{abstract}

\begingroup
\parskip0ex
\tableofcontents
\endgroup

%%%%%%%%%%%%%%%%%%%%%%%%%%%%%%%%%%%%%%%%%%%%%%%%%%%%%%%%%%%%%%%%%%%%%%%%%%%%%%%%
%%%%%%%%%%%%%%%%%%%%%%%%%%%%%%%%%%%%%%%%%%%%%%%%%%%%%%%%%%%%%%%%%%%%%%%%%%%%%%%%
\section{Introduction}

\LaTeX{} provides a mechanism to structure a large document (such as a book)
into a main file and several child files (containing the chapters)
using the |\include| command.
This mechanism is beneficial for documents
which span hundreds of pages in order to
make the source file(s) more manageable.
Moreover, compilation can be restricted to
selected child files by means of the |\includeonly| command.
The latter feature can be used to reduce the compilation time while editing
(this was significantly more useful in the earlier days of \LaTeX{})
or to generate a smaller document which is easier to navigate.
Another application of |\includeonly| is to generate
documents consisting of selected parts of the complete document.

However, there are a few drawbacks of the plain |\include| mechanism:
\begin{itemize}
\item
The child files cannot be compiled on their own,
they can only be compiled via the main file.
A naive editing environment
(such as a text editor with an option
to have the current file processed by \LaTeX)
may require one to switch to the main file before compiling;
attempting to compile the child file produces errors.
\item
The main file must be modified (each time)
to adjust the |\includeonly| command
to the present needs. This easily leaves the main file in a messy state.
\item
The generated document will always carry the filename
of the main document. This is inconvenient if
several child files are to be compiled and
to be kept for distribution.
\end{itemize}

The present package provides a simple interface
to make child files individually compilable by \LaTeX{}.
Compiling a child file then has the same effect as compiling
the main file with an |\includeonly| command
to select the appropriate child.
Moreover the generated document will carry the name of the child
rather than the main file.
This resolves all three above issues.

This feature is meant to make the editing of books,
thesis documents and lecture notes somewhat more convenient.
However, the package can also be used efficiently for
composing a series of documents (such as exercise sheets)
which are typically distributed individually.
It then assists the author in generating the individual documents
(potentially in different versions)
as well as a document containing the collected series.
Another application is in developing style files
or other kinds of included material
where compilation of the style file could redirect
to a sample or test file.

%%%%%%%%%%%%%%%%%%%%%%%%%%%%%%%%%%%%%%%%%%%%%%%%%%%%%%%%%%%%%%%%%%%%%%%%%%%%%%%%
%%%%%%%%%%%%%%%%%%%%%%%%%%%%%%%%%%%%%%%%%%%%%%%%%%%%%%%%%%%%%%%%%%%%%%%%%%%%%%%%
\section{Usage}

First of all, the package \textsf{childdoc} is \emph{not} a standard
\LaTeXe{} |.sty| style file! Therefore it needs to be invoked in
a non-standard way.

%%%%%%%%%%%%%%%%%%%%%%%%%%%%%%%%%%%%%%%%%%%%%%%%%%%%%%%%%%%%%%%%%%%%%%%%%%%%%%%%
\subsection{Included Files}
\label{sec:include}

%%%%%%%%%%%%%%%%%%%%%%%%%%%%%%%%%%%%%%%%
\DescribeMacro{\childdocmain}
To use the package, add the commands
\begin{center}
\begin{tabular}{l}
|\input{childdoc.def}|\\
|\childdocmain{}|\\
\end{tabular}
\end{center}
at the very top of the main \LaTeX{} file,
in particular \emph{before} the |\documentclass| statement!
The argument of |\childdocmain| should be left empty
(but it must be present).

%%%%%%%%%%%%%%%%%%%%%%%%%%%%%%%%%%%%%%%%
\DescribeMacro{\childdocof}
Furthermore, add the commands
\begin{center}
\begin{tabular}{l}
|\input{childdoc.def}|\\
|\childdocof{|\textit{main}|}|\\
\end{tabular}
\end{center}
at the top of every child file \textit{child}
which is included by |\include{|\textit{child}|}|
from within the main file
(or at least for those files to be compiled individually).
The argument \textit{main} must be the filename of the main file.

There are a couple of
considerations in setting up the main and child documents:

%%%%%%%%%%%%%%%%%%%%%%%%%%%%%%%%%%%%%%%%
\paragraph{Restrictions.}

Please note the following restrictions:
\begin{itemize}
\item
|\childdocmain| must be called with one argument \textit{main}
to ensure compatibility with earlier version of the package.
It must either be empty (|\childdocmain{}|)
or precisely match the filename of the main file in which it is specified.
See \secref{sec:detection} for further information.
\item
The filename \textit{main} must be specified without the |.tex| extension.
\item
The filename \textit{main} is case sensitive
(even in case-insensitive file systems)
due to internal string comparison.
\item
The argument \textit{main} should be fully expanded, it cannot be a macro.
\item
Subdirectories and special characters should be avoided in filenames.
\item
The command |\childdocmain{|\textit{main}|}| must be followed by a whitespace.
It should not be followed immediately by another command
or by a comment mark `|%|'.
This is because the \TeX{} parser reads the token immediately following
the argument of |\childdocmain| and puts it
at the beginning of every child section;
however, a white\-space is ignored.
\end{itemize}

%%%%%%%%%%%%%%%%%%%%%%%%%%%%%%%%%%%%%%%%
\paragraph{Content of Main File.}

It is advisable to place all content in the child files included by |\include|.
Any output contained in the main file will appear in all child documents
unless suppressed manually;
it cannot be suppressed automatically by the |\includeonly| directive
and thus should normally be avoided.
A method to include some content in the main file
by means of conditional processing is described in \secref{sec:conditional}.

%%%%%%%%%%%%%%%%%%%%%%%%%%%%%%%%%%%%%%%%
\paragraph{Page Numbering.}

When only a part of the document is compiled,
the appropriate numbering of pages
(as well as other status parameters)
is determined from the |.aux| files.
The latter contain information from previous passes.
However this information needs to propagate through
all intermediate child documents.
Therefore the page numbering in child documents may well
be inconsistent until the complete document is compiled at least once.

A useful (if unconventional) way to always ensure a consistent
page numbering is to restart the numbering in each child document
and denote the pages by `\textit{child}|.|\textit{page}'
where \textit{child} represents the chapter/section number of the child file.
This can be achieved by the command
|\numberwithin{page}{|\textit{child}|}|
of the \textsf{amsmath} package
where \textit{child} can be |chapter| or |section|
depending on the chosen structuring.
Alternatively, one can modify the macro |\thepage| appropriately
and reset the counter |page| at the start of each child file.

%%%%%%%%%%%%%%%%%%%%%%%%%%%%%%%%%%%%%%%%%%%%%%%%%%%%%%%%%%%%%%%%%%%%%%%%%%%%%%%%
\subsection{Conditional Processing}
\label{sec:conditional}

The package provides a mechanism to compile different versions
of a document. To customise the versions further some conditional processing
can come in handy to distinguish which version is being compiled.
The package provides two macros to describe the compilation context:

%%%%%%%%%%%%%%%%%%%%%%%%%%%%%%%%%%%%%%%%
\DescribeMacro{\ifchilddoc}
The conditional |\ifchilddoc| distinguishes between the compilation of
child documents and the main document:
%
\begin{center}
|\ifchilddoc |\textit{child-code}| |[|\||else |\textit{main-code}]| \||fi|
\end{center}

%%%%%%%%%%%%%%%%%%%%%%%%%%%%%%%%%%%%%%%%
\DescribeMacro{\childdocname}
\DescribeMacro{\childdocjob}
The macro |\childdocname| contains the filename (without extension)
of the main or child file being processed.
Note that |\childdocjob| will always contain the name of the main file.

%%%%%%%%%%%%%%%%%%%%%%%%%%%%%%%%%%%%%%%%
\paragraph{Title Page.}

Conditional processing can be used to include a title or banner page
in the main document when proper precautions are taken.
Importantly, the code in the main file should ensure that the page counter
(as well as other status parameters which are stored in the |.aux| files)
takes the same value after the conditional processing.
Otherwise the page numbers may take divergent values
depending on which part is compiled.

For example, a title page could be declared by:
%
\begin{center}
\begin{tabular}{l}
|\ifchilddoc\||else|\\
|\addtocounter{page}{-1}|\\
\textit{code for title page}\\
|\newpage|\\
|\||fi|
\end{tabular}
\end{center}
%
A banner page for the child documents can be generated by:
%
\begin{center}
\begin{tabular}{l}
|\ifchilddoc|\\
|\addtocounter{page}{-1}|\\
\textit{code for banner page}\\
|\newpage|\\
|\||fi|
\end{tabular}
\end{center}
%
Here one could write a message such as:
\begin{center}
|This is the part \childdocname{} of \childdocjob{}.|
\end{center}

%%%%%%%%%%%%%%%%%%%%%%%%%%%%%%%%%%%%%%%%%%%%%%%%%%%%%%%%%%%%%%%%%%%%%%%%%%%%%%%%
\subsection{Flags}
\label{sec:flags}

The package makes it easy to generate different versions
of the main or child documents.
To this end compilation flags can be defined
and assigned different default values.
They will be particularly useful in conjunction
with the forwarding mechanism described in \secref{sec:forward}.

For example, it may be useful to have a flag |\version|
which can be set to |draft| or |final|.
The document source will contain some conditional code
depending on the value of |\version|.
Suppose further, the flag should default to |final| for the main file
and to |draft| for child files
which is a natural assignment for editing the document.
This is achieved by placing the following code
in the preamble of the main document
(below the |\childdocmain| directive):
%
\begin{center}
\begin{tabular}{l}
|\ifchilddoc|\\
|\providecommand{\version}{draft}|\\
|\||else|\\
|\providecommand{\version}{final}|\\
|\||fi|
\end{tabular}
\end{center}
%
The definition by |\providecommand| makes sure
that previous definitions are not overwritten.
Further statements |\providecommand{\version}{...}|
can thus be added before the above code to override it.

For the main file, one might add a line
(between |\childdocmain| and the above block)
%
\begin{center}
|%\ifchilddoc\||else\providecommand{\version}{draft}\||fi|
\end{center}
%
which can be uncommented to produce a draft version.
Likewise one can add a line to the very top of a child file
(above the |\childdocof{|\textit{main}|}| directive)
%
\begin{center}
|%\providecommand{\version}{final}|
\end{center}
%
which can be uncommented to produce the final version of this child document.

%%%%%%%%%%%%%%%%%%%%%%%%%%%%%%%%%%%%%%%%%%%%%%%%%%%%%%%%%%%%%%%%%%%%%%%%%%%%%%%%
\subsection{Forwarding}
\label{sec:forward}

Different versions of the main or child documents
using compilation flags as described in \secref{sec:flags}
can be (permanently) stored in different files
for convenient compilation, viewing and distribution.
To this end, the package defines a command
to pass on compilation to a different file:

%%%%%%%%%%%%%%%%%%%%%%%%%%%%%%%%%%%%%%%%
\DescribeMacro{\childdocforward}
The command |\childdocforward| redirects processing to
another source file:
%
\begin{center}
\begin{tabular}{l}
|\input{childdoc.def}|\\
|\childdocforward[|\textit{main}|]{|\textit{dest}|}|\\
\end{tabular}
\end{center}
%
The argument \textit{dest} is the destination file
(without extension).
It should be the main file or one of the child files.
Note that further \textsf{childdoc} directives
such as |\childdocof| and |\childdocforward|
in the indicated file will be processed in this form.
The optional argument \textit{main}
passes on directly to the main file \textit{main}
while pretending to compile the child \textit{dest}.
This form behaves as if \textit{dest}
issues |\childdocof{|\textit{main}|}| right away,
and no further \textsf{childdoc} directives will be processed.

%%%%%%%%%%%%%%%%%%%%%%%%%%%%%%%%%%%%%%%%
\DescribeMacro{\...prefix}
In the alternative form |\childdocforwardprefix|,
%
\begin{center}
\begin{tabular}{l}
|\input{childdoc.def}|\\
|\childdocforwardprefix[|\textit{main}|]{|\textit{prefix}|}{|\textit{dest}|}|
\end{tabular}
\end{center}
%
the destination file is determined by a pattern
depending on the current file:
To make this work, the current file must be called
`{\textit{prefix}\hspace{0.2em}\textit{suffix}}'
with \textit{prefix} matching precisely the argument.
Processing is then passed on to the file
`{\textit{dest}\hspace{0.2em}\textit{suffix}}'.
Surely, the same effect is achieved by
directly specifying the
argument `{\textit{dest}\hspace{0.2em}\textit{suffix}}'
in the first form.
However, that requires to set up a different file
for each child. With the alternative form of the command
all these files can have exactly the same content
which simplifies setting them up and maintaining them.

For example, the following file |draft.tex|
with a compilation flag |\version| as described in \secref{sec:flags}
compiles the main document as a draft:
%
\begin{center}
\begin{tabular}{l}
|\def\version{draft}|\\
|\input{childdoc.def}|\\
|\childdocforward{|\textit{main}|}|
\end{tabular}
\end{center}
%
Likewise, the following files |final|\textit{nn}|.tex|
compile the final version of the child document
|child|\textit{nn}|.tex|:
%
\begin{center}
\begin{tabular}{l}
|\def\version{final}|\\
|\input{childdoc.def}|\\
|\childdocforwardprefix{final}{child}|
\end{tabular}
\end{center}
%

Note that when several versions of a main file and/or of each child file
are to be generated, it may be convenient to set up a |Makefile| or
shell script to automatise the process.

%%%%%%%%%%%%%%%%%%%%%%%%%%%%%%%%%%%%%%%%%%%%%%%%%%%%%%%%%%%%%%%%%%%%%%%%%%%%%%%%
\subsection{Command Line Processing}
\label{sec:commandline}

The effect of redirection files can also be achieved by invoking
the \LaTeX{} compiler with a more elaborate command line.
Most conveniently this should be done as part
of a shell script or a |Makefile|.

When using \textsf{childdoc} in the main file, the following
command lines effectively perform a redirection
(note that depending on the shell being used,
backslashes may have to be doubled: `|\|' $\to$ `|\\|'):
%
\begin{center}
|... -jobname "|\textit{target}|" |\\|"|[\textit{flags}]%
|\input{childdoc.def}\childdocforward[|\textit{main}|]{|\textit{dest}|}"|
\end{center}
%
Here \textit{target} is the name of the output file,
\textit{main} is the name of the main file
and \textit{dest} is the name of the main or child file to be processed
(all filenames without extensions).
The optional argument \textit{main} can be omitted
if \textit{main} matches \textit{dest}.
Optionally, compilation \textit{flags} can be defined via |\def| commands.
This command line makes the \TeX{} engine believe
it is compiling the file \textit{target}
whose content is specified as the latter parameter.
The provided code then forwards the processing to
\textit{main} or \textit{dest} as described in \secref{sec:forward}.

%%%%%%%%%%%%%%%%%%%%%%%%%%%%%%%%%%%%%%%%%%%%%%%%%%%%%%%%%%%%%%%%%%%%%%%%%%%%%%%%
\subsection{Include by Input}
\label{sec:input}

Including child documents by |\include| has some restrictions by design.
Most notably, the content of a child document always occupies
its own set of pages; pages cannot be shared between child documents.
Usually, this behaviour makes perfect sense
because each child document contain an essential part of the document.
However, in some situations it may be desirable to compose
a document from a collection of parts
without having mandatory page breaks between then.
For this case, the package
provides a mechanism to include parts
by |\input| which can also be processed individually.
However, by construction this mechanism
requires manual handling of the content to be output.

%%%%%%%%%%%%%%%%%%%%%%%%%%%%%%%%%%%%%%%%
\DescribeMacro{\ifchilddocmanual}
The main file should be prepared as usual, see \secref{sec:include}.
However, the document body must make a distinction
between processing of an individual part and of the main document, e.g.:
%
\begin{center}
\begin{tabular}{l}
|\ifchilddocmanual|\\
|\input{\childdocname}|\\
|\||else|\\
\textit{document body with }|\input{|\textit{part}|}|\\
|\||fi|
\end{tabular}
\end{center}
%
The conditional |\ifchilddocmanual| is true whenever
a part to be included by |\input| is being compiled,
and the name of the part is stored in |\childdocname|.

%%%%%%%%%%%%%%%%%%%%%%%%%%%%%%%%%%%%%%%%
\DescribeMacro{\childdocby}
Each part to be included by |\input| should start with:
%
\begin{center}
\begin{tabular}{l}
|\input{childdoc.def}|\\
|\childdocby{|\textit{main}|}|\\
\end{tabular}
\end{center}
%
The directive |\childdocby| is similar to |\childdocof|
described in \secref{sec:include},
but the subsequent selection of content must be done manually.
To that end, both |\ifchilddoc| and |\ifchilddocmanual|
will be true upon processing of a part,
and the name of the part is stored in |\childdocname|.
Note that |\jobname| will be set to the filename of the current part
so that each part receives an individual |.aux| file
that does not interfere with the |.aux| file(s) of the main document.
This behaviour can be altered by the alternative form
|\childdocby[*]{|\textit{main}|}| (with a non-empty optional argument)
which uses the |.aux| file of the main document
by setting |\jobname| to \textit{main}.

%%%%%%%%%%%%%%%%%%%%%%%%%%%%%%%%%%%%%%%%%%%%%%%%%%%%%%%%%%%%%%%%%%%%%%%%%%%%%%%%
\subsection{Driver Development}
\label{sec:driver}

The \textsf{childdoc} mechanism can also be use for the development
of definition files such as \LaTeX{} styles or classes.
This case differs from the above setup with multiple parts
included by |\include| in that no |\includeonly| should be invoked.
This can be achieved by starting the include file
(before |\ProvidesPackage|) with:
%
\begin{center}
\begin{tabular}{l}
|\input{childdoc.def}|\\
|\childdocforward{|\textit{main}|}|\\
\end{tabular}
\end{center}
%
or alternatively with:
%
\begin{center}
\begin{tabular}{l}
|\input{childdoc.def}|\\
|\childdocby{|\textit{main}|}|\\
\end{tabular}
\end{center}
%
Both forms have slightly different effects as described above.
The main file is prepared as usual, see \secref{sec:include}.

%%%%%%%%%%%%%%%%%%%%%%%%%%%%%%%%%%%%%%%%%%%%%%%%%%%%%%%%%%%%%%%%%%%%%%%%%%%%%%%%
\subsection{Legacy Detection}
\label{sec:detection}

The directive |\childdocmain| in the main file can detect
whether the complete document or merely a child is to be compiled
even without using the directive |\childdocof|.
This method is deprecated because it is less robust
and there is no compelling reason to use it;
it is merely provided for backward compatibility
and it may be removed in future versions.

If the detection mechanism is to be used,
it is mandatory to correctly specify
the filename of the main file as the argument of |\childdocmain|:
%
\begin{center}
\begin{tabular}{l}
|\input{childdoc.def}|\\
|\childdocmain{|\textit{main}|}|\\
\end{tabular}
\end{center}
%
If |\jobname| does not match the argument \textit{main} of |\childdocmain|,
it is assumed that |\jobname| points to the child file to be compiled.
When using |\childdocmain| with the main file specified as argument,
it suffices to start a child file
with just |\input{|\textit{main}|}|
without loading of the package and using |\childdocof|.
If instead all processing is done
with the appropriate \textsf{childdoc} directives,
the argument of \textit{main} of |\childdocmain| can be empty.

An alternative version of the command line processing described
in \secref{sec:commandline} using the detection mechanism reads:
%
\begin{center}
|... -jobname "|\textit{target}|" "|[\textit{flags}]%
[|\def\jobname{|\textit{dest}|}|]|\input{|\textit{main}|}"|
\end{center}

%%%%%%%%%%%%%%%%%%%%%%%%%%%%%%%%%%%%%%%%%%%%%%%%%%%%%%%%%%%%%%%%%%%%%%%%%%%%%%%%
\subsection{Manual Code}
\label{sec:manual}

In case one cannot be certain whether the definitions file |childdoc.def|
is installed on the target \TeX{} distribution
and one prefers not to ship it,
it is conceivable to paste a few relevant commands into the sources.

To that end, drop all statements |\input{childdoc.def}|
and perform the replacements as outlined below.
Instead of |\childdocmain{|\textit{main}|}| add the following code
to the top of the main file:
%
\begin{center}
\begin{tabular}{l}
|\||ifdefined\childdocname\endinput\||fi\newif\ifchilddoc|\\
|\edef\childdocname{\scantokens\expandafter{\jobname\noexpand}}|\\
|\def\childdocmain{|\textit{main}|}\||ifx\childdocmain\childdocname\||else|\\
|\childdoctrue\includeonly{\childdocname}\let\jobname\childdocmain\||fi|\\
\end{tabular}
\end{center}
%
Instead of |\childdocof{|\textit{main}|}| just include the main file
at the top of each child file:
%
\begin{center}
|\input{|\textit{main}|}|
\end{center}
%
A simple redirection |\childdocforward{|\textit{dest}|}| is achieved by:
%
\begin{center}
|\def\jobname{|\textit{dest}|}\input{\jobname}|
\end{center}
%
The redirection with prefix
|\childdocforwardprefix[|\textit{prefix}|]{|\textit{dest}|}|
is accomplished by:
%
\begin{center}
\begin{tabular}{l}
|{\edef\jobname{\scantokens\expandafter{\jobname\noexpand}}|\\
|\def\redirectjob |\textit{prefix}|#1~~~{\gdef\jobname{|\textit{dest}|#1}}|\\
|\expandafter\redirectjob\jobname~~~}\input{\jobname}|
\end{tabular}
\end{center}

In an alternative approach,
child documents can be compiled by a specific command line
without additional code or specific definitions:
%
\begin{center}
|... -jobname "|\textit{target}|" "|[\textit{flags}]%
|\includeonly{|\textit{dest}|}\input{|\textit{main}|}"|
\end{center}
%

%%%%%%%%%%%%%%%%%%%%%%%%%%%%%%%%%%%%%%%%%%%%%%%%%%%%%%%%%%%%%%%%%%%%%%%%%%%%%%%%
%%%%%%%%%%%%%%%%%%%%%%%%%%%%%%%%%%%%%%%%%%%%%%%%%%%%%%%%%%%%%%%%%%%%%%%%%%%%%%%%
\section{Information}

%%%%%%%%%%%%%%%%%%%%%%%%%%%%%%%%%%%%%%%%%%%%%%%%%%%%%%%%%%%%%%%%%%%%%%%%%%%%%%%%
\subsection{Copyright}

Copyright \copyright{} 2017--2018 Niklas Beisert

This work may be distributed and/or modified under the
conditions of the \LaTeX{} Project Public License, either version 1.3
of this license or (at your option) any later version.
The latest version of this license is in
  \url{http://www.latex-project.org/lppl.txt}
and version 1.3 or later is part of all distributions of \LaTeX{}
version 2005/12/01 or later.

This work has the LPPL maintenance status `maintained'.

The Current Maintainer of this work is Niklas Beisert.

This work consists of the files |README.txt|, |childdoc.ins| and |childdoc.dtx|
as well as the derived files |childdoc.def|, |cdocsamp.tex|
with |cdocsch1.tex|, |cdocsch2.tex|, |cdocspt3.tex|, |cdocspt4.tex|,
|cdocsdrf.tex|, |cdocsfn1.tex|, |cdocsfn2.tex|
as well as |childdoc.pdf|.

%%%%%%%%%%%%%%%%%%%%%%%%%%%%%%%%%%%%%%%%%%%%%%%%%%%%%%%%%%%%%%%%%%%%%%%%%%%%%%%%
\subsection{Files and Installation}

The package consists of the files:
%
\begin{center}
\begin{tabular}{ll}
    |README.txt|   & readme file \\
    |childdoc.ins| & installation file \\
    |childdoc.dtx| & source file \\
    |childdoc.def| & definition file \\
    |cdocsamp.tex| & sample main file \\
    |cdocsch1.tex| & sample include file \\
    |cdocsch2.tex| & sample include file \\
    |cdocspt3.tex| & sample part file \\
    |cdocspt4.tex| & sample part file \\
    |cdocsdrf.tex| & sample redirection file \\
    |cdocsfn1.tex| & sample redirection file \\
    |cdocsfn2.tex| & sample redirection file \\
    |childdoc.pdf| & manual
\end{tabular}
\end{center}
%
The distribution consists of the files
|README.txt|, |childdoc.ins| and |childdoc.dtx|.
%
\begin{itemize}
\item
Run (pdf)\LaTeX{} on |childdoc.dtx|
to compile the manual |childdoc.pdf| (this file).
\item
Run \LaTeX{} on |childdoc.ins| to create the definitions file |childdoc.def|
and the sample |cdocsamp.tex| with include files
|cdocsch1.tex|, |cdocsch2.tex|, |cdocspt3.tex|, |cdocspt4.tex|,
|cdocsdrf.tex|, |cdocsfn1.tex|, |cdocsfn2.tex|.
Then copy the file |childdoc.def| to an appropriate directory of your \LaTeX{}
distribution, e.g.\ \textit{texmf-root}|/tex/latex/childdoc|.
\end{itemize}

%%%%%%%%%%%%%%%%%%%%%%%%%%%%%%%%%%%%%%%%%%%%%%%%%%%%%%%%%%%%%%%%%%%%%%%%%%%%%%%%
\subsection{Related CTAN Packages}

There are several other packages which offer a similar functionality:
%
\begin{itemize}
\item
The packages
\href{http://ctan.org/pkg/docmute}{\textsf{docmute}},
\href{http://ctan.org/pkg/includex}{\textsf{includex}} and
\href{http://ctan.org/pkg/standalone}{\textsf{standalone}}
provide commands to include only the document body of
a child file thus allowing both files to be compiled individually.
\item
The packages \href{http://ctan.org/pkg/subdocs}{\textsf{subdocs}}
and \href{http://ctan.org/pkg/subfiles}{\textsf{subfiles}}
provide structures in which the main and child documents can be
encapsulated and allowing them to be compiled individually.
The inclusion mechanism is different from the conventional |\include|.
\item
The package \href{http://ctan.org/pkg/combine}{\textsf{combine}}
is an elaborate solution to combine several documents into one.
\end{itemize}
%
See also the CTAN topic \href{http://ctan.org/topic/subdocs}{\textsf{subdocs}}
for further related packages.
The present package differs from the above solutions in that
a document structure constructed with the conventional |\include| mechanism
just needs two extra commands at the top of every file
such that all constituent files can be compiled individually.

%%%%%%%%%%%%%%%%%%%%%%%%%%%%%%%%%%%%%%%%%%%%%%%%%%%%%%%%%%%%%%%%%%%%%%%%%%%%%%%%
%\subsection{Feature Suggestions}
%
%The following is a list of features which may be useful for future
%versions of this package:
%%
%\begin{itemize}
%\item
%\ldots
%\end{itemize}

%%%%%%%%%%%%%%%%%%%%%%%%%%%%%%%%%%%%%%%%%%%%%%%%%%%%%%%%%%%%%%%%%%%%%%%%%%%%%%%%
\subsection{Revision History}

%%%%%%%%%%%%%%%%%%%%%%%%%%%%%%%%%%%%%%%%
\paragraph{v2.0:} 2018/12/30

\begin{itemize}
\item
immediate forward processing
\item
added |\childdocby| mechanism
\item
manual restructured
\end{itemize}

%%%%%%%%%%%%%%%%%%%%%%%%%%%%%%%%%%%%%%%%
\paragraph{v1.6:} 2018/01/17

\begin{itemize}
\item
application for development of include files
\item
corrections to manual
\end{itemize}

%%%%%%%%%%%%%%%%%%%%%%%%%%%%%%%%%%%%%%%%
\paragraph{v1.5:} 2017/05/21

\begin{itemize}
\item
more complete structuring introduced
\item
|\childdocof| introduced
\item
|\childdoc| renamed to |\childdocmain|
\item
|\childredirect| renamed to |\childdocforward| and |\childdocforwardprefix|
and functionality expanded
\end{itemize}

%%%%%%%%%%%%%%%%%%%%%%%%%%%%%%%%%%%%%%%%
\paragraph{v1.0:} 2017/04/27

\begin{itemize}
\item
manual and install package
\item
first version published on CTAN
\end{itemize}

%%%%%%%%%%%%%%%%%%%%%%%%%%%%%%%%%%%%%%%%
\paragraph{v0.6:} 2017/04/26

\begin{itemize}
\item
redirection mechanism added
\end{itemize}

%%%%%%%%%%%%%%%%%%%%%%%%%%%%%%%%%%%%%%%%
\paragraph{v0.5:} 2017/04/26

\begin{itemize}
\item
functionality in definition file
\end{itemize}


%%%%%%%%%%%%%%%%%%%%%%%%%%%%%%%%%%%%%%%%%%%%%%%%%%%%%%%%%%%%%%%%%%%%%%%%%%%%%%%%
%%%%%%%%%%%%%%%%%%%%%%%%%%%%%%%%%%%%%%%%%%%%%%%%%%%%%%%%%%%%%%%%%%%%%%%%%%%%%%%%
%%%%%%%%%%%%%%%%%%%%%%%%%%%%%%%%%%%%%%%%%%%%%%%%%%%%%%%%%%%%%%%%%%%%%%%%%%%%%%%%
\appendix

\settowidth\MacroIndent{\rmfamily\scriptsize 000\ }

 \DocInput{childdoc.dtx}

\end{document}
%</driver>
% \fi
%
% %%%%%%%%%%%%%%%%%%%%%%%%%%%%%%%%%%%%%%%%%%%%%%%%%%%%%%%%%%%%%%%%%%%%%%%%%%%%%%
% %%%%%%%%%%%%%%%%%%%%%%%%%%%%%%%%%%%%%%%%%%%%%%%%%%%%%%%%%%%%%%%%%%%%%%%%%%%%%%
% \section{Sample}
%\iffalse
%<*samplemain>
%\fi
%
% The following presents a sample document
% with two chapters, two parts, a title page,
% a compile flag as well as three forwarding files to set the flag.
% It consists of eight |.tex| files:
% \begin{center}
% \begin{tabular}{ll}
% |cdocsamp.tex|&main file\\
% |cdocsch1.tex|&include file for chapter 1\\
% |cdocsch2.tex|&include file for chapter 2\\
% |cdocspt3.tex|&include file for part 3\\
% |cdocspt4.tex|&include file for part 4\\
% |cdocsdrf.tex|&forwarding file for main file in draft mode\\
% |cdocsfi1.tex|&forwarding file for final version of chapter 1\\
% |cdocsfi2.tex|&forwarding file for final version of chapter 2\\
% \end{tabular}
% \end{center}
% Each of the eight files can be compiled directly by the \LaTeX{} compiler.
%
% %%%%%%%%%%%%%%%%%%%%%%%%%%%%%%%%%%%%%%
% \paragraph{Main File.}
%
% The main file is called |cdocsamp.tex|.
%
% Load the \textsf{childdoc} definitions and
% declare the filename for the main document:
%    \begin{macrocode}
\input{childdoc.def}
\childdocmain{}
%    \end{macrocode}

% Optional override for |\version| flag:
%    \begin{macrocode}
%%\ifchilddoc\else\providecommand{\version}{draft}\fi
%    \end{macrocode}

% Define the default values for the |\version| flag
% (|final| for the main file and |draft| for childs):
%    \begin{macrocode}
\ifchilddoc
\providecommand{\version}{draft}
\else
\providecommand{\version}{final}
\fi
%    \end{macrocode}

% Load the standard document class:
%    \begin{macrocode}
\documentclass[12pt]{article}
%    \end{macrocode}

% Start the document body:
%    \begin{macrocode}
\begin{document}
%    \end{macrocode}

% Declare a title page.
% Print title, part of document being processed and version flag:
%    \begin{macrocode}
\addtocounter{page}{-1}
\begin{center}
{\LARGE\bfseries{}childdoc example\par}
\vspace{1cm}
\ifchilddoc
\ifchilddocmanual part\else chapter\fi:
`\childdocname' of `\childdocjob'\par
\else
main document: `\childdocjob'\par
\fi
version: \version\par
\end{center}
\newpage
%    \end{macrocode}

% Manually include selected file,
% otherwise process as usual:
%    \begin{macrocode}
\ifchilddocmanual
\section*{part `\childdocname'}
\input{\childdocname}
\else
%    \end{macrocode}

% Include the two chapters:
%    \begin{macrocode}
\include{cdocsch1}
\include{cdocsch2}
%    \end{macrocode}

% Include the two parts unless only chapters should be displayed:
%    \begin{macrocode}
\ifchilddoc\else
\section{part three}
\input{cdocspt3}
\section{part four}
\input{cdocspt4}
\fi
%    \end{macrocode}

% Process as usual until here:
%    \begin{macrocode}
\fi
%    \end{macrocode}

% End of document body:
%    \begin{macrocode}
\end{document}
%    \end{macrocode}
%\iffalse
%</samplemain>
%\fi
%
% %%%%%%%%%%%%%%%%%%%%%%%%%%%%%%%%%%%%%%
% \paragraph{Chapter Include Files.}
%
% The include files are called |cdocsch1.tex| and |cdocsch2.tex|.
%
%\iffalse
%<*samplechap1|samplechap2>
%\fi

% Optional override for |\version| flag:
%    \begin{macrocode}
%%\providecommand{\version}{final}
%    \end{macrocode}

% Include the main document:
%    \begin{macrocode}
\input{childdoc.def}
\childdocof{cdocsamp}
%    \end{macrocode}

%\iffalse
%</samplechap1|samplechap2>
%\fi
%
%\iffalse
%<*samplechap1>
%\fi
% Some text for chapter 1:
%    \begin{macrocode}
\section{one}
some text in chapter one
%    \end{macrocode}

%\iffalse
%</samplechap1>
%\fi
% Some text for chapter 2:
%\iffalse
%<*samplechap2>
%\fi
%    \begin{macrocode}
\section{two}
more text in chapter two
%    \end{macrocode}

%\iffalse
%</samplechap2>
%\fi
%
% %%%%%%%%%%%%%%%%%%%%%%%%%%%%%%%%%%%%%%
% \paragraph{Part Include Files.}
%
% The include files are called |cdocspt3.tex| and |cdocspt4.tex|.
%
%\iffalse
%<*samplepart3|samplepart4>
%\fi

% Optional override for |\version| flag:
%    \begin{macrocode}
%%\providecommand{\version}{final}
%    \end{macrocode}

% Include the main document:
%    \begin{macrocode}
\input{childdoc.def}
\childdocby{cdocsamp}
%    \end{macrocode}

%\iffalse
%</samplepart3|samplepart4>
%\fi
%
%\iffalse
%<*samplepart3>
%\fi
% Some text for part 3:
%    \begin{macrocode}
some text in part three
%    \end{macrocode}

%\iffalse
%</samplepart3>
%\fi
% Some text for part 4:
%\iffalse
%<*samplepart4>
%\fi
%    \begin{macrocode}
more text in part four
%    \end{macrocode}

%\iffalse
%</samplepart4>
%\fi
%
% %%%%%%%%%%%%%%%%%%%%%%%%%%%%%%%%%%%%%%
% \paragraph{Forwarding for a Complete Draft.}
%
% The following forwarding file |cdocsdrf.tex|
% compiles the main document in draft mode:
%\iffalse
%<*sampledraft>
%\fi
%    \begin{macrocode}
\def\version{draft}
\input{childdoc.def}
\childdocforward{cdocsamp}
%    \end{macrocode}

%\iffalse
%</sampledraft>
%\fi
%
% %%%%%%%%%%%%%%%%%%%%%%%%%%%%%%%%%%%%%%
% \paragraph{Forwarding for Final Version of the Chapters.}
%
% The following forwarding files |cdocsfn1.tex| and |cdocsfn2.tex|
% (with identical content)
% compile the final versions of the child documents
% |cdocsch1.tex| and |cdocsch2.tex|, respectively:
%\iffalse
%<*samplefinal>
%\fi
%    \begin{macrocode}
\def\version{final}
\input{childdoc.def}
\childdocforwardprefix[cdocsamp]{cdocsfn}{cdocsch}
%    \end{macrocode}

%\iffalse
%</samplefinal>
%\fi
%
% %%%%%%%%%%%%%%%%%%%%%%%%%%%%%%%%%%%%%%
% \paragraph{Command Line Processing.}
%
% The following three command lines generate the output files
% |cdocscld|, |cdocscl1| and |cdocscl2|
% which should be identical to
% |cdocsdrf|, |cdocsch1| and |cdocsfn2|, respectively:
% \begin{center}
% \begin{tabular}{l}
% |latex -jobname cdocscld \|\\
% |  "\def\version{draft}\input{childdoc.def}\childdocforward{cdocsamp}"|\\
% |latex -jobname cdocscl1 \|\\
% |  "\input{childdoc.def}\childdocforward[cdocsamp]{cdocsch1}"|\\
% |latex -jobname cdocscl2 \|\\
% |  "\def\version{final}\input{childdoc.def}\childdocforward{cdocsch2}"|
% \end{tabular}
% \end{center}
% Note that the trailing backslash on each first line
% merely continues the input to the second line
% (for convenient cut ant paste).
% Furthermore, the command |latex| can be replaced by any
% of its alternative versions such as |pdflatex|.
%
% %%%%%%%%%%%%%%%%%%%%%%%%%%%%%%%%%%%%%%%%%%%%%%%%%%%%%%%%%%%%%%%%%%%%%%%%%%%%%%
% %%%%%%%%%%%%%%%%%%%%%%%%%%%%%%%%%%%%%%%%%%%%%%%%%%%%%%%%%%%%%%%%%%%%%%%%%%%%%%
% \section{Implementation}
%\iffalse
%<*package>
%\fi
%
% This section describes the definitions file |childdoc.def|.

% The definitions cannot be loaded using |\usepackage| or |\RequirePackage|
% which has a mechanism to prevent loading a style file more than once.
% When loading the definitions by means of |\input|
% multiple instances have to be prevented manually:
%\iffalse
%This code needs to be before the `\ProvidesFile' directive
%which is defined at the beginning of this file.
%Therefore it is also placed there and commented out here.
%</package>
%<*discard>
%\fi
%    \begin{macrocode}
\ifdefined\childdocmain\endinput\fi
%    \end{macrocode}
%\iffalse
%</discard>
%<*package>
%\fi
%
% \macro{\ifchilddoc}
% \macro{\ifchilddocmanual}
% The conditional |\ifchilddoc| tells whether a
% child (true) or main (false) document is being compiled.
% The conditional |\ifchilddocmanual| tells whether
% the |\includeonly| mechanism is used (false) or
% the selection of child files must be performed manually (true).
% The definitions initialise to false:
%    \begin{macrocode}
\newif\ifchilddoc
\newif\ifchilddocmanual
%    \end{macrocode}

% \macro{\childdocname}
% \macro{\childdocjob}
% The macro |\childdocname| stores the name of the main document
% to be compiled. The macro |\childdocjob| stores the name of
% the document on which the \LaTeX{} compiler was originally invoked.
% The content of |\jobname| cannot be compared
% to filenames specified in the source due to different catcodes.
% The following code rescans |\jobname|, stores the result
% in |\childdocname| and saves a copy in |\childdocjob|:
%    \begin{macrocode}
\edef\childdocname{\scantokens\expandafter{\jobname\noexpand}}
\let\childdocjob\childdocname
%    \end{macrocode}

% \macro{\childdocdisable}
% The macro |\childdocdisable| prevents the main file
% from being processed more than once.
% At this stage, the main document command |\childdocmain|
% is assumed to be called once again where it should do nothing.
% Any subsequent call to it should prevent
% a secondary processing of the main document
% It overwrites the forwarding commands
% |\childdocof| and |\childdocforward|
% with empty macros to prevent further inclusions of the main document:
%    \begin{macrocode}
\newcommand{\childdocdisable}
{
  \renewcommand{\childdocmain}[1]{\renewcommand{\childdocmain}[1]{\endinput}}
  \renewcommand{\childdocof}[1]{}
  \renewcommand{\childdocby}[2][]{}
  \renewcommand{\childdocforward}[2][]{}
  \renewcommand{\childdocdisable}{}
}
%    \end{macrocode}

% \macro{\childdocmain}
% The macro |\childdocmain| is to be called at the top of the main file
% with nothing or the main filename (without extension) as argument.
% First, it breaks loops.
% If the argument is not empty and does not match |\childdocname|
% (which is set by the first inclusion of |childdoc.def|),
% |\ifchilddoc| is set to true, |\includeonly| is applied to the child file
% and |\jobname| is set to the main file
% (for proper handling of |.aux| files):
%    \begin{macrocode}
\newcommand{\childdocmain}[1]
{
  \childdocdisable\childdocmain{}
  \if?#1?\else
    \begingroup
      \def\childdoctmp{#1}
      \ifx\childdoctmp\childdocname
        \def\childdoctmp{}
      \else
        \def\childdoctmp
        {
          \childdoctrue
          \includeonly{\childdocname}
          \def\childdocjob{#1}
          \def\jobname{#1}
        }
      \fi
      \expandafter
    \endgroup
    \childdoctmp
  \fi
}
%    \end{macrocode}

% \macro{\childdocof}
% The command |\childdocof| redirects
% compilation to the main file |#1|.
%    \begin{macrocode}
\newcommand{\childdocof}[1]
{
  \childdocdisable
  \childdoctrue
  \includeonly{\childdocname}
  \def\jobname{#1}
  \def\childdocjob{#1}
  \input{#1}
}
%    \end{macrocode}

% \macro{\childdocby}
% The command |\childdocby| ....
%    \begin{macrocode}
\newcommand{\childdocby}[2][]
{
  \childdocdisable
  \childdoctrue
  \childdocmanualtrue
  \if?#1?\else
    \def\jobname{#2}
  \fi
  \def\childdocjob{#2}
  \input{#2}
  \endinput
}
%    \end{macrocode}

% \macro{\childdocforward}
% The command |\childdocforward| redirects
% compilation to the main file or
% (if the optional argument is given) a child file.
% Parameters are set as if the main file
% or a child file starting with |\childdocof| was compiled.
% Then compilation is handed over to the main file:
%    \begin{macrocode}
\newcommand{\childdocforward}[2][]
{
  \begingroup
    \if?#1?
      \def\childdoctmp
      {
        \def\childdocname{#2}
        \def\childdocjob{#2}
        \def\jobname{#2}
        \input{#2}
        \endinput
      }
    \else
      \def\childdoctmp
      {
        \childdocdisable
        \def\childdocname{#2}
        \childdoctrue
        \includeonly{#2}
        \def\childdocjob{#1}
        \def\jobname{#1}
        \input{#1}
        \endinput
      }
    \fi
    \expandafter
  \endgroup
  \childdoctmp
}
%    \end{macrocode}

% \macro{\childdocforwardprefix}
% The command |\childdocforwardprefix| redirects
% compilation to the main or a child file by means of a pattern.
% The prefix |#1| in the current filename is replaced by |#2|
% and the suffix of the current filename is kept
% (it is assumed that the filename does not contain the substring `|~~~|'
% which is used as a delimiter).
% Compilation is handed over to the new file by |\childdocforward|:
%    \begin{macrocode}
\newcommand{\childdocforwardprefix}[3][]
{
  \begingroup
    \def\childdocextract #2##1~~~{\def\childdoctmp{\childdocforward[#1]{#3##1}}}
    \expandafter\childdocextract\childdocname~~~
    \expandafter
  \endgroup
  \childdoctmp
}
%    \end{macrocode}

% \macro{\childdoc}
% The deprecated macro |\childdoc| is a legacy version of |\childdocmain|:
%    \begin{macrocode}
\newcommand{\childdoc}{\childdocmain}
%    \end{macrocode}

% \macro{\childdocredirect}
% The deprecated macro |\childdocredirect| is a legacy version
% of |\childdocforward| and |\childdocforwardprefix|:
%    \begin{macrocode}
\newcommand{\childdocredirect}[2][]
{
  \begingroup
    \if?#1?
      \def\childdoctmp{\childdocforward{#2}}
    \else
      \def\childdoctmp{\childdocforwardprefix{#1}{#2}}
    \fi
    \expandafter
  \endgroup
  \childdoctmp
}
%    \end{macrocode}

%\iffalse
%</package>
%\fi
%
\endinput
|\\
|\childdocmain{|\textit{main}|}|\\
\end{tabular}
\end{center}
%
If |\jobname| does not match the argument \textit{main} of |\childdocmain|,
it is assumed that |\jobname| points to the child file to be compiled.
When using |\childdocmain| with the main file specified as argument,
it suffices to start a child file
with just |\input{|\textit{main}|}|
without loading of the package and using |\childdocof|.
If instead all processing is done
with the appropriate \textsf{childdoc} directives,
the argument of \textit{main} of |\childdocmain| can be empty.

An alternative version of the command line processing described
in \secref{sec:commandline} using the detection mechanism reads:
%
\begin{center}
|... -jobname "|\textit{target}|" "|[\textit{flags}]%
[|\def\jobname{|\textit{dest}|}|]|\input{|\textit{main}|}"|
\end{center}

%%%%%%%%%%%%%%%%%%%%%%%%%%%%%%%%%%%%%%%%%%%%%%%%%%%%%%%%%%%%%%%%%%%%%%%%%%%%%%%%
\subsection{Manual Code}
\label{sec:manual}

In case one cannot be certain whether the definitions file |childdoc.def|
is installed on the target \TeX{} distribution
and one prefers not to ship it,
it is conceivable to paste a few relevant commands into the sources.

To that end, drop all statements |% \iffalse
%
% childdoc.dtx Copyright (C) 2017-2018 Niklas Beisert
%
% This work may be distributed and/or modified under the
% conditions of the LaTeX Project Public License, either version 1.3
% of this license or (at your option) any later version.
% The latest version of this license is in
%   http://www.latex-project.org/lppl.txt
% and version 1.3 or later is part of all distributions of LaTeX
% version 2005/12/01 or later.
%
% This work has the LPPL maintenance status `maintained'.
%
% The Current Maintainer of this work is Niklas Beisert.
%
% This work consists of the files childdoc.dtx and childdoc.ins
% and the derived files childdoc.def and cdocsamp.tex with
% cdocsch1.tex, cdocsch2.tex, cdocsdrf.tex, cdocsfn1.tex, cdocsfn2.tex.
%
%<package>\ifdefined\childdocmain\endinput\fi
%<package>\ProvidesFile{childdoc.def}[2018/12/30 v2.0 child document driver]
%<samplemain>\ProvidesFile{cdocsamp.tex}[2018/12/30 v2.0 sample for childdoc]
%<*driver>
%\ProvidesFile{childdoc.drv}[2018/12/30 v2.0 childdoc reference manual file]
\PassOptionsToClass{10pt,a4paper}{article}
\documentclass{ltxdoc}

\usepackage[margin=35mm]{geometry}
\usepackage{hyperref}
\usepackage{hyperxmp}
\usepackage[usenames]{color}

\hypersetup{colorlinks=true}
\hypersetup{pdfstartview=FitH}
\hypersetup{pdfpagemode=UseNone}
\hypersetup{pdfsource={}}
\hypersetup{pdflang={en-UK}}
\hypersetup{pdfcopyright={Copyright 2017-2018 Niklas Beisert.
  This work may be distributed and/or modified under the
  conditions of the LaTeX Project Public License, either version 1.3
  of this license or (at your option) any later version.}}
\hypersetup{pdflicenseurl={http://www.latex-project.org/lppl.txt}}
\hypersetup{pdfcontactaddress={ETH Zurich, ITP, HIT K,
  Wolfgang-Pauli-Strasse 27}}
\hypersetup{pdfcontactpostcode={8093}}
\hypersetup{pdfcontactcity={Zurich}}
\hypersetup{pdfcontactcountry={Switzerland}}
\hypersetup{pdfcontactemail={nbeisert@itp.phys.ethz.ch}}
\hypersetup{pdfcontacturl={http://people.phys.ethz.ch/\xmptilde nbeisert/}}

\newcommand{\secref}[1]{\hyperref[#1]{section \ref*{#1}}}

\parskip1ex
\parindent0pt
\let\olditemize\itemize
\def\itemize{\olditemize\parskip0pt}

\begin{document}

\title{The \textsf{childdoc} Package}
\hypersetup{pdftitle={The childdoc Package}}
\author{Niklas Beisert\\[2ex]
  Institut f\"ur Theoretische Physik\\
  Eidgen\"ossische Technische Hochschule Z\"urich\\
  Wolfgang-Pauli-Strasse 27, 8093 Z\"urich, Switzerland\\[1ex]
  \href{mailto:nbeisert@itp.phys.ethz.ch}
  {\texttt{nbeisert@itp.phys.ethz.ch}}}
\hypersetup{pdfauthor={Niklas Beisert}}
\hypersetup{pdfsubject={Manual for the LaTeX2e Package childdoc}}
\date{30 December 2018, \textsf{v2.0}}
\maketitle

\begin{abstract}\noindent
\textsf{childdoc} is a \LaTeXe{} package
that enables the direct compilation
of document sections included by |\include|
to individual files.
\end{abstract}

\begingroup
\parskip0ex
\tableofcontents
\endgroup

%%%%%%%%%%%%%%%%%%%%%%%%%%%%%%%%%%%%%%%%%%%%%%%%%%%%%%%%%%%%%%%%%%%%%%%%%%%%%%%%
%%%%%%%%%%%%%%%%%%%%%%%%%%%%%%%%%%%%%%%%%%%%%%%%%%%%%%%%%%%%%%%%%%%%%%%%%%%%%%%%
\section{Introduction}

\LaTeX{} provides a mechanism to structure a large document (such as a book)
into a main file and several child files (containing the chapters)
using the |\include| command.
This mechanism is beneficial for documents
which span hundreds of pages in order to
make the source file(s) more manageable.
Moreover, compilation can be restricted to
selected child files by means of the |\includeonly| command.
The latter feature can be used to reduce the compilation time while editing
(this was significantly more useful in the earlier days of \LaTeX{})
or to generate a smaller document which is easier to navigate.
Another application of |\includeonly| is to generate
documents consisting of selected parts of the complete document.

However, there are a few drawbacks of the plain |\include| mechanism:
\begin{itemize}
\item
The child files cannot be compiled on their own,
they can only be compiled via the main file.
A naive editing environment
(such as a text editor with an option
to have the current file processed by \LaTeX)
may require one to switch to the main file before compiling;
attempting to compile the child file produces errors.
\item
The main file must be modified (each time)
to adjust the |\includeonly| command
to the present needs. This easily leaves the main file in a messy state.
\item
The generated document will always carry the filename
of the main document. This is inconvenient if
several child files are to be compiled and
to be kept for distribution.
\end{itemize}

The present package provides a simple interface
to make child files individually compilable by \LaTeX{}.
Compiling a child file then has the same effect as compiling
the main file with an |\includeonly| command
to select the appropriate child.
Moreover the generated document will carry the name of the child
rather than the main file.
This resolves all three above issues.

This feature is meant to make the editing of books,
thesis documents and lecture notes somewhat more convenient.
However, the package can also be used efficiently for
composing a series of documents (such as exercise sheets)
which are typically distributed individually.
It then assists the author in generating the individual documents
(potentially in different versions)
as well as a document containing the collected series.
Another application is in developing style files
or other kinds of included material
where compilation of the style file could redirect
to a sample or test file.

%%%%%%%%%%%%%%%%%%%%%%%%%%%%%%%%%%%%%%%%%%%%%%%%%%%%%%%%%%%%%%%%%%%%%%%%%%%%%%%%
%%%%%%%%%%%%%%%%%%%%%%%%%%%%%%%%%%%%%%%%%%%%%%%%%%%%%%%%%%%%%%%%%%%%%%%%%%%%%%%%
\section{Usage}

First of all, the package \textsf{childdoc} is \emph{not} a standard
\LaTeXe{} |.sty| style file! Therefore it needs to be invoked in
a non-standard way.

%%%%%%%%%%%%%%%%%%%%%%%%%%%%%%%%%%%%%%%%%%%%%%%%%%%%%%%%%%%%%%%%%%%%%%%%%%%%%%%%
\subsection{Included Files}
\label{sec:include}

%%%%%%%%%%%%%%%%%%%%%%%%%%%%%%%%%%%%%%%%
\DescribeMacro{\childdocmain}
To use the package, add the commands
\begin{center}
\begin{tabular}{l}
|\input{childdoc.def}|\\
|\childdocmain{}|\\
\end{tabular}
\end{center}
at the very top of the main \LaTeX{} file,
in particular \emph{before} the |\documentclass| statement!
The argument of |\childdocmain| should be left empty
(but it must be present).

%%%%%%%%%%%%%%%%%%%%%%%%%%%%%%%%%%%%%%%%
\DescribeMacro{\childdocof}
Furthermore, add the commands
\begin{center}
\begin{tabular}{l}
|\input{childdoc.def}|\\
|\childdocof{|\textit{main}|}|\\
\end{tabular}
\end{center}
at the top of every child file \textit{child}
which is included by |\include{|\textit{child}|}|
from within the main file
(or at least for those files to be compiled individually).
The argument \textit{main} must be the filename of the main file.

There are a couple of
considerations in setting up the main and child documents:

%%%%%%%%%%%%%%%%%%%%%%%%%%%%%%%%%%%%%%%%
\paragraph{Restrictions.}

Please note the following restrictions:
\begin{itemize}
\item
|\childdocmain| must be called with one argument \textit{main}
to ensure compatibility with earlier version of the package.
It must either be empty (|\childdocmain{}|)
or precisely match the filename of the main file in which it is specified.
See \secref{sec:detection} for further information.
\item
The filename \textit{main} must be specified without the |.tex| extension.
\item
The filename \textit{main} is case sensitive
(even in case-insensitive file systems)
due to internal string comparison.
\item
The argument \textit{main} should be fully expanded, it cannot be a macro.
\item
Subdirectories and special characters should be avoided in filenames.
\item
The command |\childdocmain{|\textit{main}|}| must be followed by a whitespace.
It should not be followed immediately by another command
or by a comment mark `|%|'.
This is because the \TeX{} parser reads the token immediately following
the argument of |\childdocmain| and puts it
at the beginning of every child section;
however, a white\-space is ignored.
\end{itemize}

%%%%%%%%%%%%%%%%%%%%%%%%%%%%%%%%%%%%%%%%
\paragraph{Content of Main File.}

It is advisable to place all content in the child files included by |\include|.
Any output contained in the main file will appear in all child documents
unless suppressed manually;
it cannot be suppressed automatically by the |\includeonly| directive
and thus should normally be avoided.
A method to include some content in the main file
by means of conditional processing is described in \secref{sec:conditional}.

%%%%%%%%%%%%%%%%%%%%%%%%%%%%%%%%%%%%%%%%
\paragraph{Page Numbering.}

When only a part of the document is compiled,
the appropriate numbering of pages
(as well as other status parameters)
is determined from the |.aux| files.
The latter contain information from previous passes.
However this information needs to propagate through
all intermediate child documents.
Therefore the page numbering in child documents may well
be inconsistent until the complete document is compiled at least once.

A useful (if unconventional) way to always ensure a consistent
page numbering is to restart the numbering in each child document
and denote the pages by `\textit{child}|.|\textit{page}'
where \textit{child} represents the chapter/section number of the child file.
This can be achieved by the command
|\numberwithin{page}{|\textit{child}|}|
of the \textsf{amsmath} package
where \textit{child} can be |chapter| or |section|
depending on the chosen structuring.
Alternatively, one can modify the macro |\thepage| appropriately
and reset the counter |page| at the start of each child file.

%%%%%%%%%%%%%%%%%%%%%%%%%%%%%%%%%%%%%%%%%%%%%%%%%%%%%%%%%%%%%%%%%%%%%%%%%%%%%%%%
\subsection{Conditional Processing}
\label{sec:conditional}

The package provides a mechanism to compile different versions
of a document. To customise the versions further some conditional processing
can come in handy to distinguish which version is being compiled.
The package provides two macros to describe the compilation context:

%%%%%%%%%%%%%%%%%%%%%%%%%%%%%%%%%%%%%%%%
\DescribeMacro{\ifchilddoc}
The conditional |\ifchilddoc| distinguishes between the compilation of
child documents and the main document:
%
\begin{center}
|\ifchilddoc |\textit{child-code}| |[|\||else |\textit{main-code}]| \||fi|
\end{center}

%%%%%%%%%%%%%%%%%%%%%%%%%%%%%%%%%%%%%%%%
\DescribeMacro{\childdocname}
\DescribeMacro{\childdocjob}
The macro |\childdocname| contains the filename (without extension)
of the main or child file being processed.
Note that |\childdocjob| will always contain the name of the main file.

%%%%%%%%%%%%%%%%%%%%%%%%%%%%%%%%%%%%%%%%
\paragraph{Title Page.}

Conditional processing can be used to include a title or banner page
in the main document when proper precautions are taken.
Importantly, the code in the main file should ensure that the page counter
(as well as other status parameters which are stored in the |.aux| files)
takes the same value after the conditional processing.
Otherwise the page numbers may take divergent values
depending on which part is compiled.

For example, a title page could be declared by:
%
\begin{center}
\begin{tabular}{l}
|\ifchilddoc\||else|\\
|\addtocounter{page}{-1}|\\
\textit{code for title page}\\
|\newpage|\\
|\||fi|
\end{tabular}
\end{center}
%
A banner page for the child documents can be generated by:
%
\begin{center}
\begin{tabular}{l}
|\ifchilddoc|\\
|\addtocounter{page}{-1}|\\
\textit{code for banner page}\\
|\newpage|\\
|\||fi|
\end{tabular}
\end{center}
%
Here one could write a message such as:
\begin{center}
|This is the part \childdocname{} of \childdocjob{}.|
\end{center}

%%%%%%%%%%%%%%%%%%%%%%%%%%%%%%%%%%%%%%%%%%%%%%%%%%%%%%%%%%%%%%%%%%%%%%%%%%%%%%%%
\subsection{Flags}
\label{sec:flags}

The package makes it easy to generate different versions
of the main or child documents.
To this end compilation flags can be defined
and assigned different default values.
They will be particularly useful in conjunction
with the forwarding mechanism described in \secref{sec:forward}.

For example, it may be useful to have a flag |\version|
which can be set to |draft| or |final|.
The document source will contain some conditional code
depending on the value of |\version|.
Suppose further, the flag should default to |final| for the main file
and to |draft| for child files
which is a natural assignment for editing the document.
This is achieved by placing the following code
in the preamble of the main document
(below the |\childdocmain| directive):
%
\begin{center}
\begin{tabular}{l}
|\ifchilddoc|\\
|\providecommand{\version}{draft}|\\
|\||else|\\
|\providecommand{\version}{final}|\\
|\||fi|
\end{tabular}
\end{center}
%
The definition by |\providecommand| makes sure
that previous definitions are not overwritten.
Further statements |\providecommand{\version}{...}|
can thus be added before the above code to override it.

For the main file, one might add a line
(between |\childdocmain| and the above block)
%
\begin{center}
|%\ifchilddoc\||else\providecommand{\version}{draft}\||fi|
\end{center}
%
which can be uncommented to produce a draft version.
Likewise one can add a line to the very top of a child file
(above the |\childdocof{|\textit{main}|}| directive)
%
\begin{center}
|%\providecommand{\version}{final}|
\end{center}
%
which can be uncommented to produce the final version of this child document.

%%%%%%%%%%%%%%%%%%%%%%%%%%%%%%%%%%%%%%%%%%%%%%%%%%%%%%%%%%%%%%%%%%%%%%%%%%%%%%%%
\subsection{Forwarding}
\label{sec:forward}

Different versions of the main or child documents
using compilation flags as described in \secref{sec:flags}
can be (permanently) stored in different files
for convenient compilation, viewing and distribution.
To this end, the package defines a command
to pass on compilation to a different file:

%%%%%%%%%%%%%%%%%%%%%%%%%%%%%%%%%%%%%%%%
\DescribeMacro{\childdocforward}
The command |\childdocforward| redirects processing to
another source file:
%
\begin{center}
\begin{tabular}{l}
|\input{childdoc.def}|\\
|\childdocforward[|\textit{main}|]{|\textit{dest}|}|\\
\end{tabular}
\end{center}
%
The argument \textit{dest} is the destination file
(without extension).
It should be the main file or one of the child files.
Note that further \textsf{childdoc} directives
such as |\childdocof| and |\childdocforward|
in the indicated file will be processed in this form.
The optional argument \textit{main}
passes on directly to the main file \textit{main}
while pretending to compile the child \textit{dest}.
This form behaves as if \textit{dest}
issues |\childdocof{|\textit{main}|}| right away,
and no further \textsf{childdoc} directives will be processed.

%%%%%%%%%%%%%%%%%%%%%%%%%%%%%%%%%%%%%%%%
\DescribeMacro{\...prefix}
In the alternative form |\childdocforwardprefix|,
%
\begin{center}
\begin{tabular}{l}
|\input{childdoc.def}|\\
|\childdocforwardprefix[|\textit{main}|]{|\textit{prefix}|}{|\textit{dest}|}|
\end{tabular}
\end{center}
%
the destination file is determined by a pattern
depending on the current file:
To make this work, the current file must be called
`{\textit{prefix}\hspace{0.2em}\textit{suffix}}'
with \textit{prefix} matching precisely the argument.
Processing is then passed on to the file
`{\textit{dest}\hspace{0.2em}\textit{suffix}}'.
Surely, the same effect is achieved by
directly specifying the
argument `{\textit{dest}\hspace{0.2em}\textit{suffix}}'
in the first form.
However, that requires to set up a different file
for each child. With the alternative form of the command
all these files can have exactly the same content
which simplifies setting them up and maintaining them.

For example, the following file |draft.tex|
with a compilation flag |\version| as described in \secref{sec:flags}
compiles the main document as a draft:
%
\begin{center}
\begin{tabular}{l}
|\def\version{draft}|\\
|\input{childdoc.def}|\\
|\childdocforward{|\textit{main}|}|
\end{tabular}
\end{center}
%
Likewise, the following files |final|\textit{nn}|.tex|
compile the final version of the child document
|child|\textit{nn}|.tex|:
%
\begin{center}
\begin{tabular}{l}
|\def\version{final}|\\
|\input{childdoc.def}|\\
|\childdocforwardprefix{final}{child}|
\end{tabular}
\end{center}
%

Note that when several versions of a main file and/or of each child file
are to be generated, it may be convenient to set up a |Makefile| or
shell script to automatise the process.

%%%%%%%%%%%%%%%%%%%%%%%%%%%%%%%%%%%%%%%%%%%%%%%%%%%%%%%%%%%%%%%%%%%%%%%%%%%%%%%%
\subsection{Command Line Processing}
\label{sec:commandline}

The effect of redirection files can also be achieved by invoking
the \LaTeX{} compiler with a more elaborate command line.
Most conveniently this should be done as part
of a shell script or a |Makefile|.

When using \textsf{childdoc} in the main file, the following
command lines effectively perform a redirection
(note that depending on the shell being used,
backslashes may have to be doubled: `|\|' $\to$ `|\\|'):
%
\begin{center}
|... -jobname "|\textit{target}|" |\\|"|[\textit{flags}]%
|\input{childdoc.def}\childdocforward[|\textit{main}|]{|\textit{dest}|}"|
\end{center}
%
Here \textit{target} is the name of the output file,
\textit{main} is the name of the main file
and \textit{dest} is the name of the main or child file to be processed
(all filenames without extensions).
The optional argument \textit{main} can be omitted
if \textit{main} matches \textit{dest}.
Optionally, compilation \textit{flags} can be defined via |\def| commands.
This command line makes the \TeX{} engine believe
it is compiling the file \textit{target}
whose content is specified as the latter parameter.
The provided code then forwards the processing to
\textit{main} or \textit{dest} as described in \secref{sec:forward}.

%%%%%%%%%%%%%%%%%%%%%%%%%%%%%%%%%%%%%%%%%%%%%%%%%%%%%%%%%%%%%%%%%%%%%%%%%%%%%%%%
\subsection{Include by Input}
\label{sec:input}

Including child documents by |\include| has some restrictions by design.
Most notably, the content of a child document always occupies
its own set of pages; pages cannot be shared between child documents.
Usually, this behaviour makes perfect sense
because each child document contain an essential part of the document.
However, in some situations it may be desirable to compose
a document from a collection of parts
without having mandatory page breaks between then.
For this case, the package
provides a mechanism to include parts
by |\input| which can also be processed individually.
However, by construction this mechanism
requires manual handling of the content to be output.

%%%%%%%%%%%%%%%%%%%%%%%%%%%%%%%%%%%%%%%%
\DescribeMacro{\ifchilddocmanual}
The main file should be prepared as usual, see \secref{sec:include}.
However, the document body must make a distinction
between processing of an individual part and of the main document, e.g.:
%
\begin{center}
\begin{tabular}{l}
|\ifchilddocmanual|\\
|\input{\childdocname}|\\
|\||else|\\
\textit{document body with }|\input{|\textit{part}|}|\\
|\||fi|
\end{tabular}
\end{center}
%
The conditional |\ifchilddocmanual| is true whenever
a part to be included by |\input| is being compiled,
and the name of the part is stored in |\childdocname|.

%%%%%%%%%%%%%%%%%%%%%%%%%%%%%%%%%%%%%%%%
\DescribeMacro{\childdocby}
Each part to be included by |\input| should start with:
%
\begin{center}
\begin{tabular}{l}
|\input{childdoc.def}|\\
|\childdocby{|\textit{main}|}|\\
\end{tabular}
\end{center}
%
The directive |\childdocby| is similar to |\childdocof|
described in \secref{sec:include},
but the subsequent selection of content must be done manually.
To that end, both |\ifchilddoc| and |\ifchilddocmanual|
will be true upon processing of a part,
and the name of the part is stored in |\childdocname|.
Note that |\jobname| will be set to the filename of the current part
so that each part receives an individual |.aux| file
that does not interfere with the |.aux| file(s) of the main document.
This behaviour can be altered by the alternative form
|\childdocby[*]{|\textit{main}|}| (with a non-empty optional argument)
which uses the |.aux| file of the main document
by setting |\jobname| to \textit{main}.

%%%%%%%%%%%%%%%%%%%%%%%%%%%%%%%%%%%%%%%%%%%%%%%%%%%%%%%%%%%%%%%%%%%%%%%%%%%%%%%%
\subsection{Driver Development}
\label{sec:driver}

The \textsf{childdoc} mechanism can also be use for the development
of definition files such as \LaTeX{} styles or classes.
This case differs from the above setup with multiple parts
included by |\include| in that no |\includeonly| should be invoked.
This can be achieved by starting the include file
(before |\ProvidesPackage|) with:
%
\begin{center}
\begin{tabular}{l}
|\input{childdoc.def}|\\
|\childdocforward{|\textit{main}|}|\\
\end{tabular}
\end{center}
%
or alternatively with:
%
\begin{center}
\begin{tabular}{l}
|\input{childdoc.def}|\\
|\childdocby{|\textit{main}|}|\\
\end{tabular}
\end{center}
%
Both forms have slightly different effects as described above.
The main file is prepared as usual, see \secref{sec:include}.

%%%%%%%%%%%%%%%%%%%%%%%%%%%%%%%%%%%%%%%%%%%%%%%%%%%%%%%%%%%%%%%%%%%%%%%%%%%%%%%%
\subsection{Legacy Detection}
\label{sec:detection}

The directive |\childdocmain| in the main file can detect
whether the complete document or merely a child is to be compiled
even without using the directive |\childdocof|.
This method is deprecated because it is less robust
and there is no compelling reason to use it;
it is merely provided for backward compatibility
and it may be removed in future versions.

If the detection mechanism is to be used,
it is mandatory to correctly specify
the filename of the main file as the argument of |\childdocmain|:
%
\begin{center}
\begin{tabular}{l}
|\input{childdoc.def}|\\
|\childdocmain{|\textit{main}|}|\\
\end{tabular}
\end{center}
%
If |\jobname| does not match the argument \textit{main} of |\childdocmain|,
it is assumed that |\jobname| points to the child file to be compiled.
When using |\childdocmain| with the main file specified as argument,
it suffices to start a child file
with just |\input{|\textit{main}|}|
without loading of the package and using |\childdocof|.
If instead all processing is done
with the appropriate \textsf{childdoc} directives,
the argument of \textit{main} of |\childdocmain| can be empty.

An alternative version of the command line processing described
in \secref{sec:commandline} using the detection mechanism reads:
%
\begin{center}
|... -jobname "|\textit{target}|" "|[\textit{flags}]%
[|\def\jobname{|\textit{dest}|}|]|\input{|\textit{main}|}"|
\end{center}

%%%%%%%%%%%%%%%%%%%%%%%%%%%%%%%%%%%%%%%%%%%%%%%%%%%%%%%%%%%%%%%%%%%%%%%%%%%%%%%%
\subsection{Manual Code}
\label{sec:manual}

In case one cannot be certain whether the definitions file |childdoc.def|
is installed on the target \TeX{} distribution
and one prefers not to ship it,
it is conceivable to paste a few relevant commands into the sources.

To that end, drop all statements |\input{childdoc.def}|
and perform the replacements as outlined below.
Instead of |\childdocmain{|\textit{main}|}| add the following code
to the top of the main file:
%
\begin{center}
\begin{tabular}{l}
|\||ifdefined\childdocname\endinput\||fi\newif\ifchilddoc|\\
|\edef\childdocname{\scantokens\expandafter{\jobname\noexpand}}|\\
|\def\childdocmain{|\textit{main}|}\||ifx\childdocmain\childdocname\||else|\\
|\childdoctrue\includeonly{\childdocname}\let\jobname\childdocmain\||fi|\\
\end{tabular}
\end{center}
%
Instead of |\childdocof{|\textit{main}|}| just include the main file
at the top of each child file:
%
\begin{center}
|\input{|\textit{main}|}|
\end{center}
%
A simple redirection |\childdocforward{|\textit{dest}|}| is achieved by:
%
\begin{center}
|\def\jobname{|\textit{dest}|}\input{\jobname}|
\end{center}
%
The redirection with prefix
|\childdocforwardprefix[|\textit{prefix}|]{|\textit{dest}|}|
is accomplished by:
%
\begin{center}
\begin{tabular}{l}
|{\edef\jobname{\scantokens\expandafter{\jobname\noexpand}}|\\
|\def\redirectjob |\textit{prefix}|#1~~~{\gdef\jobname{|\textit{dest}|#1}}|\\
|\expandafter\redirectjob\jobname~~~}\input{\jobname}|
\end{tabular}
\end{center}

In an alternative approach,
child documents can be compiled by a specific command line
without additional code or specific definitions:
%
\begin{center}
|... -jobname "|\textit{target}|" "|[\textit{flags}]%
|\includeonly{|\textit{dest}|}\input{|\textit{main}|}"|
\end{center}
%

%%%%%%%%%%%%%%%%%%%%%%%%%%%%%%%%%%%%%%%%%%%%%%%%%%%%%%%%%%%%%%%%%%%%%%%%%%%%%%%%
%%%%%%%%%%%%%%%%%%%%%%%%%%%%%%%%%%%%%%%%%%%%%%%%%%%%%%%%%%%%%%%%%%%%%%%%%%%%%%%%
\section{Information}

%%%%%%%%%%%%%%%%%%%%%%%%%%%%%%%%%%%%%%%%%%%%%%%%%%%%%%%%%%%%%%%%%%%%%%%%%%%%%%%%
\subsection{Copyright}

Copyright \copyright{} 2017--2018 Niklas Beisert

This work may be distributed and/or modified under the
conditions of the \LaTeX{} Project Public License, either version 1.3
of this license or (at your option) any later version.
The latest version of this license is in
  \url{http://www.latex-project.org/lppl.txt}
and version 1.3 or later is part of all distributions of \LaTeX{}
version 2005/12/01 or later.

This work has the LPPL maintenance status `maintained'.

The Current Maintainer of this work is Niklas Beisert.

This work consists of the files |README.txt|, |childdoc.ins| and |childdoc.dtx|
as well as the derived files |childdoc.def|, |cdocsamp.tex|
with |cdocsch1.tex|, |cdocsch2.tex|, |cdocspt3.tex|, |cdocspt4.tex|,
|cdocsdrf.tex|, |cdocsfn1.tex|, |cdocsfn2.tex|
as well as |childdoc.pdf|.

%%%%%%%%%%%%%%%%%%%%%%%%%%%%%%%%%%%%%%%%%%%%%%%%%%%%%%%%%%%%%%%%%%%%%%%%%%%%%%%%
\subsection{Files and Installation}

The package consists of the files:
%
\begin{center}
\begin{tabular}{ll}
    |README.txt|   & readme file \\
    |childdoc.ins| & installation file \\
    |childdoc.dtx| & source file \\
    |childdoc.def| & definition file \\
    |cdocsamp.tex| & sample main file \\
    |cdocsch1.tex| & sample include file \\
    |cdocsch2.tex| & sample include file \\
    |cdocspt3.tex| & sample part file \\
    |cdocspt4.tex| & sample part file \\
    |cdocsdrf.tex| & sample redirection file \\
    |cdocsfn1.tex| & sample redirection file \\
    |cdocsfn2.tex| & sample redirection file \\
    |childdoc.pdf| & manual
\end{tabular}
\end{center}
%
The distribution consists of the files
|README.txt|, |childdoc.ins| and |childdoc.dtx|.
%
\begin{itemize}
\item
Run (pdf)\LaTeX{} on |childdoc.dtx|
to compile the manual |childdoc.pdf| (this file).
\item
Run \LaTeX{} on |childdoc.ins| to create the definitions file |childdoc.def|
and the sample |cdocsamp.tex| with include files
|cdocsch1.tex|, |cdocsch2.tex|, |cdocspt3.tex|, |cdocspt4.tex|,
|cdocsdrf.tex|, |cdocsfn1.tex|, |cdocsfn2.tex|.
Then copy the file |childdoc.def| to an appropriate directory of your \LaTeX{}
distribution, e.g.\ \textit{texmf-root}|/tex/latex/childdoc|.
\end{itemize}

%%%%%%%%%%%%%%%%%%%%%%%%%%%%%%%%%%%%%%%%%%%%%%%%%%%%%%%%%%%%%%%%%%%%%%%%%%%%%%%%
\subsection{Related CTAN Packages}

There are several other packages which offer a similar functionality:
%
\begin{itemize}
\item
The packages
\href{http://ctan.org/pkg/docmute}{\textsf{docmute}},
\href{http://ctan.org/pkg/includex}{\textsf{includex}} and
\href{http://ctan.org/pkg/standalone}{\textsf{standalone}}
provide commands to include only the document body of
a child file thus allowing both files to be compiled individually.
\item
The packages \href{http://ctan.org/pkg/subdocs}{\textsf{subdocs}}
and \href{http://ctan.org/pkg/subfiles}{\textsf{subfiles}}
provide structures in which the main and child documents can be
encapsulated and allowing them to be compiled individually.
The inclusion mechanism is different from the conventional |\include|.
\item
The package \href{http://ctan.org/pkg/combine}{\textsf{combine}}
is an elaborate solution to combine several documents into one.
\end{itemize}
%
See also the CTAN topic \href{http://ctan.org/topic/subdocs}{\textsf{subdocs}}
for further related packages.
The present package differs from the above solutions in that
a document structure constructed with the conventional |\include| mechanism
just needs two extra commands at the top of every file
such that all constituent files can be compiled individually.

%%%%%%%%%%%%%%%%%%%%%%%%%%%%%%%%%%%%%%%%%%%%%%%%%%%%%%%%%%%%%%%%%%%%%%%%%%%%%%%%
%\subsection{Feature Suggestions}
%
%The following is a list of features which may be useful for future
%versions of this package:
%%
%\begin{itemize}
%\item
%\ldots
%\end{itemize}

%%%%%%%%%%%%%%%%%%%%%%%%%%%%%%%%%%%%%%%%%%%%%%%%%%%%%%%%%%%%%%%%%%%%%%%%%%%%%%%%
\subsection{Revision History}

%%%%%%%%%%%%%%%%%%%%%%%%%%%%%%%%%%%%%%%%
\paragraph{v2.0:} 2018/12/30

\begin{itemize}
\item
immediate forward processing
\item
added |\childdocby| mechanism
\item
manual restructured
\end{itemize}

%%%%%%%%%%%%%%%%%%%%%%%%%%%%%%%%%%%%%%%%
\paragraph{v1.6:} 2018/01/17

\begin{itemize}
\item
application for development of include files
\item
corrections to manual
\end{itemize}

%%%%%%%%%%%%%%%%%%%%%%%%%%%%%%%%%%%%%%%%
\paragraph{v1.5:} 2017/05/21

\begin{itemize}
\item
more complete structuring introduced
\item
|\childdocof| introduced
\item
|\childdoc| renamed to |\childdocmain|
\item
|\childredirect| renamed to |\childdocforward| and |\childdocforwardprefix|
and functionality expanded
\end{itemize}

%%%%%%%%%%%%%%%%%%%%%%%%%%%%%%%%%%%%%%%%
\paragraph{v1.0:} 2017/04/27

\begin{itemize}
\item
manual and install package
\item
first version published on CTAN
\end{itemize}

%%%%%%%%%%%%%%%%%%%%%%%%%%%%%%%%%%%%%%%%
\paragraph{v0.6:} 2017/04/26

\begin{itemize}
\item
redirection mechanism added
\end{itemize}

%%%%%%%%%%%%%%%%%%%%%%%%%%%%%%%%%%%%%%%%
\paragraph{v0.5:} 2017/04/26

\begin{itemize}
\item
functionality in definition file
\end{itemize}


%%%%%%%%%%%%%%%%%%%%%%%%%%%%%%%%%%%%%%%%%%%%%%%%%%%%%%%%%%%%%%%%%%%%%%%%%%%%%%%%
%%%%%%%%%%%%%%%%%%%%%%%%%%%%%%%%%%%%%%%%%%%%%%%%%%%%%%%%%%%%%%%%%%%%%%%%%%%%%%%%
%%%%%%%%%%%%%%%%%%%%%%%%%%%%%%%%%%%%%%%%%%%%%%%%%%%%%%%%%%%%%%%%%%%%%%%%%%%%%%%%
\appendix

\settowidth\MacroIndent{\rmfamily\scriptsize 000\ }

 \DocInput{childdoc.dtx}

\end{document}
%</driver>
% \fi
%
% %%%%%%%%%%%%%%%%%%%%%%%%%%%%%%%%%%%%%%%%%%%%%%%%%%%%%%%%%%%%%%%%%%%%%%%%%%%%%%
% %%%%%%%%%%%%%%%%%%%%%%%%%%%%%%%%%%%%%%%%%%%%%%%%%%%%%%%%%%%%%%%%%%%%%%%%%%%%%%
% \section{Sample}
%\iffalse
%<*samplemain>
%\fi
%
% The following presents a sample document
% with two chapters, two parts, a title page,
% a compile flag as well as three forwarding files to set the flag.
% It consists of eight |.tex| files:
% \begin{center}
% \begin{tabular}{ll}
% |cdocsamp.tex|&main file\\
% |cdocsch1.tex|&include file for chapter 1\\
% |cdocsch2.tex|&include file for chapter 2\\
% |cdocspt3.tex|&include file for part 3\\
% |cdocspt4.tex|&include file for part 4\\
% |cdocsdrf.tex|&forwarding file for main file in draft mode\\
% |cdocsfi1.tex|&forwarding file for final version of chapter 1\\
% |cdocsfi2.tex|&forwarding file for final version of chapter 2\\
% \end{tabular}
% \end{center}
% Each of the eight files can be compiled directly by the \LaTeX{} compiler.
%
% %%%%%%%%%%%%%%%%%%%%%%%%%%%%%%%%%%%%%%
% \paragraph{Main File.}
%
% The main file is called |cdocsamp.tex|.
%
% Load the \textsf{childdoc} definitions and
% declare the filename for the main document:
%    \begin{macrocode}
\input{childdoc.def}
\childdocmain{}
%    \end{macrocode}

% Optional override for |\version| flag:
%    \begin{macrocode}
%%\ifchilddoc\else\providecommand{\version}{draft}\fi
%    \end{macrocode}

% Define the default values for the |\version| flag
% (|final| for the main file and |draft| for childs):
%    \begin{macrocode}
\ifchilddoc
\providecommand{\version}{draft}
\else
\providecommand{\version}{final}
\fi
%    \end{macrocode}

% Load the standard document class:
%    \begin{macrocode}
\documentclass[12pt]{article}
%    \end{macrocode}

% Start the document body:
%    \begin{macrocode}
\begin{document}
%    \end{macrocode}

% Declare a title page.
% Print title, part of document being processed and version flag:
%    \begin{macrocode}
\addtocounter{page}{-1}
\begin{center}
{\LARGE\bfseries{}childdoc example\par}
\vspace{1cm}
\ifchilddoc
\ifchilddocmanual part\else chapter\fi:
`\childdocname' of `\childdocjob'\par
\else
main document: `\childdocjob'\par
\fi
version: \version\par
\end{center}
\newpage
%    \end{macrocode}

% Manually include selected file,
% otherwise process as usual:
%    \begin{macrocode}
\ifchilddocmanual
\section*{part `\childdocname'}
\input{\childdocname}
\else
%    \end{macrocode}

% Include the two chapters:
%    \begin{macrocode}
\include{cdocsch1}
\include{cdocsch2}
%    \end{macrocode}

% Include the two parts unless only chapters should be displayed:
%    \begin{macrocode}
\ifchilddoc\else
\section{part three}
\input{cdocspt3}
\section{part four}
\input{cdocspt4}
\fi
%    \end{macrocode}

% Process as usual until here:
%    \begin{macrocode}
\fi
%    \end{macrocode}

% End of document body:
%    \begin{macrocode}
\end{document}
%    \end{macrocode}
%\iffalse
%</samplemain>
%\fi
%
% %%%%%%%%%%%%%%%%%%%%%%%%%%%%%%%%%%%%%%
% \paragraph{Chapter Include Files.}
%
% The include files are called |cdocsch1.tex| and |cdocsch2.tex|.
%
%\iffalse
%<*samplechap1|samplechap2>
%\fi

% Optional override for |\version| flag:
%    \begin{macrocode}
%%\providecommand{\version}{final}
%    \end{macrocode}

% Include the main document:
%    \begin{macrocode}
\input{childdoc.def}
\childdocof{cdocsamp}
%    \end{macrocode}

%\iffalse
%</samplechap1|samplechap2>
%\fi
%
%\iffalse
%<*samplechap1>
%\fi
% Some text for chapter 1:
%    \begin{macrocode}
\section{one}
some text in chapter one
%    \end{macrocode}

%\iffalse
%</samplechap1>
%\fi
% Some text for chapter 2:
%\iffalse
%<*samplechap2>
%\fi
%    \begin{macrocode}
\section{two}
more text in chapter two
%    \end{macrocode}

%\iffalse
%</samplechap2>
%\fi
%
% %%%%%%%%%%%%%%%%%%%%%%%%%%%%%%%%%%%%%%
% \paragraph{Part Include Files.}
%
% The include files are called |cdocspt3.tex| and |cdocspt4.tex|.
%
%\iffalse
%<*samplepart3|samplepart4>
%\fi

% Optional override for |\version| flag:
%    \begin{macrocode}
%%\providecommand{\version}{final}
%    \end{macrocode}

% Include the main document:
%    \begin{macrocode}
\input{childdoc.def}
\childdocby{cdocsamp}
%    \end{macrocode}

%\iffalse
%</samplepart3|samplepart4>
%\fi
%
%\iffalse
%<*samplepart3>
%\fi
% Some text for part 3:
%    \begin{macrocode}
some text in part three
%    \end{macrocode}

%\iffalse
%</samplepart3>
%\fi
% Some text for part 4:
%\iffalse
%<*samplepart4>
%\fi
%    \begin{macrocode}
more text in part four
%    \end{macrocode}

%\iffalse
%</samplepart4>
%\fi
%
% %%%%%%%%%%%%%%%%%%%%%%%%%%%%%%%%%%%%%%
% \paragraph{Forwarding for a Complete Draft.}
%
% The following forwarding file |cdocsdrf.tex|
% compiles the main document in draft mode:
%\iffalse
%<*sampledraft>
%\fi
%    \begin{macrocode}
\def\version{draft}
\input{childdoc.def}
\childdocforward{cdocsamp}
%    \end{macrocode}

%\iffalse
%</sampledraft>
%\fi
%
% %%%%%%%%%%%%%%%%%%%%%%%%%%%%%%%%%%%%%%
% \paragraph{Forwarding for Final Version of the Chapters.}
%
% The following forwarding files |cdocsfn1.tex| and |cdocsfn2.tex|
% (with identical content)
% compile the final versions of the child documents
% |cdocsch1.tex| and |cdocsch2.tex|, respectively:
%\iffalse
%<*samplefinal>
%\fi
%    \begin{macrocode}
\def\version{final}
\input{childdoc.def}
\childdocforwardprefix[cdocsamp]{cdocsfn}{cdocsch}
%    \end{macrocode}

%\iffalse
%</samplefinal>
%\fi
%
% %%%%%%%%%%%%%%%%%%%%%%%%%%%%%%%%%%%%%%
% \paragraph{Command Line Processing.}
%
% The following three command lines generate the output files
% |cdocscld|, |cdocscl1| and |cdocscl2|
% which should be identical to
% |cdocsdrf|, |cdocsch1| and |cdocsfn2|, respectively:
% \begin{center}
% \begin{tabular}{l}
% |latex -jobname cdocscld \|\\
% |  "\def\version{draft}\input{childdoc.def}\childdocforward{cdocsamp}"|\\
% |latex -jobname cdocscl1 \|\\
% |  "\input{childdoc.def}\childdocforward[cdocsamp]{cdocsch1}"|\\
% |latex -jobname cdocscl2 \|\\
% |  "\def\version{final}\input{childdoc.def}\childdocforward{cdocsch2}"|
% \end{tabular}
% \end{center}
% Note that the trailing backslash on each first line
% merely continues the input to the second line
% (for convenient cut ant paste).
% Furthermore, the command |latex| can be replaced by any
% of its alternative versions such as |pdflatex|.
%
% %%%%%%%%%%%%%%%%%%%%%%%%%%%%%%%%%%%%%%%%%%%%%%%%%%%%%%%%%%%%%%%%%%%%%%%%%%%%%%
% %%%%%%%%%%%%%%%%%%%%%%%%%%%%%%%%%%%%%%%%%%%%%%%%%%%%%%%%%%%%%%%%%%%%%%%%%%%%%%
% \section{Implementation}
%\iffalse
%<*package>
%\fi
%
% This section describes the definitions file |childdoc.def|.

% The definitions cannot be loaded using |\usepackage| or |\RequirePackage|
% which has a mechanism to prevent loading a style file more than once.
% When loading the definitions by means of |\input|
% multiple instances have to be prevented manually:
%\iffalse
%This code needs to be before the `\ProvidesFile' directive
%which is defined at the beginning of this file.
%Therefore it is also placed there and commented out here.
%</package>
%<*discard>
%\fi
%    \begin{macrocode}
\ifdefined\childdocmain\endinput\fi
%    \end{macrocode}
%\iffalse
%</discard>
%<*package>
%\fi
%
% \macro{\ifchilddoc}
% \macro{\ifchilddocmanual}
% The conditional |\ifchilddoc| tells whether a
% child (true) or main (false) document is being compiled.
% The conditional |\ifchilddocmanual| tells whether
% the |\includeonly| mechanism is used (false) or
% the selection of child files must be performed manually (true).
% The definitions initialise to false:
%    \begin{macrocode}
\newif\ifchilddoc
\newif\ifchilddocmanual
%    \end{macrocode}

% \macro{\childdocname}
% \macro{\childdocjob}
% The macro |\childdocname| stores the name of the main document
% to be compiled. The macro |\childdocjob| stores the name of
% the document on which the \LaTeX{} compiler was originally invoked.
% The content of |\jobname| cannot be compared
% to filenames specified in the source due to different catcodes.
% The following code rescans |\jobname|, stores the result
% in |\childdocname| and saves a copy in |\childdocjob|:
%    \begin{macrocode}
\edef\childdocname{\scantokens\expandafter{\jobname\noexpand}}
\let\childdocjob\childdocname
%    \end{macrocode}

% \macro{\childdocdisable}
% The macro |\childdocdisable| prevents the main file
% from being processed more than once.
% At this stage, the main document command |\childdocmain|
% is assumed to be called once again where it should do nothing.
% Any subsequent call to it should prevent
% a secondary processing of the main document
% It overwrites the forwarding commands
% |\childdocof| and |\childdocforward|
% with empty macros to prevent further inclusions of the main document:
%    \begin{macrocode}
\newcommand{\childdocdisable}
{
  \renewcommand{\childdocmain}[1]{\renewcommand{\childdocmain}[1]{\endinput}}
  \renewcommand{\childdocof}[1]{}
  \renewcommand{\childdocby}[2][]{}
  \renewcommand{\childdocforward}[2][]{}
  \renewcommand{\childdocdisable}{}
}
%    \end{macrocode}

% \macro{\childdocmain}
% The macro |\childdocmain| is to be called at the top of the main file
% with nothing or the main filename (without extension) as argument.
% First, it breaks loops.
% If the argument is not empty and does not match |\childdocname|
% (which is set by the first inclusion of |childdoc.def|),
% |\ifchilddoc| is set to true, |\includeonly| is applied to the child file
% and |\jobname| is set to the main file
% (for proper handling of |.aux| files):
%    \begin{macrocode}
\newcommand{\childdocmain}[1]
{
  \childdocdisable\childdocmain{}
  \if?#1?\else
    \begingroup
      \def\childdoctmp{#1}
      \ifx\childdoctmp\childdocname
        \def\childdoctmp{}
      \else
        \def\childdoctmp
        {
          \childdoctrue
          \includeonly{\childdocname}
          \def\childdocjob{#1}
          \def\jobname{#1}
        }
      \fi
      \expandafter
    \endgroup
    \childdoctmp
  \fi
}
%    \end{macrocode}

% \macro{\childdocof}
% The command |\childdocof| redirects
% compilation to the main file |#1|.
%    \begin{macrocode}
\newcommand{\childdocof}[1]
{
  \childdocdisable
  \childdoctrue
  \includeonly{\childdocname}
  \def\jobname{#1}
  \def\childdocjob{#1}
  \input{#1}
}
%    \end{macrocode}

% \macro{\childdocby}
% The command |\childdocby| ....
%    \begin{macrocode}
\newcommand{\childdocby}[2][]
{
  \childdocdisable
  \childdoctrue
  \childdocmanualtrue
  \if?#1?\else
    \def\jobname{#2}
  \fi
  \def\childdocjob{#2}
  \input{#2}
  \endinput
}
%    \end{macrocode}

% \macro{\childdocforward}
% The command |\childdocforward| redirects
% compilation to the main file or
% (if the optional argument is given) a child file.
% Parameters are set as if the main file
% or a child file starting with |\childdocof| was compiled.
% Then compilation is handed over to the main file:
%    \begin{macrocode}
\newcommand{\childdocforward}[2][]
{
  \begingroup
    \if?#1?
      \def\childdoctmp
      {
        \def\childdocname{#2}
        \def\childdocjob{#2}
        \def\jobname{#2}
        \input{#2}
        \endinput
      }
    \else
      \def\childdoctmp
      {
        \childdocdisable
        \def\childdocname{#2}
        \childdoctrue
        \includeonly{#2}
        \def\childdocjob{#1}
        \def\jobname{#1}
        \input{#1}
        \endinput
      }
    \fi
    \expandafter
  \endgroup
  \childdoctmp
}
%    \end{macrocode}

% \macro{\childdocforwardprefix}
% The command |\childdocforwardprefix| redirects
% compilation to the main or a child file by means of a pattern.
% The prefix |#1| in the current filename is replaced by |#2|
% and the suffix of the current filename is kept
% (it is assumed that the filename does not contain the substring `|~~~|'
% which is used as a delimiter).
% Compilation is handed over to the new file by |\childdocforward|:
%    \begin{macrocode}
\newcommand{\childdocforwardprefix}[3][]
{
  \begingroup
    \def\childdocextract #2##1~~~{\def\childdoctmp{\childdocforward[#1]{#3##1}}}
    \expandafter\childdocextract\childdocname~~~
    \expandafter
  \endgroup
  \childdoctmp
}
%    \end{macrocode}

% \macro{\childdoc}
% The deprecated macro |\childdoc| is a legacy version of |\childdocmain|:
%    \begin{macrocode}
\newcommand{\childdoc}{\childdocmain}
%    \end{macrocode}

% \macro{\childdocredirect}
% The deprecated macro |\childdocredirect| is a legacy version
% of |\childdocforward| and |\childdocforwardprefix|:
%    \begin{macrocode}
\newcommand{\childdocredirect}[2][]
{
  \begingroup
    \if?#1?
      \def\childdoctmp{\childdocforward{#2}}
    \else
      \def\childdoctmp{\childdocforwardprefix{#1}{#2}}
    \fi
    \expandafter
  \endgroup
  \childdoctmp
}
%    \end{macrocode}

%\iffalse
%</package>
%\fi
%
\endinput
|
and perform the replacements as outlined below.
Instead of |\childdocmain{|\textit{main}|}| add the following code
to the top of the main file:
%
\begin{center}
\begin{tabular}{l}
|\||ifdefined\childdocname\endinput\||fi\newif\ifchilddoc|\\
|\edef\childdocname{\scantokens\expandafter{\jobname\noexpand}}|\\
|\def\childdocmain{|\textit{main}|}\||ifx\childdocmain\childdocname\||else|\\
|\childdoctrue\includeonly{\childdocname}\let\jobname\childdocmain\||fi|\\
\end{tabular}
\end{center}
%
Instead of |\childdocof{|\textit{main}|}| just include the main file
at the top of each child file:
%
\begin{center}
|\input{|\textit{main}|}|
\end{center}
%
A simple redirection |\childdocforward{|\textit{dest}|}| is achieved by:
%
\begin{center}
|\def\jobname{|\textit{dest}|}\input{\jobname}|
\end{center}
%
The redirection with prefix
|\childdocforwardprefix[|\textit{prefix}|]{|\textit{dest}|}|
is accomplished by:
%
\begin{center}
\begin{tabular}{l}
|{\edef\jobname{\scantokens\expandafter{\jobname\noexpand}}|\\
|\def\redirectjob |\textit{prefix}|#1~~~{\gdef\jobname{|\textit{dest}|#1}}|\\
|\expandafter\redirectjob\jobname~~~}\input{\jobname}|
\end{tabular}
\end{center}

In an alternative approach,
child documents can be compiled by a specific command line
without additional code or specific definitions:
%
\begin{center}
|... -jobname "|\textit{target}|" "|[\textit{flags}]%
|\includeonly{|\textit{dest}|}\input{|\textit{main}|}"|
\end{center}
%

%%%%%%%%%%%%%%%%%%%%%%%%%%%%%%%%%%%%%%%%%%%%%%%%%%%%%%%%%%%%%%%%%%%%%%%%%%%%%%%%
%%%%%%%%%%%%%%%%%%%%%%%%%%%%%%%%%%%%%%%%%%%%%%%%%%%%%%%%%%%%%%%%%%%%%%%%%%%%%%%%
\section{Information}

%%%%%%%%%%%%%%%%%%%%%%%%%%%%%%%%%%%%%%%%%%%%%%%%%%%%%%%%%%%%%%%%%%%%%%%%%%%%%%%%
\subsection{Copyright}

Copyright \copyright{} 2017--2018 Niklas Beisert

This work may be distributed and/or modified under the
conditions of the \LaTeX{} Project Public License, either version 1.3
of this license or (at your option) any later version.
The latest version of this license is in
  \url{http://www.latex-project.org/lppl.txt}
and version 1.3 or later is part of all distributions of \LaTeX{}
version 2005/12/01 or later.

This work has the LPPL maintenance status `maintained'.

The Current Maintainer of this work is Niklas Beisert.

This work consists of the files |README.txt|, |childdoc.ins| and |childdoc.dtx|
as well as the derived files |childdoc.def|, |cdocsamp.tex|
with |cdocsch1.tex|, |cdocsch2.tex|, |cdocspt3.tex|, |cdocspt4.tex|,
|cdocsdrf.tex|, |cdocsfn1.tex|, |cdocsfn2.tex|
as well as |childdoc.pdf|.

%%%%%%%%%%%%%%%%%%%%%%%%%%%%%%%%%%%%%%%%%%%%%%%%%%%%%%%%%%%%%%%%%%%%%%%%%%%%%%%%
\subsection{Files and Installation}

The package consists of the files:
%
\begin{center}
\begin{tabular}{ll}
    |README.txt|   & readme file \\
    |childdoc.ins| & installation file \\
    |childdoc.dtx| & source file \\
    |childdoc.def| & definition file \\
    |cdocsamp.tex| & sample main file \\
    |cdocsch1.tex| & sample include file \\
    |cdocsch2.tex| & sample include file \\
    |cdocspt3.tex| & sample part file \\
    |cdocspt4.tex| & sample part file \\
    |cdocsdrf.tex| & sample redirection file \\
    |cdocsfn1.tex| & sample redirection file \\
    |cdocsfn2.tex| & sample redirection file \\
    |childdoc.pdf| & manual
\end{tabular}
\end{center}
%
The distribution consists of the files
|README.txt|, |childdoc.ins| and |childdoc.dtx|.
%
\begin{itemize}
\item
Run (pdf)\LaTeX{} on |childdoc.dtx|
to compile the manual |childdoc.pdf| (this file).
\item
Run \LaTeX{} on |childdoc.ins| to create the definitions file |childdoc.def|
and the sample |cdocsamp.tex| with include files
|cdocsch1.tex|, |cdocsch2.tex|, |cdocspt3.tex|, |cdocspt4.tex|,
|cdocsdrf.tex|, |cdocsfn1.tex|, |cdocsfn2.tex|.
Then copy the file |childdoc.def| to an appropriate directory of your \LaTeX{}
distribution, e.g.\ \textit{texmf-root}|/tex/latex/childdoc|.
\end{itemize}

%%%%%%%%%%%%%%%%%%%%%%%%%%%%%%%%%%%%%%%%%%%%%%%%%%%%%%%%%%%%%%%%%%%%%%%%%%%%%%%%
\subsection{Related CTAN Packages}

There are several other packages which offer a similar functionality:
%
\begin{itemize}
\item
The packages
\href{http://ctan.org/pkg/docmute}{\textsf{docmute}},
\href{http://ctan.org/pkg/includex}{\textsf{includex}} and
\href{http://ctan.org/pkg/standalone}{\textsf{standalone}}
provide commands to include only the document body of
a child file thus allowing both files to be compiled individually.
\item
The packages \href{http://ctan.org/pkg/subdocs}{\textsf{subdocs}}
and \href{http://ctan.org/pkg/subfiles}{\textsf{subfiles}}
provide structures in which the main and child documents can be
encapsulated and allowing them to be compiled individually.
The inclusion mechanism is different from the conventional |\include|.
\item
The package \href{http://ctan.org/pkg/combine}{\textsf{combine}}
is an elaborate solution to combine several documents into one.
\end{itemize}
%
See also the CTAN topic \href{http://ctan.org/topic/subdocs}{\textsf{subdocs}}
for further related packages.
The present package differs from the above solutions in that
a document structure constructed with the conventional |\include| mechanism
just needs two extra commands at the top of every file
such that all constituent files can be compiled individually.

%%%%%%%%%%%%%%%%%%%%%%%%%%%%%%%%%%%%%%%%%%%%%%%%%%%%%%%%%%%%%%%%%%%%%%%%%%%%%%%%
%\subsection{Feature Suggestions}
%
%The following is a list of features which may be useful for future
%versions of this package:
%%
%\begin{itemize}
%\item
%\ldots
%\end{itemize}

%%%%%%%%%%%%%%%%%%%%%%%%%%%%%%%%%%%%%%%%%%%%%%%%%%%%%%%%%%%%%%%%%%%%%%%%%%%%%%%%
\subsection{Revision History}

%%%%%%%%%%%%%%%%%%%%%%%%%%%%%%%%%%%%%%%%
\paragraph{v2.0:} 2018/12/30

\begin{itemize}
\item
immediate forward processing
\item
added |\childdocby| mechanism
\item
manual restructured
\end{itemize}

%%%%%%%%%%%%%%%%%%%%%%%%%%%%%%%%%%%%%%%%
\paragraph{v1.6:} 2018/01/17

\begin{itemize}
\item
application for development of include files
\item
corrections to manual
\end{itemize}

%%%%%%%%%%%%%%%%%%%%%%%%%%%%%%%%%%%%%%%%
\paragraph{v1.5:} 2017/05/21

\begin{itemize}
\item
more complete structuring introduced
\item
|\childdocof| introduced
\item
|\childdoc| renamed to |\childdocmain|
\item
|\childredirect| renamed to |\childdocforward| and |\childdocforwardprefix|
and functionality expanded
\end{itemize}

%%%%%%%%%%%%%%%%%%%%%%%%%%%%%%%%%%%%%%%%
\paragraph{v1.0:} 2017/04/27

\begin{itemize}
\item
manual and install package
\item
first version published on CTAN
\end{itemize}

%%%%%%%%%%%%%%%%%%%%%%%%%%%%%%%%%%%%%%%%
\paragraph{v0.6:} 2017/04/26

\begin{itemize}
\item
redirection mechanism added
\end{itemize}

%%%%%%%%%%%%%%%%%%%%%%%%%%%%%%%%%%%%%%%%
\paragraph{v0.5:} 2017/04/26

\begin{itemize}
\item
functionality in definition file
\end{itemize}


%%%%%%%%%%%%%%%%%%%%%%%%%%%%%%%%%%%%%%%%%%%%%%%%%%%%%%%%%%%%%%%%%%%%%%%%%%%%%%%%
%%%%%%%%%%%%%%%%%%%%%%%%%%%%%%%%%%%%%%%%%%%%%%%%%%%%%%%%%%%%%%%%%%%%%%%%%%%%%%%%
%%%%%%%%%%%%%%%%%%%%%%%%%%%%%%%%%%%%%%%%%%%%%%%%%%%%%%%%%%%%%%%%%%%%%%%%%%%%%%%%
\appendix

\settowidth\MacroIndent{\rmfamily\scriptsize 000\ }

 \DocInput{childdoc.dtx}

\end{document}
%</driver>
% \fi
%
% %%%%%%%%%%%%%%%%%%%%%%%%%%%%%%%%%%%%%%%%%%%%%%%%%%%%%%%%%%%%%%%%%%%%%%%%%%%%%%
% %%%%%%%%%%%%%%%%%%%%%%%%%%%%%%%%%%%%%%%%%%%%%%%%%%%%%%%%%%%%%%%%%%%%%%%%%%%%%%
% \section{Sample}
%\iffalse
%<*samplemain>
%\fi
%
% The following presents a sample document
% with two chapters, two parts, a title page,
% a compile flag as well as three forwarding files to set the flag.
% It consists of eight |.tex| files:
% \begin{center}
% \begin{tabular}{ll}
% |cdocsamp.tex|&main file\\
% |cdocsch1.tex|&include file for chapter 1\\
% |cdocsch2.tex|&include file for chapter 2\\
% |cdocspt3.tex|&include file for part 3\\
% |cdocspt4.tex|&include file for part 4\\
% |cdocsdrf.tex|&forwarding file for main file in draft mode\\
% |cdocsfi1.tex|&forwarding file for final version of chapter 1\\
% |cdocsfi2.tex|&forwarding file for final version of chapter 2\\
% \end{tabular}
% \end{center}
% Each of the eight files can be compiled directly by the \LaTeX{} compiler.
%
% %%%%%%%%%%%%%%%%%%%%%%%%%%%%%%%%%%%%%%
% \paragraph{Main File.}
%
% The main file is called |cdocsamp.tex|.
%
% Load the \textsf{childdoc} definitions and
% declare the filename for the main document:
%    \begin{macrocode}
% \iffalse
%
% childdoc.dtx Copyright (C) 2017-2018 Niklas Beisert
%
% This work may be distributed and/or modified under the
% conditions of the LaTeX Project Public License, either version 1.3
% of this license or (at your option) any later version.
% The latest version of this license is in
%   http://www.latex-project.org/lppl.txt
% and version 1.3 or later is part of all distributions of LaTeX
% version 2005/12/01 or later.
%
% This work has the LPPL maintenance status `maintained'.
%
% The Current Maintainer of this work is Niklas Beisert.
%
% This work consists of the files childdoc.dtx and childdoc.ins
% and the derived files childdoc.def and cdocsamp.tex with
% cdocsch1.tex, cdocsch2.tex, cdocsdrf.tex, cdocsfn1.tex, cdocsfn2.tex.
%
%<package>\ifdefined\childdocmain\endinput\fi
%<package>\ProvidesFile{childdoc.def}[2018/12/30 v2.0 child document driver]
%<samplemain>\ProvidesFile{cdocsamp.tex}[2018/12/30 v2.0 sample for childdoc]
%<*driver>
%\ProvidesFile{childdoc.drv}[2018/12/30 v2.0 childdoc reference manual file]
\PassOptionsToClass{10pt,a4paper}{article}
\documentclass{ltxdoc}

\usepackage[margin=35mm]{geometry}
\usepackage{hyperref}
\usepackage{hyperxmp}
\usepackage[usenames]{color}

\hypersetup{colorlinks=true}
\hypersetup{pdfstartview=FitH}
\hypersetup{pdfpagemode=UseNone}
\hypersetup{pdfsource={}}
\hypersetup{pdflang={en-UK}}
\hypersetup{pdfcopyright={Copyright 2017-2018 Niklas Beisert.
  This work may be distributed and/or modified under the
  conditions of the LaTeX Project Public License, either version 1.3
  of this license or (at your option) any later version.}}
\hypersetup{pdflicenseurl={http://www.latex-project.org/lppl.txt}}
\hypersetup{pdfcontactaddress={ETH Zurich, ITP, HIT K,
  Wolfgang-Pauli-Strasse 27}}
\hypersetup{pdfcontactpostcode={8093}}
\hypersetup{pdfcontactcity={Zurich}}
\hypersetup{pdfcontactcountry={Switzerland}}
\hypersetup{pdfcontactemail={nbeisert@itp.phys.ethz.ch}}
\hypersetup{pdfcontacturl={http://people.phys.ethz.ch/\xmptilde nbeisert/}}

\newcommand{\secref}[1]{\hyperref[#1]{section \ref*{#1}}}

\parskip1ex
\parindent0pt
\let\olditemize\itemize
\def\itemize{\olditemize\parskip0pt}

\begin{document}

\title{The \textsf{childdoc} Package}
\hypersetup{pdftitle={The childdoc Package}}
\author{Niklas Beisert\\[2ex]
  Institut f\"ur Theoretische Physik\\
  Eidgen\"ossische Technische Hochschule Z\"urich\\
  Wolfgang-Pauli-Strasse 27, 8093 Z\"urich, Switzerland\\[1ex]
  \href{mailto:nbeisert@itp.phys.ethz.ch}
  {\texttt{nbeisert@itp.phys.ethz.ch}}}
\hypersetup{pdfauthor={Niklas Beisert}}
\hypersetup{pdfsubject={Manual for the LaTeX2e Package childdoc}}
\date{30 December 2018, \textsf{v2.0}}
\maketitle

\begin{abstract}\noindent
\textsf{childdoc} is a \LaTeXe{} package
that enables the direct compilation
of document sections included by |\include|
to individual files.
\end{abstract}

\begingroup
\parskip0ex
\tableofcontents
\endgroup

%%%%%%%%%%%%%%%%%%%%%%%%%%%%%%%%%%%%%%%%%%%%%%%%%%%%%%%%%%%%%%%%%%%%%%%%%%%%%%%%
%%%%%%%%%%%%%%%%%%%%%%%%%%%%%%%%%%%%%%%%%%%%%%%%%%%%%%%%%%%%%%%%%%%%%%%%%%%%%%%%
\section{Introduction}

\LaTeX{} provides a mechanism to structure a large document (such as a book)
into a main file and several child files (containing the chapters)
using the |\include| command.
This mechanism is beneficial for documents
which span hundreds of pages in order to
make the source file(s) more manageable.
Moreover, compilation can be restricted to
selected child files by means of the |\includeonly| command.
The latter feature can be used to reduce the compilation time while editing
(this was significantly more useful in the earlier days of \LaTeX{})
or to generate a smaller document which is easier to navigate.
Another application of |\includeonly| is to generate
documents consisting of selected parts of the complete document.

However, there are a few drawbacks of the plain |\include| mechanism:
\begin{itemize}
\item
The child files cannot be compiled on their own,
they can only be compiled via the main file.
A naive editing environment
(such as a text editor with an option
to have the current file processed by \LaTeX)
may require one to switch to the main file before compiling;
attempting to compile the child file produces errors.
\item
The main file must be modified (each time)
to adjust the |\includeonly| command
to the present needs. This easily leaves the main file in a messy state.
\item
The generated document will always carry the filename
of the main document. This is inconvenient if
several child files are to be compiled and
to be kept for distribution.
\end{itemize}

The present package provides a simple interface
to make child files individually compilable by \LaTeX{}.
Compiling a child file then has the same effect as compiling
the main file with an |\includeonly| command
to select the appropriate child.
Moreover the generated document will carry the name of the child
rather than the main file.
This resolves all three above issues.

This feature is meant to make the editing of books,
thesis documents and lecture notes somewhat more convenient.
However, the package can also be used efficiently for
composing a series of documents (such as exercise sheets)
which are typically distributed individually.
It then assists the author in generating the individual documents
(potentially in different versions)
as well as a document containing the collected series.
Another application is in developing style files
or other kinds of included material
where compilation of the style file could redirect
to a sample or test file.

%%%%%%%%%%%%%%%%%%%%%%%%%%%%%%%%%%%%%%%%%%%%%%%%%%%%%%%%%%%%%%%%%%%%%%%%%%%%%%%%
%%%%%%%%%%%%%%%%%%%%%%%%%%%%%%%%%%%%%%%%%%%%%%%%%%%%%%%%%%%%%%%%%%%%%%%%%%%%%%%%
\section{Usage}

First of all, the package \textsf{childdoc} is \emph{not} a standard
\LaTeXe{} |.sty| style file! Therefore it needs to be invoked in
a non-standard way.

%%%%%%%%%%%%%%%%%%%%%%%%%%%%%%%%%%%%%%%%%%%%%%%%%%%%%%%%%%%%%%%%%%%%%%%%%%%%%%%%
\subsection{Included Files}
\label{sec:include}

%%%%%%%%%%%%%%%%%%%%%%%%%%%%%%%%%%%%%%%%
\DescribeMacro{\childdocmain}
To use the package, add the commands
\begin{center}
\begin{tabular}{l}
|\input{childdoc.def}|\\
|\childdocmain{}|\\
\end{tabular}
\end{center}
at the very top of the main \LaTeX{} file,
in particular \emph{before} the |\documentclass| statement!
The argument of |\childdocmain| should be left empty
(but it must be present).

%%%%%%%%%%%%%%%%%%%%%%%%%%%%%%%%%%%%%%%%
\DescribeMacro{\childdocof}
Furthermore, add the commands
\begin{center}
\begin{tabular}{l}
|\input{childdoc.def}|\\
|\childdocof{|\textit{main}|}|\\
\end{tabular}
\end{center}
at the top of every child file \textit{child}
which is included by |\include{|\textit{child}|}|
from within the main file
(or at least for those files to be compiled individually).
The argument \textit{main} must be the filename of the main file.

There are a couple of
considerations in setting up the main and child documents:

%%%%%%%%%%%%%%%%%%%%%%%%%%%%%%%%%%%%%%%%
\paragraph{Restrictions.}

Please note the following restrictions:
\begin{itemize}
\item
|\childdocmain| must be called with one argument \textit{main}
to ensure compatibility with earlier version of the package.
It must either be empty (|\childdocmain{}|)
or precisely match the filename of the main file in which it is specified.
See \secref{sec:detection} for further information.
\item
The filename \textit{main} must be specified without the |.tex| extension.
\item
The filename \textit{main} is case sensitive
(even in case-insensitive file systems)
due to internal string comparison.
\item
The argument \textit{main} should be fully expanded, it cannot be a macro.
\item
Subdirectories and special characters should be avoided in filenames.
\item
The command |\childdocmain{|\textit{main}|}| must be followed by a whitespace.
It should not be followed immediately by another command
or by a comment mark `|%|'.
This is because the \TeX{} parser reads the token immediately following
the argument of |\childdocmain| and puts it
at the beginning of every child section;
however, a white\-space is ignored.
\end{itemize}

%%%%%%%%%%%%%%%%%%%%%%%%%%%%%%%%%%%%%%%%
\paragraph{Content of Main File.}

It is advisable to place all content in the child files included by |\include|.
Any output contained in the main file will appear in all child documents
unless suppressed manually;
it cannot be suppressed automatically by the |\includeonly| directive
and thus should normally be avoided.
A method to include some content in the main file
by means of conditional processing is described in \secref{sec:conditional}.

%%%%%%%%%%%%%%%%%%%%%%%%%%%%%%%%%%%%%%%%
\paragraph{Page Numbering.}

When only a part of the document is compiled,
the appropriate numbering of pages
(as well as other status parameters)
is determined from the |.aux| files.
The latter contain information from previous passes.
However this information needs to propagate through
all intermediate child documents.
Therefore the page numbering in child documents may well
be inconsistent until the complete document is compiled at least once.

A useful (if unconventional) way to always ensure a consistent
page numbering is to restart the numbering in each child document
and denote the pages by `\textit{child}|.|\textit{page}'
where \textit{child} represents the chapter/section number of the child file.
This can be achieved by the command
|\numberwithin{page}{|\textit{child}|}|
of the \textsf{amsmath} package
where \textit{child} can be |chapter| or |section|
depending on the chosen structuring.
Alternatively, one can modify the macro |\thepage| appropriately
and reset the counter |page| at the start of each child file.

%%%%%%%%%%%%%%%%%%%%%%%%%%%%%%%%%%%%%%%%%%%%%%%%%%%%%%%%%%%%%%%%%%%%%%%%%%%%%%%%
\subsection{Conditional Processing}
\label{sec:conditional}

The package provides a mechanism to compile different versions
of a document. To customise the versions further some conditional processing
can come in handy to distinguish which version is being compiled.
The package provides two macros to describe the compilation context:

%%%%%%%%%%%%%%%%%%%%%%%%%%%%%%%%%%%%%%%%
\DescribeMacro{\ifchilddoc}
The conditional |\ifchilddoc| distinguishes between the compilation of
child documents and the main document:
%
\begin{center}
|\ifchilddoc |\textit{child-code}| |[|\||else |\textit{main-code}]| \||fi|
\end{center}

%%%%%%%%%%%%%%%%%%%%%%%%%%%%%%%%%%%%%%%%
\DescribeMacro{\childdocname}
\DescribeMacro{\childdocjob}
The macro |\childdocname| contains the filename (without extension)
of the main or child file being processed.
Note that |\childdocjob| will always contain the name of the main file.

%%%%%%%%%%%%%%%%%%%%%%%%%%%%%%%%%%%%%%%%
\paragraph{Title Page.}

Conditional processing can be used to include a title or banner page
in the main document when proper precautions are taken.
Importantly, the code in the main file should ensure that the page counter
(as well as other status parameters which are stored in the |.aux| files)
takes the same value after the conditional processing.
Otherwise the page numbers may take divergent values
depending on which part is compiled.

For example, a title page could be declared by:
%
\begin{center}
\begin{tabular}{l}
|\ifchilddoc\||else|\\
|\addtocounter{page}{-1}|\\
\textit{code for title page}\\
|\newpage|\\
|\||fi|
\end{tabular}
\end{center}
%
A banner page for the child documents can be generated by:
%
\begin{center}
\begin{tabular}{l}
|\ifchilddoc|\\
|\addtocounter{page}{-1}|\\
\textit{code for banner page}\\
|\newpage|\\
|\||fi|
\end{tabular}
\end{center}
%
Here one could write a message such as:
\begin{center}
|This is the part \childdocname{} of \childdocjob{}.|
\end{center}

%%%%%%%%%%%%%%%%%%%%%%%%%%%%%%%%%%%%%%%%%%%%%%%%%%%%%%%%%%%%%%%%%%%%%%%%%%%%%%%%
\subsection{Flags}
\label{sec:flags}

The package makes it easy to generate different versions
of the main or child documents.
To this end compilation flags can be defined
and assigned different default values.
They will be particularly useful in conjunction
with the forwarding mechanism described in \secref{sec:forward}.

For example, it may be useful to have a flag |\version|
which can be set to |draft| or |final|.
The document source will contain some conditional code
depending on the value of |\version|.
Suppose further, the flag should default to |final| for the main file
and to |draft| for child files
which is a natural assignment for editing the document.
This is achieved by placing the following code
in the preamble of the main document
(below the |\childdocmain| directive):
%
\begin{center}
\begin{tabular}{l}
|\ifchilddoc|\\
|\providecommand{\version}{draft}|\\
|\||else|\\
|\providecommand{\version}{final}|\\
|\||fi|
\end{tabular}
\end{center}
%
The definition by |\providecommand| makes sure
that previous definitions are not overwritten.
Further statements |\providecommand{\version}{...}|
can thus be added before the above code to override it.

For the main file, one might add a line
(between |\childdocmain| and the above block)
%
\begin{center}
|%\ifchilddoc\||else\providecommand{\version}{draft}\||fi|
\end{center}
%
which can be uncommented to produce a draft version.
Likewise one can add a line to the very top of a child file
(above the |\childdocof{|\textit{main}|}| directive)
%
\begin{center}
|%\providecommand{\version}{final}|
\end{center}
%
which can be uncommented to produce the final version of this child document.

%%%%%%%%%%%%%%%%%%%%%%%%%%%%%%%%%%%%%%%%%%%%%%%%%%%%%%%%%%%%%%%%%%%%%%%%%%%%%%%%
\subsection{Forwarding}
\label{sec:forward}

Different versions of the main or child documents
using compilation flags as described in \secref{sec:flags}
can be (permanently) stored in different files
for convenient compilation, viewing and distribution.
To this end, the package defines a command
to pass on compilation to a different file:

%%%%%%%%%%%%%%%%%%%%%%%%%%%%%%%%%%%%%%%%
\DescribeMacro{\childdocforward}
The command |\childdocforward| redirects processing to
another source file:
%
\begin{center}
\begin{tabular}{l}
|\input{childdoc.def}|\\
|\childdocforward[|\textit{main}|]{|\textit{dest}|}|\\
\end{tabular}
\end{center}
%
The argument \textit{dest} is the destination file
(without extension).
It should be the main file or one of the child files.
Note that further \textsf{childdoc} directives
such as |\childdocof| and |\childdocforward|
in the indicated file will be processed in this form.
The optional argument \textit{main}
passes on directly to the main file \textit{main}
while pretending to compile the child \textit{dest}.
This form behaves as if \textit{dest}
issues |\childdocof{|\textit{main}|}| right away,
and no further \textsf{childdoc} directives will be processed.

%%%%%%%%%%%%%%%%%%%%%%%%%%%%%%%%%%%%%%%%
\DescribeMacro{\...prefix}
In the alternative form |\childdocforwardprefix|,
%
\begin{center}
\begin{tabular}{l}
|\input{childdoc.def}|\\
|\childdocforwardprefix[|\textit{main}|]{|\textit{prefix}|}{|\textit{dest}|}|
\end{tabular}
\end{center}
%
the destination file is determined by a pattern
depending on the current file:
To make this work, the current file must be called
`{\textit{prefix}\hspace{0.2em}\textit{suffix}}'
with \textit{prefix} matching precisely the argument.
Processing is then passed on to the file
`{\textit{dest}\hspace{0.2em}\textit{suffix}}'.
Surely, the same effect is achieved by
directly specifying the
argument `{\textit{dest}\hspace{0.2em}\textit{suffix}}'
in the first form.
However, that requires to set up a different file
for each child. With the alternative form of the command
all these files can have exactly the same content
which simplifies setting them up and maintaining them.

For example, the following file |draft.tex|
with a compilation flag |\version| as described in \secref{sec:flags}
compiles the main document as a draft:
%
\begin{center}
\begin{tabular}{l}
|\def\version{draft}|\\
|\input{childdoc.def}|\\
|\childdocforward{|\textit{main}|}|
\end{tabular}
\end{center}
%
Likewise, the following files |final|\textit{nn}|.tex|
compile the final version of the child document
|child|\textit{nn}|.tex|:
%
\begin{center}
\begin{tabular}{l}
|\def\version{final}|\\
|\input{childdoc.def}|\\
|\childdocforwardprefix{final}{child}|
\end{tabular}
\end{center}
%

Note that when several versions of a main file and/or of each child file
are to be generated, it may be convenient to set up a |Makefile| or
shell script to automatise the process.

%%%%%%%%%%%%%%%%%%%%%%%%%%%%%%%%%%%%%%%%%%%%%%%%%%%%%%%%%%%%%%%%%%%%%%%%%%%%%%%%
\subsection{Command Line Processing}
\label{sec:commandline}

The effect of redirection files can also be achieved by invoking
the \LaTeX{} compiler with a more elaborate command line.
Most conveniently this should be done as part
of a shell script or a |Makefile|.

When using \textsf{childdoc} in the main file, the following
command lines effectively perform a redirection
(note that depending on the shell being used,
backslashes may have to be doubled: `|\|' $\to$ `|\\|'):
%
\begin{center}
|... -jobname "|\textit{target}|" |\\|"|[\textit{flags}]%
|\input{childdoc.def}\childdocforward[|\textit{main}|]{|\textit{dest}|}"|
\end{center}
%
Here \textit{target} is the name of the output file,
\textit{main} is the name of the main file
and \textit{dest} is the name of the main or child file to be processed
(all filenames without extensions).
The optional argument \textit{main} can be omitted
if \textit{main} matches \textit{dest}.
Optionally, compilation \textit{flags} can be defined via |\def| commands.
This command line makes the \TeX{} engine believe
it is compiling the file \textit{target}
whose content is specified as the latter parameter.
The provided code then forwards the processing to
\textit{main} or \textit{dest} as described in \secref{sec:forward}.

%%%%%%%%%%%%%%%%%%%%%%%%%%%%%%%%%%%%%%%%%%%%%%%%%%%%%%%%%%%%%%%%%%%%%%%%%%%%%%%%
\subsection{Include by Input}
\label{sec:input}

Including child documents by |\include| has some restrictions by design.
Most notably, the content of a child document always occupies
its own set of pages; pages cannot be shared between child documents.
Usually, this behaviour makes perfect sense
because each child document contain an essential part of the document.
However, in some situations it may be desirable to compose
a document from a collection of parts
without having mandatory page breaks between then.
For this case, the package
provides a mechanism to include parts
by |\input| which can also be processed individually.
However, by construction this mechanism
requires manual handling of the content to be output.

%%%%%%%%%%%%%%%%%%%%%%%%%%%%%%%%%%%%%%%%
\DescribeMacro{\ifchilddocmanual}
The main file should be prepared as usual, see \secref{sec:include}.
However, the document body must make a distinction
between processing of an individual part and of the main document, e.g.:
%
\begin{center}
\begin{tabular}{l}
|\ifchilddocmanual|\\
|\input{\childdocname}|\\
|\||else|\\
\textit{document body with }|\input{|\textit{part}|}|\\
|\||fi|
\end{tabular}
\end{center}
%
The conditional |\ifchilddocmanual| is true whenever
a part to be included by |\input| is being compiled,
and the name of the part is stored in |\childdocname|.

%%%%%%%%%%%%%%%%%%%%%%%%%%%%%%%%%%%%%%%%
\DescribeMacro{\childdocby}
Each part to be included by |\input| should start with:
%
\begin{center}
\begin{tabular}{l}
|\input{childdoc.def}|\\
|\childdocby{|\textit{main}|}|\\
\end{tabular}
\end{center}
%
The directive |\childdocby| is similar to |\childdocof|
described in \secref{sec:include},
but the subsequent selection of content must be done manually.
To that end, both |\ifchilddoc| and |\ifchilddocmanual|
will be true upon processing of a part,
and the name of the part is stored in |\childdocname|.
Note that |\jobname| will be set to the filename of the current part
so that each part receives an individual |.aux| file
that does not interfere with the |.aux| file(s) of the main document.
This behaviour can be altered by the alternative form
|\childdocby[*]{|\textit{main}|}| (with a non-empty optional argument)
which uses the |.aux| file of the main document
by setting |\jobname| to \textit{main}.

%%%%%%%%%%%%%%%%%%%%%%%%%%%%%%%%%%%%%%%%%%%%%%%%%%%%%%%%%%%%%%%%%%%%%%%%%%%%%%%%
\subsection{Driver Development}
\label{sec:driver}

The \textsf{childdoc} mechanism can also be use for the development
of definition files such as \LaTeX{} styles or classes.
This case differs from the above setup with multiple parts
included by |\include| in that no |\includeonly| should be invoked.
This can be achieved by starting the include file
(before |\ProvidesPackage|) with:
%
\begin{center}
\begin{tabular}{l}
|\input{childdoc.def}|\\
|\childdocforward{|\textit{main}|}|\\
\end{tabular}
\end{center}
%
or alternatively with:
%
\begin{center}
\begin{tabular}{l}
|\input{childdoc.def}|\\
|\childdocby{|\textit{main}|}|\\
\end{tabular}
\end{center}
%
Both forms have slightly different effects as described above.
The main file is prepared as usual, see \secref{sec:include}.

%%%%%%%%%%%%%%%%%%%%%%%%%%%%%%%%%%%%%%%%%%%%%%%%%%%%%%%%%%%%%%%%%%%%%%%%%%%%%%%%
\subsection{Legacy Detection}
\label{sec:detection}

The directive |\childdocmain| in the main file can detect
whether the complete document or merely a child is to be compiled
even without using the directive |\childdocof|.
This method is deprecated because it is less robust
and there is no compelling reason to use it;
it is merely provided for backward compatibility
and it may be removed in future versions.

If the detection mechanism is to be used,
it is mandatory to correctly specify
the filename of the main file as the argument of |\childdocmain|:
%
\begin{center}
\begin{tabular}{l}
|\input{childdoc.def}|\\
|\childdocmain{|\textit{main}|}|\\
\end{tabular}
\end{center}
%
If |\jobname| does not match the argument \textit{main} of |\childdocmain|,
it is assumed that |\jobname| points to the child file to be compiled.
When using |\childdocmain| with the main file specified as argument,
it suffices to start a child file
with just |\input{|\textit{main}|}|
without loading of the package and using |\childdocof|.
If instead all processing is done
with the appropriate \textsf{childdoc} directives,
the argument of \textit{main} of |\childdocmain| can be empty.

An alternative version of the command line processing described
in \secref{sec:commandline} using the detection mechanism reads:
%
\begin{center}
|... -jobname "|\textit{target}|" "|[\textit{flags}]%
[|\def\jobname{|\textit{dest}|}|]|\input{|\textit{main}|}"|
\end{center}

%%%%%%%%%%%%%%%%%%%%%%%%%%%%%%%%%%%%%%%%%%%%%%%%%%%%%%%%%%%%%%%%%%%%%%%%%%%%%%%%
\subsection{Manual Code}
\label{sec:manual}

In case one cannot be certain whether the definitions file |childdoc.def|
is installed on the target \TeX{} distribution
and one prefers not to ship it,
it is conceivable to paste a few relevant commands into the sources.

To that end, drop all statements |\input{childdoc.def}|
and perform the replacements as outlined below.
Instead of |\childdocmain{|\textit{main}|}| add the following code
to the top of the main file:
%
\begin{center}
\begin{tabular}{l}
|\||ifdefined\childdocname\endinput\||fi\newif\ifchilddoc|\\
|\edef\childdocname{\scantokens\expandafter{\jobname\noexpand}}|\\
|\def\childdocmain{|\textit{main}|}\||ifx\childdocmain\childdocname\||else|\\
|\childdoctrue\includeonly{\childdocname}\let\jobname\childdocmain\||fi|\\
\end{tabular}
\end{center}
%
Instead of |\childdocof{|\textit{main}|}| just include the main file
at the top of each child file:
%
\begin{center}
|\input{|\textit{main}|}|
\end{center}
%
A simple redirection |\childdocforward{|\textit{dest}|}| is achieved by:
%
\begin{center}
|\def\jobname{|\textit{dest}|}\input{\jobname}|
\end{center}
%
The redirection with prefix
|\childdocforwardprefix[|\textit{prefix}|]{|\textit{dest}|}|
is accomplished by:
%
\begin{center}
\begin{tabular}{l}
|{\edef\jobname{\scantokens\expandafter{\jobname\noexpand}}|\\
|\def\redirectjob |\textit{prefix}|#1~~~{\gdef\jobname{|\textit{dest}|#1}}|\\
|\expandafter\redirectjob\jobname~~~}\input{\jobname}|
\end{tabular}
\end{center}

In an alternative approach,
child documents can be compiled by a specific command line
without additional code or specific definitions:
%
\begin{center}
|... -jobname "|\textit{target}|" "|[\textit{flags}]%
|\includeonly{|\textit{dest}|}\input{|\textit{main}|}"|
\end{center}
%

%%%%%%%%%%%%%%%%%%%%%%%%%%%%%%%%%%%%%%%%%%%%%%%%%%%%%%%%%%%%%%%%%%%%%%%%%%%%%%%%
%%%%%%%%%%%%%%%%%%%%%%%%%%%%%%%%%%%%%%%%%%%%%%%%%%%%%%%%%%%%%%%%%%%%%%%%%%%%%%%%
\section{Information}

%%%%%%%%%%%%%%%%%%%%%%%%%%%%%%%%%%%%%%%%%%%%%%%%%%%%%%%%%%%%%%%%%%%%%%%%%%%%%%%%
\subsection{Copyright}

Copyright \copyright{} 2017--2018 Niklas Beisert

This work may be distributed and/or modified under the
conditions of the \LaTeX{} Project Public License, either version 1.3
of this license or (at your option) any later version.
The latest version of this license is in
  \url{http://www.latex-project.org/lppl.txt}
and version 1.3 or later is part of all distributions of \LaTeX{}
version 2005/12/01 or later.

This work has the LPPL maintenance status `maintained'.

The Current Maintainer of this work is Niklas Beisert.

This work consists of the files |README.txt|, |childdoc.ins| and |childdoc.dtx|
as well as the derived files |childdoc.def|, |cdocsamp.tex|
with |cdocsch1.tex|, |cdocsch2.tex|, |cdocspt3.tex|, |cdocspt4.tex|,
|cdocsdrf.tex|, |cdocsfn1.tex|, |cdocsfn2.tex|
as well as |childdoc.pdf|.

%%%%%%%%%%%%%%%%%%%%%%%%%%%%%%%%%%%%%%%%%%%%%%%%%%%%%%%%%%%%%%%%%%%%%%%%%%%%%%%%
\subsection{Files and Installation}

The package consists of the files:
%
\begin{center}
\begin{tabular}{ll}
    |README.txt|   & readme file \\
    |childdoc.ins| & installation file \\
    |childdoc.dtx| & source file \\
    |childdoc.def| & definition file \\
    |cdocsamp.tex| & sample main file \\
    |cdocsch1.tex| & sample include file \\
    |cdocsch2.tex| & sample include file \\
    |cdocspt3.tex| & sample part file \\
    |cdocspt4.tex| & sample part file \\
    |cdocsdrf.tex| & sample redirection file \\
    |cdocsfn1.tex| & sample redirection file \\
    |cdocsfn2.tex| & sample redirection file \\
    |childdoc.pdf| & manual
\end{tabular}
\end{center}
%
The distribution consists of the files
|README.txt|, |childdoc.ins| and |childdoc.dtx|.
%
\begin{itemize}
\item
Run (pdf)\LaTeX{} on |childdoc.dtx|
to compile the manual |childdoc.pdf| (this file).
\item
Run \LaTeX{} on |childdoc.ins| to create the definitions file |childdoc.def|
and the sample |cdocsamp.tex| with include files
|cdocsch1.tex|, |cdocsch2.tex|, |cdocspt3.tex|, |cdocspt4.tex|,
|cdocsdrf.tex|, |cdocsfn1.tex|, |cdocsfn2.tex|.
Then copy the file |childdoc.def| to an appropriate directory of your \LaTeX{}
distribution, e.g.\ \textit{texmf-root}|/tex/latex/childdoc|.
\end{itemize}

%%%%%%%%%%%%%%%%%%%%%%%%%%%%%%%%%%%%%%%%%%%%%%%%%%%%%%%%%%%%%%%%%%%%%%%%%%%%%%%%
\subsection{Related CTAN Packages}

There are several other packages which offer a similar functionality:
%
\begin{itemize}
\item
The packages
\href{http://ctan.org/pkg/docmute}{\textsf{docmute}},
\href{http://ctan.org/pkg/includex}{\textsf{includex}} and
\href{http://ctan.org/pkg/standalone}{\textsf{standalone}}
provide commands to include only the document body of
a child file thus allowing both files to be compiled individually.
\item
The packages \href{http://ctan.org/pkg/subdocs}{\textsf{subdocs}}
and \href{http://ctan.org/pkg/subfiles}{\textsf{subfiles}}
provide structures in which the main and child documents can be
encapsulated and allowing them to be compiled individually.
The inclusion mechanism is different from the conventional |\include|.
\item
The package \href{http://ctan.org/pkg/combine}{\textsf{combine}}
is an elaborate solution to combine several documents into one.
\end{itemize}
%
See also the CTAN topic \href{http://ctan.org/topic/subdocs}{\textsf{subdocs}}
for further related packages.
The present package differs from the above solutions in that
a document structure constructed with the conventional |\include| mechanism
just needs two extra commands at the top of every file
such that all constituent files can be compiled individually.

%%%%%%%%%%%%%%%%%%%%%%%%%%%%%%%%%%%%%%%%%%%%%%%%%%%%%%%%%%%%%%%%%%%%%%%%%%%%%%%%
%\subsection{Feature Suggestions}
%
%The following is a list of features which may be useful for future
%versions of this package:
%%
%\begin{itemize}
%\item
%\ldots
%\end{itemize}

%%%%%%%%%%%%%%%%%%%%%%%%%%%%%%%%%%%%%%%%%%%%%%%%%%%%%%%%%%%%%%%%%%%%%%%%%%%%%%%%
\subsection{Revision History}

%%%%%%%%%%%%%%%%%%%%%%%%%%%%%%%%%%%%%%%%
\paragraph{v2.0:} 2018/12/30

\begin{itemize}
\item
immediate forward processing
\item
added |\childdocby| mechanism
\item
manual restructured
\end{itemize}

%%%%%%%%%%%%%%%%%%%%%%%%%%%%%%%%%%%%%%%%
\paragraph{v1.6:} 2018/01/17

\begin{itemize}
\item
application for development of include files
\item
corrections to manual
\end{itemize}

%%%%%%%%%%%%%%%%%%%%%%%%%%%%%%%%%%%%%%%%
\paragraph{v1.5:} 2017/05/21

\begin{itemize}
\item
more complete structuring introduced
\item
|\childdocof| introduced
\item
|\childdoc| renamed to |\childdocmain|
\item
|\childredirect| renamed to |\childdocforward| and |\childdocforwardprefix|
and functionality expanded
\end{itemize}

%%%%%%%%%%%%%%%%%%%%%%%%%%%%%%%%%%%%%%%%
\paragraph{v1.0:} 2017/04/27

\begin{itemize}
\item
manual and install package
\item
first version published on CTAN
\end{itemize}

%%%%%%%%%%%%%%%%%%%%%%%%%%%%%%%%%%%%%%%%
\paragraph{v0.6:} 2017/04/26

\begin{itemize}
\item
redirection mechanism added
\end{itemize}

%%%%%%%%%%%%%%%%%%%%%%%%%%%%%%%%%%%%%%%%
\paragraph{v0.5:} 2017/04/26

\begin{itemize}
\item
functionality in definition file
\end{itemize}


%%%%%%%%%%%%%%%%%%%%%%%%%%%%%%%%%%%%%%%%%%%%%%%%%%%%%%%%%%%%%%%%%%%%%%%%%%%%%%%%
%%%%%%%%%%%%%%%%%%%%%%%%%%%%%%%%%%%%%%%%%%%%%%%%%%%%%%%%%%%%%%%%%%%%%%%%%%%%%%%%
%%%%%%%%%%%%%%%%%%%%%%%%%%%%%%%%%%%%%%%%%%%%%%%%%%%%%%%%%%%%%%%%%%%%%%%%%%%%%%%%
\appendix

\settowidth\MacroIndent{\rmfamily\scriptsize 000\ }

 \DocInput{childdoc.dtx}

\end{document}
%</driver>
% \fi
%
% %%%%%%%%%%%%%%%%%%%%%%%%%%%%%%%%%%%%%%%%%%%%%%%%%%%%%%%%%%%%%%%%%%%%%%%%%%%%%%
% %%%%%%%%%%%%%%%%%%%%%%%%%%%%%%%%%%%%%%%%%%%%%%%%%%%%%%%%%%%%%%%%%%%%%%%%%%%%%%
% \section{Sample}
%\iffalse
%<*samplemain>
%\fi
%
% The following presents a sample document
% with two chapters, two parts, a title page,
% a compile flag as well as three forwarding files to set the flag.
% It consists of eight |.tex| files:
% \begin{center}
% \begin{tabular}{ll}
% |cdocsamp.tex|&main file\\
% |cdocsch1.tex|&include file for chapter 1\\
% |cdocsch2.tex|&include file for chapter 2\\
% |cdocspt3.tex|&include file for part 3\\
% |cdocspt4.tex|&include file for part 4\\
% |cdocsdrf.tex|&forwarding file for main file in draft mode\\
% |cdocsfi1.tex|&forwarding file for final version of chapter 1\\
% |cdocsfi2.tex|&forwarding file for final version of chapter 2\\
% \end{tabular}
% \end{center}
% Each of the eight files can be compiled directly by the \LaTeX{} compiler.
%
% %%%%%%%%%%%%%%%%%%%%%%%%%%%%%%%%%%%%%%
% \paragraph{Main File.}
%
% The main file is called |cdocsamp.tex|.
%
% Load the \textsf{childdoc} definitions and
% declare the filename for the main document:
%    \begin{macrocode}
\input{childdoc.def}
\childdocmain{}
%    \end{macrocode}

% Optional override for |\version| flag:
%    \begin{macrocode}
%%\ifchilddoc\else\providecommand{\version}{draft}\fi
%    \end{macrocode}

% Define the default values for the |\version| flag
% (|final| for the main file and |draft| for childs):
%    \begin{macrocode}
\ifchilddoc
\providecommand{\version}{draft}
\else
\providecommand{\version}{final}
\fi
%    \end{macrocode}

% Load the standard document class:
%    \begin{macrocode}
\documentclass[12pt]{article}
%    \end{macrocode}

% Start the document body:
%    \begin{macrocode}
\begin{document}
%    \end{macrocode}

% Declare a title page.
% Print title, part of document being processed and version flag:
%    \begin{macrocode}
\addtocounter{page}{-1}
\begin{center}
{\LARGE\bfseries{}childdoc example\par}
\vspace{1cm}
\ifchilddoc
\ifchilddocmanual part\else chapter\fi:
`\childdocname' of `\childdocjob'\par
\else
main document: `\childdocjob'\par
\fi
version: \version\par
\end{center}
\newpage
%    \end{macrocode}

% Manually include selected file,
% otherwise process as usual:
%    \begin{macrocode}
\ifchilddocmanual
\section*{part `\childdocname'}
\input{\childdocname}
\else
%    \end{macrocode}

% Include the two chapters:
%    \begin{macrocode}
\include{cdocsch1}
\include{cdocsch2}
%    \end{macrocode}

% Include the two parts unless only chapters should be displayed:
%    \begin{macrocode}
\ifchilddoc\else
\section{part three}
\input{cdocspt3}
\section{part four}
\input{cdocspt4}
\fi
%    \end{macrocode}

% Process as usual until here:
%    \begin{macrocode}
\fi
%    \end{macrocode}

% End of document body:
%    \begin{macrocode}
\end{document}
%    \end{macrocode}
%\iffalse
%</samplemain>
%\fi
%
% %%%%%%%%%%%%%%%%%%%%%%%%%%%%%%%%%%%%%%
% \paragraph{Chapter Include Files.}
%
% The include files are called |cdocsch1.tex| and |cdocsch2.tex|.
%
%\iffalse
%<*samplechap1|samplechap2>
%\fi

% Optional override for |\version| flag:
%    \begin{macrocode}
%%\providecommand{\version}{final}
%    \end{macrocode}

% Include the main document:
%    \begin{macrocode}
\input{childdoc.def}
\childdocof{cdocsamp}
%    \end{macrocode}

%\iffalse
%</samplechap1|samplechap2>
%\fi
%
%\iffalse
%<*samplechap1>
%\fi
% Some text for chapter 1:
%    \begin{macrocode}
\section{one}
some text in chapter one
%    \end{macrocode}

%\iffalse
%</samplechap1>
%\fi
% Some text for chapter 2:
%\iffalse
%<*samplechap2>
%\fi
%    \begin{macrocode}
\section{two}
more text in chapter two
%    \end{macrocode}

%\iffalse
%</samplechap2>
%\fi
%
% %%%%%%%%%%%%%%%%%%%%%%%%%%%%%%%%%%%%%%
% \paragraph{Part Include Files.}
%
% The include files are called |cdocspt3.tex| and |cdocspt4.tex|.
%
%\iffalse
%<*samplepart3|samplepart4>
%\fi

% Optional override for |\version| flag:
%    \begin{macrocode}
%%\providecommand{\version}{final}
%    \end{macrocode}

% Include the main document:
%    \begin{macrocode}
\input{childdoc.def}
\childdocby{cdocsamp}
%    \end{macrocode}

%\iffalse
%</samplepart3|samplepart4>
%\fi
%
%\iffalse
%<*samplepart3>
%\fi
% Some text for part 3:
%    \begin{macrocode}
some text in part three
%    \end{macrocode}

%\iffalse
%</samplepart3>
%\fi
% Some text for part 4:
%\iffalse
%<*samplepart4>
%\fi
%    \begin{macrocode}
more text in part four
%    \end{macrocode}

%\iffalse
%</samplepart4>
%\fi
%
% %%%%%%%%%%%%%%%%%%%%%%%%%%%%%%%%%%%%%%
% \paragraph{Forwarding for a Complete Draft.}
%
% The following forwarding file |cdocsdrf.tex|
% compiles the main document in draft mode:
%\iffalse
%<*sampledraft>
%\fi
%    \begin{macrocode}
\def\version{draft}
\input{childdoc.def}
\childdocforward{cdocsamp}
%    \end{macrocode}

%\iffalse
%</sampledraft>
%\fi
%
% %%%%%%%%%%%%%%%%%%%%%%%%%%%%%%%%%%%%%%
% \paragraph{Forwarding for Final Version of the Chapters.}
%
% The following forwarding files |cdocsfn1.tex| and |cdocsfn2.tex|
% (with identical content)
% compile the final versions of the child documents
% |cdocsch1.tex| and |cdocsch2.tex|, respectively:
%\iffalse
%<*samplefinal>
%\fi
%    \begin{macrocode}
\def\version{final}
\input{childdoc.def}
\childdocforwardprefix[cdocsamp]{cdocsfn}{cdocsch}
%    \end{macrocode}

%\iffalse
%</samplefinal>
%\fi
%
% %%%%%%%%%%%%%%%%%%%%%%%%%%%%%%%%%%%%%%
% \paragraph{Command Line Processing.}
%
% The following three command lines generate the output files
% |cdocscld|, |cdocscl1| and |cdocscl2|
% which should be identical to
% |cdocsdrf|, |cdocsch1| and |cdocsfn2|, respectively:
% \begin{center}
% \begin{tabular}{l}
% |latex -jobname cdocscld \|\\
% |  "\def\version{draft}\input{childdoc.def}\childdocforward{cdocsamp}"|\\
% |latex -jobname cdocscl1 \|\\
% |  "\input{childdoc.def}\childdocforward[cdocsamp]{cdocsch1}"|\\
% |latex -jobname cdocscl2 \|\\
% |  "\def\version{final}\input{childdoc.def}\childdocforward{cdocsch2}"|
% \end{tabular}
% \end{center}
% Note that the trailing backslash on each first line
% merely continues the input to the second line
% (for convenient cut ant paste).
% Furthermore, the command |latex| can be replaced by any
% of its alternative versions such as |pdflatex|.
%
% %%%%%%%%%%%%%%%%%%%%%%%%%%%%%%%%%%%%%%%%%%%%%%%%%%%%%%%%%%%%%%%%%%%%%%%%%%%%%%
% %%%%%%%%%%%%%%%%%%%%%%%%%%%%%%%%%%%%%%%%%%%%%%%%%%%%%%%%%%%%%%%%%%%%%%%%%%%%%%
% \section{Implementation}
%\iffalse
%<*package>
%\fi
%
% This section describes the definitions file |childdoc.def|.

% The definitions cannot be loaded using |\usepackage| or |\RequirePackage|
% which has a mechanism to prevent loading a style file more than once.
% When loading the definitions by means of |\input|
% multiple instances have to be prevented manually:
%\iffalse
%This code needs to be before the `\ProvidesFile' directive
%which is defined at the beginning of this file.
%Therefore it is also placed there and commented out here.
%</package>
%<*discard>
%\fi
%    \begin{macrocode}
\ifdefined\childdocmain\endinput\fi
%    \end{macrocode}
%\iffalse
%</discard>
%<*package>
%\fi
%
% \macro{\ifchilddoc}
% \macro{\ifchilddocmanual}
% The conditional |\ifchilddoc| tells whether a
% child (true) or main (false) document is being compiled.
% The conditional |\ifchilddocmanual| tells whether
% the |\includeonly| mechanism is used (false) or
% the selection of child files must be performed manually (true).
% The definitions initialise to false:
%    \begin{macrocode}
\newif\ifchilddoc
\newif\ifchilddocmanual
%    \end{macrocode}

% \macro{\childdocname}
% \macro{\childdocjob}
% The macro |\childdocname| stores the name of the main document
% to be compiled. The macro |\childdocjob| stores the name of
% the document on which the \LaTeX{} compiler was originally invoked.
% The content of |\jobname| cannot be compared
% to filenames specified in the source due to different catcodes.
% The following code rescans |\jobname|, stores the result
% in |\childdocname| and saves a copy in |\childdocjob|:
%    \begin{macrocode}
\edef\childdocname{\scantokens\expandafter{\jobname\noexpand}}
\let\childdocjob\childdocname
%    \end{macrocode}

% \macro{\childdocdisable}
% The macro |\childdocdisable| prevents the main file
% from being processed more than once.
% At this stage, the main document command |\childdocmain|
% is assumed to be called once again where it should do nothing.
% Any subsequent call to it should prevent
% a secondary processing of the main document
% It overwrites the forwarding commands
% |\childdocof| and |\childdocforward|
% with empty macros to prevent further inclusions of the main document:
%    \begin{macrocode}
\newcommand{\childdocdisable}
{
  \renewcommand{\childdocmain}[1]{\renewcommand{\childdocmain}[1]{\endinput}}
  \renewcommand{\childdocof}[1]{}
  \renewcommand{\childdocby}[2][]{}
  \renewcommand{\childdocforward}[2][]{}
  \renewcommand{\childdocdisable}{}
}
%    \end{macrocode}

% \macro{\childdocmain}
% The macro |\childdocmain| is to be called at the top of the main file
% with nothing or the main filename (without extension) as argument.
% First, it breaks loops.
% If the argument is not empty and does not match |\childdocname|
% (which is set by the first inclusion of |childdoc.def|),
% |\ifchilddoc| is set to true, |\includeonly| is applied to the child file
% and |\jobname| is set to the main file
% (for proper handling of |.aux| files):
%    \begin{macrocode}
\newcommand{\childdocmain}[1]
{
  \childdocdisable\childdocmain{}
  \if?#1?\else
    \begingroup
      \def\childdoctmp{#1}
      \ifx\childdoctmp\childdocname
        \def\childdoctmp{}
      \else
        \def\childdoctmp
        {
          \childdoctrue
          \includeonly{\childdocname}
          \def\childdocjob{#1}
          \def\jobname{#1}
        }
      \fi
      \expandafter
    \endgroup
    \childdoctmp
  \fi
}
%    \end{macrocode}

% \macro{\childdocof}
% The command |\childdocof| redirects
% compilation to the main file |#1|.
%    \begin{macrocode}
\newcommand{\childdocof}[1]
{
  \childdocdisable
  \childdoctrue
  \includeonly{\childdocname}
  \def\jobname{#1}
  \def\childdocjob{#1}
  \input{#1}
}
%    \end{macrocode}

% \macro{\childdocby}
% The command |\childdocby| ....
%    \begin{macrocode}
\newcommand{\childdocby}[2][]
{
  \childdocdisable
  \childdoctrue
  \childdocmanualtrue
  \if?#1?\else
    \def\jobname{#2}
  \fi
  \def\childdocjob{#2}
  \input{#2}
  \endinput
}
%    \end{macrocode}

% \macro{\childdocforward}
% The command |\childdocforward| redirects
% compilation to the main file or
% (if the optional argument is given) a child file.
% Parameters are set as if the main file
% or a child file starting with |\childdocof| was compiled.
% Then compilation is handed over to the main file:
%    \begin{macrocode}
\newcommand{\childdocforward}[2][]
{
  \begingroup
    \if?#1?
      \def\childdoctmp
      {
        \def\childdocname{#2}
        \def\childdocjob{#2}
        \def\jobname{#2}
        \input{#2}
        \endinput
      }
    \else
      \def\childdoctmp
      {
        \childdocdisable
        \def\childdocname{#2}
        \childdoctrue
        \includeonly{#2}
        \def\childdocjob{#1}
        \def\jobname{#1}
        \input{#1}
        \endinput
      }
    \fi
    \expandafter
  \endgroup
  \childdoctmp
}
%    \end{macrocode}

% \macro{\childdocforwardprefix}
% The command |\childdocforwardprefix| redirects
% compilation to the main or a child file by means of a pattern.
% The prefix |#1| in the current filename is replaced by |#2|
% and the suffix of the current filename is kept
% (it is assumed that the filename does not contain the substring `|~~~|'
% which is used as a delimiter).
% Compilation is handed over to the new file by |\childdocforward|:
%    \begin{macrocode}
\newcommand{\childdocforwardprefix}[3][]
{
  \begingroup
    \def\childdocextract #2##1~~~{\def\childdoctmp{\childdocforward[#1]{#3##1}}}
    \expandafter\childdocextract\childdocname~~~
    \expandafter
  \endgroup
  \childdoctmp
}
%    \end{macrocode}

% \macro{\childdoc}
% The deprecated macro |\childdoc| is a legacy version of |\childdocmain|:
%    \begin{macrocode}
\newcommand{\childdoc}{\childdocmain}
%    \end{macrocode}

% \macro{\childdocredirect}
% The deprecated macro |\childdocredirect| is a legacy version
% of |\childdocforward| and |\childdocforwardprefix|:
%    \begin{macrocode}
\newcommand{\childdocredirect}[2][]
{
  \begingroup
    \if?#1?
      \def\childdoctmp{\childdocforward{#2}}
    \else
      \def\childdoctmp{\childdocforwardprefix{#1}{#2}}
    \fi
    \expandafter
  \endgroup
  \childdoctmp
}
%    \end{macrocode}

%\iffalse
%</package>
%\fi
%
\endinput

\childdocmain{}
%    \end{macrocode}

% Optional override for |\version| flag:
%    \begin{macrocode}
%%\ifchilddoc\else\providecommand{\version}{draft}\fi
%    \end{macrocode}

% Define the default values for the |\version| flag
% (|final| for the main file and |draft| for childs):
%    \begin{macrocode}
\ifchilddoc
\providecommand{\version}{draft}
\else
\providecommand{\version}{final}
\fi
%    \end{macrocode}

% Load the standard document class:
%    \begin{macrocode}
\documentclass[12pt]{article}
%    \end{macrocode}

% Start the document body:
%    \begin{macrocode}
\begin{document}
%    \end{macrocode}

% Declare a title page.
% Print title, part of document being processed and version flag:
%    \begin{macrocode}
\addtocounter{page}{-1}
\begin{center}
{\LARGE\bfseries{}childdoc example\par}
\vspace{1cm}
\ifchilddoc
\ifchilddocmanual part\else chapter\fi:
`\childdocname' of `\childdocjob'\par
\else
main document: `\childdocjob'\par
\fi
version: \version\par
\end{center}
\newpage
%    \end{macrocode}

% Manually include selected file,
% otherwise process as usual:
%    \begin{macrocode}
\ifchilddocmanual
\section*{part `\childdocname'}
\input{\childdocname}
\else
%    \end{macrocode}

% Include the two chapters:
%    \begin{macrocode}
\include{cdocsch1}
\include{cdocsch2}
%    \end{macrocode}

% Include the two parts unless only chapters should be displayed:
%    \begin{macrocode}
\ifchilddoc\else
\section{part three}
\input{cdocspt3}
\section{part four}
\input{cdocspt4}
\fi
%    \end{macrocode}

% Process as usual until here:
%    \begin{macrocode}
\fi
%    \end{macrocode}

% End of document body:
%    \begin{macrocode}
\end{document}
%    \end{macrocode}
%\iffalse
%</samplemain>
%\fi
%
% %%%%%%%%%%%%%%%%%%%%%%%%%%%%%%%%%%%%%%
% \paragraph{Chapter Include Files.}
%
% The include files are called |cdocsch1.tex| and |cdocsch2.tex|.
%
%\iffalse
%<*samplechap1|samplechap2>
%\fi

% Optional override for |\version| flag:
%    \begin{macrocode}
%%\providecommand{\version}{final}
%    \end{macrocode}

% Include the main document:
%    \begin{macrocode}
% \iffalse
%
% childdoc.dtx Copyright (C) 2017-2018 Niklas Beisert
%
% This work may be distributed and/or modified under the
% conditions of the LaTeX Project Public License, either version 1.3
% of this license or (at your option) any later version.
% The latest version of this license is in
%   http://www.latex-project.org/lppl.txt
% and version 1.3 or later is part of all distributions of LaTeX
% version 2005/12/01 or later.
%
% This work has the LPPL maintenance status `maintained'.
%
% The Current Maintainer of this work is Niklas Beisert.
%
% This work consists of the files childdoc.dtx and childdoc.ins
% and the derived files childdoc.def and cdocsamp.tex with
% cdocsch1.tex, cdocsch2.tex, cdocsdrf.tex, cdocsfn1.tex, cdocsfn2.tex.
%
%<package>\ifdefined\childdocmain\endinput\fi
%<package>\ProvidesFile{childdoc.def}[2018/12/30 v2.0 child document driver]
%<samplemain>\ProvidesFile{cdocsamp.tex}[2018/12/30 v2.0 sample for childdoc]
%<*driver>
%\ProvidesFile{childdoc.drv}[2018/12/30 v2.0 childdoc reference manual file]
\PassOptionsToClass{10pt,a4paper}{article}
\documentclass{ltxdoc}

\usepackage[margin=35mm]{geometry}
\usepackage{hyperref}
\usepackage{hyperxmp}
\usepackage[usenames]{color}

\hypersetup{colorlinks=true}
\hypersetup{pdfstartview=FitH}
\hypersetup{pdfpagemode=UseNone}
\hypersetup{pdfsource={}}
\hypersetup{pdflang={en-UK}}
\hypersetup{pdfcopyright={Copyright 2017-2018 Niklas Beisert.
  This work may be distributed and/or modified under the
  conditions of the LaTeX Project Public License, either version 1.3
  of this license or (at your option) any later version.}}
\hypersetup{pdflicenseurl={http://www.latex-project.org/lppl.txt}}
\hypersetup{pdfcontactaddress={ETH Zurich, ITP, HIT K,
  Wolfgang-Pauli-Strasse 27}}
\hypersetup{pdfcontactpostcode={8093}}
\hypersetup{pdfcontactcity={Zurich}}
\hypersetup{pdfcontactcountry={Switzerland}}
\hypersetup{pdfcontactemail={nbeisert@itp.phys.ethz.ch}}
\hypersetup{pdfcontacturl={http://people.phys.ethz.ch/\xmptilde nbeisert/}}

\newcommand{\secref}[1]{\hyperref[#1]{section \ref*{#1}}}

\parskip1ex
\parindent0pt
\let\olditemize\itemize
\def\itemize{\olditemize\parskip0pt}

\begin{document}

\title{The \textsf{childdoc} Package}
\hypersetup{pdftitle={The childdoc Package}}
\author{Niklas Beisert\\[2ex]
  Institut f\"ur Theoretische Physik\\
  Eidgen\"ossische Technische Hochschule Z\"urich\\
  Wolfgang-Pauli-Strasse 27, 8093 Z\"urich, Switzerland\\[1ex]
  \href{mailto:nbeisert@itp.phys.ethz.ch}
  {\texttt{nbeisert@itp.phys.ethz.ch}}}
\hypersetup{pdfauthor={Niklas Beisert}}
\hypersetup{pdfsubject={Manual for the LaTeX2e Package childdoc}}
\date{30 December 2018, \textsf{v2.0}}
\maketitle

\begin{abstract}\noindent
\textsf{childdoc} is a \LaTeXe{} package
that enables the direct compilation
of document sections included by |\include|
to individual files.
\end{abstract}

\begingroup
\parskip0ex
\tableofcontents
\endgroup

%%%%%%%%%%%%%%%%%%%%%%%%%%%%%%%%%%%%%%%%%%%%%%%%%%%%%%%%%%%%%%%%%%%%%%%%%%%%%%%%
%%%%%%%%%%%%%%%%%%%%%%%%%%%%%%%%%%%%%%%%%%%%%%%%%%%%%%%%%%%%%%%%%%%%%%%%%%%%%%%%
\section{Introduction}

\LaTeX{} provides a mechanism to structure a large document (such as a book)
into a main file and several child files (containing the chapters)
using the |\include| command.
This mechanism is beneficial for documents
which span hundreds of pages in order to
make the source file(s) more manageable.
Moreover, compilation can be restricted to
selected child files by means of the |\includeonly| command.
The latter feature can be used to reduce the compilation time while editing
(this was significantly more useful in the earlier days of \LaTeX{})
or to generate a smaller document which is easier to navigate.
Another application of |\includeonly| is to generate
documents consisting of selected parts of the complete document.

However, there are a few drawbacks of the plain |\include| mechanism:
\begin{itemize}
\item
The child files cannot be compiled on their own,
they can only be compiled via the main file.
A naive editing environment
(such as a text editor with an option
to have the current file processed by \LaTeX)
may require one to switch to the main file before compiling;
attempting to compile the child file produces errors.
\item
The main file must be modified (each time)
to adjust the |\includeonly| command
to the present needs. This easily leaves the main file in a messy state.
\item
The generated document will always carry the filename
of the main document. This is inconvenient if
several child files are to be compiled and
to be kept for distribution.
\end{itemize}

The present package provides a simple interface
to make child files individually compilable by \LaTeX{}.
Compiling a child file then has the same effect as compiling
the main file with an |\includeonly| command
to select the appropriate child.
Moreover the generated document will carry the name of the child
rather than the main file.
This resolves all three above issues.

This feature is meant to make the editing of books,
thesis documents and lecture notes somewhat more convenient.
However, the package can also be used efficiently for
composing a series of documents (such as exercise sheets)
which are typically distributed individually.
It then assists the author in generating the individual documents
(potentially in different versions)
as well as a document containing the collected series.
Another application is in developing style files
or other kinds of included material
where compilation of the style file could redirect
to a sample or test file.

%%%%%%%%%%%%%%%%%%%%%%%%%%%%%%%%%%%%%%%%%%%%%%%%%%%%%%%%%%%%%%%%%%%%%%%%%%%%%%%%
%%%%%%%%%%%%%%%%%%%%%%%%%%%%%%%%%%%%%%%%%%%%%%%%%%%%%%%%%%%%%%%%%%%%%%%%%%%%%%%%
\section{Usage}

First of all, the package \textsf{childdoc} is \emph{not} a standard
\LaTeXe{} |.sty| style file! Therefore it needs to be invoked in
a non-standard way.

%%%%%%%%%%%%%%%%%%%%%%%%%%%%%%%%%%%%%%%%%%%%%%%%%%%%%%%%%%%%%%%%%%%%%%%%%%%%%%%%
\subsection{Included Files}
\label{sec:include}

%%%%%%%%%%%%%%%%%%%%%%%%%%%%%%%%%%%%%%%%
\DescribeMacro{\childdocmain}
To use the package, add the commands
\begin{center}
\begin{tabular}{l}
|\input{childdoc.def}|\\
|\childdocmain{}|\\
\end{tabular}
\end{center}
at the very top of the main \LaTeX{} file,
in particular \emph{before} the |\documentclass| statement!
The argument of |\childdocmain| should be left empty
(but it must be present).

%%%%%%%%%%%%%%%%%%%%%%%%%%%%%%%%%%%%%%%%
\DescribeMacro{\childdocof}
Furthermore, add the commands
\begin{center}
\begin{tabular}{l}
|\input{childdoc.def}|\\
|\childdocof{|\textit{main}|}|\\
\end{tabular}
\end{center}
at the top of every child file \textit{child}
which is included by |\include{|\textit{child}|}|
from within the main file
(or at least for those files to be compiled individually).
The argument \textit{main} must be the filename of the main file.

There are a couple of
considerations in setting up the main and child documents:

%%%%%%%%%%%%%%%%%%%%%%%%%%%%%%%%%%%%%%%%
\paragraph{Restrictions.}

Please note the following restrictions:
\begin{itemize}
\item
|\childdocmain| must be called with one argument \textit{main}
to ensure compatibility with earlier version of the package.
It must either be empty (|\childdocmain{}|)
or precisely match the filename of the main file in which it is specified.
See \secref{sec:detection} for further information.
\item
The filename \textit{main} must be specified without the |.tex| extension.
\item
The filename \textit{main} is case sensitive
(even in case-insensitive file systems)
due to internal string comparison.
\item
The argument \textit{main} should be fully expanded, it cannot be a macro.
\item
Subdirectories and special characters should be avoided in filenames.
\item
The command |\childdocmain{|\textit{main}|}| must be followed by a whitespace.
It should not be followed immediately by another command
or by a comment mark `|%|'.
This is because the \TeX{} parser reads the token immediately following
the argument of |\childdocmain| and puts it
at the beginning of every child section;
however, a white\-space is ignored.
\end{itemize}

%%%%%%%%%%%%%%%%%%%%%%%%%%%%%%%%%%%%%%%%
\paragraph{Content of Main File.}

It is advisable to place all content in the child files included by |\include|.
Any output contained in the main file will appear in all child documents
unless suppressed manually;
it cannot be suppressed automatically by the |\includeonly| directive
and thus should normally be avoided.
A method to include some content in the main file
by means of conditional processing is described in \secref{sec:conditional}.

%%%%%%%%%%%%%%%%%%%%%%%%%%%%%%%%%%%%%%%%
\paragraph{Page Numbering.}

When only a part of the document is compiled,
the appropriate numbering of pages
(as well as other status parameters)
is determined from the |.aux| files.
The latter contain information from previous passes.
However this information needs to propagate through
all intermediate child documents.
Therefore the page numbering in child documents may well
be inconsistent until the complete document is compiled at least once.

A useful (if unconventional) way to always ensure a consistent
page numbering is to restart the numbering in each child document
and denote the pages by `\textit{child}|.|\textit{page}'
where \textit{child} represents the chapter/section number of the child file.
This can be achieved by the command
|\numberwithin{page}{|\textit{child}|}|
of the \textsf{amsmath} package
where \textit{child} can be |chapter| or |section|
depending on the chosen structuring.
Alternatively, one can modify the macro |\thepage| appropriately
and reset the counter |page| at the start of each child file.

%%%%%%%%%%%%%%%%%%%%%%%%%%%%%%%%%%%%%%%%%%%%%%%%%%%%%%%%%%%%%%%%%%%%%%%%%%%%%%%%
\subsection{Conditional Processing}
\label{sec:conditional}

The package provides a mechanism to compile different versions
of a document. To customise the versions further some conditional processing
can come in handy to distinguish which version is being compiled.
The package provides two macros to describe the compilation context:

%%%%%%%%%%%%%%%%%%%%%%%%%%%%%%%%%%%%%%%%
\DescribeMacro{\ifchilddoc}
The conditional |\ifchilddoc| distinguishes between the compilation of
child documents and the main document:
%
\begin{center}
|\ifchilddoc |\textit{child-code}| |[|\||else |\textit{main-code}]| \||fi|
\end{center}

%%%%%%%%%%%%%%%%%%%%%%%%%%%%%%%%%%%%%%%%
\DescribeMacro{\childdocname}
\DescribeMacro{\childdocjob}
The macro |\childdocname| contains the filename (without extension)
of the main or child file being processed.
Note that |\childdocjob| will always contain the name of the main file.

%%%%%%%%%%%%%%%%%%%%%%%%%%%%%%%%%%%%%%%%
\paragraph{Title Page.}

Conditional processing can be used to include a title or banner page
in the main document when proper precautions are taken.
Importantly, the code in the main file should ensure that the page counter
(as well as other status parameters which are stored in the |.aux| files)
takes the same value after the conditional processing.
Otherwise the page numbers may take divergent values
depending on which part is compiled.

For example, a title page could be declared by:
%
\begin{center}
\begin{tabular}{l}
|\ifchilddoc\||else|\\
|\addtocounter{page}{-1}|\\
\textit{code for title page}\\
|\newpage|\\
|\||fi|
\end{tabular}
\end{center}
%
A banner page for the child documents can be generated by:
%
\begin{center}
\begin{tabular}{l}
|\ifchilddoc|\\
|\addtocounter{page}{-1}|\\
\textit{code for banner page}\\
|\newpage|\\
|\||fi|
\end{tabular}
\end{center}
%
Here one could write a message such as:
\begin{center}
|This is the part \childdocname{} of \childdocjob{}.|
\end{center}

%%%%%%%%%%%%%%%%%%%%%%%%%%%%%%%%%%%%%%%%%%%%%%%%%%%%%%%%%%%%%%%%%%%%%%%%%%%%%%%%
\subsection{Flags}
\label{sec:flags}

The package makes it easy to generate different versions
of the main or child documents.
To this end compilation flags can be defined
and assigned different default values.
They will be particularly useful in conjunction
with the forwarding mechanism described in \secref{sec:forward}.

For example, it may be useful to have a flag |\version|
which can be set to |draft| or |final|.
The document source will contain some conditional code
depending on the value of |\version|.
Suppose further, the flag should default to |final| for the main file
and to |draft| for child files
which is a natural assignment for editing the document.
This is achieved by placing the following code
in the preamble of the main document
(below the |\childdocmain| directive):
%
\begin{center}
\begin{tabular}{l}
|\ifchilddoc|\\
|\providecommand{\version}{draft}|\\
|\||else|\\
|\providecommand{\version}{final}|\\
|\||fi|
\end{tabular}
\end{center}
%
The definition by |\providecommand| makes sure
that previous definitions are not overwritten.
Further statements |\providecommand{\version}{...}|
can thus be added before the above code to override it.

For the main file, one might add a line
(between |\childdocmain| and the above block)
%
\begin{center}
|%\ifchilddoc\||else\providecommand{\version}{draft}\||fi|
\end{center}
%
which can be uncommented to produce a draft version.
Likewise one can add a line to the very top of a child file
(above the |\childdocof{|\textit{main}|}| directive)
%
\begin{center}
|%\providecommand{\version}{final}|
\end{center}
%
which can be uncommented to produce the final version of this child document.

%%%%%%%%%%%%%%%%%%%%%%%%%%%%%%%%%%%%%%%%%%%%%%%%%%%%%%%%%%%%%%%%%%%%%%%%%%%%%%%%
\subsection{Forwarding}
\label{sec:forward}

Different versions of the main or child documents
using compilation flags as described in \secref{sec:flags}
can be (permanently) stored in different files
for convenient compilation, viewing and distribution.
To this end, the package defines a command
to pass on compilation to a different file:

%%%%%%%%%%%%%%%%%%%%%%%%%%%%%%%%%%%%%%%%
\DescribeMacro{\childdocforward}
The command |\childdocforward| redirects processing to
another source file:
%
\begin{center}
\begin{tabular}{l}
|\input{childdoc.def}|\\
|\childdocforward[|\textit{main}|]{|\textit{dest}|}|\\
\end{tabular}
\end{center}
%
The argument \textit{dest} is the destination file
(without extension).
It should be the main file or one of the child files.
Note that further \textsf{childdoc} directives
such as |\childdocof| and |\childdocforward|
in the indicated file will be processed in this form.
The optional argument \textit{main}
passes on directly to the main file \textit{main}
while pretending to compile the child \textit{dest}.
This form behaves as if \textit{dest}
issues |\childdocof{|\textit{main}|}| right away,
and no further \textsf{childdoc} directives will be processed.

%%%%%%%%%%%%%%%%%%%%%%%%%%%%%%%%%%%%%%%%
\DescribeMacro{\...prefix}
In the alternative form |\childdocforwardprefix|,
%
\begin{center}
\begin{tabular}{l}
|\input{childdoc.def}|\\
|\childdocforwardprefix[|\textit{main}|]{|\textit{prefix}|}{|\textit{dest}|}|
\end{tabular}
\end{center}
%
the destination file is determined by a pattern
depending on the current file:
To make this work, the current file must be called
`{\textit{prefix}\hspace{0.2em}\textit{suffix}}'
with \textit{prefix} matching precisely the argument.
Processing is then passed on to the file
`{\textit{dest}\hspace{0.2em}\textit{suffix}}'.
Surely, the same effect is achieved by
directly specifying the
argument `{\textit{dest}\hspace{0.2em}\textit{suffix}}'
in the first form.
However, that requires to set up a different file
for each child. With the alternative form of the command
all these files can have exactly the same content
which simplifies setting them up and maintaining them.

For example, the following file |draft.tex|
with a compilation flag |\version| as described in \secref{sec:flags}
compiles the main document as a draft:
%
\begin{center}
\begin{tabular}{l}
|\def\version{draft}|\\
|\input{childdoc.def}|\\
|\childdocforward{|\textit{main}|}|
\end{tabular}
\end{center}
%
Likewise, the following files |final|\textit{nn}|.tex|
compile the final version of the child document
|child|\textit{nn}|.tex|:
%
\begin{center}
\begin{tabular}{l}
|\def\version{final}|\\
|\input{childdoc.def}|\\
|\childdocforwardprefix{final}{child}|
\end{tabular}
\end{center}
%

Note that when several versions of a main file and/or of each child file
are to be generated, it may be convenient to set up a |Makefile| or
shell script to automatise the process.

%%%%%%%%%%%%%%%%%%%%%%%%%%%%%%%%%%%%%%%%%%%%%%%%%%%%%%%%%%%%%%%%%%%%%%%%%%%%%%%%
\subsection{Command Line Processing}
\label{sec:commandline}

The effect of redirection files can also be achieved by invoking
the \LaTeX{} compiler with a more elaborate command line.
Most conveniently this should be done as part
of a shell script or a |Makefile|.

When using \textsf{childdoc} in the main file, the following
command lines effectively perform a redirection
(note that depending on the shell being used,
backslashes may have to be doubled: `|\|' $\to$ `|\\|'):
%
\begin{center}
|... -jobname "|\textit{target}|" |\\|"|[\textit{flags}]%
|\input{childdoc.def}\childdocforward[|\textit{main}|]{|\textit{dest}|}"|
\end{center}
%
Here \textit{target} is the name of the output file,
\textit{main} is the name of the main file
and \textit{dest} is the name of the main or child file to be processed
(all filenames without extensions).
The optional argument \textit{main} can be omitted
if \textit{main} matches \textit{dest}.
Optionally, compilation \textit{flags} can be defined via |\def| commands.
This command line makes the \TeX{} engine believe
it is compiling the file \textit{target}
whose content is specified as the latter parameter.
The provided code then forwards the processing to
\textit{main} or \textit{dest} as described in \secref{sec:forward}.

%%%%%%%%%%%%%%%%%%%%%%%%%%%%%%%%%%%%%%%%%%%%%%%%%%%%%%%%%%%%%%%%%%%%%%%%%%%%%%%%
\subsection{Include by Input}
\label{sec:input}

Including child documents by |\include| has some restrictions by design.
Most notably, the content of a child document always occupies
its own set of pages; pages cannot be shared between child documents.
Usually, this behaviour makes perfect sense
because each child document contain an essential part of the document.
However, in some situations it may be desirable to compose
a document from a collection of parts
without having mandatory page breaks between then.
For this case, the package
provides a mechanism to include parts
by |\input| which can also be processed individually.
However, by construction this mechanism
requires manual handling of the content to be output.

%%%%%%%%%%%%%%%%%%%%%%%%%%%%%%%%%%%%%%%%
\DescribeMacro{\ifchilddocmanual}
The main file should be prepared as usual, see \secref{sec:include}.
However, the document body must make a distinction
between processing of an individual part and of the main document, e.g.:
%
\begin{center}
\begin{tabular}{l}
|\ifchilddocmanual|\\
|\input{\childdocname}|\\
|\||else|\\
\textit{document body with }|\input{|\textit{part}|}|\\
|\||fi|
\end{tabular}
\end{center}
%
The conditional |\ifchilddocmanual| is true whenever
a part to be included by |\input| is being compiled,
and the name of the part is stored in |\childdocname|.

%%%%%%%%%%%%%%%%%%%%%%%%%%%%%%%%%%%%%%%%
\DescribeMacro{\childdocby}
Each part to be included by |\input| should start with:
%
\begin{center}
\begin{tabular}{l}
|\input{childdoc.def}|\\
|\childdocby{|\textit{main}|}|\\
\end{tabular}
\end{center}
%
The directive |\childdocby| is similar to |\childdocof|
described in \secref{sec:include},
but the subsequent selection of content must be done manually.
To that end, both |\ifchilddoc| and |\ifchilddocmanual|
will be true upon processing of a part,
and the name of the part is stored in |\childdocname|.
Note that |\jobname| will be set to the filename of the current part
so that each part receives an individual |.aux| file
that does not interfere with the |.aux| file(s) of the main document.
This behaviour can be altered by the alternative form
|\childdocby[*]{|\textit{main}|}| (with a non-empty optional argument)
which uses the |.aux| file of the main document
by setting |\jobname| to \textit{main}.

%%%%%%%%%%%%%%%%%%%%%%%%%%%%%%%%%%%%%%%%%%%%%%%%%%%%%%%%%%%%%%%%%%%%%%%%%%%%%%%%
\subsection{Driver Development}
\label{sec:driver}

The \textsf{childdoc} mechanism can also be use for the development
of definition files such as \LaTeX{} styles or classes.
This case differs from the above setup with multiple parts
included by |\include| in that no |\includeonly| should be invoked.
This can be achieved by starting the include file
(before |\ProvidesPackage|) with:
%
\begin{center}
\begin{tabular}{l}
|\input{childdoc.def}|\\
|\childdocforward{|\textit{main}|}|\\
\end{tabular}
\end{center}
%
or alternatively with:
%
\begin{center}
\begin{tabular}{l}
|\input{childdoc.def}|\\
|\childdocby{|\textit{main}|}|\\
\end{tabular}
\end{center}
%
Both forms have slightly different effects as described above.
The main file is prepared as usual, see \secref{sec:include}.

%%%%%%%%%%%%%%%%%%%%%%%%%%%%%%%%%%%%%%%%%%%%%%%%%%%%%%%%%%%%%%%%%%%%%%%%%%%%%%%%
\subsection{Legacy Detection}
\label{sec:detection}

The directive |\childdocmain| in the main file can detect
whether the complete document or merely a child is to be compiled
even without using the directive |\childdocof|.
This method is deprecated because it is less robust
and there is no compelling reason to use it;
it is merely provided for backward compatibility
and it may be removed in future versions.

If the detection mechanism is to be used,
it is mandatory to correctly specify
the filename of the main file as the argument of |\childdocmain|:
%
\begin{center}
\begin{tabular}{l}
|\input{childdoc.def}|\\
|\childdocmain{|\textit{main}|}|\\
\end{tabular}
\end{center}
%
If |\jobname| does not match the argument \textit{main} of |\childdocmain|,
it is assumed that |\jobname| points to the child file to be compiled.
When using |\childdocmain| with the main file specified as argument,
it suffices to start a child file
with just |\input{|\textit{main}|}|
without loading of the package and using |\childdocof|.
If instead all processing is done
with the appropriate \textsf{childdoc} directives,
the argument of \textit{main} of |\childdocmain| can be empty.

An alternative version of the command line processing described
in \secref{sec:commandline} using the detection mechanism reads:
%
\begin{center}
|... -jobname "|\textit{target}|" "|[\textit{flags}]%
[|\def\jobname{|\textit{dest}|}|]|\input{|\textit{main}|}"|
\end{center}

%%%%%%%%%%%%%%%%%%%%%%%%%%%%%%%%%%%%%%%%%%%%%%%%%%%%%%%%%%%%%%%%%%%%%%%%%%%%%%%%
\subsection{Manual Code}
\label{sec:manual}

In case one cannot be certain whether the definitions file |childdoc.def|
is installed on the target \TeX{} distribution
and one prefers not to ship it,
it is conceivable to paste a few relevant commands into the sources.

To that end, drop all statements |\input{childdoc.def}|
and perform the replacements as outlined below.
Instead of |\childdocmain{|\textit{main}|}| add the following code
to the top of the main file:
%
\begin{center}
\begin{tabular}{l}
|\||ifdefined\childdocname\endinput\||fi\newif\ifchilddoc|\\
|\edef\childdocname{\scantokens\expandafter{\jobname\noexpand}}|\\
|\def\childdocmain{|\textit{main}|}\||ifx\childdocmain\childdocname\||else|\\
|\childdoctrue\includeonly{\childdocname}\let\jobname\childdocmain\||fi|\\
\end{tabular}
\end{center}
%
Instead of |\childdocof{|\textit{main}|}| just include the main file
at the top of each child file:
%
\begin{center}
|\input{|\textit{main}|}|
\end{center}
%
A simple redirection |\childdocforward{|\textit{dest}|}| is achieved by:
%
\begin{center}
|\def\jobname{|\textit{dest}|}\input{\jobname}|
\end{center}
%
The redirection with prefix
|\childdocforwardprefix[|\textit{prefix}|]{|\textit{dest}|}|
is accomplished by:
%
\begin{center}
\begin{tabular}{l}
|{\edef\jobname{\scantokens\expandafter{\jobname\noexpand}}|\\
|\def\redirectjob |\textit{prefix}|#1~~~{\gdef\jobname{|\textit{dest}|#1}}|\\
|\expandafter\redirectjob\jobname~~~}\input{\jobname}|
\end{tabular}
\end{center}

In an alternative approach,
child documents can be compiled by a specific command line
without additional code or specific definitions:
%
\begin{center}
|... -jobname "|\textit{target}|" "|[\textit{flags}]%
|\includeonly{|\textit{dest}|}\input{|\textit{main}|}"|
\end{center}
%

%%%%%%%%%%%%%%%%%%%%%%%%%%%%%%%%%%%%%%%%%%%%%%%%%%%%%%%%%%%%%%%%%%%%%%%%%%%%%%%%
%%%%%%%%%%%%%%%%%%%%%%%%%%%%%%%%%%%%%%%%%%%%%%%%%%%%%%%%%%%%%%%%%%%%%%%%%%%%%%%%
\section{Information}

%%%%%%%%%%%%%%%%%%%%%%%%%%%%%%%%%%%%%%%%%%%%%%%%%%%%%%%%%%%%%%%%%%%%%%%%%%%%%%%%
\subsection{Copyright}

Copyright \copyright{} 2017--2018 Niklas Beisert

This work may be distributed and/or modified under the
conditions of the \LaTeX{} Project Public License, either version 1.3
of this license or (at your option) any later version.
The latest version of this license is in
  \url{http://www.latex-project.org/lppl.txt}
and version 1.3 or later is part of all distributions of \LaTeX{}
version 2005/12/01 or later.

This work has the LPPL maintenance status `maintained'.

The Current Maintainer of this work is Niklas Beisert.

This work consists of the files |README.txt|, |childdoc.ins| and |childdoc.dtx|
as well as the derived files |childdoc.def|, |cdocsamp.tex|
with |cdocsch1.tex|, |cdocsch2.tex|, |cdocspt3.tex|, |cdocspt4.tex|,
|cdocsdrf.tex|, |cdocsfn1.tex|, |cdocsfn2.tex|
as well as |childdoc.pdf|.

%%%%%%%%%%%%%%%%%%%%%%%%%%%%%%%%%%%%%%%%%%%%%%%%%%%%%%%%%%%%%%%%%%%%%%%%%%%%%%%%
\subsection{Files and Installation}

The package consists of the files:
%
\begin{center}
\begin{tabular}{ll}
    |README.txt|   & readme file \\
    |childdoc.ins| & installation file \\
    |childdoc.dtx| & source file \\
    |childdoc.def| & definition file \\
    |cdocsamp.tex| & sample main file \\
    |cdocsch1.tex| & sample include file \\
    |cdocsch2.tex| & sample include file \\
    |cdocspt3.tex| & sample part file \\
    |cdocspt4.tex| & sample part file \\
    |cdocsdrf.tex| & sample redirection file \\
    |cdocsfn1.tex| & sample redirection file \\
    |cdocsfn2.tex| & sample redirection file \\
    |childdoc.pdf| & manual
\end{tabular}
\end{center}
%
The distribution consists of the files
|README.txt|, |childdoc.ins| and |childdoc.dtx|.
%
\begin{itemize}
\item
Run (pdf)\LaTeX{} on |childdoc.dtx|
to compile the manual |childdoc.pdf| (this file).
\item
Run \LaTeX{} on |childdoc.ins| to create the definitions file |childdoc.def|
and the sample |cdocsamp.tex| with include files
|cdocsch1.tex|, |cdocsch2.tex|, |cdocspt3.tex|, |cdocspt4.tex|,
|cdocsdrf.tex|, |cdocsfn1.tex|, |cdocsfn2.tex|.
Then copy the file |childdoc.def| to an appropriate directory of your \LaTeX{}
distribution, e.g.\ \textit{texmf-root}|/tex/latex/childdoc|.
\end{itemize}

%%%%%%%%%%%%%%%%%%%%%%%%%%%%%%%%%%%%%%%%%%%%%%%%%%%%%%%%%%%%%%%%%%%%%%%%%%%%%%%%
\subsection{Related CTAN Packages}

There are several other packages which offer a similar functionality:
%
\begin{itemize}
\item
The packages
\href{http://ctan.org/pkg/docmute}{\textsf{docmute}},
\href{http://ctan.org/pkg/includex}{\textsf{includex}} and
\href{http://ctan.org/pkg/standalone}{\textsf{standalone}}
provide commands to include only the document body of
a child file thus allowing both files to be compiled individually.
\item
The packages \href{http://ctan.org/pkg/subdocs}{\textsf{subdocs}}
and \href{http://ctan.org/pkg/subfiles}{\textsf{subfiles}}
provide structures in which the main and child documents can be
encapsulated and allowing them to be compiled individually.
The inclusion mechanism is different from the conventional |\include|.
\item
The package \href{http://ctan.org/pkg/combine}{\textsf{combine}}
is an elaborate solution to combine several documents into one.
\end{itemize}
%
See also the CTAN topic \href{http://ctan.org/topic/subdocs}{\textsf{subdocs}}
for further related packages.
The present package differs from the above solutions in that
a document structure constructed with the conventional |\include| mechanism
just needs two extra commands at the top of every file
such that all constituent files can be compiled individually.

%%%%%%%%%%%%%%%%%%%%%%%%%%%%%%%%%%%%%%%%%%%%%%%%%%%%%%%%%%%%%%%%%%%%%%%%%%%%%%%%
%\subsection{Feature Suggestions}
%
%The following is a list of features which may be useful for future
%versions of this package:
%%
%\begin{itemize}
%\item
%\ldots
%\end{itemize}

%%%%%%%%%%%%%%%%%%%%%%%%%%%%%%%%%%%%%%%%%%%%%%%%%%%%%%%%%%%%%%%%%%%%%%%%%%%%%%%%
\subsection{Revision History}

%%%%%%%%%%%%%%%%%%%%%%%%%%%%%%%%%%%%%%%%
\paragraph{v2.0:} 2018/12/30

\begin{itemize}
\item
immediate forward processing
\item
added |\childdocby| mechanism
\item
manual restructured
\end{itemize}

%%%%%%%%%%%%%%%%%%%%%%%%%%%%%%%%%%%%%%%%
\paragraph{v1.6:} 2018/01/17

\begin{itemize}
\item
application for development of include files
\item
corrections to manual
\end{itemize}

%%%%%%%%%%%%%%%%%%%%%%%%%%%%%%%%%%%%%%%%
\paragraph{v1.5:} 2017/05/21

\begin{itemize}
\item
more complete structuring introduced
\item
|\childdocof| introduced
\item
|\childdoc| renamed to |\childdocmain|
\item
|\childredirect| renamed to |\childdocforward| and |\childdocforwardprefix|
and functionality expanded
\end{itemize}

%%%%%%%%%%%%%%%%%%%%%%%%%%%%%%%%%%%%%%%%
\paragraph{v1.0:} 2017/04/27

\begin{itemize}
\item
manual and install package
\item
first version published on CTAN
\end{itemize}

%%%%%%%%%%%%%%%%%%%%%%%%%%%%%%%%%%%%%%%%
\paragraph{v0.6:} 2017/04/26

\begin{itemize}
\item
redirection mechanism added
\end{itemize}

%%%%%%%%%%%%%%%%%%%%%%%%%%%%%%%%%%%%%%%%
\paragraph{v0.5:} 2017/04/26

\begin{itemize}
\item
functionality in definition file
\end{itemize}


%%%%%%%%%%%%%%%%%%%%%%%%%%%%%%%%%%%%%%%%%%%%%%%%%%%%%%%%%%%%%%%%%%%%%%%%%%%%%%%%
%%%%%%%%%%%%%%%%%%%%%%%%%%%%%%%%%%%%%%%%%%%%%%%%%%%%%%%%%%%%%%%%%%%%%%%%%%%%%%%%
%%%%%%%%%%%%%%%%%%%%%%%%%%%%%%%%%%%%%%%%%%%%%%%%%%%%%%%%%%%%%%%%%%%%%%%%%%%%%%%%
\appendix

\settowidth\MacroIndent{\rmfamily\scriptsize 000\ }

 \DocInput{childdoc.dtx}

\end{document}
%</driver>
% \fi
%
% %%%%%%%%%%%%%%%%%%%%%%%%%%%%%%%%%%%%%%%%%%%%%%%%%%%%%%%%%%%%%%%%%%%%%%%%%%%%%%
% %%%%%%%%%%%%%%%%%%%%%%%%%%%%%%%%%%%%%%%%%%%%%%%%%%%%%%%%%%%%%%%%%%%%%%%%%%%%%%
% \section{Sample}
%\iffalse
%<*samplemain>
%\fi
%
% The following presents a sample document
% with two chapters, two parts, a title page,
% a compile flag as well as three forwarding files to set the flag.
% It consists of eight |.tex| files:
% \begin{center}
% \begin{tabular}{ll}
% |cdocsamp.tex|&main file\\
% |cdocsch1.tex|&include file for chapter 1\\
% |cdocsch2.tex|&include file for chapter 2\\
% |cdocspt3.tex|&include file for part 3\\
% |cdocspt4.tex|&include file for part 4\\
% |cdocsdrf.tex|&forwarding file for main file in draft mode\\
% |cdocsfi1.tex|&forwarding file for final version of chapter 1\\
% |cdocsfi2.tex|&forwarding file for final version of chapter 2\\
% \end{tabular}
% \end{center}
% Each of the eight files can be compiled directly by the \LaTeX{} compiler.
%
% %%%%%%%%%%%%%%%%%%%%%%%%%%%%%%%%%%%%%%
% \paragraph{Main File.}
%
% The main file is called |cdocsamp.tex|.
%
% Load the \textsf{childdoc} definitions and
% declare the filename for the main document:
%    \begin{macrocode}
\input{childdoc.def}
\childdocmain{}
%    \end{macrocode}

% Optional override for |\version| flag:
%    \begin{macrocode}
%%\ifchilddoc\else\providecommand{\version}{draft}\fi
%    \end{macrocode}

% Define the default values for the |\version| flag
% (|final| for the main file and |draft| for childs):
%    \begin{macrocode}
\ifchilddoc
\providecommand{\version}{draft}
\else
\providecommand{\version}{final}
\fi
%    \end{macrocode}

% Load the standard document class:
%    \begin{macrocode}
\documentclass[12pt]{article}
%    \end{macrocode}

% Start the document body:
%    \begin{macrocode}
\begin{document}
%    \end{macrocode}

% Declare a title page.
% Print title, part of document being processed and version flag:
%    \begin{macrocode}
\addtocounter{page}{-1}
\begin{center}
{\LARGE\bfseries{}childdoc example\par}
\vspace{1cm}
\ifchilddoc
\ifchilddocmanual part\else chapter\fi:
`\childdocname' of `\childdocjob'\par
\else
main document: `\childdocjob'\par
\fi
version: \version\par
\end{center}
\newpage
%    \end{macrocode}

% Manually include selected file,
% otherwise process as usual:
%    \begin{macrocode}
\ifchilddocmanual
\section*{part `\childdocname'}
\input{\childdocname}
\else
%    \end{macrocode}

% Include the two chapters:
%    \begin{macrocode}
\include{cdocsch1}
\include{cdocsch2}
%    \end{macrocode}

% Include the two parts unless only chapters should be displayed:
%    \begin{macrocode}
\ifchilddoc\else
\section{part three}
\input{cdocspt3}
\section{part four}
\input{cdocspt4}
\fi
%    \end{macrocode}

% Process as usual until here:
%    \begin{macrocode}
\fi
%    \end{macrocode}

% End of document body:
%    \begin{macrocode}
\end{document}
%    \end{macrocode}
%\iffalse
%</samplemain>
%\fi
%
% %%%%%%%%%%%%%%%%%%%%%%%%%%%%%%%%%%%%%%
% \paragraph{Chapter Include Files.}
%
% The include files are called |cdocsch1.tex| and |cdocsch2.tex|.
%
%\iffalse
%<*samplechap1|samplechap2>
%\fi

% Optional override for |\version| flag:
%    \begin{macrocode}
%%\providecommand{\version}{final}
%    \end{macrocode}

% Include the main document:
%    \begin{macrocode}
\input{childdoc.def}
\childdocof{cdocsamp}
%    \end{macrocode}

%\iffalse
%</samplechap1|samplechap2>
%\fi
%
%\iffalse
%<*samplechap1>
%\fi
% Some text for chapter 1:
%    \begin{macrocode}
\section{one}
some text in chapter one
%    \end{macrocode}

%\iffalse
%</samplechap1>
%\fi
% Some text for chapter 2:
%\iffalse
%<*samplechap2>
%\fi
%    \begin{macrocode}
\section{two}
more text in chapter two
%    \end{macrocode}

%\iffalse
%</samplechap2>
%\fi
%
% %%%%%%%%%%%%%%%%%%%%%%%%%%%%%%%%%%%%%%
% \paragraph{Part Include Files.}
%
% The include files are called |cdocspt3.tex| and |cdocspt4.tex|.
%
%\iffalse
%<*samplepart3|samplepart4>
%\fi

% Optional override for |\version| flag:
%    \begin{macrocode}
%%\providecommand{\version}{final}
%    \end{macrocode}

% Include the main document:
%    \begin{macrocode}
\input{childdoc.def}
\childdocby{cdocsamp}
%    \end{macrocode}

%\iffalse
%</samplepart3|samplepart4>
%\fi
%
%\iffalse
%<*samplepart3>
%\fi
% Some text for part 3:
%    \begin{macrocode}
some text in part three
%    \end{macrocode}

%\iffalse
%</samplepart3>
%\fi
% Some text for part 4:
%\iffalse
%<*samplepart4>
%\fi
%    \begin{macrocode}
more text in part four
%    \end{macrocode}

%\iffalse
%</samplepart4>
%\fi
%
% %%%%%%%%%%%%%%%%%%%%%%%%%%%%%%%%%%%%%%
% \paragraph{Forwarding for a Complete Draft.}
%
% The following forwarding file |cdocsdrf.tex|
% compiles the main document in draft mode:
%\iffalse
%<*sampledraft>
%\fi
%    \begin{macrocode}
\def\version{draft}
\input{childdoc.def}
\childdocforward{cdocsamp}
%    \end{macrocode}

%\iffalse
%</sampledraft>
%\fi
%
% %%%%%%%%%%%%%%%%%%%%%%%%%%%%%%%%%%%%%%
% \paragraph{Forwarding for Final Version of the Chapters.}
%
% The following forwarding files |cdocsfn1.tex| and |cdocsfn2.tex|
% (with identical content)
% compile the final versions of the child documents
% |cdocsch1.tex| and |cdocsch2.tex|, respectively:
%\iffalse
%<*samplefinal>
%\fi
%    \begin{macrocode}
\def\version{final}
\input{childdoc.def}
\childdocforwardprefix[cdocsamp]{cdocsfn}{cdocsch}
%    \end{macrocode}

%\iffalse
%</samplefinal>
%\fi
%
% %%%%%%%%%%%%%%%%%%%%%%%%%%%%%%%%%%%%%%
% \paragraph{Command Line Processing.}
%
% The following three command lines generate the output files
% |cdocscld|, |cdocscl1| and |cdocscl2|
% which should be identical to
% |cdocsdrf|, |cdocsch1| and |cdocsfn2|, respectively:
% \begin{center}
% \begin{tabular}{l}
% |latex -jobname cdocscld \|\\
% |  "\def\version{draft}\input{childdoc.def}\childdocforward{cdocsamp}"|\\
% |latex -jobname cdocscl1 \|\\
% |  "\input{childdoc.def}\childdocforward[cdocsamp]{cdocsch1}"|\\
% |latex -jobname cdocscl2 \|\\
% |  "\def\version{final}\input{childdoc.def}\childdocforward{cdocsch2}"|
% \end{tabular}
% \end{center}
% Note that the trailing backslash on each first line
% merely continues the input to the second line
% (for convenient cut ant paste).
% Furthermore, the command |latex| can be replaced by any
% of its alternative versions such as |pdflatex|.
%
% %%%%%%%%%%%%%%%%%%%%%%%%%%%%%%%%%%%%%%%%%%%%%%%%%%%%%%%%%%%%%%%%%%%%%%%%%%%%%%
% %%%%%%%%%%%%%%%%%%%%%%%%%%%%%%%%%%%%%%%%%%%%%%%%%%%%%%%%%%%%%%%%%%%%%%%%%%%%%%
% \section{Implementation}
%\iffalse
%<*package>
%\fi
%
% This section describes the definitions file |childdoc.def|.

% The definitions cannot be loaded using |\usepackage| or |\RequirePackage|
% which has a mechanism to prevent loading a style file more than once.
% When loading the definitions by means of |\input|
% multiple instances have to be prevented manually:
%\iffalse
%This code needs to be before the `\ProvidesFile' directive
%which is defined at the beginning of this file.
%Therefore it is also placed there and commented out here.
%</package>
%<*discard>
%\fi
%    \begin{macrocode}
\ifdefined\childdocmain\endinput\fi
%    \end{macrocode}
%\iffalse
%</discard>
%<*package>
%\fi
%
% \macro{\ifchilddoc}
% \macro{\ifchilddocmanual}
% The conditional |\ifchilddoc| tells whether a
% child (true) or main (false) document is being compiled.
% The conditional |\ifchilddocmanual| tells whether
% the |\includeonly| mechanism is used (false) or
% the selection of child files must be performed manually (true).
% The definitions initialise to false:
%    \begin{macrocode}
\newif\ifchilddoc
\newif\ifchilddocmanual
%    \end{macrocode}

% \macro{\childdocname}
% \macro{\childdocjob}
% The macro |\childdocname| stores the name of the main document
% to be compiled. The macro |\childdocjob| stores the name of
% the document on which the \LaTeX{} compiler was originally invoked.
% The content of |\jobname| cannot be compared
% to filenames specified in the source due to different catcodes.
% The following code rescans |\jobname|, stores the result
% in |\childdocname| and saves a copy in |\childdocjob|:
%    \begin{macrocode}
\edef\childdocname{\scantokens\expandafter{\jobname\noexpand}}
\let\childdocjob\childdocname
%    \end{macrocode}

% \macro{\childdocdisable}
% The macro |\childdocdisable| prevents the main file
% from being processed more than once.
% At this stage, the main document command |\childdocmain|
% is assumed to be called once again where it should do nothing.
% Any subsequent call to it should prevent
% a secondary processing of the main document
% It overwrites the forwarding commands
% |\childdocof| and |\childdocforward|
% with empty macros to prevent further inclusions of the main document:
%    \begin{macrocode}
\newcommand{\childdocdisable}
{
  \renewcommand{\childdocmain}[1]{\renewcommand{\childdocmain}[1]{\endinput}}
  \renewcommand{\childdocof}[1]{}
  \renewcommand{\childdocby}[2][]{}
  \renewcommand{\childdocforward}[2][]{}
  \renewcommand{\childdocdisable}{}
}
%    \end{macrocode}

% \macro{\childdocmain}
% The macro |\childdocmain| is to be called at the top of the main file
% with nothing or the main filename (without extension) as argument.
% First, it breaks loops.
% If the argument is not empty and does not match |\childdocname|
% (which is set by the first inclusion of |childdoc.def|),
% |\ifchilddoc| is set to true, |\includeonly| is applied to the child file
% and |\jobname| is set to the main file
% (for proper handling of |.aux| files):
%    \begin{macrocode}
\newcommand{\childdocmain}[1]
{
  \childdocdisable\childdocmain{}
  \if?#1?\else
    \begingroup
      \def\childdoctmp{#1}
      \ifx\childdoctmp\childdocname
        \def\childdoctmp{}
      \else
        \def\childdoctmp
        {
          \childdoctrue
          \includeonly{\childdocname}
          \def\childdocjob{#1}
          \def\jobname{#1}
        }
      \fi
      \expandafter
    \endgroup
    \childdoctmp
  \fi
}
%    \end{macrocode}

% \macro{\childdocof}
% The command |\childdocof| redirects
% compilation to the main file |#1|.
%    \begin{macrocode}
\newcommand{\childdocof}[1]
{
  \childdocdisable
  \childdoctrue
  \includeonly{\childdocname}
  \def\jobname{#1}
  \def\childdocjob{#1}
  \input{#1}
}
%    \end{macrocode}

% \macro{\childdocby}
% The command |\childdocby| ....
%    \begin{macrocode}
\newcommand{\childdocby}[2][]
{
  \childdocdisable
  \childdoctrue
  \childdocmanualtrue
  \if?#1?\else
    \def\jobname{#2}
  \fi
  \def\childdocjob{#2}
  \input{#2}
  \endinput
}
%    \end{macrocode}

% \macro{\childdocforward}
% The command |\childdocforward| redirects
% compilation to the main file or
% (if the optional argument is given) a child file.
% Parameters are set as if the main file
% or a child file starting with |\childdocof| was compiled.
% Then compilation is handed over to the main file:
%    \begin{macrocode}
\newcommand{\childdocforward}[2][]
{
  \begingroup
    \if?#1?
      \def\childdoctmp
      {
        \def\childdocname{#2}
        \def\childdocjob{#2}
        \def\jobname{#2}
        \input{#2}
        \endinput
      }
    \else
      \def\childdoctmp
      {
        \childdocdisable
        \def\childdocname{#2}
        \childdoctrue
        \includeonly{#2}
        \def\childdocjob{#1}
        \def\jobname{#1}
        \input{#1}
        \endinput
      }
    \fi
    \expandafter
  \endgroup
  \childdoctmp
}
%    \end{macrocode}

% \macro{\childdocforwardprefix}
% The command |\childdocforwardprefix| redirects
% compilation to the main or a child file by means of a pattern.
% The prefix |#1| in the current filename is replaced by |#2|
% and the suffix of the current filename is kept
% (it is assumed that the filename does not contain the substring `|~~~|'
% which is used as a delimiter).
% Compilation is handed over to the new file by |\childdocforward|:
%    \begin{macrocode}
\newcommand{\childdocforwardprefix}[3][]
{
  \begingroup
    \def\childdocextract #2##1~~~{\def\childdoctmp{\childdocforward[#1]{#3##1}}}
    \expandafter\childdocextract\childdocname~~~
    \expandafter
  \endgroup
  \childdoctmp
}
%    \end{macrocode}

% \macro{\childdoc}
% The deprecated macro |\childdoc| is a legacy version of |\childdocmain|:
%    \begin{macrocode}
\newcommand{\childdoc}{\childdocmain}
%    \end{macrocode}

% \macro{\childdocredirect}
% The deprecated macro |\childdocredirect| is a legacy version
% of |\childdocforward| and |\childdocforwardprefix|:
%    \begin{macrocode}
\newcommand{\childdocredirect}[2][]
{
  \begingroup
    \if?#1?
      \def\childdoctmp{\childdocforward{#2}}
    \else
      \def\childdoctmp{\childdocforwardprefix{#1}{#2}}
    \fi
    \expandafter
  \endgroup
  \childdoctmp
}
%    \end{macrocode}

%\iffalse
%</package>
%\fi
%
\endinput

\childdocof{cdocsamp}
%    \end{macrocode}

%\iffalse
%</samplechap1|samplechap2>
%\fi
%
%\iffalse
%<*samplechap1>
%\fi
% Some text for chapter 1:
%    \begin{macrocode}
\section{one}
some text in chapter one
%    \end{macrocode}

%\iffalse
%</samplechap1>
%\fi
% Some text for chapter 2:
%\iffalse
%<*samplechap2>
%\fi
%    \begin{macrocode}
\section{two}
more text in chapter two
%    \end{macrocode}

%\iffalse
%</samplechap2>
%\fi
%
% %%%%%%%%%%%%%%%%%%%%%%%%%%%%%%%%%%%%%%
% \paragraph{Part Include Files.}
%
% The include files are called |cdocspt3.tex| and |cdocspt4.tex|.
%
%\iffalse
%<*samplepart3|samplepart4>
%\fi

% Optional override for |\version| flag:
%    \begin{macrocode}
%%\providecommand{\version}{final}
%    \end{macrocode}

% Include the main document:
%    \begin{macrocode}
% \iffalse
%
% childdoc.dtx Copyright (C) 2017-2018 Niklas Beisert
%
% This work may be distributed and/or modified under the
% conditions of the LaTeX Project Public License, either version 1.3
% of this license or (at your option) any later version.
% The latest version of this license is in
%   http://www.latex-project.org/lppl.txt
% and version 1.3 or later is part of all distributions of LaTeX
% version 2005/12/01 or later.
%
% This work has the LPPL maintenance status `maintained'.
%
% The Current Maintainer of this work is Niklas Beisert.
%
% This work consists of the files childdoc.dtx and childdoc.ins
% and the derived files childdoc.def and cdocsamp.tex with
% cdocsch1.tex, cdocsch2.tex, cdocsdrf.tex, cdocsfn1.tex, cdocsfn2.tex.
%
%<package>\ifdefined\childdocmain\endinput\fi
%<package>\ProvidesFile{childdoc.def}[2018/12/30 v2.0 child document driver]
%<samplemain>\ProvidesFile{cdocsamp.tex}[2018/12/30 v2.0 sample for childdoc]
%<*driver>
%\ProvidesFile{childdoc.drv}[2018/12/30 v2.0 childdoc reference manual file]
\PassOptionsToClass{10pt,a4paper}{article}
\documentclass{ltxdoc}

\usepackage[margin=35mm]{geometry}
\usepackage{hyperref}
\usepackage{hyperxmp}
\usepackage[usenames]{color}

\hypersetup{colorlinks=true}
\hypersetup{pdfstartview=FitH}
\hypersetup{pdfpagemode=UseNone}
\hypersetup{pdfsource={}}
\hypersetup{pdflang={en-UK}}
\hypersetup{pdfcopyright={Copyright 2017-2018 Niklas Beisert.
  This work may be distributed and/or modified under the
  conditions of the LaTeX Project Public License, either version 1.3
  of this license or (at your option) any later version.}}
\hypersetup{pdflicenseurl={http://www.latex-project.org/lppl.txt}}
\hypersetup{pdfcontactaddress={ETH Zurich, ITP, HIT K,
  Wolfgang-Pauli-Strasse 27}}
\hypersetup{pdfcontactpostcode={8093}}
\hypersetup{pdfcontactcity={Zurich}}
\hypersetup{pdfcontactcountry={Switzerland}}
\hypersetup{pdfcontactemail={nbeisert@itp.phys.ethz.ch}}
\hypersetup{pdfcontacturl={http://people.phys.ethz.ch/\xmptilde nbeisert/}}

\newcommand{\secref}[1]{\hyperref[#1]{section \ref*{#1}}}

\parskip1ex
\parindent0pt
\let\olditemize\itemize
\def\itemize{\olditemize\parskip0pt}

\begin{document}

\title{The \textsf{childdoc} Package}
\hypersetup{pdftitle={The childdoc Package}}
\author{Niklas Beisert\\[2ex]
  Institut f\"ur Theoretische Physik\\
  Eidgen\"ossische Technische Hochschule Z\"urich\\
  Wolfgang-Pauli-Strasse 27, 8093 Z\"urich, Switzerland\\[1ex]
  \href{mailto:nbeisert@itp.phys.ethz.ch}
  {\texttt{nbeisert@itp.phys.ethz.ch}}}
\hypersetup{pdfauthor={Niklas Beisert}}
\hypersetup{pdfsubject={Manual for the LaTeX2e Package childdoc}}
\date{30 December 2018, \textsf{v2.0}}
\maketitle

\begin{abstract}\noindent
\textsf{childdoc} is a \LaTeXe{} package
that enables the direct compilation
of document sections included by |\include|
to individual files.
\end{abstract}

\begingroup
\parskip0ex
\tableofcontents
\endgroup

%%%%%%%%%%%%%%%%%%%%%%%%%%%%%%%%%%%%%%%%%%%%%%%%%%%%%%%%%%%%%%%%%%%%%%%%%%%%%%%%
%%%%%%%%%%%%%%%%%%%%%%%%%%%%%%%%%%%%%%%%%%%%%%%%%%%%%%%%%%%%%%%%%%%%%%%%%%%%%%%%
\section{Introduction}

\LaTeX{} provides a mechanism to structure a large document (such as a book)
into a main file and several child files (containing the chapters)
using the |\include| command.
This mechanism is beneficial for documents
which span hundreds of pages in order to
make the source file(s) more manageable.
Moreover, compilation can be restricted to
selected child files by means of the |\includeonly| command.
The latter feature can be used to reduce the compilation time while editing
(this was significantly more useful in the earlier days of \LaTeX{})
or to generate a smaller document which is easier to navigate.
Another application of |\includeonly| is to generate
documents consisting of selected parts of the complete document.

However, there are a few drawbacks of the plain |\include| mechanism:
\begin{itemize}
\item
The child files cannot be compiled on their own,
they can only be compiled via the main file.
A naive editing environment
(such as a text editor with an option
to have the current file processed by \LaTeX)
may require one to switch to the main file before compiling;
attempting to compile the child file produces errors.
\item
The main file must be modified (each time)
to adjust the |\includeonly| command
to the present needs. This easily leaves the main file in a messy state.
\item
The generated document will always carry the filename
of the main document. This is inconvenient if
several child files are to be compiled and
to be kept for distribution.
\end{itemize}

The present package provides a simple interface
to make child files individually compilable by \LaTeX{}.
Compiling a child file then has the same effect as compiling
the main file with an |\includeonly| command
to select the appropriate child.
Moreover the generated document will carry the name of the child
rather than the main file.
This resolves all three above issues.

This feature is meant to make the editing of books,
thesis documents and lecture notes somewhat more convenient.
However, the package can also be used efficiently for
composing a series of documents (such as exercise sheets)
which are typically distributed individually.
It then assists the author in generating the individual documents
(potentially in different versions)
as well as a document containing the collected series.
Another application is in developing style files
or other kinds of included material
where compilation of the style file could redirect
to a sample or test file.

%%%%%%%%%%%%%%%%%%%%%%%%%%%%%%%%%%%%%%%%%%%%%%%%%%%%%%%%%%%%%%%%%%%%%%%%%%%%%%%%
%%%%%%%%%%%%%%%%%%%%%%%%%%%%%%%%%%%%%%%%%%%%%%%%%%%%%%%%%%%%%%%%%%%%%%%%%%%%%%%%
\section{Usage}

First of all, the package \textsf{childdoc} is \emph{not} a standard
\LaTeXe{} |.sty| style file! Therefore it needs to be invoked in
a non-standard way.

%%%%%%%%%%%%%%%%%%%%%%%%%%%%%%%%%%%%%%%%%%%%%%%%%%%%%%%%%%%%%%%%%%%%%%%%%%%%%%%%
\subsection{Included Files}
\label{sec:include}

%%%%%%%%%%%%%%%%%%%%%%%%%%%%%%%%%%%%%%%%
\DescribeMacro{\childdocmain}
To use the package, add the commands
\begin{center}
\begin{tabular}{l}
|\input{childdoc.def}|\\
|\childdocmain{}|\\
\end{tabular}
\end{center}
at the very top of the main \LaTeX{} file,
in particular \emph{before} the |\documentclass| statement!
The argument of |\childdocmain| should be left empty
(but it must be present).

%%%%%%%%%%%%%%%%%%%%%%%%%%%%%%%%%%%%%%%%
\DescribeMacro{\childdocof}
Furthermore, add the commands
\begin{center}
\begin{tabular}{l}
|\input{childdoc.def}|\\
|\childdocof{|\textit{main}|}|\\
\end{tabular}
\end{center}
at the top of every child file \textit{child}
which is included by |\include{|\textit{child}|}|
from within the main file
(or at least for those files to be compiled individually).
The argument \textit{main} must be the filename of the main file.

There are a couple of
considerations in setting up the main and child documents:

%%%%%%%%%%%%%%%%%%%%%%%%%%%%%%%%%%%%%%%%
\paragraph{Restrictions.}

Please note the following restrictions:
\begin{itemize}
\item
|\childdocmain| must be called with one argument \textit{main}
to ensure compatibility with earlier version of the package.
It must either be empty (|\childdocmain{}|)
or precisely match the filename of the main file in which it is specified.
See \secref{sec:detection} for further information.
\item
The filename \textit{main} must be specified without the |.tex| extension.
\item
The filename \textit{main} is case sensitive
(even in case-insensitive file systems)
due to internal string comparison.
\item
The argument \textit{main} should be fully expanded, it cannot be a macro.
\item
Subdirectories and special characters should be avoided in filenames.
\item
The command |\childdocmain{|\textit{main}|}| must be followed by a whitespace.
It should not be followed immediately by another command
or by a comment mark `|%|'.
This is because the \TeX{} parser reads the token immediately following
the argument of |\childdocmain| and puts it
at the beginning of every child section;
however, a white\-space is ignored.
\end{itemize}

%%%%%%%%%%%%%%%%%%%%%%%%%%%%%%%%%%%%%%%%
\paragraph{Content of Main File.}

It is advisable to place all content in the child files included by |\include|.
Any output contained in the main file will appear in all child documents
unless suppressed manually;
it cannot be suppressed automatically by the |\includeonly| directive
and thus should normally be avoided.
A method to include some content in the main file
by means of conditional processing is described in \secref{sec:conditional}.

%%%%%%%%%%%%%%%%%%%%%%%%%%%%%%%%%%%%%%%%
\paragraph{Page Numbering.}

When only a part of the document is compiled,
the appropriate numbering of pages
(as well as other status parameters)
is determined from the |.aux| files.
The latter contain information from previous passes.
However this information needs to propagate through
all intermediate child documents.
Therefore the page numbering in child documents may well
be inconsistent until the complete document is compiled at least once.

A useful (if unconventional) way to always ensure a consistent
page numbering is to restart the numbering in each child document
and denote the pages by `\textit{child}|.|\textit{page}'
where \textit{child} represents the chapter/section number of the child file.
This can be achieved by the command
|\numberwithin{page}{|\textit{child}|}|
of the \textsf{amsmath} package
where \textit{child} can be |chapter| or |section|
depending on the chosen structuring.
Alternatively, one can modify the macro |\thepage| appropriately
and reset the counter |page| at the start of each child file.

%%%%%%%%%%%%%%%%%%%%%%%%%%%%%%%%%%%%%%%%%%%%%%%%%%%%%%%%%%%%%%%%%%%%%%%%%%%%%%%%
\subsection{Conditional Processing}
\label{sec:conditional}

The package provides a mechanism to compile different versions
of a document. To customise the versions further some conditional processing
can come in handy to distinguish which version is being compiled.
The package provides two macros to describe the compilation context:

%%%%%%%%%%%%%%%%%%%%%%%%%%%%%%%%%%%%%%%%
\DescribeMacro{\ifchilddoc}
The conditional |\ifchilddoc| distinguishes between the compilation of
child documents and the main document:
%
\begin{center}
|\ifchilddoc |\textit{child-code}| |[|\||else |\textit{main-code}]| \||fi|
\end{center}

%%%%%%%%%%%%%%%%%%%%%%%%%%%%%%%%%%%%%%%%
\DescribeMacro{\childdocname}
\DescribeMacro{\childdocjob}
The macro |\childdocname| contains the filename (without extension)
of the main or child file being processed.
Note that |\childdocjob| will always contain the name of the main file.

%%%%%%%%%%%%%%%%%%%%%%%%%%%%%%%%%%%%%%%%
\paragraph{Title Page.}

Conditional processing can be used to include a title or banner page
in the main document when proper precautions are taken.
Importantly, the code in the main file should ensure that the page counter
(as well as other status parameters which are stored in the |.aux| files)
takes the same value after the conditional processing.
Otherwise the page numbers may take divergent values
depending on which part is compiled.

For example, a title page could be declared by:
%
\begin{center}
\begin{tabular}{l}
|\ifchilddoc\||else|\\
|\addtocounter{page}{-1}|\\
\textit{code for title page}\\
|\newpage|\\
|\||fi|
\end{tabular}
\end{center}
%
A banner page for the child documents can be generated by:
%
\begin{center}
\begin{tabular}{l}
|\ifchilddoc|\\
|\addtocounter{page}{-1}|\\
\textit{code for banner page}\\
|\newpage|\\
|\||fi|
\end{tabular}
\end{center}
%
Here one could write a message such as:
\begin{center}
|This is the part \childdocname{} of \childdocjob{}.|
\end{center}

%%%%%%%%%%%%%%%%%%%%%%%%%%%%%%%%%%%%%%%%%%%%%%%%%%%%%%%%%%%%%%%%%%%%%%%%%%%%%%%%
\subsection{Flags}
\label{sec:flags}

The package makes it easy to generate different versions
of the main or child documents.
To this end compilation flags can be defined
and assigned different default values.
They will be particularly useful in conjunction
with the forwarding mechanism described in \secref{sec:forward}.

For example, it may be useful to have a flag |\version|
which can be set to |draft| or |final|.
The document source will contain some conditional code
depending on the value of |\version|.
Suppose further, the flag should default to |final| for the main file
and to |draft| for child files
which is a natural assignment for editing the document.
This is achieved by placing the following code
in the preamble of the main document
(below the |\childdocmain| directive):
%
\begin{center}
\begin{tabular}{l}
|\ifchilddoc|\\
|\providecommand{\version}{draft}|\\
|\||else|\\
|\providecommand{\version}{final}|\\
|\||fi|
\end{tabular}
\end{center}
%
The definition by |\providecommand| makes sure
that previous definitions are not overwritten.
Further statements |\providecommand{\version}{...}|
can thus be added before the above code to override it.

For the main file, one might add a line
(between |\childdocmain| and the above block)
%
\begin{center}
|%\ifchilddoc\||else\providecommand{\version}{draft}\||fi|
\end{center}
%
which can be uncommented to produce a draft version.
Likewise one can add a line to the very top of a child file
(above the |\childdocof{|\textit{main}|}| directive)
%
\begin{center}
|%\providecommand{\version}{final}|
\end{center}
%
which can be uncommented to produce the final version of this child document.

%%%%%%%%%%%%%%%%%%%%%%%%%%%%%%%%%%%%%%%%%%%%%%%%%%%%%%%%%%%%%%%%%%%%%%%%%%%%%%%%
\subsection{Forwarding}
\label{sec:forward}

Different versions of the main or child documents
using compilation flags as described in \secref{sec:flags}
can be (permanently) stored in different files
for convenient compilation, viewing and distribution.
To this end, the package defines a command
to pass on compilation to a different file:

%%%%%%%%%%%%%%%%%%%%%%%%%%%%%%%%%%%%%%%%
\DescribeMacro{\childdocforward}
The command |\childdocforward| redirects processing to
another source file:
%
\begin{center}
\begin{tabular}{l}
|\input{childdoc.def}|\\
|\childdocforward[|\textit{main}|]{|\textit{dest}|}|\\
\end{tabular}
\end{center}
%
The argument \textit{dest} is the destination file
(without extension).
It should be the main file or one of the child files.
Note that further \textsf{childdoc} directives
such as |\childdocof| and |\childdocforward|
in the indicated file will be processed in this form.
The optional argument \textit{main}
passes on directly to the main file \textit{main}
while pretending to compile the child \textit{dest}.
This form behaves as if \textit{dest}
issues |\childdocof{|\textit{main}|}| right away,
and no further \textsf{childdoc} directives will be processed.

%%%%%%%%%%%%%%%%%%%%%%%%%%%%%%%%%%%%%%%%
\DescribeMacro{\...prefix}
In the alternative form |\childdocforwardprefix|,
%
\begin{center}
\begin{tabular}{l}
|\input{childdoc.def}|\\
|\childdocforwardprefix[|\textit{main}|]{|\textit{prefix}|}{|\textit{dest}|}|
\end{tabular}
\end{center}
%
the destination file is determined by a pattern
depending on the current file:
To make this work, the current file must be called
`{\textit{prefix}\hspace{0.2em}\textit{suffix}}'
with \textit{prefix} matching precisely the argument.
Processing is then passed on to the file
`{\textit{dest}\hspace{0.2em}\textit{suffix}}'.
Surely, the same effect is achieved by
directly specifying the
argument `{\textit{dest}\hspace{0.2em}\textit{suffix}}'
in the first form.
However, that requires to set up a different file
for each child. With the alternative form of the command
all these files can have exactly the same content
which simplifies setting them up and maintaining them.

For example, the following file |draft.tex|
with a compilation flag |\version| as described in \secref{sec:flags}
compiles the main document as a draft:
%
\begin{center}
\begin{tabular}{l}
|\def\version{draft}|\\
|\input{childdoc.def}|\\
|\childdocforward{|\textit{main}|}|
\end{tabular}
\end{center}
%
Likewise, the following files |final|\textit{nn}|.tex|
compile the final version of the child document
|child|\textit{nn}|.tex|:
%
\begin{center}
\begin{tabular}{l}
|\def\version{final}|\\
|\input{childdoc.def}|\\
|\childdocforwardprefix{final}{child}|
\end{tabular}
\end{center}
%

Note that when several versions of a main file and/or of each child file
are to be generated, it may be convenient to set up a |Makefile| or
shell script to automatise the process.

%%%%%%%%%%%%%%%%%%%%%%%%%%%%%%%%%%%%%%%%%%%%%%%%%%%%%%%%%%%%%%%%%%%%%%%%%%%%%%%%
\subsection{Command Line Processing}
\label{sec:commandline}

The effect of redirection files can also be achieved by invoking
the \LaTeX{} compiler with a more elaborate command line.
Most conveniently this should be done as part
of a shell script or a |Makefile|.

When using \textsf{childdoc} in the main file, the following
command lines effectively perform a redirection
(note that depending on the shell being used,
backslashes may have to be doubled: `|\|' $\to$ `|\\|'):
%
\begin{center}
|... -jobname "|\textit{target}|" |\\|"|[\textit{flags}]%
|\input{childdoc.def}\childdocforward[|\textit{main}|]{|\textit{dest}|}"|
\end{center}
%
Here \textit{target} is the name of the output file,
\textit{main} is the name of the main file
and \textit{dest} is the name of the main or child file to be processed
(all filenames without extensions).
The optional argument \textit{main} can be omitted
if \textit{main} matches \textit{dest}.
Optionally, compilation \textit{flags} can be defined via |\def| commands.
This command line makes the \TeX{} engine believe
it is compiling the file \textit{target}
whose content is specified as the latter parameter.
The provided code then forwards the processing to
\textit{main} or \textit{dest} as described in \secref{sec:forward}.

%%%%%%%%%%%%%%%%%%%%%%%%%%%%%%%%%%%%%%%%%%%%%%%%%%%%%%%%%%%%%%%%%%%%%%%%%%%%%%%%
\subsection{Include by Input}
\label{sec:input}

Including child documents by |\include| has some restrictions by design.
Most notably, the content of a child document always occupies
its own set of pages; pages cannot be shared between child documents.
Usually, this behaviour makes perfect sense
because each child document contain an essential part of the document.
However, in some situations it may be desirable to compose
a document from a collection of parts
without having mandatory page breaks between then.
For this case, the package
provides a mechanism to include parts
by |\input| which can also be processed individually.
However, by construction this mechanism
requires manual handling of the content to be output.

%%%%%%%%%%%%%%%%%%%%%%%%%%%%%%%%%%%%%%%%
\DescribeMacro{\ifchilddocmanual}
The main file should be prepared as usual, see \secref{sec:include}.
However, the document body must make a distinction
between processing of an individual part and of the main document, e.g.:
%
\begin{center}
\begin{tabular}{l}
|\ifchilddocmanual|\\
|\input{\childdocname}|\\
|\||else|\\
\textit{document body with }|\input{|\textit{part}|}|\\
|\||fi|
\end{tabular}
\end{center}
%
The conditional |\ifchilddocmanual| is true whenever
a part to be included by |\input| is being compiled,
and the name of the part is stored in |\childdocname|.

%%%%%%%%%%%%%%%%%%%%%%%%%%%%%%%%%%%%%%%%
\DescribeMacro{\childdocby}
Each part to be included by |\input| should start with:
%
\begin{center}
\begin{tabular}{l}
|\input{childdoc.def}|\\
|\childdocby{|\textit{main}|}|\\
\end{tabular}
\end{center}
%
The directive |\childdocby| is similar to |\childdocof|
described in \secref{sec:include},
but the subsequent selection of content must be done manually.
To that end, both |\ifchilddoc| and |\ifchilddocmanual|
will be true upon processing of a part,
and the name of the part is stored in |\childdocname|.
Note that |\jobname| will be set to the filename of the current part
so that each part receives an individual |.aux| file
that does not interfere with the |.aux| file(s) of the main document.
This behaviour can be altered by the alternative form
|\childdocby[*]{|\textit{main}|}| (with a non-empty optional argument)
which uses the |.aux| file of the main document
by setting |\jobname| to \textit{main}.

%%%%%%%%%%%%%%%%%%%%%%%%%%%%%%%%%%%%%%%%%%%%%%%%%%%%%%%%%%%%%%%%%%%%%%%%%%%%%%%%
\subsection{Driver Development}
\label{sec:driver}

The \textsf{childdoc} mechanism can also be use for the development
of definition files such as \LaTeX{} styles or classes.
This case differs from the above setup with multiple parts
included by |\include| in that no |\includeonly| should be invoked.
This can be achieved by starting the include file
(before |\ProvidesPackage|) with:
%
\begin{center}
\begin{tabular}{l}
|\input{childdoc.def}|\\
|\childdocforward{|\textit{main}|}|\\
\end{tabular}
\end{center}
%
or alternatively with:
%
\begin{center}
\begin{tabular}{l}
|\input{childdoc.def}|\\
|\childdocby{|\textit{main}|}|\\
\end{tabular}
\end{center}
%
Both forms have slightly different effects as described above.
The main file is prepared as usual, see \secref{sec:include}.

%%%%%%%%%%%%%%%%%%%%%%%%%%%%%%%%%%%%%%%%%%%%%%%%%%%%%%%%%%%%%%%%%%%%%%%%%%%%%%%%
\subsection{Legacy Detection}
\label{sec:detection}

The directive |\childdocmain| in the main file can detect
whether the complete document or merely a child is to be compiled
even without using the directive |\childdocof|.
This method is deprecated because it is less robust
and there is no compelling reason to use it;
it is merely provided for backward compatibility
and it may be removed in future versions.

If the detection mechanism is to be used,
it is mandatory to correctly specify
the filename of the main file as the argument of |\childdocmain|:
%
\begin{center}
\begin{tabular}{l}
|\input{childdoc.def}|\\
|\childdocmain{|\textit{main}|}|\\
\end{tabular}
\end{center}
%
If |\jobname| does not match the argument \textit{main} of |\childdocmain|,
it is assumed that |\jobname| points to the child file to be compiled.
When using |\childdocmain| with the main file specified as argument,
it suffices to start a child file
with just |\input{|\textit{main}|}|
without loading of the package and using |\childdocof|.
If instead all processing is done
with the appropriate \textsf{childdoc} directives,
the argument of \textit{main} of |\childdocmain| can be empty.

An alternative version of the command line processing described
in \secref{sec:commandline} using the detection mechanism reads:
%
\begin{center}
|... -jobname "|\textit{target}|" "|[\textit{flags}]%
[|\def\jobname{|\textit{dest}|}|]|\input{|\textit{main}|}"|
\end{center}

%%%%%%%%%%%%%%%%%%%%%%%%%%%%%%%%%%%%%%%%%%%%%%%%%%%%%%%%%%%%%%%%%%%%%%%%%%%%%%%%
\subsection{Manual Code}
\label{sec:manual}

In case one cannot be certain whether the definitions file |childdoc.def|
is installed on the target \TeX{} distribution
and one prefers not to ship it,
it is conceivable to paste a few relevant commands into the sources.

To that end, drop all statements |\input{childdoc.def}|
and perform the replacements as outlined below.
Instead of |\childdocmain{|\textit{main}|}| add the following code
to the top of the main file:
%
\begin{center}
\begin{tabular}{l}
|\||ifdefined\childdocname\endinput\||fi\newif\ifchilddoc|\\
|\edef\childdocname{\scantokens\expandafter{\jobname\noexpand}}|\\
|\def\childdocmain{|\textit{main}|}\||ifx\childdocmain\childdocname\||else|\\
|\childdoctrue\includeonly{\childdocname}\let\jobname\childdocmain\||fi|\\
\end{tabular}
\end{center}
%
Instead of |\childdocof{|\textit{main}|}| just include the main file
at the top of each child file:
%
\begin{center}
|\input{|\textit{main}|}|
\end{center}
%
A simple redirection |\childdocforward{|\textit{dest}|}| is achieved by:
%
\begin{center}
|\def\jobname{|\textit{dest}|}\input{\jobname}|
\end{center}
%
The redirection with prefix
|\childdocforwardprefix[|\textit{prefix}|]{|\textit{dest}|}|
is accomplished by:
%
\begin{center}
\begin{tabular}{l}
|{\edef\jobname{\scantokens\expandafter{\jobname\noexpand}}|\\
|\def\redirectjob |\textit{prefix}|#1~~~{\gdef\jobname{|\textit{dest}|#1}}|\\
|\expandafter\redirectjob\jobname~~~}\input{\jobname}|
\end{tabular}
\end{center}

In an alternative approach,
child documents can be compiled by a specific command line
without additional code or specific definitions:
%
\begin{center}
|... -jobname "|\textit{target}|" "|[\textit{flags}]%
|\includeonly{|\textit{dest}|}\input{|\textit{main}|}"|
\end{center}
%

%%%%%%%%%%%%%%%%%%%%%%%%%%%%%%%%%%%%%%%%%%%%%%%%%%%%%%%%%%%%%%%%%%%%%%%%%%%%%%%%
%%%%%%%%%%%%%%%%%%%%%%%%%%%%%%%%%%%%%%%%%%%%%%%%%%%%%%%%%%%%%%%%%%%%%%%%%%%%%%%%
\section{Information}

%%%%%%%%%%%%%%%%%%%%%%%%%%%%%%%%%%%%%%%%%%%%%%%%%%%%%%%%%%%%%%%%%%%%%%%%%%%%%%%%
\subsection{Copyright}

Copyright \copyright{} 2017--2018 Niklas Beisert

This work may be distributed and/or modified under the
conditions of the \LaTeX{} Project Public License, either version 1.3
of this license or (at your option) any later version.
The latest version of this license is in
  \url{http://www.latex-project.org/lppl.txt}
and version 1.3 or later is part of all distributions of \LaTeX{}
version 2005/12/01 or later.

This work has the LPPL maintenance status `maintained'.

The Current Maintainer of this work is Niklas Beisert.

This work consists of the files |README.txt|, |childdoc.ins| and |childdoc.dtx|
as well as the derived files |childdoc.def|, |cdocsamp.tex|
with |cdocsch1.tex|, |cdocsch2.tex|, |cdocspt3.tex|, |cdocspt4.tex|,
|cdocsdrf.tex|, |cdocsfn1.tex|, |cdocsfn2.tex|
as well as |childdoc.pdf|.

%%%%%%%%%%%%%%%%%%%%%%%%%%%%%%%%%%%%%%%%%%%%%%%%%%%%%%%%%%%%%%%%%%%%%%%%%%%%%%%%
\subsection{Files and Installation}

The package consists of the files:
%
\begin{center}
\begin{tabular}{ll}
    |README.txt|   & readme file \\
    |childdoc.ins| & installation file \\
    |childdoc.dtx| & source file \\
    |childdoc.def| & definition file \\
    |cdocsamp.tex| & sample main file \\
    |cdocsch1.tex| & sample include file \\
    |cdocsch2.tex| & sample include file \\
    |cdocspt3.tex| & sample part file \\
    |cdocspt4.tex| & sample part file \\
    |cdocsdrf.tex| & sample redirection file \\
    |cdocsfn1.tex| & sample redirection file \\
    |cdocsfn2.tex| & sample redirection file \\
    |childdoc.pdf| & manual
\end{tabular}
\end{center}
%
The distribution consists of the files
|README.txt|, |childdoc.ins| and |childdoc.dtx|.
%
\begin{itemize}
\item
Run (pdf)\LaTeX{} on |childdoc.dtx|
to compile the manual |childdoc.pdf| (this file).
\item
Run \LaTeX{} on |childdoc.ins| to create the definitions file |childdoc.def|
and the sample |cdocsamp.tex| with include files
|cdocsch1.tex|, |cdocsch2.tex|, |cdocspt3.tex|, |cdocspt4.tex|,
|cdocsdrf.tex|, |cdocsfn1.tex|, |cdocsfn2.tex|.
Then copy the file |childdoc.def| to an appropriate directory of your \LaTeX{}
distribution, e.g.\ \textit{texmf-root}|/tex/latex/childdoc|.
\end{itemize}

%%%%%%%%%%%%%%%%%%%%%%%%%%%%%%%%%%%%%%%%%%%%%%%%%%%%%%%%%%%%%%%%%%%%%%%%%%%%%%%%
\subsection{Related CTAN Packages}

There are several other packages which offer a similar functionality:
%
\begin{itemize}
\item
The packages
\href{http://ctan.org/pkg/docmute}{\textsf{docmute}},
\href{http://ctan.org/pkg/includex}{\textsf{includex}} and
\href{http://ctan.org/pkg/standalone}{\textsf{standalone}}
provide commands to include only the document body of
a child file thus allowing both files to be compiled individually.
\item
The packages \href{http://ctan.org/pkg/subdocs}{\textsf{subdocs}}
and \href{http://ctan.org/pkg/subfiles}{\textsf{subfiles}}
provide structures in which the main and child documents can be
encapsulated and allowing them to be compiled individually.
The inclusion mechanism is different from the conventional |\include|.
\item
The package \href{http://ctan.org/pkg/combine}{\textsf{combine}}
is an elaborate solution to combine several documents into one.
\end{itemize}
%
See also the CTAN topic \href{http://ctan.org/topic/subdocs}{\textsf{subdocs}}
for further related packages.
The present package differs from the above solutions in that
a document structure constructed with the conventional |\include| mechanism
just needs two extra commands at the top of every file
such that all constituent files can be compiled individually.

%%%%%%%%%%%%%%%%%%%%%%%%%%%%%%%%%%%%%%%%%%%%%%%%%%%%%%%%%%%%%%%%%%%%%%%%%%%%%%%%
%\subsection{Feature Suggestions}
%
%The following is a list of features which may be useful for future
%versions of this package:
%%
%\begin{itemize}
%\item
%\ldots
%\end{itemize}

%%%%%%%%%%%%%%%%%%%%%%%%%%%%%%%%%%%%%%%%%%%%%%%%%%%%%%%%%%%%%%%%%%%%%%%%%%%%%%%%
\subsection{Revision History}

%%%%%%%%%%%%%%%%%%%%%%%%%%%%%%%%%%%%%%%%
\paragraph{v2.0:} 2018/12/30

\begin{itemize}
\item
immediate forward processing
\item
added |\childdocby| mechanism
\item
manual restructured
\end{itemize}

%%%%%%%%%%%%%%%%%%%%%%%%%%%%%%%%%%%%%%%%
\paragraph{v1.6:} 2018/01/17

\begin{itemize}
\item
application for development of include files
\item
corrections to manual
\end{itemize}

%%%%%%%%%%%%%%%%%%%%%%%%%%%%%%%%%%%%%%%%
\paragraph{v1.5:} 2017/05/21

\begin{itemize}
\item
more complete structuring introduced
\item
|\childdocof| introduced
\item
|\childdoc| renamed to |\childdocmain|
\item
|\childredirect| renamed to |\childdocforward| and |\childdocforwardprefix|
and functionality expanded
\end{itemize}

%%%%%%%%%%%%%%%%%%%%%%%%%%%%%%%%%%%%%%%%
\paragraph{v1.0:} 2017/04/27

\begin{itemize}
\item
manual and install package
\item
first version published on CTAN
\end{itemize}

%%%%%%%%%%%%%%%%%%%%%%%%%%%%%%%%%%%%%%%%
\paragraph{v0.6:} 2017/04/26

\begin{itemize}
\item
redirection mechanism added
\end{itemize}

%%%%%%%%%%%%%%%%%%%%%%%%%%%%%%%%%%%%%%%%
\paragraph{v0.5:} 2017/04/26

\begin{itemize}
\item
functionality in definition file
\end{itemize}


%%%%%%%%%%%%%%%%%%%%%%%%%%%%%%%%%%%%%%%%%%%%%%%%%%%%%%%%%%%%%%%%%%%%%%%%%%%%%%%%
%%%%%%%%%%%%%%%%%%%%%%%%%%%%%%%%%%%%%%%%%%%%%%%%%%%%%%%%%%%%%%%%%%%%%%%%%%%%%%%%
%%%%%%%%%%%%%%%%%%%%%%%%%%%%%%%%%%%%%%%%%%%%%%%%%%%%%%%%%%%%%%%%%%%%%%%%%%%%%%%%
\appendix

\settowidth\MacroIndent{\rmfamily\scriptsize 000\ }

 \DocInput{childdoc.dtx}

\end{document}
%</driver>
% \fi
%
% %%%%%%%%%%%%%%%%%%%%%%%%%%%%%%%%%%%%%%%%%%%%%%%%%%%%%%%%%%%%%%%%%%%%%%%%%%%%%%
% %%%%%%%%%%%%%%%%%%%%%%%%%%%%%%%%%%%%%%%%%%%%%%%%%%%%%%%%%%%%%%%%%%%%%%%%%%%%%%
% \section{Sample}
%\iffalse
%<*samplemain>
%\fi
%
% The following presents a sample document
% with two chapters, two parts, a title page,
% a compile flag as well as three forwarding files to set the flag.
% It consists of eight |.tex| files:
% \begin{center}
% \begin{tabular}{ll}
% |cdocsamp.tex|&main file\\
% |cdocsch1.tex|&include file for chapter 1\\
% |cdocsch2.tex|&include file for chapter 2\\
% |cdocspt3.tex|&include file for part 3\\
% |cdocspt4.tex|&include file for part 4\\
% |cdocsdrf.tex|&forwarding file for main file in draft mode\\
% |cdocsfi1.tex|&forwarding file for final version of chapter 1\\
% |cdocsfi2.tex|&forwarding file for final version of chapter 2\\
% \end{tabular}
% \end{center}
% Each of the eight files can be compiled directly by the \LaTeX{} compiler.
%
% %%%%%%%%%%%%%%%%%%%%%%%%%%%%%%%%%%%%%%
% \paragraph{Main File.}
%
% The main file is called |cdocsamp.tex|.
%
% Load the \textsf{childdoc} definitions and
% declare the filename for the main document:
%    \begin{macrocode}
\input{childdoc.def}
\childdocmain{}
%    \end{macrocode}

% Optional override for |\version| flag:
%    \begin{macrocode}
%%\ifchilddoc\else\providecommand{\version}{draft}\fi
%    \end{macrocode}

% Define the default values for the |\version| flag
% (|final| for the main file and |draft| for childs):
%    \begin{macrocode}
\ifchilddoc
\providecommand{\version}{draft}
\else
\providecommand{\version}{final}
\fi
%    \end{macrocode}

% Load the standard document class:
%    \begin{macrocode}
\documentclass[12pt]{article}
%    \end{macrocode}

% Start the document body:
%    \begin{macrocode}
\begin{document}
%    \end{macrocode}

% Declare a title page.
% Print title, part of document being processed and version flag:
%    \begin{macrocode}
\addtocounter{page}{-1}
\begin{center}
{\LARGE\bfseries{}childdoc example\par}
\vspace{1cm}
\ifchilddoc
\ifchilddocmanual part\else chapter\fi:
`\childdocname' of `\childdocjob'\par
\else
main document: `\childdocjob'\par
\fi
version: \version\par
\end{center}
\newpage
%    \end{macrocode}

% Manually include selected file,
% otherwise process as usual:
%    \begin{macrocode}
\ifchilddocmanual
\section*{part `\childdocname'}
\input{\childdocname}
\else
%    \end{macrocode}

% Include the two chapters:
%    \begin{macrocode}
\include{cdocsch1}
\include{cdocsch2}
%    \end{macrocode}

% Include the two parts unless only chapters should be displayed:
%    \begin{macrocode}
\ifchilddoc\else
\section{part three}
\input{cdocspt3}
\section{part four}
\input{cdocspt4}
\fi
%    \end{macrocode}

% Process as usual until here:
%    \begin{macrocode}
\fi
%    \end{macrocode}

% End of document body:
%    \begin{macrocode}
\end{document}
%    \end{macrocode}
%\iffalse
%</samplemain>
%\fi
%
% %%%%%%%%%%%%%%%%%%%%%%%%%%%%%%%%%%%%%%
% \paragraph{Chapter Include Files.}
%
% The include files are called |cdocsch1.tex| and |cdocsch2.tex|.
%
%\iffalse
%<*samplechap1|samplechap2>
%\fi

% Optional override for |\version| flag:
%    \begin{macrocode}
%%\providecommand{\version}{final}
%    \end{macrocode}

% Include the main document:
%    \begin{macrocode}
\input{childdoc.def}
\childdocof{cdocsamp}
%    \end{macrocode}

%\iffalse
%</samplechap1|samplechap2>
%\fi
%
%\iffalse
%<*samplechap1>
%\fi
% Some text for chapter 1:
%    \begin{macrocode}
\section{one}
some text in chapter one
%    \end{macrocode}

%\iffalse
%</samplechap1>
%\fi
% Some text for chapter 2:
%\iffalse
%<*samplechap2>
%\fi
%    \begin{macrocode}
\section{two}
more text in chapter two
%    \end{macrocode}

%\iffalse
%</samplechap2>
%\fi
%
% %%%%%%%%%%%%%%%%%%%%%%%%%%%%%%%%%%%%%%
% \paragraph{Part Include Files.}
%
% The include files are called |cdocspt3.tex| and |cdocspt4.tex|.
%
%\iffalse
%<*samplepart3|samplepart4>
%\fi

% Optional override for |\version| flag:
%    \begin{macrocode}
%%\providecommand{\version}{final}
%    \end{macrocode}

% Include the main document:
%    \begin{macrocode}
\input{childdoc.def}
\childdocby{cdocsamp}
%    \end{macrocode}

%\iffalse
%</samplepart3|samplepart4>
%\fi
%
%\iffalse
%<*samplepart3>
%\fi
% Some text for part 3:
%    \begin{macrocode}
some text in part three
%    \end{macrocode}

%\iffalse
%</samplepart3>
%\fi
% Some text for part 4:
%\iffalse
%<*samplepart4>
%\fi
%    \begin{macrocode}
more text in part four
%    \end{macrocode}

%\iffalse
%</samplepart4>
%\fi
%
% %%%%%%%%%%%%%%%%%%%%%%%%%%%%%%%%%%%%%%
% \paragraph{Forwarding for a Complete Draft.}
%
% The following forwarding file |cdocsdrf.tex|
% compiles the main document in draft mode:
%\iffalse
%<*sampledraft>
%\fi
%    \begin{macrocode}
\def\version{draft}
\input{childdoc.def}
\childdocforward{cdocsamp}
%    \end{macrocode}

%\iffalse
%</sampledraft>
%\fi
%
% %%%%%%%%%%%%%%%%%%%%%%%%%%%%%%%%%%%%%%
% \paragraph{Forwarding for Final Version of the Chapters.}
%
% The following forwarding files |cdocsfn1.tex| and |cdocsfn2.tex|
% (with identical content)
% compile the final versions of the child documents
% |cdocsch1.tex| and |cdocsch2.tex|, respectively:
%\iffalse
%<*samplefinal>
%\fi
%    \begin{macrocode}
\def\version{final}
\input{childdoc.def}
\childdocforwardprefix[cdocsamp]{cdocsfn}{cdocsch}
%    \end{macrocode}

%\iffalse
%</samplefinal>
%\fi
%
% %%%%%%%%%%%%%%%%%%%%%%%%%%%%%%%%%%%%%%
% \paragraph{Command Line Processing.}
%
% The following three command lines generate the output files
% |cdocscld|, |cdocscl1| and |cdocscl2|
% which should be identical to
% |cdocsdrf|, |cdocsch1| and |cdocsfn2|, respectively:
% \begin{center}
% \begin{tabular}{l}
% |latex -jobname cdocscld \|\\
% |  "\def\version{draft}\input{childdoc.def}\childdocforward{cdocsamp}"|\\
% |latex -jobname cdocscl1 \|\\
% |  "\input{childdoc.def}\childdocforward[cdocsamp]{cdocsch1}"|\\
% |latex -jobname cdocscl2 \|\\
% |  "\def\version{final}\input{childdoc.def}\childdocforward{cdocsch2}"|
% \end{tabular}
% \end{center}
% Note that the trailing backslash on each first line
% merely continues the input to the second line
% (for convenient cut ant paste).
% Furthermore, the command |latex| can be replaced by any
% of its alternative versions such as |pdflatex|.
%
% %%%%%%%%%%%%%%%%%%%%%%%%%%%%%%%%%%%%%%%%%%%%%%%%%%%%%%%%%%%%%%%%%%%%%%%%%%%%%%
% %%%%%%%%%%%%%%%%%%%%%%%%%%%%%%%%%%%%%%%%%%%%%%%%%%%%%%%%%%%%%%%%%%%%%%%%%%%%%%
% \section{Implementation}
%\iffalse
%<*package>
%\fi
%
% This section describes the definitions file |childdoc.def|.

% The definitions cannot be loaded using |\usepackage| or |\RequirePackage|
% which has a mechanism to prevent loading a style file more than once.
% When loading the definitions by means of |\input|
% multiple instances have to be prevented manually:
%\iffalse
%This code needs to be before the `\ProvidesFile' directive
%which is defined at the beginning of this file.
%Therefore it is also placed there and commented out here.
%</package>
%<*discard>
%\fi
%    \begin{macrocode}
\ifdefined\childdocmain\endinput\fi
%    \end{macrocode}
%\iffalse
%</discard>
%<*package>
%\fi
%
% \macro{\ifchilddoc}
% \macro{\ifchilddocmanual}
% The conditional |\ifchilddoc| tells whether a
% child (true) or main (false) document is being compiled.
% The conditional |\ifchilddocmanual| tells whether
% the |\includeonly| mechanism is used (false) or
% the selection of child files must be performed manually (true).
% The definitions initialise to false:
%    \begin{macrocode}
\newif\ifchilddoc
\newif\ifchilddocmanual
%    \end{macrocode}

% \macro{\childdocname}
% \macro{\childdocjob}
% The macro |\childdocname| stores the name of the main document
% to be compiled. The macro |\childdocjob| stores the name of
% the document on which the \LaTeX{} compiler was originally invoked.
% The content of |\jobname| cannot be compared
% to filenames specified in the source due to different catcodes.
% The following code rescans |\jobname|, stores the result
% in |\childdocname| and saves a copy in |\childdocjob|:
%    \begin{macrocode}
\edef\childdocname{\scantokens\expandafter{\jobname\noexpand}}
\let\childdocjob\childdocname
%    \end{macrocode}

% \macro{\childdocdisable}
% The macro |\childdocdisable| prevents the main file
% from being processed more than once.
% At this stage, the main document command |\childdocmain|
% is assumed to be called once again where it should do nothing.
% Any subsequent call to it should prevent
% a secondary processing of the main document
% It overwrites the forwarding commands
% |\childdocof| and |\childdocforward|
% with empty macros to prevent further inclusions of the main document:
%    \begin{macrocode}
\newcommand{\childdocdisable}
{
  \renewcommand{\childdocmain}[1]{\renewcommand{\childdocmain}[1]{\endinput}}
  \renewcommand{\childdocof}[1]{}
  \renewcommand{\childdocby}[2][]{}
  \renewcommand{\childdocforward}[2][]{}
  \renewcommand{\childdocdisable}{}
}
%    \end{macrocode}

% \macro{\childdocmain}
% The macro |\childdocmain| is to be called at the top of the main file
% with nothing or the main filename (without extension) as argument.
% First, it breaks loops.
% If the argument is not empty and does not match |\childdocname|
% (which is set by the first inclusion of |childdoc.def|),
% |\ifchilddoc| is set to true, |\includeonly| is applied to the child file
% and |\jobname| is set to the main file
% (for proper handling of |.aux| files):
%    \begin{macrocode}
\newcommand{\childdocmain}[1]
{
  \childdocdisable\childdocmain{}
  \if?#1?\else
    \begingroup
      \def\childdoctmp{#1}
      \ifx\childdoctmp\childdocname
        \def\childdoctmp{}
      \else
        \def\childdoctmp
        {
          \childdoctrue
          \includeonly{\childdocname}
          \def\childdocjob{#1}
          \def\jobname{#1}
        }
      \fi
      \expandafter
    \endgroup
    \childdoctmp
  \fi
}
%    \end{macrocode}

% \macro{\childdocof}
% The command |\childdocof| redirects
% compilation to the main file |#1|.
%    \begin{macrocode}
\newcommand{\childdocof}[1]
{
  \childdocdisable
  \childdoctrue
  \includeonly{\childdocname}
  \def\jobname{#1}
  \def\childdocjob{#1}
  \input{#1}
}
%    \end{macrocode}

% \macro{\childdocby}
% The command |\childdocby| ....
%    \begin{macrocode}
\newcommand{\childdocby}[2][]
{
  \childdocdisable
  \childdoctrue
  \childdocmanualtrue
  \if?#1?\else
    \def\jobname{#2}
  \fi
  \def\childdocjob{#2}
  \input{#2}
  \endinput
}
%    \end{macrocode}

% \macro{\childdocforward}
% The command |\childdocforward| redirects
% compilation to the main file or
% (if the optional argument is given) a child file.
% Parameters are set as if the main file
% or a child file starting with |\childdocof| was compiled.
% Then compilation is handed over to the main file:
%    \begin{macrocode}
\newcommand{\childdocforward}[2][]
{
  \begingroup
    \if?#1?
      \def\childdoctmp
      {
        \def\childdocname{#2}
        \def\childdocjob{#2}
        \def\jobname{#2}
        \input{#2}
        \endinput
      }
    \else
      \def\childdoctmp
      {
        \childdocdisable
        \def\childdocname{#2}
        \childdoctrue
        \includeonly{#2}
        \def\childdocjob{#1}
        \def\jobname{#1}
        \input{#1}
        \endinput
      }
    \fi
    \expandafter
  \endgroup
  \childdoctmp
}
%    \end{macrocode}

% \macro{\childdocforwardprefix}
% The command |\childdocforwardprefix| redirects
% compilation to the main or a child file by means of a pattern.
% The prefix |#1| in the current filename is replaced by |#2|
% and the suffix of the current filename is kept
% (it is assumed that the filename does not contain the substring `|~~~|'
% which is used as a delimiter).
% Compilation is handed over to the new file by |\childdocforward|:
%    \begin{macrocode}
\newcommand{\childdocforwardprefix}[3][]
{
  \begingroup
    \def\childdocextract #2##1~~~{\def\childdoctmp{\childdocforward[#1]{#3##1}}}
    \expandafter\childdocextract\childdocname~~~
    \expandafter
  \endgroup
  \childdoctmp
}
%    \end{macrocode}

% \macro{\childdoc}
% The deprecated macro |\childdoc| is a legacy version of |\childdocmain|:
%    \begin{macrocode}
\newcommand{\childdoc}{\childdocmain}
%    \end{macrocode}

% \macro{\childdocredirect}
% The deprecated macro |\childdocredirect| is a legacy version
% of |\childdocforward| and |\childdocforwardprefix|:
%    \begin{macrocode}
\newcommand{\childdocredirect}[2][]
{
  \begingroup
    \if?#1?
      \def\childdoctmp{\childdocforward{#2}}
    \else
      \def\childdoctmp{\childdocforwardprefix{#1}{#2}}
    \fi
    \expandafter
  \endgroup
  \childdoctmp
}
%    \end{macrocode}

%\iffalse
%</package>
%\fi
%
\endinput

\childdocby{cdocsamp}
%    \end{macrocode}

%\iffalse
%</samplepart3|samplepart4>
%\fi
%
%\iffalse
%<*samplepart3>
%\fi
% Some text for part 3:
%    \begin{macrocode}
some text in part three
%    \end{macrocode}

%\iffalse
%</samplepart3>
%\fi
% Some text for part 4:
%\iffalse
%<*samplepart4>
%\fi
%    \begin{macrocode}
more text in part four
%    \end{macrocode}

%\iffalse
%</samplepart4>
%\fi
%
% %%%%%%%%%%%%%%%%%%%%%%%%%%%%%%%%%%%%%%
% \paragraph{Forwarding for a Complete Draft.}
%
% The following forwarding file |cdocsdrf.tex|
% compiles the main document in draft mode:
%\iffalse
%<*sampledraft>
%\fi
%    \begin{macrocode}
\def\version{draft}
% \iffalse
%
% childdoc.dtx Copyright (C) 2017-2018 Niklas Beisert
%
% This work may be distributed and/or modified under the
% conditions of the LaTeX Project Public License, either version 1.3
% of this license or (at your option) any later version.
% The latest version of this license is in
%   http://www.latex-project.org/lppl.txt
% and version 1.3 or later is part of all distributions of LaTeX
% version 2005/12/01 or later.
%
% This work has the LPPL maintenance status `maintained'.
%
% The Current Maintainer of this work is Niklas Beisert.
%
% This work consists of the files childdoc.dtx and childdoc.ins
% and the derived files childdoc.def and cdocsamp.tex with
% cdocsch1.tex, cdocsch2.tex, cdocsdrf.tex, cdocsfn1.tex, cdocsfn2.tex.
%
%<package>\ifdefined\childdocmain\endinput\fi
%<package>\ProvidesFile{childdoc.def}[2018/12/30 v2.0 child document driver]
%<samplemain>\ProvidesFile{cdocsamp.tex}[2018/12/30 v2.0 sample for childdoc]
%<*driver>
%\ProvidesFile{childdoc.drv}[2018/12/30 v2.0 childdoc reference manual file]
\PassOptionsToClass{10pt,a4paper}{article}
\documentclass{ltxdoc}

\usepackage[margin=35mm]{geometry}
\usepackage{hyperref}
\usepackage{hyperxmp}
\usepackage[usenames]{color}

\hypersetup{colorlinks=true}
\hypersetup{pdfstartview=FitH}
\hypersetup{pdfpagemode=UseNone}
\hypersetup{pdfsource={}}
\hypersetup{pdflang={en-UK}}
\hypersetup{pdfcopyright={Copyright 2017-2018 Niklas Beisert.
  This work may be distributed and/or modified under the
  conditions of the LaTeX Project Public License, either version 1.3
  of this license or (at your option) any later version.}}
\hypersetup{pdflicenseurl={http://www.latex-project.org/lppl.txt}}
\hypersetup{pdfcontactaddress={ETH Zurich, ITP, HIT K,
  Wolfgang-Pauli-Strasse 27}}
\hypersetup{pdfcontactpostcode={8093}}
\hypersetup{pdfcontactcity={Zurich}}
\hypersetup{pdfcontactcountry={Switzerland}}
\hypersetup{pdfcontactemail={nbeisert@itp.phys.ethz.ch}}
\hypersetup{pdfcontacturl={http://people.phys.ethz.ch/\xmptilde nbeisert/}}

\newcommand{\secref}[1]{\hyperref[#1]{section \ref*{#1}}}

\parskip1ex
\parindent0pt
\let\olditemize\itemize
\def\itemize{\olditemize\parskip0pt}

\begin{document}

\title{The \textsf{childdoc} Package}
\hypersetup{pdftitle={The childdoc Package}}
\author{Niklas Beisert\\[2ex]
  Institut f\"ur Theoretische Physik\\
  Eidgen\"ossische Technische Hochschule Z\"urich\\
  Wolfgang-Pauli-Strasse 27, 8093 Z\"urich, Switzerland\\[1ex]
  \href{mailto:nbeisert@itp.phys.ethz.ch}
  {\texttt{nbeisert@itp.phys.ethz.ch}}}
\hypersetup{pdfauthor={Niklas Beisert}}
\hypersetup{pdfsubject={Manual for the LaTeX2e Package childdoc}}
\date{30 December 2018, \textsf{v2.0}}
\maketitle

\begin{abstract}\noindent
\textsf{childdoc} is a \LaTeXe{} package
that enables the direct compilation
of document sections included by |\include|
to individual files.
\end{abstract}

\begingroup
\parskip0ex
\tableofcontents
\endgroup

%%%%%%%%%%%%%%%%%%%%%%%%%%%%%%%%%%%%%%%%%%%%%%%%%%%%%%%%%%%%%%%%%%%%%%%%%%%%%%%%
%%%%%%%%%%%%%%%%%%%%%%%%%%%%%%%%%%%%%%%%%%%%%%%%%%%%%%%%%%%%%%%%%%%%%%%%%%%%%%%%
\section{Introduction}

\LaTeX{} provides a mechanism to structure a large document (such as a book)
into a main file and several child files (containing the chapters)
using the |\include| command.
This mechanism is beneficial for documents
which span hundreds of pages in order to
make the source file(s) more manageable.
Moreover, compilation can be restricted to
selected child files by means of the |\includeonly| command.
The latter feature can be used to reduce the compilation time while editing
(this was significantly more useful in the earlier days of \LaTeX{})
or to generate a smaller document which is easier to navigate.
Another application of |\includeonly| is to generate
documents consisting of selected parts of the complete document.

However, there are a few drawbacks of the plain |\include| mechanism:
\begin{itemize}
\item
The child files cannot be compiled on their own,
they can only be compiled via the main file.
A naive editing environment
(such as a text editor with an option
to have the current file processed by \LaTeX)
may require one to switch to the main file before compiling;
attempting to compile the child file produces errors.
\item
The main file must be modified (each time)
to adjust the |\includeonly| command
to the present needs. This easily leaves the main file in a messy state.
\item
The generated document will always carry the filename
of the main document. This is inconvenient if
several child files are to be compiled and
to be kept for distribution.
\end{itemize}

The present package provides a simple interface
to make child files individually compilable by \LaTeX{}.
Compiling a child file then has the same effect as compiling
the main file with an |\includeonly| command
to select the appropriate child.
Moreover the generated document will carry the name of the child
rather than the main file.
This resolves all three above issues.

This feature is meant to make the editing of books,
thesis documents and lecture notes somewhat more convenient.
However, the package can also be used efficiently for
composing a series of documents (such as exercise sheets)
which are typically distributed individually.
It then assists the author in generating the individual documents
(potentially in different versions)
as well as a document containing the collected series.
Another application is in developing style files
or other kinds of included material
where compilation of the style file could redirect
to a sample or test file.

%%%%%%%%%%%%%%%%%%%%%%%%%%%%%%%%%%%%%%%%%%%%%%%%%%%%%%%%%%%%%%%%%%%%%%%%%%%%%%%%
%%%%%%%%%%%%%%%%%%%%%%%%%%%%%%%%%%%%%%%%%%%%%%%%%%%%%%%%%%%%%%%%%%%%%%%%%%%%%%%%
\section{Usage}

First of all, the package \textsf{childdoc} is \emph{not} a standard
\LaTeXe{} |.sty| style file! Therefore it needs to be invoked in
a non-standard way.

%%%%%%%%%%%%%%%%%%%%%%%%%%%%%%%%%%%%%%%%%%%%%%%%%%%%%%%%%%%%%%%%%%%%%%%%%%%%%%%%
\subsection{Included Files}
\label{sec:include}

%%%%%%%%%%%%%%%%%%%%%%%%%%%%%%%%%%%%%%%%
\DescribeMacro{\childdocmain}
To use the package, add the commands
\begin{center}
\begin{tabular}{l}
|\input{childdoc.def}|\\
|\childdocmain{}|\\
\end{tabular}
\end{center}
at the very top of the main \LaTeX{} file,
in particular \emph{before} the |\documentclass| statement!
The argument of |\childdocmain| should be left empty
(but it must be present).

%%%%%%%%%%%%%%%%%%%%%%%%%%%%%%%%%%%%%%%%
\DescribeMacro{\childdocof}
Furthermore, add the commands
\begin{center}
\begin{tabular}{l}
|\input{childdoc.def}|\\
|\childdocof{|\textit{main}|}|\\
\end{tabular}
\end{center}
at the top of every child file \textit{child}
which is included by |\include{|\textit{child}|}|
from within the main file
(or at least for those files to be compiled individually).
The argument \textit{main} must be the filename of the main file.

There are a couple of
considerations in setting up the main and child documents:

%%%%%%%%%%%%%%%%%%%%%%%%%%%%%%%%%%%%%%%%
\paragraph{Restrictions.}

Please note the following restrictions:
\begin{itemize}
\item
|\childdocmain| must be called with one argument \textit{main}
to ensure compatibility with earlier version of the package.
It must either be empty (|\childdocmain{}|)
or precisely match the filename of the main file in which it is specified.
See \secref{sec:detection} for further information.
\item
The filename \textit{main} must be specified without the |.tex| extension.
\item
The filename \textit{main} is case sensitive
(even in case-insensitive file systems)
due to internal string comparison.
\item
The argument \textit{main} should be fully expanded, it cannot be a macro.
\item
Subdirectories and special characters should be avoided in filenames.
\item
The command |\childdocmain{|\textit{main}|}| must be followed by a whitespace.
It should not be followed immediately by another command
or by a comment mark `|%|'.
This is because the \TeX{} parser reads the token immediately following
the argument of |\childdocmain| and puts it
at the beginning of every child section;
however, a white\-space is ignored.
\end{itemize}

%%%%%%%%%%%%%%%%%%%%%%%%%%%%%%%%%%%%%%%%
\paragraph{Content of Main File.}

It is advisable to place all content in the child files included by |\include|.
Any output contained in the main file will appear in all child documents
unless suppressed manually;
it cannot be suppressed automatically by the |\includeonly| directive
and thus should normally be avoided.
A method to include some content in the main file
by means of conditional processing is described in \secref{sec:conditional}.

%%%%%%%%%%%%%%%%%%%%%%%%%%%%%%%%%%%%%%%%
\paragraph{Page Numbering.}

When only a part of the document is compiled,
the appropriate numbering of pages
(as well as other status parameters)
is determined from the |.aux| files.
The latter contain information from previous passes.
However this information needs to propagate through
all intermediate child documents.
Therefore the page numbering in child documents may well
be inconsistent until the complete document is compiled at least once.

A useful (if unconventional) way to always ensure a consistent
page numbering is to restart the numbering in each child document
and denote the pages by `\textit{child}|.|\textit{page}'
where \textit{child} represents the chapter/section number of the child file.
This can be achieved by the command
|\numberwithin{page}{|\textit{child}|}|
of the \textsf{amsmath} package
where \textit{child} can be |chapter| or |section|
depending on the chosen structuring.
Alternatively, one can modify the macro |\thepage| appropriately
and reset the counter |page| at the start of each child file.

%%%%%%%%%%%%%%%%%%%%%%%%%%%%%%%%%%%%%%%%%%%%%%%%%%%%%%%%%%%%%%%%%%%%%%%%%%%%%%%%
\subsection{Conditional Processing}
\label{sec:conditional}

The package provides a mechanism to compile different versions
of a document. To customise the versions further some conditional processing
can come in handy to distinguish which version is being compiled.
The package provides two macros to describe the compilation context:

%%%%%%%%%%%%%%%%%%%%%%%%%%%%%%%%%%%%%%%%
\DescribeMacro{\ifchilddoc}
The conditional |\ifchilddoc| distinguishes between the compilation of
child documents and the main document:
%
\begin{center}
|\ifchilddoc |\textit{child-code}| |[|\||else |\textit{main-code}]| \||fi|
\end{center}

%%%%%%%%%%%%%%%%%%%%%%%%%%%%%%%%%%%%%%%%
\DescribeMacro{\childdocname}
\DescribeMacro{\childdocjob}
The macro |\childdocname| contains the filename (without extension)
of the main or child file being processed.
Note that |\childdocjob| will always contain the name of the main file.

%%%%%%%%%%%%%%%%%%%%%%%%%%%%%%%%%%%%%%%%
\paragraph{Title Page.}

Conditional processing can be used to include a title or banner page
in the main document when proper precautions are taken.
Importantly, the code in the main file should ensure that the page counter
(as well as other status parameters which are stored in the |.aux| files)
takes the same value after the conditional processing.
Otherwise the page numbers may take divergent values
depending on which part is compiled.

For example, a title page could be declared by:
%
\begin{center}
\begin{tabular}{l}
|\ifchilddoc\||else|\\
|\addtocounter{page}{-1}|\\
\textit{code for title page}\\
|\newpage|\\
|\||fi|
\end{tabular}
\end{center}
%
A banner page for the child documents can be generated by:
%
\begin{center}
\begin{tabular}{l}
|\ifchilddoc|\\
|\addtocounter{page}{-1}|\\
\textit{code for banner page}\\
|\newpage|\\
|\||fi|
\end{tabular}
\end{center}
%
Here one could write a message such as:
\begin{center}
|This is the part \childdocname{} of \childdocjob{}.|
\end{center}

%%%%%%%%%%%%%%%%%%%%%%%%%%%%%%%%%%%%%%%%%%%%%%%%%%%%%%%%%%%%%%%%%%%%%%%%%%%%%%%%
\subsection{Flags}
\label{sec:flags}

The package makes it easy to generate different versions
of the main or child documents.
To this end compilation flags can be defined
and assigned different default values.
They will be particularly useful in conjunction
with the forwarding mechanism described in \secref{sec:forward}.

For example, it may be useful to have a flag |\version|
which can be set to |draft| or |final|.
The document source will contain some conditional code
depending on the value of |\version|.
Suppose further, the flag should default to |final| for the main file
and to |draft| for child files
which is a natural assignment for editing the document.
This is achieved by placing the following code
in the preamble of the main document
(below the |\childdocmain| directive):
%
\begin{center}
\begin{tabular}{l}
|\ifchilddoc|\\
|\providecommand{\version}{draft}|\\
|\||else|\\
|\providecommand{\version}{final}|\\
|\||fi|
\end{tabular}
\end{center}
%
The definition by |\providecommand| makes sure
that previous definitions are not overwritten.
Further statements |\providecommand{\version}{...}|
can thus be added before the above code to override it.

For the main file, one might add a line
(between |\childdocmain| and the above block)
%
\begin{center}
|%\ifchilddoc\||else\providecommand{\version}{draft}\||fi|
\end{center}
%
which can be uncommented to produce a draft version.
Likewise one can add a line to the very top of a child file
(above the |\childdocof{|\textit{main}|}| directive)
%
\begin{center}
|%\providecommand{\version}{final}|
\end{center}
%
which can be uncommented to produce the final version of this child document.

%%%%%%%%%%%%%%%%%%%%%%%%%%%%%%%%%%%%%%%%%%%%%%%%%%%%%%%%%%%%%%%%%%%%%%%%%%%%%%%%
\subsection{Forwarding}
\label{sec:forward}

Different versions of the main or child documents
using compilation flags as described in \secref{sec:flags}
can be (permanently) stored in different files
for convenient compilation, viewing and distribution.
To this end, the package defines a command
to pass on compilation to a different file:

%%%%%%%%%%%%%%%%%%%%%%%%%%%%%%%%%%%%%%%%
\DescribeMacro{\childdocforward}
The command |\childdocforward| redirects processing to
another source file:
%
\begin{center}
\begin{tabular}{l}
|\input{childdoc.def}|\\
|\childdocforward[|\textit{main}|]{|\textit{dest}|}|\\
\end{tabular}
\end{center}
%
The argument \textit{dest} is the destination file
(without extension).
It should be the main file or one of the child files.
Note that further \textsf{childdoc} directives
such as |\childdocof| and |\childdocforward|
in the indicated file will be processed in this form.
The optional argument \textit{main}
passes on directly to the main file \textit{main}
while pretending to compile the child \textit{dest}.
This form behaves as if \textit{dest}
issues |\childdocof{|\textit{main}|}| right away,
and no further \textsf{childdoc} directives will be processed.

%%%%%%%%%%%%%%%%%%%%%%%%%%%%%%%%%%%%%%%%
\DescribeMacro{\...prefix}
In the alternative form |\childdocforwardprefix|,
%
\begin{center}
\begin{tabular}{l}
|\input{childdoc.def}|\\
|\childdocforwardprefix[|\textit{main}|]{|\textit{prefix}|}{|\textit{dest}|}|
\end{tabular}
\end{center}
%
the destination file is determined by a pattern
depending on the current file:
To make this work, the current file must be called
`{\textit{prefix}\hspace{0.2em}\textit{suffix}}'
with \textit{prefix} matching precisely the argument.
Processing is then passed on to the file
`{\textit{dest}\hspace{0.2em}\textit{suffix}}'.
Surely, the same effect is achieved by
directly specifying the
argument `{\textit{dest}\hspace{0.2em}\textit{suffix}}'
in the first form.
However, that requires to set up a different file
for each child. With the alternative form of the command
all these files can have exactly the same content
which simplifies setting them up and maintaining them.

For example, the following file |draft.tex|
with a compilation flag |\version| as described in \secref{sec:flags}
compiles the main document as a draft:
%
\begin{center}
\begin{tabular}{l}
|\def\version{draft}|\\
|\input{childdoc.def}|\\
|\childdocforward{|\textit{main}|}|
\end{tabular}
\end{center}
%
Likewise, the following files |final|\textit{nn}|.tex|
compile the final version of the child document
|child|\textit{nn}|.tex|:
%
\begin{center}
\begin{tabular}{l}
|\def\version{final}|\\
|\input{childdoc.def}|\\
|\childdocforwardprefix{final}{child}|
\end{tabular}
\end{center}
%

Note that when several versions of a main file and/or of each child file
are to be generated, it may be convenient to set up a |Makefile| or
shell script to automatise the process.

%%%%%%%%%%%%%%%%%%%%%%%%%%%%%%%%%%%%%%%%%%%%%%%%%%%%%%%%%%%%%%%%%%%%%%%%%%%%%%%%
\subsection{Command Line Processing}
\label{sec:commandline}

The effect of redirection files can also be achieved by invoking
the \LaTeX{} compiler with a more elaborate command line.
Most conveniently this should be done as part
of a shell script or a |Makefile|.

When using \textsf{childdoc} in the main file, the following
command lines effectively perform a redirection
(note that depending on the shell being used,
backslashes may have to be doubled: `|\|' $\to$ `|\\|'):
%
\begin{center}
|... -jobname "|\textit{target}|" |\\|"|[\textit{flags}]%
|\input{childdoc.def}\childdocforward[|\textit{main}|]{|\textit{dest}|}"|
\end{center}
%
Here \textit{target} is the name of the output file,
\textit{main} is the name of the main file
and \textit{dest} is the name of the main or child file to be processed
(all filenames without extensions).
The optional argument \textit{main} can be omitted
if \textit{main} matches \textit{dest}.
Optionally, compilation \textit{flags} can be defined via |\def| commands.
This command line makes the \TeX{} engine believe
it is compiling the file \textit{target}
whose content is specified as the latter parameter.
The provided code then forwards the processing to
\textit{main} or \textit{dest} as described in \secref{sec:forward}.

%%%%%%%%%%%%%%%%%%%%%%%%%%%%%%%%%%%%%%%%%%%%%%%%%%%%%%%%%%%%%%%%%%%%%%%%%%%%%%%%
\subsection{Include by Input}
\label{sec:input}

Including child documents by |\include| has some restrictions by design.
Most notably, the content of a child document always occupies
its own set of pages; pages cannot be shared between child documents.
Usually, this behaviour makes perfect sense
because each child document contain an essential part of the document.
However, in some situations it may be desirable to compose
a document from a collection of parts
without having mandatory page breaks between then.
For this case, the package
provides a mechanism to include parts
by |\input| which can also be processed individually.
However, by construction this mechanism
requires manual handling of the content to be output.

%%%%%%%%%%%%%%%%%%%%%%%%%%%%%%%%%%%%%%%%
\DescribeMacro{\ifchilddocmanual}
The main file should be prepared as usual, see \secref{sec:include}.
However, the document body must make a distinction
between processing of an individual part and of the main document, e.g.:
%
\begin{center}
\begin{tabular}{l}
|\ifchilddocmanual|\\
|\input{\childdocname}|\\
|\||else|\\
\textit{document body with }|\input{|\textit{part}|}|\\
|\||fi|
\end{tabular}
\end{center}
%
The conditional |\ifchilddocmanual| is true whenever
a part to be included by |\input| is being compiled,
and the name of the part is stored in |\childdocname|.

%%%%%%%%%%%%%%%%%%%%%%%%%%%%%%%%%%%%%%%%
\DescribeMacro{\childdocby}
Each part to be included by |\input| should start with:
%
\begin{center}
\begin{tabular}{l}
|\input{childdoc.def}|\\
|\childdocby{|\textit{main}|}|\\
\end{tabular}
\end{center}
%
The directive |\childdocby| is similar to |\childdocof|
described in \secref{sec:include},
but the subsequent selection of content must be done manually.
To that end, both |\ifchilddoc| and |\ifchilddocmanual|
will be true upon processing of a part,
and the name of the part is stored in |\childdocname|.
Note that |\jobname| will be set to the filename of the current part
so that each part receives an individual |.aux| file
that does not interfere with the |.aux| file(s) of the main document.
This behaviour can be altered by the alternative form
|\childdocby[*]{|\textit{main}|}| (with a non-empty optional argument)
which uses the |.aux| file of the main document
by setting |\jobname| to \textit{main}.

%%%%%%%%%%%%%%%%%%%%%%%%%%%%%%%%%%%%%%%%%%%%%%%%%%%%%%%%%%%%%%%%%%%%%%%%%%%%%%%%
\subsection{Driver Development}
\label{sec:driver}

The \textsf{childdoc} mechanism can also be use for the development
of definition files such as \LaTeX{} styles or classes.
This case differs from the above setup with multiple parts
included by |\include| in that no |\includeonly| should be invoked.
This can be achieved by starting the include file
(before |\ProvidesPackage|) with:
%
\begin{center}
\begin{tabular}{l}
|\input{childdoc.def}|\\
|\childdocforward{|\textit{main}|}|\\
\end{tabular}
\end{center}
%
or alternatively with:
%
\begin{center}
\begin{tabular}{l}
|\input{childdoc.def}|\\
|\childdocby{|\textit{main}|}|\\
\end{tabular}
\end{center}
%
Both forms have slightly different effects as described above.
The main file is prepared as usual, see \secref{sec:include}.

%%%%%%%%%%%%%%%%%%%%%%%%%%%%%%%%%%%%%%%%%%%%%%%%%%%%%%%%%%%%%%%%%%%%%%%%%%%%%%%%
\subsection{Legacy Detection}
\label{sec:detection}

The directive |\childdocmain| in the main file can detect
whether the complete document or merely a child is to be compiled
even without using the directive |\childdocof|.
This method is deprecated because it is less robust
and there is no compelling reason to use it;
it is merely provided for backward compatibility
and it may be removed in future versions.

If the detection mechanism is to be used,
it is mandatory to correctly specify
the filename of the main file as the argument of |\childdocmain|:
%
\begin{center}
\begin{tabular}{l}
|\input{childdoc.def}|\\
|\childdocmain{|\textit{main}|}|\\
\end{tabular}
\end{center}
%
If |\jobname| does not match the argument \textit{main} of |\childdocmain|,
it is assumed that |\jobname| points to the child file to be compiled.
When using |\childdocmain| with the main file specified as argument,
it suffices to start a child file
with just |\input{|\textit{main}|}|
without loading of the package and using |\childdocof|.
If instead all processing is done
with the appropriate \textsf{childdoc} directives,
the argument of \textit{main} of |\childdocmain| can be empty.

An alternative version of the command line processing described
in \secref{sec:commandline} using the detection mechanism reads:
%
\begin{center}
|... -jobname "|\textit{target}|" "|[\textit{flags}]%
[|\def\jobname{|\textit{dest}|}|]|\input{|\textit{main}|}"|
\end{center}

%%%%%%%%%%%%%%%%%%%%%%%%%%%%%%%%%%%%%%%%%%%%%%%%%%%%%%%%%%%%%%%%%%%%%%%%%%%%%%%%
\subsection{Manual Code}
\label{sec:manual}

In case one cannot be certain whether the definitions file |childdoc.def|
is installed on the target \TeX{} distribution
and one prefers not to ship it,
it is conceivable to paste a few relevant commands into the sources.

To that end, drop all statements |\input{childdoc.def}|
and perform the replacements as outlined below.
Instead of |\childdocmain{|\textit{main}|}| add the following code
to the top of the main file:
%
\begin{center}
\begin{tabular}{l}
|\||ifdefined\childdocname\endinput\||fi\newif\ifchilddoc|\\
|\edef\childdocname{\scantokens\expandafter{\jobname\noexpand}}|\\
|\def\childdocmain{|\textit{main}|}\||ifx\childdocmain\childdocname\||else|\\
|\childdoctrue\includeonly{\childdocname}\let\jobname\childdocmain\||fi|\\
\end{tabular}
\end{center}
%
Instead of |\childdocof{|\textit{main}|}| just include the main file
at the top of each child file:
%
\begin{center}
|\input{|\textit{main}|}|
\end{center}
%
A simple redirection |\childdocforward{|\textit{dest}|}| is achieved by:
%
\begin{center}
|\def\jobname{|\textit{dest}|}\input{\jobname}|
\end{center}
%
The redirection with prefix
|\childdocforwardprefix[|\textit{prefix}|]{|\textit{dest}|}|
is accomplished by:
%
\begin{center}
\begin{tabular}{l}
|{\edef\jobname{\scantokens\expandafter{\jobname\noexpand}}|\\
|\def\redirectjob |\textit{prefix}|#1~~~{\gdef\jobname{|\textit{dest}|#1}}|\\
|\expandafter\redirectjob\jobname~~~}\input{\jobname}|
\end{tabular}
\end{center}

In an alternative approach,
child documents can be compiled by a specific command line
without additional code or specific definitions:
%
\begin{center}
|... -jobname "|\textit{target}|" "|[\textit{flags}]%
|\includeonly{|\textit{dest}|}\input{|\textit{main}|}"|
\end{center}
%

%%%%%%%%%%%%%%%%%%%%%%%%%%%%%%%%%%%%%%%%%%%%%%%%%%%%%%%%%%%%%%%%%%%%%%%%%%%%%%%%
%%%%%%%%%%%%%%%%%%%%%%%%%%%%%%%%%%%%%%%%%%%%%%%%%%%%%%%%%%%%%%%%%%%%%%%%%%%%%%%%
\section{Information}

%%%%%%%%%%%%%%%%%%%%%%%%%%%%%%%%%%%%%%%%%%%%%%%%%%%%%%%%%%%%%%%%%%%%%%%%%%%%%%%%
\subsection{Copyright}

Copyright \copyright{} 2017--2018 Niklas Beisert

This work may be distributed and/or modified under the
conditions of the \LaTeX{} Project Public License, either version 1.3
of this license or (at your option) any later version.
The latest version of this license is in
  \url{http://www.latex-project.org/lppl.txt}
and version 1.3 or later is part of all distributions of \LaTeX{}
version 2005/12/01 or later.

This work has the LPPL maintenance status `maintained'.

The Current Maintainer of this work is Niklas Beisert.

This work consists of the files |README.txt|, |childdoc.ins| and |childdoc.dtx|
as well as the derived files |childdoc.def|, |cdocsamp.tex|
with |cdocsch1.tex|, |cdocsch2.tex|, |cdocspt3.tex|, |cdocspt4.tex|,
|cdocsdrf.tex|, |cdocsfn1.tex|, |cdocsfn2.tex|
as well as |childdoc.pdf|.

%%%%%%%%%%%%%%%%%%%%%%%%%%%%%%%%%%%%%%%%%%%%%%%%%%%%%%%%%%%%%%%%%%%%%%%%%%%%%%%%
\subsection{Files and Installation}

The package consists of the files:
%
\begin{center}
\begin{tabular}{ll}
    |README.txt|   & readme file \\
    |childdoc.ins| & installation file \\
    |childdoc.dtx| & source file \\
    |childdoc.def| & definition file \\
    |cdocsamp.tex| & sample main file \\
    |cdocsch1.tex| & sample include file \\
    |cdocsch2.tex| & sample include file \\
    |cdocspt3.tex| & sample part file \\
    |cdocspt4.tex| & sample part file \\
    |cdocsdrf.tex| & sample redirection file \\
    |cdocsfn1.tex| & sample redirection file \\
    |cdocsfn2.tex| & sample redirection file \\
    |childdoc.pdf| & manual
\end{tabular}
\end{center}
%
The distribution consists of the files
|README.txt|, |childdoc.ins| and |childdoc.dtx|.
%
\begin{itemize}
\item
Run (pdf)\LaTeX{} on |childdoc.dtx|
to compile the manual |childdoc.pdf| (this file).
\item
Run \LaTeX{} on |childdoc.ins| to create the definitions file |childdoc.def|
and the sample |cdocsamp.tex| with include files
|cdocsch1.tex|, |cdocsch2.tex|, |cdocspt3.tex|, |cdocspt4.tex|,
|cdocsdrf.tex|, |cdocsfn1.tex|, |cdocsfn2.tex|.
Then copy the file |childdoc.def| to an appropriate directory of your \LaTeX{}
distribution, e.g.\ \textit{texmf-root}|/tex/latex/childdoc|.
\end{itemize}

%%%%%%%%%%%%%%%%%%%%%%%%%%%%%%%%%%%%%%%%%%%%%%%%%%%%%%%%%%%%%%%%%%%%%%%%%%%%%%%%
\subsection{Related CTAN Packages}

There are several other packages which offer a similar functionality:
%
\begin{itemize}
\item
The packages
\href{http://ctan.org/pkg/docmute}{\textsf{docmute}},
\href{http://ctan.org/pkg/includex}{\textsf{includex}} and
\href{http://ctan.org/pkg/standalone}{\textsf{standalone}}
provide commands to include only the document body of
a child file thus allowing both files to be compiled individually.
\item
The packages \href{http://ctan.org/pkg/subdocs}{\textsf{subdocs}}
and \href{http://ctan.org/pkg/subfiles}{\textsf{subfiles}}
provide structures in which the main and child documents can be
encapsulated and allowing them to be compiled individually.
The inclusion mechanism is different from the conventional |\include|.
\item
The package \href{http://ctan.org/pkg/combine}{\textsf{combine}}
is an elaborate solution to combine several documents into one.
\end{itemize}
%
See also the CTAN topic \href{http://ctan.org/topic/subdocs}{\textsf{subdocs}}
for further related packages.
The present package differs from the above solutions in that
a document structure constructed with the conventional |\include| mechanism
just needs two extra commands at the top of every file
such that all constituent files can be compiled individually.

%%%%%%%%%%%%%%%%%%%%%%%%%%%%%%%%%%%%%%%%%%%%%%%%%%%%%%%%%%%%%%%%%%%%%%%%%%%%%%%%
%\subsection{Feature Suggestions}
%
%The following is a list of features which may be useful for future
%versions of this package:
%%
%\begin{itemize}
%\item
%\ldots
%\end{itemize}

%%%%%%%%%%%%%%%%%%%%%%%%%%%%%%%%%%%%%%%%%%%%%%%%%%%%%%%%%%%%%%%%%%%%%%%%%%%%%%%%
\subsection{Revision History}

%%%%%%%%%%%%%%%%%%%%%%%%%%%%%%%%%%%%%%%%
\paragraph{v2.0:} 2018/12/30

\begin{itemize}
\item
immediate forward processing
\item
added |\childdocby| mechanism
\item
manual restructured
\end{itemize}

%%%%%%%%%%%%%%%%%%%%%%%%%%%%%%%%%%%%%%%%
\paragraph{v1.6:} 2018/01/17

\begin{itemize}
\item
application for development of include files
\item
corrections to manual
\end{itemize}

%%%%%%%%%%%%%%%%%%%%%%%%%%%%%%%%%%%%%%%%
\paragraph{v1.5:} 2017/05/21

\begin{itemize}
\item
more complete structuring introduced
\item
|\childdocof| introduced
\item
|\childdoc| renamed to |\childdocmain|
\item
|\childredirect| renamed to |\childdocforward| and |\childdocforwardprefix|
and functionality expanded
\end{itemize}

%%%%%%%%%%%%%%%%%%%%%%%%%%%%%%%%%%%%%%%%
\paragraph{v1.0:} 2017/04/27

\begin{itemize}
\item
manual and install package
\item
first version published on CTAN
\end{itemize}

%%%%%%%%%%%%%%%%%%%%%%%%%%%%%%%%%%%%%%%%
\paragraph{v0.6:} 2017/04/26

\begin{itemize}
\item
redirection mechanism added
\end{itemize}

%%%%%%%%%%%%%%%%%%%%%%%%%%%%%%%%%%%%%%%%
\paragraph{v0.5:} 2017/04/26

\begin{itemize}
\item
functionality in definition file
\end{itemize}


%%%%%%%%%%%%%%%%%%%%%%%%%%%%%%%%%%%%%%%%%%%%%%%%%%%%%%%%%%%%%%%%%%%%%%%%%%%%%%%%
%%%%%%%%%%%%%%%%%%%%%%%%%%%%%%%%%%%%%%%%%%%%%%%%%%%%%%%%%%%%%%%%%%%%%%%%%%%%%%%%
%%%%%%%%%%%%%%%%%%%%%%%%%%%%%%%%%%%%%%%%%%%%%%%%%%%%%%%%%%%%%%%%%%%%%%%%%%%%%%%%
\appendix

\settowidth\MacroIndent{\rmfamily\scriptsize 000\ }

 \DocInput{childdoc.dtx}

\end{document}
%</driver>
% \fi
%
% %%%%%%%%%%%%%%%%%%%%%%%%%%%%%%%%%%%%%%%%%%%%%%%%%%%%%%%%%%%%%%%%%%%%%%%%%%%%%%
% %%%%%%%%%%%%%%%%%%%%%%%%%%%%%%%%%%%%%%%%%%%%%%%%%%%%%%%%%%%%%%%%%%%%%%%%%%%%%%
% \section{Sample}
%\iffalse
%<*samplemain>
%\fi
%
% The following presents a sample document
% with two chapters, two parts, a title page,
% a compile flag as well as three forwarding files to set the flag.
% It consists of eight |.tex| files:
% \begin{center}
% \begin{tabular}{ll}
% |cdocsamp.tex|&main file\\
% |cdocsch1.tex|&include file for chapter 1\\
% |cdocsch2.tex|&include file for chapter 2\\
% |cdocspt3.tex|&include file for part 3\\
% |cdocspt4.tex|&include file for part 4\\
% |cdocsdrf.tex|&forwarding file for main file in draft mode\\
% |cdocsfi1.tex|&forwarding file for final version of chapter 1\\
% |cdocsfi2.tex|&forwarding file for final version of chapter 2\\
% \end{tabular}
% \end{center}
% Each of the eight files can be compiled directly by the \LaTeX{} compiler.
%
% %%%%%%%%%%%%%%%%%%%%%%%%%%%%%%%%%%%%%%
% \paragraph{Main File.}
%
% The main file is called |cdocsamp.tex|.
%
% Load the \textsf{childdoc} definitions and
% declare the filename for the main document:
%    \begin{macrocode}
\input{childdoc.def}
\childdocmain{}
%    \end{macrocode}

% Optional override for |\version| flag:
%    \begin{macrocode}
%%\ifchilddoc\else\providecommand{\version}{draft}\fi
%    \end{macrocode}

% Define the default values for the |\version| flag
% (|final| for the main file and |draft| for childs):
%    \begin{macrocode}
\ifchilddoc
\providecommand{\version}{draft}
\else
\providecommand{\version}{final}
\fi
%    \end{macrocode}

% Load the standard document class:
%    \begin{macrocode}
\documentclass[12pt]{article}
%    \end{macrocode}

% Start the document body:
%    \begin{macrocode}
\begin{document}
%    \end{macrocode}

% Declare a title page.
% Print title, part of document being processed and version flag:
%    \begin{macrocode}
\addtocounter{page}{-1}
\begin{center}
{\LARGE\bfseries{}childdoc example\par}
\vspace{1cm}
\ifchilddoc
\ifchilddocmanual part\else chapter\fi:
`\childdocname' of `\childdocjob'\par
\else
main document: `\childdocjob'\par
\fi
version: \version\par
\end{center}
\newpage
%    \end{macrocode}

% Manually include selected file,
% otherwise process as usual:
%    \begin{macrocode}
\ifchilddocmanual
\section*{part `\childdocname'}
\input{\childdocname}
\else
%    \end{macrocode}

% Include the two chapters:
%    \begin{macrocode}
\include{cdocsch1}
\include{cdocsch2}
%    \end{macrocode}

% Include the two parts unless only chapters should be displayed:
%    \begin{macrocode}
\ifchilddoc\else
\section{part three}
\input{cdocspt3}
\section{part four}
\input{cdocspt4}
\fi
%    \end{macrocode}

% Process as usual until here:
%    \begin{macrocode}
\fi
%    \end{macrocode}

% End of document body:
%    \begin{macrocode}
\end{document}
%    \end{macrocode}
%\iffalse
%</samplemain>
%\fi
%
% %%%%%%%%%%%%%%%%%%%%%%%%%%%%%%%%%%%%%%
% \paragraph{Chapter Include Files.}
%
% The include files are called |cdocsch1.tex| and |cdocsch2.tex|.
%
%\iffalse
%<*samplechap1|samplechap2>
%\fi

% Optional override for |\version| flag:
%    \begin{macrocode}
%%\providecommand{\version}{final}
%    \end{macrocode}

% Include the main document:
%    \begin{macrocode}
\input{childdoc.def}
\childdocof{cdocsamp}
%    \end{macrocode}

%\iffalse
%</samplechap1|samplechap2>
%\fi
%
%\iffalse
%<*samplechap1>
%\fi
% Some text for chapter 1:
%    \begin{macrocode}
\section{one}
some text in chapter one
%    \end{macrocode}

%\iffalse
%</samplechap1>
%\fi
% Some text for chapter 2:
%\iffalse
%<*samplechap2>
%\fi
%    \begin{macrocode}
\section{two}
more text in chapter two
%    \end{macrocode}

%\iffalse
%</samplechap2>
%\fi
%
% %%%%%%%%%%%%%%%%%%%%%%%%%%%%%%%%%%%%%%
% \paragraph{Part Include Files.}
%
% The include files are called |cdocspt3.tex| and |cdocspt4.tex|.
%
%\iffalse
%<*samplepart3|samplepart4>
%\fi

% Optional override for |\version| flag:
%    \begin{macrocode}
%%\providecommand{\version}{final}
%    \end{macrocode}

% Include the main document:
%    \begin{macrocode}
\input{childdoc.def}
\childdocby{cdocsamp}
%    \end{macrocode}

%\iffalse
%</samplepart3|samplepart4>
%\fi
%
%\iffalse
%<*samplepart3>
%\fi
% Some text for part 3:
%    \begin{macrocode}
some text in part three
%    \end{macrocode}

%\iffalse
%</samplepart3>
%\fi
% Some text for part 4:
%\iffalse
%<*samplepart4>
%\fi
%    \begin{macrocode}
more text in part four
%    \end{macrocode}

%\iffalse
%</samplepart4>
%\fi
%
% %%%%%%%%%%%%%%%%%%%%%%%%%%%%%%%%%%%%%%
% \paragraph{Forwarding for a Complete Draft.}
%
% The following forwarding file |cdocsdrf.tex|
% compiles the main document in draft mode:
%\iffalse
%<*sampledraft>
%\fi
%    \begin{macrocode}
\def\version{draft}
\input{childdoc.def}
\childdocforward{cdocsamp}
%    \end{macrocode}

%\iffalse
%</sampledraft>
%\fi
%
% %%%%%%%%%%%%%%%%%%%%%%%%%%%%%%%%%%%%%%
% \paragraph{Forwarding for Final Version of the Chapters.}
%
% The following forwarding files |cdocsfn1.tex| and |cdocsfn2.tex|
% (with identical content)
% compile the final versions of the child documents
% |cdocsch1.tex| and |cdocsch2.tex|, respectively:
%\iffalse
%<*samplefinal>
%\fi
%    \begin{macrocode}
\def\version{final}
\input{childdoc.def}
\childdocforwardprefix[cdocsamp]{cdocsfn}{cdocsch}
%    \end{macrocode}

%\iffalse
%</samplefinal>
%\fi
%
% %%%%%%%%%%%%%%%%%%%%%%%%%%%%%%%%%%%%%%
% \paragraph{Command Line Processing.}
%
% The following three command lines generate the output files
% |cdocscld|, |cdocscl1| and |cdocscl2|
% which should be identical to
% |cdocsdrf|, |cdocsch1| and |cdocsfn2|, respectively:
% \begin{center}
% \begin{tabular}{l}
% |latex -jobname cdocscld \|\\
% |  "\def\version{draft}\input{childdoc.def}\childdocforward{cdocsamp}"|\\
% |latex -jobname cdocscl1 \|\\
% |  "\input{childdoc.def}\childdocforward[cdocsamp]{cdocsch1}"|\\
% |latex -jobname cdocscl2 \|\\
% |  "\def\version{final}\input{childdoc.def}\childdocforward{cdocsch2}"|
% \end{tabular}
% \end{center}
% Note that the trailing backslash on each first line
% merely continues the input to the second line
% (for convenient cut ant paste).
% Furthermore, the command |latex| can be replaced by any
% of its alternative versions such as |pdflatex|.
%
% %%%%%%%%%%%%%%%%%%%%%%%%%%%%%%%%%%%%%%%%%%%%%%%%%%%%%%%%%%%%%%%%%%%%%%%%%%%%%%
% %%%%%%%%%%%%%%%%%%%%%%%%%%%%%%%%%%%%%%%%%%%%%%%%%%%%%%%%%%%%%%%%%%%%%%%%%%%%%%
% \section{Implementation}
%\iffalse
%<*package>
%\fi
%
% This section describes the definitions file |childdoc.def|.

% The definitions cannot be loaded using |\usepackage| or |\RequirePackage|
% which has a mechanism to prevent loading a style file more than once.
% When loading the definitions by means of |\input|
% multiple instances have to be prevented manually:
%\iffalse
%This code needs to be before the `\ProvidesFile' directive
%which is defined at the beginning of this file.
%Therefore it is also placed there and commented out here.
%</package>
%<*discard>
%\fi
%    \begin{macrocode}
\ifdefined\childdocmain\endinput\fi
%    \end{macrocode}
%\iffalse
%</discard>
%<*package>
%\fi
%
% \macro{\ifchilddoc}
% \macro{\ifchilddocmanual}
% The conditional |\ifchilddoc| tells whether a
% child (true) or main (false) document is being compiled.
% The conditional |\ifchilddocmanual| tells whether
% the |\includeonly| mechanism is used (false) or
% the selection of child files must be performed manually (true).
% The definitions initialise to false:
%    \begin{macrocode}
\newif\ifchilddoc
\newif\ifchilddocmanual
%    \end{macrocode}

% \macro{\childdocname}
% \macro{\childdocjob}
% The macro |\childdocname| stores the name of the main document
% to be compiled. The macro |\childdocjob| stores the name of
% the document on which the \LaTeX{} compiler was originally invoked.
% The content of |\jobname| cannot be compared
% to filenames specified in the source due to different catcodes.
% The following code rescans |\jobname|, stores the result
% in |\childdocname| and saves a copy in |\childdocjob|:
%    \begin{macrocode}
\edef\childdocname{\scantokens\expandafter{\jobname\noexpand}}
\let\childdocjob\childdocname
%    \end{macrocode}

% \macro{\childdocdisable}
% The macro |\childdocdisable| prevents the main file
% from being processed more than once.
% At this stage, the main document command |\childdocmain|
% is assumed to be called once again where it should do nothing.
% Any subsequent call to it should prevent
% a secondary processing of the main document
% It overwrites the forwarding commands
% |\childdocof| and |\childdocforward|
% with empty macros to prevent further inclusions of the main document:
%    \begin{macrocode}
\newcommand{\childdocdisable}
{
  \renewcommand{\childdocmain}[1]{\renewcommand{\childdocmain}[1]{\endinput}}
  \renewcommand{\childdocof}[1]{}
  \renewcommand{\childdocby}[2][]{}
  \renewcommand{\childdocforward}[2][]{}
  \renewcommand{\childdocdisable}{}
}
%    \end{macrocode}

% \macro{\childdocmain}
% The macro |\childdocmain| is to be called at the top of the main file
% with nothing or the main filename (without extension) as argument.
% First, it breaks loops.
% If the argument is not empty and does not match |\childdocname|
% (which is set by the first inclusion of |childdoc.def|),
% |\ifchilddoc| is set to true, |\includeonly| is applied to the child file
% and |\jobname| is set to the main file
% (for proper handling of |.aux| files):
%    \begin{macrocode}
\newcommand{\childdocmain}[1]
{
  \childdocdisable\childdocmain{}
  \if?#1?\else
    \begingroup
      \def\childdoctmp{#1}
      \ifx\childdoctmp\childdocname
        \def\childdoctmp{}
      \else
        \def\childdoctmp
        {
          \childdoctrue
          \includeonly{\childdocname}
          \def\childdocjob{#1}
          \def\jobname{#1}
        }
      \fi
      \expandafter
    \endgroup
    \childdoctmp
  \fi
}
%    \end{macrocode}

% \macro{\childdocof}
% The command |\childdocof| redirects
% compilation to the main file |#1|.
%    \begin{macrocode}
\newcommand{\childdocof}[1]
{
  \childdocdisable
  \childdoctrue
  \includeonly{\childdocname}
  \def\jobname{#1}
  \def\childdocjob{#1}
  \input{#1}
}
%    \end{macrocode}

% \macro{\childdocby}
% The command |\childdocby| ....
%    \begin{macrocode}
\newcommand{\childdocby}[2][]
{
  \childdocdisable
  \childdoctrue
  \childdocmanualtrue
  \if?#1?\else
    \def\jobname{#2}
  \fi
  \def\childdocjob{#2}
  \input{#2}
  \endinput
}
%    \end{macrocode}

% \macro{\childdocforward}
% The command |\childdocforward| redirects
% compilation to the main file or
% (if the optional argument is given) a child file.
% Parameters are set as if the main file
% or a child file starting with |\childdocof| was compiled.
% Then compilation is handed over to the main file:
%    \begin{macrocode}
\newcommand{\childdocforward}[2][]
{
  \begingroup
    \if?#1?
      \def\childdoctmp
      {
        \def\childdocname{#2}
        \def\childdocjob{#2}
        \def\jobname{#2}
        \input{#2}
        \endinput
      }
    \else
      \def\childdoctmp
      {
        \childdocdisable
        \def\childdocname{#2}
        \childdoctrue
        \includeonly{#2}
        \def\childdocjob{#1}
        \def\jobname{#1}
        \input{#1}
        \endinput
      }
    \fi
    \expandafter
  \endgroup
  \childdoctmp
}
%    \end{macrocode}

% \macro{\childdocforwardprefix}
% The command |\childdocforwardprefix| redirects
% compilation to the main or a child file by means of a pattern.
% The prefix |#1| in the current filename is replaced by |#2|
% and the suffix of the current filename is kept
% (it is assumed that the filename does not contain the substring `|~~~|'
% which is used as a delimiter).
% Compilation is handed over to the new file by |\childdocforward|:
%    \begin{macrocode}
\newcommand{\childdocforwardprefix}[3][]
{
  \begingroup
    \def\childdocextract #2##1~~~{\def\childdoctmp{\childdocforward[#1]{#3##1}}}
    \expandafter\childdocextract\childdocname~~~
    \expandafter
  \endgroup
  \childdoctmp
}
%    \end{macrocode}

% \macro{\childdoc}
% The deprecated macro |\childdoc| is a legacy version of |\childdocmain|:
%    \begin{macrocode}
\newcommand{\childdoc}{\childdocmain}
%    \end{macrocode}

% \macro{\childdocredirect}
% The deprecated macro |\childdocredirect| is a legacy version
% of |\childdocforward| and |\childdocforwardprefix|:
%    \begin{macrocode}
\newcommand{\childdocredirect}[2][]
{
  \begingroup
    \if?#1?
      \def\childdoctmp{\childdocforward{#2}}
    \else
      \def\childdoctmp{\childdocforwardprefix{#1}{#2}}
    \fi
    \expandafter
  \endgroup
  \childdoctmp
}
%    \end{macrocode}

%\iffalse
%</package>
%\fi
%
\endinput

\childdocforward{cdocsamp}
%    \end{macrocode}

%\iffalse
%</sampledraft>
%\fi
%
% %%%%%%%%%%%%%%%%%%%%%%%%%%%%%%%%%%%%%%
% \paragraph{Forwarding for Final Version of the Chapters.}
%
% The following forwarding files |cdocsfn1.tex| and |cdocsfn2.tex|
% (with identical content)
% compile the final versions of the child documents
% |cdocsch1.tex| and |cdocsch2.tex|, respectively:
%\iffalse
%<*samplefinal>
%\fi
%    \begin{macrocode}
\def\version{final}
% \iffalse
%
% childdoc.dtx Copyright (C) 2017-2018 Niklas Beisert
%
% This work may be distributed and/or modified under the
% conditions of the LaTeX Project Public License, either version 1.3
% of this license or (at your option) any later version.
% The latest version of this license is in
%   http://www.latex-project.org/lppl.txt
% and version 1.3 or later is part of all distributions of LaTeX
% version 2005/12/01 or later.
%
% This work has the LPPL maintenance status `maintained'.
%
% The Current Maintainer of this work is Niklas Beisert.
%
% This work consists of the files childdoc.dtx and childdoc.ins
% and the derived files childdoc.def and cdocsamp.tex with
% cdocsch1.tex, cdocsch2.tex, cdocsdrf.tex, cdocsfn1.tex, cdocsfn2.tex.
%
%<package>\ifdefined\childdocmain\endinput\fi
%<package>\ProvidesFile{childdoc.def}[2018/12/30 v2.0 child document driver]
%<samplemain>\ProvidesFile{cdocsamp.tex}[2018/12/30 v2.0 sample for childdoc]
%<*driver>
%\ProvidesFile{childdoc.drv}[2018/12/30 v2.0 childdoc reference manual file]
\PassOptionsToClass{10pt,a4paper}{article}
\documentclass{ltxdoc}

\usepackage[margin=35mm]{geometry}
\usepackage{hyperref}
\usepackage{hyperxmp}
\usepackage[usenames]{color}

\hypersetup{colorlinks=true}
\hypersetup{pdfstartview=FitH}
\hypersetup{pdfpagemode=UseNone}
\hypersetup{pdfsource={}}
\hypersetup{pdflang={en-UK}}
\hypersetup{pdfcopyright={Copyright 2017-2018 Niklas Beisert.
  This work may be distributed and/or modified under the
  conditions of the LaTeX Project Public License, either version 1.3
  of this license or (at your option) any later version.}}
\hypersetup{pdflicenseurl={http://www.latex-project.org/lppl.txt}}
\hypersetup{pdfcontactaddress={ETH Zurich, ITP, HIT K,
  Wolfgang-Pauli-Strasse 27}}
\hypersetup{pdfcontactpostcode={8093}}
\hypersetup{pdfcontactcity={Zurich}}
\hypersetup{pdfcontactcountry={Switzerland}}
\hypersetup{pdfcontactemail={nbeisert@itp.phys.ethz.ch}}
\hypersetup{pdfcontacturl={http://people.phys.ethz.ch/\xmptilde nbeisert/}}

\newcommand{\secref}[1]{\hyperref[#1]{section \ref*{#1}}}

\parskip1ex
\parindent0pt
\let\olditemize\itemize
\def\itemize{\olditemize\parskip0pt}

\begin{document}

\title{The \textsf{childdoc} Package}
\hypersetup{pdftitle={The childdoc Package}}
\author{Niklas Beisert\\[2ex]
  Institut f\"ur Theoretische Physik\\
  Eidgen\"ossische Technische Hochschule Z\"urich\\
  Wolfgang-Pauli-Strasse 27, 8093 Z\"urich, Switzerland\\[1ex]
  \href{mailto:nbeisert@itp.phys.ethz.ch}
  {\texttt{nbeisert@itp.phys.ethz.ch}}}
\hypersetup{pdfauthor={Niklas Beisert}}
\hypersetup{pdfsubject={Manual for the LaTeX2e Package childdoc}}
\date{30 December 2018, \textsf{v2.0}}
\maketitle

\begin{abstract}\noindent
\textsf{childdoc} is a \LaTeXe{} package
that enables the direct compilation
of document sections included by |\include|
to individual files.
\end{abstract}

\begingroup
\parskip0ex
\tableofcontents
\endgroup

%%%%%%%%%%%%%%%%%%%%%%%%%%%%%%%%%%%%%%%%%%%%%%%%%%%%%%%%%%%%%%%%%%%%%%%%%%%%%%%%
%%%%%%%%%%%%%%%%%%%%%%%%%%%%%%%%%%%%%%%%%%%%%%%%%%%%%%%%%%%%%%%%%%%%%%%%%%%%%%%%
\section{Introduction}

\LaTeX{} provides a mechanism to structure a large document (such as a book)
into a main file and several child files (containing the chapters)
using the |\include| command.
This mechanism is beneficial for documents
which span hundreds of pages in order to
make the source file(s) more manageable.
Moreover, compilation can be restricted to
selected child files by means of the |\includeonly| command.
The latter feature can be used to reduce the compilation time while editing
(this was significantly more useful in the earlier days of \LaTeX{})
or to generate a smaller document which is easier to navigate.
Another application of |\includeonly| is to generate
documents consisting of selected parts of the complete document.

However, there are a few drawbacks of the plain |\include| mechanism:
\begin{itemize}
\item
The child files cannot be compiled on their own,
they can only be compiled via the main file.
A naive editing environment
(such as a text editor with an option
to have the current file processed by \LaTeX)
may require one to switch to the main file before compiling;
attempting to compile the child file produces errors.
\item
The main file must be modified (each time)
to adjust the |\includeonly| command
to the present needs. This easily leaves the main file in a messy state.
\item
The generated document will always carry the filename
of the main document. This is inconvenient if
several child files are to be compiled and
to be kept for distribution.
\end{itemize}

The present package provides a simple interface
to make child files individually compilable by \LaTeX{}.
Compiling a child file then has the same effect as compiling
the main file with an |\includeonly| command
to select the appropriate child.
Moreover the generated document will carry the name of the child
rather than the main file.
This resolves all three above issues.

This feature is meant to make the editing of books,
thesis documents and lecture notes somewhat more convenient.
However, the package can also be used efficiently for
composing a series of documents (such as exercise sheets)
which are typically distributed individually.
It then assists the author in generating the individual documents
(potentially in different versions)
as well as a document containing the collected series.
Another application is in developing style files
or other kinds of included material
where compilation of the style file could redirect
to a sample or test file.

%%%%%%%%%%%%%%%%%%%%%%%%%%%%%%%%%%%%%%%%%%%%%%%%%%%%%%%%%%%%%%%%%%%%%%%%%%%%%%%%
%%%%%%%%%%%%%%%%%%%%%%%%%%%%%%%%%%%%%%%%%%%%%%%%%%%%%%%%%%%%%%%%%%%%%%%%%%%%%%%%
\section{Usage}

First of all, the package \textsf{childdoc} is \emph{not} a standard
\LaTeXe{} |.sty| style file! Therefore it needs to be invoked in
a non-standard way.

%%%%%%%%%%%%%%%%%%%%%%%%%%%%%%%%%%%%%%%%%%%%%%%%%%%%%%%%%%%%%%%%%%%%%%%%%%%%%%%%
\subsection{Included Files}
\label{sec:include}

%%%%%%%%%%%%%%%%%%%%%%%%%%%%%%%%%%%%%%%%
\DescribeMacro{\childdocmain}
To use the package, add the commands
\begin{center}
\begin{tabular}{l}
|\input{childdoc.def}|\\
|\childdocmain{}|\\
\end{tabular}
\end{center}
at the very top of the main \LaTeX{} file,
in particular \emph{before} the |\documentclass| statement!
The argument of |\childdocmain| should be left empty
(but it must be present).

%%%%%%%%%%%%%%%%%%%%%%%%%%%%%%%%%%%%%%%%
\DescribeMacro{\childdocof}
Furthermore, add the commands
\begin{center}
\begin{tabular}{l}
|\input{childdoc.def}|\\
|\childdocof{|\textit{main}|}|\\
\end{tabular}
\end{center}
at the top of every child file \textit{child}
which is included by |\include{|\textit{child}|}|
from within the main file
(or at least for those files to be compiled individually).
The argument \textit{main} must be the filename of the main file.

There are a couple of
considerations in setting up the main and child documents:

%%%%%%%%%%%%%%%%%%%%%%%%%%%%%%%%%%%%%%%%
\paragraph{Restrictions.}

Please note the following restrictions:
\begin{itemize}
\item
|\childdocmain| must be called with one argument \textit{main}
to ensure compatibility with earlier version of the package.
It must either be empty (|\childdocmain{}|)
or precisely match the filename of the main file in which it is specified.
See \secref{sec:detection} for further information.
\item
The filename \textit{main} must be specified without the |.tex| extension.
\item
The filename \textit{main} is case sensitive
(even in case-insensitive file systems)
due to internal string comparison.
\item
The argument \textit{main} should be fully expanded, it cannot be a macro.
\item
Subdirectories and special characters should be avoided in filenames.
\item
The command |\childdocmain{|\textit{main}|}| must be followed by a whitespace.
It should not be followed immediately by another command
or by a comment mark `|%|'.
This is because the \TeX{} parser reads the token immediately following
the argument of |\childdocmain| and puts it
at the beginning of every child section;
however, a white\-space is ignored.
\end{itemize}

%%%%%%%%%%%%%%%%%%%%%%%%%%%%%%%%%%%%%%%%
\paragraph{Content of Main File.}

It is advisable to place all content in the child files included by |\include|.
Any output contained in the main file will appear in all child documents
unless suppressed manually;
it cannot be suppressed automatically by the |\includeonly| directive
and thus should normally be avoided.
A method to include some content in the main file
by means of conditional processing is described in \secref{sec:conditional}.

%%%%%%%%%%%%%%%%%%%%%%%%%%%%%%%%%%%%%%%%
\paragraph{Page Numbering.}

When only a part of the document is compiled,
the appropriate numbering of pages
(as well as other status parameters)
is determined from the |.aux| files.
The latter contain information from previous passes.
However this information needs to propagate through
all intermediate child documents.
Therefore the page numbering in child documents may well
be inconsistent until the complete document is compiled at least once.

A useful (if unconventional) way to always ensure a consistent
page numbering is to restart the numbering in each child document
and denote the pages by `\textit{child}|.|\textit{page}'
where \textit{child} represents the chapter/section number of the child file.
This can be achieved by the command
|\numberwithin{page}{|\textit{child}|}|
of the \textsf{amsmath} package
where \textit{child} can be |chapter| or |section|
depending on the chosen structuring.
Alternatively, one can modify the macro |\thepage| appropriately
and reset the counter |page| at the start of each child file.

%%%%%%%%%%%%%%%%%%%%%%%%%%%%%%%%%%%%%%%%%%%%%%%%%%%%%%%%%%%%%%%%%%%%%%%%%%%%%%%%
\subsection{Conditional Processing}
\label{sec:conditional}

The package provides a mechanism to compile different versions
of a document. To customise the versions further some conditional processing
can come in handy to distinguish which version is being compiled.
The package provides two macros to describe the compilation context:

%%%%%%%%%%%%%%%%%%%%%%%%%%%%%%%%%%%%%%%%
\DescribeMacro{\ifchilddoc}
The conditional |\ifchilddoc| distinguishes between the compilation of
child documents and the main document:
%
\begin{center}
|\ifchilddoc |\textit{child-code}| |[|\||else |\textit{main-code}]| \||fi|
\end{center}

%%%%%%%%%%%%%%%%%%%%%%%%%%%%%%%%%%%%%%%%
\DescribeMacro{\childdocname}
\DescribeMacro{\childdocjob}
The macro |\childdocname| contains the filename (without extension)
of the main or child file being processed.
Note that |\childdocjob| will always contain the name of the main file.

%%%%%%%%%%%%%%%%%%%%%%%%%%%%%%%%%%%%%%%%
\paragraph{Title Page.}

Conditional processing can be used to include a title or banner page
in the main document when proper precautions are taken.
Importantly, the code in the main file should ensure that the page counter
(as well as other status parameters which are stored in the |.aux| files)
takes the same value after the conditional processing.
Otherwise the page numbers may take divergent values
depending on which part is compiled.

For example, a title page could be declared by:
%
\begin{center}
\begin{tabular}{l}
|\ifchilddoc\||else|\\
|\addtocounter{page}{-1}|\\
\textit{code for title page}\\
|\newpage|\\
|\||fi|
\end{tabular}
\end{center}
%
A banner page for the child documents can be generated by:
%
\begin{center}
\begin{tabular}{l}
|\ifchilddoc|\\
|\addtocounter{page}{-1}|\\
\textit{code for banner page}\\
|\newpage|\\
|\||fi|
\end{tabular}
\end{center}
%
Here one could write a message such as:
\begin{center}
|This is the part \childdocname{} of \childdocjob{}.|
\end{center}

%%%%%%%%%%%%%%%%%%%%%%%%%%%%%%%%%%%%%%%%%%%%%%%%%%%%%%%%%%%%%%%%%%%%%%%%%%%%%%%%
\subsection{Flags}
\label{sec:flags}

The package makes it easy to generate different versions
of the main or child documents.
To this end compilation flags can be defined
and assigned different default values.
They will be particularly useful in conjunction
with the forwarding mechanism described in \secref{sec:forward}.

For example, it may be useful to have a flag |\version|
which can be set to |draft| or |final|.
The document source will contain some conditional code
depending on the value of |\version|.
Suppose further, the flag should default to |final| for the main file
and to |draft| for child files
which is a natural assignment for editing the document.
This is achieved by placing the following code
in the preamble of the main document
(below the |\childdocmain| directive):
%
\begin{center}
\begin{tabular}{l}
|\ifchilddoc|\\
|\providecommand{\version}{draft}|\\
|\||else|\\
|\providecommand{\version}{final}|\\
|\||fi|
\end{tabular}
\end{center}
%
The definition by |\providecommand| makes sure
that previous definitions are not overwritten.
Further statements |\providecommand{\version}{...}|
can thus be added before the above code to override it.

For the main file, one might add a line
(between |\childdocmain| and the above block)
%
\begin{center}
|%\ifchilddoc\||else\providecommand{\version}{draft}\||fi|
\end{center}
%
which can be uncommented to produce a draft version.
Likewise one can add a line to the very top of a child file
(above the |\childdocof{|\textit{main}|}| directive)
%
\begin{center}
|%\providecommand{\version}{final}|
\end{center}
%
which can be uncommented to produce the final version of this child document.

%%%%%%%%%%%%%%%%%%%%%%%%%%%%%%%%%%%%%%%%%%%%%%%%%%%%%%%%%%%%%%%%%%%%%%%%%%%%%%%%
\subsection{Forwarding}
\label{sec:forward}

Different versions of the main or child documents
using compilation flags as described in \secref{sec:flags}
can be (permanently) stored in different files
for convenient compilation, viewing and distribution.
To this end, the package defines a command
to pass on compilation to a different file:

%%%%%%%%%%%%%%%%%%%%%%%%%%%%%%%%%%%%%%%%
\DescribeMacro{\childdocforward}
The command |\childdocforward| redirects processing to
another source file:
%
\begin{center}
\begin{tabular}{l}
|\input{childdoc.def}|\\
|\childdocforward[|\textit{main}|]{|\textit{dest}|}|\\
\end{tabular}
\end{center}
%
The argument \textit{dest} is the destination file
(without extension).
It should be the main file or one of the child files.
Note that further \textsf{childdoc} directives
such as |\childdocof| and |\childdocforward|
in the indicated file will be processed in this form.
The optional argument \textit{main}
passes on directly to the main file \textit{main}
while pretending to compile the child \textit{dest}.
This form behaves as if \textit{dest}
issues |\childdocof{|\textit{main}|}| right away,
and no further \textsf{childdoc} directives will be processed.

%%%%%%%%%%%%%%%%%%%%%%%%%%%%%%%%%%%%%%%%
\DescribeMacro{\...prefix}
In the alternative form |\childdocforwardprefix|,
%
\begin{center}
\begin{tabular}{l}
|\input{childdoc.def}|\\
|\childdocforwardprefix[|\textit{main}|]{|\textit{prefix}|}{|\textit{dest}|}|
\end{tabular}
\end{center}
%
the destination file is determined by a pattern
depending on the current file:
To make this work, the current file must be called
`{\textit{prefix}\hspace{0.2em}\textit{suffix}}'
with \textit{prefix} matching precisely the argument.
Processing is then passed on to the file
`{\textit{dest}\hspace{0.2em}\textit{suffix}}'.
Surely, the same effect is achieved by
directly specifying the
argument `{\textit{dest}\hspace{0.2em}\textit{suffix}}'
in the first form.
However, that requires to set up a different file
for each child. With the alternative form of the command
all these files can have exactly the same content
which simplifies setting them up and maintaining them.

For example, the following file |draft.tex|
with a compilation flag |\version| as described in \secref{sec:flags}
compiles the main document as a draft:
%
\begin{center}
\begin{tabular}{l}
|\def\version{draft}|\\
|\input{childdoc.def}|\\
|\childdocforward{|\textit{main}|}|
\end{tabular}
\end{center}
%
Likewise, the following files |final|\textit{nn}|.tex|
compile the final version of the child document
|child|\textit{nn}|.tex|:
%
\begin{center}
\begin{tabular}{l}
|\def\version{final}|\\
|\input{childdoc.def}|\\
|\childdocforwardprefix{final}{child}|
\end{tabular}
\end{center}
%

Note that when several versions of a main file and/or of each child file
are to be generated, it may be convenient to set up a |Makefile| or
shell script to automatise the process.

%%%%%%%%%%%%%%%%%%%%%%%%%%%%%%%%%%%%%%%%%%%%%%%%%%%%%%%%%%%%%%%%%%%%%%%%%%%%%%%%
\subsection{Command Line Processing}
\label{sec:commandline}

The effect of redirection files can also be achieved by invoking
the \LaTeX{} compiler with a more elaborate command line.
Most conveniently this should be done as part
of a shell script or a |Makefile|.

When using \textsf{childdoc} in the main file, the following
command lines effectively perform a redirection
(note that depending on the shell being used,
backslashes may have to be doubled: `|\|' $\to$ `|\\|'):
%
\begin{center}
|... -jobname "|\textit{target}|" |\\|"|[\textit{flags}]%
|\input{childdoc.def}\childdocforward[|\textit{main}|]{|\textit{dest}|}"|
\end{center}
%
Here \textit{target} is the name of the output file,
\textit{main} is the name of the main file
and \textit{dest} is the name of the main or child file to be processed
(all filenames without extensions).
The optional argument \textit{main} can be omitted
if \textit{main} matches \textit{dest}.
Optionally, compilation \textit{flags} can be defined via |\def| commands.
This command line makes the \TeX{} engine believe
it is compiling the file \textit{target}
whose content is specified as the latter parameter.
The provided code then forwards the processing to
\textit{main} or \textit{dest} as described in \secref{sec:forward}.

%%%%%%%%%%%%%%%%%%%%%%%%%%%%%%%%%%%%%%%%%%%%%%%%%%%%%%%%%%%%%%%%%%%%%%%%%%%%%%%%
\subsection{Include by Input}
\label{sec:input}

Including child documents by |\include| has some restrictions by design.
Most notably, the content of a child document always occupies
its own set of pages; pages cannot be shared between child documents.
Usually, this behaviour makes perfect sense
because each child document contain an essential part of the document.
However, in some situations it may be desirable to compose
a document from a collection of parts
without having mandatory page breaks between then.
For this case, the package
provides a mechanism to include parts
by |\input| which can also be processed individually.
However, by construction this mechanism
requires manual handling of the content to be output.

%%%%%%%%%%%%%%%%%%%%%%%%%%%%%%%%%%%%%%%%
\DescribeMacro{\ifchilddocmanual}
The main file should be prepared as usual, see \secref{sec:include}.
However, the document body must make a distinction
between processing of an individual part and of the main document, e.g.:
%
\begin{center}
\begin{tabular}{l}
|\ifchilddocmanual|\\
|\input{\childdocname}|\\
|\||else|\\
\textit{document body with }|\input{|\textit{part}|}|\\
|\||fi|
\end{tabular}
\end{center}
%
The conditional |\ifchilddocmanual| is true whenever
a part to be included by |\input| is being compiled,
and the name of the part is stored in |\childdocname|.

%%%%%%%%%%%%%%%%%%%%%%%%%%%%%%%%%%%%%%%%
\DescribeMacro{\childdocby}
Each part to be included by |\input| should start with:
%
\begin{center}
\begin{tabular}{l}
|\input{childdoc.def}|\\
|\childdocby{|\textit{main}|}|\\
\end{tabular}
\end{center}
%
The directive |\childdocby| is similar to |\childdocof|
described in \secref{sec:include},
but the subsequent selection of content must be done manually.
To that end, both |\ifchilddoc| and |\ifchilddocmanual|
will be true upon processing of a part,
and the name of the part is stored in |\childdocname|.
Note that |\jobname| will be set to the filename of the current part
so that each part receives an individual |.aux| file
that does not interfere with the |.aux| file(s) of the main document.
This behaviour can be altered by the alternative form
|\childdocby[*]{|\textit{main}|}| (with a non-empty optional argument)
which uses the |.aux| file of the main document
by setting |\jobname| to \textit{main}.

%%%%%%%%%%%%%%%%%%%%%%%%%%%%%%%%%%%%%%%%%%%%%%%%%%%%%%%%%%%%%%%%%%%%%%%%%%%%%%%%
\subsection{Driver Development}
\label{sec:driver}

The \textsf{childdoc} mechanism can also be use for the development
of definition files such as \LaTeX{} styles or classes.
This case differs from the above setup with multiple parts
included by |\include| in that no |\includeonly| should be invoked.
This can be achieved by starting the include file
(before |\ProvidesPackage|) with:
%
\begin{center}
\begin{tabular}{l}
|\input{childdoc.def}|\\
|\childdocforward{|\textit{main}|}|\\
\end{tabular}
\end{center}
%
or alternatively with:
%
\begin{center}
\begin{tabular}{l}
|\input{childdoc.def}|\\
|\childdocby{|\textit{main}|}|\\
\end{tabular}
\end{center}
%
Both forms have slightly different effects as described above.
The main file is prepared as usual, see \secref{sec:include}.

%%%%%%%%%%%%%%%%%%%%%%%%%%%%%%%%%%%%%%%%%%%%%%%%%%%%%%%%%%%%%%%%%%%%%%%%%%%%%%%%
\subsection{Legacy Detection}
\label{sec:detection}

The directive |\childdocmain| in the main file can detect
whether the complete document or merely a child is to be compiled
even without using the directive |\childdocof|.
This method is deprecated because it is less robust
and there is no compelling reason to use it;
it is merely provided for backward compatibility
and it may be removed in future versions.

If the detection mechanism is to be used,
it is mandatory to correctly specify
the filename of the main file as the argument of |\childdocmain|:
%
\begin{center}
\begin{tabular}{l}
|\input{childdoc.def}|\\
|\childdocmain{|\textit{main}|}|\\
\end{tabular}
\end{center}
%
If |\jobname| does not match the argument \textit{main} of |\childdocmain|,
it is assumed that |\jobname| points to the child file to be compiled.
When using |\childdocmain| with the main file specified as argument,
it suffices to start a child file
with just |\input{|\textit{main}|}|
without loading of the package and using |\childdocof|.
If instead all processing is done
with the appropriate \textsf{childdoc} directives,
the argument of \textit{main} of |\childdocmain| can be empty.

An alternative version of the command line processing described
in \secref{sec:commandline} using the detection mechanism reads:
%
\begin{center}
|... -jobname "|\textit{target}|" "|[\textit{flags}]%
[|\def\jobname{|\textit{dest}|}|]|\input{|\textit{main}|}"|
\end{center}

%%%%%%%%%%%%%%%%%%%%%%%%%%%%%%%%%%%%%%%%%%%%%%%%%%%%%%%%%%%%%%%%%%%%%%%%%%%%%%%%
\subsection{Manual Code}
\label{sec:manual}

In case one cannot be certain whether the definitions file |childdoc.def|
is installed on the target \TeX{} distribution
and one prefers not to ship it,
it is conceivable to paste a few relevant commands into the sources.

To that end, drop all statements |\input{childdoc.def}|
and perform the replacements as outlined below.
Instead of |\childdocmain{|\textit{main}|}| add the following code
to the top of the main file:
%
\begin{center}
\begin{tabular}{l}
|\||ifdefined\childdocname\endinput\||fi\newif\ifchilddoc|\\
|\edef\childdocname{\scantokens\expandafter{\jobname\noexpand}}|\\
|\def\childdocmain{|\textit{main}|}\||ifx\childdocmain\childdocname\||else|\\
|\childdoctrue\includeonly{\childdocname}\let\jobname\childdocmain\||fi|\\
\end{tabular}
\end{center}
%
Instead of |\childdocof{|\textit{main}|}| just include the main file
at the top of each child file:
%
\begin{center}
|\input{|\textit{main}|}|
\end{center}
%
A simple redirection |\childdocforward{|\textit{dest}|}| is achieved by:
%
\begin{center}
|\def\jobname{|\textit{dest}|}\input{\jobname}|
\end{center}
%
The redirection with prefix
|\childdocforwardprefix[|\textit{prefix}|]{|\textit{dest}|}|
is accomplished by:
%
\begin{center}
\begin{tabular}{l}
|{\edef\jobname{\scantokens\expandafter{\jobname\noexpand}}|\\
|\def\redirectjob |\textit{prefix}|#1~~~{\gdef\jobname{|\textit{dest}|#1}}|\\
|\expandafter\redirectjob\jobname~~~}\input{\jobname}|
\end{tabular}
\end{center}

In an alternative approach,
child documents can be compiled by a specific command line
without additional code or specific definitions:
%
\begin{center}
|... -jobname "|\textit{target}|" "|[\textit{flags}]%
|\includeonly{|\textit{dest}|}\input{|\textit{main}|}"|
\end{center}
%

%%%%%%%%%%%%%%%%%%%%%%%%%%%%%%%%%%%%%%%%%%%%%%%%%%%%%%%%%%%%%%%%%%%%%%%%%%%%%%%%
%%%%%%%%%%%%%%%%%%%%%%%%%%%%%%%%%%%%%%%%%%%%%%%%%%%%%%%%%%%%%%%%%%%%%%%%%%%%%%%%
\section{Information}

%%%%%%%%%%%%%%%%%%%%%%%%%%%%%%%%%%%%%%%%%%%%%%%%%%%%%%%%%%%%%%%%%%%%%%%%%%%%%%%%
\subsection{Copyright}

Copyright \copyright{} 2017--2018 Niklas Beisert

This work may be distributed and/or modified under the
conditions of the \LaTeX{} Project Public License, either version 1.3
of this license or (at your option) any later version.
The latest version of this license is in
  \url{http://www.latex-project.org/lppl.txt}
and version 1.3 or later is part of all distributions of \LaTeX{}
version 2005/12/01 or later.

This work has the LPPL maintenance status `maintained'.

The Current Maintainer of this work is Niklas Beisert.

This work consists of the files |README.txt|, |childdoc.ins| and |childdoc.dtx|
as well as the derived files |childdoc.def|, |cdocsamp.tex|
with |cdocsch1.tex|, |cdocsch2.tex|, |cdocspt3.tex|, |cdocspt4.tex|,
|cdocsdrf.tex|, |cdocsfn1.tex|, |cdocsfn2.tex|
as well as |childdoc.pdf|.

%%%%%%%%%%%%%%%%%%%%%%%%%%%%%%%%%%%%%%%%%%%%%%%%%%%%%%%%%%%%%%%%%%%%%%%%%%%%%%%%
\subsection{Files and Installation}

The package consists of the files:
%
\begin{center}
\begin{tabular}{ll}
    |README.txt|   & readme file \\
    |childdoc.ins| & installation file \\
    |childdoc.dtx| & source file \\
    |childdoc.def| & definition file \\
    |cdocsamp.tex| & sample main file \\
    |cdocsch1.tex| & sample include file \\
    |cdocsch2.tex| & sample include file \\
    |cdocspt3.tex| & sample part file \\
    |cdocspt4.tex| & sample part file \\
    |cdocsdrf.tex| & sample redirection file \\
    |cdocsfn1.tex| & sample redirection file \\
    |cdocsfn2.tex| & sample redirection file \\
    |childdoc.pdf| & manual
\end{tabular}
\end{center}
%
The distribution consists of the files
|README.txt|, |childdoc.ins| and |childdoc.dtx|.
%
\begin{itemize}
\item
Run (pdf)\LaTeX{} on |childdoc.dtx|
to compile the manual |childdoc.pdf| (this file).
\item
Run \LaTeX{} on |childdoc.ins| to create the definitions file |childdoc.def|
and the sample |cdocsamp.tex| with include files
|cdocsch1.tex|, |cdocsch2.tex|, |cdocspt3.tex|, |cdocspt4.tex|,
|cdocsdrf.tex|, |cdocsfn1.tex|, |cdocsfn2.tex|.
Then copy the file |childdoc.def| to an appropriate directory of your \LaTeX{}
distribution, e.g.\ \textit{texmf-root}|/tex/latex/childdoc|.
\end{itemize}

%%%%%%%%%%%%%%%%%%%%%%%%%%%%%%%%%%%%%%%%%%%%%%%%%%%%%%%%%%%%%%%%%%%%%%%%%%%%%%%%
\subsection{Related CTAN Packages}

There are several other packages which offer a similar functionality:
%
\begin{itemize}
\item
The packages
\href{http://ctan.org/pkg/docmute}{\textsf{docmute}},
\href{http://ctan.org/pkg/includex}{\textsf{includex}} and
\href{http://ctan.org/pkg/standalone}{\textsf{standalone}}
provide commands to include only the document body of
a child file thus allowing both files to be compiled individually.
\item
The packages \href{http://ctan.org/pkg/subdocs}{\textsf{subdocs}}
and \href{http://ctan.org/pkg/subfiles}{\textsf{subfiles}}
provide structures in which the main and child documents can be
encapsulated and allowing them to be compiled individually.
The inclusion mechanism is different from the conventional |\include|.
\item
The package \href{http://ctan.org/pkg/combine}{\textsf{combine}}
is an elaborate solution to combine several documents into one.
\end{itemize}
%
See also the CTAN topic \href{http://ctan.org/topic/subdocs}{\textsf{subdocs}}
for further related packages.
The present package differs from the above solutions in that
a document structure constructed with the conventional |\include| mechanism
just needs two extra commands at the top of every file
such that all constituent files can be compiled individually.

%%%%%%%%%%%%%%%%%%%%%%%%%%%%%%%%%%%%%%%%%%%%%%%%%%%%%%%%%%%%%%%%%%%%%%%%%%%%%%%%
%\subsection{Feature Suggestions}
%
%The following is a list of features which may be useful for future
%versions of this package:
%%
%\begin{itemize}
%\item
%\ldots
%\end{itemize}

%%%%%%%%%%%%%%%%%%%%%%%%%%%%%%%%%%%%%%%%%%%%%%%%%%%%%%%%%%%%%%%%%%%%%%%%%%%%%%%%
\subsection{Revision History}

%%%%%%%%%%%%%%%%%%%%%%%%%%%%%%%%%%%%%%%%
\paragraph{v2.0:} 2018/12/30

\begin{itemize}
\item
immediate forward processing
\item
added |\childdocby| mechanism
\item
manual restructured
\end{itemize}

%%%%%%%%%%%%%%%%%%%%%%%%%%%%%%%%%%%%%%%%
\paragraph{v1.6:} 2018/01/17

\begin{itemize}
\item
application for development of include files
\item
corrections to manual
\end{itemize}

%%%%%%%%%%%%%%%%%%%%%%%%%%%%%%%%%%%%%%%%
\paragraph{v1.5:} 2017/05/21

\begin{itemize}
\item
more complete structuring introduced
\item
|\childdocof| introduced
\item
|\childdoc| renamed to |\childdocmain|
\item
|\childredirect| renamed to |\childdocforward| and |\childdocforwardprefix|
and functionality expanded
\end{itemize}

%%%%%%%%%%%%%%%%%%%%%%%%%%%%%%%%%%%%%%%%
\paragraph{v1.0:} 2017/04/27

\begin{itemize}
\item
manual and install package
\item
first version published on CTAN
\end{itemize}

%%%%%%%%%%%%%%%%%%%%%%%%%%%%%%%%%%%%%%%%
\paragraph{v0.6:} 2017/04/26

\begin{itemize}
\item
redirection mechanism added
\end{itemize}

%%%%%%%%%%%%%%%%%%%%%%%%%%%%%%%%%%%%%%%%
\paragraph{v0.5:} 2017/04/26

\begin{itemize}
\item
functionality in definition file
\end{itemize}


%%%%%%%%%%%%%%%%%%%%%%%%%%%%%%%%%%%%%%%%%%%%%%%%%%%%%%%%%%%%%%%%%%%%%%%%%%%%%%%%
%%%%%%%%%%%%%%%%%%%%%%%%%%%%%%%%%%%%%%%%%%%%%%%%%%%%%%%%%%%%%%%%%%%%%%%%%%%%%%%%
%%%%%%%%%%%%%%%%%%%%%%%%%%%%%%%%%%%%%%%%%%%%%%%%%%%%%%%%%%%%%%%%%%%%%%%%%%%%%%%%
\appendix

\settowidth\MacroIndent{\rmfamily\scriptsize 000\ }

 \DocInput{childdoc.dtx}

\end{document}
%</driver>
% \fi
%
% %%%%%%%%%%%%%%%%%%%%%%%%%%%%%%%%%%%%%%%%%%%%%%%%%%%%%%%%%%%%%%%%%%%%%%%%%%%%%%
% %%%%%%%%%%%%%%%%%%%%%%%%%%%%%%%%%%%%%%%%%%%%%%%%%%%%%%%%%%%%%%%%%%%%%%%%%%%%%%
% \section{Sample}
%\iffalse
%<*samplemain>
%\fi
%
% The following presents a sample document
% with two chapters, two parts, a title page,
% a compile flag as well as three forwarding files to set the flag.
% It consists of eight |.tex| files:
% \begin{center}
% \begin{tabular}{ll}
% |cdocsamp.tex|&main file\\
% |cdocsch1.tex|&include file for chapter 1\\
% |cdocsch2.tex|&include file for chapter 2\\
% |cdocspt3.tex|&include file for part 3\\
% |cdocspt4.tex|&include file for part 4\\
% |cdocsdrf.tex|&forwarding file for main file in draft mode\\
% |cdocsfi1.tex|&forwarding file for final version of chapter 1\\
% |cdocsfi2.tex|&forwarding file for final version of chapter 2\\
% \end{tabular}
% \end{center}
% Each of the eight files can be compiled directly by the \LaTeX{} compiler.
%
% %%%%%%%%%%%%%%%%%%%%%%%%%%%%%%%%%%%%%%
% \paragraph{Main File.}
%
% The main file is called |cdocsamp.tex|.
%
% Load the \textsf{childdoc} definitions and
% declare the filename for the main document:
%    \begin{macrocode}
\input{childdoc.def}
\childdocmain{}
%    \end{macrocode}

% Optional override for |\version| flag:
%    \begin{macrocode}
%%\ifchilddoc\else\providecommand{\version}{draft}\fi
%    \end{macrocode}

% Define the default values for the |\version| flag
% (|final| for the main file and |draft| for childs):
%    \begin{macrocode}
\ifchilddoc
\providecommand{\version}{draft}
\else
\providecommand{\version}{final}
\fi
%    \end{macrocode}

% Load the standard document class:
%    \begin{macrocode}
\documentclass[12pt]{article}
%    \end{macrocode}

% Start the document body:
%    \begin{macrocode}
\begin{document}
%    \end{macrocode}

% Declare a title page.
% Print title, part of document being processed and version flag:
%    \begin{macrocode}
\addtocounter{page}{-1}
\begin{center}
{\LARGE\bfseries{}childdoc example\par}
\vspace{1cm}
\ifchilddoc
\ifchilddocmanual part\else chapter\fi:
`\childdocname' of `\childdocjob'\par
\else
main document: `\childdocjob'\par
\fi
version: \version\par
\end{center}
\newpage
%    \end{macrocode}

% Manually include selected file,
% otherwise process as usual:
%    \begin{macrocode}
\ifchilddocmanual
\section*{part `\childdocname'}
\input{\childdocname}
\else
%    \end{macrocode}

% Include the two chapters:
%    \begin{macrocode}
\include{cdocsch1}
\include{cdocsch2}
%    \end{macrocode}

% Include the two parts unless only chapters should be displayed:
%    \begin{macrocode}
\ifchilddoc\else
\section{part three}
\input{cdocspt3}
\section{part four}
\input{cdocspt4}
\fi
%    \end{macrocode}

% Process as usual until here:
%    \begin{macrocode}
\fi
%    \end{macrocode}

% End of document body:
%    \begin{macrocode}
\end{document}
%    \end{macrocode}
%\iffalse
%</samplemain>
%\fi
%
% %%%%%%%%%%%%%%%%%%%%%%%%%%%%%%%%%%%%%%
% \paragraph{Chapter Include Files.}
%
% The include files are called |cdocsch1.tex| and |cdocsch2.tex|.
%
%\iffalse
%<*samplechap1|samplechap2>
%\fi

% Optional override for |\version| flag:
%    \begin{macrocode}
%%\providecommand{\version}{final}
%    \end{macrocode}

% Include the main document:
%    \begin{macrocode}
\input{childdoc.def}
\childdocof{cdocsamp}
%    \end{macrocode}

%\iffalse
%</samplechap1|samplechap2>
%\fi
%
%\iffalse
%<*samplechap1>
%\fi
% Some text for chapter 1:
%    \begin{macrocode}
\section{one}
some text in chapter one
%    \end{macrocode}

%\iffalse
%</samplechap1>
%\fi
% Some text for chapter 2:
%\iffalse
%<*samplechap2>
%\fi
%    \begin{macrocode}
\section{two}
more text in chapter two
%    \end{macrocode}

%\iffalse
%</samplechap2>
%\fi
%
% %%%%%%%%%%%%%%%%%%%%%%%%%%%%%%%%%%%%%%
% \paragraph{Part Include Files.}
%
% The include files are called |cdocspt3.tex| and |cdocspt4.tex|.
%
%\iffalse
%<*samplepart3|samplepart4>
%\fi

% Optional override for |\version| flag:
%    \begin{macrocode}
%%\providecommand{\version}{final}
%    \end{macrocode}

% Include the main document:
%    \begin{macrocode}
\input{childdoc.def}
\childdocby{cdocsamp}
%    \end{macrocode}

%\iffalse
%</samplepart3|samplepart4>
%\fi
%
%\iffalse
%<*samplepart3>
%\fi
% Some text for part 3:
%    \begin{macrocode}
some text in part three
%    \end{macrocode}

%\iffalse
%</samplepart3>
%\fi
% Some text for part 4:
%\iffalse
%<*samplepart4>
%\fi
%    \begin{macrocode}
more text in part four
%    \end{macrocode}

%\iffalse
%</samplepart4>
%\fi
%
% %%%%%%%%%%%%%%%%%%%%%%%%%%%%%%%%%%%%%%
% \paragraph{Forwarding for a Complete Draft.}
%
% The following forwarding file |cdocsdrf.tex|
% compiles the main document in draft mode:
%\iffalse
%<*sampledraft>
%\fi
%    \begin{macrocode}
\def\version{draft}
\input{childdoc.def}
\childdocforward{cdocsamp}
%    \end{macrocode}

%\iffalse
%</sampledraft>
%\fi
%
% %%%%%%%%%%%%%%%%%%%%%%%%%%%%%%%%%%%%%%
% \paragraph{Forwarding for Final Version of the Chapters.}
%
% The following forwarding files |cdocsfn1.tex| and |cdocsfn2.tex|
% (with identical content)
% compile the final versions of the child documents
% |cdocsch1.tex| and |cdocsch2.tex|, respectively:
%\iffalse
%<*samplefinal>
%\fi
%    \begin{macrocode}
\def\version{final}
\input{childdoc.def}
\childdocforwardprefix[cdocsamp]{cdocsfn}{cdocsch}
%    \end{macrocode}

%\iffalse
%</samplefinal>
%\fi
%
% %%%%%%%%%%%%%%%%%%%%%%%%%%%%%%%%%%%%%%
% \paragraph{Command Line Processing.}
%
% The following three command lines generate the output files
% |cdocscld|, |cdocscl1| and |cdocscl2|
% which should be identical to
% |cdocsdrf|, |cdocsch1| and |cdocsfn2|, respectively:
% \begin{center}
% \begin{tabular}{l}
% |latex -jobname cdocscld \|\\
% |  "\def\version{draft}\input{childdoc.def}\childdocforward{cdocsamp}"|\\
% |latex -jobname cdocscl1 \|\\
% |  "\input{childdoc.def}\childdocforward[cdocsamp]{cdocsch1}"|\\
% |latex -jobname cdocscl2 \|\\
% |  "\def\version{final}\input{childdoc.def}\childdocforward{cdocsch2}"|
% \end{tabular}
% \end{center}
% Note that the trailing backslash on each first line
% merely continues the input to the second line
% (for convenient cut ant paste).
% Furthermore, the command |latex| can be replaced by any
% of its alternative versions such as |pdflatex|.
%
% %%%%%%%%%%%%%%%%%%%%%%%%%%%%%%%%%%%%%%%%%%%%%%%%%%%%%%%%%%%%%%%%%%%%%%%%%%%%%%
% %%%%%%%%%%%%%%%%%%%%%%%%%%%%%%%%%%%%%%%%%%%%%%%%%%%%%%%%%%%%%%%%%%%%%%%%%%%%%%
% \section{Implementation}
%\iffalse
%<*package>
%\fi
%
% This section describes the definitions file |childdoc.def|.

% The definitions cannot be loaded using |\usepackage| or |\RequirePackage|
% which has a mechanism to prevent loading a style file more than once.
% When loading the definitions by means of |\input|
% multiple instances have to be prevented manually:
%\iffalse
%This code needs to be before the `\ProvidesFile' directive
%which is defined at the beginning of this file.
%Therefore it is also placed there and commented out here.
%</package>
%<*discard>
%\fi
%    \begin{macrocode}
\ifdefined\childdocmain\endinput\fi
%    \end{macrocode}
%\iffalse
%</discard>
%<*package>
%\fi
%
% \macro{\ifchilddoc}
% \macro{\ifchilddocmanual}
% The conditional |\ifchilddoc| tells whether a
% child (true) or main (false) document is being compiled.
% The conditional |\ifchilddocmanual| tells whether
% the |\includeonly| mechanism is used (false) or
% the selection of child files must be performed manually (true).
% The definitions initialise to false:
%    \begin{macrocode}
\newif\ifchilddoc
\newif\ifchilddocmanual
%    \end{macrocode}

% \macro{\childdocname}
% \macro{\childdocjob}
% The macro |\childdocname| stores the name of the main document
% to be compiled. The macro |\childdocjob| stores the name of
% the document on which the \LaTeX{} compiler was originally invoked.
% The content of |\jobname| cannot be compared
% to filenames specified in the source due to different catcodes.
% The following code rescans |\jobname|, stores the result
% in |\childdocname| and saves a copy in |\childdocjob|:
%    \begin{macrocode}
\edef\childdocname{\scantokens\expandafter{\jobname\noexpand}}
\let\childdocjob\childdocname
%    \end{macrocode}

% \macro{\childdocdisable}
% The macro |\childdocdisable| prevents the main file
% from being processed more than once.
% At this stage, the main document command |\childdocmain|
% is assumed to be called once again where it should do nothing.
% Any subsequent call to it should prevent
% a secondary processing of the main document
% It overwrites the forwarding commands
% |\childdocof| and |\childdocforward|
% with empty macros to prevent further inclusions of the main document:
%    \begin{macrocode}
\newcommand{\childdocdisable}
{
  \renewcommand{\childdocmain}[1]{\renewcommand{\childdocmain}[1]{\endinput}}
  \renewcommand{\childdocof}[1]{}
  \renewcommand{\childdocby}[2][]{}
  \renewcommand{\childdocforward}[2][]{}
  \renewcommand{\childdocdisable}{}
}
%    \end{macrocode}

% \macro{\childdocmain}
% The macro |\childdocmain| is to be called at the top of the main file
% with nothing or the main filename (without extension) as argument.
% First, it breaks loops.
% If the argument is not empty and does not match |\childdocname|
% (which is set by the first inclusion of |childdoc.def|),
% |\ifchilddoc| is set to true, |\includeonly| is applied to the child file
% and |\jobname| is set to the main file
% (for proper handling of |.aux| files):
%    \begin{macrocode}
\newcommand{\childdocmain}[1]
{
  \childdocdisable\childdocmain{}
  \if?#1?\else
    \begingroup
      \def\childdoctmp{#1}
      \ifx\childdoctmp\childdocname
        \def\childdoctmp{}
      \else
        \def\childdoctmp
        {
          \childdoctrue
          \includeonly{\childdocname}
          \def\childdocjob{#1}
          \def\jobname{#1}
        }
      \fi
      \expandafter
    \endgroup
    \childdoctmp
  \fi
}
%    \end{macrocode}

% \macro{\childdocof}
% The command |\childdocof| redirects
% compilation to the main file |#1|.
%    \begin{macrocode}
\newcommand{\childdocof}[1]
{
  \childdocdisable
  \childdoctrue
  \includeonly{\childdocname}
  \def\jobname{#1}
  \def\childdocjob{#1}
  \input{#1}
}
%    \end{macrocode}

% \macro{\childdocby}
% The command |\childdocby| ....
%    \begin{macrocode}
\newcommand{\childdocby}[2][]
{
  \childdocdisable
  \childdoctrue
  \childdocmanualtrue
  \if?#1?\else
    \def\jobname{#2}
  \fi
  \def\childdocjob{#2}
  \input{#2}
  \endinput
}
%    \end{macrocode}

% \macro{\childdocforward}
% The command |\childdocforward| redirects
% compilation to the main file or
% (if the optional argument is given) a child file.
% Parameters are set as if the main file
% or a child file starting with |\childdocof| was compiled.
% Then compilation is handed over to the main file:
%    \begin{macrocode}
\newcommand{\childdocforward}[2][]
{
  \begingroup
    \if?#1?
      \def\childdoctmp
      {
        \def\childdocname{#2}
        \def\childdocjob{#2}
        \def\jobname{#2}
        \input{#2}
        \endinput
      }
    \else
      \def\childdoctmp
      {
        \childdocdisable
        \def\childdocname{#2}
        \childdoctrue
        \includeonly{#2}
        \def\childdocjob{#1}
        \def\jobname{#1}
        \input{#1}
        \endinput
      }
    \fi
    \expandafter
  \endgroup
  \childdoctmp
}
%    \end{macrocode}

% \macro{\childdocforwardprefix}
% The command |\childdocforwardprefix| redirects
% compilation to the main or a child file by means of a pattern.
% The prefix |#1| in the current filename is replaced by |#2|
% and the suffix of the current filename is kept
% (it is assumed that the filename does not contain the substring `|~~~|'
% which is used as a delimiter).
% Compilation is handed over to the new file by |\childdocforward|:
%    \begin{macrocode}
\newcommand{\childdocforwardprefix}[3][]
{
  \begingroup
    \def\childdocextract #2##1~~~{\def\childdoctmp{\childdocforward[#1]{#3##1}}}
    \expandafter\childdocextract\childdocname~~~
    \expandafter
  \endgroup
  \childdoctmp
}
%    \end{macrocode}

% \macro{\childdoc}
% The deprecated macro |\childdoc| is a legacy version of |\childdocmain|:
%    \begin{macrocode}
\newcommand{\childdoc}{\childdocmain}
%    \end{macrocode}

% \macro{\childdocredirect}
% The deprecated macro |\childdocredirect| is a legacy version
% of |\childdocforward| and |\childdocforwardprefix|:
%    \begin{macrocode}
\newcommand{\childdocredirect}[2][]
{
  \begingroup
    \if?#1?
      \def\childdoctmp{\childdocforward{#2}}
    \else
      \def\childdoctmp{\childdocforwardprefix{#1}{#2}}
    \fi
    \expandafter
  \endgroup
  \childdoctmp
}
%    \end{macrocode}

%\iffalse
%</package>
%\fi
%
\endinput

\childdocforwardprefix[cdocsamp]{cdocsfn}{cdocsch}
%    \end{macrocode}

%\iffalse
%</samplefinal>
%\fi
%
% %%%%%%%%%%%%%%%%%%%%%%%%%%%%%%%%%%%%%%
% \paragraph{Command Line Processing.}
%
% The following three command lines generate the output files
% |cdocscld|, |cdocscl1| and |cdocscl2|
% which should be identical to
% |cdocsdrf|, |cdocsch1| and |cdocsfn2|, respectively:
% \begin{center}
% \begin{tabular}{l}
% |latex -jobname cdocscld \|\\
% |  "\def\version{draft}% \iffalse
%
% childdoc.dtx Copyright (C) 2017-2018 Niklas Beisert
%
% This work may be distributed and/or modified under the
% conditions of the LaTeX Project Public License, either version 1.3
% of this license or (at your option) any later version.
% The latest version of this license is in
%   http://www.latex-project.org/lppl.txt
% and version 1.3 or later is part of all distributions of LaTeX
% version 2005/12/01 or later.
%
% This work has the LPPL maintenance status `maintained'.
%
% The Current Maintainer of this work is Niklas Beisert.
%
% This work consists of the files childdoc.dtx and childdoc.ins
% and the derived files childdoc.def and cdocsamp.tex with
% cdocsch1.tex, cdocsch2.tex, cdocsdrf.tex, cdocsfn1.tex, cdocsfn2.tex.
%
%<package>\ifdefined\childdocmain\endinput\fi
%<package>\ProvidesFile{childdoc.def}[2018/12/30 v2.0 child document driver]
%<samplemain>\ProvidesFile{cdocsamp.tex}[2018/12/30 v2.0 sample for childdoc]
%<*driver>
%\ProvidesFile{childdoc.drv}[2018/12/30 v2.0 childdoc reference manual file]
\PassOptionsToClass{10pt,a4paper}{article}
\documentclass{ltxdoc}

\usepackage[margin=35mm]{geometry}
\usepackage{hyperref}
\usepackage{hyperxmp}
\usepackage[usenames]{color}

\hypersetup{colorlinks=true}
\hypersetup{pdfstartview=FitH}
\hypersetup{pdfpagemode=UseNone}
\hypersetup{pdfsource={}}
\hypersetup{pdflang={en-UK}}
\hypersetup{pdfcopyright={Copyright 2017-2018 Niklas Beisert.
  This work may be distributed and/or modified under the
  conditions of the LaTeX Project Public License, either version 1.3
  of this license or (at your option) any later version.}}
\hypersetup{pdflicenseurl={http://www.latex-project.org/lppl.txt}}
\hypersetup{pdfcontactaddress={ETH Zurich, ITP, HIT K,
  Wolfgang-Pauli-Strasse 27}}
\hypersetup{pdfcontactpostcode={8093}}
\hypersetup{pdfcontactcity={Zurich}}
\hypersetup{pdfcontactcountry={Switzerland}}
\hypersetup{pdfcontactemail={nbeisert@itp.phys.ethz.ch}}
\hypersetup{pdfcontacturl={http://people.phys.ethz.ch/\xmptilde nbeisert/}}

\newcommand{\secref}[1]{\hyperref[#1]{section \ref*{#1}}}

\parskip1ex
\parindent0pt
\let\olditemize\itemize
\def\itemize{\olditemize\parskip0pt}

\begin{document}

\title{The \textsf{childdoc} Package}
\hypersetup{pdftitle={The childdoc Package}}
\author{Niklas Beisert\\[2ex]
  Institut f\"ur Theoretische Physik\\
  Eidgen\"ossische Technische Hochschule Z\"urich\\
  Wolfgang-Pauli-Strasse 27, 8093 Z\"urich, Switzerland\\[1ex]
  \href{mailto:nbeisert@itp.phys.ethz.ch}
  {\texttt{nbeisert@itp.phys.ethz.ch}}}
\hypersetup{pdfauthor={Niklas Beisert}}
\hypersetup{pdfsubject={Manual for the LaTeX2e Package childdoc}}
\date{30 December 2018, \textsf{v2.0}}
\maketitle

\begin{abstract}\noindent
\textsf{childdoc} is a \LaTeXe{} package
that enables the direct compilation
of document sections included by |\include|
to individual files.
\end{abstract}

\begingroup
\parskip0ex
\tableofcontents
\endgroup

%%%%%%%%%%%%%%%%%%%%%%%%%%%%%%%%%%%%%%%%%%%%%%%%%%%%%%%%%%%%%%%%%%%%%%%%%%%%%%%%
%%%%%%%%%%%%%%%%%%%%%%%%%%%%%%%%%%%%%%%%%%%%%%%%%%%%%%%%%%%%%%%%%%%%%%%%%%%%%%%%
\section{Introduction}

\LaTeX{} provides a mechanism to structure a large document (such as a book)
into a main file and several child files (containing the chapters)
using the |\include| command.
This mechanism is beneficial for documents
which span hundreds of pages in order to
make the source file(s) more manageable.
Moreover, compilation can be restricted to
selected child files by means of the |\includeonly| command.
The latter feature can be used to reduce the compilation time while editing
(this was significantly more useful in the earlier days of \LaTeX{})
or to generate a smaller document which is easier to navigate.
Another application of |\includeonly| is to generate
documents consisting of selected parts of the complete document.

However, there are a few drawbacks of the plain |\include| mechanism:
\begin{itemize}
\item
The child files cannot be compiled on their own,
they can only be compiled via the main file.
A naive editing environment
(such as a text editor with an option
to have the current file processed by \LaTeX)
may require one to switch to the main file before compiling;
attempting to compile the child file produces errors.
\item
The main file must be modified (each time)
to adjust the |\includeonly| command
to the present needs. This easily leaves the main file in a messy state.
\item
The generated document will always carry the filename
of the main document. This is inconvenient if
several child files are to be compiled and
to be kept for distribution.
\end{itemize}

The present package provides a simple interface
to make child files individually compilable by \LaTeX{}.
Compiling a child file then has the same effect as compiling
the main file with an |\includeonly| command
to select the appropriate child.
Moreover the generated document will carry the name of the child
rather than the main file.
This resolves all three above issues.

This feature is meant to make the editing of books,
thesis documents and lecture notes somewhat more convenient.
However, the package can also be used efficiently for
composing a series of documents (such as exercise sheets)
which are typically distributed individually.
It then assists the author in generating the individual documents
(potentially in different versions)
as well as a document containing the collected series.
Another application is in developing style files
or other kinds of included material
where compilation of the style file could redirect
to a sample or test file.

%%%%%%%%%%%%%%%%%%%%%%%%%%%%%%%%%%%%%%%%%%%%%%%%%%%%%%%%%%%%%%%%%%%%%%%%%%%%%%%%
%%%%%%%%%%%%%%%%%%%%%%%%%%%%%%%%%%%%%%%%%%%%%%%%%%%%%%%%%%%%%%%%%%%%%%%%%%%%%%%%
\section{Usage}

First of all, the package \textsf{childdoc} is \emph{not} a standard
\LaTeXe{} |.sty| style file! Therefore it needs to be invoked in
a non-standard way.

%%%%%%%%%%%%%%%%%%%%%%%%%%%%%%%%%%%%%%%%%%%%%%%%%%%%%%%%%%%%%%%%%%%%%%%%%%%%%%%%
\subsection{Included Files}
\label{sec:include}

%%%%%%%%%%%%%%%%%%%%%%%%%%%%%%%%%%%%%%%%
\DescribeMacro{\childdocmain}
To use the package, add the commands
\begin{center}
\begin{tabular}{l}
|\input{childdoc.def}|\\
|\childdocmain{}|\\
\end{tabular}
\end{center}
at the very top of the main \LaTeX{} file,
in particular \emph{before} the |\documentclass| statement!
The argument of |\childdocmain| should be left empty
(but it must be present).

%%%%%%%%%%%%%%%%%%%%%%%%%%%%%%%%%%%%%%%%
\DescribeMacro{\childdocof}
Furthermore, add the commands
\begin{center}
\begin{tabular}{l}
|\input{childdoc.def}|\\
|\childdocof{|\textit{main}|}|\\
\end{tabular}
\end{center}
at the top of every child file \textit{child}
which is included by |\include{|\textit{child}|}|
from within the main file
(or at least for those files to be compiled individually).
The argument \textit{main} must be the filename of the main file.

There are a couple of
considerations in setting up the main and child documents:

%%%%%%%%%%%%%%%%%%%%%%%%%%%%%%%%%%%%%%%%
\paragraph{Restrictions.}

Please note the following restrictions:
\begin{itemize}
\item
|\childdocmain| must be called with one argument \textit{main}
to ensure compatibility with earlier version of the package.
It must either be empty (|\childdocmain{}|)
or precisely match the filename of the main file in which it is specified.
See \secref{sec:detection} for further information.
\item
The filename \textit{main} must be specified without the |.tex| extension.
\item
The filename \textit{main} is case sensitive
(even in case-insensitive file systems)
due to internal string comparison.
\item
The argument \textit{main} should be fully expanded, it cannot be a macro.
\item
Subdirectories and special characters should be avoided in filenames.
\item
The command |\childdocmain{|\textit{main}|}| must be followed by a whitespace.
It should not be followed immediately by another command
or by a comment mark `|%|'.
This is because the \TeX{} parser reads the token immediately following
the argument of |\childdocmain| and puts it
at the beginning of every child section;
however, a white\-space is ignored.
\end{itemize}

%%%%%%%%%%%%%%%%%%%%%%%%%%%%%%%%%%%%%%%%
\paragraph{Content of Main File.}

It is advisable to place all content in the child files included by |\include|.
Any output contained in the main file will appear in all child documents
unless suppressed manually;
it cannot be suppressed automatically by the |\includeonly| directive
and thus should normally be avoided.
A method to include some content in the main file
by means of conditional processing is described in \secref{sec:conditional}.

%%%%%%%%%%%%%%%%%%%%%%%%%%%%%%%%%%%%%%%%
\paragraph{Page Numbering.}

When only a part of the document is compiled,
the appropriate numbering of pages
(as well as other status parameters)
is determined from the |.aux| files.
The latter contain information from previous passes.
However this information needs to propagate through
all intermediate child documents.
Therefore the page numbering in child documents may well
be inconsistent until the complete document is compiled at least once.

A useful (if unconventional) way to always ensure a consistent
page numbering is to restart the numbering in each child document
and denote the pages by `\textit{child}|.|\textit{page}'
where \textit{child} represents the chapter/section number of the child file.
This can be achieved by the command
|\numberwithin{page}{|\textit{child}|}|
of the \textsf{amsmath} package
where \textit{child} can be |chapter| or |section|
depending on the chosen structuring.
Alternatively, one can modify the macro |\thepage| appropriately
and reset the counter |page| at the start of each child file.

%%%%%%%%%%%%%%%%%%%%%%%%%%%%%%%%%%%%%%%%%%%%%%%%%%%%%%%%%%%%%%%%%%%%%%%%%%%%%%%%
\subsection{Conditional Processing}
\label{sec:conditional}

The package provides a mechanism to compile different versions
of a document. To customise the versions further some conditional processing
can come in handy to distinguish which version is being compiled.
The package provides two macros to describe the compilation context:

%%%%%%%%%%%%%%%%%%%%%%%%%%%%%%%%%%%%%%%%
\DescribeMacro{\ifchilddoc}
The conditional |\ifchilddoc| distinguishes between the compilation of
child documents and the main document:
%
\begin{center}
|\ifchilddoc |\textit{child-code}| |[|\||else |\textit{main-code}]| \||fi|
\end{center}

%%%%%%%%%%%%%%%%%%%%%%%%%%%%%%%%%%%%%%%%
\DescribeMacro{\childdocname}
\DescribeMacro{\childdocjob}
The macro |\childdocname| contains the filename (without extension)
of the main or child file being processed.
Note that |\childdocjob| will always contain the name of the main file.

%%%%%%%%%%%%%%%%%%%%%%%%%%%%%%%%%%%%%%%%
\paragraph{Title Page.}

Conditional processing can be used to include a title or banner page
in the main document when proper precautions are taken.
Importantly, the code in the main file should ensure that the page counter
(as well as other status parameters which are stored in the |.aux| files)
takes the same value after the conditional processing.
Otherwise the page numbers may take divergent values
depending on which part is compiled.

For example, a title page could be declared by:
%
\begin{center}
\begin{tabular}{l}
|\ifchilddoc\||else|\\
|\addtocounter{page}{-1}|\\
\textit{code for title page}\\
|\newpage|\\
|\||fi|
\end{tabular}
\end{center}
%
A banner page for the child documents can be generated by:
%
\begin{center}
\begin{tabular}{l}
|\ifchilddoc|\\
|\addtocounter{page}{-1}|\\
\textit{code for banner page}\\
|\newpage|\\
|\||fi|
\end{tabular}
\end{center}
%
Here one could write a message such as:
\begin{center}
|This is the part \childdocname{} of \childdocjob{}.|
\end{center}

%%%%%%%%%%%%%%%%%%%%%%%%%%%%%%%%%%%%%%%%%%%%%%%%%%%%%%%%%%%%%%%%%%%%%%%%%%%%%%%%
\subsection{Flags}
\label{sec:flags}

The package makes it easy to generate different versions
of the main or child documents.
To this end compilation flags can be defined
and assigned different default values.
They will be particularly useful in conjunction
with the forwarding mechanism described in \secref{sec:forward}.

For example, it may be useful to have a flag |\version|
which can be set to |draft| or |final|.
The document source will contain some conditional code
depending on the value of |\version|.
Suppose further, the flag should default to |final| for the main file
and to |draft| for child files
which is a natural assignment for editing the document.
This is achieved by placing the following code
in the preamble of the main document
(below the |\childdocmain| directive):
%
\begin{center}
\begin{tabular}{l}
|\ifchilddoc|\\
|\providecommand{\version}{draft}|\\
|\||else|\\
|\providecommand{\version}{final}|\\
|\||fi|
\end{tabular}
\end{center}
%
The definition by |\providecommand| makes sure
that previous definitions are not overwritten.
Further statements |\providecommand{\version}{...}|
can thus be added before the above code to override it.

For the main file, one might add a line
(between |\childdocmain| and the above block)
%
\begin{center}
|%\ifchilddoc\||else\providecommand{\version}{draft}\||fi|
\end{center}
%
which can be uncommented to produce a draft version.
Likewise one can add a line to the very top of a child file
(above the |\childdocof{|\textit{main}|}| directive)
%
\begin{center}
|%\providecommand{\version}{final}|
\end{center}
%
which can be uncommented to produce the final version of this child document.

%%%%%%%%%%%%%%%%%%%%%%%%%%%%%%%%%%%%%%%%%%%%%%%%%%%%%%%%%%%%%%%%%%%%%%%%%%%%%%%%
\subsection{Forwarding}
\label{sec:forward}

Different versions of the main or child documents
using compilation flags as described in \secref{sec:flags}
can be (permanently) stored in different files
for convenient compilation, viewing and distribution.
To this end, the package defines a command
to pass on compilation to a different file:

%%%%%%%%%%%%%%%%%%%%%%%%%%%%%%%%%%%%%%%%
\DescribeMacro{\childdocforward}
The command |\childdocforward| redirects processing to
another source file:
%
\begin{center}
\begin{tabular}{l}
|\input{childdoc.def}|\\
|\childdocforward[|\textit{main}|]{|\textit{dest}|}|\\
\end{tabular}
\end{center}
%
The argument \textit{dest} is the destination file
(without extension).
It should be the main file or one of the child files.
Note that further \textsf{childdoc} directives
such as |\childdocof| and |\childdocforward|
in the indicated file will be processed in this form.
The optional argument \textit{main}
passes on directly to the main file \textit{main}
while pretending to compile the child \textit{dest}.
This form behaves as if \textit{dest}
issues |\childdocof{|\textit{main}|}| right away,
and no further \textsf{childdoc} directives will be processed.

%%%%%%%%%%%%%%%%%%%%%%%%%%%%%%%%%%%%%%%%
\DescribeMacro{\...prefix}
In the alternative form |\childdocforwardprefix|,
%
\begin{center}
\begin{tabular}{l}
|\input{childdoc.def}|\\
|\childdocforwardprefix[|\textit{main}|]{|\textit{prefix}|}{|\textit{dest}|}|
\end{tabular}
\end{center}
%
the destination file is determined by a pattern
depending on the current file:
To make this work, the current file must be called
`{\textit{prefix}\hspace{0.2em}\textit{suffix}}'
with \textit{prefix} matching precisely the argument.
Processing is then passed on to the file
`{\textit{dest}\hspace{0.2em}\textit{suffix}}'.
Surely, the same effect is achieved by
directly specifying the
argument `{\textit{dest}\hspace{0.2em}\textit{suffix}}'
in the first form.
However, that requires to set up a different file
for each child. With the alternative form of the command
all these files can have exactly the same content
which simplifies setting them up and maintaining them.

For example, the following file |draft.tex|
with a compilation flag |\version| as described in \secref{sec:flags}
compiles the main document as a draft:
%
\begin{center}
\begin{tabular}{l}
|\def\version{draft}|\\
|\input{childdoc.def}|\\
|\childdocforward{|\textit{main}|}|
\end{tabular}
\end{center}
%
Likewise, the following files |final|\textit{nn}|.tex|
compile the final version of the child document
|child|\textit{nn}|.tex|:
%
\begin{center}
\begin{tabular}{l}
|\def\version{final}|\\
|\input{childdoc.def}|\\
|\childdocforwardprefix{final}{child}|
\end{tabular}
\end{center}
%

Note that when several versions of a main file and/or of each child file
are to be generated, it may be convenient to set up a |Makefile| or
shell script to automatise the process.

%%%%%%%%%%%%%%%%%%%%%%%%%%%%%%%%%%%%%%%%%%%%%%%%%%%%%%%%%%%%%%%%%%%%%%%%%%%%%%%%
\subsection{Command Line Processing}
\label{sec:commandline}

The effect of redirection files can also be achieved by invoking
the \LaTeX{} compiler with a more elaborate command line.
Most conveniently this should be done as part
of a shell script or a |Makefile|.

When using \textsf{childdoc} in the main file, the following
command lines effectively perform a redirection
(note that depending on the shell being used,
backslashes may have to be doubled: `|\|' $\to$ `|\\|'):
%
\begin{center}
|... -jobname "|\textit{target}|" |\\|"|[\textit{flags}]%
|\input{childdoc.def}\childdocforward[|\textit{main}|]{|\textit{dest}|}"|
\end{center}
%
Here \textit{target} is the name of the output file,
\textit{main} is the name of the main file
and \textit{dest} is the name of the main or child file to be processed
(all filenames without extensions).
The optional argument \textit{main} can be omitted
if \textit{main} matches \textit{dest}.
Optionally, compilation \textit{flags} can be defined via |\def| commands.
This command line makes the \TeX{} engine believe
it is compiling the file \textit{target}
whose content is specified as the latter parameter.
The provided code then forwards the processing to
\textit{main} or \textit{dest} as described in \secref{sec:forward}.

%%%%%%%%%%%%%%%%%%%%%%%%%%%%%%%%%%%%%%%%%%%%%%%%%%%%%%%%%%%%%%%%%%%%%%%%%%%%%%%%
\subsection{Include by Input}
\label{sec:input}

Including child documents by |\include| has some restrictions by design.
Most notably, the content of a child document always occupies
its own set of pages; pages cannot be shared between child documents.
Usually, this behaviour makes perfect sense
because each child document contain an essential part of the document.
However, in some situations it may be desirable to compose
a document from a collection of parts
without having mandatory page breaks between then.
For this case, the package
provides a mechanism to include parts
by |\input| which can also be processed individually.
However, by construction this mechanism
requires manual handling of the content to be output.

%%%%%%%%%%%%%%%%%%%%%%%%%%%%%%%%%%%%%%%%
\DescribeMacro{\ifchilddocmanual}
The main file should be prepared as usual, see \secref{sec:include}.
However, the document body must make a distinction
between processing of an individual part and of the main document, e.g.:
%
\begin{center}
\begin{tabular}{l}
|\ifchilddocmanual|\\
|\input{\childdocname}|\\
|\||else|\\
\textit{document body with }|\input{|\textit{part}|}|\\
|\||fi|
\end{tabular}
\end{center}
%
The conditional |\ifchilddocmanual| is true whenever
a part to be included by |\input| is being compiled,
and the name of the part is stored in |\childdocname|.

%%%%%%%%%%%%%%%%%%%%%%%%%%%%%%%%%%%%%%%%
\DescribeMacro{\childdocby}
Each part to be included by |\input| should start with:
%
\begin{center}
\begin{tabular}{l}
|\input{childdoc.def}|\\
|\childdocby{|\textit{main}|}|\\
\end{tabular}
\end{center}
%
The directive |\childdocby| is similar to |\childdocof|
described in \secref{sec:include},
but the subsequent selection of content must be done manually.
To that end, both |\ifchilddoc| and |\ifchilddocmanual|
will be true upon processing of a part,
and the name of the part is stored in |\childdocname|.
Note that |\jobname| will be set to the filename of the current part
so that each part receives an individual |.aux| file
that does not interfere with the |.aux| file(s) of the main document.
This behaviour can be altered by the alternative form
|\childdocby[*]{|\textit{main}|}| (with a non-empty optional argument)
which uses the |.aux| file of the main document
by setting |\jobname| to \textit{main}.

%%%%%%%%%%%%%%%%%%%%%%%%%%%%%%%%%%%%%%%%%%%%%%%%%%%%%%%%%%%%%%%%%%%%%%%%%%%%%%%%
\subsection{Driver Development}
\label{sec:driver}

The \textsf{childdoc} mechanism can also be use for the development
of definition files such as \LaTeX{} styles or classes.
This case differs from the above setup with multiple parts
included by |\include| in that no |\includeonly| should be invoked.
This can be achieved by starting the include file
(before |\ProvidesPackage|) with:
%
\begin{center}
\begin{tabular}{l}
|\input{childdoc.def}|\\
|\childdocforward{|\textit{main}|}|\\
\end{tabular}
\end{center}
%
or alternatively with:
%
\begin{center}
\begin{tabular}{l}
|\input{childdoc.def}|\\
|\childdocby{|\textit{main}|}|\\
\end{tabular}
\end{center}
%
Both forms have slightly different effects as described above.
The main file is prepared as usual, see \secref{sec:include}.

%%%%%%%%%%%%%%%%%%%%%%%%%%%%%%%%%%%%%%%%%%%%%%%%%%%%%%%%%%%%%%%%%%%%%%%%%%%%%%%%
\subsection{Legacy Detection}
\label{sec:detection}

The directive |\childdocmain| in the main file can detect
whether the complete document or merely a child is to be compiled
even without using the directive |\childdocof|.
This method is deprecated because it is less robust
and there is no compelling reason to use it;
it is merely provided for backward compatibility
and it may be removed in future versions.

If the detection mechanism is to be used,
it is mandatory to correctly specify
the filename of the main file as the argument of |\childdocmain|:
%
\begin{center}
\begin{tabular}{l}
|\input{childdoc.def}|\\
|\childdocmain{|\textit{main}|}|\\
\end{tabular}
\end{center}
%
If |\jobname| does not match the argument \textit{main} of |\childdocmain|,
it is assumed that |\jobname| points to the child file to be compiled.
When using |\childdocmain| with the main file specified as argument,
it suffices to start a child file
with just |\input{|\textit{main}|}|
without loading of the package and using |\childdocof|.
If instead all processing is done
with the appropriate \textsf{childdoc} directives,
the argument of \textit{main} of |\childdocmain| can be empty.

An alternative version of the command line processing described
in \secref{sec:commandline} using the detection mechanism reads:
%
\begin{center}
|... -jobname "|\textit{target}|" "|[\textit{flags}]%
[|\def\jobname{|\textit{dest}|}|]|\input{|\textit{main}|}"|
\end{center}

%%%%%%%%%%%%%%%%%%%%%%%%%%%%%%%%%%%%%%%%%%%%%%%%%%%%%%%%%%%%%%%%%%%%%%%%%%%%%%%%
\subsection{Manual Code}
\label{sec:manual}

In case one cannot be certain whether the definitions file |childdoc.def|
is installed on the target \TeX{} distribution
and one prefers not to ship it,
it is conceivable to paste a few relevant commands into the sources.

To that end, drop all statements |\input{childdoc.def}|
and perform the replacements as outlined below.
Instead of |\childdocmain{|\textit{main}|}| add the following code
to the top of the main file:
%
\begin{center}
\begin{tabular}{l}
|\||ifdefined\childdocname\endinput\||fi\newif\ifchilddoc|\\
|\edef\childdocname{\scantokens\expandafter{\jobname\noexpand}}|\\
|\def\childdocmain{|\textit{main}|}\||ifx\childdocmain\childdocname\||else|\\
|\childdoctrue\includeonly{\childdocname}\let\jobname\childdocmain\||fi|\\
\end{tabular}
\end{center}
%
Instead of |\childdocof{|\textit{main}|}| just include the main file
at the top of each child file:
%
\begin{center}
|\input{|\textit{main}|}|
\end{center}
%
A simple redirection |\childdocforward{|\textit{dest}|}| is achieved by:
%
\begin{center}
|\def\jobname{|\textit{dest}|}\input{\jobname}|
\end{center}
%
The redirection with prefix
|\childdocforwardprefix[|\textit{prefix}|]{|\textit{dest}|}|
is accomplished by:
%
\begin{center}
\begin{tabular}{l}
|{\edef\jobname{\scantokens\expandafter{\jobname\noexpand}}|\\
|\def\redirectjob |\textit{prefix}|#1~~~{\gdef\jobname{|\textit{dest}|#1}}|\\
|\expandafter\redirectjob\jobname~~~}\input{\jobname}|
\end{tabular}
\end{center}

In an alternative approach,
child documents can be compiled by a specific command line
without additional code or specific definitions:
%
\begin{center}
|... -jobname "|\textit{target}|" "|[\textit{flags}]%
|\includeonly{|\textit{dest}|}\input{|\textit{main}|}"|
\end{center}
%

%%%%%%%%%%%%%%%%%%%%%%%%%%%%%%%%%%%%%%%%%%%%%%%%%%%%%%%%%%%%%%%%%%%%%%%%%%%%%%%%
%%%%%%%%%%%%%%%%%%%%%%%%%%%%%%%%%%%%%%%%%%%%%%%%%%%%%%%%%%%%%%%%%%%%%%%%%%%%%%%%
\section{Information}

%%%%%%%%%%%%%%%%%%%%%%%%%%%%%%%%%%%%%%%%%%%%%%%%%%%%%%%%%%%%%%%%%%%%%%%%%%%%%%%%
\subsection{Copyright}

Copyright \copyright{} 2017--2018 Niklas Beisert

This work may be distributed and/or modified under the
conditions of the \LaTeX{} Project Public License, either version 1.3
of this license or (at your option) any later version.
The latest version of this license is in
  \url{http://www.latex-project.org/lppl.txt}
and version 1.3 or later is part of all distributions of \LaTeX{}
version 2005/12/01 or later.

This work has the LPPL maintenance status `maintained'.

The Current Maintainer of this work is Niklas Beisert.

This work consists of the files |README.txt|, |childdoc.ins| and |childdoc.dtx|
as well as the derived files |childdoc.def|, |cdocsamp.tex|
with |cdocsch1.tex|, |cdocsch2.tex|, |cdocspt3.tex|, |cdocspt4.tex|,
|cdocsdrf.tex|, |cdocsfn1.tex|, |cdocsfn2.tex|
as well as |childdoc.pdf|.

%%%%%%%%%%%%%%%%%%%%%%%%%%%%%%%%%%%%%%%%%%%%%%%%%%%%%%%%%%%%%%%%%%%%%%%%%%%%%%%%
\subsection{Files and Installation}

The package consists of the files:
%
\begin{center}
\begin{tabular}{ll}
    |README.txt|   & readme file \\
    |childdoc.ins| & installation file \\
    |childdoc.dtx| & source file \\
    |childdoc.def| & definition file \\
    |cdocsamp.tex| & sample main file \\
    |cdocsch1.tex| & sample include file \\
    |cdocsch2.tex| & sample include file \\
    |cdocspt3.tex| & sample part file \\
    |cdocspt4.tex| & sample part file \\
    |cdocsdrf.tex| & sample redirection file \\
    |cdocsfn1.tex| & sample redirection file \\
    |cdocsfn2.tex| & sample redirection file \\
    |childdoc.pdf| & manual
\end{tabular}
\end{center}
%
The distribution consists of the files
|README.txt|, |childdoc.ins| and |childdoc.dtx|.
%
\begin{itemize}
\item
Run (pdf)\LaTeX{} on |childdoc.dtx|
to compile the manual |childdoc.pdf| (this file).
\item
Run \LaTeX{} on |childdoc.ins| to create the definitions file |childdoc.def|
and the sample |cdocsamp.tex| with include files
|cdocsch1.tex|, |cdocsch2.tex|, |cdocspt3.tex|, |cdocspt4.tex|,
|cdocsdrf.tex|, |cdocsfn1.tex|, |cdocsfn2.tex|.
Then copy the file |childdoc.def| to an appropriate directory of your \LaTeX{}
distribution, e.g.\ \textit{texmf-root}|/tex/latex/childdoc|.
\end{itemize}

%%%%%%%%%%%%%%%%%%%%%%%%%%%%%%%%%%%%%%%%%%%%%%%%%%%%%%%%%%%%%%%%%%%%%%%%%%%%%%%%
\subsection{Related CTAN Packages}

There are several other packages which offer a similar functionality:
%
\begin{itemize}
\item
The packages
\href{http://ctan.org/pkg/docmute}{\textsf{docmute}},
\href{http://ctan.org/pkg/includex}{\textsf{includex}} and
\href{http://ctan.org/pkg/standalone}{\textsf{standalone}}
provide commands to include only the document body of
a child file thus allowing both files to be compiled individually.
\item
The packages \href{http://ctan.org/pkg/subdocs}{\textsf{subdocs}}
and \href{http://ctan.org/pkg/subfiles}{\textsf{subfiles}}
provide structures in which the main and child documents can be
encapsulated and allowing them to be compiled individually.
The inclusion mechanism is different from the conventional |\include|.
\item
The package \href{http://ctan.org/pkg/combine}{\textsf{combine}}
is an elaborate solution to combine several documents into one.
\end{itemize}
%
See also the CTAN topic \href{http://ctan.org/topic/subdocs}{\textsf{subdocs}}
for further related packages.
The present package differs from the above solutions in that
a document structure constructed with the conventional |\include| mechanism
just needs two extra commands at the top of every file
such that all constituent files can be compiled individually.

%%%%%%%%%%%%%%%%%%%%%%%%%%%%%%%%%%%%%%%%%%%%%%%%%%%%%%%%%%%%%%%%%%%%%%%%%%%%%%%%
%\subsection{Feature Suggestions}
%
%The following is a list of features which may be useful for future
%versions of this package:
%%
%\begin{itemize}
%\item
%\ldots
%\end{itemize}

%%%%%%%%%%%%%%%%%%%%%%%%%%%%%%%%%%%%%%%%%%%%%%%%%%%%%%%%%%%%%%%%%%%%%%%%%%%%%%%%
\subsection{Revision History}

%%%%%%%%%%%%%%%%%%%%%%%%%%%%%%%%%%%%%%%%
\paragraph{v2.0:} 2018/12/30

\begin{itemize}
\item
immediate forward processing
\item
added |\childdocby| mechanism
\item
manual restructured
\end{itemize}

%%%%%%%%%%%%%%%%%%%%%%%%%%%%%%%%%%%%%%%%
\paragraph{v1.6:} 2018/01/17

\begin{itemize}
\item
application for development of include files
\item
corrections to manual
\end{itemize}

%%%%%%%%%%%%%%%%%%%%%%%%%%%%%%%%%%%%%%%%
\paragraph{v1.5:} 2017/05/21

\begin{itemize}
\item
more complete structuring introduced
\item
|\childdocof| introduced
\item
|\childdoc| renamed to |\childdocmain|
\item
|\childredirect| renamed to |\childdocforward| and |\childdocforwardprefix|
and functionality expanded
\end{itemize}

%%%%%%%%%%%%%%%%%%%%%%%%%%%%%%%%%%%%%%%%
\paragraph{v1.0:} 2017/04/27

\begin{itemize}
\item
manual and install package
\item
first version published on CTAN
\end{itemize}

%%%%%%%%%%%%%%%%%%%%%%%%%%%%%%%%%%%%%%%%
\paragraph{v0.6:} 2017/04/26

\begin{itemize}
\item
redirection mechanism added
\end{itemize}

%%%%%%%%%%%%%%%%%%%%%%%%%%%%%%%%%%%%%%%%
\paragraph{v0.5:} 2017/04/26

\begin{itemize}
\item
functionality in definition file
\end{itemize}


%%%%%%%%%%%%%%%%%%%%%%%%%%%%%%%%%%%%%%%%%%%%%%%%%%%%%%%%%%%%%%%%%%%%%%%%%%%%%%%%
%%%%%%%%%%%%%%%%%%%%%%%%%%%%%%%%%%%%%%%%%%%%%%%%%%%%%%%%%%%%%%%%%%%%%%%%%%%%%%%%
%%%%%%%%%%%%%%%%%%%%%%%%%%%%%%%%%%%%%%%%%%%%%%%%%%%%%%%%%%%%%%%%%%%%%%%%%%%%%%%%
\appendix

\settowidth\MacroIndent{\rmfamily\scriptsize 000\ }

 \DocInput{childdoc.dtx}

\end{document}
%</driver>
% \fi
%
% %%%%%%%%%%%%%%%%%%%%%%%%%%%%%%%%%%%%%%%%%%%%%%%%%%%%%%%%%%%%%%%%%%%%%%%%%%%%%%
% %%%%%%%%%%%%%%%%%%%%%%%%%%%%%%%%%%%%%%%%%%%%%%%%%%%%%%%%%%%%%%%%%%%%%%%%%%%%%%
% \section{Sample}
%\iffalse
%<*samplemain>
%\fi
%
% The following presents a sample document
% with two chapters, two parts, a title page,
% a compile flag as well as three forwarding files to set the flag.
% It consists of eight |.tex| files:
% \begin{center}
% \begin{tabular}{ll}
% |cdocsamp.tex|&main file\\
% |cdocsch1.tex|&include file for chapter 1\\
% |cdocsch2.tex|&include file for chapter 2\\
% |cdocspt3.tex|&include file for part 3\\
% |cdocspt4.tex|&include file for part 4\\
% |cdocsdrf.tex|&forwarding file for main file in draft mode\\
% |cdocsfi1.tex|&forwarding file for final version of chapter 1\\
% |cdocsfi2.tex|&forwarding file for final version of chapter 2\\
% \end{tabular}
% \end{center}
% Each of the eight files can be compiled directly by the \LaTeX{} compiler.
%
% %%%%%%%%%%%%%%%%%%%%%%%%%%%%%%%%%%%%%%
% \paragraph{Main File.}
%
% The main file is called |cdocsamp.tex|.
%
% Load the \textsf{childdoc} definitions and
% declare the filename for the main document:
%    \begin{macrocode}
\input{childdoc.def}
\childdocmain{}
%    \end{macrocode}

% Optional override for |\version| flag:
%    \begin{macrocode}
%%\ifchilddoc\else\providecommand{\version}{draft}\fi
%    \end{macrocode}

% Define the default values for the |\version| flag
% (|final| for the main file and |draft| for childs):
%    \begin{macrocode}
\ifchilddoc
\providecommand{\version}{draft}
\else
\providecommand{\version}{final}
\fi
%    \end{macrocode}

% Load the standard document class:
%    \begin{macrocode}
\documentclass[12pt]{article}
%    \end{macrocode}

% Start the document body:
%    \begin{macrocode}
\begin{document}
%    \end{macrocode}

% Declare a title page.
% Print title, part of document being processed and version flag:
%    \begin{macrocode}
\addtocounter{page}{-1}
\begin{center}
{\LARGE\bfseries{}childdoc example\par}
\vspace{1cm}
\ifchilddoc
\ifchilddocmanual part\else chapter\fi:
`\childdocname' of `\childdocjob'\par
\else
main document: `\childdocjob'\par
\fi
version: \version\par
\end{center}
\newpage
%    \end{macrocode}

% Manually include selected file,
% otherwise process as usual:
%    \begin{macrocode}
\ifchilddocmanual
\section*{part `\childdocname'}
\input{\childdocname}
\else
%    \end{macrocode}

% Include the two chapters:
%    \begin{macrocode}
\include{cdocsch1}
\include{cdocsch2}
%    \end{macrocode}

% Include the two parts unless only chapters should be displayed:
%    \begin{macrocode}
\ifchilddoc\else
\section{part three}
\input{cdocspt3}
\section{part four}
\input{cdocspt4}
\fi
%    \end{macrocode}

% Process as usual until here:
%    \begin{macrocode}
\fi
%    \end{macrocode}

% End of document body:
%    \begin{macrocode}
\end{document}
%    \end{macrocode}
%\iffalse
%</samplemain>
%\fi
%
% %%%%%%%%%%%%%%%%%%%%%%%%%%%%%%%%%%%%%%
% \paragraph{Chapter Include Files.}
%
% The include files are called |cdocsch1.tex| and |cdocsch2.tex|.
%
%\iffalse
%<*samplechap1|samplechap2>
%\fi

% Optional override for |\version| flag:
%    \begin{macrocode}
%%\providecommand{\version}{final}
%    \end{macrocode}

% Include the main document:
%    \begin{macrocode}
\input{childdoc.def}
\childdocof{cdocsamp}
%    \end{macrocode}

%\iffalse
%</samplechap1|samplechap2>
%\fi
%
%\iffalse
%<*samplechap1>
%\fi
% Some text for chapter 1:
%    \begin{macrocode}
\section{one}
some text in chapter one
%    \end{macrocode}

%\iffalse
%</samplechap1>
%\fi
% Some text for chapter 2:
%\iffalse
%<*samplechap2>
%\fi
%    \begin{macrocode}
\section{two}
more text in chapter two
%    \end{macrocode}

%\iffalse
%</samplechap2>
%\fi
%
% %%%%%%%%%%%%%%%%%%%%%%%%%%%%%%%%%%%%%%
% \paragraph{Part Include Files.}
%
% The include files are called |cdocspt3.tex| and |cdocspt4.tex|.
%
%\iffalse
%<*samplepart3|samplepart4>
%\fi

% Optional override for |\version| flag:
%    \begin{macrocode}
%%\providecommand{\version}{final}
%    \end{macrocode}

% Include the main document:
%    \begin{macrocode}
\input{childdoc.def}
\childdocby{cdocsamp}
%    \end{macrocode}

%\iffalse
%</samplepart3|samplepart4>
%\fi
%
%\iffalse
%<*samplepart3>
%\fi
% Some text for part 3:
%    \begin{macrocode}
some text in part three
%    \end{macrocode}

%\iffalse
%</samplepart3>
%\fi
% Some text for part 4:
%\iffalse
%<*samplepart4>
%\fi
%    \begin{macrocode}
more text in part four
%    \end{macrocode}

%\iffalse
%</samplepart4>
%\fi
%
% %%%%%%%%%%%%%%%%%%%%%%%%%%%%%%%%%%%%%%
% \paragraph{Forwarding for a Complete Draft.}
%
% The following forwarding file |cdocsdrf.tex|
% compiles the main document in draft mode:
%\iffalse
%<*sampledraft>
%\fi
%    \begin{macrocode}
\def\version{draft}
\input{childdoc.def}
\childdocforward{cdocsamp}
%    \end{macrocode}

%\iffalse
%</sampledraft>
%\fi
%
% %%%%%%%%%%%%%%%%%%%%%%%%%%%%%%%%%%%%%%
% \paragraph{Forwarding for Final Version of the Chapters.}
%
% The following forwarding files |cdocsfn1.tex| and |cdocsfn2.tex|
% (with identical content)
% compile the final versions of the child documents
% |cdocsch1.tex| and |cdocsch2.tex|, respectively:
%\iffalse
%<*samplefinal>
%\fi
%    \begin{macrocode}
\def\version{final}
\input{childdoc.def}
\childdocforwardprefix[cdocsamp]{cdocsfn}{cdocsch}
%    \end{macrocode}

%\iffalse
%</samplefinal>
%\fi
%
% %%%%%%%%%%%%%%%%%%%%%%%%%%%%%%%%%%%%%%
% \paragraph{Command Line Processing.}
%
% The following three command lines generate the output files
% |cdocscld|, |cdocscl1| and |cdocscl2|
% which should be identical to
% |cdocsdrf|, |cdocsch1| and |cdocsfn2|, respectively:
% \begin{center}
% \begin{tabular}{l}
% |latex -jobname cdocscld \|\\
% |  "\def\version{draft}\input{childdoc.def}\childdocforward{cdocsamp}"|\\
% |latex -jobname cdocscl1 \|\\
% |  "\input{childdoc.def}\childdocforward[cdocsamp]{cdocsch1}"|\\
% |latex -jobname cdocscl2 \|\\
% |  "\def\version{final}\input{childdoc.def}\childdocforward{cdocsch2}"|
% \end{tabular}
% \end{center}
% Note that the trailing backslash on each first line
% merely continues the input to the second line
% (for convenient cut ant paste).
% Furthermore, the command |latex| can be replaced by any
% of its alternative versions such as |pdflatex|.
%
% %%%%%%%%%%%%%%%%%%%%%%%%%%%%%%%%%%%%%%%%%%%%%%%%%%%%%%%%%%%%%%%%%%%%%%%%%%%%%%
% %%%%%%%%%%%%%%%%%%%%%%%%%%%%%%%%%%%%%%%%%%%%%%%%%%%%%%%%%%%%%%%%%%%%%%%%%%%%%%
% \section{Implementation}
%\iffalse
%<*package>
%\fi
%
% This section describes the definitions file |childdoc.def|.

% The definitions cannot be loaded using |\usepackage| or |\RequirePackage|
% which has a mechanism to prevent loading a style file more than once.
% When loading the definitions by means of |\input|
% multiple instances have to be prevented manually:
%\iffalse
%This code needs to be before the `\ProvidesFile' directive
%which is defined at the beginning of this file.
%Therefore it is also placed there and commented out here.
%</package>
%<*discard>
%\fi
%    \begin{macrocode}
\ifdefined\childdocmain\endinput\fi
%    \end{macrocode}
%\iffalse
%</discard>
%<*package>
%\fi
%
% \macro{\ifchilddoc}
% \macro{\ifchilddocmanual}
% The conditional |\ifchilddoc| tells whether a
% child (true) or main (false) document is being compiled.
% The conditional |\ifchilddocmanual| tells whether
% the |\includeonly| mechanism is used (false) or
% the selection of child files must be performed manually (true).
% The definitions initialise to false:
%    \begin{macrocode}
\newif\ifchilddoc
\newif\ifchilddocmanual
%    \end{macrocode}

% \macro{\childdocname}
% \macro{\childdocjob}
% The macro |\childdocname| stores the name of the main document
% to be compiled. The macro |\childdocjob| stores the name of
% the document on which the \LaTeX{} compiler was originally invoked.
% The content of |\jobname| cannot be compared
% to filenames specified in the source due to different catcodes.
% The following code rescans |\jobname|, stores the result
% in |\childdocname| and saves a copy in |\childdocjob|:
%    \begin{macrocode}
\edef\childdocname{\scantokens\expandafter{\jobname\noexpand}}
\let\childdocjob\childdocname
%    \end{macrocode}

% \macro{\childdocdisable}
% The macro |\childdocdisable| prevents the main file
% from being processed more than once.
% At this stage, the main document command |\childdocmain|
% is assumed to be called once again where it should do nothing.
% Any subsequent call to it should prevent
% a secondary processing of the main document
% It overwrites the forwarding commands
% |\childdocof| and |\childdocforward|
% with empty macros to prevent further inclusions of the main document:
%    \begin{macrocode}
\newcommand{\childdocdisable}
{
  \renewcommand{\childdocmain}[1]{\renewcommand{\childdocmain}[1]{\endinput}}
  \renewcommand{\childdocof}[1]{}
  \renewcommand{\childdocby}[2][]{}
  \renewcommand{\childdocforward}[2][]{}
  \renewcommand{\childdocdisable}{}
}
%    \end{macrocode}

% \macro{\childdocmain}
% The macro |\childdocmain| is to be called at the top of the main file
% with nothing or the main filename (without extension) as argument.
% First, it breaks loops.
% If the argument is not empty and does not match |\childdocname|
% (which is set by the first inclusion of |childdoc.def|),
% |\ifchilddoc| is set to true, |\includeonly| is applied to the child file
% and |\jobname| is set to the main file
% (for proper handling of |.aux| files):
%    \begin{macrocode}
\newcommand{\childdocmain}[1]
{
  \childdocdisable\childdocmain{}
  \if?#1?\else
    \begingroup
      \def\childdoctmp{#1}
      \ifx\childdoctmp\childdocname
        \def\childdoctmp{}
      \else
        \def\childdoctmp
        {
          \childdoctrue
          \includeonly{\childdocname}
          \def\childdocjob{#1}
          \def\jobname{#1}
        }
      \fi
      \expandafter
    \endgroup
    \childdoctmp
  \fi
}
%    \end{macrocode}

% \macro{\childdocof}
% The command |\childdocof| redirects
% compilation to the main file |#1|.
%    \begin{macrocode}
\newcommand{\childdocof}[1]
{
  \childdocdisable
  \childdoctrue
  \includeonly{\childdocname}
  \def\jobname{#1}
  \def\childdocjob{#1}
  \input{#1}
}
%    \end{macrocode}

% \macro{\childdocby}
% The command |\childdocby| ....
%    \begin{macrocode}
\newcommand{\childdocby}[2][]
{
  \childdocdisable
  \childdoctrue
  \childdocmanualtrue
  \if?#1?\else
    \def\jobname{#2}
  \fi
  \def\childdocjob{#2}
  \input{#2}
  \endinput
}
%    \end{macrocode}

% \macro{\childdocforward}
% The command |\childdocforward| redirects
% compilation to the main file or
% (if the optional argument is given) a child file.
% Parameters are set as if the main file
% or a child file starting with |\childdocof| was compiled.
% Then compilation is handed over to the main file:
%    \begin{macrocode}
\newcommand{\childdocforward}[2][]
{
  \begingroup
    \if?#1?
      \def\childdoctmp
      {
        \def\childdocname{#2}
        \def\childdocjob{#2}
        \def\jobname{#2}
        \input{#2}
        \endinput
      }
    \else
      \def\childdoctmp
      {
        \childdocdisable
        \def\childdocname{#2}
        \childdoctrue
        \includeonly{#2}
        \def\childdocjob{#1}
        \def\jobname{#1}
        \input{#1}
        \endinput
      }
    \fi
    \expandafter
  \endgroup
  \childdoctmp
}
%    \end{macrocode}

% \macro{\childdocforwardprefix}
% The command |\childdocforwardprefix| redirects
% compilation to the main or a child file by means of a pattern.
% The prefix |#1| in the current filename is replaced by |#2|
% and the suffix of the current filename is kept
% (it is assumed that the filename does not contain the substring `|~~~|'
% which is used as a delimiter).
% Compilation is handed over to the new file by |\childdocforward|:
%    \begin{macrocode}
\newcommand{\childdocforwardprefix}[3][]
{
  \begingroup
    \def\childdocextract #2##1~~~{\def\childdoctmp{\childdocforward[#1]{#3##1}}}
    \expandafter\childdocextract\childdocname~~~
    \expandafter
  \endgroup
  \childdoctmp
}
%    \end{macrocode}

% \macro{\childdoc}
% The deprecated macro |\childdoc| is a legacy version of |\childdocmain|:
%    \begin{macrocode}
\newcommand{\childdoc}{\childdocmain}
%    \end{macrocode}

% \macro{\childdocredirect}
% The deprecated macro |\childdocredirect| is a legacy version
% of |\childdocforward| and |\childdocforwardprefix|:
%    \begin{macrocode}
\newcommand{\childdocredirect}[2][]
{
  \begingroup
    \if?#1?
      \def\childdoctmp{\childdocforward{#2}}
    \else
      \def\childdoctmp{\childdocforwardprefix{#1}{#2}}
    \fi
    \expandafter
  \endgroup
  \childdoctmp
}
%    \end{macrocode}

%\iffalse
%</package>
%\fi
%
\endinput
\childdocforward{cdocsamp}"|\\
% |latex -jobname cdocscl1 \|\\
% |  "% \iffalse
%
% childdoc.dtx Copyright (C) 2017-2018 Niklas Beisert
%
% This work may be distributed and/or modified under the
% conditions of the LaTeX Project Public License, either version 1.3
% of this license or (at your option) any later version.
% The latest version of this license is in
%   http://www.latex-project.org/lppl.txt
% and version 1.3 or later is part of all distributions of LaTeX
% version 2005/12/01 or later.
%
% This work has the LPPL maintenance status `maintained'.
%
% The Current Maintainer of this work is Niklas Beisert.
%
% This work consists of the files childdoc.dtx and childdoc.ins
% and the derived files childdoc.def and cdocsamp.tex with
% cdocsch1.tex, cdocsch2.tex, cdocsdrf.tex, cdocsfn1.tex, cdocsfn2.tex.
%
%<package>\ifdefined\childdocmain\endinput\fi
%<package>\ProvidesFile{childdoc.def}[2018/12/30 v2.0 child document driver]
%<samplemain>\ProvidesFile{cdocsamp.tex}[2018/12/30 v2.0 sample for childdoc]
%<*driver>
%\ProvidesFile{childdoc.drv}[2018/12/30 v2.0 childdoc reference manual file]
\PassOptionsToClass{10pt,a4paper}{article}
\documentclass{ltxdoc}

\usepackage[margin=35mm]{geometry}
\usepackage{hyperref}
\usepackage{hyperxmp}
\usepackage[usenames]{color}

\hypersetup{colorlinks=true}
\hypersetup{pdfstartview=FitH}
\hypersetup{pdfpagemode=UseNone}
\hypersetup{pdfsource={}}
\hypersetup{pdflang={en-UK}}
\hypersetup{pdfcopyright={Copyright 2017-2018 Niklas Beisert.
  This work may be distributed and/or modified under the
  conditions of the LaTeX Project Public License, either version 1.3
  of this license or (at your option) any later version.}}
\hypersetup{pdflicenseurl={http://www.latex-project.org/lppl.txt}}
\hypersetup{pdfcontactaddress={ETH Zurich, ITP, HIT K,
  Wolfgang-Pauli-Strasse 27}}
\hypersetup{pdfcontactpostcode={8093}}
\hypersetup{pdfcontactcity={Zurich}}
\hypersetup{pdfcontactcountry={Switzerland}}
\hypersetup{pdfcontactemail={nbeisert@itp.phys.ethz.ch}}
\hypersetup{pdfcontacturl={http://people.phys.ethz.ch/\xmptilde nbeisert/}}

\newcommand{\secref}[1]{\hyperref[#1]{section \ref*{#1}}}

\parskip1ex
\parindent0pt
\let\olditemize\itemize
\def\itemize{\olditemize\parskip0pt}

\begin{document}

\title{The \textsf{childdoc} Package}
\hypersetup{pdftitle={The childdoc Package}}
\author{Niklas Beisert\\[2ex]
  Institut f\"ur Theoretische Physik\\
  Eidgen\"ossische Technische Hochschule Z\"urich\\
  Wolfgang-Pauli-Strasse 27, 8093 Z\"urich, Switzerland\\[1ex]
  \href{mailto:nbeisert@itp.phys.ethz.ch}
  {\texttt{nbeisert@itp.phys.ethz.ch}}}
\hypersetup{pdfauthor={Niklas Beisert}}
\hypersetup{pdfsubject={Manual for the LaTeX2e Package childdoc}}
\date{30 December 2018, \textsf{v2.0}}
\maketitle

\begin{abstract}\noindent
\textsf{childdoc} is a \LaTeXe{} package
that enables the direct compilation
of document sections included by |\include|
to individual files.
\end{abstract}

\begingroup
\parskip0ex
\tableofcontents
\endgroup

%%%%%%%%%%%%%%%%%%%%%%%%%%%%%%%%%%%%%%%%%%%%%%%%%%%%%%%%%%%%%%%%%%%%%%%%%%%%%%%%
%%%%%%%%%%%%%%%%%%%%%%%%%%%%%%%%%%%%%%%%%%%%%%%%%%%%%%%%%%%%%%%%%%%%%%%%%%%%%%%%
\section{Introduction}

\LaTeX{} provides a mechanism to structure a large document (such as a book)
into a main file and several child files (containing the chapters)
using the |\include| command.
This mechanism is beneficial for documents
which span hundreds of pages in order to
make the source file(s) more manageable.
Moreover, compilation can be restricted to
selected child files by means of the |\includeonly| command.
The latter feature can be used to reduce the compilation time while editing
(this was significantly more useful in the earlier days of \LaTeX{})
or to generate a smaller document which is easier to navigate.
Another application of |\includeonly| is to generate
documents consisting of selected parts of the complete document.

However, there are a few drawbacks of the plain |\include| mechanism:
\begin{itemize}
\item
The child files cannot be compiled on their own,
they can only be compiled via the main file.
A naive editing environment
(such as a text editor with an option
to have the current file processed by \LaTeX)
may require one to switch to the main file before compiling;
attempting to compile the child file produces errors.
\item
The main file must be modified (each time)
to adjust the |\includeonly| command
to the present needs. This easily leaves the main file in a messy state.
\item
The generated document will always carry the filename
of the main document. This is inconvenient if
several child files are to be compiled and
to be kept for distribution.
\end{itemize}

The present package provides a simple interface
to make child files individually compilable by \LaTeX{}.
Compiling a child file then has the same effect as compiling
the main file with an |\includeonly| command
to select the appropriate child.
Moreover the generated document will carry the name of the child
rather than the main file.
This resolves all three above issues.

This feature is meant to make the editing of books,
thesis documents and lecture notes somewhat more convenient.
However, the package can also be used efficiently for
composing a series of documents (such as exercise sheets)
which are typically distributed individually.
It then assists the author in generating the individual documents
(potentially in different versions)
as well as a document containing the collected series.
Another application is in developing style files
or other kinds of included material
where compilation of the style file could redirect
to a sample or test file.

%%%%%%%%%%%%%%%%%%%%%%%%%%%%%%%%%%%%%%%%%%%%%%%%%%%%%%%%%%%%%%%%%%%%%%%%%%%%%%%%
%%%%%%%%%%%%%%%%%%%%%%%%%%%%%%%%%%%%%%%%%%%%%%%%%%%%%%%%%%%%%%%%%%%%%%%%%%%%%%%%
\section{Usage}

First of all, the package \textsf{childdoc} is \emph{not} a standard
\LaTeXe{} |.sty| style file! Therefore it needs to be invoked in
a non-standard way.

%%%%%%%%%%%%%%%%%%%%%%%%%%%%%%%%%%%%%%%%%%%%%%%%%%%%%%%%%%%%%%%%%%%%%%%%%%%%%%%%
\subsection{Included Files}
\label{sec:include}

%%%%%%%%%%%%%%%%%%%%%%%%%%%%%%%%%%%%%%%%
\DescribeMacro{\childdocmain}
To use the package, add the commands
\begin{center}
\begin{tabular}{l}
|\input{childdoc.def}|\\
|\childdocmain{}|\\
\end{tabular}
\end{center}
at the very top of the main \LaTeX{} file,
in particular \emph{before} the |\documentclass| statement!
The argument of |\childdocmain| should be left empty
(but it must be present).

%%%%%%%%%%%%%%%%%%%%%%%%%%%%%%%%%%%%%%%%
\DescribeMacro{\childdocof}
Furthermore, add the commands
\begin{center}
\begin{tabular}{l}
|\input{childdoc.def}|\\
|\childdocof{|\textit{main}|}|\\
\end{tabular}
\end{center}
at the top of every child file \textit{child}
which is included by |\include{|\textit{child}|}|
from within the main file
(or at least for those files to be compiled individually).
The argument \textit{main} must be the filename of the main file.

There are a couple of
considerations in setting up the main and child documents:

%%%%%%%%%%%%%%%%%%%%%%%%%%%%%%%%%%%%%%%%
\paragraph{Restrictions.}

Please note the following restrictions:
\begin{itemize}
\item
|\childdocmain| must be called with one argument \textit{main}
to ensure compatibility with earlier version of the package.
It must either be empty (|\childdocmain{}|)
or precisely match the filename of the main file in which it is specified.
See \secref{sec:detection} for further information.
\item
The filename \textit{main} must be specified without the |.tex| extension.
\item
The filename \textit{main} is case sensitive
(even in case-insensitive file systems)
due to internal string comparison.
\item
The argument \textit{main} should be fully expanded, it cannot be a macro.
\item
Subdirectories and special characters should be avoided in filenames.
\item
The command |\childdocmain{|\textit{main}|}| must be followed by a whitespace.
It should not be followed immediately by another command
or by a comment mark `|%|'.
This is because the \TeX{} parser reads the token immediately following
the argument of |\childdocmain| and puts it
at the beginning of every child section;
however, a white\-space is ignored.
\end{itemize}

%%%%%%%%%%%%%%%%%%%%%%%%%%%%%%%%%%%%%%%%
\paragraph{Content of Main File.}

It is advisable to place all content in the child files included by |\include|.
Any output contained in the main file will appear in all child documents
unless suppressed manually;
it cannot be suppressed automatically by the |\includeonly| directive
and thus should normally be avoided.
A method to include some content in the main file
by means of conditional processing is described in \secref{sec:conditional}.

%%%%%%%%%%%%%%%%%%%%%%%%%%%%%%%%%%%%%%%%
\paragraph{Page Numbering.}

When only a part of the document is compiled,
the appropriate numbering of pages
(as well as other status parameters)
is determined from the |.aux| files.
The latter contain information from previous passes.
However this information needs to propagate through
all intermediate child documents.
Therefore the page numbering in child documents may well
be inconsistent until the complete document is compiled at least once.

A useful (if unconventional) way to always ensure a consistent
page numbering is to restart the numbering in each child document
and denote the pages by `\textit{child}|.|\textit{page}'
where \textit{child} represents the chapter/section number of the child file.
This can be achieved by the command
|\numberwithin{page}{|\textit{child}|}|
of the \textsf{amsmath} package
where \textit{child} can be |chapter| or |section|
depending on the chosen structuring.
Alternatively, one can modify the macro |\thepage| appropriately
and reset the counter |page| at the start of each child file.

%%%%%%%%%%%%%%%%%%%%%%%%%%%%%%%%%%%%%%%%%%%%%%%%%%%%%%%%%%%%%%%%%%%%%%%%%%%%%%%%
\subsection{Conditional Processing}
\label{sec:conditional}

The package provides a mechanism to compile different versions
of a document. To customise the versions further some conditional processing
can come in handy to distinguish which version is being compiled.
The package provides two macros to describe the compilation context:

%%%%%%%%%%%%%%%%%%%%%%%%%%%%%%%%%%%%%%%%
\DescribeMacro{\ifchilddoc}
The conditional |\ifchilddoc| distinguishes between the compilation of
child documents and the main document:
%
\begin{center}
|\ifchilddoc |\textit{child-code}| |[|\||else |\textit{main-code}]| \||fi|
\end{center}

%%%%%%%%%%%%%%%%%%%%%%%%%%%%%%%%%%%%%%%%
\DescribeMacro{\childdocname}
\DescribeMacro{\childdocjob}
The macro |\childdocname| contains the filename (without extension)
of the main or child file being processed.
Note that |\childdocjob| will always contain the name of the main file.

%%%%%%%%%%%%%%%%%%%%%%%%%%%%%%%%%%%%%%%%
\paragraph{Title Page.}

Conditional processing can be used to include a title or banner page
in the main document when proper precautions are taken.
Importantly, the code in the main file should ensure that the page counter
(as well as other status parameters which are stored in the |.aux| files)
takes the same value after the conditional processing.
Otherwise the page numbers may take divergent values
depending on which part is compiled.

For example, a title page could be declared by:
%
\begin{center}
\begin{tabular}{l}
|\ifchilddoc\||else|\\
|\addtocounter{page}{-1}|\\
\textit{code for title page}\\
|\newpage|\\
|\||fi|
\end{tabular}
\end{center}
%
A banner page for the child documents can be generated by:
%
\begin{center}
\begin{tabular}{l}
|\ifchilddoc|\\
|\addtocounter{page}{-1}|\\
\textit{code for banner page}\\
|\newpage|\\
|\||fi|
\end{tabular}
\end{center}
%
Here one could write a message such as:
\begin{center}
|This is the part \childdocname{} of \childdocjob{}.|
\end{center}

%%%%%%%%%%%%%%%%%%%%%%%%%%%%%%%%%%%%%%%%%%%%%%%%%%%%%%%%%%%%%%%%%%%%%%%%%%%%%%%%
\subsection{Flags}
\label{sec:flags}

The package makes it easy to generate different versions
of the main or child documents.
To this end compilation flags can be defined
and assigned different default values.
They will be particularly useful in conjunction
with the forwarding mechanism described in \secref{sec:forward}.

For example, it may be useful to have a flag |\version|
which can be set to |draft| or |final|.
The document source will contain some conditional code
depending on the value of |\version|.
Suppose further, the flag should default to |final| for the main file
and to |draft| for child files
which is a natural assignment for editing the document.
This is achieved by placing the following code
in the preamble of the main document
(below the |\childdocmain| directive):
%
\begin{center}
\begin{tabular}{l}
|\ifchilddoc|\\
|\providecommand{\version}{draft}|\\
|\||else|\\
|\providecommand{\version}{final}|\\
|\||fi|
\end{tabular}
\end{center}
%
The definition by |\providecommand| makes sure
that previous definitions are not overwritten.
Further statements |\providecommand{\version}{...}|
can thus be added before the above code to override it.

For the main file, one might add a line
(between |\childdocmain| and the above block)
%
\begin{center}
|%\ifchilddoc\||else\providecommand{\version}{draft}\||fi|
\end{center}
%
which can be uncommented to produce a draft version.
Likewise one can add a line to the very top of a child file
(above the |\childdocof{|\textit{main}|}| directive)
%
\begin{center}
|%\providecommand{\version}{final}|
\end{center}
%
which can be uncommented to produce the final version of this child document.

%%%%%%%%%%%%%%%%%%%%%%%%%%%%%%%%%%%%%%%%%%%%%%%%%%%%%%%%%%%%%%%%%%%%%%%%%%%%%%%%
\subsection{Forwarding}
\label{sec:forward}

Different versions of the main or child documents
using compilation flags as described in \secref{sec:flags}
can be (permanently) stored in different files
for convenient compilation, viewing and distribution.
To this end, the package defines a command
to pass on compilation to a different file:

%%%%%%%%%%%%%%%%%%%%%%%%%%%%%%%%%%%%%%%%
\DescribeMacro{\childdocforward}
The command |\childdocforward| redirects processing to
another source file:
%
\begin{center}
\begin{tabular}{l}
|\input{childdoc.def}|\\
|\childdocforward[|\textit{main}|]{|\textit{dest}|}|\\
\end{tabular}
\end{center}
%
The argument \textit{dest} is the destination file
(without extension).
It should be the main file or one of the child files.
Note that further \textsf{childdoc} directives
such as |\childdocof| and |\childdocforward|
in the indicated file will be processed in this form.
The optional argument \textit{main}
passes on directly to the main file \textit{main}
while pretending to compile the child \textit{dest}.
This form behaves as if \textit{dest}
issues |\childdocof{|\textit{main}|}| right away,
and no further \textsf{childdoc} directives will be processed.

%%%%%%%%%%%%%%%%%%%%%%%%%%%%%%%%%%%%%%%%
\DescribeMacro{\...prefix}
In the alternative form |\childdocforwardprefix|,
%
\begin{center}
\begin{tabular}{l}
|\input{childdoc.def}|\\
|\childdocforwardprefix[|\textit{main}|]{|\textit{prefix}|}{|\textit{dest}|}|
\end{tabular}
\end{center}
%
the destination file is determined by a pattern
depending on the current file:
To make this work, the current file must be called
`{\textit{prefix}\hspace{0.2em}\textit{suffix}}'
with \textit{prefix} matching precisely the argument.
Processing is then passed on to the file
`{\textit{dest}\hspace{0.2em}\textit{suffix}}'.
Surely, the same effect is achieved by
directly specifying the
argument `{\textit{dest}\hspace{0.2em}\textit{suffix}}'
in the first form.
However, that requires to set up a different file
for each child. With the alternative form of the command
all these files can have exactly the same content
which simplifies setting them up and maintaining them.

For example, the following file |draft.tex|
with a compilation flag |\version| as described in \secref{sec:flags}
compiles the main document as a draft:
%
\begin{center}
\begin{tabular}{l}
|\def\version{draft}|\\
|\input{childdoc.def}|\\
|\childdocforward{|\textit{main}|}|
\end{tabular}
\end{center}
%
Likewise, the following files |final|\textit{nn}|.tex|
compile the final version of the child document
|child|\textit{nn}|.tex|:
%
\begin{center}
\begin{tabular}{l}
|\def\version{final}|\\
|\input{childdoc.def}|\\
|\childdocforwardprefix{final}{child}|
\end{tabular}
\end{center}
%

Note that when several versions of a main file and/or of each child file
are to be generated, it may be convenient to set up a |Makefile| or
shell script to automatise the process.

%%%%%%%%%%%%%%%%%%%%%%%%%%%%%%%%%%%%%%%%%%%%%%%%%%%%%%%%%%%%%%%%%%%%%%%%%%%%%%%%
\subsection{Command Line Processing}
\label{sec:commandline}

The effect of redirection files can also be achieved by invoking
the \LaTeX{} compiler with a more elaborate command line.
Most conveniently this should be done as part
of a shell script or a |Makefile|.

When using \textsf{childdoc} in the main file, the following
command lines effectively perform a redirection
(note that depending on the shell being used,
backslashes may have to be doubled: `|\|' $\to$ `|\\|'):
%
\begin{center}
|... -jobname "|\textit{target}|" |\\|"|[\textit{flags}]%
|\input{childdoc.def}\childdocforward[|\textit{main}|]{|\textit{dest}|}"|
\end{center}
%
Here \textit{target} is the name of the output file,
\textit{main} is the name of the main file
and \textit{dest} is the name of the main or child file to be processed
(all filenames without extensions).
The optional argument \textit{main} can be omitted
if \textit{main} matches \textit{dest}.
Optionally, compilation \textit{flags} can be defined via |\def| commands.
This command line makes the \TeX{} engine believe
it is compiling the file \textit{target}
whose content is specified as the latter parameter.
The provided code then forwards the processing to
\textit{main} or \textit{dest} as described in \secref{sec:forward}.

%%%%%%%%%%%%%%%%%%%%%%%%%%%%%%%%%%%%%%%%%%%%%%%%%%%%%%%%%%%%%%%%%%%%%%%%%%%%%%%%
\subsection{Include by Input}
\label{sec:input}

Including child documents by |\include| has some restrictions by design.
Most notably, the content of a child document always occupies
its own set of pages; pages cannot be shared between child documents.
Usually, this behaviour makes perfect sense
because each child document contain an essential part of the document.
However, in some situations it may be desirable to compose
a document from a collection of parts
without having mandatory page breaks between then.
For this case, the package
provides a mechanism to include parts
by |\input| which can also be processed individually.
However, by construction this mechanism
requires manual handling of the content to be output.

%%%%%%%%%%%%%%%%%%%%%%%%%%%%%%%%%%%%%%%%
\DescribeMacro{\ifchilddocmanual}
The main file should be prepared as usual, see \secref{sec:include}.
However, the document body must make a distinction
between processing of an individual part and of the main document, e.g.:
%
\begin{center}
\begin{tabular}{l}
|\ifchilddocmanual|\\
|\input{\childdocname}|\\
|\||else|\\
\textit{document body with }|\input{|\textit{part}|}|\\
|\||fi|
\end{tabular}
\end{center}
%
The conditional |\ifchilddocmanual| is true whenever
a part to be included by |\input| is being compiled,
and the name of the part is stored in |\childdocname|.

%%%%%%%%%%%%%%%%%%%%%%%%%%%%%%%%%%%%%%%%
\DescribeMacro{\childdocby}
Each part to be included by |\input| should start with:
%
\begin{center}
\begin{tabular}{l}
|\input{childdoc.def}|\\
|\childdocby{|\textit{main}|}|\\
\end{tabular}
\end{center}
%
The directive |\childdocby| is similar to |\childdocof|
described in \secref{sec:include},
but the subsequent selection of content must be done manually.
To that end, both |\ifchilddoc| and |\ifchilddocmanual|
will be true upon processing of a part,
and the name of the part is stored in |\childdocname|.
Note that |\jobname| will be set to the filename of the current part
so that each part receives an individual |.aux| file
that does not interfere with the |.aux| file(s) of the main document.
This behaviour can be altered by the alternative form
|\childdocby[*]{|\textit{main}|}| (with a non-empty optional argument)
which uses the |.aux| file of the main document
by setting |\jobname| to \textit{main}.

%%%%%%%%%%%%%%%%%%%%%%%%%%%%%%%%%%%%%%%%%%%%%%%%%%%%%%%%%%%%%%%%%%%%%%%%%%%%%%%%
\subsection{Driver Development}
\label{sec:driver}

The \textsf{childdoc} mechanism can also be use for the development
of definition files such as \LaTeX{} styles or classes.
This case differs from the above setup with multiple parts
included by |\include| in that no |\includeonly| should be invoked.
This can be achieved by starting the include file
(before |\ProvidesPackage|) with:
%
\begin{center}
\begin{tabular}{l}
|\input{childdoc.def}|\\
|\childdocforward{|\textit{main}|}|\\
\end{tabular}
\end{center}
%
or alternatively with:
%
\begin{center}
\begin{tabular}{l}
|\input{childdoc.def}|\\
|\childdocby{|\textit{main}|}|\\
\end{tabular}
\end{center}
%
Both forms have slightly different effects as described above.
The main file is prepared as usual, see \secref{sec:include}.

%%%%%%%%%%%%%%%%%%%%%%%%%%%%%%%%%%%%%%%%%%%%%%%%%%%%%%%%%%%%%%%%%%%%%%%%%%%%%%%%
\subsection{Legacy Detection}
\label{sec:detection}

The directive |\childdocmain| in the main file can detect
whether the complete document or merely a child is to be compiled
even without using the directive |\childdocof|.
This method is deprecated because it is less robust
and there is no compelling reason to use it;
it is merely provided for backward compatibility
and it may be removed in future versions.

If the detection mechanism is to be used,
it is mandatory to correctly specify
the filename of the main file as the argument of |\childdocmain|:
%
\begin{center}
\begin{tabular}{l}
|\input{childdoc.def}|\\
|\childdocmain{|\textit{main}|}|\\
\end{tabular}
\end{center}
%
If |\jobname| does not match the argument \textit{main} of |\childdocmain|,
it is assumed that |\jobname| points to the child file to be compiled.
When using |\childdocmain| with the main file specified as argument,
it suffices to start a child file
with just |\input{|\textit{main}|}|
without loading of the package and using |\childdocof|.
If instead all processing is done
with the appropriate \textsf{childdoc} directives,
the argument of \textit{main} of |\childdocmain| can be empty.

An alternative version of the command line processing described
in \secref{sec:commandline} using the detection mechanism reads:
%
\begin{center}
|... -jobname "|\textit{target}|" "|[\textit{flags}]%
[|\def\jobname{|\textit{dest}|}|]|\input{|\textit{main}|}"|
\end{center}

%%%%%%%%%%%%%%%%%%%%%%%%%%%%%%%%%%%%%%%%%%%%%%%%%%%%%%%%%%%%%%%%%%%%%%%%%%%%%%%%
\subsection{Manual Code}
\label{sec:manual}

In case one cannot be certain whether the definitions file |childdoc.def|
is installed on the target \TeX{} distribution
and one prefers not to ship it,
it is conceivable to paste a few relevant commands into the sources.

To that end, drop all statements |\input{childdoc.def}|
and perform the replacements as outlined below.
Instead of |\childdocmain{|\textit{main}|}| add the following code
to the top of the main file:
%
\begin{center}
\begin{tabular}{l}
|\||ifdefined\childdocname\endinput\||fi\newif\ifchilddoc|\\
|\edef\childdocname{\scantokens\expandafter{\jobname\noexpand}}|\\
|\def\childdocmain{|\textit{main}|}\||ifx\childdocmain\childdocname\||else|\\
|\childdoctrue\includeonly{\childdocname}\let\jobname\childdocmain\||fi|\\
\end{tabular}
\end{center}
%
Instead of |\childdocof{|\textit{main}|}| just include the main file
at the top of each child file:
%
\begin{center}
|\input{|\textit{main}|}|
\end{center}
%
A simple redirection |\childdocforward{|\textit{dest}|}| is achieved by:
%
\begin{center}
|\def\jobname{|\textit{dest}|}\input{\jobname}|
\end{center}
%
The redirection with prefix
|\childdocforwardprefix[|\textit{prefix}|]{|\textit{dest}|}|
is accomplished by:
%
\begin{center}
\begin{tabular}{l}
|{\edef\jobname{\scantokens\expandafter{\jobname\noexpand}}|\\
|\def\redirectjob |\textit{prefix}|#1~~~{\gdef\jobname{|\textit{dest}|#1}}|\\
|\expandafter\redirectjob\jobname~~~}\input{\jobname}|
\end{tabular}
\end{center}

In an alternative approach,
child documents can be compiled by a specific command line
without additional code or specific definitions:
%
\begin{center}
|... -jobname "|\textit{target}|" "|[\textit{flags}]%
|\includeonly{|\textit{dest}|}\input{|\textit{main}|}"|
\end{center}
%

%%%%%%%%%%%%%%%%%%%%%%%%%%%%%%%%%%%%%%%%%%%%%%%%%%%%%%%%%%%%%%%%%%%%%%%%%%%%%%%%
%%%%%%%%%%%%%%%%%%%%%%%%%%%%%%%%%%%%%%%%%%%%%%%%%%%%%%%%%%%%%%%%%%%%%%%%%%%%%%%%
\section{Information}

%%%%%%%%%%%%%%%%%%%%%%%%%%%%%%%%%%%%%%%%%%%%%%%%%%%%%%%%%%%%%%%%%%%%%%%%%%%%%%%%
\subsection{Copyright}

Copyright \copyright{} 2017--2018 Niklas Beisert

This work may be distributed and/or modified under the
conditions of the \LaTeX{} Project Public License, either version 1.3
of this license or (at your option) any later version.
The latest version of this license is in
  \url{http://www.latex-project.org/lppl.txt}
and version 1.3 or later is part of all distributions of \LaTeX{}
version 2005/12/01 or later.

This work has the LPPL maintenance status `maintained'.

The Current Maintainer of this work is Niklas Beisert.

This work consists of the files |README.txt|, |childdoc.ins| and |childdoc.dtx|
as well as the derived files |childdoc.def|, |cdocsamp.tex|
with |cdocsch1.tex|, |cdocsch2.tex|, |cdocspt3.tex|, |cdocspt4.tex|,
|cdocsdrf.tex|, |cdocsfn1.tex|, |cdocsfn2.tex|
as well as |childdoc.pdf|.

%%%%%%%%%%%%%%%%%%%%%%%%%%%%%%%%%%%%%%%%%%%%%%%%%%%%%%%%%%%%%%%%%%%%%%%%%%%%%%%%
\subsection{Files and Installation}

The package consists of the files:
%
\begin{center}
\begin{tabular}{ll}
    |README.txt|   & readme file \\
    |childdoc.ins| & installation file \\
    |childdoc.dtx| & source file \\
    |childdoc.def| & definition file \\
    |cdocsamp.tex| & sample main file \\
    |cdocsch1.tex| & sample include file \\
    |cdocsch2.tex| & sample include file \\
    |cdocspt3.tex| & sample part file \\
    |cdocspt4.tex| & sample part file \\
    |cdocsdrf.tex| & sample redirection file \\
    |cdocsfn1.tex| & sample redirection file \\
    |cdocsfn2.tex| & sample redirection file \\
    |childdoc.pdf| & manual
\end{tabular}
\end{center}
%
The distribution consists of the files
|README.txt|, |childdoc.ins| and |childdoc.dtx|.
%
\begin{itemize}
\item
Run (pdf)\LaTeX{} on |childdoc.dtx|
to compile the manual |childdoc.pdf| (this file).
\item
Run \LaTeX{} on |childdoc.ins| to create the definitions file |childdoc.def|
and the sample |cdocsamp.tex| with include files
|cdocsch1.tex|, |cdocsch2.tex|, |cdocspt3.tex|, |cdocspt4.tex|,
|cdocsdrf.tex|, |cdocsfn1.tex|, |cdocsfn2.tex|.
Then copy the file |childdoc.def| to an appropriate directory of your \LaTeX{}
distribution, e.g.\ \textit{texmf-root}|/tex/latex/childdoc|.
\end{itemize}

%%%%%%%%%%%%%%%%%%%%%%%%%%%%%%%%%%%%%%%%%%%%%%%%%%%%%%%%%%%%%%%%%%%%%%%%%%%%%%%%
\subsection{Related CTAN Packages}

There are several other packages which offer a similar functionality:
%
\begin{itemize}
\item
The packages
\href{http://ctan.org/pkg/docmute}{\textsf{docmute}},
\href{http://ctan.org/pkg/includex}{\textsf{includex}} and
\href{http://ctan.org/pkg/standalone}{\textsf{standalone}}
provide commands to include only the document body of
a child file thus allowing both files to be compiled individually.
\item
The packages \href{http://ctan.org/pkg/subdocs}{\textsf{subdocs}}
and \href{http://ctan.org/pkg/subfiles}{\textsf{subfiles}}
provide structures in which the main and child documents can be
encapsulated and allowing them to be compiled individually.
The inclusion mechanism is different from the conventional |\include|.
\item
The package \href{http://ctan.org/pkg/combine}{\textsf{combine}}
is an elaborate solution to combine several documents into one.
\end{itemize}
%
See also the CTAN topic \href{http://ctan.org/topic/subdocs}{\textsf{subdocs}}
for further related packages.
The present package differs from the above solutions in that
a document structure constructed with the conventional |\include| mechanism
just needs two extra commands at the top of every file
such that all constituent files can be compiled individually.

%%%%%%%%%%%%%%%%%%%%%%%%%%%%%%%%%%%%%%%%%%%%%%%%%%%%%%%%%%%%%%%%%%%%%%%%%%%%%%%%
%\subsection{Feature Suggestions}
%
%The following is a list of features which may be useful for future
%versions of this package:
%%
%\begin{itemize}
%\item
%\ldots
%\end{itemize}

%%%%%%%%%%%%%%%%%%%%%%%%%%%%%%%%%%%%%%%%%%%%%%%%%%%%%%%%%%%%%%%%%%%%%%%%%%%%%%%%
\subsection{Revision History}

%%%%%%%%%%%%%%%%%%%%%%%%%%%%%%%%%%%%%%%%
\paragraph{v2.0:} 2018/12/30

\begin{itemize}
\item
immediate forward processing
\item
added |\childdocby| mechanism
\item
manual restructured
\end{itemize}

%%%%%%%%%%%%%%%%%%%%%%%%%%%%%%%%%%%%%%%%
\paragraph{v1.6:} 2018/01/17

\begin{itemize}
\item
application for development of include files
\item
corrections to manual
\end{itemize}

%%%%%%%%%%%%%%%%%%%%%%%%%%%%%%%%%%%%%%%%
\paragraph{v1.5:} 2017/05/21

\begin{itemize}
\item
more complete structuring introduced
\item
|\childdocof| introduced
\item
|\childdoc| renamed to |\childdocmain|
\item
|\childredirect| renamed to |\childdocforward| and |\childdocforwardprefix|
and functionality expanded
\end{itemize}

%%%%%%%%%%%%%%%%%%%%%%%%%%%%%%%%%%%%%%%%
\paragraph{v1.0:} 2017/04/27

\begin{itemize}
\item
manual and install package
\item
first version published on CTAN
\end{itemize}

%%%%%%%%%%%%%%%%%%%%%%%%%%%%%%%%%%%%%%%%
\paragraph{v0.6:} 2017/04/26

\begin{itemize}
\item
redirection mechanism added
\end{itemize}

%%%%%%%%%%%%%%%%%%%%%%%%%%%%%%%%%%%%%%%%
\paragraph{v0.5:} 2017/04/26

\begin{itemize}
\item
functionality in definition file
\end{itemize}


%%%%%%%%%%%%%%%%%%%%%%%%%%%%%%%%%%%%%%%%%%%%%%%%%%%%%%%%%%%%%%%%%%%%%%%%%%%%%%%%
%%%%%%%%%%%%%%%%%%%%%%%%%%%%%%%%%%%%%%%%%%%%%%%%%%%%%%%%%%%%%%%%%%%%%%%%%%%%%%%%
%%%%%%%%%%%%%%%%%%%%%%%%%%%%%%%%%%%%%%%%%%%%%%%%%%%%%%%%%%%%%%%%%%%%%%%%%%%%%%%%
\appendix

\settowidth\MacroIndent{\rmfamily\scriptsize 000\ }

 \DocInput{childdoc.dtx}

\end{document}
%</driver>
% \fi
%
% %%%%%%%%%%%%%%%%%%%%%%%%%%%%%%%%%%%%%%%%%%%%%%%%%%%%%%%%%%%%%%%%%%%%%%%%%%%%%%
% %%%%%%%%%%%%%%%%%%%%%%%%%%%%%%%%%%%%%%%%%%%%%%%%%%%%%%%%%%%%%%%%%%%%%%%%%%%%%%
% \section{Sample}
%\iffalse
%<*samplemain>
%\fi
%
% The following presents a sample document
% with two chapters, two parts, a title page,
% a compile flag as well as three forwarding files to set the flag.
% It consists of eight |.tex| files:
% \begin{center}
% \begin{tabular}{ll}
% |cdocsamp.tex|&main file\\
% |cdocsch1.tex|&include file for chapter 1\\
% |cdocsch2.tex|&include file for chapter 2\\
% |cdocspt3.tex|&include file for part 3\\
% |cdocspt4.tex|&include file for part 4\\
% |cdocsdrf.tex|&forwarding file for main file in draft mode\\
% |cdocsfi1.tex|&forwarding file for final version of chapter 1\\
% |cdocsfi2.tex|&forwarding file for final version of chapter 2\\
% \end{tabular}
% \end{center}
% Each of the eight files can be compiled directly by the \LaTeX{} compiler.
%
% %%%%%%%%%%%%%%%%%%%%%%%%%%%%%%%%%%%%%%
% \paragraph{Main File.}
%
% The main file is called |cdocsamp.tex|.
%
% Load the \textsf{childdoc} definitions and
% declare the filename for the main document:
%    \begin{macrocode}
\input{childdoc.def}
\childdocmain{}
%    \end{macrocode}

% Optional override for |\version| flag:
%    \begin{macrocode}
%%\ifchilddoc\else\providecommand{\version}{draft}\fi
%    \end{macrocode}

% Define the default values for the |\version| flag
% (|final| for the main file and |draft| for childs):
%    \begin{macrocode}
\ifchilddoc
\providecommand{\version}{draft}
\else
\providecommand{\version}{final}
\fi
%    \end{macrocode}

% Load the standard document class:
%    \begin{macrocode}
\documentclass[12pt]{article}
%    \end{macrocode}

% Start the document body:
%    \begin{macrocode}
\begin{document}
%    \end{macrocode}

% Declare a title page.
% Print title, part of document being processed and version flag:
%    \begin{macrocode}
\addtocounter{page}{-1}
\begin{center}
{\LARGE\bfseries{}childdoc example\par}
\vspace{1cm}
\ifchilddoc
\ifchilddocmanual part\else chapter\fi:
`\childdocname' of `\childdocjob'\par
\else
main document: `\childdocjob'\par
\fi
version: \version\par
\end{center}
\newpage
%    \end{macrocode}

% Manually include selected file,
% otherwise process as usual:
%    \begin{macrocode}
\ifchilddocmanual
\section*{part `\childdocname'}
\input{\childdocname}
\else
%    \end{macrocode}

% Include the two chapters:
%    \begin{macrocode}
\include{cdocsch1}
\include{cdocsch2}
%    \end{macrocode}

% Include the two parts unless only chapters should be displayed:
%    \begin{macrocode}
\ifchilddoc\else
\section{part three}
\input{cdocspt3}
\section{part four}
\input{cdocspt4}
\fi
%    \end{macrocode}

% Process as usual until here:
%    \begin{macrocode}
\fi
%    \end{macrocode}

% End of document body:
%    \begin{macrocode}
\end{document}
%    \end{macrocode}
%\iffalse
%</samplemain>
%\fi
%
% %%%%%%%%%%%%%%%%%%%%%%%%%%%%%%%%%%%%%%
% \paragraph{Chapter Include Files.}
%
% The include files are called |cdocsch1.tex| and |cdocsch2.tex|.
%
%\iffalse
%<*samplechap1|samplechap2>
%\fi

% Optional override for |\version| flag:
%    \begin{macrocode}
%%\providecommand{\version}{final}
%    \end{macrocode}

% Include the main document:
%    \begin{macrocode}
\input{childdoc.def}
\childdocof{cdocsamp}
%    \end{macrocode}

%\iffalse
%</samplechap1|samplechap2>
%\fi
%
%\iffalse
%<*samplechap1>
%\fi
% Some text for chapter 1:
%    \begin{macrocode}
\section{one}
some text in chapter one
%    \end{macrocode}

%\iffalse
%</samplechap1>
%\fi
% Some text for chapter 2:
%\iffalse
%<*samplechap2>
%\fi
%    \begin{macrocode}
\section{two}
more text in chapter two
%    \end{macrocode}

%\iffalse
%</samplechap2>
%\fi
%
% %%%%%%%%%%%%%%%%%%%%%%%%%%%%%%%%%%%%%%
% \paragraph{Part Include Files.}
%
% The include files are called |cdocspt3.tex| and |cdocspt4.tex|.
%
%\iffalse
%<*samplepart3|samplepart4>
%\fi

% Optional override for |\version| flag:
%    \begin{macrocode}
%%\providecommand{\version}{final}
%    \end{macrocode}

% Include the main document:
%    \begin{macrocode}
\input{childdoc.def}
\childdocby{cdocsamp}
%    \end{macrocode}

%\iffalse
%</samplepart3|samplepart4>
%\fi
%
%\iffalse
%<*samplepart3>
%\fi
% Some text for part 3:
%    \begin{macrocode}
some text in part three
%    \end{macrocode}

%\iffalse
%</samplepart3>
%\fi
% Some text for part 4:
%\iffalse
%<*samplepart4>
%\fi
%    \begin{macrocode}
more text in part four
%    \end{macrocode}

%\iffalse
%</samplepart4>
%\fi
%
% %%%%%%%%%%%%%%%%%%%%%%%%%%%%%%%%%%%%%%
% \paragraph{Forwarding for a Complete Draft.}
%
% The following forwarding file |cdocsdrf.tex|
% compiles the main document in draft mode:
%\iffalse
%<*sampledraft>
%\fi
%    \begin{macrocode}
\def\version{draft}
\input{childdoc.def}
\childdocforward{cdocsamp}
%    \end{macrocode}

%\iffalse
%</sampledraft>
%\fi
%
% %%%%%%%%%%%%%%%%%%%%%%%%%%%%%%%%%%%%%%
% \paragraph{Forwarding for Final Version of the Chapters.}
%
% The following forwarding files |cdocsfn1.tex| and |cdocsfn2.tex|
% (with identical content)
% compile the final versions of the child documents
% |cdocsch1.tex| and |cdocsch2.tex|, respectively:
%\iffalse
%<*samplefinal>
%\fi
%    \begin{macrocode}
\def\version{final}
\input{childdoc.def}
\childdocforwardprefix[cdocsamp]{cdocsfn}{cdocsch}
%    \end{macrocode}

%\iffalse
%</samplefinal>
%\fi
%
% %%%%%%%%%%%%%%%%%%%%%%%%%%%%%%%%%%%%%%
% \paragraph{Command Line Processing.}
%
% The following three command lines generate the output files
% |cdocscld|, |cdocscl1| and |cdocscl2|
% which should be identical to
% |cdocsdrf|, |cdocsch1| and |cdocsfn2|, respectively:
% \begin{center}
% \begin{tabular}{l}
% |latex -jobname cdocscld \|\\
% |  "\def\version{draft}\input{childdoc.def}\childdocforward{cdocsamp}"|\\
% |latex -jobname cdocscl1 \|\\
% |  "\input{childdoc.def}\childdocforward[cdocsamp]{cdocsch1}"|\\
% |latex -jobname cdocscl2 \|\\
% |  "\def\version{final}\input{childdoc.def}\childdocforward{cdocsch2}"|
% \end{tabular}
% \end{center}
% Note that the trailing backslash on each first line
% merely continues the input to the second line
% (for convenient cut ant paste).
% Furthermore, the command |latex| can be replaced by any
% of its alternative versions such as |pdflatex|.
%
% %%%%%%%%%%%%%%%%%%%%%%%%%%%%%%%%%%%%%%%%%%%%%%%%%%%%%%%%%%%%%%%%%%%%%%%%%%%%%%
% %%%%%%%%%%%%%%%%%%%%%%%%%%%%%%%%%%%%%%%%%%%%%%%%%%%%%%%%%%%%%%%%%%%%%%%%%%%%%%
% \section{Implementation}
%\iffalse
%<*package>
%\fi
%
% This section describes the definitions file |childdoc.def|.

% The definitions cannot be loaded using |\usepackage| or |\RequirePackage|
% which has a mechanism to prevent loading a style file more than once.
% When loading the definitions by means of |\input|
% multiple instances have to be prevented manually:
%\iffalse
%This code needs to be before the `\ProvidesFile' directive
%which is defined at the beginning of this file.
%Therefore it is also placed there and commented out here.
%</package>
%<*discard>
%\fi
%    \begin{macrocode}
\ifdefined\childdocmain\endinput\fi
%    \end{macrocode}
%\iffalse
%</discard>
%<*package>
%\fi
%
% \macro{\ifchilddoc}
% \macro{\ifchilddocmanual}
% The conditional |\ifchilddoc| tells whether a
% child (true) or main (false) document is being compiled.
% The conditional |\ifchilddocmanual| tells whether
% the |\includeonly| mechanism is used (false) or
% the selection of child files must be performed manually (true).
% The definitions initialise to false:
%    \begin{macrocode}
\newif\ifchilddoc
\newif\ifchilddocmanual
%    \end{macrocode}

% \macro{\childdocname}
% \macro{\childdocjob}
% The macro |\childdocname| stores the name of the main document
% to be compiled. The macro |\childdocjob| stores the name of
% the document on which the \LaTeX{} compiler was originally invoked.
% The content of |\jobname| cannot be compared
% to filenames specified in the source due to different catcodes.
% The following code rescans |\jobname|, stores the result
% in |\childdocname| and saves a copy in |\childdocjob|:
%    \begin{macrocode}
\edef\childdocname{\scantokens\expandafter{\jobname\noexpand}}
\let\childdocjob\childdocname
%    \end{macrocode}

% \macro{\childdocdisable}
% The macro |\childdocdisable| prevents the main file
% from being processed more than once.
% At this stage, the main document command |\childdocmain|
% is assumed to be called once again where it should do nothing.
% Any subsequent call to it should prevent
% a secondary processing of the main document
% It overwrites the forwarding commands
% |\childdocof| and |\childdocforward|
% with empty macros to prevent further inclusions of the main document:
%    \begin{macrocode}
\newcommand{\childdocdisable}
{
  \renewcommand{\childdocmain}[1]{\renewcommand{\childdocmain}[1]{\endinput}}
  \renewcommand{\childdocof}[1]{}
  \renewcommand{\childdocby}[2][]{}
  \renewcommand{\childdocforward}[2][]{}
  \renewcommand{\childdocdisable}{}
}
%    \end{macrocode}

% \macro{\childdocmain}
% The macro |\childdocmain| is to be called at the top of the main file
% with nothing or the main filename (without extension) as argument.
% First, it breaks loops.
% If the argument is not empty and does not match |\childdocname|
% (which is set by the first inclusion of |childdoc.def|),
% |\ifchilddoc| is set to true, |\includeonly| is applied to the child file
% and |\jobname| is set to the main file
% (for proper handling of |.aux| files):
%    \begin{macrocode}
\newcommand{\childdocmain}[1]
{
  \childdocdisable\childdocmain{}
  \if?#1?\else
    \begingroup
      \def\childdoctmp{#1}
      \ifx\childdoctmp\childdocname
        \def\childdoctmp{}
      \else
        \def\childdoctmp
        {
          \childdoctrue
          \includeonly{\childdocname}
          \def\childdocjob{#1}
          \def\jobname{#1}
        }
      \fi
      \expandafter
    \endgroup
    \childdoctmp
  \fi
}
%    \end{macrocode}

% \macro{\childdocof}
% The command |\childdocof| redirects
% compilation to the main file |#1|.
%    \begin{macrocode}
\newcommand{\childdocof}[1]
{
  \childdocdisable
  \childdoctrue
  \includeonly{\childdocname}
  \def\jobname{#1}
  \def\childdocjob{#1}
  \input{#1}
}
%    \end{macrocode}

% \macro{\childdocby}
% The command |\childdocby| ....
%    \begin{macrocode}
\newcommand{\childdocby}[2][]
{
  \childdocdisable
  \childdoctrue
  \childdocmanualtrue
  \if?#1?\else
    \def\jobname{#2}
  \fi
  \def\childdocjob{#2}
  \input{#2}
  \endinput
}
%    \end{macrocode}

% \macro{\childdocforward}
% The command |\childdocforward| redirects
% compilation to the main file or
% (if the optional argument is given) a child file.
% Parameters are set as if the main file
% or a child file starting with |\childdocof| was compiled.
% Then compilation is handed over to the main file:
%    \begin{macrocode}
\newcommand{\childdocforward}[2][]
{
  \begingroup
    \if?#1?
      \def\childdoctmp
      {
        \def\childdocname{#2}
        \def\childdocjob{#2}
        \def\jobname{#2}
        \input{#2}
        \endinput
      }
    \else
      \def\childdoctmp
      {
        \childdocdisable
        \def\childdocname{#2}
        \childdoctrue
        \includeonly{#2}
        \def\childdocjob{#1}
        \def\jobname{#1}
        \input{#1}
        \endinput
      }
    \fi
    \expandafter
  \endgroup
  \childdoctmp
}
%    \end{macrocode}

% \macro{\childdocforwardprefix}
% The command |\childdocforwardprefix| redirects
% compilation to the main or a child file by means of a pattern.
% The prefix |#1| in the current filename is replaced by |#2|
% and the suffix of the current filename is kept
% (it is assumed that the filename does not contain the substring `|~~~|'
% which is used as a delimiter).
% Compilation is handed over to the new file by |\childdocforward|:
%    \begin{macrocode}
\newcommand{\childdocforwardprefix}[3][]
{
  \begingroup
    \def\childdocextract #2##1~~~{\def\childdoctmp{\childdocforward[#1]{#3##1}}}
    \expandafter\childdocextract\childdocname~~~
    \expandafter
  \endgroup
  \childdoctmp
}
%    \end{macrocode}

% \macro{\childdoc}
% The deprecated macro |\childdoc| is a legacy version of |\childdocmain|:
%    \begin{macrocode}
\newcommand{\childdoc}{\childdocmain}
%    \end{macrocode}

% \macro{\childdocredirect}
% The deprecated macro |\childdocredirect| is a legacy version
% of |\childdocforward| and |\childdocforwardprefix|:
%    \begin{macrocode}
\newcommand{\childdocredirect}[2][]
{
  \begingroup
    \if?#1?
      \def\childdoctmp{\childdocforward{#2}}
    \else
      \def\childdoctmp{\childdocforwardprefix{#1}{#2}}
    \fi
    \expandafter
  \endgroup
  \childdoctmp
}
%    \end{macrocode}

%\iffalse
%</package>
%\fi
%
\endinput
\childdocforward[cdocsamp]{cdocsch1}"|\\
% |latex -jobname cdocscl2 \|\\
% |  "\def\version{final}% \iffalse
%
% childdoc.dtx Copyright (C) 2017-2018 Niklas Beisert
%
% This work may be distributed and/or modified under the
% conditions of the LaTeX Project Public License, either version 1.3
% of this license or (at your option) any later version.
% The latest version of this license is in
%   http://www.latex-project.org/lppl.txt
% and version 1.3 or later is part of all distributions of LaTeX
% version 2005/12/01 or later.
%
% This work has the LPPL maintenance status `maintained'.
%
% The Current Maintainer of this work is Niklas Beisert.
%
% This work consists of the files childdoc.dtx and childdoc.ins
% and the derived files childdoc.def and cdocsamp.tex with
% cdocsch1.tex, cdocsch2.tex, cdocsdrf.tex, cdocsfn1.tex, cdocsfn2.tex.
%
%<package>\ifdefined\childdocmain\endinput\fi
%<package>\ProvidesFile{childdoc.def}[2018/12/30 v2.0 child document driver]
%<samplemain>\ProvidesFile{cdocsamp.tex}[2018/12/30 v2.0 sample for childdoc]
%<*driver>
%\ProvidesFile{childdoc.drv}[2018/12/30 v2.0 childdoc reference manual file]
\PassOptionsToClass{10pt,a4paper}{article}
\documentclass{ltxdoc}

\usepackage[margin=35mm]{geometry}
\usepackage{hyperref}
\usepackage{hyperxmp}
\usepackage[usenames]{color}

\hypersetup{colorlinks=true}
\hypersetup{pdfstartview=FitH}
\hypersetup{pdfpagemode=UseNone}
\hypersetup{pdfsource={}}
\hypersetup{pdflang={en-UK}}
\hypersetup{pdfcopyright={Copyright 2017-2018 Niklas Beisert.
  This work may be distributed and/or modified under the
  conditions of the LaTeX Project Public License, either version 1.3
  of this license or (at your option) any later version.}}
\hypersetup{pdflicenseurl={http://www.latex-project.org/lppl.txt}}
\hypersetup{pdfcontactaddress={ETH Zurich, ITP, HIT K,
  Wolfgang-Pauli-Strasse 27}}
\hypersetup{pdfcontactpostcode={8093}}
\hypersetup{pdfcontactcity={Zurich}}
\hypersetup{pdfcontactcountry={Switzerland}}
\hypersetup{pdfcontactemail={nbeisert@itp.phys.ethz.ch}}
\hypersetup{pdfcontacturl={http://people.phys.ethz.ch/\xmptilde nbeisert/}}

\newcommand{\secref}[1]{\hyperref[#1]{section \ref*{#1}}}

\parskip1ex
\parindent0pt
\let\olditemize\itemize
\def\itemize{\olditemize\parskip0pt}

\begin{document}

\title{The \textsf{childdoc} Package}
\hypersetup{pdftitle={The childdoc Package}}
\author{Niklas Beisert\\[2ex]
  Institut f\"ur Theoretische Physik\\
  Eidgen\"ossische Technische Hochschule Z\"urich\\
  Wolfgang-Pauli-Strasse 27, 8093 Z\"urich, Switzerland\\[1ex]
  \href{mailto:nbeisert@itp.phys.ethz.ch}
  {\texttt{nbeisert@itp.phys.ethz.ch}}}
\hypersetup{pdfauthor={Niklas Beisert}}
\hypersetup{pdfsubject={Manual for the LaTeX2e Package childdoc}}
\date{30 December 2018, \textsf{v2.0}}
\maketitle

\begin{abstract}\noindent
\textsf{childdoc} is a \LaTeXe{} package
that enables the direct compilation
of document sections included by |\include|
to individual files.
\end{abstract}

\begingroup
\parskip0ex
\tableofcontents
\endgroup

%%%%%%%%%%%%%%%%%%%%%%%%%%%%%%%%%%%%%%%%%%%%%%%%%%%%%%%%%%%%%%%%%%%%%%%%%%%%%%%%
%%%%%%%%%%%%%%%%%%%%%%%%%%%%%%%%%%%%%%%%%%%%%%%%%%%%%%%%%%%%%%%%%%%%%%%%%%%%%%%%
\section{Introduction}

\LaTeX{} provides a mechanism to structure a large document (such as a book)
into a main file and several child files (containing the chapters)
using the |\include| command.
This mechanism is beneficial for documents
which span hundreds of pages in order to
make the source file(s) more manageable.
Moreover, compilation can be restricted to
selected child files by means of the |\includeonly| command.
The latter feature can be used to reduce the compilation time while editing
(this was significantly more useful in the earlier days of \LaTeX{})
or to generate a smaller document which is easier to navigate.
Another application of |\includeonly| is to generate
documents consisting of selected parts of the complete document.

However, there are a few drawbacks of the plain |\include| mechanism:
\begin{itemize}
\item
The child files cannot be compiled on their own,
they can only be compiled via the main file.
A naive editing environment
(such as a text editor with an option
to have the current file processed by \LaTeX)
may require one to switch to the main file before compiling;
attempting to compile the child file produces errors.
\item
The main file must be modified (each time)
to adjust the |\includeonly| command
to the present needs. This easily leaves the main file in a messy state.
\item
The generated document will always carry the filename
of the main document. This is inconvenient if
several child files are to be compiled and
to be kept for distribution.
\end{itemize}

The present package provides a simple interface
to make child files individually compilable by \LaTeX{}.
Compiling a child file then has the same effect as compiling
the main file with an |\includeonly| command
to select the appropriate child.
Moreover the generated document will carry the name of the child
rather than the main file.
This resolves all three above issues.

This feature is meant to make the editing of books,
thesis documents and lecture notes somewhat more convenient.
However, the package can also be used efficiently for
composing a series of documents (such as exercise sheets)
which are typically distributed individually.
It then assists the author in generating the individual documents
(potentially in different versions)
as well as a document containing the collected series.
Another application is in developing style files
or other kinds of included material
where compilation of the style file could redirect
to a sample or test file.

%%%%%%%%%%%%%%%%%%%%%%%%%%%%%%%%%%%%%%%%%%%%%%%%%%%%%%%%%%%%%%%%%%%%%%%%%%%%%%%%
%%%%%%%%%%%%%%%%%%%%%%%%%%%%%%%%%%%%%%%%%%%%%%%%%%%%%%%%%%%%%%%%%%%%%%%%%%%%%%%%
\section{Usage}

First of all, the package \textsf{childdoc} is \emph{not} a standard
\LaTeXe{} |.sty| style file! Therefore it needs to be invoked in
a non-standard way.

%%%%%%%%%%%%%%%%%%%%%%%%%%%%%%%%%%%%%%%%%%%%%%%%%%%%%%%%%%%%%%%%%%%%%%%%%%%%%%%%
\subsection{Included Files}
\label{sec:include}

%%%%%%%%%%%%%%%%%%%%%%%%%%%%%%%%%%%%%%%%
\DescribeMacro{\childdocmain}
To use the package, add the commands
\begin{center}
\begin{tabular}{l}
|\input{childdoc.def}|\\
|\childdocmain{}|\\
\end{tabular}
\end{center}
at the very top of the main \LaTeX{} file,
in particular \emph{before} the |\documentclass| statement!
The argument of |\childdocmain| should be left empty
(but it must be present).

%%%%%%%%%%%%%%%%%%%%%%%%%%%%%%%%%%%%%%%%
\DescribeMacro{\childdocof}
Furthermore, add the commands
\begin{center}
\begin{tabular}{l}
|\input{childdoc.def}|\\
|\childdocof{|\textit{main}|}|\\
\end{tabular}
\end{center}
at the top of every child file \textit{child}
which is included by |\include{|\textit{child}|}|
from within the main file
(or at least for those files to be compiled individually).
The argument \textit{main} must be the filename of the main file.

There are a couple of
considerations in setting up the main and child documents:

%%%%%%%%%%%%%%%%%%%%%%%%%%%%%%%%%%%%%%%%
\paragraph{Restrictions.}

Please note the following restrictions:
\begin{itemize}
\item
|\childdocmain| must be called with one argument \textit{main}
to ensure compatibility with earlier version of the package.
It must either be empty (|\childdocmain{}|)
or precisely match the filename of the main file in which it is specified.
See \secref{sec:detection} for further information.
\item
The filename \textit{main} must be specified without the |.tex| extension.
\item
The filename \textit{main} is case sensitive
(even in case-insensitive file systems)
due to internal string comparison.
\item
The argument \textit{main} should be fully expanded, it cannot be a macro.
\item
Subdirectories and special characters should be avoided in filenames.
\item
The command |\childdocmain{|\textit{main}|}| must be followed by a whitespace.
It should not be followed immediately by another command
or by a comment mark `|%|'.
This is because the \TeX{} parser reads the token immediately following
the argument of |\childdocmain| and puts it
at the beginning of every child section;
however, a white\-space is ignored.
\end{itemize}

%%%%%%%%%%%%%%%%%%%%%%%%%%%%%%%%%%%%%%%%
\paragraph{Content of Main File.}

It is advisable to place all content in the child files included by |\include|.
Any output contained in the main file will appear in all child documents
unless suppressed manually;
it cannot be suppressed automatically by the |\includeonly| directive
and thus should normally be avoided.
A method to include some content in the main file
by means of conditional processing is described in \secref{sec:conditional}.

%%%%%%%%%%%%%%%%%%%%%%%%%%%%%%%%%%%%%%%%
\paragraph{Page Numbering.}

When only a part of the document is compiled,
the appropriate numbering of pages
(as well as other status parameters)
is determined from the |.aux| files.
The latter contain information from previous passes.
However this information needs to propagate through
all intermediate child documents.
Therefore the page numbering in child documents may well
be inconsistent until the complete document is compiled at least once.

A useful (if unconventional) way to always ensure a consistent
page numbering is to restart the numbering in each child document
and denote the pages by `\textit{child}|.|\textit{page}'
where \textit{child} represents the chapter/section number of the child file.
This can be achieved by the command
|\numberwithin{page}{|\textit{child}|}|
of the \textsf{amsmath} package
where \textit{child} can be |chapter| or |section|
depending on the chosen structuring.
Alternatively, one can modify the macro |\thepage| appropriately
and reset the counter |page| at the start of each child file.

%%%%%%%%%%%%%%%%%%%%%%%%%%%%%%%%%%%%%%%%%%%%%%%%%%%%%%%%%%%%%%%%%%%%%%%%%%%%%%%%
\subsection{Conditional Processing}
\label{sec:conditional}

The package provides a mechanism to compile different versions
of a document. To customise the versions further some conditional processing
can come in handy to distinguish which version is being compiled.
The package provides two macros to describe the compilation context:

%%%%%%%%%%%%%%%%%%%%%%%%%%%%%%%%%%%%%%%%
\DescribeMacro{\ifchilddoc}
The conditional |\ifchilddoc| distinguishes between the compilation of
child documents and the main document:
%
\begin{center}
|\ifchilddoc |\textit{child-code}| |[|\||else |\textit{main-code}]| \||fi|
\end{center}

%%%%%%%%%%%%%%%%%%%%%%%%%%%%%%%%%%%%%%%%
\DescribeMacro{\childdocname}
\DescribeMacro{\childdocjob}
The macro |\childdocname| contains the filename (without extension)
of the main or child file being processed.
Note that |\childdocjob| will always contain the name of the main file.

%%%%%%%%%%%%%%%%%%%%%%%%%%%%%%%%%%%%%%%%
\paragraph{Title Page.}

Conditional processing can be used to include a title or banner page
in the main document when proper precautions are taken.
Importantly, the code in the main file should ensure that the page counter
(as well as other status parameters which are stored in the |.aux| files)
takes the same value after the conditional processing.
Otherwise the page numbers may take divergent values
depending on which part is compiled.

For example, a title page could be declared by:
%
\begin{center}
\begin{tabular}{l}
|\ifchilddoc\||else|\\
|\addtocounter{page}{-1}|\\
\textit{code for title page}\\
|\newpage|\\
|\||fi|
\end{tabular}
\end{center}
%
A banner page for the child documents can be generated by:
%
\begin{center}
\begin{tabular}{l}
|\ifchilddoc|\\
|\addtocounter{page}{-1}|\\
\textit{code for banner page}\\
|\newpage|\\
|\||fi|
\end{tabular}
\end{center}
%
Here one could write a message such as:
\begin{center}
|This is the part \childdocname{} of \childdocjob{}.|
\end{center}

%%%%%%%%%%%%%%%%%%%%%%%%%%%%%%%%%%%%%%%%%%%%%%%%%%%%%%%%%%%%%%%%%%%%%%%%%%%%%%%%
\subsection{Flags}
\label{sec:flags}

The package makes it easy to generate different versions
of the main or child documents.
To this end compilation flags can be defined
and assigned different default values.
They will be particularly useful in conjunction
with the forwarding mechanism described in \secref{sec:forward}.

For example, it may be useful to have a flag |\version|
which can be set to |draft| or |final|.
The document source will contain some conditional code
depending on the value of |\version|.
Suppose further, the flag should default to |final| for the main file
and to |draft| for child files
which is a natural assignment for editing the document.
This is achieved by placing the following code
in the preamble of the main document
(below the |\childdocmain| directive):
%
\begin{center}
\begin{tabular}{l}
|\ifchilddoc|\\
|\providecommand{\version}{draft}|\\
|\||else|\\
|\providecommand{\version}{final}|\\
|\||fi|
\end{tabular}
\end{center}
%
The definition by |\providecommand| makes sure
that previous definitions are not overwritten.
Further statements |\providecommand{\version}{...}|
can thus be added before the above code to override it.

For the main file, one might add a line
(between |\childdocmain| and the above block)
%
\begin{center}
|%\ifchilddoc\||else\providecommand{\version}{draft}\||fi|
\end{center}
%
which can be uncommented to produce a draft version.
Likewise one can add a line to the very top of a child file
(above the |\childdocof{|\textit{main}|}| directive)
%
\begin{center}
|%\providecommand{\version}{final}|
\end{center}
%
which can be uncommented to produce the final version of this child document.

%%%%%%%%%%%%%%%%%%%%%%%%%%%%%%%%%%%%%%%%%%%%%%%%%%%%%%%%%%%%%%%%%%%%%%%%%%%%%%%%
\subsection{Forwarding}
\label{sec:forward}

Different versions of the main or child documents
using compilation flags as described in \secref{sec:flags}
can be (permanently) stored in different files
for convenient compilation, viewing and distribution.
To this end, the package defines a command
to pass on compilation to a different file:

%%%%%%%%%%%%%%%%%%%%%%%%%%%%%%%%%%%%%%%%
\DescribeMacro{\childdocforward}
The command |\childdocforward| redirects processing to
another source file:
%
\begin{center}
\begin{tabular}{l}
|\input{childdoc.def}|\\
|\childdocforward[|\textit{main}|]{|\textit{dest}|}|\\
\end{tabular}
\end{center}
%
The argument \textit{dest} is the destination file
(without extension).
It should be the main file or one of the child files.
Note that further \textsf{childdoc} directives
such as |\childdocof| and |\childdocforward|
in the indicated file will be processed in this form.
The optional argument \textit{main}
passes on directly to the main file \textit{main}
while pretending to compile the child \textit{dest}.
This form behaves as if \textit{dest}
issues |\childdocof{|\textit{main}|}| right away,
and no further \textsf{childdoc} directives will be processed.

%%%%%%%%%%%%%%%%%%%%%%%%%%%%%%%%%%%%%%%%
\DescribeMacro{\...prefix}
In the alternative form |\childdocforwardprefix|,
%
\begin{center}
\begin{tabular}{l}
|\input{childdoc.def}|\\
|\childdocforwardprefix[|\textit{main}|]{|\textit{prefix}|}{|\textit{dest}|}|
\end{tabular}
\end{center}
%
the destination file is determined by a pattern
depending on the current file:
To make this work, the current file must be called
`{\textit{prefix}\hspace{0.2em}\textit{suffix}}'
with \textit{prefix} matching precisely the argument.
Processing is then passed on to the file
`{\textit{dest}\hspace{0.2em}\textit{suffix}}'.
Surely, the same effect is achieved by
directly specifying the
argument `{\textit{dest}\hspace{0.2em}\textit{suffix}}'
in the first form.
However, that requires to set up a different file
for each child. With the alternative form of the command
all these files can have exactly the same content
which simplifies setting them up and maintaining them.

For example, the following file |draft.tex|
with a compilation flag |\version| as described in \secref{sec:flags}
compiles the main document as a draft:
%
\begin{center}
\begin{tabular}{l}
|\def\version{draft}|\\
|\input{childdoc.def}|\\
|\childdocforward{|\textit{main}|}|
\end{tabular}
\end{center}
%
Likewise, the following files |final|\textit{nn}|.tex|
compile the final version of the child document
|child|\textit{nn}|.tex|:
%
\begin{center}
\begin{tabular}{l}
|\def\version{final}|\\
|\input{childdoc.def}|\\
|\childdocforwardprefix{final}{child}|
\end{tabular}
\end{center}
%

Note that when several versions of a main file and/or of each child file
are to be generated, it may be convenient to set up a |Makefile| or
shell script to automatise the process.

%%%%%%%%%%%%%%%%%%%%%%%%%%%%%%%%%%%%%%%%%%%%%%%%%%%%%%%%%%%%%%%%%%%%%%%%%%%%%%%%
\subsection{Command Line Processing}
\label{sec:commandline}

The effect of redirection files can also be achieved by invoking
the \LaTeX{} compiler with a more elaborate command line.
Most conveniently this should be done as part
of a shell script or a |Makefile|.

When using \textsf{childdoc} in the main file, the following
command lines effectively perform a redirection
(note that depending on the shell being used,
backslashes may have to be doubled: `|\|' $\to$ `|\\|'):
%
\begin{center}
|... -jobname "|\textit{target}|" |\\|"|[\textit{flags}]%
|\input{childdoc.def}\childdocforward[|\textit{main}|]{|\textit{dest}|}"|
\end{center}
%
Here \textit{target} is the name of the output file,
\textit{main} is the name of the main file
and \textit{dest} is the name of the main or child file to be processed
(all filenames without extensions).
The optional argument \textit{main} can be omitted
if \textit{main} matches \textit{dest}.
Optionally, compilation \textit{flags} can be defined via |\def| commands.
This command line makes the \TeX{} engine believe
it is compiling the file \textit{target}
whose content is specified as the latter parameter.
The provided code then forwards the processing to
\textit{main} or \textit{dest} as described in \secref{sec:forward}.

%%%%%%%%%%%%%%%%%%%%%%%%%%%%%%%%%%%%%%%%%%%%%%%%%%%%%%%%%%%%%%%%%%%%%%%%%%%%%%%%
\subsection{Include by Input}
\label{sec:input}

Including child documents by |\include| has some restrictions by design.
Most notably, the content of a child document always occupies
its own set of pages; pages cannot be shared between child documents.
Usually, this behaviour makes perfect sense
because each child document contain an essential part of the document.
However, in some situations it may be desirable to compose
a document from a collection of parts
without having mandatory page breaks between then.
For this case, the package
provides a mechanism to include parts
by |\input| which can also be processed individually.
However, by construction this mechanism
requires manual handling of the content to be output.

%%%%%%%%%%%%%%%%%%%%%%%%%%%%%%%%%%%%%%%%
\DescribeMacro{\ifchilddocmanual}
The main file should be prepared as usual, see \secref{sec:include}.
However, the document body must make a distinction
between processing of an individual part and of the main document, e.g.:
%
\begin{center}
\begin{tabular}{l}
|\ifchilddocmanual|\\
|\input{\childdocname}|\\
|\||else|\\
\textit{document body with }|\input{|\textit{part}|}|\\
|\||fi|
\end{tabular}
\end{center}
%
The conditional |\ifchilddocmanual| is true whenever
a part to be included by |\input| is being compiled,
and the name of the part is stored in |\childdocname|.

%%%%%%%%%%%%%%%%%%%%%%%%%%%%%%%%%%%%%%%%
\DescribeMacro{\childdocby}
Each part to be included by |\input| should start with:
%
\begin{center}
\begin{tabular}{l}
|\input{childdoc.def}|\\
|\childdocby{|\textit{main}|}|\\
\end{tabular}
\end{center}
%
The directive |\childdocby| is similar to |\childdocof|
described in \secref{sec:include},
but the subsequent selection of content must be done manually.
To that end, both |\ifchilddoc| and |\ifchilddocmanual|
will be true upon processing of a part,
and the name of the part is stored in |\childdocname|.
Note that |\jobname| will be set to the filename of the current part
so that each part receives an individual |.aux| file
that does not interfere with the |.aux| file(s) of the main document.
This behaviour can be altered by the alternative form
|\childdocby[*]{|\textit{main}|}| (with a non-empty optional argument)
which uses the |.aux| file of the main document
by setting |\jobname| to \textit{main}.

%%%%%%%%%%%%%%%%%%%%%%%%%%%%%%%%%%%%%%%%%%%%%%%%%%%%%%%%%%%%%%%%%%%%%%%%%%%%%%%%
\subsection{Driver Development}
\label{sec:driver}

The \textsf{childdoc} mechanism can also be use for the development
of definition files such as \LaTeX{} styles or classes.
This case differs from the above setup with multiple parts
included by |\include| in that no |\includeonly| should be invoked.
This can be achieved by starting the include file
(before |\ProvidesPackage|) with:
%
\begin{center}
\begin{tabular}{l}
|\input{childdoc.def}|\\
|\childdocforward{|\textit{main}|}|\\
\end{tabular}
\end{center}
%
or alternatively with:
%
\begin{center}
\begin{tabular}{l}
|\input{childdoc.def}|\\
|\childdocby{|\textit{main}|}|\\
\end{tabular}
\end{center}
%
Both forms have slightly different effects as described above.
The main file is prepared as usual, see \secref{sec:include}.

%%%%%%%%%%%%%%%%%%%%%%%%%%%%%%%%%%%%%%%%%%%%%%%%%%%%%%%%%%%%%%%%%%%%%%%%%%%%%%%%
\subsection{Legacy Detection}
\label{sec:detection}

The directive |\childdocmain| in the main file can detect
whether the complete document or merely a child is to be compiled
even without using the directive |\childdocof|.
This method is deprecated because it is less robust
and there is no compelling reason to use it;
it is merely provided for backward compatibility
and it may be removed in future versions.

If the detection mechanism is to be used,
it is mandatory to correctly specify
the filename of the main file as the argument of |\childdocmain|:
%
\begin{center}
\begin{tabular}{l}
|\input{childdoc.def}|\\
|\childdocmain{|\textit{main}|}|\\
\end{tabular}
\end{center}
%
If |\jobname| does not match the argument \textit{main} of |\childdocmain|,
it is assumed that |\jobname| points to the child file to be compiled.
When using |\childdocmain| with the main file specified as argument,
it suffices to start a child file
with just |\input{|\textit{main}|}|
without loading of the package and using |\childdocof|.
If instead all processing is done
with the appropriate \textsf{childdoc} directives,
the argument of \textit{main} of |\childdocmain| can be empty.

An alternative version of the command line processing described
in \secref{sec:commandline} using the detection mechanism reads:
%
\begin{center}
|... -jobname "|\textit{target}|" "|[\textit{flags}]%
[|\def\jobname{|\textit{dest}|}|]|\input{|\textit{main}|}"|
\end{center}

%%%%%%%%%%%%%%%%%%%%%%%%%%%%%%%%%%%%%%%%%%%%%%%%%%%%%%%%%%%%%%%%%%%%%%%%%%%%%%%%
\subsection{Manual Code}
\label{sec:manual}

In case one cannot be certain whether the definitions file |childdoc.def|
is installed on the target \TeX{} distribution
and one prefers not to ship it,
it is conceivable to paste a few relevant commands into the sources.

To that end, drop all statements |\input{childdoc.def}|
and perform the replacements as outlined below.
Instead of |\childdocmain{|\textit{main}|}| add the following code
to the top of the main file:
%
\begin{center}
\begin{tabular}{l}
|\||ifdefined\childdocname\endinput\||fi\newif\ifchilddoc|\\
|\edef\childdocname{\scantokens\expandafter{\jobname\noexpand}}|\\
|\def\childdocmain{|\textit{main}|}\||ifx\childdocmain\childdocname\||else|\\
|\childdoctrue\includeonly{\childdocname}\let\jobname\childdocmain\||fi|\\
\end{tabular}
\end{center}
%
Instead of |\childdocof{|\textit{main}|}| just include the main file
at the top of each child file:
%
\begin{center}
|\input{|\textit{main}|}|
\end{center}
%
A simple redirection |\childdocforward{|\textit{dest}|}| is achieved by:
%
\begin{center}
|\def\jobname{|\textit{dest}|}\input{\jobname}|
\end{center}
%
The redirection with prefix
|\childdocforwardprefix[|\textit{prefix}|]{|\textit{dest}|}|
is accomplished by:
%
\begin{center}
\begin{tabular}{l}
|{\edef\jobname{\scantokens\expandafter{\jobname\noexpand}}|\\
|\def\redirectjob |\textit{prefix}|#1~~~{\gdef\jobname{|\textit{dest}|#1}}|\\
|\expandafter\redirectjob\jobname~~~}\input{\jobname}|
\end{tabular}
\end{center}

In an alternative approach,
child documents can be compiled by a specific command line
without additional code or specific definitions:
%
\begin{center}
|... -jobname "|\textit{target}|" "|[\textit{flags}]%
|\includeonly{|\textit{dest}|}\input{|\textit{main}|}"|
\end{center}
%

%%%%%%%%%%%%%%%%%%%%%%%%%%%%%%%%%%%%%%%%%%%%%%%%%%%%%%%%%%%%%%%%%%%%%%%%%%%%%%%%
%%%%%%%%%%%%%%%%%%%%%%%%%%%%%%%%%%%%%%%%%%%%%%%%%%%%%%%%%%%%%%%%%%%%%%%%%%%%%%%%
\section{Information}

%%%%%%%%%%%%%%%%%%%%%%%%%%%%%%%%%%%%%%%%%%%%%%%%%%%%%%%%%%%%%%%%%%%%%%%%%%%%%%%%
\subsection{Copyright}

Copyright \copyright{} 2017--2018 Niklas Beisert

This work may be distributed and/or modified under the
conditions of the \LaTeX{} Project Public License, either version 1.3
of this license or (at your option) any later version.
The latest version of this license is in
  \url{http://www.latex-project.org/lppl.txt}
and version 1.3 or later is part of all distributions of \LaTeX{}
version 2005/12/01 or later.

This work has the LPPL maintenance status `maintained'.

The Current Maintainer of this work is Niklas Beisert.

This work consists of the files |README.txt|, |childdoc.ins| and |childdoc.dtx|
as well as the derived files |childdoc.def|, |cdocsamp.tex|
with |cdocsch1.tex|, |cdocsch2.tex|, |cdocspt3.tex|, |cdocspt4.tex|,
|cdocsdrf.tex|, |cdocsfn1.tex|, |cdocsfn2.tex|
as well as |childdoc.pdf|.

%%%%%%%%%%%%%%%%%%%%%%%%%%%%%%%%%%%%%%%%%%%%%%%%%%%%%%%%%%%%%%%%%%%%%%%%%%%%%%%%
\subsection{Files and Installation}

The package consists of the files:
%
\begin{center}
\begin{tabular}{ll}
    |README.txt|   & readme file \\
    |childdoc.ins| & installation file \\
    |childdoc.dtx| & source file \\
    |childdoc.def| & definition file \\
    |cdocsamp.tex| & sample main file \\
    |cdocsch1.tex| & sample include file \\
    |cdocsch2.tex| & sample include file \\
    |cdocspt3.tex| & sample part file \\
    |cdocspt4.tex| & sample part file \\
    |cdocsdrf.tex| & sample redirection file \\
    |cdocsfn1.tex| & sample redirection file \\
    |cdocsfn2.tex| & sample redirection file \\
    |childdoc.pdf| & manual
\end{tabular}
\end{center}
%
The distribution consists of the files
|README.txt|, |childdoc.ins| and |childdoc.dtx|.
%
\begin{itemize}
\item
Run (pdf)\LaTeX{} on |childdoc.dtx|
to compile the manual |childdoc.pdf| (this file).
\item
Run \LaTeX{} on |childdoc.ins| to create the definitions file |childdoc.def|
and the sample |cdocsamp.tex| with include files
|cdocsch1.tex|, |cdocsch2.tex|, |cdocspt3.tex|, |cdocspt4.tex|,
|cdocsdrf.tex|, |cdocsfn1.tex|, |cdocsfn2.tex|.
Then copy the file |childdoc.def| to an appropriate directory of your \LaTeX{}
distribution, e.g.\ \textit{texmf-root}|/tex/latex/childdoc|.
\end{itemize}

%%%%%%%%%%%%%%%%%%%%%%%%%%%%%%%%%%%%%%%%%%%%%%%%%%%%%%%%%%%%%%%%%%%%%%%%%%%%%%%%
\subsection{Related CTAN Packages}

There are several other packages which offer a similar functionality:
%
\begin{itemize}
\item
The packages
\href{http://ctan.org/pkg/docmute}{\textsf{docmute}},
\href{http://ctan.org/pkg/includex}{\textsf{includex}} and
\href{http://ctan.org/pkg/standalone}{\textsf{standalone}}
provide commands to include only the document body of
a child file thus allowing both files to be compiled individually.
\item
The packages \href{http://ctan.org/pkg/subdocs}{\textsf{subdocs}}
and \href{http://ctan.org/pkg/subfiles}{\textsf{subfiles}}
provide structures in which the main and child documents can be
encapsulated and allowing them to be compiled individually.
The inclusion mechanism is different from the conventional |\include|.
\item
The package \href{http://ctan.org/pkg/combine}{\textsf{combine}}
is an elaborate solution to combine several documents into one.
\end{itemize}
%
See also the CTAN topic \href{http://ctan.org/topic/subdocs}{\textsf{subdocs}}
for further related packages.
The present package differs from the above solutions in that
a document structure constructed with the conventional |\include| mechanism
just needs two extra commands at the top of every file
such that all constituent files can be compiled individually.

%%%%%%%%%%%%%%%%%%%%%%%%%%%%%%%%%%%%%%%%%%%%%%%%%%%%%%%%%%%%%%%%%%%%%%%%%%%%%%%%
%\subsection{Feature Suggestions}
%
%The following is a list of features which may be useful for future
%versions of this package:
%%
%\begin{itemize}
%\item
%\ldots
%\end{itemize}

%%%%%%%%%%%%%%%%%%%%%%%%%%%%%%%%%%%%%%%%%%%%%%%%%%%%%%%%%%%%%%%%%%%%%%%%%%%%%%%%
\subsection{Revision History}

%%%%%%%%%%%%%%%%%%%%%%%%%%%%%%%%%%%%%%%%
\paragraph{v2.0:} 2018/12/30

\begin{itemize}
\item
immediate forward processing
\item
added |\childdocby| mechanism
\item
manual restructured
\end{itemize}

%%%%%%%%%%%%%%%%%%%%%%%%%%%%%%%%%%%%%%%%
\paragraph{v1.6:} 2018/01/17

\begin{itemize}
\item
application for development of include files
\item
corrections to manual
\end{itemize}

%%%%%%%%%%%%%%%%%%%%%%%%%%%%%%%%%%%%%%%%
\paragraph{v1.5:} 2017/05/21

\begin{itemize}
\item
more complete structuring introduced
\item
|\childdocof| introduced
\item
|\childdoc| renamed to |\childdocmain|
\item
|\childredirect| renamed to |\childdocforward| and |\childdocforwardprefix|
and functionality expanded
\end{itemize}

%%%%%%%%%%%%%%%%%%%%%%%%%%%%%%%%%%%%%%%%
\paragraph{v1.0:} 2017/04/27

\begin{itemize}
\item
manual and install package
\item
first version published on CTAN
\end{itemize}

%%%%%%%%%%%%%%%%%%%%%%%%%%%%%%%%%%%%%%%%
\paragraph{v0.6:} 2017/04/26

\begin{itemize}
\item
redirection mechanism added
\end{itemize}

%%%%%%%%%%%%%%%%%%%%%%%%%%%%%%%%%%%%%%%%
\paragraph{v0.5:} 2017/04/26

\begin{itemize}
\item
functionality in definition file
\end{itemize}


%%%%%%%%%%%%%%%%%%%%%%%%%%%%%%%%%%%%%%%%%%%%%%%%%%%%%%%%%%%%%%%%%%%%%%%%%%%%%%%%
%%%%%%%%%%%%%%%%%%%%%%%%%%%%%%%%%%%%%%%%%%%%%%%%%%%%%%%%%%%%%%%%%%%%%%%%%%%%%%%%
%%%%%%%%%%%%%%%%%%%%%%%%%%%%%%%%%%%%%%%%%%%%%%%%%%%%%%%%%%%%%%%%%%%%%%%%%%%%%%%%
\appendix

\settowidth\MacroIndent{\rmfamily\scriptsize 000\ }

 \DocInput{childdoc.dtx}

\end{document}
%</driver>
% \fi
%
% %%%%%%%%%%%%%%%%%%%%%%%%%%%%%%%%%%%%%%%%%%%%%%%%%%%%%%%%%%%%%%%%%%%%%%%%%%%%%%
% %%%%%%%%%%%%%%%%%%%%%%%%%%%%%%%%%%%%%%%%%%%%%%%%%%%%%%%%%%%%%%%%%%%%%%%%%%%%%%
% \section{Sample}
%\iffalse
%<*samplemain>
%\fi
%
% The following presents a sample document
% with two chapters, two parts, a title page,
% a compile flag as well as three forwarding files to set the flag.
% It consists of eight |.tex| files:
% \begin{center}
% \begin{tabular}{ll}
% |cdocsamp.tex|&main file\\
% |cdocsch1.tex|&include file for chapter 1\\
% |cdocsch2.tex|&include file for chapter 2\\
% |cdocspt3.tex|&include file for part 3\\
% |cdocspt4.tex|&include file for part 4\\
% |cdocsdrf.tex|&forwarding file for main file in draft mode\\
% |cdocsfi1.tex|&forwarding file for final version of chapter 1\\
% |cdocsfi2.tex|&forwarding file for final version of chapter 2\\
% \end{tabular}
% \end{center}
% Each of the eight files can be compiled directly by the \LaTeX{} compiler.
%
% %%%%%%%%%%%%%%%%%%%%%%%%%%%%%%%%%%%%%%
% \paragraph{Main File.}
%
% The main file is called |cdocsamp.tex|.
%
% Load the \textsf{childdoc} definitions and
% declare the filename for the main document:
%    \begin{macrocode}
\input{childdoc.def}
\childdocmain{}
%    \end{macrocode}

% Optional override for |\version| flag:
%    \begin{macrocode}
%%\ifchilddoc\else\providecommand{\version}{draft}\fi
%    \end{macrocode}

% Define the default values for the |\version| flag
% (|final| for the main file and |draft| for childs):
%    \begin{macrocode}
\ifchilddoc
\providecommand{\version}{draft}
\else
\providecommand{\version}{final}
\fi
%    \end{macrocode}

% Load the standard document class:
%    \begin{macrocode}
\documentclass[12pt]{article}
%    \end{macrocode}

% Start the document body:
%    \begin{macrocode}
\begin{document}
%    \end{macrocode}

% Declare a title page.
% Print title, part of document being processed and version flag:
%    \begin{macrocode}
\addtocounter{page}{-1}
\begin{center}
{\LARGE\bfseries{}childdoc example\par}
\vspace{1cm}
\ifchilddoc
\ifchilddocmanual part\else chapter\fi:
`\childdocname' of `\childdocjob'\par
\else
main document: `\childdocjob'\par
\fi
version: \version\par
\end{center}
\newpage
%    \end{macrocode}

% Manually include selected file,
% otherwise process as usual:
%    \begin{macrocode}
\ifchilddocmanual
\section*{part `\childdocname'}
\input{\childdocname}
\else
%    \end{macrocode}

% Include the two chapters:
%    \begin{macrocode}
\include{cdocsch1}
\include{cdocsch2}
%    \end{macrocode}

% Include the two parts unless only chapters should be displayed:
%    \begin{macrocode}
\ifchilddoc\else
\section{part three}
\input{cdocspt3}
\section{part four}
\input{cdocspt4}
\fi
%    \end{macrocode}

% Process as usual until here:
%    \begin{macrocode}
\fi
%    \end{macrocode}

% End of document body:
%    \begin{macrocode}
\end{document}
%    \end{macrocode}
%\iffalse
%</samplemain>
%\fi
%
% %%%%%%%%%%%%%%%%%%%%%%%%%%%%%%%%%%%%%%
% \paragraph{Chapter Include Files.}
%
% The include files are called |cdocsch1.tex| and |cdocsch2.tex|.
%
%\iffalse
%<*samplechap1|samplechap2>
%\fi

% Optional override for |\version| flag:
%    \begin{macrocode}
%%\providecommand{\version}{final}
%    \end{macrocode}

% Include the main document:
%    \begin{macrocode}
\input{childdoc.def}
\childdocof{cdocsamp}
%    \end{macrocode}

%\iffalse
%</samplechap1|samplechap2>
%\fi
%
%\iffalse
%<*samplechap1>
%\fi
% Some text for chapter 1:
%    \begin{macrocode}
\section{one}
some text in chapter one
%    \end{macrocode}

%\iffalse
%</samplechap1>
%\fi
% Some text for chapter 2:
%\iffalse
%<*samplechap2>
%\fi
%    \begin{macrocode}
\section{two}
more text in chapter two
%    \end{macrocode}

%\iffalse
%</samplechap2>
%\fi
%
% %%%%%%%%%%%%%%%%%%%%%%%%%%%%%%%%%%%%%%
% \paragraph{Part Include Files.}
%
% The include files are called |cdocspt3.tex| and |cdocspt4.tex|.
%
%\iffalse
%<*samplepart3|samplepart4>
%\fi

% Optional override for |\version| flag:
%    \begin{macrocode}
%%\providecommand{\version}{final}
%    \end{macrocode}

% Include the main document:
%    \begin{macrocode}
\input{childdoc.def}
\childdocby{cdocsamp}
%    \end{macrocode}

%\iffalse
%</samplepart3|samplepart4>
%\fi
%
%\iffalse
%<*samplepart3>
%\fi
% Some text for part 3:
%    \begin{macrocode}
some text in part three
%    \end{macrocode}

%\iffalse
%</samplepart3>
%\fi
% Some text for part 4:
%\iffalse
%<*samplepart4>
%\fi
%    \begin{macrocode}
more text in part four
%    \end{macrocode}

%\iffalse
%</samplepart4>
%\fi
%
% %%%%%%%%%%%%%%%%%%%%%%%%%%%%%%%%%%%%%%
% \paragraph{Forwarding for a Complete Draft.}
%
% The following forwarding file |cdocsdrf.tex|
% compiles the main document in draft mode:
%\iffalse
%<*sampledraft>
%\fi
%    \begin{macrocode}
\def\version{draft}
\input{childdoc.def}
\childdocforward{cdocsamp}
%    \end{macrocode}

%\iffalse
%</sampledraft>
%\fi
%
% %%%%%%%%%%%%%%%%%%%%%%%%%%%%%%%%%%%%%%
% \paragraph{Forwarding for Final Version of the Chapters.}
%
% The following forwarding files |cdocsfn1.tex| and |cdocsfn2.tex|
% (with identical content)
% compile the final versions of the child documents
% |cdocsch1.tex| and |cdocsch2.tex|, respectively:
%\iffalse
%<*samplefinal>
%\fi
%    \begin{macrocode}
\def\version{final}
\input{childdoc.def}
\childdocforwardprefix[cdocsamp]{cdocsfn}{cdocsch}
%    \end{macrocode}

%\iffalse
%</samplefinal>
%\fi
%
% %%%%%%%%%%%%%%%%%%%%%%%%%%%%%%%%%%%%%%
% \paragraph{Command Line Processing.}
%
% The following three command lines generate the output files
% |cdocscld|, |cdocscl1| and |cdocscl2|
% which should be identical to
% |cdocsdrf|, |cdocsch1| and |cdocsfn2|, respectively:
% \begin{center}
% \begin{tabular}{l}
% |latex -jobname cdocscld \|\\
% |  "\def\version{draft}\input{childdoc.def}\childdocforward{cdocsamp}"|\\
% |latex -jobname cdocscl1 \|\\
% |  "\input{childdoc.def}\childdocforward[cdocsamp]{cdocsch1}"|\\
% |latex -jobname cdocscl2 \|\\
% |  "\def\version{final}\input{childdoc.def}\childdocforward{cdocsch2}"|
% \end{tabular}
% \end{center}
% Note that the trailing backslash on each first line
% merely continues the input to the second line
% (for convenient cut ant paste).
% Furthermore, the command |latex| can be replaced by any
% of its alternative versions such as |pdflatex|.
%
% %%%%%%%%%%%%%%%%%%%%%%%%%%%%%%%%%%%%%%%%%%%%%%%%%%%%%%%%%%%%%%%%%%%%%%%%%%%%%%
% %%%%%%%%%%%%%%%%%%%%%%%%%%%%%%%%%%%%%%%%%%%%%%%%%%%%%%%%%%%%%%%%%%%%%%%%%%%%%%
% \section{Implementation}
%\iffalse
%<*package>
%\fi
%
% This section describes the definitions file |childdoc.def|.

% The definitions cannot be loaded using |\usepackage| or |\RequirePackage|
% which has a mechanism to prevent loading a style file more than once.
% When loading the definitions by means of |\input|
% multiple instances have to be prevented manually:
%\iffalse
%This code needs to be before the `\ProvidesFile' directive
%which is defined at the beginning of this file.
%Therefore it is also placed there and commented out here.
%</package>
%<*discard>
%\fi
%    \begin{macrocode}
\ifdefined\childdocmain\endinput\fi
%    \end{macrocode}
%\iffalse
%</discard>
%<*package>
%\fi
%
% \macro{\ifchilddoc}
% \macro{\ifchilddocmanual}
% The conditional |\ifchilddoc| tells whether a
% child (true) or main (false) document is being compiled.
% The conditional |\ifchilddocmanual| tells whether
% the |\includeonly| mechanism is used (false) or
% the selection of child files must be performed manually (true).
% The definitions initialise to false:
%    \begin{macrocode}
\newif\ifchilddoc
\newif\ifchilddocmanual
%    \end{macrocode}

% \macro{\childdocname}
% \macro{\childdocjob}
% The macro |\childdocname| stores the name of the main document
% to be compiled. The macro |\childdocjob| stores the name of
% the document on which the \LaTeX{} compiler was originally invoked.
% The content of |\jobname| cannot be compared
% to filenames specified in the source due to different catcodes.
% The following code rescans |\jobname|, stores the result
% in |\childdocname| and saves a copy in |\childdocjob|:
%    \begin{macrocode}
\edef\childdocname{\scantokens\expandafter{\jobname\noexpand}}
\let\childdocjob\childdocname
%    \end{macrocode}

% \macro{\childdocdisable}
% The macro |\childdocdisable| prevents the main file
% from being processed more than once.
% At this stage, the main document command |\childdocmain|
% is assumed to be called once again where it should do nothing.
% Any subsequent call to it should prevent
% a secondary processing of the main document
% It overwrites the forwarding commands
% |\childdocof| and |\childdocforward|
% with empty macros to prevent further inclusions of the main document:
%    \begin{macrocode}
\newcommand{\childdocdisable}
{
  \renewcommand{\childdocmain}[1]{\renewcommand{\childdocmain}[1]{\endinput}}
  \renewcommand{\childdocof}[1]{}
  \renewcommand{\childdocby}[2][]{}
  \renewcommand{\childdocforward}[2][]{}
  \renewcommand{\childdocdisable}{}
}
%    \end{macrocode}

% \macro{\childdocmain}
% The macro |\childdocmain| is to be called at the top of the main file
% with nothing or the main filename (without extension) as argument.
% First, it breaks loops.
% If the argument is not empty and does not match |\childdocname|
% (which is set by the first inclusion of |childdoc.def|),
% |\ifchilddoc| is set to true, |\includeonly| is applied to the child file
% and |\jobname| is set to the main file
% (for proper handling of |.aux| files):
%    \begin{macrocode}
\newcommand{\childdocmain}[1]
{
  \childdocdisable\childdocmain{}
  \if?#1?\else
    \begingroup
      \def\childdoctmp{#1}
      \ifx\childdoctmp\childdocname
        \def\childdoctmp{}
      \else
        \def\childdoctmp
        {
          \childdoctrue
          \includeonly{\childdocname}
          \def\childdocjob{#1}
          \def\jobname{#1}
        }
      \fi
      \expandafter
    \endgroup
    \childdoctmp
  \fi
}
%    \end{macrocode}

% \macro{\childdocof}
% The command |\childdocof| redirects
% compilation to the main file |#1|.
%    \begin{macrocode}
\newcommand{\childdocof}[1]
{
  \childdocdisable
  \childdoctrue
  \includeonly{\childdocname}
  \def\jobname{#1}
  \def\childdocjob{#1}
  \input{#1}
}
%    \end{macrocode}

% \macro{\childdocby}
% The command |\childdocby| ....
%    \begin{macrocode}
\newcommand{\childdocby}[2][]
{
  \childdocdisable
  \childdoctrue
  \childdocmanualtrue
  \if?#1?\else
    \def\jobname{#2}
  \fi
  \def\childdocjob{#2}
  \input{#2}
  \endinput
}
%    \end{macrocode}

% \macro{\childdocforward}
% The command |\childdocforward| redirects
% compilation to the main file or
% (if the optional argument is given) a child file.
% Parameters are set as if the main file
% or a child file starting with |\childdocof| was compiled.
% Then compilation is handed over to the main file:
%    \begin{macrocode}
\newcommand{\childdocforward}[2][]
{
  \begingroup
    \if?#1?
      \def\childdoctmp
      {
        \def\childdocname{#2}
        \def\childdocjob{#2}
        \def\jobname{#2}
        \input{#2}
        \endinput
      }
    \else
      \def\childdoctmp
      {
        \childdocdisable
        \def\childdocname{#2}
        \childdoctrue
        \includeonly{#2}
        \def\childdocjob{#1}
        \def\jobname{#1}
        \input{#1}
        \endinput
      }
    \fi
    \expandafter
  \endgroup
  \childdoctmp
}
%    \end{macrocode}

% \macro{\childdocforwardprefix}
% The command |\childdocforwardprefix| redirects
% compilation to the main or a child file by means of a pattern.
% The prefix |#1| in the current filename is replaced by |#2|
% and the suffix of the current filename is kept
% (it is assumed that the filename does not contain the substring `|~~~|'
% which is used as a delimiter).
% Compilation is handed over to the new file by |\childdocforward|:
%    \begin{macrocode}
\newcommand{\childdocforwardprefix}[3][]
{
  \begingroup
    \def\childdocextract #2##1~~~{\def\childdoctmp{\childdocforward[#1]{#3##1}}}
    \expandafter\childdocextract\childdocname~~~
    \expandafter
  \endgroup
  \childdoctmp
}
%    \end{macrocode}

% \macro{\childdoc}
% The deprecated macro |\childdoc| is a legacy version of |\childdocmain|:
%    \begin{macrocode}
\newcommand{\childdoc}{\childdocmain}
%    \end{macrocode}

% \macro{\childdocredirect}
% The deprecated macro |\childdocredirect| is a legacy version
% of |\childdocforward| and |\childdocforwardprefix|:
%    \begin{macrocode}
\newcommand{\childdocredirect}[2][]
{
  \begingroup
    \if?#1?
      \def\childdoctmp{\childdocforward{#2}}
    \else
      \def\childdoctmp{\childdocforwardprefix{#1}{#2}}
    \fi
    \expandafter
  \endgroup
  \childdoctmp
}
%    \end{macrocode}

%\iffalse
%</package>
%\fi
%
\endinput
\childdocforward{cdocsch2}"|
% \end{tabular}
% \end{center}
% Note that the trailing backslash on each first line
% merely continues the input to the second line
% (for convenient cut ant paste).
% Furthermore, the command |latex| can be replaced by any
% of its alternative versions such as |pdflatex|.
%
% %%%%%%%%%%%%%%%%%%%%%%%%%%%%%%%%%%%%%%%%%%%%%%%%%%%%%%%%%%%%%%%%%%%%%%%%%%%%%%
% %%%%%%%%%%%%%%%%%%%%%%%%%%%%%%%%%%%%%%%%%%%%%%%%%%%%%%%%%%%%%%%%%%%%%%%%%%%%%%
% \section{Implementation}
%\iffalse
%<*package>
%\fi
%
% This section describes the definitions file |childdoc.def|.

% The definitions cannot be loaded using |\usepackage| or |\RequirePackage|
% which has a mechanism to prevent loading a style file more than once.
% When loading the definitions by means of |\input|
% multiple instances have to be prevented manually:
%\iffalse
%This code needs to be before the `\ProvidesFile' directive
%which is defined at the beginning of this file.
%Therefore it is also placed there and commented out here.
%</package>
%<*discard>
%\fi
%    \begin{macrocode}
\ifdefined\childdocmain\endinput\fi
%    \end{macrocode}
%\iffalse
%</discard>
%<*package>
%\fi
%
% \macro{\ifchilddoc}
% \macro{\ifchilddocmanual}
% The conditional |\ifchilddoc| tells whether a
% child (true) or main (false) document is being compiled.
% The conditional |\ifchilddocmanual| tells whether
% the |\includeonly| mechanism is used (false) or
% the selection of child files must be performed manually (true).
% The definitions initialise to false:
%    \begin{macrocode}
\newif\ifchilddoc
\newif\ifchilddocmanual
%    \end{macrocode}

% \macro{\childdocname}
% \macro{\childdocjob}
% The macro |\childdocname| stores the name of the main document
% to be compiled. The macro |\childdocjob| stores the name of
% the document on which the \LaTeX{} compiler was originally invoked.
% The content of |\jobname| cannot be compared
% to filenames specified in the source due to different catcodes.
% The following code rescans |\jobname|, stores the result
% in |\childdocname| and saves a copy in |\childdocjob|:
%    \begin{macrocode}
\edef\childdocname{\scantokens\expandafter{\jobname\noexpand}}
\let\childdocjob\childdocname
%    \end{macrocode}

% \macro{\childdocdisable}
% The macro |\childdocdisable| prevents the main file
% from being processed more than once.
% At this stage, the main document command |\childdocmain|
% is assumed to be called once again where it should do nothing.
% Any subsequent call to it should prevent
% a secondary processing of the main document
% It overwrites the forwarding commands
% |\childdocof| and |\childdocforward|
% with empty macros to prevent further inclusions of the main document:
%    \begin{macrocode}
\newcommand{\childdocdisable}
{
  \renewcommand{\childdocmain}[1]{\renewcommand{\childdocmain}[1]{\endinput}}
  \renewcommand{\childdocof}[1]{}
  \renewcommand{\childdocby}[2][]{}
  \renewcommand{\childdocforward}[2][]{}
  \renewcommand{\childdocdisable}{}
}
%    \end{macrocode}

% \macro{\childdocmain}
% The macro |\childdocmain| is to be called at the top of the main file
% with nothing or the main filename (without extension) as argument.
% First, it breaks loops.
% If the argument is not empty and does not match |\childdocname|
% (which is set by the first inclusion of |childdoc.def|),
% |\ifchilddoc| is set to true, |\includeonly| is applied to the child file
% and |\jobname| is set to the main file
% (for proper handling of |.aux| files):
%    \begin{macrocode}
\newcommand{\childdocmain}[1]
{
  \childdocdisable\childdocmain{}
  \if?#1?\else
    \begingroup
      \def\childdoctmp{#1}
      \ifx\childdoctmp\childdocname
        \def\childdoctmp{}
      \else
        \def\childdoctmp
        {
          \childdoctrue
          \includeonly{\childdocname}
          \def\childdocjob{#1}
          \def\jobname{#1}
        }
      \fi
      \expandafter
    \endgroup
    \childdoctmp
  \fi
}
%    \end{macrocode}

% \macro{\childdocof}
% The command |\childdocof| redirects
% compilation to the main file |#1|.
%    \begin{macrocode}
\newcommand{\childdocof}[1]
{
  \childdocdisable
  \childdoctrue
  \includeonly{\childdocname}
  \def\jobname{#1}
  \def\childdocjob{#1}
  \input{#1}
}
%    \end{macrocode}

% \macro{\childdocby}
% The command |\childdocby| ....
%    \begin{macrocode}
\newcommand{\childdocby}[2][]
{
  \childdocdisable
  \childdoctrue
  \childdocmanualtrue
  \if?#1?\else
    \def\jobname{#2}
  \fi
  \def\childdocjob{#2}
  \input{#2}
  \endinput
}
%    \end{macrocode}

% \macro{\childdocforward}
% The command |\childdocforward| redirects
% compilation to the main file or
% (if the optional argument is given) a child file.
% Parameters are set as if the main file
% or a child file starting with |\childdocof| was compiled.
% Then compilation is handed over to the main file:
%    \begin{macrocode}
\newcommand{\childdocforward}[2][]
{
  \begingroup
    \if?#1?
      \def\childdoctmp
      {
        \def\childdocname{#2}
        \def\childdocjob{#2}
        \def\jobname{#2}
        \input{#2}
        \endinput
      }
    \else
      \def\childdoctmp
      {
        \childdocdisable
        \def\childdocname{#2}
        \childdoctrue
        \includeonly{#2}
        \def\childdocjob{#1}
        \def\jobname{#1}
        \input{#1}
        \endinput
      }
    \fi
    \expandafter
  \endgroup
  \childdoctmp
}
%    \end{macrocode}

% \macro{\childdocforwardprefix}
% The command |\childdocforwardprefix| redirects
% compilation to the main or a child file by means of a pattern.
% The prefix |#1| in the current filename is replaced by |#2|
% and the suffix of the current filename is kept
% (it is assumed that the filename does not contain the substring `|~~~|'
% which is used as a delimiter).
% Compilation is handed over to the new file by |\childdocforward|:
%    \begin{macrocode}
\newcommand{\childdocforwardprefix}[3][]
{
  \begingroup
    \def\childdocextract #2##1~~~{\def\childdoctmp{\childdocforward[#1]{#3##1}}}
    \expandafter\childdocextract\childdocname~~~
    \expandafter
  \endgroup
  \childdoctmp
}
%    \end{macrocode}

% \macro{\childdoc}
% The deprecated macro |\childdoc| is a legacy version of |\childdocmain|:
%    \begin{macrocode}
\newcommand{\childdoc}{\childdocmain}
%    \end{macrocode}

% \macro{\childdocredirect}
% The deprecated macro |\childdocredirect| is a legacy version
% of |\childdocforward| and |\childdocforwardprefix|:
%    \begin{macrocode}
\newcommand{\childdocredirect}[2][]
{
  \begingroup
    \if?#1?
      \def\childdoctmp{\childdocforward{#2}}
    \else
      \def\childdoctmp{\childdocforwardprefix{#1}{#2}}
    \fi
    \expandafter
  \endgroup
  \childdoctmp
}
%    \end{macrocode}

%\iffalse
%</package>
%\fi
%
\endinput
|\\
|\childdocby{|\textit{main}|}|\\
\end{tabular}
\end{center}
%
The directive |\childdocby| is similar to |\childdocof|
described in \secref{sec:include},
but the subsequent selection of content must be done manually.
To that end, both |\ifchilddoc| and |\ifchilddocmanual|
will be true upon processing of a part,
and the name of the part is stored in |\childdocname|.
Note that |\jobname| will be set to the filename of the current part
so that each part receives an individual |.aux| file
that does not interfere with the |.aux| file(s) of the main document.
This behaviour can be altered by the alternative form
|\childdocby[*]{|\textit{main}|}| (with a non-empty optional argument)
which uses the |.aux| file of the main document
by setting |\jobname| to \textit{main}.

%%%%%%%%%%%%%%%%%%%%%%%%%%%%%%%%%%%%%%%%%%%%%%%%%%%%%%%%%%%%%%%%%%%%%%%%%%%%%%%%
\subsection{Driver Development}
\label{sec:driver}

The \textsf{childdoc} mechanism can also be use for the development
of definition files such as \LaTeX{} styles or classes.
This case differs from the above setup with multiple parts
included by |\include| in that no |\includeonly| should be invoked.
This can be achieved by starting the include file
(before |\ProvidesPackage|) with:
%
\begin{center}
\begin{tabular}{l}
|% \iffalse
%
% childdoc.dtx Copyright (C) 2017-2018 Niklas Beisert
%
% This work may be distributed and/or modified under the
% conditions of the LaTeX Project Public License, either version 1.3
% of this license or (at your option) any later version.
% The latest version of this license is in
%   http://www.latex-project.org/lppl.txt
% and version 1.3 or later is part of all distributions of LaTeX
% version 2005/12/01 or later.
%
% This work has the LPPL maintenance status `maintained'.
%
% The Current Maintainer of this work is Niklas Beisert.
%
% This work consists of the files childdoc.dtx and childdoc.ins
% and the derived files childdoc.def and cdocsamp.tex with
% cdocsch1.tex, cdocsch2.tex, cdocsdrf.tex, cdocsfn1.tex, cdocsfn2.tex.
%
%<package>\ifdefined\childdocmain\endinput\fi
%<package>\ProvidesFile{childdoc.def}[2018/12/30 v2.0 child document driver]
%<samplemain>\ProvidesFile{cdocsamp.tex}[2018/12/30 v2.0 sample for childdoc]
%<*driver>
%\ProvidesFile{childdoc.drv}[2018/12/30 v2.0 childdoc reference manual file]
\PassOptionsToClass{10pt,a4paper}{article}
\documentclass{ltxdoc}

\usepackage[margin=35mm]{geometry}
\usepackage{hyperref}
\usepackage{hyperxmp}
\usepackage[usenames]{color}

\hypersetup{colorlinks=true}
\hypersetup{pdfstartview=FitH}
\hypersetup{pdfpagemode=UseNone}
\hypersetup{pdfsource={}}
\hypersetup{pdflang={en-UK}}
\hypersetup{pdfcopyright={Copyright 2017-2018 Niklas Beisert.
  This work may be distributed and/or modified under the
  conditions of the LaTeX Project Public License, either version 1.3
  of this license or (at your option) any later version.}}
\hypersetup{pdflicenseurl={http://www.latex-project.org/lppl.txt}}
\hypersetup{pdfcontactaddress={ETH Zurich, ITP, HIT K,
  Wolfgang-Pauli-Strasse 27}}
\hypersetup{pdfcontactpostcode={8093}}
\hypersetup{pdfcontactcity={Zurich}}
\hypersetup{pdfcontactcountry={Switzerland}}
\hypersetup{pdfcontactemail={nbeisert@itp.phys.ethz.ch}}
\hypersetup{pdfcontacturl={http://people.phys.ethz.ch/\xmptilde nbeisert/}}

\newcommand{\secref}[1]{\hyperref[#1]{section \ref*{#1}}}

\parskip1ex
\parindent0pt
\let\olditemize\itemize
\def\itemize{\olditemize\parskip0pt}

\begin{document}

\title{The \textsf{childdoc} Package}
\hypersetup{pdftitle={The childdoc Package}}
\author{Niklas Beisert\\[2ex]
  Institut f\"ur Theoretische Physik\\
  Eidgen\"ossische Technische Hochschule Z\"urich\\
  Wolfgang-Pauli-Strasse 27, 8093 Z\"urich, Switzerland\\[1ex]
  \href{mailto:nbeisert@itp.phys.ethz.ch}
  {\texttt{nbeisert@itp.phys.ethz.ch}}}
\hypersetup{pdfauthor={Niklas Beisert}}
\hypersetup{pdfsubject={Manual for the LaTeX2e Package childdoc}}
\date{30 December 2018, \textsf{v2.0}}
\maketitle

\begin{abstract}\noindent
\textsf{childdoc} is a \LaTeXe{} package
that enables the direct compilation
of document sections included by |\include|
to individual files.
\end{abstract}

\begingroup
\parskip0ex
\tableofcontents
\endgroup

%%%%%%%%%%%%%%%%%%%%%%%%%%%%%%%%%%%%%%%%%%%%%%%%%%%%%%%%%%%%%%%%%%%%%%%%%%%%%%%%
%%%%%%%%%%%%%%%%%%%%%%%%%%%%%%%%%%%%%%%%%%%%%%%%%%%%%%%%%%%%%%%%%%%%%%%%%%%%%%%%
\section{Introduction}

\LaTeX{} provides a mechanism to structure a large document (such as a book)
into a main file and several child files (containing the chapters)
using the |\include| command.
This mechanism is beneficial for documents
which span hundreds of pages in order to
make the source file(s) more manageable.
Moreover, compilation can be restricted to
selected child files by means of the |\includeonly| command.
The latter feature can be used to reduce the compilation time while editing
(this was significantly more useful in the earlier days of \LaTeX{})
or to generate a smaller document which is easier to navigate.
Another application of |\includeonly| is to generate
documents consisting of selected parts of the complete document.

However, there are a few drawbacks of the plain |\include| mechanism:
\begin{itemize}
\item
The child files cannot be compiled on their own,
they can only be compiled via the main file.
A naive editing environment
(such as a text editor with an option
to have the current file processed by \LaTeX)
may require one to switch to the main file before compiling;
attempting to compile the child file produces errors.
\item
The main file must be modified (each time)
to adjust the |\includeonly| command
to the present needs. This easily leaves the main file in a messy state.
\item
The generated document will always carry the filename
of the main document. This is inconvenient if
several child files are to be compiled and
to be kept for distribution.
\end{itemize}

The present package provides a simple interface
to make child files individually compilable by \LaTeX{}.
Compiling a child file then has the same effect as compiling
the main file with an |\includeonly| command
to select the appropriate child.
Moreover the generated document will carry the name of the child
rather than the main file.
This resolves all three above issues.

This feature is meant to make the editing of books,
thesis documents and lecture notes somewhat more convenient.
However, the package can also be used efficiently for
composing a series of documents (such as exercise sheets)
which are typically distributed individually.
It then assists the author in generating the individual documents
(potentially in different versions)
as well as a document containing the collected series.
Another application is in developing style files
or other kinds of included material
where compilation of the style file could redirect
to a sample or test file.

%%%%%%%%%%%%%%%%%%%%%%%%%%%%%%%%%%%%%%%%%%%%%%%%%%%%%%%%%%%%%%%%%%%%%%%%%%%%%%%%
%%%%%%%%%%%%%%%%%%%%%%%%%%%%%%%%%%%%%%%%%%%%%%%%%%%%%%%%%%%%%%%%%%%%%%%%%%%%%%%%
\section{Usage}

First of all, the package \textsf{childdoc} is \emph{not} a standard
\LaTeXe{} |.sty| style file! Therefore it needs to be invoked in
a non-standard way.

%%%%%%%%%%%%%%%%%%%%%%%%%%%%%%%%%%%%%%%%%%%%%%%%%%%%%%%%%%%%%%%%%%%%%%%%%%%%%%%%
\subsection{Included Files}
\label{sec:include}

%%%%%%%%%%%%%%%%%%%%%%%%%%%%%%%%%%%%%%%%
\DescribeMacro{\childdocmain}
To use the package, add the commands
\begin{center}
\begin{tabular}{l}
|% \iffalse
%
% childdoc.dtx Copyright (C) 2017-2018 Niklas Beisert
%
% This work may be distributed and/or modified under the
% conditions of the LaTeX Project Public License, either version 1.3
% of this license or (at your option) any later version.
% The latest version of this license is in
%   http://www.latex-project.org/lppl.txt
% and version 1.3 or later is part of all distributions of LaTeX
% version 2005/12/01 or later.
%
% This work has the LPPL maintenance status `maintained'.
%
% The Current Maintainer of this work is Niklas Beisert.
%
% This work consists of the files childdoc.dtx and childdoc.ins
% and the derived files childdoc.def and cdocsamp.tex with
% cdocsch1.tex, cdocsch2.tex, cdocsdrf.tex, cdocsfn1.tex, cdocsfn2.tex.
%
%<package>\ifdefined\childdocmain\endinput\fi
%<package>\ProvidesFile{childdoc.def}[2018/12/30 v2.0 child document driver]
%<samplemain>\ProvidesFile{cdocsamp.tex}[2018/12/30 v2.0 sample for childdoc]
%<*driver>
%\ProvidesFile{childdoc.drv}[2018/12/30 v2.0 childdoc reference manual file]
\PassOptionsToClass{10pt,a4paper}{article}
\documentclass{ltxdoc}

\usepackage[margin=35mm]{geometry}
\usepackage{hyperref}
\usepackage{hyperxmp}
\usepackage[usenames]{color}

\hypersetup{colorlinks=true}
\hypersetup{pdfstartview=FitH}
\hypersetup{pdfpagemode=UseNone}
\hypersetup{pdfsource={}}
\hypersetup{pdflang={en-UK}}
\hypersetup{pdfcopyright={Copyright 2017-2018 Niklas Beisert.
  This work may be distributed and/or modified under the
  conditions of the LaTeX Project Public License, either version 1.3
  of this license or (at your option) any later version.}}
\hypersetup{pdflicenseurl={http://www.latex-project.org/lppl.txt}}
\hypersetup{pdfcontactaddress={ETH Zurich, ITP, HIT K,
  Wolfgang-Pauli-Strasse 27}}
\hypersetup{pdfcontactpostcode={8093}}
\hypersetup{pdfcontactcity={Zurich}}
\hypersetup{pdfcontactcountry={Switzerland}}
\hypersetup{pdfcontactemail={nbeisert@itp.phys.ethz.ch}}
\hypersetup{pdfcontacturl={http://people.phys.ethz.ch/\xmptilde nbeisert/}}

\newcommand{\secref}[1]{\hyperref[#1]{section \ref*{#1}}}

\parskip1ex
\parindent0pt
\let\olditemize\itemize
\def\itemize{\olditemize\parskip0pt}

\begin{document}

\title{The \textsf{childdoc} Package}
\hypersetup{pdftitle={The childdoc Package}}
\author{Niklas Beisert\\[2ex]
  Institut f\"ur Theoretische Physik\\
  Eidgen\"ossische Technische Hochschule Z\"urich\\
  Wolfgang-Pauli-Strasse 27, 8093 Z\"urich, Switzerland\\[1ex]
  \href{mailto:nbeisert@itp.phys.ethz.ch}
  {\texttt{nbeisert@itp.phys.ethz.ch}}}
\hypersetup{pdfauthor={Niklas Beisert}}
\hypersetup{pdfsubject={Manual for the LaTeX2e Package childdoc}}
\date{30 December 2018, \textsf{v2.0}}
\maketitle

\begin{abstract}\noindent
\textsf{childdoc} is a \LaTeXe{} package
that enables the direct compilation
of document sections included by |\include|
to individual files.
\end{abstract}

\begingroup
\parskip0ex
\tableofcontents
\endgroup

%%%%%%%%%%%%%%%%%%%%%%%%%%%%%%%%%%%%%%%%%%%%%%%%%%%%%%%%%%%%%%%%%%%%%%%%%%%%%%%%
%%%%%%%%%%%%%%%%%%%%%%%%%%%%%%%%%%%%%%%%%%%%%%%%%%%%%%%%%%%%%%%%%%%%%%%%%%%%%%%%
\section{Introduction}

\LaTeX{} provides a mechanism to structure a large document (such as a book)
into a main file and several child files (containing the chapters)
using the |\include| command.
This mechanism is beneficial for documents
which span hundreds of pages in order to
make the source file(s) more manageable.
Moreover, compilation can be restricted to
selected child files by means of the |\includeonly| command.
The latter feature can be used to reduce the compilation time while editing
(this was significantly more useful in the earlier days of \LaTeX{})
or to generate a smaller document which is easier to navigate.
Another application of |\includeonly| is to generate
documents consisting of selected parts of the complete document.

However, there are a few drawbacks of the plain |\include| mechanism:
\begin{itemize}
\item
The child files cannot be compiled on their own,
they can only be compiled via the main file.
A naive editing environment
(such as a text editor with an option
to have the current file processed by \LaTeX)
may require one to switch to the main file before compiling;
attempting to compile the child file produces errors.
\item
The main file must be modified (each time)
to adjust the |\includeonly| command
to the present needs. This easily leaves the main file in a messy state.
\item
The generated document will always carry the filename
of the main document. This is inconvenient if
several child files are to be compiled and
to be kept for distribution.
\end{itemize}

The present package provides a simple interface
to make child files individually compilable by \LaTeX{}.
Compiling a child file then has the same effect as compiling
the main file with an |\includeonly| command
to select the appropriate child.
Moreover the generated document will carry the name of the child
rather than the main file.
This resolves all three above issues.

This feature is meant to make the editing of books,
thesis documents and lecture notes somewhat more convenient.
However, the package can also be used efficiently for
composing a series of documents (such as exercise sheets)
which are typically distributed individually.
It then assists the author in generating the individual documents
(potentially in different versions)
as well as a document containing the collected series.
Another application is in developing style files
or other kinds of included material
where compilation of the style file could redirect
to a sample or test file.

%%%%%%%%%%%%%%%%%%%%%%%%%%%%%%%%%%%%%%%%%%%%%%%%%%%%%%%%%%%%%%%%%%%%%%%%%%%%%%%%
%%%%%%%%%%%%%%%%%%%%%%%%%%%%%%%%%%%%%%%%%%%%%%%%%%%%%%%%%%%%%%%%%%%%%%%%%%%%%%%%
\section{Usage}

First of all, the package \textsf{childdoc} is \emph{not} a standard
\LaTeXe{} |.sty| style file! Therefore it needs to be invoked in
a non-standard way.

%%%%%%%%%%%%%%%%%%%%%%%%%%%%%%%%%%%%%%%%%%%%%%%%%%%%%%%%%%%%%%%%%%%%%%%%%%%%%%%%
\subsection{Included Files}
\label{sec:include}

%%%%%%%%%%%%%%%%%%%%%%%%%%%%%%%%%%%%%%%%
\DescribeMacro{\childdocmain}
To use the package, add the commands
\begin{center}
\begin{tabular}{l}
|\input{childdoc.def}|\\
|\childdocmain{}|\\
\end{tabular}
\end{center}
at the very top of the main \LaTeX{} file,
in particular \emph{before} the |\documentclass| statement!
The argument of |\childdocmain| should be left empty
(but it must be present).

%%%%%%%%%%%%%%%%%%%%%%%%%%%%%%%%%%%%%%%%
\DescribeMacro{\childdocof}
Furthermore, add the commands
\begin{center}
\begin{tabular}{l}
|\input{childdoc.def}|\\
|\childdocof{|\textit{main}|}|\\
\end{tabular}
\end{center}
at the top of every child file \textit{child}
which is included by |\include{|\textit{child}|}|
from within the main file
(or at least for those files to be compiled individually).
The argument \textit{main} must be the filename of the main file.

There are a couple of
considerations in setting up the main and child documents:

%%%%%%%%%%%%%%%%%%%%%%%%%%%%%%%%%%%%%%%%
\paragraph{Restrictions.}

Please note the following restrictions:
\begin{itemize}
\item
|\childdocmain| must be called with one argument \textit{main}
to ensure compatibility with earlier version of the package.
It must either be empty (|\childdocmain{}|)
or precisely match the filename of the main file in which it is specified.
See \secref{sec:detection} for further information.
\item
The filename \textit{main} must be specified without the |.tex| extension.
\item
The filename \textit{main} is case sensitive
(even in case-insensitive file systems)
due to internal string comparison.
\item
The argument \textit{main} should be fully expanded, it cannot be a macro.
\item
Subdirectories and special characters should be avoided in filenames.
\item
The command |\childdocmain{|\textit{main}|}| must be followed by a whitespace.
It should not be followed immediately by another command
or by a comment mark `|%|'.
This is because the \TeX{} parser reads the token immediately following
the argument of |\childdocmain| and puts it
at the beginning of every child section;
however, a white\-space is ignored.
\end{itemize}

%%%%%%%%%%%%%%%%%%%%%%%%%%%%%%%%%%%%%%%%
\paragraph{Content of Main File.}

It is advisable to place all content in the child files included by |\include|.
Any output contained in the main file will appear in all child documents
unless suppressed manually;
it cannot be suppressed automatically by the |\includeonly| directive
and thus should normally be avoided.
A method to include some content in the main file
by means of conditional processing is described in \secref{sec:conditional}.

%%%%%%%%%%%%%%%%%%%%%%%%%%%%%%%%%%%%%%%%
\paragraph{Page Numbering.}

When only a part of the document is compiled,
the appropriate numbering of pages
(as well as other status parameters)
is determined from the |.aux| files.
The latter contain information from previous passes.
However this information needs to propagate through
all intermediate child documents.
Therefore the page numbering in child documents may well
be inconsistent until the complete document is compiled at least once.

A useful (if unconventional) way to always ensure a consistent
page numbering is to restart the numbering in each child document
and denote the pages by `\textit{child}|.|\textit{page}'
where \textit{child} represents the chapter/section number of the child file.
This can be achieved by the command
|\numberwithin{page}{|\textit{child}|}|
of the \textsf{amsmath} package
where \textit{child} can be |chapter| or |section|
depending on the chosen structuring.
Alternatively, one can modify the macro |\thepage| appropriately
and reset the counter |page| at the start of each child file.

%%%%%%%%%%%%%%%%%%%%%%%%%%%%%%%%%%%%%%%%%%%%%%%%%%%%%%%%%%%%%%%%%%%%%%%%%%%%%%%%
\subsection{Conditional Processing}
\label{sec:conditional}

The package provides a mechanism to compile different versions
of a document. To customise the versions further some conditional processing
can come in handy to distinguish which version is being compiled.
The package provides two macros to describe the compilation context:

%%%%%%%%%%%%%%%%%%%%%%%%%%%%%%%%%%%%%%%%
\DescribeMacro{\ifchilddoc}
The conditional |\ifchilddoc| distinguishes between the compilation of
child documents and the main document:
%
\begin{center}
|\ifchilddoc |\textit{child-code}| |[|\||else |\textit{main-code}]| \||fi|
\end{center}

%%%%%%%%%%%%%%%%%%%%%%%%%%%%%%%%%%%%%%%%
\DescribeMacro{\childdocname}
\DescribeMacro{\childdocjob}
The macro |\childdocname| contains the filename (without extension)
of the main or child file being processed.
Note that |\childdocjob| will always contain the name of the main file.

%%%%%%%%%%%%%%%%%%%%%%%%%%%%%%%%%%%%%%%%
\paragraph{Title Page.}

Conditional processing can be used to include a title or banner page
in the main document when proper precautions are taken.
Importantly, the code in the main file should ensure that the page counter
(as well as other status parameters which are stored in the |.aux| files)
takes the same value after the conditional processing.
Otherwise the page numbers may take divergent values
depending on which part is compiled.

For example, a title page could be declared by:
%
\begin{center}
\begin{tabular}{l}
|\ifchilddoc\||else|\\
|\addtocounter{page}{-1}|\\
\textit{code for title page}\\
|\newpage|\\
|\||fi|
\end{tabular}
\end{center}
%
A banner page for the child documents can be generated by:
%
\begin{center}
\begin{tabular}{l}
|\ifchilddoc|\\
|\addtocounter{page}{-1}|\\
\textit{code for banner page}\\
|\newpage|\\
|\||fi|
\end{tabular}
\end{center}
%
Here one could write a message such as:
\begin{center}
|This is the part \childdocname{} of \childdocjob{}.|
\end{center}

%%%%%%%%%%%%%%%%%%%%%%%%%%%%%%%%%%%%%%%%%%%%%%%%%%%%%%%%%%%%%%%%%%%%%%%%%%%%%%%%
\subsection{Flags}
\label{sec:flags}

The package makes it easy to generate different versions
of the main or child documents.
To this end compilation flags can be defined
and assigned different default values.
They will be particularly useful in conjunction
with the forwarding mechanism described in \secref{sec:forward}.

For example, it may be useful to have a flag |\version|
which can be set to |draft| or |final|.
The document source will contain some conditional code
depending on the value of |\version|.
Suppose further, the flag should default to |final| for the main file
and to |draft| for child files
which is a natural assignment for editing the document.
This is achieved by placing the following code
in the preamble of the main document
(below the |\childdocmain| directive):
%
\begin{center}
\begin{tabular}{l}
|\ifchilddoc|\\
|\providecommand{\version}{draft}|\\
|\||else|\\
|\providecommand{\version}{final}|\\
|\||fi|
\end{tabular}
\end{center}
%
The definition by |\providecommand| makes sure
that previous definitions are not overwritten.
Further statements |\providecommand{\version}{...}|
can thus be added before the above code to override it.

For the main file, one might add a line
(between |\childdocmain| and the above block)
%
\begin{center}
|%\ifchilddoc\||else\providecommand{\version}{draft}\||fi|
\end{center}
%
which can be uncommented to produce a draft version.
Likewise one can add a line to the very top of a child file
(above the |\childdocof{|\textit{main}|}| directive)
%
\begin{center}
|%\providecommand{\version}{final}|
\end{center}
%
which can be uncommented to produce the final version of this child document.

%%%%%%%%%%%%%%%%%%%%%%%%%%%%%%%%%%%%%%%%%%%%%%%%%%%%%%%%%%%%%%%%%%%%%%%%%%%%%%%%
\subsection{Forwarding}
\label{sec:forward}

Different versions of the main or child documents
using compilation flags as described in \secref{sec:flags}
can be (permanently) stored in different files
for convenient compilation, viewing and distribution.
To this end, the package defines a command
to pass on compilation to a different file:

%%%%%%%%%%%%%%%%%%%%%%%%%%%%%%%%%%%%%%%%
\DescribeMacro{\childdocforward}
The command |\childdocforward| redirects processing to
another source file:
%
\begin{center}
\begin{tabular}{l}
|\input{childdoc.def}|\\
|\childdocforward[|\textit{main}|]{|\textit{dest}|}|\\
\end{tabular}
\end{center}
%
The argument \textit{dest} is the destination file
(without extension).
It should be the main file or one of the child files.
Note that further \textsf{childdoc} directives
such as |\childdocof| and |\childdocforward|
in the indicated file will be processed in this form.
The optional argument \textit{main}
passes on directly to the main file \textit{main}
while pretending to compile the child \textit{dest}.
This form behaves as if \textit{dest}
issues |\childdocof{|\textit{main}|}| right away,
and no further \textsf{childdoc} directives will be processed.

%%%%%%%%%%%%%%%%%%%%%%%%%%%%%%%%%%%%%%%%
\DescribeMacro{\...prefix}
In the alternative form |\childdocforwardprefix|,
%
\begin{center}
\begin{tabular}{l}
|\input{childdoc.def}|\\
|\childdocforwardprefix[|\textit{main}|]{|\textit{prefix}|}{|\textit{dest}|}|
\end{tabular}
\end{center}
%
the destination file is determined by a pattern
depending on the current file:
To make this work, the current file must be called
`{\textit{prefix}\hspace{0.2em}\textit{suffix}}'
with \textit{prefix} matching precisely the argument.
Processing is then passed on to the file
`{\textit{dest}\hspace{0.2em}\textit{suffix}}'.
Surely, the same effect is achieved by
directly specifying the
argument `{\textit{dest}\hspace{0.2em}\textit{suffix}}'
in the first form.
However, that requires to set up a different file
for each child. With the alternative form of the command
all these files can have exactly the same content
which simplifies setting them up and maintaining them.

For example, the following file |draft.tex|
with a compilation flag |\version| as described in \secref{sec:flags}
compiles the main document as a draft:
%
\begin{center}
\begin{tabular}{l}
|\def\version{draft}|\\
|\input{childdoc.def}|\\
|\childdocforward{|\textit{main}|}|
\end{tabular}
\end{center}
%
Likewise, the following files |final|\textit{nn}|.tex|
compile the final version of the child document
|child|\textit{nn}|.tex|:
%
\begin{center}
\begin{tabular}{l}
|\def\version{final}|\\
|\input{childdoc.def}|\\
|\childdocforwardprefix{final}{child}|
\end{tabular}
\end{center}
%

Note that when several versions of a main file and/or of each child file
are to be generated, it may be convenient to set up a |Makefile| or
shell script to automatise the process.

%%%%%%%%%%%%%%%%%%%%%%%%%%%%%%%%%%%%%%%%%%%%%%%%%%%%%%%%%%%%%%%%%%%%%%%%%%%%%%%%
\subsection{Command Line Processing}
\label{sec:commandline}

The effect of redirection files can also be achieved by invoking
the \LaTeX{} compiler with a more elaborate command line.
Most conveniently this should be done as part
of a shell script or a |Makefile|.

When using \textsf{childdoc} in the main file, the following
command lines effectively perform a redirection
(note that depending on the shell being used,
backslashes may have to be doubled: `|\|' $\to$ `|\\|'):
%
\begin{center}
|... -jobname "|\textit{target}|" |\\|"|[\textit{flags}]%
|\input{childdoc.def}\childdocforward[|\textit{main}|]{|\textit{dest}|}"|
\end{center}
%
Here \textit{target} is the name of the output file,
\textit{main} is the name of the main file
and \textit{dest} is the name of the main or child file to be processed
(all filenames without extensions).
The optional argument \textit{main} can be omitted
if \textit{main} matches \textit{dest}.
Optionally, compilation \textit{flags} can be defined via |\def| commands.
This command line makes the \TeX{} engine believe
it is compiling the file \textit{target}
whose content is specified as the latter parameter.
The provided code then forwards the processing to
\textit{main} or \textit{dest} as described in \secref{sec:forward}.

%%%%%%%%%%%%%%%%%%%%%%%%%%%%%%%%%%%%%%%%%%%%%%%%%%%%%%%%%%%%%%%%%%%%%%%%%%%%%%%%
\subsection{Include by Input}
\label{sec:input}

Including child documents by |\include| has some restrictions by design.
Most notably, the content of a child document always occupies
its own set of pages; pages cannot be shared between child documents.
Usually, this behaviour makes perfect sense
because each child document contain an essential part of the document.
However, in some situations it may be desirable to compose
a document from a collection of parts
without having mandatory page breaks between then.
For this case, the package
provides a mechanism to include parts
by |\input| which can also be processed individually.
However, by construction this mechanism
requires manual handling of the content to be output.

%%%%%%%%%%%%%%%%%%%%%%%%%%%%%%%%%%%%%%%%
\DescribeMacro{\ifchilddocmanual}
The main file should be prepared as usual, see \secref{sec:include}.
However, the document body must make a distinction
between processing of an individual part and of the main document, e.g.:
%
\begin{center}
\begin{tabular}{l}
|\ifchilddocmanual|\\
|\input{\childdocname}|\\
|\||else|\\
\textit{document body with }|\input{|\textit{part}|}|\\
|\||fi|
\end{tabular}
\end{center}
%
The conditional |\ifchilddocmanual| is true whenever
a part to be included by |\input| is being compiled,
and the name of the part is stored in |\childdocname|.

%%%%%%%%%%%%%%%%%%%%%%%%%%%%%%%%%%%%%%%%
\DescribeMacro{\childdocby}
Each part to be included by |\input| should start with:
%
\begin{center}
\begin{tabular}{l}
|\input{childdoc.def}|\\
|\childdocby{|\textit{main}|}|\\
\end{tabular}
\end{center}
%
The directive |\childdocby| is similar to |\childdocof|
described in \secref{sec:include},
but the subsequent selection of content must be done manually.
To that end, both |\ifchilddoc| and |\ifchilddocmanual|
will be true upon processing of a part,
and the name of the part is stored in |\childdocname|.
Note that |\jobname| will be set to the filename of the current part
so that each part receives an individual |.aux| file
that does not interfere with the |.aux| file(s) of the main document.
This behaviour can be altered by the alternative form
|\childdocby[*]{|\textit{main}|}| (with a non-empty optional argument)
which uses the |.aux| file of the main document
by setting |\jobname| to \textit{main}.

%%%%%%%%%%%%%%%%%%%%%%%%%%%%%%%%%%%%%%%%%%%%%%%%%%%%%%%%%%%%%%%%%%%%%%%%%%%%%%%%
\subsection{Driver Development}
\label{sec:driver}

The \textsf{childdoc} mechanism can also be use for the development
of definition files such as \LaTeX{} styles or classes.
This case differs from the above setup with multiple parts
included by |\include| in that no |\includeonly| should be invoked.
This can be achieved by starting the include file
(before |\ProvidesPackage|) with:
%
\begin{center}
\begin{tabular}{l}
|\input{childdoc.def}|\\
|\childdocforward{|\textit{main}|}|\\
\end{tabular}
\end{center}
%
or alternatively with:
%
\begin{center}
\begin{tabular}{l}
|\input{childdoc.def}|\\
|\childdocby{|\textit{main}|}|\\
\end{tabular}
\end{center}
%
Both forms have slightly different effects as described above.
The main file is prepared as usual, see \secref{sec:include}.

%%%%%%%%%%%%%%%%%%%%%%%%%%%%%%%%%%%%%%%%%%%%%%%%%%%%%%%%%%%%%%%%%%%%%%%%%%%%%%%%
\subsection{Legacy Detection}
\label{sec:detection}

The directive |\childdocmain| in the main file can detect
whether the complete document or merely a child is to be compiled
even without using the directive |\childdocof|.
This method is deprecated because it is less robust
and there is no compelling reason to use it;
it is merely provided for backward compatibility
and it may be removed in future versions.

If the detection mechanism is to be used,
it is mandatory to correctly specify
the filename of the main file as the argument of |\childdocmain|:
%
\begin{center}
\begin{tabular}{l}
|\input{childdoc.def}|\\
|\childdocmain{|\textit{main}|}|\\
\end{tabular}
\end{center}
%
If |\jobname| does not match the argument \textit{main} of |\childdocmain|,
it is assumed that |\jobname| points to the child file to be compiled.
When using |\childdocmain| with the main file specified as argument,
it suffices to start a child file
with just |\input{|\textit{main}|}|
without loading of the package and using |\childdocof|.
If instead all processing is done
with the appropriate \textsf{childdoc} directives,
the argument of \textit{main} of |\childdocmain| can be empty.

An alternative version of the command line processing described
in \secref{sec:commandline} using the detection mechanism reads:
%
\begin{center}
|... -jobname "|\textit{target}|" "|[\textit{flags}]%
[|\def\jobname{|\textit{dest}|}|]|\input{|\textit{main}|}"|
\end{center}

%%%%%%%%%%%%%%%%%%%%%%%%%%%%%%%%%%%%%%%%%%%%%%%%%%%%%%%%%%%%%%%%%%%%%%%%%%%%%%%%
\subsection{Manual Code}
\label{sec:manual}

In case one cannot be certain whether the definitions file |childdoc.def|
is installed on the target \TeX{} distribution
and one prefers not to ship it,
it is conceivable to paste a few relevant commands into the sources.

To that end, drop all statements |\input{childdoc.def}|
and perform the replacements as outlined below.
Instead of |\childdocmain{|\textit{main}|}| add the following code
to the top of the main file:
%
\begin{center}
\begin{tabular}{l}
|\||ifdefined\childdocname\endinput\||fi\newif\ifchilddoc|\\
|\edef\childdocname{\scantokens\expandafter{\jobname\noexpand}}|\\
|\def\childdocmain{|\textit{main}|}\||ifx\childdocmain\childdocname\||else|\\
|\childdoctrue\includeonly{\childdocname}\let\jobname\childdocmain\||fi|\\
\end{tabular}
\end{center}
%
Instead of |\childdocof{|\textit{main}|}| just include the main file
at the top of each child file:
%
\begin{center}
|\input{|\textit{main}|}|
\end{center}
%
A simple redirection |\childdocforward{|\textit{dest}|}| is achieved by:
%
\begin{center}
|\def\jobname{|\textit{dest}|}\input{\jobname}|
\end{center}
%
The redirection with prefix
|\childdocforwardprefix[|\textit{prefix}|]{|\textit{dest}|}|
is accomplished by:
%
\begin{center}
\begin{tabular}{l}
|{\edef\jobname{\scantokens\expandafter{\jobname\noexpand}}|\\
|\def\redirectjob |\textit{prefix}|#1~~~{\gdef\jobname{|\textit{dest}|#1}}|\\
|\expandafter\redirectjob\jobname~~~}\input{\jobname}|
\end{tabular}
\end{center}

In an alternative approach,
child documents can be compiled by a specific command line
without additional code or specific definitions:
%
\begin{center}
|... -jobname "|\textit{target}|" "|[\textit{flags}]%
|\includeonly{|\textit{dest}|}\input{|\textit{main}|}"|
\end{center}
%

%%%%%%%%%%%%%%%%%%%%%%%%%%%%%%%%%%%%%%%%%%%%%%%%%%%%%%%%%%%%%%%%%%%%%%%%%%%%%%%%
%%%%%%%%%%%%%%%%%%%%%%%%%%%%%%%%%%%%%%%%%%%%%%%%%%%%%%%%%%%%%%%%%%%%%%%%%%%%%%%%
\section{Information}

%%%%%%%%%%%%%%%%%%%%%%%%%%%%%%%%%%%%%%%%%%%%%%%%%%%%%%%%%%%%%%%%%%%%%%%%%%%%%%%%
\subsection{Copyright}

Copyright \copyright{} 2017--2018 Niklas Beisert

This work may be distributed and/or modified under the
conditions of the \LaTeX{} Project Public License, either version 1.3
of this license or (at your option) any later version.
The latest version of this license is in
  \url{http://www.latex-project.org/lppl.txt}
and version 1.3 or later is part of all distributions of \LaTeX{}
version 2005/12/01 or later.

This work has the LPPL maintenance status `maintained'.

The Current Maintainer of this work is Niklas Beisert.

This work consists of the files |README.txt|, |childdoc.ins| and |childdoc.dtx|
as well as the derived files |childdoc.def|, |cdocsamp.tex|
with |cdocsch1.tex|, |cdocsch2.tex|, |cdocspt3.tex|, |cdocspt4.tex|,
|cdocsdrf.tex|, |cdocsfn1.tex|, |cdocsfn2.tex|
as well as |childdoc.pdf|.

%%%%%%%%%%%%%%%%%%%%%%%%%%%%%%%%%%%%%%%%%%%%%%%%%%%%%%%%%%%%%%%%%%%%%%%%%%%%%%%%
\subsection{Files and Installation}

The package consists of the files:
%
\begin{center}
\begin{tabular}{ll}
    |README.txt|   & readme file \\
    |childdoc.ins| & installation file \\
    |childdoc.dtx| & source file \\
    |childdoc.def| & definition file \\
    |cdocsamp.tex| & sample main file \\
    |cdocsch1.tex| & sample include file \\
    |cdocsch2.tex| & sample include file \\
    |cdocspt3.tex| & sample part file \\
    |cdocspt4.tex| & sample part file \\
    |cdocsdrf.tex| & sample redirection file \\
    |cdocsfn1.tex| & sample redirection file \\
    |cdocsfn2.tex| & sample redirection file \\
    |childdoc.pdf| & manual
\end{tabular}
\end{center}
%
The distribution consists of the files
|README.txt|, |childdoc.ins| and |childdoc.dtx|.
%
\begin{itemize}
\item
Run (pdf)\LaTeX{} on |childdoc.dtx|
to compile the manual |childdoc.pdf| (this file).
\item
Run \LaTeX{} on |childdoc.ins| to create the definitions file |childdoc.def|
and the sample |cdocsamp.tex| with include files
|cdocsch1.tex|, |cdocsch2.tex|, |cdocspt3.tex|, |cdocspt4.tex|,
|cdocsdrf.tex|, |cdocsfn1.tex|, |cdocsfn2.tex|.
Then copy the file |childdoc.def| to an appropriate directory of your \LaTeX{}
distribution, e.g.\ \textit{texmf-root}|/tex/latex/childdoc|.
\end{itemize}

%%%%%%%%%%%%%%%%%%%%%%%%%%%%%%%%%%%%%%%%%%%%%%%%%%%%%%%%%%%%%%%%%%%%%%%%%%%%%%%%
\subsection{Related CTAN Packages}

There are several other packages which offer a similar functionality:
%
\begin{itemize}
\item
The packages
\href{http://ctan.org/pkg/docmute}{\textsf{docmute}},
\href{http://ctan.org/pkg/includex}{\textsf{includex}} and
\href{http://ctan.org/pkg/standalone}{\textsf{standalone}}
provide commands to include only the document body of
a child file thus allowing both files to be compiled individually.
\item
The packages \href{http://ctan.org/pkg/subdocs}{\textsf{subdocs}}
and \href{http://ctan.org/pkg/subfiles}{\textsf{subfiles}}
provide structures in which the main and child documents can be
encapsulated and allowing them to be compiled individually.
The inclusion mechanism is different from the conventional |\include|.
\item
The package \href{http://ctan.org/pkg/combine}{\textsf{combine}}
is an elaborate solution to combine several documents into one.
\end{itemize}
%
See also the CTAN topic \href{http://ctan.org/topic/subdocs}{\textsf{subdocs}}
for further related packages.
The present package differs from the above solutions in that
a document structure constructed with the conventional |\include| mechanism
just needs two extra commands at the top of every file
such that all constituent files can be compiled individually.

%%%%%%%%%%%%%%%%%%%%%%%%%%%%%%%%%%%%%%%%%%%%%%%%%%%%%%%%%%%%%%%%%%%%%%%%%%%%%%%%
%\subsection{Feature Suggestions}
%
%The following is a list of features which may be useful for future
%versions of this package:
%%
%\begin{itemize}
%\item
%\ldots
%\end{itemize}

%%%%%%%%%%%%%%%%%%%%%%%%%%%%%%%%%%%%%%%%%%%%%%%%%%%%%%%%%%%%%%%%%%%%%%%%%%%%%%%%
\subsection{Revision History}

%%%%%%%%%%%%%%%%%%%%%%%%%%%%%%%%%%%%%%%%
\paragraph{v2.0:} 2018/12/30

\begin{itemize}
\item
immediate forward processing
\item
added |\childdocby| mechanism
\item
manual restructured
\end{itemize}

%%%%%%%%%%%%%%%%%%%%%%%%%%%%%%%%%%%%%%%%
\paragraph{v1.6:} 2018/01/17

\begin{itemize}
\item
application for development of include files
\item
corrections to manual
\end{itemize}

%%%%%%%%%%%%%%%%%%%%%%%%%%%%%%%%%%%%%%%%
\paragraph{v1.5:} 2017/05/21

\begin{itemize}
\item
more complete structuring introduced
\item
|\childdocof| introduced
\item
|\childdoc| renamed to |\childdocmain|
\item
|\childredirect| renamed to |\childdocforward| and |\childdocforwardprefix|
and functionality expanded
\end{itemize}

%%%%%%%%%%%%%%%%%%%%%%%%%%%%%%%%%%%%%%%%
\paragraph{v1.0:} 2017/04/27

\begin{itemize}
\item
manual and install package
\item
first version published on CTAN
\end{itemize}

%%%%%%%%%%%%%%%%%%%%%%%%%%%%%%%%%%%%%%%%
\paragraph{v0.6:} 2017/04/26

\begin{itemize}
\item
redirection mechanism added
\end{itemize}

%%%%%%%%%%%%%%%%%%%%%%%%%%%%%%%%%%%%%%%%
\paragraph{v0.5:} 2017/04/26

\begin{itemize}
\item
functionality in definition file
\end{itemize}


%%%%%%%%%%%%%%%%%%%%%%%%%%%%%%%%%%%%%%%%%%%%%%%%%%%%%%%%%%%%%%%%%%%%%%%%%%%%%%%%
%%%%%%%%%%%%%%%%%%%%%%%%%%%%%%%%%%%%%%%%%%%%%%%%%%%%%%%%%%%%%%%%%%%%%%%%%%%%%%%%
%%%%%%%%%%%%%%%%%%%%%%%%%%%%%%%%%%%%%%%%%%%%%%%%%%%%%%%%%%%%%%%%%%%%%%%%%%%%%%%%
\appendix

\settowidth\MacroIndent{\rmfamily\scriptsize 000\ }

 \DocInput{childdoc.dtx}

\end{document}
%</driver>
% \fi
%
% %%%%%%%%%%%%%%%%%%%%%%%%%%%%%%%%%%%%%%%%%%%%%%%%%%%%%%%%%%%%%%%%%%%%%%%%%%%%%%
% %%%%%%%%%%%%%%%%%%%%%%%%%%%%%%%%%%%%%%%%%%%%%%%%%%%%%%%%%%%%%%%%%%%%%%%%%%%%%%
% \section{Sample}
%\iffalse
%<*samplemain>
%\fi
%
% The following presents a sample document
% with two chapters, two parts, a title page,
% a compile flag as well as three forwarding files to set the flag.
% It consists of eight |.tex| files:
% \begin{center}
% \begin{tabular}{ll}
% |cdocsamp.tex|&main file\\
% |cdocsch1.tex|&include file for chapter 1\\
% |cdocsch2.tex|&include file for chapter 2\\
% |cdocspt3.tex|&include file for part 3\\
% |cdocspt4.tex|&include file for part 4\\
% |cdocsdrf.tex|&forwarding file for main file in draft mode\\
% |cdocsfi1.tex|&forwarding file for final version of chapter 1\\
% |cdocsfi2.tex|&forwarding file for final version of chapter 2\\
% \end{tabular}
% \end{center}
% Each of the eight files can be compiled directly by the \LaTeX{} compiler.
%
% %%%%%%%%%%%%%%%%%%%%%%%%%%%%%%%%%%%%%%
% \paragraph{Main File.}
%
% The main file is called |cdocsamp.tex|.
%
% Load the \textsf{childdoc} definitions and
% declare the filename for the main document:
%    \begin{macrocode}
\input{childdoc.def}
\childdocmain{}
%    \end{macrocode}

% Optional override for |\version| flag:
%    \begin{macrocode}
%%\ifchilddoc\else\providecommand{\version}{draft}\fi
%    \end{macrocode}

% Define the default values for the |\version| flag
% (|final| for the main file and |draft| for childs):
%    \begin{macrocode}
\ifchilddoc
\providecommand{\version}{draft}
\else
\providecommand{\version}{final}
\fi
%    \end{macrocode}

% Load the standard document class:
%    \begin{macrocode}
\documentclass[12pt]{article}
%    \end{macrocode}

% Start the document body:
%    \begin{macrocode}
\begin{document}
%    \end{macrocode}

% Declare a title page.
% Print title, part of document being processed and version flag:
%    \begin{macrocode}
\addtocounter{page}{-1}
\begin{center}
{\LARGE\bfseries{}childdoc example\par}
\vspace{1cm}
\ifchilddoc
\ifchilddocmanual part\else chapter\fi:
`\childdocname' of `\childdocjob'\par
\else
main document: `\childdocjob'\par
\fi
version: \version\par
\end{center}
\newpage
%    \end{macrocode}

% Manually include selected file,
% otherwise process as usual:
%    \begin{macrocode}
\ifchilddocmanual
\section*{part `\childdocname'}
\input{\childdocname}
\else
%    \end{macrocode}

% Include the two chapters:
%    \begin{macrocode}
\include{cdocsch1}
\include{cdocsch2}
%    \end{macrocode}

% Include the two parts unless only chapters should be displayed:
%    \begin{macrocode}
\ifchilddoc\else
\section{part three}
\input{cdocspt3}
\section{part four}
\input{cdocspt4}
\fi
%    \end{macrocode}

% Process as usual until here:
%    \begin{macrocode}
\fi
%    \end{macrocode}

% End of document body:
%    \begin{macrocode}
\end{document}
%    \end{macrocode}
%\iffalse
%</samplemain>
%\fi
%
% %%%%%%%%%%%%%%%%%%%%%%%%%%%%%%%%%%%%%%
% \paragraph{Chapter Include Files.}
%
% The include files are called |cdocsch1.tex| and |cdocsch2.tex|.
%
%\iffalse
%<*samplechap1|samplechap2>
%\fi

% Optional override for |\version| flag:
%    \begin{macrocode}
%%\providecommand{\version}{final}
%    \end{macrocode}

% Include the main document:
%    \begin{macrocode}
\input{childdoc.def}
\childdocof{cdocsamp}
%    \end{macrocode}

%\iffalse
%</samplechap1|samplechap2>
%\fi
%
%\iffalse
%<*samplechap1>
%\fi
% Some text for chapter 1:
%    \begin{macrocode}
\section{one}
some text in chapter one
%    \end{macrocode}

%\iffalse
%</samplechap1>
%\fi
% Some text for chapter 2:
%\iffalse
%<*samplechap2>
%\fi
%    \begin{macrocode}
\section{two}
more text in chapter two
%    \end{macrocode}

%\iffalse
%</samplechap2>
%\fi
%
% %%%%%%%%%%%%%%%%%%%%%%%%%%%%%%%%%%%%%%
% \paragraph{Part Include Files.}
%
% The include files are called |cdocspt3.tex| and |cdocspt4.tex|.
%
%\iffalse
%<*samplepart3|samplepart4>
%\fi

% Optional override for |\version| flag:
%    \begin{macrocode}
%%\providecommand{\version}{final}
%    \end{macrocode}

% Include the main document:
%    \begin{macrocode}
\input{childdoc.def}
\childdocby{cdocsamp}
%    \end{macrocode}

%\iffalse
%</samplepart3|samplepart4>
%\fi
%
%\iffalse
%<*samplepart3>
%\fi
% Some text for part 3:
%    \begin{macrocode}
some text in part three
%    \end{macrocode}

%\iffalse
%</samplepart3>
%\fi
% Some text for part 4:
%\iffalse
%<*samplepart4>
%\fi
%    \begin{macrocode}
more text in part four
%    \end{macrocode}

%\iffalse
%</samplepart4>
%\fi
%
% %%%%%%%%%%%%%%%%%%%%%%%%%%%%%%%%%%%%%%
% \paragraph{Forwarding for a Complete Draft.}
%
% The following forwarding file |cdocsdrf.tex|
% compiles the main document in draft mode:
%\iffalse
%<*sampledraft>
%\fi
%    \begin{macrocode}
\def\version{draft}
\input{childdoc.def}
\childdocforward{cdocsamp}
%    \end{macrocode}

%\iffalse
%</sampledraft>
%\fi
%
% %%%%%%%%%%%%%%%%%%%%%%%%%%%%%%%%%%%%%%
% \paragraph{Forwarding for Final Version of the Chapters.}
%
% The following forwarding files |cdocsfn1.tex| and |cdocsfn2.tex|
% (with identical content)
% compile the final versions of the child documents
% |cdocsch1.tex| and |cdocsch2.tex|, respectively:
%\iffalse
%<*samplefinal>
%\fi
%    \begin{macrocode}
\def\version{final}
\input{childdoc.def}
\childdocforwardprefix[cdocsamp]{cdocsfn}{cdocsch}
%    \end{macrocode}

%\iffalse
%</samplefinal>
%\fi
%
% %%%%%%%%%%%%%%%%%%%%%%%%%%%%%%%%%%%%%%
% \paragraph{Command Line Processing.}
%
% The following three command lines generate the output files
% |cdocscld|, |cdocscl1| and |cdocscl2|
% which should be identical to
% |cdocsdrf|, |cdocsch1| and |cdocsfn2|, respectively:
% \begin{center}
% \begin{tabular}{l}
% |latex -jobname cdocscld \|\\
% |  "\def\version{draft}\input{childdoc.def}\childdocforward{cdocsamp}"|\\
% |latex -jobname cdocscl1 \|\\
% |  "\input{childdoc.def}\childdocforward[cdocsamp]{cdocsch1}"|\\
% |latex -jobname cdocscl2 \|\\
% |  "\def\version{final}\input{childdoc.def}\childdocforward{cdocsch2}"|
% \end{tabular}
% \end{center}
% Note that the trailing backslash on each first line
% merely continues the input to the second line
% (for convenient cut ant paste).
% Furthermore, the command |latex| can be replaced by any
% of its alternative versions such as |pdflatex|.
%
% %%%%%%%%%%%%%%%%%%%%%%%%%%%%%%%%%%%%%%%%%%%%%%%%%%%%%%%%%%%%%%%%%%%%%%%%%%%%%%
% %%%%%%%%%%%%%%%%%%%%%%%%%%%%%%%%%%%%%%%%%%%%%%%%%%%%%%%%%%%%%%%%%%%%%%%%%%%%%%
% \section{Implementation}
%\iffalse
%<*package>
%\fi
%
% This section describes the definitions file |childdoc.def|.

% The definitions cannot be loaded using |\usepackage| or |\RequirePackage|
% which has a mechanism to prevent loading a style file more than once.
% When loading the definitions by means of |\input|
% multiple instances have to be prevented manually:
%\iffalse
%This code needs to be before the `\ProvidesFile' directive
%which is defined at the beginning of this file.
%Therefore it is also placed there and commented out here.
%</package>
%<*discard>
%\fi
%    \begin{macrocode}
\ifdefined\childdocmain\endinput\fi
%    \end{macrocode}
%\iffalse
%</discard>
%<*package>
%\fi
%
% \macro{\ifchilddoc}
% \macro{\ifchilddocmanual}
% The conditional |\ifchilddoc| tells whether a
% child (true) or main (false) document is being compiled.
% The conditional |\ifchilddocmanual| tells whether
% the |\includeonly| mechanism is used (false) or
% the selection of child files must be performed manually (true).
% The definitions initialise to false:
%    \begin{macrocode}
\newif\ifchilddoc
\newif\ifchilddocmanual
%    \end{macrocode}

% \macro{\childdocname}
% \macro{\childdocjob}
% The macro |\childdocname| stores the name of the main document
% to be compiled. The macro |\childdocjob| stores the name of
% the document on which the \LaTeX{} compiler was originally invoked.
% The content of |\jobname| cannot be compared
% to filenames specified in the source due to different catcodes.
% The following code rescans |\jobname|, stores the result
% in |\childdocname| and saves a copy in |\childdocjob|:
%    \begin{macrocode}
\edef\childdocname{\scantokens\expandafter{\jobname\noexpand}}
\let\childdocjob\childdocname
%    \end{macrocode}

% \macro{\childdocdisable}
% The macro |\childdocdisable| prevents the main file
% from being processed more than once.
% At this stage, the main document command |\childdocmain|
% is assumed to be called once again where it should do nothing.
% Any subsequent call to it should prevent
% a secondary processing of the main document
% It overwrites the forwarding commands
% |\childdocof| and |\childdocforward|
% with empty macros to prevent further inclusions of the main document:
%    \begin{macrocode}
\newcommand{\childdocdisable}
{
  \renewcommand{\childdocmain}[1]{\renewcommand{\childdocmain}[1]{\endinput}}
  \renewcommand{\childdocof}[1]{}
  \renewcommand{\childdocby}[2][]{}
  \renewcommand{\childdocforward}[2][]{}
  \renewcommand{\childdocdisable}{}
}
%    \end{macrocode}

% \macro{\childdocmain}
% The macro |\childdocmain| is to be called at the top of the main file
% with nothing or the main filename (without extension) as argument.
% First, it breaks loops.
% If the argument is not empty and does not match |\childdocname|
% (which is set by the first inclusion of |childdoc.def|),
% |\ifchilddoc| is set to true, |\includeonly| is applied to the child file
% and |\jobname| is set to the main file
% (for proper handling of |.aux| files):
%    \begin{macrocode}
\newcommand{\childdocmain}[1]
{
  \childdocdisable\childdocmain{}
  \if?#1?\else
    \begingroup
      \def\childdoctmp{#1}
      \ifx\childdoctmp\childdocname
        \def\childdoctmp{}
      \else
        \def\childdoctmp
        {
          \childdoctrue
          \includeonly{\childdocname}
          \def\childdocjob{#1}
          \def\jobname{#1}
        }
      \fi
      \expandafter
    \endgroup
    \childdoctmp
  \fi
}
%    \end{macrocode}

% \macro{\childdocof}
% The command |\childdocof| redirects
% compilation to the main file |#1|.
%    \begin{macrocode}
\newcommand{\childdocof}[1]
{
  \childdocdisable
  \childdoctrue
  \includeonly{\childdocname}
  \def\jobname{#1}
  \def\childdocjob{#1}
  \input{#1}
}
%    \end{macrocode}

% \macro{\childdocby}
% The command |\childdocby| ....
%    \begin{macrocode}
\newcommand{\childdocby}[2][]
{
  \childdocdisable
  \childdoctrue
  \childdocmanualtrue
  \if?#1?\else
    \def\jobname{#2}
  \fi
  \def\childdocjob{#2}
  \input{#2}
  \endinput
}
%    \end{macrocode}

% \macro{\childdocforward}
% The command |\childdocforward| redirects
% compilation to the main file or
% (if the optional argument is given) a child file.
% Parameters are set as if the main file
% or a child file starting with |\childdocof| was compiled.
% Then compilation is handed over to the main file:
%    \begin{macrocode}
\newcommand{\childdocforward}[2][]
{
  \begingroup
    \if?#1?
      \def\childdoctmp
      {
        \def\childdocname{#2}
        \def\childdocjob{#2}
        \def\jobname{#2}
        \input{#2}
        \endinput
      }
    \else
      \def\childdoctmp
      {
        \childdocdisable
        \def\childdocname{#2}
        \childdoctrue
        \includeonly{#2}
        \def\childdocjob{#1}
        \def\jobname{#1}
        \input{#1}
        \endinput
      }
    \fi
    \expandafter
  \endgroup
  \childdoctmp
}
%    \end{macrocode}

% \macro{\childdocforwardprefix}
% The command |\childdocforwardprefix| redirects
% compilation to the main or a child file by means of a pattern.
% The prefix |#1| in the current filename is replaced by |#2|
% and the suffix of the current filename is kept
% (it is assumed that the filename does not contain the substring `|~~~|'
% which is used as a delimiter).
% Compilation is handed over to the new file by |\childdocforward|:
%    \begin{macrocode}
\newcommand{\childdocforwardprefix}[3][]
{
  \begingroup
    \def\childdocextract #2##1~~~{\def\childdoctmp{\childdocforward[#1]{#3##1}}}
    \expandafter\childdocextract\childdocname~~~
    \expandafter
  \endgroup
  \childdoctmp
}
%    \end{macrocode}

% \macro{\childdoc}
% The deprecated macro |\childdoc| is a legacy version of |\childdocmain|:
%    \begin{macrocode}
\newcommand{\childdoc}{\childdocmain}
%    \end{macrocode}

% \macro{\childdocredirect}
% The deprecated macro |\childdocredirect| is a legacy version
% of |\childdocforward| and |\childdocforwardprefix|:
%    \begin{macrocode}
\newcommand{\childdocredirect}[2][]
{
  \begingroup
    \if?#1?
      \def\childdoctmp{\childdocforward{#2}}
    \else
      \def\childdoctmp{\childdocforwardprefix{#1}{#2}}
    \fi
    \expandafter
  \endgroup
  \childdoctmp
}
%    \end{macrocode}

%\iffalse
%</package>
%\fi
%
\endinput
|\\
|\childdocmain{}|\\
\end{tabular}
\end{center}
at the very top of the main \LaTeX{} file,
in particular \emph{before} the |\documentclass| statement!
The argument of |\childdocmain| should be left empty
(but it must be present).

%%%%%%%%%%%%%%%%%%%%%%%%%%%%%%%%%%%%%%%%
\DescribeMacro{\childdocof}
Furthermore, add the commands
\begin{center}
\begin{tabular}{l}
|% \iffalse
%
% childdoc.dtx Copyright (C) 2017-2018 Niklas Beisert
%
% This work may be distributed and/or modified under the
% conditions of the LaTeX Project Public License, either version 1.3
% of this license or (at your option) any later version.
% The latest version of this license is in
%   http://www.latex-project.org/lppl.txt
% and version 1.3 or later is part of all distributions of LaTeX
% version 2005/12/01 or later.
%
% This work has the LPPL maintenance status `maintained'.
%
% The Current Maintainer of this work is Niklas Beisert.
%
% This work consists of the files childdoc.dtx and childdoc.ins
% and the derived files childdoc.def and cdocsamp.tex with
% cdocsch1.tex, cdocsch2.tex, cdocsdrf.tex, cdocsfn1.tex, cdocsfn2.tex.
%
%<package>\ifdefined\childdocmain\endinput\fi
%<package>\ProvidesFile{childdoc.def}[2018/12/30 v2.0 child document driver]
%<samplemain>\ProvidesFile{cdocsamp.tex}[2018/12/30 v2.0 sample for childdoc]
%<*driver>
%\ProvidesFile{childdoc.drv}[2018/12/30 v2.0 childdoc reference manual file]
\PassOptionsToClass{10pt,a4paper}{article}
\documentclass{ltxdoc}

\usepackage[margin=35mm]{geometry}
\usepackage{hyperref}
\usepackage{hyperxmp}
\usepackage[usenames]{color}

\hypersetup{colorlinks=true}
\hypersetup{pdfstartview=FitH}
\hypersetup{pdfpagemode=UseNone}
\hypersetup{pdfsource={}}
\hypersetup{pdflang={en-UK}}
\hypersetup{pdfcopyright={Copyright 2017-2018 Niklas Beisert.
  This work may be distributed and/or modified under the
  conditions of the LaTeX Project Public License, either version 1.3
  of this license or (at your option) any later version.}}
\hypersetup{pdflicenseurl={http://www.latex-project.org/lppl.txt}}
\hypersetup{pdfcontactaddress={ETH Zurich, ITP, HIT K,
  Wolfgang-Pauli-Strasse 27}}
\hypersetup{pdfcontactpostcode={8093}}
\hypersetup{pdfcontactcity={Zurich}}
\hypersetup{pdfcontactcountry={Switzerland}}
\hypersetup{pdfcontactemail={nbeisert@itp.phys.ethz.ch}}
\hypersetup{pdfcontacturl={http://people.phys.ethz.ch/\xmptilde nbeisert/}}

\newcommand{\secref}[1]{\hyperref[#1]{section \ref*{#1}}}

\parskip1ex
\parindent0pt
\let\olditemize\itemize
\def\itemize{\olditemize\parskip0pt}

\begin{document}

\title{The \textsf{childdoc} Package}
\hypersetup{pdftitle={The childdoc Package}}
\author{Niklas Beisert\\[2ex]
  Institut f\"ur Theoretische Physik\\
  Eidgen\"ossische Technische Hochschule Z\"urich\\
  Wolfgang-Pauli-Strasse 27, 8093 Z\"urich, Switzerland\\[1ex]
  \href{mailto:nbeisert@itp.phys.ethz.ch}
  {\texttt{nbeisert@itp.phys.ethz.ch}}}
\hypersetup{pdfauthor={Niklas Beisert}}
\hypersetup{pdfsubject={Manual for the LaTeX2e Package childdoc}}
\date{30 December 2018, \textsf{v2.0}}
\maketitle

\begin{abstract}\noindent
\textsf{childdoc} is a \LaTeXe{} package
that enables the direct compilation
of document sections included by |\include|
to individual files.
\end{abstract}

\begingroup
\parskip0ex
\tableofcontents
\endgroup

%%%%%%%%%%%%%%%%%%%%%%%%%%%%%%%%%%%%%%%%%%%%%%%%%%%%%%%%%%%%%%%%%%%%%%%%%%%%%%%%
%%%%%%%%%%%%%%%%%%%%%%%%%%%%%%%%%%%%%%%%%%%%%%%%%%%%%%%%%%%%%%%%%%%%%%%%%%%%%%%%
\section{Introduction}

\LaTeX{} provides a mechanism to structure a large document (such as a book)
into a main file and several child files (containing the chapters)
using the |\include| command.
This mechanism is beneficial for documents
which span hundreds of pages in order to
make the source file(s) more manageable.
Moreover, compilation can be restricted to
selected child files by means of the |\includeonly| command.
The latter feature can be used to reduce the compilation time while editing
(this was significantly more useful in the earlier days of \LaTeX{})
or to generate a smaller document which is easier to navigate.
Another application of |\includeonly| is to generate
documents consisting of selected parts of the complete document.

However, there are a few drawbacks of the plain |\include| mechanism:
\begin{itemize}
\item
The child files cannot be compiled on their own,
they can only be compiled via the main file.
A naive editing environment
(such as a text editor with an option
to have the current file processed by \LaTeX)
may require one to switch to the main file before compiling;
attempting to compile the child file produces errors.
\item
The main file must be modified (each time)
to adjust the |\includeonly| command
to the present needs. This easily leaves the main file in a messy state.
\item
The generated document will always carry the filename
of the main document. This is inconvenient if
several child files are to be compiled and
to be kept for distribution.
\end{itemize}

The present package provides a simple interface
to make child files individually compilable by \LaTeX{}.
Compiling a child file then has the same effect as compiling
the main file with an |\includeonly| command
to select the appropriate child.
Moreover the generated document will carry the name of the child
rather than the main file.
This resolves all three above issues.

This feature is meant to make the editing of books,
thesis documents and lecture notes somewhat more convenient.
However, the package can also be used efficiently for
composing a series of documents (such as exercise sheets)
which are typically distributed individually.
It then assists the author in generating the individual documents
(potentially in different versions)
as well as a document containing the collected series.
Another application is in developing style files
or other kinds of included material
where compilation of the style file could redirect
to a sample or test file.

%%%%%%%%%%%%%%%%%%%%%%%%%%%%%%%%%%%%%%%%%%%%%%%%%%%%%%%%%%%%%%%%%%%%%%%%%%%%%%%%
%%%%%%%%%%%%%%%%%%%%%%%%%%%%%%%%%%%%%%%%%%%%%%%%%%%%%%%%%%%%%%%%%%%%%%%%%%%%%%%%
\section{Usage}

First of all, the package \textsf{childdoc} is \emph{not} a standard
\LaTeXe{} |.sty| style file! Therefore it needs to be invoked in
a non-standard way.

%%%%%%%%%%%%%%%%%%%%%%%%%%%%%%%%%%%%%%%%%%%%%%%%%%%%%%%%%%%%%%%%%%%%%%%%%%%%%%%%
\subsection{Included Files}
\label{sec:include}

%%%%%%%%%%%%%%%%%%%%%%%%%%%%%%%%%%%%%%%%
\DescribeMacro{\childdocmain}
To use the package, add the commands
\begin{center}
\begin{tabular}{l}
|\input{childdoc.def}|\\
|\childdocmain{}|\\
\end{tabular}
\end{center}
at the very top of the main \LaTeX{} file,
in particular \emph{before} the |\documentclass| statement!
The argument of |\childdocmain| should be left empty
(but it must be present).

%%%%%%%%%%%%%%%%%%%%%%%%%%%%%%%%%%%%%%%%
\DescribeMacro{\childdocof}
Furthermore, add the commands
\begin{center}
\begin{tabular}{l}
|\input{childdoc.def}|\\
|\childdocof{|\textit{main}|}|\\
\end{tabular}
\end{center}
at the top of every child file \textit{child}
which is included by |\include{|\textit{child}|}|
from within the main file
(or at least for those files to be compiled individually).
The argument \textit{main} must be the filename of the main file.

There are a couple of
considerations in setting up the main and child documents:

%%%%%%%%%%%%%%%%%%%%%%%%%%%%%%%%%%%%%%%%
\paragraph{Restrictions.}

Please note the following restrictions:
\begin{itemize}
\item
|\childdocmain| must be called with one argument \textit{main}
to ensure compatibility with earlier version of the package.
It must either be empty (|\childdocmain{}|)
or precisely match the filename of the main file in which it is specified.
See \secref{sec:detection} for further information.
\item
The filename \textit{main} must be specified without the |.tex| extension.
\item
The filename \textit{main} is case sensitive
(even in case-insensitive file systems)
due to internal string comparison.
\item
The argument \textit{main} should be fully expanded, it cannot be a macro.
\item
Subdirectories and special characters should be avoided in filenames.
\item
The command |\childdocmain{|\textit{main}|}| must be followed by a whitespace.
It should not be followed immediately by another command
or by a comment mark `|%|'.
This is because the \TeX{} parser reads the token immediately following
the argument of |\childdocmain| and puts it
at the beginning of every child section;
however, a white\-space is ignored.
\end{itemize}

%%%%%%%%%%%%%%%%%%%%%%%%%%%%%%%%%%%%%%%%
\paragraph{Content of Main File.}

It is advisable to place all content in the child files included by |\include|.
Any output contained in the main file will appear in all child documents
unless suppressed manually;
it cannot be suppressed automatically by the |\includeonly| directive
and thus should normally be avoided.
A method to include some content in the main file
by means of conditional processing is described in \secref{sec:conditional}.

%%%%%%%%%%%%%%%%%%%%%%%%%%%%%%%%%%%%%%%%
\paragraph{Page Numbering.}

When only a part of the document is compiled,
the appropriate numbering of pages
(as well as other status parameters)
is determined from the |.aux| files.
The latter contain information from previous passes.
However this information needs to propagate through
all intermediate child documents.
Therefore the page numbering in child documents may well
be inconsistent until the complete document is compiled at least once.

A useful (if unconventional) way to always ensure a consistent
page numbering is to restart the numbering in each child document
and denote the pages by `\textit{child}|.|\textit{page}'
where \textit{child} represents the chapter/section number of the child file.
This can be achieved by the command
|\numberwithin{page}{|\textit{child}|}|
of the \textsf{amsmath} package
where \textit{child} can be |chapter| or |section|
depending on the chosen structuring.
Alternatively, one can modify the macro |\thepage| appropriately
and reset the counter |page| at the start of each child file.

%%%%%%%%%%%%%%%%%%%%%%%%%%%%%%%%%%%%%%%%%%%%%%%%%%%%%%%%%%%%%%%%%%%%%%%%%%%%%%%%
\subsection{Conditional Processing}
\label{sec:conditional}

The package provides a mechanism to compile different versions
of a document. To customise the versions further some conditional processing
can come in handy to distinguish which version is being compiled.
The package provides two macros to describe the compilation context:

%%%%%%%%%%%%%%%%%%%%%%%%%%%%%%%%%%%%%%%%
\DescribeMacro{\ifchilddoc}
The conditional |\ifchilddoc| distinguishes between the compilation of
child documents and the main document:
%
\begin{center}
|\ifchilddoc |\textit{child-code}| |[|\||else |\textit{main-code}]| \||fi|
\end{center}

%%%%%%%%%%%%%%%%%%%%%%%%%%%%%%%%%%%%%%%%
\DescribeMacro{\childdocname}
\DescribeMacro{\childdocjob}
The macro |\childdocname| contains the filename (without extension)
of the main or child file being processed.
Note that |\childdocjob| will always contain the name of the main file.

%%%%%%%%%%%%%%%%%%%%%%%%%%%%%%%%%%%%%%%%
\paragraph{Title Page.}

Conditional processing can be used to include a title or banner page
in the main document when proper precautions are taken.
Importantly, the code in the main file should ensure that the page counter
(as well as other status parameters which are stored in the |.aux| files)
takes the same value after the conditional processing.
Otherwise the page numbers may take divergent values
depending on which part is compiled.

For example, a title page could be declared by:
%
\begin{center}
\begin{tabular}{l}
|\ifchilddoc\||else|\\
|\addtocounter{page}{-1}|\\
\textit{code for title page}\\
|\newpage|\\
|\||fi|
\end{tabular}
\end{center}
%
A banner page for the child documents can be generated by:
%
\begin{center}
\begin{tabular}{l}
|\ifchilddoc|\\
|\addtocounter{page}{-1}|\\
\textit{code for banner page}\\
|\newpage|\\
|\||fi|
\end{tabular}
\end{center}
%
Here one could write a message such as:
\begin{center}
|This is the part \childdocname{} of \childdocjob{}.|
\end{center}

%%%%%%%%%%%%%%%%%%%%%%%%%%%%%%%%%%%%%%%%%%%%%%%%%%%%%%%%%%%%%%%%%%%%%%%%%%%%%%%%
\subsection{Flags}
\label{sec:flags}

The package makes it easy to generate different versions
of the main or child documents.
To this end compilation flags can be defined
and assigned different default values.
They will be particularly useful in conjunction
with the forwarding mechanism described in \secref{sec:forward}.

For example, it may be useful to have a flag |\version|
which can be set to |draft| or |final|.
The document source will contain some conditional code
depending on the value of |\version|.
Suppose further, the flag should default to |final| for the main file
and to |draft| for child files
which is a natural assignment for editing the document.
This is achieved by placing the following code
in the preamble of the main document
(below the |\childdocmain| directive):
%
\begin{center}
\begin{tabular}{l}
|\ifchilddoc|\\
|\providecommand{\version}{draft}|\\
|\||else|\\
|\providecommand{\version}{final}|\\
|\||fi|
\end{tabular}
\end{center}
%
The definition by |\providecommand| makes sure
that previous definitions are not overwritten.
Further statements |\providecommand{\version}{...}|
can thus be added before the above code to override it.

For the main file, one might add a line
(between |\childdocmain| and the above block)
%
\begin{center}
|%\ifchilddoc\||else\providecommand{\version}{draft}\||fi|
\end{center}
%
which can be uncommented to produce a draft version.
Likewise one can add a line to the very top of a child file
(above the |\childdocof{|\textit{main}|}| directive)
%
\begin{center}
|%\providecommand{\version}{final}|
\end{center}
%
which can be uncommented to produce the final version of this child document.

%%%%%%%%%%%%%%%%%%%%%%%%%%%%%%%%%%%%%%%%%%%%%%%%%%%%%%%%%%%%%%%%%%%%%%%%%%%%%%%%
\subsection{Forwarding}
\label{sec:forward}

Different versions of the main or child documents
using compilation flags as described in \secref{sec:flags}
can be (permanently) stored in different files
for convenient compilation, viewing and distribution.
To this end, the package defines a command
to pass on compilation to a different file:

%%%%%%%%%%%%%%%%%%%%%%%%%%%%%%%%%%%%%%%%
\DescribeMacro{\childdocforward}
The command |\childdocforward| redirects processing to
another source file:
%
\begin{center}
\begin{tabular}{l}
|\input{childdoc.def}|\\
|\childdocforward[|\textit{main}|]{|\textit{dest}|}|\\
\end{tabular}
\end{center}
%
The argument \textit{dest} is the destination file
(without extension).
It should be the main file or one of the child files.
Note that further \textsf{childdoc} directives
such as |\childdocof| and |\childdocforward|
in the indicated file will be processed in this form.
The optional argument \textit{main}
passes on directly to the main file \textit{main}
while pretending to compile the child \textit{dest}.
This form behaves as if \textit{dest}
issues |\childdocof{|\textit{main}|}| right away,
and no further \textsf{childdoc} directives will be processed.

%%%%%%%%%%%%%%%%%%%%%%%%%%%%%%%%%%%%%%%%
\DescribeMacro{\...prefix}
In the alternative form |\childdocforwardprefix|,
%
\begin{center}
\begin{tabular}{l}
|\input{childdoc.def}|\\
|\childdocforwardprefix[|\textit{main}|]{|\textit{prefix}|}{|\textit{dest}|}|
\end{tabular}
\end{center}
%
the destination file is determined by a pattern
depending on the current file:
To make this work, the current file must be called
`{\textit{prefix}\hspace{0.2em}\textit{suffix}}'
with \textit{prefix} matching precisely the argument.
Processing is then passed on to the file
`{\textit{dest}\hspace{0.2em}\textit{suffix}}'.
Surely, the same effect is achieved by
directly specifying the
argument `{\textit{dest}\hspace{0.2em}\textit{suffix}}'
in the first form.
However, that requires to set up a different file
for each child. With the alternative form of the command
all these files can have exactly the same content
which simplifies setting them up and maintaining them.

For example, the following file |draft.tex|
with a compilation flag |\version| as described in \secref{sec:flags}
compiles the main document as a draft:
%
\begin{center}
\begin{tabular}{l}
|\def\version{draft}|\\
|\input{childdoc.def}|\\
|\childdocforward{|\textit{main}|}|
\end{tabular}
\end{center}
%
Likewise, the following files |final|\textit{nn}|.tex|
compile the final version of the child document
|child|\textit{nn}|.tex|:
%
\begin{center}
\begin{tabular}{l}
|\def\version{final}|\\
|\input{childdoc.def}|\\
|\childdocforwardprefix{final}{child}|
\end{tabular}
\end{center}
%

Note that when several versions of a main file and/or of each child file
are to be generated, it may be convenient to set up a |Makefile| or
shell script to automatise the process.

%%%%%%%%%%%%%%%%%%%%%%%%%%%%%%%%%%%%%%%%%%%%%%%%%%%%%%%%%%%%%%%%%%%%%%%%%%%%%%%%
\subsection{Command Line Processing}
\label{sec:commandline}

The effect of redirection files can also be achieved by invoking
the \LaTeX{} compiler with a more elaborate command line.
Most conveniently this should be done as part
of a shell script or a |Makefile|.

When using \textsf{childdoc} in the main file, the following
command lines effectively perform a redirection
(note that depending on the shell being used,
backslashes may have to be doubled: `|\|' $\to$ `|\\|'):
%
\begin{center}
|... -jobname "|\textit{target}|" |\\|"|[\textit{flags}]%
|\input{childdoc.def}\childdocforward[|\textit{main}|]{|\textit{dest}|}"|
\end{center}
%
Here \textit{target} is the name of the output file,
\textit{main} is the name of the main file
and \textit{dest} is the name of the main or child file to be processed
(all filenames without extensions).
The optional argument \textit{main} can be omitted
if \textit{main} matches \textit{dest}.
Optionally, compilation \textit{flags} can be defined via |\def| commands.
This command line makes the \TeX{} engine believe
it is compiling the file \textit{target}
whose content is specified as the latter parameter.
The provided code then forwards the processing to
\textit{main} or \textit{dest} as described in \secref{sec:forward}.

%%%%%%%%%%%%%%%%%%%%%%%%%%%%%%%%%%%%%%%%%%%%%%%%%%%%%%%%%%%%%%%%%%%%%%%%%%%%%%%%
\subsection{Include by Input}
\label{sec:input}

Including child documents by |\include| has some restrictions by design.
Most notably, the content of a child document always occupies
its own set of pages; pages cannot be shared between child documents.
Usually, this behaviour makes perfect sense
because each child document contain an essential part of the document.
However, in some situations it may be desirable to compose
a document from a collection of parts
without having mandatory page breaks between then.
For this case, the package
provides a mechanism to include parts
by |\input| which can also be processed individually.
However, by construction this mechanism
requires manual handling of the content to be output.

%%%%%%%%%%%%%%%%%%%%%%%%%%%%%%%%%%%%%%%%
\DescribeMacro{\ifchilddocmanual}
The main file should be prepared as usual, see \secref{sec:include}.
However, the document body must make a distinction
between processing of an individual part and of the main document, e.g.:
%
\begin{center}
\begin{tabular}{l}
|\ifchilddocmanual|\\
|\input{\childdocname}|\\
|\||else|\\
\textit{document body with }|\input{|\textit{part}|}|\\
|\||fi|
\end{tabular}
\end{center}
%
The conditional |\ifchilddocmanual| is true whenever
a part to be included by |\input| is being compiled,
and the name of the part is stored in |\childdocname|.

%%%%%%%%%%%%%%%%%%%%%%%%%%%%%%%%%%%%%%%%
\DescribeMacro{\childdocby}
Each part to be included by |\input| should start with:
%
\begin{center}
\begin{tabular}{l}
|\input{childdoc.def}|\\
|\childdocby{|\textit{main}|}|\\
\end{tabular}
\end{center}
%
The directive |\childdocby| is similar to |\childdocof|
described in \secref{sec:include},
but the subsequent selection of content must be done manually.
To that end, both |\ifchilddoc| and |\ifchilddocmanual|
will be true upon processing of a part,
and the name of the part is stored in |\childdocname|.
Note that |\jobname| will be set to the filename of the current part
so that each part receives an individual |.aux| file
that does not interfere with the |.aux| file(s) of the main document.
This behaviour can be altered by the alternative form
|\childdocby[*]{|\textit{main}|}| (with a non-empty optional argument)
which uses the |.aux| file of the main document
by setting |\jobname| to \textit{main}.

%%%%%%%%%%%%%%%%%%%%%%%%%%%%%%%%%%%%%%%%%%%%%%%%%%%%%%%%%%%%%%%%%%%%%%%%%%%%%%%%
\subsection{Driver Development}
\label{sec:driver}

The \textsf{childdoc} mechanism can also be use for the development
of definition files such as \LaTeX{} styles or classes.
This case differs from the above setup with multiple parts
included by |\include| in that no |\includeonly| should be invoked.
This can be achieved by starting the include file
(before |\ProvidesPackage|) with:
%
\begin{center}
\begin{tabular}{l}
|\input{childdoc.def}|\\
|\childdocforward{|\textit{main}|}|\\
\end{tabular}
\end{center}
%
or alternatively with:
%
\begin{center}
\begin{tabular}{l}
|\input{childdoc.def}|\\
|\childdocby{|\textit{main}|}|\\
\end{tabular}
\end{center}
%
Both forms have slightly different effects as described above.
The main file is prepared as usual, see \secref{sec:include}.

%%%%%%%%%%%%%%%%%%%%%%%%%%%%%%%%%%%%%%%%%%%%%%%%%%%%%%%%%%%%%%%%%%%%%%%%%%%%%%%%
\subsection{Legacy Detection}
\label{sec:detection}

The directive |\childdocmain| in the main file can detect
whether the complete document or merely a child is to be compiled
even without using the directive |\childdocof|.
This method is deprecated because it is less robust
and there is no compelling reason to use it;
it is merely provided for backward compatibility
and it may be removed in future versions.

If the detection mechanism is to be used,
it is mandatory to correctly specify
the filename of the main file as the argument of |\childdocmain|:
%
\begin{center}
\begin{tabular}{l}
|\input{childdoc.def}|\\
|\childdocmain{|\textit{main}|}|\\
\end{tabular}
\end{center}
%
If |\jobname| does not match the argument \textit{main} of |\childdocmain|,
it is assumed that |\jobname| points to the child file to be compiled.
When using |\childdocmain| with the main file specified as argument,
it suffices to start a child file
with just |\input{|\textit{main}|}|
without loading of the package and using |\childdocof|.
If instead all processing is done
with the appropriate \textsf{childdoc} directives,
the argument of \textit{main} of |\childdocmain| can be empty.

An alternative version of the command line processing described
in \secref{sec:commandline} using the detection mechanism reads:
%
\begin{center}
|... -jobname "|\textit{target}|" "|[\textit{flags}]%
[|\def\jobname{|\textit{dest}|}|]|\input{|\textit{main}|}"|
\end{center}

%%%%%%%%%%%%%%%%%%%%%%%%%%%%%%%%%%%%%%%%%%%%%%%%%%%%%%%%%%%%%%%%%%%%%%%%%%%%%%%%
\subsection{Manual Code}
\label{sec:manual}

In case one cannot be certain whether the definitions file |childdoc.def|
is installed on the target \TeX{} distribution
and one prefers not to ship it,
it is conceivable to paste a few relevant commands into the sources.

To that end, drop all statements |\input{childdoc.def}|
and perform the replacements as outlined below.
Instead of |\childdocmain{|\textit{main}|}| add the following code
to the top of the main file:
%
\begin{center}
\begin{tabular}{l}
|\||ifdefined\childdocname\endinput\||fi\newif\ifchilddoc|\\
|\edef\childdocname{\scantokens\expandafter{\jobname\noexpand}}|\\
|\def\childdocmain{|\textit{main}|}\||ifx\childdocmain\childdocname\||else|\\
|\childdoctrue\includeonly{\childdocname}\let\jobname\childdocmain\||fi|\\
\end{tabular}
\end{center}
%
Instead of |\childdocof{|\textit{main}|}| just include the main file
at the top of each child file:
%
\begin{center}
|\input{|\textit{main}|}|
\end{center}
%
A simple redirection |\childdocforward{|\textit{dest}|}| is achieved by:
%
\begin{center}
|\def\jobname{|\textit{dest}|}\input{\jobname}|
\end{center}
%
The redirection with prefix
|\childdocforwardprefix[|\textit{prefix}|]{|\textit{dest}|}|
is accomplished by:
%
\begin{center}
\begin{tabular}{l}
|{\edef\jobname{\scantokens\expandafter{\jobname\noexpand}}|\\
|\def\redirectjob |\textit{prefix}|#1~~~{\gdef\jobname{|\textit{dest}|#1}}|\\
|\expandafter\redirectjob\jobname~~~}\input{\jobname}|
\end{tabular}
\end{center}

In an alternative approach,
child documents can be compiled by a specific command line
without additional code or specific definitions:
%
\begin{center}
|... -jobname "|\textit{target}|" "|[\textit{flags}]%
|\includeonly{|\textit{dest}|}\input{|\textit{main}|}"|
\end{center}
%

%%%%%%%%%%%%%%%%%%%%%%%%%%%%%%%%%%%%%%%%%%%%%%%%%%%%%%%%%%%%%%%%%%%%%%%%%%%%%%%%
%%%%%%%%%%%%%%%%%%%%%%%%%%%%%%%%%%%%%%%%%%%%%%%%%%%%%%%%%%%%%%%%%%%%%%%%%%%%%%%%
\section{Information}

%%%%%%%%%%%%%%%%%%%%%%%%%%%%%%%%%%%%%%%%%%%%%%%%%%%%%%%%%%%%%%%%%%%%%%%%%%%%%%%%
\subsection{Copyright}

Copyright \copyright{} 2017--2018 Niklas Beisert

This work may be distributed and/or modified under the
conditions of the \LaTeX{} Project Public License, either version 1.3
of this license or (at your option) any later version.
The latest version of this license is in
  \url{http://www.latex-project.org/lppl.txt}
and version 1.3 or later is part of all distributions of \LaTeX{}
version 2005/12/01 or later.

This work has the LPPL maintenance status `maintained'.

The Current Maintainer of this work is Niklas Beisert.

This work consists of the files |README.txt|, |childdoc.ins| and |childdoc.dtx|
as well as the derived files |childdoc.def|, |cdocsamp.tex|
with |cdocsch1.tex|, |cdocsch2.tex|, |cdocspt3.tex|, |cdocspt4.tex|,
|cdocsdrf.tex|, |cdocsfn1.tex|, |cdocsfn2.tex|
as well as |childdoc.pdf|.

%%%%%%%%%%%%%%%%%%%%%%%%%%%%%%%%%%%%%%%%%%%%%%%%%%%%%%%%%%%%%%%%%%%%%%%%%%%%%%%%
\subsection{Files and Installation}

The package consists of the files:
%
\begin{center}
\begin{tabular}{ll}
    |README.txt|   & readme file \\
    |childdoc.ins| & installation file \\
    |childdoc.dtx| & source file \\
    |childdoc.def| & definition file \\
    |cdocsamp.tex| & sample main file \\
    |cdocsch1.tex| & sample include file \\
    |cdocsch2.tex| & sample include file \\
    |cdocspt3.tex| & sample part file \\
    |cdocspt4.tex| & sample part file \\
    |cdocsdrf.tex| & sample redirection file \\
    |cdocsfn1.tex| & sample redirection file \\
    |cdocsfn2.tex| & sample redirection file \\
    |childdoc.pdf| & manual
\end{tabular}
\end{center}
%
The distribution consists of the files
|README.txt|, |childdoc.ins| and |childdoc.dtx|.
%
\begin{itemize}
\item
Run (pdf)\LaTeX{} on |childdoc.dtx|
to compile the manual |childdoc.pdf| (this file).
\item
Run \LaTeX{} on |childdoc.ins| to create the definitions file |childdoc.def|
and the sample |cdocsamp.tex| with include files
|cdocsch1.tex|, |cdocsch2.tex|, |cdocspt3.tex|, |cdocspt4.tex|,
|cdocsdrf.tex|, |cdocsfn1.tex|, |cdocsfn2.tex|.
Then copy the file |childdoc.def| to an appropriate directory of your \LaTeX{}
distribution, e.g.\ \textit{texmf-root}|/tex/latex/childdoc|.
\end{itemize}

%%%%%%%%%%%%%%%%%%%%%%%%%%%%%%%%%%%%%%%%%%%%%%%%%%%%%%%%%%%%%%%%%%%%%%%%%%%%%%%%
\subsection{Related CTAN Packages}

There are several other packages which offer a similar functionality:
%
\begin{itemize}
\item
The packages
\href{http://ctan.org/pkg/docmute}{\textsf{docmute}},
\href{http://ctan.org/pkg/includex}{\textsf{includex}} and
\href{http://ctan.org/pkg/standalone}{\textsf{standalone}}
provide commands to include only the document body of
a child file thus allowing both files to be compiled individually.
\item
The packages \href{http://ctan.org/pkg/subdocs}{\textsf{subdocs}}
and \href{http://ctan.org/pkg/subfiles}{\textsf{subfiles}}
provide structures in which the main and child documents can be
encapsulated and allowing them to be compiled individually.
The inclusion mechanism is different from the conventional |\include|.
\item
The package \href{http://ctan.org/pkg/combine}{\textsf{combine}}
is an elaborate solution to combine several documents into one.
\end{itemize}
%
See also the CTAN topic \href{http://ctan.org/topic/subdocs}{\textsf{subdocs}}
for further related packages.
The present package differs from the above solutions in that
a document structure constructed with the conventional |\include| mechanism
just needs two extra commands at the top of every file
such that all constituent files can be compiled individually.

%%%%%%%%%%%%%%%%%%%%%%%%%%%%%%%%%%%%%%%%%%%%%%%%%%%%%%%%%%%%%%%%%%%%%%%%%%%%%%%%
%\subsection{Feature Suggestions}
%
%The following is a list of features which may be useful for future
%versions of this package:
%%
%\begin{itemize}
%\item
%\ldots
%\end{itemize}

%%%%%%%%%%%%%%%%%%%%%%%%%%%%%%%%%%%%%%%%%%%%%%%%%%%%%%%%%%%%%%%%%%%%%%%%%%%%%%%%
\subsection{Revision History}

%%%%%%%%%%%%%%%%%%%%%%%%%%%%%%%%%%%%%%%%
\paragraph{v2.0:} 2018/12/30

\begin{itemize}
\item
immediate forward processing
\item
added |\childdocby| mechanism
\item
manual restructured
\end{itemize}

%%%%%%%%%%%%%%%%%%%%%%%%%%%%%%%%%%%%%%%%
\paragraph{v1.6:} 2018/01/17

\begin{itemize}
\item
application for development of include files
\item
corrections to manual
\end{itemize}

%%%%%%%%%%%%%%%%%%%%%%%%%%%%%%%%%%%%%%%%
\paragraph{v1.5:} 2017/05/21

\begin{itemize}
\item
more complete structuring introduced
\item
|\childdocof| introduced
\item
|\childdoc| renamed to |\childdocmain|
\item
|\childredirect| renamed to |\childdocforward| and |\childdocforwardprefix|
and functionality expanded
\end{itemize}

%%%%%%%%%%%%%%%%%%%%%%%%%%%%%%%%%%%%%%%%
\paragraph{v1.0:} 2017/04/27

\begin{itemize}
\item
manual and install package
\item
first version published on CTAN
\end{itemize}

%%%%%%%%%%%%%%%%%%%%%%%%%%%%%%%%%%%%%%%%
\paragraph{v0.6:} 2017/04/26

\begin{itemize}
\item
redirection mechanism added
\end{itemize}

%%%%%%%%%%%%%%%%%%%%%%%%%%%%%%%%%%%%%%%%
\paragraph{v0.5:} 2017/04/26

\begin{itemize}
\item
functionality in definition file
\end{itemize}


%%%%%%%%%%%%%%%%%%%%%%%%%%%%%%%%%%%%%%%%%%%%%%%%%%%%%%%%%%%%%%%%%%%%%%%%%%%%%%%%
%%%%%%%%%%%%%%%%%%%%%%%%%%%%%%%%%%%%%%%%%%%%%%%%%%%%%%%%%%%%%%%%%%%%%%%%%%%%%%%%
%%%%%%%%%%%%%%%%%%%%%%%%%%%%%%%%%%%%%%%%%%%%%%%%%%%%%%%%%%%%%%%%%%%%%%%%%%%%%%%%
\appendix

\settowidth\MacroIndent{\rmfamily\scriptsize 000\ }

 \DocInput{childdoc.dtx}

\end{document}
%</driver>
% \fi
%
% %%%%%%%%%%%%%%%%%%%%%%%%%%%%%%%%%%%%%%%%%%%%%%%%%%%%%%%%%%%%%%%%%%%%%%%%%%%%%%
% %%%%%%%%%%%%%%%%%%%%%%%%%%%%%%%%%%%%%%%%%%%%%%%%%%%%%%%%%%%%%%%%%%%%%%%%%%%%%%
% \section{Sample}
%\iffalse
%<*samplemain>
%\fi
%
% The following presents a sample document
% with two chapters, two parts, a title page,
% a compile flag as well as three forwarding files to set the flag.
% It consists of eight |.tex| files:
% \begin{center}
% \begin{tabular}{ll}
% |cdocsamp.tex|&main file\\
% |cdocsch1.tex|&include file for chapter 1\\
% |cdocsch2.tex|&include file for chapter 2\\
% |cdocspt3.tex|&include file for part 3\\
% |cdocspt4.tex|&include file for part 4\\
% |cdocsdrf.tex|&forwarding file for main file in draft mode\\
% |cdocsfi1.tex|&forwarding file for final version of chapter 1\\
% |cdocsfi2.tex|&forwarding file for final version of chapter 2\\
% \end{tabular}
% \end{center}
% Each of the eight files can be compiled directly by the \LaTeX{} compiler.
%
% %%%%%%%%%%%%%%%%%%%%%%%%%%%%%%%%%%%%%%
% \paragraph{Main File.}
%
% The main file is called |cdocsamp.tex|.
%
% Load the \textsf{childdoc} definitions and
% declare the filename for the main document:
%    \begin{macrocode}
\input{childdoc.def}
\childdocmain{}
%    \end{macrocode}

% Optional override for |\version| flag:
%    \begin{macrocode}
%%\ifchilddoc\else\providecommand{\version}{draft}\fi
%    \end{macrocode}

% Define the default values for the |\version| flag
% (|final| for the main file and |draft| for childs):
%    \begin{macrocode}
\ifchilddoc
\providecommand{\version}{draft}
\else
\providecommand{\version}{final}
\fi
%    \end{macrocode}

% Load the standard document class:
%    \begin{macrocode}
\documentclass[12pt]{article}
%    \end{macrocode}

% Start the document body:
%    \begin{macrocode}
\begin{document}
%    \end{macrocode}

% Declare a title page.
% Print title, part of document being processed and version flag:
%    \begin{macrocode}
\addtocounter{page}{-1}
\begin{center}
{\LARGE\bfseries{}childdoc example\par}
\vspace{1cm}
\ifchilddoc
\ifchilddocmanual part\else chapter\fi:
`\childdocname' of `\childdocjob'\par
\else
main document: `\childdocjob'\par
\fi
version: \version\par
\end{center}
\newpage
%    \end{macrocode}

% Manually include selected file,
% otherwise process as usual:
%    \begin{macrocode}
\ifchilddocmanual
\section*{part `\childdocname'}
\input{\childdocname}
\else
%    \end{macrocode}

% Include the two chapters:
%    \begin{macrocode}
\include{cdocsch1}
\include{cdocsch2}
%    \end{macrocode}

% Include the two parts unless only chapters should be displayed:
%    \begin{macrocode}
\ifchilddoc\else
\section{part three}
\input{cdocspt3}
\section{part four}
\input{cdocspt4}
\fi
%    \end{macrocode}

% Process as usual until here:
%    \begin{macrocode}
\fi
%    \end{macrocode}

% End of document body:
%    \begin{macrocode}
\end{document}
%    \end{macrocode}
%\iffalse
%</samplemain>
%\fi
%
% %%%%%%%%%%%%%%%%%%%%%%%%%%%%%%%%%%%%%%
% \paragraph{Chapter Include Files.}
%
% The include files are called |cdocsch1.tex| and |cdocsch2.tex|.
%
%\iffalse
%<*samplechap1|samplechap2>
%\fi

% Optional override for |\version| flag:
%    \begin{macrocode}
%%\providecommand{\version}{final}
%    \end{macrocode}

% Include the main document:
%    \begin{macrocode}
\input{childdoc.def}
\childdocof{cdocsamp}
%    \end{macrocode}

%\iffalse
%</samplechap1|samplechap2>
%\fi
%
%\iffalse
%<*samplechap1>
%\fi
% Some text for chapter 1:
%    \begin{macrocode}
\section{one}
some text in chapter one
%    \end{macrocode}

%\iffalse
%</samplechap1>
%\fi
% Some text for chapter 2:
%\iffalse
%<*samplechap2>
%\fi
%    \begin{macrocode}
\section{two}
more text in chapter two
%    \end{macrocode}

%\iffalse
%</samplechap2>
%\fi
%
% %%%%%%%%%%%%%%%%%%%%%%%%%%%%%%%%%%%%%%
% \paragraph{Part Include Files.}
%
% The include files are called |cdocspt3.tex| and |cdocspt4.tex|.
%
%\iffalse
%<*samplepart3|samplepart4>
%\fi

% Optional override for |\version| flag:
%    \begin{macrocode}
%%\providecommand{\version}{final}
%    \end{macrocode}

% Include the main document:
%    \begin{macrocode}
\input{childdoc.def}
\childdocby{cdocsamp}
%    \end{macrocode}

%\iffalse
%</samplepart3|samplepart4>
%\fi
%
%\iffalse
%<*samplepart3>
%\fi
% Some text for part 3:
%    \begin{macrocode}
some text in part three
%    \end{macrocode}

%\iffalse
%</samplepart3>
%\fi
% Some text for part 4:
%\iffalse
%<*samplepart4>
%\fi
%    \begin{macrocode}
more text in part four
%    \end{macrocode}

%\iffalse
%</samplepart4>
%\fi
%
% %%%%%%%%%%%%%%%%%%%%%%%%%%%%%%%%%%%%%%
% \paragraph{Forwarding for a Complete Draft.}
%
% The following forwarding file |cdocsdrf.tex|
% compiles the main document in draft mode:
%\iffalse
%<*sampledraft>
%\fi
%    \begin{macrocode}
\def\version{draft}
\input{childdoc.def}
\childdocforward{cdocsamp}
%    \end{macrocode}

%\iffalse
%</sampledraft>
%\fi
%
% %%%%%%%%%%%%%%%%%%%%%%%%%%%%%%%%%%%%%%
% \paragraph{Forwarding for Final Version of the Chapters.}
%
% The following forwarding files |cdocsfn1.tex| and |cdocsfn2.tex|
% (with identical content)
% compile the final versions of the child documents
% |cdocsch1.tex| and |cdocsch2.tex|, respectively:
%\iffalse
%<*samplefinal>
%\fi
%    \begin{macrocode}
\def\version{final}
\input{childdoc.def}
\childdocforwardprefix[cdocsamp]{cdocsfn}{cdocsch}
%    \end{macrocode}

%\iffalse
%</samplefinal>
%\fi
%
% %%%%%%%%%%%%%%%%%%%%%%%%%%%%%%%%%%%%%%
% \paragraph{Command Line Processing.}
%
% The following three command lines generate the output files
% |cdocscld|, |cdocscl1| and |cdocscl2|
% which should be identical to
% |cdocsdrf|, |cdocsch1| and |cdocsfn2|, respectively:
% \begin{center}
% \begin{tabular}{l}
% |latex -jobname cdocscld \|\\
% |  "\def\version{draft}\input{childdoc.def}\childdocforward{cdocsamp}"|\\
% |latex -jobname cdocscl1 \|\\
% |  "\input{childdoc.def}\childdocforward[cdocsamp]{cdocsch1}"|\\
% |latex -jobname cdocscl2 \|\\
% |  "\def\version{final}\input{childdoc.def}\childdocforward{cdocsch2}"|
% \end{tabular}
% \end{center}
% Note that the trailing backslash on each first line
% merely continues the input to the second line
% (for convenient cut ant paste).
% Furthermore, the command |latex| can be replaced by any
% of its alternative versions such as |pdflatex|.
%
% %%%%%%%%%%%%%%%%%%%%%%%%%%%%%%%%%%%%%%%%%%%%%%%%%%%%%%%%%%%%%%%%%%%%%%%%%%%%%%
% %%%%%%%%%%%%%%%%%%%%%%%%%%%%%%%%%%%%%%%%%%%%%%%%%%%%%%%%%%%%%%%%%%%%%%%%%%%%%%
% \section{Implementation}
%\iffalse
%<*package>
%\fi
%
% This section describes the definitions file |childdoc.def|.

% The definitions cannot be loaded using |\usepackage| or |\RequirePackage|
% which has a mechanism to prevent loading a style file more than once.
% When loading the definitions by means of |\input|
% multiple instances have to be prevented manually:
%\iffalse
%This code needs to be before the `\ProvidesFile' directive
%which is defined at the beginning of this file.
%Therefore it is also placed there and commented out here.
%</package>
%<*discard>
%\fi
%    \begin{macrocode}
\ifdefined\childdocmain\endinput\fi
%    \end{macrocode}
%\iffalse
%</discard>
%<*package>
%\fi
%
% \macro{\ifchilddoc}
% \macro{\ifchilddocmanual}
% The conditional |\ifchilddoc| tells whether a
% child (true) or main (false) document is being compiled.
% The conditional |\ifchilddocmanual| tells whether
% the |\includeonly| mechanism is used (false) or
% the selection of child files must be performed manually (true).
% The definitions initialise to false:
%    \begin{macrocode}
\newif\ifchilddoc
\newif\ifchilddocmanual
%    \end{macrocode}

% \macro{\childdocname}
% \macro{\childdocjob}
% The macro |\childdocname| stores the name of the main document
% to be compiled. The macro |\childdocjob| stores the name of
% the document on which the \LaTeX{} compiler was originally invoked.
% The content of |\jobname| cannot be compared
% to filenames specified in the source due to different catcodes.
% The following code rescans |\jobname|, stores the result
% in |\childdocname| and saves a copy in |\childdocjob|:
%    \begin{macrocode}
\edef\childdocname{\scantokens\expandafter{\jobname\noexpand}}
\let\childdocjob\childdocname
%    \end{macrocode}

% \macro{\childdocdisable}
% The macro |\childdocdisable| prevents the main file
% from being processed more than once.
% At this stage, the main document command |\childdocmain|
% is assumed to be called once again where it should do nothing.
% Any subsequent call to it should prevent
% a secondary processing of the main document
% It overwrites the forwarding commands
% |\childdocof| and |\childdocforward|
% with empty macros to prevent further inclusions of the main document:
%    \begin{macrocode}
\newcommand{\childdocdisable}
{
  \renewcommand{\childdocmain}[1]{\renewcommand{\childdocmain}[1]{\endinput}}
  \renewcommand{\childdocof}[1]{}
  \renewcommand{\childdocby}[2][]{}
  \renewcommand{\childdocforward}[2][]{}
  \renewcommand{\childdocdisable}{}
}
%    \end{macrocode}

% \macro{\childdocmain}
% The macro |\childdocmain| is to be called at the top of the main file
% with nothing or the main filename (without extension) as argument.
% First, it breaks loops.
% If the argument is not empty and does not match |\childdocname|
% (which is set by the first inclusion of |childdoc.def|),
% |\ifchilddoc| is set to true, |\includeonly| is applied to the child file
% and |\jobname| is set to the main file
% (for proper handling of |.aux| files):
%    \begin{macrocode}
\newcommand{\childdocmain}[1]
{
  \childdocdisable\childdocmain{}
  \if?#1?\else
    \begingroup
      \def\childdoctmp{#1}
      \ifx\childdoctmp\childdocname
        \def\childdoctmp{}
      \else
        \def\childdoctmp
        {
          \childdoctrue
          \includeonly{\childdocname}
          \def\childdocjob{#1}
          \def\jobname{#1}
        }
      \fi
      \expandafter
    \endgroup
    \childdoctmp
  \fi
}
%    \end{macrocode}

% \macro{\childdocof}
% The command |\childdocof| redirects
% compilation to the main file |#1|.
%    \begin{macrocode}
\newcommand{\childdocof}[1]
{
  \childdocdisable
  \childdoctrue
  \includeonly{\childdocname}
  \def\jobname{#1}
  \def\childdocjob{#1}
  \input{#1}
}
%    \end{macrocode}

% \macro{\childdocby}
% The command |\childdocby| ....
%    \begin{macrocode}
\newcommand{\childdocby}[2][]
{
  \childdocdisable
  \childdoctrue
  \childdocmanualtrue
  \if?#1?\else
    \def\jobname{#2}
  \fi
  \def\childdocjob{#2}
  \input{#2}
  \endinput
}
%    \end{macrocode}

% \macro{\childdocforward}
% The command |\childdocforward| redirects
% compilation to the main file or
% (if the optional argument is given) a child file.
% Parameters are set as if the main file
% or a child file starting with |\childdocof| was compiled.
% Then compilation is handed over to the main file:
%    \begin{macrocode}
\newcommand{\childdocforward}[2][]
{
  \begingroup
    \if?#1?
      \def\childdoctmp
      {
        \def\childdocname{#2}
        \def\childdocjob{#2}
        \def\jobname{#2}
        \input{#2}
        \endinput
      }
    \else
      \def\childdoctmp
      {
        \childdocdisable
        \def\childdocname{#2}
        \childdoctrue
        \includeonly{#2}
        \def\childdocjob{#1}
        \def\jobname{#1}
        \input{#1}
        \endinput
      }
    \fi
    \expandafter
  \endgroup
  \childdoctmp
}
%    \end{macrocode}

% \macro{\childdocforwardprefix}
% The command |\childdocforwardprefix| redirects
% compilation to the main or a child file by means of a pattern.
% The prefix |#1| in the current filename is replaced by |#2|
% and the suffix of the current filename is kept
% (it is assumed that the filename does not contain the substring `|~~~|'
% which is used as a delimiter).
% Compilation is handed over to the new file by |\childdocforward|:
%    \begin{macrocode}
\newcommand{\childdocforwardprefix}[3][]
{
  \begingroup
    \def\childdocextract #2##1~~~{\def\childdoctmp{\childdocforward[#1]{#3##1}}}
    \expandafter\childdocextract\childdocname~~~
    \expandafter
  \endgroup
  \childdoctmp
}
%    \end{macrocode}

% \macro{\childdoc}
% The deprecated macro |\childdoc| is a legacy version of |\childdocmain|:
%    \begin{macrocode}
\newcommand{\childdoc}{\childdocmain}
%    \end{macrocode}

% \macro{\childdocredirect}
% The deprecated macro |\childdocredirect| is a legacy version
% of |\childdocforward| and |\childdocforwardprefix|:
%    \begin{macrocode}
\newcommand{\childdocredirect}[2][]
{
  \begingroup
    \if?#1?
      \def\childdoctmp{\childdocforward{#2}}
    \else
      \def\childdoctmp{\childdocforwardprefix{#1}{#2}}
    \fi
    \expandafter
  \endgroup
  \childdoctmp
}
%    \end{macrocode}

%\iffalse
%</package>
%\fi
%
\endinput
|\\
|\childdocof{|\textit{main}|}|\\
\end{tabular}
\end{center}
at the top of every child file \textit{child}
which is included by |\include{|\textit{child}|}|
from within the main file
(or at least for those files to be compiled individually).
The argument \textit{main} must be the filename of the main file.

There are a couple of
considerations in setting up the main and child documents:

%%%%%%%%%%%%%%%%%%%%%%%%%%%%%%%%%%%%%%%%
\paragraph{Restrictions.}

Please note the following restrictions:
\begin{itemize}
\item
|\childdocmain| must be called with one argument \textit{main}
to ensure compatibility with earlier version of the package.
It must either be empty (|\childdocmain{}|)
or precisely match the filename of the main file in which it is specified.
See \secref{sec:detection} for further information.
\item
The filename \textit{main} must be specified without the |.tex| extension.
\item
The filename \textit{main} is case sensitive
(even in case-insensitive file systems)
due to internal string comparison.
\item
The argument \textit{main} should be fully expanded, it cannot be a macro.
\item
Subdirectories and special characters should be avoided in filenames.
\item
The command |\childdocmain{|\textit{main}|}| must be followed by a whitespace.
It should not be followed immediately by another command
or by a comment mark `|%|'.
This is because the \TeX{} parser reads the token immediately following
the argument of |\childdocmain| and puts it
at the beginning of every child section;
however, a white\-space is ignored.
\end{itemize}

%%%%%%%%%%%%%%%%%%%%%%%%%%%%%%%%%%%%%%%%
\paragraph{Content of Main File.}

It is advisable to place all content in the child files included by |\include|.
Any output contained in the main file will appear in all child documents
unless suppressed manually;
it cannot be suppressed automatically by the |\includeonly| directive
and thus should normally be avoided.
A method to include some content in the main file
by means of conditional processing is described in \secref{sec:conditional}.

%%%%%%%%%%%%%%%%%%%%%%%%%%%%%%%%%%%%%%%%
\paragraph{Page Numbering.}

When only a part of the document is compiled,
the appropriate numbering of pages
(as well as other status parameters)
is determined from the |.aux| files.
The latter contain information from previous passes.
However this information needs to propagate through
all intermediate child documents.
Therefore the page numbering in child documents may well
be inconsistent until the complete document is compiled at least once.

A useful (if unconventional) way to always ensure a consistent
page numbering is to restart the numbering in each child document
and denote the pages by `\textit{child}|.|\textit{page}'
where \textit{child} represents the chapter/section number of the child file.
This can be achieved by the command
|\numberwithin{page}{|\textit{child}|}|
of the \textsf{amsmath} package
where \textit{child} can be |chapter| or |section|
depending on the chosen structuring.
Alternatively, one can modify the macro |\thepage| appropriately
and reset the counter |page| at the start of each child file.

%%%%%%%%%%%%%%%%%%%%%%%%%%%%%%%%%%%%%%%%%%%%%%%%%%%%%%%%%%%%%%%%%%%%%%%%%%%%%%%%
\subsection{Conditional Processing}
\label{sec:conditional}

The package provides a mechanism to compile different versions
of a document. To customise the versions further some conditional processing
can come in handy to distinguish which version is being compiled.
The package provides two macros to describe the compilation context:

%%%%%%%%%%%%%%%%%%%%%%%%%%%%%%%%%%%%%%%%
\DescribeMacro{\ifchilddoc}
The conditional |\ifchilddoc| distinguishes between the compilation of
child documents and the main document:
%
\begin{center}
|\ifchilddoc |\textit{child-code}| |[|\||else |\textit{main-code}]| \||fi|
\end{center}

%%%%%%%%%%%%%%%%%%%%%%%%%%%%%%%%%%%%%%%%
\DescribeMacro{\childdocname}
\DescribeMacro{\childdocjob}
The macro |\childdocname| contains the filename (without extension)
of the main or child file being processed.
Note that |\childdocjob| will always contain the name of the main file.

%%%%%%%%%%%%%%%%%%%%%%%%%%%%%%%%%%%%%%%%
\paragraph{Title Page.}

Conditional processing can be used to include a title or banner page
in the main document when proper precautions are taken.
Importantly, the code in the main file should ensure that the page counter
(as well as other status parameters which are stored in the |.aux| files)
takes the same value after the conditional processing.
Otherwise the page numbers may take divergent values
depending on which part is compiled.

For example, a title page could be declared by:
%
\begin{center}
\begin{tabular}{l}
|\ifchilddoc\||else|\\
|\addtocounter{page}{-1}|\\
\textit{code for title page}\\
|\newpage|\\
|\||fi|
\end{tabular}
\end{center}
%
A banner page for the child documents can be generated by:
%
\begin{center}
\begin{tabular}{l}
|\ifchilddoc|\\
|\addtocounter{page}{-1}|\\
\textit{code for banner page}\\
|\newpage|\\
|\||fi|
\end{tabular}
\end{center}
%
Here one could write a message such as:
\begin{center}
|This is the part \childdocname{} of \childdocjob{}.|
\end{center}

%%%%%%%%%%%%%%%%%%%%%%%%%%%%%%%%%%%%%%%%%%%%%%%%%%%%%%%%%%%%%%%%%%%%%%%%%%%%%%%%
\subsection{Flags}
\label{sec:flags}

The package makes it easy to generate different versions
of the main or child documents.
To this end compilation flags can be defined
and assigned different default values.
They will be particularly useful in conjunction
with the forwarding mechanism described in \secref{sec:forward}.

For example, it may be useful to have a flag |\version|
which can be set to |draft| or |final|.
The document source will contain some conditional code
depending on the value of |\version|.
Suppose further, the flag should default to |final| for the main file
and to |draft| for child files
which is a natural assignment for editing the document.
This is achieved by placing the following code
in the preamble of the main document
(below the |\childdocmain| directive):
%
\begin{center}
\begin{tabular}{l}
|\ifchilddoc|\\
|\providecommand{\version}{draft}|\\
|\||else|\\
|\providecommand{\version}{final}|\\
|\||fi|
\end{tabular}
\end{center}
%
The definition by |\providecommand| makes sure
that previous definitions are not overwritten.
Further statements |\providecommand{\version}{...}|
can thus be added before the above code to override it.

For the main file, one might add a line
(between |\childdocmain| and the above block)
%
\begin{center}
|%\ifchilddoc\||else\providecommand{\version}{draft}\||fi|
\end{center}
%
which can be uncommented to produce a draft version.
Likewise one can add a line to the very top of a child file
(above the |\childdocof{|\textit{main}|}| directive)
%
\begin{center}
|%\providecommand{\version}{final}|
\end{center}
%
which can be uncommented to produce the final version of this child document.

%%%%%%%%%%%%%%%%%%%%%%%%%%%%%%%%%%%%%%%%%%%%%%%%%%%%%%%%%%%%%%%%%%%%%%%%%%%%%%%%
\subsection{Forwarding}
\label{sec:forward}

Different versions of the main or child documents
using compilation flags as described in \secref{sec:flags}
can be (permanently) stored in different files
for convenient compilation, viewing and distribution.
To this end, the package defines a command
to pass on compilation to a different file:

%%%%%%%%%%%%%%%%%%%%%%%%%%%%%%%%%%%%%%%%
\DescribeMacro{\childdocforward}
The command |\childdocforward| redirects processing to
another source file:
%
\begin{center}
\begin{tabular}{l}
|% \iffalse
%
% childdoc.dtx Copyright (C) 2017-2018 Niklas Beisert
%
% This work may be distributed and/or modified under the
% conditions of the LaTeX Project Public License, either version 1.3
% of this license or (at your option) any later version.
% The latest version of this license is in
%   http://www.latex-project.org/lppl.txt
% and version 1.3 or later is part of all distributions of LaTeX
% version 2005/12/01 or later.
%
% This work has the LPPL maintenance status `maintained'.
%
% The Current Maintainer of this work is Niklas Beisert.
%
% This work consists of the files childdoc.dtx and childdoc.ins
% and the derived files childdoc.def and cdocsamp.tex with
% cdocsch1.tex, cdocsch2.tex, cdocsdrf.tex, cdocsfn1.tex, cdocsfn2.tex.
%
%<package>\ifdefined\childdocmain\endinput\fi
%<package>\ProvidesFile{childdoc.def}[2018/12/30 v2.0 child document driver]
%<samplemain>\ProvidesFile{cdocsamp.tex}[2018/12/30 v2.0 sample for childdoc]
%<*driver>
%\ProvidesFile{childdoc.drv}[2018/12/30 v2.0 childdoc reference manual file]
\PassOptionsToClass{10pt,a4paper}{article}
\documentclass{ltxdoc}

\usepackage[margin=35mm]{geometry}
\usepackage{hyperref}
\usepackage{hyperxmp}
\usepackage[usenames]{color}

\hypersetup{colorlinks=true}
\hypersetup{pdfstartview=FitH}
\hypersetup{pdfpagemode=UseNone}
\hypersetup{pdfsource={}}
\hypersetup{pdflang={en-UK}}
\hypersetup{pdfcopyright={Copyright 2017-2018 Niklas Beisert.
  This work may be distributed and/or modified under the
  conditions of the LaTeX Project Public License, either version 1.3
  of this license or (at your option) any later version.}}
\hypersetup{pdflicenseurl={http://www.latex-project.org/lppl.txt}}
\hypersetup{pdfcontactaddress={ETH Zurich, ITP, HIT K,
  Wolfgang-Pauli-Strasse 27}}
\hypersetup{pdfcontactpostcode={8093}}
\hypersetup{pdfcontactcity={Zurich}}
\hypersetup{pdfcontactcountry={Switzerland}}
\hypersetup{pdfcontactemail={nbeisert@itp.phys.ethz.ch}}
\hypersetup{pdfcontacturl={http://people.phys.ethz.ch/\xmptilde nbeisert/}}

\newcommand{\secref}[1]{\hyperref[#1]{section \ref*{#1}}}

\parskip1ex
\parindent0pt
\let\olditemize\itemize
\def\itemize{\olditemize\parskip0pt}

\begin{document}

\title{The \textsf{childdoc} Package}
\hypersetup{pdftitle={The childdoc Package}}
\author{Niklas Beisert\\[2ex]
  Institut f\"ur Theoretische Physik\\
  Eidgen\"ossische Technische Hochschule Z\"urich\\
  Wolfgang-Pauli-Strasse 27, 8093 Z\"urich, Switzerland\\[1ex]
  \href{mailto:nbeisert@itp.phys.ethz.ch}
  {\texttt{nbeisert@itp.phys.ethz.ch}}}
\hypersetup{pdfauthor={Niklas Beisert}}
\hypersetup{pdfsubject={Manual for the LaTeX2e Package childdoc}}
\date{30 December 2018, \textsf{v2.0}}
\maketitle

\begin{abstract}\noindent
\textsf{childdoc} is a \LaTeXe{} package
that enables the direct compilation
of document sections included by |\include|
to individual files.
\end{abstract}

\begingroup
\parskip0ex
\tableofcontents
\endgroup

%%%%%%%%%%%%%%%%%%%%%%%%%%%%%%%%%%%%%%%%%%%%%%%%%%%%%%%%%%%%%%%%%%%%%%%%%%%%%%%%
%%%%%%%%%%%%%%%%%%%%%%%%%%%%%%%%%%%%%%%%%%%%%%%%%%%%%%%%%%%%%%%%%%%%%%%%%%%%%%%%
\section{Introduction}

\LaTeX{} provides a mechanism to structure a large document (such as a book)
into a main file and several child files (containing the chapters)
using the |\include| command.
This mechanism is beneficial for documents
which span hundreds of pages in order to
make the source file(s) more manageable.
Moreover, compilation can be restricted to
selected child files by means of the |\includeonly| command.
The latter feature can be used to reduce the compilation time while editing
(this was significantly more useful in the earlier days of \LaTeX{})
or to generate a smaller document which is easier to navigate.
Another application of |\includeonly| is to generate
documents consisting of selected parts of the complete document.

However, there are a few drawbacks of the plain |\include| mechanism:
\begin{itemize}
\item
The child files cannot be compiled on their own,
they can only be compiled via the main file.
A naive editing environment
(such as a text editor with an option
to have the current file processed by \LaTeX)
may require one to switch to the main file before compiling;
attempting to compile the child file produces errors.
\item
The main file must be modified (each time)
to adjust the |\includeonly| command
to the present needs. This easily leaves the main file in a messy state.
\item
The generated document will always carry the filename
of the main document. This is inconvenient if
several child files are to be compiled and
to be kept for distribution.
\end{itemize}

The present package provides a simple interface
to make child files individually compilable by \LaTeX{}.
Compiling a child file then has the same effect as compiling
the main file with an |\includeonly| command
to select the appropriate child.
Moreover the generated document will carry the name of the child
rather than the main file.
This resolves all three above issues.

This feature is meant to make the editing of books,
thesis documents and lecture notes somewhat more convenient.
However, the package can also be used efficiently for
composing a series of documents (such as exercise sheets)
which are typically distributed individually.
It then assists the author in generating the individual documents
(potentially in different versions)
as well as a document containing the collected series.
Another application is in developing style files
or other kinds of included material
where compilation of the style file could redirect
to a sample or test file.

%%%%%%%%%%%%%%%%%%%%%%%%%%%%%%%%%%%%%%%%%%%%%%%%%%%%%%%%%%%%%%%%%%%%%%%%%%%%%%%%
%%%%%%%%%%%%%%%%%%%%%%%%%%%%%%%%%%%%%%%%%%%%%%%%%%%%%%%%%%%%%%%%%%%%%%%%%%%%%%%%
\section{Usage}

First of all, the package \textsf{childdoc} is \emph{not} a standard
\LaTeXe{} |.sty| style file! Therefore it needs to be invoked in
a non-standard way.

%%%%%%%%%%%%%%%%%%%%%%%%%%%%%%%%%%%%%%%%%%%%%%%%%%%%%%%%%%%%%%%%%%%%%%%%%%%%%%%%
\subsection{Included Files}
\label{sec:include}

%%%%%%%%%%%%%%%%%%%%%%%%%%%%%%%%%%%%%%%%
\DescribeMacro{\childdocmain}
To use the package, add the commands
\begin{center}
\begin{tabular}{l}
|\input{childdoc.def}|\\
|\childdocmain{}|\\
\end{tabular}
\end{center}
at the very top of the main \LaTeX{} file,
in particular \emph{before} the |\documentclass| statement!
The argument of |\childdocmain| should be left empty
(but it must be present).

%%%%%%%%%%%%%%%%%%%%%%%%%%%%%%%%%%%%%%%%
\DescribeMacro{\childdocof}
Furthermore, add the commands
\begin{center}
\begin{tabular}{l}
|\input{childdoc.def}|\\
|\childdocof{|\textit{main}|}|\\
\end{tabular}
\end{center}
at the top of every child file \textit{child}
which is included by |\include{|\textit{child}|}|
from within the main file
(or at least for those files to be compiled individually).
The argument \textit{main} must be the filename of the main file.

There are a couple of
considerations in setting up the main and child documents:

%%%%%%%%%%%%%%%%%%%%%%%%%%%%%%%%%%%%%%%%
\paragraph{Restrictions.}

Please note the following restrictions:
\begin{itemize}
\item
|\childdocmain| must be called with one argument \textit{main}
to ensure compatibility with earlier version of the package.
It must either be empty (|\childdocmain{}|)
or precisely match the filename of the main file in which it is specified.
See \secref{sec:detection} for further information.
\item
The filename \textit{main} must be specified without the |.tex| extension.
\item
The filename \textit{main} is case sensitive
(even in case-insensitive file systems)
due to internal string comparison.
\item
The argument \textit{main} should be fully expanded, it cannot be a macro.
\item
Subdirectories and special characters should be avoided in filenames.
\item
The command |\childdocmain{|\textit{main}|}| must be followed by a whitespace.
It should not be followed immediately by another command
or by a comment mark `|%|'.
This is because the \TeX{} parser reads the token immediately following
the argument of |\childdocmain| and puts it
at the beginning of every child section;
however, a white\-space is ignored.
\end{itemize}

%%%%%%%%%%%%%%%%%%%%%%%%%%%%%%%%%%%%%%%%
\paragraph{Content of Main File.}

It is advisable to place all content in the child files included by |\include|.
Any output contained in the main file will appear in all child documents
unless suppressed manually;
it cannot be suppressed automatically by the |\includeonly| directive
and thus should normally be avoided.
A method to include some content in the main file
by means of conditional processing is described in \secref{sec:conditional}.

%%%%%%%%%%%%%%%%%%%%%%%%%%%%%%%%%%%%%%%%
\paragraph{Page Numbering.}

When only a part of the document is compiled,
the appropriate numbering of pages
(as well as other status parameters)
is determined from the |.aux| files.
The latter contain information from previous passes.
However this information needs to propagate through
all intermediate child documents.
Therefore the page numbering in child documents may well
be inconsistent until the complete document is compiled at least once.

A useful (if unconventional) way to always ensure a consistent
page numbering is to restart the numbering in each child document
and denote the pages by `\textit{child}|.|\textit{page}'
where \textit{child} represents the chapter/section number of the child file.
This can be achieved by the command
|\numberwithin{page}{|\textit{child}|}|
of the \textsf{amsmath} package
where \textit{child} can be |chapter| or |section|
depending on the chosen structuring.
Alternatively, one can modify the macro |\thepage| appropriately
and reset the counter |page| at the start of each child file.

%%%%%%%%%%%%%%%%%%%%%%%%%%%%%%%%%%%%%%%%%%%%%%%%%%%%%%%%%%%%%%%%%%%%%%%%%%%%%%%%
\subsection{Conditional Processing}
\label{sec:conditional}

The package provides a mechanism to compile different versions
of a document. To customise the versions further some conditional processing
can come in handy to distinguish which version is being compiled.
The package provides two macros to describe the compilation context:

%%%%%%%%%%%%%%%%%%%%%%%%%%%%%%%%%%%%%%%%
\DescribeMacro{\ifchilddoc}
The conditional |\ifchilddoc| distinguishes between the compilation of
child documents and the main document:
%
\begin{center}
|\ifchilddoc |\textit{child-code}| |[|\||else |\textit{main-code}]| \||fi|
\end{center}

%%%%%%%%%%%%%%%%%%%%%%%%%%%%%%%%%%%%%%%%
\DescribeMacro{\childdocname}
\DescribeMacro{\childdocjob}
The macro |\childdocname| contains the filename (without extension)
of the main or child file being processed.
Note that |\childdocjob| will always contain the name of the main file.

%%%%%%%%%%%%%%%%%%%%%%%%%%%%%%%%%%%%%%%%
\paragraph{Title Page.}

Conditional processing can be used to include a title or banner page
in the main document when proper precautions are taken.
Importantly, the code in the main file should ensure that the page counter
(as well as other status parameters which are stored in the |.aux| files)
takes the same value after the conditional processing.
Otherwise the page numbers may take divergent values
depending on which part is compiled.

For example, a title page could be declared by:
%
\begin{center}
\begin{tabular}{l}
|\ifchilddoc\||else|\\
|\addtocounter{page}{-1}|\\
\textit{code for title page}\\
|\newpage|\\
|\||fi|
\end{tabular}
\end{center}
%
A banner page for the child documents can be generated by:
%
\begin{center}
\begin{tabular}{l}
|\ifchilddoc|\\
|\addtocounter{page}{-1}|\\
\textit{code for banner page}\\
|\newpage|\\
|\||fi|
\end{tabular}
\end{center}
%
Here one could write a message such as:
\begin{center}
|This is the part \childdocname{} of \childdocjob{}.|
\end{center}

%%%%%%%%%%%%%%%%%%%%%%%%%%%%%%%%%%%%%%%%%%%%%%%%%%%%%%%%%%%%%%%%%%%%%%%%%%%%%%%%
\subsection{Flags}
\label{sec:flags}

The package makes it easy to generate different versions
of the main or child documents.
To this end compilation flags can be defined
and assigned different default values.
They will be particularly useful in conjunction
with the forwarding mechanism described in \secref{sec:forward}.

For example, it may be useful to have a flag |\version|
which can be set to |draft| or |final|.
The document source will contain some conditional code
depending on the value of |\version|.
Suppose further, the flag should default to |final| for the main file
and to |draft| for child files
which is a natural assignment for editing the document.
This is achieved by placing the following code
in the preamble of the main document
(below the |\childdocmain| directive):
%
\begin{center}
\begin{tabular}{l}
|\ifchilddoc|\\
|\providecommand{\version}{draft}|\\
|\||else|\\
|\providecommand{\version}{final}|\\
|\||fi|
\end{tabular}
\end{center}
%
The definition by |\providecommand| makes sure
that previous definitions are not overwritten.
Further statements |\providecommand{\version}{...}|
can thus be added before the above code to override it.

For the main file, one might add a line
(between |\childdocmain| and the above block)
%
\begin{center}
|%\ifchilddoc\||else\providecommand{\version}{draft}\||fi|
\end{center}
%
which can be uncommented to produce a draft version.
Likewise one can add a line to the very top of a child file
(above the |\childdocof{|\textit{main}|}| directive)
%
\begin{center}
|%\providecommand{\version}{final}|
\end{center}
%
which can be uncommented to produce the final version of this child document.

%%%%%%%%%%%%%%%%%%%%%%%%%%%%%%%%%%%%%%%%%%%%%%%%%%%%%%%%%%%%%%%%%%%%%%%%%%%%%%%%
\subsection{Forwarding}
\label{sec:forward}

Different versions of the main or child documents
using compilation flags as described in \secref{sec:flags}
can be (permanently) stored in different files
for convenient compilation, viewing and distribution.
To this end, the package defines a command
to pass on compilation to a different file:

%%%%%%%%%%%%%%%%%%%%%%%%%%%%%%%%%%%%%%%%
\DescribeMacro{\childdocforward}
The command |\childdocforward| redirects processing to
another source file:
%
\begin{center}
\begin{tabular}{l}
|\input{childdoc.def}|\\
|\childdocforward[|\textit{main}|]{|\textit{dest}|}|\\
\end{tabular}
\end{center}
%
The argument \textit{dest} is the destination file
(without extension).
It should be the main file or one of the child files.
Note that further \textsf{childdoc} directives
such as |\childdocof| and |\childdocforward|
in the indicated file will be processed in this form.
The optional argument \textit{main}
passes on directly to the main file \textit{main}
while pretending to compile the child \textit{dest}.
This form behaves as if \textit{dest}
issues |\childdocof{|\textit{main}|}| right away,
and no further \textsf{childdoc} directives will be processed.

%%%%%%%%%%%%%%%%%%%%%%%%%%%%%%%%%%%%%%%%
\DescribeMacro{\...prefix}
In the alternative form |\childdocforwardprefix|,
%
\begin{center}
\begin{tabular}{l}
|\input{childdoc.def}|\\
|\childdocforwardprefix[|\textit{main}|]{|\textit{prefix}|}{|\textit{dest}|}|
\end{tabular}
\end{center}
%
the destination file is determined by a pattern
depending on the current file:
To make this work, the current file must be called
`{\textit{prefix}\hspace{0.2em}\textit{suffix}}'
with \textit{prefix} matching precisely the argument.
Processing is then passed on to the file
`{\textit{dest}\hspace{0.2em}\textit{suffix}}'.
Surely, the same effect is achieved by
directly specifying the
argument `{\textit{dest}\hspace{0.2em}\textit{suffix}}'
in the first form.
However, that requires to set up a different file
for each child. With the alternative form of the command
all these files can have exactly the same content
which simplifies setting them up and maintaining them.

For example, the following file |draft.tex|
with a compilation flag |\version| as described in \secref{sec:flags}
compiles the main document as a draft:
%
\begin{center}
\begin{tabular}{l}
|\def\version{draft}|\\
|\input{childdoc.def}|\\
|\childdocforward{|\textit{main}|}|
\end{tabular}
\end{center}
%
Likewise, the following files |final|\textit{nn}|.tex|
compile the final version of the child document
|child|\textit{nn}|.tex|:
%
\begin{center}
\begin{tabular}{l}
|\def\version{final}|\\
|\input{childdoc.def}|\\
|\childdocforwardprefix{final}{child}|
\end{tabular}
\end{center}
%

Note that when several versions of a main file and/or of each child file
are to be generated, it may be convenient to set up a |Makefile| or
shell script to automatise the process.

%%%%%%%%%%%%%%%%%%%%%%%%%%%%%%%%%%%%%%%%%%%%%%%%%%%%%%%%%%%%%%%%%%%%%%%%%%%%%%%%
\subsection{Command Line Processing}
\label{sec:commandline}

The effect of redirection files can also be achieved by invoking
the \LaTeX{} compiler with a more elaborate command line.
Most conveniently this should be done as part
of a shell script or a |Makefile|.

When using \textsf{childdoc} in the main file, the following
command lines effectively perform a redirection
(note that depending on the shell being used,
backslashes may have to be doubled: `|\|' $\to$ `|\\|'):
%
\begin{center}
|... -jobname "|\textit{target}|" |\\|"|[\textit{flags}]%
|\input{childdoc.def}\childdocforward[|\textit{main}|]{|\textit{dest}|}"|
\end{center}
%
Here \textit{target} is the name of the output file,
\textit{main} is the name of the main file
and \textit{dest} is the name of the main or child file to be processed
(all filenames without extensions).
The optional argument \textit{main} can be omitted
if \textit{main} matches \textit{dest}.
Optionally, compilation \textit{flags} can be defined via |\def| commands.
This command line makes the \TeX{} engine believe
it is compiling the file \textit{target}
whose content is specified as the latter parameter.
The provided code then forwards the processing to
\textit{main} or \textit{dest} as described in \secref{sec:forward}.

%%%%%%%%%%%%%%%%%%%%%%%%%%%%%%%%%%%%%%%%%%%%%%%%%%%%%%%%%%%%%%%%%%%%%%%%%%%%%%%%
\subsection{Include by Input}
\label{sec:input}

Including child documents by |\include| has some restrictions by design.
Most notably, the content of a child document always occupies
its own set of pages; pages cannot be shared between child documents.
Usually, this behaviour makes perfect sense
because each child document contain an essential part of the document.
However, in some situations it may be desirable to compose
a document from a collection of parts
without having mandatory page breaks between then.
For this case, the package
provides a mechanism to include parts
by |\input| which can also be processed individually.
However, by construction this mechanism
requires manual handling of the content to be output.

%%%%%%%%%%%%%%%%%%%%%%%%%%%%%%%%%%%%%%%%
\DescribeMacro{\ifchilddocmanual}
The main file should be prepared as usual, see \secref{sec:include}.
However, the document body must make a distinction
between processing of an individual part and of the main document, e.g.:
%
\begin{center}
\begin{tabular}{l}
|\ifchilddocmanual|\\
|\input{\childdocname}|\\
|\||else|\\
\textit{document body with }|\input{|\textit{part}|}|\\
|\||fi|
\end{tabular}
\end{center}
%
The conditional |\ifchilddocmanual| is true whenever
a part to be included by |\input| is being compiled,
and the name of the part is stored in |\childdocname|.

%%%%%%%%%%%%%%%%%%%%%%%%%%%%%%%%%%%%%%%%
\DescribeMacro{\childdocby}
Each part to be included by |\input| should start with:
%
\begin{center}
\begin{tabular}{l}
|\input{childdoc.def}|\\
|\childdocby{|\textit{main}|}|\\
\end{tabular}
\end{center}
%
The directive |\childdocby| is similar to |\childdocof|
described in \secref{sec:include},
but the subsequent selection of content must be done manually.
To that end, both |\ifchilddoc| and |\ifchilddocmanual|
will be true upon processing of a part,
and the name of the part is stored in |\childdocname|.
Note that |\jobname| will be set to the filename of the current part
so that each part receives an individual |.aux| file
that does not interfere with the |.aux| file(s) of the main document.
This behaviour can be altered by the alternative form
|\childdocby[*]{|\textit{main}|}| (with a non-empty optional argument)
which uses the |.aux| file of the main document
by setting |\jobname| to \textit{main}.

%%%%%%%%%%%%%%%%%%%%%%%%%%%%%%%%%%%%%%%%%%%%%%%%%%%%%%%%%%%%%%%%%%%%%%%%%%%%%%%%
\subsection{Driver Development}
\label{sec:driver}

The \textsf{childdoc} mechanism can also be use for the development
of definition files such as \LaTeX{} styles or classes.
This case differs from the above setup with multiple parts
included by |\include| in that no |\includeonly| should be invoked.
This can be achieved by starting the include file
(before |\ProvidesPackage|) with:
%
\begin{center}
\begin{tabular}{l}
|\input{childdoc.def}|\\
|\childdocforward{|\textit{main}|}|\\
\end{tabular}
\end{center}
%
or alternatively with:
%
\begin{center}
\begin{tabular}{l}
|\input{childdoc.def}|\\
|\childdocby{|\textit{main}|}|\\
\end{tabular}
\end{center}
%
Both forms have slightly different effects as described above.
The main file is prepared as usual, see \secref{sec:include}.

%%%%%%%%%%%%%%%%%%%%%%%%%%%%%%%%%%%%%%%%%%%%%%%%%%%%%%%%%%%%%%%%%%%%%%%%%%%%%%%%
\subsection{Legacy Detection}
\label{sec:detection}

The directive |\childdocmain| in the main file can detect
whether the complete document or merely a child is to be compiled
even without using the directive |\childdocof|.
This method is deprecated because it is less robust
and there is no compelling reason to use it;
it is merely provided for backward compatibility
and it may be removed in future versions.

If the detection mechanism is to be used,
it is mandatory to correctly specify
the filename of the main file as the argument of |\childdocmain|:
%
\begin{center}
\begin{tabular}{l}
|\input{childdoc.def}|\\
|\childdocmain{|\textit{main}|}|\\
\end{tabular}
\end{center}
%
If |\jobname| does not match the argument \textit{main} of |\childdocmain|,
it is assumed that |\jobname| points to the child file to be compiled.
When using |\childdocmain| with the main file specified as argument,
it suffices to start a child file
with just |\input{|\textit{main}|}|
without loading of the package and using |\childdocof|.
If instead all processing is done
with the appropriate \textsf{childdoc} directives,
the argument of \textit{main} of |\childdocmain| can be empty.

An alternative version of the command line processing described
in \secref{sec:commandline} using the detection mechanism reads:
%
\begin{center}
|... -jobname "|\textit{target}|" "|[\textit{flags}]%
[|\def\jobname{|\textit{dest}|}|]|\input{|\textit{main}|}"|
\end{center}

%%%%%%%%%%%%%%%%%%%%%%%%%%%%%%%%%%%%%%%%%%%%%%%%%%%%%%%%%%%%%%%%%%%%%%%%%%%%%%%%
\subsection{Manual Code}
\label{sec:manual}

In case one cannot be certain whether the definitions file |childdoc.def|
is installed on the target \TeX{} distribution
and one prefers not to ship it,
it is conceivable to paste a few relevant commands into the sources.

To that end, drop all statements |\input{childdoc.def}|
and perform the replacements as outlined below.
Instead of |\childdocmain{|\textit{main}|}| add the following code
to the top of the main file:
%
\begin{center}
\begin{tabular}{l}
|\||ifdefined\childdocname\endinput\||fi\newif\ifchilddoc|\\
|\edef\childdocname{\scantokens\expandafter{\jobname\noexpand}}|\\
|\def\childdocmain{|\textit{main}|}\||ifx\childdocmain\childdocname\||else|\\
|\childdoctrue\includeonly{\childdocname}\let\jobname\childdocmain\||fi|\\
\end{tabular}
\end{center}
%
Instead of |\childdocof{|\textit{main}|}| just include the main file
at the top of each child file:
%
\begin{center}
|\input{|\textit{main}|}|
\end{center}
%
A simple redirection |\childdocforward{|\textit{dest}|}| is achieved by:
%
\begin{center}
|\def\jobname{|\textit{dest}|}\input{\jobname}|
\end{center}
%
The redirection with prefix
|\childdocforwardprefix[|\textit{prefix}|]{|\textit{dest}|}|
is accomplished by:
%
\begin{center}
\begin{tabular}{l}
|{\edef\jobname{\scantokens\expandafter{\jobname\noexpand}}|\\
|\def\redirectjob |\textit{prefix}|#1~~~{\gdef\jobname{|\textit{dest}|#1}}|\\
|\expandafter\redirectjob\jobname~~~}\input{\jobname}|
\end{tabular}
\end{center}

In an alternative approach,
child documents can be compiled by a specific command line
without additional code or specific definitions:
%
\begin{center}
|... -jobname "|\textit{target}|" "|[\textit{flags}]%
|\includeonly{|\textit{dest}|}\input{|\textit{main}|}"|
\end{center}
%

%%%%%%%%%%%%%%%%%%%%%%%%%%%%%%%%%%%%%%%%%%%%%%%%%%%%%%%%%%%%%%%%%%%%%%%%%%%%%%%%
%%%%%%%%%%%%%%%%%%%%%%%%%%%%%%%%%%%%%%%%%%%%%%%%%%%%%%%%%%%%%%%%%%%%%%%%%%%%%%%%
\section{Information}

%%%%%%%%%%%%%%%%%%%%%%%%%%%%%%%%%%%%%%%%%%%%%%%%%%%%%%%%%%%%%%%%%%%%%%%%%%%%%%%%
\subsection{Copyright}

Copyright \copyright{} 2017--2018 Niklas Beisert

This work may be distributed and/or modified under the
conditions of the \LaTeX{} Project Public License, either version 1.3
of this license or (at your option) any later version.
The latest version of this license is in
  \url{http://www.latex-project.org/lppl.txt}
and version 1.3 or later is part of all distributions of \LaTeX{}
version 2005/12/01 or later.

This work has the LPPL maintenance status `maintained'.

The Current Maintainer of this work is Niklas Beisert.

This work consists of the files |README.txt|, |childdoc.ins| and |childdoc.dtx|
as well as the derived files |childdoc.def|, |cdocsamp.tex|
with |cdocsch1.tex|, |cdocsch2.tex|, |cdocspt3.tex|, |cdocspt4.tex|,
|cdocsdrf.tex|, |cdocsfn1.tex|, |cdocsfn2.tex|
as well as |childdoc.pdf|.

%%%%%%%%%%%%%%%%%%%%%%%%%%%%%%%%%%%%%%%%%%%%%%%%%%%%%%%%%%%%%%%%%%%%%%%%%%%%%%%%
\subsection{Files and Installation}

The package consists of the files:
%
\begin{center}
\begin{tabular}{ll}
    |README.txt|   & readme file \\
    |childdoc.ins| & installation file \\
    |childdoc.dtx| & source file \\
    |childdoc.def| & definition file \\
    |cdocsamp.tex| & sample main file \\
    |cdocsch1.tex| & sample include file \\
    |cdocsch2.tex| & sample include file \\
    |cdocspt3.tex| & sample part file \\
    |cdocspt4.tex| & sample part file \\
    |cdocsdrf.tex| & sample redirection file \\
    |cdocsfn1.tex| & sample redirection file \\
    |cdocsfn2.tex| & sample redirection file \\
    |childdoc.pdf| & manual
\end{tabular}
\end{center}
%
The distribution consists of the files
|README.txt|, |childdoc.ins| and |childdoc.dtx|.
%
\begin{itemize}
\item
Run (pdf)\LaTeX{} on |childdoc.dtx|
to compile the manual |childdoc.pdf| (this file).
\item
Run \LaTeX{} on |childdoc.ins| to create the definitions file |childdoc.def|
and the sample |cdocsamp.tex| with include files
|cdocsch1.tex|, |cdocsch2.tex|, |cdocspt3.tex|, |cdocspt4.tex|,
|cdocsdrf.tex|, |cdocsfn1.tex|, |cdocsfn2.tex|.
Then copy the file |childdoc.def| to an appropriate directory of your \LaTeX{}
distribution, e.g.\ \textit{texmf-root}|/tex/latex/childdoc|.
\end{itemize}

%%%%%%%%%%%%%%%%%%%%%%%%%%%%%%%%%%%%%%%%%%%%%%%%%%%%%%%%%%%%%%%%%%%%%%%%%%%%%%%%
\subsection{Related CTAN Packages}

There are several other packages which offer a similar functionality:
%
\begin{itemize}
\item
The packages
\href{http://ctan.org/pkg/docmute}{\textsf{docmute}},
\href{http://ctan.org/pkg/includex}{\textsf{includex}} and
\href{http://ctan.org/pkg/standalone}{\textsf{standalone}}
provide commands to include only the document body of
a child file thus allowing both files to be compiled individually.
\item
The packages \href{http://ctan.org/pkg/subdocs}{\textsf{subdocs}}
and \href{http://ctan.org/pkg/subfiles}{\textsf{subfiles}}
provide structures in which the main and child documents can be
encapsulated and allowing them to be compiled individually.
The inclusion mechanism is different from the conventional |\include|.
\item
The package \href{http://ctan.org/pkg/combine}{\textsf{combine}}
is an elaborate solution to combine several documents into one.
\end{itemize}
%
See also the CTAN topic \href{http://ctan.org/topic/subdocs}{\textsf{subdocs}}
for further related packages.
The present package differs from the above solutions in that
a document structure constructed with the conventional |\include| mechanism
just needs two extra commands at the top of every file
such that all constituent files can be compiled individually.

%%%%%%%%%%%%%%%%%%%%%%%%%%%%%%%%%%%%%%%%%%%%%%%%%%%%%%%%%%%%%%%%%%%%%%%%%%%%%%%%
%\subsection{Feature Suggestions}
%
%The following is a list of features which may be useful for future
%versions of this package:
%%
%\begin{itemize}
%\item
%\ldots
%\end{itemize}

%%%%%%%%%%%%%%%%%%%%%%%%%%%%%%%%%%%%%%%%%%%%%%%%%%%%%%%%%%%%%%%%%%%%%%%%%%%%%%%%
\subsection{Revision History}

%%%%%%%%%%%%%%%%%%%%%%%%%%%%%%%%%%%%%%%%
\paragraph{v2.0:} 2018/12/30

\begin{itemize}
\item
immediate forward processing
\item
added |\childdocby| mechanism
\item
manual restructured
\end{itemize}

%%%%%%%%%%%%%%%%%%%%%%%%%%%%%%%%%%%%%%%%
\paragraph{v1.6:} 2018/01/17

\begin{itemize}
\item
application for development of include files
\item
corrections to manual
\end{itemize}

%%%%%%%%%%%%%%%%%%%%%%%%%%%%%%%%%%%%%%%%
\paragraph{v1.5:} 2017/05/21

\begin{itemize}
\item
more complete structuring introduced
\item
|\childdocof| introduced
\item
|\childdoc| renamed to |\childdocmain|
\item
|\childredirect| renamed to |\childdocforward| and |\childdocforwardprefix|
and functionality expanded
\end{itemize}

%%%%%%%%%%%%%%%%%%%%%%%%%%%%%%%%%%%%%%%%
\paragraph{v1.0:} 2017/04/27

\begin{itemize}
\item
manual and install package
\item
first version published on CTAN
\end{itemize}

%%%%%%%%%%%%%%%%%%%%%%%%%%%%%%%%%%%%%%%%
\paragraph{v0.6:} 2017/04/26

\begin{itemize}
\item
redirection mechanism added
\end{itemize}

%%%%%%%%%%%%%%%%%%%%%%%%%%%%%%%%%%%%%%%%
\paragraph{v0.5:} 2017/04/26

\begin{itemize}
\item
functionality in definition file
\end{itemize}


%%%%%%%%%%%%%%%%%%%%%%%%%%%%%%%%%%%%%%%%%%%%%%%%%%%%%%%%%%%%%%%%%%%%%%%%%%%%%%%%
%%%%%%%%%%%%%%%%%%%%%%%%%%%%%%%%%%%%%%%%%%%%%%%%%%%%%%%%%%%%%%%%%%%%%%%%%%%%%%%%
%%%%%%%%%%%%%%%%%%%%%%%%%%%%%%%%%%%%%%%%%%%%%%%%%%%%%%%%%%%%%%%%%%%%%%%%%%%%%%%%
\appendix

\settowidth\MacroIndent{\rmfamily\scriptsize 000\ }

 \DocInput{childdoc.dtx}

\end{document}
%</driver>
% \fi
%
% %%%%%%%%%%%%%%%%%%%%%%%%%%%%%%%%%%%%%%%%%%%%%%%%%%%%%%%%%%%%%%%%%%%%%%%%%%%%%%
% %%%%%%%%%%%%%%%%%%%%%%%%%%%%%%%%%%%%%%%%%%%%%%%%%%%%%%%%%%%%%%%%%%%%%%%%%%%%%%
% \section{Sample}
%\iffalse
%<*samplemain>
%\fi
%
% The following presents a sample document
% with two chapters, two parts, a title page,
% a compile flag as well as three forwarding files to set the flag.
% It consists of eight |.tex| files:
% \begin{center}
% \begin{tabular}{ll}
% |cdocsamp.tex|&main file\\
% |cdocsch1.tex|&include file for chapter 1\\
% |cdocsch2.tex|&include file for chapter 2\\
% |cdocspt3.tex|&include file for part 3\\
% |cdocspt4.tex|&include file for part 4\\
% |cdocsdrf.tex|&forwarding file for main file in draft mode\\
% |cdocsfi1.tex|&forwarding file for final version of chapter 1\\
% |cdocsfi2.tex|&forwarding file for final version of chapter 2\\
% \end{tabular}
% \end{center}
% Each of the eight files can be compiled directly by the \LaTeX{} compiler.
%
% %%%%%%%%%%%%%%%%%%%%%%%%%%%%%%%%%%%%%%
% \paragraph{Main File.}
%
% The main file is called |cdocsamp.tex|.
%
% Load the \textsf{childdoc} definitions and
% declare the filename for the main document:
%    \begin{macrocode}
\input{childdoc.def}
\childdocmain{}
%    \end{macrocode}

% Optional override for |\version| flag:
%    \begin{macrocode}
%%\ifchilddoc\else\providecommand{\version}{draft}\fi
%    \end{macrocode}

% Define the default values for the |\version| flag
% (|final| for the main file and |draft| for childs):
%    \begin{macrocode}
\ifchilddoc
\providecommand{\version}{draft}
\else
\providecommand{\version}{final}
\fi
%    \end{macrocode}

% Load the standard document class:
%    \begin{macrocode}
\documentclass[12pt]{article}
%    \end{macrocode}

% Start the document body:
%    \begin{macrocode}
\begin{document}
%    \end{macrocode}

% Declare a title page.
% Print title, part of document being processed and version flag:
%    \begin{macrocode}
\addtocounter{page}{-1}
\begin{center}
{\LARGE\bfseries{}childdoc example\par}
\vspace{1cm}
\ifchilddoc
\ifchilddocmanual part\else chapter\fi:
`\childdocname' of `\childdocjob'\par
\else
main document: `\childdocjob'\par
\fi
version: \version\par
\end{center}
\newpage
%    \end{macrocode}

% Manually include selected file,
% otherwise process as usual:
%    \begin{macrocode}
\ifchilddocmanual
\section*{part `\childdocname'}
\input{\childdocname}
\else
%    \end{macrocode}

% Include the two chapters:
%    \begin{macrocode}
\include{cdocsch1}
\include{cdocsch2}
%    \end{macrocode}

% Include the two parts unless only chapters should be displayed:
%    \begin{macrocode}
\ifchilddoc\else
\section{part three}
\input{cdocspt3}
\section{part four}
\input{cdocspt4}
\fi
%    \end{macrocode}

% Process as usual until here:
%    \begin{macrocode}
\fi
%    \end{macrocode}

% End of document body:
%    \begin{macrocode}
\end{document}
%    \end{macrocode}
%\iffalse
%</samplemain>
%\fi
%
% %%%%%%%%%%%%%%%%%%%%%%%%%%%%%%%%%%%%%%
% \paragraph{Chapter Include Files.}
%
% The include files are called |cdocsch1.tex| and |cdocsch2.tex|.
%
%\iffalse
%<*samplechap1|samplechap2>
%\fi

% Optional override for |\version| flag:
%    \begin{macrocode}
%%\providecommand{\version}{final}
%    \end{macrocode}

% Include the main document:
%    \begin{macrocode}
\input{childdoc.def}
\childdocof{cdocsamp}
%    \end{macrocode}

%\iffalse
%</samplechap1|samplechap2>
%\fi
%
%\iffalse
%<*samplechap1>
%\fi
% Some text for chapter 1:
%    \begin{macrocode}
\section{one}
some text in chapter one
%    \end{macrocode}

%\iffalse
%</samplechap1>
%\fi
% Some text for chapter 2:
%\iffalse
%<*samplechap2>
%\fi
%    \begin{macrocode}
\section{two}
more text in chapter two
%    \end{macrocode}

%\iffalse
%</samplechap2>
%\fi
%
% %%%%%%%%%%%%%%%%%%%%%%%%%%%%%%%%%%%%%%
% \paragraph{Part Include Files.}
%
% The include files are called |cdocspt3.tex| and |cdocspt4.tex|.
%
%\iffalse
%<*samplepart3|samplepart4>
%\fi

% Optional override for |\version| flag:
%    \begin{macrocode}
%%\providecommand{\version}{final}
%    \end{macrocode}

% Include the main document:
%    \begin{macrocode}
\input{childdoc.def}
\childdocby{cdocsamp}
%    \end{macrocode}

%\iffalse
%</samplepart3|samplepart4>
%\fi
%
%\iffalse
%<*samplepart3>
%\fi
% Some text for part 3:
%    \begin{macrocode}
some text in part three
%    \end{macrocode}

%\iffalse
%</samplepart3>
%\fi
% Some text for part 4:
%\iffalse
%<*samplepart4>
%\fi
%    \begin{macrocode}
more text in part four
%    \end{macrocode}

%\iffalse
%</samplepart4>
%\fi
%
% %%%%%%%%%%%%%%%%%%%%%%%%%%%%%%%%%%%%%%
% \paragraph{Forwarding for a Complete Draft.}
%
% The following forwarding file |cdocsdrf.tex|
% compiles the main document in draft mode:
%\iffalse
%<*sampledraft>
%\fi
%    \begin{macrocode}
\def\version{draft}
\input{childdoc.def}
\childdocforward{cdocsamp}
%    \end{macrocode}

%\iffalse
%</sampledraft>
%\fi
%
% %%%%%%%%%%%%%%%%%%%%%%%%%%%%%%%%%%%%%%
% \paragraph{Forwarding for Final Version of the Chapters.}
%
% The following forwarding files |cdocsfn1.tex| and |cdocsfn2.tex|
% (with identical content)
% compile the final versions of the child documents
% |cdocsch1.tex| and |cdocsch2.tex|, respectively:
%\iffalse
%<*samplefinal>
%\fi
%    \begin{macrocode}
\def\version{final}
\input{childdoc.def}
\childdocforwardprefix[cdocsamp]{cdocsfn}{cdocsch}
%    \end{macrocode}

%\iffalse
%</samplefinal>
%\fi
%
% %%%%%%%%%%%%%%%%%%%%%%%%%%%%%%%%%%%%%%
% \paragraph{Command Line Processing.}
%
% The following three command lines generate the output files
% |cdocscld|, |cdocscl1| and |cdocscl2|
% which should be identical to
% |cdocsdrf|, |cdocsch1| and |cdocsfn2|, respectively:
% \begin{center}
% \begin{tabular}{l}
% |latex -jobname cdocscld \|\\
% |  "\def\version{draft}\input{childdoc.def}\childdocforward{cdocsamp}"|\\
% |latex -jobname cdocscl1 \|\\
% |  "\input{childdoc.def}\childdocforward[cdocsamp]{cdocsch1}"|\\
% |latex -jobname cdocscl2 \|\\
% |  "\def\version{final}\input{childdoc.def}\childdocforward{cdocsch2}"|
% \end{tabular}
% \end{center}
% Note that the trailing backslash on each first line
% merely continues the input to the second line
% (for convenient cut ant paste).
% Furthermore, the command |latex| can be replaced by any
% of its alternative versions such as |pdflatex|.
%
% %%%%%%%%%%%%%%%%%%%%%%%%%%%%%%%%%%%%%%%%%%%%%%%%%%%%%%%%%%%%%%%%%%%%%%%%%%%%%%
% %%%%%%%%%%%%%%%%%%%%%%%%%%%%%%%%%%%%%%%%%%%%%%%%%%%%%%%%%%%%%%%%%%%%%%%%%%%%%%
% \section{Implementation}
%\iffalse
%<*package>
%\fi
%
% This section describes the definitions file |childdoc.def|.

% The definitions cannot be loaded using |\usepackage| or |\RequirePackage|
% which has a mechanism to prevent loading a style file more than once.
% When loading the definitions by means of |\input|
% multiple instances have to be prevented manually:
%\iffalse
%This code needs to be before the `\ProvidesFile' directive
%which is defined at the beginning of this file.
%Therefore it is also placed there and commented out here.
%</package>
%<*discard>
%\fi
%    \begin{macrocode}
\ifdefined\childdocmain\endinput\fi
%    \end{macrocode}
%\iffalse
%</discard>
%<*package>
%\fi
%
% \macro{\ifchilddoc}
% \macro{\ifchilddocmanual}
% The conditional |\ifchilddoc| tells whether a
% child (true) or main (false) document is being compiled.
% The conditional |\ifchilddocmanual| tells whether
% the |\includeonly| mechanism is used (false) or
% the selection of child files must be performed manually (true).
% The definitions initialise to false:
%    \begin{macrocode}
\newif\ifchilddoc
\newif\ifchilddocmanual
%    \end{macrocode}

% \macro{\childdocname}
% \macro{\childdocjob}
% The macro |\childdocname| stores the name of the main document
% to be compiled. The macro |\childdocjob| stores the name of
% the document on which the \LaTeX{} compiler was originally invoked.
% The content of |\jobname| cannot be compared
% to filenames specified in the source due to different catcodes.
% The following code rescans |\jobname|, stores the result
% in |\childdocname| and saves a copy in |\childdocjob|:
%    \begin{macrocode}
\edef\childdocname{\scantokens\expandafter{\jobname\noexpand}}
\let\childdocjob\childdocname
%    \end{macrocode}

% \macro{\childdocdisable}
% The macro |\childdocdisable| prevents the main file
% from being processed more than once.
% At this stage, the main document command |\childdocmain|
% is assumed to be called once again where it should do nothing.
% Any subsequent call to it should prevent
% a secondary processing of the main document
% It overwrites the forwarding commands
% |\childdocof| and |\childdocforward|
% with empty macros to prevent further inclusions of the main document:
%    \begin{macrocode}
\newcommand{\childdocdisable}
{
  \renewcommand{\childdocmain}[1]{\renewcommand{\childdocmain}[1]{\endinput}}
  \renewcommand{\childdocof}[1]{}
  \renewcommand{\childdocby}[2][]{}
  \renewcommand{\childdocforward}[2][]{}
  \renewcommand{\childdocdisable}{}
}
%    \end{macrocode}

% \macro{\childdocmain}
% The macro |\childdocmain| is to be called at the top of the main file
% with nothing or the main filename (without extension) as argument.
% First, it breaks loops.
% If the argument is not empty and does not match |\childdocname|
% (which is set by the first inclusion of |childdoc.def|),
% |\ifchilddoc| is set to true, |\includeonly| is applied to the child file
% and |\jobname| is set to the main file
% (for proper handling of |.aux| files):
%    \begin{macrocode}
\newcommand{\childdocmain}[1]
{
  \childdocdisable\childdocmain{}
  \if?#1?\else
    \begingroup
      \def\childdoctmp{#1}
      \ifx\childdoctmp\childdocname
        \def\childdoctmp{}
      \else
        \def\childdoctmp
        {
          \childdoctrue
          \includeonly{\childdocname}
          \def\childdocjob{#1}
          \def\jobname{#1}
        }
      \fi
      \expandafter
    \endgroup
    \childdoctmp
  \fi
}
%    \end{macrocode}

% \macro{\childdocof}
% The command |\childdocof| redirects
% compilation to the main file |#1|.
%    \begin{macrocode}
\newcommand{\childdocof}[1]
{
  \childdocdisable
  \childdoctrue
  \includeonly{\childdocname}
  \def\jobname{#1}
  \def\childdocjob{#1}
  \input{#1}
}
%    \end{macrocode}

% \macro{\childdocby}
% The command |\childdocby| ....
%    \begin{macrocode}
\newcommand{\childdocby}[2][]
{
  \childdocdisable
  \childdoctrue
  \childdocmanualtrue
  \if?#1?\else
    \def\jobname{#2}
  \fi
  \def\childdocjob{#2}
  \input{#2}
  \endinput
}
%    \end{macrocode}

% \macro{\childdocforward}
% The command |\childdocforward| redirects
% compilation to the main file or
% (if the optional argument is given) a child file.
% Parameters are set as if the main file
% or a child file starting with |\childdocof| was compiled.
% Then compilation is handed over to the main file:
%    \begin{macrocode}
\newcommand{\childdocforward}[2][]
{
  \begingroup
    \if?#1?
      \def\childdoctmp
      {
        \def\childdocname{#2}
        \def\childdocjob{#2}
        \def\jobname{#2}
        \input{#2}
        \endinput
      }
    \else
      \def\childdoctmp
      {
        \childdocdisable
        \def\childdocname{#2}
        \childdoctrue
        \includeonly{#2}
        \def\childdocjob{#1}
        \def\jobname{#1}
        \input{#1}
        \endinput
      }
    \fi
    \expandafter
  \endgroup
  \childdoctmp
}
%    \end{macrocode}

% \macro{\childdocforwardprefix}
% The command |\childdocforwardprefix| redirects
% compilation to the main or a child file by means of a pattern.
% The prefix |#1| in the current filename is replaced by |#2|
% and the suffix of the current filename is kept
% (it is assumed that the filename does not contain the substring `|~~~|'
% which is used as a delimiter).
% Compilation is handed over to the new file by |\childdocforward|:
%    \begin{macrocode}
\newcommand{\childdocforwardprefix}[3][]
{
  \begingroup
    \def\childdocextract #2##1~~~{\def\childdoctmp{\childdocforward[#1]{#3##1}}}
    \expandafter\childdocextract\childdocname~~~
    \expandafter
  \endgroup
  \childdoctmp
}
%    \end{macrocode}

% \macro{\childdoc}
% The deprecated macro |\childdoc| is a legacy version of |\childdocmain|:
%    \begin{macrocode}
\newcommand{\childdoc}{\childdocmain}
%    \end{macrocode}

% \macro{\childdocredirect}
% The deprecated macro |\childdocredirect| is a legacy version
% of |\childdocforward| and |\childdocforwardprefix|:
%    \begin{macrocode}
\newcommand{\childdocredirect}[2][]
{
  \begingroup
    \if?#1?
      \def\childdoctmp{\childdocforward{#2}}
    \else
      \def\childdoctmp{\childdocforwardprefix{#1}{#2}}
    \fi
    \expandafter
  \endgroup
  \childdoctmp
}
%    \end{macrocode}

%\iffalse
%</package>
%\fi
%
\endinput
|\\
|\childdocforward[|\textit{main}|]{|\textit{dest}|}|\\
\end{tabular}
\end{center}
%
The argument \textit{dest} is the destination file
(without extension).
It should be the main file or one of the child files.
Note that further \textsf{childdoc} directives
such as |\childdocof| and |\childdocforward|
in the indicated file will be processed in this form.
The optional argument \textit{main}
passes on directly to the main file \textit{main}
while pretending to compile the child \textit{dest}.
This form behaves as if \textit{dest}
issues |\childdocof{|\textit{main}|}| right away,
and no further \textsf{childdoc} directives will be processed.

%%%%%%%%%%%%%%%%%%%%%%%%%%%%%%%%%%%%%%%%
\DescribeMacro{\...prefix}
In the alternative form |\childdocforwardprefix|,
%
\begin{center}
\begin{tabular}{l}
|% \iffalse
%
% childdoc.dtx Copyright (C) 2017-2018 Niklas Beisert
%
% This work may be distributed and/or modified under the
% conditions of the LaTeX Project Public License, either version 1.3
% of this license or (at your option) any later version.
% The latest version of this license is in
%   http://www.latex-project.org/lppl.txt
% and version 1.3 or later is part of all distributions of LaTeX
% version 2005/12/01 or later.
%
% This work has the LPPL maintenance status `maintained'.
%
% The Current Maintainer of this work is Niklas Beisert.
%
% This work consists of the files childdoc.dtx and childdoc.ins
% and the derived files childdoc.def and cdocsamp.tex with
% cdocsch1.tex, cdocsch2.tex, cdocsdrf.tex, cdocsfn1.tex, cdocsfn2.tex.
%
%<package>\ifdefined\childdocmain\endinput\fi
%<package>\ProvidesFile{childdoc.def}[2018/12/30 v2.0 child document driver]
%<samplemain>\ProvidesFile{cdocsamp.tex}[2018/12/30 v2.0 sample for childdoc]
%<*driver>
%\ProvidesFile{childdoc.drv}[2018/12/30 v2.0 childdoc reference manual file]
\PassOptionsToClass{10pt,a4paper}{article}
\documentclass{ltxdoc}

\usepackage[margin=35mm]{geometry}
\usepackage{hyperref}
\usepackage{hyperxmp}
\usepackage[usenames]{color}

\hypersetup{colorlinks=true}
\hypersetup{pdfstartview=FitH}
\hypersetup{pdfpagemode=UseNone}
\hypersetup{pdfsource={}}
\hypersetup{pdflang={en-UK}}
\hypersetup{pdfcopyright={Copyright 2017-2018 Niklas Beisert.
  This work may be distributed and/or modified under the
  conditions of the LaTeX Project Public License, either version 1.3
  of this license or (at your option) any later version.}}
\hypersetup{pdflicenseurl={http://www.latex-project.org/lppl.txt}}
\hypersetup{pdfcontactaddress={ETH Zurich, ITP, HIT K,
  Wolfgang-Pauli-Strasse 27}}
\hypersetup{pdfcontactpostcode={8093}}
\hypersetup{pdfcontactcity={Zurich}}
\hypersetup{pdfcontactcountry={Switzerland}}
\hypersetup{pdfcontactemail={nbeisert@itp.phys.ethz.ch}}
\hypersetup{pdfcontacturl={http://people.phys.ethz.ch/\xmptilde nbeisert/}}

\newcommand{\secref}[1]{\hyperref[#1]{section \ref*{#1}}}

\parskip1ex
\parindent0pt
\let\olditemize\itemize
\def\itemize{\olditemize\parskip0pt}

\begin{document}

\title{The \textsf{childdoc} Package}
\hypersetup{pdftitle={The childdoc Package}}
\author{Niklas Beisert\\[2ex]
  Institut f\"ur Theoretische Physik\\
  Eidgen\"ossische Technische Hochschule Z\"urich\\
  Wolfgang-Pauli-Strasse 27, 8093 Z\"urich, Switzerland\\[1ex]
  \href{mailto:nbeisert@itp.phys.ethz.ch}
  {\texttt{nbeisert@itp.phys.ethz.ch}}}
\hypersetup{pdfauthor={Niklas Beisert}}
\hypersetup{pdfsubject={Manual for the LaTeX2e Package childdoc}}
\date{30 December 2018, \textsf{v2.0}}
\maketitle

\begin{abstract}\noindent
\textsf{childdoc} is a \LaTeXe{} package
that enables the direct compilation
of document sections included by |\include|
to individual files.
\end{abstract}

\begingroup
\parskip0ex
\tableofcontents
\endgroup

%%%%%%%%%%%%%%%%%%%%%%%%%%%%%%%%%%%%%%%%%%%%%%%%%%%%%%%%%%%%%%%%%%%%%%%%%%%%%%%%
%%%%%%%%%%%%%%%%%%%%%%%%%%%%%%%%%%%%%%%%%%%%%%%%%%%%%%%%%%%%%%%%%%%%%%%%%%%%%%%%
\section{Introduction}

\LaTeX{} provides a mechanism to structure a large document (such as a book)
into a main file and several child files (containing the chapters)
using the |\include| command.
This mechanism is beneficial for documents
which span hundreds of pages in order to
make the source file(s) more manageable.
Moreover, compilation can be restricted to
selected child files by means of the |\includeonly| command.
The latter feature can be used to reduce the compilation time while editing
(this was significantly more useful in the earlier days of \LaTeX{})
or to generate a smaller document which is easier to navigate.
Another application of |\includeonly| is to generate
documents consisting of selected parts of the complete document.

However, there are a few drawbacks of the plain |\include| mechanism:
\begin{itemize}
\item
The child files cannot be compiled on their own,
they can only be compiled via the main file.
A naive editing environment
(such as a text editor with an option
to have the current file processed by \LaTeX)
may require one to switch to the main file before compiling;
attempting to compile the child file produces errors.
\item
The main file must be modified (each time)
to adjust the |\includeonly| command
to the present needs. This easily leaves the main file in a messy state.
\item
The generated document will always carry the filename
of the main document. This is inconvenient if
several child files are to be compiled and
to be kept for distribution.
\end{itemize}

The present package provides a simple interface
to make child files individually compilable by \LaTeX{}.
Compiling a child file then has the same effect as compiling
the main file with an |\includeonly| command
to select the appropriate child.
Moreover the generated document will carry the name of the child
rather than the main file.
This resolves all three above issues.

This feature is meant to make the editing of books,
thesis documents and lecture notes somewhat more convenient.
However, the package can also be used efficiently for
composing a series of documents (such as exercise sheets)
which are typically distributed individually.
It then assists the author in generating the individual documents
(potentially in different versions)
as well as a document containing the collected series.
Another application is in developing style files
or other kinds of included material
where compilation of the style file could redirect
to a sample or test file.

%%%%%%%%%%%%%%%%%%%%%%%%%%%%%%%%%%%%%%%%%%%%%%%%%%%%%%%%%%%%%%%%%%%%%%%%%%%%%%%%
%%%%%%%%%%%%%%%%%%%%%%%%%%%%%%%%%%%%%%%%%%%%%%%%%%%%%%%%%%%%%%%%%%%%%%%%%%%%%%%%
\section{Usage}

First of all, the package \textsf{childdoc} is \emph{not} a standard
\LaTeXe{} |.sty| style file! Therefore it needs to be invoked in
a non-standard way.

%%%%%%%%%%%%%%%%%%%%%%%%%%%%%%%%%%%%%%%%%%%%%%%%%%%%%%%%%%%%%%%%%%%%%%%%%%%%%%%%
\subsection{Included Files}
\label{sec:include}

%%%%%%%%%%%%%%%%%%%%%%%%%%%%%%%%%%%%%%%%
\DescribeMacro{\childdocmain}
To use the package, add the commands
\begin{center}
\begin{tabular}{l}
|\input{childdoc.def}|\\
|\childdocmain{}|\\
\end{tabular}
\end{center}
at the very top of the main \LaTeX{} file,
in particular \emph{before} the |\documentclass| statement!
The argument of |\childdocmain| should be left empty
(but it must be present).

%%%%%%%%%%%%%%%%%%%%%%%%%%%%%%%%%%%%%%%%
\DescribeMacro{\childdocof}
Furthermore, add the commands
\begin{center}
\begin{tabular}{l}
|\input{childdoc.def}|\\
|\childdocof{|\textit{main}|}|\\
\end{tabular}
\end{center}
at the top of every child file \textit{child}
which is included by |\include{|\textit{child}|}|
from within the main file
(or at least for those files to be compiled individually).
The argument \textit{main} must be the filename of the main file.

There are a couple of
considerations in setting up the main and child documents:

%%%%%%%%%%%%%%%%%%%%%%%%%%%%%%%%%%%%%%%%
\paragraph{Restrictions.}

Please note the following restrictions:
\begin{itemize}
\item
|\childdocmain| must be called with one argument \textit{main}
to ensure compatibility with earlier version of the package.
It must either be empty (|\childdocmain{}|)
or precisely match the filename of the main file in which it is specified.
See \secref{sec:detection} for further information.
\item
The filename \textit{main} must be specified without the |.tex| extension.
\item
The filename \textit{main} is case sensitive
(even in case-insensitive file systems)
due to internal string comparison.
\item
The argument \textit{main} should be fully expanded, it cannot be a macro.
\item
Subdirectories and special characters should be avoided in filenames.
\item
The command |\childdocmain{|\textit{main}|}| must be followed by a whitespace.
It should not be followed immediately by another command
or by a comment mark `|%|'.
This is because the \TeX{} parser reads the token immediately following
the argument of |\childdocmain| and puts it
at the beginning of every child section;
however, a white\-space is ignored.
\end{itemize}

%%%%%%%%%%%%%%%%%%%%%%%%%%%%%%%%%%%%%%%%
\paragraph{Content of Main File.}

It is advisable to place all content in the child files included by |\include|.
Any output contained in the main file will appear in all child documents
unless suppressed manually;
it cannot be suppressed automatically by the |\includeonly| directive
and thus should normally be avoided.
A method to include some content in the main file
by means of conditional processing is described in \secref{sec:conditional}.

%%%%%%%%%%%%%%%%%%%%%%%%%%%%%%%%%%%%%%%%
\paragraph{Page Numbering.}

When only a part of the document is compiled,
the appropriate numbering of pages
(as well as other status parameters)
is determined from the |.aux| files.
The latter contain information from previous passes.
However this information needs to propagate through
all intermediate child documents.
Therefore the page numbering in child documents may well
be inconsistent until the complete document is compiled at least once.

A useful (if unconventional) way to always ensure a consistent
page numbering is to restart the numbering in each child document
and denote the pages by `\textit{child}|.|\textit{page}'
where \textit{child} represents the chapter/section number of the child file.
This can be achieved by the command
|\numberwithin{page}{|\textit{child}|}|
of the \textsf{amsmath} package
where \textit{child} can be |chapter| or |section|
depending on the chosen structuring.
Alternatively, one can modify the macro |\thepage| appropriately
and reset the counter |page| at the start of each child file.

%%%%%%%%%%%%%%%%%%%%%%%%%%%%%%%%%%%%%%%%%%%%%%%%%%%%%%%%%%%%%%%%%%%%%%%%%%%%%%%%
\subsection{Conditional Processing}
\label{sec:conditional}

The package provides a mechanism to compile different versions
of a document. To customise the versions further some conditional processing
can come in handy to distinguish which version is being compiled.
The package provides two macros to describe the compilation context:

%%%%%%%%%%%%%%%%%%%%%%%%%%%%%%%%%%%%%%%%
\DescribeMacro{\ifchilddoc}
The conditional |\ifchilddoc| distinguishes between the compilation of
child documents and the main document:
%
\begin{center}
|\ifchilddoc |\textit{child-code}| |[|\||else |\textit{main-code}]| \||fi|
\end{center}

%%%%%%%%%%%%%%%%%%%%%%%%%%%%%%%%%%%%%%%%
\DescribeMacro{\childdocname}
\DescribeMacro{\childdocjob}
The macro |\childdocname| contains the filename (without extension)
of the main or child file being processed.
Note that |\childdocjob| will always contain the name of the main file.

%%%%%%%%%%%%%%%%%%%%%%%%%%%%%%%%%%%%%%%%
\paragraph{Title Page.}

Conditional processing can be used to include a title or banner page
in the main document when proper precautions are taken.
Importantly, the code in the main file should ensure that the page counter
(as well as other status parameters which are stored in the |.aux| files)
takes the same value after the conditional processing.
Otherwise the page numbers may take divergent values
depending on which part is compiled.

For example, a title page could be declared by:
%
\begin{center}
\begin{tabular}{l}
|\ifchilddoc\||else|\\
|\addtocounter{page}{-1}|\\
\textit{code for title page}\\
|\newpage|\\
|\||fi|
\end{tabular}
\end{center}
%
A banner page for the child documents can be generated by:
%
\begin{center}
\begin{tabular}{l}
|\ifchilddoc|\\
|\addtocounter{page}{-1}|\\
\textit{code for banner page}\\
|\newpage|\\
|\||fi|
\end{tabular}
\end{center}
%
Here one could write a message such as:
\begin{center}
|This is the part \childdocname{} of \childdocjob{}.|
\end{center}

%%%%%%%%%%%%%%%%%%%%%%%%%%%%%%%%%%%%%%%%%%%%%%%%%%%%%%%%%%%%%%%%%%%%%%%%%%%%%%%%
\subsection{Flags}
\label{sec:flags}

The package makes it easy to generate different versions
of the main or child documents.
To this end compilation flags can be defined
and assigned different default values.
They will be particularly useful in conjunction
with the forwarding mechanism described in \secref{sec:forward}.

For example, it may be useful to have a flag |\version|
which can be set to |draft| or |final|.
The document source will contain some conditional code
depending on the value of |\version|.
Suppose further, the flag should default to |final| for the main file
and to |draft| for child files
which is a natural assignment for editing the document.
This is achieved by placing the following code
in the preamble of the main document
(below the |\childdocmain| directive):
%
\begin{center}
\begin{tabular}{l}
|\ifchilddoc|\\
|\providecommand{\version}{draft}|\\
|\||else|\\
|\providecommand{\version}{final}|\\
|\||fi|
\end{tabular}
\end{center}
%
The definition by |\providecommand| makes sure
that previous definitions are not overwritten.
Further statements |\providecommand{\version}{...}|
can thus be added before the above code to override it.

For the main file, one might add a line
(between |\childdocmain| and the above block)
%
\begin{center}
|%\ifchilddoc\||else\providecommand{\version}{draft}\||fi|
\end{center}
%
which can be uncommented to produce a draft version.
Likewise one can add a line to the very top of a child file
(above the |\childdocof{|\textit{main}|}| directive)
%
\begin{center}
|%\providecommand{\version}{final}|
\end{center}
%
which can be uncommented to produce the final version of this child document.

%%%%%%%%%%%%%%%%%%%%%%%%%%%%%%%%%%%%%%%%%%%%%%%%%%%%%%%%%%%%%%%%%%%%%%%%%%%%%%%%
\subsection{Forwarding}
\label{sec:forward}

Different versions of the main or child documents
using compilation flags as described in \secref{sec:flags}
can be (permanently) stored in different files
for convenient compilation, viewing and distribution.
To this end, the package defines a command
to pass on compilation to a different file:

%%%%%%%%%%%%%%%%%%%%%%%%%%%%%%%%%%%%%%%%
\DescribeMacro{\childdocforward}
The command |\childdocforward| redirects processing to
another source file:
%
\begin{center}
\begin{tabular}{l}
|\input{childdoc.def}|\\
|\childdocforward[|\textit{main}|]{|\textit{dest}|}|\\
\end{tabular}
\end{center}
%
The argument \textit{dest} is the destination file
(without extension).
It should be the main file or one of the child files.
Note that further \textsf{childdoc} directives
such as |\childdocof| and |\childdocforward|
in the indicated file will be processed in this form.
The optional argument \textit{main}
passes on directly to the main file \textit{main}
while pretending to compile the child \textit{dest}.
This form behaves as if \textit{dest}
issues |\childdocof{|\textit{main}|}| right away,
and no further \textsf{childdoc} directives will be processed.

%%%%%%%%%%%%%%%%%%%%%%%%%%%%%%%%%%%%%%%%
\DescribeMacro{\...prefix}
In the alternative form |\childdocforwardprefix|,
%
\begin{center}
\begin{tabular}{l}
|\input{childdoc.def}|\\
|\childdocforwardprefix[|\textit{main}|]{|\textit{prefix}|}{|\textit{dest}|}|
\end{tabular}
\end{center}
%
the destination file is determined by a pattern
depending on the current file:
To make this work, the current file must be called
`{\textit{prefix}\hspace{0.2em}\textit{suffix}}'
with \textit{prefix} matching precisely the argument.
Processing is then passed on to the file
`{\textit{dest}\hspace{0.2em}\textit{suffix}}'.
Surely, the same effect is achieved by
directly specifying the
argument `{\textit{dest}\hspace{0.2em}\textit{suffix}}'
in the first form.
However, that requires to set up a different file
for each child. With the alternative form of the command
all these files can have exactly the same content
which simplifies setting them up and maintaining them.

For example, the following file |draft.tex|
with a compilation flag |\version| as described in \secref{sec:flags}
compiles the main document as a draft:
%
\begin{center}
\begin{tabular}{l}
|\def\version{draft}|\\
|\input{childdoc.def}|\\
|\childdocforward{|\textit{main}|}|
\end{tabular}
\end{center}
%
Likewise, the following files |final|\textit{nn}|.tex|
compile the final version of the child document
|child|\textit{nn}|.tex|:
%
\begin{center}
\begin{tabular}{l}
|\def\version{final}|\\
|\input{childdoc.def}|\\
|\childdocforwardprefix{final}{child}|
\end{tabular}
\end{center}
%

Note that when several versions of a main file and/or of each child file
are to be generated, it may be convenient to set up a |Makefile| or
shell script to automatise the process.

%%%%%%%%%%%%%%%%%%%%%%%%%%%%%%%%%%%%%%%%%%%%%%%%%%%%%%%%%%%%%%%%%%%%%%%%%%%%%%%%
\subsection{Command Line Processing}
\label{sec:commandline}

The effect of redirection files can also be achieved by invoking
the \LaTeX{} compiler with a more elaborate command line.
Most conveniently this should be done as part
of a shell script or a |Makefile|.

When using \textsf{childdoc} in the main file, the following
command lines effectively perform a redirection
(note that depending on the shell being used,
backslashes may have to be doubled: `|\|' $\to$ `|\\|'):
%
\begin{center}
|... -jobname "|\textit{target}|" |\\|"|[\textit{flags}]%
|\input{childdoc.def}\childdocforward[|\textit{main}|]{|\textit{dest}|}"|
\end{center}
%
Here \textit{target} is the name of the output file,
\textit{main} is the name of the main file
and \textit{dest} is the name of the main or child file to be processed
(all filenames without extensions).
The optional argument \textit{main} can be omitted
if \textit{main} matches \textit{dest}.
Optionally, compilation \textit{flags} can be defined via |\def| commands.
This command line makes the \TeX{} engine believe
it is compiling the file \textit{target}
whose content is specified as the latter parameter.
The provided code then forwards the processing to
\textit{main} or \textit{dest} as described in \secref{sec:forward}.

%%%%%%%%%%%%%%%%%%%%%%%%%%%%%%%%%%%%%%%%%%%%%%%%%%%%%%%%%%%%%%%%%%%%%%%%%%%%%%%%
\subsection{Include by Input}
\label{sec:input}

Including child documents by |\include| has some restrictions by design.
Most notably, the content of a child document always occupies
its own set of pages; pages cannot be shared between child documents.
Usually, this behaviour makes perfect sense
because each child document contain an essential part of the document.
However, in some situations it may be desirable to compose
a document from a collection of parts
without having mandatory page breaks between then.
For this case, the package
provides a mechanism to include parts
by |\input| which can also be processed individually.
However, by construction this mechanism
requires manual handling of the content to be output.

%%%%%%%%%%%%%%%%%%%%%%%%%%%%%%%%%%%%%%%%
\DescribeMacro{\ifchilddocmanual}
The main file should be prepared as usual, see \secref{sec:include}.
However, the document body must make a distinction
between processing of an individual part and of the main document, e.g.:
%
\begin{center}
\begin{tabular}{l}
|\ifchilddocmanual|\\
|\input{\childdocname}|\\
|\||else|\\
\textit{document body with }|\input{|\textit{part}|}|\\
|\||fi|
\end{tabular}
\end{center}
%
The conditional |\ifchilddocmanual| is true whenever
a part to be included by |\input| is being compiled,
and the name of the part is stored in |\childdocname|.

%%%%%%%%%%%%%%%%%%%%%%%%%%%%%%%%%%%%%%%%
\DescribeMacro{\childdocby}
Each part to be included by |\input| should start with:
%
\begin{center}
\begin{tabular}{l}
|\input{childdoc.def}|\\
|\childdocby{|\textit{main}|}|\\
\end{tabular}
\end{center}
%
The directive |\childdocby| is similar to |\childdocof|
described in \secref{sec:include},
but the subsequent selection of content must be done manually.
To that end, both |\ifchilddoc| and |\ifchilddocmanual|
will be true upon processing of a part,
and the name of the part is stored in |\childdocname|.
Note that |\jobname| will be set to the filename of the current part
so that each part receives an individual |.aux| file
that does not interfere with the |.aux| file(s) of the main document.
This behaviour can be altered by the alternative form
|\childdocby[*]{|\textit{main}|}| (with a non-empty optional argument)
which uses the |.aux| file of the main document
by setting |\jobname| to \textit{main}.

%%%%%%%%%%%%%%%%%%%%%%%%%%%%%%%%%%%%%%%%%%%%%%%%%%%%%%%%%%%%%%%%%%%%%%%%%%%%%%%%
\subsection{Driver Development}
\label{sec:driver}

The \textsf{childdoc} mechanism can also be use for the development
of definition files such as \LaTeX{} styles or classes.
This case differs from the above setup with multiple parts
included by |\include| in that no |\includeonly| should be invoked.
This can be achieved by starting the include file
(before |\ProvidesPackage|) with:
%
\begin{center}
\begin{tabular}{l}
|\input{childdoc.def}|\\
|\childdocforward{|\textit{main}|}|\\
\end{tabular}
\end{center}
%
or alternatively with:
%
\begin{center}
\begin{tabular}{l}
|\input{childdoc.def}|\\
|\childdocby{|\textit{main}|}|\\
\end{tabular}
\end{center}
%
Both forms have slightly different effects as described above.
The main file is prepared as usual, see \secref{sec:include}.

%%%%%%%%%%%%%%%%%%%%%%%%%%%%%%%%%%%%%%%%%%%%%%%%%%%%%%%%%%%%%%%%%%%%%%%%%%%%%%%%
\subsection{Legacy Detection}
\label{sec:detection}

The directive |\childdocmain| in the main file can detect
whether the complete document or merely a child is to be compiled
even without using the directive |\childdocof|.
This method is deprecated because it is less robust
and there is no compelling reason to use it;
it is merely provided for backward compatibility
and it may be removed in future versions.

If the detection mechanism is to be used,
it is mandatory to correctly specify
the filename of the main file as the argument of |\childdocmain|:
%
\begin{center}
\begin{tabular}{l}
|\input{childdoc.def}|\\
|\childdocmain{|\textit{main}|}|\\
\end{tabular}
\end{center}
%
If |\jobname| does not match the argument \textit{main} of |\childdocmain|,
it is assumed that |\jobname| points to the child file to be compiled.
When using |\childdocmain| with the main file specified as argument,
it suffices to start a child file
with just |\input{|\textit{main}|}|
without loading of the package and using |\childdocof|.
If instead all processing is done
with the appropriate \textsf{childdoc} directives,
the argument of \textit{main} of |\childdocmain| can be empty.

An alternative version of the command line processing described
in \secref{sec:commandline} using the detection mechanism reads:
%
\begin{center}
|... -jobname "|\textit{target}|" "|[\textit{flags}]%
[|\def\jobname{|\textit{dest}|}|]|\input{|\textit{main}|}"|
\end{center}

%%%%%%%%%%%%%%%%%%%%%%%%%%%%%%%%%%%%%%%%%%%%%%%%%%%%%%%%%%%%%%%%%%%%%%%%%%%%%%%%
\subsection{Manual Code}
\label{sec:manual}

In case one cannot be certain whether the definitions file |childdoc.def|
is installed on the target \TeX{} distribution
and one prefers not to ship it,
it is conceivable to paste a few relevant commands into the sources.

To that end, drop all statements |\input{childdoc.def}|
and perform the replacements as outlined below.
Instead of |\childdocmain{|\textit{main}|}| add the following code
to the top of the main file:
%
\begin{center}
\begin{tabular}{l}
|\||ifdefined\childdocname\endinput\||fi\newif\ifchilddoc|\\
|\edef\childdocname{\scantokens\expandafter{\jobname\noexpand}}|\\
|\def\childdocmain{|\textit{main}|}\||ifx\childdocmain\childdocname\||else|\\
|\childdoctrue\includeonly{\childdocname}\let\jobname\childdocmain\||fi|\\
\end{tabular}
\end{center}
%
Instead of |\childdocof{|\textit{main}|}| just include the main file
at the top of each child file:
%
\begin{center}
|\input{|\textit{main}|}|
\end{center}
%
A simple redirection |\childdocforward{|\textit{dest}|}| is achieved by:
%
\begin{center}
|\def\jobname{|\textit{dest}|}\input{\jobname}|
\end{center}
%
The redirection with prefix
|\childdocforwardprefix[|\textit{prefix}|]{|\textit{dest}|}|
is accomplished by:
%
\begin{center}
\begin{tabular}{l}
|{\edef\jobname{\scantokens\expandafter{\jobname\noexpand}}|\\
|\def\redirectjob |\textit{prefix}|#1~~~{\gdef\jobname{|\textit{dest}|#1}}|\\
|\expandafter\redirectjob\jobname~~~}\input{\jobname}|
\end{tabular}
\end{center}

In an alternative approach,
child documents can be compiled by a specific command line
without additional code or specific definitions:
%
\begin{center}
|... -jobname "|\textit{target}|" "|[\textit{flags}]%
|\includeonly{|\textit{dest}|}\input{|\textit{main}|}"|
\end{center}
%

%%%%%%%%%%%%%%%%%%%%%%%%%%%%%%%%%%%%%%%%%%%%%%%%%%%%%%%%%%%%%%%%%%%%%%%%%%%%%%%%
%%%%%%%%%%%%%%%%%%%%%%%%%%%%%%%%%%%%%%%%%%%%%%%%%%%%%%%%%%%%%%%%%%%%%%%%%%%%%%%%
\section{Information}

%%%%%%%%%%%%%%%%%%%%%%%%%%%%%%%%%%%%%%%%%%%%%%%%%%%%%%%%%%%%%%%%%%%%%%%%%%%%%%%%
\subsection{Copyright}

Copyright \copyright{} 2017--2018 Niklas Beisert

This work may be distributed and/or modified under the
conditions of the \LaTeX{} Project Public License, either version 1.3
of this license or (at your option) any later version.
The latest version of this license is in
  \url{http://www.latex-project.org/lppl.txt}
and version 1.3 or later is part of all distributions of \LaTeX{}
version 2005/12/01 or later.

This work has the LPPL maintenance status `maintained'.

The Current Maintainer of this work is Niklas Beisert.

This work consists of the files |README.txt|, |childdoc.ins| and |childdoc.dtx|
as well as the derived files |childdoc.def|, |cdocsamp.tex|
with |cdocsch1.tex|, |cdocsch2.tex|, |cdocspt3.tex|, |cdocspt4.tex|,
|cdocsdrf.tex|, |cdocsfn1.tex|, |cdocsfn2.tex|
as well as |childdoc.pdf|.

%%%%%%%%%%%%%%%%%%%%%%%%%%%%%%%%%%%%%%%%%%%%%%%%%%%%%%%%%%%%%%%%%%%%%%%%%%%%%%%%
\subsection{Files and Installation}

The package consists of the files:
%
\begin{center}
\begin{tabular}{ll}
    |README.txt|   & readme file \\
    |childdoc.ins| & installation file \\
    |childdoc.dtx| & source file \\
    |childdoc.def| & definition file \\
    |cdocsamp.tex| & sample main file \\
    |cdocsch1.tex| & sample include file \\
    |cdocsch2.tex| & sample include file \\
    |cdocspt3.tex| & sample part file \\
    |cdocspt4.tex| & sample part file \\
    |cdocsdrf.tex| & sample redirection file \\
    |cdocsfn1.tex| & sample redirection file \\
    |cdocsfn2.tex| & sample redirection file \\
    |childdoc.pdf| & manual
\end{tabular}
\end{center}
%
The distribution consists of the files
|README.txt|, |childdoc.ins| and |childdoc.dtx|.
%
\begin{itemize}
\item
Run (pdf)\LaTeX{} on |childdoc.dtx|
to compile the manual |childdoc.pdf| (this file).
\item
Run \LaTeX{} on |childdoc.ins| to create the definitions file |childdoc.def|
and the sample |cdocsamp.tex| with include files
|cdocsch1.tex|, |cdocsch2.tex|, |cdocspt3.tex|, |cdocspt4.tex|,
|cdocsdrf.tex|, |cdocsfn1.tex|, |cdocsfn2.tex|.
Then copy the file |childdoc.def| to an appropriate directory of your \LaTeX{}
distribution, e.g.\ \textit{texmf-root}|/tex/latex/childdoc|.
\end{itemize}

%%%%%%%%%%%%%%%%%%%%%%%%%%%%%%%%%%%%%%%%%%%%%%%%%%%%%%%%%%%%%%%%%%%%%%%%%%%%%%%%
\subsection{Related CTAN Packages}

There are several other packages which offer a similar functionality:
%
\begin{itemize}
\item
The packages
\href{http://ctan.org/pkg/docmute}{\textsf{docmute}},
\href{http://ctan.org/pkg/includex}{\textsf{includex}} and
\href{http://ctan.org/pkg/standalone}{\textsf{standalone}}
provide commands to include only the document body of
a child file thus allowing both files to be compiled individually.
\item
The packages \href{http://ctan.org/pkg/subdocs}{\textsf{subdocs}}
and \href{http://ctan.org/pkg/subfiles}{\textsf{subfiles}}
provide structures in which the main and child documents can be
encapsulated and allowing them to be compiled individually.
The inclusion mechanism is different from the conventional |\include|.
\item
The package \href{http://ctan.org/pkg/combine}{\textsf{combine}}
is an elaborate solution to combine several documents into one.
\end{itemize}
%
See also the CTAN topic \href{http://ctan.org/topic/subdocs}{\textsf{subdocs}}
for further related packages.
The present package differs from the above solutions in that
a document structure constructed with the conventional |\include| mechanism
just needs two extra commands at the top of every file
such that all constituent files can be compiled individually.

%%%%%%%%%%%%%%%%%%%%%%%%%%%%%%%%%%%%%%%%%%%%%%%%%%%%%%%%%%%%%%%%%%%%%%%%%%%%%%%%
%\subsection{Feature Suggestions}
%
%The following is a list of features which may be useful for future
%versions of this package:
%%
%\begin{itemize}
%\item
%\ldots
%\end{itemize}

%%%%%%%%%%%%%%%%%%%%%%%%%%%%%%%%%%%%%%%%%%%%%%%%%%%%%%%%%%%%%%%%%%%%%%%%%%%%%%%%
\subsection{Revision History}

%%%%%%%%%%%%%%%%%%%%%%%%%%%%%%%%%%%%%%%%
\paragraph{v2.0:} 2018/12/30

\begin{itemize}
\item
immediate forward processing
\item
added |\childdocby| mechanism
\item
manual restructured
\end{itemize}

%%%%%%%%%%%%%%%%%%%%%%%%%%%%%%%%%%%%%%%%
\paragraph{v1.6:} 2018/01/17

\begin{itemize}
\item
application for development of include files
\item
corrections to manual
\end{itemize}

%%%%%%%%%%%%%%%%%%%%%%%%%%%%%%%%%%%%%%%%
\paragraph{v1.5:} 2017/05/21

\begin{itemize}
\item
more complete structuring introduced
\item
|\childdocof| introduced
\item
|\childdoc| renamed to |\childdocmain|
\item
|\childredirect| renamed to |\childdocforward| and |\childdocforwardprefix|
and functionality expanded
\end{itemize}

%%%%%%%%%%%%%%%%%%%%%%%%%%%%%%%%%%%%%%%%
\paragraph{v1.0:} 2017/04/27

\begin{itemize}
\item
manual and install package
\item
first version published on CTAN
\end{itemize}

%%%%%%%%%%%%%%%%%%%%%%%%%%%%%%%%%%%%%%%%
\paragraph{v0.6:} 2017/04/26

\begin{itemize}
\item
redirection mechanism added
\end{itemize}

%%%%%%%%%%%%%%%%%%%%%%%%%%%%%%%%%%%%%%%%
\paragraph{v0.5:} 2017/04/26

\begin{itemize}
\item
functionality in definition file
\end{itemize}


%%%%%%%%%%%%%%%%%%%%%%%%%%%%%%%%%%%%%%%%%%%%%%%%%%%%%%%%%%%%%%%%%%%%%%%%%%%%%%%%
%%%%%%%%%%%%%%%%%%%%%%%%%%%%%%%%%%%%%%%%%%%%%%%%%%%%%%%%%%%%%%%%%%%%%%%%%%%%%%%%
%%%%%%%%%%%%%%%%%%%%%%%%%%%%%%%%%%%%%%%%%%%%%%%%%%%%%%%%%%%%%%%%%%%%%%%%%%%%%%%%
\appendix

\settowidth\MacroIndent{\rmfamily\scriptsize 000\ }

 \DocInput{childdoc.dtx}

\end{document}
%</driver>
% \fi
%
% %%%%%%%%%%%%%%%%%%%%%%%%%%%%%%%%%%%%%%%%%%%%%%%%%%%%%%%%%%%%%%%%%%%%%%%%%%%%%%
% %%%%%%%%%%%%%%%%%%%%%%%%%%%%%%%%%%%%%%%%%%%%%%%%%%%%%%%%%%%%%%%%%%%%%%%%%%%%%%
% \section{Sample}
%\iffalse
%<*samplemain>
%\fi
%
% The following presents a sample document
% with two chapters, two parts, a title page,
% a compile flag as well as three forwarding files to set the flag.
% It consists of eight |.tex| files:
% \begin{center}
% \begin{tabular}{ll}
% |cdocsamp.tex|&main file\\
% |cdocsch1.tex|&include file for chapter 1\\
% |cdocsch2.tex|&include file for chapter 2\\
% |cdocspt3.tex|&include file for part 3\\
% |cdocspt4.tex|&include file for part 4\\
% |cdocsdrf.tex|&forwarding file for main file in draft mode\\
% |cdocsfi1.tex|&forwarding file for final version of chapter 1\\
% |cdocsfi2.tex|&forwarding file for final version of chapter 2\\
% \end{tabular}
% \end{center}
% Each of the eight files can be compiled directly by the \LaTeX{} compiler.
%
% %%%%%%%%%%%%%%%%%%%%%%%%%%%%%%%%%%%%%%
% \paragraph{Main File.}
%
% The main file is called |cdocsamp.tex|.
%
% Load the \textsf{childdoc} definitions and
% declare the filename for the main document:
%    \begin{macrocode}
\input{childdoc.def}
\childdocmain{}
%    \end{macrocode}

% Optional override for |\version| flag:
%    \begin{macrocode}
%%\ifchilddoc\else\providecommand{\version}{draft}\fi
%    \end{macrocode}

% Define the default values for the |\version| flag
% (|final| for the main file and |draft| for childs):
%    \begin{macrocode}
\ifchilddoc
\providecommand{\version}{draft}
\else
\providecommand{\version}{final}
\fi
%    \end{macrocode}

% Load the standard document class:
%    \begin{macrocode}
\documentclass[12pt]{article}
%    \end{macrocode}

% Start the document body:
%    \begin{macrocode}
\begin{document}
%    \end{macrocode}

% Declare a title page.
% Print title, part of document being processed and version flag:
%    \begin{macrocode}
\addtocounter{page}{-1}
\begin{center}
{\LARGE\bfseries{}childdoc example\par}
\vspace{1cm}
\ifchilddoc
\ifchilddocmanual part\else chapter\fi:
`\childdocname' of `\childdocjob'\par
\else
main document: `\childdocjob'\par
\fi
version: \version\par
\end{center}
\newpage
%    \end{macrocode}

% Manually include selected file,
% otherwise process as usual:
%    \begin{macrocode}
\ifchilddocmanual
\section*{part `\childdocname'}
\input{\childdocname}
\else
%    \end{macrocode}

% Include the two chapters:
%    \begin{macrocode}
\include{cdocsch1}
\include{cdocsch2}
%    \end{macrocode}

% Include the two parts unless only chapters should be displayed:
%    \begin{macrocode}
\ifchilddoc\else
\section{part three}
\input{cdocspt3}
\section{part four}
\input{cdocspt4}
\fi
%    \end{macrocode}

% Process as usual until here:
%    \begin{macrocode}
\fi
%    \end{macrocode}

% End of document body:
%    \begin{macrocode}
\end{document}
%    \end{macrocode}
%\iffalse
%</samplemain>
%\fi
%
% %%%%%%%%%%%%%%%%%%%%%%%%%%%%%%%%%%%%%%
% \paragraph{Chapter Include Files.}
%
% The include files are called |cdocsch1.tex| and |cdocsch2.tex|.
%
%\iffalse
%<*samplechap1|samplechap2>
%\fi

% Optional override for |\version| flag:
%    \begin{macrocode}
%%\providecommand{\version}{final}
%    \end{macrocode}

% Include the main document:
%    \begin{macrocode}
\input{childdoc.def}
\childdocof{cdocsamp}
%    \end{macrocode}

%\iffalse
%</samplechap1|samplechap2>
%\fi
%
%\iffalse
%<*samplechap1>
%\fi
% Some text for chapter 1:
%    \begin{macrocode}
\section{one}
some text in chapter one
%    \end{macrocode}

%\iffalse
%</samplechap1>
%\fi
% Some text for chapter 2:
%\iffalse
%<*samplechap2>
%\fi
%    \begin{macrocode}
\section{two}
more text in chapter two
%    \end{macrocode}

%\iffalse
%</samplechap2>
%\fi
%
% %%%%%%%%%%%%%%%%%%%%%%%%%%%%%%%%%%%%%%
% \paragraph{Part Include Files.}
%
% The include files are called |cdocspt3.tex| and |cdocspt4.tex|.
%
%\iffalse
%<*samplepart3|samplepart4>
%\fi

% Optional override for |\version| flag:
%    \begin{macrocode}
%%\providecommand{\version}{final}
%    \end{macrocode}

% Include the main document:
%    \begin{macrocode}
\input{childdoc.def}
\childdocby{cdocsamp}
%    \end{macrocode}

%\iffalse
%</samplepart3|samplepart4>
%\fi
%
%\iffalse
%<*samplepart3>
%\fi
% Some text for part 3:
%    \begin{macrocode}
some text in part three
%    \end{macrocode}

%\iffalse
%</samplepart3>
%\fi
% Some text for part 4:
%\iffalse
%<*samplepart4>
%\fi
%    \begin{macrocode}
more text in part four
%    \end{macrocode}

%\iffalse
%</samplepart4>
%\fi
%
% %%%%%%%%%%%%%%%%%%%%%%%%%%%%%%%%%%%%%%
% \paragraph{Forwarding for a Complete Draft.}
%
% The following forwarding file |cdocsdrf.tex|
% compiles the main document in draft mode:
%\iffalse
%<*sampledraft>
%\fi
%    \begin{macrocode}
\def\version{draft}
\input{childdoc.def}
\childdocforward{cdocsamp}
%    \end{macrocode}

%\iffalse
%</sampledraft>
%\fi
%
% %%%%%%%%%%%%%%%%%%%%%%%%%%%%%%%%%%%%%%
% \paragraph{Forwarding for Final Version of the Chapters.}
%
% The following forwarding files |cdocsfn1.tex| and |cdocsfn2.tex|
% (with identical content)
% compile the final versions of the child documents
% |cdocsch1.tex| and |cdocsch2.tex|, respectively:
%\iffalse
%<*samplefinal>
%\fi
%    \begin{macrocode}
\def\version{final}
\input{childdoc.def}
\childdocforwardprefix[cdocsamp]{cdocsfn}{cdocsch}
%    \end{macrocode}

%\iffalse
%</samplefinal>
%\fi
%
% %%%%%%%%%%%%%%%%%%%%%%%%%%%%%%%%%%%%%%
% \paragraph{Command Line Processing.}
%
% The following three command lines generate the output files
% |cdocscld|, |cdocscl1| and |cdocscl2|
% which should be identical to
% |cdocsdrf|, |cdocsch1| and |cdocsfn2|, respectively:
% \begin{center}
% \begin{tabular}{l}
% |latex -jobname cdocscld \|\\
% |  "\def\version{draft}\input{childdoc.def}\childdocforward{cdocsamp}"|\\
% |latex -jobname cdocscl1 \|\\
% |  "\input{childdoc.def}\childdocforward[cdocsamp]{cdocsch1}"|\\
% |latex -jobname cdocscl2 \|\\
% |  "\def\version{final}\input{childdoc.def}\childdocforward{cdocsch2}"|
% \end{tabular}
% \end{center}
% Note that the trailing backslash on each first line
% merely continues the input to the second line
% (for convenient cut ant paste).
% Furthermore, the command |latex| can be replaced by any
% of its alternative versions such as |pdflatex|.
%
% %%%%%%%%%%%%%%%%%%%%%%%%%%%%%%%%%%%%%%%%%%%%%%%%%%%%%%%%%%%%%%%%%%%%%%%%%%%%%%
% %%%%%%%%%%%%%%%%%%%%%%%%%%%%%%%%%%%%%%%%%%%%%%%%%%%%%%%%%%%%%%%%%%%%%%%%%%%%%%
% \section{Implementation}
%\iffalse
%<*package>
%\fi
%
% This section describes the definitions file |childdoc.def|.

% The definitions cannot be loaded using |\usepackage| or |\RequirePackage|
% which has a mechanism to prevent loading a style file more than once.
% When loading the definitions by means of |\input|
% multiple instances have to be prevented manually:
%\iffalse
%This code needs to be before the `\ProvidesFile' directive
%which is defined at the beginning of this file.
%Therefore it is also placed there and commented out here.
%</package>
%<*discard>
%\fi
%    \begin{macrocode}
\ifdefined\childdocmain\endinput\fi
%    \end{macrocode}
%\iffalse
%</discard>
%<*package>
%\fi
%
% \macro{\ifchilddoc}
% \macro{\ifchilddocmanual}
% The conditional |\ifchilddoc| tells whether a
% child (true) or main (false) document is being compiled.
% The conditional |\ifchilddocmanual| tells whether
% the |\includeonly| mechanism is used (false) or
% the selection of child files must be performed manually (true).
% The definitions initialise to false:
%    \begin{macrocode}
\newif\ifchilddoc
\newif\ifchilddocmanual
%    \end{macrocode}

% \macro{\childdocname}
% \macro{\childdocjob}
% The macro |\childdocname| stores the name of the main document
% to be compiled. The macro |\childdocjob| stores the name of
% the document on which the \LaTeX{} compiler was originally invoked.
% The content of |\jobname| cannot be compared
% to filenames specified in the source due to different catcodes.
% The following code rescans |\jobname|, stores the result
% in |\childdocname| and saves a copy in |\childdocjob|:
%    \begin{macrocode}
\edef\childdocname{\scantokens\expandafter{\jobname\noexpand}}
\let\childdocjob\childdocname
%    \end{macrocode}

% \macro{\childdocdisable}
% The macro |\childdocdisable| prevents the main file
% from being processed more than once.
% At this stage, the main document command |\childdocmain|
% is assumed to be called once again where it should do nothing.
% Any subsequent call to it should prevent
% a secondary processing of the main document
% It overwrites the forwarding commands
% |\childdocof| and |\childdocforward|
% with empty macros to prevent further inclusions of the main document:
%    \begin{macrocode}
\newcommand{\childdocdisable}
{
  \renewcommand{\childdocmain}[1]{\renewcommand{\childdocmain}[1]{\endinput}}
  \renewcommand{\childdocof}[1]{}
  \renewcommand{\childdocby}[2][]{}
  \renewcommand{\childdocforward}[2][]{}
  \renewcommand{\childdocdisable}{}
}
%    \end{macrocode}

% \macro{\childdocmain}
% The macro |\childdocmain| is to be called at the top of the main file
% with nothing or the main filename (without extension) as argument.
% First, it breaks loops.
% If the argument is not empty and does not match |\childdocname|
% (which is set by the first inclusion of |childdoc.def|),
% |\ifchilddoc| is set to true, |\includeonly| is applied to the child file
% and |\jobname| is set to the main file
% (for proper handling of |.aux| files):
%    \begin{macrocode}
\newcommand{\childdocmain}[1]
{
  \childdocdisable\childdocmain{}
  \if?#1?\else
    \begingroup
      \def\childdoctmp{#1}
      \ifx\childdoctmp\childdocname
        \def\childdoctmp{}
      \else
        \def\childdoctmp
        {
          \childdoctrue
          \includeonly{\childdocname}
          \def\childdocjob{#1}
          \def\jobname{#1}
        }
      \fi
      \expandafter
    \endgroup
    \childdoctmp
  \fi
}
%    \end{macrocode}

% \macro{\childdocof}
% The command |\childdocof| redirects
% compilation to the main file |#1|.
%    \begin{macrocode}
\newcommand{\childdocof}[1]
{
  \childdocdisable
  \childdoctrue
  \includeonly{\childdocname}
  \def\jobname{#1}
  \def\childdocjob{#1}
  \input{#1}
}
%    \end{macrocode}

% \macro{\childdocby}
% The command |\childdocby| ....
%    \begin{macrocode}
\newcommand{\childdocby}[2][]
{
  \childdocdisable
  \childdoctrue
  \childdocmanualtrue
  \if?#1?\else
    \def\jobname{#2}
  \fi
  \def\childdocjob{#2}
  \input{#2}
  \endinput
}
%    \end{macrocode}

% \macro{\childdocforward}
% The command |\childdocforward| redirects
% compilation to the main file or
% (if the optional argument is given) a child file.
% Parameters are set as if the main file
% or a child file starting with |\childdocof| was compiled.
% Then compilation is handed over to the main file:
%    \begin{macrocode}
\newcommand{\childdocforward}[2][]
{
  \begingroup
    \if?#1?
      \def\childdoctmp
      {
        \def\childdocname{#2}
        \def\childdocjob{#2}
        \def\jobname{#2}
        \input{#2}
        \endinput
      }
    \else
      \def\childdoctmp
      {
        \childdocdisable
        \def\childdocname{#2}
        \childdoctrue
        \includeonly{#2}
        \def\childdocjob{#1}
        \def\jobname{#1}
        \input{#1}
        \endinput
      }
    \fi
    \expandafter
  \endgroup
  \childdoctmp
}
%    \end{macrocode}

% \macro{\childdocforwardprefix}
% The command |\childdocforwardprefix| redirects
% compilation to the main or a child file by means of a pattern.
% The prefix |#1| in the current filename is replaced by |#2|
% and the suffix of the current filename is kept
% (it is assumed that the filename does not contain the substring `|~~~|'
% which is used as a delimiter).
% Compilation is handed over to the new file by |\childdocforward|:
%    \begin{macrocode}
\newcommand{\childdocforwardprefix}[3][]
{
  \begingroup
    \def\childdocextract #2##1~~~{\def\childdoctmp{\childdocforward[#1]{#3##1}}}
    \expandafter\childdocextract\childdocname~~~
    \expandafter
  \endgroup
  \childdoctmp
}
%    \end{macrocode}

% \macro{\childdoc}
% The deprecated macro |\childdoc| is a legacy version of |\childdocmain|:
%    \begin{macrocode}
\newcommand{\childdoc}{\childdocmain}
%    \end{macrocode}

% \macro{\childdocredirect}
% The deprecated macro |\childdocredirect| is a legacy version
% of |\childdocforward| and |\childdocforwardprefix|:
%    \begin{macrocode}
\newcommand{\childdocredirect}[2][]
{
  \begingroup
    \if?#1?
      \def\childdoctmp{\childdocforward{#2}}
    \else
      \def\childdoctmp{\childdocforwardprefix{#1}{#2}}
    \fi
    \expandafter
  \endgroup
  \childdoctmp
}
%    \end{macrocode}

%\iffalse
%</package>
%\fi
%
\endinput
|\\
|\childdocforwardprefix[|\textit{main}|]{|\textit{prefix}|}{|\textit{dest}|}|
\end{tabular}
\end{center}
%
the destination file is determined by a pattern
depending on the current file:
To make this work, the current file must be called
`{\textit{prefix}\hspace{0.2em}\textit{suffix}}'
with \textit{prefix} matching precisely the argument.
Processing is then passed on to the file
`{\textit{dest}\hspace{0.2em}\textit{suffix}}'.
Surely, the same effect is achieved by
directly specifying the
argument `{\textit{dest}\hspace{0.2em}\textit{suffix}}'
in the first form.
However, that requires to set up a different file
for each child. With the alternative form of the command
all these files can have exactly the same content
which simplifies setting them up and maintaining them.

For example, the following file |draft.tex|
with a compilation flag |\version| as described in \secref{sec:flags}
compiles the main document as a draft:
%
\begin{center}
\begin{tabular}{l}
|\def\version{draft}|\\
|% \iffalse
%
% childdoc.dtx Copyright (C) 2017-2018 Niklas Beisert
%
% This work may be distributed and/or modified under the
% conditions of the LaTeX Project Public License, either version 1.3
% of this license or (at your option) any later version.
% The latest version of this license is in
%   http://www.latex-project.org/lppl.txt
% and version 1.3 or later is part of all distributions of LaTeX
% version 2005/12/01 or later.
%
% This work has the LPPL maintenance status `maintained'.
%
% The Current Maintainer of this work is Niklas Beisert.
%
% This work consists of the files childdoc.dtx and childdoc.ins
% and the derived files childdoc.def and cdocsamp.tex with
% cdocsch1.tex, cdocsch2.tex, cdocsdrf.tex, cdocsfn1.tex, cdocsfn2.tex.
%
%<package>\ifdefined\childdocmain\endinput\fi
%<package>\ProvidesFile{childdoc.def}[2018/12/30 v2.0 child document driver]
%<samplemain>\ProvidesFile{cdocsamp.tex}[2018/12/30 v2.0 sample for childdoc]
%<*driver>
%\ProvidesFile{childdoc.drv}[2018/12/30 v2.0 childdoc reference manual file]
\PassOptionsToClass{10pt,a4paper}{article}
\documentclass{ltxdoc}

\usepackage[margin=35mm]{geometry}
\usepackage{hyperref}
\usepackage{hyperxmp}
\usepackage[usenames]{color}

\hypersetup{colorlinks=true}
\hypersetup{pdfstartview=FitH}
\hypersetup{pdfpagemode=UseNone}
\hypersetup{pdfsource={}}
\hypersetup{pdflang={en-UK}}
\hypersetup{pdfcopyright={Copyright 2017-2018 Niklas Beisert.
  This work may be distributed and/or modified under the
  conditions of the LaTeX Project Public License, either version 1.3
  of this license or (at your option) any later version.}}
\hypersetup{pdflicenseurl={http://www.latex-project.org/lppl.txt}}
\hypersetup{pdfcontactaddress={ETH Zurich, ITP, HIT K,
  Wolfgang-Pauli-Strasse 27}}
\hypersetup{pdfcontactpostcode={8093}}
\hypersetup{pdfcontactcity={Zurich}}
\hypersetup{pdfcontactcountry={Switzerland}}
\hypersetup{pdfcontactemail={nbeisert@itp.phys.ethz.ch}}
\hypersetup{pdfcontacturl={http://people.phys.ethz.ch/\xmptilde nbeisert/}}

\newcommand{\secref}[1]{\hyperref[#1]{section \ref*{#1}}}

\parskip1ex
\parindent0pt
\let\olditemize\itemize
\def\itemize{\olditemize\parskip0pt}

\begin{document}

\title{The \textsf{childdoc} Package}
\hypersetup{pdftitle={The childdoc Package}}
\author{Niklas Beisert\\[2ex]
  Institut f\"ur Theoretische Physik\\
  Eidgen\"ossische Technische Hochschule Z\"urich\\
  Wolfgang-Pauli-Strasse 27, 8093 Z\"urich, Switzerland\\[1ex]
  \href{mailto:nbeisert@itp.phys.ethz.ch}
  {\texttt{nbeisert@itp.phys.ethz.ch}}}
\hypersetup{pdfauthor={Niklas Beisert}}
\hypersetup{pdfsubject={Manual for the LaTeX2e Package childdoc}}
\date{30 December 2018, \textsf{v2.0}}
\maketitle

\begin{abstract}\noindent
\textsf{childdoc} is a \LaTeXe{} package
that enables the direct compilation
of document sections included by |\include|
to individual files.
\end{abstract}

\begingroup
\parskip0ex
\tableofcontents
\endgroup

%%%%%%%%%%%%%%%%%%%%%%%%%%%%%%%%%%%%%%%%%%%%%%%%%%%%%%%%%%%%%%%%%%%%%%%%%%%%%%%%
%%%%%%%%%%%%%%%%%%%%%%%%%%%%%%%%%%%%%%%%%%%%%%%%%%%%%%%%%%%%%%%%%%%%%%%%%%%%%%%%
\section{Introduction}

\LaTeX{} provides a mechanism to structure a large document (such as a book)
into a main file and several child files (containing the chapters)
using the |\include| command.
This mechanism is beneficial for documents
which span hundreds of pages in order to
make the source file(s) more manageable.
Moreover, compilation can be restricted to
selected child files by means of the |\includeonly| command.
The latter feature can be used to reduce the compilation time while editing
(this was significantly more useful in the earlier days of \LaTeX{})
or to generate a smaller document which is easier to navigate.
Another application of |\includeonly| is to generate
documents consisting of selected parts of the complete document.

However, there are a few drawbacks of the plain |\include| mechanism:
\begin{itemize}
\item
The child files cannot be compiled on their own,
they can only be compiled via the main file.
A naive editing environment
(such as a text editor with an option
to have the current file processed by \LaTeX)
may require one to switch to the main file before compiling;
attempting to compile the child file produces errors.
\item
The main file must be modified (each time)
to adjust the |\includeonly| command
to the present needs. This easily leaves the main file in a messy state.
\item
The generated document will always carry the filename
of the main document. This is inconvenient if
several child files are to be compiled and
to be kept for distribution.
\end{itemize}

The present package provides a simple interface
to make child files individually compilable by \LaTeX{}.
Compiling a child file then has the same effect as compiling
the main file with an |\includeonly| command
to select the appropriate child.
Moreover the generated document will carry the name of the child
rather than the main file.
This resolves all three above issues.

This feature is meant to make the editing of books,
thesis documents and lecture notes somewhat more convenient.
However, the package can also be used efficiently for
composing a series of documents (such as exercise sheets)
which are typically distributed individually.
It then assists the author in generating the individual documents
(potentially in different versions)
as well as a document containing the collected series.
Another application is in developing style files
or other kinds of included material
where compilation of the style file could redirect
to a sample or test file.

%%%%%%%%%%%%%%%%%%%%%%%%%%%%%%%%%%%%%%%%%%%%%%%%%%%%%%%%%%%%%%%%%%%%%%%%%%%%%%%%
%%%%%%%%%%%%%%%%%%%%%%%%%%%%%%%%%%%%%%%%%%%%%%%%%%%%%%%%%%%%%%%%%%%%%%%%%%%%%%%%
\section{Usage}

First of all, the package \textsf{childdoc} is \emph{not} a standard
\LaTeXe{} |.sty| style file! Therefore it needs to be invoked in
a non-standard way.

%%%%%%%%%%%%%%%%%%%%%%%%%%%%%%%%%%%%%%%%%%%%%%%%%%%%%%%%%%%%%%%%%%%%%%%%%%%%%%%%
\subsection{Included Files}
\label{sec:include}

%%%%%%%%%%%%%%%%%%%%%%%%%%%%%%%%%%%%%%%%
\DescribeMacro{\childdocmain}
To use the package, add the commands
\begin{center}
\begin{tabular}{l}
|\input{childdoc.def}|\\
|\childdocmain{}|\\
\end{tabular}
\end{center}
at the very top of the main \LaTeX{} file,
in particular \emph{before} the |\documentclass| statement!
The argument of |\childdocmain| should be left empty
(but it must be present).

%%%%%%%%%%%%%%%%%%%%%%%%%%%%%%%%%%%%%%%%
\DescribeMacro{\childdocof}
Furthermore, add the commands
\begin{center}
\begin{tabular}{l}
|\input{childdoc.def}|\\
|\childdocof{|\textit{main}|}|\\
\end{tabular}
\end{center}
at the top of every child file \textit{child}
which is included by |\include{|\textit{child}|}|
from within the main file
(or at least for those files to be compiled individually).
The argument \textit{main} must be the filename of the main file.

There are a couple of
considerations in setting up the main and child documents:

%%%%%%%%%%%%%%%%%%%%%%%%%%%%%%%%%%%%%%%%
\paragraph{Restrictions.}

Please note the following restrictions:
\begin{itemize}
\item
|\childdocmain| must be called with one argument \textit{main}
to ensure compatibility with earlier version of the package.
It must either be empty (|\childdocmain{}|)
or precisely match the filename of the main file in which it is specified.
See \secref{sec:detection} for further information.
\item
The filename \textit{main} must be specified without the |.tex| extension.
\item
The filename \textit{main} is case sensitive
(even in case-insensitive file systems)
due to internal string comparison.
\item
The argument \textit{main} should be fully expanded, it cannot be a macro.
\item
Subdirectories and special characters should be avoided in filenames.
\item
The command |\childdocmain{|\textit{main}|}| must be followed by a whitespace.
It should not be followed immediately by another command
or by a comment mark `|%|'.
This is because the \TeX{} parser reads the token immediately following
the argument of |\childdocmain| and puts it
at the beginning of every child section;
however, a white\-space is ignored.
\end{itemize}

%%%%%%%%%%%%%%%%%%%%%%%%%%%%%%%%%%%%%%%%
\paragraph{Content of Main File.}

It is advisable to place all content in the child files included by |\include|.
Any output contained in the main file will appear in all child documents
unless suppressed manually;
it cannot be suppressed automatically by the |\includeonly| directive
and thus should normally be avoided.
A method to include some content in the main file
by means of conditional processing is described in \secref{sec:conditional}.

%%%%%%%%%%%%%%%%%%%%%%%%%%%%%%%%%%%%%%%%
\paragraph{Page Numbering.}

When only a part of the document is compiled,
the appropriate numbering of pages
(as well as other status parameters)
is determined from the |.aux| files.
The latter contain information from previous passes.
However this information needs to propagate through
all intermediate child documents.
Therefore the page numbering in child documents may well
be inconsistent until the complete document is compiled at least once.

A useful (if unconventional) way to always ensure a consistent
page numbering is to restart the numbering in each child document
and denote the pages by `\textit{child}|.|\textit{page}'
where \textit{child} represents the chapter/section number of the child file.
This can be achieved by the command
|\numberwithin{page}{|\textit{child}|}|
of the \textsf{amsmath} package
where \textit{child} can be |chapter| or |section|
depending on the chosen structuring.
Alternatively, one can modify the macro |\thepage| appropriately
and reset the counter |page| at the start of each child file.

%%%%%%%%%%%%%%%%%%%%%%%%%%%%%%%%%%%%%%%%%%%%%%%%%%%%%%%%%%%%%%%%%%%%%%%%%%%%%%%%
\subsection{Conditional Processing}
\label{sec:conditional}

The package provides a mechanism to compile different versions
of a document. To customise the versions further some conditional processing
can come in handy to distinguish which version is being compiled.
The package provides two macros to describe the compilation context:

%%%%%%%%%%%%%%%%%%%%%%%%%%%%%%%%%%%%%%%%
\DescribeMacro{\ifchilddoc}
The conditional |\ifchilddoc| distinguishes between the compilation of
child documents and the main document:
%
\begin{center}
|\ifchilddoc |\textit{child-code}| |[|\||else |\textit{main-code}]| \||fi|
\end{center}

%%%%%%%%%%%%%%%%%%%%%%%%%%%%%%%%%%%%%%%%
\DescribeMacro{\childdocname}
\DescribeMacro{\childdocjob}
The macro |\childdocname| contains the filename (without extension)
of the main or child file being processed.
Note that |\childdocjob| will always contain the name of the main file.

%%%%%%%%%%%%%%%%%%%%%%%%%%%%%%%%%%%%%%%%
\paragraph{Title Page.}

Conditional processing can be used to include a title or banner page
in the main document when proper precautions are taken.
Importantly, the code in the main file should ensure that the page counter
(as well as other status parameters which are stored in the |.aux| files)
takes the same value after the conditional processing.
Otherwise the page numbers may take divergent values
depending on which part is compiled.

For example, a title page could be declared by:
%
\begin{center}
\begin{tabular}{l}
|\ifchilddoc\||else|\\
|\addtocounter{page}{-1}|\\
\textit{code for title page}\\
|\newpage|\\
|\||fi|
\end{tabular}
\end{center}
%
A banner page for the child documents can be generated by:
%
\begin{center}
\begin{tabular}{l}
|\ifchilddoc|\\
|\addtocounter{page}{-1}|\\
\textit{code for banner page}\\
|\newpage|\\
|\||fi|
\end{tabular}
\end{center}
%
Here one could write a message such as:
\begin{center}
|This is the part \childdocname{} of \childdocjob{}.|
\end{center}

%%%%%%%%%%%%%%%%%%%%%%%%%%%%%%%%%%%%%%%%%%%%%%%%%%%%%%%%%%%%%%%%%%%%%%%%%%%%%%%%
\subsection{Flags}
\label{sec:flags}

The package makes it easy to generate different versions
of the main or child documents.
To this end compilation flags can be defined
and assigned different default values.
They will be particularly useful in conjunction
with the forwarding mechanism described in \secref{sec:forward}.

For example, it may be useful to have a flag |\version|
which can be set to |draft| or |final|.
The document source will contain some conditional code
depending on the value of |\version|.
Suppose further, the flag should default to |final| for the main file
and to |draft| for child files
which is a natural assignment for editing the document.
This is achieved by placing the following code
in the preamble of the main document
(below the |\childdocmain| directive):
%
\begin{center}
\begin{tabular}{l}
|\ifchilddoc|\\
|\providecommand{\version}{draft}|\\
|\||else|\\
|\providecommand{\version}{final}|\\
|\||fi|
\end{tabular}
\end{center}
%
The definition by |\providecommand| makes sure
that previous definitions are not overwritten.
Further statements |\providecommand{\version}{...}|
can thus be added before the above code to override it.

For the main file, one might add a line
(between |\childdocmain| and the above block)
%
\begin{center}
|%\ifchilddoc\||else\providecommand{\version}{draft}\||fi|
\end{center}
%
which can be uncommented to produce a draft version.
Likewise one can add a line to the very top of a child file
(above the |\childdocof{|\textit{main}|}| directive)
%
\begin{center}
|%\providecommand{\version}{final}|
\end{center}
%
which can be uncommented to produce the final version of this child document.

%%%%%%%%%%%%%%%%%%%%%%%%%%%%%%%%%%%%%%%%%%%%%%%%%%%%%%%%%%%%%%%%%%%%%%%%%%%%%%%%
\subsection{Forwarding}
\label{sec:forward}

Different versions of the main or child documents
using compilation flags as described in \secref{sec:flags}
can be (permanently) stored in different files
for convenient compilation, viewing and distribution.
To this end, the package defines a command
to pass on compilation to a different file:

%%%%%%%%%%%%%%%%%%%%%%%%%%%%%%%%%%%%%%%%
\DescribeMacro{\childdocforward}
The command |\childdocforward| redirects processing to
another source file:
%
\begin{center}
\begin{tabular}{l}
|\input{childdoc.def}|\\
|\childdocforward[|\textit{main}|]{|\textit{dest}|}|\\
\end{tabular}
\end{center}
%
The argument \textit{dest} is the destination file
(without extension).
It should be the main file or one of the child files.
Note that further \textsf{childdoc} directives
such as |\childdocof| and |\childdocforward|
in the indicated file will be processed in this form.
The optional argument \textit{main}
passes on directly to the main file \textit{main}
while pretending to compile the child \textit{dest}.
This form behaves as if \textit{dest}
issues |\childdocof{|\textit{main}|}| right away,
and no further \textsf{childdoc} directives will be processed.

%%%%%%%%%%%%%%%%%%%%%%%%%%%%%%%%%%%%%%%%
\DescribeMacro{\...prefix}
In the alternative form |\childdocforwardprefix|,
%
\begin{center}
\begin{tabular}{l}
|\input{childdoc.def}|\\
|\childdocforwardprefix[|\textit{main}|]{|\textit{prefix}|}{|\textit{dest}|}|
\end{tabular}
\end{center}
%
the destination file is determined by a pattern
depending on the current file:
To make this work, the current file must be called
`{\textit{prefix}\hspace{0.2em}\textit{suffix}}'
with \textit{prefix} matching precisely the argument.
Processing is then passed on to the file
`{\textit{dest}\hspace{0.2em}\textit{suffix}}'.
Surely, the same effect is achieved by
directly specifying the
argument `{\textit{dest}\hspace{0.2em}\textit{suffix}}'
in the first form.
However, that requires to set up a different file
for each child. With the alternative form of the command
all these files can have exactly the same content
which simplifies setting them up and maintaining them.

For example, the following file |draft.tex|
with a compilation flag |\version| as described in \secref{sec:flags}
compiles the main document as a draft:
%
\begin{center}
\begin{tabular}{l}
|\def\version{draft}|\\
|\input{childdoc.def}|\\
|\childdocforward{|\textit{main}|}|
\end{tabular}
\end{center}
%
Likewise, the following files |final|\textit{nn}|.tex|
compile the final version of the child document
|child|\textit{nn}|.tex|:
%
\begin{center}
\begin{tabular}{l}
|\def\version{final}|\\
|\input{childdoc.def}|\\
|\childdocforwardprefix{final}{child}|
\end{tabular}
\end{center}
%

Note that when several versions of a main file and/or of each child file
are to be generated, it may be convenient to set up a |Makefile| or
shell script to automatise the process.

%%%%%%%%%%%%%%%%%%%%%%%%%%%%%%%%%%%%%%%%%%%%%%%%%%%%%%%%%%%%%%%%%%%%%%%%%%%%%%%%
\subsection{Command Line Processing}
\label{sec:commandline}

The effect of redirection files can also be achieved by invoking
the \LaTeX{} compiler with a more elaborate command line.
Most conveniently this should be done as part
of a shell script or a |Makefile|.

When using \textsf{childdoc} in the main file, the following
command lines effectively perform a redirection
(note that depending on the shell being used,
backslashes may have to be doubled: `|\|' $\to$ `|\\|'):
%
\begin{center}
|... -jobname "|\textit{target}|" |\\|"|[\textit{flags}]%
|\input{childdoc.def}\childdocforward[|\textit{main}|]{|\textit{dest}|}"|
\end{center}
%
Here \textit{target} is the name of the output file,
\textit{main} is the name of the main file
and \textit{dest} is the name of the main or child file to be processed
(all filenames without extensions).
The optional argument \textit{main} can be omitted
if \textit{main} matches \textit{dest}.
Optionally, compilation \textit{flags} can be defined via |\def| commands.
This command line makes the \TeX{} engine believe
it is compiling the file \textit{target}
whose content is specified as the latter parameter.
The provided code then forwards the processing to
\textit{main} or \textit{dest} as described in \secref{sec:forward}.

%%%%%%%%%%%%%%%%%%%%%%%%%%%%%%%%%%%%%%%%%%%%%%%%%%%%%%%%%%%%%%%%%%%%%%%%%%%%%%%%
\subsection{Include by Input}
\label{sec:input}

Including child documents by |\include| has some restrictions by design.
Most notably, the content of a child document always occupies
its own set of pages; pages cannot be shared between child documents.
Usually, this behaviour makes perfect sense
because each child document contain an essential part of the document.
However, in some situations it may be desirable to compose
a document from a collection of parts
without having mandatory page breaks between then.
For this case, the package
provides a mechanism to include parts
by |\input| which can also be processed individually.
However, by construction this mechanism
requires manual handling of the content to be output.

%%%%%%%%%%%%%%%%%%%%%%%%%%%%%%%%%%%%%%%%
\DescribeMacro{\ifchilddocmanual}
The main file should be prepared as usual, see \secref{sec:include}.
However, the document body must make a distinction
between processing of an individual part and of the main document, e.g.:
%
\begin{center}
\begin{tabular}{l}
|\ifchilddocmanual|\\
|\input{\childdocname}|\\
|\||else|\\
\textit{document body with }|\input{|\textit{part}|}|\\
|\||fi|
\end{tabular}
\end{center}
%
The conditional |\ifchilddocmanual| is true whenever
a part to be included by |\input| is being compiled,
and the name of the part is stored in |\childdocname|.

%%%%%%%%%%%%%%%%%%%%%%%%%%%%%%%%%%%%%%%%
\DescribeMacro{\childdocby}
Each part to be included by |\input| should start with:
%
\begin{center}
\begin{tabular}{l}
|\input{childdoc.def}|\\
|\childdocby{|\textit{main}|}|\\
\end{tabular}
\end{center}
%
The directive |\childdocby| is similar to |\childdocof|
described in \secref{sec:include},
but the subsequent selection of content must be done manually.
To that end, both |\ifchilddoc| and |\ifchilddocmanual|
will be true upon processing of a part,
and the name of the part is stored in |\childdocname|.
Note that |\jobname| will be set to the filename of the current part
so that each part receives an individual |.aux| file
that does not interfere with the |.aux| file(s) of the main document.
This behaviour can be altered by the alternative form
|\childdocby[*]{|\textit{main}|}| (with a non-empty optional argument)
which uses the |.aux| file of the main document
by setting |\jobname| to \textit{main}.

%%%%%%%%%%%%%%%%%%%%%%%%%%%%%%%%%%%%%%%%%%%%%%%%%%%%%%%%%%%%%%%%%%%%%%%%%%%%%%%%
\subsection{Driver Development}
\label{sec:driver}

The \textsf{childdoc} mechanism can also be use for the development
of definition files such as \LaTeX{} styles or classes.
This case differs from the above setup with multiple parts
included by |\include| in that no |\includeonly| should be invoked.
This can be achieved by starting the include file
(before |\ProvidesPackage|) with:
%
\begin{center}
\begin{tabular}{l}
|\input{childdoc.def}|\\
|\childdocforward{|\textit{main}|}|\\
\end{tabular}
\end{center}
%
or alternatively with:
%
\begin{center}
\begin{tabular}{l}
|\input{childdoc.def}|\\
|\childdocby{|\textit{main}|}|\\
\end{tabular}
\end{center}
%
Both forms have slightly different effects as described above.
The main file is prepared as usual, see \secref{sec:include}.

%%%%%%%%%%%%%%%%%%%%%%%%%%%%%%%%%%%%%%%%%%%%%%%%%%%%%%%%%%%%%%%%%%%%%%%%%%%%%%%%
\subsection{Legacy Detection}
\label{sec:detection}

The directive |\childdocmain| in the main file can detect
whether the complete document or merely a child is to be compiled
even without using the directive |\childdocof|.
This method is deprecated because it is less robust
and there is no compelling reason to use it;
it is merely provided for backward compatibility
and it may be removed in future versions.

If the detection mechanism is to be used,
it is mandatory to correctly specify
the filename of the main file as the argument of |\childdocmain|:
%
\begin{center}
\begin{tabular}{l}
|\input{childdoc.def}|\\
|\childdocmain{|\textit{main}|}|\\
\end{tabular}
\end{center}
%
If |\jobname| does not match the argument \textit{main} of |\childdocmain|,
it is assumed that |\jobname| points to the child file to be compiled.
When using |\childdocmain| with the main file specified as argument,
it suffices to start a child file
with just |\input{|\textit{main}|}|
without loading of the package and using |\childdocof|.
If instead all processing is done
with the appropriate \textsf{childdoc} directives,
the argument of \textit{main} of |\childdocmain| can be empty.

An alternative version of the command line processing described
in \secref{sec:commandline} using the detection mechanism reads:
%
\begin{center}
|... -jobname "|\textit{target}|" "|[\textit{flags}]%
[|\def\jobname{|\textit{dest}|}|]|\input{|\textit{main}|}"|
\end{center}

%%%%%%%%%%%%%%%%%%%%%%%%%%%%%%%%%%%%%%%%%%%%%%%%%%%%%%%%%%%%%%%%%%%%%%%%%%%%%%%%
\subsection{Manual Code}
\label{sec:manual}

In case one cannot be certain whether the definitions file |childdoc.def|
is installed on the target \TeX{} distribution
and one prefers not to ship it,
it is conceivable to paste a few relevant commands into the sources.

To that end, drop all statements |\input{childdoc.def}|
and perform the replacements as outlined below.
Instead of |\childdocmain{|\textit{main}|}| add the following code
to the top of the main file:
%
\begin{center}
\begin{tabular}{l}
|\||ifdefined\childdocname\endinput\||fi\newif\ifchilddoc|\\
|\edef\childdocname{\scantokens\expandafter{\jobname\noexpand}}|\\
|\def\childdocmain{|\textit{main}|}\||ifx\childdocmain\childdocname\||else|\\
|\childdoctrue\includeonly{\childdocname}\let\jobname\childdocmain\||fi|\\
\end{tabular}
\end{center}
%
Instead of |\childdocof{|\textit{main}|}| just include the main file
at the top of each child file:
%
\begin{center}
|\input{|\textit{main}|}|
\end{center}
%
A simple redirection |\childdocforward{|\textit{dest}|}| is achieved by:
%
\begin{center}
|\def\jobname{|\textit{dest}|}\input{\jobname}|
\end{center}
%
The redirection with prefix
|\childdocforwardprefix[|\textit{prefix}|]{|\textit{dest}|}|
is accomplished by:
%
\begin{center}
\begin{tabular}{l}
|{\edef\jobname{\scantokens\expandafter{\jobname\noexpand}}|\\
|\def\redirectjob |\textit{prefix}|#1~~~{\gdef\jobname{|\textit{dest}|#1}}|\\
|\expandafter\redirectjob\jobname~~~}\input{\jobname}|
\end{tabular}
\end{center}

In an alternative approach,
child documents can be compiled by a specific command line
without additional code or specific definitions:
%
\begin{center}
|... -jobname "|\textit{target}|" "|[\textit{flags}]%
|\includeonly{|\textit{dest}|}\input{|\textit{main}|}"|
\end{center}
%

%%%%%%%%%%%%%%%%%%%%%%%%%%%%%%%%%%%%%%%%%%%%%%%%%%%%%%%%%%%%%%%%%%%%%%%%%%%%%%%%
%%%%%%%%%%%%%%%%%%%%%%%%%%%%%%%%%%%%%%%%%%%%%%%%%%%%%%%%%%%%%%%%%%%%%%%%%%%%%%%%
\section{Information}

%%%%%%%%%%%%%%%%%%%%%%%%%%%%%%%%%%%%%%%%%%%%%%%%%%%%%%%%%%%%%%%%%%%%%%%%%%%%%%%%
\subsection{Copyright}

Copyright \copyright{} 2017--2018 Niklas Beisert

This work may be distributed and/or modified under the
conditions of the \LaTeX{} Project Public License, either version 1.3
of this license or (at your option) any later version.
The latest version of this license is in
  \url{http://www.latex-project.org/lppl.txt}
and version 1.3 or later is part of all distributions of \LaTeX{}
version 2005/12/01 or later.

This work has the LPPL maintenance status `maintained'.

The Current Maintainer of this work is Niklas Beisert.

This work consists of the files |README.txt|, |childdoc.ins| and |childdoc.dtx|
as well as the derived files |childdoc.def|, |cdocsamp.tex|
with |cdocsch1.tex|, |cdocsch2.tex|, |cdocspt3.tex|, |cdocspt4.tex|,
|cdocsdrf.tex|, |cdocsfn1.tex|, |cdocsfn2.tex|
as well as |childdoc.pdf|.

%%%%%%%%%%%%%%%%%%%%%%%%%%%%%%%%%%%%%%%%%%%%%%%%%%%%%%%%%%%%%%%%%%%%%%%%%%%%%%%%
\subsection{Files and Installation}

The package consists of the files:
%
\begin{center}
\begin{tabular}{ll}
    |README.txt|   & readme file \\
    |childdoc.ins| & installation file \\
    |childdoc.dtx| & source file \\
    |childdoc.def| & definition file \\
    |cdocsamp.tex| & sample main file \\
    |cdocsch1.tex| & sample include file \\
    |cdocsch2.tex| & sample include file \\
    |cdocspt3.tex| & sample part file \\
    |cdocspt4.tex| & sample part file \\
    |cdocsdrf.tex| & sample redirection file \\
    |cdocsfn1.tex| & sample redirection file \\
    |cdocsfn2.tex| & sample redirection file \\
    |childdoc.pdf| & manual
\end{tabular}
\end{center}
%
The distribution consists of the files
|README.txt|, |childdoc.ins| and |childdoc.dtx|.
%
\begin{itemize}
\item
Run (pdf)\LaTeX{} on |childdoc.dtx|
to compile the manual |childdoc.pdf| (this file).
\item
Run \LaTeX{} on |childdoc.ins| to create the definitions file |childdoc.def|
and the sample |cdocsamp.tex| with include files
|cdocsch1.tex|, |cdocsch2.tex|, |cdocspt3.tex|, |cdocspt4.tex|,
|cdocsdrf.tex|, |cdocsfn1.tex|, |cdocsfn2.tex|.
Then copy the file |childdoc.def| to an appropriate directory of your \LaTeX{}
distribution, e.g.\ \textit{texmf-root}|/tex/latex/childdoc|.
\end{itemize}

%%%%%%%%%%%%%%%%%%%%%%%%%%%%%%%%%%%%%%%%%%%%%%%%%%%%%%%%%%%%%%%%%%%%%%%%%%%%%%%%
\subsection{Related CTAN Packages}

There are several other packages which offer a similar functionality:
%
\begin{itemize}
\item
The packages
\href{http://ctan.org/pkg/docmute}{\textsf{docmute}},
\href{http://ctan.org/pkg/includex}{\textsf{includex}} and
\href{http://ctan.org/pkg/standalone}{\textsf{standalone}}
provide commands to include only the document body of
a child file thus allowing both files to be compiled individually.
\item
The packages \href{http://ctan.org/pkg/subdocs}{\textsf{subdocs}}
and \href{http://ctan.org/pkg/subfiles}{\textsf{subfiles}}
provide structures in which the main and child documents can be
encapsulated and allowing them to be compiled individually.
The inclusion mechanism is different from the conventional |\include|.
\item
The package \href{http://ctan.org/pkg/combine}{\textsf{combine}}
is an elaborate solution to combine several documents into one.
\end{itemize}
%
See also the CTAN topic \href{http://ctan.org/topic/subdocs}{\textsf{subdocs}}
for further related packages.
The present package differs from the above solutions in that
a document structure constructed with the conventional |\include| mechanism
just needs two extra commands at the top of every file
such that all constituent files can be compiled individually.

%%%%%%%%%%%%%%%%%%%%%%%%%%%%%%%%%%%%%%%%%%%%%%%%%%%%%%%%%%%%%%%%%%%%%%%%%%%%%%%%
%\subsection{Feature Suggestions}
%
%The following is a list of features which may be useful for future
%versions of this package:
%%
%\begin{itemize}
%\item
%\ldots
%\end{itemize}

%%%%%%%%%%%%%%%%%%%%%%%%%%%%%%%%%%%%%%%%%%%%%%%%%%%%%%%%%%%%%%%%%%%%%%%%%%%%%%%%
\subsection{Revision History}

%%%%%%%%%%%%%%%%%%%%%%%%%%%%%%%%%%%%%%%%
\paragraph{v2.0:} 2018/12/30

\begin{itemize}
\item
immediate forward processing
\item
added |\childdocby| mechanism
\item
manual restructured
\end{itemize}

%%%%%%%%%%%%%%%%%%%%%%%%%%%%%%%%%%%%%%%%
\paragraph{v1.6:} 2018/01/17

\begin{itemize}
\item
application for development of include files
\item
corrections to manual
\end{itemize}

%%%%%%%%%%%%%%%%%%%%%%%%%%%%%%%%%%%%%%%%
\paragraph{v1.5:} 2017/05/21

\begin{itemize}
\item
more complete structuring introduced
\item
|\childdocof| introduced
\item
|\childdoc| renamed to |\childdocmain|
\item
|\childredirect| renamed to |\childdocforward| and |\childdocforwardprefix|
and functionality expanded
\end{itemize}

%%%%%%%%%%%%%%%%%%%%%%%%%%%%%%%%%%%%%%%%
\paragraph{v1.0:} 2017/04/27

\begin{itemize}
\item
manual and install package
\item
first version published on CTAN
\end{itemize}

%%%%%%%%%%%%%%%%%%%%%%%%%%%%%%%%%%%%%%%%
\paragraph{v0.6:} 2017/04/26

\begin{itemize}
\item
redirection mechanism added
\end{itemize}

%%%%%%%%%%%%%%%%%%%%%%%%%%%%%%%%%%%%%%%%
\paragraph{v0.5:} 2017/04/26

\begin{itemize}
\item
functionality in definition file
\end{itemize}


%%%%%%%%%%%%%%%%%%%%%%%%%%%%%%%%%%%%%%%%%%%%%%%%%%%%%%%%%%%%%%%%%%%%%%%%%%%%%%%%
%%%%%%%%%%%%%%%%%%%%%%%%%%%%%%%%%%%%%%%%%%%%%%%%%%%%%%%%%%%%%%%%%%%%%%%%%%%%%%%%
%%%%%%%%%%%%%%%%%%%%%%%%%%%%%%%%%%%%%%%%%%%%%%%%%%%%%%%%%%%%%%%%%%%%%%%%%%%%%%%%
\appendix

\settowidth\MacroIndent{\rmfamily\scriptsize 000\ }

 \DocInput{childdoc.dtx}

\end{document}
%</driver>
% \fi
%
% %%%%%%%%%%%%%%%%%%%%%%%%%%%%%%%%%%%%%%%%%%%%%%%%%%%%%%%%%%%%%%%%%%%%%%%%%%%%%%
% %%%%%%%%%%%%%%%%%%%%%%%%%%%%%%%%%%%%%%%%%%%%%%%%%%%%%%%%%%%%%%%%%%%%%%%%%%%%%%
% \section{Sample}
%\iffalse
%<*samplemain>
%\fi
%
% The following presents a sample document
% with two chapters, two parts, a title page,
% a compile flag as well as three forwarding files to set the flag.
% It consists of eight |.tex| files:
% \begin{center}
% \begin{tabular}{ll}
% |cdocsamp.tex|&main file\\
% |cdocsch1.tex|&include file for chapter 1\\
% |cdocsch2.tex|&include file for chapter 2\\
% |cdocspt3.tex|&include file for part 3\\
% |cdocspt4.tex|&include file for part 4\\
% |cdocsdrf.tex|&forwarding file for main file in draft mode\\
% |cdocsfi1.tex|&forwarding file for final version of chapter 1\\
% |cdocsfi2.tex|&forwarding file for final version of chapter 2\\
% \end{tabular}
% \end{center}
% Each of the eight files can be compiled directly by the \LaTeX{} compiler.
%
% %%%%%%%%%%%%%%%%%%%%%%%%%%%%%%%%%%%%%%
% \paragraph{Main File.}
%
% The main file is called |cdocsamp.tex|.
%
% Load the \textsf{childdoc} definitions and
% declare the filename for the main document:
%    \begin{macrocode}
\input{childdoc.def}
\childdocmain{}
%    \end{macrocode}

% Optional override for |\version| flag:
%    \begin{macrocode}
%%\ifchilddoc\else\providecommand{\version}{draft}\fi
%    \end{macrocode}

% Define the default values for the |\version| flag
% (|final| for the main file and |draft| for childs):
%    \begin{macrocode}
\ifchilddoc
\providecommand{\version}{draft}
\else
\providecommand{\version}{final}
\fi
%    \end{macrocode}

% Load the standard document class:
%    \begin{macrocode}
\documentclass[12pt]{article}
%    \end{macrocode}

% Start the document body:
%    \begin{macrocode}
\begin{document}
%    \end{macrocode}

% Declare a title page.
% Print title, part of document being processed and version flag:
%    \begin{macrocode}
\addtocounter{page}{-1}
\begin{center}
{\LARGE\bfseries{}childdoc example\par}
\vspace{1cm}
\ifchilddoc
\ifchilddocmanual part\else chapter\fi:
`\childdocname' of `\childdocjob'\par
\else
main document: `\childdocjob'\par
\fi
version: \version\par
\end{center}
\newpage
%    \end{macrocode}

% Manually include selected file,
% otherwise process as usual:
%    \begin{macrocode}
\ifchilddocmanual
\section*{part `\childdocname'}
\input{\childdocname}
\else
%    \end{macrocode}

% Include the two chapters:
%    \begin{macrocode}
\include{cdocsch1}
\include{cdocsch2}
%    \end{macrocode}

% Include the two parts unless only chapters should be displayed:
%    \begin{macrocode}
\ifchilddoc\else
\section{part three}
\input{cdocspt3}
\section{part four}
\input{cdocspt4}
\fi
%    \end{macrocode}

% Process as usual until here:
%    \begin{macrocode}
\fi
%    \end{macrocode}

% End of document body:
%    \begin{macrocode}
\end{document}
%    \end{macrocode}
%\iffalse
%</samplemain>
%\fi
%
% %%%%%%%%%%%%%%%%%%%%%%%%%%%%%%%%%%%%%%
% \paragraph{Chapter Include Files.}
%
% The include files are called |cdocsch1.tex| and |cdocsch2.tex|.
%
%\iffalse
%<*samplechap1|samplechap2>
%\fi

% Optional override for |\version| flag:
%    \begin{macrocode}
%%\providecommand{\version}{final}
%    \end{macrocode}

% Include the main document:
%    \begin{macrocode}
\input{childdoc.def}
\childdocof{cdocsamp}
%    \end{macrocode}

%\iffalse
%</samplechap1|samplechap2>
%\fi
%
%\iffalse
%<*samplechap1>
%\fi
% Some text for chapter 1:
%    \begin{macrocode}
\section{one}
some text in chapter one
%    \end{macrocode}

%\iffalse
%</samplechap1>
%\fi
% Some text for chapter 2:
%\iffalse
%<*samplechap2>
%\fi
%    \begin{macrocode}
\section{two}
more text in chapter two
%    \end{macrocode}

%\iffalse
%</samplechap2>
%\fi
%
% %%%%%%%%%%%%%%%%%%%%%%%%%%%%%%%%%%%%%%
% \paragraph{Part Include Files.}
%
% The include files are called |cdocspt3.tex| and |cdocspt4.tex|.
%
%\iffalse
%<*samplepart3|samplepart4>
%\fi

% Optional override for |\version| flag:
%    \begin{macrocode}
%%\providecommand{\version}{final}
%    \end{macrocode}

% Include the main document:
%    \begin{macrocode}
\input{childdoc.def}
\childdocby{cdocsamp}
%    \end{macrocode}

%\iffalse
%</samplepart3|samplepart4>
%\fi
%
%\iffalse
%<*samplepart3>
%\fi
% Some text for part 3:
%    \begin{macrocode}
some text in part three
%    \end{macrocode}

%\iffalse
%</samplepart3>
%\fi
% Some text for part 4:
%\iffalse
%<*samplepart4>
%\fi
%    \begin{macrocode}
more text in part four
%    \end{macrocode}

%\iffalse
%</samplepart4>
%\fi
%
% %%%%%%%%%%%%%%%%%%%%%%%%%%%%%%%%%%%%%%
% \paragraph{Forwarding for a Complete Draft.}
%
% The following forwarding file |cdocsdrf.tex|
% compiles the main document in draft mode:
%\iffalse
%<*sampledraft>
%\fi
%    \begin{macrocode}
\def\version{draft}
\input{childdoc.def}
\childdocforward{cdocsamp}
%    \end{macrocode}

%\iffalse
%</sampledraft>
%\fi
%
% %%%%%%%%%%%%%%%%%%%%%%%%%%%%%%%%%%%%%%
% \paragraph{Forwarding for Final Version of the Chapters.}
%
% The following forwarding files |cdocsfn1.tex| and |cdocsfn2.tex|
% (with identical content)
% compile the final versions of the child documents
% |cdocsch1.tex| and |cdocsch2.tex|, respectively:
%\iffalse
%<*samplefinal>
%\fi
%    \begin{macrocode}
\def\version{final}
\input{childdoc.def}
\childdocforwardprefix[cdocsamp]{cdocsfn}{cdocsch}
%    \end{macrocode}

%\iffalse
%</samplefinal>
%\fi
%
% %%%%%%%%%%%%%%%%%%%%%%%%%%%%%%%%%%%%%%
% \paragraph{Command Line Processing.}
%
% The following three command lines generate the output files
% |cdocscld|, |cdocscl1| and |cdocscl2|
% which should be identical to
% |cdocsdrf|, |cdocsch1| and |cdocsfn2|, respectively:
% \begin{center}
% \begin{tabular}{l}
% |latex -jobname cdocscld \|\\
% |  "\def\version{draft}\input{childdoc.def}\childdocforward{cdocsamp}"|\\
% |latex -jobname cdocscl1 \|\\
% |  "\input{childdoc.def}\childdocforward[cdocsamp]{cdocsch1}"|\\
% |latex -jobname cdocscl2 \|\\
% |  "\def\version{final}\input{childdoc.def}\childdocforward{cdocsch2}"|
% \end{tabular}
% \end{center}
% Note that the trailing backslash on each first line
% merely continues the input to the second line
% (for convenient cut ant paste).
% Furthermore, the command |latex| can be replaced by any
% of its alternative versions such as |pdflatex|.
%
% %%%%%%%%%%%%%%%%%%%%%%%%%%%%%%%%%%%%%%%%%%%%%%%%%%%%%%%%%%%%%%%%%%%%%%%%%%%%%%
% %%%%%%%%%%%%%%%%%%%%%%%%%%%%%%%%%%%%%%%%%%%%%%%%%%%%%%%%%%%%%%%%%%%%%%%%%%%%%%
% \section{Implementation}
%\iffalse
%<*package>
%\fi
%
% This section describes the definitions file |childdoc.def|.

% The definitions cannot be loaded using |\usepackage| or |\RequirePackage|
% which has a mechanism to prevent loading a style file more than once.
% When loading the definitions by means of |\input|
% multiple instances have to be prevented manually:
%\iffalse
%This code needs to be before the `\ProvidesFile' directive
%which is defined at the beginning of this file.
%Therefore it is also placed there and commented out here.
%</package>
%<*discard>
%\fi
%    \begin{macrocode}
\ifdefined\childdocmain\endinput\fi
%    \end{macrocode}
%\iffalse
%</discard>
%<*package>
%\fi
%
% \macro{\ifchilddoc}
% \macro{\ifchilddocmanual}
% The conditional |\ifchilddoc| tells whether a
% child (true) or main (false) document is being compiled.
% The conditional |\ifchilddocmanual| tells whether
% the |\includeonly| mechanism is used (false) or
% the selection of child files must be performed manually (true).
% The definitions initialise to false:
%    \begin{macrocode}
\newif\ifchilddoc
\newif\ifchilddocmanual
%    \end{macrocode}

% \macro{\childdocname}
% \macro{\childdocjob}
% The macro |\childdocname| stores the name of the main document
% to be compiled. The macro |\childdocjob| stores the name of
% the document on which the \LaTeX{} compiler was originally invoked.
% The content of |\jobname| cannot be compared
% to filenames specified in the source due to different catcodes.
% The following code rescans |\jobname|, stores the result
% in |\childdocname| and saves a copy in |\childdocjob|:
%    \begin{macrocode}
\edef\childdocname{\scantokens\expandafter{\jobname\noexpand}}
\let\childdocjob\childdocname
%    \end{macrocode}

% \macro{\childdocdisable}
% The macro |\childdocdisable| prevents the main file
% from being processed more than once.
% At this stage, the main document command |\childdocmain|
% is assumed to be called once again where it should do nothing.
% Any subsequent call to it should prevent
% a secondary processing of the main document
% It overwrites the forwarding commands
% |\childdocof| and |\childdocforward|
% with empty macros to prevent further inclusions of the main document:
%    \begin{macrocode}
\newcommand{\childdocdisable}
{
  \renewcommand{\childdocmain}[1]{\renewcommand{\childdocmain}[1]{\endinput}}
  \renewcommand{\childdocof}[1]{}
  \renewcommand{\childdocby}[2][]{}
  \renewcommand{\childdocforward}[2][]{}
  \renewcommand{\childdocdisable}{}
}
%    \end{macrocode}

% \macro{\childdocmain}
% The macro |\childdocmain| is to be called at the top of the main file
% with nothing or the main filename (without extension) as argument.
% First, it breaks loops.
% If the argument is not empty and does not match |\childdocname|
% (which is set by the first inclusion of |childdoc.def|),
% |\ifchilddoc| is set to true, |\includeonly| is applied to the child file
% and |\jobname| is set to the main file
% (for proper handling of |.aux| files):
%    \begin{macrocode}
\newcommand{\childdocmain}[1]
{
  \childdocdisable\childdocmain{}
  \if?#1?\else
    \begingroup
      \def\childdoctmp{#1}
      \ifx\childdoctmp\childdocname
        \def\childdoctmp{}
      \else
        \def\childdoctmp
        {
          \childdoctrue
          \includeonly{\childdocname}
          \def\childdocjob{#1}
          \def\jobname{#1}
        }
      \fi
      \expandafter
    \endgroup
    \childdoctmp
  \fi
}
%    \end{macrocode}

% \macro{\childdocof}
% The command |\childdocof| redirects
% compilation to the main file |#1|.
%    \begin{macrocode}
\newcommand{\childdocof}[1]
{
  \childdocdisable
  \childdoctrue
  \includeonly{\childdocname}
  \def\jobname{#1}
  \def\childdocjob{#1}
  \input{#1}
}
%    \end{macrocode}

% \macro{\childdocby}
% The command |\childdocby| ....
%    \begin{macrocode}
\newcommand{\childdocby}[2][]
{
  \childdocdisable
  \childdoctrue
  \childdocmanualtrue
  \if?#1?\else
    \def\jobname{#2}
  \fi
  \def\childdocjob{#2}
  \input{#2}
  \endinput
}
%    \end{macrocode}

% \macro{\childdocforward}
% The command |\childdocforward| redirects
% compilation to the main file or
% (if the optional argument is given) a child file.
% Parameters are set as if the main file
% or a child file starting with |\childdocof| was compiled.
% Then compilation is handed over to the main file:
%    \begin{macrocode}
\newcommand{\childdocforward}[2][]
{
  \begingroup
    \if?#1?
      \def\childdoctmp
      {
        \def\childdocname{#2}
        \def\childdocjob{#2}
        \def\jobname{#2}
        \input{#2}
        \endinput
      }
    \else
      \def\childdoctmp
      {
        \childdocdisable
        \def\childdocname{#2}
        \childdoctrue
        \includeonly{#2}
        \def\childdocjob{#1}
        \def\jobname{#1}
        \input{#1}
        \endinput
      }
    \fi
    \expandafter
  \endgroup
  \childdoctmp
}
%    \end{macrocode}

% \macro{\childdocforwardprefix}
% The command |\childdocforwardprefix| redirects
% compilation to the main or a child file by means of a pattern.
% The prefix |#1| in the current filename is replaced by |#2|
% and the suffix of the current filename is kept
% (it is assumed that the filename does not contain the substring `|~~~|'
% which is used as a delimiter).
% Compilation is handed over to the new file by |\childdocforward|:
%    \begin{macrocode}
\newcommand{\childdocforwardprefix}[3][]
{
  \begingroup
    \def\childdocextract #2##1~~~{\def\childdoctmp{\childdocforward[#1]{#3##1}}}
    \expandafter\childdocextract\childdocname~~~
    \expandafter
  \endgroup
  \childdoctmp
}
%    \end{macrocode}

% \macro{\childdoc}
% The deprecated macro |\childdoc| is a legacy version of |\childdocmain|:
%    \begin{macrocode}
\newcommand{\childdoc}{\childdocmain}
%    \end{macrocode}

% \macro{\childdocredirect}
% The deprecated macro |\childdocredirect| is a legacy version
% of |\childdocforward| and |\childdocforwardprefix|:
%    \begin{macrocode}
\newcommand{\childdocredirect}[2][]
{
  \begingroup
    \if?#1?
      \def\childdoctmp{\childdocforward{#2}}
    \else
      \def\childdoctmp{\childdocforwardprefix{#1}{#2}}
    \fi
    \expandafter
  \endgroup
  \childdoctmp
}
%    \end{macrocode}

%\iffalse
%</package>
%\fi
%
\endinput
|\\
|\childdocforward{|\textit{main}|}|
\end{tabular}
\end{center}
%
Likewise, the following files |final|\textit{nn}|.tex|
compile the final version of the child document
|child|\textit{nn}|.tex|:
%
\begin{center}
\begin{tabular}{l}
|\def\version{final}|\\
|% \iffalse
%
% childdoc.dtx Copyright (C) 2017-2018 Niklas Beisert
%
% This work may be distributed and/or modified under the
% conditions of the LaTeX Project Public License, either version 1.3
% of this license or (at your option) any later version.
% The latest version of this license is in
%   http://www.latex-project.org/lppl.txt
% and version 1.3 or later is part of all distributions of LaTeX
% version 2005/12/01 or later.
%
% This work has the LPPL maintenance status `maintained'.
%
% The Current Maintainer of this work is Niklas Beisert.
%
% This work consists of the files childdoc.dtx and childdoc.ins
% and the derived files childdoc.def and cdocsamp.tex with
% cdocsch1.tex, cdocsch2.tex, cdocsdrf.tex, cdocsfn1.tex, cdocsfn2.tex.
%
%<package>\ifdefined\childdocmain\endinput\fi
%<package>\ProvidesFile{childdoc.def}[2018/12/30 v2.0 child document driver]
%<samplemain>\ProvidesFile{cdocsamp.tex}[2018/12/30 v2.0 sample for childdoc]
%<*driver>
%\ProvidesFile{childdoc.drv}[2018/12/30 v2.0 childdoc reference manual file]
\PassOptionsToClass{10pt,a4paper}{article}
\documentclass{ltxdoc}

\usepackage[margin=35mm]{geometry}
\usepackage{hyperref}
\usepackage{hyperxmp}
\usepackage[usenames]{color}

\hypersetup{colorlinks=true}
\hypersetup{pdfstartview=FitH}
\hypersetup{pdfpagemode=UseNone}
\hypersetup{pdfsource={}}
\hypersetup{pdflang={en-UK}}
\hypersetup{pdfcopyright={Copyright 2017-2018 Niklas Beisert.
  This work may be distributed and/or modified under the
  conditions of the LaTeX Project Public License, either version 1.3
  of this license or (at your option) any later version.}}
\hypersetup{pdflicenseurl={http://www.latex-project.org/lppl.txt}}
\hypersetup{pdfcontactaddress={ETH Zurich, ITP, HIT K,
  Wolfgang-Pauli-Strasse 27}}
\hypersetup{pdfcontactpostcode={8093}}
\hypersetup{pdfcontactcity={Zurich}}
\hypersetup{pdfcontactcountry={Switzerland}}
\hypersetup{pdfcontactemail={nbeisert@itp.phys.ethz.ch}}
\hypersetup{pdfcontacturl={http://people.phys.ethz.ch/\xmptilde nbeisert/}}

\newcommand{\secref}[1]{\hyperref[#1]{section \ref*{#1}}}

\parskip1ex
\parindent0pt
\let\olditemize\itemize
\def\itemize{\olditemize\parskip0pt}

\begin{document}

\title{The \textsf{childdoc} Package}
\hypersetup{pdftitle={The childdoc Package}}
\author{Niklas Beisert\\[2ex]
  Institut f\"ur Theoretische Physik\\
  Eidgen\"ossische Technische Hochschule Z\"urich\\
  Wolfgang-Pauli-Strasse 27, 8093 Z\"urich, Switzerland\\[1ex]
  \href{mailto:nbeisert@itp.phys.ethz.ch}
  {\texttt{nbeisert@itp.phys.ethz.ch}}}
\hypersetup{pdfauthor={Niklas Beisert}}
\hypersetup{pdfsubject={Manual for the LaTeX2e Package childdoc}}
\date{30 December 2018, \textsf{v2.0}}
\maketitle

\begin{abstract}\noindent
\textsf{childdoc} is a \LaTeXe{} package
that enables the direct compilation
of document sections included by |\include|
to individual files.
\end{abstract}

\begingroup
\parskip0ex
\tableofcontents
\endgroup

%%%%%%%%%%%%%%%%%%%%%%%%%%%%%%%%%%%%%%%%%%%%%%%%%%%%%%%%%%%%%%%%%%%%%%%%%%%%%%%%
%%%%%%%%%%%%%%%%%%%%%%%%%%%%%%%%%%%%%%%%%%%%%%%%%%%%%%%%%%%%%%%%%%%%%%%%%%%%%%%%
\section{Introduction}

\LaTeX{} provides a mechanism to structure a large document (such as a book)
into a main file and several child files (containing the chapters)
using the |\include| command.
This mechanism is beneficial for documents
which span hundreds of pages in order to
make the source file(s) more manageable.
Moreover, compilation can be restricted to
selected child files by means of the |\includeonly| command.
The latter feature can be used to reduce the compilation time while editing
(this was significantly more useful in the earlier days of \LaTeX{})
or to generate a smaller document which is easier to navigate.
Another application of |\includeonly| is to generate
documents consisting of selected parts of the complete document.

However, there are a few drawbacks of the plain |\include| mechanism:
\begin{itemize}
\item
The child files cannot be compiled on their own,
they can only be compiled via the main file.
A naive editing environment
(such as a text editor with an option
to have the current file processed by \LaTeX)
may require one to switch to the main file before compiling;
attempting to compile the child file produces errors.
\item
The main file must be modified (each time)
to adjust the |\includeonly| command
to the present needs. This easily leaves the main file in a messy state.
\item
The generated document will always carry the filename
of the main document. This is inconvenient if
several child files are to be compiled and
to be kept for distribution.
\end{itemize}

The present package provides a simple interface
to make child files individually compilable by \LaTeX{}.
Compiling a child file then has the same effect as compiling
the main file with an |\includeonly| command
to select the appropriate child.
Moreover the generated document will carry the name of the child
rather than the main file.
This resolves all three above issues.

This feature is meant to make the editing of books,
thesis documents and lecture notes somewhat more convenient.
However, the package can also be used efficiently for
composing a series of documents (such as exercise sheets)
which are typically distributed individually.
It then assists the author in generating the individual documents
(potentially in different versions)
as well as a document containing the collected series.
Another application is in developing style files
or other kinds of included material
where compilation of the style file could redirect
to a sample or test file.

%%%%%%%%%%%%%%%%%%%%%%%%%%%%%%%%%%%%%%%%%%%%%%%%%%%%%%%%%%%%%%%%%%%%%%%%%%%%%%%%
%%%%%%%%%%%%%%%%%%%%%%%%%%%%%%%%%%%%%%%%%%%%%%%%%%%%%%%%%%%%%%%%%%%%%%%%%%%%%%%%
\section{Usage}

First of all, the package \textsf{childdoc} is \emph{not} a standard
\LaTeXe{} |.sty| style file! Therefore it needs to be invoked in
a non-standard way.

%%%%%%%%%%%%%%%%%%%%%%%%%%%%%%%%%%%%%%%%%%%%%%%%%%%%%%%%%%%%%%%%%%%%%%%%%%%%%%%%
\subsection{Included Files}
\label{sec:include}

%%%%%%%%%%%%%%%%%%%%%%%%%%%%%%%%%%%%%%%%
\DescribeMacro{\childdocmain}
To use the package, add the commands
\begin{center}
\begin{tabular}{l}
|\input{childdoc.def}|\\
|\childdocmain{}|\\
\end{tabular}
\end{center}
at the very top of the main \LaTeX{} file,
in particular \emph{before} the |\documentclass| statement!
The argument of |\childdocmain| should be left empty
(but it must be present).

%%%%%%%%%%%%%%%%%%%%%%%%%%%%%%%%%%%%%%%%
\DescribeMacro{\childdocof}
Furthermore, add the commands
\begin{center}
\begin{tabular}{l}
|\input{childdoc.def}|\\
|\childdocof{|\textit{main}|}|\\
\end{tabular}
\end{center}
at the top of every child file \textit{child}
which is included by |\include{|\textit{child}|}|
from within the main file
(or at least for those files to be compiled individually).
The argument \textit{main} must be the filename of the main file.

There are a couple of
considerations in setting up the main and child documents:

%%%%%%%%%%%%%%%%%%%%%%%%%%%%%%%%%%%%%%%%
\paragraph{Restrictions.}

Please note the following restrictions:
\begin{itemize}
\item
|\childdocmain| must be called with one argument \textit{main}
to ensure compatibility with earlier version of the package.
It must either be empty (|\childdocmain{}|)
or precisely match the filename of the main file in which it is specified.
See \secref{sec:detection} for further information.
\item
The filename \textit{main} must be specified without the |.tex| extension.
\item
The filename \textit{main} is case sensitive
(even in case-insensitive file systems)
due to internal string comparison.
\item
The argument \textit{main} should be fully expanded, it cannot be a macro.
\item
Subdirectories and special characters should be avoided in filenames.
\item
The command |\childdocmain{|\textit{main}|}| must be followed by a whitespace.
It should not be followed immediately by another command
or by a comment mark `|%|'.
This is because the \TeX{} parser reads the token immediately following
the argument of |\childdocmain| and puts it
at the beginning of every child section;
however, a white\-space is ignored.
\end{itemize}

%%%%%%%%%%%%%%%%%%%%%%%%%%%%%%%%%%%%%%%%
\paragraph{Content of Main File.}

It is advisable to place all content in the child files included by |\include|.
Any output contained in the main file will appear in all child documents
unless suppressed manually;
it cannot be suppressed automatically by the |\includeonly| directive
and thus should normally be avoided.
A method to include some content in the main file
by means of conditional processing is described in \secref{sec:conditional}.

%%%%%%%%%%%%%%%%%%%%%%%%%%%%%%%%%%%%%%%%
\paragraph{Page Numbering.}

When only a part of the document is compiled,
the appropriate numbering of pages
(as well as other status parameters)
is determined from the |.aux| files.
The latter contain information from previous passes.
However this information needs to propagate through
all intermediate child documents.
Therefore the page numbering in child documents may well
be inconsistent until the complete document is compiled at least once.

A useful (if unconventional) way to always ensure a consistent
page numbering is to restart the numbering in each child document
and denote the pages by `\textit{child}|.|\textit{page}'
where \textit{child} represents the chapter/section number of the child file.
This can be achieved by the command
|\numberwithin{page}{|\textit{child}|}|
of the \textsf{amsmath} package
where \textit{child} can be |chapter| or |section|
depending on the chosen structuring.
Alternatively, one can modify the macro |\thepage| appropriately
and reset the counter |page| at the start of each child file.

%%%%%%%%%%%%%%%%%%%%%%%%%%%%%%%%%%%%%%%%%%%%%%%%%%%%%%%%%%%%%%%%%%%%%%%%%%%%%%%%
\subsection{Conditional Processing}
\label{sec:conditional}

The package provides a mechanism to compile different versions
of a document. To customise the versions further some conditional processing
can come in handy to distinguish which version is being compiled.
The package provides two macros to describe the compilation context:

%%%%%%%%%%%%%%%%%%%%%%%%%%%%%%%%%%%%%%%%
\DescribeMacro{\ifchilddoc}
The conditional |\ifchilddoc| distinguishes between the compilation of
child documents and the main document:
%
\begin{center}
|\ifchilddoc |\textit{child-code}| |[|\||else |\textit{main-code}]| \||fi|
\end{center}

%%%%%%%%%%%%%%%%%%%%%%%%%%%%%%%%%%%%%%%%
\DescribeMacro{\childdocname}
\DescribeMacro{\childdocjob}
The macro |\childdocname| contains the filename (without extension)
of the main or child file being processed.
Note that |\childdocjob| will always contain the name of the main file.

%%%%%%%%%%%%%%%%%%%%%%%%%%%%%%%%%%%%%%%%
\paragraph{Title Page.}

Conditional processing can be used to include a title or banner page
in the main document when proper precautions are taken.
Importantly, the code in the main file should ensure that the page counter
(as well as other status parameters which are stored in the |.aux| files)
takes the same value after the conditional processing.
Otherwise the page numbers may take divergent values
depending on which part is compiled.

For example, a title page could be declared by:
%
\begin{center}
\begin{tabular}{l}
|\ifchilddoc\||else|\\
|\addtocounter{page}{-1}|\\
\textit{code for title page}\\
|\newpage|\\
|\||fi|
\end{tabular}
\end{center}
%
A banner page for the child documents can be generated by:
%
\begin{center}
\begin{tabular}{l}
|\ifchilddoc|\\
|\addtocounter{page}{-1}|\\
\textit{code for banner page}\\
|\newpage|\\
|\||fi|
\end{tabular}
\end{center}
%
Here one could write a message such as:
\begin{center}
|This is the part \childdocname{} of \childdocjob{}.|
\end{center}

%%%%%%%%%%%%%%%%%%%%%%%%%%%%%%%%%%%%%%%%%%%%%%%%%%%%%%%%%%%%%%%%%%%%%%%%%%%%%%%%
\subsection{Flags}
\label{sec:flags}

The package makes it easy to generate different versions
of the main or child documents.
To this end compilation flags can be defined
and assigned different default values.
They will be particularly useful in conjunction
with the forwarding mechanism described in \secref{sec:forward}.

For example, it may be useful to have a flag |\version|
which can be set to |draft| or |final|.
The document source will contain some conditional code
depending on the value of |\version|.
Suppose further, the flag should default to |final| for the main file
and to |draft| for child files
which is a natural assignment for editing the document.
This is achieved by placing the following code
in the preamble of the main document
(below the |\childdocmain| directive):
%
\begin{center}
\begin{tabular}{l}
|\ifchilddoc|\\
|\providecommand{\version}{draft}|\\
|\||else|\\
|\providecommand{\version}{final}|\\
|\||fi|
\end{tabular}
\end{center}
%
The definition by |\providecommand| makes sure
that previous definitions are not overwritten.
Further statements |\providecommand{\version}{...}|
can thus be added before the above code to override it.

For the main file, one might add a line
(between |\childdocmain| and the above block)
%
\begin{center}
|%\ifchilddoc\||else\providecommand{\version}{draft}\||fi|
\end{center}
%
which can be uncommented to produce a draft version.
Likewise one can add a line to the very top of a child file
(above the |\childdocof{|\textit{main}|}| directive)
%
\begin{center}
|%\providecommand{\version}{final}|
\end{center}
%
which can be uncommented to produce the final version of this child document.

%%%%%%%%%%%%%%%%%%%%%%%%%%%%%%%%%%%%%%%%%%%%%%%%%%%%%%%%%%%%%%%%%%%%%%%%%%%%%%%%
\subsection{Forwarding}
\label{sec:forward}

Different versions of the main or child documents
using compilation flags as described in \secref{sec:flags}
can be (permanently) stored in different files
for convenient compilation, viewing and distribution.
To this end, the package defines a command
to pass on compilation to a different file:

%%%%%%%%%%%%%%%%%%%%%%%%%%%%%%%%%%%%%%%%
\DescribeMacro{\childdocforward}
The command |\childdocforward| redirects processing to
another source file:
%
\begin{center}
\begin{tabular}{l}
|\input{childdoc.def}|\\
|\childdocforward[|\textit{main}|]{|\textit{dest}|}|\\
\end{tabular}
\end{center}
%
The argument \textit{dest} is the destination file
(without extension).
It should be the main file or one of the child files.
Note that further \textsf{childdoc} directives
such as |\childdocof| and |\childdocforward|
in the indicated file will be processed in this form.
The optional argument \textit{main}
passes on directly to the main file \textit{main}
while pretending to compile the child \textit{dest}.
This form behaves as if \textit{dest}
issues |\childdocof{|\textit{main}|}| right away,
and no further \textsf{childdoc} directives will be processed.

%%%%%%%%%%%%%%%%%%%%%%%%%%%%%%%%%%%%%%%%
\DescribeMacro{\...prefix}
In the alternative form |\childdocforwardprefix|,
%
\begin{center}
\begin{tabular}{l}
|\input{childdoc.def}|\\
|\childdocforwardprefix[|\textit{main}|]{|\textit{prefix}|}{|\textit{dest}|}|
\end{tabular}
\end{center}
%
the destination file is determined by a pattern
depending on the current file:
To make this work, the current file must be called
`{\textit{prefix}\hspace{0.2em}\textit{suffix}}'
with \textit{prefix} matching precisely the argument.
Processing is then passed on to the file
`{\textit{dest}\hspace{0.2em}\textit{suffix}}'.
Surely, the same effect is achieved by
directly specifying the
argument `{\textit{dest}\hspace{0.2em}\textit{suffix}}'
in the first form.
However, that requires to set up a different file
for each child. With the alternative form of the command
all these files can have exactly the same content
which simplifies setting them up and maintaining them.

For example, the following file |draft.tex|
with a compilation flag |\version| as described in \secref{sec:flags}
compiles the main document as a draft:
%
\begin{center}
\begin{tabular}{l}
|\def\version{draft}|\\
|\input{childdoc.def}|\\
|\childdocforward{|\textit{main}|}|
\end{tabular}
\end{center}
%
Likewise, the following files |final|\textit{nn}|.tex|
compile the final version of the child document
|child|\textit{nn}|.tex|:
%
\begin{center}
\begin{tabular}{l}
|\def\version{final}|\\
|\input{childdoc.def}|\\
|\childdocforwardprefix{final}{child}|
\end{tabular}
\end{center}
%

Note that when several versions of a main file and/or of each child file
are to be generated, it may be convenient to set up a |Makefile| or
shell script to automatise the process.

%%%%%%%%%%%%%%%%%%%%%%%%%%%%%%%%%%%%%%%%%%%%%%%%%%%%%%%%%%%%%%%%%%%%%%%%%%%%%%%%
\subsection{Command Line Processing}
\label{sec:commandline}

The effect of redirection files can also be achieved by invoking
the \LaTeX{} compiler with a more elaborate command line.
Most conveniently this should be done as part
of a shell script or a |Makefile|.

When using \textsf{childdoc} in the main file, the following
command lines effectively perform a redirection
(note that depending on the shell being used,
backslashes may have to be doubled: `|\|' $\to$ `|\\|'):
%
\begin{center}
|... -jobname "|\textit{target}|" |\\|"|[\textit{flags}]%
|\input{childdoc.def}\childdocforward[|\textit{main}|]{|\textit{dest}|}"|
\end{center}
%
Here \textit{target} is the name of the output file,
\textit{main} is the name of the main file
and \textit{dest} is the name of the main or child file to be processed
(all filenames without extensions).
The optional argument \textit{main} can be omitted
if \textit{main} matches \textit{dest}.
Optionally, compilation \textit{flags} can be defined via |\def| commands.
This command line makes the \TeX{} engine believe
it is compiling the file \textit{target}
whose content is specified as the latter parameter.
The provided code then forwards the processing to
\textit{main} or \textit{dest} as described in \secref{sec:forward}.

%%%%%%%%%%%%%%%%%%%%%%%%%%%%%%%%%%%%%%%%%%%%%%%%%%%%%%%%%%%%%%%%%%%%%%%%%%%%%%%%
\subsection{Include by Input}
\label{sec:input}

Including child documents by |\include| has some restrictions by design.
Most notably, the content of a child document always occupies
its own set of pages; pages cannot be shared between child documents.
Usually, this behaviour makes perfect sense
because each child document contain an essential part of the document.
However, in some situations it may be desirable to compose
a document from a collection of parts
without having mandatory page breaks between then.
For this case, the package
provides a mechanism to include parts
by |\input| which can also be processed individually.
However, by construction this mechanism
requires manual handling of the content to be output.

%%%%%%%%%%%%%%%%%%%%%%%%%%%%%%%%%%%%%%%%
\DescribeMacro{\ifchilddocmanual}
The main file should be prepared as usual, see \secref{sec:include}.
However, the document body must make a distinction
between processing of an individual part and of the main document, e.g.:
%
\begin{center}
\begin{tabular}{l}
|\ifchilddocmanual|\\
|\input{\childdocname}|\\
|\||else|\\
\textit{document body with }|\input{|\textit{part}|}|\\
|\||fi|
\end{tabular}
\end{center}
%
The conditional |\ifchilddocmanual| is true whenever
a part to be included by |\input| is being compiled,
and the name of the part is stored in |\childdocname|.

%%%%%%%%%%%%%%%%%%%%%%%%%%%%%%%%%%%%%%%%
\DescribeMacro{\childdocby}
Each part to be included by |\input| should start with:
%
\begin{center}
\begin{tabular}{l}
|\input{childdoc.def}|\\
|\childdocby{|\textit{main}|}|\\
\end{tabular}
\end{center}
%
The directive |\childdocby| is similar to |\childdocof|
described in \secref{sec:include},
but the subsequent selection of content must be done manually.
To that end, both |\ifchilddoc| and |\ifchilddocmanual|
will be true upon processing of a part,
and the name of the part is stored in |\childdocname|.
Note that |\jobname| will be set to the filename of the current part
so that each part receives an individual |.aux| file
that does not interfere with the |.aux| file(s) of the main document.
This behaviour can be altered by the alternative form
|\childdocby[*]{|\textit{main}|}| (with a non-empty optional argument)
which uses the |.aux| file of the main document
by setting |\jobname| to \textit{main}.

%%%%%%%%%%%%%%%%%%%%%%%%%%%%%%%%%%%%%%%%%%%%%%%%%%%%%%%%%%%%%%%%%%%%%%%%%%%%%%%%
\subsection{Driver Development}
\label{sec:driver}

The \textsf{childdoc} mechanism can also be use for the development
of definition files such as \LaTeX{} styles or classes.
This case differs from the above setup with multiple parts
included by |\include| in that no |\includeonly| should be invoked.
This can be achieved by starting the include file
(before |\ProvidesPackage|) with:
%
\begin{center}
\begin{tabular}{l}
|\input{childdoc.def}|\\
|\childdocforward{|\textit{main}|}|\\
\end{tabular}
\end{center}
%
or alternatively with:
%
\begin{center}
\begin{tabular}{l}
|\input{childdoc.def}|\\
|\childdocby{|\textit{main}|}|\\
\end{tabular}
\end{center}
%
Both forms have slightly different effects as described above.
The main file is prepared as usual, see \secref{sec:include}.

%%%%%%%%%%%%%%%%%%%%%%%%%%%%%%%%%%%%%%%%%%%%%%%%%%%%%%%%%%%%%%%%%%%%%%%%%%%%%%%%
\subsection{Legacy Detection}
\label{sec:detection}

The directive |\childdocmain| in the main file can detect
whether the complete document or merely a child is to be compiled
even without using the directive |\childdocof|.
This method is deprecated because it is less robust
and there is no compelling reason to use it;
it is merely provided for backward compatibility
and it may be removed in future versions.

If the detection mechanism is to be used,
it is mandatory to correctly specify
the filename of the main file as the argument of |\childdocmain|:
%
\begin{center}
\begin{tabular}{l}
|\input{childdoc.def}|\\
|\childdocmain{|\textit{main}|}|\\
\end{tabular}
\end{center}
%
If |\jobname| does not match the argument \textit{main} of |\childdocmain|,
it is assumed that |\jobname| points to the child file to be compiled.
When using |\childdocmain| with the main file specified as argument,
it suffices to start a child file
with just |\input{|\textit{main}|}|
without loading of the package and using |\childdocof|.
If instead all processing is done
with the appropriate \textsf{childdoc} directives,
the argument of \textit{main} of |\childdocmain| can be empty.

An alternative version of the command line processing described
in \secref{sec:commandline} using the detection mechanism reads:
%
\begin{center}
|... -jobname "|\textit{target}|" "|[\textit{flags}]%
[|\def\jobname{|\textit{dest}|}|]|\input{|\textit{main}|}"|
\end{center}

%%%%%%%%%%%%%%%%%%%%%%%%%%%%%%%%%%%%%%%%%%%%%%%%%%%%%%%%%%%%%%%%%%%%%%%%%%%%%%%%
\subsection{Manual Code}
\label{sec:manual}

In case one cannot be certain whether the definitions file |childdoc.def|
is installed on the target \TeX{} distribution
and one prefers not to ship it,
it is conceivable to paste a few relevant commands into the sources.

To that end, drop all statements |\input{childdoc.def}|
and perform the replacements as outlined below.
Instead of |\childdocmain{|\textit{main}|}| add the following code
to the top of the main file:
%
\begin{center}
\begin{tabular}{l}
|\||ifdefined\childdocname\endinput\||fi\newif\ifchilddoc|\\
|\edef\childdocname{\scantokens\expandafter{\jobname\noexpand}}|\\
|\def\childdocmain{|\textit{main}|}\||ifx\childdocmain\childdocname\||else|\\
|\childdoctrue\includeonly{\childdocname}\let\jobname\childdocmain\||fi|\\
\end{tabular}
\end{center}
%
Instead of |\childdocof{|\textit{main}|}| just include the main file
at the top of each child file:
%
\begin{center}
|\input{|\textit{main}|}|
\end{center}
%
A simple redirection |\childdocforward{|\textit{dest}|}| is achieved by:
%
\begin{center}
|\def\jobname{|\textit{dest}|}\input{\jobname}|
\end{center}
%
The redirection with prefix
|\childdocforwardprefix[|\textit{prefix}|]{|\textit{dest}|}|
is accomplished by:
%
\begin{center}
\begin{tabular}{l}
|{\edef\jobname{\scantokens\expandafter{\jobname\noexpand}}|\\
|\def\redirectjob |\textit{prefix}|#1~~~{\gdef\jobname{|\textit{dest}|#1}}|\\
|\expandafter\redirectjob\jobname~~~}\input{\jobname}|
\end{tabular}
\end{center}

In an alternative approach,
child documents can be compiled by a specific command line
without additional code or specific definitions:
%
\begin{center}
|... -jobname "|\textit{target}|" "|[\textit{flags}]%
|\includeonly{|\textit{dest}|}\input{|\textit{main}|}"|
\end{center}
%

%%%%%%%%%%%%%%%%%%%%%%%%%%%%%%%%%%%%%%%%%%%%%%%%%%%%%%%%%%%%%%%%%%%%%%%%%%%%%%%%
%%%%%%%%%%%%%%%%%%%%%%%%%%%%%%%%%%%%%%%%%%%%%%%%%%%%%%%%%%%%%%%%%%%%%%%%%%%%%%%%
\section{Information}

%%%%%%%%%%%%%%%%%%%%%%%%%%%%%%%%%%%%%%%%%%%%%%%%%%%%%%%%%%%%%%%%%%%%%%%%%%%%%%%%
\subsection{Copyright}

Copyright \copyright{} 2017--2018 Niklas Beisert

This work may be distributed and/or modified under the
conditions of the \LaTeX{} Project Public License, either version 1.3
of this license or (at your option) any later version.
The latest version of this license is in
  \url{http://www.latex-project.org/lppl.txt}
and version 1.3 or later is part of all distributions of \LaTeX{}
version 2005/12/01 or later.

This work has the LPPL maintenance status `maintained'.

The Current Maintainer of this work is Niklas Beisert.

This work consists of the files |README.txt|, |childdoc.ins| and |childdoc.dtx|
as well as the derived files |childdoc.def|, |cdocsamp.tex|
with |cdocsch1.tex|, |cdocsch2.tex|, |cdocspt3.tex|, |cdocspt4.tex|,
|cdocsdrf.tex|, |cdocsfn1.tex|, |cdocsfn2.tex|
as well as |childdoc.pdf|.

%%%%%%%%%%%%%%%%%%%%%%%%%%%%%%%%%%%%%%%%%%%%%%%%%%%%%%%%%%%%%%%%%%%%%%%%%%%%%%%%
\subsection{Files and Installation}

The package consists of the files:
%
\begin{center}
\begin{tabular}{ll}
    |README.txt|   & readme file \\
    |childdoc.ins| & installation file \\
    |childdoc.dtx| & source file \\
    |childdoc.def| & definition file \\
    |cdocsamp.tex| & sample main file \\
    |cdocsch1.tex| & sample include file \\
    |cdocsch2.tex| & sample include file \\
    |cdocspt3.tex| & sample part file \\
    |cdocspt4.tex| & sample part file \\
    |cdocsdrf.tex| & sample redirection file \\
    |cdocsfn1.tex| & sample redirection file \\
    |cdocsfn2.tex| & sample redirection file \\
    |childdoc.pdf| & manual
\end{tabular}
\end{center}
%
The distribution consists of the files
|README.txt|, |childdoc.ins| and |childdoc.dtx|.
%
\begin{itemize}
\item
Run (pdf)\LaTeX{} on |childdoc.dtx|
to compile the manual |childdoc.pdf| (this file).
\item
Run \LaTeX{} on |childdoc.ins| to create the definitions file |childdoc.def|
and the sample |cdocsamp.tex| with include files
|cdocsch1.tex|, |cdocsch2.tex|, |cdocspt3.tex|, |cdocspt4.tex|,
|cdocsdrf.tex|, |cdocsfn1.tex|, |cdocsfn2.tex|.
Then copy the file |childdoc.def| to an appropriate directory of your \LaTeX{}
distribution, e.g.\ \textit{texmf-root}|/tex/latex/childdoc|.
\end{itemize}

%%%%%%%%%%%%%%%%%%%%%%%%%%%%%%%%%%%%%%%%%%%%%%%%%%%%%%%%%%%%%%%%%%%%%%%%%%%%%%%%
\subsection{Related CTAN Packages}

There are several other packages which offer a similar functionality:
%
\begin{itemize}
\item
The packages
\href{http://ctan.org/pkg/docmute}{\textsf{docmute}},
\href{http://ctan.org/pkg/includex}{\textsf{includex}} and
\href{http://ctan.org/pkg/standalone}{\textsf{standalone}}
provide commands to include only the document body of
a child file thus allowing both files to be compiled individually.
\item
The packages \href{http://ctan.org/pkg/subdocs}{\textsf{subdocs}}
and \href{http://ctan.org/pkg/subfiles}{\textsf{subfiles}}
provide structures in which the main and child documents can be
encapsulated and allowing them to be compiled individually.
The inclusion mechanism is different from the conventional |\include|.
\item
The package \href{http://ctan.org/pkg/combine}{\textsf{combine}}
is an elaborate solution to combine several documents into one.
\end{itemize}
%
See also the CTAN topic \href{http://ctan.org/topic/subdocs}{\textsf{subdocs}}
for further related packages.
The present package differs from the above solutions in that
a document structure constructed with the conventional |\include| mechanism
just needs two extra commands at the top of every file
such that all constituent files can be compiled individually.

%%%%%%%%%%%%%%%%%%%%%%%%%%%%%%%%%%%%%%%%%%%%%%%%%%%%%%%%%%%%%%%%%%%%%%%%%%%%%%%%
%\subsection{Feature Suggestions}
%
%The following is a list of features which may be useful for future
%versions of this package:
%%
%\begin{itemize}
%\item
%\ldots
%\end{itemize}

%%%%%%%%%%%%%%%%%%%%%%%%%%%%%%%%%%%%%%%%%%%%%%%%%%%%%%%%%%%%%%%%%%%%%%%%%%%%%%%%
\subsection{Revision History}

%%%%%%%%%%%%%%%%%%%%%%%%%%%%%%%%%%%%%%%%
\paragraph{v2.0:} 2018/12/30

\begin{itemize}
\item
immediate forward processing
\item
added |\childdocby| mechanism
\item
manual restructured
\end{itemize}

%%%%%%%%%%%%%%%%%%%%%%%%%%%%%%%%%%%%%%%%
\paragraph{v1.6:} 2018/01/17

\begin{itemize}
\item
application for development of include files
\item
corrections to manual
\end{itemize}

%%%%%%%%%%%%%%%%%%%%%%%%%%%%%%%%%%%%%%%%
\paragraph{v1.5:} 2017/05/21

\begin{itemize}
\item
more complete structuring introduced
\item
|\childdocof| introduced
\item
|\childdoc| renamed to |\childdocmain|
\item
|\childredirect| renamed to |\childdocforward| and |\childdocforwardprefix|
and functionality expanded
\end{itemize}

%%%%%%%%%%%%%%%%%%%%%%%%%%%%%%%%%%%%%%%%
\paragraph{v1.0:} 2017/04/27

\begin{itemize}
\item
manual and install package
\item
first version published on CTAN
\end{itemize}

%%%%%%%%%%%%%%%%%%%%%%%%%%%%%%%%%%%%%%%%
\paragraph{v0.6:} 2017/04/26

\begin{itemize}
\item
redirection mechanism added
\end{itemize}

%%%%%%%%%%%%%%%%%%%%%%%%%%%%%%%%%%%%%%%%
\paragraph{v0.5:} 2017/04/26

\begin{itemize}
\item
functionality in definition file
\end{itemize}


%%%%%%%%%%%%%%%%%%%%%%%%%%%%%%%%%%%%%%%%%%%%%%%%%%%%%%%%%%%%%%%%%%%%%%%%%%%%%%%%
%%%%%%%%%%%%%%%%%%%%%%%%%%%%%%%%%%%%%%%%%%%%%%%%%%%%%%%%%%%%%%%%%%%%%%%%%%%%%%%%
%%%%%%%%%%%%%%%%%%%%%%%%%%%%%%%%%%%%%%%%%%%%%%%%%%%%%%%%%%%%%%%%%%%%%%%%%%%%%%%%
\appendix

\settowidth\MacroIndent{\rmfamily\scriptsize 000\ }

 \DocInput{childdoc.dtx}

\end{document}
%</driver>
% \fi
%
% %%%%%%%%%%%%%%%%%%%%%%%%%%%%%%%%%%%%%%%%%%%%%%%%%%%%%%%%%%%%%%%%%%%%%%%%%%%%%%
% %%%%%%%%%%%%%%%%%%%%%%%%%%%%%%%%%%%%%%%%%%%%%%%%%%%%%%%%%%%%%%%%%%%%%%%%%%%%%%
% \section{Sample}
%\iffalse
%<*samplemain>
%\fi
%
% The following presents a sample document
% with two chapters, two parts, a title page,
% a compile flag as well as three forwarding files to set the flag.
% It consists of eight |.tex| files:
% \begin{center}
% \begin{tabular}{ll}
% |cdocsamp.tex|&main file\\
% |cdocsch1.tex|&include file for chapter 1\\
% |cdocsch2.tex|&include file for chapter 2\\
% |cdocspt3.tex|&include file for part 3\\
% |cdocspt4.tex|&include file for part 4\\
% |cdocsdrf.tex|&forwarding file for main file in draft mode\\
% |cdocsfi1.tex|&forwarding file for final version of chapter 1\\
% |cdocsfi2.tex|&forwarding file for final version of chapter 2\\
% \end{tabular}
% \end{center}
% Each of the eight files can be compiled directly by the \LaTeX{} compiler.
%
% %%%%%%%%%%%%%%%%%%%%%%%%%%%%%%%%%%%%%%
% \paragraph{Main File.}
%
% The main file is called |cdocsamp.tex|.
%
% Load the \textsf{childdoc} definitions and
% declare the filename for the main document:
%    \begin{macrocode}
\input{childdoc.def}
\childdocmain{}
%    \end{macrocode}

% Optional override for |\version| flag:
%    \begin{macrocode}
%%\ifchilddoc\else\providecommand{\version}{draft}\fi
%    \end{macrocode}

% Define the default values for the |\version| flag
% (|final| for the main file and |draft| for childs):
%    \begin{macrocode}
\ifchilddoc
\providecommand{\version}{draft}
\else
\providecommand{\version}{final}
\fi
%    \end{macrocode}

% Load the standard document class:
%    \begin{macrocode}
\documentclass[12pt]{article}
%    \end{macrocode}

% Start the document body:
%    \begin{macrocode}
\begin{document}
%    \end{macrocode}

% Declare a title page.
% Print title, part of document being processed and version flag:
%    \begin{macrocode}
\addtocounter{page}{-1}
\begin{center}
{\LARGE\bfseries{}childdoc example\par}
\vspace{1cm}
\ifchilddoc
\ifchilddocmanual part\else chapter\fi:
`\childdocname' of `\childdocjob'\par
\else
main document: `\childdocjob'\par
\fi
version: \version\par
\end{center}
\newpage
%    \end{macrocode}

% Manually include selected file,
% otherwise process as usual:
%    \begin{macrocode}
\ifchilddocmanual
\section*{part `\childdocname'}
\input{\childdocname}
\else
%    \end{macrocode}

% Include the two chapters:
%    \begin{macrocode}
\include{cdocsch1}
\include{cdocsch2}
%    \end{macrocode}

% Include the two parts unless only chapters should be displayed:
%    \begin{macrocode}
\ifchilddoc\else
\section{part three}
\input{cdocspt3}
\section{part four}
\input{cdocspt4}
\fi
%    \end{macrocode}

% Process as usual until here:
%    \begin{macrocode}
\fi
%    \end{macrocode}

% End of document body:
%    \begin{macrocode}
\end{document}
%    \end{macrocode}
%\iffalse
%</samplemain>
%\fi
%
% %%%%%%%%%%%%%%%%%%%%%%%%%%%%%%%%%%%%%%
% \paragraph{Chapter Include Files.}
%
% The include files are called |cdocsch1.tex| and |cdocsch2.tex|.
%
%\iffalse
%<*samplechap1|samplechap2>
%\fi

% Optional override for |\version| flag:
%    \begin{macrocode}
%%\providecommand{\version}{final}
%    \end{macrocode}

% Include the main document:
%    \begin{macrocode}
\input{childdoc.def}
\childdocof{cdocsamp}
%    \end{macrocode}

%\iffalse
%</samplechap1|samplechap2>
%\fi
%
%\iffalse
%<*samplechap1>
%\fi
% Some text for chapter 1:
%    \begin{macrocode}
\section{one}
some text in chapter one
%    \end{macrocode}

%\iffalse
%</samplechap1>
%\fi
% Some text for chapter 2:
%\iffalse
%<*samplechap2>
%\fi
%    \begin{macrocode}
\section{two}
more text in chapter two
%    \end{macrocode}

%\iffalse
%</samplechap2>
%\fi
%
% %%%%%%%%%%%%%%%%%%%%%%%%%%%%%%%%%%%%%%
% \paragraph{Part Include Files.}
%
% The include files are called |cdocspt3.tex| and |cdocspt4.tex|.
%
%\iffalse
%<*samplepart3|samplepart4>
%\fi

% Optional override for |\version| flag:
%    \begin{macrocode}
%%\providecommand{\version}{final}
%    \end{macrocode}

% Include the main document:
%    \begin{macrocode}
\input{childdoc.def}
\childdocby{cdocsamp}
%    \end{macrocode}

%\iffalse
%</samplepart3|samplepart4>
%\fi
%
%\iffalse
%<*samplepart3>
%\fi
% Some text for part 3:
%    \begin{macrocode}
some text in part three
%    \end{macrocode}

%\iffalse
%</samplepart3>
%\fi
% Some text for part 4:
%\iffalse
%<*samplepart4>
%\fi
%    \begin{macrocode}
more text in part four
%    \end{macrocode}

%\iffalse
%</samplepart4>
%\fi
%
% %%%%%%%%%%%%%%%%%%%%%%%%%%%%%%%%%%%%%%
% \paragraph{Forwarding for a Complete Draft.}
%
% The following forwarding file |cdocsdrf.tex|
% compiles the main document in draft mode:
%\iffalse
%<*sampledraft>
%\fi
%    \begin{macrocode}
\def\version{draft}
\input{childdoc.def}
\childdocforward{cdocsamp}
%    \end{macrocode}

%\iffalse
%</sampledraft>
%\fi
%
% %%%%%%%%%%%%%%%%%%%%%%%%%%%%%%%%%%%%%%
% \paragraph{Forwarding for Final Version of the Chapters.}
%
% The following forwarding files |cdocsfn1.tex| and |cdocsfn2.tex|
% (with identical content)
% compile the final versions of the child documents
% |cdocsch1.tex| and |cdocsch2.tex|, respectively:
%\iffalse
%<*samplefinal>
%\fi
%    \begin{macrocode}
\def\version{final}
\input{childdoc.def}
\childdocforwardprefix[cdocsamp]{cdocsfn}{cdocsch}
%    \end{macrocode}

%\iffalse
%</samplefinal>
%\fi
%
% %%%%%%%%%%%%%%%%%%%%%%%%%%%%%%%%%%%%%%
% \paragraph{Command Line Processing.}
%
% The following three command lines generate the output files
% |cdocscld|, |cdocscl1| and |cdocscl2|
% which should be identical to
% |cdocsdrf|, |cdocsch1| and |cdocsfn2|, respectively:
% \begin{center}
% \begin{tabular}{l}
% |latex -jobname cdocscld \|\\
% |  "\def\version{draft}\input{childdoc.def}\childdocforward{cdocsamp}"|\\
% |latex -jobname cdocscl1 \|\\
% |  "\input{childdoc.def}\childdocforward[cdocsamp]{cdocsch1}"|\\
% |latex -jobname cdocscl2 \|\\
% |  "\def\version{final}\input{childdoc.def}\childdocforward{cdocsch2}"|
% \end{tabular}
% \end{center}
% Note that the trailing backslash on each first line
% merely continues the input to the second line
% (for convenient cut ant paste).
% Furthermore, the command |latex| can be replaced by any
% of its alternative versions such as |pdflatex|.
%
% %%%%%%%%%%%%%%%%%%%%%%%%%%%%%%%%%%%%%%%%%%%%%%%%%%%%%%%%%%%%%%%%%%%%%%%%%%%%%%
% %%%%%%%%%%%%%%%%%%%%%%%%%%%%%%%%%%%%%%%%%%%%%%%%%%%%%%%%%%%%%%%%%%%%%%%%%%%%%%
% \section{Implementation}
%\iffalse
%<*package>
%\fi
%
% This section describes the definitions file |childdoc.def|.

% The definitions cannot be loaded using |\usepackage| or |\RequirePackage|
% which has a mechanism to prevent loading a style file more than once.
% When loading the definitions by means of |\input|
% multiple instances have to be prevented manually:
%\iffalse
%This code needs to be before the `\ProvidesFile' directive
%which is defined at the beginning of this file.
%Therefore it is also placed there and commented out here.
%</package>
%<*discard>
%\fi
%    \begin{macrocode}
\ifdefined\childdocmain\endinput\fi
%    \end{macrocode}
%\iffalse
%</discard>
%<*package>
%\fi
%
% \macro{\ifchilddoc}
% \macro{\ifchilddocmanual}
% The conditional |\ifchilddoc| tells whether a
% child (true) or main (false) document is being compiled.
% The conditional |\ifchilddocmanual| tells whether
% the |\includeonly| mechanism is used (false) or
% the selection of child files must be performed manually (true).
% The definitions initialise to false:
%    \begin{macrocode}
\newif\ifchilddoc
\newif\ifchilddocmanual
%    \end{macrocode}

% \macro{\childdocname}
% \macro{\childdocjob}
% The macro |\childdocname| stores the name of the main document
% to be compiled. The macro |\childdocjob| stores the name of
% the document on which the \LaTeX{} compiler was originally invoked.
% The content of |\jobname| cannot be compared
% to filenames specified in the source due to different catcodes.
% The following code rescans |\jobname|, stores the result
% in |\childdocname| and saves a copy in |\childdocjob|:
%    \begin{macrocode}
\edef\childdocname{\scantokens\expandafter{\jobname\noexpand}}
\let\childdocjob\childdocname
%    \end{macrocode}

% \macro{\childdocdisable}
% The macro |\childdocdisable| prevents the main file
% from being processed more than once.
% At this stage, the main document command |\childdocmain|
% is assumed to be called once again where it should do nothing.
% Any subsequent call to it should prevent
% a secondary processing of the main document
% It overwrites the forwarding commands
% |\childdocof| and |\childdocforward|
% with empty macros to prevent further inclusions of the main document:
%    \begin{macrocode}
\newcommand{\childdocdisable}
{
  \renewcommand{\childdocmain}[1]{\renewcommand{\childdocmain}[1]{\endinput}}
  \renewcommand{\childdocof}[1]{}
  \renewcommand{\childdocby}[2][]{}
  \renewcommand{\childdocforward}[2][]{}
  \renewcommand{\childdocdisable}{}
}
%    \end{macrocode}

% \macro{\childdocmain}
% The macro |\childdocmain| is to be called at the top of the main file
% with nothing or the main filename (without extension) as argument.
% First, it breaks loops.
% If the argument is not empty and does not match |\childdocname|
% (which is set by the first inclusion of |childdoc.def|),
% |\ifchilddoc| is set to true, |\includeonly| is applied to the child file
% and |\jobname| is set to the main file
% (for proper handling of |.aux| files):
%    \begin{macrocode}
\newcommand{\childdocmain}[1]
{
  \childdocdisable\childdocmain{}
  \if?#1?\else
    \begingroup
      \def\childdoctmp{#1}
      \ifx\childdoctmp\childdocname
        \def\childdoctmp{}
      \else
        \def\childdoctmp
        {
          \childdoctrue
          \includeonly{\childdocname}
          \def\childdocjob{#1}
          \def\jobname{#1}
        }
      \fi
      \expandafter
    \endgroup
    \childdoctmp
  \fi
}
%    \end{macrocode}

% \macro{\childdocof}
% The command |\childdocof| redirects
% compilation to the main file |#1|.
%    \begin{macrocode}
\newcommand{\childdocof}[1]
{
  \childdocdisable
  \childdoctrue
  \includeonly{\childdocname}
  \def\jobname{#1}
  \def\childdocjob{#1}
  \input{#1}
}
%    \end{macrocode}

% \macro{\childdocby}
% The command |\childdocby| ....
%    \begin{macrocode}
\newcommand{\childdocby}[2][]
{
  \childdocdisable
  \childdoctrue
  \childdocmanualtrue
  \if?#1?\else
    \def\jobname{#2}
  \fi
  \def\childdocjob{#2}
  \input{#2}
  \endinput
}
%    \end{macrocode}

% \macro{\childdocforward}
% The command |\childdocforward| redirects
% compilation to the main file or
% (if the optional argument is given) a child file.
% Parameters are set as if the main file
% or a child file starting with |\childdocof| was compiled.
% Then compilation is handed over to the main file:
%    \begin{macrocode}
\newcommand{\childdocforward}[2][]
{
  \begingroup
    \if?#1?
      \def\childdoctmp
      {
        \def\childdocname{#2}
        \def\childdocjob{#2}
        \def\jobname{#2}
        \input{#2}
        \endinput
      }
    \else
      \def\childdoctmp
      {
        \childdocdisable
        \def\childdocname{#2}
        \childdoctrue
        \includeonly{#2}
        \def\childdocjob{#1}
        \def\jobname{#1}
        \input{#1}
        \endinput
      }
    \fi
    \expandafter
  \endgroup
  \childdoctmp
}
%    \end{macrocode}

% \macro{\childdocforwardprefix}
% The command |\childdocforwardprefix| redirects
% compilation to the main or a child file by means of a pattern.
% The prefix |#1| in the current filename is replaced by |#2|
% and the suffix of the current filename is kept
% (it is assumed that the filename does not contain the substring `|~~~|'
% which is used as a delimiter).
% Compilation is handed over to the new file by |\childdocforward|:
%    \begin{macrocode}
\newcommand{\childdocforwardprefix}[3][]
{
  \begingroup
    \def\childdocextract #2##1~~~{\def\childdoctmp{\childdocforward[#1]{#3##1}}}
    \expandafter\childdocextract\childdocname~~~
    \expandafter
  \endgroup
  \childdoctmp
}
%    \end{macrocode}

% \macro{\childdoc}
% The deprecated macro |\childdoc| is a legacy version of |\childdocmain|:
%    \begin{macrocode}
\newcommand{\childdoc}{\childdocmain}
%    \end{macrocode}

% \macro{\childdocredirect}
% The deprecated macro |\childdocredirect| is a legacy version
% of |\childdocforward| and |\childdocforwardprefix|:
%    \begin{macrocode}
\newcommand{\childdocredirect}[2][]
{
  \begingroup
    \if?#1?
      \def\childdoctmp{\childdocforward{#2}}
    \else
      \def\childdoctmp{\childdocforwardprefix{#1}{#2}}
    \fi
    \expandafter
  \endgroup
  \childdoctmp
}
%    \end{macrocode}

%\iffalse
%</package>
%\fi
%
\endinput
|\\
|\childdocforwardprefix{final}{child}|
\end{tabular}
\end{center}
%

Note that when several versions of a main file and/or of each child file
are to be generated, it may be convenient to set up a |Makefile| or
shell script to automatise the process.

%%%%%%%%%%%%%%%%%%%%%%%%%%%%%%%%%%%%%%%%%%%%%%%%%%%%%%%%%%%%%%%%%%%%%%%%%%%%%%%%
\subsection{Command Line Processing}
\label{sec:commandline}

The effect of redirection files can also be achieved by invoking
the \LaTeX{} compiler with a more elaborate command line.
Most conveniently this should be done as part
of a shell script or a |Makefile|.

When using \textsf{childdoc} in the main file, the following
command lines effectively perform a redirection
(note that depending on the shell being used,
backslashes may have to be doubled: `|\|' $\to$ `|\\|'):
%
\begin{center}
|... -jobname "|\textit{target}|" |\\|"|[\textit{flags}]%
|% \iffalse
%
% childdoc.dtx Copyright (C) 2017-2018 Niklas Beisert
%
% This work may be distributed and/or modified under the
% conditions of the LaTeX Project Public License, either version 1.3
% of this license or (at your option) any later version.
% The latest version of this license is in
%   http://www.latex-project.org/lppl.txt
% and version 1.3 or later is part of all distributions of LaTeX
% version 2005/12/01 or later.
%
% This work has the LPPL maintenance status `maintained'.
%
% The Current Maintainer of this work is Niklas Beisert.
%
% This work consists of the files childdoc.dtx and childdoc.ins
% and the derived files childdoc.def and cdocsamp.tex with
% cdocsch1.tex, cdocsch2.tex, cdocsdrf.tex, cdocsfn1.tex, cdocsfn2.tex.
%
%<package>\ifdefined\childdocmain\endinput\fi
%<package>\ProvidesFile{childdoc.def}[2018/12/30 v2.0 child document driver]
%<samplemain>\ProvidesFile{cdocsamp.tex}[2018/12/30 v2.0 sample for childdoc]
%<*driver>
%\ProvidesFile{childdoc.drv}[2018/12/30 v2.0 childdoc reference manual file]
\PassOptionsToClass{10pt,a4paper}{article}
\documentclass{ltxdoc}

\usepackage[margin=35mm]{geometry}
\usepackage{hyperref}
\usepackage{hyperxmp}
\usepackage[usenames]{color}

\hypersetup{colorlinks=true}
\hypersetup{pdfstartview=FitH}
\hypersetup{pdfpagemode=UseNone}
\hypersetup{pdfsource={}}
\hypersetup{pdflang={en-UK}}
\hypersetup{pdfcopyright={Copyright 2017-2018 Niklas Beisert.
  This work may be distributed and/or modified under the
  conditions of the LaTeX Project Public License, either version 1.3
  of this license or (at your option) any later version.}}
\hypersetup{pdflicenseurl={http://www.latex-project.org/lppl.txt}}
\hypersetup{pdfcontactaddress={ETH Zurich, ITP, HIT K,
  Wolfgang-Pauli-Strasse 27}}
\hypersetup{pdfcontactpostcode={8093}}
\hypersetup{pdfcontactcity={Zurich}}
\hypersetup{pdfcontactcountry={Switzerland}}
\hypersetup{pdfcontactemail={nbeisert@itp.phys.ethz.ch}}
\hypersetup{pdfcontacturl={http://people.phys.ethz.ch/\xmptilde nbeisert/}}

\newcommand{\secref}[1]{\hyperref[#1]{section \ref*{#1}}}

\parskip1ex
\parindent0pt
\let\olditemize\itemize
\def\itemize{\olditemize\parskip0pt}

\begin{document}

\title{The \textsf{childdoc} Package}
\hypersetup{pdftitle={The childdoc Package}}
\author{Niklas Beisert\\[2ex]
  Institut f\"ur Theoretische Physik\\
  Eidgen\"ossische Technische Hochschule Z\"urich\\
  Wolfgang-Pauli-Strasse 27, 8093 Z\"urich, Switzerland\\[1ex]
  \href{mailto:nbeisert@itp.phys.ethz.ch}
  {\texttt{nbeisert@itp.phys.ethz.ch}}}
\hypersetup{pdfauthor={Niklas Beisert}}
\hypersetup{pdfsubject={Manual for the LaTeX2e Package childdoc}}
\date{30 December 2018, \textsf{v2.0}}
\maketitle

\begin{abstract}\noindent
\textsf{childdoc} is a \LaTeXe{} package
that enables the direct compilation
of document sections included by |\include|
to individual files.
\end{abstract}

\begingroup
\parskip0ex
\tableofcontents
\endgroup

%%%%%%%%%%%%%%%%%%%%%%%%%%%%%%%%%%%%%%%%%%%%%%%%%%%%%%%%%%%%%%%%%%%%%%%%%%%%%%%%
%%%%%%%%%%%%%%%%%%%%%%%%%%%%%%%%%%%%%%%%%%%%%%%%%%%%%%%%%%%%%%%%%%%%%%%%%%%%%%%%
\section{Introduction}

\LaTeX{} provides a mechanism to structure a large document (such as a book)
into a main file and several child files (containing the chapters)
using the |\include| command.
This mechanism is beneficial for documents
which span hundreds of pages in order to
make the source file(s) more manageable.
Moreover, compilation can be restricted to
selected child files by means of the |\includeonly| command.
The latter feature can be used to reduce the compilation time while editing
(this was significantly more useful in the earlier days of \LaTeX{})
or to generate a smaller document which is easier to navigate.
Another application of |\includeonly| is to generate
documents consisting of selected parts of the complete document.

However, there are a few drawbacks of the plain |\include| mechanism:
\begin{itemize}
\item
The child files cannot be compiled on their own,
they can only be compiled via the main file.
A naive editing environment
(such as a text editor with an option
to have the current file processed by \LaTeX)
may require one to switch to the main file before compiling;
attempting to compile the child file produces errors.
\item
The main file must be modified (each time)
to adjust the |\includeonly| command
to the present needs. This easily leaves the main file in a messy state.
\item
The generated document will always carry the filename
of the main document. This is inconvenient if
several child files are to be compiled and
to be kept for distribution.
\end{itemize}

The present package provides a simple interface
to make child files individually compilable by \LaTeX{}.
Compiling a child file then has the same effect as compiling
the main file with an |\includeonly| command
to select the appropriate child.
Moreover the generated document will carry the name of the child
rather than the main file.
This resolves all three above issues.

This feature is meant to make the editing of books,
thesis documents and lecture notes somewhat more convenient.
However, the package can also be used efficiently for
composing a series of documents (such as exercise sheets)
which are typically distributed individually.
It then assists the author in generating the individual documents
(potentially in different versions)
as well as a document containing the collected series.
Another application is in developing style files
or other kinds of included material
where compilation of the style file could redirect
to a sample or test file.

%%%%%%%%%%%%%%%%%%%%%%%%%%%%%%%%%%%%%%%%%%%%%%%%%%%%%%%%%%%%%%%%%%%%%%%%%%%%%%%%
%%%%%%%%%%%%%%%%%%%%%%%%%%%%%%%%%%%%%%%%%%%%%%%%%%%%%%%%%%%%%%%%%%%%%%%%%%%%%%%%
\section{Usage}

First of all, the package \textsf{childdoc} is \emph{not} a standard
\LaTeXe{} |.sty| style file! Therefore it needs to be invoked in
a non-standard way.

%%%%%%%%%%%%%%%%%%%%%%%%%%%%%%%%%%%%%%%%%%%%%%%%%%%%%%%%%%%%%%%%%%%%%%%%%%%%%%%%
\subsection{Included Files}
\label{sec:include}

%%%%%%%%%%%%%%%%%%%%%%%%%%%%%%%%%%%%%%%%
\DescribeMacro{\childdocmain}
To use the package, add the commands
\begin{center}
\begin{tabular}{l}
|\input{childdoc.def}|\\
|\childdocmain{}|\\
\end{tabular}
\end{center}
at the very top of the main \LaTeX{} file,
in particular \emph{before} the |\documentclass| statement!
The argument of |\childdocmain| should be left empty
(but it must be present).

%%%%%%%%%%%%%%%%%%%%%%%%%%%%%%%%%%%%%%%%
\DescribeMacro{\childdocof}
Furthermore, add the commands
\begin{center}
\begin{tabular}{l}
|\input{childdoc.def}|\\
|\childdocof{|\textit{main}|}|\\
\end{tabular}
\end{center}
at the top of every child file \textit{child}
which is included by |\include{|\textit{child}|}|
from within the main file
(or at least for those files to be compiled individually).
The argument \textit{main} must be the filename of the main file.

There are a couple of
considerations in setting up the main and child documents:

%%%%%%%%%%%%%%%%%%%%%%%%%%%%%%%%%%%%%%%%
\paragraph{Restrictions.}

Please note the following restrictions:
\begin{itemize}
\item
|\childdocmain| must be called with one argument \textit{main}
to ensure compatibility with earlier version of the package.
It must either be empty (|\childdocmain{}|)
or precisely match the filename of the main file in which it is specified.
See \secref{sec:detection} for further information.
\item
The filename \textit{main} must be specified without the |.tex| extension.
\item
The filename \textit{main} is case sensitive
(even in case-insensitive file systems)
due to internal string comparison.
\item
The argument \textit{main} should be fully expanded, it cannot be a macro.
\item
Subdirectories and special characters should be avoided in filenames.
\item
The command |\childdocmain{|\textit{main}|}| must be followed by a whitespace.
It should not be followed immediately by another command
or by a comment mark `|%|'.
This is because the \TeX{} parser reads the token immediately following
the argument of |\childdocmain| and puts it
at the beginning of every child section;
however, a white\-space is ignored.
\end{itemize}

%%%%%%%%%%%%%%%%%%%%%%%%%%%%%%%%%%%%%%%%
\paragraph{Content of Main File.}

It is advisable to place all content in the child files included by |\include|.
Any output contained in the main file will appear in all child documents
unless suppressed manually;
it cannot be suppressed automatically by the |\includeonly| directive
and thus should normally be avoided.
A method to include some content in the main file
by means of conditional processing is described in \secref{sec:conditional}.

%%%%%%%%%%%%%%%%%%%%%%%%%%%%%%%%%%%%%%%%
\paragraph{Page Numbering.}

When only a part of the document is compiled,
the appropriate numbering of pages
(as well as other status parameters)
is determined from the |.aux| files.
The latter contain information from previous passes.
However this information needs to propagate through
all intermediate child documents.
Therefore the page numbering in child documents may well
be inconsistent until the complete document is compiled at least once.

A useful (if unconventional) way to always ensure a consistent
page numbering is to restart the numbering in each child document
and denote the pages by `\textit{child}|.|\textit{page}'
where \textit{child} represents the chapter/section number of the child file.
This can be achieved by the command
|\numberwithin{page}{|\textit{child}|}|
of the \textsf{amsmath} package
where \textit{child} can be |chapter| or |section|
depending on the chosen structuring.
Alternatively, one can modify the macro |\thepage| appropriately
and reset the counter |page| at the start of each child file.

%%%%%%%%%%%%%%%%%%%%%%%%%%%%%%%%%%%%%%%%%%%%%%%%%%%%%%%%%%%%%%%%%%%%%%%%%%%%%%%%
\subsection{Conditional Processing}
\label{sec:conditional}

The package provides a mechanism to compile different versions
of a document. To customise the versions further some conditional processing
can come in handy to distinguish which version is being compiled.
The package provides two macros to describe the compilation context:

%%%%%%%%%%%%%%%%%%%%%%%%%%%%%%%%%%%%%%%%
\DescribeMacro{\ifchilddoc}
The conditional |\ifchilddoc| distinguishes between the compilation of
child documents and the main document:
%
\begin{center}
|\ifchilddoc |\textit{child-code}| |[|\||else |\textit{main-code}]| \||fi|
\end{center}

%%%%%%%%%%%%%%%%%%%%%%%%%%%%%%%%%%%%%%%%
\DescribeMacro{\childdocname}
\DescribeMacro{\childdocjob}
The macro |\childdocname| contains the filename (without extension)
of the main or child file being processed.
Note that |\childdocjob| will always contain the name of the main file.

%%%%%%%%%%%%%%%%%%%%%%%%%%%%%%%%%%%%%%%%
\paragraph{Title Page.}

Conditional processing can be used to include a title or banner page
in the main document when proper precautions are taken.
Importantly, the code in the main file should ensure that the page counter
(as well as other status parameters which are stored in the |.aux| files)
takes the same value after the conditional processing.
Otherwise the page numbers may take divergent values
depending on which part is compiled.

For example, a title page could be declared by:
%
\begin{center}
\begin{tabular}{l}
|\ifchilddoc\||else|\\
|\addtocounter{page}{-1}|\\
\textit{code for title page}\\
|\newpage|\\
|\||fi|
\end{tabular}
\end{center}
%
A banner page for the child documents can be generated by:
%
\begin{center}
\begin{tabular}{l}
|\ifchilddoc|\\
|\addtocounter{page}{-1}|\\
\textit{code for banner page}\\
|\newpage|\\
|\||fi|
\end{tabular}
\end{center}
%
Here one could write a message such as:
\begin{center}
|This is the part \childdocname{} of \childdocjob{}.|
\end{center}

%%%%%%%%%%%%%%%%%%%%%%%%%%%%%%%%%%%%%%%%%%%%%%%%%%%%%%%%%%%%%%%%%%%%%%%%%%%%%%%%
\subsection{Flags}
\label{sec:flags}

The package makes it easy to generate different versions
of the main or child documents.
To this end compilation flags can be defined
and assigned different default values.
They will be particularly useful in conjunction
with the forwarding mechanism described in \secref{sec:forward}.

For example, it may be useful to have a flag |\version|
which can be set to |draft| or |final|.
The document source will contain some conditional code
depending on the value of |\version|.
Suppose further, the flag should default to |final| for the main file
and to |draft| for child files
which is a natural assignment for editing the document.
This is achieved by placing the following code
in the preamble of the main document
(below the |\childdocmain| directive):
%
\begin{center}
\begin{tabular}{l}
|\ifchilddoc|\\
|\providecommand{\version}{draft}|\\
|\||else|\\
|\providecommand{\version}{final}|\\
|\||fi|
\end{tabular}
\end{center}
%
The definition by |\providecommand| makes sure
that previous definitions are not overwritten.
Further statements |\providecommand{\version}{...}|
can thus be added before the above code to override it.

For the main file, one might add a line
(between |\childdocmain| and the above block)
%
\begin{center}
|%\ifchilddoc\||else\providecommand{\version}{draft}\||fi|
\end{center}
%
which can be uncommented to produce a draft version.
Likewise one can add a line to the very top of a child file
(above the |\childdocof{|\textit{main}|}| directive)
%
\begin{center}
|%\providecommand{\version}{final}|
\end{center}
%
which can be uncommented to produce the final version of this child document.

%%%%%%%%%%%%%%%%%%%%%%%%%%%%%%%%%%%%%%%%%%%%%%%%%%%%%%%%%%%%%%%%%%%%%%%%%%%%%%%%
\subsection{Forwarding}
\label{sec:forward}

Different versions of the main or child documents
using compilation flags as described in \secref{sec:flags}
can be (permanently) stored in different files
for convenient compilation, viewing and distribution.
To this end, the package defines a command
to pass on compilation to a different file:

%%%%%%%%%%%%%%%%%%%%%%%%%%%%%%%%%%%%%%%%
\DescribeMacro{\childdocforward}
The command |\childdocforward| redirects processing to
another source file:
%
\begin{center}
\begin{tabular}{l}
|\input{childdoc.def}|\\
|\childdocforward[|\textit{main}|]{|\textit{dest}|}|\\
\end{tabular}
\end{center}
%
The argument \textit{dest} is the destination file
(without extension).
It should be the main file or one of the child files.
Note that further \textsf{childdoc} directives
such as |\childdocof| and |\childdocforward|
in the indicated file will be processed in this form.
The optional argument \textit{main}
passes on directly to the main file \textit{main}
while pretending to compile the child \textit{dest}.
This form behaves as if \textit{dest}
issues |\childdocof{|\textit{main}|}| right away,
and no further \textsf{childdoc} directives will be processed.

%%%%%%%%%%%%%%%%%%%%%%%%%%%%%%%%%%%%%%%%
\DescribeMacro{\...prefix}
In the alternative form |\childdocforwardprefix|,
%
\begin{center}
\begin{tabular}{l}
|\input{childdoc.def}|\\
|\childdocforwardprefix[|\textit{main}|]{|\textit{prefix}|}{|\textit{dest}|}|
\end{tabular}
\end{center}
%
the destination file is determined by a pattern
depending on the current file:
To make this work, the current file must be called
`{\textit{prefix}\hspace{0.2em}\textit{suffix}}'
with \textit{prefix} matching precisely the argument.
Processing is then passed on to the file
`{\textit{dest}\hspace{0.2em}\textit{suffix}}'.
Surely, the same effect is achieved by
directly specifying the
argument `{\textit{dest}\hspace{0.2em}\textit{suffix}}'
in the first form.
However, that requires to set up a different file
for each child. With the alternative form of the command
all these files can have exactly the same content
which simplifies setting them up and maintaining them.

For example, the following file |draft.tex|
with a compilation flag |\version| as described in \secref{sec:flags}
compiles the main document as a draft:
%
\begin{center}
\begin{tabular}{l}
|\def\version{draft}|\\
|\input{childdoc.def}|\\
|\childdocforward{|\textit{main}|}|
\end{tabular}
\end{center}
%
Likewise, the following files |final|\textit{nn}|.tex|
compile the final version of the child document
|child|\textit{nn}|.tex|:
%
\begin{center}
\begin{tabular}{l}
|\def\version{final}|\\
|\input{childdoc.def}|\\
|\childdocforwardprefix{final}{child}|
\end{tabular}
\end{center}
%

Note that when several versions of a main file and/or of each child file
are to be generated, it may be convenient to set up a |Makefile| or
shell script to automatise the process.

%%%%%%%%%%%%%%%%%%%%%%%%%%%%%%%%%%%%%%%%%%%%%%%%%%%%%%%%%%%%%%%%%%%%%%%%%%%%%%%%
\subsection{Command Line Processing}
\label{sec:commandline}

The effect of redirection files can also be achieved by invoking
the \LaTeX{} compiler with a more elaborate command line.
Most conveniently this should be done as part
of a shell script or a |Makefile|.

When using \textsf{childdoc} in the main file, the following
command lines effectively perform a redirection
(note that depending on the shell being used,
backslashes may have to be doubled: `|\|' $\to$ `|\\|'):
%
\begin{center}
|... -jobname "|\textit{target}|" |\\|"|[\textit{flags}]%
|\input{childdoc.def}\childdocforward[|\textit{main}|]{|\textit{dest}|}"|
\end{center}
%
Here \textit{target} is the name of the output file,
\textit{main} is the name of the main file
and \textit{dest} is the name of the main or child file to be processed
(all filenames without extensions).
The optional argument \textit{main} can be omitted
if \textit{main} matches \textit{dest}.
Optionally, compilation \textit{flags} can be defined via |\def| commands.
This command line makes the \TeX{} engine believe
it is compiling the file \textit{target}
whose content is specified as the latter parameter.
The provided code then forwards the processing to
\textit{main} or \textit{dest} as described in \secref{sec:forward}.

%%%%%%%%%%%%%%%%%%%%%%%%%%%%%%%%%%%%%%%%%%%%%%%%%%%%%%%%%%%%%%%%%%%%%%%%%%%%%%%%
\subsection{Include by Input}
\label{sec:input}

Including child documents by |\include| has some restrictions by design.
Most notably, the content of a child document always occupies
its own set of pages; pages cannot be shared between child documents.
Usually, this behaviour makes perfect sense
because each child document contain an essential part of the document.
However, in some situations it may be desirable to compose
a document from a collection of parts
without having mandatory page breaks between then.
For this case, the package
provides a mechanism to include parts
by |\input| which can also be processed individually.
However, by construction this mechanism
requires manual handling of the content to be output.

%%%%%%%%%%%%%%%%%%%%%%%%%%%%%%%%%%%%%%%%
\DescribeMacro{\ifchilddocmanual}
The main file should be prepared as usual, see \secref{sec:include}.
However, the document body must make a distinction
between processing of an individual part and of the main document, e.g.:
%
\begin{center}
\begin{tabular}{l}
|\ifchilddocmanual|\\
|\input{\childdocname}|\\
|\||else|\\
\textit{document body with }|\input{|\textit{part}|}|\\
|\||fi|
\end{tabular}
\end{center}
%
The conditional |\ifchilddocmanual| is true whenever
a part to be included by |\input| is being compiled,
and the name of the part is stored in |\childdocname|.

%%%%%%%%%%%%%%%%%%%%%%%%%%%%%%%%%%%%%%%%
\DescribeMacro{\childdocby}
Each part to be included by |\input| should start with:
%
\begin{center}
\begin{tabular}{l}
|\input{childdoc.def}|\\
|\childdocby{|\textit{main}|}|\\
\end{tabular}
\end{center}
%
The directive |\childdocby| is similar to |\childdocof|
described in \secref{sec:include},
but the subsequent selection of content must be done manually.
To that end, both |\ifchilddoc| and |\ifchilddocmanual|
will be true upon processing of a part,
and the name of the part is stored in |\childdocname|.
Note that |\jobname| will be set to the filename of the current part
so that each part receives an individual |.aux| file
that does not interfere with the |.aux| file(s) of the main document.
This behaviour can be altered by the alternative form
|\childdocby[*]{|\textit{main}|}| (with a non-empty optional argument)
which uses the |.aux| file of the main document
by setting |\jobname| to \textit{main}.

%%%%%%%%%%%%%%%%%%%%%%%%%%%%%%%%%%%%%%%%%%%%%%%%%%%%%%%%%%%%%%%%%%%%%%%%%%%%%%%%
\subsection{Driver Development}
\label{sec:driver}

The \textsf{childdoc} mechanism can also be use for the development
of definition files such as \LaTeX{} styles or classes.
This case differs from the above setup with multiple parts
included by |\include| in that no |\includeonly| should be invoked.
This can be achieved by starting the include file
(before |\ProvidesPackage|) with:
%
\begin{center}
\begin{tabular}{l}
|\input{childdoc.def}|\\
|\childdocforward{|\textit{main}|}|\\
\end{tabular}
\end{center}
%
or alternatively with:
%
\begin{center}
\begin{tabular}{l}
|\input{childdoc.def}|\\
|\childdocby{|\textit{main}|}|\\
\end{tabular}
\end{center}
%
Both forms have slightly different effects as described above.
The main file is prepared as usual, see \secref{sec:include}.

%%%%%%%%%%%%%%%%%%%%%%%%%%%%%%%%%%%%%%%%%%%%%%%%%%%%%%%%%%%%%%%%%%%%%%%%%%%%%%%%
\subsection{Legacy Detection}
\label{sec:detection}

The directive |\childdocmain| in the main file can detect
whether the complete document or merely a child is to be compiled
even without using the directive |\childdocof|.
This method is deprecated because it is less robust
and there is no compelling reason to use it;
it is merely provided for backward compatibility
and it may be removed in future versions.

If the detection mechanism is to be used,
it is mandatory to correctly specify
the filename of the main file as the argument of |\childdocmain|:
%
\begin{center}
\begin{tabular}{l}
|\input{childdoc.def}|\\
|\childdocmain{|\textit{main}|}|\\
\end{tabular}
\end{center}
%
If |\jobname| does not match the argument \textit{main} of |\childdocmain|,
it is assumed that |\jobname| points to the child file to be compiled.
When using |\childdocmain| with the main file specified as argument,
it suffices to start a child file
with just |\input{|\textit{main}|}|
without loading of the package and using |\childdocof|.
If instead all processing is done
with the appropriate \textsf{childdoc} directives,
the argument of \textit{main} of |\childdocmain| can be empty.

An alternative version of the command line processing described
in \secref{sec:commandline} using the detection mechanism reads:
%
\begin{center}
|... -jobname "|\textit{target}|" "|[\textit{flags}]%
[|\def\jobname{|\textit{dest}|}|]|\input{|\textit{main}|}"|
\end{center}

%%%%%%%%%%%%%%%%%%%%%%%%%%%%%%%%%%%%%%%%%%%%%%%%%%%%%%%%%%%%%%%%%%%%%%%%%%%%%%%%
\subsection{Manual Code}
\label{sec:manual}

In case one cannot be certain whether the definitions file |childdoc.def|
is installed on the target \TeX{} distribution
and one prefers not to ship it,
it is conceivable to paste a few relevant commands into the sources.

To that end, drop all statements |\input{childdoc.def}|
and perform the replacements as outlined below.
Instead of |\childdocmain{|\textit{main}|}| add the following code
to the top of the main file:
%
\begin{center}
\begin{tabular}{l}
|\||ifdefined\childdocname\endinput\||fi\newif\ifchilddoc|\\
|\edef\childdocname{\scantokens\expandafter{\jobname\noexpand}}|\\
|\def\childdocmain{|\textit{main}|}\||ifx\childdocmain\childdocname\||else|\\
|\childdoctrue\includeonly{\childdocname}\let\jobname\childdocmain\||fi|\\
\end{tabular}
\end{center}
%
Instead of |\childdocof{|\textit{main}|}| just include the main file
at the top of each child file:
%
\begin{center}
|\input{|\textit{main}|}|
\end{center}
%
A simple redirection |\childdocforward{|\textit{dest}|}| is achieved by:
%
\begin{center}
|\def\jobname{|\textit{dest}|}\input{\jobname}|
\end{center}
%
The redirection with prefix
|\childdocforwardprefix[|\textit{prefix}|]{|\textit{dest}|}|
is accomplished by:
%
\begin{center}
\begin{tabular}{l}
|{\edef\jobname{\scantokens\expandafter{\jobname\noexpand}}|\\
|\def\redirectjob |\textit{prefix}|#1~~~{\gdef\jobname{|\textit{dest}|#1}}|\\
|\expandafter\redirectjob\jobname~~~}\input{\jobname}|
\end{tabular}
\end{center}

In an alternative approach,
child documents can be compiled by a specific command line
without additional code or specific definitions:
%
\begin{center}
|... -jobname "|\textit{target}|" "|[\textit{flags}]%
|\includeonly{|\textit{dest}|}\input{|\textit{main}|}"|
\end{center}
%

%%%%%%%%%%%%%%%%%%%%%%%%%%%%%%%%%%%%%%%%%%%%%%%%%%%%%%%%%%%%%%%%%%%%%%%%%%%%%%%%
%%%%%%%%%%%%%%%%%%%%%%%%%%%%%%%%%%%%%%%%%%%%%%%%%%%%%%%%%%%%%%%%%%%%%%%%%%%%%%%%
\section{Information}

%%%%%%%%%%%%%%%%%%%%%%%%%%%%%%%%%%%%%%%%%%%%%%%%%%%%%%%%%%%%%%%%%%%%%%%%%%%%%%%%
\subsection{Copyright}

Copyright \copyright{} 2017--2018 Niklas Beisert

This work may be distributed and/or modified under the
conditions of the \LaTeX{} Project Public License, either version 1.3
of this license or (at your option) any later version.
The latest version of this license is in
  \url{http://www.latex-project.org/lppl.txt}
and version 1.3 or later is part of all distributions of \LaTeX{}
version 2005/12/01 or later.

This work has the LPPL maintenance status `maintained'.

The Current Maintainer of this work is Niklas Beisert.

This work consists of the files |README.txt|, |childdoc.ins| and |childdoc.dtx|
as well as the derived files |childdoc.def|, |cdocsamp.tex|
with |cdocsch1.tex|, |cdocsch2.tex|, |cdocspt3.tex|, |cdocspt4.tex|,
|cdocsdrf.tex|, |cdocsfn1.tex|, |cdocsfn2.tex|
as well as |childdoc.pdf|.

%%%%%%%%%%%%%%%%%%%%%%%%%%%%%%%%%%%%%%%%%%%%%%%%%%%%%%%%%%%%%%%%%%%%%%%%%%%%%%%%
\subsection{Files and Installation}

The package consists of the files:
%
\begin{center}
\begin{tabular}{ll}
    |README.txt|   & readme file \\
    |childdoc.ins| & installation file \\
    |childdoc.dtx| & source file \\
    |childdoc.def| & definition file \\
    |cdocsamp.tex| & sample main file \\
    |cdocsch1.tex| & sample include file \\
    |cdocsch2.tex| & sample include file \\
    |cdocspt3.tex| & sample part file \\
    |cdocspt4.tex| & sample part file \\
    |cdocsdrf.tex| & sample redirection file \\
    |cdocsfn1.tex| & sample redirection file \\
    |cdocsfn2.tex| & sample redirection file \\
    |childdoc.pdf| & manual
\end{tabular}
\end{center}
%
The distribution consists of the files
|README.txt|, |childdoc.ins| and |childdoc.dtx|.
%
\begin{itemize}
\item
Run (pdf)\LaTeX{} on |childdoc.dtx|
to compile the manual |childdoc.pdf| (this file).
\item
Run \LaTeX{} on |childdoc.ins| to create the definitions file |childdoc.def|
and the sample |cdocsamp.tex| with include files
|cdocsch1.tex|, |cdocsch2.tex|, |cdocspt3.tex|, |cdocspt4.tex|,
|cdocsdrf.tex|, |cdocsfn1.tex|, |cdocsfn2.tex|.
Then copy the file |childdoc.def| to an appropriate directory of your \LaTeX{}
distribution, e.g.\ \textit{texmf-root}|/tex/latex/childdoc|.
\end{itemize}

%%%%%%%%%%%%%%%%%%%%%%%%%%%%%%%%%%%%%%%%%%%%%%%%%%%%%%%%%%%%%%%%%%%%%%%%%%%%%%%%
\subsection{Related CTAN Packages}

There are several other packages which offer a similar functionality:
%
\begin{itemize}
\item
The packages
\href{http://ctan.org/pkg/docmute}{\textsf{docmute}},
\href{http://ctan.org/pkg/includex}{\textsf{includex}} and
\href{http://ctan.org/pkg/standalone}{\textsf{standalone}}
provide commands to include only the document body of
a child file thus allowing both files to be compiled individually.
\item
The packages \href{http://ctan.org/pkg/subdocs}{\textsf{subdocs}}
and \href{http://ctan.org/pkg/subfiles}{\textsf{subfiles}}
provide structures in which the main and child documents can be
encapsulated and allowing them to be compiled individually.
The inclusion mechanism is different from the conventional |\include|.
\item
The package \href{http://ctan.org/pkg/combine}{\textsf{combine}}
is an elaborate solution to combine several documents into one.
\end{itemize}
%
See also the CTAN topic \href{http://ctan.org/topic/subdocs}{\textsf{subdocs}}
for further related packages.
The present package differs from the above solutions in that
a document structure constructed with the conventional |\include| mechanism
just needs two extra commands at the top of every file
such that all constituent files can be compiled individually.

%%%%%%%%%%%%%%%%%%%%%%%%%%%%%%%%%%%%%%%%%%%%%%%%%%%%%%%%%%%%%%%%%%%%%%%%%%%%%%%%
%\subsection{Feature Suggestions}
%
%The following is a list of features which may be useful for future
%versions of this package:
%%
%\begin{itemize}
%\item
%\ldots
%\end{itemize}

%%%%%%%%%%%%%%%%%%%%%%%%%%%%%%%%%%%%%%%%%%%%%%%%%%%%%%%%%%%%%%%%%%%%%%%%%%%%%%%%
\subsection{Revision History}

%%%%%%%%%%%%%%%%%%%%%%%%%%%%%%%%%%%%%%%%
\paragraph{v2.0:} 2018/12/30

\begin{itemize}
\item
immediate forward processing
\item
added |\childdocby| mechanism
\item
manual restructured
\end{itemize}

%%%%%%%%%%%%%%%%%%%%%%%%%%%%%%%%%%%%%%%%
\paragraph{v1.6:} 2018/01/17

\begin{itemize}
\item
application for development of include files
\item
corrections to manual
\end{itemize}

%%%%%%%%%%%%%%%%%%%%%%%%%%%%%%%%%%%%%%%%
\paragraph{v1.5:} 2017/05/21

\begin{itemize}
\item
more complete structuring introduced
\item
|\childdocof| introduced
\item
|\childdoc| renamed to |\childdocmain|
\item
|\childredirect| renamed to |\childdocforward| and |\childdocforwardprefix|
and functionality expanded
\end{itemize}

%%%%%%%%%%%%%%%%%%%%%%%%%%%%%%%%%%%%%%%%
\paragraph{v1.0:} 2017/04/27

\begin{itemize}
\item
manual and install package
\item
first version published on CTAN
\end{itemize}

%%%%%%%%%%%%%%%%%%%%%%%%%%%%%%%%%%%%%%%%
\paragraph{v0.6:} 2017/04/26

\begin{itemize}
\item
redirection mechanism added
\end{itemize}

%%%%%%%%%%%%%%%%%%%%%%%%%%%%%%%%%%%%%%%%
\paragraph{v0.5:} 2017/04/26

\begin{itemize}
\item
functionality in definition file
\end{itemize}


%%%%%%%%%%%%%%%%%%%%%%%%%%%%%%%%%%%%%%%%%%%%%%%%%%%%%%%%%%%%%%%%%%%%%%%%%%%%%%%%
%%%%%%%%%%%%%%%%%%%%%%%%%%%%%%%%%%%%%%%%%%%%%%%%%%%%%%%%%%%%%%%%%%%%%%%%%%%%%%%%
%%%%%%%%%%%%%%%%%%%%%%%%%%%%%%%%%%%%%%%%%%%%%%%%%%%%%%%%%%%%%%%%%%%%%%%%%%%%%%%%
\appendix

\settowidth\MacroIndent{\rmfamily\scriptsize 000\ }

 \DocInput{childdoc.dtx}

\end{document}
%</driver>
% \fi
%
% %%%%%%%%%%%%%%%%%%%%%%%%%%%%%%%%%%%%%%%%%%%%%%%%%%%%%%%%%%%%%%%%%%%%%%%%%%%%%%
% %%%%%%%%%%%%%%%%%%%%%%%%%%%%%%%%%%%%%%%%%%%%%%%%%%%%%%%%%%%%%%%%%%%%%%%%%%%%%%
% \section{Sample}
%\iffalse
%<*samplemain>
%\fi
%
% The following presents a sample document
% with two chapters, two parts, a title page,
% a compile flag as well as three forwarding files to set the flag.
% It consists of eight |.tex| files:
% \begin{center}
% \begin{tabular}{ll}
% |cdocsamp.tex|&main file\\
% |cdocsch1.tex|&include file for chapter 1\\
% |cdocsch2.tex|&include file for chapter 2\\
% |cdocspt3.tex|&include file for part 3\\
% |cdocspt4.tex|&include file for part 4\\
% |cdocsdrf.tex|&forwarding file for main file in draft mode\\
% |cdocsfi1.tex|&forwarding file for final version of chapter 1\\
% |cdocsfi2.tex|&forwarding file for final version of chapter 2\\
% \end{tabular}
% \end{center}
% Each of the eight files can be compiled directly by the \LaTeX{} compiler.
%
% %%%%%%%%%%%%%%%%%%%%%%%%%%%%%%%%%%%%%%
% \paragraph{Main File.}
%
% The main file is called |cdocsamp.tex|.
%
% Load the \textsf{childdoc} definitions and
% declare the filename for the main document:
%    \begin{macrocode}
\input{childdoc.def}
\childdocmain{}
%    \end{macrocode}

% Optional override for |\version| flag:
%    \begin{macrocode}
%%\ifchilddoc\else\providecommand{\version}{draft}\fi
%    \end{macrocode}

% Define the default values for the |\version| flag
% (|final| for the main file and |draft| for childs):
%    \begin{macrocode}
\ifchilddoc
\providecommand{\version}{draft}
\else
\providecommand{\version}{final}
\fi
%    \end{macrocode}

% Load the standard document class:
%    \begin{macrocode}
\documentclass[12pt]{article}
%    \end{macrocode}

% Start the document body:
%    \begin{macrocode}
\begin{document}
%    \end{macrocode}

% Declare a title page.
% Print title, part of document being processed and version flag:
%    \begin{macrocode}
\addtocounter{page}{-1}
\begin{center}
{\LARGE\bfseries{}childdoc example\par}
\vspace{1cm}
\ifchilddoc
\ifchilddocmanual part\else chapter\fi:
`\childdocname' of `\childdocjob'\par
\else
main document: `\childdocjob'\par
\fi
version: \version\par
\end{center}
\newpage
%    \end{macrocode}

% Manually include selected file,
% otherwise process as usual:
%    \begin{macrocode}
\ifchilddocmanual
\section*{part `\childdocname'}
\input{\childdocname}
\else
%    \end{macrocode}

% Include the two chapters:
%    \begin{macrocode}
\include{cdocsch1}
\include{cdocsch2}
%    \end{macrocode}

% Include the two parts unless only chapters should be displayed:
%    \begin{macrocode}
\ifchilddoc\else
\section{part three}
\input{cdocspt3}
\section{part four}
\input{cdocspt4}
\fi
%    \end{macrocode}

% Process as usual until here:
%    \begin{macrocode}
\fi
%    \end{macrocode}

% End of document body:
%    \begin{macrocode}
\end{document}
%    \end{macrocode}
%\iffalse
%</samplemain>
%\fi
%
% %%%%%%%%%%%%%%%%%%%%%%%%%%%%%%%%%%%%%%
% \paragraph{Chapter Include Files.}
%
% The include files are called |cdocsch1.tex| and |cdocsch2.tex|.
%
%\iffalse
%<*samplechap1|samplechap2>
%\fi

% Optional override for |\version| flag:
%    \begin{macrocode}
%%\providecommand{\version}{final}
%    \end{macrocode}

% Include the main document:
%    \begin{macrocode}
\input{childdoc.def}
\childdocof{cdocsamp}
%    \end{macrocode}

%\iffalse
%</samplechap1|samplechap2>
%\fi
%
%\iffalse
%<*samplechap1>
%\fi
% Some text for chapter 1:
%    \begin{macrocode}
\section{one}
some text in chapter one
%    \end{macrocode}

%\iffalse
%</samplechap1>
%\fi
% Some text for chapter 2:
%\iffalse
%<*samplechap2>
%\fi
%    \begin{macrocode}
\section{two}
more text in chapter two
%    \end{macrocode}

%\iffalse
%</samplechap2>
%\fi
%
% %%%%%%%%%%%%%%%%%%%%%%%%%%%%%%%%%%%%%%
% \paragraph{Part Include Files.}
%
% The include files are called |cdocspt3.tex| and |cdocspt4.tex|.
%
%\iffalse
%<*samplepart3|samplepart4>
%\fi

% Optional override for |\version| flag:
%    \begin{macrocode}
%%\providecommand{\version}{final}
%    \end{macrocode}

% Include the main document:
%    \begin{macrocode}
\input{childdoc.def}
\childdocby{cdocsamp}
%    \end{macrocode}

%\iffalse
%</samplepart3|samplepart4>
%\fi
%
%\iffalse
%<*samplepart3>
%\fi
% Some text for part 3:
%    \begin{macrocode}
some text in part three
%    \end{macrocode}

%\iffalse
%</samplepart3>
%\fi
% Some text for part 4:
%\iffalse
%<*samplepart4>
%\fi
%    \begin{macrocode}
more text in part four
%    \end{macrocode}

%\iffalse
%</samplepart4>
%\fi
%
% %%%%%%%%%%%%%%%%%%%%%%%%%%%%%%%%%%%%%%
% \paragraph{Forwarding for a Complete Draft.}
%
% The following forwarding file |cdocsdrf.tex|
% compiles the main document in draft mode:
%\iffalse
%<*sampledraft>
%\fi
%    \begin{macrocode}
\def\version{draft}
\input{childdoc.def}
\childdocforward{cdocsamp}
%    \end{macrocode}

%\iffalse
%</sampledraft>
%\fi
%
% %%%%%%%%%%%%%%%%%%%%%%%%%%%%%%%%%%%%%%
% \paragraph{Forwarding for Final Version of the Chapters.}
%
% The following forwarding files |cdocsfn1.tex| and |cdocsfn2.tex|
% (with identical content)
% compile the final versions of the child documents
% |cdocsch1.tex| and |cdocsch2.tex|, respectively:
%\iffalse
%<*samplefinal>
%\fi
%    \begin{macrocode}
\def\version{final}
\input{childdoc.def}
\childdocforwardprefix[cdocsamp]{cdocsfn}{cdocsch}
%    \end{macrocode}

%\iffalse
%</samplefinal>
%\fi
%
% %%%%%%%%%%%%%%%%%%%%%%%%%%%%%%%%%%%%%%
% \paragraph{Command Line Processing.}
%
% The following three command lines generate the output files
% |cdocscld|, |cdocscl1| and |cdocscl2|
% which should be identical to
% |cdocsdrf|, |cdocsch1| and |cdocsfn2|, respectively:
% \begin{center}
% \begin{tabular}{l}
% |latex -jobname cdocscld \|\\
% |  "\def\version{draft}\input{childdoc.def}\childdocforward{cdocsamp}"|\\
% |latex -jobname cdocscl1 \|\\
% |  "\input{childdoc.def}\childdocforward[cdocsamp]{cdocsch1}"|\\
% |latex -jobname cdocscl2 \|\\
% |  "\def\version{final}\input{childdoc.def}\childdocforward{cdocsch2}"|
% \end{tabular}
% \end{center}
% Note that the trailing backslash on each first line
% merely continues the input to the second line
% (for convenient cut ant paste).
% Furthermore, the command |latex| can be replaced by any
% of its alternative versions such as |pdflatex|.
%
% %%%%%%%%%%%%%%%%%%%%%%%%%%%%%%%%%%%%%%%%%%%%%%%%%%%%%%%%%%%%%%%%%%%%%%%%%%%%%%
% %%%%%%%%%%%%%%%%%%%%%%%%%%%%%%%%%%%%%%%%%%%%%%%%%%%%%%%%%%%%%%%%%%%%%%%%%%%%%%
% \section{Implementation}
%\iffalse
%<*package>
%\fi
%
% This section describes the definitions file |childdoc.def|.

% The definitions cannot be loaded using |\usepackage| or |\RequirePackage|
% which has a mechanism to prevent loading a style file more than once.
% When loading the definitions by means of |\input|
% multiple instances have to be prevented manually:
%\iffalse
%This code needs to be before the `\ProvidesFile' directive
%which is defined at the beginning of this file.
%Therefore it is also placed there and commented out here.
%</package>
%<*discard>
%\fi
%    \begin{macrocode}
\ifdefined\childdocmain\endinput\fi
%    \end{macrocode}
%\iffalse
%</discard>
%<*package>
%\fi
%
% \macro{\ifchilddoc}
% \macro{\ifchilddocmanual}
% The conditional |\ifchilddoc| tells whether a
% child (true) or main (false) document is being compiled.
% The conditional |\ifchilddocmanual| tells whether
% the |\includeonly| mechanism is used (false) or
% the selection of child files must be performed manually (true).
% The definitions initialise to false:
%    \begin{macrocode}
\newif\ifchilddoc
\newif\ifchilddocmanual
%    \end{macrocode}

% \macro{\childdocname}
% \macro{\childdocjob}
% The macro |\childdocname| stores the name of the main document
% to be compiled. The macro |\childdocjob| stores the name of
% the document on which the \LaTeX{} compiler was originally invoked.
% The content of |\jobname| cannot be compared
% to filenames specified in the source due to different catcodes.
% The following code rescans |\jobname|, stores the result
% in |\childdocname| and saves a copy in |\childdocjob|:
%    \begin{macrocode}
\edef\childdocname{\scantokens\expandafter{\jobname\noexpand}}
\let\childdocjob\childdocname
%    \end{macrocode}

% \macro{\childdocdisable}
% The macro |\childdocdisable| prevents the main file
% from being processed more than once.
% At this stage, the main document command |\childdocmain|
% is assumed to be called once again where it should do nothing.
% Any subsequent call to it should prevent
% a secondary processing of the main document
% It overwrites the forwarding commands
% |\childdocof| and |\childdocforward|
% with empty macros to prevent further inclusions of the main document:
%    \begin{macrocode}
\newcommand{\childdocdisable}
{
  \renewcommand{\childdocmain}[1]{\renewcommand{\childdocmain}[1]{\endinput}}
  \renewcommand{\childdocof}[1]{}
  \renewcommand{\childdocby}[2][]{}
  \renewcommand{\childdocforward}[2][]{}
  \renewcommand{\childdocdisable}{}
}
%    \end{macrocode}

% \macro{\childdocmain}
% The macro |\childdocmain| is to be called at the top of the main file
% with nothing or the main filename (without extension) as argument.
% First, it breaks loops.
% If the argument is not empty and does not match |\childdocname|
% (which is set by the first inclusion of |childdoc.def|),
% |\ifchilddoc| is set to true, |\includeonly| is applied to the child file
% and |\jobname| is set to the main file
% (for proper handling of |.aux| files):
%    \begin{macrocode}
\newcommand{\childdocmain}[1]
{
  \childdocdisable\childdocmain{}
  \if?#1?\else
    \begingroup
      \def\childdoctmp{#1}
      \ifx\childdoctmp\childdocname
        \def\childdoctmp{}
      \else
        \def\childdoctmp
        {
          \childdoctrue
          \includeonly{\childdocname}
          \def\childdocjob{#1}
          \def\jobname{#1}
        }
      \fi
      \expandafter
    \endgroup
    \childdoctmp
  \fi
}
%    \end{macrocode}

% \macro{\childdocof}
% The command |\childdocof| redirects
% compilation to the main file |#1|.
%    \begin{macrocode}
\newcommand{\childdocof}[1]
{
  \childdocdisable
  \childdoctrue
  \includeonly{\childdocname}
  \def\jobname{#1}
  \def\childdocjob{#1}
  \input{#1}
}
%    \end{macrocode}

% \macro{\childdocby}
% The command |\childdocby| ....
%    \begin{macrocode}
\newcommand{\childdocby}[2][]
{
  \childdocdisable
  \childdoctrue
  \childdocmanualtrue
  \if?#1?\else
    \def\jobname{#2}
  \fi
  \def\childdocjob{#2}
  \input{#2}
  \endinput
}
%    \end{macrocode}

% \macro{\childdocforward}
% The command |\childdocforward| redirects
% compilation to the main file or
% (if the optional argument is given) a child file.
% Parameters are set as if the main file
% or a child file starting with |\childdocof| was compiled.
% Then compilation is handed over to the main file:
%    \begin{macrocode}
\newcommand{\childdocforward}[2][]
{
  \begingroup
    \if?#1?
      \def\childdoctmp
      {
        \def\childdocname{#2}
        \def\childdocjob{#2}
        \def\jobname{#2}
        \input{#2}
        \endinput
      }
    \else
      \def\childdoctmp
      {
        \childdocdisable
        \def\childdocname{#2}
        \childdoctrue
        \includeonly{#2}
        \def\childdocjob{#1}
        \def\jobname{#1}
        \input{#1}
        \endinput
      }
    \fi
    \expandafter
  \endgroup
  \childdoctmp
}
%    \end{macrocode}

% \macro{\childdocforwardprefix}
% The command |\childdocforwardprefix| redirects
% compilation to the main or a child file by means of a pattern.
% The prefix |#1| in the current filename is replaced by |#2|
% and the suffix of the current filename is kept
% (it is assumed that the filename does not contain the substring `|~~~|'
% which is used as a delimiter).
% Compilation is handed over to the new file by |\childdocforward|:
%    \begin{macrocode}
\newcommand{\childdocforwardprefix}[3][]
{
  \begingroup
    \def\childdocextract #2##1~~~{\def\childdoctmp{\childdocforward[#1]{#3##1}}}
    \expandafter\childdocextract\childdocname~~~
    \expandafter
  \endgroup
  \childdoctmp
}
%    \end{macrocode}

% \macro{\childdoc}
% The deprecated macro |\childdoc| is a legacy version of |\childdocmain|:
%    \begin{macrocode}
\newcommand{\childdoc}{\childdocmain}
%    \end{macrocode}

% \macro{\childdocredirect}
% The deprecated macro |\childdocredirect| is a legacy version
% of |\childdocforward| and |\childdocforwardprefix|:
%    \begin{macrocode}
\newcommand{\childdocredirect}[2][]
{
  \begingroup
    \if?#1?
      \def\childdoctmp{\childdocforward{#2}}
    \else
      \def\childdoctmp{\childdocforwardprefix{#1}{#2}}
    \fi
    \expandafter
  \endgroup
  \childdoctmp
}
%    \end{macrocode}

%\iffalse
%</package>
%\fi
%
\endinput
\childdocforward[|\textit{main}|]{|\textit{dest}|}"|
\end{center}
%
Here \textit{target} is the name of the output file,
\textit{main} is the name of the main file
and \textit{dest} is the name of the main or child file to be processed
(all filenames without extensions).
The optional argument \textit{main} can be omitted
if \textit{main} matches \textit{dest}.
Optionally, compilation \textit{flags} can be defined via |\def| commands.
This command line makes the \TeX{} engine believe
it is compiling the file \textit{target}
whose content is specified as the latter parameter.
The provided code then forwards the processing to
\textit{main} or \textit{dest} as described in \secref{sec:forward}.

%%%%%%%%%%%%%%%%%%%%%%%%%%%%%%%%%%%%%%%%%%%%%%%%%%%%%%%%%%%%%%%%%%%%%%%%%%%%%%%%
\subsection{Include by Input}
\label{sec:input}

Including child documents by |\include| has some restrictions by design.
Most notably, the content of a child document always occupies
its own set of pages; pages cannot be shared between child documents.
Usually, this behaviour makes perfect sense
because each child document contain an essential part of the document.
However, in some situations it may be desirable to compose
a document from a collection of parts
without having mandatory page breaks between then.
For this case, the package
provides a mechanism to include parts
by |\input| which can also be processed individually.
However, by construction this mechanism
requires manual handling of the content to be output.

%%%%%%%%%%%%%%%%%%%%%%%%%%%%%%%%%%%%%%%%
\DescribeMacro{\ifchilddocmanual}
The main file should be prepared as usual, see \secref{sec:include}.
However, the document body must make a distinction
between processing of an individual part and of the main document, e.g.:
%
\begin{center}
\begin{tabular}{l}
|\ifchilddocmanual|\\
|\input{\childdocname}|\\
|\||else|\\
\textit{document body with }|\input{|\textit{part}|}|\\
|\||fi|
\end{tabular}
\end{center}
%
The conditional |\ifchilddocmanual| is true whenever
a part to be included by |\input| is being compiled,
and the name of the part is stored in |\childdocname|.

%%%%%%%%%%%%%%%%%%%%%%%%%%%%%%%%%%%%%%%%
\DescribeMacro{\childdocby}
Each part to be included by |\input| should start with:
%
\begin{center}
\begin{tabular}{l}
|% \iffalse
%
% childdoc.dtx Copyright (C) 2017-2018 Niklas Beisert
%
% This work may be distributed and/or modified under the
% conditions of the LaTeX Project Public License, either version 1.3
% of this license or (at your option) any later version.
% The latest version of this license is in
%   http://www.latex-project.org/lppl.txt
% and version 1.3 or later is part of all distributions of LaTeX
% version 2005/12/01 or later.
%
% This work has the LPPL maintenance status `maintained'.
%
% The Current Maintainer of this work is Niklas Beisert.
%
% This work consists of the files childdoc.dtx and childdoc.ins
% and the derived files childdoc.def and cdocsamp.tex with
% cdocsch1.tex, cdocsch2.tex, cdocsdrf.tex, cdocsfn1.tex, cdocsfn2.tex.
%
%<package>\ifdefined\childdocmain\endinput\fi
%<package>\ProvidesFile{childdoc.def}[2018/12/30 v2.0 child document driver]
%<samplemain>\ProvidesFile{cdocsamp.tex}[2018/12/30 v2.0 sample for childdoc]
%<*driver>
%\ProvidesFile{childdoc.drv}[2018/12/30 v2.0 childdoc reference manual file]
\PassOptionsToClass{10pt,a4paper}{article}
\documentclass{ltxdoc}

\usepackage[margin=35mm]{geometry}
\usepackage{hyperref}
\usepackage{hyperxmp}
\usepackage[usenames]{color}

\hypersetup{colorlinks=true}
\hypersetup{pdfstartview=FitH}
\hypersetup{pdfpagemode=UseNone}
\hypersetup{pdfsource={}}
\hypersetup{pdflang={en-UK}}
\hypersetup{pdfcopyright={Copyright 2017-2018 Niklas Beisert.
  This work may be distributed and/or modified under the
  conditions of the LaTeX Project Public License, either version 1.3
  of this license or (at your option) any later version.}}
\hypersetup{pdflicenseurl={http://www.latex-project.org/lppl.txt}}
\hypersetup{pdfcontactaddress={ETH Zurich, ITP, HIT K,
  Wolfgang-Pauli-Strasse 27}}
\hypersetup{pdfcontactpostcode={8093}}
\hypersetup{pdfcontactcity={Zurich}}
\hypersetup{pdfcontactcountry={Switzerland}}
\hypersetup{pdfcontactemail={nbeisert@itp.phys.ethz.ch}}
\hypersetup{pdfcontacturl={http://people.phys.ethz.ch/\xmptilde nbeisert/}}

\newcommand{\secref}[1]{\hyperref[#1]{section \ref*{#1}}}

\parskip1ex
\parindent0pt
\let\olditemize\itemize
\def\itemize{\olditemize\parskip0pt}

\begin{document}

\title{The \textsf{childdoc} Package}
\hypersetup{pdftitle={The childdoc Package}}
\author{Niklas Beisert\\[2ex]
  Institut f\"ur Theoretische Physik\\
  Eidgen\"ossische Technische Hochschule Z\"urich\\
  Wolfgang-Pauli-Strasse 27, 8093 Z\"urich, Switzerland\\[1ex]
  \href{mailto:nbeisert@itp.phys.ethz.ch}
  {\texttt{nbeisert@itp.phys.ethz.ch}}}
\hypersetup{pdfauthor={Niklas Beisert}}
\hypersetup{pdfsubject={Manual for the LaTeX2e Package childdoc}}
\date{30 December 2018, \textsf{v2.0}}
\maketitle

\begin{abstract}\noindent
\textsf{childdoc} is a \LaTeXe{} package
that enables the direct compilation
of document sections included by |\include|
to individual files.
\end{abstract}

\begingroup
\parskip0ex
\tableofcontents
\endgroup

%%%%%%%%%%%%%%%%%%%%%%%%%%%%%%%%%%%%%%%%%%%%%%%%%%%%%%%%%%%%%%%%%%%%%%%%%%%%%%%%
%%%%%%%%%%%%%%%%%%%%%%%%%%%%%%%%%%%%%%%%%%%%%%%%%%%%%%%%%%%%%%%%%%%%%%%%%%%%%%%%
\section{Introduction}

\LaTeX{} provides a mechanism to structure a large document (such as a book)
into a main file and several child files (containing the chapters)
using the |\include| command.
This mechanism is beneficial for documents
which span hundreds of pages in order to
make the source file(s) more manageable.
Moreover, compilation can be restricted to
selected child files by means of the |\includeonly| command.
The latter feature can be used to reduce the compilation time while editing
(this was significantly more useful in the earlier days of \LaTeX{})
or to generate a smaller document which is easier to navigate.
Another application of |\includeonly| is to generate
documents consisting of selected parts of the complete document.

However, there are a few drawbacks of the plain |\include| mechanism:
\begin{itemize}
\item
The child files cannot be compiled on their own,
they can only be compiled via the main file.
A naive editing environment
(such as a text editor with an option
to have the current file processed by \LaTeX)
may require one to switch to the main file before compiling;
attempting to compile the child file produces errors.
\item
The main file must be modified (each time)
to adjust the |\includeonly| command
to the present needs. This easily leaves the main file in a messy state.
\item
The generated document will always carry the filename
of the main document. This is inconvenient if
several child files are to be compiled and
to be kept for distribution.
\end{itemize}

The present package provides a simple interface
to make child files individually compilable by \LaTeX{}.
Compiling a child file then has the same effect as compiling
the main file with an |\includeonly| command
to select the appropriate child.
Moreover the generated document will carry the name of the child
rather than the main file.
This resolves all three above issues.

This feature is meant to make the editing of books,
thesis documents and lecture notes somewhat more convenient.
However, the package can also be used efficiently for
composing a series of documents (such as exercise sheets)
which are typically distributed individually.
It then assists the author in generating the individual documents
(potentially in different versions)
as well as a document containing the collected series.
Another application is in developing style files
or other kinds of included material
where compilation of the style file could redirect
to a sample or test file.

%%%%%%%%%%%%%%%%%%%%%%%%%%%%%%%%%%%%%%%%%%%%%%%%%%%%%%%%%%%%%%%%%%%%%%%%%%%%%%%%
%%%%%%%%%%%%%%%%%%%%%%%%%%%%%%%%%%%%%%%%%%%%%%%%%%%%%%%%%%%%%%%%%%%%%%%%%%%%%%%%
\section{Usage}

First of all, the package \textsf{childdoc} is \emph{not} a standard
\LaTeXe{} |.sty| style file! Therefore it needs to be invoked in
a non-standard way.

%%%%%%%%%%%%%%%%%%%%%%%%%%%%%%%%%%%%%%%%%%%%%%%%%%%%%%%%%%%%%%%%%%%%%%%%%%%%%%%%
\subsection{Included Files}
\label{sec:include}

%%%%%%%%%%%%%%%%%%%%%%%%%%%%%%%%%%%%%%%%
\DescribeMacro{\childdocmain}
To use the package, add the commands
\begin{center}
\begin{tabular}{l}
|\input{childdoc.def}|\\
|\childdocmain{}|\\
\end{tabular}
\end{center}
at the very top of the main \LaTeX{} file,
in particular \emph{before} the |\documentclass| statement!
The argument of |\childdocmain| should be left empty
(but it must be present).

%%%%%%%%%%%%%%%%%%%%%%%%%%%%%%%%%%%%%%%%
\DescribeMacro{\childdocof}
Furthermore, add the commands
\begin{center}
\begin{tabular}{l}
|\input{childdoc.def}|\\
|\childdocof{|\textit{main}|}|\\
\end{tabular}
\end{center}
at the top of every child file \textit{child}
which is included by |\include{|\textit{child}|}|
from within the main file
(or at least for those files to be compiled individually).
The argument \textit{main} must be the filename of the main file.

There are a couple of
considerations in setting up the main and child documents:

%%%%%%%%%%%%%%%%%%%%%%%%%%%%%%%%%%%%%%%%
\paragraph{Restrictions.}

Please note the following restrictions:
\begin{itemize}
\item
|\childdocmain| must be called with one argument \textit{main}
to ensure compatibility with earlier version of the package.
It must either be empty (|\childdocmain{}|)
or precisely match the filename of the main file in which it is specified.
See \secref{sec:detection} for further information.
\item
The filename \textit{main} must be specified without the |.tex| extension.
\item
The filename \textit{main} is case sensitive
(even in case-insensitive file systems)
due to internal string comparison.
\item
The argument \textit{main} should be fully expanded, it cannot be a macro.
\item
Subdirectories and special characters should be avoided in filenames.
\item
The command |\childdocmain{|\textit{main}|}| must be followed by a whitespace.
It should not be followed immediately by another command
or by a comment mark `|%|'.
This is because the \TeX{} parser reads the token immediately following
the argument of |\childdocmain| and puts it
at the beginning of every child section;
however, a white\-space is ignored.
\end{itemize}

%%%%%%%%%%%%%%%%%%%%%%%%%%%%%%%%%%%%%%%%
\paragraph{Content of Main File.}

It is advisable to place all content in the child files included by |\include|.
Any output contained in the main file will appear in all child documents
unless suppressed manually;
it cannot be suppressed automatically by the |\includeonly| directive
and thus should normally be avoided.
A method to include some content in the main file
by means of conditional processing is described in \secref{sec:conditional}.

%%%%%%%%%%%%%%%%%%%%%%%%%%%%%%%%%%%%%%%%
\paragraph{Page Numbering.}

When only a part of the document is compiled,
the appropriate numbering of pages
(as well as other status parameters)
is determined from the |.aux| files.
The latter contain information from previous passes.
However this information needs to propagate through
all intermediate child documents.
Therefore the page numbering in child documents may well
be inconsistent until the complete document is compiled at least once.

A useful (if unconventional) way to always ensure a consistent
page numbering is to restart the numbering in each child document
and denote the pages by `\textit{child}|.|\textit{page}'
where \textit{child} represents the chapter/section number of the child file.
This can be achieved by the command
|\numberwithin{page}{|\textit{child}|}|
of the \textsf{amsmath} package
where \textit{child} can be |chapter| or |section|
depending on the chosen structuring.
Alternatively, one can modify the macro |\thepage| appropriately
and reset the counter |page| at the start of each child file.

%%%%%%%%%%%%%%%%%%%%%%%%%%%%%%%%%%%%%%%%%%%%%%%%%%%%%%%%%%%%%%%%%%%%%%%%%%%%%%%%
\subsection{Conditional Processing}
\label{sec:conditional}

The package provides a mechanism to compile different versions
of a document. To customise the versions further some conditional processing
can come in handy to distinguish which version is being compiled.
The package provides two macros to describe the compilation context:

%%%%%%%%%%%%%%%%%%%%%%%%%%%%%%%%%%%%%%%%
\DescribeMacro{\ifchilddoc}
The conditional |\ifchilddoc| distinguishes between the compilation of
child documents and the main document:
%
\begin{center}
|\ifchilddoc |\textit{child-code}| |[|\||else |\textit{main-code}]| \||fi|
\end{center}

%%%%%%%%%%%%%%%%%%%%%%%%%%%%%%%%%%%%%%%%
\DescribeMacro{\childdocname}
\DescribeMacro{\childdocjob}
The macro |\childdocname| contains the filename (without extension)
of the main or child file being processed.
Note that |\childdocjob| will always contain the name of the main file.

%%%%%%%%%%%%%%%%%%%%%%%%%%%%%%%%%%%%%%%%
\paragraph{Title Page.}

Conditional processing can be used to include a title or banner page
in the main document when proper precautions are taken.
Importantly, the code in the main file should ensure that the page counter
(as well as other status parameters which are stored in the |.aux| files)
takes the same value after the conditional processing.
Otherwise the page numbers may take divergent values
depending on which part is compiled.

For example, a title page could be declared by:
%
\begin{center}
\begin{tabular}{l}
|\ifchilddoc\||else|\\
|\addtocounter{page}{-1}|\\
\textit{code for title page}\\
|\newpage|\\
|\||fi|
\end{tabular}
\end{center}
%
A banner page for the child documents can be generated by:
%
\begin{center}
\begin{tabular}{l}
|\ifchilddoc|\\
|\addtocounter{page}{-1}|\\
\textit{code for banner page}\\
|\newpage|\\
|\||fi|
\end{tabular}
\end{center}
%
Here one could write a message such as:
\begin{center}
|This is the part \childdocname{} of \childdocjob{}.|
\end{center}

%%%%%%%%%%%%%%%%%%%%%%%%%%%%%%%%%%%%%%%%%%%%%%%%%%%%%%%%%%%%%%%%%%%%%%%%%%%%%%%%
\subsection{Flags}
\label{sec:flags}

The package makes it easy to generate different versions
of the main or child documents.
To this end compilation flags can be defined
and assigned different default values.
They will be particularly useful in conjunction
with the forwarding mechanism described in \secref{sec:forward}.

For example, it may be useful to have a flag |\version|
which can be set to |draft| or |final|.
The document source will contain some conditional code
depending on the value of |\version|.
Suppose further, the flag should default to |final| for the main file
and to |draft| for child files
which is a natural assignment for editing the document.
This is achieved by placing the following code
in the preamble of the main document
(below the |\childdocmain| directive):
%
\begin{center}
\begin{tabular}{l}
|\ifchilddoc|\\
|\providecommand{\version}{draft}|\\
|\||else|\\
|\providecommand{\version}{final}|\\
|\||fi|
\end{tabular}
\end{center}
%
The definition by |\providecommand| makes sure
that previous definitions are not overwritten.
Further statements |\providecommand{\version}{...}|
can thus be added before the above code to override it.

For the main file, one might add a line
(between |\childdocmain| and the above block)
%
\begin{center}
|%\ifchilddoc\||else\providecommand{\version}{draft}\||fi|
\end{center}
%
which can be uncommented to produce a draft version.
Likewise one can add a line to the very top of a child file
(above the |\childdocof{|\textit{main}|}| directive)
%
\begin{center}
|%\providecommand{\version}{final}|
\end{center}
%
which can be uncommented to produce the final version of this child document.

%%%%%%%%%%%%%%%%%%%%%%%%%%%%%%%%%%%%%%%%%%%%%%%%%%%%%%%%%%%%%%%%%%%%%%%%%%%%%%%%
\subsection{Forwarding}
\label{sec:forward}

Different versions of the main or child documents
using compilation flags as described in \secref{sec:flags}
can be (permanently) stored in different files
for convenient compilation, viewing and distribution.
To this end, the package defines a command
to pass on compilation to a different file:

%%%%%%%%%%%%%%%%%%%%%%%%%%%%%%%%%%%%%%%%
\DescribeMacro{\childdocforward}
The command |\childdocforward| redirects processing to
another source file:
%
\begin{center}
\begin{tabular}{l}
|\input{childdoc.def}|\\
|\childdocforward[|\textit{main}|]{|\textit{dest}|}|\\
\end{tabular}
\end{center}
%
The argument \textit{dest} is the destination file
(without extension).
It should be the main file or one of the child files.
Note that further \textsf{childdoc} directives
such as |\childdocof| and |\childdocforward|
in the indicated file will be processed in this form.
The optional argument \textit{main}
passes on directly to the main file \textit{main}
while pretending to compile the child \textit{dest}.
This form behaves as if \textit{dest}
issues |\childdocof{|\textit{main}|}| right away,
and no further \textsf{childdoc} directives will be processed.

%%%%%%%%%%%%%%%%%%%%%%%%%%%%%%%%%%%%%%%%
\DescribeMacro{\...prefix}
In the alternative form |\childdocforwardprefix|,
%
\begin{center}
\begin{tabular}{l}
|\input{childdoc.def}|\\
|\childdocforwardprefix[|\textit{main}|]{|\textit{prefix}|}{|\textit{dest}|}|
\end{tabular}
\end{center}
%
the destination file is determined by a pattern
depending on the current file:
To make this work, the current file must be called
`{\textit{prefix}\hspace{0.2em}\textit{suffix}}'
with \textit{prefix} matching precisely the argument.
Processing is then passed on to the file
`{\textit{dest}\hspace{0.2em}\textit{suffix}}'.
Surely, the same effect is achieved by
directly specifying the
argument `{\textit{dest}\hspace{0.2em}\textit{suffix}}'
in the first form.
However, that requires to set up a different file
for each child. With the alternative form of the command
all these files can have exactly the same content
which simplifies setting them up and maintaining them.

For example, the following file |draft.tex|
with a compilation flag |\version| as described in \secref{sec:flags}
compiles the main document as a draft:
%
\begin{center}
\begin{tabular}{l}
|\def\version{draft}|\\
|\input{childdoc.def}|\\
|\childdocforward{|\textit{main}|}|
\end{tabular}
\end{center}
%
Likewise, the following files |final|\textit{nn}|.tex|
compile the final version of the child document
|child|\textit{nn}|.tex|:
%
\begin{center}
\begin{tabular}{l}
|\def\version{final}|\\
|\input{childdoc.def}|\\
|\childdocforwardprefix{final}{child}|
\end{tabular}
\end{center}
%

Note that when several versions of a main file and/or of each child file
are to be generated, it may be convenient to set up a |Makefile| or
shell script to automatise the process.

%%%%%%%%%%%%%%%%%%%%%%%%%%%%%%%%%%%%%%%%%%%%%%%%%%%%%%%%%%%%%%%%%%%%%%%%%%%%%%%%
\subsection{Command Line Processing}
\label{sec:commandline}

The effect of redirection files can also be achieved by invoking
the \LaTeX{} compiler with a more elaborate command line.
Most conveniently this should be done as part
of a shell script or a |Makefile|.

When using \textsf{childdoc} in the main file, the following
command lines effectively perform a redirection
(note that depending on the shell being used,
backslashes may have to be doubled: `|\|' $\to$ `|\\|'):
%
\begin{center}
|... -jobname "|\textit{target}|" |\\|"|[\textit{flags}]%
|\input{childdoc.def}\childdocforward[|\textit{main}|]{|\textit{dest}|}"|
\end{center}
%
Here \textit{target} is the name of the output file,
\textit{main} is the name of the main file
and \textit{dest} is the name of the main or child file to be processed
(all filenames without extensions).
The optional argument \textit{main} can be omitted
if \textit{main} matches \textit{dest}.
Optionally, compilation \textit{flags} can be defined via |\def| commands.
This command line makes the \TeX{} engine believe
it is compiling the file \textit{target}
whose content is specified as the latter parameter.
The provided code then forwards the processing to
\textit{main} or \textit{dest} as described in \secref{sec:forward}.

%%%%%%%%%%%%%%%%%%%%%%%%%%%%%%%%%%%%%%%%%%%%%%%%%%%%%%%%%%%%%%%%%%%%%%%%%%%%%%%%
\subsection{Include by Input}
\label{sec:input}

Including child documents by |\include| has some restrictions by design.
Most notably, the content of a child document always occupies
its own set of pages; pages cannot be shared between child documents.
Usually, this behaviour makes perfect sense
because each child document contain an essential part of the document.
However, in some situations it may be desirable to compose
a document from a collection of parts
without having mandatory page breaks between then.
For this case, the package
provides a mechanism to include parts
by |\input| which can also be processed individually.
However, by construction this mechanism
requires manual handling of the content to be output.

%%%%%%%%%%%%%%%%%%%%%%%%%%%%%%%%%%%%%%%%
\DescribeMacro{\ifchilddocmanual}
The main file should be prepared as usual, see \secref{sec:include}.
However, the document body must make a distinction
between processing of an individual part and of the main document, e.g.:
%
\begin{center}
\begin{tabular}{l}
|\ifchilddocmanual|\\
|\input{\childdocname}|\\
|\||else|\\
\textit{document body with }|\input{|\textit{part}|}|\\
|\||fi|
\end{tabular}
\end{center}
%
The conditional |\ifchilddocmanual| is true whenever
a part to be included by |\input| is being compiled,
and the name of the part is stored in |\childdocname|.

%%%%%%%%%%%%%%%%%%%%%%%%%%%%%%%%%%%%%%%%
\DescribeMacro{\childdocby}
Each part to be included by |\input| should start with:
%
\begin{center}
\begin{tabular}{l}
|\input{childdoc.def}|\\
|\childdocby{|\textit{main}|}|\\
\end{tabular}
\end{center}
%
The directive |\childdocby| is similar to |\childdocof|
described in \secref{sec:include},
but the subsequent selection of content must be done manually.
To that end, both |\ifchilddoc| and |\ifchilddocmanual|
will be true upon processing of a part,
and the name of the part is stored in |\childdocname|.
Note that |\jobname| will be set to the filename of the current part
so that each part receives an individual |.aux| file
that does not interfere with the |.aux| file(s) of the main document.
This behaviour can be altered by the alternative form
|\childdocby[*]{|\textit{main}|}| (with a non-empty optional argument)
which uses the |.aux| file of the main document
by setting |\jobname| to \textit{main}.

%%%%%%%%%%%%%%%%%%%%%%%%%%%%%%%%%%%%%%%%%%%%%%%%%%%%%%%%%%%%%%%%%%%%%%%%%%%%%%%%
\subsection{Driver Development}
\label{sec:driver}

The \textsf{childdoc} mechanism can also be use for the development
of definition files such as \LaTeX{} styles or classes.
This case differs from the above setup with multiple parts
included by |\include| in that no |\includeonly| should be invoked.
This can be achieved by starting the include file
(before |\ProvidesPackage|) with:
%
\begin{center}
\begin{tabular}{l}
|\input{childdoc.def}|\\
|\childdocforward{|\textit{main}|}|\\
\end{tabular}
\end{center}
%
or alternatively with:
%
\begin{center}
\begin{tabular}{l}
|\input{childdoc.def}|\\
|\childdocby{|\textit{main}|}|\\
\end{tabular}
\end{center}
%
Both forms have slightly different effects as described above.
The main file is prepared as usual, see \secref{sec:include}.

%%%%%%%%%%%%%%%%%%%%%%%%%%%%%%%%%%%%%%%%%%%%%%%%%%%%%%%%%%%%%%%%%%%%%%%%%%%%%%%%
\subsection{Legacy Detection}
\label{sec:detection}

The directive |\childdocmain| in the main file can detect
whether the complete document or merely a child is to be compiled
even without using the directive |\childdocof|.
This method is deprecated because it is less robust
and there is no compelling reason to use it;
it is merely provided for backward compatibility
and it may be removed in future versions.

If the detection mechanism is to be used,
it is mandatory to correctly specify
the filename of the main file as the argument of |\childdocmain|:
%
\begin{center}
\begin{tabular}{l}
|\input{childdoc.def}|\\
|\childdocmain{|\textit{main}|}|\\
\end{tabular}
\end{center}
%
If |\jobname| does not match the argument \textit{main} of |\childdocmain|,
it is assumed that |\jobname| points to the child file to be compiled.
When using |\childdocmain| with the main file specified as argument,
it suffices to start a child file
with just |\input{|\textit{main}|}|
without loading of the package and using |\childdocof|.
If instead all processing is done
with the appropriate \textsf{childdoc} directives,
the argument of \textit{main} of |\childdocmain| can be empty.

An alternative version of the command line processing described
in \secref{sec:commandline} using the detection mechanism reads:
%
\begin{center}
|... -jobname "|\textit{target}|" "|[\textit{flags}]%
[|\def\jobname{|\textit{dest}|}|]|\input{|\textit{main}|}"|
\end{center}

%%%%%%%%%%%%%%%%%%%%%%%%%%%%%%%%%%%%%%%%%%%%%%%%%%%%%%%%%%%%%%%%%%%%%%%%%%%%%%%%
\subsection{Manual Code}
\label{sec:manual}

In case one cannot be certain whether the definitions file |childdoc.def|
is installed on the target \TeX{} distribution
and one prefers not to ship it,
it is conceivable to paste a few relevant commands into the sources.

To that end, drop all statements |\input{childdoc.def}|
and perform the replacements as outlined below.
Instead of |\childdocmain{|\textit{main}|}| add the following code
to the top of the main file:
%
\begin{center}
\begin{tabular}{l}
|\||ifdefined\childdocname\endinput\||fi\newif\ifchilddoc|\\
|\edef\childdocname{\scantokens\expandafter{\jobname\noexpand}}|\\
|\def\childdocmain{|\textit{main}|}\||ifx\childdocmain\childdocname\||else|\\
|\childdoctrue\includeonly{\childdocname}\let\jobname\childdocmain\||fi|\\
\end{tabular}
\end{center}
%
Instead of |\childdocof{|\textit{main}|}| just include the main file
at the top of each child file:
%
\begin{center}
|\input{|\textit{main}|}|
\end{center}
%
A simple redirection |\childdocforward{|\textit{dest}|}| is achieved by:
%
\begin{center}
|\def\jobname{|\textit{dest}|}\input{\jobname}|
\end{center}
%
The redirection with prefix
|\childdocforwardprefix[|\textit{prefix}|]{|\textit{dest}|}|
is accomplished by:
%
\begin{center}
\begin{tabular}{l}
|{\edef\jobname{\scantokens\expandafter{\jobname\noexpand}}|\\
|\def\redirectjob |\textit{prefix}|#1~~~{\gdef\jobname{|\textit{dest}|#1}}|\\
|\expandafter\redirectjob\jobname~~~}\input{\jobname}|
\end{tabular}
\end{center}

In an alternative approach,
child documents can be compiled by a specific command line
without additional code or specific definitions:
%
\begin{center}
|... -jobname "|\textit{target}|" "|[\textit{flags}]%
|\includeonly{|\textit{dest}|}\input{|\textit{main}|}"|
\end{center}
%

%%%%%%%%%%%%%%%%%%%%%%%%%%%%%%%%%%%%%%%%%%%%%%%%%%%%%%%%%%%%%%%%%%%%%%%%%%%%%%%%
%%%%%%%%%%%%%%%%%%%%%%%%%%%%%%%%%%%%%%%%%%%%%%%%%%%%%%%%%%%%%%%%%%%%%%%%%%%%%%%%
\section{Information}

%%%%%%%%%%%%%%%%%%%%%%%%%%%%%%%%%%%%%%%%%%%%%%%%%%%%%%%%%%%%%%%%%%%%%%%%%%%%%%%%
\subsection{Copyright}

Copyright \copyright{} 2017--2018 Niklas Beisert

This work may be distributed and/or modified under the
conditions of the \LaTeX{} Project Public License, either version 1.3
of this license or (at your option) any later version.
The latest version of this license is in
  \url{http://www.latex-project.org/lppl.txt}
and version 1.3 or later is part of all distributions of \LaTeX{}
version 2005/12/01 or later.

This work has the LPPL maintenance status `maintained'.

The Current Maintainer of this work is Niklas Beisert.

This work consists of the files |README.txt|, |childdoc.ins| and |childdoc.dtx|
as well as the derived files |childdoc.def|, |cdocsamp.tex|
with |cdocsch1.tex|, |cdocsch2.tex|, |cdocspt3.tex|, |cdocspt4.tex|,
|cdocsdrf.tex|, |cdocsfn1.tex|, |cdocsfn2.tex|
as well as |childdoc.pdf|.

%%%%%%%%%%%%%%%%%%%%%%%%%%%%%%%%%%%%%%%%%%%%%%%%%%%%%%%%%%%%%%%%%%%%%%%%%%%%%%%%
\subsection{Files and Installation}

The package consists of the files:
%
\begin{center}
\begin{tabular}{ll}
    |README.txt|   & readme file \\
    |childdoc.ins| & installation file \\
    |childdoc.dtx| & source file \\
    |childdoc.def| & definition file \\
    |cdocsamp.tex| & sample main file \\
    |cdocsch1.tex| & sample include file \\
    |cdocsch2.tex| & sample include file \\
    |cdocspt3.tex| & sample part file \\
    |cdocspt4.tex| & sample part file \\
    |cdocsdrf.tex| & sample redirection file \\
    |cdocsfn1.tex| & sample redirection file \\
    |cdocsfn2.tex| & sample redirection file \\
    |childdoc.pdf| & manual
\end{tabular}
\end{center}
%
The distribution consists of the files
|README.txt|, |childdoc.ins| and |childdoc.dtx|.
%
\begin{itemize}
\item
Run (pdf)\LaTeX{} on |childdoc.dtx|
to compile the manual |childdoc.pdf| (this file).
\item
Run \LaTeX{} on |childdoc.ins| to create the definitions file |childdoc.def|
and the sample |cdocsamp.tex| with include files
|cdocsch1.tex|, |cdocsch2.tex|, |cdocspt3.tex|, |cdocspt4.tex|,
|cdocsdrf.tex|, |cdocsfn1.tex|, |cdocsfn2.tex|.
Then copy the file |childdoc.def| to an appropriate directory of your \LaTeX{}
distribution, e.g.\ \textit{texmf-root}|/tex/latex/childdoc|.
\end{itemize}

%%%%%%%%%%%%%%%%%%%%%%%%%%%%%%%%%%%%%%%%%%%%%%%%%%%%%%%%%%%%%%%%%%%%%%%%%%%%%%%%
\subsection{Related CTAN Packages}

There are several other packages which offer a similar functionality:
%
\begin{itemize}
\item
The packages
\href{http://ctan.org/pkg/docmute}{\textsf{docmute}},
\href{http://ctan.org/pkg/includex}{\textsf{includex}} and
\href{http://ctan.org/pkg/standalone}{\textsf{standalone}}
provide commands to include only the document body of
a child file thus allowing both files to be compiled individually.
\item
The packages \href{http://ctan.org/pkg/subdocs}{\textsf{subdocs}}
and \href{http://ctan.org/pkg/subfiles}{\textsf{subfiles}}
provide structures in which the main and child documents can be
encapsulated and allowing them to be compiled individually.
The inclusion mechanism is different from the conventional |\include|.
\item
The package \href{http://ctan.org/pkg/combine}{\textsf{combine}}
is an elaborate solution to combine several documents into one.
\end{itemize}
%
See also the CTAN topic \href{http://ctan.org/topic/subdocs}{\textsf{subdocs}}
for further related packages.
The present package differs from the above solutions in that
a document structure constructed with the conventional |\include| mechanism
just needs two extra commands at the top of every file
such that all constituent files can be compiled individually.

%%%%%%%%%%%%%%%%%%%%%%%%%%%%%%%%%%%%%%%%%%%%%%%%%%%%%%%%%%%%%%%%%%%%%%%%%%%%%%%%
%\subsection{Feature Suggestions}
%
%The following is a list of features which may be useful for future
%versions of this package:
%%
%\begin{itemize}
%\item
%\ldots
%\end{itemize}

%%%%%%%%%%%%%%%%%%%%%%%%%%%%%%%%%%%%%%%%%%%%%%%%%%%%%%%%%%%%%%%%%%%%%%%%%%%%%%%%
\subsection{Revision History}

%%%%%%%%%%%%%%%%%%%%%%%%%%%%%%%%%%%%%%%%
\paragraph{v2.0:} 2018/12/30

\begin{itemize}
\item
immediate forward processing
\item
added |\childdocby| mechanism
\item
manual restructured
\end{itemize}

%%%%%%%%%%%%%%%%%%%%%%%%%%%%%%%%%%%%%%%%
\paragraph{v1.6:} 2018/01/17

\begin{itemize}
\item
application for development of include files
\item
corrections to manual
\end{itemize}

%%%%%%%%%%%%%%%%%%%%%%%%%%%%%%%%%%%%%%%%
\paragraph{v1.5:} 2017/05/21

\begin{itemize}
\item
more complete structuring introduced
\item
|\childdocof| introduced
\item
|\childdoc| renamed to |\childdocmain|
\item
|\childredirect| renamed to |\childdocforward| and |\childdocforwardprefix|
and functionality expanded
\end{itemize}

%%%%%%%%%%%%%%%%%%%%%%%%%%%%%%%%%%%%%%%%
\paragraph{v1.0:} 2017/04/27

\begin{itemize}
\item
manual and install package
\item
first version published on CTAN
\end{itemize}

%%%%%%%%%%%%%%%%%%%%%%%%%%%%%%%%%%%%%%%%
\paragraph{v0.6:} 2017/04/26

\begin{itemize}
\item
redirection mechanism added
\end{itemize}

%%%%%%%%%%%%%%%%%%%%%%%%%%%%%%%%%%%%%%%%
\paragraph{v0.5:} 2017/04/26

\begin{itemize}
\item
functionality in definition file
\end{itemize}


%%%%%%%%%%%%%%%%%%%%%%%%%%%%%%%%%%%%%%%%%%%%%%%%%%%%%%%%%%%%%%%%%%%%%%%%%%%%%%%%
%%%%%%%%%%%%%%%%%%%%%%%%%%%%%%%%%%%%%%%%%%%%%%%%%%%%%%%%%%%%%%%%%%%%%%%%%%%%%%%%
%%%%%%%%%%%%%%%%%%%%%%%%%%%%%%%%%%%%%%%%%%%%%%%%%%%%%%%%%%%%%%%%%%%%%%%%%%%%%%%%
\appendix

\settowidth\MacroIndent{\rmfamily\scriptsize 000\ }

 \DocInput{childdoc.dtx}

\end{document}
%</driver>
% \fi
%
% %%%%%%%%%%%%%%%%%%%%%%%%%%%%%%%%%%%%%%%%%%%%%%%%%%%%%%%%%%%%%%%%%%%%%%%%%%%%%%
% %%%%%%%%%%%%%%%%%%%%%%%%%%%%%%%%%%%%%%%%%%%%%%%%%%%%%%%%%%%%%%%%%%%%%%%%%%%%%%
% \section{Sample}
%\iffalse
%<*samplemain>
%\fi
%
% The following presents a sample document
% with two chapters, two parts, a title page,
% a compile flag as well as three forwarding files to set the flag.
% It consists of eight |.tex| files:
% \begin{center}
% \begin{tabular}{ll}
% |cdocsamp.tex|&main file\\
% |cdocsch1.tex|&include file for chapter 1\\
% |cdocsch2.tex|&include file for chapter 2\\
% |cdocspt3.tex|&include file for part 3\\
% |cdocspt4.tex|&include file for part 4\\
% |cdocsdrf.tex|&forwarding file for main file in draft mode\\
% |cdocsfi1.tex|&forwarding file for final version of chapter 1\\
% |cdocsfi2.tex|&forwarding file for final version of chapter 2\\
% \end{tabular}
% \end{center}
% Each of the eight files can be compiled directly by the \LaTeX{} compiler.
%
% %%%%%%%%%%%%%%%%%%%%%%%%%%%%%%%%%%%%%%
% \paragraph{Main File.}
%
% The main file is called |cdocsamp.tex|.
%
% Load the \textsf{childdoc} definitions and
% declare the filename for the main document:
%    \begin{macrocode}
\input{childdoc.def}
\childdocmain{}
%    \end{macrocode}

% Optional override for |\version| flag:
%    \begin{macrocode}
%%\ifchilddoc\else\providecommand{\version}{draft}\fi
%    \end{macrocode}

% Define the default values for the |\version| flag
% (|final| for the main file and |draft| for childs):
%    \begin{macrocode}
\ifchilddoc
\providecommand{\version}{draft}
\else
\providecommand{\version}{final}
\fi
%    \end{macrocode}

% Load the standard document class:
%    \begin{macrocode}
\documentclass[12pt]{article}
%    \end{macrocode}

% Start the document body:
%    \begin{macrocode}
\begin{document}
%    \end{macrocode}

% Declare a title page.
% Print title, part of document being processed and version flag:
%    \begin{macrocode}
\addtocounter{page}{-1}
\begin{center}
{\LARGE\bfseries{}childdoc example\par}
\vspace{1cm}
\ifchilddoc
\ifchilddocmanual part\else chapter\fi:
`\childdocname' of `\childdocjob'\par
\else
main document: `\childdocjob'\par
\fi
version: \version\par
\end{center}
\newpage
%    \end{macrocode}

% Manually include selected file,
% otherwise process as usual:
%    \begin{macrocode}
\ifchilddocmanual
\section*{part `\childdocname'}
\input{\childdocname}
\else
%    \end{macrocode}

% Include the two chapters:
%    \begin{macrocode}
\include{cdocsch1}
\include{cdocsch2}
%    \end{macrocode}

% Include the two parts unless only chapters should be displayed:
%    \begin{macrocode}
\ifchilddoc\else
\section{part three}
\input{cdocspt3}
\section{part four}
\input{cdocspt4}
\fi
%    \end{macrocode}

% Process as usual until here:
%    \begin{macrocode}
\fi
%    \end{macrocode}

% End of document body:
%    \begin{macrocode}
\end{document}
%    \end{macrocode}
%\iffalse
%</samplemain>
%\fi
%
% %%%%%%%%%%%%%%%%%%%%%%%%%%%%%%%%%%%%%%
% \paragraph{Chapter Include Files.}
%
% The include files are called |cdocsch1.tex| and |cdocsch2.tex|.
%
%\iffalse
%<*samplechap1|samplechap2>
%\fi

% Optional override for |\version| flag:
%    \begin{macrocode}
%%\providecommand{\version}{final}
%    \end{macrocode}

% Include the main document:
%    \begin{macrocode}
\input{childdoc.def}
\childdocof{cdocsamp}
%    \end{macrocode}

%\iffalse
%</samplechap1|samplechap2>
%\fi
%
%\iffalse
%<*samplechap1>
%\fi
% Some text for chapter 1:
%    \begin{macrocode}
\section{one}
some text in chapter one
%    \end{macrocode}

%\iffalse
%</samplechap1>
%\fi
% Some text for chapter 2:
%\iffalse
%<*samplechap2>
%\fi
%    \begin{macrocode}
\section{two}
more text in chapter two
%    \end{macrocode}

%\iffalse
%</samplechap2>
%\fi
%
% %%%%%%%%%%%%%%%%%%%%%%%%%%%%%%%%%%%%%%
% \paragraph{Part Include Files.}
%
% The include files are called |cdocspt3.tex| and |cdocspt4.tex|.
%
%\iffalse
%<*samplepart3|samplepart4>
%\fi

% Optional override for |\version| flag:
%    \begin{macrocode}
%%\providecommand{\version}{final}
%    \end{macrocode}

% Include the main document:
%    \begin{macrocode}
\input{childdoc.def}
\childdocby{cdocsamp}
%    \end{macrocode}

%\iffalse
%</samplepart3|samplepart4>
%\fi
%
%\iffalse
%<*samplepart3>
%\fi
% Some text for part 3:
%    \begin{macrocode}
some text in part three
%    \end{macrocode}

%\iffalse
%</samplepart3>
%\fi
% Some text for part 4:
%\iffalse
%<*samplepart4>
%\fi
%    \begin{macrocode}
more text in part four
%    \end{macrocode}

%\iffalse
%</samplepart4>
%\fi
%
% %%%%%%%%%%%%%%%%%%%%%%%%%%%%%%%%%%%%%%
% \paragraph{Forwarding for a Complete Draft.}
%
% The following forwarding file |cdocsdrf.tex|
% compiles the main document in draft mode:
%\iffalse
%<*sampledraft>
%\fi
%    \begin{macrocode}
\def\version{draft}
\input{childdoc.def}
\childdocforward{cdocsamp}
%    \end{macrocode}

%\iffalse
%</sampledraft>
%\fi
%
% %%%%%%%%%%%%%%%%%%%%%%%%%%%%%%%%%%%%%%
% \paragraph{Forwarding for Final Version of the Chapters.}
%
% The following forwarding files |cdocsfn1.tex| and |cdocsfn2.tex|
% (with identical content)
% compile the final versions of the child documents
% |cdocsch1.tex| and |cdocsch2.tex|, respectively:
%\iffalse
%<*samplefinal>
%\fi
%    \begin{macrocode}
\def\version{final}
\input{childdoc.def}
\childdocforwardprefix[cdocsamp]{cdocsfn}{cdocsch}
%    \end{macrocode}

%\iffalse
%</samplefinal>
%\fi
%
% %%%%%%%%%%%%%%%%%%%%%%%%%%%%%%%%%%%%%%
% \paragraph{Command Line Processing.}
%
% The following three command lines generate the output files
% |cdocscld|, |cdocscl1| and |cdocscl2|
% which should be identical to
% |cdocsdrf|, |cdocsch1| and |cdocsfn2|, respectively:
% \begin{center}
% \begin{tabular}{l}
% |latex -jobname cdocscld \|\\
% |  "\def\version{draft}\input{childdoc.def}\childdocforward{cdocsamp}"|\\
% |latex -jobname cdocscl1 \|\\
% |  "\input{childdoc.def}\childdocforward[cdocsamp]{cdocsch1}"|\\
% |latex -jobname cdocscl2 \|\\
% |  "\def\version{final}\input{childdoc.def}\childdocforward{cdocsch2}"|
% \end{tabular}
% \end{center}
% Note that the trailing backslash on each first line
% merely continues the input to the second line
% (for convenient cut ant paste).
% Furthermore, the command |latex| can be replaced by any
% of its alternative versions such as |pdflatex|.
%
% %%%%%%%%%%%%%%%%%%%%%%%%%%%%%%%%%%%%%%%%%%%%%%%%%%%%%%%%%%%%%%%%%%%%%%%%%%%%%%
% %%%%%%%%%%%%%%%%%%%%%%%%%%%%%%%%%%%%%%%%%%%%%%%%%%%%%%%%%%%%%%%%%%%%%%%%%%%%%%
% \section{Implementation}
%\iffalse
%<*package>
%\fi
%
% This section describes the definitions file |childdoc.def|.

% The definitions cannot be loaded using |\usepackage| or |\RequirePackage|
% which has a mechanism to prevent loading a style file more than once.
% When loading the definitions by means of |\input|
% multiple instances have to be prevented manually:
%\iffalse
%This code needs to be before the `\ProvidesFile' directive
%which is defined at the beginning of this file.
%Therefore it is also placed there and commented out here.
%</package>
%<*discard>
%\fi
%    \begin{macrocode}
\ifdefined\childdocmain\endinput\fi
%    \end{macrocode}
%\iffalse
%</discard>
%<*package>
%\fi
%
% \macro{\ifchilddoc}
% \macro{\ifchilddocmanual}
% The conditional |\ifchilddoc| tells whether a
% child (true) or main (false) document is being compiled.
% The conditional |\ifchilddocmanual| tells whether
% the |\includeonly| mechanism is used (false) or
% the selection of child files must be performed manually (true).
% The definitions initialise to false:
%    \begin{macrocode}
\newif\ifchilddoc
\newif\ifchilddocmanual
%    \end{macrocode}

% \macro{\childdocname}
% \macro{\childdocjob}
% The macro |\childdocname| stores the name of the main document
% to be compiled. The macro |\childdocjob| stores the name of
% the document on which the \LaTeX{} compiler was originally invoked.
% The content of |\jobname| cannot be compared
% to filenames specified in the source due to different catcodes.
% The following code rescans |\jobname|, stores the result
% in |\childdocname| and saves a copy in |\childdocjob|:
%    \begin{macrocode}
\edef\childdocname{\scantokens\expandafter{\jobname\noexpand}}
\let\childdocjob\childdocname
%    \end{macrocode}

% \macro{\childdocdisable}
% The macro |\childdocdisable| prevents the main file
% from being processed more than once.
% At this stage, the main document command |\childdocmain|
% is assumed to be called once again where it should do nothing.
% Any subsequent call to it should prevent
% a secondary processing of the main document
% It overwrites the forwarding commands
% |\childdocof| and |\childdocforward|
% with empty macros to prevent further inclusions of the main document:
%    \begin{macrocode}
\newcommand{\childdocdisable}
{
  \renewcommand{\childdocmain}[1]{\renewcommand{\childdocmain}[1]{\endinput}}
  \renewcommand{\childdocof}[1]{}
  \renewcommand{\childdocby}[2][]{}
  \renewcommand{\childdocforward}[2][]{}
  \renewcommand{\childdocdisable}{}
}
%    \end{macrocode}

% \macro{\childdocmain}
% The macro |\childdocmain| is to be called at the top of the main file
% with nothing or the main filename (without extension) as argument.
% First, it breaks loops.
% If the argument is not empty and does not match |\childdocname|
% (which is set by the first inclusion of |childdoc.def|),
% |\ifchilddoc| is set to true, |\includeonly| is applied to the child file
% and |\jobname| is set to the main file
% (for proper handling of |.aux| files):
%    \begin{macrocode}
\newcommand{\childdocmain}[1]
{
  \childdocdisable\childdocmain{}
  \if?#1?\else
    \begingroup
      \def\childdoctmp{#1}
      \ifx\childdoctmp\childdocname
        \def\childdoctmp{}
      \else
        \def\childdoctmp
        {
          \childdoctrue
          \includeonly{\childdocname}
          \def\childdocjob{#1}
          \def\jobname{#1}
        }
      \fi
      \expandafter
    \endgroup
    \childdoctmp
  \fi
}
%    \end{macrocode}

% \macro{\childdocof}
% The command |\childdocof| redirects
% compilation to the main file |#1|.
%    \begin{macrocode}
\newcommand{\childdocof}[1]
{
  \childdocdisable
  \childdoctrue
  \includeonly{\childdocname}
  \def\jobname{#1}
  \def\childdocjob{#1}
  \input{#1}
}
%    \end{macrocode}

% \macro{\childdocby}
% The command |\childdocby| ....
%    \begin{macrocode}
\newcommand{\childdocby}[2][]
{
  \childdocdisable
  \childdoctrue
  \childdocmanualtrue
  \if?#1?\else
    \def\jobname{#2}
  \fi
  \def\childdocjob{#2}
  \input{#2}
  \endinput
}
%    \end{macrocode}

% \macro{\childdocforward}
% The command |\childdocforward| redirects
% compilation to the main file or
% (if the optional argument is given) a child file.
% Parameters are set as if the main file
% or a child file starting with |\childdocof| was compiled.
% Then compilation is handed over to the main file:
%    \begin{macrocode}
\newcommand{\childdocforward}[2][]
{
  \begingroup
    \if?#1?
      \def\childdoctmp
      {
        \def\childdocname{#2}
        \def\childdocjob{#2}
        \def\jobname{#2}
        \input{#2}
        \endinput
      }
    \else
      \def\childdoctmp
      {
        \childdocdisable
        \def\childdocname{#2}
        \childdoctrue
        \includeonly{#2}
        \def\childdocjob{#1}
        \def\jobname{#1}
        \input{#1}
        \endinput
      }
    \fi
    \expandafter
  \endgroup
  \childdoctmp
}
%    \end{macrocode}

% \macro{\childdocforwardprefix}
% The command |\childdocforwardprefix| redirects
% compilation to the main or a child file by means of a pattern.
% The prefix |#1| in the current filename is replaced by |#2|
% and the suffix of the current filename is kept
% (it is assumed that the filename does not contain the substring `|~~~|'
% which is used as a delimiter).
% Compilation is handed over to the new file by |\childdocforward|:
%    \begin{macrocode}
\newcommand{\childdocforwardprefix}[3][]
{
  \begingroup
    \def\childdocextract #2##1~~~{\def\childdoctmp{\childdocforward[#1]{#3##1}}}
    \expandafter\childdocextract\childdocname~~~
    \expandafter
  \endgroup
  \childdoctmp
}
%    \end{macrocode}

% \macro{\childdoc}
% The deprecated macro |\childdoc| is a legacy version of |\childdocmain|:
%    \begin{macrocode}
\newcommand{\childdoc}{\childdocmain}
%    \end{macrocode}

% \macro{\childdocredirect}
% The deprecated macro |\childdocredirect| is a legacy version
% of |\childdocforward| and |\childdocforwardprefix|:
%    \begin{macrocode}
\newcommand{\childdocredirect}[2][]
{
  \begingroup
    \if?#1?
      \def\childdoctmp{\childdocforward{#2}}
    \else
      \def\childdoctmp{\childdocforwardprefix{#1}{#2}}
    \fi
    \expandafter
  \endgroup
  \childdoctmp
}
%    \end{macrocode}

%\iffalse
%</package>
%\fi
%
\endinput
|\\
|\childdocby{|\textit{main}|}|\\
\end{tabular}
\end{center}
%
The directive |\childdocby| is similar to |\childdocof|
described in \secref{sec:include},
but the subsequent selection of content must be done manually.
To that end, both |\ifchilddoc| and |\ifchilddocmanual|
will be true upon processing of a part,
and the name of the part is stored in |\childdocname|.
Note that |\jobname| will be set to the filename of the current part
so that each part receives an individual |.aux| file
that does not interfere with the |.aux| file(s) of the main document.
This behaviour can be altered by the alternative form
|\childdocby[*]{|\textit{main}|}| (with a non-empty optional argument)
which uses the |.aux| file of the main document
by setting |\jobname| to \textit{main}.

%%%%%%%%%%%%%%%%%%%%%%%%%%%%%%%%%%%%%%%%%%%%%%%%%%%%%%%%%%%%%%%%%%%%%%%%%%%%%%%%
\subsection{Driver Development}
\label{sec:driver}

The \textsf{childdoc} mechanism can also be use for the development
of definition files such as \LaTeX{} styles or classes.
This case differs from the above setup with multiple parts
included by |\include| in that no |\includeonly| should be invoked.
This can be achieved by starting the include file
(before |\ProvidesPackage|) with:
%
\begin{center}
\begin{tabular}{l}
|% \iffalse
%
% childdoc.dtx Copyright (C) 2017-2018 Niklas Beisert
%
% This work may be distributed and/or modified under the
% conditions of the LaTeX Project Public License, either version 1.3
% of this license or (at your option) any later version.
% The latest version of this license is in
%   http://www.latex-project.org/lppl.txt
% and version 1.3 or later is part of all distributions of LaTeX
% version 2005/12/01 or later.
%
% This work has the LPPL maintenance status `maintained'.
%
% The Current Maintainer of this work is Niklas Beisert.
%
% This work consists of the files childdoc.dtx and childdoc.ins
% and the derived files childdoc.def and cdocsamp.tex with
% cdocsch1.tex, cdocsch2.tex, cdocsdrf.tex, cdocsfn1.tex, cdocsfn2.tex.
%
%<package>\ifdefined\childdocmain\endinput\fi
%<package>\ProvidesFile{childdoc.def}[2018/12/30 v2.0 child document driver]
%<samplemain>\ProvidesFile{cdocsamp.tex}[2018/12/30 v2.0 sample for childdoc]
%<*driver>
%\ProvidesFile{childdoc.drv}[2018/12/30 v2.0 childdoc reference manual file]
\PassOptionsToClass{10pt,a4paper}{article}
\documentclass{ltxdoc}

\usepackage[margin=35mm]{geometry}
\usepackage{hyperref}
\usepackage{hyperxmp}
\usepackage[usenames]{color}

\hypersetup{colorlinks=true}
\hypersetup{pdfstartview=FitH}
\hypersetup{pdfpagemode=UseNone}
\hypersetup{pdfsource={}}
\hypersetup{pdflang={en-UK}}
\hypersetup{pdfcopyright={Copyright 2017-2018 Niklas Beisert.
  This work may be distributed and/or modified under the
  conditions of the LaTeX Project Public License, either version 1.3
  of this license or (at your option) any later version.}}
\hypersetup{pdflicenseurl={http://www.latex-project.org/lppl.txt}}
\hypersetup{pdfcontactaddress={ETH Zurich, ITP, HIT K,
  Wolfgang-Pauli-Strasse 27}}
\hypersetup{pdfcontactpostcode={8093}}
\hypersetup{pdfcontactcity={Zurich}}
\hypersetup{pdfcontactcountry={Switzerland}}
\hypersetup{pdfcontactemail={nbeisert@itp.phys.ethz.ch}}
\hypersetup{pdfcontacturl={http://people.phys.ethz.ch/\xmptilde nbeisert/}}

\newcommand{\secref}[1]{\hyperref[#1]{section \ref*{#1}}}

\parskip1ex
\parindent0pt
\let\olditemize\itemize
\def\itemize{\olditemize\parskip0pt}

\begin{document}

\title{The \textsf{childdoc} Package}
\hypersetup{pdftitle={The childdoc Package}}
\author{Niklas Beisert\\[2ex]
  Institut f\"ur Theoretische Physik\\
  Eidgen\"ossische Technische Hochschule Z\"urich\\
  Wolfgang-Pauli-Strasse 27, 8093 Z\"urich, Switzerland\\[1ex]
  \href{mailto:nbeisert@itp.phys.ethz.ch}
  {\texttt{nbeisert@itp.phys.ethz.ch}}}
\hypersetup{pdfauthor={Niklas Beisert}}
\hypersetup{pdfsubject={Manual for the LaTeX2e Package childdoc}}
\date{30 December 2018, \textsf{v2.0}}
\maketitle

\begin{abstract}\noindent
\textsf{childdoc} is a \LaTeXe{} package
that enables the direct compilation
of document sections included by |\include|
to individual files.
\end{abstract}

\begingroup
\parskip0ex
\tableofcontents
\endgroup

%%%%%%%%%%%%%%%%%%%%%%%%%%%%%%%%%%%%%%%%%%%%%%%%%%%%%%%%%%%%%%%%%%%%%%%%%%%%%%%%
%%%%%%%%%%%%%%%%%%%%%%%%%%%%%%%%%%%%%%%%%%%%%%%%%%%%%%%%%%%%%%%%%%%%%%%%%%%%%%%%
\section{Introduction}

\LaTeX{} provides a mechanism to structure a large document (such as a book)
into a main file and several child files (containing the chapters)
using the |\include| command.
This mechanism is beneficial for documents
which span hundreds of pages in order to
make the source file(s) more manageable.
Moreover, compilation can be restricted to
selected child files by means of the |\includeonly| command.
The latter feature can be used to reduce the compilation time while editing
(this was significantly more useful in the earlier days of \LaTeX{})
or to generate a smaller document which is easier to navigate.
Another application of |\includeonly| is to generate
documents consisting of selected parts of the complete document.

However, there are a few drawbacks of the plain |\include| mechanism:
\begin{itemize}
\item
The child files cannot be compiled on their own,
they can only be compiled via the main file.
A naive editing environment
(such as a text editor with an option
to have the current file processed by \LaTeX)
may require one to switch to the main file before compiling;
attempting to compile the child file produces errors.
\item
The main file must be modified (each time)
to adjust the |\includeonly| command
to the present needs. This easily leaves the main file in a messy state.
\item
The generated document will always carry the filename
of the main document. This is inconvenient if
several child files are to be compiled and
to be kept for distribution.
\end{itemize}

The present package provides a simple interface
to make child files individually compilable by \LaTeX{}.
Compiling a child file then has the same effect as compiling
the main file with an |\includeonly| command
to select the appropriate child.
Moreover the generated document will carry the name of the child
rather than the main file.
This resolves all three above issues.

This feature is meant to make the editing of books,
thesis documents and lecture notes somewhat more convenient.
However, the package can also be used efficiently for
composing a series of documents (such as exercise sheets)
which are typically distributed individually.
It then assists the author in generating the individual documents
(potentially in different versions)
as well as a document containing the collected series.
Another application is in developing style files
or other kinds of included material
where compilation of the style file could redirect
to a sample or test file.

%%%%%%%%%%%%%%%%%%%%%%%%%%%%%%%%%%%%%%%%%%%%%%%%%%%%%%%%%%%%%%%%%%%%%%%%%%%%%%%%
%%%%%%%%%%%%%%%%%%%%%%%%%%%%%%%%%%%%%%%%%%%%%%%%%%%%%%%%%%%%%%%%%%%%%%%%%%%%%%%%
\section{Usage}

First of all, the package \textsf{childdoc} is \emph{not} a standard
\LaTeXe{} |.sty| style file! Therefore it needs to be invoked in
a non-standard way.

%%%%%%%%%%%%%%%%%%%%%%%%%%%%%%%%%%%%%%%%%%%%%%%%%%%%%%%%%%%%%%%%%%%%%%%%%%%%%%%%
\subsection{Included Files}
\label{sec:include}

%%%%%%%%%%%%%%%%%%%%%%%%%%%%%%%%%%%%%%%%
\DescribeMacro{\childdocmain}
To use the package, add the commands
\begin{center}
\begin{tabular}{l}
|\input{childdoc.def}|\\
|\childdocmain{}|\\
\end{tabular}
\end{center}
at the very top of the main \LaTeX{} file,
in particular \emph{before} the |\documentclass| statement!
The argument of |\childdocmain| should be left empty
(but it must be present).

%%%%%%%%%%%%%%%%%%%%%%%%%%%%%%%%%%%%%%%%
\DescribeMacro{\childdocof}
Furthermore, add the commands
\begin{center}
\begin{tabular}{l}
|\input{childdoc.def}|\\
|\childdocof{|\textit{main}|}|\\
\end{tabular}
\end{center}
at the top of every child file \textit{child}
which is included by |\include{|\textit{child}|}|
from within the main file
(or at least for those files to be compiled individually).
The argument \textit{main} must be the filename of the main file.

There are a couple of
considerations in setting up the main and child documents:

%%%%%%%%%%%%%%%%%%%%%%%%%%%%%%%%%%%%%%%%
\paragraph{Restrictions.}

Please note the following restrictions:
\begin{itemize}
\item
|\childdocmain| must be called with one argument \textit{main}
to ensure compatibility with earlier version of the package.
It must either be empty (|\childdocmain{}|)
or precisely match the filename of the main file in which it is specified.
See \secref{sec:detection} for further information.
\item
The filename \textit{main} must be specified without the |.tex| extension.
\item
The filename \textit{main} is case sensitive
(even in case-insensitive file systems)
due to internal string comparison.
\item
The argument \textit{main} should be fully expanded, it cannot be a macro.
\item
Subdirectories and special characters should be avoided in filenames.
\item
The command |\childdocmain{|\textit{main}|}| must be followed by a whitespace.
It should not be followed immediately by another command
or by a comment mark `|%|'.
This is because the \TeX{} parser reads the token immediately following
the argument of |\childdocmain| and puts it
at the beginning of every child section;
however, a white\-space is ignored.
\end{itemize}

%%%%%%%%%%%%%%%%%%%%%%%%%%%%%%%%%%%%%%%%
\paragraph{Content of Main File.}

It is advisable to place all content in the child files included by |\include|.
Any output contained in the main file will appear in all child documents
unless suppressed manually;
it cannot be suppressed automatically by the |\includeonly| directive
and thus should normally be avoided.
A method to include some content in the main file
by means of conditional processing is described in \secref{sec:conditional}.

%%%%%%%%%%%%%%%%%%%%%%%%%%%%%%%%%%%%%%%%
\paragraph{Page Numbering.}

When only a part of the document is compiled,
the appropriate numbering of pages
(as well as other status parameters)
is determined from the |.aux| files.
The latter contain information from previous passes.
However this information needs to propagate through
all intermediate child documents.
Therefore the page numbering in child documents may well
be inconsistent until the complete document is compiled at least once.

A useful (if unconventional) way to always ensure a consistent
page numbering is to restart the numbering in each child document
and denote the pages by `\textit{child}|.|\textit{page}'
where \textit{child} represents the chapter/section number of the child file.
This can be achieved by the command
|\numberwithin{page}{|\textit{child}|}|
of the \textsf{amsmath} package
where \textit{child} can be |chapter| or |section|
depending on the chosen structuring.
Alternatively, one can modify the macro |\thepage| appropriately
and reset the counter |page| at the start of each child file.

%%%%%%%%%%%%%%%%%%%%%%%%%%%%%%%%%%%%%%%%%%%%%%%%%%%%%%%%%%%%%%%%%%%%%%%%%%%%%%%%
\subsection{Conditional Processing}
\label{sec:conditional}

The package provides a mechanism to compile different versions
of a document. To customise the versions further some conditional processing
can come in handy to distinguish which version is being compiled.
The package provides two macros to describe the compilation context:

%%%%%%%%%%%%%%%%%%%%%%%%%%%%%%%%%%%%%%%%
\DescribeMacro{\ifchilddoc}
The conditional |\ifchilddoc| distinguishes between the compilation of
child documents and the main document:
%
\begin{center}
|\ifchilddoc |\textit{child-code}| |[|\||else |\textit{main-code}]| \||fi|
\end{center}

%%%%%%%%%%%%%%%%%%%%%%%%%%%%%%%%%%%%%%%%
\DescribeMacro{\childdocname}
\DescribeMacro{\childdocjob}
The macro |\childdocname| contains the filename (without extension)
of the main or child file being processed.
Note that |\childdocjob| will always contain the name of the main file.

%%%%%%%%%%%%%%%%%%%%%%%%%%%%%%%%%%%%%%%%
\paragraph{Title Page.}

Conditional processing can be used to include a title or banner page
in the main document when proper precautions are taken.
Importantly, the code in the main file should ensure that the page counter
(as well as other status parameters which are stored in the |.aux| files)
takes the same value after the conditional processing.
Otherwise the page numbers may take divergent values
depending on which part is compiled.

For example, a title page could be declared by:
%
\begin{center}
\begin{tabular}{l}
|\ifchilddoc\||else|\\
|\addtocounter{page}{-1}|\\
\textit{code for title page}\\
|\newpage|\\
|\||fi|
\end{tabular}
\end{center}
%
A banner page for the child documents can be generated by:
%
\begin{center}
\begin{tabular}{l}
|\ifchilddoc|\\
|\addtocounter{page}{-1}|\\
\textit{code for banner page}\\
|\newpage|\\
|\||fi|
\end{tabular}
\end{center}
%
Here one could write a message such as:
\begin{center}
|This is the part \childdocname{} of \childdocjob{}.|
\end{center}

%%%%%%%%%%%%%%%%%%%%%%%%%%%%%%%%%%%%%%%%%%%%%%%%%%%%%%%%%%%%%%%%%%%%%%%%%%%%%%%%
\subsection{Flags}
\label{sec:flags}

The package makes it easy to generate different versions
of the main or child documents.
To this end compilation flags can be defined
and assigned different default values.
They will be particularly useful in conjunction
with the forwarding mechanism described in \secref{sec:forward}.

For example, it may be useful to have a flag |\version|
which can be set to |draft| or |final|.
The document source will contain some conditional code
depending on the value of |\version|.
Suppose further, the flag should default to |final| for the main file
and to |draft| for child files
which is a natural assignment for editing the document.
This is achieved by placing the following code
in the preamble of the main document
(below the |\childdocmain| directive):
%
\begin{center}
\begin{tabular}{l}
|\ifchilddoc|\\
|\providecommand{\version}{draft}|\\
|\||else|\\
|\providecommand{\version}{final}|\\
|\||fi|
\end{tabular}
\end{center}
%
The definition by |\providecommand| makes sure
that previous definitions are not overwritten.
Further statements |\providecommand{\version}{...}|
can thus be added before the above code to override it.

For the main file, one might add a line
(between |\childdocmain| and the above block)
%
\begin{center}
|%\ifchilddoc\||else\providecommand{\version}{draft}\||fi|
\end{center}
%
which can be uncommented to produce a draft version.
Likewise one can add a line to the very top of a child file
(above the |\childdocof{|\textit{main}|}| directive)
%
\begin{center}
|%\providecommand{\version}{final}|
\end{center}
%
which can be uncommented to produce the final version of this child document.

%%%%%%%%%%%%%%%%%%%%%%%%%%%%%%%%%%%%%%%%%%%%%%%%%%%%%%%%%%%%%%%%%%%%%%%%%%%%%%%%
\subsection{Forwarding}
\label{sec:forward}

Different versions of the main or child documents
using compilation flags as described in \secref{sec:flags}
can be (permanently) stored in different files
for convenient compilation, viewing and distribution.
To this end, the package defines a command
to pass on compilation to a different file:

%%%%%%%%%%%%%%%%%%%%%%%%%%%%%%%%%%%%%%%%
\DescribeMacro{\childdocforward}
The command |\childdocforward| redirects processing to
another source file:
%
\begin{center}
\begin{tabular}{l}
|\input{childdoc.def}|\\
|\childdocforward[|\textit{main}|]{|\textit{dest}|}|\\
\end{tabular}
\end{center}
%
The argument \textit{dest} is the destination file
(without extension).
It should be the main file or one of the child files.
Note that further \textsf{childdoc} directives
such as |\childdocof| and |\childdocforward|
in the indicated file will be processed in this form.
The optional argument \textit{main}
passes on directly to the main file \textit{main}
while pretending to compile the child \textit{dest}.
This form behaves as if \textit{dest}
issues |\childdocof{|\textit{main}|}| right away,
and no further \textsf{childdoc} directives will be processed.

%%%%%%%%%%%%%%%%%%%%%%%%%%%%%%%%%%%%%%%%
\DescribeMacro{\...prefix}
In the alternative form |\childdocforwardprefix|,
%
\begin{center}
\begin{tabular}{l}
|\input{childdoc.def}|\\
|\childdocforwardprefix[|\textit{main}|]{|\textit{prefix}|}{|\textit{dest}|}|
\end{tabular}
\end{center}
%
the destination file is determined by a pattern
depending on the current file:
To make this work, the current file must be called
`{\textit{prefix}\hspace{0.2em}\textit{suffix}}'
with \textit{prefix} matching precisely the argument.
Processing is then passed on to the file
`{\textit{dest}\hspace{0.2em}\textit{suffix}}'.
Surely, the same effect is achieved by
directly specifying the
argument `{\textit{dest}\hspace{0.2em}\textit{suffix}}'
in the first form.
However, that requires to set up a different file
for each child. With the alternative form of the command
all these files can have exactly the same content
which simplifies setting them up and maintaining them.

For example, the following file |draft.tex|
with a compilation flag |\version| as described in \secref{sec:flags}
compiles the main document as a draft:
%
\begin{center}
\begin{tabular}{l}
|\def\version{draft}|\\
|\input{childdoc.def}|\\
|\childdocforward{|\textit{main}|}|
\end{tabular}
\end{center}
%
Likewise, the following files |final|\textit{nn}|.tex|
compile the final version of the child document
|child|\textit{nn}|.tex|:
%
\begin{center}
\begin{tabular}{l}
|\def\version{final}|\\
|\input{childdoc.def}|\\
|\childdocforwardprefix{final}{child}|
\end{tabular}
\end{center}
%

Note that when several versions of a main file and/or of each child file
are to be generated, it may be convenient to set up a |Makefile| or
shell script to automatise the process.

%%%%%%%%%%%%%%%%%%%%%%%%%%%%%%%%%%%%%%%%%%%%%%%%%%%%%%%%%%%%%%%%%%%%%%%%%%%%%%%%
\subsection{Command Line Processing}
\label{sec:commandline}

The effect of redirection files can also be achieved by invoking
the \LaTeX{} compiler with a more elaborate command line.
Most conveniently this should be done as part
of a shell script or a |Makefile|.

When using \textsf{childdoc} in the main file, the following
command lines effectively perform a redirection
(note that depending on the shell being used,
backslashes may have to be doubled: `|\|' $\to$ `|\\|'):
%
\begin{center}
|... -jobname "|\textit{target}|" |\\|"|[\textit{flags}]%
|\input{childdoc.def}\childdocforward[|\textit{main}|]{|\textit{dest}|}"|
\end{center}
%
Here \textit{target} is the name of the output file,
\textit{main} is the name of the main file
and \textit{dest} is the name of the main or child file to be processed
(all filenames without extensions).
The optional argument \textit{main} can be omitted
if \textit{main} matches \textit{dest}.
Optionally, compilation \textit{flags} can be defined via |\def| commands.
This command line makes the \TeX{} engine believe
it is compiling the file \textit{target}
whose content is specified as the latter parameter.
The provided code then forwards the processing to
\textit{main} or \textit{dest} as described in \secref{sec:forward}.

%%%%%%%%%%%%%%%%%%%%%%%%%%%%%%%%%%%%%%%%%%%%%%%%%%%%%%%%%%%%%%%%%%%%%%%%%%%%%%%%
\subsection{Include by Input}
\label{sec:input}

Including child documents by |\include| has some restrictions by design.
Most notably, the content of a child document always occupies
its own set of pages; pages cannot be shared between child documents.
Usually, this behaviour makes perfect sense
because each child document contain an essential part of the document.
However, in some situations it may be desirable to compose
a document from a collection of parts
without having mandatory page breaks between then.
For this case, the package
provides a mechanism to include parts
by |\input| which can also be processed individually.
However, by construction this mechanism
requires manual handling of the content to be output.

%%%%%%%%%%%%%%%%%%%%%%%%%%%%%%%%%%%%%%%%
\DescribeMacro{\ifchilddocmanual}
The main file should be prepared as usual, see \secref{sec:include}.
However, the document body must make a distinction
between processing of an individual part and of the main document, e.g.:
%
\begin{center}
\begin{tabular}{l}
|\ifchilddocmanual|\\
|\input{\childdocname}|\\
|\||else|\\
\textit{document body with }|\input{|\textit{part}|}|\\
|\||fi|
\end{tabular}
\end{center}
%
The conditional |\ifchilddocmanual| is true whenever
a part to be included by |\input| is being compiled,
and the name of the part is stored in |\childdocname|.

%%%%%%%%%%%%%%%%%%%%%%%%%%%%%%%%%%%%%%%%
\DescribeMacro{\childdocby}
Each part to be included by |\input| should start with:
%
\begin{center}
\begin{tabular}{l}
|\input{childdoc.def}|\\
|\childdocby{|\textit{main}|}|\\
\end{tabular}
\end{center}
%
The directive |\childdocby| is similar to |\childdocof|
described in \secref{sec:include},
but the subsequent selection of content must be done manually.
To that end, both |\ifchilddoc| and |\ifchilddocmanual|
will be true upon processing of a part,
and the name of the part is stored in |\childdocname|.
Note that |\jobname| will be set to the filename of the current part
so that each part receives an individual |.aux| file
that does not interfere with the |.aux| file(s) of the main document.
This behaviour can be altered by the alternative form
|\childdocby[*]{|\textit{main}|}| (with a non-empty optional argument)
which uses the |.aux| file of the main document
by setting |\jobname| to \textit{main}.

%%%%%%%%%%%%%%%%%%%%%%%%%%%%%%%%%%%%%%%%%%%%%%%%%%%%%%%%%%%%%%%%%%%%%%%%%%%%%%%%
\subsection{Driver Development}
\label{sec:driver}

The \textsf{childdoc} mechanism can also be use for the development
of definition files such as \LaTeX{} styles or classes.
This case differs from the above setup with multiple parts
included by |\include| in that no |\includeonly| should be invoked.
This can be achieved by starting the include file
(before |\ProvidesPackage|) with:
%
\begin{center}
\begin{tabular}{l}
|\input{childdoc.def}|\\
|\childdocforward{|\textit{main}|}|\\
\end{tabular}
\end{center}
%
or alternatively with:
%
\begin{center}
\begin{tabular}{l}
|\input{childdoc.def}|\\
|\childdocby{|\textit{main}|}|\\
\end{tabular}
\end{center}
%
Both forms have slightly different effects as described above.
The main file is prepared as usual, see \secref{sec:include}.

%%%%%%%%%%%%%%%%%%%%%%%%%%%%%%%%%%%%%%%%%%%%%%%%%%%%%%%%%%%%%%%%%%%%%%%%%%%%%%%%
\subsection{Legacy Detection}
\label{sec:detection}

The directive |\childdocmain| in the main file can detect
whether the complete document or merely a child is to be compiled
even without using the directive |\childdocof|.
This method is deprecated because it is less robust
and there is no compelling reason to use it;
it is merely provided for backward compatibility
and it may be removed in future versions.

If the detection mechanism is to be used,
it is mandatory to correctly specify
the filename of the main file as the argument of |\childdocmain|:
%
\begin{center}
\begin{tabular}{l}
|\input{childdoc.def}|\\
|\childdocmain{|\textit{main}|}|\\
\end{tabular}
\end{center}
%
If |\jobname| does not match the argument \textit{main} of |\childdocmain|,
it is assumed that |\jobname| points to the child file to be compiled.
When using |\childdocmain| with the main file specified as argument,
it suffices to start a child file
with just |\input{|\textit{main}|}|
without loading of the package and using |\childdocof|.
If instead all processing is done
with the appropriate \textsf{childdoc} directives,
the argument of \textit{main} of |\childdocmain| can be empty.

An alternative version of the command line processing described
in \secref{sec:commandline} using the detection mechanism reads:
%
\begin{center}
|... -jobname "|\textit{target}|" "|[\textit{flags}]%
[|\def\jobname{|\textit{dest}|}|]|\input{|\textit{main}|}"|
\end{center}

%%%%%%%%%%%%%%%%%%%%%%%%%%%%%%%%%%%%%%%%%%%%%%%%%%%%%%%%%%%%%%%%%%%%%%%%%%%%%%%%
\subsection{Manual Code}
\label{sec:manual}

In case one cannot be certain whether the definitions file |childdoc.def|
is installed on the target \TeX{} distribution
and one prefers not to ship it,
it is conceivable to paste a few relevant commands into the sources.

To that end, drop all statements |\input{childdoc.def}|
and perform the replacements as outlined below.
Instead of |\childdocmain{|\textit{main}|}| add the following code
to the top of the main file:
%
\begin{center}
\begin{tabular}{l}
|\||ifdefined\childdocname\endinput\||fi\newif\ifchilddoc|\\
|\edef\childdocname{\scantokens\expandafter{\jobname\noexpand}}|\\
|\def\childdocmain{|\textit{main}|}\||ifx\childdocmain\childdocname\||else|\\
|\childdoctrue\includeonly{\childdocname}\let\jobname\childdocmain\||fi|\\
\end{tabular}
\end{center}
%
Instead of |\childdocof{|\textit{main}|}| just include the main file
at the top of each child file:
%
\begin{center}
|\input{|\textit{main}|}|
\end{center}
%
A simple redirection |\childdocforward{|\textit{dest}|}| is achieved by:
%
\begin{center}
|\def\jobname{|\textit{dest}|}\input{\jobname}|
\end{center}
%
The redirection with prefix
|\childdocforwardprefix[|\textit{prefix}|]{|\textit{dest}|}|
is accomplished by:
%
\begin{center}
\begin{tabular}{l}
|{\edef\jobname{\scantokens\expandafter{\jobname\noexpand}}|\\
|\def\redirectjob |\textit{prefix}|#1~~~{\gdef\jobname{|\textit{dest}|#1}}|\\
|\expandafter\redirectjob\jobname~~~}\input{\jobname}|
\end{tabular}
\end{center}

In an alternative approach,
child documents can be compiled by a specific command line
without additional code or specific definitions:
%
\begin{center}
|... -jobname "|\textit{target}|" "|[\textit{flags}]%
|\includeonly{|\textit{dest}|}\input{|\textit{main}|}"|
\end{center}
%

%%%%%%%%%%%%%%%%%%%%%%%%%%%%%%%%%%%%%%%%%%%%%%%%%%%%%%%%%%%%%%%%%%%%%%%%%%%%%%%%
%%%%%%%%%%%%%%%%%%%%%%%%%%%%%%%%%%%%%%%%%%%%%%%%%%%%%%%%%%%%%%%%%%%%%%%%%%%%%%%%
\section{Information}

%%%%%%%%%%%%%%%%%%%%%%%%%%%%%%%%%%%%%%%%%%%%%%%%%%%%%%%%%%%%%%%%%%%%%%%%%%%%%%%%
\subsection{Copyright}

Copyright \copyright{} 2017--2018 Niklas Beisert

This work may be distributed and/or modified under the
conditions of the \LaTeX{} Project Public License, either version 1.3
of this license or (at your option) any later version.
The latest version of this license is in
  \url{http://www.latex-project.org/lppl.txt}
and version 1.3 or later is part of all distributions of \LaTeX{}
version 2005/12/01 or later.

This work has the LPPL maintenance status `maintained'.

The Current Maintainer of this work is Niklas Beisert.

This work consists of the files |README.txt|, |childdoc.ins| and |childdoc.dtx|
as well as the derived files |childdoc.def|, |cdocsamp.tex|
with |cdocsch1.tex|, |cdocsch2.tex|, |cdocspt3.tex|, |cdocspt4.tex|,
|cdocsdrf.tex|, |cdocsfn1.tex|, |cdocsfn2.tex|
as well as |childdoc.pdf|.

%%%%%%%%%%%%%%%%%%%%%%%%%%%%%%%%%%%%%%%%%%%%%%%%%%%%%%%%%%%%%%%%%%%%%%%%%%%%%%%%
\subsection{Files and Installation}

The package consists of the files:
%
\begin{center}
\begin{tabular}{ll}
    |README.txt|   & readme file \\
    |childdoc.ins| & installation file \\
    |childdoc.dtx| & source file \\
    |childdoc.def| & definition file \\
    |cdocsamp.tex| & sample main file \\
    |cdocsch1.tex| & sample include file \\
    |cdocsch2.tex| & sample include file \\
    |cdocspt3.tex| & sample part file \\
    |cdocspt4.tex| & sample part file \\
    |cdocsdrf.tex| & sample redirection file \\
    |cdocsfn1.tex| & sample redirection file \\
    |cdocsfn2.tex| & sample redirection file \\
    |childdoc.pdf| & manual
\end{tabular}
\end{center}
%
The distribution consists of the files
|README.txt|, |childdoc.ins| and |childdoc.dtx|.
%
\begin{itemize}
\item
Run (pdf)\LaTeX{} on |childdoc.dtx|
to compile the manual |childdoc.pdf| (this file).
\item
Run \LaTeX{} on |childdoc.ins| to create the definitions file |childdoc.def|
and the sample |cdocsamp.tex| with include files
|cdocsch1.tex|, |cdocsch2.tex|, |cdocspt3.tex|, |cdocspt4.tex|,
|cdocsdrf.tex|, |cdocsfn1.tex|, |cdocsfn2.tex|.
Then copy the file |childdoc.def| to an appropriate directory of your \LaTeX{}
distribution, e.g.\ \textit{texmf-root}|/tex/latex/childdoc|.
\end{itemize}

%%%%%%%%%%%%%%%%%%%%%%%%%%%%%%%%%%%%%%%%%%%%%%%%%%%%%%%%%%%%%%%%%%%%%%%%%%%%%%%%
\subsection{Related CTAN Packages}

There are several other packages which offer a similar functionality:
%
\begin{itemize}
\item
The packages
\href{http://ctan.org/pkg/docmute}{\textsf{docmute}},
\href{http://ctan.org/pkg/includex}{\textsf{includex}} and
\href{http://ctan.org/pkg/standalone}{\textsf{standalone}}
provide commands to include only the document body of
a child file thus allowing both files to be compiled individually.
\item
The packages \href{http://ctan.org/pkg/subdocs}{\textsf{subdocs}}
and \href{http://ctan.org/pkg/subfiles}{\textsf{subfiles}}
provide structures in which the main and child documents can be
encapsulated and allowing them to be compiled individually.
The inclusion mechanism is different from the conventional |\include|.
\item
The package \href{http://ctan.org/pkg/combine}{\textsf{combine}}
is an elaborate solution to combine several documents into one.
\end{itemize}
%
See also the CTAN topic \href{http://ctan.org/topic/subdocs}{\textsf{subdocs}}
for further related packages.
The present package differs from the above solutions in that
a document structure constructed with the conventional |\include| mechanism
just needs two extra commands at the top of every file
such that all constituent files can be compiled individually.

%%%%%%%%%%%%%%%%%%%%%%%%%%%%%%%%%%%%%%%%%%%%%%%%%%%%%%%%%%%%%%%%%%%%%%%%%%%%%%%%
%\subsection{Feature Suggestions}
%
%The following is a list of features which may be useful for future
%versions of this package:
%%
%\begin{itemize}
%\item
%\ldots
%\end{itemize}

%%%%%%%%%%%%%%%%%%%%%%%%%%%%%%%%%%%%%%%%%%%%%%%%%%%%%%%%%%%%%%%%%%%%%%%%%%%%%%%%
\subsection{Revision History}

%%%%%%%%%%%%%%%%%%%%%%%%%%%%%%%%%%%%%%%%
\paragraph{v2.0:} 2018/12/30

\begin{itemize}
\item
immediate forward processing
\item
added |\childdocby| mechanism
\item
manual restructured
\end{itemize}

%%%%%%%%%%%%%%%%%%%%%%%%%%%%%%%%%%%%%%%%
\paragraph{v1.6:} 2018/01/17

\begin{itemize}
\item
application for development of include files
\item
corrections to manual
\end{itemize}

%%%%%%%%%%%%%%%%%%%%%%%%%%%%%%%%%%%%%%%%
\paragraph{v1.5:} 2017/05/21

\begin{itemize}
\item
more complete structuring introduced
\item
|\childdocof| introduced
\item
|\childdoc| renamed to |\childdocmain|
\item
|\childredirect| renamed to |\childdocforward| and |\childdocforwardprefix|
and functionality expanded
\end{itemize}

%%%%%%%%%%%%%%%%%%%%%%%%%%%%%%%%%%%%%%%%
\paragraph{v1.0:} 2017/04/27

\begin{itemize}
\item
manual and install package
\item
first version published on CTAN
\end{itemize}

%%%%%%%%%%%%%%%%%%%%%%%%%%%%%%%%%%%%%%%%
\paragraph{v0.6:} 2017/04/26

\begin{itemize}
\item
redirection mechanism added
\end{itemize}

%%%%%%%%%%%%%%%%%%%%%%%%%%%%%%%%%%%%%%%%
\paragraph{v0.5:} 2017/04/26

\begin{itemize}
\item
functionality in definition file
\end{itemize}


%%%%%%%%%%%%%%%%%%%%%%%%%%%%%%%%%%%%%%%%%%%%%%%%%%%%%%%%%%%%%%%%%%%%%%%%%%%%%%%%
%%%%%%%%%%%%%%%%%%%%%%%%%%%%%%%%%%%%%%%%%%%%%%%%%%%%%%%%%%%%%%%%%%%%%%%%%%%%%%%%
%%%%%%%%%%%%%%%%%%%%%%%%%%%%%%%%%%%%%%%%%%%%%%%%%%%%%%%%%%%%%%%%%%%%%%%%%%%%%%%%
\appendix

\settowidth\MacroIndent{\rmfamily\scriptsize 000\ }

 \DocInput{childdoc.dtx}

\end{document}
%</driver>
% \fi
%
% %%%%%%%%%%%%%%%%%%%%%%%%%%%%%%%%%%%%%%%%%%%%%%%%%%%%%%%%%%%%%%%%%%%%%%%%%%%%%%
% %%%%%%%%%%%%%%%%%%%%%%%%%%%%%%%%%%%%%%%%%%%%%%%%%%%%%%%%%%%%%%%%%%%%%%%%%%%%%%
% \section{Sample}
%\iffalse
%<*samplemain>
%\fi
%
% The following presents a sample document
% with two chapters, two parts, a title page,
% a compile flag as well as three forwarding files to set the flag.
% It consists of eight |.tex| files:
% \begin{center}
% \begin{tabular}{ll}
% |cdocsamp.tex|&main file\\
% |cdocsch1.tex|&include file for chapter 1\\
% |cdocsch2.tex|&include file for chapter 2\\
% |cdocspt3.tex|&include file for part 3\\
% |cdocspt4.tex|&include file for part 4\\
% |cdocsdrf.tex|&forwarding file for main file in draft mode\\
% |cdocsfi1.tex|&forwarding file for final version of chapter 1\\
% |cdocsfi2.tex|&forwarding file for final version of chapter 2\\
% \end{tabular}
% \end{center}
% Each of the eight files can be compiled directly by the \LaTeX{} compiler.
%
% %%%%%%%%%%%%%%%%%%%%%%%%%%%%%%%%%%%%%%
% \paragraph{Main File.}
%
% The main file is called |cdocsamp.tex|.
%
% Load the \textsf{childdoc} definitions and
% declare the filename for the main document:
%    \begin{macrocode}
\input{childdoc.def}
\childdocmain{}
%    \end{macrocode}

% Optional override for |\version| flag:
%    \begin{macrocode}
%%\ifchilddoc\else\providecommand{\version}{draft}\fi
%    \end{macrocode}

% Define the default values for the |\version| flag
% (|final| for the main file and |draft| for childs):
%    \begin{macrocode}
\ifchilddoc
\providecommand{\version}{draft}
\else
\providecommand{\version}{final}
\fi
%    \end{macrocode}

% Load the standard document class:
%    \begin{macrocode}
\documentclass[12pt]{article}
%    \end{macrocode}

% Start the document body:
%    \begin{macrocode}
\begin{document}
%    \end{macrocode}

% Declare a title page.
% Print title, part of document being processed and version flag:
%    \begin{macrocode}
\addtocounter{page}{-1}
\begin{center}
{\LARGE\bfseries{}childdoc example\par}
\vspace{1cm}
\ifchilddoc
\ifchilddocmanual part\else chapter\fi:
`\childdocname' of `\childdocjob'\par
\else
main document: `\childdocjob'\par
\fi
version: \version\par
\end{center}
\newpage
%    \end{macrocode}

% Manually include selected file,
% otherwise process as usual:
%    \begin{macrocode}
\ifchilddocmanual
\section*{part `\childdocname'}
\input{\childdocname}
\else
%    \end{macrocode}

% Include the two chapters:
%    \begin{macrocode}
\include{cdocsch1}
\include{cdocsch2}
%    \end{macrocode}

% Include the two parts unless only chapters should be displayed:
%    \begin{macrocode}
\ifchilddoc\else
\section{part three}
\input{cdocspt3}
\section{part four}
\input{cdocspt4}
\fi
%    \end{macrocode}

% Process as usual until here:
%    \begin{macrocode}
\fi
%    \end{macrocode}

% End of document body:
%    \begin{macrocode}
\end{document}
%    \end{macrocode}
%\iffalse
%</samplemain>
%\fi
%
% %%%%%%%%%%%%%%%%%%%%%%%%%%%%%%%%%%%%%%
% \paragraph{Chapter Include Files.}
%
% The include files are called |cdocsch1.tex| and |cdocsch2.tex|.
%
%\iffalse
%<*samplechap1|samplechap2>
%\fi

% Optional override for |\version| flag:
%    \begin{macrocode}
%%\providecommand{\version}{final}
%    \end{macrocode}

% Include the main document:
%    \begin{macrocode}
\input{childdoc.def}
\childdocof{cdocsamp}
%    \end{macrocode}

%\iffalse
%</samplechap1|samplechap2>
%\fi
%
%\iffalse
%<*samplechap1>
%\fi
% Some text for chapter 1:
%    \begin{macrocode}
\section{one}
some text in chapter one
%    \end{macrocode}

%\iffalse
%</samplechap1>
%\fi
% Some text for chapter 2:
%\iffalse
%<*samplechap2>
%\fi
%    \begin{macrocode}
\section{two}
more text in chapter two
%    \end{macrocode}

%\iffalse
%</samplechap2>
%\fi
%
% %%%%%%%%%%%%%%%%%%%%%%%%%%%%%%%%%%%%%%
% \paragraph{Part Include Files.}
%
% The include files are called |cdocspt3.tex| and |cdocspt4.tex|.
%
%\iffalse
%<*samplepart3|samplepart4>
%\fi

% Optional override for |\version| flag:
%    \begin{macrocode}
%%\providecommand{\version}{final}
%    \end{macrocode}

% Include the main document:
%    \begin{macrocode}
\input{childdoc.def}
\childdocby{cdocsamp}
%    \end{macrocode}

%\iffalse
%</samplepart3|samplepart4>
%\fi
%
%\iffalse
%<*samplepart3>
%\fi
% Some text for part 3:
%    \begin{macrocode}
some text in part three
%    \end{macrocode}

%\iffalse
%</samplepart3>
%\fi
% Some text for part 4:
%\iffalse
%<*samplepart4>
%\fi
%    \begin{macrocode}
more text in part four
%    \end{macrocode}

%\iffalse
%</samplepart4>
%\fi
%
% %%%%%%%%%%%%%%%%%%%%%%%%%%%%%%%%%%%%%%
% \paragraph{Forwarding for a Complete Draft.}
%
% The following forwarding file |cdocsdrf.tex|
% compiles the main document in draft mode:
%\iffalse
%<*sampledraft>
%\fi
%    \begin{macrocode}
\def\version{draft}
\input{childdoc.def}
\childdocforward{cdocsamp}
%    \end{macrocode}

%\iffalse
%</sampledraft>
%\fi
%
% %%%%%%%%%%%%%%%%%%%%%%%%%%%%%%%%%%%%%%
% \paragraph{Forwarding for Final Version of the Chapters.}
%
% The following forwarding files |cdocsfn1.tex| and |cdocsfn2.tex|
% (with identical content)
% compile the final versions of the child documents
% |cdocsch1.tex| and |cdocsch2.tex|, respectively:
%\iffalse
%<*samplefinal>
%\fi
%    \begin{macrocode}
\def\version{final}
\input{childdoc.def}
\childdocforwardprefix[cdocsamp]{cdocsfn}{cdocsch}
%    \end{macrocode}

%\iffalse
%</samplefinal>
%\fi
%
% %%%%%%%%%%%%%%%%%%%%%%%%%%%%%%%%%%%%%%
% \paragraph{Command Line Processing.}
%
% The following three command lines generate the output files
% |cdocscld|, |cdocscl1| and |cdocscl2|
% which should be identical to
% |cdocsdrf|, |cdocsch1| and |cdocsfn2|, respectively:
% \begin{center}
% \begin{tabular}{l}
% |latex -jobname cdocscld \|\\
% |  "\def\version{draft}\input{childdoc.def}\childdocforward{cdocsamp}"|\\
% |latex -jobname cdocscl1 \|\\
% |  "\input{childdoc.def}\childdocforward[cdocsamp]{cdocsch1}"|\\
% |latex -jobname cdocscl2 \|\\
% |  "\def\version{final}\input{childdoc.def}\childdocforward{cdocsch2}"|
% \end{tabular}
% \end{center}
% Note that the trailing backslash on each first line
% merely continues the input to the second line
% (for convenient cut ant paste).
% Furthermore, the command |latex| can be replaced by any
% of its alternative versions such as |pdflatex|.
%
% %%%%%%%%%%%%%%%%%%%%%%%%%%%%%%%%%%%%%%%%%%%%%%%%%%%%%%%%%%%%%%%%%%%%%%%%%%%%%%
% %%%%%%%%%%%%%%%%%%%%%%%%%%%%%%%%%%%%%%%%%%%%%%%%%%%%%%%%%%%%%%%%%%%%%%%%%%%%%%
% \section{Implementation}
%\iffalse
%<*package>
%\fi
%
% This section describes the definitions file |childdoc.def|.

% The definitions cannot be loaded using |\usepackage| or |\RequirePackage|
% which has a mechanism to prevent loading a style file more than once.
% When loading the definitions by means of |\input|
% multiple instances have to be prevented manually:
%\iffalse
%This code needs to be before the `\ProvidesFile' directive
%which is defined at the beginning of this file.
%Therefore it is also placed there and commented out here.
%</package>
%<*discard>
%\fi
%    \begin{macrocode}
\ifdefined\childdocmain\endinput\fi
%    \end{macrocode}
%\iffalse
%</discard>
%<*package>
%\fi
%
% \macro{\ifchilddoc}
% \macro{\ifchilddocmanual}
% The conditional |\ifchilddoc| tells whether a
% child (true) or main (false) document is being compiled.
% The conditional |\ifchilddocmanual| tells whether
% the |\includeonly| mechanism is used (false) or
% the selection of child files must be performed manually (true).
% The definitions initialise to false:
%    \begin{macrocode}
\newif\ifchilddoc
\newif\ifchilddocmanual
%    \end{macrocode}

% \macro{\childdocname}
% \macro{\childdocjob}
% The macro |\childdocname| stores the name of the main document
% to be compiled. The macro |\childdocjob| stores the name of
% the document on which the \LaTeX{} compiler was originally invoked.
% The content of |\jobname| cannot be compared
% to filenames specified in the source due to different catcodes.
% The following code rescans |\jobname|, stores the result
% in |\childdocname| and saves a copy in |\childdocjob|:
%    \begin{macrocode}
\edef\childdocname{\scantokens\expandafter{\jobname\noexpand}}
\let\childdocjob\childdocname
%    \end{macrocode}

% \macro{\childdocdisable}
% The macro |\childdocdisable| prevents the main file
% from being processed more than once.
% At this stage, the main document command |\childdocmain|
% is assumed to be called once again where it should do nothing.
% Any subsequent call to it should prevent
% a secondary processing of the main document
% It overwrites the forwarding commands
% |\childdocof| and |\childdocforward|
% with empty macros to prevent further inclusions of the main document:
%    \begin{macrocode}
\newcommand{\childdocdisable}
{
  \renewcommand{\childdocmain}[1]{\renewcommand{\childdocmain}[1]{\endinput}}
  \renewcommand{\childdocof}[1]{}
  \renewcommand{\childdocby}[2][]{}
  \renewcommand{\childdocforward}[2][]{}
  \renewcommand{\childdocdisable}{}
}
%    \end{macrocode}

% \macro{\childdocmain}
% The macro |\childdocmain| is to be called at the top of the main file
% with nothing or the main filename (without extension) as argument.
% First, it breaks loops.
% If the argument is not empty and does not match |\childdocname|
% (which is set by the first inclusion of |childdoc.def|),
% |\ifchilddoc| is set to true, |\includeonly| is applied to the child file
% and |\jobname| is set to the main file
% (for proper handling of |.aux| files):
%    \begin{macrocode}
\newcommand{\childdocmain}[1]
{
  \childdocdisable\childdocmain{}
  \if?#1?\else
    \begingroup
      \def\childdoctmp{#1}
      \ifx\childdoctmp\childdocname
        \def\childdoctmp{}
      \else
        \def\childdoctmp
        {
          \childdoctrue
          \includeonly{\childdocname}
          \def\childdocjob{#1}
          \def\jobname{#1}
        }
      \fi
      \expandafter
    \endgroup
    \childdoctmp
  \fi
}
%    \end{macrocode}

% \macro{\childdocof}
% The command |\childdocof| redirects
% compilation to the main file |#1|.
%    \begin{macrocode}
\newcommand{\childdocof}[1]
{
  \childdocdisable
  \childdoctrue
  \includeonly{\childdocname}
  \def\jobname{#1}
  \def\childdocjob{#1}
  \input{#1}
}
%    \end{macrocode}

% \macro{\childdocby}
% The command |\childdocby| ....
%    \begin{macrocode}
\newcommand{\childdocby}[2][]
{
  \childdocdisable
  \childdoctrue
  \childdocmanualtrue
  \if?#1?\else
    \def\jobname{#2}
  \fi
  \def\childdocjob{#2}
  \input{#2}
  \endinput
}
%    \end{macrocode}

% \macro{\childdocforward}
% The command |\childdocforward| redirects
% compilation to the main file or
% (if the optional argument is given) a child file.
% Parameters are set as if the main file
% or a child file starting with |\childdocof| was compiled.
% Then compilation is handed over to the main file:
%    \begin{macrocode}
\newcommand{\childdocforward}[2][]
{
  \begingroup
    \if?#1?
      \def\childdoctmp
      {
        \def\childdocname{#2}
        \def\childdocjob{#2}
        \def\jobname{#2}
        \input{#2}
        \endinput
      }
    \else
      \def\childdoctmp
      {
        \childdocdisable
        \def\childdocname{#2}
        \childdoctrue
        \includeonly{#2}
        \def\childdocjob{#1}
        \def\jobname{#1}
        \input{#1}
        \endinput
      }
    \fi
    \expandafter
  \endgroup
  \childdoctmp
}
%    \end{macrocode}

% \macro{\childdocforwardprefix}
% The command |\childdocforwardprefix| redirects
% compilation to the main or a child file by means of a pattern.
% The prefix |#1| in the current filename is replaced by |#2|
% and the suffix of the current filename is kept
% (it is assumed that the filename does not contain the substring `|~~~|'
% which is used as a delimiter).
% Compilation is handed over to the new file by |\childdocforward|:
%    \begin{macrocode}
\newcommand{\childdocforwardprefix}[3][]
{
  \begingroup
    \def\childdocextract #2##1~~~{\def\childdoctmp{\childdocforward[#1]{#3##1}}}
    \expandafter\childdocextract\childdocname~~~
    \expandafter
  \endgroup
  \childdoctmp
}
%    \end{macrocode}

% \macro{\childdoc}
% The deprecated macro |\childdoc| is a legacy version of |\childdocmain|:
%    \begin{macrocode}
\newcommand{\childdoc}{\childdocmain}
%    \end{macrocode}

% \macro{\childdocredirect}
% The deprecated macro |\childdocredirect| is a legacy version
% of |\childdocforward| and |\childdocforwardprefix|:
%    \begin{macrocode}
\newcommand{\childdocredirect}[2][]
{
  \begingroup
    \if?#1?
      \def\childdoctmp{\childdocforward{#2}}
    \else
      \def\childdoctmp{\childdocforwardprefix{#1}{#2}}
    \fi
    \expandafter
  \endgroup
  \childdoctmp
}
%    \end{macrocode}

%\iffalse
%</package>
%\fi
%
\endinput
|\\
|\childdocforward{|\textit{main}|}|\\
\end{tabular}
\end{center}
%
or alternatively with:
%
\begin{center}
\begin{tabular}{l}
|% \iffalse
%
% childdoc.dtx Copyright (C) 2017-2018 Niklas Beisert
%
% This work may be distributed and/or modified under the
% conditions of the LaTeX Project Public License, either version 1.3
% of this license or (at your option) any later version.
% The latest version of this license is in
%   http://www.latex-project.org/lppl.txt
% and version 1.3 or later is part of all distributions of LaTeX
% version 2005/12/01 or later.
%
% This work has the LPPL maintenance status `maintained'.
%
% The Current Maintainer of this work is Niklas Beisert.
%
% This work consists of the files childdoc.dtx and childdoc.ins
% and the derived files childdoc.def and cdocsamp.tex with
% cdocsch1.tex, cdocsch2.tex, cdocsdrf.tex, cdocsfn1.tex, cdocsfn2.tex.
%
%<package>\ifdefined\childdocmain\endinput\fi
%<package>\ProvidesFile{childdoc.def}[2018/12/30 v2.0 child document driver]
%<samplemain>\ProvidesFile{cdocsamp.tex}[2018/12/30 v2.0 sample for childdoc]
%<*driver>
%\ProvidesFile{childdoc.drv}[2018/12/30 v2.0 childdoc reference manual file]
\PassOptionsToClass{10pt,a4paper}{article}
\documentclass{ltxdoc}

\usepackage[margin=35mm]{geometry}
\usepackage{hyperref}
\usepackage{hyperxmp}
\usepackage[usenames]{color}

\hypersetup{colorlinks=true}
\hypersetup{pdfstartview=FitH}
\hypersetup{pdfpagemode=UseNone}
\hypersetup{pdfsource={}}
\hypersetup{pdflang={en-UK}}
\hypersetup{pdfcopyright={Copyright 2017-2018 Niklas Beisert.
  This work may be distributed and/or modified under the
  conditions of the LaTeX Project Public License, either version 1.3
  of this license or (at your option) any later version.}}
\hypersetup{pdflicenseurl={http://www.latex-project.org/lppl.txt}}
\hypersetup{pdfcontactaddress={ETH Zurich, ITP, HIT K,
  Wolfgang-Pauli-Strasse 27}}
\hypersetup{pdfcontactpostcode={8093}}
\hypersetup{pdfcontactcity={Zurich}}
\hypersetup{pdfcontactcountry={Switzerland}}
\hypersetup{pdfcontactemail={nbeisert@itp.phys.ethz.ch}}
\hypersetup{pdfcontacturl={http://people.phys.ethz.ch/\xmptilde nbeisert/}}

\newcommand{\secref}[1]{\hyperref[#1]{section \ref*{#1}}}

\parskip1ex
\parindent0pt
\let\olditemize\itemize
\def\itemize{\olditemize\parskip0pt}

\begin{document}

\title{The \textsf{childdoc} Package}
\hypersetup{pdftitle={The childdoc Package}}
\author{Niklas Beisert\\[2ex]
  Institut f\"ur Theoretische Physik\\
  Eidgen\"ossische Technische Hochschule Z\"urich\\
  Wolfgang-Pauli-Strasse 27, 8093 Z\"urich, Switzerland\\[1ex]
  \href{mailto:nbeisert@itp.phys.ethz.ch}
  {\texttt{nbeisert@itp.phys.ethz.ch}}}
\hypersetup{pdfauthor={Niklas Beisert}}
\hypersetup{pdfsubject={Manual for the LaTeX2e Package childdoc}}
\date{30 December 2018, \textsf{v2.0}}
\maketitle

\begin{abstract}\noindent
\textsf{childdoc} is a \LaTeXe{} package
that enables the direct compilation
of document sections included by |\include|
to individual files.
\end{abstract}

\begingroup
\parskip0ex
\tableofcontents
\endgroup

%%%%%%%%%%%%%%%%%%%%%%%%%%%%%%%%%%%%%%%%%%%%%%%%%%%%%%%%%%%%%%%%%%%%%%%%%%%%%%%%
%%%%%%%%%%%%%%%%%%%%%%%%%%%%%%%%%%%%%%%%%%%%%%%%%%%%%%%%%%%%%%%%%%%%%%%%%%%%%%%%
\section{Introduction}

\LaTeX{} provides a mechanism to structure a large document (such as a book)
into a main file and several child files (containing the chapters)
using the |\include| command.
This mechanism is beneficial for documents
which span hundreds of pages in order to
make the source file(s) more manageable.
Moreover, compilation can be restricted to
selected child files by means of the |\includeonly| command.
The latter feature can be used to reduce the compilation time while editing
(this was significantly more useful in the earlier days of \LaTeX{})
or to generate a smaller document which is easier to navigate.
Another application of |\includeonly| is to generate
documents consisting of selected parts of the complete document.

However, there are a few drawbacks of the plain |\include| mechanism:
\begin{itemize}
\item
The child files cannot be compiled on their own,
they can only be compiled via the main file.
A naive editing environment
(such as a text editor with an option
to have the current file processed by \LaTeX)
may require one to switch to the main file before compiling;
attempting to compile the child file produces errors.
\item
The main file must be modified (each time)
to adjust the |\includeonly| command
to the present needs. This easily leaves the main file in a messy state.
\item
The generated document will always carry the filename
of the main document. This is inconvenient if
several child files are to be compiled and
to be kept for distribution.
\end{itemize}

The present package provides a simple interface
to make child files individually compilable by \LaTeX{}.
Compiling a child file then has the same effect as compiling
the main file with an |\includeonly| command
to select the appropriate child.
Moreover the generated document will carry the name of the child
rather than the main file.
This resolves all three above issues.

This feature is meant to make the editing of books,
thesis documents and lecture notes somewhat more convenient.
However, the package can also be used efficiently for
composing a series of documents (such as exercise sheets)
which are typically distributed individually.
It then assists the author in generating the individual documents
(potentially in different versions)
as well as a document containing the collected series.
Another application is in developing style files
or other kinds of included material
where compilation of the style file could redirect
to a sample or test file.

%%%%%%%%%%%%%%%%%%%%%%%%%%%%%%%%%%%%%%%%%%%%%%%%%%%%%%%%%%%%%%%%%%%%%%%%%%%%%%%%
%%%%%%%%%%%%%%%%%%%%%%%%%%%%%%%%%%%%%%%%%%%%%%%%%%%%%%%%%%%%%%%%%%%%%%%%%%%%%%%%
\section{Usage}

First of all, the package \textsf{childdoc} is \emph{not} a standard
\LaTeXe{} |.sty| style file! Therefore it needs to be invoked in
a non-standard way.

%%%%%%%%%%%%%%%%%%%%%%%%%%%%%%%%%%%%%%%%%%%%%%%%%%%%%%%%%%%%%%%%%%%%%%%%%%%%%%%%
\subsection{Included Files}
\label{sec:include}

%%%%%%%%%%%%%%%%%%%%%%%%%%%%%%%%%%%%%%%%
\DescribeMacro{\childdocmain}
To use the package, add the commands
\begin{center}
\begin{tabular}{l}
|\input{childdoc.def}|\\
|\childdocmain{}|\\
\end{tabular}
\end{center}
at the very top of the main \LaTeX{} file,
in particular \emph{before} the |\documentclass| statement!
The argument of |\childdocmain| should be left empty
(but it must be present).

%%%%%%%%%%%%%%%%%%%%%%%%%%%%%%%%%%%%%%%%
\DescribeMacro{\childdocof}
Furthermore, add the commands
\begin{center}
\begin{tabular}{l}
|\input{childdoc.def}|\\
|\childdocof{|\textit{main}|}|\\
\end{tabular}
\end{center}
at the top of every child file \textit{child}
which is included by |\include{|\textit{child}|}|
from within the main file
(or at least for those files to be compiled individually).
The argument \textit{main} must be the filename of the main file.

There are a couple of
considerations in setting up the main and child documents:

%%%%%%%%%%%%%%%%%%%%%%%%%%%%%%%%%%%%%%%%
\paragraph{Restrictions.}

Please note the following restrictions:
\begin{itemize}
\item
|\childdocmain| must be called with one argument \textit{main}
to ensure compatibility with earlier version of the package.
It must either be empty (|\childdocmain{}|)
or precisely match the filename of the main file in which it is specified.
See \secref{sec:detection} for further information.
\item
The filename \textit{main} must be specified without the |.tex| extension.
\item
The filename \textit{main} is case sensitive
(even in case-insensitive file systems)
due to internal string comparison.
\item
The argument \textit{main} should be fully expanded, it cannot be a macro.
\item
Subdirectories and special characters should be avoided in filenames.
\item
The command |\childdocmain{|\textit{main}|}| must be followed by a whitespace.
It should not be followed immediately by another command
or by a comment mark `|%|'.
This is because the \TeX{} parser reads the token immediately following
the argument of |\childdocmain| and puts it
at the beginning of every child section;
however, a white\-space is ignored.
\end{itemize}

%%%%%%%%%%%%%%%%%%%%%%%%%%%%%%%%%%%%%%%%
\paragraph{Content of Main File.}

It is advisable to place all content in the child files included by |\include|.
Any output contained in the main file will appear in all child documents
unless suppressed manually;
it cannot be suppressed automatically by the |\includeonly| directive
and thus should normally be avoided.
A method to include some content in the main file
by means of conditional processing is described in \secref{sec:conditional}.

%%%%%%%%%%%%%%%%%%%%%%%%%%%%%%%%%%%%%%%%
\paragraph{Page Numbering.}

When only a part of the document is compiled,
the appropriate numbering of pages
(as well as other status parameters)
is determined from the |.aux| files.
The latter contain information from previous passes.
However this information needs to propagate through
all intermediate child documents.
Therefore the page numbering in child documents may well
be inconsistent until the complete document is compiled at least once.

A useful (if unconventional) way to always ensure a consistent
page numbering is to restart the numbering in each child document
and denote the pages by `\textit{child}|.|\textit{page}'
where \textit{child} represents the chapter/section number of the child file.
This can be achieved by the command
|\numberwithin{page}{|\textit{child}|}|
of the \textsf{amsmath} package
where \textit{child} can be |chapter| or |section|
depending on the chosen structuring.
Alternatively, one can modify the macro |\thepage| appropriately
and reset the counter |page| at the start of each child file.

%%%%%%%%%%%%%%%%%%%%%%%%%%%%%%%%%%%%%%%%%%%%%%%%%%%%%%%%%%%%%%%%%%%%%%%%%%%%%%%%
\subsection{Conditional Processing}
\label{sec:conditional}

The package provides a mechanism to compile different versions
of a document. To customise the versions further some conditional processing
can come in handy to distinguish which version is being compiled.
The package provides two macros to describe the compilation context:

%%%%%%%%%%%%%%%%%%%%%%%%%%%%%%%%%%%%%%%%
\DescribeMacro{\ifchilddoc}
The conditional |\ifchilddoc| distinguishes between the compilation of
child documents and the main document:
%
\begin{center}
|\ifchilddoc |\textit{child-code}| |[|\||else |\textit{main-code}]| \||fi|
\end{center}

%%%%%%%%%%%%%%%%%%%%%%%%%%%%%%%%%%%%%%%%
\DescribeMacro{\childdocname}
\DescribeMacro{\childdocjob}
The macro |\childdocname| contains the filename (without extension)
of the main or child file being processed.
Note that |\childdocjob| will always contain the name of the main file.

%%%%%%%%%%%%%%%%%%%%%%%%%%%%%%%%%%%%%%%%
\paragraph{Title Page.}

Conditional processing can be used to include a title or banner page
in the main document when proper precautions are taken.
Importantly, the code in the main file should ensure that the page counter
(as well as other status parameters which are stored in the |.aux| files)
takes the same value after the conditional processing.
Otherwise the page numbers may take divergent values
depending on which part is compiled.

For example, a title page could be declared by:
%
\begin{center}
\begin{tabular}{l}
|\ifchilddoc\||else|\\
|\addtocounter{page}{-1}|\\
\textit{code for title page}\\
|\newpage|\\
|\||fi|
\end{tabular}
\end{center}
%
A banner page for the child documents can be generated by:
%
\begin{center}
\begin{tabular}{l}
|\ifchilddoc|\\
|\addtocounter{page}{-1}|\\
\textit{code for banner page}\\
|\newpage|\\
|\||fi|
\end{tabular}
\end{center}
%
Here one could write a message such as:
\begin{center}
|This is the part \childdocname{} of \childdocjob{}.|
\end{center}

%%%%%%%%%%%%%%%%%%%%%%%%%%%%%%%%%%%%%%%%%%%%%%%%%%%%%%%%%%%%%%%%%%%%%%%%%%%%%%%%
\subsection{Flags}
\label{sec:flags}

The package makes it easy to generate different versions
of the main or child documents.
To this end compilation flags can be defined
and assigned different default values.
They will be particularly useful in conjunction
with the forwarding mechanism described in \secref{sec:forward}.

For example, it may be useful to have a flag |\version|
which can be set to |draft| or |final|.
The document source will contain some conditional code
depending on the value of |\version|.
Suppose further, the flag should default to |final| for the main file
and to |draft| for child files
which is a natural assignment for editing the document.
This is achieved by placing the following code
in the preamble of the main document
(below the |\childdocmain| directive):
%
\begin{center}
\begin{tabular}{l}
|\ifchilddoc|\\
|\providecommand{\version}{draft}|\\
|\||else|\\
|\providecommand{\version}{final}|\\
|\||fi|
\end{tabular}
\end{center}
%
The definition by |\providecommand| makes sure
that previous definitions are not overwritten.
Further statements |\providecommand{\version}{...}|
can thus be added before the above code to override it.

For the main file, one might add a line
(between |\childdocmain| and the above block)
%
\begin{center}
|%\ifchilddoc\||else\providecommand{\version}{draft}\||fi|
\end{center}
%
which can be uncommented to produce a draft version.
Likewise one can add a line to the very top of a child file
(above the |\childdocof{|\textit{main}|}| directive)
%
\begin{center}
|%\providecommand{\version}{final}|
\end{center}
%
which can be uncommented to produce the final version of this child document.

%%%%%%%%%%%%%%%%%%%%%%%%%%%%%%%%%%%%%%%%%%%%%%%%%%%%%%%%%%%%%%%%%%%%%%%%%%%%%%%%
\subsection{Forwarding}
\label{sec:forward}

Different versions of the main or child documents
using compilation flags as described in \secref{sec:flags}
can be (permanently) stored in different files
for convenient compilation, viewing and distribution.
To this end, the package defines a command
to pass on compilation to a different file:

%%%%%%%%%%%%%%%%%%%%%%%%%%%%%%%%%%%%%%%%
\DescribeMacro{\childdocforward}
The command |\childdocforward| redirects processing to
another source file:
%
\begin{center}
\begin{tabular}{l}
|\input{childdoc.def}|\\
|\childdocforward[|\textit{main}|]{|\textit{dest}|}|\\
\end{tabular}
\end{center}
%
The argument \textit{dest} is the destination file
(without extension).
It should be the main file or one of the child files.
Note that further \textsf{childdoc} directives
such as |\childdocof| and |\childdocforward|
in the indicated file will be processed in this form.
The optional argument \textit{main}
passes on directly to the main file \textit{main}
while pretending to compile the child \textit{dest}.
This form behaves as if \textit{dest}
issues |\childdocof{|\textit{main}|}| right away,
and no further \textsf{childdoc} directives will be processed.

%%%%%%%%%%%%%%%%%%%%%%%%%%%%%%%%%%%%%%%%
\DescribeMacro{\...prefix}
In the alternative form |\childdocforwardprefix|,
%
\begin{center}
\begin{tabular}{l}
|\input{childdoc.def}|\\
|\childdocforwardprefix[|\textit{main}|]{|\textit{prefix}|}{|\textit{dest}|}|
\end{tabular}
\end{center}
%
the destination file is determined by a pattern
depending on the current file:
To make this work, the current file must be called
`{\textit{prefix}\hspace{0.2em}\textit{suffix}}'
with \textit{prefix} matching precisely the argument.
Processing is then passed on to the file
`{\textit{dest}\hspace{0.2em}\textit{suffix}}'.
Surely, the same effect is achieved by
directly specifying the
argument `{\textit{dest}\hspace{0.2em}\textit{suffix}}'
in the first form.
However, that requires to set up a different file
for each child. With the alternative form of the command
all these files can have exactly the same content
which simplifies setting them up and maintaining them.

For example, the following file |draft.tex|
with a compilation flag |\version| as described in \secref{sec:flags}
compiles the main document as a draft:
%
\begin{center}
\begin{tabular}{l}
|\def\version{draft}|\\
|\input{childdoc.def}|\\
|\childdocforward{|\textit{main}|}|
\end{tabular}
\end{center}
%
Likewise, the following files |final|\textit{nn}|.tex|
compile the final version of the child document
|child|\textit{nn}|.tex|:
%
\begin{center}
\begin{tabular}{l}
|\def\version{final}|\\
|\input{childdoc.def}|\\
|\childdocforwardprefix{final}{child}|
\end{tabular}
\end{center}
%

Note that when several versions of a main file and/or of each child file
are to be generated, it may be convenient to set up a |Makefile| or
shell script to automatise the process.

%%%%%%%%%%%%%%%%%%%%%%%%%%%%%%%%%%%%%%%%%%%%%%%%%%%%%%%%%%%%%%%%%%%%%%%%%%%%%%%%
\subsection{Command Line Processing}
\label{sec:commandline}

The effect of redirection files can also be achieved by invoking
the \LaTeX{} compiler with a more elaborate command line.
Most conveniently this should be done as part
of a shell script or a |Makefile|.

When using \textsf{childdoc} in the main file, the following
command lines effectively perform a redirection
(note that depending on the shell being used,
backslashes may have to be doubled: `|\|' $\to$ `|\\|'):
%
\begin{center}
|... -jobname "|\textit{target}|" |\\|"|[\textit{flags}]%
|\input{childdoc.def}\childdocforward[|\textit{main}|]{|\textit{dest}|}"|
\end{center}
%
Here \textit{target} is the name of the output file,
\textit{main} is the name of the main file
and \textit{dest} is the name of the main or child file to be processed
(all filenames without extensions).
The optional argument \textit{main} can be omitted
if \textit{main} matches \textit{dest}.
Optionally, compilation \textit{flags} can be defined via |\def| commands.
This command line makes the \TeX{} engine believe
it is compiling the file \textit{target}
whose content is specified as the latter parameter.
The provided code then forwards the processing to
\textit{main} or \textit{dest} as described in \secref{sec:forward}.

%%%%%%%%%%%%%%%%%%%%%%%%%%%%%%%%%%%%%%%%%%%%%%%%%%%%%%%%%%%%%%%%%%%%%%%%%%%%%%%%
\subsection{Include by Input}
\label{sec:input}

Including child documents by |\include| has some restrictions by design.
Most notably, the content of a child document always occupies
its own set of pages; pages cannot be shared between child documents.
Usually, this behaviour makes perfect sense
because each child document contain an essential part of the document.
However, in some situations it may be desirable to compose
a document from a collection of parts
without having mandatory page breaks between then.
For this case, the package
provides a mechanism to include parts
by |\input| which can also be processed individually.
However, by construction this mechanism
requires manual handling of the content to be output.

%%%%%%%%%%%%%%%%%%%%%%%%%%%%%%%%%%%%%%%%
\DescribeMacro{\ifchilddocmanual}
The main file should be prepared as usual, see \secref{sec:include}.
However, the document body must make a distinction
between processing of an individual part and of the main document, e.g.:
%
\begin{center}
\begin{tabular}{l}
|\ifchilddocmanual|\\
|\input{\childdocname}|\\
|\||else|\\
\textit{document body with }|\input{|\textit{part}|}|\\
|\||fi|
\end{tabular}
\end{center}
%
The conditional |\ifchilddocmanual| is true whenever
a part to be included by |\input| is being compiled,
and the name of the part is stored in |\childdocname|.

%%%%%%%%%%%%%%%%%%%%%%%%%%%%%%%%%%%%%%%%
\DescribeMacro{\childdocby}
Each part to be included by |\input| should start with:
%
\begin{center}
\begin{tabular}{l}
|\input{childdoc.def}|\\
|\childdocby{|\textit{main}|}|\\
\end{tabular}
\end{center}
%
The directive |\childdocby| is similar to |\childdocof|
described in \secref{sec:include},
but the subsequent selection of content must be done manually.
To that end, both |\ifchilddoc| and |\ifchilddocmanual|
will be true upon processing of a part,
and the name of the part is stored in |\childdocname|.
Note that |\jobname| will be set to the filename of the current part
so that each part receives an individual |.aux| file
that does not interfere with the |.aux| file(s) of the main document.
This behaviour can be altered by the alternative form
|\childdocby[*]{|\textit{main}|}| (with a non-empty optional argument)
which uses the |.aux| file of the main document
by setting |\jobname| to \textit{main}.

%%%%%%%%%%%%%%%%%%%%%%%%%%%%%%%%%%%%%%%%%%%%%%%%%%%%%%%%%%%%%%%%%%%%%%%%%%%%%%%%
\subsection{Driver Development}
\label{sec:driver}

The \textsf{childdoc} mechanism can also be use for the development
of definition files such as \LaTeX{} styles or classes.
This case differs from the above setup with multiple parts
included by |\include| in that no |\includeonly| should be invoked.
This can be achieved by starting the include file
(before |\ProvidesPackage|) with:
%
\begin{center}
\begin{tabular}{l}
|\input{childdoc.def}|\\
|\childdocforward{|\textit{main}|}|\\
\end{tabular}
\end{center}
%
or alternatively with:
%
\begin{center}
\begin{tabular}{l}
|\input{childdoc.def}|\\
|\childdocby{|\textit{main}|}|\\
\end{tabular}
\end{center}
%
Both forms have slightly different effects as described above.
The main file is prepared as usual, see \secref{sec:include}.

%%%%%%%%%%%%%%%%%%%%%%%%%%%%%%%%%%%%%%%%%%%%%%%%%%%%%%%%%%%%%%%%%%%%%%%%%%%%%%%%
\subsection{Legacy Detection}
\label{sec:detection}

The directive |\childdocmain| in the main file can detect
whether the complete document or merely a child is to be compiled
even without using the directive |\childdocof|.
This method is deprecated because it is less robust
and there is no compelling reason to use it;
it is merely provided for backward compatibility
and it may be removed in future versions.

If the detection mechanism is to be used,
it is mandatory to correctly specify
the filename of the main file as the argument of |\childdocmain|:
%
\begin{center}
\begin{tabular}{l}
|\input{childdoc.def}|\\
|\childdocmain{|\textit{main}|}|\\
\end{tabular}
\end{center}
%
If |\jobname| does not match the argument \textit{main} of |\childdocmain|,
it is assumed that |\jobname| points to the child file to be compiled.
When using |\childdocmain| with the main file specified as argument,
it suffices to start a child file
with just |\input{|\textit{main}|}|
without loading of the package and using |\childdocof|.
If instead all processing is done
with the appropriate \textsf{childdoc} directives,
the argument of \textit{main} of |\childdocmain| can be empty.

An alternative version of the command line processing described
in \secref{sec:commandline} using the detection mechanism reads:
%
\begin{center}
|... -jobname "|\textit{target}|" "|[\textit{flags}]%
[|\def\jobname{|\textit{dest}|}|]|\input{|\textit{main}|}"|
\end{center}

%%%%%%%%%%%%%%%%%%%%%%%%%%%%%%%%%%%%%%%%%%%%%%%%%%%%%%%%%%%%%%%%%%%%%%%%%%%%%%%%
\subsection{Manual Code}
\label{sec:manual}

In case one cannot be certain whether the definitions file |childdoc.def|
is installed on the target \TeX{} distribution
and one prefers not to ship it,
it is conceivable to paste a few relevant commands into the sources.

To that end, drop all statements |\input{childdoc.def}|
and perform the replacements as outlined below.
Instead of |\childdocmain{|\textit{main}|}| add the following code
to the top of the main file:
%
\begin{center}
\begin{tabular}{l}
|\||ifdefined\childdocname\endinput\||fi\newif\ifchilddoc|\\
|\edef\childdocname{\scantokens\expandafter{\jobname\noexpand}}|\\
|\def\childdocmain{|\textit{main}|}\||ifx\childdocmain\childdocname\||else|\\
|\childdoctrue\includeonly{\childdocname}\let\jobname\childdocmain\||fi|\\
\end{tabular}
\end{center}
%
Instead of |\childdocof{|\textit{main}|}| just include the main file
at the top of each child file:
%
\begin{center}
|\input{|\textit{main}|}|
\end{center}
%
A simple redirection |\childdocforward{|\textit{dest}|}| is achieved by:
%
\begin{center}
|\def\jobname{|\textit{dest}|}\input{\jobname}|
\end{center}
%
The redirection with prefix
|\childdocforwardprefix[|\textit{prefix}|]{|\textit{dest}|}|
is accomplished by:
%
\begin{center}
\begin{tabular}{l}
|{\edef\jobname{\scantokens\expandafter{\jobname\noexpand}}|\\
|\def\redirectjob |\textit{prefix}|#1~~~{\gdef\jobname{|\textit{dest}|#1}}|\\
|\expandafter\redirectjob\jobname~~~}\input{\jobname}|
\end{tabular}
\end{center}

In an alternative approach,
child documents can be compiled by a specific command line
without additional code or specific definitions:
%
\begin{center}
|... -jobname "|\textit{target}|" "|[\textit{flags}]%
|\includeonly{|\textit{dest}|}\input{|\textit{main}|}"|
\end{center}
%

%%%%%%%%%%%%%%%%%%%%%%%%%%%%%%%%%%%%%%%%%%%%%%%%%%%%%%%%%%%%%%%%%%%%%%%%%%%%%%%%
%%%%%%%%%%%%%%%%%%%%%%%%%%%%%%%%%%%%%%%%%%%%%%%%%%%%%%%%%%%%%%%%%%%%%%%%%%%%%%%%
\section{Information}

%%%%%%%%%%%%%%%%%%%%%%%%%%%%%%%%%%%%%%%%%%%%%%%%%%%%%%%%%%%%%%%%%%%%%%%%%%%%%%%%
\subsection{Copyright}

Copyright \copyright{} 2017--2018 Niklas Beisert

This work may be distributed and/or modified under the
conditions of the \LaTeX{} Project Public License, either version 1.3
of this license or (at your option) any later version.
The latest version of this license is in
  \url{http://www.latex-project.org/lppl.txt}
and version 1.3 or later is part of all distributions of \LaTeX{}
version 2005/12/01 or later.

This work has the LPPL maintenance status `maintained'.

The Current Maintainer of this work is Niklas Beisert.

This work consists of the files |README.txt|, |childdoc.ins| and |childdoc.dtx|
as well as the derived files |childdoc.def|, |cdocsamp.tex|
with |cdocsch1.tex|, |cdocsch2.tex|, |cdocspt3.tex|, |cdocspt4.tex|,
|cdocsdrf.tex|, |cdocsfn1.tex|, |cdocsfn2.tex|
as well as |childdoc.pdf|.

%%%%%%%%%%%%%%%%%%%%%%%%%%%%%%%%%%%%%%%%%%%%%%%%%%%%%%%%%%%%%%%%%%%%%%%%%%%%%%%%
\subsection{Files and Installation}

The package consists of the files:
%
\begin{center}
\begin{tabular}{ll}
    |README.txt|   & readme file \\
    |childdoc.ins| & installation file \\
    |childdoc.dtx| & source file \\
    |childdoc.def| & definition file \\
    |cdocsamp.tex| & sample main file \\
    |cdocsch1.tex| & sample include file \\
    |cdocsch2.tex| & sample include file \\
    |cdocspt3.tex| & sample part file \\
    |cdocspt4.tex| & sample part file \\
    |cdocsdrf.tex| & sample redirection file \\
    |cdocsfn1.tex| & sample redirection file \\
    |cdocsfn2.tex| & sample redirection file \\
    |childdoc.pdf| & manual
\end{tabular}
\end{center}
%
The distribution consists of the files
|README.txt|, |childdoc.ins| and |childdoc.dtx|.
%
\begin{itemize}
\item
Run (pdf)\LaTeX{} on |childdoc.dtx|
to compile the manual |childdoc.pdf| (this file).
\item
Run \LaTeX{} on |childdoc.ins| to create the definitions file |childdoc.def|
and the sample |cdocsamp.tex| with include files
|cdocsch1.tex|, |cdocsch2.tex|, |cdocspt3.tex|, |cdocspt4.tex|,
|cdocsdrf.tex|, |cdocsfn1.tex|, |cdocsfn2.tex|.
Then copy the file |childdoc.def| to an appropriate directory of your \LaTeX{}
distribution, e.g.\ \textit{texmf-root}|/tex/latex/childdoc|.
\end{itemize}

%%%%%%%%%%%%%%%%%%%%%%%%%%%%%%%%%%%%%%%%%%%%%%%%%%%%%%%%%%%%%%%%%%%%%%%%%%%%%%%%
\subsection{Related CTAN Packages}

There are several other packages which offer a similar functionality:
%
\begin{itemize}
\item
The packages
\href{http://ctan.org/pkg/docmute}{\textsf{docmute}},
\href{http://ctan.org/pkg/includex}{\textsf{includex}} and
\href{http://ctan.org/pkg/standalone}{\textsf{standalone}}
provide commands to include only the document body of
a child file thus allowing both files to be compiled individually.
\item
The packages \href{http://ctan.org/pkg/subdocs}{\textsf{subdocs}}
and \href{http://ctan.org/pkg/subfiles}{\textsf{subfiles}}
provide structures in which the main and child documents can be
encapsulated and allowing them to be compiled individually.
The inclusion mechanism is different from the conventional |\include|.
\item
The package \href{http://ctan.org/pkg/combine}{\textsf{combine}}
is an elaborate solution to combine several documents into one.
\end{itemize}
%
See also the CTAN topic \href{http://ctan.org/topic/subdocs}{\textsf{subdocs}}
for further related packages.
The present package differs from the above solutions in that
a document structure constructed with the conventional |\include| mechanism
just needs two extra commands at the top of every file
such that all constituent files can be compiled individually.

%%%%%%%%%%%%%%%%%%%%%%%%%%%%%%%%%%%%%%%%%%%%%%%%%%%%%%%%%%%%%%%%%%%%%%%%%%%%%%%%
%\subsection{Feature Suggestions}
%
%The following is a list of features which may be useful for future
%versions of this package:
%%
%\begin{itemize}
%\item
%\ldots
%\end{itemize}

%%%%%%%%%%%%%%%%%%%%%%%%%%%%%%%%%%%%%%%%%%%%%%%%%%%%%%%%%%%%%%%%%%%%%%%%%%%%%%%%
\subsection{Revision History}

%%%%%%%%%%%%%%%%%%%%%%%%%%%%%%%%%%%%%%%%
\paragraph{v2.0:} 2018/12/30

\begin{itemize}
\item
immediate forward processing
\item
added |\childdocby| mechanism
\item
manual restructured
\end{itemize}

%%%%%%%%%%%%%%%%%%%%%%%%%%%%%%%%%%%%%%%%
\paragraph{v1.6:} 2018/01/17

\begin{itemize}
\item
application for development of include files
\item
corrections to manual
\end{itemize}

%%%%%%%%%%%%%%%%%%%%%%%%%%%%%%%%%%%%%%%%
\paragraph{v1.5:} 2017/05/21

\begin{itemize}
\item
more complete structuring introduced
\item
|\childdocof| introduced
\item
|\childdoc| renamed to |\childdocmain|
\item
|\childredirect| renamed to |\childdocforward| and |\childdocforwardprefix|
and functionality expanded
\end{itemize}

%%%%%%%%%%%%%%%%%%%%%%%%%%%%%%%%%%%%%%%%
\paragraph{v1.0:} 2017/04/27

\begin{itemize}
\item
manual and install package
\item
first version published on CTAN
\end{itemize}

%%%%%%%%%%%%%%%%%%%%%%%%%%%%%%%%%%%%%%%%
\paragraph{v0.6:} 2017/04/26

\begin{itemize}
\item
redirection mechanism added
\end{itemize}

%%%%%%%%%%%%%%%%%%%%%%%%%%%%%%%%%%%%%%%%
\paragraph{v0.5:} 2017/04/26

\begin{itemize}
\item
functionality in definition file
\end{itemize}


%%%%%%%%%%%%%%%%%%%%%%%%%%%%%%%%%%%%%%%%%%%%%%%%%%%%%%%%%%%%%%%%%%%%%%%%%%%%%%%%
%%%%%%%%%%%%%%%%%%%%%%%%%%%%%%%%%%%%%%%%%%%%%%%%%%%%%%%%%%%%%%%%%%%%%%%%%%%%%%%%
%%%%%%%%%%%%%%%%%%%%%%%%%%%%%%%%%%%%%%%%%%%%%%%%%%%%%%%%%%%%%%%%%%%%%%%%%%%%%%%%
\appendix

\settowidth\MacroIndent{\rmfamily\scriptsize 000\ }

 \DocInput{childdoc.dtx}

\end{document}
%</driver>
% \fi
%
% %%%%%%%%%%%%%%%%%%%%%%%%%%%%%%%%%%%%%%%%%%%%%%%%%%%%%%%%%%%%%%%%%%%%%%%%%%%%%%
% %%%%%%%%%%%%%%%%%%%%%%%%%%%%%%%%%%%%%%%%%%%%%%%%%%%%%%%%%%%%%%%%%%%%%%%%%%%%%%
% \section{Sample}
%\iffalse
%<*samplemain>
%\fi
%
% The following presents a sample document
% with two chapters, two parts, a title page,
% a compile flag as well as three forwarding files to set the flag.
% It consists of eight |.tex| files:
% \begin{center}
% \begin{tabular}{ll}
% |cdocsamp.tex|&main file\\
% |cdocsch1.tex|&include file for chapter 1\\
% |cdocsch2.tex|&include file for chapter 2\\
% |cdocspt3.tex|&include file for part 3\\
% |cdocspt4.tex|&include file for part 4\\
% |cdocsdrf.tex|&forwarding file for main file in draft mode\\
% |cdocsfi1.tex|&forwarding file for final version of chapter 1\\
% |cdocsfi2.tex|&forwarding file for final version of chapter 2\\
% \end{tabular}
% \end{center}
% Each of the eight files can be compiled directly by the \LaTeX{} compiler.
%
% %%%%%%%%%%%%%%%%%%%%%%%%%%%%%%%%%%%%%%
% \paragraph{Main File.}
%
% The main file is called |cdocsamp.tex|.
%
% Load the \textsf{childdoc} definitions and
% declare the filename for the main document:
%    \begin{macrocode}
\input{childdoc.def}
\childdocmain{}
%    \end{macrocode}

% Optional override for |\version| flag:
%    \begin{macrocode}
%%\ifchilddoc\else\providecommand{\version}{draft}\fi
%    \end{macrocode}

% Define the default values for the |\version| flag
% (|final| for the main file and |draft| for childs):
%    \begin{macrocode}
\ifchilddoc
\providecommand{\version}{draft}
\else
\providecommand{\version}{final}
\fi
%    \end{macrocode}

% Load the standard document class:
%    \begin{macrocode}
\documentclass[12pt]{article}
%    \end{macrocode}

% Start the document body:
%    \begin{macrocode}
\begin{document}
%    \end{macrocode}

% Declare a title page.
% Print title, part of document being processed and version flag:
%    \begin{macrocode}
\addtocounter{page}{-1}
\begin{center}
{\LARGE\bfseries{}childdoc example\par}
\vspace{1cm}
\ifchilddoc
\ifchilddocmanual part\else chapter\fi:
`\childdocname' of `\childdocjob'\par
\else
main document: `\childdocjob'\par
\fi
version: \version\par
\end{center}
\newpage
%    \end{macrocode}

% Manually include selected file,
% otherwise process as usual:
%    \begin{macrocode}
\ifchilddocmanual
\section*{part `\childdocname'}
\input{\childdocname}
\else
%    \end{macrocode}

% Include the two chapters:
%    \begin{macrocode}
\include{cdocsch1}
\include{cdocsch2}
%    \end{macrocode}

% Include the two parts unless only chapters should be displayed:
%    \begin{macrocode}
\ifchilddoc\else
\section{part three}
\input{cdocspt3}
\section{part four}
\input{cdocspt4}
\fi
%    \end{macrocode}

% Process as usual until here:
%    \begin{macrocode}
\fi
%    \end{macrocode}

% End of document body:
%    \begin{macrocode}
\end{document}
%    \end{macrocode}
%\iffalse
%</samplemain>
%\fi
%
% %%%%%%%%%%%%%%%%%%%%%%%%%%%%%%%%%%%%%%
% \paragraph{Chapter Include Files.}
%
% The include files are called |cdocsch1.tex| and |cdocsch2.tex|.
%
%\iffalse
%<*samplechap1|samplechap2>
%\fi

% Optional override for |\version| flag:
%    \begin{macrocode}
%%\providecommand{\version}{final}
%    \end{macrocode}

% Include the main document:
%    \begin{macrocode}
\input{childdoc.def}
\childdocof{cdocsamp}
%    \end{macrocode}

%\iffalse
%</samplechap1|samplechap2>
%\fi
%
%\iffalse
%<*samplechap1>
%\fi
% Some text for chapter 1:
%    \begin{macrocode}
\section{one}
some text in chapter one
%    \end{macrocode}

%\iffalse
%</samplechap1>
%\fi
% Some text for chapter 2:
%\iffalse
%<*samplechap2>
%\fi
%    \begin{macrocode}
\section{two}
more text in chapter two
%    \end{macrocode}

%\iffalse
%</samplechap2>
%\fi
%
% %%%%%%%%%%%%%%%%%%%%%%%%%%%%%%%%%%%%%%
% \paragraph{Part Include Files.}
%
% The include files are called |cdocspt3.tex| and |cdocspt4.tex|.
%
%\iffalse
%<*samplepart3|samplepart4>
%\fi

% Optional override for |\version| flag:
%    \begin{macrocode}
%%\providecommand{\version}{final}
%    \end{macrocode}

% Include the main document:
%    \begin{macrocode}
\input{childdoc.def}
\childdocby{cdocsamp}
%    \end{macrocode}

%\iffalse
%</samplepart3|samplepart4>
%\fi
%
%\iffalse
%<*samplepart3>
%\fi
% Some text for part 3:
%    \begin{macrocode}
some text in part three
%    \end{macrocode}

%\iffalse
%</samplepart3>
%\fi
% Some text for part 4:
%\iffalse
%<*samplepart4>
%\fi
%    \begin{macrocode}
more text in part four
%    \end{macrocode}

%\iffalse
%</samplepart4>
%\fi
%
% %%%%%%%%%%%%%%%%%%%%%%%%%%%%%%%%%%%%%%
% \paragraph{Forwarding for a Complete Draft.}
%
% The following forwarding file |cdocsdrf.tex|
% compiles the main document in draft mode:
%\iffalse
%<*sampledraft>
%\fi
%    \begin{macrocode}
\def\version{draft}
\input{childdoc.def}
\childdocforward{cdocsamp}
%    \end{macrocode}

%\iffalse
%</sampledraft>
%\fi
%
% %%%%%%%%%%%%%%%%%%%%%%%%%%%%%%%%%%%%%%
% \paragraph{Forwarding for Final Version of the Chapters.}
%
% The following forwarding files |cdocsfn1.tex| and |cdocsfn2.tex|
% (with identical content)
% compile the final versions of the child documents
% |cdocsch1.tex| and |cdocsch2.tex|, respectively:
%\iffalse
%<*samplefinal>
%\fi
%    \begin{macrocode}
\def\version{final}
\input{childdoc.def}
\childdocforwardprefix[cdocsamp]{cdocsfn}{cdocsch}
%    \end{macrocode}

%\iffalse
%</samplefinal>
%\fi
%
% %%%%%%%%%%%%%%%%%%%%%%%%%%%%%%%%%%%%%%
% \paragraph{Command Line Processing.}
%
% The following three command lines generate the output files
% |cdocscld|, |cdocscl1| and |cdocscl2|
% which should be identical to
% |cdocsdrf|, |cdocsch1| and |cdocsfn2|, respectively:
% \begin{center}
% \begin{tabular}{l}
% |latex -jobname cdocscld \|\\
% |  "\def\version{draft}\input{childdoc.def}\childdocforward{cdocsamp}"|\\
% |latex -jobname cdocscl1 \|\\
% |  "\input{childdoc.def}\childdocforward[cdocsamp]{cdocsch1}"|\\
% |latex -jobname cdocscl2 \|\\
% |  "\def\version{final}\input{childdoc.def}\childdocforward{cdocsch2}"|
% \end{tabular}
% \end{center}
% Note that the trailing backslash on each first line
% merely continues the input to the second line
% (for convenient cut ant paste).
% Furthermore, the command |latex| can be replaced by any
% of its alternative versions such as |pdflatex|.
%
% %%%%%%%%%%%%%%%%%%%%%%%%%%%%%%%%%%%%%%%%%%%%%%%%%%%%%%%%%%%%%%%%%%%%%%%%%%%%%%
% %%%%%%%%%%%%%%%%%%%%%%%%%%%%%%%%%%%%%%%%%%%%%%%%%%%%%%%%%%%%%%%%%%%%%%%%%%%%%%
% \section{Implementation}
%\iffalse
%<*package>
%\fi
%
% This section describes the definitions file |childdoc.def|.

% The definitions cannot be loaded using |\usepackage| or |\RequirePackage|
% which has a mechanism to prevent loading a style file more than once.
% When loading the definitions by means of |\input|
% multiple instances have to be prevented manually:
%\iffalse
%This code needs to be before the `\ProvidesFile' directive
%which is defined at the beginning of this file.
%Therefore it is also placed there and commented out here.
%</package>
%<*discard>
%\fi
%    \begin{macrocode}
\ifdefined\childdocmain\endinput\fi
%    \end{macrocode}
%\iffalse
%</discard>
%<*package>
%\fi
%
% \macro{\ifchilddoc}
% \macro{\ifchilddocmanual}
% The conditional |\ifchilddoc| tells whether a
% child (true) or main (false) document is being compiled.
% The conditional |\ifchilddocmanual| tells whether
% the |\includeonly| mechanism is used (false) or
% the selection of child files must be performed manually (true).
% The definitions initialise to false:
%    \begin{macrocode}
\newif\ifchilddoc
\newif\ifchilddocmanual
%    \end{macrocode}

% \macro{\childdocname}
% \macro{\childdocjob}
% The macro |\childdocname| stores the name of the main document
% to be compiled. The macro |\childdocjob| stores the name of
% the document on which the \LaTeX{} compiler was originally invoked.
% The content of |\jobname| cannot be compared
% to filenames specified in the source due to different catcodes.
% The following code rescans |\jobname|, stores the result
% in |\childdocname| and saves a copy in |\childdocjob|:
%    \begin{macrocode}
\edef\childdocname{\scantokens\expandafter{\jobname\noexpand}}
\let\childdocjob\childdocname
%    \end{macrocode}

% \macro{\childdocdisable}
% The macro |\childdocdisable| prevents the main file
% from being processed more than once.
% At this stage, the main document command |\childdocmain|
% is assumed to be called once again where it should do nothing.
% Any subsequent call to it should prevent
% a secondary processing of the main document
% It overwrites the forwarding commands
% |\childdocof| and |\childdocforward|
% with empty macros to prevent further inclusions of the main document:
%    \begin{macrocode}
\newcommand{\childdocdisable}
{
  \renewcommand{\childdocmain}[1]{\renewcommand{\childdocmain}[1]{\endinput}}
  \renewcommand{\childdocof}[1]{}
  \renewcommand{\childdocby}[2][]{}
  \renewcommand{\childdocforward}[2][]{}
  \renewcommand{\childdocdisable}{}
}
%    \end{macrocode}

% \macro{\childdocmain}
% The macro |\childdocmain| is to be called at the top of the main file
% with nothing or the main filename (without extension) as argument.
% First, it breaks loops.
% If the argument is not empty and does not match |\childdocname|
% (which is set by the first inclusion of |childdoc.def|),
% |\ifchilddoc| is set to true, |\includeonly| is applied to the child file
% and |\jobname| is set to the main file
% (for proper handling of |.aux| files):
%    \begin{macrocode}
\newcommand{\childdocmain}[1]
{
  \childdocdisable\childdocmain{}
  \if?#1?\else
    \begingroup
      \def\childdoctmp{#1}
      \ifx\childdoctmp\childdocname
        \def\childdoctmp{}
      \else
        \def\childdoctmp
        {
          \childdoctrue
          \includeonly{\childdocname}
          \def\childdocjob{#1}
          \def\jobname{#1}
        }
      \fi
      \expandafter
    \endgroup
    \childdoctmp
  \fi
}
%    \end{macrocode}

% \macro{\childdocof}
% The command |\childdocof| redirects
% compilation to the main file |#1|.
%    \begin{macrocode}
\newcommand{\childdocof}[1]
{
  \childdocdisable
  \childdoctrue
  \includeonly{\childdocname}
  \def\jobname{#1}
  \def\childdocjob{#1}
  \input{#1}
}
%    \end{macrocode}

% \macro{\childdocby}
% The command |\childdocby| ....
%    \begin{macrocode}
\newcommand{\childdocby}[2][]
{
  \childdocdisable
  \childdoctrue
  \childdocmanualtrue
  \if?#1?\else
    \def\jobname{#2}
  \fi
  \def\childdocjob{#2}
  \input{#2}
  \endinput
}
%    \end{macrocode}

% \macro{\childdocforward}
% The command |\childdocforward| redirects
% compilation to the main file or
% (if the optional argument is given) a child file.
% Parameters are set as if the main file
% or a child file starting with |\childdocof| was compiled.
% Then compilation is handed over to the main file:
%    \begin{macrocode}
\newcommand{\childdocforward}[2][]
{
  \begingroup
    \if?#1?
      \def\childdoctmp
      {
        \def\childdocname{#2}
        \def\childdocjob{#2}
        \def\jobname{#2}
        \input{#2}
        \endinput
      }
    \else
      \def\childdoctmp
      {
        \childdocdisable
        \def\childdocname{#2}
        \childdoctrue
        \includeonly{#2}
        \def\childdocjob{#1}
        \def\jobname{#1}
        \input{#1}
        \endinput
      }
    \fi
    \expandafter
  \endgroup
  \childdoctmp
}
%    \end{macrocode}

% \macro{\childdocforwardprefix}
% The command |\childdocforwardprefix| redirects
% compilation to the main or a child file by means of a pattern.
% The prefix |#1| in the current filename is replaced by |#2|
% and the suffix of the current filename is kept
% (it is assumed that the filename does not contain the substring `|~~~|'
% which is used as a delimiter).
% Compilation is handed over to the new file by |\childdocforward|:
%    \begin{macrocode}
\newcommand{\childdocforwardprefix}[3][]
{
  \begingroup
    \def\childdocextract #2##1~~~{\def\childdoctmp{\childdocforward[#1]{#3##1}}}
    \expandafter\childdocextract\childdocname~~~
    \expandafter
  \endgroup
  \childdoctmp
}
%    \end{macrocode}

% \macro{\childdoc}
% The deprecated macro |\childdoc| is a legacy version of |\childdocmain|:
%    \begin{macrocode}
\newcommand{\childdoc}{\childdocmain}
%    \end{macrocode}

% \macro{\childdocredirect}
% The deprecated macro |\childdocredirect| is a legacy version
% of |\childdocforward| and |\childdocforwardprefix|:
%    \begin{macrocode}
\newcommand{\childdocredirect}[2][]
{
  \begingroup
    \if?#1?
      \def\childdoctmp{\childdocforward{#2}}
    \else
      \def\childdoctmp{\childdocforwardprefix{#1}{#2}}
    \fi
    \expandafter
  \endgroup
  \childdoctmp
}
%    \end{macrocode}

%\iffalse
%</package>
%\fi
%
\endinput
|\\
|\childdocby{|\textit{main}|}|\\
\end{tabular}
\end{center}
%
Both forms have slightly different effects as described above.
The main file is prepared as usual, see \secref{sec:include}.

%%%%%%%%%%%%%%%%%%%%%%%%%%%%%%%%%%%%%%%%%%%%%%%%%%%%%%%%%%%%%%%%%%%%%%%%%%%%%%%%
\subsection{Legacy Detection}
\label{sec:detection}

The directive |\childdocmain| in the main file can detect
whether the complete document or merely a child is to be compiled
even without using the directive |\childdocof|.
This method is deprecated because it is less robust
and there is no compelling reason to use it;
it is merely provided for backward compatibility
and it may be removed in future versions.

If the detection mechanism is to be used,
it is mandatory to correctly specify
the filename of the main file as the argument of |\childdocmain|:
%
\begin{center}
\begin{tabular}{l}
|% \iffalse
%
% childdoc.dtx Copyright (C) 2017-2018 Niklas Beisert
%
% This work may be distributed and/or modified under the
% conditions of the LaTeX Project Public License, either version 1.3
% of this license or (at your option) any later version.
% The latest version of this license is in
%   http://www.latex-project.org/lppl.txt
% and version 1.3 or later is part of all distributions of LaTeX
% version 2005/12/01 or later.
%
% This work has the LPPL maintenance status `maintained'.
%
% The Current Maintainer of this work is Niklas Beisert.
%
% This work consists of the files childdoc.dtx and childdoc.ins
% and the derived files childdoc.def and cdocsamp.tex with
% cdocsch1.tex, cdocsch2.tex, cdocsdrf.tex, cdocsfn1.tex, cdocsfn2.tex.
%
%<package>\ifdefined\childdocmain\endinput\fi
%<package>\ProvidesFile{childdoc.def}[2018/12/30 v2.0 child document driver]
%<samplemain>\ProvidesFile{cdocsamp.tex}[2018/12/30 v2.0 sample for childdoc]
%<*driver>
%\ProvidesFile{childdoc.drv}[2018/12/30 v2.0 childdoc reference manual file]
\PassOptionsToClass{10pt,a4paper}{article}
\documentclass{ltxdoc}

\usepackage[margin=35mm]{geometry}
\usepackage{hyperref}
\usepackage{hyperxmp}
\usepackage[usenames]{color}

\hypersetup{colorlinks=true}
\hypersetup{pdfstartview=FitH}
\hypersetup{pdfpagemode=UseNone}
\hypersetup{pdfsource={}}
\hypersetup{pdflang={en-UK}}
\hypersetup{pdfcopyright={Copyright 2017-2018 Niklas Beisert.
  This work may be distributed and/or modified under the
  conditions of the LaTeX Project Public License, either version 1.3
  of this license or (at your option) any later version.}}
\hypersetup{pdflicenseurl={http://www.latex-project.org/lppl.txt}}
\hypersetup{pdfcontactaddress={ETH Zurich, ITP, HIT K,
  Wolfgang-Pauli-Strasse 27}}
\hypersetup{pdfcontactpostcode={8093}}
\hypersetup{pdfcontactcity={Zurich}}
\hypersetup{pdfcontactcountry={Switzerland}}
\hypersetup{pdfcontactemail={nbeisert@itp.phys.ethz.ch}}
\hypersetup{pdfcontacturl={http://people.phys.ethz.ch/\xmptilde nbeisert/}}

\newcommand{\secref}[1]{\hyperref[#1]{section \ref*{#1}}}

\parskip1ex
\parindent0pt
\let\olditemize\itemize
\def\itemize{\olditemize\parskip0pt}

\begin{document}

\title{The \textsf{childdoc} Package}
\hypersetup{pdftitle={The childdoc Package}}
\author{Niklas Beisert\\[2ex]
  Institut f\"ur Theoretische Physik\\
  Eidgen\"ossische Technische Hochschule Z\"urich\\
  Wolfgang-Pauli-Strasse 27, 8093 Z\"urich, Switzerland\\[1ex]
  \href{mailto:nbeisert@itp.phys.ethz.ch}
  {\texttt{nbeisert@itp.phys.ethz.ch}}}
\hypersetup{pdfauthor={Niklas Beisert}}
\hypersetup{pdfsubject={Manual for the LaTeX2e Package childdoc}}
\date{30 December 2018, \textsf{v2.0}}
\maketitle

\begin{abstract}\noindent
\textsf{childdoc} is a \LaTeXe{} package
that enables the direct compilation
of document sections included by |\include|
to individual files.
\end{abstract}

\begingroup
\parskip0ex
\tableofcontents
\endgroup

%%%%%%%%%%%%%%%%%%%%%%%%%%%%%%%%%%%%%%%%%%%%%%%%%%%%%%%%%%%%%%%%%%%%%%%%%%%%%%%%
%%%%%%%%%%%%%%%%%%%%%%%%%%%%%%%%%%%%%%%%%%%%%%%%%%%%%%%%%%%%%%%%%%%%%%%%%%%%%%%%
\section{Introduction}

\LaTeX{} provides a mechanism to structure a large document (such as a book)
into a main file and several child files (containing the chapters)
using the |\include| command.
This mechanism is beneficial for documents
which span hundreds of pages in order to
make the source file(s) more manageable.
Moreover, compilation can be restricted to
selected child files by means of the |\includeonly| command.
The latter feature can be used to reduce the compilation time while editing
(this was significantly more useful in the earlier days of \LaTeX{})
or to generate a smaller document which is easier to navigate.
Another application of |\includeonly| is to generate
documents consisting of selected parts of the complete document.

However, there are a few drawbacks of the plain |\include| mechanism:
\begin{itemize}
\item
The child files cannot be compiled on their own,
they can only be compiled via the main file.
A naive editing environment
(such as a text editor with an option
to have the current file processed by \LaTeX)
may require one to switch to the main file before compiling;
attempting to compile the child file produces errors.
\item
The main file must be modified (each time)
to adjust the |\includeonly| command
to the present needs. This easily leaves the main file in a messy state.
\item
The generated document will always carry the filename
of the main document. This is inconvenient if
several child files are to be compiled and
to be kept for distribution.
\end{itemize}

The present package provides a simple interface
to make child files individually compilable by \LaTeX{}.
Compiling a child file then has the same effect as compiling
the main file with an |\includeonly| command
to select the appropriate child.
Moreover the generated document will carry the name of the child
rather than the main file.
This resolves all three above issues.

This feature is meant to make the editing of books,
thesis documents and lecture notes somewhat more convenient.
However, the package can also be used efficiently for
composing a series of documents (such as exercise sheets)
which are typically distributed individually.
It then assists the author in generating the individual documents
(potentially in different versions)
as well as a document containing the collected series.
Another application is in developing style files
or other kinds of included material
where compilation of the style file could redirect
to a sample or test file.

%%%%%%%%%%%%%%%%%%%%%%%%%%%%%%%%%%%%%%%%%%%%%%%%%%%%%%%%%%%%%%%%%%%%%%%%%%%%%%%%
%%%%%%%%%%%%%%%%%%%%%%%%%%%%%%%%%%%%%%%%%%%%%%%%%%%%%%%%%%%%%%%%%%%%%%%%%%%%%%%%
\section{Usage}

First of all, the package \textsf{childdoc} is \emph{not} a standard
\LaTeXe{} |.sty| style file! Therefore it needs to be invoked in
a non-standard way.

%%%%%%%%%%%%%%%%%%%%%%%%%%%%%%%%%%%%%%%%%%%%%%%%%%%%%%%%%%%%%%%%%%%%%%%%%%%%%%%%
\subsection{Included Files}
\label{sec:include}

%%%%%%%%%%%%%%%%%%%%%%%%%%%%%%%%%%%%%%%%
\DescribeMacro{\childdocmain}
To use the package, add the commands
\begin{center}
\begin{tabular}{l}
|\input{childdoc.def}|\\
|\childdocmain{}|\\
\end{tabular}
\end{center}
at the very top of the main \LaTeX{} file,
in particular \emph{before} the |\documentclass| statement!
The argument of |\childdocmain| should be left empty
(but it must be present).

%%%%%%%%%%%%%%%%%%%%%%%%%%%%%%%%%%%%%%%%
\DescribeMacro{\childdocof}
Furthermore, add the commands
\begin{center}
\begin{tabular}{l}
|\input{childdoc.def}|\\
|\childdocof{|\textit{main}|}|\\
\end{tabular}
\end{center}
at the top of every child file \textit{child}
which is included by |\include{|\textit{child}|}|
from within the main file
(or at least for those files to be compiled individually).
The argument \textit{main} must be the filename of the main file.

There are a couple of
considerations in setting up the main and child documents:

%%%%%%%%%%%%%%%%%%%%%%%%%%%%%%%%%%%%%%%%
\paragraph{Restrictions.}

Please note the following restrictions:
\begin{itemize}
\item
|\childdocmain| must be called with one argument \textit{main}
to ensure compatibility with earlier version of the package.
It must either be empty (|\childdocmain{}|)
or precisely match the filename of the main file in which it is specified.
See \secref{sec:detection} for further information.
\item
The filename \textit{main} must be specified without the |.tex| extension.
\item
The filename \textit{main} is case sensitive
(even in case-insensitive file systems)
due to internal string comparison.
\item
The argument \textit{main} should be fully expanded, it cannot be a macro.
\item
Subdirectories and special characters should be avoided in filenames.
\item
The command |\childdocmain{|\textit{main}|}| must be followed by a whitespace.
It should not be followed immediately by another command
or by a comment mark `|%|'.
This is because the \TeX{} parser reads the token immediately following
the argument of |\childdocmain| and puts it
at the beginning of every child section;
however, a white\-space is ignored.
\end{itemize}

%%%%%%%%%%%%%%%%%%%%%%%%%%%%%%%%%%%%%%%%
\paragraph{Content of Main File.}

It is advisable to place all content in the child files included by |\include|.
Any output contained in the main file will appear in all child documents
unless suppressed manually;
it cannot be suppressed automatically by the |\includeonly| directive
and thus should normally be avoided.
A method to include some content in the main file
by means of conditional processing is described in \secref{sec:conditional}.

%%%%%%%%%%%%%%%%%%%%%%%%%%%%%%%%%%%%%%%%
\paragraph{Page Numbering.}

When only a part of the document is compiled,
the appropriate numbering of pages
(as well as other status parameters)
is determined from the |.aux| files.
The latter contain information from previous passes.
However this information needs to propagate through
all intermediate child documents.
Therefore the page numbering in child documents may well
be inconsistent until the complete document is compiled at least once.

A useful (if unconventional) way to always ensure a consistent
page numbering is to restart the numbering in each child document
and denote the pages by `\textit{child}|.|\textit{page}'
where \textit{child} represents the chapter/section number of the child file.
This can be achieved by the command
|\numberwithin{page}{|\textit{child}|}|
of the \textsf{amsmath} package
where \textit{child} can be |chapter| or |section|
depending on the chosen structuring.
Alternatively, one can modify the macro |\thepage| appropriately
and reset the counter |page| at the start of each child file.

%%%%%%%%%%%%%%%%%%%%%%%%%%%%%%%%%%%%%%%%%%%%%%%%%%%%%%%%%%%%%%%%%%%%%%%%%%%%%%%%
\subsection{Conditional Processing}
\label{sec:conditional}

The package provides a mechanism to compile different versions
of a document. To customise the versions further some conditional processing
can come in handy to distinguish which version is being compiled.
The package provides two macros to describe the compilation context:

%%%%%%%%%%%%%%%%%%%%%%%%%%%%%%%%%%%%%%%%
\DescribeMacro{\ifchilddoc}
The conditional |\ifchilddoc| distinguishes between the compilation of
child documents and the main document:
%
\begin{center}
|\ifchilddoc |\textit{child-code}| |[|\||else |\textit{main-code}]| \||fi|
\end{center}

%%%%%%%%%%%%%%%%%%%%%%%%%%%%%%%%%%%%%%%%
\DescribeMacro{\childdocname}
\DescribeMacro{\childdocjob}
The macro |\childdocname| contains the filename (without extension)
of the main or child file being processed.
Note that |\childdocjob| will always contain the name of the main file.

%%%%%%%%%%%%%%%%%%%%%%%%%%%%%%%%%%%%%%%%
\paragraph{Title Page.}

Conditional processing can be used to include a title or banner page
in the main document when proper precautions are taken.
Importantly, the code in the main file should ensure that the page counter
(as well as other status parameters which are stored in the |.aux| files)
takes the same value after the conditional processing.
Otherwise the page numbers may take divergent values
depending on which part is compiled.

For example, a title page could be declared by:
%
\begin{center}
\begin{tabular}{l}
|\ifchilddoc\||else|\\
|\addtocounter{page}{-1}|\\
\textit{code for title page}\\
|\newpage|\\
|\||fi|
\end{tabular}
\end{center}
%
A banner page for the child documents can be generated by:
%
\begin{center}
\begin{tabular}{l}
|\ifchilddoc|\\
|\addtocounter{page}{-1}|\\
\textit{code for banner page}\\
|\newpage|\\
|\||fi|
\end{tabular}
\end{center}
%
Here one could write a message such as:
\begin{center}
|This is the part \childdocname{} of \childdocjob{}.|
\end{center}

%%%%%%%%%%%%%%%%%%%%%%%%%%%%%%%%%%%%%%%%%%%%%%%%%%%%%%%%%%%%%%%%%%%%%%%%%%%%%%%%
\subsection{Flags}
\label{sec:flags}

The package makes it easy to generate different versions
of the main or child documents.
To this end compilation flags can be defined
and assigned different default values.
They will be particularly useful in conjunction
with the forwarding mechanism described in \secref{sec:forward}.

For example, it may be useful to have a flag |\version|
which can be set to |draft| or |final|.
The document source will contain some conditional code
depending on the value of |\version|.
Suppose further, the flag should default to |final| for the main file
and to |draft| for child files
which is a natural assignment for editing the document.
This is achieved by placing the following code
in the preamble of the main document
(below the |\childdocmain| directive):
%
\begin{center}
\begin{tabular}{l}
|\ifchilddoc|\\
|\providecommand{\version}{draft}|\\
|\||else|\\
|\providecommand{\version}{final}|\\
|\||fi|
\end{tabular}
\end{center}
%
The definition by |\providecommand| makes sure
that previous definitions are not overwritten.
Further statements |\providecommand{\version}{...}|
can thus be added before the above code to override it.

For the main file, one might add a line
(between |\childdocmain| and the above block)
%
\begin{center}
|%\ifchilddoc\||else\providecommand{\version}{draft}\||fi|
\end{center}
%
which can be uncommented to produce a draft version.
Likewise one can add a line to the very top of a child file
(above the |\childdocof{|\textit{main}|}| directive)
%
\begin{center}
|%\providecommand{\version}{final}|
\end{center}
%
which can be uncommented to produce the final version of this child document.

%%%%%%%%%%%%%%%%%%%%%%%%%%%%%%%%%%%%%%%%%%%%%%%%%%%%%%%%%%%%%%%%%%%%%%%%%%%%%%%%
\subsection{Forwarding}
\label{sec:forward}

Different versions of the main or child documents
using compilation flags as described in \secref{sec:flags}
can be (permanently) stored in different files
for convenient compilation, viewing and distribution.
To this end, the package defines a command
to pass on compilation to a different file:

%%%%%%%%%%%%%%%%%%%%%%%%%%%%%%%%%%%%%%%%
\DescribeMacro{\childdocforward}
The command |\childdocforward| redirects processing to
another source file:
%
\begin{center}
\begin{tabular}{l}
|\input{childdoc.def}|\\
|\childdocforward[|\textit{main}|]{|\textit{dest}|}|\\
\end{tabular}
\end{center}
%
The argument \textit{dest} is the destination file
(without extension).
It should be the main file or one of the child files.
Note that further \textsf{childdoc} directives
such as |\childdocof| and |\childdocforward|
in the indicated file will be processed in this form.
The optional argument \textit{main}
passes on directly to the main file \textit{main}
while pretending to compile the child \textit{dest}.
This form behaves as if \textit{dest}
issues |\childdocof{|\textit{main}|}| right away,
and no further \textsf{childdoc} directives will be processed.

%%%%%%%%%%%%%%%%%%%%%%%%%%%%%%%%%%%%%%%%
\DescribeMacro{\...prefix}
In the alternative form |\childdocforwardprefix|,
%
\begin{center}
\begin{tabular}{l}
|\input{childdoc.def}|\\
|\childdocforwardprefix[|\textit{main}|]{|\textit{prefix}|}{|\textit{dest}|}|
\end{tabular}
\end{center}
%
the destination file is determined by a pattern
depending on the current file:
To make this work, the current file must be called
`{\textit{prefix}\hspace{0.2em}\textit{suffix}}'
with \textit{prefix} matching precisely the argument.
Processing is then passed on to the file
`{\textit{dest}\hspace{0.2em}\textit{suffix}}'.
Surely, the same effect is achieved by
directly specifying the
argument `{\textit{dest}\hspace{0.2em}\textit{suffix}}'
in the first form.
However, that requires to set up a different file
for each child. With the alternative form of the command
all these files can have exactly the same content
which simplifies setting them up and maintaining them.

For example, the following file |draft.tex|
with a compilation flag |\version| as described in \secref{sec:flags}
compiles the main document as a draft:
%
\begin{center}
\begin{tabular}{l}
|\def\version{draft}|\\
|\input{childdoc.def}|\\
|\childdocforward{|\textit{main}|}|
\end{tabular}
\end{center}
%
Likewise, the following files |final|\textit{nn}|.tex|
compile the final version of the child document
|child|\textit{nn}|.tex|:
%
\begin{center}
\begin{tabular}{l}
|\def\version{final}|\\
|\input{childdoc.def}|\\
|\childdocforwardprefix{final}{child}|
\end{tabular}
\end{center}
%

Note that when several versions of a main file and/or of each child file
are to be generated, it may be convenient to set up a |Makefile| or
shell script to automatise the process.

%%%%%%%%%%%%%%%%%%%%%%%%%%%%%%%%%%%%%%%%%%%%%%%%%%%%%%%%%%%%%%%%%%%%%%%%%%%%%%%%
\subsection{Command Line Processing}
\label{sec:commandline}

The effect of redirection files can also be achieved by invoking
the \LaTeX{} compiler with a more elaborate command line.
Most conveniently this should be done as part
of a shell script or a |Makefile|.

When using \textsf{childdoc} in the main file, the following
command lines effectively perform a redirection
(note that depending on the shell being used,
backslashes may have to be doubled: `|\|' $\to$ `|\\|'):
%
\begin{center}
|... -jobname "|\textit{target}|" |\\|"|[\textit{flags}]%
|\input{childdoc.def}\childdocforward[|\textit{main}|]{|\textit{dest}|}"|
\end{center}
%
Here \textit{target} is the name of the output file,
\textit{main} is the name of the main file
and \textit{dest} is the name of the main or child file to be processed
(all filenames without extensions).
The optional argument \textit{main} can be omitted
if \textit{main} matches \textit{dest}.
Optionally, compilation \textit{flags} can be defined via |\def| commands.
This command line makes the \TeX{} engine believe
it is compiling the file \textit{target}
whose content is specified as the latter parameter.
The provided code then forwards the processing to
\textit{main} or \textit{dest} as described in \secref{sec:forward}.

%%%%%%%%%%%%%%%%%%%%%%%%%%%%%%%%%%%%%%%%%%%%%%%%%%%%%%%%%%%%%%%%%%%%%%%%%%%%%%%%
\subsection{Include by Input}
\label{sec:input}

Including child documents by |\include| has some restrictions by design.
Most notably, the content of a child document always occupies
its own set of pages; pages cannot be shared between child documents.
Usually, this behaviour makes perfect sense
because each child document contain an essential part of the document.
However, in some situations it may be desirable to compose
a document from a collection of parts
without having mandatory page breaks between then.
For this case, the package
provides a mechanism to include parts
by |\input| which can also be processed individually.
However, by construction this mechanism
requires manual handling of the content to be output.

%%%%%%%%%%%%%%%%%%%%%%%%%%%%%%%%%%%%%%%%
\DescribeMacro{\ifchilddocmanual}
The main file should be prepared as usual, see \secref{sec:include}.
However, the document body must make a distinction
between processing of an individual part and of the main document, e.g.:
%
\begin{center}
\begin{tabular}{l}
|\ifchilddocmanual|\\
|\input{\childdocname}|\\
|\||else|\\
\textit{document body with }|\input{|\textit{part}|}|\\
|\||fi|
\end{tabular}
\end{center}
%
The conditional |\ifchilddocmanual| is true whenever
a part to be included by |\input| is being compiled,
and the name of the part is stored in |\childdocname|.

%%%%%%%%%%%%%%%%%%%%%%%%%%%%%%%%%%%%%%%%
\DescribeMacro{\childdocby}
Each part to be included by |\input| should start with:
%
\begin{center}
\begin{tabular}{l}
|\input{childdoc.def}|\\
|\childdocby{|\textit{main}|}|\\
\end{tabular}
\end{center}
%
The directive |\childdocby| is similar to |\childdocof|
described in \secref{sec:include},
but the subsequent selection of content must be done manually.
To that end, both |\ifchilddoc| and |\ifchilddocmanual|
will be true upon processing of a part,
and the name of the part is stored in |\childdocname|.
Note that |\jobname| will be set to the filename of the current part
so that each part receives an individual |.aux| file
that does not interfere with the |.aux| file(s) of the main document.
This behaviour can be altered by the alternative form
|\childdocby[*]{|\textit{main}|}| (with a non-empty optional argument)
which uses the |.aux| file of the main document
by setting |\jobname| to \textit{main}.

%%%%%%%%%%%%%%%%%%%%%%%%%%%%%%%%%%%%%%%%%%%%%%%%%%%%%%%%%%%%%%%%%%%%%%%%%%%%%%%%
\subsection{Driver Development}
\label{sec:driver}

The \textsf{childdoc} mechanism can also be use for the development
of definition files such as \LaTeX{} styles or classes.
This case differs from the above setup with multiple parts
included by |\include| in that no |\includeonly| should be invoked.
This can be achieved by starting the include file
(before |\ProvidesPackage|) with:
%
\begin{center}
\begin{tabular}{l}
|\input{childdoc.def}|\\
|\childdocforward{|\textit{main}|}|\\
\end{tabular}
\end{center}
%
or alternatively with:
%
\begin{center}
\begin{tabular}{l}
|\input{childdoc.def}|\\
|\childdocby{|\textit{main}|}|\\
\end{tabular}
\end{center}
%
Both forms have slightly different effects as described above.
The main file is prepared as usual, see \secref{sec:include}.

%%%%%%%%%%%%%%%%%%%%%%%%%%%%%%%%%%%%%%%%%%%%%%%%%%%%%%%%%%%%%%%%%%%%%%%%%%%%%%%%
\subsection{Legacy Detection}
\label{sec:detection}

The directive |\childdocmain| in the main file can detect
whether the complete document or merely a child is to be compiled
even without using the directive |\childdocof|.
This method is deprecated because it is less robust
and there is no compelling reason to use it;
it is merely provided for backward compatibility
and it may be removed in future versions.

If the detection mechanism is to be used,
it is mandatory to correctly specify
the filename of the main file as the argument of |\childdocmain|:
%
\begin{center}
\begin{tabular}{l}
|\input{childdoc.def}|\\
|\childdocmain{|\textit{main}|}|\\
\end{tabular}
\end{center}
%
If |\jobname| does not match the argument \textit{main} of |\childdocmain|,
it is assumed that |\jobname| points to the child file to be compiled.
When using |\childdocmain| with the main file specified as argument,
it suffices to start a child file
with just |\input{|\textit{main}|}|
without loading of the package and using |\childdocof|.
If instead all processing is done
with the appropriate \textsf{childdoc} directives,
the argument of \textit{main} of |\childdocmain| can be empty.

An alternative version of the command line processing described
in \secref{sec:commandline} using the detection mechanism reads:
%
\begin{center}
|... -jobname "|\textit{target}|" "|[\textit{flags}]%
[|\def\jobname{|\textit{dest}|}|]|\input{|\textit{main}|}"|
\end{center}

%%%%%%%%%%%%%%%%%%%%%%%%%%%%%%%%%%%%%%%%%%%%%%%%%%%%%%%%%%%%%%%%%%%%%%%%%%%%%%%%
\subsection{Manual Code}
\label{sec:manual}

In case one cannot be certain whether the definitions file |childdoc.def|
is installed on the target \TeX{} distribution
and one prefers not to ship it,
it is conceivable to paste a few relevant commands into the sources.

To that end, drop all statements |\input{childdoc.def}|
and perform the replacements as outlined below.
Instead of |\childdocmain{|\textit{main}|}| add the following code
to the top of the main file:
%
\begin{center}
\begin{tabular}{l}
|\||ifdefined\childdocname\endinput\||fi\newif\ifchilddoc|\\
|\edef\childdocname{\scantokens\expandafter{\jobname\noexpand}}|\\
|\def\childdocmain{|\textit{main}|}\||ifx\childdocmain\childdocname\||else|\\
|\childdoctrue\includeonly{\childdocname}\let\jobname\childdocmain\||fi|\\
\end{tabular}
\end{center}
%
Instead of |\childdocof{|\textit{main}|}| just include the main file
at the top of each child file:
%
\begin{center}
|\input{|\textit{main}|}|
\end{center}
%
A simple redirection |\childdocforward{|\textit{dest}|}| is achieved by:
%
\begin{center}
|\def\jobname{|\textit{dest}|}\input{\jobname}|
\end{center}
%
The redirection with prefix
|\childdocforwardprefix[|\textit{prefix}|]{|\textit{dest}|}|
is accomplished by:
%
\begin{center}
\begin{tabular}{l}
|{\edef\jobname{\scantokens\expandafter{\jobname\noexpand}}|\\
|\def\redirectjob |\textit{prefix}|#1~~~{\gdef\jobname{|\textit{dest}|#1}}|\\
|\expandafter\redirectjob\jobname~~~}\input{\jobname}|
\end{tabular}
\end{center}

In an alternative approach,
child documents can be compiled by a specific command line
without additional code or specific definitions:
%
\begin{center}
|... -jobname "|\textit{target}|" "|[\textit{flags}]%
|\includeonly{|\textit{dest}|}\input{|\textit{main}|}"|
\end{center}
%

%%%%%%%%%%%%%%%%%%%%%%%%%%%%%%%%%%%%%%%%%%%%%%%%%%%%%%%%%%%%%%%%%%%%%%%%%%%%%%%%
%%%%%%%%%%%%%%%%%%%%%%%%%%%%%%%%%%%%%%%%%%%%%%%%%%%%%%%%%%%%%%%%%%%%%%%%%%%%%%%%
\section{Information}

%%%%%%%%%%%%%%%%%%%%%%%%%%%%%%%%%%%%%%%%%%%%%%%%%%%%%%%%%%%%%%%%%%%%%%%%%%%%%%%%
\subsection{Copyright}

Copyright \copyright{} 2017--2018 Niklas Beisert

This work may be distributed and/or modified under the
conditions of the \LaTeX{} Project Public License, either version 1.3
of this license or (at your option) any later version.
The latest version of this license is in
  \url{http://www.latex-project.org/lppl.txt}
and version 1.3 or later is part of all distributions of \LaTeX{}
version 2005/12/01 or later.

This work has the LPPL maintenance status `maintained'.

The Current Maintainer of this work is Niklas Beisert.

This work consists of the files |README.txt|, |childdoc.ins| and |childdoc.dtx|
as well as the derived files |childdoc.def|, |cdocsamp.tex|
with |cdocsch1.tex|, |cdocsch2.tex|, |cdocspt3.tex|, |cdocspt4.tex|,
|cdocsdrf.tex|, |cdocsfn1.tex|, |cdocsfn2.tex|
as well as |childdoc.pdf|.

%%%%%%%%%%%%%%%%%%%%%%%%%%%%%%%%%%%%%%%%%%%%%%%%%%%%%%%%%%%%%%%%%%%%%%%%%%%%%%%%
\subsection{Files and Installation}

The package consists of the files:
%
\begin{center}
\begin{tabular}{ll}
    |README.txt|   & readme file \\
    |childdoc.ins| & installation file \\
    |childdoc.dtx| & source file \\
    |childdoc.def| & definition file \\
    |cdocsamp.tex| & sample main file \\
    |cdocsch1.tex| & sample include file \\
    |cdocsch2.tex| & sample include file \\
    |cdocspt3.tex| & sample part file \\
    |cdocspt4.tex| & sample part file \\
    |cdocsdrf.tex| & sample redirection file \\
    |cdocsfn1.tex| & sample redirection file \\
    |cdocsfn2.tex| & sample redirection file \\
    |childdoc.pdf| & manual
\end{tabular}
\end{center}
%
The distribution consists of the files
|README.txt|, |childdoc.ins| and |childdoc.dtx|.
%
\begin{itemize}
\item
Run (pdf)\LaTeX{} on |childdoc.dtx|
to compile the manual |childdoc.pdf| (this file).
\item
Run \LaTeX{} on |childdoc.ins| to create the definitions file |childdoc.def|
and the sample |cdocsamp.tex| with include files
|cdocsch1.tex|, |cdocsch2.tex|, |cdocspt3.tex|, |cdocspt4.tex|,
|cdocsdrf.tex|, |cdocsfn1.tex|, |cdocsfn2.tex|.
Then copy the file |childdoc.def| to an appropriate directory of your \LaTeX{}
distribution, e.g.\ \textit{texmf-root}|/tex/latex/childdoc|.
\end{itemize}

%%%%%%%%%%%%%%%%%%%%%%%%%%%%%%%%%%%%%%%%%%%%%%%%%%%%%%%%%%%%%%%%%%%%%%%%%%%%%%%%
\subsection{Related CTAN Packages}

There are several other packages which offer a similar functionality:
%
\begin{itemize}
\item
The packages
\href{http://ctan.org/pkg/docmute}{\textsf{docmute}},
\href{http://ctan.org/pkg/includex}{\textsf{includex}} and
\href{http://ctan.org/pkg/standalone}{\textsf{standalone}}
provide commands to include only the document body of
a child file thus allowing both files to be compiled individually.
\item
The packages \href{http://ctan.org/pkg/subdocs}{\textsf{subdocs}}
and \href{http://ctan.org/pkg/subfiles}{\textsf{subfiles}}
provide structures in which the main and child documents can be
encapsulated and allowing them to be compiled individually.
The inclusion mechanism is different from the conventional |\include|.
\item
The package \href{http://ctan.org/pkg/combine}{\textsf{combine}}
is an elaborate solution to combine several documents into one.
\end{itemize}
%
See also the CTAN topic \href{http://ctan.org/topic/subdocs}{\textsf{subdocs}}
for further related packages.
The present package differs from the above solutions in that
a document structure constructed with the conventional |\include| mechanism
just needs two extra commands at the top of every file
such that all constituent files can be compiled individually.

%%%%%%%%%%%%%%%%%%%%%%%%%%%%%%%%%%%%%%%%%%%%%%%%%%%%%%%%%%%%%%%%%%%%%%%%%%%%%%%%
%\subsection{Feature Suggestions}
%
%The following is a list of features which may be useful for future
%versions of this package:
%%
%\begin{itemize}
%\item
%\ldots
%\end{itemize}

%%%%%%%%%%%%%%%%%%%%%%%%%%%%%%%%%%%%%%%%%%%%%%%%%%%%%%%%%%%%%%%%%%%%%%%%%%%%%%%%
\subsection{Revision History}

%%%%%%%%%%%%%%%%%%%%%%%%%%%%%%%%%%%%%%%%
\paragraph{v2.0:} 2018/12/30

\begin{itemize}
\item
immediate forward processing
\item
added |\childdocby| mechanism
\item
manual restructured
\end{itemize}

%%%%%%%%%%%%%%%%%%%%%%%%%%%%%%%%%%%%%%%%
\paragraph{v1.6:} 2018/01/17

\begin{itemize}
\item
application for development of include files
\item
corrections to manual
\end{itemize}

%%%%%%%%%%%%%%%%%%%%%%%%%%%%%%%%%%%%%%%%
\paragraph{v1.5:} 2017/05/21

\begin{itemize}
\item
more complete structuring introduced
\item
|\childdocof| introduced
\item
|\childdoc| renamed to |\childdocmain|
\item
|\childredirect| renamed to |\childdocforward| and |\childdocforwardprefix|
and functionality expanded
\end{itemize}

%%%%%%%%%%%%%%%%%%%%%%%%%%%%%%%%%%%%%%%%
\paragraph{v1.0:} 2017/04/27

\begin{itemize}
\item
manual and install package
\item
first version published on CTAN
\end{itemize}

%%%%%%%%%%%%%%%%%%%%%%%%%%%%%%%%%%%%%%%%
\paragraph{v0.6:} 2017/04/26

\begin{itemize}
\item
redirection mechanism added
\end{itemize}

%%%%%%%%%%%%%%%%%%%%%%%%%%%%%%%%%%%%%%%%
\paragraph{v0.5:} 2017/04/26

\begin{itemize}
\item
functionality in definition file
\end{itemize}


%%%%%%%%%%%%%%%%%%%%%%%%%%%%%%%%%%%%%%%%%%%%%%%%%%%%%%%%%%%%%%%%%%%%%%%%%%%%%%%%
%%%%%%%%%%%%%%%%%%%%%%%%%%%%%%%%%%%%%%%%%%%%%%%%%%%%%%%%%%%%%%%%%%%%%%%%%%%%%%%%
%%%%%%%%%%%%%%%%%%%%%%%%%%%%%%%%%%%%%%%%%%%%%%%%%%%%%%%%%%%%%%%%%%%%%%%%%%%%%%%%
\appendix

\settowidth\MacroIndent{\rmfamily\scriptsize 000\ }

 \DocInput{childdoc.dtx}

\end{document}
%</driver>
% \fi
%
% %%%%%%%%%%%%%%%%%%%%%%%%%%%%%%%%%%%%%%%%%%%%%%%%%%%%%%%%%%%%%%%%%%%%%%%%%%%%%%
% %%%%%%%%%%%%%%%%%%%%%%%%%%%%%%%%%%%%%%%%%%%%%%%%%%%%%%%%%%%%%%%%%%%%%%%%%%%%%%
% \section{Sample}
%\iffalse
%<*samplemain>
%\fi
%
% The following presents a sample document
% with two chapters, two parts, a title page,
% a compile flag as well as three forwarding files to set the flag.
% It consists of eight |.tex| files:
% \begin{center}
% \begin{tabular}{ll}
% |cdocsamp.tex|&main file\\
% |cdocsch1.tex|&include file for chapter 1\\
% |cdocsch2.tex|&include file for chapter 2\\
% |cdocspt3.tex|&include file for part 3\\
% |cdocspt4.tex|&include file for part 4\\
% |cdocsdrf.tex|&forwarding file for main file in draft mode\\
% |cdocsfi1.tex|&forwarding file for final version of chapter 1\\
% |cdocsfi2.tex|&forwarding file for final version of chapter 2\\
% \end{tabular}
% \end{center}
% Each of the eight files can be compiled directly by the \LaTeX{} compiler.
%
% %%%%%%%%%%%%%%%%%%%%%%%%%%%%%%%%%%%%%%
% \paragraph{Main File.}
%
% The main file is called |cdocsamp.tex|.
%
% Load the \textsf{childdoc} definitions and
% declare the filename for the main document:
%    \begin{macrocode}
\input{childdoc.def}
\childdocmain{}
%    \end{macrocode}

% Optional override for |\version| flag:
%    \begin{macrocode}
%%\ifchilddoc\else\providecommand{\version}{draft}\fi
%    \end{macrocode}

% Define the default values for the |\version| flag
% (|final| for the main file and |draft| for childs):
%    \begin{macrocode}
\ifchilddoc
\providecommand{\version}{draft}
\else
\providecommand{\version}{final}
\fi
%    \end{macrocode}

% Load the standard document class:
%    \begin{macrocode}
\documentclass[12pt]{article}
%    \end{macrocode}

% Start the document body:
%    \begin{macrocode}
\begin{document}
%    \end{macrocode}

% Declare a title page.
% Print title, part of document being processed and version flag:
%    \begin{macrocode}
\addtocounter{page}{-1}
\begin{center}
{\LARGE\bfseries{}childdoc example\par}
\vspace{1cm}
\ifchilddoc
\ifchilddocmanual part\else chapter\fi:
`\childdocname' of `\childdocjob'\par
\else
main document: `\childdocjob'\par
\fi
version: \version\par
\end{center}
\newpage
%    \end{macrocode}

% Manually include selected file,
% otherwise process as usual:
%    \begin{macrocode}
\ifchilddocmanual
\section*{part `\childdocname'}
\input{\childdocname}
\else
%    \end{macrocode}

% Include the two chapters:
%    \begin{macrocode}
\include{cdocsch1}
\include{cdocsch2}
%    \end{macrocode}

% Include the two parts unless only chapters should be displayed:
%    \begin{macrocode}
\ifchilddoc\else
\section{part three}
\input{cdocspt3}
\section{part four}
\input{cdocspt4}
\fi
%    \end{macrocode}

% Process as usual until here:
%    \begin{macrocode}
\fi
%    \end{macrocode}

% End of document body:
%    \begin{macrocode}
\end{document}
%    \end{macrocode}
%\iffalse
%</samplemain>
%\fi
%
% %%%%%%%%%%%%%%%%%%%%%%%%%%%%%%%%%%%%%%
% \paragraph{Chapter Include Files.}
%
% The include files are called |cdocsch1.tex| and |cdocsch2.tex|.
%
%\iffalse
%<*samplechap1|samplechap2>
%\fi

% Optional override for |\version| flag:
%    \begin{macrocode}
%%\providecommand{\version}{final}
%    \end{macrocode}

% Include the main document:
%    \begin{macrocode}
\input{childdoc.def}
\childdocof{cdocsamp}
%    \end{macrocode}

%\iffalse
%</samplechap1|samplechap2>
%\fi
%
%\iffalse
%<*samplechap1>
%\fi
% Some text for chapter 1:
%    \begin{macrocode}
\section{one}
some text in chapter one
%    \end{macrocode}

%\iffalse
%</samplechap1>
%\fi
% Some text for chapter 2:
%\iffalse
%<*samplechap2>
%\fi
%    \begin{macrocode}
\section{two}
more text in chapter two
%    \end{macrocode}

%\iffalse
%</samplechap2>
%\fi
%
% %%%%%%%%%%%%%%%%%%%%%%%%%%%%%%%%%%%%%%
% \paragraph{Part Include Files.}
%
% The include files are called |cdocspt3.tex| and |cdocspt4.tex|.
%
%\iffalse
%<*samplepart3|samplepart4>
%\fi

% Optional override for |\version| flag:
%    \begin{macrocode}
%%\providecommand{\version}{final}
%    \end{macrocode}

% Include the main document:
%    \begin{macrocode}
\input{childdoc.def}
\childdocby{cdocsamp}
%    \end{macrocode}

%\iffalse
%</samplepart3|samplepart4>
%\fi
%
%\iffalse
%<*samplepart3>
%\fi
% Some text for part 3:
%    \begin{macrocode}
some text in part three
%    \end{macrocode}

%\iffalse
%</samplepart3>
%\fi
% Some text for part 4:
%\iffalse
%<*samplepart4>
%\fi
%    \begin{macrocode}
more text in part four
%    \end{macrocode}

%\iffalse
%</samplepart4>
%\fi
%
% %%%%%%%%%%%%%%%%%%%%%%%%%%%%%%%%%%%%%%
% \paragraph{Forwarding for a Complete Draft.}
%
% The following forwarding file |cdocsdrf.tex|
% compiles the main document in draft mode:
%\iffalse
%<*sampledraft>
%\fi
%    \begin{macrocode}
\def\version{draft}
\input{childdoc.def}
\childdocforward{cdocsamp}
%    \end{macrocode}

%\iffalse
%</sampledraft>
%\fi
%
% %%%%%%%%%%%%%%%%%%%%%%%%%%%%%%%%%%%%%%
% \paragraph{Forwarding for Final Version of the Chapters.}
%
% The following forwarding files |cdocsfn1.tex| and |cdocsfn2.tex|
% (with identical content)
% compile the final versions of the child documents
% |cdocsch1.tex| and |cdocsch2.tex|, respectively:
%\iffalse
%<*samplefinal>
%\fi
%    \begin{macrocode}
\def\version{final}
\input{childdoc.def}
\childdocforwardprefix[cdocsamp]{cdocsfn}{cdocsch}
%    \end{macrocode}

%\iffalse
%</samplefinal>
%\fi
%
% %%%%%%%%%%%%%%%%%%%%%%%%%%%%%%%%%%%%%%
% \paragraph{Command Line Processing.}
%
% The following three command lines generate the output files
% |cdocscld|, |cdocscl1| and |cdocscl2|
% which should be identical to
% |cdocsdrf|, |cdocsch1| and |cdocsfn2|, respectively:
% \begin{center}
% \begin{tabular}{l}
% |latex -jobname cdocscld \|\\
% |  "\def\version{draft}\input{childdoc.def}\childdocforward{cdocsamp}"|\\
% |latex -jobname cdocscl1 \|\\
% |  "\input{childdoc.def}\childdocforward[cdocsamp]{cdocsch1}"|\\
% |latex -jobname cdocscl2 \|\\
% |  "\def\version{final}\input{childdoc.def}\childdocforward{cdocsch2}"|
% \end{tabular}
% \end{center}
% Note that the trailing backslash on each first line
% merely continues the input to the second line
% (for convenient cut ant paste).
% Furthermore, the command |latex| can be replaced by any
% of its alternative versions such as |pdflatex|.
%
% %%%%%%%%%%%%%%%%%%%%%%%%%%%%%%%%%%%%%%%%%%%%%%%%%%%%%%%%%%%%%%%%%%%%%%%%%%%%%%
% %%%%%%%%%%%%%%%%%%%%%%%%%%%%%%%%%%%%%%%%%%%%%%%%%%%%%%%%%%%%%%%%%%%%%%%%%%%%%%
% \section{Implementation}
%\iffalse
%<*package>
%\fi
%
% This section describes the definitions file |childdoc.def|.

% The definitions cannot be loaded using |\usepackage| or |\RequirePackage|
% which has a mechanism to prevent loading a style file more than once.
% When loading the definitions by means of |\input|
% multiple instances have to be prevented manually:
%\iffalse
%This code needs to be before the `\ProvidesFile' directive
%which is defined at the beginning of this file.
%Therefore it is also placed there and commented out here.
%</package>
%<*discard>
%\fi
%    \begin{macrocode}
\ifdefined\childdocmain\endinput\fi
%    \end{macrocode}
%\iffalse
%</discard>
%<*package>
%\fi
%
% \macro{\ifchilddoc}
% \macro{\ifchilddocmanual}
% The conditional |\ifchilddoc| tells whether a
% child (true) or main (false) document is being compiled.
% The conditional |\ifchilddocmanual| tells whether
% the |\includeonly| mechanism is used (false) or
% the selection of child files must be performed manually (true).
% The definitions initialise to false:
%    \begin{macrocode}
\newif\ifchilddoc
\newif\ifchilddocmanual
%    \end{macrocode}

% \macro{\childdocname}
% \macro{\childdocjob}
% The macro |\childdocname| stores the name of the main document
% to be compiled. The macro |\childdocjob| stores the name of
% the document on which the \LaTeX{} compiler was originally invoked.
% The content of |\jobname| cannot be compared
% to filenames specified in the source due to different catcodes.
% The following code rescans |\jobname|, stores the result
% in |\childdocname| and saves a copy in |\childdocjob|:
%    \begin{macrocode}
\edef\childdocname{\scantokens\expandafter{\jobname\noexpand}}
\let\childdocjob\childdocname
%    \end{macrocode}

% \macro{\childdocdisable}
% The macro |\childdocdisable| prevents the main file
% from being processed more than once.
% At this stage, the main document command |\childdocmain|
% is assumed to be called once again where it should do nothing.
% Any subsequent call to it should prevent
% a secondary processing of the main document
% It overwrites the forwarding commands
% |\childdocof| and |\childdocforward|
% with empty macros to prevent further inclusions of the main document:
%    \begin{macrocode}
\newcommand{\childdocdisable}
{
  \renewcommand{\childdocmain}[1]{\renewcommand{\childdocmain}[1]{\endinput}}
  \renewcommand{\childdocof}[1]{}
  \renewcommand{\childdocby}[2][]{}
  \renewcommand{\childdocforward}[2][]{}
  \renewcommand{\childdocdisable}{}
}
%    \end{macrocode}

% \macro{\childdocmain}
% The macro |\childdocmain| is to be called at the top of the main file
% with nothing or the main filename (without extension) as argument.
% First, it breaks loops.
% If the argument is not empty and does not match |\childdocname|
% (which is set by the first inclusion of |childdoc.def|),
% |\ifchilddoc| is set to true, |\includeonly| is applied to the child file
% and |\jobname| is set to the main file
% (for proper handling of |.aux| files):
%    \begin{macrocode}
\newcommand{\childdocmain}[1]
{
  \childdocdisable\childdocmain{}
  \if?#1?\else
    \begingroup
      \def\childdoctmp{#1}
      \ifx\childdoctmp\childdocname
        \def\childdoctmp{}
      \else
        \def\childdoctmp
        {
          \childdoctrue
          \includeonly{\childdocname}
          \def\childdocjob{#1}
          \def\jobname{#1}
        }
      \fi
      \expandafter
    \endgroup
    \childdoctmp
  \fi
}
%    \end{macrocode}

% \macro{\childdocof}
% The command |\childdocof| redirects
% compilation to the main file |#1|.
%    \begin{macrocode}
\newcommand{\childdocof}[1]
{
  \childdocdisable
  \childdoctrue
  \includeonly{\childdocname}
  \def\jobname{#1}
  \def\childdocjob{#1}
  \input{#1}
}
%    \end{macrocode}

% \macro{\childdocby}
% The command |\childdocby| ....
%    \begin{macrocode}
\newcommand{\childdocby}[2][]
{
  \childdocdisable
  \childdoctrue
  \childdocmanualtrue
  \if?#1?\else
    \def\jobname{#2}
  \fi
  \def\childdocjob{#2}
  \input{#2}
  \endinput
}
%    \end{macrocode}

% \macro{\childdocforward}
% The command |\childdocforward| redirects
% compilation to the main file or
% (if the optional argument is given) a child file.
% Parameters are set as if the main file
% or a child file starting with |\childdocof| was compiled.
% Then compilation is handed over to the main file:
%    \begin{macrocode}
\newcommand{\childdocforward}[2][]
{
  \begingroup
    \if?#1?
      \def\childdoctmp
      {
        \def\childdocname{#2}
        \def\childdocjob{#2}
        \def\jobname{#2}
        \input{#2}
        \endinput
      }
    \else
      \def\childdoctmp
      {
        \childdocdisable
        \def\childdocname{#2}
        \childdoctrue
        \includeonly{#2}
        \def\childdocjob{#1}
        \def\jobname{#1}
        \input{#1}
        \endinput
      }
    \fi
    \expandafter
  \endgroup
  \childdoctmp
}
%    \end{macrocode}

% \macro{\childdocforwardprefix}
% The command |\childdocforwardprefix| redirects
% compilation to the main or a child file by means of a pattern.
% The prefix |#1| in the current filename is replaced by |#2|
% and the suffix of the current filename is kept
% (it is assumed that the filename does not contain the substring `|~~~|'
% which is used as a delimiter).
% Compilation is handed over to the new file by |\childdocforward|:
%    \begin{macrocode}
\newcommand{\childdocforwardprefix}[3][]
{
  \begingroup
    \def\childdocextract #2##1~~~{\def\childdoctmp{\childdocforward[#1]{#3##1}}}
    \expandafter\childdocextract\childdocname~~~
    \expandafter
  \endgroup
  \childdoctmp
}
%    \end{macrocode}

% \macro{\childdoc}
% The deprecated macro |\childdoc| is a legacy version of |\childdocmain|:
%    \begin{macrocode}
\newcommand{\childdoc}{\childdocmain}
%    \end{macrocode}

% \macro{\childdocredirect}
% The deprecated macro |\childdocredirect| is a legacy version
% of |\childdocforward| and |\childdocforwardprefix|:
%    \begin{macrocode}
\newcommand{\childdocredirect}[2][]
{
  \begingroup
    \if?#1?
      \def\childdoctmp{\childdocforward{#2}}
    \else
      \def\childdoctmp{\childdocforwardprefix{#1}{#2}}
    \fi
    \expandafter
  \endgroup
  \childdoctmp
}
%    \end{macrocode}

%\iffalse
%</package>
%\fi
%
\endinput
|\\
|\childdocmain{|\textit{main}|}|\\
\end{tabular}
\end{center}
%
If |\jobname| does not match the argument \textit{main} of |\childdocmain|,
it is assumed that |\jobname| points to the child file to be compiled.
When using |\childdocmain| with the main file specified as argument,
it suffices to start a child file
with just |\input{|\textit{main}|}|
without loading of the package and using |\childdocof|.
If instead all processing is done
with the appropriate \textsf{childdoc} directives,
the argument of \textit{main} of |\childdocmain| can be empty.

An alternative version of the command line processing described
in \secref{sec:commandline} using the detection mechanism reads:
%
\begin{center}
|... -jobname "|\textit{target}|" "|[\textit{flags}]%
[|\def\jobname{|\textit{dest}|}|]|\input{|\textit{main}|}"|
\end{center}

%%%%%%%%%%%%%%%%%%%%%%%%%%%%%%%%%%%%%%%%%%%%%%%%%%%%%%%%%%%%%%%%%%%%%%%%%%%%%%%%
\subsection{Manual Code}
\label{sec:manual}

In case one cannot be certain whether the definitions file |childdoc.def|
is installed on the target \TeX{} distribution
and one prefers not to ship it,
it is conceivable to paste a few relevant commands into the sources.

To that end, drop all statements |% \iffalse
%
% childdoc.dtx Copyright (C) 2017-2018 Niklas Beisert
%
% This work may be distributed and/or modified under the
% conditions of the LaTeX Project Public License, either version 1.3
% of this license or (at your option) any later version.
% The latest version of this license is in
%   http://www.latex-project.org/lppl.txt
% and version 1.3 or later is part of all distributions of LaTeX
% version 2005/12/01 or later.
%
% This work has the LPPL maintenance status `maintained'.
%
% The Current Maintainer of this work is Niklas Beisert.
%
% This work consists of the files childdoc.dtx and childdoc.ins
% and the derived files childdoc.def and cdocsamp.tex with
% cdocsch1.tex, cdocsch2.tex, cdocsdrf.tex, cdocsfn1.tex, cdocsfn2.tex.
%
%<package>\ifdefined\childdocmain\endinput\fi
%<package>\ProvidesFile{childdoc.def}[2018/12/30 v2.0 child document driver]
%<samplemain>\ProvidesFile{cdocsamp.tex}[2018/12/30 v2.0 sample for childdoc]
%<*driver>
%\ProvidesFile{childdoc.drv}[2018/12/30 v2.0 childdoc reference manual file]
\PassOptionsToClass{10pt,a4paper}{article}
\documentclass{ltxdoc}

\usepackage[margin=35mm]{geometry}
\usepackage{hyperref}
\usepackage{hyperxmp}
\usepackage[usenames]{color}

\hypersetup{colorlinks=true}
\hypersetup{pdfstartview=FitH}
\hypersetup{pdfpagemode=UseNone}
\hypersetup{pdfsource={}}
\hypersetup{pdflang={en-UK}}
\hypersetup{pdfcopyright={Copyright 2017-2018 Niklas Beisert.
  This work may be distributed and/or modified under the
  conditions of the LaTeX Project Public License, either version 1.3
  of this license or (at your option) any later version.}}
\hypersetup{pdflicenseurl={http://www.latex-project.org/lppl.txt}}
\hypersetup{pdfcontactaddress={ETH Zurich, ITP, HIT K,
  Wolfgang-Pauli-Strasse 27}}
\hypersetup{pdfcontactpostcode={8093}}
\hypersetup{pdfcontactcity={Zurich}}
\hypersetup{pdfcontactcountry={Switzerland}}
\hypersetup{pdfcontactemail={nbeisert@itp.phys.ethz.ch}}
\hypersetup{pdfcontacturl={http://people.phys.ethz.ch/\xmptilde nbeisert/}}

\newcommand{\secref}[1]{\hyperref[#1]{section \ref*{#1}}}

\parskip1ex
\parindent0pt
\let\olditemize\itemize
\def\itemize{\olditemize\parskip0pt}

\begin{document}

\title{The \textsf{childdoc} Package}
\hypersetup{pdftitle={The childdoc Package}}
\author{Niklas Beisert\\[2ex]
  Institut f\"ur Theoretische Physik\\
  Eidgen\"ossische Technische Hochschule Z\"urich\\
  Wolfgang-Pauli-Strasse 27, 8093 Z\"urich, Switzerland\\[1ex]
  \href{mailto:nbeisert@itp.phys.ethz.ch}
  {\texttt{nbeisert@itp.phys.ethz.ch}}}
\hypersetup{pdfauthor={Niklas Beisert}}
\hypersetup{pdfsubject={Manual for the LaTeX2e Package childdoc}}
\date{30 December 2018, \textsf{v2.0}}
\maketitle

\begin{abstract}\noindent
\textsf{childdoc} is a \LaTeXe{} package
that enables the direct compilation
of document sections included by |\include|
to individual files.
\end{abstract}

\begingroup
\parskip0ex
\tableofcontents
\endgroup

%%%%%%%%%%%%%%%%%%%%%%%%%%%%%%%%%%%%%%%%%%%%%%%%%%%%%%%%%%%%%%%%%%%%%%%%%%%%%%%%
%%%%%%%%%%%%%%%%%%%%%%%%%%%%%%%%%%%%%%%%%%%%%%%%%%%%%%%%%%%%%%%%%%%%%%%%%%%%%%%%
\section{Introduction}

\LaTeX{} provides a mechanism to structure a large document (such as a book)
into a main file and several child files (containing the chapters)
using the |\include| command.
This mechanism is beneficial for documents
which span hundreds of pages in order to
make the source file(s) more manageable.
Moreover, compilation can be restricted to
selected child files by means of the |\includeonly| command.
The latter feature can be used to reduce the compilation time while editing
(this was significantly more useful in the earlier days of \LaTeX{})
or to generate a smaller document which is easier to navigate.
Another application of |\includeonly| is to generate
documents consisting of selected parts of the complete document.

However, there are a few drawbacks of the plain |\include| mechanism:
\begin{itemize}
\item
The child files cannot be compiled on their own,
they can only be compiled via the main file.
A naive editing environment
(such as a text editor with an option
to have the current file processed by \LaTeX)
may require one to switch to the main file before compiling;
attempting to compile the child file produces errors.
\item
The main file must be modified (each time)
to adjust the |\includeonly| command
to the present needs. This easily leaves the main file in a messy state.
\item
The generated document will always carry the filename
of the main document. This is inconvenient if
several child files are to be compiled and
to be kept for distribution.
\end{itemize}

The present package provides a simple interface
to make child files individually compilable by \LaTeX{}.
Compiling a child file then has the same effect as compiling
the main file with an |\includeonly| command
to select the appropriate child.
Moreover the generated document will carry the name of the child
rather than the main file.
This resolves all three above issues.

This feature is meant to make the editing of books,
thesis documents and lecture notes somewhat more convenient.
However, the package can also be used efficiently for
composing a series of documents (such as exercise sheets)
which are typically distributed individually.
It then assists the author in generating the individual documents
(potentially in different versions)
as well as a document containing the collected series.
Another application is in developing style files
or other kinds of included material
where compilation of the style file could redirect
to a sample or test file.

%%%%%%%%%%%%%%%%%%%%%%%%%%%%%%%%%%%%%%%%%%%%%%%%%%%%%%%%%%%%%%%%%%%%%%%%%%%%%%%%
%%%%%%%%%%%%%%%%%%%%%%%%%%%%%%%%%%%%%%%%%%%%%%%%%%%%%%%%%%%%%%%%%%%%%%%%%%%%%%%%
\section{Usage}

First of all, the package \textsf{childdoc} is \emph{not} a standard
\LaTeXe{} |.sty| style file! Therefore it needs to be invoked in
a non-standard way.

%%%%%%%%%%%%%%%%%%%%%%%%%%%%%%%%%%%%%%%%%%%%%%%%%%%%%%%%%%%%%%%%%%%%%%%%%%%%%%%%
\subsection{Included Files}
\label{sec:include}

%%%%%%%%%%%%%%%%%%%%%%%%%%%%%%%%%%%%%%%%
\DescribeMacro{\childdocmain}
To use the package, add the commands
\begin{center}
\begin{tabular}{l}
|\input{childdoc.def}|\\
|\childdocmain{}|\\
\end{tabular}
\end{center}
at the very top of the main \LaTeX{} file,
in particular \emph{before} the |\documentclass| statement!
The argument of |\childdocmain| should be left empty
(but it must be present).

%%%%%%%%%%%%%%%%%%%%%%%%%%%%%%%%%%%%%%%%
\DescribeMacro{\childdocof}
Furthermore, add the commands
\begin{center}
\begin{tabular}{l}
|\input{childdoc.def}|\\
|\childdocof{|\textit{main}|}|\\
\end{tabular}
\end{center}
at the top of every child file \textit{child}
which is included by |\include{|\textit{child}|}|
from within the main file
(or at least for those files to be compiled individually).
The argument \textit{main} must be the filename of the main file.

There are a couple of
considerations in setting up the main and child documents:

%%%%%%%%%%%%%%%%%%%%%%%%%%%%%%%%%%%%%%%%
\paragraph{Restrictions.}

Please note the following restrictions:
\begin{itemize}
\item
|\childdocmain| must be called with one argument \textit{main}
to ensure compatibility with earlier version of the package.
It must either be empty (|\childdocmain{}|)
or precisely match the filename of the main file in which it is specified.
See \secref{sec:detection} for further information.
\item
The filename \textit{main} must be specified without the |.tex| extension.
\item
The filename \textit{main} is case sensitive
(even in case-insensitive file systems)
due to internal string comparison.
\item
The argument \textit{main} should be fully expanded, it cannot be a macro.
\item
Subdirectories and special characters should be avoided in filenames.
\item
The command |\childdocmain{|\textit{main}|}| must be followed by a whitespace.
It should not be followed immediately by another command
or by a comment mark `|%|'.
This is because the \TeX{} parser reads the token immediately following
the argument of |\childdocmain| and puts it
at the beginning of every child section;
however, a white\-space is ignored.
\end{itemize}

%%%%%%%%%%%%%%%%%%%%%%%%%%%%%%%%%%%%%%%%
\paragraph{Content of Main File.}

It is advisable to place all content in the child files included by |\include|.
Any output contained in the main file will appear in all child documents
unless suppressed manually;
it cannot be suppressed automatically by the |\includeonly| directive
and thus should normally be avoided.
A method to include some content in the main file
by means of conditional processing is described in \secref{sec:conditional}.

%%%%%%%%%%%%%%%%%%%%%%%%%%%%%%%%%%%%%%%%
\paragraph{Page Numbering.}

When only a part of the document is compiled,
the appropriate numbering of pages
(as well as other status parameters)
is determined from the |.aux| files.
The latter contain information from previous passes.
However this information needs to propagate through
all intermediate child documents.
Therefore the page numbering in child documents may well
be inconsistent until the complete document is compiled at least once.

A useful (if unconventional) way to always ensure a consistent
page numbering is to restart the numbering in each child document
and denote the pages by `\textit{child}|.|\textit{page}'
where \textit{child} represents the chapter/section number of the child file.
This can be achieved by the command
|\numberwithin{page}{|\textit{child}|}|
of the \textsf{amsmath} package
where \textit{child} can be |chapter| or |section|
depending on the chosen structuring.
Alternatively, one can modify the macro |\thepage| appropriately
and reset the counter |page| at the start of each child file.

%%%%%%%%%%%%%%%%%%%%%%%%%%%%%%%%%%%%%%%%%%%%%%%%%%%%%%%%%%%%%%%%%%%%%%%%%%%%%%%%
\subsection{Conditional Processing}
\label{sec:conditional}

The package provides a mechanism to compile different versions
of a document. To customise the versions further some conditional processing
can come in handy to distinguish which version is being compiled.
The package provides two macros to describe the compilation context:

%%%%%%%%%%%%%%%%%%%%%%%%%%%%%%%%%%%%%%%%
\DescribeMacro{\ifchilddoc}
The conditional |\ifchilddoc| distinguishes between the compilation of
child documents and the main document:
%
\begin{center}
|\ifchilddoc |\textit{child-code}| |[|\||else |\textit{main-code}]| \||fi|
\end{center}

%%%%%%%%%%%%%%%%%%%%%%%%%%%%%%%%%%%%%%%%
\DescribeMacro{\childdocname}
\DescribeMacro{\childdocjob}
The macro |\childdocname| contains the filename (without extension)
of the main or child file being processed.
Note that |\childdocjob| will always contain the name of the main file.

%%%%%%%%%%%%%%%%%%%%%%%%%%%%%%%%%%%%%%%%
\paragraph{Title Page.}

Conditional processing can be used to include a title or banner page
in the main document when proper precautions are taken.
Importantly, the code in the main file should ensure that the page counter
(as well as other status parameters which are stored in the |.aux| files)
takes the same value after the conditional processing.
Otherwise the page numbers may take divergent values
depending on which part is compiled.

For example, a title page could be declared by:
%
\begin{center}
\begin{tabular}{l}
|\ifchilddoc\||else|\\
|\addtocounter{page}{-1}|\\
\textit{code for title page}\\
|\newpage|\\
|\||fi|
\end{tabular}
\end{center}
%
A banner page for the child documents can be generated by:
%
\begin{center}
\begin{tabular}{l}
|\ifchilddoc|\\
|\addtocounter{page}{-1}|\\
\textit{code for banner page}\\
|\newpage|\\
|\||fi|
\end{tabular}
\end{center}
%
Here one could write a message such as:
\begin{center}
|This is the part \childdocname{} of \childdocjob{}.|
\end{center}

%%%%%%%%%%%%%%%%%%%%%%%%%%%%%%%%%%%%%%%%%%%%%%%%%%%%%%%%%%%%%%%%%%%%%%%%%%%%%%%%
\subsection{Flags}
\label{sec:flags}

The package makes it easy to generate different versions
of the main or child documents.
To this end compilation flags can be defined
and assigned different default values.
They will be particularly useful in conjunction
with the forwarding mechanism described in \secref{sec:forward}.

For example, it may be useful to have a flag |\version|
which can be set to |draft| or |final|.
The document source will contain some conditional code
depending on the value of |\version|.
Suppose further, the flag should default to |final| for the main file
and to |draft| for child files
which is a natural assignment for editing the document.
This is achieved by placing the following code
in the preamble of the main document
(below the |\childdocmain| directive):
%
\begin{center}
\begin{tabular}{l}
|\ifchilddoc|\\
|\providecommand{\version}{draft}|\\
|\||else|\\
|\providecommand{\version}{final}|\\
|\||fi|
\end{tabular}
\end{center}
%
The definition by |\providecommand| makes sure
that previous definitions are not overwritten.
Further statements |\providecommand{\version}{...}|
can thus be added before the above code to override it.

For the main file, one might add a line
(between |\childdocmain| and the above block)
%
\begin{center}
|%\ifchilddoc\||else\providecommand{\version}{draft}\||fi|
\end{center}
%
which can be uncommented to produce a draft version.
Likewise one can add a line to the very top of a child file
(above the |\childdocof{|\textit{main}|}| directive)
%
\begin{center}
|%\providecommand{\version}{final}|
\end{center}
%
which can be uncommented to produce the final version of this child document.

%%%%%%%%%%%%%%%%%%%%%%%%%%%%%%%%%%%%%%%%%%%%%%%%%%%%%%%%%%%%%%%%%%%%%%%%%%%%%%%%
\subsection{Forwarding}
\label{sec:forward}

Different versions of the main or child documents
using compilation flags as described in \secref{sec:flags}
can be (permanently) stored in different files
for convenient compilation, viewing and distribution.
To this end, the package defines a command
to pass on compilation to a different file:

%%%%%%%%%%%%%%%%%%%%%%%%%%%%%%%%%%%%%%%%
\DescribeMacro{\childdocforward}
The command |\childdocforward| redirects processing to
another source file:
%
\begin{center}
\begin{tabular}{l}
|\input{childdoc.def}|\\
|\childdocforward[|\textit{main}|]{|\textit{dest}|}|\\
\end{tabular}
\end{center}
%
The argument \textit{dest} is the destination file
(without extension).
It should be the main file or one of the child files.
Note that further \textsf{childdoc} directives
such as |\childdocof| and |\childdocforward|
in the indicated file will be processed in this form.
The optional argument \textit{main}
passes on directly to the main file \textit{main}
while pretending to compile the child \textit{dest}.
This form behaves as if \textit{dest}
issues |\childdocof{|\textit{main}|}| right away,
and no further \textsf{childdoc} directives will be processed.

%%%%%%%%%%%%%%%%%%%%%%%%%%%%%%%%%%%%%%%%
\DescribeMacro{\...prefix}
In the alternative form |\childdocforwardprefix|,
%
\begin{center}
\begin{tabular}{l}
|\input{childdoc.def}|\\
|\childdocforwardprefix[|\textit{main}|]{|\textit{prefix}|}{|\textit{dest}|}|
\end{tabular}
\end{center}
%
the destination file is determined by a pattern
depending on the current file:
To make this work, the current file must be called
`{\textit{prefix}\hspace{0.2em}\textit{suffix}}'
with \textit{prefix} matching precisely the argument.
Processing is then passed on to the file
`{\textit{dest}\hspace{0.2em}\textit{suffix}}'.
Surely, the same effect is achieved by
directly specifying the
argument `{\textit{dest}\hspace{0.2em}\textit{suffix}}'
in the first form.
However, that requires to set up a different file
for each child. With the alternative form of the command
all these files can have exactly the same content
which simplifies setting them up and maintaining them.

For example, the following file |draft.tex|
with a compilation flag |\version| as described in \secref{sec:flags}
compiles the main document as a draft:
%
\begin{center}
\begin{tabular}{l}
|\def\version{draft}|\\
|\input{childdoc.def}|\\
|\childdocforward{|\textit{main}|}|
\end{tabular}
\end{center}
%
Likewise, the following files |final|\textit{nn}|.tex|
compile the final version of the child document
|child|\textit{nn}|.tex|:
%
\begin{center}
\begin{tabular}{l}
|\def\version{final}|\\
|\input{childdoc.def}|\\
|\childdocforwardprefix{final}{child}|
\end{tabular}
\end{center}
%

Note that when several versions of a main file and/or of each child file
are to be generated, it may be convenient to set up a |Makefile| or
shell script to automatise the process.

%%%%%%%%%%%%%%%%%%%%%%%%%%%%%%%%%%%%%%%%%%%%%%%%%%%%%%%%%%%%%%%%%%%%%%%%%%%%%%%%
\subsection{Command Line Processing}
\label{sec:commandline}

The effect of redirection files can also be achieved by invoking
the \LaTeX{} compiler with a more elaborate command line.
Most conveniently this should be done as part
of a shell script or a |Makefile|.

When using \textsf{childdoc} in the main file, the following
command lines effectively perform a redirection
(note that depending on the shell being used,
backslashes may have to be doubled: `|\|' $\to$ `|\\|'):
%
\begin{center}
|... -jobname "|\textit{target}|" |\\|"|[\textit{flags}]%
|\input{childdoc.def}\childdocforward[|\textit{main}|]{|\textit{dest}|}"|
\end{center}
%
Here \textit{target} is the name of the output file,
\textit{main} is the name of the main file
and \textit{dest} is the name of the main or child file to be processed
(all filenames without extensions).
The optional argument \textit{main} can be omitted
if \textit{main} matches \textit{dest}.
Optionally, compilation \textit{flags} can be defined via |\def| commands.
This command line makes the \TeX{} engine believe
it is compiling the file \textit{target}
whose content is specified as the latter parameter.
The provided code then forwards the processing to
\textit{main} or \textit{dest} as described in \secref{sec:forward}.

%%%%%%%%%%%%%%%%%%%%%%%%%%%%%%%%%%%%%%%%%%%%%%%%%%%%%%%%%%%%%%%%%%%%%%%%%%%%%%%%
\subsection{Include by Input}
\label{sec:input}

Including child documents by |\include| has some restrictions by design.
Most notably, the content of a child document always occupies
its own set of pages; pages cannot be shared between child documents.
Usually, this behaviour makes perfect sense
because each child document contain an essential part of the document.
However, in some situations it may be desirable to compose
a document from a collection of parts
without having mandatory page breaks between then.
For this case, the package
provides a mechanism to include parts
by |\input| which can also be processed individually.
However, by construction this mechanism
requires manual handling of the content to be output.

%%%%%%%%%%%%%%%%%%%%%%%%%%%%%%%%%%%%%%%%
\DescribeMacro{\ifchilddocmanual}
The main file should be prepared as usual, see \secref{sec:include}.
However, the document body must make a distinction
between processing of an individual part and of the main document, e.g.:
%
\begin{center}
\begin{tabular}{l}
|\ifchilddocmanual|\\
|\input{\childdocname}|\\
|\||else|\\
\textit{document body with }|\input{|\textit{part}|}|\\
|\||fi|
\end{tabular}
\end{center}
%
The conditional |\ifchilddocmanual| is true whenever
a part to be included by |\input| is being compiled,
and the name of the part is stored in |\childdocname|.

%%%%%%%%%%%%%%%%%%%%%%%%%%%%%%%%%%%%%%%%
\DescribeMacro{\childdocby}
Each part to be included by |\input| should start with:
%
\begin{center}
\begin{tabular}{l}
|\input{childdoc.def}|\\
|\childdocby{|\textit{main}|}|\\
\end{tabular}
\end{center}
%
The directive |\childdocby| is similar to |\childdocof|
described in \secref{sec:include},
but the subsequent selection of content must be done manually.
To that end, both |\ifchilddoc| and |\ifchilddocmanual|
will be true upon processing of a part,
and the name of the part is stored in |\childdocname|.
Note that |\jobname| will be set to the filename of the current part
so that each part receives an individual |.aux| file
that does not interfere with the |.aux| file(s) of the main document.
This behaviour can be altered by the alternative form
|\childdocby[*]{|\textit{main}|}| (with a non-empty optional argument)
which uses the |.aux| file of the main document
by setting |\jobname| to \textit{main}.

%%%%%%%%%%%%%%%%%%%%%%%%%%%%%%%%%%%%%%%%%%%%%%%%%%%%%%%%%%%%%%%%%%%%%%%%%%%%%%%%
\subsection{Driver Development}
\label{sec:driver}

The \textsf{childdoc} mechanism can also be use for the development
of definition files such as \LaTeX{} styles or classes.
This case differs from the above setup with multiple parts
included by |\include| in that no |\includeonly| should be invoked.
This can be achieved by starting the include file
(before |\ProvidesPackage|) with:
%
\begin{center}
\begin{tabular}{l}
|\input{childdoc.def}|\\
|\childdocforward{|\textit{main}|}|\\
\end{tabular}
\end{center}
%
or alternatively with:
%
\begin{center}
\begin{tabular}{l}
|\input{childdoc.def}|\\
|\childdocby{|\textit{main}|}|\\
\end{tabular}
\end{center}
%
Both forms have slightly different effects as described above.
The main file is prepared as usual, see \secref{sec:include}.

%%%%%%%%%%%%%%%%%%%%%%%%%%%%%%%%%%%%%%%%%%%%%%%%%%%%%%%%%%%%%%%%%%%%%%%%%%%%%%%%
\subsection{Legacy Detection}
\label{sec:detection}

The directive |\childdocmain| in the main file can detect
whether the complete document or merely a child is to be compiled
even without using the directive |\childdocof|.
This method is deprecated because it is less robust
and there is no compelling reason to use it;
it is merely provided for backward compatibility
and it may be removed in future versions.

If the detection mechanism is to be used,
it is mandatory to correctly specify
the filename of the main file as the argument of |\childdocmain|:
%
\begin{center}
\begin{tabular}{l}
|\input{childdoc.def}|\\
|\childdocmain{|\textit{main}|}|\\
\end{tabular}
\end{center}
%
If |\jobname| does not match the argument \textit{main} of |\childdocmain|,
it is assumed that |\jobname| points to the child file to be compiled.
When using |\childdocmain| with the main file specified as argument,
it suffices to start a child file
with just |\input{|\textit{main}|}|
without loading of the package and using |\childdocof|.
If instead all processing is done
with the appropriate \textsf{childdoc} directives,
the argument of \textit{main} of |\childdocmain| can be empty.

An alternative version of the command line processing described
in \secref{sec:commandline} using the detection mechanism reads:
%
\begin{center}
|... -jobname "|\textit{target}|" "|[\textit{flags}]%
[|\def\jobname{|\textit{dest}|}|]|\input{|\textit{main}|}"|
\end{center}

%%%%%%%%%%%%%%%%%%%%%%%%%%%%%%%%%%%%%%%%%%%%%%%%%%%%%%%%%%%%%%%%%%%%%%%%%%%%%%%%
\subsection{Manual Code}
\label{sec:manual}

In case one cannot be certain whether the definitions file |childdoc.def|
is installed on the target \TeX{} distribution
and one prefers not to ship it,
it is conceivable to paste a few relevant commands into the sources.

To that end, drop all statements |\input{childdoc.def}|
and perform the replacements as outlined below.
Instead of |\childdocmain{|\textit{main}|}| add the following code
to the top of the main file:
%
\begin{center}
\begin{tabular}{l}
|\||ifdefined\childdocname\endinput\||fi\newif\ifchilddoc|\\
|\edef\childdocname{\scantokens\expandafter{\jobname\noexpand}}|\\
|\def\childdocmain{|\textit{main}|}\||ifx\childdocmain\childdocname\||else|\\
|\childdoctrue\includeonly{\childdocname}\let\jobname\childdocmain\||fi|\\
\end{tabular}
\end{center}
%
Instead of |\childdocof{|\textit{main}|}| just include the main file
at the top of each child file:
%
\begin{center}
|\input{|\textit{main}|}|
\end{center}
%
A simple redirection |\childdocforward{|\textit{dest}|}| is achieved by:
%
\begin{center}
|\def\jobname{|\textit{dest}|}\input{\jobname}|
\end{center}
%
The redirection with prefix
|\childdocforwardprefix[|\textit{prefix}|]{|\textit{dest}|}|
is accomplished by:
%
\begin{center}
\begin{tabular}{l}
|{\edef\jobname{\scantokens\expandafter{\jobname\noexpand}}|\\
|\def\redirectjob |\textit{prefix}|#1~~~{\gdef\jobname{|\textit{dest}|#1}}|\\
|\expandafter\redirectjob\jobname~~~}\input{\jobname}|
\end{tabular}
\end{center}

In an alternative approach,
child documents can be compiled by a specific command line
without additional code or specific definitions:
%
\begin{center}
|... -jobname "|\textit{target}|" "|[\textit{flags}]%
|\includeonly{|\textit{dest}|}\input{|\textit{main}|}"|
\end{center}
%

%%%%%%%%%%%%%%%%%%%%%%%%%%%%%%%%%%%%%%%%%%%%%%%%%%%%%%%%%%%%%%%%%%%%%%%%%%%%%%%%
%%%%%%%%%%%%%%%%%%%%%%%%%%%%%%%%%%%%%%%%%%%%%%%%%%%%%%%%%%%%%%%%%%%%%%%%%%%%%%%%
\section{Information}

%%%%%%%%%%%%%%%%%%%%%%%%%%%%%%%%%%%%%%%%%%%%%%%%%%%%%%%%%%%%%%%%%%%%%%%%%%%%%%%%
\subsection{Copyright}

Copyright \copyright{} 2017--2018 Niklas Beisert

This work may be distributed and/or modified under the
conditions of the \LaTeX{} Project Public License, either version 1.3
of this license or (at your option) any later version.
The latest version of this license is in
  \url{http://www.latex-project.org/lppl.txt}
and version 1.3 or later is part of all distributions of \LaTeX{}
version 2005/12/01 or later.

This work has the LPPL maintenance status `maintained'.

The Current Maintainer of this work is Niklas Beisert.

This work consists of the files |README.txt|, |childdoc.ins| and |childdoc.dtx|
as well as the derived files |childdoc.def|, |cdocsamp.tex|
with |cdocsch1.tex|, |cdocsch2.tex|, |cdocspt3.tex|, |cdocspt4.tex|,
|cdocsdrf.tex|, |cdocsfn1.tex|, |cdocsfn2.tex|
as well as |childdoc.pdf|.

%%%%%%%%%%%%%%%%%%%%%%%%%%%%%%%%%%%%%%%%%%%%%%%%%%%%%%%%%%%%%%%%%%%%%%%%%%%%%%%%
\subsection{Files and Installation}

The package consists of the files:
%
\begin{center}
\begin{tabular}{ll}
    |README.txt|   & readme file \\
    |childdoc.ins| & installation file \\
    |childdoc.dtx| & source file \\
    |childdoc.def| & definition file \\
    |cdocsamp.tex| & sample main file \\
    |cdocsch1.tex| & sample include file \\
    |cdocsch2.tex| & sample include file \\
    |cdocspt3.tex| & sample part file \\
    |cdocspt4.tex| & sample part file \\
    |cdocsdrf.tex| & sample redirection file \\
    |cdocsfn1.tex| & sample redirection file \\
    |cdocsfn2.tex| & sample redirection file \\
    |childdoc.pdf| & manual
\end{tabular}
\end{center}
%
The distribution consists of the files
|README.txt|, |childdoc.ins| and |childdoc.dtx|.
%
\begin{itemize}
\item
Run (pdf)\LaTeX{} on |childdoc.dtx|
to compile the manual |childdoc.pdf| (this file).
\item
Run \LaTeX{} on |childdoc.ins| to create the definitions file |childdoc.def|
and the sample |cdocsamp.tex| with include files
|cdocsch1.tex|, |cdocsch2.tex|, |cdocspt3.tex|, |cdocspt4.tex|,
|cdocsdrf.tex|, |cdocsfn1.tex|, |cdocsfn2.tex|.
Then copy the file |childdoc.def| to an appropriate directory of your \LaTeX{}
distribution, e.g.\ \textit{texmf-root}|/tex/latex/childdoc|.
\end{itemize}

%%%%%%%%%%%%%%%%%%%%%%%%%%%%%%%%%%%%%%%%%%%%%%%%%%%%%%%%%%%%%%%%%%%%%%%%%%%%%%%%
\subsection{Related CTAN Packages}

There are several other packages which offer a similar functionality:
%
\begin{itemize}
\item
The packages
\href{http://ctan.org/pkg/docmute}{\textsf{docmute}},
\href{http://ctan.org/pkg/includex}{\textsf{includex}} and
\href{http://ctan.org/pkg/standalone}{\textsf{standalone}}
provide commands to include only the document body of
a child file thus allowing both files to be compiled individually.
\item
The packages \href{http://ctan.org/pkg/subdocs}{\textsf{subdocs}}
and \href{http://ctan.org/pkg/subfiles}{\textsf{subfiles}}
provide structures in which the main and child documents can be
encapsulated and allowing them to be compiled individually.
The inclusion mechanism is different from the conventional |\include|.
\item
The package \href{http://ctan.org/pkg/combine}{\textsf{combine}}
is an elaborate solution to combine several documents into one.
\end{itemize}
%
See also the CTAN topic \href{http://ctan.org/topic/subdocs}{\textsf{subdocs}}
for further related packages.
The present package differs from the above solutions in that
a document structure constructed with the conventional |\include| mechanism
just needs two extra commands at the top of every file
such that all constituent files can be compiled individually.

%%%%%%%%%%%%%%%%%%%%%%%%%%%%%%%%%%%%%%%%%%%%%%%%%%%%%%%%%%%%%%%%%%%%%%%%%%%%%%%%
%\subsection{Feature Suggestions}
%
%The following is a list of features which may be useful for future
%versions of this package:
%%
%\begin{itemize}
%\item
%\ldots
%\end{itemize}

%%%%%%%%%%%%%%%%%%%%%%%%%%%%%%%%%%%%%%%%%%%%%%%%%%%%%%%%%%%%%%%%%%%%%%%%%%%%%%%%
\subsection{Revision History}

%%%%%%%%%%%%%%%%%%%%%%%%%%%%%%%%%%%%%%%%
\paragraph{v2.0:} 2018/12/30

\begin{itemize}
\item
immediate forward processing
\item
added |\childdocby| mechanism
\item
manual restructured
\end{itemize}

%%%%%%%%%%%%%%%%%%%%%%%%%%%%%%%%%%%%%%%%
\paragraph{v1.6:} 2018/01/17

\begin{itemize}
\item
application for development of include files
\item
corrections to manual
\end{itemize}

%%%%%%%%%%%%%%%%%%%%%%%%%%%%%%%%%%%%%%%%
\paragraph{v1.5:} 2017/05/21

\begin{itemize}
\item
more complete structuring introduced
\item
|\childdocof| introduced
\item
|\childdoc| renamed to |\childdocmain|
\item
|\childredirect| renamed to |\childdocforward| and |\childdocforwardprefix|
and functionality expanded
\end{itemize}

%%%%%%%%%%%%%%%%%%%%%%%%%%%%%%%%%%%%%%%%
\paragraph{v1.0:} 2017/04/27

\begin{itemize}
\item
manual and install package
\item
first version published on CTAN
\end{itemize}

%%%%%%%%%%%%%%%%%%%%%%%%%%%%%%%%%%%%%%%%
\paragraph{v0.6:} 2017/04/26

\begin{itemize}
\item
redirection mechanism added
\end{itemize}

%%%%%%%%%%%%%%%%%%%%%%%%%%%%%%%%%%%%%%%%
\paragraph{v0.5:} 2017/04/26

\begin{itemize}
\item
functionality in definition file
\end{itemize}


%%%%%%%%%%%%%%%%%%%%%%%%%%%%%%%%%%%%%%%%%%%%%%%%%%%%%%%%%%%%%%%%%%%%%%%%%%%%%%%%
%%%%%%%%%%%%%%%%%%%%%%%%%%%%%%%%%%%%%%%%%%%%%%%%%%%%%%%%%%%%%%%%%%%%%%%%%%%%%%%%
%%%%%%%%%%%%%%%%%%%%%%%%%%%%%%%%%%%%%%%%%%%%%%%%%%%%%%%%%%%%%%%%%%%%%%%%%%%%%%%%
\appendix

\settowidth\MacroIndent{\rmfamily\scriptsize 000\ }

 \DocInput{childdoc.dtx}

\end{document}
%</driver>
% \fi
%
% %%%%%%%%%%%%%%%%%%%%%%%%%%%%%%%%%%%%%%%%%%%%%%%%%%%%%%%%%%%%%%%%%%%%%%%%%%%%%%
% %%%%%%%%%%%%%%%%%%%%%%%%%%%%%%%%%%%%%%%%%%%%%%%%%%%%%%%%%%%%%%%%%%%%%%%%%%%%%%
% \section{Sample}
%\iffalse
%<*samplemain>
%\fi
%
% The following presents a sample document
% with two chapters, two parts, a title page,
% a compile flag as well as three forwarding files to set the flag.
% It consists of eight |.tex| files:
% \begin{center}
% \begin{tabular}{ll}
% |cdocsamp.tex|&main file\\
% |cdocsch1.tex|&include file for chapter 1\\
% |cdocsch2.tex|&include file for chapter 2\\
% |cdocspt3.tex|&include file for part 3\\
% |cdocspt4.tex|&include file for part 4\\
% |cdocsdrf.tex|&forwarding file for main file in draft mode\\
% |cdocsfi1.tex|&forwarding file for final version of chapter 1\\
% |cdocsfi2.tex|&forwarding file for final version of chapter 2\\
% \end{tabular}
% \end{center}
% Each of the eight files can be compiled directly by the \LaTeX{} compiler.
%
% %%%%%%%%%%%%%%%%%%%%%%%%%%%%%%%%%%%%%%
% \paragraph{Main File.}
%
% The main file is called |cdocsamp.tex|.
%
% Load the \textsf{childdoc} definitions and
% declare the filename for the main document:
%    \begin{macrocode}
\input{childdoc.def}
\childdocmain{}
%    \end{macrocode}

% Optional override for |\version| flag:
%    \begin{macrocode}
%%\ifchilddoc\else\providecommand{\version}{draft}\fi
%    \end{macrocode}

% Define the default values for the |\version| flag
% (|final| for the main file and |draft| for childs):
%    \begin{macrocode}
\ifchilddoc
\providecommand{\version}{draft}
\else
\providecommand{\version}{final}
\fi
%    \end{macrocode}

% Load the standard document class:
%    \begin{macrocode}
\documentclass[12pt]{article}
%    \end{macrocode}

% Start the document body:
%    \begin{macrocode}
\begin{document}
%    \end{macrocode}

% Declare a title page.
% Print title, part of document being processed and version flag:
%    \begin{macrocode}
\addtocounter{page}{-1}
\begin{center}
{\LARGE\bfseries{}childdoc example\par}
\vspace{1cm}
\ifchilddoc
\ifchilddocmanual part\else chapter\fi:
`\childdocname' of `\childdocjob'\par
\else
main document: `\childdocjob'\par
\fi
version: \version\par
\end{center}
\newpage
%    \end{macrocode}

% Manually include selected file,
% otherwise process as usual:
%    \begin{macrocode}
\ifchilddocmanual
\section*{part `\childdocname'}
\input{\childdocname}
\else
%    \end{macrocode}

% Include the two chapters:
%    \begin{macrocode}
\include{cdocsch1}
\include{cdocsch2}
%    \end{macrocode}

% Include the two parts unless only chapters should be displayed:
%    \begin{macrocode}
\ifchilddoc\else
\section{part three}
\input{cdocspt3}
\section{part four}
\input{cdocspt4}
\fi
%    \end{macrocode}

% Process as usual until here:
%    \begin{macrocode}
\fi
%    \end{macrocode}

% End of document body:
%    \begin{macrocode}
\end{document}
%    \end{macrocode}
%\iffalse
%</samplemain>
%\fi
%
% %%%%%%%%%%%%%%%%%%%%%%%%%%%%%%%%%%%%%%
% \paragraph{Chapter Include Files.}
%
% The include files are called |cdocsch1.tex| and |cdocsch2.tex|.
%
%\iffalse
%<*samplechap1|samplechap2>
%\fi

% Optional override for |\version| flag:
%    \begin{macrocode}
%%\providecommand{\version}{final}
%    \end{macrocode}

% Include the main document:
%    \begin{macrocode}
\input{childdoc.def}
\childdocof{cdocsamp}
%    \end{macrocode}

%\iffalse
%</samplechap1|samplechap2>
%\fi
%
%\iffalse
%<*samplechap1>
%\fi
% Some text for chapter 1:
%    \begin{macrocode}
\section{one}
some text in chapter one
%    \end{macrocode}

%\iffalse
%</samplechap1>
%\fi
% Some text for chapter 2:
%\iffalse
%<*samplechap2>
%\fi
%    \begin{macrocode}
\section{two}
more text in chapter two
%    \end{macrocode}

%\iffalse
%</samplechap2>
%\fi
%
% %%%%%%%%%%%%%%%%%%%%%%%%%%%%%%%%%%%%%%
% \paragraph{Part Include Files.}
%
% The include files are called |cdocspt3.tex| and |cdocspt4.tex|.
%
%\iffalse
%<*samplepart3|samplepart4>
%\fi

% Optional override for |\version| flag:
%    \begin{macrocode}
%%\providecommand{\version}{final}
%    \end{macrocode}

% Include the main document:
%    \begin{macrocode}
\input{childdoc.def}
\childdocby{cdocsamp}
%    \end{macrocode}

%\iffalse
%</samplepart3|samplepart4>
%\fi
%
%\iffalse
%<*samplepart3>
%\fi
% Some text for part 3:
%    \begin{macrocode}
some text in part three
%    \end{macrocode}

%\iffalse
%</samplepart3>
%\fi
% Some text for part 4:
%\iffalse
%<*samplepart4>
%\fi
%    \begin{macrocode}
more text in part four
%    \end{macrocode}

%\iffalse
%</samplepart4>
%\fi
%
% %%%%%%%%%%%%%%%%%%%%%%%%%%%%%%%%%%%%%%
% \paragraph{Forwarding for a Complete Draft.}
%
% The following forwarding file |cdocsdrf.tex|
% compiles the main document in draft mode:
%\iffalse
%<*sampledraft>
%\fi
%    \begin{macrocode}
\def\version{draft}
\input{childdoc.def}
\childdocforward{cdocsamp}
%    \end{macrocode}

%\iffalse
%</sampledraft>
%\fi
%
% %%%%%%%%%%%%%%%%%%%%%%%%%%%%%%%%%%%%%%
% \paragraph{Forwarding for Final Version of the Chapters.}
%
% The following forwarding files |cdocsfn1.tex| and |cdocsfn2.tex|
% (with identical content)
% compile the final versions of the child documents
% |cdocsch1.tex| and |cdocsch2.tex|, respectively:
%\iffalse
%<*samplefinal>
%\fi
%    \begin{macrocode}
\def\version{final}
\input{childdoc.def}
\childdocforwardprefix[cdocsamp]{cdocsfn}{cdocsch}
%    \end{macrocode}

%\iffalse
%</samplefinal>
%\fi
%
% %%%%%%%%%%%%%%%%%%%%%%%%%%%%%%%%%%%%%%
% \paragraph{Command Line Processing.}
%
% The following three command lines generate the output files
% |cdocscld|, |cdocscl1| and |cdocscl2|
% which should be identical to
% |cdocsdrf|, |cdocsch1| and |cdocsfn2|, respectively:
% \begin{center}
% \begin{tabular}{l}
% |latex -jobname cdocscld \|\\
% |  "\def\version{draft}\input{childdoc.def}\childdocforward{cdocsamp}"|\\
% |latex -jobname cdocscl1 \|\\
% |  "\input{childdoc.def}\childdocforward[cdocsamp]{cdocsch1}"|\\
% |latex -jobname cdocscl2 \|\\
% |  "\def\version{final}\input{childdoc.def}\childdocforward{cdocsch2}"|
% \end{tabular}
% \end{center}
% Note that the trailing backslash on each first line
% merely continues the input to the second line
% (for convenient cut ant paste).
% Furthermore, the command |latex| can be replaced by any
% of its alternative versions such as |pdflatex|.
%
% %%%%%%%%%%%%%%%%%%%%%%%%%%%%%%%%%%%%%%%%%%%%%%%%%%%%%%%%%%%%%%%%%%%%%%%%%%%%%%
% %%%%%%%%%%%%%%%%%%%%%%%%%%%%%%%%%%%%%%%%%%%%%%%%%%%%%%%%%%%%%%%%%%%%%%%%%%%%%%
% \section{Implementation}
%\iffalse
%<*package>
%\fi
%
% This section describes the definitions file |childdoc.def|.

% The definitions cannot be loaded using |\usepackage| or |\RequirePackage|
% which has a mechanism to prevent loading a style file more than once.
% When loading the definitions by means of |\input|
% multiple instances have to be prevented manually:
%\iffalse
%This code needs to be before the `\ProvidesFile' directive
%which is defined at the beginning of this file.
%Therefore it is also placed there and commented out here.
%</package>
%<*discard>
%\fi
%    \begin{macrocode}
\ifdefined\childdocmain\endinput\fi
%    \end{macrocode}
%\iffalse
%</discard>
%<*package>
%\fi
%
% \macro{\ifchilddoc}
% \macro{\ifchilddocmanual}
% The conditional |\ifchilddoc| tells whether a
% child (true) or main (false) document is being compiled.
% The conditional |\ifchilddocmanual| tells whether
% the |\includeonly| mechanism is used (false) or
% the selection of child files must be performed manually (true).
% The definitions initialise to false:
%    \begin{macrocode}
\newif\ifchilddoc
\newif\ifchilddocmanual
%    \end{macrocode}

% \macro{\childdocname}
% \macro{\childdocjob}
% The macro |\childdocname| stores the name of the main document
% to be compiled. The macro |\childdocjob| stores the name of
% the document on which the \LaTeX{} compiler was originally invoked.
% The content of |\jobname| cannot be compared
% to filenames specified in the source due to different catcodes.
% The following code rescans |\jobname|, stores the result
% in |\childdocname| and saves a copy in |\childdocjob|:
%    \begin{macrocode}
\edef\childdocname{\scantokens\expandafter{\jobname\noexpand}}
\let\childdocjob\childdocname
%    \end{macrocode}

% \macro{\childdocdisable}
% The macro |\childdocdisable| prevents the main file
% from being processed more than once.
% At this stage, the main document command |\childdocmain|
% is assumed to be called once again where it should do nothing.
% Any subsequent call to it should prevent
% a secondary processing of the main document
% It overwrites the forwarding commands
% |\childdocof| and |\childdocforward|
% with empty macros to prevent further inclusions of the main document:
%    \begin{macrocode}
\newcommand{\childdocdisable}
{
  \renewcommand{\childdocmain}[1]{\renewcommand{\childdocmain}[1]{\endinput}}
  \renewcommand{\childdocof}[1]{}
  \renewcommand{\childdocby}[2][]{}
  \renewcommand{\childdocforward}[2][]{}
  \renewcommand{\childdocdisable}{}
}
%    \end{macrocode}

% \macro{\childdocmain}
% The macro |\childdocmain| is to be called at the top of the main file
% with nothing or the main filename (without extension) as argument.
% First, it breaks loops.
% If the argument is not empty and does not match |\childdocname|
% (which is set by the first inclusion of |childdoc.def|),
% |\ifchilddoc| is set to true, |\includeonly| is applied to the child file
% and |\jobname| is set to the main file
% (for proper handling of |.aux| files):
%    \begin{macrocode}
\newcommand{\childdocmain}[1]
{
  \childdocdisable\childdocmain{}
  \if?#1?\else
    \begingroup
      \def\childdoctmp{#1}
      \ifx\childdoctmp\childdocname
        \def\childdoctmp{}
      \else
        \def\childdoctmp
        {
          \childdoctrue
          \includeonly{\childdocname}
          \def\childdocjob{#1}
          \def\jobname{#1}
        }
      \fi
      \expandafter
    \endgroup
    \childdoctmp
  \fi
}
%    \end{macrocode}

% \macro{\childdocof}
% The command |\childdocof| redirects
% compilation to the main file |#1|.
%    \begin{macrocode}
\newcommand{\childdocof}[1]
{
  \childdocdisable
  \childdoctrue
  \includeonly{\childdocname}
  \def\jobname{#1}
  \def\childdocjob{#1}
  \input{#1}
}
%    \end{macrocode}

% \macro{\childdocby}
% The command |\childdocby| ....
%    \begin{macrocode}
\newcommand{\childdocby}[2][]
{
  \childdocdisable
  \childdoctrue
  \childdocmanualtrue
  \if?#1?\else
    \def\jobname{#2}
  \fi
  \def\childdocjob{#2}
  \input{#2}
  \endinput
}
%    \end{macrocode}

% \macro{\childdocforward}
% The command |\childdocforward| redirects
% compilation to the main file or
% (if the optional argument is given) a child file.
% Parameters are set as if the main file
% or a child file starting with |\childdocof| was compiled.
% Then compilation is handed over to the main file:
%    \begin{macrocode}
\newcommand{\childdocforward}[2][]
{
  \begingroup
    \if?#1?
      \def\childdoctmp
      {
        \def\childdocname{#2}
        \def\childdocjob{#2}
        \def\jobname{#2}
        \input{#2}
        \endinput
      }
    \else
      \def\childdoctmp
      {
        \childdocdisable
        \def\childdocname{#2}
        \childdoctrue
        \includeonly{#2}
        \def\childdocjob{#1}
        \def\jobname{#1}
        \input{#1}
        \endinput
      }
    \fi
    \expandafter
  \endgroup
  \childdoctmp
}
%    \end{macrocode}

% \macro{\childdocforwardprefix}
% The command |\childdocforwardprefix| redirects
% compilation to the main or a child file by means of a pattern.
% The prefix |#1| in the current filename is replaced by |#2|
% and the suffix of the current filename is kept
% (it is assumed that the filename does not contain the substring `|~~~|'
% which is used as a delimiter).
% Compilation is handed over to the new file by |\childdocforward|:
%    \begin{macrocode}
\newcommand{\childdocforwardprefix}[3][]
{
  \begingroup
    \def\childdocextract #2##1~~~{\def\childdoctmp{\childdocforward[#1]{#3##1}}}
    \expandafter\childdocextract\childdocname~~~
    \expandafter
  \endgroup
  \childdoctmp
}
%    \end{macrocode}

% \macro{\childdoc}
% The deprecated macro |\childdoc| is a legacy version of |\childdocmain|:
%    \begin{macrocode}
\newcommand{\childdoc}{\childdocmain}
%    \end{macrocode}

% \macro{\childdocredirect}
% The deprecated macro |\childdocredirect| is a legacy version
% of |\childdocforward| and |\childdocforwardprefix|:
%    \begin{macrocode}
\newcommand{\childdocredirect}[2][]
{
  \begingroup
    \if?#1?
      \def\childdoctmp{\childdocforward{#2}}
    \else
      \def\childdoctmp{\childdocforwardprefix{#1}{#2}}
    \fi
    \expandafter
  \endgroup
  \childdoctmp
}
%    \end{macrocode}

%\iffalse
%</package>
%\fi
%
\endinput
|
and perform the replacements as outlined below.
Instead of |\childdocmain{|\textit{main}|}| add the following code
to the top of the main file:
%
\begin{center}
\begin{tabular}{l}
|\||ifdefined\childdocname\endinput\||fi\newif\ifchilddoc|\\
|\edef\childdocname{\scantokens\expandafter{\jobname\noexpand}}|\\
|\def\childdocmain{|\textit{main}|}\||ifx\childdocmain\childdocname\||else|\\
|\childdoctrue\includeonly{\childdocname}\let\jobname\childdocmain\||fi|\\
\end{tabular}
\end{center}
%
Instead of |\childdocof{|\textit{main}|}| just include the main file
at the top of each child file:
%
\begin{center}
|\input{|\textit{main}|}|
\end{center}
%
A simple redirection |\childdocforward{|\textit{dest}|}| is achieved by:
%
\begin{center}
|\def\jobname{|\textit{dest}|}\input{\jobname}|
\end{center}
%
The redirection with prefix
|\childdocforwardprefix[|\textit{prefix}|]{|\textit{dest}|}|
is accomplished by:
%
\begin{center}
\begin{tabular}{l}
|{\edef\jobname{\scantokens\expandafter{\jobname\noexpand}}|\\
|\def\redirectjob |\textit{prefix}|#1~~~{\gdef\jobname{|\textit{dest}|#1}}|\\
|\expandafter\redirectjob\jobname~~~}\input{\jobname}|
\end{tabular}
\end{center}

In an alternative approach,
child documents can be compiled by a specific command line
without additional code or specific definitions:
%
\begin{center}
|... -jobname "|\textit{target}|" "|[\textit{flags}]%
|\includeonly{|\textit{dest}|}\input{|\textit{main}|}"|
\end{center}
%

%%%%%%%%%%%%%%%%%%%%%%%%%%%%%%%%%%%%%%%%%%%%%%%%%%%%%%%%%%%%%%%%%%%%%%%%%%%%%%%%
%%%%%%%%%%%%%%%%%%%%%%%%%%%%%%%%%%%%%%%%%%%%%%%%%%%%%%%%%%%%%%%%%%%%%%%%%%%%%%%%
\section{Information}

%%%%%%%%%%%%%%%%%%%%%%%%%%%%%%%%%%%%%%%%%%%%%%%%%%%%%%%%%%%%%%%%%%%%%%%%%%%%%%%%
\subsection{Copyright}

Copyright \copyright{} 2017--2018 Niklas Beisert

This work may be distributed and/or modified under the
conditions of the \LaTeX{} Project Public License, either version 1.3
of this license or (at your option) any later version.
The latest version of this license is in
  \url{http://www.latex-project.org/lppl.txt}
and version 1.3 or later is part of all distributions of \LaTeX{}
version 2005/12/01 or later.

This work has the LPPL maintenance status `maintained'.

The Current Maintainer of this work is Niklas Beisert.

This work consists of the files |README.txt|, |childdoc.ins| and |childdoc.dtx|
as well as the derived files |childdoc.def|, |cdocsamp.tex|
with |cdocsch1.tex|, |cdocsch2.tex|, |cdocspt3.tex|, |cdocspt4.tex|,
|cdocsdrf.tex|, |cdocsfn1.tex|, |cdocsfn2.tex|
as well as |childdoc.pdf|.

%%%%%%%%%%%%%%%%%%%%%%%%%%%%%%%%%%%%%%%%%%%%%%%%%%%%%%%%%%%%%%%%%%%%%%%%%%%%%%%%
\subsection{Files and Installation}

The package consists of the files:
%
\begin{center}
\begin{tabular}{ll}
    |README.txt|   & readme file \\
    |childdoc.ins| & installation file \\
    |childdoc.dtx| & source file \\
    |childdoc.def| & definition file \\
    |cdocsamp.tex| & sample main file \\
    |cdocsch1.tex| & sample include file \\
    |cdocsch2.tex| & sample include file \\
    |cdocspt3.tex| & sample part file \\
    |cdocspt4.tex| & sample part file \\
    |cdocsdrf.tex| & sample redirection file \\
    |cdocsfn1.tex| & sample redirection file \\
    |cdocsfn2.tex| & sample redirection file \\
    |childdoc.pdf| & manual
\end{tabular}
\end{center}
%
The distribution consists of the files
|README.txt|, |childdoc.ins| and |childdoc.dtx|.
%
\begin{itemize}
\item
Run (pdf)\LaTeX{} on |childdoc.dtx|
to compile the manual |childdoc.pdf| (this file).
\item
Run \LaTeX{} on |childdoc.ins| to create the definitions file |childdoc.def|
and the sample |cdocsamp.tex| with include files
|cdocsch1.tex|, |cdocsch2.tex|, |cdocspt3.tex|, |cdocspt4.tex|,
|cdocsdrf.tex|, |cdocsfn1.tex|, |cdocsfn2.tex|.
Then copy the file |childdoc.def| to an appropriate directory of your \LaTeX{}
distribution, e.g.\ \textit{texmf-root}|/tex/latex/childdoc|.
\end{itemize}

%%%%%%%%%%%%%%%%%%%%%%%%%%%%%%%%%%%%%%%%%%%%%%%%%%%%%%%%%%%%%%%%%%%%%%%%%%%%%%%%
\subsection{Related CTAN Packages}

There are several other packages which offer a similar functionality:
%
\begin{itemize}
\item
The packages
\href{http://ctan.org/pkg/docmute}{\textsf{docmute}},
\href{http://ctan.org/pkg/includex}{\textsf{includex}} and
\href{http://ctan.org/pkg/standalone}{\textsf{standalone}}
provide commands to include only the document body of
a child file thus allowing both files to be compiled individually.
\item
The packages \href{http://ctan.org/pkg/subdocs}{\textsf{subdocs}}
and \href{http://ctan.org/pkg/subfiles}{\textsf{subfiles}}
provide structures in which the main and child documents can be
encapsulated and allowing them to be compiled individually.
The inclusion mechanism is different from the conventional |\include|.
\item
The package \href{http://ctan.org/pkg/combine}{\textsf{combine}}
is an elaborate solution to combine several documents into one.
\end{itemize}
%
See also the CTAN topic \href{http://ctan.org/topic/subdocs}{\textsf{subdocs}}
for further related packages.
The present package differs from the above solutions in that
a document structure constructed with the conventional |\include| mechanism
just needs two extra commands at the top of every file
such that all constituent files can be compiled individually.

%%%%%%%%%%%%%%%%%%%%%%%%%%%%%%%%%%%%%%%%%%%%%%%%%%%%%%%%%%%%%%%%%%%%%%%%%%%%%%%%
%\subsection{Feature Suggestions}
%
%The following is a list of features which may be useful for future
%versions of this package:
%%
%\begin{itemize}
%\item
%\ldots
%\end{itemize}

%%%%%%%%%%%%%%%%%%%%%%%%%%%%%%%%%%%%%%%%%%%%%%%%%%%%%%%%%%%%%%%%%%%%%%%%%%%%%%%%
\subsection{Revision History}

%%%%%%%%%%%%%%%%%%%%%%%%%%%%%%%%%%%%%%%%
\paragraph{v2.0:} 2018/12/30

\begin{itemize}
\item
immediate forward processing
\item
added |\childdocby| mechanism
\item
manual restructured
\end{itemize}

%%%%%%%%%%%%%%%%%%%%%%%%%%%%%%%%%%%%%%%%
\paragraph{v1.6:} 2018/01/17

\begin{itemize}
\item
application for development of include files
\item
corrections to manual
\end{itemize}

%%%%%%%%%%%%%%%%%%%%%%%%%%%%%%%%%%%%%%%%
\paragraph{v1.5:} 2017/05/21

\begin{itemize}
\item
more complete structuring introduced
\item
|\childdocof| introduced
\item
|\childdoc| renamed to |\childdocmain|
\item
|\childredirect| renamed to |\childdocforward| and |\childdocforwardprefix|
and functionality expanded
\end{itemize}

%%%%%%%%%%%%%%%%%%%%%%%%%%%%%%%%%%%%%%%%
\paragraph{v1.0:} 2017/04/27

\begin{itemize}
\item
manual and install package
\item
first version published on CTAN
\end{itemize}

%%%%%%%%%%%%%%%%%%%%%%%%%%%%%%%%%%%%%%%%
\paragraph{v0.6:} 2017/04/26

\begin{itemize}
\item
redirection mechanism added
\end{itemize}

%%%%%%%%%%%%%%%%%%%%%%%%%%%%%%%%%%%%%%%%
\paragraph{v0.5:} 2017/04/26

\begin{itemize}
\item
functionality in definition file
\end{itemize}


%%%%%%%%%%%%%%%%%%%%%%%%%%%%%%%%%%%%%%%%%%%%%%%%%%%%%%%%%%%%%%%%%%%%%%%%%%%%%%%%
%%%%%%%%%%%%%%%%%%%%%%%%%%%%%%%%%%%%%%%%%%%%%%%%%%%%%%%%%%%%%%%%%%%%%%%%%%%%%%%%
%%%%%%%%%%%%%%%%%%%%%%%%%%%%%%%%%%%%%%%%%%%%%%%%%%%%%%%%%%%%%%%%%%%%%%%%%%%%%%%%
\appendix

\settowidth\MacroIndent{\rmfamily\scriptsize 000\ }

 \DocInput{childdoc.dtx}

\end{document}
%</driver>
% \fi
%
% %%%%%%%%%%%%%%%%%%%%%%%%%%%%%%%%%%%%%%%%%%%%%%%%%%%%%%%%%%%%%%%%%%%%%%%%%%%%%%
% %%%%%%%%%%%%%%%%%%%%%%%%%%%%%%%%%%%%%%%%%%%%%%%%%%%%%%%%%%%%%%%%%%%%%%%%%%%%%%
% \section{Sample}
%\iffalse
%<*samplemain>
%\fi
%
% The following presents a sample document
% with two chapters, two parts, a title page,
% a compile flag as well as three forwarding files to set the flag.
% It consists of eight |.tex| files:
% \begin{center}
% \begin{tabular}{ll}
% |cdocsamp.tex|&main file\\
% |cdocsch1.tex|&include file for chapter 1\\
% |cdocsch2.tex|&include file for chapter 2\\
% |cdocspt3.tex|&include file for part 3\\
% |cdocspt4.tex|&include file for part 4\\
% |cdocsdrf.tex|&forwarding file for main file in draft mode\\
% |cdocsfi1.tex|&forwarding file for final version of chapter 1\\
% |cdocsfi2.tex|&forwarding file for final version of chapter 2\\
% \end{tabular}
% \end{center}
% Each of the eight files can be compiled directly by the \LaTeX{} compiler.
%
% %%%%%%%%%%%%%%%%%%%%%%%%%%%%%%%%%%%%%%
% \paragraph{Main File.}
%
% The main file is called |cdocsamp.tex|.
%
% Load the \textsf{childdoc} definitions and
% declare the filename for the main document:
%    \begin{macrocode}
% \iffalse
%
% childdoc.dtx Copyright (C) 2017-2018 Niklas Beisert
%
% This work may be distributed and/or modified under the
% conditions of the LaTeX Project Public License, either version 1.3
% of this license or (at your option) any later version.
% The latest version of this license is in
%   http://www.latex-project.org/lppl.txt
% and version 1.3 or later is part of all distributions of LaTeX
% version 2005/12/01 or later.
%
% This work has the LPPL maintenance status `maintained'.
%
% The Current Maintainer of this work is Niklas Beisert.
%
% This work consists of the files childdoc.dtx and childdoc.ins
% and the derived files childdoc.def and cdocsamp.tex with
% cdocsch1.tex, cdocsch2.tex, cdocsdrf.tex, cdocsfn1.tex, cdocsfn2.tex.
%
%<package>\ifdefined\childdocmain\endinput\fi
%<package>\ProvidesFile{childdoc.def}[2018/12/30 v2.0 child document driver]
%<samplemain>\ProvidesFile{cdocsamp.tex}[2018/12/30 v2.0 sample for childdoc]
%<*driver>
%\ProvidesFile{childdoc.drv}[2018/12/30 v2.0 childdoc reference manual file]
\PassOptionsToClass{10pt,a4paper}{article}
\documentclass{ltxdoc}

\usepackage[margin=35mm]{geometry}
\usepackage{hyperref}
\usepackage{hyperxmp}
\usepackage[usenames]{color}

\hypersetup{colorlinks=true}
\hypersetup{pdfstartview=FitH}
\hypersetup{pdfpagemode=UseNone}
\hypersetup{pdfsource={}}
\hypersetup{pdflang={en-UK}}
\hypersetup{pdfcopyright={Copyright 2017-2018 Niklas Beisert.
  This work may be distributed and/or modified under the
  conditions of the LaTeX Project Public License, either version 1.3
  of this license or (at your option) any later version.}}
\hypersetup{pdflicenseurl={http://www.latex-project.org/lppl.txt}}
\hypersetup{pdfcontactaddress={ETH Zurich, ITP, HIT K,
  Wolfgang-Pauli-Strasse 27}}
\hypersetup{pdfcontactpostcode={8093}}
\hypersetup{pdfcontactcity={Zurich}}
\hypersetup{pdfcontactcountry={Switzerland}}
\hypersetup{pdfcontactemail={nbeisert@itp.phys.ethz.ch}}
\hypersetup{pdfcontacturl={http://people.phys.ethz.ch/\xmptilde nbeisert/}}

\newcommand{\secref}[1]{\hyperref[#1]{section \ref*{#1}}}

\parskip1ex
\parindent0pt
\let\olditemize\itemize
\def\itemize{\olditemize\parskip0pt}

\begin{document}

\title{The \textsf{childdoc} Package}
\hypersetup{pdftitle={The childdoc Package}}
\author{Niklas Beisert\\[2ex]
  Institut f\"ur Theoretische Physik\\
  Eidgen\"ossische Technische Hochschule Z\"urich\\
  Wolfgang-Pauli-Strasse 27, 8093 Z\"urich, Switzerland\\[1ex]
  \href{mailto:nbeisert@itp.phys.ethz.ch}
  {\texttt{nbeisert@itp.phys.ethz.ch}}}
\hypersetup{pdfauthor={Niklas Beisert}}
\hypersetup{pdfsubject={Manual for the LaTeX2e Package childdoc}}
\date{30 December 2018, \textsf{v2.0}}
\maketitle

\begin{abstract}\noindent
\textsf{childdoc} is a \LaTeXe{} package
that enables the direct compilation
of document sections included by |\include|
to individual files.
\end{abstract}

\begingroup
\parskip0ex
\tableofcontents
\endgroup

%%%%%%%%%%%%%%%%%%%%%%%%%%%%%%%%%%%%%%%%%%%%%%%%%%%%%%%%%%%%%%%%%%%%%%%%%%%%%%%%
%%%%%%%%%%%%%%%%%%%%%%%%%%%%%%%%%%%%%%%%%%%%%%%%%%%%%%%%%%%%%%%%%%%%%%%%%%%%%%%%
\section{Introduction}

\LaTeX{} provides a mechanism to structure a large document (such as a book)
into a main file and several child files (containing the chapters)
using the |\include| command.
This mechanism is beneficial for documents
which span hundreds of pages in order to
make the source file(s) more manageable.
Moreover, compilation can be restricted to
selected child files by means of the |\includeonly| command.
The latter feature can be used to reduce the compilation time while editing
(this was significantly more useful in the earlier days of \LaTeX{})
or to generate a smaller document which is easier to navigate.
Another application of |\includeonly| is to generate
documents consisting of selected parts of the complete document.

However, there are a few drawbacks of the plain |\include| mechanism:
\begin{itemize}
\item
The child files cannot be compiled on their own,
they can only be compiled via the main file.
A naive editing environment
(such as a text editor with an option
to have the current file processed by \LaTeX)
may require one to switch to the main file before compiling;
attempting to compile the child file produces errors.
\item
The main file must be modified (each time)
to adjust the |\includeonly| command
to the present needs. This easily leaves the main file in a messy state.
\item
The generated document will always carry the filename
of the main document. This is inconvenient if
several child files are to be compiled and
to be kept for distribution.
\end{itemize}

The present package provides a simple interface
to make child files individually compilable by \LaTeX{}.
Compiling a child file then has the same effect as compiling
the main file with an |\includeonly| command
to select the appropriate child.
Moreover the generated document will carry the name of the child
rather than the main file.
This resolves all three above issues.

This feature is meant to make the editing of books,
thesis documents and lecture notes somewhat more convenient.
However, the package can also be used efficiently for
composing a series of documents (such as exercise sheets)
which are typically distributed individually.
It then assists the author in generating the individual documents
(potentially in different versions)
as well as a document containing the collected series.
Another application is in developing style files
or other kinds of included material
where compilation of the style file could redirect
to a sample or test file.

%%%%%%%%%%%%%%%%%%%%%%%%%%%%%%%%%%%%%%%%%%%%%%%%%%%%%%%%%%%%%%%%%%%%%%%%%%%%%%%%
%%%%%%%%%%%%%%%%%%%%%%%%%%%%%%%%%%%%%%%%%%%%%%%%%%%%%%%%%%%%%%%%%%%%%%%%%%%%%%%%
\section{Usage}

First of all, the package \textsf{childdoc} is \emph{not} a standard
\LaTeXe{} |.sty| style file! Therefore it needs to be invoked in
a non-standard way.

%%%%%%%%%%%%%%%%%%%%%%%%%%%%%%%%%%%%%%%%%%%%%%%%%%%%%%%%%%%%%%%%%%%%%%%%%%%%%%%%
\subsection{Included Files}
\label{sec:include}

%%%%%%%%%%%%%%%%%%%%%%%%%%%%%%%%%%%%%%%%
\DescribeMacro{\childdocmain}
To use the package, add the commands
\begin{center}
\begin{tabular}{l}
|\input{childdoc.def}|\\
|\childdocmain{}|\\
\end{tabular}
\end{center}
at the very top of the main \LaTeX{} file,
in particular \emph{before} the |\documentclass| statement!
The argument of |\childdocmain| should be left empty
(but it must be present).

%%%%%%%%%%%%%%%%%%%%%%%%%%%%%%%%%%%%%%%%
\DescribeMacro{\childdocof}
Furthermore, add the commands
\begin{center}
\begin{tabular}{l}
|\input{childdoc.def}|\\
|\childdocof{|\textit{main}|}|\\
\end{tabular}
\end{center}
at the top of every child file \textit{child}
which is included by |\include{|\textit{child}|}|
from within the main file
(or at least for those files to be compiled individually).
The argument \textit{main} must be the filename of the main file.

There are a couple of
considerations in setting up the main and child documents:

%%%%%%%%%%%%%%%%%%%%%%%%%%%%%%%%%%%%%%%%
\paragraph{Restrictions.}

Please note the following restrictions:
\begin{itemize}
\item
|\childdocmain| must be called with one argument \textit{main}
to ensure compatibility with earlier version of the package.
It must either be empty (|\childdocmain{}|)
or precisely match the filename of the main file in which it is specified.
See \secref{sec:detection} for further information.
\item
The filename \textit{main} must be specified without the |.tex| extension.
\item
The filename \textit{main} is case sensitive
(even in case-insensitive file systems)
due to internal string comparison.
\item
The argument \textit{main} should be fully expanded, it cannot be a macro.
\item
Subdirectories and special characters should be avoided in filenames.
\item
The command |\childdocmain{|\textit{main}|}| must be followed by a whitespace.
It should not be followed immediately by another command
or by a comment mark `|%|'.
This is because the \TeX{} parser reads the token immediately following
the argument of |\childdocmain| and puts it
at the beginning of every child section;
however, a white\-space is ignored.
\end{itemize}

%%%%%%%%%%%%%%%%%%%%%%%%%%%%%%%%%%%%%%%%
\paragraph{Content of Main File.}

It is advisable to place all content in the child files included by |\include|.
Any output contained in the main file will appear in all child documents
unless suppressed manually;
it cannot be suppressed automatically by the |\includeonly| directive
and thus should normally be avoided.
A method to include some content in the main file
by means of conditional processing is described in \secref{sec:conditional}.

%%%%%%%%%%%%%%%%%%%%%%%%%%%%%%%%%%%%%%%%
\paragraph{Page Numbering.}

When only a part of the document is compiled,
the appropriate numbering of pages
(as well as other status parameters)
is determined from the |.aux| files.
The latter contain information from previous passes.
However this information needs to propagate through
all intermediate child documents.
Therefore the page numbering in child documents may well
be inconsistent until the complete document is compiled at least once.

A useful (if unconventional) way to always ensure a consistent
page numbering is to restart the numbering in each child document
and denote the pages by `\textit{child}|.|\textit{page}'
where \textit{child} represents the chapter/section number of the child file.
This can be achieved by the command
|\numberwithin{page}{|\textit{child}|}|
of the \textsf{amsmath} package
where \textit{child} can be |chapter| or |section|
depending on the chosen structuring.
Alternatively, one can modify the macro |\thepage| appropriately
and reset the counter |page| at the start of each child file.

%%%%%%%%%%%%%%%%%%%%%%%%%%%%%%%%%%%%%%%%%%%%%%%%%%%%%%%%%%%%%%%%%%%%%%%%%%%%%%%%
\subsection{Conditional Processing}
\label{sec:conditional}

The package provides a mechanism to compile different versions
of a document. To customise the versions further some conditional processing
can come in handy to distinguish which version is being compiled.
The package provides two macros to describe the compilation context:

%%%%%%%%%%%%%%%%%%%%%%%%%%%%%%%%%%%%%%%%
\DescribeMacro{\ifchilddoc}
The conditional |\ifchilddoc| distinguishes between the compilation of
child documents and the main document:
%
\begin{center}
|\ifchilddoc |\textit{child-code}| |[|\||else |\textit{main-code}]| \||fi|
\end{center}

%%%%%%%%%%%%%%%%%%%%%%%%%%%%%%%%%%%%%%%%
\DescribeMacro{\childdocname}
\DescribeMacro{\childdocjob}
The macro |\childdocname| contains the filename (without extension)
of the main or child file being processed.
Note that |\childdocjob| will always contain the name of the main file.

%%%%%%%%%%%%%%%%%%%%%%%%%%%%%%%%%%%%%%%%
\paragraph{Title Page.}

Conditional processing can be used to include a title or banner page
in the main document when proper precautions are taken.
Importantly, the code in the main file should ensure that the page counter
(as well as other status parameters which are stored in the |.aux| files)
takes the same value after the conditional processing.
Otherwise the page numbers may take divergent values
depending on which part is compiled.

For example, a title page could be declared by:
%
\begin{center}
\begin{tabular}{l}
|\ifchilddoc\||else|\\
|\addtocounter{page}{-1}|\\
\textit{code for title page}\\
|\newpage|\\
|\||fi|
\end{tabular}
\end{center}
%
A banner page for the child documents can be generated by:
%
\begin{center}
\begin{tabular}{l}
|\ifchilddoc|\\
|\addtocounter{page}{-1}|\\
\textit{code for banner page}\\
|\newpage|\\
|\||fi|
\end{tabular}
\end{center}
%
Here one could write a message such as:
\begin{center}
|This is the part \childdocname{} of \childdocjob{}.|
\end{center}

%%%%%%%%%%%%%%%%%%%%%%%%%%%%%%%%%%%%%%%%%%%%%%%%%%%%%%%%%%%%%%%%%%%%%%%%%%%%%%%%
\subsection{Flags}
\label{sec:flags}

The package makes it easy to generate different versions
of the main or child documents.
To this end compilation flags can be defined
and assigned different default values.
They will be particularly useful in conjunction
with the forwarding mechanism described in \secref{sec:forward}.

For example, it may be useful to have a flag |\version|
which can be set to |draft| or |final|.
The document source will contain some conditional code
depending on the value of |\version|.
Suppose further, the flag should default to |final| for the main file
and to |draft| for child files
which is a natural assignment for editing the document.
This is achieved by placing the following code
in the preamble of the main document
(below the |\childdocmain| directive):
%
\begin{center}
\begin{tabular}{l}
|\ifchilddoc|\\
|\providecommand{\version}{draft}|\\
|\||else|\\
|\providecommand{\version}{final}|\\
|\||fi|
\end{tabular}
\end{center}
%
The definition by |\providecommand| makes sure
that previous definitions are not overwritten.
Further statements |\providecommand{\version}{...}|
can thus be added before the above code to override it.

For the main file, one might add a line
(between |\childdocmain| and the above block)
%
\begin{center}
|%\ifchilddoc\||else\providecommand{\version}{draft}\||fi|
\end{center}
%
which can be uncommented to produce a draft version.
Likewise one can add a line to the very top of a child file
(above the |\childdocof{|\textit{main}|}| directive)
%
\begin{center}
|%\providecommand{\version}{final}|
\end{center}
%
which can be uncommented to produce the final version of this child document.

%%%%%%%%%%%%%%%%%%%%%%%%%%%%%%%%%%%%%%%%%%%%%%%%%%%%%%%%%%%%%%%%%%%%%%%%%%%%%%%%
\subsection{Forwarding}
\label{sec:forward}

Different versions of the main or child documents
using compilation flags as described in \secref{sec:flags}
can be (permanently) stored in different files
for convenient compilation, viewing and distribution.
To this end, the package defines a command
to pass on compilation to a different file:

%%%%%%%%%%%%%%%%%%%%%%%%%%%%%%%%%%%%%%%%
\DescribeMacro{\childdocforward}
The command |\childdocforward| redirects processing to
another source file:
%
\begin{center}
\begin{tabular}{l}
|\input{childdoc.def}|\\
|\childdocforward[|\textit{main}|]{|\textit{dest}|}|\\
\end{tabular}
\end{center}
%
The argument \textit{dest} is the destination file
(without extension).
It should be the main file or one of the child files.
Note that further \textsf{childdoc} directives
such as |\childdocof| and |\childdocforward|
in the indicated file will be processed in this form.
The optional argument \textit{main}
passes on directly to the main file \textit{main}
while pretending to compile the child \textit{dest}.
This form behaves as if \textit{dest}
issues |\childdocof{|\textit{main}|}| right away,
and no further \textsf{childdoc} directives will be processed.

%%%%%%%%%%%%%%%%%%%%%%%%%%%%%%%%%%%%%%%%
\DescribeMacro{\...prefix}
In the alternative form |\childdocforwardprefix|,
%
\begin{center}
\begin{tabular}{l}
|\input{childdoc.def}|\\
|\childdocforwardprefix[|\textit{main}|]{|\textit{prefix}|}{|\textit{dest}|}|
\end{tabular}
\end{center}
%
the destination file is determined by a pattern
depending on the current file:
To make this work, the current file must be called
`{\textit{prefix}\hspace{0.2em}\textit{suffix}}'
with \textit{prefix} matching precisely the argument.
Processing is then passed on to the file
`{\textit{dest}\hspace{0.2em}\textit{suffix}}'.
Surely, the same effect is achieved by
directly specifying the
argument `{\textit{dest}\hspace{0.2em}\textit{suffix}}'
in the first form.
However, that requires to set up a different file
for each child. With the alternative form of the command
all these files can have exactly the same content
which simplifies setting them up and maintaining them.

For example, the following file |draft.tex|
with a compilation flag |\version| as described in \secref{sec:flags}
compiles the main document as a draft:
%
\begin{center}
\begin{tabular}{l}
|\def\version{draft}|\\
|\input{childdoc.def}|\\
|\childdocforward{|\textit{main}|}|
\end{tabular}
\end{center}
%
Likewise, the following files |final|\textit{nn}|.tex|
compile the final version of the child document
|child|\textit{nn}|.tex|:
%
\begin{center}
\begin{tabular}{l}
|\def\version{final}|\\
|\input{childdoc.def}|\\
|\childdocforwardprefix{final}{child}|
\end{tabular}
\end{center}
%

Note that when several versions of a main file and/or of each child file
are to be generated, it may be convenient to set up a |Makefile| or
shell script to automatise the process.

%%%%%%%%%%%%%%%%%%%%%%%%%%%%%%%%%%%%%%%%%%%%%%%%%%%%%%%%%%%%%%%%%%%%%%%%%%%%%%%%
\subsection{Command Line Processing}
\label{sec:commandline}

The effect of redirection files can also be achieved by invoking
the \LaTeX{} compiler with a more elaborate command line.
Most conveniently this should be done as part
of a shell script or a |Makefile|.

When using \textsf{childdoc} in the main file, the following
command lines effectively perform a redirection
(note that depending on the shell being used,
backslashes may have to be doubled: `|\|' $\to$ `|\\|'):
%
\begin{center}
|... -jobname "|\textit{target}|" |\\|"|[\textit{flags}]%
|\input{childdoc.def}\childdocforward[|\textit{main}|]{|\textit{dest}|}"|
\end{center}
%
Here \textit{target} is the name of the output file,
\textit{main} is the name of the main file
and \textit{dest} is the name of the main or child file to be processed
(all filenames without extensions).
The optional argument \textit{main} can be omitted
if \textit{main} matches \textit{dest}.
Optionally, compilation \textit{flags} can be defined via |\def| commands.
This command line makes the \TeX{} engine believe
it is compiling the file \textit{target}
whose content is specified as the latter parameter.
The provided code then forwards the processing to
\textit{main} or \textit{dest} as described in \secref{sec:forward}.

%%%%%%%%%%%%%%%%%%%%%%%%%%%%%%%%%%%%%%%%%%%%%%%%%%%%%%%%%%%%%%%%%%%%%%%%%%%%%%%%
\subsection{Include by Input}
\label{sec:input}

Including child documents by |\include| has some restrictions by design.
Most notably, the content of a child document always occupies
its own set of pages; pages cannot be shared between child documents.
Usually, this behaviour makes perfect sense
because each child document contain an essential part of the document.
However, in some situations it may be desirable to compose
a document from a collection of parts
without having mandatory page breaks between then.
For this case, the package
provides a mechanism to include parts
by |\input| which can also be processed individually.
However, by construction this mechanism
requires manual handling of the content to be output.

%%%%%%%%%%%%%%%%%%%%%%%%%%%%%%%%%%%%%%%%
\DescribeMacro{\ifchilddocmanual}
The main file should be prepared as usual, see \secref{sec:include}.
However, the document body must make a distinction
between processing of an individual part and of the main document, e.g.:
%
\begin{center}
\begin{tabular}{l}
|\ifchilddocmanual|\\
|\input{\childdocname}|\\
|\||else|\\
\textit{document body with }|\input{|\textit{part}|}|\\
|\||fi|
\end{tabular}
\end{center}
%
The conditional |\ifchilddocmanual| is true whenever
a part to be included by |\input| is being compiled,
and the name of the part is stored in |\childdocname|.

%%%%%%%%%%%%%%%%%%%%%%%%%%%%%%%%%%%%%%%%
\DescribeMacro{\childdocby}
Each part to be included by |\input| should start with:
%
\begin{center}
\begin{tabular}{l}
|\input{childdoc.def}|\\
|\childdocby{|\textit{main}|}|\\
\end{tabular}
\end{center}
%
The directive |\childdocby| is similar to |\childdocof|
described in \secref{sec:include},
but the subsequent selection of content must be done manually.
To that end, both |\ifchilddoc| and |\ifchilddocmanual|
will be true upon processing of a part,
and the name of the part is stored in |\childdocname|.
Note that |\jobname| will be set to the filename of the current part
so that each part receives an individual |.aux| file
that does not interfere with the |.aux| file(s) of the main document.
This behaviour can be altered by the alternative form
|\childdocby[*]{|\textit{main}|}| (with a non-empty optional argument)
which uses the |.aux| file of the main document
by setting |\jobname| to \textit{main}.

%%%%%%%%%%%%%%%%%%%%%%%%%%%%%%%%%%%%%%%%%%%%%%%%%%%%%%%%%%%%%%%%%%%%%%%%%%%%%%%%
\subsection{Driver Development}
\label{sec:driver}

The \textsf{childdoc} mechanism can also be use for the development
of definition files such as \LaTeX{} styles or classes.
This case differs from the above setup with multiple parts
included by |\include| in that no |\includeonly| should be invoked.
This can be achieved by starting the include file
(before |\ProvidesPackage|) with:
%
\begin{center}
\begin{tabular}{l}
|\input{childdoc.def}|\\
|\childdocforward{|\textit{main}|}|\\
\end{tabular}
\end{center}
%
or alternatively with:
%
\begin{center}
\begin{tabular}{l}
|\input{childdoc.def}|\\
|\childdocby{|\textit{main}|}|\\
\end{tabular}
\end{center}
%
Both forms have slightly different effects as described above.
The main file is prepared as usual, see \secref{sec:include}.

%%%%%%%%%%%%%%%%%%%%%%%%%%%%%%%%%%%%%%%%%%%%%%%%%%%%%%%%%%%%%%%%%%%%%%%%%%%%%%%%
\subsection{Legacy Detection}
\label{sec:detection}

The directive |\childdocmain| in the main file can detect
whether the complete document or merely a child is to be compiled
even without using the directive |\childdocof|.
This method is deprecated because it is less robust
and there is no compelling reason to use it;
it is merely provided for backward compatibility
and it may be removed in future versions.

If the detection mechanism is to be used,
it is mandatory to correctly specify
the filename of the main file as the argument of |\childdocmain|:
%
\begin{center}
\begin{tabular}{l}
|\input{childdoc.def}|\\
|\childdocmain{|\textit{main}|}|\\
\end{tabular}
\end{center}
%
If |\jobname| does not match the argument \textit{main} of |\childdocmain|,
it is assumed that |\jobname| points to the child file to be compiled.
When using |\childdocmain| with the main file specified as argument,
it suffices to start a child file
with just |\input{|\textit{main}|}|
without loading of the package and using |\childdocof|.
If instead all processing is done
with the appropriate \textsf{childdoc} directives,
the argument of \textit{main} of |\childdocmain| can be empty.

An alternative version of the command line processing described
in \secref{sec:commandline} using the detection mechanism reads:
%
\begin{center}
|... -jobname "|\textit{target}|" "|[\textit{flags}]%
[|\def\jobname{|\textit{dest}|}|]|\input{|\textit{main}|}"|
\end{center}

%%%%%%%%%%%%%%%%%%%%%%%%%%%%%%%%%%%%%%%%%%%%%%%%%%%%%%%%%%%%%%%%%%%%%%%%%%%%%%%%
\subsection{Manual Code}
\label{sec:manual}

In case one cannot be certain whether the definitions file |childdoc.def|
is installed on the target \TeX{} distribution
and one prefers not to ship it,
it is conceivable to paste a few relevant commands into the sources.

To that end, drop all statements |\input{childdoc.def}|
and perform the replacements as outlined below.
Instead of |\childdocmain{|\textit{main}|}| add the following code
to the top of the main file:
%
\begin{center}
\begin{tabular}{l}
|\||ifdefined\childdocname\endinput\||fi\newif\ifchilddoc|\\
|\edef\childdocname{\scantokens\expandafter{\jobname\noexpand}}|\\
|\def\childdocmain{|\textit{main}|}\||ifx\childdocmain\childdocname\||else|\\
|\childdoctrue\includeonly{\childdocname}\let\jobname\childdocmain\||fi|\\
\end{tabular}
\end{center}
%
Instead of |\childdocof{|\textit{main}|}| just include the main file
at the top of each child file:
%
\begin{center}
|\input{|\textit{main}|}|
\end{center}
%
A simple redirection |\childdocforward{|\textit{dest}|}| is achieved by:
%
\begin{center}
|\def\jobname{|\textit{dest}|}\input{\jobname}|
\end{center}
%
The redirection with prefix
|\childdocforwardprefix[|\textit{prefix}|]{|\textit{dest}|}|
is accomplished by:
%
\begin{center}
\begin{tabular}{l}
|{\edef\jobname{\scantokens\expandafter{\jobname\noexpand}}|\\
|\def\redirectjob |\textit{prefix}|#1~~~{\gdef\jobname{|\textit{dest}|#1}}|\\
|\expandafter\redirectjob\jobname~~~}\input{\jobname}|
\end{tabular}
\end{center}

In an alternative approach,
child documents can be compiled by a specific command line
without additional code or specific definitions:
%
\begin{center}
|... -jobname "|\textit{target}|" "|[\textit{flags}]%
|\includeonly{|\textit{dest}|}\input{|\textit{main}|}"|
\end{center}
%

%%%%%%%%%%%%%%%%%%%%%%%%%%%%%%%%%%%%%%%%%%%%%%%%%%%%%%%%%%%%%%%%%%%%%%%%%%%%%%%%
%%%%%%%%%%%%%%%%%%%%%%%%%%%%%%%%%%%%%%%%%%%%%%%%%%%%%%%%%%%%%%%%%%%%%%%%%%%%%%%%
\section{Information}

%%%%%%%%%%%%%%%%%%%%%%%%%%%%%%%%%%%%%%%%%%%%%%%%%%%%%%%%%%%%%%%%%%%%%%%%%%%%%%%%
\subsection{Copyright}

Copyright \copyright{} 2017--2018 Niklas Beisert

This work may be distributed and/or modified under the
conditions of the \LaTeX{} Project Public License, either version 1.3
of this license or (at your option) any later version.
The latest version of this license is in
  \url{http://www.latex-project.org/lppl.txt}
and version 1.3 or later is part of all distributions of \LaTeX{}
version 2005/12/01 or later.

This work has the LPPL maintenance status `maintained'.

The Current Maintainer of this work is Niklas Beisert.

This work consists of the files |README.txt|, |childdoc.ins| and |childdoc.dtx|
as well as the derived files |childdoc.def|, |cdocsamp.tex|
with |cdocsch1.tex|, |cdocsch2.tex|, |cdocspt3.tex|, |cdocspt4.tex|,
|cdocsdrf.tex|, |cdocsfn1.tex|, |cdocsfn2.tex|
as well as |childdoc.pdf|.

%%%%%%%%%%%%%%%%%%%%%%%%%%%%%%%%%%%%%%%%%%%%%%%%%%%%%%%%%%%%%%%%%%%%%%%%%%%%%%%%
\subsection{Files and Installation}

The package consists of the files:
%
\begin{center}
\begin{tabular}{ll}
    |README.txt|   & readme file \\
    |childdoc.ins| & installation file \\
    |childdoc.dtx| & source file \\
    |childdoc.def| & definition file \\
    |cdocsamp.tex| & sample main file \\
    |cdocsch1.tex| & sample include file \\
    |cdocsch2.tex| & sample include file \\
    |cdocspt3.tex| & sample part file \\
    |cdocspt4.tex| & sample part file \\
    |cdocsdrf.tex| & sample redirection file \\
    |cdocsfn1.tex| & sample redirection file \\
    |cdocsfn2.tex| & sample redirection file \\
    |childdoc.pdf| & manual
\end{tabular}
\end{center}
%
The distribution consists of the files
|README.txt|, |childdoc.ins| and |childdoc.dtx|.
%
\begin{itemize}
\item
Run (pdf)\LaTeX{} on |childdoc.dtx|
to compile the manual |childdoc.pdf| (this file).
\item
Run \LaTeX{} on |childdoc.ins| to create the definitions file |childdoc.def|
and the sample |cdocsamp.tex| with include files
|cdocsch1.tex|, |cdocsch2.tex|, |cdocspt3.tex|, |cdocspt4.tex|,
|cdocsdrf.tex|, |cdocsfn1.tex|, |cdocsfn2.tex|.
Then copy the file |childdoc.def| to an appropriate directory of your \LaTeX{}
distribution, e.g.\ \textit{texmf-root}|/tex/latex/childdoc|.
\end{itemize}

%%%%%%%%%%%%%%%%%%%%%%%%%%%%%%%%%%%%%%%%%%%%%%%%%%%%%%%%%%%%%%%%%%%%%%%%%%%%%%%%
\subsection{Related CTAN Packages}

There are several other packages which offer a similar functionality:
%
\begin{itemize}
\item
The packages
\href{http://ctan.org/pkg/docmute}{\textsf{docmute}},
\href{http://ctan.org/pkg/includex}{\textsf{includex}} and
\href{http://ctan.org/pkg/standalone}{\textsf{standalone}}
provide commands to include only the document body of
a child file thus allowing both files to be compiled individually.
\item
The packages \href{http://ctan.org/pkg/subdocs}{\textsf{subdocs}}
and \href{http://ctan.org/pkg/subfiles}{\textsf{subfiles}}
provide structures in which the main and child documents can be
encapsulated and allowing them to be compiled individually.
The inclusion mechanism is different from the conventional |\include|.
\item
The package \href{http://ctan.org/pkg/combine}{\textsf{combine}}
is an elaborate solution to combine several documents into one.
\end{itemize}
%
See also the CTAN topic \href{http://ctan.org/topic/subdocs}{\textsf{subdocs}}
for further related packages.
The present package differs from the above solutions in that
a document structure constructed with the conventional |\include| mechanism
just needs two extra commands at the top of every file
such that all constituent files can be compiled individually.

%%%%%%%%%%%%%%%%%%%%%%%%%%%%%%%%%%%%%%%%%%%%%%%%%%%%%%%%%%%%%%%%%%%%%%%%%%%%%%%%
%\subsection{Feature Suggestions}
%
%The following is a list of features which may be useful for future
%versions of this package:
%%
%\begin{itemize}
%\item
%\ldots
%\end{itemize}

%%%%%%%%%%%%%%%%%%%%%%%%%%%%%%%%%%%%%%%%%%%%%%%%%%%%%%%%%%%%%%%%%%%%%%%%%%%%%%%%
\subsection{Revision History}

%%%%%%%%%%%%%%%%%%%%%%%%%%%%%%%%%%%%%%%%
\paragraph{v2.0:} 2018/12/30

\begin{itemize}
\item
immediate forward processing
\item
added |\childdocby| mechanism
\item
manual restructured
\end{itemize}

%%%%%%%%%%%%%%%%%%%%%%%%%%%%%%%%%%%%%%%%
\paragraph{v1.6:} 2018/01/17

\begin{itemize}
\item
application for development of include files
\item
corrections to manual
\end{itemize}

%%%%%%%%%%%%%%%%%%%%%%%%%%%%%%%%%%%%%%%%
\paragraph{v1.5:} 2017/05/21

\begin{itemize}
\item
more complete structuring introduced
\item
|\childdocof| introduced
\item
|\childdoc| renamed to |\childdocmain|
\item
|\childredirect| renamed to |\childdocforward| and |\childdocforwardprefix|
and functionality expanded
\end{itemize}

%%%%%%%%%%%%%%%%%%%%%%%%%%%%%%%%%%%%%%%%
\paragraph{v1.0:} 2017/04/27

\begin{itemize}
\item
manual and install package
\item
first version published on CTAN
\end{itemize}

%%%%%%%%%%%%%%%%%%%%%%%%%%%%%%%%%%%%%%%%
\paragraph{v0.6:} 2017/04/26

\begin{itemize}
\item
redirection mechanism added
\end{itemize}

%%%%%%%%%%%%%%%%%%%%%%%%%%%%%%%%%%%%%%%%
\paragraph{v0.5:} 2017/04/26

\begin{itemize}
\item
functionality in definition file
\end{itemize}


%%%%%%%%%%%%%%%%%%%%%%%%%%%%%%%%%%%%%%%%%%%%%%%%%%%%%%%%%%%%%%%%%%%%%%%%%%%%%%%%
%%%%%%%%%%%%%%%%%%%%%%%%%%%%%%%%%%%%%%%%%%%%%%%%%%%%%%%%%%%%%%%%%%%%%%%%%%%%%%%%
%%%%%%%%%%%%%%%%%%%%%%%%%%%%%%%%%%%%%%%%%%%%%%%%%%%%%%%%%%%%%%%%%%%%%%%%%%%%%%%%
\appendix

\settowidth\MacroIndent{\rmfamily\scriptsize 000\ }

 \DocInput{childdoc.dtx}

\end{document}
%</driver>
% \fi
%
% %%%%%%%%%%%%%%%%%%%%%%%%%%%%%%%%%%%%%%%%%%%%%%%%%%%%%%%%%%%%%%%%%%%%%%%%%%%%%%
% %%%%%%%%%%%%%%%%%%%%%%%%%%%%%%%%%%%%%%%%%%%%%%%%%%%%%%%%%%%%%%%%%%%%%%%%%%%%%%
% \section{Sample}
%\iffalse
%<*samplemain>
%\fi
%
% The following presents a sample document
% with two chapters, two parts, a title page,
% a compile flag as well as three forwarding files to set the flag.
% It consists of eight |.tex| files:
% \begin{center}
% \begin{tabular}{ll}
% |cdocsamp.tex|&main file\\
% |cdocsch1.tex|&include file for chapter 1\\
% |cdocsch2.tex|&include file for chapter 2\\
% |cdocspt3.tex|&include file for part 3\\
% |cdocspt4.tex|&include file for part 4\\
% |cdocsdrf.tex|&forwarding file for main file in draft mode\\
% |cdocsfi1.tex|&forwarding file for final version of chapter 1\\
% |cdocsfi2.tex|&forwarding file for final version of chapter 2\\
% \end{tabular}
% \end{center}
% Each of the eight files can be compiled directly by the \LaTeX{} compiler.
%
% %%%%%%%%%%%%%%%%%%%%%%%%%%%%%%%%%%%%%%
% \paragraph{Main File.}
%
% The main file is called |cdocsamp.tex|.
%
% Load the \textsf{childdoc} definitions and
% declare the filename for the main document:
%    \begin{macrocode}
\input{childdoc.def}
\childdocmain{}
%    \end{macrocode}

% Optional override for |\version| flag:
%    \begin{macrocode}
%%\ifchilddoc\else\providecommand{\version}{draft}\fi
%    \end{macrocode}

% Define the default values for the |\version| flag
% (|final| for the main file and |draft| for childs):
%    \begin{macrocode}
\ifchilddoc
\providecommand{\version}{draft}
\else
\providecommand{\version}{final}
\fi
%    \end{macrocode}

% Load the standard document class:
%    \begin{macrocode}
\documentclass[12pt]{article}
%    \end{macrocode}

% Start the document body:
%    \begin{macrocode}
\begin{document}
%    \end{macrocode}

% Declare a title page.
% Print title, part of document being processed and version flag:
%    \begin{macrocode}
\addtocounter{page}{-1}
\begin{center}
{\LARGE\bfseries{}childdoc example\par}
\vspace{1cm}
\ifchilddoc
\ifchilddocmanual part\else chapter\fi:
`\childdocname' of `\childdocjob'\par
\else
main document: `\childdocjob'\par
\fi
version: \version\par
\end{center}
\newpage
%    \end{macrocode}

% Manually include selected file,
% otherwise process as usual:
%    \begin{macrocode}
\ifchilddocmanual
\section*{part `\childdocname'}
\input{\childdocname}
\else
%    \end{macrocode}

% Include the two chapters:
%    \begin{macrocode}
\include{cdocsch1}
\include{cdocsch2}
%    \end{macrocode}

% Include the two parts unless only chapters should be displayed:
%    \begin{macrocode}
\ifchilddoc\else
\section{part three}
\input{cdocspt3}
\section{part four}
\input{cdocspt4}
\fi
%    \end{macrocode}

% Process as usual until here:
%    \begin{macrocode}
\fi
%    \end{macrocode}

% End of document body:
%    \begin{macrocode}
\end{document}
%    \end{macrocode}
%\iffalse
%</samplemain>
%\fi
%
% %%%%%%%%%%%%%%%%%%%%%%%%%%%%%%%%%%%%%%
% \paragraph{Chapter Include Files.}
%
% The include files are called |cdocsch1.tex| and |cdocsch2.tex|.
%
%\iffalse
%<*samplechap1|samplechap2>
%\fi

% Optional override for |\version| flag:
%    \begin{macrocode}
%%\providecommand{\version}{final}
%    \end{macrocode}

% Include the main document:
%    \begin{macrocode}
\input{childdoc.def}
\childdocof{cdocsamp}
%    \end{macrocode}

%\iffalse
%</samplechap1|samplechap2>
%\fi
%
%\iffalse
%<*samplechap1>
%\fi
% Some text for chapter 1:
%    \begin{macrocode}
\section{one}
some text in chapter one
%    \end{macrocode}

%\iffalse
%</samplechap1>
%\fi
% Some text for chapter 2:
%\iffalse
%<*samplechap2>
%\fi
%    \begin{macrocode}
\section{two}
more text in chapter two
%    \end{macrocode}

%\iffalse
%</samplechap2>
%\fi
%
% %%%%%%%%%%%%%%%%%%%%%%%%%%%%%%%%%%%%%%
% \paragraph{Part Include Files.}
%
% The include files are called |cdocspt3.tex| and |cdocspt4.tex|.
%
%\iffalse
%<*samplepart3|samplepart4>
%\fi

% Optional override for |\version| flag:
%    \begin{macrocode}
%%\providecommand{\version}{final}
%    \end{macrocode}

% Include the main document:
%    \begin{macrocode}
\input{childdoc.def}
\childdocby{cdocsamp}
%    \end{macrocode}

%\iffalse
%</samplepart3|samplepart4>
%\fi
%
%\iffalse
%<*samplepart3>
%\fi
% Some text for part 3:
%    \begin{macrocode}
some text in part three
%    \end{macrocode}

%\iffalse
%</samplepart3>
%\fi
% Some text for part 4:
%\iffalse
%<*samplepart4>
%\fi
%    \begin{macrocode}
more text in part four
%    \end{macrocode}

%\iffalse
%</samplepart4>
%\fi
%
% %%%%%%%%%%%%%%%%%%%%%%%%%%%%%%%%%%%%%%
% \paragraph{Forwarding for a Complete Draft.}
%
% The following forwarding file |cdocsdrf.tex|
% compiles the main document in draft mode:
%\iffalse
%<*sampledraft>
%\fi
%    \begin{macrocode}
\def\version{draft}
\input{childdoc.def}
\childdocforward{cdocsamp}
%    \end{macrocode}

%\iffalse
%</sampledraft>
%\fi
%
% %%%%%%%%%%%%%%%%%%%%%%%%%%%%%%%%%%%%%%
% \paragraph{Forwarding for Final Version of the Chapters.}
%
% The following forwarding files |cdocsfn1.tex| and |cdocsfn2.tex|
% (with identical content)
% compile the final versions of the child documents
% |cdocsch1.tex| and |cdocsch2.tex|, respectively:
%\iffalse
%<*samplefinal>
%\fi
%    \begin{macrocode}
\def\version{final}
\input{childdoc.def}
\childdocforwardprefix[cdocsamp]{cdocsfn}{cdocsch}
%    \end{macrocode}

%\iffalse
%</samplefinal>
%\fi
%
% %%%%%%%%%%%%%%%%%%%%%%%%%%%%%%%%%%%%%%
% \paragraph{Command Line Processing.}
%
% The following three command lines generate the output files
% |cdocscld|, |cdocscl1| and |cdocscl2|
% which should be identical to
% |cdocsdrf|, |cdocsch1| and |cdocsfn2|, respectively:
% \begin{center}
% \begin{tabular}{l}
% |latex -jobname cdocscld \|\\
% |  "\def\version{draft}\input{childdoc.def}\childdocforward{cdocsamp}"|\\
% |latex -jobname cdocscl1 \|\\
% |  "\input{childdoc.def}\childdocforward[cdocsamp]{cdocsch1}"|\\
% |latex -jobname cdocscl2 \|\\
% |  "\def\version{final}\input{childdoc.def}\childdocforward{cdocsch2}"|
% \end{tabular}
% \end{center}
% Note that the trailing backslash on each first line
% merely continues the input to the second line
% (for convenient cut ant paste).
% Furthermore, the command |latex| can be replaced by any
% of its alternative versions such as |pdflatex|.
%
% %%%%%%%%%%%%%%%%%%%%%%%%%%%%%%%%%%%%%%%%%%%%%%%%%%%%%%%%%%%%%%%%%%%%%%%%%%%%%%
% %%%%%%%%%%%%%%%%%%%%%%%%%%%%%%%%%%%%%%%%%%%%%%%%%%%%%%%%%%%%%%%%%%%%%%%%%%%%%%
% \section{Implementation}
%\iffalse
%<*package>
%\fi
%
% This section describes the definitions file |childdoc.def|.

% The definitions cannot be loaded using |\usepackage| or |\RequirePackage|
% which has a mechanism to prevent loading a style file more than once.
% When loading the definitions by means of |\input|
% multiple instances have to be prevented manually:
%\iffalse
%This code needs to be before the `\ProvidesFile' directive
%which is defined at the beginning of this file.
%Therefore it is also placed there and commented out here.
%</package>
%<*discard>
%\fi
%    \begin{macrocode}
\ifdefined\childdocmain\endinput\fi
%    \end{macrocode}
%\iffalse
%</discard>
%<*package>
%\fi
%
% \macro{\ifchilddoc}
% \macro{\ifchilddocmanual}
% The conditional |\ifchilddoc| tells whether a
% child (true) or main (false) document is being compiled.
% The conditional |\ifchilddocmanual| tells whether
% the |\includeonly| mechanism is used (false) or
% the selection of child files must be performed manually (true).
% The definitions initialise to false:
%    \begin{macrocode}
\newif\ifchilddoc
\newif\ifchilddocmanual
%    \end{macrocode}

% \macro{\childdocname}
% \macro{\childdocjob}
% The macro |\childdocname| stores the name of the main document
% to be compiled. The macro |\childdocjob| stores the name of
% the document on which the \LaTeX{} compiler was originally invoked.
% The content of |\jobname| cannot be compared
% to filenames specified in the source due to different catcodes.
% The following code rescans |\jobname|, stores the result
% in |\childdocname| and saves a copy in |\childdocjob|:
%    \begin{macrocode}
\edef\childdocname{\scantokens\expandafter{\jobname\noexpand}}
\let\childdocjob\childdocname
%    \end{macrocode}

% \macro{\childdocdisable}
% The macro |\childdocdisable| prevents the main file
% from being processed more than once.
% At this stage, the main document command |\childdocmain|
% is assumed to be called once again where it should do nothing.
% Any subsequent call to it should prevent
% a secondary processing of the main document
% It overwrites the forwarding commands
% |\childdocof| and |\childdocforward|
% with empty macros to prevent further inclusions of the main document:
%    \begin{macrocode}
\newcommand{\childdocdisable}
{
  \renewcommand{\childdocmain}[1]{\renewcommand{\childdocmain}[1]{\endinput}}
  \renewcommand{\childdocof}[1]{}
  \renewcommand{\childdocby}[2][]{}
  \renewcommand{\childdocforward}[2][]{}
  \renewcommand{\childdocdisable}{}
}
%    \end{macrocode}

% \macro{\childdocmain}
% The macro |\childdocmain| is to be called at the top of the main file
% with nothing or the main filename (without extension) as argument.
% First, it breaks loops.
% If the argument is not empty and does not match |\childdocname|
% (which is set by the first inclusion of |childdoc.def|),
% |\ifchilddoc| is set to true, |\includeonly| is applied to the child file
% and |\jobname| is set to the main file
% (for proper handling of |.aux| files):
%    \begin{macrocode}
\newcommand{\childdocmain}[1]
{
  \childdocdisable\childdocmain{}
  \if?#1?\else
    \begingroup
      \def\childdoctmp{#1}
      \ifx\childdoctmp\childdocname
        \def\childdoctmp{}
      \else
        \def\childdoctmp
        {
          \childdoctrue
          \includeonly{\childdocname}
          \def\childdocjob{#1}
          \def\jobname{#1}
        }
      \fi
      \expandafter
    \endgroup
    \childdoctmp
  \fi
}
%    \end{macrocode}

% \macro{\childdocof}
% The command |\childdocof| redirects
% compilation to the main file |#1|.
%    \begin{macrocode}
\newcommand{\childdocof}[1]
{
  \childdocdisable
  \childdoctrue
  \includeonly{\childdocname}
  \def\jobname{#1}
  \def\childdocjob{#1}
  \input{#1}
}
%    \end{macrocode}

% \macro{\childdocby}
% The command |\childdocby| ....
%    \begin{macrocode}
\newcommand{\childdocby}[2][]
{
  \childdocdisable
  \childdoctrue
  \childdocmanualtrue
  \if?#1?\else
    \def\jobname{#2}
  \fi
  \def\childdocjob{#2}
  \input{#2}
  \endinput
}
%    \end{macrocode}

% \macro{\childdocforward}
% The command |\childdocforward| redirects
% compilation to the main file or
% (if the optional argument is given) a child file.
% Parameters are set as if the main file
% or a child file starting with |\childdocof| was compiled.
% Then compilation is handed over to the main file:
%    \begin{macrocode}
\newcommand{\childdocforward}[2][]
{
  \begingroup
    \if?#1?
      \def\childdoctmp
      {
        \def\childdocname{#2}
        \def\childdocjob{#2}
        \def\jobname{#2}
        \input{#2}
        \endinput
      }
    \else
      \def\childdoctmp
      {
        \childdocdisable
        \def\childdocname{#2}
        \childdoctrue
        \includeonly{#2}
        \def\childdocjob{#1}
        \def\jobname{#1}
        \input{#1}
        \endinput
      }
    \fi
    \expandafter
  \endgroup
  \childdoctmp
}
%    \end{macrocode}

% \macro{\childdocforwardprefix}
% The command |\childdocforwardprefix| redirects
% compilation to the main or a child file by means of a pattern.
% The prefix |#1| in the current filename is replaced by |#2|
% and the suffix of the current filename is kept
% (it is assumed that the filename does not contain the substring `|~~~|'
% which is used as a delimiter).
% Compilation is handed over to the new file by |\childdocforward|:
%    \begin{macrocode}
\newcommand{\childdocforwardprefix}[3][]
{
  \begingroup
    \def\childdocextract #2##1~~~{\def\childdoctmp{\childdocforward[#1]{#3##1}}}
    \expandafter\childdocextract\childdocname~~~
    \expandafter
  \endgroup
  \childdoctmp
}
%    \end{macrocode}

% \macro{\childdoc}
% The deprecated macro |\childdoc| is a legacy version of |\childdocmain|:
%    \begin{macrocode}
\newcommand{\childdoc}{\childdocmain}
%    \end{macrocode}

% \macro{\childdocredirect}
% The deprecated macro |\childdocredirect| is a legacy version
% of |\childdocforward| and |\childdocforwardprefix|:
%    \begin{macrocode}
\newcommand{\childdocredirect}[2][]
{
  \begingroup
    \if?#1?
      \def\childdoctmp{\childdocforward{#2}}
    \else
      \def\childdoctmp{\childdocforwardprefix{#1}{#2}}
    \fi
    \expandafter
  \endgroup
  \childdoctmp
}
%    \end{macrocode}

%\iffalse
%</package>
%\fi
%
\endinput

\childdocmain{}
%    \end{macrocode}

% Optional override for |\version| flag:
%    \begin{macrocode}
%%\ifchilddoc\else\providecommand{\version}{draft}\fi
%    \end{macrocode}

% Define the default values for the |\version| flag
% (|final| for the main file and |draft| for childs):
%    \begin{macrocode}
\ifchilddoc
\providecommand{\version}{draft}
\else
\providecommand{\version}{final}
\fi
%    \end{macrocode}

% Load the standard document class:
%    \begin{macrocode}
\documentclass[12pt]{article}
%    \end{macrocode}

% Start the document body:
%    \begin{macrocode}
\begin{document}
%    \end{macrocode}

% Declare a title page.
% Print title, part of document being processed and version flag:
%    \begin{macrocode}
\addtocounter{page}{-1}
\begin{center}
{\LARGE\bfseries{}childdoc example\par}
\vspace{1cm}
\ifchilddoc
\ifchilddocmanual part\else chapter\fi:
`\childdocname' of `\childdocjob'\par
\else
main document: `\childdocjob'\par
\fi
version: \version\par
\end{center}
\newpage
%    \end{macrocode}

% Manually include selected file,
% otherwise process as usual:
%    \begin{macrocode}
\ifchilddocmanual
\section*{part `\childdocname'}
\input{\childdocname}
\else
%    \end{macrocode}

% Include the two chapters:
%    \begin{macrocode}
\include{cdocsch1}
\include{cdocsch2}
%    \end{macrocode}

% Include the two parts unless only chapters should be displayed:
%    \begin{macrocode}
\ifchilddoc\else
\section{part three}
\input{cdocspt3}
\section{part four}
\input{cdocspt4}
\fi
%    \end{macrocode}

% Process as usual until here:
%    \begin{macrocode}
\fi
%    \end{macrocode}

% End of document body:
%    \begin{macrocode}
\end{document}
%    \end{macrocode}
%\iffalse
%</samplemain>
%\fi
%
% %%%%%%%%%%%%%%%%%%%%%%%%%%%%%%%%%%%%%%
% \paragraph{Chapter Include Files.}
%
% The include files are called |cdocsch1.tex| and |cdocsch2.tex|.
%
%\iffalse
%<*samplechap1|samplechap2>
%\fi

% Optional override for |\version| flag:
%    \begin{macrocode}
%%\providecommand{\version}{final}
%    \end{macrocode}

% Include the main document:
%    \begin{macrocode}
% \iffalse
%
% childdoc.dtx Copyright (C) 2017-2018 Niklas Beisert
%
% This work may be distributed and/or modified under the
% conditions of the LaTeX Project Public License, either version 1.3
% of this license or (at your option) any later version.
% The latest version of this license is in
%   http://www.latex-project.org/lppl.txt
% and version 1.3 or later is part of all distributions of LaTeX
% version 2005/12/01 or later.
%
% This work has the LPPL maintenance status `maintained'.
%
% The Current Maintainer of this work is Niklas Beisert.
%
% This work consists of the files childdoc.dtx and childdoc.ins
% and the derived files childdoc.def and cdocsamp.tex with
% cdocsch1.tex, cdocsch2.tex, cdocsdrf.tex, cdocsfn1.tex, cdocsfn2.tex.
%
%<package>\ifdefined\childdocmain\endinput\fi
%<package>\ProvidesFile{childdoc.def}[2018/12/30 v2.0 child document driver]
%<samplemain>\ProvidesFile{cdocsamp.tex}[2018/12/30 v2.0 sample for childdoc]
%<*driver>
%\ProvidesFile{childdoc.drv}[2018/12/30 v2.0 childdoc reference manual file]
\PassOptionsToClass{10pt,a4paper}{article}
\documentclass{ltxdoc}

\usepackage[margin=35mm]{geometry}
\usepackage{hyperref}
\usepackage{hyperxmp}
\usepackage[usenames]{color}

\hypersetup{colorlinks=true}
\hypersetup{pdfstartview=FitH}
\hypersetup{pdfpagemode=UseNone}
\hypersetup{pdfsource={}}
\hypersetup{pdflang={en-UK}}
\hypersetup{pdfcopyright={Copyright 2017-2018 Niklas Beisert.
  This work may be distributed and/or modified under the
  conditions of the LaTeX Project Public License, either version 1.3
  of this license or (at your option) any later version.}}
\hypersetup{pdflicenseurl={http://www.latex-project.org/lppl.txt}}
\hypersetup{pdfcontactaddress={ETH Zurich, ITP, HIT K,
  Wolfgang-Pauli-Strasse 27}}
\hypersetup{pdfcontactpostcode={8093}}
\hypersetup{pdfcontactcity={Zurich}}
\hypersetup{pdfcontactcountry={Switzerland}}
\hypersetup{pdfcontactemail={nbeisert@itp.phys.ethz.ch}}
\hypersetup{pdfcontacturl={http://people.phys.ethz.ch/\xmptilde nbeisert/}}

\newcommand{\secref}[1]{\hyperref[#1]{section \ref*{#1}}}

\parskip1ex
\parindent0pt
\let\olditemize\itemize
\def\itemize{\olditemize\parskip0pt}

\begin{document}

\title{The \textsf{childdoc} Package}
\hypersetup{pdftitle={The childdoc Package}}
\author{Niklas Beisert\\[2ex]
  Institut f\"ur Theoretische Physik\\
  Eidgen\"ossische Technische Hochschule Z\"urich\\
  Wolfgang-Pauli-Strasse 27, 8093 Z\"urich, Switzerland\\[1ex]
  \href{mailto:nbeisert@itp.phys.ethz.ch}
  {\texttt{nbeisert@itp.phys.ethz.ch}}}
\hypersetup{pdfauthor={Niklas Beisert}}
\hypersetup{pdfsubject={Manual for the LaTeX2e Package childdoc}}
\date{30 December 2018, \textsf{v2.0}}
\maketitle

\begin{abstract}\noindent
\textsf{childdoc} is a \LaTeXe{} package
that enables the direct compilation
of document sections included by |\include|
to individual files.
\end{abstract}

\begingroup
\parskip0ex
\tableofcontents
\endgroup

%%%%%%%%%%%%%%%%%%%%%%%%%%%%%%%%%%%%%%%%%%%%%%%%%%%%%%%%%%%%%%%%%%%%%%%%%%%%%%%%
%%%%%%%%%%%%%%%%%%%%%%%%%%%%%%%%%%%%%%%%%%%%%%%%%%%%%%%%%%%%%%%%%%%%%%%%%%%%%%%%
\section{Introduction}

\LaTeX{} provides a mechanism to structure a large document (such as a book)
into a main file and several child files (containing the chapters)
using the |\include| command.
This mechanism is beneficial for documents
which span hundreds of pages in order to
make the source file(s) more manageable.
Moreover, compilation can be restricted to
selected child files by means of the |\includeonly| command.
The latter feature can be used to reduce the compilation time while editing
(this was significantly more useful in the earlier days of \LaTeX{})
or to generate a smaller document which is easier to navigate.
Another application of |\includeonly| is to generate
documents consisting of selected parts of the complete document.

However, there are a few drawbacks of the plain |\include| mechanism:
\begin{itemize}
\item
The child files cannot be compiled on their own,
they can only be compiled via the main file.
A naive editing environment
(such as a text editor with an option
to have the current file processed by \LaTeX)
may require one to switch to the main file before compiling;
attempting to compile the child file produces errors.
\item
The main file must be modified (each time)
to adjust the |\includeonly| command
to the present needs. This easily leaves the main file in a messy state.
\item
The generated document will always carry the filename
of the main document. This is inconvenient if
several child files are to be compiled and
to be kept for distribution.
\end{itemize}

The present package provides a simple interface
to make child files individually compilable by \LaTeX{}.
Compiling a child file then has the same effect as compiling
the main file with an |\includeonly| command
to select the appropriate child.
Moreover the generated document will carry the name of the child
rather than the main file.
This resolves all three above issues.

This feature is meant to make the editing of books,
thesis documents and lecture notes somewhat more convenient.
However, the package can also be used efficiently for
composing a series of documents (such as exercise sheets)
which are typically distributed individually.
It then assists the author in generating the individual documents
(potentially in different versions)
as well as a document containing the collected series.
Another application is in developing style files
or other kinds of included material
where compilation of the style file could redirect
to a sample or test file.

%%%%%%%%%%%%%%%%%%%%%%%%%%%%%%%%%%%%%%%%%%%%%%%%%%%%%%%%%%%%%%%%%%%%%%%%%%%%%%%%
%%%%%%%%%%%%%%%%%%%%%%%%%%%%%%%%%%%%%%%%%%%%%%%%%%%%%%%%%%%%%%%%%%%%%%%%%%%%%%%%
\section{Usage}

First of all, the package \textsf{childdoc} is \emph{not} a standard
\LaTeXe{} |.sty| style file! Therefore it needs to be invoked in
a non-standard way.

%%%%%%%%%%%%%%%%%%%%%%%%%%%%%%%%%%%%%%%%%%%%%%%%%%%%%%%%%%%%%%%%%%%%%%%%%%%%%%%%
\subsection{Included Files}
\label{sec:include}

%%%%%%%%%%%%%%%%%%%%%%%%%%%%%%%%%%%%%%%%
\DescribeMacro{\childdocmain}
To use the package, add the commands
\begin{center}
\begin{tabular}{l}
|\input{childdoc.def}|\\
|\childdocmain{}|\\
\end{tabular}
\end{center}
at the very top of the main \LaTeX{} file,
in particular \emph{before} the |\documentclass| statement!
The argument of |\childdocmain| should be left empty
(but it must be present).

%%%%%%%%%%%%%%%%%%%%%%%%%%%%%%%%%%%%%%%%
\DescribeMacro{\childdocof}
Furthermore, add the commands
\begin{center}
\begin{tabular}{l}
|\input{childdoc.def}|\\
|\childdocof{|\textit{main}|}|\\
\end{tabular}
\end{center}
at the top of every child file \textit{child}
which is included by |\include{|\textit{child}|}|
from within the main file
(or at least for those files to be compiled individually).
The argument \textit{main} must be the filename of the main file.

There are a couple of
considerations in setting up the main and child documents:

%%%%%%%%%%%%%%%%%%%%%%%%%%%%%%%%%%%%%%%%
\paragraph{Restrictions.}

Please note the following restrictions:
\begin{itemize}
\item
|\childdocmain| must be called with one argument \textit{main}
to ensure compatibility with earlier version of the package.
It must either be empty (|\childdocmain{}|)
or precisely match the filename of the main file in which it is specified.
See \secref{sec:detection} for further information.
\item
The filename \textit{main} must be specified without the |.tex| extension.
\item
The filename \textit{main} is case sensitive
(even in case-insensitive file systems)
due to internal string comparison.
\item
The argument \textit{main} should be fully expanded, it cannot be a macro.
\item
Subdirectories and special characters should be avoided in filenames.
\item
The command |\childdocmain{|\textit{main}|}| must be followed by a whitespace.
It should not be followed immediately by another command
or by a comment mark `|%|'.
This is because the \TeX{} parser reads the token immediately following
the argument of |\childdocmain| and puts it
at the beginning of every child section;
however, a white\-space is ignored.
\end{itemize}

%%%%%%%%%%%%%%%%%%%%%%%%%%%%%%%%%%%%%%%%
\paragraph{Content of Main File.}

It is advisable to place all content in the child files included by |\include|.
Any output contained in the main file will appear in all child documents
unless suppressed manually;
it cannot be suppressed automatically by the |\includeonly| directive
and thus should normally be avoided.
A method to include some content in the main file
by means of conditional processing is described in \secref{sec:conditional}.

%%%%%%%%%%%%%%%%%%%%%%%%%%%%%%%%%%%%%%%%
\paragraph{Page Numbering.}

When only a part of the document is compiled,
the appropriate numbering of pages
(as well as other status parameters)
is determined from the |.aux| files.
The latter contain information from previous passes.
However this information needs to propagate through
all intermediate child documents.
Therefore the page numbering in child documents may well
be inconsistent until the complete document is compiled at least once.

A useful (if unconventional) way to always ensure a consistent
page numbering is to restart the numbering in each child document
and denote the pages by `\textit{child}|.|\textit{page}'
where \textit{child} represents the chapter/section number of the child file.
This can be achieved by the command
|\numberwithin{page}{|\textit{child}|}|
of the \textsf{amsmath} package
where \textit{child} can be |chapter| or |section|
depending on the chosen structuring.
Alternatively, one can modify the macro |\thepage| appropriately
and reset the counter |page| at the start of each child file.

%%%%%%%%%%%%%%%%%%%%%%%%%%%%%%%%%%%%%%%%%%%%%%%%%%%%%%%%%%%%%%%%%%%%%%%%%%%%%%%%
\subsection{Conditional Processing}
\label{sec:conditional}

The package provides a mechanism to compile different versions
of a document. To customise the versions further some conditional processing
can come in handy to distinguish which version is being compiled.
The package provides two macros to describe the compilation context:

%%%%%%%%%%%%%%%%%%%%%%%%%%%%%%%%%%%%%%%%
\DescribeMacro{\ifchilddoc}
The conditional |\ifchilddoc| distinguishes between the compilation of
child documents and the main document:
%
\begin{center}
|\ifchilddoc |\textit{child-code}| |[|\||else |\textit{main-code}]| \||fi|
\end{center}

%%%%%%%%%%%%%%%%%%%%%%%%%%%%%%%%%%%%%%%%
\DescribeMacro{\childdocname}
\DescribeMacro{\childdocjob}
The macro |\childdocname| contains the filename (without extension)
of the main or child file being processed.
Note that |\childdocjob| will always contain the name of the main file.

%%%%%%%%%%%%%%%%%%%%%%%%%%%%%%%%%%%%%%%%
\paragraph{Title Page.}

Conditional processing can be used to include a title or banner page
in the main document when proper precautions are taken.
Importantly, the code in the main file should ensure that the page counter
(as well as other status parameters which are stored in the |.aux| files)
takes the same value after the conditional processing.
Otherwise the page numbers may take divergent values
depending on which part is compiled.

For example, a title page could be declared by:
%
\begin{center}
\begin{tabular}{l}
|\ifchilddoc\||else|\\
|\addtocounter{page}{-1}|\\
\textit{code for title page}\\
|\newpage|\\
|\||fi|
\end{tabular}
\end{center}
%
A banner page for the child documents can be generated by:
%
\begin{center}
\begin{tabular}{l}
|\ifchilddoc|\\
|\addtocounter{page}{-1}|\\
\textit{code for banner page}\\
|\newpage|\\
|\||fi|
\end{tabular}
\end{center}
%
Here one could write a message such as:
\begin{center}
|This is the part \childdocname{} of \childdocjob{}.|
\end{center}

%%%%%%%%%%%%%%%%%%%%%%%%%%%%%%%%%%%%%%%%%%%%%%%%%%%%%%%%%%%%%%%%%%%%%%%%%%%%%%%%
\subsection{Flags}
\label{sec:flags}

The package makes it easy to generate different versions
of the main or child documents.
To this end compilation flags can be defined
and assigned different default values.
They will be particularly useful in conjunction
with the forwarding mechanism described in \secref{sec:forward}.

For example, it may be useful to have a flag |\version|
which can be set to |draft| or |final|.
The document source will contain some conditional code
depending on the value of |\version|.
Suppose further, the flag should default to |final| for the main file
and to |draft| for child files
which is a natural assignment for editing the document.
This is achieved by placing the following code
in the preamble of the main document
(below the |\childdocmain| directive):
%
\begin{center}
\begin{tabular}{l}
|\ifchilddoc|\\
|\providecommand{\version}{draft}|\\
|\||else|\\
|\providecommand{\version}{final}|\\
|\||fi|
\end{tabular}
\end{center}
%
The definition by |\providecommand| makes sure
that previous definitions are not overwritten.
Further statements |\providecommand{\version}{...}|
can thus be added before the above code to override it.

For the main file, one might add a line
(between |\childdocmain| and the above block)
%
\begin{center}
|%\ifchilddoc\||else\providecommand{\version}{draft}\||fi|
\end{center}
%
which can be uncommented to produce a draft version.
Likewise one can add a line to the very top of a child file
(above the |\childdocof{|\textit{main}|}| directive)
%
\begin{center}
|%\providecommand{\version}{final}|
\end{center}
%
which can be uncommented to produce the final version of this child document.

%%%%%%%%%%%%%%%%%%%%%%%%%%%%%%%%%%%%%%%%%%%%%%%%%%%%%%%%%%%%%%%%%%%%%%%%%%%%%%%%
\subsection{Forwarding}
\label{sec:forward}

Different versions of the main or child documents
using compilation flags as described in \secref{sec:flags}
can be (permanently) stored in different files
for convenient compilation, viewing and distribution.
To this end, the package defines a command
to pass on compilation to a different file:

%%%%%%%%%%%%%%%%%%%%%%%%%%%%%%%%%%%%%%%%
\DescribeMacro{\childdocforward}
The command |\childdocforward| redirects processing to
another source file:
%
\begin{center}
\begin{tabular}{l}
|\input{childdoc.def}|\\
|\childdocforward[|\textit{main}|]{|\textit{dest}|}|\\
\end{tabular}
\end{center}
%
The argument \textit{dest} is the destination file
(without extension).
It should be the main file or one of the child files.
Note that further \textsf{childdoc} directives
such as |\childdocof| and |\childdocforward|
in the indicated file will be processed in this form.
The optional argument \textit{main}
passes on directly to the main file \textit{main}
while pretending to compile the child \textit{dest}.
This form behaves as if \textit{dest}
issues |\childdocof{|\textit{main}|}| right away,
and no further \textsf{childdoc} directives will be processed.

%%%%%%%%%%%%%%%%%%%%%%%%%%%%%%%%%%%%%%%%
\DescribeMacro{\...prefix}
In the alternative form |\childdocforwardprefix|,
%
\begin{center}
\begin{tabular}{l}
|\input{childdoc.def}|\\
|\childdocforwardprefix[|\textit{main}|]{|\textit{prefix}|}{|\textit{dest}|}|
\end{tabular}
\end{center}
%
the destination file is determined by a pattern
depending on the current file:
To make this work, the current file must be called
`{\textit{prefix}\hspace{0.2em}\textit{suffix}}'
with \textit{prefix} matching precisely the argument.
Processing is then passed on to the file
`{\textit{dest}\hspace{0.2em}\textit{suffix}}'.
Surely, the same effect is achieved by
directly specifying the
argument `{\textit{dest}\hspace{0.2em}\textit{suffix}}'
in the first form.
However, that requires to set up a different file
for each child. With the alternative form of the command
all these files can have exactly the same content
which simplifies setting them up and maintaining them.

For example, the following file |draft.tex|
with a compilation flag |\version| as described in \secref{sec:flags}
compiles the main document as a draft:
%
\begin{center}
\begin{tabular}{l}
|\def\version{draft}|\\
|\input{childdoc.def}|\\
|\childdocforward{|\textit{main}|}|
\end{tabular}
\end{center}
%
Likewise, the following files |final|\textit{nn}|.tex|
compile the final version of the child document
|child|\textit{nn}|.tex|:
%
\begin{center}
\begin{tabular}{l}
|\def\version{final}|\\
|\input{childdoc.def}|\\
|\childdocforwardprefix{final}{child}|
\end{tabular}
\end{center}
%

Note that when several versions of a main file and/or of each child file
are to be generated, it may be convenient to set up a |Makefile| or
shell script to automatise the process.

%%%%%%%%%%%%%%%%%%%%%%%%%%%%%%%%%%%%%%%%%%%%%%%%%%%%%%%%%%%%%%%%%%%%%%%%%%%%%%%%
\subsection{Command Line Processing}
\label{sec:commandline}

The effect of redirection files can also be achieved by invoking
the \LaTeX{} compiler with a more elaborate command line.
Most conveniently this should be done as part
of a shell script or a |Makefile|.

When using \textsf{childdoc} in the main file, the following
command lines effectively perform a redirection
(note that depending on the shell being used,
backslashes may have to be doubled: `|\|' $\to$ `|\\|'):
%
\begin{center}
|... -jobname "|\textit{target}|" |\\|"|[\textit{flags}]%
|\input{childdoc.def}\childdocforward[|\textit{main}|]{|\textit{dest}|}"|
\end{center}
%
Here \textit{target} is the name of the output file,
\textit{main} is the name of the main file
and \textit{dest} is the name of the main or child file to be processed
(all filenames without extensions).
The optional argument \textit{main} can be omitted
if \textit{main} matches \textit{dest}.
Optionally, compilation \textit{flags} can be defined via |\def| commands.
This command line makes the \TeX{} engine believe
it is compiling the file \textit{target}
whose content is specified as the latter parameter.
The provided code then forwards the processing to
\textit{main} or \textit{dest} as described in \secref{sec:forward}.

%%%%%%%%%%%%%%%%%%%%%%%%%%%%%%%%%%%%%%%%%%%%%%%%%%%%%%%%%%%%%%%%%%%%%%%%%%%%%%%%
\subsection{Include by Input}
\label{sec:input}

Including child documents by |\include| has some restrictions by design.
Most notably, the content of a child document always occupies
its own set of pages; pages cannot be shared between child documents.
Usually, this behaviour makes perfect sense
because each child document contain an essential part of the document.
However, in some situations it may be desirable to compose
a document from a collection of parts
without having mandatory page breaks between then.
For this case, the package
provides a mechanism to include parts
by |\input| which can also be processed individually.
However, by construction this mechanism
requires manual handling of the content to be output.

%%%%%%%%%%%%%%%%%%%%%%%%%%%%%%%%%%%%%%%%
\DescribeMacro{\ifchilddocmanual}
The main file should be prepared as usual, see \secref{sec:include}.
However, the document body must make a distinction
between processing of an individual part and of the main document, e.g.:
%
\begin{center}
\begin{tabular}{l}
|\ifchilddocmanual|\\
|\input{\childdocname}|\\
|\||else|\\
\textit{document body with }|\input{|\textit{part}|}|\\
|\||fi|
\end{tabular}
\end{center}
%
The conditional |\ifchilddocmanual| is true whenever
a part to be included by |\input| is being compiled,
and the name of the part is stored in |\childdocname|.

%%%%%%%%%%%%%%%%%%%%%%%%%%%%%%%%%%%%%%%%
\DescribeMacro{\childdocby}
Each part to be included by |\input| should start with:
%
\begin{center}
\begin{tabular}{l}
|\input{childdoc.def}|\\
|\childdocby{|\textit{main}|}|\\
\end{tabular}
\end{center}
%
The directive |\childdocby| is similar to |\childdocof|
described in \secref{sec:include},
but the subsequent selection of content must be done manually.
To that end, both |\ifchilddoc| and |\ifchilddocmanual|
will be true upon processing of a part,
and the name of the part is stored in |\childdocname|.
Note that |\jobname| will be set to the filename of the current part
so that each part receives an individual |.aux| file
that does not interfere with the |.aux| file(s) of the main document.
This behaviour can be altered by the alternative form
|\childdocby[*]{|\textit{main}|}| (with a non-empty optional argument)
which uses the |.aux| file of the main document
by setting |\jobname| to \textit{main}.

%%%%%%%%%%%%%%%%%%%%%%%%%%%%%%%%%%%%%%%%%%%%%%%%%%%%%%%%%%%%%%%%%%%%%%%%%%%%%%%%
\subsection{Driver Development}
\label{sec:driver}

The \textsf{childdoc} mechanism can also be use for the development
of definition files such as \LaTeX{} styles or classes.
This case differs from the above setup with multiple parts
included by |\include| in that no |\includeonly| should be invoked.
This can be achieved by starting the include file
(before |\ProvidesPackage|) with:
%
\begin{center}
\begin{tabular}{l}
|\input{childdoc.def}|\\
|\childdocforward{|\textit{main}|}|\\
\end{tabular}
\end{center}
%
or alternatively with:
%
\begin{center}
\begin{tabular}{l}
|\input{childdoc.def}|\\
|\childdocby{|\textit{main}|}|\\
\end{tabular}
\end{center}
%
Both forms have slightly different effects as described above.
The main file is prepared as usual, see \secref{sec:include}.

%%%%%%%%%%%%%%%%%%%%%%%%%%%%%%%%%%%%%%%%%%%%%%%%%%%%%%%%%%%%%%%%%%%%%%%%%%%%%%%%
\subsection{Legacy Detection}
\label{sec:detection}

The directive |\childdocmain| in the main file can detect
whether the complete document or merely a child is to be compiled
even without using the directive |\childdocof|.
This method is deprecated because it is less robust
and there is no compelling reason to use it;
it is merely provided for backward compatibility
and it may be removed in future versions.

If the detection mechanism is to be used,
it is mandatory to correctly specify
the filename of the main file as the argument of |\childdocmain|:
%
\begin{center}
\begin{tabular}{l}
|\input{childdoc.def}|\\
|\childdocmain{|\textit{main}|}|\\
\end{tabular}
\end{center}
%
If |\jobname| does not match the argument \textit{main} of |\childdocmain|,
it is assumed that |\jobname| points to the child file to be compiled.
When using |\childdocmain| with the main file specified as argument,
it suffices to start a child file
with just |\input{|\textit{main}|}|
without loading of the package and using |\childdocof|.
If instead all processing is done
with the appropriate \textsf{childdoc} directives,
the argument of \textit{main} of |\childdocmain| can be empty.

An alternative version of the command line processing described
in \secref{sec:commandline} using the detection mechanism reads:
%
\begin{center}
|... -jobname "|\textit{target}|" "|[\textit{flags}]%
[|\def\jobname{|\textit{dest}|}|]|\input{|\textit{main}|}"|
\end{center}

%%%%%%%%%%%%%%%%%%%%%%%%%%%%%%%%%%%%%%%%%%%%%%%%%%%%%%%%%%%%%%%%%%%%%%%%%%%%%%%%
\subsection{Manual Code}
\label{sec:manual}

In case one cannot be certain whether the definitions file |childdoc.def|
is installed on the target \TeX{} distribution
and one prefers not to ship it,
it is conceivable to paste a few relevant commands into the sources.

To that end, drop all statements |\input{childdoc.def}|
and perform the replacements as outlined below.
Instead of |\childdocmain{|\textit{main}|}| add the following code
to the top of the main file:
%
\begin{center}
\begin{tabular}{l}
|\||ifdefined\childdocname\endinput\||fi\newif\ifchilddoc|\\
|\edef\childdocname{\scantokens\expandafter{\jobname\noexpand}}|\\
|\def\childdocmain{|\textit{main}|}\||ifx\childdocmain\childdocname\||else|\\
|\childdoctrue\includeonly{\childdocname}\let\jobname\childdocmain\||fi|\\
\end{tabular}
\end{center}
%
Instead of |\childdocof{|\textit{main}|}| just include the main file
at the top of each child file:
%
\begin{center}
|\input{|\textit{main}|}|
\end{center}
%
A simple redirection |\childdocforward{|\textit{dest}|}| is achieved by:
%
\begin{center}
|\def\jobname{|\textit{dest}|}\input{\jobname}|
\end{center}
%
The redirection with prefix
|\childdocforwardprefix[|\textit{prefix}|]{|\textit{dest}|}|
is accomplished by:
%
\begin{center}
\begin{tabular}{l}
|{\edef\jobname{\scantokens\expandafter{\jobname\noexpand}}|\\
|\def\redirectjob |\textit{prefix}|#1~~~{\gdef\jobname{|\textit{dest}|#1}}|\\
|\expandafter\redirectjob\jobname~~~}\input{\jobname}|
\end{tabular}
\end{center}

In an alternative approach,
child documents can be compiled by a specific command line
without additional code or specific definitions:
%
\begin{center}
|... -jobname "|\textit{target}|" "|[\textit{flags}]%
|\includeonly{|\textit{dest}|}\input{|\textit{main}|}"|
\end{center}
%

%%%%%%%%%%%%%%%%%%%%%%%%%%%%%%%%%%%%%%%%%%%%%%%%%%%%%%%%%%%%%%%%%%%%%%%%%%%%%%%%
%%%%%%%%%%%%%%%%%%%%%%%%%%%%%%%%%%%%%%%%%%%%%%%%%%%%%%%%%%%%%%%%%%%%%%%%%%%%%%%%
\section{Information}

%%%%%%%%%%%%%%%%%%%%%%%%%%%%%%%%%%%%%%%%%%%%%%%%%%%%%%%%%%%%%%%%%%%%%%%%%%%%%%%%
\subsection{Copyright}

Copyright \copyright{} 2017--2018 Niklas Beisert

This work may be distributed and/or modified under the
conditions of the \LaTeX{} Project Public License, either version 1.3
of this license or (at your option) any later version.
The latest version of this license is in
  \url{http://www.latex-project.org/lppl.txt}
and version 1.3 or later is part of all distributions of \LaTeX{}
version 2005/12/01 or later.

This work has the LPPL maintenance status `maintained'.

The Current Maintainer of this work is Niklas Beisert.

This work consists of the files |README.txt|, |childdoc.ins| and |childdoc.dtx|
as well as the derived files |childdoc.def|, |cdocsamp.tex|
with |cdocsch1.tex|, |cdocsch2.tex|, |cdocspt3.tex|, |cdocspt4.tex|,
|cdocsdrf.tex|, |cdocsfn1.tex|, |cdocsfn2.tex|
as well as |childdoc.pdf|.

%%%%%%%%%%%%%%%%%%%%%%%%%%%%%%%%%%%%%%%%%%%%%%%%%%%%%%%%%%%%%%%%%%%%%%%%%%%%%%%%
\subsection{Files and Installation}

The package consists of the files:
%
\begin{center}
\begin{tabular}{ll}
    |README.txt|   & readme file \\
    |childdoc.ins| & installation file \\
    |childdoc.dtx| & source file \\
    |childdoc.def| & definition file \\
    |cdocsamp.tex| & sample main file \\
    |cdocsch1.tex| & sample include file \\
    |cdocsch2.tex| & sample include file \\
    |cdocspt3.tex| & sample part file \\
    |cdocspt4.tex| & sample part file \\
    |cdocsdrf.tex| & sample redirection file \\
    |cdocsfn1.tex| & sample redirection file \\
    |cdocsfn2.tex| & sample redirection file \\
    |childdoc.pdf| & manual
\end{tabular}
\end{center}
%
The distribution consists of the files
|README.txt|, |childdoc.ins| and |childdoc.dtx|.
%
\begin{itemize}
\item
Run (pdf)\LaTeX{} on |childdoc.dtx|
to compile the manual |childdoc.pdf| (this file).
\item
Run \LaTeX{} on |childdoc.ins| to create the definitions file |childdoc.def|
and the sample |cdocsamp.tex| with include files
|cdocsch1.tex|, |cdocsch2.tex|, |cdocspt3.tex|, |cdocspt4.tex|,
|cdocsdrf.tex|, |cdocsfn1.tex|, |cdocsfn2.tex|.
Then copy the file |childdoc.def| to an appropriate directory of your \LaTeX{}
distribution, e.g.\ \textit{texmf-root}|/tex/latex/childdoc|.
\end{itemize}

%%%%%%%%%%%%%%%%%%%%%%%%%%%%%%%%%%%%%%%%%%%%%%%%%%%%%%%%%%%%%%%%%%%%%%%%%%%%%%%%
\subsection{Related CTAN Packages}

There are several other packages which offer a similar functionality:
%
\begin{itemize}
\item
The packages
\href{http://ctan.org/pkg/docmute}{\textsf{docmute}},
\href{http://ctan.org/pkg/includex}{\textsf{includex}} and
\href{http://ctan.org/pkg/standalone}{\textsf{standalone}}
provide commands to include only the document body of
a child file thus allowing both files to be compiled individually.
\item
The packages \href{http://ctan.org/pkg/subdocs}{\textsf{subdocs}}
and \href{http://ctan.org/pkg/subfiles}{\textsf{subfiles}}
provide structures in which the main and child documents can be
encapsulated and allowing them to be compiled individually.
The inclusion mechanism is different from the conventional |\include|.
\item
The package \href{http://ctan.org/pkg/combine}{\textsf{combine}}
is an elaborate solution to combine several documents into one.
\end{itemize}
%
See also the CTAN topic \href{http://ctan.org/topic/subdocs}{\textsf{subdocs}}
for further related packages.
The present package differs from the above solutions in that
a document structure constructed with the conventional |\include| mechanism
just needs two extra commands at the top of every file
such that all constituent files can be compiled individually.

%%%%%%%%%%%%%%%%%%%%%%%%%%%%%%%%%%%%%%%%%%%%%%%%%%%%%%%%%%%%%%%%%%%%%%%%%%%%%%%%
%\subsection{Feature Suggestions}
%
%The following is a list of features which may be useful for future
%versions of this package:
%%
%\begin{itemize}
%\item
%\ldots
%\end{itemize}

%%%%%%%%%%%%%%%%%%%%%%%%%%%%%%%%%%%%%%%%%%%%%%%%%%%%%%%%%%%%%%%%%%%%%%%%%%%%%%%%
\subsection{Revision History}

%%%%%%%%%%%%%%%%%%%%%%%%%%%%%%%%%%%%%%%%
\paragraph{v2.0:} 2018/12/30

\begin{itemize}
\item
immediate forward processing
\item
added |\childdocby| mechanism
\item
manual restructured
\end{itemize}

%%%%%%%%%%%%%%%%%%%%%%%%%%%%%%%%%%%%%%%%
\paragraph{v1.6:} 2018/01/17

\begin{itemize}
\item
application for development of include files
\item
corrections to manual
\end{itemize}

%%%%%%%%%%%%%%%%%%%%%%%%%%%%%%%%%%%%%%%%
\paragraph{v1.5:} 2017/05/21

\begin{itemize}
\item
more complete structuring introduced
\item
|\childdocof| introduced
\item
|\childdoc| renamed to |\childdocmain|
\item
|\childredirect| renamed to |\childdocforward| and |\childdocforwardprefix|
and functionality expanded
\end{itemize}

%%%%%%%%%%%%%%%%%%%%%%%%%%%%%%%%%%%%%%%%
\paragraph{v1.0:} 2017/04/27

\begin{itemize}
\item
manual and install package
\item
first version published on CTAN
\end{itemize}

%%%%%%%%%%%%%%%%%%%%%%%%%%%%%%%%%%%%%%%%
\paragraph{v0.6:} 2017/04/26

\begin{itemize}
\item
redirection mechanism added
\end{itemize}

%%%%%%%%%%%%%%%%%%%%%%%%%%%%%%%%%%%%%%%%
\paragraph{v0.5:} 2017/04/26

\begin{itemize}
\item
functionality in definition file
\end{itemize}


%%%%%%%%%%%%%%%%%%%%%%%%%%%%%%%%%%%%%%%%%%%%%%%%%%%%%%%%%%%%%%%%%%%%%%%%%%%%%%%%
%%%%%%%%%%%%%%%%%%%%%%%%%%%%%%%%%%%%%%%%%%%%%%%%%%%%%%%%%%%%%%%%%%%%%%%%%%%%%%%%
%%%%%%%%%%%%%%%%%%%%%%%%%%%%%%%%%%%%%%%%%%%%%%%%%%%%%%%%%%%%%%%%%%%%%%%%%%%%%%%%
\appendix

\settowidth\MacroIndent{\rmfamily\scriptsize 000\ }

 \DocInput{childdoc.dtx}

\end{document}
%</driver>
% \fi
%
% %%%%%%%%%%%%%%%%%%%%%%%%%%%%%%%%%%%%%%%%%%%%%%%%%%%%%%%%%%%%%%%%%%%%%%%%%%%%%%
% %%%%%%%%%%%%%%%%%%%%%%%%%%%%%%%%%%%%%%%%%%%%%%%%%%%%%%%%%%%%%%%%%%%%%%%%%%%%%%
% \section{Sample}
%\iffalse
%<*samplemain>
%\fi
%
% The following presents a sample document
% with two chapters, two parts, a title page,
% a compile flag as well as three forwarding files to set the flag.
% It consists of eight |.tex| files:
% \begin{center}
% \begin{tabular}{ll}
% |cdocsamp.tex|&main file\\
% |cdocsch1.tex|&include file for chapter 1\\
% |cdocsch2.tex|&include file for chapter 2\\
% |cdocspt3.tex|&include file for part 3\\
% |cdocspt4.tex|&include file for part 4\\
% |cdocsdrf.tex|&forwarding file for main file in draft mode\\
% |cdocsfi1.tex|&forwarding file for final version of chapter 1\\
% |cdocsfi2.tex|&forwarding file for final version of chapter 2\\
% \end{tabular}
% \end{center}
% Each of the eight files can be compiled directly by the \LaTeX{} compiler.
%
% %%%%%%%%%%%%%%%%%%%%%%%%%%%%%%%%%%%%%%
% \paragraph{Main File.}
%
% The main file is called |cdocsamp.tex|.
%
% Load the \textsf{childdoc} definitions and
% declare the filename for the main document:
%    \begin{macrocode}
\input{childdoc.def}
\childdocmain{}
%    \end{macrocode}

% Optional override for |\version| flag:
%    \begin{macrocode}
%%\ifchilddoc\else\providecommand{\version}{draft}\fi
%    \end{macrocode}

% Define the default values for the |\version| flag
% (|final| for the main file and |draft| for childs):
%    \begin{macrocode}
\ifchilddoc
\providecommand{\version}{draft}
\else
\providecommand{\version}{final}
\fi
%    \end{macrocode}

% Load the standard document class:
%    \begin{macrocode}
\documentclass[12pt]{article}
%    \end{macrocode}

% Start the document body:
%    \begin{macrocode}
\begin{document}
%    \end{macrocode}

% Declare a title page.
% Print title, part of document being processed and version flag:
%    \begin{macrocode}
\addtocounter{page}{-1}
\begin{center}
{\LARGE\bfseries{}childdoc example\par}
\vspace{1cm}
\ifchilddoc
\ifchilddocmanual part\else chapter\fi:
`\childdocname' of `\childdocjob'\par
\else
main document: `\childdocjob'\par
\fi
version: \version\par
\end{center}
\newpage
%    \end{macrocode}

% Manually include selected file,
% otherwise process as usual:
%    \begin{macrocode}
\ifchilddocmanual
\section*{part `\childdocname'}
\input{\childdocname}
\else
%    \end{macrocode}

% Include the two chapters:
%    \begin{macrocode}
\include{cdocsch1}
\include{cdocsch2}
%    \end{macrocode}

% Include the two parts unless only chapters should be displayed:
%    \begin{macrocode}
\ifchilddoc\else
\section{part three}
\input{cdocspt3}
\section{part four}
\input{cdocspt4}
\fi
%    \end{macrocode}

% Process as usual until here:
%    \begin{macrocode}
\fi
%    \end{macrocode}

% End of document body:
%    \begin{macrocode}
\end{document}
%    \end{macrocode}
%\iffalse
%</samplemain>
%\fi
%
% %%%%%%%%%%%%%%%%%%%%%%%%%%%%%%%%%%%%%%
% \paragraph{Chapter Include Files.}
%
% The include files are called |cdocsch1.tex| and |cdocsch2.tex|.
%
%\iffalse
%<*samplechap1|samplechap2>
%\fi

% Optional override for |\version| flag:
%    \begin{macrocode}
%%\providecommand{\version}{final}
%    \end{macrocode}

% Include the main document:
%    \begin{macrocode}
\input{childdoc.def}
\childdocof{cdocsamp}
%    \end{macrocode}

%\iffalse
%</samplechap1|samplechap2>
%\fi
%
%\iffalse
%<*samplechap1>
%\fi
% Some text for chapter 1:
%    \begin{macrocode}
\section{one}
some text in chapter one
%    \end{macrocode}

%\iffalse
%</samplechap1>
%\fi
% Some text for chapter 2:
%\iffalse
%<*samplechap2>
%\fi
%    \begin{macrocode}
\section{two}
more text in chapter two
%    \end{macrocode}

%\iffalse
%</samplechap2>
%\fi
%
% %%%%%%%%%%%%%%%%%%%%%%%%%%%%%%%%%%%%%%
% \paragraph{Part Include Files.}
%
% The include files are called |cdocspt3.tex| and |cdocspt4.tex|.
%
%\iffalse
%<*samplepart3|samplepart4>
%\fi

% Optional override for |\version| flag:
%    \begin{macrocode}
%%\providecommand{\version}{final}
%    \end{macrocode}

% Include the main document:
%    \begin{macrocode}
\input{childdoc.def}
\childdocby{cdocsamp}
%    \end{macrocode}

%\iffalse
%</samplepart3|samplepart4>
%\fi
%
%\iffalse
%<*samplepart3>
%\fi
% Some text for part 3:
%    \begin{macrocode}
some text in part three
%    \end{macrocode}

%\iffalse
%</samplepart3>
%\fi
% Some text for part 4:
%\iffalse
%<*samplepart4>
%\fi
%    \begin{macrocode}
more text in part four
%    \end{macrocode}

%\iffalse
%</samplepart4>
%\fi
%
% %%%%%%%%%%%%%%%%%%%%%%%%%%%%%%%%%%%%%%
% \paragraph{Forwarding for a Complete Draft.}
%
% The following forwarding file |cdocsdrf.tex|
% compiles the main document in draft mode:
%\iffalse
%<*sampledraft>
%\fi
%    \begin{macrocode}
\def\version{draft}
\input{childdoc.def}
\childdocforward{cdocsamp}
%    \end{macrocode}

%\iffalse
%</sampledraft>
%\fi
%
% %%%%%%%%%%%%%%%%%%%%%%%%%%%%%%%%%%%%%%
% \paragraph{Forwarding for Final Version of the Chapters.}
%
% The following forwarding files |cdocsfn1.tex| and |cdocsfn2.tex|
% (with identical content)
% compile the final versions of the child documents
% |cdocsch1.tex| and |cdocsch2.tex|, respectively:
%\iffalse
%<*samplefinal>
%\fi
%    \begin{macrocode}
\def\version{final}
\input{childdoc.def}
\childdocforwardprefix[cdocsamp]{cdocsfn}{cdocsch}
%    \end{macrocode}

%\iffalse
%</samplefinal>
%\fi
%
% %%%%%%%%%%%%%%%%%%%%%%%%%%%%%%%%%%%%%%
% \paragraph{Command Line Processing.}
%
% The following three command lines generate the output files
% |cdocscld|, |cdocscl1| and |cdocscl2|
% which should be identical to
% |cdocsdrf|, |cdocsch1| and |cdocsfn2|, respectively:
% \begin{center}
% \begin{tabular}{l}
% |latex -jobname cdocscld \|\\
% |  "\def\version{draft}\input{childdoc.def}\childdocforward{cdocsamp}"|\\
% |latex -jobname cdocscl1 \|\\
% |  "\input{childdoc.def}\childdocforward[cdocsamp]{cdocsch1}"|\\
% |latex -jobname cdocscl2 \|\\
% |  "\def\version{final}\input{childdoc.def}\childdocforward{cdocsch2}"|
% \end{tabular}
% \end{center}
% Note that the trailing backslash on each first line
% merely continues the input to the second line
% (for convenient cut ant paste).
% Furthermore, the command |latex| can be replaced by any
% of its alternative versions such as |pdflatex|.
%
% %%%%%%%%%%%%%%%%%%%%%%%%%%%%%%%%%%%%%%%%%%%%%%%%%%%%%%%%%%%%%%%%%%%%%%%%%%%%%%
% %%%%%%%%%%%%%%%%%%%%%%%%%%%%%%%%%%%%%%%%%%%%%%%%%%%%%%%%%%%%%%%%%%%%%%%%%%%%%%
% \section{Implementation}
%\iffalse
%<*package>
%\fi
%
% This section describes the definitions file |childdoc.def|.

% The definitions cannot be loaded using |\usepackage| or |\RequirePackage|
% which has a mechanism to prevent loading a style file more than once.
% When loading the definitions by means of |\input|
% multiple instances have to be prevented manually:
%\iffalse
%This code needs to be before the `\ProvidesFile' directive
%which is defined at the beginning of this file.
%Therefore it is also placed there and commented out here.
%</package>
%<*discard>
%\fi
%    \begin{macrocode}
\ifdefined\childdocmain\endinput\fi
%    \end{macrocode}
%\iffalse
%</discard>
%<*package>
%\fi
%
% \macro{\ifchilddoc}
% \macro{\ifchilddocmanual}
% The conditional |\ifchilddoc| tells whether a
% child (true) or main (false) document is being compiled.
% The conditional |\ifchilddocmanual| tells whether
% the |\includeonly| mechanism is used (false) or
% the selection of child files must be performed manually (true).
% The definitions initialise to false:
%    \begin{macrocode}
\newif\ifchilddoc
\newif\ifchilddocmanual
%    \end{macrocode}

% \macro{\childdocname}
% \macro{\childdocjob}
% The macro |\childdocname| stores the name of the main document
% to be compiled. The macro |\childdocjob| stores the name of
% the document on which the \LaTeX{} compiler was originally invoked.
% The content of |\jobname| cannot be compared
% to filenames specified in the source due to different catcodes.
% The following code rescans |\jobname|, stores the result
% in |\childdocname| and saves a copy in |\childdocjob|:
%    \begin{macrocode}
\edef\childdocname{\scantokens\expandafter{\jobname\noexpand}}
\let\childdocjob\childdocname
%    \end{macrocode}

% \macro{\childdocdisable}
% The macro |\childdocdisable| prevents the main file
% from being processed more than once.
% At this stage, the main document command |\childdocmain|
% is assumed to be called once again where it should do nothing.
% Any subsequent call to it should prevent
% a secondary processing of the main document
% It overwrites the forwarding commands
% |\childdocof| and |\childdocforward|
% with empty macros to prevent further inclusions of the main document:
%    \begin{macrocode}
\newcommand{\childdocdisable}
{
  \renewcommand{\childdocmain}[1]{\renewcommand{\childdocmain}[1]{\endinput}}
  \renewcommand{\childdocof}[1]{}
  \renewcommand{\childdocby}[2][]{}
  \renewcommand{\childdocforward}[2][]{}
  \renewcommand{\childdocdisable}{}
}
%    \end{macrocode}

% \macro{\childdocmain}
% The macro |\childdocmain| is to be called at the top of the main file
% with nothing or the main filename (without extension) as argument.
% First, it breaks loops.
% If the argument is not empty and does not match |\childdocname|
% (which is set by the first inclusion of |childdoc.def|),
% |\ifchilddoc| is set to true, |\includeonly| is applied to the child file
% and |\jobname| is set to the main file
% (for proper handling of |.aux| files):
%    \begin{macrocode}
\newcommand{\childdocmain}[1]
{
  \childdocdisable\childdocmain{}
  \if?#1?\else
    \begingroup
      \def\childdoctmp{#1}
      \ifx\childdoctmp\childdocname
        \def\childdoctmp{}
      \else
        \def\childdoctmp
        {
          \childdoctrue
          \includeonly{\childdocname}
          \def\childdocjob{#1}
          \def\jobname{#1}
        }
      \fi
      \expandafter
    \endgroup
    \childdoctmp
  \fi
}
%    \end{macrocode}

% \macro{\childdocof}
% The command |\childdocof| redirects
% compilation to the main file |#1|.
%    \begin{macrocode}
\newcommand{\childdocof}[1]
{
  \childdocdisable
  \childdoctrue
  \includeonly{\childdocname}
  \def\jobname{#1}
  \def\childdocjob{#1}
  \input{#1}
}
%    \end{macrocode}

% \macro{\childdocby}
% The command |\childdocby| ....
%    \begin{macrocode}
\newcommand{\childdocby}[2][]
{
  \childdocdisable
  \childdoctrue
  \childdocmanualtrue
  \if?#1?\else
    \def\jobname{#2}
  \fi
  \def\childdocjob{#2}
  \input{#2}
  \endinput
}
%    \end{macrocode}

% \macro{\childdocforward}
% The command |\childdocforward| redirects
% compilation to the main file or
% (if the optional argument is given) a child file.
% Parameters are set as if the main file
% or a child file starting with |\childdocof| was compiled.
% Then compilation is handed over to the main file:
%    \begin{macrocode}
\newcommand{\childdocforward}[2][]
{
  \begingroup
    \if?#1?
      \def\childdoctmp
      {
        \def\childdocname{#2}
        \def\childdocjob{#2}
        \def\jobname{#2}
        \input{#2}
        \endinput
      }
    \else
      \def\childdoctmp
      {
        \childdocdisable
        \def\childdocname{#2}
        \childdoctrue
        \includeonly{#2}
        \def\childdocjob{#1}
        \def\jobname{#1}
        \input{#1}
        \endinput
      }
    \fi
    \expandafter
  \endgroup
  \childdoctmp
}
%    \end{macrocode}

% \macro{\childdocforwardprefix}
% The command |\childdocforwardprefix| redirects
% compilation to the main or a child file by means of a pattern.
% The prefix |#1| in the current filename is replaced by |#2|
% and the suffix of the current filename is kept
% (it is assumed that the filename does not contain the substring `|~~~|'
% which is used as a delimiter).
% Compilation is handed over to the new file by |\childdocforward|:
%    \begin{macrocode}
\newcommand{\childdocforwardprefix}[3][]
{
  \begingroup
    \def\childdocextract #2##1~~~{\def\childdoctmp{\childdocforward[#1]{#3##1}}}
    \expandafter\childdocextract\childdocname~~~
    \expandafter
  \endgroup
  \childdoctmp
}
%    \end{macrocode}

% \macro{\childdoc}
% The deprecated macro |\childdoc| is a legacy version of |\childdocmain|:
%    \begin{macrocode}
\newcommand{\childdoc}{\childdocmain}
%    \end{macrocode}

% \macro{\childdocredirect}
% The deprecated macro |\childdocredirect| is a legacy version
% of |\childdocforward| and |\childdocforwardprefix|:
%    \begin{macrocode}
\newcommand{\childdocredirect}[2][]
{
  \begingroup
    \if?#1?
      \def\childdoctmp{\childdocforward{#2}}
    \else
      \def\childdoctmp{\childdocforwardprefix{#1}{#2}}
    \fi
    \expandafter
  \endgroup
  \childdoctmp
}
%    \end{macrocode}

%\iffalse
%</package>
%\fi
%
\endinput

\childdocof{cdocsamp}
%    \end{macrocode}

%\iffalse
%</samplechap1|samplechap2>
%\fi
%
%\iffalse
%<*samplechap1>
%\fi
% Some text for chapter 1:
%    \begin{macrocode}
\section{one}
some text in chapter one
%    \end{macrocode}

%\iffalse
%</samplechap1>
%\fi
% Some text for chapter 2:
%\iffalse
%<*samplechap2>
%\fi
%    \begin{macrocode}
\section{two}
more text in chapter two
%    \end{macrocode}

%\iffalse
%</samplechap2>
%\fi
%
% %%%%%%%%%%%%%%%%%%%%%%%%%%%%%%%%%%%%%%
% \paragraph{Part Include Files.}
%
% The include files are called |cdocspt3.tex| and |cdocspt4.tex|.
%
%\iffalse
%<*samplepart3|samplepart4>
%\fi

% Optional override for |\version| flag:
%    \begin{macrocode}
%%\providecommand{\version}{final}
%    \end{macrocode}

% Include the main document:
%    \begin{macrocode}
% \iffalse
%
% childdoc.dtx Copyright (C) 2017-2018 Niklas Beisert
%
% This work may be distributed and/or modified under the
% conditions of the LaTeX Project Public License, either version 1.3
% of this license or (at your option) any later version.
% The latest version of this license is in
%   http://www.latex-project.org/lppl.txt
% and version 1.3 or later is part of all distributions of LaTeX
% version 2005/12/01 or later.
%
% This work has the LPPL maintenance status `maintained'.
%
% The Current Maintainer of this work is Niklas Beisert.
%
% This work consists of the files childdoc.dtx and childdoc.ins
% and the derived files childdoc.def and cdocsamp.tex with
% cdocsch1.tex, cdocsch2.tex, cdocsdrf.tex, cdocsfn1.tex, cdocsfn2.tex.
%
%<package>\ifdefined\childdocmain\endinput\fi
%<package>\ProvidesFile{childdoc.def}[2018/12/30 v2.0 child document driver]
%<samplemain>\ProvidesFile{cdocsamp.tex}[2018/12/30 v2.0 sample for childdoc]
%<*driver>
%\ProvidesFile{childdoc.drv}[2018/12/30 v2.0 childdoc reference manual file]
\PassOptionsToClass{10pt,a4paper}{article}
\documentclass{ltxdoc}

\usepackage[margin=35mm]{geometry}
\usepackage{hyperref}
\usepackage{hyperxmp}
\usepackage[usenames]{color}

\hypersetup{colorlinks=true}
\hypersetup{pdfstartview=FitH}
\hypersetup{pdfpagemode=UseNone}
\hypersetup{pdfsource={}}
\hypersetup{pdflang={en-UK}}
\hypersetup{pdfcopyright={Copyright 2017-2018 Niklas Beisert.
  This work may be distributed and/or modified under the
  conditions of the LaTeX Project Public License, either version 1.3
  of this license or (at your option) any later version.}}
\hypersetup{pdflicenseurl={http://www.latex-project.org/lppl.txt}}
\hypersetup{pdfcontactaddress={ETH Zurich, ITP, HIT K,
  Wolfgang-Pauli-Strasse 27}}
\hypersetup{pdfcontactpostcode={8093}}
\hypersetup{pdfcontactcity={Zurich}}
\hypersetup{pdfcontactcountry={Switzerland}}
\hypersetup{pdfcontactemail={nbeisert@itp.phys.ethz.ch}}
\hypersetup{pdfcontacturl={http://people.phys.ethz.ch/\xmptilde nbeisert/}}

\newcommand{\secref}[1]{\hyperref[#1]{section \ref*{#1}}}

\parskip1ex
\parindent0pt
\let\olditemize\itemize
\def\itemize{\olditemize\parskip0pt}

\begin{document}

\title{The \textsf{childdoc} Package}
\hypersetup{pdftitle={The childdoc Package}}
\author{Niklas Beisert\\[2ex]
  Institut f\"ur Theoretische Physik\\
  Eidgen\"ossische Technische Hochschule Z\"urich\\
  Wolfgang-Pauli-Strasse 27, 8093 Z\"urich, Switzerland\\[1ex]
  \href{mailto:nbeisert@itp.phys.ethz.ch}
  {\texttt{nbeisert@itp.phys.ethz.ch}}}
\hypersetup{pdfauthor={Niklas Beisert}}
\hypersetup{pdfsubject={Manual for the LaTeX2e Package childdoc}}
\date{30 December 2018, \textsf{v2.0}}
\maketitle

\begin{abstract}\noindent
\textsf{childdoc} is a \LaTeXe{} package
that enables the direct compilation
of document sections included by |\include|
to individual files.
\end{abstract}

\begingroup
\parskip0ex
\tableofcontents
\endgroup

%%%%%%%%%%%%%%%%%%%%%%%%%%%%%%%%%%%%%%%%%%%%%%%%%%%%%%%%%%%%%%%%%%%%%%%%%%%%%%%%
%%%%%%%%%%%%%%%%%%%%%%%%%%%%%%%%%%%%%%%%%%%%%%%%%%%%%%%%%%%%%%%%%%%%%%%%%%%%%%%%
\section{Introduction}

\LaTeX{} provides a mechanism to structure a large document (such as a book)
into a main file and several child files (containing the chapters)
using the |\include| command.
This mechanism is beneficial for documents
which span hundreds of pages in order to
make the source file(s) more manageable.
Moreover, compilation can be restricted to
selected child files by means of the |\includeonly| command.
The latter feature can be used to reduce the compilation time while editing
(this was significantly more useful in the earlier days of \LaTeX{})
or to generate a smaller document which is easier to navigate.
Another application of |\includeonly| is to generate
documents consisting of selected parts of the complete document.

However, there are a few drawbacks of the plain |\include| mechanism:
\begin{itemize}
\item
The child files cannot be compiled on their own,
they can only be compiled via the main file.
A naive editing environment
(such as a text editor with an option
to have the current file processed by \LaTeX)
may require one to switch to the main file before compiling;
attempting to compile the child file produces errors.
\item
The main file must be modified (each time)
to adjust the |\includeonly| command
to the present needs. This easily leaves the main file in a messy state.
\item
The generated document will always carry the filename
of the main document. This is inconvenient if
several child files are to be compiled and
to be kept for distribution.
\end{itemize}

The present package provides a simple interface
to make child files individually compilable by \LaTeX{}.
Compiling a child file then has the same effect as compiling
the main file with an |\includeonly| command
to select the appropriate child.
Moreover the generated document will carry the name of the child
rather than the main file.
This resolves all three above issues.

This feature is meant to make the editing of books,
thesis documents and lecture notes somewhat more convenient.
However, the package can also be used efficiently for
composing a series of documents (such as exercise sheets)
which are typically distributed individually.
It then assists the author in generating the individual documents
(potentially in different versions)
as well as a document containing the collected series.
Another application is in developing style files
or other kinds of included material
where compilation of the style file could redirect
to a sample or test file.

%%%%%%%%%%%%%%%%%%%%%%%%%%%%%%%%%%%%%%%%%%%%%%%%%%%%%%%%%%%%%%%%%%%%%%%%%%%%%%%%
%%%%%%%%%%%%%%%%%%%%%%%%%%%%%%%%%%%%%%%%%%%%%%%%%%%%%%%%%%%%%%%%%%%%%%%%%%%%%%%%
\section{Usage}

First of all, the package \textsf{childdoc} is \emph{not} a standard
\LaTeXe{} |.sty| style file! Therefore it needs to be invoked in
a non-standard way.

%%%%%%%%%%%%%%%%%%%%%%%%%%%%%%%%%%%%%%%%%%%%%%%%%%%%%%%%%%%%%%%%%%%%%%%%%%%%%%%%
\subsection{Included Files}
\label{sec:include}

%%%%%%%%%%%%%%%%%%%%%%%%%%%%%%%%%%%%%%%%
\DescribeMacro{\childdocmain}
To use the package, add the commands
\begin{center}
\begin{tabular}{l}
|\input{childdoc.def}|\\
|\childdocmain{}|\\
\end{tabular}
\end{center}
at the very top of the main \LaTeX{} file,
in particular \emph{before} the |\documentclass| statement!
The argument of |\childdocmain| should be left empty
(but it must be present).

%%%%%%%%%%%%%%%%%%%%%%%%%%%%%%%%%%%%%%%%
\DescribeMacro{\childdocof}
Furthermore, add the commands
\begin{center}
\begin{tabular}{l}
|\input{childdoc.def}|\\
|\childdocof{|\textit{main}|}|\\
\end{tabular}
\end{center}
at the top of every child file \textit{child}
which is included by |\include{|\textit{child}|}|
from within the main file
(or at least for those files to be compiled individually).
The argument \textit{main} must be the filename of the main file.

There are a couple of
considerations in setting up the main and child documents:

%%%%%%%%%%%%%%%%%%%%%%%%%%%%%%%%%%%%%%%%
\paragraph{Restrictions.}

Please note the following restrictions:
\begin{itemize}
\item
|\childdocmain| must be called with one argument \textit{main}
to ensure compatibility with earlier version of the package.
It must either be empty (|\childdocmain{}|)
or precisely match the filename of the main file in which it is specified.
See \secref{sec:detection} for further information.
\item
The filename \textit{main} must be specified without the |.tex| extension.
\item
The filename \textit{main} is case sensitive
(even in case-insensitive file systems)
due to internal string comparison.
\item
The argument \textit{main} should be fully expanded, it cannot be a macro.
\item
Subdirectories and special characters should be avoided in filenames.
\item
The command |\childdocmain{|\textit{main}|}| must be followed by a whitespace.
It should not be followed immediately by another command
or by a comment mark `|%|'.
This is because the \TeX{} parser reads the token immediately following
the argument of |\childdocmain| and puts it
at the beginning of every child section;
however, a white\-space is ignored.
\end{itemize}

%%%%%%%%%%%%%%%%%%%%%%%%%%%%%%%%%%%%%%%%
\paragraph{Content of Main File.}

It is advisable to place all content in the child files included by |\include|.
Any output contained in the main file will appear in all child documents
unless suppressed manually;
it cannot be suppressed automatically by the |\includeonly| directive
and thus should normally be avoided.
A method to include some content in the main file
by means of conditional processing is described in \secref{sec:conditional}.

%%%%%%%%%%%%%%%%%%%%%%%%%%%%%%%%%%%%%%%%
\paragraph{Page Numbering.}

When only a part of the document is compiled,
the appropriate numbering of pages
(as well as other status parameters)
is determined from the |.aux| files.
The latter contain information from previous passes.
However this information needs to propagate through
all intermediate child documents.
Therefore the page numbering in child documents may well
be inconsistent until the complete document is compiled at least once.

A useful (if unconventional) way to always ensure a consistent
page numbering is to restart the numbering in each child document
and denote the pages by `\textit{child}|.|\textit{page}'
where \textit{child} represents the chapter/section number of the child file.
This can be achieved by the command
|\numberwithin{page}{|\textit{child}|}|
of the \textsf{amsmath} package
where \textit{child} can be |chapter| or |section|
depending on the chosen structuring.
Alternatively, one can modify the macro |\thepage| appropriately
and reset the counter |page| at the start of each child file.

%%%%%%%%%%%%%%%%%%%%%%%%%%%%%%%%%%%%%%%%%%%%%%%%%%%%%%%%%%%%%%%%%%%%%%%%%%%%%%%%
\subsection{Conditional Processing}
\label{sec:conditional}

The package provides a mechanism to compile different versions
of a document. To customise the versions further some conditional processing
can come in handy to distinguish which version is being compiled.
The package provides two macros to describe the compilation context:

%%%%%%%%%%%%%%%%%%%%%%%%%%%%%%%%%%%%%%%%
\DescribeMacro{\ifchilddoc}
The conditional |\ifchilddoc| distinguishes between the compilation of
child documents and the main document:
%
\begin{center}
|\ifchilddoc |\textit{child-code}| |[|\||else |\textit{main-code}]| \||fi|
\end{center}

%%%%%%%%%%%%%%%%%%%%%%%%%%%%%%%%%%%%%%%%
\DescribeMacro{\childdocname}
\DescribeMacro{\childdocjob}
The macro |\childdocname| contains the filename (without extension)
of the main or child file being processed.
Note that |\childdocjob| will always contain the name of the main file.

%%%%%%%%%%%%%%%%%%%%%%%%%%%%%%%%%%%%%%%%
\paragraph{Title Page.}

Conditional processing can be used to include a title or banner page
in the main document when proper precautions are taken.
Importantly, the code in the main file should ensure that the page counter
(as well as other status parameters which are stored in the |.aux| files)
takes the same value after the conditional processing.
Otherwise the page numbers may take divergent values
depending on which part is compiled.

For example, a title page could be declared by:
%
\begin{center}
\begin{tabular}{l}
|\ifchilddoc\||else|\\
|\addtocounter{page}{-1}|\\
\textit{code for title page}\\
|\newpage|\\
|\||fi|
\end{tabular}
\end{center}
%
A banner page for the child documents can be generated by:
%
\begin{center}
\begin{tabular}{l}
|\ifchilddoc|\\
|\addtocounter{page}{-1}|\\
\textit{code for banner page}\\
|\newpage|\\
|\||fi|
\end{tabular}
\end{center}
%
Here one could write a message such as:
\begin{center}
|This is the part \childdocname{} of \childdocjob{}.|
\end{center}

%%%%%%%%%%%%%%%%%%%%%%%%%%%%%%%%%%%%%%%%%%%%%%%%%%%%%%%%%%%%%%%%%%%%%%%%%%%%%%%%
\subsection{Flags}
\label{sec:flags}

The package makes it easy to generate different versions
of the main or child documents.
To this end compilation flags can be defined
and assigned different default values.
They will be particularly useful in conjunction
with the forwarding mechanism described in \secref{sec:forward}.

For example, it may be useful to have a flag |\version|
which can be set to |draft| or |final|.
The document source will contain some conditional code
depending on the value of |\version|.
Suppose further, the flag should default to |final| for the main file
and to |draft| for child files
which is a natural assignment for editing the document.
This is achieved by placing the following code
in the preamble of the main document
(below the |\childdocmain| directive):
%
\begin{center}
\begin{tabular}{l}
|\ifchilddoc|\\
|\providecommand{\version}{draft}|\\
|\||else|\\
|\providecommand{\version}{final}|\\
|\||fi|
\end{tabular}
\end{center}
%
The definition by |\providecommand| makes sure
that previous definitions are not overwritten.
Further statements |\providecommand{\version}{...}|
can thus be added before the above code to override it.

For the main file, one might add a line
(between |\childdocmain| and the above block)
%
\begin{center}
|%\ifchilddoc\||else\providecommand{\version}{draft}\||fi|
\end{center}
%
which can be uncommented to produce a draft version.
Likewise one can add a line to the very top of a child file
(above the |\childdocof{|\textit{main}|}| directive)
%
\begin{center}
|%\providecommand{\version}{final}|
\end{center}
%
which can be uncommented to produce the final version of this child document.

%%%%%%%%%%%%%%%%%%%%%%%%%%%%%%%%%%%%%%%%%%%%%%%%%%%%%%%%%%%%%%%%%%%%%%%%%%%%%%%%
\subsection{Forwarding}
\label{sec:forward}

Different versions of the main or child documents
using compilation flags as described in \secref{sec:flags}
can be (permanently) stored in different files
for convenient compilation, viewing and distribution.
To this end, the package defines a command
to pass on compilation to a different file:

%%%%%%%%%%%%%%%%%%%%%%%%%%%%%%%%%%%%%%%%
\DescribeMacro{\childdocforward}
The command |\childdocforward| redirects processing to
another source file:
%
\begin{center}
\begin{tabular}{l}
|\input{childdoc.def}|\\
|\childdocforward[|\textit{main}|]{|\textit{dest}|}|\\
\end{tabular}
\end{center}
%
The argument \textit{dest} is the destination file
(without extension).
It should be the main file or one of the child files.
Note that further \textsf{childdoc} directives
such as |\childdocof| and |\childdocforward|
in the indicated file will be processed in this form.
The optional argument \textit{main}
passes on directly to the main file \textit{main}
while pretending to compile the child \textit{dest}.
This form behaves as if \textit{dest}
issues |\childdocof{|\textit{main}|}| right away,
and no further \textsf{childdoc} directives will be processed.

%%%%%%%%%%%%%%%%%%%%%%%%%%%%%%%%%%%%%%%%
\DescribeMacro{\...prefix}
In the alternative form |\childdocforwardprefix|,
%
\begin{center}
\begin{tabular}{l}
|\input{childdoc.def}|\\
|\childdocforwardprefix[|\textit{main}|]{|\textit{prefix}|}{|\textit{dest}|}|
\end{tabular}
\end{center}
%
the destination file is determined by a pattern
depending on the current file:
To make this work, the current file must be called
`{\textit{prefix}\hspace{0.2em}\textit{suffix}}'
with \textit{prefix} matching precisely the argument.
Processing is then passed on to the file
`{\textit{dest}\hspace{0.2em}\textit{suffix}}'.
Surely, the same effect is achieved by
directly specifying the
argument `{\textit{dest}\hspace{0.2em}\textit{suffix}}'
in the first form.
However, that requires to set up a different file
for each child. With the alternative form of the command
all these files can have exactly the same content
which simplifies setting them up and maintaining them.

For example, the following file |draft.tex|
with a compilation flag |\version| as described in \secref{sec:flags}
compiles the main document as a draft:
%
\begin{center}
\begin{tabular}{l}
|\def\version{draft}|\\
|\input{childdoc.def}|\\
|\childdocforward{|\textit{main}|}|
\end{tabular}
\end{center}
%
Likewise, the following files |final|\textit{nn}|.tex|
compile the final version of the child document
|child|\textit{nn}|.tex|:
%
\begin{center}
\begin{tabular}{l}
|\def\version{final}|\\
|\input{childdoc.def}|\\
|\childdocforwardprefix{final}{child}|
\end{tabular}
\end{center}
%

Note that when several versions of a main file and/or of each child file
are to be generated, it may be convenient to set up a |Makefile| or
shell script to automatise the process.

%%%%%%%%%%%%%%%%%%%%%%%%%%%%%%%%%%%%%%%%%%%%%%%%%%%%%%%%%%%%%%%%%%%%%%%%%%%%%%%%
\subsection{Command Line Processing}
\label{sec:commandline}

The effect of redirection files can also be achieved by invoking
the \LaTeX{} compiler with a more elaborate command line.
Most conveniently this should be done as part
of a shell script or a |Makefile|.

When using \textsf{childdoc} in the main file, the following
command lines effectively perform a redirection
(note that depending on the shell being used,
backslashes may have to be doubled: `|\|' $\to$ `|\\|'):
%
\begin{center}
|... -jobname "|\textit{target}|" |\\|"|[\textit{flags}]%
|\input{childdoc.def}\childdocforward[|\textit{main}|]{|\textit{dest}|}"|
\end{center}
%
Here \textit{target} is the name of the output file,
\textit{main} is the name of the main file
and \textit{dest} is the name of the main or child file to be processed
(all filenames without extensions).
The optional argument \textit{main} can be omitted
if \textit{main} matches \textit{dest}.
Optionally, compilation \textit{flags} can be defined via |\def| commands.
This command line makes the \TeX{} engine believe
it is compiling the file \textit{target}
whose content is specified as the latter parameter.
The provided code then forwards the processing to
\textit{main} or \textit{dest} as described in \secref{sec:forward}.

%%%%%%%%%%%%%%%%%%%%%%%%%%%%%%%%%%%%%%%%%%%%%%%%%%%%%%%%%%%%%%%%%%%%%%%%%%%%%%%%
\subsection{Include by Input}
\label{sec:input}

Including child documents by |\include| has some restrictions by design.
Most notably, the content of a child document always occupies
its own set of pages; pages cannot be shared between child documents.
Usually, this behaviour makes perfect sense
because each child document contain an essential part of the document.
However, in some situations it may be desirable to compose
a document from a collection of parts
without having mandatory page breaks between then.
For this case, the package
provides a mechanism to include parts
by |\input| which can also be processed individually.
However, by construction this mechanism
requires manual handling of the content to be output.

%%%%%%%%%%%%%%%%%%%%%%%%%%%%%%%%%%%%%%%%
\DescribeMacro{\ifchilddocmanual}
The main file should be prepared as usual, see \secref{sec:include}.
However, the document body must make a distinction
between processing of an individual part and of the main document, e.g.:
%
\begin{center}
\begin{tabular}{l}
|\ifchilddocmanual|\\
|\input{\childdocname}|\\
|\||else|\\
\textit{document body with }|\input{|\textit{part}|}|\\
|\||fi|
\end{tabular}
\end{center}
%
The conditional |\ifchilddocmanual| is true whenever
a part to be included by |\input| is being compiled,
and the name of the part is stored in |\childdocname|.

%%%%%%%%%%%%%%%%%%%%%%%%%%%%%%%%%%%%%%%%
\DescribeMacro{\childdocby}
Each part to be included by |\input| should start with:
%
\begin{center}
\begin{tabular}{l}
|\input{childdoc.def}|\\
|\childdocby{|\textit{main}|}|\\
\end{tabular}
\end{center}
%
The directive |\childdocby| is similar to |\childdocof|
described in \secref{sec:include},
but the subsequent selection of content must be done manually.
To that end, both |\ifchilddoc| and |\ifchilddocmanual|
will be true upon processing of a part,
and the name of the part is stored in |\childdocname|.
Note that |\jobname| will be set to the filename of the current part
so that each part receives an individual |.aux| file
that does not interfere with the |.aux| file(s) of the main document.
This behaviour can be altered by the alternative form
|\childdocby[*]{|\textit{main}|}| (with a non-empty optional argument)
which uses the |.aux| file of the main document
by setting |\jobname| to \textit{main}.

%%%%%%%%%%%%%%%%%%%%%%%%%%%%%%%%%%%%%%%%%%%%%%%%%%%%%%%%%%%%%%%%%%%%%%%%%%%%%%%%
\subsection{Driver Development}
\label{sec:driver}

The \textsf{childdoc} mechanism can also be use for the development
of definition files such as \LaTeX{} styles or classes.
This case differs from the above setup with multiple parts
included by |\include| in that no |\includeonly| should be invoked.
This can be achieved by starting the include file
(before |\ProvidesPackage|) with:
%
\begin{center}
\begin{tabular}{l}
|\input{childdoc.def}|\\
|\childdocforward{|\textit{main}|}|\\
\end{tabular}
\end{center}
%
or alternatively with:
%
\begin{center}
\begin{tabular}{l}
|\input{childdoc.def}|\\
|\childdocby{|\textit{main}|}|\\
\end{tabular}
\end{center}
%
Both forms have slightly different effects as described above.
The main file is prepared as usual, see \secref{sec:include}.

%%%%%%%%%%%%%%%%%%%%%%%%%%%%%%%%%%%%%%%%%%%%%%%%%%%%%%%%%%%%%%%%%%%%%%%%%%%%%%%%
\subsection{Legacy Detection}
\label{sec:detection}

The directive |\childdocmain| in the main file can detect
whether the complete document or merely a child is to be compiled
even without using the directive |\childdocof|.
This method is deprecated because it is less robust
and there is no compelling reason to use it;
it is merely provided for backward compatibility
and it may be removed in future versions.

If the detection mechanism is to be used,
it is mandatory to correctly specify
the filename of the main file as the argument of |\childdocmain|:
%
\begin{center}
\begin{tabular}{l}
|\input{childdoc.def}|\\
|\childdocmain{|\textit{main}|}|\\
\end{tabular}
\end{center}
%
If |\jobname| does not match the argument \textit{main} of |\childdocmain|,
it is assumed that |\jobname| points to the child file to be compiled.
When using |\childdocmain| with the main file specified as argument,
it suffices to start a child file
with just |\input{|\textit{main}|}|
without loading of the package and using |\childdocof|.
If instead all processing is done
with the appropriate \textsf{childdoc} directives,
the argument of \textit{main} of |\childdocmain| can be empty.

An alternative version of the command line processing described
in \secref{sec:commandline} using the detection mechanism reads:
%
\begin{center}
|... -jobname "|\textit{target}|" "|[\textit{flags}]%
[|\def\jobname{|\textit{dest}|}|]|\input{|\textit{main}|}"|
\end{center}

%%%%%%%%%%%%%%%%%%%%%%%%%%%%%%%%%%%%%%%%%%%%%%%%%%%%%%%%%%%%%%%%%%%%%%%%%%%%%%%%
\subsection{Manual Code}
\label{sec:manual}

In case one cannot be certain whether the definitions file |childdoc.def|
is installed on the target \TeX{} distribution
and one prefers not to ship it,
it is conceivable to paste a few relevant commands into the sources.

To that end, drop all statements |\input{childdoc.def}|
and perform the replacements as outlined below.
Instead of |\childdocmain{|\textit{main}|}| add the following code
to the top of the main file:
%
\begin{center}
\begin{tabular}{l}
|\||ifdefined\childdocname\endinput\||fi\newif\ifchilddoc|\\
|\edef\childdocname{\scantokens\expandafter{\jobname\noexpand}}|\\
|\def\childdocmain{|\textit{main}|}\||ifx\childdocmain\childdocname\||else|\\
|\childdoctrue\includeonly{\childdocname}\let\jobname\childdocmain\||fi|\\
\end{tabular}
\end{center}
%
Instead of |\childdocof{|\textit{main}|}| just include the main file
at the top of each child file:
%
\begin{center}
|\input{|\textit{main}|}|
\end{center}
%
A simple redirection |\childdocforward{|\textit{dest}|}| is achieved by:
%
\begin{center}
|\def\jobname{|\textit{dest}|}\input{\jobname}|
\end{center}
%
The redirection with prefix
|\childdocforwardprefix[|\textit{prefix}|]{|\textit{dest}|}|
is accomplished by:
%
\begin{center}
\begin{tabular}{l}
|{\edef\jobname{\scantokens\expandafter{\jobname\noexpand}}|\\
|\def\redirectjob |\textit{prefix}|#1~~~{\gdef\jobname{|\textit{dest}|#1}}|\\
|\expandafter\redirectjob\jobname~~~}\input{\jobname}|
\end{tabular}
\end{center}

In an alternative approach,
child documents can be compiled by a specific command line
without additional code or specific definitions:
%
\begin{center}
|... -jobname "|\textit{target}|" "|[\textit{flags}]%
|\includeonly{|\textit{dest}|}\input{|\textit{main}|}"|
\end{center}
%

%%%%%%%%%%%%%%%%%%%%%%%%%%%%%%%%%%%%%%%%%%%%%%%%%%%%%%%%%%%%%%%%%%%%%%%%%%%%%%%%
%%%%%%%%%%%%%%%%%%%%%%%%%%%%%%%%%%%%%%%%%%%%%%%%%%%%%%%%%%%%%%%%%%%%%%%%%%%%%%%%
\section{Information}

%%%%%%%%%%%%%%%%%%%%%%%%%%%%%%%%%%%%%%%%%%%%%%%%%%%%%%%%%%%%%%%%%%%%%%%%%%%%%%%%
\subsection{Copyright}

Copyright \copyright{} 2017--2018 Niklas Beisert

This work may be distributed and/or modified under the
conditions of the \LaTeX{} Project Public License, either version 1.3
of this license or (at your option) any later version.
The latest version of this license is in
  \url{http://www.latex-project.org/lppl.txt}
and version 1.3 or later is part of all distributions of \LaTeX{}
version 2005/12/01 or later.

This work has the LPPL maintenance status `maintained'.

The Current Maintainer of this work is Niklas Beisert.

This work consists of the files |README.txt|, |childdoc.ins| and |childdoc.dtx|
as well as the derived files |childdoc.def|, |cdocsamp.tex|
with |cdocsch1.tex|, |cdocsch2.tex|, |cdocspt3.tex|, |cdocspt4.tex|,
|cdocsdrf.tex|, |cdocsfn1.tex|, |cdocsfn2.tex|
as well as |childdoc.pdf|.

%%%%%%%%%%%%%%%%%%%%%%%%%%%%%%%%%%%%%%%%%%%%%%%%%%%%%%%%%%%%%%%%%%%%%%%%%%%%%%%%
\subsection{Files and Installation}

The package consists of the files:
%
\begin{center}
\begin{tabular}{ll}
    |README.txt|   & readme file \\
    |childdoc.ins| & installation file \\
    |childdoc.dtx| & source file \\
    |childdoc.def| & definition file \\
    |cdocsamp.tex| & sample main file \\
    |cdocsch1.tex| & sample include file \\
    |cdocsch2.tex| & sample include file \\
    |cdocspt3.tex| & sample part file \\
    |cdocspt4.tex| & sample part file \\
    |cdocsdrf.tex| & sample redirection file \\
    |cdocsfn1.tex| & sample redirection file \\
    |cdocsfn2.tex| & sample redirection file \\
    |childdoc.pdf| & manual
\end{tabular}
\end{center}
%
The distribution consists of the files
|README.txt|, |childdoc.ins| and |childdoc.dtx|.
%
\begin{itemize}
\item
Run (pdf)\LaTeX{} on |childdoc.dtx|
to compile the manual |childdoc.pdf| (this file).
\item
Run \LaTeX{} on |childdoc.ins| to create the definitions file |childdoc.def|
and the sample |cdocsamp.tex| with include files
|cdocsch1.tex|, |cdocsch2.tex|, |cdocspt3.tex|, |cdocspt4.tex|,
|cdocsdrf.tex|, |cdocsfn1.tex|, |cdocsfn2.tex|.
Then copy the file |childdoc.def| to an appropriate directory of your \LaTeX{}
distribution, e.g.\ \textit{texmf-root}|/tex/latex/childdoc|.
\end{itemize}

%%%%%%%%%%%%%%%%%%%%%%%%%%%%%%%%%%%%%%%%%%%%%%%%%%%%%%%%%%%%%%%%%%%%%%%%%%%%%%%%
\subsection{Related CTAN Packages}

There are several other packages which offer a similar functionality:
%
\begin{itemize}
\item
The packages
\href{http://ctan.org/pkg/docmute}{\textsf{docmute}},
\href{http://ctan.org/pkg/includex}{\textsf{includex}} and
\href{http://ctan.org/pkg/standalone}{\textsf{standalone}}
provide commands to include only the document body of
a child file thus allowing both files to be compiled individually.
\item
The packages \href{http://ctan.org/pkg/subdocs}{\textsf{subdocs}}
and \href{http://ctan.org/pkg/subfiles}{\textsf{subfiles}}
provide structures in which the main and child documents can be
encapsulated and allowing them to be compiled individually.
The inclusion mechanism is different from the conventional |\include|.
\item
The package \href{http://ctan.org/pkg/combine}{\textsf{combine}}
is an elaborate solution to combine several documents into one.
\end{itemize}
%
See also the CTAN topic \href{http://ctan.org/topic/subdocs}{\textsf{subdocs}}
for further related packages.
The present package differs from the above solutions in that
a document structure constructed with the conventional |\include| mechanism
just needs two extra commands at the top of every file
such that all constituent files can be compiled individually.

%%%%%%%%%%%%%%%%%%%%%%%%%%%%%%%%%%%%%%%%%%%%%%%%%%%%%%%%%%%%%%%%%%%%%%%%%%%%%%%%
%\subsection{Feature Suggestions}
%
%The following is a list of features which may be useful for future
%versions of this package:
%%
%\begin{itemize}
%\item
%\ldots
%\end{itemize}

%%%%%%%%%%%%%%%%%%%%%%%%%%%%%%%%%%%%%%%%%%%%%%%%%%%%%%%%%%%%%%%%%%%%%%%%%%%%%%%%
\subsection{Revision History}

%%%%%%%%%%%%%%%%%%%%%%%%%%%%%%%%%%%%%%%%
\paragraph{v2.0:} 2018/12/30

\begin{itemize}
\item
immediate forward processing
\item
added |\childdocby| mechanism
\item
manual restructured
\end{itemize}

%%%%%%%%%%%%%%%%%%%%%%%%%%%%%%%%%%%%%%%%
\paragraph{v1.6:} 2018/01/17

\begin{itemize}
\item
application for development of include files
\item
corrections to manual
\end{itemize}

%%%%%%%%%%%%%%%%%%%%%%%%%%%%%%%%%%%%%%%%
\paragraph{v1.5:} 2017/05/21

\begin{itemize}
\item
more complete structuring introduced
\item
|\childdocof| introduced
\item
|\childdoc| renamed to |\childdocmain|
\item
|\childredirect| renamed to |\childdocforward| and |\childdocforwardprefix|
and functionality expanded
\end{itemize}

%%%%%%%%%%%%%%%%%%%%%%%%%%%%%%%%%%%%%%%%
\paragraph{v1.0:} 2017/04/27

\begin{itemize}
\item
manual and install package
\item
first version published on CTAN
\end{itemize}

%%%%%%%%%%%%%%%%%%%%%%%%%%%%%%%%%%%%%%%%
\paragraph{v0.6:} 2017/04/26

\begin{itemize}
\item
redirection mechanism added
\end{itemize}

%%%%%%%%%%%%%%%%%%%%%%%%%%%%%%%%%%%%%%%%
\paragraph{v0.5:} 2017/04/26

\begin{itemize}
\item
functionality in definition file
\end{itemize}


%%%%%%%%%%%%%%%%%%%%%%%%%%%%%%%%%%%%%%%%%%%%%%%%%%%%%%%%%%%%%%%%%%%%%%%%%%%%%%%%
%%%%%%%%%%%%%%%%%%%%%%%%%%%%%%%%%%%%%%%%%%%%%%%%%%%%%%%%%%%%%%%%%%%%%%%%%%%%%%%%
%%%%%%%%%%%%%%%%%%%%%%%%%%%%%%%%%%%%%%%%%%%%%%%%%%%%%%%%%%%%%%%%%%%%%%%%%%%%%%%%
\appendix

\settowidth\MacroIndent{\rmfamily\scriptsize 000\ }

 \DocInput{childdoc.dtx}

\end{document}
%</driver>
% \fi
%
% %%%%%%%%%%%%%%%%%%%%%%%%%%%%%%%%%%%%%%%%%%%%%%%%%%%%%%%%%%%%%%%%%%%%%%%%%%%%%%
% %%%%%%%%%%%%%%%%%%%%%%%%%%%%%%%%%%%%%%%%%%%%%%%%%%%%%%%%%%%%%%%%%%%%%%%%%%%%%%
% \section{Sample}
%\iffalse
%<*samplemain>
%\fi
%
% The following presents a sample document
% with two chapters, two parts, a title page,
% a compile flag as well as three forwarding files to set the flag.
% It consists of eight |.tex| files:
% \begin{center}
% \begin{tabular}{ll}
% |cdocsamp.tex|&main file\\
% |cdocsch1.tex|&include file for chapter 1\\
% |cdocsch2.tex|&include file for chapter 2\\
% |cdocspt3.tex|&include file for part 3\\
% |cdocspt4.tex|&include file for part 4\\
% |cdocsdrf.tex|&forwarding file for main file in draft mode\\
% |cdocsfi1.tex|&forwarding file for final version of chapter 1\\
% |cdocsfi2.tex|&forwarding file for final version of chapter 2\\
% \end{tabular}
% \end{center}
% Each of the eight files can be compiled directly by the \LaTeX{} compiler.
%
% %%%%%%%%%%%%%%%%%%%%%%%%%%%%%%%%%%%%%%
% \paragraph{Main File.}
%
% The main file is called |cdocsamp.tex|.
%
% Load the \textsf{childdoc} definitions and
% declare the filename for the main document:
%    \begin{macrocode}
\input{childdoc.def}
\childdocmain{}
%    \end{macrocode}

% Optional override for |\version| flag:
%    \begin{macrocode}
%%\ifchilddoc\else\providecommand{\version}{draft}\fi
%    \end{macrocode}

% Define the default values for the |\version| flag
% (|final| for the main file and |draft| for childs):
%    \begin{macrocode}
\ifchilddoc
\providecommand{\version}{draft}
\else
\providecommand{\version}{final}
\fi
%    \end{macrocode}

% Load the standard document class:
%    \begin{macrocode}
\documentclass[12pt]{article}
%    \end{macrocode}

% Start the document body:
%    \begin{macrocode}
\begin{document}
%    \end{macrocode}

% Declare a title page.
% Print title, part of document being processed and version flag:
%    \begin{macrocode}
\addtocounter{page}{-1}
\begin{center}
{\LARGE\bfseries{}childdoc example\par}
\vspace{1cm}
\ifchilddoc
\ifchilddocmanual part\else chapter\fi:
`\childdocname' of `\childdocjob'\par
\else
main document: `\childdocjob'\par
\fi
version: \version\par
\end{center}
\newpage
%    \end{macrocode}

% Manually include selected file,
% otherwise process as usual:
%    \begin{macrocode}
\ifchilddocmanual
\section*{part `\childdocname'}
\input{\childdocname}
\else
%    \end{macrocode}

% Include the two chapters:
%    \begin{macrocode}
\include{cdocsch1}
\include{cdocsch2}
%    \end{macrocode}

% Include the two parts unless only chapters should be displayed:
%    \begin{macrocode}
\ifchilddoc\else
\section{part three}
\input{cdocspt3}
\section{part four}
\input{cdocspt4}
\fi
%    \end{macrocode}

% Process as usual until here:
%    \begin{macrocode}
\fi
%    \end{macrocode}

% End of document body:
%    \begin{macrocode}
\end{document}
%    \end{macrocode}
%\iffalse
%</samplemain>
%\fi
%
% %%%%%%%%%%%%%%%%%%%%%%%%%%%%%%%%%%%%%%
% \paragraph{Chapter Include Files.}
%
% The include files are called |cdocsch1.tex| and |cdocsch2.tex|.
%
%\iffalse
%<*samplechap1|samplechap2>
%\fi

% Optional override for |\version| flag:
%    \begin{macrocode}
%%\providecommand{\version}{final}
%    \end{macrocode}

% Include the main document:
%    \begin{macrocode}
\input{childdoc.def}
\childdocof{cdocsamp}
%    \end{macrocode}

%\iffalse
%</samplechap1|samplechap2>
%\fi
%
%\iffalse
%<*samplechap1>
%\fi
% Some text for chapter 1:
%    \begin{macrocode}
\section{one}
some text in chapter one
%    \end{macrocode}

%\iffalse
%</samplechap1>
%\fi
% Some text for chapter 2:
%\iffalse
%<*samplechap2>
%\fi
%    \begin{macrocode}
\section{two}
more text in chapter two
%    \end{macrocode}

%\iffalse
%</samplechap2>
%\fi
%
% %%%%%%%%%%%%%%%%%%%%%%%%%%%%%%%%%%%%%%
% \paragraph{Part Include Files.}
%
% The include files are called |cdocspt3.tex| and |cdocspt4.tex|.
%
%\iffalse
%<*samplepart3|samplepart4>
%\fi

% Optional override for |\version| flag:
%    \begin{macrocode}
%%\providecommand{\version}{final}
%    \end{macrocode}

% Include the main document:
%    \begin{macrocode}
\input{childdoc.def}
\childdocby{cdocsamp}
%    \end{macrocode}

%\iffalse
%</samplepart3|samplepart4>
%\fi
%
%\iffalse
%<*samplepart3>
%\fi
% Some text for part 3:
%    \begin{macrocode}
some text in part three
%    \end{macrocode}

%\iffalse
%</samplepart3>
%\fi
% Some text for part 4:
%\iffalse
%<*samplepart4>
%\fi
%    \begin{macrocode}
more text in part four
%    \end{macrocode}

%\iffalse
%</samplepart4>
%\fi
%
% %%%%%%%%%%%%%%%%%%%%%%%%%%%%%%%%%%%%%%
% \paragraph{Forwarding for a Complete Draft.}
%
% The following forwarding file |cdocsdrf.tex|
% compiles the main document in draft mode:
%\iffalse
%<*sampledraft>
%\fi
%    \begin{macrocode}
\def\version{draft}
\input{childdoc.def}
\childdocforward{cdocsamp}
%    \end{macrocode}

%\iffalse
%</sampledraft>
%\fi
%
% %%%%%%%%%%%%%%%%%%%%%%%%%%%%%%%%%%%%%%
% \paragraph{Forwarding for Final Version of the Chapters.}
%
% The following forwarding files |cdocsfn1.tex| and |cdocsfn2.tex|
% (with identical content)
% compile the final versions of the child documents
% |cdocsch1.tex| and |cdocsch2.tex|, respectively:
%\iffalse
%<*samplefinal>
%\fi
%    \begin{macrocode}
\def\version{final}
\input{childdoc.def}
\childdocforwardprefix[cdocsamp]{cdocsfn}{cdocsch}
%    \end{macrocode}

%\iffalse
%</samplefinal>
%\fi
%
% %%%%%%%%%%%%%%%%%%%%%%%%%%%%%%%%%%%%%%
% \paragraph{Command Line Processing.}
%
% The following three command lines generate the output files
% |cdocscld|, |cdocscl1| and |cdocscl2|
% which should be identical to
% |cdocsdrf|, |cdocsch1| and |cdocsfn2|, respectively:
% \begin{center}
% \begin{tabular}{l}
% |latex -jobname cdocscld \|\\
% |  "\def\version{draft}\input{childdoc.def}\childdocforward{cdocsamp}"|\\
% |latex -jobname cdocscl1 \|\\
% |  "\input{childdoc.def}\childdocforward[cdocsamp]{cdocsch1}"|\\
% |latex -jobname cdocscl2 \|\\
% |  "\def\version{final}\input{childdoc.def}\childdocforward{cdocsch2}"|
% \end{tabular}
% \end{center}
% Note that the trailing backslash on each first line
% merely continues the input to the second line
% (for convenient cut ant paste).
% Furthermore, the command |latex| can be replaced by any
% of its alternative versions such as |pdflatex|.
%
% %%%%%%%%%%%%%%%%%%%%%%%%%%%%%%%%%%%%%%%%%%%%%%%%%%%%%%%%%%%%%%%%%%%%%%%%%%%%%%
% %%%%%%%%%%%%%%%%%%%%%%%%%%%%%%%%%%%%%%%%%%%%%%%%%%%%%%%%%%%%%%%%%%%%%%%%%%%%%%
% \section{Implementation}
%\iffalse
%<*package>
%\fi
%
% This section describes the definitions file |childdoc.def|.

% The definitions cannot be loaded using |\usepackage| or |\RequirePackage|
% which has a mechanism to prevent loading a style file more than once.
% When loading the definitions by means of |\input|
% multiple instances have to be prevented manually:
%\iffalse
%This code needs to be before the `\ProvidesFile' directive
%which is defined at the beginning of this file.
%Therefore it is also placed there and commented out here.
%</package>
%<*discard>
%\fi
%    \begin{macrocode}
\ifdefined\childdocmain\endinput\fi
%    \end{macrocode}
%\iffalse
%</discard>
%<*package>
%\fi
%
% \macro{\ifchilddoc}
% \macro{\ifchilddocmanual}
% The conditional |\ifchilddoc| tells whether a
% child (true) or main (false) document is being compiled.
% The conditional |\ifchilddocmanual| tells whether
% the |\includeonly| mechanism is used (false) or
% the selection of child files must be performed manually (true).
% The definitions initialise to false:
%    \begin{macrocode}
\newif\ifchilddoc
\newif\ifchilddocmanual
%    \end{macrocode}

% \macro{\childdocname}
% \macro{\childdocjob}
% The macro |\childdocname| stores the name of the main document
% to be compiled. The macro |\childdocjob| stores the name of
% the document on which the \LaTeX{} compiler was originally invoked.
% The content of |\jobname| cannot be compared
% to filenames specified in the source due to different catcodes.
% The following code rescans |\jobname|, stores the result
% in |\childdocname| and saves a copy in |\childdocjob|:
%    \begin{macrocode}
\edef\childdocname{\scantokens\expandafter{\jobname\noexpand}}
\let\childdocjob\childdocname
%    \end{macrocode}

% \macro{\childdocdisable}
% The macro |\childdocdisable| prevents the main file
% from being processed more than once.
% At this stage, the main document command |\childdocmain|
% is assumed to be called once again where it should do nothing.
% Any subsequent call to it should prevent
% a secondary processing of the main document
% It overwrites the forwarding commands
% |\childdocof| and |\childdocforward|
% with empty macros to prevent further inclusions of the main document:
%    \begin{macrocode}
\newcommand{\childdocdisable}
{
  \renewcommand{\childdocmain}[1]{\renewcommand{\childdocmain}[1]{\endinput}}
  \renewcommand{\childdocof}[1]{}
  \renewcommand{\childdocby}[2][]{}
  \renewcommand{\childdocforward}[2][]{}
  \renewcommand{\childdocdisable}{}
}
%    \end{macrocode}

% \macro{\childdocmain}
% The macro |\childdocmain| is to be called at the top of the main file
% with nothing or the main filename (without extension) as argument.
% First, it breaks loops.
% If the argument is not empty and does not match |\childdocname|
% (which is set by the first inclusion of |childdoc.def|),
% |\ifchilddoc| is set to true, |\includeonly| is applied to the child file
% and |\jobname| is set to the main file
% (for proper handling of |.aux| files):
%    \begin{macrocode}
\newcommand{\childdocmain}[1]
{
  \childdocdisable\childdocmain{}
  \if?#1?\else
    \begingroup
      \def\childdoctmp{#1}
      \ifx\childdoctmp\childdocname
        \def\childdoctmp{}
      \else
        \def\childdoctmp
        {
          \childdoctrue
          \includeonly{\childdocname}
          \def\childdocjob{#1}
          \def\jobname{#1}
        }
      \fi
      \expandafter
    \endgroup
    \childdoctmp
  \fi
}
%    \end{macrocode}

% \macro{\childdocof}
% The command |\childdocof| redirects
% compilation to the main file |#1|.
%    \begin{macrocode}
\newcommand{\childdocof}[1]
{
  \childdocdisable
  \childdoctrue
  \includeonly{\childdocname}
  \def\jobname{#1}
  \def\childdocjob{#1}
  \input{#1}
}
%    \end{macrocode}

% \macro{\childdocby}
% The command |\childdocby| ....
%    \begin{macrocode}
\newcommand{\childdocby}[2][]
{
  \childdocdisable
  \childdoctrue
  \childdocmanualtrue
  \if?#1?\else
    \def\jobname{#2}
  \fi
  \def\childdocjob{#2}
  \input{#2}
  \endinput
}
%    \end{macrocode}

% \macro{\childdocforward}
% The command |\childdocforward| redirects
% compilation to the main file or
% (if the optional argument is given) a child file.
% Parameters are set as if the main file
% or a child file starting with |\childdocof| was compiled.
% Then compilation is handed over to the main file:
%    \begin{macrocode}
\newcommand{\childdocforward}[2][]
{
  \begingroup
    \if?#1?
      \def\childdoctmp
      {
        \def\childdocname{#2}
        \def\childdocjob{#2}
        \def\jobname{#2}
        \input{#2}
        \endinput
      }
    \else
      \def\childdoctmp
      {
        \childdocdisable
        \def\childdocname{#2}
        \childdoctrue
        \includeonly{#2}
        \def\childdocjob{#1}
        \def\jobname{#1}
        \input{#1}
        \endinput
      }
    \fi
    \expandafter
  \endgroup
  \childdoctmp
}
%    \end{macrocode}

% \macro{\childdocforwardprefix}
% The command |\childdocforwardprefix| redirects
% compilation to the main or a child file by means of a pattern.
% The prefix |#1| in the current filename is replaced by |#2|
% and the suffix of the current filename is kept
% (it is assumed that the filename does not contain the substring `|~~~|'
% which is used as a delimiter).
% Compilation is handed over to the new file by |\childdocforward|:
%    \begin{macrocode}
\newcommand{\childdocforwardprefix}[3][]
{
  \begingroup
    \def\childdocextract #2##1~~~{\def\childdoctmp{\childdocforward[#1]{#3##1}}}
    \expandafter\childdocextract\childdocname~~~
    \expandafter
  \endgroup
  \childdoctmp
}
%    \end{macrocode}

% \macro{\childdoc}
% The deprecated macro |\childdoc| is a legacy version of |\childdocmain|:
%    \begin{macrocode}
\newcommand{\childdoc}{\childdocmain}
%    \end{macrocode}

% \macro{\childdocredirect}
% The deprecated macro |\childdocredirect| is a legacy version
% of |\childdocforward| and |\childdocforwardprefix|:
%    \begin{macrocode}
\newcommand{\childdocredirect}[2][]
{
  \begingroup
    \if?#1?
      \def\childdoctmp{\childdocforward{#2}}
    \else
      \def\childdoctmp{\childdocforwardprefix{#1}{#2}}
    \fi
    \expandafter
  \endgroup
  \childdoctmp
}
%    \end{macrocode}

%\iffalse
%</package>
%\fi
%
\endinput

\childdocby{cdocsamp}
%    \end{macrocode}

%\iffalse
%</samplepart3|samplepart4>
%\fi
%
%\iffalse
%<*samplepart3>
%\fi
% Some text for part 3:
%    \begin{macrocode}
some text in part three
%    \end{macrocode}

%\iffalse
%</samplepart3>
%\fi
% Some text for part 4:
%\iffalse
%<*samplepart4>
%\fi
%    \begin{macrocode}
more text in part four
%    \end{macrocode}

%\iffalse
%</samplepart4>
%\fi
%
% %%%%%%%%%%%%%%%%%%%%%%%%%%%%%%%%%%%%%%
% \paragraph{Forwarding for a Complete Draft.}
%
% The following forwarding file |cdocsdrf.tex|
% compiles the main document in draft mode:
%\iffalse
%<*sampledraft>
%\fi
%    \begin{macrocode}
\def\version{draft}
% \iffalse
%
% childdoc.dtx Copyright (C) 2017-2018 Niklas Beisert
%
% This work may be distributed and/or modified under the
% conditions of the LaTeX Project Public License, either version 1.3
% of this license or (at your option) any later version.
% The latest version of this license is in
%   http://www.latex-project.org/lppl.txt
% and version 1.3 or later is part of all distributions of LaTeX
% version 2005/12/01 or later.
%
% This work has the LPPL maintenance status `maintained'.
%
% The Current Maintainer of this work is Niklas Beisert.
%
% This work consists of the files childdoc.dtx and childdoc.ins
% and the derived files childdoc.def and cdocsamp.tex with
% cdocsch1.tex, cdocsch2.tex, cdocsdrf.tex, cdocsfn1.tex, cdocsfn2.tex.
%
%<package>\ifdefined\childdocmain\endinput\fi
%<package>\ProvidesFile{childdoc.def}[2018/12/30 v2.0 child document driver]
%<samplemain>\ProvidesFile{cdocsamp.tex}[2018/12/30 v2.0 sample for childdoc]
%<*driver>
%\ProvidesFile{childdoc.drv}[2018/12/30 v2.0 childdoc reference manual file]
\PassOptionsToClass{10pt,a4paper}{article}
\documentclass{ltxdoc}

\usepackage[margin=35mm]{geometry}
\usepackage{hyperref}
\usepackage{hyperxmp}
\usepackage[usenames]{color}

\hypersetup{colorlinks=true}
\hypersetup{pdfstartview=FitH}
\hypersetup{pdfpagemode=UseNone}
\hypersetup{pdfsource={}}
\hypersetup{pdflang={en-UK}}
\hypersetup{pdfcopyright={Copyright 2017-2018 Niklas Beisert.
  This work may be distributed and/or modified under the
  conditions of the LaTeX Project Public License, either version 1.3
  of this license or (at your option) any later version.}}
\hypersetup{pdflicenseurl={http://www.latex-project.org/lppl.txt}}
\hypersetup{pdfcontactaddress={ETH Zurich, ITP, HIT K,
  Wolfgang-Pauli-Strasse 27}}
\hypersetup{pdfcontactpostcode={8093}}
\hypersetup{pdfcontactcity={Zurich}}
\hypersetup{pdfcontactcountry={Switzerland}}
\hypersetup{pdfcontactemail={nbeisert@itp.phys.ethz.ch}}
\hypersetup{pdfcontacturl={http://people.phys.ethz.ch/\xmptilde nbeisert/}}

\newcommand{\secref}[1]{\hyperref[#1]{section \ref*{#1}}}

\parskip1ex
\parindent0pt
\let\olditemize\itemize
\def\itemize{\olditemize\parskip0pt}

\begin{document}

\title{The \textsf{childdoc} Package}
\hypersetup{pdftitle={The childdoc Package}}
\author{Niklas Beisert\\[2ex]
  Institut f\"ur Theoretische Physik\\
  Eidgen\"ossische Technische Hochschule Z\"urich\\
  Wolfgang-Pauli-Strasse 27, 8093 Z\"urich, Switzerland\\[1ex]
  \href{mailto:nbeisert@itp.phys.ethz.ch}
  {\texttt{nbeisert@itp.phys.ethz.ch}}}
\hypersetup{pdfauthor={Niklas Beisert}}
\hypersetup{pdfsubject={Manual for the LaTeX2e Package childdoc}}
\date{30 December 2018, \textsf{v2.0}}
\maketitle

\begin{abstract}\noindent
\textsf{childdoc} is a \LaTeXe{} package
that enables the direct compilation
of document sections included by |\include|
to individual files.
\end{abstract}

\begingroup
\parskip0ex
\tableofcontents
\endgroup

%%%%%%%%%%%%%%%%%%%%%%%%%%%%%%%%%%%%%%%%%%%%%%%%%%%%%%%%%%%%%%%%%%%%%%%%%%%%%%%%
%%%%%%%%%%%%%%%%%%%%%%%%%%%%%%%%%%%%%%%%%%%%%%%%%%%%%%%%%%%%%%%%%%%%%%%%%%%%%%%%
\section{Introduction}

\LaTeX{} provides a mechanism to structure a large document (such as a book)
into a main file and several child files (containing the chapters)
using the |\include| command.
This mechanism is beneficial for documents
which span hundreds of pages in order to
make the source file(s) more manageable.
Moreover, compilation can be restricted to
selected child files by means of the |\includeonly| command.
The latter feature can be used to reduce the compilation time while editing
(this was significantly more useful in the earlier days of \LaTeX{})
or to generate a smaller document which is easier to navigate.
Another application of |\includeonly| is to generate
documents consisting of selected parts of the complete document.

However, there are a few drawbacks of the plain |\include| mechanism:
\begin{itemize}
\item
The child files cannot be compiled on their own,
they can only be compiled via the main file.
A naive editing environment
(such as a text editor with an option
to have the current file processed by \LaTeX)
may require one to switch to the main file before compiling;
attempting to compile the child file produces errors.
\item
The main file must be modified (each time)
to adjust the |\includeonly| command
to the present needs. This easily leaves the main file in a messy state.
\item
The generated document will always carry the filename
of the main document. This is inconvenient if
several child files are to be compiled and
to be kept for distribution.
\end{itemize}

The present package provides a simple interface
to make child files individually compilable by \LaTeX{}.
Compiling a child file then has the same effect as compiling
the main file with an |\includeonly| command
to select the appropriate child.
Moreover the generated document will carry the name of the child
rather than the main file.
This resolves all three above issues.

This feature is meant to make the editing of books,
thesis documents and lecture notes somewhat more convenient.
However, the package can also be used efficiently for
composing a series of documents (such as exercise sheets)
which are typically distributed individually.
It then assists the author in generating the individual documents
(potentially in different versions)
as well as a document containing the collected series.
Another application is in developing style files
or other kinds of included material
where compilation of the style file could redirect
to a sample or test file.

%%%%%%%%%%%%%%%%%%%%%%%%%%%%%%%%%%%%%%%%%%%%%%%%%%%%%%%%%%%%%%%%%%%%%%%%%%%%%%%%
%%%%%%%%%%%%%%%%%%%%%%%%%%%%%%%%%%%%%%%%%%%%%%%%%%%%%%%%%%%%%%%%%%%%%%%%%%%%%%%%
\section{Usage}

First of all, the package \textsf{childdoc} is \emph{not} a standard
\LaTeXe{} |.sty| style file! Therefore it needs to be invoked in
a non-standard way.

%%%%%%%%%%%%%%%%%%%%%%%%%%%%%%%%%%%%%%%%%%%%%%%%%%%%%%%%%%%%%%%%%%%%%%%%%%%%%%%%
\subsection{Included Files}
\label{sec:include}

%%%%%%%%%%%%%%%%%%%%%%%%%%%%%%%%%%%%%%%%
\DescribeMacro{\childdocmain}
To use the package, add the commands
\begin{center}
\begin{tabular}{l}
|\input{childdoc.def}|\\
|\childdocmain{}|\\
\end{tabular}
\end{center}
at the very top of the main \LaTeX{} file,
in particular \emph{before} the |\documentclass| statement!
The argument of |\childdocmain| should be left empty
(but it must be present).

%%%%%%%%%%%%%%%%%%%%%%%%%%%%%%%%%%%%%%%%
\DescribeMacro{\childdocof}
Furthermore, add the commands
\begin{center}
\begin{tabular}{l}
|\input{childdoc.def}|\\
|\childdocof{|\textit{main}|}|\\
\end{tabular}
\end{center}
at the top of every child file \textit{child}
which is included by |\include{|\textit{child}|}|
from within the main file
(or at least for those files to be compiled individually).
The argument \textit{main} must be the filename of the main file.

There are a couple of
considerations in setting up the main and child documents:

%%%%%%%%%%%%%%%%%%%%%%%%%%%%%%%%%%%%%%%%
\paragraph{Restrictions.}

Please note the following restrictions:
\begin{itemize}
\item
|\childdocmain| must be called with one argument \textit{main}
to ensure compatibility with earlier version of the package.
It must either be empty (|\childdocmain{}|)
or precisely match the filename of the main file in which it is specified.
See \secref{sec:detection} for further information.
\item
The filename \textit{main} must be specified without the |.tex| extension.
\item
The filename \textit{main} is case sensitive
(even in case-insensitive file systems)
due to internal string comparison.
\item
The argument \textit{main} should be fully expanded, it cannot be a macro.
\item
Subdirectories and special characters should be avoided in filenames.
\item
The command |\childdocmain{|\textit{main}|}| must be followed by a whitespace.
It should not be followed immediately by another command
or by a comment mark `|%|'.
This is because the \TeX{} parser reads the token immediately following
the argument of |\childdocmain| and puts it
at the beginning of every child section;
however, a white\-space is ignored.
\end{itemize}

%%%%%%%%%%%%%%%%%%%%%%%%%%%%%%%%%%%%%%%%
\paragraph{Content of Main File.}

It is advisable to place all content in the child files included by |\include|.
Any output contained in the main file will appear in all child documents
unless suppressed manually;
it cannot be suppressed automatically by the |\includeonly| directive
and thus should normally be avoided.
A method to include some content in the main file
by means of conditional processing is described in \secref{sec:conditional}.

%%%%%%%%%%%%%%%%%%%%%%%%%%%%%%%%%%%%%%%%
\paragraph{Page Numbering.}

When only a part of the document is compiled,
the appropriate numbering of pages
(as well as other status parameters)
is determined from the |.aux| files.
The latter contain information from previous passes.
However this information needs to propagate through
all intermediate child documents.
Therefore the page numbering in child documents may well
be inconsistent until the complete document is compiled at least once.

A useful (if unconventional) way to always ensure a consistent
page numbering is to restart the numbering in each child document
and denote the pages by `\textit{child}|.|\textit{page}'
where \textit{child} represents the chapter/section number of the child file.
This can be achieved by the command
|\numberwithin{page}{|\textit{child}|}|
of the \textsf{amsmath} package
where \textit{child} can be |chapter| or |section|
depending on the chosen structuring.
Alternatively, one can modify the macro |\thepage| appropriately
and reset the counter |page| at the start of each child file.

%%%%%%%%%%%%%%%%%%%%%%%%%%%%%%%%%%%%%%%%%%%%%%%%%%%%%%%%%%%%%%%%%%%%%%%%%%%%%%%%
\subsection{Conditional Processing}
\label{sec:conditional}

The package provides a mechanism to compile different versions
of a document. To customise the versions further some conditional processing
can come in handy to distinguish which version is being compiled.
The package provides two macros to describe the compilation context:

%%%%%%%%%%%%%%%%%%%%%%%%%%%%%%%%%%%%%%%%
\DescribeMacro{\ifchilddoc}
The conditional |\ifchilddoc| distinguishes between the compilation of
child documents and the main document:
%
\begin{center}
|\ifchilddoc |\textit{child-code}| |[|\||else |\textit{main-code}]| \||fi|
\end{center}

%%%%%%%%%%%%%%%%%%%%%%%%%%%%%%%%%%%%%%%%
\DescribeMacro{\childdocname}
\DescribeMacro{\childdocjob}
The macro |\childdocname| contains the filename (without extension)
of the main or child file being processed.
Note that |\childdocjob| will always contain the name of the main file.

%%%%%%%%%%%%%%%%%%%%%%%%%%%%%%%%%%%%%%%%
\paragraph{Title Page.}

Conditional processing can be used to include a title or banner page
in the main document when proper precautions are taken.
Importantly, the code in the main file should ensure that the page counter
(as well as other status parameters which are stored in the |.aux| files)
takes the same value after the conditional processing.
Otherwise the page numbers may take divergent values
depending on which part is compiled.

For example, a title page could be declared by:
%
\begin{center}
\begin{tabular}{l}
|\ifchilddoc\||else|\\
|\addtocounter{page}{-1}|\\
\textit{code for title page}\\
|\newpage|\\
|\||fi|
\end{tabular}
\end{center}
%
A banner page for the child documents can be generated by:
%
\begin{center}
\begin{tabular}{l}
|\ifchilddoc|\\
|\addtocounter{page}{-1}|\\
\textit{code for banner page}\\
|\newpage|\\
|\||fi|
\end{tabular}
\end{center}
%
Here one could write a message such as:
\begin{center}
|This is the part \childdocname{} of \childdocjob{}.|
\end{center}

%%%%%%%%%%%%%%%%%%%%%%%%%%%%%%%%%%%%%%%%%%%%%%%%%%%%%%%%%%%%%%%%%%%%%%%%%%%%%%%%
\subsection{Flags}
\label{sec:flags}

The package makes it easy to generate different versions
of the main or child documents.
To this end compilation flags can be defined
and assigned different default values.
They will be particularly useful in conjunction
with the forwarding mechanism described in \secref{sec:forward}.

For example, it may be useful to have a flag |\version|
which can be set to |draft| or |final|.
The document source will contain some conditional code
depending on the value of |\version|.
Suppose further, the flag should default to |final| for the main file
and to |draft| for child files
which is a natural assignment for editing the document.
This is achieved by placing the following code
in the preamble of the main document
(below the |\childdocmain| directive):
%
\begin{center}
\begin{tabular}{l}
|\ifchilddoc|\\
|\providecommand{\version}{draft}|\\
|\||else|\\
|\providecommand{\version}{final}|\\
|\||fi|
\end{tabular}
\end{center}
%
The definition by |\providecommand| makes sure
that previous definitions are not overwritten.
Further statements |\providecommand{\version}{...}|
can thus be added before the above code to override it.

For the main file, one might add a line
(between |\childdocmain| and the above block)
%
\begin{center}
|%\ifchilddoc\||else\providecommand{\version}{draft}\||fi|
\end{center}
%
which can be uncommented to produce a draft version.
Likewise one can add a line to the very top of a child file
(above the |\childdocof{|\textit{main}|}| directive)
%
\begin{center}
|%\providecommand{\version}{final}|
\end{center}
%
which can be uncommented to produce the final version of this child document.

%%%%%%%%%%%%%%%%%%%%%%%%%%%%%%%%%%%%%%%%%%%%%%%%%%%%%%%%%%%%%%%%%%%%%%%%%%%%%%%%
\subsection{Forwarding}
\label{sec:forward}

Different versions of the main or child documents
using compilation flags as described in \secref{sec:flags}
can be (permanently) stored in different files
for convenient compilation, viewing and distribution.
To this end, the package defines a command
to pass on compilation to a different file:

%%%%%%%%%%%%%%%%%%%%%%%%%%%%%%%%%%%%%%%%
\DescribeMacro{\childdocforward}
The command |\childdocforward| redirects processing to
another source file:
%
\begin{center}
\begin{tabular}{l}
|\input{childdoc.def}|\\
|\childdocforward[|\textit{main}|]{|\textit{dest}|}|\\
\end{tabular}
\end{center}
%
The argument \textit{dest} is the destination file
(without extension).
It should be the main file or one of the child files.
Note that further \textsf{childdoc} directives
such as |\childdocof| and |\childdocforward|
in the indicated file will be processed in this form.
The optional argument \textit{main}
passes on directly to the main file \textit{main}
while pretending to compile the child \textit{dest}.
This form behaves as if \textit{dest}
issues |\childdocof{|\textit{main}|}| right away,
and no further \textsf{childdoc} directives will be processed.

%%%%%%%%%%%%%%%%%%%%%%%%%%%%%%%%%%%%%%%%
\DescribeMacro{\...prefix}
In the alternative form |\childdocforwardprefix|,
%
\begin{center}
\begin{tabular}{l}
|\input{childdoc.def}|\\
|\childdocforwardprefix[|\textit{main}|]{|\textit{prefix}|}{|\textit{dest}|}|
\end{tabular}
\end{center}
%
the destination file is determined by a pattern
depending on the current file:
To make this work, the current file must be called
`{\textit{prefix}\hspace{0.2em}\textit{suffix}}'
with \textit{prefix} matching precisely the argument.
Processing is then passed on to the file
`{\textit{dest}\hspace{0.2em}\textit{suffix}}'.
Surely, the same effect is achieved by
directly specifying the
argument `{\textit{dest}\hspace{0.2em}\textit{suffix}}'
in the first form.
However, that requires to set up a different file
for each child. With the alternative form of the command
all these files can have exactly the same content
which simplifies setting them up and maintaining them.

For example, the following file |draft.tex|
with a compilation flag |\version| as described in \secref{sec:flags}
compiles the main document as a draft:
%
\begin{center}
\begin{tabular}{l}
|\def\version{draft}|\\
|\input{childdoc.def}|\\
|\childdocforward{|\textit{main}|}|
\end{tabular}
\end{center}
%
Likewise, the following files |final|\textit{nn}|.tex|
compile the final version of the child document
|child|\textit{nn}|.tex|:
%
\begin{center}
\begin{tabular}{l}
|\def\version{final}|\\
|\input{childdoc.def}|\\
|\childdocforwardprefix{final}{child}|
\end{tabular}
\end{center}
%

Note that when several versions of a main file and/or of each child file
are to be generated, it may be convenient to set up a |Makefile| or
shell script to automatise the process.

%%%%%%%%%%%%%%%%%%%%%%%%%%%%%%%%%%%%%%%%%%%%%%%%%%%%%%%%%%%%%%%%%%%%%%%%%%%%%%%%
\subsection{Command Line Processing}
\label{sec:commandline}

The effect of redirection files can also be achieved by invoking
the \LaTeX{} compiler with a more elaborate command line.
Most conveniently this should be done as part
of a shell script or a |Makefile|.

When using \textsf{childdoc} in the main file, the following
command lines effectively perform a redirection
(note that depending on the shell being used,
backslashes may have to be doubled: `|\|' $\to$ `|\\|'):
%
\begin{center}
|... -jobname "|\textit{target}|" |\\|"|[\textit{flags}]%
|\input{childdoc.def}\childdocforward[|\textit{main}|]{|\textit{dest}|}"|
\end{center}
%
Here \textit{target} is the name of the output file,
\textit{main} is the name of the main file
and \textit{dest} is the name of the main or child file to be processed
(all filenames without extensions).
The optional argument \textit{main} can be omitted
if \textit{main} matches \textit{dest}.
Optionally, compilation \textit{flags} can be defined via |\def| commands.
This command line makes the \TeX{} engine believe
it is compiling the file \textit{target}
whose content is specified as the latter parameter.
The provided code then forwards the processing to
\textit{main} or \textit{dest} as described in \secref{sec:forward}.

%%%%%%%%%%%%%%%%%%%%%%%%%%%%%%%%%%%%%%%%%%%%%%%%%%%%%%%%%%%%%%%%%%%%%%%%%%%%%%%%
\subsection{Include by Input}
\label{sec:input}

Including child documents by |\include| has some restrictions by design.
Most notably, the content of a child document always occupies
its own set of pages; pages cannot be shared between child documents.
Usually, this behaviour makes perfect sense
because each child document contain an essential part of the document.
However, in some situations it may be desirable to compose
a document from a collection of parts
without having mandatory page breaks between then.
For this case, the package
provides a mechanism to include parts
by |\input| which can also be processed individually.
However, by construction this mechanism
requires manual handling of the content to be output.

%%%%%%%%%%%%%%%%%%%%%%%%%%%%%%%%%%%%%%%%
\DescribeMacro{\ifchilddocmanual}
The main file should be prepared as usual, see \secref{sec:include}.
However, the document body must make a distinction
between processing of an individual part and of the main document, e.g.:
%
\begin{center}
\begin{tabular}{l}
|\ifchilddocmanual|\\
|\input{\childdocname}|\\
|\||else|\\
\textit{document body with }|\input{|\textit{part}|}|\\
|\||fi|
\end{tabular}
\end{center}
%
The conditional |\ifchilddocmanual| is true whenever
a part to be included by |\input| is being compiled,
and the name of the part is stored in |\childdocname|.

%%%%%%%%%%%%%%%%%%%%%%%%%%%%%%%%%%%%%%%%
\DescribeMacro{\childdocby}
Each part to be included by |\input| should start with:
%
\begin{center}
\begin{tabular}{l}
|\input{childdoc.def}|\\
|\childdocby{|\textit{main}|}|\\
\end{tabular}
\end{center}
%
The directive |\childdocby| is similar to |\childdocof|
described in \secref{sec:include},
but the subsequent selection of content must be done manually.
To that end, both |\ifchilddoc| and |\ifchilddocmanual|
will be true upon processing of a part,
and the name of the part is stored in |\childdocname|.
Note that |\jobname| will be set to the filename of the current part
so that each part receives an individual |.aux| file
that does not interfere with the |.aux| file(s) of the main document.
This behaviour can be altered by the alternative form
|\childdocby[*]{|\textit{main}|}| (with a non-empty optional argument)
which uses the |.aux| file of the main document
by setting |\jobname| to \textit{main}.

%%%%%%%%%%%%%%%%%%%%%%%%%%%%%%%%%%%%%%%%%%%%%%%%%%%%%%%%%%%%%%%%%%%%%%%%%%%%%%%%
\subsection{Driver Development}
\label{sec:driver}

The \textsf{childdoc} mechanism can also be use for the development
of definition files such as \LaTeX{} styles or classes.
This case differs from the above setup with multiple parts
included by |\include| in that no |\includeonly| should be invoked.
This can be achieved by starting the include file
(before |\ProvidesPackage|) with:
%
\begin{center}
\begin{tabular}{l}
|\input{childdoc.def}|\\
|\childdocforward{|\textit{main}|}|\\
\end{tabular}
\end{center}
%
or alternatively with:
%
\begin{center}
\begin{tabular}{l}
|\input{childdoc.def}|\\
|\childdocby{|\textit{main}|}|\\
\end{tabular}
\end{center}
%
Both forms have slightly different effects as described above.
The main file is prepared as usual, see \secref{sec:include}.

%%%%%%%%%%%%%%%%%%%%%%%%%%%%%%%%%%%%%%%%%%%%%%%%%%%%%%%%%%%%%%%%%%%%%%%%%%%%%%%%
\subsection{Legacy Detection}
\label{sec:detection}

The directive |\childdocmain| in the main file can detect
whether the complete document or merely a child is to be compiled
even without using the directive |\childdocof|.
This method is deprecated because it is less robust
and there is no compelling reason to use it;
it is merely provided for backward compatibility
and it may be removed in future versions.

If the detection mechanism is to be used,
it is mandatory to correctly specify
the filename of the main file as the argument of |\childdocmain|:
%
\begin{center}
\begin{tabular}{l}
|\input{childdoc.def}|\\
|\childdocmain{|\textit{main}|}|\\
\end{tabular}
\end{center}
%
If |\jobname| does not match the argument \textit{main} of |\childdocmain|,
it is assumed that |\jobname| points to the child file to be compiled.
When using |\childdocmain| with the main file specified as argument,
it suffices to start a child file
with just |\input{|\textit{main}|}|
without loading of the package and using |\childdocof|.
If instead all processing is done
with the appropriate \textsf{childdoc} directives,
the argument of \textit{main} of |\childdocmain| can be empty.

An alternative version of the command line processing described
in \secref{sec:commandline} using the detection mechanism reads:
%
\begin{center}
|... -jobname "|\textit{target}|" "|[\textit{flags}]%
[|\def\jobname{|\textit{dest}|}|]|\input{|\textit{main}|}"|
\end{center}

%%%%%%%%%%%%%%%%%%%%%%%%%%%%%%%%%%%%%%%%%%%%%%%%%%%%%%%%%%%%%%%%%%%%%%%%%%%%%%%%
\subsection{Manual Code}
\label{sec:manual}

In case one cannot be certain whether the definitions file |childdoc.def|
is installed on the target \TeX{} distribution
and one prefers not to ship it,
it is conceivable to paste a few relevant commands into the sources.

To that end, drop all statements |\input{childdoc.def}|
and perform the replacements as outlined below.
Instead of |\childdocmain{|\textit{main}|}| add the following code
to the top of the main file:
%
\begin{center}
\begin{tabular}{l}
|\||ifdefined\childdocname\endinput\||fi\newif\ifchilddoc|\\
|\edef\childdocname{\scantokens\expandafter{\jobname\noexpand}}|\\
|\def\childdocmain{|\textit{main}|}\||ifx\childdocmain\childdocname\||else|\\
|\childdoctrue\includeonly{\childdocname}\let\jobname\childdocmain\||fi|\\
\end{tabular}
\end{center}
%
Instead of |\childdocof{|\textit{main}|}| just include the main file
at the top of each child file:
%
\begin{center}
|\input{|\textit{main}|}|
\end{center}
%
A simple redirection |\childdocforward{|\textit{dest}|}| is achieved by:
%
\begin{center}
|\def\jobname{|\textit{dest}|}\input{\jobname}|
\end{center}
%
The redirection with prefix
|\childdocforwardprefix[|\textit{prefix}|]{|\textit{dest}|}|
is accomplished by:
%
\begin{center}
\begin{tabular}{l}
|{\edef\jobname{\scantokens\expandafter{\jobname\noexpand}}|\\
|\def\redirectjob |\textit{prefix}|#1~~~{\gdef\jobname{|\textit{dest}|#1}}|\\
|\expandafter\redirectjob\jobname~~~}\input{\jobname}|
\end{tabular}
\end{center}

In an alternative approach,
child documents can be compiled by a specific command line
without additional code or specific definitions:
%
\begin{center}
|... -jobname "|\textit{target}|" "|[\textit{flags}]%
|\includeonly{|\textit{dest}|}\input{|\textit{main}|}"|
\end{center}
%

%%%%%%%%%%%%%%%%%%%%%%%%%%%%%%%%%%%%%%%%%%%%%%%%%%%%%%%%%%%%%%%%%%%%%%%%%%%%%%%%
%%%%%%%%%%%%%%%%%%%%%%%%%%%%%%%%%%%%%%%%%%%%%%%%%%%%%%%%%%%%%%%%%%%%%%%%%%%%%%%%
\section{Information}

%%%%%%%%%%%%%%%%%%%%%%%%%%%%%%%%%%%%%%%%%%%%%%%%%%%%%%%%%%%%%%%%%%%%%%%%%%%%%%%%
\subsection{Copyright}

Copyright \copyright{} 2017--2018 Niklas Beisert

This work may be distributed and/or modified under the
conditions of the \LaTeX{} Project Public License, either version 1.3
of this license or (at your option) any later version.
The latest version of this license is in
  \url{http://www.latex-project.org/lppl.txt}
and version 1.3 or later is part of all distributions of \LaTeX{}
version 2005/12/01 or later.

This work has the LPPL maintenance status `maintained'.

The Current Maintainer of this work is Niklas Beisert.

This work consists of the files |README.txt|, |childdoc.ins| and |childdoc.dtx|
as well as the derived files |childdoc.def|, |cdocsamp.tex|
with |cdocsch1.tex|, |cdocsch2.tex|, |cdocspt3.tex|, |cdocspt4.tex|,
|cdocsdrf.tex|, |cdocsfn1.tex|, |cdocsfn2.tex|
as well as |childdoc.pdf|.

%%%%%%%%%%%%%%%%%%%%%%%%%%%%%%%%%%%%%%%%%%%%%%%%%%%%%%%%%%%%%%%%%%%%%%%%%%%%%%%%
\subsection{Files and Installation}

The package consists of the files:
%
\begin{center}
\begin{tabular}{ll}
    |README.txt|   & readme file \\
    |childdoc.ins| & installation file \\
    |childdoc.dtx| & source file \\
    |childdoc.def| & definition file \\
    |cdocsamp.tex| & sample main file \\
    |cdocsch1.tex| & sample include file \\
    |cdocsch2.tex| & sample include file \\
    |cdocspt3.tex| & sample part file \\
    |cdocspt4.tex| & sample part file \\
    |cdocsdrf.tex| & sample redirection file \\
    |cdocsfn1.tex| & sample redirection file \\
    |cdocsfn2.tex| & sample redirection file \\
    |childdoc.pdf| & manual
\end{tabular}
\end{center}
%
The distribution consists of the files
|README.txt|, |childdoc.ins| and |childdoc.dtx|.
%
\begin{itemize}
\item
Run (pdf)\LaTeX{} on |childdoc.dtx|
to compile the manual |childdoc.pdf| (this file).
\item
Run \LaTeX{} on |childdoc.ins| to create the definitions file |childdoc.def|
and the sample |cdocsamp.tex| with include files
|cdocsch1.tex|, |cdocsch2.tex|, |cdocspt3.tex|, |cdocspt4.tex|,
|cdocsdrf.tex|, |cdocsfn1.tex|, |cdocsfn2.tex|.
Then copy the file |childdoc.def| to an appropriate directory of your \LaTeX{}
distribution, e.g.\ \textit{texmf-root}|/tex/latex/childdoc|.
\end{itemize}

%%%%%%%%%%%%%%%%%%%%%%%%%%%%%%%%%%%%%%%%%%%%%%%%%%%%%%%%%%%%%%%%%%%%%%%%%%%%%%%%
\subsection{Related CTAN Packages}

There are several other packages which offer a similar functionality:
%
\begin{itemize}
\item
The packages
\href{http://ctan.org/pkg/docmute}{\textsf{docmute}},
\href{http://ctan.org/pkg/includex}{\textsf{includex}} and
\href{http://ctan.org/pkg/standalone}{\textsf{standalone}}
provide commands to include only the document body of
a child file thus allowing both files to be compiled individually.
\item
The packages \href{http://ctan.org/pkg/subdocs}{\textsf{subdocs}}
and \href{http://ctan.org/pkg/subfiles}{\textsf{subfiles}}
provide structures in which the main and child documents can be
encapsulated and allowing them to be compiled individually.
The inclusion mechanism is different from the conventional |\include|.
\item
The package \href{http://ctan.org/pkg/combine}{\textsf{combine}}
is an elaborate solution to combine several documents into one.
\end{itemize}
%
See also the CTAN topic \href{http://ctan.org/topic/subdocs}{\textsf{subdocs}}
for further related packages.
The present package differs from the above solutions in that
a document structure constructed with the conventional |\include| mechanism
just needs two extra commands at the top of every file
such that all constituent files can be compiled individually.

%%%%%%%%%%%%%%%%%%%%%%%%%%%%%%%%%%%%%%%%%%%%%%%%%%%%%%%%%%%%%%%%%%%%%%%%%%%%%%%%
%\subsection{Feature Suggestions}
%
%The following is a list of features which may be useful for future
%versions of this package:
%%
%\begin{itemize}
%\item
%\ldots
%\end{itemize}

%%%%%%%%%%%%%%%%%%%%%%%%%%%%%%%%%%%%%%%%%%%%%%%%%%%%%%%%%%%%%%%%%%%%%%%%%%%%%%%%
\subsection{Revision History}

%%%%%%%%%%%%%%%%%%%%%%%%%%%%%%%%%%%%%%%%
\paragraph{v2.0:} 2018/12/30

\begin{itemize}
\item
immediate forward processing
\item
added |\childdocby| mechanism
\item
manual restructured
\end{itemize}

%%%%%%%%%%%%%%%%%%%%%%%%%%%%%%%%%%%%%%%%
\paragraph{v1.6:} 2018/01/17

\begin{itemize}
\item
application for development of include files
\item
corrections to manual
\end{itemize}

%%%%%%%%%%%%%%%%%%%%%%%%%%%%%%%%%%%%%%%%
\paragraph{v1.5:} 2017/05/21

\begin{itemize}
\item
more complete structuring introduced
\item
|\childdocof| introduced
\item
|\childdoc| renamed to |\childdocmain|
\item
|\childredirect| renamed to |\childdocforward| and |\childdocforwardprefix|
and functionality expanded
\end{itemize}

%%%%%%%%%%%%%%%%%%%%%%%%%%%%%%%%%%%%%%%%
\paragraph{v1.0:} 2017/04/27

\begin{itemize}
\item
manual and install package
\item
first version published on CTAN
\end{itemize}

%%%%%%%%%%%%%%%%%%%%%%%%%%%%%%%%%%%%%%%%
\paragraph{v0.6:} 2017/04/26

\begin{itemize}
\item
redirection mechanism added
\end{itemize}

%%%%%%%%%%%%%%%%%%%%%%%%%%%%%%%%%%%%%%%%
\paragraph{v0.5:} 2017/04/26

\begin{itemize}
\item
functionality in definition file
\end{itemize}


%%%%%%%%%%%%%%%%%%%%%%%%%%%%%%%%%%%%%%%%%%%%%%%%%%%%%%%%%%%%%%%%%%%%%%%%%%%%%%%%
%%%%%%%%%%%%%%%%%%%%%%%%%%%%%%%%%%%%%%%%%%%%%%%%%%%%%%%%%%%%%%%%%%%%%%%%%%%%%%%%
%%%%%%%%%%%%%%%%%%%%%%%%%%%%%%%%%%%%%%%%%%%%%%%%%%%%%%%%%%%%%%%%%%%%%%%%%%%%%%%%
\appendix

\settowidth\MacroIndent{\rmfamily\scriptsize 000\ }

 \DocInput{childdoc.dtx}

\end{document}
%</driver>
% \fi
%
% %%%%%%%%%%%%%%%%%%%%%%%%%%%%%%%%%%%%%%%%%%%%%%%%%%%%%%%%%%%%%%%%%%%%%%%%%%%%%%
% %%%%%%%%%%%%%%%%%%%%%%%%%%%%%%%%%%%%%%%%%%%%%%%%%%%%%%%%%%%%%%%%%%%%%%%%%%%%%%
% \section{Sample}
%\iffalse
%<*samplemain>
%\fi
%
% The following presents a sample document
% with two chapters, two parts, a title page,
% a compile flag as well as three forwarding files to set the flag.
% It consists of eight |.tex| files:
% \begin{center}
% \begin{tabular}{ll}
% |cdocsamp.tex|&main file\\
% |cdocsch1.tex|&include file for chapter 1\\
% |cdocsch2.tex|&include file for chapter 2\\
% |cdocspt3.tex|&include file for part 3\\
% |cdocspt4.tex|&include file for part 4\\
% |cdocsdrf.tex|&forwarding file for main file in draft mode\\
% |cdocsfi1.tex|&forwarding file for final version of chapter 1\\
% |cdocsfi2.tex|&forwarding file for final version of chapter 2\\
% \end{tabular}
% \end{center}
% Each of the eight files can be compiled directly by the \LaTeX{} compiler.
%
% %%%%%%%%%%%%%%%%%%%%%%%%%%%%%%%%%%%%%%
% \paragraph{Main File.}
%
% The main file is called |cdocsamp.tex|.
%
% Load the \textsf{childdoc} definitions and
% declare the filename for the main document:
%    \begin{macrocode}
\input{childdoc.def}
\childdocmain{}
%    \end{macrocode}

% Optional override for |\version| flag:
%    \begin{macrocode}
%%\ifchilddoc\else\providecommand{\version}{draft}\fi
%    \end{macrocode}

% Define the default values for the |\version| flag
% (|final| for the main file and |draft| for childs):
%    \begin{macrocode}
\ifchilddoc
\providecommand{\version}{draft}
\else
\providecommand{\version}{final}
\fi
%    \end{macrocode}

% Load the standard document class:
%    \begin{macrocode}
\documentclass[12pt]{article}
%    \end{macrocode}

% Start the document body:
%    \begin{macrocode}
\begin{document}
%    \end{macrocode}

% Declare a title page.
% Print title, part of document being processed and version flag:
%    \begin{macrocode}
\addtocounter{page}{-1}
\begin{center}
{\LARGE\bfseries{}childdoc example\par}
\vspace{1cm}
\ifchilddoc
\ifchilddocmanual part\else chapter\fi:
`\childdocname' of `\childdocjob'\par
\else
main document: `\childdocjob'\par
\fi
version: \version\par
\end{center}
\newpage
%    \end{macrocode}

% Manually include selected file,
% otherwise process as usual:
%    \begin{macrocode}
\ifchilddocmanual
\section*{part `\childdocname'}
\input{\childdocname}
\else
%    \end{macrocode}

% Include the two chapters:
%    \begin{macrocode}
\include{cdocsch1}
\include{cdocsch2}
%    \end{macrocode}

% Include the two parts unless only chapters should be displayed:
%    \begin{macrocode}
\ifchilddoc\else
\section{part three}
\input{cdocspt3}
\section{part four}
\input{cdocspt4}
\fi
%    \end{macrocode}

% Process as usual until here:
%    \begin{macrocode}
\fi
%    \end{macrocode}

% End of document body:
%    \begin{macrocode}
\end{document}
%    \end{macrocode}
%\iffalse
%</samplemain>
%\fi
%
% %%%%%%%%%%%%%%%%%%%%%%%%%%%%%%%%%%%%%%
% \paragraph{Chapter Include Files.}
%
% The include files are called |cdocsch1.tex| and |cdocsch2.tex|.
%
%\iffalse
%<*samplechap1|samplechap2>
%\fi

% Optional override for |\version| flag:
%    \begin{macrocode}
%%\providecommand{\version}{final}
%    \end{macrocode}

% Include the main document:
%    \begin{macrocode}
\input{childdoc.def}
\childdocof{cdocsamp}
%    \end{macrocode}

%\iffalse
%</samplechap1|samplechap2>
%\fi
%
%\iffalse
%<*samplechap1>
%\fi
% Some text for chapter 1:
%    \begin{macrocode}
\section{one}
some text in chapter one
%    \end{macrocode}

%\iffalse
%</samplechap1>
%\fi
% Some text for chapter 2:
%\iffalse
%<*samplechap2>
%\fi
%    \begin{macrocode}
\section{two}
more text in chapter two
%    \end{macrocode}

%\iffalse
%</samplechap2>
%\fi
%
% %%%%%%%%%%%%%%%%%%%%%%%%%%%%%%%%%%%%%%
% \paragraph{Part Include Files.}
%
% The include files are called |cdocspt3.tex| and |cdocspt4.tex|.
%
%\iffalse
%<*samplepart3|samplepart4>
%\fi

% Optional override for |\version| flag:
%    \begin{macrocode}
%%\providecommand{\version}{final}
%    \end{macrocode}

% Include the main document:
%    \begin{macrocode}
\input{childdoc.def}
\childdocby{cdocsamp}
%    \end{macrocode}

%\iffalse
%</samplepart3|samplepart4>
%\fi
%
%\iffalse
%<*samplepart3>
%\fi
% Some text for part 3:
%    \begin{macrocode}
some text in part three
%    \end{macrocode}

%\iffalse
%</samplepart3>
%\fi
% Some text for part 4:
%\iffalse
%<*samplepart4>
%\fi
%    \begin{macrocode}
more text in part four
%    \end{macrocode}

%\iffalse
%</samplepart4>
%\fi
%
% %%%%%%%%%%%%%%%%%%%%%%%%%%%%%%%%%%%%%%
% \paragraph{Forwarding for a Complete Draft.}
%
% The following forwarding file |cdocsdrf.tex|
% compiles the main document in draft mode:
%\iffalse
%<*sampledraft>
%\fi
%    \begin{macrocode}
\def\version{draft}
\input{childdoc.def}
\childdocforward{cdocsamp}
%    \end{macrocode}

%\iffalse
%</sampledraft>
%\fi
%
% %%%%%%%%%%%%%%%%%%%%%%%%%%%%%%%%%%%%%%
% \paragraph{Forwarding for Final Version of the Chapters.}
%
% The following forwarding files |cdocsfn1.tex| and |cdocsfn2.tex|
% (with identical content)
% compile the final versions of the child documents
% |cdocsch1.tex| and |cdocsch2.tex|, respectively:
%\iffalse
%<*samplefinal>
%\fi
%    \begin{macrocode}
\def\version{final}
\input{childdoc.def}
\childdocforwardprefix[cdocsamp]{cdocsfn}{cdocsch}
%    \end{macrocode}

%\iffalse
%</samplefinal>
%\fi
%
% %%%%%%%%%%%%%%%%%%%%%%%%%%%%%%%%%%%%%%
% \paragraph{Command Line Processing.}
%
% The following three command lines generate the output files
% |cdocscld|, |cdocscl1| and |cdocscl2|
% which should be identical to
% |cdocsdrf|, |cdocsch1| and |cdocsfn2|, respectively:
% \begin{center}
% \begin{tabular}{l}
% |latex -jobname cdocscld \|\\
% |  "\def\version{draft}\input{childdoc.def}\childdocforward{cdocsamp}"|\\
% |latex -jobname cdocscl1 \|\\
% |  "\input{childdoc.def}\childdocforward[cdocsamp]{cdocsch1}"|\\
% |latex -jobname cdocscl2 \|\\
% |  "\def\version{final}\input{childdoc.def}\childdocforward{cdocsch2}"|
% \end{tabular}
% \end{center}
% Note that the trailing backslash on each first line
% merely continues the input to the second line
% (for convenient cut ant paste).
% Furthermore, the command |latex| can be replaced by any
% of its alternative versions such as |pdflatex|.
%
% %%%%%%%%%%%%%%%%%%%%%%%%%%%%%%%%%%%%%%%%%%%%%%%%%%%%%%%%%%%%%%%%%%%%%%%%%%%%%%
% %%%%%%%%%%%%%%%%%%%%%%%%%%%%%%%%%%%%%%%%%%%%%%%%%%%%%%%%%%%%%%%%%%%%%%%%%%%%%%
% \section{Implementation}
%\iffalse
%<*package>
%\fi
%
% This section describes the definitions file |childdoc.def|.

% The definitions cannot be loaded using |\usepackage| or |\RequirePackage|
% which has a mechanism to prevent loading a style file more than once.
% When loading the definitions by means of |\input|
% multiple instances have to be prevented manually:
%\iffalse
%This code needs to be before the `\ProvidesFile' directive
%which is defined at the beginning of this file.
%Therefore it is also placed there and commented out here.
%</package>
%<*discard>
%\fi
%    \begin{macrocode}
\ifdefined\childdocmain\endinput\fi
%    \end{macrocode}
%\iffalse
%</discard>
%<*package>
%\fi
%
% \macro{\ifchilddoc}
% \macro{\ifchilddocmanual}
% The conditional |\ifchilddoc| tells whether a
% child (true) or main (false) document is being compiled.
% The conditional |\ifchilddocmanual| tells whether
% the |\includeonly| mechanism is used (false) or
% the selection of child files must be performed manually (true).
% The definitions initialise to false:
%    \begin{macrocode}
\newif\ifchilddoc
\newif\ifchilddocmanual
%    \end{macrocode}

% \macro{\childdocname}
% \macro{\childdocjob}
% The macro |\childdocname| stores the name of the main document
% to be compiled. The macro |\childdocjob| stores the name of
% the document on which the \LaTeX{} compiler was originally invoked.
% The content of |\jobname| cannot be compared
% to filenames specified in the source due to different catcodes.
% The following code rescans |\jobname|, stores the result
% in |\childdocname| and saves a copy in |\childdocjob|:
%    \begin{macrocode}
\edef\childdocname{\scantokens\expandafter{\jobname\noexpand}}
\let\childdocjob\childdocname
%    \end{macrocode}

% \macro{\childdocdisable}
% The macro |\childdocdisable| prevents the main file
% from being processed more than once.
% At this stage, the main document command |\childdocmain|
% is assumed to be called once again where it should do nothing.
% Any subsequent call to it should prevent
% a secondary processing of the main document
% It overwrites the forwarding commands
% |\childdocof| and |\childdocforward|
% with empty macros to prevent further inclusions of the main document:
%    \begin{macrocode}
\newcommand{\childdocdisable}
{
  \renewcommand{\childdocmain}[1]{\renewcommand{\childdocmain}[1]{\endinput}}
  \renewcommand{\childdocof}[1]{}
  \renewcommand{\childdocby}[2][]{}
  \renewcommand{\childdocforward}[2][]{}
  \renewcommand{\childdocdisable}{}
}
%    \end{macrocode}

% \macro{\childdocmain}
% The macro |\childdocmain| is to be called at the top of the main file
% with nothing or the main filename (without extension) as argument.
% First, it breaks loops.
% If the argument is not empty and does not match |\childdocname|
% (which is set by the first inclusion of |childdoc.def|),
% |\ifchilddoc| is set to true, |\includeonly| is applied to the child file
% and |\jobname| is set to the main file
% (for proper handling of |.aux| files):
%    \begin{macrocode}
\newcommand{\childdocmain}[1]
{
  \childdocdisable\childdocmain{}
  \if?#1?\else
    \begingroup
      \def\childdoctmp{#1}
      \ifx\childdoctmp\childdocname
        \def\childdoctmp{}
      \else
        \def\childdoctmp
        {
          \childdoctrue
          \includeonly{\childdocname}
          \def\childdocjob{#1}
          \def\jobname{#1}
        }
      \fi
      \expandafter
    \endgroup
    \childdoctmp
  \fi
}
%    \end{macrocode}

% \macro{\childdocof}
% The command |\childdocof| redirects
% compilation to the main file |#1|.
%    \begin{macrocode}
\newcommand{\childdocof}[1]
{
  \childdocdisable
  \childdoctrue
  \includeonly{\childdocname}
  \def\jobname{#1}
  \def\childdocjob{#1}
  \input{#1}
}
%    \end{macrocode}

% \macro{\childdocby}
% The command |\childdocby| ....
%    \begin{macrocode}
\newcommand{\childdocby}[2][]
{
  \childdocdisable
  \childdoctrue
  \childdocmanualtrue
  \if?#1?\else
    \def\jobname{#2}
  \fi
  \def\childdocjob{#2}
  \input{#2}
  \endinput
}
%    \end{macrocode}

% \macro{\childdocforward}
% The command |\childdocforward| redirects
% compilation to the main file or
% (if the optional argument is given) a child file.
% Parameters are set as if the main file
% or a child file starting with |\childdocof| was compiled.
% Then compilation is handed over to the main file:
%    \begin{macrocode}
\newcommand{\childdocforward}[2][]
{
  \begingroup
    \if?#1?
      \def\childdoctmp
      {
        \def\childdocname{#2}
        \def\childdocjob{#2}
        \def\jobname{#2}
        \input{#2}
        \endinput
      }
    \else
      \def\childdoctmp
      {
        \childdocdisable
        \def\childdocname{#2}
        \childdoctrue
        \includeonly{#2}
        \def\childdocjob{#1}
        \def\jobname{#1}
        \input{#1}
        \endinput
      }
    \fi
    \expandafter
  \endgroup
  \childdoctmp
}
%    \end{macrocode}

% \macro{\childdocforwardprefix}
% The command |\childdocforwardprefix| redirects
% compilation to the main or a child file by means of a pattern.
% The prefix |#1| in the current filename is replaced by |#2|
% and the suffix of the current filename is kept
% (it is assumed that the filename does not contain the substring `|~~~|'
% which is used as a delimiter).
% Compilation is handed over to the new file by |\childdocforward|:
%    \begin{macrocode}
\newcommand{\childdocforwardprefix}[3][]
{
  \begingroup
    \def\childdocextract #2##1~~~{\def\childdoctmp{\childdocforward[#1]{#3##1}}}
    \expandafter\childdocextract\childdocname~~~
    \expandafter
  \endgroup
  \childdoctmp
}
%    \end{macrocode}

% \macro{\childdoc}
% The deprecated macro |\childdoc| is a legacy version of |\childdocmain|:
%    \begin{macrocode}
\newcommand{\childdoc}{\childdocmain}
%    \end{macrocode}

% \macro{\childdocredirect}
% The deprecated macro |\childdocredirect| is a legacy version
% of |\childdocforward| and |\childdocforwardprefix|:
%    \begin{macrocode}
\newcommand{\childdocredirect}[2][]
{
  \begingroup
    \if?#1?
      \def\childdoctmp{\childdocforward{#2}}
    \else
      \def\childdoctmp{\childdocforwardprefix{#1}{#2}}
    \fi
    \expandafter
  \endgroup
  \childdoctmp
}
%    \end{macrocode}

%\iffalse
%</package>
%\fi
%
\endinput

\childdocforward{cdocsamp}
%    \end{macrocode}

%\iffalse
%</sampledraft>
%\fi
%
% %%%%%%%%%%%%%%%%%%%%%%%%%%%%%%%%%%%%%%
% \paragraph{Forwarding for Final Version of the Chapters.}
%
% The following forwarding files |cdocsfn1.tex| and |cdocsfn2.tex|
% (with identical content)
% compile the final versions of the child documents
% |cdocsch1.tex| and |cdocsch2.tex|, respectively:
%\iffalse
%<*samplefinal>
%\fi
%    \begin{macrocode}
\def\version{final}
% \iffalse
%
% childdoc.dtx Copyright (C) 2017-2018 Niklas Beisert
%
% This work may be distributed and/or modified under the
% conditions of the LaTeX Project Public License, either version 1.3
% of this license or (at your option) any later version.
% The latest version of this license is in
%   http://www.latex-project.org/lppl.txt
% and version 1.3 or later is part of all distributions of LaTeX
% version 2005/12/01 or later.
%
% This work has the LPPL maintenance status `maintained'.
%
% The Current Maintainer of this work is Niklas Beisert.
%
% This work consists of the files childdoc.dtx and childdoc.ins
% and the derived files childdoc.def and cdocsamp.tex with
% cdocsch1.tex, cdocsch2.tex, cdocsdrf.tex, cdocsfn1.tex, cdocsfn2.tex.
%
%<package>\ifdefined\childdocmain\endinput\fi
%<package>\ProvidesFile{childdoc.def}[2018/12/30 v2.0 child document driver]
%<samplemain>\ProvidesFile{cdocsamp.tex}[2018/12/30 v2.0 sample for childdoc]
%<*driver>
%\ProvidesFile{childdoc.drv}[2018/12/30 v2.0 childdoc reference manual file]
\PassOptionsToClass{10pt,a4paper}{article}
\documentclass{ltxdoc}

\usepackage[margin=35mm]{geometry}
\usepackage{hyperref}
\usepackage{hyperxmp}
\usepackage[usenames]{color}

\hypersetup{colorlinks=true}
\hypersetup{pdfstartview=FitH}
\hypersetup{pdfpagemode=UseNone}
\hypersetup{pdfsource={}}
\hypersetup{pdflang={en-UK}}
\hypersetup{pdfcopyright={Copyright 2017-2018 Niklas Beisert.
  This work may be distributed and/or modified under the
  conditions of the LaTeX Project Public License, either version 1.3
  of this license or (at your option) any later version.}}
\hypersetup{pdflicenseurl={http://www.latex-project.org/lppl.txt}}
\hypersetup{pdfcontactaddress={ETH Zurich, ITP, HIT K,
  Wolfgang-Pauli-Strasse 27}}
\hypersetup{pdfcontactpostcode={8093}}
\hypersetup{pdfcontactcity={Zurich}}
\hypersetup{pdfcontactcountry={Switzerland}}
\hypersetup{pdfcontactemail={nbeisert@itp.phys.ethz.ch}}
\hypersetup{pdfcontacturl={http://people.phys.ethz.ch/\xmptilde nbeisert/}}

\newcommand{\secref}[1]{\hyperref[#1]{section \ref*{#1}}}

\parskip1ex
\parindent0pt
\let\olditemize\itemize
\def\itemize{\olditemize\parskip0pt}

\begin{document}

\title{The \textsf{childdoc} Package}
\hypersetup{pdftitle={The childdoc Package}}
\author{Niklas Beisert\\[2ex]
  Institut f\"ur Theoretische Physik\\
  Eidgen\"ossische Technische Hochschule Z\"urich\\
  Wolfgang-Pauli-Strasse 27, 8093 Z\"urich, Switzerland\\[1ex]
  \href{mailto:nbeisert@itp.phys.ethz.ch}
  {\texttt{nbeisert@itp.phys.ethz.ch}}}
\hypersetup{pdfauthor={Niklas Beisert}}
\hypersetup{pdfsubject={Manual for the LaTeX2e Package childdoc}}
\date{30 December 2018, \textsf{v2.0}}
\maketitle

\begin{abstract}\noindent
\textsf{childdoc} is a \LaTeXe{} package
that enables the direct compilation
of document sections included by |\include|
to individual files.
\end{abstract}

\begingroup
\parskip0ex
\tableofcontents
\endgroup

%%%%%%%%%%%%%%%%%%%%%%%%%%%%%%%%%%%%%%%%%%%%%%%%%%%%%%%%%%%%%%%%%%%%%%%%%%%%%%%%
%%%%%%%%%%%%%%%%%%%%%%%%%%%%%%%%%%%%%%%%%%%%%%%%%%%%%%%%%%%%%%%%%%%%%%%%%%%%%%%%
\section{Introduction}

\LaTeX{} provides a mechanism to structure a large document (such as a book)
into a main file and several child files (containing the chapters)
using the |\include| command.
This mechanism is beneficial for documents
which span hundreds of pages in order to
make the source file(s) more manageable.
Moreover, compilation can be restricted to
selected child files by means of the |\includeonly| command.
The latter feature can be used to reduce the compilation time while editing
(this was significantly more useful in the earlier days of \LaTeX{})
or to generate a smaller document which is easier to navigate.
Another application of |\includeonly| is to generate
documents consisting of selected parts of the complete document.

However, there are a few drawbacks of the plain |\include| mechanism:
\begin{itemize}
\item
The child files cannot be compiled on their own,
they can only be compiled via the main file.
A naive editing environment
(such as a text editor with an option
to have the current file processed by \LaTeX)
may require one to switch to the main file before compiling;
attempting to compile the child file produces errors.
\item
The main file must be modified (each time)
to adjust the |\includeonly| command
to the present needs. This easily leaves the main file in a messy state.
\item
The generated document will always carry the filename
of the main document. This is inconvenient if
several child files are to be compiled and
to be kept for distribution.
\end{itemize}

The present package provides a simple interface
to make child files individually compilable by \LaTeX{}.
Compiling a child file then has the same effect as compiling
the main file with an |\includeonly| command
to select the appropriate child.
Moreover the generated document will carry the name of the child
rather than the main file.
This resolves all three above issues.

This feature is meant to make the editing of books,
thesis documents and lecture notes somewhat more convenient.
However, the package can also be used efficiently for
composing a series of documents (such as exercise sheets)
which are typically distributed individually.
It then assists the author in generating the individual documents
(potentially in different versions)
as well as a document containing the collected series.
Another application is in developing style files
or other kinds of included material
where compilation of the style file could redirect
to a sample or test file.

%%%%%%%%%%%%%%%%%%%%%%%%%%%%%%%%%%%%%%%%%%%%%%%%%%%%%%%%%%%%%%%%%%%%%%%%%%%%%%%%
%%%%%%%%%%%%%%%%%%%%%%%%%%%%%%%%%%%%%%%%%%%%%%%%%%%%%%%%%%%%%%%%%%%%%%%%%%%%%%%%
\section{Usage}

First of all, the package \textsf{childdoc} is \emph{not} a standard
\LaTeXe{} |.sty| style file! Therefore it needs to be invoked in
a non-standard way.

%%%%%%%%%%%%%%%%%%%%%%%%%%%%%%%%%%%%%%%%%%%%%%%%%%%%%%%%%%%%%%%%%%%%%%%%%%%%%%%%
\subsection{Included Files}
\label{sec:include}

%%%%%%%%%%%%%%%%%%%%%%%%%%%%%%%%%%%%%%%%
\DescribeMacro{\childdocmain}
To use the package, add the commands
\begin{center}
\begin{tabular}{l}
|\input{childdoc.def}|\\
|\childdocmain{}|\\
\end{tabular}
\end{center}
at the very top of the main \LaTeX{} file,
in particular \emph{before} the |\documentclass| statement!
The argument of |\childdocmain| should be left empty
(but it must be present).

%%%%%%%%%%%%%%%%%%%%%%%%%%%%%%%%%%%%%%%%
\DescribeMacro{\childdocof}
Furthermore, add the commands
\begin{center}
\begin{tabular}{l}
|\input{childdoc.def}|\\
|\childdocof{|\textit{main}|}|\\
\end{tabular}
\end{center}
at the top of every child file \textit{child}
which is included by |\include{|\textit{child}|}|
from within the main file
(or at least for those files to be compiled individually).
The argument \textit{main} must be the filename of the main file.

There are a couple of
considerations in setting up the main and child documents:

%%%%%%%%%%%%%%%%%%%%%%%%%%%%%%%%%%%%%%%%
\paragraph{Restrictions.}

Please note the following restrictions:
\begin{itemize}
\item
|\childdocmain| must be called with one argument \textit{main}
to ensure compatibility with earlier version of the package.
It must either be empty (|\childdocmain{}|)
or precisely match the filename of the main file in which it is specified.
See \secref{sec:detection} for further information.
\item
The filename \textit{main} must be specified without the |.tex| extension.
\item
The filename \textit{main} is case sensitive
(even in case-insensitive file systems)
due to internal string comparison.
\item
The argument \textit{main} should be fully expanded, it cannot be a macro.
\item
Subdirectories and special characters should be avoided in filenames.
\item
The command |\childdocmain{|\textit{main}|}| must be followed by a whitespace.
It should not be followed immediately by another command
or by a comment mark `|%|'.
This is because the \TeX{} parser reads the token immediately following
the argument of |\childdocmain| and puts it
at the beginning of every child section;
however, a white\-space is ignored.
\end{itemize}

%%%%%%%%%%%%%%%%%%%%%%%%%%%%%%%%%%%%%%%%
\paragraph{Content of Main File.}

It is advisable to place all content in the child files included by |\include|.
Any output contained in the main file will appear in all child documents
unless suppressed manually;
it cannot be suppressed automatically by the |\includeonly| directive
and thus should normally be avoided.
A method to include some content in the main file
by means of conditional processing is described in \secref{sec:conditional}.

%%%%%%%%%%%%%%%%%%%%%%%%%%%%%%%%%%%%%%%%
\paragraph{Page Numbering.}

When only a part of the document is compiled,
the appropriate numbering of pages
(as well as other status parameters)
is determined from the |.aux| files.
The latter contain information from previous passes.
However this information needs to propagate through
all intermediate child documents.
Therefore the page numbering in child documents may well
be inconsistent until the complete document is compiled at least once.

A useful (if unconventional) way to always ensure a consistent
page numbering is to restart the numbering in each child document
and denote the pages by `\textit{child}|.|\textit{page}'
where \textit{child} represents the chapter/section number of the child file.
This can be achieved by the command
|\numberwithin{page}{|\textit{child}|}|
of the \textsf{amsmath} package
where \textit{child} can be |chapter| or |section|
depending on the chosen structuring.
Alternatively, one can modify the macro |\thepage| appropriately
and reset the counter |page| at the start of each child file.

%%%%%%%%%%%%%%%%%%%%%%%%%%%%%%%%%%%%%%%%%%%%%%%%%%%%%%%%%%%%%%%%%%%%%%%%%%%%%%%%
\subsection{Conditional Processing}
\label{sec:conditional}

The package provides a mechanism to compile different versions
of a document. To customise the versions further some conditional processing
can come in handy to distinguish which version is being compiled.
The package provides two macros to describe the compilation context:

%%%%%%%%%%%%%%%%%%%%%%%%%%%%%%%%%%%%%%%%
\DescribeMacro{\ifchilddoc}
The conditional |\ifchilddoc| distinguishes between the compilation of
child documents and the main document:
%
\begin{center}
|\ifchilddoc |\textit{child-code}| |[|\||else |\textit{main-code}]| \||fi|
\end{center}

%%%%%%%%%%%%%%%%%%%%%%%%%%%%%%%%%%%%%%%%
\DescribeMacro{\childdocname}
\DescribeMacro{\childdocjob}
The macro |\childdocname| contains the filename (without extension)
of the main or child file being processed.
Note that |\childdocjob| will always contain the name of the main file.

%%%%%%%%%%%%%%%%%%%%%%%%%%%%%%%%%%%%%%%%
\paragraph{Title Page.}

Conditional processing can be used to include a title or banner page
in the main document when proper precautions are taken.
Importantly, the code in the main file should ensure that the page counter
(as well as other status parameters which are stored in the |.aux| files)
takes the same value after the conditional processing.
Otherwise the page numbers may take divergent values
depending on which part is compiled.

For example, a title page could be declared by:
%
\begin{center}
\begin{tabular}{l}
|\ifchilddoc\||else|\\
|\addtocounter{page}{-1}|\\
\textit{code for title page}\\
|\newpage|\\
|\||fi|
\end{tabular}
\end{center}
%
A banner page for the child documents can be generated by:
%
\begin{center}
\begin{tabular}{l}
|\ifchilddoc|\\
|\addtocounter{page}{-1}|\\
\textit{code for banner page}\\
|\newpage|\\
|\||fi|
\end{tabular}
\end{center}
%
Here one could write a message such as:
\begin{center}
|This is the part \childdocname{} of \childdocjob{}.|
\end{center}

%%%%%%%%%%%%%%%%%%%%%%%%%%%%%%%%%%%%%%%%%%%%%%%%%%%%%%%%%%%%%%%%%%%%%%%%%%%%%%%%
\subsection{Flags}
\label{sec:flags}

The package makes it easy to generate different versions
of the main or child documents.
To this end compilation flags can be defined
and assigned different default values.
They will be particularly useful in conjunction
with the forwarding mechanism described in \secref{sec:forward}.

For example, it may be useful to have a flag |\version|
which can be set to |draft| or |final|.
The document source will contain some conditional code
depending on the value of |\version|.
Suppose further, the flag should default to |final| for the main file
and to |draft| for child files
which is a natural assignment for editing the document.
This is achieved by placing the following code
in the preamble of the main document
(below the |\childdocmain| directive):
%
\begin{center}
\begin{tabular}{l}
|\ifchilddoc|\\
|\providecommand{\version}{draft}|\\
|\||else|\\
|\providecommand{\version}{final}|\\
|\||fi|
\end{tabular}
\end{center}
%
The definition by |\providecommand| makes sure
that previous definitions are not overwritten.
Further statements |\providecommand{\version}{...}|
can thus be added before the above code to override it.

For the main file, one might add a line
(between |\childdocmain| and the above block)
%
\begin{center}
|%\ifchilddoc\||else\providecommand{\version}{draft}\||fi|
\end{center}
%
which can be uncommented to produce a draft version.
Likewise one can add a line to the very top of a child file
(above the |\childdocof{|\textit{main}|}| directive)
%
\begin{center}
|%\providecommand{\version}{final}|
\end{center}
%
which can be uncommented to produce the final version of this child document.

%%%%%%%%%%%%%%%%%%%%%%%%%%%%%%%%%%%%%%%%%%%%%%%%%%%%%%%%%%%%%%%%%%%%%%%%%%%%%%%%
\subsection{Forwarding}
\label{sec:forward}

Different versions of the main or child documents
using compilation flags as described in \secref{sec:flags}
can be (permanently) stored in different files
for convenient compilation, viewing and distribution.
To this end, the package defines a command
to pass on compilation to a different file:

%%%%%%%%%%%%%%%%%%%%%%%%%%%%%%%%%%%%%%%%
\DescribeMacro{\childdocforward}
The command |\childdocforward| redirects processing to
another source file:
%
\begin{center}
\begin{tabular}{l}
|\input{childdoc.def}|\\
|\childdocforward[|\textit{main}|]{|\textit{dest}|}|\\
\end{tabular}
\end{center}
%
The argument \textit{dest} is the destination file
(without extension).
It should be the main file or one of the child files.
Note that further \textsf{childdoc} directives
such as |\childdocof| and |\childdocforward|
in the indicated file will be processed in this form.
The optional argument \textit{main}
passes on directly to the main file \textit{main}
while pretending to compile the child \textit{dest}.
This form behaves as if \textit{dest}
issues |\childdocof{|\textit{main}|}| right away,
and no further \textsf{childdoc} directives will be processed.

%%%%%%%%%%%%%%%%%%%%%%%%%%%%%%%%%%%%%%%%
\DescribeMacro{\...prefix}
In the alternative form |\childdocforwardprefix|,
%
\begin{center}
\begin{tabular}{l}
|\input{childdoc.def}|\\
|\childdocforwardprefix[|\textit{main}|]{|\textit{prefix}|}{|\textit{dest}|}|
\end{tabular}
\end{center}
%
the destination file is determined by a pattern
depending on the current file:
To make this work, the current file must be called
`{\textit{prefix}\hspace{0.2em}\textit{suffix}}'
with \textit{prefix} matching precisely the argument.
Processing is then passed on to the file
`{\textit{dest}\hspace{0.2em}\textit{suffix}}'.
Surely, the same effect is achieved by
directly specifying the
argument `{\textit{dest}\hspace{0.2em}\textit{suffix}}'
in the first form.
However, that requires to set up a different file
for each child. With the alternative form of the command
all these files can have exactly the same content
which simplifies setting them up and maintaining them.

For example, the following file |draft.tex|
with a compilation flag |\version| as described in \secref{sec:flags}
compiles the main document as a draft:
%
\begin{center}
\begin{tabular}{l}
|\def\version{draft}|\\
|\input{childdoc.def}|\\
|\childdocforward{|\textit{main}|}|
\end{tabular}
\end{center}
%
Likewise, the following files |final|\textit{nn}|.tex|
compile the final version of the child document
|child|\textit{nn}|.tex|:
%
\begin{center}
\begin{tabular}{l}
|\def\version{final}|\\
|\input{childdoc.def}|\\
|\childdocforwardprefix{final}{child}|
\end{tabular}
\end{center}
%

Note that when several versions of a main file and/or of each child file
are to be generated, it may be convenient to set up a |Makefile| or
shell script to automatise the process.

%%%%%%%%%%%%%%%%%%%%%%%%%%%%%%%%%%%%%%%%%%%%%%%%%%%%%%%%%%%%%%%%%%%%%%%%%%%%%%%%
\subsection{Command Line Processing}
\label{sec:commandline}

The effect of redirection files can also be achieved by invoking
the \LaTeX{} compiler with a more elaborate command line.
Most conveniently this should be done as part
of a shell script or a |Makefile|.

When using \textsf{childdoc} in the main file, the following
command lines effectively perform a redirection
(note that depending on the shell being used,
backslashes may have to be doubled: `|\|' $\to$ `|\\|'):
%
\begin{center}
|... -jobname "|\textit{target}|" |\\|"|[\textit{flags}]%
|\input{childdoc.def}\childdocforward[|\textit{main}|]{|\textit{dest}|}"|
\end{center}
%
Here \textit{target} is the name of the output file,
\textit{main} is the name of the main file
and \textit{dest} is the name of the main or child file to be processed
(all filenames without extensions).
The optional argument \textit{main} can be omitted
if \textit{main} matches \textit{dest}.
Optionally, compilation \textit{flags} can be defined via |\def| commands.
This command line makes the \TeX{} engine believe
it is compiling the file \textit{target}
whose content is specified as the latter parameter.
The provided code then forwards the processing to
\textit{main} or \textit{dest} as described in \secref{sec:forward}.

%%%%%%%%%%%%%%%%%%%%%%%%%%%%%%%%%%%%%%%%%%%%%%%%%%%%%%%%%%%%%%%%%%%%%%%%%%%%%%%%
\subsection{Include by Input}
\label{sec:input}

Including child documents by |\include| has some restrictions by design.
Most notably, the content of a child document always occupies
its own set of pages; pages cannot be shared between child documents.
Usually, this behaviour makes perfect sense
because each child document contain an essential part of the document.
However, in some situations it may be desirable to compose
a document from a collection of parts
without having mandatory page breaks between then.
For this case, the package
provides a mechanism to include parts
by |\input| which can also be processed individually.
However, by construction this mechanism
requires manual handling of the content to be output.

%%%%%%%%%%%%%%%%%%%%%%%%%%%%%%%%%%%%%%%%
\DescribeMacro{\ifchilddocmanual}
The main file should be prepared as usual, see \secref{sec:include}.
However, the document body must make a distinction
between processing of an individual part and of the main document, e.g.:
%
\begin{center}
\begin{tabular}{l}
|\ifchilddocmanual|\\
|\input{\childdocname}|\\
|\||else|\\
\textit{document body with }|\input{|\textit{part}|}|\\
|\||fi|
\end{tabular}
\end{center}
%
The conditional |\ifchilddocmanual| is true whenever
a part to be included by |\input| is being compiled,
and the name of the part is stored in |\childdocname|.

%%%%%%%%%%%%%%%%%%%%%%%%%%%%%%%%%%%%%%%%
\DescribeMacro{\childdocby}
Each part to be included by |\input| should start with:
%
\begin{center}
\begin{tabular}{l}
|\input{childdoc.def}|\\
|\childdocby{|\textit{main}|}|\\
\end{tabular}
\end{center}
%
The directive |\childdocby| is similar to |\childdocof|
described in \secref{sec:include},
but the subsequent selection of content must be done manually.
To that end, both |\ifchilddoc| and |\ifchilddocmanual|
will be true upon processing of a part,
and the name of the part is stored in |\childdocname|.
Note that |\jobname| will be set to the filename of the current part
so that each part receives an individual |.aux| file
that does not interfere with the |.aux| file(s) of the main document.
This behaviour can be altered by the alternative form
|\childdocby[*]{|\textit{main}|}| (with a non-empty optional argument)
which uses the |.aux| file of the main document
by setting |\jobname| to \textit{main}.

%%%%%%%%%%%%%%%%%%%%%%%%%%%%%%%%%%%%%%%%%%%%%%%%%%%%%%%%%%%%%%%%%%%%%%%%%%%%%%%%
\subsection{Driver Development}
\label{sec:driver}

The \textsf{childdoc} mechanism can also be use for the development
of definition files such as \LaTeX{} styles or classes.
This case differs from the above setup with multiple parts
included by |\include| in that no |\includeonly| should be invoked.
This can be achieved by starting the include file
(before |\ProvidesPackage|) with:
%
\begin{center}
\begin{tabular}{l}
|\input{childdoc.def}|\\
|\childdocforward{|\textit{main}|}|\\
\end{tabular}
\end{center}
%
or alternatively with:
%
\begin{center}
\begin{tabular}{l}
|\input{childdoc.def}|\\
|\childdocby{|\textit{main}|}|\\
\end{tabular}
\end{center}
%
Both forms have slightly different effects as described above.
The main file is prepared as usual, see \secref{sec:include}.

%%%%%%%%%%%%%%%%%%%%%%%%%%%%%%%%%%%%%%%%%%%%%%%%%%%%%%%%%%%%%%%%%%%%%%%%%%%%%%%%
\subsection{Legacy Detection}
\label{sec:detection}

The directive |\childdocmain| in the main file can detect
whether the complete document or merely a child is to be compiled
even without using the directive |\childdocof|.
This method is deprecated because it is less robust
and there is no compelling reason to use it;
it is merely provided for backward compatibility
and it may be removed in future versions.

If the detection mechanism is to be used,
it is mandatory to correctly specify
the filename of the main file as the argument of |\childdocmain|:
%
\begin{center}
\begin{tabular}{l}
|\input{childdoc.def}|\\
|\childdocmain{|\textit{main}|}|\\
\end{tabular}
\end{center}
%
If |\jobname| does not match the argument \textit{main} of |\childdocmain|,
it is assumed that |\jobname| points to the child file to be compiled.
When using |\childdocmain| with the main file specified as argument,
it suffices to start a child file
with just |\input{|\textit{main}|}|
without loading of the package and using |\childdocof|.
If instead all processing is done
with the appropriate \textsf{childdoc} directives,
the argument of \textit{main} of |\childdocmain| can be empty.

An alternative version of the command line processing described
in \secref{sec:commandline} using the detection mechanism reads:
%
\begin{center}
|... -jobname "|\textit{target}|" "|[\textit{flags}]%
[|\def\jobname{|\textit{dest}|}|]|\input{|\textit{main}|}"|
\end{center}

%%%%%%%%%%%%%%%%%%%%%%%%%%%%%%%%%%%%%%%%%%%%%%%%%%%%%%%%%%%%%%%%%%%%%%%%%%%%%%%%
\subsection{Manual Code}
\label{sec:manual}

In case one cannot be certain whether the definitions file |childdoc.def|
is installed on the target \TeX{} distribution
and one prefers not to ship it,
it is conceivable to paste a few relevant commands into the sources.

To that end, drop all statements |\input{childdoc.def}|
and perform the replacements as outlined below.
Instead of |\childdocmain{|\textit{main}|}| add the following code
to the top of the main file:
%
\begin{center}
\begin{tabular}{l}
|\||ifdefined\childdocname\endinput\||fi\newif\ifchilddoc|\\
|\edef\childdocname{\scantokens\expandafter{\jobname\noexpand}}|\\
|\def\childdocmain{|\textit{main}|}\||ifx\childdocmain\childdocname\||else|\\
|\childdoctrue\includeonly{\childdocname}\let\jobname\childdocmain\||fi|\\
\end{tabular}
\end{center}
%
Instead of |\childdocof{|\textit{main}|}| just include the main file
at the top of each child file:
%
\begin{center}
|\input{|\textit{main}|}|
\end{center}
%
A simple redirection |\childdocforward{|\textit{dest}|}| is achieved by:
%
\begin{center}
|\def\jobname{|\textit{dest}|}\input{\jobname}|
\end{center}
%
The redirection with prefix
|\childdocforwardprefix[|\textit{prefix}|]{|\textit{dest}|}|
is accomplished by:
%
\begin{center}
\begin{tabular}{l}
|{\edef\jobname{\scantokens\expandafter{\jobname\noexpand}}|\\
|\def\redirectjob |\textit{prefix}|#1~~~{\gdef\jobname{|\textit{dest}|#1}}|\\
|\expandafter\redirectjob\jobname~~~}\input{\jobname}|
\end{tabular}
\end{center}

In an alternative approach,
child documents can be compiled by a specific command line
without additional code or specific definitions:
%
\begin{center}
|... -jobname "|\textit{target}|" "|[\textit{flags}]%
|\includeonly{|\textit{dest}|}\input{|\textit{main}|}"|
\end{center}
%

%%%%%%%%%%%%%%%%%%%%%%%%%%%%%%%%%%%%%%%%%%%%%%%%%%%%%%%%%%%%%%%%%%%%%%%%%%%%%%%%
%%%%%%%%%%%%%%%%%%%%%%%%%%%%%%%%%%%%%%%%%%%%%%%%%%%%%%%%%%%%%%%%%%%%%%%%%%%%%%%%
\section{Information}

%%%%%%%%%%%%%%%%%%%%%%%%%%%%%%%%%%%%%%%%%%%%%%%%%%%%%%%%%%%%%%%%%%%%%%%%%%%%%%%%
\subsection{Copyright}

Copyright \copyright{} 2017--2018 Niklas Beisert

This work may be distributed and/or modified under the
conditions of the \LaTeX{} Project Public License, either version 1.3
of this license or (at your option) any later version.
The latest version of this license is in
  \url{http://www.latex-project.org/lppl.txt}
and version 1.3 or later is part of all distributions of \LaTeX{}
version 2005/12/01 or later.

This work has the LPPL maintenance status `maintained'.

The Current Maintainer of this work is Niklas Beisert.

This work consists of the files |README.txt|, |childdoc.ins| and |childdoc.dtx|
as well as the derived files |childdoc.def|, |cdocsamp.tex|
with |cdocsch1.tex|, |cdocsch2.tex|, |cdocspt3.tex|, |cdocspt4.tex|,
|cdocsdrf.tex|, |cdocsfn1.tex|, |cdocsfn2.tex|
as well as |childdoc.pdf|.

%%%%%%%%%%%%%%%%%%%%%%%%%%%%%%%%%%%%%%%%%%%%%%%%%%%%%%%%%%%%%%%%%%%%%%%%%%%%%%%%
\subsection{Files and Installation}

The package consists of the files:
%
\begin{center}
\begin{tabular}{ll}
    |README.txt|   & readme file \\
    |childdoc.ins| & installation file \\
    |childdoc.dtx| & source file \\
    |childdoc.def| & definition file \\
    |cdocsamp.tex| & sample main file \\
    |cdocsch1.tex| & sample include file \\
    |cdocsch2.tex| & sample include file \\
    |cdocspt3.tex| & sample part file \\
    |cdocspt4.tex| & sample part file \\
    |cdocsdrf.tex| & sample redirection file \\
    |cdocsfn1.tex| & sample redirection file \\
    |cdocsfn2.tex| & sample redirection file \\
    |childdoc.pdf| & manual
\end{tabular}
\end{center}
%
The distribution consists of the files
|README.txt|, |childdoc.ins| and |childdoc.dtx|.
%
\begin{itemize}
\item
Run (pdf)\LaTeX{} on |childdoc.dtx|
to compile the manual |childdoc.pdf| (this file).
\item
Run \LaTeX{} on |childdoc.ins| to create the definitions file |childdoc.def|
and the sample |cdocsamp.tex| with include files
|cdocsch1.tex|, |cdocsch2.tex|, |cdocspt3.tex|, |cdocspt4.tex|,
|cdocsdrf.tex|, |cdocsfn1.tex|, |cdocsfn2.tex|.
Then copy the file |childdoc.def| to an appropriate directory of your \LaTeX{}
distribution, e.g.\ \textit{texmf-root}|/tex/latex/childdoc|.
\end{itemize}

%%%%%%%%%%%%%%%%%%%%%%%%%%%%%%%%%%%%%%%%%%%%%%%%%%%%%%%%%%%%%%%%%%%%%%%%%%%%%%%%
\subsection{Related CTAN Packages}

There are several other packages which offer a similar functionality:
%
\begin{itemize}
\item
The packages
\href{http://ctan.org/pkg/docmute}{\textsf{docmute}},
\href{http://ctan.org/pkg/includex}{\textsf{includex}} and
\href{http://ctan.org/pkg/standalone}{\textsf{standalone}}
provide commands to include only the document body of
a child file thus allowing both files to be compiled individually.
\item
The packages \href{http://ctan.org/pkg/subdocs}{\textsf{subdocs}}
and \href{http://ctan.org/pkg/subfiles}{\textsf{subfiles}}
provide structures in which the main and child documents can be
encapsulated and allowing them to be compiled individually.
The inclusion mechanism is different from the conventional |\include|.
\item
The package \href{http://ctan.org/pkg/combine}{\textsf{combine}}
is an elaborate solution to combine several documents into one.
\end{itemize}
%
See also the CTAN topic \href{http://ctan.org/topic/subdocs}{\textsf{subdocs}}
for further related packages.
The present package differs from the above solutions in that
a document structure constructed with the conventional |\include| mechanism
just needs two extra commands at the top of every file
such that all constituent files can be compiled individually.

%%%%%%%%%%%%%%%%%%%%%%%%%%%%%%%%%%%%%%%%%%%%%%%%%%%%%%%%%%%%%%%%%%%%%%%%%%%%%%%%
%\subsection{Feature Suggestions}
%
%The following is a list of features which may be useful for future
%versions of this package:
%%
%\begin{itemize}
%\item
%\ldots
%\end{itemize}

%%%%%%%%%%%%%%%%%%%%%%%%%%%%%%%%%%%%%%%%%%%%%%%%%%%%%%%%%%%%%%%%%%%%%%%%%%%%%%%%
\subsection{Revision History}

%%%%%%%%%%%%%%%%%%%%%%%%%%%%%%%%%%%%%%%%
\paragraph{v2.0:} 2018/12/30

\begin{itemize}
\item
immediate forward processing
\item
added |\childdocby| mechanism
\item
manual restructured
\end{itemize}

%%%%%%%%%%%%%%%%%%%%%%%%%%%%%%%%%%%%%%%%
\paragraph{v1.6:} 2018/01/17

\begin{itemize}
\item
application for development of include files
\item
corrections to manual
\end{itemize}

%%%%%%%%%%%%%%%%%%%%%%%%%%%%%%%%%%%%%%%%
\paragraph{v1.5:} 2017/05/21

\begin{itemize}
\item
more complete structuring introduced
\item
|\childdocof| introduced
\item
|\childdoc| renamed to |\childdocmain|
\item
|\childredirect| renamed to |\childdocforward| and |\childdocforwardprefix|
and functionality expanded
\end{itemize}

%%%%%%%%%%%%%%%%%%%%%%%%%%%%%%%%%%%%%%%%
\paragraph{v1.0:} 2017/04/27

\begin{itemize}
\item
manual and install package
\item
first version published on CTAN
\end{itemize}

%%%%%%%%%%%%%%%%%%%%%%%%%%%%%%%%%%%%%%%%
\paragraph{v0.6:} 2017/04/26

\begin{itemize}
\item
redirection mechanism added
\end{itemize}

%%%%%%%%%%%%%%%%%%%%%%%%%%%%%%%%%%%%%%%%
\paragraph{v0.5:} 2017/04/26

\begin{itemize}
\item
functionality in definition file
\end{itemize}


%%%%%%%%%%%%%%%%%%%%%%%%%%%%%%%%%%%%%%%%%%%%%%%%%%%%%%%%%%%%%%%%%%%%%%%%%%%%%%%%
%%%%%%%%%%%%%%%%%%%%%%%%%%%%%%%%%%%%%%%%%%%%%%%%%%%%%%%%%%%%%%%%%%%%%%%%%%%%%%%%
%%%%%%%%%%%%%%%%%%%%%%%%%%%%%%%%%%%%%%%%%%%%%%%%%%%%%%%%%%%%%%%%%%%%%%%%%%%%%%%%
\appendix

\settowidth\MacroIndent{\rmfamily\scriptsize 000\ }

 \DocInput{childdoc.dtx}

\end{document}
%</driver>
% \fi
%
% %%%%%%%%%%%%%%%%%%%%%%%%%%%%%%%%%%%%%%%%%%%%%%%%%%%%%%%%%%%%%%%%%%%%%%%%%%%%%%
% %%%%%%%%%%%%%%%%%%%%%%%%%%%%%%%%%%%%%%%%%%%%%%%%%%%%%%%%%%%%%%%%%%%%%%%%%%%%%%
% \section{Sample}
%\iffalse
%<*samplemain>
%\fi
%
% The following presents a sample document
% with two chapters, two parts, a title page,
% a compile flag as well as three forwarding files to set the flag.
% It consists of eight |.tex| files:
% \begin{center}
% \begin{tabular}{ll}
% |cdocsamp.tex|&main file\\
% |cdocsch1.tex|&include file for chapter 1\\
% |cdocsch2.tex|&include file for chapter 2\\
% |cdocspt3.tex|&include file for part 3\\
% |cdocspt4.tex|&include file for part 4\\
% |cdocsdrf.tex|&forwarding file for main file in draft mode\\
% |cdocsfi1.tex|&forwarding file for final version of chapter 1\\
% |cdocsfi2.tex|&forwarding file for final version of chapter 2\\
% \end{tabular}
% \end{center}
% Each of the eight files can be compiled directly by the \LaTeX{} compiler.
%
% %%%%%%%%%%%%%%%%%%%%%%%%%%%%%%%%%%%%%%
% \paragraph{Main File.}
%
% The main file is called |cdocsamp.tex|.
%
% Load the \textsf{childdoc} definitions and
% declare the filename for the main document:
%    \begin{macrocode}
\input{childdoc.def}
\childdocmain{}
%    \end{macrocode}

% Optional override for |\version| flag:
%    \begin{macrocode}
%%\ifchilddoc\else\providecommand{\version}{draft}\fi
%    \end{macrocode}

% Define the default values for the |\version| flag
% (|final| for the main file and |draft| for childs):
%    \begin{macrocode}
\ifchilddoc
\providecommand{\version}{draft}
\else
\providecommand{\version}{final}
\fi
%    \end{macrocode}

% Load the standard document class:
%    \begin{macrocode}
\documentclass[12pt]{article}
%    \end{macrocode}

% Start the document body:
%    \begin{macrocode}
\begin{document}
%    \end{macrocode}

% Declare a title page.
% Print title, part of document being processed and version flag:
%    \begin{macrocode}
\addtocounter{page}{-1}
\begin{center}
{\LARGE\bfseries{}childdoc example\par}
\vspace{1cm}
\ifchilddoc
\ifchilddocmanual part\else chapter\fi:
`\childdocname' of `\childdocjob'\par
\else
main document: `\childdocjob'\par
\fi
version: \version\par
\end{center}
\newpage
%    \end{macrocode}

% Manually include selected file,
% otherwise process as usual:
%    \begin{macrocode}
\ifchilddocmanual
\section*{part `\childdocname'}
\input{\childdocname}
\else
%    \end{macrocode}

% Include the two chapters:
%    \begin{macrocode}
\include{cdocsch1}
\include{cdocsch2}
%    \end{macrocode}

% Include the two parts unless only chapters should be displayed:
%    \begin{macrocode}
\ifchilddoc\else
\section{part three}
\input{cdocspt3}
\section{part four}
\input{cdocspt4}
\fi
%    \end{macrocode}

% Process as usual until here:
%    \begin{macrocode}
\fi
%    \end{macrocode}

% End of document body:
%    \begin{macrocode}
\end{document}
%    \end{macrocode}
%\iffalse
%</samplemain>
%\fi
%
% %%%%%%%%%%%%%%%%%%%%%%%%%%%%%%%%%%%%%%
% \paragraph{Chapter Include Files.}
%
% The include files are called |cdocsch1.tex| and |cdocsch2.tex|.
%
%\iffalse
%<*samplechap1|samplechap2>
%\fi

% Optional override for |\version| flag:
%    \begin{macrocode}
%%\providecommand{\version}{final}
%    \end{macrocode}

% Include the main document:
%    \begin{macrocode}
\input{childdoc.def}
\childdocof{cdocsamp}
%    \end{macrocode}

%\iffalse
%</samplechap1|samplechap2>
%\fi
%
%\iffalse
%<*samplechap1>
%\fi
% Some text for chapter 1:
%    \begin{macrocode}
\section{one}
some text in chapter one
%    \end{macrocode}

%\iffalse
%</samplechap1>
%\fi
% Some text for chapter 2:
%\iffalse
%<*samplechap2>
%\fi
%    \begin{macrocode}
\section{two}
more text in chapter two
%    \end{macrocode}

%\iffalse
%</samplechap2>
%\fi
%
% %%%%%%%%%%%%%%%%%%%%%%%%%%%%%%%%%%%%%%
% \paragraph{Part Include Files.}
%
% The include files are called |cdocspt3.tex| and |cdocspt4.tex|.
%
%\iffalse
%<*samplepart3|samplepart4>
%\fi

% Optional override for |\version| flag:
%    \begin{macrocode}
%%\providecommand{\version}{final}
%    \end{macrocode}

% Include the main document:
%    \begin{macrocode}
\input{childdoc.def}
\childdocby{cdocsamp}
%    \end{macrocode}

%\iffalse
%</samplepart3|samplepart4>
%\fi
%
%\iffalse
%<*samplepart3>
%\fi
% Some text for part 3:
%    \begin{macrocode}
some text in part three
%    \end{macrocode}

%\iffalse
%</samplepart3>
%\fi
% Some text for part 4:
%\iffalse
%<*samplepart4>
%\fi
%    \begin{macrocode}
more text in part four
%    \end{macrocode}

%\iffalse
%</samplepart4>
%\fi
%
% %%%%%%%%%%%%%%%%%%%%%%%%%%%%%%%%%%%%%%
% \paragraph{Forwarding for a Complete Draft.}
%
% The following forwarding file |cdocsdrf.tex|
% compiles the main document in draft mode:
%\iffalse
%<*sampledraft>
%\fi
%    \begin{macrocode}
\def\version{draft}
\input{childdoc.def}
\childdocforward{cdocsamp}
%    \end{macrocode}

%\iffalse
%</sampledraft>
%\fi
%
% %%%%%%%%%%%%%%%%%%%%%%%%%%%%%%%%%%%%%%
% \paragraph{Forwarding for Final Version of the Chapters.}
%
% The following forwarding files |cdocsfn1.tex| and |cdocsfn2.tex|
% (with identical content)
% compile the final versions of the child documents
% |cdocsch1.tex| and |cdocsch2.tex|, respectively:
%\iffalse
%<*samplefinal>
%\fi
%    \begin{macrocode}
\def\version{final}
\input{childdoc.def}
\childdocforwardprefix[cdocsamp]{cdocsfn}{cdocsch}
%    \end{macrocode}

%\iffalse
%</samplefinal>
%\fi
%
% %%%%%%%%%%%%%%%%%%%%%%%%%%%%%%%%%%%%%%
% \paragraph{Command Line Processing.}
%
% The following three command lines generate the output files
% |cdocscld|, |cdocscl1| and |cdocscl2|
% which should be identical to
% |cdocsdrf|, |cdocsch1| and |cdocsfn2|, respectively:
% \begin{center}
% \begin{tabular}{l}
% |latex -jobname cdocscld \|\\
% |  "\def\version{draft}\input{childdoc.def}\childdocforward{cdocsamp}"|\\
% |latex -jobname cdocscl1 \|\\
% |  "\input{childdoc.def}\childdocforward[cdocsamp]{cdocsch1}"|\\
% |latex -jobname cdocscl2 \|\\
% |  "\def\version{final}\input{childdoc.def}\childdocforward{cdocsch2}"|
% \end{tabular}
% \end{center}
% Note that the trailing backslash on each first line
% merely continues the input to the second line
% (for convenient cut ant paste).
% Furthermore, the command |latex| can be replaced by any
% of its alternative versions such as |pdflatex|.
%
% %%%%%%%%%%%%%%%%%%%%%%%%%%%%%%%%%%%%%%%%%%%%%%%%%%%%%%%%%%%%%%%%%%%%%%%%%%%%%%
% %%%%%%%%%%%%%%%%%%%%%%%%%%%%%%%%%%%%%%%%%%%%%%%%%%%%%%%%%%%%%%%%%%%%%%%%%%%%%%
% \section{Implementation}
%\iffalse
%<*package>
%\fi
%
% This section describes the definitions file |childdoc.def|.

% The definitions cannot be loaded using |\usepackage| or |\RequirePackage|
% which has a mechanism to prevent loading a style file more than once.
% When loading the definitions by means of |\input|
% multiple instances have to be prevented manually:
%\iffalse
%This code needs to be before the `\ProvidesFile' directive
%which is defined at the beginning of this file.
%Therefore it is also placed there and commented out here.
%</package>
%<*discard>
%\fi
%    \begin{macrocode}
\ifdefined\childdocmain\endinput\fi
%    \end{macrocode}
%\iffalse
%</discard>
%<*package>
%\fi
%
% \macro{\ifchilddoc}
% \macro{\ifchilddocmanual}
% The conditional |\ifchilddoc| tells whether a
% child (true) or main (false) document is being compiled.
% The conditional |\ifchilddocmanual| tells whether
% the |\includeonly| mechanism is used (false) or
% the selection of child files must be performed manually (true).
% The definitions initialise to false:
%    \begin{macrocode}
\newif\ifchilddoc
\newif\ifchilddocmanual
%    \end{macrocode}

% \macro{\childdocname}
% \macro{\childdocjob}
% The macro |\childdocname| stores the name of the main document
% to be compiled. The macro |\childdocjob| stores the name of
% the document on which the \LaTeX{} compiler was originally invoked.
% The content of |\jobname| cannot be compared
% to filenames specified in the source due to different catcodes.
% The following code rescans |\jobname|, stores the result
% in |\childdocname| and saves a copy in |\childdocjob|:
%    \begin{macrocode}
\edef\childdocname{\scantokens\expandafter{\jobname\noexpand}}
\let\childdocjob\childdocname
%    \end{macrocode}

% \macro{\childdocdisable}
% The macro |\childdocdisable| prevents the main file
% from being processed more than once.
% At this stage, the main document command |\childdocmain|
% is assumed to be called once again where it should do nothing.
% Any subsequent call to it should prevent
% a secondary processing of the main document
% It overwrites the forwarding commands
% |\childdocof| and |\childdocforward|
% with empty macros to prevent further inclusions of the main document:
%    \begin{macrocode}
\newcommand{\childdocdisable}
{
  \renewcommand{\childdocmain}[1]{\renewcommand{\childdocmain}[1]{\endinput}}
  \renewcommand{\childdocof}[1]{}
  \renewcommand{\childdocby}[2][]{}
  \renewcommand{\childdocforward}[2][]{}
  \renewcommand{\childdocdisable}{}
}
%    \end{macrocode}

% \macro{\childdocmain}
% The macro |\childdocmain| is to be called at the top of the main file
% with nothing or the main filename (without extension) as argument.
% First, it breaks loops.
% If the argument is not empty and does not match |\childdocname|
% (which is set by the first inclusion of |childdoc.def|),
% |\ifchilddoc| is set to true, |\includeonly| is applied to the child file
% and |\jobname| is set to the main file
% (for proper handling of |.aux| files):
%    \begin{macrocode}
\newcommand{\childdocmain}[1]
{
  \childdocdisable\childdocmain{}
  \if?#1?\else
    \begingroup
      \def\childdoctmp{#1}
      \ifx\childdoctmp\childdocname
        \def\childdoctmp{}
      \else
        \def\childdoctmp
        {
          \childdoctrue
          \includeonly{\childdocname}
          \def\childdocjob{#1}
          \def\jobname{#1}
        }
      \fi
      \expandafter
    \endgroup
    \childdoctmp
  \fi
}
%    \end{macrocode}

% \macro{\childdocof}
% The command |\childdocof| redirects
% compilation to the main file |#1|.
%    \begin{macrocode}
\newcommand{\childdocof}[1]
{
  \childdocdisable
  \childdoctrue
  \includeonly{\childdocname}
  \def\jobname{#1}
  \def\childdocjob{#1}
  \input{#1}
}
%    \end{macrocode}

% \macro{\childdocby}
% The command |\childdocby| ....
%    \begin{macrocode}
\newcommand{\childdocby}[2][]
{
  \childdocdisable
  \childdoctrue
  \childdocmanualtrue
  \if?#1?\else
    \def\jobname{#2}
  \fi
  \def\childdocjob{#2}
  \input{#2}
  \endinput
}
%    \end{macrocode}

% \macro{\childdocforward}
% The command |\childdocforward| redirects
% compilation to the main file or
% (if the optional argument is given) a child file.
% Parameters are set as if the main file
% or a child file starting with |\childdocof| was compiled.
% Then compilation is handed over to the main file:
%    \begin{macrocode}
\newcommand{\childdocforward}[2][]
{
  \begingroup
    \if?#1?
      \def\childdoctmp
      {
        \def\childdocname{#2}
        \def\childdocjob{#2}
        \def\jobname{#2}
        \input{#2}
        \endinput
      }
    \else
      \def\childdoctmp
      {
        \childdocdisable
        \def\childdocname{#2}
        \childdoctrue
        \includeonly{#2}
        \def\childdocjob{#1}
        \def\jobname{#1}
        \input{#1}
        \endinput
      }
    \fi
    \expandafter
  \endgroup
  \childdoctmp
}
%    \end{macrocode}

% \macro{\childdocforwardprefix}
% The command |\childdocforwardprefix| redirects
% compilation to the main or a child file by means of a pattern.
% The prefix |#1| in the current filename is replaced by |#2|
% and the suffix of the current filename is kept
% (it is assumed that the filename does not contain the substring `|~~~|'
% which is used as a delimiter).
% Compilation is handed over to the new file by |\childdocforward|:
%    \begin{macrocode}
\newcommand{\childdocforwardprefix}[3][]
{
  \begingroup
    \def\childdocextract #2##1~~~{\def\childdoctmp{\childdocforward[#1]{#3##1}}}
    \expandafter\childdocextract\childdocname~~~
    \expandafter
  \endgroup
  \childdoctmp
}
%    \end{macrocode}

% \macro{\childdoc}
% The deprecated macro |\childdoc| is a legacy version of |\childdocmain|:
%    \begin{macrocode}
\newcommand{\childdoc}{\childdocmain}
%    \end{macrocode}

% \macro{\childdocredirect}
% The deprecated macro |\childdocredirect| is a legacy version
% of |\childdocforward| and |\childdocforwardprefix|:
%    \begin{macrocode}
\newcommand{\childdocredirect}[2][]
{
  \begingroup
    \if?#1?
      \def\childdoctmp{\childdocforward{#2}}
    \else
      \def\childdoctmp{\childdocforwardprefix{#1}{#2}}
    \fi
    \expandafter
  \endgroup
  \childdoctmp
}
%    \end{macrocode}

%\iffalse
%</package>
%\fi
%
\endinput

\childdocforwardprefix[cdocsamp]{cdocsfn}{cdocsch}
%    \end{macrocode}

%\iffalse
%</samplefinal>
%\fi
%
% %%%%%%%%%%%%%%%%%%%%%%%%%%%%%%%%%%%%%%
% \paragraph{Command Line Processing.}
%
% The following three command lines generate the output files
% |cdocscld|, |cdocscl1| and |cdocscl2|
% which should be identical to
% |cdocsdrf|, |cdocsch1| and |cdocsfn2|, respectively:
% \begin{center}
% \begin{tabular}{l}
% |latex -jobname cdocscld \|\\
% |  "\def\version{draft}% \iffalse
%
% childdoc.dtx Copyright (C) 2017-2018 Niklas Beisert
%
% This work may be distributed and/or modified under the
% conditions of the LaTeX Project Public License, either version 1.3
% of this license or (at your option) any later version.
% The latest version of this license is in
%   http://www.latex-project.org/lppl.txt
% and version 1.3 or later is part of all distributions of LaTeX
% version 2005/12/01 or later.
%
% This work has the LPPL maintenance status `maintained'.
%
% The Current Maintainer of this work is Niklas Beisert.
%
% This work consists of the files childdoc.dtx and childdoc.ins
% and the derived files childdoc.def and cdocsamp.tex with
% cdocsch1.tex, cdocsch2.tex, cdocsdrf.tex, cdocsfn1.tex, cdocsfn2.tex.
%
%<package>\ifdefined\childdocmain\endinput\fi
%<package>\ProvidesFile{childdoc.def}[2018/12/30 v2.0 child document driver]
%<samplemain>\ProvidesFile{cdocsamp.tex}[2018/12/30 v2.0 sample for childdoc]
%<*driver>
%\ProvidesFile{childdoc.drv}[2018/12/30 v2.0 childdoc reference manual file]
\PassOptionsToClass{10pt,a4paper}{article}
\documentclass{ltxdoc}

\usepackage[margin=35mm]{geometry}
\usepackage{hyperref}
\usepackage{hyperxmp}
\usepackage[usenames]{color}

\hypersetup{colorlinks=true}
\hypersetup{pdfstartview=FitH}
\hypersetup{pdfpagemode=UseNone}
\hypersetup{pdfsource={}}
\hypersetup{pdflang={en-UK}}
\hypersetup{pdfcopyright={Copyright 2017-2018 Niklas Beisert.
  This work may be distributed and/or modified under the
  conditions of the LaTeX Project Public License, either version 1.3
  of this license or (at your option) any later version.}}
\hypersetup{pdflicenseurl={http://www.latex-project.org/lppl.txt}}
\hypersetup{pdfcontactaddress={ETH Zurich, ITP, HIT K,
  Wolfgang-Pauli-Strasse 27}}
\hypersetup{pdfcontactpostcode={8093}}
\hypersetup{pdfcontactcity={Zurich}}
\hypersetup{pdfcontactcountry={Switzerland}}
\hypersetup{pdfcontactemail={nbeisert@itp.phys.ethz.ch}}
\hypersetup{pdfcontacturl={http://people.phys.ethz.ch/\xmptilde nbeisert/}}

\newcommand{\secref}[1]{\hyperref[#1]{section \ref*{#1}}}

\parskip1ex
\parindent0pt
\let\olditemize\itemize
\def\itemize{\olditemize\parskip0pt}

\begin{document}

\title{The \textsf{childdoc} Package}
\hypersetup{pdftitle={The childdoc Package}}
\author{Niklas Beisert\\[2ex]
  Institut f\"ur Theoretische Physik\\
  Eidgen\"ossische Technische Hochschule Z\"urich\\
  Wolfgang-Pauli-Strasse 27, 8093 Z\"urich, Switzerland\\[1ex]
  \href{mailto:nbeisert@itp.phys.ethz.ch}
  {\texttt{nbeisert@itp.phys.ethz.ch}}}
\hypersetup{pdfauthor={Niklas Beisert}}
\hypersetup{pdfsubject={Manual for the LaTeX2e Package childdoc}}
\date{30 December 2018, \textsf{v2.0}}
\maketitle

\begin{abstract}\noindent
\textsf{childdoc} is a \LaTeXe{} package
that enables the direct compilation
of document sections included by |\include|
to individual files.
\end{abstract}

\begingroup
\parskip0ex
\tableofcontents
\endgroup

%%%%%%%%%%%%%%%%%%%%%%%%%%%%%%%%%%%%%%%%%%%%%%%%%%%%%%%%%%%%%%%%%%%%%%%%%%%%%%%%
%%%%%%%%%%%%%%%%%%%%%%%%%%%%%%%%%%%%%%%%%%%%%%%%%%%%%%%%%%%%%%%%%%%%%%%%%%%%%%%%
\section{Introduction}

\LaTeX{} provides a mechanism to structure a large document (such as a book)
into a main file and several child files (containing the chapters)
using the |\include| command.
This mechanism is beneficial for documents
which span hundreds of pages in order to
make the source file(s) more manageable.
Moreover, compilation can be restricted to
selected child files by means of the |\includeonly| command.
The latter feature can be used to reduce the compilation time while editing
(this was significantly more useful in the earlier days of \LaTeX{})
or to generate a smaller document which is easier to navigate.
Another application of |\includeonly| is to generate
documents consisting of selected parts of the complete document.

However, there are a few drawbacks of the plain |\include| mechanism:
\begin{itemize}
\item
The child files cannot be compiled on their own,
they can only be compiled via the main file.
A naive editing environment
(such as a text editor with an option
to have the current file processed by \LaTeX)
may require one to switch to the main file before compiling;
attempting to compile the child file produces errors.
\item
The main file must be modified (each time)
to adjust the |\includeonly| command
to the present needs. This easily leaves the main file in a messy state.
\item
The generated document will always carry the filename
of the main document. This is inconvenient if
several child files are to be compiled and
to be kept for distribution.
\end{itemize}

The present package provides a simple interface
to make child files individually compilable by \LaTeX{}.
Compiling a child file then has the same effect as compiling
the main file with an |\includeonly| command
to select the appropriate child.
Moreover the generated document will carry the name of the child
rather than the main file.
This resolves all three above issues.

This feature is meant to make the editing of books,
thesis documents and lecture notes somewhat more convenient.
However, the package can also be used efficiently for
composing a series of documents (such as exercise sheets)
which are typically distributed individually.
It then assists the author in generating the individual documents
(potentially in different versions)
as well as a document containing the collected series.
Another application is in developing style files
or other kinds of included material
where compilation of the style file could redirect
to a sample or test file.

%%%%%%%%%%%%%%%%%%%%%%%%%%%%%%%%%%%%%%%%%%%%%%%%%%%%%%%%%%%%%%%%%%%%%%%%%%%%%%%%
%%%%%%%%%%%%%%%%%%%%%%%%%%%%%%%%%%%%%%%%%%%%%%%%%%%%%%%%%%%%%%%%%%%%%%%%%%%%%%%%
\section{Usage}

First of all, the package \textsf{childdoc} is \emph{not} a standard
\LaTeXe{} |.sty| style file! Therefore it needs to be invoked in
a non-standard way.

%%%%%%%%%%%%%%%%%%%%%%%%%%%%%%%%%%%%%%%%%%%%%%%%%%%%%%%%%%%%%%%%%%%%%%%%%%%%%%%%
\subsection{Included Files}
\label{sec:include}

%%%%%%%%%%%%%%%%%%%%%%%%%%%%%%%%%%%%%%%%
\DescribeMacro{\childdocmain}
To use the package, add the commands
\begin{center}
\begin{tabular}{l}
|\input{childdoc.def}|\\
|\childdocmain{}|\\
\end{tabular}
\end{center}
at the very top of the main \LaTeX{} file,
in particular \emph{before} the |\documentclass| statement!
The argument of |\childdocmain| should be left empty
(but it must be present).

%%%%%%%%%%%%%%%%%%%%%%%%%%%%%%%%%%%%%%%%
\DescribeMacro{\childdocof}
Furthermore, add the commands
\begin{center}
\begin{tabular}{l}
|\input{childdoc.def}|\\
|\childdocof{|\textit{main}|}|\\
\end{tabular}
\end{center}
at the top of every child file \textit{child}
which is included by |\include{|\textit{child}|}|
from within the main file
(or at least for those files to be compiled individually).
The argument \textit{main} must be the filename of the main file.

There are a couple of
considerations in setting up the main and child documents:

%%%%%%%%%%%%%%%%%%%%%%%%%%%%%%%%%%%%%%%%
\paragraph{Restrictions.}

Please note the following restrictions:
\begin{itemize}
\item
|\childdocmain| must be called with one argument \textit{main}
to ensure compatibility with earlier version of the package.
It must either be empty (|\childdocmain{}|)
or precisely match the filename of the main file in which it is specified.
See \secref{sec:detection} for further information.
\item
The filename \textit{main} must be specified without the |.tex| extension.
\item
The filename \textit{main} is case sensitive
(even in case-insensitive file systems)
due to internal string comparison.
\item
The argument \textit{main} should be fully expanded, it cannot be a macro.
\item
Subdirectories and special characters should be avoided in filenames.
\item
The command |\childdocmain{|\textit{main}|}| must be followed by a whitespace.
It should not be followed immediately by another command
or by a comment mark `|%|'.
This is because the \TeX{} parser reads the token immediately following
the argument of |\childdocmain| and puts it
at the beginning of every child section;
however, a white\-space is ignored.
\end{itemize}

%%%%%%%%%%%%%%%%%%%%%%%%%%%%%%%%%%%%%%%%
\paragraph{Content of Main File.}

It is advisable to place all content in the child files included by |\include|.
Any output contained in the main file will appear in all child documents
unless suppressed manually;
it cannot be suppressed automatically by the |\includeonly| directive
and thus should normally be avoided.
A method to include some content in the main file
by means of conditional processing is described in \secref{sec:conditional}.

%%%%%%%%%%%%%%%%%%%%%%%%%%%%%%%%%%%%%%%%
\paragraph{Page Numbering.}

When only a part of the document is compiled,
the appropriate numbering of pages
(as well as other status parameters)
is determined from the |.aux| files.
The latter contain information from previous passes.
However this information needs to propagate through
all intermediate child documents.
Therefore the page numbering in child documents may well
be inconsistent until the complete document is compiled at least once.

A useful (if unconventional) way to always ensure a consistent
page numbering is to restart the numbering in each child document
and denote the pages by `\textit{child}|.|\textit{page}'
where \textit{child} represents the chapter/section number of the child file.
This can be achieved by the command
|\numberwithin{page}{|\textit{child}|}|
of the \textsf{amsmath} package
where \textit{child} can be |chapter| or |section|
depending on the chosen structuring.
Alternatively, one can modify the macro |\thepage| appropriately
and reset the counter |page| at the start of each child file.

%%%%%%%%%%%%%%%%%%%%%%%%%%%%%%%%%%%%%%%%%%%%%%%%%%%%%%%%%%%%%%%%%%%%%%%%%%%%%%%%
\subsection{Conditional Processing}
\label{sec:conditional}

The package provides a mechanism to compile different versions
of a document. To customise the versions further some conditional processing
can come in handy to distinguish which version is being compiled.
The package provides two macros to describe the compilation context:

%%%%%%%%%%%%%%%%%%%%%%%%%%%%%%%%%%%%%%%%
\DescribeMacro{\ifchilddoc}
The conditional |\ifchilddoc| distinguishes between the compilation of
child documents and the main document:
%
\begin{center}
|\ifchilddoc |\textit{child-code}| |[|\||else |\textit{main-code}]| \||fi|
\end{center}

%%%%%%%%%%%%%%%%%%%%%%%%%%%%%%%%%%%%%%%%
\DescribeMacro{\childdocname}
\DescribeMacro{\childdocjob}
The macro |\childdocname| contains the filename (without extension)
of the main or child file being processed.
Note that |\childdocjob| will always contain the name of the main file.

%%%%%%%%%%%%%%%%%%%%%%%%%%%%%%%%%%%%%%%%
\paragraph{Title Page.}

Conditional processing can be used to include a title or banner page
in the main document when proper precautions are taken.
Importantly, the code in the main file should ensure that the page counter
(as well as other status parameters which are stored in the |.aux| files)
takes the same value after the conditional processing.
Otherwise the page numbers may take divergent values
depending on which part is compiled.

For example, a title page could be declared by:
%
\begin{center}
\begin{tabular}{l}
|\ifchilddoc\||else|\\
|\addtocounter{page}{-1}|\\
\textit{code for title page}\\
|\newpage|\\
|\||fi|
\end{tabular}
\end{center}
%
A banner page for the child documents can be generated by:
%
\begin{center}
\begin{tabular}{l}
|\ifchilddoc|\\
|\addtocounter{page}{-1}|\\
\textit{code for banner page}\\
|\newpage|\\
|\||fi|
\end{tabular}
\end{center}
%
Here one could write a message such as:
\begin{center}
|This is the part \childdocname{} of \childdocjob{}.|
\end{center}

%%%%%%%%%%%%%%%%%%%%%%%%%%%%%%%%%%%%%%%%%%%%%%%%%%%%%%%%%%%%%%%%%%%%%%%%%%%%%%%%
\subsection{Flags}
\label{sec:flags}

The package makes it easy to generate different versions
of the main or child documents.
To this end compilation flags can be defined
and assigned different default values.
They will be particularly useful in conjunction
with the forwarding mechanism described in \secref{sec:forward}.

For example, it may be useful to have a flag |\version|
which can be set to |draft| or |final|.
The document source will contain some conditional code
depending on the value of |\version|.
Suppose further, the flag should default to |final| for the main file
and to |draft| for child files
which is a natural assignment for editing the document.
This is achieved by placing the following code
in the preamble of the main document
(below the |\childdocmain| directive):
%
\begin{center}
\begin{tabular}{l}
|\ifchilddoc|\\
|\providecommand{\version}{draft}|\\
|\||else|\\
|\providecommand{\version}{final}|\\
|\||fi|
\end{tabular}
\end{center}
%
The definition by |\providecommand| makes sure
that previous definitions are not overwritten.
Further statements |\providecommand{\version}{...}|
can thus be added before the above code to override it.

For the main file, one might add a line
(between |\childdocmain| and the above block)
%
\begin{center}
|%\ifchilddoc\||else\providecommand{\version}{draft}\||fi|
\end{center}
%
which can be uncommented to produce a draft version.
Likewise one can add a line to the very top of a child file
(above the |\childdocof{|\textit{main}|}| directive)
%
\begin{center}
|%\providecommand{\version}{final}|
\end{center}
%
which can be uncommented to produce the final version of this child document.

%%%%%%%%%%%%%%%%%%%%%%%%%%%%%%%%%%%%%%%%%%%%%%%%%%%%%%%%%%%%%%%%%%%%%%%%%%%%%%%%
\subsection{Forwarding}
\label{sec:forward}

Different versions of the main or child documents
using compilation flags as described in \secref{sec:flags}
can be (permanently) stored in different files
for convenient compilation, viewing and distribution.
To this end, the package defines a command
to pass on compilation to a different file:

%%%%%%%%%%%%%%%%%%%%%%%%%%%%%%%%%%%%%%%%
\DescribeMacro{\childdocforward}
The command |\childdocforward| redirects processing to
another source file:
%
\begin{center}
\begin{tabular}{l}
|\input{childdoc.def}|\\
|\childdocforward[|\textit{main}|]{|\textit{dest}|}|\\
\end{tabular}
\end{center}
%
The argument \textit{dest} is the destination file
(without extension).
It should be the main file or one of the child files.
Note that further \textsf{childdoc} directives
such as |\childdocof| and |\childdocforward|
in the indicated file will be processed in this form.
The optional argument \textit{main}
passes on directly to the main file \textit{main}
while pretending to compile the child \textit{dest}.
This form behaves as if \textit{dest}
issues |\childdocof{|\textit{main}|}| right away,
and no further \textsf{childdoc} directives will be processed.

%%%%%%%%%%%%%%%%%%%%%%%%%%%%%%%%%%%%%%%%
\DescribeMacro{\...prefix}
In the alternative form |\childdocforwardprefix|,
%
\begin{center}
\begin{tabular}{l}
|\input{childdoc.def}|\\
|\childdocforwardprefix[|\textit{main}|]{|\textit{prefix}|}{|\textit{dest}|}|
\end{tabular}
\end{center}
%
the destination file is determined by a pattern
depending on the current file:
To make this work, the current file must be called
`{\textit{prefix}\hspace{0.2em}\textit{suffix}}'
with \textit{prefix} matching precisely the argument.
Processing is then passed on to the file
`{\textit{dest}\hspace{0.2em}\textit{suffix}}'.
Surely, the same effect is achieved by
directly specifying the
argument `{\textit{dest}\hspace{0.2em}\textit{suffix}}'
in the first form.
However, that requires to set up a different file
for each child. With the alternative form of the command
all these files can have exactly the same content
which simplifies setting them up and maintaining them.

For example, the following file |draft.tex|
with a compilation flag |\version| as described in \secref{sec:flags}
compiles the main document as a draft:
%
\begin{center}
\begin{tabular}{l}
|\def\version{draft}|\\
|\input{childdoc.def}|\\
|\childdocforward{|\textit{main}|}|
\end{tabular}
\end{center}
%
Likewise, the following files |final|\textit{nn}|.tex|
compile the final version of the child document
|child|\textit{nn}|.tex|:
%
\begin{center}
\begin{tabular}{l}
|\def\version{final}|\\
|\input{childdoc.def}|\\
|\childdocforwardprefix{final}{child}|
\end{tabular}
\end{center}
%

Note that when several versions of a main file and/or of each child file
are to be generated, it may be convenient to set up a |Makefile| or
shell script to automatise the process.

%%%%%%%%%%%%%%%%%%%%%%%%%%%%%%%%%%%%%%%%%%%%%%%%%%%%%%%%%%%%%%%%%%%%%%%%%%%%%%%%
\subsection{Command Line Processing}
\label{sec:commandline}

The effect of redirection files can also be achieved by invoking
the \LaTeX{} compiler with a more elaborate command line.
Most conveniently this should be done as part
of a shell script or a |Makefile|.

When using \textsf{childdoc} in the main file, the following
command lines effectively perform a redirection
(note that depending on the shell being used,
backslashes may have to be doubled: `|\|' $\to$ `|\\|'):
%
\begin{center}
|... -jobname "|\textit{target}|" |\\|"|[\textit{flags}]%
|\input{childdoc.def}\childdocforward[|\textit{main}|]{|\textit{dest}|}"|
\end{center}
%
Here \textit{target} is the name of the output file,
\textit{main} is the name of the main file
and \textit{dest} is the name of the main or child file to be processed
(all filenames without extensions).
The optional argument \textit{main} can be omitted
if \textit{main} matches \textit{dest}.
Optionally, compilation \textit{flags} can be defined via |\def| commands.
This command line makes the \TeX{} engine believe
it is compiling the file \textit{target}
whose content is specified as the latter parameter.
The provided code then forwards the processing to
\textit{main} or \textit{dest} as described in \secref{sec:forward}.

%%%%%%%%%%%%%%%%%%%%%%%%%%%%%%%%%%%%%%%%%%%%%%%%%%%%%%%%%%%%%%%%%%%%%%%%%%%%%%%%
\subsection{Include by Input}
\label{sec:input}

Including child documents by |\include| has some restrictions by design.
Most notably, the content of a child document always occupies
its own set of pages; pages cannot be shared between child documents.
Usually, this behaviour makes perfect sense
because each child document contain an essential part of the document.
However, in some situations it may be desirable to compose
a document from a collection of parts
without having mandatory page breaks between then.
For this case, the package
provides a mechanism to include parts
by |\input| which can also be processed individually.
However, by construction this mechanism
requires manual handling of the content to be output.

%%%%%%%%%%%%%%%%%%%%%%%%%%%%%%%%%%%%%%%%
\DescribeMacro{\ifchilddocmanual}
The main file should be prepared as usual, see \secref{sec:include}.
However, the document body must make a distinction
between processing of an individual part and of the main document, e.g.:
%
\begin{center}
\begin{tabular}{l}
|\ifchilddocmanual|\\
|\input{\childdocname}|\\
|\||else|\\
\textit{document body with }|\input{|\textit{part}|}|\\
|\||fi|
\end{tabular}
\end{center}
%
The conditional |\ifchilddocmanual| is true whenever
a part to be included by |\input| is being compiled,
and the name of the part is stored in |\childdocname|.

%%%%%%%%%%%%%%%%%%%%%%%%%%%%%%%%%%%%%%%%
\DescribeMacro{\childdocby}
Each part to be included by |\input| should start with:
%
\begin{center}
\begin{tabular}{l}
|\input{childdoc.def}|\\
|\childdocby{|\textit{main}|}|\\
\end{tabular}
\end{center}
%
The directive |\childdocby| is similar to |\childdocof|
described in \secref{sec:include},
but the subsequent selection of content must be done manually.
To that end, both |\ifchilddoc| and |\ifchilddocmanual|
will be true upon processing of a part,
and the name of the part is stored in |\childdocname|.
Note that |\jobname| will be set to the filename of the current part
so that each part receives an individual |.aux| file
that does not interfere with the |.aux| file(s) of the main document.
This behaviour can be altered by the alternative form
|\childdocby[*]{|\textit{main}|}| (with a non-empty optional argument)
which uses the |.aux| file of the main document
by setting |\jobname| to \textit{main}.

%%%%%%%%%%%%%%%%%%%%%%%%%%%%%%%%%%%%%%%%%%%%%%%%%%%%%%%%%%%%%%%%%%%%%%%%%%%%%%%%
\subsection{Driver Development}
\label{sec:driver}

The \textsf{childdoc} mechanism can also be use for the development
of definition files such as \LaTeX{} styles or classes.
This case differs from the above setup with multiple parts
included by |\include| in that no |\includeonly| should be invoked.
This can be achieved by starting the include file
(before |\ProvidesPackage|) with:
%
\begin{center}
\begin{tabular}{l}
|\input{childdoc.def}|\\
|\childdocforward{|\textit{main}|}|\\
\end{tabular}
\end{center}
%
or alternatively with:
%
\begin{center}
\begin{tabular}{l}
|\input{childdoc.def}|\\
|\childdocby{|\textit{main}|}|\\
\end{tabular}
\end{center}
%
Both forms have slightly different effects as described above.
The main file is prepared as usual, see \secref{sec:include}.

%%%%%%%%%%%%%%%%%%%%%%%%%%%%%%%%%%%%%%%%%%%%%%%%%%%%%%%%%%%%%%%%%%%%%%%%%%%%%%%%
\subsection{Legacy Detection}
\label{sec:detection}

The directive |\childdocmain| in the main file can detect
whether the complete document or merely a child is to be compiled
even without using the directive |\childdocof|.
This method is deprecated because it is less robust
and there is no compelling reason to use it;
it is merely provided for backward compatibility
and it may be removed in future versions.

If the detection mechanism is to be used,
it is mandatory to correctly specify
the filename of the main file as the argument of |\childdocmain|:
%
\begin{center}
\begin{tabular}{l}
|\input{childdoc.def}|\\
|\childdocmain{|\textit{main}|}|\\
\end{tabular}
\end{center}
%
If |\jobname| does not match the argument \textit{main} of |\childdocmain|,
it is assumed that |\jobname| points to the child file to be compiled.
When using |\childdocmain| with the main file specified as argument,
it suffices to start a child file
with just |\input{|\textit{main}|}|
without loading of the package and using |\childdocof|.
If instead all processing is done
with the appropriate \textsf{childdoc} directives,
the argument of \textit{main} of |\childdocmain| can be empty.

An alternative version of the command line processing described
in \secref{sec:commandline} using the detection mechanism reads:
%
\begin{center}
|... -jobname "|\textit{target}|" "|[\textit{flags}]%
[|\def\jobname{|\textit{dest}|}|]|\input{|\textit{main}|}"|
\end{center}

%%%%%%%%%%%%%%%%%%%%%%%%%%%%%%%%%%%%%%%%%%%%%%%%%%%%%%%%%%%%%%%%%%%%%%%%%%%%%%%%
\subsection{Manual Code}
\label{sec:manual}

In case one cannot be certain whether the definitions file |childdoc.def|
is installed on the target \TeX{} distribution
and one prefers not to ship it,
it is conceivable to paste a few relevant commands into the sources.

To that end, drop all statements |\input{childdoc.def}|
and perform the replacements as outlined below.
Instead of |\childdocmain{|\textit{main}|}| add the following code
to the top of the main file:
%
\begin{center}
\begin{tabular}{l}
|\||ifdefined\childdocname\endinput\||fi\newif\ifchilddoc|\\
|\edef\childdocname{\scantokens\expandafter{\jobname\noexpand}}|\\
|\def\childdocmain{|\textit{main}|}\||ifx\childdocmain\childdocname\||else|\\
|\childdoctrue\includeonly{\childdocname}\let\jobname\childdocmain\||fi|\\
\end{tabular}
\end{center}
%
Instead of |\childdocof{|\textit{main}|}| just include the main file
at the top of each child file:
%
\begin{center}
|\input{|\textit{main}|}|
\end{center}
%
A simple redirection |\childdocforward{|\textit{dest}|}| is achieved by:
%
\begin{center}
|\def\jobname{|\textit{dest}|}\input{\jobname}|
\end{center}
%
The redirection with prefix
|\childdocforwardprefix[|\textit{prefix}|]{|\textit{dest}|}|
is accomplished by:
%
\begin{center}
\begin{tabular}{l}
|{\edef\jobname{\scantokens\expandafter{\jobname\noexpand}}|\\
|\def\redirectjob |\textit{prefix}|#1~~~{\gdef\jobname{|\textit{dest}|#1}}|\\
|\expandafter\redirectjob\jobname~~~}\input{\jobname}|
\end{tabular}
\end{center}

In an alternative approach,
child documents can be compiled by a specific command line
without additional code or specific definitions:
%
\begin{center}
|... -jobname "|\textit{target}|" "|[\textit{flags}]%
|\includeonly{|\textit{dest}|}\input{|\textit{main}|}"|
\end{center}
%

%%%%%%%%%%%%%%%%%%%%%%%%%%%%%%%%%%%%%%%%%%%%%%%%%%%%%%%%%%%%%%%%%%%%%%%%%%%%%%%%
%%%%%%%%%%%%%%%%%%%%%%%%%%%%%%%%%%%%%%%%%%%%%%%%%%%%%%%%%%%%%%%%%%%%%%%%%%%%%%%%
\section{Information}

%%%%%%%%%%%%%%%%%%%%%%%%%%%%%%%%%%%%%%%%%%%%%%%%%%%%%%%%%%%%%%%%%%%%%%%%%%%%%%%%
\subsection{Copyright}

Copyright \copyright{} 2017--2018 Niklas Beisert

This work may be distributed and/or modified under the
conditions of the \LaTeX{} Project Public License, either version 1.3
of this license or (at your option) any later version.
The latest version of this license is in
  \url{http://www.latex-project.org/lppl.txt}
and version 1.3 or later is part of all distributions of \LaTeX{}
version 2005/12/01 or later.

This work has the LPPL maintenance status `maintained'.

The Current Maintainer of this work is Niklas Beisert.

This work consists of the files |README.txt|, |childdoc.ins| and |childdoc.dtx|
as well as the derived files |childdoc.def|, |cdocsamp.tex|
with |cdocsch1.tex|, |cdocsch2.tex|, |cdocspt3.tex|, |cdocspt4.tex|,
|cdocsdrf.tex|, |cdocsfn1.tex|, |cdocsfn2.tex|
as well as |childdoc.pdf|.

%%%%%%%%%%%%%%%%%%%%%%%%%%%%%%%%%%%%%%%%%%%%%%%%%%%%%%%%%%%%%%%%%%%%%%%%%%%%%%%%
\subsection{Files and Installation}

The package consists of the files:
%
\begin{center}
\begin{tabular}{ll}
    |README.txt|   & readme file \\
    |childdoc.ins| & installation file \\
    |childdoc.dtx| & source file \\
    |childdoc.def| & definition file \\
    |cdocsamp.tex| & sample main file \\
    |cdocsch1.tex| & sample include file \\
    |cdocsch2.tex| & sample include file \\
    |cdocspt3.tex| & sample part file \\
    |cdocspt4.tex| & sample part file \\
    |cdocsdrf.tex| & sample redirection file \\
    |cdocsfn1.tex| & sample redirection file \\
    |cdocsfn2.tex| & sample redirection file \\
    |childdoc.pdf| & manual
\end{tabular}
\end{center}
%
The distribution consists of the files
|README.txt|, |childdoc.ins| and |childdoc.dtx|.
%
\begin{itemize}
\item
Run (pdf)\LaTeX{} on |childdoc.dtx|
to compile the manual |childdoc.pdf| (this file).
\item
Run \LaTeX{} on |childdoc.ins| to create the definitions file |childdoc.def|
and the sample |cdocsamp.tex| with include files
|cdocsch1.tex|, |cdocsch2.tex|, |cdocspt3.tex|, |cdocspt4.tex|,
|cdocsdrf.tex|, |cdocsfn1.tex|, |cdocsfn2.tex|.
Then copy the file |childdoc.def| to an appropriate directory of your \LaTeX{}
distribution, e.g.\ \textit{texmf-root}|/tex/latex/childdoc|.
\end{itemize}

%%%%%%%%%%%%%%%%%%%%%%%%%%%%%%%%%%%%%%%%%%%%%%%%%%%%%%%%%%%%%%%%%%%%%%%%%%%%%%%%
\subsection{Related CTAN Packages}

There are several other packages which offer a similar functionality:
%
\begin{itemize}
\item
The packages
\href{http://ctan.org/pkg/docmute}{\textsf{docmute}},
\href{http://ctan.org/pkg/includex}{\textsf{includex}} and
\href{http://ctan.org/pkg/standalone}{\textsf{standalone}}
provide commands to include only the document body of
a child file thus allowing both files to be compiled individually.
\item
The packages \href{http://ctan.org/pkg/subdocs}{\textsf{subdocs}}
and \href{http://ctan.org/pkg/subfiles}{\textsf{subfiles}}
provide structures in which the main and child documents can be
encapsulated and allowing them to be compiled individually.
The inclusion mechanism is different from the conventional |\include|.
\item
The package \href{http://ctan.org/pkg/combine}{\textsf{combine}}
is an elaborate solution to combine several documents into one.
\end{itemize}
%
See also the CTAN topic \href{http://ctan.org/topic/subdocs}{\textsf{subdocs}}
for further related packages.
The present package differs from the above solutions in that
a document structure constructed with the conventional |\include| mechanism
just needs two extra commands at the top of every file
such that all constituent files can be compiled individually.

%%%%%%%%%%%%%%%%%%%%%%%%%%%%%%%%%%%%%%%%%%%%%%%%%%%%%%%%%%%%%%%%%%%%%%%%%%%%%%%%
%\subsection{Feature Suggestions}
%
%The following is a list of features which may be useful for future
%versions of this package:
%%
%\begin{itemize}
%\item
%\ldots
%\end{itemize}

%%%%%%%%%%%%%%%%%%%%%%%%%%%%%%%%%%%%%%%%%%%%%%%%%%%%%%%%%%%%%%%%%%%%%%%%%%%%%%%%
\subsection{Revision History}

%%%%%%%%%%%%%%%%%%%%%%%%%%%%%%%%%%%%%%%%
\paragraph{v2.0:} 2018/12/30

\begin{itemize}
\item
immediate forward processing
\item
added |\childdocby| mechanism
\item
manual restructured
\end{itemize}

%%%%%%%%%%%%%%%%%%%%%%%%%%%%%%%%%%%%%%%%
\paragraph{v1.6:} 2018/01/17

\begin{itemize}
\item
application for development of include files
\item
corrections to manual
\end{itemize}

%%%%%%%%%%%%%%%%%%%%%%%%%%%%%%%%%%%%%%%%
\paragraph{v1.5:} 2017/05/21

\begin{itemize}
\item
more complete structuring introduced
\item
|\childdocof| introduced
\item
|\childdoc| renamed to |\childdocmain|
\item
|\childredirect| renamed to |\childdocforward| and |\childdocforwardprefix|
and functionality expanded
\end{itemize}

%%%%%%%%%%%%%%%%%%%%%%%%%%%%%%%%%%%%%%%%
\paragraph{v1.0:} 2017/04/27

\begin{itemize}
\item
manual and install package
\item
first version published on CTAN
\end{itemize}

%%%%%%%%%%%%%%%%%%%%%%%%%%%%%%%%%%%%%%%%
\paragraph{v0.6:} 2017/04/26

\begin{itemize}
\item
redirection mechanism added
\end{itemize}

%%%%%%%%%%%%%%%%%%%%%%%%%%%%%%%%%%%%%%%%
\paragraph{v0.5:} 2017/04/26

\begin{itemize}
\item
functionality in definition file
\end{itemize}


%%%%%%%%%%%%%%%%%%%%%%%%%%%%%%%%%%%%%%%%%%%%%%%%%%%%%%%%%%%%%%%%%%%%%%%%%%%%%%%%
%%%%%%%%%%%%%%%%%%%%%%%%%%%%%%%%%%%%%%%%%%%%%%%%%%%%%%%%%%%%%%%%%%%%%%%%%%%%%%%%
%%%%%%%%%%%%%%%%%%%%%%%%%%%%%%%%%%%%%%%%%%%%%%%%%%%%%%%%%%%%%%%%%%%%%%%%%%%%%%%%
\appendix

\settowidth\MacroIndent{\rmfamily\scriptsize 000\ }

 \DocInput{childdoc.dtx}

\end{document}
%</driver>
% \fi
%
% %%%%%%%%%%%%%%%%%%%%%%%%%%%%%%%%%%%%%%%%%%%%%%%%%%%%%%%%%%%%%%%%%%%%%%%%%%%%%%
% %%%%%%%%%%%%%%%%%%%%%%%%%%%%%%%%%%%%%%%%%%%%%%%%%%%%%%%%%%%%%%%%%%%%%%%%%%%%%%
% \section{Sample}
%\iffalse
%<*samplemain>
%\fi
%
% The following presents a sample document
% with two chapters, two parts, a title page,
% a compile flag as well as three forwarding files to set the flag.
% It consists of eight |.tex| files:
% \begin{center}
% \begin{tabular}{ll}
% |cdocsamp.tex|&main file\\
% |cdocsch1.tex|&include file for chapter 1\\
% |cdocsch2.tex|&include file for chapter 2\\
% |cdocspt3.tex|&include file for part 3\\
% |cdocspt4.tex|&include file for part 4\\
% |cdocsdrf.tex|&forwarding file for main file in draft mode\\
% |cdocsfi1.tex|&forwarding file for final version of chapter 1\\
% |cdocsfi2.tex|&forwarding file for final version of chapter 2\\
% \end{tabular}
% \end{center}
% Each of the eight files can be compiled directly by the \LaTeX{} compiler.
%
% %%%%%%%%%%%%%%%%%%%%%%%%%%%%%%%%%%%%%%
% \paragraph{Main File.}
%
% The main file is called |cdocsamp.tex|.
%
% Load the \textsf{childdoc} definitions and
% declare the filename for the main document:
%    \begin{macrocode}
\input{childdoc.def}
\childdocmain{}
%    \end{macrocode}

% Optional override for |\version| flag:
%    \begin{macrocode}
%%\ifchilddoc\else\providecommand{\version}{draft}\fi
%    \end{macrocode}

% Define the default values for the |\version| flag
% (|final| for the main file and |draft| for childs):
%    \begin{macrocode}
\ifchilddoc
\providecommand{\version}{draft}
\else
\providecommand{\version}{final}
\fi
%    \end{macrocode}

% Load the standard document class:
%    \begin{macrocode}
\documentclass[12pt]{article}
%    \end{macrocode}

% Start the document body:
%    \begin{macrocode}
\begin{document}
%    \end{macrocode}

% Declare a title page.
% Print title, part of document being processed and version flag:
%    \begin{macrocode}
\addtocounter{page}{-1}
\begin{center}
{\LARGE\bfseries{}childdoc example\par}
\vspace{1cm}
\ifchilddoc
\ifchilddocmanual part\else chapter\fi:
`\childdocname' of `\childdocjob'\par
\else
main document: `\childdocjob'\par
\fi
version: \version\par
\end{center}
\newpage
%    \end{macrocode}

% Manually include selected file,
% otherwise process as usual:
%    \begin{macrocode}
\ifchilddocmanual
\section*{part `\childdocname'}
\input{\childdocname}
\else
%    \end{macrocode}

% Include the two chapters:
%    \begin{macrocode}
\include{cdocsch1}
\include{cdocsch2}
%    \end{macrocode}

% Include the two parts unless only chapters should be displayed:
%    \begin{macrocode}
\ifchilddoc\else
\section{part three}
\input{cdocspt3}
\section{part four}
\input{cdocspt4}
\fi
%    \end{macrocode}

% Process as usual until here:
%    \begin{macrocode}
\fi
%    \end{macrocode}

% End of document body:
%    \begin{macrocode}
\end{document}
%    \end{macrocode}
%\iffalse
%</samplemain>
%\fi
%
% %%%%%%%%%%%%%%%%%%%%%%%%%%%%%%%%%%%%%%
% \paragraph{Chapter Include Files.}
%
% The include files are called |cdocsch1.tex| and |cdocsch2.tex|.
%
%\iffalse
%<*samplechap1|samplechap2>
%\fi

% Optional override for |\version| flag:
%    \begin{macrocode}
%%\providecommand{\version}{final}
%    \end{macrocode}

% Include the main document:
%    \begin{macrocode}
\input{childdoc.def}
\childdocof{cdocsamp}
%    \end{macrocode}

%\iffalse
%</samplechap1|samplechap2>
%\fi
%
%\iffalse
%<*samplechap1>
%\fi
% Some text for chapter 1:
%    \begin{macrocode}
\section{one}
some text in chapter one
%    \end{macrocode}

%\iffalse
%</samplechap1>
%\fi
% Some text for chapter 2:
%\iffalse
%<*samplechap2>
%\fi
%    \begin{macrocode}
\section{two}
more text in chapter two
%    \end{macrocode}

%\iffalse
%</samplechap2>
%\fi
%
% %%%%%%%%%%%%%%%%%%%%%%%%%%%%%%%%%%%%%%
% \paragraph{Part Include Files.}
%
% The include files are called |cdocspt3.tex| and |cdocspt4.tex|.
%
%\iffalse
%<*samplepart3|samplepart4>
%\fi

% Optional override for |\version| flag:
%    \begin{macrocode}
%%\providecommand{\version}{final}
%    \end{macrocode}

% Include the main document:
%    \begin{macrocode}
\input{childdoc.def}
\childdocby{cdocsamp}
%    \end{macrocode}

%\iffalse
%</samplepart3|samplepart4>
%\fi
%
%\iffalse
%<*samplepart3>
%\fi
% Some text for part 3:
%    \begin{macrocode}
some text in part three
%    \end{macrocode}

%\iffalse
%</samplepart3>
%\fi
% Some text for part 4:
%\iffalse
%<*samplepart4>
%\fi
%    \begin{macrocode}
more text in part four
%    \end{macrocode}

%\iffalse
%</samplepart4>
%\fi
%
% %%%%%%%%%%%%%%%%%%%%%%%%%%%%%%%%%%%%%%
% \paragraph{Forwarding for a Complete Draft.}
%
% The following forwarding file |cdocsdrf.tex|
% compiles the main document in draft mode:
%\iffalse
%<*sampledraft>
%\fi
%    \begin{macrocode}
\def\version{draft}
\input{childdoc.def}
\childdocforward{cdocsamp}
%    \end{macrocode}

%\iffalse
%</sampledraft>
%\fi
%
% %%%%%%%%%%%%%%%%%%%%%%%%%%%%%%%%%%%%%%
% \paragraph{Forwarding for Final Version of the Chapters.}
%
% The following forwarding files |cdocsfn1.tex| and |cdocsfn2.tex|
% (with identical content)
% compile the final versions of the child documents
% |cdocsch1.tex| and |cdocsch2.tex|, respectively:
%\iffalse
%<*samplefinal>
%\fi
%    \begin{macrocode}
\def\version{final}
\input{childdoc.def}
\childdocforwardprefix[cdocsamp]{cdocsfn}{cdocsch}
%    \end{macrocode}

%\iffalse
%</samplefinal>
%\fi
%
% %%%%%%%%%%%%%%%%%%%%%%%%%%%%%%%%%%%%%%
% \paragraph{Command Line Processing.}
%
% The following three command lines generate the output files
% |cdocscld|, |cdocscl1| and |cdocscl2|
% which should be identical to
% |cdocsdrf|, |cdocsch1| and |cdocsfn2|, respectively:
% \begin{center}
% \begin{tabular}{l}
% |latex -jobname cdocscld \|\\
% |  "\def\version{draft}\input{childdoc.def}\childdocforward{cdocsamp}"|\\
% |latex -jobname cdocscl1 \|\\
% |  "\input{childdoc.def}\childdocforward[cdocsamp]{cdocsch1}"|\\
% |latex -jobname cdocscl2 \|\\
% |  "\def\version{final}\input{childdoc.def}\childdocforward{cdocsch2}"|
% \end{tabular}
% \end{center}
% Note that the trailing backslash on each first line
% merely continues the input to the second line
% (for convenient cut ant paste).
% Furthermore, the command |latex| can be replaced by any
% of its alternative versions such as |pdflatex|.
%
% %%%%%%%%%%%%%%%%%%%%%%%%%%%%%%%%%%%%%%%%%%%%%%%%%%%%%%%%%%%%%%%%%%%%%%%%%%%%%%
% %%%%%%%%%%%%%%%%%%%%%%%%%%%%%%%%%%%%%%%%%%%%%%%%%%%%%%%%%%%%%%%%%%%%%%%%%%%%%%
% \section{Implementation}
%\iffalse
%<*package>
%\fi
%
% This section describes the definitions file |childdoc.def|.

% The definitions cannot be loaded using |\usepackage| or |\RequirePackage|
% which has a mechanism to prevent loading a style file more than once.
% When loading the definitions by means of |\input|
% multiple instances have to be prevented manually:
%\iffalse
%This code needs to be before the `\ProvidesFile' directive
%which is defined at the beginning of this file.
%Therefore it is also placed there and commented out here.
%</package>
%<*discard>
%\fi
%    \begin{macrocode}
\ifdefined\childdocmain\endinput\fi
%    \end{macrocode}
%\iffalse
%</discard>
%<*package>
%\fi
%
% \macro{\ifchilddoc}
% \macro{\ifchilddocmanual}
% The conditional |\ifchilddoc| tells whether a
% child (true) or main (false) document is being compiled.
% The conditional |\ifchilddocmanual| tells whether
% the |\includeonly| mechanism is used (false) or
% the selection of child files must be performed manually (true).
% The definitions initialise to false:
%    \begin{macrocode}
\newif\ifchilddoc
\newif\ifchilddocmanual
%    \end{macrocode}

% \macro{\childdocname}
% \macro{\childdocjob}
% The macro |\childdocname| stores the name of the main document
% to be compiled. The macro |\childdocjob| stores the name of
% the document on which the \LaTeX{} compiler was originally invoked.
% The content of |\jobname| cannot be compared
% to filenames specified in the source due to different catcodes.
% The following code rescans |\jobname|, stores the result
% in |\childdocname| and saves a copy in |\childdocjob|:
%    \begin{macrocode}
\edef\childdocname{\scantokens\expandafter{\jobname\noexpand}}
\let\childdocjob\childdocname
%    \end{macrocode}

% \macro{\childdocdisable}
% The macro |\childdocdisable| prevents the main file
% from being processed more than once.
% At this stage, the main document command |\childdocmain|
% is assumed to be called once again where it should do nothing.
% Any subsequent call to it should prevent
% a secondary processing of the main document
% It overwrites the forwarding commands
% |\childdocof| and |\childdocforward|
% with empty macros to prevent further inclusions of the main document:
%    \begin{macrocode}
\newcommand{\childdocdisable}
{
  \renewcommand{\childdocmain}[1]{\renewcommand{\childdocmain}[1]{\endinput}}
  \renewcommand{\childdocof}[1]{}
  \renewcommand{\childdocby}[2][]{}
  \renewcommand{\childdocforward}[2][]{}
  \renewcommand{\childdocdisable}{}
}
%    \end{macrocode}

% \macro{\childdocmain}
% The macro |\childdocmain| is to be called at the top of the main file
% with nothing or the main filename (without extension) as argument.
% First, it breaks loops.
% If the argument is not empty and does not match |\childdocname|
% (which is set by the first inclusion of |childdoc.def|),
% |\ifchilddoc| is set to true, |\includeonly| is applied to the child file
% and |\jobname| is set to the main file
% (for proper handling of |.aux| files):
%    \begin{macrocode}
\newcommand{\childdocmain}[1]
{
  \childdocdisable\childdocmain{}
  \if?#1?\else
    \begingroup
      \def\childdoctmp{#1}
      \ifx\childdoctmp\childdocname
        \def\childdoctmp{}
      \else
        \def\childdoctmp
        {
          \childdoctrue
          \includeonly{\childdocname}
          \def\childdocjob{#1}
          \def\jobname{#1}
        }
      \fi
      \expandafter
    \endgroup
    \childdoctmp
  \fi
}
%    \end{macrocode}

% \macro{\childdocof}
% The command |\childdocof| redirects
% compilation to the main file |#1|.
%    \begin{macrocode}
\newcommand{\childdocof}[1]
{
  \childdocdisable
  \childdoctrue
  \includeonly{\childdocname}
  \def\jobname{#1}
  \def\childdocjob{#1}
  \input{#1}
}
%    \end{macrocode}

% \macro{\childdocby}
% The command |\childdocby| ....
%    \begin{macrocode}
\newcommand{\childdocby}[2][]
{
  \childdocdisable
  \childdoctrue
  \childdocmanualtrue
  \if?#1?\else
    \def\jobname{#2}
  \fi
  \def\childdocjob{#2}
  \input{#2}
  \endinput
}
%    \end{macrocode}

% \macro{\childdocforward}
% The command |\childdocforward| redirects
% compilation to the main file or
% (if the optional argument is given) a child file.
% Parameters are set as if the main file
% or a child file starting with |\childdocof| was compiled.
% Then compilation is handed over to the main file:
%    \begin{macrocode}
\newcommand{\childdocforward}[2][]
{
  \begingroup
    \if?#1?
      \def\childdoctmp
      {
        \def\childdocname{#2}
        \def\childdocjob{#2}
        \def\jobname{#2}
        \input{#2}
        \endinput
      }
    \else
      \def\childdoctmp
      {
        \childdocdisable
        \def\childdocname{#2}
        \childdoctrue
        \includeonly{#2}
        \def\childdocjob{#1}
        \def\jobname{#1}
        \input{#1}
        \endinput
      }
    \fi
    \expandafter
  \endgroup
  \childdoctmp
}
%    \end{macrocode}

% \macro{\childdocforwardprefix}
% The command |\childdocforwardprefix| redirects
% compilation to the main or a child file by means of a pattern.
% The prefix |#1| in the current filename is replaced by |#2|
% and the suffix of the current filename is kept
% (it is assumed that the filename does not contain the substring `|~~~|'
% which is used as a delimiter).
% Compilation is handed over to the new file by |\childdocforward|:
%    \begin{macrocode}
\newcommand{\childdocforwardprefix}[3][]
{
  \begingroup
    \def\childdocextract #2##1~~~{\def\childdoctmp{\childdocforward[#1]{#3##1}}}
    \expandafter\childdocextract\childdocname~~~
    \expandafter
  \endgroup
  \childdoctmp
}
%    \end{macrocode}

% \macro{\childdoc}
% The deprecated macro |\childdoc| is a legacy version of |\childdocmain|:
%    \begin{macrocode}
\newcommand{\childdoc}{\childdocmain}
%    \end{macrocode}

% \macro{\childdocredirect}
% The deprecated macro |\childdocredirect| is a legacy version
% of |\childdocforward| and |\childdocforwardprefix|:
%    \begin{macrocode}
\newcommand{\childdocredirect}[2][]
{
  \begingroup
    \if?#1?
      \def\childdoctmp{\childdocforward{#2}}
    \else
      \def\childdoctmp{\childdocforwardprefix{#1}{#2}}
    \fi
    \expandafter
  \endgroup
  \childdoctmp
}
%    \end{macrocode}

%\iffalse
%</package>
%\fi
%
\endinput
\childdocforward{cdocsamp}"|\\
% |latex -jobname cdocscl1 \|\\
% |  "% \iffalse
%
% childdoc.dtx Copyright (C) 2017-2018 Niklas Beisert
%
% This work may be distributed and/or modified under the
% conditions of the LaTeX Project Public License, either version 1.3
% of this license or (at your option) any later version.
% The latest version of this license is in
%   http://www.latex-project.org/lppl.txt
% and version 1.3 or later is part of all distributions of LaTeX
% version 2005/12/01 or later.
%
% This work has the LPPL maintenance status `maintained'.
%
% The Current Maintainer of this work is Niklas Beisert.
%
% This work consists of the files childdoc.dtx and childdoc.ins
% and the derived files childdoc.def and cdocsamp.tex with
% cdocsch1.tex, cdocsch2.tex, cdocsdrf.tex, cdocsfn1.tex, cdocsfn2.tex.
%
%<package>\ifdefined\childdocmain\endinput\fi
%<package>\ProvidesFile{childdoc.def}[2018/12/30 v2.0 child document driver]
%<samplemain>\ProvidesFile{cdocsamp.tex}[2018/12/30 v2.0 sample for childdoc]
%<*driver>
%\ProvidesFile{childdoc.drv}[2018/12/30 v2.0 childdoc reference manual file]
\PassOptionsToClass{10pt,a4paper}{article}
\documentclass{ltxdoc}

\usepackage[margin=35mm]{geometry}
\usepackage{hyperref}
\usepackage{hyperxmp}
\usepackage[usenames]{color}

\hypersetup{colorlinks=true}
\hypersetup{pdfstartview=FitH}
\hypersetup{pdfpagemode=UseNone}
\hypersetup{pdfsource={}}
\hypersetup{pdflang={en-UK}}
\hypersetup{pdfcopyright={Copyright 2017-2018 Niklas Beisert.
  This work may be distributed and/or modified under the
  conditions of the LaTeX Project Public License, either version 1.3
  of this license or (at your option) any later version.}}
\hypersetup{pdflicenseurl={http://www.latex-project.org/lppl.txt}}
\hypersetup{pdfcontactaddress={ETH Zurich, ITP, HIT K,
  Wolfgang-Pauli-Strasse 27}}
\hypersetup{pdfcontactpostcode={8093}}
\hypersetup{pdfcontactcity={Zurich}}
\hypersetup{pdfcontactcountry={Switzerland}}
\hypersetup{pdfcontactemail={nbeisert@itp.phys.ethz.ch}}
\hypersetup{pdfcontacturl={http://people.phys.ethz.ch/\xmptilde nbeisert/}}

\newcommand{\secref}[1]{\hyperref[#1]{section \ref*{#1}}}

\parskip1ex
\parindent0pt
\let\olditemize\itemize
\def\itemize{\olditemize\parskip0pt}

\begin{document}

\title{The \textsf{childdoc} Package}
\hypersetup{pdftitle={The childdoc Package}}
\author{Niklas Beisert\\[2ex]
  Institut f\"ur Theoretische Physik\\
  Eidgen\"ossische Technische Hochschule Z\"urich\\
  Wolfgang-Pauli-Strasse 27, 8093 Z\"urich, Switzerland\\[1ex]
  \href{mailto:nbeisert@itp.phys.ethz.ch}
  {\texttt{nbeisert@itp.phys.ethz.ch}}}
\hypersetup{pdfauthor={Niklas Beisert}}
\hypersetup{pdfsubject={Manual for the LaTeX2e Package childdoc}}
\date{30 December 2018, \textsf{v2.0}}
\maketitle

\begin{abstract}\noindent
\textsf{childdoc} is a \LaTeXe{} package
that enables the direct compilation
of document sections included by |\include|
to individual files.
\end{abstract}

\begingroup
\parskip0ex
\tableofcontents
\endgroup

%%%%%%%%%%%%%%%%%%%%%%%%%%%%%%%%%%%%%%%%%%%%%%%%%%%%%%%%%%%%%%%%%%%%%%%%%%%%%%%%
%%%%%%%%%%%%%%%%%%%%%%%%%%%%%%%%%%%%%%%%%%%%%%%%%%%%%%%%%%%%%%%%%%%%%%%%%%%%%%%%
\section{Introduction}

\LaTeX{} provides a mechanism to structure a large document (such as a book)
into a main file and several child files (containing the chapters)
using the |\include| command.
This mechanism is beneficial for documents
which span hundreds of pages in order to
make the source file(s) more manageable.
Moreover, compilation can be restricted to
selected child files by means of the |\includeonly| command.
The latter feature can be used to reduce the compilation time while editing
(this was significantly more useful in the earlier days of \LaTeX{})
or to generate a smaller document which is easier to navigate.
Another application of |\includeonly| is to generate
documents consisting of selected parts of the complete document.

However, there are a few drawbacks of the plain |\include| mechanism:
\begin{itemize}
\item
The child files cannot be compiled on their own,
they can only be compiled via the main file.
A naive editing environment
(such as a text editor with an option
to have the current file processed by \LaTeX)
may require one to switch to the main file before compiling;
attempting to compile the child file produces errors.
\item
The main file must be modified (each time)
to adjust the |\includeonly| command
to the present needs. This easily leaves the main file in a messy state.
\item
The generated document will always carry the filename
of the main document. This is inconvenient if
several child files are to be compiled and
to be kept for distribution.
\end{itemize}

The present package provides a simple interface
to make child files individually compilable by \LaTeX{}.
Compiling a child file then has the same effect as compiling
the main file with an |\includeonly| command
to select the appropriate child.
Moreover the generated document will carry the name of the child
rather than the main file.
This resolves all three above issues.

This feature is meant to make the editing of books,
thesis documents and lecture notes somewhat more convenient.
However, the package can also be used efficiently for
composing a series of documents (such as exercise sheets)
which are typically distributed individually.
It then assists the author in generating the individual documents
(potentially in different versions)
as well as a document containing the collected series.
Another application is in developing style files
or other kinds of included material
where compilation of the style file could redirect
to a sample or test file.

%%%%%%%%%%%%%%%%%%%%%%%%%%%%%%%%%%%%%%%%%%%%%%%%%%%%%%%%%%%%%%%%%%%%%%%%%%%%%%%%
%%%%%%%%%%%%%%%%%%%%%%%%%%%%%%%%%%%%%%%%%%%%%%%%%%%%%%%%%%%%%%%%%%%%%%%%%%%%%%%%
\section{Usage}

First of all, the package \textsf{childdoc} is \emph{not} a standard
\LaTeXe{} |.sty| style file! Therefore it needs to be invoked in
a non-standard way.

%%%%%%%%%%%%%%%%%%%%%%%%%%%%%%%%%%%%%%%%%%%%%%%%%%%%%%%%%%%%%%%%%%%%%%%%%%%%%%%%
\subsection{Included Files}
\label{sec:include}

%%%%%%%%%%%%%%%%%%%%%%%%%%%%%%%%%%%%%%%%
\DescribeMacro{\childdocmain}
To use the package, add the commands
\begin{center}
\begin{tabular}{l}
|\input{childdoc.def}|\\
|\childdocmain{}|\\
\end{tabular}
\end{center}
at the very top of the main \LaTeX{} file,
in particular \emph{before} the |\documentclass| statement!
The argument of |\childdocmain| should be left empty
(but it must be present).

%%%%%%%%%%%%%%%%%%%%%%%%%%%%%%%%%%%%%%%%
\DescribeMacro{\childdocof}
Furthermore, add the commands
\begin{center}
\begin{tabular}{l}
|\input{childdoc.def}|\\
|\childdocof{|\textit{main}|}|\\
\end{tabular}
\end{center}
at the top of every child file \textit{child}
which is included by |\include{|\textit{child}|}|
from within the main file
(or at least for those files to be compiled individually).
The argument \textit{main} must be the filename of the main file.

There are a couple of
considerations in setting up the main and child documents:

%%%%%%%%%%%%%%%%%%%%%%%%%%%%%%%%%%%%%%%%
\paragraph{Restrictions.}

Please note the following restrictions:
\begin{itemize}
\item
|\childdocmain| must be called with one argument \textit{main}
to ensure compatibility with earlier version of the package.
It must either be empty (|\childdocmain{}|)
or precisely match the filename of the main file in which it is specified.
See \secref{sec:detection} for further information.
\item
The filename \textit{main} must be specified without the |.tex| extension.
\item
The filename \textit{main} is case sensitive
(even in case-insensitive file systems)
due to internal string comparison.
\item
The argument \textit{main} should be fully expanded, it cannot be a macro.
\item
Subdirectories and special characters should be avoided in filenames.
\item
The command |\childdocmain{|\textit{main}|}| must be followed by a whitespace.
It should not be followed immediately by another command
or by a comment mark `|%|'.
This is because the \TeX{} parser reads the token immediately following
the argument of |\childdocmain| and puts it
at the beginning of every child section;
however, a white\-space is ignored.
\end{itemize}

%%%%%%%%%%%%%%%%%%%%%%%%%%%%%%%%%%%%%%%%
\paragraph{Content of Main File.}

It is advisable to place all content in the child files included by |\include|.
Any output contained in the main file will appear in all child documents
unless suppressed manually;
it cannot be suppressed automatically by the |\includeonly| directive
and thus should normally be avoided.
A method to include some content in the main file
by means of conditional processing is described in \secref{sec:conditional}.

%%%%%%%%%%%%%%%%%%%%%%%%%%%%%%%%%%%%%%%%
\paragraph{Page Numbering.}

When only a part of the document is compiled,
the appropriate numbering of pages
(as well as other status parameters)
is determined from the |.aux| files.
The latter contain information from previous passes.
However this information needs to propagate through
all intermediate child documents.
Therefore the page numbering in child documents may well
be inconsistent until the complete document is compiled at least once.

A useful (if unconventional) way to always ensure a consistent
page numbering is to restart the numbering in each child document
and denote the pages by `\textit{child}|.|\textit{page}'
where \textit{child} represents the chapter/section number of the child file.
This can be achieved by the command
|\numberwithin{page}{|\textit{child}|}|
of the \textsf{amsmath} package
where \textit{child} can be |chapter| or |section|
depending on the chosen structuring.
Alternatively, one can modify the macro |\thepage| appropriately
and reset the counter |page| at the start of each child file.

%%%%%%%%%%%%%%%%%%%%%%%%%%%%%%%%%%%%%%%%%%%%%%%%%%%%%%%%%%%%%%%%%%%%%%%%%%%%%%%%
\subsection{Conditional Processing}
\label{sec:conditional}

The package provides a mechanism to compile different versions
of a document. To customise the versions further some conditional processing
can come in handy to distinguish which version is being compiled.
The package provides two macros to describe the compilation context:

%%%%%%%%%%%%%%%%%%%%%%%%%%%%%%%%%%%%%%%%
\DescribeMacro{\ifchilddoc}
The conditional |\ifchilddoc| distinguishes between the compilation of
child documents and the main document:
%
\begin{center}
|\ifchilddoc |\textit{child-code}| |[|\||else |\textit{main-code}]| \||fi|
\end{center}

%%%%%%%%%%%%%%%%%%%%%%%%%%%%%%%%%%%%%%%%
\DescribeMacro{\childdocname}
\DescribeMacro{\childdocjob}
The macro |\childdocname| contains the filename (without extension)
of the main or child file being processed.
Note that |\childdocjob| will always contain the name of the main file.

%%%%%%%%%%%%%%%%%%%%%%%%%%%%%%%%%%%%%%%%
\paragraph{Title Page.}

Conditional processing can be used to include a title or banner page
in the main document when proper precautions are taken.
Importantly, the code in the main file should ensure that the page counter
(as well as other status parameters which are stored in the |.aux| files)
takes the same value after the conditional processing.
Otherwise the page numbers may take divergent values
depending on which part is compiled.

For example, a title page could be declared by:
%
\begin{center}
\begin{tabular}{l}
|\ifchilddoc\||else|\\
|\addtocounter{page}{-1}|\\
\textit{code for title page}\\
|\newpage|\\
|\||fi|
\end{tabular}
\end{center}
%
A banner page for the child documents can be generated by:
%
\begin{center}
\begin{tabular}{l}
|\ifchilddoc|\\
|\addtocounter{page}{-1}|\\
\textit{code for banner page}\\
|\newpage|\\
|\||fi|
\end{tabular}
\end{center}
%
Here one could write a message such as:
\begin{center}
|This is the part \childdocname{} of \childdocjob{}.|
\end{center}

%%%%%%%%%%%%%%%%%%%%%%%%%%%%%%%%%%%%%%%%%%%%%%%%%%%%%%%%%%%%%%%%%%%%%%%%%%%%%%%%
\subsection{Flags}
\label{sec:flags}

The package makes it easy to generate different versions
of the main or child documents.
To this end compilation flags can be defined
and assigned different default values.
They will be particularly useful in conjunction
with the forwarding mechanism described in \secref{sec:forward}.

For example, it may be useful to have a flag |\version|
which can be set to |draft| or |final|.
The document source will contain some conditional code
depending on the value of |\version|.
Suppose further, the flag should default to |final| for the main file
and to |draft| for child files
which is a natural assignment for editing the document.
This is achieved by placing the following code
in the preamble of the main document
(below the |\childdocmain| directive):
%
\begin{center}
\begin{tabular}{l}
|\ifchilddoc|\\
|\providecommand{\version}{draft}|\\
|\||else|\\
|\providecommand{\version}{final}|\\
|\||fi|
\end{tabular}
\end{center}
%
The definition by |\providecommand| makes sure
that previous definitions are not overwritten.
Further statements |\providecommand{\version}{...}|
can thus be added before the above code to override it.

For the main file, one might add a line
(between |\childdocmain| and the above block)
%
\begin{center}
|%\ifchilddoc\||else\providecommand{\version}{draft}\||fi|
\end{center}
%
which can be uncommented to produce a draft version.
Likewise one can add a line to the very top of a child file
(above the |\childdocof{|\textit{main}|}| directive)
%
\begin{center}
|%\providecommand{\version}{final}|
\end{center}
%
which can be uncommented to produce the final version of this child document.

%%%%%%%%%%%%%%%%%%%%%%%%%%%%%%%%%%%%%%%%%%%%%%%%%%%%%%%%%%%%%%%%%%%%%%%%%%%%%%%%
\subsection{Forwarding}
\label{sec:forward}

Different versions of the main or child documents
using compilation flags as described in \secref{sec:flags}
can be (permanently) stored in different files
for convenient compilation, viewing and distribution.
To this end, the package defines a command
to pass on compilation to a different file:

%%%%%%%%%%%%%%%%%%%%%%%%%%%%%%%%%%%%%%%%
\DescribeMacro{\childdocforward}
The command |\childdocforward| redirects processing to
another source file:
%
\begin{center}
\begin{tabular}{l}
|\input{childdoc.def}|\\
|\childdocforward[|\textit{main}|]{|\textit{dest}|}|\\
\end{tabular}
\end{center}
%
The argument \textit{dest} is the destination file
(without extension).
It should be the main file or one of the child files.
Note that further \textsf{childdoc} directives
such as |\childdocof| and |\childdocforward|
in the indicated file will be processed in this form.
The optional argument \textit{main}
passes on directly to the main file \textit{main}
while pretending to compile the child \textit{dest}.
This form behaves as if \textit{dest}
issues |\childdocof{|\textit{main}|}| right away,
and no further \textsf{childdoc} directives will be processed.

%%%%%%%%%%%%%%%%%%%%%%%%%%%%%%%%%%%%%%%%
\DescribeMacro{\...prefix}
In the alternative form |\childdocforwardprefix|,
%
\begin{center}
\begin{tabular}{l}
|\input{childdoc.def}|\\
|\childdocforwardprefix[|\textit{main}|]{|\textit{prefix}|}{|\textit{dest}|}|
\end{tabular}
\end{center}
%
the destination file is determined by a pattern
depending on the current file:
To make this work, the current file must be called
`{\textit{prefix}\hspace{0.2em}\textit{suffix}}'
with \textit{prefix} matching precisely the argument.
Processing is then passed on to the file
`{\textit{dest}\hspace{0.2em}\textit{suffix}}'.
Surely, the same effect is achieved by
directly specifying the
argument `{\textit{dest}\hspace{0.2em}\textit{suffix}}'
in the first form.
However, that requires to set up a different file
for each child. With the alternative form of the command
all these files can have exactly the same content
which simplifies setting them up and maintaining them.

For example, the following file |draft.tex|
with a compilation flag |\version| as described in \secref{sec:flags}
compiles the main document as a draft:
%
\begin{center}
\begin{tabular}{l}
|\def\version{draft}|\\
|\input{childdoc.def}|\\
|\childdocforward{|\textit{main}|}|
\end{tabular}
\end{center}
%
Likewise, the following files |final|\textit{nn}|.tex|
compile the final version of the child document
|child|\textit{nn}|.tex|:
%
\begin{center}
\begin{tabular}{l}
|\def\version{final}|\\
|\input{childdoc.def}|\\
|\childdocforwardprefix{final}{child}|
\end{tabular}
\end{center}
%

Note that when several versions of a main file and/or of each child file
are to be generated, it may be convenient to set up a |Makefile| or
shell script to automatise the process.

%%%%%%%%%%%%%%%%%%%%%%%%%%%%%%%%%%%%%%%%%%%%%%%%%%%%%%%%%%%%%%%%%%%%%%%%%%%%%%%%
\subsection{Command Line Processing}
\label{sec:commandline}

The effect of redirection files can also be achieved by invoking
the \LaTeX{} compiler with a more elaborate command line.
Most conveniently this should be done as part
of a shell script or a |Makefile|.

When using \textsf{childdoc} in the main file, the following
command lines effectively perform a redirection
(note that depending on the shell being used,
backslashes may have to be doubled: `|\|' $\to$ `|\\|'):
%
\begin{center}
|... -jobname "|\textit{target}|" |\\|"|[\textit{flags}]%
|\input{childdoc.def}\childdocforward[|\textit{main}|]{|\textit{dest}|}"|
\end{center}
%
Here \textit{target} is the name of the output file,
\textit{main} is the name of the main file
and \textit{dest} is the name of the main or child file to be processed
(all filenames without extensions).
The optional argument \textit{main} can be omitted
if \textit{main} matches \textit{dest}.
Optionally, compilation \textit{flags} can be defined via |\def| commands.
This command line makes the \TeX{} engine believe
it is compiling the file \textit{target}
whose content is specified as the latter parameter.
The provided code then forwards the processing to
\textit{main} or \textit{dest} as described in \secref{sec:forward}.

%%%%%%%%%%%%%%%%%%%%%%%%%%%%%%%%%%%%%%%%%%%%%%%%%%%%%%%%%%%%%%%%%%%%%%%%%%%%%%%%
\subsection{Include by Input}
\label{sec:input}

Including child documents by |\include| has some restrictions by design.
Most notably, the content of a child document always occupies
its own set of pages; pages cannot be shared between child documents.
Usually, this behaviour makes perfect sense
because each child document contain an essential part of the document.
However, in some situations it may be desirable to compose
a document from a collection of parts
without having mandatory page breaks between then.
For this case, the package
provides a mechanism to include parts
by |\input| which can also be processed individually.
However, by construction this mechanism
requires manual handling of the content to be output.

%%%%%%%%%%%%%%%%%%%%%%%%%%%%%%%%%%%%%%%%
\DescribeMacro{\ifchilddocmanual}
The main file should be prepared as usual, see \secref{sec:include}.
However, the document body must make a distinction
between processing of an individual part and of the main document, e.g.:
%
\begin{center}
\begin{tabular}{l}
|\ifchilddocmanual|\\
|\input{\childdocname}|\\
|\||else|\\
\textit{document body with }|\input{|\textit{part}|}|\\
|\||fi|
\end{tabular}
\end{center}
%
The conditional |\ifchilddocmanual| is true whenever
a part to be included by |\input| is being compiled,
and the name of the part is stored in |\childdocname|.

%%%%%%%%%%%%%%%%%%%%%%%%%%%%%%%%%%%%%%%%
\DescribeMacro{\childdocby}
Each part to be included by |\input| should start with:
%
\begin{center}
\begin{tabular}{l}
|\input{childdoc.def}|\\
|\childdocby{|\textit{main}|}|\\
\end{tabular}
\end{center}
%
The directive |\childdocby| is similar to |\childdocof|
described in \secref{sec:include},
but the subsequent selection of content must be done manually.
To that end, both |\ifchilddoc| and |\ifchilddocmanual|
will be true upon processing of a part,
and the name of the part is stored in |\childdocname|.
Note that |\jobname| will be set to the filename of the current part
so that each part receives an individual |.aux| file
that does not interfere with the |.aux| file(s) of the main document.
This behaviour can be altered by the alternative form
|\childdocby[*]{|\textit{main}|}| (with a non-empty optional argument)
which uses the |.aux| file of the main document
by setting |\jobname| to \textit{main}.

%%%%%%%%%%%%%%%%%%%%%%%%%%%%%%%%%%%%%%%%%%%%%%%%%%%%%%%%%%%%%%%%%%%%%%%%%%%%%%%%
\subsection{Driver Development}
\label{sec:driver}

The \textsf{childdoc} mechanism can also be use for the development
of definition files such as \LaTeX{} styles or classes.
This case differs from the above setup with multiple parts
included by |\include| in that no |\includeonly| should be invoked.
This can be achieved by starting the include file
(before |\ProvidesPackage|) with:
%
\begin{center}
\begin{tabular}{l}
|\input{childdoc.def}|\\
|\childdocforward{|\textit{main}|}|\\
\end{tabular}
\end{center}
%
or alternatively with:
%
\begin{center}
\begin{tabular}{l}
|\input{childdoc.def}|\\
|\childdocby{|\textit{main}|}|\\
\end{tabular}
\end{center}
%
Both forms have slightly different effects as described above.
The main file is prepared as usual, see \secref{sec:include}.

%%%%%%%%%%%%%%%%%%%%%%%%%%%%%%%%%%%%%%%%%%%%%%%%%%%%%%%%%%%%%%%%%%%%%%%%%%%%%%%%
\subsection{Legacy Detection}
\label{sec:detection}

The directive |\childdocmain| in the main file can detect
whether the complete document or merely a child is to be compiled
even without using the directive |\childdocof|.
This method is deprecated because it is less robust
and there is no compelling reason to use it;
it is merely provided for backward compatibility
and it may be removed in future versions.

If the detection mechanism is to be used,
it is mandatory to correctly specify
the filename of the main file as the argument of |\childdocmain|:
%
\begin{center}
\begin{tabular}{l}
|\input{childdoc.def}|\\
|\childdocmain{|\textit{main}|}|\\
\end{tabular}
\end{center}
%
If |\jobname| does not match the argument \textit{main} of |\childdocmain|,
it is assumed that |\jobname| points to the child file to be compiled.
When using |\childdocmain| with the main file specified as argument,
it suffices to start a child file
with just |\input{|\textit{main}|}|
without loading of the package and using |\childdocof|.
If instead all processing is done
with the appropriate \textsf{childdoc} directives,
the argument of \textit{main} of |\childdocmain| can be empty.

An alternative version of the command line processing described
in \secref{sec:commandline} using the detection mechanism reads:
%
\begin{center}
|... -jobname "|\textit{target}|" "|[\textit{flags}]%
[|\def\jobname{|\textit{dest}|}|]|\input{|\textit{main}|}"|
\end{center}

%%%%%%%%%%%%%%%%%%%%%%%%%%%%%%%%%%%%%%%%%%%%%%%%%%%%%%%%%%%%%%%%%%%%%%%%%%%%%%%%
\subsection{Manual Code}
\label{sec:manual}

In case one cannot be certain whether the definitions file |childdoc.def|
is installed on the target \TeX{} distribution
and one prefers not to ship it,
it is conceivable to paste a few relevant commands into the sources.

To that end, drop all statements |\input{childdoc.def}|
and perform the replacements as outlined below.
Instead of |\childdocmain{|\textit{main}|}| add the following code
to the top of the main file:
%
\begin{center}
\begin{tabular}{l}
|\||ifdefined\childdocname\endinput\||fi\newif\ifchilddoc|\\
|\edef\childdocname{\scantokens\expandafter{\jobname\noexpand}}|\\
|\def\childdocmain{|\textit{main}|}\||ifx\childdocmain\childdocname\||else|\\
|\childdoctrue\includeonly{\childdocname}\let\jobname\childdocmain\||fi|\\
\end{tabular}
\end{center}
%
Instead of |\childdocof{|\textit{main}|}| just include the main file
at the top of each child file:
%
\begin{center}
|\input{|\textit{main}|}|
\end{center}
%
A simple redirection |\childdocforward{|\textit{dest}|}| is achieved by:
%
\begin{center}
|\def\jobname{|\textit{dest}|}\input{\jobname}|
\end{center}
%
The redirection with prefix
|\childdocforwardprefix[|\textit{prefix}|]{|\textit{dest}|}|
is accomplished by:
%
\begin{center}
\begin{tabular}{l}
|{\edef\jobname{\scantokens\expandafter{\jobname\noexpand}}|\\
|\def\redirectjob |\textit{prefix}|#1~~~{\gdef\jobname{|\textit{dest}|#1}}|\\
|\expandafter\redirectjob\jobname~~~}\input{\jobname}|
\end{tabular}
\end{center}

In an alternative approach,
child documents can be compiled by a specific command line
without additional code or specific definitions:
%
\begin{center}
|... -jobname "|\textit{target}|" "|[\textit{flags}]%
|\includeonly{|\textit{dest}|}\input{|\textit{main}|}"|
\end{center}
%

%%%%%%%%%%%%%%%%%%%%%%%%%%%%%%%%%%%%%%%%%%%%%%%%%%%%%%%%%%%%%%%%%%%%%%%%%%%%%%%%
%%%%%%%%%%%%%%%%%%%%%%%%%%%%%%%%%%%%%%%%%%%%%%%%%%%%%%%%%%%%%%%%%%%%%%%%%%%%%%%%
\section{Information}

%%%%%%%%%%%%%%%%%%%%%%%%%%%%%%%%%%%%%%%%%%%%%%%%%%%%%%%%%%%%%%%%%%%%%%%%%%%%%%%%
\subsection{Copyright}

Copyright \copyright{} 2017--2018 Niklas Beisert

This work may be distributed and/or modified under the
conditions of the \LaTeX{} Project Public License, either version 1.3
of this license or (at your option) any later version.
The latest version of this license is in
  \url{http://www.latex-project.org/lppl.txt}
and version 1.3 or later is part of all distributions of \LaTeX{}
version 2005/12/01 or later.

This work has the LPPL maintenance status `maintained'.

The Current Maintainer of this work is Niklas Beisert.

This work consists of the files |README.txt|, |childdoc.ins| and |childdoc.dtx|
as well as the derived files |childdoc.def|, |cdocsamp.tex|
with |cdocsch1.tex|, |cdocsch2.tex|, |cdocspt3.tex|, |cdocspt4.tex|,
|cdocsdrf.tex|, |cdocsfn1.tex|, |cdocsfn2.tex|
as well as |childdoc.pdf|.

%%%%%%%%%%%%%%%%%%%%%%%%%%%%%%%%%%%%%%%%%%%%%%%%%%%%%%%%%%%%%%%%%%%%%%%%%%%%%%%%
\subsection{Files and Installation}

The package consists of the files:
%
\begin{center}
\begin{tabular}{ll}
    |README.txt|   & readme file \\
    |childdoc.ins| & installation file \\
    |childdoc.dtx| & source file \\
    |childdoc.def| & definition file \\
    |cdocsamp.tex| & sample main file \\
    |cdocsch1.tex| & sample include file \\
    |cdocsch2.tex| & sample include file \\
    |cdocspt3.tex| & sample part file \\
    |cdocspt4.tex| & sample part file \\
    |cdocsdrf.tex| & sample redirection file \\
    |cdocsfn1.tex| & sample redirection file \\
    |cdocsfn2.tex| & sample redirection file \\
    |childdoc.pdf| & manual
\end{tabular}
\end{center}
%
The distribution consists of the files
|README.txt|, |childdoc.ins| and |childdoc.dtx|.
%
\begin{itemize}
\item
Run (pdf)\LaTeX{} on |childdoc.dtx|
to compile the manual |childdoc.pdf| (this file).
\item
Run \LaTeX{} on |childdoc.ins| to create the definitions file |childdoc.def|
and the sample |cdocsamp.tex| with include files
|cdocsch1.tex|, |cdocsch2.tex|, |cdocspt3.tex|, |cdocspt4.tex|,
|cdocsdrf.tex|, |cdocsfn1.tex|, |cdocsfn2.tex|.
Then copy the file |childdoc.def| to an appropriate directory of your \LaTeX{}
distribution, e.g.\ \textit{texmf-root}|/tex/latex/childdoc|.
\end{itemize}

%%%%%%%%%%%%%%%%%%%%%%%%%%%%%%%%%%%%%%%%%%%%%%%%%%%%%%%%%%%%%%%%%%%%%%%%%%%%%%%%
\subsection{Related CTAN Packages}

There are several other packages which offer a similar functionality:
%
\begin{itemize}
\item
The packages
\href{http://ctan.org/pkg/docmute}{\textsf{docmute}},
\href{http://ctan.org/pkg/includex}{\textsf{includex}} and
\href{http://ctan.org/pkg/standalone}{\textsf{standalone}}
provide commands to include only the document body of
a child file thus allowing both files to be compiled individually.
\item
The packages \href{http://ctan.org/pkg/subdocs}{\textsf{subdocs}}
and \href{http://ctan.org/pkg/subfiles}{\textsf{subfiles}}
provide structures in which the main and child documents can be
encapsulated and allowing them to be compiled individually.
The inclusion mechanism is different from the conventional |\include|.
\item
The package \href{http://ctan.org/pkg/combine}{\textsf{combine}}
is an elaborate solution to combine several documents into one.
\end{itemize}
%
See also the CTAN topic \href{http://ctan.org/topic/subdocs}{\textsf{subdocs}}
for further related packages.
The present package differs from the above solutions in that
a document structure constructed with the conventional |\include| mechanism
just needs two extra commands at the top of every file
such that all constituent files can be compiled individually.

%%%%%%%%%%%%%%%%%%%%%%%%%%%%%%%%%%%%%%%%%%%%%%%%%%%%%%%%%%%%%%%%%%%%%%%%%%%%%%%%
%\subsection{Feature Suggestions}
%
%The following is a list of features which may be useful for future
%versions of this package:
%%
%\begin{itemize}
%\item
%\ldots
%\end{itemize}

%%%%%%%%%%%%%%%%%%%%%%%%%%%%%%%%%%%%%%%%%%%%%%%%%%%%%%%%%%%%%%%%%%%%%%%%%%%%%%%%
\subsection{Revision History}

%%%%%%%%%%%%%%%%%%%%%%%%%%%%%%%%%%%%%%%%
\paragraph{v2.0:} 2018/12/30

\begin{itemize}
\item
immediate forward processing
\item
added |\childdocby| mechanism
\item
manual restructured
\end{itemize}

%%%%%%%%%%%%%%%%%%%%%%%%%%%%%%%%%%%%%%%%
\paragraph{v1.6:} 2018/01/17

\begin{itemize}
\item
application for development of include files
\item
corrections to manual
\end{itemize}

%%%%%%%%%%%%%%%%%%%%%%%%%%%%%%%%%%%%%%%%
\paragraph{v1.5:} 2017/05/21

\begin{itemize}
\item
more complete structuring introduced
\item
|\childdocof| introduced
\item
|\childdoc| renamed to |\childdocmain|
\item
|\childredirect| renamed to |\childdocforward| and |\childdocforwardprefix|
and functionality expanded
\end{itemize}

%%%%%%%%%%%%%%%%%%%%%%%%%%%%%%%%%%%%%%%%
\paragraph{v1.0:} 2017/04/27

\begin{itemize}
\item
manual and install package
\item
first version published on CTAN
\end{itemize}

%%%%%%%%%%%%%%%%%%%%%%%%%%%%%%%%%%%%%%%%
\paragraph{v0.6:} 2017/04/26

\begin{itemize}
\item
redirection mechanism added
\end{itemize}

%%%%%%%%%%%%%%%%%%%%%%%%%%%%%%%%%%%%%%%%
\paragraph{v0.5:} 2017/04/26

\begin{itemize}
\item
functionality in definition file
\end{itemize}


%%%%%%%%%%%%%%%%%%%%%%%%%%%%%%%%%%%%%%%%%%%%%%%%%%%%%%%%%%%%%%%%%%%%%%%%%%%%%%%%
%%%%%%%%%%%%%%%%%%%%%%%%%%%%%%%%%%%%%%%%%%%%%%%%%%%%%%%%%%%%%%%%%%%%%%%%%%%%%%%%
%%%%%%%%%%%%%%%%%%%%%%%%%%%%%%%%%%%%%%%%%%%%%%%%%%%%%%%%%%%%%%%%%%%%%%%%%%%%%%%%
\appendix

\settowidth\MacroIndent{\rmfamily\scriptsize 000\ }

 \DocInput{childdoc.dtx}

\end{document}
%</driver>
% \fi
%
% %%%%%%%%%%%%%%%%%%%%%%%%%%%%%%%%%%%%%%%%%%%%%%%%%%%%%%%%%%%%%%%%%%%%%%%%%%%%%%
% %%%%%%%%%%%%%%%%%%%%%%%%%%%%%%%%%%%%%%%%%%%%%%%%%%%%%%%%%%%%%%%%%%%%%%%%%%%%%%
% \section{Sample}
%\iffalse
%<*samplemain>
%\fi
%
% The following presents a sample document
% with two chapters, two parts, a title page,
% a compile flag as well as three forwarding files to set the flag.
% It consists of eight |.tex| files:
% \begin{center}
% \begin{tabular}{ll}
% |cdocsamp.tex|&main file\\
% |cdocsch1.tex|&include file for chapter 1\\
% |cdocsch2.tex|&include file for chapter 2\\
% |cdocspt3.tex|&include file for part 3\\
% |cdocspt4.tex|&include file for part 4\\
% |cdocsdrf.tex|&forwarding file for main file in draft mode\\
% |cdocsfi1.tex|&forwarding file for final version of chapter 1\\
% |cdocsfi2.tex|&forwarding file for final version of chapter 2\\
% \end{tabular}
% \end{center}
% Each of the eight files can be compiled directly by the \LaTeX{} compiler.
%
% %%%%%%%%%%%%%%%%%%%%%%%%%%%%%%%%%%%%%%
% \paragraph{Main File.}
%
% The main file is called |cdocsamp.tex|.
%
% Load the \textsf{childdoc} definitions and
% declare the filename for the main document:
%    \begin{macrocode}
\input{childdoc.def}
\childdocmain{}
%    \end{macrocode}

% Optional override for |\version| flag:
%    \begin{macrocode}
%%\ifchilddoc\else\providecommand{\version}{draft}\fi
%    \end{macrocode}

% Define the default values for the |\version| flag
% (|final| for the main file and |draft| for childs):
%    \begin{macrocode}
\ifchilddoc
\providecommand{\version}{draft}
\else
\providecommand{\version}{final}
\fi
%    \end{macrocode}

% Load the standard document class:
%    \begin{macrocode}
\documentclass[12pt]{article}
%    \end{macrocode}

% Start the document body:
%    \begin{macrocode}
\begin{document}
%    \end{macrocode}

% Declare a title page.
% Print title, part of document being processed and version flag:
%    \begin{macrocode}
\addtocounter{page}{-1}
\begin{center}
{\LARGE\bfseries{}childdoc example\par}
\vspace{1cm}
\ifchilddoc
\ifchilddocmanual part\else chapter\fi:
`\childdocname' of `\childdocjob'\par
\else
main document: `\childdocjob'\par
\fi
version: \version\par
\end{center}
\newpage
%    \end{macrocode}

% Manually include selected file,
% otherwise process as usual:
%    \begin{macrocode}
\ifchilddocmanual
\section*{part `\childdocname'}
\input{\childdocname}
\else
%    \end{macrocode}

% Include the two chapters:
%    \begin{macrocode}
\include{cdocsch1}
\include{cdocsch2}
%    \end{macrocode}

% Include the two parts unless only chapters should be displayed:
%    \begin{macrocode}
\ifchilddoc\else
\section{part three}
\input{cdocspt3}
\section{part four}
\input{cdocspt4}
\fi
%    \end{macrocode}

% Process as usual until here:
%    \begin{macrocode}
\fi
%    \end{macrocode}

% End of document body:
%    \begin{macrocode}
\end{document}
%    \end{macrocode}
%\iffalse
%</samplemain>
%\fi
%
% %%%%%%%%%%%%%%%%%%%%%%%%%%%%%%%%%%%%%%
% \paragraph{Chapter Include Files.}
%
% The include files are called |cdocsch1.tex| and |cdocsch2.tex|.
%
%\iffalse
%<*samplechap1|samplechap2>
%\fi

% Optional override for |\version| flag:
%    \begin{macrocode}
%%\providecommand{\version}{final}
%    \end{macrocode}

% Include the main document:
%    \begin{macrocode}
\input{childdoc.def}
\childdocof{cdocsamp}
%    \end{macrocode}

%\iffalse
%</samplechap1|samplechap2>
%\fi
%
%\iffalse
%<*samplechap1>
%\fi
% Some text for chapter 1:
%    \begin{macrocode}
\section{one}
some text in chapter one
%    \end{macrocode}

%\iffalse
%</samplechap1>
%\fi
% Some text for chapter 2:
%\iffalse
%<*samplechap2>
%\fi
%    \begin{macrocode}
\section{two}
more text in chapter two
%    \end{macrocode}

%\iffalse
%</samplechap2>
%\fi
%
% %%%%%%%%%%%%%%%%%%%%%%%%%%%%%%%%%%%%%%
% \paragraph{Part Include Files.}
%
% The include files are called |cdocspt3.tex| and |cdocspt4.tex|.
%
%\iffalse
%<*samplepart3|samplepart4>
%\fi

% Optional override for |\version| flag:
%    \begin{macrocode}
%%\providecommand{\version}{final}
%    \end{macrocode}

% Include the main document:
%    \begin{macrocode}
\input{childdoc.def}
\childdocby{cdocsamp}
%    \end{macrocode}

%\iffalse
%</samplepart3|samplepart4>
%\fi
%
%\iffalse
%<*samplepart3>
%\fi
% Some text for part 3:
%    \begin{macrocode}
some text in part three
%    \end{macrocode}

%\iffalse
%</samplepart3>
%\fi
% Some text for part 4:
%\iffalse
%<*samplepart4>
%\fi
%    \begin{macrocode}
more text in part four
%    \end{macrocode}

%\iffalse
%</samplepart4>
%\fi
%
% %%%%%%%%%%%%%%%%%%%%%%%%%%%%%%%%%%%%%%
% \paragraph{Forwarding for a Complete Draft.}
%
% The following forwarding file |cdocsdrf.tex|
% compiles the main document in draft mode:
%\iffalse
%<*sampledraft>
%\fi
%    \begin{macrocode}
\def\version{draft}
\input{childdoc.def}
\childdocforward{cdocsamp}
%    \end{macrocode}

%\iffalse
%</sampledraft>
%\fi
%
% %%%%%%%%%%%%%%%%%%%%%%%%%%%%%%%%%%%%%%
% \paragraph{Forwarding for Final Version of the Chapters.}
%
% The following forwarding files |cdocsfn1.tex| and |cdocsfn2.tex|
% (with identical content)
% compile the final versions of the child documents
% |cdocsch1.tex| and |cdocsch2.tex|, respectively:
%\iffalse
%<*samplefinal>
%\fi
%    \begin{macrocode}
\def\version{final}
\input{childdoc.def}
\childdocforwardprefix[cdocsamp]{cdocsfn}{cdocsch}
%    \end{macrocode}

%\iffalse
%</samplefinal>
%\fi
%
% %%%%%%%%%%%%%%%%%%%%%%%%%%%%%%%%%%%%%%
% \paragraph{Command Line Processing.}
%
% The following three command lines generate the output files
% |cdocscld|, |cdocscl1| and |cdocscl2|
% which should be identical to
% |cdocsdrf|, |cdocsch1| and |cdocsfn2|, respectively:
% \begin{center}
% \begin{tabular}{l}
% |latex -jobname cdocscld \|\\
% |  "\def\version{draft}\input{childdoc.def}\childdocforward{cdocsamp}"|\\
% |latex -jobname cdocscl1 \|\\
% |  "\input{childdoc.def}\childdocforward[cdocsamp]{cdocsch1}"|\\
% |latex -jobname cdocscl2 \|\\
% |  "\def\version{final}\input{childdoc.def}\childdocforward{cdocsch2}"|
% \end{tabular}
% \end{center}
% Note that the trailing backslash on each first line
% merely continues the input to the second line
% (for convenient cut ant paste).
% Furthermore, the command |latex| can be replaced by any
% of its alternative versions such as |pdflatex|.
%
% %%%%%%%%%%%%%%%%%%%%%%%%%%%%%%%%%%%%%%%%%%%%%%%%%%%%%%%%%%%%%%%%%%%%%%%%%%%%%%
% %%%%%%%%%%%%%%%%%%%%%%%%%%%%%%%%%%%%%%%%%%%%%%%%%%%%%%%%%%%%%%%%%%%%%%%%%%%%%%
% \section{Implementation}
%\iffalse
%<*package>
%\fi
%
% This section describes the definitions file |childdoc.def|.

% The definitions cannot be loaded using |\usepackage| or |\RequirePackage|
% which has a mechanism to prevent loading a style file more than once.
% When loading the definitions by means of |\input|
% multiple instances have to be prevented manually:
%\iffalse
%This code needs to be before the `\ProvidesFile' directive
%which is defined at the beginning of this file.
%Therefore it is also placed there and commented out here.
%</package>
%<*discard>
%\fi
%    \begin{macrocode}
\ifdefined\childdocmain\endinput\fi
%    \end{macrocode}
%\iffalse
%</discard>
%<*package>
%\fi
%
% \macro{\ifchilddoc}
% \macro{\ifchilddocmanual}
% The conditional |\ifchilddoc| tells whether a
% child (true) or main (false) document is being compiled.
% The conditional |\ifchilddocmanual| tells whether
% the |\includeonly| mechanism is used (false) or
% the selection of child files must be performed manually (true).
% The definitions initialise to false:
%    \begin{macrocode}
\newif\ifchilddoc
\newif\ifchilddocmanual
%    \end{macrocode}

% \macro{\childdocname}
% \macro{\childdocjob}
% The macro |\childdocname| stores the name of the main document
% to be compiled. The macro |\childdocjob| stores the name of
% the document on which the \LaTeX{} compiler was originally invoked.
% The content of |\jobname| cannot be compared
% to filenames specified in the source due to different catcodes.
% The following code rescans |\jobname|, stores the result
% in |\childdocname| and saves a copy in |\childdocjob|:
%    \begin{macrocode}
\edef\childdocname{\scantokens\expandafter{\jobname\noexpand}}
\let\childdocjob\childdocname
%    \end{macrocode}

% \macro{\childdocdisable}
% The macro |\childdocdisable| prevents the main file
% from being processed more than once.
% At this stage, the main document command |\childdocmain|
% is assumed to be called once again where it should do nothing.
% Any subsequent call to it should prevent
% a secondary processing of the main document
% It overwrites the forwarding commands
% |\childdocof| and |\childdocforward|
% with empty macros to prevent further inclusions of the main document:
%    \begin{macrocode}
\newcommand{\childdocdisable}
{
  \renewcommand{\childdocmain}[1]{\renewcommand{\childdocmain}[1]{\endinput}}
  \renewcommand{\childdocof}[1]{}
  \renewcommand{\childdocby}[2][]{}
  \renewcommand{\childdocforward}[2][]{}
  \renewcommand{\childdocdisable}{}
}
%    \end{macrocode}

% \macro{\childdocmain}
% The macro |\childdocmain| is to be called at the top of the main file
% with nothing or the main filename (without extension) as argument.
% First, it breaks loops.
% If the argument is not empty and does not match |\childdocname|
% (which is set by the first inclusion of |childdoc.def|),
% |\ifchilddoc| is set to true, |\includeonly| is applied to the child file
% and |\jobname| is set to the main file
% (for proper handling of |.aux| files):
%    \begin{macrocode}
\newcommand{\childdocmain}[1]
{
  \childdocdisable\childdocmain{}
  \if?#1?\else
    \begingroup
      \def\childdoctmp{#1}
      \ifx\childdoctmp\childdocname
        \def\childdoctmp{}
      \else
        \def\childdoctmp
        {
          \childdoctrue
          \includeonly{\childdocname}
          \def\childdocjob{#1}
          \def\jobname{#1}
        }
      \fi
      \expandafter
    \endgroup
    \childdoctmp
  \fi
}
%    \end{macrocode}

% \macro{\childdocof}
% The command |\childdocof| redirects
% compilation to the main file |#1|.
%    \begin{macrocode}
\newcommand{\childdocof}[1]
{
  \childdocdisable
  \childdoctrue
  \includeonly{\childdocname}
  \def\jobname{#1}
  \def\childdocjob{#1}
  \input{#1}
}
%    \end{macrocode}

% \macro{\childdocby}
% The command |\childdocby| ....
%    \begin{macrocode}
\newcommand{\childdocby}[2][]
{
  \childdocdisable
  \childdoctrue
  \childdocmanualtrue
  \if?#1?\else
    \def\jobname{#2}
  \fi
  \def\childdocjob{#2}
  \input{#2}
  \endinput
}
%    \end{macrocode}

% \macro{\childdocforward}
% The command |\childdocforward| redirects
% compilation to the main file or
% (if the optional argument is given) a child file.
% Parameters are set as if the main file
% or a child file starting with |\childdocof| was compiled.
% Then compilation is handed over to the main file:
%    \begin{macrocode}
\newcommand{\childdocforward}[2][]
{
  \begingroup
    \if?#1?
      \def\childdoctmp
      {
        \def\childdocname{#2}
        \def\childdocjob{#2}
        \def\jobname{#2}
        \input{#2}
        \endinput
      }
    \else
      \def\childdoctmp
      {
        \childdocdisable
        \def\childdocname{#2}
        \childdoctrue
        \includeonly{#2}
        \def\childdocjob{#1}
        \def\jobname{#1}
        \input{#1}
        \endinput
      }
    \fi
    \expandafter
  \endgroup
  \childdoctmp
}
%    \end{macrocode}

% \macro{\childdocforwardprefix}
% The command |\childdocforwardprefix| redirects
% compilation to the main or a child file by means of a pattern.
% The prefix |#1| in the current filename is replaced by |#2|
% and the suffix of the current filename is kept
% (it is assumed that the filename does not contain the substring `|~~~|'
% which is used as a delimiter).
% Compilation is handed over to the new file by |\childdocforward|:
%    \begin{macrocode}
\newcommand{\childdocforwardprefix}[3][]
{
  \begingroup
    \def\childdocextract #2##1~~~{\def\childdoctmp{\childdocforward[#1]{#3##1}}}
    \expandafter\childdocextract\childdocname~~~
    \expandafter
  \endgroup
  \childdoctmp
}
%    \end{macrocode}

% \macro{\childdoc}
% The deprecated macro |\childdoc| is a legacy version of |\childdocmain|:
%    \begin{macrocode}
\newcommand{\childdoc}{\childdocmain}
%    \end{macrocode}

% \macro{\childdocredirect}
% The deprecated macro |\childdocredirect| is a legacy version
% of |\childdocforward| and |\childdocforwardprefix|:
%    \begin{macrocode}
\newcommand{\childdocredirect}[2][]
{
  \begingroup
    \if?#1?
      \def\childdoctmp{\childdocforward{#2}}
    \else
      \def\childdoctmp{\childdocforwardprefix{#1}{#2}}
    \fi
    \expandafter
  \endgroup
  \childdoctmp
}
%    \end{macrocode}

%\iffalse
%</package>
%\fi
%
\endinput
\childdocforward[cdocsamp]{cdocsch1}"|\\
% |latex -jobname cdocscl2 \|\\
% |  "\def\version{final}% \iffalse
%
% childdoc.dtx Copyright (C) 2017-2018 Niklas Beisert
%
% This work may be distributed and/or modified under the
% conditions of the LaTeX Project Public License, either version 1.3
% of this license or (at your option) any later version.
% The latest version of this license is in
%   http://www.latex-project.org/lppl.txt
% and version 1.3 or later is part of all distributions of LaTeX
% version 2005/12/01 or later.
%
% This work has the LPPL maintenance status `maintained'.
%
% The Current Maintainer of this work is Niklas Beisert.
%
% This work consists of the files childdoc.dtx and childdoc.ins
% and the derived files childdoc.def and cdocsamp.tex with
% cdocsch1.tex, cdocsch2.tex, cdocsdrf.tex, cdocsfn1.tex, cdocsfn2.tex.
%
%<package>\ifdefined\childdocmain\endinput\fi
%<package>\ProvidesFile{childdoc.def}[2018/12/30 v2.0 child document driver]
%<samplemain>\ProvidesFile{cdocsamp.tex}[2018/12/30 v2.0 sample for childdoc]
%<*driver>
%\ProvidesFile{childdoc.drv}[2018/12/30 v2.0 childdoc reference manual file]
\PassOptionsToClass{10pt,a4paper}{article}
\documentclass{ltxdoc}

\usepackage[margin=35mm]{geometry}
\usepackage{hyperref}
\usepackage{hyperxmp}
\usepackage[usenames]{color}

\hypersetup{colorlinks=true}
\hypersetup{pdfstartview=FitH}
\hypersetup{pdfpagemode=UseNone}
\hypersetup{pdfsource={}}
\hypersetup{pdflang={en-UK}}
\hypersetup{pdfcopyright={Copyright 2017-2018 Niklas Beisert.
  This work may be distributed and/or modified under the
  conditions of the LaTeX Project Public License, either version 1.3
  of this license or (at your option) any later version.}}
\hypersetup{pdflicenseurl={http://www.latex-project.org/lppl.txt}}
\hypersetup{pdfcontactaddress={ETH Zurich, ITP, HIT K,
  Wolfgang-Pauli-Strasse 27}}
\hypersetup{pdfcontactpostcode={8093}}
\hypersetup{pdfcontactcity={Zurich}}
\hypersetup{pdfcontactcountry={Switzerland}}
\hypersetup{pdfcontactemail={nbeisert@itp.phys.ethz.ch}}
\hypersetup{pdfcontacturl={http://people.phys.ethz.ch/\xmptilde nbeisert/}}

\newcommand{\secref}[1]{\hyperref[#1]{section \ref*{#1}}}

\parskip1ex
\parindent0pt
\let\olditemize\itemize
\def\itemize{\olditemize\parskip0pt}

\begin{document}

\title{The \textsf{childdoc} Package}
\hypersetup{pdftitle={The childdoc Package}}
\author{Niklas Beisert\\[2ex]
  Institut f\"ur Theoretische Physik\\
  Eidgen\"ossische Technische Hochschule Z\"urich\\
  Wolfgang-Pauli-Strasse 27, 8093 Z\"urich, Switzerland\\[1ex]
  \href{mailto:nbeisert@itp.phys.ethz.ch}
  {\texttt{nbeisert@itp.phys.ethz.ch}}}
\hypersetup{pdfauthor={Niklas Beisert}}
\hypersetup{pdfsubject={Manual for the LaTeX2e Package childdoc}}
\date{30 December 2018, \textsf{v2.0}}
\maketitle

\begin{abstract}\noindent
\textsf{childdoc} is a \LaTeXe{} package
that enables the direct compilation
of document sections included by |\include|
to individual files.
\end{abstract}

\begingroup
\parskip0ex
\tableofcontents
\endgroup

%%%%%%%%%%%%%%%%%%%%%%%%%%%%%%%%%%%%%%%%%%%%%%%%%%%%%%%%%%%%%%%%%%%%%%%%%%%%%%%%
%%%%%%%%%%%%%%%%%%%%%%%%%%%%%%%%%%%%%%%%%%%%%%%%%%%%%%%%%%%%%%%%%%%%%%%%%%%%%%%%
\section{Introduction}

\LaTeX{} provides a mechanism to structure a large document (such as a book)
into a main file and several child files (containing the chapters)
using the |\include| command.
This mechanism is beneficial for documents
which span hundreds of pages in order to
make the source file(s) more manageable.
Moreover, compilation can be restricted to
selected child files by means of the |\includeonly| command.
The latter feature can be used to reduce the compilation time while editing
(this was significantly more useful in the earlier days of \LaTeX{})
or to generate a smaller document which is easier to navigate.
Another application of |\includeonly| is to generate
documents consisting of selected parts of the complete document.

However, there are a few drawbacks of the plain |\include| mechanism:
\begin{itemize}
\item
The child files cannot be compiled on their own,
they can only be compiled via the main file.
A naive editing environment
(such as a text editor with an option
to have the current file processed by \LaTeX)
may require one to switch to the main file before compiling;
attempting to compile the child file produces errors.
\item
The main file must be modified (each time)
to adjust the |\includeonly| command
to the present needs. This easily leaves the main file in a messy state.
\item
The generated document will always carry the filename
of the main document. This is inconvenient if
several child files are to be compiled and
to be kept for distribution.
\end{itemize}

The present package provides a simple interface
to make child files individually compilable by \LaTeX{}.
Compiling a child file then has the same effect as compiling
the main file with an |\includeonly| command
to select the appropriate child.
Moreover the generated document will carry the name of the child
rather than the main file.
This resolves all three above issues.

This feature is meant to make the editing of books,
thesis documents and lecture notes somewhat more convenient.
However, the package can also be used efficiently for
composing a series of documents (such as exercise sheets)
which are typically distributed individually.
It then assists the author in generating the individual documents
(potentially in different versions)
as well as a document containing the collected series.
Another application is in developing style files
or other kinds of included material
where compilation of the style file could redirect
to a sample or test file.

%%%%%%%%%%%%%%%%%%%%%%%%%%%%%%%%%%%%%%%%%%%%%%%%%%%%%%%%%%%%%%%%%%%%%%%%%%%%%%%%
%%%%%%%%%%%%%%%%%%%%%%%%%%%%%%%%%%%%%%%%%%%%%%%%%%%%%%%%%%%%%%%%%%%%%%%%%%%%%%%%
\section{Usage}

First of all, the package \textsf{childdoc} is \emph{not} a standard
\LaTeXe{} |.sty| style file! Therefore it needs to be invoked in
a non-standard way.

%%%%%%%%%%%%%%%%%%%%%%%%%%%%%%%%%%%%%%%%%%%%%%%%%%%%%%%%%%%%%%%%%%%%%%%%%%%%%%%%
\subsection{Included Files}
\label{sec:include}

%%%%%%%%%%%%%%%%%%%%%%%%%%%%%%%%%%%%%%%%
\DescribeMacro{\childdocmain}
To use the package, add the commands
\begin{center}
\begin{tabular}{l}
|\input{childdoc.def}|\\
|\childdocmain{}|\\
\end{tabular}
\end{center}
at the very top of the main \LaTeX{} file,
in particular \emph{before} the |\documentclass| statement!
The argument of |\childdocmain| should be left empty
(but it must be present).

%%%%%%%%%%%%%%%%%%%%%%%%%%%%%%%%%%%%%%%%
\DescribeMacro{\childdocof}
Furthermore, add the commands
\begin{center}
\begin{tabular}{l}
|\input{childdoc.def}|\\
|\childdocof{|\textit{main}|}|\\
\end{tabular}
\end{center}
at the top of every child file \textit{child}
which is included by |\include{|\textit{child}|}|
from within the main file
(or at least for those files to be compiled individually).
The argument \textit{main} must be the filename of the main file.

There are a couple of
considerations in setting up the main and child documents:

%%%%%%%%%%%%%%%%%%%%%%%%%%%%%%%%%%%%%%%%
\paragraph{Restrictions.}

Please note the following restrictions:
\begin{itemize}
\item
|\childdocmain| must be called with one argument \textit{main}
to ensure compatibility with earlier version of the package.
It must either be empty (|\childdocmain{}|)
or precisely match the filename of the main file in which it is specified.
See \secref{sec:detection} for further information.
\item
The filename \textit{main} must be specified without the |.tex| extension.
\item
The filename \textit{main} is case sensitive
(even in case-insensitive file systems)
due to internal string comparison.
\item
The argument \textit{main} should be fully expanded, it cannot be a macro.
\item
Subdirectories and special characters should be avoided in filenames.
\item
The command |\childdocmain{|\textit{main}|}| must be followed by a whitespace.
It should not be followed immediately by another command
or by a comment mark `|%|'.
This is because the \TeX{} parser reads the token immediately following
the argument of |\childdocmain| and puts it
at the beginning of every child section;
however, a white\-space is ignored.
\end{itemize}

%%%%%%%%%%%%%%%%%%%%%%%%%%%%%%%%%%%%%%%%
\paragraph{Content of Main File.}

It is advisable to place all content in the child files included by |\include|.
Any output contained in the main file will appear in all child documents
unless suppressed manually;
it cannot be suppressed automatically by the |\includeonly| directive
and thus should normally be avoided.
A method to include some content in the main file
by means of conditional processing is described in \secref{sec:conditional}.

%%%%%%%%%%%%%%%%%%%%%%%%%%%%%%%%%%%%%%%%
\paragraph{Page Numbering.}

When only a part of the document is compiled,
the appropriate numbering of pages
(as well as other status parameters)
is determined from the |.aux| files.
The latter contain information from previous passes.
However this information needs to propagate through
all intermediate child documents.
Therefore the page numbering in child documents may well
be inconsistent until the complete document is compiled at least once.

A useful (if unconventional) way to always ensure a consistent
page numbering is to restart the numbering in each child document
and denote the pages by `\textit{child}|.|\textit{page}'
where \textit{child} represents the chapter/section number of the child file.
This can be achieved by the command
|\numberwithin{page}{|\textit{child}|}|
of the \textsf{amsmath} package
where \textit{child} can be |chapter| or |section|
depending on the chosen structuring.
Alternatively, one can modify the macro |\thepage| appropriately
and reset the counter |page| at the start of each child file.

%%%%%%%%%%%%%%%%%%%%%%%%%%%%%%%%%%%%%%%%%%%%%%%%%%%%%%%%%%%%%%%%%%%%%%%%%%%%%%%%
\subsection{Conditional Processing}
\label{sec:conditional}

The package provides a mechanism to compile different versions
of a document. To customise the versions further some conditional processing
can come in handy to distinguish which version is being compiled.
The package provides two macros to describe the compilation context:

%%%%%%%%%%%%%%%%%%%%%%%%%%%%%%%%%%%%%%%%
\DescribeMacro{\ifchilddoc}
The conditional |\ifchilddoc| distinguishes between the compilation of
child documents and the main document:
%
\begin{center}
|\ifchilddoc |\textit{child-code}| |[|\||else |\textit{main-code}]| \||fi|
\end{center}

%%%%%%%%%%%%%%%%%%%%%%%%%%%%%%%%%%%%%%%%
\DescribeMacro{\childdocname}
\DescribeMacro{\childdocjob}
The macro |\childdocname| contains the filename (without extension)
of the main or child file being processed.
Note that |\childdocjob| will always contain the name of the main file.

%%%%%%%%%%%%%%%%%%%%%%%%%%%%%%%%%%%%%%%%
\paragraph{Title Page.}

Conditional processing can be used to include a title or banner page
in the main document when proper precautions are taken.
Importantly, the code in the main file should ensure that the page counter
(as well as other status parameters which are stored in the |.aux| files)
takes the same value after the conditional processing.
Otherwise the page numbers may take divergent values
depending on which part is compiled.

For example, a title page could be declared by:
%
\begin{center}
\begin{tabular}{l}
|\ifchilddoc\||else|\\
|\addtocounter{page}{-1}|\\
\textit{code for title page}\\
|\newpage|\\
|\||fi|
\end{tabular}
\end{center}
%
A banner page for the child documents can be generated by:
%
\begin{center}
\begin{tabular}{l}
|\ifchilddoc|\\
|\addtocounter{page}{-1}|\\
\textit{code for banner page}\\
|\newpage|\\
|\||fi|
\end{tabular}
\end{center}
%
Here one could write a message such as:
\begin{center}
|This is the part \childdocname{} of \childdocjob{}.|
\end{center}

%%%%%%%%%%%%%%%%%%%%%%%%%%%%%%%%%%%%%%%%%%%%%%%%%%%%%%%%%%%%%%%%%%%%%%%%%%%%%%%%
\subsection{Flags}
\label{sec:flags}

The package makes it easy to generate different versions
of the main or child documents.
To this end compilation flags can be defined
and assigned different default values.
They will be particularly useful in conjunction
with the forwarding mechanism described in \secref{sec:forward}.

For example, it may be useful to have a flag |\version|
which can be set to |draft| or |final|.
The document source will contain some conditional code
depending on the value of |\version|.
Suppose further, the flag should default to |final| for the main file
and to |draft| for child files
which is a natural assignment for editing the document.
This is achieved by placing the following code
in the preamble of the main document
(below the |\childdocmain| directive):
%
\begin{center}
\begin{tabular}{l}
|\ifchilddoc|\\
|\providecommand{\version}{draft}|\\
|\||else|\\
|\providecommand{\version}{final}|\\
|\||fi|
\end{tabular}
\end{center}
%
The definition by |\providecommand| makes sure
that previous definitions are not overwritten.
Further statements |\providecommand{\version}{...}|
can thus be added before the above code to override it.

For the main file, one might add a line
(between |\childdocmain| and the above block)
%
\begin{center}
|%\ifchilddoc\||else\providecommand{\version}{draft}\||fi|
\end{center}
%
which can be uncommented to produce a draft version.
Likewise one can add a line to the very top of a child file
(above the |\childdocof{|\textit{main}|}| directive)
%
\begin{center}
|%\providecommand{\version}{final}|
\end{center}
%
which can be uncommented to produce the final version of this child document.

%%%%%%%%%%%%%%%%%%%%%%%%%%%%%%%%%%%%%%%%%%%%%%%%%%%%%%%%%%%%%%%%%%%%%%%%%%%%%%%%
\subsection{Forwarding}
\label{sec:forward}

Different versions of the main or child documents
using compilation flags as described in \secref{sec:flags}
can be (permanently) stored in different files
for convenient compilation, viewing and distribution.
To this end, the package defines a command
to pass on compilation to a different file:

%%%%%%%%%%%%%%%%%%%%%%%%%%%%%%%%%%%%%%%%
\DescribeMacro{\childdocforward}
The command |\childdocforward| redirects processing to
another source file:
%
\begin{center}
\begin{tabular}{l}
|\input{childdoc.def}|\\
|\childdocforward[|\textit{main}|]{|\textit{dest}|}|\\
\end{tabular}
\end{center}
%
The argument \textit{dest} is the destination file
(without extension).
It should be the main file or one of the child files.
Note that further \textsf{childdoc} directives
such as |\childdocof| and |\childdocforward|
in the indicated file will be processed in this form.
The optional argument \textit{main}
passes on directly to the main file \textit{main}
while pretending to compile the child \textit{dest}.
This form behaves as if \textit{dest}
issues |\childdocof{|\textit{main}|}| right away,
and no further \textsf{childdoc} directives will be processed.

%%%%%%%%%%%%%%%%%%%%%%%%%%%%%%%%%%%%%%%%
\DescribeMacro{\...prefix}
In the alternative form |\childdocforwardprefix|,
%
\begin{center}
\begin{tabular}{l}
|\input{childdoc.def}|\\
|\childdocforwardprefix[|\textit{main}|]{|\textit{prefix}|}{|\textit{dest}|}|
\end{tabular}
\end{center}
%
the destination file is determined by a pattern
depending on the current file:
To make this work, the current file must be called
`{\textit{prefix}\hspace{0.2em}\textit{suffix}}'
with \textit{prefix} matching precisely the argument.
Processing is then passed on to the file
`{\textit{dest}\hspace{0.2em}\textit{suffix}}'.
Surely, the same effect is achieved by
directly specifying the
argument `{\textit{dest}\hspace{0.2em}\textit{suffix}}'
in the first form.
However, that requires to set up a different file
for each child. With the alternative form of the command
all these files can have exactly the same content
which simplifies setting them up and maintaining them.

For example, the following file |draft.tex|
with a compilation flag |\version| as described in \secref{sec:flags}
compiles the main document as a draft:
%
\begin{center}
\begin{tabular}{l}
|\def\version{draft}|\\
|\input{childdoc.def}|\\
|\childdocforward{|\textit{main}|}|
\end{tabular}
\end{center}
%
Likewise, the following files |final|\textit{nn}|.tex|
compile the final version of the child document
|child|\textit{nn}|.tex|:
%
\begin{center}
\begin{tabular}{l}
|\def\version{final}|\\
|\input{childdoc.def}|\\
|\childdocforwardprefix{final}{child}|
\end{tabular}
\end{center}
%

Note that when several versions of a main file and/or of each child file
are to be generated, it may be convenient to set up a |Makefile| or
shell script to automatise the process.

%%%%%%%%%%%%%%%%%%%%%%%%%%%%%%%%%%%%%%%%%%%%%%%%%%%%%%%%%%%%%%%%%%%%%%%%%%%%%%%%
\subsection{Command Line Processing}
\label{sec:commandline}

The effect of redirection files can also be achieved by invoking
the \LaTeX{} compiler with a more elaborate command line.
Most conveniently this should be done as part
of a shell script or a |Makefile|.

When using \textsf{childdoc} in the main file, the following
command lines effectively perform a redirection
(note that depending on the shell being used,
backslashes may have to be doubled: `|\|' $\to$ `|\\|'):
%
\begin{center}
|... -jobname "|\textit{target}|" |\\|"|[\textit{flags}]%
|\input{childdoc.def}\childdocforward[|\textit{main}|]{|\textit{dest}|}"|
\end{center}
%
Here \textit{target} is the name of the output file,
\textit{main} is the name of the main file
and \textit{dest} is the name of the main or child file to be processed
(all filenames without extensions).
The optional argument \textit{main} can be omitted
if \textit{main} matches \textit{dest}.
Optionally, compilation \textit{flags} can be defined via |\def| commands.
This command line makes the \TeX{} engine believe
it is compiling the file \textit{target}
whose content is specified as the latter parameter.
The provided code then forwards the processing to
\textit{main} or \textit{dest} as described in \secref{sec:forward}.

%%%%%%%%%%%%%%%%%%%%%%%%%%%%%%%%%%%%%%%%%%%%%%%%%%%%%%%%%%%%%%%%%%%%%%%%%%%%%%%%
\subsection{Include by Input}
\label{sec:input}

Including child documents by |\include| has some restrictions by design.
Most notably, the content of a child document always occupies
its own set of pages; pages cannot be shared between child documents.
Usually, this behaviour makes perfect sense
because each child document contain an essential part of the document.
However, in some situations it may be desirable to compose
a document from a collection of parts
without having mandatory page breaks between then.
For this case, the package
provides a mechanism to include parts
by |\input| which can also be processed individually.
However, by construction this mechanism
requires manual handling of the content to be output.

%%%%%%%%%%%%%%%%%%%%%%%%%%%%%%%%%%%%%%%%
\DescribeMacro{\ifchilddocmanual}
The main file should be prepared as usual, see \secref{sec:include}.
However, the document body must make a distinction
between processing of an individual part and of the main document, e.g.:
%
\begin{center}
\begin{tabular}{l}
|\ifchilddocmanual|\\
|\input{\childdocname}|\\
|\||else|\\
\textit{document body with }|\input{|\textit{part}|}|\\
|\||fi|
\end{tabular}
\end{center}
%
The conditional |\ifchilddocmanual| is true whenever
a part to be included by |\input| is being compiled,
and the name of the part is stored in |\childdocname|.

%%%%%%%%%%%%%%%%%%%%%%%%%%%%%%%%%%%%%%%%
\DescribeMacro{\childdocby}
Each part to be included by |\input| should start with:
%
\begin{center}
\begin{tabular}{l}
|\input{childdoc.def}|\\
|\childdocby{|\textit{main}|}|\\
\end{tabular}
\end{center}
%
The directive |\childdocby| is similar to |\childdocof|
described in \secref{sec:include},
but the subsequent selection of content must be done manually.
To that end, both |\ifchilddoc| and |\ifchilddocmanual|
will be true upon processing of a part,
and the name of the part is stored in |\childdocname|.
Note that |\jobname| will be set to the filename of the current part
so that each part receives an individual |.aux| file
that does not interfere with the |.aux| file(s) of the main document.
This behaviour can be altered by the alternative form
|\childdocby[*]{|\textit{main}|}| (with a non-empty optional argument)
which uses the |.aux| file of the main document
by setting |\jobname| to \textit{main}.

%%%%%%%%%%%%%%%%%%%%%%%%%%%%%%%%%%%%%%%%%%%%%%%%%%%%%%%%%%%%%%%%%%%%%%%%%%%%%%%%
\subsection{Driver Development}
\label{sec:driver}

The \textsf{childdoc} mechanism can also be use for the development
of definition files such as \LaTeX{} styles or classes.
This case differs from the above setup with multiple parts
included by |\include| in that no |\includeonly| should be invoked.
This can be achieved by starting the include file
(before |\ProvidesPackage|) with:
%
\begin{center}
\begin{tabular}{l}
|\input{childdoc.def}|\\
|\childdocforward{|\textit{main}|}|\\
\end{tabular}
\end{center}
%
or alternatively with:
%
\begin{center}
\begin{tabular}{l}
|\input{childdoc.def}|\\
|\childdocby{|\textit{main}|}|\\
\end{tabular}
\end{center}
%
Both forms have slightly different effects as described above.
The main file is prepared as usual, see \secref{sec:include}.

%%%%%%%%%%%%%%%%%%%%%%%%%%%%%%%%%%%%%%%%%%%%%%%%%%%%%%%%%%%%%%%%%%%%%%%%%%%%%%%%
\subsection{Legacy Detection}
\label{sec:detection}

The directive |\childdocmain| in the main file can detect
whether the complete document or merely a child is to be compiled
even without using the directive |\childdocof|.
This method is deprecated because it is less robust
and there is no compelling reason to use it;
it is merely provided for backward compatibility
and it may be removed in future versions.

If the detection mechanism is to be used,
it is mandatory to correctly specify
the filename of the main file as the argument of |\childdocmain|:
%
\begin{center}
\begin{tabular}{l}
|\input{childdoc.def}|\\
|\childdocmain{|\textit{main}|}|\\
\end{tabular}
\end{center}
%
If |\jobname| does not match the argument \textit{main} of |\childdocmain|,
it is assumed that |\jobname| points to the child file to be compiled.
When using |\childdocmain| with the main file specified as argument,
it suffices to start a child file
with just |\input{|\textit{main}|}|
without loading of the package and using |\childdocof|.
If instead all processing is done
with the appropriate \textsf{childdoc} directives,
the argument of \textit{main} of |\childdocmain| can be empty.

An alternative version of the command line processing described
in \secref{sec:commandline} using the detection mechanism reads:
%
\begin{center}
|... -jobname "|\textit{target}|" "|[\textit{flags}]%
[|\def\jobname{|\textit{dest}|}|]|\input{|\textit{main}|}"|
\end{center}

%%%%%%%%%%%%%%%%%%%%%%%%%%%%%%%%%%%%%%%%%%%%%%%%%%%%%%%%%%%%%%%%%%%%%%%%%%%%%%%%
\subsection{Manual Code}
\label{sec:manual}

In case one cannot be certain whether the definitions file |childdoc.def|
is installed on the target \TeX{} distribution
and one prefers not to ship it,
it is conceivable to paste a few relevant commands into the sources.

To that end, drop all statements |\input{childdoc.def}|
and perform the replacements as outlined below.
Instead of |\childdocmain{|\textit{main}|}| add the following code
to the top of the main file:
%
\begin{center}
\begin{tabular}{l}
|\||ifdefined\childdocname\endinput\||fi\newif\ifchilddoc|\\
|\edef\childdocname{\scantokens\expandafter{\jobname\noexpand}}|\\
|\def\childdocmain{|\textit{main}|}\||ifx\childdocmain\childdocname\||else|\\
|\childdoctrue\includeonly{\childdocname}\let\jobname\childdocmain\||fi|\\
\end{tabular}
\end{center}
%
Instead of |\childdocof{|\textit{main}|}| just include the main file
at the top of each child file:
%
\begin{center}
|\input{|\textit{main}|}|
\end{center}
%
A simple redirection |\childdocforward{|\textit{dest}|}| is achieved by:
%
\begin{center}
|\def\jobname{|\textit{dest}|}\input{\jobname}|
\end{center}
%
The redirection with prefix
|\childdocforwardprefix[|\textit{prefix}|]{|\textit{dest}|}|
is accomplished by:
%
\begin{center}
\begin{tabular}{l}
|{\edef\jobname{\scantokens\expandafter{\jobname\noexpand}}|\\
|\def\redirectjob |\textit{prefix}|#1~~~{\gdef\jobname{|\textit{dest}|#1}}|\\
|\expandafter\redirectjob\jobname~~~}\input{\jobname}|
\end{tabular}
\end{center}

In an alternative approach,
child documents can be compiled by a specific command line
without additional code or specific definitions:
%
\begin{center}
|... -jobname "|\textit{target}|" "|[\textit{flags}]%
|\includeonly{|\textit{dest}|}\input{|\textit{main}|}"|
\end{center}
%

%%%%%%%%%%%%%%%%%%%%%%%%%%%%%%%%%%%%%%%%%%%%%%%%%%%%%%%%%%%%%%%%%%%%%%%%%%%%%%%%
%%%%%%%%%%%%%%%%%%%%%%%%%%%%%%%%%%%%%%%%%%%%%%%%%%%%%%%%%%%%%%%%%%%%%%%%%%%%%%%%
\section{Information}

%%%%%%%%%%%%%%%%%%%%%%%%%%%%%%%%%%%%%%%%%%%%%%%%%%%%%%%%%%%%%%%%%%%%%%%%%%%%%%%%
\subsection{Copyright}

Copyright \copyright{} 2017--2018 Niklas Beisert

This work may be distributed and/or modified under the
conditions of the \LaTeX{} Project Public License, either version 1.3
of this license or (at your option) any later version.
The latest version of this license is in
  \url{http://www.latex-project.org/lppl.txt}
and version 1.3 or later is part of all distributions of \LaTeX{}
version 2005/12/01 or later.

This work has the LPPL maintenance status `maintained'.

The Current Maintainer of this work is Niklas Beisert.

This work consists of the files |README.txt|, |childdoc.ins| and |childdoc.dtx|
as well as the derived files |childdoc.def|, |cdocsamp.tex|
with |cdocsch1.tex|, |cdocsch2.tex|, |cdocspt3.tex|, |cdocspt4.tex|,
|cdocsdrf.tex|, |cdocsfn1.tex|, |cdocsfn2.tex|
as well as |childdoc.pdf|.

%%%%%%%%%%%%%%%%%%%%%%%%%%%%%%%%%%%%%%%%%%%%%%%%%%%%%%%%%%%%%%%%%%%%%%%%%%%%%%%%
\subsection{Files and Installation}

The package consists of the files:
%
\begin{center}
\begin{tabular}{ll}
    |README.txt|   & readme file \\
    |childdoc.ins| & installation file \\
    |childdoc.dtx| & source file \\
    |childdoc.def| & definition file \\
    |cdocsamp.tex| & sample main file \\
    |cdocsch1.tex| & sample include file \\
    |cdocsch2.tex| & sample include file \\
    |cdocspt3.tex| & sample part file \\
    |cdocspt4.tex| & sample part file \\
    |cdocsdrf.tex| & sample redirection file \\
    |cdocsfn1.tex| & sample redirection file \\
    |cdocsfn2.tex| & sample redirection file \\
    |childdoc.pdf| & manual
\end{tabular}
\end{center}
%
The distribution consists of the files
|README.txt|, |childdoc.ins| and |childdoc.dtx|.
%
\begin{itemize}
\item
Run (pdf)\LaTeX{} on |childdoc.dtx|
to compile the manual |childdoc.pdf| (this file).
\item
Run \LaTeX{} on |childdoc.ins| to create the definitions file |childdoc.def|
and the sample |cdocsamp.tex| with include files
|cdocsch1.tex|, |cdocsch2.tex|, |cdocspt3.tex|, |cdocspt4.tex|,
|cdocsdrf.tex|, |cdocsfn1.tex|, |cdocsfn2.tex|.
Then copy the file |childdoc.def| to an appropriate directory of your \LaTeX{}
distribution, e.g.\ \textit{texmf-root}|/tex/latex/childdoc|.
\end{itemize}

%%%%%%%%%%%%%%%%%%%%%%%%%%%%%%%%%%%%%%%%%%%%%%%%%%%%%%%%%%%%%%%%%%%%%%%%%%%%%%%%
\subsection{Related CTAN Packages}

There are several other packages which offer a similar functionality:
%
\begin{itemize}
\item
The packages
\href{http://ctan.org/pkg/docmute}{\textsf{docmute}},
\href{http://ctan.org/pkg/includex}{\textsf{includex}} and
\href{http://ctan.org/pkg/standalone}{\textsf{standalone}}
provide commands to include only the document body of
a child file thus allowing both files to be compiled individually.
\item
The packages \href{http://ctan.org/pkg/subdocs}{\textsf{subdocs}}
and \href{http://ctan.org/pkg/subfiles}{\textsf{subfiles}}
provide structures in which the main and child documents can be
encapsulated and allowing them to be compiled individually.
The inclusion mechanism is different from the conventional |\include|.
\item
The package \href{http://ctan.org/pkg/combine}{\textsf{combine}}
is an elaborate solution to combine several documents into one.
\end{itemize}
%
See also the CTAN topic \href{http://ctan.org/topic/subdocs}{\textsf{subdocs}}
for further related packages.
The present package differs from the above solutions in that
a document structure constructed with the conventional |\include| mechanism
just needs two extra commands at the top of every file
such that all constituent files can be compiled individually.

%%%%%%%%%%%%%%%%%%%%%%%%%%%%%%%%%%%%%%%%%%%%%%%%%%%%%%%%%%%%%%%%%%%%%%%%%%%%%%%%
%\subsection{Feature Suggestions}
%
%The following is a list of features which may be useful for future
%versions of this package:
%%
%\begin{itemize}
%\item
%\ldots
%\end{itemize}

%%%%%%%%%%%%%%%%%%%%%%%%%%%%%%%%%%%%%%%%%%%%%%%%%%%%%%%%%%%%%%%%%%%%%%%%%%%%%%%%
\subsection{Revision History}

%%%%%%%%%%%%%%%%%%%%%%%%%%%%%%%%%%%%%%%%
\paragraph{v2.0:} 2018/12/30

\begin{itemize}
\item
immediate forward processing
\item
added |\childdocby| mechanism
\item
manual restructured
\end{itemize}

%%%%%%%%%%%%%%%%%%%%%%%%%%%%%%%%%%%%%%%%
\paragraph{v1.6:} 2018/01/17

\begin{itemize}
\item
application for development of include files
\item
corrections to manual
\end{itemize}

%%%%%%%%%%%%%%%%%%%%%%%%%%%%%%%%%%%%%%%%
\paragraph{v1.5:} 2017/05/21

\begin{itemize}
\item
more complete structuring introduced
\item
|\childdocof| introduced
\item
|\childdoc| renamed to |\childdocmain|
\item
|\childredirect| renamed to |\childdocforward| and |\childdocforwardprefix|
and functionality expanded
\end{itemize}

%%%%%%%%%%%%%%%%%%%%%%%%%%%%%%%%%%%%%%%%
\paragraph{v1.0:} 2017/04/27

\begin{itemize}
\item
manual and install package
\item
first version published on CTAN
\end{itemize}

%%%%%%%%%%%%%%%%%%%%%%%%%%%%%%%%%%%%%%%%
\paragraph{v0.6:} 2017/04/26

\begin{itemize}
\item
redirection mechanism added
\end{itemize}

%%%%%%%%%%%%%%%%%%%%%%%%%%%%%%%%%%%%%%%%
\paragraph{v0.5:} 2017/04/26

\begin{itemize}
\item
functionality in definition file
\end{itemize}


%%%%%%%%%%%%%%%%%%%%%%%%%%%%%%%%%%%%%%%%%%%%%%%%%%%%%%%%%%%%%%%%%%%%%%%%%%%%%%%%
%%%%%%%%%%%%%%%%%%%%%%%%%%%%%%%%%%%%%%%%%%%%%%%%%%%%%%%%%%%%%%%%%%%%%%%%%%%%%%%%
%%%%%%%%%%%%%%%%%%%%%%%%%%%%%%%%%%%%%%%%%%%%%%%%%%%%%%%%%%%%%%%%%%%%%%%%%%%%%%%%
\appendix

\settowidth\MacroIndent{\rmfamily\scriptsize 000\ }

 \DocInput{childdoc.dtx}

\end{document}
%</driver>
% \fi
%
% %%%%%%%%%%%%%%%%%%%%%%%%%%%%%%%%%%%%%%%%%%%%%%%%%%%%%%%%%%%%%%%%%%%%%%%%%%%%%%
% %%%%%%%%%%%%%%%%%%%%%%%%%%%%%%%%%%%%%%%%%%%%%%%%%%%%%%%%%%%%%%%%%%%%%%%%%%%%%%
% \section{Sample}
%\iffalse
%<*samplemain>
%\fi
%
% The following presents a sample document
% with two chapters, two parts, a title page,
% a compile flag as well as three forwarding files to set the flag.
% It consists of eight |.tex| files:
% \begin{center}
% \begin{tabular}{ll}
% |cdocsamp.tex|&main file\\
% |cdocsch1.tex|&include file for chapter 1\\
% |cdocsch2.tex|&include file for chapter 2\\
% |cdocspt3.tex|&include file for part 3\\
% |cdocspt4.tex|&include file for part 4\\
% |cdocsdrf.tex|&forwarding file for main file in draft mode\\
% |cdocsfi1.tex|&forwarding file for final version of chapter 1\\
% |cdocsfi2.tex|&forwarding file for final version of chapter 2\\
% \end{tabular}
% \end{center}
% Each of the eight files can be compiled directly by the \LaTeX{} compiler.
%
% %%%%%%%%%%%%%%%%%%%%%%%%%%%%%%%%%%%%%%
% \paragraph{Main File.}
%
% The main file is called |cdocsamp.tex|.
%
% Load the \textsf{childdoc} definitions and
% declare the filename for the main document:
%    \begin{macrocode}
\input{childdoc.def}
\childdocmain{}
%    \end{macrocode}

% Optional override for |\version| flag:
%    \begin{macrocode}
%%\ifchilddoc\else\providecommand{\version}{draft}\fi
%    \end{macrocode}

% Define the default values for the |\version| flag
% (|final| for the main file and |draft| for childs):
%    \begin{macrocode}
\ifchilddoc
\providecommand{\version}{draft}
\else
\providecommand{\version}{final}
\fi
%    \end{macrocode}

% Load the standard document class:
%    \begin{macrocode}
\documentclass[12pt]{article}
%    \end{macrocode}

% Start the document body:
%    \begin{macrocode}
\begin{document}
%    \end{macrocode}

% Declare a title page.
% Print title, part of document being processed and version flag:
%    \begin{macrocode}
\addtocounter{page}{-1}
\begin{center}
{\LARGE\bfseries{}childdoc example\par}
\vspace{1cm}
\ifchilddoc
\ifchilddocmanual part\else chapter\fi:
`\childdocname' of `\childdocjob'\par
\else
main document: `\childdocjob'\par
\fi
version: \version\par
\end{center}
\newpage
%    \end{macrocode}

% Manually include selected file,
% otherwise process as usual:
%    \begin{macrocode}
\ifchilddocmanual
\section*{part `\childdocname'}
\input{\childdocname}
\else
%    \end{macrocode}

% Include the two chapters:
%    \begin{macrocode}
\include{cdocsch1}
\include{cdocsch2}
%    \end{macrocode}

% Include the two parts unless only chapters should be displayed:
%    \begin{macrocode}
\ifchilddoc\else
\section{part three}
\input{cdocspt3}
\section{part four}
\input{cdocspt4}
\fi
%    \end{macrocode}

% Process as usual until here:
%    \begin{macrocode}
\fi
%    \end{macrocode}

% End of document body:
%    \begin{macrocode}
\end{document}
%    \end{macrocode}
%\iffalse
%</samplemain>
%\fi
%
% %%%%%%%%%%%%%%%%%%%%%%%%%%%%%%%%%%%%%%
% \paragraph{Chapter Include Files.}
%
% The include files are called |cdocsch1.tex| and |cdocsch2.tex|.
%
%\iffalse
%<*samplechap1|samplechap2>
%\fi

% Optional override for |\version| flag:
%    \begin{macrocode}
%%\providecommand{\version}{final}
%    \end{macrocode}

% Include the main document:
%    \begin{macrocode}
\input{childdoc.def}
\childdocof{cdocsamp}
%    \end{macrocode}

%\iffalse
%</samplechap1|samplechap2>
%\fi
%
%\iffalse
%<*samplechap1>
%\fi
% Some text for chapter 1:
%    \begin{macrocode}
\section{one}
some text in chapter one
%    \end{macrocode}

%\iffalse
%</samplechap1>
%\fi
% Some text for chapter 2:
%\iffalse
%<*samplechap2>
%\fi
%    \begin{macrocode}
\section{two}
more text in chapter two
%    \end{macrocode}

%\iffalse
%</samplechap2>
%\fi
%
% %%%%%%%%%%%%%%%%%%%%%%%%%%%%%%%%%%%%%%
% \paragraph{Part Include Files.}
%
% The include files are called |cdocspt3.tex| and |cdocspt4.tex|.
%
%\iffalse
%<*samplepart3|samplepart4>
%\fi

% Optional override for |\version| flag:
%    \begin{macrocode}
%%\providecommand{\version}{final}
%    \end{macrocode}

% Include the main document:
%    \begin{macrocode}
\input{childdoc.def}
\childdocby{cdocsamp}
%    \end{macrocode}

%\iffalse
%</samplepart3|samplepart4>
%\fi
%
%\iffalse
%<*samplepart3>
%\fi
% Some text for part 3:
%    \begin{macrocode}
some text in part three
%    \end{macrocode}

%\iffalse
%</samplepart3>
%\fi
% Some text for part 4:
%\iffalse
%<*samplepart4>
%\fi
%    \begin{macrocode}
more text in part four
%    \end{macrocode}

%\iffalse
%</samplepart4>
%\fi
%
% %%%%%%%%%%%%%%%%%%%%%%%%%%%%%%%%%%%%%%
% \paragraph{Forwarding for a Complete Draft.}
%
% The following forwarding file |cdocsdrf.tex|
% compiles the main document in draft mode:
%\iffalse
%<*sampledraft>
%\fi
%    \begin{macrocode}
\def\version{draft}
\input{childdoc.def}
\childdocforward{cdocsamp}
%    \end{macrocode}

%\iffalse
%</sampledraft>
%\fi
%
% %%%%%%%%%%%%%%%%%%%%%%%%%%%%%%%%%%%%%%
% \paragraph{Forwarding for Final Version of the Chapters.}
%
% The following forwarding files |cdocsfn1.tex| and |cdocsfn2.tex|
% (with identical content)
% compile the final versions of the child documents
% |cdocsch1.tex| and |cdocsch2.tex|, respectively:
%\iffalse
%<*samplefinal>
%\fi
%    \begin{macrocode}
\def\version{final}
\input{childdoc.def}
\childdocforwardprefix[cdocsamp]{cdocsfn}{cdocsch}
%    \end{macrocode}

%\iffalse
%</samplefinal>
%\fi
%
% %%%%%%%%%%%%%%%%%%%%%%%%%%%%%%%%%%%%%%
% \paragraph{Command Line Processing.}
%
% The following three command lines generate the output files
% |cdocscld|, |cdocscl1| and |cdocscl2|
% which should be identical to
% |cdocsdrf|, |cdocsch1| and |cdocsfn2|, respectively:
% \begin{center}
% \begin{tabular}{l}
% |latex -jobname cdocscld \|\\
% |  "\def\version{draft}\input{childdoc.def}\childdocforward{cdocsamp}"|\\
% |latex -jobname cdocscl1 \|\\
% |  "\input{childdoc.def}\childdocforward[cdocsamp]{cdocsch1}"|\\
% |latex -jobname cdocscl2 \|\\
% |  "\def\version{final}\input{childdoc.def}\childdocforward{cdocsch2}"|
% \end{tabular}
% \end{center}
% Note that the trailing backslash on each first line
% merely continues the input to the second line
% (for convenient cut ant paste).
% Furthermore, the command |latex| can be replaced by any
% of its alternative versions such as |pdflatex|.
%
% %%%%%%%%%%%%%%%%%%%%%%%%%%%%%%%%%%%%%%%%%%%%%%%%%%%%%%%%%%%%%%%%%%%%%%%%%%%%%%
% %%%%%%%%%%%%%%%%%%%%%%%%%%%%%%%%%%%%%%%%%%%%%%%%%%%%%%%%%%%%%%%%%%%%%%%%%%%%%%
% \section{Implementation}
%\iffalse
%<*package>
%\fi
%
% This section describes the definitions file |childdoc.def|.

% The definitions cannot be loaded using |\usepackage| or |\RequirePackage|
% which has a mechanism to prevent loading a style file more than once.
% When loading the definitions by means of |\input|
% multiple instances have to be prevented manually:
%\iffalse
%This code needs to be before the `\ProvidesFile' directive
%which is defined at the beginning of this file.
%Therefore it is also placed there and commented out here.
%</package>
%<*discard>
%\fi
%    \begin{macrocode}
\ifdefined\childdocmain\endinput\fi
%    \end{macrocode}
%\iffalse
%</discard>
%<*package>
%\fi
%
% \macro{\ifchilddoc}
% \macro{\ifchilddocmanual}
% The conditional |\ifchilddoc| tells whether a
% child (true) or main (false) document is being compiled.
% The conditional |\ifchilddocmanual| tells whether
% the |\includeonly| mechanism is used (false) or
% the selection of child files must be performed manually (true).
% The definitions initialise to false:
%    \begin{macrocode}
\newif\ifchilddoc
\newif\ifchilddocmanual
%    \end{macrocode}

% \macro{\childdocname}
% \macro{\childdocjob}
% The macro |\childdocname| stores the name of the main document
% to be compiled. The macro |\childdocjob| stores the name of
% the document on which the \LaTeX{} compiler was originally invoked.
% The content of |\jobname| cannot be compared
% to filenames specified in the source due to different catcodes.
% The following code rescans |\jobname|, stores the result
% in |\childdocname| and saves a copy in |\childdocjob|:
%    \begin{macrocode}
\edef\childdocname{\scantokens\expandafter{\jobname\noexpand}}
\let\childdocjob\childdocname
%    \end{macrocode}

% \macro{\childdocdisable}
% The macro |\childdocdisable| prevents the main file
% from being processed more than once.
% At this stage, the main document command |\childdocmain|
% is assumed to be called once again where it should do nothing.
% Any subsequent call to it should prevent
% a secondary processing of the main document
% It overwrites the forwarding commands
% |\childdocof| and |\childdocforward|
% with empty macros to prevent further inclusions of the main document:
%    \begin{macrocode}
\newcommand{\childdocdisable}
{
  \renewcommand{\childdocmain}[1]{\renewcommand{\childdocmain}[1]{\endinput}}
  \renewcommand{\childdocof}[1]{}
  \renewcommand{\childdocby}[2][]{}
  \renewcommand{\childdocforward}[2][]{}
  \renewcommand{\childdocdisable}{}
}
%    \end{macrocode}

% \macro{\childdocmain}
% The macro |\childdocmain| is to be called at the top of the main file
% with nothing or the main filename (without extension) as argument.
% First, it breaks loops.
% If the argument is not empty and does not match |\childdocname|
% (which is set by the first inclusion of |childdoc.def|),
% |\ifchilddoc| is set to true, |\includeonly| is applied to the child file
% and |\jobname| is set to the main file
% (for proper handling of |.aux| files):
%    \begin{macrocode}
\newcommand{\childdocmain}[1]
{
  \childdocdisable\childdocmain{}
  \if?#1?\else
    \begingroup
      \def\childdoctmp{#1}
      \ifx\childdoctmp\childdocname
        \def\childdoctmp{}
      \else
        \def\childdoctmp
        {
          \childdoctrue
          \includeonly{\childdocname}
          \def\childdocjob{#1}
          \def\jobname{#1}
        }
      \fi
      \expandafter
    \endgroup
    \childdoctmp
  \fi
}
%    \end{macrocode}

% \macro{\childdocof}
% The command |\childdocof| redirects
% compilation to the main file |#1|.
%    \begin{macrocode}
\newcommand{\childdocof}[1]
{
  \childdocdisable
  \childdoctrue
  \includeonly{\childdocname}
  \def\jobname{#1}
  \def\childdocjob{#1}
  \input{#1}
}
%    \end{macrocode}

% \macro{\childdocby}
% The command |\childdocby| ....
%    \begin{macrocode}
\newcommand{\childdocby}[2][]
{
  \childdocdisable
  \childdoctrue
  \childdocmanualtrue
  \if?#1?\else
    \def\jobname{#2}
  \fi
  \def\childdocjob{#2}
  \input{#2}
  \endinput
}
%    \end{macrocode}

% \macro{\childdocforward}
% The command |\childdocforward| redirects
% compilation to the main file or
% (if the optional argument is given) a child file.
% Parameters are set as if the main file
% or a child file starting with |\childdocof| was compiled.
% Then compilation is handed over to the main file:
%    \begin{macrocode}
\newcommand{\childdocforward}[2][]
{
  \begingroup
    \if?#1?
      \def\childdoctmp
      {
        \def\childdocname{#2}
        \def\childdocjob{#2}
        \def\jobname{#2}
        \input{#2}
        \endinput
      }
    \else
      \def\childdoctmp
      {
        \childdocdisable
        \def\childdocname{#2}
        \childdoctrue
        \includeonly{#2}
        \def\childdocjob{#1}
        \def\jobname{#1}
        \input{#1}
        \endinput
      }
    \fi
    \expandafter
  \endgroup
  \childdoctmp
}
%    \end{macrocode}

% \macro{\childdocforwardprefix}
% The command |\childdocforwardprefix| redirects
% compilation to the main or a child file by means of a pattern.
% The prefix |#1| in the current filename is replaced by |#2|
% and the suffix of the current filename is kept
% (it is assumed that the filename does not contain the substring `|~~~|'
% which is used as a delimiter).
% Compilation is handed over to the new file by |\childdocforward|:
%    \begin{macrocode}
\newcommand{\childdocforwardprefix}[3][]
{
  \begingroup
    \def\childdocextract #2##1~~~{\def\childdoctmp{\childdocforward[#1]{#3##1}}}
    \expandafter\childdocextract\childdocname~~~
    \expandafter
  \endgroup
  \childdoctmp
}
%    \end{macrocode}

% \macro{\childdoc}
% The deprecated macro |\childdoc| is a legacy version of |\childdocmain|:
%    \begin{macrocode}
\newcommand{\childdoc}{\childdocmain}
%    \end{macrocode}

% \macro{\childdocredirect}
% The deprecated macro |\childdocredirect| is a legacy version
% of |\childdocforward| and |\childdocforwardprefix|:
%    \begin{macrocode}
\newcommand{\childdocredirect}[2][]
{
  \begingroup
    \if?#1?
      \def\childdoctmp{\childdocforward{#2}}
    \else
      \def\childdoctmp{\childdocforwardprefix{#1}{#2}}
    \fi
    \expandafter
  \endgroup
  \childdoctmp
}
%    \end{macrocode}

%\iffalse
%</package>
%\fi
%
\endinput
\childdocforward{cdocsch2}"|
% \end{tabular}
% \end{center}
% Note that the trailing backslash on each first line
% merely continues the input to the second line
% (for convenient cut ant paste).
% Furthermore, the command |latex| can be replaced by any
% of its alternative versions such as |pdflatex|.
%
% %%%%%%%%%%%%%%%%%%%%%%%%%%%%%%%%%%%%%%%%%%%%%%%%%%%%%%%%%%%%%%%%%%%%%%%%%%%%%%
% %%%%%%%%%%%%%%%%%%%%%%%%%%%%%%%%%%%%%%%%%%%%%%%%%%%%%%%%%%%%%%%%%%%%%%%%%%%%%%
% \section{Implementation}
%\iffalse
%<*package>
%\fi
%
% This section describes the definitions file |childdoc.def|.

% The definitions cannot be loaded using |\usepackage| or |\RequirePackage|
% which has a mechanism to prevent loading a style file more than once.
% When loading the definitions by means of |\input|
% multiple instances have to be prevented manually:
%\iffalse
%This code needs to be before the `\ProvidesFile' directive
%which is defined at the beginning of this file.
%Therefore it is also placed there and commented out here.
%</package>
%<*discard>
%\fi
%    \begin{macrocode}
\ifdefined\childdocmain\endinput\fi
%    \end{macrocode}
%\iffalse
%</discard>
%<*package>
%\fi
%
% \macro{\ifchilddoc}
% \macro{\ifchilddocmanual}
% The conditional |\ifchilddoc| tells whether a
% child (true) or main (false) document is being compiled.
% The conditional |\ifchilddocmanual| tells whether
% the |\includeonly| mechanism is used (false) or
% the selection of child files must be performed manually (true).
% The definitions initialise to false:
%    \begin{macrocode}
\newif\ifchilddoc
\newif\ifchilddocmanual
%    \end{macrocode}

% \macro{\childdocname}
% \macro{\childdocjob}
% The macro |\childdocname| stores the name of the main document
% to be compiled. The macro |\childdocjob| stores the name of
% the document on which the \LaTeX{} compiler was originally invoked.
% The content of |\jobname| cannot be compared
% to filenames specified in the source due to different catcodes.
% The following code rescans |\jobname|, stores the result
% in |\childdocname| and saves a copy in |\childdocjob|:
%    \begin{macrocode}
\edef\childdocname{\scantokens\expandafter{\jobname\noexpand}}
\let\childdocjob\childdocname
%    \end{macrocode}

% \macro{\childdocdisable}
% The macro |\childdocdisable| prevents the main file
% from being processed more than once.
% At this stage, the main document command |\childdocmain|
% is assumed to be called once again where it should do nothing.
% Any subsequent call to it should prevent
% a secondary processing of the main document
% It overwrites the forwarding commands
% |\childdocof| and |\childdocforward|
% with empty macros to prevent further inclusions of the main document:
%    \begin{macrocode}
\newcommand{\childdocdisable}
{
  \renewcommand{\childdocmain}[1]{\renewcommand{\childdocmain}[1]{\endinput}}
  \renewcommand{\childdocof}[1]{}
  \renewcommand{\childdocby}[2][]{}
  \renewcommand{\childdocforward}[2][]{}
  \renewcommand{\childdocdisable}{}
}
%    \end{macrocode}

% \macro{\childdocmain}
% The macro |\childdocmain| is to be called at the top of the main file
% with nothing or the main filename (without extension) as argument.
% First, it breaks loops.
% If the argument is not empty and does not match |\childdocname|
% (which is set by the first inclusion of |childdoc.def|),
% |\ifchilddoc| is set to true, |\includeonly| is applied to the child file
% and |\jobname| is set to the main file
% (for proper handling of |.aux| files):
%    \begin{macrocode}
\newcommand{\childdocmain}[1]
{
  \childdocdisable\childdocmain{}
  \if?#1?\else
    \begingroup
      \def\childdoctmp{#1}
      \ifx\childdoctmp\childdocname
        \def\childdoctmp{}
      \else
        \def\childdoctmp
        {
          \childdoctrue
          \includeonly{\childdocname}
          \def\childdocjob{#1}
          \def\jobname{#1}
        }
      \fi
      \expandafter
    \endgroup
    \childdoctmp
  \fi
}
%    \end{macrocode}

% \macro{\childdocof}
% The command |\childdocof| redirects
% compilation to the main file |#1|.
%    \begin{macrocode}
\newcommand{\childdocof}[1]
{
  \childdocdisable
  \childdoctrue
  \includeonly{\childdocname}
  \def\jobname{#1}
  \def\childdocjob{#1}
  \input{#1}
}
%    \end{macrocode}

% \macro{\childdocby}
% The command |\childdocby| ....
%    \begin{macrocode}
\newcommand{\childdocby}[2][]
{
  \childdocdisable
  \childdoctrue
  \childdocmanualtrue
  \if?#1?\else
    \def\jobname{#2}
  \fi
  \def\childdocjob{#2}
  \input{#2}
  \endinput
}
%    \end{macrocode}

% \macro{\childdocforward}
% The command |\childdocforward| redirects
% compilation to the main file or
% (if the optional argument is given) a child file.
% Parameters are set as if the main file
% or a child file starting with |\childdocof| was compiled.
% Then compilation is handed over to the main file:
%    \begin{macrocode}
\newcommand{\childdocforward}[2][]
{
  \begingroup
    \if?#1?
      \def\childdoctmp
      {
        \def\childdocname{#2}
        \def\childdocjob{#2}
        \def\jobname{#2}
        \input{#2}
        \endinput
      }
    \else
      \def\childdoctmp
      {
        \childdocdisable
        \def\childdocname{#2}
        \childdoctrue
        \includeonly{#2}
        \def\childdocjob{#1}
        \def\jobname{#1}
        \input{#1}
        \endinput
      }
    \fi
    \expandafter
  \endgroup
  \childdoctmp
}
%    \end{macrocode}

% \macro{\childdocforwardprefix}
% The command |\childdocforwardprefix| redirects
% compilation to the main or a child file by means of a pattern.
% The prefix |#1| in the current filename is replaced by |#2|
% and the suffix of the current filename is kept
% (it is assumed that the filename does not contain the substring `|~~~|'
% which is used as a delimiter).
% Compilation is handed over to the new file by |\childdocforward|:
%    \begin{macrocode}
\newcommand{\childdocforwardprefix}[3][]
{
  \begingroup
    \def\childdocextract #2##1~~~{\def\childdoctmp{\childdocforward[#1]{#3##1}}}
    \expandafter\childdocextract\childdocname~~~
    \expandafter
  \endgroup
  \childdoctmp
}
%    \end{macrocode}

% \macro{\childdoc}
% The deprecated macro |\childdoc| is a legacy version of |\childdocmain|:
%    \begin{macrocode}
\newcommand{\childdoc}{\childdocmain}
%    \end{macrocode}

% \macro{\childdocredirect}
% The deprecated macro |\childdocredirect| is a legacy version
% of |\childdocforward| and |\childdocforwardprefix|:
%    \begin{macrocode}
\newcommand{\childdocredirect}[2][]
{
  \begingroup
    \if?#1?
      \def\childdoctmp{\childdocforward{#2}}
    \else
      \def\childdoctmp{\childdocforwardprefix{#1}{#2}}
    \fi
    \expandafter
  \endgroup
  \childdoctmp
}
%    \end{macrocode}

%\iffalse
%</package>
%\fi
%
\endinput
|\\
|\childdocforward{|\textit{main}|}|\\
\end{tabular}
\end{center}
%
or alternatively with:
%
\begin{center}
\begin{tabular}{l}
|% \iffalse
%
% childdoc.dtx Copyright (C) 2017-2018 Niklas Beisert
%
% This work may be distributed and/or modified under the
% conditions of the LaTeX Project Public License, either version 1.3
% of this license or (at your option) any later version.
% The latest version of this license is in
%   http://www.latex-project.org/lppl.txt
% and version 1.3 or later is part of all distributions of LaTeX
% version 2005/12/01 or later.
%
% This work has the LPPL maintenance status `maintained'.
%
% The Current Maintainer of this work is Niklas Beisert.
%
% This work consists of the files childdoc.dtx and childdoc.ins
% and the derived files childdoc.def and cdocsamp.tex with
% cdocsch1.tex, cdocsch2.tex, cdocsdrf.tex, cdocsfn1.tex, cdocsfn2.tex.
%
%<package>\ifdefined\childdocmain\endinput\fi
%<package>\ProvidesFile{childdoc.def}[2018/12/30 v2.0 child document driver]
%<samplemain>\ProvidesFile{cdocsamp.tex}[2018/12/30 v2.0 sample for childdoc]
%<*driver>
%\ProvidesFile{childdoc.drv}[2018/12/30 v2.0 childdoc reference manual file]
\PassOptionsToClass{10pt,a4paper}{article}
\documentclass{ltxdoc}

\usepackage[margin=35mm]{geometry}
\usepackage{hyperref}
\usepackage{hyperxmp}
\usepackage[usenames]{color}

\hypersetup{colorlinks=true}
\hypersetup{pdfstartview=FitH}
\hypersetup{pdfpagemode=UseNone}
\hypersetup{pdfsource={}}
\hypersetup{pdflang={en-UK}}
\hypersetup{pdfcopyright={Copyright 2017-2018 Niklas Beisert.
  This work may be distributed and/or modified under the
  conditions of the LaTeX Project Public License, either version 1.3
  of this license or (at your option) any later version.}}
\hypersetup{pdflicenseurl={http://www.latex-project.org/lppl.txt}}
\hypersetup{pdfcontactaddress={ETH Zurich, ITP, HIT K,
  Wolfgang-Pauli-Strasse 27}}
\hypersetup{pdfcontactpostcode={8093}}
\hypersetup{pdfcontactcity={Zurich}}
\hypersetup{pdfcontactcountry={Switzerland}}
\hypersetup{pdfcontactemail={nbeisert@itp.phys.ethz.ch}}
\hypersetup{pdfcontacturl={http://people.phys.ethz.ch/\xmptilde nbeisert/}}

\newcommand{\secref}[1]{\hyperref[#1]{section \ref*{#1}}}

\parskip1ex
\parindent0pt
\let\olditemize\itemize
\def\itemize{\olditemize\parskip0pt}

\begin{document}

\title{The \textsf{childdoc} Package}
\hypersetup{pdftitle={The childdoc Package}}
\author{Niklas Beisert\\[2ex]
  Institut f\"ur Theoretische Physik\\
  Eidgen\"ossische Technische Hochschule Z\"urich\\
  Wolfgang-Pauli-Strasse 27, 8093 Z\"urich, Switzerland\\[1ex]
  \href{mailto:nbeisert@itp.phys.ethz.ch}
  {\texttt{nbeisert@itp.phys.ethz.ch}}}
\hypersetup{pdfauthor={Niklas Beisert}}
\hypersetup{pdfsubject={Manual for the LaTeX2e Package childdoc}}
\date{30 December 2018, \textsf{v2.0}}
\maketitle

\begin{abstract}\noindent
\textsf{childdoc} is a \LaTeXe{} package
that enables the direct compilation
of document sections included by |\include|
to individual files.
\end{abstract}

\begingroup
\parskip0ex
\tableofcontents
\endgroup

%%%%%%%%%%%%%%%%%%%%%%%%%%%%%%%%%%%%%%%%%%%%%%%%%%%%%%%%%%%%%%%%%%%%%%%%%%%%%%%%
%%%%%%%%%%%%%%%%%%%%%%%%%%%%%%%%%%%%%%%%%%%%%%%%%%%%%%%%%%%%%%%%%%%%%%%%%%%%%%%%
\section{Introduction}

\LaTeX{} provides a mechanism to structure a large document (such as a book)
into a main file and several child files (containing the chapters)
using the |\include| command.
This mechanism is beneficial for documents
which span hundreds of pages in order to
make the source file(s) more manageable.
Moreover, compilation can be restricted to
selected child files by means of the |\includeonly| command.
The latter feature can be used to reduce the compilation time while editing
(this was significantly more useful in the earlier days of \LaTeX{})
or to generate a smaller document which is easier to navigate.
Another application of |\includeonly| is to generate
documents consisting of selected parts of the complete document.

However, there are a few drawbacks of the plain |\include| mechanism:
\begin{itemize}
\item
The child files cannot be compiled on their own,
they can only be compiled via the main file.
A naive editing environment
(such as a text editor with an option
to have the current file processed by \LaTeX)
may require one to switch to the main file before compiling;
attempting to compile the child file produces errors.
\item
The main file must be modified (each time)
to adjust the |\includeonly| command
to the present needs. This easily leaves the main file in a messy state.
\item
The generated document will always carry the filename
of the main document. This is inconvenient if
several child files are to be compiled and
to be kept for distribution.
\end{itemize}

The present package provides a simple interface
to make child files individually compilable by \LaTeX{}.
Compiling a child file then has the same effect as compiling
the main file with an |\includeonly| command
to select the appropriate child.
Moreover the generated document will carry the name of the child
rather than the main file.
This resolves all three above issues.

This feature is meant to make the editing of books,
thesis documents and lecture notes somewhat more convenient.
However, the package can also be used efficiently for
composing a series of documents (such as exercise sheets)
which are typically distributed individually.
It then assists the author in generating the individual documents
(potentially in different versions)
as well as a document containing the collected series.
Another application is in developing style files
or other kinds of included material
where compilation of the style file could redirect
to a sample or test file.

%%%%%%%%%%%%%%%%%%%%%%%%%%%%%%%%%%%%%%%%%%%%%%%%%%%%%%%%%%%%%%%%%%%%%%%%%%%%%%%%
%%%%%%%%%%%%%%%%%%%%%%%%%%%%%%%%%%%%%%%%%%%%%%%%%%%%%%%%%%%%%%%%%%%%%%%%%%%%%%%%
\section{Usage}

First of all, the package \textsf{childdoc} is \emph{not} a standard
\LaTeXe{} |.sty| style file! Therefore it needs to be invoked in
a non-standard way.

%%%%%%%%%%%%%%%%%%%%%%%%%%%%%%%%%%%%%%%%%%%%%%%%%%%%%%%%%%%%%%%%%%%%%%%%%%%%%%%%
\subsection{Included Files}
\label{sec:include}

%%%%%%%%%%%%%%%%%%%%%%%%%%%%%%%%%%%%%%%%
\DescribeMacro{\childdocmain}
To use the package, add the commands
\begin{center}
\begin{tabular}{l}
|% \iffalse
%
% childdoc.dtx Copyright (C) 2017-2018 Niklas Beisert
%
% This work may be distributed and/or modified under the
% conditions of the LaTeX Project Public License, either version 1.3
% of this license or (at your option) any later version.
% The latest version of this license is in
%   http://www.latex-project.org/lppl.txt
% and version 1.3 or later is part of all distributions of LaTeX
% version 2005/12/01 or later.
%
% This work has the LPPL maintenance status `maintained'.
%
% The Current Maintainer of this work is Niklas Beisert.
%
% This work consists of the files childdoc.dtx and childdoc.ins
% and the derived files childdoc.def and cdocsamp.tex with
% cdocsch1.tex, cdocsch2.tex, cdocsdrf.tex, cdocsfn1.tex, cdocsfn2.tex.
%
%<package>\ifdefined\childdocmain\endinput\fi
%<package>\ProvidesFile{childdoc.def}[2018/12/30 v2.0 child document driver]
%<samplemain>\ProvidesFile{cdocsamp.tex}[2018/12/30 v2.0 sample for childdoc]
%<*driver>
%\ProvidesFile{childdoc.drv}[2018/12/30 v2.0 childdoc reference manual file]
\PassOptionsToClass{10pt,a4paper}{article}
\documentclass{ltxdoc}

\usepackage[margin=35mm]{geometry}
\usepackage{hyperref}
\usepackage{hyperxmp}
\usepackage[usenames]{color}

\hypersetup{colorlinks=true}
\hypersetup{pdfstartview=FitH}
\hypersetup{pdfpagemode=UseNone}
\hypersetup{pdfsource={}}
\hypersetup{pdflang={en-UK}}
\hypersetup{pdfcopyright={Copyright 2017-2018 Niklas Beisert.
  This work may be distributed and/or modified under the
  conditions of the LaTeX Project Public License, either version 1.3
  of this license or (at your option) any later version.}}
\hypersetup{pdflicenseurl={http://www.latex-project.org/lppl.txt}}
\hypersetup{pdfcontactaddress={ETH Zurich, ITP, HIT K,
  Wolfgang-Pauli-Strasse 27}}
\hypersetup{pdfcontactpostcode={8093}}
\hypersetup{pdfcontactcity={Zurich}}
\hypersetup{pdfcontactcountry={Switzerland}}
\hypersetup{pdfcontactemail={nbeisert@itp.phys.ethz.ch}}
\hypersetup{pdfcontacturl={http://people.phys.ethz.ch/\xmptilde nbeisert/}}

\newcommand{\secref}[1]{\hyperref[#1]{section \ref*{#1}}}

\parskip1ex
\parindent0pt
\let\olditemize\itemize
\def\itemize{\olditemize\parskip0pt}

\begin{document}

\title{The \textsf{childdoc} Package}
\hypersetup{pdftitle={The childdoc Package}}
\author{Niklas Beisert\\[2ex]
  Institut f\"ur Theoretische Physik\\
  Eidgen\"ossische Technische Hochschule Z\"urich\\
  Wolfgang-Pauli-Strasse 27, 8093 Z\"urich, Switzerland\\[1ex]
  \href{mailto:nbeisert@itp.phys.ethz.ch}
  {\texttt{nbeisert@itp.phys.ethz.ch}}}
\hypersetup{pdfauthor={Niklas Beisert}}
\hypersetup{pdfsubject={Manual for the LaTeX2e Package childdoc}}
\date{30 December 2018, \textsf{v2.0}}
\maketitle

\begin{abstract}\noindent
\textsf{childdoc} is a \LaTeXe{} package
that enables the direct compilation
of document sections included by |\include|
to individual files.
\end{abstract}

\begingroup
\parskip0ex
\tableofcontents
\endgroup

%%%%%%%%%%%%%%%%%%%%%%%%%%%%%%%%%%%%%%%%%%%%%%%%%%%%%%%%%%%%%%%%%%%%%%%%%%%%%%%%
%%%%%%%%%%%%%%%%%%%%%%%%%%%%%%%%%%%%%%%%%%%%%%%%%%%%%%%%%%%%%%%%%%%%%%%%%%%%%%%%
\section{Introduction}

\LaTeX{} provides a mechanism to structure a large document (such as a book)
into a main file and several child files (containing the chapters)
using the |\include| command.
This mechanism is beneficial for documents
which span hundreds of pages in order to
make the source file(s) more manageable.
Moreover, compilation can be restricted to
selected child files by means of the |\includeonly| command.
The latter feature can be used to reduce the compilation time while editing
(this was significantly more useful in the earlier days of \LaTeX{})
or to generate a smaller document which is easier to navigate.
Another application of |\includeonly| is to generate
documents consisting of selected parts of the complete document.

However, there are a few drawbacks of the plain |\include| mechanism:
\begin{itemize}
\item
The child files cannot be compiled on their own,
they can only be compiled via the main file.
A naive editing environment
(such as a text editor with an option
to have the current file processed by \LaTeX)
may require one to switch to the main file before compiling;
attempting to compile the child file produces errors.
\item
The main file must be modified (each time)
to adjust the |\includeonly| command
to the present needs. This easily leaves the main file in a messy state.
\item
The generated document will always carry the filename
of the main document. This is inconvenient if
several child files are to be compiled and
to be kept for distribution.
\end{itemize}

The present package provides a simple interface
to make child files individually compilable by \LaTeX{}.
Compiling a child file then has the same effect as compiling
the main file with an |\includeonly| command
to select the appropriate child.
Moreover the generated document will carry the name of the child
rather than the main file.
This resolves all three above issues.

This feature is meant to make the editing of books,
thesis documents and lecture notes somewhat more convenient.
However, the package can also be used efficiently for
composing a series of documents (such as exercise sheets)
which are typically distributed individually.
It then assists the author in generating the individual documents
(potentially in different versions)
as well as a document containing the collected series.
Another application is in developing style files
or other kinds of included material
where compilation of the style file could redirect
to a sample or test file.

%%%%%%%%%%%%%%%%%%%%%%%%%%%%%%%%%%%%%%%%%%%%%%%%%%%%%%%%%%%%%%%%%%%%%%%%%%%%%%%%
%%%%%%%%%%%%%%%%%%%%%%%%%%%%%%%%%%%%%%%%%%%%%%%%%%%%%%%%%%%%%%%%%%%%%%%%%%%%%%%%
\section{Usage}

First of all, the package \textsf{childdoc} is \emph{not} a standard
\LaTeXe{} |.sty| style file! Therefore it needs to be invoked in
a non-standard way.

%%%%%%%%%%%%%%%%%%%%%%%%%%%%%%%%%%%%%%%%%%%%%%%%%%%%%%%%%%%%%%%%%%%%%%%%%%%%%%%%
\subsection{Included Files}
\label{sec:include}

%%%%%%%%%%%%%%%%%%%%%%%%%%%%%%%%%%%%%%%%
\DescribeMacro{\childdocmain}
To use the package, add the commands
\begin{center}
\begin{tabular}{l}
|\input{childdoc.def}|\\
|\childdocmain{}|\\
\end{tabular}
\end{center}
at the very top of the main \LaTeX{} file,
in particular \emph{before} the |\documentclass| statement!
The argument of |\childdocmain| should be left empty
(but it must be present).

%%%%%%%%%%%%%%%%%%%%%%%%%%%%%%%%%%%%%%%%
\DescribeMacro{\childdocof}
Furthermore, add the commands
\begin{center}
\begin{tabular}{l}
|\input{childdoc.def}|\\
|\childdocof{|\textit{main}|}|\\
\end{tabular}
\end{center}
at the top of every child file \textit{child}
which is included by |\include{|\textit{child}|}|
from within the main file
(or at least for those files to be compiled individually).
The argument \textit{main} must be the filename of the main file.

There are a couple of
considerations in setting up the main and child documents:

%%%%%%%%%%%%%%%%%%%%%%%%%%%%%%%%%%%%%%%%
\paragraph{Restrictions.}

Please note the following restrictions:
\begin{itemize}
\item
|\childdocmain| must be called with one argument \textit{main}
to ensure compatibility with earlier version of the package.
It must either be empty (|\childdocmain{}|)
or precisely match the filename of the main file in which it is specified.
See \secref{sec:detection} for further information.
\item
The filename \textit{main} must be specified without the |.tex| extension.
\item
The filename \textit{main} is case sensitive
(even in case-insensitive file systems)
due to internal string comparison.
\item
The argument \textit{main} should be fully expanded, it cannot be a macro.
\item
Subdirectories and special characters should be avoided in filenames.
\item
The command |\childdocmain{|\textit{main}|}| must be followed by a whitespace.
It should not be followed immediately by another command
or by a comment mark `|%|'.
This is because the \TeX{} parser reads the token immediately following
the argument of |\childdocmain| and puts it
at the beginning of every child section;
however, a white\-space is ignored.
\end{itemize}

%%%%%%%%%%%%%%%%%%%%%%%%%%%%%%%%%%%%%%%%
\paragraph{Content of Main File.}

It is advisable to place all content in the child files included by |\include|.
Any output contained in the main file will appear in all child documents
unless suppressed manually;
it cannot be suppressed automatically by the |\includeonly| directive
and thus should normally be avoided.
A method to include some content in the main file
by means of conditional processing is described in \secref{sec:conditional}.

%%%%%%%%%%%%%%%%%%%%%%%%%%%%%%%%%%%%%%%%
\paragraph{Page Numbering.}

When only a part of the document is compiled,
the appropriate numbering of pages
(as well as other status parameters)
is determined from the |.aux| files.
The latter contain information from previous passes.
However this information needs to propagate through
all intermediate child documents.
Therefore the page numbering in child documents may well
be inconsistent until the complete document is compiled at least once.

A useful (if unconventional) way to always ensure a consistent
page numbering is to restart the numbering in each child document
and denote the pages by `\textit{child}|.|\textit{page}'
where \textit{child} represents the chapter/section number of the child file.
This can be achieved by the command
|\numberwithin{page}{|\textit{child}|}|
of the \textsf{amsmath} package
where \textit{child} can be |chapter| or |section|
depending on the chosen structuring.
Alternatively, one can modify the macro |\thepage| appropriately
and reset the counter |page| at the start of each child file.

%%%%%%%%%%%%%%%%%%%%%%%%%%%%%%%%%%%%%%%%%%%%%%%%%%%%%%%%%%%%%%%%%%%%%%%%%%%%%%%%
\subsection{Conditional Processing}
\label{sec:conditional}

The package provides a mechanism to compile different versions
of a document. To customise the versions further some conditional processing
can come in handy to distinguish which version is being compiled.
The package provides two macros to describe the compilation context:

%%%%%%%%%%%%%%%%%%%%%%%%%%%%%%%%%%%%%%%%
\DescribeMacro{\ifchilddoc}
The conditional |\ifchilddoc| distinguishes between the compilation of
child documents and the main document:
%
\begin{center}
|\ifchilddoc |\textit{child-code}| |[|\||else |\textit{main-code}]| \||fi|
\end{center}

%%%%%%%%%%%%%%%%%%%%%%%%%%%%%%%%%%%%%%%%
\DescribeMacro{\childdocname}
\DescribeMacro{\childdocjob}
The macro |\childdocname| contains the filename (without extension)
of the main or child file being processed.
Note that |\childdocjob| will always contain the name of the main file.

%%%%%%%%%%%%%%%%%%%%%%%%%%%%%%%%%%%%%%%%
\paragraph{Title Page.}

Conditional processing can be used to include a title or banner page
in the main document when proper precautions are taken.
Importantly, the code in the main file should ensure that the page counter
(as well as other status parameters which are stored in the |.aux| files)
takes the same value after the conditional processing.
Otherwise the page numbers may take divergent values
depending on which part is compiled.

For example, a title page could be declared by:
%
\begin{center}
\begin{tabular}{l}
|\ifchilddoc\||else|\\
|\addtocounter{page}{-1}|\\
\textit{code for title page}\\
|\newpage|\\
|\||fi|
\end{tabular}
\end{center}
%
A banner page for the child documents can be generated by:
%
\begin{center}
\begin{tabular}{l}
|\ifchilddoc|\\
|\addtocounter{page}{-1}|\\
\textit{code for banner page}\\
|\newpage|\\
|\||fi|
\end{tabular}
\end{center}
%
Here one could write a message such as:
\begin{center}
|This is the part \childdocname{} of \childdocjob{}.|
\end{center}

%%%%%%%%%%%%%%%%%%%%%%%%%%%%%%%%%%%%%%%%%%%%%%%%%%%%%%%%%%%%%%%%%%%%%%%%%%%%%%%%
\subsection{Flags}
\label{sec:flags}

The package makes it easy to generate different versions
of the main or child documents.
To this end compilation flags can be defined
and assigned different default values.
They will be particularly useful in conjunction
with the forwarding mechanism described in \secref{sec:forward}.

For example, it may be useful to have a flag |\version|
which can be set to |draft| or |final|.
The document source will contain some conditional code
depending on the value of |\version|.
Suppose further, the flag should default to |final| for the main file
and to |draft| for child files
which is a natural assignment for editing the document.
This is achieved by placing the following code
in the preamble of the main document
(below the |\childdocmain| directive):
%
\begin{center}
\begin{tabular}{l}
|\ifchilddoc|\\
|\providecommand{\version}{draft}|\\
|\||else|\\
|\providecommand{\version}{final}|\\
|\||fi|
\end{tabular}
\end{center}
%
The definition by |\providecommand| makes sure
that previous definitions are not overwritten.
Further statements |\providecommand{\version}{...}|
can thus be added before the above code to override it.

For the main file, one might add a line
(between |\childdocmain| and the above block)
%
\begin{center}
|%\ifchilddoc\||else\providecommand{\version}{draft}\||fi|
\end{center}
%
which can be uncommented to produce a draft version.
Likewise one can add a line to the very top of a child file
(above the |\childdocof{|\textit{main}|}| directive)
%
\begin{center}
|%\providecommand{\version}{final}|
\end{center}
%
which can be uncommented to produce the final version of this child document.

%%%%%%%%%%%%%%%%%%%%%%%%%%%%%%%%%%%%%%%%%%%%%%%%%%%%%%%%%%%%%%%%%%%%%%%%%%%%%%%%
\subsection{Forwarding}
\label{sec:forward}

Different versions of the main or child documents
using compilation flags as described in \secref{sec:flags}
can be (permanently) stored in different files
for convenient compilation, viewing and distribution.
To this end, the package defines a command
to pass on compilation to a different file:

%%%%%%%%%%%%%%%%%%%%%%%%%%%%%%%%%%%%%%%%
\DescribeMacro{\childdocforward}
The command |\childdocforward| redirects processing to
another source file:
%
\begin{center}
\begin{tabular}{l}
|\input{childdoc.def}|\\
|\childdocforward[|\textit{main}|]{|\textit{dest}|}|\\
\end{tabular}
\end{center}
%
The argument \textit{dest} is the destination file
(without extension).
It should be the main file or one of the child files.
Note that further \textsf{childdoc} directives
such as |\childdocof| and |\childdocforward|
in the indicated file will be processed in this form.
The optional argument \textit{main}
passes on directly to the main file \textit{main}
while pretending to compile the child \textit{dest}.
This form behaves as if \textit{dest}
issues |\childdocof{|\textit{main}|}| right away,
and no further \textsf{childdoc} directives will be processed.

%%%%%%%%%%%%%%%%%%%%%%%%%%%%%%%%%%%%%%%%
\DescribeMacro{\...prefix}
In the alternative form |\childdocforwardprefix|,
%
\begin{center}
\begin{tabular}{l}
|\input{childdoc.def}|\\
|\childdocforwardprefix[|\textit{main}|]{|\textit{prefix}|}{|\textit{dest}|}|
\end{tabular}
\end{center}
%
the destination file is determined by a pattern
depending on the current file:
To make this work, the current file must be called
`{\textit{prefix}\hspace{0.2em}\textit{suffix}}'
with \textit{prefix} matching precisely the argument.
Processing is then passed on to the file
`{\textit{dest}\hspace{0.2em}\textit{suffix}}'.
Surely, the same effect is achieved by
directly specifying the
argument `{\textit{dest}\hspace{0.2em}\textit{suffix}}'
in the first form.
However, that requires to set up a different file
for each child. With the alternative form of the command
all these files can have exactly the same content
which simplifies setting them up and maintaining them.

For example, the following file |draft.tex|
with a compilation flag |\version| as described in \secref{sec:flags}
compiles the main document as a draft:
%
\begin{center}
\begin{tabular}{l}
|\def\version{draft}|\\
|\input{childdoc.def}|\\
|\childdocforward{|\textit{main}|}|
\end{tabular}
\end{center}
%
Likewise, the following files |final|\textit{nn}|.tex|
compile the final version of the child document
|child|\textit{nn}|.tex|:
%
\begin{center}
\begin{tabular}{l}
|\def\version{final}|\\
|\input{childdoc.def}|\\
|\childdocforwardprefix{final}{child}|
\end{tabular}
\end{center}
%

Note that when several versions of a main file and/or of each child file
are to be generated, it may be convenient to set up a |Makefile| or
shell script to automatise the process.

%%%%%%%%%%%%%%%%%%%%%%%%%%%%%%%%%%%%%%%%%%%%%%%%%%%%%%%%%%%%%%%%%%%%%%%%%%%%%%%%
\subsection{Command Line Processing}
\label{sec:commandline}

The effect of redirection files can also be achieved by invoking
the \LaTeX{} compiler with a more elaborate command line.
Most conveniently this should be done as part
of a shell script or a |Makefile|.

When using \textsf{childdoc} in the main file, the following
command lines effectively perform a redirection
(note that depending on the shell being used,
backslashes may have to be doubled: `|\|' $\to$ `|\\|'):
%
\begin{center}
|... -jobname "|\textit{target}|" |\\|"|[\textit{flags}]%
|\input{childdoc.def}\childdocforward[|\textit{main}|]{|\textit{dest}|}"|
\end{center}
%
Here \textit{target} is the name of the output file,
\textit{main} is the name of the main file
and \textit{dest} is the name of the main or child file to be processed
(all filenames without extensions).
The optional argument \textit{main} can be omitted
if \textit{main} matches \textit{dest}.
Optionally, compilation \textit{flags} can be defined via |\def| commands.
This command line makes the \TeX{} engine believe
it is compiling the file \textit{target}
whose content is specified as the latter parameter.
The provided code then forwards the processing to
\textit{main} or \textit{dest} as described in \secref{sec:forward}.

%%%%%%%%%%%%%%%%%%%%%%%%%%%%%%%%%%%%%%%%%%%%%%%%%%%%%%%%%%%%%%%%%%%%%%%%%%%%%%%%
\subsection{Include by Input}
\label{sec:input}

Including child documents by |\include| has some restrictions by design.
Most notably, the content of a child document always occupies
its own set of pages; pages cannot be shared between child documents.
Usually, this behaviour makes perfect sense
because each child document contain an essential part of the document.
However, in some situations it may be desirable to compose
a document from a collection of parts
without having mandatory page breaks between then.
For this case, the package
provides a mechanism to include parts
by |\input| which can also be processed individually.
However, by construction this mechanism
requires manual handling of the content to be output.

%%%%%%%%%%%%%%%%%%%%%%%%%%%%%%%%%%%%%%%%
\DescribeMacro{\ifchilddocmanual}
The main file should be prepared as usual, see \secref{sec:include}.
However, the document body must make a distinction
between processing of an individual part and of the main document, e.g.:
%
\begin{center}
\begin{tabular}{l}
|\ifchilddocmanual|\\
|\input{\childdocname}|\\
|\||else|\\
\textit{document body with }|\input{|\textit{part}|}|\\
|\||fi|
\end{tabular}
\end{center}
%
The conditional |\ifchilddocmanual| is true whenever
a part to be included by |\input| is being compiled,
and the name of the part is stored in |\childdocname|.

%%%%%%%%%%%%%%%%%%%%%%%%%%%%%%%%%%%%%%%%
\DescribeMacro{\childdocby}
Each part to be included by |\input| should start with:
%
\begin{center}
\begin{tabular}{l}
|\input{childdoc.def}|\\
|\childdocby{|\textit{main}|}|\\
\end{tabular}
\end{center}
%
The directive |\childdocby| is similar to |\childdocof|
described in \secref{sec:include},
but the subsequent selection of content must be done manually.
To that end, both |\ifchilddoc| and |\ifchilddocmanual|
will be true upon processing of a part,
and the name of the part is stored in |\childdocname|.
Note that |\jobname| will be set to the filename of the current part
so that each part receives an individual |.aux| file
that does not interfere with the |.aux| file(s) of the main document.
This behaviour can be altered by the alternative form
|\childdocby[*]{|\textit{main}|}| (with a non-empty optional argument)
which uses the |.aux| file of the main document
by setting |\jobname| to \textit{main}.

%%%%%%%%%%%%%%%%%%%%%%%%%%%%%%%%%%%%%%%%%%%%%%%%%%%%%%%%%%%%%%%%%%%%%%%%%%%%%%%%
\subsection{Driver Development}
\label{sec:driver}

The \textsf{childdoc} mechanism can also be use for the development
of definition files such as \LaTeX{} styles or classes.
This case differs from the above setup with multiple parts
included by |\include| in that no |\includeonly| should be invoked.
This can be achieved by starting the include file
(before |\ProvidesPackage|) with:
%
\begin{center}
\begin{tabular}{l}
|\input{childdoc.def}|\\
|\childdocforward{|\textit{main}|}|\\
\end{tabular}
\end{center}
%
or alternatively with:
%
\begin{center}
\begin{tabular}{l}
|\input{childdoc.def}|\\
|\childdocby{|\textit{main}|}|\\
\end{tabular}
\end{center}
%
Both forms have slightly different effects as described above.
The main file is prepared as usual, see \secref{sec:include}.

%%%%%%%%%%%%%%%%%%%%%%%%%%%%%%%%%%%%%%%%%%%%%%%%%%%%%%%%%%%%%%%%%%%%%%%%%%%%%%%%
\subsection{Legacy Detection}
\label{sec:detection}

The directive |\childdocmain| in the main file can detect
whether the complete document or merely a child is to be compiled
even without using the directive |\childdocof|.
This method is deprecated because it is less robust
and there is no compelling reason to use it;
it is merely provided for backward compatibility
and it may be removed in future versions.

If the detection mechanism is to be used,
it is mandatory to correctly specify
the filename of the main file as the argument of |\childdocmain|:
%
\begin{center}
\begin{tabular}{l}
|\input{childdoc.def}|\\
|\childdocmain{|\textit{main}|}|\\
\end{tabular}
\end{center}
%
If |\jobname| does not match the argument \textit{main} of |\childdocmain|,
it is assumed that |\jobname| points to the child file to be compiled.
When using |\childdocmain| with the main file specified as argument,
it suffices to start a child file
with just |\input{|\textit{main}|}|
without loading of the package and using |\childdocof|.
If instead all processing is done
with the appropriate \textsf{childdoc} directives,
the argument of \textit{main} of |\childdocmain| can be empty.

An alternative version of the command line processing described
in \secref{sec:commandline} using the detection mechanism reads:
%
\begin{center}
|... -jobname "|\textit{target}|" "|[\textit{flags}]%
[|\def\jobname{|\textit{dest}|}|]|\input{|\textit{main}|}"|
\end{center}

%%%%%%%%%%%%%%%%%%%%%%%%%%%%%%%%%%%%%%%%%%%%%%%%%%%%%%%%%%%%%%%%%%%%%%%%%%%%%%%%
\subsection{Manual Code}
\label{sec:manual}

In case one cannot be certain whether the definitions file |childdoc.def|
is installed on the target \TeX{} distribution
and one prefers not to ship it,
it is conceivable to paste a few relevant commands into the sources.

To that end, drop all statements |\input{childdoc.def}|
and perform the replacements as outlined below.
Instead of |\childdocmain{|\textit{main}|}| add the following code
to the top of the main file:
%
\begin{center}
\begin{tabular}{l}
|\||ifdefined\childdocname\endinput\||fi\newif\ifchilddoc|\\
|\edef\childdocname{\scantokens\expandafter{\jobname\noexpand}}|\\
|\def\childdocmain{|\textit{main}|}\||ifx\childdocmain\childdocname\||else|\\
|\childdoctrue\includeonly{\childdocname}\let\jobname\childdocmain\||fi|\\
\end{tabular}
\end{center}
%
Instead of |\childdocof{|\textit{main}|}| just include the main file
at the top of each child file:
%
\begin{center}
|\input{|\textit{main}|}|
\end{center}
%
A simple redirection |\childdocforward{|\textit{dest}|}| is achieved by:
%
\begin{center}
|\def\jobname{|\textit{dest}|}\input{\jobname}|
\end{center}
%
The redirection with prefix
|\childdocforwardprefix[|\textit{prefix}|]{|\textit{dest}|}|
is accomplished by:
%
\begin{center}
\begin{tabular}{l}
|{\edef\jobname{\scantokens\expandafter{\jobname\noexpand}}|\\
|\def\redirectjob |\textit{prefix}|#1~~~{\gdef\jobname{|\textit{dest}|#1}}|\\
|\expandafter\redirectjob\jobname~~~}\input{\jobname}|
\end{tabular}
\end{center}

In an alternative approach,
child documents can be compiled by a specific command line
without additional code or specific definitions:
%
\begin{center}
|... -jobname "|\textit{target}|" "|[\textit{flags}]%
|\includeonly{|\textit{dest}|}\input{|\textit{main}|}"|
\end{center}
%

%%%%%%%%%%%%%%%%%%%%%%%%%%%%%%%%%%%%%%%%%%%%%%%%%%%%%%%%%%%%%%%%%%%%%%%%%%%%%%%%
%%%%%%%%%%%%%%%%%%%%%%%%%%%%%%%%%%%%%%%%%%%%%%%%%%%%%%%%%%%%%%%%%%%%%%%%%%%%%%%%
\section{Information}

%%%%%%%%%%%%%%%%%%%%%%%%%%%%%%%%%%%%%%%%%%%%%%%%%%%%%%%%%%%%%%%%%%%%%%%%%%%%%%%%
\subsection{Copyright}

Copyright \copyright{} 2017--2018 Niklas Beisert

This work may be distributed and/or modified under the
conditions of the \LaTeX{} Project Public License, either version 1.3
of this license or (at your option) any later version.
The latest version of this license is in
  \url{http://www.latex-project.org/lppl.txt}
and version 1.3 or later is part of all distributions of \LaTeX{}
version 2005/12/01 or later.

This work has the LPPL maintenance status `maintained'.

The Current Maintainer of this work is Niklas Beisert.

This work consists of the files |README.txt|, |childdoc.ins| and |childdoc.dtx|
as well as the derived files |childdoc.def|, |cdocsamp.tex|
with |cdocsch1.tex|, |cdocsch2.tex|, |cdocspt3.tex|, |cdocspt4.tex|,
|cdocsdrf.tex|, |cdocsfn1.tex|, |cdocsfn2.tex|
as well as |childdoc.pdf|.

%%%%%%%%%%%%%%%%%%%%%%%%%%%%%%%%%%%%%%%%%%%%%%%%%%%%%%%%%%%%%%%%%%%%%%%%%%%%%%%%
\subsection{Files and Installation}

The package consists of the files:
%
\begin{center}
\begin{tabular}{ll}
    |README.txt|   & readme file \\
    |childdoc.ins| & installation file \\
    |childdoc.dtx| & source file \\
    |childdoc.def| & definition file \\
    |cdocsamp.tex| & sample main file \\
    |cdocsch1.tex| & sample include file \\
    |cdocsch2.tex| & sample include file \\
    |cdocspt3.tex| & sample part file \\
    |cdocspt4.tex| & sample part file \\
    |cdocsdrf.tex| & sample redirection file \\
    |cdocsfn1.tex| & sample redirection file \\
    |cdocsfn2.tex| & sample redirection file \\
    |childdoc.pdf| & manual
\end{tabular}
\end{center}
%
The distribution consists of the files
|README.txt|, |childdoc.ins| and |childdoc.dtx|.
%
\begin{itemize}
\item
Run (pdf)\LaTeX{} on |childdoc.dtx|
to compile the manual |childdoc.pdf| (this file).
\item
Run \LaTeX{} on |childdoc.ins| to create the definitions file |childdoc.def|
and the sample |cdocsamp.tex| with include files
|cdocsch1.tex|, |cdocsch2.tex|, |cdocspt3.tex|, |cdocspt4.tex|,
|cdocsdrf.tex|, |cdocsfn1.tex|, |cdocsfn2.tex|.
Then copy the file |childdoc.def| to an appropriate directory of your \LaTeX{}
distribution, e.g.\ \textit{texmf-root}|/tex/latex/childdoc|.
\end{itemize}

%%%%%%%%%%%%%%%%%%%%%%%%%%%%%%%%%%%%%%%%%%%%%%%%%%%%%%%%%%%%%%%%%%%%%%%%%%%%%%%%
\subsection{Related CTAN Packages}

There are several other packages which offer a similar functionality:
%
\begin{itemize}
\item
The packages
\href{http://ctan.org/pkg/docmute}{\textsf{docmute}},
\href{http://ctan.org/pkg/includex}{\textsf{includex}} and
\href{http://ctan.org/pkg/standalone}{\textsf{standalone}}
provide commands to include only the document body of
a child file thus allowing both files to be compiled individually.
\item
The packages \href{http://ctan.org/pkg/subdocs}{\textsf{subdocs}}
and \href{http://ctan.org/pkg/subfiles}{\textsf{subfiles}}
provide structures in which the main and child documents can be
encapsulated and allowing them to be compiled individually.
The inclusion mechanism is different from the conventional |\include|.
\item
The package \href{http://ctan.org/pkg/combine}{\textsf{combine}}
is an elaborate solution to combine several documents into one.
\end{itemize}
%
See also the CTAN topic \href{http://ctan.org/topic/subdocs}{\textsf{subdocs}}
for further related packages.
The present package differs from the above solutions in that
a document structure constructed with the conventional |\include| mechanism
just needs two extra commands at the top of every file
such that all constituent files can be compiled individually.

%%%%%%%%%%%%%%%%%%%%%%%%%%%%%%%%%%%%%%%%%%%%%%%%%%%%%%%%%%%%%%%%%%%%%%%%%%%%%%%%
%\subsection{Feature Suggestions}
%
%The following is a list of features which may be useful for future
%versions of this package:
%%
%\begin{itemize}
%\item
%\ldots
%\end{itemize}

%%%%%%%%%%%%%%%%%%%%%%%%%%%%%%%%%%%%%%%%%%%%%%%%%%%%%%%%%%%%%%%%%%%%%%%%%%%%%%%%
\subsection{Revision History}

%%%%%%%%%%%%%%%%%%%%%%%%%%%%%%%%%%%%%%%%
\paragraph{v2.0:} 2018/12/30

\begin{itemize}
\item
immediate forward processing
\item
added |\childdocby| mechanism
\item
manual restructured
\end{itemize}

%%%%%%%%%%%%%%%%%%%%%%%%%%%%%%%%%%%%%%%%
\paragraph{v1.6:} 2018/01/17

\begin{itemize}
\item
application for development of include files
\item
corrections to manual
\end{itemize}

%%%%%%%%%%%%%%%%%%%%%%%%%%%%%%%%%%%%%%%%
\paragraph{v1.5:} 2017/05/21

\begin{itemize}
\item
more complete structuring introduced
\item
|\childdocof| introduced
\item
|\childdoc| renamed to |\childdocmain|
\item
|\childredirect| renamed to |\childdocforward| and |\childdocforwardprefix|
and functionality expanded
\end{itemize}

%%%%%%%%%%%%%%%%%%%%%%%%%%%%%%%%%%%%%%%%
\paragraph{v1.0:} 2017/04/27

\begin{itemize}
\item
manual and install package
\item
first version published on CTAN
\end{itemize}

%%%%%%%%%%%%%%%%%%%%%%%%%%%%%%%%%%%%%%%%
\paragraph{v0.6:} 2017/04/26

\begin{itemize}
\item
redirection mechanism added
\end{itemize}

%%%%%%%%%%%%%%%%%%%%%%%%%%%%%%%%%%%%%%%%
\paragraph{v0.5:} 2017/04/26

\begin{itemize}
\item
functionality in definition file
\end{itemize}


%%%%%%%%%%%%%%%%%%%%%%%%%%%%%%%%%%%%%%%%%%%%%%%%%%%%%%%%%%%%%%%%%%%%%%%%%%%%%%%%
%%%%%%%%%%%%%%%%%%%%%%%%%%%%%%%%%%%%%%%%%%%%%%%%%%%%%%%%%%%%%%%%%%%%%%%%%%%%%%%%
%%%%%%%%%%%%%%%%%%%%%%%%%%%%%%%%%%%%%%%%%%%%%%%%%%%%%%%%%%%%%%%%%%%%%%%%%%%%%%%%
\appendix

\settowidth\MacroIndent{\rmfamily\scriptsize 000\ }

 \DocInput{childdoc.dtx}

\end{document}
%</driver>
% \fi
%
% %%%%%%%%%%%%%%%%%%%%%%%%%%%%%%%%%%%%%%%%%%%%%%%%%%%%%%%%%%%%%%%%%%%%%%%%%%%%%%
% %%%%%%%%%%%%%%%%%%%%%%%%%%%%%%%%%%%%%%%%%%%%%%%%%%%%%%%%%%%%%%%%%%%%%%%%%%%%%%
% \section{Sample}
%\iffalse
%<*samplemain>
%\fi
%
% The following presents a sample document
% with two chapters, two parts, a title page,
% a compile flag as well as three forwarding files to set the flag.
% It consists of eight |.tex| files:
% \begin{center}
% \begin{tabular}{ll}
% |cdocsamp.tex|&main file\\
% |cdocsch1.tex|&include file for chapter 1\\
% |cdocsch2.tex|&include file for chapter 2\\
% |cdocspt3.tex|&include file for part 3\\
% |cdocspt4.tex|&include file for part 4\\
% |cdocsdrf.tex|&forwarding file for main file in draft mode\\
% |cdocsfi1.tex|&forwarding file for final version of chapter 1\\
% |cdocsfi2.tex|&forwarding file for final version of chapter 2\\
% \end{tabular}
% \end{center}
% Each of the eight files can be compiled directly by the \LaTeX{} compiler.
%
% %%%%%%%%%%%%%%%%%%%%%%%%%%%%%%%%%%%%%%
% \paragraph{Main File.}
%
% The main file is called |cdocsamp.tex|.
%
% Load the \textsf{childdoc} definitions and
% declare the filename for the main document:
%    \begin{macrocode}
\input{childdoc.def}
\childdocmain{}
%    \end{macrocode}

% Optional override for |\version| flag:
%    \begin{macrocode}
%%\ifchilddoc\else\providecommand{\version}{draft}\fi
%    \end{macrocode}

% Define the default values for the |\version| flag
% (|final| for the main file and |draft| for childs):
%    \begin{macrocode}
\ifchilddoc
\providecommand{\version}{draft}
\else
\providecommand{\version}{final}
\fi
%    \end{macrocode}

% Load the standard document class:
%    \begin{macrocode}
\documentclass[12pt]{article}
%    \end{macrocode}

% Start the document body:
%    \begin{macrocode}
\begin{document}
%    \end{macrocode}

% Declare a title page.
% Print title, part of document being processed and version flag:
%    \begin{macrocode}
\addtocounter{page}{-1}
\begin{center}
{\LARGE\bfseries{}childdoc example\par}
\vspace{1cm}
\ifchilddoc
\ifchilddocmanual part\else chapter\fi:
`\childdocname' of `\childdocjob'\par
\else
main document: `\childdocjob'\par
\fi
version: \version\par
\end{center}
\newpage
%    \end{macrocode}

% Manually include selected file,
% otherwise process as usual:
%    \begin{macrocode}
\ifchilddocmanual
\section*{part `\childdocname'}
\input{\childdocname}
\else
%    \end{macrocode}

% Include the two chapters:
%    \begin{macrocode}
\include{cdocsch1}
\include{cdocsch2}
%    \end{macrocode}

% Include the two parts unless only chapters should be displayed:
%    \begin{macrocode}
\ifchilddoc\else
\section{part three}
\input{cdocspt3}
\section{part four}
\input{cdocspt4}
\fi
%    \end{macrocode}

% Process as usual until here:
%    \begin{macrocode}
\fi
%    \end{macrocode}

% End of document body:
%    \begin{macrocode}
\end{document}
%    \end{macrocode}
%\iffalse
%</samplemain>
%\fi
%
% %%%%%%%%%%%%%%%%%%%%%%%%%%%%%%%%%%%%%%
% \paragraph{Chapter Include Files.}
%
% The include files are called |cdocsch1.tex| and |cdocsch2.tex|.
%
%\iffalse
%<*samplechap1|samplechap2>
%\fi

% Optional override for |\version| flag:
%    \begin{macrocode}
%%\providecommand{\version}{final}
%    \end{macrocode}

% Include the main document:
%    \begin{macrocode}
\input{childdoc.def}
\childdocof{cdocsamp}
%    \end{macrocode}

%\iffalse
%</samplechap1|samplechap2>
%\fi
%
%\iffalse
%<*samplechap1>
%\fi
% Some text for chapter 1:
%    \begin{macrocode}
\section{one}
some text in chapter one
%    \end{macrocode}

%\iffalse
%</samplechap1>
%\fi
% Some text for chapter 2:
%\iffalse
%<*samplechap2>
%\fi
%    \begin{macrocode}
\section{two}
more text in chapter two
%    \end{macrocode}

%\iffalse
%</samplechap2>
%\fi
%
% %%%%%%%%%%%%%%%%%%%%%%%%%%%%%%%%%%%%%%
% \paragraph{Part Include Files.}
%
% The include files are called |cdocspt3.tex| and |cdocspt4.tex|.
%
%\iffalse
%<*samplepart3|samplepart4>
%\fi

% Optional override for |\version| flag:
%    \begin{macrocode}
%%\providecommand{\version}{final}
%    \end{macrocode}

% Include the main document:
%    \begin{macrocode}
\input{childdoc.def}
\childdocby{cdocsamp}
%    \end{macrocode}

%\iffalse
%</samplepart3|samplepart4>
%\fi
%
%\iffalse
%<*samplepart3>
%\fi
% Some text for part 3:
%    \begin{macrocode}
some text in part three
%    \end{macrocode}

%\iffalse
%</samplepart3>
%\fi
% Some text for part 4:
%\iffalse
%<*samplepart4>
%\fi
%    \begin{macrocode}
more text in part four
%    \end{macrocode}

%\iffalse
%</samplepart4>
%\fi
%
% %%%%%%%%%%%%%%%%%%%%%%%%%%%%%%%%%%%%%%
% \paragraph{Forwarding for a Complete Draft.}
%
% The following forwarding file |cdocsdrf.tex|
% compiles the main document in draft mode:
%\iffalse
%<*sampledraft>
%\fi
%    \begin{macrocode}
\def\version{draft}
\input{childdoc.def}
\childdocforward{cdocsamp}
%    \end{macrocode}

%\iffalse
%</sampledraft>
%\fi
%
% %%%%%%%%%%%%%%%%%%%%%%%%%%%%%%%%%%%%%%
% \paragraph{Forwarding for Final Version of the Chapters.}
%
% The following forwarding files |cdocsfn1.tex| and |cdocsfn2.tex|
% (with identical content)
% compile the final versions of the child documents
% |cdocsch1.tex| and |cdocsch2.tex|, respectively:
%\iffalse
%<*samplefinal>
%\fi
%    \begin{macrocode}
\def\version{final}
\input{childdoc.def}
\childdocforwardprefix[cdocsamp]{cdocsfn}{cdocsch}
%    \end{macrocode}

%\iffalse
%</samplefinal>
%\fi
%
% %%%%%%%%%%%%%%%%%%%%%%%%%%%%%%%%%%%%%%
% \paragraph{Command Line Processing.}
%
% The following three command lines generate the output files
% |cdocscld|, |cdocscl1| and |cdocscl2|
% which should be identical to
% |cdocsdrf|, |cdocsch1| and |cdocsfn2|, respectively:
% \begin{center}
% \begin{tabular}{l}
% |latex -jobname cdocscld \|\\
% |  "\def\version{draft}\input{childdoc.def}\childdocforward{cdocsamp}"|\\
% |latex -jobname cdocscl1 \|\\
% |  "\input{childdoc.def}\childdocforward[cdocsamp]{cdocsch1}"|\\
% |latex -jobname cdocscl2 \|\\
% |  "\def\version{final}\input{childdoc.def}\childdocforward{cdocsch2}"|
% \end{tabular}
% \end{center}
% Note that the trailing backslash on each first line
% merely continues the input to the second line
% (for convenient cut ant paste).
% Furthermore, the command |latex| can be replaced by any
% of its alternative versions such as |pdflatex|.
%
% %%%%%%%%%%%%%%%%%%%%%%%%%%%%%%%%%%%%%%%%%%%%%%%%%%%%%%%%%%%%%%%%%%%%%%%%%%%%%%
% %%%%%%%%%%%%%%%%%%%%%%%%%%%%%%%%%%%%%%%%%%%%%%%%%%%%%%%%%%%%%%%%%%%%%%%%%%%%%%
% \section{Implementation}
%\iffalse
%<*package>
%\fi
%
% This section describes the definitions file |childdoc.def|.

% The definitions cannot be loaded using |\usepackage| or |\RequirePackage|
% which has a mechanism to prevent loading a style file more than once.
% When loading the definitions by means of |\input|
% multiple instances have to be prevented manually:
%\iffalse
%This code needs to be before the `\ProvidesFile' directive
%which is defined at the beginning of this file.
%Therefore it is also placed there and commented out here.
%</package>
%<*discard>
%\fi
%    \begin{macrocode}
\ifdefined\childdocmain\endinput\fi
%    \end{macrocode}
%\iffalse
%</discard>
%<*package>
%\fi
%
% \macro{\ifchilddoc}
% \macro{\ifchilddocmanual}
% The conditional |\ifchilddoc| tells whether a
% child (true) or main (false) document is being compiled.
% The conditional |\ifchilddocmanual| tells whether
% the |\includeonly| mechanism is used (false) or
% the selection of child files must be performed manually (true).
% The definitions initialise to false:
%    \begin{macrocode}
\newif\ifchilddoc
\newif\ifchilddocmanual
%    \end{macrocode}

% \macro{\childdocname}
% \macro{\childdocjob}
% The macro |\childdocname| stores the name of the main document
% to be compiled. The macro |\childdocjob| stores the name of
% the document on which the \LaTeX{} compiler was originally invoked.
% The content of |\jobname| cannot be compared
% to filenames specified in the source due to different catcodes.
% The following code rescans |\jobname|, stores the result
% in |\childdocname| and saves a copy in |\childdocjob|:
%    \begin{macrocode}
\edef\childdocname{\scantokens\expandafter{\jobname\noexpand}}
\let\childdocjob\childdocname
%    \end{macrocode}

% \macro{\childdocdisable}
% The macro |\childdocdisable| prevents the main file
% from being processed more than once.
% At this stage, the main document command |\childdocmain|
% is assumed to be called once again where it should do nothing.
% Any subsequent call to it should prevent
% a secondary processing of the main document
% It overwrites the forwarding commands
% |\childdocof| and |\childdocforward|
% with empty macros to prevent further inclusions of the main document:
%    \begin{macrocode}
\newcommand{\childdocdisable}
{
  \renewcommand{\childdocmain}[1]{\renewcommand{\childdocmain}[1]{\endinput}}
  \renewcommand{\childdocof}[1]{}
  \renewcommand{\childdocby}[2][]{}
  \renewcommand{\childdocforward}[2][]{}
  \renewcommand{\childdocdisable}{}
}
%    \end{macrocode}

% \macro{\childdocmain}
% The macro |\childdocmain| is to be called at the top of the main file
% with nothing or the main filename (without extension) as argument.
% First, it breaks loops.
% If the argument is not empty and does not match |\childdocname|
% (which is set by the first inclusion of |childdoc.def|),
% |\ifchilddoc| is set to true, |\includeonly| is applied to the child file
% and |\jobname| is set to the main file
% (for proper handling of |.aux| files):
%    \begin{macrocode}
\newcommand{\childdocmain}[1]
{
  \childdocdisable\childdocmain{}
  \if?#1?\else
    \begingroup
      \def\childdoctmp{#1}
      \ifx\childdoctmp\childdocname
        \def\childdoctmp{}
      \else
        \def\childdoctmp
        {
          \childdoctrue
          \includeonly{\childdocname}
          \def\childdocjob{#1}
          \def\jobname{#1}
        }
      \fi
      \expandafter
    \endgroup
    \childdoctmp
  \fi
}
%    \end{macrocode}

% \macro{\childdocof}
% The command |\childdocof| redirects
% compilation to the main file |#1|.
%    \begin{macrocode}
\newcommand{\childdocof}[1]
{
  \childdocdisable
  \childdoctrue
  \includeonly{\childdocname}
  \def\jobname{#1}
  \def\childdocjob{#1}
  \input{#1}
}
%    \end{macrocode}

% \macro{\childdocby}
% The command |\childdocby| ....
%    \begin{macrocode}
\newcommand{\childdocby}[2][]
{
  \childdocdisable
  \childdoctrue
  \childdocmanualtrue
  \if?#1?\else
    \def\jobname{#2}
  \fi
  \def\childdocjob{#2}
  \input{#2}
  \endinput
}
%    \end{macrocode}

% \macro{\childdocforward}
% The command |\childdocforward| redirects
% compilation to the main file or
% (if the optional argument is given) a child file.
% Parameters are set as if the main file
% or a child file starting with |\childdocof| was compiled.
% Then compilation is handed over to the main file:
%    \begin{macrocode}
\newcommand{\childdocforward}[2][]
{
  \begingroup
    \if?#1?
      \def\childdoctmp
      {
        \def\childdocname{#2}
        \def\childdocjob{#2}
        \def\jobname{#2}
        \input{#2}
        \endinput
      }
    \else
      \def\childdoctmp
      {
        \childdocdisable
        \def\childdocname{#2}
        \childdoctrue
        \includeonly{#2}
        \def\childdocjob{#1}
        \def\jobname{#1}
        \input{#1}
        \endinput
      }
    \fi
    \expandafter
  \endgroup
  \childdoctmp
}
%    \end{macrocode}

% \macro{\childdocforwardprefix}
% The command |\childdocforwardprefix| redirects
% compilation to the main or a child file by means of a pattern.
% The prefix |#1| in the current filename is replaced by |#2|
% and the suffix of the current filename is kept
% (it is assumed that the filename does not contain the substring `|~~~|'
% which is used as a delimiter).
% Compilation is handed over to the new file by |\childdocforward|:
%    \begin{macrocode}
\newcommand{\childdocforwardprefix}[3][]
{
  \begingroup
    \def\childdocextract #2##1~~~{\def\childdoctmp{\childdocforward[#1]{#3##1}}}
    \expandafter\childdocextract\childdocname~~~
    \expandafter
  \endgroup
  \childdoctmp
}
%    \end{macrocode}

% \macro{\childdoc}
% The deprecated macro |\childdoc| is a legacy version of |\childdocmain|:
%    \begin{macrocode}
\newcommand{\childdoc}{\childdocmain}
%    \end{macrocode}

% \macro{\childdocredirect}
% The deprecated macro |\childdocredirect| is a legacy version
% of |\childdocforward| and |\childdocforwardprefix|:
%    \begin{macrocode}
\newcommand{\childdocredirect}[2][]
{
  \begingroup
    \if?#1?
      \def\childdoctmp{\childdocforward{#2}}
    \else
      \def\childdoctmp{\childdocforwardprefix{#1}{#2}}
    \fi
    \expandafter
  \endgroup
  \childdoctmp
}
%    \end{macrocode}

%\iffalse
%</package>
%\fi
%
\endinput
|\\
|\childdocmain{}|\\
\end{tabular}
\end{center}
at the very top of the main \LaTeX{} file,
in particular \emph{before} the |\documentclass| statement!
The argument of |\childdocmain| should be left empty
(but it must be present).

%%%%%%%%%%%%%%%%%%%%%%%%%%%%%%%%%%%%%%%%
\DescribeMacro{\childdocof}
Furthermore, add the commands
\begin{center}
\begin{tabular}{l}
|% \iffalse
%
% childdoc.dtx Copyright (C) 2017-2018 Niklas Beisert
%
% This work may be distributed and/or modified under the
% conditions of the LaTeX Project Public License, either version 1.3
% of this license or (at your option) any later version.
% The latest version of this license is in
%   http://www.latex-project.org/lppl.txt
% and version 1.3 or later is part of all distributions of LaTeX
% version 2005/12/01 or later.
%
% This work has the LPPL maintenance status `maintained'.
%
% The Current Maintainer of this work is Niklas Beisert.
%
% This work consists of the files childdoc.dtx and childdoc.ins
% and the derived files childdoc.def and cdocsamp.tex with
% cdocsch1.tex, cdocsch2.tex, cdocsdrf.tex, cdocsfn1.tex, cdocsfn2.tex.
%
%<package>\ifdefined\childdocmain\endinput\fi
%<package>\ProvidesFile{childdoc.def}[2018/12/30 v2.0 child document driver]
%<samplemain>\ProvidesFile{cdocsamp.tex}[2018/12/30 v2.0 sample for childdoc]
%<*driver>
%\ProvidesFile{childdoc.drv}[2018/12/30 v2.0 childdoc reference manual file]
\PassOptionsToClass{10pt,a4paper}{article}
\documentclass{ltxdoc}

\usepackage[margin=35mm]{geometry}
\usepackage{hyperref}
\usepackage{hyperxmp}
\usepackage[usenames]{color}

\hypersetup{colorlinks=true}
\hypersetup{pdfstartview=FitH}
\hypersetup{pdfpagemode=UseNone}
\hypersetup{pdfsource={}}
\hypersetup{pdflang={en-UK}}
\hypersetup{pdfcopyright={Copyright 2017-2018 Niklas Beisert.
  This work may be distributed and/or modified under the
  conditions of the LaTeX Project Public License, either version 1.3
  of this license or (at your option) any later version.}}
\hypersetup{pdflicenseurl={http://www.latex-project.org/lppl.txt}}
\hypersetup{pdfcontactaddress={ETH Zurich, ITP, HIT K,
  Wolfgang-Pauli-Strasse 27}}
\hypersetup{pdfcontactpostcode={8093}}
\hypersetup{pdfcontactcity={Zurich}}
\hypersetup{pdfcontactcountry={Switzerland}}
\hypersetup{pdfcontactemail={nbeisert@itp.phys.ethz.ch}}
\hypersetup{pdfcontacturl={http://people.phys.ethz.ch/\xmptilde nbeisert/}}

\newcommand{\secref}[1]{\hyperref[#1]{section \ref*{#1}}}

\parskip1ex
\parindent0pt
\let\olditemize\itemize
\def\itemize{\olditemize\parskip0pt}

\begin{document}

\title{The \textsf{childdoc} Package}
\hypersetup{pdftitle={The childdoc Package}}
\author{Niklas Beisert\\[2ex]
  Institut f\"ur Theoretische Physik\\
  Eidgen\"ossische Technische Hochschule Z\"urich\\
  Wolfgang-Pauli-Strasse 27, 8093 Z\"urich, Switzerland\\[1ex]
  \href{mailto:nbeisert@itp.phys.ethz.ch}
  {\texttt{nbeisert@itp.phys.ethz.ch}}}
\hypersetup{pdfauthor={Niklas Beisert}}
\hypersetup{pdfsubject={Manual for the LaTeX2e Package childdoc}}
\date{30 December 2018, \textsf{v2.0}}
\maketitle

\begin{abstract}\noindent
\textsf{childdoc} is a \LaTeXe{} package
that enables the direct compilation
of document sections included by |\include|
to individual files.
\end{abstract}

\begingroup
\parskip0ex
\tableofcontents
\endgroup

%%%%%%%%%%%%%%%%%%%%%%%%%%%%%%%%%%%%%%%%%%%%%%%%%%%%%%%%%%%%%%%%%%%%%%%%%%%%%%%%
%%%%%%%%%%%%%%%%%%%%%%%%%%%%%%%%%%%%%%%%%%%%%%%%%%%%%%%%%%%%%%%%%%%%%%%%%%%%%%%%
\section{Introduction}

\LaTeX{} provides a mechanism to structure a large document (such as a book)
into a main file and several child files (containing the chapters)
using the |\include| command.
This mechanism is beneficial for documents
which span hundreds of pages in order to
make the source file(s) more manageable.
Moreover, compilation can be restricted to
selected child files by means of the |\includeonly| command.
The latter feature can be used to reduce the compilation time while editing
(this was significantly more useful in the earlier days of \LaTeX{})
or to generate a smaller document which is easier to navigate.
Another application of |\includeonly| is to generate
documents consisting of selected parts of the complete document.

However, there are a few drawbacks of the plain |\include| mechanism:
\begin{itemize}
\item
The child files cannot be compiled on their own,
they can only be compiled via the main file.
A naive editing environment
(such as a text editor with an option
to have the current file processed by \LaTeX)
may require one to switch to the main file before compiling;
attempting to compile the child file produces errors.
\item
The main file must be modified (each time)
to adjust the |\includeonly| command
to the present needs. This easily leaves the main file in a messy state.
\item
The generated document will always carry the filename
of the main document. This is inconvenient if
several child files are to be compiled and
to be kept for distribution.
\end{itemize}

The present package provides a simple interface
to make child files individually compilable by \LaTeX{}.
Compiling a child file then has the same effect as compiling
the main file with an |\includeonly| command
to select the appropriate child.
Moreover the generated document will carry the name of the child
rather than the main file.
This resolves all three above issues.

This feature is meant to make the editing of books,
thesis documents and lecture notes somewhat more convenient.
However, the package can also be used efficiently for
composing a series of documents (such as exercise sheets)
which are typically distributed individually.
It then assists the author in generating the individual documents
(potentially in different versions)
as well as a document containing the collected series.
Another application is in developing style files
or other kinds of included material
where compilation of the style file could redirect
to a sample or test file.

%%%%%%%%%%%%%%%%%%%%%%%%%%%%%%%%%%%%%%%%%%%%%%%%%%%%%%%%%%%%%%%%%%%%%%%%%%%%%%%%
%%%%%%%%%%%%%%%%%%%%%%%%%%%%%%%%%%%%%%%%%%%%%%%%%%%%%%%%%%%%%%%%%%%%%%%%%%%%%%%%
\section{Usage}

First of all, the package \textsf{childdoc} is \emph{not} a standard
\LaTeXe{} |.sty| style file! Therefore it needs to be invoked in
a non-standard way.

%%%%%%%%%%%%%%%%%%%%%%%%%%%%%%%%%%%%%%%%%%%%%%%%%%%%%%%%%%%%%%%%%%%%%%%%%%%%%%%%
\subsection{Included Files}
\label{sec:include}

%%%%%%%%%%%%%%%%%%%%%%%%%%%%%%%%%%%%%%%%
\DescribeMacro{\childdocmain}
To use the package, add the commands
\begin{center}
\begin{tabular}{l}
|\input{childdoc.def}|\\
|\childdocmain{}|\\
\end{tabular}
\end{center}
at the very top of the main \LaTeX{} file,
in particular \emph{before} the |\documentclass| statement!
The argument of |\childdocmain| should be left empty
(but it must be present).

%%%%%%%%%%%%%%%%%%%%%%%%%%%%%%%%%%%%%%%%
\DescribeMacro{\childdocof}
Furthermore, add the commands
\begin{center}
\begin{tabular}{l}
|\input{childdoc.def}|\\
|\childdocof{|\textit{main}|}|\\
\end{tabular}
\end{center}
at the top of every child file \textit{child}
which is included by |\include{|\textit{child}|}|
from within the main file
(or at least for those files to be compiled individually).
The argument \textit{main} must be the filename of the main file.

There are a couple of
considerations in setting up the main and child documents:

%%%%%%%%%%%%%%%%%%%%%%%%%%%%%%%%%%%%%%%%
\paragraph{Restrictions.}

Please note the following restrictions:
\begin{itemize}
\item
|\childdocmain| must be called with one argument \textit{main}
to ensure compatibility with earlier version of the package.
It must either be empty (|\childdocmain{}|)
or precisely match the filename of the main file in which it is specified.
See \secref{sec:detection} for further information.
\item
The filename \textit{main} must be specified without the |.tex| extension.
\item
The filename \textit{main} is case sensitive
(even in case-insensitive file systems)
due to internal string comparison.
\item
The argument \textit{main} should be fully expanded, it cannot be a macro.
\item
Subdirectories and special characters should be avoided in filenames.
\item
The command |\childdocmain{|\textit{main}|}| must be followed by a whitespace.
It should not be followed immediately by another command
or by a comment mark `|%|'.
This is because the \TeX{} parser reads the token immediately following
the argument of |\childdocmain| and puts it
at the beginning of every child section;
however, a white\-space is ignored.
\end{itemize}

%%%%%%%%%%%%%%%%%%%%%%%%%%%%%%%%%%%%%%%%
\paragraph{Content of Main File.}

It is advisable to place all content in the child files included by |\include|.
Any output contained in the main file will appear in all child documents
unless suppressed manually;
it cannot be suppressed automatically by the |\includeonly| directive
and thus should normally be avoided.
A method to include some content in the main file
by means of conditional processing is described in \secref{sec:conditional}.

%%%%%%%%%%%%%%%%%%%%%%%%%%%%%%%%%%%%%%%%
\paragraph{Page Numbering.}

When only a part of the document is compiled,
the appropriate numbering of pages
(as well as other status parameters)
is determined from the |.aux| files.
The latter contain information from previous passes.
However this information needs to propagate through
all intermediate child documents.
Therefore the page numbering in child documents may well
be inconsistent until the complete document is compiled at least once.

A useful (if unconventional) way to always ensure a consistent
page numbering is to restart the numbering in each child document
and denote the pages by `\textit{child}|.|\textit{page}'
where \textit{child} represents the chapter/section number of the child file.
This can be achieved by the command
|\numberwithin{page}{|\textit{child}|}|
of the \textsf{amsmath} package
where \textit{child} can be |chapter| or |section|
depending on the chosen structuring.
Alternatively, one can modify the macro |\thepage| appropriately
and reset the counter |page| at the start of each child file.

%%%%%%%%%%%%%%%%%%%%%%%%%%%%%%%%%%%%%%%%%%%%%%%%%%%%%%%%%%%%%%%%%%%%%%%%%%%%%%%%
\subsection{Conditional Processing}
\label{sec:conditional}

The package provides a mechanism to compile different versions
of a document. To customise the versions further some conditional processing
can come in handy to distinguish which version is being compiled.
The package provides two macros to describe the compilation context:

%%%%%%%%%%%%%%%%%%%%%%%%%%%%%%%%%%%%%%%%
\DescribeMacro{\ifchilddoc}
The conditional |\ifchilddoc| distinguishes between the compilation of
child documents and the main document:
%
\begin{center}
|\ifchilddoc |\textit{child-code}| |[|\||else |\textit{main-code}]| \||fi|
\end{center}

%%%%%%%%%%%%%%%%%%%%%%%%%%%%%%%%%%%%%%%%
\DescribeMacro{\childdocname}
\DescribeMacro{\childdocjob}
The macro |\childdocname| contains the filename (without extension)
of the main or child file being processed.
Note that |\childdocjob| will always contain the name of the main file.

%%%%%%%%%%%%%%%%%%%%%%%%%%%%%%%%%%%%%%%%
\paragraph{Title Page.}

Conditional processing can be used to include a title or banner page
in the main document when proper precautions are taken.
Importantly, the code in the main file should ensure that the page counter
(as well as other status parameters which are stored in the |.aux| files)
takes the same value after the conditional processing.
Otherwise the page numbers may take divergent values
depending on which part is compiled.

For example, a title page could be declared by:
%
\begin{center}
\begin{tabular}{l}
|\ifchilddoc\||else|\\
|\addtocounter{page}{-1}|\\
\textit{code for title page}\\
|\newpage|\\
|\||fi|
\end{tabular}
\end{center}
%
A banner page for the child documents can be generated by:
%
\begin{center}
\begin{tabular}{l}
|\ifchilddoc|\\
|\addtocounter{page}{-1}|\\
\textit{code for banner page}\\
|\newpage|\\
|\||fi|
\end{tabular}
\end{center}
%
Here one could write a message such as:
\begin{center}
|This is the part \childdocname{} of \childdocjob{}.|
\end{center}

%%%%%%%%%%%%%%%%%%%%%%%%%%%%%%%%%%%%%%%%%%%%%%%%%%%%%%%%%%%%%%%%%%%%%%%%%%%%%%%%
\subsection{Flags}
\label{sec:flags}

The package makes it easy to generate different versions
of the main or child documents.
To this end compilation flags can be defined
and assigned different default values.
They will be particularly useful in conjunction
with the forwarding mechanism described in \secref{sec:forward}.

For example, it may be useful to have a flag |\version|
which can be set to |draft| or |final|.
The document source will contain some conditional code
depending on the value of |\version|.
Suppose further, the flag should default to |final| for the main file
and to |draft| for child files
which is a natural assignment for editing the document.
This is achieved by placing the following code
in the preamble of the main document
(below the |\childdocmain| directive):
%
\begin{center}
\begin{tabular}{l}
|\ifchilddoc|\\
|\providecommand{\version}{draft}|\\
|\||else|\\
|\providecommand{\version}{final}|\\
|\||fi|
\end{tabular}
\end{center}
%
The definition by |\providecommand| makes sure
that previous definitions are not overwritten.
Further statements |\providecommand{\version}{...}|
can thus be added before the above code to override it.

For the main file, one might add a line
(between |\childdocmain| and the above block)
%
\begin{center}
|%\ifchilddoc\||else\providecommand{\version}{draft}\||fi|
\end{center}
%
which can be uncommented to produce a draft version.
Likewise one can add a line to the very top of a child file
(above the |\childdocof{|\textit{main}|}| directive)
%
\begin{center}
|%\providecommand{\version}{final}|
\end{center}
%
which can be uncommented to produce the final version of this child document.

%%%%%%%%%%%%%%%%%%%%%%%%%%%%%%%%%%%%%%%%%%%%%%%%%%%%%%%%%%%%%%%%%%%%%%%%%%%%%%%%
\subsection{Forwarding}
\label{sec:forward}

Different versions of the main or child documents
using compilation flags as described in \secref{sec:flags}
can be (permanently) stored in different files
for convenient compilation, viewing and distribution.
To this end, the package defines a command
to pass on compilation to a different file:

%%%%%%%%%%%%%%%%%%%%%%%%%%%%%%%%%%%%%%%%
\DescribeMacro{\childdocforward}
The command |\childdocforward| redirects processing to
another source file:
%
\begin{center}
\begin{tabular}{l}
|\input{childdoc.def}|\\
|\childdocforward[|\textit{main}|]{|\textit{dest}|}|\\
\end{tabular}
\end{center}
%
The argument \textit{dest} is the destination file
(without extension).
It should be the main file or one of the child files.
Note that further \textsf{childdoc} directives
such as |\childdocof| and |\childdocforward|
in the indicated file will be processed in this form.
The optional argument \textit{main}
passes on directly to the main file \textit{main}
while pretending to compile the child \textit{dest}.
This form behaves as if \textit{dest}
issues |\childdocof{|\textit{main}|}| right away,
and no further \textsf{childdoc} directives will be processed.

%%%%%%%%%%%%%%%%%%%%%%%%%%%%%%%%%%%%%%%%
\DescribeMacro{\...prefix}
In the alternative form |\childdocforwardprefix|,
%
\begin{center}
\begin{tabular}{l}
|\input{childdoc.def}|\\
|\childdocforwardprefix[|\textit{main}|]{|\textit{prefix}|}{|\textit{dest}|}|
\end{tabular}
\end{center}
%
the destination file is determined by a pattern
depending on the current file:
To make this work, the current file must be called
`{\textit{prefix}\hspace{0.2em}\textit{suffix}}'
with \textit{prefix} matching precisely the argument.
Processing is then passed on to the file
`{\textit{dest}\hspace{0.2em}\textit{suffix}}'.
Surely, the same effect is achieved by
directly specifying the
argument `{\textit{dest}\hspace{0.2em}\textit{suffix}}'
in the first form.
However, that requires to set up a different file
for each child. With the alternative form of the command
all these files can have exactly the same content
which simplifies setting them up and maintaining them.

For example, the following file |draft.tex|
with a compilation flag |\version| as described in \secref{sec:flags}
compiles the main document as a draft:
%
\begin{center}
\begin{tabular}{l}
|\def\version{draft}|\\
|\input{childdoc.def}|\\
|\childdocforward{|\textit{main}|}|
\end{tabular}
\end{center}
%
Likewise, the following files |final|\textit{nn}|.tex|
compile the final version of the child document
|child|\textit{nn}|.tex|:
%
\begin{center}
\begin{tabular}{l}
|\def\version{final}|\\
|\input{childdoc.def}|\\
|\childdocforwardprefix{final}{child}|
\end{tabular}
\end{center}
%

Note that when several versions of a main file and/or of each child file
are to be generated, it may be convenient to set up a |Makefile| or
shell script to automatise the process.

%%%%%%%%%%%%%%%%%%%%%%%%%%%%%%%%%%%%%%%%%%%%%%%%%%%%%%%%%%%%%%%%%%%%%%%%%%%%%%%%
\subsection{Command Line Processing}
\label{sec:commandline}

The effect of redirection files can also be achieved by invoking
the \LaTeX{} compiler with a more elaborate command line.
Most conveniently this should be done as part
of a shell script or a |Makefile|.

When using \textsf{childdoc} in the main file, the following
command lines effectively perform a redirection
(note that depending on the shell being used,
backslashes may have to be doubled: `|\|' $\to$ `|\\|'):
%
\begin{center}
|... -jobname "|\textit{target}|" |\\|"|[\textit{flags}]%
|\input{childdoc.def}\childdocforward[|\textit{main}|]{|\textit{dest}|}"|
\end{center}
%
Here \textit{target} is the name of the output file,
\textit{main} is the name of the main file
and \textit{dest} is the name of the main or child file to be processed
(all filenames without extensions).
The optional argument \textit{main} can be omitted
if \textit{main} matches \textit{dest}.
Optionally, compilation \textit{flags} can be defined via |\def| commands.
This command line makes the \TeX{} engine believe
it is compiling the file \textit{target}
whose content is specified as the latter parameter.
The provided code then forwards the processing to
\textit{main} or \textit{dest} as described in \secref{sec:forward}.

%%%%%%%%%%%%%%%%%%%%%%%%%%%%%%%%%%%%%%%%%%%%%%%%%%%%%%%%%%%%%%%%%%%%%%%%%%%%%%%%
\subsection{Include by Input}
\label{sec:input}

Including child documents by |\include| has some restrictions by design.
Most notably, the content of a child document always occupies
its own set of pages; pages cannot be shared between child documents.
Usually, this behaviour makes perfect sense
because each child document contain an essential part of the document.
However, in some situations it may be desirable to compose
a document from a collection of parts
without having mandatory page breaks between then.
For this case, the package
provides a mechanism to include parts
by |\input| which can also be processed individually.
However, by construction this mechanism
requires manual handling of the content to be output.

%%%%%%%%%%%%%%%%%%%%%%%%%%%%%%%%%%%%%%%%
\DescribeMacro{\ifchilddocmanual}
The main file should be prepared as usual, see \secref{sec:include}.
However, the document body must make a distinction
between processing of an individual part and of the main document, e.g.:
%
\begin{center}
\begin{tabular}{l}
|\ifchilddocmanual|\\
|\input{\childdocname}|\\
|\||else|\\
\textit{document body with }|\input{|\textit{part}|}|\\
|\||fi|
\end{tabular}
\end{center}
%
The conditional |\ifchilddocmanual| is true whenever
a part to be included by |\input| is being compiled,
and the name of the part is stored in |\childdocname|.

%%%%%%%%%%%%%%%%%%%%%%%%%%%%%%%%%%%%%%%%
\DescribeMacro{\childdocby}
Each part to be included by |\input| should start with:
%
\begin{center}
\begin{tabular}{l}
|\input{childdoc.def}|\\
|\childdocby{|\textit{main}|}|\\
\end{tabular}
\end{center}
%
The directive |\childdocby| is similar to |\childdocof|
described in \secref{sec:include},
but the subsequent selection of content must be done manually.
To that end, both |\ifchilddoc| and |\ifchilddocmanual|
will be true upon processing of a part,
and the name of the part is stored in |\childdocname|.
Note that |\jobname| will be set to the filename of the current part
so that each part receives an individual |.aux| file
that does not interfere with the |.aux| file(s) of the main document.
This behaviour can be altered by the alternative form
|\childdocby[*]{|\textit{main}|}| (with a non-empty optional argument)
which uses the |.aux| file of the main document
by setting |\jobname| to \textit{main}.

%%%%%%%%%%%%%%%%%%%%%%%%%%%%%%%%%%%%%%%%%%%%%%%%%%%%%%%%%%%%%%%%%%%%%%%%%%%%%%%%
\subsection{Driver Development}
\label{sec:driver}

The \textsf{childdoc} mechanism can also be use for the development
of definition files such as \LaTeX{} styles or classes.
This case differs from the above setup with multiple parts
included by |\include| in that no |\includeonly| should be invoked.
This can be achieved by starting the include file
(before |\ProvidesPackage|) with:
%
\begin{center}
\begin{tabular}{l}
|\input{childdoc.def}|\\
|\childdocforward{|\textit{main}|}|\\
\end{tabular}
\end{center}
%
or alternatively with:
%
\begin{center}
\begin{tabular}{l}
|\input{childdoc.def}|\\
|\childdocby{|\textit{main}|}|\\
\end{tabular}
\end{center}
%
Both forms have slightly different effects as described above.
The main file is prepared as usual, see \secref{sec:include}.

%%%%%%%%%%%%%%%%%%%%%%%%%%%%%%%%%%%%%%%%%%%%%%%%%%%%%%%%%%%%%%%%%%%%%%%%%%%%%%%%
\subsection{Legacy Detection}
\label{sec:detection}

The directive |\childdocmain| in the main file can detect
whether the complete document or merely a child is to be compiled
even without using the directive |\childdocof|.
This method is deprecated because it is less robust
and there is no compelling reason to use it;
it is merely provided for backward compatibility
and it may be removed in future versions.

If the detection mechanism is to be used,
it is mandatory to correctly specify
the filename of the main file as the argument of |\childdocmain|:
%
\begin{center}
\begin{tabular}{l}
|\input{childdoc.def}|\\
|\childdocmain{|\textit{main}|}|\\
\end{tabular}
\end{center}
%
If |\jobname| does not match the argument \textit{main} of |\childdocmain|,
it is assumed that |\jobname| points to the child file to be compiled.
When using |\childdocmain| with the main file specified as argument,
it suffices to start a child file
with just |\input{|\textit{main}|}|
without loading of the package and using |\childdocof|.
If instead all processing is done
with the appropriate \textsf{childdoc} directives,
the argument of \textit{main} of |\childdocmain| can be empty.

An alternative version of the command line processing described
in \secref{sec:commandline} using the detection mechanism reads:
%
\begin{center}
|... -jobname "|\textit{target}|" "|[\textit{flags}]%
[|\def\jobname{|\textit{dest}|}|]|\input{|\textit{main}|}"|
\end{center}

%%%%%%%%%%%%%%%%%%%%%%%%%%%%%%%%%%%%%%%%%%%%%%%%%%%%%%%%%%%%%%%%%%%%%%%%%%%%%%%%
\subsection{Manual Code}
\label{sec:manual}

In case one cannot be certain whether the definitions file |childdoc.def|
is installed on the target \TeX{} distribution
and one prefers not to ship it,
it is conceivable to paste a few relevant commands into the sources.

To that end, drop all statements |\input{childdoc.def}|
and perform the replacements as outlined below.
Instead of |\childdocmain{|\textit{main}|}| add the following code
to the top of the main file:
%
\begin{center}
\begin{tabular}{l}
|\||ifdefined\childdocname\endinput\||fi\newif\ifchilddoc|\\
|\edef\childdocname{\scantokens\expandafter{\jobname\noexpand}}|\\
|\def\childdocmain{|\textit{main}|}\||ifx\childdocmain\childdocname\||else|\\
|\childdoctrue\includeonly{\childdocname}\let\jobname\childdocmain\||fi|\\
\end{tabular}
\end{center}
%
Instead of |\childdocof{|\textit{main}|}| just include the main file
at the top of each child file:
%
\begin{center}
|\input{|\textit{main}|}|
\end{center}
%
A simple redirection |\childdocforward{|\textit{dest}|}| is achieved by:
%
\begin{center}
|\def\jobname{|\textit{dest}|}\input{\jobname}|
\end{center}
%
The redirection with prefix
|\childdocforwardprefix[|\textit{prefix}|]{|\textit{dest}|}|
is accomplished by:
%
\begin{center}
\begin{tabular}{l}
|{\edef\jobname{\scantokens\expandafter{\jobname\noexpand}}|\\
|\def\redirectjob |\textit{prefix}|#1~~~{\gdef\jobname{|\textit{dest}|#1}}|\\
|\expandafter\redirectjob\jobname~~~}\input{\jobname}|
\end{tabular}
\end{center}

In an alternative approach,
child documents can be compiled by a specific command line
without additional code or specific definitions:
%
\begin{center}
|... -jobname "|\textit{target}|" "|[\textit{flags}]%
|\includeonly{|\textit{dest}|}\input{|\textit{main}|}"|
\end{center}
%

%%%%%%%%%%%%%%%%%%%%%%%%%%%%%%%%%%%%%%%%%%%%%%%%%%%%%%%%%%%%%%%%%%%%%%%%%%%%%%%%
%%%%%%%%%%%%%%%%%%%%%%%%%%%%%%%%%%%%%%%%%%%%%%%%%%%%%%%%%%%%%%%%%%%%%%%%%%%%%%%%
\section{Information}

%%%%%%%%%%%%%%%%%%%%%%%%%%%%%%%%%%%%%%%%%%%%%%%%%%%%%%%%%%%%%%%%%%%%%%%%%%%%%%%%
\subsection{Copyright}

Copyright \copyright{} 2017--2018 Niklas Beisert

This work may be distributed and/or modified under the
conditions of the \LaTeX{} Project Public License, either version 1.3
of this license or (at your option) any later version.
The latest version of this license is in
  \url{http://www.latex-project.org/lppl.txt}
and version 1.3 or later is part of all distributions of \LaTeX{}
version 2005/12/01 or later.

This work has the LPPL maintenance status `maintained'.

The Current Maintainer of this work is Niklas Beisert.

This work consists of the files |README.txt|, |childdoc.ins| and |childdoc.dtx|
as well as the derived files |childdoc.def|, |cdocsamp.tex|
with |cdocsch1.tex|, |cdocsch2.tex|, |cdocspt3.tex|, |cdocspt4.tex|,
|cdocsdrf.tex|, |cdocsfn1.tex|, |cdocsfn2.tex|
as well as |childdoc.pdf|.

%%%%%%%%%%%%%%%%%%%%%%%%%%%%%%%%%%%%%%%%%%%%%%%%%%%%%%%%%%%%%%%%%%%%%%%%%%%%%%%%
\subsection{Files and Installation}

The package consists of the files:
%
\begin{center}
\begin{tabular}{ll}
    |README.txt|   & readme file \\
    |childdoc.ins| & installation file \\
    |childdoc.dtx| & source file \\
    |childdoc.def| & definition file \\
    |cdocsamp.tex| & sample main file \\
    |cdocsch1.tex| & sample include file \\
    |cdocsch2.tex| & sample include file \\
    |cdocspt3.tex| & sample part file \\
    |cdocspt4.tex| & sample part file \\
    |cdocsdrf.tex| & sample redirection file \\
    |cdocsfn1.tex| & sample redirection file \\
    |cdocsfn2.tex| & sample redirection file \\
    |childdoc.pdf| & manual
\end{tabular}
\end{center}
%
The distribution consists of the files
|README.txt|, |childdoc.ins| and |childdoc.dtx|.
%
\begin{itemize}
\item
Run (pdf)\LaTeX{} on |childdoc.dtx|
to compile the manual |childdoc.pdf| (this file).
\item
Run \LaTeX{} on |childdoc.ins| to create the definitions file |childdoc.def|
and the sample |cdocsamp.tex| with include files
|cdocsch1.tex|, |cdocsch2.tex|, |cdocspt3.tex|, |cdocspt4.tex|,
|cdocsdrf.tex|, |cdocsfn1.tex|, |cdocsfn2.tex|.
Then copy the file |childdoc.def| to an appropriate directory of your \LaTeX{}
distribution, e.g.\ \textit{texmf-root}|/tex/latex/childdoc|.
\end{itemize}

%%%%%%%%%%%%%%%%%%%%%%%%%%%%%%%%%%%%%%%%%%%%%%%%%%%%%%%%%%%%%%%%%%%%%%%%%%%%%%%%
\subsection{Related CTAN Packages}

There are several other packages which offer a similar functionality:
%
\begin{itemize}
\item
The packages
\href{http://ctan.org/pkg/docmute}{\textsf{docmute}},
\href{http://ctan.org/pkg/includex}{\textsf{includex}} and
\href{http://ctan.org/pkg/standalone}{\textsf{standalone}}
provide commands to include only the document body of
a child file thus allowing both files to be compiled individually.
\item
The packages \href{http://ctan.org/pkg/subdocs}{\textsf{subdocs}}
and \href{http://ctan.org/pkg/subfiles}{\textsf{subfiles}}
provide structures in which the main and child documents can be
encapsulated and allowing them to be compiled individually.
The inclusion mechanism is different from the conventional |\include|.
\item
The package \href{http://ctan.org/pkg/combine}{\textsf{combine}}
is an elaborate solution to combine several documents into one.
\end{itemize}
%
See also the CTAN topic \href{http://ctan.org/topic/subdocs}{\textsf{subdocs}}
for further related packages.
The present package differs from the above solutions in that
a document structure constructed with the conventional |\include| mechanism
just needs two extra commands at the top of every file
such that all constituent files can be compiled individually.

%%%%%%%%%%%%%%%%%%%%%%%%%%%%%%%%%%%%%%%%%%%%%%%%%%%%%%%%%%%%%%%%%%%%%%%%%%%%%%%%
%\subsection{Feature Suggestions}
%
%The following is a list of features which may be useful for future
%versions of this package:
%%
%\begin{itemize}
%\item
%\ldots
%\end{itemize}

%%%%%%%%%%%%%%%%%%%%%%%%%%%%%%%%%%%%%%%%%%%%%%%%%%%%%%%%%%%%%%%%%%%%%%%%%%%%%%%%
\subsection{Revision History}

%%%%%%%%%%%%%%%%%%%%%%%%%%%%%%%%%%%%%%%%
\paragraph{v2.0:} 2018/12/30

\begin{itemize}
\item
immediate forward processing
\item
added |\childdocby| mechanism
\item
manual restructured
\end{itemize}

%%%%%%%%%%%%%%%%%%%%%%%%%%%%%%%%%%%%%%%%
\paragraph{v1.6:} 2018/01/17

\begin{itemize}
\item
application for development of include files
\item
corrections to manual
\end{itemize}

%%%%%%%%%%%%%%%%%%%%%%%%%%%%%%%%%%%%%%%%
\paragraph{v1.5:} 2017/05/21

\begin{itemize}
\item
more complete structuring introduced
\item
|\childdocof| introduced
\item
|\childdoc| renamed to |\childdocmain|
\item
|\childredirect| renamed to |\childdocforward| and |\childdocforwardprefix|
and functionality expanded
\end{itemize}

%%%%%%%%%%%%%%%%%%%%%%%%%%%%%%%%%%%%%%%%
\paragraph{v1.0:} 2017/04/27

\begin{itemize}
\item
manual and install package
\item
first version published on CTAN
\end{itemize}

%%%%%%%%%%%%%%%%%%%%%%%%%%%%%%%%%%%%%%%%
\paragraph{v0.6:} 2017/04/26

\begin{itemize}
\item
redirection mechanism added
\end{itemize}

%%%%%%%%%%%%%%%%%%%%%%%%%%%%%%%%%%%%%%%%
\paragraph{v0.5:} 2017/04/26

\begin{itemize}
\item
functionality in definition file
\end{itemize}


%%%%%%%%%%%%%%%%%%%%%%%%%%%%%%%%%%%%%%%%%%%%%%%%%%%%%%%%%%%%%%%%%%%%%%%%%%%%%%%%
%%%%%%%%%%%%%%%%%%%%%%%%%%%%%%%%%%%%%%%%%%%%%%%%%%%%%%%%%%%%%%%%%%%%%%%%%%%%%%%%
%%%%%%%%%%%%%%%%%%%%%%%%%%%%%%%%%%%%%%%%%%%%%%%%%%%%%%%%%%%%%%%%%%%%%%%%%%%%%%%%
\appendix

\settowidth\MacroIndent{\rmfamily\scriptsize 000\ }

 \DocInput{childdoc.dtx}

\end{document}
%</driver>
% \fi
%
% %%%%%%%%%%%%%%%%%%%%%%%%%%%%%%%%%%%%%%%%%%%%%%%%%%%%%%%%%%%%%%%%%%%%%%%%%%%%%%
% %%%%%%%%%%%%%%%%%%%%%%%%%%%%%%%%%%%%%%%%%%%%%%%%%%%%%%%%%%%%%%%%%%%%%%%%%%%%%%
% \section{Sample}
%\iffalse
%<*samplemain>
%\fi
%
% The following presents a sample document
% with two chapters, two parts, a title page,
% a compile flag as well as three forwarding files to set the flag.
% It consists of eight |.tex| files:
% \begin{center}
% \begin{tabular}{ll}
% |cdocsamp.tex|&main file\\
% |cdocsch1.tex|&include file for chapter 1\\
% |cdocsch2.tex|&include file for chapter 2\\
% |cdocspt3.tex|&include file for part 3\\
% |cdocspt4.tex|&include file for part 4\\
% |cdocsdrf.tex|&forwarding file for main file in draft mode\\
% |cdocsfi1.tex|&forwarding file for final version of chapter 1\\
% |cdocsfi2.tex|&forwarding file for final version of chapter 2\\
% \end{tabular}
% \end{center}
% Each of the eight files can be compiled directly by the \LaTeX{} compiler.
%
% %%%%%%%%%%%%%%%%%%%%%%%%%%%%%%%%%%%%%%
% \paragraph{Main File.}
%
% The main file is called |cdocsamp.tex|.
%
% Load the \textsf{childdoc} definitions and
% declare the filename for the main document:
%    \begin{macrocode}
\input{childdoc.def}
\childdocmain{}
%    \end{macrocode}

% Optional override for |\version| flag:
%    \begin{macrocode}
%%\ifchilddoc\else\providecommand{\version}{draft}\fi
%    \end{macrocode}

% Define the default values for the |\version| flag
% (|final| for the main file and |draft| for childs):
%    \begin{macrocode}
\ifchilddoc
\providecommand{\version}{draft}
\else
\providecommand{\version}{final}
\fi
%    \end{macrocode}

% Load the standard document class:
%    \begin{macrocode}
\documentclass[12pt]{article}
%    \end{macrocode}

% Start the document body:
%    \begin{macrocode}
\begin{document}
%    \end{macrocode}

% Declare a title page.
% Print title, part of document being processed and version flag:
%    \begin{macrocode}
\addtocounter{page}{-1}
\begin{center}
{\LARGE\bfseries{}childdoc example\par}
\vspace{1cm}
\ifchilddoc
\ifchilddocmanual part\else chapter\fi:
`\childdocname' of `\childdocjob'\par
\else
main document: `\childdocjob'\par
\fi
version: \version\par
\end{center}
\newpage
%    \end{macrocode}

% Manually include selected file,
% otherwise process as usual:
%    \begin{macrocode}
\ifchilddocmanual
\section*{part `\childdocname'}
\input{\childdocname}
\else
%    \end{macrocode}

% Include the two chapters:
%    \begin{macrocode}
\include{cdocsch1}
\include{cdocsch2}
%    \end{macrocode}

% Include the two parts unless only chapters should be displayed:
%    \begin{macrocode}
\ifchilddoc\else
\section{part three}
\input{cdocspt3}
\section{part four}
\input{cdocspt4}
\fi
%    \end{macrocode}

% Process as usual until here:
%    \begin{macrocode}
\fi
%    \end{macrocode}

% End of document body:
%    \begin{macrocode}
\end{document}
%    \end{macrocode}
%\iffalse
%</samplemain>
%\fi
%
% %%%%%%%%%%%%%%%%%%%%%%%%%%%%%%%%%%%%%%
% \paragraph{Chapter Include Files.}
%
% The include files are called |cdocsch1.tex| and |cdocsch2.tex|.
%
%\iffalse
%<*samplechap1|samplechap2>
%\fi

% Optional override for |\version| flag:
%    \begin{macrocode}
%%\providecommand{\version}{final}
%    \end{macrocode}

% Include the main document:
%    \begin{macrocode}
\input{childdoc.def}
\childdocof{cdocsamp}
%    \end{macrocode}

%\iffalse
%</samplechap1|samplechap2>
%\fi
%
%\iffalse
%<*samplechap1>
%\fi
% Some text for chapter 1:
%    \begin{macrocode}
\section{one}
some text in chapter one
%    \end{macrocode}

%\iffalse
%</samplechap1>
%\fi
% Some text for chapter 2:
%\iffalse
%<*samplechap2>
%\fi
%    \begin{macrocode}
\section{two}
more text in chapter two
%    \end{macrocode}

%\iffalse
%</samplechap2>
%\fi
%
% %%%%%%%%%%%%%%%%%%%%%%%%%%%%%%%%%%%%%%
% \paragraph{Part Include Files.}
%
% The include files are called |cdocspt3.tex| and |cdocspt4.tex|.
%
%\iffalse
%<*samplepart3|samplepart4>
%\fi

% Optional override for |\version| flag:
%    \begin{macrocode}
%%\providecommand{\version}{final}
%    \end{macrocode}

% Include the main document:
%    \begin{macrocode}
\input{childdoc.def}
\childdocby{cdocsamp}
%    \end{macrocode}

%\iffalse
%</samplepart3|samplepart4>
%\fi
%
%\iffalse
%<*samplepart3>
%\fi
% Some text for part 3:
%    \begin{macrocode}
some text in part three
%    \end{macrocode}

%\iffalse
%</samplepart3>
%\fi
% Some text for part 4:
%\iffalse
%<*samplepart4>
%\fi
%    \begin{macrocode}
more text in part four
%    \end{macrocode}

%\iffalse
%</samplepart4>
%\fi
%
% %%%%%%%%%%%%%%%%%%%%%%%%%%%%%%%%%%%%%%
% \paragraph{Forwarding for a Complete Draft.}
%
% The following forwarding file |cdocsdrf.tex|
% compiles the main document in draft mode:
%\iffalse
%<*sampledraft>
%\fi
%    \begin{macrocode}
\def\version{draft}
\input{childdoc.def}
\childdocforward{cdocsamp}
%    \end{macrocode}

%\iffalse
%</sampledraft>
%\fi
%
% %%%%%%%%%%%%%%%%%%%%%%%%%%%%%%%%%%%%%%
% \paragraph{Forwarding for Final Version of the Chapters.}
%
% The following forwarding files |cdocsfn1.tex| and |cdocsfn2.tex|
% (with identical content)
% compile the final versions of the child documents
% |cdocsch1.tex| and |cdocsch2.tex|, respectively:
%\iffalse
%<*samplefinal>
%\fi
%    \begin{macrocode}
\def\version{final}
\input{childdoc.def}
\childdocforwardprefix[cdocsamp]{cdocsfn}{cdocsch}
%    \end{macrocode}

%\iffalse
%</samplefinal>
%\fi
%
% %%%%%%%%%%%%%%%%%%%%%%%%%%%%%%%%%%%%%%
% \paragraph{Command Line Processing.}
%
% The following three command lines generate the output files
% |cdocscld|, |cdocscl1| and |cdocscl2|
% which should be identical to
% |cdocsdrf|, |cdocsch1| and |cdocsfn2|, respectively:
% \begin{center}
% \begin{tabular}{l}
% |latex -jobname cdocscld \|\\
% |  "\def\version{draft}\input{childdoc.def}\childdocforward{cdocsamp}"|\\
% |latex -jobname cdocscl1 \|\\
% |  "\input{childdoc.def}\childdocforward[cdocsamp]{cdocsch1}"|\\
% |latex -jobname cdocscl2 \|\\
% |  "\def\version{final}\input{childdoc.def}\childdocforward{cdocsch2}"|
% \end{tabular}
% \end{center}
% Note that the trailing backslash on each first line
% merely continues the input to the second line
% (for convenient cut ant paste).
% Furthermore, the command |latex| can be replaced by any
% of its alternative versions such as |pdflatex|.
%
% %%%%%%%%%%%%%%%%%%%%%%%%%%%%%%%%%%%%%%%%%%%%%%%%%%%%%%%%%%%%%%%%%%%%%%%%%%%%%%
% %%%%%%%%%%%%%%%%%%%%%%%%%%%%%%%%%%%%%%%%%%%%%%%%%%%%%%%%%%%%%%%%%%%%%%%%%%%%%%
% \section{Implementation}
%\iffalse
%<*package>
%\fi
%
% This section describes the definitions file |childdoc.def|.

% The definitions cannot be loaded using |\usepackage| or |\RequirePackage|
% which has a mechanism to prevent loading a style file more than once.
% When loading the definitions by means of |\input|
% multiple instances have to be prevented manually:
%\iffalse
%This code needs to be before the `\ProvidesFile' directive
%which is defined at the beginning of this file.
%Therefore it is also placed there and commented out here.
%</package>
%<*discard>
%\fi
%    \begin{macrocode}
\ifdefined\childdocmain\endinput\fi
%    \end{macrocode}
%\iffalse
%</discard>
%<*package>
%\fi
%
% \macro{\ifchilddoc}
% \macro{\ifchilddocmanual}
% The conditional |\ifchilddoc| tells whether a
% child (true) or main (false) document is being compiled.
% The conditional |\ifchilddocmanual| tells whether
% the |\includeonly| mechanism is used (false) or
% the selection of child files must be performed manually (true).
% The definitions initialise to false:
%    \begin{macrocode}
\newif\ifchilddoc
\newif\ifchilddocmanual
%    \end{macrocode}

% \macro{\childdocname}
% \macro{\childdocjob}
% The macro |\childdocname| stores the name of the main document
% to be compiled. The macro |\childdocjob| stores the name of
% the document on which the \LaTeX{} compiler was originally invoked.
% The content of |\jobname| cannot be compared
% to filenames specified in the source due to different catcodes.
% The following code rescans |\jobname|, stores the result
% in |\childdocname| and saves a copy in |\childdocjob|:
%    \begin{macrocode}
\edef\childdocname{\scantokens\expandafter{\jobname\noexpand}}
\let\childdocjob\childdocname
%    \end{macrocode}

% \macro{\childdocdisable}
% The macro |\childdocdisable| prevents the main file
% from being processed more than once.
% At this stage, the main document command |\childdocmain|
% is assumed to be called once again where it should do nothing.
% Any subsequent call to it should prevent
% a secondary processing of the main document
% It overwrites the forwarding commands
% |\childdocof| and |\childdocforward|
% with empty macros to prevent further inclusions of the main document:
%    \begin{macrocode}
\newcommand{\childdocdisable}
{
  \renewcommand{\childdocmain}[1]{\renewcommand{\childdocmain}[1]{\endinput}}
  \renewcommand{\childdocof}[1]{}
  \renewcommand{\childdocby}[2][]{}
  \renewcommand{\childdocforward}[2][]{}
  \renewcommand{\childdocdisable}{}
}
%    \end{macrocode}

% \macro{\childdocmain}
% The macro |\childdocmain| is to be called at the top of the main file
% with nothing or the main filename (without extension) as argument.
% First, it breaks loops.
% If the argument is not empty and does not match |\childdocname|
% (which is set by the first inclusion of |childdoc.def|),
% |\ifchilddoc| is set to true, |\includeonly| is applied to the child file
% and |\jobname| is set to the main file
% (for proper handling of |.aux| files):
%    \begin{macrocode}
\newcommand{\childdocmain}[1]
{
  \childdocdisable\childdocmain{}
  \if?#1?\else
    \begingroup
      \def\childdoctmp{#1}
      \ifx\childdoctmp\childdocname
        \def\childdoctmp{}
      \else
        \def\childdoctmp
        {
          \childdoctrue
          \includeonly{\childdocname}
          \def\childdocjob{#1}
          \def\jobname{#1}
        }
      \fi
      \expandafter
    \endgroup
    \childdoctmp
  \fi
}
%    \end{macrocode}

% \macro{\childdocof}
% The command |\childdocof| redirects
% compilation to the main file |#1|.
%    \begin{macrocode}
\newcommand{\childdocof}[1]
{
  \childdocdisable
  \childdoctrue
  \includeonly{\childdocname}
  \def\jobname{#1}
  \def\childdocjob{#1}
  \input{#1}
}
%    \end{macrocode}

% \macro{\childdocby}
% The command |\childdocby| ....
%    \begin{macrocode}
\newcommand{\childdocby}[2][]
{
  \childdocdisable
  \childdoctrue
  \childdocmanualtrue
  \if?#1?\else
    \def\jobname{#2}
  \fi
  \def\childdocjob{#2}
  \input{#2}
  \endinput
}
%    \end{macrocode}

% \macro{\childdocforward}
% The command |\childdocforward| redirects
% compilation to the main file or
% (if the optional argument is given) a child file.
% Parameters are set as if the main file
% or a child file starting with |\childdocof| was compiled.
% Then compilation is handed over to the main file:
%    \begin{macrocode}
\newcommand{\childdocforward}[2][]
{
  \begingroup
    \if?#1?
      \def\childdoctmp
      {
        \def\childdocname{#2}
        \def\childdocjob{#2}
        \def\jobname{#2}
        \input{#2}
        \endinput
      }
    \else
      \def\childdoctmp
      {
        \childdocdisable
        \def\childdocname{#2}
        \childdoctrue
        \includeonly{#2}
        \def\childdocjob{#1}
        \def\jobname{#1}
        \input{#1}
        \endinput
      }
    \fi
    \expandafter
  \endgroup
  \childdoctmp
}
%    \end{macrocode}

% \macro{\childdocforwardprefix}
% The command |\childdocforwardprefix| redirects
% compilation to the main or a child file by means of a pattern.
% The prefix |#1| in the current filename is replaced by |#2|
% and the suffix of the current filename is kept
% (it is assumed that the filename does not contain the substring `|~~~|'
% which is used as a delimiter).
% Compilation is handed over to the new file by |\childdocforward|:
%    \begin{macrocode}
\newcommand{\childdocforwardprefix}[3][]
{
  \begingroup
    \def\childdocextract #2##1~~~{\def\childdoctmp{\childdocforward[#1]{#3##1}}}
    \expandafter\childdocextract\childdocname~~~
    \expandafter
  \endgroup
  \childdoctmp
}
%    \end{macrocode}

% \macro{\childdoc}
% The deprecated macro |\childdoc| is a legacy version of |\childdocmain|:
%    \begin{macrocode}
\newcommand{\childdoc}{\childdocmain}
%    \end{macrocode}

% \macro{\childdocredirect}
% The deprecated macro |\childdocredirect| is a legacy version
% of |\childdocforward| and |\childdocforwardprefix|:
%    \begin{macrocode}
\newcommand{\childdocredirect}[2][]
{
  \begingroup
    \if?#1?
      \def\childdoctmp{\childdocforward{#2}}
    \else
      \def\childdoctmp{\childdocforwardprefix{#1}{#2}}
    \fi
    \expandafter
  \endgroup
  \childdoctmp
}
%    \end{macrocode}

%\iffalse
%</package>
%\fi
%
\endinput
|\\
|\childdocof{|\textit{main}|}|\\
\end{tabular}
\end{center}
at the top of every child file \textit{child}
which is included by |\include{|\textit{child}|}|
from within the main file
(or at least for those files to be compiled individually).
The argument \textit{main} must be the filename of the main file.

There are a couple of
considerations in setting up the main and child documents:

%%%%%%%%%%%%%%%%%%%%%%%%%%%%%%%%%%%%%%%%
\paragraph{Restrictions.}

Please note the following restrictions:
\begin{itemize}
\item
|\childdocmain| must be called with one argument \textit{main}
to ensure compatibility with earlier version of the package.
It must either be empty (|\childdocmain{}|)
or precisely match the filename of the main file in which it is specified.
See \secref{sec:detection} for further information.
\item
The filename \textit{main} must be specified without the |.tex| extension.
\item
The filename \textit{main} is case sensitive
(even in case-insensitive file systems)
due to internal string comparison.
\item
The argument \textit{main} should be fully expanded, it cannot be a macro.
\item
Subdirectories and special characters should be avoided in filenames.
\item
The command |\childdocmain{|\textit{main}|}| must be followed by a whitespace.
It should not be followed immediately by another command
or by a comment mark `|%|'.
This is because the \TeX{} parser reads the token immediately following
the argument of |\childdocmain| and puts it
at the beginning of every child section;
however, a white\-space is ignored.
\end{itemize}

%%%%%%%%%%%%%%%%%%%%%%%%%%%%%%%%%%%%%%%%
\paragraph{Content of Main File.}

It is advisable to place all content in the child files included by |\include|.
Any output contained in the main file will appear in all child documents
unless suppressed manually;
it cannot be suppressed automatically by the |\includeonly| directive
and thus should normally be avoided.
A method to include some content in the main file
by means of conditional processing is described in \secref{sec:conditional}.

%%%%%%%%%%%%%%%%%%%%%%%%%%%%%%%%%%%%%%%%
\paragraph{Page Numbering.}

When only a part of the document is compiled,
the appropriate numbering of pages
(as well as other status parameters)
is determined from the |.aux| files.
The latter contain information from previous passes.
However this information needs to propagate through
all intermediate child documents.
Therefore the page numbering in child documents may well
be inconsistent until the complete document is compiled at least once.

A useful (if unconventional) way to always ensure a consistent
page numbering is to restart the numbering in each child document
and denote the pages by `\textit{child}|.|\textit{page}'
where \textit{child} represents the chapter/section number of the child file.
This can be achieved by the command
|\numberwithin{page}{|\textit{child}|}|
of the \textsf{amsmath} package
where \textit{child} can be |chapter| or |section|
depending on the chosen structuring.
Alternatively, one can modify the macro |\thepage| appropriately
and reset the counter |page| at the start of each child file.

%%%%%%%%%%%%%%%%%%%%%%%%%%%%%%%%%%%%%%%%%%%%%%%%%%%%%%%%%%%%%%%%%%%%%%%%%%%%%%%%
\subsection{Conditional Processing}
\label{sec:conditional}

The package provides a mechanism to compile different versions
of a document. To customise the versions further some conditional processing
can come in handy to distinguish which version is being compiled.
The package provides two macros to describe the compilation context:

%%%%%%%%%%%%%%%%%%%%%%%%%%%%%%%%%%%%%%%%
\DescribeMacro{\ifchilddoc}
The conditional |\ifchilddoc| distinguishes between the compilation of
child documents and the main document:
%
\begin{center}
|\ifchilddoc |\textit{child-code}| |[|\||else |\textit{main-code}]| \||fi|
\end{center}

%%%%%%%%%%%%%%%%%%%%%%%%%%%%%%%%%%%%%%%%
\DescribeMacro{\childdocname}
\DescribeMacro{\childdocjob}
The macro |\childdocname| contains the filename (without extension)
of the main or child file being processed.
Note that |\childdocjob| will always contain the name of the main file.

%%%%%%%%%%%%%%%%%%%%%%%%%%%%%%%%%%%%%%%%
\paragraph{Title Page.}

Conditional processing can be used to include a title or banner page
in the main document when proper precautions are taken.
Importantly, the code in the main file should ensure that the page counter
(as well as other status parameters which are stored in the |.aux| files)
takes the same value after the conditional processing.
Otherwise the page numbers may take divergent values
depending on which part is compiled.

For example, a title page could be declared by:
%
\begin{center}
\begin{tabular}{l}
|\ifchilddoc\||else|\\
|\addtocounter{page}{-1}|\\
\textit{code for title page}\\
|\newpage|\\
|\||fi|
\end{tabular}
\end{center}
%
A banner page for the child documents can be generated by:
%
\begin{center}
\begin{tabular}{l}
|\ifchilddoc|\\
|\addtocounter{page}{-1}|\\
\textit{code for banner page}\\
|\newpage|\\
|\||fi|
\end{tabular}
\end{center}
%
Here one could write a message such as:
\begin{center}
|This is the part \childdocname{} of \childdocjob{}.|
\end{center}

%%%%%%%%%%%%%%%%%%%%%%%%%%%%%%%%%%%%%%%%%%%%%%%%%%%%%%%%%%%%%%%%%%%%%%%%%%%%%%%%
\subsection{Flags}
\label{sec:flags}

The package makes it easy to generate different versions
of the main or child documents.
To this end compilation flags can be defined
and assigned different default values.
They will be particularly useful in conjunction
with the forwarding mechanism described in \secref{sec:forward}.

For example, it may be useful to have a flag |\version|
which can be set to |draft| or |final|.
The document source will contain some conditional code
depending on the value of |\version|.
Suppose further, the flag should default to |final| for the main file
and to |draft| for child files
which is a natural assignment for editing the document.
This is achieved by placing the following code
in the preamble of the main document
(below the |\childdocmain| directive):
%
\begin{center}
\begin{tabular}{l}
|\ifchilddoc|\\
|\providecommand{\version}{draft}|\\
|\||else|\\
|\providecommand{\version}{final}|\\
|\||fi|
\end{tabular}
\end{center}
%
The definition by |\providecommand| makes sure
that previous definitions are not overwritten.
Further statements |\providecommand{\version}{...}|
can thus be added before the above code to override it.

For the main file, one might add a line
(between |\childdocmain| and the above block)
%
\begin{center}
|%\ifchilddoc\||else\providecommand{\version}{draft}\||fi|
\end{center}
%
which can be uncommented to produce a draft version.
Likewise one can add a line to the very top of a child file
(above the |\childdocof{|\textit{main}|}| directive)
%
\begin{center}
|%\providecommand{\version}{final}|
\end{center}
%
which can be uncommented to produce the final version of this child document.

%%%%%%%%%%%%%%%%%%%%%%%%%%%%%%%%%%%%%%%%%%%%%%%%%%%%%%%%%%%%%%%%%%%%%%%%%%%%%%%%
\subsection{Forwarding}
\label{sec:forward}

Different versions of the main or child documents
using compilation flags as described in \secref{sec:flags}
can be (permanently) stored in different files
for convenient compilation, viewing and distribution.
To this end, the package defines a command
to pass on compilation to a different file:

%%%%%%%%%%%%%%%%%%%%%%%%%%%%%%%%%%%%%%%%
\DescribeMacro{\childdocforward}
The command |\childdocforward| redirects processing to
another source file:
%
\begin{center}
\begin{tabular}{l}
|% \iffalse
%
% childdoc.dtx Copyright (C) 2017-2018 Niklas Beisert
%
% This work may be distributed and/or modified under the
% conditions of the LaTeX Project Public License, either version 1.3
% of this license or (at your option) any later version.
% The latest version of this license is in
%   http://www.latex-project.org/lppl.txt
% and version 1.3 or later is part of all distributions of LaTeX
% version 2005/12/01 or later.
%
% This work has the LPPL maintenance status `maintained'.
%
% The Current Maintainer of this work is Niklas Beisert.
%
% This work consists of the files childdoc.dtx and childdoc.ins
% and the derived files childdoc.def and cdocsamp.tex with
% cdocsch1.tex, cdocsch2.tex, cdocsdrf.tex, cdocsfn1.tex, cdocsfn2.tex.
%
%<package>\ifdefined\childdocmain\endinput\fi
%<package>\ProvidesFile{childdoc.def}[2018/12/30 v2.0 child document driver]
%<samplemain>\ProvidesFile{cdocsamp.tex}[2018/12/30 v2.0 sample for childdoc]
%<*driver>
%\ProvidesFile{childdoc.drv}[2018/12/30 v2.0 childdoc reference manual file]
\PassOptionsToClass{10pt,a4paper}{article}
\documentclass{ltxdoc}

\usepackage[margin=35mm]{geometry}
\usepackage{hyperref}
\usepackage{hyperxmp}
\usepackage[usenames]{color}

\hypersetup{colorlinks=true}
\hypersetup{pdfstartview=FitH}
\hypersetup{pdfpagemode=UseNone}
\hypersetup{pdfsource={}}
\hypersetup{pdflang={en-UK}}
\hypersetup{pdfcopyright={Copyright 2017-2018 Niklas Beisert.
  This work may be distributed and/or modified under the
  conditions of the LaTeX Project Public License, either version 1.3
  of this license or (at your option) any later version.}}
\hypersetup{pdflicenseurl={http://www.latex-project.org/lppl.txt}}
\hypersetup{pdfcontactaddress={ETH Zurich, ITP, HIT K,
  Wolfgang-Pauli-Strasse 27}}
\hypersetup{pdfcontactpostcode={8093}}
\hypersetup{pdfcontactcity={Zurich}}
\hypersetup{pdfcontactcountry={Switzerland}}
\hypersetup{pdfcontactemail={nbeisert@itp.phys.ethz.ch}}
\hypersetup{pdfcontacturl={http://people.phys.ethz.ch/\xmptilde nbeisert/}}

\newcommand{\secref}[1]{\hyperref[#1]{section \ref*{#1}}}

\parskip1ex
\parindent0pt
\let\olditemize\itemize
\def\itemize{\olditemize\parskip0pt}

\begin{document}

\title{The \textsf{childdoc} Package}
\hypersetup{pdftitle={The childdoc Package}}
\author{Niklas Beisert\\[2ex]
  Institut f\"ur Theoretische Physik\\
  Eidgen\"ossische Technische Hochschule Z\"urich\\
  Wolfgang-Pauli-Strasse 27, 8093 Z\"urich, Switzerland\\[1ex]
  \href{mailto:nbeisert@itp.phys.ethz.ch}
  {\texttt{nbeisert@itp.phys.ethz.ch}}}
\hypersetup{pdfauthor={Niklas Beisert}}
\hypersetup{pdfsubject={Manual for the LaTeX2e Package childdoc}}
\date{30 December 2018, \textsf{v2.0}}
\maketitle

\begin{abstract}\noindent
\textsf{childdoc} is a \LaTeXe{} package
that enables the direct compilation
of document sections included by |\include|
to individual files.
\end{abstract}

\begingroup
\parskip0ex
\tableofcontents
\endgroup

%%%%%%%%%%%%%%%%%%%%%%%%%%%%%%%%%%%%%%%%%%%%%%%%%%%%%%%%%%%%%%%%%%%%%%%%%%%%%%%%
%%%%%%%%%%%%%%%%%%%%%%%%%%%%%%%%%%%%%%%%%%%%%%%%%%%%%%%%%%%%%%%%%%%%%%%%%%%%%%%%
\section{Introduction}

\LaTeX{} provides a mechanism to structure a large document (such as a book)
into a main file and several child files (containing the chapters)
using the |\include| command.
This mechanism is beneficial for documents
which span hundreds of pages in order to
make the source file(s) more manageable.
Moreover, compilation can be restricted to
selected child files by means of the |\includeonly| command.
The latter feature can be used to reduce the compilation time while editing
(this was significantly more useful in the earlier days of \LaTeX{})
or to generate a smaller document which is easier to navigate.
Another application of |\includeonly| is to generate
documents consisting of selected parts of the complete document.

However, there are a few drawbacks of the plain |\include| mechanism:
\begin{itemize}
\item
The child files cannot be compiled on their own,
they can only be compiled via the main file.
A naive editing environment
(such as a text editor with an option
to have the current file processed by \LaTeX)
may require one to switch to the main file before compiling;
attempting to compile the child file produces errors.
\item
The main file must be modified (each time)
to adjust the |\includeonly| command
to the present needs. This easily leaves the main file in a messy state.
\item
The generated document will always carry the filename
of the main document. This is inconvenient if
several child files are to be compiled and
to be kept for distribution.
\end{itemize}

The present package provides a simple interface
to make child files individually compilable by \LaTeX{}.
Compiling a child file then has the same effect as compiling
the main file with an |\includeonly| command
to select the appropriate child.
Moreover the generated document will carry the name of the child
rather than the main file.
This resolves all three above issues.

This feature is meant to make the editing of books,
thesis documents and lecture notes somewhat more convenient.
However, the package can also be used efficiently for
composing a series of documents (such as exercise sheets)
which are typically distributed individually.
It then assists the author in generating the individual documents
(potentially in different versions)
as well as a document containing the collected series.
Another application is in developing style files
or other kinds of included material
where compilation of the style file could redirect
to a sample or test file.

%%%%%%%%%%%%%%%%%%%%%%%%%%%%%%%%%%%%%%%%%%%%%%%%%%%%%%%%%%%%%%%%%%%%%%%%%%%%%%%%
%%%%%%%%%%%%%%%%%%%%%%%%%%%%%%%%%%%%%%%%%%%%%%%%%%%%%%%%%%%%%%%%%%%%%%%%%%%%%%%%
\section{Usage}

First of all, the package \textsf{childdoc} is \emph{not} a standard
\LaTeXe{} |.sty| style file! Therefore it needs to be invoked in
a non-standard way.

%%%%%%%%%%%%%%%%%%%%%%%%%%%%%%%%%%%%%%%%%%%%%%%%%%%%%%%%%%%%%%%%%%%%%%%%%%%%%%%%
\subsection{Included Files}
\label{sec:include}

%%%%%%%%%%%%%%%%%%%%%%%%%%%%%%%%%%%%%%%%
\DescribeMacro{\childdocmain}
To use the package, add the commands
\begin{center}
\begin{tabular}{l}
|\input{childdoc.def}|\\
|\childdocmain{}|\\
\end{tabular}
\end{center}
at the very top of the main \LaTeX{} file,
in particular \emph{before} the |\documentclass| statement!
The argument of |\childdocmain| should be left empty
(but it must be present).

%%%%%%%%%%%%%%%%%%%%%%%%%%%%%%%%%%%%%%%%
\DescribeMacro{\childdocof}
Furthermore, add the commands
\begin{center}
\begin{tabular}{l}
|\input{childdoc.def}|\\
|\childdocof{|\textit{main}|}|\\
\end{tabular}
\end{center}
at the top of every child file \textit{child}
which is included by |\include{|\textit{child}|}|
from within the main file
(or at least for those files to be compiled individually).
The argument \textit{main} must be the filename of the main file.

There are a couple of
considerations in setting up the main and child documents:

%%%%%%%%%%%%%%%%%%%%%%%%%%%%%%%%%%%%%%%%
\paragraph{Restrictions.}

Please note the following restrictions:
\begin{itemize}
\item
|\childdocmain| must be called with one argument \textit{main}
to ensure compatibility with earlier version of the package.
It must either be empty (|\childdocmain{}|)
or precisely match the filename of the main file in which it is specified.
See \secref{sec:detection} for further information.
\item
The filename \textit{main} must be specified without the |.tex| extension.
\item
The filename \textit{main} is case sensitive
(even in case-insensitive file systems)
due to internal string comparison.
\item
The argument \textit{main} should be fully expanded, it cannot be a macro.
\item
Subdirectories and special characters should be avoided in filenames.
\item
The command |\childdocmain{|\textit{main}|}| must be followed by a whitespace.
It should not be followed immediately by another command
or by a comment mark `|%|'.
This is because the \TeX{} parser reads the token immediately following
the argument of |\childdocmain| and puts it
at the beginning of every child section;
however, a white\-space is ignored.
\end{itemize}

%%%%%%%%%%%%%%%%%%%%%%%%%%%%%%%%%%%%%%%%
\paragraph{Content of Main File.}

It is advisable to place all content in the child files included by |\include|.
Any output contained in the main file will appear in all child documents
unless suppressed manually;
it cannot be suppressed automatically by the |\includeonly| directive
and thus should normally be avoided.
A method to include some content in the main file
by means of conditional processing is described in \secref{sec:conditional}.

%%%%%%%%%%%%%%%%%%%%%%%%%%%%%%%%%%%%%%%%
\paragraph{Page Numbering.}

When only a part of the document is compiled,
the appropriate numbering of pages
(as well as other status parameters)
is determined from the |.aux| files.
The latter contain information from previous passes.
However this information needs to propagate through
all intermediate child documents.
Therefore the page numbering in child documents may well
be inconsistent until the complete document is compiled at least once.

A useful (if unconventional) way to always ensure a consistent
page numbering is to restart the numbering in each child document
and denote the pages by `\textit{child}|.|\textit{page}'
where \textit{child} represents the chapter/section number of the child file.
This can be achieved by the command
|\numberwithin{page}{|\textit{child}|}|
of the \textsf{amsmath} package
where \textit{child} can be |chapter| or |section|
depending on the chosen structuring.
Alternatively, one can modify the macro |\thepage| appropriately
and reset the counter |page| at the start of each child file.

%%%%%%%%%%%%%%%%%%%%%%%%%%%%%%%%%%%%%%%%%%%%%%%%%%%%%%%%%%%%%%%%%%%%%%%%%%%%%%%%
\subsection{Conditional Processing}
\label{sec:conditional}

The package provides a mechanism to compile different versions
of a document. To customise the versions further some conditional processing
can come in handy to distinguish which version is being compiled.
The package provides two macros to describe the compilation context:

%%%%%%%%%%%%%%%%%%%%%%%%%%%%%%%%%%%%%%%%
\DescribeMacro{\ifchilddoc}
The conditional |\ifchilddoc| distinguishes between the compilation of
child documents and the main document:
%
\begin{center}
|\ifchilddoc |\textit{child-code}| |[|\||else |\textit{main-code}]| \||fi|
\end{center}

%%%%%%%%%%%%%%%%%%%%%%%%%%%%%%%%%%%%%%%%
\DescribeMacro{\childdocname}
\DescribeMacro{\childdocjob}
The macro |\childdocname| contains the filename (without extension)
of the main or child file being processed.
Note that |\childdocjob| will always contain the name of the main file.

%%%%%%%%%%%%%%%%%%%%%%%%%%%%%%%%%%%%%%%%
\paragraph{Title Page.}

Conditional processing can be used to include a title or banner page
in the main document when proper precautions are taken.
Importantly, the code in the main file should ensure that the page counter
(as well as other status parameters which are stored in the |.aux| files)
takes the same value after the conditional processing.
Otherwise the page numbers may take divergent values
depending on which part is compiled.

For example, a title page could be declared by:
%
\begin{center}
\begin{tabular}{l}
|\ifchilddoc\||else|\\
|\addtocounter{page}{-1}|\\
\textit{code for title page}\\
|\newpage|\\
|\||fi|
\end{tabular}
\end{center}
%
A banner page for the child documents can be generated by:
%
\begin{center}
\begin{tabular}{l}
|\ifchilddoc|\\
|\addtocounter{page}{-1}|\\
\textit{code for banner page}\\
|\newpage|\\
|\||fi|
\end{tabular}
\end{center}
%
Here one could write a message such as:
\begin{center}
|This is the part \childdocname{} of \childdocjob{}.|
\end{center}

%%%%%%%%%%%%%%%%%%%%%%%%%%%%%%%%%%%%%%%%%%%%%%%%%%%%%%%%%%%%%%%%%%%%%%%%%%%%%%%%
\subsection{Flags}
\label{sec:flags}

The package makes it easy to generate different versions
of the main or child documents.
To this end compilation flags can be defined
and assigned different default values.
They will be particularly useful in conjunction
with the forwarding mechanism described in \secref{sec:forward}.

For example, it may be useful to have a flag |\version|
which can be set to |draft| or |final|.
The document source will contain some conditional code
depending on the value of |\version|.
Suppose further, the flag should default to |final| for the main file
and to |draft| for child files
which is a natural assignment for editing the document.
This is achieved by placing the following code
in the preamble of the main document
(below the |\childdocmain| directive):
%
\begin{center}
\begin{tabular}{l}
|\ifchilddoc|\\
|\providecommand{\version}{draft}|\\
|\||else|\\
|\providecommand{\version}{final}|\\
|\||fi|
\end{tabular}
\end{center}
%
The definition by |\providecommand| makes sure
that previous definitions are not overwritten.
Further statements |\providecommand{\version}{...}|
can thus be added before the above code to override it.

For the main file, one might add a line
(between |\childdocmain| and the above block)
%
\begin{center}
|%\ifchilddoc\||else\providecommand{\version}{draft}\||fi|
\end{center}
%
which can be uncommented to produce a draft version.
Likewise one can add a line to the very top of a child file
(above the |\childdocof{|\textit{main}|}| directive)
%
\begin{center}
|%\providecommand{\version}{final}|
\end{center}
%
which can be uncommented to produce the final version of this child document.

%%%%%%%%%%%%%%%%%%%%%%%%%%%%%%%%%%%%%%%%%%%%%%%%%%%%%%%%%%%%%%%%%%%%%%%%%%%%%%%%
\subsection{Forwarding}
\label{sec:forward}

Different versions of the main or child documents
using compilation flags as described in \secref{sec:flags}
can be (permanently) stored in different files
for convenient compilation, viewing and distribution.
To this end, the package defines a command
to pass on compilation to a different file:

%%%%%%%%%%%%%%%%%%%%%%%%%%%%%%%%%%%%%%%%
\DescribeMacro{\childdocforward}
The command |\childdocforward| redirects processing to
another source file:
%
\begin{center}
\begin{tabular}{l}
|\input{childdoc.def}|\\
|\childdocforward[|\textit{main}|]{|\textit{dest}|}|\\
\end{tabular}
\end{center}
%
The argument \textit{dest} is the destination file
(without extension).
It should be the main file or one of the child files.
Note that further \textsf{childdoc} directives
such as |\childdocof| and |\childdocforward|
in the indicated file will be processed in this form.
The optional argument \textit{main}
passes on directly to the main file \textit{main}
while pretending to compile the child \textit{dest}.
This form behaves as if \textit{dest}
issues |\childdocof{|\textit{main}|}| right away,
and no further \textsf{childdoc} directives will be processed.

%%%%%%%%%%%%%%%%%%%%%%%%%%%%%%%%%%%%%%%%
\DescribeMacro{\...prefix}
In the alternative form |\childdocforwardprefix|,
%
\begin{center}
\begin{tabular}{l}
|\input{childdoc.def}|\\
|\childdocforwardprefix[|\textit{main}|]{|\textit{prefix}|}{|\textit{dest}|}|
\end{tabular}
\end{center}
%
the destination file is determined by a pattern
depending on the current file:
To make this work, the current file must be called
`{\textit{prefix}\hspace{0.2em}\textit{suffix}}'
with \textit{prefix} matching precisely the argument.
Processing is then passed on to the file
`{\textit{dest}\hspace{0.2em}\textit{suffix}}'.
Surely, the same effect is achieved by
directly specifying the
argument `{\textit{dest}\hspace{0.2em}\textit{suffix}}'
in the first form.
However, that requires to set up a different file
for each child. With the alternative form of the command
all these files can have exactly the same content
which simplifies setting them up and maintaining them.

For example, the following file |draft.tex|
with a compilation flag |\version| as described in \secref{sec:flags}
compiles the main document as a draft:
%
\begin{center}
\begin{tabular}{l}
|\def\version{draft}|\\
|\input{childdoc.def}|\\
|\childdocforward{|\textit{main}|}|
\end{tabular}
\end{center}
%
Likewise, the following files |final|\textit{nn}|.tex|
compile the final version of the child document
|child|\textit{nn}|.tex|:
%
\begin{center}
\begin{tabular}{l}
|\def\version{final}|\\
|\input{childdoc.def}|\\
|\childdocforwardprefix{final}{child}|
\end{tabular}
\end{center}
%

Note that when several versions of a main file and/or of each child file
are to be generated, it may be convenient to set up a |Makefile| or
shell script to automatise the process.

%%%%%%%%%%%%%%%%%%%%%%%%%%%%%%%%%%%%%%%%%%%%%%%%%%%%%%%%%%%%%%%%%%%%%%%%%%%%%%%%
\subsection{Command Line Processing}
\label{sec:commandline}

The effect of redirection files can also be achieved by invoking
the \LaTeX{} compiler with a more elaborate command line.
Most conveniently this should be done as part
of a shell script or a |Makefile|.

When using \textsf{childdoc} in the main file, the following
command lines effectively perform a redirection
(note that depending on the shell being used,
backslashes may have to be doubled: `|\|' $\to$ `|\\|'):
%
\begin{center}
|... -jobname "|\textit{target}|" |\\|"|[\textit{flags}]%
|\input{childdoc.def}\childdocforward[|\textit{main}|]{|\textit{dest}|}"|
\end{center}
%
Here \textit{target} is the name of the output file,
\textit{main} is the name of the main file
and \textit{dest} is the name of the main or child file to be processed
(all filenames without extensions).
The optional argument \textit{main} can be omitted
if \textit{main} matches \textit{dest}.
Optionally, compilation \textit{flags} can be defined via |\def| commands.
This command line makes the \TeX{} engine believe
it is compiling the file \textit{target}
whose content is specified as the latter parameter.
The provided code then forwards the processing to
\textit{main} or \textit{dest} as described in \secref{sec:forward}.

%%%%%%%%%%%%%%%%%%%%%%%%%%%%%%%%%%%%%%%%%%%%%%%%%%%%%%%%%%%%%%%%%%%%%%%%%%%%%%%%
\subsection{Include by Input}
\label{sec:input}

Including child documents by |\include| has some restrictions by design.
Most notably, the content of a child document always occupies
its own set of pages; pages cannot be shared between child documents.
Usually, this behaviour makes perfect sense
because each child document contain an essential part of the document.
However, in some situations it may be desirable to compose
a document from a collection of parts
without having mandatory page breaks between then.
For this case, the package
provides a mechanism to include parts
by |\input| which can also be processed individually.
However, by construction this mechanism
requires manual handling of the content to be output.

%%%%%%%%%%%%%%%%%%%%%%%%%%%%%%%%%%%%%%%%
\DescribeMacro{\ifchilddocmanual}
The main file should be prepared as usual, see \secref{sec:include}.
However, the document body must make a distinction
between processing of an individual part and of the main document, e.g.:
%
\begin{center}
\begin{tabular}{l}
|\ifchilddocmanual|\\
|\input{\childdocname}|\\
|\||else|\\
\textit{document body with }|\input{|\textit{part}|}|\\
|\||fi|
\end{tabular}
\end{center}
%
The conditional |\ifchilddocmanual| is true whenever
a part to be included by |\input| is being compiled,
and the name of the part is stored in |\childdocname|.

%%%%%%%%%%%%%%%%%%%%%%%%%%%%%%%%%%%%%%%%
\DescribeMacro{\childdocby}
Each part to be included by |\input| should start with:
%
\begin{center}
\begin{tabular}{l}
|\input{childdoc.def}|\\
|\childdocby{|\textit{main}|}|\\
\end{tabular}
\end{center}
%
The directive |\childdocby| is similar to |\childdocof|
described in \secref{sec:include},
but the subsequent selection of content must be done manually.
To that end, both |\ifchilddoc| and |\ifchilddocmanual|
will be true upon processing of a part,
and the name of the part is stored in |\childdocname|.
Note that |\jobname| will be set to the filename of the current part
so that each part receives an individual |.aux| file
that does not interfere with the |.aux| file(s) of the main document.
This behaviour can be altered by the alternative form
|\childdocby[*]{|\textit{main}|}| (with a non-empty optional argument)
which uses the |.aux| file of the main document
by setting |\jobname| to \textit{main}.

%%%%%%%%%%%%%%%%%%%%%%%%%%%%%%%%%%%%%%%%%%%%%%%%%%%%%%%%%%%%%%%%%%%%%%%%%%%%%%%%
\subsection{Driver Development}
\label{sec:driver}

The \textsf{childdoc} mechanism can also be use for the development
of definition files such as \LaTeX{} styles or classes.
This case differs from the above setup with multiple parts
included by |\include| in that no |\includeonly| should be invoked.
This can be achieved by starting the include file
(before |\ProvidesPackage|) with:
%
\begin{center}
\begin{tabular}{l}
|\input{childdoc.def}|\\
|\childdocforward{|\textit{main}|}|\\
\end{tabular}
\end{center}
%
or alternatively with:
%
\begin{center}
\begin{tabular}{l}
|\input{childdoc.def}|\\
|\childdocby{|\textit{main}|}|\\
\end{tabular}
\end{center}
%
Both forms have slightly different effects as described above.
The main file is prepared as usual, see \secref{sec:include}.

%%%%%%%%%%%%%%%%%%%%%%%%%%%%%%%%%%%%%%%%%%%%%%%%%%%%%%%%%%%%%%%%%%%%%%%%%%%%%%%%
\subsection{Legacy Detection}
\label{sec:detection}

The directive |\childdocmain| in the main file can detect
whether the complete document or merely a child is to be compiled
even without using the directive |\childdocof|.
This method is deprecated because it is less robust
and there is no compelling reason to use it;
it is merely provided for backward compatibility
and it may be removed in future versions.

If the detection mechanism is to be used,
it is mandatory to correctly specify
the filename of the main file as the argument of |\childdocmain|:
%
\begin{center}
\begin{tabular}{l}
|\input{childdoc.def}|\\
|\childdocmain{|\textit{main}|}|\\
\end{tabular}
\end{center}
%
If |\jobname| does not match the argument \textit{main} of |\childdocmain|,
it is assumed that |\jobname| points to the child file to be compiled.
When using |\childdocmain| with the main file specified as argument,
it suffices to start a child file
with just |\input{|\textit{main}|}|
without loading of the package and using |\childdocof|.
If instead all processing is done
with the appropriate \textsf{childdoc} directives,
the argument of \textit{main} of |\childdocmain| can be empty.

An alternative version of the command line processing described
in \secref{sec:commandline} using the detection mechanism reads:
%
\begin{center}
|... -jobname "|\textit{target}|" "|[\textit{flags}]%
[|\def\jobname{|\textit{dest}|}|]|\input{|\textit{main}|}"|
\end{center}

%%%%%%%%%%%%%%%%%%%%%%%%%%%%%%%%%%%%%%%%%%%%%%%%%%%%%%%%%%%%%%%%%%%%%%%%%%%%%%%%
\subsection{Manual Code}
\label{sec:manual}

In case one cannot be certain whether the definitions file |childdoc.def|
is installed on the target \TeX{} distribution
and one prefers not to ship it,
it is conceivable to paste a few relevant commands into the sources.

To that end, drop all statements |\input{childdoc.def}|
and perform the replacements as outlined below.
Instead of |\childdocmain{|\textit{main}|}| add the following code
to the top of the main file:
%
\begin{center}
\begin{tabular}{l}
|\||ifdefined\childdocname\endinput\||fi\newif\ifchilddoc|\\
|\edef\childdocname{\scantokens\expandafter{\jobname\noexpand}}|\\
|\def\childdocmain{|\textit{main}|}\||ifx\childdocmain\childdocname\||else|\\
|\childdoctrue\includeonly{\childdocname}\let\jobname\childdocmain\||fi|\\
\end{tabular}
\end{center}
%
Instead of |\childdocof{|\textit{main}|}| just include the main file
at the top of each child file:
%
\begin{center}
|\input{|\textit{main}|}|
\end{center}
%
A simple redirection |\childdocforward{|\textit{dest}|}| is achieved by:
%
\begin{center}
|\def\jobname{|\textit{dest}|}\input{\jobname}|
\end{center}
%
The redirection with prefix
|\childdocforwardprefix[|\textit{prefix}|]{|\textit{dest}|}|
is accomplished by:
%
\begin{center}
\begin{tabular}{l}
|{\edef\jobname{\scantokens\expandafter{\jobname\noexpand}}|\\
|\def\redirectjob |\textit{prefix}|#1~~~{\gdef\jobname{|\textit{dest}|#1}}|\\
|\expandafter\redirectjob\jobname~~~}\input{\jobname}|
\end{tabular}
\end{center}

In an alternative approach,
child documents can be compiled by a specific command line
without additional code or specific definitions:
%
\begin{center}
|... -jobname "|\textit{target}|" "|[\textit{flags}]%
|\includeonly{|\textit{dest}|}\input{|\textit{main}|}"|
\end{center}
%

%%%%%%%%%%%%%%%%%%%%%%%%%%%%%%%%%%%%%%%%%%%%%%%%%%%%%%%%%%%%%%%%%%%%%%%%%%%%%%%%
%%%%%%%%%%%%%%%%%%%%%%%%%%%%%%%%%%%%%%%%%%%%%%%%%%%%%%%%%%%%%%%%%%%%%%%%%%%%%%%%
\section{Information}

%%%%%%%%%%%%%%%%%%%%%%%%%%%%%%%%%%%%%%%%%%%%%%%%%%%%%%%%%%%%%%%%%%%%%%%%%%%%%%%%
\subsection{Copyright}

Copyright \copyright{} 2017--2018 Niklas Beisert

This work may be distributed and/or modified under the
conditions of the \LaTeX{} Project Public License, either version 1.3
of this license or (at your option) any later version.
The latest version of this license is in
  \url{http://www.latex-project.org/lppl.txt}
and version 1.3 or later is part of all distributions of \LaTeX{}
version 2005/12/01 or later.

This work has the LPPL maintenance status `maintained'.

The Current Maintainer of this work is Niklas Beisert.

This work consists of the files |README.txt|, |childdoc.ins| and |childdoc.dtx|
as well as the derived files |childdoc.def|, |cdocsamp.tex|
with |cdocsch1.tex|, |cdocsch2.tex|, |cdocspt3.tex|, |cdocspt4.tex|,
|cdocsdrf.tex|, |cdocsfn1.tex|, |cdocsfn2.tex|
as well as |childdoc.pdf|.

%%%%%%%%%%%%%%%%%%%%%%%%%%%%%%%%%%%%%%%%%%%%%%%%%%%%%%%%%%%%%%%%%%%%%%%%%%%%%%%%
\subsection{Files and Installation}

The package consists of the files:
%
\begin{center}
\begin{tabular}{ll}
    |README.txt|   & readme file \\
    |childdoc.ins| & installation file \\
    |childdoc.dtx| & source file \\
    |childdoc.def| & definition file \\
    |cdocsamp.tex| & sample main file \\
    |cdocsch1.tex| & sample include file \\
    |cdocsch2.tex| & sample include file \\
    |cdocspt3.tex| & sample part file \\
    |cdocspt4.tex| & sample part file \\
    |cdocsdrf.tex| & sample redirection file \\
    |cdocsfn1.tex| & sample redirection file \\
    |cdocsfn2.tex| & sample redirection file \\
    |childdoc.pdf| & manual
\end{tabular}
\end{center}
%
The distribution consists of the files
|README.txt|, |childdoc.ins| and |childdoc.dtx|.
%
\begin{itemize}
\item
Run (pdf)\LaTeX{} on |childdoc.dtx|
to compile the manual |childdoc.pdf| (this file).
\item
Run \LaTeX{} on |childdoc.ins| to create the definitions file |childdoc.def|
and the sample |cdocsamp.tex| with include files
|cdocsch1.tex|, |cdocsch2.tex|, |cdocspt3.tex|, |cdocspt4.tex|,
|cdocsdrf.tex|, |cdocsfn1.tex|, |cdocsfn2.tex|.
Then copy the file |childdoc.def| to an appropriate directory of your \LaTeX{}
distribution, e.g.\ \textit{texmf-root}|/tex/latex/childdoc|.
\end{itemize}

%%%%%%%%%%%%%%%%%%%%%%%%%%%%%%%%%%%%%%%%%%%%%%%%%%%%%%%%%%%%%%%%%%%%%%%%%%%%%%%%
\subsection{Related CTAN Packages}

There are several other packages which offer a similar functionality:
%
\begin{itemize}
\item
The packages
\href{http://ctan.org/pkg/docmute}{\textsf{docmute}},
\href{http://ctan.org/pkg/includex}{\textsf{includex}} and
\href{http://ctan.org/pkg/standalone}{\textsf{standalone}}
provide commands to include only the document body of
a child file thus allowing both files to be compiled individually.
\item
The packages \href{http://ctan.org/pkg/subdocs}{\textsf{subdocs}}
and \href{http://ctan.org/pkg/subfiles}{\textsf{subfiles}}
provide structures in which the main and child documents can be
encapsulated and allowing them to be compiled individually.
The inclusion mechanism is different from the conventional |\include|.
\item
The package \href{http://ctan.org/pkg/combine}{\textsf{combine}}
is an elaborate solution to combine several documents into one.
\end{itemize}
%
See also the CTAN topic \href{http://ctan.org/topic/subdocs}{\textsf{subdocs}}
for further related packages.
The present package differs from the above solutions in that
a document structure constructed with the conventional |\include| mechanism
just needs two extra commands at the top of every file
such that all constituent files can be compiled individually.

%%%%%%%%%%%%%%%%%%%%%%%%%%%%%%%%%%%%%%%%%%%%%%%%%%%%%%%%%%%%%%%%%%%%%%%%%%%%%%%%
%\subsection{Feature Suggestions}
%
%The following is a list of features which may be useful for future
%versions of this package:
%%
%\begin{itemize}
%\item
%\ldots
%\end{itemize}

%%%%%%%%%%%%%%%%%%%%%%%%%%%%%%%%%%%%%%%%%%%%%%%%%%%%%%%%%%%%%%%%%%%%%%%%%%%%%%%%
\subsection{Revision History}

%%%%%%%%%%%%%%%%%%%%%%%%%%%%%%%%%%%%%%%%
\paragraph{v2.0:} 2018/12/30

\begin{itemize}
\item
immediate forward processing
\item
added |\childdocby| mechanism
\item
manual restructured
\end{itemize}

%%%%%%%%%%%%%%%%%%%%%%%%%%%%%%%%%%%%%%%%
\paragraph{v1.6:} 2018/01/17

\begin{itemize}
\item
application for development of include files
\item
corrections to manual
\end{itemize}

%%%%%%%%%%%%%%%%%%%%%%%%%%%%%%%%%%%%%%%%
\paragraph{v1.5:} 2017/05/21

\begin{itemize}
\item
more complete structuring introduced
\item
|\childdocof| introduced
\item
|\childdoc| renamed to |\childdocmain|
\item
|\childredirect| renamed to |\childdocforward| and |\childdocforwardprefix|
and functionality expanded
\end{itemize}

%%%%%%%%%%%%%%%%%%%%%%%%%%%%%%%%%%%%%%%%
\paragraph{v1.0:} 2017/04/27

\begin{itemize}
\item
manual and install package
\item
first version published on CTAN
\end{itemize}

%%%%%%%%%%%%%%%%%%%%%%%%%%%%%%%%%%%%%%%%
\paragraph{v0.6:} 2017/04/26

\begin{itemize}
\item
redirection mechanism added
\end{itemize}

%%%%%%%%%%%%%%%%%%%%%%%%%%%%%%%%%%%%%%%%
\paragraph{v0.5:} 2017/04/26

\begin{itemize}
\item
functionality in definition file
\end{itemize}


%%%%%%%%%%%%%%%%%%%%%%%%%%%%%%%%%%%%%%%%%%%%%%%%%%%%%%%%%%%%%%%%%%%%%%%%%%%%%%%%
%%%%%%%%%%%%%%%%%%%%%%%%%%%%%%%%%%%%%%%%%%%%%%%%%%%%%%%%%%%%%%%%%%%%%%%%%%%%%%%%
%%%%%%%%%%%%%%%%%%%%%%%%%%%%%%%%%%%%%%%%%%%%%%%%%%%%%%%%%%%%%%%%%%%%%%%%%%%%%%%%
\appendix

\settowidth\MacroIndent{\rmfamily\scriptsize 000\ }

 \DocInput{childdoc.dtx}

\end{document}
%</driver>
% \fi
%
% %%%%%%%%%%%%%%%%%%%%%%%%%%%%%%%%%%%%%%%%%%%%%%%%%%%%%%%%%%%%%%%%%%%%%%%%%%%%%%
% %%%%%%%%%%%%%%%%%%%%%%%%%%%%%%%%%%%%%%%%%%%%%%%%%%%%%%%%%%%%%%%%%%%%%%%%%%%%%%
% \section{Sample}
%\iffalse
%<*samplemain>
%\fi
%
% The following presents a sample document
% with two chapters, two parts, a title page,
% a compile flag as well as three forwarding files to set the flag.
% It consists of eight |.tex| files:
% \begin{center}
% \begin{tabular}{ll}
% |cdocsamp.tex|&main file\\
% |cdocsch1.tex|&include file for chapter 1\\
% |cdocsch2.tex|&include file for chapter 2\\
% |cdocspt3.tex|&include file for part 3\\
% |cdocspt4.tex|&include file for part 4\\
% |cdocsdrf.tex|&forwarding file for main file in draft mode\\
% |cdocsfi1.tex|&forwarding file for final version of chapter 1\\
% |cdocsfi2.tex|&forwarding file for final version of chapter 2\\
% \end{tabular}
% \end{center}
% Each of the eight files can be compiled directly by the \LaTeX{} compiler.
%
% %%%%%%%%%%%%%%%%%%%%%%%%%%%%%%%%%%%%%%
% \paragraph{Main File.}
%
% The main file is called |cdocsamp.tex|.
%
% Load the \textsf{childdoc} definitions and
% declare the filename for the main document:
%    \begin{macrocode}
\input{childdoc.def}
\childdocmain{}
%    \end{macrocode}

% Optional override for |\version| flag:
%    \begin{macrocode}
%%\ifchilddoc\else\providecommand{\version}{draft}\fi
%    \end{macrocode}

% Define the default values for the |\version| flag
% (|final| for the main file and |draft| for childs):
%    \begin{macrocode}
\ifchilddoc
\providecommand{\version}{draft}
\else
\providecommand{\version}{final}
\fi
%    \end{macrocode}

% Load the standard document class:
%    \begin{macrocode}
\documentclass[12pt]{article}
%    \end{macrocode}

% Start the document body:
%    \begin{macrocode}
\begin{document}
%    \end{macrocode}

% Declare a title page.
% Print title, part of document being processed and version flag:
%    \begin{macrocode}
\addtocounter{page}{-1}
\begin{center}
{\LARGE\bfseries{}childdoc example\par}
\vspace{1cm}
\ifchilddoc
\ifchilddocmanual part\else chapter\fi:
`\childdocname' of `\childdocjob'\par
\else
main document: `\childdocjob'\par
\fi
version: \version\par
\end{center}
\newpage
%    \end{macrocode}

% Manually include selected file,
% otherwise process as usual:
%    \begin{macrocode}
\ifchilddocmanual
\section*{part `\childdocname'}
\input{\childdocname}
\else
%    \end{macrocode}

% Include the two chapters:
%    \begin{macrocode}
\include{cdocsch1}
\include{cdocsch2}
%    \end{macrocode}

% Include the two parts unless only chapters should be displayed:
%    \begin{macrocode}
\ifchilddoc\else
\section{part three}
\input{cdocspt3}
\section{part four}
\input{cdocspt4}
\fi
%    \end{macrocode}

% Process as usual until here:
%    \begin{macrocode}
\fi
%    \end{macrocode}

% End of document body:
%    \begin{macrocode}
\end{document}
%    \end{macrocode}
%\iffalse
%</samplemain>
%\fi
%
% %%%%%%%%%%%%%%%%%%%%%%%%%%%%%%%%%%%%%%
% \paragraph{Chapter Include Files.}
%
% The include files are called |cdocsch1.tex| and |cdocsch2.tex|.
%
%\iffalse
%<*samplechap1|samplechap2>
%\fi

% Optional override for |\version| flag:
%    \begin{macrocode}
%%\providecommand{\version}{final}
%    \end{macrocode}

% Include the main document:
%    \begin{macrocode}
\input{childdoc.def}
\childdocof{cdocsamp}
%    \end{macrocode}

%\iffalse
%</samplechap1|samplechap2>
%\fi
%
%\iffalse
%<*samplechap1>
%\fi
% Some text for chapter 1:
%    \begin{macrocode}
\section{one}
some text in chapter one
%    \end{macrocode}

%\iffalse
%</samplechap1>
%\fi
% Some text for chapter 2:
%\iffalse
%<*samplechap2>
%\fi
%    \begin{macrocode}
\section{two}
more text in chapter two
%    \end{macrocode}

%\iffalse
%</samplechap2>
%\fi
%
% %%%%%%%%%%%%%%%%%%%%%%%%%%%%%%%%%%%%%%
% \paragraph{Part Include Files.}
%
% The include files are called |cdocspt3.tex| and |cdocspt4.tex|.
%
%\iffalse
%<*samplepart3|samplepart4>
%\fi

% Optional override for |\version| flag:
%    \begin{macrocode}
%%\providecommand{\version}{final}
%    \end{macrocode}

% Include the main document:
%    \begin{macrocode}
\input{childdoc.def}
\childdocby{cdocsamp}
%    \end{macrocode}

%\iffalse
%</samplepart3|samplepart4>
%\fi
%
%\iffalse
%<*samplepart3>
%\fi
% Some text for part 3:
%    \begin{macrocode}
some text in part three
%    \end{macrocode}

%\iffalse
%</samplepart3>
%\fi
% Some text for part 4:
%\iffalse
%<*samplepart4>
%\fi
%    \begin{macrocode}
more text in part four
%    \end{macrocode}

%\iffalse
%</samplepart4>
%\fi
%
% %%%%%%%%%%%%%%%%%%%%%%%%%%%%%%%%%%%%%%
% \paragraph{Forwarding for a Complete Draft.}
%
% The following forwarding file |cdocsdrf.tex|
% compiles the main document in draft mode:
%\iffalse
%<*sampledraft>
%\fi
%    \begin{macrocode}
\def\version{draft}
\input{childdoc.def}
\childdocforward{cdocsamp}
%    \end{macrocode}

%\iffalse
%</sampledraft>
%\fi
%
% %%%%%%%%%%%%%%%%%%%%%%%%%%%%%%%%%%%%%%
% \paragraph{Forwarding for Final Version of the Chapters.}
%
% The following forwarding files |cdocsfn1.tex| and |cdocsfn2.tex|
% (with identical content)
% compile the final versions of the child documents
% |cdocsch1.tex| and |cdocsch2.tex|, respectively:
%\iffalse
%<*samplefinal>
%\fi
%    \begin{macrocode}
\def\version{final}
\input{childdoc.def}
\childdocforwardprefix[cdocsamp]{cdocsfn}{cdocsch}
%    \end{macrocode}

%\iffalse
%</samplefinal>
%\fi
%
% %%%%%%%%%%%%%%%%%%%%%%%%%%%%%%%%%%%%%%
% \paragraph{Command Line Processing.}
%
% The following three command lines generate the output files
% |cdocscld|, |cdocscl1| and |cdocscl2|
% which should be identical to
% |cdocsdrf|, |cdocsch1| and |cdocsfn2|, respectively:
% \begin{center}
% \begin{tabular}{l}
% |latex -jobname cdocscld \|\\
% |  "\def\version{draft}\input{childdoc.def}\childdocforward{cdocsamp}"|\\
% |latex -jobname cdocscl1 \|\\
% |  "\input{childdoc.def}\childdocforward[cdocsamp]{cdocsch1}"|\\
% |latex -jobname cdocscl2 \|\\
% |  "\def\version{final}\input{childdoc.def}\childdocforward{cdocsch2}"|
% \end{tabular}
% \end{center}
% Note that the trailing backslash on each first line
% merely continues the input to the second line
% (for convenient cut ant paste).
% Furthermore, the command |latex| can be replaced by any
% of its alternative versions such as |pdflatex|.
%
% %%%%%%%%%%%%%%%%%%%%%%%%%%%%%%%%%%%%%%%%%%%%%%%%%%%%%%%%%%%%%%%%%%%%%%%%%%%%%%
% %%%%%%%%%%%%%%%%%%%%%%%%%%%%%%%%%%%%%%%%%%%%%%%%%%%%%%%%%%%%%%%%%%%%%%%%%%%%%%
% \section{Implementation}
%\iffalse
%<*package>
%\fi
%
% This section describes the definitions file |childdoc.def|.

% The definitions cannot be loaded using |\usepackage| or |\RequirePackage|
% which has a mechanism to prevent loading a style file more than once.
% When loading the definitions by means of |\input|
% multiple instances have to be prevented manually:
%\iffalse
%This code needs to be before the `\ProvidesFile' directive
%which is defined at the beginning of this file.
%Therefore it is also placed there and commented out here.
%</package>
%<*discard>
%\fi
%    \begin{macrocode}
\ifdefined\childdocmain\endinput\fi
%    \end{macrocode}
%\iffalse
%</discard>
%<*package>
%\fi
%
% \macro{\ifchilddoc}
% \macro{\ifchilddocmanual}
% The conditional |\ifchilddoc| tells whether a
% child (true) or main (false) document is being compiled.
% The conditional |\ifchilddocmanual| tells whether
% the |\includeonly| mechanism is used (false) or
% the selection of child files must be performed manually (true).
% The definitions initialise to false:
%    \begin{macrocode}
\newif\ifchilddoc
\newif\ifchilddocmanual
%    \end{macrocode}

% \macro{\childdocname}
% \macro{\childdocjob}
% The macro |\childdocname| stores the name of the main document
% to be compiled. The macro |\childdocjob| stores the name of
% the document on which the \LaTeX{} compiler was originally invoked.
% The content of |\jobname| cannot be compared
% to filenames specified in the source due to different catcodes.
% The following code rescans |\jobname|, stores the result
% in |\childdocname| and saves a copy in |\childdocjob|:
%    \begin{macrocode}
\edef\childdocname{\scantokens\expandafter{\jobname\noexpand}}
\let\childdocjob\childdocname
%    \end{macrocode}

% \macro{\childdocdisable}
% The macro |\childdocdisable| prevents the main file
% from being processed more than once.
% At this stage, the main document command |\childdocmain|
% is assumed to be called once again where it should do nothing.
% Any subsequent call to it should prevent
% a secondary processing of the main document
% It overwrites the forwarding commands
% |\childdocof| and |\childdocforward|
% with empty macros to prevent further inclusions of the main document:
%    \begin{macrocode}
\newcommand{\childdocdisable}
{
  \renewcommand{\childdocmain}[1]{\renewcommand{\childdocmain}[1]{\endinput}}
  \renewcommand{\childdocof}[1]{}
  \renewcommand{\childdocby}[2][]{}
  \renewcommand{\childdocforward}[2][]{}
  \renewcommand{\childdocdisable}{}
}
%    \end{macrocode}

% \macro{\childdocmain}
% The macro |\childdocmain| is to be called at the top of the main file
% with nothing or the main filename (without extension) as argument.
% First, it breaks loops.
% If the argument is not empty and does not match |\childdocname|
% (which is set by the first inclusion of |childdoc.def|),
% |\ifchilddoc| is set to true, |\includeonly| is applied to the child file
% and |\jobname| is set to the main file
% (for proper handling of |.aux| files):
%    \begin{macrocode}
\newcommand{\childdocmain}[1]
{
  \childdocdisable\childdocmain{}
  \if?#1?\else
    \begingroup
      \def\childdoctmp{#1}
      \ifx\childdoctmp\childdocname
        \def\childdoctmp{}
      \else
        \def\childdoctmp
        {
          \childdoctrue
          \includeonly{\childdocname}
          \def\childdocjob{#1}
          \def\jobname{#1}
        }
      \fi
      \expandafter
    \endgroup
    \childdoctmp
  \fi
}
%    \end{macrocode}

% \macro{\childdocof}
% The command |\childdocof| redirects
% compilation to the main file |#1|.
%    \begin{macrocode}
\newcommand{\childdocof}[1]
{
  \childdocdisable
  \childdoctrue
  \includeonly{\childdocname}
  \def\jobname{#1}
  \def\childdocjob{#1}
  \input{#1}
}
%    \end{macrocode}

% \macro{\childdocby}
% The command |\childdocby| ....
%    \begin{macrocode}
\newcommand{\childdocby}[2][]
{
  \childdocdisable
  \childdoctrue
  \childdocmanualtrue
  \if?#1?\else
    \def\jobname{#2}
  \fi
  \def\childdocjob{#2}
  \input{#2}
  \endinput
}
%    \end{macrocode}

% \macro{\childdocforward}
% The command |\childdocforward| redirects
% compilation to the main file or
% (if the optional argument is given) a child file.
% Parameters are set as if the main file
% or a child file starting with |\childdocof| was compiled.
% Then compilation is handed over to the main file:
%    \begin{macrocode}
\newcommand{\childdocforward}[2][]
{
  \begingroup
    \if?#1?
      \def\childdoctmp
      {
        \def\childdocname{#2}
        \def\childdocjob{#2}
        \def\jobname{#2}
        \input{#2}
        \endinput
      }
    \else
      \def\childdoctmp
      {
        \childdocdisable
        \def\childdocname{#2}
        \childdoctrue
        \includeonly{#2}
        \def\childdocjob{#1}
        \def\jobname{#1}
        \input{#1}
        \endinput
      }
    \fi
    \expandafter
  \endgroup
  \childdoctmp
}
%    \end{macrocode}

% \macro{\childdocforwardprefix}
% The command |\childdocforwardprefix| redirects
% compilation to the main or a child file by means of a pattern.
% The prefix |#1| in the current filename is replaced by |#2|
% and the suffix of the current filename is kept
% (it is assumed that the filename does not contain the substring `|~~~|'
% which is used as a delimiter).
% Compilation is handed over to the new file by |\childdocforward|:
%    \begin{macrocode}
\newcommand{\childdocforwardprefix}[3][]
{
  \begingroup
    \def\childdocextract #2##1~~~{\def\childdoctmp{\childdocforward[#1]{#3##1}}}
    \expandafter\childdocextract\childdocname~~~
    \expandafter
  \endgroup
  \childdoctmp
}
%    \end{macrocode}

% \macro{\childdoc}
% The deprecated macro |\childdoc| is a legacy version of |\childdocmain|:
%    \begin{macrocode}
\newcommand{\childdoc}{\childdocmain}
%    \end{macrocode}

% \macro{\childdocredirect}
% The deprecated macro |\childdocredirect| is a legacy version
% of |\childdocforward| and |\childdocforwardprefix|:
%    \begin{macrocode}
\newcommand{\childdocredirect}[2][]
{
  \begingroup
    \if?#1?
      \def\childdoctmp{\childdocforward{#2}}
    \else
      \def\childdoctmp{\childdocforwardprefix{#1}{#2}}
    \fi
    \expandafter
  \endgroup
  \childdoctmp
}
%    \end{macrocode}

%\iffalse
%</package>
%\fi
%
\endinput
|\\
|\childdocforward[|\textit{main}|]{|\textit{dest}|}|\\
\end{tabular}
\end{center}
%
The argument \textit{dest} is the destination file
(without extension).
It should be the main file or one of the child files.
Note that further \textsf{childdoc} directives
such as |\childdocof| and |\childdocforward|
in the indicated file will be processed in this form.
The optional argument \textit{main}
passes on directly to the main file \textit{main}
while pretending to compile the child \textit{dest}.
This form behaves as if \textit{dest}
issues |\childdocof{|\textit{main}|}| right away,
and no further \textsf{childdoc} directives will be processed.

%%%%%%%%%%%%%%%%%%%%%%%%%%%%%%%%%%%%%%%%
\DescribeMacro{\...prefix}
In the alternative form |\childdocforwardprefix|,
%
\begin{center}
\begin{tabular}{l}
|% \iffalse
%
% childdoc.dtx Copyright (C) 2017-2018 Niklas Beisert
%
% This work may be distributed and/or modified under the
% conditions of the LaTeX Project Public License, either version 1.3
% of this license or (at your option) any later version.
% The latest version of this license is in
%   http://www.latex-project.org/lppl.txt
% and version 1.3 or later is part of all distributions of LaTeX
% version 2005/12/01 or later.
%
% This work has the LPPL maintenance status `maintained'.
%
% The Current Maintainer of this work is Niklas Beisert.
%
% This work consists of the files childdoc.dtx and childdoc.ins
% and the derived files childdoc.def and cdocsamp.tex with
% cdocsch1.tex, cdocsch2.tex, cdocsdrf.tex, cdocsfn1.tex, cdocsfn2.tex.
%
%<package>\ifdefined\childdocmain\endinput\fi
%<package>\ProvidesFile{childdoc.def}[2018/12/30 v2.0 child document driver]
%<samplemain>\ProvidesFile{cdocsamp.tex}[2018/12/30 v2.0 sample for childdoc]
%<*driver>
%\ProvidesFile{childdoc.drv}[2018/12/30 v2.0 childdoc reference manual file]
\PassOptionsToClass{10pt,a4paper}{article}
\documentclass{ltxdoc}

\usepackage[margin=35mm]{geometry}
\usepackage{hyperref}
\usepackage{hyperxmp}
\usepackage[usenames]{color}

\hypersetup{colorlinks=true}
\hypersetup{pdfstartview=FitH}
\hypersetup{pdfpagemode=UseNone}
\hypersetup{pdfsource={}}
\hypersetup{pdflang={en-UK}}
\hypersetup{pdfcopyright={Copyright 2017-2018 Niklas Beisert.
  This work may be distributed and/or modified under the
  conditions of the LaTeX Project Public License, either version 1.3
  of this license or (at your option) any later version.}}
\hypersetup{pdflicenseurl={http://www.latex-project.org/lppl.txt}}
\hypersetup{pdfcontactaddress={ETH Zurich, ITP, HIT K,
  Wolfgang-Pauli-Strasse 27}}
\hypersetup{pdfcontactpostcode={8093}}
\hypersetup{pdfcontactcity={Zurich}}
\hypersetup{pdfcontactcountry={Switzerland}}
\hypersetup{pdfcontactemail={nbeisert@itp.phys.ethz.ch}}
\hypersetup{pdfcontacturl={http://people.phys.ethz.ch/\xmptilde nbeisert/}}

\newcommand{\secref}[1]{\hyperref[#1]{section \ref*{#1}}}

\parskip1ex
\parindent0pt
\let\olditemize\itemize
\def\itemize{\olditemize\parskip0pt}

\begin{document}

\title{The \textsf{childdoc} Package}
\hypersetup{pdftitle={The childdoc Package}}
\author{Niklas Beisert\\[2ex]
  Institut f\"ur Theoretische Physik\\
  Eidgen\"ossische Technische Hochschule Z\"urich\\
  Wolfgang-Pauli-Strasse 27, 8093 Z\"urich, Switzerland\\[1ex]
  \href{mailto:nbeisert@itp.phys.ethz.ch}
  {\texttt{nbeisert@itp.phys.ethz.ch}}}
\hypersetup{pdfauthor={Niklas Beisert}}
\hypersetup{pdfsubject={Manual for the LaTeX2e Package childdoc}}
\date{30 December 2018, \textsf{v2.0}}
\maketitle

\begin{abstract}\noindent
\textsf{childdoc} is a \LaTeXe{} package
that enables the direct compilation
of document sections included by |\include|
to individual files.
\end{abstract}

\begingroup
\parskip0ex
\tableofcontents
\endgroup

%%%%%%%%%%%%%%%%%%%%%%%%%%%%%%%%%%%%%%%%%%%%%%%%%%%%%%%%%%%%%%%%%%%%%%%%%%%%%%%%
%%%%%%%%%%%%%%%%%%%%%%%%%%%%%%%%%%%%%%%%%%%%%%%%%%%%%%%%%%%%%%%%%%%%%%%%%%%%%%%%
\section{Introduction}

\LaTeX{} provides a mechanism to structure a large document (such as a book)
into a main file and several child files (containing the chapters)
using the |\include| command.
This mechanism is beneficial for documents
which span hundreds of pages in order to
make the source file(s) more manageable.
Moreover, compilation can be restricted to
selected child files by means of the |\includeonly| command.
The latter feature can be used to reduce the compilation time while editing
(this was significantly more useful in the earlier days of \LaTeX{})
or to generate a smaller document which is easier to navigate.
Another application of |\includeonly| is to generate
documents consisting of selected parts of the complete document.

However, there are a few drawbacks of the plain |\include| mechanism:
\begin{itemize}
\item
The child files cannot be compiled on their own,
they can only be compiled via the main file.
A naive editing environment
(such as a text editor with an option
to have the current file processed by \LaTeX)
may require one to switch to the main file before compiling;
attempting to compile the child file produces errors.
\item
The main file must be modified (each time)
to adjust the |\includeonly| command
to the present needs. This easily leaves the main file in a messy state.
\item
The generated document will always carry the filename
of the main document. This is inconvenient if
several child files are to be compiled and
to be kept for distribution.
\end{itemize}

The present package provides a simple interface
to make child files individually compilable by \LaTeX{}.
Compiling a child file then has the same effect as compiling
the main file with an |\includeonly| command
to select the appropriate child.
Moreover the generated document will carry the name of the child
rather than the main file.
This resolves all three above issues.

This feature is meant to make the editing of books,
thesis documents and lecture notes somewhat more convenient.
However, the package can also be used efficiently for
composing a series of documents (such as exercise sheets)
which are typically distributed individually.
It then assists the author in generating the individual documents
(potentially in different versions)
as well as a document containing the collected series.
Another application is in developing style files
or other kinds of included material
where compilation of the style file could redirect
to a sample or test file.

%%%%%%%%%%%%%%%%%%%%%%%%%%%%%%%%%%%%%%%%%%%%%%%%%%%%%%%%%%%%%%%%%%%%%%%%%%%%%%%%
%%%%%%%%%%%%%%%%%%%%%%%%%%%%%%%%%%%%%%%%%%%%%%%%%%%%%%%%%%%%%%%%%%%%%%%%%%%%%%%%
\section{Usage}

First of all, the package \textsf{childdoc} is \emph{not} a standard
\LaTeXe{} |.sty| style file! Therefore it needs to be invoked in
a non-standard way.

%%%%%%%%%%%%%%%%%%%%%%%%%%%%%%%%%%%%%%%%%%%%%%%%%%%%%%%%%%%%%%%%%%%%%%%%%%%%%%%%
\subsection{Included Files}
\label{sec:include}

%%%%%%%%%%%%%%%%%%%%%%%%%%%%%%%%%%%%%%%%
\DescribeMacro{\childdocmain}
To use the package, add the commands
\begin{center}
\begin{tabular}{l}
|\input{childdoc.def}|\\
|\childdocmain{}|\\
\end{tabular}
\end{center}
at the very top of the main \LaTeX{} file,
in particular \emph{before} the |\documentclass| statement!
The argument of |\childdocmain| should be left empty
(but it must be present).

%%%%%%%%%%%%%%%%%%%%%%%%%%%%%%%%%%%%%%%%
\DescribeMacro{\childdocof}
Furthermore, add the commands
\begin{center}
\begin{tabular}{l}
|\input{childdoc.def}|\\
|\childdocof{|\textit{main}|}|\\
\end{tabular}
\end{center}
at the top of every child file \textit{child}
which is included by |\include{|\textit{child}|}|
from within the main file
(or at least for those files to be compiled individually).
The argument \textit{main} must be the filename of the main file.

There are a couple of
considerations in setting up the main and child documents:

%%%%%%%%%%%%%%%%%%%%%%%%%%%%%%%%%%%%%%%%
\paragraph{Restrictions.}

Please note the following restrictions:
\begin{itemize}
\item
|\childdocmain| must be called with one argument \textit{main}
to ensure compatibility with earlier version of the package.
It must either be empty (|\childdocmain{}|)
or precisely match the filename of the main file in which it is specified.
See \secref{sec:detection} for further information.
\item
The filename \textit{main} must be specified without the |.tex| extension.
\item
The filename \textit{main} is case sensitive
(even in case-insensitive file systems)
due to internal string comparison.
\item
The argument \textit{main} should be fully expanded, it cannot be a macro.
\item
Subdirectories and special characters should be avoided in filenames.
\item
The command |\childdocmain{|\textit{main}|}| must be followed by a whitespace.
It should not be followed immediately by another command
or by a comment mark `|%|'.
This is because the \TeX{} parser reads the token immediately following
the argument of |\childdocmain| and puts it
at the beginning of every child section;
however, a white\-space is ignored.
\end{itemize}

%%%%%%%%%%%%%%%%%%%%%%%%%%%%%%%%%%%%%%%%
\paragraph{Content of Main File.}

It is advisable to place all content in the child files included by |\include|.
Any output contained in the main file will appear in all child documents
unless suppressed manually;
it cannot be suppressed automatically by the |\includeonly| directive
and thus should normally be avoided.
A method to include some content in the main file
by means of conditional processing is described in \secref{sec:conditional}.

%%%%%%%%%%%%%%%%%%%%%%%%%%%%%%%%%%%%%%%%
\paragraph{Page Numbering.}

When only a part of the document is compiled,
the appropriate numbering of pages
(as well as other status parameters)
is determined from the |.aux| files.
The latter contain information from previous passes.
However this information needs to propagate through
all intermediate child documents.
Therefore the page numbering in child documents may well
be inconsistent until the complete document is compiled at least once.

A useful (if unconventional) way to always ensure a consistent
page numbering is to restart the numbering in each child document
and denote the pages by `\textit{child}|.|\textit{page}'
where \textit{child} represents the chapter/section number of the child file.
This can be achieved by the command
|\numberwithin{page}{|\textit{child}|}|
of the \textsf{amsmath} package
where \textit{child} can be |chapter| or |section|
depending on the chosen structuring.
Alternatively, one can modify the macro |\thepage| appropriately
and reset the counter |page| at the start of each child file.

%%%%%%%%%%%%%%%%%%%%%%%%%%%%%%%%%%%%%%%%%%%%%%%%%%%%%%%%%%%%%%%%%%%%%%%%%%%%%%%%
\subsection{Conditional Processing}
\label{sec:conditional}

The package provides a mechanism to compile different versions
of a document. To customise the versions further some conditional processing
can come in handy to distinguish which version is being compiled.
The package provides two macros to describe the compilation context:

%%%%%%%%%%%%%%%%%%%%%%%%%%%%%%%%%%%%%%%%
\DescribeMacro{\ifchilddoc}
The conditional |\ifchilddoc| distinguishes between the compilation of
child documents and the main document:
%
\begin{center}
|\ifchilddoc |\textit{child-code}| |[|\||else |\textit{main-code}]| \||fi|
\end{center}

%%%%%%%%%%%%%%%%%%%%%%%%%%%%%%%%%%%%%%%%
\DescribeMacro{\childdocname}
\DescribeMacro{\childdocjob}
The macro |\childdocname| contains the filename (without extension)
of the main or child file being processed.
Note that |\childdocjob| will always contain the name of the main file.

%%%%%%%%%%%%%%%%%%%%%%%%%%%%%%%%%%%%%%%%
\paragraph{Title Page.}

Conditional processing can be used to include a title or banner page
in the main document when proper precautions are taken.
Importantly, the code in the main file should ensure that the page counter
(as well as other status parameters which are stored in the |.aux| files)
takes the same value after the conditional processing.
Otherwise the page numbers may take divergent values
depending on which part is compiled.

For example, a title page could be declared by:
%
\begin{center}
\begin{tabular}{l}
|\ifchilddoc\||else|\\
|\addtocounter{page}{-1}|\\
\textit{code for title page}\\
|\newpage|\\
|\||fi|
\end{tabular}
\end{center}
%
A banner page for the child documents can be generated by:
%
\begin{center}
\begin{tabular}{l}
|\ifchilddoc|\\
|\addtocounter{page}{-1}|\\
\textit{code for banner page}\\
|\newpage|\\
|\||fi|
\end{tabular}
\end{center}
%
Here one could write a message such as:
\begin{center}
|This is the part \childdocname{} of \childdocjob{}.|
\end{center}

%%%%%%%%%%%%%%%%%%%%%%%%%%%%%%%%%%%%%%%%%%%%%%%%%%%%%%%%%%%%%%%%%%%%%%%%%%%%%%%%
\subsection{Flags}
\label{sec:flags}

The package makes it easy to generate different versions
of the main or child documents.
To this end compilation flags can be defined
and assigned different default values.
They will be particularly useful in conjunction
with the forwarding mechanism described in \secref{sec:forward}.

For example, it may be useful to have a flag |\version|
which can be set to |draft| or |final|.
The document source will contain some conditional code
depending on the value of |\version|.
Suppose further, the flag should default to |final| for the main file
and to |draft| for child files
which is a natural assignment for editing the document.
This is achieved by placing the following code
in the preamble of the main document
(below the |\childdocmain| directive):
%
\begin{center}
\begin{tabular}{l}
|\ifchilddoc|\\
|\providecommand{\version}{draft}|\\
|\||else|\\
|\providecommand{\version}{final}|\\
|\||fi|
\end{tabular}
\end{center}
%
The definition by |\providecommand| makes sure
that previous definitions are not overwritten.
Further statements |\providecommand{\version}{...}|
can thus be added before the above code to override it.

For the main file, one might add a line
(between |\childdocmain| and the above block)
%
\begin{center}
|%\ifchilddoc\||else\providecommand{\version}{draft}\||fi|
\end{center}
%
which can be uncommented to produce a draft version.
Likewise one can add a line to the very top of a child file
(above the |\childdocof{|\textit{main}|}| directive)
%
\begin{center}
|%\providecommand{\version}{final}|
\end{center}
%
which can be uncommented to produce the final version of this child document.

%%%%%%%%%%%%%%%%%%%%%%%%%%%%%%%%%%%%%%%%%%%%%%%%%%%%%%%%%%%%%%%%%%%%%%%%%%%%%%%%
\subsection{Forwarding}
\label{sec:forward}

Different versions of the main or child documents
using compilation flags as described in \secref{sec:flags}
can be (permanently) stored in different files
for convenient compilation, viewing and distribution.
To this end, the package defines a command
to pass on compilation to a different file:

%%%%%%%%%%%%%%%%%%%%%%%%%%%%%%%%%%%%%%%%
\DescribeMacro{\childdocforward}
The command |\childdocforward| redirects processing to
another source file:
%
\begin{center}
\begin{tabular}{l}
|\input{childdoc.def}|\\
|\childdocforward[|\textit{main}|]{|\textit{dest}|}|\\
\end{tabular}
\end{center}
%
The argument \textit{dest} is the destination file
(without extension).
It should be the main file or one of the child files.
Note that further \textsf{childdoc} directives
such as |\childdocof| and |\childdocforward|
in the indicated file will be processed in this form.
The optional argument \textit{main}
passes on directly to the main file \textit{main}
while pretending to compile the child \textit{dest}.
This form behaves as if \textit{dest}
issues |\childdocof{|\textit{main}|}| right away,
and no further \textsf{childdoc} directives will be processed.

%%%%%%%%%%%%%%%%%%%%%%%%%%%%%%%%%%%%%%%%
\DescribeMacro{\...prefix}
In the alternative form |\childdocforwardprefix|,
%
\begin{center}
\begin{tabular}{l}
|\input{childdoc.def}|\\
|\childdocforwardprefix[|\textit{main}|]{|\textit{prefix}|}{|\textit{dest}|}|
\end{tabular}
\end{center}
%
the destination file is determined by a pattern
depending on the current file:
To make this work, the current file must be called
`{\textit{prefix}\hspace{0.2em}\textit{suffix}}'
with \textit{prefix} matching precisely the argument.
Processing is then passed on to the file
`{\textit{dest}\hspace{0.2em}\textit{suffix}}'.
Surely, the same effect is achieved by
directly specifying the
argument `{\textit{dest}\hspace{0.2em}\textit{suffix}}'
in the first form.
However, that requires to set up a different file
for each child. With the alternative form of the command
all these files can have exactly the same content
which simplifies setting them up and maintaining them.

For example, the following file |draft.tex|
with a compilation flag |\version| as described in \secref{sec:flags}
compiles the main document as a draft:
%
\begin{center}
\begin{tabular}{l}
|\def\version{draft}|\\
|\input{childdoc.def}|\\
|\childdocforward{|\textit{main}|}|
\end{tabular}
\end{center}
%
Likewise, the following files |final|\textit{nn}|.tex|
compile the final version of the child document
|child|\textit{nn}|.tex|:
%
\begin{center}
\begin{tabular}{l}
|\def\version{final}|\\
|\input{childdoc.def}|\\
|\childdocforwardprefix{final}{child}|
\end{tabular}
\end{center}
%

Note that when several versions of a main file and/or of each child file
are to be generated, it may be convenient to set up a |Makefile| or
shell script to automatise the process.

%%%%%%%%%%%%%%%%%%%%%%%%%%%%%%%%%%%%%%%%%%%%%%%%%%%%%%%%%%%%%%%%%%%%%%%%%%%%%%%%
\subsection{Command Line Processing}
\label{sec:commandline}

The effect of redirection files can also be achieved by invoking
the \LaTeX{} compiler with a more elaborate command line.
Most conveniently this should be done as part
of a shell script or a |Makefile|.

When using \textsf{childdoc} in the main file, the following
command lines effectively perform a redirection
(note that depending on the shell being used,
backslashes may have to be doubled: `|\|' $\to$ `|\\|'):
%
\begin{center}
|... -jobname "|\textit{target}|" |\\|"|[\textit{flags}]%
|\input{childdoc.def}\childdocforward[|\textit{main}|]{|\textit{dest}|}"|
\end{center}
%
Here \textit{target} is the name of the output file,
\textit{main} is the name of the main file
and \textit{dest} is the name of the main or child file to be processed
(all filenames without extensions).
The optional argument \textit{main} can be omitted
if \textit{main} matches \textit{dest}.
Optionally, compilation \textit{flags} can be defined via |\def| commands.
This command line makes the \TeX{} engine believe
it is compiling the file \textit{target}
whose content is specified as the latter parameter.
The provided code then forwards the processing to
\textit{main} or \textit{dest} as described in \secref{sec:forward}.

%%%%%%%%%%%%%%%%%%%%%%%%%%%%%%%%%%%%%%%%%%%%%%%%%%%%%%%%%%%%%%%%%%%%%%%%%%%%%%%%
\subsection{Include by Input}
\label{sec:input}

Including child documents by |\include| has some restrictions by design.
Most notably, the content of a child document always occupies
its own set of pages; pages cannot be shared between child documents.
Usually, this behaviour makes perfect sense
because each child document contain an essential part of the document.
However, in some situations it may be desirable to compose
a document from a collection of parts
without having mandatory page breaks between then.
For this case, the package
provides a mechanism to include parts
by |\input| which can also be processed individually.
However, by construction this mechanism
requires manual handling of the content to be output.

%%%%%%%%%%%%%%%%%%%%%%%%%%%%%%%%%%%%%%%%
\DescribeMacro{\ifchilddocmanual}
The main file should be prepared as usual, see \secref{sec:include}.
However, the document body must make a distinction
between processing of an individual part and of the main document, e.g.:
%
\begin{center}
\begin{tabular}{l}
|\ifchilddocmanual|\\
|\input{\childdocname}|\\
|\||else|\\
\textit{document body with }|\input{|\textit{part}|}|\\
|\||fi|
\end{tabular}
\end{center}
%
The conditional |\ifchilddocmanual| is true whenever
a part to be included by |\input| is being compiled,
and the name of the part is stored in |\childdocname|.

%%%%%%%%%%%%%%%%%%%%%%%%%%%%%%%%%%%%%%%%
\DescribeMacro{\childdocby}
Each part to be included by |\input| should start with:
%
\begin{center}
\begin{tabular}{l}
|\input{childdoc.def}|\\
|\childdocby{|\textit{main}|}|\\
\end{tabular}
\end{center}
%
The directive |\childdocby| is similar to |\childdocof|
described in \secref{sec:include},
but the subsequent selection of content must be done manually.
To that end, both |\ifchilddoc| and |\ifchilddocmanual|
will be true upon processing of a part,
and the name of the part is stored in |\childdocname|.
Note that |\jobname| will be set to the filename of the current part
so that each part receives an individual |.aux| file
that does not interfere with the |.aux| file(s) of the main document.
This behaviour can be altered by the alternative form
|\childdocby[*]{|\textit{main}|}| (with a non-empty optional argument)
which uses the |.aux| file of the main document
by setting |\jobname| to \textit{main}.

%%%%%%%%%%%%%%%%%%%%%%%%%%%%%%%%%%%%%%%%%%%%%%%%%%%%%%%%%%%%%%%%%%%%%%%%%%%%%%%%
\subsection{Driver Development}
\label{sec:driver}

The \textsf{childdoc} mechanism can also be use for the development
of definition files such as \LaTeX{} styles or classes.
This case differs from the above setup with multiple parts
included by |\include| in that no |\includeonly| should be invoked.
This can be achieved by starting the include file
(before |\ProvidesPackage|) with:
%
\begin{center}
\begin{tabular}{l}
|\input{childdoc.def}|\\
|\childdocforward{|\textit{main}|}|\\
\end{tabular}
\end{center}
%
or alternatively with:
%
\begin{center}
\begin{tabular}{l}
|\input{childdoc.def}|\\
|\childdocby{|\textit{main}|}|\\
\end{tabular}
\end{center}
%
Both forms have slightly different effects as described above.
The main file is prepared as usual, see \secref{sec:include}.

%%%%%%%%%%%%%%%%%%%%%%%%%%%%%%%%%%%%%%%%%%%%%%%%%%%%%%%%%%%%%%%%%%%%%%%%%%%%%%%%
\subsection{Legacy Detection}
\label{sec:detection}

The directive |\childdocmain| in the main file can detect
whether the complete document or merely a child is to be compiled
even without using the directive |\childdocof|.
This method is deprecated because it is less robust
and there is no compelling reason to use it;
it is merely provided for backward compatibility
and it may be removed in future versions.

If the detection mechanism is to be used,
it is mandatory to correctly specify
the filename of the main file as the argument of |\childdocmain|:
%
\begin{center}
\begin{tabular}{l}
|\input{childdoc.def}|\\
|\childdocmain{|\textit{main}|}|\\
\end{tabular}
\end{center}
%
If |\jobname| does not match the argument \textit{main} of |\childdocmain|,
it is assumed that |\jobname| points to the child file to be compiled.
When using |\childdocmain| with the main file specified as argument,
it suffices to start a child file
with just |\input{|\textit{main}|}|
without loading of the package and using |\childdocof|.
If instead all processing is done
with the appropriate \textsf{childdoc} directives,
the argument of \textit{main} of |\childdocmain| can be empty.

An alternative version of the command line processing described
in \secref{sec:commandline} using the detection mechanism reads:
%
\begin{center}
|... -jobname "|\textit{target}|" "|[\textit{flags}]%
[|\def\jobname{|\textit{dest}|}|]|\input{|\textit{main}|}"|
\end{center}

%%%%%%%%%%%%%%%%%%%%%%%%%%%%%%%%%%%%%%%%%%%%%%%%%%%%%%%%%%%%%%%%%%%%%%%%%%%%%%%%
\subsection{Manual Code}
\label{sec:manual}

In case one cannot be certain whether the definitions file |childdoc.def|
is installed on the target \TeX{} distribution
and one prefers not to ship it,
it is conceivable to paste a few relevant commands into the sources.

To that end, drop all statements |\input{childdoc.def}|
and perform the replacements as outlined below.
Instead of |\childdocmain{|\textit{main}|}| add the following code
to the top of the main file:
%
\begin{center}
\begin{tabular}{l}
|\||ifdefined\childdocname\endinput\||fi\newif\ifchilddoc|\\
|\edef\childdocname{\scantokens\expandafter{\jobname\noexpand}}|\\
|\def\childdocmain{|\textit{main}|}\||ifx\childdocmain\childdocname\||else|\\
|\childdoctrue\includeonly{\childdocname}\let\jobname\childdocmain\||fi|\\
\end{tabular}
\end{center}
%
Instead of |\childdocof{|\textit{main}|}| just include the main file
at the top of each child file:
%
\begin{center}
|\input{|\textit{main}|}|
\end{center}
%
A simple redirection |\childdocforward{|\textit{dest}|}| is achieved by:
%
\begin{center}
|\def\jobname{|\textit{dest}|}\input{\jobname}|
\end{center}
%
The redirection with prefix
|\childdocforwardprefix[|\textit{prefix}|]{|\textit{dest}|}|
is accomplished by:
%
\begin{center}
\begin{tabular}{l}
|{\edef\jobname{\scantokens\expandafter{\jobname\noexpand}}|\\
|\def\redirectjob |\textit{prefix}|#1~~~{\gdef\jobname{|\textit{dest}|#1}}|\\
|\expandafter\redirectjob\jobname~~~}\input{\jobname}|
\end{tabular}
\end{center}

In an alternative approach,
child documents can be compiled by a specific command line
without additional code or specific definitions:
%
\begin{center}
|... -jobname "|\textit{target}|" "|[\textit{flags}]%
|\includeonly{|\textit{dest}|}\input{|\textit{main}|}"|
\end{center}
%

%%%%%%%%%%%%%%%%%%%%%%%%%%%%%%%%%%%%%%%%%%%%%%%%%%%%%%%%%%%%%%%%%%%%%%%%%%%%%%%%
%%%%%%%%%%%%%%%%%%%%%%%%%%%%%%%%%%%%%%%%%%%%%%%%%%%%%%%%%%%%%%%%%%%%%%%%%%%%%%%%
\section{Information}

%%%%%%%%%%%%%%%%%%%%%%%%%%%%%%%%%%%%%%%%%%%%%%%%%%%%%%%%%%%%%%%%%%%%%%%%%%%%%%%%
\subsection{Copyright}

Copyright \copyright{} 2017--2018 Niklas Beisert

This work may be distributed and/or modified under the
conditions of the \LaTeX{} Project Public License, either version 1.3
of this license or (at your option) any later version.
The latest version of this license is in
  \url{http://www.latex-project.org/lppl.txt}
and version 1.3 or later is part of all distributions of \LaTeX{}
version 2005/12/01 or later.

This work has the LPPL maintenance status `maintained'.

The Current Maintainer of this work is Niklas Beisert.

This work consists of the files |README.txt|, |childdoc.ins| and |childdoc.dtx|
as well as the derived files |childdoc.def|, |cdocsamp.tex|
with |cdocsch1.tex|, |cdocsch2.tex|, |cdocspt3.tex|, |cdocspt4.tex|,
|cdocsdrf.tex|, |cdocsfn1.tex|, |cdocsfn2.tex|
as well as |childdoc.pdf|.

%%%%%%%%%%%%%%%%%%%%%%%%%%%%%%%%%%%%%%%%%%%%%%%%%%%%%%%%%%%%%%%%%%%%%%%%%%%%%%%%
\subsection{Files and Installation}

The package consists of the files:
%
\begin{center}
\begin{tabular}{ll}
    |README.txt|   & readme file \\
    |childdoc.ins| & installation file \\
    |childdoc.dtx| & source file \\
    |childdoc.def| & definition file \\
    |cdocsamp.tex| & sample main file \\
    |cdocsch1.tex| & sample include file \\
    |cdocsch2.tex| & sample include file \\
    |cdocspt3.tex| & sample part file \\
    |cdocspt4.tex| & sample part file \\
    |cdocsdrf.tex| & sample redirection file \\
    |cdocsfn1.tex| & sample redirection file \\
    |cdocsfn2.tex| & sample redirection file \\
    |childdoc.pdf| & manual
\end{tabular}
\end{center}
%
The distribution consists of the files
|README.txt|, |childdoc.ins| and |childdoc.dtx|.
%
\begin{itemize}
\item
Run (pdf)\LaTeX{} on |childdoc.dtx|
to compile the manual |childdoc.pdf| (this file).
\item
Run \LaTeX{} on |childdoc.ins| to create the definitions file |childdoc.def|
and the sample |cdocsamp.tex| with include files
|cdocsch1.tex|, |cdocsch2.tex|, |cdocspt3.tex|, |cdocspt4.tex|,
|cdocsdrf.tex|, |cdocsfn1.tex|, |cdocsfn2.tex|.
Then copy the file |childdoc.def| to an appropriate directory of your \LaTeX{}
distribution, e.g.\ \textit{texmf-root}|/tex/latex/childdoc|.
\end{itemize}

%%%%%%%%%%%%%%%%%%%%%%%%%%%%%%%%%%%%%%%%%%%%%%%%%%%%%%%%%%%%%%%%%%%%%%%%%%%%%%%%
\subsection{Related CTAN Packages}

There are several other packages which offer a similar functionality:
%
\begin{itemize}
\item
The packages
\href{http://ctan.org/pkg/docmute}{\textsf{docmute}},
\href{http://ctan.org/pkg/includex}{\textsf{includex}} and
\href{http://ctan.org/pkg/standalone}{\textsf{standalone}}
provide commands to include only the document body of
a child file thus allowing both files to be compiled individually.
\item
The packages \href{http://ctan.org/pkg/subdocs}{\textsf{subdocs}}
and \href{http://ctan.org/pkg/subfiles}{\textsf{subfiles}}
provide structures in which the main and child documents can be
encapsulated and allowing them to be compiled individually.
The inclusion mechanism is different from the conventional |\include|.
\item
The package \href{http://ctan.org/pkg/combine}{\textsf{combine}}
is an elaborate solution to combine several documents into one.
\end{itemize}
%
See also the CTAN topic \href{http://ctan.org/topic/subdocs}{\textsf{subdocs}}
for further related packages.
The present package differs from the above solutions in that
a document structure constructed with the conventional |\include| mechanism
just needs two extra commands at the top of every file
such that all constituent files can be compiled individually.

%%%%%%%%%%%%%%%%%%%%%%%%%%%%%%%%%%%%%%%%%%%%%%%%%%%%%%%%%%%%%%%%%%%%%%%%%%%%%%%%
%\subsection{Feature Suggestions}
%
%The following is a list of features which may be useful for future
%versions of this package:
%%
%\begin{itemize}
%\item
%\ldots
%\end{itemize}

%%%%%%%%%%%%%%%%%%%%%%%%%%%%%%%%%%%%%%%%%%%%%%%%%%%%%%%%%%%%%%%%%%%%%%%%%%%%%%%%
\subsection{Revision History}

%%%%%%%%%%%%%%%%%%%%%%%%%%%%%%%%%%%%%%%%
\paragraph{v2.0:} 2018/12/30

\begin{itemize}
\item
immediate forward processing
\item
added |\childdocby| mechanism
\item
manual restructured
\end{itemize}

%%%%%%%%%%%%%%%%%%%%%%%%%%%%%%%%%%%%%%%%
\paragraph{v1.6:} 2018/01/17

\begin{itemize}
\item
application for development of include files
\item
corrections to manual
\end{itemize}

%%%%%%%%%%%%%%%%%%%%%%%%%%%%%%%%%%%%%%%%
\paragraph{v1.5:} 2017/05/21

\begin{itemize}
\item
more complete structuring introduced
\item
|\childdocof| introduced
\item
|\childdoc| renamed to |\childdocmain|
\item
|\childredirect| renamed to |\childdocforward| and |\childdocforwardprefix|
and functionality expanded
\end{itemize}

%%%%%%%%%%%%%%%%%%%%%%%%%%%%%%%%%%%%%%%%
\paragraph{v1.0:} 2017/04/27

\begin{itemize}
\item
manual and install package
\item
first version published on CTAN
\end{itemize}

%%%%%%%%%%%%%%%%%%%%%%%%%%%%%%%%%%%%%%%%
\paragraph{v0.6:} 2017/04/26

\begin{itemize}
\item
redirection mechanism added
\end{itemize}

%%%%%%%%%%%%%%%%%%%%%%%%%%%%%%%%%%%%%%%%
\paragraph{v0.5:} 2017/04/26

\begin{itemize}
\item
functionality in definition file
\end{itemize}


%%%%%%%%%%%%%%%%%%%%%%%%%%%%%%%%%%%%%%%%%%%%%%%%%%%%%%%%%%%%%%%%%%%%%%%%%%%%%%%%
%%%%%%%%%%%%%%%%%%%%%%%%%%%%%%%%%%%%%%%%%%%%%%%%%%%%%%%%%%%%%%%%%%%%%%%%%%%%%%%%
%%%%%%%%%%%%%%%%%%%%%%%%%%%%%%%%%%%%%%%%%%%%%%%%%%%%%%%%%%%%%%%%%%%%%%%%%%%%%%%%
\appendix

\settowidth\MacroIndent{\rmfamily\scriptsize 000\ }

 \DocInput{childdoc.dtx}

\end{document}
%</driver>
% \fi
%
% %%%%%%%%%%%%%%%%%%%%%%%%%%%%%%%%%%%%%%%%%%%%%%%%%%%%%%%%%%%%%%%%%%%%%%%%%%%%%%
% %%%%%%%%%%%%%%%%%%%%%%%%%%%%%%%%%%%%%%%%%%%%%%%%%%%%%%%%%%%%%%%%%%%%%%%%%%%%%%
% \section{Sample}
%\iffalse
%<*samplemain>
%\fi
%
% The following presents a sample document
% with two chapters, two parts, a title page,
% a compile flag as well as three forwarding files to set the flag.
% It consists of eight |.tex| files:
% \begin{center}
% \begin{tabular}{ll}
% |cdocsamp.tex|&main file\\
% |cdocsch1.tex|&include file for chapter 1\\
% |cdocsch2.tex|&include file for chapter 2\\
% |cdocspt3.tex|&include file for part 3\\
% |cdocspt4.tex|&include file for part 4\\
% |cdocsdrf.tex|&forwarding file for main file in draft mode\\
% |cdocsfi1.tex|&forwarding file for final version of chapter 1\\
% |cdocsfi2.tex|&forwarding file for final version of chapter 2\\
% \end{tabular}
% \end{center}
% Each of the eight files can be compiled directly by the \LaTeX{} compiler.
%
% %%%%%%%%%%%%%%%%%%%%%%%%%%%%%%%%%%%%%%
% \paragraph{Main File.}
%
% The main file is called |cdocsamp.tex|.
%
% Load the \textsf{childdoc} definitions and
% declare the filename for the main document:
%    \begin{macrocode}
\input{childdoc.def}
\childdocmain{}
%    \end{macrocode}

% Optional override for |\version| flag:
%    \begin{macrocode}
%%\ifchilddoc\else\providecommand{\version}{draft}\fi
%    \end{macrocode}

% Define the default values for the |\version| flag
% (|final| for the main file and |draft| for childs):
%    \begin{macrocode}
\ifchilddoc
\providecommand{\version}{draft}
\else
\providecommand{\version}{final}
\fi
%    \end{macrocode}

% Load the standard document class:
%    \begin{macrocode}
\documentclass[12pt]{article}
%    \end{macrocode}

% Start the document body:
%    \begin{macrocode}
\begin{document}
%    \end{macrocode}

% Declare a title page.
% Print title, part of document being processed and version flag:
%    \begin{macrocode}
\addtocounter{page}{-1}
\begin{center}
{\LARGE\bfseries{}childdoc example\par}
\vspace{1cm}
\ifchilddoc
\ifchilddocmanual part\else chapter\fi:
`\childdocname' of `\childdocjob'\par
\else
main document: `\childdocjob'\par
\fi
version: \version\par
\end{center}
\newpage
%    \end{macrocode}

% Manually include selected file,
% otherwise process as usual:
%    \begin{macrocode}
\ifchilddocmanual
\section*{part `\childdocname'}
\input{\childdocname}
\else
%    \end{macrocode}

% Include the two chapters:
%    \begin{macrocode}
\include{cdocsch1}
\include{cdocsch2}
%    \end{macrocode}

% Include the two parts unless only chapters should be displayed:
%    \begin{macrocode}
\ifchilddoc\else
\section{part three}
\input{cdocspt3}
\section{part four}
\input{cdocspt4}
\fi
%    \end{macrocode}

% Process as usual until here:
%    \begin{macrocode}
\fi
%    \end{macrocode}

% End of document body:
%    \begin{macrocode}
\end{document}
%    \end{macrocode}
%\iffalse
%</samplemain>
%\fi
%
% %%%%%%%%%%%%%%%%%%%%%%%%%%%%%%%%%%%%%%
% \paragraph{Chapter Include Files.}
%
% The include files are called |cdocsch1.tex| and |cdocsch2.tex|.
%
%\iffalse
%<*samplechap1|samplechap2>
%\fi

% Optional override for |\version| flag:
%    \begin{macrocode}
%%\providecommand{\version}{final}
%    \end{macrocode}

% Include the main document:
%    \begin{macrocode}
\input{childdoc.def}
\childdocof{cdocsamp}
%    \end{macrocode}

%\iffalse
%</samplechap1|samplechap2>
%\fi
%
%\iffalse
%<*samplechap1>
%\fi
% Some text for chapter 1:
%    \begin{macrocode}
\section{one}
some text in chapter one
%    \end{macrocode}

%\iffalse
%</samplechap1>
%\fi
% Some text for chapter 2:
%\iffalse
%<*samplechap2>
%\fi
%    \begin{macrocode}
\section{two}
more text in chapter two
%    \end{macrocode}

%\iffalse
%</samplechap2>
%\fi
%
% %%%%%%%%%%%%%%%%%%%%%%%%%%%%%%%%%%%%%%
% \paragraph{Part Include Files.}
%
% The include files are called |cdocspt3.tex| and |cdocspt4.tex|.
%
%\iffalse
%<*samplepart3|samplepart4>
%\fi

% Optional override for |\version| flag:
%    \begin{macrocode}
%%\providecommand{\version}{final}
%    \end{macrocode}

% Include the main document:
%    \begin{macrocode}
\input{childdoc.def}
\childdocby{cdocsamp}
%    \end{macrocode}

%\iffalse
%</samplepart3|samplepart4>
%\fi
%
%\iffalse
%<*samplepart3>
%\fi
% Some text for part 3:
%    \begin{macrocode}
some text in part three
%    \end{macrocode}

%\iffalse
%</samplepart3>
%\fi
% Some text for part 4:
%\iffalse
%<*samplepart4>
%\fi
%    \begin{macrocode}
more text in part four
%    \end{macrocode}

%\iffalse
%</samplepart4>
%\fi
%
% %%%%%%%%%%%%%%%%%%%%%%%%%%%%%%%%%%%%%%
% \paragraph{Forwarding for a Complete Draft.}
%
% The following forwarding file |cdocsdrf.tex|
% compiles the main document in draft mode:
%\iffalse
%<*sampledraft>
%\fi
%    \begin{macrocode}
\def\version{draft}
\input{childdoc.def}
\childdocforward{cdocsamp}
%    \end{macrocode}

%\iffalse
%</sampledraft>
%\fi
%
% %%%%%%%%%%%%%%%%%%%%%%%%%%%%%%%%%%%%%%
% \paragraph{Forwarding for Final Version of the Chapters.}
%
% The following forwarding files |cdocsfn1.tex| and |cdocsfn2.tex|
% (with identical content)
% compile the final versions of the child documents
% |cdocsch1.tex| and |cdocsch2.tex|, respectively:
%\iffalse
%<*samplefinal>
%\fi
%    \begin{macrocode}
\def\version{final}
\input{childdoc.def}
\childdocforwardprefix[cdocsamp]{cdocsfn}{cdocsch}
%    \end{macrocode}

%\iffalse
%</samplefinal>
%\fi
%
% %%%%%%%%%%%%%%%%%%%%%%%%%%%%%%%%%%%%%%
% \paragraph{Command Line Processing.}
%
% The following three command lines generate the output files
% |cdocscld|, |cdocscl1| and |cdocscl2|
% which should be identical to
% |cdocsdrf|, |cdocsch1| and |cdocsfn2|, respectively:
% \begin{center}
% \begin{tabular}{l}
% |latex -jobname cdocscld \|\\
% |  "\def\version{draft}\input{childdoc.def}\childdocforward{cdocsamp}"|\\
% |latex -jobname cdocscl1 \|\\
% |  "\input{childdoc.def}\childdocforward[cdocsamp]{cdocsch1}"|\\
% |latex -jobname cdocscl2 \|\\
% |  "\def\version{final}\input{childdoc.def}\childdocforward{cdocsch2}"|
% \end{tabular}
% \end{center}
% Note that the trailing backslash on each first line
% merely continues the input to the second line
% (for convenient cut ant paste).
% Furthermore, the command |latex| can be replaced by any
% of its alternative versions such as |pdflatex|.
%
% %%%%%%%%%%%%%%%%%%%%%%%%%%%%%%%%%%%%%%%%%%%%%%%%%%%%%%%%%%%%%%%%%%%%%%%%%%%%%%
% %%%%%%%%%%%%%%%%%%%%%%%%%%%%%%%%%%%%%%%%%%%%%%%%%%%%%%%%%%%%%%%%%%%%%%%%%%%%%%
% \section{Implementation}
%\iffalse
%<*package>
%\fi
%
% This section describes the definitions file |childdoc.def|.

% The definitions cannot be loaded using |\usepackage| or |\RequirePackage|
% which has a mechanism to prevent loading a style file more than once.
% When loading the definitions by means of |\input|
% multiple instances have to be prevented manually:
%\iffalse
%This code needs to be before the `\ProvidesFile' directive
%which is defined at the beginning of this file.
%Therefore it is also placed there and commented out here.
%</package>
%<*discard>
%\fi
%    \begin{macrocode}
\ifdefined\childdocmain\endinput\fi
%    \end{macrocode}
%\iffalse
%</discard>
%<*package>
%\fi
%
% \macro{\ifchilddoc}
% \macro{\ifchilddocmanual}
% The conditional |\ifchilddoc| tells whether a
% child (true) or main (false) document is being compiled.
% The conditional |\ifchilddocmanual| tells whether
% the |\includeonly| mechanism is used (false) or
% the selection of child files must be performed manually (true).
% The definitions initialise to false:
%    \begin{macrocode}
\newif\ifchilddoc
\newif\ifchilddocmanual
%    \end{macrocode}

% \macro{\childdocname}
% \macro{\childdocjob}
% The macro |\childdocname| stores the name of the main document
% to be compiled. The macro |\childdocjob| stores the name of
% the document on which the \LaTeX{} compiler was originally invoked.
% The content of |\jobname| cannot be compared
% to filenames specified in the source due to different catcodes.
% The following code rescans |\jobname|, stores the result
% in |\childdocname| and saves a copy in |\childdocjob|:
%    \begin{macrocode}
\edef\childdocname{\scantokens\expandafter{\jobname\noexpand}}
\let\childdocjob\childdocname
%    \end{macrocode}

% \macro{\childdocdisable}
% The macro |\childdocdisable| prevents the main file
% from being processed more than once.
% At this stage, the main document command |\childdocmain|
% is assumed to be called once again where it should do nothing.
% Any subsequent call to it should prevent
% a secondary processing of the main document
% It overwrites the forwarding commands
% |\childdocof| and |\childdocforward|
% with empty macros to prevent further inclusions of the main document:
%    \begin{macrocode}
\newcommand{\childdocdisable}
{
  \renewcommand{\childdocmain}[1]{\renewcommand{\childdocmain}[1]{\endinput}}
  \renewcommand{\childdocof}[1]{}
  \renewcommand{\childdocby}[2][]{}
  \renewcommand{\childdocforward}[2][]{}
  \renewcommand{\childdocdisable}{}
}
%    \end{macrocode}

% \macro{\childdocmain}
% The macro |\childdocmain| is to be called at the top of the main file
% with nothing or the main filename (without extension) as argument.
% First, it breaks loops.
% If the argument is not empty and does not match |\childdocname|
% (which is set by the first inclusion of |childdoc.def|),
% |\ifchilddoc| is set to true, |\includeonly| is applied to the child file
% and |\jobname| is set to the main file
% (for proper handling of |.aux| files):
%    \begin{macrocode}
\newcommand{\childdocmain}[1]
{
  \childdocdisable\childdocmain{}
  \if?#1?\else
    \begingroup
      \def\childdoctmp{#1}
      \ifx\childdoctmp\childdocname
        \def\childdoctmp{}
      \else
        \def\childdoctmp
        {
          \childdoctrue
          \includeonly{\childdocname}
          \def\childdocjob{#1}
          \def\jobname{#1}
        }
      \fi
      \expandafter
    \endgroup
    \childdoctmp
  \fi
}
%    \end{macrocode}

% \macro{\childdocof}
% The command |\childdocof| redirects
% compilation to the main file |#1|.
%    \begin{macrocode}
\newcommand{\childdocof}[1]
{
  \childdocdisable
  \childdoctrue
  \includeonly{\childdocname}
  \def\jobname{#1}
  \def\childdocjob{#1}
  \input{#1}
}
%    \end{macrocode}

% \macro{\childdocby}
% The command |\childdocby| ....
%    \begin{macrocode}
\newcommand{\childdocby}[2][]
{
  \childdocdisable
  \childdoctrue
  \childdocmanualtrue
  \if?#1?\else
    \def\jobname{#2}
  \fi
  \def\childdocjob{#2}
  \input{#2}
  \endinput
}
%    \end{macrocode}

% \macro{\childdocforward}
% The command |\childdocforward| redirects
% compilation to the main file or
% (if the optional argument is given) a child file.
% Parameters are set as if the main file
% or a child file starting with |\childdocof| was compiled.
% Then compilation is handed over to the main file:
%    \begin{macrocode}
\newcommand{\childdocforward}[2][]
{
  \begingroup
    \if?#1?
      \def\childdoctmp
      {
        \def\childdocname{#2}
        \def\childdocjob{#2}
        \def\jobname{#2}
        \input{#2}
        \endinput
      }
    \else
      \def\childdoctmp
      {
        \childdocdisable
        \def\childdocname{#2}
        \childdoctrue
        \includeonly{#2}
        \def\childdocjob{#1}
        \def\jobname{#1}
        \input{#1}
        \endinput
      }
    \fi
    \expandafter
  \endgroup
  \childdoctmp
}
%    \end{macrocode}

% \macro{\childdocforwardprefix}
% The command |\childdocforwardprefix| redirects
% compilation to the main or a child file by means of a pattern.
% The prefix |#1| in the current filename is replaced by |#2|
% and the suffix of the current filename is kept
% (it is assumed that the filename does not contain the substring `|~~~|'
% which is used as a delimiter).
% Compilation is handed over to the new file by |\childdocforward|:
%    \begin{macrocode}
\newcommand{\childdocforwardprefix}[3][]
{
  \begingroup
    \def\childdocextract #2##1~~~{\def\childdoctmp{\childdocforward[#1]{#3##1}}}
    \expandafter\childdocextract\childdocname~~~
    \expandafter
  \endgroup
  \childdoctmp
}
%    \end{macrocode}

% \macro{\childdoc}
% The deprecated macro |\childdoc| is a legacy version of |\childdocmain|:
%    \begin{macrocode}
\newcommand{\childdoc}{\childdocmain}
%    \end{macrocode}

% \macro{\childdocredirect}
% The deprecated macro |\childdocredirect| is a legacy version
% of |\childdocforward| and |\childdocforwardprefix|:
%    \begin{macrocode}
\newcommand{\childdocredirect}[2][]
{
  \begingroup
    \if?#1?
      \def\childdoctmp{\childdocforward{#2}}
    \else
      \def\childdoctmp{\childdocforwardprefix{#1}{#2}}
    \fi
    \expandafter
  \endgroup
  \childdoctmp
}
%    \end{macrocode}

%\iffalse
%</package>
%\fi
%
\endinput
|\\
|\childdocforwardprefix[|\textit{main}|]{|\textit{prefix}|}{|\textit{dest}|}|
\end{tabular}
\end{center}
%
the destination file is determined by a pattern
depending on the current file:
To make this work, the current file must be called
`{\textit{prefix}\hspace{0.2em}\textit{suffix}}'
with \textit{prefix} matching precisely the argument.
Processing is then passed on to the file
`{\textit{dest}\hspace{0.2em}\textit{suffix}}'.
Surely, the same effect is achieved by
directly specifying the
argument `{\textit{dest}\hspace{0.2em}\textit{suffix}}'
in the first form.
However, that requires to set up a different file
for each child. With the alternative form of the command
all these files can have exactly the same content
which simplifies setting them up and maintaining them.

For example, the following file |draft.tex|
with a compilation flag |\version| as described in \secref{sec:flags}
compiles the main document as a draft:
%
\begin{center}
\begin{tabular}{l}
|\def\version{draft}|\\
|% \iffalse
%
% childdoc.dtx Copyright (C) 2017-2018 Niklas Beisert
%
% This work may be distributed and/or modified under the
% conditions of the LaTeX Project Public License, either version 1.3
% of this license or (at your option) any later version.
% The latest version of this license is in
%   http://www.latex-project.org/lppl.txt
% and version 1.3 or later is part of all distributions of LaTeX
% version 2005/12/01 or later.
%
% This work has the LPPL maintenance status `maintained'.
%
% The Current Maintainer of this work is Niklas Beisert.
%
% This work consists of the files childdoc.dtx and childdoc.ins
% and the derived files childdoc.def and cdocsamp.tex with
% cdocsch1.tex, cdocsch2.tex, cdocsdrf.tex, cdocsfn1.tex, cdocsfn2.tex.
%
%<package>\ifdefined\childdocmain\endinput\fi
%<package>\ProvidesFile{childdoc.def}[2018/12/30 v2.0 child document driver]
%<samplemain>\ProvidesFile{cdocsamp.tex}[2018/12/30 v2.0 sample for childdoc]
%<*driver>
%\ProvidesFile{childdoc.drv}[2018/12/30 v2.0 childdoc reference manual file]
\PassOptionsToClass{10pt,a4paper}{article}
\documentclass{ltxdoc}

\usepackage[margin=35mm]{geometry}
\usepackage{hyperref}
\usepackage{hyperxmp}
\usepackage[usenames]{color}

\hypersetup{colorlinks=true}
\hypersetup{pdfstartview=FitH}
\hypersetup{pdfpagemode=UseNone}
\hypersetup{pdfsource={}}
\hypersetup{pdflang={en-UK}}
\hypersetup{pdfcopyright={Copyright 2017-2018 Niklas Beisert.
  This work may be distributed and/or modified under the
  conditions of the LaTeX Project Public License, either version 1.3
  of this license or (at your option) any later version.}}
\hypersetup{pdflicenseurl={http://www.latex-project.org/lppl.txt}}
\hypersetup{pdfcontactaddress={ETH Zurich, ITP, HIT K,
  Wolfgang-Pauli-Strasse 27}}
\hypersetup{pdfcontactpostcode={8093}}
\hypersetup{pdfcontactcity={Zurich}}
\hypersetup{pdfcontactcountry={Switzerland}}
\hypersetup{pdfcontactemail={nbeisert@itp.phys.ethz.ch}}
\hypersetup{pdfcontacturl={http://people.phys.ethz.ch/\xmptilde nbeisert/}}

\newcommand{\secref}[1]{\hyperref[#1]{section \ref*{#1}}}

\parskip1ex
\parindent0pt
\let\olditemize\itemize
\def\itemize{\olditemize\parskip0pt}

\begin{document}

\title{The \textsf{childdoc} Package}
\hypersetup{pdftitle={The childdoc Package}}
\author{Niklas Beisert\\[2ex]
  Institut f\"ur Theoretische Physik\\
  Eidgen\"ossische Technische Hochschule Z\"urich\\
  Wolfgang-Pauli-Strasse 27, 8093 Z\"urich, Switzerland\\[1ex]
  \href{mailto:nbeisert@itp.phys.ethz.ch}
  {\texttt{nbeisert@itp.phys.ethz.ch}}}
\hypersetup{pdfauthor={Niklas Beisert}}
\hypersetup{pdfsubject={Manual for the LaTeX2e Package childdoc}}
\date{30 December 2018, \textsf{v2.0}}
\maketitle

\begin{abstract}\noindent
\textsf{childdoc} is a \LaTeXe{} package
that enables the direct compilation
of document sections included by |\include|
to individual files.
\end{abstract}

\begingroup
\parskip0ex
\tableofcontents
\endgroup

%%%%%%%%%%%%%%%%%%%%%%%%%%%%%%%%%%%%%%%%%%%%%%%%%%%%%%%%%%%%%%%%%%%%%%%%%%%%%%%%
%%%%%%%%%%%%%%%%%%%%%%%%%%%%%%%%%%%%%%%%%%%%%%%%%%%%%%%%%%%%%%%%%%%%%%%%%%%%%%%%
\section{Introduction}

\LaTeX{} provides a mechanism to structure a large document (such as a book)
into a main file and several child files (containing the chapters)
using the |\include| command.
This mechanism is beneficial for documents
which span hundreds of pages in order to
make the source file(s) more manageable.
Moreover, compilation can be restricted to
selected child files by means of the |\includeonly| command.
The latter feature can be used to reduce the compilation time while editing
(this was significantly more useful in the earlier days of \LaTeX{})
or to generate a smaller document which is easier to navigate.
Another application of |\includeonly| is to generate
documents consisting of selected parts of the complete document.

However, there are a few drawbacks of the plain |\include| mechanism:
\begin{itemize}
\item
The child files cannot be compiled on their own,
they can only be compiled via the main file.
A naive editing environment
(such as a text editor with an option
to have the current file processed by \LaTeX)
may require one to switch to the main file before compiling;
attempting to compile the child file produces errors.
\item
The main file must be modified (each time)
to adjust the |\includeonly| command
to the present needs. This easily leaves the main file in a messy state.
\item
The generated document will always carry the filename
of the main document. This is inconvenient if
several child files are to be compiled and
to be kept for distribution.
\end{itemize}

The present package provides a simple interface
to make child files individually compilable by \LaTeX{}.
Compiling a child file then has the same effect as compiling
the main file with an |\includeonly| command
to select the appropriate child.
Moreover the generated document will carry the name of the child
rather than the main file.
This resolves all three above issues.

This feature is meant to make the editing of books,
thesis documents and lecture notes somewhat more convenient.
However, the package can also be used efficiently for
composing a series of documents (such as exercise sheets)
which are typically distributed individually.
It then assists the author in generating the individual documents
(potentially in different versions)
as well as a document containing the collected series.
Another application is in developing style files
or other kinds of included material
where compilation of the style file could redirect
to a sample or test file.

%%%%%%%%%%%%%%%%%%%%%%%%%%%%%%%%%%%%%%%%%%%%%%%%%%%%%%%%%%%%%%%%%%%%%%%%%%%%%%%%
%%%%%%%%%%%%%%%%%%%%%%%%%%%%%%%%%%%%%%%%%%%%%%%%%%%%%%%%%%%%%%%%%%%%%%%%%%%%%%%%
\section{Usage}

First of all, the package \textsf{childdoc} is \emph{not} a standard
\LaTeXe{} |.sty| style file! Therefore it needs to be invoked in
a non-standard way.

%%%%%%%%%%%%%%%%%%%%%%%%%%%%%%%%%%%%%%%%%%%%%%%%%%%%%%%%%%%%%%%%%%%%%%%%%%%%%%%%
\subsection{Included Files}
\label{sec:include}

%%%%%%%%%%%%%%%%%%%%%%%%%%%%%%%%%%%%%%%%
\DescribeMacro{\childdocmain}
To use the package, add the commands
\begin{center}
\begin{tabular}{l}
|\input{childdoc.def}|\\
|\childdocmain{}|\\
\end{tabular}
\end{center}
at the very top of the main \LaTeX{} file,
in particular \emph{before} the |\documentclass| statement!
The argument of |\childdocmain| should be left empty
(but it must be present).

%%%%%%%%%%%%%%%%%%%%%%%%%%%%%%%%%%%%%%%%
\DescribeMacro{\childdocof}
Furthermore, add the commands
\begin{center}
\begin{tabular}{l}
|\input{childdoc.def}|\\
|\childdocof{|\textit{main}|}|\\
\end{tabular}
\end{center}
at the top of every child file \textit{child}
which is included by |\include{|\textit{child}|}|
from within the main file
(or at least for those files to be compiled individually).
The argument \textit{main} must be the filename of the main file.

There are a couple of
considerations in setting up the main and child documents:

%%%%%%%%%%%%%%%%%%%%%%%%%%%%%%%%%%%%%%%%
\paragraph{Restrictions.}

Please note the following restrictions:
\begin{itemize}
\item
|\childdocmain| must be called with one argument \textit{main}
to ensure compatibility with earlier version of the package.
It must either be empty (|\childdocmain{}|)
or precisely match the filename of the main file in which it is specified.
See \secref{sec:detection} for further information.
\item
The filename \textit{main} must be specified without the |.tex| extension.
\item
The filename \textit{main} is case sensitive
(even in case-insensitive file systems)
due to internal string comparison.
\item
The argument \textit{main} should be fully expanded, it cannot be a macro.
\item
Subdirectories and special characters should be avoided in filenames.
\item
The command |\childdocmain{|\textit{main}|}| must be followed by a whitespace.
It should not be followed immediately by another command
or by a comment mark `|%|'.
This is because the \TeX{} parser reads the token immediately following
the argument of |\childdocmain| and puts it
at the beginning of every child section;
however, a white\-space is ignored.
\end{itemize}

%%%%%%%%%%%%%%%%%%%%%%%%%%%%%%%%%%%%%%%%
\paragraph{Content of Main File.}

It is advisable to place all content in the child files included by |\include|.
Any output contained in the main file will appear in all child documents
unless suppressed manually;
it cannot be suppressed automatically by the |\includeonly| directive
and thus should normally be avoided.
A method to include some content in the main file
by means of conditional processing is described in \secref{sec:conditional}.

%%%%%%%%%%%%%%%%%%%%%%%%%%%%%%%%%%%%%%%%
\paragraph{Page Numbering.}

When only a part of the document is compiled,
the appropriate numbering of pages
(as well as other status parameters)
is determined from the |.aux| files.
The latter contain information from previous passes.
However this information needs to propagate through
all intermediate child documents.
Therefore the page numbering in child documents may well
be inconsistent until the complete document is compiled at least once.

A useful (if unconventional) way to always ensure a consistent
page numbering is to restart the numbering in each child document
and denote the pages by `\textit{child}|.|\textit{page}'
where \textit{child} represents the chapter/section number of the child file.
This can be achieved by the command
|\numberwithin{page}{|\textit{child}|}|
of the \textsf{amsmath} package
where \textit{child} can be |chapter| or |section|
depending on the chosen structuring.
Alternatively, one can modify the macro |\thepage| appropriately
and reset the counter |page| at the start of each child file.

%%%%%%%%%%%%%%%%%%%%%%%%%%%%%%%%%%%%%%%%%%%%%%%%%%%%%%%%%%%%%%%%%%%%%%%%%%%%%%%%
\subsection{Conditional Processing}
\label{sec:conditional}

The package provides a mechanism to compile different versions
of a document. To customise the versions further some conditional processing
can come in handy to distinguish which version is being compiled.
The package provides two macros to describe the compilation context:

%%%%%%%%%%%%%%%%%%%%%%%%%%%%%%%%%%%%%%%%
\DescribeMacro{\ifchilddoc}
The conditional |\ifchilddoc| distinguishes between the compilation of
child documents and the main document:
%
\begin{center}
|\ifchilddoc |\textit{child-code}| |[|\||else |\textit{main-code}]| \||fi|
\end{center}

%%%%%%%%%%%%%%%%%%%%%%%%%%%%%%%%%%%%%%%%
\DescribeMacro{\childdocname}
\DescribeMacro{\childdocjob}
The macro |\childdocname| contains the filename (without extension)
of the main or child file being processed.
Note that |\childdocjob| will always contain the name of the main file.

%%%%%%%%%%%%%%%%%%%%%%%%%%%%%%%%%%%%%%%%
\paragraph{Title Page.}

Conditional processing can be used to include a title or banner page
in the main document when proper precautions are taken.
Importantly, the code in the main file should ensure that the page counter
(as well as other status parameters which are stored in the |.aux| files)
takes the same value after the conditional processing.
Otherwise the page numbers may take divergent values
depending on which part is compiled.

For example, a title page could be declared by:
%
\begin{center}
\begin{tabular}{l}
|\ifchilddoc\||else|\\
|\addtocounter{page}{-1}|\\
\textit{code for title page}\\
|\newpage|\\
|\||fi|
\end{tabular}
\end{center}
%
A banner page for the child documents can be generated by:
%
\begin{center}
\begin{tabular}{l}
|\ifchilddoc|\\
|\addtocounter{page}{-1}|\\
\textit{code for banner page}\\
|\newpage|\\
|\||fi|
\end{tabular}
\end{center}
%
Here one could write a message such as:
\begin{center}
|This is the part \childdocname{} of \childdocjob{}.|
\end{center}

%%%%%%%%%%%%%%%%%%%%%%%%%%%%%%%%%%%%%%%%%%%%%%%%%%%%%%%%%%%%%%%%%%%%%%%%%%%%%%%%
\subsection{Flags}
\label{sec:flags}

The package makes it easy to generate different versions
of the main or child documents.
To this end compilation flags can be defined
and assigned different default values.
They will be particularly useful in conjunction
with the forwarding mechanism described in \secref{sec:forward}.

For example, it may be useful to have a flag |\version|
which can be set to |draft| or |final|.
The document source will contain some conditional code
depending on the value of |\version|.
Suppose further, the flag should default to |final| for the main file
and to |draft| for child files
which is a natural assignment for editing the document.
This is achieved by placing the following code
in the preamble of the main document
(below the |\childdocmain| directive):
%
\begin{center}
\begin{tabular}{l}
|\ifchilddoc|\\
|\providecommand{\version}{draft}|\\
|\||else|\\
|\providecommand{\version}{final}|\\
|\||fi|
\end{tabular}
\end{center}
%
The definition by |\providecommand| makes sure
that previous definitions are not overwritten.
Further statements |\providecommand{\version}{...}|
can thus be added before the above code to override it.

For the main file, one might add a line
(between |\childdocmain| and the above block)
%
\begin{center}
|%\ifchilddoc\||else\providecommand{\version}{draft}\||fi|
\end{center}
%
which can be uncommented to produce a draft version.
Likewise one can add a line to the very top of a child file
(above the |\childdocof{|\textit{main}|}| directive)
%
\begin{center}
|%\providecommand{\version}{final}|
\end{center}
%
which can be uncommented to produce the final version of this child document.

%%%%%%%%%%%%%%%%%%%%%%%%%%%%%%%%%%%%%%%%%%%%%%%%%%%%%%%%%%%%%%%%%%%%%%%%%%%%%%%%
\subsection{Forwarding}
\label{sec:forward}

Different versions of the main or child documents
using compilation flags as described in \secref{sec:flags}
can be (permanently) stored in different files
for convenient compilation, viewing and distribution.
To this end, the package defines a command
to pass on compilation to a different file:

%%%%%%%%%%%%%%%%%%%%%%%%%%%%%%%%%%%%%%%%
\DescribeMacro{\childdocforward}
The command |\childdocforward| redirects processing to
another source file:
%
\begin{center}
\begin{tabular}{l}
|\input{childdoc.def}|\\
|\childdocforward[|\textit{main}|]{|\textit{dest}|}|\\
\end{tabular}
\end{center}
%
The argument \textit{dest} is the destination file
(without extension).
It should be the main file or one of the child files.
Note that further \textsf{childdoc} directives
such as |\childdocof| and |\childdocforward|
in the indicated file will be processed in this form.
The optional argument \textit{main}
passes on directly to the main file \textit{main}
while pretending to compile the child \textit{dest}.
This form behaves as if \textit{dest}
issues |\childdocof{|\textit{main}|}| right away,
and no further \textsf{childdoc} directives will be processed.

%%%%%%%%%%%%%%%%%%%%%%%%%%%%%%%%%%%%%%%%
\DescribeMacro{\...prefix}
In the alternative form |\childdocforwardprefix|,
%
\begin{center}
\begin{tabular}{l}
|\input{childdoc.def}|\\
|\childdocforwardprefix[|\textit{main}|]{|\textit{prefix}|}{|\textit{dest}|}|
\end{tabular}
\end{center}
%
the destination file is determined by a pattern
depending on the current file:
To make this work, the current file must be called
`{\textit{prefix}\hspace{0.2em}\textit{suffix}}'
with \textit{prefix} matching precisely the argument.
Processing is then passed on to the file
`{\textit{dest}\hspace{0.2em}\textit{suffix}}'.
Surely, the same effect is achieved by
directly specifying the
argument `{\textit{dest}\hspace{0.2em}\textit{suffix}}'
in the first form.
However, that requires to set up a different file
for each child. With the alternative form of the command
all these files can have exactly the same content
which simplifies setting them up and maintaining them.

For example, the following file |draft.tex|
with a compilation flag |\version| as described in \secref{sec:flags}
compiles the main document as a draft:
%
\begin{center}
\begin{tabular}{l}
|\def\version{draft}|\\
|\input{childdoc.def}|\\
|\childdocforward{|\textit{main}|}|
\end{tabular}
\end{center}
%
Likewise, the following files |final|\textit{nn}|.tex|
compile the final version of the child document
|child|\textit{nn}|.tex|:
%
\begin{center}
\begin{tabular}{l}
|\def\version{final}|\\
|\input{childdoc.def}|\\
|\childdocforwardprefix{final}{child}|
\end{tabular}
\end{center}
%

Note that when several versions of a main file and/or of each child file
are to be generated, it may be convenient to set up a |Makefile| or
shell script to automatise the process.

%%%%%%%%%%%%%%%%%%%%%%%%%%%%%%%%%%%%%%%%%%%%%%%%%%%%%%%%%%%%%%%%%%%%%%%%%%%%%%%%
\subsection{Command Line Processing}
\label{sec:commandline}

The effect of redirection files can also be achieved by invoking
the \LaTeX{} compiler with a more elaborate command line.
Most conveniently this should be done as part
of a shell script or a |Makefile|.

When using \textsf{childdoc} in the main file, the following
command lines effectively perform a redirection
(note that depending on the shell being used,
backslashes may have to be doubled: `|\|' $\to$ `|\\|'):
%
\begin{center}
|... -jobname "|\textit{target}|" |\\|"|[\textit{flags}]%
|\input{childdoc.def}\childdocforward[|\textit{main}|]{|\textit{dest}|}"|
\end{center}
%
Here \textit{target} is the name of the output file,
\textit{main} is the name of the main file
and \textit{dest} is the name of the main or child file to be processed
(all filenames without extensions).
The optional argument \textit{main} can be omitted
if \textit{main} matches \textit{dest}.
Optionally, compilation \textit{flags} can be defined via |\def| commands.
This command line makes the \TeX{} engine believe
it is compiling the file \textit{target}
whose content is specified as the latter parameter.
The provided code then forwards the processing to
\textit{main} or \textit{dest} as described in \secref{sec:forward}.

%%%%%%%%%%%%%%%%%%%%%%%%%%%%%%%%%%%%%%%%%%%%%%%%%%%%%%%%%%%%%%%%%%%%%%%%%%%%%%%%
\subsection{Include by Input}
\label{sec:input}

Including child documents by |\include| has some restrictions by design.
Most notably, the content of a child document always occupies
its own set of pages; pages cannot be shared between child documents.
Usually, this behaviour makes perfect sense
because each child document contain an essential part of the document.
However, in some situations it may be desirable to compose
a document from a collection of parts
without having mandatory page breaks between then.
For this case, the package
provides a mechanism to include parts
by |\input| which can also be processed individually.
However, by construction this mechanism
requires manual handling of the content to be output.

%%%%%%%%%%%%%%%%%%%%%%%%%%%%%%%%%%%%%%%%
\DescribeMacro{\ifchilddocmanual}
The main file should be prepared as usual, see \secref{sec:include}.
However, the document body must make a distinction
between processing of an individual part and of the main document, e.g.:
%
\begin{center}
\begin{tabular}{l}
|\ifchilddocmanual|\\
|\input{\childdocname}|\\
|\||else|\\
\textit{document body with }|\input{|\textit{part}|}|\\
|\||fi|
\end{tabular}
\end{center}
%
The conditional |\ifchilddocmanual| is true whenever
a part to be included by |\input| is being compiled,
and the name of the part is stored in |\childdocname|.

%%%%%%%%%%%%%%%%%%%%%%%%%%%%%%%%%%%%%%%%
\DescribeMacro{\childdocby}
Each part to be included by |\input| should start with:
%
\begin{center}
\begin{tabular}{l}
|\input{childdoc.def}|\\
|\childdocby{|\textit{main}|}|\\
\end{tabular}
\end{center}
%
The directive |\childdocby| is similar to |\childdocof|
described in \secref{sec:include},
but the subsequent selection of content must be done manually.
To that end, both |\ifchilddoc| and |\ifchilddocmanual|
will be true upon processing of a part,
and the name of the part is stored in |\childdocname|.
Note that |\jobname| will be set to the filename of the current part
so that each part receives an individual |.aux| file
that does not interfere with the |.aux| file(s) of the main document.
This behaviour can be altered by the alternative form
|\childdocby[*]{|\textit{main}|}| (with a non-empty optional argument)
which uses the |.aux| file of the main document
by setting |\jobname| to \textit{main}.

%%%%%%%%%%%%%%%%%%%%%%%%%%%%%%%%%%%%%%%%%%%%%%%%%%%%%%%%%%%%%%%%%%%%%%%%%%%%%%%%
\subsection{Driver Development}
\label{sec:driver}

The \textsf{childdoc} mechanism can also be use for the development
of definition files such as \LaTeX{} styles or classes.
This case differs from the above setup with multiple parts
included by |\include| in that no |\includeonly| should be invoked.
This can be achieved by starting the include file
(before |\ProvidesPackage|) with:
%
\begin{center}
\begin{tabular}{l}
|\input{childdoc.def}|\\
|\childdocforward{|\textit{main}|}|\\
\end{tabular}
\end{center}
%
or alternatively with:
%
\begin{center}
\begin{tabular}{l}
|\input{childdoc.def}|\\
|\childdocby{|\textit{main}|}|\\
\end{tabular}
\end{center}
%
Both forms have slightly different effects as described above.
The main file is prepared as usual, see \secref{sec:include}.

%%%%%%%%%%%%%%%%%%%%%%%%%%%%%%%%%%%%%%%%%%%%%%%%%%%%%%%%%%%%%%%%%%%%%%%%%%%%%%%%
\subsection{Legacy Detection}
\label{sec:detection}

The directive |\childdocmain| in the main file can detect
whether the complete document or merely a child is to be compiled
even without using the directive |\childdocof|.
This method is deprecated because it is less robust
and there is no compelling reason to use it;
it is merely provided for backward compatibility
and it may be removed in future versions.

If the detection mechanism is to be used,
it is mandatory to correctly specify
the filename of the main file as the argument of |\childdocmain|:
%
\begin{center}
\begin{tabular}{l}
|\input{childdoc.def}|\\
|\childdocmain{|\textit{main}|}|\\
\end{tabular}
\end{center}
%
If |\jobname| does not match the argument \textit{main} of |\childdocmain|,
it is assumed that |\jobname| points to the child file to be compiled.
When using |\childdocmain| with the main file specified as argument,
it suffices to start a child file
with just |\input{|\textit{main}|}|
without loading of the package and using |\childdocof|.
If instead all processing is done
with the appropriate \textsf{childdoc} directives,
the argument of \textit{main} of |\childdocmain| can be empty.

An alternative version of the command line processing described
in \secref{sec:commandline} using the detection mechanism reads:
%
\begin{center}
|... -jobname "|\textit{target}|" "|[\textit{flags}]%
[|\def\jobname{|\textit{dest}|}|]|\input{|\textit{main}|}"|
\end{center}

%%%%%%%%%%%%%%%%%%%%%%%%%%%%%%%%%%%%%%%%%%%%%%%%%%%%%%%%%%%%%%%%%%%%%%%%%%%%%%%%
\subsection{Manual Code}
\label{sec:manual}

In case one cannot be certain whether the definitions file |childdoc.def|
is installed on the target \TeX{} distribution
and one prefers not to ship it,
it is conceivable to paste a few relevant commands into the sources.

To that end, drop all statements |\input{childdoc.def}|
and perform the replacements as outlined below.
Instead of |\childdocmain{|\textit{main}|}| add the following code
to the top of the main file:
%
\begin{center}
\begin{tabular}{l}
|\||ifdefined\childdocname\endinput\||fi\newif\ifchilddoc|\\
|\edef\childdocname{\scantokens\expandafter{\jobname\noexpand}}|\\
|\def\childdocmain{|\textit{main}|}\||ifx\childdocmain\childdocname\||else|\\
|\childdoctrue\includeonly{\childdocname}\let\jobname\childdocmain\||fi|\\
\end{tabular}
\end{center}
%
Instead of |\childdocof{|\textit{main}|}| just include the main file
at the top of each child file:
%
\begin{center}
|\input{|\textit{main}|}|
\end{center}
%
A simple redirection |\childdocforward{|\textit{dest}|}| is achieved by:
%
\begin{center}
|\def\jobname{|\textit{dest}|}\input{\jobname}|
\end{center}
%
The redirection with prefix
|\childdocforwardprefix[|\textit{prefix}|]{|\textit{dest}|}|
is accomplished by:
%
\begin{center}
\begin{tabular}{l}
|{\edef\jobname{\scantokens\expandafter{\jobname\noexpand}}|\\
|\def\redirectjob |\textit{prefix}|#1~~~{\gdef\jobname{|\textit{dest}|#1}}|\\
|\expandafter\redirectjob\jobname~~~}\input{\jobname}|
\end{tabular}
\end{center}

In an alternative approach,
child documents can be compiled by a specific command line
without additional code or specific definitions:
%
\begin{center}
|... -jobname "|\textit{target}|" "|[\textit{flags}]%
|\includeonly{|\textit{dest}|}\input{|\textit{main}|}"|
\end{center}
%

%%%%%%%%%%%%%%%%%%%%%%%%%%%%%%%%%%%%%%%%%%%%%%%%%%%%%%%%%%%%%%%%%%%%%%%%%%%%%%%%
%%%%%%%%%%%%%%%%%%%%%%%%%%%%%%%%%%%%%%%%%%%%%%%%%%%%%%%%%%%%%%%%%%%%%%%%%%%%%%%%
\section{Information}

%%%%%%%%%%%%%%%%%%%%%%%%%%%%%%%%%%%%%%%%%%%%%%%%%%%%%%%%%%%%%%%%%%%%%%%%%%%%%%%%
\subsection{Copyright}

Copyright \copyright{} 2017--2018 Niklas Beisert

This work may be distributed and/or modified under the
conditions of the \LaTeX{} Project Public License, either version 1.3
of this license or (at your option) any later version.
The latest version of this license is in
  \url{http://www.latex-project.org/lppl.txt}
and version 1.3 or later is part of all distributions of \LaTeX{}
version 2005/12/01 or later.

This work has the LPPL maintenance status `maintained'.

The Current Maintainer of this work is Niklas Beisert.

This work consists of the files |README.txt|, |childdoc.ins| and |childdoc.dtx|
as well as the derived files |childdoc.def|, |cdocsamp.tex|
with |cdocsch1.tex|, |cdocsch2.tex|, |cdocspt3.tex|, |cdocspt4.tex|,
|cdocsdrf.tex|, |cdocsfn1.tex|, |cdocsfn2.tex|
as well as |childdoc.pdf|.

%%%%%%%%%%%%%%%%%%%%%%%%%%%%%%%%%%%%%%%%%%%%%%%%%%%%%%%%%%%%%%%%%%%%%%%%%%%%%%%%
\subsection{Files and Installation}

The package consists of the files:
%
\begin{center}
\begin{tabular}{ll}
    |README.txt|   & readme file \\
    |childdoc.ins| & installation file \\
    |childdoc.dtx| & source file \\
    |childdoc.def| & definition file \\
    |cdocsamp.tex| & sample main file \\
    |cdocsch1.tex| & sample include file \\
    |cdocsch2.tex| & sample include file \\
    |cdocspt3.tex| & sample part file \\
    |cdocspt4.tex| & sample part file \\
    |cdocsdrf.tex| & sample redirection file \\
    |cdocsfn1.tex| & sample redirection file \\
    |cdocsfn2.tex| & sample redirection file \\
    |childdoc.pdf| & manual
\end{tabular}
\end{center}
%
The distribution consists of the files
|README.txt|, |childdoc.ins| and |childdoc.dtx|.
%
\begin{itemize}
\item
Run (pdf)\LaTeX{} on |childdoc.dtx|
to compile the manual |childdoc.pdf| (this file).
\item
Run \LaTeX{} on |childdoc.ins| to create the definitions file |childdoc.def|
and the sample |cdocsamp.tex| with include files
|cdocsch1.tex|, |cdocsch2.tex|, |cdocspt3.tex|, |cdocspt4.tex|,
|cdocsdrf.tex|, |cdocsfn1.tex|, |cdocsfn2.tex|.
Then copy the file |childdoc.def| to an appropriate directory of your \LaTeX{}
distribution, e.g.\ \textit{texmf-root}|/tex/latex/childdoc|.
\end{itemize}

%%%%%%%%%%%%%%%%%%%%%%%%%%%%%%%%%%%%%%%%%%%%%%%%%%%%%%%%%%%%%%%%%%%%%%%%%%%%%%%%
\subsection{Related CTAN Packages}

There are several other packages which offer a similar functionality:
%
\begin{itemize}
\item
The packages
\href{http://ctan.org/pkg/docmute}{\textsf{docmute}},
\href{http://ctan.org/pkg/includex}{\textsf{includex}} and
\href{http://ctan.org/pkg/standalone}{\textsf{standalone}}
provide commands to include only the document body of
a child file thus allowing both files to be compiled individually.
\item
The packages \href{http://ctan.org/pkg/subdocs}{\textsf{subdocs}}
and \href{http://ctan.org/pkg/subfiles}{\textsf{subfiles}}
provide structures in which the main and child documents can be
encapsulated and allowing them to be compiled individually.
The inclusion mechanism is different from the conventional |\include|.
\item
The package \href{http://ctan.org/pkg/combine}{\textsf{combine}}
is an elaborate solution to combine several documents into one.
\end{itemize}
%
See also the CTAN topic \href{http://ctan.org/topic/subdocs}{\textsf{subdocs}}
for further related packages.
The present package differs from the above solutions in that
a document structure constructed with the conventional |\include| mechanism
just needs two extra commands at the top of every file
such that all constituent files can be compiled individually.

%%%%%%%%%%%%%%%%%%%%%%%%%%%%%%%%%%%%%%%%%%%%%%%%%%%%%%%%%%%%%%%%%%%%%%%%%%%%%%%%
%\subsection{Feature Suggestions}
%
%The following is a list of features which may be useful for future
%versions of this package:
%%
%\begin{itemize}
%\item
%\ldots
%\end{itemize}

%%%%%%%%%%%%%%%%%%%%%%%%%%%%%%%%%%%%%%%%%%%%%%%%%%%%%%%%%%%%%%%%%%%%%%%%%%%%%%%%
\subsection{Revision History}

%%%%%%%%%%%%%%%%%%%%%%%%%%%%%%%%%%%%%%%%
\paragraph{v2.0:} 2018/12/30

\begin{itemize}
\item
immediate forward processing
\item
added |\childdocby| mechanism
\item
manual restructured
\end{itemize}

%%%%%%%%%%%%%%%%%%%%%%%%%%%%%%%%%%%%%%%%
\paragraph{v1.6:} 2018/01/17

\begin{itemize}
\item
application for development of include files
\item
corrections to manual
\end{itemize}

%%%%%%%%%%%%%%%%%%%%%%%%%%%%%%%%%%%%%%%%
\paragraph{v1.5:} 2017/05/21

\begin{itemize}
\item
more complete structuring introduced
\item
|\childdocof| introduced
\item
|\childdoc| renamed to |\childdocmain|
\item
|\childredirect| renamed to |\childdocforward| and |\childdocforwardprefix|
and functionality expanded
\end{itemize}

%%%%%%%%%%%%%%%%%%%%%%%%%%%%%%%%%%%%%%%%
\paragraph{v1.0:} 2017/04/27

\begin{itemize}
\item
manual and install package
\item
first version published on CTAN
\end{itemize}

%%%%%%%%%%%%%%%%%%%%%%%%%%%%%%%%%%%%%%%%
\paragraph{v0.6:} 2017/04/26

\begin{itemize}
\item
redirection mechanism added
\end{itemize}

%%%%%%%%%%%%%%%%%%%%%%%%%%%%%%%%%%%%%%%%
\paragraph{v0.5:} 2017/04/26

\begin{itemize}
\item
functionality in definition file
\end{itemize}


%%%%%%%%%%%%%%%%%%%%%%%%%%%%%%%%%%%%%%%%%%%%%%%%%%%%%%%%%%%%%%%%%%%%%%%%%%%%%%%%
%%%%%%%%%%%%%%%%%%%%%%%%%%%%%%%%%%%%%%%%%%%%%%%%%%%%%%%%%%%%%%%%%%%%%%%%%%%%%%%%
%%%%%%%%%%%%%%%%%%%%%%%%%%%%%%%%%%%%%%%%%%%%%%%%%%%%%%%%%%%%%%%%%%%%%%%%%%%%%%%%
\appendix

\settowidth\MacroIndent{\rmfamily\scriptsize 000\ }

 \DocInput{childdoc.dtx}

\end{document}
%</driver>
% \fi
%
% %%%%%%%%%%%%%%%%%%%%%%%%%%%%%%%%%%%%%%%%%%%%%%%%%%%%%%%%%%%%%%%%%%%%%%%%%%%%%%
% %%%%%%%%%%%%%%%%%%%%%%%%%%%%%%%%%%%%%%%%%%%%%%%%%%%%%%%%%%%%%%%%%%%%%%%%%%%%%%
% \section{Sample}
%\iffalse
%<*samplemain>
%\fi
%
% The following presents a sample document
% with two chapters, two parts, a title page,
% a compile flag as well as three forwarding files to set the flag.
% It consists of eight |.tex| files:
% \begin{center}
% \begin{tabular}{ll}
% |cdocsamp.tex|&main file\\
% |cdocsch1.tex|&include file for chapter 1\\
% |cdocsch2.tex|&include file for chapter 2\\
% |cdocspt3.tex|&include file for part 3\\
% |cdocspt4.tex|&include file for part 4\\
% |cdocsdrf.tex|&forwarding file for main file in draft mode\\
% |cdocsfi1.tex|&forwarding file for final version of chapter 1\\
% |cdocsfi2.tex|&forwarding file for final version of chapter 2\\
% \end{tabular}
% \end{center}
% Each of the eight files can be compiled directly by the \LaTeX{} compiler.
%
% %%%%%%%%%%%%%%%%%%%%%%%%%%%%%%%%%%%%%%
% \paragraph{Main File.}
%
% The main file is called |cdocsamp.tex|.
%
% Load the \textsf{childdoc} definitions and
% declare the filename for the main document:
%    \begin{macrocode}
\input{childdoc.def}
\childdocmain{}
%    \end{macrocode}

% Optional override for |\version| flag:
%    \begin{macrocode}
%%\ifchilddoc\else\providecommand{\version}{draft}\fi
%    \end{macrocode}

% Define the default values for the |\version| flag
% (|final| for the main file and |draft| for childs):
%    \begin{macrocode}
\ifchilddoc
\providecommand{\version}{draft}
\else
\providecommand{\version}{final}
\fi
%    \end{macrocode}

% Load the standard document class:
%    \begin{macrocode}
\documentclass[12pt]{article}
%    \end{macrocode}

% Start the document body:
%    \begin{macrocode}
\begin{document}
%    \end{macrocode}

% Declare a title page.
% Print title, part of document being processed and version flag:
%    \begin{macrocode}
\addtocounter{page}{-1}
\begin{center}
{\LARGE\bfseries{}childdoc example\par}
\vspace{1cm}
\ifchilddoc
\ifchilddocmanual part\else chapter\fi:
`\childdocname' of `\childdocjob'\par
\else
main document: `\childdocjob'\par
\fi
version: \version\par
\end{center}
\newpage
%    \end{macrocode}

% Manually include selected file,
% otherwise process as usual:
%    \begin{macrocode}
\ifchilddocmanual
\section*{part `\childdocname'}
\input{\childdocname}
\else
%    \end{macrocode}

% Include the two chapters:
%    \begin{macrocode}
\include{cdocsch1}
\include{cdocsch2}
%    \end{macrocode}

% Include the two parts unless only chapters should be displayed:
%    \begin{macrocode}
\ifchilddoc\else
\section{part three}
\input{cdocspt3}
\section{part four}
\input{cdocspt4}
\fi
%    \end{macrocode}

% Process as usual until here:
%    \begin{macrocode}
\fi
%    \end{macrocode}

% End of document body:
%    \begin{macrocode}
\end{document}
%    \end{macrocode}
%\iffalse
%</samplemain>
%\fi
%
% %%%%%%%%%%%%%%%%%%%%%%%%%%%%%%%%%%%%%%
% \paragraph{Chapter Include Files.}
%
% The include files are called |cdocsch1.tex| and |cdocsch2.tex|.
%
%\iffalse
%<*samplechap1|samplechap2>
%\fi

% Optional override for |\version| flag:
%    \begin{macrocode}
%%\providecommand{\version}{final}
%    \end{macrocode}

% Include the main document:
%    \begin{macrocode}
\input{childdoc.def}
\childdocof{cdocsamp}
%    \end{macrocode}

%\iffalse
%</samplechap1|samplechap2>
%\fi
%
%\iffalse
%<*samplechap1>
%\fi
% Some text for chapter 1:
%    \begin{macrocode}
\section{one}
some text in chapter one
%    \end{macrocode}

%\iffalse
%</samplechap1>
%\fi
% Some text for chapter 2:
%\iffalse
%<*samplechap2>
%\fi
%    \begin{macrocode}
\section{two}
more text in chapter two
%    \end{macrocode}

%\iffalse
%</samplechap2>
%\fi
%
% %%%%%%%%%%%%%%%%%%%%%%%%%%%%%%%%%%%%%%
% \paragraph{Part Include Files.}
%
% The include files are called |cdocspt3.tex| and |cdocspt4.tex|.
%
%\iffalse
%<*samplepart3|samplepart4>
%\fi

% Optional override for |\version| flag:
%    \begin{macrocode}
%%\providecommand{\version}{final}
%    \end{macrocode}

% Include the main document:
%    \begin{macrocode}
\input{childdoc.def}
\childdocby{cdocsamp}
%    \end{macrocode}

%\iffalse
%</samplepart3|samplepart4>
%\fi
%
%\iffalse
%<*samplepart3>
%\fi
% Some text for part 3:
%    \begin{macrocode}
some text in part three
%    \end{macrocode}

%\iffalse
%</samplepart3>
%\fi
% Some text for part 4:
%\iffalse
%<*samplepart4>
%\fi
%    \begin{macrocode}
more text in part four
%    \end{macrocode}

%\iffalse
%</samplepart4>
%\fi
%
% %%%%%%%%%%%%%%%%%%%%%%%%%%%%%%%%%%%%%%
% \paragraph{Forwarding for a Complete Draft.}
%
% The following forwarding file |cdocsdrf.tex|
% compiles the main document in draft mode:
%\iffalse
%<*sampledraft>
%\fi
%    \begin{macrocode}
\def\version{draft}
\input{childdoc.def}
\childdocforward{cdocsamp}
%    \end{macrocode}

%\iffalse
%</sampledraft>
%\fi
%
% %%%%%%%%%%%%%%%%%%%%%%%%%%%%%%%%%%%%%%
% \paragraph{Forwarding for Final Version of the Chapters.}
%
% The following forwarding files |cdocsfn1.tex| and |cdocsfn2.tex|
% (with identical content)
% compile the final versions of the child documents
% |cdocsch1.tex| and |cdocsch2.tex|, respectively:
%\iffalse
%<*samplefinal>
%\fi
%    \begin{macrocode}
\def\version{final}
\input{childdoc.def}
\childdocforwardprefix[cdocsamp]{cdocsfn}{cdocsch}
%    \end{macrocode}

%\iffalse
%</samplefinal>
%\fi
%
% %%%%%%%%%%%%%%%%%%%%%%%%%%%%%%%%%%%%%%
% \paragraph{Command Line Processing.}
%
% The following three command lines generate the output files
% |cdocscld|, |cdocscl1| and |cdocscl2|
% which should be identical to
% |cdocsdrf|, |cdocsch1| and |cdocsfn2|, respectively:
% \begin{center}
% \begin{tabular}{l}
% |latex -jobname cdocscld \|\\
% |  "\def\version{draft}\input{childdoc.def}\childdocforward{cdocsamp}"|\\
% |latex -jobname cdocscl1 \|\\
% |  "\input{childdoc.def}\childdocforward[cdocsamp]{cdocsch1}"|\\
% |latex -jobname cdocscl2 \|\\
% |  "\def\version{final}\input{childdoc.def}\childdocforward{cdocsch2}"|
% \end{tabular}
% \end{center}
% Note that the trailing backslash on each first line
% merely continues the input to the second line
% (for convenient cut ant paste).
% Furthermore, the command |latex| can be replaced by any
% of its alternative versions such as |pdflatex|.
%
% %%%%%%%%%%%%%%%%%%%%%%%%%%%%%%%%%%%%%%%%%%%%%%%%%%%%%%%%%%%%%%%%%%%%%%%%%%%%%%
% %%%%%%%%%%%%%%%%%%%%%%%%%%%%%%%%%%%%%%%%%%%%%%%%%%%%%%%%%%%%%%%%%%%%%%%%%%%%%%
% \section{Implementation}
%\iffalse
%<*package>
%\fi
%
% This section describes the definitions file |childdoc.def|.

% The definitions cannot be loaded using |\usepackage| or |\RequirePackage|
% which has a mechanism to prevent loading a style file more than once.
% When loading the definitions by means of |\input|
% multiple instances have to be prevented manually:
%\iffalse
%This code needs to be before the `\ProvidesFile' directive
%which is defined at the beginning of this file.
%Therefore it is also placed there and commented out here.
%</package>
%<*discard>
%\fi
%    \begin{macrocode}
\ifdefined\childdocmain\endinput\fi
%    \end{macrocode}
%\iffalse
%</discard>
%<*package>
%\fi
%
% \macro{\ifchilddoc}
% \macro{\ifchilddocmanual}
% The conditional |\ifchilddoc| tells whether a
% child (true) or main (false) document is being compiled.
% The conditional |\ifchilddocmanual| tells whether
% the |\includeonly| mechanism is used (false) or
% the selection of child files must be performed manually (true).
% The definitions initialise to false:
%    \begin{macrocode}
\newif\ifchilddoc
\newif\ifchilddocmanual
%    \end{macrocode}

% \macro{\childdocname}
% \macro{\childdocjob}
% The macro |\childdocname| stores the name of the main document
% to be compiled. The macro |\childdocjob| stores the name of
% the document on which the \LaTeX{} compiler was originally invoked.
% The content of |\jobname| cannot be compared
% to filenames specified in the source due to different catcodes.
% The following code rescans |\jobname|, stores the result
% in |\childdocname| and saves a copy in |\childdocjob|:
%    \begin{macrocode}
\edef\childdocname{\scantokens\expandafter{\jobname\noexpand}}
\let\childdocjob\childdocname
%    \end{macrocode}

% \macro{\childdocdisable}
% The macro |\childdocdisable| prevents the main file
% from being processed more than once.
% At this stage, the main document command |\childdocmain|
% is assumed to be called once again where it should do nothing.
% Any subsequent call to it should prevent
% a secondary processing of the main document
% It overwrites the forwarding commands
% |\childdocof| and |\childdocforward|
% with empty macros to prevent further inclusions of the main document:
%    \begin{macrocode}
\newcommand{\childdocdisable}
{
  \renewcommand{\childdocmain}[1]{\renewcommand{\childdocmain}[1]{\endinput}}
  \renewcommand{\childdocof}[1]{}
  \renewcommand{\childdocby}[2][]{}
  \renewcommand{\childdocforward}[2][]{}
  \renewcommand{\childdocdisable}{}
}
%    \end{macrocode}

% \macro{\childdocmain}
% The macro |\childdocmain| is to be called at the top of the main file
% with nothing or the main filename (without extension) as argument.
% First, it breaks loops.
% If the argument is not empty and does not match |\childdocname|
% (which is set by the first inclusion of |childdoc.def|),
% |\ifchilddoc| is set to true, |\includeonly| is applied to the child file
% and |\jobname| is set to the main file
% (for proper handling of |.aux| files):
%    \begin{macrocode}
\newcommand{\childdocmain}[1]
{
  \childdocdisable\childdocmain{}
  \if?#1?\else
    \begingroup
      \def\childdoctmp{#1}
      \ifx\childdoctmp\childdocname
        \def\childdoctmp{}
      \else
        \def\childdoctmp
        {
          \childdoctrue
          \includeonly{\childdocname}
          \def\childdocjob{#1}
          \def\jobname{#1}
        }
      \fi
      \expandafter
    \endgroup
    \childdoctmp
  \fi
}
%    \end{macrocode}

% \macro{\childdocof}
% The command |\childdocof| redirects
% compilation to the main file |#1|.
%    \begin{macrocode}
\newcommand{\childdocof}[1]
{
  \childdocdisable
  \childdoctrue
  \includeonly{\childdocname}
  \def\jobname{#1}
  \def\childdocjob{#1}
  \input{#1}
}
%    \end{macrocode}

% \macro{\childdocby}
% The command |\childdocby| ....
%    \begin{macrocode}
\newcommand{\childdocby}[2][]
{
  \childdocdisable
  \childdoctrue
  \childdocmanualtrue
  \if?#1?\else
    \def\jobname{#2}
  \fi
  \def\childdocjob{#2}
  \input{#2}
  \endinput
}
%    \end{macrocode}

% \macro{\childdocforward}
% The command |\childdocforward| redirects
% compilation to the main file or
% (if the optional argument is given) a child file.
% Parameters are set as if the main file
% or a child file starting with |\childdocof| was compiled.
% Then compilation is handed over to the main file:
%    \begin{macrocode}
\newcommand{\childdocforward}[2][]
{
  \begingroup
    \if?#1?
      \def\childdoctmp
      {
        \def\childdocname{#2}
        \def\childdocjob{#2}
        \def\jobname{#2}
        \input{#2}
        \endinput
      }
    \else
      \def\childdoctmp
      {
        \childdocdisable
        \def\childdocname{#2}
        \childdoctrue
        \includeonly{#2}
        \def\childdocjob{#1}
        \def\jobname{#1}
        \input{#1}
        \endinput
      }
    \fi
    \expandafter
  \endgroup
  \childdoctmp
}
%    \end{macrocode}

% \macro{\childdocforwardprefix}
% The command |\childdocforwardprefix| redirects
% compilation to the main or a child file by means of a pattern.
% The prefix |#1| in the current filename is replaced by |#2|
% and the suffix of the current filename is kept
% (it is assumed that the filename does not contain the substring `|~~~|'
% which is used as a delimiter).
% Compilation is handed over to the new file by |\childdocforward|:
%    \begin{macrocode}
\newcommand{\childdocforwardprefix}[3][]
{
  \begingroup
    \def\childdocextract #2##1~~~{\def\childdoctmp{\childdocforward[#1]{#3##1}}}
    \expandafter\childdocextract\childdocname~~~
    \expandafter
  \endgroup
  \childdoctmp
}
%    \end{macrocode}

% \macro{\childdoc}
% The deprecated macro |\childdoc| is a legacy version of |\childdocmain|:
%    \begin{macrocode}
\newcommand{\childdoc}{\childdocmain}
%    \end{macrocode}

% \macro{\childdocredirect}
% The deprecated macro |\childdocredirect| is a legacy version
% of |\childdocforward| and |\childdocforwardprefix|:
%    \begin{macrocode}
\newcommand{\childdocredirect}[2][]
{
  \begingroup
    \if?#1?
      \def\childdoctmp{\childdocforward{#2}}
    \else
      \def\childdoctmp{\childdocforwardprefix{#1}{#2}}
    \fi
    \expandafter
  \endgroup
  \childdoctmp
}
%    \end{macrocode}

%\iffalse
%</package>
%\fi
%
\endinput
|\\
|\childdocforward{|\textit{main}|}|
\end{tabular}
\end{center}
%
Likewise, the following files |final|\textit{nn}|.tex|
compile the final version of the child document
|child|\textit{nn}|.tex|:
%
\begin{center}
\begin{tabular}{l}
|\def\version{final}|\\
|% \iffalse
%
% childdoc.dtx Copyright (C) 2017-2018 Niklas Beisert
%
% This work may be distributed and/or modified under the
% conditions of the LaTeX Project Public License, either version 1.3
% of this license or (at your option) any later version.
% The latest version of this license is in
%   http://www.latex-project.org/lppl.txt
% and version 1.3 or later is part of all distributions of LaTeX
% version 2005/12/01 or later.
%
% This work has the LPPL maintenance status `maintained'.
%
% The Current Maintainer of this work is Niklas Beisert.
%
% This work consists of the files childdoc.dtx and childdoc.ins
% and the derived files childdoc.def and cdocsamp.tex with
% cdocsch1.tex, cdocsch2.tex, cdocsdrf.tex, cdocsfn1.tex, cdocsfn2.tex.
%
%<package>\ifdefined\childdocmain\endinput\fi
%<package>\ProvidesFile{childdoc.def}[2018/12/30 v2.0 child document driver]
%<samplemain>\ProvidesFile{cdocsamp.tex}[2018/12/30 v2.0 sample for childdoc]
%<*driver>
%\ProvidesFile{childdoc.drv}[2018/12/30 v2.0 childdoc reference manual file]
\PassOptionsToClass{10pt,a4paper}{article}
\documentclass{ltxdoc}

\usepackage[margin=35mm]{geometry}
\usepackage{hyperref}
\usepackage{hyperxmp}
\usepackage[usenames]{color}

\hypersetup{colorlinks=true}
\hypersetup{pdfstartview=FitH}
\hypersetup{pdfpagemode=UseNone}
\hypersetup{pdfsource={}}
\hypersetup{pdflang={en-UK}}
\hypersetup{pdfcopyright={Copyright 2017-2018 Niklas Beisert.
  This work may be distributed and/or modified under the
  conditions of the LaTeX Project Public License, either version 1.3
  of this license or (at your option) any later version.}}
\hypersetup{pdflicenseurl={http://www.latex-project.org/lppl.txt}}
\hypersetup{pdfcontactaddress={ETH Zurich, ITP, HIT K,
  Wolfgang-Pauli-Strasse 27}}
\hypersetup{pdfcontactpostcode={8093}}
\hypersetup{pdfcontactcity={Zurich}}
\hypersetup{pdfcontactcountry={Switzerland}}
\hypersetup{pdfcontactemail={nbeisert@itp.phys.ethz.ch}}
\hypersetup{pdfcontacturl={http://people.phys.ethz.ch/\xmptilde nbeisert/}}

\newcommand{\secref}[1]{\hyperref[#1]{section \ref*{#1}}}

\parskip1ex
\parindent0pt
\let\olditemize\itemize
\def\itemize{\olditemize\parskip0pt}

\begin{document}

\title{The \textsf{childdoc} Package}
\hypersetup{pdftitle={The childdoc Package}}
\author{Niklas Beisert\\[2ex]
  Institut f\"ur Theoretische Physik\\
  Eidgen\"ossische Technische Hochschule Z\"urich\\
  Wolfgang-Pauli-Strasse 27, 8093 Z\"urich, Switzerland\\[1ex]
  \href{mailto:nbeisert@itp.phys.ethz.ch}
  {\texttt{nbeisert@itp.phys.ethz.ch}}}
\hypersetup{pdfauthor={Niklas Beisert}}
\hypersetup{pdfsubject={Manual for the LaTeX2e Package childdoc}}
\date{30 December 2018, \textsf{v2.0}}
\maketitle

\begin{abstract}\noindent
\textsf{childdoc} is a \LaTeXe{} package
that enables the direct compilation
of document sections included by |\include|
to individual files.
\end{abstract}

\begingroup
\parskip0ex
\tableofcontents
\endgroup

%%%%%%%%%%%%%%%%%%%%%%%%%%%%%%%%%%%%%%%%%%%%%%%%%%%%%%%%%%%%%%%%%%%%%%%%%%%%%%%%
%%%%%%%%%%%%%%%%%%%%%%%%%%%%%%%%%%%%%%%%%%%%%%%%%%%%%%%%%%%%%%%%%%%%%%%%%%%%%%%%
\section{Introduction}

\LaTeX{} provides a mechanism to structure a large document (such as a book)
into a main file and several child files (containing the chapters)
using the |\include| command.
This mechanism is beneficial for documents
which span hundreds of pages in order to
make the source file(s) more manageable.
Moreover, compilation can be restricted to
selected child files by means of the |\includeonly| command.
The latter feature can be used to reduce the compilation time while editing
(this was significantly more useful in the earlier days of \LaTeX{})
or to generate a smaller document which is easier to navigate.
Another application of |\includeonly| is to generate
documents consisting of selected parts of the complete document.

However, there are a few drawbacks of the plain |\include| mechanism:
\begin{itemize}
\item
The child files cannot be compiled on their own,
they can only be compiled via the main file.
A naive editing environment
(such as a text editor with an option
to have the current file processed by \LaTeX)
may require one to switch to the main file before compiling;
attempting to compile the child file produces errors.
\item
The main file must be modified (each time)
to adjust the |\includeonly| command
to the present needs. This easily leaves the main file in a messy state.
\item
The generated document will always carry the filename
of the main document. This is inconvenient if
several child files are to be compiled and
to be kept for distribution.
\end{itemize}

The present package provides a simple interface
to make child files individually compilable by \LaTeX{}.
Compiling a child file then has the same effect as compiling
the main file with an |\includeonly| command
to select the appropriate child.
Moreover the generated document will carry the name of the child
rather than the main file.
This resolves all three above issues.

This feature is meant to make the editing of books,
thesis documents and lecture notes somewhat more convenient.
However, the package can also be used efficiently for
composing a series of documents (such as exercise sheets)
which are typically distributed individually.
It then assists the author in generating the individual documents
(potentially in different versions)
as well as a document containing the collected series.
Another application is in developing style files
or other kinds of included material
where compilation of the style file could redirect
to a sample or test file.

%%%%%%%%%%%%%%%%%%%%%%%%%%%%%%%%%%%%%%%%%%%%%%%%%%%%%%%%%%%%%%%%%%%%%%%%%%%%%%%%
%%%%%%%%%%%%%%%%%%%%%%%%%%%%%%%%%%%%%%%%%%%%%%%%%%%%%%%%%%%%%%%%%%%%%%%%%%%%%%%%
\section{Usage}

First of all, the package \textsf{childdoc} is \emph{not} a standard
\LaTeXe{} |.sty| style file! Therefore it needs to be invoked in
a non-standard way.

%%%%%%%%%%%%%%%%%%%%%%%%%%%%%%%%%%%%%%%%%%%%%%%%%%%%%%%%%%%%%%%%%%%%%%%%%%%%%%%%
\subsection{Included Files}
\label{sec:include}

%%%%%%%%%%%%%%%%%%%%%%%%%%%%%%%%%%%%%%%%
\DescribeMacro{\childdocmain}
To use the package, add the commands
\begin{center}
\begin{tabular}{l}
|\input{childdoc.def}|\\
|\childdocmain{}|\\
\end{tabular}
\end{center}
at the very top of the main \LaTeX{} file,
in particular \emph{before} the |\documentclass| statement!
The argument of |\childdocmain| should be left empty
(but it must be present).

%%%%%%%%%%%%%%%%%%%%%%%%%%%%%%%%%%%%%%%%
\DescribeMacro{\childdocof}
Furthermore, add the commands
\begin{center}
\begin{tabular}{l}
|\input{childdoc.def}|\\
|\childdocof{|\textit{main}|}|\\
\end{tabular}
\end{center}
at the top of every child file \textit{child}
which is included by |\include{|\textit{child}|}|
from within the main file
(or at least for those files to be compiled individually).
The argument \textit{main} must be the filename of the main file.

There are a couple of
considerations in setting up the main and child documents:

%%%%%%%%%%%%%%%%%%%%%%%%%%%%%%%%%%%%%%%%
\paragraph{Restrictions.}

Please note the following restrictions:
\begin{itemize}
\item
|\childdocmain| must be called with one argument \textit{main}
to ensure compatibility with earlier version of the package.
It must either be empty (|\childdocmain{}|)
or precisely match the filename of the main file in which it is specified.
See \secref{sec:detection} for further information.
\item
The filename \textit{main} must be specified without the |.tex| extension.
\item
The filename \textit{main} is case sensitive
(even in case-insensitive file systems)
due to internal string comparison.
\item
The argument \textit{main} should be fully expanded, it cannot be a macro.
\item
Subdirectories and special characters should be avoided in filenames.
\item
The command |\childdocmain{|\textit{main}|}| must be followed by a whitespace.
It should not be followed immediately by another command
or by a comment mark `|%|'.
This is because the \TeX{} parser reads the token immediately following
the argument of |\childdocmain| and puts it
at the beginning of every child section;
however, a white\-space is ignored.
\end{itemize}

%%%%%%%%%%%%%%%%%%%%%%%%%%%%%%%%%%%%%%%%
\paragraph{Content of Main File.}

It is advisable to place all content in the child files included by |\include|.
Any output contained in the main file will appear in all child documents
unless suppressed manually;
it cannot be suppressed automatically by the |\includeonly| directive
and thus should normally be avoided.
A method to include some content in the main file
by means of conditional processing is described in \secref{sec:conditional}.

%%%%%%%%%%%%%%%%%%%%%%%%%%%%%%%%%%%%%%%%
\paragraph{Page Numbering.}

When only a part of the document is compiled,
the appropriate numbering of pages
(as well as other status parameters)
is determined from the |.aux| files.
The latter contain information from previous passes.
However this information needs to propagate through
all intermediate child documents.
Therefore the page numbering in child documents may well
be inconsistent until the complete document is compiled at least once.

A useful (if unconventional) way to always ensure a consistent
page numbering is to restart the numbering in each child document
and denote the pages by `\textit{child}|.|\textit{page}'
where \textit{child} represents the chapter/section number of the child file.
This can be achieved by the command
|\numberwithin{page}{|\textit{child}|}|
of the \textsf{amsmath} package
where \textit{child} can be |chapter| or |section|
depending on the chosen structuring.
Alternatively, one can modify the macro |\thepage| appropriately
and reset the counter |page| at the start of each child file.

%%%%%%%%%%%%%%%%%%%%%%%%%%%%%%%%%%%%%%%%%%%%%%%%%%%%%%%%%%%%%%%%%%%%%%%%%%%%%%%%
\subsection{Conditional Processing}
\label{sec:conditional}

The package provides a mechanism to compile different versions
of a document. To customise the versions further some conditional processing
can come in handy to distinguish which version is being compiled.
The package provides two macros to describe the compilation context:

%%%%%%%%%%%%%%%%%%%%%%%%%%%%%%%%%%%%%%%%
\DescribeMacro{\ifchilddoc}
The conditional |\ifchilddoc| distinguishes between the compilation of
child documents and the main document:
%
\begin{center}
|\ifchilddoc |\textit{child-code}| |[|\||else |\textit{main-code}]| \||fi|
\end{center}

%%%%%%%%%%%%%%%%%%%%%%%%%%%%%%%%%%%%%%%%
\DescribeMacro{\childdocname}
\DescribeMacro{\childdocjob}
The macro |\childdocname| contains the filename (without extension)
of the main or child file being processed.
Note that |\childdocjob| will always contain the name of the main file.

%%%%%%%%%%%%%%%%%%%%%%%%%%%%%%%%%%%%%%%%
\paragraph{Title Page.}

Conditional processing can be used to include a title or banner page
in the main document when proper precautions are taken.
Importantly, the code in the main file should ensure that the page counter
(as well as other status parameters which are stored in the |.aux| files)
takes the same value after the conditional processing.
Otherwise the page numbers may take divergent values
depending on which part is compiled.

For example, a title page could be declared by:
%
\begin{center}
\begin{tabular}{l}
|\ifchilddoc\||else|\\
|\addtocounter{page}{-1}|\\
\textit{code for title page}\\
|\newpage|\\
|\||fi|
\end{tabular}
\end{center}
%
A banner page for the child documents can be generated by:
%
\begin{center}
\begin{tabular}{l}
|\ifchilddoc|\\
|\addtocounter{page}{-1}|\\
\textit{code for banner page}\\
|\newpage|\\
|\||fi|
\end{tabular}
\end{center}
%
Here one could write a message such as:
\begin{center}
|This is the part \childdocname{} of \childdocjob{}.|
\end{center}

%%%%%%%%%%%%%%%%%%%%%%%%%%%%%%%%%%%%%%%%%%%%%%%%%%%%%%%%%%%%%%%%%%%%%%%%%%%%%%%%
\subsection{Flags}
\label{sec:flags}

The package makes it easy to generate different versions
of the main or child documents.
To this end compilation flags can be defined
and assigned different default values.
They will be particularly useful in conjunction
with the forwarding mechanism described in \secref{sec:forward}.

For example, it may be useful to have a flag |\version|
which can be set to |draft| or |final|.
The document source will contain some conditional code
depending on the value of |\version|.
Suppose further, the flag should default to |final| for the main file
and to |draft| for child files
which is a natural assignment for editing the document.
This is achieved by placing the following code
in the preamble of the main document
(below the |\childdocmain| directive):
%
\begin{center}
\begin{tabular}{l}
|\ifchilddoc|\\
|\providecommand{\version}{draft}|\\
|\||else|\\
|\providecommand{\version}{final}|\\
|\||fi|
\end{tabular}
\end{center}
%
The definition by |\providecommand| makes sure
that previous definitions are not overwritten.
Further statements |\providecommand{\version}{...}|
can thus be added before the above code to override it.

For the main file, one might add a line
(between |\childdocmain| and the above block)
%
\begin{center}
|%\ifchilddoc\||else\providecommand{\version}{draft}\||fi|
\end{center}
%
which can be uncommented to produce a draft version.
Likewise one can add a line to the very top of a child file
(above the |\childdocof{|\textit{main}|}| directive)
%
\begin{center}
|%\providecommand{\version}{final}|
\end{center}
%
which can be uncommented to produce the final version of this child document.

%%%%%%%%%%%%%%%%%%%%%%%%%%%%%%%%%%%%%%%%%%%%%%%%%%%%%%%%%%%%%%%%%%%%%%%%%%%%%%%%
\subsection{Forwarding}
\label{sec:forward}

Different versions of the main or child documents
using compilation flags as described in \secref{sec:flags}
can be (permanently) stored in different files
for convenient compilation, viewing and distribution.
To this end, the package defines a command
to pass on compilation to a different file:

%%%%%%%%%%%%%%%%%%%%%%%%%%%%%%%%%%%%%%%%
\DescribeMacro{\childdocforward}
The command |\childdocforward| redirects processing to
another source file:
%
\begin{center}
\begin{tabular}{l}
|\input{childdoc.def}|\\
|\childdocforward[|\textit{main}|]{|\textit{dest}|}|\\
\end{tabular}
\end{center}
%
The argument \textit{dest} is the destination file
(without extension).
It should be the main file or one of the child files.
Note that further \textsf{childdoc} directives
such as |\childdocof| and |\childdocforward|
in the indicated file will be processed in this form.
The optional argument \textit{main}
passes on directly to the main file \textit{main}
while pretending to compile the child \textit{dest}.
This form behaves as if \textit{dest}
issues |\childdocof{|\textit{main}|}| right away,
and no further \textsf{childdoc} directives will be processed.

%%%%%%%%%%%%%%%%%%%%%%%%%%%%%%%%%%%%%%%%
\DescribeMacro{\...prefix}
In the alternative form |\childdocforwardprefix|,
%
\begin{center}
\begin{tabular}{l}
|\input{childdoc.def}|\\
|\childdocforwardprefix[|\textit{main}|]{|\textit{prefix}|}{|\textit{dest}|}|
\end{tabular}
\end{center}
%
the destination file is determined by a pattern
depending on the current file:
To make this work, the current file must be called
`{\textit{prefix}\hspace{0.2em}\textit{suffix}}'
with \textit{prefix} matching precisely the argument.
Processing is then passed on to the file
`{\textit{dest}\hspace{0.2em}\textit{suffix}}'.
Surely, the same effect is achieved by
directly specifying the
argument `{\textit{dest}\hspace{0.2em}\textit{suffix}}'
in the first form.
However, that requires to set up a different file
for each child. With the alternative form of the command
all these files can have exactly the same content
which simplifies setting them up and maintaining them.

For example, the following file |draft.tex|
with a compilation flag |\version| as described in \secref{sec:flags}
compiles the main document as a draft:
%
\begin{center}
\begin{tabular}{l}
|\def\version{draft}|\\
|\input{childdoc.def}|\\
|\childdocforward{|\textit{main}|}|
\end{tabular}
\end{center}
%
Likewise, the following files |final|\textit{nn}|.tex|
compile the final version of the child document
|child|\textit{nn}|.tex|:
%
\begin{center}
\begin{tabular}{l}
|\def\version{final}|\\
|\input{childdoc.def}|\\
|\childdocforwardprefix{final}{child}|
\end{tabular}
\end{center}
%

Note that when several versions of a main file and/or of each child file
are to be generated, it may be convenient to set up a |Makefile| or
shell script to automatise the process.

%%%%%%%%%%%%%%%%%%%%%%%%%%%%%%%%%%%%%%%%%%%%%%%%%%%%%%%%%%%%%%%%%%%%%%%%%%%%%%%%
\subsection{Command Line Processing}
\label{sec:commandline}

The effect of redirection files can also be achieved by invoking
the \LaTeX{} compiler with a more elaborate command line.
Most conveniently this should be done as part
of a shell script or a |Makefile|.

When using \textsf{childdoc} in the main file, the following
command lines effectively perform a redirection
(note that depending on the shell being used,
backslashes may have to be doubled: `|\|' $\to$ `|\\|'):
%
\begin{center}
|... -jobname "|\textit{target}|" |\\|"|[\textit{flags}]%
|\input{childdoc.def}\childdocforward[|\textit{main}|]{|\textit{dest}|}"|
\end{center}
%
Here \textit{target} is the name of the output file,
\textit{main} is the name of the main file
and \textit{dest} is the name of the main or child file to be processed
(all filenames without extensions).
The optional argument \textit{main} can be omitted
if \textit{main} matches \textit{dest}.
Optionally, compilation \textit{flags} can be defined via |\def| commands.
This command line makes the \TeX{} engine believe
it is compiling the file \textit{target}
whose content is specified as the latter parameter.
The provided code then forwards the processing to
\textit{main} or \textit{dest} as described in \secref{sec:forward}.

%%%%%%%%%%%%%%%%%%%%%%%%%%%%%%%%%%%%%%%%%%%%%%%%%%%%%%%%%%%%%%%%%%%%%%%%%%%%%%%%
\subsection{Include by Input}
\label{sec:input}

Including child documents by |\include| has some restrictions by design.
Most notably, the content of a child document always occupies
its own set of pages; pages cannot be shared between child documents.
Usually, this behaviour makes perfect sense
because each child document contain an essential part of the document.
However, in some situations it may be desirable to compose
a document from a collection of parts
without having mandatory page breaks between then.
For this case, the package
provides a mechanism to include parts
by |\input| which can also be processed individually.
However, by construction this mechanism
requires manual handling of the content to be output.

%%%%%%%%%%%%%%%%%%%%%%%%%%%%%%%%%%%%%%%%
\DescribeMacro{\ifchilddocmanual}
The main file should be prepared as usual, see \secref{sec:include}.
However, the document body must make a distinction
between processing of an individual part and of the main document, e.g.:
%
\begin{center}
\begin{tabular}{l}
|\ifchilddocmanual|\\
|\input{\childdocname}|\\
|\||else|\\
\textit{document body with }|\input{|\textit{part}|}|\\
|\||fi|
\end{tabular}
\end{center}
%
The conditional |\ifchilddocmanual| is true whenever
a part to be included by |\input| is being compiled,
and the name of the part is stored in |\childdocname|.

%%%%%%%%%%%%%%%%%%%%%%%%%%%%%%%%%%%%%%%%
\DescribeMacro{\childdocby}
Each part to be included by |\input| should start with:
%
\begin{center}
\begin{tabular}{l}
|\input{childdoc.def}|\\
|\childdocby{|\textit{main}|}|\\
\end{tabular}
\end{center}
%
The directive |\childdocby| is similar to |\childdocof|
described in \secref{sec:include},
but the subsequent selection of content must be done manually.
To that end, both |\ifchilddoc| and |\ifchilddocmanual|
will be true upon processing of a part,
and the name of the part is stored in |\childdocname|.
Note that |\jobname| will be set to the filename of the current part
so that each part receives an individual |.aux| file
that does not interfere with the |.aux| file(s) of the main document.
This behaviour can be altered by the alternative form
|\childdocby[*]{|\textit{main}|}| (with a non-empty optional argument)
which uses the |.aux| file of the main document
by setting |\jobname| to \textit{main}.

%%%%%%%%%%%%%%%%%%%%%%%%%%%%%%%%%%%%%%%%%%%%%%%%%%%%%%%%%%%%%%%%%%%%%%%%%%%%%%%%
\subsection{Driver Development}
\label{sec:driver}

The \textsf{childdoc} mechanism can also be use for the development
of definition files such as \LaTeX{} styles or classes.
This case differs from the above setup with multiple parts
included by |\include| in that no |\includeonly| should be invoked.
This can be achieved by starting the include file
(before |\ProvidesPackage|) with:
%
\begin{center}
\begin{tabular}{l}
|\input{childdoc.def}|\\
|\childdocforward{|\textit{main}|}|\\
\end{tabular}
\end{center}
%
or alternatively with:
%
\begin{center}
\begin{tabular}{l}
|\input{childdoc.def}|\\
|\childdocby{|\textit{main}|}|\\
\end{tabular}
\end{center}
%
Both forms have slightly different effects as described above.
The main file is prepared as usual, see \secref{sec:include}.

%%%%%%%%%%%%%%%%%%%%%%%%%%%%%%%%%%%%%%%%%%%%%%%%%%%%%%%%%%%%%%%%%%%%%%%%%%%%%%%%
\subsection{Legacy Detection}
\label{sec:detection}

The directive |\childdocmain| in the main file can detect
whether the complete document or merely a child is to be compiled
even without using the directive |\childdocof|.
This method is deprecated because it is less robust
and there is no compelling reason to use it;
it is merely provided for backward compatibility
and it may be removed in future versions.

If the detection mechanism is to be used,
it is mandatory to correctly specify
the filename of the main file as the argument of |\childdocmain|:
%
\begin{center}
\begin{tabular}{l}
|\input{childdoc.def}|\\
|\childdocmain{|\textit{main}|}|\\
\end{tabular}
\end{center}
%
If |\jobname| does not match the argument \textit{main} of |\childdocmain|,
it is assumed that |\jobname| points to the child file to be compiled.
When using |\childdocmain| with the main file specified as argument,
it suffices to start a child file
with just |\input{|\textit{main}|}|
without loading of the package and using |\childdocof|.
If instead all processing is done
with the appropriate \textsf{childdoc} directives,
the argument of \textit{main} of |\childdocmain| can be empty.

An alternative version of the command line processing described
in \secref{sec:commandline} using the detection mechanism reads:
%
\begin{center}
|... -jobname "|\textit{target}|" "|[\textit{flags}]%
[|\def\jobname{|\textit{dest}|}|]|\input{|\textit{main}|}"|
\end{center}

%%%%%%%%%%%%%%%%%%%%%%%%%%%%%%%%%%%%%%%%%%%%%%%%%%%%%%%%%%%%%%%%%%%%%%%%%%%%%%%%
\subsection{Manual Code}
\label{sec:manual}

In case one cannot be certain whether the definitions file |childdoc.def|
is installed on the target \TeX{} distribution
and one prefers not to ship it,
it is conceivable to paste a few relevant commands into the sources.

To that end, drop all statements |\input{childdoc.def}|
and perform the replacements as outlined below.
Instead of |\childdocmain{|\textit{main}|}| add the following code
to the top of the main file:
%
\begin{center}
\begin{tabular}{l}
|\||ifdefined\childdocname\endinput\||fi\newif\ifchilddoc|\\
|\edef\childdocname{\scantokens\expandafter{\jobname\noexpand}}|\\
|\def\childdocmain{|\textit{main}|}\||ifx\childdocmain\childdocname\||else|\\
|\childdoctrue\includeonly{\childdocname}\let\jobname\childdocmain\||fi|\\
\end{tabular}
\end{center}
%
Instead of |\childdocof{|\textit{main}|}| just include the main file
at the top of each child file:
%
\begin{center}
|\input{|\textit{main}|}|
\end{center}
%
A simple redirection |\childdocforward{|\textit{dest}|}| is achieved by:
%
\begin{center}
|\def\jobname{|\textit{dest}|}\input{\jobname}|
\end{center}
%
The redirection with prefix
|\childdocforwardprefix[|\textit{prefix}|]{|\textit{dest}|}|
is accomplished by:
%
\begin{center}
\begin{tabular}{l}
|{\edef\jobname{\scantokens\expandafter{\jobname\noexpand}}|\\
|\def\redirectjob |\textit{prefix}|#1~~~{\gdef\jobname{|\textit{dest}|#1}}|\\
|\expandafter\redirectjob\jobname~~~}\input{\jobname}|
\end{tabular}
\end{center}

In an alternative approach,
child documents can be compiled by a specific command line
without additional code or specific definitions:
%
\begin{center}
|... -jobname "|\textit{target}|" "|[\textit{flags}]%
|\includeonly{|\textit{dest}|}\input{|\textit{main}|}"|
\end{center}
%

%%%%%%%%%%%%%%%%%%%%%%%%%%%%%%%%%%%%%%%%%%%%%%%%%%%%%%%%%%%%%%%%%%%%%%%%%%%%%%%%
%%%%%%%%%%%%%%%%%%%%%%%%%%%%%%%%%%%%%%%%%%%%%%%%%%%%%%%%%%%%%%%%%%%%%%%%%%%%%%%%
\section{Information}

%%%%%%%%%%%%%%%%%%%%%%%%%%%%%%%%%%%%%%%%%%%%%%%%%%%%%%%%%%%%%%%%%%%%%%%%%%%%%%%%
\subsection{Copyright}

Copyright \copyright{} 2017--2018 Niklas Beisert

This work may be distributed and/or modified under the
conditions of the \LaTeX{} Project Public License, either version 1.3
of this license or (at your option) any later version.
The latest version of this license is in
  \url{http://www.latex-project.org/lppl.txt}
and version 1.3 or later is part of all distributions of \LaTeX{}
version 2005/12/01 or later.

This work has the LPPL maintenance status `maintained'.

The Current Maintainer of this work is Niklas Beisert.

This work consists of the files |README.txt|, |childdoc.ins| and |childdoc.dtx|
as well as the derived files |childdoc.def|, |cdocsamp.tex|
with |cdocsch1.tex|, |cdocsch2.tex|, |cdocspt3.tex|, |cdocspt4.tex|,
|cdocsdrf.tex|, |cdocsfn1.tex|, |cdocsfn2.tex|
as well as |childdoc.pdf|.

%%%%%%%%%%%%%%%%%%%%%%%%%%%%%%%%%%%%%%%%%%%%%%%%%%%%%%%%%%%%%%%%%%%%%%%%%%%%%%%%
\subsection{Files and Installation}

The package consists of the files:
%
\begin{center}
\begin{tabular}{ll}
    |README.txt|   & readme file \\
    |childdoc.ins| & installation file \\
    |childdoc.dtx| & source file \\
    |childdoc.def| & definition file \\
    |cdocsamp.tex| & sample main file \\
    |cdocsch1.tex| & sample include file \\
    |cdocsch2.tex| & sample include file \\
    |cdocspt3.tex| & sample part file \\
    |cdocspt4.tex| & sample part file \\
    |cdocsdrf.tex| & sample redirection file \\
    |cdocsfn1.tex| & sample redirection file \\
    |cdocsfn2.tex| & sample redirection file \\
    |childdoc.pdf| & manual
\end{tabular}
\end{center}
%
The distribution consists of the files
|README.txt|, |childdoc.ins| and |childdoc.dtx|.
%
\begin{itemize}
\item
Run (pdf)\LaTeX{} on |childdoc.dtx|
to compile the manual |childdoc.pdf| (this file).
\item
Run \LaTeX{} on |childdoc.ins| to create the definitions file |childdoc.def|
and the sample |cdocsamp.tex| with include files
|cdocsch1.tex|, |cdocsch2.tex|, |cdocspt3.tex|, |cdocspt4.tex|,
|cdocsdrf.tex|, |cdocsfn1.tex|, |cdocsfn2.tex|.
Then copy the file |childdoc.def| to an appropriate directory of your \LaTeX{}
distribution, e.g.\ \textit{texmf-root}|/tex/latex/childdoc|.
\end{itemize}

%%%%%%%%%%%%%%%%%%%%%%%%%%%%%%%%%%%%%%%%%%%%%%%%%%%%%%%%%%%%%%%%%%%%%%%%%%%%%%%%
\subsection{Related CTAN Packages}

There are several other packages which offer a similar functionality:
%
\begin{itemize}
\item
The packages
\href{http://ctan.org/pkg/docmute}{\textsf{docmute}},
\href{http://ctan.org/pkg/includex}{\textsf{includex}} and
\href{http://ctan.org/pkg/standalone}{\textsf{standalone}}
provide commands to include only the document body of
a child file thus allowing both files to be compiled individually.
\item
The packages \href{http://ctan.org/pkg/subdocs}{\textsf{subdocs}}
and \href{http://ctan.org/pkg/subfiles}{\textsf{subfiles}}
provide structures in which the main and child documents can be
encapsulated and allowing them to be compiled individually.
The inclusion mechanism is different from the conventional |\include|.
\item
The package \href{http://ctan.org/pkg/combine}{\textsf{combine}}
is an elaborate solution to combine several documents into one.
\end{itemize}
%
See also the CTAN topic \href{http://ctan.org/topic/subdocs}{\textsf{subdocs}}
for further related packages.
The present package differs from the above solutions in that
a document structure constructed with the conventional |\include| mechanism
just needs two extra commands at the top of every file
such that all constituent files can be compiled individually.

%%%%%%%%%%%%%%%%%%%%%%%%%%%%%%%%%%%%%%%%%%%%%%%%%%%%%%%%%%%%%%%%%%%%%%%%%%%%%%%%
%\subsection{Feature Suggestions}
%
%The following is a list of features which may be useful for future
%versions of this package:
%%
%\begin{itemize}
%\item
%\ldots
%\end{itemize}

%%%%%%%%%%%%%%%%%%%%%%%%%%%%%%%%%%%%%%%%%%%%%%%%%%%%%%%%%%%%%%%%%%%%%%%%%%%%%%%%
\subsection{Revision History}

%%%%%%%%%%%%%%%%%%%%%%%%%%%%%%%%%%%%%%%%
\paragraph{v2.0:} 2018/12/30

\begin{itemize}
\item
immediate forward processing
\item
added |\childdocby| mechanism
\item
manual restructured
\end{itemize}

%%%%%%%%%%%%%%%%%%%%%%%%%%%%%%%%%%%%%%%%
\paragraph{v1.6:} 2018/01/17

\begin{itemize}
\item
application for development of include files
\item
corrections to manual
\end{itemize}

%%%%%%%%%%%%%%%%%%%%%%%%%%%%%%%%%%%%%%%%
\paragraph{v1.5:} 2017/05/21

\begin{itemize}
\item
more complete structuring introduced
\item
|\childdocof| introduced
\item
|\childdoc| renamed to |\childdocmain|
\item
|\childredirect| renamed to |\childdocforward| and |\childdocforwardprefix|
and functionality expanded
\end{itemize}

%%%%%%%%%%%%%%%%%%%%%%%%%%%%%%%%%%%%%%%%
\paragraph{v1.0:} 2017/04/27

\begin{itemize}
\item
manual and install package
\item
first version published on CTAN
\end{itemize}

%%%%%%%%%%%%%%%%%%%%%%%%%%%%%%%%%%%%%%%%
\paragraph{v0.6:} 2017/04/26

\begin{itemize}
\item
redirection mechanism added
\end{itemize}

%%%%%%%%%%%%%%%%%%%%%%%%%%%%%%%%%%%%%%%%
\paragraph{v0.5:} 2017/04/26

\begin{itemize}
\item
functionality in definition file
\end{itemize}


%%%%%%%%%%%%%%%%%%%%%%%%%%%%%%%%%%%%%%%%%%%%%%%%%%%%%%%%%%%%%%%%%%%%%%%%%%%%%%%%
%%%%%%%%%%%%%%%%%%%%%%%%%%%%%%%%%%%%%%%%%%%%%%%%%%%%%%%%%%%%%%%%%%%%%%%%%%%%%%%%
%%%%%%%%%%%%%%%%%%%%%%%%%%%%%%%%%%%%%%%%%%%%%%%%%%%%%%%%%%%%%%%%%%%%%%%%%%%%%%%%
\appendix

\settowidth\MacroIndent{\rmfamily\scriptsize 000\ }

 \DocInput{childdoc.dtx}

\end{document}
%</driver>
% \fi
%
% %%%%%%%%%%%%%%%%%%%%%%%%%%%%%%%%%%%%%%%%%%%%%%%%%%%%%%%%%%%%%%%%%%%%%%%%%%%%%%
% %%%%%%%%%%%%%%%%%%%%%%%%%%%%%%%%%%%%%%%%%%%%%%%%%%%%%%%%%%%%%%%%%%%%%%%%%%%%%%
% \section{Sample}
%\iffalse
%<*samplemain>
%\fi
%
% The following presents a sample document
% with two chapters, two parts, a title page,
% a compile flag as well as three forwarding files to set the flag.
% It consists of eight |.tex| files:
% \begin{center}
% \begin{tabular}{ll}
% |cdocsamp.tex|&main file\\
% |cdocsch1.tex|&include file for chapter 1\\
% |cdocsch2.tex|&include file for chapter 2\\
% |cdocspt3.tex|&include file for part 3\\
% |cdocspt4.tex|&include file for part 4\\
% |cdocsdrf.tex|&forwarding file for main file in draft mode\\
% |cdocsfi1.tex|&forwarding file for final version of chapter 1\\
% |cdocsfi2.tex|&forwarding file for final version of chapter 2\\
% \end{tabular}
% \end{center}
% Each of the eight files can be compiled directly by the \LaTeX{} compiler.
%
% %%%%%%%%%%%%%%%%%%%%%%%%%%%%%%%%%%%%%%
% \paragraph{Main File.}
%
% The main file is called |cdocsamp.tex|.
%
% Load the \textsf{childdoc} definitions and
% declare the filename for the main document:
%    \begin{macrocode}
\input{childdoc.def}
\childdocmain{}
%    \end{macrocode}

% Optional override for |\version| flag:
%    \begin{macrocode}
%%\ifchilddoc\else\providecommand{\version}{draft}\fi
%    \end{macrocode}

% Define the default values for the |\version| flag
% (|final| for the main file and |draft| for childs):
%    \begin{macrocode}
\ifchilddoc
\providecommand{\version}{draft}
\else
\providecommand{\version}{final}
\fi
%    \end{macrocode}

% Load the standard document class:
%    \begin{macrocode}
\documentclass[12pt]{article}
%    \end{macrocode}

% Start the document body:
%    \begin{macrocode}
\begin{document}
%    \end{macrocode}

% Declare a title page.
% Print title, part of document being processed and version flag:
%    \begin{macrocode}
\addtocounter{page}{-1}
\begin{center}
{\LARGE\bfseries{}childdoc example\par}
\vspace{1cm}
\ifchilddoc
\ifchilddocmanual part\else chapter\fi:
`\childdocname' of `\childdocjob'\par
\else
main document: `\childdocjob'\par
\fi
version: \version\par
\end{center}
\newpage
%    \end{macrocode}

% Manually include selected file,
% otherwise process as usual:
%    \begin{macrocode}
\ifchilddocmanual
\section*{part `\childdocname'}
\input{\childdocname}
\else
%    \end{macrocode}

% Include the two chapters:
%    \begin{macrocode}
\include{cdocsch1}
\include{cdocsch2}
%    \end{macrocode}

% Include the two parts unless only chapters should be displayed:
%    \begin{macrocode}
\ifchilddoc\else
\section{part three}
\input{cdocspt3}
\section{part four}
\input{cdocspt4}
\fi
%    \end{macrocode}

% Process as usual until here:
%    \begin{macrocode}
\fi
%    \end{macrocode}

% End of document body:
%    \begin{macrocode}
\end{document}
%    \end{macrocode}
%\iffalse
%</samplemain>
%\fi
%
% %%%%%%%%%%%%%%%%%%%%%%%%%%%%%%%%%%%%%%
% \paragraph{Chapter Include Files.}
%
% The include files are called |cdocsch1.tex| and |cdocsch2.tex|.
%
%\iffalse
%<*samplechap1|samplechap2>
%\fi

% Optional override for |\version| flag:
%    \begin{macrocode}
%%\providecommand{\version}{final}
%    \end{macrocode}

% Include the main document:
%    \begin{macrocode}
\input{childdoc.def}
\childdocof{cdocsamp}
%    \end{macrocode}

%\iffalse
%</samplechap1|samplechap2>
%\fi
%
%\iffalse
%<*samplechap1>
%\fi
% Some text for chapter 1:
%    \begin{macrocode}
\section{one}
some text in chapter one
%    \end{macrocode}

%\iffalse
%</samplechap1>
%\fi
% Some text for chapter 2:
%\iffalse
%<*samplechap2>
%\fi
%    \begin{macrocode}
\section{two}
more text in chapter two
%    \end{macrocode}

%\iffalse
%</samplechap2>
%\fi
%
% %%%%%%%%%%%%%%%%%%%%%%%%%%%%%%%%%%%%%%
% \paragraph{Part Include Files.}
%
% The include files are called |cdocspt3.tex| and |cdocspt4.tex|.
%
%\iffalse
%<*samplepart3|samplepart4>
%\fi

% Optional override for |\version| flag:
%    \begin{macrocode}
%%\providecommand{\version}{final}
%    \end{macrocode}

% Include the main document:
%    \begin{macrocode}
\input{childdoc.def}
\childdocby{cdocsamp}
%    \end{macrocode}

%\iffalse
%</samplepart3|samplepart4>
%\fi
%
%\iffalse
%<*samplepart3>
%\fi
% Some text for part 3:
%    \begin{macrocode}
some text in part three
%    \end{macrocode}

%\iffalse
%</samplepart3>
%\fi
% Some text for part 4:
%\iffalse
%<*samplepart4>
%\fi
%    \begin{macrocode}
more text in part four
%    \end{macrocode}

%\iffalse
%</samplepart4>
%\fi
%
% %%%%%%%%%%%%%%%%%%%%%%%%%%%%%%%%%%%%%%
% \paragraph{Forwarding for a Complete Draft.}
%
% The following forwarding file |cdocsdrf.tex|
% compiles the main document in draft mode:
%\iffalse
%<*sampledraft>
%\fi
%    \begin{macrocode}
\def\version{draft}
\input{childdoc.def}
\childdocforward{cdocsamp}
%    \end{macrocode}

%\iffalse
%</sampledraft>
%\fi
%
% %%%%%%%%%%%%%%%%%%%%%%%%%%%%%%%%%%%%%%
% \paragraph{Forwarding for Final Version of the Chapters.}
%
% The following forwarding files |cdocsfn1.tex| and |cdocsfn2.tex|
% (with identical content)
% compile the final versions of the child documents
% |cdocsch1.tex| and |cdocsch2.tex|, respectively:
%\iffalse
%<*samplefinal>
%\fi
%    \begin{macrocode}
\def\version{final}
\input{childdoc.def}
\childdocforwardprefix[cdocsamp]{cdocsfn}{cdocsch}
%    \end{macrocode}

%\iffalse
%</samplefinal>
%\fi
%
% %%%%%%%%%%%%%%%%%%%%%%%%%%%%%%%%%%%%%%
% \paragraph{Command Line Processing.}
%
% The following three command lines generate the output files
% |cdocscld|, |cdocscl1| and |cdocscl2|
% which should be identical to
% |cdocsdrf|, |cdocsch1| and |cdocsfn2|, respectively:
% \begin{center}
% \begin{tabular}{l}
% |latex -jobname cdocscld \|\\
% |  "\def\version{draft}\input{childdoc.def}\childdocforward{cdocsamp}"|\\
% |latex -jobname cdocscl1 \|\\
% |  "\input{childdoc.def}\childdocforward[cdocsamp]{cdocsch1}"|\\
% |latex -jobname cdocscl2 \|\\
% |  "\def\version{final}\input{childdoc.def}\childdocforward{cdocsch2}"|
% \end{tabular}
% \end{center}
% Note that the trailing backslash on each first line
% merely continues the input to the second line
% (for convenient cut ant paste).
% Furthermore, the command |latex| can be replaced by any
% of its alternative versions such as |pdflatex|.
%
% %%%%%%%%%%%%%%%%%%%%%%%%%%%%%%%%%%%%%%%%%%%%%%%%%%%%%%%%%%%%%%%%%%%%%%%%%%%%%%
% %%%%%%%%%%%%%%%%%%%%%%%%%%%%%%%%%%%%%%%%%%%%%%%%%%%%%%%%%%%%%%%%%%%%%%%%%%%%%%
% \section{Implementation}
%\iffalse
%<*package>
%\fi
%
% This section describes the definitions file |childdoc.def|.

% The definitions cannot be loaded using |\usepackage| or |\RequirePackage|
% which has a mechanism to prevent loading a style file more than once.
% When loading the definitions by means of |\input|
% multiple instances have to be prevented manually:
%\iffalse
%This code needs to be before the `\ProvidesFile' directive
%which is defined at the beginning of this file.
%Therefore it is also placed there and commented out here.
%</package>
%<*discard>
%\fi
%    \begin{macrocode}
\ifdefined\childdocmain\endinput\fi
%    \end{macrocode}
%\iffalse
%</discard>
%<*package>
%\fi
%
% \macro{\ifchilddoc}
% \macro{\ifchilddocmanual}
% The conditional |\ifchilddoc| tells whether a
% child (true) or main (false) document is being compiled.
% The conditional |\ifchilddocmanual| tells whether
% the |\includeonly| mechanism is used (false) or
% the selection of child files must be performed manually (true).
% The definitions initialise to false:
%    \begin{macrocode}
\newif\ifchilddoc
\newif\ifchilddocmanual
%    \end{macrocode}

% \macro{\childdocname}
% \macro{\childdocjob}
% The macro |\childdocname| stores the name of the main document
% to be compiled. The macro |\childdocjob| stores the name of
% the document on which the \LaTeX{} compiler was originally invoked.
% The content of |\jobname| cannot be compared
% to filenames specified in the source due to different catcodes.
% The following code rescans |\jobname|, stores the result
% in |\childdocname| and saves a copy in |\childdocjob|:
%    \begin{macrocode}
\edef\childdocname{\scantokens\expandafter{\jobname\noexpand}}
\let\childdocjob\childdocname
%    \end{macrocode}

% \macro{\childdocdisable}
% The macro |\childdocdisable| prevents the main file
% from being processed more than once.
% At this stage, the main document command |\childdocmain|
% is assumed to be called once again where it should do nothing.
% Any subsequent call to it should prevent
% a secondary processing of the main document
% It overwrites the forwarding commands
% |\childdocof| and |\childdocforward|
% with empty macros to prevent further inclusions of the main document:
%    \begin{macrocode}
\newcommand{\childdocdisable}
{
  \renewcommand{\childdocmain}[1]{\renewcommand{\childdocmain}[1]{\endinput}}
  \renewcommand{\childdocof}[1]{}
  \renewcommand{\childdocby}[2][]{}
  \renewcommand{\childdocforward}[2][]{}
  \renewcommand{\childdocdisable}{}
}
%    \end{macrocode}

% \macro{\childdocmain}
% The macro |\childdocmain| is to be called at the top of the main file
% with nothing or the main filename (without extension) as argument.
% First, it breaks loops.
% If the argument is not empty and does not match |\childdocname|
% (which is set by the first inclusion of |childdoc.def|),
% |\ifchilddoc| is set to true, |\includeonly| is applied to the child file
% and |\jobname| is set to the main file
% (for proper handling of |.aux| files):
%    \begin{macrocode}
\newcommand{\childdocmain}[1]
{
  \childdocdisable\childdocmain{}
  \if?#1?\else
    \begingroup
      \def\childdoctmp{#1}
      \ifx\childdoctmp\childdocname
        \def\childdoctmp{}
      \else
        \def\childdoctmp
        {
          \childdoctrue
          \includeonly{\childdocname}
          \def\childdocjob{#1}
          \def\jobname{#1}
        }
      \fi
      \expandafter
    \endgroup
    \childdoctmp
  \fi
}
%    \end{macrocode}

% \macro{\childdocof}
% The command |\childdocof| redirects
% compilation to the main file |#1|.
%    \begin{macrocode}
\newcommand{\childdocof}[1]
{
  \childdocdisable
  \childdoctrue
  \includeonly{\childdocname}
  \def\jobname{#1}
  \def\childdocjob{#1}
  \input{#1}
}
%    \end{macrocode}

% \macro{\childdocby}
% The command |\childdocby| ....
%    \begin{macrocode}
\newcommand{\childdocby}[2][]
{
  \childdocdisable
  \childdoctrue
  \childdocmanualtrue
  \if?#1?\else
    \def\jobname{#2}
  \fi
  \def\childdocjob{#2}
  \input{#2}
  \endinput
}
%    \end{macrocode}

% \macro{\childdocforward}
% The command |\childdocforward| redirects
% compilation to the main file or
% (if the optional argument is given) a child file.
% Parameters are set as if the main file
% or a child file starting with |\childdocof| was compiled.
% Then compilation is handed over to the main file:
%    \begin{macrocode}
\newcommand{\childdocforward}[2][]
{
  \begingroup
    \if?#1?
      \def\childdoctmp
      {
        \def\childdocname{#2}
        \def\childdocjob{#2}
        \def\jobname{#2}
        \input{#2}
        \endinput
      }
    \else
      \def\childdoctmp
      {
        \childdocdisable
        \def\childdocname{#2}
        \childdoctrue
        \includeonly{#2}
        \def\childdocjob{#1}
        \def\jobname{#1}
        \input{#1}
        \endinput
      }
    \fi
    \expandafter
  \endgroup
  \childdoctmp
}
%    \end{macrocode}

% \macro{\childdocforwardprefix}
% The command |\childdocforwardprefix| redirects
% compilation to the main or a child file by means of a pattern.
% The prefix |#1| in the current filename is replaced by |#2|
% and the suffix of the current filename is kept
% (it is assumed that the filename does not contain the substring `|~~~|'
% which is used as a delimiter).
% Compilation is handed over to the new file by |\childdocforward|:
%    \begin{macrocode}
\newcommand{\childdocforwardprefix}[3][]
{
  \begingroup
    \def\childdocextract #2##1~~~{\def\childdoctmp{\childdocforward[#1]{#3##1}}}
    \expandafter\childdocextract\childdocname~~~
    \expandafter
  \endgroup
  \childdoctmp
}
%    \end{macrocode}

% \macro{\childdoc}
% The deprecated macro |\childdoc| is a legacy version of |\childdocmain|:
%    \begin{macrocode}
\newcommand{\childdoc}{\childdocmain}
%    \end{macrocode}

% \macro{\childdocredirect}
% The deprecated macro |\childdocredirect| is a legacy version
% of |\childdocforward| and |\childdocforwardprefix|:
%    \begin{macrocode}
\newcommand{\childdocredirect}[2][]
{
  \begingroup
    \if?#1?
      \def\childdoctmp{\childdocforward{#2}}
    \else
      \def\childdoctmp{\childdocforwardprefix{#1}{#2}}
    \fi
    \expandafter
  \endgroup
  \childdoctmp
}
%    \end{macrocode}

%\iffalse
%</package>
%\fi
%
\endinput
|\\
|\childdocforwardprefix{final}{child}|
\end{tabular}
\end{center}
%

Note that when several versions of a main file and/or of each child file
are to be generated, it may be convenient to set up a |Makefile| or
shell script to automatise the process.

%%%%%%%%%%%%%%%%%%%%%%%%%%%%%%%%%%%%%%%%%%%%%%%%%%%%%%%%%%%%%%%%%%%%%%%%%%%%%%%%
\subsection{Command Line Processing}
\label{sec:commandline}

The effect of redirection files can also be achieved by invoking
the \LaTeX{} compiler with a more elaborate command line.
Most conveniently this should be done as part
of a shell script or a |Makefile|.

When using \textsf{childdoc} in the main file, the following
command lines effectively perform a redirection
(note that depending on the shell being used,
backslashes may have to be doubled: `|\|' $\to$ `|\\|'):
%
\begin{center}
|... -jobname "|\textit{target}|" |\\|"|[\textit{flags}]%
|% \iffalse
%
% childdoc.dtx Copyright (C) 2017-2018 Niklas Beisert
%
% This work may be distributed and/or modified under the
% conditions of the LaTeX Project Public License, either version 1.3
% of this license or (at your option) any later version.
% The latest version of this license is in
%   http://www.latex-project.org/lppl.txt
% and version 1.3 or later is part of all distributions of LaTeX
% version 2005/12/01 or later.
%
% This work has the LPPL maintenance status `maintained'.
%
% The Current Maintainer of this work is Niklas Beisert.
%
% This work consists of the files childdoc.dtx and childdoc.ins
% and the derived files childdoc.def and cdocsamp.tex with
% cdocsch1.tex, cdocsch2.tex, cdocsdrf.tex, cdocsfn1.tex, cdocsfn2.tex.
%
%<package>\ifdefined\childdocmain\endinput\fi
%<package>\ProvidesFile{childdoc.def}[2018/12/30 v2.0 child document driver]
%<samplemain>\ProvidesFile{cdocsamp.tex}[2018/12/30 v2.0 sample for childdoc]
%<*driver>
%\ProvidesFile{childdoc.drv}[2018/12/30 v2.0 childdoc reference manual file]
\PassOptionsToClass{10pt,a4paper}{article}
\documentclass{ltxdoc}

\usepackage[margin=35mm]{geometry}
\usepackage{hyperref}
\usepackage{hyperxmp}
\usepackage[usenames]{color}

\hypersetup{colorlinks=true}
\hypersetup{pdfstartview=FitH}
\hypersetup{pdfpagemode=UseNone}
\hypersetup{pdfsource={}}
\hypersetup{pdflang={en-UK}}
\hypersetup{pdfcopyright={Copyright 2017-2018 Niklas Beisert.
  This work may be distributed and/or modified under the
  conditions of the LaTeX Project Public License, either version 1.3
  of this license or (at your option) any later version.}}
\hypersetup{pdflicenseurl={http://www.latex-project.org/lppl.txt}}
\hypersetup{pdfcontactaddress={ETH Zurich, ITP, HIT K,
  Wolfgang-Pauli-Strasse 27}}
\hypersetup{pdfcontactpostcode={8093}}
\hypersetup{pdfcontactcity={Zurich}}
\hypersetup{pdfcontactcountry={Switzerland}}
\hypersetup{pdfcontactemail={nbeisert@itp.phys.ethz.ch}}
\hypersetup{pdfcontacturl={http://people.phys.ethz.ch/\xmptilde nbeisert/}}

\newcommand{\secref}[1]{\hyperref[#1]{section \ref*{#1}}}

\parskip1ex
\parindent0pt
\let\olditemize\itemize
\def\itemize{\olditemize\parskip0pt}

\begin{document}

\title{The \textsf{childdoc} Package}
\hypersetup{pdftitle={The childdoc Package}}
\author{Niklas Beisert\\[2ex]
  Institut f\"ur Theoretische Physik\\
  Eidgen\"ossische Technische Hochschule Z\"urich\\
  Wolfgang-Pauli-Strasse 27, 8093 Z\"urich, Switzerland\\[1ex]
  \href{mailto:nbeisert@itp.phys.ethz.ch}
  {\texttt{nbeisert@itp.phys.ethz.ch}}}
\hypersetup{pdfauthor={Niklas Beisert}}
\hypersetup{pdfsubject={Manual for the LaTeX2e Package childdoc}}
\date{30 December 2018, \textsf{v2.0}}
\maketitle

\begin{abstract}\noindent
\textsf{childdoc} is a \LaTeXe{} package
that enables the direct compilation
of document sections included by |\include|
to individual files.
\end{abstract}

\begingroup
\parskip0ex
\tableofcontents
\endgroup

%%%%%%%%%%%%%%%%%%%%%%%%%%%%%%%%%%%%%%%%%%%%%%%%%%%%%%%%%%%%%%%%%%%%%%%%%%%%%%%%
%%%%%%%%%%%%%%%%%%%%%%%%%%%%%%%%%%%%%%%%%%%%%%%%%%%%%%%%%%%%%%%%%%%%%%%%%%%%%%%%
\section{Introduction}

\LaTeX{} provides a mechanism to structure a large document (such as a book)
into a main file and several child files (containing the chapters)
using the |\include| command.
This mechanism is beneficial for documents
which span hundreds of pages in order to
make the source file(s) more manageable.
Moreover, compilation can be restricted to
selected child files by means of the |\includeonly| command.
The latter feature can be used to reduce the compilation time while editing
(this was significantly more useful in the earlier days of \LaTeX{})
or to generate a smaller document which is easier to navigate.
Another application of |\includeonly| is to generate
documents consisting of selected parts of the complete document.

However, there are a few drawbacks of the plain |\include| mechanism:
\begin{itemize}
\item
The child files cannot be compiled on their own,
they can only be compiled via the main file.
A naive editing environment
(such as a text editor with an option
to have the current file processed by \LaTeX)
may require one to switch to the main file before compiling;
attempting to compile the child file produces errors.
\item
The main file must be modified (each time)
to adjust the |\includeonly| command
to the present needs. This easily leaves the main file in a messy state.
\item
The generated document will always carry the filename
of the main document. This is inconvenient if
several child files are to be compiled and
to be kept for distribution.
\end{itemize}

The present package provides a simple interface
to make child files individually compilable by \LaTeX{}.
Compiling a child file then has the same effect as compiling
the main file with an |\includeonly| command
to select the appropriate child.
Moreover the generated document will carry the name of the child
rather than the main file.
This resolves all three above issues.

This feature is meant to make the editing of books,
thesis documents and lecture notes somewhat more convenient.
However, the package can also be used efficiently for
composing a series of documents (such as exercise sheets)
which are typically distributed individually.
It then assists the author in generating the individual documents
(potentially in different versions)
as well as a document containing the collected series.
Another application is in developing style files
or other kinds of included material
where compilation of the style file could redirect
to a sample or test file.

%%%%%%%%%%%%%%%%%%%%%%%%%%%%%%%%%%%%%%%%%%%%%%%%%%%%%%%%%%%%%%%%%%%%%%%%%%%%%%%%
%%%%%%%%%%%%%%%%%%%%%%%%%%%%%%%%%%%%%%%%%%%%%%%%%%%%%%%%%%%%%%%%%%%%%%%%%%%%%%%%
\section{Usage}

First of all, the package \textsf{childdoc} is \emph{not} a standard
\LaTeXe{} |.sty| style file! Therefore it needs to be invoked in
a non-standard way.

%%%%%%%%%%%%%%%%%%%%%%%%%%%%%%%%%%%%%%%%%%%%%%%%%%%%%%%%%%%%%%%%%%%%%%%%%%%%%%%%
\subsection{Included Files}
\label{sec:include}

%%%%%%%%%%%%%%%%%%%%%%%%%%%%%%%%%%%%%%%%
\DescribeMacro{\childdocmain}
To use the package, add the commands
\begin{center}
\begin{tabular}{l}
|\input{childdoc.def}|\\
|\childdocmain{}|\\
\end{tabular}
\end{center}
at the very top of the main \LaTeX{} file,
in particular \emph{before} the |\documentclass| statement!
The argument of |\childdocmain| should be left empty
(but it must be present).

%%%%%%%%%%%%%%%%%%%%%%%%%%%%%%%%%%%%%%%%
\DescribeMacro{\childdocof}
Furthermore, add the commands
\begin{center}
\begin{tabular}{l}
|\input{childdoc.def}|\\
|\childdocof{|\textit{main}|}|\\
\end{tabular}
\end{center}
at the top of every child file \textit{child}
which is included by |\include{|\textit{child}|}|
from within the main file
(or at least for those files to be compiled individually).
The argument \textit{main} must be the filename of the main file.

There are a couple of
considerations in setting up the main and child documents:

%%%%%%%%%%%%%%%%%%%%%%%%%%%%%%%%%%%%%%%%
\paragraph{Restrictions.}

Please note the following restrictions:
\begin{itemize}
\item
|\childdocmain| must be called with one argument \textit{main}
to ensure compatibility with earlier version of the package.
It must either be empty (|\childdocmain{}|)
or precisely match the filename of the main file in which it is specified.
See \secref{sec:detection} for further information.
\item
The filename \textit{main} must be specified without the |.tex| extension.
\item
The filename \textit{main} is case sensitive
(even in case-insensitive file systems)
due to internal string comparison.
\item
The argument \textit{main} should be fully expanded, it cannot be a macro.
\item
Subdirectories and special characters should be avoided in filenames.
\item
The command |\childdocmain{|\textit{main}|}| must be followed by a whitespace.
It should not be followed immediately by another command
or by a comment mark `|%|'.
This is because the \TeX{} parser reads the token immediately following
the argument of |\childdocmain| and puts it
at the beginning of every child section;
however, a white\-space is ignored.
\end{itemize}

%%%%%%%%%%%%%%%%%%%%%%%%%%%%%%%%%%%%%%%%
\paragraph{Content of Main File.}

It is advisable to place all content in the child files included by |\include|.
Any output contained in the main file will appear in all child documents
unless suppressed manually;
it cannot be suppressed automatically by the |\includeonly| directive
and thus should normally be avoided.
A method to include some content in the main file
by means of conditional processing is described in \secref{sec:conditional}.

%%%%%%%%%%%%%%%%%%%%%%%%%%%%%%%%%%%%%%%%
\paragraph{Page Numbering.}

When only a part of the document is compiled,
the appropriate numbering of pages
(as well as other status parameters)
is determined from the |.aux| files.
The latter contain information from previous passes.
However this information needs to propagate through
all intermediate child documents.
Therefore the page numbering in child documents may well
be inconsistent until the complete document is compiled at least once.

A useful (if unconventional) way to always ensure a consistent
page numbering is to restart the numbering in each child document
and denote the pages by `\textit{child}|.|\textit{page}'
where \textit{child} represents the chapter/section number of the child file.
This can be achieved by the command
|\numberwithin{page}{|\textit{child}|}|
of the \textsf{amsmath} package
where \textit{child} can be |chapter| or |section|
depending on the chosen structuring.
Alternatively, one can modify the macro |\thepage| appropriately
and reset the counter |page| at the start of each child file.

%%%%%%%%%%%%%%%%%%%%%%%%%%%%%%%%%%%%%%%%%%%%%%%%%%%%%%%%%%%%%%%%%%%%%%%%%%%%%%%%
\subsection{Conditional Processing}
\label{sec:conditional}

The package provides a mechanism to compile different versions
of a document. To customise the versions further some conditional processing
can come in handy to distinguish which version is being compiled.
The package provides two macros to describe the compilation context:

%%%%%%%%%%%%%%%%%%%%%%%%%%%%%%%%%%%%%%%%
\DescribeMacro{\ifchilddoc}
The conditional |\ifchilddoc| distinguishes between the compilation of
child documents and the main document:
%
\begin{center}
|\ifchilddoc |\textit{child-code}| |[|\||else |\textit{main-code}]| \||fi|
\end{center}

%%%%%%%%%%%%%%%%%%%%%%%%%%%%%%%%%%%%%%%%
\DescribeMacro{\childdocname}
\DescribeMacro{\childdocjob}
The macro |\childdocname| contains the filename (without extension)
of the main or child file being processed.
Note that |\childdocjob| will always contain the name of the main file.

%%%%%%%%%%%%%%%%%%%%%%%%%%%%%%%%%%%%%%%%
\paragraph{Title Page.}

Conditional processing can be used to include a title or banner page
in the main document when proper precautions are taken.
Importantly, the code in the main file should ensure that the page counter
(as well as other status parameters which are stored in the |.aux| files)
takes the same value after the conditional processing.
Otherwise the page numbers may take divergent values
depending on which part is compiled.

For example, a title page could be declared by:
%
\begin{center}
\begin{tabular}{l}
|\ifchilddoc\||else|\\
|\addtocounter{page}{-1}|\\
\textit{code for title page}\\
|\newpage|\\
|\||fi|
\end{tabular}
\end{center}
%
A banner page for the child documents can be generated by:
%
\begin{center}
\begin{tabular}{l}
|\ifchilddoc|\\
|\addtocounter{page}{-1}|\\
\textit{code for banner page}\\
|\newpage|\\
|\||fi|
\end{tabular}
\end{center}
%
Here one could write a message such as:
\begin{center}
|This is the part \childdocname{} of \childdocjob{}.|
\end{center}

%%%%%%%%%%%%%%%%%%%%%%%%%%%%%%%%%%%%%%%%%%%%%%%%%%%%%%%%%%%%%%%%%%%%%%%%%%%%%%%%
\subsection{Flags}
\label{sec:flags}

The package makes it easy to generate different versions
of the main or child documents.
To this end compilation flags can be defined
and assigned different default values.
They will be particularly useful in conjunction
with the forwarding mechanism described in \secref{sec:forward}.

For example, it may be useful to have a flag |\version|
which can be set to |draft| or |final|.
The document source will contain some conditional code
depending on the value of |\version|.
Suppose further, the flag should default to |final| for the main file
and to |draft| for child files
which is a natural assignment for editing the document.
This is achieved by placing the following code
in the preamble of the main document
(below the |\childdocmain| directive):
%
\begin{center}
\begin{tabular}{l}
|\ifchilddoc|\\
|\providecommand{\version}{draft}|\\
|\||else|\\
|\providecommand{\version}{final}|\\
|\||fi|
\end{tabular}
\end{center}
%
The definition by |\providecommand| makes sure
that previous definitions are not overwritten.
Further statements |\providecommand{\version}{...}|
can thus be added before the above code to override it.

For the main file, one might add a line
(between |\childdocmain| and the above block)
%
\begin{center}
|%\ifchilddoc\||else\providecommand{\version}{draft}\||fi|
\end{center}
%
which can be uncommented to produce a draft version.
Likewise one can add a line to the very top of a child file
(above the |\childdocof{|\textit{main}|}| directive)
%
\begin{center}
|%\providecommand{\version}{final}|
\end{center}
%
which can be uncommented to produce the final version of this child document.

%%%%%%%%%%%%%%%%%%%%%%%%%%%%%%%%%%%%%%%%%%%%%%%%%%%%%%%%%%%%%%%%%%%%%%%%%%%%%%%%
\subsection{Forwarding}
\label{sec:forward}

Different versions of the main or child documents
using compilation flags as described in \secref{sec:flags}
can be (permanently) stored in different files
for convenient compilation, viewing and distribution.
To this end, the package defines a command
to pass on compilation to a different file:

%%%%%%%%%%%%%%%%%%%%%%%%%%%%%%%%%%%%%%%%
\DescribeMacro{\childdocforward}
The command |\childdocforward| redirects processing to
another source file:
%
\begin{center}
\begin{tabular}{l}
|\input{childdoc.def}|\\
|\childdocforward[|\textit{main}|]{|\textit{dest}|}|\\
\end{tabular}
\end{center}
%
The argument \textit{dest} is the destination file
(without extension).
It should be the main file or one of the child files.
Note that further \textsf{childdoc} directives
such as |\childdocof| and |\childdocforward|
in the indicated file will be processed in this form.
The optional argument \textit{main}
passes on directly to the main file \textit{main}
while pretending to compile the child \textit{dest}.
This form behaves as if \textit{dest}
issues |\childdocof{|\textit{main}|}| right away,
and no further \textsf{childdoc} directives will be processed.

%%%%%%%%%%%%%%%%%%%%%%%%%%%%%%%%%%%%%%%%
\DescribeMacro{\...prefix}
In the alternative form |\childdocforwardprefix|,
%
\begin{center}
\begin{tabular}{l}
|\input{childdoc.def}|\\
|\childdocforwardprefix[|\textit{main}|]{|\textit{prefix}|}{|\textit{dest}|}|
\end{tabular}
\end{center}
%
the destination file is determined by a pattern
depending on the current file:
To make this work, the current file must be called
`{\textit{prefix}\hspace{0.2em}\textit{suffix}}'
with \textit{prefix} matching precisely the argument.
Processing is then passed on to the file
`{\textit{dest}\hspace{0.2em}\textit{suffix}}'.
Surely, the same effect is achieved by
directly specifying the
argument `{\textit{dest}\hspace{0.2em}\textit{suffix}}'
in the first form.
However, that requires to set up a different file
for each child. With the alternative form of the command
all these files can have exactly the same content
which simplifies setting them up and maintaining them.

For example, the following file |draft.tex|
with a compilation flag |\version| as described in \secref{sec:flags}
compiles the main document as a draft:
%
\begin{center}
\begin{tabular}{l}
|\def\version{draft}|\\
|\input{childdoc.def}|\\
|\childdocforward{|\textit{main}|}|
\end{tabular}
\end{center}
%
Likewise, the following files |final|\textit{nn}|.tex|
compile the final version of the child document
|child|\textit{nn}|.tex|:
%
\begin{center}
\begin{tabular}{l}
|\def\version{final}|\\
|\input{childdoc.def}|\\
|\childdocforwardprefix{final}{child}|
\end{tabular}
\end{center}
%

Note that when several versions of a main file and/or of each child file
are to be generated, it may be convenient to set up a |Makefile| or
shell script to automatise the process.

%%%%%%%%%%%%%%%%%%%%%%%%%%%%%%%%%%%%%%%%%%%%%%%%%%%%%%%%%%%%%%%%%%%%%%%%%%%%%%%%
\subsection{Command Line Processing}
\label{sec:commandline}

The effect of redirection files can also be achieved by invoking
the \LaTeX{} compiler with a more elaborate command line.
Most conveniently this should be done as part
of a shell script or a |Makefile|.

When using \textsf{childdoc} in the main file, the following
command lines effectively perform a redirection
(note that depending on the shell being used,
backslashes may have to be doubled: `|\|' $\to$ `|\\|'):
%
\begin{center}
|... -jobname "|\textit{target}|" |\\|"|[\textit{flags}]%
|\input{childdoc.def}\childdocforward[|\textit{main}|]{|\textit{dest}|}"|
\end{center}
%
Here \textit{target} is the name of the output file,
\textit{main} is the name of the main file
and \textit{dest} is the name of the main or child file to be processed
(all filenames without extensions).
The optional argument \textit{main} can be omitted
if \textit{main} matches \textit{dest}.
Optionally, compilation \textit{flags} can be defined via |\def| commands.
This command line makes the \TeX{} engine believe
it is compiling the file \textit{target}
whose content is specified as the latter parameter.
The provided code then forwards the processing to
\textit{main} or \textit{dest} as described in \secref{sec:forward}.

%%%%%%%%%%%%%%%%%%%%%%%%%%%%%%%%%%%%%%%%%%%%%%%%%%%%%%%%%%%%%%%%%%%%%%%%%%%%%%%%
\subsection{Include by Input}
\label{sec:input}

Including child documents by |\include| has some restrictions by design.
Most notably, the content of a child document always occupies
its own set of pages; pages cannot be shared between child documents.
Usually, this behaviour makes perfect sense
because each child document contain an essential part of the document.
However, in some situations it may be desirable to compose
a document from a collection of parts
without having mandatory page breaks between then.
For this case, the package
provides a mechanism to include parts
by |\input| which can also be processed individually.
However, by construction this mechanism
requires manual handling of the content to be output.

%%%%%%%%%%%%%%%%%%%%%%%%%%%%%%%%%%%%%%%%
\DescribeMacro{\ifchilddocmanual}
The main file should be prepared as usual, see \secref{sec:include}.
However, the document body must make a distinction
between processing of an individual part and of the main document, e.g.:
%
\begin{center}
\begin{tabular}{l}
|\ifchilddocmanual|\\
|\input{\childdocname}|\\
|\||else|\\
\textit{document body with }|\input{|\textit{part}|}|\\
|\||fi|
\end{tabular}
\end{center}
%
The conditional |\ifchilddocmanual| is true whenever
a part to be included by |\input| is being compiled,
and the name of the part is stored in |\childdocname|.

%%%%%%%%%%%%%%%%%%%%%%%%%%%%%%%%%%%%%%%%
\DescribeMacro{\childdocby}
Each part to be included by |\input| should start with:
%
\begin{center}
\begin{tabular}{l}
|\input{childdoc.def}|\\
|\childdocby{|\textit{main}|}|\\
\end{tabular}
\end{center}
%
The directive |\childdocby| is similar to |\childdocof|
described in \secref{sec:include},
but the subsequent selection of content must be done manually.
To that end, both |\ifchilddoc| and |\ifchilddocmanual|
will be true upon processing of a part,
and the name of the part is stored in |\childdocname|.
Note that |\jobname| will be set to the filename of the current part
so that each part receives an individual |.aux| file
that does not interfere with the |.aux| file(s) of the main document.
This behaviour can be altered by the alternative form
|\childdocby[*]{|\textit{main}|}| (with a non-empty optional argument)
which uses the |.aux| file of the main document
by setting |\jobname| to \textit{main}.

%%%%%%%%%%%%%%%%%%%%%%%%%%%%%%%%%%%%%%%%%%%%%%%%%%%%%%%%%%%%%%%%%%%%%%%%%%%%%%%%
\subsection{Driver Development}
\label{sec:driver}

The \textsf{childdoc} mechanism can also be use for the development
of definition files such as \LaTeX{} styles or classes.
This case differs from the above setup with multiple parts
included by |\include| in that no |\includeonly| should be invoked.
This can be achieved by starting the include file
(before |\ProvidesPackage|) with:
%
\begin{center}
\begin{tabular}{l}
|\input{childdoc.def}|\\
|\childdocforward{|\textit{main}|}|\\
\end{tabular}
\end{center}
%
or alternatively with:
%
\begin{center}
\begin{tabular}{l}
|\input{childdoc.def}|\\
|\childdocby{|\textit{main}|}|\\
\end{tabular}
\end{center}
%
Both forms have slightly different effects as described above.
The main file is prepared as usual, see \secref{sec:include}.

%%%%%%%%%%%%%%%%%%%%%%%%%%%%%%%%%%%%%%%%%%%%%%%%%%%%%%%%%%%%%%%%%%%%%%%%%%%%%%%%
\subsection{Legacy Detection}
\label{sec:detection}

The directive |\childdocmain| in the main file can detect
whether the complete document or merely a child is to be compiled
even without using the directive |\childdocof|.
This method is deprecated because it is less robust
and there is no compelling reason to use it;
it is merely provided for backward compatibility
and it may be removed in future versions.

If the detection mechanism is to be used,
it is mandatory to correctly specify
the filename of the main file as the argument of |\childdocmain|:
%
\begin{center}
\begin{tabular}{l}
|\input{childdoc.def}|\\
|\childdocmain{|\textit{main}|}|\\
\end{tabular}
\end{center}
%
If |\jobname| does not match the argument \textit{main} of |\childdocmain|,
it is assumed that |\jobname| points to the child file to be compiled.
When using |\childdocmain| with the main file specified as argument,
it suffices to start a child file
with just |\input{|\textit{main}|}|
without loading of the package and using |\childdocof|.
If instead all processing is done
with the appropriate \textsf{childdoc} directives,
the argument of \textit{main} of |\childdocmain| can be empty.

An alternative version of the command line processing described
in \secref{sec:commandline} using the detection mechanism reads:
%
\begin{center}
|... -jobname "|\textit{target}|" "|[\textit{flags}]%
[|\def\jobname{|\textit{dest}|}|]|\input{|\textit{main}|}"|
\end{center}

%%%%%%%%%%%%%%%%%%%%%%%%%%%%%%%%%%%%%%%%%%%%%%%%%%%%%%%%%%%%%%%%%%%%%%%%%%%%%%%%
\subsection{Manual Code}
\label{sec:manual}

In case one cannot be certain whether the definitions file |childdoc.def|
is installed on the target \TeX{} distribution
and one prefers not to ship it,
it is conceivable to paste a few relevant commands into the sources.

To that end, drop all statements |\input{childdoc.def}|
and perform the replacements as outlined below.
Instead of |\childdocmain{|\textit{main}|}| add the following code
to the top of the main file:
%
\begin{center}
\begin{tabular}{l}
|\||ifdefined\childdocname\endinput\||fi\newif\ifchilddoc|\\
|\edef\childdocname{\scantokens\expandafter{\jobname\noexpand}}|\\
|\def\childdocmain{|\textit{main}|}\||ifx\childdocmain\childdocname\||else|\\
|\childdoctrue\includeonly{\childdocname}\let\jobname\childdocmain\||fi|\\
\end{tabular}
\end{center}
%
Instead of |\childdocof{|\textit{main}|}| just include the main file
at the top of each child file:
%
\begin{center}
|\input{|\textit{main}|}|
\end{center}
%
A simple redirection |\childdocforward{|\textit{dest}|}| is achieved by:
%
\begin{center}
|\def\jobname{|\textit{dest}|}\input{\jobname}|
\end{center}
%
The redirection with prefix
|\childdocforwardprefix[|\textit{prefix}|]{|\textit{dest}|}|
is accomplished by:
%
\begin{center}
\begin{tabular}{l}
|{\edef\jobname{\scantokens\expandafter{\jobname\noexpand}}|\\
|\def\redirectjob |\textit{prefix}|#1~~~{\gdef\jobname{|\textit{dest}|#1}}|\\
|\expandafter\redirectjob\jobname~~~}\input{\jobname}|
\end{tabular}
\end{center}

In an alternative approach,
child documents can be compiled by a specific command line
without additional code or specific definitions:
%
\begin{center}
|... -jobname "|\textit{target}|" "|[\textit{flags}]%
|\includeonly{|\textit{dest}|}\input{|\textit{main}|}"|
\end{center}
%

%%%%%%%%%%%%%%%%%%%%%%%%%%%%%%%%%%%%%%%%%%%%%%%%%%%%%%%%%%%%%%%%%%%%%%%%%%%%%%%%
%%%%%%%%%%%%%%%%%%%%%%%%%%%%%%%%%%%%%%%%%%%%%%%%%%%%%%%%%%%%%%%%%%%%%%%%%%%%%%%%
\section{Information}

%%%%%%%%%%%%%%%%%%%%%%%%%%%%%%%%%%%%%%%%%%%%%%%%%%%%%%%%%%%%%%%%%%%%%%%%%%%%%%%%
\subsection{Copyright}

Copyright \copyright{} 2017--2018 Niklas Beisert

This work may be distributed and/or modified under the
conditions of the \LaTeX{} Project Public License, either version 1.3
of this license or (at your option) any later version.
The latest version of this license is in
  \url{http://www.latex-project.org/lppl.txt}
and version 1.3 or later is part of all distributions of \LaTeX{}
version 2005/12/01 or later.

This work has the LPPL maintenance status `maintained'.

The Current Maintainer of this work is Niklas Beisert.

This work consists of the files |README.txt|, |childdoc.ins| and |childdoc.dtx|
as well as the derived files |childdoc.def|, |cdocsamp.tex|
with |cdocsch1.tex|, |cdocsch2.tex|, |cdocspt3.tex|, |cdocspt4.tex|,
|cdocsdrf.tex|, |cdocsfn1.tex|, |cdocsfn2.tex|
as well as |childdoc.pdf|.

%%%%%%%%%%%%%%%%%%%%%%%%%%%%%%%%%%%%%%%%%%%%%%%%%%%%%%%%%%%%%%%%%%%%%%%%%%%%%%%%
\subsection{Files and Installation}

The package consists of the files:
%
\begin{center}
\begin{tabular}{ll}
    |README.txt|   & readme file \\
    |childdoc.ins| & installation file \\
    |childdoc.dtx| & source file \\
    |childdoc.def| & definition file \\
    |cdocsamp.tex| & sample main file \\
    |cdocsch1.tex| & sample include file \\
    |cdocsch2.tex| & sample include file \\
    |cdocspt3.tex| & sample part file \\
    |cdocspt4.tex| & sample part file \\
    |cdocsdrf.tex| & sample redirection file \\
    |cdocsfn1.tex| & sample redirection file \\
    |cdocsfn2.tex| & sample redirection file \\
    |childdoc.pdf| & manual
\end{tabular}
\end{center}
%
The distribution consists of the files
|README.txt|, |childdoc.ins| and |childdoc.dtx|.
%
\begin{itemize}
\item
Run (pdf)\LaTeX{} on |childdoc.dtx|
to compile the manual |childdoc.pdf| (this file).
\item
Run \LaTeX{} on |childdoc.ins| to create the definitions file |childdoc.def|
and the sample |cdocsamp.tex| with include files
|cdocsch1.tex|, |cdocsch2.tex|, |cdocspt3.tex|, |cdocspt4.tex|,
|cdocsdrf.tex|, |cdocsfn1.tex|, |cdocsfn2.tex|.
Then copy the file |childdoc.def| to an appropriate directory of your \LaTeX{}
distribution, e.g.\ \textit{texmf-root}|/tex/latex/childdoc|.
\end{itemize}

%%%%%%%%%%%%%%%%%%%%%%%%%%%%%%%%%%%%%%%%%%%%%%%%%%%%%%%%%%%%%%%%%%%%%%%%%%%%%%%%
\subsection{Related CTAN Packages}

There are several other packages which offer a similar functionality:
%
\begin{itemize}
\item
The packages
\href{http://ctan.org/pkg/docmute}{\textsf{docmute}},
\href{http://ctan.org/pkg/includex}{\textsf{includex}} and
\href{http://ctan.org/pkg/standalone}{\textsf{standalone}}
provide commands to include only the document body of
a child file thus allowing both files to be compiled individually.
\item
The packages \href{http://ctan.org/pkg/subdocs}{\textsf{subdocs}}
and \href{http://ctan.org/pkg/subfiles}{\textsf{subfiles}}
provide structures in which the main and child documents can be
encapsulated and allowing them to be compiled individually.
The inclusion mechanism is different from the conventional |\include|.
\item
The package \href{http://ctan.org/pkg/combine}{\textsf{combine}}
is an elaborate solution to combine several documents into one.
\end{itemize}
%
See also the CTAN topic \href{http://ctan.org/topic/subdocs}{\textsf{subdocs}}
for further related packages.
The present package differs from the above solutions in that
a document structure constructed with the conventional |\include| mechanism
just needs two extra commands at the top of every file
such that all constituent files can be compiled individually.

%%%%%%%%%%%%%%%%%%%%%%%%%%%%%%%%%%%%%%%%%%%%%%%%%%%%%%%%%%%%%%%%%%%%%%%%%%%%%%%%
%\subsection{Feature Suggestions}
%
%The following is a list of features which may be useful for future
%versions of this package:
%%
%\begin{itemize}
%\item
%\ldots
%\end{itemize}

%%%%%%%%%%%%%%%%%%%%%%%%%%%%%%%%%%%%%%%%%%%%%%%%%%%%%%%%%%%%%%%%%%%%%%%%%%%%%%%%
\subsection{Revision History}

%%%%%%%%%%%%%%%%%%%%%%%%%%%%%%%%%%%%%%%%
\paragraph{v2.0:} 2018/12/30

\begin{itemize}
\item
immediate forward processing
\item
added |\childdocby| mechanism
\item
manual restructured
\end{itemize}

%%%%%%%%%%%%%%%%%%%%%%%%%%%%%%%%%%%%%%%%
\paragraph{v1.6:} 2018/01/17

\begin{itemize}
\item
application for development of include files
\item
corrections to manual
\end{itemize}

%%%%%%%%%%%%%%%%%%%%%%%%%%%%%%%%%%%%%%%%
\paragraph{v1.5:} 2017/05/21

\begin{itemize}
\item
more complete structuring introduced
\item
|\childdocof| introduced
\item
|\childdoc| renamed to |\childdocmain|
\item
|\childredirect| renamed to |\childdocforward| and |\childdocforwardprefix|
and functionality expanded
\end{itemize}

%%%%%%%%%%%%%%%%%%%%%%%%%%%%%%%%%%%%%%%%
\paragraph{v1.0:} 2017/04/27

\begin{itemize}
\item
manual and install package
\item
first version published on CTAN
\end{itemize}

%%%%%%%%%%%%%%%%%%%%%%%%%%%%%%%%%%%%%%%%
\paragraph{v0.6:} 2017/04/26

\begin{itemize}
\item
redirection mechanism added
\end{itemize}

%%%%%%%%%%%%%%%%%%%%%%%%%%%%%%%%%%%%%%%%
\paragraph{v0.5:} 2017/04/26

\begin{itemize}
\item
functionality in definition file
\end{itemize}


%%%%%%%%%%%%%%%%%%%%%%%%%%%%%%%%%%%%%%%%%%%%%%%%%%%%%%%%%%%%%%%%%%%%%%%%%%%%%%%%
%%%%%%%%%%%%%%%%%%%%%%%%%%%%%%%%%%%%%%%%%%%%%%%%%%%%%%%%%%%%%%%%%%%%%%%%%%%%%%%%
%%%%%%%%%%%%%%%%%%%%%%%%%%%%%%%%%%%%%%%%%%%%%%%%%%%%%%%%%%%%%%%%%%%%%%%%%%%%%%%%
\appendix

\settowidth\MacroIndent{\rmfamily\scriptsize 000\ }

 \DocInput{childdoc.dtx}

\end{document}
%</driver>
% \fi
%
% %%%%%%%%%%%%%%%%%%%%%%%%%%%%%%%%%%%%%%%%%%%%%%%%%%%%%%%%%%%%%%%%%%%%%%%%%%%%%%
% %%%%%%%%%%%%%%%%%%%%%%%%%%%%%%%%%%%%%%%%%%%%%%%%%%%%%%%%%%%%%%%%%%%%%%%%%%%%%%
% \section{Sample}
%\iffalse
%<*samplemain>
%\fi
%
% The following presents a sample document
% with two chapters, two parts, a title page,
% a compile flag as well as three forwarding files to set the flag.
% It consists of eight |.tex| files:
% \begin{center}
% \begin{tabular}{ll}
% |cdocsamp.tex|&main file\\
% |cdocsch1.tex|&include file for chapter 1\\
% |cdocsch2.tex|&include file for chapter 2\\
% |cdocspt3.tex|&include file for part 3\\
% |cdocspt4.tex|&include file for part 4\\
% |cdocsdrf.tex|&forwarding file for main file in draft mode\\
% |cdocsfi1.tex|&forwarding file for final version of chapter 1\\
% |cdocsfi2.tex|&forwarding file for final version of chapter 2\\
% \end{tabular}
% \end{center}
% Each of the eight files can be compiled directly by the \LaTeX{} compiler.
%
% %%%%%%%%%%%%%%%%%%%%%%%%%%%%%%%%%%%%%%
% \paragraph{Main File.}
%
% The main file is called |cdocsamp.tex|.
%
% Load the \textsf{childdoc} definitions and
% declare the filename for the main document:
%    \begin{macrocode}
\input{childdoc.def}
\childdocmain{}
%    \end{macrocode}

% Optional override for |\version| flag:
%    \begin{macrocode}
%%\ifchilddoc\else\providecommand{\version}{draft}\fi
%    \end{macrocode}

% Define the default values for the |\version| flag
% (|final| for the main file and |draft| for childs):
%    \begin{macrocode}
\ifchilddoc
\providecommand{\version}{draft}
\else
\providecommand{\version}{final}
\fi
%    \end{macrocode}

% Load the standard document class:
%    \begin{macrocode}
\documentclass[12pt]{article}
%    \end{macrocode}

% Start the document body:
%    \begin{macrocode}
\begin{document}
%    \end{macrocode}

% Declare a title page.
% Print title, part of document being processed and version flag:
%    \begin{macrocode}
\addtocounter{page}{-1}
\begin{center}
{\LARGE\bfseries{}childdoc example\par}
\vspace{1cm}
\ifchilddoc
\ifchilddocmanual part\else chapter\fi:
`\childdocname' of `\childdocjob'\par
\else
main document: `\childdocjob'\par
\fi
version: \version\par
\end{center}
\newpage
%    \end{macrocode}

% Manually include selected file,
% otherwise process as usual:
%    \begin{macrocode}
\ifchilddocmanual
\section*{part `\childdocname'}
\input{\childdocname}
\else
%    \end{macrocode}

% Include the two chapters:
%    \begin{macrocode}
\include{cdocsch1}
\include{cdocsch2}
%    \end{macrocode}

% Include the two parts unless only chapters should be displayed:
%    \begin{macrocode}
\ifchilddoc\else
\section{part three}
\input{cdocspt3}
\section{part four}
\input{cdocspt4}
\fi
%    \end{macrocode}

% Process as usual until here:
%    \begin{macrocode}
\fi
%    \end{macrocode}

% End of document body:
%    \begin{macrocode}
\end{document}
%    \end{macrocode}
%\iffalse
%</samplemain>
%\fi
%
% %%%%%%%%%%%%%%%%%%%%%%%%%%%%%%%%%%%%%%
% \paragraph{Chapter Include Files.}
%
% The include files are called |cdocsch1.tex| and |cdocsch2.tex|.
%
%\iffalse
%<*samplechap1|samplechap2>
%\fi

% Optional override for |\version| flag:
%    \begin{macrocode}
%%\providecommand{\version}{final}
%    \end{macrocode}

% Include the main document:
%    \begin{macrocode}
\input{childdoc.def}
\childdocof{cdocsamp}
%    \end{macrocode}

%\iffalse
%</samplechap1|samplechap2>
%\fi
%
%\iffalse
%<*samplechap1>
%\fi
% Some text for chapter 1:
%    \begin{macrocode}
\section{one}
some text in chapter one
%    \end{macrocode}

%\iffalse
%</samplechap1>
%\fi
% Some text for chapter 2:
%\iffalse
%<*samplechap2>
%\fi
%    \begin{macrocode}
\section{two}
more text in chapter two
%    \end{macrocode}

%\iffalse
%</samplechap2>
%\fi
%
% %%%%%%%%%%%%%%%%%%%%%%%%%%%%%%%%%%%%%%
% \paragraph{Part Include Files.}
%
% The include files are called |cdocspt3.tex| and |cdocspt4.tex|.
%
%\iffalse
%<*samplepart3|samplepart4>
%\fi

% Optional override for |\version| flag:
%    \begin{macrocode}
%%\providecommand{\version}{final}
%    \end{macrocode}

% Include the main document:
%    \begin{macrocode}
\input{childdoc.def}
\childdocby{cdocsamp}
%    \end{macrocode}

%\iffalse
%</samplepart3|samplepart4>
%\fi
%
%\iffalse
%<*samplepart3>
%\fi
% Some text for part 3:
%    \begin{macrocode}
some text in part three
%    \end{macrocode}

%\iffalse
%</samplepart3>
%\fi
% Some text for part 4:
%\iffalse
%<*samplepart4>
%\fi
%    \begin{macrocode}
more text in part four
%    \end{macrocode}

%\iffalse
%</samplepart4>
%\fi
%
% %%%%%%%%%%%%%%%%%%%%%%%%%%%%%%%%%%%%%%
% \paragraph{Forwarding for a Complete Draft.}
%
% The following forwarding file |cdocsdrf.tex|
% compiles the main document in draft mode:
%\iffalse
%<*sampledraft>
%\fi
%    \begin{macrocode}
\def\version{draft}
\input{childdoc.def}
\childdocforward{cdocsamp}
%    \end{macrocode}

%\iffalse
%</sampledraft>
%\fi
%
% %%%%%%%%%%%%%%%%%%%%%%%%%%%%%%%%%%%%%%
% \paragraph{Forwarding for Final Version of the Chapters.}
%
% The following forwarding files |cdocsfn1.tex| and |cdocsfn2.tex|
% (with identical content)
% compile the final versions of the child documents
% |cdocsch1.tex| and |cdocsch2.tex|, respectively:
%\iffalse
%<*samplefinal>
%\fi
%    \begin{macrocode}
\def\version{final}
\input{childdoc.def}
\childdocforwardprefix[cdocsamp]{cdocsfn}{cdocsch}
%    \end{macrocode}

%\iffalse
%</samplefinal>
%\fi
%
% %%%%%%%%%%%%%%%%%%%%%%%%%%%%%%%%%%%%%%
% \paragraph{Command Line Processing.}
%
% The following three command lines generate the output files
% |cdocscld|, |cdocscl1| and |cdocscl2|
% which should be identical to
% |cdocsdrf|, |cdocsch1| and |cdocsfn2|, respectively:
% \begin{center}
% \begin{tabular}{l}
% |latex -jobname cdocscld \|\\
% |  "\def\version{draft}\input{childdoc.def}\childdocforward{cdocsamp}"|\\
% |latex -jobname cdocscl1 \|\\
% |  "\input{childdoc.def}\childdocforward[cdocsamp]{cdocsch1}"|\\
% |latex -jobname cdocscl2 \|\\
% |  "\def\version{final}\input{childdoc.def}\childdocforward{cdocsch2}"|
% \end{tabular}
% \end{center}
% Note that the trailing backslash on each first line
% merely continues the input to the second line
% (for convenient cut ant paste).
% Furthermore, the command |latex| can be replaced by any
% of its alternative versions such as |pdflatex|.
%
% %%%%%%%%%%%%%%%%%%%%%%%%%%%%%%%%%%%%%%%%%%%%%%%%%%%%%%%%%%%%%%%%%%%%%%%%%%%%%%
% %%%%%%%%%%%%%%%%%%%%%%%%%%%%%%%%%%%%%%%%%%%%%%%%%%%%%%%%%%%%%%%%%%%%%%%%%%%%%%
% \section{Implementation}
%\iffalse
%<*package>
%\fi
%
% This section describes the definitions file |childdoc.def|.

% The definitions cannot be loaded using |\usepackage| or |\RequirePackage|
% which has a mechanism to prevent loading a style file more than once.
% When loading the definitions by means of |\input|
% multiple instances have to be prevented manually:
%\iffalse
%This code needs to be before the `\ProvidesFile' directive
%which is defined at the beginning of this file.
%Therefore it is also placed there and commented out here.
%</package>
%<*discard>
%\fi
%    \begin{macrocode}
\ifdefined\childdocmain\endinput\fi
%    \end{macrocode}
%\iffalse
%</discard>
%<*package>
%\fi
%
% \macro{\ifchilddoc}
% \macro{\ifchilddocmanual}
% The conditional |\ifchilddoc| tells whether a
% child (true) or main (false) document is being compiled.
% The conditional |\ifchilddocmanual| tells whether
% the |\includeonly| mechanism is used (false) or
% the selection of child files must be performed manually (true).
% The definitions initialise to false:
%    \begin{macrocode}
\newif\ifchilddoc
\newif\ifchilddocmanual
%    \end{macrocode}

% \macro{\childdocname}
% \macro{\childdocjob}
% The macro |\childdocname| stores the name of the main document
% to be compiled. The macro |\childdocjob| stores the name of
% the document on which the \LaTeX{} compiler was originally invoked.
% The content of |\jobname| cannot be compared
% to filenames specified in the source due to different catcodes.
% The following code rescans |\jobname|, stores the result
% in |\childdocname| and saves a copy in |\childdocjob|:
%    \begin{macrocode}
\edef\childdocname{\scantokens\expandafter{\jobname\noexpand}}
\let\childdocjob\childdocname
%    \end{macrocode}

% \macro{\childdocdisable}
% The macro |\childdocdisable| prevents the main file
% from being processed more than once.
% At this stage, the main document command |\childdocmain|
% is assumed to be called once again where it should do nothing.
% Any subsequent call to it should prevent
% a secondary processing of the main document
% It overwrites the forwarding commands
% |\childdocof| and |\childdocforward|
% with empty macros to prevent further inclusions of the main document:
%    \begin{macrocode}
\newcommand{\childdocdisable}
{
  \renewcommand{\childdocmain}[1]{\renewcommand{\childdocmain}[1]{\endinput}}
  \renewcommand{\childdocof}[1]{}
  \renewcommand{\childdocby}[2][]{}
  \renewcommand{\childdocforward}[2][]{}
  \renewcommand{\childdocdisable}{}
}
%    \end{macrocode}

% \macro{\childdocmain}
% The macro |\childdocmain| is to be called at the top of the main file
% with nothing or the main filename (without extension) as argument.
% First, it breaks loops.
% If the argument is not empty and does not match |\childdocname|
% (which is set by the first inclusion of |childdoc.def|),
% |\ifchilddoc| is set to true, |\includeonly| is applied to the child file
% and |\jobname| is set to the main file
% (for proper handling of |.aux| files):
%    \begin{macrocode}
\newcommand{\childdocmain}[1]
{
  \childdocdisable\childdocmain{}
  \if?#1?\else
    \begingroup
      \def\childdoctmp{#1}
      \ifx\childdoctmp\childdocname
        \def\childdoctmp{}
      \else
        \def\childdoctmp
        {
          \childdoctrue
          \includeonly{\childdocname}
          \def\childdocjob{#1}
          \def\jobname{#1}
        }
      \fi
      \expandafter
    \endgroup
    \childdoctmp
  \fi
}
%    \end{macrocode}

% \macro{\childdocof}
% The command |\childdocof| redirects
% compilation to the main file |#1|.
%    \begin{macrocode}
\newcommand{\childdocof}[1]
{
  \childdocdisable
  \childdoctrue
  \includeonly{\childdocname}
  \def\jobname{#1}
  \def\childdocjob{#1}
  \input{#1}
}
%    \end{macrocode}

% \macro{\childdocby}
% The command |\childdocby| ....
%    \begin{macrocode}
\newcommand{\childdocby}[2][]
{
  \childdocdisable
  \childdoctrue
  \childdocmanualtrue
  \if?#1?\else
    \def\jobname{#2}
  \fi
  \def\childdocjob{#2}
  \input{#2}
  \endinput
}
%    \end{macrocode}

% \macro{\childdocforward}
% The command |\childdocforward| redirects
% compilation to the main file or
% (if the optional argument is given) a child file.
% Parameters are set as if the main file
% or a child file starting with |\childdocof| was compiled.
% Then compilation is handed over to the main file:
%    \begin{macrocode}
\newcommand{\childdocforward}[2][]
{
  \begingroup
    \if?#1?
      \def\childdoctmp
      {
        \def\childdocname{#2}
        \def\childdocjob{#2}
        \def\jobname{#2}
        \input{#2}
        \endinput
      }
    \else
      \def\childdoctmp
      {
        \childdocdisable
        \def\childdocname{#2}
        \childdoctrue
        \includeonly{#2}
        \def\childdocjob{#1}
        \def\jobname{#1}
        \input{#1}
        \endinput
      }
    \fi
    \expandafter
  \endgroup
  \childdoctmp
}
%    \end{macrocode}

% \macro{\childdocforwardprefix}
% The command |\childdocforwardprefix| redirects
% compilation to the main or a child file by means of a pattern.
% The prefix |#1| in the current filename is replaced by |#2|
% and the suffix of the current filename is kept
% (it is assumed that the filename does not contain the substring `|~~~|'
% which is used as a delimiter).
% Compilation is handed over to the new file by |\childdocforward|:
%    \begin{macrocode}
\newcommand{\childdocforwardprefix}[3][]
{
  \begingroup
    \def\childdocextract #2##1~~~{\def\childdoctmp{\childdocforward[#1]{#3##1}}}
    \expandafter\childdocextract\childdocname~~~
    \expandafter
  \endgroup
  \childdoctmp
}
%    \end{macrocode}

% \macro{\childdoc}
% The deprecated macro |\childdoc| is a legacy version of |\childdocmain|:
%    \begin{macrocode}
\newcommand{\childdoc}{\childdocmain}
%    \end{macrocode}

% \macro{\childdocredirect}
% The deprecated macro |\childdocredirect| is a legacy version
% of |\childdocforward| and |\childdocforwardprefix|:
%    \begin{macrocode}
\newcommand{\childdocredirect}[2][]
{
  \begingroup
    \if?#1?
      \def\childdoctmp{\childdocforward{#2}}
    \else
      \def\childdoctmp{\childdocforwardprefix{#1}{#2}}
    \fi
    \expandafter
  \endgroup
  \childdoctmp
}
%    \end{macrocode}

%\iffalse
%</package>
%\fi
%
\endinput
\childdocforward[|\textit{main}|]{|\textit{dest}|}"|
\end{center}
%
Here \textit{target} is the name of the output file,
\textit{main} is the name of the main file
and \textit{dest} is the name of the main or child file to be processed
(all filenames without extensions).
The optional argument \textit{main} can be omitted
if \textit{main} matches \textit{dest}.
Optionally, compilation \textit{flags} can be defined via |\def| commands.
This command line makes the \TeX{} engine believe
it is compiling the file \textit{target}
whose content is specified as the latter parameter.
The provided code then forwards the processing to
\textit{main} or \textit{dest} as described in \secref{sec:forward}.

%%%%%%%%%%%%%%%%%%%%%%%%%%%%%%%%%%%%%%%%%%%%%%%%%%%%%%%%%%%%%%%%%%%%%%%%%%%%%%%%
\subsection{Include by Input}
\label{sec:input}

Including child documents by |\include| has some restrictions by design.
Most notably, the content of a child document always occupies
its own set of pages; pages cannot be shared between child documents.
Usually, this behaviour makes perfect sense
because each child document contain an essential part of the document.
However, in some situations it may be desirable to compose
a document from a collection of parts
without having mandatory page breaks between then.
For this case, the package
provides a mechanism to include parts
by |\input| which can also be processed individually.
However, by construction this mechanism
requires manual handling of the content to be output.

%%%%%%%%%%%%%%%%%%%%%%%%%%%%%%%%%%%%%%%%
\DescribeMacro{\ifchilddocmanual}
The main file should be prepared as usual, see \secref{sec:include}.
However, the document body must make a distinction
between processing of an individual part and of the main document, e.g.:
%
\begin{center}
\begin{tabular}{l}
|\ifchilddocmanual|\\
|\input{\childdocname}|\\
|\||else|\\
\textit{document body with }|\input{|\textit{part}|}|\\
|\||fi|
\end{tabular}
\end{center}
%
The conditional |\ifchilddocmanual| is true whenever
a part to be included by |\input| is being compiled,
and the name of the part is stored in |\childdocname|.

%%%%%%%%%%%%%%%%%%%%%%%%%%%%%%%%%%%%%%%%
\DescribeMacro{\childdocby}
Each part to be included by |\input| should start with:
%
\begin{center}
\begin{tabular}{l}
|% \iffalse
%
% childdoc.dtx Copyright (C) 2017-2018 Niklas Beisert
%
% This work may be distributed and/or modified under the
% conditions of the LaTeX Project Public License, either version 1.3
% of this license or (at your option) any later version.
% The latest version of this license is in
%   http://www.latex-project.org/lppl.txt
% and version 1.3 or later is part of all distributions of LaTeX
% version 2005/12/01 or later.
%
% This work has the LPPL maintenance status `maintained'.
%
% The Current Maintainer of this work is Niklas Beisert.
%
% This work consists of the files childdoc.dtx and childdoc.ins
% and the derived files childdoc.def and cdocsamp.tex with
% cdocsch1.tex, cdocsch2.tex, cdocsdrf.tex, cdocsfn1.tex, cdocsfn2.tex.
%
%<package>\ifdefined\childdocmain\endinput\fi
%<package>\ProvidesFile{childdoc.def}[2018/12/30 v2.0 child document driver]
%<samplemain>\ProvidesFile{cdocsamp.tex}[2018/12/30 v2.0 sample for childdoc]
%<*driver>
%\ProvidesFile{childdoc.drv}[2018/12/30 v2.0 childdoc reference manual file]
\PassOptionsToClass{10pt,a4paper}{article}
\documentclass{ltxdoc}

\usepackage[margin=35mm]{geometry}
\usepackage{hyperref}
\usepackage{hyperxmp}
\usepackage[usenames]{color}

\hypersetup{colorlinks=true}
\hypersetup{pdfstartview=FitH}
\hypersetup{pdfpagemode=UseNone}
\hypersetup{pdfsource={}}
\hypersetup{pdflang={en-UK}}
\hypersetup{pdfcopyright={Copyright 2017-2018 Niklas Beisert.
  This work may be distributed and/or modified under the
  conditions of the LaTeX Project Public License, either version 1.3
  of this license or (at your option) any later version.}}
\hypersetup{pdflicenseurl={http://www.latex-project.org/lppl.txt}}
\hypersetup{pdfcontactaddress={ETH Zurich, ITP, HIT K,
  Wolfgang-Pauli-Strasse 27}}
\hypersetup{pdfcontactpostcode={8093}}
\hypersetup{pdfcontactcity={Zurich}}
\hypersetup{pdfcontactcountry={Switzerland}}
\hypersetup{pdfcontactemail={nbeisert@itp.phys.ethz.ch}}
\hypersetup{pdfcontacturl={http://people.phys.ethz.ch/\xmptilde nbeisert/}}

\newcommand{\secref}[1]{\hyperref[#1]{section \ref*{#1}}}

\parskip1ex
\parindent0pt
\let\olditemize\itemize
\def\itemize{\olditemize\parskip0pt}

\begin{document}

\title{The \textsf{childdoc} Package}
\hypersetup{pdftitle={The childdoc Package}}
\author{Niklas Beisert\\[2ex]
  Institut f\"ur Theoretische Physik\\
  Eidgen\"ossische Technische Hochschule Z\"urich\\
  Wolfgang-Pauli-Strasse 27, 8093 Z\"urich, Switzerland\\[1ex]
  \href{mailto:nbeisert@itp.phys.ethz.ch}
  {\texttt{nbeisert@itp.phys.ethz.ch}}}
\hypersetup{pdfauthor={Niklas Beisert}}
\hypersetup{pdfsubject={Manual for the LaTeX2e Package childdoc}}
\date{30 December 2018, \textsf{v2.0}}
\maketitle

\begin{abstract}\noindent
\textsf{childdoc} is a \LaTeXe{} package
that enables the direct compilation
of document sections included by |\include|
to individual files.
\end{abstract}

\begingroup
\parskip0ex
\tableofcontents
\endgroup

%%%%%%%%%%%%%%%%%%%%%%%%%%%%%%%%%%%%%%%%%%%%%%%%%%%%%%%%%%%%%%%%%%%%%%%%%%%%%%%%
%%%%%%%%%%%%%%%%%%%%%%%%%%%%%%%%%%%%%%%%%%%%%%%%%%%%%%%%%%%%%%%%%%%%%%%%%%%%%%%%
\section{Introduction}

\LaTeX{} provides a mechanism to structure a large document (such as a book)
into a main file and several child files (containing the chapters)
using the |\include| command.
This mechanism is beneficial for documents
which span hundreds of pages in order to
make the source file(s) more manageable.
Moreover, compilation can be restricted to
selected child files by means of the |\includeonly| command.
The latter feature can be used to reduce the compilation time while editing
(this was significantly more useful in the earlier days of \LaTeX{})
or to generate a smaller document which is easier to navigate.
Another application of |\includeonly| is to generate
documents consisting of selected parts of the complete document.

However, there are a few drawbacks of the plain |\include| mechanism:
\begin{itemize}
\item
The child files cannot be compiled on their own,
they can only be compiled via the main file.
A naive editing environment
(such as a text editor with an option
to have the current file processed by \LaTeX)
may require one to switch to the main file before compiling;
attempting to compile the child file produces errors.
\item
The main file must be modified (each time)
to adjust the |\includeonly| command
to the present needs. This easily leaves the main file in a messy state.
\item
The generated document will always carry the filename
of the main document. This is inconvenient if
several child files are to be compiled and
to be kept for distribution.
\end{itemize}

The present package provides a simple interface
to make child files individually compilable by \LaTeX{}.
Compiling a child file then has the same effect as compiling
the main file with an |\includeonly| command
to select the appropriate child.
Moreover the generated document will carry the name of the child
rather than the main file.
This resolves all three above issues.

This feature is meant to make the editing of books,
thesis documents and lecture notes somewhat more convenient.
However, the package can also be used efficiently for
composing a series of documents (such as exercise sheets)
which are typically distributed individually.
It then assists the author in generating the individual documents
(potentially in different versions)
as well as a document containing the collected series.
Another application is in developing style files
or other kinds of included material
where compilation of the style file could redirect
to a sample or test file.

%%%%%%%%%%%%%%%%%%%%%%%%%%%%%%%%%%%%%%%%%%%%%%%%%%%%%%%%%%%%%%%%%%%%%%%%%%%%%%%%
%%%%%%%%%%%%%%%%%%%%%%%%%%%%%%%%%%%%%%%%%%%%%%%%%%%%%%%%%%%%%%%%%%%%%%%%%%%%%%%%
\section{Usage}

First of all, the package \textsf{childdoc} is \emph{not} a standard
\LaTeXe{} |.sty| style file! Therefore it needs to be invoked in
a non-standard way.

%%%%%%%%%%%%%%%%%%%%%%%%%%%%%%%%%%%%%%%%%%%%%%%%%%%%%%%%%%%%%%%%%%%%%%%%%%%%%%%%
\subsection{Included Files}
\label{sec:include}

%%%%%%%%%%%%%%%%%%%%%%%%%%%%%%%%%%%%%%%%
\DescribeMacro{\childdocmain}
To use the package, add the commands
\begin{center}
\begin{tabular}{l}
|\input{childdoc.def}|\\
|\childdocmain{}|\\
\end{tabular}
\end{center}
at the very top of the main \LaTeX{} file,
in particular \emph{before} the |\documentclass| statement!
The argument of |\childdocmain| should be left empty
(but it must be present).

%%%%%%%%%%%%%%%%%%%%%%%%%%%%%%%%%%%%%%%%
\DescribeMacro{\childdocof}
Furthermore, add the commands
\begin{center}
\begin{tabular}{l}
|\input{childdoc.def}|\\
|\childdocof{|\textit{main}|}|\\
\end{tabular}
\end{center}
at the top of every child file \textit{child}
which is included by |\include{|\textit{child}|}|
from within the main file
(or at least for those files to be compiled individually).
The argument \textit{main} must be the filename of the main file.

There are a couple of
considerations in setting up the main and child documents:

%%%%%%%%%%%%%%%%%%%%%%%%%%%%%%%%%%%%%%%%
\paragraph{Restrictions.}

Please note the following restrictions:
\begin{itemize}
\item
|\childdocmain| must be called with one argument \textit{main}
to ensure compatibility with earlier version of the package.
It must either be empty (|\childdocmain{}|)
or precisely match the filename of the main file in which it is specified.
See \secref{sec:detection} for further information.
\item
The filename \textit{main} must be specified without the |.tex| extension.
\item
The filename \textit{main} is case sensitive
(even in case-insensitive file systems)
due to internal string comparison.
\item
The argument \textit{main} should be fully expanded, it cannot be a macro.
\item
Subdirectories and special characters should be avoided in filenames.
\item
The command |\childdocmain{|\textit{main}|}| must be followed by a whitespace.
It should not be followed immediately by another command
or by a comment mark `|%|'.
This is because the \TeX{} parser reads the token immediately following
the argument of |\childdocmain| and puts it
at the beginning of every child section;
however, a white\-space is ignored.
\end{itemize}

%%%%%%%%%%%%%%%%%%%%%%%%%%%%%%%%%%%%%%%%
\paragraph{Content of Main File.}

It is advisable to place all content in the child files included by |\include|.
Any output contained in the main file will appear in all child documents
unless suppressed manually;
it cannot be suppressed automatically by the |\includeonly| directive
and thus should normally be avoided.
A method to include some content in the main file
by means of conditional processing is described in \secref{sec:conditional}.

%%%%%%%%%%%%%%%%%%%%%%%%%%%%%%%%%%%%%%%%
\paragraph{Page Numbering.}

When only a part of the document is compiled,
the appropriate numbering of pages
(as well as other status parameters)
is determined from the |.aux| files.
The latter contain information from previous passes.
However this information needs to propagate through
all intermediate child documents.
Therefore the page numbering in child documents may well
be inconsistent until the complete document is compiled at least once.

A useful (if unconventional) way to always ensure a consistent
page numbering is to restart the numbering in each child document
and denote the pages by `\textit{child}|.|\textit{page}'
where \textit{child} represents the chapter/section number of the child file.
This can be achieved by the command
|\numberwithin{page}{|\textit{child}|}|
of the \textsf{amsmath} package
where \textit{child} can be |chapter| or |section|
depending on the chosen structuring.
Alternatively, one can modify the macro |\thepage| appropriately
and reset the counter |page| at the start of each child file.

%%%%%%%%%%%%%%%%%%%%%%%%%%%%%%%%%%%%%%%%%%%%%%%%%%%%%%%%%%%%%%%%%%%%%%%%%%%%%%%%
\subsection{Conditional Processing}
\label{sec:conditional}

The package provides a mechanism to compile different versions
of a document. To customise the versions further some conditional processing
can come in handy to distinguish which version is being compiled.
The package provides two macros to describe the compilation context:

%%%%%%%%%%%%%%%%%%%%%%%%%%%%%%%%%%%%%%%%
\DescribeMacro{\ifchilddoc}
The conditional |\ifchilddoc| distinguishes between the compilation of
child documents and the main document:
%
\begin{center}
|\ifchilddoc |\textit{child-code}| |[|\||else |\textit{main-code}]| \||fi|
\end{center}

%%%%%%%%%%%%%%%%%%%%%%%%%%%%%%%%%%%%%%%%
\DescribeMacro{\childdocname}
\DescribeMacro{\childdocjob}
The macro |\childdocname| contains the filename (without extension)
of the main or child file being processed.
Note that |\childdocjob| will always contain the name of the main file.

%%%%%%%%%%%%%%%%%%%%%%%%%%%%%%%%%%%%%%%%
\paragraph{Title Page.}

Conditional processing can be used to include a title or banner page
in the main document when proper precautions are taken.
Importantly, the code in the main file should ensure that the page counter
(as well as other status parameters which are stored in the |.aux| files)
takes the same value after the conditional processing.
Otherwise the page numbers may take divergent values
depending on which part is compiled.

For example, a title page could be declared by:
%
\begin{center}
\begin{tabular}{l}
|\ifchilddoc\||else|\\
|\addtocounter{page}{-1}|\\
\textit{code for title page}\\
|\newpage|\\
|\||fi|
\end{tabular}
\end{center}
%
A banner page for the child documents can be generated by:
%
\begin{center}
\begin{tabular}{l}
|\ifchilddoc|\\
|\addtocounter{page}{-1}|\\
\textit{code for banner page}\\
|\newpage|\\
|\||fi|
\end{tabular}
\end{center}
%
Here one could write a message such as:
\begin{center}
|This is the part \childdocname{} of \childdocjob{}.|
\end{center}

%%%%%%%%%%%%%%%%%%%%%%%%%%%%%%%%%%%%%%%%%%%%%%%%%%%%%%%%%%%%%%%%%%%%%%%%%%%%%%%%
\subsection{Flags}
\label{sec:flags}

The package makes it easy to generate different versions
of the main or child documents.
To this end compilation flags can be defined
and assigned different default values.
They will be particularly useful in conjunction
with the forwarding mechanism described in \secref{sec:forward}.

For example, it may be useful to have a flag |\version|
which can be set to |draft| or |final|.
The document source will contain some conditional code
depending on the value of |\version|.
Suppose further, the flag should default to |final| for the main file
and to |draft| for child files
which is a natural assignment for editing the document.
This is achieved by placing the following code
in the preamble of the main document
(below the |\childdocmain| directive):
%
\begin{center}
\begin{tabular}{l}
|\ifchilddoc|\\
|\providecommand{\version}{draft}|\\
|\||else|\\
|\providecommand{\version}{final}|\\
|\||fi|
\end{tabular}
\end{center}
%
The definition by |\providecommand| makes sure
that previous definitions are not overwritten.
Further statements |\providecommand{\version}{...}|
can thus be added before the above code to override it.

For the main file, one might add a line
(between |\childdocmain| and the above block)
%
\begin{center}
|%\ifchilddoc\||else\providecommand{\version}{draft}\||fi|
\end{center}
%
which can be uncommented to produce a draft version.
Likewise one can add a line to the very top of a child file
(above the |\childdocof{|\textit{main}|}| directive)
%
\begin{center}
|%\providecommand{\version}{final}|
\end{center}
%
which can be uncommented to produce the final version of this child document.

%%%%%%%%%%%%%%%%%%%%%%%%%%%%%%%%%%%%%%%%%%%%%%%%%%%%%%%%%%%%%%%%%%%%%%%%%%%%%%%%
\subsection{Forwarding}
\label{sec:forward}

Different versions of the main or child documents
using compilation flags as described in \secref{sec:flags}
can be (permanently) stored in different files
for convenient compilation, viewing and distribution.
To this end, the package defines a command
to pass on compilation to a different file:

%%%%%%%%%%%%%%%%%%%%%%%%%%%%%%%%%%%%%%%%
\DescribeMacro{\childdocforward}
The command |\childdocforward| redirects processing to
another source file:
%
\begin{center}
\begin{tabular}{l}
|\input{childdoc.def}|\\
|\childdocforward[|\textit{main}|]{|\textit{dest}|}|\\
\end{tabular}
\end{center}
%
The argument \textit{dest} is the destination file
(without extension).
It should be the main file or one of the child files.
Note that further \textsf{childdoc} directives
such as |\childdocof| and |\childdocforward|
in the indicated file will be processed in this form.
The optional argument \textit{main}
passes on directly to the main file \textit{main}
while pretending to compile the child \textit{dest}.
This form behaves as if \textit{dest}
issues |\childdocof{|\textit{main}|}| right away,
and no further \textsf{childdoc} directives will be processed.

%%%%%%%%%%%%%%%%%%%%%%%%%%%%%%%%%%%%%%%%
\DescribeMacro{\...prefix}
In the alternative form |\childdocforwardprefix|,
%
\begin{center}
\begin{tabular}{l}
|\input{childdoc.def}|\\
|\childdocforwardprefix[|\textit{main}|]{|\textit{prefix}|}{|\textit{dest}|}|
\end{tabular}
\end{center}
%
the destination file is determined by a pattern
depending on the current file:
To make this work, the current file must be called
`{\textit{prefix}\hspace{0.2em}\textit{suffix}}'
with \textit{prefix} matching precisely the argument.
Processing is then passed on to the file
`{\textit{dest}\hspace{0.2em}\textit{suffix}}'.
Surely, the same effect is achieved by
directly specifying the
argument `{\textit{dest}\hspace{0.2em}\textit{suffix}}'
in the first form.
However, that requires to set up a different file
for each child. With the alternative form of the command
all these files can have exactly the same content
which simplifies setting them up and maintaining them.

For example, the following file |draft.tex|
with a compilation flag |\version| as described in \secref{sec:flags}
compiles the main document as a draft:
%
\begin{center}
\begin{tabular}{l}
|\def\version{draft}|\\
|\input{childdoc.def}|\\
|\childdocforward{|\textit{main}|}|
\end{tabular}
\end{center}
%
Likewise, the following files |final|\textit{nn}|.tex|
compile the final version of the child document
|child|\textit{nn}|.tex|:
%
\begin{center}
\begin{tabular}{l}
|\def\version{final}|\\
|\input{childdoc.def}|\\
|\childdocforwardprefix{final}{child}|
\end{tabular}
\end{center}
%

Note that when several versions of a main file and/or of each child file
are to be generated, it may be convenient to set up a |Makefile| or
shell script to automatise the process.

%%%%%%%%%%%%%%%%%%%%%%%%%%%%%%%%%%%%%%%%%%%%%%%%%%%%%%%%%%%%%%%%%%%%%%%%%%%%%%%%
\subsection{Command Line Processing}
\label{sec:commandline}

The effect of redirection files can also be achieved by invoking
the \LaTeX{} compiler with a more elaborate command line.
Most conveniently this should be done as part
of a shell script or a |Makefile|.

When using \textsf{childdoc} in the main file, the following
command lines effectively perform a redirection
(note that depending on the shell being used,
backslashes may have to be doubled: `|\|' $\to$ `|\\|'):
%
\begin{center}
|... -jobname "|\textit{target}|" |\\|"|[\textit{flags}]%
|\input{childdoc.def}\childdocforward[|\textit{main}|]{|\textit{dest}|}"|
\end{center}
%
Here \textit{target} is the name of the output file,
\textit{main} is the name of the main file
and \textit{dest} is the name of the main or child file to be processed
(all filenames without extensions).
The optional argument \textit{main} can be omitted
if \textit{main} matches \textit{dest}.
Optionally, compilation \textit{flags} can be defined via |\def| commands.
This command line makes the \TeX{} engine believe
it is compiling the file \textit{target}
whose content is specified as the latter parameter.
The provided code then forwards the processing to
\textit{main} or \textit{dest} as described in \secref{sec:forward}.

%%%%%%%%%%%%%%%%%%%%%%%%%%%%%%%%%%%%%%%%%%%%%%%%%%%%%%%%%%%%%%%%%%%%%%%%%%%%%%%%
\subsection{Include by Input}
\label{sec:input}

Including child documents by |\include| has some restrictions by design.
Most notably, the content of a child document always occupies
its own set of pages; pages cannot be shared between child documents.
Usually, this behaviour makes perfect sense
because each child document contain an essential part of the document.
However, in some situations it may be desirable to compose
a document from a collection of parts
without having mandatory page breaks between then.
For this case, the package
provides a mechanism to include parts
by |\input| which can also be processed individually.
However, by construction this mechanism
requires manual handling of the content to be output.

%%%%%%%%%%%%%%%%%%%%%%%%%%%%%%%%%%%%%%%%
\DescribeMacro{\ifchilddocmanual}
The main file should be prepared as usual, see \secref{sec:include}.
However, the document body must make a distinction
between processing of an individual part and of the main document, e.g.:
%
\begin{center}
\begin{tabular}{l}
|\ifchilddocmanual|\\
|\input{\childdocname}|\\
|\||else|\\
\textit{document body with }|\input{|\textit{part}|}|\\
|\||fi|
\end{tabular}
\end{center}
%
The conditional |\ifchilddocmanual| is true whenever
a part to be included by |\input| is being compiled,
and the name of the part is stored in |\childdocname|.

%%%%%%%%%%%%%%%%%%%%%%%%%%%%%%%%%%%%%%%%
\DescribeMacro{\childdocby}
Each part to be included by |\input| should start with:
%
\begin{center}
\begin{tabular}{l}
|\input{childdoc.def}|\\
|\childdocby{|\textit{main}|}|\\
\end{tabular}
\end{center}
%
The directive |\childdocby| is similar to |\childdocof|
described in \secref{sec:include},
but the subsequent selection of content must be done manually.
To that end, both |\ifchilddoc| and |\ifchilddocmanual|
will be true upon processing of a part,
and the name of the part is stored in |\childdocname|.
Note that |\jobname| will be set to the filename of the current part
so that each part receives an individual |.aux| file
that does not interfere with the |.aux| file(s) of the main document.
This behaviour can be altered by the alternative form
|\childdocby[*]{|\textit{main}|}| (with a non-empty optional argument)
which uses the |.aux| file of the main document
by setting |\jobname| to \textit{main}.

%%%%%%%%%%%%%%%%%%%%%%%%%%%%%%%%%%%%%%%%%%%%%%%%%%%%%%%%%%%%%%%%%%%%%%%%%%%%%%%%
\subsection{Driver Development}
\label{sec:driver}

The \textsf{childdoc} mechanism can also be use for the development
of definition files such as \LaTeX{} styles or classes.
This case differs from the above setup with multiple parts
included by |\include| in that no |\includeonly| should be invoked.
This can be achieved by starting the include file
(before |\ProvidesPackage|) with:
%
\begin{center}
\begin{tabular}{l}
|\input{childdoc.def}|\\
|\childdocforward{|\textit{main}|}|\\
\end{tabular}
\end{center}
%
or alternatively with:
%
\begin{center}
\begin{tabular}{l}
|\input{childdoc.def}|\\
|\childdocby{|\textit{main}|}|\\
\end{tabular}
\end{center}
%
Both forms have slightly different effects as described above.
The main file is prepared as usual, see \secref{sec:include}.

%%%%%%%%%%%%%%%%%%%%%%%%%%%%%%%%%%%%%%%%%%%%%%%%%%%%%%%%%%%%%%%%%%%%%%%%%%%%%%%%
\subsection{Legacy Detection}
\label{sec:detection}

The directive |\childdocmain| in the main file can detect
whether the complete document or merely a child is to be compiled
even without using the directive |\childdocof|.
This method is deprecated because it is less robust
and there is no compelling reason to use it;
it is merely provided for backward compatibility
and it may be removed in future versions.

If the detection mechanism is to be used,
it is mandatory to correctly specify
the filename of the main file as the argument of |\childdocmain|:
%
\begin{center}
\begin{tabular}{l}
|\input{childdoc.def}|\\
|\childdocmain{|\textit{main}|}|\\
\end{tabular}
\end{center}
%
If |\jobname| does not match the argument \textit{main} of |\childdocmain|,
it is assumed that |\jobname| points to the child file to be compiled.
When using |\childdocmain| with the main file specified as argument,
it suffices to start a child file
with just |\input{|\textit{main}|}|
without loading of the package and using |\childdocof|.
If instead all processing is done
with the appropriate \textsf{childdoc} directives,
the argument of \textit{main} of |\childdocmain| can be empty.

An alternative version of the command line processing described
in \secref{sec:commandline} using the detection mechanism reads:
%
\begin{center}
|... -jobname "|\textit{target}|" "|[\textit{flags}]%
[|\def\jobname{|\textit{dest}|}|]|\input{|\textit{main}|}"|
\end{center}

%%%%%%%%%%%%%%%%%%%%%%%%%%%%%%%%%%%%%%%%%%%%%%%%%%%%%%%%%%%%%%%%%%%%%%%%%%%%%%%%
\subsection{Manual Code}
\label{sec:manual}

In case one cannot be certain whether the definitions file |childdoc.def|
is installed on the target \TeX{} distribution
and one prefers not to ship it,
it is conceivable to paste a few relevant commands into the sources.

To that end, drop all statements |\input{childdoc.def}|
and perform the replacements as outlined below.
Instead of |\childdocmain{|\textit{main}|}| add the following code
to the top of the main file:
%
\begin{center}
\begin{tabular}{l}
|\||ifdefined\childdocname\endinput\||fi\newif\ifchilddoc|\\
|\edef\childdocname{\scantokens\expandafter{\jobname\noexpand}}|\\
|\def\childdocmain{|\textit{main}|}\||ifx\childdocmain\childdocname\||else|\\
|\childdoctrue\includeonly{\childdocname}\let\jobname\childdocmain\||fi|\\
\end{tabular}
\end{center}
%
Instead of |\childdocof{|\textit{main}|}| just include the main file
at the top of each child file:
%
\begin{center}
|\input{|\textit{main}|}|
\end{center}
%
A simple redirection |\childdocforward{|\textit{dest}|}| is achieved by:
%
\begin{center}
|\def\jobname{|\textit{dest}|}\input{\jobname}|
\end{center}
%
The redirection with prefix
|\childdocforwardprefix[|\textit{prefix}|]{|\textit{dest}|}|
is accomplished by:
%
\begin{center}
\begin{tabular}{l}
|{\edef\jobname{\scantokens\expandafter{\jobname\noexpand}}|\\
|\def\redirectjob |\textit{prefix}|#1~~~{\gdef\jobname{|\textit{dest}|#1}}|\\
|\expandafter\redirectjob\jobname~~~}\input{\jobname}|
\end{tabular}
\end{center}

In an alternative approach,
child documents can be compiled by a specific command line
without additional code or specific definitions:
%
\begin{center}
|... -jobname "|\textit{target}|" "|[\textit{flags}]%
|\includeonly{|\textit{dest}|}\input{|\textit{main}|}"|
\end{center}
%

%%%%%%%%%%%%%%%%%%%%%%%%%%%%%%%%%%%%%%%%%%%%%%%%%%%%%%%%%%%%%%%%%%%%%%%%%%%%%%%%
%%%%%%%%%%%%%%%%%%%%%%%%%%%%%%%%%%%%%%%%%%%%%%%%%%%%%%%%%%%%%%%%%%%%%%%%%%%%%%%%
\section{Information}

%%%%%%%%%%%%%%%%%%%%%%%%%%%%%%%%%%%%%%%%%%%%%%%%%%%%%%%%%%%%%%%%%%%%%%%%%%%%%%%%
\subsection{Copyright}

Copyright \copyright{} 2017--2018 Niklas Beisert

This work may be distributed and/or modified under the
conditions of the \LaTeX{} Project Public License, either version 1.3
of this license or (at your option) any later version.
The latest version of this license is in
  \url{http://www.latex-project.org/lppl.txt}
and version 1.3 or later is part of all distributions of \LaTeX{}
version 2005/12/01 or later.

This work has the LPPL maintenance status `maintained'.

The Current Maintainer of this work is Niklas Beisert.

This work consists of the files |README.txt|, |childdoc.ins| and |childdoc.dtx|
as well as the derived files |childdoc.def|, |cdocsamp.tex|
with |cdocsch1.tex|, |cdocsch2.tex|, |cdocspt3.tex|, |cdocspt4.tex|,
|cdocsdrf.tex|, |cdocsfn1.tex|, |cdocsfn2.tex|
as well as |childdoc.pdf|.

%%%%%%%%%%%%%%%%%%%%%%%%%%%%%%%%%%%%%%%%%%%%%%%%%%%%%%%%%%%%%%%%%%%%%%%%%%%%%%%%
\subsection{Files and Installation}

The package consists of the files:
%
\begin{center}
\begin{tabular}{ll}
    |README.txt|   & readme file \\
    |childdoc.ins| & installation file \\
    |childdoc.dtx| & source file \\
    |childdoc.def| & definition file \\
    |cdocsamp.tex| & sample main file \\
    |cdocsch1.tex| & sample include file \\
    |cdocsch2.tex| & sample include file \\
    |cdocspt3.tex| & sample part file \\
    |cdocspt4.tex| & sample part file \\
    |cdocsdrf.tex| & sample redirection file \\
    |cdocsfn1.tex| & sample redirection file \\
    |cdocsfn2.tex| & sample redirection file \\
    |childdoc.pdf| & manual
\end{tabular}
\end{center}
%
The distribution consists of the files
|README.txt|, |childdoc.ins| and |childdoc.dtx|.
%
\begin{itemize}
\item
Run (pdf)\LaTeX{} on |childdoc.dtx|
to compile the manual |childdoc.pdf| (this file).
\item
Run \LaTeX{} on |childdoc.ins| to create the definitions file |childdoc.def|
and the sample |cdocsamp.tex| with include files
|cdocsch1.tex|, |cdocsch2.tex|, |cdocspt3.tex|, |cdocspt4.tex|,
|cdocsdrf.tex|, |cdocsfn1.tex|, |cdocsfn2.tex|.
Then copy the file |childdoc.def| to an appropriate directory of your \LaTeX{}
distribution, e.g.\ \textit{texmf-root}|/tex/latex/childdoc|.
\end{itemize}

%%%%%%%%%%%%%%%%%%%%%%%%%%%%%%%%%%%%%%%%%%%%%%%%%%%%%%%%%%%%%%%%%%%%%%%%%%%%%%%%
\subsection{Related CTAN Packages}

There are several other packages which offer a similar functionality:
%
\begin{itemize}
\item
The packages
\href{http://ctan.org/pkg/docmute}{\textsf{docmute}},
\href{http://ctan.org/pkg/includex}{\textsf{includex}} and
\href{http://ctan.org/pkg/standalone}{\textsf{standalone}}
provide commands to include only the document body of
a child file thus allowing both files to be compiled individually.
\item
The packages \href{http://ctan.org/pkg/subdocs}{\textsf{subdocs}}
and \href{http://ctan.org/pkg/subfiles}{\textsf{subfiles}}
provide structures in which the main and child documents can be
encapsulated and allowing them to be compiled individually.
The inclusion mechanism is different from the conventional |\include|.
\item
The package \href{http://ctan.org/pkg/combine}{\textsf{combine}}
is an elaborate solution to combine several documents into one.
\end{itemize}
%
See also the CTAN topic \href{http://ctan.org/topic/subdocs}{\textsf{subdocs}}
for further related packages.
The present package differs from the above solutions in that
a document structure constructed with the conventional |\include| mechanism
just needs two extra commands at the top of every file
such that all constituent files can be compiled individually.

%%%%%%%%%%%%%%%%%%%%%%%%%%%%%%%%%%%%%%%%%%%%%%%%%%%%%%%%%%%%%%%%%%%%%%%%%%%%%%%%
%\subsection{Feature Suggestions}
%
%The following is a list of features which may be useful for future
%versions of this package:
%%
%\begin{itemize}
%\item
%\ldots
%\end{itemize}

%%%%%%%%%%%%%%%%%%%%%%%%%%%%%%%%%%%%%%%%%%%%%%%%%%%%%%%%%%%%%%%%%%%%%%%%%%%%%%%%
\subsection{Revision History}

%%%%%%%%%%%%%%%%%%%%%%%%%%%%%%%%%%%%%%%%
\paragraph{v2.0:} 2018/12/30

\begin{itemize}
\item
immediate forward processing
\item
added |\childdocby| mechanism
\item
manual restructured
\end{itemize}

%%%%%%%%%%%%%%%%%%%%%%%%%%%%%%%%%%%%%%%%
\paragraph{v1.6:} 2018/01/17

\begin{itemize}
\item
application for development of include files
\item
corrections to manual
\end{itemize}

%%%%%%%%%%%%%%%%%%%%%%%%%%%%%%%%%%%%%%%%
\paragraph{v1.5:} 2017/05/21

\begin{itemize}
\item
more complete structuring introduced
\item
|\childdocof| introduced
\item
|\childdoc| renamed to |\childdocmain|
\item
|\childredirect| renamed to |\childdocforward| and |\childdocforwardprefix|
and functionality expanded
\end{itemize}

%%%%%%%%%%%%%%%%%%%%%%%%%%%%%%%%%%%%%%%%
\paragraph{v1.0:} 2017/04/27

\begin{itemize}
\item
manual and install package
\item
first version published on CTAN
\end{itemize}

%%%%%%%%%%%%%%%%%%%%%%%%%%%%%%%%%%%%%%%%
\paragraph{v0.6:} 2017/04/26

\begin{itemize}
\item
redirection mechanism added
\end{itemize}

%%%%%%%%%%%%%%%%%%%%%%%%%%%%%%%%%%%%%%%%
\paragraph{v0.5:} 2017/04/26

\begin{itemize}
\item
functionality in definition file
\end{itemize}


%%%%%%%%%%%%%%%%%%%%%%%%%%%%%%%%%%%%%%%%%%%%%%%%%%%%%%%%%%%%%%%%%%%%%%%%%%%%%%%%
%%%%%%%%%%%%%%%%%%%%%%%%%%%%%%%%%%%%%%%%%%%%%%%%%%%%%%%%%%%%%%%%%%%%%%%%%%%%%%%%
%%%%%%%%%%%%%%%%%%%%%%%%%%%%%%%%%%%%%%%%%%%%%%%%%%%%%%%%%%%%%%%%%%%%%%%%%%%%%%%%
\appendix

\settowidth\MacroIndent{\rmfamily\scriptsize 000\ }

 \DocInput{childdoc.dtx}

\end{document}
%</driver>
% \fi
%
% %%%%%%%%%%%%%%%%%%%%%%%%%%%%%%%%%%%%%%%%%%%%%%%%%%%%%%%%%%%%%%%%%%%%%%%%%%%%%%
% %%%%%%%%%%%%%%%%%%%%%%%%%%%%%%%%%%%%%%%%%%%%%%%%%%%%%%%%%%%%%%%%%%%%%%%%%%%%%%
% \section{Sample}
%\iffalse
%<*samplemain>
%\fi
%
% The following presents a sample document
% with two chapters, two parts, a title page,
% a compile flag as well as three forwarding files to set the flag.
% It consists of eight |.tex| files:
% \begin{center}
% \begin{tabular}{ll}
% |cdocsamp.tex|&main file\\
% |cdocsch1.tex|&include file for chapter 1\\
% |cdocsch2.tex|&include file for chapter 2\\
% |cdocspt3.tex|&include file for part 3\\
% |cdocspt4.tex|&include file for part 4\\
% |cdocsdrf.tex|&forwarding file for main file in draft mode\\
% |cdocsfi1.tex|&forwarding file for final version of chapter 1\\
% |cdocsfi2.tex|&forwarding file for final version of chapter 2\\
% \end{tabular}
% \end{center}
% Each of the eight files can be compiled directly by the \LaTeX{} compiler.
%
% %%%%%%%%%%%%%%%%%%%%%%%%%%%%%%%%%%%%%%
% \paragraph{Main File.}
%
% The main file is called |cdocsamp.tex|.
%
% Load the \textsf{childdoc} definitions and
% declare the filename for the main document:
%    \begin{macrocode}
\input{childdoc.def}
\childdocmain{}
%    \end{macrocode}

% Optional override for |\version| flag:
%    \begin{macrocode}
%%\ifchilddoc\else\providecommand{\version}{draft}\fi
%    \end{macrocode}

% Define the default values for the |\version| flag
% (|final| for the main file and |draft| for childs):
%    \begin{macrocode}
\ifchilddoc
\providecommand{\version}{draft}
\else
\providecommand{\version}{final}
\fi
%    \end{macrocode}

% Load the standard document class:
%    \begin{macrocode}
\documentclass[12pt]{article}
%    \end{macrocode}

% Start the document body:
%    \begin{macrocode}
\begin{document}
%    \end{macrocode}

% Declare a title page.
% Print title, part of document being processed and version flag:
%    \begin{macrocode}
\addtocounter{page}{-1}
\begin{center}
{\LARGE\bfseries{}childdoc example\par}
\vspace{1cm}
\ifchilddoc
\ifchilddocmanual part\else chapter\fi:
`\childdocname' of `\childdocjob'\par
\else
main document: `\childdocjob'\par
\fi
version: \version\par
\end{center}
\newpage
%    \end{macrocode}

% Manually include selected file,
% otherwise process as usual:
%    \begin{macrocode}
\ifchilddocmanual
\section*{part `\childdocname'}
\input{\childdocname}
\else
%    \end{macrocode}

% Include the two chapters:
%    \begin{macrocode}
\include{cdocsch1}
\include{cdocsch2}
%    \end{macrocode}

% Include the two parts unless only chapters should be displayed:
%    \begin{macrocode}
\ifchilddoc\else
\section{part three}
\input{cdocspt3}
\section{part four}
\input{cdocspt4}
\fi
%    \end{macrocode}

% Process as usual until here:
%    \begin{macrocode}
\fi
%    \end{macrocode}

% End of document body:
%    \begin{macrocode}
\end{document}
%    \end{macrocode}
%\iffalse
%</samplemain>
%\fi
%
% %%%%%%%%%%%%%%%%%%%%%%%%%%%%%%%%%%%%%%
% \paragraph{Chapter Include Files.}
%
% The include files are called |cdocsch1.tex| and |cdocsch2.tex|.
%
%\iffalse
%<*samplechap1|samplechap2>
%\fi

% Optional override for |\version| flag:
%    \begin{macrocode}
%%\providecommand{\version}{final}
%    \end{macrocode}

% Include the main document:
%    \begin{macrocode}
\input{childdoc.def}
\childdocof{cdocsamp}
%    \end{macrocode}

%\iffalse
%</samplechap1|samplechap2>
%\fi
%
%\iffalse
%<*samplechap1>
%\fi
% Some text for chapter 1:
%    \begin{macrocode}
\section{one}
some text in chapter one
%    \end{macrocode}

%\iffalse
%</samplechap1>
%\fi
% Some text for chapter 2:
%\iffalse
%<*samplechap2>
%\fi
%    \begin{macrocode}
\section{two}
more text in chapter two
%    \end{macrocode}

%\iffalse
%</samplechap2>
%\fi
%
% %%%%%%%%%%%%%%%%%%%%%%%%%%%%%%%%%%%%%%
% \paragraph{Part Include Files.}
%
% The include files are called |cdocspt3.tex| and |cdocspt4.tex|.
%
%\iffalse
%<*samplepart3|samplepart4>
%\fi

% Optional override for |\version| flag:
%    \begin{macrocode}
%%\providecommand{\version}{final}
%    \end{macrocode}

% Include the main document:
%    \begin{macrocode}
\input{childdoc.def}
\childdocby{cdocsamp}
%    \end{macrocode}

%\iffalse
%</samplepart3|samplepart4>
%\fi
%
%\iffalse
%<*samplepart3>
%\fi
% Some text for part 3:
%    \begin{macrocode}
some text in part three
%    \end{macrocode}

%\iffalse
%</samplepart3>
%\fi
% Some text for part 4:
%\iffalse
%<*samplepart4>
%\fi
%    \begin{macrocode}
more text in part four
%    \end{macrocode}

%\iffalse
%</samplepart4>
%\fi
%
% %%%%%%%%%%%%%%%%%%%%%%%%%%%%%%%%%%%%%%
% \paragraph{Forwarding for a Complete Draft.}
%
% The following forwarding file |cdocsdrf.tex|
% compiles the main document in draft mode:
%\iffalse
%<*sampledraft>
%\fi
%    \begin{macrocode}
\def\version{draft}
\input{childdoc.def}
\childdocforward{cdocsamp}
%    \end{macrocode}

%\iffalse
%</sampledraft>
%\fi
%
% %%%%%%%%%%%%%%%%%%%%%%%%%%%%%%%%%%%%%%
% \paragraph{Forwarding for Final Version of the Chapters.}
%
% The following forwarding files |cdocsfn1.tex| and |cdocsfn2.tex|
% (with identical content)
% compile the final versions of the child documents
% |cdocsch1.tex| and |cdocsch2.tex|, respectively:
%\iffalse
%<*samplefinal>
%\fi
%    \begin{macrocode}
\def\version{final}
\input{childdoc.def}
\childdocforwardprefix[cdocsamp]{cdocsfn}{cdocsch}
%    \end{macrocode}

%\iffalse
%</samplefinal>
%\fi
%
% %%%%%%%%%%%%%%%%%%%%%%%%%%%%%%%%%%%%%%
% \paragraph{Command Line Processing.}
%
% The following three command lines generate the output files
% |cdocscld|, |cdocscl1| and |cdocscl2|
% which should be identical to
% |cdocsdrf|, |cdocsch1| and |cdocsfn2|, respectively:
% \begin{center}
% \begin{tabular}{l}
% |latex -jobname cdocscld \|\\
% |  "\def\version{draft}\input{childdoc.def}\childdocforward{cdocsamp}"|\\
% |latex -jobname cdocscl1 \|\\
% |  "\input{childdoc.def}\childdocforward[cdocsamp]{cdocsch1}"|\\
% |latex -jobname cdocscl2 \|\\
% |  "\def\version{final}\input{childdoc.def}\childdocforward{cdocsch2}"|
% \end{tabular}
% \end{center}
% Note that the trailing backslash on each first line
% merely continues the input to the second line
% (for convenient cut ant paste).
% Furthermore, the command |latex| can be replaced by any
% of its alternative versions such as |pdflatex|.
%
% %%%%%%%%%%%%%%%%%%%%%%%%%%%%%%%%%%%%%%%%%%%%%%%%%%%%%%%%%%%%%%%%%%%%%%%%%%%%%%
% %%%%%%%%%%%%%%%%%%%%%%%%%%%%%%%%%%%%%%%%%%%%%%%%%%%%%%%%%%%%%%%%%%%%%%%%%%%%%%
% \section{Implementation}
%\iffalse
%<*package>
%\fi
%
% This section describes the definitions file |childdoc.def|.

% The definitions cannot be loaded using |\usepackage| or |\RequirePackage|
% which has a mechanism to prevent loading a style file more than once.
% When loading the definitions by means of |\input|
% multiple instances have to be prevented manually:
%\iffalse
%This code needs to be before the `\ProvidesFile' directive
%which is defined at the beginning of this file.
%Therefore it is also placed there and commented out here.
%</package>
%<*discard>
%\fi
%    \begin{macrocode}
\ifdefined\childdocmain\endinput\fi
%    \end{macrocode}
%\iffalse
%</discard>
%<*package>
%\fi
%
% \macro{\ifchilddoc}
% \macro{\ifchilddocmanual}
% The conditional |\ifchilddoc| tells whether a
% child (true) or main (false) document is being compiled.
% The conditional |\ifchilddocmanual| tells whether
% the |\includeonly| mechanism is used (false) or
% the selection of child files must be performed manually (true).
% The definitions initialise to false:
%    \begin{macrocode}
\newif\ifchilddoc
\newif\ifchilddocmanual
%    \end{macrocode}

% \macro{\childdocname}
% \macro{\childdocjob}
% The macro |\childdocname| stores the name of the main document
% to be compiled. The macro |\childdocjob| stores the name of
% the document on which the \LaTeX{} compiler was originally invoked.
% The content of |\jobname| cannot be compared
% to filenames specified in the source due to different catcodes.
% The following code rescans |\jobname|, stores the result
% in |\childdocname| and saves a copy in |\childdocjob|:
%    \begin{macrocode}
\edef\childdocname{\scantokens\expandafter{\jobname\noexpand}}
\let\childdocjob\childdocname
%    \end{macrocode}

% \macro{\childdocdisable}
% The macro |\childdocdisable| prevents the main file
% from being processed more than once.
% At this stage, the main document command |\childdocmain|
% is assumed to be called once again where it should do nothing.
% Any subsequent call to it should prevent
% a secondary processing of the main document
% It overwrites the forwarding commands
% |\childdocof| and |\childdocforward|
% with empty macros to prevent further inclusions of the main document:
%    \begin{macrocode}
\newcommand{\childdocdisable}
{
  \renewcommand{\childdocmain}[1]{\renewcommand{\childdocmain}[1]{\endinput}}
  \renewcommand{\childdocof}[1]{}
  \renewcommand{\childdocby}[2][]{}
  \renewcommand{\childdocforward}[2][]{}
  \renewcommand{\childdocdisable}{}
}
%    \end{macrocode}

% \macro{\childdocmain}
% The macro |\childdocmain| is to be called at the top of the main file
% with nothing or the main filename (without extension) as argument.
% First, it breaks loops.
% If the argument is not empty and does not match |\childdocname|
% (which is set by the first inclusion of |childdoc.def|),
% |\ifchilddoc| is set to true, |\includeonly| is applied to the child file
% and |\jobname| is set to the main file
% (for proper handling of |.aux| files):
%    \begin{macrocode}
\newcommand{\childdocmain}[1]
{
  \childdocdisable\childdocmain{}
  \if?#1?\else
    \begingroup
      \def\childdoctmp{#1}
      \ifx\childdoctmp\childdocname
        \def\childdoctmp{}
      \else
        \def\childdoctmp
        {
          \childdoctrue
          \includeonly{\childdocname}
          \def\childdocjob{#1}
          \def\jobname{#1}
        }
      \fi
      \expandafter
    \endgroup
    \childdoctmp
  \fi
}
%    \end{macrocode}

% \macro{\childdocof}
% The command |\childdocof| redirects
% compilation to the main file |#1|.
%    \begin{macrocode}
\newcommand{\childdocof}[1]
{
  \childdocdisable
  \childdoctrue
  \includeonly{\childdocname}
  \def\jobname{#1}
  \def\childdocjob{#1}
  \input{#1}
}
%    \end{macrocode}

% \macro{\childdocby}
% The command |\childdocby| ....
%    \begin{macrocode}
\newcommand{\childdocby}[2][]
{
  \childdocdisable
  \childdoctrue
  \childdocmanualtrue
  \if?#1?\else
    \def\jobname{#2}
  \fi
  \def\childdocjob{#2}
  \input{#2}
  \endinput
}
%    \end{macrocode}

% \macro{\childdocforward}
% The command |\childdocforward| redirects
% compilation to the main file or
% (if the optional argument is given) a child file.
% Parameters are set as if the main file
% or a child file starting with |\childdocof| was compiled.
% Then compilation is handed over to the main file:
%    \begin{macrocode}
\newcommand{\childdocforward}[2][]
{
  \begingroup
    \if?#1?
      \def\childdoctmp
      {
        \def\childdocname{#2}
        \def\childdocjob{#2}
        \def\jobname{#2}
        \input{#2}
        \endinput
      }
    \else
      \def\childdoctmp
      {
        \childdocdisable
        \def\childdocname{#2}
        \childdoctrue
        \includeonly{#2}
        \def\childdocjob{#1}
        \def\jobname{#1}
        \input{#1}
        \endinput
      }
    \fi
    \expandafter
  \endgroup
  \childdoctmp
}
%    \end{macrocode}

% \macro{\childdocforwardprefix}
% The command |\childdocforwardprefix| redirects
% compilation to the main or a child file by means of a pattern.
% The prefix |#1| in the current filename is replaced by |#2|
% and the suffix of the current filename is kept
% (it is assumed that the filename does not contain the substring `|~~~|'
% which is used as a delimiter).
% Compilation is handed over to the new file by |\childdocforward|:
%    \begin{macrocode}
\newcommand{\childdocforwardprefix}[3][]
{
  \begingroup
    \def\childdocextract #2##1~~~{\def\childdoctmp{\childdocforward[#1]{#3##1}}}
    \expandafter\childdocextract\childdocname~~~
    \expandafter
  \endgroup
  \childdoctmp
}
%    \end{macrocode}

% \macro{\childdoc}
% The deprecated macro |\childdoc| is a legacy version of |\childdocmain|:
%    \begin{macrocode}
\newcommand{\childdoc}{\childdocmain}
%    \end{macrocode}

% \macro{\childdocredirect}
% The deprecated macro |\childdocredirect| is a legacy version
% of |\childdocforward| and |\childdocforwardprefix|:
%    \begin{macrocode}
\newcommand{\childdocredirect}[2][]
{
  \begingroup
    \if?#1?
      \def\childdoctmp{\childdocforward{#2}}
    \else
      \def\childdoctmp{\childdocforwardprefix{#1}{#2}}
    \fi
    \expandafter
  \endgroup
  \childdoctmp
}
%    \end{macrocode}

%\iffalse
%</package>
%\fi
%
\endinput
|\\
|\childdocby{|\textit{main}|}|\\
\end{tabular}
\end{center}
%
The directive |\childdocby| is similar to |\childdocof|
described in \secref{sec:include},
but the subsequent selection of content must be done manually.
To that end, both |\ifchilddoc| and |\ifchilddocmanual|
will be true upon processing of a part,
and the name of the part is stored in |\childdocname|.
Note that |\jobname| will be set to the filename of the current part
so that each part receives an individual |.aux| file
that does not interfere with the |.aux| file(s) of the main document.
This behaviour can be altered by the alternative form
|\childdocby[*]{|\textit{main}|}| (with a non-empty optional argument)
which uses the |.aux| file of the main document
by setting |\jobname| to \textit{main}.

%%%%%%%%%%%%%%%%%%%%%%%%%%%%%%%%%%%%%%%%%%%%%%%%%%%%%%%%%%%%%%%%%%%%%%%%%%%%%%%%
\subsection{Driver Development}
\label{sec:driver}

The \textsf{childdoc} mechanism can also be use for the development
of definition files such as \LaTeX{} styles or classes.
This case differs from the above setup with multiple parts
included by |\include| in that no |\includeonly| should be invoked.
This can be achieved by starting the include file
(before |\ProvidesPackage|) with:
%
\begin{center}
\begin{tabular}{l}
|% \iffalse
%
% childdoc.dtx Copyright (C) 2017-2018 Niklas Beisert
%
% This work may be distributed and/or modified under the
% conditions of the LaTeX Project Public License, either version 1.3
% of this license or (at your option) any later version.
% The latest version of this license is in
%   http://www.latex-project.org/lppl.txt
% and version 1.3 or later is part of all distributions of LaTeX
% version 2005/12/01 or later.
%
% This work has the LPPL maintenance status `maintained'.
%
% The Current Maintainer of this work is Niklas Beisert.
%
% This work consists of the files childdoc.dtx and childdoc.ins
% and the derived files childdoc.def and cdocsamp.tex with
% cdocsch1.tex, cdocsch2.tex, cdocsdrf.tex, cdocsfn1.tex, cdocsfn2.tex.
%
%<package>\ifdefined\childdocmain\endinput\fi
%<package>\ProvidesFile{childdoc.def}[2018/12/30 v2.0 child document driver]
%<samplemain>\ProvidesFile{cdocsamp.tex}[2018/12/30 v2.0 sample for childdoc]
%<*driver>
%\ProvidesFile{childdoc.drv}[2018/12/30 v2.0 childdoc reference manual file]
\PassOptionsToClass{10pt,a4paper}{article}
\documentclass{ltxdoc}

\usepackage[margin=35mm]{geometry}
\usepackage{hyperref}
\usepackage{hyperxmp}
\usepackage[usenames]{color}

\hypersetup{colorlinks=true}
\hypersetup{pdfstartview=FitH}
\hypersetup{pdfpagemode=UseNone}
\hypersetup{pdfsource={}}
\hypersetup{pdflang={en-UK}}
\hypersetup{pdfcopyright={Copyright 2017-2018 Niklas Beisert.
  This work may be distributed and/or modified under the
  conditions of the LaTeX Project Public License, either version 1.3
  of this license or (at your option) any later version.}}
\hypersetup{pdflicenseurl={http://www.latex-project.org/lppl.txt}}
\hypersetup{pdfcontactaddress={ETH Zurich, ITP, HIT K,
  Wolfgang-Pauli-Strasse 27}}
\hypersetup{pdfcontactpostcode={8093}}
\hypersetup{pdfcontactcity={Zurich}}
\hypersetup{pdfcontactcountry={Switzerland}}
\hypersetup{pdfcontactemail={nbeisert@itp.phys.ethz.ch}}
\hypersetup{pdfcontacturl={http://people.phys.ethz.ch/\xmptilde nbeisert/}}

\newcommand{\secref}[1]{\hyperref[#1]{section \ref*{#1}}}

\parskip1ex
\parindent0pt
\let\olditemize\itemize
\def\itemize{\olditemize\parskip0pt}

\begin{document}

\title{The \textsf{childdoc} Package}
\hypersetup{pdftitle={The childdoc Package}}
\author{Niklas Beisert\\[2ex]
  Institut f\"ur Theoretische Physik\\
  Eidgen\"ossische Technische Hochschule Z\"urich\\
  Wolfgang-Pauli-Strasse 27, 8093 Z\"urich, Switzerland\\[1ex]
  \href{mailto:nbeisert@itp.phys.ethz.ch}
  {\texttt{nbeisert@itp.phys.ethz.ch}}}
\hypersetup{pdfauthor={Niklas Beisert}}
\hypersetup{pdfsubject={Manual for the LaTeX2e Package childdoc}}
\date{30 December 2018, \textsf{v2.0}}
\maketitle

\begin{abstract}\noindent
\textsf{childdoc} is a \LaTeXe{} package
that enables the direct compilation
of document sections included by |\include|
to individual files.
\end{abstract}

\begingroup
\parskip0ex
\tableofcontents
\endgroup

%%%%%%%%%%%%%%%%%%%%%%%%%%%%%%%%%%%%%%%%%%%%%%%%%%%%%%%%%%%%%%%%%%%%%%%%%%%%%%%%
%%%%%%%%%%%%%%%%%%%%%%%%%%%%%%%%%%%%%%%%%%%%%%%%%%%%%%%%%%%%%%%%%%%%%%%%%%%%%%%%
\section{Introduction}

\LaTeX{} provides a mechanism to structure a large document (such as a book)
into a main file and several child files (containing the chapters)
using the |\include| command.
This mechanism is beneficial for documents
which span hundreds of pages in order to
make the source file(s) more manageable.
Moreover, compilation can be restricted to
selected child files by means of the |\includeonly| command.
The latter feature can be used to reduce the compilation time while editing
(this was significantly more useful in the earlier days of \LaTeX{})
or to generate a smaller document which is easier to navigate.
Another application of |\includeonly| is to generate
documents consisting of selected parts of the complete document.

However, there are a few drawbacks of the plain |\include| mechanism:
\begin{itemize}
\item
The child files cannot be compiled on their own,
they can only be compiled via the main file.
A naive editing environment
(such as a text editor with an option
to have the current file processed by \LaTeX)
may require one to switch to the main file before compiling;
attempting to compile the child file produces errors.
\item
The main file must be modified (each time)
to adjust the |\includeonly| command
to the present needs. This easily leaves the main file in a messy state.
\item
The generated document will always carry the filename
of the main document. This is inconvenient if
several child files are to be compiled and
to be kept for distribution.
\end{itemize}

The present package provides a simple interface
to make child files individually compilable by \LaTeX{}.
Compiling a child file then has the same effect as compiling
the main file with an |\includeonly| command
to select the appropriate child.
Moreover the generated document will carry the name of the child
rather than the main file.
This resolves all three above issues.

This feature is meant to make the editing of books,
thesis documents and lecture notes somewhat more convenient.
However, the package can also be used efficiently for
composing a series of documents (such as exercise sheets)
which are typically distributed individually.
It then assists the author in generating the individual documents
(potentially in different versions)
as well as a document containing the collected series.
Another application is in developing style files
or other kinds of included material
where compilation of the style file could redirect
to a sample or test file.

%%%%%%%%%%%%%%%%%%%%%%%%%%%%%%%%%%%%%%%%%%%%%%%%%%%%%%%%%%%%%%%%%%%%%%%%%%%%%%%%
%%%%%%%%%%%%%%%%%%%%%%%%%%%%%%%%%%%%%%%%%%%%%%%%%%%%%%%%%%%%%%%%%%%%%%%%%%%%%%%%
\section{Usage}

First of all, the package \textsf{childdoc} is \emph{not} a standard
\LaTeXe{} |.sty| style file! Therefore it needs to be invoked in
a non-standard way.

%%%%%%%%%%%%%%%%%%%%%%%%%%%%%%%%%%%%%%%%%%%%%%%%%%%%%%%%%%%%%%%%%%%%%%%%%%%%%%%%
\subsection{Included Files}
\label{sec:include}

%%%%%%%%%%%%%%%%%%%%%%%%%%%%%%%%%%%%%%%%
\DescribeMacro{\childdocmain}
To use the package, add the commands
\begin{center}
\begin{tabular}{l}
|\input{childdoc.def}|\\
|\childdocmain{}|\\
\end{tabular}
\end{center}
at the very top of the main \LaTeX{} file,
in particular \emph{before} the |\documentclass| statement!
The argument of |\childdocmain| should be left empty
(but it must be present).

%%%%%%%%%%%%%%%%%%%%%%%%%%%%%%%%%%%%%%%%
\DescribeMacro{\childdocof}
Furthermore, add the commands
\begin{center}
\begin{tabular}{l}
|\input{childdoc.def}|\\
|\childdocof{|\textit{main}|}|\\
\end{tabular}
\end{center}
at the top of every child file \textit{child}
which is included by |\include{|\textit{child}|}|
from within the main file
(or at least for those files to be compiled individually).
The argument \textit{main} must be the filename of the main file.

There are a couple of
considerations in setting up the main and child documents:

%%%%%%%%%%%%%%%%%%%%%%%%%%%%%%%%%%%%%%%%
\paragraph{Restrictions.}

Please note the following restrictions:
\begin{itemize}
\item
|\childdocmain| must be called with one argument \textit{main}
to ensure compatibility with earlier version of the package.
It must either be empty (|\childdocmain{}|)
or precisely match the filename of the main file in which it is specified.
See \secref{sec:detection} for further information.
\item
The filename \textit{main} must be specified without the |.tex| extension.
\item
The filename \textit{main} is case sensitive
(even in case-insensitive file systems)
due to internal string comparison.
\item
The argument \textit{main} should be fully expanded, it cannot be a macro.
\item
Subdirectories and special characters should be avoided in filenames.
\item
The command |\childdocmain{|\textit{main}|}| must be followed by a whitespace.
It should not be followed immediately by another command
or by a comment mark `|%|'.
This is because the \TeX{} parser reads the token immediately following
the argument of |\childdocmain| and puts it
at the beginning of every child section;
however, a white\-space is ignored.
\end{itemize}

%%%%%%%%%%%%%%%%%%%%%%%%%%%%%%%%%%%%%%%%
\paragraph{Content of Main File.}

It is advisable to place all content in the child files included by |\include|.
Any output contained in the main file will appear in all child documents
unless suppressed manually;
it cannot be suppressed automatically by the |\includeonly| directive
and thus should normally be avoided.
A method to include some content in the main file
by means of conditional processing is described in \secref{sec:conditional}.

%%%%%%%%%%%%%%%%%%%%%%%%%%%%%%%%%%%%%%%%
\paragraph{Page Numbering.}

When only a part of the document is compiled,
the appropriate numbering of pages
(as well as other status parameters)
is determined from the |.aux| files.
The latter contain information from previous passes.
However this information needs to propagate through
all intermediate child documents.
Therefore the page numbering in child documents may well
be inconsistent until the complete document is compiled at least once.

A useful (if unconventional) way to always ensure a consistent
page numbering is to restart the numbering in each child document
and denote the pages by `\textit{child}|.|\textit{page}'
where \textit{child} represents the chapter/section number of the child file.
This can be achieved by the command
|\numberwithin{page}{|\textit{child}|}|
of the \textsf{amsmath} package
where \textit{child} can be |chapter| or |section|
depending on the chosen structuring.
Alternatively, one can modify the macro |\thepage| appropriately
and reset the counter |page| at the start of each child file.

%%%%%%%%%%%%%%%%%%%%%%%%%%%%%%%%%%%%%%%%%%%%%%%%%%%%%%%%%%%%%%%%%%%%%%%%%%%%%%%%
\subsection{Conditional Processing}
\label{sec:conditional}

The package provides a mechanism to compile different versions
of a document. To customise the versions further some conditional processing
can come in handy to distinguish which version is being compiled.
The package provides two macros to describe the compilation context:

%%%%%%%%%%%%%%%%%%%%%%%%%%%%%%%%%%%%%%%%
\DescribeMacro{\ifchilddoc}
The conditional |\ifchilddoc| distinguishes between the compilation of
child documents and the main document:
%
\begin{center}
|\ifchilddoc |\textit{child-code}| |[|\||else |\textit{main-code}]| \||fi|
\end{center}

%%%%%%%%%%%%%%%%%%%%%%%%%%%%%%%%%%%%%%%%
\DescribeMacro{\childdocname}
\DescribeMacro{\childdocjob}
The macro |\childdocname| contains the filename (without extension)
of the main or child file being processed.
Note that |\childdocjob| will always contain the name of the main file.

%%%%%%%%%%%%%%%%%%%%%%%%%%%%%%%%%%%%%%%%
\paragraph{Title Page.}

Conditional processing can be used to include a title or banner page
in the main document when proper precautions are taken.
Importantly, the code in the main file should ensure that the page counter
(as well as other status parameters which are stored in the |.aux| files)
takes the same value after the conditional processing.
Otherwise the page numbers may take divergent values
depending on which part is compiled.

For example, a title page could be declared by:
%
\begin{center}
\begin{tabular}{l}
|\ifchilddoc\||else|\\
|\addtocounter{page}{-1}|\\
\textit{code for title page}\\
|\newpage|\\
|\||fi|
\end{tabular}
\end{center}
%
A banner page for the child documents can be generated by:
%
\begin{center}
\begin{tabular}{l}
|\ifchilddoc|\\
|\addtocounter{page}{-1}|\\
\textit{code for banner page}\\
|\newpage|\\
|\||fi|
\end{tabular}
\end{center}
%
Here one could write a message such as:
\begin{center}
|This is the part \childdocname{} of \childdocjob{}.|
\end{center}

%%%%%%%%%%%%%%%%%%%%%%%%%%%%%%%%%%%%%%%%%%%%%%%%%%%%%%%%%%%%%%%%%%%%%%%%%%%%%%%%
\subsection{Flags}
\label{sec:flags}

The package makes it easy to generate different versions
of the main or child documents.
To this end compilation flags can be defined
and assigned different default values.
They will be particularly useful in conjunction
with the forwarding mechanism described in \secref{sec:forward}.

For example, it may be useful to have a flag |\version|
which can be set to |draft| or |final|.
The document source will contain some conditional code
depending on the value of |\version|.
Suppose further, the flag should default to |final| for the main file
and to |draft| for child files
which is a natural assignment for editing the document.
This is achieved by placing the following code
in the preamble of the main document
(below the |\childdocmain| directive):
%
\begin{center}
\begin{tabular}{l}
|\ifchilddoc|\\
|\providecommand{\version}{draft}|\\
|\||else|\\
|\providecommand{\version}{final}|\\
|\||fi|
\end{tabular}
\end{center}
%
The definition by |\providecommand| makes sure
that previous definitions are not overwritten.
Further statements |\providecommand{\version}{...}|
can thus be added before the above code to override it.

For the main file, one might add a line
(between |\childdocmain| and the above block)
%
\begin{center}
|%\ifchilddoc\||else\providecommand{\version}{draft}\||fi|
\end{center}
%
which can be uncommented to produce a draft version.
Likewise one can add a line to the very top of a child file
(above the |\childdocof{|\textit{main}|}| directive)
%
\begin{center}
|%\providecommand{\version}{final}|
\end{center}
%
which can be uncommented to produce the final version of this child document.

%%%%%%%%%%%%%%%%%%%%%%%%%%%%%%%%%%%%%%%%%%%%%%%%%%%%%%%%%%%%%%%%%%%%%%%%%%%%%%%%
\subsection{Forwarding}
\label{sec:forward}

Different versions of the main or child documents
using compilation flags as described in \secref{sec:flags}
can be (permanently) stored in different files
for convenient compilation, viewing and distribution.
To this end, the package defines a command
to pass on compilation to a different file:

%%%%%%%%%%%%%%%%%%%%%%%%%%%%%%%%%%%%%%%%
\DescribeMacro{\childdocforward}
The command |\childdocforward| redirects processing to
another source file:
%
\begin{center}
\begin{tabular}{l}
|\input{childdoc.def}|\\
|\childdocforward[|\textit{main}|]{|\textit{dest}|}|\\
\end{tabular}
\end{center}
%
The argument \textit{dest} is the destination file
(without extension).
It should be the main file or one of the child files.
Note that further \textsf{childdoc} directives
such as |\childdocof| and |\childdocforward|
in the indicated file will be processed in this form.
The optional argument \textit{main}
passes on directly to the main file \textit{main}
while pretending to compile the child \textit{dest}.
This form behaves as if \textit{dest}
issues |\childdocof{|\textit{main}|}| right away,
and no further \textsf{childdoc} directives will be processed.

%%%%%%%%%%%%%%%%%%%%%%%%%%%%%%%%%%%%%%%%
\DescribeMacro{\...prefix}
In the alternative form |\childdocforwardprefix|,
%
\begin{center}
\begin{tabular}{l}
|\input{childdoc.def}|\\
|\childdocforwardprefix[|\textit{main}|]{|\textit{prefix}|}{|\textit{dest}|}|
\end{tabular}
\end{center}
%
the destination file is determined by a pattern
depending on the current file:
To make this work, the current file must be called
`{\textit{prefix}\hspace{0.2em}\textit{suffix}}'
with \textit{prefix} matching precisely the argument.
Processing is then passed on to the file
`{\textit{dest}\hspace{0.2em}\textit{suffix}}'.
Surely, the same effect is achieved by
directly specifying the
argument `{\textit{dest}\hspace{0.2em}\textit{suffix}}'
in the first form.
However, that requires to set up a different file
for each child. With the alternative form of the command
all these files can have exactly the same content
which simplifies setting them up and maintaining them.

For example, the following file |draft.tex|
with a compilation flag |\version| as described in \secref{sec:flags}
compiles the main document as a draft:
%
\begin{center}
\begin{tabular}{l}
|\def\version{draft}|\\
|\input{childdoc.def}|\\
|\childdocforward{|\textit{main}|}|
\end{tabular}
\end{center}
%
Likewise, the following files |final|\textit{nn}|.tex|
compile the final version of the child document
|child|\textit{nn}|.tex|:
%
\begin{center}
\begin{tabular}{l}
|\def\version{final}|\\
|\input{childdoc.def}|\\
|\childdocforwardprefix{final}{child}|
\end{tabular}
\end{center}
%

Note that when several versions of a main file and/or of each child file
are to be generated, it may be convenient to set up a |Makefile| or
shell script to automatise the process.

%%%%%%%%%%%%%%%%%%%%%%%%%%%%%%%%%%%%%%%%%%%%%%%%%%%%%%%%%%%%%%%%%%%%%%%%%%%%%%%%
\subsection{Command Line Processing}
\label{sec:commandline}

The effect of redirection files can also be achieved by invoking
the \LaTeX{} compiler with a more elaborate command line.
Most conveniently this should be done as part
of a shell script or a |Makefile|.

When using \textsf{childdoc} in the main file, the following
command lines effectively perform a redirection
(note that depending on the shell being used,
backslashes may have to be doubled: `|\|' $\to$ `|\\|'):
%
\begin{center}
|... -jobname "|\textit{target}|" |\\|"|[\textit{flags}]%
|\input{childdoc.def}\childdocforward[|\textit{main}|]{|\textit{dest}|}"|
\end{center}
%
Here \textit{target} is the name of the output file,
\textit{main} is the name of the main file
and \textit{dest} is the name of the main or child file to be processed
(all filenames without extensions).
The optional argument \textit{main} can be omitted
if \textit{main} matches \textit{dest}.
Optionally, compilation \textit{flags} can be defined via |\def| commands.
This command line makes the \TeX{} engine believe
it is compiling the file \textit{target}
whose content is specified as the latter parameter.
The provided code then forwards the processing to
\textit{main} or \textit{dest} as described in \secref{sec:forward}.

%%%%%%%%%%%%%%%%%%%%%%%%%%%%%%%%%%%%%%%%%%%%%%%%%%%%%%%%%%%%%%%%%%%%%%%%%%%%%%%%
\subsection{Include by Input}
\label{sec:input}

Including child documents by |\include| has some restrictions by design.
Most notably, the content of a child document always occupies
its own set of pages; pages cannot be shared between child documents.
Usually, this behaviour makes perfect sense
because each child document contain an essential part of the document.
However, in some situations it may be desirable to compose
a document from a collection of parts
without having mandatory page breaks between then.
For this case, the package
provides a mechanism to include parts
by |\input| which can also be processed individually.
However, by construction this mechanism
requires manual handling of the content to be output.

%%%%%%%%%%%%%%%%%%%%%%%%%%%%%%%%%%%%%%%%
\DescribeMacro{\ifchilddocmanual}
The main file should be prepared as usual, see \secref{sec:include}.
However, the document body must make a distinction
between processing of an individual part and of the main document, e.g.:
%
\begin{center}
\begin{tabular}{l}
|\ifchilddocmanual|\\
|\input{\childdocname}|\\
|\||else|\\
\textit{document body with }|\input{|\textit{part}|}|\\
|\||fi|
\end{tabular}
\end{center}
%
The conditional |\ifchilddocmanual| is true whenever
a part to be included by |\input| is being compiled,
and the name of the part is stored in |\childdocname|.

%%%%%%%%%%%%%%%%%%%%%%%%%%%%%%%%%%%%%%%%
\DescribeMacro{\childdocby}
Each part to be included by |\input| should start with:
%
\begin{center}
\begin{tabular}{l}
|\input{childdoc.def}|\\
|\childdocby{|\textit{main}|}|\\
\end{tabular}
\end{center}
%
The directive |\childdocby| is similar to |\childdocof|
described in \secref{sec:include},
but the subsequent selection of content must be done manually.
To that end, both |\ifchilddoc| and |\ifchilddocmanual|
will be true upon processing of a part,
and the name of the part is stored in |\childdocname|.
Note that |\jobname| will be set to the filename of the current part
so that each part receives an individual |.aux| file
that does not interfere with the |.aux| file(s) of the main document.
This behaviour can be altered by the alternative form
|\childdocby[*]{|\textit{main}|}| (with a non-empty optional argument)
which uses the |.aux| file of the main document
by setting |\jobname| to \textit{main}.

%%%%%%%%%%%%%%%%%%%%%%%%%%%%%%%%%%%%%%%%%%%%%%%%%%%%%%%%%%%%%%%%%%%%%%%%%%%%%%%%
\subsection{Driver Development}
\label{sec:driver}

The \textsf{childdoc} mechanism can also be use for the development
of definition files such as \LaTeX{} styles or classes.
This case differs from the above setup with multiple parts
included by |\include| in that no |\includeonly| should be invoked.
This can be achieved by starting the include file
(before |\ProvidesPackage|) with:
%
\begin{center}
\begin{tabular}{l}
|\input{childdoc.def}|\\
|\childdocforward{|\textit{main}|}|\\
\end{tabular}
\end{center}
%
or alternatively with:
%
\begin{center}
\begin{tabular}{l}
|\input{childdoc.def}|\\
|\childdocby{|\textit{main}|}|\\
\end{tabular}
\end{center}
%
Both forms have slightly different effects as described above.
The main file is prepared as usual, see \secref{sec:include}.

%%%%%%%%%%%%%%%%%%%%%%%%%%%%%%%%%%%%%%%%%%%%%%%%%%%%%%%%%%%%%%%%%%%%%%%%%%%%%%%%
\subsection{Legacy Detection}
\label{sec:detection}

The directive |\childdocmain| in the main file can detect
whether the complete document or merely a child is to be compiled
even without using the directive |\childdocof|.
This method is deprecated because it is less robust
and there is no compelling reason to use it;
it is merely provided for backward compatibility
and it may be removed in future versions.

If the detection mechanism is to be used,
it is mandatory to correctly specify
the filename of the main file as the argument of |\childdocmain|:
%
\begin{center}
\begin{tabular}{l}
|\input{childdoc.def}|\\
|\childdocmain{|\textit{main}|}|\\
\end{tabular}
\end{center}
%
If |\jobname| does not match the argument \textit{main} of |\childdocmain|,
it is assumed that |\jobname| points to the child file to be compiled.
When using |\childdocmain| with the main file specified as argument,
it suffices to start a child file
with just |\input{|\textit{main}|}|
without loading of the package and using |\childdocof|.
If instead all processing is done
with the appropriate \textsf{childdoc} directives,
the argument of \textit{main} of |\childdocmain| can be empty.

An alternative version of the command line processing described
in \secref{sec:commandline} using the detection mechanism reads:
%
\begin{center}
|... -jobname "|\textit{target}|" "|[\textit{flags}]%
[|\def\jobname{|\textit{dest}|}|]|\input{|\textit{main}|}"|
\end{center}

%%%%%%%%%%%%%%%%%%%%%%%%%%%%%%%%%%%%%%%%%%%%%%%%%%%%%%%%%%%%%%%%%%%%%%%%%%%%%%%%
\subsection{Manual Code}
\label{sec:manual}

In case one cannot be certain whether the definitions file |childdoc.def|
is installed on the target \TeX{} distribution
and one prefers not to ship it,
it is conceivable to paste a few relevant commands into the sources.

To that end, drop all statements |\input{childdoc.def}|
and perform the replacements as outlined below.
Instead of |\childdocmain{|\textit{main}|}| add the following code
to the top of the main file:
%
\begin{center}
\begin{tabular}{l}
|\||ifdefined\childdocname\endinput\||fi\newif\ifchilddoc|\\
|\edef\childdocname{\scantokens\expandafter{\jobname\noexpand}}|\\
|\def\childdocmain{|\textit{main}|}\||ifx\childdocmain\childdocname\||else|\\
|\childdoctrue\includeonly{\childdocname}\let\jobname\childdocmain\||fi|\\
\end{tabular}
\end{center}
%
Instead of |\childdocof{|\textit{main}|}| just include the main file
at the top of each child file:
%
\begin{center}
|\input{|\textit{main}|}|
\end{center}
%
A simple redirection |\childdocforward{|\textit{dest}|}| is achieved by:
%
\begin{center}
|\def\jobname{|\textit{dest}|}\input{\jobname}|
\end{center}
%
The redirection with prefix
|\childdocforwardprefix[|\textit{prefix}|]{|\textit{dest}|}|
is accomplished by:
%
\begin{center}
\begin{tabular}{l}
|{\edef\jobname{\scantokens\expandafter{\jobname\noexpand}}|\\
|\def\redirectjob |\textit{prefix}|#1~~~{\gdef\jobname{|\textit{dest}|#1}}|\\
|\expandafter\redirectjob\jobname~~~}\input{\jobname}|
\end{tabular}
\end{center}

In an alternative approach,
child documents can be compiled by a specific command line
without additional code or specific definitions:
%
\begin{center}
|... -jobname "|\textit{target}|" "|[\textit{flags}]%
|\includeonly{|\textit{dest}|}\input{|\textit{main}|}"|
\end{center}
%

%%%%%%%%%%%%%%%%%%%%%%%%%%%%%%%%%%%%%%%%%%%%%%%%%%%%%%%%%%%%%%%%%%%%%%%%%%%%%%%%
%%%%%%%%%%%%%%%%%%%%%%%%%%%%%%%%%%%%%%%%%%%%%%%%%%%%%%%%%%%%%%%%%%%%%%%%%%%%%%%%
\section{Information}

%%%%%%%%%%%%%%%%%%%%%%%%%%%%%%%%%%%%%%%%%%%%%%%%%%%%%%%%%%%%%%%%%%%%%%%%%%%%%%%%
\subsection{Copyright}

Copyright \copyright{} 2017--2018 Niklas Beisert

This work may be distributed and/or modified under the
conditions of the \LaTeX{} Project Public License, either version 1.3
of this license or (at your option) any later version.
The latest version of this license is in
  \url{http://www.latex-project.org/lppl.txt}
and version 1.3 or later is part of all distributions of \LaTeX{}
version 2005/12/01 or later.

This work has the LPPL maintenance status `maintained'.

The Current Maintainer of this work is Niklas Beisert.

This work consists of the files |README.txt|, |childdoc.ins| and |childdoc.dtx|
as well as the derived files |childdoc.def|, |cdocsamp.tex|
with |cdocsch1.tex|, |cdocsch2.tex|, |cdocspt3.tex|, |cdocspt4.tex|,
|cdocsdrf.tex|, |cdocsfn1.tex|, |cdocsfn2.tex|
as well as |childdoc.pdf|.

%%%%%%%%%%%%%%%%%%%%%%%%%%%%%%%%%%%%%%%%%%%%%%%%%%%%%%%%%%%%%%%%%%%%%%%%%%%%%%%%
\subsection{Files and Installation}

The package consists of the files:
%
\begin{center}
\begin{tabular}{ll}
    |README.txt|   & readme file \\
    |childdoc.ins| & installation file \\
    |childdoc.dtx| & source file \\
    |childdoc.def| & definition file \\
    |cdocsamp.tex| & sample main file \\
    |cdocsch1.tex| & sample include file \\
    |cdocsch2.tex| & sample include file \\
    |cdocspt3.tex| & sample part file \\
    |cdocspt4.tex| & sample part file \\
    |cdocsdrf.tex| & sample redirection file \\
    |cdocsfn1.tex| & sample redirection file \\
    |cdocsfn2.tex| & sample redirection file \\
    |childdoc.pdf| & manual
\end{tabular}
\end{center}
%
The distribution consists of the files
|README.txt|, |childdoc.ins| and |childdoc.dtx|.
%
\begin{itemize}
\item
Run (pdf)\LaTeX{} on |childdoc.dtx|
to compile the manual |childdoc.pdf| (this file).
\item
Run \LaTeX{} on |childdoc.ins| to create the definitions file |childdoc.def|
and the sample |cdocsamp.tex| with include files
|cdocsch1.tex|, |cdocsch2.tex|, |cdocspt3.tex|, |cdocspt4.tex|,
|cdocsdrf.tex|, |cdocsfn1.tex|, |cdocsfn2.tex|.
Then copy the file |childdoc.def| to an appropriate directory of your \LaTeX{}
distribution, e.g.\ \textit{texmf-root}|/tex/latex/childdoc|.
\end{itemize}

%%%%%%%%%%%%%%%%%%%%%%%%%%%%%%%%%%%%%%%%%%%%%%%%%%%%%%%%%%%%%%%%%%%%%%%%%%%%%%%%
\subsection{Related CTAN Packages}

There are several other packages which offer a similar functionality:
%
\begin{itemize}
\item
The packages
\href{http://ctan.org/pkg/docmute}{\textsf{docmute}},
\href{http://ctan.org/pkg/includex}{\textsf{includex}} and
\href{http://ctan.org/pkg/standalone}{\textsf{standalone}}
provide commands to include only the document body of
a child file thus allowing both files to be compiled individually.
\item
The packages \href{http://ctan.org/pkg/subdocs}{\textsf{subdocs}}
and \href{http://ctan.org/pkg/subfiles}{\textsf{subfiles}}
provide structures in which the main and child documents can be
encapsulated and allowing them to be compiled individually.
The inclusion mechanism is different from the conventional |\include|.
\item
The package \href{http://ctan.org/pkg/combine}{\textsf{combine}}
is an elaborate solution to combine several documents into one.
\end{itemize}
%
See also the CTAN topic \href{http://ctan.org/topic/subdocs}{\textsf{subdocs}}
for further related packages.
The present package differs from the above solutions in that
a document structure constructed with the conventional |\include| mechanism
just needs two extra commands at the top of every file
such that all constituent files can be compiled individually.

%%%%%%%%%%%%%%%%%%%%%%%%%%%%%%%%%%%%%%%%%%%%%%%%%%%%%%%%%%%%%%%%%%%%%%%%%%%%%%%%
%\subsection{Feature Suggestions}
%
%The following is a list of features which may be useful for future
%versions of this package:
%%
%\begin{itemize}
%\item
%\ldots
%\end{itemize}

%%%%%%%%%%%%%%%%%%%%%%%%%%%%%%%%%%%%%%%%%%%%%%%%%%%%%%%%%%%%%%%%%%%%%%%%%%%%%%%%
\subsection{Revision History}

%%%%%%%%%%%%%%%%%%%%%%%%%%%%%%%%%%%%%%%%
\paragraph{v2.0:} 2018/12/30

\begin{itemize}
\item
immediate forward processing
\item
added |\childdocby| mechanism
\item
manual restructured
\end{itemize}

%%%%%%%%%%%%%%%%%%%%%%%%%%%%%%%%%%%%%%%%
\paragraph{v1.6:} 2018/01/17

\begin{itemize}
\item
application for development of include files
\item
corrections to manual
\end{itemize}

%%%%%%%%%%%%%%%%%%%%%%%%%%%%%%%%%%%%%%%%
\paragraph{v1.5:} 2017/05/21

\begin{itemize}
\item
more complete structuring introduced
\item
|\childdocof| introduced
\item
|\childdoc| renamed to |\childdocmain|
\item
|\childredirect| renamed to |\childdocforward| and |\childdocforwardprefix|
and functionality expanded
\end{itemize}

%%%%%%%%%%%%%%%%%%%%%%%%%%%%%%%%%%%%%%%%
\paragraph{v1.0:} 2017/04/27

\begin{itemize}
\item
manual and install package
\item
first version published on CTAN
\end{itemize}

%%%%%%%%%%%%%%%%%%%%%%%%%%%%%%%%%%%%%%%%
\paragraph{v0.6:} 2017/04/26

\begin{itemize}
\item
redirection mechanism added
\end{itemize}

%%%%%%%%%%%%%%%%%%%%%%%%%%%%%%%%%%%%%%%%
\paragraph{v0.5:} 2017/04/26

\begin{itemize}
\item
functionality in definition file
\end{itemize}


%%%%%%%%%%%%%%%%%%%%%%%%%%%%%%%%%%%%%%%%%%%%%%%%%%%%%%%%%%%%%%%%%%%%%%%%%%%%%%%%
%%%%%%%%%%%%%%%%%%%%%%%%%%%%%%%%%%%%%%%%%%%%%%%%%%%%%%%%%%%%%%%%%%%%%%%%%%%%%%%%
%%%%%%%%%%%%%%%%%%%%%%%%%%%%%%%%%%%%%%%%%%%%%%%%%%%%%%%%%%%%%%%%%%%%%%%%%%%%%%%%
\appendix

\settowidth\MacroIndent{\rmfamily\scriptsize 000\ }

 \DocInput{childdoc.dtx}

\end{document}
%</driver>
% \fi
%
% %%%%%%%%%%%%%%%%%%%%%%%%%%%%%%%%%%%%%%%%%%%%%%%%%%%%%%%%%%%%%%%%%%%%%%%%%%%%%%
% %%%%%%%%%%%%%%%%%%%%%%%%%%%%%%%%%%%%%%%%%%%%%%%%%%%%%%%%%%%%%%%%%%%%%%%%%%%%%%
% \section{Sample}
%\iffalse
%<*samplemain>
%\fi
%
% The following presents a sample document
% with two chapters, two parts, a title page,
% a compile flag as well as three forwarding files to set the flag.
% It consists of eight |.tex| files:
% \begin{center}
% \begin{tabular}{ll}
% |cdocsamp.tex|&main file\\
% |cdocsch1.tex|&include file for chapter 1\\
% |cdocsch2.tex|&include file for chapter 2\\
% |cdocspt3.tex|&include file for part 3\\
% |cdocspt4.tex|&include file for part 4\\
% |cdocsdrf.tex|&forwarding file for main file in draft mode\\
% |cdocsfi1.tex|&forwarding file for final version of chapter 1\\
% |cdocsfi2.tex|&forwarding file for final version of chapter 2\\
% \end{tabular}
% \end{center}
% Each of the eight files can be compiled directly by the \LaTeX{} compiler.
%
% %%%%%%%%%%%%%%%%%%%%%%%%%%%%%%%%%%%%%%
% \paragraph{Main File.}
%
% The main file is called |cdocsamp.tex|.
%
% Load the \textsf{childdoc} definitions and
% declare the filename for the main document:
%    \begin{macrocode}
\input{childdoc.def}
\childdocmain{}
%    \end{macrocode}

% Optional override for |\version| flag:
%    \begin{macrocode}
%%\ifchilddoc\else\providecommand{\version}{draft}\fi
%    \end{macrocode}

% Define the default values for the |\version| flag
% (|final| for the main file and |draft| for childs):
%    \begin{macrocode}
\ifchilddoc
\providecommand{\version}{draft}
\else
\providecommand{\version}{final}
\fi
%    \end{macrocode}

% Load the standard document class:
%    \begin{macrocode}
\documentclass[12pt]{article}
%    \end{macrocode}

% Start the document body:
%    \begin{macrocode}
\begin{document}
%    \end{macrocode}

% Declare a title page.
% Print title, part of document being processed and version flag:
%    \begin{macrocode}
\addtocounter{page}{-1}
\begin{center}
{\LARGE\bfseries{}childdoc example\par}
\vspace{1cm}
\ifchilddoc
\ifchilddocmanual part\else chapter\fi:
`\childdocname' of `\childdocjob'\par
\else
main document: `\childdocjob'\par
\fi
version: \version\par
\end{center}
\newpage
%    \end{macrocode}

% Manually include selected file,
% otherwise process as usual:
%    \begin{macrocode}
\ifchilddocmanual
\section*{part `\childdocname'}
\input{\childdocname}
\else
%    \end{macrocode}

% Include the two chapters:
%    \begin{macrocode}
\include{cdocsch1}
\include{cdocsch2}
%    \end{macrocode}

% Include the two parts unless only chapters should be displayed:
%    \begin{macrocode}
\ifchilddoc\else
\section{part three}
\input{cdocspt3}
\section{part four}
\input{cdocspt4}
\fi
%    \end{macrocode}

% Process as usual until here:
%    \begin{macrocode}
\fi
%    \end{macrocode}

% End of document body:
%    \begin{macrocode}
\end{document}
%    \end{macrocode}
%\iffalse
%</samplemain>
%\fi
%
% %%%%%%%%%%%%%%%%%%%%%%%%%%%%%%%%%%%%%%
% \paragraph{Chapter Include Files.}
%
% The include files are called |cdocsch1.tex| and |cdocsch2.tex|.
%
%\iffalse
%<*samplechap1|samplechap2>
%\fi

% Optional override for |\version| flag:
%    \begin{macrocode}
%%\providecommand{\version}{final}
%    \end{macrocode}

% Include the main document:
%    \begin{macrocode}
\input{childdoc.def}
\childdocof{cdocsamp}
%    \end{macrocode}

%\iffalse
%</samplechap1|samplechap2>
%\fi
%
%\iffalse
%<*samplechap1>
%\fi
% Some text for chapter 1:
%    \begin{macrocode}
\section{one}
some text in chapter one
%    \end{macrocode}

%\iffalse
%</samplechap1>
%\fi
% Some text for chapter 2:
%\iffalse
%<*samplechap2>
%\fi
%    \begin{macrocode}
\section{two}
more text in chapter two
%    \end{macrocode}

%\iffalse
%</samplechap2>
%\fi
%
% %%%%%%%%%%%%%%%%%%%%%%%%%%%%%%%%%%%%%%
% \paragraph{Part Include Files.}
%
% The include files are called |cdocspt3.tex| and |cdocspt4.tex|.
%
%\iffalse
%<*samplepart3|samplepart4>
%\fi

% Optional override for |\version| flag:
%    \begin{macrocode}
%%\providecommand{\version}{final}
%    \end{macrocode}

% Include the main document:
%    \begin{macrocode}
\input{childdoc.def}
\childdocby{cdocsamp}
%    \end{macrocode}

%\iffalse
%</samplepart3|samplepart4>
%\fi
%
%\iffalse
%<*samplepart3>
%\fi
% Some text for part 3:
%    \begin{macrocode}
some text in part three
%    \end{macrocode}

%\iffalse
%</samplepart3>
%\fi
% Some text for part 4:
%\iffalse
%<*samplepart4>
%\fi
%    \begin{macrocode}
more text in part four
%    \end{macrocode}

%\iffalse
%</samplepart4>
%\fi
%
% %%%%%%%%%%%%%%%%%%%%%%%%%%%%%%%%%%%%%%
% \paragraph{Forwarding for a Complete Draft.}
%
% The following forwarding file |cdocsdrf.tex|
% compiles the main document in draft mode:
%\iffalse
%<*sampledraft>
%\fi
%    \begin{macrocode}
\def\version{draft}
\input{childdoc.def}
\childdocforward{cdocsamp}
%    \end{macrocode}

%\iffalse
%</sampledraft>
%\fi
%
% %%%%%%%%%%%%%%%%%%%%%%%%%%%%%%%%%%%%%%
% \paragraph{Forwarding for Final Version of the Chapters.}
%
% The following forwarding files |cdocsfn1.tex| and |cdocsfn2.tex|
% (with identical content)
% compile the final versions of the child documents
% |cdocsch1.tex| and |cdocsch2.tex|, respectively:
%\iffalse
%<*samplefinal>
%\fi
%    \begin{macrocode}
\def\version{final}
\input{childdoc.def}
\childdocforwardprefix[cdocsamp]{cdocsfn}{cdocsch}
%    \end{macrocode}

%\iffalse
%</samplefinal>
%\fi
%
% %%%%%%%%%%%%%%%%%%%%%%%%%%%%%%%%%%%%%%
% \paragraph{Command Line Processing.}
%
% The following three command lines generate the output files
% |cdocscld|, |cdocscl1| and |cdocscl2|
% which should be identical to
% |cdocsdrf|, |cdocsch1| and |cdocsfn2|, respectively:
% \begin{center}
% \begin{tabular}{l}
% |latex -jobname cdocscld \|\\
% |  "\def\version{draft}\input{childdoc.def}\childdocforward{cdocsamp}"|\\
% |latex -jobname cdocscl1 \|\\
% |  "\input{childdoc.def}\childdocforward[cdocsamp]{cdocsch1}"|\\
% |latex -jobname cdocscl2 \|\\
% |  "\def\version{final}\input{childdoc.def}\childdocforward{cdocsch2}"|
% \end{tabular}
% \end{center}
% Note that the trailing backslash on each first line
% merely continues the input to the second line
% (for convenient cut ant paste).
% Furthermore, the command |latex| can be replaced by any
% of its alternative versions such as |pdflatex|.
%
% %%%%%%%%%%%%%%%%%%%%%%%%%%%%%%%%%%%%%%%%%%%%%%%%%%%%%%%%%%%%%%%%%%%%%%%%%%%%%%
% %%%%%%%%%%%%%%%%%%%%%%%%%%%%%%%%%%%%%%%%%%%%%%%%%%%%%%%%%%%%%%%%%%%%%%%%%%%%%%
% \section{Implementation}
%\iffalse
%<*package>
%\fi
%
% This section describes the definitions file |childdoc.def|.

% The definitions cannot be loaded using |\usepackage| or |\RequirePackage|
% which has a mechanism to prevent loading a style file more than once.
% When loading the definitions by means of |\input|
% multiple instances have to be prevented manually:
%\iffalse
%This code needs to be before the `\ProvidesFile' directive
%which is defined at the beginning of this file.
%Therefore it is also placed there and commented out here.
%</package>
%<*discard>
%\fi
%    \begin{macrocode}
\ifdefined\childdocmain\endinput\fi
%    \end{macrocode}
%\iffalse
%</discard>
%<*package>
%\fi
%
% \macro{\ifchilddoc}
% \macro{\ifchilddocmanual}
% The conditional |\ifchilddoc| tells whether a
% child (true) or main (false) document is being compiled.
% The conditional |\ifchilddocmanual| tells whether
% the |\includeonly| mechanism is used (false) or
% the selection of child files must be performed manually (true).
% The definitions initialise to false:
%    \begin{macrocode}
\newif\ifchilddoc
\newif\ifchilddocmanual
%    \end{macrocode}

% \macro{\childdocname}
% \macro{\childdocjob}
% The macro |\childdocname| stores the name of the main document
% to be compiled. The macro |\childdocjob| stores the name of
% the document on which the \LaTeX{} compiler was originally invoked.
% The content of |\jobname| cannot be compared
% to filenames specified in the source due to different catcodes.
% The following code rescans |\jobname|, stores the result
% in |\childdocname| and saves a copy in |\childdocjob|:
%    \begin{macrocode}
\edef\childdocname{\scantokens\expandafter{\jobname\noexpand}}
\let\childdocjob\childdocname
%    \end{macrocode}

% \macro{\childdocdisable}
% The macro |\childdocdisable| prevents the main file
% from being processed more than once.
% At this stage, the main document command |\childdocmain|
% is assumed to be called once again where it should do nothing.
% Any subsequent call to it should prevent
% a secondary processing of the main document
% It overwrites the forwarding commands
% |\childdocof| and |\childdocforward|
% with empty macros to prevent further inclusions of the main document:
%    \begin{macrocode}
\newcommand{\childdocdisable}
{
  \renewcommand{\childdocmain}[1]{\renewcommand{\childdocmain}[1]{\endinput}}
  \renewcommand{\childdocof}[1]{}
  \renewcommand{\childdocby}[2][]{}
  \renewcommand{\childdocforward}[2][]{}
  \renewcommand{\childdocdisable}{}
}
%    \end{macrocode}

% \macro{\childdocmain}
% The macro |\childdocmain| is to be called at the top of the main file
% with nothing or the main filename (without extension) as argument.
% First, it breaks loops.
% If the argument is not empty and does not match |\childdocname|
% (which is set by the first inclusion of |childdoc.def|),
% |\ifchilddoc| is set to true, |\includeonly| is applied to the child file
% and |\jobname| is set to the main file
% (for proper handling of |.aux| files):
%    \begin{macrocode}
\newcommand{\childdocmain}[1]
{
  \childdocdisable\childdocmain{}
  \if?#1?\else
    \begingroup
      \def\childdoctmp{#1}
      \ifx\childdoctmp\childdocname
        \def\childdoctmp{}
      \else
        \def\childdoctmp
        {
          \childdoctrue
          \includeonly{\childdocname}
          \def\childdocjob{#1}
          \def\jobname{#1}
        }
      \fi
      \expandafter
    \endgroup
    \childdoctmp
  \fi
}
%    \end{macrocode}

% \macro{\childdocof}
% The command |\childdocof| redirects
% compilation to the main file |#1|.
%    \begin{macrocode}
\newcommand{\childdocof}[1]
{
  \childdocdisable
  \childdoctrue
  \includeonly{\childdocname}
  \def\jobname{#1}
  \def\childdocjob{#1}
  \input{#1}
}
%    \end{macrocode}

% \macro{\childdocby}
% The command |\childdocby| ....
%    \begin{macrocode}
\newcommand{\childdocby}[2][]
{
  \childdocdisable
  \childdoctrue
  \childdocmanualtrue
  \if?#1?\else
    \def\jobname{#2}
  \fi
  \def\childdocjob{#2}
  \input{#2}
  \endinput
}
%    \end{macrocode}

% \macro{\childdocforward}
% The command |\childdocforward| redirects
% compilation to the main file or
% (if the optional argument is given) a child file.
% Parameters are set as if the main file
% or a child file starting with |\childdocof| was compiled.
% Then compilation is handed over to the main file:
%    \begin{macrocode}
\newcommand{\childdocforward}[2][]
{
  \begingroup
    \if?#1?
      \def\childdoctmp
      {
        \def\childdocname{#2}
        \def\childdocjob{#2}
        \def\jobname{#2}
        \input{#2}
        \endinput
      }
    \else
      \def\childdoctmp
      {
        \childdocdisable
        \def\childdocname{#2}
        \childdoctrue
        \includeonly{#2}
        \def\childdocjob{#1}
        \def\jobname{#1}
        \input{#1}
        \endinput
      }
    \fi
    \expandafter
  \endgroup
  \childdoctmp
}
%    \end{macrocode}

% \macro{\childdocforwardprefix}
% The command |\childdocforwardprefix| redirects
% compilation to the main or a child file by means of a pattern.
% The prefix |#1| in the current filename is replaced by |#2|
% and the suffix of the current filename is kept
% (it is assumed that the filename does not contain the substring `|~~~|'
% which is used as a delimiter).
% Compilation is handed over to the new file by |\childdocforward|:
%    \begin{macrocode}
\newcommand{\childdocforwardprefix}[3][]
{
  \begingroup
    \def\childdocextract #2##1~~~{\def\childdoctmp{\childdocforward[#1]{#3##1}}}
    \expandafter\childdocextract\childdocname~~~
    \expandafter
  \endgroup
  \childdoctmp
}
%    \end{macrocode}

% \macro{\childdoc}
% The deprecated macro |\childdoc| is a legacy version of |\childdocmain|:
%    \begin{macrocode}
\newcommand{\childdoc}{\childdocmain}
%    \end{macrocode}

% \macro{\childdocredirect}
% The deprecated macro |\childdocredirect| is a legacy version
% of |\childdocforward| and |\childdocforwardprefix|:
%    \begin{macrocode}
\newcommand{\childdocredirect}[2][]
{
  \begingroup
    \if?#1?
      \def\childdoctmp{\childdocforward{#2}}
    \else
      \def\childdoctmp{\childdocforwardprefix{#1}{#2}}
    \fi
    \expandafter
  \endgroup
  \childdoctmp
}
%    \end{macrocode}

%\iffalse
%</package>
%\fi
%
\endinput
|\\
|\childdocforward{|\textit{main}|}|\\
\end{tabular}
\end{center}
%
or alternatively with:
%
\begin{center}
\begin{tabular}{l}
|% \iffalse
%
% childdoc.dtx Copyright (C) 2017-2018 Niklas Beisert
%
% This work may be distributed and/or modified under the
% conditions of the LaTeX Project Public License, either version 1.3
% of this license or (at your option) any later version.
% The latest version of this license is in
%   http://www.latex-project.org/lppl.txt
% and version 1.3 or later is part of all distributions of LaTeX
% version 2005/12/01 or later.
%
% This work has the LPPL maintenance status `maintained'.
%
% The Current Maintainer of this work is Niklas Beisert.
%
% This work consists of the files childdoc.dtx and childdoc.ins
% and the derived files childdoc.def and cdocsamp.tex with
% cdocsch1.tex, cdocsch2.tex, cdocsdrf.tex, cdocsfn1.tex, cdocsfn2.tex.
%
%<package>\ifdefined\childdocmain\endinput\fi
%<package>\ProvidesFile{childdoc.def}[2018/12/30 v2.0 child document driver]
%<samplemain>\ProvidesFile{cdocsamp.tex}[2018/12/30 v2.0 sample for childdoc]
%<*driver>
%\ProvidesFile{childdoc.drv}[2018/12/30 v2.0 childdoc reference manual file]
\PassOptionsToClass{10pt,a4paper}{article}
\documentclass{ltxdoc}

\usepackage[margin=35mm]{geometry}
\usepackage{hyperref}
\usepackage{hyperxmp}
\usepackage[usenames]{color}

\hypersetup{colorlinks=true}
\hypersetup{pdfstartview=FitH}
\hypersetup{pdfpagemode=UseNone}
\hypersetup{pdfsource={}}
\hypersetup{pdflang={en-UK}}
\hypersetup{pdfcopyright={Copyright 2017-2018 Niklas Beisert.
  This work may be distributed and/or modified under the
  conditions of the LaTeX Project Public License, either version 1.3
  of this license or (at your option) any later version.}}
\hypersetup{pdflicenseurl={http://www.latex-project.org/lppl.txt}}
\hypersetup{pdfcontactaddress={ETH Zurich, ITP, HIT K,
  Wolfgang-Pauli-Strasse 27}}
\hypersetup{pdfcontactpostcode={8093}}
\hypersetup{pdfcontactcity={Zurich}}
\hypersetup{pdfcontactcountry={Switzerland}}
\hypersetup{pdfcontactemail={nbeisert@itp.phys.ethz.ch}}
\hypersetup{pdfcontacturl={http://people.phys.ethz.ch/\xmptilde nbeisert/}}

\newcommand{\secref}[1]{\hyperref[#1]{section \ref*{#1}}}

\parskip1ex
\parindent0pt
\let\olditemize\itemize
\def\itemize{\olditemize\parskip0pt}

\begin{document}

\title{The \textsf{childdoc} Package}
\hypersetup{pdftitle={The childdoc Package}}
\author{Niklas Beisert\\[2ex]
  Institut f\"ur Theoretische Physik\\
  Eidgen\"ossische Technische Hochschule Z\"urich\\
  Wolfgang-Pauli-Strasse 27, 8093 Z\"urich, Switzerland\\[1ex]
  \href{mailto:nbeisert@itp.phys.ethz.ch}
  {\texttt{nbeisert@itp.phys.ethz.ch}}}
\hypersetup{pdfauthor={Niklas Beisert}}
\hypersetup{pdfsubject={Manual for the LaTeX2e Package childdoc}}
\date{30 December 2018, \textsf{v2.0}}
\maketitle

\begin{abstract}\noindent
\textsf{childdoc} is a \LaTeXe{} package
that enables the direct compilation
of document sections included by |\include|
to individual files.
\end{abstract}

\begingroup
\parskip0ex
\tableofcontents
\endgroup

%%%%%%%%%%%%%%%%%%%%%%%%%%%%%%%%%%%%%%%%%%%%%%%%%%%%%%%%%%%%%%%%%%%%%%%%%%%%%%%%
%%%%%%%%%%%%%%%%%%%%%%%%%%%%%%%%%%%%%%%%%%%%%%%%%%%%%%%%%%%%%%%%%%%%%%%%%%%%%%%%
\section{Introduction}

\LaTeX{} provides a mechanism to structure a large document (such as a book)
into a main file and several child files (containing the chapters)
using the |\include| command.
This mechanism is beneficial for documents
which span hundreds of pages in order to
make the source file(s) more manageable.
Moreover, compilation can be restricted to
selected child files by means of the |\includeonly| command.
The latter feature can be used to reduce the compilation time while editing
(this was significantly more useful in the earlier days of \LaTeX{})
or to generate a smaller document which is easier to navigate.
Another application of |\includeonly| is to generate
documents consisting of selected parts of the complete document.

However, there are a few drawbacks of the plain |\include| mechanism:
\begin{itemize}
\item
The child files cannot be compiled on their own,
they can only be compiled via the main file.
A naive editing environment
(such as a text editor with an option
to have the current file processed by \LaTeX)
may require one to switch to the main file before compiling;
attempting to compile the child file produces errors.
\item
The main file must be modified (each time)
to adjust the |\includeonly| command
to the present needs. This easily leaves the main file in a messy state.
\item
The generated document will always carry the filename
of the main document. This is inconvenient if
several child files are to be compiled and
to be kept for distribution.
\end{itemize}

The present package provides a simple interface
to make child files individually compilable by \LaTeX{}.
Compiling a child file then has the same effect as compiling
the main file with an |\includeonly| command
to select the appropriate child.
Moreover the generated document will carry the name of the child
rather than the main file.
This resolves all three above issues.

This feature is meant to make the editing of books,
thesis documents and lecture notes somewhat more convenient.
However, the package can also be used efficiently for
composing a series of documents (such as exercise sheets)
which are typically distributed individually.
It then assists the author in generating the individual documents
(potentially in different versions)
as well as a document containing the collected series.
Another application is in developing style files
or other kinds of included material
where compilation of the style file could redirect
to a sample or test file.

%%%%%%%%%%%%%%%%%%%%%%%%%%%%%%%%%%%%%%%%%%%%%%%%%%%%%%%%%%%%%%%%%%%%%%%%%%%%%%%%
%%%%%%%%%%%%%%%%%%%%%%%%%%%%%%%%%%%%%%%%%%%%%%%%%%%%%%%%%%%%%%%%%%%%%%%%%%%%%%%%
\section{Usage}

First of all, the package \textsf{childdoc} is \emph{not} a standard
\LaTeXe{} |.sty| style file! Therefore it needs to be invoked in
a non-standard way.

%%%%%%%%%%%%%%%%%%%%%%%%%%%%%%%%%%%%%%%%%%%%%%%%%%%%%%%%%%%%%%%%%%%%%%%%%%%%%%%%
\subsection{Included Files}
\label{sec:include}

%%%%%%%%%%%%%%%%%%%%%%%%%%%%%%%%%%%%%%%%
\DescribeMacro{\childdocmain}
To use the package, add the commands
\begin{center}
\begin{tabular}{l}
|\input{childdoc.def}|\\
|\childdocmain{}|\\
\end{tabular}
\end{center}
at the very top of the main \LaTeX{} file,
in particular \emph{before} the |\documentclass| statement!
The argument of |\childdocmain| should be left empty
(but it must be present).

%%%%%%%%%%%%%%%%%%%%%%%%%%%%%%%%%%%%%%%%
\DescribeMacro{\childdocof}
Furthermore, add the commands
\begin{center}
\begin{tabular}{l}
|\input{childdoc.def}|\\
|\childdocof{|\textit{main}|}|\\
\end{tabular}
\end{center}
at the top of every child file \textit{child}
which is included by |\include{|\textit{child}|}|
from within the main file
(or at least for those files to be compiled individually).
The argument \textit{main} must be the filename of the main file.

There are a couple of
considerations in setting up the main and child documents:

%%%%%%%%%%%%%%%%%%%%%%%%%%%%%%%%%%%%%%%%
\paragraph{Restrictions.}

Please note the following restrictions:
\begin{itemize}
\item
|\childdocmain| must be called with one argument \textit{main}
to ensure compatibility with earlier version of the package.
It must either be empty (|\childdocmain{}|)
or precisely match the filename of the main file in which it is specified.
See \secref{sec:detection} for further information.
\item
The filename \textit{main} must be specified without the |.tex| extension.
\item
The filename \textit{main} is case sensitive
(even in case-insensitive file systems)
due to internal string comparison.
\item
The argument \textit{main} should be fully expanded, it cannot be a macro.
\item
Subdirectories and special characters should be avoided in filenames.
\item
The command |\childdocmain{|\textit{main}|}| must be followed by a whitespace.
It should not be followed immediately by another command
or by a comment mark `|%|'.
This is because the \TeX{} parser reads the token immediately following
the argument of |\childdocmain| and puts it
at the beginning of every child section;
however, a white\-space is ignored.
\end{itemize}

%%%%%%%%%%%%%%%%%%%%%%%%%%%%%%%%%%%%%%%%
\paragraph{Content of Main File.}

It is advisable to place all content in the child files included by |\include|.
Any output contained in the main file will appear in all child documents
unless suppressed manually;
it cannot be suppressed automatically by the |\includeonly| directive
and thus should normally be avoided.
A method to include some content in the main file
by means of conditional processing is described in \secref{sec:conditional}.

%%%%%%%%%%%%%%%%%%%%%%%%%%%%%%%%%%%%%%%%
\paragraph{Page Numbering.}

When only a part of the document is compiled,
the appropriate numbering of pages
(as well as other status parameters)
is determined from the |.aux| files.
The latter contain information from previous passes.
However this information needs to propagate through
all intermediate child documents.
Therefore the page numbering in child documents may well
be inconsistent until the complete document is compiled at least once.

A useful (if unconventional) way to always ensure a consistent
page numbering is to restart the numbering in each child document
and denote the pages by `\textit{child}|.|\textit{page}'
where \textit{child} represents the chapter/section number of the child file.
This can be achieved by the command
|\numberwithin{page}{|\textit{child}|}|
of the \textsf{amsmath} package
where \textit{child} can be |chapter| or |section|
depending on the chosen structuring.
Alternatively, one can modify the macro |\thepage| appropriately
and reset the counter |page| at the start of each child file.

%%%%%%%%%%%%%%%%%%%%%%%%%%%%%%%%%%%%%%%%%%%%%%%%%%%%%%%%%%%%%%%%%%%%%%%%%%%%%%%%
\subsection{Conditional Processing}
\label{sec:conditional}

The package provides a mechanism to compile different versions
of a document. To customise the versions further some conditional processing
can come in handy to distinguish which version is being compiled.
The package provides two macros to describe the compilation context:

%%%%%%%%%%%%%%%%%%%%%%%%%%%%%%%%%%%%%%%%
\DescribeMacro{\ifchilddoc}
The conditional |\ifchilddoc| distinguishes between the compilation of
child documents and the main document:
%
\begin{center}
|\ifchilddoc |\textit{child-code}| |[|\||else |\textit{main-code}]| \||fi|
\end{center}

%%%%%%%%%%%%%%%%%%%%%%%%%%%%%%%%%%%%%%%%
\DescribeMacro{\childdocname}
\DescribeMacro{\childdocjob}
The macro |\childdocname| contains the filename (without extension)
of the main or child file being processed.
Note that |\childdocjob| will always contain the name of the main file.

%%%%%%%%%%%%%%%%%%%%%%%%%%%%%%%%%%%%%%%%
\paragraph{Title Page.}

Conditional processing can be used to include a title or banner page
in the main document when proper precautions are taken.
Importantly, the code in the main file should ensure that the page counter
(as well as other status parameters which are stored in the |.aux| files)
takes the same value after the conditional processing.
Otherwise the page numbers may take divergent values
depending on which part is compiled.

For example, a title page could be declared by:
%
\begin{center}
\begin{tabular}{l}
|\ifchilddoc\||else|\\
|\addtocounter{page}{-1}|\\
\textit{code for title page}\\
|\newpage|\\
|\||fi|
\end{tabular}
\end{center}
%
A banner page for the child documents can be generated by:
%
\begin{center}
\begin{tabular}{l}
|\ifchilddoc|\\
|\addtocounter{page}{-1}|\\
\textit{code for banner page}\\
|\newpage|\\
|\||fi|
\end{tabular}
\end{center}
%
Here one could write a message such as:
\begin{center}
|This is the part \childdocname{} of \childdocjob{}.|
\end{center}

%%%%%%%%%%%%%%%%%%%%%%%%%%%%%%%%%%%%%%%%%%%%%%%%%%%%%%%%%%%%%%%%%%%%%%%%%%%%%%%%
\subsection{Flags}
\label{sec:flags}

The package makes it easy to generate different versions
of the main or child documents.
To this end compilation flags can be defined
and assigned different default values.
They will be particularly useful in conjunction
with the forwarding mechanism described in \secref{sec:forward}.

For example, it may be useful to have a flag |\version|
which can be set to |draft| or |final|.
The document source will contain some conditional code
depending on the value of |\version|.
Suppose further, the flag should default to |final| for the main file
and to |draft| for child files
which is a natural assignment for editing the document.
This is achieved by placing the following code
in the preamble of the main document
(below the |\childdocmain| directive):
%
\begin{center}
\begin{tabular}{l}
|\ifchilddoc|\\
|\providecommand{\version}{draft}|\\
|\||else|\\
|\providecommand{\version}{final}|\\
|\||fi|
\end{tabular}
\end{center}
%
The definition by |\providecommand| makes sure
that previous definitions are not overwritten.
Further statements |\providecommand{\version}{...}|
can thus be added before the above code to override it.

For the main file, one might add a line
(between |\childdocmain| and the above block)
%
\begin{center}
|%\ifchilddoc\||else\providecommand{\version}{draft}\||fi|
\end{center}
%
which can be uncommented to produce a draft version.
Likewise one can add a line to the very top of a child file
(above the |\childdocof{|\textit{main}|}| directive)
%
\begin{center}
|%\providecommand{\version}{final}|
\end{center}
%
which can be uncommented to produce the final version of this child document.

%%%%%%%%%%%%%%%%%%%%%%%%%%%%%%%%%%%%%%%%%%%%%%%%%%%%%%%%%%%%%%%%%%%%%%%%%%%%%%%%
\subsection{Forwarding}
\label{sec:forward}

Different versions of the main or child documents
using compilation flags as described in \secref{sec:flags}
can be (permanently) stored in different files
for convenient compilation, viewing and distribution.
To this end, the package defines a command
to pass on compilation to a different file:

%%%%%%%%%%%%%%%%%%%%%%%%%%%%%%%%%%%%%%%%
\DescribeMacro{\childdocforward}
The command |\childdocforward| redirects processing to
another source file:
%
\begin{center}
\begin{tabular}{l}
|\input{childdoc.def}|\\
|\childdocforward[|\textit{main}|]{|\textit{dest}|}|\\
\end{tabular}
\end{center}
%
The argument \textit{dest} is the destination file
(without extension).
It should be the main file or one of the child files.
Note that further \textsf{childdoc} directives
such as |\childdocof| and |\childdocforward|
in the indicated file will be processed in this form.
The optional argument \textit{main}
passes on directly to the main file \textit{main}
while pretending to compile the child \textit{dest}.
This form behaves as if \textit{dest}
issues |\childdocof{|\textit{main}|}| right away,
and no further \textsf{childdoc} directives will be processed.

%%%%%%%%%%%%%%%%%%%%%%%%%%%%%%%%%%%%%%%%
\DescribeMacro{\...prefix}
In the alternative form |\childdocforwardprefix|,
%
\begin{center}
\begin{tabular}{l}
|\input{childdoc.def}|\\
|\childdocforwardprefix[|\textit{main}|]{|\textit{prefix}|}{|\textit{dest}|}|
\end{tabular}
\end{center}
%
the destination file is determined by a pattern
depending on the current file:
To make this work, the current file must be called
`{\textit{prefix}\hspace{0.2em}\textit{suffix}}'
with \textit{prefix} matching precisely the argument.
Processing is then passed on to the file
`{\textit{dest}\hspace{0.2em}\textit{suffix}}'.
Surely, the same effect is achieved by
directly specifying the
argument `{\textit{dest}\hspace{0.2em}\textit{suffix}}'
in the first form.
However, that requires to set up a different file
for each child. With the alternative form of the command
all these files can have exactly the same content
which simplifies setting them up and maintaining them.

For example, the following file |draft.tex|
with a compilation flag |\version| as described in \secref{sec:flags}
compiles the main document as a draft:
%
\begin{center}
\begin{tabular}{l}
|\def\version{draft}|\\
|\input{childdoc.def}|\\
|\childdocforward{|\textit{main}|}|
\end{tabular}
\end{center}
%
Likewise, the following files |final|\textit{nn}|.tex|
compile the final version of the child document
|child|\textit{nn}|.tex|:
%
\begin{center}
\begin{tabular}{l}
|\def\version{final}|\\
|\input{childdoc.def}|\\
|\childdocforwardprefix{final}{child}|
\end{tabular}
\end{center}
%

Note that when several versions of a main file and/or of each child file
are to be generated, it may be convenient to set up a |Makefile| or
shell script to automatise the process.

%%%%%%%%%%%%%%%%%%%%%%%%%%%%%%%%%%%%%%%%%%%%%%%%%%%%%%%%%%%%%%%%%%%%%%%%%%%%%%%%
\subsection{Command Line Processing}
\label{sec:commandline}

The effect of redirection files can also be achieved by invoking
the \LaTeX{} compiler with a more elaborate command line.
Most conveniently this should be done as part
of a shell script or a |Makefile|.

When using \textsf{childdoc} in the main file, the following
command lines effectively perform a redirection
(note that depending on the shell being used,
backslashes may have to be doubled: `|\|' $\to$ `|\\|'):
%
\begin{center}
|... -jobname "|\textit{target}|" |\\|"|[\textit{flags}]%
|\input{childdoc.def}\childdocforward[|\textit{main}|]{|\textit{dest}|}"|
\end{center}
%
Here \textit{target} is the name of the output file,
\textit{main} is the name of the main file
and \textit{dest} is the name of the main or child file to be processed
(all filenames without extensions).
The optional argument \textit{main} can be omitted
if \textit{main} matches \textit{dest}.
Optionally, compilation \textit{flags} can be defined via |\def| commands.
This command line makes the \TeX{} engine believe
it is compiling the file \textit{target}
whose content is specified as the latter parameter.
The provided code then forwards the processing to
\textit{main} or \textit{dest} as described in \secref{sec:forward}.

%%%%%%%%%%%%%%%%%%%%%%%%%%%%%%%%%%%%%%%%%%%%%%%%%%%%%%%%%%%%%%%%%%%%%%%%%%%%%%%%
\subsection{Include by Input}
\label{sec:input}

Including child documents by |\include| has some restrictions by design.
Most notably, the content of a child document always occupies
its own set of pages; pages cannot be shared between child documents.
Usually, this behaviour makes perfect sense
because each child document contain an essential part of the document.
However, in some situations it may be desirable to compose
a document from a collection of parts
without having mandatory page breaks between then.
For this case, the package
provides a mechanism to include parts
by |\input| which can also be processed individually.
However, by construction this mechanism
requires manual handling of the content to be output.

%%%%%%%%%%%%%%%%%%%%%%%%%%%%%%%%%%%%%%%%
\DescribeMacro{\ifchilddocmanual}
The main file should be prepared as usual, see \secref{sec:include}.
However, the document body must make a distinction
between processing of an individual part and of the main document, e.g.:
%
\begin{center}
\begin{tabular}{l}
|\ifchilddocmanual|\\
|\input{\childdocname}|\\
|\||else|\\
\textit{document body with }|\input{|\textit{part}|}|\\
|\||fi|
\end{tabular}
\end{center}
%
The conditional |\ifchilddocmanual| is true whenever
a part to be included by |\input| is being compiled,
and the name of the part is stored in |\childdocname|.

%%%%%%%%%%%%%%%%%%%%%%%%%%%%%%%%%%%%%%%%
\DescribeMacro{\childdocby}
Each part to be included by |\input| should start with:
%
\begin{center}
\begin{tabular}{l}
|\input{childdoc.def}|\\
|\childdocby{|\textit{main}|}|\\
\end{tabular}
\end{center}
%
The directive |\childdocby| is similar to |\childdocof|
described in \secref{sec:include},
but the subsequent selection of content must be done manually.
To that end, both |\ifchilddoc| and |\ifchilddocmanual|
will be true upon processing of a part,
and the name of the part is stored in |\childdocname|.
Note that |\jobname| will be set to the filename of the current part
so that each part receives an individual |.aux| file
that does not interfere with the |.aux| file(s) of the main document.
This behaviour can be altered by the alternative form
|\childdocby[*]{|\textit{main}|}| (with a non-empty optional argument)
which uses the |.aux| file of the main document
by setting |\jobname| to \textit{main}.

%%%%%%%%%%%%%%%%%%%%%%%%%%%%%%%%%%%%%%%%%%%%%%%%%%%%%%%%%%%%%%%%%%%%%%%%%%%%%%%%
\subsection{Driver Development}
\label{sec:driver}

The \textsf{childdoc} mechanism can also be use for the development
of definition files such as \LaTeX{} styles or classes.
This case differs from the above setup with multiple parts
included by |\include| in that no |\includeonly| should be invoked.
This can be achieved by starting the include file
(before |\ProvidesPackage|) with:
%
\begin{center}
\begin{tabular}{l}
|\input{childdoc.def}|\\
|\childdocforward{|\textit{main}|}|\\
\end{tabular}
\end{center}
%
or alternatively with:
%
\begin{center}
\begin{tabular}{l}
|\input{childdoc.def}|\\
|\childdocby{|\textit{main}|}|\\
\end{tabular}
\end{center}
%
Both forms have slightly different effects as described above.
The main file is prepared as usual, see \secref{sec:include}.

%%%%%%%%%%%%%%%%%%%%%%%%%%%%%%%%%%%%%%%%%%%%%%%%%%%%%%%%%%%%%%%%%%%%%%%%%%%%%%%%
\subsection{Legacy Detection}
\label{sec:detection}

The directive |\childdocmain| in the main file can detect
whether the complete document or merely a child is to be compiled
even without using the directive |\childdocof|.
This method is deprecated because it is less robust
and there is no compelling reason to use it;
it is merely provided for backward compatibility
and it may be removed in future versions.

If the detection mechanism is to be used,
it is mandatory to correctly specify
the filename of the main file as the argument of |\childdocmain|:
%
\begin{center}
\begin{tabular}{l}
|\input{childdoc.def}|\\
|\childdocmain{|\textit{main}|}|\\
\end{tabular}
\end{center}
%
If |\jobname| does not match the argument \textit{main} of |\childdocmain|,
it is assumed that |\jobname| points to the child file to be compiled.
When using |\childdocmain| with the main file specified as argument,
it suffices to start a child file
with just |\input{|\textit{main}|}|
without loading of the package and using |\childdocof|.
If instead all processing is done
with the appropriate \textsf{childdoc} directives,
the argument of \textit{main} of |\childdocmain| can be empty.

An alternative version of the command line processing described
in \secref{sec:commandline} using the detection mechanism reads:
%
\begin{center}
|... -jobname "|\textit{target}|" "|[\textit{flags}]%
[|\def\jobname{|\textit{dest}|}|]|\input{|\textit{main}|}"|
\end{center}

%%%%%%%%%%%%%%%%%%%%%%%%%%%%%%%%%%%%%%%%%%%%%%%%%%%%%%%%%%%%%%%%%%%%%%%%%%%%%%%%
\subsection{Manual Code}
\label{sec:manual}

In case one cannot be certain whether the definitions file |childdoc.def|
is installed on the target \TeX{} distribution
and one prefers not to ship it,
it is conceivable to paste a few relevant commands into the sources.

To that end, drop all statements |\input{childdoc.def}|
and perform the replacements as outlined below.
Instead of |\childdocmain{|\textit{main}|}| add the following code
to the top of the main file:
%
\begin{center}
\begin{tabular}{l}
|\||ifdefined\childdocname\endinput\||fi\newif\ifchilddoc|\\
|\edef\childdocname{\scantokens\expandafter{\jobname\noexpand}}|\\
|\def\childdocmain{|\textit{main}|}\||ifx\childdocmain\childdocname\||else|\\
|\childdoctrue\includeonly{\childdocname}\let\jobname\childdocmain\||fi|\\
\end{tabular}
\end{center}
%
Instead of |\childdocof{|\textit{main}|}| just include the main file
at the top of each child file:
%
\begin{center}
|\input{|\textit{main}|}|
\end{center}
%
A simple redirection |\childdocforward{|\textit{dest}|}| is achieved by:
%
\begin{center}
|\def\jobname{|\textit{dest}|}\input{\jobname}|
\end{center}
%
The redirection with prefix
|\childdocforwardprefix[|\textit{prefix}|]{|\textit{dest}|}|
is accomplished by:
%
\begin{center}
\begin{tabular}{l}
|{\edef\jobname{\scantokens\expandafter{\jobname\noexpand}}|\\
|\def\redirectjob |\textit{prefix}|#1~~~{\gdef\jobname{|\textit{dest}|#1}}|\\
|\expandafter\redirectjob\jobname~~~}\input{\jobname}|
\end{tabular}
\end{center}

In an alternative approach,
child documents can be compiled by a specific command line
without additional code or specific definitions:
%
\begin{center}
|... -jobname "|\textit{target}|" "|[\textit{flags}]%
|\includeonly{|\textit{dest}|}\input{|\textit{main}|}"|
\end{center}
%

%%%%%%%%%%%%%%%%%%%%%%%%%%%%%%%%%%%%%%%%%%%%%%%%%%%%%%%%%%%%%%%%%%%%%%%%%%%%%%%%
%%%%%%%%%%%%%%%%%%%%%%%%%%%%%%%%%%%%%%%%%%%%%%%%%%%%%%%%%%%%%%%%%%%%%%%%%%%%%%%%
\section{Information}

%%%%%%%%%%%%%%%%%%%%%%%%%%%%%%%%%%%%%%%%%%%%%%%%%%%%%%%%%%%%%%%%%%%%%%%%%%%%%%%%
\subsection{Copyright}

Copyright \copyright{} 2017--2018 Niklas Beisert

This work may be distributed and/or modified under the
conditions of the \LaTeX{} Project Public License, either version 1.3
of this license or (at your option) any later version.
The latest version of this license is in
  \url{http://www.latex-project.org/lppl.txt}
and version 1.3 or later is part of all distributions of \LaTeX{}
version 2005/12/01 or later.

This work has the LPPL maintenance status `maintained'.

The Current Maintainer of this work is Niklas Beisert.

This work consists of the files |README.txt|, |childdoc.ins| and |childdoc.dtx|
as well as the derived files |childdoc.def|, |cdocsamp.tex|
with |cdocsch1.tex|, |cdocsch2.tex|, |cdocspt3.tex|, |cdocspt4.tex|,
|cdocsdrf.tex|, |cdocsfn1.tex|, |cdocsfn2.tex|
as well as |childdoc.pdf|.

%%%%%%%%%%%%%%%%%%%%%%%%%%%%%%%%%%%%%%%%%%%%%%%%%%%%%%%%%%%%%%%%%%%%%%%%%%%%%%%%
\subsection{Files and Installation}

The package consists of the files:
%
\begin{center}
\begin{tabular}{ll}
    |README.txt|   & readme file \\
    |childdoc.ins| & installation file \\
    |childdoc.dtx| & source file \\
    |childdoc.def| & definition file \\
    |cdocsamp.tex| & sample main file \\
    |cdocsch1.tex| & sample include file \\
    |cdocsch2.tex| & sample include file \\
    |cdocspt3.tex| & sample part file \\
    |cdocspt4.tex| & sample part file \\
    |cdocsdrf.tex| & sample redirection file \\
    |cdocsfn1.tex| & sample redirection file \\
    |cdocsfn2.tex| & sample redirection file \\
    |childdoc.pdf| & manual
\end{tabular}
\end{center}
%
The distribution consists of the files
|README.txt|, |childdoc.ins| and |childdoc.dtx|.
%
\begin{itemize}
\item
Run (pdf)\LaTeX{} on |childdoc.dtx|
to compile the manual |childdoc.pdf| (this file).
\item
Run \LaTeX{} on |childdoc.ins| to create the definitions file |childdoc.def|
and the sample |cdocsamp.tex| with include files
|cdocsch1.tex|, |cdocsch2.tex|, |cdocspt3.tex|, |cdocspt4.tex|,
|cdocsdrf.tex|, |cdocsfn1.tex|, |cdocsfn2.tex|.
Then copy the file |childdoc.def| to an appropriate directory of your \LaTeX{}
distribution, e.g.\ \textit{texmf-root}|/tex/latex/childdoc|.
\end{itemize}

%%%%%%%%%%%%%%%%%%%%%%%%%%%%%%%%%%%%%%%%%%%%%%%%%%%%%%%%%%%%%%%%%%%%%%%%%%%%%%%%
\subsection{Related CTAN Packages}

There are several other packages which offer a similar functionality:
%
\begin{itemize}
\item
The packages
\href{http://ctan.org/pkg/docmute}{\textsf{docmute}},
\href{http://ctan.org/pkg/includex}{\textsf{includex}} and
\href{http://ctan.org/pkg/standalone}{\textsf{standalone}}
provide commands to include only the document body of
a child file thus allowing both files to be compiled individually.
\item
The packages \href{http://ctan.org/pkg/subdocs}{\textsf{subdocs}}
and \href{http://ctan.org/pkg/subfiles}{\textsf{subfiles}}
provide structures in which the main and child documents can be
encapsulated and allowing them to be compiled individually.
The inclusion mechanism is different from the conventional |\include|.
\item
The package \href{http://ctan.org/pkg/combine}{\textsf{combine}}
is an elaborate solution to combine several documents into one.
\end{itemize}
%
See also the CTAN topic \href{http://ctan.org/topic/subdocs}{\textsf{subdocs}}
for further related packages.
The present package differs from the above solutions in that
a document structure constructed with the conventional |\include| mechanism
just needs two extra commands at the top of every file
such that all constituent files can be compiled individually.

%%%%%%%%%%%%%%%%%%%%%%%%%%%%%%%%%%%%%%%%%%%%%%%%%%%%%%%%%%%%%%%%%%%%%%%%%%%%%%%%
%\subsection{Feature Suggestions}
%
%The following is a list of features which may be useful for future
%versions of this package:
%%
%\begin{itemize}
%\item
%\ldots
%\end{itemize}

%%%%%%%%%%%%%%%%%%%%%%%%%%%%%%%%%%%%%%%%%%%%%%%%%%%%%%%%%%%%%%%%%%%%%%%%%%%%%%%%
\subsection{Revision History}

%%%%%%%%%%%%%%%%%%%%%%%%%%%%%%%%%%%%%%%%
\paragraph{v2.0:} 2018/12/30

\begin{itemize}
\item
immediate forward processing
\item
added |\childdocby| mechanism
\item
manual restructured
\end{itemize}

%%%%%%%%%%%%%%%%%%%%%%%%%%%%%%%%%%%%%%%%
\paragraph{v1.6:} 2018/01/17

\begin{itemize}
\item
application for development of include files
\item
corrections to manual
\end{itemize}

%%%%%%%%%%%%%%%%%%%%%%%%%%%%%%%%%%%%%%%%
\paragraph{v1.5:} 2017/05/21

\begin{itemize}
\item
more complete structuring introduced
\item
|\childdocof| introduced
\item
|\childdoc| renamed to |\childdocmain|
\item
|\childredirect| renamed to |\childdocforward| and |\childdocforwardprefix|
and functionality expanded
\end{itemize}

%%%%%%%%%%%%%%%%%%%%%%%%%%%%%%%%%%%%%%%%
\paragraph{v1.0:} 2017/04/27

\begin{itemize}
\item
manual and install package
\item
first version published on CTAN
\end{itemize}

%%%%%%%%%%%%%%%%%%%%%%%%%%%%%%%%%%%%%%%%
\paragraph{v0.6:} 2017/04/26

\begin{itemize}
\item
redirection mechanism added
\end{itemize}

%%%%%%%%%%%%%%%%%%%%%%%%%%%%%%%%%%%%%%%%
\paragraph{v0.5:} 2017/04/26

\begin{itemize}
\item
functionality in definition file
\end{itemize}


%%%%%%%%%%%%%%%%%%%%%%%%%%%%%%%%%%%%%%%%%%%%%%%%%%%%%%%%%%%%%%%%%%%%%%%%%%%%%%%%
%%%%%%%%%%%%%%%%%%%%%%%%%%%%%%%%%%%%%%%%%%%%%%%%%%%%%%%%%%%%%%%%%%%%%%%%%%%%%%%%
%%%%%%%%%%%%%%%%%%%%%%%%%%%%%%%%%%%%%%%%%%%%%%%%%%%%%%%%%%%%%%%%%%%%%%%%%%%%%%%%
\appendix

\settowidth\MacroIndent{\rmfamily\scriptsize 000\ }

 \DocInput{childdoc.dtx}

\end{document}
%</driver>
% \fi
%
% %%%%%%%%%%%%%%%%%%%%%%%%%%%%%%%%%%%%%%%%%%%%%%%%%%%%%%%%%%%%%%%%%%%%%%%%%%%%%%
% %%%%%%%%%%%%%%%%%%%%%%%%%%%%%%%%%%%%%%%%%%%%%%%%%%%%%%%%%%%%%%%%%%%%%%%%%%%%%%
% \section{Sample}
%\iffalse
%<*samplemain>
%\fi
%
% The following presents a sample document
% with two chapters, two parts, a title page,
% a compile flag as well as three forwarding files to set the flag.
% It consists of eight |.tex| files:
% \begin{center}
% \begin{tabular}{ll}
% |cdocsamp.tex|&main file\\
% |cdocsch1.tex|&include file for chapter 1\\
% |cdocsch2.tex|&include file for chapter 2\\
% |cdocspt3.tex|&include file for part 3\\
% |cdocspt4.tex|&include file for part 4\\
% |cdocsdrf.tex|&forwarding file for main file in draft mode\\
% |cdocsfi1.tex|&forwarding file for final version of chapter 1\\
% |cdocsfi2.tex|&forwarding file for final version of chapter 2\\
% \end{tabular}
% \end{center}
% Each of the eight files can be compiled directly by the \LaTeX{} compiler.
%
% %%%%%%%%%%%%%%%%%%%%%%%%%%%%%%%%%%%%%%
% \paragraph{Main File.}
%
% The main file is called |cdocsamp.tex|.
%
% Load the \textsf{childdoc} definitions and
% declare the filename for the main document:
%    \begin{macrocode}
\input{childdoc.def}
\childdocmain{}
%    \end{macrocode}

% Optional override for |\version| flag:
%    \begin{macrocode}
%%\ifchilddoc\else\providecommand{\version}{draft}\fi
%    \end{macrocode}

% Define the default values for the |\version| flag
% (|final| for the main file and |draft| for childs):
%    \begin{macrocode}
\ifchilddoc
\providecommand{\version}{draft}
\else
\providecommand{\version}{final}
\fi
%    \end{macrocode}

% Load the standard document class:
%    \begin{macrocode}
\documentclass[12pt]{article}
%    \end{macrocode}

% Start the document body:
%    \begin{macrocode}
\begin{document}
%    \end{macrocode}

% Declare a title page.
% Print title, part of document being processed and version flag:
%    \begin{macrocode}
\addtocounter{page}{-1}
\begin{center}
{\LARGE\bfseries{}childdoc example\par}
\vspace{1cm}
\ifchilddoc
\ifchilddocmanual part\else chapter\fi:
`\childdocname' of `\childdocjob'\par
\else
main document: `\childdocjob'\par
\fi
version: \version\par
\end{center}
\newpage
%    \end{macrocode}

% Manually include selected file,
% otherwise process as usual:
%    \begin{macrocode}
\ifchilddocmanual
\section*{part `\childdocname'}
\input{\childdocname}
\else
%    \end{macrocode}

% Include the two chapters:
%    \begin{macrocode}
\include{cdocsch1}
\include{cdocsch2}
%    \end{macrocode}

% Include the two parts unless only chapters should be displayed:
%    \begin{macrocode}
\ifchilddoc\else
\section{part three}
\input{cdocspt3}
\section{part four}
\input{cdocspt4}
\fi
%    \end{macrocode}

% Process as usual until here:
%    \begin{macrocode}
\fi
%    \end{macrocode}

% End of document body:
%    \begin{macrocode}
\end{document}
%    \end{macrocode}
%\iffalse
%</samplemain>
%\fi
%
% %%%%%%%%%%%%%%%%%%%%%%%%%%%%%%%%%%%%%%
% \paragraph{Chapter Include Files.}
%
% The include files are called |cdocsch1.tex| and |cdocsch2.tex|.
%
%\iffalse
%<*samplechap1|samplechap2>
%\fi

% Optional override for |\version| flag:
%    \begin{macrocode}
%%\providecommand{\version}{final}
%    \end{macrocode}

% Include the main document:
%    \begin{macrocode}
\input{childdoc.def}
\childdocof{cdocsamp}
%    \end{macrocode}

%\iffalse
%</samplechap1|samplechap2>
%\fi
%
%\iffalse
%<*samplechap1>
%\fi
% Some text for chapter 1:
%    \begin{macrocode}
\section{one}
some text in chapter one
%    \end{macrocode}

%\iffalse
%</samplechap1>
%\fi
% Some text for chapter 2:
%\iffalse
%<*samplechap2>
%\fi
%    \begin{macrocode}
\section{two}
more text in chapter two
%    \end{macrocode}

%\iffalse
%</samplechap2>
%\fi
%
% %%%%%%%%%%%%%%%%%%%%%%%%%%%%%%%%%%%%%%
% \paragraph{Part Include Files.}
%
% The include files are called |cdocspt3.tex| and |cdocspt4.tex|.
%
%\iffalse
%<*samplepart3|samplepart4>
%\fi

% Optional override for |\version| flag:
%    \begin{macrocode}
%%\providecommand{\version}{final}
%    \end{macrocode}

% Include the main document:
%    \begin{macrocode}
\input{childdoc.def}
\childdocby{cdocsamp}
%    \end{macrocode}

%\iffalse
%</samplepart3|samplepart4>
%\fi
%
%\iffalse
%<*samplepart3>
%\fi
% Some text for part 3:
%    \begin{macrocode}
some text in part three
%    \end{macrocode}

%\iffalse
%</samplepart3>
%\fi
% Some text for part 4:
%\iffalse
%<*samplepart4>
%\fi
%    \begin{macrocode}
more text in part four
%    \end{macrocode}

%\iffalse
%</samplepart4>
%\fi
%
% %%%%%%%%%%%%%%%%%%%%%%%%%%%%%%%%%%%%%%
% \paragraph{Forwarding for a Complete Draft.}
%
% The following forwarding file |cdocsdrf.tex|
% compiles the main document in draft mode:
%\iffalse
%<*sampledraft>
%\fi
%    \begin{macrocode}
\def\version{draft}
\input{childdoc.def}
\childdocforward{cdocsamp}
%    \end{macrocode}

%\iffalse
%</sampledraft>
%\fi
%
% %%%%%%%%%%%%%%%%%%%%%%%%%%%%%%%%%%%%%%
% \paragraph{Forwarding for Final Version of the Chapters.}
%
% The following forwarding files |cdocsfn1.tex| and |cdocsfn2.tex|
% (with identical content)
% compile the final versions of the child documents
% |cdocsch1.tex| and |cdocsch2.tex|, respectively:
%\iffalse
%<*samplefinal>
%\fi
%    \begin{macrocode}
\def\version{final}
\input{childdoc.def}
\childdocforwardprefix[cdocsamp]{cdocsfn}{cdocsch}
%    \end{macrocode}

%\iffalse
%</samplefinal>
%\fi
%
% %%%%%%%%%%%%%%%%%%%%%%%%%%%%%%%%%%%%%%
% \paragraph{Command Line Processing.}
%
% The following three command lines generate the output files
% |cdocscld|, |cdocscl1| and |cdocscl2|
% which should be identical to
% |cdocsdrf|, |cdocsch1| and |cdocsfn2|, respectively:
% \begin{center}
% \begin{tabular}{l}
% |latex -jobname cdocscld \|\\
% |  "\def\version{draft}\input{childdoc.def}\childdocforward{cdocsamp}"|\\
% |latex -jobname cdocscl1 \|\\
% |  "\input{childdoc.def}\childdocforward[cdocsamp]{cdocsch1}"|\\
% |latex -jobname cdocscl2 \|\\
% |  "\def\version{final}\input{childdoc.def}\childdocforward{cdocsch2}"|
% \end{tabular}
% \end{center}
% Note that the trailing backslash on each first line
% merely continues the input to the second line
% (for convenient cut ant paste).
% Furthermore, the command |latex| can be replaced by any
% of its alternative versions such as |pdflatex|.
%
% %%%%%%%%%%%%%%%%%%%%%%%%%%%%%%%%%%%%%%%%%%%%%%%%%%%%%%%%%%%%%%%%%%%%%%%%%%%%%%
% %%%%%%%%%%%%%%%%%%%%%%%%%%%%%%%%%%%%%%%%%%%%%%%%%%%%%%%%%%%%%%%%%%%%%%%%%%%%%%
% \section{Implementation}
%\iffalse
%<*package>
%\fi
%
% This section describes the definitions file |childdoc.def|.

% The definitions cannot be loaded using |\usepackage| or |\RequirePackage|
% which has a mechanism to prevent loading a style file more than once.
% When loading the definitions by means of |\input|
% multiple instances have to be prevented manually:
%\iffalse
%This code needs to be before the `\ProvidesFile' directive
%which is defined at the beginning of this file.
%Therefore it is also placed there and commented out here.
%</package>
%<*discard>
%\fi
%    \begin{macrocode}
\ifdefined\childdocmain\endinput\fi
%    \end{macrocode}
%\iffalse
%</discard>
%<*package>
%\fi
%
% \macro{\ifchilddoc}
% \macro{\ifchilddocmanual}
% The conditional |\ifchilddoc| tells whether a
% child (true) or main (false) document is being compiled.
% The conditional |\ifchilddocmanual| tells whether
% the |\includeonly| mechanism is used (false) or
% the selection of child files must be performed manually (true).
% The definitions initialise to false:
%    \begin{macrocode}
\newif\ifchilddoc
\newif\ifchilddocmanual
%    \end{macrocode}

% \macro{\childdocname}
% \macro{\childdocjob}
% The macro |\childdocname| stores the name of the main document
% to be compiled. The macro |\childdocjob| stores the name of
% the document on which the \LaTeX{} compiler was originally invoked.
% The content of |\jobname| cannot be compared
% to filenames specified in the source due to different catcodes.
% The following code rescans |\jobname|, stores the result
% in |\childdocname| and saves a copy in |\childdocjob|:
%    \begin{macrocode}
\edef\childdocname{\scantokens\expandafter{\jobname\noexpand}}
\let\childdocjob\childdocname
%    \end{macrocode}

% \macro{\childdocdisable}
% The macro |\childdocdisable| prevents the main file
% from being processed more than once.
% At this stage, the main document command |\childdocmain|
% is assumed to be called once again where it should do nothing.
% Any subsequent call to it should prevent
% a secondary processing of the main document
% It overwrites the forwarding commands
% |\childdocof| and |\childdocforward|
% with empty macros to prevent further inclusions of the main document:
%    \begin{macrocode}
\newcommand{\childdocdisable}
{
  \renewcommand{\childdocmain}[1]{\renewcommand{\childdocmain}[1]{\endinput}}
  \renewcommand{\childdocof}[1]{}
  \renewcommand{\childdocby}[2][]{}
  \renewcommand{\childdocforward}[2][]{}
  \renewcommand{\childdocdisable}{}
}
%    \end{macrocode}

% \macro{\childdocmain}
% The macro |\childdocmain| is to be called at the top of the main file
% with nothing or the main filename (without extension) as argument.
% First, it breaks loops.
% If the argument is not empty and does not match |\childdocname|
% (which is set by the first inclusion of |childdoc.def|),
% |\ifchilddoc| is set to true, |\includeonly| is applied to the child file
% and |\jobname| is set to the main file
% (for proper handling of |.aux| files):
%    \begin{macrocode}
\newcommand{\childdocmain}[1]
{
  \childdocdisable\childdocmain{}
  \if?#1?\else
    \begingroup
      \def\childdoctmp{#1}
      \ifx\childdoctmp\childdocname
        \def\childdoctmp{}
      \else
        \def\childdoctmp
        {
          \childdoctrue
          \includeonly{\childdocname}
          \def\childdocjob{#1}
          \def\jobname{#1}
        }
      \fi
      \expandafter
    \endgroup
    \childdoctmp
  \fi
}
%    \end{macrocode}

% \macro{\childdocof}
% The command |\childdocof| redirects
% compilation to the main file |#1|.
%    \begin{macrocode}
\newcommand{\childdocof}[1]
{
  \childdocdisable
  \childdoctrue
  \includeonly{\childdocname}
  \def\jobname{#1}
  \def\childdocjob{#1}
  \input{#1}
}
%    \end{macrocode}

% \macro{\childdocby}
% The command |\childdocby| ....
%    \begin{macrocode}
\newcommand{\childdocby}[2][]
{
  \childdocdisable
  \childdoctrue
  \childdocmanualtrue
  \if?#1?\else
    \def\jobname{#2}
  \fi
  \def\childdocjob{#2}
  \input{#2}
  \endinput
}
%    \end{macrocode}

% \macro{\childdocforward}
% The command |\childdocforward| redirects
% compilation to the main file or
% (if the optional argument is given) a child file.
% Parameters are set as if the main file
% or a child file starting with |\childdocof| was compiled.
% Then compilation is handed over to the main file:
%    \begin{macrocode}
\newcommand{\childdocforward}[2][]
{
  \begingroup
    \if?#1?
      \def\childdoctmp
      {
        \def\childdocname{#2}
        \def\childdocjob{#2}
        \def\jobname{#2}
        \input{#2}
        \endinput
      }
    \else
      \def\childdoctmp
      {
        \childdocdisable
        \def\childdocname{#2}
        \childdoctrue
        \includeonly{#2}
        \def\childdocjob{#1}
        \def\jobname{#1}
        \input{#1}
        \endinput
      }
    \fi
    \expandafter
  \endgroup
  \childdoctmp
}
%    \end{macrocode}

% \macro{\childdocforwardprefix}
% The command |\childdocforwardprefix| redirects
% compilation to the main or a child file by means of a pattern.
% The prefix |#1| in the current filename is replaced by |#2|
% and the suffix of the current filename is kept
% (it is assumed that the filename does not contain the substring `|~~~|'
% which is used as a delimiter).
% Compilation is handed over to the new file by |\childdocforward|:
%    \begin{macrocode}
\newcommand{\childdocforwardprefix}[3][]
{
  \begingroup
    \def\childdocextract #2##1~~~{\def\childdoctmp{\childdocforward[#1]{#3##1}}}
    \expandafter\childdocextract\childdocname~~~
    \expandafter
  \endgroup
  \childdoctmp
}
%    \end{macrocode}

% \macro{\childdoc}
% The deprecated macro |\childdoc| is a legacy version of |\childdocmain|:
%    \begin{macrocode}
\newcommand{\childdoc}{\childdocmain}
%    \end{macrocode}

% \macro{\childdocredirect}
% The deprecated macro |\childdocredirect| is a legacy version
% of |\childdocforward| and |\childdocforwardprefix|:
%    \begin{macrocode}
\newcommand{\childdocredirect}[2][]
{
  \begingroup
    \if?#1?
      \def\childdoctmp{\childdocforward{#2}}
    \else
      \def\childdoctmp{\childdocforwardprefix{#1}{#2}}
    \fi
    \expandafter
  \endgroup
  \childdoctmp
}
%    \end{macrocode}

%\iffalse
%</package>
%\fi
%
\endinput
|\\
|\childdocby{|\textit{main}|}|\\
\end{tabular}
\end{center}
%
Both forms have slightly different effects as described above.
The main file is prepared as usual, see \secref{sec:include}.

%%%%%%%%%%%%%%%%%%%%%%%%%%%%%%%%%%%%%%%%%%%%%%%%%%%%%%%%%%%%%%%%%%%%%%%%%%%%%%%%
\subsection{Legacy Detection}
\label{sec:detection}

The directive |\childdocmain| in the main file can detect
whether the complete document or merely a child is to be compiled
even without using the directive |\childdocof|.
This method is deprecated because it is less robust
and there is no compelling reason to use it;
it is merely provided for backward compatibility
and it may be removed in future versions.

If the detection mechanism is to be used,
it is mandatory to correctly specify
the filename of the main file as the argument of |\childdocmain|:
%
\begin{center}
\begin{tabular}{l}
|% \iffalse
%
% childdoc.dtx Copyright (C) 2017-2018 Niklas Beisert
%
% This work may be distributed and/or modified under the
% conditions of the LaTeX Project Public License, either version 1.3
% of this license or (at your option) any later version.
% The latest version of this license is in
%   http://www.latex-project.org/lppl.txt
% and version 1.3 or later is part of all distributions of LaTeX
% version 2005/12/01 or later.
%
% This work has the LPPL maintenance status `maintained'.
%
% The Current Maintainer of this work is Niklas Beisert.
%
% This work consists of the files childdoc.dtx and childdoc.ins
% and the derived files childdoc.def and cdocsamp.tex with
% cdocsch1.tex, cdocsch2.tex, cdocsdrf.tex, cdocsfn1.tex, cdocsfn2.tex.
%
%<package>\ifdefined\childdocmain\endinput\fi
%<package>\ProvidesFile{childdoc.def}[2018/12/30 v2.0 child document driver]
%<samplemain>\ProvidesFile{cdocsamp.tex}[2018/12/30 v2.0 sample for childdoc]
%<*driver>
%\ProvidesFile{childdoc.drv}[2018/12/30 v2.0 childdoc reference manual file]
\PassOptionsToClass{10pt,a4paper}{article}
\documentclass{ltxdoc}

\usepackage[margin=35mm]{geometry}
\usepackage{hyperref}
\usepackage{hyperxmp}
\usepackage[usenames]{color}

\hypersetup{colorlinks=true}
\hypersetup{pdfstartview=FitH}
\hypersetup{pdfpagemode=UseNone}
\hypersetup{pdfsource={}}
\hypersetup{pdflang={en-UK}}
\hypersetup{pdfcopyright={Copyright 2017-2018 Niklas Beisert.
  This work may be distributed and/or modified under the
  conditions of the LaTeX Project Public License, either version 1.3
  of this license or (at your option) any later version.}}
\hypersetup{pdflicenseurl={http://www.latex-project.org/lppl.txt}}
\hypersetup{pdfcontactaddress={ETH Zurich, ITP, HIT K,
  Wolfgang-Pauli-Strasse 27}}
\hypersetup{pdfcontactpostcode={8093}}
\hypersetup{pdfcontactcity={Zurich}}
\hypersetup{pdfcontactcountry={Switzerland}}
\hypersetup{pdfcontactemail={nbeisert@itp.phys.ethz.ch}}
\hypersetup{pdfcontacturl={http://people.phys.ethz.ch/\xmptilde nbeisert/}}

\newcommand{\secref}[1]{\hyperref[#1]{section \ref*{#1}}}

\parskip1ex
\parindent0pt
\let\olditemize\itemize
\def\itemize{\olditemize\parskip0pt}

\begin{document}

\title{The \textsf{childdoc} Package}
\hypersetup{pdftitle={The childdoc Package}}
\author{Niklas Beisert\\[2ex]
  Institut f\"ur Theoretische Physik\\
  Eidgen\"ossische Technische Hochschule Z\"urich\\
  Wolfgang-Pauli-Strasse 27, 8093 Z\"urich, Switzerland\\[1ex]
  \href{mailto:nbeisert@itp.phys.ethz.ch}
  {\texttt{nbeisert@itp.phys.ethz.ch}}}
\hypersetup{pdfauthor={Niklas Beisert}}
\hypersetup{pdfsubject={Manual for the LaTeX2e Package childdoc}}
\date{30 December 2018, \textsf{v2.0}}
\maketitle

\begin{abstract}\noindent
\textsf{childdoc} is a \LaTeXe{} package
that enables the direct compilation
of document sections included by |\include|
to individual files.
\end{abstract}

\begingroup
\parskip0ex
\tableofcontents
\endgroup

%%%%%%%%%%%%%%%%%%%%%%%%%%%%%%%%%%%%%%%%%%%%%%%%%%%%%%%%%%%%%%%%%%%%%%%%%%%%%%%%
%%%%%%%%%%%%%%%%%%%%%%%%%%%%%%%%%%%%%%%%%%%%%%%%%%%%%%%%%%%%%%%%%%%%%%%%%%%%%%%%
\section{Introduction}

\LaTeX{} provides a mechanism to structure a large document (such as a book)
into a main file and several child files (containing the chapters)
using the |\include| command.
This mechanism is beneficial for documents
which span hundreds of pages in order to
make the source file(s) more manageable.
Moreover, compilation can be restricted to
selected child files by means of the |\includeonly| command.
The latter feature can be used to reduce the compilation time while editing
(this was significantly more useful in the earlier days of \LaTeX{})
or to generate a smaller document which is easier to navigate.
Another application of |\includeonly| is to generate
documents consisting of selected parts of the complete document.

However, there are a few drawbacks of the plain |\include| mechanism:
\begin{itemize}
\item
The child files cannot be compiled on their own,
they can only be compiled via the main file.
A naive editing environment
(such as a text editor with an option
to have the current file processed by \LaTeX)
may require one to switch to the main file before compiling;
attempting to compile the child file produces errors.
\item
The main file must be modified (each time)
to adjust the |\includeonly| command
to the present needs. This easily leaves the main file in a messy state.
\item
The generated document will always carry the filename
of the main document. This is inconvenient if
several child files are to be compiled and
to be kept for distribution.
\end{itemize}

The present package provides a simple interface
to make child files individually compilable by \LaTeX{}.
Compiling a child file then has the same effect as compiling
the main file with an |\includeonly| command
to select the appropriate child.
Moreover the generated document will carry the name of the child
rather than the main file.
This resolves all three above issues.

This feature is meant to make the editing of books,
thesis documents and lecture notes somewhat more convenient.
However, the package can also be used efficiently for
composing a series of documents (such as exercise sheets)
which are typically distributed individually.
It then assists the author in generating the individual documents
(potentially in different versions)
as well as a document containing the collected series.
Another application is in developing style files
or other kinds of included material
where compilation of the style file could redirect
to a sample or test file.

%%%%%%%%%%%%%%%%%%%%%%%%%%%%%%%%%%%%%%%%%%%%%%%%%%%%%%%%%%%%%%%%%%%%%%%%%%%%%%%%
%%%%%%%%%%%%%%%%%%%%%%%%%%%%%%%%%%%%%%%%%%%%%%%%%%%%%%%%%%%%%%%%%%%%%%%%%%%%%%%%
\section{Usage}

First of all, the package \textsf{childdoc} is \emph{not} a standard
\LaTeXe{} |.sty| style file! Therefore it needs to be invoked in
a non-standard way.

%%%%%%%%%%%%%%%%%%%%%%%%%%%%%%%%%%%%%%%%%%%%%%%%%%%%%%%%%%%%%%%%%%%%%%%%%%%%%%%%
\subsection{Included Files}
\label{sec:include}

%%%%%%%%%%%%%%%%%%%%%%%%%%%%%%%%%%%%%%%%
\DescribeMacro{\childdocmain}
To use the package, add the commands
\begin{center}
\begin{tabular}{l}
|\input{childdoc.def}|\\
|\childdocmain{}|\\
\end{tabular}
\end{center}
at the very top of the main \LaTeX{} file,
in particular \emph{before} the |\documentclass| statement!
The argument of |\childdocmain| should be left empty
(but it must be present).

%%%%%%%%%%%%%%%%%%%%%%%%%%%%%%%%%%%%%%%%
\DescribeMacro{\childdocof}
Furthermore, add the commands
\begin{center}
\begin{tabular}{l}
|\input{childdoc.def}|\\
|\childdocof{|\textit{main}|}|\\
\end{tabular}
\end{center}
at the top of every child file \textit{child}
which is included by |\include{|\textit{child}|}|
from within the main file
(or at least for those files to be compiled individually).
The argument \textit{main} must be the filename of the main file.

There are a couple of
considerations in setting up the main and child documents:

%%%%%%%%%%%%%%%%%%%%%%%%%%%%%%%%%%%%%%%%
\paragraph{Restrictions.}

Please note the following restrictions:
\begin{itemize}
\item
|\childdocmain| must be called with one argument \textit{main}
to ensure compatibility with earlier version of the package.
It must either be empty (|\childdocmain{}|)
or precisely match the filename of the main file in which it is specified.
See \secref{sec:detection} for further information.
\item
The filename \textit{main} must be specified without the |.tex| extension.
\item
The filename \textit{main} is case sensitive
(even in case-insensitive file systems)
due to internal string comparison.
\item
The argument \textit{main} should be fully expanded, it cannot be a macro.
\item
Subdirectories and special characters should be avoided in filenames.
\item
The command |\childdocmain{|\textit{main}|}| must be followed by a whitespace.
It should not be followed immediately by another command
or by a comment mark `|%|'.
This is because the \TeX{} parser reads the token immediately following
the argument of |\childdocmain| and puts it
at the beginning of every child section;
however, a white\-space is ignored.
\end{itemize}

%%%%%%%%%%%%%%%%%%%%%%%%%%%%%%%%%%%%%%%%
\paragraph{Content of Main File.}

It is advisable to place all content in the child files included by |\include|.
Any output contained in the main file will appear in all child documents
unless suppressed manually;
it cannot be suppressed automatically by the |\includeonly| directive
and thus should normally be avoided.
A method to include some content in the main file
by means of conditional processing is described in \secref{sec:conditional}.

%%%%%%%%%%%%%%%%%%%%%%%%%%%%%%%%%%%%%%%%
\paragraph{Page Numbering.}

When only a part of the document is compiled,
the appropriate numbering of pages
(as well as other status parameters)
is determined from the |.aux| files.
The latter contain information from previous passes.
However this information needs to propagate through
all intermediate child documents.
Therefore the page numbering in child documents may well
be inconsistent until the complete document is compiled at least once.

A useful (if unconventional) way to always ensure a consistent
page numbering is to restart the numbering in each child document
and denote the pages by `\textit{child}|.|\textit{page}'
where \textit{child} represents the chapter/section number of the child file.
This can be achieved by the command
|\numberwithin{page}{|\textit{child}|}|
of the \textsf{amsmath} package
where \textit{child} can be |chapter| or |section|
depending on the chosen structuring.
Alternatively, one can modify the macro |\thepage| appropriately
and reset the counter |page| at the start of each child file.

%%%%%%%%%%%%%%%%%%%%%%%%%%%%%%%%%%%%%%%%%%%%%%%%%%%%%%%%%%%%%%%%%%%%%%%%%%%%%%%%
\subsection{Conditional Processing}
\label{sec:conditional}

The package provides a mechanism to compile different versions
of a document. To customise the versions further some conditional processing
can come in handy to distinguish which version is being compiled.
The package provides two macros to describe the compilation context:

%%%%%%%%%%%%%%%%%%%%%%%%%%%%%%%%%%%%%%%%
\DescribeMacro{\ifchilddoc}
The conditional |\ifchilddoc| distinguishes between the compilation of
child documents and the main document:
%
\begin{center}
|\ifchilddoc |\textit{child-code}| |[|\||else |\textit{main-code}]| \||fi|
\end{center}

%%%%%%%%%%%%%%%%%%%%%%%%%%%%%%%%%%%%%%%%
\DescribeMacro{\childdocname}
\DescribeMacro{\childdocjob}
The macro |\childdocname| contains the filename (without extension)
of the main or child file being processed.
Note that |\childdocjob| will always contain the name of the main file.

%%%%%%%%%%%%%%%%%%%%%%%%%%%%%%%%%%%%%%%%
\paragraph{Title Page.}

Conditional processing can be used to include a title or banner page
in the main document when proper precautions are taken.
Importantly, the code in the main file should ensure that the page counter
(as well as other status parameters which are stored in the |.aux| files)
takes the same value after the conditional processing.
Otherwise the page numbers may take divergent values
depending on which part is compiled.

For example, a title page could be declared by:
%
\begin{center}
\begin{tabular}{l}
|\ifchilddoc\||else|\\
|\addtocounter{page}{-1}|\\
\textit{code for title page}\\
|\newpage|\\
|\||fi|
\end{tabular}
\end{center}
%
A banner page for the child documents can be generated by:
%
\begin{center}
\begin{tabular}{l}
|\ifchilddoc|\\
|\addtocounter{page}{-1}|\\
\textit{code for banner page}\\
|\newpage|\\
|\||fi|
\end{tabular}
\end{center}
%
Here one could write a message such as:
\begin{center}
|This is the part \childdocname{} of \childdocjob{}.|
\end{center}

%%%%%%%%%%%%%%%%%%%%%%%%%%%%%%%%%%%%%%%%%%%%%%%%%%%%%%%%%%%%%%%%%%%%%%%%%%%%%%%%
\subsection{Flags}
\label{sec:flags}

The package makes it easy to generate different versions
of the main or child documents.
To this end compilation flags can be defined
and assigned different default values.
They will be particularly useful in conjunction
with the forwarding mechanism described in \secref{sec:forward}.

For example, it may be useful to have a flag |\version|
which can be set to |draft| or |final|.
The document source will contain some conditional code
depending on the value of |\version|.
Suppose further, the flag should default to |final| for the main file
and to |draft| for child files
which is a natural assignment for editing the document.
This is achieved by placing the following code
in the preamble of the main document
(below the |\childdocmain| directive):
%
\begin{center}
\begin{tabular}{l}
|\ifchilddoc|\\
|\providecommand{\version}{draft}|\\
|\||else|\\
|\providecommand{\version}{final}|\\
|\||fi|
\end{tabular}
\end{center}
%
The definition by |\providecommand| makes sure
that previous definitions are not overwritten.
Further statements |\providecommand{\version}{...}|
can thus be added before the above code to override it.

For the main file, one might add a line
(between |\childdocmain| and the above block)
%
\begin{center}
|%\ifchilddoc\||else\providecommand{\version}{draft}\||fi|
\end{center}
%
which can be uncommented to produce a draft version.
Likewise one can add a line to the very top of a child file
(above the |\childdocof{|\textit{main}|}| directive)
%
\begin{center}
|%\providecommand{\version}{final}|
\end{center}
%
which can be uncommented to produce the final version of this child document.

%%%%%%%%%%%%%%%%%%%%%%%%%%%%%%%%%%%%%%%%%%%%%%%%%%%%%%%%%%%%%%%%%%%%%%%%%%%%%%%%
\subsection{Forwarding}
\label{sec:forward}

Different versions of the main or child documents
using compilation flags as described in \secref{sec:flags}
can be (permanently) stored in different files
for convenient compilation, viewing and distribution.
To this end, the package defines a command
to pass on compilation to a different file:

%%%%%%%%%%%%%%%%%%%%%%%%%%%%%%%%%%%%%%%%
\DescribeMacro{\childdocforward}
The command |\childdocforward| redirects processing to
another source file:
%
\begin{center}
\begin{tabular}{l}
|\input{childdoc.def}|\\
|\childdocforward[|\textit{main}|]{|\textit{dest}|}|\\
\end{tabular}
\end{center}
%
The argument \textit{dest} is the destination file
(without extension).
It should be the main file or one of the child files.
Note that further \textsf{childdoc} directives
such as |\childdocof| and |\childdocforward|
in the indicated file will be processed in this form.
The optional argument \textit{main}
passes on directly to the main file \textit{main}
while pretending to compile the child \textit{dest}.
This form behaves as if \textit{dest}
issues |\childdocof{|\textit{main}|}| right away,
and no further \textsf{childdoc} directives will be processed.

%%%%%%%%%%%%%%%%%%%%%%%%%%%%%%%%%%%%%%%%
\DescribeMacro{\...prefix}
In the alternative form |\childdocforwardprefix|,
%
\begin{center}
\begin{tabular}{l}
|\input{childdoc.def}|\\
|\childdocforwardprefix[|\textit{main}|]{|\textit{prefix}|}{|\textit{dest}|}|
\end{tabular}
\end{center}
%
the destination file is determined by a pattern
depending on the current file:
To make this work, the current file must be called
`{\textit{prefix}\hspace{0.2em}\textit{suffix}}'
with \textit{prefix} matching precisely the argument.
Processing is then passed on to the file
`{\textit{dest}\hspace{0.2em}\textit{suffix}}'.
Surely, the same effect is achieved by
directly specifying the
argument `{\textit{dest}\hspace{0.2em}\textit{suffix}}'
in the first form.
However, that requires to set up a different file
for each child. With the alternative form of the command
all these files can have exactly the same content
which simplifies setting them up and maintaining them.

For example, the following file |draft.tex|
with a compilation flag |\version| as described in \secref{sec:flags}
compiles the main document as a draft:
%
\begin{center}
\begin{tabular}{l}
|\def\version{draft}|\\
|\input{childdoc.def}|\\
|\childdocforward{|\textit{main}|}|
\end{tabular}
\end{center}
%
Likewise, the following files |final|\textit{nn}|.tex|
compile the final version of the child document
|child|\textit{nn}|.tex|:
%
\begin{center}
\begin{tabular}{l}
|\def\version{final}|\\
|\input{childdoc.def}|\\
|\childdocforwardprefix{final}{child}|
\end{tabular}
\end{center}
%

Note that when several versions of a main file and/or of each child file
are to be generated, it may be convenient to set up a |Makefile| or
shell script to automatise the process.

%%%%%%%%%%%%%%%%%%%%%%%%%%%%%%%%%%%%%%%%%%%%%%%%%%%%%%%%%%%%%%%%%%%%%%%%%%%%%%%%
\subsection{Command Line Processing}
\label{sec:commandline}

The effect of redirection files can also be achieved by invoking
the \LaTeX{} compiler with a more elaborate command line.
Most conveniently this should be done as part
of a shell script or a |Makefile|.

When using \textsf{childdoc} in the main file, the following
command lines effectively perform a redirection
(note that depending on the shell being used,
backslashes may have to be doubled: `|\|' $\to$ `|\\|'):
%
\begin{center}
|... -jobname "|\textit{target}|" |\\|"|[\textit{flags}]%
|\input{childdoc.def}\childdocforward[|\textit{main}|]{|\textit{dest}|}"|
\end{center}
%
Here \textit{target} is the name of the output file,
\textit{main} is the name of the main file
and \textit{dest} is the name of the main or child file to be processed
(all filenames without extensions).
The optional argument \textit{main} can be omitted
if \textit{main} matches \textit{dest}.
Optionally, compilation \textit{flags} can be defined via |\def| commands.
This command line makes the \TeX{} engine believe
it is compiling the file \textit{target}
whose content is specified as the latter parameter.
The provided code then forwards the processing to
\textit{main} or \textit{dest} as described in \secref{sec:forward}.

%%%%%%%%%%%%%%%%%%%%%%%%%%%%%%%%%%%%%%%%%%%%%%%%%%%%%%%%%%%%%%%%%%%%%%%%%%%%%%%%
\subsection{Include by Input}
\label{sec:input}

Including child documents by |\include| has some restrictions by design.
Most notably, the content of a child document always occupies
its own set of pages; pages cannot be shared between child documents.
Usually, this behaviour makes perfect sense
because each child document contain an essential part of the document.
However, in some situations it may be desirable to compose
a document from a collection of parts
without having mandatory page breaks between then.
For this case, the package
provides a mechanism to include parts
by |\input| which can also be processed individually.
However, by construction this mechanism
requires manual handling of the content to be output.

%%%%%%%%%%%%%%%%%%%%%%%%%%%%%%%%%%%%%%%%
\DescribeMacro{\ifchilddocmanual}
The main file should be prepared as usual, see \secref{sec:include}.
However, the document body must make a distinction
between processing of an individual part and of the main document, e.g.:
%
\begin{center}
\begin{tabular}{l}
|\ifchilddocmanual|\\
|\input{\childdocname}|\\
|\||else|\\
\textit{document body with }|\input{|\textit{part}|}|\\
|\||fi|
\end{tabular}
\end{center}
%
The conditional |\ifchilddocmanual| is true whenever
a part to be included by |\input| is being compiled,
and the name of the part is stored in |\childdocname|.

%%%%%%%%%%%%%%%%%%%%%%%%%%%%%%%%%%%%%%%%
\DescribeMacro{\childdocby}
Each part to be included by |\input| should start with:
%
\begin{center}
\begin{tabular}{l}
|\input{childdoc.def}|\\
|\childdocby{|\textit{main}|}|\\
\end{tabular}
\end{center}
%
The directive |\childdocby| is similar to |\childdocof|
described in \secref{sec:include},
but the subsequent selection of content must be done manually.
To that end, both |\ifchilddoc| and |\ifchilddocmanual|
will be true upon processing of a part,
and the name of the part is stored in |\childdocname|.
Note that |\jobname| will be set to the filename of the current part
so that each part receives an individual |.aux| file
that does not interfere with the |.aux| file(s) of the main document.
This behaviour can be altered by the alternative form
|\childdocby[*]{|\textit{main}|}| (with a non-empty optional argument)
which uses the |.aux| file of the main document
by setting |\jobname| to \textit{main}.

%%%%%%%%%%%%%%%%%%%%%%%%%%%%%%%%%%%%%%%%%%%%%%%%%%%%%%%%%%%%%%%%%%%%%%%%%%%%%%%%
\subsection{Driver Development}
\label{sec:driver}

The \textsf{childdoc} mechanism can also be use for the development
of definition files such as \LaTeX{} styles or classes.
This case differs from the above setup with multiple parts
included by |\include| in that no |\includeonly| should be invoked.
This can be achieved by starting the include file
(before |\ProvidesPackage|) with:
%
\begin{center}
\begin{tabular}{l}
|\input{childdoc.def}|\\
|\childdocforward{|\textit{main}|}|\\
\end{tabular}
\end{center}
%
or alternatively with:
%
\begin{center}
\begin{tabular}{l}
|\input{childdoc.def}|\\
|\childdocby{|\textit{main}|}|\\
\end{tabular}
\end{center}
%
Both forms have slightly different effects as described above.
The main file is prepared as usual, see \secref{sec:include}.

%%%%%%%%%%%%%%%%%%%%%%%%%%%%%%%%%%%%%%%%%%%%%%%%%%%%%%%%%%%%%%%%%%%%%%%%%%%%%%%%
\subsection{Legacy Detection}
\label{sec:detection}

The directive |\childdocmain| in the main file can detect
whether the complete document or merely a child is to be compiled
even without using the directive |\childdocof|.
This method is deprecated because it is less robust
and there is no compelling reason to use it;
it is merely provided for backward compatibility
and it may be removed in future versions.

If the detection mechanism is to be used,
it is mandatory to correctly specify
the filename of the main file as the argument of |\childdocmain|:
%
\begin{center}
\begin{tabular}{l}
|\input{childdoc.def}|\\
|\childdocmain{|\textit{main}|}|\\
\end{tabular}
\end{center}
%
If |\jobname| does not match the argument \textit{main} of |\childdocmain|,
it is assumed that |\jobname| points to the child file to be compiled.
When using |\childdocmain| with the main file specified as argument,
it suffices to start a child file
with just |\input{|\textit{main}|}|
without loading of the package and using |\childdocof|.
If instead all processing is done
with the appropriate \textsf{childdoc} directives,
the argument of \textit{main} of |\childdocmain| can be empty.

An alternative version of the command line processing described
in \secref{sec:commandline} using the detection mechanism reads:
%
\begin{center}
|... -jobname "|\textit{target}|" "|[\textit{flags}]%
[|\def\jobname{|\textit{dest}|}|]|\input{|\textit{main}|}"|
\end{center}

%%%%%%%%%%%%%%%%%%%%%%%%%%%%%%%%%%%%%%%%%%%%%%%%%%%%%%%%%%%%%%%%%%%%%%%%%%%%%%%%
\subsection{Manual Code}
\label{sec:manual}

In case one cannot be certain whether the definitions file |childdoc.def|
is installed on the target \TeX{} distribution
and one prefers not to ship it,
it is conceivable to paste a few relevant commands into the sources.

To that end, drop all statements |\input{childdoc.def}|
and perform the replacements as outlined below.
Instead of |\childdocmain{|\textit{main}|}| add the following code
to the top of the main file:
%
\begin{center}
\begin{tabular}{l}
|\||ifdefined\childdocname\endinput\||fi\newif\ifchilddoc|\\
|\edef\childdocname{\scantokens\expandafter{\jobname\noexpand}}|\\
|\def\childdocmain{|\textit{main}|}\||ifx\childdocmain\childdocname\||else|\\
|\childdoctrue\includeonly{\childdocname}\let\jobname\childdocmain\||fi|\\
\end{tabular}
\end{center}
%
Instead of |\childdocof{|\textit{main}|}| just include the main file
at the top of each child file:
%
\begin{center}
|\input{|\textit{main}|}|
\end{center}
%
A simple redirection |\childdocforward{|\textit{dest}|}| is achieved by:
%
\begin{center}
|\def\jobname{|\textit{dest}|}\input{\jobname}|
\end{center}
%
The redirection with prefix
|\childdocforwardprefix[|\textit{prefix}|]{|\textit{dest}|}|
is accomplished by:
%
\begin{center}
\begin{tabular}{l}
|{\edef\jobname{\scantokens\expandafter{\jobname\noexpand}}|\\
|\def\redirectjob |\textit{prefix}|#1~~~{\gdef\jobname{|\textit{dest}|#1}}|\\
|\expandafter\redirectjob\jobname~~~}\input{\jobname}|
\end{tabular}
\end{center}

In an alternative approach,
child documents can be compiled by a specific command line
without additional code or specific definitions:
%
\begin{center}
|... -jobname "|\textit{target}|" "|[\textit{flags}]%
|\includeonly{|\textit{dest}|}\input{|\textit{main}|}"|
\end{center}
%

%%%%%%%%%%%%%%%%%%%%%%%%%%%%%%%%%%%%%%%%%%%%%%%%%%%%%%%%%%%%%%%%%%%%%%%%%%%%%%%%
%%%%%%%%%%%%%%%%%%%%%%%%%%%%%%%%%%%%%%%%%%%%%%%%%%%%%%%%%%%%%%%%%%%%%%%%%%%%%%%%
\section{Information}

%%%%%%%%%%%%%%%%%%%%%%%%%%%%%%%%%%%%%%%%%%%%%%%%%%%%%%%%%%%%%%%%%%%%%%%%%%%%%%%%
\subsection{Copyright}

Copyright \copyright{} 2017--2018 Niklas Beisert

This work may be distributed and/or modified under the
conditions of the \LaTeX{} Project Public License, either version 1.3
of this license or (at your option) any later version.
The latest version of this license is in
  \url{http://www.latex-project.org/lppl.txt}
and version 1.3 or later is part of all distributions of \LaTeX{}
version 2005/12/01 or later.

This work has the LPPL maintenance status `maintained'.

The Current Maintainer of this work is Niklas Beisert.

This work consists of the files |README.txt|, |childdoc.ins| and |childdoc.dtx|
as well as the derived files |childdoc.def|, |cdocsamp.tex|
with |cdocsch1.tex|, |cdocsch2.tex|, |cdocspt3.tex|, |cdocspt4.tex|,
|cdocsdrf.tex|, |cdocsfn1.tex|, |cdocsfn2.tex|
as well as |childdoc.pdf|.

%%%%%%%%%%%%%%%%%%%%%%%%%%%%%%%%%%%%%%%%%%%%%%%%%%%%%%%%%%%%%%%%%%%%%%%%%%%%%%%%
\subsection{Files and Installation}

The package consists of the files:
%
\begin{center}
\begin{tabular}{ll}
    |README.txt|   & readme file \\
    |childdoc.ins| & installation file \\
    |childdoc.dtx| & source file \\
    |childdoc.def| & definition file \\
    |cdocsamp.tex| & sample main file \\
    |cdocsch1.tex| & sample include file \\
    |cdocsch2.tex| & sample include file \\
    |cdocspt3.tex| & sample part file \\
    |cdocspt4.tex| & sample part file \\
    |cdocsdrf.tex| & sample redirection file \\
    |cdocsfn1.tex| & sample redirection file \\
    |cdocsfn2.tex| & sample redirection file \\
    |childdoc.pdf| & manual
\end{tabular}
\end{center}
%
The distribution consists of the files
|README.txt|, |childdoc.ins| and |childdoc.dtx|.
%
\begin{itemize}
\item
Run (pdf)\LaTeX{} on |childdoc.dtx|
to compile the manual |childdoc.pdf| (this file).
\item
Run \LaTeX{} on |childdoc.ins| to create the definitions file |childdoc.def|
and the sample |cdocsamp.tex| with include files
|cdocsch1.tex|, |cdocsch2.tex|, |cdocspt3.tex|, |cdocspt4.tex|,
|cdocsdrf.tex|, |cdocsfn1.tex|, |cdocsfn2.tex|.
Then copy the file |childdoc.def| to an appropriate directory of your \LaTeX{}
distribution, e.g.\ \textit{texmf-root}|/tex/latex/childdoc|.
\end{itemize}

%%%%%%%%%%%%%%%%%%%%%%%%%%%%%%%%%%%%%%%%%%%%%%%%%%%%%%%%%%%%%%%%%%%%%%%%%%%%%%%%
\subsection{Related CTAN Packages}

There are several other packages which offer a similar functionality:
%
\begin{itemize}
\item
The packages
\href{http://ctan.org/pkg/docmute}{\textsf{docmute}},
\href{http://ctan.org/pkg/includex}{\textsf{includex}} and
\href{http://ctan.org/pkg/standalone}{\textsf{standalone}}
provide commands to include only the document body of
a child file thus allowing both files to be compiled individually.
\item
The packages \href{http://ctan.org/pkg/subdocs}{\textsf{subdocs}}
and \href{http://ctan.org/pkg/subfiles}{\textsf{subfiles}}
provide structures in which the main and child documents can be
encapsulated and allowing them to be compiled individually.
The inclusion mechanism is different from the conventional |\include|.
\item
The package \href{http://ctan.org/pkg/combine}{\textsf{combine}}
is an elaborate solution to combine several documents into one.
\end{itemize}
%
See also the CTAN topic \href{http://ctan.org/topic/subdocs}{\textsf{subdocs}}
for further related packages.
The present package differs from the above solutions in that
a document structure constructed with the conventional |\include| mechanism
just needs two extra commands at the top of every file
such that all constituent files can be compiled individually.

%%%%%%%%%%%%%%%%%%%%%%%%%%%%%%%%%%%%%%%%%%%%%%%%%%%%%%%%%%%%%%%%%%%%%%%%%%%%%%%%
%\subsection{Feature Suggestions}
%
%The following is a list of features which may be useful for future
%versions of this package:
%%
%\begin{itemize}
%\item
%\ldots
%\end{itemize}

%%%%%%%%%%%%%%%%%%%%%%%%%%%%%%%%%%%%%%%%%%%%%%%%%%%%%%%%%%%%%%%%%%%%%%%%%%%%%%%%
\subsection{Revision History}

%%%%%%%%%%%%%%%%%%%%%%%%%%%%%%%%%%%%%%%%
\paragraph{v2.0:} 2018/12/30

\begin{itemize}
\item
immediate forward processing
\item
added |\childdocby| mechanism
\item
manual restructured
\end{itemize}

%%%%%%%%%%%%%%%%%%%%%%%%%%%%%%%%%%%%%%%%
\paragraph{v1.6:} 2018/01/17

\begin{itemize}
\item
application for development of include files
\item
corrections to manual
\end{itemize}

%%%%%%%%%%%%%%%%%%%%%%%%%%%%%%%%%%%%%%%%
\paragraph{v1.5:} 2017/05/21

\begin{itemize}
\item
more complete structuring introduced
\item
|\childdocof| introduced
\item
|\childdoc| renamed to |\childdocmain|
\item
|\childredirect| renamed to |\childdocforward| and |\childdocforwardprefix|
and functionality expanded
\end{itemize}

%%%%%%%%%%%%%%%%%%%%%%%%%%%%%%%%%%%%%%%%
\paragraph{v1.0:} 2017/04/27

\begin{itemize}
\item
manual and install package
\item
first version published on CTAN
\end{itemize}

%%%%%%%%%%%%%%%%%%%%%%%%%%%%%%%%%%%%%%%%
\paragraph{v0.6:} 2017/04/26

\begin{itemize}
\item
redirection mechanism added
\end{itemize}

%%%%%%%%%%%%%%%%%%%%%%%%%%%%%%%%%%%%%%%%
\paragraph{v0.5:} 2017/04/26

\begin{itemize}
\item
functionality in definition file
\end{itemize}


%%%%%%%%%%%%%%%%%%%%%%%%%%%%%%%%%%%%%%%%%%%%%%%%%%%%%%%%%%%%%%%%%%%%%%%%%%%%%%%%
%%%%%%%%%%%%%%%%%%%%%%%%%%%%%%%%%%%%%%%%%%%%%%%%%%%%%%%%%%%%%%%%%%%%%%%%%%%%%%%%
%%%%%%%%%%%%%%%%%%%%%%%%%%%%%%%%%%%%%%%%%%%%%%%%%%%%%%%%%%%%%%%%%%%%%%%%%%%%%%%%
\appendix

\settowidth\MacroIndent{\rmfamily\scriptsize 000\ }

 \DocInput{childdoc.dtx}

\end{document}
%</driver>
% \fi
%
% %%%%%%%%%%%%%%%%%%%%%%%%%%%%%%%%%%%%%%%%%%%%%%%%%%%%%%%%%%%%%%%%%%%%%%%%%%%%%%
% %%%%%%%%%%%%%%%%%%%%%%%%%%%%%%%%%%%%%%%%%%%%%%%%%%%%%%%%%%%%%%%%%%%%%%%%%%%%%%
% \section{Sample}
%\iffalse
%<*samplemain>
%\fi
%
% The following presents a sample document
% with two chapters, two parts, a title page,
% a compile flag as well as three forwarding files to set the flag.
% It consists of eight |.tex| files:
% \begin{center}
% \begin{tabular}{ll}
% |cdocsamp.tex|&main file\\
% |cdocsch1.tex|&include file for chapter 1\\
% |cdocsch2.tex|&include file for chapter 2\\
% |cdocspt3.tex|&include file for part 3\\
% |cdocspt4.tex|&include file for part 4\\
% |cdocsdrf.tex|&forwarding file for main file in draft mode\\
% |cdocsfi1.tex|&forwarding file for final version of chapter 1\\
% |cdocsfi2.tex|&forwarding file for final version of chapter 2\\
% \end{tabular}
% \end{center}
% Each of the eight files can be compiled directly by the \LaTeX{} compiler.
%
% %%%%%%%%%%%%%%%%%%%%%%%%%%%%%%%%%%%%%%
% \paragraph{Main File.}
%
% The main file is called |cdocsamp.tex|.
%
% Load the \textsf{childdoc} definitions and
% declare the filename for the main document:
%    \begin{macrocode}
\input{childdoc.def}
\childdocmain{}
%    \end{macrocode}

% Optional override for |\version| flag:
%    \begin{macrocode}
%%\ifchilddoc\else\providecommand{\version}{draft}\fi
%    \end{macrocode}

% Define the default values for the |\version| flag
% (|final| for the main file and |draft| for childs):
%    \begin{macrocode}
\ifchilddoc
\providecommand{\version}{draft}
\else
\providecommand{\version}{final}
\fi
%    \end{macrocode}

% Load the standard document class:
%    \begin{macrocode}
\documentclass[12pt]{article}
%    \end{macrocode}

% Start the document body:
%    \begin{macrocode}
\begin{document}
%    \end{macrocode}

% Declare a title page.
% Print title, part of document being processed and version flag:
%    \begin{macrocode}
\addtocounter{page}{-1}
\begin{center}
{\LARGE\bfseries{}childdoc example\par}
\vspace{1cm}
\ifchilddoc
\ifchilddocmanual part\else chapter\fi:
`\childdocname' of `\childdocjob'\par
\else
main document: `\childdocjob'\par
\fi
version: \version\par
\end{center}
\newpage
%    \end{macrocode}

% Manually include selected file,
% otherwise process as usual:
%    \begin{macrocode}
\ifchilddocmanual
\section*{part `\childdocname'}
\input{\childdocname}
\else
%    \end{macrocode}

% Include the two chapters:
%    \begin{macrocode}
\include{cdocsch1}
\include{cdocsch2}
%    \end{macrocode}

% Include the two parts unless only chapters should be displayed:
%    \begin{macrocode}
\ifchilddoc\else
\section{part three}
\input{cdocspt3}
\section{part four}
\input{cdocspt4}
\fi
%    \end{macrocode}

% Process as usual until here:
%    \begin{macrocode}
\fi
%    \end{macrocode}

% End of document body:
%    \begin{macrocode}
\end{document}
%    \end{macrocode}
%\iffalse
%</samplemain>
%\fi
%
% %%%%%%%%%%%%%%%%%%%%%%%%%%%%%%%%%%%%%%
% \paragraph{Chapter Include Files.}
%
% The include files are called |cdocsch1.tex| and |cdocsch2.tex|.
%
%\iffalse
%<*samplechap1|samplechap2>
%\fi

% Optional override for |\version| flag:
%    \begin{macrocode}
%%\providecommand{\version}{final}
%    \end{macrocode}

% Include the main document:
%    \begin{macrocode}
\input{childdoc.def}
\childdocof{cdocsamp}
%    \end{macrocode}

%\iffalse
%</samplechap1|samplechap2>
%\fi
%
%\iffalse
%<*samplechap1>
%\fi
% Some text for chapter 1:
%    \begin{macrocode}
\section{one}
some text in chapter one
%    \end{macrocode}

%\iffalse
%</samplechap1>
%\fi
% Some text for chapter 2:
%\iffalse
%<*samplechap2>
%\fi
%    \begin{macrocode}
\section{two}
more text in chapter two
%    \end{macrocode}

%\iffalse
%</samplechap2>
%\fi
%
% %%%%%%%%%%%%%%%%%%%%%%%%%%%%%%%%%%%%%%
% \paragraph{Part Include Files.}
%
% The include files are called |cdocspt3.tex| and |cdocspt4.tex|.
%
%\iffalse
%<*samplepart3|samplepart4>
%\fi

% Optional override for |\version| flag:
%    \begin{macrocode}
%%\providecommand{\version}{final}
%    \end{macrocode}

% Include the main document:
%    \begin{macrocode}
\input{childdoc.def}
\childdocby{cdocsamp}
%    \end{macrocode}

%\iffalse
%</samplepart3|samplepart4>
%\fi
%
%\iffalse
%<*samplepart3>
%\fi
% Some text for part 3:
%    \begin{macrocode}
some text in part three
%    \end{macrocode}

%\iffalse
%</samplepart3>
%\fi
% Some text for part 4:
%\iffalse
%<*samplepart4>
%\fi
%    \begin{macrocode}
more text in part four
%    \end{macrocode}

%\iffalse
%</samplepart4>
%\fi
%
% %%%%%%%%%%%%%%%%%%%%%%%%%%%%%%%%%%%%%%
% \paragraph{Forwarding for a Complete Draft.}
%
% The following forwarding file |cdocsdrf.tex|
% compiles the main document in draft mode:
%\iffalse
%<*sampledraft>
%\fi
%    \begin{macrocode}
\def\version{draft}
\input{childdoc.def}
\childdocforward{cdocsamp}
%    \end{macrocode}

%\iffalse
%</sampledraft>
%\fi
%
% %%%%%%%%%%%%%%%%%%%%%%%%%%%%%%%%%%%%%%
% \paragraph{Forwarding for Final Version of the Chapters.}
%
% The following forwarding files |cdocsfn1.tex| and |cdocsfn2.tex|
% (with identical content)
% compile the final versions of the child documents
% |cdocsch1.tex| and |cdocsch2.tex|, respectively:
%\iffalse
%<*samplefinal>
%\fi
%    \begin{macrocode}
\def\version{final}
\input{childdoc.def}
\childdocforwardprefix[cdocsamp]{cdocsfn}{cdocsch}
%    \end{macrocode}

%\iffalse
%</samplefinal>
%\fi
%
% %%%%%%%%%%%%%%%%%%%%%%%%%%%%%%%%%%%%%%
% \paragraph{Command Line Processing.}
%
% The following three command lines generate the output files
% |cdocscld|, |cdocscl1| and |cdocscl2|
% which should be identical to
% |cdocsdrf|, |cdocsch1| and |cdocsfn2|, respectively:
% \begin{center}
% \begin{tabular}{l}
% |latex -jobname cdocscld \|\\
% |  "\def\version{draft}\input{childdoc.def}\childdocforward{cdocsamp}"|\\
% |latex -jobname cdocscl1 \|\\
% |  "\input{childdoc.def}\childdocforward[cdocsamp]{cdocsch1}"|\\
% |latex -jobname cdocscl2 \|\\
% |  "\def\version{final}\input{childdoc.def}\childdocforward{cdocsch2}"|
% \end{tabular}
% \end{center}
% Note that the trailing backslash on each first line
% merely continues the input to the second line
% (for convenient cut ant paste).
% Furthermore, the command |latex| can be replaced by any
% of its alternative versions such as |pdflatex|.
%
% %%%%%%%%%%%%%%%%%%%%%%%%%%%%%%%%%%%%%%%%%%%%%%%%%%%%%%%%%%%%%%%%%%%%%%%%%%%%%%
% %%%%%%%%%%%%%%%%%%%%%%%%%%%%%%%%%%%%%%%%%%%%%%%%%%%%%%%%%%%%%%%%%%%%%%%%%%%%%%
% \section{Implementation}
%\iffalse
%<*package>
%\fi
%
% This section describes the definitions file |childdoc.def|.

% The definitions cannot be loaded using |\usepackage| or |\RequirePackage|
% which has a mechanism to prevent loading a style file more than once.
% When loading the definitions by means of |\input|
% multiple instances have to be prevented manually:
%\iffalse
%This code needs to be before the `\ProvidesFile' directive
%which is defined at the beginning of this file.
%Therefore it is also placed there and commented out here.
%</package>
%<*discard>
%\fi
%    \begin{macrocode}
\ifdefined\childdocmain\endinput\fi
%    \end{macrocode}
%\iffalse
%</discard>
%<*package>
%\fi
%
% \macro{\ifchilddoc}
% \macro{\ifchilddocmanual}
% The conditional |\ifchilddoc| tells whether a
% child (true) or main (false) document is being compiled.
% The conditional |\ifchilddocmanual| tells whether
% the |\includeonly| mechanism is used (false) or
% the selection of child files must be performed manually (true).
% The definitions initialise to false:
%    \begin{macrocode}
\newif\ifchilddoc
\newif\ifchilddocmanual
%    \end{macrocode}

% \macro{\childdocname}
% \macro{\childdocjob}
% The macro |\childdocname| stores the name of the main document
% to be compiled. The macro |\childdocjob| stores the name of
% the document on which the \LaTeX{} compiler was originally invoked.
% The content of |\jobname| cannot be compared
% to filenames specified in the source due to different catcodes.
% The following code rescans |\jobname|, stores the result
% in |\childdocname| and saves a copy in |\childdocjob|:
%    \begin{macrocode}
\edef\childdocname{\scantokens\expandafter{\jobname\noexpand}}
\let\childdocjob\childdocname
%    \end{macrocode}

% \macro{\childdocdisable}
% The macro |\childdocdisable| prevents the main file
% from being processed more than once.
% At this stage, the main document command |\childdocmain|
% is assumed to be called once again where it should do nothing.
% Any subsequent call to it should prevent
% a secondary processing of the main document
% It overwrites the forwarding commands
% |\childdocof| and |\childdocforward|
% with empty macros to prevent further inclusions of the main document:
%    \begin{macrocode}
\newcommand{\childdocdisable}
{
  \renewcommand{\childdocmain}[1]{\renewcommand{\childdocmain}[1]{\endinput}}
  \renewcommand{\childdocof}[1]{}
  \renewcommand{\childdocby}[2][]{}
  \renewcommand{\childdocforward}[2][]{}
  \renewcommand{\childdocdisable}{}
}
%    \end{macrocode}

% \macro{\childdocmain}
% The macro |\childdocmain| is to be called at the top of the main file
% with nothing or the main filename (without extension) as argument.
% First, it breaks loops.
% If the argument is not empty and does not match |\childdocname|
% (which is set by the first inclusion of |childdoc.def|),
% |\ifchilddoc| is set to true, |\includeonly| is applied to the child file
% and |\jobname| is set to the main file
% (for proper handling of |.aux| files):
%    \begin{macrocode}
\newcommand{\childdocmain}[1]
{
  \childdocdisable\childdocmain{}
  \if?#1?\else
    \begingroup
      \def\childdoctmp{#1}
      \ifx\childdoctmp\childdocname
        \def\childdoctmp{}
      \else
        \def\childdoctmp
        {
          \childdoctrue
          \includeonly{\childdocname}
          \def\childdocjob{#1}
          \def\jobname{#1}
        }
      \fi
      \expandafter
    \endgroup
    \childdoctmp
  \fi
}
%    \end{macrocode}

% \macro{\childdocof}
% The command |\childdocof| redirects
% compilation to the main file |#1|.
%    \begin{macrocode}
\newcommand{\childdocof}[1]
{
  \childdocdisable
  \childdoctrue
  \includeonly{\childdocname}
  \def\jobname{#1}
  \def\childdocjob{#1}
  \input{#1}
}
%    \end{macrocode}

% \macro{\childdocby}
% The command |\childdocby| ....
%    \begin{macrocode}
\newcommand{\childdocby}[2][]
{
  \childdocdisable
  \childdoctrue
  \childdocmanualtrue
  \if?#1?\else
    \def\jobname{#2}
  \fi
  \def\childdocjob{#2}
  \input{#2}
  \endinput
}
%    \end{macrocode}

% \macro{\childdocforward}
% The command |\childdocforward| redirects
% compilation to the main file or
% (if the optional argument is given) a child file.
% Parameters are set as if the main file
% or a child file starting with |\childdocof| was compiled.
% Then compilation is handed over to the main file:
%    \begin{macrocode}
\newcommand{\childdocforward}[2][]
{
  \begingroup
    \if?#1?
      \def\childdoctmp
      {
        \def\childdocname{#2}
        \def\childdocjob{#2}
        \def\jobname{#2}
        \input{#2}
        \endinput
      }
    \else
      \def\childdoctmp
      {
        \childdocdisable
        \def\childdocname{#2}
        \childdoctrue
        \includeonly{#2}
        \def\childdocjob{#1}
        \def\jobname{#1}
        \input{#1}
        \endinput
      }
    \fi
    \expandafter
  \endgroup
  \childdoctmp
}
%    \end{macrocode}

% \macro{\childdocforwardprefix}
% The command |\childdocforwardprefix| redirects
% compilation to the main or a child file by means of a pattern.
% The prefix |#1| in the current filename is replaced by |#2|
% and the suffix of the current filename is kept
% (it is assumed that the filename does not contain the substring `|~~~|'
% which is used as a delimiter).
% Compilation is handed over to the new file by |\childdocforward|:
%    \begin{macrocode}
\newcommand{\childdocforwardprefix}[3][]
{
  \begingroup
    \def\childdocextract #2##1~~~{\def\childdoctmp{\childdocforward[#1]{#3##1}}}
    \expandafter\childdocextract\childdocname~~~
    \expandafter
  \endgroup
  \childdoctmp
}
%    \end{macrocode}

% \macro{\childdoc}
% The deprecated macro |\childdoc| is a legacy version of |\childdocmain|:
%    \begin{macrocode}
\newcommand{\childdoc}{\childdocmain}
%    \end{macrocode}

% \macro{\childdocredirect}
% The deprecated macro |\childdocredirect| is a legacy version
% of |\childdocforward| and |\childdocforwardprefix|:
%    \begin{macrocode}
\newcommand{\childdocredirect}[2][]
{
  \begingroup
    \if?#1?
      \def\childdoctmp{\childdocforward{#2}}
    \else
      \def\childdoctmp{\childdocforwardprefix{#1}{#2}}
    \fi
    \expandafter
  \endgroup
  \childdoctmp
}
%    \end{macrocode}

%\iffalse
%</package>
%\fi
%
\endinput
|\\
|\childdocmain{|\textit{main}|}|\\
\end{tabular}
\end{center}
%
If |\jobname| does not match the argument \textit{main} of |\childdocmain|,
it is assumed that |\jobname| points to the child file to be compiled.
When using |\childdocmain| with the main file specified as argument,
it suffices to start a child file
with just |\input{|\textit{main}|}|
without loading of the package and using |\childdocof|.
If instead all processing is done
with the appropriate \textsf{childdoc} directives,
the argument of \textit{main} of |\childdocmain| can be empty.

An alternative version of the command line processing described
in \secref{sec:commandline} using the detection mechanism reads:
%
\begin{center}
|... -jobname "|\textit{target}|" "|[\textit{flags}]%
[|\def\jobname{|\textit{dest}|}|]|\input{|\textit{main}|}"|
\end{center}

%%%%%%%%%%%%%%%%%%%%%%%%%%%%%%%%%%%%%%%%%%%%%%%%%%%%%%%%%%%%%%%%%%%%%%%%%%%%%%%%
\subsection{Manual Code}
\label{sec:manual}

In case one cannot be certain whether the definitions file |childdoc.def|
is installed on the target \TeX{} distribution
and one prefers not to ship it,
it is conceivable to paste a few relevant commands into the sources.

To that end, drop all statements |% \iffalse
%
% childdoc.dtx Copyright (C) 2017-2018 Niklas Beisert
%
% This work may be distributed and/or modified under the
% conditions of the LaTeX Project Public License, either version 1.3
% of this license or (at your option) any later version.
% The latest version of this license is in
%   http://www.latex-project.org/lppl.txt
% and version 1.3 or later is part of all distributions of LaTeX
% version 2005/12/01 or later.
%
% This work has the LPPL maintenance status `maintained'.
%
% The Current Maintainer of this work is Niklas Beisert.
%
% This work consists of the files childdoc.dtx and childdoc.ins
% and the derived files childdoc.def and cdocsamp.tex with
% cdocsch1.tex, cdocsch2.tex, cdocsdrf.tex, cdocsfn1.tex, cdocsfn2.tex.
%
%<package>\ifdefined\childdocmain\endinput\fi
%<package>\ProvidesFile{childdoc.def}[2018/12/30 v2.0 child document driver]
%<samplemain>\ProvidesFile{cdocsamp.tex}[2018/12/30 v2.0 sample for childdoc]
%<*driver>
%\ProvidesFile{childdoc.drv}[2018/12/30 v2.0 childdoc reference manual file]
\PassOptionsToClass{10pt,a4paper}{article}
\documentclass{ltxdoc}

\usepackage[margin=35mm]{geometry}
\usepackage{hyperref}
\usepackage{hyperxmp}
\usepackage[usenames]{color}

\hypersetup{colorlinks=true}
\hypersetup{pdfstartview=FitH}
\hypersetup{pdfpagemode=UseNone}
\hypersetup{pdfsource={}}
\hypersetup{pdflang={en-UK}}
\hypersetup{pdfcopyright={Copyright 2017-2018 Niklas Beisert.
  This work may be distributed and/or modified under the
  conditions of the LaTeX Project Public License, either version 1.3
  of this license or (at your option) any later version.}}
\hypersetup{pdflicenseurl={http://www.latex-project.org/lppl.txt}}
\hypersetup{pdfcontactaddress={ETH Zurich, ITP, HIT K,
  Wolfgang-Pauli-Strasse 27}}
\hypersetup{pdfcontactpostcode={8093}}
\hypersetup{pdfcontactcity={Zurich}}
\hypersetup{pdfcontactcountry={Switzerland}}
\hypersetup{pdfcontactemail={nbeisert@itp.phys.ethz.ch}}
\hypersetup{pdfcontacturl={http://people.phys.ethz.ch/\xmptilde nbeisert/}}

\newcommand{\secref}[1]{\hyperref[#1]{section \ref*{#1}}}

\parskip1ex
\parindent0pt
\let\olditemize\itemize
\def\itemize{\olditemize\parskip0pt}

\begin{document}

\title{The \textsf{childdoc} Package}
\hypersetup{pdftitle={The childdoc Package}}
\author{Niklas Beisert\\[2ex]
  Institut f\"ur Theoretische Physik\\
  Eidgen\"ossische Technische Hochschule Z\"urich\\
  Wolfgang-Pauli-Strasse 27, 8093 Z\"urich, Switzerland\\[1ex]
  \href{mailto:nbeisert@itp.phys.ethz.ch}
  {\texttt{nbeisert@itp.phys.ethz.ch}}}
\hypersetup{pdfauthor={Niklas Beisert}}
\hypersetup{pdfsubject={Manual for the LaTeX2e Package childdoc}}
\date{30 December 2018, \textsf{v2.0}}
\maketitle

\begin{abstract}\noindent
\textsf{childdoc} is a \LaTeXe{} package
that enables the direct compilation
of document sections included by |\include|
to individual files.
\end{abstract}

\begingroup
\parskip0ex
\tableofcontents
\endgroup

%%%%%%%%%%%%%%%%%%%%%%%%%%%%%%%%%%%%%%%%%%%%%%%%%%%%%%%%%%%%%%%%%%%%%%%%%%%%%%%%
%%%%%%%%%%%%%%%%%%%%%%%%%%%%%%%%%%%%%%%%%%%%%%%%%%%%%%%%%%%%%%%%%%%%%%%%%%%%%%%%
\section{Introduction}

\LaTeX{} provides a mechanism to structure a large document (such as a book)
into a main file and several child files (containing the chapters)
using the |\include| command.
This mechanism is beneficial for documents
which span hundreds of pages in order to
make the source file(s) more manageable.
Moreover, compilation can be restricted to
selected child files by means of the |\includeonly| command.
The latter feature can be used to reduce the compilation time while editing
(this was significantly more useful in the earlier days of \LaTeX{})
or to generate a smaller document which is easier to navigate.
Another application of |\includeonly| is to generate
documents consisting of selected parts of the complete document.

However, there are a few drawbacks of the plain |\include| mechanism:
\begin{itemize}
\item
The child files cannot be compiled on their own,
they can only be compiled via the main file.
A naive editing environment
(such as a text editor with an option
to have the current file processed by \LaTeX)
may require one to switch to the main file before compiling;
attempting to compile the child file produces errors.
\item
The main file must be modified (each time)
to adjust the |\includeonly| command
to the present needs. This easily leaves the main file in a messy state.
\item
The generated document will always carry the filename
of the main document. This is inconvenient if
several child files are to be compiled and
to be kept for distribution.
\end{itemize}

The present package provides a simple interface
to make child files individually compilable by \LaTeX{}.
Compiling a child file then has the same effect as compiling
the main file with an |\includeonly| command
to select the appropriate child.
Moreover the generated document will carry the name of the child
rather than the main file.
This resolves all three above issues.

This feature is meant to make the editing of books,
thesis documents and lecture notes somewhat more convenient.
However, the package can also be used efficiently for
composing a series of documents (such as exercise sheets)
which are typically distributed individually.
It then assists the author in generating the individual documents
(potentially in different versions)
as well as a document containing the collected series.
Another application is in developing style files
or other kinds of included material
where compilation of the style file could redirect
to a sample or test file.

%%%%%%%%%%%%%%%%%%%%%%%%%%%%%%%%%%%%%%%%%%%%%%%%%%%%%%%%%%%%%%%%%%%%%%%%%%%%%%%%
%%%%%%%%%%%%%%%%%%%%%%%%%%%%%%%%%%%%%%%%%%%%%%%%%%%%%%%%%%%%%%%%%%%%%%%%%%%%%%%%
\section{Usage}

First of all, the package \textsf{childdoc} is \emph{not} a standard
\LaTeXe{} |.sty| style file! Therefore it needs to be invoked in
a non-standard way.

%%%%%%%%%%%%%%%%%%%%%%%%%%%%%%%%%%%%%%%%%%%%%%%%%%%%%%%%%%%%%%%%%%%%%%%%%%%%%%%%
\subsection{Included Files}
\label{sec:include}

%%%%%%%%%%%%%%%%%%%%%%%%%%%%%%%%%%%%%%%%
\DescribeMacro{\childdocmain}
To use the package, add the commands
\begin{center}
\begin{tabular}{l}
|\input{childdoc.def}|\\
|\childdocmain{}|\\
\end{tabular}
\end{center}
at the very top of the main \LaTeX{} file,
in particular \emph{before} the |\documentclass| statement!
The argument of |\childdocmain| should be left empty
(but it must be present).

%%%%%%%%%%%%%%%%%%%%%%%%%%%%%%%%%%%%%%%%
\DescribeMacro{\childdocof}
Furthermore, add the commands
\begin{center}
\begin{tabular}{l}
|\input{childdoc.def}|\\
|\childdocof{|\textit{main}|}|\\
\end{tabular}
\end{center}
at the top of every child file \textit{child}
which is included by |\include{|\textit{child}|}|
from within the main file
(or at least for those files to be compiled individually).
The argument \textit{main} must be the filename of the main file.

There are a couple of
considerations in setting up the main and child documents:

%%%%%%%%%%%%%%%%%%%%%%%%%%%%%%%%%%%%%%%%
\paragraph{Restrictions.}

Please note the following restrictions:
\begin{itemize}
\item
|\childdocmain| must be called with one argument \textit{main}
to ensure compatibility with earlier version of the package.
It must either be empty (|\childdocmain{}|)
or precisely match the filename of the main file in which it is specified.
See \secref{sec:detection} for further information.
\item
The filename \textit{main} must be specified without the |.tex| extension.
\item
The filename \textit{main} is case sensitive
(even in case-insensitive file systems)
due to internal string comparison.
\item
The argument \textit{main} should be fully expanded, it cannot be a macro.
\item
Subdirectories and special characters should be avoided in filenames.
\item
The command |\childdocmain{|\textit{main}|}| must be followed by a whitespace.
It should not be followed immediately by another command
or by a comment mark `|%|'.
This is because the \TeX{} parser reads the token immediately following
the argument of |\childdocmain| and puts it
at the beginning of every child section;
however, a white\-space is ignored.
\end{itemize}

%%%%%%%%%%%%%%%%%%%%%%%%%%%%%%%%%%%%%%%%
\paragraph{Content of Main File.}

It is advisable to place all content in the child files included by |\include|.
Any output contained in the main file will appear in all child documents
unless suppressed manually;
it cannot be suppressed automatically by the |\includeonly| directive
and thus should normally be avoided.
A method to include some content in the main file
by means of conditional processing is described in \secref{sec:conditional}.

%%%%%%%%%%%%%%%%%%%%%%%%%%%%%%%%%%%%%%%%
\paragraph{Page Numbering.}

When only a part of the document is compiled,
the appropriate numbering of pages
(as well as other status parameters)
is determined from the |.aux| files.
The latter contain information from previous passes.
However this information needs to propagate through
all intermediate child documents.
Therefore the page numbering in child documents may well
be inconsistent until the complete document is compiled at least once.

A useful (if unconventional) way to always ensure a consistent
page numbering is to restart the numbering in each child document
and denote the pages by `\textit{child}|.|\textit{page}'
where \textit{child} represents the chapter/section number of the child file.
This can be achieved by the command
|\numberwithin{page}{|\textit{child}|}|
of the \textsf{amsmath} package
where \textit{child} can be |chapter| or |section|
depending on the chosen structuring.
Alternatively, one can modify the macro |\thepage| appropriately
and reset the counter |page| at the start of each child file.

%%%%%%%%%%%%%%%%%%%%%%%%%%%%%%%%%%%%%%%%%%%%%%%%%%%%%%%%%%%%%%%%%%%%%%%%%%%%%%%%
\subsection{Conditional Processing}
\label{sec:conditional}

The package provides a mechanism to compile different versions
of a document. To customise the versions further some conditional processing
can come in handy to distinguish which version is being compiled.
The package provides two macros to describe the compilation context:

%%%%%%%%%%%%%%%%%%%%%%%%%%%%%%%%%%%%%%%%
\DescribeMacro{\ifchilddoc}
The conditional |\ifchilddoc| distinguishes between the compilation of
child documents and the main document:
%
\begin{center}
|\ifchilddoc |\textit{child-code}| |[|\||else |\textit{main-code}]| \||fi|
\end{center}

%%%%%%%%%%%%%%%%%%%%%%%%%%%%%%%%%%%%%%%%
\DescribeMacro{\childdocname}
\DescribeMacro{\childdocjob}
The macro |\childdocname| contains the filename (without extension)
of the main or child file being processed.
Note that |\childdocjob| will always contain the name of the main file.

%%%%%%%%%%%%%%%%%%%%%%%%%%%%%%%%%%%%%%%%
\paragraph{Title Page.}

Conditional processing can be used to include a title or banner page
in the main document when proper precautions are taken.
Importantly, the code in the main file should ensure that the page counter
(as well as other status parameters which are stored in the |.aux| files)
takes the same value after the conditional processing.
Otherwise the page numbers may take divergent values
depending on which part is compiled.

For example, a title page could be declared by:
%
\begin{center}
\begin{tabular}{l}
|\ifchilddoc\||else|\\
|\addtocounter{page}{-1}|\\
\textit{code for title page}\\
|\newpage|\\
|\||fi|
\end{tabular}
\end{center}
%
A banner page for the child documents can be generated by:
%
\begin{center}
\begin{tabular}{l}
|\ifchilddoc|\\
|\addtocounter{page}{-1}|\\
\textit{code for banner page}\\
|\newpage|\\
|\||fi|
\end{tabular}
\end{center}
%
Here one could write a message such as:
\begin{center}
|This is the part \childdocname{} of \childdocjob{}.|
\end{center}

%%%%%%%%%%%%%%%%%%%%%%%%%%%%%%%%%%%%%%%%%%%%%%%%%%%%%%%%%%%%%%%%%%%%%%%%%%%%%%%%
\subsection{Flags}
\label{sec:flags}

The package makes it easy to generate different versions
of the main or child documents.
To this end compilation flags can be defined
and assigned different default values.
They will be particularly useful in conjunction
with the forwarding mechanism described in \secref{sec:forward}.

For example, it may be useful to have a flag |\version|
which can be set to |draft| or |final|.
The document source will contain some conditional code
depending on the value of |\version|.
Suppose further, the flag should default to |final| for the main file
and to |draft| for child files
which is a natural assignment for editing the document.
This is achieved by placing the following code
in the preamble of the main document
(below the |\childdocmain| directive):
%
\begin{center}
\begin{tabular}{l}
|\ifchilddoc|\\
|\providecommand{\version}{draft}|\\
|\||else|\\
|\providecommand{\version}{final}|\\
|\||fi|
\end{tabular}
\end{center}
%
The definition by |\providecommand| makes sure
that previous definitions are not overwritten.
Further statements |\providecommand{\version}{...}|
can thus be added before the above code to override it.

For the main file, one might add a line
(between |\childdocmain| and the above block)
%
\begin{center}
|%\ifchilddoc\||else\providecommand{\version}{draft}\||fi|
\end{center}
%
which can be uncommented to produce a draft version.
Likewise one can add a line to the very top of a child file
(above the |\childdocof{|\textit{main}|}| directive)
%
\begin{center}
|%\providecommand{\version}{final}|
\end{center}
%
which can be uncommented to produce the final version of this child document.

%%%%%%%%%%%%%%%%%%%%%%%%%%%%%%%%%%%%%%%%%%%%%%%%%%%%%%%%%%%%%%%%%%%%%%%%%%%%%%%%
\subsection{Forwarding}
\label{sec:forward}

Different versions of the main or child documents
using compilation flags as described in \secref{sec:flags}
can be (permanently) stored in different files
for convenient compilation, viewing and distribution.
To this end, the package defines a command
to pass on compilation to a different file:

%%%%%%%%%%%%%%%%%%%%%%%%%%%%%%%%%%%%%%%%
\DescribeMacro{\childdocforward}
The command |\childdocforward| redirects processing to
another source file:
%
\begin{center}
\begin{tabular}{l}
|\input{childdoc.def}|\\
|\childdocforward[|\textit{main}|]{|\textit{dest}|}|\\
\end{tabular}
\end{center}
%
The argument \textit{dest} is the destination file
(without extension).
It should be the main file or one of the child files.
Note that further \textsf{childdoc} directives
such as |\childdocof| and |\childdocforward|
in the indicated file will be processed in this form.
The optional argument \textit{main}
passes on directly to the main file \textit{main}
while pretending to compile the child \textit{dest}.
This form behaves as if \textit{dest}
issues |\childdocof{|\textit{main}|}| right away,
and no further \textsf{childdoc} directives will be processed.

%%%%%%%%%%%%%%%%%%%%%%%%%%%%%%%%%%%%%%%%
\DescribeMacro{\...prefix}
In the alternative form |\childdocforwardprefix|,
%
\begin{center}
\begin{tabular}{l}
|\input{childdoc.def}|\\
|\childdocforwardprefix[|\textit{main}|]{|\textit{prefix}|}{|\textit{dest}|}|
\end{tabular}
\end{center}
%
the destination file is determined by a pattern
depending on the current file:
To make this work, the current file must be called
`{\textit{prefix}\hspace{0.2em}\textit{suffix}}'
with \textit{prefix} matching precisely the argument.
Processing is then passed on to the file
`{\textit{dest}\hspace{0.2em}\textit{suffix}}'.
Surely, the same effect is achieved by
directly specifying the
argument `{\textit{dest}\hspace{0.2em}\textit{suffix}}'
in the first form.
However, that requires to set up a different file
for each child. With the alternative form of the command
all these files can have exactly the same content
which simplifies setting them up and maintaining them.

For example, the following file |draft.tex|
with a compilation flag |\version| as described in \secref{sec:flags}
compiles the main document as a draft:
%
\begin{center}
\begin{tabular}{l}
|\def\version{draft}|\\
|\input{childdoc.def}|\\
|\childdocforward{|\textit{main}|}|
\end{tabular}
\end{center}
%
Likewise, the following files |final|\textit{nn}|.tex|
compile the final version of the child document
|child|\textit{nn}|.tex|:
%
\begin{center}
\begin{tabular}{l}
|\def\version{final}|\\
|\input{childdoc.def}|\\
|\childdocforwardprefix{final}{child}|
\end{tabular}
\end{center}
%

Note that when several versions of a main file and/or of each child file
are to be generated, it may be convenient to set up a |Makefile| or
shell script to automatise the process.

%%%%%%%%%%%%%%%%%%%%%%%%%%%%%%%%%%%%%%%%%%%%%%%%%%%%%%%%%%%%%%%%%%%%%%%%%%%%%%%%
\subsection{Command Line Processing}
\label{sec:commandline}

The effect of redirection files can also be achieved by invoking
the \LaTeX{} compiler with a more elaborate command line.
Most conveniently this should be done as part
of a shell script or a |Makefile|.

When using \textsf{childdoc} in the main file, the following
command lines effectively perform a redirection
(note that depending on the shell being used,
backslashes may have to be doubled: `|\|' $\to$ `|\\|'):
%
\begin{center}
|... -jobname "|\textit{target}|" |\\|"|[\textit{flags}]%
|\input{childdoc.def}\childdocforward[|\textit{main}|]{|\textit{dest}|}"|
\end{center}
%
Here \textit{target} is the name of the output file,
\textit{main} is the name of the main file
and \textit{dest} is the name of the main or child file to be processed
(all filenames without extensions).
The optional argument \textit{main} can be omitted
if \textit{main} matches \textit{dest}.
Optionally, compilation \textit{flags} can be defined via |\def| commands.
This command line makes the \TeX{} engine believe
it is compiling the file \textit{target}
whose content is specified as the latter parameter.
The provided code then forwards the processing to
\textit{main} or \textit{dest} as described in \secref{sec:forward}.

%%%%%%%%%%%%%%%%%%%%%%%%%%%%%%%%%%%%%%%%%%%%%%%%%%%%%%%%%%%%%%%%%%%%%%%%%%%%%%%%
\subsection{Include by Input}
\label{sec:input}

Including child documents by |\include| has some restrictions by design.
Most notably, the content of a child document always occupies
its own set of pages; pages cannot be shared between child documents.
Usually, this behaviour makes perfect sense
because each child document contain an essential part of the document.
However, in some situations it may be desirable to compose
a document from a collection of parts
without having mandatory page breaks between then.
For this case, the package
provides a mechanism to include parts
by |\input| which can also be processed individually.
However, by construction this mechanism
requires manual handling of the content to be output.

%%%%%%%%%%%%%%%%%%%%%%%%%%%%%%%%%%%%%%%%
\DescribeMacro{\ifchilddocmanual}
The main file should be prepared as usual, see \secref{sec:include}.
However, the document body must make a distinction
between processing of an individual part and of the main document, e.g.:
%
\begin{center}
\begin{tabular}{l}
|\ifchilddocmanual|\\
|\input{\childdocname}|\\
|\||else|\\
\textit{document body with }|\input{|\textit{part}|}|\\
|\||fi|
\end{tabular}
\end{center}
%
The conditional |\ifchilddocmanual| is true whenever
a part to be included by |\input| is being compiled,
and the name of the part is stored in |\childdocname|.

%%%%%%%%%%%%%%%%%%%%%%%%%%%%%%%%%%%%%%%%
\DescribeMacro{\childdocby}
Each part to be included by |\input| should start with:
%
\begin{center}
\begin{tabular}{l}
|\input{childdoc.def}|\\
|\childdocby{|\textit{main}|}|\\
\end{tabular}
\end{center}
%
The directive |\childdocby| is similar to |\childdocof|
described in \secref{sec:include},
but the subsequent selection of content must be done manually.
To that end, both |\ifchilddoc| and |\ifchilddocmanual|
will be true upon processing of a part,
and the name of the part is stored in |\childdocname|.
Note that |\jobname| will be set to the filename of the current part
so that each part receives an individual |.aux| file
that does not interfere with the |.aux| file(s) of the main document.
This behaviour can be altered by the alternative form
|\childdocby[*]{|\textit{main}|}| (with a non-empty optional argument)
which uses the |.aux| file of the main document
by setting |\jobname| to \textit{main}.

%%%%%%%%%%%%%%%%%%%%%%%%%%%%%%%%%%%%%%%%%%%%%%%%%%%%%%%%%%%%%%%%%%%%%%%%%%%%%%%%
\subsection{Driver Development}
\label{sec:driver}

The \textsf{childdoc} mechanism can also be use for the development
of definition files such as \LaTeX{} styles or classes.
This case differs from the above setup with multiple parts
included by |\include| in that no |\includeonly| should be invoked.
This can be achieved by starting the include file
(before |\ProvidesPackage|) with:
%
\begin{center}
\begin{tabular}{l}
|\input{childdoc.def}|\\
|\childdocforward{|\textit{main}|}|\\
\end{tabular}
\end{center}
%
or alternatively with:
%
\begin{center}
\begin{tabular}{l}
|\input{childdoc.def}|\\
|\childdocby{|\textit{main}|}|\\
\end{tabular}
\end{center}
%
Both forms have slightly different effects as described above.
The main file is prepared as usual, see \secref{sec:include}.

%%%%%%%%%%%%%%%%%%%%%%%%%%%%%%%%%%%%%%%%%%%%%%%%%%%%%%%%%%%%%%%%%%%%%%%%%%%%%%%%
\subsection{Legacy Detection}
\label{sec:detection}

The directive |\childdocmain| in the main file can detect
whether the complete document or merely a child is to be compiled
even without using the directive |\childdocof|.
This method is deprecated because it is less robust
and there is no compelling reason to use it;
it is merely provided for backward compatibility
and it may be removed in future versions.

If the detection mechanism is to be used,
it is mandatory to correctly specify
the filename of the main file as the argument of |\childdocmain|:
%
\begin{center}
\begin{tabular}{l}
|\input{childdoc.def}|\\
|\childdocmain{|\textit{main}|}|\\
\end{tabular}
\end{center}
%
If |\jobname| does not match the argument \textit{main} of |\childdocmain|,
it is assumed that |\jobname| points to the child file to be compiled.
When using |\childdocmain| with the main file specified as argument,
it suffices to start a child file
with just |\input{|\textit{main}|}|
without loading of the package and using |\childdocof|.
If instead all processing is done
with the appropriate \textsf{childdoc} directives,
the argument of \textit{main} of |\childdocmain| can be empty.

An alternative version of the command line processing described
in \secref{sec:commandline} using the detection mechanism reads:
%
\begin{center}
|... -jobname "|\textit{target}|" "|[\textit{flags}]%
[|\def\jobname{|\textit{dest}|}|]|\input{|\textit{main}|}"|
\end{center}

%%%%%%%%%%%%%%%%%%%%%%%%%%%%%%%%%%%%%%%%%%%%%%%%%%%%%%%%%%%%%%%%%%%%%%%%%%%%%%%%
\subsection{Manual Code}
\label{sec:manual}

In case one cannot be certain whether the definitions file |childdoc.def|
is installed on the target \TeX{} distribution
and one prefers not to ship it,
it is conceivable to paste a few relevant commands into the sources.

To that end, drop all statements |\input{childdoc.def}|
and perform the replacements as outlined below.
Instead of |\childdocmain{|\textit{main}|}| add the following code
to the top of the main file:
%
\begin{center}
\begin{tabular}{l}
|\||ifdefined\childdocname\endinput\||fi\newif\ifchilddoc|\\
|\edef\childdocname{\scantokens\expandafter{\jobname\noexpand}}|\\
|\def\childdocmain{|\textit{main}|}\||ifx\childdocmain\childdocname\||else|\\
|\childdoctrue\includeonly{\childdocname}\let\jobname\childdocmain\||fi|\\
\end{tabular}
\end{center}
%
Instead of |\childdocof{|\textit{main}|}| just include the main file
at the top of each child file:
%
\begin{center}
|\input{|\textit{main}|}|
\end{center}
%
A simple redirection |\childdocforward{|\textit{dest}|}| is achieved by:
%
\begin{center}
|\def\jobname{|\textit{dest}|}\input{\jobname}|
\end{center}
%
The redirection with prefix
|\childdocforwardprefix[|\textit{prefix}|]{|\textit{dest}|}|
is accomplished by:
%
\begin{center}
\begin{tabular}{l}
|{\edef\jobname{\scantokens\expandafter{\jobname\noexpand}}|\\
|\def\redirectjob |\textit{prefix}|#1~~~{\gdef\jobname{|\textit{dest}|#1}}|\\
|\expandafter\redirectjob\jobname~~~}\input{\jobname}|
\end{tabular}
\end{center}

In an alternative approach,
child documents can be compiled by a specific command line
without additional code or specific definitions:
%
\begin{center}
|... -jobname "|\textit{target}|" "|[\textit{flags}]%
|\includeonly{|\textit{dest}|}\input{|\textit{main}|}"|
\end{center}
%

%%%%%%%%%%%%%%%%%%%%%%%%%%%%%%%%%%%%%%%%%%%%%%%%%%%%%%%%%%%%%%%%%%%%%%%%%%%%%%%%
%%%%%%%%%%%%%%%%%%%%%%%%%%%%%%%%%%%%%%%%%%%%%%%%%%%%%%%%%%%%%%%%%%%%%%%%%%%%%%%%
\section{Information}

%%%%%%%%%%%%%%%%%%%%%%%%%%%%%%%%%%%%%%%%%%%%%%%%%%%%%%%%%%%%%%%%%%%%%%%%%%%%%%%%
\subsection{Copyright}

Copyright \copyright{} 2017--2018 Niklas Beisert

This work may be distributed and/or modified under the
conditions of the \LaTeX{} Project Public License, either version 1.3
of this license or (at your option) any later version.
The latest version of this license is in
  \url{http://www.latex-project.org/lppl.txt}
and version 1.3 or later is part of all distributions of \LaTeX{}
version 2005/12/01 or later.

This work has the LPPL maintenance status `maintained'.

The Current Maintainer of this work is Niklas Beisert.

This work consists of the files |README.txt|, |childdoc.ins| and |childdoc.dtx|
as well as the derived files |childdoc.def|, |cdocsamp.tex|
with |cdocsch1.tex|, |cdocsch2.tex|, |cdocspt3.tex|, |cdocspt4.tex|,
|cdocsdrf.tex|, |cdocsfn1.tex|, |cdocsfn2.tex|
as well as |childdoc.pdf|.

%%%%%%%%%%%%%%%%%%%%%%%%%%%%%%%%%%%%%%%%%%%%%%%%%%%%%%%%%%%%%%%%%%%%%%%%%%%%%%%%
\subsection{Files and Installation}

The package consists of the files:
%
\begin{center}
\begin{tabular}{ll}
    |README.txt|   & readme file \\
    |childdoc.ins| & installation file \\
    |childdoc.dtx| & source file \\
    |childdoc.def| & definition file \\
    |cdocsamp.tex| & sample main file \\
    |cdocsch1.tex| & sample include file \\
    |cdocsch2.tex| & sample include file \\
    |cdocspt3.tex| & sample part file \\
    |cdocspt4.tex| & sample part file \\
    |cdocsdrf.tex| & sample redirection file \\
    |cdocsfn1.tex| & sample redirection file \\
    |cdocsfn2.tex| & sample redirection file \\
    |childdoc.pdf| & manual
\end{tabular}
\end{center}
%
The distribution consists of the files
|README.txt|, |childdoc.ins| and |childdoc.dtx|.
%
\begin{itemize}
\item
Run (pdf)\LaTeX{} on |childdoc.dtx|
to compile the manual |childdoc.pdf| (this file).
\item
Run \LaTeX{} on |childdoc.ins| to create the definitions file |childdoc.def|
and the sample |cdocsamp.tex| with include files
|cdocsch1.tex|, |cdocsch2.tex|, |cdocspt3.tex|, |cdocspt4.tex|,
|cdocsdrf.tex|, |cdocsfn1.tex|, |cdocsfn2.tex|.
Then copy the file |childdoc.def| to an appropriate directory of your \LaTeX{}
distribution, e.g.\ \textit{texmf-root}|/tex/latex/childdoc|.
\end{itemize}

%%%%%%%%%%%%%%%%%%%%%%%%%%%%%%%%%%%%%%%%%%%%%%%%%%%%%%%%%%%%%%%%%%%%%%%%%%%%%%%%
\subsection{Related CTAN Packages}

There are several other packages which offer a similar functionality:
%
\begin{itemize}
\item
The packages
\href{http://ctan.org/pkg/docmute}{\textsf{docmute}},
\href{http://ctan.org/pkg/includex}{\textsf{includex}} and
\href{http://ctan.org/pkg/standalone}{\textsf{standalone}}
provide commands to include only the document body of
a child file thus allowing both files to be compiled individually.
\item
The packages \href{http://ctan.org/pkg/subdocs}{\textsf{subdocs}}
and \href{http://ctan.org/pkg/subfiles}{\textsf{subfiles}}
provide structures in which the main and child documents can be
encapsulated and allowing them to be compiled individually.
The inclusion mechanism is different from the conventional |\include|.
\item
The package \href{http://ctan.org/pkg/combine}{\textsf{combine}}
is an elaborate solution to combine several documents into one.
\end{itemize}
%
See also the CTAN topic \href{http://ctan.org/topic/subdocs}{\textsf{subdocs}}
for further related packages.
The present package differs from the above solutions in that
a document structure constructed with the conventional |\include| mechanism
just needs two extra commands at the top of every file
such that all constituent files can be compiled individually.

%%%%%%%%%%%%%%%%%%%%%%%%%%%%%%%%%%%%%%%%%%%%%%%%%%%%%%%%%%%%%%%%%%%%%%%%%%%%%%%%
%\subsection{Feature Suggestions}
%
%The following is a list of features which may be useful for future
%versions of this package:
%%
%\begin{itemize}
%\item
%\ldots
%\end{itemize}

%%%%%%%%%%%%%%%%%%%%%%%%%%%%%%%%%%%%%%%%%%%%%%%%%%%%%%%%%%%%%%%%%%%%%%%%%%%%%%%%
\subsection{Revision History}

%%%%%%%%%%%%%%%%%%%%%%%%%%%%%%%%%%%%%%%%
\paragraph{v2.0:} 2018/12/30

\begin{itemize}
\item
immediate forward processing
\item
added |\childdocby| mechanism
\item
manual restructured
\end{itemize}

%%%%%%%%%%%%%%%%%%%%%%%%%%%%%%%%%%%%%%%%
\paragraph{v1.6:} 2018/01/17

\begin{itemize}
\item
application for development of include files
\item
corrections to manual
\end{itemize}

%%%%%%%%%%%%%%%%%%%%%%%%%%%%%%%%%%%%%%%%
\paragraph{v1.5:} 2017/05/21

\begin{itemize}
\item
more complete structuring introduced
\item
|\childdocof| introduced
\item
|\childdoc| renamed to |\childdocmain|
\item
|\childredirect| renamed to |\childdocforward| and |\childdocforwardprefix|
and functionality expanded
\end{itemize}

%%%%%%%%%%%%%%%%%%%%%%%%%%%%%%%%%%%%%%%%
\paragraph{v1.0:} 2017/04/27

\begin{itemize}
\item
manual and install package
\item
first version published on CTAN
\end{itemize}

%%%%%%%%%%%%%%%%%%%%%%%%%%%%%%%%%%%%%%%%
\paragraph{v0.6:} 2017/04/26

\begin{itemize}
\item
redirection mechanism added
\end{itemize}

%%%%%%%%%%%%%%%%%%%%%%%%%%%%%%%%%%%%%%%%
\paragraph{v0.5:} 2017/04/26

\begin{itemize}
\item
functionality in definition file
\end{itemize}


%%%%%%%%%%%%%%%%%%%%%%%%%%%%%%%%%%%%%%%%%%%%%%%%%%%%%%%%%%%%%%%%%%%%%%%%%%%%%%%%
%%%%%%%%%%%%%%%%%%%%%%%%%%%%%%%%%%%%%%%%%%%%%%%%%%%%%%%%%%%%%%%%%%%%%%%%%%%%%%%%
%%%%%%%%%%%%%%%%%%%%%%%%%%%%%%%%%%%%%%%%%%%%%%%%%%%%%%%%%%%%%%%%%%%%%%%%%%%%%%%%
\appendix

\settowidth\MacroIndent{\rmfamily\scriptsize 000\ }

 \DocInput{childdoc.dtx}

\end{document}
%</driver>
% \fi
%
% %%%%%%%%%%%%%%%%%%%%%%%%%%%%%%%%%%%%%%%%%%%%%%%%%%%%%%%%%%%%%%%%%%%%%%%%%%%%%%
% %%%%%%%%%%%%%%%%%%%%%%%%%%%%%%%%%%%%%%%%%%%%%%%%%%%%%%%%%%%%%%%%%%%%%%%%%%%%%%
% \section{Sample}
%\iffalse
%<*samplemain>
%\fi
%
% The following presents a sample document
% with two chapters, two parts, a title page,
% a compile flag as well as three forwarding files to set the flag.
% It consists of eight |.tex| files:
% \begin{center}
% \begin{tabular}{ll}
% |cdocsamp.tex|&main file\\
% |cdocsch1.tex|&include file for chapter 1\\
% |cdocsch2.tex|&include file for chapter 2\\
% |cdocspt3.tex|&include file for part 3\\
% |cdocspt4.tex|&include file for part 4\\
% |cdocsdrf.tex|&forwarding file for main file in draft mode\\
% |cdocsfi1.tex|&forwarding file for final version of chapter 1\\
% |cdocsfi2.tex|&forwarding file for final version of chapter 2\\
% \end{tabular}
% \end{center}
% Each of the eight files can be compiled directly by the \LaTeX{} compiler.
%
% %%%%%%%%%%%%%%%%%%%%%%%%%%%%%%%%%%%%%%
% \paragraph{Main File.}
%
% The main file is called |cdocsamp.tex|.
%
% Load the \textsf{childdoc} definitions and
% declare the filename for the main document:
%    \begin{macrocode}
\input{childdoc.def}
\childdocmain{}
%    \end{macrocode}

% Optional override for |\version| flag:
%    \begin{macrocode}
%%\ifchilddoc\else\providecommand{\version}{draft}\fi
%    \end{macrocode}

% Define the default values for the |\version| flag
% (|final| for the main file and |draft| for childs):
%    \begin{macrocode}
\ifchilddoc
\providecommand{\version}{draft}
\else
\providecommand{\version}{final}
\fi
%    \end{macrocode}

% Load the standard document class:
%    \begin{macrocode}
\documentclass[12pt]{article}
%    \end{macrocode}

% Start the document body:
%    \begin{macrocode}
\begin{document}
%    \end{macrocode}

% Declare a title page.
% Print title, part of document being processed and version flag:
%    \begin{macrocode}
\addtocounter{page}{-1}
\begin{center}
{\LARGE\bfseries{}childdoc example\par}
\vspace{1cm}
\ifchilddoc
\ifchilddocmanual part\else chapter\fi:
`\childdocname' of `\childdocjob'\par
\else
main document: `\childdocjob'\par
\fi
version: \version\par
\end{center}
\newpage
%    \end{macrocode}

% Manually include selected file,
% otherwise process as usual:
%    \begin{macrocode}
\ifchilddocmanual
\section*{part `\childdocname'}
\input{\childdocname}
\else
%    \end{macrocode}

% Include the two chapters:
%    \begin{macrocode}
\include{cdocsch1}
\include{cdocsch2}
%    \end{macrocode}

% Include the two parts unless only chapters should be displayed:
%    \begin{macrocode}
\ifchilddoc\else
\section{part three}
\input{cdocspt3}
\section{part four}
\input{cdocspt4}
\fi
%    \end{macrocode}

% Process as usual until here:
%    \begin{macrocode}
\fi
%    \end{macrocode}

% End of document body:
%    \begin{macrocode}
\end{document}
%    \end{macrocode}
%\iffalse
%</samplemain>
%\fi
%
% %%%%%%%%%%%%%%%%%%%%%%%%%%%%%%%%%%%%%%
% \paragraph{Chapter Include Files.}
%
% The include files are called |cdocsch1.tex| and |cdocsch2.tex|.
%
%\iffalse
%<*samplechap1|samplechap2>
%\fi

% Optional override for |\version| flag:
%    \begin{macrocode}
%%\providecommand{\version}{final}
%    \end{macrocode}

% Include the main document:
%    \begin{macrocode}
\input{childdoc.def}
\childdocof{cdocsamp}
%    \end{macrocode}

%\iffalse
%</samplechap1|samplechap2>
%\fi
%
%\iffalse
%<*samplechap1>
%\fi
% Some text for chapter 1:
%    \begin{macrocode}
\section{one}
some text in chapter one
%    \end{macrocode}

%\iffalse
%</samplechap1>
%\fi
% Some text for chapter 2:
%\iffalse
%<*samplechap2>
%\fi
%    \begin{macrocode}
\section{two}
more text in chapter two
%    \end{macrocode}

%\iffalse
%</samplechap2>
%\fi
%
% %%%%%%%%%%%%%%%%%%%%%%%%%%%%%%%%%%%%%%
% \paragraph{Part Include Files.}
%
% The include files are called |cdocspt3.tex| and |cdocspt4.tex|.
%
%\iffalse
%<*samplepart3|samplepart4>
%\fi

% Optional override for |\version| flag:
%    \begin{macrocode}
%%\providecommand{\version}{final}
%    \end{macrocode}

% Include the main document:
%    \begin{macrocode}
\input{childdoc.def}
\childdocby{cdocsamp}
%    \end{macrocode}

%\iffalse
%</samplepart3|samplepart4>
%\fi
%
%\iffalse
%<*samplepart3>
%\fi
% Some text for part 3:
%    \begin{macrocode}
some text in part three
%    \end{macrocode}

%\iffalse
%</samplepart3>
%\fi
% Some text for part 4:
%\iffalse
%<*samplepart4>
%\fi
%    \begin{macrocode}
more text in part four
%    \end{macrocode}

%\iffalse
%</samplepart4>
%\fi
%
% %%%%%%%%%%%%%%%%%%%%%%%%%%%%%%%%%%%%%%
% \paragraph{Forwarding for a Complete Draft.}
%
% The following forwarding file |cdocsdrf.tex|
% compiles the main document in draft mode:
%\iffalse
%<*sampledraft>
%\fi
%    \begin{macrocode}
\def\version{draft}
\input{childdoc.def}
\childdocforward{cdocsamp}
%    \end{macrocode}

%\iffalse
%</sampledraft>
%\fi
%
% %%%%%%%%%%%%%%%%%%%%%%%%%%%%%%%%%%%%%%
% \paragraph{Forwarding for Final Version of the Chapters.}
%
% The following forwarding files |cdocsfn1.tex| and |cdocsfn2.tex|
% (with identical content)
% compile the final versions of the child documents
% |cdocsch1.tex| and |cdocsch2.tex|, respectively:
%\iffalse
%<*samplefinal>
%\fi
%    \begin{macrocode}
\def\version{final}
\input{childdoc.def}
\childdocforwardprefix[cdocsamp]{cdocsfn}{cdocsch}
%    \end{macrocode}

%\iffalse
%</samplefinal>
%\fi
%
% %%%%%%%%%%%%%%%%%%%%%%%%%%%%%%%%%%%%%%
% \paragraph{Command Line Processing.}
%
% The following three command lines generate the output files
% |cdocscld|, |cdocscl1| and |cdocscl2|
% which should be identical to
% |cdocsdrf|, |cdocsch1| and |cdocsfn2|, respectively:
% \begin{center}
% \begin{tabular}{l}
% |latex -jobname cdocscld \|\\
% |  "\def\version{draft}\input{childdoc.def}\childdocforward{cdocsamp}"|\\
% |latex -jobname cdocscl1 \|\\
% |  "\input{childdoc.def}\childdocforward[cdocsamp]{cdocsch1}"|\\
% |latex -jobname cdocscl2 \|\\
% |  "\def\version{final}\input{childdoc.def}\childdocforward{cdocsch2}"|
% \end{tabular}
% \end{center}
% Note that the trailing backslash on each first line
% merely continues the input to the second line
% (for convenient cut ant paste).
% Furthermore, the command |latex| can be replaced by any
% of its alternative versions such as |pdflatex|.
%
% %%%%%%%%%%%%%%%%%%%%%%%%%%%%%%%%%%%%%%%%%%%%%%%%%%%%%%%%%%%%%%%%%%%%%%%%%%%%%%
% %%%%%%%%%%%%%%%%%%%%%%%%%%%%%%%%%%%%%%%%%%%%%%%%%%%%%%%%%%%%%%%%%%%%%%%%%%%%%%
% \section{Implementation}
%\iffalse
%<*package>
%\fi
%
% This section describes the definitions file |childdoc.def|.

% The definitions cannot be loaded using |\usepackage| or |\RequirePackage|
% which has a mechanism to prevent loading a style file more than once.
% When loading the definitions by means of |\input|
% multiple instances have to be prevented manually:
%\iffalse
%This code needs to be before the `\ProvidesFile' directive
%which is defined at the beginning of this file.
%Therefore it is also placed there and commented out here.
%</package>
%<*discard>
%\fi
%    \begin{macrocode}
\ifdefined\childdocmain\endinput\fi
%    \end{macrocode}
%\iffalse
%</discard>
%<*package>
%\fi
%
% \macro{\ifchilddoc}
% \macro{\ifchilddocmanual}
% The conditional |\ifchilddoc| tells whether a
% child (true) or main (false) document is being compiled.
% The conditional |\ifchilddocmanual| tells whether
% the |\includeonly| mechanism is used (false) or
% the selection of child files must be performed manually (true).
% The definitions initialise to false:
%    \begin{macrocode}
\newif\ifchilddoc
\newif\ifchilddocmanual
%    \end{macrocode}

% \macro{\childdocname}
% \macro{\childdocjob}
% The macro |\childdocname| stores the name of the main document
% to be compiled. The macro |\childdocjob| stores the name of
% the document on which the \LaTeX{} compiler was originally invoked.
% The content of |\jobname| cannot be compared
% to filenames specified in the source due to different catcodes.
% The following code rescans |\jobname|, stores the result
% in |\childdocname| and saves a copy in |\childdocjob|:
%    \begin{macrocode}
\edef\childdocname{\scantokens\expandafter{\jobname\noexpand}}
\let\childdocjob\childdocname
%    \end{macrocode}

% \macro{\childdocdisable}
% The macro |\childdocdisable| prevents the main file
% from being processed more than once.
% At this stage, the main document command |\childdocmain|
% is assumed to be called once again where it should do nothing.
% Any subsequent call to it should prevent
% a secondary processing of the main document
% It overwrites the forwarding commands
% |\childdocof| and |\childdocforward|
% with empty macros to prevent further inclusions of the main document:
%    \begin{macrocode}
\newcommand{\childdocdisable}
{
  \renewcommand{\childdocmain}[1]{\renewcommand{\childdocmain}[1]{\endinput}}
  \renewcommand{\childdocof}[1]{}
  \renewcommand{\childdocby}[2][]{}
  \renewcommand{\childdocforward}[2][]{}
  \renewcommand{\childdocdisable}{}
}
%    \end{macrocode}

% \macro{\childdocmain}
% The macro |\childdocmain| is to be called at the top of the main file
% with nothing or the main filename (without extension) as argument.
% First, it breaks loops.
% If the argument is not empty and does not match |\childdocname|
% (which is set by the first inclusion of |childdoc.def|),
% |\ifchilddoc| is set to true, |\includeonly| is applied to the child file
% and |\jobname| is set to the main file
% (for proper handling of |.aux| files):
%    \begin{macrocode}
\newcommand{\childdocmain}[1]
{
  \childdocdisable\childdocmain{}
  \if?#1?\else
    \begingroup
      \def\childdoctmp{#1}
      \ifx\childdoctmp\childdocname
        \def\childdoctmp{}
      \else
        \def\childdoctmp
        {
          \childdoctrue
          \includeonly{\childdocname}
          \def\childdocjob{#1}
          \def\jobname{#1}
        }
      \fi
      \expandafter
    \endgroup
    \childdoctmp
  \fi
}
%    \end{macrocode}

% \macro{\childdocof}
% The command |\childdocof| redirects
% compilation to the main file |#1|.
%    \begin{macrocode}
\newcommand{\childdocof}[1]
{
  \childdocdisable
  \childdoctrue
  \includeonly{\childdocname}
  \def\jobname{#1}
  \def\childdocjob{#1}
  \input{#1}
}
%    \end{macrocode}

% \macro{\childdocby}
% The command |\childdocby| ....
%    \begin{macrocode}
\newcommand{\childdocby}[2][]
{
  \childdocdisable
  \childdoctrue
  \childdocmanualtrue
  \if?#1?\else
    \def\jobname{#2}
  \fi
  \def\childdocjob{#2}
  \input{#2}
  \endinput
}
%    \end{macrocode}

% \macro{\childdocforward}
% The command |\childdocforward| redirects
% compilation to the main file or
% (if the optional argument is given) a child file.
% Parameters are set as if the main file
% or a child file starting with |\childdocof| was compiled.
% Then compilation is handed over to the main file:
%    \begin{macrocode}
\newcommand{\childdocforward}[2][]
{
  \begingroup
    \if?#1?
      \def\childdoctmp
      {
        \def\childdocname{#2}
        \def\childdocjob{#2}
        \def\jobname{#2}
        \input{#2}
        \endinput
      }
    \else
      \def\childdoctmp
      {
        \childdocdisable
        \def\childdocname{#2}
        \childdoctrue
        \includeonly{#2}
        \def\childdocjob{#1}
        \def\jobname{#1}
        \input{#1}
        \endinput
      }
    \fi
    \expandafter
  \endgroup
  \childdoctmp
}
%    \end{macrocode}

% \macro{\childdocforwardprefix}
% The command |\childdocforwardprefix| redirects
% compilation to the main or a child file by means of a pattern.
% The prefix |#1| in the current filename is replaced by |#2|
% and the suffix of the current filename is kept
% (it is assumed that the filename does not contain the substring `|~~~|'
% which is used as a delimiter).
% Compilation is handed over to the new file by |\childdocforward|:
%    \begin{macrocode}
\newcommand{\childdocforwardprefix}[3][]
{
  \begingroup
    \def\childdocextract #2##1~~~{\def\childdoctmp{\childdocforward[#1]{#3##1}}}
    \expandafter\childdocextract\childdocname~~~
    \expandafter
  \endgroup
  \childdoctmp
}
%    \end{macrocode}

% \macro{\childdoc}
% The deprecated macro |\childdoc| is a legacy version of |\childdocmain|:
%    \begin{macrocode}
\newcommand{\childdoc}{\childdocmain}
%    \end{macrocode}

% \macro{\childdocredirect}
% The deprecated macro |\childdocredirect| is a legacy version
% of |\childdocforward| and |\childdocforwardprefix|:
%    \begin{macrocode}
\newcommand{\childdocredirect}[2][]
{
  \begingroup
    \if?#1?
      \def\childdoctmp{\childdocforward{#2}}
    \else
      \def\childdoctmp{\childdocforwardprefix{#1}{#2}}
    \fi
    \expandafter
  \endgroup
  \childdoctmp
}
%    \end{macrocode}

%\iffalse
%</package>
%\fi
%
\endinput
|
and perform the replacements as outlined below.
Instead of |\childdocmain{|\textit{main}|}| add the following code
to the top of the main file:
%
\begin{center}
\begin{tabular}{l}
|\||ifdefined\childdocname\endinput\||fi\newif\ifchilddoc|\\
|\edef\childdocname{\scantokens\expandafter{\jobname\noexpand}}|\\
|\def\childdocmain{|\textit{main}|}\||ifx\childdocmain\childdocname\||else|\\
|\childdoctrue\includeonly{\childdocname}\let\jobname\childdocmain\||fi|\\
\end{tabular}
\end{center}
%
Instead of |\childdocof{|\textit{main}|}| just include the main file
at the top of each child file:
%
\begin{center}
|\input{|\textit{main}|}|
\end{center}
%
A simple redirection |\childdocforward{|\textit{dest}|}| is achieved by:
%
\begin{center}
|\def\jobname{|\textit{dest}|}\input{\jobname}|
\end{center}
%
The redirection with prefix
|\childdocforwardprefix[|\textit{prefix}|]{|\textit{dest}|}|
is accomplished by:
%
\begin{center}
\begin{tabular}{l}
|{\edef\jobname{\scantokens\expandafter{\jobname\noexpand}}|\\
|\def\redirectjob |\textit{prefix}|#1~~~{\gdef\jobname{|\textit{dest}|#1}}|\\
|\expandafter\redirectjob\jobname~~~}\input{\jobname}|
\end{tabular}
\end{center}

In an alternative approach,
child documents can be compiled by a specific command line
without additional code or specific definitions:
%
\begin{center}
|... -jobname "|\textit{target}|" "|[\textit{flags}]%
|\includeonly{|\textit{dest}|}\input{|\textit{main}|}"|
\end{center}
%

%%%%%%%%%%%%%%%%%%%%%%%%%%%%%%%%%%%%%%%%%%%%%%%%%%%%%%%%%%%%%%%%%%%%%%%%%%%%%%%%
%%%%%%%%%%%%%%%%%%%%%%%%%%%%%%%%%%%%%%%%%%%%%%%%%%%%%%%%%%%%%%%%%%%%%%%%%%%%%%%%
\section{Information}

%%%%%%%%%%%%%%%%%%%%%%%%%%%%%%%%%%%%%%%%%%%%%%%%%%%%%%%%%%%%%%%%%%%%%%%%%%%%%%%%
\subsection{Copyright}

Copyright \copyright{} 2017--2018 Niklas Beisert

This work may be distributed and/or modified under the
conditions of the \LaTeX{} Project Public License, either version 1.3
of this license or (at your option) any later version.
The latest version of this license is in
  \url{http://www.latex-project.org/lppl.txt}
and version 1.3 or later is part of all distributions of \LaTeX{}
version 2005/12/01 or later.

This work has the LPPL maintenance status `maintained'.

The Current Maintainer of this work is Niklas Beisert.

This work consists of the files |README.txt|, |childdoc.ins| and |childdoc.dtx|
as well as the derived files |childdoc.def|, |cdocsamp.tex|
with |cdocsch1.tex|, |cdocsch2.tex|, |cdocspt3.tex|, |cdocspt4.tex|,
|cdocsdrf.tex|, |cdocsfn1.tex|, |cdocsfn2.tex|
as well as |childdoc.pdf|.

%%%%%%%%%%%%%%%%%%%%%%%%%%%%%%%%%%%%%%%%%%%%%%%%%%%%%%%%%%%%%%%%%%%%%%%%%%%%%%%%
\subsection{Files and Installation}

The package consists of the files:
%
\begin{center}
\begin{tabular}{ll}
    |README.txt|   & readme file \\
    |childdoc.ins| & installation file \\
    |childdoc.dtx| & source file \\
    |childdoc.def| & definition file \\
    |cdocsamp.tex| & sample main file \\
    |cdocsch1.tex| & sample include file \\
    |cdocsch2.tex| & sample include file \\
    |cdocspt3.tex| & sample part file \\
    |cdocspt4.tex| & sample part file \\
    |cdocsdrf.tex| & sample redirection file \\
    |cdocsfn1.tex| & sample redirection file \\
    |cdocsfn2.tex| & sample redirection file \\
    |childdoc.pdf| & manual
\end{tabular}
\end{center}
%
The distribution consists of the files
|README.txt|, |childdoc.ins| and |childdoc.dtx|.
%
\begin{itemize}
\item
Run (pdf)\LaTeX{} on |childdoc.dtx|
to compile the manual |childdoc.pdf| (this file).
\item
Run \LaTeX{} on |childdoc.ins| to create the definitions file |childdoc.def|
and the sample |cdocsamp.tex| with include files
|cdocsch1.tex|, |cdocsch2.tex|, |cdocspt3.tex|, |cdocspt4.tex|,
|cdocsdrf.tex|, |cdocsfn1.tex|, |cdocsfn2.tex|.
Then copy the file |childdoc.def| to an appropriate directory of your \LaTeX{}
distribution, e.g.\ \textit{texmf-root}|/tex/latex/childdoc|.
\end{itemize}

%%%%%%%%%%%%%%%%%%%%%%%%%%%%%%%%%%%%%%%%%%%%%%%%%%%%%%%%%%%%%%%%%%%%%%%%%%%%%%%%
\subsection{Related CTAN Packages}

There are several other packages which offer a similar functionality:
%
\begin{itemize}
\item
The packages
\href{http://ctan.org/pkg/docmute}{\textsf{docmute}},
\href{http://ctan.org/pkg/includex}{\textsf{includex}} and
\href{http://ctan.org/pkg/standalone}{\textsf{standalone}}
provide commands to include only the document body of
a child file thus allowing both files to be compiled individually.
\item
The packages \href{http://ctan.org/pkg/subdocs}{\textsf{subdocs}}
and \href{http://ctan.org/pkg/subfiles}{\textsf{subfiles}}
provide structures in which the main and child documents can be
encapsulated and allowing them to be compiled individually.
The inclusion mechanism is different from the conventional |\include|.
\item
The package \href{http://ctan.org/pkg/combine}{\textsf{combine}}
is an elaborate solution to combine several documents into one.
\end{itemize}
%
See also the CTAN topic \href{http://ctan.org/topic/subdocs}{\textsf{subdocs}}
for further related packages.
The present package differs from the above solutions in that
a document structure constructed with the conventional |\include| mechanism
just needs two extra commands at the top of every file
such that all constituent files can be compiled individually.

%%%%%%%%%%%%%%%%%%%%%%%%%%%%%%%%%%%%%%%%%%%%%%%%%%%%%%%%%%%%%%%%%%%%%%%%%%%%%%%%
%\subsection{Feature Suggestions}
%
%The following is a list of features which may be useful for future
%versions of this package:
%%
%\begin{itemize}
%\item
%\ldots
%\end{itemize}

%%%%%%%%%%%%%%%%%%%%%%%%%%%%%%%%%%%%%%%%%%%%%%%%%%%%%%%%%%%%%%%%%%%%%%%%%%%%%%%%
\subsection{Revision History}

%%%%%%%%%%%%%%%%%%%%%%%%%%%%%%%%%%%%%%%%
\paragraph{v2.0:} 2018/12/30

\begin{itemize}
\item
immediate forward processing
\item
added |\childdocby| mechanism
\item
manual restructured
\end{itemize}

%%%%%%%%%%%%%%%%%%%%%%%%%%%%%%%%%%%%%%%%
\paragraph{v1.6:} 2018/01/17

\begin{itemize}
\item
application for development of include files
\item
corrections to manual
\end{itemize}

%%%%%%%%%%%%%%%%%%%%%%%%%%%%%%%%%%%%%%%%
\paragraph{v1.5:} 2017/05/21

\begin{itemize}
\item
more complete structuring introduced
\item
|\childdocof| introduced
\item
|\childdoc| renamed to |\childdocmain|
\item
|\childredirect| renamed to |\childdocforward| and |\childdocforwardprefix|
and functionality expanded
\end{itemize}

%%%%%%%%%%%%%%%%%%%%%%%%%%%%%%%%%%%%%%%%
\paragraph{v1.0:} 2017/04/27

\begin{itemize}
\item
manual and install package
\item
first version published on CTAN
\end{itemize}

%%%%%%%%%%%%%%%%%%%%%%%%%%%%%%%%%%%%%%%%
\paragraph{v0.6:} 2017/04/26

\begin{itemize}
\item
redirection mechanism added
\end{itemize}

%%%%%%%%%%%%%%%%%%%%%%%%%%%%%%%%%%%%%%%%
\paragraph{v0.5:} 2017/04/26

\begin{itemize}
\item
functionality in definition file
\end{itemize}


%%%%%%%%%%%%%%%%%%%%%%%%%%%%%%%%%%%%%%%%%%%%%%%%%%%%%%%%%%%%%%%%%%%%%%%%%%%%%%%%
%%%%%%%%%%%%%%%%%%%%%%%%%%%%%%%%%%%%%%%%%%%%%%%%%%%%%%%%%%%%%%%%%%%%%%%%%%%%%%%%
%%%%%%%%%%%%%%%%%%%%%%%%%%%%%%%%%%%%%%%%%%%%%%%%%%%%%%%%%%%%%%%%%%%%%%%%%%%%%%%%
\appendix

\settowidth\MacroIndent{\rmfamily\scriptsize 000\ }

 \DocInput{childdoc.dtx}

\end{document}
%</driver>
% \fi
%
% %%%%%%%%%%%%%%%%%%%%%%%%%%%%%%%%%%%%%%%%%%%%%%%%%%%%%%%%%%%%%%%%%%%%%%%%%%%%%%
% %%%%%%%%%%%%%%%%%%%%%%%%%%%%%%%%%%%%%%%%%%%%%%%%%%%%%%%%%%%%%%%%%%%%%%%%%%%%%%
% \section{Sample}
%\iffalse
%<*samplemain>
%\fi
%
% The following presents a sample document
% with two chapters, two parts, a title page,
% a compile flag as well as three forwarding files to set the flag.
% It consists of eight |.tex| files:
% \begin{center}
% \begin{tabular}{ll}
% |cdocsamp.tex|&main file\\
% |cdocsch1.tex|&include file for chapter 1\\
% |cdocsch2.tex|&include file for chapter 2\\
% |cdocspt3.tex|&include file for part 3\\
% |cdocspt4.tex|&include file for part 4\\
% |cdocsdrf.tex|&forwarding file for main file in draft mode\\
% |cdocsfi1.tex|&forwarding file for final version of chapter 1\\
% |cdocsfi2.tex|&forwarding file for final version of chapter 2\\
% \end{tabular}
% \end{center}
% Each of the eight files can be compiled directly by the \LaTeX{} compiler.
%
% %%%%%%%%%%%%%%%%%%%%%%%%%%%%%%%%%%%%%%
% \paragraph{Main File.}
%
% The main file is called |cdocsamp.tex|.
%
% Load the \textsf{childdoc} definitions and
% declare the filename for the main document:
%    \begin{macrocode}
% \iffalse
%
% childdoc.dtx Copyright (C) 2017-2018 Niklas Beisert
%
% This work may be distributed and/or modified under the
% conditions of the LaTeX Project Public License, either version 1.3
% of this license or (at your option) any later version.
% The latest version of this license is in
%   http://www.latex-project.org/lppl.txt
% and version 1.3 or later is part of all distributions of LaTeX
% version 2005/12/01 or later.
%
% This work has the LPPL maintenance status `maintained'.
%
% The Current Maintainer of this work is Niklas Beisert.
%
% This work consists of the files childdoc.dtx and childdoc.ins
% and the derived files childdoc.def and cdocsamp.tex with
% cdocsch1.tex, cdocsch2.tex, cdocsdrf.tex, cdocsfn1.tex, cdocsfn2.tex.
%
%<package>\ifdefined\childdocmain\endinput\fi
%<package>\ProvidesFile{childdoc.def}[2018/12/30 v2.0 child document driver]
%<samplemain>\ProvidesFile{cdocsamp.tex}[2018/12/30 v2.0 sample for childdoc]
%<*driver>
%\ProvidesFile{childdoc.drv}[2018/12/30 v2.0 childdoc reference manual file]
\PassOptionsToClass{10pt,a4paper}{article}
\documentclass{ltxdoc}

\usepackage[margin=35mm]{geometry}
\usepackage{hyperref}
\usepackage{hyperxmp}
\usepackage[usenames]{color}

\hypersetup{colorlinks=true}
\hypersetup{pdfstartview=FitH}
\hypersetup{pdfpagemode=UseNone}
\hypersetup{pdfsource={}}
\hypersetup{pdflang={en-UK}}
\hypersetup{pdfcopyright={Copyright 2017-2018 Niklas Beisert.
  This work may be distributed and/or modified under the
  conditions of the LaTeX Project Public License, either version 1.3
  of this license or (at your option) any later version.}}
\hypersetup{pdflicenseurl={http://www.latex-project.org/lppl.txt}}
\hypersetup{pdfcontactaddress={ETH Zurich, ITP, HIT K,
  Wolfgang-Pauli-Strasse 27}}
\hypersetup{pdfcontactpostcode={8093}}
\hypersetup{pdfcontactcity={Zurich}}
\hypersetup{pdfcontactcountry={Switzerland}}
\hypersetup{pdfcontactemail={nbeisert@itp.phys.ethz.ch}}
\hypersetup{pdfcontacturl={http://people.phys.ethz.ch/\xmptilde nbeisert/}}

\newcommand{\secref}[1]{\hyperref[#1]{section \ref*{#1}}}

\parskip1ex
\parindent0pt
\let\olditemize\itemize
\def\itemize{\olditemize\parskip0pt}

\begin{document}

\title{The \textsf{childdoc} Package}
\hypersetup{pdftitle={The childdoc Package}}
\author{Niklas Beisert\\[2ex]
  Institut f\"ur Theoretische Physik\\
  Eidgen\"ossische Technische Hochschule Z\"urich\\
  Wolfgang-Pauli-Strasse 27, 8093 Z\"urich, Switzerland\\[1ex]
  \href{mailto:nbeisert@itp.phys.ethz.ch}
  {\texttt{nbeisert@itp.phys.ethz.ch}}}
\hypersetup{pdfauthor={Niklas Beisert}}
\hypersetup{pdfsubject={Manual for the LaTeX2e Package childdoc}}
\date{30 December 2018, \textsf{v2.0}}
\maketitle

\begin{abstract}\noindent
\textsf{childdoc} is a \LaTeXe{} package
that enables the direct compilation
of document sections included by |\include|
to individual files.
\end{abstract}

\begingroup
\parskip0ex
\tableofcontents
\endgroup

%%%%%%%%%%%%%%%%%%%%%%%%%%%%%%%%%%%%%%%%%%%%%%%%%%%%%%%%%%%%%%%%%%%%%%%%%%%%%%%%
%%%%%%%%%%%%%%%%%%%%%%%%%%%%%%%%%%%%%%%%%%%%%%%%%%%%%%%%%%%%%%%%%%%%%%%%%%%%%%%%
\section{Introduction}

\LaTeX{} provides a mechanism to structure a large document (such as a book)
into a main file and several child files (containing the chapters)
using the |\include| command.
This mechanism is beneficial for documents
which span hundreds of pages in order to
make the source file(s) more manageable.
Moreover, compilation can be restricted to
selected child files by means of the |\includeonly| command.
The latter feature can be used to reduce the compilation time while editing
(this was significantly more useful in the earlier days of \LaTeX{})
or to generate a smaller document which is easier to navigate.
Another application of |\includeonly| is to generate
documents consisting of selected parts of the complete document.

However, there are a few drawbacks of the plain |\include| mechanism:
\begin{itemize}
\item
The child files cannot be compiled on their own,
they can only be compiled via the main file.
A naive editing environment
(such as a text editor with an option
to have the current file processed by \LaTeX)
may require one to switch to the main file before compiling;
attempting to compile the child file produces errors.
\item
The main file must be modified (each time)
to adjust the |\includeonly| command
to the present needs. This easily leaves the main file in a messy state.
\item
The generated document will always carry the filename
of the main document. This is inconvenient if
several child files are to be compiled and
to be kept for distribution.
\end{itemize}

The present package provides a simple interface
to make child files individually compilable by \LaTeX{}.
Compiling a child file then has the same effect as compiling
the main file with an |\includeonly| command
to select the appropriate child.
Moreover the generated document will carry the name of the child
rather than the main file.
This resolves all three above issues.

This feature is meant to make the editing of books,
thesis documents and lecture notes somewhat more convenient.
However, the package can also be used efficiently for
composing a series of documents (such as exercise sheets)
which are typically distributed individually.
It then assists the author in generating the individual documents
(potentially in different versions)
as well as a document containing the collected series.
Another application is in developing style files
or other kinds of included material
where compilation of the style file could redirect
to a sample or test file.

%%%%%%%%%%%%%%%%%%%%%%%%%%%%%%%%%%%%%%%%%%%%%%%%%%%%%%%%%%%%%%%%%%%%%%%%%%%%%%%%
%%%%%%%%%%%%%%%%%%%%%%%%%%%%%%%%%%%%%%%%%%%%%%%%%%%%%%%%%%%%%%%%%%%%%%%%%%%%%%%%
\section{Usage}

First of all, the package \textsf{childdoc} is \emph{not} a standard
\LaTeXe{} |.sty| style file! Therefore it needs to be invoked in
a non-standard way.

%%%%%%%%%%%%%%%%%%%%%%%%%%%%%%%%%%%%%%%%%%%%%%%%%%%%%%%%%%%%%%%%%%%%%%%%%%%%%%%%
\subsection{Included Files}
\label{sec:include}

%%%%%%%%%%%%%%%%%%%%%%%%%%%%%%%%%%%%%%%%
\DescribeMacro{\childdocmain}
To use the package, add the commands
\begin{center}
\begin{tabular}{l}
|\input{childdoc.def}|\\
|\childdocmain{}|\\
\end{tabular}
\end{center}
at the very top of the main \LaTeX{} file,
in particular \emph{before} the |\documentclass| statement!
The argument of |\childdocmain| should be left empty
(but it must be present).

%%%%%%%%%%%%%%%%%%%%%%%%%%%%%%%%%%%%%%%%
\DescribeMacro{\childdocof}
Furthermore, add the commands
\begin{center}
\begin{tabular}{l}
|\input{childdoc.def}|\\
|\childdocof{|\textit{main}|}|\\
\end{tabular}
\end{center}
at the top of every child file \textit{child}
which is included by |\include{|\textit{child}|}|
from within the main file
(or at least for those files to be compiled individually).
The argument \textit{main} must be the filename of the main file.

There are a couple of
considerations in setting up the main and child documents:

%%%%%%%%%%%%%%%%%%%%%%%%%%%%%%%%%%%%%%%%
\paragraph{Restrictions.}

Please note the following restrictions:
\begin{itemize}
\item
|\childdocmain| must be called with one argument \textit{main}
to ensure compatibility with earlier version of the package.
It must either be empty (|\childdocmain{}|)
or precisely match the filename of the main file in which it is specified.
See \secref{sec:detection} for further information.
\item
The filename \textit{main} must be specified without the |.tex| extension.
\item
The filename \textit{main} is case sensitive
(even in case-insensitive file systems)
due to internal string comparison.
\item
The argument \textit{main} should be fully expanded, it cannot be a macro.
\item
Subdirectories and special characters should be avoided in filenames.
\item
The command |\childdocmain{|\textit{main}|}| must be followed by a whitespace.
It should not be followed immediately by another command
or by a comment mark `|%|'.
This is because the \TeX{} parser reads the token immediately following
the argument of |\childdocmain| and puts it
at the beginning of every child section;
however, a white\-space is ignored.
\end{itemize}

%%%%%%%%%%%%%%%%%%%%%%%%%%%%%%%%%%%%%%%%
\paragraph{Content of Main File.}

It is advisable to place all content in the child files included by |\include|.
Any output contained in the main file will appear in all child documents
unless suppressed manually;
it cannot be suppressed automatically by the |\includeonly| directive
and thus should normally be avoided.
A method to include some content in the main file
by means of conditional processing is described in \secref{sec:conditional}.

%%%%%%%%%%%%%%%%%%%%%%%%%%%%%%%%%%%%%%%%
\paragraph{Page Numbering.}

When only a part of the document is compiled,
the appropriate numbering of pages
(as well as other status parameters)
is determined from the |.aux| files.
The latter contain information from previous passes.
However this information needs to propagate through
all intermediate child documents.
Therefore the page numbering in child documents may well
be inconsistent until the complete document is compiled at least once.

A useful (if unconventional) way to always ensure a consistent
page numbering is to restart the numbering in each child document
and denote the pages by `\textit{child}|.|\textit{page}'
where \textit{child} represents the chapter/section number of the child file.
This can be achieved by the command
|\numberwithin{page}{|\textit{child}|}|
of the \textsf{amsmath} package
where \textit{child} can be |chapter| or |section|
depending on the chosen structuring.
Alternatively, one can modify the macro |\thepage| appropriately
and reset the counter |page| at the start of each child file.

%%%%%%%%%%%%%%%%%%%%%%%%%%%%%%%%%%%%%%%%%%%%%%%%%%%%%%%%%%%%%%%%%%%%%%%%%%%%%%%%
\subsection{Conditional Processing}
\label{sec:conditional}

The package provides a mechanism to compile different versions
of a document. To customise the versions further some conditional processing
can come in handy to distinguish which version is being compiled.
The package provides two macros to describe the compilation context:

%%%%%%%%%%%%%%%%%%%%%%%%%%%%%%%%%%%%%%%%
\DescribeMacro{\ifchilddoc}
The conditional |\ifchilddoc| distinguishes between the compilation of
child documents and the main document:
%
\begin{center}
|\ifchilddoc |\textit{child-code}| |[|\||else |\textit{main-code}]| \||fi|
\end{center}

%%%%%%%%%%%%%%%%%%%%%%%%%%%%%%%%%%%%%%%%
\DescribeMacro{\childdocname}
\DescribeMacro{\childdocjob}
The macro |\childdocname| contains the filename (without extension)
of the main or child file being processed.
Note that |\childdocjob| will always contain the name of the main file.

%%%%%%%%%%%%%%%%%%%%%%%%%%%%%%%%%%%%%%%%
\paragraph{Title Page.}

Conditional processing can be used to include a title or banner page
in the main document when proper precautions are taken.
Importantly, the code in the main file should ensure that the page counter
(as well as other status parameters which are stored in the |.aux| files)
takes the same value after the conditional processing.
Otherwise the page numbers may take divergent values
depending on which part is compiled.

For example, a title page could be declared by:
%
\begin{center}
\begin{tabular}{l}
|\ifchilddoc\||else|\\
|\addtocounter{page}{-1}|\\
\textit{code for title page}\\
|\newpage|\\
|\||fi|
\end{tabular}
\end{center}
%
A banner page for the child documents can be generated by:
%
\begin{center}
\begin{tabular}{l}
|\ifchilddoc|\\
|\addtocounter{page}{-1}|\\
\textit{code for banner page}\\
|\newpage|\\
|\||fi|
\end{tabular}
\end{center}
%
Here one could write a message such as:
\begin{center}
|This is the part \childdocname{} of \childdocjob{}.|
\end{center}

%%%%%%%%%%%%%%%%%%%%%%%%%%%%%%%%%%%%%%%%%%%%%%%%%%%%%%%%%%%%%%%%%%%%%%%%%%%%%%%%
\subsection{Flags}
\label{sec:flags}

The package makes it easy to generate different versions
of the main or child documents.
To this end compilation flags can be defined
and assigned different default values.
They will be particularly useful in conjunction
with the forwarding mechanism described in \secref{sec:forward}.

For example, it may be useful to have a flag |\version|
which can be set to |draft| or |final|.
The document source will contain some conditional code
depending on the value of |\version|.
Suppose further, the flag should default to |final| for the main file
and to |draft| for child files
which is a natural assignment for editing the document.
This is achieved by placing the following code
in the preamble of the main document
(below the |\childdocmain| directive):
%
\begin{center}
\begin{tabular}{l}
|\ifchilddoc|\\
|\providecommand{\version}{draft}|\\
|\||else|\\
|\providecommand{\version}{final}|\\
|\||fi|
\end{tabular}
\end{center}
%
The definition by |\providecommand| makes sure
that previous definitions are not overwritten.
Further statements |\providecommand{\version}{...}|
can thus be added before the above code to override it.

For the main file, one might add a line
(between |\childdocmain| and the above block)
%
\begin{center}
|%\ifchilddoc\||else\providecommand{\version}{draft}\||fi|
\end{center}
%
which can be uncommented to produce a draft version.
Likewise one can add a line to the very top of a child file
(above the |\childdocof{|\textit{main}|}| directive)
%
\begin{center}
|%\providecommand{\version}{final}|
\end{center}
%
which can be uncommented to produce the final version of this child document.

%%%%%%%%%%%%%%%%%%%%%%%%%%%%%%%%%%%%%%%%%%%%%%%%%%%%%%%%%%%%%%%%%%%%%%%%%%%%%%%%
\subsection{Forwarding}
\label{sec:forward}

Different versions of the main or child documents
using compilation flags as described in \secref{sec:flags}
can be (permanently) stored in different files
for convenient compilation, viewing and distribution.
To this end, the package defines a command
to pass on compilation to a different file:

%%%%%%%%%%%%%%%%%%%%%%%%%%%%%%%%%%%%%%%%
\DescribeMacro{\childdocforward}
The command |\childdocforward| redirects processing to
another source file:
%
\begin{center}
\begin{tabular}{l}
|\input{childdoc.def}|\\
|\childdocforward[|\textit{main}|]{|\textit{dest}|}|\\
\end{tabular}
\end{center}
%
The argument \textit{dest} is the destination file
(without extension).
It should be the main file or one of the child files.
Note that further \textsf{childdoc} directives
such as |\childdocof| and |\childdocforward|
in the indicated file will be processed in this form.
The optional argument \textit{main}
passes on directly to the main file \textit{main}
while pretending to compile the child \textit{dest}.
This form behaves as if \textit{dest}
issues |\childdocof{|\textit{main}|}| right away,
and no further \textsf{childdoc} directives will be processed.

%%%%%%%%%%%%%%%%%%%%%%%%%%%%%%%%%%%%%%%%
\DescribeMacro{\...prefix}
In the alternative form |\childdocforwardprefix|,
%
\begin{center}
\begin{tabular}{l}
|\input{childdoc.def}|\\
|\childdocforwardprefix[|\textit{main}|]{|\textit{prefix}|}{|\textit{dest}|}|
\end{tabular}
\end{center}
%
the destination file is determined by a pattern
depending on the current file:
To make this work, the current file must be called
`{\textit{prefix}\hspace{0.2em}\textit{suffix}}'
with \textit{prefix} matching precisely the argument.
Processing is then passed on to the file
`{\textit{dest}\hspace{0.2em}\textit{suffix}}'.
Surely, the same effect is achieved by
directly specifying the
argument `{\textit{dest}\hspace{0.2em}\textit{suffix}}'
in the first form.
However, that requires to set up a different file
for each child. With the alternative form of the command
all these files can have exactly the same content
which simplifies setting them up and maintaining them.

For example, the following file |draft.tex|
with a compilation flag |\version| as described in \secref{sec:flags}
compiles the main document as a draft:
%
\begin{center}
\begin{tabular}{l}
|\def\version{draft}|\\
|\input{childdoc.def}|\\
|\childdocforward{|\textit{main}|}|
\end{tabular}
\end{center}
%
Likewise, the following files |final|\textit{nn}|.tex|
compile the final version of the child document
|child|\textit{nn}|.tex|:
%
\begin{center}
\begin{tabular}{l}
|\def\version{final}|\\
|\input{childdoc.def}|\\
|\childdocforwardprefix{final}{child}|
\end{tabular}
\end{center}
%

Note that when several versions of a main file and/or of each child file
are to be generated, it may be convenient to set up a |Makefile| or
shell script to automatise the process.

%%%%%%%%%%%%%%%%%%%%%%%%%%%%%%%%%%%%%%%%%%%%%%%%%%%%%%%%%%%%%%%%%%%%%%%%%%%%%%%%
\subsection{Command Line Processing}
\label{sec:commandline}

The effect of redirection files can also be achieved by invoking
the \LaTeX{} compiler with a more elaborate command line.
Most conveniently this should be done as part
of a shell script or a |Makefile|.

When using \textsf{childdoc} in the main file, the following
command lines effectively perform a redirection
(note that depending on the shell being used,
backslashes may have to be doubled: `|\|' $\to$ `|\\|'):
%
\begin{center}
|... -jobname "|\textit{target}|" |\\|"|[\textit{flags}]%
|\input{childdoc.def}\childdocforward[|\textit{main}|]{|\textit{dest}|}"|
\end{center}
%
Here \textit{target} is the name of the output file,
\textit{main} is the name of the main file
and \textit{dest} is the name of the main or child file to be processed
(all filenames without extensions).
The optional argument \textit{main} can be omitted
if \textit{main} matches \textit{dest}.
Optionally, compilation \textit{flags} can be defined via |\def| commands.
This command line makes the \TeX{} engine believe
it is compiling the file \textit{target}
whose content is specified as the latter parameter.
The provided code then forwards the processing to
\textit{main} or \textit{dest} as described in \secref{sec:forward}.

%%%%%%%%%%%%%%%%%%%%%%%%%%%%%%%%%%%%%%%%%%%%%%%%%%%%%%%%%%%%%%%%%%%%%%%%%%%%%%%%
\subsection{Include by Input}
\label{sec:input}

Including child documents by |\include| has some restrictions by design.
Most notably, the content of a child document always occupies
its own set of pages; pages cannot be shared between child documents.
Usually, this behaviour makes perfect sense
because each child document contain an essential part of the document.
However, in some situations it may be desirable to compose
a document from a collection of parts
without having mandatory page breaks between then.
For this case, the package
provides a mechanism to include parts
by |\input| which can also be processed individually.
However, by construction this mechanism
requires manual handling of the content to be output.

%%%%%%%%%%%%%%%%%%%%%%%%%%%%%%%%%%%%%%%%
\DescribeMacro{\ifchilddocmanual}
The main file should be prepared as usual, see \secref{sec:include}.
However, the document body must make a distinction
between processing of an individual part and of the main document, e.g.:
%
\begin{center}
\begin{tabular}{l}
|\ifchilddocmanual|\\
|\input{\childdocname}|\\
|\||else|\\
\textit{document body with }|\input{|\textit{part}|}|\\
|\||fi|
\end{tabular}
\end{center}
%
The conditional |\ifchilddocmanual| is true whenever
a part to be included by |\input| is being compiled,
and the name of the part is stored in |\childdocname|.

%%%%%%%%%%%%%%%%%%%%%%%%%%%%%%%%%%%%%%%%
\DescribeMacro{\childdocby}
Each part to be included by |\input| should start with:
%
\begin{center}
\begin{tabular}{l}
|\input{childdoc.def}|\\
|\childdocby{|\textit{main}|}|\\
\end{tabular}
\end{center}
%
The directive |\childdocby| is similar to |\childdocof|
described in \secref{sec:include},
but the subsequent selection of content must be done manually.
To that end, both |\ifchilddoc| and |\ifchilddocmanual|
will be true upon processing of a part,
and the name of the part is stored in |\childdocname|.
Note that |\jobname| will be set to the filename of the current part
so that each part receives an individual |.aux| file
that does not interfere with the |.aux| file(s) of the main document.
This behaviour can be altered by the alternative form
|\childdocby[*]{|\textit{main}|}| (with a non-empty optional argument)
which uses the |.aux| file of the main document
by setting |\jobname| to \textit{main}.

%%%%%%%%%%%%%%%%%%%%%%%%%%%%%%%%%%%%%%%%%%%%%%%%%%%%%%%%%%%%%%%%%%%%%%%%%%%%%%%%
\subsection{Driver Development}
\label{sec:driver}

The \textsf{childdoc} mechanism can also be use for the development
of definition files such as \LaTeX{} styles or classes.
This case differs from the above setup with multiple parts
included by |\include| in that no |\includeonly| should be invoked.
This can be achieved by starting the include file
(before |\ProvidesPackage|) with:
%
\begin{center}
\begin{tabular}{l}
|\input{childdoc.def}|\\
|\childdocforward{|\textit{main}|}|\\
\end{tabular}
\end{center}
%
or alternatively with:
%
\begin{center}
\begin{tabular}{l}
|\input{childdoc.def}|\\
|\childdocby{|\textit{main}|}|\\
\end{tabular}
\end{center}
%
Both forms have slightly different effects as described above.
The main file is prepared as usual, see \secref{sec:include}.

%%%%%%%%%%%%%%%%%%%%%%%%%%%%%%%%%%%%%%%%%%%%%%%%%%%%%%%%%%%%%%%%%%%%%%%%%%%%%%%%
\subsection{Legacy Detection}
\label{sec:detection}

The directive |\childdocmain| in the main file can detect
whether the complete document or merely a child is to be compiled
even without using the directive |\childdocof|.
This method is deprecated because it is less robust
and there is no compelling reason to use it;
it is merely provided for backward compatibility
and it may be removed in future versions.

If the detection mechanism is to be used,
it is mandatory to correctly specify
the filename of the main file as the argument of |\childdocmain|:
%
\begin{center}
\begin{tabular}{l}
|\input{childdoc.def}|\\
|\childdocmain{|\textit{main}|}|\\
\end{tabular}
\end{center}
%
If |\jobname| does not match the argument \textit{main} of |\childdocmain|,
it is assumed that |\jobname| points to the child file to be compiled.
When using |\childdocmain| with the main file specified as argument,
it suffices to start a child file
with just |\input{|\textit{main}|}|
without loading of the package and using |\childdocof|.
If instead all processing is done
with the appropriate \textsf{childdoc} directives,
the argument of \textit{main} of |\childdocmain| can be empty.

An alternative version of the command line processing described
in \secref{sec:commandline} using the detection mechanism reads:
%
\begin{center}
|... -jobname "|\textit{target}|" "|[\textit{flags}]%
[|\def\jobname{|\textit{dest}|}|]|\input{|\textit{main}|}"|
\end{center}

%%%%%%%%%%%%%%%%%%%%%%%%%%%%%%%%%%%%%%%%%%%%%%%%%%%%%%%%%%%%%%%%%%%%%%%%%%%%%%%%
\subsection{Manual Code}
\label{sec:manual}

In case one cannot be certain whether the definitions file |childdoc.def|
is installed on the target \TeX{} distribution
and one prefers not to ship it,
it is conceivable to paste a few relevant commands into the sources.

To that end, drop all statements |\input{childdoc.def}|
and perform the replacements as outlined below.
Instead of |\childdocmain{|\textit{main}|}| add the following code
to the top of the main file:
%
\begin{center}
\begin{tabular}{l}
|\||ifdefined\childdocname\endinput\||fi\newif\ifchilddoc|\\
|\edef\childdocname{\scantokens\expandafter{\jobname\noexpand}}|\\
|\def\childdocmain{|\textit{main}|}\||ifx\childdocmain\childdocname\||else|\\
|\childdoctrue\includeonly{\childdocname}\let\jobname\childdocmain\||fi|\\
\end{tabular}
\end{center}
%
Instead of |\childdocof{|\textit{main}|}| just include the main file
at the top of each child file:
%
\begin{center}
|\input{|\textit{main}|}|
\end{center}
%
A simple redirection |\childdocforward{|\textit{dest}|}| is achieved by:
%
\begin{center}
|\def\jobname{|\textit{dest}|}\input{\jobname}|
\end{center}
%
The redirection with prefix
|\childdocforwardprefix[|\textit{prefix}|]{|\textit{dest}|}|
is accomplished by:
%
\begin{center}
\begin{tabular}{l}
|{\edef\jobname{\scantokens\expandafter{\jobname\noexpand}}|\\
|\def\redirectjob |\textit{prefix}|#1~~~{\gdef\jobname{|\textit{dest}|#1}}|\\
|\expandafter\redirectjob\jobname~~~}\input{\jobname}|
\end{tabular}
\end{center}

In an alternative approach,
child documents can be compiled by a specific command line
without additional code or specific definitions:
%
\begin{center}
|... -jobname "|\textit{target}|" "|[\textit{flags}]%
|\includeonly{|\textit{dest}|}\input{|\textit{main}|}"|
\end{center}
%

%%%%%%%%%%%%%%%%%%%%%%%%%%%%%%%%%%%%%%%%%%%%%%%%%%%%%%%%%%%%%%%%%%%%%%%%%%%%%%%%
%%%%%%%%%%%%%%%%%%%%%%%%%%%%%%%%%%%%%%%%%%%%%%%%%%%%%%%%%%%%%%%%%%%%%%%%%%%%%%%%
\section{Information}

%%%%%%%%%%%%%%%%%%%%%%%%%%%%%%%%%%%%%%%%%%%%%%%%%%%%%%%%%%%%%%%%%%%%%%%%%%%%%%%%
\subsection{Copyright}

Copyright \copyright{} 2017--2018 Niklas Beisert

This work may be distributed and/or modified under the
conditions of the \LaTeX{} Project Public License, either version 1.3
of this license or (at your option) any later version.
The latest version of this license is in
  \url{http://www.latex-project.org/lppl.txt}
and version 1.3 or later is part of all distributions of \LaTeX{}
version 2005/12/01 or later.

This work has the LPPL maintenance status `maintained'.

The Current Maintainer of this work is Niklas Beisert.

This work consists of the files |README.txt|, |childdoc.ins| and |childdoc.dtx|
as well as the derived files |childdoc.def|, |cdocsamp.tex|
with |cdocsch1.tex|, |cdocsch2.tex|, |cdocspt3.tex|, |cdocspt4.tex|,
|cdocsdrf.tex|, |cdocsfn1.tex|, |cdocsfn2.tex|
as well as |childdoc.pdf|.

%%%%%%%%%%%%%%%%%%%%%%%%%%%%%%%%%%%%%%%%%%%%%%%%%%%%%%%%%%%%%%%%%%%%%%%%%%%%%%%%
\subsection{Files and Installation}

The package consists of the files:
%
\begin{center}
\begin{tabular}{ll}
    |README.txt|   & readme file \\
    |childdoc.ins| & installation file \\
    |childdoc.dtx| & source file \\
    |childdoc.def| & definition file \\
    |cdocsamp.tex| & sample main file \\
    |cdocsch1.tex| & sample include file \\
    |cdocsch2.tex| & sample include file \\
    |cdocspt3.tex| & sample part file \\
    |cdocspt4.tex| & sample part file \\
    |cdocsdrf.tex| & sample redirection file \\
    |cdocsfn1.tex| & sample redirection file \\
    |cdocsfn2.tex| & sample redirection file \\
    |childdoc.pdf| & manual
\end{tabular}
\end{center}
%
The distribution consists of the files
|README.txt|, |childdoc.ins| and |childdoc.dtx|.
%
\begin{itemize}
\item
Run (pdf)\LaTeX{} on |childdoc.dtx|
to compile the manual |childdoc.pdf| (this file).
\item
Run \LaTeX{} on |childdoc.ins| to create the definitions file |childdoc.def|
and the sample |cdocsamp.tex| with include files
|cdocsch1.tex|, |cdocsch2.tex|, |cdocspt3.tex|, |cdocspt4.tex|,
|cdocsdrf.tex|, |cdocsfn1.tex|, |cdocsfn2.tex|.
Then copy the file |childdoc.def| to an appropriate directory of your \LaTeX{}
distribution, e.g.\ \textit{texmf-root}|/tex/latex/childdoc|.
\end{itemize}

%%%%%%%%%%%%%%%%%%%%%%%%%%%%%%%%%%%%%%%%%%%%%%%%%%%%%%%%%%%%%%%%%%%%%%%%%%%%%%%%
\subsection{Related CTAN Packages}

There are several other packages which offer a similar functionality:
%
\begin{itemize}
\item
The packages
\href{http://ctan.org/pkg/docmute}{\textsf{docmute}},
\href{http://ctan.org/pkg/includex}{\textsf{includex}} and
\href{http://ctan.org/pkg/standalone}{\textsf{standalone}}
provide commands to include only the document body of
a child file thus allowing both files to be compiled individually.
\item
The packages \href{http://ctan.org/pkg/subdocs}{\textsf{subdocs}}
and \href{http://ctan.org/pkg/subfiles}{\textsf{subfiles}}
provide structures in which the main and child documents can be
encapsulated and allowing them to be compiled individually.
The inclusion mechanism is different from the conventional |\include|.
\item
The package \href{http://ctan.org/pkg/combine}{\textsf{combine}}
is an elaborate solution to combine several documents into one.
\end{itemize}
%
See also the CTAN topic \href{http://ctan.org/topic/subdocs}{\textsf{subdocs}}
for further related packages.
The present package differs from the above solutions in that
a document structure constructed with the conventional |\include| mechanism
just needs two extra commands at the top of every file
such that all constituent files can be compiled individually.

%%%%%%%%%%%%%%%%%%%%%%%%%%%%%%%%%%%%%%%%%%%%%%%%%%%%%%%%%%%%%%%%%%%%%%%%%%%%%%%%
%\subsection{Feature Suggestions}
%
%The following is a list of features which may be useful for future
%versions of this package:
%%
%\begin{itemize}
%\item
%\ldots
%\end{itemize}

%%%%%%%%%%%%%%%%%%%%%%%%%%%%%%%%%%%%%%%%%%%%%%%%%%%%%%%%%%%%%%%%%%%%%%%%%%%%%%%%
\subsection{Revision History}

%%%%%%%%%%%%%%%%%%%%%%%%%%%%%%%%%%%%%%%%
\paragraph{v2.0:} 2018/12/30

\begin{itemize}
\item
immediate forward processing
\item
added |\childdocby| mechanism
\item
manual restructured
\end{itemize}

%%%%%%%%%%%%%%%%%%%%%%%%%%%%%%%%%%%%%%%%
\paragraph{v1.6:} 2018/01/17

\begin{itemize}
\item
application for development of include files
\item
corrections to manual
\end{itemize}

%%%%%%%%%%%%%%%%%%%%%%%%%%%%%%%%%%%%%%%%
\paragraph{v1.5:} 2017/05/21

\begin{itemize}
\item
more complete structuring introduced
\item
|\childdocof| introduced
\item
|\childdoc| renamed to |\childdocmain|
\item
|\childredirect| renamed to |\childdocforward| and |\childdocforwardprefix|
and functionality expanded
\end{itemize}

%%%%%%%%%%%%%%%%%%%%%%%%%%%%%%%%%%%%%%%%
\paragraph{v1.0:} 2017/04/27

\begin{itemize}
\item
manual and install package
\item
first version published on CTAN
\end{itemize}

%%%%%%%%%%%%%%%%%%%%%%%%%%%%%%%%%%%%%%%%
\paragraph{v0.6:} 2017/04/26

\begin{itemize}
\item
redirection mechanism added
\end{itemize}

%%%%%%%%%%%%%%%%%%%%%%%%%%%%%%%%%%%%%%%%
\paragraph{v0.5:} 2017/04/26

\begin{itemize}
\item
functionality in definition file
\end{itemize}


%%%%%%%%%%%%%%%%%%%%%%%%%%%%%%%%%%%%%%%%%%%%%%%%%%%%%%%%%%%%%%%%%%%%%%%%%%%%%%%%
%%%%%%%%%%%%%%%%%%%%%%%%%%%%%%%%%%%%%%%%%%%%%%%%%%%%%%%%%%%%%%%%%%%%%%%%%%%%%%%%
%%%%%%%%%%%%%%%%%%%%%%%%%%%%%%%%%%%%%%%%%%%%%%%%%%%%%%%%%%%%%%%%%%%%%%%%%%%%%%%%
\appendix

\settowidth\MacroIndent{\rmfamily\scriptsize 000\ }

 \DocInput{childdoc.dtx}

\end{document}
%</driver>
% \fi
%
% %%%%%%%%%%%%%%%%%%%%%%%%%%%%%%%%%%%%%%%%%%%%%%%%%%%%%%%%%%%%%%%%%%%%%%%%%%%%%%
% %%%%%%%%%%%%%%%%%%%%%%%%%%%%%%%%%%%%%%%%%%%%%%%%%%%%%%%%%%%%%%%%%%%%%%%%%%%%%%
% \section{Sample}
%\iffalse
%<*samplemain>
%\fi
%
% The following presents a sample document
% with two chapters, two parts, a title page,
% a compile flag as well as three forwarding files to set the flag.
% It consists of eight |.tex| files:
% \begin{center}
% \begin{tabular}{ll}
% |cdocsamp.tex|&main file\\
% |cdocsch1.tex|&include file for chapter 1\\
% |cdocsch2.tex|&include file for chapter 2\\
% |cdocspt3.tex|&include file for part 3\\
% |cdocspt4.tex|&include file for part 4\\
% |cdocsdrf.tex|&forwarding file for main file in draft mode\\
% |cdocsfi1.tex|&forwarding file for final version of chapter 1\\
% |cdocsfi2.tex|&forwarding file for final version of chapter 2\\
% \end{tabular}
% \end{center}
% Each of the eight files can be compiled directly by the \LaTeX{} compiler.
%
% %%%%%%%%%%%%%%%%%%%%%%%%%%%%%%%%%%%%%%
% \paragraph{Main File.}
%
% The main file is called |cdocsamp.tex|.
%
% Load the \textsf{childdoc} definitions and
% declare the filename for the main document:
%    \begin{macrocode}
\input{childdoc.def}
\childdocmain{}
%    \end{macrocode}

% Optional override for |\version| flag:
%    \begin{macrocode}
%%\ifchilddoc\else\providecommand{\version}{draft}\fi
%    \end{macrocode}

% Define the default values for the |\version| flag
% (|final| for the main file and |draft| for childs):
%    \begin{macrocode}
\ifchilddoc
\providecommand{\version}{draft}
\else
\providecommand{\version}{final}
\fi
%    \end{macrocode}

% Load the standard document class:
%    \begin{macrocode}
\documentclass[12pt]{article}
%    \end{macrocode}

% Start the document body:
%    \begin{macrocode}
\begin{document}
%    \end{macrocode}

% Declare a title page.
% Print title, part of document being processed and version flag:
%    \begin{macrocode}
\addtocounter{page}{-1}
\begin{center}
{\LARGE\bfseries{}childdoc example\par}
\vspace{1cm}
\ifchilddoc
\ifchilddocmanual part\else chapter\fi:
`\childdocname' of `\childdocjob'\par
\else
main document: `\childdocjob'\par
\fi
version: \version\par
\end{center}
\newpage
%    \end{macrocode}

% Manually include selected file,
% otherwise process as usual:
%    \begin{macrocode}
\ifchilddocmanual
\section*{part `\childdocname'}
\input{\childdocname}
\else
%    \end{macrocode}

% Include the two chapters:
%    \begin{macrocode}
\include{cdocsch1}
\include{cdocsch2}
%    \end{macrocode}

% Include the two parts unless only chapters should be displayed:
%    \begin{macrocode}
\ifchilddoc\else
\section{part three}
\input{cdocspt3}
\section{part four}
\input{cdocspt4}
\fi
%    \end{macrocode}

% Process as usual until here:
%    \begin{macrocode}
\fi
%    \end{macrocode}

% End of document body:
%    \begin{macrocode}
\end{document}
%    \end{macrocode}
%\iffalse
%</samplemain>
%\fi
%
% %%%%%%%%%%%%%%%%%%%%%%%%%%%%%%%%%%%%%%
% \paragraph{Chapter Include Files.}
%
% The include files are called |cdocsch1.tex| and |cdocsch2.tex|.
%
%\iffalse
%<*samplechap1|samplechap2>
%\fi

% Optional override for |\version| flag:
%    \begin{macrocode}
%%\providecommand{\version}{final}
%    \end{macrocode}

% Include the main document:
%    \begin{macrocode}
\input{childdoc.def}
\childdocof{cdocsamp}
%    \end{macrocode}

%\iffalse
%</samplechap1|samplechap2>
%\fi
%
%\iffalse
%<*samplechap1>
%\fi
% Some text for chapter 1:
%    \begin{macrocode}
\section{one}
some text in chapter one
%    \end{macrocode}

%\iffalse
%</samplechap1>
%\fi
% Some text for chapter 2:
%\iffalse
%<*samplechap2>
%\fi
%    \begin{macrocode}
\section{two}
more text in chapter two
%    \end{macrocode}

%\iffalse
%</samplechap2>
%\fi
%
% %%%%%%%%%%%%%%%%%%%%%%%%%%%%%%%%%%%%%%
% \paragraph{Part Include Files.}
%
% The include files are called |cdocspt3.tex| and |cdocspt4.tex|.
%
%\iffalse
%<*samplepart3|samplepart4>
%\fi

% Optional override for |\version| flag:
%    \begin{macrocode}
%%\providecommand{\version}{final}
%    \end{macrocode}

% Include the main document:
%    \begin{macrocode}
\input{childdoc.def}
\childdocby{cdocsamp}
%    \end{macrocode}

%\iffalse
%</samplepart3|samplepart4>
%\fi
%
%\iffalse
%<*samplepart3>
%\fi
% Some text for part 3:
%    \begin{macrocode}
some text in part three
%    \end{macrocode}

%\iffalse
%</samplepart3>
%\fi
% Some text for part 4:
%\iffalse
%<*samplepart4>
%\fi
%    \begin{macrocode}
more text in part four
%    \end{macrocode}

%\iffalse
%</samplepart4>
%\fi
%
% %%%%%%%%%%%%%%%%%%%%%%%%%%%%%%%%%%%%%%
% \paragraph{Forwarding for a Complete Draft.}
%
% The following forwarding file |cdocsdrf.tex|
% compiles the main document in draft mode:
%\iffalse
%<*sampledraft>
%\fi
%    \begin{macrocode}
\def\version{draft}
\input{childdoc.def}
\childdocforward{cdocsamp}
%    \end{macrocode}

%\iffalse
%</sampledraft>
%\fi
%
% %%%%%%%%%%%%%%%%%%%%%%%%%%%%%%%%%%%%%%
% \paragraph{Forwarding for Final Version of the Chapters.}
%
% The following forwarding files |cdocsfn1.tex| and |cdocsfn2.tex|
% (with identical content)
% compile the final versions of the child documents
% |cdocsch1.tex| and |cdocsch2.tex|, respectively:
%\iffalse
%<*samplefinal>
%\fi
%    \begin{macrocode}
\def\version{final}
\input{childdoc.def}
\childdocforwardprefix[cdocsamp]{cdocsfn}{cdocsch}
%    \end{macrocode}

%\iffalse
%</samplefinal>
%\fi
%
% %%%%%%%%%%%%%%%%%%%%%%%%%%%%%%%%%%%%%%
% \paragraph{Command Line Processing.}
%
% The following three command lines generate the output files
% |cdocscld|, |cdocscl1| and |cdocscl2|
% which should be identical to
% |cdocsdrf|, |cdocsch1| and |cdocsfn2|, respectively:
% \begin{center}
% \begin{tabular}{l}
% |latex -jobname cdocscld \|\\
% |  "\def\version{draft}\input{childdoc.def}\childdocforward{cdocsamp}"|\\
% |latex -jobname cdocscl1 \|\\
% |  "\input{childdoc.def}\childdocforward[cdocsamp]{cdocsch1}"|\\
% |latex -jobname cdocscl2 \|\\
% |  "\def\version{final}\input{childdoc.def}\childdocforward{cdocsch2}"|
% \end{tabular}
% \end{center}
% Note that the trailing backslash on each first line
% merely continues the input to the second line
% (for convenient cut ant paste).
% Furthermore, the command |latex| can be replaced by any
% of its alternative versions such as |pdflatex|.
%
% %%%%%%%%%%%%%%%%%%%%%%%%%%%%%%%%%%%%%%%%%%%%%%%%%%%%%%%%%%%%%%%%%%%%%%%%%%%%%%
% %%%%%%%%%%%%%%%%%%%%%%%%%%%%%%%%%%%%%%%%%%%%%%%%%%%%%%%%%%%%%%%%%%%%%%%%%%%%%%
% \section{Implementation}
%\iffalse
%<*package>
%\fi
%
% This section describes the definitions file |childdoc.def|.

% The definitions cannot be loaded using |\usepackage| or |\RequirePackage|
% which has a mechanism to prevent loading a style file more than once.
% When loading the definitions by means of |\input|
% multiple instances have to be prevented manually:
%\iffalse
%This code needs to be before the `\ProvidesFile' directive
%which is defined at the beginning of this file.
%Therefore it is also placed there and commented out here.
%</package>
%<*discard>
%\fi
%    \begin{macrocode}
\ifdefined\childdocmain\endinput\fi
%    \end{macrocode}
%\iffalse
%</discard>
%<*package>
%\fi
%
% \macro{\ifchilddoc}
% \macro{\ifchilddocmanual}
% The conditional |\ifchilddoc| tells whether a
% child (true) or main (false) document is being compiled.
% The conditional |\ifchilddocmanual| tells whether
% the |\includeonly| mechanism is used (false) or
% the selection of child files must be performed manually (true).
% The definitions initialise to false:
%    \begin{macrocode}
\newif\ifchilddoc
\newif\ifchilddocmanual
%    \end{macrocode}

% \macro{\childdocname}
% \macro{\childdocjob}
% The macro |\childdocname| stores the name of the main document
% to be compiled. The macro |\childdocjob| stores the name of
% the document on which the \LaTeX{} compiler was originally invoked.
% The content of |\jobname| cannot be compared
% to filenames specified in the source due to different catcodes.
% The following code rescans |\jobname|, stores the result
% in |\childdocname| and saves a copy in |\childdocjob|:
%    \begin{macrocode}
\edef\childdocname{\scantokens\expandafter{\jobname\noexpand}}
\let\childdocjob\childdocname
%    \end{macrocode}

% \macro{\childdocdisable}
% The macro |\childdocdisable| prevents the main file
% from being processed more than once.
% At this stage, the main document command |\childdocmain|
% is assumed to be called once again where it should do nothing.
% Any subsequent call to it should prevent
% a secondary processing of the main document
% It overwrites the forwarding commands
% |\childdocof| and |\childdocforward|
% with empty macros to prevent further inclusions of the main document:
%    \begin{macrocode}
\newcommand{\childdocdisable}
{
  \renewcommand{\childdocmain}[1]{\renewcommand{\childdocmain}[1]{\endinput}}
  \renewcommand{\childdocof}[1]{}
  \renewcommand{\childdocby}[2][]{}
  \renewcommand{\childdocforward}[2][]{}
  \renewcommand{\childdocdisable}{}
}
%    \end{macrocode}

% \macro{\childdocmain}
% The macro |\childdocmain| is to be called at the top of the main file
% with nothing or the main filename (without extension) as argument.
% First, it breaks loops.
% If the argument is not empty and does not match |\childdocname|
% (which is set by the first inclusion of |childdoc.def|),
% |\ifchilddoc| is set to true, |\includeonly| is applied to the child file
% and |\jobname| is set to the main file
% (for proper handling of |.aux| files):
%    \begin{macrocode}
\newcommand{\childdocmain}[1]
{
  \childdocdisable\childdocmain{}
  \if?#1?\else
    \begingroup
      \def\childdoctmp{#1}
      \ifx\childdoctmp\childdocname
        \def\childdoctmp{}
      \else
        \def\childdoctmp
        {
          \childdoctrue
          \includeonly{\childdocname}
          \def\childdocjob{#1}
          \def\jobname{#1}
        }
      \fi
      \expandafter
    \endgroup
    \childdoctmp
  \fi
}
%    \end{macrocode}

% \macro{\childdocof}
% The command |\childdocof| redirects
% compilation to the main file |#1|.
%    \begin{macrocode}
\newcommand{\childdocof}[1]
{
  \childdocdisable
  \childdoctrue
  \includeonly{\childdocname}
  \def\jobname{#1}
  \def\childdocjob{#1}
  \input{#1}
}
%    \end{macrocode}

% \macro{\childdocby}
% The command |\childdocby| ....
%    \begin{macrocode}
\newcommand{\childdocby}[2][]
{
  \childdocdisable
  \childdoctrue
  \childdocmanualtrue
  \if?#1?\else
    \def\jobname{#2}
  \fi
  \def\childdocjob{#2}
  \input{#2}
  \endinput
}
%    \end{macrocode}

% \macro{\childdocforward}
% The command |\childdocforward| redirects
% compilation to the main file or
% (if the optional argument is given) a child file.
% Parameters are set as if the main file
% or a child file starting with |\childdocof| was compiled.
% Then compilation is handed over to the main file:
%    \begin{macrocode}
\newcommand{\childdocforward}[2][]
{
  \begingroup
    \if?#1?
      \def\childdoctmp
      {
        \def\childdocname{#2}
        \def\childdocjob{#2}
        \def\jobname{#2}
        \input{#2}
        \endinput
      }
    \else
      \def\childdoctmp
      {
        \childdocdisable
        \def\childdocname{#2}
        \childdoctrue
        \includeonly{#2}
        \def\childdocjob{#1}
        \def\jobname{#1}
        \input{#1}
        \endinput
      }
    \fi
    \expandafter
  \endgroup
  \childdoctmp
}
%    \end{macrocode}

% \macro{\childdocforwardprefix}
% The command |\childdocforwardprefix| redirects
% compilation to the main or a child file by means of a pattern.
% The prefix |#1| in the current filename is replaced by |#2|
% and the suffix of the current filename is kept
% (it is assumed that the filename does not contain the substring `|~~~|'
% which is used as a delimiter).
% Compilation is handed over to the new file by |\childdocforward|:
%    \begin{macrocode}
\newcommand{\childdocforwardprefix}[3][]
{
  \begingroup
    \def\childdocextract #2##1~~~{\def\childdoctmp{\childdocforward[#1]{#3##1}}}
    \expandafter\childdocextract\childdocname~~~
    \expandafter
  \endgroup
  \childdoctmp
}
%    \end{macrocode}

% \macro{\childdoc}
% The deprecated macro |\childdoc| is a legacy version of |\childdocmain|:
%    \begin{macrocode}
\newcommand{\childdoc}{\childdocmain}
%    \end{macrocode}

% \macro{\childdocredirect}
% The deprecated macro |\childdocredirect| is a legacy version
% of |\childdocforward| and |\childdocforwardprefix|:
%    \begin{macrocode}
\newcommand{\childdocredirect}[2][]
{
  \begingroup
    \if?#1?
      \def\childdoctmp{\childdocforward{#2}}
    \else
      \def\childdoctmp{\childdocforwardprefix{#1}{#2}}
    \fi
    \expandafter
  \endgroup
  \childdoctmp
}
%    \end{macrocode}

%\iffalse
%</package>
%\fi
%
\endinput

\childdocmain{}
%    \end{macrocode}

% Optional override for |\version| flag:
%    \begin{macrocode}
%%\ifchilddoc\else\providecommand{\version}{draft}\fi
%    \end{macrocode}

% Define the default values for the |\version| flag
% (|final| for the main file and |draft| for childs):
%    \begin{macrocode}
\ifchilddoc
\providecommand{\version}{draft}
\else
\providecommand{\version}{final}
\fi
%    \end{macrocode}

% Load the standard document class:
%    \begin{macrocode}
\documentclass[12pt]{article}
%    \end{macrocode}

% Start the document body:
%    \begin{macrocode}
\begin{document}
%    \end{macrocode}

% Declare a title page.
% Print title, part of document being processed and version flag:
%    \begin{macrocode}
\addtocounter{page}{-1}
\begin{center}
{\LARGE\bfseries{}childdoc example\par}
\vspace{1cm}
\ifchilddoc
\ifchilddocmanual part\else chapter\fi:
`\childdocname' of `\childdocjob'\par
\else
main document: `\childdocjob'\par
\fi
version: \version\par
\end{center}
\newpage
%    \end{macrocode}

% Manually include selected file,
% otherwise process as usual:
%    \begin{macrocode}
\ifchilddocmanual
\section*{part `\childdocname'}
\input{\childdocname}
\else
%    \end{macrocode}

% Include the two chapters:
%    \begin{macrocode}
\include{cdocsch1}
\include{cdocsch2}
%    \end{macrocode}

% Include the two parts unless only chapters should be displayed:
%    \begin{macrocode}
\ifchilddoc\else
\section{part three}
\input{cdocspt3}
\section{part four}
\input{cdocspt4}
\fi
%    \end{macrocode}

% Process as usual until here:
%    \begin{macrocode}
\fi
%    \end{macrocode}

% End of document body:
%    \begin{macrocode}
\end{document}
%    \end{macrocode}
%\iffalse
%</samplemain>
%\fi
%
% %%%%%%%%%%%%%%%%%%%%%%%%%%%%%%%%%%%%%%
% \paragraph{Chapter Include Files.}
%
% The include files are called |cdocsch1.tex| and |cdocsch2.tex|.
%
%\iffalse
%<*samplechap1|samplechap2>
%\fi

% Optional override for |\version| flag:
%    \begin{macrocode}
%%\providecommand{\version}{final}
%    \end{macrocode}

% Include the main document:
%    \begin{macrocode}
% \iffalse
%
% childdoc.dtx Copyright (C) 2017-2018 Niklas Beisert
%
% This work may be distributed and/or modified under the
% conditions of the LaTeX Project Public License, either version 1.3
% of this license or (at your option) any later version.
% The latest version of this license is in
%   http://www.latex-project.org/lppl.txt
% and version 1.3 or later is part of all distributions of LaTeX
% version 2005/12/01 or later.
%
% This work has the LPPL maintenance status `maintained'.
%
% The Current Maintainer of this work is Niklas Beisert.
%
% This work consists of the files childdoc.dtx and childdoc.ins
% and the derived files childdoc.def and cdocsamp.tex with
% cdocsch1.tex, cdocsch2.tex, cdocsdrf.tex, cdocsfn1.tex, cdocsfn2.tex.
%
%<package>\ifdefined\childdocmain\endinput\fi
%<package>\ProvidesFile{childdoc.def}[2018/12/30 v2.0 child document driver]
%<samplemain>\ProvidesFile{cdocsamp.tex}[2018/12/30 v2.0 sample for childdoc]
%<*driver>
%\ProvidesFile{childdoc.drv}[2018/12/30 v2.0 childdoc reference manual file]
\PassOptionsToClass{10pt,a4paper}{article}
\documentclass{ltxdoc}

\usepackage[margin=35mm]{geometry}
\usepackage{hyperref}
\usepackage{hyperxmp}
\usepackage[usenames]{color}

\hypersetup{colorlinks=true}
\hypersetup{pdfstartview=FitH}
\hypersetup{pdfpagemode=UseNone}
\hypersetup{pdfsource={}}
\hypersetup{pdflang={en-UK}}
\hypersetup{pdfcopyright={Copyright 2017-2018 Niklas Beisert.
  This work may be distributed and/or modified under the
  conditions of the LaTeX Project Public License, either version 1.3
  of this license or (at your option) any later version.}}
\hypersetup{pdflicenseurl={http://www.latex-project.org/lppl.txt}}
\hypersetup{pdfcontactaddress={ETH Zurich, ITP, HIT K,
  Wolfgang-Pauli-Strasse 27}}
\hypersetup{pdfcontactpostcode={8093}}
\hypersetup{pdfcontactcity={Zurich}}
\hypersetup{pdfcontactcountry={Switzerland}}
\hypersetup{pdfcontactemail={nbeisert@itp.phys.ethz.ch}}
\hypersetup{pdfcontacturl={http://people.phys.ethz.ch/\xmptilde nbeisert/}}

\newcommand{\secref}[1]{\hyperref[#1]{section \ref*{#1}}}

\parskip1ex
\parindent0pt
\let\olditemize\itemize
\def\itemize{\olditemize\parskip0pt}

\begin{document}

\title{The \textsf{childdoc} Package}
\hypersetup{pdftitle={The childdoc Package}}
\author{Niklas Beisert\\[2ex]
  Institut f\"ur Theoretische Physik\\
  Eidgen\"ossische Technische Hochschule Z\"urich\\
  Wolfgang-Pauli-Strasse 27, 8093 Z\"urich, Switzerland\\[1ex]
  \href{mailto:nbeisert@itp.phys.ethz.ch}
  {\texttt{nbeisert@itp.phys.ethz.ch}}}
\hypersetup{pdfauthor={Niklas Beisert}}
\hypersetup{pdfsubject={Manual for the LaTeX2e Package childdoc}}
\date{30 December 2018, \textsf{v2.0}}
\maketitle

\begin{abstract}\noindent
\textsf{childdoc} is a \LaTeXe{} package
that enables the direct compilation
of document sections included by |\include|
to individual files.
\end{abstract}

\begingroup
\parskip0ex
\tableofcontents
\endgroup

%%%%%%%%%%%%%%%%%%%%%%%%%%%%%%%%%%%%%%%%%%%%%%%%%%%%%%%%%%%%%%%%%%%%%%%%%%%%%%%%
%%%%%%%%%%%%%%%%%%%%%%%%%%%%%%%%%%%%%%%%%%%%%%%%%%%%%%%%%%%%%%%%%%%%%%%%%%%%%%%%
\section{Introduction}

\LaTeX{} provides a mechanism to structure a large document (such as a book)
into a main file and several child files (containing the chapters)
using the |\include| command.
This mechanism is beneficial for documents
which span hundreds of pages in order to
make the source file(s) more manageable.
Moreover, compilation can be restricted to
selected child files by means of the |\includeonly| command.
The latter feature can be used to reduce the compilation time while editing
(this was significantly more useful in the earlier days of \LaTeX{})
or to generate a smaller document which is easier to navigate.
Another application of |\includeonly| is to generate
documents consisting of selected parts of the complete document.

However, there are a few drawbacks of the plain |\include| mechanism:
\begin{itemize}
\item
The child files cannot be compiled on their own,
they can only be compiled via the main file.
A naive editing environment
(such as a text editor with an option
to have the current file processed by \LaTeX)
may require one to switch to the main file before compiling;
attempting to compile the child file produces errors.
\item
The main file must be modified (each time)
to adjust the |\includeonly| command
to the present needs. This easily leaves the main file in a messy state.
\item
The generated document will always carry the filename
of the main document. This is inconvenient if
several child files are to be compiled and
to be kept for distribution.
\end{itemize}

The present package provides a simple interface
to make child files individually compilable by \LaTeX{}.
Compiling a child file then has the same effect as compiling
the main file with an |\includeonly| command
to select the appropriate child.
Moreover the generated document will carry the name of the child
rather than the main file.
This resolves all three above issues.

This feature is meant to make the editing of books,
thesis documents and lecture notes somewhat more convenient.
However, the package can also be used efficiently for
composing a series of documents (such as exercise sheets)
which are typically distributed individually.
It then assists the author in generating the individual documents
(potentially in different versions)
as well as a document containing the collected series.
Another application is in developing style files
or other kinds of included material
where compilation of the style file could redirect
to a sample or test file.

%%%%%%%%%%%%%%%%%%%%%%%%%%%%%%%%%%%%%%%%%%%%%%%%%%%%%%%%%%%%%%%%%%%%%%%%%%%%%%%%
%%%%%%%%%%%%%%%%%%%%%%%%%%%%%%%%%%%%%%%%%%%%%%%%%%%%%%%%%%%%%%%%%%%%%%%%%%%%%%%%
\section{Usage}

First of all, the package \textsf{childdoc} is \emph{not} a standard
\LaTeXe{} |.sty| style file! Therefore it needs to be invoked in
a non-standard way.

%%%%%%%%%%%%%%%%%%%%%%%%%%%%%%%%%%%%%%%%%%%%%%%%%%%%%%%%%%%%%%%%%%%%%%%%%%%%%%%%
\subsection{Included Files}
\label{sec:include}

%%%%%%%%%%%%%%%%%%%%%%%%%%%%%%%%%%%%%%%%
\DescribeMacro{\childdocmain}
To use the package, add the commands
\begin{center}
\begin{tabular}{l}
|\input{childdoc.def}|\\
|\childdocmain{}|\\
\end{tabular}
\end{center}
at the very top of the main \LaTeX{} file,
in particular \emph{before} the |\documentclass| statement!
The argument of |\childdocmain| should be left empty
(but it must be present).

%%%%%%%%%%%%%%%%%%%%%%%%%%%%%%%%%%%%%%%%
\DescribeMacro{\childdocof}
Furthermore, add the commands
\begin{center}
\begin{tabular}{l}
|\input{childdoc.def}|\\
|\childdocof{|\textit{main}|}|\\
\end{tabular}
\end{center}
at the top of every child file \textit{child}
which is included by |\include{|\textit{child}|}|
from within the main file
(or at least for those files to be compiled individually).
The argument \textit{main} must be the filename of the main file.

There are a couple of
considerations in setting up the main and child documents:

%%%%%%%%%%%%%%%%%%%%%%%%%%%%%%%%%%%%%%%%
\paragraph{Restrictions.}

Please note the following restrictions:
\begin{itemize}
\item
|\childdocmain| must be called with one argument \textit{main}
to ensure compatibility with earlier version of the package.
It must either be empty (|\childdocmain{}|)
or precisely match the filename of the main file in which it is specified.
See \secref{sec:detection} for further information.
\item
The filename \textit{main} must be specified without the |.tex| extension.
\item
The filename \textit{main} is case sensitive
(even in case-insensitive file systems)
due to internal string comparison.
\item
The argument \textit{main} should be fully expanded, it cannot be a macro.
\item
Subdirectories and special characters should be avoided in filenames.
\item
The command |\childdocmain{|\textit{main}|}| must be followed by a whitespace.
It should not be followed immediately by another command
or by a comment mark `|%|'.
This is because the \TeX{} parser reads the token immediately following
the argument of |\childdocmain| and puts it
at the beginning of every child section;
however, a white\-space is ignored.
\end{itemize}

%%%%%%%%%%%%%%%%%%%%%%%%%%%%%%%%%%%%%%%%
\paragraph{Content of Main File.}

It is advisable to place all content in the child files included by |\include|.
Any output contained in the main file will appear in all child documents
unless suppressed manually;
it cannot be suppressed automatically by the |\includeonly| directive
and thus should normally be avoided.
A method to include some content in the main file
by means of conditional processing is described in \secref{sec:conditional}.

%%%%%%%%%%%%%%%%%%%%%%%%%%%%%%%%%%%%%%%%
\paragraph{Page Numbering.}

When only a part of the document is compiled,
the appropriate numbering of pages
(as well as other status parameters)
is determined from the |.aux| files.
The latter contain information from previous passes.
However this information needs to propagate through
all intermediate child documents.
Therefore the page numbering in child documents may well
be inconsistent until the complete document is compiled at least once.

A useful (if unconventional) way to always ensure a consistent
page numbering is to restart the numbering in each child document
and denote the pages by `\textit{child}|.|\textit{page}'
where \textit{child} represents the chapter/section number of the child file.
This can be achieved by the command
|\numberwithin{page}{|\textit{child}|}|
of the \textsf{amsmath} package
where \textit{child} can be |chapter| or |section|
depending on the chosen structuring.
Alternatively, one can modify the macro |\thepage| appropriately
and reset the counter |page| at the start of each child file.

%%%%%%%%%%%%%%%%%%%%%%%%%%%%%%%%%%%%%%%%%%%%%%%%%%%%%%%%%%%%%%%%%%%%%%%%%%%%%%%%
\subsection{Conditional Processing}
\label{sec:conditional}

The package provides a mechanism to compile different versions
of a document. To customise the versions further some conditional processing
can come in handy to distinguish which version is being compiled.
The package provides two macros to describe the compilation context:

%%%%%%%%%%%%%%%%%%%%%%%%%%%%%%%%%%%%%%%%
\DescribeMacro{\ifchilddoc}
The conditional |\ifchilddoc| distinguishes between the compilation of
child documents and the main document:
%
\begin{center}
|\ifchilddoc |\textit{child-code}| |[|\||else |\textit{main-code}]| \||fi|
\end{center}

%%%%%%%%%%%%%%%%%%%%%%%%%%%%%%%%%%%%%%%%
\DescribeMacro{\childdocname}
\DescribeMacro{\childdocjob}
The macro |\childdocname| contains the filename (without extension)
of the main or child file being processed.
Note that |\childdocjob| will always contain the name of the main file.

%%%%%%%%%%%%%%%%%%%%%%%%%%%%%%%%%%%%%%%%
\paragraph{Title Page.}

Conditional processing can be used to include a title or banner page
in the main document when proper precautions are taken.
Importantly, the code in the main file should ensure that the page counter
(as well as other status parameters which are stored in the |.aux| files)
takes the same value after the conditional processing.
Otherwise the page numbers may take divergent values
depending on which part is compiled.

For example, a title page could be declared by:
%
\begin{center}
\begin{tabular}{l}
|\ifchilddoc\||else|\\
|\addtocounter{page}{-1}|\\
\textit{code for title page}\\
|\newpage|\\
|\||fi|
\end{tabular}
\end{center}
%
A banner page for the child documents can be generated by:
%
\begin{center}
\begin{tabular}{l}
|\ifchilddoc|\\
|\addtocounter{page}{-1}|\\
\textit{code for banner page}\\
|\newpage|\\
|\||fi|
\end{tabular}
\end{center}
%
Here one could write a message such as:
\begin{center}
|This is the part \childdocname{} of \childdocjob{}.|
\end{center}

%%%%%%%%%%%%%%%%%%%%%%%%%%%%%%%%%%%%%%%%%%%%%%%%%%%%%%%%%%%%%%%%%%%%%%%%%%%%%%%%
\subsection{Flags}
\label{sec:flags}

The package makes it easy to generate different versions
of the main or child documents.
To this end compilation flags can be defined
and assigned different default values.
They will be particularly useful in conjunction
with the forwarding mechanism described in \secref{sec:forward}.

For example, it may be useful to have a flag |\version|
which can be set to |draft| or |final|.
The document source will contain some conditional code
depending on the value of |\version|.
Suppose further, the flag should default to |final| for the main file
and to |draft| for child files
which is a natural assignment for editing the document.
This is achieved by placing the following code
in the preamble of the main document
(below the |\childdocmain| directive):
%
\begin{center}
\begin{tabular}{l}
|\ifchilddoc|\\
|\providecommand{\version}{draft}|\\
|\||else|\\
|\providecommand{\version}{final}|\\
|\||fi|
\end{tabular}
\end{center}
%
The definition by |\providecommand| makes sure
that previous definitions are not overwritten.
Further statements |\providecommand{\version}{...}|
can thus be added before the above code to override it.

For the main file, one might add a line
(between |\childdocmain| and the above block)
%
\begin{center}
|%\ifchilddoc\||else\providecommand{\version}{draft}\||fi|
\end{center}
%
which can be uncommented to produce a draft version.
Likewise one can add a line to the very top of a child file
(above the |\childdocof{|\textit{main}|}| directive)
%
\begin{center}
|%\providecommand{\version}{final}|
\end{center}
%
which can be uncommented to produce the final version of this child document.

%%%%%%%%%%%%%%%%%%%%%%%%%%%%%%%%%%%%%%%%%%%%%%%%%%%%%%%%%%%%%%%%%%%%%%%%%%%%%%%%
\subsection{Forwarding}
\label{sec:forward}

Different versions of the main or child documents
using compilation flags as described in \secref{sec:flags}
can be (permanently) stored in different files
for convenient compilation, viewing and distribution.
To this end, the package defines a command
to pass on compilation to a different file:

%%%%%%%%%%%%%%%%%%%%%%%%%%%%%%%%%%%%%%%%
\DescribeMacro{\childdocforward}
The command |\childdocforward| redirects processing to
another source file:
%
\begin{center}
\begin{tabular}{l}
|\input{childdoc.def}|\\
|\childdocforward[|\textit{main}|]{|\textit{dest}|}|\\
\end{tabular}
\end{center}
%
The argument \textit{dest} is the destination file
(without extension).
It should be the main file or one of the child files.
Note that further \textsf{childdoc} directives
such as |\childdocof| and |\childdocforward|
in the indicated file will be processed in this form.
The optional argument \textit{main}
passes on directly to the main file \textit{main}
while pretending to compile the child \textit{dest}.
This form behaves as if \textit{dest}
issues |\childdocof{|\textit{main}|}| right away,
and no further \textsf{childdoc} directives will be processed.

%%%%%%%%%%%%%%%%%%%%%%%%%%%%%%%%%%%%%%%%
\DescribeMacro{\...prefix}
In the alternative form |\childdocforwardprefix|,
%
\begin{center}
\begin{tabular}{l}
|\input{childdoc.def}|\\
|\childdocforwardprefix[|\textit{main}|]{|\textit{prefix}|}{|\textit{dest}|}|
\end{tabular}
\end{center}
%
the destination file is determined by a pattern
depending on the current file:
To make this work, the current file must be called
`{\textit{prefix}\hspace{0.2em}\textit{suffix}}'
with \textit{prefix} matching precisely the argument.
Processing is then passed on to the file
`{\textit{dest}\hspace{0.2em}\textit{suffix}}'.
Surely, the same effect is achieved by
directly specifying the
argument `{\textit{dest}\hspace{0.2em}\textit{suffix}}'
in the first form.
However, that requires to set up a different file
for each child. With the alternative form of the command
all these files can have exactly the same content
which simplifies setting them up and maintaining them.

For example, the following file |draft.tex|
with a compilation flag |\version| as described in \secref{sec:flags}
compiles the main document as a draft:
%
\begin{center}
\begin{tabular}{l}
|\def\version{draft}|\\
|\input{childdoc.def}|\\
|\childdocforward{|\textit{main}|}|
\end{tabular}
\end{center}
%
Likewise, the following files |final|\textit{nn}|.tex|
compile the final version of the child document
|child|\textit{nn}|.tex|:
%
\begin{center}
\begin{tabular}{l}
|\def\version{final}|\\
|\input{childdoc.def}|\\
|\childdocforwardprefix{final}{child}|
\end{tabular}
\end{center}
%

Note that when several versions of a main file and/or of each child file
are to be generated, it may be convenient to set up a |Makefile| or
shell script to automatise the process.

%%%%%%%%%%%%%%%%%%%%%%%%%%%%%%%%%%%%%%%%%%%%%%%%%%%%%%%%%%%%%%%%%%%%%%%%%%%%%%%%
\subsection{Command Line Processing}
\label{sec:commandline}

The effect of redirection files can also be achieved by invoking
the \LaTeX{} compiler with a more elaborate command line.
Most conveniently this should be done as part
of a shell script or a |Makefile|.

When using \textsf{childdoc} in the main file, the following
command lines effectively perform a redirection
(note that depending on the shell being used,
backslashes may have to be doubled: `|\|' $\to$ `|\\|'):
%
\begin{center}
|... -jobname "|\textit{target}|" |\\|"|[\textit{flags}]%
|\input{childdoc.def}\childdocforward[|\textit{main}|]{|\textit{dest}|}"|
\end{center}
%
Here \textit{target} is the name of the output file,
\textit{main} is the name of the main file
and \textit{dest} is the name of the main or child file to be processed
(all filenames without extensions).
The optional argument \textit{main} can be omitted
if \textit{main} matches \textit{dest}.
Optionally, compilation \textit{flags} can be defined via |\def| commands.
This command line makes the \TeX{} engine believe
it is compiling the file \textit{target}
whose content is specified as the latter parameter.
The provided code then forwards the processing to
\textit{main} or \textit{dest} as described in \secref{sec:forward}.

%%%%%%%%%%%%%%%%%%%%%%%%%%%%%%%%%%%%%%%%%%%%%%%%%%%%%%%%%%%%%%%%%%%%%%%%%%%%%%%%
\subsection{Include by Input}
\label{sec:input}

Including child documents by |\include| has some restrictions by design.
Most notably, the content of a child document always occupies
its own set of pages; pages cannot be shared between child documents.
Usually, this behaviour makes perfect sense
because each child document contain an essential part of the document.
However, in some situations it may be desirable to compose
a document from a collection of parts
without having mandatory page breaks between then.
For this case, the package
provides a mechanism to include parts
by |\input| which can also be processed individually.
However, by construction this mechanism
requires manual handling of the content to be output.

%%%%%%%%%%%%%%%%%%%%%%%%%%%%%%%%%%%%%%%%
\DescribeMacro{\ifchilddocmanual}
The main file should be prepared as usual, see \secref{sec:include}.
However, the document body must make a distinction
between processing of an individual part and of the main document, e.g.:
%
\begin{center}
\begin{tabular}{l}
|\ifchilddocmanual|\\
|\input{\childdocname}|\\
|\||else|\\
\textit{document body with }|\input{|\textit{part}|}|\\
|\||fi|
\end{tabular}
\end{center}
%
The conditional |\ifchilddocmanual| is true whenever
a part to be included by |\input| is being compiled,
and the name of the part is stored in |\childdocname|.

%%%%%%%%%%%%%%%%%%%%%%%%%%%%%%%%%%%%%%%%
\DescribeMacro{\childdocby}
Each part to be included by |\input| should start with:
%
\begin{center}
\begin{tabular}{l}
|\input{childdoc.def}|\\
|\childdocby{|\textit{main}|}|\\
\end{tabular}
\end{center}
%
The directive |\childdocby| is similar to |\childdocof|
described in \secref{sec:include},
but the subsequent selection of content must be done manually.
To that end, both |\ifchilddoc| and |\ifchilddocmanual|
will be true upon processing of a part,
and the name of the part is stored in |\childdocname|.
Note that |\jobname| will be set to the filename of the current part
so that each part receives an individual |.aux| file
that does not interfere with the |.aux| file(s) of the main document.
This behaviour can be altered by the alternative form
|\childdocby[*]{|\textit{main}|}| (with a non-empty optional argument)
which uses the |.aux| file of the main document
by setting |\jobname| to \textit{main}.

%%%%%%%%%%%%%%%%%%%%%%%%%%%%%%%%%%%%%%%%%%%%%%%%%%%%%%%%%%%%%%%%%%%%%%%%%%%%%%%%
\subsection{Driver Development}
\label{sec:driver}

The \textsf{childdoc} mechanism can also be use for the development
of definition files such as \LaTeX{} styles or classes.
This case differs from the above setup with multiple parts
included by |\include| in that no |\includeonly| should be invoked.
This can be achieved by starting the include file
(before |\ProvidesPackage|) with:
%
\begin{center}
\begin{tabular}{l}
|\input{childdoc.def}|\\
|\childdocforward{|\textit{main}|}|\\
\end{tabular}
\end{center}
%
or alternatively with:
%
\begin{center}
\begin{tabular}{l}
|\input{childdoc.def}|\\
|\childdocby{|\textit{main}|}|\\
\end{tabular}
\end{center}
%
Both forms have slightly different effects as described above.
The main file is prepared as usual, see \secref{sec:include}.

%%%%%%%%%%%%%%%%%%%%%%%%%%%%%%%%%%%%%%%%%%%%%%%%%%%%%%%%%%%%%%%%%%%%%%%%%%%%%%%%
\subsection{Legacy Detection}
\label{sec:detection}

The directive |\childdocmain| in the main file can detect
whether the complete document or merely a child is to be compiled
even without using the directive |\childdocof|.
This method is deprecated because it is less robust
and there is no compelling reason to use it;
it is merely provided for backward compatibility
and it may be removed in future versions.

If the detection mechanism is to be used,
it is mandatory to correctly specify
the filename of the main file as the argument of |\childdocmain|:
%
\begin{center}
\begin{tabular}{l}
|\input{childdoc.def}|\\
|\childdocmain{|\textit{main}|}|\\
\end{tabular}
\end{center}
%
If |\jobname| does not match the argument \textit{main} of |\childdocmain|,
it is assumed that |\jobname| points to the child file to be compiled.
When using |\childdocmain| with the main file specified as argument,
it suffices to start a child file
with just |\input{|\textit{main}|}|
without loading of the package and using |\childdocof|.
If instead all processing is done
with the appropriate \textsf{childdoc} directives,
the argument of \textit{main} of |\childdocmain| can be empty.

An alternative version of the command line processing described
in \secref{sec:commandline} using the detection mechanism reads:
%
\begin{center}
|... -jobname "|\textit{target}|" "|[\textit{flags}]%
[|\def\jobname{|\textit{dest}|}|]|\input{|\textit{main}|}"|
\end{center}

%%%%%%%%%%%%%%%%%%%%%%%%%%%%%%%%%%%%%%%%%%%%%%%%%%%%%%%%%%%%%%%%%%%%%%%%%%%%%%%%
\subsection{Manual Code}
\label{sec:manual}

In case one cannot be certain whether the definitions file |childdoc.def|
is installed on the target \TeX{} distribution
and one prefers not to ship it,
it is conceivable to paste a few relevant commands into the sources.

To that end, drop all statements |\input{childdoc.def}|
and perform the replacements as outlined below.
Instead of |\childdocmain{|\textit{main}|}| add the following code
to the top of the main file:
%
\begin{center}
\begin{tabular}{l}
|\||ifdefined\childdocname\endinput\||fi\newif\ifchilddoc|\\
|\edef\childdocname{\scantokens\expandafter{\jobname\noexpand}}|\\
|\def\childdocmain{|\textit{main}|}\||ifx\childdocmain\childdocname\||else|\\
|\childdoctrue\includeonly{\childdocname}\let\jobname\childdocmain\||fi|\\
\end{tabular}
\end{center}
%
Instead of |\childdocof{|\textit{main}|}| just include the main file
at the top of each child file:
%
\begin{center}
|\input{|\textit{main}|}|
\end{center}
%
A simple redirection |\childdocforward{|\textit{dest}|}| is achieved by:
%
\begin{center}
|\def\jobname{|\textit{dest}|}\input{\jobname}|
\end{center}
%
The redirection with prefix
|\childdocforwardprefix[|\textit{prefix}|]{|\textit{dest}|}|
is accomplished by:
%
\begin{center}
\begin{tabular}{l}
|{\edef\jobname{\scantokens\expandafter{\jobname\noexpand}}|\\
|\def\redirectjob |\textit{prefix}|#1~~~{\gdef\jobname{|\textit{dest}|#1}}|\\
|\expandafter\redirectjob\jobname~~~}\input{\jobname}|
\end{tabular}
\end{center}

In an alternative approach,
child documents can be compiled by a specific command line
without additional code or specific definitions:
%
\begin{center}
|... -jobname "|\textit{target}|" "|[\textit{flags}]%
|\includeonly{|\textit{dest}|}\input{|\textit{main}|}"|
\end{center}
%

%%%%%%%%%%%%%%%%%%%%%%%%%%%%%%%%%%%%%%%%%%%%%%%%%%%%%%%%%%%%%%%%%%%%%%%%%%%%%%%%
%%%%%%%%%%%%%%%%%%%%%%%%%%%%%%%%%%%%%%%%%%%%%%%%%%%%%%%%%%%%%%%%%%%%%%%%%%%%%%%%
\section{Information}

%%%%%%%%%%%%%%%%%%%%%%%%%%%%%%%%%%%%%%%%%%%%%%%%%%%%%%%%%%%%%%%%%%%%%%%%%%%%%%%%
\subsection{Copyright}

Copyright \copyright{} 2017--2018 Niklas Beisert

This work may be distributed and/or modified under the
conditions of the \LaTeX{} Project Public License, either version 1.3
of this license or (at your option) any later version.
The latest version of this license is in
  \url{http://www.latex-project.org/lppl.txt}
and version 1.3 or later is part of all distributions of \LaTeX{}
version 2005/12/01 or later.

This work has the LPPL maintenance status `maintained'.

The Current Maintainer of this work is Niklas Beisert.

This work consists of the files |README.txt|, |childdoc.ins| and |childdoc.dtx|
as well as the derived files |childdoc.def|, |cdocsamp.tex|
with |cdocsch1.tex|, |cdocsch2.tex|, |cdocspt3.tex|, |cdocspt4.tex|,
|cdocsdrf.tex|, |cdocsfn1.tex|, |cdocsfn2.tex|
as well as |childdoc.pdf|.

%%%%%%%%%%%%%%%%%%%%%%%%%%%%%%%%%%%%%%%%%%%%%%%%%%%%%%%%%%%%%%%%%%%%%%%%%%%%%%%%
\subsection{Files and Installation}

The package consists of the files:
%
\begin{center}
\begin{tabular}{ll}
    |README.txt|   & readme file \\
    |childdoc.ins| & installation file \\
    |childdoc.dtx| & source file \\
    |childdoc.def| & definition file \\
    |cdocsamp.tex| & sample main file \\
    |cdocsch1.tex| & sample include file \\
    |cdocsch2.tex| & sample include file \\
    |cdocspt3.tex| & sample part file \\
    |cdocspt4.tex| & sample part file \\
    |cdocsdrf.tex| & sample redirection file \\
    |cdocsfn1.tex| & sample redirection file \\
    |cdocsfn2.tex| & sample redirection file \\
    |childdoc.pdf| & manual
\end{tabular}
\end{center}
%
The distribution consists of the files
|README.txt|, |childdoc.ins| and |childdoc.dtx|.
%
\begin{itemize}
\item
Run (pdf)\LaTeX{} on |childdoc.dtx|
to compile the manual |childdoc.pdf| (this file).
\item
Run \LaTeX{} on |childdoc.ins| to create the definitions file |childdoc.def|
and the sample |cdocsamp.tex| with include files
|cdocsch1.tex|, |cdocsch2.tex|, |cdocspt3.tex|, |cdocspt4.tex|,
|cdocsdrf.tex|, |cdocsfn1.tex|, |cdocsfn2.tex|.
Then copy the file |childdoc.def| to an appropriate directory of your \LaTeX{}
distribution, e.g.\ \textit{texmf-root}|/tex/latex/childdoc|.
\end{itemize}

%%%%%%%%%%%%%%%%%%%%%%%%%%%%%%%%%%%%%%%%%%%%%%%%%%%%%%%%%%%%%%%%%%%%%%%%%%%%%%%%
\subsection{Related CTAN Packages}

There are several other packages which offer a similar functionality:
%
\begin{itemize}
\item
The packages
\href{http://ctan.org/pkg/docmute}{\textsf{docmute}},
\href{http://ctan.org/pkg/includex}{\textsf{includex}} and
\href{http://ctan.org/pkg/standalone}{\textsf{standalone}}
provide commands to include only the document body of
a child file thus allowing both files to be compiled individually.
\item
The packages \href{http://ctan.org/pkg/subdocs}{\textsf{subdocs}}
and \href{http://ctan.org/pkg/subfiles}{\textsf{subfiles}}
provide structures in which the main and child documents can be
encapsulated and allowing them to be compiled individually.
The inclusion mechanism is different from the conventional |\include|.
\item
The package \href{http://ctan.org/pkg/combine}{\textsf{combine}}
is an elaborate solution to combine several documents into one.
\end{itemize}
%
See also the CTAN topic \href{http://ctan.org/topic/subdocs}{\textsf{subdocs}}
for further related packages.
The present package differs from the above solutions in that
a document structure constructed with the conventional |\include| mechanism
just needs two extra commands at the top of every file
such that all constituent files can be compiled individually.

%%%%%%%%%%%%%%%%%%%%%%%%%%%%%%%%%%%%%%%%%%%%%%%%%%%%%%%%%%%%%%%%%%%%%%%%%%%%%%%%
%\subsection{Feature Suggestions}
%
%The following is a list of features which may be useful for future
%versions of this package:
%%
%\begin{itemize}
%\item
%\ldots
%\end{itemize}

%%%%%%%%%%%%%%%%%%%%%%%%%%%%%%%%%%%%%%%%%%%%%%%%%%%%%%%%%%%%%%%%%%%%%%%%%%%%%%%%
\subsection{Revision History}

%%%%%%%%%%%%%%%%%%%%%%%%%%%%%%%%%%%%%%%%
\paragraph{v2.0:} 2018/12/30

\begin{itemize}
\item
immediate forward processing
\item
added |\childdocby| mechanism
\item
manual restructured
\end{itemize}

%%%%%%%%%%%%%%%%%%%%%%%%%%%%%%%%%%%%%%%%
\paragraph{v1.6:} 2018/01/17

\begin{itemize}
\item
application for development of include files
\item
corrections to manual
\end{itemize}

%%%%%%%%%%%%%%%%%%%%%%%%%%%%%%%%%%%%%%%%
\paragraph{v1.5:} 2017/05/21

\begin{itemize}
\item
more complete structuring introduced
\item
|\childdocof| introduced
\item
|\childdoc| renamed to |\childdocmain|
\item
|\childredirect| renamed to |\childdocforward| and |\childdocforwardprefix|
and functionality expanded
\end{itemize}

%%%%%%%%%%%%%%%%%%%%%%%%%%%%%%%%%%%%%%%%
\paragraph{v1.0:} 2017/04/27

\begin{itemize}
\item
manual and install package
\item
first version published on CTAN
\end{itemize}

%%%%%%%%%%%%%%%%%%%%%%%%%%%%%%%%%%%%%%%%
\paragraph{v0.6:} 2017/04/26

\begin{itemize}
\item
redirection mechanism added
\end{itemize}

%%%%%%%%%%%%%%%%%%%%%%%%%%%%%%%%%%%%%%%%
\paragraph{v0.5:} 2017/04/26

\begin{itemize}
\item
functionality in definition file
\end{itemize}


%%%%%%%%%%%%%%%%%%%%%%%%%%%%%%%%%%%%%%%%%%%%%%%%%%%%%%%%%%%%%%%%%%%%%%%%%%%%%%%%
%%%%%%%%%%%%%%%%%%%%%%%%%%%%%%%%%%%%%%%%%%%%%%%%%%%%%%%%%%%%%%%%%%%%%%%%%%%%%%%%
%%%%%%%%%%%%%%%%%%%%%%%%%%%%%%%%%%%%%%%%%%%%%%%%%%%%%%%%%%%%%%%%%%%%%%%%%%%%%%%%
\appendix

\settowidth\MacroIndent{\rmfamily\scriptsize 000\ }

 \DocInput{childdoc.dtx}

\end{document}
%</driver>
% \fi
%
% %%%%%%%%%%%%%%%%%%%%%%%%%%%%%%%%%%%%%%%%%%%%%%%%%%%%%%%%%%%%%%%%%%%%%%%%%%%%%%
% %%%%%%%%%%%%%%%%%%%%%%%%%%%%%%%%%%%%%%%%%%%%%%%%%%%%%%%%%%%%%%%%%%%%%%%%%%%%%%
% \section{Sample}
%\iffalse
%<*samplemain>
%\fi
%
% The following presents a sample document
% with two chapters, two parts, a title page,
% a compile flag as well as three forwarding files to set the flag.
% It consists of eight |.tex| files:
% \begin{center}
% \begin{tabular}{ll}
% |cdocsamp.tex|&main file\\
% |cdocsch1.tex|&include file for chapter 1\\
% |cdocsch2.tex|&include file for chapter 2\\
% |cdocspt3.tex|&include file for part 3\\
% |cdocspt4.tex|&include file for part 4\\
% |cdocsdrf.tex|&forwarding file for main file in draft mode\\
% |cdocsfi1.tex|&forwarding file for final version of chapter 1\\
% |cdocsfi2.tex|&forwarding file for final version of chapter 2\\
% \end{tabular}
% \end{center}
% Each of the eight files can be compiled directly by the \LaTeX{} compiler.
%
% %%%%%%%%%%%%%%%%%%%%%%%%%%%%%%%%%%%%%%
% \paragraph{Main File.}
%
% The main file is called |cdocsamp.tex|.
%
% Load the \textsf{childdoc} definitions and
% declare the filename for the main document:
%    \begin{macrocode}
\input{childdoc.def}
\childdocmain{}
%    \end{macrocode}

% Optional override for |\version| flag:
%    \begin{macrocode}
%%\ifchilddoc\else\providecommand{\version}{draft}\fi
%    \end{macrocode}

% Define the default values for the |\version| flag
% (|final| for the main file and |draft| for childs):
%    \begin{macrocode}
\ifchilddoc
\providecommand{\version}{draft}
\else
\providecommand{\version}{final}
\fi
%    \end{macrocode}

% Load the standard document class:
%    \begin{macrocode}
\documentclass[12pt]{article}
%    \end{macrocode}

% Start the document body:
%    \begin{macrocode}
\begin{document}
%    \end{macrocode}

% Declare a title page.
% Print title, part of document being processed and version flag:
%    \begin{macrocode}
\addtocounter{page}{-1}
\begin{center}
{\LARGE\bfseries{}childdoc example\par}
\vspace{1cm}
\ifchilddoc
\ifchilddocmanual part\else chapter\fi:
`\childdocname' of `\childdocjob'\par
\else
main document: `\childdocjob'\par
\fi
version: \version\par
\end{center}
\newpage
%    \end{macrocode}

% Manually include selected file,
% otherwise process as usual:
%    \begin{macrocode}
\ifchilddocmanual
\section*{part `\childdocname'}
\input{\childdocname}
\else
%    \end{macrocode}

% Include the two chapters:
%    \begin{macrocode}
\include{cdocsch1}
\include{cdocsch2}
%    \end{macrocode}

% Include the two parts unless only chapters should be displayed:
%    \begin{macrocode}
\ifchilddoc\else
\section{part three}
\input{cdocspt3}
\section{part four}
\input{cdocspt4}
\fi
%    \end{macrocode}

% Process as usual until here:
%    \begin{macrocode}
\fi
%    \end{macrocode}

% End of document body:
%    \begin{macrocode}
\end{document}
%    \end{macrocode}
%\iffalse
%</samplemain>
%\fi
%
% %%%%%%%%%%%%%%%%%%%%%%%%%%%%%%%%%%%%%%
% \paragraph{Chapter Include Files.}
%
% The include files are called |cdocsch1.tex| and |cdocsch2.tex|.
%
%\iffalse
%<*samplechap1|samplechap2>
%\fi

% Optional override for |\version| flag:
%    \begin{macrocode}
%%\providecommand{\version}{final}
%    \end{macrocode}

% Include the main document:
%    \begin{macrocode}
\input{childdoc.def}
\childdocof{cdocsamp}
%    \end{macrocode}

%\iffalse
%</samplechap1|samplechap2>
%\fi
%
%\iffalse
%<*samplechap1>
%\fi
% Some text for chapter 1:
%    \begin{macrocode}
\section{one}
some text in chapter one
%    \end{macrocode}

%\iffalse
%</samplechap1>
%\fi
% Some text for chapter 2:
%\iffalse
%<*samplechap2>
%\fi
%    \begin{macrocode}
\section{two}
more text in chapter two
%    \end{macrocode}

%\iffalse
%</samplechap2>
%\fi
%
% %%%%%%%%%%%%%%%%%%%%%%%%%%%%%%%%%%%%%%
% \paragraph{Part Include Files.}
%
% The include files are called |cdocspt3.tex| and |cdocspt4.tex|.
%
%\iffalse
%<*samplepart3|samplepart4>
%\fi

% Optional override for |\version| flag:
%    \begin{macrocode}
%%\providecommand{\version}{final}
%    \end{macrocode}

% Include the main document:
%    \begin{macrocode}
\input{childdoc.def}
\childdocby{cdocsamp}
%    \end{macrocode}

%\iffalse
%</samplepart3|samplepart4>
%\fi
%
%\iffalse
%<*samplepart3>
%\fi
% Some text for part 3:
%    \begin{macrocode}
some text in part three
%    \end{macrocode}

%\iffalse
%</samplepart3>
%\fi
% Some text for part 4:
%\iffalse
%<*samplepart4>
%\fi
%    \begin{macrocode}
more text in part four
%    \end{macrocode}

%\iffalse
%</samplepart4>
%\fi
%
% %%%%%%%%%%%%%%%%%%%%%%%%%%%%%%%%%%%%%%
% \paragraph{Forwarding for a Complete Draft.}
%
% The following forwarding file |cdocsdrf.tex|
% compiles the main document in draft mode:
%\iffalse
%<*sampledraft>
%\fi
%    \begin{macrocode}
\def\version{draft}
\input{childdoc.def}
\childdocforward{cdocsamp}
%    \end{macrocode}

%\iffalse
%</sampledraft>
%\fi
%
% %%%%%%%%%%%%%%%%%%%%%%%%%%%%%%%%%%%%%%
% \paragraph{Forwarding for Final Version of the Chapters.}
%
% The following forwarding files |cdocsfn1.tex| and |cdocsfn2.tex|
% (with identical content)
% compile the final versions of the child documents
% |cdocsch1.tex| and |cdocsch2.tex|, respectively:
%\iffalse
%<*samplefinal>
%\fi
%    \begin{macrocode}
\def\version{final}
\input{childdoc.def}
\childdocforwardprefix[cdocsamp]{cdocsfn}{cdocsch}
%    \end{macrocode}

%\iffalse
%</samplefinal>
%\fi
%
% %%%%%%%%%%%%%%%%%%%%%%%%%%%%%%%%%%%%%%
% \paragraph{Command Line Processing.}
%
% The following three command lines generate the output files
% |cdocscld|, |cdocscl1| and |cdocscl2|
% which should be identical to
% |cdocsdrf|, |cdocsch1| and |cdocsfn2|, respectively:
% \begin{center}
% \begin{tabular}{l}
% |latex -jobname cdocscld \|\\
% |  "\def\version{draft}\input{childdoc.def}\childdocforward{cdocsamp}"|\\
% |latex -jobname cdocscl1 \|\\
% |  "\input{childdoc.def}\childdocforward[cdocsamp]{cdocsch1}"|\\
% |latex -jobname cdocscl2 \|\\
% |  "\def\version{final}\input{childdoc.def}\childdocforward{cdocsch2}"|
% \end{tabular}
% \end{center}
% Note that the trailing backslash on each first line
% merely continues the input to the second line
% (for convenient cut ant paste).
% Furthermore, the command |latex| can be replaced by any
% of its alternative versions such as |pdflatex|.
%
% %%%%%%%%%%%%%%%%%%%%%%%%%%%%%%%%%%%%%%%%%%%%%%%%%%%%%%%%%%%%%%%%%%%%%%%%%%%%%%
% %%%%%%%%%%%%%%%%%%%%%%%%%%%%%%%%%%%%%%%%%%%%%%%%%%%%%%%%%%%%%%%%%%%%%%%%%%%%%%
% \section{Implementation}
%\iffalse
%<*package>
%\fi
%
% This section describes the definitions file |childdoc.def|.

% The definitions cannot be loaded using |\usepackage| or |\RequirePackage|
% which has a mechanism to prevent loading a style file more than once.
% When loading the definitions by means of |\input|
% multiple instances have to be prevented manually:
%\iffalse
%This code needs to be before the `\ProvidesFile' directive
%which is defined at the beginning of this file.
%Therefore it is also placed there and commented out here.
%</package>
%<*discard>
%\fi
%    \begin{macrocode}
\ifdefined\childdocmain\endinput\fi
%    \end{macrocode}
%\iffalse
%</discard>
%<*package>
%\fi
%
% \macro{\ifchilddoc}
% \macro{\ifchilddocmanual}
% The conditional |\ifchilddoc| tells whether a
% child (true) or main (false) document is being compiled.
% The conditional |\ifchilddocmanual| tells whether
% the |\includeonly| mechanism is used (false) or
% the selection of child files must be performed manually (true).
% The definitions initialise to false:
%    \begin{macrocode}
\newif\ifchilddoc
\newif\ifchilddocmanual
%    \end{macrocode}

% \macro{\childdocname}
% \macro{\childdocjob}
% The macro |\childdocname| stores the name of the main document
% to be compiled. The macro |\childdocjob| stores the name of
% the document on which the \LaTeX{} compiler was originally invoked.
% The content of |\jobname| cannot be compared
% to filenames specified in the source due to different catcodes.
% The following code rescans |\jobname|, stores the result
% in |\childdocname| and saves a copy in |\childdocjob|:
%    \begin{macrocode}
\edef\childdocname{\scantokens\expandafter{\jobname\noexpand}}
\let\childdocjob\childdocname
%    \end{macrocode}

% \macro{\childdocdisable}
% The macro |\childdocdisable| prevents the main file
% from being processed more than once.
% At this stage, the main document command |\childdocmain|
% is assumed to be called once again where it should do nothing.
% Any subsequent call to it should prevent
% a secondary processing of the main document
% It overwrites the forwarding commands
% |\childdocof| and |\childdocforward|
% with empty macros to prevent further inclusions of the main document:
%    \begin{macrocode}
\newcommand{\childdocdisable}
{
  \renewcommand{\childdocmain}[1]{\renewcommand{\childdocmain}[1]{\endinput}}
  \renewcommand{\childdocof}[1]{}
  \renewcommand{\childdocby}[2][]{}
  \renewcommand{\childdocforward}[2][]{}
  \renewcommand{\childdocdisable}{}
}
%    \end{macrocode}

% \macro{\childdocmain}
% The macro |\childdocmain| is to be called at the top of the main file
% with nothing or the main filename (without extension) as argument.
% First, it breaks loops.
% If the argument is not empty and does not match |\childdocname|
% (which is set by the first inclusion of |childdoc.def|),
% |\ifchilddoc| is set to true, |\includeonly| is applied to the child file
% and |\jobname| is set to the main file
% (for proper handling of |.aux| files):
%    \begin{macrocode}
\newcommand{\childdocmain}[1]
{
  \childdocdisable\childdocmain{}
  \if?#1?\else
    \begingroup
      \def\childdoctmp{#1}
      \ifx\childdoctmp\childdocname
        \def\childdoctmp{}
      \else
        \def\childdoctmp
        {
          \childdoctrue
          \includeonly{\childdocname}
          \def\childdocjob{#1}
          \def\jobname{#1}
        }
      \fi
      \expandafter
    \endgroup
    \childdoctmp
  \fi
}
%    \end{macrocode}

% \macro{\childdocof}
% The command |\childdocof| redirects
% compilation to the main file |#1|.
%    \begin{macrocode}
\newcommand{\childdocof}[1]
{
  \childdocdisable
  \childdoctrue
  \includeonly{\childdocname}
  \def\jobname{#1}
  \def\childdocjob{#1}
  \input{#1}
}
%    \end{macrocode}

% \macro{\childdocby}
% The command |\childdocby| ....
%    \begin{macrocode}
\newcommand{\childdocby}[2][]
{
  \childdocdisable
  \childdoctrue
  \childdocmanualtrue
  \if?#1?\else
    \def\jobname{#2}
  \fi
  \def\childdocjob{#2}
  \input{#2}
  \endinput
}
%    \end{macrocode}

% \macro{\childdocforward}
% The command |\childdocforward| redirects
% compilation to the main file or
% (if the optional argument is given) a child file.
% Parameters are set as if the main file
% or a child file starting with |\childdocof| was compiled.
% Then compilation is handed over to the main file:
%    \begin{macrocode}
\newcommand{\childdocforward}[2][]
{
  \begingroup
    \if?#1?
      \def\childdoctmp
      {
        \def\childdocname{#2}
        \def\childdocjob{#2}
        \def\jobname{#2}
        \input{#2}
        \endinput
      }
    \else
      \def\childdoctmp
      {
        \childdocdisable
        \def\childdocname{#2}
        \childdoctrue
        \includeonly{#2}
        \def\childdocjob{#1}
        \def\jobname{#1}
        \input{#1}
        \endinput
      }
    \fi
    \expandafter
  \endgroup
  \childdoctmp
}
%    \end{macrocode}

% \macro{\childdocforwardprefix}
% The command |\childdocforwardprefix| redirects
% compilation to the main or a child file by means of a pattern.
% The prefix |#1| in the current filename is replaced by |#2|
% and the suffix of the current filename is kept
% (it is assumed that the filename does not contain the substring `|~~~|'
% which is used as a delimiter).
% Compilation is handed over to the new file by |\childdocforward|:
%    \begin{macrocode}
\newcommand{\childdocforwardprefix}[3][]
{
  \begingroup
    \def\childdocextract #2##1~~~{\def\childdoctmp{\childdocforward[#1]{#3##1}}}
    \expandafter\childdocextract\childdocname~~~
    \expandafter
  \endgroup
  \childdoctmp
}
%    \end{macrocode}

% \macro{\childdoc}
% The deprecated macro |\childdoc| is a legacy version of |\childdocmain|:
%    \begin{macrocode}
\newcommand{\childdoc}{\childdocmain}
%    \end{macrocode}

% \macro{\childdocredirect}
% The deprecated macro |\childdocredirect| is a legacy version
% of |\childdocforward| and |\childdocforwardprefix|:
%    \begin{macrocode}
\newcommand{\childdocredirect}[2][]
{
  \begingroup
    \if?#1?
      \def\childdoctmp{\childdocforward{#2}}
    \else
      \def\childdoctmp{\childdocforwardprefix{#1}{#2}}
    \fi
    \expandafter
  \endgroup
  \childdoctmp
}
%    \end{macrocode}

%\iffalse
%</package>
%\fi
%
\endinput

\childdocof{cdocsamp}
%    \end{macrocode}

%\iffalse
%</samplechap1|samplechap2>
%\fi
%
%\iffalse
%<*samplechap1>
%\fi
% Some text for chapter 1:
%    \begin{macrocode}
\section{one}
some text in chapter one
%    \end{macrocode}

%\iffalse
%</samplechap1>
%\fi
% Some text for chapter 2:
%\iffalse
%<*samplechap2>
%\fi
%    \begin{macrocode}
\section{two}
more text in chapter two
%    \end{macrocode}

%\iffalse
%</samplechap2>
%\fi
%
% %%%%%%%%%%%%%%%%%%%%%%%%%%%%%%%%%%%%%%
% \paragraph{Part Include Files.}
%
% The include files are called |cdocspt3.tex| and |cdocspt4.tex|.
%
%\iffalse
%<*samplepart3|samplepart4>
%\fi

% Optional override for |\version| flag:
%    \begin{macrocode}
%%\providecommand{\version}{final}
%    \end{macrocode}

% Include the main document:
%    \begin{macrocode}
% \iffalse
%
% childdoc.dtx Copyright (C) 2017-2018 Niklas Beisert
%
% This work may be distributed and/or modified under the
% conditions of the LaTeX Project Public License, either version 1.3
% of this license or (at your option) any later version.
% The latest version of this license is in
%   http://www.latex-project.org/lppl.txt
% and version 1.3 or later is part of all distributions of LaTeX
% version 2005/12/01 or later.
%
% This work has the LPPL maintenance status `maintained'.
%
% The Current Maintainer of this work is Niklas Beisert.
%
% This work consists of the files childdoc.dtx and childdoc.ins
% and the derived files childdoc.def and cdocsamp.tex with
% cdocsch1.tex, cdocsch2.tex, cdocsdrf.tex, cdocsfn1.tex, cdocsfn2.tex.
%
%<package>\ifdefined\childdocmain\endinput\fi
%<package>\ProvidesFile{childdoc.def}[2018/12/30 v2.0 child document driver]
%<samplemain>\ProvidesFile{cdocsamp.tex}[2018/12/30 v2.0 sample for childdoc]
%<*driver>
%\ProvidesFile{childdoc.drv}[2018/12/30 v2.0 childdoc reference manual file]
\PassOptionsToClass{10pt,a4paper}{article}
\documentclass{ltxdoc}

\usepackage[margin=35mm]{geometry}
\usepackage{hyperref}
\usepackage{hyperxmp}
\usepackage[usenames]{color}

\hypersetup{colorlinks=true}
\hypersetup{pdfstartview=FitH}
\hypersetup{pdfpagemode=UseNone}
\hypersetup{pdfsource={}}
\hypersetup{pdflang={en-UK}}
\hypersetup{pdfcopyright={Copyright 2017-2018 Niklas Beisert.
  This work may be distributed and/or modified under the
  conditions of the LaTeX Project Public License, either version 1.3
  of this license or (at your option) any later version.}}
\hypersetup{pdflicenseurl={http://www.latex-project.org/lppl.txt}}
\hypersetup{pdfcontactaddress={ETH Zurich, ITP, HIT K,
  Wolfgang-Pauli-Strasse 27}}
\hypersetup{pdfcontactpostcode={8093}}
\hypersetup{pdfcontactcity={Zurich}}
\hypersetup{pdfcontactcountry={Switzerland}}
\hypersetup{pdfcontactemail={nbeisert@itp.phys.ethz.ch}}
\hypersetup{pdfcontacturl={http://people.phys.ethz.ch/\xmptilde nbeisert/}}

\newcommand{\secref}[1]{\hyperref[#1]{section \ref*{#1}}}

\parskip1ex
\parindent0pt
\let\olditemize\itemize
\def\itemize{\olditemize\parskip0pt}

\begin{document}

\title{The \textsf{childdoc} Package}
\hypersetup{pdftitle={The childdoc Package}}
\author{Niklas Beisert\\[2ex]
  Institut f\"ur Theoretische Physik\\
  Eidgen\"ossische Technische Hochschule Z\"urich\\
  Wolfgang-Pauli-Strasse 27, 8093 Z\"urich, Switzerland\\[1ex]
  \href{mailto:nbeisert@itp.phys.ethz.ch}
  {\texttt{nbeisert@itp.phys.ethz.ch}}}
\hypersetup{pdfauthor={Niklas Beisert}}
\hypersetup{pdfsubject={Manual for the LaTeX2e Package childdoc}}
\date{30 December 2018, \textsf{v2.0}}
\maketitle

\begin{abstract}\noindent
\textsf{childdoc} is a \LaTeXe{} package
that enables the direct compilation
of document sections included by |\include|
to individual files.
\end{abstract}

\begingroup
\parskip0ex
\tableofcontents
\endgroup

%%%%%%%%%%%%%%%%%%%%%%%%%%%%%%%%%%%%%%%%%%%%%%%%%%%%%%%%%%%%%%%%%%%%%%%%%%%%%%%%
%%%%%%%%%%%%%%%%%%%%%%%%%%%%%%%%%%%%%%%%%%%%%%%%%%%%%%%%%%%%%%%%%%%%%%%%%%%%%%%%
\section{Introduction}

\LaTeX{} provides a mechanism to structure a large document (such as a book)
into a main file and several child files (containing the chapters)
using the |\include| command.
This mechanism is beneficial for documents
which span hundreds of pages in order to
make the source file(s) more manageable.
Moreover, compilation can be restricted to
selected child files by means of the |\includeonly| command.
The latter feature can be used to reduce the compilation time while editing
(this was significantly more useful in the earlier days of \LaTeX{})
or to generate a smaller document which is easier to navigate.
Another application of |\includeonly| is to generate
documents consisting of selected parts of the complete document.

However, there are a few drawbacks of the plain |\include| mechanism:
\begin{itemize}
\item
The child files cannot be compiled on their own,
they can only be compiled via the main file.
A naive editing environment
(such as a text editor with an option
to have the current file processed by \LaTeX)
may require one to switch to the main file before compiling;
attempting to compile the child file produces errors.
\item
The main file must be modified (each time)
to adjust the |\includeonly| command
to the present needs. This easily leaves the main file in a messy state.
\item
The generated document will always carry the filename
of the main document. This is inconvenient if
several child files are to be compiled and
to be kept for distribution.
\end{itemize}

The present package provides a simple interface
to make child files individually compilable by \LaTeX{}.
Compiling a child file then has the same effect as compiling
the main file with an |\includeonly| command
to select the appropriate child.
Moreover the generated document will carry the name of the child
rather than the main file.
This resolves all three above issues.

This feature is meant to make the editing of books,
thesis documents and lecture notes somewhat more convenient.
However, the package can also be used efficiently for
composing a series of documents (such as exercise sheets)
which are typically distributed individually.
It then assists the author in generating the individual documents
(potentially in different versions)
as well as a document containing the collected series.
Another application is in developing style files
or other kinds of included material
where compilation of the style file could redirect
to a sample or test file.

%%%%%%%%%%%%%%%%%%%%%%%%%%%%%%%%%%%%%%%%%%%%%%%%%%%%%%%%%%%%%%%%%%%%%%%%%%%%%%%%
%%%%%%%%%%%%%%%%%%%%%%%%%%%%%%%%%%%%%%%%%%%%%%%%%%%%%%%%%%%%%%%%%%%%%%%%%%%%%%%%
\section{Usage}

First of all, the package \textsf{childdoc} is \emph{not} a standard
\LaTeXe{} |.sty| style file! Therefore it needs to be invoked in
a non-standard way.

%%%%%%%%%%%%%%%%%%%%%%%%%%%%%%%%%%%%%%%%%%%%%%%%%%%%%%%%%%%%%%%%%%%%%%%%%%%%%%%%
\subsection{Included Files}
\label{sec:include}

%%%%%%%%%%%%%%%%%%%%%%%%%%%%%%%%%%%%%%%%
\DescribeMacro{\childdocmain}
To use the package, add the commands
\begin{center}
\begin{tabular}{l}
|\input{childdoc.def}|\\
|\childdocmain{}|\\
\end{tabular}
\end{center}
at the very top of the main \LaTeX{} file,
in particular \emph{before} the |\documentclass| statement!
The argument of |\childdocmain| should be left empty
(but it must be present).

%%%%%%%%%%%%%%%%%%%%%%%%%%%%%%%%%%%%%%%%
\DescribeMacro{\childdocof}
Furthermore, add the commands
\begin{center}
\begin{tabular}{l}
|\input{childdoc.def}|\\
|\childdocof{|\textit{main}|}|\\
\end{tabular}
\end{center}
at the top of every child file \textit{child}
which is included by |\include{|\textit{child}|}|
from within the main file
(or at least for those files to be compiled individually).
The argument \textit{main} must be the filename of the main file.

There are a couple of
considerations in setting up the main and child documents:

%%%%%%%%%%%%%%%%%%%%%%%%%%%%%%%%%%%%%%%%
\paragraph{Restrictions.}

Please note the following restrictions:
\begin{itemize}
\item
|\childdocmain| must be called with one argument \textit{main}
to ensure compatibility with earlier version of the package.
It must either be empty (|\childdocmain{}|)
or precisely match the filename of the main file in which it is specified.
See \secref{sec:detection} for further information.
\item
The filename \textit{main} must be specified without the |.tex| extension.
\item
The filename \textit{main} is case sensitive
(even in case-insensitive file systems)
due to internal string comparison.
\item
The argument \textit{main} should be fully expanded, it cannot be a macro.
\item
Subdirectories and special characters should be avoided in filenames.
\item
The command |\childdocmain{|\textit{main}|}| must be followed by a whitespace.
It should not be followed immediately by another command
or by a comment mark `|%|'.
This is because the \TeX{} parser reads the token immediately following
the argument of |\childdocmain| and puts it
at the beginning of every child section;
however, a white\-space is ignored.
\end{itemize}

%%%%%%%%%%%%%%%%%%%%%%%%%%%%%%%%%%%%%%%%
\paragraph{Content of Main File.}

It is advisable to place all content in the child files included by |\include|.
Any output contained in the main file will appear in all child documents
unless suppressed manually;
it cannot be suppressed automatically by the |\includeonly| directive
and thus should normally be avoided.
A method to include some content in the main file
by means of conditional processing is described in \secref{sec:conditional}.

%%%%%%%%%%%%%%%%%%%%%%%%%%%%%%%%%%%%%%%%
\paragraph{Page Numbering.}

When only a part of the document is compiled,
the appropriate numbering of pages
(as well as other status parameters)
is determined from the |.aux| files.
The latter contain information from previous passes.
However this information needs to propagate through
all intermediate child documents.
Therefore the page numbering in child documents may well
be inconsistent until the complete document is compiled at least once.

A useful (if unconventional) way to always ensure a consistent
page numbering is to restart the numbering in each child document
and denote the pages by `\textit{child}|.|\textit{page}'
where \textit{child} represents the chapter/section number of the child file.
This can be achieved by the command
|\numberwithin{page}{|\textit{child}|}|
of the \textsf{amsmath} package
where \textit{child} can be |chapter| or |section|
depending on the chosen structuring.
Alternatively, one can modify the macro |\thepage| appropriately
and reset the counter |page| at the start of each child file.

%%%%%%%%%%%%%%%%%%%%%%%%%%%%%%%%%%%%%%%%%%%%%%%%%%%%%%%%%%%%%%%%%%%%%%%%%%%%%%%%
\subsection{Conditional Processing}
\label{sec:conditional}

The package provides a mechanism to compile different versions
of a document. To customise the versions further some conditional processing
can come in handy to distinguish which version is being compiled.
The package provides two macros to describe the compilation context:

%%%%%%%%%%%%%%%%%%%%%%%%%%%%%%%%%%%%%%%%
\DescribeMacro{\ifchilddoc}
The conditional |\ifchilddoc| distinguishes between the compilation of
child documents and the main document:
%
\begin{center}
|\ifchilddoc |\textit{child-code}| |[|\||else |\textit{main-code}]| \||fi|
\end{center}

%%%%%%%%%%%%%%%%%%%%%%%%%%%%%%%%%%%%%%%%
\DescribeMacro{\childdocname}
\DescribeMacro{\childdocjob}
The macro |\childdocname| contains the filename (without extension)
of the main or child file being processed.
Note that |\childdocjob| will always contain the name of the main file.

%%%%%%%%%%%%%%%%%%%%%%%%%%%%%%%%%%%%%%%%
\paragraph{Title Page.}

Conditional processing can be used to include a title or banner page
in the main document when proper precautions are taken.
Importantly, the code in the main file should ensure that the page counter
(as well as other status parameters which are stored in the |.aux| files)
takes the same value after the conditional processing.
Otherwise the page numbers may take divergent values
depending on which part is compiled.

For example, a title page could be declared by:
%
\begin{center}
\begin{tabular}{l}
|\ifchilddoc\||else|\\
|\addtocounter{page}{-1}|\\
\textit{code for title page}\\
|\newpage|\\
|\||fi|
\end{tabular}
\end{center}
%
A banner page for the child documents can be generated by:
%
\begin{center}
\begin{tabular}{l}
|\ifchilddoc|\\
|\addtocounter{page}{-1}|\\
\textit{code for banner page}\\
|\newpage|\\
|\||fi|
\end{tabular}
\end{center}
%
Here one could write a message such as:
\begin{center}
|This is the part \childdocname{} of \childdocjob{}.|
\end{center}

%%%%%%%%%%%%%%%%%%%%%%%%%%%%%%%%%%%%%%%%%%%%%%%%%%%%%%%%%%%%%%%%%%%%%%%%%%%%%%%%
\subsection{Flags}
\label{sec:flags}

The package makes it easy to generate different versions
of the main or child documents.
To this end compilation flags can be defined
and assigned different default values.
They will be particularly useful in conjunction
with the forwarding mechanism described in \secref{sec:forward}.

For example, it may be useful to have a flag |\version|
which can be set to |draft| or |final|.
The document source will contain some conditional code
depending on the value of |\version|.
Suppose further, the flag should default to |final| for the main file
and to |draft| for child files
which is a natural assignment for editing the document.
This is achieved by placing the following code
in the preamble of the main document
(below the |\childdocmain| directive):
%
\begin{center}
\begin{tabular}{l}
|\ifchilddoc|\\
|\providecommand{\version}{draft}|\\
|\||else|\\
|\providecommand{\version}{final}|\\
|\||fi|
\end{tabular}
\end{center}
%
The definition by |\providecommand| makes sure
that previous definitions are not overwritten.
Further statements |\providecommand{\version}{...}|
can thus be added before the above code to override it.

For the main file, one might add a line
(between |\childdocmain| and the above block)
%
\begin{center}
|%\ifchilddoc\||else\providecommand{\version}{draft}\||fi|
\end{center}
%
which can be uncommented to produce a draft version.
Likewise one can add a line to the very top of a child file
(above the |\childdocof{|\textit{main}|}| directive)
%
\begin{center}
|%\providecommand{\version}{final}|
\end{center}
%
which can be uncommented to produce the final version of this child document.

%%%%%%%%%%%%%%%%%%%%%%%%%%%%%%%%%%%%%%%%%%%%%%%%%%%%%%%%%%%%%%%%%%%%%%%%%%%%%%%%
\subsection{Forwarding}
\label{sec:forward}

Different versions of the main or child documents
using compilation flags as described in \secref{sec:flags}
can be (permanently) stored in different files
for convenient compilation, viewing and distribution.
To this end, the package defines a command
to pass on compilation to a different file:

%%%%%%%%%%%%%%%%%%%%%%%%%%%%%%%%%%%%%%%%
\DescribeMacro{\childdocforward}
The command |\childdocforward| redirects processing to
another source file:
%
\begin{center}
\begin{tabular}{l}
|\input{childdoc.def}|\\
|\childdocforward[|\textit{main}|]{|\textit{dest}|}|\\
\end{tabular}
\end{center}
%
The argument \textit{dest} is the destination file
(without extension).
It should be the main file or one of the child files.
Note that further \textsf{childdoc} directives
such as |\childdocof| and |\childdocforward|
in the indicated file will be processed in this form.
The optional argument \textit{main}
passes on directly to the main file \textit{main}
while pretending to compile the child \textit{dest}.
This form behaves as if \textit{dest}
issues |\childdocof{|\textit{main}|}| right away,
and no further \textsf{childdoc} directives will be processed.

%%%%%%%%%%%%%%%%%%%%%%%%%%%%%%%%%%%%%%%%
\DescribeMacro{\...prefix}
In the alternative form |\childdocforwardprefix|,
%
\begin{center}
\begin{tabular}{l}
|\input{childdoc.def}|\\
|\childdocforwardprefix[|\textit{main}|]{|\textit{prefix}|}{|\textit{dest}|}|
\end{tabular}
\end{center}
%
the destination file is determined by a pattern
depending on the current file:
To make this work, the current file must be called
`{\textit{prefix}\hspace{0.2em}\textit{suffix}}'
with \textit{prefix} matching precisely the argument.
Processing is then passed on to the file
`{\textit{dest}\hspace{0.2em}\textit{suffix}}'.
Surely, the same effect is achieved by
directly specifying the
argument `{\textit{dest}\hspace{0.2em}\textit{suffix}}'
in the first form.
However, that requires to set up a different file
for each child. With the alternative form of the command
all these files can have exactly the same content
which simplifies setting them up and maintaining them.

For example, the following file |draft.tex|
with a compilation flag |\version| as described in \secref{sec:flags}
compiles the main document as a draft:
%
\begin{center}
\begin{tabular}{l}
|\def\version{draft}|\\
|\input{childdoc.def}|\\
|\childdocforward{|\textit{main}|}|
\end{tabular}
\end{center}
%
Likewise, the following files |final|\textit{nn}|.tex|
compile the final version of the child document
|child|\textit{nn}|.tex|:
%
\begin{center}
\begin{tabular}{l}
|\def\version{final}|\\
|\input{childdoc.def}|\\
|\childdocforwardprefix{final}{child}|
\end{tabular}
\end{center}
%

Note that when several versions of a main file and/or of each child file
are to be generated, it may be convenient to set up a |Makefile| or
shell script to automatise the process.

%%%%%%%%%%%%%%%%%%%%%%%%%%%%%%%%%%%%%%%%%%%%%%%%%%%%%%%%%%%%%%%%%%%%%%%%%%%%%%%%
\subsection{Command Line Processing}
\label{sec:commandline}

The effect of redirection files can also be achieved by invoking
the \LaTeX{} compiler with a more elaborate command line.
Most conveniently this should be done as part
of a shell script or a |Makefile|.

When using \textsf{childdoc} in the main file, the following
command lines effectively perform a redirection
(note that depending on the shell being used,
backslashes may have to be doubled: `|\|' $\to$ `|\\|'):
%
\begin{center}
|... -jobname "|\textit{target}|" |\\|"|[\textit{flags}]%
|\input{childdoc.def}\childdocforward[|\textit{main}|]{|\textit{dest}|}"|
\end{center}
%
Here \textit{target} is the name of the output file,
\textit{main} is the name of the main file
and \textit{dest} is the name of the main or child file to be processed
(all filenames without extensions).
The optional argument \textit{main} can be omitted
if \textit{main} matches \textit{dest}.
Optionally, compilation \textit{flags} can be defined via |\def| commands.
This command line makes the \TeX{} engine believe
it is compiling the file \textit{target}
whose content is specified as the latter parameter.
The provided code then forwards the processing to
\textit{main} or \textit{dest} as described in \secref{sec:forward}.

%%%%%%%%%%%%%%%%%%%%%%%%%%%%%%%%%%%%%%%%%%%%%%%%%%%%%%%%%%%%%%%%%%%%%%%%%%%%%%%%
\subsection{Include by Input}
\label{sec:input}

Including child documents by |\include| has some restrictions by design.
Most notably, the content of a child document always occupies
its own set of pages; pages cannot be shared between child documents.
Usually, this behaviour makes perfect sense
because each child document contain an essential part of the document.
However, in some situations it may be desirable to compose
a document from a collection of parts
without having mandatory page breaks between then.
For this case, the package
provides a mechanism to include parts
by |\input| which can also be processed individually.
However, by construction this mechanism
requires manual handling of the content to be output.

%%%%%%%%%%%%%%%%%%%%%%%%%%%%%%%%%%%%%%%%
\DescribeMacro{\ifchilddocmanual}
The main file should be prepared as usual, see \secref{sec:include}.
However, the document body must make a distinction
between processing of an individual part and of the main document, e.g.:
%
\begin{center}
\begin{tabular}{l}
|\ifchilddocmanual|\\
|\input{\childdocname}|\\
|\||else|\\
\textit{document body with }|\input{|\textit{part}|}|\\
|\||fi|
\end{tabular}
\end{center}
%
The conditional |\ifchilddocmanual| is true whenever
a part to be included by |\input| is being compiled,
and the name of the part is stored in |\childdocname|.

%%%%%%%%%%%%%%%%%%%%%%%%%%%%%%%%%%%%%%%%
\DescribeMacro{\childdocby}
Each part to be included by |\input| should start with:
%
\begin{center}
\begin{tabular}{l}
|\input{childdoc.def}|\\
|\childdocby{|\textit{main}|}|\\
\end{tabular}
\end{center}
%
The directive |\childdocby| is similar to |\childdocof|
described in \secref{sec:include},
but the subsequent selection of content must be done manually.
To that end, both |\ifchilddoc| and |\ifchilddocmanual|
will be true upon processing of a part,
and the name of the part is stored in |\childdocname|.
Note that |\jobname| will be set to the filename of the current part
so that each part receives an individual |.aux| file
that does not interfere with the |.aux| file(s) of the main document.
This behaviour can be altered by the alternative form
|\childdocby[*]{|\textit{main}|}| (with a non-empty optional argument)
which uses the |.aux| file of the main document
by setting |\jobname| to \textit{main}.

%%%%%%%%%%%%%%%%%%%%%%%%%%%%%%%%%%%%%%%%%%%%%%%%%%%%%%%%%%%%%%%%%%%%%%%%%%%%%%%%
\subsection{Driver Development}
\label{sec:driver}

The \textsf{childdoc} mechanism can also be use for the development
of definition files such as \LaTeX{} styles or classes.
This case differs from the above setup with multiple parts
included by |\include| in that no |\includeonly| should be invoked.
This can be achieved by starting the include file
(before |\ProvidesPackage|) with:
%
\begin{center}
\begin{tabular}{l}
|\input{childdoc.def}|\\
|\childdocforward{|\textit{main}|}|\\
\end{tabular}
\end{center}
%
or alternatively with:
%
\begin{center}
\begin{tabular}{l}
|\input{childdoc.def}|\\
|\childdocby{|\textit{main}|}|\\
\end{tabular}
\end{center}
%
Both forms have slightly different effects as described above.
The main file is prepared as usual, see \secref{sec:include}.

%%%%%%%%%%%%%%%%%%%%%%%%%%%%%%%%%%%%%%%%%%%%%%%%%%%%%%%%%%%%%%%%%%%%%%%%%%%%%%%%
\subsection{Legacy Detection}
\label{sec:detection}

The directive |\childdocmain| in the main file can detect
whether the complete document or merely a child is to be compiled
even without using the directive |\childdocof|.
This method is deprecated because it is less robust
and there is no compelling reason to use it;
it is merely provided for backward compatibility
and it may be removed in future versions.

If the detection mechanism is to be used,
it is mandatory to correctly specify
the filename of the main file as the argument of |\childdocmain|:
%
\begin{center}
\begin{tabular}{l}
|\input{childdoc.def}|\\
|\childdocmain{|\textit{main}|}|\\
\end{tabular}
\end{center}
%
If |\jobname| does not match the argument \textit{main} of |\childdocmain|,
it is assumed that |\jobname| points to the child file to be compiled.
When using |\childdocmain| with the main file specified as argument,
it suffices to start a child file
with just |\input{|\textit{main}|}|
without loading of the package and using |\childdocof|.
If instead all processing is done
with the appropriate \textsf{childdoc} directives,
the argument of \textit{main} of |\childdocmain| can be empty.

An alternative version of the command line processing described
in \secref{sec:commandline} using the detection mechanism reads:
%
\begin{center}
|... -jobname "|\textit{target}|" "|[\textit{flags}]%
[|\def\jobname{|\textit{dest}|}|]|\input{|\textit{main}|}"|
\end{center}

%%%%%%%%%%%%%%%%%%%%%%%%%%%%%%%%%%%%%%%%%%%%%%%%%%%%%%%%%%%%%%%%%%%%%%%%%%%%%%%%
\subsection{Manual Code}
\label{sec:manual}

In case one cannot be certain whether the definitions file |childdoc.def|
is installed on the target \TeX{} distribution
and one prefers not to ship it,
it is conceivable to paste a few relevant commands into the sources.

To that end, drop all statements |\input{childdoc.def}|
and perform the replacements as outlined below.
Instead of |\childdocmain{|\textit{main}|}| add the following code
to the top of the main file:
%
\begin{center}
\begin{tabular}{l}
|\||ifdefined\childdocname\endinput\||fi\newif\ifchilddoc|\\
|\edef\childdocname{\scantokens\expandafter{\jobname\noexpand}}|\\
|\def\childdocmain{|\textit{main}|}\||ifx\childdocmain\childdocname\||else|\\
|\childdoctrue\includeonly{\childdocname}\let\jobname\childdocmain\||fi|\\
\end{tabular}
\end{center}
%
Instead of |\childdocof{|\textit{main}|}| just include the main file
at the top of each child file:
%
\begin{center}
|\input{|\textit{main}|}|
\end{center}
%
A simple redirection |\childdocforward{|\textit{dest}|}| is achieved by:
%
\begin{center}
|\def\jobname{|\textit{dest}|}\input{\jobname}|
\end{center}
%
The redirection with prefix
|\childdocforwardprefix[|\textit{prefix}|]{|\textit{dest}|}|
is accomplished by:
%
\begin{center}
\begin{tabular}{l}
|{\edef\jobname{\scantokens\expandafter{\jobname\noexpand}}|\\
|\def\redirectjob |\textit{prefix}|#1~~~{\gdef\jobname{|\textit{dest}|#1}}|\\
|\expandafter\redirectjob\jobname~~~}\input{\jobname}|
\end{tabular}
\end{center}

In an alternative approach,
child documents can be compiled by a specific command line
without additional code or specific definitions:
%
\begin{center}
|... -jobname "|\textit{target}|" "|[\textit{flags}]%
|\includeonly{|\textit{dest}|}\input{|\textit{main}|}"|
\end{center}
%

%%%%%%%%%%%%%%%%%%%%%%%%%%%%%%%%%%%%%%%%%%%%%%%%%%%%%%%%%%%%%%%%%%%%%%%%%%%%%%%%
%%%%%%%%%%%%%%%%%%%%%%%%%%%%%%%%%%%%%%%%%%%%%%%%%%%%%%%%%%%%%%%%%%%%%%%%%%%%%%%%
\section{Information}

%%%%%%%%%%%%%%%%%%%%%%%%%%%%%%%%%%%%%%%%%%%%%%%%%%%%%%%%%%%%%%%%%%%%%%%%%%%%%%%%
\subsection{Copyright}

Copyright \copyright{} 2017--2018 Niklas Beisert

This work may be distributed and/or modified under the
conditions of the \LaTeX{} Project Public License, either version 1.3
of this license or (at your option) any later version.
The latest version of this license is in
  \url{http://www.latex-project.org/lppl.txt}
and version 1.3 or later is part of all distributions of \LaTeX{}
version 2005/12/01 or later.

This work has the LPPL maintenance status `maintained'.

The Current Maintainer of this work is Niklas Beisert.

This work consists of the files |README.txt|, |childdoc.ins| and |childdoc.dtx|
as well as the derived files |childdoc.def|, |cdocsamp.tex|
with |cdocsch1.tex|, |cdocsch2.tex|, |cdocspt3.tex|, |cdocspt4.tex|,
|cdocsdrf.tex|, |cdocsfn1.tex|, |cdocsfn2.tex|
as well as |childdoc.pdf|.

%%%%%%%%%%%%%%%%%%%%%%%%%%%%%%%%%%%%%%%%%%%%%%%%%%%%%%%%%%%%%%%%%%%%%%%%%%%%%%%%
\subsection{Files and Installation}

The package consists of the files:
%
\begin{center}
\begin{tabular}{ll}
    |README.txt|   & readme file \\
    |childdoc.ins| & installation file \\
    |childdoc.dtx| & source file \\
    |childdoc.def| & definition file \\
    |cdocsamp.tex| & sample main file \\
    |cdocsch1.tex| & sample include file \\
    |cdocsch2.tex| & sample include file \\
    |cdocspt3.tex| & sample part file \\
    |cdocspt4.tex| & sample part file \\
    |cdocsdrf.tex| & sample redirection file \\
    |cdocsfn1.tex| & sample redirection file \\
    |cdocsfn2.tex| & sample redirection file \\
    |childdoc.pdf| & manual
\end{tabular}
\end{center}
%
The distribution consists of the files
|README.txt|, |childdoc.ins| and |childdoc.dtx|.
%
\begin{itemize}
\item
Run (pdf)\LaTeX{} on |childdoc.dtx|
to compile the manual |childdoc.pdf| (this file).
\item
Run \LaTeX{} on |childdoc.ins| to create the definitions file |childdoc.def|
and the sample |cdocsamp.tex| with include files
|cdocsch1.tex|, |cdocsch2.tex|, |cdocspt3.tex|, |cdocspt4.tex|,
|cdocsdrf.tex|, |cdocsfn1.tex|, |cdocsfn2.tex|.
Then copy the file |childdoc.def| to an appropriate directory of your \LaTeX{}
distribution, e.g.\ \textit{texmf-root}|/tex/latex/childdoc|.
\end{itemize}

%%%%%%%%%%%%%%%%%%%%%%%%%%%%%%%%%%%%%%%%%%%%%%%%%%%%%%%%%%%%%%%%%%%%%%%%%%%%%%%%
\subsection{Related CTAN Packages}

There are several other packages which offer a similar functionality:
%
\begin{itemize}
\item
The packages
\href{http://ctan.org/pkg/docmute}{\textsf{docmute}},
\href{http://ctan.org/pkg/includex}{\textsf{includex}} and
\href{http://ctan.org/pkg/standalone}{\textsf{standalone}}
provide commands to include only the document body of
a child file thus allowing both files to be compiled individually.
\item
The packages \href{http://ctan.org/pkg/subdocs}{\textsf{subdocs}}
and \href{http://ctan.org/pkg/subfiles}{\textsf{subfiles}}
provide structures in which the main and child documents can be
encapsulated and allowing them to be compiled individually.
The inclusion mechanism is different from the conventional |\include|.
\item
The package \href{http://ctan.org/pkg/combine}{\textsf{combine}}
is an elaborate solution to combine several documents into one.
\end{itemize}
%
See also the CTAN topic \href{http://ctan.org/topic/subdocs}{\textsf{subdocs}}
for further related packages.
The present package differs from the above solutions in that
a document structure constructed with the conventional |\include| mechanism
just needs two extra commands at the top of every file
such that all constituent files can be compiled individually.

%%%%%%%%%%%%%%%%%%%%%%%%%%%%%%%%%%%%%%%%%%%%%%%%%%%%%%%%%%%%%%%%%%%%%%%%%%%%%%%%
%\subsection{Feature Suggestions}
%
%The following is a list of features which may be useful for future
%versions of this package:
%%
%\begin{itemize}
%\item
%\ldots
%\end{itemize}

%%%%%%%%%%%%%%%%%%%%%%%%%%%%%%%%%%%%%%%%%%%%%%%%%%%%%%%%%%%%%%%%%%%%%%%%%%%%%%%%
\subsection{Revision History}

%%%%%%%%%%%%%%%%%%%%%%%%%%%%%%%%%%%%%%%%
\paragraph{v2.0:} 2018/12/30

\begin{itemize}
\item
immediate forward processing
\item
added |\childdocby| mechanism
\item
manual restructured
\end{itemize}

%%%%%%%%%%%%%%%%%%%%%%%%%%%%%%%%%%%%%%%%
\paragraph{v1.6:} 2018/01/17

\begin{itemize}
\item
application for development of include files
\item
corrections to manual
\end{itemize}

%%%%%%%%%%%%%%%%%%%%%%%%%%%%%%%%%%%%%%%%
\paragraph{v1.5:} 2017/05/21

\begin{itemize}
\item
more complete structuring introduced
\item
|\childdocof| introduced
\item
|\childdoc| renamed to |\childdocmain|
\item
|\childredirect| renamed to |\childdocforward| and |\childdocforwardprefix|
and functionality expanded
\end{itemize}

%%%%%%%%%%%%%%%%%%%%%%%%%%%%%%%%%%%%%%%%
\paragraph{v1.0:} 2017/04/27

\begin{itemize}
\item
manual and install package
\item
first version published on CTAN
\end{itemize}

%%%%%%%%%%%%%%%%%%%%%%%%%%%%%%%%%%%%%%%%
\paragraph{v0.6:} 2017/04/26

\begin{itemize}
\item
redirection mechanism added
\end{itemize}

%%%%%%%%%%%%%%%%%%%%%%%%%%%%%%%%%%%%%%%%
\paragraph{v0.5:} 2017/04/26

\begin{itemize}
\item
functionality in definition file
\end{itemize}


%%%%%%%%%%%%%%%%%%%%%%%%%%%%%%%%%%%%%%%%%%%%%%%%%%%%%%%%%%%%%%%%%%%%%%%%%%%%%%%%
%%%%%%%%%%%%%%%%%%%%%%%%%%%%%%%%%%%%%%%%%%%%%%%%%%%%%%%%%%%%%%%%%%%%%%%%%%%%%%%%
%%%%%%%%%%%%%%%%%%%%%%%%%%%%%%%%%%%%%%%%%%%%%%%%%%%%%%%%%%%%%%%%%%%%%%%%%%%%%%%%
\appendix

\settowidth\MacroIndent{\rmfamily\scriptsize 000\ }

 \DocInput{childdoc.dtx}

\end{document}
%</driver>
% \fi
%
% %%%%%%%%%%%%%%%%%%%%%%%%%%%%%%%%%%%%%%%%%%%%%%%%%%%%%%%%%%%%%%%%%%%%%%%%%%%%%%
% %%%%%%%%%%%%%%%%%%%%%%%%%%%%%%%%%%%%%%%%%%%%%%%%%%%%%%%%%%%%%%%%%%%%%%%%%%%%%%
% \section{Sample}
%\iffalse
%<*samplemain>
%\fi
%
% The following presents a sample document
% with two chapters, two parts, a title page,
% a compile flag as well as three forwarding files to set the flag.
% It consists of eight |.tex| files:
% \begin{center}
% \begin{tabular}{ll}
% |cdocsamp.tex|&main file\\
% |cdocsch1.tex|&include file for chapter 1\\
% |cdocsch2.tex|&include file for chapter 2\\
% |cdocspt3.tex|&include file for part 3\\
% |cdocspt4.tex|&include file for part 4\\
% |cdocsdrf.tex|&forwarding file for main file in draft mode\\
% |cdocsfi1.tex|&forwarding file for final version of chapter 1\\
% |cdocsfi2.tex|&forwarding file for final version of chapter 2\\
% \end{tabular}
% \end{center}
% Each of the eight files can be compiled directly by the \LaTeX{} compiler.
%
% %%%%%%%%%%%%%%%%%%%%%%%%%%%%%%%%%%%%%%
% \paragraph{Main File.}
%
% The main file is called |cdocsamp.tex|.
%
% Load the \textsf{childdoc} definitions and
% declare the filename for the main document:
%    \begin{macrocode}
\input{childdoc.def}
\childdocmain{}
%    \end{macrocode}

% Optional override for |\version| flag:
%    \begin{macrocode}
%%\ifchilddoc\else\providecommand{\version}{draft}\fi
%    \end{macrocode}

% Define the default values for the |\version| flag
% (|final| for the main file and |draft| for childs):
%    \begin{macrocode}
\ifchilddoc
\providecommand{\version}{draft}
\else
\providecommand{\version}{final}
\fi
%    \end{macrocode}

% Load the standard document class:
%    \begin{macrocode}
\documentclass[12pt]{article}
%    \end{macrocode}

% Start the document body:
%    \begin{macrocode}
\begin{document}
%    \end{macrocode}

% Declare a title page.
% Print title, part of document being processed and version flag:
%    \begin{macrocode}
\addtocounter{page}{-1}
\begin{center}
{\LARGE\bfseries{}childdoc example\par}
\vspace{1cm}
\ifchilddoc
\ifchilddocmanual part\else chapter\fi:
`\childdocname' of `\childdocjob'\par
\else
main document: `\childdocjob'\par
\fi
version: \version\par
\end{center}
\newpage
%    \end{macrocode}

% Manually include selected file,
% otherwise process as usual:
%    \begin{macrocode}
\ifchilddocmanual
\section*{part `\childdocname'}
\input{\childdocname}
\else
%    \end{macrocode}

% Include the two chapters:
%    \begin{macrocode}
\include{cdocsch1}
\include{cdocsch2}
%    \end{macrocode}

% Include the two parts unless only chapters should be displayed:
%    \begin{macrocode}
\ifchilddoc\else
\section{part three}
\input{cdocspt3}
\section{part four}
\input{cdocspt4}
\fi
%    \end{macrocode}

% Process as usual until here:
%    \begin{macrocode}
\fi
%    \end{macrocode}

% End of document body:
%    \begin{macrocode}
\end{document}
%    \end{macrocode}
%\iffalse
%</samplemain>
%\fi
%
% %%%%%%%%%%%%%%%%%%%%%%%%%%%%%%%%%%%%%%
% \paragraph{Chapter Include Files.}
%
% The include files are called |cdocsch1.tex| and |cdocsch2.tex|.
%
%\iffalse
%<*samplechap1|samplechap2>
%\fi

% Optional override for |\version| flag:
%    \begin{macrocode}
%%\providecommand{\version}{final}
%    \end{macrocode}

% Include the main document:
%    \begin{macrocode}
\input{childdoc.def}
\childdocof{cdocsamp}
%    \end{macrocode}

%\iffalse
%</samplechap1|samplechap2>
%\fi
%
%\iffalse
%<*samplechap1>
%\fi
% Some text for chapter 1:
%    \begin{macrocode}
\section{one}
some text in chapter one
%    \end{macrocode}

%\iffalse
%</samplechap1>
%\fi
% Some text for chapter 2:
%\iffalse
%<*samplechap2>
%\fi
%    \begin{macrocode}
\section{two}
more text in chapter two
%    \end{macrocode}

%\iffalse
%</samplechap2>
%\fi
%
% %%%%%%%%%%%%%%%%%%%%%%%%%%%%%%%%%%%%%%
% \paragraph{Part Include Files.}
%
% The include files are called |cdocspt3.tex| and |cdocspt4.tex|.
%
%\iffalse
%<*samplepart3|samplepart4>
%\fi

% Optional override for |\version| flag:
%    \begin{macrocode}
%%\providecommand{\version}{final}
%    \end{macrocode}

% Include the main document:
%    \begin{macrocode}
\input{childdoc.def}
\childdocby{cdocsamp}
%    \end{macrocode}

%\iffalse
%</samplepart3|samplepart4>
%\fi
%
%\iffalse
%<*samplepart3>
%\fi
% Some text for part 3:
%    \begin{macrocode}
some text in part three
%    \end{macrocode}

%\iffalse
%</samplepart3>
%\fi
% Some text for part 4:
%\iffalse
%<*samplepart4>
%\fi
%    \begin{macrocode}
more text in part four
%    \end{macrocode}

%\iffalse
%</samplepart4>
%\fi
%
% %%%%%%%%%%%%%%%%%%%%%%%%%%%%%%%%%%%%%%
% \paragraph{Forwarding for a Complete Draft.}
%
% The following forwarding file |cdocsdrf.tex|
% compiles the main document in draft mode:
%\iffalse
%<*sampledraft>
%\fi
%    \begin{macrocode}
\def\version{draft}
\input{childdoc.def}
\childdocforward{cdocsamp}
%    \end{macrocode}

%\iffalse
%</sampledraft>
%\fi
%
% %%%%%%%%%%%%%%%%%%%%%%%%%%%%%%%%%%%%%%
% \paragraph{Forwarding for Final Version of the Chapters.}
%
% The following forwarding files |cdocsfn1.tex| and |cdocsfn2.tex|
% (with identical content)
% compile the final versions of the child documents
% |cdocsch1.tex| and |cdocsch2.tex|, respectively:
%\iffalse
%<*samplefinal>
%\fi
%    \begin{macrocode}
\def\version{final}
\input{childdoc.def}
\childdocforwardprefix[cdocsamp]{cdocsfn}{cdocsch}
%    \end{macrocode}

%\iffalse
%</samplefinal>
%\fi
%
% %%%%%%%%%%%%%%%%%%%%%%%%%%%%%%%%%%%%%%
% \paragraph{Command Line Processing.}
%
% The following three command lines generate the output files
% |cdocscld|, |cdocscl1| and |cdocscl2|
% which should be identical to
% |cdocsdrf|, |cdocsch1| and |cdocsfn2|, respectively:
% \begin{center}
% \begin{tabular}{l}
% |latex -jobname cdocscld \|\\
% |  "\def\version{draft}\input{childdoc.def}\childdocforward{cdocsamp}"|\\
% |latex -jobname cdocscl1 \|\\
% |  "\input{childdoc.def}\childdocforward[cdocsamp]{cdocsch1}"|\\
% |latex -jobname cdocscl2 \|\\
% |  "\def\version{final}\input{childdoc.def}\childdocforward{cdocsch2}"|
% \end{tabular}
% \end{center}
% Note that the trailing backslash on each first line
% merely continues the input to the second line
% (for convenient cut ant paste).
% Furthermore, the command |latex| can be replaced by any
% of its alternative versions such as |pdflatex|.
%
% %%%%%%%%%%%%%%%%%%%%%%%%%%%%%%%%%%%%%%%%%%%%%%%%%%%%%%%%%%%%%%%%%%%%%%%%%%%%%%
% %%%%%%%%%%%%%%%%%%%%%%%%%%%%%%%%%%%%%%%%%%%%%%%%%%%%%%%%%%%%%%%%%%%%%%%%%%%%%%
% \section{Implementation}
%\iffalse
%<*package>
%\fi
%
% This section describes the definitions file |childdoc.def|.

% The definitions cannot be loaded using |\usepackage| or |\RequirePackage|
% which has a mechanism to prevent loading a style file more than once.
% When loading the definitions by means of |\input|
% multiple instances have to be prevented manually:
%\iffalse
%This code needs to be before the `\ProvidesFile' directive
%which is defined at the beginning of this file.
%Therefore it is also placed there and commented out here.
%</package>
%<*discard>
%\fi
%    \begin{macrocode}
\ifdefined\childdocmain\endinput\fi
%    \end{macrocode}
%\iffalse
%</discard>
%<*package>
%\fi
%
% \macro{\ifchilddoc}
% \macro{\ifchilddocmanual}
% The conditional |\ifchilddoc| tells whether a
% child (true) or main (false) document is being compiled.
% The conditional |\ifchilddocmanual| tells whether
% the |\includeonly| mechanism is used (false) or
% the selection of child files must be performed manually (true).
% The definitions initialise to false:
%    \begin{macrocode}
\newif\ifchilddoc
\newif\ifchilddocmanual
%    \end{macrocode}

% \macro{\childdocname}
% \macro{\childdocjob}
% The macro |\childdocname| stores the name of the main document
% to be compiled. The macro |\childdocjob| stores the name of
% the document on which the \LaTeX{} compiler was originally invoked.
% The content of |\jobname| cannot be compared
% to filenames specified in the source due to different catcodes.
% The following code rescans |\jobname|, stores the result
% in |\childdocname| and saves a copy in |\childdocjob|:
%    \begin{macrocode}
\edef\childdocname{\scantokens\expandafter{\jobname\noexpand}}
\let\childdocjob\childdocname
%    \end{macrocode}

% \macro{\childdocdisable}
% The macro |\childdocdisable| prevents the main file
% from being processed more than once.
% At this stage, the main document command |\childdocmain|
% is assumed to be called once again where it should do nothing.
% Any subsequent call to it should prevent
% a secondary processing of the main document
% It overwrites the forwarding commands
% |\childdocof| and |\childdocforward|
% with empty macros to prevent further inclusions of the main document:
%    \begin{macrocode}
\newcommand{\childdocdisable}
{
  \renewcommand{\childdocmain}[1]{\renewcommand{\childdocmain}[1]{\endinput}}
  \renewcommand{\childdocof}[1]{}
  \renewcommand{\childdocby}[2][]{}
  \renewcommand{\childdocforward}[2][]{}
  \renewcommand{\childdocdisable}{}
}
%    \end{macrocode}

% \macro{\childdocmain}
% The macro |\childdocmain| is to be called at the top of the main file
% with nothing or the main filename (without extension) as argument.
% First, it breaks loops.
% If the argument is not empty and does not match |\childdocname|
% (which is set by the first inclusion of |childdoc.def|),
% |\ifchilddoc| is set to true, |\includeonly| is applied to the child file
% and |\jobname| is set to the main file
% (for proper handling of |.aux| files):
%    \begin{macrocode}
\newcommand{\childdocmain}[1]
{
  \childdocdisable\childdocmain{}
  \if?#1?\else
    \begingroup
      \def\childdoctmp{#1}
      \ifx\childdoctmp\childdocname
        \def\childdoctmp{}
      \else
        \def\childdoctmp
        {
          \childdoctrue
          \includeonly{\childdocname}
          \def\childdocjob{#1}
          \def\jobname{#1}
        }
      \fi
      \expandafter
    \endgroup
    \childdoctmp
  \fi
}
%    \end{macrocode}

% \macro{\childdocof}
% The command |\childdocof| redirects
% compilation to the main file |#1|.
%    \begin{macrocode}
\newcommand{\childdocof}[1]
{
  \childdocdisable
  \childdoctrue
  \includeonly{\childdocname}
  \def\jobname{#1}
  \def\childdocjob{#1}
  \input{#1}
}
%    \end{macrocode}

% \macro{\childdocby}
% The command |\childdocby| ....
%    \begin{macrocode}
\newcommand{\childdocby}[2][]
{
  \childdocdisable
  \childdoctrue
  \childdocmanualtrue
  \if?#1?\else
    \def\jobname{#2}
  \fi
  \def\childdocjob{#2}
  \input{#2}
  \endinput
}
%    \end{macrocode}

% \macro{\childdocforward}
% The command |\childdocforward| redirects
% compilation to the main file or
% (if the optional argument is given) a child file.
% Parameters are set as if the main file
% or a child file starting with |\childdocof| was compiled.
% Then compilation is handed over to the main file:
%    \begin{macrocode}
\newcommand{\childdocforward}[2][]
{
  \begingroup
    \if?#1?
      \def\childdoctmp
      {
        \def\childdocname{#2}
        \def\childdocjob{#2}
        \def\jobname{#2}
        \input{#2}
        \endinput
      }
    \else
      \def\childdoctmp
      {
        \childdocdisable
        \def\childdocname{#2}
        \childdoctrue
        \includeonly{#2}
        \def\childdocjob{#1}
        \def\jobname{#1}
        \input{#1}
        \endinput
      }
    \fi
    \expandafter
  \endgroup
  \childdoctmp
}
%    \end{macrocode}

% \macro{\childdocforwardprefix}
% The command |\childdocforwardprefix| redirects
% compilation to the main or a child file by means of a pattern.
% The prefix |#1| in the current filename is replaced by |#2|
% and the suffix of the current filename is kept
% (it is assumed that the filename does not contain the substring `|~~~|'
% which is used as a delimiter).
% Compilation is handed over to the new file by |\childdocforward|:
%    \begin{macrocode}
\newcommand{\childdocforwardprefix}[3][]
{
  \begingroup
    \def\childdocextract #2##1~~~{\def\childdoctmp{\childdocforward[#1]{#3##1}}}
    \expandafter\childdocextract\childdocname~~~
    \expandafter
  \endgroup
  \childdoctmp
}
%    \end{macrocode}

% \macro{\childdoc}
% The deprecated macro |\childdoc| is a legacy version of |\childdocmain|:
%    \begin{macrocode}
\newcommand{\childdoc}{\childdocmain}
%    \end{macrocode}

% \macro{\childdocredirect}
% The deprecated macro |\childdocredirect| is a legacy version
% of |\childdocforward| and |\childdocforwardprefix|:
%    \begin{macrocode}
\newcommand{\childdocredirect}[2][]
{
  \begingroup
    \if?#1?
      \def\childdoctmp{\childdocforward{#2}}
    \else
      \def\childdoctmp{\childdocforwardprefix{#1}{#2}}
    \fi
    \expandafter
  \endgroup
  \childdoctmp
}
%    \end{macrocode}

%\iffalse
%</package>
%\fi
%
\endinput

\childdocby{cdocsamp}
%    \end{macrocode}

%\iffalse
%</samplepart3|samplepart4>
%\fi
%
%\iffalse
%<*samplepart3>
%\fi
% Some text for part 3:
%    \begin{macrocode}
some text in part three
%    \end{macrocode}

%\iffalse
%</samplepart3>
%\fi
% Some text for part 4:
%\iffalse
%<*samplepart4>
%\fi
%    \begin{macrocode}
more text in part four
%    \end{macrocode}

%\iffalse
%</samplepart4>
%\fi
%
% %%%%%%%%%%%%%%%%%%%%%%%%%%%%%%%%%%%%%%
% \paragraph{Forwarding for a Complete Draft.}
%
% The following forwarding file |cdocsdrf.tex|
% compiles the main document in draft mode:
%\iffalse
%<*sampledraft>
%\fi
%    \begin{macrocode}
\def\version{draft}
% \iffalse
%
% childdoc.dtx Copyright (C) 2017-2018 Niklas Beisert
%
% This work may be distributed and/or modified under the
% conditions of the LaTeX Project Public License, either version 1.3
% of this license or (at your option) any later version.
% The latest version of this license is in
%   http://www.latex-project.org/lppl.txt
% and version 1.3 or later is part of all distributions of LaTeX
% version 2005/12/01 or later.
%
% This work has the LPPL maintenance status `maintained'.
%
% The Current Maintainer of this work is Niklas Beisert.
%
% This work consists of the files childdoc.dtx and childdoc.ins
% and the derived files childdoc.def and cdocsamp.tex with
% cdocsch1.tex, cdocsch2.tex, cdocsdrf.tex, cdocsfn1.tex, cdocsfn2.tex.
%
%<package>\ifdefined\childdocmain\endinput\fi
%<package>\ProvidesFile{childdoc.def}[2018/12/30 v2.0 child document driver]
%<samplemain>\ProvidesFile{cdocsamp.tex}[2018/12/30 v2.0 sample for childdoc]
%<*driver>
%\ProvidesFile{childdoc.drv}[2018/12/30 v2.0 childdoc reference manual file]
\PassOptionsToClass{10pt,a4paper}{article}
\documentclass{ltxdoc}

\usepackage[margin=35mm]{geometry}
\usepackage{hyperref}
\usepackage{hyperxmp}
\usepackage[usenames]{color}

\hypersetup{colorlinks=true}
\hypersetup{pdfstartview=FitH}
\hypersetup{pdfpagemode=UseNone}
\hypersetup{pdfsource={}}
\hypersetup{pdflang={en-UK}}
\hypersetup{pdfcopyright={Copyright 2017-2018 Niklas Beisert.
  This work may be distributed and/or modified under the
  conditions of the LaTeX Project Public License, either version 1.3
  of this license or (at your option) any later version.}}
\hypersetup{pdflicenseurl={http://www.latex-project.org/lppl.txt}}
\hypersetup{pdfcontactaddress={ETH Zurich, ITP, HIT K,
  Wolfgang-Pauli-Strasse 27}}
\hypersetup{pdfcontactpostcode={8093}}
\hypersetup{pdfcontactcity={Zurich}}
\hypersetup{pdfcontactcountry={Switzerland}}
\hypersetup{pdfcontactemail={nbeisert@itp.phys.ethz.ch}}
\hypersetup{pdfcontacturl={http://people.phys.ethz.ch/\xmptilde nbeisert/}}

\newcommand{\secref}[1]{\hyperref[#1]{section \ref*{#1}}}

\parskip1ex
\parindent0pt
\let\olditemize\itemize
\def\itemize{\olditemize\parskip0pt}

\begin{document}

\title{The \textsf{childdoc} Package}
\hypersetup{pdftitle={The childdoc Package}}
\author{Niklas Beisert\\[2ex]
  Institut f\"ur Theoretische Physik\\
  Eidgen\"ossische Technische Hochschule Z\"urich\\
  Wolfgang-Pauli-Strasse 27, 8093 Z\"urich, Switzerland\\[1ex]
  \href{mailto:nbeisert@itp.phys.ethz.ch}
  {\texttt{nbeisert@itp.phys.ethz.ch}}}
\hypersetup{pdfauthor={Niklas Beisert}}
\hypersetup{pdfsubject={Manual for the LaTeX2e Package childdoc}}
\date{30 December 2018, \textsf{v2.0}}
\maketitle

\begin{abstract}\noindent
\textsf{childdoc} is a \LaTeXe{} package
that enables the direct compilation
of document sections included by |\include|
to individual files.
\end{abstract}

\begingroup
\parskip0ex
\tableofcontents
\endgroup

%%%%%%%%%%%%%%%%%%%%%%%%%%%%%%%%%%%%%%%%%%%%%%%%%%%%%%%%%%%%%%%%%%%%%%%%%%%%%%%%
%%%%%%%%%%%%%%%%%%%%%%%%%%%%%%%%%%%%%%%%%%%%%%%%%%%%%%%%%%%%%%%%%%%%%%%%%%%%%%%%
\section{Introduction}

\LaTeX{} provides a mechanism to structure a large document (such as a book)
into a main file and several child files (containing the chapters)
using the |\include| command.
This mechanism is beneficial for documents
which span hundreds of pages in order to
make the source file(s) more manageable.
Moreover, compilation can be restricted to
selected child files by means of the |\includeonly| command.
The latter feature can be used to reduce the compilation time while editing
(this was significantly more useful in the earlier days of \LaTeX{})
or to generate a smaller document which is easier to navigate.
Another application of |\includeonly| is to generate
documents consisting of selected parts of the complete document.

However, there are a few drawbacks of the plain |\include| mechanism:
\begin{itemize}
\item
The child files cannot be compiled on their own,
they can only be compiled via the main file.
A naive editing environment
(such as a text editor with an option
to have the current file processed by \LaTeX)
may require one to switch to the main file before compiling;
attempting to compile the child file produces errors.
\item
The main file must be modified (each time)
to adjust the |\includeonly| command
to the present needs. This easily leaves the main file in a messy state.
\item
The generated document will always carry the filename
of the main document. This is inconvenient if
several child files are to be compiled and
to be kept for distribution.
\end{itemize}

The present package provides a simple interface
to make child files individually compilable by \LaTeX{}.
Compiling a child file then has the same effect as compiling
the main file with an |\includeonly| command
to select the appropriate child.
Moreover the generated document will carry the name of the child
rather than the main file.
This resolves all three above issues.

This feature is meant to make the editing of books,
thesis documents and lecture notes somewhat more convenient.
However, the package can also be used efficiently for
composing a series of documents (such as exercise sheets)
which are typically distributed individually.
It then assists the author in generating the individual documents
(potentially in different versions)
as well as a document containing the collected series.
Another application is in developing style files
or other kinds of included material
where compilation of the style file could redirect
to a sample or test file.

%%%%%%%%%%%%%%%%%%%%%%%%%%%%%%%%%%%%%%%%%%%%%%%%%%%%%%%%%%%%%%%%%%%%%%%%%%%%%%%%
%%%%%%%%%%%%%%%%%%%%%%%%%%%%%%%%%%%%%%%%%%%%%%%%%%%%%%%%%%%%%%%%%%%%%%%%%%%%%%%%
\section{Usage}

First of all, the package \textsf{childdoc} is \emph{not} a standard
\LaTeXe{} |.sty| style file! Therefore it needs to be invoked in
a non-standard way.

%%%%%%%%%%%%%%%%%%%%%%%%%%%%%%%%%%%%%%%%%%%%%%%%%%%%%%%%%%%%%%%%%%%%%%%%%%%%%%%%
\subsection{Included Files}
\label{sec:include}

%%%%%%%%%%%%%%%%%%%%%%%%%%%%%%%%%%%%%%%%
\DescribeMacro{\childdocmain}
To use the package, add the commands
\begin{center}
\begin{tabular}{l}
|\input{childdoc.def}|\\
|\childdocmain{}|\\
\end{tabular}
\end{center}
at the very top of the main \LaTeX{} file,
in particular \emph{before} the |\documentclass| statement!
The argument of |\childdocmain| should be left empty
(but it must be present).

%%%%%%%%%%%%%%%%%%%%%%%%%%%%%%%%%%%%%%%%
\DescribeMacro{\childdocof}
Furthermore, add the commands
\begin{center}
\begin{tabular}{l}
|\input{childdoc.def}|\\
|\childdocof{|\textit{main}|}|\\
\end{tabular}
\end{center}
at the top of every child file \textit{child}
which is included by |\include{|\textit{child}|}|
from within the main file
(or at least for those files to be compiled individually).
The argument \textit{main} must be the filename of the main file.

There are a couple of
considerations in setting up the main and child documents:

%%%%%%%%%%%%%%%%%%%%%%%%%%%%%%%%%%%%%%%%
\paragraph{Restrictions.}

Please note the following restrictions:
\begin{itemize}
\item
|\childdocmain| must be called with one argument \textit{main}
to ensure compatibility with earlier version of the package.
It must either be empty (|\childdocmain{}|)
or precisely match the filename of the main file in which it is specified.
See \secref{sec:detection} for further information.
\item
The filename \textit{main} must be specified without the |.tex| extension.
\item
The filename \textit{main} is case sensitive
(even in case-insensitive file systems)
due to internal string comparison.
\item
The argument \textit{main} should be fully expanded, it cannot be a macro.
\item
Subdirectories and special characters should be avoided in filenames.
\item
The command |\childdocmain{|\textit{main}|}| must be followed by a whitespace.
It should not be followed immediately by another command
or by a comment mark `|%|'.
This is because the \TeX{} parser reads the token immediately following
the argument of |\childdocmain| and puts it
at the beginning of every child section;
however, a white\-space is ignored.
\end{itemize}

%%%%%%%%%%%%%%%%%%%%%%%%%%%%%%%%%%%%%%%%
\paragraph{Content of Main File.}

It is advisable to place all content in the child files included by |\include|.
Any output contained in the main file will appear in all child documents
unless suppressed manually;
it cannot be suppressed automatically by the |\includeonly| directive
and thus should normally be avoided.
A method to include some content in the main file
by means of conditional processing is described in \secref{sec:conditional}.

%%%%%%%%%%%%%%%%%%%%%%%%%%%%%%%%%%%%%%%%
\paragraph{Page Numbering.}

When only a part of the document is compiled,
the appropriate numbering of pages
(as well as other status parameters)
is determined from the |.aux| files.
The latter contain information from previous passes.
However this information needs to propagate through
all intermediate child documents.
Therefore the page numbering in child documents may well
be inconsistent until the complete document is compiled at least once.

A useful (if unconventional) way to always ensure a consistent
page numbering is to restart the numbering in each child document
and denote the pages by `\textit{child}|.|\textit{page}'
where \textit{child} represents the chapter/section number of the child file.
This can be achieved by the command
|\numberwithin{page}{|\textit{child}|}|
of the \textsf{amsmath} package
where \textit{child} can be |chapter| or |section|
depending on the chosen structuring.
Alternatively, one can modify the macro |\thepage| appropriately
and reset the counter |page| at the start of each child file.

%%%%%%%%%%%%%%%%%%%%%%%%%%%%%%%%%%%%%%%%%%%%%%%%%%%%%%%%%%%%%%%%%%%%%%%%%%%%%%%%
\subsection{Conditional Processing}
\label{sec:conditional}

The package provides a mechanism to compile different versions
of a document. To customise the versions further some conditional processing
can come in handy to distinguish which version is being compiled.
The package provides two macros to describe the compilation context:

%%%%%%%%%%%%%%%%%%%%%%%%%%%%%%%%%%%%%%%%
\DescribeMacro{\ifchilddoc}
The conditional |\ifchilddoc| distinguishes between the compilation of
child documents and the main document:
%
\begin{center}
|\ifchilddoc |\textit{child-code}| |[|\||else |\textit{main-code}]| \||fi|
\end{center}

%%%%%%%%%%%%%%%%%%%%%%%%%%%%%%%%%%%%%%%%
\DescribeMacro{\childdocname}
\DescribeMacro{\childdocjob}
The macro |\childdocname| contains the filename (without extension)
of the main or child file being processed.
Note that |\childdocjob| will always contain the name of the main file.

%%%%%%%%%%%%%%%%%%%%%%%%%%%%%%%%%%%%%%%%
\paragraph{Title Page.}

Conditional processing can be used to include a title or banner page
in the main document when proper precautions are taken.
Importantly, the code in the main file should ensure that the page counter
(as well as other status parameters which are stored in the |.aux| files)
takes the same value after the conditional processing.
Otherwise the page numbers may take divergent values
depending on which part is compiled.

For example, a title page could be declared by:
%
\begin{center}
\begin{tabular}{l}
|\ifchilddoc\||else|\\
|\addtocounter{page}{-1}|\\
\textit{code for title page}\\
|\newpage|\\
|\||fi|
\end{tabular}
\end{center}
%
A banner page for the child documents can be generated by:
%
\begin{center}
\begin{tabular}{l}
|\ifchilddoc|\\
|\addtocounter{page}{-1}|\\
\textit{code for banner page}\\
|\newpage|\\
|\||fi|
\end{tabular}
\end{center}
%
Here one could write a message such as:
\begin{center}
|This is the part \childdocname{} of \childdocjob{}.|
\end{center}

%%%%%%%%%%%%%%%%%%%%%%%%%%%%%%%%%%%%%%%%%%%%%%%%%%%%%%%%%%%%%%%%%%%%%%%%%%%%%%%%
\subsection{Flags}
\label{sec:flags}

The package makes it easy to generate different versions
of the main or child documents.
To this end compilation flags can be defined
and assigned different default values.
They will be particularly useful in conjunction
with the forwarding mechanism described in \secref{sec:forward}.

For example, it may be useful to have a flag |\version|
which can be set to |draft| or |final|.
The document source will contain some conditional code
depending on the value of |\version|.
Suppose further, the flag should default to |final| for the main file
and to |draft| for child files
which is a natural assignment for editing the document.
This is achieved by placing the following code
in the preamble of the main document
(below the |\childdocmain| directive):
%
\begin{center}
\begin{tabular}{l}
|\ifchilddoc|\\
|\providecommand{\version}{draft}|\\
|\||else|\\
|\providecommand{\version}{final}|\\
|\||fi|
\end{tabular}
\end{center}
%
The definition by |\providecommand| makes sure
that previous definitions are not overwritten.
Further statements |\providecommand{\version}{...}|
can thus be added before the above code to override it.

For the main file, one might add a line
(between |\childdocmain| and the above block)
%
\begin{center}
|%\ifchilddoc\||else\providecommand{\version}{draft}\||fi|
\end{center}
%
which can be uncommented to produce a draft version.
Likewise one can add a line to the very top of a child file
(above the |\childdocof{|\textit{main}|}| directive)
%
\begin{center}
|%\providecommand{\version}{final}|
\end{center}
%
which can be uncommented to produce the final version of this child document.

%%%%%%%%%%%%%%%%%%%%%%%%%%%%%%%%%%%%%%%%%%%%%%%%%%%%%%%%%%%%%%%%%%%%%%%%%%%%%%%%
\subsection{Forwarding}
\label{sec:forward}

Different versions of the main or child documents
using compilation flags as described in \secref{sec:flags}
can be (permanently) stored in different files
for convenient compilation, viewing and distribution.
To this end, the package defines a command
to pass on compilation to a different file:

%%%%%%%%%%%%%%%%%%%%%%%%%%%%%%%%%%%%%%%%
\DescribeMacro{\childdocforward}
The command |\childdocforward| redirects processing to
another source file:
%
\begin{center}
\begin{tabular}{l}
|\input{childdoc.def}|\\
|\childdocforward[|\textit{main}|]{|\textit{dest}|}|\\
\end{tabular}
\end{center}
%
The argument \textit{dest} is the destination file
(without extension).
It should be the main file or one of the child files.
Note that further \textsf{childdoc} directives
such as |\childdocof| and |\childdocforward|
in the indicated file will be processed in this form.
The optional argument \textit{main}
passes on directly to the main file \textit{main}
while pretending to compile the child \textit{dest}.
This form behaves as if \textit{dest}
issues |\childdocof{|\textit{main}|}| right away,
and no further \textsf{childdoc} directives will be processed.

%%%%%%%%%%%%%%%%%%%%%%%%%%%%%%%%%%%%%%%%
\DescribeMacro{\...prefix}
In the alternative form |\childdocforwardprefix|,
%
\begin{center}
\begin{tabular}{l}
|\input{childdoc.def}|\\
|\childdocforwardprefix[|\textit{main}|]{|\textit{prefix}|}{|\textit{dest}|}|
\end{tabular}
\end{center}
%
the destination file is determined by a pattern
depending on the current file:
To make this work, the current file must be called
`{\textit{prefix}\hspace{0.2em}\textit{suffix}}'
with \textit{prefix} matching precisely the argument.
Processing is then passed on to the file
`{\textit{dest}\hspace{0.2em}\textit{suffix}}'.
Surely, the same effect is achieved by
directly specifying the
argument `{\textit{dest}\hspace{0.2em}\textit{suffix}}'
in the first form.
However, that requires to set up a different file
for each child. With the alternative form of the command
all these files can have exactly the same content
which simplifies setting them up and maintaining them.

For example, the following file |draft.tex|
with a compilation flag |\version| as described in \secref{sec:flags}
compiles the main document as a draft:
%
\begin{center}
\begin{tabular}{l}
|\def\version{draft}|\\
|\input{childdoc.def}|\\
|\childdocforward{|\textit{main}|}|
\end{tabular}
\end{center}
%
Likewise, the following files |final|\textit{nn}|.tex|
compile the final version of the child document
|child|\textit{nn}|.tex|:
%
\begin{center}
\begin{tabular}{l}
|\def\version{final}|\\
|\input{childdoc.def}|\\
|\childdocforwardprefix{final}{child}|
\end{tabular}
\end{center}
%

Note that when several versions of a main file and/or of each child file
are to be generated, it may be convenient to set up a |Makefile| or
shell script to automatise the process.

%%%%%%%%%%%%%%%%%%%%%%%%%%%%%%%%%%%%%%%%%%%%%%%%%%%%%%%%%%%%%%%%%%%%%%%%%%%%%%%%
\subsection{Command Line Processing}
\label{sec:commandline}

The effect of redirection files can also be achieved by invoking
the \LaTeX{} compiler with a more elaborate command line.
Most conveniently this should be done as part
of a shell script or a |Makefile|.

When using \textsf{childdoc} in the main file, the following
command lines effectively perform a redirection
(note that depending on the shell being used,
backslashes may have to be doubled: `|\|' $\to$ `|\\|'):
%
\begin{center}
|... -jobname "|\textit{target}|" |\\|"|[\textit{flags}]%
|\input{childdoc.def}\childdocforward[|\textit{main}|]{|\textit{dest}|}"|
\end{center}
%
Here \textit{target} is the name of the output file,
\textit{main} is the name of the main file
and \textit{dest} is the name of the main or child file to be processed
(all filenames without extensions).
The optional argument \textit{main} can be omitted
if \textit{main} matches \textit{dest}.
Optionally, compilation \textit{flags} can be defined via |\def| commands.
This command line makes the \TeX{} engine believe
it is compiling the file \textit{target}
whose content is specified as the latter parameter.
The provided code then forwards the processing to
\textit{main} or \textit{dest} as described in \secref{sec:forward}.

%%%%%%%%%%%%%%%%%%%%%%%%%%%%%%%%%%%%%%%%%%%%%%%%%%%%%%%%%%%%%%%%%%%%%%%%%%%%%%%%
\subsection{Include by Input}
\label{sec:input}

Including child documents by |\include| has some restrictions by design.
Most notably, the content of a child document always occupies
its own set of pages; pages cannot be shared between child documents.
Usually, this behaviour makes perfect sense
because each child document contain an essential part of the document.
However, in some situations it may be desirable to compose
a document from a collection of parts
without having mandatory page breaks between then.
For this case, the package
provides a mechanism to include parts
by |\input| which can also be processed individually.
However, by construction this mechanism
requires manual handling of the content to be output.

%%%%%%%%%%%%%%%%%%%%%%%%%%%%%%%%%%%%%%%%
\DescribeMacro{\ifchilddocmanual}
The main file should be prepared as usual, see \secref{sec:include}.
However, the document body must make a distinction
between processing of an individual part and of the main document, e.g.:
%
\begin{center}
\begin{tabular}{l}
|\ifchilddocmanual|\\
|\input{\childdocname}|\\
|\||else|\\
\textit{document body with }|\input{|\textit{part}|}|\\
|\||fi|
\end{tabular}
\end{center}
%
The conditional |\ifchilddocmanual| is true whenever
a part to be included by |\input| is being compiled,
and the name of the part is stored in |\childdocname|.

%%%%%%%%%%%%%%%%%%%%%%%%%%%%%%%%%%%%%%%%
\DescribeMacro{\childdocby}
Each part to be included by |\input| should start with:
%
\begin{center}
\begin{tabular}{l}
|\input{childdoc.def}|\\
|\childdocby{|\textit{main}|}|\\
\end{tabular}
\end{center}
%
The directive |\childdocby| is similar to |\childdocof|
described in \secref{sec:include},
but the subsequent selection of content must be done manually.
To that end, both |\ifchilddoc| and |\ifchilddocmanual|
will be true upon processing of a part,
and the name of the part is stored in |\childdocname|.
Note that |\jobname| will be set to the filename of the current part
so that each part receives an individual |.aux| file
that does not interfere with the |.aux| file(s) of the main document.
This behaviour can be altered by the alternative form
|\childdocby[*]{|\textit{main}|}| (with a non-empty optional argument)
which uses the |.aux| file of the main document
by setting |\jobname| to \textit{main}.

%%%%%%%%%%%%%%%%%%%%%%%%%%%%%%%%%%%%%%%%%%%%%%%%%%%%%%%%%%%%%%%%%%%%%%%%%%%%%%%%
\subsection{Driver Development}
\label{sec:driver}

The \textsf{childdoc} mechanism can also be use for the development
of definition files such as \LaTeX{} styles or classes.
This case differs from the above setup with multiple parts
included by |\include| in that no |\includeonly| should be invoked.
This can be achieved by starting the include file
(before |\ProvidesPackage|) with:
%
\begin{center}
\begin{tabular}{l}
|\input{childdoc.def}|\\
|\childdocforward{|\textit{main}|}|\\
\end{tabular}
\end{center}
%
or alternatively with:
%
\begin{center}
\begin{tabular}{l}
|\input{childdoc.def}|\\
|\childdocby{|\textit{main}|}|\\
\end{tabular}
\end{center}
%
Both forms have slightly different effects as described above.
The main file is prepared as usual, see \secref{sec:include}.

%%%%%%%%%%%%%%%%%%%%%%%%%%%%%%%%%%%%%%%%%%%%%%%%%%%%%%%%%%%%%%%%%%%%%%%%%%%%%%%%
\subsection{Legacy Detection}
\label{sec:detection}

The directive |\childdocmain| in the main file can detect
whether the complete document or merely a child is to be compiled
even without using the directive |\childdocof|.
This method is deprecated because it is less robust
and there is no compelling reason to use it;
it is merely provided for backward compatibility
and it may be removed in future versions.

If the detection mechanism is to be used,
it is mandatory to correctly specify
the filename of the main file as the argument of |\childdocmain|:
%
\begin{center}
\begin{tabular}{l}
|\input{childdoc.def}|\\
|\childdocmain{|\textit{main}|}|\\
\end{tabular}
\end{center}
%
If |\jobname| does not match the argument \textit{main} of |\childdocmain|,
it is assumed that |\jobname| points to the child file to be compiled.
When using |\childdocmain| with the main file specified as argument,
it suffices to start a child file
with just |\input{|\textit{main}|}|
without loading of the package and using |\childdocof|.
If instead all processing is done
with the appropriate \textsf{childdoc} directives,
the argument of \textit{main} of |\childdocmain| can be empty.

An alternative version of the command line processing described
in \secref{sec:commandline} using the detection mechanism reads:
%
\begin{center}
|... -jobname "|\textit{target}|" "|[\textit{flags}]%
[|\def\jobname{|\textit{dest}|}|]|\input{|\textit{main}|}"|
\end{center}

%%%%%%%%%%%%%%%%%%%%%%%%%%%%%%%%%%%%%%%%%%%%%%%%%%%%%%%%%%%%%%%%%%%%%%%%%%%%%%%%
\subsection{Manual Code}
\label{sec:manual}

In case one cannot be certain whether the definitions file |childdoc.def|
is installed on the target \TeX{} distribution
and one prefers not to ship it,
it is conceivable to paste a few relevant commands into the sources.

To that end, drop all statements |\input{childdoc.def}|
and perform the replacements as outlined below.
Instead of |\childdocmain{|\textit{main}|}| add the following code
to the top of the main file:
%
\begin{center}
\begin{tabular}{l}
|\||ifdefined\childdocname\endinput\||fi\newif\ifchilddoc|\\
|\edef\childdocname{\scantokens\expandafter{\jobname\noexpand}}|\\
|\def\childdocmain{|\textit{main}|}\||ifx\childdocmain\childdocname\||else|\\
|\childdoctrue\includeonly{\childdocname}\let\jobname\childdocmain\||fi|\\
\end{tabular}
\end{center}
%
Instead of |\childdocof{|\textit{main}|}| just include the main file
at the top of each child file:
%
\begin{center}
|\input{|\textit{main}|}|
\end{center}
%
A simple redirection |\childdocforward{|\textit{dest}|}| is achieved by:
%
\begin{center}
|\def\jobname{|\textit{dest}|}\input{\jobname}|
\end{center}
%
The redirection with prefix
|\childdocforwardprefix[|\textit{prefix}|]{|\textit{dest}|}|
is accomplished by:
%
\begin{center}
\begin{tabular}{l}
|{\edef\jobname{\scantokens\expandafter{\jobname\noexpand}}|\\
|\def\redirectjob |\textit{prefix}|#1~~~{\gdef\jobname{|\textit{dest}|#1}}|\\
|\expandafter\redirectjob\jobname~~~}\input{\jobname}|
\end{tabular}
\end{center}

In an alternative approach,
child documents can be compiled by a specific command line
without additional code or specific definitions:
%
\begin{center}
|... -jobname "|\textit{target}|" "|[\textit{flags}]%
|\includeonly{|\textit{dest}|}\input{|\textit{main}|}"|
\end{center}
%

%%%%%%%%%%%%%%%%%%%%%%%%%%%%%%%%%%%%%%%%%%%%%%%%%%%%%%%%%%%%%%%%%%%%%%%%%%%%%%%%
%%%%%%%%%%%%%%%%%%%%%%%%%%%%%%%%%%%%%%%%%%%%%%%%%%%%%%%%%%%%%%%%%%%%%%%%%%%%%%%%
\section{Information}

%%%%%%%%%%%%%%%%%%%%%%%%%%%%%%%%%%%%%%%%%%%%%%%%%%%%%%%%%%%%%%%%%%%%%%%%%%%%%%%%
\subsection{Copyright}

Copyright \copyright{} 2017--2018 Niklas Beisert

This work may be distributed and/or modified under the
conditions of the \LaTeX{} Project Public License, either version 1.3
of this license or (at your option) any later version.
The latest version of this license is in
  \url{http://www.latex-project.org/lppl.txt}
and version 1.3 or later is part of all distributions of \LaTeX{}
version 2005/12/01 or later.

This work has the LPPL maintenance status `maintained'.

The Current Maintainer of this work is Niklas Beisert.

This work consists of the files |README.txt|, |childdoc.ins| and |childdoc.dtx|
as well as the derived files |childdoc.def|, |cdocsamp.tex|
with |cdocsch1.tex|, |cdocsch2.tex|, |cdocspt3.tex|, |cdocspt4.tex|,
|cdocsdrf.tex|, |cdocsfn1.tex|, |cdocsfn2.tex|
as well as |childdoc.pdf|.

%%%%%%%%%%%%%%%%%%%%%%%%%%%%%%%%%%%%%%%%%%%%%%%%%%%%%%%%%%%%%%%%%%%%%%%%%%%%%%%%
\subsection{Files and Installation}

The package consists of the files:
%
\begin{center}
\begin{tabular}{ll}
    |README.txt|   & readme file \\
    |childdoc.ins| & installation file \\
    |childdoc.dtx| & source file \\
    |childdoc.def| & definition file \\
    |cdocsamp.tex| & sample main file \\
    |cdocsch1.tex| & sample include file \\
    |cdocsch2.tex| & sample include file \\
    |cdocspt3.tex| & sample part file \\
    |cdocspt4.tex| & sample part file \\
    |cdocsdrf.tex| & sample redirection file \\
    |cdocsfn1.tex| & sample redirection file \\
    |cdocsfn2.tex| & sample redirection file \\
    |childdoc.pdf| & manual
\end{tabular}
\end{center}
%
The distribution consists of the files
|README.txt|, |childdoc.ins| and |childdoc.dtx|.
%
\begin{itemize}
\item
Run (pdf)\LaTeX{} on |childdoc.dtx|
to compile the manual |childdoc.pdf| (this file).
\item
Run \LaTeX{} on |childdoc.ins| to create the definitions file |childdoc.def|
and the sample |cdocsamp.tex| with include files
|cdocsch1.tex|, |cdocsch2.tex|, |cdocspt3.tex|, |cdocspt4.tex|,
|cdocsdrf.tex|, |cdocsfn1.tex|, |cdocsfn2.tex|.
Then copy the file |childdoc.def| to an appropriate directory of your \LaTeX{}
distribution, e.g.\ \textit{texmf-root}|/tex/latex/childdoc|.
\end{itemize}

%%%%%%%%%%%%%%%%%%%%%%%%%%%%%%%%%%%%%%%%%%%%%%%%%%%%%%%%%%%%%%%%%%%%%%%%%%%%%%%%
\subsection{Related CTAN Packages}

There are several other packages which offer a similar functionality:
%
\begin{itemize}
\item
The packages
\href{http://ctan.org/pkg/docmute}{\textsf{docmute}},
\href{http://ctan.org/pkg/includex}{\textsf{includex}} and
\href{http://ctan.org/pkg/standalone}{\textsf{standalone}}
provide commands to include only the document body of
a child file thus allowing both files to be compiled individually.
\item
The packages \href{http://ctan.org/pkg/subdocs}{\textsf{subdocs}}
and \href{http://ctan.org/pkg/subfiles}{\textsf{subfiles}}
provide structures in which the main and child documents can be
encapsulated and allowing them to be compiled individually.
The inclusion mechanism is different from the conventional |\include|.
\item
The package \href{http://ctan.org/pkg/combine}{\textsf{combine}}
is an elaborate solution to combine several documents into one.
\end{itemize}
%
See also the CTAN topic \href{http://ctan.org/topic/subdocs}{\textsf{subdocs}}
for further related packages.
The present package differs from the above solutions in that
a document structure constructed with the conventional |\include| mechanism
just needs two extra commands at the top of every file
such that all constituent files can be compiled individually.

%%%%%%%%%%%%%%%%%%%%%%%%%%%%%%%%%%%%%%%%%%%%%%%%%%%%%%%%%%%%%%%%%%%%%%%%%%%%%%%%
%\subsection{Feature Suggestions}
%
%The following is a list of features which may be useful for future
%versions of this package:
%%
%\begin{itemize}
%\item
%\ldots
%\end{itemize}

%%%%%%%%%%%%%%%%%%%%%%%%%%%%%%%%%%%%%%%%%%%%%%%%%%%%%%%%%%%%%%%%%%%%%%%%%%%%%%%%
\subsection{Revision History}

%%%%%%%%%%%%%%%%%%%%%%%%%%%%%%%%%%%%%%%%
\paragraph{v2.0:} 2018/12/30

\begin{itemize}
\item
immediate forward processing
\item
added |\childdocby| mechanism
\item
manual restructured
\end{itemize}

%%%%%%%%%%%%%%%%%%%%%%%%%%%%%%%%%%%%%%%%
\paragraph{v1.6:} 2018/01/17

\begin{itemize}
\item
application for development of include files
\item
corrections to manual
\end{itemize}

%%%%%%%%%%%%%%%%%%%%%%%%%%%%%%%%%%%%%%%%
\paragraph{v1.5:} 2017/05/21

\begin{itemize}
\item
more complete structuring introduced
\item
|\childdocof| introduced
\item
|\childdoc| renamed to |\childdocmain|
\item
|\childredirect| renamed to |\childdocforward| and |\childdocforwardprefix|
and functionality expanded
\end{itemize}

%%%%%%%%%%%%%%%%%%%%%%%%%%%%%%%%%%%%%%%%
\paragraph{v1.0:} 2017/04/27

\begin{itemize}
\item
manual and install package
\item
first version published on CTAN
\end{itemize}

%%%%%%%%%%%%%%%%%%%%%%%%%%%%%%%%%%%%%%%%
\paragraph{v0.6:} 2017/04/26

\begin{itemize}
\item
redirection mechanism added
\end{itemize}

%%%%%%%%%%%%%%%%%%%%%%%%%%%%%%%%%%%%%%%%
\paragraph{v0.5:} 2017/04/26

\begin{itemize}
\item
functionality in definition file
\end{itemize}


%%%%%%%%%%%%%%%%%%%%%%%%%%%%%%%%%%%%%%%%%%%%%%%%%%%%%%%%%%%%%%%%%%%%%%%%%%%%%%%%
%%%%%%%%%%%%%%%%%%%%%%%%%%%%%%%%%%%%%%%%%%%%%%%%%%%%%%%%%%%%%%%%%%%%%%%%%%%%%%%%
%%%%%%%%%%%%%%%%%%%%%%%%%%%%%%%%%%%%%%%%%%%%%%%%%%%%%%%%%%%%%%%%%%%%%%%%%%%%%%%%
\appendix

\settowidth\MacroIndent{\rmfamily\scriptsize 000\ }

 \DocInput{childdoc.dtx}

\end{document}
%</driver>
% \fi
%
% %%%%%%%%%%%%%%%%%%%%%%%%%%%%%%%%%%%%%%%%%%%%%%%%%%%%%%%%%%%%%%%%%%%%%%%%%%%%%%
% %%%%%%%%%%%%%%%%%%%%%%%%%%%%%%%%%%%%%%%%%%%%%%%%%%%%%%%%%%%%%%%%%%%%%%%%%%%%%%
% \section{Sample}
%\iffalse
%<*samplemain>
%\fi
%
% The following presents a sample document
% with two chapters, two parts, a title page,
% a compile flag as well as three forwarding files to set the flag.
% It consists of eight |.tex| files:
% \begin{center}
% \begin{tabular}{ll}
% |cdocsamp.tex|&main file\\
% |cdocsch1.tex|&include file for chapter 1\\
% |cdocsch2.tex|&include file for chapter 2\\
% |cdocspt3.tex|&include file for part 3\\
% |cdocspt4.tex|&include file for part 4\\
% |cdocsdrf.tex|&forwarding file for main file in draft mode\\
% |cdocsfi1.tex|&forwarding file for final version of chapter 1\\
% |cdocsfi2.tex|&forwarding file for final version of chapter 2\\
% \end{tabular}
% \end{center}
% Each of the eight files can be compiled directly by the \LaTeX{} compiler.
%
% %%%%%%%%%%%%%%%%%%%%%%%%%%%%%%%%%%%%%%
% \paragraph{Main File.}
%
% The main file is called |cdocsamp.tex|.
%
% Load the \textsf{childdoc} definitions and
% declare the filename for the main document:
%    \begin{macrocode}
\input{childdoc.def}
\childdocmain{}
%    \end{macrocode}

% Optional override for |\version| flag:
%    \begin{macrocode}
%%\ifchilddoc\else\providecommand{\version}{draft}\fi
%    \end{macrocode}

% Define the default values for the |\version| flag
% (|final| for the main file and |draft| for childs):
%    \begin{macrocode}
\ifchilddoc
\providecommand{\version}{draft}
\else
\providecommand{\version}{final}
\fi
%    \end{macrocode}

% Load the standard document class:
%    \begin{macrocode}
\documentclass[12pt]{article}
%    \end{macrocode}

% Start the document body:
%    \begin{macrocode}
\begin{document}
%    \end{macrocode}

% Declare a title page.
% Print title, part of document being processed and version flag:
%    \begin{macrocode}
\addtocounter{page}{-1}
\begin{center}
{\LARGE\bfseries{}childdoc example\par}
\vspace{1cm}
\ifchilddoc
\ifchilddocmanual part\else chapter\fi:
`\childdocname' of `\childdocjob'\par
\else
main document: `\childdocjob'\par
\fi
version: \version\par
\end{center}
\newpage
%    \end{macrocode}

% Manually include selected file,
% otherwise process as usual:
%    \begin{macrocode}
\ifchilddocmanual
\section*{part `\childdocname'}
\input{\childdocname}
\else
%    \end{macrocode}

% Include the two chapters:
%    \begin{macrocode}
\include{cdocsch1}
\include{cdocsch2}
%    \end{macrocode}

% Include the two parts unless only chapters should be displayed:
%    \begin{macrocode}
\ifchilddoc\else
\section{part three}
\input{cdocspt3}
\section{part four}
\input{cdocspt4}
\fi
%    \end{macrocode}

% Process as usual until here:
%    \begin{macrocode}
\fi
%    \end{macrocode}

% End of document body:
%    \begin{macrocode}
\end{document}
%    \end{macrocode}
%\iffalse
%</samplemain>
%\fi
%
% %%%%%%%%%%%%%%%%%%%%%%%%%%%%%%%%%%%%%%
% \paragraph{Chapter Include Files.}
%
% The include files are called |cdocsch1.tex| and |cdocsch2.tex|.
%
%\iffalse
%<*samplechap1|samplechap2>
%\fi

% Optional override for |\version| flag:
%    \begin{macrocode}
%%\providecommand{\version}{final}
%    \end{macrocode}

% Include the main document:
%    \begin{macrocode}
\input{childdoc.def}
\childdocof{cdocsamp}
%    \end{macrocode}

%\iffalse
%</samplechap1|samplechap2>
%\fi
%
%\iffalse
%<*samplechap1>
%\fi
% Some text for chapter 1:
%    \begin{macrocode}
\section{one}
some text in chapter one
%    \end{macrocode}

%\iffalse
%</samplechap1>
%\fi
% Some text for chapter 2:
%\iffalse
%<*samplechap2>
%\fi
%    \begin{macrocode}
\section{two}
more text in chapter two
%    \end{macrocode}

%\iffalse
%</samplechap2>
%\fi
%
% %%%%%%%%%%%%%%%%%%%%%%%%%%%%%%%%%%%%%%
% \paragraph{Part Include Files.}
%
% The include files are called |cdocspt3.tex| and |cdocspt4.tex|.
%
%\iffalse
%<*samplepart3|samplepart4>
%\fi

% Optional override for |\version| flag:
%    \begin{macrocode}
%%\providecommand{\version}{final}
%    \end{macrocode}

% Include the main document:
%    \begin{macrocode}
\input{childdoc.def}
\childdocby{cdocsamp}
%    \end{macrocode}

%\iffalse
%</samplepart3|samplepart4>
%\fi
%
%\iffalse
%<*samplepart3>
%\fi
% Some text for part 3:
%    \begin{macrocode}
some text in part three
%    \end{macrocode}

%\iffalse
%</samplepart3>
%\fi
% Some text for part 4:
%\iffalse
%<*samplepart4>
%\fi
%    \begin{macrocode}
more text in part four
%    \end{macrocode}

%\iffalse
%</samplepart4>
%\fi
%
% %%%%%%%%%%%%%%%%%%%%%%%%%%%%%%%%%%%%%%
% \paragraph{Forwarding for a Complete Draft.}
%
% The following forwarding file |cdocsdrf.tex|
% compiles the main document in draft mode:
%\iffalse
%<*sampledraft>
%\fi
%    \begin{macrocode}
\def\version{draft}
\input{childdoc.def}
\childdocforward{cdocsamp}
%    \end{macrocode}

%\iffalse
%</sampledraft>
%\fi
%
% %%%%%%%%%%%%%%%%%%%%%%%%%%%%%%%%%%%%%%
% \paragraph{Forwarding for Final Version of the Chapters.}
%
% The following forwarding files |cdocsfn1.tex| and |cdocsfn2.tex|
% (with identical content)
% compile the final versions of the child documents
% |cdocsch1.tex| and |cdocsch2.tex|, respectively:
%\iffalse
%<*samplefinal>
%\fi
%    \begin{macrocode}
\def\version{final}
\input{childdoc.def}
\childdocforwardprefix[cdocsamp]{cdocsfn}{cdocsch}
%    \end{macrocode}

%\iffalse
%</samplefinal>
%\fi
%
% %%%%%%%%%%%%%%%%%%%%%%%%%%%%%%%%%%%%%%
% \paragraph{Command Line Processing.}
%
% The following three command lines generate the output files
% |cdocscld|, |cdocscl1| and |cdocscl2|
% which should be identical to
% |cdocsdrf|, |cdocsch1| and |cdocsfn2|, respectively:
% \begin{center}
% \begin{tabular}{l}
% |latex -jobname cdocscld \|\\
% |  "\def\version{draft}\input{childdoc.def}\childdocforward{cdocsamp}"|\\
% |latex -jobname cdocscl1 \|\\
% |  "\input{childdoc.def}\childdocforward[cdocsamp]{cdocsch1}"|\\
% |latex -jobname cdocscl2 \|\\
% |  "\def\version{final}\input{childdoc.def}\childdocforward{cdocsch2}"|
% \end{tabular}
% \end{center}
% Note that the trailing backslash on each first line
% merely continues the input to the second line
% (for convenient cut ant paste).
% Furthermore, the command |latex| can be replaced by any
% of its alternative versions such as |pdflatex|.
%
% %%%%%%%%%%%%%%%%%%%%%%%%%%%%%%%%%%%%%%%%%%%%%%%%%%%%%%%%%%%%%%%%%%%%%%%%%%%%%%
% %%%%%%%%%%%%%%%%%%%%%%%%%%%%%%%%%%%%%%%%%%%%%%%%%%%%%%%%%%%%%%%%%%%%%%%%%%%%%%
% \section{Implementation}
%\iffalse
%<*package>
%\fi
%
% This section describes the definitions file |childdoc.def|.

% The definitions cannot be loaded using |\usepackage| or |\RequirePackage|
% which has a mechanism to prevent loading a style file more than once.
% When loading the definitions by means of |\input|
% multiple instances have to be prevented manually:
%\iffalse
%This code needs to be before the `\ProvidesFile' directive
%which is defined at the beginning of this file.
%Therefore it is also placed there and commented out here.
%</package>
%<*discard>
%\fi
%    \begin{macrocode}
\ifdefined\childdocmain\endinput\fi
%    \end{macrocode}
%\iffalse
%</discard>
%<*package>
%\fi
%
% \macro{\ifchilddoc}
% \macro{\ifchilddocmanual}
% The conditional |\ifchilddoc| tells whether a
% child (true) or main (false) document is being compiled.
% The conditional |\ifchilddocmanual| tells whether
% the |\includeonly| mechanism is used (false) or
% the selection of child files must be performed manually (true).
% The definitions initialise to false:
%    \begin{macrocode}
\newif\ifchilddoc
\newif\ifchilddocmanual
%    \end{macrocode}

% \macro{\childdocname}
% \macro{\childdocjob}
% The macro |\childdocname| stores the name of the main document
% to be compiled. The macro |\childdocjob| stores the name of
% the document on which the \LaTeX{} compiler was originally invoked.
% The content of |\jobname| cannot be compared
% to filenames specified in the source due to different catcodes.
% The following code rescans |\jobname|, stores the result
% in |\childdocname| and saves a copy in |\childdocjob|:
%    \begin{macrocode}
\edef\childdocname{\scantokens\expandafter{\jobname\noexpand}}
\let\childdocjob\childdocname
%    \end{macrocode}

% \macro{\childdocdisable}
% The macro |\childdocdisable| prevents the main file
% from being processed more than once.
% At this stage, the main document command |\childdocmain|
% is assumed to be called once again where it should do nothing.
% Any subsequent call to it should prevent
% a secondary processing of the main document
% It overwrites the forwarding commands
% |\childdocof| and |\childdocforward|
% with empty macros to prevent further inclusions of the main document:
%    \begin{macrocode}
\newcommand{\childdocdisable}
{
  \renewcommand{\childdocmain}[1]{\renewcommand{\childdocmain}[1]{\endinput}}
  \renewcommand{\childdocof}[1]{}
  \renewcommand{\childdocby}[2][]{}
  \renewcommand{\childdocforward}[2][]{}
  \renewcommand{\childdocdisable}{}
}
%    \end{macrocode}

% \macro{\childdocmain}
% The macro |\childdocmain| is to be called at the top of the main file
% with nothing or the main filename (without extension) as argument.
% First, it breaks loops.
% If the argument is not empty and does not match |\childdocname|
% (which is set by the first inclusion of |childdoc.def|),
% |\ifchilddoc| is set to true, |\includeonly| is applied to the child file
% and |\jobname| is set to the main file
% (for proper handling of |.aux| files):
%    \begin{macrocode}
\newcommand{\childdocmain}[1]
{
  \childdocdisable\childdocmain{}
  \if?#1?\else
    \begingroup
      \def\childdoctmp{#1}
      \ifx\childdoctmp\childdocname
        \def\childdoctmp{}
      \else
        \def\childdoctmp
        {
          \childdoctrue
          \includeonly{\childdocname}
          \def\childdocjob{#1}
          \def\jobname{#1}
        }
      \fi
      \expandafter
    \endgroup
    \childdoctmp
  \fi
}
%    \end{macrocode}

% \macro{\childdocof}
% The command |\childdocof| redirects
% compilation to the main file |#1|.
%    \begin{macrocode}
\newcommand{\childdocof}[1]
{
  \childdocdisable
  \childdoctrue
  \includeonly{\childdocname}
  \def\jobname{#1}
  \def\childdocjob{#1}
  \input{#1}
}
%    \end{macrocode}

% \macro{\childdocby}
% The command |\childdocby| ....
%    \begin{macrocode}
\newcommand{\childdocby}[2][]
{
  \childdocdisable
  \childdoctrue
  \childdocmanualtrue
  \if?#1?\else
    \def\jobname{#2}
  \fi
  \def\childdocjob{#2}
  \input{#2}
  \endinput
}
%    \end{macrocode}

% \macro{\childdocforward}
% The command |\childdocforward| redirects
% compilation to the main file or
% (if the optional argument is given) a child file.
% Parameters are set as if the main file
% or a child file starting with |\childdocof| was compiled.
% Then compilation is handed over to the main file:
%    \begin{macrocode}
\newcommand{\childdocforward}[2][]
{
  \begingroup
    \if?#1?
      \def\childdoctmp
      {
        \def\childdocname{#2}
        \def\childdocjob{#2}
        \def\jobname{#2}
        \input{#2}
        \endinput
      }
    \else
      \def\childdoctmp
      {
        \childdocdisable
        \def\childdocname{#2}
        \childdoctrue
        \includeonly{#2}
        \def\childdocjob{#1}
        \def\jobname{#1}
        \input{#1}
        \endinput
      }
    \fi
    \expandafter
  \endgroup
  \childdoctmp
}
%    \end{macrocode}

% \macro{\childdocforwardprefix}
% The command |\childdocforwardprefix| redirects
% compilation to the main or a child file by means of a pattern.
% The prefix |#1| in the current filename is replaced by |#2|
% and the suffix of the current filename is kept
% (it is assumed that the filename does not contain the substring `|~~~|'
% which is used as a delimiter).
% Compilation is handed over to the new file by |\childdocforward|:
%    \begin{macrocode}
\newcommand{\childdocforwardprefix}[3][]
{
  \begingroup
    \def\childdocextract #2##1~~~{\def\childdoctmp{\childdocforward[#1]{#3##1}}}
    \expandafter\childdocextract\childdocname~~~
    \expandafter
  \endgroup
  \childdoctmp
}
%    \end{macrocode}

% \macro{\childdoc}
% The deprecated macro |\childdoc| is a legacy version of |\childdocmain|:
%    \begin{macrocode}
\newcommand{\childdoc}{\childdocmain}
%    \end{macrocode}

% \macro{\childdocredirect}
% The deprecated macro |\childdocredirect| is a legacy version
% of |\childdocforward| and |\childdocforwardprefix|:
%    \begin{macrocode}
\newcommand{\childdocredirect}[2][]
{
  \begingroup
    \if?#1?
      \def\childdoctmp{\childdocforward{#2}}
    \else
      \def\childdoctmp{\childdocforwardprefix{#1}{#2}}
    \fi
    \expandafter
  \endgroup
  \childdoctmp
}
%    \end{macrocode}

%\iffalse
%</package>
%\fi
%
\endinput

\childdocforward{cdocsamp}
%    \end{macrocode}

%\iffalse
%</sampledraft>
%\fi
%
% %%%%%%%%%%%%%%%%%%%%%%%%%%%%%%%%%%%%%%
% \paragraph{Forwarding for Final Version of the Chapters.}
%
% The following forwarding files |cdocsfn1.tex| and |cdocsfn2.tex|
% (with identical content)
% compile the final versions of the child documents
% |cdocsch1.tex| and |cdocsch2.tex|, respectively:
%\iffalse
%<*samplefinal>
%\fi
%    \begin{macrocode}
\def\version{final}
% \iffalse
%
% childdoc.dtx Copyright (C) 2017-2018 Niklas Beisert
%
% This work may be distributed and/or modified under the
% conditions of the LaTeX Project Public License, either version 1.3
% of this license or (at your option) any later version.
% The latest version of this license is in
%   http://www.latex-project.org/lppl.txt
% and version 1.3 or later is part of all distributions of LaTeX
% version 2005/12/01 or later.
%
% This work has the LPPL maintenance status `maintained'.
%
% The Current Maintainer of this work is Niklas Beisert.
%
% This work consists of the files childdoc.dtx and childdoc.ins
% and the derived files childdoc.def and cdocsamp.tex with
% cdocsch1.tex, cdocsch2.tex, cdocsdrf.tex, cdocsfn1.tex, cdocsfn2.tex.
%
%<package>\ifdefined\childdocmain\endinput\fi
%<package>\ProvidesFile{childdoc.def}[2018/12/30 v2.0 child document driver]
%<samplemain>\ProvidesFile{cdocsamp.tex}[2018/12/30 v2.0 sample for childdoc]
%<*driver>
%\ProvidesFile{childdoc.drv}[2018/12/30 v2.0 childdoc reference manual file]
\PassOptionsToClass{10pt,a4paper}{article}
\documentclass{ltxdoc}

\usepackage[margin=35mm]{geometry}
\usepackage{hyperref}
\usepackage{hyperxmp}
\usepackage[usenames]{color}

\hypersetup{colorlinks=true}
\hypersetup{pdfstartview=FitH}
\hypersetup{pdfpagemode=UseNone}
\hypersetup{pdfsource={}}
\hypersetup{pdflang={en-UK}}
\hypersetup{pdfcopyright={Copyright 2017-2018 Niklas Beisert.
  This work may be distributed and/or modified under the
  conditions of the LaTeX Project Public License, either version 1.3
  of this license or (at your option) any later version.}}
\hypersetup{pdflicenseurl={http://www.latex-project.org/lppl.txt}}
\hypersetup{pdfcontactaddress={ETH Zurich, ITP, HIT K,
  Wolfgang-Pauli-Strasse 27}}
\hypersetup{pdfcontactpostcode={8093}}
\hypersetup{pdfcontactcity={Zurich}}
\hypersetup{pdfcontactcountry={Switzerland}}
\hypersetup{pdfcontactemail={nbeisert@itp.phys.ethz.ch}}
\hypersetup{pdfcontacturl={http://people.phys.ethz.ch/\xmptilde nbeisert/}}

\newcommand{\secref}[1]{\hyperref[#1]{section \ref*{#1}}}

\parskip1ex
\parindent0pt
\let\olditemize\itemize
\def\itemize{\olditemize\parskip0pt}

\begin{document}

\title{The \textsf{childdoc} Package}
\hypersetup{pdftitle={The childdoc Package}}
\author{Niklas Beisert\\[2ex]
  Institut f\"ur Theoretische Physik\\
  Eidgen\"ossische Technische Hochschule Z\"urich\\
  Wolfgang-Pauli-Strasse 27, 8093 Z\"urich, Switzerland\\[1ex]
  \href{mailto:nbeisert@itp.phys.ethz.ch}
  {\texttt{nbeisert@itp.phys.ethz.ch}}}
\hypersetup{pdfauthor={Niklas Beisert}}
\hypersetup{pdfsubject={Manual for the LaTeX2e Package childdoc}}
\date{30 December 2018, \textsf{v2.0}}
\maketitle

\begin{abstract}\noindent
\textsf{childdoc} is a \LaTeXe{} package
that enables the direct compilation
of document sections included by |\include|
to individual files.
\end{abstract}

\begingroup
\parskip0ex
\tableofcontents
\endgroup

%%%%%%%%%%%%%%%%%%%%%%%%%%%%%%%%%%%%%%%%%%%%%%%%%%%%%%%%%%%%%%%%%%%%%%%%%%%%%%%%
%%%%%%%%%%%%%%%%%%%%%%%%%%%%%%%%%%%%%%%%%%%%%%%%%%%%%%%%%%%%%%%%%%%%%%%%%%%%%%%%
\section{Introduction}

\LaTeX{} provides a mechanism to structure a large document (such as a book)
into a main file and several child files (containing the chapters)
using the |\include| command.
This mechanism is beneficial for documents
which span hundreds of pages in order to
make the source file(s) more manageable.
Moreover, compilation can be restricted to
selected child files by means of the |\includeonly| command.
The latter feature can be used to reduce the compilation time while editing
(this was significantly more useful in the earlier days of \LaTeX{})
or to generate a smaller document which is easier to navigate.
Another application of |\includeonly| is to generate
documents consisting of selected parts of the complete document.

However, there are a few drawbacks of the plain |\include| mechanism:
\begin{itemize}
\item
The child files cannot be compiled on their own,
they can only be compiled via the main file.
A naive editing environment
(such as a text editor with an option
to have the current file processed by \LaTeX)
may require one to switch to the main file before compiling;
attempting to compile the child file produces errors.
\item
The main file must be modified (each time)
to adjust the |\includeonly| command
to the present needs. This easily leaves the main file in a messy state.
\item
The generated document will always carry the filename
of the main document. This is inconvenient if
several child files are to be compiled and
to be kept for distribution.
\end{itemize}

The present package provides a simple interface
to make child files individually compilable by \LaTeX{}.
Compiling a child file then has the same effect as compiling
the main file with an |\includeonly| command
to select the appropriate child.
Moreover the generated document will carry the name of the child
rather than the main file.
This resolves all three above issues.

This feature is meant to make the editing of books,
thesis documents and lecture notes somewhat more convenient.
However, the package can also be used efficiently for
composing a series of documents (such as exercise sheets)
which are typically distributed individually.
It then assists the author in generating the individual documents
(potentially in different versions)
as well as a document containing the collected series.
Another application is in developing style files
or other kinds of included material
where compilation of the style file could redirect
to a sample or test file.

%%%%%%%%%%%%%%%%%%%%%%%%%%%%%%%%%%%%%%%%%%%%%%%%%%%%%%%%%%%%%%%%%%%%%%%%%%%%%%%%
%%%%%%%%%%%%%%%%%%%%%%%%%%%%%%%%%%%%%%%%%%%%%%%%%%%%%%%%%%%%%%%%%%%%%%%%%%%%%%%%
\section{Usage}

First of all, the package \textsf{childdoc} is \emph{not} a standard
\LaTeXe{} |.sty| style file! Therefore it needs to be invoked in
a non-standard way.

%%%%%%%%%%%%%%%%%%%%%%%%%%%%%%%%%%%%%%%%%%%%%%%%%%%%%%%%%%%%%%%%%%%%%%%%%%%%%%%%
\subsection{Included Files}
\label{sec:include}

%%%%%%%%%%%%%%%%%%%%%%%%%%%%%%%%%%%%%%%%
\DescribeMacro{\childdocmain}
To use the package, add the commands
\begin{center}
\begin{tabular}{l}
|\input{childdoc.def}|\\
|\childdocmain{}|\\
\end{tabular}
\end{center}
at the very top of the main \LaTeX{} file,
in particular \emph{before} the |\documentclass| statement!
The argument of |\childdocmain| should be left empty
(but it must be present).

%%%%%%%%%%%%%%%%%%%%%%%%%%%%%%%%%%%%%%%%
\DescribeMacro{\childdocof}
Furthermore, add the commands
\begin{center}
\begin{tabular}{l}
|\input{childdoc.def}|\\
|\childdocof{|\textit{main}|}|\\
\end{tabular}
\end{center}
at the top of every child file \textit{child}
which is included by |\include{|\textit{child}|}|
from within the main file
(or at least for those files to be compiled individually).
The argument \textit{main} must be the filename of the main file.

There are a couple of
considerations in setting up the main and child documents:

%%%%%%%%%%%%%%%%%%%%%%%%%%%%%%%%%%%%%%%%
\paragraph{Restrictions.}

Please note the following restrictions:
\begin{itemize}
\item
|\childdocmain| must be called with one argument \textit{main}
to ensure compatibility with earlier version of the package.
It must either be empty (|\childdocmain{}|)
or precisely match the filename of the main file in which it is specified.
See \secref{sec:detection} for further information.
\item
The filename \textit{main} must be specified without the |.tex| extension.
\item
The filename \textit{main} is case sensitive
(even in case-insensitive file systems)
due to internal string comparison.
\item
The argument \textit{main} should be fully expanded, it cannot be a macro.
\item
Subdirectories and special characters should be avoided in filenames.
\item
The command |\childdocmain{|\textit{main}|}| must be followed by a whitespace.
It should not be followed immediately by another command
or by a comment mark `|%|'.
This is because the \TeX{} parser reads the token immediately following
the argument of |\childdocmain| and puts it
at the beginning of every child section;
however, a white\-space is ignored.
\end{itemize}

%%%%%%%%%%%%%%%%%%%%%%%%%%%%%%%%%%%%%%%%
\paragraph{Content of Main File.}

It is advisable to place all content in the child files included by |\include|.
Any output contained in the main file will appear in all child documents
unless suppressed manually;
it cannot be suppressed automatically by the |\includeonly| directive
and thus should normally be avoided.
A method to include some content in the main file
by means of conditional processing is described in \secref{sec:conditional}.

%%%%%%%%%%%%%%%%%%%%%%%%%%%%%%%%%%%%%%%%
\paragraph{Page Numbering.}

When only a part of the document is compiled,
the appropriate numbering of pages
(as well as other status parameters)
is determined from the |.aux| files.
The latter contain information from previous passes.
However this information needs to propagate through
all intermediate child documents.
Therefore the page numbering in child documents may well
be inconsistent until the complete document is compiled at least once.

A useful (if unconventional) way to always ensure a consistent
page numbering is to restart the numbering in each child document
and denote the pages by `\textit{child}|.|\textit{page}'
where \textit{child} represents the chapter/section number of the child file.
This can be achieved by the command
|\numberwithin{page}{|\textit{child}|}|
of the \textsf{amsmath} package
where \textit{child} can be |chapter| or |section|
depending on the chosen structuring.
Alternatively, one can modify the macro |\thepage| appropriately
and reset the counter |page| at the start of each child file.

%%%%%%%%%%%%%%%%%%%%%%%%%%%%%%%%%%%%%%%%%%%%%%%%%%%%%%%%%%%%%%%%%%%%%%%%%%%%%%%%
\subsection{Conditional Processing}
\label{sec:conditional}

The package provides a mechanism to compile different versions
of a document. To customise the versions further some conditional processing
can come in handy to distinguish which version is being compiled.
The package provides two macros to describe the compilation context:

%%%%%%%%%%%%%%%%%%%%%%%%%%%%%%%%%%%%%%%%
\DescribeMacro{\ifchilddoc}
The conditional |\ifchilddoc| distinguishes between the compilation of
child documents and the main document:
%
\begin{center}
|\ifchilddoc |\textit{child-code}| |[|\||else |\textit{main-code}]| \||fi|
\end{center}

%%%%%%%%%%%%%%%%%%%%%%%%%%%%%%%%%%%%%%%%
\DescribeMacro{\childdocname}
\DescribeMacro{\childdocjob}
The macro |\childdocname| contains the filename (without extension)
of the main or child file being processed.
Note that |\childdocjob| will always contain the name of the main file.

%%%%%%%%%%%%%%%%%%%%%%%%%%%%%%%%%%%%%%%%
\paragraph{Title Page.}

Conditional processing can be used to include a title or banner page
in the main document when proper precautions are taken.
Importantly, the code in the main file should ensure that the page counter
(as well as other status parameters which are stored in the |.aux| files)
takes the same value after the conditional processing.
Otherwise the page numbers may take divergent values
depending on which part is compiled.

For example, a title page could be declared by:
%
\begin{center}
\begin{tabular}{l}
|\ifchilddoc\||else|\\
|\addtocounter{page}{-1}|\\
\textit{code for title page}\\
|\newpage|\\
|\||fi|
\end{tabular}
\end{center}
%
A banner page for the child documents can be generated by:
%
\begin{center}
\begin{tabular}{l}
|\ifchilddoc|\\
|\addtocounter{page}{-1}|\\
\textit{code for banner page}\\
|\newpage|\\
|\||fi|
\end{tabular}
\end{center}
%
Here one could write a message such as:
\begin{center}
|This is the part \childdocname{} of \childdocjob{}.|
\end{center}

%%%%%%%%%%%%%%%%%%%%%%%%%%%%%%%%%%%%%%%%%%%%%%%%%%%%%%%%%%%%%%%%%%%%%%%%%%%%%%%%
\subsection{Flags}
\label{sec:flags}

The package makes it easy to generate different versions
of the main or child documents.
To this end compilation flags can be defined
and assigned different default values.
They will be particularly useful in conjunction
with the forwarding mechanism described in \secref{sec:forward}.

For example, it may be useful to have a flag |\version|
which can be set to |draft| or |final|.
The document source will contain some conditional code
depending on the value of |\version|.
Suppose further, the flag should default to |final| for the main file
and to |draft| for child files
which is a natural assignment for editing the document.
This is achieved by placing the following code
in the preamble of the main document
(below the |\childdocmain| directive):
%
\begin{center}
\begin{tabular}{l}
|\ifchilddoc|\\
|\providecommand{\version}{draft}|\\
|\||else|\\
|\providecommand{\version}{final}|\\
|\||fi|
\end{tabular}
\end{center}
%
The definition by |\providecommand| makes sure
that previous definitions are not overwritten.
Further statements |\providecommand{\version}{...}|
can thus be added before the above code to override it.

For the main file, one might add a line
(between |\childdocmain| and the above block)
%
\begin{center}
|%\ifchilddoc\||else\providecommand{\version}{draft}\||fi|
\end{center}
%
which can be uncommented to produce a draft version.
Likewise one can add a line to the very top of a child file
(above the |\childdocof{|\textit{main}|}| directive)
%
\begin{center}
|%\providecommand{\version}{final}|
\end{center}
%
which can be uncommented to produce the final version of this child document.

%%%%%%%%%%%%%%%%%%%%%%%%%%%%%%%%%%%%%%%%%%%%%%%%%%%%%%%%%%%%%%%%%%%%%%%%%%%%%%%%
\subsection{Forwarding}
\label{sec:forward}

Different versions of the main or child documents
using compilation flags as described in \secref{sec:flags}
can be (permanently) stored in different files
for convenient compilation, viewing and distribution.
To this end, the package defines a command
to pass on compilation to a different file:

%%%%%%%%%%%%%%%%%%%%%%%%%%%%%%%%%%%%%%%%
\DescribeMacro{\childdocforward}
The command |\childdocforward| redirects processing to
another source file:
%
\begin{center}
\begin{tabular}{l}
|\input{childdoc.def}|\\
|\childdocforward[|\textit{main}|]{|\textit{dest}|}|\\
\end{tabular}
\end{center}
%
The argument \textit{dest} is the destination file
(without extension).
It should be the main file or one of the child files.
Note that further \textsf{childdoc} directives
such as |\childdocof| and |\childdocforward|
in the indicated file will be processed in this form.
The optional argument \textit{main}
passes on directly to the main file \textit{main}
while pretending to compile the child \textit{dest}.
This form behaves as if \textit{dest}
issues |\childdocof{|\textit{main}|}| right away,
and no further \textsf{childdoc} directives will be processed.

%%%%%%%%%%%%%%%%%%%%%%%%%%%%%%%%%%%%%%%%
\DescribeMacro{\...prefix}
In the alternative form |\childdocforwardprefix|,
%
\begin{center}
\begin{tabular}{l}
|\input{childdoc.def}|\\
|\childdocforwardprefix[|\textit{main}|]{|\textit{prefix}|}{|\textit{dest}|}|
\end{tabular}
\end{center}
%
the destination file is determined by a pattern
depending on the current file:
To make this work, the current file must be called
`{\textit{prefix}\hspace{0.2em}\textit{suffix}}'
with \textit{prefix} matching precisely the argument.
Processing is then passed on to the file
`{\textit{dest}\hspace{0.2em}\textit{suffix}}'.
Surely, the same effect is achieved by
directly specifying the
argument `{\textit{dest}\hspace{0.2em}\textit{suffix}}'
in the first form.
However, that requires to set up a different file
for each child. With the alternative form of the command
all these files can have exactly the same content
which simplifies setting them up and maintaining them.

For example, the following file |draft.tex|
with a compilation flag |\version| as described in \secref{sec:flags}
compiles the main document as a draft:
%
\begin{center}
\begin{tabular}{l}
|\def\version{draft}|\\
|\input{childdoc.def}|\\
|\childdocforward{|\textit{main}|}|
\end{tabular}
\end{center}
%
Likewise, the following files |final|\textit{nn}|.tex|
compile the final version of the child document
|child|\textit{nn}|.tex|:
%
\begin{center}
\begin{tabular}{l}
|\def\version{final}|\\
|\input{childdoc.def}|\\
|\childdocforwardprefix{final}{child}|
\end{tabular}
\end{center}
%

Note that when several versions of a main file and/or of each child file
are to be generated, it may be convenient to set up a |Makefile| or
shell script to automatise the process.

%%%%%%%%%%%%%%%%%%%%%%%%%%%%%%%%%%%%%%%%%%%%%%%%%%%%%%%%%%%%%%%%%%%%%%%%%%%%%%%%
\subsection{Command Line Processing}
\label{sec:commandline}

The effect of redirection files can also be achieved by invoking
the \LaTeX{} compiler with a more elaborate command line.
Most conveniently this should be done as part
of a shell script or a |Makefile|.

When using \textsf{childdoc} in the main file, the following
command lines effectively perform a redirection
(note that depending on the shell being used,
backslashes may have to be doubled: `|\|' $\to$ `|\\|'):
%
\begin{center}
|... -jobname "|\textit{target}|" |\\|"|[\textit{flags}]%
|\input{childdoc.def}\childdocforward[|\textit{main}|]{|\textit{dest}|}"|
\end{center}
%
Here \textit{target} is the name of the output file,
\textit{main} is the name of the main file
and \textit{dest} is the name of the main or child file to be processed
(all filenames without extensions).
The optional argument \textit{main} can be omitted
if \textit{main} matches \textit{dest}.
Optionally, compilation \textit{flags} can be defined via |\def| commands.
This command line makes the \TeX{} engine believe
it is compiling the file \textit{target}
whose content is specified as the latter parameter.
The provided code then forwards the processing to
\textit{main} or \textit{dest} as described in \secref{sec:forward}.

%%%%%%%%%%%%%%%%%%%%%%%%%%%%%%%%%%%%%%%%%%%%%%%%%%%%%%%%%%%%%%%%%%%%%%%%%%%%%%%%
\subsection{Include by Input}
\label{sec:input}

Including child documents by |\include| has some restrictions by design.
Most notably, the content of a child document always occupies
its own set of pages; pages cannot be shared between child documents.
Usually, this behaviour makes perfect sense
because each child document contain an essential part of the document.
However, in some situations it may be desirable to compose
a document from a collection of parts
without having mandatory page breaks between then.
For this case, the package
provides a mechanism to include parts
by |\input| which can also be processed individually.
However, by construction this mechanism
requires manual handling of the content to be output.

%%%%%%%%%%%%%%%%%%%%%%%%%%%%%%%%%%%%%%%%
\DescribeMacro{\ifchilddocmanual}
The main file should be prepared as usual, see \secref{sec:include}.
However, the document body must make a distinction
between processing of an individual part and of the main document, e.g.:
%
\begin{center}
\begin{tabular}{l}
|\ifchilddocmanual|\\
|\input{\childdocname}|\\
|\||else|\\
\textit{document body with }|\input{|\textit{part}|}|\\
|\||fi|
\end{tabular}
\end{center}
%
The conditional |\ifchilddocmanual| is true whenever
a part to be included by |\input| is being compiled,
and the name of the part is stored in |\childdocname|.

%%%%%%%%%%%%%%%%%%%%%%%%%%%%%%%%%%%%%%%%
\DescribeMacro{\childdocby}
Each part to be included by |\input| should start with:
%
\begin{center}
\begin{tabular}{l}
|\input{childdoc.def}|\\
|\childdocby{|\textit{main}|}|\\
\end{tabular}
\end{center}
%
The directive |\childdocby| is similar to |\childdocof|
described in \secref{sec:include},
but the subsequent selection of content must be done manually.
To that end, both |\ifchilddoc| and |\ifchilddocmanual|
will be true upon processing of a part,
and the name of the part is stored in |\childdocname|.
Note that |\jobname| will be set to the filename of the current part
so that each part receives an individual |.aux| file
that does not interfere with the |.aux| file(s) of the main document.
This behaviour can be altered by the alternative form
|\childdocby[*]{|\textit{main}|}| (with a non-empty optional argument)
which uses the |.aux| file of the main document
by setting |\jobname| to \textit{main}.

%%%%%%%%%%%%%%%%%%%%%%%%%%%%%%%%%%%%%%%%%%%%%%%%%%%%%%%%%%%%%%%%%%%%%%%%%%%%%%%%
\subsection{Driver Development}
\label{sec:driver}

The \textsf{childdoc} mechanism can also be use for the development
of definition files such as \LaTeX{} styles or classes.
This case differs from the above setup with multiple parts
included by |\include| in that no |\includeonly| should be invoked.
This can be achieved by starting the include file
(before |\ProvidesPackage|) with:
%
\begin{center}
\begin{tabular}{l}
|\input{childdoc.def}|\\
|\childdocforward{|\textit{main}|}|\\
\end{tabular}
\end{center}
%
or alternatively with:
%
\begin{center}
\begin{tabular}{l}
|\input{childdoc.def}|\\
|\childdocby{|\textit{main}|}|\\
\end{tabular}
\end{center}
%
Both forms have slightly different effects as described above.
The main file is prepared as usual, see \secref{sec:include}.

%%%%%%%%%%%%%%%%%%%%%%%%%%%%%%%%%%%%%%%%%%%%%%%%%%%%%%%%%%%%%%%%%%%%%%%%%%%%%%%%
\subsection{Legacy Detection}
\label{sec:detection}

The directive |\childdocmain| in the main file can detect
whether the complete document or merely a child is to be compiled
even without using the directive |\childdocof|.
This method is deprecated because it is less robust
and there is no compelling reason to use it;
it is merely provided for backward compatibility
and it may be removed in future versions.

If the detection mechanism is to be used,
it is mandatory to correctly specify
the filename of the main file as the argument of |\childdocmain|:
%
\begin{center}
\begin{tabular}{l}
|\input{childdoc.def}|\\
|\childdocmain{|\textit{main}|}|\\
\end{tabular}
\end{center}
%
If |\jobname| does not match the argument \textit{main} of |\childdocmain|,
it is assumed that |\jobname| points to the child file to be compiled.
When using |\childdocmain| with the main file specified as argument,
it suffices to start a child file
with just |\input{|\textit{main}|}|
without loading of the package and using |\childdocof|.
If instead all processing is done
with the appropriate \textsf{childdoc} directives,
the argument of \textit{main} of |\childdocmain| can be empty.

An alternative version of the command line processing described
in \secref{sec:commandline} using the detection mechanism reads:
%
\begin{center}
|... -jobname "|\textit{target}|" "|[\textit{flags}]%
[|\def\jobname{|\textit{dest}|}|]|\input{|\textit{main}|}"|
\end{center}

%%%%%%%%%%%%%%%%%%%%%%%%%%%%%%%%%%%%%%%%%%%%%%%%%%%%%%%%%%%%%%%%%%%%%%%%%%%%%%%%
\subsection{Manual Code}
\label{sec:manual}

In case one cannot be certain whether the definitions file |childdoc.def|
is installed on the target \TeX{} distribution
and one prefers not to ship it,
it is conceivable to paste a few relevant commands into the sources.

To that end, drop all statements |\input{childdoc.def}|
and perform the replacements as outlined below.
Instead of |\childdocmain{|\textit{main}|}| add the following code
to the top of the main file:
%
\begin{center}
\begin{tabular}{l}
|\||ifdefined\childdocname\endinput\||fi\newif\ifchilddoc|\\
|\edef\childdocname{\scantokens\expandafter{\jobname\noexpand}}|\\
|\def\childdocmain{|\textit{main}|}\||ifx\childdocmain\childdocname\||else|\\
|\childdoctrue\includeonly{\childdocname}\let\jobname\childdocmain\||fi|\\
\end{tabular}
\end{center}
%
Instead of |\childdocof{|\textit{main}|}| just include the main file
at the top of each child file:
%
\begin{center}
|\input{|\textit{main}|}|
\end{center}
%
A simple redirection |\childdocforward{|\textit{dest}|}| is achieved by:
%
\begin{center}
|\def\jobname{|\textit{dest}|}\input{\jobname}|
\end{center}
%
The redirection with prefix
|\childdocforwardprefix[|\textit{prefix}|]{|\textit{dest}|}|
is accomplished by:
%
\begin{center}
\begin{tabular}{l}
|{\edef\jobname{\scantokens\expandafter{\jobname\noexpand}}|\\
|\def\redirectjob |\textit{prefix}|#1~~~{\gdef\jobname{|\textit{dest}|#1}}|\\
|\expandafter\redirectjob\jobname~~~}\input{\jobname}|
\end{tabular}
\end{center}

In an alternative approach,
child documents can be compiled by a specific command line
without additional code or specific definitions:
%
\begin{center}
|... -jobname "|\textit{target}|" "|[\textit{flags}]%
|\includeonly{|\textit{dest}|}\input{|\textit{main}|}"|
\end{center}
%

%%%%%%%%%%%%%%%%%%%%%%%%%%%%%%%%%%%%%%%%%%%%%%%%%%%%%%%%%%%%%%%%%%%%%%%%%%%%%%%%
%%%%%%%%%%%%%%%%%%%%%%%%%%%%%%%%%%%%%%%%%%%%%%%%%%%%%%%%%%%%%%%%%%%%%%%%%%%%%%%%
\section{Information}

%%%%%%%%%%%%%%%%%%%%%%%%%%%%%%%%%%%%%%%%%%%%%%%%%%%%%%%%%%%%%%%%%%%%%%%%%%%%%%%%
\subsection{Copyright}

Copyright \copyright{} 2017--2018 Niklas Beisert

This work may be distributed and/or modified under the
conditions of the \LaTeX{} Project Public License, either version 1.3
of this license or (at your option) any later version.
The latest version of this license is in
  \url{http://www.latex-project.org/lppl.txt}
and version 1.3 or later is part of all distributions of \LaTeX{}
version 2005/12/01 or later.

This work has the LPPL maintenance status `maintained'.

The Current Maintainer of this work is Niklas Beisert.

This work consists of the files |README.txt|, |childdoc.ins| and |childdoc.dtx|
as well as the derived files |childdoc.def|, |cdocsamp.tex|
with |cdocsch1.tex|, |cdocsch2.tex|, |cdocspt3.tex|, |cdocspt4.tex|,
|cdocsdrf.tex|, |cdocsfn1.tex|, |cdocsfn2.tex|
as well as |childdoc.pdf|.

%%%%%%%%%%%%%%%%%%%%%%%%%%%%%%%%%%%%%%%%%%%%%%%%%%%%%%%%%%%%%%%%%%%%%%%%%%%%%%%%
\subsection{Files and Installation}

The package consists of the files:
%
\begin{center}
\begin{tabular}{ll}
    |README.txt|   & readme file \\
    |childdoc.ins| & installation file \\
    |childdoc.dtx| & source file \\
    |childdoc.def| & definition file \\
    |cdocsamp.tex| & sample main file \\
    |cdocsch1.tex| & sample include file \\
    |cdocsch2.tex| & sample include file \\
    |cdocspt3.tex| & sample part file \\
    |cdocspt4.tex| & sample part file \\
    |cdocsdrf.tex| & sample redirection file \\
    |cdocsfn1.tex| & sample redirection file \\
    |cdocsfn2.tex| & sample redirection file \\
    |childdoc.pdf| & manual
\end{tabular}
\end{center}
%
The distribution consists of the files
|README.txt|, |childdoc.ins| and |childdoc.dtx|.
%
\begin{itemize}
\item
Run (pdf)\LaTeX{} on |childdoc.dtx|
to compile the manual |childdoc.pdf| (this file).
\item
Run \LaTeX{} on |childdoc.ins| to create the definitions file |childdoc.def|
and the sample |cdocsamp.tex| with include files
|cdocsch1.tex|, |cdocsch2.tex|, |cdocspt3.tex|, |cdocspt4.tex|,
|cdocsdrf.tex|, |cdocsfn1.tex|, |cdocsfn2.tex|.
Then copy the file |childdoc.def| to an appropriate directory of your \LaTeX{}
distribution, e.g.\ \textit{texmf-root}|/tex/latex/childdoc|.
\end{itemize}

%%%%%%%%%%%%%%%%%%%%%%%%%%%%%%%%%%%%%%%%%%%%%%%%%%%%%%%%%%%%%%%%%%%%%%%%%%%%%%%%
\subsection{Related CTAN Packages}

There are several other packages which offer a similar functionality:
%
\begin{itemize}
\item
The packages
\href{http://ctan.org/pkg/docmute}{\textsf{docmute}},
\href{http://ctan.org/pkg/includex}{\textsf{includex}} and
\href{http://ctan.org/pkg/standalone}{\textsf{standalone}}
provide commands to include only the document body of
a child file thus allowing both files to be compiled individually.
\item
The packages \href{http://ctan.org/pkg/subdocs}{\textsf{subdocs}}
and \href{http://ctan.org/pkg/subfiles}{\textsf{subfiles}}
provide structures in which the main and child documents can be
encapsulated and allowing them to be compiled individually.
The inclusion mechanism is different from the conventional |\include|.
\item
The package \href{http://ctan.org/pkg/combine}{\textsf{combine}}
is an elaborate solution to combine several documents into one.
\end{itemize}
%
See also the CTAN topic \href{http://ctan.org/topic/subdocs}{\textsf{subdocs}}
for further related packages.
The present package differs from the above solutions in that
a document structure constructed with the conventional |\include| mechanism
just needs two extra commands at the top of every file
such that all constituent files can be compiled individually.

%%%%%%%%%%%%%%%%%%%%%%%%%%%%%%%%%%%%%%%%%%%%%%%%%%%%%%%%%%%%%%%%%%%%%%%%%%%%%%%%
%\subsection{Feature Suggestions}
%
%The following is a list of features which may be useful for future
%versions of this package:
%%
%\begin{itemize}
%\item
%\ldots
%\end{itemize}

%%%%%%%%%%%%%%%%%%%%%%%%%%%%%%%%%%%%%%%%%%%%%%%%%%%%%%%%%%%%%%%%%%%%%%%%%%%%%%%%
\subsection{Revision History}

%%%%%%%%%%%%%%%%%%%%%%%%%%%%%%%%%%%%%%%%
\paragraph{v2.0:} 2018/12/30

\begin{itemize}
\item
immediate forward processing
\item
added |\childdocby| mechanism
\item
manual restructured
\end{itemize}

%%%%%%%%%%%%%%%%%%%%%%%%%%%%%%%%%%%%%%%%
\paragraph{v1.6:} 2018/01/17

\begin{itemize}
\item
application for development of include files
\item
corrections to manual
\end{itemize}

%%%%%%%%%%%%%%%%%%%%%%%%%%%%%%%%%%%%%%%%
\paragraph{v1.5:} 2017/05/21

\begin{itemize}
\item
more complete structuring introduced
\item
|\childdocof| introduced
\item
|\childdoc| renamed to |\childdocmain|
\item
|\childredirect| renamed to |\childdocforward| and |\childdocforwardprefix|
and functionality expanded
\end{itemize}

%%%%%%%%%%%%%%%%%%%%%%%%%%%%%%%%%%%%%%%%
\paragraph{v1.0:} 2017/04/27

\begin{itemize}
\item
manual and install package
\item
first version published on CTAN
\end{itemize}

%%%%%%%%%%%%%%%%%%%%%%%%%%%%%%%%%%%%%%%%
\paragraph{v0.6:} 2017/04/26

\begin{itemize}
\item
redirection mechanism added
\end{itemize}

%%%%%%%%%%%%%%%%%%%%%%%%%%%%%%%%%%%%%%%%
\paragraph{v0.5:} 2017/04/26

\begin{itemize}
\item
functionality in definition file
\end{itemize}


%%%%%%%%%%%%%%%%%%%%%%%%%%%%%%%%%%%%%%%%%%%%%%%%%%%%%%%%%%%%%%%%%%%%%%%%%%%%%%%%
%%%%%%%%%%%%%%%%%%%%%%%%%%%%%%%%%%%%%%%%%%%%%%%%%%%%%%%%%%%%%%%%%%%%%%%%%%%%%%%%
%%%%%%%%%%%%%%%%%%%%%%%%%%%%%%%%%%%%%%%%%%%%%%%%%%%%%%%%%%%%%%%%%%%%%%%%%%%%%%%%
\appendix

\settowidth\MacroIndent{\rmfamily\scriptsize 000\ }

 \DocInput{childdoc.dtx}

\end{document}
%</driver>
% \fi
%
% %%%%%%%%%%%%%%%%%%%%%%%%%%%%%%%%%%%%%%%%%%%%%%%%%%%%%%%%%%%%%%%%%%%%%%%%%%%%%%
% %%%%%%%%%%%%%%%%%%%%%%%%%%%%%%%%%%%%%%%%%%%%%%%%%%%%%%%%%%%%%%%%%%%%%%%%%%%%%%
% \section{Sample}
%\iffalse
%<*samplemain>
%\fi
%
% The following presents a sample document
% with two chapters, two parts, a title page,
% a compile flag as well as three forwarding files to set the flag.
% It consists of eight |.tex| files:
% \begin{center}
% \begin{tabular}{ll}
% |cdocsamp.tex|&main file\\
% |cdocsch1.tex|&include file for chapter 1\\
% |cdocsch2.tex|&include file for chapter 2\\
% |cdocspt3.tex|&include file for part 3\\
% |cdocspt4.tex|&include file for part 4\\
% |cdocsdrf.tex|&forwarding file for main file in draft mode\\
% |cdocsfi1.tex|&forwarding file for final version of chapter 1\\
% |cdocsfi2.tex|&forwarding file for final version of chapter 2\\
% \end{tabular}
% \end{center}
% Each of the eight files can be compiled directly by the \LaTeX{} compiler.
%
% %%%%%%%%%%%%%%%%%%%%%%%%%%%%%%%%%%%%%%
% \paragraph{Main File.}
%
% The main file is called |cdocsamp.tex|.
%
% Load the \textsf{childdoc} definitions and
% declare the filename for the main document:
%    \begin{macrocode}
\input{childdoc.def}
\childdocmain{}
%    \end{macrocode}

% Optional override for |\version| flag:
%    \begin{macrocode}
%%\ifchilddoc\else\providecommand{\version}{draft}\fi
%    \end{macrocode}

% Define the default values for the |\version| flag
% (|final| for the main file and |draft| for childs):
%    \begin{macrocode}
\ifchilddoc
\providecommand{\version}{draft}
\else
\providecommand{\version}{final}
\fi
%    \end{macrocode}

% Load the standard document class:
%    \begin{macrocode}
\documentclass[12pt]{article}
%    \end{macrocode}

% Start the document body:
%    \begin{macrocode}
\begin{document}
%    \end{macrocode}

% Declare a title page.
% Print title, part of document being processed and version flag:
%    \begin{macrocode}
\addtocounter{page}{-1}
\begin{center}
{\LARGE\bfseries{}childdoc example\par}
\vspace{1cm}
\ifchilddoc
\ifchilddocmanual part\else chapter\fi:
`\childdocname' of `\childdocjob'\par
\else
main document: `\childdocjob'\par
\fi
version: \version\par
\end{center}
\newpage
%    \end{macrocode}

% Manually include selected file,
% otherwise process as usual:
%    \begin{macrocode}
\ifchilddocmanual
\section*{part `\childdocname'}
\input{\childdocname}
\else
%    \end{macrocode}

% Include the two chapters:
%    \begin{macrocode}
\include{cdocsch1}
\include{cdocsch2}
%    \end{macrocode}

% Include the two parts unless only chapters should be displayed:
%    \begin{macrocode}
\ifchilddoc\else
\section{part three}
\input{cdocspt3}
\section{part four}
\input{cdocspt4}
\fi
%    \end{macrocode}

% Process as usual until here:
%    \begin{macrocode}
\fi
%    \end{macrocode}

% End of document body:
%    \begin{macrocode}
\end{document}
%    \end{macrocode}
%\iffalse
%</samplemain>
%\fi
%
% %%%%%%%%%%%%%%%%%%%%%%%%%%%%%%%%%%%%%%
% \paragraph{Chapter Include Files.}
%
% The include files are called |cdocsch1.tex| and |cdocsch2.tex|.
%
%\iffalse
%<*samplechap1|samplechap2>
%\fi

% Optional override for |\version| flag:
%    \begin{macrocode}
%%\providecommand{\version}{final}
%    \end{macrocode}

% Include the main document:
%    \begin{macrocode}
\input{childdoc.def}
\childdocof{cdocsamp}
%    \end{macrocode}

%\iffalse
%</samplechap1|samplechap2>
%\fi
%
%\iffalse
%<*samplechap1>
%\fi
% Some text for chapter 1:
%    \begin{macrocode}
\section{one}
some text in chapter one
%    \end{macrocode}

%\iffalse
%</samplechap1>
%\fi
% Some text for chapter 2:
%\iffalse
%<*samplechap2>
%\fi
%    \begin{macrocode}
\section{two}
more text in chapter two
%    \end{macrocode}

%\iffalse
%</samplechap2>
%\fi
%
% %%%%%%%%%%%%%%%%%%%%%%%%%%%%%%%%%%%%%%
% \paragraph{Part Include Files.}
%
% The include files are called |cdocspt3.tex| and |cdocspt4.tex|.
%
%\iffalse
%<*samplepart3|samplepart4>
%\fi

% Optional override for |\version| flag:
%    \begin{macrocode}
%%\providecommand{\version}{final}
%    \end{macrocode}

% Include the main document:
%    \begin{macrocode}
\input{childdoc.def}
\childdocby{cdocsamp}
%    \end{macrocode}

%\iffalse
%</samplepart3|samplepart4>
%\fi
%
%\iffalse
%<*samplepart3>
%\fi
% Some text for part 3:
%    \begin{macrocode}
some text in part three
%    \end{macrocode}

%\iffalse
%</samplepart3>
%\fi
% Some text for part 4:
%\iffalse
%<*samplepart4>
%\fi
%    \begin{macrocode}
more text in part four
%    \end{macrocode}

%\iffalse
%</samplepart4>
%\fi
%
% %%%%%%%%%%%%%%%%%%%%%%%%%%%%%%%%%%%%%%
% \paragraph{Forwarding for a Complete Draft.}
%
% The following forwarding file |cdocsdrf.tex|
% compiles the main document in draft mode:
%\iffalse
%<*sampledraft>
%\fi
%    \begin{macrocode}
\def\version{draft}
\input{childdoc.def}
\childdocforward{cdocsamp}
%    \end{macrocode}

%\iffalse
%</sampledraft>
%\fi
%
% %%%%%%%%%%%%%%%%%%%%%%%%%%%%%%%%%%%%%%
% \paragraph{Forwarding for Final Version of the Chapters.}
%
% The following forwarding files |cdocsfn1.tex| and |cdocsfn2.tex|
% (with identical content)
% compile the final versions of the child documents
% |cdocsch1.tex| and |cdocsch2.tex|, respectively:
%\iffalse
%<*samplefinal>
%\fi
%    \begin{macrocode}
\def\version{final}
\input{childdoc.def}
\childdocforwardprefix[cdocsamp]{cdocsfn}{cdocsch}
%    \end{macrocode}

%\iffalse
%</samplefinal>
%\fi
%
% %%%%%%%%%%%%%%%%%%%%%%%%%%%%%%%%%%%%%%
% \paragraph{Command Line Processing.}
%
% The following three command lines generate the output files
% |cdocscld|, |cdocscl1| and |cdocscl2|
% which should be identical to
% |cdocsdrf|, |cdocsch1| and |cdocsfn2|, respectively:
% \begin{center}
% \begin{tabular}{l}
% |latex -jobname cdocscld \|\\
% |  "\def\version{draft}\input{childdoc.def}\childdocforward{cdocsamp}"|\\
% |latex -jobname cdocscl1 \|\\
% |  "\input{childdoc.def}\childdocforward[cdocsamp]{cdocsch1}"|\\
% |latex -jobname cdocscl2 \|\\
% |  "\def\version{final}\input{childdoc.def}\childdocforward{cdocsch2}"|
% \end{tabular}
% \end{center}
% Note that the trailing backslash on each first line
% merely continues the input to the second line
% (for convenient cut ant paste).
% Furthermore, the command |latex| can be replaced by any
% of its alternative versions such as |pdflatex|.
%
% %%%%%%%%%%%%%%%%%%%%%%%%%%%%%%%%%%%%%%%%%%%%%%%%%%%%%%%%%%%%%%%%%%%%%%%%%%%%%%
% %%%%%%%%%%%%%%%%%%%%%%%%%%%%%%%%%%%%%%%%%%%%%%%%%%%%%%%%%%%%%%%%%%%%%%%%%%%%%%
% \section{Implementation}
%\iffalse
%<*package>
%\fi
%
% This section describes the definitions file |childdoc.def|.

% The definitions cannot be loaded using |\usepackage| or |\RequirePackage|
% which has a mechanism to prevent loading a style file more than once.
% When loading the definitions by means of |\input|
% multiple instances have to be prevented manually:
%\iffalse
%This code needs to be before the `\ProvidesFile' directive
%which is defined at the beginning of this file.
%Therefore it is also placed there and commented out here.
%</package>
%<*discard>
%\fi
%    \begin{macrocode}
\ifdefined\childdocmain\endinput\fi
%    \end{macrocode}
%\iffalse
%</discard>
%<*package>
%\fi
%
% \macro{\ifchilddoc}
% \macro{\ifchilddocmanual}
% The conditional |\ifchilddoc| tells whether a
% child (true) or main (false) document is being compiled.
% The conditional |\ifchilddocmanual| tells whether
% the |\includeonly| mechanism is used (false) or
% the selection of child files must be performed manually (true).
% The definitions initialise to false:
%    \begin{macrocode}
\newif\ifchilddoc
\newif\ifchilddocmanual
%    \end{macrocode}

% \macro{\childdocname}
% \macro{\childdocjob}
% The macro |\childdocname| stores the name of the main document
% to be compiled. The macro |\childdocjob| stores the name of
% the document on which the \LaTeX{} compiler was originally invoked.
% The content of |\jobname| cannot be compared
% to filenames specified in the source due to different catcodes.
% The following code rescans |\jobname|, stores the result
% in |\childdocname| and saves a copy in |\childdocjob|:
%    \begin{macrocode}
\edef\childdocname{\scantokens\expandafter{\jobname\noexpand}}
\let\childdocjob\childdocname
%    \end{macrocode}

% \macro{\childdocdisable}
% The macro |\childdocdisable| prevents the main file
% from being processed more than once.
% At this stage, the main document command |\childdocmain|
% is assumed to be called once again where it should do nothing.
% Any subsequent call to it should prevent
% a secondary processing of the main document
% It overwrites the forwarding commands
% |\childdocof| and |\childdocforward|
% with empty macros to prevent further inclusions of the main document:
%    \begin{macrocode}
\newcommand{\childdocdisable}
{
  \renewcommand{\childdocmain}[1]{\renewcommand{\childdocmain}[1]{\endinput}}
  \renewcommand{\childdocof}[1]{}
  \renewcommand{\childdocby}[2][]{}
  \renewcommand{\childdocforward}[2][]{}
  \renewcommand{\childdocdisable}{}
}
%    \end{macrocode}

% \macro{\childdocmain}
% The macro |\childdocmain| is to be called at the top of the main file
% with nothing or the main filename (without extension) as argument.
% First, it breaks loops.
% If the argument is not empty and does not match |\childdocname|
% (which is set by the first inclusion of |childdoc.def|),
% |\ifchilddoc| is set to true, |\includeonly| is applied to the child file
% and |\jobname| is set to the main file
% (for proper handling of |.aux| files):
%    \begin{macrocode}
\newcommand{\childdocmain}[1]
{
  \childdocdisable\childdocmain{}
  \if?#1?\else
    \begingroup
      \def\childdoctmp{#1}
      \ifx\childdoctmp\childdocname
        \def\childdoctmp{}
      \else
        \def\childdoctmp
        {
          \childdoctrue
          \includeonly{\childdocname}
          \def\childdocjob{#1}
          \def\jobname{#1}
        }
      \fi
      \expandafter
    \endgroup
    \childdoctmp
  \fi
}
%    \end{macrocode}

% \macro{\childdocof}
% The command |\childdocof| redirects
% compilation to the main file |#1|.
%    \begin{macrocode}
\newcommand{\childdocof}[1]
{
  \childdocdisable
  \childdoctrue
  \includeonly{\childdocname}
  \def\jobname{#1}
  \def\childdocjob{#1}
  \input{#1}
}
%    \end{macrocode}

% \macro{\childdocby}
% The command |\childdocby| ....
%    \begin{macrocode}
\newcommand{\childdocby}[2][]
{
  \childdocdisable
  \childdoctrue
  \childdocmanualtrue
  \if?#1?\else
    \def\jobname{#2}
  \fi
  \def\childdocjob{#2}
  \input{#2}
  \endinput
}
%    \end{macrocode}

% \macro{\childdocforward}
% The command |\childdocforward| redirects
% compilation to the main file or
% (if the optional argument is given) a child file.
% Parameters are set as if the main file
% or a child file starting with |\childdocof| was compiled.
% Then compilation is handed over to the main file:
%    \begin{macrocode}
\newcommand{\childdocforward}[2][]
{
  \begingroup
    \if?#1?
      \def\childdoctmp
      {
        \def\childdocname{#2}
        \def\childdocjob{#2}
        \def\jobname{#2}
        \input{#2}
        \endinput
      }
    \else
      \def\childdoctmp
      {
        \childdocdisable
        \def\childdocname{#2}
        \childdoctrue
        \includeonly{#2}
        \def\childdocjob{#1}
        \def\jobname{#1}
        \input{#1}
        \endinput
      }
    \fi
    \expandafter
  \endgroup
  \childdoctmp
}
%    \end{macrocode}

% \macro{\childdocforwardprefix}
% The command |\childdocforwardprefix| redirects
% compilation to the main or a child file by means of a pattern.
% The prefix |#1| in the current filename is replaced by |#2|
% and the suffix of the current filename is kept
% (it is assumed that the filename does not contain the substring `|~~~|'
% which is used as a delimiter).
% Compilation is handed over to the new file by |\childdocforward|:
%    \begin{macrocode}
\newcommand{\childdocforwardprefix}[3][]
{
  \begingroup
    \def\childdocextract #2##1~~~{\def\childdoctmp{\childdocforward[#1]{#3##1}}}
    \expandafter\childdocextract\childdocname~~~
    \expandafter
  \endgroup
  \childdoctmp
}
%    \end{macrocode}

% \macro{\childdoc}
% The deprecated macro |\childdoc| is a legacy version of |\childdocmain|:
%    \begin{macrocode}
\newcommand{\childdoc}{\childdocmain}
%    \end{macrocode}

% \macro{\childdocredirect}
% The deprecated macro |\childdocredirect| is a legacy version
% of |\childdocforward| and |\childdocforwardprefix|:
%    \begin{macrocode}
\newcommand{\childdocredirect}[2][]
{
  \begingroup
    \if?#1?
      \def\childdoctmp{\childdocforward{#2}}
    \else
      \def\childdoctmp{\childdocforwardprefix{#1}{#2}}
    \fi
    \expandafter
  \endgroup
  \childdoctmp
}
%    \end{macrocode}

%\iffalse
%</package>
%\fi
%
\endinput

\childdocforwardprefix[cdocsamp]{cdocsfn}{cdocsch}
%    \end{macrocode}

%\iffalse
%</samplefinal>
%\fi
%
% %%%%%%%%%%%%%%%%%%%%%%%%%%%%%%%%%%%%%%
% \paragraph{Command Line Processing.}
%
% The following three command lines generate the output files
% |cdocscld|, |cdocscl1| and |cdocscl2|
% which should be identical to
% |cdocsdrf|, |cdocsch1| and |cdocsfn2|, respectively:
% \begin{center}
% \begin{tabular}{l}
% |latex -jobname cdocscld \|\\
% |  "\def\version{draft}% \iffalse
%
% childdoc.dtx Copyright (C) 2017-2018 Niklas Beisert
%
% This work may be distributed and/or modified under the
% conditions of the LaTeX Project Public License, either version 1.3
% of this license or (at your option) any later version.
% The latest version of this license is in
%   http://www.latex-project.org/lppl.txt
% and version 1.3 or later is part of all distributions of LaTeX
% version 2005/12/01 or later.
%
% This work has the LPPL maintenance status `maintained'.
%
% The Current Maintainer of this work is Niklas Beisert.
%
% This work consists of the files childdoc.dtx and childdoc.ins
% and the derived files childdoc.def and cdocsamp.tex with
% cdocsch1.tex, cdocsch2.tex, cdocsdrf.tex, cdocsfn1.tex, cdocsfn2.tex.
%
%<package>\ifdefined\childdocmain\endinput\fi
%<package>\ProvidesFile{childdoc.def}[2018/12/30 v2.0 child document driver]
%<samplemain>\ProvidesFile{cdocsamp.tex}[2018/12/30 v2.0 sample for childdoc]
%<*driver>
%\ProvidesFile{childdoc.drv}[2018/12/30 v2.0 childdoc reference manual file]
\PassOptionsToClass{10pt,a4paper}{article}
\documentclass{ltxdoc}

\usepackage[margin=35mm]{geometry}
\usepackage{hyperref}
\usepackage{hyperxmp}
\usepackage[usenames]{color}

\hypersetup{colorlinks=true}
\hypersetup{pdfstartview=FitH}
\hypersetup{pdfpagemode=UseNone}
\hypersetup{pdfsource={}}
\hypersetup{pdflang={en-UK}}
\hypersetup{pdfcopyright={Copyright 2017-2018 Niklas Beisert.
  This work may be distributed and/or modified under the
  conditions of the LaTeX Project Public License, either version 1.3
  of this license or (at your option) any later version.}}
\hypersetup{pdflicenseurl={http://www.latex-project.org/lppl.txt}}
\hypersetup{pdfcontactaddress={ETH Zurich, ITP, HIT K,
  Wolfgang-Pauli-Strasse 27}}
\hypersetup{pdfcontactpostcode={8093}}
\hypersetup{pdfcontactcity={Zurich}}
\hypersetup{pdfcontactcountry={Switzerland}}
\hypersetup{pdfcontactemail={nbeisert@itp.phys.ethz.ch}}
\hypersetup{pdfcontacturl={http://people.phys.ethz.ch/\xmptilde nbeisert/}}

\newcommand{\secref}[1]{\hyperref[#1]{section \ref*{#1}}}

\parskip1ex
\parindent0pt
\let\olditemize\itemize
\def\itemize{\olditemize\parskip0pt}

\begin{document}

\title{The \textsf{childdoc} Package}
\hypersetup{pdftitle={The childdoc Package}}
\author{Niklas Beisert\\[2ex]
  Institut f\"ur Theoretische Physik\\
  Eidgen\"ossische Technische Hochschule Z\"urich\\
  Wolfgang-Pauli-Strasse 27, 8093 Z\"urich, Switzerland\\[1ex]
  \href{mailto:nbeisert@itp.phys.ethz.ch}
  {\texttt{nbeisert@itp.phys.ethz.ch}}}
\hypersetup{pdfauthor={Niklas Beisert}}
\hypersetup{pdfsubject={Manual for the LaTeX2e Package childdoc}}
\date{30 December 2018, \textsf{v2.0}}
\maketitle

\begin{abstract}\noindent
\textsf{childdoc} is a \LaTeXe{} package
that enables the direct compilation
of document sections included by |\include|
to individual files.
\end{abstract}

\begingroup
\parskip0ex
\tableofcontents
\endgroup

%%%%%%%%%%%%%%%%%%%%%%%%%%%%%%%%%%%%%%%%%%%%%%%%%%%%%%%%%%%%%%%%%%%%%%%%%%%%%%%%
%%%%%%%%%%%%%%%%%%%%%%%%%%%%%%%%%%%%%%%%%%%%%%%%%%%%%%%%%%%%%%%%%%%%%%%%%%%%%%%%
\section{Introduction}

\LaTeX{} provides a mechanism to structure a large document (such as a book)
into a main file and several child files (containing the chapters)
using the |\include| command.
This mechanism is beneficial for documents
which span hundreds of pages in order to
make the source file(s) more manageable.
Moreover, compilation can be restricted to
selected child files by means of the |\includeonly| command.
The latter feature can be used to reduce the compilation time while editing
(this was significantly more useful in the earlier days of \LaTeX{})
or to generate a smaller document which is easier to navigate.
Another application of |\includeonly| is to generate
documents consisting of selected parts of the complete document.

However, there are a few drawbacks of the plain |\include| mechanism:
\begin{itemize}
\item
The child files cannot be compiled on their own,
they can only be compiled via the main file.
A naive editing environment
(such as a text editor with an option
to have the current file processed by \LaTeX)
may require one to switch to the main file before compiling;
attempting to compile the child file produces errors.
\item
The main file must be modified (each time)
to adjust the |\includeonly| command
to the present needs. This easily leaves the main file in a messy state.
\item
The generated document will always carry the filename
of the main document. This is inconvenient if
several child files are to be compiled and
to be kept for distribution.
\end{itemize}

The present package provides a simple interface
to make child files individually compilable by \LaTeX{}.
Compiling a child file then has the same effect as compiling
the main file with an |\includeonly| command
to select the appropriate child.
Moreover the generated document will carry the name of the child
rather than the main file.
This resolves all three above issues.

This feature is meant to make the editing of books,
thesis documents and lecture notes somewhat more convenient.
However, the package can also be used efficiently for
composing a series of documents (such as exercise sheets)
which are typically distributed individually.
It then assists the author in generating the individual documents
(potentially in different versions)
as well as a document containing the collected series.
Another application is in developing style files
or other kinds of included material
where compilation of the style file could redirect
to a sample or test file.

%%%%%%%%%%%%%%%%%%%%%%%%%%%%%%%%%%%%%%%%%%%%%%%%%%%%%%%%%%%%%%%%%%%%%%%%%%%%%%%%
%%%%%%%%%%%%%%%%%%%%%%%%%%%%%%%%%%%%%%%%%%%%%%%%%%%%%%%%%%%%%%%%%%%%%%%%%%%%%%%%
\section{Usage}

First of all, the package \textsf{childdoc} is \emph{not} a standard
\LaTeXe{} |.sty| style file! Therefore it needs to be invoked in
a non-standard way.

%%%%%%%%%%%%%%%%%%%%%%%%%%%%%%%%%%%%%%%%%%%%%%%%%%%%%%%%%%%%%%%%%%%%%%%%%%%%%%%%
\subsection{Included Files}
\label{sec:include}

%%%%%%%%%%%%%%%%%%%%%%%%%%%%%%%%%%%%%%%%
\DescribeMacro{\childdocmain}
To use the package, add the commands
\begin{center}
\begin{tabular}{l}
|\input{childdoc.def}|\\
|\childdocmain{}|\\
\end{tabular}
\end{center}
at the very top of the main \LaTeX{} file,
in particular \emph{before} the |\documentclass| statement!
The argument of |\childdocmain| should be left empty
(but it must be present).

%%%%%%%%%%%%%%%%%%%%%%%%%%%%%%%%%%%%%%%%
\DescribeMacro{\childdocof}
Furthermore, add the commands
\begin{center}
\begin{tabular}{l}
|\input{childdoc.def}|\\
|\childdocof{|\textit{main}|}|\\
\end{tabular}
\end{center}
at the top of every child file \textit{child}
which is included by |\include{|\textit{child}|}|
from within the main file
(or at least for those files to be compiled individually).
The argument \textit{main} must be the filename of the main file.

There are a couple of
considerations in setting up the main and child documents:

%%%%%%%%%%%%%%%%%%%%%%%%%%%%%%%%%%%%%%%%
\paragraph{Restrictions.}

Please note the following restrictions:
\begin{itemize}
\item
|\childdocmain| must be called with one argument \textit{main}
to ensure compatibility with earlier version of the package.
It must either be empty (|\childdocmain{}|)
or precisely match the filename of the main file in which it is specified.
See \secref{sec:detection} for further information.
\item
The filename \textit{main} must be specified without the |.tex| extension.
\item
The filename \textit{main} is case sensitive
(even in case-insensitive file systems)
due to internal string comparison.
\item
The argument \textit{main} should be fully expanded, it cannot be a macro.
\item
Subdirectories and special characters should be avoided in filenames.
\item
The command |\childdocmain{|\textit{main}|}| must be followed by a whitespace.
It should not be followed immediately by another command
or by a comment mark `|%|'.
This is because the \TeX{} parser reads the token immediately following
the argument of |\childdocmain| and puts it
at the beginning of every child section;
however, a white\-space is ignored.
\end{itemize}

%%%%%%%%%%%%%%%%%%%%%%%%%%%%%%%%%%%%%%%%
\paragraph{Content of Main File.}

It is advisable to place all content in the child files included by |\include|.
Any output contained in the main file will appear in all child documents
unless suppressed manually;
it cannot be suppressed automatically by the |\includeonly| directive
and thus should normally be avoided.
A method to include some content in the main file
by means of conditional processing is described in \secref{sec:conditional}.

%%%%%%%%%%%%%%%%%%%%%%%%%%%%%%%%%%%%%%%%
\paragraph{Page Numbering.}

When only a part of the document is compiled,
the appropriate numbering of pages
(as well as other status parameters)
is determined from the |.aux| files.
The latter contain information from previous passes.
However this information needs to propagate through
all intermediate child documents.
Therefore the page numbering in child documents may well
be inconsistent until the complete document is compiled at least once.

A useful (if unconventional) way to always ensure a consistent
page numbering is to restart the numbering in each child document
and denote the pages by `\textit{child}|.|\textit{page}'
where \textit{child} represents the chapter/section number of the child file.
This can be achieved by the command
|\numberwithin{page}{|\textit{child}|}|
of the \textsf{amsmath} package
where \textit{child} can be |chapter| or |section|
depending on the chosen structuring.
Alternatively, one can modify the macro |\thepage| appropriately
and reset the counter |page| at the start of each child file.

%%%%%%%%%%%%%%%%%%%%%%%%%%%%%%%%%%%%%%%%%%%%%%%%%%%%%%%%%%%%%%%%%%%%%%%%%%%%%%%%
\subsection{Conditional Processing}
\label{sec:conditional}

The package provides a mechanism to compile different versions
of a document. To customise the versions further some conditional processing
can come in handy to distinguish which version is being compiled.
The package provides two macros to describe the compilation context:

%%%%%%%%%%%%%%%%%%%%%%%%%%%%%%%%%%%%%%%%
\DescribeMacro{\ifchilddoc}
The conditional |\ifchilddoc| distinguishes between the compilation of
child documents and the main document:
%
\begin{center}
|\ifchilddoc |\textit{child-code}| |[|\||else |\textit{main-code}]| \||fi|
\end{center}

%%%%%%%%%%%%%%%%%%%%%%%%%%%%%%%%%%%%%%%%
\DescribeMacro{\childdocname}
\DescribeMacro{\childdocjob}
The macro |\childdocname| contains the filename (without extension)
of the main or child file being processed.
Note that |\childdocjob| will always contain the name of the main file.

%%%%%%%%%%%%%%%%%%%%%%%%%%%%%%%%%%%%%%%%
\paragraph{Title Page.}

Conditional processing can be used to include a title or banner page
in the main document when proper precautions are taken.
Importantly, the code in the main file should ensure that the page counter
(as well as other status parameters which are stored in the |.aux| files)
takes the same value after the conditional processing.
Otherwise the page numbers may take divergent values
depending on which part is compiled.

For example, a title page could be declared by:
%
\begin{center}
\begin{tabular}{l}
|\ifchilddoc\||else|\\
|\addtocounter{page}{-1}|\\
\textit{code for title page}\\
|\newpage|\\
|\||fi|
\end{tabular}
\end{center}
%
A banner page for the child documents can be generated by:
%
\begin{center}
\begin{tabular}{l}
|\ifchilddoc|\\
|\addtocounter{page}{-1}|\\
\textit{code for banner page}\\
|\newpage|\\
|\||fi|
\end{tabular}
\end{center}
%
Here one could write a message such as:
\begin{center}
|This is the part \childdocname{} of \childdocjob{}.|
\end{center}

%%%%%%%%%%%%%%%%%%%%%%%%%%%%%%%%%%%%%%%%%%%%%%%%%%%%%%%%%%%%%%%%%%%%%%%%%%%%%%%%
\subsection{Flags}
\label{sec:flags}

The package makes it easy to generate different versions
of the main or child documents.
To this end compilation flags can be defined
and assigned different default values.
They will be particularly useful in conjunction
with the forwarding mechanism described in \secref{sec:forward}.

For example, it may be useful to have a flag |\version|
which can be set to |draft| or |final|.
The document source will contain some conditional code
depending on the value of |\version|.
Suppose further, the flag should default to |final| for the main file
and to |draft| for child files
which is a natural assignment for editing the document.
This is achieved by placing the following code
in the preamble of the main document
(below the |\childdocmain| directive):
%
\begin{center}
\begin{tabular}{l}
|\ifchilddoc|\\
|\providecommand{\version}{draft}|\\
|\||else|\\
|\providecommand{\version}{final}|\\
|\||fi|
\end{tabular}
\end{center}
%
The definition by |\providecommand| makes sure
that previous definitions are not overwritten.
Further statements |\providecommand{\version}{...}|
can thus be added before the above code to override it.

For the main file, one might add a line
(between |\childdocmain| and the above block)
%
\begin{center}
|%\ifchilddoc\||else\providecommand{\version}{draft}\||fi|
\end{center}
%
which can be uncommented to produce a draft version.
Likewise one can add a line to the very top of a child file
(above the |\childdocof{|\textit{main}|}| directive)
%
\begin{center}
|%\providecommand{\version}{final}|
\end{center}
%
which can be uncommented to produce the final version of this child document.

%%%%%%%%%%%%%%%%%%%%%%%%%%%%%%%%%%%%%%%%%%%%%%%%%%%%%%%%%%%%%%%%%%%%%%%%%%%%%%%%
\subsection{Forwarding}
\label{sec:forward}

Different versions of the main or child documents
using compilation flags as described in \secref{sec:flags}
can be (permanently) stored in different files
for convenient compilation, viewing and distribution.
To this end, the package defines a command
to pass on compilation to a different file:

%%%%%%%%%%%%%%%%%%%%%%%%%%%%%%%%%%%%%%%%
\DescribeMacro{\childdocforward}
The command |\childdocforward| redirects processing to
another source file:
%
\begin{center}
\begin{tabular}{l}
|\input{childdoc.def}|\\
|\childdocforward[|\textit{main}|]{|\textit{dest}|}|\\
\end{tabular}
\end{center}
%
The argument \textit{dest} is the destination file
(without extension).
It should be the main file or one of the child files.
Note that further \textsf{childdoc} directives
such as |\childdocof| and |\childdocforward|
in the indicated file will be processed in this form.
The optional argument \textit{main}
passes on directly to the main file \textit{main}
while pretending to compile the child \textit{dest}.
This form behaves as if \textit{dest}
issues |\childdocof{|\textit{main}|}| right away,
and no further \textsf{childdoc} directives will be processed.

%%%%%%%%%%%%%%%%%%%%%%%%%%%%%%%%%%%%%%%%
\DescribeMacro{\...prefix}
In the alternative form |\childdocforwardprefix|,
%
\begin{center}
\begin{tabular}{l}
|\input{childdoc.def}|\\
|\childdocforwardprefix[|\textit{main}|]{|\textit{prefix}|}{|\textit{dest}|}|
\end{tabular}
\end{center}
%
the destination file is determined by a pattern
depending on the current file:
To make this work, the current file must be called
`{\textit{prefix}\hspace{0.2em}\textit{suffix}}'
with \textit{prefix} matching precisely the argument.
Processing is then passed on to the file
`{\textit{dest}\hspace{0.2em}\textit{suffix}}'.
Surely, the same effect is achieved by
directly specifying the
argument `{\textit{dest}\hspace{0.2em}\textit{suffix}}'
in the first form.
However, that requires to set up a different file
for each child. With the alternative form of the command
all these files can have exactly the same content
which simplifies setting them up and maintaining them.

For example, the following file |draft.tex|
with a compilation flag |\version| as described in \secref{sec:flags}
compiles the main document as a draft:
%
\begin{center}
\begin{tabular}{l}
|\def\version{draft}|\\
|\input{childdoc.def}|\\
|\childdocforward{|\textit{main}|}|
\end{tabular}
\end{center}
%
Likewise, the following files |final|\textit{nn}|.tex|
compile the final version of the child document
|child|\textit{nn}|.tex|:
%
\begin{center}
\begin{tabular}{l}
|\def\version{final}|\\
|\input{childdoc.def}|\\
|\childdocforwardprefix{final}{child}|
\end{tabular}
\end{center}
%

Note that when several versions of a main file and/or of each child file
are to be generated, it may be convenient to set up a |Makefile| or
shell script to automatise the process.

%%%%%%%%%%%%%%%%%%%%%%%%%%%%%%%%%%%%%%%%%%%%%%%%%%%%%%%%%%%%%%%%%%%%%%%%%%%%%%%%
\subsection{Command Line Processing}
\label{sec:commandline}

The effect of redirection files can also be achieved by invoking
the \LaTeX{} compiler with a more elaborate command line.
Most conveniently this should be done as part
of a shell script or a |Makefile|.

When using \textsf{childdoc} in the main file, the following
command lines effectively perform a redirection
(note that depending on the shell being used,
backslashes may have to be doubled: `|\|' $\to$ `|\\|'):
%
\begin{center}
|... -jobname "|\textit{target}|" |\\|"|[\textit{flags}]%
|\input{childdoc.def}\childdocforward[|\textit{main}|]{|\textit{dest}|}"|
\end{center}
%
Here \textit{target} is the name of the output file,
\textit{main} is the name of the main file
and \textit{dest} is the name of the main or child file to be processed
(all filenames without extensions).
The optional argument \textit{main} can be omitted
if \textit{main} matches \textit{dest}.
Optionally, compilation \textit{flags} can be defined via |\def| commands.
This command line makes the \TeX{} engine believe
it is compiling the file \textit{target}
whose content is specified as the latter parameter.
The provided code then forwards the processing to
\textit{main} or \textit{dest} as described in \secref{sec:forward}.

%%%%%%%%%%%%%%%%%%%%%%%%%%%%%%%%%%%%%%%%%%%%%%%%%%%%%%%%%%%%%%%%%%%%%%%%%%%%%%%%
\subsection{Include by Input}
\label{sec:input}

Including child documents by |\include| has some restrictions by design.
Most notably, the content of a child document always occupies
its own set of pages; pages cannot be shared between child documents.
Usually, this behaviour makes perfect sense
because each child document contain an essential part of the document.
However, in some situations it may be desirable to compose
a document from a collection of parts
without having mandatory page breaks between then.
For this case, the package
provides a mechanism to include parts
by |\input| which can also be processed individually.
However, by construction this mechanism
requires manual handling of the content to be output.

%%%%%%%%%%%%%%%%%%%%%%%%%%%%%%%%%%%%%%%%
\DescribeMacro{\ifchilddocmanual}
The main file should be prepared as usual, see \secref{sec:include}.
However, the document body must make a distinction
between processing of an individual part and of the main document, e.g.:
%
\begin{center}
\begin{tabular}{l}
|\ifchilddocmanual|\\
|\input{\childdocname}|\\
|\||else|\\
\textit{document body with }|\input{|\textit{part}|}|\\
|\||fi|
\end{tabular}
\end{center}
%
The conditional |\ifchilddocmanual| is true whenever
a part to be included by |\input| is being compiled,
and the name of the part is stored in |\childdocname|.

%%%%%%%%%%%%%%%%%%%%%%%%%%%%%%%%%%%%%%%%
\DescribeMacro{\childdocby}
Each part to be included by |\input| should start with:
%
\begin{center}
\begin{tabular}{l}
|\input{childdoc.def}|\\
|\childdocby{|\textit{main}|}|\\
\end{tabular}
\end{center}
%
The directive |\childdocby| is similar to |\childdocof|
described in \secref{sec:include},
but the subsequent selection of content must be done manually.
To that end, both |\ifchilddoc| and |\ifchilddocmanual|
will be true upon processing of a part,
and the name of the part is stored in |\childdocname|.
Note that |\jobname| will be set to the filename of the current part
so that each part receives an individual |.aux| file
that does not interfere with the |.aux| file(s) of the main document.
This behaviour can be altered by the alternative form
|\childdocby[*]{|\textit{main}|}| (with a non-empty optional argument)
which uses the |.aux| file of the main document
by setting |\jobname| to \textit{main}.

%%%%%%%%%%%%%%%%%%%%%%%%%%%%%%%%%%%%%%%%%%%%%%%%%%%%%%%%%%%%%%%%%%%%%%%%%%%%%%%%
\subsection{Driver Development}
\label{sec:driver}

The \textsf{childdoc} mechanism can also be use for the development
of definition files such as \LaTeX{} styles or classes.
This case differs from the above setup with multiple parts
included by |\include| in that no |\includeonly| should be invoked.
This can be achieved by starting the include file
(before |\ProvidesPackage|) with:
%
\begin{center}
\begin{tabular}{l}
|\input{childdoc.def}|\\
|\childdocforward{|\textit{main}|}|\\
\end{tabular}
\end{center}
%
or alternatively with:
%
\begin{center}
\begin{tabular}{l}
|\input{childdoc.def}|\\
|\childdocby{|\textit{main}|}|\\
\end{tabular}
\end{center}
%
Both forms have slightly different effects as described above.
The main file is prepared as usual, see \secref{sec:include}.

%%%%%%%%%%%%%%%%%%%%%%%%%%%%%%%%%%%%%%%%%%%%%%%%%%%%%%%%%%%%%%%%%%%%%%%%%%%%%%%%
\subsection{Legacy Detection}
\label{sec:detection}

The directive |\childdocmain| in the main file can detect
whether the complete document or merely a child is to be compiled
even without using the directive |\childdocof|.
This method is deprecated because it is less robust
and there is no compelling reason to use it;
it is merely provided for backward compatibility
and it may be removed in future versions.

If the detection mechanism is to be used,
it is mandatory to correctly specify
the filename of the main file as the argument of |\childdocmain|:
%
\begin{center}
\begin{tabular}{l}
|\input{childdoc.def}|\\
|\childdocmain{|\textit{main}|}|\\
\end{tabular}
\end{center}
%
If |\jobname| does not match the argument \textit{main} of |\childdocmain|,
it is assumed that |\jobname| points to the child file to be compiled.
When using |\childdocmain| with the main file specified as argument,
it suffices to start a child file
with just |\input{|\textit{main}|}|
without loading of the package and using |\childdocof|.
If instead all processing is done
with the appropriate \textsf{childdoc} directives,
the argument of \textit{main} of |\childdocmain| can be empty.

An alternative version of the command line processing described
in \secref{sec:commandline} using the detection mechanism reads:
%
\begin{center}
|... -jobname "|\textit{target}|" "|[\textit{flags}]%
[|\def\jobname{|\textit{dest}|}|]|\input{|\textit{main}|}"|
\end{center}

%%%%%%%%%%%%%%%%%%%%%%%%%%%%%%%%%%%%%%%%%%%%%%%%%%%%%%%%%%%%%%%%%%%%%%%%%%%%%%%%
\subsection{Manual Code}
\label{sec:manual}

In case one cannot be certain whether the definitions file |childdoc.def|
is installed on the target \TeX{} distribution
and one prefers not to ship it,
it is conceivable to paste a few relevant commands into the sources.

To that end, drop all statements |\input{childdoc.def}|
and perform the replacements as outlined below.
Instead of |\childdocmain{|\textit{main}|}| add the following code
to the top of the main file:
%
\begin{center}
\begin{tabular}{l}
|\||ifdefined\childdocname\endinput\||fi\newif\ifchilddoc|\\
|\edef\childdocname{\scantokens\expandafter{\jobname\noexpand}}|\\
|\def\childdocmain{|\textit{main}|}\||ifx\childdocmain\childdocname\||else|\\
|\childdoctrue\includeonly{\childdocname}\let\jobname\childdocmain\||fi|\\
\end{tabular}
\end{center}
%
Instead of |\childdocof{|\textit{main}|}| just include the main file
at the top of each child file:
%
\begin{center}
|\input{|\textit{main}|}|
\end{center}
%
A simple redirection |\childdocforward{|\textit{dest}|}| is achieved by:
%
\begin{center}
|\def\jobname{|\textit{dest}|}\input{\jobname}|
\end{center}
%
The redirection with prefix
|\childdocforwardprefix[|\textit{prefix}|]{|\textit{dest}|}|
is accomplished by:
%
\begin{center}
\begin{tabular}{l}
|{\edef\jobname{\scantokens\expandafter{\jobname\noexpand}}|\\
|\def\redirectjob |\textit{prefix}|#1~~~{\gdef\jobname{|\textit{dest}|#1}}|\\
|\expandafter\redirectjob\jobname~~~}\input{\jobname}|
\end{tabular}
\end{center}

In an alternative approach,
child documents can be compiled by a specific command line
without additional code or specific definitions:
%
\begin{center}
|... -jobname "|\textit{target}|" "|[\textit{flags}]%
|\includeonly{|\textit{dest}|}\input{|\textit{main}|}"|
\end{center}
%

%%%%%%%%%%%%%%%%%%%%%%%%%%%%%%%%%%%%%%%%%%%%%%%%%%%%%%%%%%%%%%%%%%%%%%%%%%%%%%%%
%%%%%%%%%%%%%%%%%%%%%%%%%%%%%%%%%%%%%%%%%%%%%%%%%%%%%%%%%%%%%%%%%%%%%%%%%%%%%%%%
\section{Information}

%%%%%%%%%%%%%%%%%%%%%%%%%%%%%%%%%%%%%%%%%%%%%%%%%%%%%%%%%%%%%%%%%%%%%%%%%%%%%%%%
\subsection{Copyright}

Copyright \copyright{} 2017--2018 Niklas Beisert

This work may be distributed and/or modified under the
conditions of the \LaTeX{} Project Public License, either version 1.3
of this license or (at your option) any later version.
The latest version of this license is in
  \url{http://www.latex-project.org/lppl.txt}
and version 1.3 or later is part of all distributions of \LaTeX{}
version 2005/12/01 or later.

This work has the LPPL maintenance status `maintained'.

The Current Maintainer of this work is Niklas Beisert.

This work consists of the files |README.txt|, |childdoc.ins| and |childdoc.dtx|
as well as the derived files |childdoc.def|, |cdocsamp.tex|
with |cdocsch1.tex|, |cdocsch2.tex|, |cdocspt3.tex|, |cdocspt4.tex|,
|cdocsdrf.tex|, |cdocsfn1.tex|, |cdocsfn2.tex|
as well as |childdoc.pdf|.

%%%%%%%%%%%%%%%%%%%%%%%%%%%%%%%%%%%%%%%%%%%%%%%%%%%%%%%%%%%%%%%%%%%%%%%%%%%%%%%%
\subsection{Files and Installation}

The package consists of the files:
%
\begin{center}
\begin{tabular}{ll}
    |README.txt|   & readme file \\
    |childdoc.ins| & installation file \\
    |childdoc.dtx| & source file \\
    |childdoc.def| & definition file \\
    |cdocsamp.tex| & sample main file \\
    |cdocsch1.tex| & sample include file \\
    |cdocsch2.tex| & sample include file \\
    |cdocspt3.tex| & sample part file \\
    |cdocspt4.tex| & sample part file \\
    |cdocsdrf.tex| & sample redirection file \\
    |cdocsfn1.tex| & sample redirection file \\
    |cdocsfn2.tex| & sample redirection file \\
    |childdoc.pdf| & manual
\end{tabular}
\end{center}
%
The distribution consists of the files
|README.txt|, |childdoc.ins| and |childdoc.dtx|.
%
\begin{itemize}
\item
Run (pdf)\LaTeX{} on |childdoc.dtx|
to compile the manual |childdoc.pdf| (this file).
\item
Run \LaTeX{} on |childdoc.ins| to create the definitions file |childdoc.def|
and the sample |cdocsamp.tex| with include files
|cdocsch1.tex|, |cdocsch2.tex|, |cdocspt3.tex|, |cdocspt4.tex|,
|cdocsdrf.tex|, |cdocsfn1.tex|, |cdocsfn2.tex|.
Then copy the file |childdoc.def| to an appropriate directory of your \LaTeX{}
distribution, e.g.\ \textit{texmf-root}|/tex/latex/childdoc|.
\end{itemize}

%%%%%%%%%%%%%%%%%%%%%%%%%%%%%%%%%%%%%%%%%%%%%%%%%%%%%%%%%%%%%%%%%%%%%%%%%%%%%%%%
\subsection{Related CTAN Packages}

There are several other packages which offer a similar functionality:
%
\begin{itemize}
\item
The packages
\href{http://ctan.org/pkg/docmute}{\textsf{docmute}},
\href{http://ctan.org/pkg/includex}{\textsf{includex}} and
\href{http://ctan.org/pkg/standalone}{\textsf{standalone}}
provide commands to include only the document body of
a child file thus allowing both files to be compiled individually.
\item
The packages \href{http://ctan.org/pkg/subdocs}{\textsf{subdocs}}
and \href{http://ctan.org/pkg/subfiles}{\textsf{subfiles}}
provide structures in which the main and child documents can be
encapsulated and allowing them to be compiled individually.
The inclusion mechanism is different from the conventional |\include|.
\item
The package \href{http://ctan.org/pkg/combine}{\textsf{combine}}
is an elaborate solution to combine several documents into one.
\end{itemize}
%
See also the CTAN topic \href{http://ctan.org/topic/subdocs}{\textsf{subdocs}}
for further related packages.
The present package differs from the above solutions in that
a document structure constructed with the conventional |\include| mechanism
just needs two extra commands at the top of every file
such that all constituent files can be compiled individually.

%%%%%%%%%%%%%%%%%%%%%%%%%%%%%%%%%%%%%%%%%%%%%%%%%%%%%%%%%%%%%%%%%%%%%%%%%%%%%%%%
%\subsection{Feature Suggestions}
%
%The following is a list of features which may be useful for future
%versions of this package:
%%
%\begin{itemize}
%\item
%\ldots
%\end{itemize}

%%%%%%%%%%%%%%%%%%%%%%%%%%%%%%%%%%%%%%%%%%%%%%%%%%%%%%%%%%%%%%%%%%%%%%%%%%%%%%%%
\subsection{Revision History}

%%%%%%%%%%%%%%%%%%%%%%%%%%%%%%%%%%%%%%%%
\paragraph{v2.0:} 2018/12/30

\begin{itemize}
\item
immediate forward processing
\item
added |\childdocby| mechanism
\item
manual restructured
\end{itemize}

%%%%%%%%%%%%%%%%%%%%%%%%%%%%%%%%%%%%%%%%
\paragraph{v1.6:} 2018/01/17

\begin{itemize}
\item
application for development of include files
\item
corrections to manual
\end{itemize}

%%%%%%%%%%%%%%%%%%%%%%%%%%%%%%%%%%%%%%%%
\paragraph{v1.5:} 2017/05/21

\begin{itemize}
\item
more complete structuring introduced
\item
|\childdocof| introduced
\item
|\childdoc| renamed to |\childdocmain|
\item
|\childredirect| renamed to |\childdocforward| and |\childdocforwardprefix|
and functionality expanded
\end{itemize}

%%%%%%%%%%%%%%%%%%%%%%%%%%%%%%%%%%%%%%%%
\paragraph{v1.0:} 2017/04/27

\begin{itemize}
\item
manual and install package
\item
first version published on CTAN
\end{itemize}

%%%%%%%%%%%%%%%%%%%%%%%%%%%%%%%%%%%%%%%%
\paragraph{v0.6:} 2017/04/26

\begin{itemize}
\item
redirection mechanism added
\end{itemize}

%%%%%%%%%%%%%%%%%%%%%%%%%%%%%%%%%%%%%%%%
\paragraph{v0.5:} 2017/04/26

\begin{itemize}
\item
functionality in definition file
\end{itemize}


%%%%%%%%%%%%%%%%%%%%%%%%%%%%%%%%%%%%%%%%%%%%%%%%%%%%%%%%%%%%%%%%%%%%%%%%%%%%%%%%
%%%%%%%%%%%%%%%%%%%%%%%%%%%%%%%%%%%%%%%%%%%%%%%%%%%%%%%%%%%%%%%%%%%%%%%%%%%%%%%%
%%%%%%%%%%%%%%%%%%%%%%%%%%%%%%%%%%%%%%%%%%%%%%%%%%%%%%%%%%%%%%%%%%%%%%%%%%%%%%%%
\appendix

\settowidth\MacroIndent{\rmfamily\scriptsize 000\ }

 \DocInput{childdoc.dtx}

\end{document}
%</driver>
% \fi
%
% %%%%%%%%%%%%%%%%%%%%%%%%%%%%%%%%%%%%%%%%%%%%%%%%%%%%%%%%%%%%%%%%%%%%%%%%%%%%%%
% %%%%%%%%%%%%%%%%%%%%%%%%%%%%%%%%%%%%%%%%%%%%%%%%%%%%%%%%%%%%%%%%%%%%%%%%%%%%%%
% \section{Sample}
%\iffalse
%<*samplemain>
%\fi
%
% The following presents a sample document
% with two chapters, two parts, a title page,
% a compile flag as well as three forwarding files to set the flag.
% It consists of eight |.tex| files:
% \begin{center}
% \begin{tabular}{ll}
% |cdocsamp.tex|&main file\\
% |cdocsch1.tex|&include file for chapter 1\\
% |cdocsch2.tex|&include file for chapter 2\\
% |cdocspt3.tex|&include file for part 3\\
% |cdocspt4.tex|&include file for part 4\\
% |cdocsdrf.tex|&forwarding file for main file in draft mode\\
% |cdocsfi1.tex|&forwarding file for final version of chapter 1\\
% |cdocsfi2.tex|&forwarding file for final version of chapter 2\\
% \end{tabular}
% \end{center}
% Each of the eight files can be compiled directly by the \LaTeX{} compiler.
%
% %%%%%%%%%%%%%%%%%%%%%%%%%%%%%%%%%%%%%%
% \paragraph{Main File.}
%
% The main file is called |cdocsamp.tex|.
%
% Load the \textsf{childdoc} definitions and
% declare the filename for the main document:
%    \begin{macrocode}
\input{childdoc.def}
\childdocmain{}
%    \end{macrocode}

% Optional override for |\version| flag:
%    \begin{macrocode}
%%\ifchilddoc\else\providecommand{\version}{draft}\fi
%    \end{macrocode}

% Define the default values for the |\version| flag
% (|final| for the main file and |draft| for childs):
%    \begin{macrocode}
\ifchilddoc
\providecommand{\version}{draft}
\else
\providecommand{\version}{final}
\fi
%    \end{macrocode}

% Load the standard document class:
%    \begin{macrocode}
\documentclass[12pt]{article}
%    \end{macrocode}

% Start the document body:
%    \begin{macrocode}
\begin{document}
%    \end{macrocode}

% Declare a title page.
% Print title, part of document being processed and version flag:
%    \begin{macrocode}
\addtocounter{page}{-1}
\begin{center}
{\LARGE\bfseries{}childdoc example\par}
\vspace{1cm}
\ifchilddoc
\ifchilddocmanual part\else chapter\fi:
`\childdocname' of `\childdocjob'\par
\else
main document: `\childdocjob'\par
\fi
version: \version\par
\end{center}
\newpage
%    \end{macrocode}

% Manually include selected file,
% otherwise process as usual:
%    \begin{macrocode}
\ifchilddocmanual
\section*{part `\childdocname'}
\input{\childdocname}
\else
%    \end{macrocode}

% Include the two chapters:
%    \begin{macrocode}
\include{cdocsch1}
\include{cdocsch2}
%    \end{macrocode}

% Include the two parts unless only chapters should be displayed:
%    \begin{macrocode}
\ifchilddoc\else
\section{part three}
\input{cdocspt3}
\section{part four}
\input{cdocspt4}
\fi
%    \end{macrocode}

% Process as usual until here:
%    \begin{macrocode}
\fi
%    \end{macrocode}

% End of document body:
%    \begin{macrocode}
\end{document}
%    \end{macrocode}
%\iffalse
%</samplemain>
%\fi
%
% %%%%%%%%%%%%%%%%%%%%%%%%%%%%%%%%%%%%%%
% \paragraph{Chapter Include Files.}
%
% The include files are called |cdocsch1.tex| and |cdocsch2.tex|.
%
%\iffalse
%<*samplechap1|samplechap2>
%\fi

% Optional override for |\version| flag:
%    \begin{macrocode}
%%\providecommand{\version}{final}
%    \end{macrocode}

% Include the main document:
%    \begin{macrocode}
\input{childdoc.def}
\childdocof{cdocsamp}
%    \end{macrocode}

%\iffalse
%</samplechap1|samplechap2>
%\fi
%
%\iffalse
%<*samplechap1>
%\fi
% Some text for chapter 1:
%    \begin{macrocode}
\section{one}
some text in chapter one
%    \end{macrocode}

%\iffalse
%</samplechap1>
%\fi
% Some text for chapter 2:
%\iffalse
%<*samplechap2>
%\fi
%    \begin{macrocode}
\section{two}
more text in chapter two
%    \end{macrocode}

%\iffalse
%</samplechap2>
%\fi
%
% %%%%%%%%%%%%%%%%%%%%%%%%%%%%%%%%%%%%%%
% \paragraph{Part Include Files.}
%
% The include files are called |cdocspt3.tex| and |cdocspt4.tex|.
%
%\iffalse
%<*samplepart3|samplepart4>
%\fi

% Optional override for |\version| flag:
%    \begin{macrocode}
%%\providecommand{\version}{final}
%    \end{macrocode}

% Include the main document:
%    \begin{macrocode}
\input{childdoc.def}
\childdocby{cdocsamp}
%    \end{macrocode}

%\iffalse
%</samplepart3|samplepart4>
%\fi
%
%\iffalse
%<*samplepart3>
%\fi
% Some text for part 3:
%    \begin{macrocode}
some text in part three
%    \end{macrocode}

%\iffalse
%</samplepart3>
%\fi
% Some text for part 4:
%\iffalse
%<*samplepart4>
%\fi
%    \begin{macrocode}
more text in part four
%    \end{macrocode}

%\iffalse
%</samplepart4>
%\fi
%
% %%%%%%%%%%%%%%%%%%%%%%%%%%%%%%%%%%%%%%
% \paragraph{Forwarding for a Complete Draft.}
%
% The following forwarding file |cdocsdrf.tex|
% compiles the main document in draft mode:
%\iffalse
%<*sampledraft>
%\fi
%    \begin{macrocode}
\def\version{draft}
\input{childdoc.def}
\childdocforward{cdocsamp}
%    \end{macrocode}

%\iffalse
%</sampledraft>
%\fi
%
% %%%%%%%%%%%%%%%%%%%%%%%%%%%%%%%%%%%%%%
% \paragraph{Forwarding for Final Version of the Chapters.}
%
% The following forwarding files |cdocsfn1.tex| and |cdocsfn2.tex|
% (with identical content)
% compile the final versions of the child documents
% |cdocsch1.tex| and |cdocsch2.tex|, respectively:
%\iffalse
%<*samplefinal>
%\fi
%    \begin{macrocode}
\def\version{final}
\input{childdoc.def}
\childdocforwardprefix[cdocsamp]{cdocsfn}{cdocsch}
%    \end{macrocode}

%\iffalse
%</samplefinal>
%\fi
%
% %%%%%%%%%%%%%%%%%%%%%%%%%%%%%%%%%%%%%%
% \paragraph{Command Line Processing.}
%
% The following three command lines generate the output files
% |cdocscld|, |cdocscl1| and |cdocscl2|
% which should be identical to
% |cdocsdrf|, |cdocsch1| and |cdocsfn2|, respectively:
% \begin{center}
% \begin{tabular}{l}
% |latex -jobname cdocscld \|\\
% |  "\def\version{draft}\input{childdoc.def}\childdocforward{cdocsamp}"|\\
% |latex -jobname cdocscl1 \|\\
% |  "\input{childdoc.def}\childdocforward[cdocsamp]{cdocsch1}"|\\
% |latex -jobname cdocscl2 \|\\
% |  "\def\version{final}\input{childdoc.def}\childdocforward{cdocsch2}"|
% \end{tabular}
% \end{center}
% Note that the trailing backslash on each first line
% merely continues the input to the second line
% (for convenient cut ant paste).
% Furthermore, the command |latex| can be replaced by any
% of its alternative versions such as |pdflatex|.
%
% %%%%%%%%%%%%%%%%%%%%%%%%%%%%%%%%%%%%%%%%%%%%%%%%%%%%%%%%%%%%%%%%%%%%%%%%%%%%%%
% %%%%%%%%%%%%%%%%%%%%%%%%%%%%%%%%%%%%%%%%%%%%%%%%%%%%%%%%%%%%%%%%%%%%%%%%%%%%%%
% \section{Implementation}
%\iffalse
%<*package>
%\fi
%
% This section describes the definitions file |childdoc.def|.

% The definitions cannot be loaded using |\usepackage| or |\RequirePackage|
% which has a mechanism to prevent loading a style file more than once.
% When loading the definitions by means of |\input|
% multiple instances have to be prevented manually:
%\iffalse
%This code needs to be before the `\ProvidesFile' directive
%which is defined at the beginning of this file.
%Therefore it is also placed there and commented out here.
%</package>
%<*discard>
%\fi
%    \begin{macrocode}
\ifdefined\childdocmain\endinput\fi
%    \end{macrocode}
%\iffalse
%</discard>
%<*package>
%\fi
%
% \macro{\ifchilddoc}
% \macro{\ifchilddocmanual}
% The conditional |\ifchilddoc| tells whether a
% child (true) or main (false) document is being compiled.
% The conditional |\ifchilddocmanual| tells whether
% the |\includeonly| mechanism is used (false) or
% the selection of child files must be performed manually (true).
% The definitions initialise to false:
%    \begin{macrocode}
\newif\ifchilddoc
\newif\ifchilddocmanual
%    \end{macrocode}

% \macro{\childdocname}
% \macro{\childdocjob}
% The macro |\childdocname| stores the name of the main document
% to be compiled. The macro |\childdocjob| stores the name of
% the document on which the \LaTeX{} compiler was originally invoked.
% The content of |\jobname| cannot be compared
% to filenames specified in the source due to different catcodes.
% The following code rescans |\jobname|, stores the result
% in |\childdocname| and saves a copy in |\childdocjob|:
%    \begin{macrocode}
\edef\childdocname{\scantokens\expandafter{\jobname\noexpand}}
\let\childdocjob\childdocname
%    \end{macrocode}

% \macro{\childdocdisable}
% The macro |\childdocdisable| prevents the main file
% from being processed more than once.
% At this stage, the main document command |\childdocmain|
% is assumed to be called once again where it should do nothing.
% Any subsequent call to it should prevent
% a secondary processing of the main document
% It overwrites the forwarding commands
% |\childdocof| and |\childdocforward|
% with empty macros to prevent further inclusions of the main document:
%    \begin{macrocode}
\newcommand{\childdocdisable}
{
  \renewcommand{\childdocmain}[1]{\renewcommand{\childdocmain}[1]{\endinput}}
  \renewcommand{\childdocof}[1]{}
  \renewcommand{\childdocby}[2][]{}
  \renewcommand{\childdocforward}[2][]{}
  \renewcommand{\childdocdisable}{}
}
%    \end{macrocode}

% \macro{\childdocmain}
% The macro |\childdocmain| is to be called at the top of the main file
% with nothing or the main filename (without extension) as argument.
% First, it breaks loops.
% If the argument is not empty and does not match |\childdocname|
% (which is set by the first inclusion of |childdoc.def|),
% |\ifchilddoc| is set to true, |\includeonly| is applied to the child file
% and |\jobname| is set to the main file
% (for proper handling of |.aux| files):
%    \begin{macrocode}
\newcommand{\childdocmain}[1]
{
  \childdocdisable\childdocmain{}
  \if?#1?\else
    \begingroup
      \def\childdoctmp{#1}
      \ifx\childdoctmp\childdocname
        \def\childdoctmp{}
      \else
        \def\childdoctmp
        {
          \childdoctrue
          \includeonly{\childdocname}
          \def\childdocjob{#1}
          \def\jobname{#1}
        }
      \fi
      \expandafter
    \endgroup
    \childdoctmp
  \fi
}
%    \end{macrocode}

% \macro{\childdocof}
% The command |\childdocof| redirects
% compilation to the main file |#1|.
%    \begin{macrocode}
\newcommand{\childdocof}[1]
{
  \childdocdisable
  \childdoctrue
  \includeonly{\childdocname}
  \def\jobname{#1}
  \def\childdocjob{#1}
  \input{#1}
}
%    \end{macrocode}

% \macro{\childdocby}
% The command |\childdocby| ....
%    \begin{macrocode}
\newcommand{\childdocby}[2][]
{
  \childdocdisable
  \childdoctrue
  \childdocmanualtrue
  \if?#1?\else
    \def\jobname{#2}
  \fi
  \def\childdocjob{#2}
  \input{#2}
  \endinput
}
%    \end{macrocode}

% \macro{\childdocforward}
% The command |\childdocforward| redirects
% compilation to the main file or
% (if the optional argument is given) a child file.
% Parameters are set as if the main file
% or a child file starting with |\childdocof| was compiled.
% Then compilation is handed over to the main file:
%    \begin{macrocode}
\newcommand{\childdocforward}[2][]
{
  \begingroup
    \if?#1?
      \def\childdoctmp
      {
        \def\childdocname{#2}
        \def\childdocjob{#2}
        \def\jobname{#2}
        \input{#2}
        \endinput
      }
    \else
      \def\childdoctmp
      {
        \childdocdisable
        \def\childdocname{#2}
        \childdoctrue
        \includeonly{#2}
        \def\childdocjob{#1}
        \def\jobname{#1}
        \input{#1}
        \endinput
      }
    \fi
    \expandafter
  \endgroup
  \childdoctmp
}
%    \end{macrocode}

% \macro{\childdocforwardprefix}
% The command |\childdocforwardprefix| redirects
% compilation to the main or a child file by means of a pattern.
% The prefix |#1| in the current filename is replaced by |#2|
% and the suffix of the current filename is kept
% (it is assumed that the filename does not contain the substring `|~~~|'
% which is used as a delimiter).
% Compilation is handed over to the new file by |\childdocforward|:
%    \begin{macrocode}
\newcommand{\childdocforwardprefix}[3][]
{
  \begingroup
    \def\childdocextract #2##1~~~{\def\childdoctmp{\childdocforward[#1]{#3##1}}}
    \expandafter\childdocextract\childdocname~~~
    \expandafter
  \endgroup
  \childdoctmp
}
%    \end{macrocode}

% \macro{\childdoc}
% The deprecated macro |\childdoc| is a legacy version of |\childdocmain|:
%    \begin{macrocode}
\newcommand{\childdoc}{\childdocmain}
%    \end{macrocode}

% \macro{\childdocredirect}
% The deprecated macro |\childdocredirect| is a legacy version
% of |\childdocforward| and |\childdocforwardprefix|:
%    \begin{macrocode}
\newcommand{\childdocredirect}[2][]
{
  \begingroup
    \if?#1?
      \def\childdoctmp{\childdocforward{#2}}
    \else
      \def\childdoctmp{\childdocforwardprefix{#1}{#2}}
    \fi
    \expandafter
  \endgroup
  \childdoctmp
}
%    \end{macrocode}

%\iffalse
%</package>
%\fi
%
\endinput
\childdocforward{cdocsamp}"|\\
% |latex -jobname cdocscl1 \|\\
% |  "% \iffalse
%
% childdoc.dtx Copyright (C) 2017-2018 Niklas Beisert
%
% This work may be distributed and/or modified under the
% conditions of the LaTeX Project Public License, either version 1.3
% of this license or (at your option) any later version.
% The latest version of this license is in
%   http://www.latex-project.org/lppl.txt
% and version 1.3 or later is part of all distributions of LaTeX
% version 2005/12/01 or later.
%
% This work has the LPPL maintenance status `maintained'.
%
% The Current Maintainer of this work is Niklas Beisert.
%
% This work consists of the files childdoc.dtx and childdoc.ins
% and the derived files childdoc.def and cdocsamp.tex with
% cdocsch1.tex, cdocsch2.tex, cdocsdrf.tex, cdocsfn1.tex, cdocsfn2.tex.
%
%<package>\ifdefined\childdocmain\endinput\fi
%<package>\ProvidesFile{childdoc.def}[2018/12/30 v2.0 child document driver]
%<samplemain>\ProvidesFile{cdocsamp.tex}[2018/12/30 v2.0 sample for childdoc]
%<*driver>
%\ProvidesFile{childdoc.drv}[2018/12/30 v2.0 childdoc reference manual file]
\PassOptionsToClass{10pt,a4paper}{article}
\documentclass{ltxdoc}

\usepackage[margin=35mm]{geometry}
\usepackage{hyperref}
\usepackage{hyperxmp}
\usepackage[usenames]{color}

\hypersetup{colorlinks=true}
\hypersetup{pdfstartview=FitH}
\hypersetup{pdfpagemode=UseNone}
\hypersetup{pdfsource={}}
\hypersetup{pdflang={en-UK}}
\hypersetup{pdfcopyright={Copyright 2017-2018 Niklas Beisert.
  This work may be distributed and/or modified under the
  conditions of the LaTeX Project Public License, either version 1.3
  of this license or (at your option) any later version.}}
\hypersetup{pdflicenseurl={http://www.latex-project.org/lppl.txt}}
\hypersetup{pdfcontactaddress={ETH Zurich, ITP, HIT K,
  Wolfgang-Pauli-Strasse 27}}
\hypersetup{pdfcontactpostcode={8093}}
\hypersetup{pdfcontactcity={Zurich}}
\hypersetup{pdfcontactcountry={Switzerland}}
\hypersetup{pdfcontactemail={nbeisert@itp.phys.ethz.ch}}
\hypersetup{pdfcontacturl={http://people.phys.ethz.ch/\xmptilde nbeisert/}}

\newcommand{\secref}[1]{\hyperref[#1]{section \ref*{#1}}}

\parskip1ex
\parindent0pt
\let\olditemize\itemize
\def\itemize{\olditemize\parskip0pt}

\begin{document}

\title{The \textsf{childdoc} Package}
\hypersetup{pdftitle={The childdoc Package}}
\author{Niklas Beisert\\[2ex]
  Institut f\"ur Theoretische Physik\\
  Eidgen\"ossische Technische Hochschule Z\"urich\\
  Wolfgang-Pauli-Strasse 27, 8093 Z\"urich, Switzerland\\[1ex]
  \href{mailto:nbeisert@itp.phys.ethz.ch}
  {\texttt{nbeisert@itp.phys.ethz.ch}}}
\hypersetup{pdfauthor={Niklas Beisert}}
\hypersetup{pdfsubject={Manual for the LaTeX2e Package childdoc}}
\date{30 December 2018, \textsf{v2.0}}
\maketitle

\begin{abstract}\noindent
\textsf{childdoc} is a \LaTeXe{} package
that enables the direct compilation
of document sections included by |\include|
to individual files.
\end{abstract}

\begingroup
\parskip0ex
\tableofcontents
\endgroup

%%%%%%%%%%%%%%%%%%%%%%%%%%%%%%%%%%%%%%%%%%%%%%%%%%%%%%%%%%%%%%%%%%%%%%%%%%%%%%%%
%%%%%%%%%%%%%%%%%%%%%%%%%%%%%%%%%%%%%%%%%%%%%%%%%%%%%%%%%%%%%%%%%%%%%%%%%%%%%%%%
\section{Introduction}

\LaTeX{} provides a mechanism to structure a large document (such as a book)
into a main file and several child files (containing the chapters)
using the |\include| command.
This mechanism is beneficial for documents
which span hundreds of pages in order to
make the source file(s) more manageable.
Moreover, compilation can be restricted to
selected child files by means of the |\includeonly| command.
The latter feature can be used to reduce the compilation time while editing
(this was significantly more useful in the earlier days of \LaTeX{})
or to generate a smaller document which is easier to navigate.
Another application of |\includeonly| is to generate
documents consisting of selected parts of the complete document.

However, there are a few drawbacks of the plain |\include| mechanism:
\begin{itemize}
\item
The child files cannot be compiled on their own,
they can only be compiled via the main file.
A naive editing environment
(such as a text editor with an option
to have the current file processed by \LaTeX)
may require one to switch to the main file before compiling;
attempting to compile the child file produces errors.
\item
The main file must be modified (each time)
to adjust the |\includeonly| command
to the present needs. This easily leaves the main file in a messy state.
\item
The generated document will always carry the filename
of the main document. This is inconvenient if
several child files are to be compiled and
to be kept for distribution.
\end{itemize}

The present package provides a simple interface
to make child files individually compilable by \LaTeX{}.
Compiling a child file then has the same effect as compiling
the main file with an |\includeonly| command
to select the appropriate child.
Moreover the generated document will carry the name of the child
rather than the main file.
This resolves all three above issues.

This feature is meant to make the editing of books,
thesis documents and lecture notes somewhat more convenient.
However, the package can also be used efficiently for
composing a series of documents (such as exercise sheets)
which are typically distributed individually.
It then assists the author in generating the individual documents
(potentially in different versions)
as well as a document containing the collected series.
Another application is in developing style files
or other kinds of included material
where compilation of the style file could redirect
to a sample or test file.

%%%%%%%%%%%%%%%%%%%%%%%%%%%%%%%%%%%%%%%%%%%%%%%%%%%%%%%%%%%%%%%%%%%%%%%%%%%%%%%%
%%%%%%%%%%%%%%%%%%%%%%%%%%%%%%%%%%%%%%%%%%%%%%%%%%%%%%%%%%%%%%%%%%%%%%%%%%%%%%%%
\section{Usage}

First of all, the package \textsf{childdoc} is \emph{not} a standard
\LaTeXe{} |.sty| style file! Therefore it needs to be invoked in
a non-standard way.

%%%%%%%%%%%%%%%%%%%%%%%%%%%%%%%%%%%%%%%%%%%%%%%%%%%%%%%%%%%%%%%%%%%%%%%%%%%%%%%%
\subsection{Included Files}
\label{sec:include}

%%%%%%%%%%%%%%%%%%%%%%%%%%%%%%%%%%%%%%%%
\DescribeMacro{\childdocmain}
To use the package, add the commands
\begin{center}
\begin{tabular}{l}
|\input{childdoc.def}|\\
|\childdocmain{}|\\
\end{tabular}
\end{center}
at the very top of the main \LaTeX{} file,
in particular \emph{before} the |\documentclass| statement!
The argument of |\childdocmain| should be left empty
(but it must be present).

%%%%%%%%%%%%%%%%%%%%%%%%%%%%%%%%%%%%%%%%
\DescribeMacro{\childdocof}
Furthermore, add the commands
\begin{center}
\begin{tabular}{l}
|\input{childdoc.def}|\\
|\childdocof{|\textit{main}|}|\\
\end{tabular}
\end{center}
at the top of every child file \textit{child}
which is included by |\include{|\textit{child}|}|
from within the main file
(or at least for those files to be compiled individually).
The argument \textit{main} must be the filename of the main file.

There are a couple of
considerations in setting up the main and child documents:

%%%%%%%%%%%%%%%%%%%%%%%%%%%%%%%%%%%%%%%%
\paragraph{Restrictions.}

Please note the following restrictions:
\begin{itemize}
\item
|\childdocmain| must be called with one argument \textit{main}
to ensure compatibility with earlier version of the package.
It must either be empty (|\childdocmain{}|)
or precisely match the filename of the main file in which it is specified.
See \secref{sec:detection} for further information.
\item
The filename \textit{main} must be specified without the |.tex| extension.
\item
The filename \textit{main} is case sensitive
(even in case-insensitive file systems)
due to internal string comparison.
\item
The argument \textit{main} should be fully expanded, it cannot be a macro.
\item
Subdirectories and special characters should be avoided in filenames.
\item
The command |\childdocmain{|\textit{main}|}| must be followed by a whitespace.
It should not be followed immediately by another command
or by a comment mark `|%|'.
This is because the \TeX{} parser reads the token immediately following
the argument of |\childdocmain| and puts it
at the beginning of every child section;
however, a white\-space is ignored.
\end{itemize}

%%%%%%%%%%%%%%%%%%%%%%%%%%%%%%%%%%%%%%%%
\paragraph{Content of Main File.}

It is advisable to place all content in the child files included by |\include|.
Any output contained in the main file will appear in all child documents
unless suppressed manually;
it cannot be suppressed automatically by the |\includeonly| directive
and thus should normally be avoided.
A method to include some content in the main file
by means of conditional processing is described in \secref{sec:conditional}.

%%%%%%%%%%%%%%%%%%%%%%%%%%%%%%%%%%%%%%%%
\paragraph{Page Numbering.}

When only a part of the document is compiled,
the appropriate numbering of pages
(as well as other status parameters)
is determined from the |.aux| files.
The latter contain information from previous passes.
However this information needs to propagate through
all intermediate child documents.
Therefore the page numbering in child documents may well
be inconsistent until the complete document is compiled at least once.

A useful (if unconventional) way to always ensure a consistent
page numbering is to restart the numbering in each child document
and denote the pages by `\textit{child}|.|\textit{page}'
where \textit{child} represents the chapter/section number of the child file.
This can be achieved by the command
|\numberwithin{page}{|\textit{child}|}|
of the \textsf{amsmath} package
where \textit{child} can be |chapter| or |section|
depending on the chosen structuring.
Alternatively, one can modify the macro |\thepage| appropriately
and reset the counter |page| at the start of each child file.

%%%%%%%%%%%%%%%%%%%%%%%%%%%%%%%%%%%%%%%%%%%%%%%%%%%%%%%%%%%%%%%%%%%%%%%%%%%%%%%%
\subsection{Conditional Processing}
\label{sec:conditional}

The package provides a mechanism to compile different versions
of a document. To customise the versions further some conditional processing
can come in handy to distinguish which version is being compiled.
The package provides two macros to describe the compilation context:

%%%%%%%%%%%%%%%%%%%%%%%%%%%%%%%%%%%%%%%%
\DescribeMacro{\ifchilddoc}
The conditional |\ifchilddoc| distinguishes between the compilation of
child documents and the main document:
%
\begin{center}
|\ifchilddoc |\textit{child-code}| |[|\||else |\textit{main-code}]| \||fi|
\end{center}

%%%%%%%%%%%%%%%%%%%%%%%%%%%%%%%%%%%%%%%%
\DescribeMacro{\childdocname}
\DescribeMacro{\childdocjob}
The macro |\childdocname| contains the filename (without extension)
of the main or child file being processed.
Note that |\childdocjob| will always contain the name of the main file.

%%%%%%%%%%%%%%%%%%%%%%%%%%%%%%%%%%%%%%%%
\paragraph{Title Page.}

Conditional processing can be used to include a title or banner page
in the main document when proper precautions are taken.
Importantly, the code in the main file should ensure that the page counter
(as well as other status parameters which are stored in the |.aux| files)
takes the same value after the conditional processing.
Otherwise the page numbers may take divergent values
depending on which part is compiled.

For example, a title page could be declared by:
%
\begin{center}
\begin{tabular}{l}
|\ifchilddoc\||else|\\
|\addtocounter{page}{-1}|\\
\textit{code for title page}\\
|\newpage|\\
|\||fi|
\end{tabular}
\end{center}
%
A banner page for the child documents can be generated by:
%
\begin{center}
\begin{tabular}{l}
|\ifchilddoc|\\
|\addtocounter{page}{-1}|\\
\textit{code for banner page}\\
|\newpage|\\
|\||fi|
\end{tabular}
\end{center}
%
Here one could write a message such as:
\begin{center}
|This is the part \childdocname{} of \childdocjob{}.|
\end{center}

%%%%%%%%%%%%%%%%%%%%%%%%%%%%%%%%%%%%%%%%%%%%%%%%%%%%%%%%%%%%%%%%%%%%%%%%%%%%%%%%
\subsection{Flags}
\label{sec:flags}

The package makes it easy to generate different versions
of the main or child documents.
To this end compilation flags can be defined
and assigned different default values.
They will be particularly useful in conjunction
with the forwarding mechanism described in \secref{sec:forward}.

For example, it may be useful to have a flag |\version|
which can be set to |draft| or |final|.
The document source will contain some conditional code
depending on the value of |\version|.
Suppose further, the flag should default to |final| for the main file
and to |draft| for child files
which is a natural assignment for editing the document.
This is achieved by placing the following code
in the preamble of the main document
(below the |\childdocmain| directive):
%
\begin{center}
\begin{tabular}{l}
|\ifchilddoc|\\
|\providecommand{\version}{draft}|\\
|\||else|\\
|\providecommand{\version}{final}|\\
|\||fi|
\end{tabular}
\end{center}
%
The definition by |\providecommand| makes sure
that previous definitions are not overwritten.
Further statements |\providecommand{\version}{...}|
can thus be added before the above code to override it.

For the main file, one might add a line
(between |\childdocmain| and the above block)
%
\begin{center}
|%\ifchilddoc\||else\providecommand{\version}{draft}\||fi|
\end{center}
%
which can be uncommented to produce a draft version.
Likewise one can add a line to the very top of a child file
(above the |\childdocof{|\textit{main}|}| directive)
%
\begin{center}
|%\providecommand{\version}{final}|
\end{center}
%
which can be uncommented to produce the final version of this child document.

%%%%%%%%%%%%%%%%%%%%%%%%%%%%%%%%%%%%%%%%%%%%%%%%%%%%%%%%%%%%%%%%%%%%%%%%%%%%%%%%
\subsection{Forwarding}
\label{sec:forward}

Different versions of the main or child documents
using compilation flags as described in \secref{sec:flags}
can be (permanently) stored in different files
for convenient compilation, viewing and distribution.
To this end, the package defines a command
to pass on compilation to a different file:

%%%%%%%%%%%%%%%%%%%%%%%%%%%%%%%%%%%%%%%%
\DescribeMacro{\childdocforward}
The command |\childdocforward| redirects processing to
another source file:
%
\begin{center}
\begin{tabular}{l}
|\input{childdoc.def}|\\
|\childdocforward[|\textit{main}|]{|\textit{dest}|}|\\
\end{tabular}
\end{center}
%
The argument \textit{dest} is the destination file
(without extension).
It should be the main file or one of the child files.
Note that further \textsf{childdoc} directives
such as |\childdocof| and |\childdocforward|
in the indicated file will be processed in this form.
The optional argument \textit{main}
passes on directly to the main file \textit{main}
while pretending to compile the child \textit{dest}.
This form behaves as if \textit{dest}
issues |\childdocof{|\textit{main}|}| right away,
and no further \textsf{childdoc} directives will be processed.

%%%%%%%%%%%%%%%%%%%%%%%%%%%%%%%%%%%%%%%%
\DescribeMacro{\...prefix}
In the alternative form |\childdocforwardprefix|,
%
\begin{center}
\begin{tabular}{l}
|\input{childdoc.def}|\\
|\childdocforwardprefix[|\textit{main}|]{|\textit{prefix}|}{|\textit{dest}|}|
\end{tabular}
\end{center}
%
the destination file is determined by a pattern
depending on the current file:
To make this work, the current file must be called
`{\textit{prefix}\hspace{0.2em}\textit{suffix}}'
with \textit{prefix} matching precisely the argument.
Processing is then passed on to the file
`{\textit{dest}\hspace{0.2em}\textit{suffix}}'.
Surely, the same effect is achieved by
directly specifying the
argument `{\textit{dest}\hspace{0.2em}\textit{suffix}}'
in the first form.
However, that requires to set up a different file
for each child. With the alternative form of the command
all these files can have exactly the same content
which simplifies setting them up and maintaining them.

For example, the following file |draft.tex|
with a compilation flag |\version| as described in \secref{sec:flags}
compiles the main document as a draft:
%
\begin{center}
\begin{tabular}{l}
|\def\version{draft}|\\
|\input{childdoc.def}|\\
|\childdocforward{|\textit{main}|}|
\end{tabular}
\end{center}
%
Likewise, the following files |final|\textit{nn}|.tex|
compile the final version of the child document
|child|\textit{nn}|.tex|:
%
\begin{center}
\begin{tabular}{l}
|\def\version{final}|\\
|\input{childdoc.def}|\\
|\childdocforwardprefix{final}{child}|
\end{tabular}
\end{center}
%

Note that when several versions of a main file and/or of each child file
are to be generated, it may be convenient to set up a |Makefile| or
shell script to automatise the process.

%%%%%%%%%%%%%%%%%%%%%%%%%%%%%%%%%%%%%%%%%%%%%%%%%%%%%%%%%%%%%%%%%%%%%%%%%%%%%%%%
\subsection{Command Line Processing}
\label{sec:commandline}

The effect of redirection files can also be achieved by invoking
the \LaTeX{} compiler with a more elaborate command line.
Most conveniently this should be done as part
of a shell script or a |Makefile|.

When using \textsf{childdoc} in the main file, the following
command lines effectively perform a redirection
(note that depending on the shell being used,
backslashes may have to be doubled: `|\|' $\to$ `|\\|'):
%
\begin{center}
|... -jobname "|\textit{target}|" |\\|"|[\textit{flags}]%
|\input{childdoc.def}\childdocforward[|\textit{main}|]{|\textit{dest}|}"|
\end{center}
%
Here \textit{target} is the name of the output file,
\textit{main} is the name of the main file
and \textit{dest} is the name of the main or child file to be processed
(all filenames without extensions).
The optional argument \textit{main} can be omitted
if \textit{main} matches \textit{dest}.
Optionally, compilation \textit{flags} can be defined via |\def| commands.
This command line makes the \TeX{} engine believe
it is compiling the file \textit{target}
whose content is specified as the latter parameter.
The provided code then forwards the processing to
\textit{main} or \textit{dest} as described in \secref{sec:forward}.

%%%%%%%%%%%%%%%%%%%%%%%%%%%%%%%%%%%%%%%%%%%%%%%%%%%%%%%%%%%%%%%%%%%%%%%%%%%%%%%%
\subsection{Include by Input}
\label{sec:input}

Including child documents by |\include| has some restrictions by design.
Most notably, the content of a child document always occupies
its own set of pages; pages cannot be shared between child documents.
Usually, this behaviour makes perfect sense
because each child document contain an essential part of the document.
However, in some situations it may be desirable to compose
a document from a collection of parts
without having mandatory page breaks between then.
For this case, the package
provides a mechanism to include parts
by |\input| which can also be processed individually.
However, by construction this mechanism
requires manual handling of the content to be output.

%%%%%%%%%%%%%%%%%%%%%%%%%%%%%%%%%%%%%%%%
\DescribeMacro{\ifchilddocmanual}
The main file should be prepared as usual, see \secref{sec:include}.
However, the document body must make a distinction
between processing of an individual part and of the main document, e.g.:
%
\begin{center}
\begin{tabular}{l}
|\ifchilddocmanual|\\
|\input{\childdocname}|\\
|\||else|\\
\textit{document body with }|\input{|\textit{part}|}|\\
|\||fi|
\end{tabular}
\end{center}
%
The conditional |\ifchilddocmanual| is true whenever
a part to be included by |\input| is being compiled,
and the name of the part is stored in |\childdocname|.

%%%%%%%%%%%%%%%%%%%%%%%%%%%%%%%%%%%%%%%%
\DescribeMacro{\childdocby}
Each part to be included by |\input| should start with:
%
\begin{center}
\begin{tabular}{l}
|\input{childdoc.def}|\\
|\childdocby{|\textit{main}|}|\\
\end{tabular}
\end{center}
%
The directive |\childdocby| is similar to |\childdocof|
described in \secref{sec:include},
but the subsequent selection of content must be done manually.
To that end, both |\ifchilddoc| and |\ifchilddocmanual|
will be true upon processing of a part,
and the name of the part is stored in |\childdocname|.
Note that |\jobname| will be set to the filename of the current part
so that each part receives an individual |.aux| file
that does not interfere with the |.aux| file(s) of the main document.
This behaviour can be altered by the alternative form
|\childdocby[*]{|\textit{main}|}| (with a non-empty optional argument)
which uses the |.aux| file of the main document
by setting |\jobname| to \textit{main}.

%%%%%%%%%%%%%%%%%%%%%%%%%%%%%%%%%%%%%%%%%%%%%%%%%%%%%%%%%%%%%%%%%%%%%%%%%%%%%%%%
\subsection{Driver Development}
\label{sec:driver}

The \textsf{childdoc} mechanism can also be use for the development
of definition files such as \LaTeX{} styles or classes.
This case differs from the above setup with multiple parts
included by |\include| in that no |\includeonly| should be invoked.
This can be achieved by starting the include file
(before |\ProvidesPackage|) with:
%
\begin{center}
\begin{tabular}{l}
|\input{childdoc.def}|\\
|\childdocforward{|\textit{main}|}|\\
\end{tabular}
\end{center}
%
or alternatively with:
%
\begin{center}
\begin{tabular}{l}
|\input{childdoc.def}|\\
|\childdocby{|\textit{main}|}|\\
\end{tabular}
\end{center}
%
Both forms have slightly different effects as described above.
The main file is prepared as usual, see \secref{sec:include}.

%%%%%%%%%%%%%%%%%%%%%%%%%%%%%%%%%%%%%%%%%%%%%%%%%%%%%%%%%%%%%%%%%%%%%%%%%%%%%%%%
\subsection{Legacy Detection}
\label{sec:detection}

The directive |\childdocmain| in the main file can detect
whether the complete document or merely a child is to be compiled
even without using the directive |\childdocof|.
This method is deprecated because it is less robust
and there is no compelling reason to use it;
it is merely provided for backward compatibility
and it may be removed in future versions.

If the detection mechanism is to be used,
it is mandatory to correctly specify
the filename of the main file as the argument of |\childdocmain|:
%
\begin{center}
\begin{tabular}{l}
|\input{childdoc.def}|\\
|\childdocmain{|\textit{main}|}|\\
\end{tabular}
\end{center}
%
If |\jobname| does not match the argument \textit{main} of |\childdocmain|,
it is assumed that |\jobname| points to the child file to be compiled.
When using |\childdocmain| with the main file specified as argument,
it suffices to start a child file
with just |\input{|\textit{main}|}|
without loading of the package and using |\childdocof|.
If instead all processing is done
with the appropriate \textsf{childdoc} directives,
the argument of \textit{main} of |\childdocmain| can be empty.

An alternative version of the command line processing described
in \secref{sec:commandline} using the detection mechanism reads:
%
\begin{center}
|... -jobname "|\textit{target}|" "|[\textit{flags}]%
[|\def\jobname{|\textit{dest}|}|]|\input{|\textit{main}|}"|
\end{center}

%%%%%%%%%%%%%%%%%%%%%%%%%%%%%%%%%%%%%%%%%%%%%%%%%%%%%%%%%%%%%%%%%%%%%%%%%%%%%%%%
\subsection{Manual Code}
\label{sec:manual}

In case one cannot be certain whether the definitions file |childdoc.def|
is installed on the target \TeX{} distribution
and one prefers not to ship it,
it is conceivable to paste a few relevant commands into the sources.

To that end, drop all statements |\input{childdoc.def}|
and perform the replacements as outlined below.
Instead of |\childdocmain{|\textit{main}|}| add the following code
to the top of the main file:
%
\begin{center}
\begin{tabular}{l}
|\||ifdefined\childdocname\endinput\||fi\newif\ifchilddoc|\\
|\edef\childdocname{\scantokens\expandafter{\jobname\noexpand}}|\\
|\def\childdocmain{|\textit{main}|}\||ifx\childdocmain\childdocname\||else|\\
|\childdoctrue\includeonly{\childdocname}\let\jobname\childdocmain\||fi|\\
\end{tabular}
\end{center}
%
Instead of |\childdocof{|\textit{main}|}| just include the main file
at the top of each child file:
%
\begin{center}
|\input{|\textit{main}|}|
\end{center}
%
A simple redirection |\childdocforward{|\textit{dest}|}| is achieved by:
%
\begin{center}
|\def\jobname{|\textit{dest}|}\input{\jobname}|
\end{center}
%
The redirection with prefix
|\childdocforwardprefix[|\textit{prefix}|]{|\textit{dest}|}|
is accomplished by:
%
\begin{center}
\begin{tabular}{l}
|{\edef\jobname{\scantokens\expandafter{\jobname\noexpand}}|\\
|\def\redirectjob |\textit{prefix}|#1~~~{\gdef\jobname{|\textit{dest}|#1}}|\\
|\expandafter\redirectjob\jobname~~~}\input{\jobname}|
\end{tabular}
\end{center}

In an alternative approach,
child documents can be compiled by a specific command line
without additional code or specific definitions:
%
\begin{center}
|... -jobname "|\textit{target}|" "|[\textit{flags}]%
|\includeonly{|\textit{dest}|}\input{|\textit{main}|}"|
\end{center}
%

%%%%%%%%%%%%%%%%%%%%%%%%%%%%%%%%%%%%%%%%%%%%%%%%%%%%%%%%%%%%%%%%%%%%%%%%%%%%%%%%
%%%%%%%%%%%%%%%%%%%%%%%%%%%%%%%%%%%%%%%%%%%%%%%%%%%%%%%%%%%%%%%%%%%%%%%%%%%%%%%%
\section{Information}

%%%%%%%%%%%%%%%%%%%%%%%%%%%%%%%%%%%%%%%%%%%%%%%%%%%%%%%%%%%%%%%%%%%%%%%%%%%%%%%%
\subsection{Copyright}

Copyright \copyright{} 2017--2018 Niklas Beisert

This work may be distributed and/or modified under the
conditions of the \LaTeX{} Project Public License, either version 1.3
of this license or (at your option) any later version.
The latest version of this license is in
  \url{http://www.latex-project.org/lppl.txt}
and version 1.3 or later is part of all distributions of \LaTeX{}
version 2005/12/01 or later.

This work has the LPPL maintenance status `maintained'.

The Current Maintainer of this work is Niklas Beisert.

This work consists of the files |README.txt|, |childdoc.ins| and |childdoc.dtx|
as well as the derived files |childdoc.def|, |cdocsamp.tex|
with |cdocsch1.tex|, |cdocsch2.tex|, |cdocspt3.tex|, |cdocspt4.tex|,
|cdocsdrf.tex|, |cdocsfn1.tex|, |cdocsfn2.tex|
as well as |childdoc.pdf|.

%%%%%%%%%%%%%%%%%%%%%%%%%%%%%%%%%%%%%%%%%%%%%%%%%%%%%%%%%%%%%%%%%%%%%%%%%%%%%%%%
\subsection{Files and Installation}

The package consists of the files:
%
\begin{center}
\begin{tabular}{ll}
    |README.txt|   & readme file \\
    |childdoc.ins| & installation file \\
    |childdoc.dtx| & source file \\
    |childdoc.def| & definition file \\
    |cdocsamp.tex| & sample main file \\
    |cdocsch1.tex| & sample include file \\
    |cdocsch2.tex| & sample include file \\
    |cdocspt3.tex| & sample part file \\
    |cdocspt4.tex| & sample part file \\
    |cdocsdrf.tex| & sample redirection file \\
    |cdocsfn1.tex| & sample redirection file \\
    |cdocsfn2.tex| & sample redirection file \\
    |childdoc.pdf| & manual
\end{tabular}
\end{center}
%
The distribution consists of the files
|README.txt|, |childdoc.ins| and |childdoc.dtx|.
%
\begin{itemize}
\item
Run (pdf)\LaTeX{} on |childdoc.dtx|
to compile the manual |childdoc.pdf| (this file).
\item
Run \LaTeX{} on |childdoc.ins| to create the definitions file |childdoc.def|
and the sample |cdocsamp.tex| with include files
|cdocsch1.tex|, |cdocsch2.tex|, |cdocspt3.tex|, |cdocspt4.tex|,
|cdocsdrf.tex|, |cdocsfn1.tex|, |cdocsfn2.tex|.
Then copy the file |childdoc.def| to an appropriate directory of your \LaTeX{}
distribution, e.g.\ \textit{texmf-root}|/tex/latex/childdoc|.
\end{itemize}

%%%%%%%%%%%%%%%%%%%%%%%%%%%%%%%%%%%%%%%%%%%%%%%%%%%%%%%%%%%%%%%%%%%%%%%%%%%%%%%%
\subsection{Related CTAN Packages}

There are several other packages which offer a similar functionality:
%
\begin{itemize}
\item
The packages
\href{http://ctan.org/pkg/docmute}{\textsf{docmute}},
\href{http://ctan.org/pkg/includex}{\textsf{includex}} and
\href{http://ctan.org/pkg/standalone}{\textsf{standalone}}
provide commands to include only the document body of
a child file thus allowing both files to be compiled individually.
\item
The packages \href{http://ctan.org/pkg/subdocs}{\textsf{subdocs}}
and \href{http://ctan.org/pkg/subfiles}{\textsf{subfiles}}
provide structures in which the main and child documents can be
encapsulated and allowing them to be compiled individually.
The inclusion mechanism is different from the conventional |\include|.
\item
The package \href{http://ctan.org/pkg/combine}{\textsf{combine}}
is an elaborate solution to combine several documents into one.
\end{itemize}
%
See also the CTAN topic \href{http://ctan.org/topic/subdocs}{\textsf{subdocs}}
for further related packages.
The present package differs from the above solutions in that
a document structure constructed with the conventional |\include| mechanism
just needs two extra commands at the top of every file
such that all constituent files can be compiled individually.

%%%%%%%%%%%%%%%%%%%%%%%%%%%%%%%%%%%%%%%%%%%%%%%%%%%%%%%%%%%%%%%%%%%%%%%%%%%%%%%%
%\subsection{Feature Suggestions}
%
%The following is a list of features which may be useful for future
%versions of this package:
%%
%\begin{itemize}
%\item
%\ldots
%\end{itemize}

%%%%%%%%%%%%%%%%%%%%%%%%%%%%%%%%%%%%%%%%%%%%%%%%%%%%%%%%%%%%%%%%%%%%%%%%%%%%%%%%
\subsection{Revision History}

%%%%%%%%%%%%%%%%%%%%%%%%%%%%%%%%%%%%%%%%
\paragraph{v2.0:} 2018/12/30

\begin{itemize}
\item
immediate forward processing
\item
added |\childdocby| mechanism
\item
manual restructured
\end{itemize}

%%%%%%%%%%%%%%%%%%%%%%%%%%%%%%%%%%%%%%%%
\paragraph{v1.6:} 2018/01/17

\begin{itemize}
\item
application for development of include files
\item
corrections to manual
\end{itemize}

%%%%%%%%%%%%%%%%%%%%%%%%%%%%%%%%%%%%%%%%
\paragraph{v1.5:} 2017/05/21

\begin{itemize}
\item
more complete structuring introduced
\item
|\childdocof| introduced
\item
|\childdoc| renamed to |\childdocmain|
\item
|\childredirect| renamed to |\childdocforward| and |\childdocforwardprefix|
and functionality expanded
\end{itemize}

%%%%%%%%%%%%%%%%%%%%%%%%%%%%%%%%%%%%%%%%
\paragraph{v1.0:} 2017/04/27

\begin{itemize}
\item
manual and install package
\item
first version published on CTAN
\end{itemize}

%%%%%%%%%%%%%%%%%%%%%%%%%%%%%%%%%%%%%%%%
\paragraph{v0.6:} 2017/04/26

\begin{itemize}
\item
redirection mechanism added
\end{itemize}

%%%%%%%%%%%%%%%%%%%%%%%%%%%%%%%%%%%%%%%%
\paragraph{v0.5:} 2017/04/26

\begin{itemize}
\item
functionality in definition file
\end{itemize}


%%%%%%%%%%%%%%%%%%%%%%%%%%%%%%%%%%%%%%%%%%%%%%%%%%%%%%%%%%%%%%%%%%%%%%%%%%%%%%%%
%%%%%%%%%%%%%%%%%%%%%%%%%%%%%%%%%%%%%%%%%%%%%%%%%%%%%%%%%%%%%%%%%%%%%%%%%%%%%%%%
%%%%%%%%%%%%%%%%%%%%%%%%%%%%%%%%%%%%%%%%%%%%%%%%%%%%%%%%%%%%%%%%%%%%%%%%%%%%%%%%
\appendix

\settowidth\MacroIndent{\rmfamily\scriptsize 000\ }

 \DocInput{childdoc.dtx}

\end{document}
%</driver>
% \fi
%
% %%%%%%%%%%%%%%%%%%%%%%%%%%%%%%%%%%%%%%%%%%%%%%%%%%%%%%%%%%%%%%%%%%%%%%%%%%%%%%
% %%%%%%%%%%%%%%%%%%%%%%%%%%%%%%%%%%%%%%%%%%%%%%%%%%%%%%%%%%%%%%%%%%%%%%%%%%%%%%
% \section{Sample}
%\iffalse
%<*samplemain>
%\fi
%
% The following presents a sample document
% with two chapters, two parts, a title page,
% a compile flag as well as three forwarding files to set the flag.
% It consists of eight |.tex| files:
% \begin{center}
% \begin{tabular}{ll}
% |cdocsamp.tex|&main file\\
% |cdocsch1.tex|&include file for chapter 1\\
% |cdocsch2.tex|&include file for chapter 2\\
% |cdocspt3.tex|&include file for part 3\\
% |cdocspt4.tex|&include file for part 4\\
% |cdocsdrf.tex|&forwarding file for main file in draft mode\\
% |cdocsfi1.tex|&forwarding file for final version of chapter 1\\
% |cdocsfi2.tex|&forwarding file for final version of chapter 2\\
% \end{tabular}
% \end{center}
% Each of the eight files can be compiled directly by the \LaTeX{} compiler.
%
% %%%%%%%%%%%%%%%%%%%%%%%%%%%%%%%%%%%%%%
% \paragraph{Main File.}
%
% The main file is called |cdocsamp.tex|.
%
% Load the \textsf{childdoc} definitions and
% declare the filename for the main document:
%    \begin{macrocode}
\input{childdoc.def}
\childdocmain{}
%    \end{macrocode}

% Optional override for |\version| flag:
%    \begin{macrocode}
%%\ifchilddoc\else\providecommand{\version}{draft}\fi
%    \end{macrocode}

% Define the default values for the |\version| flag
% (|final| for the main file and |draft| for childs):
%    \begin{macrocode}
\ifchilddoc
\providecommand{\version}{draft}
\else
\providecommand{\version}{final}
\fi
%    \end{macrocode}

% Load the standard document class:
%    \begin{macrocode}
\documentclass[12pt]{article}
%    \end{macrocode}

% Start the document body:
%    \begin{macrocode}
\begin{document}
%    \end{macrocode}

% Declare a title page.
% Print title, part of document being processed and version flag:
%    \begin{macrocode}
\addtocounter{page}{-1}
\begin{center}
{\LARGE\bfseries{}childdoc example\par}
\vspace{1cm}
\ifchilddoc
\ifchilddocmanual part\else chapter\fi:
`\childdocname' of `\childdocjob'\par
\else
main document: `\childdocjob'\par
\fi
version: \version\par
\end{center}
\newpage
%    \end{macrocode}

% Manually include selected file,
% otherwise process as usual:
%    \begin{macrocode}
\ifchilddocmanual
\section*{part `\childdocname'}
\input{\childdocname}
\else
%    \end{macrocode}

% Include the two chapters:
%    \begin{macrocode}
\include{cdocsch1}
\include{cdocsch2}
%    \end{macrocode}

% Include the two parts unless only chapters should be displayed:
%    \begin{macrocode}
\ifchilddoc\else
\section{part three}
\input{cdocspt3}
\section{part four}
\input{cdocspt4}
\fi
%    \end{macrocode}

% Process as usual until here:
%    \begin{macrocode}
\fi
%    \end{macrocode}

% End of document body:
%    \begin{macrocode}
\end{document}
%    \end{macrocode}
%\iffalse
%</samplemain>
%\fi
%
% %%%%%%%%%%%%%%%%%%%%%%%%%%%%%%%%%%%%%%
% \paragraph{Chapter Include Files.}
%
% The include files are called |cdocsch1.tex| and |cdocsch2.tex|.
%
%\iffalse
%<*samplechap1|samplechap2>
%\fi

% Optional override for |\version| flag:
%    \begin{macrocode}
%%\providecommand{\version}{final}
%    \end{macrocode}

% Include the main document:
%    \begin{macrocode}
\input{childdoc.def}
\childdocof{cdocsamp}
%    \end{macrocode}

%\iffalse
%</samplechap1|samplechap2>
%\fi
%
%\iffalse
%<*samplechap1>
%\fi
% Some text for chapter 1:
%    \begin{macrocode}
\section{one}
some text in chapter one
%    \end{macrocode}

%\iffalse
%</samplechap1>
%\fi
% Some text for chapter 2:
%\iffalse
%<*samplechap2>
%\fi
%    \begin{macrocode}
\section{two}
more text in chapter two
%    \end{macrocode}

%\iffalse
%</samplechap2>
%\fi
%
% %%%%%%%%%%%%%%%%%%%%%%%%%%%%%%%%%%%%%%
% \paragraph{Part Include Files.}
%
% The include files are called |cdocspt3.tex| and |cdocspt4.tex|.
%
%\iffalse
%<*samplepart3|samplepart4>
%\fi

% Optional override for |\version| flag:
%    \begin{macrocode}
%%\providecommand{\version}{final}
%    \end{macrocode}

% Include the main document:
%    \begin{macrocode}
\input{childdoc.def}
\childdocby{cdocsamp}
%    \end{macrocode}

%\iffalse
%</samplepart3|samplepart4>
%\fi
%
%\iffalse
%<*samplepart3>
%\fi
% Some text for part 3:
%    \begin{macrocode}
some text in part three
%    \end{macrocode}

%\iffalse
%</samplepart3>
%\fi
% Some text for part 4:
%\iffalse
%<*samplepart4>
%\fi
%    \begin{macrocode}
more text in part four
%    \end{macrocode}

%\iffalse
%</samplepart4>
%\fi
%
% %%%%%%%%%%%%%%%%%%%%%%%%%%%%%%%%%%%%%%
% \paragraph{Forwarding for a Complete Draft.}
%
% The following forwarding file |cdocsdrf.tex|
% compiles the main document in draft mode:
%\iffalse
%<*sampledraft>
%\fi
%    \begin{macrocode}
\def\version{draft}
\input{childdoc.def}
\childdocforward{cdocsamp}
%    \end{macrocode}

%\iffalse
%</sampledraft>
%\fi
%
% %%%%%%%%%%%%%%%%%%%%%%%%%%%%%%%%%%%%%%
% \paragraph{Forwarding for Final Version of the Chapters.}
%
% The following forwarding files |cdocsfn1.tex| and |cdocsfn2.tex|
% (with identical content)
% compile the final versions of the child documents
% |cdocsch1.tex| and |cdocsch2.tex|, respectively:
%\iffalse
%<*samplefinal>
%\fi
%    \begin{macrocode}
\def\version{final}
\input{childdoc.def}
\childdocforwardprefix[cdocsamp]{cdocsfn}{cdocsch}
%    \end{macrocode}

%\iffalse
%</samplefinal>
%\fi
%
% %%%%%%%%%%%%%%%%%%%%%%%%%%%%%%%%%%%%%%
% \paragraph{Command Line Processing.}
%
% The following three command lines generate the output files
% |cdocscld|, |cdocscl1| and |cdocscl2|
% which should be identical to
% |cdocsdrf|, |cdocsch1| and |cdocsfn2|, respectively:
% \begin{center}
% \begin{tabular}{l}
% |latex -jobname cdocscld \|\\
% |  "\def\version{draft}\input{childdoc.def}\childdocforward{cdocsamp}"|\\
% |latex -jobname cdocscl1 \|\\
% |  "\input{childdoc.def}\childdocforward[cdocsamp]{cdocsch1}"|\\
% |latex -jobname cdocscl2 \|\\
% |  "\def\version{final}\input{childdoc.def}\childdocforward{cdocsch2}"|
% \end{tabular}
% \end{center}
% Note that the trailing backslash on each first line
% merely continues the input to the second line
% (for convenient cut ant paste).
% Furthermore, the command |latex| can be replaced by any
% of its alternative versions such as |pdflatex|.
%
% %%%%%%%%%%%%%%%%%%%%%%%%%%%%%%%%%%%%%%%%%%%%%%%%%%%%%%%%%%%%%%%%%%%%%%%%%%%%%%
% %%%%%%%%%%%%%%%%%%%%%%%%%%%%%%%%%%%%%%%%%%%%%%%%%%%%%%%%%%%%%%%%%%%%%%%%%%%%%%
% \section{Implementation}
%\iffalse
%<*package>
%\fi
%
% This section describes the definitions file |childdoc.def|.

% The definitions cannot be loaded using |\usepackage| or |\RequirePackage|
% which has a mechanism to prevent loading a style file more than once.
% When loading the definitions by means of |\input|
% multiple instances have to be prevented manually:
%\iffalse
%This code needs to be before the `\ProvidesFile' directive
%which is defined at the beginning of this file.
%Therefore it is also placed there and commented out here.
%</package>
%<*discard>
%\fi
%    \begin{macrocode}
\ifdefined\childdocmain\endinput\fi
%    \end{macrocode}
%\iffalse
%</discard>
%<*package>
%\fi
%
% \macro{\ifchilddoc}
% \macro{\ifchilddocmanual}
% The conditional |\ifchilddoc| tells whether a
% child (true) or main (false) document is being compiled.
% The conditional |\ifchilddocmanual| tells whether
% the |\includeonly| mechanism is used (false) or
% the selection of child files must be performed manually (true).
% The definitions initialise to false:
%    \begin{macrocode}
\newif\ifchilddoc
\newif\ifchilddocmanual
%    \end{macrocode}

% \macro{\childdocname}
% \macro{\childdocjob}
% The macro |\childdocname| stores the name of the main document
% to be compiled. The macro |\childdocjob| stores the name of
% the document on which the \LaTeX{} compiler was originally invoked.
% The content of |\jobname| cannot be compared
% to filenames specified in the source due to different catcodes.
% The following code rescans |\jobname|, stores the result
% in |\childdocname| and saves a copy in |\childdocjob|:
%    \begin{macrocode}
\edef\childdocname{\scantokens\expandafter{\jobname\noexpand}}
\let\childdocjob\childdocname
%    \end{macrocode}

% \macro{\childdocdisable}
% The macro |\childdocdisable| prevents the main file
% from being processed more than once.
% At this stage, the main document command |\childdocmain|
% is assumed to be called once again where it should do nothing.
% Any subsequent call to it should prevent
% a secondary processing of the main document
% It overwrites the forwarding commands
% |\childdocof| and |\childdocforward|
% with empty macros to prevent further inclusions of the main document:
%    \begin{macrocode}
\newcommand{\childdocdisable}
{
  \renewcommand{\childdocmain}[1]{\renewcommand{\childdocmain}[1]{\endinput}}
  \renewcommand{\childdocof}[1]{}
  \renewcommand{\childdocby}[2][]{}
  \renewcommand{\childdocforward}[2][]{}
  \renewcommand{\childdocdisable}{}
}
%    \end{macrocode}

% \macro{\childdocmain}
% The macro |\childdocmain| is to be called at the top of the main file
% with nothing or the main filename (without extension) as argument.
% First, it breaks loops.
% If the argument is not empty and does not match |\childdocname|
% (which is set by the first inclusion of |childdoc.def|),
% |\ifchilddoc| is set to true, |\includeonly| is applied to the child file
% and |\jobname| is set to the main file
% (for proper handling of |.aux| files):
%    \begin{macrocode}
\newcommand{\childdocmain}[1]
{
  \childdocdisable\childdocmain{}
  \if?#1?\else
    \begingroup
      \def\childdoctmp{#1}
      \ifx\childdoctmp\childdocname
        \def\childdoctmp{}
      \else
        \def\childdoctmp
        {
          \childdoctrue
          \includeonly{\childdocname}
          \def\childdocjob{#1}
          \def\jobname{#1}
        }
      \fi
      \expandafter
    \endgroup
    \childdoctmp
  \fi
}
%    \end{macrocode}

% \macro{\childdocof}
% The command |\childdocof| redirects
% compilation to the main file |#1|.
%    \begin{macrocode}
\newcommand{\childdocof}[1]
{
  \childdocdisable
  \childdoctrue
  \includeonly{\childdocname}
  \def\jobname{#1}
  \def\childdocjob{#1}
  \input{#1}
}
%    \end{macrocode}

% \macro{\childdocby}
% The command |\childdocby| ....
%    \begin{macrocode}
\newcommand{\childdocby}[2][]
{
  \childdocdisable
  \childdoctrue
  \childdocmanualtrue
  \if?#1?\else
    \def\jobname{#2}
  \fi
  \def\childdocjob{#2}
  \input{#2}
  \endinput
}
%    \end{macrocode}

% \macro{\childdocforward}
% The command |\childdocforward| redirects
% compilation to the main file or
% (if the optional argument is given) a child file.
% Parameters are set as if the main file
% or a child file starting with |\childdocof| was compiled.
% Then compilation is handed over to the main file:
%    \begin{macrocode}
\newcommand{\childdocforward}[2][]
{
  \begingroup
    \if?#1?
      \def\childdoctmp
      {
        \def\childdocname{#2}
        \def\childdocjob{#2}
        \def\jobname{#2}
        \input{#2}
        \endinput
      }
    \else
      \def\childdoctmp
      {
        \childdocdisable
        \def\childdocname{#2}
        \childdoctrue
        \includeonly{#2}
        \def\childdocjob{#1}
        \def\jobname{#1}
        \input{#1}
        \endinput
      }
    \fi
    \expandafter
  \endgroup
  \childdoctmp
}
%    \end{macrocode}

% \macro{\childdocforwardprefix}
% The command |\childdocforwardprefix| redirects
% compilation to the main or a child file by means of a pattern.
% The prefix |#1| in the current filename is replaced by |#2|
% and the suffix of the current filename is kept
% (it is assumed that the filename does not contain the substring `|~~~|'
% which is used as a delimiter).
% Compilation is handed over to the new file by |\childdocforward|:
%    \begin{macrocode}
\newcommand{\childdocforwardprefix}[3][]
{
  \begingroup
    \def\childdocextract #2##1~~~{\def\childdoctmp{\childdocforward[#1]{#3##1}}}
    \expandafter\childdocextract\childdocname~~~
    \expandafter
  \endgroup
  \childdoctmp
}
%    \end{macrocode}

% \macro{\childdoc}
% The deprecated macro |\childdoc| is a legacy version of |\childdocmain|:
%    \begin{macrocode}
\newcommand{\childdoc}{\childdocmain}
%    \end{macrocode}

% \macro{\childdocredirect}
% The deprecated macro |\childdocredirect| is a legacy version
% of |\childdocforward| and |\childdocforwardprefix|:
%    \begin{macrocode}
\newcommand{\childdocredirect}[2][]
{
  \begingroup
    \if?#1?
      \def\childdoctmp{\childdocforward{#2}}
    \else
      \def\childdoctmp{\childdocforwardprefix{#1}{#2}}
    \fi
    \expandafter
  \endgroup
  \childdoctmp
}
%    \end{macrocode}

%\iffalse
%</package>
%\fi
%
\endinput
\childdocforward[cdocsamp]{cdocsch1}"|\\
% |latex -jobname cdocscl2 \|\\
% |  "\def\version{final}% \iffalse
%
% childdoc.dtx Copyright (C) 2017-2018 Niklas Beisert
%
% This work may be distributed and/or modified under the
% conditions of the LaTeX Project Public License, either version 1.3
% of this license or (at your option) any later version.
% The latest version of this license is in
%   http://www.latex-project.org/lppl.txt
% and version 1.3 or later is part of all distributions of LaTeX
% version 2005/12/01 or later.
%
% This work has the LPPL maintenance status `maintained'.
%
% The Current Maintainer of this work is Niklas Beisert.
%
% This work consists of the files childdoc.dtx and childdoc.ins
% and the derived files childdoc.def and cdocsamp.tex with
% cdocsch1.tex, cdocsch2.tex, cdocsdrf.tex, cdocsfn1.tex, cdocsfn2.tex.
%
%<package>\ifdefined\childdocmain\endinput\fi
%<package>\ProvidesFile{childdoc.def}[2018/12/30 v2.0 child document driver]
%<samplemain>\ProvidesFile{cdocsamp.tex}[2018/12/30 v2.0 sample for childdoc]
%<*driver>
%\ProvidesFile{childdoc.drv}[2018/12/30 v2.0 childdoc reference manual file]
\PassOptionsToClass{10pt,a4paper}{article}
\documentclass{ltxdoc}

\usepackage[margin=35mm]{geometry}
\usepackage{hyperref}
\usepackage{hyperxmp}
\usepackage[usenames]{color}

\hypersetup{colorlinks=true}
\hypersetup{pdfstartview=FitH}
\hypersetup{pdfpagemode=UseNone}
\hypersetup{pdfsource={}}
\hypersetup{pdflang={en-UK}}
\hypersetup{pdfcopyright={Copyright 2017-2018 Niklas Beisert.
  This work may be distributed and/or modified under the
  conditions of the LaTeX Project Public License, either version 1.3
  of this license or (at your option) any later version.}}
\hypersetup{pdflicenseurl={http://www.latex-project.org/lppl.txt}}
\hypersetup{pdfcontactaddress={ETH Zurich, ITP, HIT K,
  Wolfgang-Pauli-Strasse 27}}
\hypersetup{pdfcontactpostcode={8093}}
\hypersetup{pdfcontactcity={Zurich}}
\hypersetup{pdfcontactcountry={Switzerland}}
\hypersetup{pdfcontactemail={nbeisert@itp.phys.ethz.ch}}
\hypersetup{pdfcontacturl={http://people.phys.ethz.ch/\xmptilde nbeisert/}}

\newcommand{\secref}[1]{\hyperref[#1]{section \ref*{#1}}}

\parskip1ex
\parindent0pt
\let\olditemize\itemize
\def\itemize{\olditemize\parskip0pt}

\begin{document}

\title{The \textsf{childdoc} Package}
\hypersetup{pdftitle={The childdoc Package}}
\author{Niklas Beisert\\[2ex]
  Institut f\"ur Theoretische Physik\\
  Eidgen\"ossische Technische Hochschule Z\"urich\\
  Wolfgang-Pauli-Strasse 27, 8093 Z\"urich, Switzerland\\[1ex]
  \href{mailto:nbeisert@itp.phys.ethz.ch}
  {\texttt{nbeisert@itp.phys.ethz.ch}}}
\hypersetup{pdfauthor={Niklas Beisert}}
\hypersetup{pdfsubject={Manual for the LaTeX2e Package childdoc}}
\date{30 December 2018, \textsf{v2.0}}
\maketitle

\begin{abstract}\noindent
\textsf{childdoc} is a \LaTeXe{} package
that enables the direct compilation
of document sections included by |\include|
to individual files.
\end{abstract}

\begingroup
\parskip0ex
\tableofcontents
\endgroup

%%%%%%%%%%%%%%%%%%%%%%%%%%%%%%%%%%%%%%%%%%%%%%%%%%%%%%%%%%%%%%%%%%%%%%%%%%%%%%%%
%%%%%%%%%%%%%%%%%%%%%%%%%%%%%%%%%%%%%%%%%%%%%%%%%%%%%%%%%%%%%%%%%%%%%%%%%%%%%%%%
\section{Introduction}

\LaTeX{} provides a mechanism to structure a large document (such as a book)
into a main file and several child files (containing the chapters)
using the |\include| command.
This mechanism is beneficial for documents
which span hundreds of pages in order to
make the source file(s) more manageable.
Moreover, compilation can be restricted to
selected child files by means of the |\includeonly| command.
The latter feature can be used to reduce the compilation time while editing
(this was significantly more useful in the earlier days of \LaTeX{})
or to generate a smaller document which is easier to navigate.
Another application of |\includeonly| is to generate
documents consisting of selected parts of the complete document.

However, there are a few drawbacks of the plain |\include| mechanism:
\begin{itemize}
\item
The child files cannot be compiled on their own,
they can only be compiled via the main file.
A naive editing environment
(such as a text editor with an option
to have the current file processed by \LaTeX)
may require one to switch to the main file before compiling;
attempting to compile the child file produces errors.
\item
The main file must be modified (each time)
to adjust the |\includeonly| command
to the present needs. This easily leaves the main file in a messy state.
\item
The generated document will always carry the filename
of the main document. This is inconvenient if
several child files are to be compiled and
to be kept for distribution.
\end{itemize}

The present package provides a simple interface
to make child files individually compilable by \LaTeX{}.
Compiling a child file then has the same effect as compiling
the main file with an |\includeonly| command
to select the appropriate child.
Moreover the generated document will carry the name of the child
rather than the main file.
This resolves all three above issues.

This feature is meant to make the editing of books,
thesis documents and lecture notes somewhat more convenient.
However, the package can also be used efficiently for
composing a series of documents (such as exercise sheets)
which are typically distributed individually.
It then assists the author in generating the individual documents
(potentially in different versions)
as well as a document containing the collected series.
Another application is in developing style files
or other kinds of included material
where compilation of the style file could redirect
to a sample or test file.

%%%%%%%%%%%%%%%%%%%%%%%%%%%%%%%%%%%%%%%%%%%%%%%%%%%%%%%%%%%%%%%%%%%%%%%%%%%%%%%%
%%%%%%%%%%%%%%%%%%%%%%%%%%%%%%%%%%%%%%%%%%%%%%%%%%%%%%%%%%%%%%%%%%%%%%%%%%%%%%%%
\section{Usage}

First of all, the package \textsf{childdoc} is \emph{not} a standard
\LaTeXe{} |.sty| style file! Therefore it needs to be invoked in
a non-standard way.

%%%%%%%%%%%%%%%%%%%%%%%%%%%%%%%%%%%%%%%%%%%%%%%%%%%%%%%%%%%%%%%%%%%%%%%%%%%%%%%%
\subsection{Included Files}
\label{sec:include}

%%%%%%%%%%%%%%%%%%%%%%%%%%%%%%%%%%%%%%%%
\DescribeMacro{\childdocmain}
To use the package, add the commands
\begin{center}
\begin{tabular}{l}
|\input{childdoc.def}|\\
|\childdocmain{}|\\
\end{tabular}
\end{center}
at the very top of the main \LaTeX{} file,
in particular \emph{before} the |\documentclass| statement!
The argument of |\childdocmain| should be left empty
(but it must be present).

%%%%%%%%%%%%%%%%%%%%%%%%%%%%%%%%%%%%%%%%
\DescribeMacro{\childdocof}
Furthermore, add the commands
\begin{center}
\begin{tabular}{l}
|\input{childdoc.def}|\\
|\childdocof{|\textit{main}|}|\\
\end{tabular}
\end{center}
at the top of every child file \textit{child}
which is included by |\include{|\textit{child}|}|
from within the main file
(or at least for those files to be compiled individually).
The argument \textit{main} must be the filename of the main file.

There are a couple of
considerations in setting up the main and child documents:

%%%%%%%%%%%%%%%%%%%%%%%%%%%%%%%%%%%%%%%%
\paragraph{Restrictions.}

Please note the following restrictions:
\begin{itemize}
\item
|\childdocmain| must be called with one argument \textit{main}
to ensure compatibility with earlier version of the package.
It must either be empty (|\childdocmain{}|)
or precisely match the filename of the main file in which it is specified.
See \secref{sec:detection} for further information.
\item
The filename \textit{main} must be specified without the |.tex| extension.
\item
The filename \textit{main} is case sensitive
(even in case-insensitive file systems)
due to internal string comparison.
\item
The argument \textit{main} should be fully expanded, it cannot be a macro.
\item
Subdirectories and special characters should be avoided in filenames.
\item
The command |\childdocmain{|\textit{main}|}| must be followed by a whitespace.
It should not be followed immediately by another command
or by a comment mark `|%|'.
This is because the \TeX{} parser reads the token immediately following
the argument of |\childdocmain| and puts it
at the beginning of every child section;
however, a white\-space is ignored.
\end{itemize}

%%%%%%%%%%%%%%%%%%%%%%%%%%%%%%%%%%%%%%%%
\paragraph{Content of Main File.}

It is advisable to place all content in the child files included by |\include|.
Any output contained in the main file will appear in all child documents
unless suppressed manually;
it cannot be suppressed automatically by the |\includeonly| directive
and thus should normally be avoided.
A method to include some content in the main file
by means of conditional processing is described in \secref{sec:conditional}.

%%%%%%%%%%%%%%%%%%%%%%%%%%%%%%%%%%%%%%%%
\paragraph{Page Numbering.}

When only a part of the document is compiled,
the appropriate numbering of pages
(as well as other status parameters)
is determined from the |.aux| files.
The latter contain information from previous passes.
However this information needs to propagate through
all intermediate child documents.
Therefore the page numbering in child documents may well
be inconsistent until the complete document is compiled at least once.

A useful (if unconventional) way to always ensure a consistent
page numbering is to restart the numbering in each child document
and denote the pages by `\textit{child}|.|\textit{page}'
where \textit{child} represents the chapter/section number of the child file.
This can be achieved by the command
|\numberwithin{page}{|\textit{child}|}|
of the \textsf{amsmath} package
where \textit{child} can be |chapter| or |section|
depending on the chosen structuring.
Alternatively, one can modify the macro |\thepage| appropriately
and reset the counter |page| at the start of each child file.

%%%%%%%%%%%%%%%%%%%%%%%%%%%%%%%%%%%%%%%%%%%%%%%%%%%%%%%%%%%%%%%%%%%%%%%%%%%%%%%%
\subsection{Conditional Processing}
\label{sec:conditional}

The package provides a mechanism to compile different versions
of a document. To customise the versions further some conditional processing
can come in handy to distinguish which version is being compiled.
The package provides two macros to describe the compilation context:

%%%%%%%%%%%%%%%%%%%%%%%%%%%%%%%%%%%%%%%%
\DescribeMacro{\ifchilddoc}
The conditional |\ifchilddoc| distinguishes between the compilation of
child documents and the main document:
%
\begin{center}
|\ifchilddoc |\textit{child-code}| |[|\||else |\textit{main-code}]| \||fi|
\end{center}

%%%%%%%%%%%%%%%%%%%%%%%%%%%%%%%%%%%%%%%%
\DescribeMacro{\childdocname}
\DescribeMacro{\childdocjob}
The macro |\childdocname| contains the filename (without extension)
of the main or child file being processed.
Note that |\childdocjob| will always contain the name of the main file.

%%%%%%%%%%%%%%%%%%%%%%%%%%%%%%%%%%%%%%%%
\paragraph{Title Page.}

Conditional processing can be used to include a title or banner page
in the main document when proper precautions are taken.
Importantly, the code in the main file should ensure that the page counter
(as well as other status parameters which are stored in the |.aux| files)
takes the same value after the conditional processing.
Otherwise the page numbers may take divergent values
depending on which part is compiled.

For example, a title page could be declared by:
%
\begin{center}
\begin{tabular}{l}
|\ifchilddoc\||else|\\
|\addtocounter{page}{-1}|\\
\textit{code for title page}\\
|\newpage|\\
|\||fi|
\end{tabular}
\end{center}
%
A banner page for the child documents can be generated by:
%
\begin{center}
\begin{tabular}{l}
|\ifchilddoc|\\
|\addtocounter{page}{-1}|\\
\textit{code for banner page}\\
|\newpage|\\
|\||fi|
\end{tabular}
\end{center}
%
Here one could write a message such as:
\begin{center}
|This is the part \childdocname{} of \childdocjob{}.|
\end{center}

%%%%%%%%%%%%%%%%%%%%%%%%%%%%%%%%%%%%%%%%%%%%%%%%%%%%%%%%%%%%%%%%%%%%%%%%%%%%%%%%
\subsection{Flags}
\label{sec:flags}

The package makes it easy to generate different versions
of the main or child documents.
To this end compilation flags can be defined
and assigned different default values.
They will be particularly useful in conjunction
with the forwarding mechanism described in \secref{sec:forward}.

For example, it may be useful to have a flag |\version|
which can be set to |draft| or |final|.
The document source will contain some conditional code
depending on the value of |\version|.
Suppose further, the flag should default to |final| for the main file
and to |draft| for child files
which is a natural assignment for editing the document.
This is achieved by placing the following code
in the preamble of the main document
(below the |\childdocmain| directive):
%
\begin{center}
\begin{tabular}{l}
|\ifchilddoc|\\
|\providecommand{\version}{draft}|\\
|\||else|\\
|\providecommand{\version}{final}|\\
|\||fi|
\end{tabular}
\end{center}
%
The definition by |\providecommand| makes sure
that previous definitions are not overwritten.
Further statements |\providecommand{\version}{...}|
can thus be added before the above code to override it.

For the main file, one might add a line
(between |\childdocmain| and the above block)
%
\begin{center}
|%\ifchilddoc\||else\providecommand{\version}{draft}\||fi|
\end{center}
%
which can be uncommented to produce a draft version.
Likewise one can add a line to the very top of a child file
(above the |\childdocof{|\textit{main}|}| directive)
%
\begin{center}
|%\providecommand{\version}{final}|
\end{center}
%
which can be uncommented to produce the final version of this child document.

%%%%%%%%%%%%%%%%%%%%%%%%%%%%%%%%%%%%%%%%%%%%%%%%%%%%%%%%%%%%%%%%%%%%%%%%%%%%%%%%
\subsection{Forwarding}
\label{sec:forward}

Different versions of the main or child documents
using compilation flags as described in \secref{sec:flags}
can be (permanently) stored in different files
for convenient compilation, viewing and distribution.
To this end, the package defines a command
to pass on compilation to a different file:

%%%%%%%%%%%%%%%%%%%%%%%%%%%%%%%%%%%%%%%%
\DescribeMacro{\childdocforward}
The command |\childdocforward| redirects processing to
another source file:
%
\begin{center}
\begin{tabular}{l}
|\input{childdoc.def}|\\
|\childdocforward[|\textit{main}|]{|\textit{dest}|}|\\
\end{tabular}
\end{center}
%
The argument \textit{dest} is the destination file
(without extension).
It should be the main file or one of the child files.
Note that further \textsf{childdoc} directives
such as |\childdocof| and |\childdocforward|
in the indicated file will be processed in this form.
The optional argument \textit{main}
passes on directly to the main file \textit{main}
while pretending to compile the child \textit{dest}.
This form behaves as if \textit{dest}
issues |\childdocof{|\textit{main}|}| right away,
and no further \textsf{childdoc} directives will be processed.

%%%%%%%%%%%%%%%%%%%%%%%%%%%%%%%%%%%%%%%%
\DescribeMacro{\...prefix}
In the alternative form |\childdocforwardprefix|,
%
\begin{center}
\begin{tabular}{l}
|\input{childdoc.def}|\\
|\childdocforwardprefix[|\textit{main}|]{|\textit{prefix}|}{|\textit{dest}|}|
\end{tabular}
\end{center}
%
the destination file is determined by a pattern
depending on the current file:
To make this work, the current file must be called
`{\textit{prefix}\hspace{0.2em}\textit{suffix}}'
with \textit{prefix} matching precisely the argument.
Processing is then passed on to the file
`{\textit{dest}\hspace{0.2em}\textit{suffix}}'.
Surely, the same effect is achieved by
directly specifying the
argument `{\textit{dest}\hspace{0.2em}\textit{suffix}}'
in the first form.
However, that requires to set up a different file
for each child. With the alternative form of the command
all these files can have exactly the same content
which simplifies setting them up and maintaining them.

For example, the following file |draft.tex|
with a compilation flag |\version| as described in \secref{sec:flags}
compiles the main document as a draft:
%
\begin{center}
\begin{tabular}{l}
|\def\version{draft}|\\
|\input{childdoc.def}|\\
|\childdocforward{|\textit{main}|}|
\end{tabular}
\end{center}
%
Likewise, the following files |final|\textit{nn}|.tex|
compile the final version of the child document
|child|\textit{nn}|.tex|:
%
\begin{center}
\begin{tabular}{l}
|\def\version{final}|\\
|\input{childdoc.def}|\\
|\childdocforwardprefix{final}{child}|
\end{tabular}
\end{center}
%

Note that when several versions of a main file and/or of each child file
are to be generated, it may be convenient to set up a |Makefile| or
shell script to automatise the process.

%%%%%%%%%%%%%%%%%%%%%%%%%%%%%%%%%%%%%%%%%%%%%%%%%%%%%%%%%%%%%%%%%%%%%%%%%%%%%%%%
\subsection{Command Line Processing}
\label{sec:commandline}

The effect of redirection files can also be achieved by invoking
the \LaTeX{} compiler with a more elaborate command line.
Most conveniently this should be done as part
of a shell script or a |Makefile|.

When using \textsf{childdoc} in the main file, the following
command lines effectively perform a redirection
(note that depending on the shell being used,
backslashes may have to be doubled: `|\|' $\to$ `|\\|'):
%
\begin{center}
|... -jobname "|\textit{target}|" |\\|"|[\textit{flags}]%
|\input{childdoc.def}\childdocforward[|\textit{main}|]{|\textit{dest}|}"|
\end{center}
%
Here \textit{target} is the name of the output file,
\textit{main} is the name of the main file
and \textit{dest} is the name of the main or child file to be processed
(all filenames without extensions).
The optional argument \textit{main} can be omitted
if \textit{main} matches \textit{dest}.
Optionally, compilation \textit{flags} can be defined via |\def| commands.
This command line makes the \TeX{} engine believe
it is compiling the file \textit{target}
whose content is specified as the latter parameter.
The provided code then forwards the processing to
\textit{main} or \textit{dest} as described in \secref{sec:forward}.

%%%%%%%%%%%%%%%%%%%%%%%%%%%%%%%%%%%%%%%%%%%%%%%%%%%%%%%%%%%%%%%%%%%%%%%%%%%%%%%%
\subsection{Include by Input}
\label{sec:input}

Including child documents by |\include| has some restrictions by design.
Most notably, the content of a child document always occupies
its own set of pages; pages cannot be shared between child documents.
Usually, this behaviour makes perfect sense
because each child document contain an essential part of the document.
However, in some situations it may be desirable to compose
a document from a collection of parts
without having mandatory page breaks between then.
For this case, the package
provides a mechanism to include parts
by |\input| which can also be processed individually.
However, by construction this mechanism
requires manual handling of the content to be output.

%%%%%%%%%%%%%%%%%%%%%%%%%%%%%%%%%%%%%%%%
\DescribeMacro{\ifchilddocmanual}
The main file should be prepared as usual, see \secref{sec:include}.
However, the document body must make a distinction
between processing of an individual part and of the main document, e.g.:
%
\begin{center}
\begin{tabular}{l}
|\ifchilddocmanual|\\
|\input{\childdocname}|\\
|\||else|\\
\textit{document body with }|\input{|\textit{part}|}|\\
|\||fi|
\end{tabular}
\end{center}
%
The conditional |\ifchilddocmanual| is true whenever
a part to be included by |\input| is being compiled,
and the name of the part is stored in |\childdocname|.

%%%%%%%%%%%%%%%%%%%%%%%%%%%%%%%%%%%%%%%%
\DescribeMacro{\childdocby}
Each part to be included by |\input| should start with:
%
\begin{center}
\begin{tabular}{l}
|\input{childdoc.def}|\\
|\childdocby{|\textit{main}|}|\\
\end{tabular}
\end{center}
%
The directive |\childdocby| is similar to |\childdocof|
described in \secref{sec:include},
but the subsequent selection of content must be done manually.
To that end, both |\ifchilddoc| and |\ifchilddocmanual|
will be true upon processing of a part,
and the name of the part is stored in |\childdocname|.
Note that |\jobname| will be set to the filename of the current part
so that each part receives an individual |.aux| file
that does not interfere with the |.aux| file(s) of the main document.
This behaviour can be altered by the alternative form
|\childdocby[*]{|\textit{main}|}| (with a non-empty optional argument)
which uses the |.aux| file of the main document
by setting |\jobname| to \textit{main}.

%%%%%%%%%%%%%%%%%%%%%%%%%%%%%%%%%%%%%%%%%%%%%%%%%%%%%%%%%%%%%%%%%%%%%%%%%%%%%%%%
\subsection{Driver Development}
\label{sec:driver}

The \textsf{childdoc} mechanism can also be use for the development
of definition files such as \LaTeX{} styles or classes.
This case differs from the above setup with multiple parts
included by |\include| in that no |\includeonly| should be invoked.
This can be achieved by starting the include file
(before |\ProvidesPackage|) with:
%
\begin{center}
\begin{tabular}{l}
|\input{childdoc.def}|\\
|\childdocforward{|\textit{main}|}|\\
\end{tabular}
\end{center}
%
or alternatively with:
%
\begin{center}
\begin{tabular}{l}
|\input{childdoc.def}|\\
|\childdocby{|\textit{main}|}|\\
\end{tabular}
\end{center}
%
Both forms have slightly different effects as described above.
The main file is prepared as usual, see \secref{sec:include}.

%%%%%%%%%%%%%%%%%%%%%%%%%%%%%%%%%%%%%%%%%%%%%%%%%%%%%%%%%%%%%%%%%%%%%%%%%%%%%%%%
\subsection{Legacy Detection}
\label{sec:detection}

The directive |\childdocmain| in the main file can detect
whether the complete document or merely a child is to be compiled
even without using the directive |\childdocof|.
This method is deprecated because it is less robust
and there is no compelling reason to use it;
it is merely provided for backward compatibility
and it may be removed in future versions.

If the detection mechanism is to be used,
it is mandatory to correctly specify
the filename of the main file as the argument of |\childdocmain|:
%
\begin{center}
\begin{tabular}{l}
|\input{childdoc.def}|\\
|\childdocmain{|\textit{main}|}|\\
\end{tabular}
\end{center}
%
If |\jobname| does not match the argument \textit{main} of |\childdocmain|,
it is assumed that |\jobname| points to the child file to be compiled.
When using |\childdocmain| with the main file specified as argument,
it suffices to start a child file
with just |\input{|\textit{main}|}|
without loading of the package and using |\childdocof|.
If instead all processing is done
with the appropriate \textsf{childdoc} directives,
the argument of \textit{main} of |\childdocmain| can be empty.

An alternative version of the command line processing described
in \secref{sec:commandline} using the detection mechanism reads:
%
\begin{center}
|... -jobname "|\textit{target}|" "|[\textit{flags}]%
[|\def\jobname{|\textit{dest}|}|]|\input{|\textit{main}|}"|
\end{center}

%%%%%%%%%%%%%%%%%%%%%%%%%%%%%%%%%%%%%%%%%%%%%%%%%%%%%%%%%%%%%%%%%%%%%%%%%%%%%%%%
\subsection{Manual Code}
\label{sec:manual}

In case one cannot be certain whether the definitions file |childdoc.def|
is installed on the target \TeX{} distribution
and one prefers not to ship it,
it is conceivable to paste a few relevant commands into the sources.

To that end, drop all statements |\input{childdoc.def}|
and perform the replacements as outlined below.
Instead of |\childdocmain{|\textit{main}|}| add the following code
to the top of the main file:
%
\begin{center}
\begin{tabular}{l}
|\||ifdefined\childdocname\endinput\||fi\newif\ifchilddoc|\\
|\edef\childdocname{\scantokens\expandafter{\jobname\noexpand}}|\\
|\def\childdocmain{|\textit{main}|}\||ifx\childdocmain\childdocname\||else|\\
|\childdoctrue\includeonly{\childdocname}\let\jobname\childdocmain\||fi|\\
\end{tabular}
\end{center}
%
Instead of |\childdocof{|\textit{main}|}| just include the main file
at the top of each child file:
%
\begin{center}
|\input{|\textit{main}|}|
\end{center}
%
A simple redirection |\childdocforward{|\textit{dest}|}| is achieved by:
%
\begin{center}
|\def\jobname{|\textit{dest}|}\input{\jobname}|
\end{center}
%
The redirection with prefix
|\childdocforwardprefix[|\textit{prefix}|]{|\textit{dest}|}|
is accomplished by:
%
\begin{center}
\begin{tabular}{l}
|{\edef\jobname{\scantokens\expandafter{\jobname\noexpand}}|\\
|\def\redirectjob |\textit{prefix}|#1~~~{\gdef\jobname{|\textit{dest}|#1}}|\\
|\expandafter\redirectjob\jobname~~~}\input{\jobname}|
\end{tabular}
\end{center}

In an alternative approach,
child documents can be compiled by a specific command line
without additional code or specific definitions:
%
\begin{center}
|... -jobname "|\textit{target}|" "|[\textit{flags}]%
|\includeonly{|\textit{dest}|}\input{|\textit{main}|}"|
\end{center}
%

%%%%%%%%%%%%%%%%%%%%%%%%%%%%%%%%%%%%%%%%%%%%%%%%%%%%%%%%%%%%%%%%%%%%%%%%%%%%%%%%
%%%%%%%%%%%%%%%%%%%%%%%%%%%%%%%%%%%%%%%%%%%%%%%%%%%%%%%%%%%%%%%%%%%%%%%%%%%%%%%%
\section{Information}

%%%%%%%%%%%%%%%%%%%%%%%%%%%%%%%%%%%%%%%%%%%%%%%%%%%%%%%%%%%%%%%%%%%%%%%%%%%%%%%%
\subsection{Copyright}

Copyright \copyright{} 2017--2018 Niklas Beisert

This work may be distributed and/or modified under the
conditions of the \LaTeX{} Project Public License, either version 1.3
of this license or (at your option) any later version.
The latest version of this license is in
  \url{http://www.latex-project.org/lppl.txt}
and version 1.3 or later is part of all distributions of \LaTeX{}
version 2005/12/01 or later.

This work has the LPPL maintenance status `maintained'.

The Current Maintainer of this work is Niklas Beisert.

This work consists of the files |README.txt|, |childdoc.ins| and |childdoc.dtx|
as well as the derived files |childdoc.def|, |cdocsamp.tex|
with |cdocsch1.tex|, |cdocsch2.tex|, |cdocspt3.tex|, |cdocspt4.tex|,
|cdocsdrf.tex|, |cdocsfn1.tex|, |cdocsfn2.tex|
as well as |childdoc.pdf|.

%%%%%%%%%%%%%%%%%%%%%%%%%%%%%%%%%%%%%%%%%%%%%%%%%%%%%%%%%%%%%%%%%%%%%%%%%%%%%%%%
\subsection{Files and Installation}

The package consists of the files:
%
\begin{center}
\begin{tabular}{ll}
    |README.txt|   & readme file \\
    |childdoc.ins| & installation file \\
    |childdoc.dtx| & source file \\
    |childdoc.def| & definition file \\
    |cdocsamp.tex| & sample main file \\
    |cdocsch1.tex| & sample include file \\
    |cdocsch2.tex| & sample include file \\
    |cdocspt3.tex| & sample part file \\
    |cdocspt4.tex| & sample part file \\
    |cdocsdrf.tex| & sample redirection file \\
    |cdocsfn1.tex| & sample redirection file \\
    |cdocsfn2.tex| & sample redirection file \\
    |childdoc.pdf| & manual
\end{tabular}
\end{center}
%
The distribution consists of the files
|README.txt|, |childdoc.ins| and |childdoc.dtx|.
%
\begin{itemize}
\item
Run (pdf)\LaTeX{} on |childdoc.dtx|
to compile the manual |childdoc.pdf| (this file).
\item
Run \LaTeX{} on |childdoc.ins| to create the definitions file |childdoc.def|
and the sample |cdocsamp.tex| with include files
|cdocsch1.tex|, |cdocsch2.tex|, |cdocspt3.tex|, |cdocspt4.tex|,
|cdocsdrf.tex|, |cdocsfn1.tex|, |cdocsfn2.tex|.
Then copy the file |childdoc.def| to an appropriate directory of your \LaTeX{}
distribution, e.g.\ \textit{texmf-root}|/tex/latex/childdoc|.
\end{itemize}

%%%%%%%%%%%%%%%%%%%%%%%%%%%%%%%%%%%%%%%%%%%%%%%%%%%%%%%%%%%%%%%%%%%%%%%%%%%%%%%%
\subsection{Related CTAN Packages}

There are several other packages which offer a similar functionality:
%
\begin{itemize}
\item
The packages
\href{http://ctan.org/pkg/docmute}{\textsf{docmute}},
\href{http://ctan.org/pkg/includex}{\textsf{includex}} and
\href{http://ctan.org/pkg/standalone}{\textsf{standalone}}
provide commands to include only the document body of
a child file thus allowing both files to be compiled individually.
\item
The packages \href{http://ctan.org/pkg/subdocs}{\textsf{subdocs}}
and \href{http://ctan.org/pkg/subfiles}{\textsf{subfiles}}
provide structures in which the main and child documents can be
encapsulated and allowing them to be compiled individually.
The inclusion mechanism is different from the conventional |\include|.
\item
The package \href{http://ctan.org/pkg/combine}{\textsf{combine}}
is an elaborate solution to combine several documents into one.
\end{itemize}
%
See also the CTAN topic \href{http://ctan.org/topic/subdocs}{\textsf{subdocs}}
for further related packages.
The present package differs from the above solutions in that
a document structure constructed with the conventional |\include| mechanism
just needs two extra commands at the top of every file
such that all constituent files can be compiled individually.

%%%%%%%%%%%%%%%%%%%%%%%%%%%%%%%%%%%%%%%%%%%%%%%%%%%%%%%%%%%%%%%%%%%%%%%%%%%%%%%%
%\subsection{Feature Suggestions}
%
%The following is a list of features which may be useful for future
%versions of this package:
%%
%\begin{itemize}
%\item
%\ldots
%\end{itemize}

%%%%%%%%%%%%%%%%%%%%%%%%%%%%%%%%%%%%%%%%%%%%%%%%%%%%%%%%%%%%%%%%%%%%%%%%%%%%%%%%
\subsection{Revision History}

%%%%%%%%%%%%%%%%%%%%%%%%%%%%%%%%%%%%%%%%
\paragraph{v2.0:} 2018/12/30

\begin{itemize}
\item
immediate forward processing
\item
added |\childdocby| mechanism
\item
manual restructured
\end{itemize}

%%%%%%%%%%%%%%%%%%%%%%%%%%%%%%%%%%%%%%%%
\paragraph{v1.6:} 2018/01/17

\begin{itemize}
\item
application for development of include files
\item
corrections to manual
\end{itemize}

%%%%%%%%%%%%%%%%%%%%%%%%%%%%%%%%%%%%%%%%
\paragraph{v1.5:} 2017/05/21

\begin{itemize}
\item
more complete structuring introduced
\item
|\childdocof| introduced
\item
|\childdoc| renamed to |\childdocmain|
\item
|\childredirect| renamed to |\childdocforward| and |\childdocforwardprefix|
and functionality expanded
\end{itemize}

%%%%%%%%%%%%%%%%%%%%%%%%%%%%%%%%%%%%%%%%
\paragraph{v1.0:} 2017/04/27

\begin{itemize}
\item
manual and install package
\item
first version published on CTAN
\end{itemize}

%%%%%%%%%%%%%%%%%%%%%%%%%%%%%%%%%%%%%%%%
\paragraph{v0.6:} 2017/04/26

\begin{itemize}
\item
redirection mechanism added
\end{itemize}

%%%%%%%%%%%%%%%%%%%%%%%%%%%%%%%%%%%%%%%%
\paragraph{v0.5:} 2017/04/26

\begin{itemize}
\item
functionality in definition file
\end{itemize}


%%%%%%%%%%%%%%%%%%%%%%%%%%%%%%%%%%%%%%%%%%%%%%%%%%%%%%%%%%%%%%%%%%%%%%%%%%%%%%%%
%%%%%%%%%%%%%%%%%%%%%%%%%%%%%%%%%%%%%%%%%%%%%%%%%%%%%%%%%%%%%%%%%%%%%%%%%%%%%%%%
%%%%%%%%%%%%%%%%%%%%%%%%%%%%%%%%%%%%%%%%%%%%%%%%%%%%%%%%%%%%%%%%%%%%%%%%%%%%%%%%
\appendix

\settowidth\MacroIndent{\rmfamily\scriptsize 000\ }

 \DocInput{childdoc.dtx}

\end{document}
%</driver>
% \fi
%
% %%%%%%%%%%%%%%%%%%%%%%%%%%%%%%%%%%%%%%%%%%%%%%%%%%%%%%%%%%%%%%%%%%%%%%%%%%%%%%
% %%%%%%%%%%%%%%%%%%%%%%%%%%%%%%%%%%%%%%%%%%%%%%%%%%%%%%%%%%%%%%%%%%%%%%%%%%%%%%
% \section{Sample}
%\iffalse
%<*samplemain>
%\fi
%
% The following presents a sample document
% with two chapters, two parts, a title page,
% a compile flag as well as three forwarding files to set the flag.
% It consists of eight |.tex| files:
% \begin{center}
% \begin{tabular}{ll}
% |cdocsamp.tex|&main file\\
% |cdocsch1.tex|&include file for chapter 1\\
% |cdocsch2.tex|&include file for chapter 2\\
% |cdocspt3.tex|&include file for part 3\\
% |cdocspt4.tex|&include file for part 4\\
% |cdocsdrf.tex|&forwarding file for main file in draft mode\\
% |cdocsfi1.tex|&forwarding file for final version of chapter 1\\
% |cdocsfi2.tex|&forwarding file for final version of chapter 2\\
% \end{tabular}
% \end{center}
% Each of the eight files can be compiled directly by the \LaTeX{} compiler.
%
% %%%%%%%%%%%%%%%%%%%%%%%%%%%%%%%%%%%%%%
% \paragraph{Main File.}
%
% The main file is called |cdocsamp.tex|.
%
% Load the \textsf{childdoc} definitions and
% declare the filename for the main document:
%    \begin{macrocode}
\input{childdoc.def}
\childdocmain{}
%    \end{macrocode}

% Optional override for |\version| flag:
%    \begin{macrocode}
%%\ifchilddoc\else\providecommand{\version}{draft}\fi
%    \end{macrocode}

% Define the default values for the |\version| flag
% (|final| for the main file and |draft| for childs):
%    \begin{macrocode}
\ifchilddoc
\providecommand{\version}{draft}
\else
\providecommand{\version}{final}
\fi
%    \end{macrocode}

% Load the standard document class:
%    \begin{macrocode}
\documentclass[12pt]{article}
%    \end{macrocode}

% Start the document body:
%    \begin{macrocode}
\begin{document}
%    \end{macrocode}

% Declare a title page.
% Print title, part of document being processed and version flag:
%    \begin{macrocode}
\addtocounter{page}{-1}
\begin{center}
{\LARGE\bfseries{}childdoc example\par}
\vspace{1cm}
\ifchilddoc
\ifchilddocmanual part\else chapter\fi:
`\childdocname' of `\childdocjob'\par
\else
main document: `\childdocjob'\par
\fi
version: \version\par
\end{center}
\newpage
%    \end{macrocode}

% Manually include selected file,
% otherwise process as usual:
%    \begin{macrocode}
\ifchilddocmanual
\section*{part `\childdocname'}
\input{\childdocname}
\else
%    \end{macrocode}

% Include the two chapters:
%    \begin{macrocode}
\include{cdocsch1}
\include{cdocsch2}
%    \end{macrocode}

% Include the two parts unless only chapters should be displayed:
%    \begin{macrocode}
\ifchilddoc\else
\section{part three}
\input{cdocspt3}
\section{part four}
\input{cdocspt4}
\fi
%    \end{macrocode}

% Process as usual until here:
%    \begin{macrocode}
\fi
%    \end{macrocode}

% End of document body:
%    \begin{macrocode}
\end{document}
%    \end{macrocode}
%\iffalse
%</samplemain>
%\fi
%
% %%%%%%%%%%%%%%%%%%%%%%%%%%%%%%%%%%%%%%
% \paragraph{Chapter Include Files.}
%
% The include files are called |cdocsch1.tex| and |cdocsch2.tex|.
%
%\iffalse
%<*samplechap1|samplechap2>
%\fi

% Optional override for |\version| flag:
%    \begin{macrocode}
%%\providecommand{\version}{final}
%    \end{macrocode}

% Include the main document:
%    \begin{macrocode}
\input{childdoc.def}
\childdocof{cdocsamp}
%    \end{macrocode}

%\iffalse
%</samplechap1|samplechap2>
%\fi
%
%\iffalse
%<*samplechap1>
%\fi
% Some text for chapter 1:
%    \begin{macrocode}
\section{one}
some text in chapter one
%    \end{macrocode}

%\iffalse
%</samplechap1>
%\fi
% Some text for chapter 2:
%\iffalse
%<*samplechap2>
%\fi
%    \begin{macrocode}
\section{two}
more text in chapter two
%    \end{macrocode}

%\iffalse
%</samplechap2>
%\fi
%
% %%%%%%%%%%%%%%%%%%%%%%%%%%%%%%%%%%%%%%
% \paragraph{Part Include Files.}
%
% The include files are called |cdocspt3.tex| and |cdocspt4.tex|.
%
%\iffalse
%<*samplepart3|samplepart4>
%\fi

% Optional override for |\version| flag:
%    \begin{macrocode}
%%\providecommand{\version}{final}
%    \end{macrocode}

% Include the main document:
%    \begin{macrocode}
\input{childdoc.def}
\childdocby{cdocsamp}
%    \end{macrocode}

%\iffalse
%</samplepart3|samplepart4>
%\fi
%
%\iffalse
%<*samplepart3>
%\fi
% Some text for part 3:
%    \begin{macrocode}
some text in part three
%    \end{macrocode}

%\iffalse
%</samplepart3>
%\fi
% Some text for part 4:
%\iffalse
%<*samplepart4>
%\fi
%    \begin{macrocode}
more text in part four
%    \end{macrocode}

%\iffalse
%</samplepart4>
%\fi
%
% %%%%%%%%%%%%%%%%%%%%%%%%%%%%%%%%%%%%%%
% \paragraph{Forwarding for a Complete Draft.}
%
% The following forwarding file |cdocsdrf.tex|
% compiles the main document in draft mode:
%\iffalse
%<*sampledraft>
%\fi
%    \begin{macrocode}
\def\version{draft}
\input{childdoc.def}
\childdocforward{cdocsamp}
%    \end{macrocode}

%\iffalse
%</sampledraft>
%\fi
%
% %%%%%%%%%%%%%%%%%%%%%%%%%%%%%%%%%%%%%%
% \paragraph{Forwarding for Final Version of the Chapters.}
%
% The following forwarding files |cdocsfn1.tex| and |cdocsfn2.tex|
% (with identical content)
% compile the final versions of the child documents
% |cdocsch1.tex| and |cdocsch2.tex|, respectively:
%\iffalse
%<*samplefinal>
%\fi
%    \begin{macrocode}
\def\version{final}
\input{childdoc.def}
\childdocforwardprefix[cdocsamp]{cdocsfn}{cdocsch}
%    \end{macrocode}

%\iffalse
%</samplefinal>
%\fi
%
% %%%%%%%%%%%%%%%%%%%%%%%%%%%%%%%%%%%%%%
% \paragraph{Command Line Processing.}
%
% The following three command lines generate the output files
% |cdocscld|, |cdocscl1| and |cdocscl2|
% which should be identical to
% |cdocsdrf|, |cdocsch1| and |cdocsfn2|, respectively:
% \begin{center}
% \begin{tabular}{l}
% |latex -jobname cdocscld \|\\
% |  "\def\version{draft}\input{childdoc.def}\childdocforward{cdocsamp}"|\\
% |latex -jobname cdocscl1 \|\\
% |  "\input{childdoc.def}\childdocforward[cdocsamp]{cdocsch1}"|\\
% |latex -jobname cdocscl2 \|\\
% |  "\def\version{final}\input{childdoc.def}\childdocforward{cdocsch2}"|
% \end{tabular}
% \end{center}
% Note that the trailing backslash on each first line
% merely continues the input to the second line
% (for convenient cut ant paste).
% Furthermore, the command |latex| can be replaced by any
% of its alternative versions such as |pdflatex|.
%
% %%%%%%%%%%%%%%%%%%%%%%%%%%%%%%%%%%%%%%%%%%%%%%%%%%%%%%%%%%%%%%%%%%%%%%%%%%%%%%
% %%%%%%%%%%%%%%%%%%%%%%%%%%%%%%%%%%%%%%%%%%%%%%%%%%%%%%%%%%%%%%%%%%%%%%%%%%%%%%
% \section{Implementation}
%\iffalse
%<*package>
%\fi
%
% This section describes the definitions file |childdoc.def|.

% The definitions cannot be loaded using |\usepackage| or |\RequirePackage|
% which has a mechanism to prevent loading a style file more than once.
% When loading the definitions by means of |\input|
% multiple instances have to be prevented manually:
%\iffalse
%This code needs to be before the `\ProvidesFile' directive
%which is defined at the beginning of this file.
%Therefore it is also placed there and commented out here.
%</package>
%<*discard>
%\fi
%    \begin{macrocode}
\ifdefined\childdocmain\endinput\fi
%    \end{macrocode}
%\iffalse
%</discard>
%<*package>
%\fi
%
% \macro{\ifchilddoc}
% \macro{\ifchilddocmanual}
% The conditional |\ifchilddoc| tells whether a
% child (true) or main (false) document is being compiled.
% The conditional |\ifchilddocmanual| tells whether
% the |\includeonly| mechanism is used (false) or
% the selection of child files must be performed manually (true).
% The definitions initialise to false:
%    \begin{macrocode}
\newif\ifchilddoc
\newif\ifchilddocmanual
%    \end{macrocode}

% \macro{\childdocname}
% \macro{\childdocjob}
% The macro |\childdocname| stores the name of the main document
% to be compiled. The macro |\childdocjob| stores the name of
% the document on which the \LaTeX{} compiler was originally invoked.
% The content of |\jobname| cannot be compared
% to filenames specified in the source due to different catcodes.
% The following code rescans |\jobname|, stores the result
% in |\childdocname| and saves a copy in |\childdocjob|:
%    \begin{macrocode}
\edef\childdocname{\scantokens\expandafter{\jobname\noexpand}}
\let\childdocjob\childdocname
%    \end{macrocode}

% \macro{\childdocdisable}
% The macro |\childdocdisable| prevents the main file
% from being processed more than once.
% At this stage, the main document command |\childdocmain|
% is assumed to be called once again where it should do nothing.
% Any subsequent call to it should prevent
% a secondary processing of the main document
% It overwrites the forwarding commands
% |\childdocof| and |\childdocforward|
% with empty macros to prevent further inclusions of the main document:
%    \begin{macrocode}
\newcommand{\childdocdisable}
{
  \renewcommand{\childdocmain}[1]{\renewcommand{\childdocmain}[1]{\endinput}}
  \renewcommand{\childdocof}[1]{}
  \renewcommand{\childdocby}[2][]{}
  \renewcommand{\childdocforward}[2][]{}
  \renewcommand{\childdocdisable}{}
}
%    \end{macrocode}

% \macro{\childdocmain}
% The macro |\childdocmain| is to be called at the top of the main file
% with nothing or the main filename (without extension) as argument.
% First, it breaks loops.
% If the argument is not empty and does not match |\childdocname|
% (which is set by the first inclusion of |childdoc.def|),
% |\ifchilddoc| is set to true, |\includeonly| is applied to the child file
% and |\jobname| is set to the main file
% (for proper handling of |.aux| files):
%    \begin{macrocode}
\newcommand{\childdocmain}[1]
{
  \childdocdisable\childdocmain{}
  \if?#1?\else
    \begingroup
      \def\childdoctmp{#1}
      \ifx\childdoctmp\childdocname
        \def\childdoctmp{}
      \else
        \def\childdoctmp
        {
          \childdoctrue
          \includeonly{\childdocname}
          \def\childdocjob{#1}
          \def\jobname{#1}
        }
      \fi
      \expandafter
    \endgroup
    \childdoctmp
  \fi
}
%    \end{macrocode}

% \macro{\childdocof}
% The command |\childdocof| redirects
% compilation to the main file |#1|.
%    \begin{macrocode}
\newcommand{\childdocof}[1]
{
  \childdocdisable
  \childdoctrue
  \includeonly{\childdocname}
  \def\jobname{#1}
  \def\childdocjob{#1}
  \input{#1}
}
%    \end{macrocode}

% \macro{\childdocby}
% The command |\childdocby| ....
%    \begin{macrocode}
\newcommand{\childdocby}[2][]
{
  \childdocdisable
  \childdoctrue
  \childdocmanualtrue
  \if?#1?\else
    \def\jobname{#2}
  \fi
  \def\childdocjob{#2}
  \input{#2}
  \endinput
}
%    \end{macrocode}

% \macro{\childdocforward}
% The command |\childdocforward| redirects
% compilation to the main file or
% (if the optional argument is given) a child file.
% Parameters are set as if the main file
% or a child file starting with |\childdocof| was compiled.
% Then compilation is handed over to the main file:
%    \begin{macrocode}
\newcommand{\childdocforward}[2][]
{
  \begingroup
    \if?#1?
      \def\childdoctmp
      {
        \def\childdocname{#2}
        \def\childdocjob{#2}
        \def\jobname{#2}
        \input{#2}
        \endinput
      }
    \else
      \def\childdoctmp
      {
        \childdocdisable
        \def\childdocname{#2}
        \childdoctrue
        \includeonly{#2}
        \def\childdocjob{#1}
        \def\jobname{#1}
        \input{#1}
        \endinput
      }
    \fi
    \expandafter
  \endgroup
  \childdoctmp
}
%    \end{macrocode}

% \macro{\childdocforwardprefix}
% The command |\childdocforwardprefix| redirects
% compilation to the main or a child file by means of a pattern.
% The prefix |#1| in the current filename is replaced by |#2|
% and the suffix of the current filename is kept
% (it is assumed that the filename does not contain the substring `|~~~|'
% which is used as a delimiter).
% Compilation is handed over to the new file by |\childdocforward|:
%    \begin{macrocode}
\newcommand{\childdocforwardprefix}[3][]
{
  \begingroup
    \def\childdocextract #2##1~~~{\def\childdoctmp{\childdocforward[#1]{#3##1}}}
    \expandafter\childdocextract\childdocname~~~
    \expandafter
  \endgroup
  \childdoctmp
}
%    \end{macrocode}

% \macro{\childdoc}
% The deprecated macro |\childdoc| is a legacy version of |\childdocmain|:
%    \begin{macrocode}
\newcommand{\childdoc}{\childdocmain}
%    \end{macrocode}

% \macro{\childdocredirect}
% The deprecated macro |\childdocredirect| is a legacy version
% of |\childdocforward| and |\childdocforwardprefix|:
%    \begin{macrocode}
\newcommand{\childdocredirect}[2][]
{
  \begingroup
    \if?#1?
      \def\childdoctmp{\childdocforward{#2}}
    \else
      \def\childdoctmp{\childdocforwardprefix{#1}{#2}}
    \fi
    \expandafter
  \endgroup
  \childdoctmp
}
%    \end{macrocode}

%\iffalse
%</package>
%\fi
%
\endinput
\childdocforward{cdocsch2}"|
% \end{tabular}
% \end{center}
% Note that the trailing backslash on each first line
% merely continues the input to the second line
% (for convenient cut ant paste).
% Furthermore, the command |latex| can be replaced by any
% of its alternative versions such as |pdflatex|.
%
% %%%%%%%%%%%%%%%%%%%%%%%%%%%%%%%%%%%%%%%%%%%%%%%%%%%%%%%%%%%%%%%%%%%%%%%%%%%%%%
% %%%%%%%%%%%%%%%%%%%%%%%%%%%%%%%%%%%%%%%%%%%%%%%%%%%%%%%%%%%%%%%%%%%%%%%%%%%%%%
% \section{Implementation}
%\iffalse
%<*package>
%\fi
%
% This section describes the definitions file |childdoc.def|.

% The definitions cannot be loaded using |\usepackage| or |\RequirePackage|
% which has a mechanism to prevent loading a style file more than once.
% When loading the definitions by means of |\input|
% multiple instances have to be prevented manually:
%\iffalse
%This code needs to be before the `\ProvidesFile' directive
%which is defined at the beginning of this file.
%Therefore it is also placed there and commented out here.
%</package>
%<*discard>
%\fi
%    \begin{macrocode}
\ifdefined\childdocmain\endinput\fi
%    \end{macrocode}
%\iffalse
%</discard>
%<*package>
%\fi
%
% \macro{\ifchilddoc}
% \macro{\ifchilddocmanual}
% The conditional |\ifchilddoc| tells whether a
% child (true) or main (false) document is being compiled.
% The conditional |\ifchilddocmanual| tells whether
% the |\includeonly| mechanism is used (false) or
% the selection of child files must be performed manually (true).
% The definitions initialise to false:
%    \begin{macrocode}
\newif\ifchilddoc
\newif\ifchilddocmanual
%    \end{macrocode}

% \macro{\childdocname}
% \macro{\childdocjob}
% The macro |\childdocname| stores the name of the main document
% to be compiled. The macro |\childdocjob| stores the name of
% the document on which the \LaTeX{} compiler was originally invoked.
% The content of |\jobname| cannot be compared
% to filenames specified in the source due to different catcodes.
% The following code rescans |\jobname|, stores the result
% in |\childdocname| and saves a copy in |\childdocjob|:
%    \begin{macrocode}
\edef\childdocname{\scantokens\expandafter{\jobname\noexpand}}
\let\childdocjob\childdocname
%    \end{macrocode}

% \macro{\childdocdisable}
% The macro |\childdocdisable| prevents the main file
% from being processed more than once.
% At this stage, the main document command |\childdocmain|
% is assumed to be called once again where it should do nothing.
% Any subsequent call to it should prevent
% a secondary processing of the main document
% It overwrites the forwarding commands
% |\childdocof| and |\childdocforward|
% with empty macros to prevent further inclusions of the main document:
%    \begin{macrocode}
\newcommand{\childdocdisable}
{
  \renewcommand{\childdocmain}[1]{\renewcommand{\childdocmain}[1]{\endinput}}
  \renewcommand{\childdocof}[1]{}
  \renewcommand{\childdocby}[2][]{}
  \renewcommand{\childdocforward}[2][]{}
  \renewcommand{\childdocdisable}{}
}
%    \end{macrocode}

% \macro{\childdocmain}
% The macro |\childdocmain| is to be called at the top of the main file
% with nothing or the main filename (without extension) as argument.
% First, it breaks loops.
% If the argument is not empty and does not match |\childdocname|
% (which is set by the first inclusion of |childdoc.def|),
% |\ifchilddoc| is set to true, |\includeonly| is applied to the child file
% and |\jobname| is set to the main file
% (for proper handling of |.aux| files):
%    \begin{macrocode}
\newcommand{\childdocmain}[1]
{
  \childdocdisable\childdocmain{}
  \if?#1?\else
    \begingroup
      \def\childdoctmp{#1}
      \ifx\childdoctmp\childdocname
        \def\childdoctmp{}
      \else
        \def\childdoctmp
        {
          \childdoctrue
          \includeonly{\childdocname}
          \def\childdocjob{#1}
          \def\jobname{#1}
        }
      \fi
      \expandafter
    \endgroup
    \childdoctmp
  \fi
}
%    \end{macrocode}

% \macro{\childdocof}
% The command |\childdocof| redirects
% compilation to the main file |#1|.
%    \begin{macrocode}
\newcommand{\childdocof}[1]
{
  \childdocdisable
  \childdoctrue
  \includeonly{\childdocname}
  \def\jobname{#1}
  \def\childdocjob{#1}
  \input{#1}
}
%    \end{macrocode}

% \macro{\childdocby}
% The command |\childdocby| ....
%    \begin{macrocode}
\newcommand{\childdocby}[2][]
{
  \childdocdisable
  \childdoctrue
  \childdocmanualtrue
  \if?#1?\else
    \def\jobname{#2}
  \fi
  \def\childdocjob{#2}
  \input{#2}
  \endinput
}
%    \end{macrocode}

% \macro{\childdocforward}
% The command |\childdocforward| redirects
% compilation to the main file or
% (if the optional argument is given) a child file.
% Parameters are set as if the main file
% or a child file starting with |\childdocof| was compiled.
% Then compilation is handed over to the main file:
%    \begin{macrocode}
\newcommand{\childdocforward}[2][]
{
  \begingroup
    \if?#1?
      \def\childdoctmp
      {
        \def\childdocname{#2}
        \def\childdocjob{#2}
        \def\jobname{#2}
        \input{#2}
        \endinput
      }
    \else
      \def\childdoctmp
      {
        \childdocdisable
        \def\childdocname{#2}
        \childdoctrue
        \includeonly{#2}
        \def\childdocjob{#1}
        \def\jobname{#1}
        \input{#1}
        \endinput
      }
    \fi
    \expandafter
  \endgroup
  \childdoctmp
}
%    \end{macrocode}

% \macro{\childdocforwardprefix}
% The command |\childdocforwardprefix| redirects
% compilation to the main or a child file by means of a pattern.
% The prefix |#1| in the current filename is replaced by |#2|
% and the suffix of the current filename is kept
% (it is assumed that the filename does not contain the substring `|~~~|'
% which is used as a delimiter).
% Compilation is handed over to the new file by |\childdocforward|:
%    \begin{macrocode}
\newcommand{\childdocforwardprefix}[3][]
{
  \begingroup
    \def\childdocextract #2##1~~~{\def\childdoctmp{\childdocforward[#1]{#3##1}}}
    \expandafter\childdocextract\childdocname~~~
    \expandafter
  \endgroup
  \childdoctmp
}
%    \end{macrocode}

% \macro{\childdoc}
% The deprecated macro |\childdoc| is a legacy version of |\childdocmain|:
%    \begin{macrocode}
\newcommand{\childdoc}{\childdocmain}
%    \end{macrocode}

% \macro{\childdocredirect}
% The deprecated macro |\childdocredirect| is a legacy version
% of |\childdocforward| and |\childdocforwardprefix|:
%    \begin{macrocode}
\newcommand{\childdocredirect}[2][]
{
  \begingroup
    \if?#1?
      \def\childdoctmp{\childdocforward{#2}}
    \else
      \def\childdoctmp{\childdocforwardprefix{#1}{#2}}
    \fi
    \expandafter
  \endgroup
  \childdoctmp
}
%    \end{macrocode}

%\iffalse
%</package>
%\fi
%
\endinput
|\\
|\childdocby{|\textit{main}|}|\\
\end{tabular}
\end{center}
%
Both forms have slightly different effects as described above.
The main file is prepared as usual, see \secref{sec:include}.

%%%%%%%%%%%%%%%%%%%%%%%%%%%%%%%%%%%%%%%%%%%%%%%%%%%%%%%%%%%%%%%%%%%%%%%%%%%%%%%%
\subsection{Legacy Detection}
\label{sec:detection}

The directive |\childdocmain| in the main file can detect
whether the complete document or merely a child is to be compiled
even without using the directive |\childdocof|.
This method is deprecated because it is less robust
and there is no compelling reason to use it;
it is merely provided for backward compatibility
and it may be removed in future versions.

If the detection mechanism is to be used,
it is mandatory to correctly specify
the filename of the main file as the argument of |\childdocmain|:
%
\begin{center}
\begin{tabular}{l}
|% \iffalse
%
% childdoc.dtx Copyright (C) 2017-2018 Niklas Beisert
%
% This work may be distributed and/or modified under the
% conditions of the LaTeX Project Public License, either version 1.3
% of this license or (at your option) any later version.
% The latest version of this license is in
%   http://www.latex-project.org/lppl.txt
% and version 1.3 or later is part of all distributions of LaTeX
% version 2005/12/01 or later.
%
% This work has the LPPL maintenance status `maintained'.
%
% The Current Maintainer of this work is Niklas Beisert.
%
% This work consists of the files childdoc.dtx and childdoc.ins
% and the derived files childdoc.def and cdocsamp.tex with
% cdocsch1.tex, cdocsch2.tex, cdocsdrf.tex, cdocsfn1.tex, cdocsfn2.tex.
%
%<package>\ifdefined\childdocmain\endinput\fi
%<package>\ProvidesFile{childdoc.def}[2018/12/30 v2.0 child document driver]
%<samplemain>\ProvidesFile{cdocsamp.tex}[2018/12/30 v2.0 sample for childdoc]
%<*driver>
%\ProvidesFile{childdoc.drv}[2018/12/30 v2.0 childdoc reference manual file]
\PassOptionsToClass{10pt,a4paper}{article}
\documentclass{ltxdoc}

\usepackage[margin=35mm]{geometry}
\usepackage{hyperref}
\usepackage{hyperxmp}
\usepackage[usenames]{color}

\hypersetup{colorlinks=true}
\hypersetup{pdfstartview=FitH}
\hypersetup{pdfpagemode=UseNone}
\hypersetup{pdfsource={}}
\hypersetup{pdflang={en-UK}}
\hypersetup{pdfcopyright={Copyright 2017-2018 Niklas Beisert.
  This work may be distributed and/or modified under the
  conditions of the LaTeX Project Public License, either version 1.3
  of this license or (at your option) any later version.}}
\hypersetup{pdflicenseurl={http://www.latex-project.org/lppl.txt}}
\hypersetup{pdfcontactaddress={ETH Zurich, ITP, HIT K,
  Wolfgang-Pauli-Strasse 27}}
\hypersetup{pdfcontactpostcode={8093}}
\hypersetup{pdfcontactcity={Zurich}}
\hypersetup{pdfcontactcountry={Switzerland}}
\hypersetup{pdfcontactemail={nbeisert@itp.phys.ethz.ch}}
\hypersetup{pdfcontacturl={http://people.phys.ethz.ch/\xmptilde nbeisert/}}

\newcommand{\secref}[1]{\hyperref[#1]{section \ref*{#1}}}

\parskip1ex
\parindent0pt
\let\olditemize\itemize
\def\itemize{\olditemize\parskip0pt}

\begin{document}

\title{The \textsf{childdoc} Package}
\hypersetup{pdftitle={The childdoc Package}}
\author{Niklas Beisert\\[2ex]
  Institut f\"ur Theoretische Physik\\
  Eidgen\"ossische Technische Hochschule Z\"urich\\
  Wolfgang-Pauli-Strasse 27, 8093 Z\"urich, Switzerland\\[1ex]
  \href{mailto:nbeisert@itp.phys.ethz.ch}
  {\texttt{nbeisert@itp.phys.ethz.ch}}}
\hypersetup{pdfauthor={Niklas Beisert}}
\hypersetup{pdfsubject={Manual for the LaTeX2e Package childdoc}}
\date{30 December 2018, \textsf{v2.0}}
\maketitle

\begin{abstract}\noindent
\textsf{childdoc} is a \LaTeXe{} package
that enables the direct compilation
of document sections included by |\include|
to individual files.
\end{abstract}

\begingroup
\parskip0ex
\tableofcontents
\endgroup

%%%%%%%%%%%%%%%%%%%%%%%%%%%%%%%%%%%%%%%%%%%%%%%%%%%%%%%%%%%%%%%%%%%%%%%%%%%%%%%%
%%%%%%%%%%%%%%%%%%%%%%%%%%%%%%%%%%%%%%%%%%%%%%%%%%%%%%%%%%%%%%%%%%%%%%%%%%%%%%%%
\section{Introduction}

\LaTeX{} provides a mechanism to structure a large document (such as a book)
into a main file and several child files (containing the chapters)
using the |\include| command.
This mechanism is beneficial for documents
which span hundreds of pages in order to
make the source file(s) more manageable.
Moreover, compilation can be restricted to
selected child files by means of the |\includeonly| command.
The latter feature can be used to reduce the compilation time while editing
(this was significantly more useful in the earlier days of \LaTeX{})
or to generate a smaller document which is easier to navigate.
Another application of |\includeonly| is to generate
documents consisting of selected parts of the complete document.

However, there are a few drawbacks of the plain |\include| mechanism:
\begin{itemize}
\item
The child files cannot be compiled on their own,
they can only be compiled via the main file.
A naive editing environment
(such as a text editor with an option
to have the current file processed by \LaTeX)
may require one to switch to the main file before compiling;
attempting to compile the child file produces errors.
\item
The main file must be modified (each time)
to adjust the |\includeonly| command
to the present needs. This easily leaves the main file in a messy state.
\item
The generated document will always carry the filename
of the main document. This is inconvenient if
several child files are to be compiled and
to be kept for distribution.
\end{itemize}

The present package provides a simple interface
to make child files individually compilable by \LaTeX{}.
Compiling a child file then has the same effect as compiling
the main file with an |\includeonly| command
to select the appropriate child.
Moreover the generated document will carry the name of the child
rather than the main file.
This resolves all three above issues.

This feature is meant to make the editing of books,
thesis documents and lecture notes somewhat more convenient.
However, the package can also be used efficiently for
composing a series of documents (such as exercise sheets)
which are typically distributed individually.
It then assists the author in generating the individual documents
(potentially in different versions)
as well as a document containing the collected series.
Another application is in developing style files
or other kinds of included material
where compilation of the style file could redirect
to a sample or test file.

%%%%%%%%%%%%%%%%%%%%%%%%%%%%%%%%%%%%%%%%%%%%%%%%%%%%%%%%%%%%%%%%%%%%%%%%%%%%%%%%
%%%%%%%%%%%%%%%%%%%%%%%%%%%%%%%%%%%%%%%%%%%%%%%%%%%%%%%%%%%%%%%%%%%%%%%%%%%%%%%%
\section{Usage}

First of all, the package \textsf{childdoc} is \emph{not} a standard
\LaTeXe{} |.sty| style file! Therefore it needs to be invoked in
a non-standard way.

%%%%%%%%%%%%%%%%%%%%%%%%%%%%%%%%%%%%%%%%%%%%%%%%%%%%%%%%%%%%%%%%%%%%%%%%%%%%%%%%
\subsection{Included Files}
\label{sec:include}

%%%%%%%%%%%%%%%%%%%%%%%%%%%%%%%%%%%%%%%%
\DescribeMacro{\childdocmain}
To use the package, add the commands
\begin{center}
\begin{tabular}{l}
|% \iffalse
%
% childdoc.dtx Copyright (C) 2017-2018 Niklas Beisert
%
% This work may be distributed and/or modified under the
% conditions of the LaTeX Project Public License, either version 1.3
% of this license or (at your option) any later version.
% The latest version of this license is in
%   http://www.latex-project.org/lppl.txt
% and version 1.3 or later is part of all distributions of LaTeX
% version 2005/12/01 or later.
%
% This work has the LPPL maintenance status `maintained'.
%
% The Current Maintainer of this work is Niklas Beisert.
%
% This work consists of the files childdoc.dtx and childdoc.ins
% and the derived files childdoc.def and cdocsamp.tex with
% cdocsch1.tex, cdocsch2.tex, cdocsdrf.tex, cdocsfn1.tex, cdocsfn2.tex.
%
%<package>\ifdefined\childdocmain\endinput\fi
%<package>\ProvidesFile{childdoc.def}[2018/12/30 v2.0 child document driver]
%<samplemain>\ProvidesFile{cdocsamp.tex}[2018/12/30 v2.0 sample for childdoc]
%<*driver>
%\ProvidesFile{childdoc.drv}[2018/12/30 v2.0 childdoc reference manual file]
\PassOptionsToClass{10pt,a4paper}{article}
\documentclass{ltxdoc}

\usepackage[margin=35mm]{geometry}
\usepackage{hyperref}
\usepackage{hyperxmp}
\usepackage[usenames]{color}

\hypersetup{colorlinks=true}
\hypersetup{pdfstartview=FitH}
\hypersetup{pdfpagemode=UseNone}
\hypersetup{pdfsource={}}
\hypersetup{pdflang={en-UK}}
\hypersetup{pdfcopyright={Copyright 2017-2018 Niklas Beisert.
  This work may be distributed and/or modified under the
  conditions of the LaTeX Project Public License, either version 1.3
  of this license or (at your option) any later version.}}
\hypersetup{pdflicenseurl={http://www.latex-project.org/lppl.txt}}
\hypersetup{pdfcontactaddress={ETH Zurich, ITP, HIT K,
  Wolfgang-Pauli-Strasse 27}}
\hypersetup{pdfcontactpostcode={8093}}
\hypersetup{pdfcontactcity={Zurich}}
\hypersetup{pdfcontactcountry={Switzerland}}
\hypersetup{pdfcontactemail={nbeisert@itp.phys.ethz.ch}}
\hypersetup{pdfcontacturl={http://people.phys.ethz.ch/\xmptilde nbeisert/}}

\newcommand{\secref}[1]{\hyperref[#1]{section \ref*{#1}}}

\parskip1ex
\parindent0pt
\let\olditemize\itemize
\def\itemize{\olditemize\parskip0pt}

\begin{document}

\title{The \textsf{childdoc} Package}
\hypersetup{pdftitle={The childdoc Package}}
\author{Niklas Beisert\\[2ex]
  Institut f\"ur Theoretische Physik\\
  Eidgen\"ossische Technische Hochschule Z\"urich\\
  Wolfgang-Pauli-Strasse 27, 8093 Z\"urich, Switzerland\\[1ex]
  \href{mailto:nbeisert@itp.phys.ethz.ch}
  {\texttt{nbeisert@itp.phys.ethz.ch}}}
\hypersetup{pdfauthor={Niklas Beisert}}
\hypersetup{pdfsubject={Manual for the LaTeX2e Package childdoc}}
\date{30 December 2018, \textsf{v2.0}}
\maketitle

\begin{abstract}\noindent
\textsf{childdoc} is a \LaTeXe{} package
that enables the direct compilation
of document sections included by |\include|
to individual files.
\end{abstract}

\begingroup
\parskip0ex
\tableofcontents
\endgroup

%%%%%%%%%%%%%%%%%%%%%%%%%%%%%%%%%%%%%%%%%%%%%%%%%%%%%%%%%%%%%%%%%%%%%%%%%%%%%%%%
%%%%%%%%%%%%%%%%%%%%%%%%%%%%%%%%%%%%%%%%%%%%%%%%%%%%%%%%%%%%%%%%%%%%%%%%%%%%%%%%
\section{Introduction}

\LaTeX{} provides a mechanism to structure a large document (such as a book)
into a main file and several child files (containing the chapters)
using the |\include| command.
This mechanism is beneficial for documents
which span hundreds of pages in order to
make the source file(s) more manageable.
Moreover, compilation can be restricted to
selected child files by means of the |\includeonly| command.
The latter feature can be used to reduce the compilation time while editing
(this was significantly more useful in the earlier days of \LaTeX{})
or to generate a smaller document which is easier to navigate.
Another application of |\includeonly| is to generate
documents consisting of selected parts of the complete document.

However, there are a few drawbacks of the plain |\include| mechanism:
\begin{itemize}
\item
The child files cannot be compiled on their own,
they can only be compiled via the main file.
A naive editing environment
(such as a text editor with an option
to have the current file processed by \LaTeX)
may require one to switch to the main file before compiling;
attempting to compile the child file produces errors.
\item
The main file must be modified (each time)
to adjust the |\includeonly| command
to the present needs. This easily leaves the main file in a messy state.
\item
The generated document will always carry the filename
of the main document. This is inconvenient if
several child files are to be compiled and
to be kept for distribution.
\end{itemize}

The present package provides a simple interface
to make child files individually compilable by \LaTeX{}.
Compiling a child file then has the same effect as compiling
the main file with an |\includeonly| command
to select the appropriate child.
Moreover the generated document will carry the name of the child
rather than the main file.
This resolves all three above issues.

This feature is meant to make the editing of books,
thesis documents and lecture notes somewhat more convenient.
However, the package can also be used efficiently for
composing a series of documents (such as exercise sheets)
which are typically distributed individually.
It then assists the author in generating the individual documents
(potentially in different versions)
as well as a document containing the collected series.
Another application is in developing style files
or other kinds of included material
where compilation of the style file could redirect
to a sample or test file.

%%%%%%%%%%%%%%%%%%%%%%%%%%%%%%%%%%%%%%%%%%%%%%%%%%%%%%%%%%%%%%%%%%%%%%%%%%%%%%%%
%%%%%%%%%%%%%%%%%%%%%%%%%%%%%%%%%%%%%%%%%%%%%%%%%%%%%%%%%%%%%%%%%%%%%%%%%%%%%%%%
\section{Usage}

First of all, the package \textsf{childdoc} is \emph{not} a standard
\LaTeXe{} |.sty| style file! Therefore it needs to be invoked in
a non-standard way.

%%%%%%%%%%%%%%%%%%%%%%%%%%%%%%%%%%%%%%%%%%%%%%%%%%%%%%%%%%%%%%%%%%%%%%%%%%%%%%%%
\subsection{Included Files}
\label{sec:include}

%%%%%%%%%%%%%%%%%%%%%%%%%%%%%%%%%%%%%%%%
\DescribeMacro{\childdocmain}
To use the package, add the commands
\begin{center}
\begin{tabular}{l}
|\input{childdoc.def}|\\
|\childdocmain{}|\\
\end{tabular}
\end{center}
at the very top of the main \LaTeX{} file,
in particular \emph{before} the |\documentclass| statement!
The argument of |\childdocmain| should be left empty
(but it must be present).

%%%%%%%%%%%%%%%%%%%%%%%%%%%%%%%%%%%%%%%%
\DescribeMacro{\childdocof}
Furthermore, add the commands
\begin{center}
\begin{tabular}{l}
|\input{childdoc.def}|\\
|\childdocof{|\textit{main}|}|\\
\end{tabular}
\end{center}
at the top of every child file \textit{child}
which is included by |\include{|\textit{child}|}|
from within the main file
(or at least for those files to be compiled individually).
The argument \textit{main} must be the filename of the main file.

There are a couple of
considerations in setting up the main and child documents:

%%%%%%%%%%%%%%%%%%%%%%%%%%%%%%%%%%%%%%%%
\paragraph{Restrictions.}

Please note the following restrictions:
\begin{itemize}
\item
|\childdocmain| must be called with one argument \textit{main}
to ensure compatibility with earlier version of the package.
It must either be empty (|\childdocmain{}|)
or precisely match the filename of the main file in which it is specified.
See \secref{sec:detection} for further information.
\item
The filename \textit{main} must be specified without the |.tex| extension.
\item
The filename \textit{main} is case sensitive
(even in case-insensitive file systems)
due to internal string comparison.
\item
The argument \textit{main} should be fully expanded, it cannot be a macro.
\item
Subdirectories and special characters should be avoided in filenames.
\item
The command |\childdocmain{|\textit{main}|}| must be followed by a whitespace.
It should not be followed immediately by another command
or by a comment mark `|%|'.
This is because the \TeX{} parser reads the token immediately following
the argument of |\childdocmain| and puts it
at the beginning of every child section;
however, a white\-space is ignored.
\end{itemize}

%%%%%%%%%%%%%%%%%%%%%%%%%%%%%%%%%%%%%%%%
\paragraph{Content of Main File.}

It is advisable to place all content in the child files included by |\include|.
Any output contained in the main file will appear in all child documents
unless suppressed manually;
it cannot be suppressed automatically by the |\includeonly| directive
and thus should normally be avoided.
A method to include some content in the main file
by means of conditional processing is described in \secref{sec:conditional}.

%%%%%%%%%%%%%%%%%%%%%%%%%%%%%%%%%%%%%%%%
\paragraph{Page Numbering.}

When only a part of the document is compiled,
the appropriate numbering of pages
(as well as other status parameters)
is determined from the |.aux| files.
The latter contain information from previous passes.
However this information needs to propagate through
all intermediate child documents.
Therefore the page numbering in child documents may well
be inconsistent until the complete document is compiled at least once.

A useful (if unconventional) way to always ensure a consistent
page numbering is to restart the numbering in each child document
and denote the pages by `\textit{child}|.|\textit{page}'
where \textit{child} represents the chapter/section number of the child file.
This can be achieved by the command
|\numberwithin{page}{|\textit{child}|}|
of the \textsf{amsmath} package
where \textit{child} can be |chapter| or |section|
depending on the chosen structuring.
Alternatively, one can modify the macro |\thepage| appropriately
and reset the counter |page| at the start of each child file.

%%%%%%%%%%%%%%%%%%%%%%%%%%%%%%%%%%%%%%%%%%%%%%%%%%%%%%%%%%%%%%%%%%%%%%%%%%%%%%%%
\subsection{Conditional Processing}
\label{sec:conditional}

The package provides a mechanism to compile different versions
of a document. To customise the versions further some conditional processing
can come in handy to distinguish which version is being compiled.
The package provides two macros to describe the compilation context:

%%%%%%%%%%%%%%%%%%%%%%%%%%%%%%%%%%%%%%%%
\DescribeMacro{\ifchilddoc}
The conditional |\ifchilddoc| distinguishes between the compilation of
child documents and the main document:
%
\begin{center}
|\ifchilddoc |\textit{child-code}| |[|\||else |\textit{main-code}]| \||fi|
\end{center}

%%%%%%%%%%%%%%%%%%%%%%%%%%%%%%%%%%%%%%%%
\DescribeMacro{\childdocname}
\DescribeMacro{\childdocjob}
The macro |\childdocname| contains the filename (without extension)
of the main or child file being processed.
Note that |\childdocjob| will always contain the name of the main file.

%%%%%%%%%%%%%%%%%%%%%%%%%%%%%%%%%%%%%%%%
\paragraph{Title Page.}

Conditional processing can be used to include a title or banner page
in the main document when proper precautions are taken.
Importantly, the code in the main file should ensure that the page counter
(as well as other status parameters which are stored in the |.aux| files)
takes the same value after the conditional processing.
Otherwise the page numbers may take divergent values
depending on which part is compiled.

For example, a title page could be declared by:
%
\begin{center}
\begin{tabular}{l}
|\ifchilddoc\||else|\\
|\addtocounter{page}{-1}|\\
\textit{code for title page}\\
|\newpage|\\
|\||fi|
\end{tabular}
\end{center}
%
A banner page for the child documents can be generated by:
%
\begin{center}
\begin{tabular}{l}
|\ifchilddoc|\\
|\addtocounter{page}{-1}|\\
\textit{code for banner page}\\
|\newpage|\\
|\||fi|
\end{tabular}
\end{center}
%
Here one could write a message such as:
\begin{center}
|This is the part \childdocname{} of \childdocjob{}.|
\end{center}

%%%%%%%%%%%%%%%%%%%%%%%%%%%%%%%%%%%%%%%%%%%%%%%%%%%%%%%%%%%%%%%%%%%%%%%%%%%%%%%%
\subsection{Flags}
\label{sec:flags}

The package makes it easy to generate different versions
of the main or child documents.
To this end compilation flags can be defined
and assigned different default values.
They will be particularly useful in conjunction
with the forwarding mechanism described in \secref{sec:forward}.

For example, it may be useful to have a flag |\version|
which can be set to |draft| or |final|.
The document source will contain some conditional code
depending on the value of |\version|.
Suppose further, the flag should default to |final| for the main file
and to |draft| for child files
which is a natural assignment for editing the document.
This is achieved by placing the following code
in the preamble of the main document
(below the |\childdocmain| directive):
%
\begin{center}
\begin{tabular}{l}
|\ifchilddoc|\\
|\providecommand{\version}{draft}|\\
|\||else|\\
|\providecommand{\version}{final}|\\
|\||fi|
\end{tabular}
\end{center}
%
The definition by |\providecommand| makes sure
that previous definitions are not overwritten.
Further statements |\providecommand{\version}{...}|
can thus be added before the above code to override it.

For the main file, one might add a line
(between |\childdocmain| and the above block)
%
\begin{center}
|%\ifchilddoc\||else\providecommand{\version}{draft}\||fi|
\end{center}
%
which can be uncommented to produce a draft version.
Likewise one can add a line to the very top of a child file
(above the |\childdocof{|\textit{main}|}| directive)
%
\begin{center}
|%\providecommand{\version}{final}|
\end{center}
%
which can be uncommented to produce the final version of this child document.

%%%%%%%%%%%%%%%%%%%%%%%%%%%%%%%%%%%%%%%%%%%%%%%%%%%%%%%%%%%%%%%%%%%%%%%%%%%%%%%%
\subsection{Forwarding}
\label{sec:forward}

Different versions of the main or child documents
using compilation flags as described in \secref{sec:flags}
can be (permanently) stored in different files
for convenient compilation, viewing and distribution.
To this end, the package defines a command
to pass on compilation to a different file:

%%%%%%%%%%%%%%%%%%%%%%%%%%%%%%%%%%%%%%%%
\DescribeMacro{\childdocforward}
The command |\childdocforward| redirects processing to
another source file:
%
\begin{center}
\begin{tabular}{l}
|\input{childdoc.def}|\\
|\childdocforward[|\textit{main}|]{|\textit{dest}|}|\\
\end{tabular}
\end{center}
%
The argument \textit{dest} is the destination file
(without extension).
It should be the main file or one of the child files.
Note that further \textsf{childdoc} directives
such as |\childdocof| and |\childdocforward|
in the indicated file will be processed in this form.
The optional argument \textit{main}
passes on directly to the main file \textit{main}
while pretending to compile the child \textit{dest}.
This form behaves as if \textit{dest}
issues |\childdocof{|\textit{main}|}| right away,
and no further \textsf{childdoc} directives will be processed.

%%%%%%%%%%%%%%%%%%%%%%%%%%%%%%%%%%%%%%%%
\DescribeMacro{\...prefix}
In the alternative form |\childdocforwardprefix|,
%
\begin{center}
\begin{tabular}{l}
|\input{childdoc.def}|\\
|\childdocforwardprefix[|\textit{main}|]{|\textit{prefix}|}{|\textit{dest}|}|
\end{tabular}
\end{center}
%
the destination file is determined by a pattern
depending on the current file:
To make this work, the current file must be called
`{\textit{prefix}\hspace{0.2em}\textit{suffix}}'
with \textit{prefix} matching precisely the argument.
Processing is then passed on to the file
`{\textit{dest}\hspace{0.2em}\textit{suffix}}'.
Surely, the same effect is achieved by
directly specifying the
argument `{\textit{dest}\hspace{0.2em}\textit{suffix}}'
in the first form.
However, that requires to set up a different file
for each child. With the alternative form of the command
all these files can have exactly the same content
which simplifies setting them up and maintaining them.

For example, the following file |draft.tex|
with a compilation flag |\version| as described in \secref{sec:flags}
compiles the main document as a draft:
%
\begin{center}
\begin{tabular}{l}
|\def\version{draft}|\\
|\input{childdoc.def}|\\
|\childdocforward{|\textit{main}|}|
\end{tabular}
\end{center}
%
Likewise, the following files |final|\textit{nn}|.tex|
compile the final version of the child document
|child|\textit{nn}|.tex|:
%
\begin{center}
\begin{tabular}{l}
|\def\version{final}|\\
|\input{childdoc.def}|\\
|\childdocforwardprefix{final}{child}|
\end{tabular}
\end{center}
%

Note that when several versions of a main file and/or of each child file
are to be generated, it may be convenient to set up a |Makefile| or
shell script to automatise the process.

%%%%%%%%%%%%%%%%%%%%%%%%%%%%%%%%%%%%%%%%%%%%%%%%%%%%%%%%%%%%%%%%%%%%%%%%%%%%%%%%
\subsection{Command Line Processing}
\label{sec:commandline}

The effect of redirection files can also be achieved by invoking
the \LaTeX{} compiler with a more elaborate command line.
Most conveniently this should be done as part
of a shell script or a |Makefile|.

When using \textsf{childdoc} in the main file, the following
command lines effectively perform a redirection
(note that depending on the shell being used,
backslashes may have to be doubled: `|\|' $\to$ `|\\|'):
%
\begin{center}
|... -jobname "|\textit{target}|" |\\|"|[\textit{flags}]%
|\input{childdoc.def}\childdocforward[|\textit{main}|]{|\textit{dest}|}"|
\end{center}
%
Here \textit{target} is the name of the output file,
\textit{main} is the name of the main file
and \textit{dest} is the name of the main or child file to be processed
(all filenames without extensions).
The optional argument \textit{main} can be omitted
if \textit{main} matches \textit{dest}.
Optionally, compilation \textit{flags} can be defined via |\def| commands.
This command line makes the \TeX{} engine believe
it is compiling the file \textit{target}
whose content is specified as the latter parameter.
The provided code then forwards the processing to
\textit{main} or \textit{dest} as described in \secref{sec:forward}.

%%%%%%%%%%%%%%%%%%%%%%%%%%%%%%%%%%%%%%%%%%%%%%%%%%%%%%%%%%%%%%%%%%%%%%%%%%%%%%%%
\subsection{Include by Input}
\label{sec:input}

Including child documents by |\include| has some restrictions by design.
Most notably, the content of a child document always occupies
its own set of pages; pages cannot be shared between child documents.
Usually, this behaviour makes perfect sense
because each child document contain an essential part of the document.
However, in some situations it may be desirable to compose
a document from a collection of parts
without having mandatory page breaks between then.
For this case, the package
provides a mechanism to include parts
by |\input| which can also be processed individually.
However, by construction this mechanism
requires manual handling of the content to be output.

%%%%%%%%%%%%%%%%%%%%%%%%%%%%%%%%%%%%%%%%
\DescribeMacro{\ifchilddocmanual}
The main file should be prepared as usual, see \secref{sec:include}.
However, the document body must make a distinction
between processing of an individual part and of the main document, e.g.:
%
\begin{center}
\begin{tabular}{l}
|\ifchilddocmanual|\\
|\input{\childdocname}|\\
|\||else|\\
\textit{document body with }|\input{|\textit{part}|}|\\
|\||fi|
\end{tabular}
\end{center}
%
The conditional |\ifchilddocmanual| is true whenever
a part to be included by |\input| is being compiled,
and the name of the part is stored in |\childdocname|.

%%%%%%%%%%%%%%%%%%%%%%%%%%%%%%%%%%%%%%%%
\DescribeMacro{\childdocby}
Each part to be included by |\input| should start with:
%
\begin{center}
\begin{tabular}{l}
|\input{childdoc.def}|\\
|\childdocby{|\textit{main}|}|\\
\end{tabular}
\end{center}
%
The directive |\childdocby| is similar to |\childdocof|
described in \secref{sec:include},
but the subsequent selection of content must be done manually.
To that end, both |\ifchilddoc| and |\ifchilddocmanual|
will be true upon processing of a part,
and the name of the part is stored in |\childdocname|.
Note that |\jobname| will be set to the filename of the current part
so that each part receives an individual |.aux| file
that does not interfere with the |.aux| file(s) of the main document.
This behaviour can be altered by the alternative form
|\childdocby[*]{|\textit{main}|}| (with a non-empty optional argument)
which uses the |.aux| file of the main document
by setting |\jobname| to \textit{main}.

%%%%%%%%%%%%%%%%%%%%%%%%%%%%%%%%%%%%%%%%%%%%%%%%%%%%%%%%%%%%%%%%%%%%%%%%%%%%%%%%
\subsection{Driver Development}
\label{sec:driver}

The \textsf{childdoc} mechanism can also be use for the development
of definition files such as \LaTeX{} styles or classes.
This case differs from the above setup with multiple parts
included by |\include| in that no |\includeonly| should be invoked.
This can be achieved by starting the include file
(before |\ProvidesPackage|) with:
%
\begin{center}
\begin{tabular}{l}
|\input{childdoc.def}|\\
|\childdocforward{|\textit{main}|}|\\
\end{tabular}
\end{center}
%
or alternatively with:
%
\begin{center}
\begin{tabular}{l}
|\input{childdoc.def}|\\
|\childdocby{|\textit{main}|}|\\
\end{tabular}
\end{center}
%
Both forms have slightly different effects as described above.
The main file is prepared as usual, see \secref{sec:include}.

%%%%%%%%%%%%%%%%%%%%%%%%%%%%%%%%%%%%%%%%%%%%%%%%%%%%%%%%%%%%%%%%%%%%%%%%%%%%%%%%
\subsection{Legacy Detection}
\label{sec:detection}

The directive |\childdocmain| in the main file can detect
whether the complete document or merely a child is to be compiled
even without using the directive |\childdocof|.
This method is deprecated because it is less robust
and there is no compelling reason to use it;
it is merely provided for backward compatibility
and it may be removed in future versions.

If the detection mechanism is to be used,
it is mandatory to correctly specify
the filename of the main file as the argument of |\childdocmain|:
%
\begin{center}
\begin{tabular}{l}
|\input{childdoc.def}|\\
|\childdocmain{|\textit{main}|}|\\
\end{tabular}
\end{center}
%
If |\jobname| does not match the argument \textit{main} of |\childdocmain|,
it is assumed that |\jobname| points to the child file to be compiled.
When using |\childdocmain| with the main file specified as argument,
it suffices to start a child file
with just |\input{|\textit{main}|}|
without loading of the package and using |\childdocof|.
If instead all processing is done
with the appropriate \textsf{childdoc} directives,
the argument of \textit{main} of |\childdocmain| can be empty.

An alternative version of the command line processing described
in \secref{sec:commandline} using the detection mechanism reads:
%
\begin{center}
|... -jobname "|\textit{target}|" "|[\textit{flags}]%
[|\def\jobname{|\textit{dest}|}|]|\input{|\textit{main}|}"|
\end{center}

%%%%%%%%%%%%%%%%%%%%%%%%%%%%%%%%%%%%%%%%%%%%%%%%%%%%%%%%%%%%%%%%%%%%%%%%%%%%%%%%
\subsection{Manual Code}
\label{sec:manual}

In case one cannot be certain whether the definitions file |childdoc.def|
is installed on the target \TeX{} distribution
and one prefers not to ship it,
it is conceivable to paste a few relevant commands into the sources.

To that end, drop all statements |\input{childdoc.def}|
and perform the replacements as outlined below.
Instead of |\childdocmain{|\textit{main}|}| add the following code
to the top of the main file:
%
\begin{center}
\begin{tabular}{l}
|\||ifdefined\childdocname\endinput\||fi\newif\ifchilddoc|\\
|\edef\childdocname{\scantokens\expandafter{\jobname\noexpand}}|\\
|\def\childdocmain{|\textit{main}|}\||ifx\childdocmain\childdocname\||else|\\
|\childdoctrue\includeonly{\childdocname}\let\jobname\childdocmain\||fi|\\
\end{tabular}
\end{center}
%
Instead of |\childdocof{|\textit{main}|}| just include the main file
at the top of each child file:
%
\begin{center}
|\input{|\textit{main}|}|
\end{center}
%
A simple redirection |\childdocforward{|\textit{dest}|}| is achieved by:
%
\begin{center}
|\def\jobname{|\textit{dest}|}\input{\jobname}|
\end{center}
%
The redirection with prefix
|\childdocforwardprefix[|\textit{prefix}|]{|\textit{dest}|}|
is accomplished by:
%
\begin{center}
\begin{tabular}{l}
|{\edef\jobname{\scantokens\expandafter{\jobname\noexpand}}|\\
|\def\redirectjob |\textit{prefix}|#1~~~{\gdef\jobname{|\textit{dest}|#1}}|\\
|\expandafter\redirectjob\jobname~~~}\input{\jobname}|
\end{tabular}
\end{center}

In an alternative approach,
child documents can be compiled by a specific command line
without additional code or specific definitions:
%
\begin{center}
|... -jobname "|\textit{target}|" "|[\textit{flags}]%
|\includeonly{|\textit{dest}|}\input{|\textit{main}|}"|
\end{center}
%

%%%%%%%%%%%%%%%%%%%%%%%%%%%%%%%%%%%%%%%%%%%%%%%%%%%%%%%%%%%%%%%%%%%%%%%%%%%%%%%%
%%%%%%%%%%%%%%%%%%%%%%%%%%%%%%%%%%%%%%%%%%%%%%%%%%%%%%%%%%%%%%%%%%%%%%%%%%%%%%%%
\section{Information}

%%%%%%%%%%%%%%%%%%%%%%%%%%%%%%%%%%%%%%%%%%%%%%%%%%%%%%%%%%%%%%%%%%%%%%%%%%%%%%%%
\subsection{Copyright}

Copyright \copyright{} 2017--2018 Niklas Beisert

This work may be distributed and/or modified under the
conditions of the \LaTeX{} Project Public License, either version 1.3
of this license or (at your option) any later version.
The latest version of this license is in
  \url{http://www.latex-project.org/lppl.txt}
and version 1.3 or later is part of all distributions of \LaTeX{}
version 2005/12/01 or later.

This work has the LPPL maintenance status `maintained'.

The Current Maintainer of this work is Niklas Beisert.

This work consists of the files |README.txt|, |childdoc.ins| and |childdoc.dtx|
as well as the derived files |childdoc.def|, |cdocsamp.tex|
with |cdocsch1.tex|, |cdocsch2.tex|, |cdocspt3.tex|, |cdocspt4.tex|,
|cdocsdrf.tex|, |cdocsfn1.tex|, |cdocsfn2.tex|
as well as |childdoc.pdf|.

%%%%%%%%%%%%%%%%%%%%%%%%%%%%%%%%%%%%%%%%%%%%%%%%%%%%%%%%%%%%%%%%%%%%%%%%%%%%%%%%
\subsection{Files and Installation}

The package consists of the files:
%
\begin{center}
\begin{tabular}{ll}
    |README.txt|   & readme file \\
    |childdoc.ins| & installation file \\
    |childdoc.dtx| & source file \\
    |childdoc.def| & definition file \\
    |cdocsamp.tex| & sample main file \\
    |cdocsch1.tex| & sample include file \\
    |cdocsch2.tex| & sample include file \\
    |cdocspt3.tex| & sample part file \\
    |cdocspt4.tex| & sample part file \\
    |cdocsdrf.tex| & sample redirection file \\
    |cdocsfn1.tex| & sample redirection file \\
    |cdocsfn2.tex| & sample redirection file \\
    |childdoc.pdf| & manual
\end{tabular}
\end{center}
%
The distribution consists of the files
|README.txt|, |childdoc.ins| and |childdoc.dtx|.
%
\begin{itemize}
\item
Run (pdf)\LaTeX{} on |childdoc.dtx|
to compile the manual |childdoc.pdf| (this file).
\item
Run \LaTeX{} on |childdoc.ins| to create the definitions file |childdoc.def|
and the sample |cdocsamp.tex| with include files
|cdocsch1.tex|, |cdocsch2.tex|, |cdocspt3.tex|, |cdocspt4.tex|,
|cdocsdrf.tex|, |cdocsfn1.tex|, |cdocsfn2.tex|.
Then copy the file |childdoc.def| to an appropriate directory of your \LaTeX{}
distribution, e.g.\ \textit{texmf-root}|/tex/latex/childdoc|.
\end{itemize}

%%%%%%%%%%%%%%%%%%%%%%%%%%%%%%%%%%%%%%%%%%%%%%%%%%%%%%%%%%%%%%%%%%%%%%%%%%%%%%%%
\subsection{Related CTAN Packages}

There are several other packages which offer a similar functionality:
%
\begin{itemize}
\item
The packages
\href{http://ctan.org/pkg/docmute}{\textsf{docmute}},
\href{http://ctan.org/pkg/includex}{\textsf{includex}} and
\href{http://ctan.org/pkg/standalone}{\textsf{standalone}}
provide commands to include only the document body of
a child file thus allowing both files to be compiled individually.
\item
The packages \href{http://ctan.org/pkg/subdocs}{\textsf{subdocs}}
and \href{http://ctan.org/pkg/subfiles}{\textsf{subfiles}}
provide structures in which the main and child documents can be
encapsulated and allowing them to be compiled individually.
The inclusion mechanism is different from the conventional |\include|.
\item
The package \href{http://ctan.org/pkg/combine}{\textsf{combine}}
is an elaborate solution to combine several documents into one.
\end{itemize}
%
See also the CTAN topic \href{http://ctan.org/topic/subdocs}{\textsf{subdocs}}
for further related packages.
The present package differs from the above solutions in that
a document structure constructed with the conventional |\include| mechanism
just needs two extra commands at the top of every file
such that all constituent files can be compiled individually.

%%%%%%%%%%%%%%%%%%%%%%%%%%%%%%%%%%%%%%%%%%%%%%%%%%%%%%%%%%%%%%%%%%%%%%%%%%%%%%%%
%\subsection{Feature Suggestions}
%
%The following is a list of features which may be useful for future
%versions of this package:
%%
%\begin{itemize}
%\item
%\ldots
%\end{itemize}

%%%%%%%%%%%%%%%%%%%%%%%%%%%%%%%%%%%%%%%%%%%%%%%%%%%%%%%%%%%%%%%%%%%%%%%%%%%%%%%%
\subsection{Revision History}

%%%%%%%%%%%%%%%%%%%%%%%%%%%%%%%%%%%%%%%%
\paragraph{v2.0:} 2018/12/30

\begin{itemize}
\item
immediate forward processing
\item
added |\childdocby| mechanism
\item
manual restructured
\end{itemize}

%%%%%%%%%%%%%%%%%%%%%%%%%%%%%%%%%%%%%%%%
\paragraph{v1.6:} 2018/01/17

\begin{itemize}
\item
application for development of include files
\item
corrections to manual
\end{itemize}

%%%%%%%%%%%%%%%%%%%%%%%%%%%%%%%%%%%%%%%%
\paragraph{v1.5:} 2017/05/21

\begin{itemize}
\item
more complete structuring introduced
\item
|\childdocof| introduced
\item
|\childdoc| renamed to |\childdocmain|
\item
|\childredirect| renamed to |\childdocforward| and |\childdocforwardprefix|
and functionality expanded
\end{itemize}

%%%%%%%%%%%%%%%%%%%%%%%%%%%%%%%%%%%%%%%%
\paragraph{v1.0:} 2017/04/27

\begin{itemize}
\item
manual and install package
\item
first version published on CTAN
\end{itemize}

%%%%%%%%%%%%%%%%%%%%%%%%%%%%%%%%%%%%%%%%
\paragraph{v0.6:} 2017/04/26

\begin{itemize}
\item
redirection mechanism added
\end{itemize}

%%%%%%%%%%%%%%%%%%%%%%%%%%%%%%%%%%%%%%%%
\paragraph{v0.5:} 2017/04/26

\begin{itemize}
\item
functionality in definition file
\end{itemize}


%%%%%%%%%%%%%%%%%%%%%%%%%%%%%%%%%%%%%%%%%%%%%%%%%%%%%%%%%%%%%%%%%%%%%%%%%%%%%%%%
%%%%%%%%%%%%%%%%%%%%%%%%%%%%%%%%%%%%%%%%%%%%%%%%%%%%%%%%%%%%%%%%%%%%%%%%%%%%%%%%
%%%%%%%%%%%%%%%%%%%%%%%%%%%%%%%%%%%%%%%%%%%%%%%%%%%%%%%%%%%%%%%%%%%%%%%%%%%%%%%%
\appendix

\settowidth\MacroIndent{\rmfamily\scriptsize 000\ }

 \DocInput{childdoc.dtx}

\end{document}
%</driver>
% \fi
%
% %%%%%%%%%%%%%%%%%%%%%%%%%%%%%%%%%%%%%%%%%%%%%%%%%%%%%%%%%%%%%%%%%%%%%%%%%%%%%%
% %%%%%%%%%%%%%%%%%%%%%%%%%%%%%%%%%%%%%%%%%%%%%%%%%%%%%%%%%%%%%%%%%%%%%%%%%%%%%%
% \section{Sample}
%\iffalse
%<*samplemain>
%\fi
%
% The following presents a sample document
% with two chapters, two parts, a title page,
% a compile flag as well as three forwarding files to set the flag.
% It consists of eight |.tex| files:
% \begin{center}
% \begin{tabular}{ll}
% |cdocsamp.tex|&main file\\
% |cdocsch1.tex|&include file for chapter 1\\
% |cdocsch2.tex|&include file for chapter 2\\
% |cdocspt3.tex|&include file for part 3\\
% |cdocspt4.tex|&include file for part 4\\
% |cdocsdrf.tex|&forwarding file for main file in draft mode\\
% |cdocsfi1.tex|&forwarding file for final version of chapter 1\\
% |cdocsfi2.tex|&forwarding file for final version of chapter 2\\
% \end{tabular}
% \end{center}
% Each of the eight files can be compiled directly by the \LaTeX{} compiler.
%
% %%%%%%%%%%%%%%%%%%%%%%%%%%%%%%%%%%%%%%
% \paragraph{Main File.}
%
% The main file is called |cdocsamp.tex|.
%
% Load the \textsf{childdoc} definitions and
% declare the filename for the main document:
%    \begin{macrocode}
\input{childdoc.def}
\childdocmain{}
%    \end{macrocode}

% Optional override for |\version| flag:
%    \begin{macrocode}
%%\ifchilddoc\else\providecommand{\version}{draft}\fi
%    \end{macrocode}

% Define the default values for the |\version| flag
% (|final| for the main file and |draft| for childs):
%    \begin{macrocode}
\ifchilddoc
\providecommand{\version}{draft}
\else
\providecommand{\version}{final}
\fi
%    \end{macrocode}

% Load the standard document class:
%    \begin{macrocode}
\documentclass[12pt]{article}
%    \end{macrocode}

% Start the document body:
%    \begin{macrocode}
\begin{document}
%    \end{macrocode}

% Declare a title page.
% Print title, part of document being processed and version flag:
%    \begin{macrocode}
\addtocounter{page}{-1}
\begin{center}
{\LARGE\bfseries{}childdoc example\par}
\vspace{1cm}
\ifchilddoc
\ifchilddocmanual part\else chapter\fi:
`\childdocname' of `\childdocjob'\par
\else
main document: `\childdocjob'\par
\fi
version: \version\par
\end{center}
\newpage
%    \end{macrocode}

% Manually include selected file,
% otherwise process as usual:
%    \begin{macrocode}
\ifchilddocmanual
\section*{part `\childdocname'}
\input{\childdocname}
\else
%    \end{macrocode}

% Include the two chapters:
%    \begin{macrocode}
\include{cdocsch1}
\include{cdocsch2}
%    \end{macrocode}

% Include the two parts unless only chapters should be displayed:
%    \begin{macrocode}
\ifchilddoc\else
\section{part three}
\input{cdocspt3}
\section{part four}
\input{cdocspt4}
\fi
%    \end{macrocode}

% Process as usual until here:
%    \begin{macrocode}
\fi
%    \end{macrocode}

% End of document body:
%    \begin{macrocode}
\end{document}
%    \end{macrocode}
%\iffalse
%</samplemain>
%\fi
%
% %%%%%%%%%%%%%%%%%%%%%%%%%%%%%%%%%%%%%%
% \paragraph{Chapter Include Files.}
%
% The include files are called |cdocsch1.tex| and |cdocsch2.tex|.
%
%\iffalse
%<*samplechap1|samplechap2>
%\fi

% Optional override for |\version| flag:
%    \begin{macrocode}
%%\providecommand{\version}{final}
%    \end{macrocode}

% Include the main document:
%    \begin{macrocode}
\input{childdoc.def}
\childdocof{cdocsamp}
%    \end{macrocode}

%\iffalse
%</samplechap1|samplechap2>
%\fi
%
%\iffalse
%<*samplechap1>
%\fi
% Some text for chapter 1:
%    \begin{macrocode}
\section{one}
some text in chapter one
%    \end{macrocode}

%\iffalse
%</samplechap1>
%\fi
% Some text for chapter 2:
%\iffalse
%<*samplechap2>
%\fi
%    \begin{macrocode}
\section{two}
more text in chapter two
%    \end{macrocode}

%\iffalse
%</samplechap2>
%\fi
%
% %%%%%%%%%%%%%%%%%%%%%%%%%%%%%%%%%%%%%%
% \paragraph{Part Include Files.}
%
% The include files are called |cdocspt3.tex| and |cdocspt4.tex|.
%
%\iffalse
%<*samplepart3|samplepart4>
%\fi

% Optional override for |\version| flag:
%    \begin{macrocode}
%%\providecommand{\version}{final}
%    \end{macrocode}

% Include the main document:
%    \begin{macrocode}
\input{childdoc.def}
\childdocby{cdocsamp}
%    \end{macrocode}

%\iffalse
%</samplepart3|samplepart4>
%\fi
%
%\iffalse
%<*samplepart3>
%\fi
% Some text for part 3:
%    \begin{macrocode}
some text in part three
%    \end{macrocode}

%\iffalse
%</samplepart3>
%\fi
% Some text for part 4:
%\iffalse
%<*samplepart4>
%\fi
%    \begin{macrocode}
more text in part four
%    \end{macrocode}

%\iffalse
%</samplepart4>
%\fi
%
% %%%%%%%%%%%%%%%%%%%%%%%%%%%%%%%%%%%%%%
% \paragraph{Forwarding for a Complete Draft.}
%
% The following forwarding file |cdocsdrf.tex|
% compiles the main document in draft mode:
%\iffalse
%<*sampledraft>
%\fi
%    \begin{macrocode}
\def\version{draft}
\input{childdoc.def}
\childdocforward{cdocsamp}
%    \end{macrocode}

%\iffalse
%</sampledraft>
%\fi
%
% %%%%%%%%%%%%%%%%%%%%%%%%%%%%%%%%%%%%%%
% \paragraph{Forwarding for Final Version of the Chapters.}
%
% The following forwarding files |cdocsfn1.tex| and |cdocsfn2.tex|
% (with identical content)
% compile the final versions of the child documents
% |cdocsch1.tex| and |cdocsch2.tex|, respectively:
%\iffalse
%<*samplefinal>
%\fi
%    \begin{macrocode}
\def\version{final}
\input{childdoc.def}
\childdocforwardprefix[cdocsamp]{cdocsfn}{cdocsch}
%    \end{macrocode}

%\iffalse
%</samplefinal>
%\fi
%
% %%%%%%%%%%%%%%%%%%%%%%%%%%%%%%%%%%%%%%
% \paragraph{Command Line Processing.}
%
% The following three command lines generate the output files
% |cdocscld|, |cdocscl1| and |cdocscl2|
% which should be identical to
% |cdocsdrf|, |cdocsch1| and |cdocsfn2|, respectively:
% \begin{center}
% \begin{tabular}{l}
% |latex -jobname cdocscld \|\\
% |  "\def\version{draft}\input{childdoc.def}\childdocforward{cdocsamp}"|\\
% |latex -jobname cdocscl1 \|\\
% |  "\input{childdoc.def}\childdocforward[cdocsamp]{cdocsch1}"|\\
% |latex -jobname cdocscl2 \|\\
% |  "\def\version{final}\input{childdoc.def}\childdocforward{cdocsch2}"|
% \end{tabular}
% \end{center}
% Note that the trailing backslash on each first line
% merely continues the input to the second line
% (for convenient cut ant paste).
% Furthermore, the command |latex| can be replaced by any
% of its alternative versions such as |pdflatex|.
%
% %%%%%%%%%%%%%%%%%%%%%%%%%%%%%%%%%%%%%%%%%%%%%%%%%%%%%%%%%%%%%%%%%%%%%%%%%%%%%%
% %%%%%%%%%%%%%%%%%%%%%%%%%%%%%%%%%%%%%%%%%%%%%%%%%%%%%%%%%%%%%%%%%%%%%%%%%%%%%%
% \section{Implementation}
%\iffalse
%<*package>
%\fi
%
% This section describes the definitions file |childdoc.def|.

% The definitions cannot be loaded using |\usepackage| or |\RequirePackage|
% which has a mechanism to prevent loading a style file more than once.
% When loading the definitions by means of |\input|
% multiple instances have to be prevented manually:
%\iffalse
%This code needs to be before the `\ProvidesFile' directive
%which is defined at the beginning of this file.
%Therefore it is also placed there and commented out here.
%</package>
%<*discard>
%\fi
%    \begin{macrocode}
\ifdefined\childdocmain\endinput\fi
%    \end{macrocode}
%\iffalse
%</discard>
%<*package>
%\fi
%
% \macro{\ifchilddoc}
% \macro{\ifchilddocmanual}
% The conditional |\ifchilddoc| tells whether a
% child (true) or main (false) document is being compiled.
% The conditional |\ifchilddocmanual| tells whether
% the |\includeonly| mechanism is used (false) or
% the selection of child files must be performed manually (true).
% The definitions initialise to false:
%    \begin{macrocode}
\newif\ifchilddoc
\newif\ifchilddocmanual
%    \end{macrocode}

% \macro{\childdocname}
% \macro{\childdocjob}
% The macro |\childdocname| stores the name of the main document
% to be compiled. The macro |\childdocjob| stores the name of
% the document on which the \LaTeX{} compiler was originally invoked.
% The content of |\jobname| cannot be compared
% to filenames specified in the source due to different catcodes.
% The following code rescans |\jobname|, stores the result
% in |\childdocname| and saves a copy in |\childdocjob|:
%    \begin{macrocode}
\edef\childdocname{\scantokens\expandafter{\jobname\noexpand}}
\let\childdocjob\childdocname
%    \end{macrocode}

% \macro{\childdocdisable}
% The macro |\childdocdisable| prevents the main file
% from being processed more than once.
% At this stage, the main document command |\childdocmain|
% is assumed to be called once again where it should do nothing.
% Any subsequent call to it should prevent
% a secondary processing of the main document
% It overwrites the forwarding commands
% |\childdocof| and |\childdocforward|
% with empty macros to prevent further inclusions of the main document:
%    \begin{macrocode}
\newcommand{\childdocdisable}
{
  \renewcommand{\childdocmain}[1]{\renewcommand{\childdocmain}[1]{\endinput}}
  \renewcommand{\childdocof}[1]{}
  \renewcommand{\childdocby}[2][]{}
  \renewcommand{\childdocforward}[2][]{}
  \renewcommand{\childdocdisable}{}
}
%    \end{macrocode}

% \macro{\childdocmain}
% The macro |\childdocmain| is to be called at the top of the main file
% with nothing or the main filename (without extension) as argument.
% First, it breaks loops.
% If the argument is not empty and does not match |\childdocname|
% (which is set by the first inclusion of |childdoc.def|),
% |\ifchilddoc| is set to true, |\includeonly| is applied to the child file
% and |\jobname| is set to the main file
% (for proper handling of |.aux| files):
%    \begin{macrocode}
\newcommand{\childdocmain}[1]
{
  \childdocdisable\childdocmain{}
  \if?#1?\else
    \begingroup
      \def\childdoctmp{#1}
      \ifx\childdoctmp\childdocname
        \def\childdoctmp{}
      \else
        \def\childdoctmp
        {
          \childdoctrue
          \includeonly{\childdocname}
          \def\childdocjob{#1}
          \def\jobname{#1}
        }
      \fi
      \expandafter
    \endgroup
    \childdoctmp
  \fi
}
%    \end{macrocode}

% \macro{\childdocof}
% The command |\childdocof| redirects
% compilation to the main file |#1|.
%    \begin{macrocode}
\newcommand{\childdocof}[1]
{
  \childdocdisable
  \childdoctrue
  \includeonly{\childdocname}
  \def\jobname{#1}
  \def\childdocjob{#1}
  \input{#1}
}
%    \end{macrocode}

% \macro{\childdocby}
% The command |\childdocby| ....
%    \begin{macrocode}
\newcommand{\childdocby}[2][]
{
  \childdocdisable
  \childdoctrue
  \childdocmanualtrue
  \if?#1?\else
    \def\jobname{#2}
  \fi
  \def\childdocjob{#2}
  \input{#2}
  \endinput
}
%    \end{macrocode}

% \macro{\childdocforward}
% The command |\childdocforward| redirects
% compilation to the main file or
% (if the optional argument is given) a child file.
% Parameters are set as if the main file
% or a child file starting with |\childdocof| was compiled.
% Then compilation is handed over to the main file:
%    \begin{macrocode}
\newcommand{\childdocforward}[2][]
{
  \begingroup
    \if?#1?
      \def\childdoctmp
      {
        \def\childdocname{#2}
        \def\childdocjob{#2}
        \def\jobname{#2}
        \input{#2}
        \endinput
      }
    \else
      \def\childdoctmp
      {
        \childdocdisable
        \def\childdocname{#2}
        \childdoctrue
        \includeonly{#2}
        \def\childdocjob{#1}
        \def\jobname{#1}
        \input{#1}
        \endinput
      }
    \fi
    \expandafter
  \endgroup
  \childdoctmp
}
%    \end{macrocode}

% \macro{\childdocforwardprefix}
% The command |\childdocforwardprefix| redirects
% compilation to the main or a child file by means of a pattern.
% The prefix |#1| in the current filename is replaced by |#2|
% and the suffix of the current filename is kept
% (it is assumed that the filename does not contain the substring `|~~~|'
% which is used as a delimiter).
% Compilation is handed over to the new file by |\childdocforward|:
%    \begin{macrocode}
\newcommand{\childdocforwardprefix}[3][]
{
  \begingroup
    \def\childdocextract #2##1~~~{\def\childdoctmp{\childdocforward[#1]{#3##1}}}
    \expandafter\childdocextract\childdocname~~~
    \expandafter
  \endgroup
  \childdoctmp
}
%    \end{macrocode}

% \macro{\childdoc}
% The deprecated macro |\childdoc| is a legacy version of |\childdocmain|:
%    \begin{macrocode}
\newcommand{\childdoc}{\childdocmain}
%    \end{macrocode}

% \macro{\childdocredirect}
% The deprecated macro |\childdocredirect| is a legacy version
% of |\childdocforward| and |\childdocforwardprefix|:
%    \begin{macrocode}
\newcommand{\childdocredirect}[2][]
{
  \begingroup
    \if?#1?
      \def\childdoctmp{\childdocforward{#2}}
    \else
      \def\childdoctmp{\childdocforwardprefix{#1}{#2}}
    \fi
    \expandafter
  \endgroup
  \childdoctmp
}
%    \end{macrocode}

%\iffalse
%</package>
%\fi
%
\endinput
|\\
|\childdocmain{}|\\
\end{tabular}
\end{center}
at the very top of the main \LaTeX{} file,
in particular \emph{before} the |\documentclass| statement!
The argument of |\childdocmain| should be left empty
(but it must be present).

%%%%%%%%%%%%%%%%%%%%%%%%%%%%%%%%%%%%%%%%
\DescribeMacro{\childdocof}
Furthermore, add the commands
\begin{center}
\begin{tabular}{l}
|% \iffalse
%
% childdoc.dtx Copyright (C) 2017-2018 Niklas Beisert
%
% This work may be distributed and/or modified under the
% conditions of the LaTeX Project Public License, either version 1.3
% of this license or (at your option) any later version.
% The latest version of this license is in
%   http://www.latex-project.org/lppl.txt
% and version 1.3 or later is part of all distributions of LaTeX
% version 2005/12/01 or later.
%
% This work has the LPPL maintenance status `maintained'.
%
% The Current Maintainer of this work is Niklas Beisert.
%
% This work consists of the files childdoc.dtx and childdoc.ins
% and the derived files childdoc.def and cdocsamp.tex with
% cdocsch1.tex, cdocsch2.tex, cdocsdrf.tex, cdocsfn1.tex, cdocsfn2.tex.
%
%<package>\ifdefined\childdocmain\endinput\fi
%<package>\ProvidesFile{childdoc.def}[2018/12/30 v2.0 child document driver]
%<samplemain>\ProvidesFile{cdocsamp.tex}[2018/12/30 v2.0 sample for childdoc]
%<*driver>
%\ProvidesFile{childdoc.drv}[2018/12/30 v2.0 childdoc reference manual file]
\PassOptionsToClass{10pt,a4paper}{article}
\documentclass{ltxdoc}

\usepackage[margin=35mm]{geometry}
\usepackage{hyperref}
\usepackage{hyperxmp}
\usepackage[usenames]{color}

\hypersetup{colorlinks=true}
\hypersetup{pdfstartview=FitH}
\hypersetup{pdfpagemode=UseNone}
\hypersetup{pdfsource={}}
\hypersetup{pdflang={en-UK}}
\hypersetup{pdfcopyright={Copyright 2017-2018 Niklas Beisert.
  This work may be distributed and/or modified under the
  conditions of the LaTeX Project Public License, either version 1.3
  of this license or (at your option) any later version.}}
\hypersetup{pdflicenseurl={http://www.latex-project.org/lppl.txt}}
\hypersetup{pdfcontactaddress={ETH Zurich, ITP, HIT K,
  Wolfgang-Pauli-Strasse 27}}
\hypersetup{pdfcontactpostcode={8093}}
\hypersetup{pdfcontactcity={Zurich}}
\hypersetup{pdfcontactcountry={Switzerland}}
\hypersetup{pdfcontactemail={nbeisert@itp.phys.ethz.ch}}
\hypersetup{pdfcontacturl={http://people.phys.ethz.ch/\xmptilde nbeisert/}}

\newcommand{\secref}[1]{\hyperref[#1]{section \ref*{#1}}}

\parskip1ex
\parindent0pt
\let\olditemize\itemize
\def\itemize{\olditemize\parskip0pt}

\begin{document}

\title{The \textsf{childdoc} Package}
\hypersetup{pdftitle={The childdoc Package}}
\author{Niklas Beisert\\[2ex]
  Institut f\"ur Theoretische Physik\\
  Eidgen\"ossische Technische Hochschule Z\"urich\\
  Wolfgang-Pauli-Strasse 27, 8093 Z\"urich, Switzerland\\[1ex]
  \href{mailto:nbeisert@itp.phys.ethz.ch}
  {\texttt{nbeisert@itp.phys.ethz.ch}}}
\hypersetup{pdfauthor={Niklas Beisert}}
\hypersetup{pdfsubject={Manual for the LaTeX2e Package childdoc}}
\date{30 December 2018, \textsf{v2.0}}
\maketitle

\begin{abstract}\noindent
\textsf{childdoc} is a \LaTeXe{} package
that enables the direct compilation
of document sections included by |\include|
to individual files.
\end{abstract}

\begingroup
\parskip0ex
\tableofcontents
\endgroup

%%%%%%%%%%%%%%%%%%%%%%%%%%%%%%%%%%%%%%%%%%%%%%%%%%%%%%%%%%%%%%%%%%%%%%%%%%%%%%%%
%%%%%%%%%%%%%%%%%%%%%%%%%%%%%%%%%%%%%%%%%%%%%%%%%%%%%%%%%%%%%%%%%%%%%%%%%%%%%%%%
\section{Introduction}

\LaTeX{} provides a mechanism to structure a large document (such as a book)
into a main file and several child files (containing the chapters)
using the |\include| command.
This mechanism is beneficial for documents
which span hundreds of pages in order to
make the source file(s) more manageable.
Moreover, compilation can be restricted to
selected child files by means of the |\includeonly| command.
The latter feature can be used to reduce the compilation time while editing
(this was significantly more useful in the earlier days of \LaTeX{})
or to generate a smaller document which is easier to navigate.
Another application of |\includeonly| is to generate
documents consisting of selected parts of the complete document.

However, there are a few drawbacks of the plain |\include| mechanism:
\begin{itemize}
\item
The child files cannot be compiled on their own,
they can only be compiled via the main file.
A naive editing environment
(such as a text editor with an option
to have the current file processed by \LaTeX)
may require one to switch to the main file before compiling;
attempting to compile the child file produces errors.
\item
The main file must be modified (each time)
to adjust the |\includeonly| command
to the present needs. This easily leaves the main file in a messy state.
\item
The generated document will always carry the filename
of the main document. This is inconvenient if
several child files are to be compiled and
to be kept for distribution.
\end{itemize}

The present package provides a simple interface
to make child files individually compilable by \LaTeX{}.
Compiling a child file then has the same effect as compiling
the main file with an |\includeonly| command
to select the appropriate child.
Moreover the generated document will carry the name of the child
rather than the main file.
This resolves all three above issues.

This feature is meant to make the editing of books,
thesis documents and lecture notes somewhat more convenient.
However, the package can also be used efficiently for
composing a series of documents (such as exercise sheets)
which are typically distributed individually.
It then assists the author in generating the individual documents
(potentially in different versions)
as well as a document containing the collected series.
Another application is in developing style files
or other kinds of included material
where compilation of the style file could redirect
to a sample or test file.

%%%%%%%%%%%%%%%%%%%%%%%%%%%%%%%%%%%%%%%%%%%%%%%%%%%%%%%%%%%%%%%%%%%%%%%%%%%%%%%%
%%%%%%%%%%%%%%%%%%%%%%%%%%%%%%%%%%%%%%%%%%%%%%%%%%%%%%%%%%%%%%%%%%%%%%%%%%%%%%%%
\section{Usage}

First of all, the package \textsf{childdoc} is \emph{not} a standard
\LaTeXe{} |.sty| style file! Therefore it needs to be invoked in
a non-standard way.

%%%%%%%%%%%%%%%%%%%%%%%%%%%%%%%%%%%%%%%%%%%%%%%%%%%%%%%%%%%%%%%%%%%%%%%%%%%%%%%%
\subsection{Included Files}
\label{sec:include}

%%%%%%%%%%%%%%%%%%%%%%%%%%%%%%%%%%%%%%%%
\DescribeMacro{\childdocmain}
To use the package, add the commands
\begin{center}
\begin{tabular}{l}
|\input{childdoc.def}|\\
|\childdocmain{}|\\
\end{tabular}
\end{center}
at the very top of the main \LaTeX{} file,
in particular \emph{before} the |\documentclass| statement!
The argument of |\childdocmain| should be left empty
(but it must be present).

%%%%%%%%%%%%%%%%%%%%%%%%%%%%%%%%%%%%%%%%
\DescribeMacro{\childdocof}
Furthermore, add the commands
\begin{center}
\begin{tabular}{l}
|\input{childdoc.def}|\\
|\childdocof{|\textit{main}|}|\\
\end{tabular}
\end{center}
at the top of every child file \textit{child}
which is included by |\include{|\textit{child}|}|
from within the main file
(or at least for those files to be compiled individually).
The argument \textit{main} must be the filename of the main file.

There are a couple of
considerations in setting up the main and child documents:

%%%%%%%%%%%%%%%%%%%%%%%%%%%%%%%%%%%%%%%%
\paragraph{Restrictions.}

Please note the following restrictions:
\begin{itemize}
\item
|\childdocmain| must be called with one argument \textit{main}
to ensure compatibility with earlier version of the package.
It must either be empty (|\childdocmain{}|)
or precisely match the filename of the main file in which it is specified.
See \secref{sec:detection} for further information.
\item
The filename \textit{main} must be specified without the |.tex| extension.
\item
The filename \textit{main} is case sensitive
(even in case-insensitive file systems)
due to internal string comparison.
\item
The argument \textit{main} should be fully expanded, it cannot be a macro.
\item
Subdirectories and special characters should be avoided in filenames.
\item
The command |\childdocmain{|\textit{main}|}| must be followed by a whitespace.
It should not be followed immediately by another command
or by a comment mark `|%|'.
This is because the \TeX{} parser reads the token immediately following
the argument of |\childdocmain| and puts it
at the beginning of every child section;
however, a white\-space is ignored.
\end{itemize}

%%%%%%%%%%%%%%%%%%%%%%%%%%%%%%%%%%%%%%%%
\paragraph{Content of Main File.}

It is advisable to place all content in the child files included by |\include|.
Any output contained in the main file will appear in all child documents
unless suppressed manually;
it cannot be suppressed automatically by the |\includeonly| directive
and thus should normally be avoided.
A method to include some content in the main file
by means of conditional processing is described in \secref{sec:conditional}.

%%%%%%%%%%%%%%%%%%%%%%%%%%%%%%%%%%%%%%%%
\paragraph{Page Numbering.}

When only a part of the document is compiled,
the appropriate numbering of pages
(as well as other status parameters)
is determined from the |.aux| files.
The latter contain information from previous passes.
However this information needs to propagate through
all intermediate child documents.
Therefore the page numbering in child documents may well
be inconsistent until the complete document is compiled at least once.

A useful (if unconventional) way to always ensure a consistent
page numbering is to restart the numbering in each child document
and denote the pages by `\textit{child}|.|\textit{page}'
where \textit{child} represents the chapter/section number of the child file.
This can be achieved by the command
|\numberwithin{page}{|\textit{child}|}|
of the \textsf{amsmath} package
where \textit{child} can be |chapter| or |section|
depending on the chosen structuring.
Alternatively, one can modify the macro |\thepage| appropriately
and reset the counter |page| at the start of each child file.

%%%%%%%%%%%%%%%%%%%%%%%%%%%%%%%%%%%%%%%%%%%%%%%%%%%%%%%%%%%%%%%%%%%%%%%%%%%%%%%%
\subsection{Conditional Processing}
\label{sec:conditional}

The package provides a mechanism to compile different versions
of a document. To customise the versions further some conditional processing
can come in handy to distinguish which version is being compiled.
The package provides two macros to describe the compilation context:

%%%%%%%%%%%%%%%%%%%%%%%%%%%%%%%%%%%%%%%%
\DescribeMacro{\ifchilddoc}
The conditional |\ifchilddoc| distinguishes between the compilation of
child documents and the main document:
%
\begin{center}
|\ifchilddoc |\textit{child-code}| |[|\||else |\textit{main-code}]| \||fi|
\end{center}

%%%%%%%%%%%%%%%%%%%%%%%%%%%%%%%%%%%%%%%%
\DescribeMacro{\childdocname}
\DescribeMacro{\childdocjob}
The macro |\childdocname| contains the filename (without extension)
of the main or child file being processed.
Note that |\childdocjob| will always contain the name of the main file.

%%%%%%%%%%%%%%%%%%%%%%%%%%%%%%%%%%%%%%%%
\paragraph{Title Page.}

Conditional processing can be used to include a title or banner page
in the main document when proper precautions are taken.
Importantly, the code in the main file should ensure that the page counter
(as well as other status parameters which are stored in the |.aux| files)
takes the same value after the conditional processing.
Otherwise the page numbers may take divergent values
depending on which part is compiled.

For example, a title page could be declared by:
%
\begin{center}
\begin{tabular}{l}
|\ifchilddoc\||else|\\
|\addtocounter{page}{-1}|\\
\textit{code for title page}\\
|\newpage|\\
|\||fi|
\end{tabular}
\end{center}
%
A banner page for the child documents can be generated by:
%
\begin{center}
\begin{tabular}{l}
|\ifchilddoc|\\
|\addtocounter{page}{-1}|\\
\textit{code for banner page}\\
|\newpage|\\
|\||fi|
\end{tabular}
\end{center}
%
Here one could write a message such as:
\begin{center}
|This is the part \childdocname{} of \childdocjob{}.|
\end{center}

%%%%%%%%%%%%%%%%%%%%%%%%%%%%%%%%%%%%%%%%%%%%%%%%%%%%%%%%%%%%%%%%%%%%%%%%%%%%%%%%
\subsection{Flags}
\label{sec:flags}

The package makes it easy to generate different versions
of the main or child documents.
To this end compilation flags can be defined
and assigned different default values.
They will be particularly useful in conjunction
with the forwarding mechanism described in \secref{sec:forward}.

For example, it may be useful to have a flag |\version|
which can be set to |draft| or |final|.
The document source will contain some conditional code
depending on the value of |\version|.
Suppose further, the flag should default to |final| for the main file
and to |draft| for child files
which is a natural assignment for editing the document.
This is achieved by placing the following code
in the preamble of the main document
(below the |\childdocmain| directive):
%
\begin{center}
\begin{tabular}{l}
|\ifchilddoc|\\
|\providecommand{\version}{draft}|\\
|\||else|\\
|\providecommand{\version}{final}|\\
|\||fi|
\end{tabular}
\end{center}
%
The definition by |\providecommand| makes sure
that previous definitions are not overwritten.
Further statements |\providecommand{\version}{...}|
can thus be added before the above code to override it.

For the main file, one might add a line
(between |\childdocmain| and the above block)
%
\begin{center}
|%\ifchilddoc\||else\providecommand{\version}{draft}\||fi|
\end{center}
%
which can be uncommented to produce a draft version.
Likewise one can add a line to the very top of a child file
(above the |\childdocof{|\textit{main}|}| directive)
%
\begin{center}
|%\providecommand{\version}{final}|
\end{center}
%
which can be uncommented to produce the final version of this child document.

%%%%%%%%%%%%%%%%%%%%%%%%%%%%%%%%%%%%%%%%%%%%%%%%%%%%%%%%%%%%%%%%%%%%%%%%%%%%%%%%
\subsection{Forwarding}
\label{sec:forward}

Different versions of the main or child documents
using compilation flags as described in \secref{sec:flags}
can be (permanently) stored in different files
for convenient compilation, viewing and distribution.
To this end, the package defines a command
to pass on compilation to a different file:

%%%%%%%%%%%%%%%%%%%%%%%%%%%%%%%%%%%%%%%%
\DescribeMacro{\childdocforward}
The command |\childdocforward| redirects processing to
another source file:
%
\begin{center}
\begin{tabular}{l}
|\input{childdoc.def}|\\
|\childdocforward[|\textit{main}|]{|\textit{dest}|}|\\
\end{tabular}
\end{center}
%
The argument \textit{dest} is the destination file
(without extension).
It should be the main file or one of the child files.
Note that further \textsf{childdoc} directives
such as |\childdocof| and |\childdocforward|
in the indicated file will be processed in this form.
The optional argument \textit{main}
passes on directly to the main file \textit{main}
while pretending to compile the child \textit{dest}.
This form behaves as if \textit{dest}
issues |\childdocof{|\textit{main}|}| right away,
and no further \textsf{childdoc} directives will be processed.

%%%%%%%%%%%%%%%%%%%%%%%%%%%%%%%%%%%%%%%%
\DescribeMacro{\...prefix}
In the alternative form |\childdocforwardprefix|,
%
\begin{center}
\begin{tabular}{l}
|\input{childdoc.def}|\\
|\childdocforwardprefix[|\textit{main}|]{|\textit{prefix}|}{|\textit{dest}|}|
\end{tabular}
\end{center}
%
the destination file is determined by a pattern
depending on the current file:
To make this work, the current file must be called
`{\textit{prefix}\hspace{0.2em}\textit{suffix}}'
with \textit{prefix} matching precisely the argument.
Processing is then passed on to the file
`{\textit{dest}\hspace{0.2em}\textit{suffix}}'.
Surely, the same effect is achieved by
directly specifying the
argument `{\textit{dest}\hspace{0.2em}\textit{suffix}}'
in the first form.
However, that requires to set up a different file
for each child. With the alternative form of the command
all these files can have exactly the same content
which simplifies setting them up and maintaining them.

For example, the following file |draft.tex|
with a compilation flag |\version| as described in \secref{sec:flags}
compiles the main document as a draft:
%
\begin{center}
\begin{tabular}{l}
|\def\version{draft}|\\
|\input{childdoc.def}|\\
|\childdocforward{|\textit{main}|}|
\end{tabular}
\end{center}
%
Likewise, the following files |final|\textit{nn}|.tex|
compile the final version of the child document
|child|\textit{nn}|.tex|:
%
\begin{center}
\begin{tabular}{l}
|\def\version{final}|\\
|\input{childdoc.def}|\\
|\childdocforwardprefix{final}{child}|
\end{tabular}
\end{center}
%

Note that when several versions of a main file and/or of each child file
are to be generated, it may be convenient to set up a |Makefile| or
shell script to automatise the process.

%%%%%%%%%%%%%%%%%%%%%%%%%%%%%%%%%%%%%%%%%%%%%%%%%%%%%%%%%%%%%%%%%%%%%%%%%%%%%%%%
\subsection{Command Line Processing}
\label{sec:commandline}

The effect of redirection files can also be achieved by invoking
the \LaTeX{} compiler with a more elaborate command line.
Most conveniently this should be done as part
of a shell script or a |Makefile|.

When using \textsf{childdoc} in the main file, the following
command lines effectively perform a redirection
(note that depending on the shell being used,
backslashes may have to be doubled: `|\|' $\to$ `|\\|'):
%
\begin{center}
|... -jobname "|\textit{target}|" |\\|"|[\textit{flags}]%
|\input{childdoc.def}\childdocforward[|\textit{main}|]{|\textit{dest}|}"|
\end{center}
%
Here \textit{target} is the name of the output file,
\textit{main} is the name of the main file
and \textit{dest} is the name of the main or child file to be processed
(all filenames without extensions).
The optional argument \textit{main} can be omitted
if \textit{main} matches \textit{dest}.
Optionally, compilation \textit{flags} can be defined via |\def| commands.
This command line makes the \TeX{} engine believe
it is compiling the file \textit{target}
whose content is specified as the latter parameter.
The provided code then forwards the processing to
\textit{main} or \textit{dest} as described in \secref{sec:forward}.

%%%%%%%%%%%%%%%%%%%%%%%%%%%%%%%%%%%%%%%%%%%%%%%%%%%%%%%%%%%%%%%%%%%%%%%%%%%%%%%%
\subsection{Include by Input}
\label{sec:input}

Including child documents by |\include| has some restrictions by design.
Most notably, the content of a child document always occupies
its own set of pages; pages cannot be shared between child documents.
Usually, this behaviour makes perfect sense
because each child document contain an essential part of the document.
However, in some situations it may be desirable to compose
a document from a collection of parts
without having mandatory page breaks between then.
For this case, the package
provides a mechanism to include parts
by |\input| which can also be processed individually.
However, by construction this mechanism
requires manual handling of the content to be output.

%%%%%%%%%%%%%%%%%%%%%%%%%%%%%%%%%%%%%%%%
\DescribeMacro{\ifchilddocmanual}
The main file should be prepared as usual, see \secref{sec:include}.
However, the document body must make a distinction
between processing of an individual part and of the main document, e.g.:
%
\begin{center}
\begin{tabular}{l}
|\ifchilddocmanual|\\
|\input{\childdocname}|\\
|\||else|\\
\textit{document body with }|\input{|\textit{part}|}|\\
|\||fi|
\end{tabular}
\end{center}
%
The conditional |\ifchilddocmanual| is true whenever
a part to be included by |\input| is being compiled,
and the name of the part is stored in |\childdocname|.

%%%%%%%%%%%%%%%%%%%%%%%%%%%%%%%%%%%%%%%%
\DescribeMacro{\childdocby}
Each part to be included by |\input| should start with:
%
\begin{center}
\begin{tabular}{l}
|\input{childdoc.def}|\\
|\childdocby{|\textit{main}|}|\\
\end{tabular}
\end{center}
%
The directive |\childdocby| is similar to |\childdocof|
described in \secref{sec:include},
but the subsequent selection of content must be done manually.
To that end, both |\ifchilddoc| and |\ifchilddocmanual|
will be true upon processing of a part,
and the name of the part is stored in |\childdocname|.
Note that |\jobname| will be set to the filename of the current part
so that each part receives an individual |.aux| file
that does not interfere with the |.aux| file(s) of the main document.
This behaviour can be altered by the alternative form
|\childdocby[*]{|\textit{main}|}| (with a non-empty optional argument)
which uses the |.aux| file of the main document
by setting |\jobname| to \textit{main}.

%%%%%%%%%%%%%%%%%%%%%%%%%%%%%%%%%%%%%%%%%%%%%%%%%%%%%%%%%%%%%%%%%%%%%%%%%%%%%%%%
\subsection{Driver Development}
\label{sec:driver}

The \textsf{childdoc} mechanism can also be use for the development
of definition files such as \LaTeX{} styles or classes.
This case differs from the above setup with multiple parts
included by |\include| in that no |\includeonly| should be invoked.
This can be achieved by starting the include file
(before |\ProvidesPackage|) with:
%
\begin{center}
\begin{tabular}{l}
|\input{childdoc.def}|\\
|\childdocforward{|\textit{main}|}|\\
\end{tabular}
\end{center}
%
or alternatively with:
%
\begin{center}
\begin{tabular}{l}
|\input{childdoc.def}|\\
|\childdocby{|\textit{main}|}|\\
\end{tabular}
\end{center}
%
Both forms have slightly different effects as described above.
The main file is prepared as usual, see \secref{sec:include}.

%%%%%%%%%%%%%%%%%%%%%%%%%%%%%%%%%%%%%%%%%%%%%%%%%%%%%%%%%%%%%%%%%%%%%%%%%%%%%%%%
\subsection{Legacy Detection}
\label{sec:detection}

The directive |\childdocmain| in the main file can detect
whether the complete document or merely a child is to be compiled
even without using the directive |\childdocof|.
This method is deprecated because it is less robust
and there is no compelling reason to use it;
it is merely provided for backward compatibility
and it may be removed in future versions.

If the detection mechanism is to be used,
it is mandatory to correctly specify
the filename of the main file as the argument of |\childdocmain|:
%
\begin{center}
\begin{tabular}{l}
|\input{childdoc.def}|\\
|\childdocmain{|\textit{main}|}|\\
\end{tabular}
\end{center}
%
If |\jobname| does not match the argument \textit{main} of |\childdocmain|,
it is assumed that |\jobname| points to the child file to be compiled.
When using |\childdocmain| with the main file specified as argument,
it suffices to start a child file
with just |\input{|\textit{main}|}|
without loading of the package and using |\childdocof|.
If instead all processing is done
with the appropriate \textsf{childdoc} directives,
the argument of \textit{main} of |\childdocmain| can be empty.

An alternative version of the command line processing described
in \secref{sec:commandline} using the detection mechanism reads:
%
\begin{center}
|... -jobname "|\textit{target}|" "|[\textit{flags}]%
[|\def\jobname{|\textit{dest}|}|]|\input{|\textit{main}|}"|
\end{center}

%%%%%%%%%%%%%%%%%%%%%%%%%%%%%%%%%%%%%%%%%%%%%%%%%%%%%%%%%%%%%%%%%%%%%%%%%%%%%%%%
\subsection{Manual Code}
\label{sec:manual}

In case one cannot be certain whether the definitions file |childdoc.def|
is installed on the target \TeX{} distribution
and one prefers not to ship it,
it is conceivable to paste a few relevant commands into the sources.

To that end, drop all statements |\input{childdoc.def}|
and perform the replacements as outlined below.
Instead of |\childdocmain{|\textit{main}|}| add the following code
to the top of the main file:
%
\begin{center}
\begin{tabular}{l}
|\||ifdefined\childdocname\endinput\||fi\newif\ifchilddoc|\\
|\edef\childdocname{\scantokens\expandafter{\jobname\noexpand}}|\\
|\def\childdocmain{|\textit{main}|}\||ifx\childdocmain\childdocname\||else|\\
|\childdoctrue\includeonly{\childdocname}\let\jobname\childdocmain\||fi|\\
\end{tabular}
\end{center}
%
Instead of |\childdocof{|\textit{main}|}| just include the main file
at the top of each child file:
%
\begin{center}
|\input{|\textit{main}|}|
\end{center}
%
A simple redirection |\childdocforward{|\textit{dest}|}| is achieved by:
%
\begin{center}
|\def\jobname{|\textit{dest}|}\input{\jobname}|
\end{center}
%
The redirection with prefix
|\childdocforwardprefix[|\textit{prefix}|]{|\textit{dest}|}|
is accomplished by:
%
\begin{center}
\begin{tabular}{l}
|{\edef\jobname{\scantokens\expandafter{\jobname\noexpand}}|\\
|\def\redirectjob |\textit{prefix}|#1~~~{\gdef\jobname{|\textit{dest}|#1}}|\\
|\expandafter\redirectjob\jobname~~~}\input{\jobname}|
\end{tabular}
\end{center}

In an alternative approach,
child documents can be compiled by a specific command line
without additional code or specific definitions:
%
\begin{center}
|... -jobname "|\textit{target}|" "|[\textit{flags}]%
|\includeonly{|\textit{dest}|}\input{|\textit{main}|}"|
\end{center}
%

%%%%%%%%%%%%%%%%%%%%%%%%%%%%%%%%%%%%%%%%%%%%%%%%%%%%%%%%%%%%%%%%%%%%%%%%%%%%%%%%
%%%%%%%%%%%%%%%%%%%%%%%%%%%%%%%%%%%%%%%%%%%%%%%%%%%%%%%%%%%%%%%%%%%%%%%%%%%%%%%%
\section{Information}

%%%%%%%%%%%%%%%%%%%%%%%%%%%%%%%%%%%%%%%%%%%%%%%%%%%%%%%%%%%%%%%%%%%%%%%%%%%%%%%%
\subsection{Copyright}

Copyright \copyright{} 2017--2018 Niklas Beisert

This work may be distributed and/or modified under the
conditions of the \LaTeX{} Project Public License, either version 1.3
of this license or (at your option) any later version.
The latest version of this license is in
  \url{http://www.latex-project.org/lppl.txt}
and version 1.3 or later is part of all distributions of \LaTeX{}
version 2005/12/01 or later.

This work has the LPPL maintenance status `maintained'.

The Current Maintainer of this work is Niklas Beisert.

This work consists of the files |README.txt|, |childdoc.ins| and |childdoc.dtx|
as well as the derived files |childdoc.def|, |cdocsamp.tex|
with |cdocsch1.tex|, |cdocsch2.tex|, |cdocspt3.tex|, |cdocspt4.tex|,
|cdocsdrf.tex|, |cdocsfn1.tex|, |cdocsfn2.tex|
as well as |childdoc.pdf|.

%%%%%%%%%%%%%%%%%%%%%%%%%%%%%%%%%%%%%%%%%%%%%%%%%%%%%%%%%%%%%%%%%%%%%%%%%%%%%%%%
\subsection{Files and Installation}

The package consists of the files:
%
\begin{center}
\begin{tabular}{ll}
    |README.txt|   & readme file \\
    |childdoc.ins| & installation file \\
    |childdoc.dtx| & source file \\
    |childdoc.def| & definition file \\
    |cdocsamp.tex| & sample main file \\
    |cdocsch1.tex| & sample include file \\
    |cdocsch2.tex| & sample include file \\
    |cdocspt3.tex| & sample part file \\
    |cdocspt4.tex| & sample part file \\
    |cdocsdrf.tex| & sample redirection file \\
    |cdocsfn1.tex| & sample redirection file \\
    |cdocsfn2.tex| & sample redirection file \\
    |childdoc.pdf| & manual
\end{tabular}
\end{center}
%
The distribution consists of the files
|README.txt|, |childdoc.ins| and |childdoc.dtx|.
%
\begin{itemize}
\item
Run (pdf)\LaTeX{} on |childdoc.dtx|
to compile the manual |childdoc.pdf| (this file).
\item
Run \LaTeX{} on |childdoc.ins| to create the definitions file |childdoc.def|
and the sample |cdocsamp.tex| with include files
|cdocsch1.tex|, |cdocsch2.tex|, |cdocspt3.tex|, |cdocspt4.tex|,
|cdocsdrf.tex|, |cdocsfn1.tex|, |cdocsfn2.tex|.
Then copy the file |childdoc.def| to an appropriate directory of your \LaTeX{}
distribution, e.g.\ \textit{texmf-root}|/tex/latex/childdoc|.
\end{itemize}

%%%%%%%%%%%%%%%%%%%%%%%%%%%%%%%%%%%%%%%%%%%%%%%%%%%%%%%%%%%%%%%%%%%%%%%%%%%%%%%%
\subsection{Related CTAN Packages}

There are several other packages which offer a similar functionality:
%
\begin{itemize}
\item
The packages
\href{http://ctan.org/pkg/docmute}{\textsf{docmute}},
\href{http://ctan.org/pkg/includex}{\textsf{includex}} and
\href{http://ctan.org/pkg/standalone}{\textsf{standalone}}
provide commands to include only the document body of
a child file thus allowing both files to be compiled individually.
\item
The packages \href{http://ctan.org/pkg/subdocs}{\textsf{subdocs}}
and \href{http://ctan.org/pkg/subfiles}{\textsf{subfiles}}
provide structures in which the main and child documents can be
encapsulated and allowing them to be compiled individually.
The inclusion mechanism is different from the conventional |\include|.
\item
The package \href{http://ctan.org/pkg/combine}{\textsf{combine}}
is an elaborate solution to combine several documents into one.
\end{itemize}
%
See also the CTAN topic \href{http://ctan.org/topic/subdocs}{\textsf{subdocs}}
for further related packages.
The present package differs from the above solutions in that
a document structure constructed with the conventional |\include| mechanism
just needs two extra commands at the top of every file
such that all constituent files can be compiled individually.

%%%%%%%%%%%%%%%%%%%%%%%%%%%%%%%%%%%%%%%%%%%%%%%%%%%%%%%%%%%%%%%%%%%%%%%%%%%%%%%%
%\subsection{Feature Suggestions}
%
%The following is a list of features which may be useful for future
%versions of this package:
%%
%\begin{itemize}
%\item
%\ldots
%\end{itemize}

%%%%%%%%%%%%%%%%%%%%%%%%%%%%%%%%%%%%%%%%%%%%%%%%%%%%%%%%%%%%%%%%%%%%%%%%%%%%%%%%
\subsection{Revision History}

%%%%%%%%%%%%%%%%%%%%%%%%%%%%%%%%%%%%%%%%
\paragraph{v2.0:} 2018/12/30

\begin{itemize}
\item
immediate forward processing
\item
added |\childdocby| mechanism
\item
manual restructured
\end{itemize}

%%%%%%%%%%%%%%%%%%%%%%%%%%%%%%%%%%%%%%%%
\paragraph{v1.6:} 2018/01/17

\begin{itemize}
\item
application for development of include files
\item
corrections to manual
\end{itemize}

%%%%%%%%%%%%%%%%%%%%%%%%%%%%%%%%%%%%%%%%
\paragraph{v1.5:} 2017/05/21

\begin{itemize}
\item
more complete structuring introduced
\item
|\childdocof| introduced
\item
|\childdoc| renamed to |\childdocmain|
\item
|\childredirect| renamed to |\childdocforward| and |\childdocforwardprefix|
and functionality expanded
\end{itemize}

%%%%%%%%%%%%%%%%%%%%%%%%%%%%%%%%%%%%%%%%
\paragraph{v1.0:} 2017/04/27

\begin{itemize}
\item
manual and install package
\item
first version published on CTAN
\end{itemize}

%%%%%%%%%%%%%%%%%%%%%%%%%%%%%%%%%%%%%%%%
\paragraph{v0.6:} 2017/04/26

\begin{itemize}
\item
redirection mechanism added
\end{itemize}

%%%%%%%%%%%%%%%%%%%%%%%%%%%%%%%%%%%%%%%%
\paragraph{v0.5:} 2017/04/26

\begin{itemize}
\item
functionality in definition file
\end{itemize}


%%%%%%%%%%%%%%%%%%%%%%%%%%%%%%%%%%%%%%%%%%%%%%%%%%%%%%%%%%%%%%%%%%%%%%%%%%%%%%%%
%%%%%%%%%%%%%%%%%%%%%%%%%%%%%%%%%%%%%%%%%%%%%%%%%%%%%%%%%%%%%%%%%%%%%%%%%%%%%%%%
%%%%%%%%%%%%%%%%%%%%%%%%%%%%%%%%%%%%%%%%%%%%%%%%%%%%%%%%%%%%%%%%%%%%%%%%%%%%%%%%
\appendix

\settowidth\MacroIndent{\rmfamily\scriptsize 000\ }

 \DocInput{childdoc.dtx}

\end{document}
%</driver>
% \fi
%
% %%%%%%%%%%%%%%%%%%%%%%%%%%%%%%%%%%%%%%%%%%%%%%%%%%%%%%%%%%%%%%%%%%%%%%%%%%%%%%
% %%%%%%%%%%%%%%%%%%%%%%%%%%%%%%%%%%%%%%%%%%%%%%%%%%%%%%%%%%%%%%%%%%%%%%%%%%%%%%
% \section{Sample}
%\iffalse
%<*samplemain>
%\fi
%
% The following presents a sample document
% with two chapters, two parts, a title page,
% a compile flag as well as three forwarding files to set the flag.
% It consists of eight |.tex| files:
% \begin{center}
% \begin{tabular}{ll}
% |cdocsamp.tex|&main file\\
% |cdocsch1.tex|&include file for chapter 1\\
% |cdocsch2.tex|&include file for chapter 2\\
% |cdocspt3.tex|&include file for part 3\\
% |cdocspt4.tex|&include file for part 4\\
% |cdocsdrf.tex|&forwarding file for main file in draft mode\\
% |cdocsfi1.tex|&forwarding file for final version of chapter 1\\
% |cdocsfi2.tex|&forwarding file for final version of chapter 2\\
% \end{tabular}
% \end{center}
% Each of the eight files can be compiled directly by the \LaTeX{} compiler.
%
% %%%%%%%%%%%%%%%%%%%%%%%%%%%%%%%%%%%%%%
% \paragraph{Main File.}
%
% The main file is called |cdocsamp.tex|.
%
% Load the \textsf{childdoc} definitions and
% declare the filename for the main document:
%    \begin{macrocode}
\input{childdoc.def}
\childdocmain{}
%    \end{macrocode}

% Optional override for |\version| flag:
%    \begin{macrocode}
%%\ifchilddoc\else\providecommand{\version}{draft}\fi
%    \end{macrocode}

% Define the default values for the |\version| flag
% (|final| for the main file and |draft| for childs):
%    \begin{macrocode}
\ifchilddoc
\providecommand{\version}{draft}
\else
\providecommand{\version}{final}
\fi
%    \end{macrocode}

% Load the standard document class:
%    \begin{macrocode}
\documentclass[12pt]{article}
%    \end{macrocode}

% Start the document body:
%    \begin{macrocode}
\begin{document}
%    \end{macrocode}

% Declare a title page.
% Print title, part of document being processed and version flag:
%    \begin{macrocode}
\addtocounter{page}{-1}
\begin{center}
{\LARGE\bfseries{}childdoc example\par}
\vspace{1cm}
\ifchilddoc
\ifchilddocmanual part\else chapter\fi:
`\childdocname' of `\childdocjob'\par
\else
main document: `\childdocjob'\par
\fi
version: \version\par
\end{center}
\newpage
%    \end{macrocode}

% Manually include selected file,
% otherwise process as usual:
%    \begin{macrocode}
\ifchilddocmanual
\section*{part `\childdocname'}
\input{\childdocname}
\else
%    \end{macrocode}

% Include the two chapters:
%    \begin{macrocode}
\include{cdocsch1}
\include{cdocsch2}
%    \end{macrocode}

% Include the two parts unless only chapters should be displayed:
%    \begin{macrocode}
\ifchilddoc\else
\section{part three}
\input{cdocspt3}
\section{part four}
\input{cdocspt4}
\fi
%    \end{macrocode}

% Process as usual until here:
%    \begin{macrocode}
\fi
%    \end{macrocode}

% End of document body:
%    \begin{macrocode}
\end{document}
%    \end{macrocode}
%\iffalse
%</samplemain>
%\fi
%
% %%%%%%%%%%%%%%%%%%%%%%%%%%%%%%%%%%%%%%
% \paragraph{Chapter Include Files.}
%
% The include files are called |cdocsch1.tex| and |cdocsch2.tex|.
%
%\iffalse
%<*samplechap1|samplechap2>
%\fi

% Optional override for |\version| flag:
%    \begin{macrocode}
%%\providecommand{\version}{final}
%    \end{macrocode}

% Include the main document:
%    \begin{macrocode}
\input{childdoc.def}
\childdocof{cdocsamp}
%    \end{macrocode}

%\iffalse
%</samplechap1|samplechap2>
%\fi
%
%\iffalse
%<*samplechap1>
%\fi
% Some text for chapter 1:
%    \begin{macrocode}
\section{one}
some text in chapter one
%    \end{macrocode}

%\iffalse
%</samplechap1>
%\fi
% Some text for chapter 2:
%\iffalse
%<*samplechap2>
%\fi
%    \begin{macrocode}
\section{two}
more text in chapter two
%    \end{macrocode}

%\iffalse
%</samplechap2>
%\fi
%
% %%%%%%%%%%%%%%%%%%%%%%%%%%%%%%%%%%%%%%
% \paragraph{Part Include Files.}
%
% The include files are called |cdocspt3.tex| and |cdocspt4.tex|.
%
%\iffalse
%<*samplepart3|samplepart4>
%\fi

% Optional override for |\version| flag:
%    \begin{macrocode}
%%\providecommand{\version}{final}
%    \end{macrocode}

% Include the main document:
%    \begin{macrocode}
\input{childdoc.def}
\childdocby{cdocsamp}
%    \end{macrocode}

%\iffalse
%</samplepart3|samplepart4>
%\fi
%
%\iffalse
%<*samplepart3>
%\fi
% Some text for part 3:
%    \begin{macrocode}
some text in part three
%    \end{macrocode}

%\iffalse
%</samplepart3>
%\fi
% Some text for part 4:
%\iffalse
%<*samplepart4>
%\fi
%    \begin{macrocode}
more text in part four
%    \end{macrocode}

%\iffalse
%</samplepart4>
%\fi
%
% %%%%%%%%%%%%%%%%%%%%%%%%%%%%%%%%%%%%%%
% \paragraph{Forwarding for a Complete Draft.}
%
% The following forwarding file |cdocsdrf.tex|
% compiles the main document in draft mode:
%\iffalse
%<*sampledraft>
%\fi
%    \begin{macrocode}
\def\version{draft}
\input{childdoc.def}
\childdocforward{cdocsamp}
%    \end{macrocode}

%\iffalse
%</sampledraft>
%\fi
%
% %%%%%%%%%%%%%%%%%%%%%%%%%%%%%%%%%%%%%%
% \paragraph{Forwarding for Final Version of the Chapters.}
%
% The following forwarding files |cdocsfn1.tex| and |cdocsfn2.tex|
% (with identical content)
% compile the final versions of the child documents
% |cdocsch1.tex| and |cdocsch2.tex|, respectively:
%\iffalse
%<*samplefinal>
%\fi
%    \begin{macrocode}
\def\version{final}
\input{childdoc.def}
\childdocforwardprefix[cdocsamp]{cdocsfn}{cdocsch}
%    \end{macrocode}

%\iffalse
%</samplefinal>
%\fi
%
% %%%%%%%%%%%%%%%%%%%%%%%%%%%%%%%%%%%%%%
% \paragraph{Command Line Processing.}
%
% The following three command lines generate the output files
% |cdocscld|, |cdocscl1| and |cdocscl2|
% which should be identical to
% |cdocsdrf|, |cdocsch1| and |cdocsfn2|, respectively:
% \begin{center}
% \begin{tabular}{l}
% |latex -jobname cdocscld \|\\
% |  "\def\version{draft}\input{childdoc.def}\childdocforward{cdocsamp}"|\\
% |latex -jobname cdocscl1 \|\\
% |  "\input{childdoc.def}\childdocforward[cdocsamp]{cdocsch1}"|\\
% |latex -jobname cdocscl2 \|\\
% |  "\def\version{final}\input{childdoc.def}\childdocforward{cdocsch2}"|
% \end{tabular}
% \end{center}
% Note that the trailing backslash on each first line
% merely continues the input to the second line
% (for convenient cut ant paste).
% Furthermore, the command |latex| can be replaced by any
% of its alternative versions such as |pdflatex|.
%
% %%%%%%%%%%%%%%%%%%%%%%%%%%%%%%%%%%%%%%%%%%%%%%%%%%%%%%%%%%%%%%%%%%%%%%%%%%%%%%
% %%%%%%%%%%%%%%%%%%%%%%%%%%%%%%%%%%%%%%%%%%%%%%%%%%%%%%%%%%%%%%%%%%%%%%%%%%%%%%
% \section{Implementation}
%\iffalse
%<*package>
%\fi
%
% This section describes the definitions file |childdoc.def|.

% The definitions cannot be loaded using |\usepackage| or |\RequirePackage|
% which has a mechanism to prevent loading a style file more than once.
% When loading the definitions by means of |\input|
% multiple instances have to be prevented manually:
%\iffalse
%This code needs to be before the `\ProvidesFile' directive
%which is defined at the beginning of this file.
%Therefore it is also placed there and commented out here.
%</package>
%<*discard>
%\fi
%    \begin{macrocode}
\ifdefined\childdocmain\endinput\fi
%    \end{macrocode}
%\iffalse
%</discard>
%<*package>
%\fi
%
% \macro{\ifchilddoc}
% \macro{\ifchilddocmanual}
% The conditional |\ifchilddoc| tells whether a
% child (true) or main (false) document is being compiled.
% The conditional |\ifchilddocmanual| tells whether
% the |\includeonly| mechanism is used (false) or
% the selection of child files must be performed manually (true).
% The definitions initialise to false:
%    \begin{macrocode}
\newif\ifchilddoc
\newif\ifchilddocmanual
%    \end{macrocode}

% \macro{\childdocname}
% \macro{\childdocjob}
% The macro |\childdocname| stores the name of the main document
% to be compiled. The macro |\childdocjob| stores the name of
% the document on which the \LaTeX{} compiler was originally invoked.
% The content of |\jobname| cannot be compared
% to filenames specified in the source due to different catcodes.
% The following code rescans |\jobname|, stores the result
% in |\childdocname| and saves a copy in |\childdocjob|:
%    \begin{macrocode}
\edef\childdocname{\scantokens\expandafter{\jobname\noexpand}}
\let\childdocjob\childdocname
%    \end{macrocode}

% \macro{\childdocdisable}
% The macro |\childdocdisable| prevents the main file
% from being processed more than once.
% At this stage, the main document command |\childdocmain|
% is assumed to be called once again where it should do nothing.
% Any subsequent call to it should prevent
% a secondary processing of the main document
% It overwrites the forwarding commands
% |\childdocof| and |\childdocforward|
% with empty macros to prevent further inclusions of the main document:
%    \begin{macrocode}
\newcommand{\childdocdisable}
{
  \renewcommand{\childdocmain}[1]{\renewcommand{\childdocmain}[1]{\endinput}}
  \renewcommand{\childdocof}[1]{}
  \renewcommand{\childdocby}[2][]{}
  \renewcommand{\childdocforward}[2][]{}
  \renewcommand{\childdocdisable}{}
}
%    \end{macrocode}

% \macro{\childdocmain}
% The macro |\childdocmain| is to be called at the top of the main file
% with nothing or the main filename (without extension) as argument.
% First, it breaks loops.
% If the argument is not empty and does not match |\childdocname|
% (which is set by the first inclusion of |childdoc.def|),
% |\ifchilddoc| is set to true, |\includeonly| is applied to the child file
% and |\jobname| is set to the main file
% (for proper handling of |.aux| files):
%    \begin{macrocode}
\newcommand{\childdocmain}[1]
{
  \childdocdisable\childdocmain{}
  \if?#1?\else
    \begingroup
      \def\childdoctmp{#1}
      \ifx\childdoctmp\childdocname
        \def\childdoctmp{}
      \else
        \def\childdoctmp
        {
          \childdoctrue
          \includeonly{\childdocname}
          \def\childdocjob{#1}
          \def\jobname{#1}
        }
      \fi
      \expandafter
    \endgroup
    \childdoctmp
  \fi
}
%    \end{macrocode}

% \macro{\childdocof}
% The command |\childdocof| redirects
% compilation to the main file |#1|.
%    \begin{macrocode}
\newcommand{\childdocof}[1]
{
  \childdocdisable
  \childdoctrue
  \includeonly{\childdocname}
  \def\jobname{#1}
  \def\childdocjob{#1}
  \input{#1}
}
%    \end{macrocode}

% \macro{\childdocby}
% The command |\childdocby| ....
%    \begin{macrocode}
\newcommand{\childdocby}[2][]
{
  \childdocdisable
  \childdoctrue
  \childdocmanualtrue
  \if?#1?\else
    \def\jobname{#2}
  \fi
  \def\childdocjob{#2}
  \input{#2}
  \endinput
}
%    \end{macrocode}

% \macro{\childdocforward}
% The command |\childdocforward| redirects
% compilation to the main file or
% (if the optional argument is given) a child file.
% Parameters are set as if the main file
% or a child file starting with |\childdocof| was compiled.
% Then compilation is handed over to the main file:
%    \begin{macrocode}
\newcommand{\childdocforward}[2][]
{
  \begingroup
    \if?#1?
      \def\childdoctmp
      {
        \def\childdocname{#2}
        \def\childdocjob{#2}
        \def\jobname{#2}
        \input{#2}
        \endinput
      }
    \else
      \def\childdoctmp
      {
        \childdocdisable
        \def\childdocname{#2}
        \childdoctrue
        \includeonly{#2}
        \def\childdocjob{#1}
        \def\jobname{#1}
        \input{#1}
        \endinput
      }
    \fi
    \expandafter
  \endgroup
  \childdoctmp
}
%    \end{macrocode}

% \macro{\childdocforwardprefix}
% The command |\childdocforwardprefix| redirects
% compilation to the main or a child file by means of a pattern.
% The prefix |#1| in the current filename is replaced by |#2|
% and the suffix of the current filename is kept
% (it is assumed that the filename does not contain the substring `|~~~|'
% which is used as a delimiter).
% Compilation is handed over to the new file by |\childdocforward|:
%    \begin{macrocode}
\newcommand{\childdocforwardprefix}[3][]
{
  \begingroup
    \def\childdocextract #2##1~~~{\def\childdoctmp{\childdocforward[#1]{#3##1}}}
    \expandafter\childdocextract\childdocname~~~
    \expandafter
  \endgroup
  \childdoctmp
}
%    \end{macrocode}

% \macro{\childdoc}
% The deprecated macro |\childdoc| is a legacy version of |\childdocmain|:
%    \begin{macrocode}
\newcommand{\childdoc}{\childdocmain}
%    \end{macrocode}

% \macro{\childdocredirect}
% The deprecated macro |\childdocredirect| is a legacy version
% of |\childdocforward| and |\childdocforwardprefix|:
%    \begin{macrocode}
\newcommand{\childdocredirect}[2][]
{
  \begingroup
    \if?#1?
      \def\childdoctmp{\childdocforward{#2}}
    \else
      \def\childdoctmp{\childdocforwardprefix{#1}{#2}}
    \fi
    \expandafter
  \endgroup
  \childdoctmp
}
%    \end{macrocode}

%\iffalse
%</package>
%\fi
%
\endinput
|\\
|\childdocof{|\textit{main}|}|\\
\end{tabular}
\end{center}
at the top of every child file \textit{child}
which is included by |\include{|\textit{child}|}|
from within the main file
(or at least for those files to be compiled individually).
The argument \textit{main} must be the filename of the main file.

There are a couple of
considerations in setting up the main and child documents:

%%%%%%%%%%%%%%%%%%%%%%%%%%%%%%%%%%%%%%%%
\paragraph{Restrictions.}

Please note the following restrictions:
\begin{itemize}
\item
|\childdocmain| must be called with one argument \textit{main}
to ensure compatibility with earlier version of the package.
It must either be empty (|\childdocmain{}|)
or precisely match the filename of the main file in which it is specified.
See \secref{sec:detection} for further information.
\item
The filename \textit{main} must be specified without the |.tex| extension.
\item
The filename \textit{main} is case sensitive
(even in case-insensitive file systems)
due to internal string comparison.
\item
The argument \textit{main} should be fully expanded, it cannot be a macro.
\item
Subdirectories and special characters should be avoided in filenames.
\item
The command |\childdocmain{|\textit{main}|}| must be followed by a whitespace.
It should not be followed immediately by another command
or by a comment mark `|%|'.
This is because the \TeX{} parser reads the token immediately following
the argument of |\childdocmain| and puts it
at the beginning of every child section;
however, a white\-space is ignored.
\end{itemize}

%%%%%%%%%%%%%%%%%%%%%%%%%%%%%%%%%%%%%%%%
\paragraph{Content of Main File.}

It is advisable to place all content in the child files included by |\include|.
Any output contained in the main file will appear in all child documents
unless suppressed manually;
it cannot be suppressed automatically by the |\includeonly| directive
and thus should normally be avoided.
A method to include some content in the main file
by means of conditional processing is described in \secref{sec:conditional}.

%%%%%%%%%%%%%%%%%%%%%%%%%%%%%%%%%%%%%%%%
\paragraph{Page Numbering.}

When only a part of the document is compiled,
the appropriate numbering of pages
(as well as other status parameters)
is determined from the |.aux| files.
The latter contain information from previous passes.
However this information needs to propagate through
all intermediate child documents.
Therefore the page numbering in child documents may well
be inconsistent until the complete document is compiled at least once.

A useful (if unconventional) way to always ensure a consistent
page numbering is to restart the numbering in each child document
and denote the pages by `\textit{child}|.|\textit{page}'
where \textit{child} represents the chapter/section number of the child file.
This can be achieved by the command
|\numberwithin{page}{|\textit{child}|}|
of the \textsf{amsmath} package
where \textit{child} can be |chapter| or |section|
depending on the chosen structuring.
Alternatively, one can modify the macro |\thepage| appropriately
and reset the counter |page| at the start of each child file.

%%%%%%%%%%%%%%%%%%%%%%%%%%%%%%%%%%%%%%%%%%%%%%%%%%%%%%%%%%%%%%%%%%%%%%%%%%%%%%%%
\subsection{Conditional Processing}
\label{sec:conditional}

The package provides a mechanism to compile different versions
of a document. To customise the versions further some conditional processing
can come in handy to distinguish which version is being compiled.
The package provides two macros to describe the compilation context:

%%%%%%%%%%%%%%%%%%%%%%%%%%%%%%%%%%%%%%%%
\DescribeMacro{\ifchilddoc}
The conditional |\ifchilddoc| distinguishes between the compilation of
child documents and the main document:
%
\begin{center}
|\ifchilddoc |\textit{child-code}| |[|\||else |\textit{main-code}]| \||fi|
\end{center}

%%%%%%%%%%%%%%%%%%%%%%%%%%%%%%%%%%%%%%%%
\DescribeMacro{\childdocname}
\DescribeMacro{\childdocjob}
The macro |\childdocname| contains the filename (without extension)
of the main or child file being processed.
Note that |\childdocjob| will always contain the name of the main file.

%%%%%%%%%%%%%%%%%%%%%%%%%%%%%%%%%%%%%%%%
\paragraph{Title Page.}

Conditional processing can be used to include a title or banner page
in the main document when proper precautions are taken.
Importantly, the code in the main file should ensure that the page counter
(as well as other status parameters which are stored in the |.aux| files)
takes the same value after the conditional processing.
Otherwise the page numbers may take divergent values
depending on which part is compiled.

For example, a title page could be declared by:
%
\begin{center}
\begin{tabular}{l}
|\ifchilddoc\||else|\\
|\addtocounter{page}{-1}|\\
\textit{code for title page}\\
|\newpage|\\
|\||fi|
\end{tabular}
\end{center}
%
A banner page for the child documents can be generated by:
%
\begin{center}
\begin{tabular}{l}
|\ifchilddoc|\\
|\addtocounter{page}{-1}|\\
\textit{code for banner page}\\
|\newpage|\\
|\||fi|
\end{tabular}
\end{center}
%
Here one could write a message such as:
\begin{center}
|This is the part \childdocname{} of \childdocjob{}.|
\end{center}

%%%%%%%%%%%%%%%%%%%%%%%%%%%%%%%%%%%%%%%%%%%%%%%%%%%%%%%%%%%%%%%%%%%%%%%%%%%%%%%%
\subsection{Flags}
\label{sec:flags}

The package makes it easy to generate different versions
of the main or child documents.
To this end compilation flags can be defined
and assigned different default values.
They will be particularly useful in conjunction
with the forwarding mechanism described in \secref{sec:forward}.

For example, it may be useful to have a flag |\version|
which can be set to |draft| or |final|.
The document source will contain some conditional code
depending on the value of |\version|.
Suppose further, the flag should default to |final| for the main file
and to |draft| for child files
which is a natural assignment for editing the document.
This is achieved by placing the following code
in the preamble of the main document
(below the |\childdocmain| directive):
%
\begin{center}
\begin{tabular}{l}
|\ifchilddoc|\\
|\providecommand{\version}{draft}|\\
|\||else|\\
|\providecommand{\version}{final}|\\
|\||fi|
\end{tabular}
\end{center}
%
The definition by |\providecommand| makes sure
that previous definitions are not overwritten.
Further statements |\providecommand{\version}{...}|
can thus be added before the above code to override it.

For the main file, one might add a line
(between |\childdocmain| and the above block)
%
\begin{center}
|%\ifchilddoc\||else\providecommand{\version}{draft}\||fi|
\end{center}
%
which can be uncommented to produce a draft version.
Likewise one can add a line to the very top of a child file
(above the |\childdocof{|\textit{main}|}| directive)
%
\begin{center}
|%\providecommand{\version}{final}|
\end{center}
%
which can be uncommented to produce the final version of this child document.

%%%%%%%%%%%%%%%%%%%%%%%%%%%%%%%%%%%%%%%%%%%%%%%%%%%%%%%%%%%%%%%%%%%%%%%%%%%%%%%%
\subsection{Forwarding}
\label{sec:forward}

Different versions of the main or child documents
using compilation flags as described in \secref{sec:flags}
can be (permanently) stored in different files
for convenient compilation, viewing and distribution.
To this end, the package defines a command
to pass on compilation to a different file:

%%%%%%%%%%%%%%%%%%%%%%%%%%%%%%%%%%%%%%%%
\DescribeMacro{\childdocforward}
The command |\childdocforward| redirects processing to
another source file:
%
\begin{center}
\begin{tabular}{l}
|% \iffalse
%
% childdoc.dtx Copyright (C) 2017-2018 Niklas Beisert
%
% This work may be distributed and/or modified under the
% conditions of the LaTeX Project Public License, either version 1.3
% of this license or (at your option) any later version.
% The latest version of this license is in
%   http://www.latex-project.org/lppl.txt
% and version 1.3 or later is part of all distributions of LaTeX
% version 2005/12/01 or later.
%
% This work has the LPPL maintenance status `maintained'.
%
% The Current Maintainer of this work is Niklas Beisert.
%
% This work consists of the files childdoc.dtx and childdoc.ins
% and the derived files childdoc.def and cdocsamp.tex with
% cdocsch1.tex, cdocsch2.tex, cdocsdrf.tex, cdocsfn1.tex, cdocsfn2.tex.
%
%<package>\ifdefined\childdocmain\endinput\fi
%<package>\ProvidesFile{childdoc.def}[2018/12/30 v2.0 child document driver]
%<samplemain>\ProvidesFile{cdocsamp.tex}[2018/12/30 v2.0 sample for childdoc]
%<*driver>
%\ProvidesFile{childdoc.drv}[2018/12/30 v2.0 childdoc reference manual file]
\PassOptionsToClass{10pt,a4paper}{article}
\documentclass{ltxdoc}

\usepackage[margin=35mm]{geometry}
\usepackage{hyperref}
\usepackage{hyperxmp}
\usepackage[usenames]{color}

\hypersetup{colorlinks=true}
\hypersetup{pdfstartview=FitH}
\hypersetup{pdfpagemode=UseNone}
\hypersetup{pdfsource={}}
\hypersetup{pdflang={en-UK}}
\hypersetup{pdfcopyright={Copyright 2017-2018 Niklas Beisert.
  This work may be distributed and/or modified under the
  conditions of the LaTeX Project Public License, either version 1.3
  of this license or (at your option) any later version.}}
\hypersetup{pdflicenseurl={http://www.latex-project.org/lppl.txt}}
\hypersetup{pdfcontactaddress={ETH Zurich, ITP, HIT K,
  Wolfgang-Pauli-Strasse 27}}
\hypersetup{pdfcontactpostcode={8093}}
\hypersetup{pdfcontactcity={Zurich}}
\hypersetup{pdfcontactcountry={Switzerland}}
\hypersetup{pdfcontactemail={nbeisert@itp.phys.ethz.ch}}
\hypersetup{pdfcontacturl={http://people.phys.ethz.ch/\xmptilde nbeisert/}}

\newcommand{\secref}[1]{\hyperref[#1]{section \ref*{#1}}}

\parskip1ex
\parindent0pt
\let\olditemize\itemize
\def\itemize{\olditemize\parskip0pt}

\begin{document}

\title{The \textsf{childdoc} Package}
\hypersetup{pdftitle={The childdoc Package}}
\author{Niklas Beisert\\[2ex]
  Institut f\"ur Theoretische Physik\\
  Eidgen\"ossische Technische Hochschule Z\"urich\\
  Wolfgang-Pauli-Strasse 27, 8093 Z\"urich, Switzerland\\[1ex]
  \href{mailto:nbeisert@itp.phys.ethz.ch}
  {\texttt{nbeisert@itp.phys.ethz.ch}}}
\hypersetup{pdfauthor={Niklas Beisert}}
\hypersetup{pdfsubject={Manual for the LaTeX2e Package childdoc}}
\date{30 December 2018, \textsf{v2.0}}
\maketitle

\begin{abstract}\noindent
\textsf{childdoc} is a \LaTeXe{} package
that enables the direct compilation
of document sections included by |\include|
to individual files.
\end{abstract}

\begingroup
\parskip0ex
\tableofcontents
\endgroup

%%%%%%%%%%%%%%%%%%%%%%%%%%%%%%%%%%%%%%%%%%%%%%%%%%%%%%%%%%%%%%%%%%%%%%%%%%%%%%%%
%%%%%%%%%%%%%%%%%%%%%%%%%%%%%%%%%%%%%%%%%%%%%%%%%%%%%%%%%%%%%%%%%%%%%%%%%%%%%%%%
\section{Introduction}

\LaTeX{} provides a mechanism to structure a large document (such as a book)
into a main file and several child files (containing the chapters)
using the |\include| command.
This mechanism is beneficial for documents
which span hundreds of pages in order to
make the source file(s) more manageable.
Moreover, compilation can be restricted to
selected child files by means of the |\includeonly| command.
The latter feature can be used to reduce the compilation time while editing
(this was significantly more useful in the earlier days of \LaTeX{})
or to generate a smaller document which is easier to navigate.
Another application of |\includeonly| is to generate
documents consisting of selected parts of the complete document.

However, there are a few drawbacks of the plain |\include| mechanism:
\begin{itemize}
\item
The child files cannot be compiled on their own,
they can only be compiled via the main file.
A naive editing environment
(such as a text editor with an option
to have the current file processed by \LaTeX)
may require one to switch to the main file before compiling;
attempting to compile the child file produces errors.
\item
The main file must be modified (each time)
to adjust the |\includeonly| command
to the present needs. This easily leaves the main file in a messy state.
\item
The generated document will always carry the filename
of the main document. This is inconvenient if
several child files are to be compiled and
to be kept for distribution.
\end{itemize}

The present package provides a simple interface
to make child files individually compilable by \LaTeX{}.
Compiling a child file then has the same effect as compiling
the main file with an |\includeonly| command
to select the appropriate child.
Moreover the generated document will carry the name of the child
rather than the main file.
This resolves all three above issues.

This feature is meant to make the editing of books,
thesis documents and lecture notes somewhat more convenient.
However, the package can also be used efficiently for
composing a series of documents (such as exercise sheets)
which are typically distributed individually.
It then assists the author in generating the individual documents
(potentially in different versions)
as well as a document containing the collected series.
Another application is in developing style files
or other kinds of included material
where compilation of the style file could redirect
to a sample or test file.

%%%%%%%%%%%%%%%%%%%%%%%%%%%%%%%%%%%%%%%%%%%%%%%%%%%%%%%%%%%%%%%%%%%%%%%%%%%%%%%%
%%%%%%%%%%%%%%%%%%%%%%%%%%%%%%%%%%%%%%%%%%%%%%%%%%%%%%%%%%%%%%%%%%%%%%%%%%%%%%%%
\section{Usage}

First of all, the package \textsf{childdoc} is \emph{not} a standard
\LaTeXe{} |.sty| style file! Therefore it needs to be invoked in
a non-standard way.

%%%%%%%%%%%%%%%%%%%%%%%%%%%%%%%%%%%%%%%%%%%%%%%%%%%%%%%%%%%%%%%%%%%%%%%%%%%%%%%%
\subsection{Included Files}
\label{sec:include}

%%%%%%%%%%%%%%%%%%%%%%%%%%%%%%%%%%%%%%%%
\DescribeMacro{\childdocmain}
To use the package, add the commands
\begin{center}
\begin{tabular}{l}
|\input{childdoc.def}|\\
|\childdocmain{}|\\
\end{tabular}
\end{center}
at the very top of the main \LaTeX{} file,
in particular \emph{before} the |\documentclass| statement!
The argument of |\childdocmain| should be left empty
(but it must be present).

%%%%%%%%%%%%%%%%%%%%%%%%%%%%%%%%%%%%%%%%
\DescribeMacro{\childdocof}
Furthermore, add the commands
\begin{center}
\begin{tabular}{l}
|\input{childdoc.def}|\\
|\childdocof{|\textit{main}|}|\\
\end{tabular}
\end{center}
at the top of every child file \textit{child}
which is included by |\include{|\textit{child}|}|
from within the main file
(or at least for those files to be compiled individually).
The argument \textit{main} must be the filename of the main file.

There are a couple of
considerations in setting up the main and child documents:

%%%%%%%%%%%%%%%%%%%%%%%%%%%%%%%%%%%%%%%%
\paragraph{Restrictions.}

Please note the following restrictions:
\begin{itemize}
\item
|\childdocmain| must be called with one argument \textit{main}
to ensure compatibility with earlier version of the package.
It must either be empty (|\childdocmain{}|)
or precisely match the filename of the main file in which it is specified.
See \secref{sec:detection} for further information.
\item
The filename \textit{main} must be specified without the |.tex| extension.
\item
The filename \textit{main} is case sensitive
(even in case-insensitive file systems)
due to internal string comparison.
\item
The argument \textit{main} should be fully expanded, it cannot be a macro.
\item
Subdirectories and special characters should be avoided in filenames.
\item
The command |\childdocmain{|\textit{main}|}| must be followed by a whitespace.
It should not be followed immediately by another command
or by a comment mark `|%|'.
This is because the \TeX{} parser reads the token immediately following
the argument of |\childdocmain| and puts it
at the beginning of every child section;
however, a white\-space is ignored.
\end{itemize}

%%%%%%%%%%%%%%%%%%%%%%%%%%%%%%%%%%%%%%%%
\paragraph{Content of Main File.}

It is advisable to place all content in the child files included by |\include|.
Any output contained in the main file will appear in all child documents
unless suppressed manually;
it cannot be suppressed automatically by the |\includeonly| directive
and thus should normally be avoided.
A method to include some content in the main file
by means of conditional processing is described in \secref{sec:conditional}.

%%%%%%%%%%%%%%%%%%%%%%%%%%%%%%%%%%%%%%%%
\paragraph{Page Numbering.}

When only a part of the document is compiled,
the appropriate numbering of pages
(as well as other status parameters)
is determined from the |.aux| files.
The latter contain information from previous passes.
However this information needs to propagate through
all intermediate child documents.
Therefore the page numbering in child documents may well
be inconsistent until the complete document is compiled at least once.

A useful (if unconventional) way to always ensure a consistent
page numbering is to restart the numbering in each child document
and denote the pages by `\textit{child}|.|\textit{page}'
where \textit{child} represents the chapter/section number of the child file.
This can be achieved by the command
|\numberwithin{page}{|\textit{child}|}|
of the \textsf{amsmath} package
where \textit{child} can be |chapter| or |section|
depending on the chosen structuring.
Alternatively, one can modify the macro |\thepage| appropriately
and reset the counter |page| at the start of each child file.

%%%%%%%%%%%%%%%%%%%%%%%%%%%%%%%%%%%%%%%%%%%%%%%%%%%%%%%%%%%%%%%%%%%%%%%%%%%%%%%%
\subsection{Conditional Processing}
\label{sec:conditional}

The package provides a mechanism to compile different versions
of a document. To customise the versions further some conditional processing
can come in handy to distinguish which version is being compiled.
The package provides two macros to describe the compilation context:

%%%%%%%%%%%%%%%%%%%%%%%%%%%%%%%%%%%%%%%%
\DescribeMacro{\ifchilddoc}
The conditional |\ifchilddoc| distinguishes between the compilation of
child documents and the main document:
%
\begin{center}
|\ifchilddoc |\textit{child-code}| |[|\||else |\textit{main-code}]| \||fi|
\end{center}

%%%%%%%%%%%%%%%%%%%%%%%%%%%%%%%%%%%%%%%%
\DescribeMacro{\childdocname}
\DescribeMacro{\childdocjob}
The macro |\childdocname| contains the filename (without extension)
of the main or child file being processed.
Note that |\childdocjob| will always contain the name of the main file.

%%%%%%%%%%%%%%%%%%%%%%%%%%%%%%%%%%%%%%%%
\paragraph{Title Page.}

Conditional processing can be used to include a title or banner page
in the main document when proper precautions are taken.
Importantly, the code in the main file should ensure that the page counter
(as well as other status parameters which are stored in the |.aux| files)
takes the same value after the conditional processing.
Otherwise the page numbers may take divergent values
depending on which part is compiled.

For example, a title page could be declared by:
%
\begin{center}
\begin{tabular}{l}
|\ifchilddoc\||else|\\
|\addtocounter{page}{-1}|\\
\textit{code for title page}\\
|\newpage|\\
|\||fi|
\end{tabular}
\end{center}
%
A banner page for the child documents can be generated by:
%
\begin{center}
\begin{tabular}{l}
|\ifchilddoc|\\
|\addtocounter{page}{-1}|\\
\textit{code for banner page}\\
|\newpage|\\
|\||fi|
\end{tabular}
\end{center}
%
Here one could write a message such as:
\begin{center}
|This is the part \childdocname{} of \childdocjob{}.|
\end{center}

%%%%%%%%%%%%%%%%%%%%%%%%%%%%%%%%%%%%%%%%%%%%%%%%%%%%%%%%%%%%%%%%%%%%%%%%%%%%%%%%
\subsection{Flags}
\label{sec:flags}

The package makes it easy to generate different versions
of the main or child documents.
To this end compilation flags can be defined
and assigned different default values.
They will be particularly useful in conjunction
with the forwarding mechanism described in \secref{sec:forward}.

For example, it may be useful to have a flag |\version|
which can be set to |draft| or |final|.
The document source will contain some conditional code
depending on the value of |\version|.
Suppose further, the flag should default to |final| for the main file
and to |draft| for child files
which is a natural assignment for editing the document.
This is achieved by placing the following code
in the preamble of the main document
(below the |\childdocmain| directive):
%
\begin{center}
\begin{tabular}{l}
|\ifchilddoc|\\
|\providecommand{\version}{draft}|\\
|\||else|\\
|\providecommand{\version}{final}|\\
|\||fi|
\end{tabular}
\end{center}
%
The definition by |\providecommand| makes sure
that previous definitions are not overwritten.
Further statements |\providecommand{\version}{...}|
can thus be added before the above code to override it.

For the main file, one might add a line
(between |\childdocmain| and the above block)
%
\begin{center}
|%\ifchilddoc\||else\providecommand{\version}{draft}\||fi|
\end{center}
%
which can be uncommented to produce a draft version.
Likewise one can add a line to the very top of a child file
(above the |\childdocof{|\textit{main}|}| directive)
%
\begin{center}
|%\providecommand{\version}{final}|
\end{center}
%
which can be uncommented to produce the final version of this child document.

%%%%%%%%%%%%%%%%%%%%%%%%%%%%%%%%%%%%%%%%%%%%%%%%%%%%%%%%%%%%%%%%%%%%%%%%%%%%%%%%
\subsection{Forwarding}
\label{sec:forward}

Different versions of the main or child documents
using compilation flags as described in \secref{sec:flags}
can be (permanently) stored in different files
for convenient compilation, viewing and distribution.
To this end, the package defines a command
to pass on compilation to a different file:

%%%%%%%%%%%%%%%%%%%%%%%%%%%%%%%%%%%%%%%%
\DescribeMacro{\childdocforward}
The command |\childdocforward| redirects processing to
another source file:
%
\begin{center}
\begin{tabular}{l}
|\input{childdoc.def}|\\
|\childdocforward[|\textit{main}|]{|\textit{dest}|}|\\
\end{tabular}
\end{center}
%
The argument \textit{dest} is the destination file
(without extension).
It should be the main file or one of the child files.
Note that further \textsf{childdoc} directives
such as |\childdocof| and |\childdocforward|
in the indicated file will be processed in this form.
The optional argument \textit{main}
passes on directly to the main file \textit{main}
while pretending to compile the child \textit{dest}.
This form behaves as if \textit{dest}
issues |\childdocof{|\textit{main}|}| right away,
and no further \textsf{childdoc} directives will be processed.

%%%%%%%%%%%%%%%%%%%%%%%%%%%%%%%%%%%%%%%%
\DescribeMacro{\...prefix}
In the alternative form |\childdocforwardprefix|,
%
\begin{center}
\begin{tabular}{l}
|\input{childdoc.def}|\\
|\childdocforwardprefix[|\textit{main}|]{|\textit{prefix}|}{|\textit{dest}|}|
\end{tabular}
\end{center}
%
the destination file is determined by a pattern
depending on the current file:
To make this work, the current file must be called
`{\textit{prefix}\hspace{0.2em}\textit{suffix}}'
with \textit{prefix} matching precisely the argument.
Processing is then passed on to the file
`{\textit{dest}\hspace{0.2em}\textit{suffix}}'.
Surely, the same effect is achieved by
directly specifying the
argument `{\textit{dest}\hspace{0.2em}\textit{suffix}}'
in the first form.
However, that requires to set up a different file
for each child. With the alternative form of the command
all these files can have exactly the same content
which simplifies setting them up and maintaining them.

For example, the following file |draft.tex|
with a compilation flag |\version| as described in \secref{sec:flags}
compiles the main document as a draft:
%
\begin{center}
\begin{tabular}{l}
|\def\version{draft}|\\
|\input{childdoc.def}|\\
|\childdocforward{|\textit{main}|}|
\end{tabular}
\end{center}
%
Likewise, the following files |final|\textit{nn}|.tex|
compile the final version of the child document
|child|\textit{nn}|.tex|:
%
\begin{center}
\begin{tabular}{l}
|\def\version{final}|\\
|\input{childdoc.def}|\\
|\childdocforwardprefix{final}{child}|
\end{tabular}
\end{center}
%

Note that when several versions of a main file and/or of each child file
are to be generated, it may be convenient to set up a |Makefile| or
shell script to automatise the process.

%%%%%%%%%%%%%%%%%%%%%%%%%%%%%%%%%%%%%%%%%%%%%%%%%%%%%%%%%%%%%%%%%%%%%%%%%%%%%%%%
\subsection{Command Line Processing}
\label{sec:commandline}

The effect of redirection files can also be achieved by invoking
the \LaTeX{} compiler with a more elaborate command line.
Most conveniently this should be done as part
of a shell script or a |Makefile|.

When using \textsf{childdoc} in the main file, the following
command lines effectively perform a redirection
(note that depending on the shell being used,
backslashes may have to be doubled: `|\|' $\to$ `|\\|'):
%
\begin{center}
|... -jobname "|\textit{target}|" |\\|"|[\textit{flags}]%
|\input{childdoc.def}\childdocforward[|\textit{main}|]{|\textit{dest}|}"|
\end{center}
%
Here \textit{target} is the name of the output file,
\textit{main} is the name of the main file
and \textit{dest} is the name of the main or child file to be processed
(all filenames without extensions).
The optional argument \textit{main} can be omitted
if \textit{main} matches \textit{dest}.
Optionally, compilation \textit{flags} can be defined via |\def| commands.
This command line makes the \TeX{} engine believe
it is compiling the file \textit{target}
whose content is specified as the latter parameter.
The provided code then forwards the processing to
\textit{main} or \textit{dest} as described in \secref{sec:forward}.

%%%%%%%%%%%%%%%%%%%%%%%%%%%%%%%%%%%%%%%%%%%%%%%%%%%%%%%%%%%%%%%%%%%%%%%%%%%%%%%%
\subsection{Include by Input}
\label{sec:input}

Including child documents by |\include| has some restrictions by design.
Most notably, the content of a child document always occupies
its own set of pages; pages cannot be shared between child documents.
Usually, this behaviour makes perfect sense
because each child document contain an essential part of the document.
However, in some situations it may be desirable to compose
a document from a collection of parts
without having mandatory page breaks between then.
For this case, the package
provides a mechanism to include parts
by |\input| which can also be processed individually.
However, by construction this mechanism
requires manual handling of the content to be output.

%%%%%%%%%%%%%%%%%%%%%%%%%%%%%%%%%%%%%%%%
\DescribeMacro{\ifchilddocmanual}
The main file should be prepared as usual, see \secref{sec:include}.
However, the document body must make a distinction
between processing of an individual part and of the main document, e.g.:
%
\begin{center}
\begin{tabular}{l}
|\ifchilddocmanual|\\
|\input{\childdocname}|\\
|\||else|\\
\textit{document body with }|\input{|\textit{part}|}|\\
|\||fi|
\end{tabular}
\end{center}
%
The conditional |\ifchilddocmanual| is true whenever
a part to be included by |\input| is being compiled,
and the name of the part is stored in |\childdocname|.

%%%%%%%%%%%%%%%%%%%%%%%%%%%%%%%%%%%%%%%%
\DescribeMacro{\childdocby}
Each part to be included by |\input| should start with:
%
\begin{center}
\begin{tabular}{l}
|\input{childdoc.def}|\\
|\childdocby{|\textit{main}|}|\\
\end{tabular}
\end{center}
%
The directive |\childdocby| is similar to |\childdocof|
described in \secref{sec:include},
but the subsequent selection of content must be done manually.
To that end, both |\ifchilddoc| and |\ifchilddocmanual|
will be true upon processing of a part,
and the name of the part is stored in |\childdocname|.
Note that |\jobname| will be set to the filename of the current part
so that each part receives an individual |.aux| file
that does not interfere with the |.aux| file(s) of the main document.
This behaviour can be altered by the alternative form
|\childdocby[*]{|\textit{main}|}| (with a non-empty optional argument)
which uses the |.aux| file of the main document
by setting |\jobname| to \textit{main}.

%%%%%%%%%%%%%%%%%%%%%%%%%%%%%%%%%%%%%%%%%%%%%%%%%%%%%%%%%%%%%%%%%%%%%%%%%%%%%%%%
\subsection{Driver Development}
\label{sec:driver}

The \textsf{childdoc} mechanism can also be use for the development
of definition files such as \LaTeX{} styles or classes.
This case differs from the above setup with multiple parts
included by |\include| in that no |\includeonly| should be invoked.
This can be achieved by starting the include file
(before |\ProvidesPackage|) with:
%
\begin{center}
\begin{tabular}{l}
|\input{childdoc.def}|\\
|\childdocforward{|\textit{main}|}|\\
\end{tabular}
\end{center}
%
or alternatively with:
%
\begin{center}
\begin{tabular}{l}
|\input{childdoc.def}|\\
|\childdocby{|\textit{main}|}|\\
\end{tabular}
\end{center}
%
Both forms have slightly different effects as described above.
The main file is prepared as usual, see \secref{sec:include}.

%%%%%%%%%%%%%%%%%%%%%%%%%%%%%%%%%%%%%%%%%%%%%%%%%%%%%%%%%%%%%%%%%%%%%%%%%%%%%%%%
\subsection{Legacy Detection}
\label{sec:detection}

The directive |\childdocmain| in the main file can detect
whether the complete document or merely a child is to be compiled
even without using the directive |\childdocof|.
This method is deprecated because it is less robust
and there is no compelling reason to use it;
it is merely provided for backward compatibility
and it may be removed in future versions.

If the detection mechanism is to be used,
it is mandatory to correctly specify
the filename of the main file as the argument of |\childdocmain|:
%
\begin{center}
\begin{tabular}{l}
|\input{childdoc.def}|\\
|\childdocmain{|\textit{main}|}|\\
\end{tabular}
\end{center}
%
If |\jobname| does not match the argument \textit{main} of |\childdocmain|,
it is assumed that |\jobname| points to the child file to be compiled.
When using |\childdocmain| with the main file specified as argument,
it suffices to start a child file
with just |\input{|\textit{main}|}|
without loading of the package and using |\childdocof|.
If instead all processing is done
with the appropriate \textsf{childdoc} directives,
the argument of \textit{main} of |\childdocmain| can be empty.

An alternative version of the command line processing described
in \secref{sec:commandline} using the detection mechanism reads:
%
\begin{center}
|... -jobname "|\textit{target}|" "|[\textit{flags}]%
[|\def\jobname{|\textit{dest}|}|]|\input{|\textit{main}|}"|
\end{center}

%%%%%%%%%%%%%%%%%%%%%%%%%%%%%%%%%%%%%%%%%%%%%%%%%%%%%%%%%%%%%%%%%%%%%%%%%%%%%%%%
\subsection{Manual Code}
\label{sec:manual}

In case one cannot be certain whether the definitions file |childdoc.def|
is installed on the target \TeX{} distribution
and one prefers not to ship it,
it is conceivable to paste a few relevant commands into the sources.

To that end, drop all statements |\input{childdoc.def}|
and perform the replacements as outlined below.
Instead of |\childdocmain{|\textit{main}|}| add the following code
to the top of the main file:
%
\begin{center}
\begin{tabular}{l}
|\||ifdefined\childdocname\endinput\||fi\newif\ifchilddoc|\\
|\edef\childdocname{\scantokens\expandafter{\jobname\noexpand}}|\\
|\def\childdocmain{|\textit{main}|}\||ifx\childdocmain\childdocname\||else|\\
|\childdoctrue\includeonly{\childdocname}\let\jobname\childdocmain\||fi|\\
\end{tabular}
\end{center}
%
Instead of |\childdocof{|\textit{main}|}| just include the main file
at the top of each child file:
%
\begin{center}
|\input{|\textit{main}|}|
\end{center}
%
A simple redirection |\childdocforward{|\textit{dest}|}| is achieved by:
%
\begin{center}
|\def\jobname{|\textit{dest}|}\input{\jobname}|
\end{center}
%
The redirection with prefix
|\childdocforwardprefix[|\textit{prefix}|]{|\textit{dest}|}|
is accomplished by:
%
\begin{center}
\begin{tabular}{l}
|{\edef\jobname{\scantokens\expandafter{\jobname\noexpand}}|\\
|\def\redirectjob |\textit{prefix}|#1~~~{\gdef\jobname{|\textit{dest}|#1}}|\\
|\expandafter\redirectjob\jobname~~~}\input{\jobname}|
\end{tabular}
\end{center}

In an alternative approach,
child documents can be compiled by a specific command line
without additional code or specific definitions:
%
\begin{center}
|... -jobname "|\textit{target}|" "|[\textit{flags}]%
|\includeonly{|\textit{dest}|}\input{|\textit{main}|}"|
\end{center}
%

%%%%%%%%%%%%%%%%%%%%%%%%%%%%%%%%%%%%%%%%%%%%%%%%%%%%%%%%%%%%%%%%%%%%%%%%%%%%%%%%
%%%%%%%%%%%%%%%%%%%%%%%%%%%%%%%%%%%%%%%%%%%%%%%%%%%%%%%%%%%%%%%%%%%%%%%%%%%%%%%%
\section{Information}

%%%%%%%%%%%%%%%%%%%%%%%%%%%%%%%%%%%%%%%%%%%%%%%%%%%%%%%%%%%%%%%%%%%%%%%%%%%%%%%%
\subsection{Copyright}

Copyright \copyright{} 2017--2018 Niklas Beisert

This work may be distributed and/or modified under the
conditions of the \LaTeX{} Project Public License, either version 1.3
of this license or (at your option) any later version.
The latest version of this license is in
  \url{http://www.latex-project.org/lppl.txt}
and version 1.3 or later is part of all distributions of \LaTeX{}
version 2005/12/01 or later.

This work has the LPPL maintenance status `maintained'.

The Current Maintainer of this work is Niklas Beisert.

This work consists of the files |README.txt|, |childdoc.ins| and |childdoc.dtx|
as well as the derived files |childdoc.def|, |cdocsamp.tex|
with |cdocsch1.tex|, |cdocsch2.tex|, |cdocspt3.tex|, |cdocspt4.tex|,
|cdocsdrf.tex|, |cdocsfn1.tex|, |cdocsfn2.tex|
as well as |childdoc.pdf|.

%%%%%%%%%%%%%%%%%%%%%%%%%%%%%%%%%%%%%%%%%%%%%%%%%%%%%%%%%%%%%%%%%%%%%%%%%%%%%%%%
\subsection{Files and Installation}

The package consists of the files:
%
\begin{center}
\begin{tabular}{ll}
    |README.txt|   & readme file \\
    |childdoc.ins| & installation file \\
    |childdoc.dtx| & source file \\
    |childdoc.def| & definition file \\
    |cdocsamp.tex| & sample main file \\
    |cdocsch1.tex| & sample include file \\
    |cdocsch2.tex| & sample include file \\
    |cdocspt3.tex| & sample part file \\
    |cdocspt4.tex| & sample part file \\
    |cdocsdrf.tex| & sample redirection file \\
    |cdocsfn1.tex| & sample redirection file \\
    |cdocsfn2.tex| & sample redirection file \\
    |childdoc.pdf| & manual
\end{tabular}
\end{center}
%
The distribution consists of the files
|README.txt|, |childdoc.ins| and |childdoc.dtx|.
%
\begin{itemize}
\item
Run (pdf)\LaTeX{} on |childdoc.dtx|
to compile the manual |childdoc.pdf| (this file).
\item
Run \LaTeX{} on |childdoc.ins| to create the definitions file |childdoc.def|
and the sample |cdocsamp.tex| with include files
|cdocsch1.tex|, |cdocsch2.tex|, |cdocspt3.tex|, |cdocspt4.tex|,
|cdocsdrf.tex|, |cdocsfn1.tex|, |cdocsfn2.tex|.
Then copy the file |childdoc.def| to an appropriate directory of your \LaTeX{}
distribution, e.g.\ \textit{texmf-root}|/tex/latex/childdoc|.
\end{itemize}

%%%%%%%%%%%%%%%%%%%%%%%%%%%%%%%%%%%%%%%%%%%%%%%%%%%%%%%%%%%%%%%%%%%%%%%%%%%%%%%%
\subsection{Related CTAN Packages}

There are several other packages which offer a similar functionality:
%
\begin{itemize}
\item
The packages
\href{http://ctan.org/pkg/docmute}{\textsf{docmute}},
\href{http://ctan.org/pkg/includex}{\textsf{includex}} and
\href{http://ctan.org/pkg/standalone}{\textsf{standalone}}
provide commands to include only the document body of
a child file thus allowing both files to be compiled individually.
\item
The packages \href{http://ctan.org/pkg/subdocs}{\textsf{subdocs}}
and \href{http://ctan.org/pkg/subfiles}{\textsf{subfiles}}
provide structures in which the main and child documents can be
encapsulated and allowing them to be compiled individually.
The inclusion mechanism is different from the conventional |\include|.
\item
The package \href{http://ctan.org/pkg/combine}{\textsf{combine}}
is an elaborate solution to combine several documents into one.
\end{itemize}
%
See also the CTAN topic \href{http://ctan.org/topic/subdocs}{\textsf{subdocs}}
for further related packages.
The present package differs from the above solutions in that
a document structure constructed with the conventional |\include| mechanism
just needs two extra commands at the top of every file
such that all constituent files can be compiled individually.

%%%%%%%%%%%%%%%%%%%%%%%%%%%%%%%%%%%%%%%%%%%%%%%%%%%%%%%%%%%%%%%%%%%%%%%%%%%%%%%%
%\subsection{Feature Suggestions}
%
%The following is a list of features which may be useful for future
%versions of this package:
%%
%\begin{itemize}
%\item
%\ldots
%\end{itemize}

%%%%%%%%%%%%%%%%%%%%%%%%%%%%%%%%%%%%%%%%%%%%%%%%%%%%%%%%%%%%%%%%%%%%%%%%%%%%%%%%
\subsection{Revision History}

%%%%%%%%%%%%%%%%%%%%%%%%%%%%%%%%%%%%%%%%
\paragraph{v2.0:} 2018/12/30

\begin{itemize}
\item
immediate forward processing
\item
added |\childdocby| mechanism
\item
manual restructured
\end{itemize}

%%%%%%%%%%%%%%%%%%%%%%%%%%%%%%%%%%%%%%%%
\paragraph{v1.6:} 2018/01/17

\begin{itemize}
\item
application for development of include files
\item
corrections to manual
\end{itemize}

%%%%%%%%%%%%%%%%%%%%%%%%%%%%%%%%%%%%%%%%
\paragraph{v1.5:} 2017/05/21

\begin{itemize}
\item
more complete structuring introduced
\item
|\childdocof| introduced
\item
|\childdoc| renamed to |\childdocmain|
\item
|\childredirect| renamed to |\childdocforward| and |\childdocforwardprefix|
and functionality expanded
\end{itemize}

%%%%%%%%%%%%%%%%%%%%%%%%%%%%%%%%%%%%%%%%
\paragraph{v1.0:} 2017/04/27

\begin{itemize}
\item
manual and install package
\item
first version published on CTAN
\end{itemize}

%%%%%%%%%%%%%%%%%%%%%%%%%%%%%%%%%%%%%%%%
\paragraph{v0.6:} 2017/04/26

\begin{itemize}
\item
redirection mechanism added
\end{itemize}

%%%%%%%%%%%%%%%%%%%%%%%%%%%%%%%%%%%%%%%%
\paragraph{v0.5:} 2017/04/26

\begin{itemize}
\item
functionality in definition file
\end{itemize}


%%%%%%%%%%%%%%%%%%%%%%%%%%%%%%%%%%%%%%%%%%%%%%%%%%%%%%%%%%%%%%%%%%%%%%%%%%%%%%%%
%%%%%%%%%%%%%%%%%%%%%%%%%%%%%%%%%%%%%%%%%%%%%%%%%%%%%%%%%%%%%%%%%%%%%%%%%%%%%%%%
%%%%%%%%%%%%%%%%%%%%%%%%%%%%%%%%%%%%%%%%%%%%%%%%%%%%%%%%%%%%%%%%%%%%%%%%%%%%%%%%
\appendix

\settowidth\MacroIndent{\rmfamily\scriptsize 000\ }

 \DocInput{childdoc.dtx}

\end{document}
%</driver>
% \fi
%
% %%%%%%%%%%%%%%%%%%%%%%%%%%%%%%%%%%%%%%%%%%%%%%%%%%%%%%%%%%%%%%%%%%%%%%%%%%%%%%
% %%%%%%%%%%%%%%%%%%%%%%%%%%%%%%%%%%%%%%%%%%%%%%%%%%%%%%%%%%%%%%%%%%%%%%%%%%%%%%
% \section{Sample}
%\iffalse
%<*samplemain>
%\fi
%
% The following presents a sample document
% with two chapters, two parts, a title page,
% a compile flag as well as three forwarding files to set the flag.
% It consists of eight |.tex| files:
% \begin{center}
% \begin{tabular}{ll}
% |cdocsamp.tex|&main file\\
% |cdocsch1.tex|&include file for chapter 1\\
% |cdocsch2.tex|&include file for chapter 2\\
% |cdocspt3.tex|&include file for part 3\\
% |cdocspt4.tex|&include file for part 4\\
% |cdocsdrf.tex|&forwarding file for main file in draft mode\\
% |cdocsfi1.tex|&forwarding file for final version of chapter 1\\
% |cdocsfi2.tex|&forwarding file for final version of chapter 2\\
% \end{tabular}
% \end{center}
% Each of the eight files can be compiled directly by the \LaTeX{} compiler.
%
% %%%%%%%%%%%%%%%%%%%%%%%%%%%%%%%%%%%%%%
% \paragraph{Main File.}
%
% The main file is called |cdocsamp.tex|.
%
% Load the \textsf{childdoc} definitions and
% declare the filename for the main document:
%    \begin{macrocode}
\input{childdoc.def}
\childdocmain{}
%    \end{macrocode}

% Optional override for |\version| flag:
%    \begin{macrocode}
%%\ifchilddoc\else\providecommand{\version}{draft}\fi
%    \end{macrocode}

% Define the default values for the |\version| flag
% (|final| for the main file and |draft| for childs):
%    \begin{macrocode}
\ifchilddoc
\providecommand{\version}{draft}
\else
\providecommand{\version}{final}
\fi
%    \end{macrocode}

% Load the standard document class:
%    \begin{macrocode}
\documentclass[12pt]{article}
%    \end{macrocode}

% Start the document body:
%    \begin{macrocode}
\begin{document}
%    \end{macrocode}

% Declare a title page.
% Print title, part of document being processed and version flag:
%    \begin{macrocode}
\addtocounter{page}{-1}
\begin{center}
{\LARGE\bfseries{}childdoc example\par}
\vspace{1cm}
\ifchilddoc
\ifchilddocmanual part\else chapter\fi:
`\childdocname' of `\childdocjob'\par
\else
main document: `\childdocjob'\par
\fi
version: \version\par
\end{center}
\newpage
%    \end{macrocode}

% Manually include selected file,
% otherwise process as usual:
%    \begin{macrocode}
\ifchilddocmanual
\section*{part `\childdocname'}
\input{\childdocname}
\else
%    \end{macrocode}

% Include the two chapters:
%    \begin{macrocode}
\include{cdocsch1}
\include{cdocsch2}
%    \end{macrocode}

% Include the two parts unless only chapters should be displayed:
%    \begin{macrocode}
\ifchilddoc\else
\section{part three}
\input{cdocspt3}
\section{part four}
\input{cdocspt4}
\fi
%    \end{macrocode}

% Process as usual until here:
%    \begin{macrocode}
\fi
%    \end{macrocode}

% End of document body:
%    \begin{macrocode}
\end{document}
%    \end{macrocode}
%\iffalse
%</samplemain>
%\fi
%
% %%%%%%%%%%%%%%%%%%%%%%%%%%%%%%%%%%%%%%
% \paragraph{Chapter Include Files.}
%
% The include files are called |cdocsch1.tex| and |cdocsch2.tex|.
%
%\iffalse
%<*samplechap1|samplechap2>
%\fi

% Optional override for |\version| flag:
%    \begin{macrocode}
%%\providecommand{\version}{final}
%    \end{macrocode}

% Include the main document:
%    \begin{macrocode}
\input{childdoc.def}
\childdocof{cdocsamp}
%    \end{macrocode}

%\iffalse
%</samplechap1|samplechap2>
%\fi
%
%\iffalse
%<*samplechap1>
%\fi
% Some text for chapter 1:
%    \begin{macrocode}
\section{one}
some text in chapter one
%    \end{macrocode}

%\iffalse
%</samplechap1>
%\fi
% Some text for chapter 2:
%\iffalse
%<*samplechap2>
%\fi
%    \begin{macrocode}
\section{two}
more text in chapter two
%    \end{macrocode}

%\iffalse
%</samplechap2>
%\fi
%
% %%%%%%%%%%%%%%%%%%%%%%%%%%%%%%%%%%%%%%
% \paragraph{Part Include Files.}
%
% The include files are called |cdocspt3.tex| and |cdocspt4.tex|.
%
%\iffalse
%<*samplepart3|samplepart4>
%\fi

% Optional override for |\version| flag:
%    \begin{macrocode}
%%\providecommand{\version}{final}
%    \end{macrocode}

% Include the main document:
%    \begin{macrocode}
\input{childdoc.def}
\childdocby{cdocsamp}
%    \end{macrocode}

%\iffalse
%</samplepart3|samplepart4>
%\fi
%
%\iffalse
%<*samplepart3>
%\fi
% Some text for part 3:
%    \begin{macrocode}
some text in part three
%    \end{macrocode}

%\iffalse
%</samplepart3>
%\fi
% Some text for part 4:
%\iffalse
%<*samplepart4>
%\fi
%    \begin{macrocode}
more text in part four
%    \end{macrocode}

%\iffalse
%</samplepart4>
%\fi
%
% %%%%%%%%%%%%%%%%%%%%%%%%%%%%%%%%%%%%%%
% \paragraph{Forwarding for a Complete Draft.}
%
% The following forwarding file |cdocsdrf.tex|
% compiles the main document in draft mode:
%\iffalse
%<*sampledraft>
%\fi
%    \begin{macrocode}
\def\version{draft}
\input{childdoc.def}
\childdocforward{cdocsamp}
%    \end{macrocode}

%\iffalse
%</sampledraft>
%\fi
%
% %%%%%%%%%%%%%%%%%%%%%%%%%%%%%%%%%%%%%%
% \paragraph{Forwarding for Final Version of the Chapters.}
%
% The following forwarding files |cdocsfn1.tex| and |cdocsfn2.tex|
% (with identical content)
% compile the final versions of the child documents
% |cdocsch1.tex| and |cdocsch2.tex|, respectively:
%\iffalse
%<*samplefinal>
%\fi
%    \begin{macrocode}
\def\version{final}
\input{childdoc.def}
\childdocforwardprefix[cdocsamp]{cdocsfn}{cdocsch}
%    \end{macrocode}

%\iffalse
%</samplefinal>
%\fi
%
% %%%%%%%%%%%%%%%%%%%%%%%%%%%%%%%%%%%%%%
% \paragraph{Command Line Processing.}
%
% The following three command lines generate the output files
% |cdocscld|, |cdocscl1| and |cdocscl2|
% which should be identical to
% |cdocsdrf|, |cdocsch1| and |cdocsfn2|, respectively:
% \begin{center}
% \begin{tabular}{l}
% |latex -jobname cdocscld \|\\
% |  "\def\version{draft}\input{childdoc.def}\childdocforward{cdocsamp}"|\\
% |latex -jobname cdocscl1 \|\\
% |  "\input{childdoc.def}\childdocforward[cdocsamp]{cdocsch1}"|\\
% |latex -jobname cdocscl2 \|\\
% |  "\def\version{final}\input{childdoc.def}\childdocforward{cdocsch2}"|
% \end{tabular}
% \end{center}
% Note that the trailing backslash on each first line
% merely continues the input to the second line
% (for convenient cut ant paste).
% Furthermore, the command |latex| can be replaced by any
% of its alternative versions such as |pdflatex|.
%
% %%%%%%%%%%%%%%%%%%%%%%%%%%%%%%%%%%%%%%%%%%%%%%%%%%%%%%%%%%%%%%%%%%%%%%%%%%%%%%
% %%%%%%%%%%%%%%%%%%%%%%%%%%%%%%%%%%%%%%%%%%%%%%%%%%%%%%%%%%%%%%%%%%%%%%%%%%%%%%
% \section{Implementation}
%\iffalse
%<*package>
%\fi
%
% This section describes the definitions file |childdoc.def|.

% The definitions cannot be loaded using |\usepackage| or |\RequirePackage|
% which has a mechanism to prevent loading a style file more than once.
% When loading the definitions by means of |\input|
% multiple instances have to be prevented manually:
%\iffalse
%This code needs to be before the `\ProvidesFile' directive
%which is defined at the beginning of this file.
%Therefore it is also placed there and commented out here.
%</package>
%<*discard>
%\fi
%    \begin{macrocode}
\ifdefined\childdocmain\endinput\fi
%    \end{macrocode}
%\iffalse
%</discard>
%<*package>
%\fi
%
% \macro{\ifchilddoc}
% \macro{\ifchilddocmanual}
% The conditional |\ifchilddoc| tells whether a
% child (true) or main (false) document is being compiled.
% The conditional |\ifchilddocmanual| tells whether
% the |\includeonly| mechanism is used (false) or
% the selection of child files must be performed manually (true).
% The definitions initialise to false:
%    \begin{macrocode}
\newif\ifchilddoc
\newif\ifchilddocmanual
%    \end{macrocode}

% \macro{\childdocname}
% \macro{\childdocjob}
% The macro |\childdocname| stores the name of the main document
% to be compiled. The macro |\childdocjob| stores the name of
% the document on which the \LaTeX{} compiler was originally invoked.
% The content of |\jobname| cannot be compared
% to filenames specified in the source due to different catcodes.
% The following code rescans |\jobname|, stores the result
% in |\childdocname| and saves a copy in |\childdocjob|:
%    \begin{macrocode}
\edef\childdocname{\scantokens\expandafter{\jobname\noexpand}}
\let\childdocjob\childdocname
%    \end{macrocode}

% \macro{\childdocdisable}
% The macro |\childdocdisable| prevents the main file
% from being processed more than once.
% At this stage, the main document command |\childdocmain|
% is assumed to be called once again where it should do nothing.
% Any subsequent call to it should prevent
% a secondary processing of the main document
% It overwrites the forwarding commands
% |\childdocof| and |\childdocforward|
% with empty macros to prevent further inclusions of the main document:
%    \begin{macrocode}
\newcommand{\childdocdisable}
{
  \renewcommand{\childdocmain}[1]{\renewcommand{\childdocmain}[1]{\endinput}}
  \renewcommand{\childdocof}[1]{}
  \renewcommand{\childdocby}[2][]{}
  \renewcommand{\childdocforward}[2][]{}
  \renewcommand{\childdocdisable}{}
}
%    \end{macrocode}

% \macro{\childdocmain}
% The macro |\childdocmain| is to be called at the top of the main file
% with nothing or the main filename (without extension) as argument.
% First, it breaks loops.
% If the argument is not empty and does not match |\childdocname|
% (which is set by the first inclusion of |childdoc.def|),
% |\ifchilddoc| is set to true, |\includeonly| is applied to the child file
% and |\jobname| is set to the main file
% (for proper handling of |.aux| files):
%    \begin{macrocode}
\newcommand{\childdocmain}[1]
{
  \childdocdisable\childdocmain{}
  \if?#1?\else
    \begingroup
      \def\childdoctmp{#1}
      \ifx\childdoctmp\childdocname
        \def\childdoctmp{}
      \else
        \def\childdoctmp
        {
          \childdoctrue
          \includeonly{\childdocname}
          \def\childdocjob{#1}
          \def\jobname{#1}
        }
      \fi
      \expandafter
    \endgroup
    \childdoctmp
  \fi
}
%    \end{macrocode}

% \macro{\childdocof}
% The command |\childdocof| redirects
% compilation to the main file |#1|.
%    \begin{macrocode}
\newcommand{\childdocof}[1]
{
  \childdocdisable
  \childdoctrue
  \includeonly{\childdocname}
  \def\jobname{#1}
  \def\childdocjob{#1}
  \input{#1}
}
%    \end{macrocode}

% \macro{\childdocby}
% The command |\childdocby| ....
%    \begin{macrocode}
\newcommand{\childdocby}[2][]
{
  \childdocdisable
  \childdoctrue
  \childdocmanualtrue
  \if?#1?\else
    \def\jobname{#2}
  \fi
  \def\childdocjob{#2}
  \input{#2}
  \endinput
}
%    \end{macrocode}

% \macro{\childdocforward}
% The command |\childdocforward| redirects
% compilation to the main file or
% (if the optional argument is given) a child file.
% Parameters are set as if the main file
% or a child file starting with |\childdocof| was compiled.
% Then compilation is handed over to the main file:
%    \begin{macrocode}
\newcommand{\childdocforward}[2][]
{
  \begingroup
    \if?#1?
      \def\childdoctmp
      {
        \def\childdocname{#2}
        \def\childdocjob{#2}
        \def\jobname{#2}
        \input{#2}
        \endinput
      }
    \else
      \def\childdoctmp
      {
        \childdocdisable
        \def\childdocname{#2}
        \childdoctrue
        \includeonly{#2}
        \def\childdocjob{#1}
        \def\jobname{#1}
        \input{#1}
        \endinput
      }
    \fi
    \expandafter
  \endgroup
  \childdoctmp
}
%    \end{macrocode}

% \macro{\childdocforwardprefix}
% The command |\childdocforwardprefix| redirects
% compilation to the main or a child file by means of a pattern.
% The prefix |#1| in the current filename is replaced by |#2|
% and the suffix of the current filename is kept
% (it is assumed that the filename does not contain the substring `|~~~|'
% which is used as a delimiter).
% Compilation is handed over to the new file by |\childdocforward|:
%    \begin{macrocode}
\newcommand{\childdocforwardprefix}[3][]
{
  \begingroup
    \def\childdocextract #2##1~~~{\def\childdoctmp{\childdocforward[#1]{#3##1}}}
    \expandafter\childdocextract\childdocname~~~
    \expandafter
  \endgroup
  \childdoctmp
}
%    \end{macrocode}

% \macro{\childdoc}
% The deprecated macro |\childdoc| is a legacy version of |\childdocmain|:
%    \begin{macrocode}
\newcommand{\childdoc}{\childdocmain}
%    \end{macrocode}

% \macro{\childdocredirect}
% The deprecated macro |\childdocredirect| is a legacy version
% of |\childdocforward| and |\childdocforwardprefix|:
%    \begin{macrocode}
\newcommand{\childdocredirect}[2][]
{
  \begingroup
    \if?#1?
      \def\childdoctmp{\childdocforward{#2}}
    \else
      \def\childdoctmp{\childdocforwardprefix{#1}{#2}}
    \fi
    \expandafter
  \endgroup
  \childdoctmp
}
%    \end{macrocode}

%\iffalse
%</package>
%\fi
%
\endinput
|\\
|\childdocforward[|\textit{main}|]{|\textit{dest}|}|\\
\end{tabular}
\end{center}
%
The argument \textit{dest} is the destination file
(without extension).
It should be the main file or one of the child files.
Note that further \textsf{childdoc} directives
such as |\childdocof| and |\childdocforward|
in the indicated file will be processed in this form.
The optional argument \textit{main}
passes on directly to the main file \textit{main}
while pretending to compile the child \textit{dest}.
This form behaves as if \textit{dest}
issues |\childdocof{|\textit{main}|}| right away,
and no further \textsf{childdoc} directives will be processed.

%%%%%%%%%%%%%%%%%%%%%%%%%%%%%%%%%%%%%%%%
\DescribeMacro{\...prefix}
In the alternative form |\childdocforwardprefix|,
%
\begin{center}
\begin{tabular}{l}
|% \iffalse
%
% childdoc.dtx Copyright (C) 2017-2018 Niklas Beisert
%
% This work may be distributed and/or modified under the
% conditions of the LaTeX Project Public License, either version 1.3
% of this license or (at your option) any later version.
% The latest version of this license is in
%   http://www.latex-project.org/lppl.txt
% and version 1.3 or later is part of all distributions of LaTeX
% version 2005/12/01 or later.
%
% This work has the LPPL maintenance status `maintained'.
%
% The Current Maintainer of this work is Niklas Beisert.
%
% This work consists of the files childdoc.dtx and childdoc.ins
% and the derived files childdoc.def and cdocsamp.tex with
% cdocsch1.tex, cdocsch2.tex, cdocsdrf.tex, cdocsfn1.tex, cdocsfn2.tex.
%
%<package>\ifdefined\childdocmain\endinput\fi
%<package>\ProvidesFile{childdoc.def}[2018/12/30 v2.0 child document driver]
%<samplemain>\ProvidesFile{cdocsamp.tex}[2018/12/30 v2.0 sample for childdoc]
%<*driver>
%\ProvidesFile{childdoc.drv}[2018/12/30 v2.0 childdoc reference manual file]
\PassOptionsToClass{10pt,a4paper}{article}
\documentclass{ltxdoc}

\usepackage[margin=35mm]{geometry}
\usepackage{hyperref}
\usepackage{hyperxmp}
\usepackage[usenames]{color}

\hypersetup{colorlinks=true}
\hypersetup{pdfstartview=FitH}
\hypersetup{pdfpagemode=UseNone}
\hypersetup{pdfsource={}}
\hypersetup{pdflang={en-UK}}
\hypersetup{pdfcopyright={Copyright 2017-2018 Niklas Beisert.
  This work may be distributed and/or modified under the
  conditions of the LaTeX Project Public License, either version 1.3
  of this license or (at your option) any later version.}}
\hypersetup{pdflicenseurl={http://www.latex-project.org/lppl.txt}}
\hypersetup{pdfcontactaddress={ETH Zurich, ITP, HIT K,
  Wolfgang-Pauli-Strasse 27}}
\hypersetup{pdfcontactpostcode={8093}}
\hypersetup{pdfcontactcity={Zurich}}
\hypersetup{pdfcontactcountry={Switzerland}}
\hypersetup{pdfcontactemail={nbeisert@itp.phys.ethz.ch}}
\hypersetup{pdfcontacturl={http://people.phys.ethz.ch/\xmptilde nbeisert/}}

\newcommand{\secref}[1]{\hyperref[#1]{section \ref*{#1}}}

\parskip1ex
\parindent0pt
\let\olditemize\itemize
\def\itemize{\olditemize\parskip0pt}

\begin{document}

\title{The \textsf{childdoc} Package}
\hypersetup{pdftitle={The childdoc Package}}
\author{Niklas Beisert\\[2ex]
  Institut f\"ur Theoretische Physik\\
  Eidgen\"ossische Technische Hochschule Z\"urich\\
  Wolfgang-Pauli-Strasse 27, 8093 Z\"urich, Switzerland\\[1ex]
  \href{mailto:nbeisert@itp.phys.ethz.ch}
  {\texttt{nbeisert@itp.phys.ethz.ch}}}
\hypersetup{pdfauthor={Niklas Beisert}}
\hypersetup{pdfsubject={Manual for the LaTeX2e Package childdoc}}
\date{30 December 2018, \textsf{v2.0}}
\maketitle

\begin{abstract}\noindent
\textsf{childdoc} is a \LaTeXe{} package
that enables the direct compilation
of document sections included by |\include|
to individual files.
\end{abstract}

\begingroup
\parskip0ex
\tableofcontents
\endgroup

%%%%%%%%%%%%%%%%%%%%%%%%%%%%%%%%%%%%%%%%%%%%%%%%%%%%%%%%%%%%%%%%%%%%%%%%%%%%%%%%
%%%%%%%%%%%%%%%%%%%%%%%%%%%%%%%%%%%%%%%%%%%%%%%%%%%%%%%%%%%%%%%%%%%%%%%%%%%%%%%%
\section{Introduction}

\LaTeX{} provides a mechanism to structure a large document (such as a book)
into a main file and several child files (containing the chapters)
using the |\include| command.
This mechanism is beneficial for documents
which span hundreds of pages in order to
make the source file(s) more manageable.
Moreover, compilation can be restricted to
selected child files by means of the |\includeonly| command.
The latter feature can be used to reduce the compilation time while editing
(this was significantly more useful in the earlier days of \LaTeX{})
or to generate a smaller document which is easier to navigate.
Another application of |\includeonly| is to generate
documents consisting of selected parts of the complete document.

However, there are a few drawbacks of the plain |\include| mechanism:
\begin{itemize}
\item
The child files cannot be compiled on their own,
they can only be compiled via the main file.
A naive editing environment
(such as a text editor with an option
to have the current file processed by \LaTeX)
may require one to switch to the main file before compiling;
attempting to compile the child file produces errors.
\item
The main file must be modified (each time)
to adjust the |\includeonly| command
to the present needs. This easily leaves the main file in a messy state.
\item
The generated document will always carry the filename
of the main document. This is inconvenient if
several child files are to be compiled and
to be kept for distribution.
\end{itemize}

The present package provides a simple interface
to make child files individually compilable by \LaTeX{}.
Compiling a child file then has the same effect as compiling
the main file with an |\includeonly| command
to select the appropriate child.
Moreover the generated document will carry the name of the child
rather than the main file.
This resolves all three above issues.

This feature is meant to make the editing of books,
thesis documents and lecture notes somewhat more convenient.
However, the package can also be used efficiently for
composing a series of documents (such as exercise sheets)
which are typically distributed individually.
It then assists the author in generating the individual documents
(potentially in different versions)
as well as a document containing the collected series.
Another application is in developing style files
or other kinds of included material
where compilation of the style file could redirect
to a sample or test file.

%%%%%%%%%%%%%%%%%%%%%%%%%%%%%%%%%%%%%%%%%%%%%%%%%%%%%%%%%%%%%%%%%%%%%%%%%%%%%%%%
%%%%%%%%%%%%%%%%%%%%%%%%%%%%%%%%%%%%%%%%%%%%%%%%%%%%%%%%%%%%%%%%%%%%%%%%%%%%%%%%
\section{Usage}

First of all, the package \textsf{childdoc} is \emph{not} a standard
\LaTeXe{} |.sty| style file! Therefore it needs to be invoked in
a non-standard way.

%%%%%%%%%%%%%%%%%%%%%%%%%%%%%%%%%%%%%%%%%%%%%%%%%%%%%%%%%%%%%%%%%%%%%%%%%%%%%%%%
\subsection{Included Files}
\label{sec:include}

%%%%%%%%%%%%%%%%%%%%%%%%%%%%%%%%%%%%%%%%
\DescribeMacro{\childdocmain}
To use the package, add the commands
\begin{center}
\begin{tabular}{l}
|\input{childdoc.def}|\\
|\childdocmain{}|\\
\end{tabular}
\end{center}
at the very top of the main \LaTeX{} file,
in particular \emph{before} the |\documentclass| statement!
The argument of |\childdocmain| should be left empty
(but it must be present).

%%%%%%%%%%%%%%%%%%%%%%%%%%%%%%%%%%%%%%%%
\DescribeMacro{\childdocof}
Furthermore, add the commands
\begin{center}
\begin{tabular}{l}
|\input{childdoc.def}|\\
|\childdocof{|\textit{main}|}|\\
\end{tabular}
\end{center}
at the top of every child file \textit{child}
which is included by |\include{|\textit{child}|}|
from within the main file
(or at least for those files to be compiled individually).
The argument \textit{main} must be the filename of the main file.

There are a couple of
considerations in setting up the main and child documents:

%%%%%%%%%%%%%%%%%%%%%%%%%%%%%%%%%%%%%%%%
\paragraph{Restrictions.}

Please note the following restrictions:
\begin{itemize}
\item
|\childdocmain| must be called with one argument \textit{main}
to ensure compatibility with earlier version of the package.
It must either be empty (|\childdocmain{}|)
or precisely match the filename of the main file in which it is specified.
See \secref{sec:detection} for further information.
\item
The filename \textit{main} must be specified without the |.tex| extension.
\item
The filename \textit{main} is case sensitive
(even in case-insensitive file systems)
due to internal string comparison.
\item
The argument \textit{main} should be fully expanded, it cannot be a macro.
\item
Subdirectories and special characters should be avoided in filenames.
\item
The command |\childdocmain{|\textit{main}|}| must be followed by a whitespace.
It should not be followed immediately by another command
or by a comment mark `|%|'.
This is because the \TeX{} parser reads the token immediately following
the argument of |\childdocmain| and puts it
at the beginning of every child section;
however, a white\-space is ignored.
\end{itemize}

%%%%%%%%%%%%%%%%%%%%%%%%%%%%%%%%%%%%%%%%
\paragraph{Content of Main File.}

It is advisable to place all content in the child files included by |\include|.
Any output contained in the main file will appear in all child documents
unless suppressed manually;
it cannot be suppressed automatically by the |\includeonly| directive
and thus should normally be avoided.
A method to include some content in the main file
by means of conditional processing is described in \secref{sec:conditional}.

%%%%%%%%%%%%%%%%%%%%%%%%%%%%%%%%%%%%%%%%
\paragraph{Page Numbering.}

When only a part of the document is compiled,
the appropriate numbering of pages
(as well as other status parameters)
is determined from the |.aux| files.
The latter contain information from previous passes.
However this information needs to propagate through
all intermediate child documents.
Therefore the page numbering in child documents may well
be inconsistent until the complete document is compiled at least once.

A useful (if unconventional) way to always ensure a consistent
page numbering is to restart the numbering in each child document
and denote the pages by `\textit{child}|.|\textit{page}'
where \textit{child} represents the chapter/section number of the child file.
This can be achieved by the command
|\numberwithin{page}{|\textit{child}|}|
of the \textsf{amsmath} package
where \textit{child} can be |chapter| or |section|
depending on the chosen structuring.
Alternatively, one can modify the macro |\thepage| appropriately
and reset the counter |page| at the start of each child file.

%%%%%%%%%%%%%%%%%%%%%%%%%%%%%%%%%%%%%%%%%%%%%%%%%%%%%%%%%%%%%%%%%%%%%%%%%%%%%%%%
\subsection{Conditional Processing}
\label{sec:conditional}

The package provides a mechanism to compile different versions
of a document. To customise the versions further some conditional processing
can come in handy to distinguish which version is being compiled.
The package provides two macros to describe the compilation context:

%%%%%%%%%%%%%%%%%%%%%%%%%%%%%%%%%%%%%%%%
\DescribeMacro{\ifchilddoc}
The conditional |\ifchilddoc| distinguishes between the compilation of
child documents and the main document:
%
\begin{center}
|\ifchilddoc |\textit{child-code}| |[|\||else |\textit{main-code}]| \||fi|
\end{center}

%%%%%%%%%%%%%%%%%%%%%%%%%%%%%%%%%%%%%%%%
\DescribeMacro{\childdocname}
\DescribeMacro{\childdocjob}
The macro |\childdocname| contains the filename (without extension)
of the main or child file being processed.
Note that |\childdocjob| will always contain the name of the main file.

%%%%%%%%%%%%%%%%%%%%%%%%%%%%%%%%%%%%%%%%
\paragraph{Title Page.}

Conditional processing can be used to include a title or banner page
in the main document when proper precautions are taken.
Importantly, the code in the main file should ensure that the page counter
(as well as other status parameters which are stored in the |.aux| files)
takes the same value after the conditional processing.
Otherwise the page numbers may take divergent values
depending on which part is compiled.

For example, a title page could be declared by:
%
\begin{center}
\begin{tabular}{l}
|\ifchilddoc\||else|\\
|\addtocounter{page}{-1}|\\
\textit{code for title page}\\
|\newpage|\\
|\||fi|
\end{tabular}
\end{center}
%
A banner page for the child documents can be generated by:
%
\begin{center}
\begin{tabular}{l}
|\ifchilddoc|\\
|\addtocounter{page}{-1}|\\
\textit{code for banner page}\\
|\newpage|\\
|\||fi|
\end{tabular}
\end{center}
%
Here one could write a message such as:
\begin{center}
|This is the part \childdocname{} of \childdocjob{}.|
\end{center}

%%%%%%%%%%%%%%%%%%%%%%%%%%%%%%%%%%%%%%%%%%%%%%%%%%%%%%%%%%%%%%%%%%%%%%%%%%%%%%%%
\subsection{Flags}
\label{sec:flags}

The package makes it easy to generate different versions
of the main or child documents.
To this end compilation flags can be defined
and assigned different default values.
They will be particularly useful in conjunction
with the forwarding mechanism described in \secref{sec:forward}.

For example, it may be useful to have a flag |\version|
which can be set to |draft| or |final|.
The document source will contain some conditional code
depending on the value of |\version|.
Suppose further, the flag should default to |final| for the main file
and to |draft| for child files
which is a natural assignment for editing the document.
This is achieved by placing the following code
in the preamble of the main document
(below the |\childdocmain| directive):
%
\begin{center}
\begin{tabular}{l}
|\ifchilddoc|\\
|\providecommand{\version}{draft}|\\
|\||else|\\
|\providecommand{\version}{final}|\\
|\||fi|
\end{tabular}
\end{center}
%
The definition by |\providecommand| makes sure
that previous definitions are not overwritten.
Further statements |\providecommand{\version}{...}|
can thus be added before the above code to override it.

For the main file, one might add a line
(between |\childdocmain| and the above block)
%
\begin{center}
|%\ifchilddoc\||else\providecommand{\version}{draft}\||fi|
\end{center}
%
which can be uncommented to produce a draft version.
Likewise one can add a line to the very top of a child file
(above the |\childdocof{|\textit{main}|}| directive)
%
\begin{center}
|%\providecommand{\version}{final}|
\end{center}
%
which can be uncommented to produce the final version of this child document.

%%%%%%%%%%%%%%%%%%%%%%%%%%%%%%%%%%%%%%%%%%%%%%%%%%%%%%%%%%%%%%%%%%%%%%%%%%%%%%%%
\subsection{Forwarding}
\label{sec:forward}

Different versions of the main or child documents
using compilation flags as described in \secref{sec:flags}
can be (permanently) stored in different files
for convenient compilation, viewing and distribution.
To this end, the package defines a command
to pass on compilation to a different file:

%%%%%%%%%%%%%%%%%%%%%%%%%%%%%%%%%%%%%%%%
\DescribeMacro{\childdocforward}
The command |\childdocforward| redirects processing to
another source file:
%
\begin{center}
\begin{tabular}{l}
|\input{childdoc.def}|\\
|\childdocforward[|\textit{main}|]{|\textit{dest}|}|\\
\end{tabular}
\end{center}
%
The argument \textit{dest} is the destination file
(without extension).
It should be the main file or one of the child files.
Note that further \textsf{childdoc} directives
such as |\childdocof| and |\childdocforward|
in the indicated file will be processed in this form.
The optional argument \textit{main}
passes on directly to the main file \textit{main}
while pretending to compile the child \textit{dest}.
This form behaves as if \textit{dest}
issues |\childdocof{|\textit{main}|}| right away,
and no further \textsf{childdoc} directives will be processed.

%%%%%%%%%%%%%%%%%%%%%%%%%%%%%%%%%%%%%%%%
\DescribeMacro{\...prefix}
In the alternative form |\childdocforwardprefix|,
%
\begin{center}
\begin{tabular}{l}
|\input{childdoc.def}|\\
|\childdocforwardprefix[|\textit{main}|]{|\textit{prefix}|}{|\textit{dest}|}|
\end{tabular}
\end{center}
%
the destination file is determined by a pattern
depending on the current file:
To make this work, the current file must be called
`{\textit{prefix}\hspace{0.2em}\textit{suffix}}'
with \textit{prefix} matching precisely the argument.
Processing is then passed on to the file
`{\textit{dest}\hspace{0.2em}\textit{suffix}}'.
Surely, the same effect is achieved by
directly specifying the
argument `{\textit{dest}\hspace{0.2em}\textit{suffix}}'
in the first form.
However, that requires to set up a different file
for each child. With the alternative form of the command
all these files can have exactly the same content
which simplifies setting them up and maintaining them.

For example, the following file |draft.tex|
with a compilation flag |\version| as described in \secref{sec:flags}
compiles the main document as a draft:
%
\begin{center}
\begin{tabular}{l}
|\def\version{draft}|\\
|\input{childdoc.def}|\\
|\childdocforward{|\textit{main}|}|
\end{tabular}
\end{center}
%
Likewise, the following files |final|\textit{nn}|.tex|
compile the final version of the child document
|child|\textit{nn}|.tex|:
%
\begin{center}
\begin{tabular}{l}
|\def\version{final}|\\
|\input{childdoc.def}|\\
|\childdocforwardprefix{final}{child}|
\end{tabular}
\end{center}
%

Note that when several versions of a main file and/or of each child file
are to be generated, it may be convenient to set up a |Makefile| or
shell script to automatise the process.

%%%%%%%%%%%%%%%%%%%%%%%%%%%%%%%%%%%%%%%%%%%%%%%%%%%%%%%%%%%%%%%%%%%%%%%%%%%%%%%%
\subsection{Command Line Processing}
\label{sec:commandline}

The effect of redirection files can also be achieved by invoking
the \LaTeX{} compiler with a more elaborate command line.
Most conveniently this should be done as part
of a shell script or a |Makefile|.

When using \textsf{childdoc} in the main file, the following
command lines effectively perform a redirection
(note that depending on the shell being used,
backslashes may have to be doubled: `|\|' $\to$ `|\\|'):
%
\begin{center}
|... -jobname "|\textit{target}|" |\\|"|[\textit{flags}]%
|\input{childdoc.def}\childdocforward[|\textit{main}|]{|\textit{dest}|}"|
\end{center}
%
Here \textit{target} is the name of the output file,
\textit{main} is the name of the main file
and \textit{dest} is the name of the main or child file to be processed
(all filenames without extensions).
The optional argument \textit{main} can be omitted
if \textit{main} matches \textit{dest}.
Optionally, compilation \textit{flags} can be defined via |\def| commands.
This command line makes the \TeX{} engine believe
it is compiling the file \textit{target}
whose content is specified as the latter parameter.
The provided code then forwards the processing to
\textit{main} or \textit{dest} as described in \secref{sec:forward}.

%%%%%%%%%%%%%%%%%%%%%%%%%%%%%%%%%%%%%%%%%%%%%%%%%%%%%%%%%%%%%%%%%%%%%%%%%%%%%%%%
\subsection{Include by Input}
\label{sec:input}

Including child documents by |\include| has some restrictions by design.
Most notably, the content of a child document always occupies
its own set of pages; pages cannot be shared between child documents.
Usually, this behaviour makes perfect sense
because each child document contain an essential part of the document.
However, in some situations it may be desirable to compose
a document from a collection of parts
without having mandatory page breaks between then.
For this case, the package
provides a mechanism to include parts
by |\input| which can also be processed individually.
However, by construction this mechanism
requires manual handling of the content to be output.

%%%%%%%%%%%%%%%%%%%%%%%%%%%%%%%%%%%%%%%%
\DescribeMacro{\ifchilddocmanual}
The main file should be prepared as usual, see \secref{sec:include}.
However, the document body must make a distinction
between processing of an individual part and of the main document, e.g.:
%
\begin{center}
\begin{tabular}{l}
|\ifchilddocmanual|\\
|\input{\childdocname}|\\
|\||else|\\
\textit{document body with }|\input{|\textit{part}|}|\\
|\||fi|
\end{tabular}
\end{center}
%
The conditional |\ifchilddocmanual| is true whenever
a part to be included by |\input| is being compiled,
and the name of the part is stored in |\childdocname|.

%%%%%%%%%%%%%%%%%%%%%%%%%%%%%%%%%%%%%%%%
\DescribeMacro{\childdocby}
Each part to be included by |\input| should start with:
%
\begin{center}
\begin{tabular}{l}
|\input{childdoc.def}|\\
|\childdocby{|\textit{main}|}|\\
\end{tabular}
\end{center}
%
The directive |\childdocby| is similar to |\childdocof|
described in \secref{sec:include},
but the subsequent selection of content must be done manually.
To that end, both |\ifchilddoc| and |\ifchilddocmanual|
will be true upon processing of a part,
and the name of the part is stored in |\childdocname|.
Note that |\jobname| will be set to the filename of the current part
so that each part receives an individual |.aux| file
that does not interfere with the |.aux| file(s) of the main document.
This behaviour can be altered by the alternative form
|\childdocby[*]{|\textit{main}|}| (with a non-empty optional argument)
which uses the |.aux| file of the main document
by setting |\jobname| to \textit{main}.

%%%%%%%%%%%%%%%%%%%%%%%%%%%%%%%%%%%%%%%%%%%%%%%%%%%%%%%%%%%%%%%%%%%%%%%%%%%%%%%%
\subsection{Driver Development}
\label{sec:driver}

The \textsf{childdoc} mechanism can also be use for the development
of definition files such as \LaTeX{} styles or classes.
This case differs from the above setup with multiple parts
included by |\include| in that no |\includeonly| should be invoked.
This can be achieved by starting the include file
(before |\ProvidesPackage|) with:
%
\begin{center}
\begin{tabular}{l}
|\input{childdoc.def}|\\
|\childdocforward{|\textit{main}|}|\\
\end{tabular}
\end{center}
%
or alternatively with:
%
\begin{center}
\begin{tabular}{l}
|\input{childdoc.def}|\\
|\childdocby{|\textit{main}|}|\\
\end{tabular}
\end{center}
%
Both forms have slightly different effects as described above.
The main file is prepared as usual, see \secref{sec:include}.

%%%%%%%%%%%%%%%%%%%%%%%%%%%%%%%%%%%%%%%%%%%%%%%%%%%%%%%%%%%%%%%%%%%%%%%%%%%%%%%%
\subsection{Legacy Detection}
\label{sec:detection}

The directive |\childdocmain| in the main file can detect
whether the complete document or merely a child is to be compiled
even without using the directive |\childdocof|.
This method is deprecated because it is less robust
and there is no compelling reason to use it;
it is merely provided for backward compatibility
and it may be removed in future versions.

If the detection mechanism is to be used,
it is mandatory to correctly specify
the filename of the main file as the argument of |\childdocmain|:
%
\begin{center}
\begin{tabular}{l}
|\input{childdoc.def}|\\
|\childdocmain{|\textit{main}|}|\\
\end{tabular}
\end{center}
%
If |\jobname| does not match the argument \textit{main} of |\childdocmain|,
it is assumed that |\jobname| points to the child file to be compiled.
When using |\childdocmain| with the main file specified as argument,
it suffices to start a child file
with just |\input{|\textit{main}|}|
without loading of the package and using |\childdocof|.
If instead all processing is done
with the appropriate \textsf{childdoc} directives,
the argument of \textit{main} of |\childdocmain| can be empty.

An alternative version of the command line processing described
in \secref{sec:commandline} using the detection mechanism reads:
%
\begin{center}
|... -jobname "|\textit{target}|" "|[\textit{flags}]%
[|\def\jobname{|\textit{dest}|}|]|\input{|\textit{main}|}"|
\end{center}

%%%%%%%%%%%%%%%%%%%%%%%%%%%%%%%%%%%%%%%%%%%%%%%%%%%%%%%%%%%%%%%%%%%%%%%%%%%%%%%%
\subsection{Manual Code}
\label{sec:manual}

In case one cannot be certain whether the definitions file |childdoc.def|
is installed on the target \TeX{} distribution
and one prefers not to ship it,
it is conceivable to paste a few relevant commands into the sources.

To that end, drop all statements |\input{childdoc.def}|
and perform the replacements as outlined below.
Instead of |\childdocmain{|\textit{main}|}| add the following code
to the top of the main file:
%
\begin{center}
\begin{tabular}{l}
|\||ifdefined\childdocname\endinput\||fi\newif\ifchilddoc|\\
|\edef\childdocname{\scantokens\expandafter{\jobname\noexpand}}|\\
|\def\childdocmain{|\textit{main}|}\||ifx\childdocmain\childdocname\||else|\\
|\childdoctrue\includeonly{\childdocname}\let\jobname\childdocmain\||fi|\\
\end{tabular}
\end{center}
%
Instead of |\childdocof{|\textit{main}|}| just include the main file
at the top of each child file:
%
\begin{center}
|\input{|\textit{main}|}|
\end{center}
%
A simple redirection |\childdocforward{|\textit{dest}|}| is achieved by:
%
\begin{center}
|\def\jobname{|\textit{dest}|}\input{\jobname}|
\end{center}
%
The redirection with prefix
|\childdocforwardprefix[|\textit{prefix}|]{|\textit{dest}|}|
is accomplished by:
%
\begin{center}
\begin{tabular}{l}
|{\edef\jobname{\scantokens\expandafter{\jobname\noexpand}}|\\
|\def\redirectjob |\textit{prefix}|#1~~~{\gdef\jobname{|\textit{dest}|#1}}|\\
|\expandafter\redirectjob\jobname~~~}\input{\jobname}|
\end{tabular}
\end{center}

In an alternative approach,
child documents can be compiled by a specific command line
without additional code or specific definitions:
%
\begin{center}
|... -jobname "|\textit{target}|" "|[\textit{flags}]%
|\includeonly{|\textit{dest}|}\input{|\textit{main}|}"|
\end{center}
%

%%%%%%%%%%%%%%%%%%%%%%%%%%%%%%%%%%%%%%%%%%%%%%%%%%%%%%%%%%%%%%%%%%%%%%%%%%%%%%%%
%%%%%%%%%%%%%%%%%%%%%%%%%%%%%%%%%%%%%%%%%%%%%%%%%%%%%%%%%%%%%%%%%%%%%%%%%%%%%%%%
\section{Information}

%%%%%%%%%%%%%%%%%%%%%%%%%%%%%%%%%%%%%%%%%%%%%%%%%%%%%%%%%%%%%%%%%%%%%%%%%%%%%%%%
\subsection{Copyright}

Copyright \copyright{} 2017--2018 Niklas Beisert

This work may be distributed and/or modified under the
conditions of the \LaTeX{} Project Public License, either version 1.3
of this license or (at your option) any later version.
The latest version of this license is in
  \url{http://www.latex-project.org/lppl.txt}
and version 1.3 or later is part of all distributions of \LaTeX{}
version 2005/12/01 or later.

This work has the LPPL maintenance status `maintained'.

The Current Maintainer of this work is Niklas Beisert.

This work consists of the files |README.txt|, |childdoc.ins| and |childdoc.dtx|
as well as the derived files |childdoc.def|, |cdocsamp.tex|
with |cdocsch1.tex|, |cdocsch2.tex|, |cdocspt3.tex|, |cdocspt4.tex|,
|cdocsdrf.tex|, |cdocsfn1.tex|, |cdocsfn2.tex|
as well as |childdoc.pdf|.

%%%%%%%%%%%%%%%%%%%%%%%%%%%%%%%%%%%%%%%%%%%%%%%%%%%%%%%%%%%%%%%%%%%%%%%%%%%%%%%%
\subsection{Files and Installation}

The package consists of the files:
%
\begin{center}
\begin{tabular}{ll}
    |README.txt|   & readme file \\
    |childdoc.ins| & installation file \\
    |childdoc.dtx| & source file \\
    |childdoc.def| & definition file \\
    |cdocsamp.tex| & sample main file \\
    |cdocsch1.tex| & sample include file \\
    |cdocsch2.tex| & sample include file \\
    |cdocspt3.tex| & sample part file \\
    |cdocspt4.tex| & sample part file \\
    |cdocsdrf.tex| & sample redirection file \\
    |cdocsfn1.tex| & sample redirection file \\
    |cdocsfn2.tex| & sample redirection file \\
    |childdoc.pdf| & manual
\end{tabular}
\end{center}
%
The distribution consists of the files
|README.txt|, |childdoc.ins| and |childdoc.dtx|.
%
\begin{itemize}
\item
Run (pdf)\LaTeX{} on |childdoc.dtx|
to compile the manual |childdoc.pdf| (this file).
\item
Run \LaTeX{} on |childdoc.ins| to create the definitions file |childdoc.def|
and the sample |cdocsamp.tex| with include files
|cdocsch1.tex|, |cdocsch2.tex|, |cdocspt3.tex|, |cdocspt4.tex|,
|cdocsdrf.tex|, |cdocsfn1.tex|, |cdocsfn2.tex|.
Then copy the file |childdoc.def| to an appropriate directory of your \LaTeX{}
distribution, e.g.\ \textit{texmf-root}|/tex/latex/childdoc|.
\end{itemize}

%%%%%%%%%%%%%%%%%%%%%%%%%%%%%%%%%%%%%%%%%%%%%%%%%%%%%%%%%%%%%%%%%%%%%%%%%%%%%%%%
\subsection{Related CTAN Packages}

There are several other packages which offer a similar functionality:
%
\begin{itemize}
\item
The packages
\href{http://ctan.org/pkg/docmute}{\textsf{docmute}},
\href{http://ctan.org/pkg/includex}{\textsf{includex}} and
\href{http://ctan.org/pkg/standalone}{\textsf{standalone}}
provide commands to include only the document body of
a child file thus allowing both files to be compiled individually.
\item
The packages \href{http://ctan.org/pkg/subdocs}{\textsf{subdocs}}
and \href{http://ctan.org/pkg/subfiles}{\textsf{subfiles}}
provide structures in which the main and child documents can be
encapsulated and allowing them to be compiled individually.
The inclusion mechanism is different from the conventional |\include|.
\item
The package \href{http://ctan.org/pkg/combine}{\textsf{combine}}
is an elaborate solution to combine several documents into one.
\end{itemize}
%
See also the CTAN topic \href{http://ctan.org/topic/subdocs}{\textsf{subdocs}}
for further related packages.
The present package differs from the above solutions in that
a document structure constructed with the conventional |\include| mechanism
just needs two extra commands at the top of every file
such that all constituent files can be compiled individually.

%%%%%%%%%%%%%%%%%%%%%%%%%%%%%%%%%%%%%%%%%%%%%%%%%%%%%%%%%%%%%%%%%%%%%%%%%%%%%%%%
%\subsection{Feature Suggestions}
%
%The following is a list of features which may be useful for future
%versions of this package:
%%
%\begin{itemize}
%\item
%\ldots
%\end{itemize}

%%%%%%%%%%%%%%%%%%%%%%%%%%%%%%%%%%%%%%%%%%%%%%%%%%%%%%%%%%%%%%%%%%%%%%%%%%%%%%%%
\subsection{Revision History}

%%%%%%%%%%%%%%%%%%%%%%%%%%%%%%%%%%%%%%%%
\paragraph{v2.0:} 2018/12/30

\begin{itemize}
\item
immediate forward processing
\item
added |\childdocby| mechanism
\item
manual restructured
\end{itemize}

%%%%%%%%%%%%%%%%%%%%%%%%%%%%%%%%%%%%%%%%
\paragraph{v1.6:} 2018/01/17

\begin{itemize}
\item
application for development of include files
\item
corrections to manual
\end{itemize}

%%%%%%%%%%%%%%%%%%%%%%%%%%%%%%%%%%%%%%%%
\paragraph{v1.5:} 2017/05/21

\begin{itemize}
\item
more complete structuring introduced
\item
|\childdocof| introduced
\item
|\childdoc| renamed to |\childdocmain|
\item
|\childredirect| renamed to |\childdocforward| and |\childdocforwardprefix|
and functionality expanded
\end{itemize}

%%%%%%%%%%%%%%%%%%%%%%%%%%%%%%%%%%%%%%%%
\paragraph{v1.0:} 2017/04/27

\begin{itemize}
\item
manual and install package
\item
first version published on CTAN
\end{itemize}

%%%%%%%%%%%%%%%%%%%%%%%%%%%%%%%%%%%%%%%%
\paragraph{v0.6:} 2017/04/26

\begin{itemize}
\item
redirection mechanism added
\end{itemize}

%%%%%%%%%%%%%%%%%%%%%%%%%%%%%%%%%%%%%%%%
\paragraph{v0.5:} 2017/04/26

\begin{itemize}
\item
functionality in definition file
\end{itemize}


%%%%%%%%%%%%%%%%%%%%%%%%%%%%%%%%%%%%%%%%%%%%%%%%%%%%%%%%%%%%%%%%%%%%%%%%%%%%%%%%
%%%%%%%%%%%%%%%%%%%%%%%%%%%%%%%%%%%%%%%%%%%%%%%%%%%%%%%%%%%%%%%%%%%%%%%%%%%%%%%%
%%%%%%%%%%%%%%%%%%%%%%%%%%%%%%%%%%%%%%%%%%%%%%%%%%%%%%%%%%%%%%%%%%%%%%%%%%%%%%%%
\appendix

\settowidth\MacroIndent{\rmfamily\scriptsize 000\ }

 \DocInput{childdoc.dtx}

\end{document}
%</driver>
% \fi
%
% %%%%%%%%%%%%%%%%%%%%%%%%%%%%%%%%%%%%%%%%%%%%%%%%%%%%%%%%%%%%%%%%%%%%%%%%%%%%%%
% %%%%%%%%%%%%%%%%%%%%%%%%%%%%%%%%%%%%%%%%%%%%%%%%%%%%%%%%%%%%%%%%%%%%%%%%%%%%%%
% \section{Sample}
%\iffalse
%<*samplemain>
%\fi
%
% The following presents a sample document
% with two chapters, two parts, a title page,
% a compile flag as well as three forwarding files to set the flag.
% It consists of eight |.tex| files:
% \begin{center}
% \begin{tabular}{ll}
% |cdocsamp.tex|&main file\\
% |cdocsch1.tex|&include file for chapter 1\\
% |cdocsch2.tex|&include file for chapter 2\\
% |cdocspt3.tex|&include file for part 3\\
% |cdocspt4.tex|&include file for part 4\\
% |cdocsdrf.tex|&forwarding file for main file in draft mode\\
% |cdocsfi1.tex|&forwarding file for final version of chapter 1\\
% |cdocsfi2.tex|&forwarding file for final version of chapter 2\\
% \end{tabular}
% \end{center}
% Each of the eight files can be compiled directly by the \LaTeX{} compiler.
%
% %%%%%%%%%%%%%%%%%%%%%%%%%%%%%%%%%%%%%%
% \paragraph{Main File.}
%
% The main file is called |cdocsamp.tex|.
%
% Load the \textsf{childdoc} definitions and
% declare the filename for the main document:
%    \begin{macrocode}
\input{childdoc.def}
\childdocmain{}
%    \end{macrocode}

% Optional override for |\version| flag:
%    \begin{macrocode}
%%\ifchilddoc\else\providecommand{\version}{draft}\fi
%    \end{macrocode}

% Define the default values for the |\version| flag
% (|final| for the main file and |draft| for childs):
%    \begin{macrocode}
\ifchilddoc
\providecommand{\version}{draft}
\else
\providecommand{\version}{final}
\fi
%    \end{macrocode}

% Load the standard document class:
%    \begin{macrocode}
\documentclass[12pt]{article}
%    \end{macrocode}

% Start the document body:
%    \begin{macrocode}
\begin{document}
%    \end{macrocode}

% Declare a title page.
% Print title, part of document being processed and version flag:
%    \begin{macrocode}
\addtocounter{page}{-1}
\begin{center}
{\LARGE\bfseries{}childdoc example\par}
\vspace{1cm}
\ifchilddoc
\ifchilddocmanual part\else chapter\fi:
`\childdocname' of `\childdocjob'\par
\else
main document: `\childdocjob'\par
\fi
version: \version\par
\end{center}
\newpage
%    \end{macrocode}

% Manually include selected file,
% otherwise process as usual:
%    \begin{macrocode}
\ifchilddocmanual
\section*{part `\childdocname'}
\input{\childdocname}
\else
%    \end{macrocode}

% Include the two chapters:
%    \begin{macrocode}
\include{cdocsch1}
\include{cdocsch2}
%    \end{macrocode}

% Include the two parts unless only chapters should be displayed:
%    \begin{macrocode}
\ifchilddoc\else
\section{part three}
\input{cdocspt3}
\section{part four}
\input{cdocspt4}
\fi
%    \end{macrocode}

% Process as usual until here:
%    \begin{macrocode}
\fi
%    \end{macrocode}

% End of document body:
%    \begin{macrocode}
\end{document}
%    \end{macrocode}
%\iffalse
%</samplemain>
%\fi
%
% %%%%%%%%%%%%%%%%%%%%%%%%%%%%%%%%%%%%%%
% \paragraph{Chapter Include Files.}
%
% The include files are called |cdocsch1.tex| and |cdocsch2.tex|.
%
%\iffalse
%<*samplechap1|samplechap2>
%\fi

% Optional override for |\version| flag:
%    \begin{macrocode}
%%\providecommand{\version}{final}
%    \end{macrocode}

% Include the main document:
%    \begin{macrocode}
\input{childdoc.def}
\childdocof{cdocsamp}
%    \end{macrocode}

%\iffalse
%</samplechap1|samplechap2>
%\fi
%
%\iffalse
%<*samplechap1>
%\fi
% Some text for chapter 1:
%    \begin{macrocode}
\section{one}
some text in chapter one
%    \end{macrocode}

%\iffalse
%</samplechap1>
%\fi
% Some text for chapter 2:
%\iffalse
%<*samplechap2>
%\fi
%    \begin{macrocode}
\section{two}
more text in chapter two
%    \end{macrocode}

%\iffalse
%</samplechap2>
%\fi
%
% %%%%%%%%%%%%%%%%%%%%%%%%%%%%%%%%%%%%%%
% \paragraph{Part Include Files.}
%
% The include files are called |cdocspt3.tex| and |cdocspt4.tex|.
%
%\iffalse
%<*samplepart3|samplepart4>
%\fi

% Optional override for |\version| flag:
%    \begin{macrocode}
%%\providecommand{\version}{final}
%    \end{macrocode}

% Include the main document:
%    \begin{macrocode}
\input{childdoc.def}
\childdocby{cdocsamp}
%    \end{macrocode}

%\iffalse
%</samplepart3|samplepart4>
%\fi
%
%\iffalse
%<*samplepart3>
%\fi
% Some text for part 3:
%    \begin{macrocode}
some text in part three
%    \end{macrocode}

%\iffalse
%</samplepart3>
%\fi
% Some text for part 4:
%\iffalse
%<*samplepart4>
%\fi
%    \begin{macrocode}
more text in part four
%    \end{macrocode}

%\iffalse
%</samplepart4>
%\fi
%
% %%%%%%%%%%%%%%%%%%%%%%%%%%%%%%%%%%%%%%
% \paragraph{Forwarding for a Complete Draft.}
%
% The following forwarding file |cdocsdrf.tex|
% compiles the main document in draft mode:
%\iffalse
%<*sampledraft>
%\fi
%    \begin{macrocode}
\def\version{draft}
\input{childdoc.def}
\childdocforward{cdocsamp}
%    \end{macrocode}

%\iffalse
%</sampledraft>
%\fi
%
% %%%%%%%%%%%%%%%%%%%%%%%%%%%%%%%%%%%%%%
% \paragraph{Forwarding for Final Version of the Chapters.}
%
% The following forwarding files |cdocsfn1.tex| and |cdocsfn2.tex|
% (with identical content)
% compile the final versions of the child documents
% |cdocsch1.tex| and |cdocsch2.tex|, respectively:
%\iffalse
%<*samplefinal>
%\fi
%    \begin{macrocode}
\def\version{final}
\input{childdoc.def}
\childdocforwardprefix[cdocsamp]{cdocsfn}{cdocsch}
%    \end{macrocode}

%\iffalse
%</samplefinal>
%\fi
%
% %%%%%%%%%%%%%%%%%%%%%%%%%%%%%%%%%%%%%%
% \paragraph{Command Line Processing.}
%
% The following three command lines generate the output files
% |cdocscld|, |cdocscl1| and |cdocscl2|
% which should be identical to
% |cdocsdrf|, |cdocsch1| and |cdocsfn2|, respectively:
% \begin{center}
% \begin{tabular}{l}
% |latex -jobname cdocscld \|\\
% |  "\def\version{draft}\input{childdoc.def}\childdocforward{cdocsamp}"|\\
% |latex -jobname cdocscl1 \|\\
% |  "\input{childdoc.def}\childdocforward[cdocsamp]{cdocsch1}"|\\
% |latex -jobname cdocscl2 \|\\
% |  "\def\version{final}\input{childdoc.def}\childdocforward{cdocsch2}"|
% \end{tabular}
% \end{center}
% Note that the trailing backslash on each first line
% merely continues the input to the second line
% (for convenient cut ant paste).
% Furthermore, the command |latex| can be replaced by any
% of its alternative versions such as |pdflatex|.
%
% %%%%%%%%%%%%%%%%%%%%%%%%%%%%%%%%%%%%%%%%%%%%%%%%%%%%%%%%%%%%%%%%%%%%%%%%%%%%%%
% %%%%%%%%%%%%%%%%%%%%%%%%%%%%%%%%%%%%%%%%%%%%%%%%%%%%%%%%%%%%%%%%%%%%%%%%%%%%%%
% \section{Implementation}
%\iffalse
%<*package>
%\fi
%
% This section describes the definitions file |childdoc.def|.

% The definitions cannot be loaded using |\usepackage| or |\RequirePackage|
% which has a mechanism to prevent loading a style file more than once.
% When loading the definitions by means of |\input|
% multiple instances have to be prevented manually:
%\iffalse
%This code needs to be before the `\ProvidesFile' directive
%which is defined at the beginning of this file.
%Therefore it is also placed there and commented out here.
%</package>
%<*discard>
%\fi
%    \begin{macrocode}
\ifdefined\childdocmain\endinput\fi
%    \end{macrocode}
%\iffalse
%</discard>
%<*package>
%\fi
%
% \macro{\ifchilddoc}
% \macro{\ifchilddocmanual}
% The conditional |\ifchilddoc| tells whether a
% child (true) or main (false) document is being compiled.
% The conditional |\ifchilddocmanual| tells whether
% the |\includeonly| mechanism is used (false) or
% the selection of child files must be performed manually (true).
% The definitions initialise to false:
%    \begin{macrocode}
\newif\ifchilddoc
\newif\ifchilddocmanual
%    \end{macrocode}

% \macro{\childdocname}
% \macro{\childdocjob}
% The macro |\childdocname| stores the name of the main document
% to be compiled. The macro |\childdocjob| stores the name of
% the document on which the \LaTeX{} compiler was originally invoked.
% The content of |\jobname| cannot be compared
% to filenames specified in the source due to different catcodes.
% The following code rescans |\jobname|, stores the result
% in |\childdocname| and saves a copy in |\childdocjob|:
%    \begin{macrocode}
\edef\childdocname{\scantokens\expandafter{\jobname\noexpand}}
\let\childdocjob\childdocname
%    \end{macrocode}

% \macro{\childdocdisable}
% The macro |\childdocdisable| prevents the main file
% from being processed more than once.
% At this stage, the main document command |\childdocmain|
% is assumed to be called once again where it should do nothing.
% Any subsequent call to it should prevent
% a secondary processing of the main document
% It overwrites the forwarding commands
% |\childdocof| and |\childdocforward|
% with empty macros to prevent further inclusions of the main document:
%    \begin{macrocode}
\newcommand{\childdocdisable}
{
  \renewcommand{\childdocmain}[1]{\renewcommand{\childdocmain}[1]{\endinput}}
  \renewcommand{\childdocof}[1]{}
  \renewcommand{\childdocby}[2][]{}
  \renewcommand{\childdocforward}[2][]{}
  \renewcommand{\childdocdisable}{}
}
%    \end{macrocode}

% \macro{\childdocmain}
% The macro |\childdocmain| is to be called at the top of the main file
% with nothing or the main filename (without extension) as argument.
% First, it breaks loops.
% If the argument is not empty and does not match |\childdocname|
% (which is set by the first inclusion of |childdoc.def|),
% |\ifchilddoc| is set to true, |\includeonly| is applied to the child file
% and |\jobname| is set to the main file
% (for proper handling of |.aux| files):
%    \begin{macrocode}
\newcommand{\childdocmain}[1]
{
  \childdocdisable\childdocmain{}
  \if?#1?\else
    \begingroup
      \def\childdoctmp{#1}
      \ifx\childdoctmp\childdocname
        \def\childdoctmp{}
      \else
        \def\childdoctmp
        {
          \childdoctrue
          \includeonly{\childdocname}
          \def\childdocjob{#1}
          \def\jobname{#1}
        }
      \fi
      \expandafter
    \endgroup
    \childdoctmp
  \fi
}
%    \end{macrocode}

% \macro{\childdocof}
% The command |\childdocof| redirects
% compilation to the main file |#1|.
%    \begin{macrocode}
\newcommand{\childdocof}[1]
{
  \childdocdisable
  \childdoctrue
  \includeonly{\childdocname}
  \def\jobname{#1}
  \def\childdocjob{#1}
  \input{#1}
}
%    \end{macrocode}

% \macro{\childdocby}
% The command |\childdocby| ....
%    \begin{macrocode}
\newcommand{\childdocby}[2][]
{
  \childdocdisable
  \childdoctrue
  \childdocmanualtrue
  \if?#1?\else
    \def\jobname{#2}
  \fi
  \def\childdocjob{#2}
  \input{#2}
  \endinput
}
%    \end{macrocode}

% \macro{\childdocforward}
% The command |\childdocforward| redirects
% compilation to the main file or
% (if the optional argument is given) a child file.
% Parameters are set as if the main file
% or a child file starting with |\childdocof| was compiled.
% Then compilation is handed over to the main file:
%    \begin{macrocode}
\newcommand{\childdocforward}[2][]
{
  \begingroup
    \if?#1?
      \def\childdoctmp
      {
        \def\childdocname{#2}
        \def\childdocjob{#2}
        \def\jobname{#2}
        \input{#2}
        \endinput
      }
    \else
      \def\childdoctmp
      {
        \childdocdisable
        \def\childdocname{#2}
        \childdoctrue
        \includeonly{#2}
        \def\childdocjob{#1}
        \def\jobname{#1}
        \input{#1}
        \endinput
      }
    \fi
    \expandafter
  \endgroup
  \childdoctmp
}
%    \end{macrocode}

% \macro{\childdocforwardprefix}
% The command |\childdocforwardprefix| redirects
% compilation to the main or a child file by means of a pattern.
% The prefix |#1| in the current filename is replaced by |#2|
% and the suffix of the current filename is kept
% (it is assumed that the filename does not contain the substring `|~~~|'
% which is used as a delimiter).
% Compilation is handed over to the new file by |\childdocforward|:
%    \begin{macrocode}
\newcommand{\childdocforwardprefix}[3][]
{
  \begingroup
    \def\childdocextract #2##1~~~{\def\childdoctmp{\childdocforward[#1]{#3##1}}}
    \expandafter\childdocextract\childdocname~~~
    \expandafter
  \endgroup
  \childdoctmp
}
%    \end{macrocode}

% \macro{\childdoc}
% The deprecated macro |\childdoc| is a legacy version of |\childdocmain|:
%    \begin{macrocode}
\newcommand{\childdoc}{\childdocmain}
%    \end{macrocode}

% \macro{\childdocredirect}
% The deprecated macro |\childdocredirect| is a legacy version
% of |\childdocforward| and |\childdocforwardprefix|:
%    \begin{macrocode}
\newcommand{\childdocredirect}[2][]
{
  \begingroup
    \if?#1?
      \def\childdoctmp{\childdocforward{#2}}
    \else
      \def\childdoctmp{\childdocforwardprefix{#1}{#2}}
    \fi
    \expandafter
  \endgroup
  \childdoctmp
}
%    \end{macrocode}

%\iffalse
%</package>
%\fi
%
\endinput
|\\
|\childdocforwardprefix[|\textit{main}|]{|\textit{prefix}|}{|\textit{dest}|}|
\end{tabular}
\end{center}
%
the destination file is determined by a pattern
depending on the current file:
To make this work, the current file must be called
`{\textit{prefix}\hspace{0.2em}\textit{suffix}}'
with \textit{prefix} matching precisely the argument.
Processing is then passed on to the file
`{\textit{dest}\hspace{0.2em}\textit{suffix}}'.
Surely, the same effect is achieved by
directly specifying the
argument `{\textit{dest}\hspace{0.2em}\textit{suffix}}'
in the first form.
However, that requires to set up a different file
for each child. With the alternative form of the command
all these files can have exactly the same content
which simplifies setting them up and maintaining them.

For example, the following file |draft.tex|
with a compilation flag |\version| as described in \secref{sec:flags}
compiles the main document as a draft:
%
\begin{center}
\begin{tabular}{l}
|\def\version{draft}|\\
|% \iffalse
%
% childdoc.dtx Copyright (C) 2017-2018 Niklas Beisert
%
% This work may be distributed and/or modified under the
% conditions of the LaTeX Project Public License, either version 1.3
% of this license or (at your option) any later version.
% The latest version of this license is in
%   http://www.latex-project.org/lppl.txt
% and version 1.3 or later is part of all distributions of LaTeX
% version 2005/12/01 or later.
%
% This work has the LPPL maintenance status `maintained'.
%
% The Current Maintainer of this work is Niklas Beisert.
%
% This work consists of the files childdoc.dtx and childdoc.ins
% and the derived files childdoc.def and cdocsamp.tex with
% cdocsch1.tex, cdocsch2.tex, cdocsdrf.tex, cdocsfn1.tex, cdocsfn2.tex.
%
%<package>\ifdefined\childdocmain\endinput\fi
%<package>\ProvidesFile{childdoc.def}[2018/12/30 v2.0 child document driver]
%<samplemain>\ProvidesFile{cdocsamp.tex}[2018/12/30 v2.0 sample for childdoc]
%<*driver>
%\ProvidesFile{childdoc.drv}[2018/12/30 v2.0 childdoc reference manual file]
\PassOptionsToClass{10pt,a4paper}{article}
\documentclass{ltxdoc}

\usepackage[margin=35mm]{geometry}
\usepackage{hyperref}
\usepackage{hyperxmp}
\usepackage[usenames]{color}

\hypersetup{colorlinks=true}
\hypersetup{pdfstartview=FitH}
\hypersetup{pdfpagemode=UseNone}
\hypersetup{pdfsource={}}
\hypersetup{pdflang={en-UK}}
\hypersetup{pdfcopyright={Copyright 2017-2018 Niklas Beisert.
  This work may be distributed and/or modified under the
  conditions of the LaTeX Project Public License, either version 1.3
  of this license or (at your option) any later version.}}
\hypersetup{pdflicenseurl={http://www.latex-project.org/lppl.txt}}
\hypersetup{pdfcontactaddress={ETH Zurich, ITP, HIT K,
  Wolfgang-Pauli-Strasse 27}}
\hypersetup{pdfcontactpostcode={8093}}
\hypersetup{pdfcontactcity={Zurich}}
\hypersetup{pdfcontactcountry={Switzerland}}
\hypersetup{pdfcontactemail={nbeisert@itp.phys.ethz.ch}}
\hypersetup{pdfcontacturl={http://people.phys.ethz.ch/\xmptilde nbeisert/}}

\newcommand{\secref}[1]{\hyperref[#1]{section \ref*{#1}}}

\parskip1ex
\parindent0pt
\let\olditemize\itemize
\def\itemize{\olditemize\parskip0pt}

\begin{document}

\title{The \textsf{childdoc} Package}
\hypersetup{pdftitle={The childdoc Package}}
\author{Niklas Beisert\\[2ex]
  Institut f\"ur Theoretische Physik\\
  Eidgen\"ossische Technische Hochschule Z\"urich\\
  Wolfgang-Pauli-Strasse 27, 8093 Z\"urich, Switzerland\\[1ex]
  \href{mailto:nbeisert@itp.phys.ethz.ch}
  {\texttt{nbeisert@itp.phys.ethz.ch}}}
\hypersetup{pdfauthor={Niklas Beisert}}
\hypersetup{pdfsubject={Manual for the LaTeX2e Package childdoc}}
\date{30 December 2018, \textsf{v2.0}}
\maketitle

\begin{abstract}\noindent
\textsf{childdoc} is a \LaTeXe{} package
that enables the direct compilation
of document sections included by |\include|
to individual files.
\end{abstract}

\begingroup
\parskip0ex
\tableofcontents
\endgroup

%%%%%%%%%%%%%%%%%%%%%%%%%%%%%%%%%%%%%%%%%%%%%%%%%%%%%%%%%%%%%%%%%%%%%%%%%%%%%%%%
%%%%%%%%%%%%%%%%%%%%%%%%%%%%%%%%%%%%%%%%%%%%%%%%%%%%%%%%%%%%%%%%%%%%%%%%%%%%%%%%
\section{Introduction}

\LaTeX{} provides a mechanism to structure a large document (such as a book)
into a main file and several child files (containing the chapters)
using the |\include| command.
This mechanism is beneficial for documents
which span hundreds of pages in order to
make the source file(s) more manageable.
Moreover, compilation can be restricted to
selected child files by means of the |\includeonly| command.
The latter feature can be used to reduce the compilation time while editing
(this was significantly more useful in the earlier days of \LaTeX{})
or to generate a smaller document which is easier to navigate.
Another application of |\includeonly| is to generate
documents consisting of selected parts of the complete document.

However, there are a few drawbacks of the plain |\include| mechanism:
\begin{itemize}
\item
The child files cannot be compiled on their own,
they can only be compiled via the main file.
A naive editing environment
(such as a text editor with an option
to have the current file processed by \LaTeX)
may require one to switch to the main file before compiling;
attempting to compile the child file produces errors.
\item
The main file must be modified (each time)
to adjust the |\includeonly| command
to the present needs. This easily leaves the main file in a messy state.
\item
The generated document will always carry the filename
of the main document. This is inconvenient if
several child files are to be compiled and
to be kept for distribution.
\end{itemize}

The present package provides a simple interface
to make child files individually compilable by \LaTeX{}.
Compiling a child file then has the same effect as compiling
the main file with an |\includeonly| command
to select the appropriate child.
Moreover the generated document will carry the name of the child
rather than the main file.
This resolves all three above issues.

This feature is meant to make the editing of books,
thesis documents and lecture notes somewhat more convenient.
However, the package can also be used efficiently for
composing a series of documents (such as exercise sheets)
which are typically distributed individually.
It then assists the author in generating the individual documents
(potentially in different versions)
as well as a document containing the collected series.
Another application is in developing style files
or other kinds of included material
where compilation of the style file could redirect
to a sample or test file.

%%%%%%%%%%%%%%%%%%%%%%%%%%%%%%%%%%%%%%%%%%%%%%%%%%%%%%%%%%%%%%%%%%%%%%%%%%%%%%%%
%%%%%%%%%%%%%%%%%%%%%%%%%%%%%%%%%%%%%%%%%%%%%%%%%%%%%%%%%%%%%%%%%%%%%%%%%%%%%%%%
\section{Usage}

First of all, the package \textsf{childdoc} is \emph{not} a standard
\LaTeXe{} |.sty| style file! Therefore it needs to be invoked in
a non-standard way.

%%%%%%%%%%%%%%%%%%%%%%%%%%%%%%%%%%%%%%%%%%%%%%%%%%%%%%%%%%%%%%%%%%%%%%%%%%%%%%%%
\subsection{Included Files}
\label{sec:include}

%%%%%%%%%%%%%%%%%%%%%%%%%%%%%%%%%%%%%%%%
\DescribeMacro{\childdocmain}
To use the package, add the commands
\begin{center}
\begin{tabular}{l}
|\input{childdoc.def}|\\
|\childdocmain{}|\\
\end{tabular}
\end{center}
at the very top of the main \LaTeX{} file,
in particular \emph{before} the |\documentclass| statement!
The argument of |\childdocmain| should be left empty
(but it must be present).

%%%%%%%%%%%%%%%%%%%%%%%%%%%%%%%%%%%%%%%%
\DescribeMacro{\childdocof}
Furthermore, add the commands
\begin{center}
\begin{tabular}{l}
|\input{childdoc.def}|\\
|\childdocof{|\textit{main}|}|\\
\end{tabular}
\end{center}
at the top of every child file \textit{child}
which is included by |\include{|\textit{child}|}|
from within the main file
(or at least for those files to be compiled individually).
The argument \textit{main} must be the filename of the main file.

There are a couple of
considerations in setting up the main and child documents:

%%%%%%%%%%%%%%%%%%%%%%%%%%%%%%%%%%%%%%%%
\paragraph{Restrictions.}

Please note the following restrictions:
\begin{itemize}
\item
|\childdocmain| must be called with one argument \textit{main}
to ensure compatibility with earlier version of the package.
It must either be empty (|\childdocmain{}|)
or precisely match the filename of the main file in which it is specified.
See \secref{sec:detection} for further information.
\item
The filename \textit{main} must be specified without the |.tex| extension.
\item
The filename \textit{main} is case sensitive
(even in case-insensitive file systems)
due to internal string comparison.
\item
The argument \textit{main} should be fully expanded, it cannot be a macro.
\item
Subdirectories and special characters should be avoided in filenames.
\item
The command |\childdocmain{|\textit{main}|}| must be followed by a whitespace.
It should not be followed immediately by another command
or by a comment mark `|%|'.
This is because the \TeX{} parser reads the token immediately following
the argument of |\childdocmain| and puts it
at the beginning of every child section;
however, a white\-space is ignored.
\end{itemize}

%%%%%%%%%%%%%%%%%%%%%%%%%%%%%%%%%%%%%%%%
\paragraph{Content of Main File.}

It is advisable to place all content in the child files included by |\include|.
Any output contained in the main file will appear in all child documents
unless suppressed manually;
it cannot be suppressed automatically by the |\includeonly| directive
and thus should normally be avoided.
A method to include some content in the main file
by means of conditional processing is described in \secref{sec:conditional}.

%%%%%%%%%%%%%%%%%%%%%%%%%%%%%%%%%%%%%%%%
\paragraph{Page Numbering.}

When only a part of the document is compiled,
the appropriate numbering of pages
(as well as other status parameters)
is determined from the |.aux| files.
The latter contain information from previous passes.
However this information needs to propagate through
all intermediate child documents.
Therefore the page numbering in child documents may well
be inconsistent until the complete document is compiled at least once.

A useful (if unconventional) way to always ensure a consistent
page numbering is to restart the numbering in each child document
and denote the pages by `\textit{child}|.|\textit{page}'
where \textit{child} represents the chapter/section number of the child file.
This can be achieved by the command
|\numberwithin{page}{|\textit{child}|}|
of the \textsf{amsmath} package
where \textit{child} can be |chapter| or |section|
depending on the chosen structuring.
Alternatively, one can modify the macro |\thepage| appropriately
and reset the counter |page| at the start of each child file.

%%%%%%%%%%%%%%%%%%%%%%%%%%%%%%%%%%%%%%%%%%%%%%%%%%%%%%%%%%%%%%%%%%%%%%%%%%%%%%%%
\subsection{Conditional Processing}
\label{sec:conditional}

The package provides a mechanism to compile different versions
of a document. To customise the versions further some conditional processing
can come in handy to distinguish which version is being compiled.
The package provides two macros to describe the compilation context:

%%%%%%%%%%%%%%%%%%%%%%%%%%%%%%%%%%%%%%%%
\DescribeMacro{\ifchilddoc}
The conditional |\ifchilddoc| distinguishes between the compilation of
child documents and the main document:
%
\begin{center}
|\ifchilddoc |\textit{child-code}| |[|\||else |\textit{main-code}]| \||fi|
\end{center}

%%%%%%%%%%%%%%%%%%%%%%%%%%%%%%%%%%%%%%%%
\DescribeMacro{\childdocname}
\DescribeMacro{\childdocjob}
The macro |\childdocname| contains the filename (without extension)
of the main or child file being processed.
Note that |\childdocjob| will always contain the name of the main file.

%%%%%%%%%%%%%%%%%%%%%%%%%%%%%%%%%%%%%%%%
\paragraph{Title Page.}

Conditional processing can be used to include a title or banner page
in the main document when proper precautions are taken.
Importantly, the code in the main file should ensure that the page counter
(as well as other status parameters which are stored in the |.aux| files)
takes the same value after the conditional processing.
Otherwise the page numbers may take divergent values
depending on which part is compiled.

For example, a title page could be declared by:
%
\begin{center}
\begin{tabular}{l}
|\ifchilddoc\||else|\\
|\addtocounter{page}{-1}|\\
\textit{code for title page}\\
|\newpage|\\
|\||fi|
\end{tabular}
\end{center}
%
A banner page for the child documents can be generated by:
%
\begin{center}
\begin{tabular}{l}
|\ifchilddoc|\\
|\addtocounter{page}{-1}|\\
\textit{code for banner page}\\
|\newpage|\\
|\||fi|
\end{tabular}
\end{center}
%
Here one could write a message such as:
\begin{center}
|This is the part \childdocname{} of \childdocjob{}.|
\end{center}

%%%%%%%%%%%%%%%%%%%%%%%%%%%%%%%%%%%%%%%%%%%%%%%%%%%%%%%%%%%%%%%%%%%%%%%%%%%%%%%%
\subsection{Flags}
\label{sec:flags}

The package makes it easy to generate different versions
of the main or child documents.
To this end compilation flags can be defined
and assigned different default values.
They will be particularly useful in conjunction
with the forwarding mechanism described in \secref{sec:forward}.

For example, it may be useful to have a flag |\version|
which can be set to |draft| or |final|.
The document source will contain some conditional code
depending on the value of |\version|.
Suppose further, the flag should default to |final| for the main file
and to |draft| for child files
which is a natural assignment for editing the document.
This is achieved by placing the following code
in the preamble of the main document
(below the |\childdocmain| directive):
%
\begin{center}
\begin{tabular}{l}
|\ifchilddoc|\\
|\providecommand{\version}{draft}|\\
|\||else|\\
|\providecommand{\version}{final}|\\
|\||fi|
\end{tabular}
\end{center}
%
The definition by |\providecommand| makes sure
that previous definitions are not overwritten.
Further statements |\providecommand{\version}{...}|
can thus be added before the above code to override it.

For the main file, one might add a line
(between |\childdocmain| and the above block)
%
\begin{center}
|%\ifchilddoc\||else\providecommand{\version}{draft}\||fi|
\end{center}
%
which can be uncommented to produce a draft version.
Likewise one can add a line to the very top of a child file
(above the |\childdocof{|\textit{main}|}| directive)
%
\begin{center}
|%\providecommand{\version}{final}|
\end{center}
%
which can be uncommented to produce the final version of this child document.

%%%%%%%%%%%%%%%%%%%%%%%%%%%%%%%%%%%%%%%%%%%%%%%%%%%%%%%%%%%%%%%%%%%%%%%%%%%%%%%%
\subsection{Forwarding}
\label{sec:forward}

Different versions of the main or child documents
using compilation flags as described in \secref{sec:flags}
can be (permanently) stored in different files
for convenient compilation, viewing and distribution.
To this end, the package defines a command
to pass on compilation to a different file:

%%%%%%%%%%%%%%%%%%%%%%%%%%%%%%%%%%%%%%%%
\DescribeMacro{\childdocforward}
The command |\childdocforward| redirects processing to
another source file:
%
\begin{center}
\begin{tabular}{l}
|\input{childdoc.def}|\\
|\childdocforward[|\textit{main}|]{|\textit{dest}|}|\\
\end{tabular}
\end{center}
%
The argument \textit{dest} is the destination file
(without extension).
It should be the main file or one of the child files.
Note that further \textsf{childdoc} directives
such as |\childdocof| and |\childdocforward|
in the indicated file will be processed in this form.
The optional argument \textit{main}
passes on directly to the main file \textit{main}
while pretending to compile the child \textit{dest}.
This form behaves as if \textit{dest}
issues |\childdocof{|\textit{main}|}| right away,
and no further \textsf{childdoc} directives will be processed.

%%%%%%%%%%%%%%%%%%%%%%%%%%%%%%%%%%%%%%%%
\DescribeMacro{\...prefix}
In the alternative form |\childdocforwardprefix|,
%
\begin{center}
\begin{tabular}{l}
|\input{childdoc.def}|\\
|\childdocforwardprefix[|\textit{main}|]{|\textit{prefix}|}{|\textit{dest}|}|
\end{tabular}
\end{center}
%
the destination file is determined by a pattern
depending on the current file:
To make this work, the current file must be called
`{\textit{prefix}\hspace{0.2em}\textit{suffix}}'
with \textit{prefix} matching precisely the argument.
Processing is then passed on to the file
`{\textit{dest}\hspace{0.2em}\textit{suffix}}'.
Surely, the same effect is achieved by
directly specifying the
argument `{\textit{dest}\hspace{0.2em}\textit{suffix}}'
in the first form.
However, that requires to set up a different file
for each child. With the alternative form of the command
all these files can have exactly the same content
which simplifies setting them up and maintaining them.

For example, the following file |draft.tex|
with a compilation flag |\version| as described in \secref{sec:flags}
compiles the main document as a draft:
%
\begin{center}
\begin{tabular}{l}
|\def\version{draft}|\\
|\input{childdoc.def}|\\
|\childdocforward{|\textit{main}|}|
\end{tabular}
\end{center}
%
Likewise, the following files |final|\textit{nn}|.tex|
compile the final version of the child document
|child|\textit{nn}|.tex|:
%
\begin{center}
\begin{tabular}{l}
|\def\version{final}|\\
|\input{childdoc.def}|\\
|\childdocforwardprefix{final}{child}|
\end{tabular}
\end{center}
%

Note that when several versions of a main file and/or of each child file
are to be generated, it may be convenient to set up a |Makefile| or
shell script to automatise the process.

%%%%%%%%%%%%%%%%%%%%%%%%%%%%%%%%%%%%%%%%%%%%%%%%%%%%%%%%%%%%%%%%%%%%%%%%%%%%%%%%
\subsection{Command Line Processing}
\label{sec:commandline}

The effect of redirection files can also be achieved by invoking
the \LaTeX{} compiler with a more elaborate command line.
Most conveniently this should be done as part
of a shell script or a |Makefile|.

When using \textsf{childdoc} in the main file, the following
command lines effectively perform a redirection
(note that depending on the shell being used,
backslashes may have to be doubled: `|\|' $\to$ `|\\|'):
%
\begin{center}
|... -jobname "|\textit{target}|" |\\|"|[\textit{flags}]%
|\input{childdoc.def}\childdocforward[|\textit{main}|]{|\textit{dest}|}"|
\end{center}
%
Here \textit{target} is the name of the output file,
\textit{main} is the name of the main file
and \textit{dest} is the name of the main or child file to be processed
(all filenames without extensions).
The optional argument \textit{main} can be omitted
if \textit{main} matches \textit{dest}.
Optionally, compilation \textit{flags} can be defined via |\def| commands.
This command line makes the \TeX{} engine believe
it is compiling the file \textit{target}
whose content is specified as the latter parameter.
The provided code then forwards the processing to
\textit{main} or \textit{dest} as described in \secref{sec:forward}.

%%%%%%%%%%%%%%%%%%%%%%%%%%%%%%%%%%%%%%%%%%%%%%%%%%%%%%%%%%%%%%%%%%%%%%%%%%%%%%%%
\subsection{Include by Input}
\label{sec:input}

Including child documents by |\include| has some restrictions by design.
Most notably, the content of a child document always occupies
its own set of pages; pages cannot be shared between child documents.
Usually, this behaviour makes perfect sense
because each child document contain an essential part of the document.
However, in some situations it may be desirable to compose
a document from a collection of parts
without having mandatory page breaks between then.
For this case, the package
provides a mechanism to include parts
by |\input| which can also be processed individually.
However, by construction this mechanism
requires manual handling of the content to be output.

%%%%%%%%%%%%%%%%%%%%%%%%%%%%%%%%%%%%%%%%
\DescribeMacro{\ifchilddocmanual}
The main file should be prepared as usual, see \secref{sec:include}.
However, the document body must make a distinction
between processing of an individual part and of the main document, e.g.:
%
\begin{center}
\begin{tabular}{l}
|\ifchilddocmanual|\\
|\input{\childdocname}|\\
|\||else|\\
\textit{document body with }|\input{|\textit{part}|}|\\
|\||fi|
\end{tabular}
\end{center}
%
The conditional |\ifchilddocmanual| is true whenever
a part to be included by |\input| is being compiled,
and the name of the part is stored in |\childdocname|.

%%%%%%%%%%%%%%%%%%%%%%%%%%%%%%%%%%%%%%%%
\DescribeMacro{\childdocby}
Each part to be included by |\input| should start with:
%
\begin{center}
\begin{tabular}{l}
|\input{childdoc.def}|\\
|\childdocby{|\textit{main}|}|\\
\end{tabular}
\end{center}
%
The directive |\childdocby| is similar to |\childdocof|
described in \secref{sec:include},
but the subsequent selection of content must be done manually.
To that end, both |\ifchilddoc| and |\ifchilddocmanual|
will be true upon processing of a part,
and the name of the part is stored in |\childdocname|.
Note that |\jobname| will be set to the filename of the current part
so that each part receives an individual |.aux| file
that does not interfere with the |.aux| file(s) of the main document.
This behaviour can be altered by the alternative form
|\childdocby[*]{|\textit{main}|}| (with a non-empty optional argument)
which uses the |.aux| file of the main document
by setting |\jobname| to \textit{main}.

%%%%%%%%%%%%%%%%%%%%%%%%%%%%%%%%%%%%%%%%%%%%%%%%%%%%%%%%%%%%%%%%%%%%%%%%%%%%%%%%
\subsection{Driver Development}
\label{sec:driver}

The \textsf{childdoc} mechanism can also be use for the development
of definition files such as \LaTeX{} styles or classes.
This case differs from the above setup with multiple parts
included by |\include| in that no |\includeonly| should be invoked.
This can be achieved by starting the include file
(before |\ProvidesPackage|) with:
%
\begin{center}
\begin{tabular}{l}
|\input{childdoc.def}|\\
|\childdocforward{|\textit{main}|}|\\
\end{tabular}
\end{center}
%
or alternatively with:
%
\begin{center}
\begin{tabular}{l}
|\input{childdoc.def}|\\
|\childdocby{|\textit{main}|}|\\
\end{tabular}
\end{center}
%
Both forms have slightly different effects as described above.
The main file is prepared as usual, see \secref{sec:include}.

%%%%%%%%%%%%%%%%%%%%%%%%%%%%%%%%%%%%%%%%%%%%%%%%%%%%%%%%%%%%%%%%%%%%%%%%%%%%%%%%
\subsection{Legacy Detection}
\label{sec:detection}

The directive |\childdocmain| in the main file can detect
whether the complete document or merely a child is to be compiled
even without using the directive |\childdocof|.
This method is deprecated because it is less robust
and there is no compelling reason to use it;
it is merely provided for backward compatibility
and it may be removed in future versions.

If the detection mechanism is to be used,
it is mandatory to correctly specify
the filename of the main file as the argument of |\childdocmain|:
%
\begin{center}
\begin{tabular}{l}
|\input{childdoc.def}|\\
|\childdocmain{|\textit{main}|}|\\
\end{tabular}
\end{center}
%
If |\jobname| does not match the argument \textit{main} of |\childdocmain|,
it is assumed that |\jobname| points to the child file to be compiled.
When using |\childdocmain| with the main file specified as argument,
it suffices to start a child file
with just |\input{|\textit{main}|}|
without loading of the package and using |\childdocof|.
If instead all processing is done
with the appropriate \textsf{childdoc} directives,
the argument of \textit{main} of |\childdocmain| can be empty.

An alternative version of the command line processing described
in \secref{sec:commandline} using the detection mechanism reads:
%
\begin{center}
|... -jobname "|\textit{target}|" "|[\textit{flags}]%
[|\def\jobname{|\textit{dest}|}|]|\input{|\textit{main}|}"|
\end{center}

%%%%%%%%%%%%%%%%%%%%%%%%%%%%%%%%%%%%%%%%%%%%%%%%%%%%%%%%%%%%%%%%%%%%%%%%%%%%%%%%
\subsection{Manual Code}
\label{sec:manual}

In case one cannot be certain whether the definitions file |childdoc.def|
is installed on the target \TeX{} distribution
and one prefers not to ship it,
it is conceivable to paste a few relevant commands into the sources.

To that end, drop all statements |\input{childdoc.def}|
and perform the replacements as outlined below.
Instead of |\childdocmain{|\textit{main}|}| add the following code
to the top of the main file:
%
\begin{center}
\begin{tabular}{l}
|\||ifdefined\childdocname\endinput\||fi\newif\ifchilddoc|\\
|\edef\childdocname{\scantokens\expandafter{\jobname\noexpand}}|\\
|\def\childdocmain{|\textit{main}|}\||ifx\childdocmain\childdocname\||else|\\
|\childdoctrue\includeonly{\childdocname}\let\jobname\childdocmain\||fi|\\
\end{tabular}
\end{center}
%
Instead of |\childdocof{|\textit{main}|}| just include the main file
at the top of each child file:
%
\begin{center}
|\input{|\textit{main}|}|
\end{center}
%
A simple redirection |\childdocforward{|\textit{dest}|}| is achieved by:
%
\begin{center}
|\def\jobname{|\textit{dest}|}\input{\jobname}|
\end{center}
%
The redirection with prefix
|\childdocforwardprefix[|\textit{prefix}|]{|\textit{dest}|}|
is accomplished by:
%
\begin{center}
\begin{tabular}{l}
|{\edef\jobname{\scantokens\expandafter{\jobname\noexpand}}|\\
|\def\redirectjob |\textit{prefix}|#1~~~{\gdef\jobname{|\textit{dest}|#1}}|\\
|\expandafter\redirectjob\jobname~~~}\input{\jobname}|
\end{tabular}
\end{center}

In an alternative approach,
child documents can be compiled by a specific command line
without additional code or specific definitions:
%
\begin{center}
|... -jobname "|\textit{target}|" "|[\textit{flags}]%
|\includeonly{|\textit{dest}|}\input{|\textit{main}|}"|
\end{center}
%

%%%%%%%%%%%%%%%%%%%%%%%%%%%%%%%%%%%%%%%%%%%%%%%%%%%%%%%%%%%%%%%%%%%%%%%%%%%%%%%%
%%%%%%%%%%%%%%%%%%%%%%%%%%%%%%%%%%%%%%%%%%%%%%%%%%%%%%%%%%%%%%%%%%%%%%%%%%%%%%%%
\section{Information}

%%%%%%%%%%%%%%%%%%%%%%%%%%%%%%%%%%%%%%%%%%%%%%%%%%%%%%%%%%%%%%%%%%%%%%%%%%%%%%%%
\subsection{Copyright}

Copyright \copyright{} 2017--2018 Niklas Beisert

This work may be distributed and/or modified under the
conditions of the \LaTeX{} Project Public License, either version 1.3
of this license or (at your option) any later version.
The latest version of this license is in
  \url{http://www.latex-project.org/lppl.txt}
and version 1.3 or later is part of all distributions of \LaTeX{}
version 2005/12/01 or later.

This work has the LPPL maintenance status `maintained'.

The Current Maintainer of this work is Niklas Beisert.

This work consists of the files |README.txt|, |childdoc.ins| and |childdoc.dtx|
as well as the derived files |childdoc.def|, |cdocsamp.tex|
with |cdocsch1.tex|, |cdocsch2.tex|, |cdocspt3.tex|, |cdocspt4.tex|,
|cdocsdrf.tex|, |cdocsfn1.tex|, |cdocsfn2.tex|
as well as |childdoc.pdf|.

%%%%%%%%%%%%%%%%%%%%%%%%%%%%%%%%%%%%%%%%%%%%%%%%%%%%%%%%%%%%%%%%%%%%%%%%%%%%%%%%
\subsection{Files and Installation}

The package consists of the files:
%
\begin{center}
\begin{tabular}{ll}
    |README.txt|   & readme file \\
    |childdoc.ins| & installation file \\
    |childdoc.dtx| & source file \\
    |childdoc.def| & definition file \\
    |cdocsamp.tex| & sample main file \\
    |cdocsch1.tex| & sample include file \\
    |cdocsch2.tex| & sample include file \\
    |cdocspt3.tex| & sample part file \\
    |cdocspt4.tex| & sample part file \\
    |cdocsdrf.tex| & sample redirection file \\
    |cdocsfn1.tex| & sample redirection file \\
    |cdocsfn2.tex| & sample redirection file \\
    |childdoc.pdf| & manual
\end{tabular}
\end{center}
%
The distribution consists of the files
|README.txt|, |childdoc.ins| and |childdoc.dtx|.
%
\begin{itemize}
\item
Run (pdf)\LaTeX{} on |childdoc.dtx|
to compile the manual |childdoc.pdf| (this file).
\item
Run \LaTeX{} on |childdoc.ins| to create the definitions file |childdoc.def|
and the sample |cdocsamp.tex| with include files
|cdocsch1.tex|, |cdocsch2.tex|, |cdocspt3.tex|, |cdocspt4.tex|,
|cdocsdrf.tex|, |cdocsfn1.tex|, |cdocsfn2.tex|.
Then copy the file |childdoc.def| to an appropriate directory of your \LaTeX{}
distribution, e.g.\ \textit{texmf-root}|/tex/latex/childdoc|.
\end{itemize}

%%%%%%%%%%%%%%%%%%%%%%%%%%%%%%%%%%%%%%%%%%%%%%%%%%%%%%%%%%%%%%%%%%%%%%%%%%%%%%%%
\subsection{Related CTAN Packages}

There are several other packages which offer a similar functionality:
%
\begin{itemize}
\item
The packages
\href{http://ctan.org/pkg/docmute}{\textsf{docmute}},
\href{http://ctan.org/pkg/includex}{\textsf{includex}} and
\href{http://ctan.org/pkg/standalone}{\textsf{standalone}}
provide commands to include only the document body of
a child file thus allowing both files to be compiled individually.
\item
The packages \href{http://ctan.org/pkg/subdocs}{\textsf{subdocs}}
and \href{http://ctan.org/pkg/subfiles}{\textsf{subfiles}}
provide structures in which the main and child documents can be
encapsulated and allowing them to be compiled individually.
The inclusion mechanism is different from the conventional |\include|.
\item
The package \href{http://ctan.org/pkg/combine}{\textsf{combine}}
is an elaborate solution to combine several documents into one.
\end{itemize}
%
See also the CTAN topic \href{http://ctan.org/topic/subdocs}{\textsf{subdocs}}
for further related packages.
The present package differs from the above solutions in that
a document structure constructed with the conventional |\include| mechanism
just needs two extra commands at the top of every file
such that all constituent files can be compiled individually.

%%%%%%%%%%%%%%%%%%%%%%%%%%%%%%%%%%%%%%%%%%%%%%%%%%%%%%%%%%%%%%%%%%%%%%%%%%%%%%%%
%\subsection{Feature Suggestions}
%
%The following is a list of features which may be useful for future
%versions of this package:
%%
%\begin{itemize}
%\item
%\ldots
%\end{itemize}

%%%%%%%%%%%%%%%%%%%%%%%%%%%%%%%%%%%%%%%%%%%%%%%%%%%%%%%%%%%%%%%%%%%%%%%%%%%%%%%%
\subsection{Revision History}

%%%%%%%%%%%%%%%%%%%%%%%%%%%%%%%%%%%%%%%%
\paragraph{v2.0:} 2018/12/30

\begin{itemize}
\item
immediate forward processing
\item
added |\childdocby| mechanism
\item
manual restructured
\end{itemize}

%%%%%%%%%%%%%%%%%%%%%%%%%%%%%%%%%%%%%%%%
\paragraph{v1.6:} 2018/01/17

\begin{itemize}
\item
application for development of include files
\item
corrections to manual
\end{itemize}

%%%%%%%%%%%%%%%%%%%%%%%%%%%%%%%%%%%%%%%%
\paragraph{v1.5:} 2017/05/21

\begin{itemize}
\item
more complete structuring introduced
\item
|\childdocof| introduced
\item
|\childdoc| renamed to |\childdocmain|
\item
|\childredirect| renamed to |\childdocforward| and |\childdocforwardprefix|
and functionality expanded
\end{itemize}

%%%%%%%%%%%%%%%%%%%%%%%%%%%%%%%%%%%%%%%%
\paragraph{v1.0:} 2017/04/27

\begin{itemize}
\item
manual and install package
\item
first version published on CTAN
\end{itemize}

%%%%%%%%%%%%%%%%%%%%%%%%%%%%%%%%%%%%%%%%
\paragraph{v0.6:} 2017/04/26

\begin{itemize}
\item
redirection mechanism added
\end{itemize}

%%%%%%%%%%%%%%%%%%%%%%%%%%%%%%%%%%%%%%%%
\paragraph{v0.5:} 2017/04/26

\begin{itemize}
\item
functionality in definition file
\end{itemize}


%%%%%%%%%%%%%%%%%%%%%%%%%%%%%%%%%%%%%%%%%%%%%%%%%%%%%%%%%%%%%%%%%%%%%%%%%%%%%%%%
%%%%%%%%%%%%%%%%%%%%%%%%%%%%%%%%%%%%%%%%%%%%%%%%%%%%%%%%%%%%%%%%%%%%%%%%%%%%%%%%
%%%%%%%%%%%%%%%%%%%%%%%%%%%%%%%%%%%%%%%%%%%%%%%%%%%%%%%%%%%%%%%%%%%%%%%%%%%%%%%%
\appendix

\settowidth\MacroIndent{\rmfamily\scriptsize 000\ }

 \DocInput{childdoc.dtx}

\end{document}
%</driver>
% \fi
%
% %%%%%%%%%%%%%%%%%%%%%%%%%%%%%%%%%%%%%%%%%%%%%%%%%%%%%%%%%%%%%%%%%%%%%%%%%%%%%%
% %%%%%%%%%%%%%%%%%%%%%%%%%%%%%%%%%%%%%%%%%%%%%%%%%%%%%%%%%%%%%%%%%%%%%%%%%%%%%%
% \section{Sample}
%\iffalse
%<*samplemain>
%\fi
%
% The following presents a sample document
% with two chapters, two parts, a title page,
% a compile flag as well as three forwarding files to set the flag.
% It consists of eight |.tex| files:
% \begin{center}
% \begin{tabular}{ll}
% |cdocsamp.tex|&main file\\
% |cdocsch1.tex|&include file for chapter 1\\
% |cdocsch2.tex|&include file for chapter 2\\
% |cdocspt3.tex|&include file for part 3\\
% |cdocspt4.tex|&include file for part 4\\
% |cdocsdrf.tex|&forwarding file for main file in draft mode\\
% |cdocsfi1.tex|&forwarding file for final version of chapter 1\\
% |cdocsfi2.tex|&forwarding file for final version of chapter 2\\
% \end{tabular}
% \end{center}
% Each of the eight files can be compiled directly by the \LaTeX{} compiler.
%
% %%%%%%%%%%%%%%%%%%%%%%%%%%%%%%%%%%%%%%
% \paragraph{Main File.}
%
% The main file is called |cdocsamp.tex|.
%
% Load the \textsf{childdoc} definitions and
% declare the filename for the main document:
%    \begin{macrocode}
\input{childdoc.def}
\childdocmain{}
%    \end{macrocode}

% Optional override for |\version| flag:
%    \begin{macrocode}
%%\ifchilddoc\else\providecommand{\version}{draft}\fi
%    \end{macrocode}

% Define the default values for the |\version| flag
% (|final| for the main file and |draft| for childs):
%    \begin{macrocode}
\ifchilddoc
\providecommand{\version}{draft}
\else
\providecommand{\version}{final}
\fi
%    \end{macrocode}

% Load the standard document class:
%    \begin{macrocode}
\documentclass[12pt]{article}
%    \end{macrocode}

% Start the document body:
%    \begin{macrocode}
\begin{document}
%    \end{macrocode}

% Declare a title page.
% Print title, part of document being processed and version flag:
%    \begin{macrocode}
\addtocounter{page}{-1}
\begin{center}
{\LARGE\bfseries{}childdoc example\par}
\vspace{1cm}
\ifchilddoc
\ifchilddocmanual part\else chapter\fi:
`\childdocname' of `\childdocjob'\par
\else
main document: `\childdocjob'\par
\fi
version: \version\par
\end{center}
\newpage
%    \end{macrocode}

% Manually include selected file,
% otherwise process as usual:
%    \begin{macrocode}
\ifchilddocmanual
\section*{part `\childdocname'}
\input{\childdocname}
\else
%    \end{macrocode}

% Include the two chapters:
%    \begin{macrocode}
\include{cdocsch1}
\include{cdocsch2}
%    \end{macrocode}

% Include the two parts unless only chapters should be displayed:
%    \begin{macrocode}
\ifchilddoc\else
\section{part three}
\input{cdocspt3}
\section{part four}
\input{cdocspt4}
\fi
%    \end{macrocode}

% Process as usual until here:
%    \begin{macrocode}
\fi
%    \end{macrocode}

% End of document body:
%    \begin{macrocode}
\end{document}
%    \end{macrocode}
%\iffalse
%</samplemain>
%\fi
%
% %%%%%%%%%%%%%%%%%%%%%%%%%%%%%%%%%%%%%%
% \paragraph{Chapter Include Files.}
%
% The include files are called |cdocsch1.tex| and |cdocsch2.tex|.
%
%\iffalse
%<*samplechap1|samplechap2>
%\fi

% Optional override for |\version| flag:
%    \begin{macrocode}
%%\providecommand{\version}{final}
%    \end{macrocode}

% Include the main document:
%    \begin{macrocode}
\input{childdoc.def}
\childdocof{cdocsamp}
%    \end{macrocode}

%\iffalse
%</samplechap1|samplechap2>
%\fi
%
%\iffalse
%<*samplechap1>
%\fi
% Some text for chapter 1:
%    \begin{macrocode}
\section{one}
some text in chapter one
%    \end{macrocode}

%\iffalse
%</samplechap1>
%\fi
% Some text for chapter 2:
%\iffalse
%<*samplechap2>
%\fi
%    \begin{macrocode}
\section{two}
more text in chapter two
%    \end{macrocode}

%\iffalse
%</samplechap2>
%\fi
%
% %%%%%%%%%%%%%%%%%%%%%%%%%%%%%%%%%%%%%%
% \paragraph{Part Include Files.}
%
% The include files are called |cdocspt3.tex| and |cdocspt4.tex|.
%
%\iffalse
%<*samplepart3|samplepart4>
%\fi

% Optional override for |\version| flag:
%    \begin{macrocode}
%%\providecommand{\version}{final}
%    \end{macrocode}

% Include the main document:
%    \begin{macrocode}
\input{childdoc.def}
\childdocby{cdocsamp}
%    \end{macrocode}

%\iffalse
%</samplepart3|samplepart4>
%\fi
%
%\iffalse
%<*samplepart3>
%\fi
% Some text for part 3:
%    \begin{macrocode}
some text in part three
%    \end{macrocode}

%\iffalse
%</samplepart3>
%\fi
% Some text for part 4:
%\iffalse
%<*samplepart4>
%\fi
%    \begin{macrocode}
more text in part four
%    \end{macrocode}

%\iffalse
%</samplepart4>
%\fi
%
% %%%%%%%%%%%%%%%%%%%%%%%%%%%%%%%%%%%%%%
% \paragraph{Forwarding for a Complete Draft.}
%
% The following forwarding file |cdocsdrf.tex|
% compiles the main document in draft mode:
%\iffalse
%<*sampledraft>
%\fi
%    \begin{macrocode}
\def\version{draft}
\input{childdoc.def}
\childdocforward{cdocsamp}
%    \end{macrocode}

%\iffalse
%</sampledraft>
%\fi
%
% %%%%%%%%%%%%%%%%%%%%%%%%%%%%%%%%%%%%%%
% \paragraph{Forwarding for Final Version of the Chapters.}
%
% The following forwarding files |cdocsfn1.tex| and |cdocsfn2.tex|
% (with identical content)
% compile the final versions of the child documents
% |cdocsch1.tex| and |cdocsch2.tex|, respectively:
%\iffalse
%<*samplefinal>
%\fi
%    \begin{macrocode}
\def\version{final}
\input{childdoc.def}
\childdocforwardprefix[cdocsamp]{cdocsfn}{cdocsch}
%    \end{macrocode}

%\iffalse
%</samplefinal>
%\fi
%
% %%%%%%%%%%%%%%%%%%%%%%%%%%%%%%%%%%%%%%
% \paragraph{Command Line Processing.}
%
% The following three command lines generate the output files
% |cdocscld|, |cdocscl1| and |cdocscl2|
% which should be identical to
% |cdocsdrf|, |cdocsch1| and |cdocsfn2|, respectively:
% \begin{center}
% \begin{tabular}{l}
% |latex -jobname cdocscld \|\\
% |  "\def\version{draft}\input{childdoc.def}\childdocforward{cdocsamp}"|\\
% |latex -jobname cdocscl1 \|\\
% |  "\input{childdoc.def}\childdocforward[cdocsamp]{cdocsch1}"|\\
% |latex -jobname cdocscl2 \|\\
% |  "\def\version{final}\input{childdoc.def}\childdocforward{cdocsch2}"|
% \end{tabular}
% \end{center}
% Note that the trailing backslash on each first line
% merely continues the input to the second line
% (for convenient cut ant paste).
% Furthermore, the command |latex| can be replaced by any
% of its alternative versions such as |pdflatex|.
%
% %%%%%%%%%%%%%%%%%%%%%%%%%%%%%%%%%%%%%%%%%%%%%%%%%%%%%%%%%%%%%%%%%%%%%%%%%%%%%%
% %%%%%%%%%%%%%%%%%%%%%%%%%%%%%%%%%%%%%%%%%%%%%%%%%%%%%%%%%%%%%%%%%%%%%%%%%%%%%%
% \section{Implementation}
%\iffalse
%<*package>
%\fi
%
% This section describes the definitions file |childdoc.def|.

% The definitions cannot be loaded using |\usepackage| or |\RequirePackage|
% which has a mechanism to prevent loading a style file more than once.
% When loading the definitions by means of |\input|
% multiple instances have to be prevented manually:
%\iffalse
%This code needs to be before the `\ProvidesFile' directive
%which is defined at the beginning of this file.
%Therefore it is also placed there and commented out here.
%</package>
%<*discard>
%\fi
%    \begin{macrocode}
\ifdefined\childdocmain\endinput\fi
%    \end{macrocode}
%\iffalse
%</discard>
%<*package>
%\fi
%
% \macro{\ifchilddoc}
% \macro{\ifchilddocmanual}
% The conditional |\ifchilddoc| tells whether a
% child (true) or main (false) document is being compiled.
% The conditional |\ifchilddocmanual| tells whether
% the |\includeonly| mechanism is used (false) or
% the selection of child files must be performed manually (true).
% The definitions initialise to false:
%    \begin{macrocode}
\newif\ifchilddoc
\newif\ifchilddocmanual
%    \end{macrocode}

% \macro{\childdocname}
% \macro{\childdocjob}
% The macro |\childdocname| stores the name of the main document
% to be compiled. The macro |\childdocjob| stores the name of
% the document on which the \LaTeX{} compiler was originally invoked.
% The content of |\jobname| cannot be compared
% to filenames specified in the source due to different catcodes.
% The following code rescans |\jobname|, stores the result
% in |\childdocname| and saves a copy in |\childdocjob|:
%    \begin{macrocode}
\edef\childdocname{\scantokens\expandafter{\jobname\noexpand}}
\let\childdocjob\childdocname
%    \end{macrocode}

% \macro{\childdocdisable}
% The macro |\childdocdisable| prevents the main file
% from being processed more than once.
% At this stage, the main document command |\childdocmain|
% is assumed to be called once again where it should do nothing.
% Any subsequent call to it should prevent
% a secondary processing of the main document
% It overwrites the forwarding commands
% |\childdocof| and |\childdocforward|
% with empty macros to prevent further inclusions of the main document:
%    \begin{macrocode}
\newcommand{\childdocdisable}
{
  \renewcommand{\childdocmain}[1]{\renewcommand{\childdocmain}[1]{\endinput}}
  \renewcommand{\childdocof}[1]{}
  \renewcommand{\childdocby}[2][]{}
  \renewcommand{\childdocforward}[2][]{}
  \renewcommand{\childdocdisable}{}
}
%    \end{macrocode}

% \macro{\childdocmain}
% The macro |\childdocmain| is to be called at the top of the main file
% with nothing or the main filename (without extension) as argument.
% First, it breaks loops.
% If the argument is not empty and does not match |\childdocname|
% (which is set by the first inclusion of |childdoc.def|),
% |\ifchilddoc| is set to true, |\includeonly| is applied to the child file
% and |\jobname| is set to the main file
% (for proper handling of |.aux| files):
%    \begin{macrocode}
\newcommand{\childdocmain}[1]
{
  \childdocdisable\childdocmain{}
  \if?#1?\else
    \begingroup
      \def\childdoctmp{#1}
      \ifx\childdoctmp\childdocname
        \def\childdoctmp{}
      \else
        \def\childdoctmp
        {
          \childdoctrue
          \includeonly{\childdocname}
          \def\childdocjob{#1}
          \def\jobname{#1}
        }
      \fi
      \expandafter
    \endgroup
    \childdoctmp
  \fi
}
%    \end{macrocode}

% \macro{\childdocof}
% The command |\childdocof| redirects
% compilation to the main file |#1|.
%    \begin{macrocode}
\newcommand{\childdocof}[1]
{
  \childdocdisable
  \childdoctrue
  \includeonly{\childdocname}
  \def\jobname{#1}
  \def\childdocjob{#1}
  \input{#1}
}
%    \end{macrocode}

% \macro{\childdocby}
% The command |\childdocby| ....
%    \begin{macrocode}
\newcommand{\childdocby}[2][]
{
  \childdocdisable
  \childdoctrue
  \childdocmanualtrue
  \if?#1?\else
    \def\jobname{#2}
  \fi
  \def\childdocjob{#2}
  \input{#2}
  \endinput
}
%    \end{macrocode}

% \macro{\childdocforward}
% The command |\childdocforward| redirects
% compilation to the main file or
% (if the optional argument is given) a child file.
% Parameters are set as if the main file
% or a child file starting with |\childdocof| was compiled.
% Then compilation is handed over to the main file:
%    \begin{macrocode}
\newcommand{\childdocforward}[2][]
{
  \begingroup
    \if?#1?
      \def\childdoctmp
      {
        \def\childdocname{#2}
        \def\childdocjob{#2}
        \def\jobname{#2}
        \input{#2}
        \endinput
      }
    \else
      \def\childdoctmp
      {
        \childdocdisable
        \def\childdocname{#2}
        \childdoctrue
        \includeonly{#2}
        \def\childdocjob{#1}
        \def\jobname{#1}
        \input{#1}
        \endinput
      }
    \fi
    \expandafter
  \endgroup
  \childdoctmp
}
%    \end{macrocode}

% \macro{\childdocforwardprefix}
% The command |\childdocforwardprefix| redirects
% compilation to the main or a child file by means of a pattern.
% The prefix |#1| in the current filename is replaced by |#2|
% and the suffix of the current filename is kept
% (it is assumed that the filename does not contain the substring `|~~~|'
% which is used as a delimiter).
% Compilation is handed over to the new file by |\childdocforward|:
%    \begin{macrocode}
\newcommand{\childdocforwardprefix}[3][]
{
  \begingroup
    \def\childdocextract #2##1~~~{\def\childdoctmp{\childdocforward[#1]{#3##1}}}
    \expandafter\childdocextract\childdocname~~~
    \expandafter
  \endgroup
  \childdoctmp
}
%    \end{macrocode}

% \macro{\childdoc}
% The deprecated macro |\childdoc| is a legacy version of |\childdocmain|:
%    \begin{macrocode}
\newcommand{\childdoc}{\childdocmain}
%    \end{macrocode}

% \macro{\childdocredirect}
% The deprecated macro |\childdocredirect| is a legacy version
% of |\childdocforward| and |\childdocforwardprefix|:
%    \begin{macrocode}
\newcommand{\childdocredirect}[2][]
{
  \begingroup
    \if?#1?
      \def\childdoctmp{\childdocforward{#2}}
    \else
      \def\childdoctmp{\childdocforwardprefix{#1}{#2}}
    \fi
    \expandafter
  \endgroup
  \childdoctmp
}
%    \end{macrocode}

%\iffalse
%</package>
%\fi
%
\endinput
|\\
|\childdocforward{|\textit{main}|}|
\end{tabular}
\end{center}
%
Likewise, the following files |final|\textit{nn}|.tex|
compile the final version of the child document
|child|\textit{nn}|.tex|:
%
\begin{center}
\begin{tabular}{l}
|\def\version{final}|\\
|% \iffalse
%
% childdoc.dtx Copyright (C) 2017-2018 Niklas Beisert
%
% This work may be distributed and/or modified under the
% conditions of the LaTeX Project Public License, either version 1.3
% of this license or (at your option) any later version.
% The latest version of this license is in
%   http://www.latex-project.org/lppl.txt
% and version 1.3 or later is part of all distributions of LaTeX
% version 2005/12/01 or later.
%
% This work has the LPPL maintenance status `maintained'.
%
% The Current Maintainer of this work is Niklas Beisert.
%
% This work consists of the files childdoc.dtx and childdoc.ins
% and the derived files childdoc.def and cdocsamp.tex with
% cdocsch1.tex, cdocsch2.tex, cdocsdrf.tex, cdocsfn1.tex, cdocsfn2.tex.
%
%<package>\ifdefined\childdocmain\endinput\fi
%<package>\ProvidesFile{childdoc.def}[2018/12/30 v2.0 child document driver]
%<samplemain>\ProvidesFile{cdocsamp.tex}[2018/12/30 v2.0 sample for childdoc]
%<*driver>
%\ProvidesFile{childdoc.drv}[2018/12/30 v2.0 childdoc reference manual file]
\PassOptionsToClass{10pt,a4paper}{article}
\documentclass{ltxdoc}

\usepackage[margin=35mm]{geometry}
\usepackage{hyperref}
\usepackage{hyperxmp}
\usepackage[usenames]{color}

\hypersetup{colorlinks=true}
\hypersetup{pdfstartview=FitH}
\hypersetup{pdfpagemode=UseNone}
\hypersetup{pdfsource={}}
\hypersetup{pdflang={en-UK}}
\hypersetup{pdfcopyright={Copyright 2017-2018 Niklas Beisert.
  This work may be distributed and/or modified under the
  conditions of the LaTeX Project Public License, either version 1.3
  of this license or (at your option) any later version.}}
\hypersetup{pdflicenseurl={http://www.latex-project.org/lppl.txt}}
\hypersetup{pdfcontactaddress={ETH Zurich, ITP, HIT K,
  Wolfgang-Pauli-Strasse 27}}
\hypersetup{pdfcontactpostcode={8093}}
\hypersetup{pdfcontactcity={Zurich}}
\hypersetup{pdfcontactcountry={Switzerland}}
\hypersetup{pdfcontactemail={nbeisert@itp.phys.ethz.ch}}
\hypersetup{pdfcontacturl={http://people.phys.ethz.ch/\xmptilde nbeisert/}}

\newcommand{\secref}[1]{\hyperref[#1]{section \ref*{#1}}}

\parskip1ex
\parindent0pt
\let\olditemize\itemize
\def\itemize{\olditemize\parskip0pt}

\begin{document}

\title{The \textsf{childdoc} Package}
\hypersetup{pdftitle={The childdoc Package}}
\author{Niklas Beisert\\[2ex]
  Institut f\"ur Theoretische Physik\\
  Eidgen\"ossische Technische Hochschule Z\"urich\\
  Wolfgang-Pauli-Strasse 27, 8093 Z\"urich, Switzerland\\[1ex]
  \href{mailto:nbeisert@itp.phys.ethz.ch}
  {\texttt{nbeisert@itp.phys.ethz.ch}}}
\hypersetup{pdfauthor={Niklas Beisert}}
\hypersetup{pdfsubject={Manual for the LaTeX2e Package childdoc}}
\date{30 December 2018, \textsf{v2.0}}
\maketitle

\begin{abstract}\noindent
\textsf{childdoc} is a \LaTeXe{} package
that enables the direct compilation
of document sections included by |\include|
to individual files.
\end{abstract}

\begingroup
\parskip0ex
\tableofcontents
\endgroup

%%%%%%%%%%%%%%%%%%%%%%%%%%%%%%%%%%%%%%%%%%%%%%%%%%%%%%%%%%%%%%%%%%%%%%%%%%%%%%%%
%%%%%%%%%%%%%%%%%%%%%%%%%%%%%%%%%%%%%%%%%%%%%%%%%%%%%%%%%%%%%%%%%%%%%%%%%%%%%%%%
\section{Introduction}

\LaTeX{} provides a mechanism to structure a large document (such as a book)
into a main file and several child files (containing the chapters)
using the |\include| command.
This mechanism is beneficial for documents
which span hundreds of pages in order to
make the source file(s) more manageable.
Moreover, compilation can be restricted to
selected child files by means of the |\includeonly| command.
The latter feature can be used to reduce the compilation time while editing
(this was significantly more useful in the earlier days of \LaTeX{})
or to generate a smaller document which is easier to navigate.
Another application of |\includeonly| is to generate
documents consisting of selected parts of the complete document.

However, there are a few drawbacks of the plain |\include| mechanism:
\begin{itemize}
\item
The child files cannot be compiled on their own,
they can only be compiled via the main file.
A naive editing environment
(such as a text editor with an option
to have the current file processed by \LaTeX)
may require one to switch to the main file before compiling;
attempting to compile the child file produces errors.
\item
The main file must be modified (each time)
to adjust the |\includeonly| command
to the present needs. This easily leaves the main file in a messy state.
\item
The generated document will always carry the filename
of the main document. This is inconvenient if
several child files are to be compiled and
to be kept for distribution.
\end{itemize}

The present package provides a simple interface
to make child files individually compilable by \LaTeX{}.
Compiling a child file then has the same effect as compiling
the main file with an |\includeonly| command
to select the appropriate child.
Moreover the generated document will carry the name of the child
rather than the main file.
This resolves all three above issues.

This feature is meant to make the editing of books,
thesis documents and lecture notes somewhat more convenient.
However, the package can also be used efficiently for
composing a series of documents (such as exercise sheets)
which are typically distributed individually.
It then assists the author in generating the individual documents
(potentially in different versions)
as well as a document containing the collected series.
Another application is in developing style files
or other kinds of included material
where compilation of the style file could redirect
to a sample or test file.

%%%%%%%%%%%%%%%%%%%%%%%%%%%%%%%%%%%%%%%%%%%%%%%%%%%%%%%%%%%%%%%%%%%%%%%%%%%%%%%%
%%%%%%%%%%%%%%%%%%%%%%%%%%%%%%%%%%%%%%%%%%%%%%%%%%%%%%%%%%%%%%%%%%%%%%%%%%%%%%%%
\section{Usage}

First of all, the package \textsf{childdoc} is \emph{not} a standard
\LaTeXe{} |.sty| style file! Therefore it needs to be invoked in
a non-standard way.

%%%%%%%%%%%%%%%%%%%%%%%%%%%%%%%%%%%%%%%%%%%%%%%%%%%%%%%%%%%%%%%%%%%%%%%%%%%%%%%%
\subsection{Included Files}
\label{sec:include}

%%%%%%%%%%%%%%%%%%%%%%%%%%%%%%%%%%%%%%%%
\DescribeMacro{\childdocmain}
To use the package, add the commands
\begin{center}
\begin{tabular}{l}
|\input{childdoc.def}|\\
|\childdocmain{}|\\
\end{tabular}
\end{center}
at the very top of the main \LaTeX{} file,
in particular \emph{before} the |\documentclass| statement!
The argument of |\childdocmain| should be left empty
(but it must be present).

%%%%%%%%%%%%%%%%%%%%%%%%%%%%%%%%%%%%%%%%
\DescribeMacro{\childdocof}
Furthermore, add the commands
\begin{center}
\begin{tabular}{l}
|\input{childdoc.def}|\\
|\childdocof{|\textit{main}|}|\\
\end{tabular}
\end{center}
at the top of every child file \textit{child}
which is included by |\include{|\textit{child}|}|
from within the main file
(or at least for those files to be compiled individually).
The argument \textit{main} must be the filename of the main file.

There are a couple of
considerations in setting up the main and child documents:

%%%%%%%%%%%%%%%%%%%%%%%%%%%%%%%%%%%%%%%%
\paragraph{Restrictions.}

Please note the following restrictions:
\begin{itemize}
\item
|\childdocmain| must be called with one argument \textit{main}
to ensure compatibility with earlier version of the package.
It must either be empty (|\childdocmain{}|)
or precisely match the filename of the main file in which it is specified.
See \secref{sec:detection} for further information.
\item
The filename \textit{main} must be specified without the |.tex| extension.
\item
The filename \textit{main} is case sensitive
(even in case-insensitive file systems)
due to internal string comparison.
\item
The argument \textit{main} should be fully expanded, it cannot be a macro.
\item
Subdirectories and special characters should be avoided in filenames.
\item
The command |\childdocmain{|\textit{main}|}| must be followed by a whitespace.
It should not be followed immediately by another command
or by a comment mark `|%|'.
This is because the \TeX{} parser reads the token immediately following
the argument of |\childdocmain| and puts it
at the beginning of every child section;
however, a white\-space is ignored.
\end{itemize}

%%%%%%%%%%%%%%%%%%%%%%%%%%%%%%%%%%%%%%%%
\paragraph{Content of Main File.}

It is advisable to place all content in the child files included by |\include|.
Any output contained in the main file will appear in all child documents
unless suppressed manually;
it cannot be suppressed automatically by the |\includeonly| directive
and thus should normally be avoided.
A method to include some content in the main file
by means of conditional processing is described in \secref{sec:conditional}.

%%%%%%%%%%%%%%%%%%%%%%%%%%%%%%%%%%%%%%%%
\paragraph{Page Numbering.}

When only a part of the document is compiled,
the appropriate numbering of pages
(as well as other status parameters)
is determined from the |.aux| files.
The latter contain information from previous passes.
However this information needs to propagate through
all intermediate child documents.
Therefore the page numbering in child documents may well
be inconsistent until the complete document is compiled at least once.

A useful (if unconventional) way to always ensure a consistent
page numbering is to restart the numbering in each child document
and denote the pages by `\textit{child}|.|\textit{page}'
where \textit{child} represents the chapter/section number of the child file.
This can be achieved by the command
|\numberwithin{page}{|\textit{child}|}|
of the \textsf{amsmath} package
where \textit{child} can be |chapter| or |section|
depending on the chosen structuring.
Alternatively, one can modify the macro |\thepage| appropriately
and reset the counter |page| at the start of each child file.

%%%%%%%%%%%%%%%%%%%%%%%%%%%%%%%%%%%%%%%%%%%%%%%%%%%%%%%%%%%%%%%%%%%%%%%%%%%%%%%%
\subsection{Conditional Processing}
\label{sec:conditional}

The package provides a mechanism to compile different versions
of a document. To customise the versions further some conditional processing
can come in handy to distinguish which version is being compiled.
The package provides two macros to describe the compilation context:

%%%%%%%%%%%%%%%%%%%%%%%%%%%%%%%%%%%%%%%%
\DescribeMacro{\ifchilddoc}
The conditional |\ifchilddoc| distinguishes between the compilation of
child documents and the main document:
%
\begin{center}
|\ifchilddoc |\textit{child-code}| |[|\||else |\textit{main-code}]| \||fi|
\end{center}

%%%%%%%%%%%%%%%%%%%%%%%%%%%%%%%%%%%%%%%%
\DescribeMacro{\childdocname}
\DescribeMacro{\childdocjob}
The macro |\childdocname| contains the filename (without extension)
of the main or child file being processed.
Note that |\childdocjob| will always contain the name of the main file.

%%%%%%%%%%%%%%%%%%%%%%%%%%%%%%%%%%%%%%%%
\paragraph{Title Page.}

Conditional processing can be used to include a title or banner page
in the main document when proper precautions are taken.
Importantly, the code in the main file should ensure that the page counter
(as well as other status parameters which are stored in the |.aux| files)
takes the same value after the conditional processing.
Otherwise the page numbers may take divergent values
depending on which part is compiled.

For example, a title page could be declared by:
%
\begin{center}
\begin{tabular}{l}
|\ifchilddoc\||else|\\
|\addtocounter{page}{-1}|\\
\textit{code for title page}\\
|\newpage|\\
|\||fi|
\end{tabular}
\end{center}
%
A banner page for the child documents can be generated by:
%
\begin{center}
\begin{tabular}{l}
|\ifchilddoc|\\
|\addtocounter{page}{-1}|\\
\textit{code for banner page}\\
|\newpage|\\
|\||fi|
\end{tabular}
\end{center}
%
Here one could write a message such as:
\begin{center}
|This is the part \childdocname{} of \childdocjob{}.|
\end{center}

%%%%%%%%%%%%%%%%%%%%%%%%%%%%%%%%%%%%%%%%%%%%%%%%%%%%%%%%%%%%%%%%%%%%%%%%%%%%%%%%
\subsection{Flags}
\label{sec:flags}

The package makes it easy to generate different versions
of the main or child documents.
To this end compilation flags can be defined
and assigned different default values.
They will be particularly useful in conjunction
with the forwarding mechanism described in \secref{sec:forward}.

For example, it may be useful to have a flag |\version|
which can be set to |draft| or |final|.
The document source will contain some conditional code
depending on the value of |\version|.
Suppose further, the flag should default to |final| for the main file
and to |draft| for child files
which is a natural assignment for editing the document.
This is achieved by placing the following code
in the preamble of the main document
(below the |\childdocmain| directive):
%
\begin{center}
\begin{tabular}{l}
|\ifchilddoc|\\
|\providecommand{\version}{draft}|\\
|\||else|\\
|\providecommand{\version}{final}|\\
|\||fi|
\end{tabular}
\end{center}
%
The definition by |\providecommand| makes sure
that previous definitions are not overwritten.
Further statements |\providecommand{\version}{...}|
can thus be added before the above code to override it.

For the main file, one might add a line
(between |\childdocmain| and the above block)
%
\begin{center}
|%\ifchilddoc\||else\providecommand{\version}{draft}\||fi|
\end{center}
%
which can be uncommented to produce a draft version.
Likewise one can add a line to the very top of a child file
(above the |\childdocof{|\textit{main}|}| directive)
%
\begin{center}
|%\providecommand{\version}{final}|
\end{center}
%
which can be uncommented to produce the final version of this child document.

%%%%%%%%%%%%%%%%%%%%%%%%%%%%%%%%%%%%%%%%%%%%%%%%%%%%%%%%%%%%%%%%%%%%%%%%%%%%%%%%
\subsection{Forwarding}
\label{sec:forward}

Different versions of the main or child documents
using compilation flags as described in \secref{sec:flags}
can be (permanently) stored in different files
for convenient compilation, viewing and distribution.
To this end, the package defines a command
to pass on compilation to a different file:

%%%%%%%%%%%%%%%%%%%%%%%%%%%%%%%%%%%%%%%%
\DescribeMacro{\childdocforward}
The command |\childdocforward| redirects processing to
another source file:
%
\begin{center}
\begin{tabular}{l}
|\input{childdoc.def}|\\
|\childdocforward[|\textit{main}|]{|\textit{dest}|}|\\
\end{tabular}
\end{center}
%
The argument \textit{dest} is the destination file
(without extension).
It should be the main file or one of the child files.
Note that further \textsf{childdoc} directives
such as |\childdocof| and |\childdocforward|
in the indicated file will be processed in this form.
The optional argument \textit{main}
passes on directly to the main file \textit{main}
while pretending to compile the child \textit{dest}.
This form behaves as if \textit{dest}
issues |\childdocof{|\textit{main}|}| right away,
and no further \textsf{childdoc} directives will be processed.

%%%%%%%%%%%%%%%%%%%%%%%%%%%%%%%%%%%%%%%%
\DescribeMacro{\...prefix}
In the alternative form |\childdocforwardprefix|,
%
\begin{center}
\begin{tabular}{l}
|\input{childdoc.def}|\\
|\childdocforwardprefix[|\textit{main}|]{|\textit{prefix}|}{|\textit{dest}|}|
\end{tabular}
\end{center}
%
the destination file is determined by a pattern
depending on the current file:
To make this work, the current file must be called
`{\textit{prefix}\hspace{0.2em}\textit{suffix}}'
with \textit{prefix} matching precisely the argument.
Processing is then passed on to the file
`{\textit{dest}\hspace{0.2em}\textit{suffix}}'.
Surely, the same effect is achieved by
directly specifying the
argument `{\textit{dest}\hspace{0.2em}\textit{suffix}}'
in the first form.
However, that requires to set up a different file
for each child. With the alternative form of the command
all these files can have exactly the same content
which simplifies setting them up and maintaining them.

For example, the following file |draft.tex|
with a compilation flag |\version| as described in \secref{sec:flags}
compiles the main document as a draft:
%
\begin{center}
\begin{tabular}{l}
|\def\version{draft}|\\
|\input{childdoc.def}|\\
|\childdocforward{|\textit{main}|}|
\end{tabular}
\end{center}
%
Likewise, the following files |final|\textit{nn}|.tex|
compile the final version of the child document
|child|\textit{nn}|.tex|:
%
\begin{center}
\begin{tabular}{l}
|\def\version{final}|\\
|\input{childdoc.def}|\\
|\childdocforwardprefix{final}{child}|
\end{tabular}
\end{center}
%

Note that when several versions of a main file and/or of each child file
are to be generated, it may be convenient to set up a |Makefile| or
shell script to automatise the process.

%%%%%%%%%%%%%%%%%%%%%%%%%%%%%%%%%%%%%%%%%%%%%%%%%%%%%%%%%%%%%%%%%%%%%%%%%%%%%%%%
\subsection{Command Line Processing}
\label{sec:commandline}

The effect of redirection files can also be achieved by invoking
the \LaTeX{} compiler with a more elaborate command line.
Most conveniently this should be done as part
of a shell script or a |Makefile|.

When using \textsf{childdoc} in the main file, the following
command lines effectively perform a redirection
(note that depending on the shell being used,
backslashes may have to be doubled: `|\|' $\to$ `|\\|'):
%
\begin{center}
|... -jobname "|\textit{target}|" |\\|"|[\textit{flags}]%
|\input{childdoc.def}\childdocforward[|\textit{main}|]{|\textit{dest}|}"|
\end{center}
%
Here \textit{target} is the name of the output file,
\textit{main} is the name of the main file
and \textit{dest} is the name of the main or child file to be processed
(all filenames without extensions).
The optional argument \textit{main} can be omitted
if \textit{main} matches \textit{dest}.
Optionally, compilation \textit{flags} can be defined via |\def| commands.
This command line makes the \TeX{} engine believe
it is compiling the file \textit{target}
whose content is specified as the latter parameter.
The provided code then forwards the processing to
\textit{main} or \textit{dest} as described in \secref{sec:forward}.

%%%%%%%%%%%%%%%%%%%%%%%%%%%%%%%%%%%%%%%%%%%%%%%%%%%%%%%%%%%%%%%%%%%%%%%%%%%%%%%%
\subsection{Include by Input}
\label{sec:input}

Including child documents by |\include| has some restrictions by design.
Most notably, the content of a child document always occupies
its own set of pages; pages cannot be shared between child documents.
Usually, this behaviour makes perfect sense
because each child document contain an essential part of the document.
However, in some situations it may be desirable to compose
a document from a collection of parts
without having mandatory page breaks between then.
For this case, the package
provides a mechanism to include parts
by |\input| which can also be processed individually.
However, by construction this mechanism
requires manual handling of the content to be output.

%%%%%%%%%%%%%%%%%%%%%%%%%%%%%%%%%%%%%%%%
\DescribeMacro{\ifchilddocmanual}
The main file should be prepared as usual, see \secref{sec:include}.
However, the document body must make a distinction
between processing of an individual part and of the main document, e.g.:
%
\begin{center}
\begin{tabular}{l}
|\ifchilddocmanual|\\
|\input{\childdocname}|\\
|\||else|\\
\textit{document body with }|\input{|\textit{part}|}|\\
|\||fi|
\end{tabular}
\end{center}
%
The conditional |\ifchilddocmanual| is true whenever
a part to be included by |\input| is being compiled,
and the name of the part is stored in |\childdocname|.

%%%%%%%%%%%%%%%%%%%%%%%%%%%%%%%%%%%%%%%%
\DescribeMacro{\childdocby}
Each part to be included by |\input| should start with:
%
\begin{center}
\begin{tabular}{l}
|\input{childdoc.def}|\\
|\childdocby{|\textit{main}|}|\\
\end{tabular}
\end{center}
%
The directive |\childdocby| is similar to |\childdocof|
described in \secref{sec:include},
but the subsequent selection of content must be done manually.
To that end, both |\ifchilddoc| and |\ifchilddocmanual|
will be true upon processing of a part,
and the name of the part is stored in |\childdocname|.
Note that |\jobname| will be set to the filename of the current part
so that each part receives an individual |.aux| file
that does not interfere with the |.aux| file(s) of the main document.
This behaviour can be altered by the alternative form
|\childdocby[*]{|\textit{main}|}| (with a non-empty optional argument)
which uses the |.aux| file of the main document
by setting |\jobname| to \textit{main}.

%%%%%%%%%%%%%%%%%%%%%%%%%%%%%%%%%%%%%%%%%%%%%%%%%%%%%%%%%%%%%%%%%%%%%%%%%%%%%%%%
\subsection{Driver Development}
\label{sec:driver}

The \textsf{childdoc} mechanism can also be use for the development
of definition files such as \LaTeX{} styles or classes.
This case differs from the above setup with multiple parts
included by |\include| in that no |\includeonly| should be invoked.
This can be achieved by starting the include file
(before |\ProvidesPackage|) with:
%
\begin{center}
\begin{tabular}{l}
|\input{childdoc.def}|\\
|\childdocforward{|\textit{main}|}|\\
\end{tabular}
\end{center}
%
or alternatively with:
%
\begin{center}
\begin{tabular}{l}
|\input{childdoc.def}|\\
|\childdocby{|\textit{main}|}|\\
\end{tabular}
\end{center}
%
Both forms have slightly different effects as described above.
The main file is prepared as usual, see \secref{sec:include}.

%%%%%%%%%%%%%%%%%%%%%%%%%%%%%%%%%%%%%%%%%%%%%%%%%%%%%%%%%%%%%%%%%%%%%%%%%%%%%%%%
\subsection{Legacy Detection}
\label{sec:detection}

The directive |\childdocmain| in the main file can detect
whether the complete document or merely a child is to be compiled
even without using the directive |\childdocof|.
This method is deprecated because it is less robust
and there is no compelling reason to use it;
it is merely provided for backward compatibility
and it may be removed in future versions.

If the detection mechanism is to be used,
it is mandatory to correctly specify
the filename of the main file as the argument of |\childdocmain|:
%
\begin{center}
\begin{tabular}{l}
|\input{childdoc.def}|\\
|\childdocmain{|\textit{main}|}|\\
\end{tabular}
\end{center}
%
If |\jobname| does not match the argument \textit{main} of |\childdocmain|,
it is assumed that |\jobname| points to the child file to be compiled.
When using |\childdocmain| with the main file specified as argument,
it suffices to start a child file
with just |\input{|\textit{main}|}|
without loading of the package and using |\childdocof|.
If instead all processing is done
with the appropriate \textsf{childdoc} directives,
the argument of \textit{main} of |\childdocmain| can be empty.

An alternative version of the command line processing described
in \secref{sec:commandline} using the detection mechanism reads:
%
\begin{center}
|... -jobname "|\textit{target}|" "|[\textit{flags}]%
[|\def\jobname{|\textit{dest}|}|]|\input{|\textit{main}|}"|
\end{center}

%%%%%%%%%%%%%%%%%%%%%%%%%%%%%%%%%%%%%%%%%%%%%%%%%%%%%%%%%%%%%%%%%%%%%%%%%%%%%%%%
\subsection{Manual Code}
\label{sec:manual}

In case one cannot be certain whether the definitions file |childdoc.def|
is installed on the target \TeX{} distribution
and one prefers not to ship it,
it is conceivable to paste a few relevant commands into the sources.

To that end, drop all statements |\input{childdoc.def}|
and perform the replacements as outlined below.
Instead of |\childdocmain{|\textit{main}|}| add the following code
to the top of the main file:
%
\begin{center}
\begin{tabular}{l}
|\||ifdefined\childdocname\endinput\||fi\newif\ifchilddoc|\\
|\edef\childdocname{\scantokens\expandafter{\jobname\noexpand}}|\\
|\def\childdocmain{|\textit{main}|}\||ifx\childdocmain\childdocname\||else|\\
|\childdoctrue\includeonly{\childdocname}\let\jobname\childdocmain\||fi|\\
\end{tabular}
\end{center}
%
Instead of |\childdocof{|\textit{main}|}| just include the main file
at the top of each child file:
%
\begin{center}
|\input{|\textit{main}|}|
\end{center}
%
A simple redirection |\childdocforward{|\textit{dest}|}| is achieved by:
%
\begin{center}
|\def\jobname{|\textit{dest}|}\input{\jobname}|
\end{center}
%
The redirection with prefix
|\childdocforwardprefix[|\textit{prefix}|]{|\textit{dest}|}|
is accomplished by:
%
\begin{center}
\begin{tabular}{l}
|{\edef\jobname{\scantokens\expandafter{\jobname\noexpand}}|\\
|\def\redirectjob |\textit{prefix}|#1~~~{\gdef\jobname{|\textit{dest}|#1}}|\\
|\expandafter\redirectjob\jobname~~~}\input{\jobname}|
\end{tabular}
\end{center}

In an alternative approach,
child documents can be compiled by a specific command line
without additional code or specific definitions:
%
\begin{center}
|... -jobname "|\textit{target}|" "|[\textit{flags}]%
|\includeonly{|\textit{dest}|}\input{|\textit{main}|}"|
\end{center}
%

%%%%%%%%%%%%%%%%%%%%%%%%%%%%%%%%%%%%%%%%%%%%%%%%%%%%%%%%%%%%%%%%%%%%%%%%%%%%%%%%
%%%%%%%%%%%%%%%%%%%%%%%%%%%%%%%%%%%%%%%%%%%%%%%%%%%%%%%%%%%%%%%%%%%%%%%%%%%%%%%%
\section{Information}

%%%%%%%%%%%%%%%%%%%%%%%%%%%%%%%%%%%%%%%%%%%%%%%%%%%%%%%%%%%%%%%%%%%%%%%%%%%%%%%%
\subsection{Copyright}

Copyright \copyright{} 2017--2018 Niklas Beisert

This work may be distributed and/or modified under the
conditions of the \LaTeX{} Project Public License, either version 1.3
of this license or (at your option) any later version.
The latest version of this license is in
  \url{http://www.latex-project.org/lppl.txt}
and version 1.3 or later is part of all distributions of \LaTeX{}
version 2005/12/01 or later.

This work has the LPPL maintenance status `maintained'.

The Current Maintainer of this work is Niklas Beisert.

This work consists of the files |README.txt|, |childdoc.ins| and |childdoc.dtx|
as well as the derived files |childdoc.def|, |cdocsamp.tex|
with |cdocsch1.tex|, |cdocsch2.tex|, |cdocspt3.tex|, |cdocspt4.tex|,
|cdocsdrf.tex|, |cdocsfn1.tex|, |cdocsfn2.tex|
as well as |childdoc.pdf|.

%%%%%%%%%%%%%%%%%%%%%%%%%%%%%%%%%%%%%%%%%%%%%%%%%%%%%%%%%%%%%%%%%%%%%%%%%%%%%%%%
\subsection{Files and Installation}

The package consists of the files:
%
\begin{center}
\begin{tabular}{ll}
    |README.txt|   & readme file \\
    |childdoc.ins| & installation file \\
    |childdoc.dtx| & source file \\
    |childdoc.def| & definition file \\
    |cdocsamp.tex| & sample main file \\
    |cdocsch1.tex| & sample include file \\
    |cdocsch2.tex| & sample include file \\
    |cdocspt3.tex| & sample part file \\
    |cdocspt4.tex| & sample part file \\
    |cdocsdrf.tex| & sample redirection file \\
    |cdocsfn1.tex| & sample redirection file \\
    |cdocsfn2.tex| & sample redirection file \\
    |childdoc.pdf| & manual
\end{tabular}
\end{center}
%
The distribution consists of the files
|README.txt|, |childdoc.ins| and |childdoc.dtx|.
%
\begin{itemize}
\item
Run (pdf)\LaTeX{} on |childdoc.dtx|
to compile the manual |childdoc.pdf| (this file).
\item
Run \LaTeX{} on |childdoc.ins| to create the definitions file |childdoc.def|
and the sample |cdocsamp.tex| with include files
|cdocsch1.tex|, |cdocsch2.tex|, |cdocspt3.tex|, |cdocspt4.tex|,
|cdocsdrf.tex|, |cdocsfn1.tex|, |cdocsfn2.tex|.
Then copy the file |childdoc.def| to an appropriate directory of your \LaTeX{}
distribution, e.g.\ \textit{texmf-root}|/tex/latex/childdoc|.
\end{itemize}

%%%%%%%%%%%%%%%%%%%%%%%%%%%%%%%%%%%%%%%%%%%%%%%%%%%%%%%%%%%%%%%%%%%%%%%%%%%%%%%%
\subsection{Related CTAN Packages}

There are several other packages which offer a similar functionality:
%
\begin{itemize}
\item
The packages
\href{http://ctan.org/pkg/docmute}{\textsf{docmute}},
\href{http://ctan.org/pkg/includex}{\textsf{includex}} and
\href{http://ctan.org/pkg/standalone}{\textsf{standalone}}
provide commands to include only the document body of
a child file thus allowing both files to be compiled individually.
\item
The packages \href{http://ctan.org/pkg/subdocs}{\textsf{subdocs}}
and \href{http://ctan.org/pkg/subfiles}{\textsf{subfiles}}
provide structures in which the main and child documents can be
encapsulated and allowing them to be compiled individually.
The inclusion mechanism is different from the conventional |\include|.
\item
The package \href{http://ctan.org/pkg/combine}{\textsf{combine}}
is an elaborate solution to combine several documents into one.
\end{itemize}
%
See also the CTAN topic \href{http://ctan.org/topic/subdocs}{\textsf{subdocs}}
for further related packages.
The present package differs from the above solutions in that
a document structure constructed with the conventional |\include| mechanism
just needs two extra commands at the top of every file
such that all constituent files can be compiled individually.

%%%%%%%%%%%%%%%%%%%%%%%%%%%%%%%%%%%%%%%%%%%%%%%%%%%%%%%%%%%%%%%%%%%%%%%%%%%%%%%%
%\subsection{Feature Suggestions}
%
%The following is a list of features which may be useful for future
%versions of this package:
%%
%\begin{itemize}
%\item
%\ldots
%\end{itemize}

%%%%%%%%%%%%%%%%%%%%%%%%%%%%%%%%%%%%%%%%%%%%%%%%%%%%%%%%%%%%%%%%%%%%%%%%%%%%%%%%
\subsection{Revision History}

%%%%%%%%%%%%%%%%%%%%%%%%%%%%%%%%%%%%%%%%
\paragraph{v2.0:} 2018/12/30

\begin{itemize}
\item
immediate forward processing
\item
added |\childdocby| mechanism
\item
manual restructured
\end{itemize}

%%%%%%%%%%%%%%%%%%%%%%%%%%%%%%%%%%%%%%%%
\paragraph{v1.6:} 2018/01/17

\begin{itemize}
\item
application for development of include files
\item
corrections to manual
\end{itemize}

%%%%%%%%%%%%%%%%%%%%%%%%%%%%%%%%%%%%%%%%
\paragraph{v1.5:} 2017/05/21

\begin{itemize}
\item
more complete structuring introduced
\item
|\childdocof| introduced
\item
|\childdoc| renamed to |\childdocmain|
\item
|\childredirect| renamed to |\childdocforward| and |\childdocforwardprefix|
and functionality expanded
\end{itemize}

%%%%%%%%%%%%%%%%%%%%%%%%%%%%%%%%%%%%%%%%
\paragraph{v1.0:} 2017/04/27

\begin{itemize}
\item
manual and install package
\item
first version published on CTAN
\end{itemize}

%%%%%%%%%%%%%%%%%%%%%%%%%%%%%%%%%%%%%%%%
\paragraph{v0.6:} 2017/04/26

\begin{itemize}
\item
redirection mechanism added
\end{itemize}

%%%%%%%%%%%%%%%%%%%%%%%%%%%%%%%%%%%%%%%%
\paragraph{v0.5:} 2017/04/26

\begin{itemize}
\item
functionality in definition file
\end{itemize}


%%%%%%%%%%%%%%%%%%%%%%%%%%%%%%%%%%%%%%%%%%%%%%%%%%%%%%%%%%%%%%%%%%%%%%%%%%%%%%%%
%%%%%%%%%%%%%%%%%%%%%%%%%%%%%%%%%%%%%%%%%%%%%%%%%%%%%%%%%%%%%%%%%%%%%%%%%%%%%%%%
%%%%%%%%%%%%%%%%%%%%%%%%%%%%%%%%%%%%%%%%%%%%%%%%%%%%%%%%%%%%%%%%%%%%%%%%%%%%%%%%
\appendix

\settowidth\MacroIndent{\rmfamily\scriptsize 000\ }

 \DocInput{childdoc.dtx}

\end{document}
%</driver>
% \fi
%
% %%%%%%%%%%%%%%%%%%%%%%%%%%%%%%%%%%%%%%%%%%%%%%%%%%%%%%%%%%%%%%%%%%%%%%%%%%%%%%
% %%%%%%%%%%%%%%%%%%%%%%%%%%%%%%%%%%%%%%%%%%%%%%%%%%%%%%%%%%%%%%%%%%%%%%%%%%%%%%
% \section{Sample}
%\iffalse
%<*samplemain>
%\fi
%
% The following presents a sample document
% with two chapters, two parts, a title page,
% a compile flag as well as three forwarding files to set the flag.
% It consists of eight |.tex| files:
% \begin{center}
% \begin{tabular}{ll}
% |cdocsamp.tex|&main file\\
% |cdocsch1.tex|&include file for chapter 1\\
% |cdocsch2.tex|&include file for chapter 2\\
% |cdocspt3.tex|&include file for part 3\\
% |cdocspt4.tex|&include file for part 4\\
% |cdocsdrf.tex|&forwarding file for main file in draft mode\\
% |cdocsfi1.tex|&forwarding file for final version of chapter 1\\
% |cdocsfi2.tex|&forwarding file for final version of chapter 2\\
% \end{tabular}
% \end{center}
% Each of the eight files can be compiled directly by the \LaTeX{} compiler.
%
% %%%%%%%%%%%%%%%%%%%%%%%%%%%%%%%%%%%%%%
% \paragraph{Main File.}
%
% The main file is called |cdocsamp.tex|.
%
% Load the \textsf{childdoc} definitions and
% declare the filename for the main document:
%    \begin{macrocode}
\input{childdoc.def}
\childdocmain{}
%    \end{macrocode}

% Optional override for |\version| flag:
%    \begin{macrocode}
%%\ifchilddoc\else\providecommand{\version}{draft}\fi
%    \end{macrocode}

% Define the default values for the |\version| flag
% (|final| for the main file and |draft| for childs):
%    \begin{macrocode}
\ifchilddoc
\providecommand{\version}{draft}
\else
\providecommand{\version}{final}
\fi
%    \end{macrocode}

% Load the standard document class:
%    \begin{macrocode}
\documentclass[12pt]{article}
%    \end{macrocode}

% Start the document body:
%    \begin{macrocode}
\begin{document}
%    \end{macrocode}

% Declare a title page.
% Print title, part of document being processed and version flag:
%    \begin{macrocode}
\addtocounter{page}{-1}
\begin{center}
{\LARGE\bfseries{}childdoc example\par}
\vspace{1cm}
\ifchilddoc
\ifchilddocmanual part\else chapter\fi:
`\childdocname' of `\childdocjob'\par
\else
main document: `\childdocjob'\par
\fi
version: \version\par
\end{center}
\newpage
%    \end{macrocode}

% Manually include selected file,
% otherwise process as usual:
%    \begin{macrocode}
\ifchilddocmanual
\section*{part `\childdocname'}
\input{\childdocname}
\else
%    \end{macrocode}

% Include the two chapters:
%    \begin{macrocode}
\include{cdocsch1}
\include{cdocsch2}
%    \end{macrocode}

% Include the two parts unless only chapters should be displayed:
%    \begin{macrocode}
\ifchilddoc\else
\section{part three}
\input{cdocspt3}
\section{part four}
\input{cdocspt4}
\fi
%    \end{macrocode}

% Process as usual until here:
%    \begin{macrocode}
\fi
%    \end{macrocode}

% End of document body:
%    \begin{macrocode}
\end{document}
%    \end{macrocode}
%\iffalse
%</samplemain>
%\fi
%
% %%%%%%%%%%%%%%%%%%%%%%%%%%%%%%%%%%%%%%
% \paragraph{Chapter Include Files.}
%
% The include files are called |cdocsch1.tex| and |cdocsch2.tex|.
%
%\iffalse
%<*samplechap1|samplechap2>
%\fi

% Optional override for |\version| flag:
%    \begin{macrocode}
%%\providecommand{\version}{final}
%    \end{macrocode}

% Include the main document:
%    \begin{macrocode}
\input{childdoc.def}
\childdocof{cdocsamp}
%    \end{macrocode}

%\iffalse
%</samplechap1|samplechap2>
%\fi
%
%\iffalse
%<*samplechap1>
%\fi
% Some text for chapter 1:
%    \begin{macrocode}
\section{one}
some text in chapter one
%    \end{macrocode}

%\iffalse
%</samplechap1>
%\fi
% Some text for chapter 2:
%\iffalse
%<*samplechap2>
%\fi
%    \begin{macrocode}
\section{two}
more text in chapter two
%    \end{macrocode}

%\iffalse
%</samplechap2>
%\fi
%
% %%%%%%%%%%%%%%%%%%%%%%%%%%%%%%%%%%%%%%
% \paragraph{Part Include Files.}
%
% The include files are called |cdocspt3.tex| and |cdocspt4.tex|.
%
%\iffalse
%<*samplepart3|samplepart4>
%\fi

% Optional override for |\version| flag:
%    \begin{macrocode}
%%\providecommand{\version}{final}
%    \end{macrocode}

% Include the main document:
%    \begin{macrocode}
\input{childdoc.def}
\childdocby{cdocsamp}
%    \end{macrocode}

%\iffalse
%</samplepart3|samplepart4>
%\fi
%
%\iffalse
%<*samplepart3>
%\fi
% Some text for part 3:
%    \begin{macrocode}
some text in part three
%    \end{macrocode}

%\iffalse
%</samplepart3>
%\fi
% Some text for part 4:
%\iffalse
%<*samplepart4>
%\fi
%    \begin{macrocode}
more text in part four
%    \end{macrocode}

%\iffalse
%</samplepart4>
%\fi
%
% %%%%%%%%%%%%%%%%%%%%%%%%%%%%%%%%%%%%%%
% \paragraph{Forwarding for a Complete Draft.}
%
% The following forwarding file |cdocsdrf.tex|
% compiles the main document in draft mode:
%\iffalse
%<*sampledraft>
%\fi
%    \begin{macrocode}
\def\version{draft}
\input{childdoc.def}
\childdocforward{cdocsamp}
%    \end{macrocode}

%\iffalse
%</sampledraft>
%\fi
%
% %%%%%%%%%%%%%%%%%%%%%%%%%%%%%%%%%%%%%%
% \paragraph{Forwarding for Final Version of the Chapters.}
%
% The following forwarding files |cdocsfn1.tex| and |cdocsfn2.tex|
% (with identical content)
% compile the final versions of the child documents
% |cdocsch1.tex| and |cdocsch2.tex|, respectively:
%\iffalse
%<*samplefinal>
%\fi
%    \begin{macrocode}
\def\version{final}
\input{childdoc.def}
\childdocforwardprefix[cdocsamp]{cdocsfn}{cdocsch}
%    \end{macrocode}

%\iffalse
%</samplefinal>
%\fi
%
% %%%%%%%%%%%%%%%%%%%%%%%%%%%%%%%%%%%%%%
% \paragraph{Command Line Processing.}
%
% The following three command lines generate the output files
% |cdocscld|, |cdocscl1| and |cdocscl2|
% which should be identical to
% |cdocsdrf|, |cdocsch1| and |cdocsfn2|, respectively:
% \begin{center}
% \begin{tabular}{l}
% |latex -jobname cdocscld \|\\
% |  "\def\version{draft}\input{childdoc.def}\childdocforward{cdocsamp}"|\\
% |latex -jobname cdocscl1 \|\\
% |  "\input{childdoc.def}\childdocforward[cdocsamp]{cdocsch1}"|\\
% |latex -jobname cdocscl2 \|\\
% |  "\def\version{final}\input{childdoc.def}\childdocforward{cdocsch2}"|
% \end{tabular}
% \end{center}
% Note that the trailing backslash on each first line
% merely continues the input to the second line
% (for convenient cut ant paste).
% Furthermore, the command |latex| can be replaced by any
% of its alternative versions such as |pdflatex|.
%
% %%%%%%%%%%%%%%%%%%%%%%%%%%%%%%%%%%%%%%%%%%%%%%%%%%%%%%%%%%%%%%%%%%%%%%%%%%%%%%
% %%%%%%%%%%%%%%%%%%%%%%%%%%%%%%%%%%%%%%%%%%%%%%%%%%%%%%%%%%%%%%%%%%%%%%%%%%%%%%
% \section{Implementation}
%\iffalse
%<*package>
%\fi
%
% This section describes the definitions file |childdoc.def|.

% The definitions cannot be loaded using |\usepackage| or |\RequirePackage|
% which has a mechanism to prevent loading a style file more than once.
% When loading the definitions by means of |\input|
% multiple instances have to be prevented manually:
%\iffalse
%This code needs to be before the `\ProvidesFile' directive
%which is defined at the beginning of this file.
%Therefore it is also placed there and commented out here.
%</package>
%<*discard>
%\fi
%    \begin{macrocode}
\ifdefined\childdocmain\endinput\fi
%    \end{macrocode}
%\iffalse
%</discard>
%<*package>
%\fi
%
% \macro{\ifchilddoc}
% \macro{\ifchilddocmanual}
% The conditional |\ifchilddoc| tells whether a
% child (true) or main (false) document is being compiled.
% The conditional |\ifchilddocmanual| tells whether
% the |\includeonly| mechanism is used (false) or
% the selection of child files must be performed manually (true).
% The definitions initialise to false:
%    \begin{macrocode}
\newif\ifchilddoc
\newif\ifchilddocmanual
%    \end{macrocode}

% \macro{\childdocname}
% \macro{\childdocjob}
% The macro |\childdocname| stores the name of the main document
% to be compiled. The macro |\childdocjob| stores the name of
% the document on which the \LaTeX{} compiler was originally invoked.
% The content of |\jobname| cannot be compared
% to filenames specified in the source due to different catcodes.
% The following code rescans |\jobname|, stores the result
% in |\childdocname| and saves a copy in |\childdocjob|:
%    \begin{macrocode}
\edef\childdocname{\scantokens\expandafter{\jobname\noexpand}}
\let\childdocjob\childdocname
%    \end{macrocode}

% \macro{\childdocdisable}
% The macro |\childdocdisable| prevents the main file
% from being processed more than once.
% At this stage, the main document command |\childdocmain|
% is assumed to be called once again where it should do nothing.
% Any subsequent call to it should prevent
% a secondary processing of the main document
% It overwrites the forwarding commands
% |\childdocof| and |\childdocforward|
% with empty macros to prevent further inclusions of the main document:
%    \begin{macrocode}
\newcommand{\childdocdisable}
{
  \renewcommand{\childdocmain}[1]{\renewcommand{\childdocmain}[1]{\endinput}}
  \renewcommand{\childdocof}[1]{}
  \renewcommand{\childdocby}[2][]{}
  \renewcommand{\childdocforward}[2][]{}
  \renewcommand{\childdocdisable}{}
}
%    \end{macrocode}

% \macro{\childdocmain}
% The macro |\childdocmain| is to be called at the top of the main file
% with nothing or the main filename (without extension) as argument.
% First, it breaks loops.
% If the argument is not empty and does not match |\childdocname|
% (which is set by the first inclusion of |childdoc.def|),
% |\ifchilddoc| is set to true, |\includeonly| is applied to the child file
% and |\jobname| is set to the main file
% (for proper handling of |.aux| files):
%    \begin{macrocode}
\newcommand{\childdocmain}[1]
{
  \childdocdisable\childdocmain{}
  \if?#1?\else
    \begingroup
      \def\childdoctmp{#1}
      \ifx\childdoctmp\childdocname
        \def\childdoctmp{}
      \else
        \def\childdoctmp
        {
          \childdoctrue
          \includeonly{\childdocname}
          \def\childdocjob{#1}
          \def\jobname{#1}
        }
      \fi
      \expandafter
    \endgroup
    \childdoctmp
  \fi
}
%    \end{macrocode}

% \macro{\childdocof}
% The command |\childdocof| redirects
% compilation to the main file |#1|.
%    \begin{macrocode}
\newcommand{\childdocof}[1]
{
  \childdocdisable
  \childdoctrue
  \includeonly{\childdocname}
  \def\jobname{#1}
  \def\childdocjob{#1}
  \input{#1}
}
%    \end{macrocode}

% \macro{\childdocby}
% The command |\childdocby| ....
%    \begin{macrocode}
\newcommand{\childdocby}[2][]
{
  \childdocdisable
  \childdoctrue
  \childdocmanualtrue
  \if?#1?\else
    \def\jobname{#2}
  \fi
  \def\childdocjob{#2}
  \input{#2}
  \endinput
}
%    \end{macrocode}

% \macro{\childdocforward}
% The command |\childdocforward| redirects
% compilation to the main file or
% (if the optional argument is given) a child file.
% Parameters are set as if the main file
% or a child file starting with |\childdocof| was compiled.
% Then compilation is handed over to the main file:
%    \begin{macrocode}
\newcommand{\childdocforward}[2][]
{
  \begingroup
    \if?#1?
      \def\childdoctmp
      {
        \def\childdocname{#2}
        \def\childdocjob{#2}
        \def\jobname{#2}
        \input{#2}
        \endinput
      }
    \else
      \def\childdoctmp
      {
        \childdocdisable
        \def\childdocname{#2}
        \childdoctrue
        \includeonly{#2}
        \def\childdocjob{#1}
        \def\jobname{#1}
        \input{#1}
        \endinput
      }
    \fi
    \expandafter
  \endgroup
  \childdoctmp
}
%    \end{macrocode}

% \macro{\childdocforwardprefix}
% The command |\childdocforwardprefix| redirects
% compilation to the main or a child file by means of a pattern.
% The prefix |#1| in the current filename is replaced by |#2|
% and the suffix of the current filename is kept
% (it is assumed that the filename does not contain the substring `|~~~|'
% which is used as a delimiter).
% Compilation is handed over to the new file by |\childdocforward|:
%    \begin{macrocode}
\newcommand{\childdocforwardprefix}[3][]
{
  \begingroup
    \def\childdocextract #2##1~~~{\def\childdoctmp{\childdocforward[#1]{#3##1}}}
    \expandafter\childdocextract\childdocname~~~
    \expandafter
  \endgroup
  \childdoctmp
}
%    \end{macrocode}

% \macro{\childdoc}
% The deprecated macro |\childdoc| is a legacy version of |\childdocmain|:
%    \begin{macrocode}
\newcommand{\childdoc}{\childdocmain}
%    \end{macrocode}

% \macro{\childdocredirect}
% The deprecated macro |\childdocredirect| is a legacy version
% of |\childdocforward| and |\childdocforwardprefix|:
%    \begin{macrocode}
\newcommand{\childdocredirect}[2][]
{
  \begingroup
    \if?#1?
      \def\childdoctmp{\childdocforward{#2}}
    \else
      \def\childdoctmp{\childdocforwardprefix{#1}{#2}}
    \fi
    \expandafter
  \endgroup
  \childdoctmp
}
%    \end{macrocode}

%\iffalse
%</package>
%\fi
%
\endinput
|\\
|\childdocforwardprefix{final}{child}|
\end{tabular}
\end{center}
%

Note that when several versions of a main file and/or of each child file
are to be generated, it may be convenient to set up a |Makefile| or
shell script to automatise the process.

%%%%%%%%%%%%%%%%%%%%%%%%%%%%%%%%%%%%%%%%%%%%%%%%%%%%%%%%%%%%%%%%%%%%%%%%%%%%%%%%
\subsection{Command Line Processing}
\label{sec:commandline}

The effect of redirection files can also be achieved by invoking
the \LaTeX{} compiler with a more elaborate command line.
Most conveniently this should be done as part
of a shell script or a |Makefile|.

When using \textsf{childdoc} in the main file, the following
command lines effectively perform a redirection
(note that depending on the shell being used,
backslashes may have to be doubled: `|\|' $\to$ `|\\|'):
%
\begin{center}
|... -jobname "|\textit{target}|" |\\|"|[\textit{flags}]%
|% \iffalse
%
% childdoc.dtx Copyright (C) 2017-2018 Niklas Beisert
%
% This work may be distributed and/or modified under the
% conditions of the LaTeX Project Public License, either version 1.3
% of this license or (at your option) any later version.
% The latest version of this license is in
%   http://www.latex-project.org/lppl.txt
% and version 1.3 or later is part of all distributions of LaTeX
% version 2005/12/01 or later.
%
% This work has the LPPL maintenance status `maintained'.
%
% The Current Maintainer of this work is Niklas Beisert.
%
% This work consists of the files childdoc.dtx and childdoc.ins
% and the derived files childdoc.def and cdocsamp.tex with
% cdocsch1.tex, cdocsch2.tex, cdocsdrf.tex, cdocsfn1.tex, cdocsfn2.tex.
%
%<package>\ifdefined\childdocmain\endinput\fi
%<package>\ProvidesFile{childdoc.def}[2018/12/30 v2.0 child document driver]
%<samplemain>\ProvidesFile{cdocsamp.tex}[2018/12/30 v2.0 sample for childdoc]
%<*driver>
%\ProvidesFile{childdoc.drv}[2018/12/30 v2.0 childdoc reference manual file]
\PassOptionsToClass{10pt,a4paper}{article}
\documentclass{ltxdoc}

\usepackage[margin=35mm]{geometry}
\usepackage{hyperref}
\usepackage{hyperxmp}
\usepackage[usenames]{color}

\hypersetup{colorlinks=true}
\hypersetup{pdfstartview=FitH}
\hypersetup{pdfpagemode=UseNone}
\hypersetup{pdfsource={}}
\hypersetup{pdflang={en-UK}}
\hypersetup{pdfcopyright={Copyright 2017-2018 Niklas Beisert.
  This work may be distributed and/or modified under the
  conditions of the LaTeX Project Public License, either version 1.3
  of this license or (at your option) any later version.}}
\hypersetup{pdflicenseurl={http://www.latex-project.org/lppl.txt}}
\hypersetup{pdfcontactaddress={ETH Zurich, ITP, HIT K,
  Wolfgang-Pauli-Strasse 27}}
\hypersetup{pdfcontactpostcode={8093}}
\hypersetup{pdfcontactcity={Zurich}}
\hypersetup{pdfcontactcountry={Switzerland}}
\hypersetup{pdfcontactemail={nbeisert@itp.phys.ethz.ch}}
\hypersetup{pdfcontacturl={http://people.phys.ethz.ch/\xmptilde nbeisert/}}

\newcommand{\secref}[1]{\hyperref[#1]{section \ref*{#1}}}

\parskip1ex
\parindent0pt
\let\olditemize\itemize
\def\itemize{\olditemize\parskip0pt}

\begin{document}

\title{The \textsf{childdoc} Package}
\hypersetup{pdftitle={The childdoc Package}}
\author{Niklas Beisert\\[2ex]
  Institut f\"ur Theoretische Physik\\
  Eidgen\"ossische Technische Hochschule Z\"urich\\
  Wolfgang-Pauli-Strasse 27, 8093 Z\"urich, Switzerland\\[1ex]
  \href{mailto:nbeisert@itp.phys.ethz.ch}
  {\texttt{nbeisert@itp.phys.ethz.ch}}}
\hypersetup{pdfauthor={Niklas Beisert}}
\hypersetup{pdfsubject={Manual for the LaTeX2e Package childdoc}}
\date{30 December 2018, \textsf{v2.0}}
\maketitle

\begin{abstract}\noindent
\textsf{childdoc} is a \LaTeXe{} package
that enables the direct compilation
of document sections included by |\include|
to individual files.
\end{abstract}

\begingroup
\parskip0ex
\tableofcontents
\endgroup

%%%%%%%%%%%%%%%%%%%%%%%%%%%%%%%%%%%%%%%%%%%%%%%%%%%%%%%%%%%%%%%%%%%%%%%%%%%%%%%%
%%%%%%%%%%%%%%%%%%%%%%%%%%%%%%%%%%%%%%%%%%%%%%%%%%%%%%%%%%%%%%%%%%%%%%%%%%%%%%%%
\section{Introduction}

\LaTeX{} provides a mechanism to structure a large document (such as a book)
into a main file and several child files (containing the chapters)
using the |\include| command.
This mechanism is beneficial for documents
which span hundreds of pages in order to
make the source file(s) more manageable.
Moreover, compilation can be restricted to
selected child files by means of the |\includeonly| command.
The latter feature can be used to reduce the compilation time while editing
(this was significantly more useful in the earlier days of \LaTeX{})
or to generate a smaller document which is easier to navigate.
Another application of |\includeonly| is to generate
documents consisting of selected parts of the complete document.

However, there are a few drawbacks of the plain |\include| mechanism:
\begin{itemize}
\item
The child files cannot be compiled on their own,
they can only be compiled via the main file.
A naive editing environment
(such as a text editor with an option
to have the current file processed by \LaTeX)
may require one to switch to the main file before compiling;
attempting to compile the child file produces errors.
\item
The main file must be modified (each time)
to adjust the |\includeonly| command
to the present needs. This easily leaves the main file in a messy state.
\item
The generated document will always carry the filename
of the main document. This is inconvenient if
several child files are to be compiled and
to be kept for distribution.
\end{itemize}

The present package provides a simple interface
to make child files individually compilable by \LaTeX{}.
Compiling a child file then has the same effect as compiling
the main file with an |\includeonly| command
to select the appropriate child.
Moreover the generated document will carry the name of the child
rather than the main file.
This resolves all three above issues.

This feature is meant to make the editing of books,
thesis documents and lecture notes somewhat more convenient.
However, the package can also be used efficiently for
composing a series of documents (such as exercise sheets)
which are typically distributed individually.
It then assists the author in generating the individual documents
(potentially in different versions)
as well as a document containing the collected series.
Another application is in developing style files
or other kinds of included material
where compilation of the style file could redirect
to a sample or test file.

%%%%%%%%%%%%%%%%%%%%%%%%%%%%%%%%%%%%%%%%%%%%%%%%%%%%%%%%%%%%%%%%%%%%%%%%%%%%%%%%
%%%%%%%%%%%%%%%%%%%%%%%%%%%%%%%%%%%%%%%%%%%%%%%%%%%%%%%%%%%%%%%%%%%%%%%%%%%%%%%%
\section{Usage}

First of all, the package \textsf{childdoc} is \emph{not} a standard
\LaTeXe{} |.sty| style file! Therefore it needs to be invoked in
a non-standard way.

%%%%%%%%%%%%%%%%%%%%%%%%%%%%%%%%%%%%%%%%%%%%%%%%%%%%%%%%%%%%%%%%%%%%%%%%%%%%%%%%
\subsection{Included Files}
\label{sec:include}

%%%%%%%%%%%%%%%%%%%%%%%%%%%%%%%%%%%%%%%%
\DescribeMacro{\childdocmain}
To use the package, add the commands
\begin{center}
\begin{tabular}{l}
|\input{childdoc.def}|\\
|\childdocmain{}|\\
\end{tabular}
\end{center}
at the very top of the main \LaTeX{} file,
in particular \emph{before} the |\documentclass| statement!
The argument of |\childdocmain| should be left empty
(but it must be present).

%%%%%%%%%%%%%%%%%%%%%%%%%%%%%%%%%%%%%%%%
\DescribeMacro{\childdocof}
Furthermore, add the commands
\begin{center}
\begin{tabular}{l}
|\input{childdoc.def}|\\
|\childdocof{|\textit{main}|}|\\
\end{tabular}
\end{center}
at the top of every child file \textit{child}
which is included by |\include{|\textit{child}|}|
from within the main file
(or at least for those files to be compiled individually).
The argument \textit{main} must be the filename of the main file.

There are a couple of
considerations in setting up the main and child documents:

%%%%%%%%%%%%%%%%%%%%%%%%%%%%%%%%%%%%%%%%
\paragraph{Restrictions.}

Please note the following restrictions:
\begin{itemize}
\item
|\childdocmain| must be called with one argument \textit{main}
to ensure compatibility with earlier version of the package.
It must either be empty (|\childdocmain{}|)
or precisely match the filename of the main file in which it is specified.
See \secref{sec:detection} for further information.
\item
The filename \textit{main} must be specified without the |.tex| extension.
\item
The filename \textit{main} is case sensitive
(even in case-insensitive file systems)
due to internal string comparison.
\item
The argument \textit{main} should be fully expanded, it cannot be a macro.
\item
Subdirectories and special characters should be avoided in filenames.
\item
The command |\childdocmain{|\textit{main}|}| must be followed by a whitespace.
It should not be followed immediately by another command
or by a comment mark `|%|'.
This is because the \TeX{} parser reads the token immediately following
the argument of |\childdocmain| and puts it
at the beginning of every child section;
however, a white\-space is ignored.
\end{itemize}

%%%%%%%%%%%%%%%%%%%%%%%%%%%%%%%%%%%%%%%%
\paragraph{Content of Main File.}

It is advisable to place all content in the child files included by |\include|.
Any output contained in the main file will appear in all child documents
unless suppressed manually;
it cannot be suppressed automatically by the |\includeonly| directive
and thus should normally be avoided.
A method to include some content in the main file
by means of conditional processing is described in \secref{sec:conditional}.

%%%%%%%%%%%%%%%%%%%%%%%%%%%%%%%%%%%%%%%%
\paragraph{Page Numbering.}

When only a part of the document is compiled,
the appropriate numbering of pages
(as well as other status parameters)
is determined from the |.aux| files.
The latter contain information from previous passes.
However this information needs to propagate through
all intermediate child documents.
Therefore the page numbering in child documents may well
be inconsistent until the complete document is compiled at least once.

A useful (if unconventional) way to always ensure a consistent
page numbering is to restart the numbering in each child document
and denote the pages by `\textit{child}|.|\textit{page}'
where \textit{child} represents the chapter/section number of the child file.
This can be achieved by the command
|\numberwithin{page}{|\textit{child}|}|
of the \textsf{amsmath} package
where \textit{child} can be |chapter| or |section|
depending on the chosen structuring.
Alternatively, one can modify the macro |\thepage| appropriately
and reset the counter |page| at the start of each child file.

%%%%%%%%%%%%%%%%%%%%%%%%%%%%%%%%%%%%%%%%%%%%%%%%%%%%%%%%%%%%%%%%%%%%%%%%%%%%%%%%
\subsection{Conditional Processing}
\label{sec:conditional}

The package provides a mechanism to compile different versions
of a document. To customise the versions further some conditional processing
can come in handy to distinguish which version is being compiled.
The package provides two macros to describe the compilation context:

%%%%%%%%%%%%%%%%%%%%%%%%%%%%%%%%%%%%%%%%
\DescribeMacro{\ifchilddoc}
The conditional |\ifchilddoc| distinguishes between the compilation of
child documents and the main document:
%
\begin{center}
|\ifchilddoc |\textit{child-code}| |[|\||else |\textit{main-code}]| \||fi|
\end{center}

%%%%%%%%%%%%%%%%%%%%%%%%%%%%%%%%%%%%%%%%
\DescribeMacro{\childdocname}
\DescribeMacro{\childdocjob}
The macro |\childdocname| contains the filename (without extension)
of the main or child file being processed.
Note that |\childdocjob| will always contain the name of the main file.

%%%%%%%%%%%%%%%%%%%%%%%%%%%%%%%%%%%%%%%%
\paragraph{Title Page.}

Conditional processing can be used to include a title or banner page
in the main document when proper precautions are taken.
Importantly, the code in the main file should ensure that the page counter
(as well as other status parameters which are stored in the |.aux| files)
takes the same value after the conditional processing.
Otherwise the page numbers may take divergent values
depending on which part is compiled.

For example, a title page could be declared by:
%
\begin{center}
\begin{tabular}{l}
|\ifchilddoc\||else|\\
|\addtocounter{page}{-1}|\\
\textit{code for title page}\\
|\newpage|\\
|\||fi|
\end{tabular}
\end{center}
%
A banner page for the child documents can be generated by:
%
\begin{center}
\begin{tabular}{l}
|\ifchilddoc|\\
|\addtocounter{page}{-1}|\\
\textit{code for banner page}\\
|\newpage|\\
|\||fi|
\end{tabular}
\end{center}
%
Here one could write a message such as:
\begin{center}
|This is the part \childdocname{} of \childdocjob{}.|
\end{center}

%%%%%%%%%%%%%%%%%%%%%%%%%%%%%%%%%%%%%%%%%%%%%%%%%%%%%%%%%%%%%%%%%%%%%%%%%%%%%%%%
\subsection{Flags}
\label{sec:flags}

The package makes it easy to generate different versions
of the main or child documents.
To this end compilation flags can be defined
and assigned different default values.
They will be particularly useful in conjunction
with the forwarding mechanism described in \secref{sec:forward}.

For example, it may be useful to have a flag |\version|
which can be set to |draft| or |final|.
The document source will contain some conditional code
depending on the value of |\version|.
Suppose further, the flag should default to |final| for the main file
and to |draft| for child files
which is a natural assignment for editing the document.
This is achieved by placing the following code
in the preamble of the main document
(below the |\childdocmain| directive):
%
\begin{center}
\begin{tabular}{l}
|\ifchilddoc|\\
|\providecommand{\version}{draft}|\\
|\||else|\\
|\providecommand{\version}{final}|\\
|\||fi|
\end{tabular}
\end{center}
%
The definition by |\providecommand| makes sure
that previous definitions are not overwritten.
Further statements |\providecommand{\version}{...}|
can thus be added before the above code to override it.

For the main file, one might add a line
(between |\childdocmain| and the above block)
%
\begin{center}
|%\ifchilddoc\||else\providecommand{\version}{draft}\||fi|
\end{center}
%
which can be uncommented to produce a draft version.
Likewise one can add a line to the very top of a child file
(above the |\childdocof{|\textit{main}|}| directive)
%
\begin{center}
|%\providecommand{\version}{final}|
\end{center}
%
which can be uncommented to produce the final version of this child document.

%%%%%%%%%%%%%%%%%%%%%%%%%%%%%%%%%%%%%%%%%%%%%%%%%%%%%%%%%%%%%%%%%%%%%%%%%%%%%%%%
\subsection{Forwarding}
\label{sec:forward}

Different versions of the main or child documents
using compilation flags as described in \secref{sec:flags}
can be (permanently) stored in different files
for convenient compilation, viewing and distribution.
To this end, the package defines a command
to pass on compilation to a different file:

%%%%%%%%%%%%%%%%%%%%%%%%%%%%%%%%%%%%%%%%
\DescribeMacro{\childdocforward}
The command |\childdocforward| redirects processing to
another source file:
%
\begin{center}
\begin{tabular}{l}
|\input{childdoc.def}|\\
|\childdocforward[|\textit{main}|]{|\textit{dest}|}|\\
\end{tabular}
\end{center}
%
The argument \textit{dest} is the destination file
(without extension).
It should be the main file or one of the child files.
Note that further \textsf{childdoc} directives
such as |\childdocof| and |\childdocforward|
in the indicated file will be processed in this form.
The optional argument \textit{main}
passes on directly to the main file \textit{main}
while pretending to compile the child \textit{dest}.
This form behaves as if \textit{dest}
issues |\childdocof{|\textit{main}|}| right away,
and no further \textsf{childdoc} directives will be processed.

%%%%%%%%%%%%%%%%%%%%%%%%%%%%%%%%%%%%%%%%
\DescribeMacro{\...prefix}
In the alternative form |\childdocforwardprefix|,
%
\begin{center}
\begin{tabular}{l}
|\input{childdoc.def}|\\
|\childdocforwardprefix[|\textit{main}|]{|\textit{prefix}|}{|\textit{dest}|}|
\end{tabular}
\end{center}
%
the destination file is determined by a pattern
depending on the current file:
To make this work, the current file must be called
`{\textit{prefix}\hspace{0.2em}\textit{suffix}}'
with \textit{prefix} matching precisely the argument.
Processing is then passed on to the file
`{\textit{dest}\hspace{0.2em}\textit{suffix}}'.
Surely, the same effect is achieved by
directly specifying the
argument `{\textit{dest}\hspace{0.2em}\textit{suffix}}'
in the first form.
However, that requires to set up a different file
for each child. With the alternative form of the command
all these files can have exactly the same content
which simplifies setting them up and maintaining them.

For example, the following file |draft.tex|
with a compilation flag |\version| as described in \secref{sec:flags}
compiles the main document as a draft:
%
\begin{center}
\begin{tabular}{l}
|\def\version{draft}|\\
|\input{childdoc.def}|\\
|\childdocforward{|\textit{main}|}|
\end{tabular}
\end{center}
%
Likewise, the following files |final|\textit{nn}|.tex|
compile the final version of the child document
|child|\textit{nn}|.tex|:
%
\begin{center}
\begin{tabular}{l}
|\def\version{final}|\\
|\input{childdoc.def}|\\
|\childdocforwardprefix{final}{child}|
\end{tabular}
\end{center}
%

Note that when several versions of a main file and/or of each child file
are to be generated, it may be convenient to set up a |Makefile| or
shell script to automatise the process.

%%%%%%%%%%%%%%%%%%%%%%%%%%%%%%%%%%%%%%%%%%%%%%%%%%%%%%%%%%%%%%%%%%%%%%%%%%%%%%%%
\subsection{Command Line Processing}
\label{sec:commandline}

The effect of redirection files can also be achieved by invoking
the \LaTeX{} compiler with a more elaborate command line.
Most conveniently this should be done as part
of a shell script or a |Makefile|.

When using \textsf{childdoc} in the main file, the following
command lines effectively perform a redirection
(note that depending on the shell being used,
backslashes may have to be doubled: `|\|' $\to$ `|\\|'):
%
\begin{center}
|... -jobname "|\textit{target}|" |\\|"|[\textit{flags}]%
|\input{childdoc.def}\childdocforward[|\textit{main}|]{|\textit{dest}|}"|
\end{center}
%
Here \textit{target} is the name of the output file,
\textit{main} is the name of the main file
and \textit{dest} is the name of the main or child file to be processed
(all filenames without extensions).
The optional argument \textit{main} can be omitted
if \textit{main} matches \textit{dest}.
Optionally, compilation \textit{flags} can be defined via |\def| commands.
This command line makes the \TeX{} engine believe
it is compiling the file \textit{target}
whose content is specified as the latter parameter.
The provided code then forwards the processing to
\textit{main} or \textit{dest} as described in \secref{sec:forward}.

%%%%%%%%%%%%%%%%%%%%%%%%%%%%%%%%%%%%%%%%%%%%%%%%%%%%%%%%%%%%%%%%%%%%%%%%%%%%%%%%
\subsection{Include by Input}
\label{sec:input}

Including child documents by |\include| has some restrictions by design.
Most notably, the content of a child document always occupies
its own set of pages; pages cannot be shared between child documents.
Usually, this behaviour makes perfect sense
because each child document contain an essential part of the document.
However, in some situations it may be desirable to compose
a document from a collection of parts
without having mandatory page breaks between then.
For this case, the package
provides a mechanism to include parts
by |\input| which can also be processed individually.
However, by construction this mechanism
requires manual handling of the content to be output.

%%%%%%%%%%%%%%%%%%%%%%%%%%%%%%%%%%%%%%%%
\DescribeMacro{\ifchilddocmanual}
The main file should be prepared as usual, see \secref{sec:include}.
However, the document body must make a distinction
between processing of an individual part and of the main document, e.g.:
%
\begin{center}
\begin{tabular}{l}
|\ifchilddocmanual|\\
|\input{\childdocname}|\\
|\||else|\\
\textit{document body with }|\input{|\textit{part}|}|\\
|\||fi|
\end{tabular}
\end{center}
%
The conditional |\ifchilddocmanual| is true whenever
a part to be included by |\input| is being compiled,
and the name of the part is stored in |\childdocname|.

%%%%%%%%%%%%%%%%%%%%%%%%%%%%%%%%%%%%%%%%
\DescribeMacro{\childdocby}
Each part to be included by |\input| should start with:
%
\begin{center}
\begin{tabular}{l}
|\input{childdoc.def}|\\
|\childdocby{|\textit{main}|}|\\
\end{tabular}
\end{center}
%
The directive |\childdocby| is similar to |\childdocof|
described in \secref{sec:include},
but the subsequent selection of content must be done manually.
To that end, both |\ifchilddoc| and |\ifchilddocmanual|
will be true upon processing of a part,
and the name of the part is stored in |\childdocname|.
Note that |\jobname| will be set to the filename of the current part
so that each part receives an individual |.aux| file
that does not interfere with the |.aux| file(s) of the main document.
This behaviour can be altered by the alternative form
|\childdocby[*]{|\textit{main}|}| (with a non-empty optional argument)
which uses the |.aux| file of the main document
by setting |\jobname| to \textit{main}.

%%%%%%%%%%%%%%%%%%%%%%%%%%%%%%%%%%%%%%%%%%%%%%%%%%%%%%%%%%%%%%%%%%%%%%%%%%%%%%%%
\subsection{Driver Development}
\label{sec:driver}

The \textsf{childdoc} mechanism can also be use for the development
of definition files such as \LaTeX{} styles or classes.
This case differs from the above setup with multiple parts
included by |\include| in that no |\includeonly| should be invoked.
This can be achieved by starting the include file
(before |\ProvidesPackage|) with:
%
\begin{center}
\begin{tabular}{l}
|\input{childdoc.def}|\\
|\childdocforward{|\textit{main}|}|\\
\end{tabular}
\end{center}
%
or alternatively with:
%
\begin{center}
\begin{tabular}{l}
|\input{childdoc.def}|\\
|\childdocby{|\textit{main}|}|\\
\end{tabular}
\end{center}
%
Both forms have slightly different effects as described above.
The main file is prepared as usual, see \secref{sec:include}.

%%%%%%%%%%%%%%%%%%%%%%%%%%%%%%%%%%%%%%%%%%%%%%%%%%%%%%%%%%%%%%%%%%%%%%%%%%%%%%%%
\subsection{Legacy Detection}
\label{sec:detection}

The directive |\childdocmain| in the main file can detect
whether the complete document or merely a child is to be compiled
even without using the directive |\childdocof|.
This method is deprecated because it is less robust
and there is no compelling reason to use it;
it is merely provided for backward compatibility
and it may be removed in future versions.

If the detection mechanism is to be used,
it is mandatory to correctly specify
the filename of the main file as the argument of |\childdocmain|:
%
\begin{center}
\begin{tabular}{l}
|\input{childdoc.def}|\\
|\childdocmain{|\textit{main}|}|\\
\end{tabular}
\end{center}
%
If |\jobname| does not match the argument \textit{main} of |\childdocmain|,
it is assumed that |\jobname| points to the child file to be compiled.
When using |\childdocmain| with the main file specified as argument,
it suffices to start a child file
with just |\input{|\textit{main}|}|
without loading of the package and using |\childdocof|.
If instead all processing is done
with the appropriate \textsf{childdoc} directives,
the argument of \textit{main} of |\childdocmain| can be empty.

An alternative version of the command line processing described
in \secref{sec:commandline} using the detection mechanism reads:
%
\begin{center}
|... -jobname "|\textit{target}|" "|[\textit{flags}]%
[|\def\jobname{|\textit{dest}|}|]|\input{|\textit{main}|}"|
\end{center}

%%%%%%%%%%%%%%%%%%%%%%%%%%%%%%%%%%%%%%%%%%%%%%%%%%%%%%%%%%%%%%%%%%%%%%%%%%%%%%%%
\subsection{Manual Code}
\label{sec:manual}

In case one cannot be certain whether the definitions file |childdoc.def|
is installed on the target \TeX{} distribution
and one prefers not to ship it,
it is conceivable to paste a few relevant commands into the sources.

To that end, drop all statements |\input{childdoc.def}|
and perform the replacements as outlined below.
Instead of |\childdocmain{|\textit{main}|}| add the following code
to the top of the main file:
%
\begin{center}
\begin{tabular}{l}
|\||ifdefined\childdocname\endinput\||fi\newif\ifchilddoc|\\
|\edef\childdocname{\scantokens\expandafter{\jobname\noexpand}}|\\
|\def\childdocmain{|\textit{main}|}\||ifx\childdocmain\childdocname\||else|\\
|\childdoctrue\includeonly{\childdocname}\let\jobname\childdocmain\||fi|\\
\end{tabular}
\end{center}
%
Instead of |\childdocof{|\textit{main}|}| just include the main file
at the top of each child file:
%
\begin{center}
|\input{|\textit{main}|}|
\end{center}
%
A simple redirection |\childdocforward{|\textit{dest}|}| is achieved by:
%
\begin{center}
|\def\jobname{|\textit{dest}|}\input{\jobname}|
\end{center}
%
The redirection with prefix
|\childdocforwardprefix[|\textit{prefix}|]{|\textit{dest}|}|
is accomplished by:
%
\begin{center}
\begin{tabular}{l}
|{\edef\jobname{\scantokens\expandafter{\jobname\noexpand}}|\\
|\def\redirectjob |\textit{prefix}|#1~~~{\gdef\jobname{|\textit{dest}|#1}}|\\
|\expandafter\redirectjob\jobname~~~}\input{\jobname}|
\end{tabular}
\end{center}

In an alternative approach,
child documents can be compiled by a specific command line
without additional code or specific definitions:
%
\begin{center}
|... -jobname "|\textit{target}|" "|[\textit{flags}]%
|\includeonly{|\textit{dest}|}\input{|\textit{main}|}"|
\end{center}
%

%%%%%%%%%%%%%%%%%%%%%%%%%%%%%%%%%%%%%%%%%%%%%%%%%%%%%%%%%%%%%%%%%%%%%%%%%%%%%%%%
%%%%%%%%%%%%%%%%%%%%%%%%%%%%%%%%%%%%%%%%%%%%%%%%%%%%%%%%%%%%%%%%%%%%%%%%%%%%%%%%
\section{Information}

%%%%%%%%%%%%%%%%%%%%%%%%%%%%%%%%%%%%%%%%%%%%%%%%%%%%%%%%%%%%%%%%%%%%%%%%%%%%%%%%
\subsection{Copyright}

Copyright \copyright{} 2017--2018 Niklas Beisert

This work may be distributed and/or modified under the
conditions of the \LaTeX{} Project Public License, either version 1.3
of this license or (at your option) any later version.
The latest version of this license is in
  \url{http://www.latex-project.org/lppl.txt}
and version 1.3 or later is part of all distributions of \LaTeX{}
version 2005/12/01 or later.

This work has the LPPL maintenance status `maintained'.

The Current Maintainer of this work is Niklas Beisert.

This work consists of the files |README.txt|, |childdoc.ins| and |childdoc.dtx|
as well as the derived files |childdoc.def|, |cdocsamp.tex|
with |cdocsch1.tex|, |cdocsch2.tex|, |cdocspt3.tex|, |cdocspt4.tex|,
|cdocsdrf.tex|, |cdocsfn1.tex|, |cdocsfn2.tex|
as well as |childdoc.pdf|.

%%%%%%%%%%%%%%%%%%%%%%%%%%%%%%%%%%%%%%%%%%%%%%%%%%%%%%%%%%%%%%%%%%%%%%%%%%%%%%%%
\subsection{Files and Installation}

The package consists of the files:
%
\begin{center}
\begin{tabular}{ll}
    |README.txt|   & readme file \\
    |childdoc.ins| & installation file \\
    |childdoc.dtx| & source file \\
    |childdoc.def| & definition file \\
    |cdocsamp.tex| & sample main file \\
    |cdocsch1.tex| & sample include file \\
    |cdocsch2.tex| & sample include file \\
    |cdocspt3.tex| & sample part file \\
    |cdocspt4.tex| & sample part file \\
    |cdocsdrf.tex| & sample redirection file \\
    |cdocsfn1.tex| & sample redirection file \\
    |cdocsfn2.tex| & sample redirection file \\
    |childdoc.pdf| & manual
\end{tabular}
\end{center}
%
The distribution consists of the files
|README.txt|, |childdoc.ins| and |childdoc.dtx|.
%
\begin{itemize}
\item
Run (pdf)\LaTeX{} on |childdoc.dtx|
to compile the manual |childdoc.pdf| (this file).
\item
Run \LaTeX{} on |childdoc.ins| to create the definitions file |childdoc.def|
and the sample |cdocsamp.tex| with include files
|cdocsch1.tex|, |cdocsch2.tex|, |cdocspt3.tex|, |cdocspt4.tex|,
|cdocsdrf.tex|, |cdocsfn1.tex|, |cdocsfn2.tex|.
Then copy the file |childdoc.def| to an appropriate directory of your \LaTeX{}
distribution, e.g.\ \textit{texmf-root}|/tex/latex/childdoc|.
\end{itemize}

%%%%%%%%%%%%%%%%%%%%%%%%%%%%%%%%%%%%%%%%%%%%%%%%%%%%%%%%%%%%%%%%%%%%%%%%%%%%%%%%
\subsection{Related CTAN Packages}

There are several other packages which offer a similar functionality:
%
\begin{itemize}
\item
The packages
\href{http://ctan.org/pkg/docmute}{\textsf{docmute}},
\href{http://ctan.org/pkg/includex}{\textsf{includex}} and
\href{http://ctan.org/pkg/standalone}{\textsf{standalone}}
provide commands to include only the document body of
a child file thus allowing both files to be compiled individually.
\item
The packages \href{http://ctan.org/pkg/subdocs}{\textsf{subdocs}}
and \href{http://ctan.org/pkg/subfiles}{\textsf{subfiles}}
provide structures in which the main and child documents can be
encapsulated and allowing them to be compiled individually.
The inclusion mechanism is different from the conventional |\include|.
\item
The package \href{http://ctan.org/pkg/combine}{\textsf{combine}}
is an elaborate solution to combine several documents into one.
\end{itemize}
%
See also the CTAN topic \href{http://ctan.org/topic/subdocs}{\textsf{subdocs}}
for further related packages.
The present package differs from the above solutions in that
a document structure constructed with the conventional |\include| mechanism
just needs two extra commands at the top of every file
such that all constituent files can be compiled individually.

%%%%%%%%%%%%%%%%%%%%%%%%%%%%%%%%%%%%%%%%%%%%%%%%%%%%%%%%%%%%%%%%%%%%%%%%%%%%%%%%
%\subsection{Feature Suggestions}
%
%The following is a list of features which may be useful for future
%versions of this package:
%%
%\begin{itemize}
%\item
%\ldots
%\end{itemize}

%%%%%%%%%%%%%%%%%%%%%%%%%%%%%%%%%%%%%%%%%%%%%%%%%%%%%%%%%%%%%%%%%%%%%%%%%%%%%%%%
\subsection{Revision History}

%%%%%%%%%%%%%%%%%%%%%%%%%%%%%%%%%%%%%%%%
\paragraph{v2.0:} 2018/12/30

\begin{itemize}
\item
immediate forward processing
\item
added |\childdocby| mechanism
\item
manual restructured
\end{itemize}

%%%%%%%%%%%%%%%%%%%%%%%%%%%%%%%%%%%%%%%%
\paragraph{v1.6:} 2018/01/17

\begin{itemize}
\item
application for development of include files
\item
corrections to manual
\end{itemize}

%%%%%%%%%%%%%%%%%%%%%%%%%%%%%%%%%%%%%%%%
\paragraph{v1.5:} 2017/05/21

\begin{itemize}
\item
more complete structuring introduced
\item
|\childdocof| introduced
\item
|\childdoc| renamed to |\childdocmain|
\item
|\childredirect| renamed to |\childdocforward| and |\childdocforwardprefix|
and functionality expanded
\end{itemize}

%%%%%%%%%%%%%%%%%%%%%%%%%%%%%%%%%%%%%%%%
\paragraph{v1.0:} 2017/04/27

\begin{itemize}
\item
manual and install package
\item
first version published on CTAN
\end{itemize}

%%%%%%%%%%%%%%%%%%%%%%%%%%%%%%%%%%%%%%%%
\paragraph{v0.6:} 2017/04/26

\begin{itemize}
\item
redirection mechanism added
\end{itemize}

%%%%%%%%%%%%%%%%%%%%%%%%%%%%%%%%%%%%%%%%
\paragraph{v0.5:} 2017/04/26

\begin{itemize}
\item
functionality in definition file
\end{itemize}


%%%%%%%%%%%%%%%%%%%%%%%%%%%%%%%%%%%%%%%%%%%%%%%%%%%%%%%%%%%%%%%%%%%%%%%%%%%%%%%%
%%%%%%%%%%%%%%%%%%%%%%%%%%%%%%%%%%%%%%%%%%%%%%%%%%%%%%%%%%%%%%%%%%%%%%%%%%%%%%%%
%%%%%%%%%%%%%%%%%%%%%%%%%%%%%%%%%%%%%%%%%%%%%%%%%%%%%%%%%%%%%%%%%%%%%%%%%%%%%%%%
\appendix

\settowidth\MacroIndent{\rmfamily\scriptsize 000\ }

 \DocInput{childdoc.dtx}

\end{document}
%</driver>
% \fi
%
% %%%%%%%%%%%%%%%%%%%%%%%%%%%%%%%%%%%%%%%%%%%%%%%%%%%%%%%%%%%%%%%%%%%%%%%%%%%%%%
% %%%%%%%%%%%%%%%%%%%%%%%%%%%%%%%%%%%%%%%%%%%%%%%%%%%%%%%%%%%%%%%%%%%%%%%%%%%%%%
% \section{Sample}
%\iffalse
%<*samplemain>
%\fi
%
% The following presents a sample document
% with two chapters, two parts, a title page,
% a compile flag as well as three forwarding files to set the flag.
% It consists of eight |.tex| files:
% \begin{center}
% \begin{tabular}{ll}
% |cdocsamp.tex|&main file\\
% |cdocsch1.tex|&include file for chapter 1\\
% |cdocsch2.tex|&include file for chapter 2\\
% |cdocspt3.tex|&include file for part 3\\
% |cdocspt4.tex|&include file for part 4\\
% |cdocsdrf.tex|&forwarding file for main file in draft mode\\
% |cdocsfi1.tex|&forwarding file for final version of chapter 1\\
% |cdocsfi2.tex|&forwarding file for final version of chapter 2\\
% \end{tabular}
% \end{center}
% Each of the eight files can be compiled directly by the \LaTeX{} compiler.
%
% %%%%%%%%%%%%%%%%%%%%%%%%%%%%%%%%%%%%%%
% \paragraph{Main File.}
%
% The main file is called |cdocsamp.tex|.
%
% Load the \textsf{childdoc} definitions and
% declare the filename for the main document:
%    \begin{macrocode}
\input{childdoc.def}
\childdocmain{}
%    \end{macrocode}

% Optional override for |\version| flag:
%    \begin{macrocode}
%%\ifchilddoc\else\providecommand{\version}{draft}\fi
%    \end{macrocode}

% Define the default values for the |\version| flag
% (|final| for the main file and |draft| for childs):
%    \begin{macrocode}
\ifchilddoc
\providecommand{\version}{draft}
\else
\providecommand{\version}{final}
\fi
%    \end{macrocode}

% Load the standard document class:
%    \begin{macrocode}
\documentclass[12pt]{article}
%    \end{macrocode}

% Start the document body:
%    \begin{macrocode}
\begin{document}
%    \end{macrocode}

% Declare a title page.
% Print title, part of document being processed and version flag:
%    \begin{macrocode}
\addtocounter{page}{-1}
\begin{center}
{\LARGE\bfseries{}childdoc example\par}
\vspace{1cm}
\ifchilddoc
\ifchilddocmanual part\else chapter\fi:
`\childdocname' of `\childdocjob'\par
\else
main document: `\childdocjob'\par
\fi
version: \version\par
\end{center}
\newpage
%    \end{macrocode}

% Manually include selected file,
% otherwise process as usual:
%    \begin{macrocode}
\ifchilddocmanual
\section*{part `\childdocname'}
\input{\childdocname}
\else
%    \end{macrocode}

% Include the two chapters:
%    \begin{macrocode}
\include{cdocsch1}
\include{cdocsch2}
%    \end{macrocode}

% Include the two parts unless only chapters should be displayed:
%    \begin{macrocode}
\ifchilddoc\else
\section{part three}
\input{cdocspt3}
\section{part four}
\input{cdocspt4}
\fi
%    \end{macrocode}

% Process as usual until here:
%    \begin{macrocode}
\fi
%    \end{macrocode}

% End of document body:
%    \begin{macrocode}
\end{document}
%    \end{macrocode}
%\iffalse
%</samplemain>
%\fi
%
% %%%%%%%%%%%%%%%%%%%%%%%%%%%%%%%%%%%%%%
% \paragraph{Chapter Include Files.}
%
% The include files are called |cdocsch1.tex| and |cdocsch2.tex|.
%
%\iffalse
%<*samplechap1|samplechap2>
%\fi

% Optional override for |\version| flag:
%    \begin{macrocode}
%%\providecommand{\version}{final}
%    \end{macrocode}

% Include the main document:
%    \begin{macrocode}
\input{childdoc.def}
\childdocof{cdocsamp}
%    \end{macrocode}

%\iffalse
%</samplechap1|samplechap2>
%\fi
%
%\iffalse
%<*samplechap1>
%\fi
% Some text for chapter 1:
%    \begin{macrocode}
\section{one}
some text in chapter one
%    \end{macrocode}

%\iffalse
%</samplechap1>
%\fi
% Some text for chapter 2:
%\iffalse
%<*samplechap2>
%\fi
%    \begin{macrocode}
\section{two}
more text in chapter two
%    \end{macrocode}

%\iffalse
%</samplechap2>
%\fi
%
% %%%%%%%%%%%%%%%%%%%%%%%%%%%%%%%%%%%%%%
% \paragraph{Part Include Files.}
%
% The include files are called |cdocspt3.tex| and |cdocspt4.tex|.
%
%\iffalse
%<*samplepart3|samplepart4>
%\fi

% Optional override for |\version| flag:
%    \begin{macrocode}
%%\providecommand{\version}{final}
%    \end{macrocode}

% Include the main document:
%    \begin{macrocode}
\input{childdoc.def}
\childdocby{cdocsamp}
%    \end{macrocode}

%\iffalse
%</samplepart3|samplepart4>
%\fi
%
%\iffalse
%<*samplepart3>
%\fi
% Some text for part 3:
%    \begin{macrocode}
some text in part three
%    \end{macrocode}

%\iffalse
%</samplepart3>
%\fi
% Some text for part 4:
%\iffalse
%<*samplepart4>
%\fi
%    \begin{macrocode}
more text in part four
%    \end{macrocode}

%\iffalse
%</samplepart4>
%\fi
%
% %%%%%%%%%%%%%%%%%%%%%%%%%%%%%%%%%%%%%%
% \paragraph{Forwarding for a Complete Draft.}
%
% The following forwarding file |cdocsdrf.tex|
% compiles the main document in draft mode:
%\iffalse
%<*sampledraft>
%\fi
%    \begin{macrocode}
\def\version{draft}
\input{childdoc.def}
\childdocforward{cdocsamp}
%    \end{macrocode}

%\iffalse
%</sampledraft>
%\fi
%
% %%%%%%%%%%%%%%%%%%%%%%%%%%%%%%%%%%%%%%
% \paragraph{Forwarding for Final Version of the Chapters.}
%
% The following forwarding files |cdocsfn1.tex| and |cdocsfn2.tex|
% (with identical content)
% compile the final versions of the child documents
% |cdocsch1.tex| and |cdocsch2.tex|, respectively:
%\iffalse
%<*samplefinal>
%\fi
%    \begin{macrocode}
\def\version{final}
\input{childdoc.def}
\childdocforwardprefix[cdocsamp]{cdocsfn}{cdocsch}
%    \end{macrocode}

%\iffalse
%</samplefinal>
%\fi
%
% %%%%%%%%%%%%%%%%%%%%%%%%%%%%%%%%%%%%%%
% \paragraph{Command Line Processing.}
%
% The following three command lines generate the output files
% |cdocscld|, |cdocscl1| and |cdocscl2|
% which should be identical to
% |cdocsdrf|, |cdocsch1| and |cdocsfn2|, respectively:
% \begin{center}
% \begin{tabular}{l}
% |latex -jobname cdocscld \|\\
% |  "\def\version{draft}\input{childdoc.def}\childdocforward{cdocsamp}"|\\
% |latex -jobname cdocscl1 \|\\
% |  "\input{childdoc.def}\childdocforward[cdocsamp]{cdocsch1}"|\\
% |latex -jobname cdocscl2 \|\\
% |  "\def\version{final}\input{childdoc.def}\childdocforward{cdocsch2}"|
% \end{tabular}
% \end{center}
% Note that the trailing backslash on each first line
% merely continues the input to the second line
% (for convenient cut ant paste).
% Furthermore, the command |latex| can be replaced by any
% of its alternative versions such as |pdflatex|.
%
% %%%%%%%%%%%%%%%%%%%%%%%%%%%%%%%%%%%%%%%%%%%%%%%%%%%%%%%%%%%%%%%%%%%%%%%%%%%%%%
% %%%%%%%%%%%%%%%%%%%%%%%%%%%%%%%%%%%%%%%%%%%%%%%%%%%%%%%%%%%%%%%%%%%%%%%%%%%%%%
% \section{Implementation}
%\iffalse
%<*package>
%\fi
%
% This section describes the definitions file |childdoc.def|.

% The definitions cannot be loaded using |\usepackage| or |\RequirePackage|
% which has a mechanism to prevent loading a style file more than once.
% When loading the definitions by means of |\input|
% multiple instances have to be prevented manually:
%\iffalse
%This code needs to be before the `\ProvidesFile' directive
%which is defined at the beginning of this file.
%Therefore it is also placed there and commented out here.
%</package>
%<*discard>
%\fi
%    \begin{macrocode}
\ifdefined\childdocmain\endinput\fi
%    \end{macrocode}
%\iffalse
%</discard>
%<*package>
%\fi
%
% \macro{\ifchilddoc}
% \macro{\ifchilddocmanual}
% The conditional |\ifchilddoc| tells whether a
% child (true) or main (false) document is being compiled.
% The conditional |\ifchilddocmanual| tells whether
% the |\includeonly| mechanism is used (false) or
% the selection of child files must be performed manually (true).
% The definitions initialise to false:
%    \begin{macrocode}
\newif\ifchilddoc
\newif\ifchilddocmanual
%    \end{macrocode}

% \macro{\childdocname}
% \macro{\childdocjob}
% The macro |\childdocname| stores the name of the main document
% to be compiled. The macro |\childdocjob| stores the name of
% the document on which the \LaTeX{} compiler was originally invoked.
% The content of |\jobname| cannot be compared
% to filenames specified in the source due to different catcodes.
% The following code rescans |\jobname|, stores the result
% in |\childdocname| and saves a copy in |\childdocjob|:
%    \begin{macrocode}
\edef\childdocname{\scantokens\expandafter{\jobname\noexpand}}
\let\childdocjob\childdocname
%    \end{macrocode}

% \macro{\childdocdisable}
% The macro |\childdocdisable| prevents the main file
% from being processed more than once.
% At this stage, the main document command |\childdocmain|
% is assumed to be called once again where it should do nothing.
% Any subsequent call to it should prevent
% a secondary processing of the main document
% It overwrites the forwarding commands
% |\childdocof| and |\childdocforward|
% with empty macros to prevent further inclusions of the main document:
%    \begin{macrocode}
\newcommand{\childdocdisable}
{
  \renewcommand{\childdocmain}[1]{\renewcommand{\childdocmain}[1]{\endinput}}
  \renewcommand{\childdocof}[1]{}
  \renewcommand{\childdocby}[2][]{}
  \renewcommand{\childdocforward}[2][]{}
  \renewcommand{\childdocdisable}{}
}
%    \end{macrocode}

% \macro{\childdocmain}
% The macro |\childdocmain| is to be called at the top of the main file
% with nothing or the main filename (without extension) as argument.
% First, it breaks loops.
% If the argument is not empty and does not match |\childdocname|
% (which is set by the first inclusion of |childdoc.def|),
% |\ifchilddoc| is set to true, |\includeonly| is applied to the child file
% and |\jobname| is set to the main file
% (for proper handling of |.aux| files):
%    \begin{macrocode}
\newcommand{\childdocmain}[1]
{
  \childdocdisable\childdocmain{}
  \if?#1?\else
    \begingroup
      \def\childdoctmp{#1}
      \ifx\childdoctmp\childdocname
        \def\childdoctmp{}
      \else
        \def\childdoctmp
        {
          \childdoctrue
          \includeonly{\childdocname}
          \def\childdocjob{#1}
          \def\jobname{#1}
        }
      \fi
      \expandafter
    \endgroup
    \childdoctmp
  \fi
}
%    \end{macrocode}

% \macro{\childdocof}
% The command |\childdocof| redirects
% compilation to the main file |#1|.
%    \begin{macrocode}
\newcommand{\childdocof}[1]
{
  \childdocdisable
  \childdoctrue
  \includeonly{\childdocname}
  \def\jobname{#1}
  \def\childdocjob{#1}
  \input{#1}
}
%    \end{macrocode}

% \macro{\childdocby}
% The command |\childdocby| ....
%    \begin{macrocode}
\newcommand{\childdocby}[2][]
{
  \childdocdisable
  \childdoctrue
  \childdocmanualtrue
  \if?#1?\else
    \def\jobname{#2}
  \fi
  \def\childdocjob{#2}
  \input{#2}
  \endinput
}
%    \end{macrocode}

% \macro{\childdocforward}
% The command |\childdocforward| redirects
% compilation to the main file or
% (if the optional argument is given) a child file.
% Parameters are set as if the main file
% or a child file starting with |\childdocof| was compiled.
% Then compilation is handed over to the main file:
%    \begin{macrocode}
\newcommand{\childdocforward}[2][]
{
  \begingroup
    \if?#1?
      \def\childdoctmp
      {
        \def\childdocname{#2}
        \def\childdocjob{#2}
        \def\jobname{#2}
        \input{#2}
        \endinput
      }
    \else
      \def\childdoctmp
      {
        \childdocdisable
        \def\childdocname{#2}
        \childdoctrue
        \includeonly{#2}
        \def\childdocjob{#1}
        \def\jobname{#1}
        \input{#1}
        \endinput
      }
    \fi
    \expandafter
  \endgroup
  \childdoctmp
}
%    \end{macrocode}

% \macro{\childdocforwardprefix}
% The command |\childdocforwardprefix| redirects
% compilation to the main or a child file by means of a pattern.
% The prefix |#1| in the current filename is replaced by |#2|
% and the suffix of the current filename is kept
% (it is assumed that the filename does not contain the substring `|~~~|'
% which is used as a delimiter).
% Compilation is handed over to the new file by |\childdocforward|:
%    \begin{macrocode}
\newcommand{\childdocforwardprefix}[3][]
{
  \begingroup
    \def\childdocextract #2##1~~~{\def\childdoctmp{\childdocforward[#1]{#3##1}}}
    \expandafter\childdocextract\childdocname~~~
    \expandafter
  \endgroup
  \childdoctmp
}
%    \end{macrocode}

% \macro{\childdoc}
% The deprecated macro |\childdoc| is a legacy version of |\childdocmain|:
%    \begin{macrocode}
\newcommand{\childdoc}{\childdocmain}
%    \end{macrocode}

% \macro{\childdocredirect}
% The deprecated macro |\childdocredirect| is a legacy version
% of |\childdocforward| and |\childdocforwardprefix|:
%    \begin{macrocode}
\newcommand{\childdocredirect}[2][]
{
  \begingroup
    \if?#1?
      \def\childdoctmp{\childdocforward{#2}}
    \else
      \def\childdoctmp{\childdocforwardprefix{#1}{#2}}
    \fi
    \expandafter
  \endgroup
  \childdoctmp
}
%    \end{macrocode}

%\iffalse
%</package>
%\fi
%
\endinput
\childdocforward[|\textit{main}|]{|\textit{dest}|}"|
\end{center}
%
Here \textit{target} is the name of the output file,
\textit{main} is the name of the main file
and \textit{dest} is the name of the main or child file to be processed
(all filenames without extensions).
The optional argument \textit{main} can be omitted
if \textit{main} matches \textit{dest}.
Optionally, compilation \textit{flags} can be defined via |\def| commands.
This command line makes the \TeX{} engine believe
it is compiling the file \textit{target}
whose content is specified as the latter parameter.
The provided code then forwards the processing to
\textit{main} or \textit{dest} as described in \secref{sec:forward}.

%%%%%%%%%%%%%%%%%%%%%%%%%%%%%%%%%%%%%%%%%%%%%%%%%%%%%%%%%%%%%%%%%%%%%%%%%%%%%%%%
\subsection{Include by Input}
\label{sec:input}

Including child documents by |\include| has some restrictions by design.
Most notably, the content of a child document always occupies
its own set of pages; pages cannot be shared between child documents.
Usually, this behaviour makes perfect sense
because each child document contain an essential part of the document.
However, in some situations it may be desirable to compose
a document from a collection of parts
without having mandatory page breaks between then.
For this case, the package
provides a mechanism to include parts
by |\input| which can also be processed individually.
However, by construction this mechanism
requires manual handling of the content to be output.

%%%%%%%%%%%%%%%%%%%%%%%%%%%%%%%%%%%%%%%%
\DescribeMacro{\ifchilddocmanual}
The main file should be prepared as usual, see \secref{sec:include}.
However, the document body must make a distinction
between processing of an individual part and of the main document, e.g.:
%
\begin{center}
\begin{tabular}{l}
|\ifchilddocmanual|\\
|\input{\childdocname}|\\
|\||else|\\
\textit{document body with }|\input{|\textit{part}|}|\\
|\||fi|
\end{tabular}
\end{center}
%
The conditional |\ifchilddocmanual| is true whenever
a part to be included by |\input| is being compiled,
and the name of the part is stored in |\childdocname|.

%%%%%%%%%%%%%%%%%%%%%%%%%%%%%%%%%%%%%%%%
\DescribeMacro{\childdocby}
Each part to be included by |\input| should start with:
%
\begin{center}
\begin{tabular}{l}
|% \iffalse
%
% childdoc.dtx Copyright (C) 2017-2018 Niklas Beisert
%
% This work may be distributed and/or modified under the
% conditions of the LaTeX Project Public License, either version 1.3
% of this license or (at your option) any later version.
% The latest version of this license is in
%   http://www.latex-project.org/lppl.txt
% and version 1.3 or later is part of all distributions of LaTeX
% version 2005/12/01 or later.
%
% This work has the LPPL maintenance status `maintained'.
%
% The Current Maintainer of this work is Niklas Beisert.
%
% This work consists of the files childdoc.dtx and childdoc.ins
% and the derived files childdoc.def and cdocsamp.tex with
% cdocsch1.tex, cdocsch2.tex, cdocsdrf.tex, cdocsfn1.tex, cdocsfn2.tex.
%
%<package>\ifdefined\childdocmain\endinput\fi
%<package>\ProvidesFile{childdoc.def}[2018/12/30 v2.0 child document driver]
%<samplemain>\ProvidesFile{cdocsamp.tex}[2018/12/30 v2.0 sample for childdoc]
%<*driver>
%\ProvidesFile{childdoc.drv}[2018/12/30 v2.0 childdoc reference manual file]
\PassOptionsToClass{10pt,a4paper}{article}
\documentclass{ltxdoc}

\usepackage[margin=35mm]{geometry}
\usepackage{hyperref}
\usepackage{hyperxmp}
\usepackage[usenames]{color}

\hypersetup{colorlinks=true}
\hypersetup{pdfstartview=FitH}
\hypersetup{pdfpagemode=UseNone}
\hypersetup{pdfsource={}}
\hypersetup{pdflang={en-UK}}
\hypersetup{pdfcopyright={Copyright 2017-2018 Niklas Beisert.
  This work may be distributed and/or modified under the
  conditions of the LaTeX Project Public License, either version 1.3
  of this license or (at your option) any later version.}}
\hypersetup{pdflicenseurl={http://www.latex-project.org/lppl.txt}}
\hypersetup{pdfcontactaddress={ETH Zurich, ITP, HIT K,
  Wolfgang-Pauli-Strasse 27}}
\hypersetup{pdfcontactpostcode={8093}}
\hypersetup{pdfcontactcity={Zurich}}
\hypersetup{pdfcontactcountry={Switzerland}}
\hypersetup{pdfcontactemail={nbeisert@itp.phys.ethz.ch}}
\hypersetup{pdfcontacturl={http://people.phys.ethz.ch/\xmptilde nbeisert/}}

\newcommand{\secref}[1]{\hyperref[#1]{section \ref*{#1}}}

\parskip1ex
\parindent0pt
\let\olditemize\itemize
\def\itemize{\olditemize\parskip0pt}

\begin{document}

\title{The \textsf{childdoc} Package}
\hypersetup{pdftitle={The childdoc Package}}
\author{Niklas Beisert\\[2ex]
  Institut f\"ur Theoretische Physik\\
  Eidgen\"ossische Technische Hochschule Z\"urich\\
  Wolfgang-Pauli-Strasse 27, 8093 Z\"urich, Switzerland\\[1ex]
  \href{mailto:nbeisert@itp.phys.ethz.ch}
  {\texttt{nbeisert@itp.phys.ethz.ch}}}
\hypersetup{pdfauthor={Niklas Beisert}}
\hypersetup{pdfsubject={Manual for the LaTeX2e Package childdoc}}
\date{30 December 2018, \textsf{v2.0}}
\maketitle

\begin{abstract}\noindent
\textsf{childdoc} is a \LaTeXe{} package
that enables the direct compilation
of document sections included by |\include|
to individual files.
\end{abstract}

\begingroup
\parskip0ex
\tableofcontents
\endgroup

%%%%%%%%%%%%%%%%%%%%%%%%%%%%%%%%%%%%%%%%%%%%%%%%%%%%%%%%%%%%%%%%%%%%%%%%%%%%%%%%
%%%%%%%%%%%%%%%%%%%%%%%%%%%%%%%%%%%%%%%%%%%%%%%%%%%%%%%%%%%%%%%%%%%%%%%%%%%%%%%%
\section{Introduction}

\LaTeX{} provides a mechanism to structure a large document (such as a book)
into a main file and several child files (containing the chapters)
using the |\include| command.
This mechanism is beneficial for documents
which span hundreds of pages in order to
make the source file(s) more manageable.
Moreover, compilation can be restricted to
selected child files by means of the |\includeonly| command.
The latter feature can be used to reduce the compilation time while editing
(this was significantly more useful in the earlier days of \LaTeX{})
or to generate a smaller document which is easier to navigate.
Another application of |\includeonly| is to generate
documents consisting of selected parts of the complete document.

However, there are a few drawbacks of the plain |\include| mechanism:
\begin{itemize}
\item
The child files cannot be compiled on their own,
they can only be compiled via the main file.
A naive editing environment
(such as a text editor with an option
to have the current file processed by \LaTeX)
may require one to switch to the main file before compiling;
attempting to compile the child file produces errors.
\item
The main file must be modified (each time)
to adjust the |\includeonly| command
to the present needs. This easily leaves the main file in a messy state.
\item
The generated document will always carry the filename
of the main document. This is inconvenient if
several child files are to be compiled and
to be kept for distribution.
\end{itemize}

The present package provides a simple interface
to make child files individually compilable by \LaTeX{}.
Compiling a child file then has the same effect as compiling
the main file with an |\includeonly| command
to select the appropriate child.
Moreover the generated document will carry the name of the child
rather than the main file.
This resolves all three above issues.

This feature is meant to make the editing of books,
thesis documents and lecture notes somewhat more convenient.
However, the package can also be used efficiently for
composing a series of documents (such as exercise sheets)
which are typically distributed individually.
It then assists the author in generating the individual documents
(potentially in different versions)
as well as a document containing the collected series.
Another application is in developing style files
or other kinds of included material
where compilation of the style file could redirect
to a sample or test file.

%%%%%%%%%%%%%%%%%%%%%%%%%%%%%%%%%%%%%%%%%%%%%%%%%%%%%%%%%%%%%%%%%%%%%%%%%%%%%%%%
%%%%%%%%%%%%%%%%%%%%%%%%%%%%%%%%%%%%%%%%%%%%%%%%%%%%%%%%%%%%%%%%%%%%%%%%%%%%%%%%
\section{Usage}

First of all, the package \textsf{childdoc} is \emph{not} a standard
\LaTeXe{} |.sty| style file! Therefore it needs to be invoked in
a non-standard way.

%%%%%%%%%%%%%%%%%%%%%%%%%%%%%%%%%%%%%%%%%%%%%%%%%%%%%%%%%%%%%%%%%%%%%%%%%%%%%%%%
\subsection{Included Files}
\label{sec:include}

%%%%%%%%%%%%%%%%%%%%%%%%%%%%%%%%%%%%%%%%
\DescribeMacro{\childdocmain}
To use the package, add the commands
\begin{center}
\begin{tabular}{l}
|\input{childdoc.def}|\\
|\childdocmain{}|\\
\end{tabular}
\end{center}
at the very top of the main \LaTeX{} file,
in particular \emph{before} the |\documentclass| statement!
The argument of |\childdocmain| should be left empty
(but it must be present).

%%%%%%%%%%%%%%%%%%%%%%%%%%%%%%%%%%%%%%%%
\DescribeMacro{\childdocof}
Furthermore, add the commands
\begin{center}
\begin{tabular}{l}
|\input{childdoc.def}|\\
|\childdocof{|\textit{main}|}|\\
\end{tabular}
\end{center}
at the top of every child file \textit{child}
which is included by |\include{|\textit{child}|}|
from within the main file
(or at least for those files to be compiled individually).
The argument \textit{main} must be the filename of the main file.

There are a couple of
considerations in setting up the main and child documents:

%%%%%%%%%%%%%%%%%%%%%%%%%%%%%%%%%%%%%%%%
\paragraph{Restrictions.}

Please note the following restrictions:
\begin{itemize}
\item
|\childdocmain| must be called with one argument \textit{main}
to ensure compatibility with earlier version of the package.
It must either be empty (|\childdocmain{}|)
or precisely match the filename of the main file in which it is specified.
See \secref{sec:detection} for further information.
\item
The filename \textit{main} must be specified without the |.tex| extension.
\item
The filename \textit{main} is case sensitive
(even in case-insensitive file systems)
due to internal string comparison.
\item
The argument \textit{main} should be fully expanded, it cannot be a macro.
\item
Subdirectories and special characters should be avoided in filenames.
\item
The command |\childdocmain{|\textit{main}|}| must be followed by a whitespace.
It should not be followed immediately by another command
or by a comment mark `|%|'.
This is because the \TeX{} parser reads the token immediately following
the argument of |\childdocmain| and puts it
at the beginning of every child section;
however, a white\-space is ignored.
\end{itemize}

%%%%%%%%%%%%%%%%%%%%%%%%%%%%%%%%%%%%%%%%
\paragraph{Content of Main File.}

It is advisable to place all content in the child files included by |\include|.
Any output contained in the main file will appear in all child documents
unless suppressed manually;
it cannot be suppressed automatically by the |\includeonly| directive
and thus should normally be avoided.
A method to include some content in the main file
by means of conditional processing is described in \secref{sec:conditional}.

%%%%%%%%%%%%%%%%%%%%%%%%%%%%%%%%%%%%%%%%
\paragraph{Page Numbering.}

When only a part of the document is compiled,
the appropriate numbering of pages
(as well as other status parameters)
is determined from the |.aux| files.
The latter contain information from previous passes.
However this information needs to propagate through
all intermediate child documents.
Therefore the page numbering in child documents may well
be inconsistent until the complete document is compiled at least once.

A useful (if unconventional) way to always ensure a consistent
page numbering is to restart the numbering in each child document
and denote the pages by `\textit{child}|.|\textit{page}'
where \textit{child} represents the chapter/section number of the child file.
This can be achieved by the command
|\numberwithin{page}{|\textit{child}|}|
of the \textsf{amsmath} package
where \textit{child} can be |chapter| or |section|
depending on the chosen structuring.
Alternatively, one can modify the macro |\thepage| appropriately
and reset the counter |page| at the start of each child file.

%%%%%%%%%%%%%%%%%%%%%%%%%%%%%%%%%%%%%%%%%%%%%%%%%%%%%%%%%%%%%%%%%%%%%%%%%%%%%%%%
\subsection{Conditional Processing}
\label{sec:conditional}

The package provides a mechanism to compile different versions
of a document. To customise the versions further some conditional processing
can come in handy to distinguish which version is being compiled.
The package provides two macros to describe the compilation context:

%%%%%%%%%%%%%%%%%%%%%%%%%%%%%%%%%%%%%%%%
\DescribeMacro{\ifchilddoc}
The conditional |\ifchilddoc| distinguishes between the compilation of
child documents and the main document:
%
\begin{center}
|\ifchilddoc |\textit{child-code}| |[|\||else |\textit{main-code}]| \||fi|
\end{center}

%%%%%%%%%%%%%%%%%%%%%%%%%%%%%%%%%%%%%%%%
\DescribeMacro{\childdocname}
\DescribeMacro{\childdocjob}
The macro |\childdocname| contains the filename (without extension)
of the main or child file being processed.
Note that |\childdocjob| will always contain the name of the main file.

%%%%%%%%%%%%%%%%%%%%%%%%%%%%%%%%%%%%%%%%
\paragraph{Title Page.}

Conditional processing can be used to include a title or banner page
in the main document when proper precautions are taken.
Importantly, the code in the main file should ensure that the page counter
(as well as other status parameters which are stored in the |.aux| files)
takes the same value after the conditional processing.
Otherwise the page numbers may take divergent values
depending on which part is compiled.

For example, a title page could be declared by:
%
\begin{center}
\begin{tabular}{l}
|\ifchilddoc\||else|\\
|\addtocounter{page}{-1}|\\
\textit{code for title page}\\
|\newpage|\\
|\||fi|
\end{tabular}
\end{center}
%
A banner page for the child documents can be generated by:
%
\begin{center}
\begin{tabular}{l}
|\ifchilddoc|\\
|\addtocounter{page}{-1}|\\
\textit{code for banner page}\\
|\newpage|\\
|\||fi|
\end{tabular}
\end{center}
%
Here one could write a message such as:
\begin{center}
|This is the part \childdocname{} of \childdocjob{}.|
\end{center}

%%%%%%%%%%%%%%%%%%%%%%%%%%%%%%%%%%%%%%%%%%%%%%%%%%%%%%%%%%%%%%%%%%%%%%%%%%%%%%%%
\subsection{Flags}
\label{sec:flags}

The package makes it easy to generate different versions
of the main or child documents.
To this end compilation flags can be defined
and assigned different default values.
They will be particularly useful in conjunction
with the forwarding mechanism described in \secref{sec:forward}.

For example, it may be useful to have a flag |\version|
which can be set to |draft| or |final|.
The document source will contain some conditional code
depending on the value of |\version|.
Suppose further, the flag should default to |final| for the main file
and to |draft| for child files
which is a natural assignment for editing the document.
This is achieved by placing the following code
in the preamble of the main document
(below the |\childdocmain| directive):
%
\begin{center}
\begin{tabular}{l}
|\ifchilddoc|\\
|\providecommand{\version}{draft}|\\
|\||else|\\
|\providecommand{\version}{final}|\\
|\||fi|
\end{tabular}
\end{center}
%
The definition by |\providecommand| makes sure
that previous definitions are not overwritten.
Further statements |\providecommand{\version}{...}|
can thus be added before the above code to override it.

For the main file, one might add a line
(between |\childdocmain| and the above block)
%
\begin{center}
|%\ifchilddoc\||else\providecommand{\version}{draft}\||fi|
\end{center}
%
which can be uncommented to produce a draft version.
Likewise one can add a line to the very top of a child file
(above the |\childdocof{|\textit{main}|}| directive)
%
\begin{center}
|%\providecommand{\version}{final}|
\end{center}
%
which can be uncommented to produce the final version of this child document.

%%%%%%%%%%%%%%%%%%%%%%%%%%%%%%%%%%%%%%%%%%%%%%%%%%%%%%%%%%%%%%%%%%%%%%%%%%%%%%%%
\subsection{Forwarding}
\label{sec:forward}

Different versions of the main or child documents
using compilation flags as described in \secref{sec:flags}
can be (permanently) stored in different files
for convenient compilation, viewing and distribution.
To this end, the package defines a command
to pass on compilation to a different file:

%%%%%%%%%%%%%%%%%%%%%%%%%%%%%%%%%%%%%%%%
\DescribeMacro{\childdocforward}
The command |\childdocforward| redirects processing to
another source file:
%
\begin{center}
\begin{tabular}{l}
|\input{childdoc.def}|\\
|\childdocforward[|\textit{main}|]{|\textit{dest}|}|\\
\end{tabular}
\end{center}
%
The argument \textit{dest} is the destination file
(without extension).
It should be the main file or one of the child files.
Note that further \textsf{childdoc} directives
such as |\childdocof| and |\childdocforward|
in the indicated file will be processed in this form.
The optional argument \textit{main}
passes on directly to the main file \textit{main}
while pretending to compile the child \textit{dest}.
This form behaves as if \textit{dest}
issues |\childdocof{|\textit{main}|}| right away,
and no further \textsf{childdoc} directives will be processed.

%%%%%%%%%%%%%%%%%%%%%%%%%%%%%%%%%%%%%%%%
\DescribeMacro{\...prefix}
In the alternative form |\childdocforwardprefix|,
%
\begin{center}
\begin{tabular}{l}
|\input{childdoc.def}|\\
|\childdocforwardprefix[|\textit{main}|]{|\textit{prefix}|}{|\textit{dest}|}|
\end{tabular}
\end{center}
%
the destination file is determined by a pattern
depending on the current file:
To make this work, the current file must be called
`{\textit{prefix}\hspace{0.2em}\textit{suffix}}'
with \textit{prefix} matching precisely the argument.
Processing is then passed on to the file
`{\textit{dest}\hspace{0.2em}\textit{suffix}}'.
Surely, the same effect is achieved by
directly specifying the
argument `{\textit{dest}\hspace{0.2em}\textit{suffix}}'
in the first form.
However, that requires to set up a different file
for each child. With the alternative form of the command
all these files can have exactly the same content
which simplifies setting them up and maintaining them.

For example, the following file |draft.tex|
with a compilation flag |\version| as described in \secref{sec:flags}
compiles the main document as a draft:
%
\begin{center}
\begin{tabular}{l}
|\def\version{draft}|\\
|\input{childdoc.def}|\\
|\childdocforward{|\textit{main}|}|
\end{tabular}
\end{center}
%
Likewise, the following files |final|\textit{nn}|.tex|
compile the final version of the child document
|child|\textit{nn}|.tex|:
%
\begin{center}
\begin{tabular}{l}
|\def\version{final}|\\
|\input{childdoc.def}|\\
|\childdocforwardprefix{final}{child}|
\end{tabular}
\end{center}
%

Note that when several versions of a main file and/or of each child file
are to be generated, it may be convenient to set up a |Makefile| or
shell script to automatise the process.

%%%%%%%%%%%%%%%%%%%%%%%%%%%%%%%%%%%%%%%%%%%%%%%%%%%%%%%%%%%%%%%%%%%%%%%%%%%%%%%%
\subsection{Command Line Processing}
\label{sec:commandline}

The effect of redirection files can also be achieved by invoking
the \LaTeX{} compiler with a more elaborate command line.
Most conveniently this should be done as part
of a shell script or a |Makefile|.

When using \textsf{childdoc} in the main file, the following
command lines effectively perform a redirection
(note that depending on the shell being used,
backslashes may have to be doubled: `|\|' $\to$ `|\\|'):
%
\begin{center}
|... -jobname "|\textit{target}|" |\\|"|[\textit{flags}]%
|\input{childdoc.def}\childdocforward[|\textit{main}|]{|\textit{dest}|}"|
\end{center}
%
Here \textit{target} is the name of the output file,
\textit{main} is the name of the main file
and \textit{dest} is the name of the main or child file to be processed
(all filenames without extensions).
The optional argument \textit{main} can be omitted
if \textit{main} matches \textit{dest}.
Optionally, compilation \textit{flags} can be defined via |\def| commands.
This command line makes the \TeX{} engine believe
it is compiling the file \textit{target}
whose content is specified as the latter parameter.
The provided code then forwards the processing to
\textit{main} or \textit{dest} as described in \secref{sec:forward}.

%%%%%%%%%%%%%%%%%%%%%%%%%%%%%%%%%%%%%%%%%%%%%%%%%%%%%%%%%%%%%%%%%%%%%%%%%%%%%%%%
\subsection{Include by Input}
\label{sec:input}

Including child documents by |\include| has some restrictions by design.
Most notably, the content of a child document always occupies
its own set of pages; pages cannot be shared between child documents.
Usually, this behaviour makes perfect sense
because each child document contain an essential part of the document.
However, in some situations it may be desirable to compose
a document from a collection of parts
without having mandatory page breaks between then.
For this case, the package
provides a mechanism to include parts
by |\input| which can also be processed individually.
However, by construction this mechanism
requires manual handling of the content to be output.

%%%%%%%%%%%%%%%%%%%%%%%%%%%%%%%%%%%%%%%%
\DescribeMacro{\ifchilddocmanual}
The main file should be prepared as usual, see \secref{sec:include}.
However, the document body must make a distinction
between processing of an individual part and of the main document, e.g.:
%
\begin{center}
\begin{tabular}{l}
|\ifchilddocmanual|\\
|\input{\childdocname}|\\
|\||else|\\
\textit{document body with }|\input{|\textit{part}|}|\\
|\||fi|
\end{tabular}
\end{center}
%
The conditional |\ifchilddocmanual| is true whenever
a part to be included by |\input| is being compiled,
and the name of the part is stored in |\childdocname|.

%%%%%%%%%%%%%%%%%%%%%%%%%%%%%%%%%%%%%%%%
\DescribeMacro{\childdocby}
Each part to be included by |\input| should start with:
%
\begin{center}
\begin{tabular}{l}
|\input{childdoc.def}|\\
|\childdocby{|\textit{main}|}|\\
\end{tabular}
\end{center}
%
The directive |\childdocby| is similar to |\childdocof|
described in \secref{sec:include},
but the subsequent selection of content must be done manually.
To that end, both |\ifchilddoc| and |\ifchilddocmanual|
will be true upon processing of a part,
and the name of the part is stored in |\childdocname|.
Note that |\jobname| will be set to the filename of the current part
so that each part receives an individual |.aux| file
that does not interfere with the |.aux| file(s) of the main document.
This behaviour can be altered by the alternative form
|\childdocby[*]{|\textit{main}|}| (with a non-empty optional argument)
which uses the |.aux| file of the main document
by setting |\jobname| to \textit{main}.

%%%%%%%%%%%%%%%%%%%%%%%%%%%%%%%%%%%%%%%%%%%%%%%%%%%%%%%%%%%%%%%%%%%%%%%%%%%%%%%%
\subsection{Driver Development}
\label{sec:driver}

The \textsf{childdoc} mechanism can also be use for the development
of definition files such as \LaTeX{} styles or classes.
This case differs from the above setup with multiple parts
included by |\include| in that no |\includeonly| should be invoked.
This can be achieved by starting the include file
(before |\ProvidesPackage|) with:
%
\begin{center}
\begin{tabular}{l}
|\input{childdoc.def}|\\
|\childdocforward{|\textit{main}|}|\\
\end{tabular}
\end{center}
%
or alternatively with:
%
\begin{center}
\begin{tabular}{l}
|\input{childdoc.def}|\\
|\childdocby{|\textit{main}|}|\\
\end{tabular}
\end{center}
%
Both forms have slightly different effects as described above.
The main file is prepared as usual, see \secref{sec:include}.

%%%%%%%%%%%%%%%%%%%%%%%%%%%%%%%%%%%%%%%%%%%%%%%%%%%%%%%%%%%%%%%%%%%%%%%%%%%%%%%%
\subsection{Legacy Detection}
\label{sec:detection}

The directive |\childdocmain| in the main file can detect
whether the complete document or merely a child is to be compiled
even without using the directive |\childdocof|.
This method is deprecated because it is less robust
and there is no compelling reason to use it;
it is merely provided for backward compatibility
and it may be removed in future versions.

If the detection mechanism is to be used,
it is mandatory to correctly specify
the filename of the main file as the argument of |\childdocmain|:
%
\begin{center}
\begin{tabular}{l}
|\input{childdoc.def}|\\
|\childdocmain{|\textit{main}|}|\\
\end{tabular}
\end{center}
%
If |\jobname| does not match the argument \textit{main} of |\childdocmain|,
it is assumed that |\jobname| points to the child file to be compiled.
When using |\childdocmain| with the main file specified as argument,
it suffices to start a child file
with just |\input{|\textit{main}|}|
without loading of the package and using |\childdocof|.
If instead all processing is done
with the appropriate \textsf{childdoc} directives,
the argument of \textit{main} of |\childdocmain| can be empty.

An alternative version of the command line processing described
in \secref{sec:commandline} using the detection mechanism reads:
%
\begin{center}
|... -jobname "|\textit{target}|" "|[\textit{flags}]%
[|\def\jobname{|\textit{dest}|}|]|\input{|\textit{main}|}"|
\end{center}

%%%%%%%%%%%%%%%%%%%%%%%%%%%%%%%%%%%%%%%%%%%%%%%%%%%%%%%%%%%%%%%%%%%%%%%%%%%%%%%%
\subsection{Manual Code}
\label{sec:manual}

In case one cannot be certain whether the definitions file |childdoc.def|
is installed on the target \TeX{} distribution
and one prefers not to ship it,
it is conceivable to paste a few relevant commands into the sources.

To that end, drop all statements |\input{childdoc.def}|
and perform the replacements as outlined below.
Instead of |\childdocmain{|\textit{main}|}| add the following code
to the top of the main file:
%
\begin{center}
\begin{tabular}{l}
|\||ifdefined\childdocname\endinput\||fi\newif\ifchilddoc|\\
|\edef\childdocname{\scantokens\expandafter{\jobname\noexpand}}|\\
|\def\childdocmain{|\textit{main}|}\||ifx\childdocmain\childdocname\||else|\\
|\childdoctrue\includeonly{\childdocname}\let\jobname\childdocmain\||fi|\\
\end{tabular}
\end{center}
%
Instead of |\childdocof{|\textit{main}|}| just include the main file
at the top of each child file:
%
\begin{center}
|\input{|\textit{main}|}|
\end{center}
%
A simple redirection |\childdocforward{|\textit{dest}|}| is achieved by:
%
\begin{center}
|\def\jobname{|\textit{dest}|}\input{\jobname}|
\end{center}
%
The redirection with prefix
|\childdocforwardprefix[|\textit{prefix}|]{|\textit{dest}|}|
is accomplished by:
%
\begin{center}
\begin{tabular}{l}
|{\edef\jobname{\scantokens\expandafter{\jobname\noexpand}}|\\
|\def\redirectjob |\textit{prefix}|#1~~~{\gdef\jobname{|\textit{dest}|#1}}|\\
|\expandafter\redirectjob\jobname~~~}\input{\jobname}|
\end{tabular}
\end{center}

In an alternative approach,
child documents can be compiled by a specific command line
without additional code or specific definitions:
%
\begin{center}
|... -jobname "|\textit{target}|" "|[\textit{flags}]%
|\includeonly{|\textit{dest}|}\input{|\textit{main}|}"|
\end{center}
%

%%%%%%%%%%%%%%%%%%%%%%%%%%%%%%%%%%%%%%%%%%%%%%%%%%%%%%%%%%%%%%%%%%%%%%%%%%%%%%%%
%%%%%%%%%%%%%%%%%%%%%%%%%%%%%%%%%%%%%%%%%%%%%%%%%%%%%%%%%%%%%%%%%%%%%%%%%%%%%%%%
\section{Information}

%%%%%%%%%%%%%%%%%%%%%%%%%%%%%%%%%%%%%%%%%%%%%%%%%%%%%%%%%%%%%%%%%%%%%%%%%%%%%%%%
\subsection{Copyright}

Copyright \copyright{} 2017--2018 Niklas Beisert

This work may be distributed and/or modified under the
conditions of the \LaTeX{} Project Public License, either version 1.3
of this license or (at your option) any later version.
The latest version of this license is in
  \url{http://www.latex-project.org/lppl.txt}
and version 1.3 or later is part of all distributions of \LaTeX{}
version 2005/12/01 or later.

This work has the LPPL maintenance status `maintained'.

The Current Maintainer of this work is Niklas Beisert.

This work consists of the files |README.txt|, |childdoc.ins| and |childdoc.dtx|
as well as the derived files |childdoc.def|, |cdocsamp.tex|
with |cdocsch1.tex|, |cdocsch2.tex|, |cdocspt3.tex|, |cdocspt4.tex|,
|cdocsdrf.tex|, |cdocsfn1.tex|, |cdocsfn2.tex|
as well as |childdoc.pdf|.

%%%%%%%%%%%%%%%%%%%%%%%%%%%%%%%%%%%%%%%%%%%%%%%%%%%%%%%%%%%%%%%%%%%%%%%%%%%%%%%%
\subsection{Files and Installation}

The package consists of the files:
%
\begin{center}
\begin{tabular}{ll}
    |README.txt|   & readme file \\
    |childdoc.ins| & installation file \\
    |childdoc.dtx| & source file \\
    |childdoc.def| & definition file \\
    |cdocsamp.tex| & sample main file \\
    |cdocsch1.tex| & sample include file \\
    |cdocsch2.tex| & sample include file \\
    |cdocspt3.tex| & sample part file \\
    |cdocspt4.tex| & sample part file \\
    |cdocsdrf.tex| & sample redirection file \\
    |cdocsfn1.tex| & sample redirection file \\
    |cdocsfn2.tex| & sample redirection file \\
    |childdoc.pdf| & manual
\end{tabular}
\end{center}
%
The distribution consists of the files
|README.txt|, |childdoc.ins| and |childdoc.dtx|.
%
\begin{itemize}
\item
Run (pdf)\LaTeX{} on |childdoc.dtx|
to compile the manual |childdoc.pdf| (this file).
\item
Run \LaTeX{} on |childdoc.ins| to create the definitions file |childdoc.def|
and the sample |cdocsamp.tex| with include files
|cdocsch1.tex|, |cdocsch2.tex|, |cdocspt3.tex|, |cdocspt4.tex|,
|cdocsdrf.tex|, |cdocsfn1.tex|, |cdocsfn2.tex|.
Then copy the file |childdoc.def| to an appropriate directory of your \LaTeX{}
distribution, e.g.\ \textit{texmf-root}|/tex/latex/childdoc|.
\end{itemize}

%%%%%%%%%%%%%%%%%%%%%%%%%%%%%%%%%%%%%%%%%%%%%%%%%%%%%%%%%%%%%%%%%%%%%%%%%%%%%%%%
\subsection{Related CTAN Packages}

There are several other packages which offer a similar functionality:
%
\begin{itemize}
\item
The packages
\href{http://ctan.org/pkg/docmute}{\textsf{docmute}},
\href{http://ctan.org/pkg/includex}{\textsf{includex}} and
\href{http://ctan.org/pkg/standalone}{\textsf{standalone}}
provide commands to include only the document body of
a child file thus allowing both files to be compiled individually.
\item
The packages \href{http://ctan.org/pkg/subdocs}{\textsf{subdocs}}
and \href{http://ctan.org/pkg/subfiles}{\textsf{subfiles}}
provide structures in which the main and child documents can be
encapsulated and allowing them to be compiled individually.
The inclusion mechanism is different from the conventional |\include|.
\item
The package \href{http://ctan.org/pkg/combine}{\textsf{combine}}
is an elaborate solution to combine several documents into one.
\end{itemize}
%
See also the CTAN topic \href{http://ctan.org/topic/subdocs}{\textsf{subdocs}}
for further related packages.
The present package differs from the above solutions in that
a document structure constructed with the conventional |\include| mechanism
just needs two extra commands at the top of every file
such that all constituent files can be compiled individually.

%%%%%%%%%%%%%%%%%%%%%%%%%%%%%%%%%%%%%%%%%%%%%%%%%%%%%%%%%%%%%%%%%%%%%%%%%%%%%%%%
%\subsection{Feature Suggestions}
%
%The following is a list of features which may be useful for future
%versions of this package:
%%
%\begin{itemize}
%\item
%\ldots
%\end{itemize}

%%%%%%%%%%%%%%%%%%%%%%%%%%%%%%%%%%%%%%%%%%%%%%%%%%%%%%%%%%%%%%%%%%%%%%%%%%%%%%%%
\subsection{Revision History}

%%%%%%%%%%%%%%%%%%%%%%%%%%%%%%%%%%%%%%%%
\paragraph{v2.0:} 2018/12/30

\begin{itemize}
\item
immediate forward processing
\item
added |\childdocby| mechanism
\item
manual restructured
\end{itemize}

%%%%%%%%%%%%%%%%%%%%%%%%%%%%%%%%%%%%%%%%
\paragraph{v1.6:} 2018/01/17

\begin{itemize}
\item
application for development of include files
\item
corrections to manual
\end{itemize}

%%%%%%%%%%%%%%%%%%%%%%%%%%%%%%%%%%%%%%%%
\paragraph{v1.5:} 2017/05/21

\begin{itemize}
\item
more complete structuring introduced
\item
|\childdocof| introduced
\item
|\childdoc| renamed to |\childdocmain|
\item
|\childredirect| renamed to |\childdocforward| and |\childdocforwardprefix|
and functionality expanded
\end{itemize}

%%%%%%%%%%%%%%%%%%%%%%%%%%%%%%%%%%%%%%%%
\paragraph{v1.0:} 2017/04/27

\begin{itemize}
\item
manual and install package
\item
first version published on CTAN
\end{itemize}

%%%%%%%%%%%%%%%%%%%%%%%%%%%%%%%%%%%%%%%%
\paragraph{v0.6:} 2017/04/26

\begin{itemize}
\item
redirection mechanism added
\end{itemize}

%%%%%%%%%%%%%%%%%%%%%%%%%%%%%%%%%%%%%%%%
\paragraph{v0.5:} 2017/04/26

\begin{itemize}
\item
functionality in definition file
\end{itemize}


%%%%%%%%%%%%%%%%%%%%%%%%%%%%%%%%%%%%%%%%%%%%%%%%%%%%%%%%%%%%%%%%%%%%%%%%%%%%%%%%
%%%%%%%%%%%%%%%%%%%%%%%%%%%%%%%%%%%%%%%%%%%%%%%%%%%%%%%%%%%%%%%%%%%%%%%%%%%%%%%%
%%%%%%%%%%%%%%%%%%%%%%%%%%%%%%%%%%%%%%%%%%%%%%%%%%%%%%%%%%%%%%%%%%%%%%%%%%%%%%%%
\appendix

\settowidth\MacroIndent{\rmfamily\scriptsize 000\ }

 \DocInput{childdoc.dtx}

\end{document}
%</driver>
% \fi
%
% %%%%%%%%%%%%%%%%%%%%%%%%%%%%%%%%%%%%%%%%%%%%%%%%%%%%%%%%%%%%%%%%%%%%%%%%%%%%%%
% %%%%%%%%%%%%%%%%%%%%%%%%%%%%%%%%%%%%%%%%%%%%%%%%%%%%%%%%%%%%%%%%%%%%%%%%%%%%%%
% \section{Sample}
%\iffalse
%<*samplemain>
%\fi
%
% The following presents a sample document
% with two chapters, two parts, a title page,
% a compile flag as well as three forwarding files to set the flag.
% It consists of eight |.tex| files:
% \begin{center}
% \begin{tabular}{ll}
% |cdocsamp.tex|&main file\\
% |cdocsch1.tex|&include file for chapter 1\\
% |cdocsch2.tex|&include file for chapter 2\\
% |cdocspt3.tex|&include file for part 3\\
% |cdocspt4.tex|&include file for part 4\\
% |cdocsdrf.tex|&forwarding file for main file in draft mode\\
% |cdocsfi1.tex|&forwarding file for final version of chapter 1\\
% |cdocsfi2.tex|&forwarding file for final version of chapter 2\\
% \end{tabular}
% \end{center}
% Each of the eight files can be compiled directly by the \LaTeX{} compiler.
%
% %%%%%%%%%%%%%%%%%%%%%%%%%%%%%%%%%%%%%%
% \paragraph{Main File.}
%
% The main file is called |cdocsamp.tex|.
%
% Load the \textsf{childdoc} definitions and
% declare the filename for the main document:
%    \begin{macrocode}
\input{childdoc.def}
\childdocmain{}
%    \end{macrocode}

% Optional override for |\version| flag:
%    \begin{macrocode}
%%\ifchilddoc\else\providecommand{\version}{draft}\fi
%    \end{macrocode}

% Define the default values for the |\version| flag
% (|final| for the main file and |draft| for childs):
%    \begin{macrocode}
\ifchilddoc
\providecommand{\version}{draft}
\else
\providecommand{\version}{final}
\fi
%    \end{macrocode}

% Load the standard document class:
%    \begin{macrocode}
\documentclass[12pt]{article}
%    \end{macrocode}

% Start the document body:
%    \begin{macrocode}
\begin{document}
%    \end{macrocode}

% Declare a title page.
% Print title, part of document being processed and version flag:
%    \begin{macrocode}
\addtocounter{page}{-1}
\begin{center}
{\LARGE\bfseries{}childdoc example\par}
\vspace{1cm}
\ifchilddoc
\ifchilddocmanual part\else chapter\fi:
`\childdocname' of `\childdocjob'\par
\else
main document: `\childdocjob'\par
\fi
version: \version\par
\end{center}
\newpage
%    \end{macrocode}

% Manually include selected file,
% otherwise process as usual:
%    \begin{macrocode}
\ifchilddocmanual
\section*{part `\childdocname'}
\input{\childdocname}
\else
%    \end{macrocode}

% Include the two chapters:
%    \begin{macrocode}
\include{cdocsch1}
\include{cdocsch2}
%    \end{macrocode}

% Include the two parts unless only chapters should be displayed:
%    \begin{macrocode}
\ifchilddoc\else
\section{part three}
\input{cdocspt3}
\section{part four}
\input{cdocspt4}
\fi
%    \end{macrocode}

% Process as usual until here:
%    \begin{macrocode}
\fi
%    \end{macrocode}

% End of document body:
%    \begin{macrocode}
\end{document}
%    \end{macrocode}
%\iffalse
%</samplemain>
%\fi
%
% %%%%%%%%%%%%%%%%%%%%%%%%%%%%%%%%%%%%%%
% \paragraph{Chapter Include Files.}
%
% The include files are called |cdocsch1.tex| and |cdocsch2.tex|.
%
%\iffalse
%<*samplechap1|samplechap2>
%\fi

% Optional override for |\version| flag:
%    \begin{macrocode}
%%\providecommand{\version}{final}
%    \end{macrocode}

% Include the main document:
%    \begin{macrocode}
\input{childdoc.def}
\childdocof{cdocsamp}
%    \end{macrocode}

%\iffalse
%</samplechap1|samplechap2>
%\fi
%
%\iffalse
%<*samplechap1>
%\fi
% Some text for chapter 1:
%    \begin{macrocode}
\section{one}
some text in chapter one
%    \end{macrocode}

%\iffalse
%</samplechap1>
%\fi
% Some text for chapter 2:
%\iffalse
%<*samplechap2>
%\fi
%    \begin{macrocode}
\section{two}
more text in chapter two
%    \end{macrocode}

%\iffalse
%</samplechap2>
%\fi
%
% %%%%%%%%%%%%%%%%%%%%%%%%%%%%%%%%%%%%%%
% \paragraph{Part Include Files.}
%
% The include files are called |cdocspt3.tex| and |cdocspt4.tex|.
%
%\iffalse
%<*samplepart3|samplepart4>
%\fi

% Optional override for |\version| flag:
%    \begin{macrocode}
%%\providecommand{\version}{final}
%    \end{macrocode}

% Include the main document:
%    \begin{macrocode}
\input{childdoc.def}
\childdocby{cdocsamp}
%    \end{macrocode}

%\iffalse
%</samplepart3|samplepart4>
%\fi
%
%\iffalse
%<*samplepart3>
%\fi
% Some text for part 3:
%    \begin{macrocode}
some text in part three
%    \end{macrocode}

%\iffalse
%</samplepart3>
%\fi
% Some text for part 4:
%\iffalse
%<*samplepart4>
%\fi
%    \begin{macrocode}
more text in part four
%    \end{macrocode}

%\iffalse
%</samplepart4>
%\fi
%
% %%%%%%%%%%%%%%%%%%%%%%%%%%%%%%%%%%%%%%
% \paragraph{Forwarding for a Complete Draft.}
%
% The following forwarding file |cdocsdrf.tex|
% compiles the main document in draft mode:
%\iffalse
%<*sampledraft>
%\fi
%    \begin{macrocode}
\def\version{draft}
\input{childdoc.def}
\childdocforward{cdocsamp}
%    \end{macrocode}

%\iffalse
%</sampledraft>
%\fi
%
% %%%%%%%%%%%%%%%%%%%%%%%%%%%%%%%%%%%%%%
% \paragraph{Forwarding for Final Version of the Chapters.}
%
% The following forwarding files |cdocsfn1.tex| and |cdocsfn2.tex|
% (with identical content)
% compile the final versions of the child documents
% |cdocsch1.tex| and |cdocsch2.tex|, respectively:
%\iffalse
%<*samplefinal>
%\fi
%    \begin{macrocode}
\def\version{final}
\input{childdoc.def}
\childdocforwardprefix[cdocsamp]{cdocsfn}{cdocsch}
%    \end{macrocode}

%\iffalse
%</samplefinal>
%\fi
%
% %%%%%%%%%%%%%%%%%%%%%%%%%%%%%%%%%%%%%%
% \paragraph{Command Line Processing.}
%
% The following three command lines generate the output files
% |cdocscld|, |cdocscl1| and |cdocscl2|
% which should be identical to
% |cdocsdrf|, |cdocsch1| and |cdocsfn2|, respectively:
% \begin{center}
% \begin{tabular}{l}
% |latex -jobname cdocscld \|\\
% |  "\def\version{draft}\input{childdoc.def}\childdocforward{cdocsamp}"|\\
% |latex -jobname cdocscl1 \|\\
% |  "\input{childdoc.def}\childdocforward[cdocsamp]{cdocsch1}"|\\
% |latex -jobname cdocscl2 \|\\
% |  "\def\version{final}\input{childdoc.def}\childdocforward{cdocsch2}"|
% \end{tabular}
% \end{center}
% Note that the trailing backslash on each first line
% merely continues the input to the second line
% (for convenient cut ant paste).
% Furthermore, the command |latex| can be replaced by any
% of its alternative versions such as |pdflatex|.
%
% %%%%%%%%%%%%%%%%%%%%%%%%%%%%%%%%%%%%%%%%%%%%%%%%%%%%%%%%%%%%%%%%%%%%%%%%%%%%%%
% %%%%%%%%%%%%%%%%%%%%%%%%%%%%%%%%%%%%%%%%%%%%%%%%%%%%%%%%%%%%%%%%%%%%%%%%%%%%%%
% \section{Implementation}
%\iffalse
%<*package>
%\fi
%
% This section describes the definitions file |childdoc.def|.

% The definitions cannot be loaded using |\usepackage| or |\RequirePackage|
% which has a mechanism to prevent loading a style file more than once.
% When loading the definitions by means of |\input|
% multiple instances have to be prevented manually:
%\iffalse
%This code needs to be before the `\ProvidesFile' directive
%which is defined at the beginning of this file.
%Therefore it is also placed there and commented out here.
%</package>
%<*discard>
%\fi
%    \begin{macrocode}
\ifdefined\childdocmain\endinput\fi
%    \end{macrocode}
%\iffalse
%</discard>
%<*package>
%\fi
%
% \macro{\ifchilddoc}
% \macro{\ifchilddocmanual}
% The conditional |\ifchilddoc| tells whether a
% child (true) or main (false) document is being compiled.
% The conditional |\ifchilddocmanual| tells whether
% the |\includeonly| mechanism is used (false) or
% the selection of child files must be performed manually (true).
% The definitions initialise to false:
%    \begin{macrocode}
\newif\ifchilddoc
\newif\ifchilddocmanual
%    \end{macrocode}

% \macro{\childdocname}
% \macro{\childdocjob}
% The macro |\childdocname| stores the name of the main document
% to be compiled. The macro |\childdocjob| stores the name of
% the document on which the \LaTeX{} compiler was originally invoked.
% The content of |\jobname| cannot be compared
% to filenames specified in the source due to different catcodes.
% The following code rescans |\jobname|, stores the result
% in |\childdocname| and saves a copy in |\childdocjob|:
%    \begin{macrocode}
\edef\childdocname{\scantokens\expandafter{\jobname\noexpand}}
\let\childdocjob\childdocname
%    \end{macrocode}

% \macro{\childdocdisable}
% The macro |\childdocdisable| prevents the main file
% from being processed more than once.
% At this stage, the main document command |\childdocmain|
% is assumed to be called once again where it should do nothing.
% Any subsequent call to it should prevent
% a secondary processing of the main document
% It overwrites the forwarding commands
% |\childdocof| and |\childdocforward|
% with empty macros to prevent further inclusions of the main document:
%    \begin{macrocode}
\newcommand{\childdocdisable}
{
  \renewcommand{\childdocmain}[1]{\renewcommand{\childdocmain}[1]{\endinput}}
  \renewcommand{\childdocof}[1]{}
  \renewcommand{\childdocby}[2][]{}
  \renewcommand{\childdocforward}[2][]{}
  \renewcommand{\childdocdisable}{}
}
%    \end{macrocode}

% \macro{\childdocmain}
% The macro |\childdocmain| is to be called at the top of the main file
% with nothing or the main filename (without extension) as argument.
% First, it breaks loops.
% If the argument is not empty and does not match |\childdocname|
% (which is set by the first inclusion of |childdoc.def|),
% |\ifchilddoc| is set to true, |\includeonly| is applied to the child file
% and |\jobname| is set to the main file
% (for proper handling of |.aux| files):
%    \begin{macrocode}
\newcommand{\childdocmain}[1]
{
  \childdocdisable\childdocmain{}
  \if?#1?\else
    \begingroup
      \def\childdoctmp{#1}
      \ifx\childdoctmp\childdocname
        \def\childdoctmp{}
      \else
        \def\childdoctmp
        {
          \childdoctrue
          \includeonly{\childdocname}
          \def\childdocjob{#1}
          \def\jobname{#1}
        }
      \fi
      \expandafter
    \endgroup
    \childdoctmp
  \fi
}
%    \end{macrocode}

% \macro{\childdocof}
% The command |\childdocof| redirects
% compilation to the main file |#1|.
%    \begin{macrocode}
\newcommand{\childdocof}[1]
{
  \childdocdisable
  \childdoctrue
  \includeonly{\childdocname}
  \def\jobname{#1}
  \def\childdocjob{#1}
  \input{#1}
}
%    \end{macrocode}

% \macro{\childdocby}
% The command |\childdocby| ....
%    \begin{macrocode}
\newcommand{\childdocby}[2][]
{
  \childdocdisable
  \childdoctrue
  \childdocmanualtrue
  \if?#1?\else
    \def\jobname{#2}
  \fi
  \def\childdocjob{#2}
  \input{#2}
  \endinput
}
%    \end{macrocode}

% \macro{\childdocforward}
% The command |\childdocforward| redirects
% compilation to the main file or
% (if the optional argument is given) a child file.
% Parameters are set as if the main file
% or a child file starting with |\childdocof| was compiled.
% Then compilation is handed over to the main file:
%    \begin{macrocode}
\newcommand{\childdocforward}[2][]
{
  \begingroup
    \if?#1?
      \def\childdoctmp
      {
        \def\childdocname{#2}
        \def\childdocjob{#2}
        \def\jobname{#2}
        \input{#2}
        \endinput
      }
    \else
      \def\childdoctmp
      {
        \childdocdisable
        \def\childdocname{#2}
        \childdoctrue
        \includeonly{#2}
        \def\childdocjob{#1}
        \def\jobname{#1}
        \input{#1}
        \endinput
      }
    \fi
    \expandafter
  \endgroup
  \childdoctmp
}
%    \end{macrocode}

% \macro{\childdocforwardprefix}
% The command |\childdocforwardprefix| redirects
% compilation to the main or a child file by means of a pattern.
% The prefix |#1| in the current filename is replaced by |#2|
% and the suffix of the current filename is kept
% (it is assumed that the filename does not contain the substring `|~~~|'
% which is used as a delimiter).
% Compilation is handed over to the new file by |\childdocforward|:
%    \begin{macrocode}
\newcommand{\childdocforwardprefix}[3][]
{
  \begingroup
    \def\childdocextract #2##1~~~{\def\childdoctmp{\childdocforward[#1]{#3##1}}}
    \expandafter\childdocextract\childdocname~~~
    \expandafter
  \endgroup
  \childdoctmp
}
%    \end{macrocode}

% \macro{\childdoc}
% The deprecated macro |\childdoc| is a legacy version of |\childdocmain|:
%    \begin{macrocode}
\newcommand{\childdoc}{\childdocmain}
%    \end{macrocode}

% \macro{\childdocredirect}
% The deprecated macro |\childdocredirect| is a legacy version
% of |\childdocforward| and |\childdocforwardprefix|:
%    \begin{macrocode}
\newcommand{\childdocredirect}[2][]
{
  \begingroup
    \if?#1?
      \def\childdoctmp{\childdocforward{#2}}
    \else
      \def\childdoctmp{\childdocforwardprefix{#1}{#2}}
    \fi
    \expandafter
  \endgroup
  \childdoctmp
}
%    \end{macrocode}

%\iffalse
%</package>
%\fi
%
\endinput
|\\
|\childdocby{|\textit{main}|}|\\
\end{tabular}
\end{center}
%
The directive |\childdocby| is similar to |\childdocof|
described in \secref{sec:include},
but the subsequent selection of content must be done manually.
To that end, both |\ifchilddoc| and |\ifchilddocmanual|
will be true upon processing of a part,
and the name of the part is stored in |\childdocname|.
Note that |\jobname| will be set to the filename of the current part
so that each part receives an individual |.aux| file
that does not interfere with the |.aux| file(s) of the main document.
This behaviour can be altered by the alternative form
|\childdocby[*]{|\textit{main}|}| (with a non-empty optional argument)
which uses the |.aux| file of the main document
by setting |\jobname| to \textit{main}.

%%%%%%%%%%%%%%%%%%%%%%%%%%%%%%%%%%%%%%%%%%%%%%%%%%%%%%%%%%%%%%%%%%%%%%%%%%%%%%%%
\subsection{Driver Development}
\label{sec:driver}

The \textsf{childdoc} mechanism can also be use for the development
of definition files such as \LaTeX{} styles or classes.
This case differs from the above setup with multiple parts
included by |\include| in that no |\includeonly| should be invoked.
This can be achieved by starting the include file
(before |\ProvidesPackage|) with:
%
\begin{center}
\begin{tabular}{l}
|% \iffalse
%
% childdoc.dtx Copyright (C) 2017-2018 Niklas Beisert
%
% This work may be distributed and/or modified under the
% conditions of the LaTeX Project Public License, either version 1.3
% of this license or (at your option) any later version.
% The latest version of this license is in
%   http://www.latex-project.org/lppl.txt
% and version 1.3 or later is part of all distributions of LaTeX
% version 2005/12/01 or later.
%
% This work has the LPPL maintenance status `maintained'.
%
% The Current Maintainer of this work is Niklas Beisert.
%
% This work consists of the files childdoc.dtx and childdoc.ins
% and the derived files childdoc.def and cdocsamp.tex with
% cdocsch1.tex, cdocsch2.tex, cdocsdrf.tex, cdocsfn1.tex, cdocsfn2.tex.
%
%<package>\ifdefined\childdocmain\endinput\fi
%<package>\ProvidesFile{childdoc.def}[2018/12/30 v2.0 child document driver]
%<samplemain>\ProvidesFile{cdocsamp.tex}[2018/12/30 v2.0 sample for childdoc]
%<*driver>
%\ProvidesFile{childdoc.drv}[2018/12/30 v2.0 childdoc reference manual file]
\PassOptionsToClass{10pt,a4paper}{article}
\documentclass{ltxdoc}

\usepackage[margin=35mm]{geometry}
\usepackage{hyperref}
\usepackage{hyperxmp}
\usepackage[usenames]{color}

\hypersetup{colorlinks=true}
\hypersetup{pdfstartview=FitH}
\hypersetup{pdfpagemode=UseNone}
\hypersetup{pdfsource={}}
\hypersetup{pdflang={en-UK}}
\hypersetup{pdfcopyright={Copyright 2017-2018 Niklas Beisert.
  This work may be distributed and/or modified under the
  conditions of the LaTeX Project Public License, either version 1.3
  of this license or (at your option) any later version.}}
\hypersetup{pdflicenseurl={http://www.latex-project.org/lppl.txt}}
\hypersetup{pdfcontactaddress={ETH Zurich, ITP, HIT K,
  Wolfgang-Pauli-Strasse 27}}
\hypersetup{pdfcontactpostcode={8093}}
\hypersetup{pdfcontactcity={Zurich}}
\hypersetup{pdfcontactcountry={Switzerland}}
\hypersetup{pdfcontactemail={nbeisert@itp.phys.ethz.ch}}
\hypersetup{pdfcontacturl={http://people.phys.ethz.ch/\xmptilde nbeisert/}}

\newcommand{\secref}[1]{\hyperref[#1]{section \ref*{#1}}}

\parskip1ex
\parindent0pt
\let\olditemize\itemize
\def\itemize{\olditemize\parskip0pt}

\begin{document}

\title{The \textsf{childdoc} Package}
\hypersetup{pdftitle={The childdoc Package}}
\author{Niklas Beisert\\[2ex]
  Institut f\"ur Theoretische Physik\\
  Eidgen\"ossische Technische Hochschule Z\"urich\\
  Wolfgang-Pauli-Strasse 27, 8093 Z\"urich, Switzerland\\[1ex]
  \href{mailto:nbeisert@itp.phys.ethz.ch}
  {\texttt{nbeisert@itp.phys.ethz.ch}}}
\hypersetup{pdfauthor={Niklas Beisert}}
\hypersetup{pdfsubject={Manual for the LaTeX2e Package childdoc}}
\date{30 December 2018, \textsf{v2.0}}
\maketitle

\begin{abstract}\noindent
\textsf{childdoc} is a \LaTeXe{} package
that enables the direct compilation
of document sections included by |\include|
to individual files.
\end{abstract}

\begingroup
\parskip0ex
\tableofcontents
\endgroup

%%%%%%%%%%%%%%%%%%%%%%%%%%%%%%%%%%%%%%%%%%%%%%%%%%%%%%%%%%%%%%%%%%%%%%%%%%%%%%%%
%%%%%%%%%%%%%%%%%%%%%%%%%%%%%%%%%%%%%%%%%%%%%%%%%%%%%%%%%%%%%%%%%%%%%%%%%%%%%%%%
\section{Introduction}

\LaTeX{} provides a mechanism to structure a large document (such as a book)
into a main file and several child files (containing the chapters)
using the |\include| command.
This mechanism is beneficial for documents
which span hundreds of pages in order to
make the source file(s) more manageable.
Moreover, compilation can be restricted to
selected child files by means of the |\includeonly| command.
The latter feature can be used to reduce the compilation time while editing
(this was significantly more useful in the earlier days of \LaTeX{})
or to generate a smaller document which is easier to navigate.
Another application of |\includeonly| is to generate
documents consisting of selected parts of the complete document.

However, there are a few drawbacks of the plain |\include| mechanism:
\begin{itemize}
\item
The child files cannot be compiled on their own,
they can only be compiled via the main file.
A naive editing environment
(such as a text editor with an option
to have the current file processed by \LaTeX)
may require one to switch to the main file before compiling;
attempting to compile the child file produces errors.
\item
The main file must be modified (each time)
to adjust the |\includeonly| command
to the present needs. This easily leaves the main file in a messy state.
\item
The generated document will always carry the filename
of the main document. This is inconvenient if
several child files are to be compiled and
to be kept for distribution.
\end{itemize}

The present package provides a simple interface
to make child files individually compilable by \LaTeX{}.
Compiling a child file then has the same effect as compiling
the main file with an |\includeonly| command
to select the appropriate child.
Moreover the generated document will carry the name of the child
rather than the main file.
This resolves all three above issues.

This feature is meant to make the editing of books,
thesis documents and lecture notes somewhat more convenient.
However, the package can also be used efficiently for
composing a series of documents (such as exercise sheets)
which are typically distributed individually.
It then assists the author in generating the individual documents
(potentially in different versions)
as well as a document containing the collected series.
Another application is in developing style files
or other kinds of included material
where compilation of the style file could redirect
to a sample or test file.

%%%%%%%%%%%%%%%%%%%%%%%%%%%%%%%%%%%%%%%%%%%%%%%%%%%%%%%%%%%%%%%%%%%%%%%%%%%%%%%%
%%%%%%%%%%%%%%%%%%%%%%%%%%%%%%%%%%%%%%%%%%%%%%%%%%%%%%%%%%%%%%%%%%%%%%%%%%%%%%%%
\section{Usage}

First of all, the package \textsf{childdoc} is \emph{not} a standard
\LaTeXe{} |.sty| style file! Therefore it needs to be invoked in
a non-standard way.

%%%%%%%%%%%%%%%%%%%%%%%%%%%%%%%%%%%%%%%%%%%%%%%%%%%%%%%%%%%%%%%%%%%%%%%%%%%%%%%%
\subsection{Included Files}
\label{sec:include}

%%%%%%%%%%%%%%%%%%%%%%%%%%%%%%%%%%%%%%%%
\DescribeMacro{\childdocmain}
To use the package, add the commands
\begin{center}
\begin{tabular}{l}
|\input{childdoc.def}|\\
|\childdocmain{}|\\
\end{tabular}
\end{center}
at the very top of the main \LaTeX{} file,
in particular \emph{before} the |\documentclass| statement!
The argument of |\childdocmain| should be left empty
(but it must be present).

%%%%%%%%%%%%%%%%%%%%%%%%%%%%%%%%%%%%%%%%
\DescribeMacro{\childdocof}
Furthermore, add the commands
\begin{center}
\begin{tabular}{l}
|\input{childdoc.def}|\\
|\childdocof{|\textit{main}|}|\\
\end{tabular}
\end{center}
at the top of every child file \textit{child}
which is included by |\include{|\textit{child}|}|
from within the main file
(or at least for those files to be compiled individually).
The argument \textit{main} must be the filename of the main file.

There are a couple of
considerations in setting up the main and child documents:

%%%%%%%%%%%%%%%%%%%%%%%%%%%%%%%%%%%%%%%%
\paragraph{Restrictions.}

Please note the following restrictions:
\begin{itemize}
\item
|\childdocmain| must be called with one argument \textit{main}
to ensure compatibility with earlier version of the package.
It must either be empty (|\childdocmain{}|)
or precisely match the filename of the main file in which it is specified.
See \secref{sec:detection} for further information.
\item
The filename \textit{main} must be specified without the |.tex| extension.
\item
The filename \textit{main} is case sensitive
(even in case-insensitive file systems)
due to internal string comparison.
\item
The argument \textit{main} should be fully expanded, it cannot be a macro.
\item
Subdirectories and special characters should be avoided in filenames.
\item
The command |\childdocmain{|\textit{main}|}| must be followed by a whitespace.
It should not be followed immediately by another command
or by a comment mark `|%|'.
This is because the \TeX{} parser reads the token immediately following
the argument of |\childdocmain| and puts it
at the beginning of every child section;
however, a white\-space is ignored.
\end{itemize}

%%%%%%%%%%%%%%%%%%%%%%%%%%%%%%%%%%%%%%%%
\paragraph{Content of Main File.}

It is advisable to place all content in the child files included by |\include|.
Any output contained in the main file will appear in all child documents
unless suppressed manually;
it cannot be suppressed automatically by the |\includeonly| directive
and thus should normally be avoided.
A method to include some content in the main file
by means of conditional processing is described in \secref{sec:conditional}.

%%%%%%%%%%%%%%%%%%%%%%%%%%%%%%%%%%%%%%%%
\paragraph{Page Numbering.}

When only a part of the document is compiled,
the appropriate numbering of pages
(as well as other status parameters)
is determined from the |.aux| files.
The latter contain information from previous passes.
However this information needs to propagate through
all intermediate child documents.
Therefore the page numbering in child documents may well
be inconsistent until the complete document is compiled at least once.

A useful (if unconventional) way to always ensure a consistent
page numbering is to restart the numbering in each child document
and denote the pages by `\textit{child}|.|\textit{page}'
where \textit{child} represents the chapter/section number of the child file.
This can be achieved by the command
|\numberwithin{page}{|\textit{child}|}|
of the \textsf{amsmath} package
where \textit{child} can be |chapter| or |section|
depending on the chosen structuring.
Alternatively, one can modify the macro |\thepage| appropriately
and reset the counter |page| at the start of each child file.

%%%%%%%%%%%%%%%%%%%%%%%%%%%%%%%%%%%%%%%%%%%%%%%%%%%%%%%%%%%%%%%%%%%%%%%%%%%%%%%%
\subsection{Conditional Processing}
\label{sec:conditional}

The package provides a mechanism to compile different versions
of a document. To customise the versions further some conditional processing
can come in handy to distinguish which version is being compiled.
The package provides two macros to describe the compilation context:

%%%%%%%%%%%%%%%%%%%%%%%%%%%%%%%%%%%%%%%%
\DescribeMacro{\ifchilddoc}
The conditional |\ifchilddoc| distinguishes between the compilation of
child documents and the main document:
%
\begin{center}
|\ifchilddoc |\textit{child-code}| |[|\||else |\textit{main-code}]| \||fi|
\end{center}

%%%%%%%%%%%%%%%%%%%%%%%%%%%%%%%%%%%%%%%%
\DescribeMacro{\childdocname}
\DescribeMacro{\childdocjob}
The macro |\childdocname| contains the filename (without extension)
of the main or child file being processed.
Note that |\childdocjob| will always contain the name of the main file.

%%%%%%%%%%%%%%%%%%%%%%%%%%%%%%%%%%%%%%%%
\paragraph{Title Page.}

Conditional processing can be used to include a title or banner page
in the main document when proper precautions are taken.
Importantly, the code in the main file should ensure that the page counter
(as well as other status parameters which are stored in the |.aux| files)
takes the same value after the conditional processing.
Otherwise the page numbers may take divergent values
depending on which part is compiled.

For example, a title page could be declared by:
%
\begin{center}
\begin{tabular}{l}
|\ifchilddoc\||else|\\
|\addtocounter{page}{-1}|\\
\textit{code for title page}\\
|\newpage|\\
|\||fi|
\end{tabular}
\end{center}
%
A banner page for the child documents can be generated by:
%
\begin{center}
\begin{tabular}{l}
|\ifchilddoc|\\
|\addtocounter{page}{-1}|\\
\textit{code for banner page}\\
|\newpage|\\
|\||fi|
\end{tabular}
\end{center}
%
Here one could write a message such as:
\begin{center}
|This is the part \childdocname{} of \childdocjob{}.|
\end{center}

%%%%%%%%%%%%%%%%%%%%%%%%%%%%%%%%%%%%%%%%%%%%%%%%%%%%%%%%%%%%%%%%%%%%%%%%%%%%%%%%
\subsection{Flags}
\label{sec:flags}

The package makes it easy to generate different versions
of the main or child documents.
To this end compilation flags can be defined
and assigned different default values.
They will be particularly useful in conjunction
with the forwarding mechanism described in \secref{sec:forward}.

For example, it may be useful to have a flag |\version|
which can be set to |draft| or |final|.
The document source will contain some conditional code
depending on the value of |\version|.
Suppose further, the flag should default to |final| for the main file
and to |draft| for child files
which is a natural assignment for editing the document.
This is achieved by placing the following code
in the preamble of the main document
(below the |\childdocmain| directive):
%
\begin{center}
\begin{tabular}{l}
|\ifchilddoc|\\
|\providecommand{\version}{draft}|\\
|\||else|\\
|\providecommand{\version}{final}|\\
|\||fi|
\end{tabular}
\end{center}
%
The definition by |\providecommand| makes sure
that previous definitions are not overwritten.
Further statements |\providecommand{\version}{...}|
can thus be added before the above code to override it.

For the main file, one might add a line
(between |\childdocmain| and the above block)
%
\begin{center}
|%\ifchilddoc\||else\providecommand{\version}{draft}\||fi|
\end{center}
%
which can be uncommented to produce a draft version.
Likewise one can add a line to the very top of a child file
(above the |\childdocof{|\textit{main}|}| directive)
%
\begin{center}
|%\providecommand{\version}{final}|
\end{center}
%
which can be uncommented to produce the final version of this child document.

%%%%%%%%%%%%%%%%%%%%%%%%%%%%%%%%%%%%%%%%%%%%%%%%%%%%%%%%%%%%%%%%%%%%%%%%%%%%%%%%
\subsection{Forwarding}
\label{sec:forward}

Different versions of the main or child documents
using compilation flags as described in \secref{sec:flags}
can be (permanently) stored in different files
for convenient compilation, viewing and distribution.
To this end, the package defines a command
to pass on compilation to a different file:

%%%%%%%%%%%%%%%%%%%%%%%%%%%%%%%%%%%%%%%%
\DescribeMacro{\childdocforward}
The command |\childdocforward| redirects processing to
another source file:
%
\begin{center}
\begin{tabular}{l}
|\input{childdoc.def}|\\
|\childdocforward[|\textit{main}|]{|\textit{dest}|}|\\
\end{tabular}
\end{center}
%
The argument \textit{dest} is the destination file
(without extension).
It should be the main file or one of the child files.
Note that further \textsf{childdoc} directives
such as |\childdocof| and |\childdocforward|
in the indicated file will be processed in this form.
The optional argument \textit{main}
passes on directly to the main file \textit{main}
while pretending to compile the child \textit{dest}.
This form behaves as if \textit{dest}
issues |\childdocof{|\textit{main}|}| right away,
and no further \textsf{childdoc} directives will be processed.

%%%%%%%%%%%%%%%%%%%%%%%%%%%%%%%%%%%%%%%%
\DescribeMacro{\...prefix}
In the alternative form |\childdocforwardprefix|,
%
\begin{center}
\begin{tabular}{l}
|\input{childdoc.def}|\\
|\childdocforwardprefix[|\textit{main}|]{|\textit{prefix}|}{|\textit{dest}|}|
\end{tabular}
\end{center}
%
the destination file is determined by a pattern
depending on the current file:
To make this work, the current file must be called
`{\textit{prefix}\hspace{0.2em}\textit{suffix}}'
with \textit{prefix} matching precisely the argument.
Processing is then passed on to the file
`{\textit{dest}\hspace{0.2em}\textit{suffix}}'.
Surely, the same effect is achieved by
directly specifying the
argument `{\textit{dest}\hspace{0.2em}\textit{suffix}}'
in the first form.
However, that requires to set up a different file
for each child. With the alternative form of the command
all these files can have exactly the same content
which simplifies setting them up and maintaining them.

For example, the following file |draft.tex|
with a compilation flag |\version| as described in \secref{sec:flags}
compiles the main document as a draft:
%
\begin{center}
\begin{tabular}{l}
|\def\version{draft}|\\
|\input{childdoc.def}|\\
|\childdocforward{|\textit{main}|}|
\end{tabular}
\end{center}
%
Likewise, the following files |final|\textit{nn}|.tex|
compile the final version of the child document
|child|\textit{nn}|.tex|:
%
\begin{center}
\begin{tabular}{l}
|\def\version{final}|\\
|\input{childdoc.def}|\\
|\childdocforwardprefix{final}{child}|
\end{tabular}
\end{center}
%

Note that when several versions of a main file and/or of each child file
are to be generated, it may be convenient to set up a |Makefile| or
shell script to automatise the process.

%%%%%%%%%%%%%%%%%%%%%%%%%%%%%%%%%%%%%%%%%%%%%%%%%%%%%%%%%%%%%%%%%%%%%%%%%%%%%%%%
\subsection{Command Line Processing}
\label{sec:commandline}

The effect of redirection files can also be achieved by invoking
the \LaTeX{} compiler with a more elaborate command line.
Most conveniently this should be done as part
of a shell script or a |Makefile|.

When using \textsf{childdoc} in the main file, the following
command lines effectively perform a redirection
(note that depending on the shell being used,
backslashes may have to be doubled: `|\|' $\to$ `|\\|'):
%
\begin{center}
|... -jobname "|\textit{target}|" |\\|"|[\textit{flags}]%
|\input{childdoc.def}\childdocforward[|\textit{main}|]{|\textit{dest}|}"|
\end{center}
%
Here \textit{target} is the name of the output file,
\textit{main} is the name of the main file
and \textit{dest} is the name of the main or child file to be processed
(all filenames without extensions).
The optional argument \textit{main} can be omitted
if \textit{main} matches \textit{dest}.
Optionally, compilation \textit{flags} can be defined via |\def| commands.
This command line makes the \TeX{} engine believe
it is compiling the file \textit{target}
whose content is specified as the latter parameter.
The provided code then forwards the processing to
\textit{main} or \textit{dest} as described in \secref{sec:forward}.

%%%%%%%%%%%%%%%%%%%%%%%%%%%%%%%%%%%%%%%%%%%%%%%%%%%%%%%%%%%%%%%%%%%%%%%%%%%%%%%%
\subsection{Include by Input}
\label{sec:input}

Including child documents by |\include| has some restrictions by design.
Most notably, the content of a child document always occupies
its own set of pages; pages cannot be shared between child documents.
Usually, this behaviour makes perfect sense
because each child document contain an essential part of the document.
However, in some situations it may be desirable to compose
a document from a collection of parts
without having mandatory page breaks between then.
For this case, the package
provides a mechanism to include parts
by |\input| which can also be processed individually.
However, by construction this mechanism
requires manual handling of the content to be output.

%%%%%%%%%%%%%%%%%%%%%%%%%%%%%%%%%%%%%%%%
\DescribeMacro{\ifchilddocmanual}
The main file should be prepared as usual, see \secref{sec:include}.
However, the document body must make a distinction
between processing of an individual part and of the main document, e.g.:
%
\begin{center}
\begin{tabular}{l}
|\ifchilddocmanual|\\
|\input{\childdocname}|\\
|\||else|\\
\textit{document body with }|\input{|\textit{part}|}|\\
|\||fi|
\end{tabular}
\end{center}
%
The conditional |\ifchilddocmanual| is true whenever
a part to be included by |\input| is being compiled,
and the name of the part is stored in |\childdocname|.

%%%%%%%%%%%%%%%%%%%%%%%%%%%%%%%%%%%%%%%%
\DescribeMacro{\childdocby}
Each part to be included by |\input| should start with:
%
\begin{center}
\begin{tabular}{l}
|\input{childdoc.def}|\\
|\childdocby{|\textit{main}|}|\\
\end{tabular}
\end{center}
%
The directive |\childdocby| is similar to |\childdocof|
described in \secref{sec:include},
but the subsequent selection of content must be done manually.
To that end, both |\ifchilddoc| and |\ifchilddocmanual|
will be true upon processing of a part,
and the name of the part is stored in |\childdocname|.
Note that |\jobname| will be set to the filename of the current part
so that each part receives an individual |.aux| file
that does not interfere with the |.aux| file(s) of the main document.
This behaviour can be altered by the alternative form
|\childdocby[*]{|\textit{main}|}| (with a non-empty optional argument)
which uses the |.aux| file of the main document
by setting |\jobname| to \textit{main}.

%%%%%%%%%%%%%%%%%%%%%%%%%%%%%%%%%%%%%%%%%%%%%%%%%%%%%%%%%%%%%%%%%%%%%%%%%%%%%%%%
\subsection{Driver Development}
\label{sec:driver}

The \textsf{childdoc} mechanism can also be use for the development
of definition files such as \LaTeX{} styles or classes.
This case differs from the above setup with multiple parts
included by |\include| in that no |\includeonly| should be invoked.
This can be achieved by starting the include file
(before |\ProvidesPackage|) with:
%
\begin{center}
\begin{tabular}{l}
|\input{childdoc.def}|\\
|\childdocforward{|\textit{main}|}|\\
\end{tabular}
\end{center}
%
or alternatively with:
%
\begin{center}
\begin{tabular}{l}
|\input{childdoc.def}|\\
|\childdocby{|\textit{main}|}|\\
\end{tabular}
\end{center}
%
Both forms have slightly different effects as described above.
The main file is prepared as usual, see \secref{sec:include}.

%%%%%%%%%%%%%%%%%%%%%%%%%%%%%%%%%%%%%%%%%%%%%%%%%%%%%%%%%%%%%%%%%%%%%%%%%%%%%%%%
\subsection{Legacy Detection}
\label{sec:detection}

The directive |\childdocmain| in the main file can detect
whether the complete document or merely a child is to be compiled
even without using the directive |\childdocof|.
This method is deprecated because it is less robust
and there is no compelling reason to use it;
it is merely provided for backward compatibility
and it may be removed in future versions.

If the detection mechanism is to be used,
it is mandatory to correctly specify
the filename of the main file as the argument of |\childdocmain|:
%
\begin{center}
\begin{tabular}{l}
|\input{childdoc.def}|\\
|\childdocmain{|\textit{main}|}|\\
\end{tabular}
\end{center}
%
If |\jobname| does not match the argument \textit{main} of |\childdocmain|,
it is assumed that |\jobname| points to the child file to be compiled.
When using |\childdocmain| with the main file specified as argument,
it suffices to start a child file
with just |\input{|\textit{main}|}|
without loading of the package and using |\childdocof|.
If instead all processing is done
with the appropriate \textsf{childdoc} directives,
the argument of \textit{main} of |\childdocmain| can be empty.

An alternative version of the command line processing described
in \secref{sec:commandline} using the detection mechanism reads:
%
\begin{center}
|... -jobname "|\textit{target}|" "|[\textit{flags}]%
[|\def\jobname{|\textit{dest}|}|]|\input{|\textit{main}|}"|
\end{center}

%%%%%%%%%%%%%%%%%%%%%%%%%%%%%%%%%%%%%%%%%%%%%%%%%%%%%%%%%%%%%%%%%%%%%%%%%%%%%%%%
\subsection{Manual Code}
\label{sec:manual}

In case one cannot be certain whether the definitions file |childdoc.def|
is installed on the target \TeX{} distribution
and one prefers not to ship it,
it is conceivable to paste a few relevant commands into the sources.

To that end, drop all statements |\input{childdoc.def}|
and perform the replacements as outlined below.
Instead of |\childdocmain{|\textit{main}|}| add the following code
to the top of the main file:
%
\begin{center}
\begin{tabular}{l}
|\||ifdefined\childdocname\endinput\||fi\newif\ifchilddoc|\\
|\edef\childdocname{\scantokens\expandafter{\jobname\noexpand}}|\\
|\def\childdocmain{|\textit{main}|}\||ifx\childdocmain\childdocname\||else|\\
|\childdoctrue\includeonly{\childdocname}\let\jobname\childdocmain\||fi|\\
\end{tabular}
\end{center}
%
Instead of |\childdocof{|\textit{main}|}| just include the main file
at the top of each child file:
%
\begin{center}
|\input{|\textit{main}|}|
\end{center}
%
A simple redirection |\childdocforward{|\textit{dest}|}| is achieved by:
%
\begin{center}
|\def\jobname{|\textit{dest}|}\input{\jobname}|
\end{center}
%
The redirection with prefix
|\childdocforwardprefix[|\textit{prefix}|]{|\textit{dest}|}|
is accomplished by:
%
\begin{center}
\begin{tabular}{l}
|{\edef\jobname{\scantokens\expandafter{\jobname\noexpand}}|\\
|\def\redirectjob |\textit{prefix}|#1~~~{\gdef\jobname{|\textit{dest}|#1}}|\\
|\expandafter\redirectjob\jobname~~~}\input{\jobname}|
\end{tabular}
\end{center}

In an alternative approach,
child documents can be compiled by a specific command line
without additional code or specific definitions:
%
\begin{center}
|... -jobname "|\textit{target}|" "|[\textit{flags}]%
|\includeonly{|\textit{dest}|}\input{|\textit{main}|}"|
\end{center}
%

%%%%%%%%%%%%%%%%%%%%%%%%%%%%%%%%%%%%%%%%%%%%%%%%%%%%%%%%%%%%%%%%%%%%%%%%%%%%%%%%
%%%%%%%%%%%%%%%%%%%%%%%%%%%%%%%%%%%%%%%%%%%%%%%%%%%%%%%%%%%%%%%%%%%%%%%%%%%%%%%%
\section{Information}

%%%%%%%%%%%%%%%%%%%%%%%%%%%%%%%%%%%%%%%%%%%%%%%%%%%%%%%%%%%%%%%%%%%%%%%%%%%%%%%%
\subsection{Copyright}

Copyright \copyright{} 2017--2018 Niklas Beisert

This work may be distributed and/or modified under the
conditions of the \LaTeX{} Project Public License, either version 1.3
of this license or (at your option) any later version.
The latest version of this license is in
  \url{http://www.latex-project.org/lppl.txt}
and version 1.3 or later is part of all distributions of \LaTeX{}
version 2005/12/01 or later.

This work has the LPPL maintenance status `maintained'.

The Current Maintainer of this work is Niklas Beisert.

This work consists of the files |README.txt|, |childdoc.ins| and |childdoc.dtx|
as well as the derived files |childdoc.def|, |cdocsamp.tex|
with |cdocsch1.tex|, |cdocsch2.tex|, |cdocspt3.tex|, |cdocspt4.tex|,
|cdocsdrf.tex|, |cdocsfn1.tex|, |cdocsfn2.tex|
as well as |childdoc.pdf|.

%%%%%%%%%%%%%%%%%%%%%%%%%%%%%%%%%%%%%%%%%%%%%%%%%%%%%%%%%%%%%%%%%%%%%%%%%%%%%%%%
\subsection{Files and Installation}

The package consists of the files:
%
\begin{center}
\begin{tabular}{ll}
    |README.txt|   & readme file \\
    |childdoc.ins| & installation file \\
    |childdoc.dtx| & source file \\
    |childdoc.def| & definition file \\
    |cdocsamp.tex| & sample main file \\
    |cdocsch1.tex| & sample include file \\
    |cdocsch2.tex| & sample include file \\
    |cdocspt3.tex| & sample part file \\
    |cdocspt4.tex| & sample part file \\
    |cdocsdrf.tex| & sample redirection file \\
    |cdocsfn1.tex| & sample redirection file \\
    |cdocsfn2.tex| & sample redirection file \\
    |childdoc.pdf| & manual
\end{tabular}
\end{center}
%
The distribution consists of the files
|README.txt|, |childdoc.ins| and |childdoc.dtx|.
%
\begin{itemize}
\item
Run (pdf)\LaTeX{} on |childdoc.dtx|
to compile the manual |childdoc.pdf| (this file).
\item
Run \LaTeX{} on |childdoc.ins| to create the definitions file |childdoc.def|
and the sample |cdocsamp.tex| with include files
|cdocsch1.tex|, |cdocsch2.tex|, |cdocspt3.tex|, |cdocspt4.tex|,
|cdocsdrf.tex|, |cdocsfn1.tex|, |cdocsfn2.tex|.
Then copy the file |childdoc.def| to an appropriate directory of your \LaTeX{}
distribution, e.g.\ \textit{texmf-root}|/tex/latex/childdoc|.
\end{itemize}

%%%%%%%%%%%%%%%%%%%%%%%%%%%%%%%%%%%%%%%%%%%%%%%%%%%%%%%%%%%%%%%%%%%%%%%%%%%%%%%%
\subsection{Related CTAN Packages}

There are several other packages which offer a similar functionality:
%
\begin{itemize}
\item
The packages
\href{http://ctan.org/pkg/docmute}{\textsf{docmute}},
\href{http://ctan.org/pkg/includex}{\textsf{includex}} and
\href{http://ctan.org/pkg/standalone}{\textsf{standalone}}
provide commands to include only the document body of
a child file thus allowing both files to be compiled individually.
\item
The packages \href{http://ctan.org/pkg/subdocs}{\textsf{subdocs}}
and \href{http://ctan.org/pkg/subfiles}{\textsf{subfiles}}
provide structures in which the main and child documents can be
encapsulated and allowing them to be compiled individually.
The inclusion mechanism is different from the conventional |\include|.
\item
The package \href{http://ctan.org/pkg/combine}{\textsf{combine}}
is an elaborate solution to combine several documents into one.
\end{itemize}
%
See also the CTAN topic \href{http://ctan.org/topic/subdocs}{\textsf{subdocs}}
for further related packages.
The present package differs from the above solutions in that
a document structure constructed with the conventional |\include| mechanism
just needs two extra commands at the top of every file
such that all constituent files can be compiled individually.

%%%%%%%%%%%%%%%%%%%%%%%%%%%%%%%%%%%%%%%%%%%%%%%%%%%%%%%%%%%%%%%%%%%%%%%%%%%%%%%%
%\subsection{Feature Suggestions}
%
%The following is a list of features which may be useful for future
%versions of this package:
%%
%\begin{itemize}
%\item
%\ldots
%\end{itemize}

%%%%%%%%%%%%%%%%%%%%%%%%%%%%%%%%%%%%%%%%%%%%%%%%%%%%%%%%%%%%%%%%%%%%%%%%%%%%%%%%
\subsection{Revision History}

%%%%%%%%%%%%%%%%%%%%%%%%%%%%%%%%%%%%%%%%
\paragraph{v2.0:} 2018/12/30

\begin{itemize}
\item
immediate forward processing
\item
added |\childdocby| mechanism
\item
manual restructured
\end{itemize}

%%%%%%%%%%%%%%%%%%%%%%%%%%%%%%%%%%%%%%%%
\paragraph{v1.6:} 2018/01/17

\begin{itemize}
\item
application for development of include files
\item
corrections to manual
\end{itemize}

%%%%%%%%%%%%%%%%%%%%%%%%%%%%%%%%%%%%%%%%
\paragraph{v1.5:} 2017/05/21

\begin{itemize}
\item
more complete structuring introduced
\item
|\childdocof| introduced
\item
|\childdoc| renamed to |\childdocmain|
\item
|\childredirect| renamed to |\childdocforward| and |\childdocforwardprefix|
and functionality expanded
\end{itemize}

%%%%%%%%%%%%%%%%%%%%%%%%%%%%%%%%%%%%%%%%
\paragraph{v1.0:} 2017/04/27

\begin{itemize}
\item
manual and install package
\item
first version published on CTAN
\end{itemize}

%%%%%%%%%%%%%%%%%%%%%%%%%%%%%%%%%%%%%%%%
\paragraph{v0.6:} 2017/04/26

\begin{itemize}
\item
redirection mechanism added
\end{itemize}

%%%%%%%%%%%%%%%%%%%%%%%%%%%%%%%%%%%%%%%%
\paragraph{v0.5:} 2017/04/26

\begin{itemize}
\item
functionality in definition file
\end{itemize}


%%%%%%%%%%%%%%%%%%%%%%%%%%%%%%%%%%%%%%%%%%%%%%%%%%%%%%%%%%%%%%%%%%%%%%%%%%%%%%%%
%%%%%%%%%%%%%%%%%%%%%%%%%%%%%%%%%%%%%%%%%%%%%%%%%%%%%%%%%%%%%%%%%%%%%%%%%%%%%%%%
%%%%%%%%%%%%%%%%%%%%%%%%%%%%%%%%%%%%%%%%%%%%%%%%%%%%%%%%%%%%%%%%%%%%%%%%%%%%%%%%
\appendix

\settowidth\MacroIndent{\rmfamily\scriptsize 000\ }

 \DocInput{childdoc.dtx}

\end{document}
%</driver>
% \fi
%
% %%%%%%%%%%%%%%%%%%%%%%%%%%%%%%%%%%%%%%%%%%%%%%%%%%%%%%%%%%%%%%%%%%%%%%%%%%%%%%
% %%%%%%%%%%%%%%%%%%%%%%%%%%%%%%%%%%%%%%%%%%%%%%%%%%%%%%%%%%%%%%%%%%%%%%%%%%%%%%
% \section{Sample}
%\iffalse
%<*samplemain>
%\fi
%
% The following presents a sample document
% with two chapters, two parts, a title page,
% a compile flag as well as three forwarding files to set the flag.
% It consists of eight |.tex| files:
% \begin{center}
% \begin{tabular}{ll}
% |cdocsamp.tex|&main file\\
% |cdocsch1.tex|&include file for chapter 1\\
% |cdocsch2.tex|&include file for chapter 2\\
% |cdocspt3.tex|&include file for part 3\\
% |cdocspt4.tex|&include file for part 4\\
% |cdocsdrf.tex|&forwarding file for main file in draft mode\\
% |cdocsfi1.tex|&forwarding file for final version of chapter 1\\
% |cdocsfi2.tex|&forwarding file for final version of chapter 2\\
% \end{tabular}
% \end{center}
% Each of the eight files can be compiled directly by the \LaTeX{} compiler.
%
% %%%%%%%%%%%%%%%%%%%%%%%%%%%%%%%%%%%%%%
% \paragraph{Main File.}
%
% The main file is called |cdocsamp.tex|.
%
% Load the \textsf{childdoc} definitions and
% declare the filename for the main document:
%    \begin{macrocode}
\input{childdoc.def}
\childdocmain{}
%    \end{macrocode}

% Optional override for |\version| flag:
%    \begin{macrocode}
%%\ifchilddoc\else\providecommand{\version}{draft}\fi
%    \end{macrocode}

% Define the default values for the |\version| flag
% (|final| for the main file and |draft| for childs):
%    \begin{macrocode}
\ifchilddoc
\providecommand{\version}{draft}
\else
\providecommand{\version}{final}
\fi
%    \end{macrocode}

% Load the standard document class:
%    \begin{macrocode}
\documentclass[12pt]{article}
%    \end{macrocode}

% Start the document body:
%    \begin{macrocode}
\begin{document}
%    \end{macrocode}

% Declare a title page.
% Print title, part of document being processed and version flag:
%    \begin{macrocode}
\addtocounter{page}{-1}
\begin{center}
{\LARGE\bfseries{}childdoc example\par}
\vspace{1cm}
\ifchilddoc
\ifchilddocmanual part\else chapter\fi:
`\childdocname' of `\childdocjob'\par
\else
main document: `\childdocjob'\par
\fi
version: \version\par
\end{center}
\newpage
%    \end{macrocode}

% Manually include selected file,
% otherwise process as usual:
%    \begin{macrocode}
\ifchilddocmanual
\section*{part `\childdocname'}
\input{\childdocname}
\else
%    \end{macrocode}

% Include the two chapters:
%    \begin{macrocode}
\include{cdocsch1}
\include{cdocsch2}
%    \end{macrocode}

% Include the two parts unless only chapters should be displayed:
%    \begin{macrocode}
\ifchilddoc\else
\section{part three}
\input{cdocspt3}
\section{part four}
\input{cdocspt4}
\fi
%    \end{macrocode}

% Process as usual until here:
%    \begin{macrocode}
\fi
%    \end{macrocode}

% End of document body:
%    \begin{macrocode}
\end{document}
%    \end{macrocode}
%\iffalse
%</samplemain>
%\fi
%
% %%%%%%%%%%%%%%%%%%%%%%%%%%%%%%%%%%%%%%
% \paragraph{Chapter Include Files.}
%
% The include files are called |cdocsch1.tex| and |cdocsch2.tex|.
%
%\iffalse
%<*samplechap1|samplechap2>
%\fi

% Optional override for |\version| flag:
%    \begin{macrocode}
%%\providecommand{\version}{final}
%    \end{macrocode}

% Include the main document:
%    \begin{macrocode}
\input{childdoc.def}
\childdocof{cdocsamp}
%    \end{macrocode}

%\iffalse
%</samplechap1|samplechap2>
%\fi
%
%\iffalse
%<*samplechap1>
%\fi
% Some text for chapter 1:
%    \begin{macrocode}
\section{one}
some text in chapter one
%    \end{macrocode}

%\iffalse
%</samplechap1>
%\fi
% Some text for chapter 2:
%\iffalse
%<*samplechap2>
%\fi
%    \begin{macrocode}
\section{two}
more text in chapter two
%    \end{macrocode}

%\iffalse
%</samplechap2>
%\fi
%
% %%%%%%%%%%%%%%%%%%%%%%%%%%%%%%%%%%%%%%
% \paragraph{Part Include Files.}
%
% The include files are called |cdocspt3.tex| and |cdocspt4.tex|.
%
%\iffalse
%<*samplepart3|samplepart4>
%\fi

% Optional override for |\version| flag:
%    \begin{macrocode}
%%\providecommand{\version}{final}
%    \end{macrocode}

% Include the main document:
%    \begin{macrocode}
\input{childdoc.def}
\childdocby{cdocsamp}
%    \end{macrocode}

%\iffalse
%</samplepart3|samplepart4>
%\fi
%
%\iffalse
%<*samplepart3>
%\fi
% Some text for part 3:
%    \begin{macrocode}
some text in part three
%    \end{macrocode}

%\iffalse
%</samplepart3>
%\fi
% Some text for part 4:
%\iffalse
%<*samplepart4>
%\fi
%    \begin{macrocode}
more text in part four
%    \end{macrocode}

%\iffalse
%</samplepart4>
%\fi
%
% %%%%%%%%%%%%%%%%%%%%%%%%%%%%%%%%%%%%%%
% \paragraph{Forwarding for a Complete Draft.}
%
% The following forwarding file |cdocsdrf.tex|
% compiles the main document in draft mode:
%\iffalse
%<*sampledraft>
%\fi
%    \begin{macrocode}
\def\version{draft}
\input{childdoc.def}
\childdocforward{cdocsamp}
%    \end{macrocode}

%\iffalse
%</sampledraft>
%\fi
%
% %%%%%%%%%%%%%%%%%%%%%%%%%%%%%%%%%%%%%%
% \paragraph{Forwarding for Final Version of the Chapters.}
%
% The following forwarding files |cdocsfn1.tex| and |cdocsfn2.tex|
% (with identical content)
% compile the final versions of the child documents
% |cdocsch1.tex| and |cdocsch2.tex|, respectively:
%\iffalse
%<*samplefinal>
%\fi
%    \begin{macrocode}
\def\version{final}
\input{childdoc.def}
\childdocforwardprefix[cdocsamp]{cdocsfn}{cdocsch}
%    \end{macrocode}

%\iffalse
%</samplefinal>
%\fi
%
% %%%%%%%%%%%%%%%%%%%%%%%%%%%%%%%%%%%%%%
% \paragraph{Command Line Processing.}
%
% The following three command lines generate the output files
% |cdocscld|, |cdocscl1| and |cdocscl2|
% which should be identical to
% |cdocsdrf|, |cdocsch1| and |cdocsfn2|, respectively:
% \begin{center}
% \begin{tabular}{l}
% |latex -jobname cdocscld \|\\
% |  "\def\version{draft}\input{childdoc.def}\childdocforward{cdocsamp}"|\\
% |latex -jobname cdocscl1 \|\\
% |  "\input{childdoc.def}\childdocforward[cdocsamp]{cdocsch1}"|\\
% |latex -jobname cdocscl2 \|\\
% |  "\def\version{final}\input{childdoc.def}\childdocforward{cdocsch2}"|
% \end{tabular}
% \end{center}
% Note that the trailing backslash on each first line
% merely continues the input to the second line
% (for convenient cut ant paste).
% Furthermore, the command |latex| can be replaced by any
% of its alternative versions such as |pdflatex|.
%
% %%%%%%%%%%%%%%%%%%%%%%%%%%%%%%%%%%%%%%%%%%%%%%%%%%%%%%%%%%%%%%%%%%%%%%%%%%%%%%
% %%%%%%%%%%%%%%%%%%%%%%%%%%%%%%%%%%%%%%%%%%%%%%%%%%%%%%%%%%%%%%%%%%%%%%%%%%%%%%
% \section{Implementation}
%\iffalse
%<*package>
%\fi
%
% This section describes the definitions file |childdoc.def|.

% The definitions cannot be loaded using |\usepackage| or |\RequirePackage|
% which has a mechanism to prevent loading a style file more than once.
% When loading the definitions by means of |\input|
% multiple instances have to be prevented manually:
%\iffalse
%This code needs to be before the `\ProvidesFile' directive
%which is defined at the beginning of this file.
%Therefore it is also placed there and commented out here.
%</package>
%<*discard>
%\fi
%    \begin{macrocode}
\ifdefined\childdocmain\endinput\fi
%    \end{macrocode}
%\iffalse
%</discard>
%<*package>
%\fi
%
% \macro{\ifchilddoc}
% \macro{\ifchilddocmanual}
% The conditional |\ifchilddoc| tells whether a
% child (true) or main (false) document is being compiled.
% The conditional |\ifchilddocmanual| tells whether
% the |\includeonly| mechanism is used (false) or
% the selection of child files must be performed manually (true).
% The definitions initialise to false:
%    \begin{macrocode}
\newif\ifchilddoc
\newif\ifchilddocmanual
%    \end{macrocode}

% \macro{\childdocname}
% \macro{\childdocjob}
% The macro |\childdocname| stores the name of the main document
% to be compiled. The macro |\childdocjob| stores the name of
% the document on which the \LaTeX{} compiler was originally invoked.
% The content of |\jobname| cannot be compared
% to filenames specified in the source due to different catcodes.
% The following code rescans |\jobname|, stores the result
% in |\childdocname| and saves a copy in |\childdocjob|:
%    \begin{macrocode}
\edef\childdocname{\scantokens\expandafter{\jobname\noexpand}}
\let\childdocjob\childdocname
%    \end{macrocode}

% \macro{\childdocdisable}
% The macro |\childdocdisable| prevents the main file
% from being processed more than once.
% At this stage, the main document command |\childdocmain|
% is assumed to be called once again where it should do nothing.
% Any subsequent call to it should prevent
% a secondary processing of the main document
% It overwrites the forwarding commands
% |\childdocof| and |\childdocforward|
% with empty macros to prevent further inclusions of the main document:
%    \begin{macrocode}
\newcommand{\childdocdisable}
{
  \renewcommand{\childdocmain}[1]{\renewcommand{\childdocmain}[1]{\endinput}}
  \renewcommand{\childdocof}[1]{}
  \renewcommand{\childdocby}[2][]{}
  \renewcommand{\childdocforward}[2][]{}
  \renewcommand{\childdocdisable}{}
}
%    \end{macrocode}

% \macro{\childdocmain}
% The macro |\childdocmain| is to be called at the top of the main file
% with nothing or the main filename (without extension) as argument.
% First, it breaks loops.
% If the argument is not empty and does not match |\childdocname|
% (which is set by the first inclusion of |childdoc.def|),
% |\ifchilddoc| is set to true, |\includeonly| is applied to the child file
% and |\jobname| is set to the main file
% (for proper handling of |.aux| files):
%    \begin{macrocode}
\newcommand{\childdocmain}[1]
{
  \childdocdisable\childdocmain{}
  \if?#1?\else
    \begingroup
      \def\childdoctmp{#1}
      \ifx\childdoctmp\childdocname
        \def\childdoctmp{}
      \else
        \def\childdoctmp
        {
          \childdoctrue
          \includeonly{\childdocname}
          \def\childdocjob{#1}
          \def\jobname{#1}
        }
      \fi
      \expandafter
    \endgroup
    \childdoctmp
  \fi
}
%    \end{macrocode}

% \macro{\childdocof}
% The command |\childdocof| redirects
% compilation to the main file |#1|.
%    \begin{macrocode}
\newcommand{\childdocof}[1]
{
  \childdocdisable
  \childdoctrue
  \includeonly{\childdocname}
  \def\jobname{#1}
  \def\childdocjob{#1}
  \input{#1}
}
%    \end{macrocode}

% \macro{\childdocby}
% The command |\childdocby| ....
%    \begin{macrocode}
\newcommand{\childdocby}[2][]
{
  \childdocdisable
  \childdoctrue
  \childdocmanualtrue
  \if?#1?\else
    \def\jobname{#2}
  \fi
  \def\childdocjob{#2}
  \input{#2}
  \endinput
}
%    \end{macrocode}

% \macro{\childdocforward}
% The command |\childdocforward| redirects
% compilation to the main file or
% (if the optional argument is given) a child file.
% Parameters are set as if the main file
% or a child file starting with |\childdocof| was compiled.
% Then compilation is handed over to the main file:
%    \begin{macrocode}
\newcommand{\childdocforward}[2][]
{
  \begingroup
    \if?#1?
      \def\childdoctmp
      {
        \def\childdocname{#2}
        \def\childdocjob{#2}
        \def\jobname{#2}
        \input{#2}
        \endinput
      }
    \else
      \def\childdoctmp
      {
        \childdocdisable
        \def\childdocname{#2}
        \childdoctrue
        \includeonly{#2}
        \def\childdocjob{#1}
        \def\jobname{#1}
        \input{#1}
        \endinput
      }
    \fi
    \expandafter
  \endgroup
  \childdoctmp
}
%    \end{macrocode}

% \macro{\childdocforwardprefix}
% The command |\childdocforwardprefix| redirects
% compilation to the main or a child file by means of a pattern.
% The prefix |#1| in the current filename is replaced by |#2|
% and the suffix of the current filename is kept
% (it is assumed that the filename does not contain the substring `|~~~|'
% which is used as a delimiter).
% Compilation is handed over to the new file by |\childdocforward|:
%    \begin{macrocode}
\newcommand{\childdocforwardprefix}[3][]
{
  \begingroup
    \def\childdocextract #2##1~~~{\def\childdoctmp{\childdocforward[#1]{#3##1}}}
    \expandafter\childdocextract\childdocname~~~
    \expandafter
  \endgroup
  \childdoctmp
}
%    \end{macrocode}

% \macro{\childdoc}
% The deprecated macro |\childdoc| is a legacy version of |\childdocmain|:
%    \begin{macrocode}
\newcommand{\childdoc}{\childdocmain}
%    \end{macrocode}

% \macro{\childdocredirect}
% The deprecated macro |\childdocredirect| is a legacy version
% of |\childdocforward| and |\childdocforwardprefix|:
%    \begin{macrocode}
\newcommand{\childdocredirect}[2][]
{
  \begingroup
    \if?#1?
      \def\childdoctmp{\childdocforward{#2}}
    \else
      \def\childdoctmp{\childdocforwardprefix{#1}{#2}}
    \fi
    \expandafter
  \endgroup
  \childdoctmp
}
%    \end{macrocode}

%\iffalse
%</package>
%\fi
%
\endinput
|\\
|\childdocforward{|\textit{main}|}|\\
\end{tabular}
\end{center}
%
or alternatively with:
%
\begin{center}
\begin{tabular}{l}
|% \iffalse
%
% childdoc.dtx Copyright (C) 2017-2018 Niklas Beisert
%
% This work may be distributed and/or modified under the
% conditions of the LaTeX Project Public License, either version 1.3
% of this license or (at your option) any later version.
% The latest version of this license is in
%   http://www.latex-project.org/lppl.txt
% and version 1.3 or later is part of all distributions of LaTeX
% version 2005/12/01 or later.
%
% This work has the LPPL maintenance status `maintained'.
%
% The Current Maintainer of this work is Niklas Beisert.
%
% This work consists of the files childdoc.dtx and childdoc.ins
% and the derived files childdoc.def and cdocsamp.tex with
% cdocsch1.tex, cdocsch2.tex, cdocsdrf.tex, cdocsfn1.tex, cdocsfn2.tex.
%
%<package>\ifdefined\childdocmain\endinput\fi
%<package>\ProvidesFile{childdoc.def}[2018/12/30 v2.0 child document driver]
%<samplemain>\ProvidesFile{cdocsamp.tex}[2018/12/30 v2.0 sample for childdoc]
%<*driver>
%\ProvidesFile{childdoc.drv}[2018/12/30 v2.0 childdoc reference manual file]
\PassOptionsToClass{10pt,a4paper}{article}
\documentclass{ltxdoc}

\usepackage[margin=35mm]{geometry}
\usepackage{hyperref}
\usepackage{hyperxmp}
\usepackage[usenames]{color}

\hypersetup{colorlinks=true}
\hypersetup{pdfstartview=FitH}
\hypersetup{pdfpagemode=UseNone}
\hypersetup{pdfsource={}}
\hypersetup{pdflang={en-UK}}
\hypersetup{pdfcopyright={Copyright 2017-2018 Niklas Beisert.
  This work may be distributed and/or modified under the
  conditions of the LaTeX Project Public License, either version 1.3
  of this license or (at your option) any later version.}}
\hypersetup{pdflicenseurl={http://www.latex-project.org/lppl.txt}}
\hypersetup{pdfcontactaddress={ETH Zurich, ITP, HIT K,
  Wolfgang-Pauli-Strasse 27}}
\hypersetup{pdfcontactpostcode={8093}}
\hypersetup{pdfcontactcity={Zurich}}
\hypersetup{pdfcontactcountry={Switzerland}}
\hypersetup{pdfcontactemail={nbeisert@itp.phys.ethz.ch}}
\hypersetup{pdfcontacturl={http://people.phys.ethz.ch/\xmptilde nbeisert/}}

\newcommand{\secref}[1]{\hyperref[#1]{section \ref*{#1}}}

\parskip1ex
\parindent0pt
\let\olditemize\itemize
\def\itemize{\olditemize\parskip0pt}

\begin{document}

\title{The \textsf{childdoc} Package}
\hypersetup{pdftitle={The childdoc Package}}
\author{Niklas Beisert\\[2ex]
  Institut f\"ur Theoretische Physik\\
  Eidgen\"ossische Technische Hochschule Z\"urich\\
  Wolfgang-Pauli-Strasse 27, 8093 Z\"urich, Switzerland\\[1ex]
  \href{mailto:nbeisert@itp.phys.ethz.ch}
  {\texttt{nbeisert@itp.phys.ethz.ch}}}
\hypersetup{pdfauthor={Niklas Beisert}}
\hypersetup{pdfsubject={Manual for the LaTeX2e Package childdoc}}
\date{30 December 2018, \textsf{v2.0}}
\maketitle

\begin{abstract}\noindent
\textsf{childdoc} is a \LaTeXe{} package
that enables the direct compilation
of document sections included by |\include|
to individual files.
\end{abstract}

\begingroup
\parskip0ex
\tableofcontents
\endgroup

%%%%%%%%%%%%%%%%%%%%%%%%%%%%%%%%%%%%%%%%%%%%%%%%%%%%%%%%%%%%%%%%%%%%%%%%%%%%%%%%
%%%%%%%%%%%%%%%%%%%%%%%%%%%%%%%%%%%%%%%%%%%%%%%%%%%%%%%%%%%%%%%%%%%%%%%%%%%%%%%%
\section{Introduction}

\LaTeX{} provides a mechanism to structure a large document (such as a book)
into a main file and several child files (containing the chapters)
using the |\include| command.
This mechanism is beneficial for documents
which span hundreds of pages in order to
make the source file(s) more manageable.
Moreover, compilation can be restricted to
selected child files by means of the |\includeonly| command.
The latter feature can be used to reduce the compilation time while editing
(this was significantly more useful in the earlier days of \LaTeX{})
or to generate a smaller document which is easier to navigate.
Another application of |\includeonly| is to generate
documents consisting of selected parts of the complete document.

However, there are a few drawbacks of the plain |\include| mechanism:
\begin{itemize}
\item
The child files cannot be compiled on their own,
they can only be compiled via the main file.
A naive editing environment
(such as a text editor with an option
to have the current file processed by \LaTeX)
may require one to switch to the main file before compiling;
attempting to compile the child file produces errors.
\item
The main file must be modified (each time)
to adjust the |\includeonly| command
to the present needs. This easily leaves the main file in a messy state.
\item
The generated document will always carry the filename
of the main document. This is inconvenient if
several child files are to be compiled and
to be kept for distribution.
\end{itemize}

The present package provides a simple interface
to make child files individually compilable by \LaTeX{}.
Compiling a child file then has the same effect as compiling
the main file with an |\includeonly| command
to select the appropriate child.
Moreover the generated document will carry the name of the child
rather than the main file.
This resolves all three above issues.

This feature is meant to make the editing of books,
thesis documents and lecture notes somewhat more convenient.
However, the package can also be used efficiently for
composing a series of documents (such as exercise sheets)
which are typically distributed individually.
It then assists the author in generating the individual documents
(potentially in different versions)
as well as a document containing the collected series.
Another application is in developing style files
or other kinds of included material
where compilation of the style file could redirect
to a sample or test file.

%%%%%%%%%%%%%%%%%%%%%%%%%%%%%%%%%%%%%%%%%%%%%%%%%%%%%%%%%%%%%%%%%%%%%%%%%%%%%%%%
%%%%%%%%%%%%%%%%%%%%%%%%%%%%%%%%%%%%%%%%%%%%%%%%%%%%%%%%%%%%%%%%%%%%%%%%%%%%%%%%
\section{Usage}

First of all, the package \textsf{childdoc} is \emph{not} a standard
\LaTeXe{} |.sty| style file! Therefore it needs to be invoked in
a non-standard way.

%%%%%%%%%%%%%%%%%%%%%%%%%%%%%%%%%%%%%%%%%%%%%%%%%%%%%%%%%%%%%%%%%%%%%%%%%%%%%%%%
\subsection{Included Files}
\label{sec:include}

%%%%%%%%%%%%%%%%%%%%%%%%%%%%%%%%%%%%%%%%
\DescribeMacro{\childdocmain}
To use the package, add the commands
\begin{center}
\begin{tabular}{l}
|\input{childdoc.def}|\\
|\childdocmain{}|\\
\end{tabular}
\end{center}
at the very top of the main \LaTeX{} file,
in particular \emph{before} the |\documentclass| statement!
The argument of |\childdocmain| should be left empty
(but it must be present).

%%%%%%%%%%%%%%%%%%%%%%%%%%%%%%%%%%%%%%%%
\DescribeMacro{\childdocof}
Furthermore, add the commands
\begin{center}
\begin{tabular}{l}
|\input{childdoc.def}|\\
|\childdocof{|\textit{main}|}|\\
\end{tabular}
\end{center}
at the top of every child file \textit{child}
which is included by |\include{|\textit{child}|}|
from within the main file
(or at least for those files to be compiled individually).
The argument \textit{main} must be the filename of the main file.

There are a couple of
considerations in setting up the main and child documents:

%%%%%%%%%%%%%%%%%%%%%%%%%%%%%%%%%%%%%%%%
\paragraph{Restrictions.}

Please note the following restrictions:
\begin{itemize}
\item
|\childdocmain| must be called with one argument \textit{main}
to ensure compatibility with earlier version of the package.
It must either be empty (|\childdocmain{}|)
or precisely match the filename of the main file in which it is specified.
See \secref{sec:detection} for further information.
\item
The filename \textit{main} must be specified without the |.tex| extension.
\item
The filename \textit{main} is case sensitive
(even in case-insensitive file systems)
due to internal string comparison.
\item
The argument \textit{main} should be fully expanded, it cannot be a macro.
\item
Subdirectories and special characters should be avoided in filenames.
\item
The command |\childdocmain{|\textit{main}|}| must be followed by a whitespace.
It should not be followed immediately by another command
or by a comment mark `|%|'.
This is because the \TeX{} parser reads the token immediately following
the argument of |\childdocmain| and puts it
at the beginning of every child section;
however, a white\-space is ignored.
\end{itemize}

%%%%%%%%%%%%%%%%%%%%%%%%%%%%%%%%%%%%%%%%
\paragraph{Content of Main File.}

It is advisable to place all content in the child files included by |\include|.
Any output contained in the main file will appear in all child documents
unless suppressed manually;
it cannot be suppressed automatically by the |\includeonly| directive
and thus should normally be avoided.
A method to include some content in the main file
by means of conditional processing is described in \secref{sec:conditional}.

%%%%%%%%%%%%%%%%%%%%%%%%%%%%%%%%%%%%%%%%
\paragraph{Page Numbering.}

When only a part of the document is compiled,
the appropriate numbering of pages
(as well as other status parameters)
is determined from the |.aux| files.
The latter contain information from previous passes.
However this information needs to propagate through
all intermediate child documents.
Therefore the page numbering in child documents may well
be inconsistent until the complete document is compiled at least once.

A useful (if unconventional) way to always ensure a consistent
page numbering is to restart the numbering in each child document
and denote the pages by `\textit{child}|.|\textit{page}'
where \textit{child} represents the chapter/section number of the child file.
This can be achieved by the command
|\numberwithin{page}{|\textit{child}|}|
of the \textsf{amsmath} package
where \textit{child} can be |chapter| or |section|
depending on the chosen structuring.
Alternatively, one can modify the macro |\thepage| appropriately
and reset the counter |page| at the start of each child file.

%%%%%%%%%%%%%%%%%%%%%%%%%%%%%%%%%%%%%%%%%%%%%%%%%%%%%%%%%%%%%%%%%%%%%%%%%%%%%%%%
\subsection{Conditional Processing}
\label{sec:conditional}

The package provides a mechanism to compile different versions
of a document. To customise the versions further some conditional processing
can come in handy to distinguish which version is being compiled.
The package provides two macros to describe the compilation context:

%%%%%%%%%%%%%%%%%%%%%%%%%%%%%%%%%%%%%%%%
\DescribeMacro{\ifchilddoc}
The conditional |\ifchilddoc| distinguishes between the compilation of
child documents and the main document:
%
\begin{center}
|\ifchilddoc |\textit{child-code}| |[|\||else |\textit{main-code}]| \||fi|
\end{center}

%%%%%%%%%%%%%%%%%%%%%%%%%%%%%%%%%%%%%%%%
\DescribeMacro{\childdocname}
\DescribeMacro{\childdocjob}
The macro |\childdocname| contains the filename (without extension)
of the main or child file being processed.
Note that |\childdocjob| will always contain the name of the main file.

%%%%%%%%%%%%%%%%%%%%%%%%%%%%%%%%%%%%%%%%
\paragraph{Title Page.}

Conditional processing can be used to include a title or banner page
in the main document when proper precautions are taken.
Importantly, the code in the main file should ensure that the page counter
(as well as other status parameters which are stored in the |.aux| files)
takes the same value after the conditional processing.
Otherwise the page numbers may take divergent values
depending on which part is compiled.

For example, a title page could be declared by:
%
\begin{center}
\begin{tabular}{l}
|\ifchilddoc\||else|\\
|\addtocounter{page}{-1}|\\
\textit{code for title page}\\
|\newpage|\\
|\||fi|
\end{tabular}
\end{center}
%
A banner page for the child documents can be generated by:
%
\begin{center}
\begin{tabular}{l}
|\ifchilddoc|\\
|\addtocounter{page}{-1}|\\
\textit{code for banner page}\\
|\newpage|\\
|\||fi|
\end{tabular}
\end{center}
%
Here one could write a message such as:
\begin{center}
|This is the part \childdocname{} of \childdocjob{}.|
\end{center}

%%%%%%%%%%%%%%%%%%%%%%%%%%%%%%%%%%%%%%%%%%%%%%%%%%%%%%%%%%%%%%%%%%%%%%%%%%%%%%%%
\subsection{Flags}
\label{sec:flags}

The package makes it easy to generate different versions
of the main or child documents.
To this end compilation flags can be defined
and assigned different default values.
They will be particularly useful in conjunction
with the forwarding mechanism described in \secref{sec:forward}.

For example, it may be useful to have a flag |\version|
which can be set to |draft| or |final|.
The document source will contain some conditional code
depending on the value of |\version|.
Suppose further, the flag should default to |final| for the main file
and to |draft| for child files
which is a natural assignment for editing the document.
This is achieved by placing the following code
in the preamble of the main document
(below the |\childdocmain| directive):
%
\begin{center}
\begin{tabular}{l}
|\ifchilddoc|\\
|\providecommand{\version}{draft}|\\
|\||else|\\
|\providecommand{\version}{final}|\\
|\||fi|
\end{tabular}
\end{center}
%
The definition by |\providecommand| makes sure
that previous definitions are not overwritten.
Further statements |\providecommand{\version}{...}|
can thus be added before the above code to override it.

For the main file, one might add a line
(between |\childdocmain| and the above block)
%
\begin{center}
|%\ifchilddoc\||else\providecommand{\version}{draft}\||fi|
\end{center}
%
which can be uncommented to produce a draft version.
Likewise one can add a line to the very top of a child file
(above the |\childdocof{|\textit{main}|}| directive)
%
\begin{center}
|%\providecommand{\version}{final}|
\end{center}
%
which can be uncommented to produce the final version of this child document.

%%%%%%%%%%%%%%%%%%%%%%%%%%%%%%%%%%%%%%%%%%%%%%%%%%%%%%%%%%%%%%%%%%%%%%%%%%%%%%%%
\subsection{Forwarding}
\label{sec:forward}

Different versions of the main or child documents
using compilation flags as described in \secref{sec:flags}
can be (permanently) stored in different files
for convenient compilation, viewing and distribution.
To this end, the package defines a command
to pass on compilation to a different file:

%%%%%%%%%%%%%%%%%%%%%%%%%%%%%%%%%%%%%%%%
\DescribeMacro{\childdocforward}
The command |\childdocforward| redirects processing to
another source file:
%
\begin{center}
\begin{tabular}{l}
|\input{childdoc.def}|\\
|\childdocforward[|\textit{main}|]{|\textit{dest}|}|\\
\end{tabular}
\end{center}
%
The argument \textit{dest} is the destination file
(without extension).
It should be the main file or one of the child files.
Note that further \textsf{childdoc} directives
such as |\childdocof| and |\childdocforward|
in the indicated file will be processed in this form.
The optional argument \textit{main}
passes on directly to the main file \textit{main}
while pretending to compile the child \textit{dest}.
This form behaves as if \textit{dest}
issues |\childdocof{|\textit{main}|}| right away,
and no further \textsf{childdoc} directives will be processed.

%%%%%%%%%%%%%%%%%%%%%%%%%%%%%%%%%%%%%%%%
\DescribeMacro{\...prefix}
In the alternative form |\childdocforwardprefix|,
%
\begin{center}
\begin{tabular}{l}
|\input{childdoc.def}|\\
|\childdocforwardprefix[|\textit{main}|]{|\textit{prefix}|}{|\textit{dest}|}|
\end{tabular}
\end{center}
%
the destination file is determined by a pattern
depending on the current file:
To make this work, the current file must be called
`{\textit{prefix}\hspace{0.2em}\textit{suffix}}'
with \textit{prefix} matching precisely the argument.
Processing is then passed on to the file
`{\textit{dest}\hspace{0.2em}\textit{suffix}}'.
Surely, the same effect is achieved by
directly specifying the
argument `{\textit{dest}\hspace{0.2em}\textit{suffix}}'
in the first form.
However, that requires to set up a different file
for each child. With the alternative form of the command
all these files can have exactly the same content
which simplifies setting them up and maintaining them.

For example, the following file |draft.tex|
with a compilation flag |\version| as described in \secref{sec:flags}
compiles the main document as a draft:
%
\begin{center}
\begin{tabular}{l}
|\def\version{draft}|\\
|\input{childdoc.def}|\\
|\childdocforward{|\textit{main}|}|
\end{tabular}
\end{center}
%
Likewise, the following files |final|\textit{nn}|.tex|
compile the final version of the child document
|child|\textit{nn}|.tex|:
%
\begin{center}
\begin{tabular}{l}
|\def\version{final}|\\
|\input{childdoc.def}|\\
|\childdocforwardprefix{final}{child}|
\end{tabular}
\end{center}
%

Note that when several versions of a main file and/or of each child file
are to be generated, it may be convenient to set up a |Makefile| or
shell script to automatise the process.

%%%%%%%%%%%%%%%%%%%%%%%%%%%%%%%%%%%%%%%%%%%%%%%%%%%%%%%%%%%%%%%%%%%%%%%%%%%%%%%%
\subsection{Command Line Processing}
\label{sec:commandline}

The effect of redirection files can also be achieved by invoking
the \LaTeX{} compiler with a more elaborate command line.
Most conveniently this should be done as part
of a shell script or a |Makefile|.

When using \textsf{childdoc} in the main file, the following
command lines effectively perform a redirection
(note that depending on the shell being used,
backslashes may have to be doubled: `|\|' $\to$ `|\\|'):
%
\begin{center}
|... -jobname "|\textit{target}|" |\\|"|[\textit{flags}]%
|\input{childdoc.def}\childdocforward[|\textit{main}|]{|\textit{dest}|}"|
\end{center}
%
Here \textit{target} is the name of the output file,
\textit{main} is the name of the main file
and \textit{dest} is the name of the main or child file to be processed
(all filenames without extensions).
The optional argument \textit{main} can be omitted
if \textit{main} matches \textit{dest}.
Optionally, compilation \textit{flags} can be defined via |\def| commands.
This command line makes the \TeX{} engine believe
it is compiling the file \textit{target}
whose content is specified as the latter parameter.
The provided code then forwards the processing to
\textit{main} or \textit{dest} as described in \secref{sec:forward}.

%%%%%%%%%%%%%%%%%%%%%%%%%%%%%%%%%%%%%%%%%%%%%%%%%%%%%%%%%%%%%%%%%%%%%%%%%%%%%%%%
\subsection{Include by Input}
\label{sec:input}

Including child documents by |\include| has some restrictions by design.
Most notably, the content of a child document always occupies
its own set of pages; pages cannot be shared between child documents.
Usually, this behaviour makes perfect sense
because each child document contain an essential part of the document.
However, in some situations it may be desirable to compose
a document from a collection of parts
without having mandatory page breaks between then.
For this case, the package
provides a mechanism to include parts
by |\input| which can also be processed individually.
However, by construction this mechanism
requires manual handling of the content to be output.

%%%%%%%%%%%%%%%%%%%%%%%%%%%%%%%%%%%%%%%%
\DescribeMacro{\ifchilddocmanual}
The main file should be prepared as usual, see \secref{sec:include}.
However, the document body must make a distinction
between processing of an individual part and of the main document, e.g.:
%
\begin{center}
\begin{tabular}{l}
|\ifchilddocmanual|\\
|\input{\childdocname}|\\
|\||else|\\
\textit{document body with }|\input{|\textit{part}|}|\\
|\||fi|
\end{tabular}
\end{center}
%
The conditional |\ifchilddocmanual| is true whenever
a part to be included by |\input| is being compiled,
and the name of the part is stored in |\childdocname|.

%%%%%%%%%%%%%%%%%%%%%%%%%%%%%%%%%%%%%%%%
\DescribeMacro{\childdocby}
Each part to be included by |\input| should start with:
%
\begin{center}
\begin{tabular}{l}
|\input{childdoc.def}|\\
|\childdocby{|\textit{main}|}|\\
\end{tabular}
\end{center}
%
The directive |\childdocby| is similar to |\childdocof|
described in \secref{sec:include},
but the subsequent selection of content must be done manually.
To that end, both |\ifchilddoc| and |\ifchilddocmanual|
will be true upon processing of a part,
and the name of the part is stored in |\childdocname|.
Note that |\jobname| will be set to the filename of the current part
so that each part receives an individual |.aux| file
that does not interfere with the |.aux| file(s) of the main document.
This behaviour can be altered by the alternative form
|\childdocby[*]{|\textit{main}|}| (with a non-empty optional argument)
which uses the |.aux| file of the main document
by setting |\jobname| to \textit{main}.

%%%%%%%%%%%%%%%%%%%%%%%%%%%%%%%%%%%%%%%%%%%%%%%%%%%%%%%%%%%%%%%%%%%%%%%%%%%%%%%%
\subsection{Driver Development}
\label{sec:driver}

The \textsf{childdoc} mechanism can also be use for the development
of definition files such as \LaTeX{} styles or classes.
This case differs from the above setup with multiple parts
included by |\include| in that no |\includeonly| should be invoked.
This can be achieved by starting the include file
(before |\ProvidesPackage|) with:
%
\begin{center}
\begin{tabular}{l}
|\input{childdoc.def}|\\
|\childdocforward{|\textit{main}|}|\\
\end{tabular}
\end{center}
%
or alternatively with:
%
\begin{center}
\begin{tabular}{l}
|\input{childdoc.def}|\\
|\childdocby{|\textit{main}|}|\\
\end{tabular}
\end{center}
%
Both forms have slightly different effects as described above.
The main file is prepared as usual, see \secref{sec:include}.

%%%%%%%%%%%%%%%%%%%%%%%%%%%%%%%%%%%%%%%%%%%%%%%%%%%%%%%%%%%%%%%%%%%%%%%%%%%%%%%%
\subsection{Legacy Detection}
\label{sec:detection}

The directive |\childdocmain| in the main file can detect
whether the complete document or merely a child is to be compiled
even without using the directive |\childdocof|.
This method is deprecated because it is less robust
and there is no compelling reason to use it;
it is merely provided for backward compatibility
and it may be removed in future versions.

If the detection mechanism is to be used,
it is mandatory to correctly specify
the filename of the main file as the argument of |\childdocmain|:
%
\begin{center}
\begin{tabular}{l}
|\input{childdoc.def}|\\
|\childdocmain{|\textit{main}|}|\\
\end{tabular}
\end{center}
%
If |\jobname| does not match the argument \textit{main} of |\childdocmain|,
it is assumed that |\jobname| points to the child file to be compiled.
When using |\childdocmain| with the main file specified as argument,
it suffices to start a child file
with just |\input{|\textit{main}|}|
without loading of the package and using |\childdocof|.
If instead all processing is done
with the appropriate \textsf{childdoc} directives,
the argument of \textit{main} of |\childdocmain| can be empty.

An alternative version of the command line processing described
in \secref{sec:commandline} using the detection mechanism reads:
%
\begin{center}
|... -jobname "|\textit{target}|" "|[\textit{flags}]%
[|\def\jobname{|\textit{dest}|}|]|\input{|\textit{main}|}"|
\end{center}

%%%%%%%%%%%%%%%%%%%%%%%%%%%%%%%%%%%%%%%%%%%%%%%%%%%%%%%%%%%%%%%%%%%%%%%%%%%%%%%%
\subsection{Manual Code}
\label{sec:manual}

In case one cannot be certain whether the definitions file |childdoc.def|
is installed on the target \TeX{} distribution
and one prefers not to ship it,
it is conceivable to paste a few relevant commands into the sources.

To that end, drop all statements |\input{childdoc.def}|
and perform the replacements as outlined below.
Instead of |\childdocmain{|\textit{main}|}| add the following code
to the top of the main file:
%
\begin{center}
\begin{tabular}{l}
|\||ifdefined\childdocname\endinput\||fi\newif\ifchilddoc|\\
|\edef\childdocname{\scantokens\expandafter{\jobname\noexpand}}|\\
|\def\childdocmain{|\textit{main}|}\||ifx\childdocmain\childdocname\||else|\\
|\childdoctrue\includeonly{\childdocname}\let\jobname\childdocmain\||fi|\\
\end{tabular}
\end{center}
%
Instead of |\childdocof{|\textit{main}|}| just include the main file
at the top of each child file:
%
\begin{center}
|\input{|\textit{main}|}|
\end{center}
%
A simple redirection |\childdocforward{|\textit{dest}|}| is achieved by:
%
\begin{center}
|\def\jobname{|\textit{dest}|}\input{\jobname}|
\end{center}
%
The redirection with prefix
|\childdocforwardprefix[|\textit{prefix}|]{|\textit{dest}|}|
is accomplished by:
%
\begin{center}
\begin{tabular}{l}
|{\edef\jobname{\scantokens\expandafter{\jobname\noexpand}}|\\
|\def\redirectjob |\textit{prefix}|#1~~~{\gdef\jobname{|\textit{dest}|#1}}|\\
|\expandafter\redirectjob\jobname~~~}\input{\jobname}|
\end{tabular}
\end{center}

In an alternative approach,
child documents can be compiled by a specific command line
without additional code or specific definitions:
%
\begin{center}
|... -jobname "|\textit{target}|" "|[\textit{flags}]%
|\includeonly{|\textit{dest}|}\input{|\textit{main}|}"|
\end{center}
%

%%%%%%%%%%%%%%%%%%%%%%%%%%%%%%%%%%%%%%%%%%%%%%%%%%%%%%%%%%%%%%%%%%%%%%%%%%%%%%%%
%%%%%%%%%%%%%%%%%%%%%%%%%%%%%%%%%%%%%%%%%%%%%%%%%%%%%%%%%%%%%%%%%%%%%%%%%%%%%%%%
\section{Information}

%%%%%%%%%%%%%%%%%%%%%%%%%%%%%%%%%%%%%%%%%%%%%%%%%%%%%%%%%%%%%%%%%%%%%%%%%%%%%%%%
\subsection{Copyright}

Copyright \copyright{} 2017--2018 Niklas Beisert

This work may be distributed and/or modified under the
conditions of the \LaTeX{} Project Public License, either version 1.3
of this license or (at your option) any later version.
The latest version of this license is in
  \url{http://www.latex-project.org/lppl.txt}
and version 1.3 or later is part of all distributions of \LaTeX{}
version 2005/12/01 or later.

This work has the LPPL maintenance status `maintained'.

The Current Maintainer of this work is Niklas Beisert.

This work consists of the files |README.txt|, |childdoc.ins| and |childdoc.dtx|
as well as the derived files |childdoc.def|, |cdocsamp.tex|
with |cdocsch1.tex|, |cdocsch2.tex|, |cdocspt3.tex|, |cdocspt4.tex|,
|cdocsdrf.tex|, |cdocsfn1.tex|, |cdocsfn2.tex|
as well as |childdoc.pdf|.

%%%%%%%%%%%%%%%%%%%%%%%%%%%%%%%%%%%%%%%%%%%%%%%%%%%%%%%%%%%%%%%%%%%%%%%%%%%%%%%%
\subsection{Files and Installation}

The package consists of the files:
%
\begin{center}
\begin{tabular}{ll}
    |README.txt|   & readme file \\
    |childdoc.ins| & installation file \\
    |childdoc.dtx| & source file \\
    |childdoc.def| & definition file \\
    |cdocsamp.tex| & sample main file \\
    |cdocsch1.tex| & sample include file \\
    |cdocsch2.tex| & sample include file \\
    |cdocspt3.tex| & sample part file \\
    |cdocspt4.tex| & sample part file \\
    |cdocsdrf.tex| & sample redirection file \\
    |cdocsfn1.tex| & sample redirection file \\
    |cdocsfn2.tex| & sample redirection file \\
    |childdoc.pdf| & manual
\end{tabular}
\end{center}
%
The distribution consists of the files
|README.txt|, |childdoc.ins| and |childdoc.dtx|.
%
\begin{itemize}
\item
Run (pdf)\LaTeX{} on |childdoc.dtx|
to compile the manual |childdoc.pdf| (this file).
\item
Run \LaTeX{} on |childdoc.ins| to create the definitions file |childdoc.def|
and the sample |cdocsamp.tex| with include files
|cdocsch1.tex|, |cdocsch2.tex|, |cdocspt3.tex|, |cdocspt4.tex|,
|cdocsdrf.tex|, |cdocsfn1.tex|, |cdocsfn2.tex|.
Then copy the file |childdoc.def| to an appropriate directory of your \LaTeX{}
distribution, e.g.\ \textit{texmf-root}|/tex/latex/childdoc|.
\end{itemize}

%%%%%%%%%%%%%%%%%%%%%%%%%%%%%%%%%%%%%%%%%%%%%%%%%%%%%%%%%%%%%%%%%%%%%%%%%%%%%%%%
\subsection{Related CTAN Packages}

There are several other packages which offer a similar functionality:
%
\begin{itemize}
\item
The packages
\href{http://ctan.org/pkg/docmute}{\textsf{docmute}},
\href{http://ctan.org/pkg/includex}{\textsf{includex}} and
\href{http://ctan.org/pkg/standalone}{\textsf{standalone}}
provide commands to include only the document body of
a child file thus allowing both files to be compiled individually.
\item
The packages \href{http://ctan.org/pkg/subdocs}{\textsf{subdocs}}
and \href{http://ctan.org/pkg/subfiles}{\textsf{subfiles}}
provide structures in which the main and child documents can be
encapsulated and allowing them to be compiled individually.
The inclusion mechanism is different from the conventional |\include|.
\item
The package \href{http://ctan.org/pkg/combine}{\textsf{combine}}
is an elaborate solution to combine several documents into one.
\end{itemize}
%
See also the CTAN topic \href{http://ctan.org/topic/subdocs}{\textsf{subdocs}}
for further related packages.
The present package differs from the above solutions in that
a document structure constructed with the conventional |\include| mechanism
just needs two extra commands at the top of every file
such that all constituent files can be compiled individually.

%%%%%%%%%%%%%%%%%%%%%%%%%%%%%%%%%%%%%%%%%%%%%%%%%%%%%%%%%%%%%%%%%%%%%%%%%%%%%%%%
%\subsection{Feature Suggestions}
%
%The following is a list of features which may be useful for future
%versions of this package:
%%
%\begin{itemize}
%\item
%\ldots
%\end{itemize}

%%%%%%%%%%%%%%%%%%%%%%%%%%%%%%%%%%%%%%%%%%%%%%%%%%%%%%%%%%%%%%%%%%%%%%%%%%%%%%%%
\subsection{Revision History}

%%%%%%%%%%%%%%%%%%%%%%%%%%%%%%%%%%%%%%%%
\paragraph{v2.0:} 2018/12/30

\begin{itemize}
\item
immediate forward processing
\item
added |\childdocby| mechanism
\item
manual restructured
\end{itemize}

%%%%%%%%%%%%%%%%%%%%%%%%%%%%%%%%%%%%%%%%
\paragraph{v1.6:} 2018/01/17

\begin{itemize}
\item
application for development of include files
\item
corrections to manual
\end{itemize}

%%%%%%%%%%%%%%%%%%%%%%%%%%%%%%%%%%%%%%%%
\paragraph{v1.5:} 2017/05/21

\begin{itemize}
\item
more complete structuring introduced
\item
|\childdocof| introduced
\item
|\childdoc| renamed to |\childdocmain|
\item
|\childredirect| renamed to |\childdocforward| and |\childdocforwardprefix|
and functionality expanded
\end{itemize}

%%%%%%%%%%%%%%%%%%%%%%%%%%%%%%%%%%%%%%%%
\paragraph{v1.0:} 2017/04/27

\begin{itemize}
\item
manual and install package
\item
first version published on CTAN
\end{itemize}

%%%%%%%%%%%%%%%%%%%%%%%%%%%%%%%%%%%%%%%%
\paragraph{v0.6:} 2017/04/26

\begin{itemize}
\item
redirection mechanism added
\end{itemize}

%%%%%%%%%%%%%%%%%%%%%%%%%%%%%%%%%%%%%%%%
\paragraph{v0.5:} 2017/04/26

\begin{itemize}
\item
functionality in definition file
\end{itemize}


%%%%%%%%%%%%%%%%%%%%%%%%%%%%%%%%%%%%%%%%%%%%%%%%%%%%%%%%%%%%%%%%%%%%%%%%%%%%%%%%
%%%%%%%%%%%%%%%%%%%%%%%%%%%%%%%%%%%%%%%%%%%%%%%%%%%%%%%%%%%%%%%%%%%%%%%%%%%%%%%%
%%%%%%%%%%%%%%%%%%%%%%%%%%%%%%%%%%%%%%%%%%%%%%%%%%%%%%%%%%%%%%%%%%%%%%%%%%%%%%%%
\appendix

\settowidth\MacroIndent{\rmfamily\scriptsize 000\ }

 \DocInput{childdoc.dtx}

\end{document}
%</driver>
% \fi
%
% %%%%%%%%%%%%%%%%%%%%%%%%%%%%%%%%%%%%%%%%%%%%%%%%%%%%%%%%%%%%%%%%%%%%%%%%%%%%%%
% %%%%%%%%%%%%%%%%%%%%%%%%%%%%%%%%%%%%%%%%%%%%%%%%%%%%%%%%%%%%%%%%%%%%%%%%%%%%%%
% \section{Sample}
%\iffalse
%<*samplemain>
%\fi
%
% The following presents a sample document
% with two chapters, two parts, a title page,
% a compile flag as well as three forwarding files to set the flag.
% It consists of eight |.tex| files:
% \begin{center}
% \begin{tabular}{ll}
% |cdocsamp.tex|&main file\\
% |cdocsch1.tex|&include file for chapter 1\\
% |cdocsch2.tex|&include file for chapter 2\\
% |cdocspt3.tex|&include file for part 3\\
% |cdocspt4.tex|&include file for part 4\\
% |cdocsdrf.tex|&forwarding file for main file in draft mode\\
% |cdocsfi1.tex|&forwarding file for final version of chapter 1\\
% |cdocsfi2.tex|&forwarding file for final version of chapter 2\\
% \end{tabular}
% \end{center}
% Each of the eight files can be compiled directly by the \LaTeX{} compiler.
%
% %%%%%%%%%%%%%%%%%%%%%%%%%%%%%%%%%%%%%%
% \paragraph{Main File.}
%
% The main file is called |cdocsamp.tex|.
%
% Load the \textsf{childdoc} definitions and
% declare the filename for the main document:
%    \begin{macrocode}
\input{childdoc.def}
\childdocmain{}
%    \end{macrocode}

% Optional override for |\version| flag:
%    \begin{macrocode}
%%\ifchilddoc\else\providecommand{\version}{draft}\fi
%    \end{macrocode}

% Define the default values for the |\version| flag
% (|final| for the main file and |draft| for childs):
%    \begin{macrocode}
\ifchilddoc
\providecommand{\version}{draft}
\else
\providecommand{\version}{final}
\fi
%    \end{macrocode}

% Load the standard document class:
%    \begin{macrocode}
\documentclass[12pt]{article}
%    \end{macrocode}

% Start the document body:
%    \begin{macrocode}
\begin{document}
%    \end{macrocode}

% Declare a title page.
% Print title, part of document being processed and version flag:
%    \begin{macrocode}
\addtocounter{page}{-1}
\begin{center}
{\LARGE\bfseries{}childdoc example\par}
\vspace{1cm}
\ifchilddoc
\ifchilddocmanual part\else chapter\fi:
`\childdocname' of `\childdocjob'\par
\else
main document: `\childdocjob'\par
\fi
version: \version\par
\end{center}
\newpage
%    \end{macrocode}

% Manually include selected file,
% otherwise process as usual:
%    \begin{macrocode}
\ifchilddocmanual
\section*{part `\childdocname'}
\input{\childdocname}
\else
%    \end{macrocode}

% Include the two chapters:
%    \begin{macrocode}
\include{cdocsch1}
\include{cdocsch2}
%    \end{macrocode}

% Include the two parts unless only chapters should be displayed:
%    \begin{macrocode}
\ifchilddoc\else
\section{part three}
\input{cdocspt3}
\section{part four}
\input{cdocspt4}
\fi
%    \end{macrocode}

% Process as usual until here:
%    \begin{macrocode}
\fi
%    \end{macrocode}

% End of document body:
%    \begin{macrocode}
\end{document}
%    \end{macrocode}
%\iffalse
%</samplemain>
%\fi
%
% %%%%%%%%%%%%%%%%%%%%%%%%%%%%%%%%%%%%%%
% \paragraph{Chapter Include Files.}
%
% The include files are called |cdocsch1.tex| and |cdocsch2.tex|.
%
%\iffalse
%<*samplechap1|samplechap2>
%\fi

% Optional override for |\version| flag:
%    \begin{macrocode}
%%\providecommand{\version}{final}
%    \end{macrocode}

% Include the main document:
%    \begin{macrocode}
\input{childdoc.def}
\childdocof{cdocsamp}
%    \end{macrocode}

%\iffalse
%</samplechap1|samplechap2>
%\fi
%
%\iffalse
%<*samplechap1>
%\fi
% Some text for chapter 1:
%    \begin{macrocode}
\section{one}
some text in chapter one
%    \end{macrocode}

%\iffalse
%</samplechap1>
%\fi
% Some text for chapter 2:
%\iffalse
%<*samplechap2>
%\fi
%    \begin{macrocode}
\section{two}
more text in chapter two
%    \end{macrocode}

%\iffalse
%</samplechap2>
%\fi
%
% %%%%%%%%%%%%%%%%%%%%%%%%%%%%%%%%%%%%%%
% \paragraph{Part Include Files.}
%
% The include files are called |cdocspt3.tex| and |cdocspt4.tex|.
%
%\iffalse
%<*samplepart3|samplepart4>
%\fi

% Optional override for |\version| flag:
%    \begin{macrocode}
%%\providecommand{\version}{final}
%    \end{macrocode}

% Include the main document:
%    \begin{macrocode}
\input{childdoc.def}
\childdocby{cdocsamp}
%    \end{macrocode}

%\iffalse
%</samplepart3|samplepart4>
%\fi
%
%\iffalse
%<*samplepart3>
%\fi
% Some text for part 3:
%    \begin{macrocode}
some text in part three
%    \end{macrocode}

%\iffalse
%</samplepart3>
%\fi
% Some text for part 4:
%\iffalse
%<*samplepart4>
%\fi
%    \begin{macrocode}
more text in part four
%    \end{macrocode}

%\iffalse
%</samplepart4>
%\fi
%
% %%%%%%%%%%%%%%%%%%%%%%%%%%%%%%%%%%%%%%
% \paragraph{Forwarding for a Complete Draft.}
%
% The following forwarding file |cdocsdrf.tex|
% compiles the main document in draft mode:
%\iffalse
%<*sampledraft>
%\fi
%    \begin{macrocode}
\def\version{draft}
\input{childdoc.def}
\childdocforward{cdocsamp}
%    \end{macrocode}

%\iffalse
%</sampledraft>
%\fi
%
% %%%%%%%%%%%%%%%%%%%%%%%%%%%%%%%%%%%%%%
% \paragraph{Forwarding for Final Version of the Chapters.}
%
% The following forwarding files |cdocsfn1.tex| and |cdocsfn2.tex|
% (with identical content)
% compile the final versions of the child documents
% |cdocsch1.tex| and |cdocsch2.tex|, respectively:
%\iffalse
%<*samplefinal>
%\fi
%    \begin{macrocode}
\def\version{final}
\input{childdoc.def}
\childdocforwardprefix[cdocsamp]{cdocsfn}{cdocsch}
%    \end{macrocode}

%\iffalse
%</samplefinal>
%\fi
%
% %%%%%%%%%%%%%%%%%%%%%%%%%%%%%%%%%%%%%%
% \paragraph{Command Line Processing.}
%
% The following three command lines generate the output files
% |cdocscld|, |cdocscl1| and |cdocscl2|
% which should be identical to
% |cdocsdrf|, |cdocsch1| and |cdocsfn2|, respectively:
% \begin{center}
% \begin{tabular}{l}
% |latex -jobname cdocscld \|\\
% |  "\def\version{draft}\input{childdoc.def}\childdocforward{cdocsamp}"|\\
% |latex -jobname cdocscl1 \|\\
% |  "\input{childdoc.def}\childdocforward[cdocsamp]{cdocsch1}"|\\
% |latex -jobname cdocscl2 \|\\
% |  "\def\version{final}\input{childdoc.def}\childdocforward{cdocsch2}"|
% \end{tabular}
% \end{center}
% Note that the trailing backslash on each first line
% merely continues the input to the second line
% (for convenient cut ant paste).
% Furthermore, the command |latex| can be replaced by any
% of its alternative versions such as |pdflatex|.
%
% %%%%%%%%%%%%%%%%%%%%%%%%%%%%%%%%%%%%%%%%%%%%%%%%%%%%%%%%%%%%%%%%%%%%%%%%%%%%%%
% %%%%%%%%%%%%%%%%%%%%%%%%%%%%%%%%%%%%%%%%%%%%%%%%%%%%%%%%%%%%%%%%%%%%%%%%%%%%%%
% \section{Implementation}
%\iffalse
%<*package>
%\fi
%
% This section describes the definitions file |childdoc.def|.

% The definitions cannot be loaded using |\usepackage| or |\RequirePackage|
% which has a mechanism to prevent loading a style file more than once.
% When loading the definitions by means of |\input|
% multiple instances have to be prevented manually:
%\iffalse
%This code needs to be before the `\ProvidesFile' directive
%which is defined at the beginning of this file.
%Therefore it is also placed there and commented out here.
%</package>
%<*discard>
%\fi
%    \begin{macrocode}
\ifdefined\childdocmain\endinput\fi
%    \end{macrocode}
%\iffalse
%</discard>
%<*package>
%\fi
%
% \macro{\ifchilddoc}
% \macro{\ifchilddocmanual}
% The conditional |\ifchilddoc| tells whether a
% child (true) or main (false) document is being compiled.
% The conditional |\ifchilddocmanual| tells whether
% the |\includeonly| mechanism is used (false) or
% the selection of child files must be performed manually (true).
% The definitions initialise to false:
%    \begin{macrocode}
\newif\ifchilddoc
\newif\ifchilddocmanual
%    \end{macrocode}

% \macro{\childdocname}
% \macro{\childdocjob}
% The macro |\childdocname| stores the name of the main document
% to be compiled. The macro |\childdocjob| stores the name of
% the document on which the \LaTeX{} compiler was originally invoked.
% The content of |\jobname| cannot be compared
% to filenames specified in the source due to different catcodes.
% The following code rescans |\jobname|, stores the result
% in |\childdocname| and saves a copy in |\childdocjob|:
%    \begin{macrocode}
\edef\childdocname{\scantokens\expandafter{\jobname\noexpand}}
\let\childdocjob\childdocname
%    \end{macrocode}

% \macro{\childdocdisable}
% The macro |\childdocdisable| prevents the main file
% from being processed more than once.
% At this stage, the main document command |\childdocmain|
% is assumed to be called once again where it should do nothing.
% Any subsequent call to it should prevent
% a secondary processing of the main document
% It overwrites the forwarding commands
% |\childdocof| and |\childdocforward|
% with empty macros to prevent further inclusions of the main document:
%    \begin{macrocode}
\newcommand{\childdocdisable}
{
  \renewcommand{\childdocmain}[1]{\renewcommand{\childdocmain}[1]{\endinput}}
  \renewcommand{\childdocof}[1]{}
  \renewcommand{\childdocby}[2][]{}
  \renewcommand{\childdocforward}[2][]{}
  \renewcommand{\childdocdisable}{}
}
%    \end{macrocode}

% \macro{\childdocmain}
% The macro |\childdocmain| is to be called at the top of the main file
% with nothing or the main filename (without extension) as argument.
% First, it breaks loops.
% If the argument is not empty and does not match |\childdocname|
% (which is set by the first inclusion of |childdoc.def|),
% |\ifchilddoc| is set to true, |\includeonly| is applied to the child file
% and |\jobname| is set to the main file
% (for proper handling of |.aux| files):
%    \begin{macrocode}
\newcommand{\childdocmain}[1]
{
  \childdocdisable\childdocmain{}
  \if?#1?\else
    \begingroup
      \def\childdoctmp{#1}
      \ifx\childdoctmp\childdocname
        \def\childdoctmp{}
      \else
        \def\childdoctmp
        {
          \childdoctrue
          \includeonly{\childdocname}
          \def\childdocjob{#1}
          \def\jobname{#1}
        }
      \fi
      \expandafter
    \endgroup
    \childdoctmp
  \fi
}
%    \end{macrocode}

% \macro{\childdocof}
% The command |\childdocof| redirects
% compilation to the main file |#1|.
%    \begin{macrocode}
\newcommand{\childdocof}[1]
{
  \childdocdisable
  \childdoctrue
  \includeonly{\childdocname}
  \def\jobname{#1}
  \def\childdocjob{#1}
  \input{#1}
}
%    \end{macrocode}

% \macro{\childdocby}
% The command |\childdocby| ....
%    \begin{macrocode}
\newcommand{\childdocby}[2][]
{
  \childdocdisable
  \childdoctrue
  \childdocmanualtrue
  \if?#1?\else
    \def\jobname{#2}
  \fi
  \def\childdocjob{#2}
  \input{#2}
  \endinput
}
%    \end{macrocode}

% \macro{\childdocforward}
% The command |\childdocforward| redirects
% compilation to the main file or
% (if the optional argument is given) a child file.
% Parameters are set as if the main file
% or a child file starting with |\childdocof| was compiled.
% Then compilation is handed over to the main file:
%    \begin{macrocode}
\newcommand{\childdocforward}[2][]
{
  \begingroup
    \if?#1?
      \def\childdoctmp
      {
        \def\childdocname{#2}
        \def\childdocjob{#2}
        \def\jobname{#2}
        \input{#2}
        \endinput
      }
    \else
      \def\childdoctmp
      {
        \childdocdisable
        \def\childdocname{#2}
        \childdoctrue
        \includeonly{#2}
        \def\childdocjob{#1}
        \def\jobname{#1}
        \input{#1}
        \endinput
      }
    \fi
    \expandafter
  \endgroup
  \childdoctmp
}
%    \end{macrocode}

% \macro{\childdocforwardprefix}
% The command |\childdocforwardprefix| redirects
% compilation to the main or a child file by means of a pattern.
% The prefix |#1| in the current filename is replaced by |#2|
% and the suffix of the current filename is kept
% (it is assumed that the filename does not contain the substring `|~~~|'
% which is used as a delimiter).
% Compilation is handed over to the new file by |\childdocforward|:
%    \begin{macrocode}
\newcommand{\childdocforwardprefix}[3][]
{
  \begingroup
    \def\childdocextract #2##1~~~{\def\childdoctmp{\childdocforward[#1]{#3##1}}}
    \expandafter\childdocextract\childdocname~~~
    \expandafter
  \endgroup
  \childdoctmp
}
%    \end{macrocode}

% \macro{\childdoc}
% The deprecated macro |\childdoc| is a legacy version of |\childdocmain|:
%    \begin{macrocode}
\newcommand{\childdoc}{\childdocmain}
%    \end{macrocode}

% \macro{\childdocredirect}
% The deprecated macro |\childdocredirect| is a legacy version
% of |\childdocforward| and |\childdocforwardprefix|:
%    \begin{macrocode}
\newcommand{\childdocredirect}[2][]
{
  \begingroup
    \if?#1?
      \def\childdoctmp{\childdocforward{#2}}
    \else
      \def\childdoctmp{\childdocforwardprefix{#1}{#2}}
    \fi
    \expandafter
  \endgroup
  \childdoctmp
}
%    \end{macrocode}

%\iffalse
%</package>
%\fi
%
\endinput
|\\
|\childdocby{|\textit{main}|}|\\
\end{tabular}
\end{center}
%
Both forms have slightly different effects as described above.
The main file is prepared as usual, see \secref{sec:include}.

%%%%%%%%%%%%%%%%%%%%%%%%%%%%%%%%%%%%%%%%%%%%%%%%%%%%%%%%%%%%%%%%%%%%%%%%%%%%%%%%
\subsection{Legacy Detection}
\label{sec:detection}

The directive |\childdocmain| in the main file can detect
whether the complete document or merely a child is to be compiled
even without using the directive |\childdocof|.
This method is deprecated because it is less robust
and there is no compelling reason to use it;
it is merely provided for backward compatibility
and it may be removed in future versions.

If the detection mechanism is to be used,
it is mandatory to correctly specify
the filename of the main file as the argument of |\childdocmain|:
%
\begin{center}
\begin{tabular}{l}
|% \iffalse
%
% childdoc.dtx Copyright (C) 2017-2018 Niklas Beisert
%
% This work may be distributed and/or modified under the
% conditions of the LaTeX Project Public License, either version 1.3
% of this license or (at your option) any later version.
% The latest version of this license is in
%   http://www.latex-project.org/lppl.txt
% and version 1.3 or later is part of all distributions of LaTeX
% version 2005/12/01 or later.
%
% This work has the LPPL maintenance status `maintained'.
%
% The Current Maintainer of this work is Niklas Beisert.
%
% This work consists of the files childdoc.dtx and childdoc.ins
% and the derived files childdoc.def and cdocsamp.tex with
% cdocsch1.tex, cdocsch2.tex, cdocsdrf.tex, cdocsfn1.tex, cdocsfn2.tex.
%
%<package>\ifdefined\childdocmain\endinput\fi
%<package>\ProvidesFile{childdoc.def}[2018/12/30 v2.0 child document driver]
%<samplemain>\ProvidesFile{cdocsamp.tex}[2018/12/30 v2.0 sample for childdoc]
%<*driver>
%\ProvidesFile{childdoc.drv}[2018/12/30 v2.0 childdoc reference manual file]
\PassOptionsToClass{10pt,a4paper}{article}
\documentclass{ltxdoc}

\usepackage[margin=35mm]{geometry}
\usepackage{hyperref}
\usepackage{hyperxmp}
\usepackage[usenames]{color}

\hypersetup{colorlinks=true}
\hypersetup{pdfstartview=FitH}
\hypersetup{pdfpagemode=UseNone}
\hypersetup{pdfsource={}}
\hypersetup{pdflang={en-UK}}
\hypersetup{pdfcopyright={Copyright 2017-2018 Niklas Beisert.
  This work may be distributed and/or modified under the
  conditions of the LaTeX Project Public License, either version 1.3
  of this license or (at your option) any later version.}}
\hypersetup{pdflicenseurl={http://www.latex-project.org/lppl.txt}}
\hypersetup{pdfcontactaddress={ETH Zurich, ITP, HIT K,
  Wolfgang-Pauli-Strasse 27}}
\hypersetup{pdfcontactpostcode={8093}}
\hypersetup{pdfcontactcity={Zurich}}
\hypersetup{pdfcontactcountry={Switzerland}}
\hypersetup{pdfcontactemail={nbeisert@itp.phys.ethz.ch}}
\hypersetup{pdfcontacturl={http://people.phys.ethz.ch/\xmptilde nbeisert/}}

\newcommand{\secref}[1]{\hyperref[#1]{section \ref*{#1}}}

\parskip1ex
\parindent0pt
\let\olditemize\itemize
\def\itemize{\olditemize\parskip0pt}

\begin{document}

\title{The \textsf{childdoc} Package}
\hypersetup{pdftitle={The childdoc Package}}
\author{Niklas Beisert\\[2ex]
  Institut f\"ur Theoretische Physik\\
  Eidgen\"ossische Technische Hochschule Z\"urich\\
  Wolfgang-Pauli-Strasse 27, 8093 Z\"urich, Switzerland\\[1ex]
  \href{mailto:nbeisert@itp.phys.ethz.ch}
  {\texttt{nbeisert@itp.phys.ethz.ch}}}
\hypersetup{pdfauthor={Niklas Beisert}}
\hypersetup{pdfsubject={Manual for the LaTeX2e Package childdoc}}
\date{30 December 2018, \textsf{v2.0}}
\maketitle

\begin{abstract}\noindent
\textsf{childdoc} is a \LaTeXe{} package
that enables the direct compilation
of document sections included by |\include|
to individual files.
\end{abstract}

\begingroup
\parskip0ex
\tableofcontents
\endgroup

%%%%%%%%%%%%%%%%%%%%%%%%%%%%%%%%%%%%%%%%%%%%%%%%%%%%%%%%%%%%%%%%%%%%%%%%%%%%%%%%
%%%%%%%%%%%%%%%%%%%%%%%%%%%%%%%%%%%%%%%%%%%%%%%%%%%%%%%%%%%%%%%%%%%%%%%%%%%%%%%%
\section{Introduction}

\LaTeX{} provides a mechanism to structure a large document (such as a book)
into a main file and several child files (containing the chapters)
using the |\include| command.
This mechanism is beneficial for documents
which span hundreds of pages in order to
make the source file(s) more manageable.
Moreover, compilation can be restricted to
selected child files by means of the |\includeonly| command.
The latter feature can be used to reduce the compilation time while editing
(this was significantly more useful in the earlier days of \LaTeX{})
or to generate a smaller document which is easier to navigate.
Another application of |\includeonly| is to generate
documents consisting of selected parts of the complete document.

However, there are a few drawbacks of the plain |\include| mechanism:
\begin{itemize}
\item
The child files cannot be compiled on their own,
they can only be compiled via the main file.
A naive editing environment
(such as a text editor with an option
to have the current file processed by \LaTeX)
may require one to switch to the main file before compiling;
attempting to compile the child file produces errors.
\item
The main file must be modified (each time)
to adjust the |\includeonly| command
to the present needs. This easily leaves the main file in a messy state.
\item
The generated document will always carry the filename
of the main document. This is inconvenient if
several child files are to be compiled and
to be kept for distribution.
\end{itemize}

The present package provides a simple interface
to make child files individually compilable by \LaTeX{}.
Compiling a child file then has the same effect as compiling
the main file with an |\includeonly| command
to select the appropriate child.
Moreover the generated document will carry the name of the child
rather than the main file.
This resolves all three above issues.

This feature is meant to make the editing of books,
thesis documents and lecture notes somewhat more convenient.
However, the package can also be used efficiently for
composing a series of documents (such as exercise sheets)
which are typically distributed individually.
It then assists the author in generating the individual documents
(potentially in different versions)
as well as a document containing the collected series.
Another application is in developing style files
or other kinds of included material
where compilation of the style file could redirect
to a sample or test file.

%%%%%%%%%%%%%%%%%%%%%%%%%%%%%%%%%%%%%%%%%%%%%%%%%%%%%%%%%%%%%%%%%%%%%%%%%%%%%%%%
%%%%%%%%%%%%%%%%%%%%%%%%%%%%%%%%%%%%%%%%%%%%%%%%%%%%%%%%%%%%%%%%%%%%%%%%%%%%%%%%
\section{Usage}

First of all, the package \textsf{childdoc} is \emph{not} a standard
\LaTeXe{} |.sty| style file! Therefore it needs to be invoked in
a non-standard way.

%%%%%%%%%%%%%%%%%%%%%%%%%%%%%%%%%%%%%%%%%%%%%%%%%%%%%%%%%%%%%%%%%%%%%%%%%%%%%%%%
\subsection{Included Files}
\label{sec:include}

%%%%%%%%%%%%%%%%%%%%%%%%%%%%%%%%%%%%%%%%
\DescribeMacro{\childdocmain}
To use the package, add the commands
\begin{center}
\begin{tabular}{l}
|\input{childdoc.def}|\\
|\childdocmain{}|\\
\end{tabular}
\end{center}
at the very top of the main \LaTeX{} file,
in particular \emph{before} the |\documentclass| statement!
The argument of |\childdocmain| should be left empty
(but it must be present).

%%%%%%%%%%%%%%%%%%%%%%%%%%%%%%%%%%%%%%%%
\DescribeMacro{\childdocof}
Furthermore, add the commands
\begin{center}
\begin{tabular}{l}
|\input{childdoc.def}|\\
|\childdocof{|\textit{main}|}|\\
\end{tabular}
\end{center}
at the top of every child file \textit{child}
which is included by |\include{|\textit{child}|}|
from within the main file
(or at least for those files to be compiled individually).
The argument \textit{main} must be the filename of the main file.

There are a couple of
considerations in setting up the main and child documents:

%%%%%%%%%%%%%%%%%%%%%%%%%%%%%%%%%%%%%%%%
\paragraph{Restrictions.}

Please note the following restrictions:
\begin{itemize}
\item
|\childdocmain| must be called with one argument \textit{main}
to ensure compatibility with earlier version of the package.
It must either be empty (|\childdocmain{}|)
or precisely match the filename of the main file in which it is specified.
See \secref{sec:detection} for further information.
\item
The filename \textit{main} must be specified without the |.tex| extension.
\item
The filename \textit{main} is case sensitive
(even in case-insensitive file systems)
due to internal string comparison.
\item
The argument \textit{main} should be fully expanded, it cannot be a macro.
\item
Subdirectories and special characters should be avoided in filenames.
\item
The command |\childdocmain{|\textit{main}|}| must be followed by a whitespace.
It should not be followed immediately by another command
or by a comment mark `|%|'.
This is because the \TeX{} parser reads the token immediately following
the argument of |\childdocmain| and puts it
at the beginning of every child section;
however, a white\-space is ignored.
\end{itemize}

%%%%%%%%%%%%%%%%%%%%%%%%%%%%%%%%%%%%%%%%
\paragraph{Content of Main File.}

It is advisable to place all content in the child files included by |\include|.
Any output contained in the main file will appear in all child documents
unless suppressed manually;
it cannot be suppressed automatically by the |\includeonly| directive
and thus should normally be avoided.
A method to include some content in the main file
by means of conditional processing is described in \secref{sec:conditional}.

%%%%%%%%%%%%%%%%%%%%%%%%%%%%%%%%%%%%%%%%
\paragraph{Page Numbering.}

When only a part of the document is compiled,
the appropriate numbering of pages
(as well as other status parameters)
is determined from the |.aux| files.
The latter contain information from previous passes.
However this information needs to propagate through
all intermediate child documents.
Therefore the page numbering in child documents may well
be inconsistent until the complete document is compiled at least once.

A useful (if unconventional) way to always ensure a consistent
page numbering is to restart the numbering in each child document
and denote the pages by `\textit{child}|.|\textit{page}'
where \textit{child} represents the chapter/section number of the child file.
This can be achieved by the command
|\numberwithin{page}{|\textit{child}|}|
of the \textsf{amsmath} package
where \textit{child} can be |chapter| or |section|
depending on the chosen structuring.
Alternatively, one can modify the macro |\thepage| appropriately
and reset the counter |page| at the start of each child file.

%%%%%%%%%%%%%%%%%%%%%%%%%%%%%%%%%%%%%%%%%%%%%%%%%%%%%%%%%%%%%%%%%%%%%%%%%%%%%%%%
\subsection{Conditional Processing}
\label{sec:conditional}

The package provides a mechanism to compile different versions
of a document. To customise the versions further some conditional processing
can come in handy to distinguish which version is being compiled.
The package provides two macros to describe the compilation context:

%%%%%%%%%%%%%%%%%%%%%%%%%%%%%%%%%%%%%%%%
\DescribeMacro{\ifchilddoc}
The conditional |\ifchilddoc| distinguishes between the compilation of
child documents and the main document:
%
\begin{center}
|\ifchilddoc |\textit{child-code}| |[|\||else |\textit{main-code}]| \||fi|
\end{center}

%%%%%%%%%%%%%%%%%%%%%%%%%%%%%%%%%%%%%%%%
\DescribeMacro{\childdocname}
\DescribeMacro{\childdocjob}
The macro |\childdocname| contains the filename (without extension)
of the main or child file being processed.
Note that |\childdocjob| will always contain the name of the main file.

%%%%%%%%%%%%%%%%%%%%%%%%%%%%%%%%%%%%%%%%
\paragraph{Title Page.}

Conditional processing can be used to include a title or banner page
in the main document when proper precautions are taken.
Importantly, the code in the main file should ensure that the page counter
(as well as other status parameters which are stored in the |.aux| files)
takes the same value after the conditional processing.
Otherwise the page numbers may take divergent values
depending on which part is compiled.

For example, a title page could be declared by:
%
\begin{center}
\begin{tabular}{l}
|\ifchilddoc\||else|\\
|\addtocounter{page}{-1}|\\
\textit{code for title page}\\
|\newpage|\\
|\||fi|
\end{tabular}
\end{center}
%
A banner page for the child documents can be generated by:
%
\begin{center}
\begin{tabular}{l}
|\ifchilddoc|\\
|\addtocounter{page}{-1}|\\
\textit{code for banner page}\\
|\newpage|\\
|\||fi|
\end{tabular}
\end{center}
%
Here one could write a message such as:
\begin{center}
|This is the part \childdocname{} of \childdocjob{}.|
\end{center}

%%%%%%%%%%%%%%%%%%%%%%%%%%%%%%%%%%%%%%%%%%%%%%%%%%%%%%%%%%%%%%%%%%%%%%%%%%%%%%%%
\subsection{Flags}
\label{sec:flags}

The package makes it easy to generate different versions
of the main or child documents.
To this end compilation flags can be defined
and assigned different default values.
They will be particularly useful in conjunction
with the forwarding mechanism described in \secref{sec:forward}.

For example, it may be useful to have a flag |\version|
which can be set to |draft| or |final|.
The document source will contain some conditional code
depending on the value of |\version|.
Suppose further, the flag should default to |final| for the main file
and to |draft| for child files
which is a natural assignment for editing the document.
This is achieved by placing the following code
in the preamble of the main document
(below the |\childdocmain| directive):
%
\begin{center}
\begin{tabular}{l}
|\ifchilddoc|\\
|\providecommand{\version}{draft}|\\
|\||else|\\
|\providecommand{\version}{final}|\\
|\||fi|
\end{tabular}
\end{center}
%
The definition by |\providecommand| makes sure
that previous definitions are not overwritten.
Further statements |\providecommand{\version}{...}|
can thus be added before the above code to override it.

For the main file, one might add a line
(between |\childdocmain| and the above block)
%
\begin{center}
|%\ifchilddoc\||else\providecommand{\version}{draft}\||fi|
\end{center}
%
which can be uncommented to produce a draft version.
Likewise one can add a line to the very top of a child file
(above the |\childdocof{|\textit{main}|}| directive)
%
\begin{center}
|%\providecommand{\version}{final}|
\end{center}
%
which can be uncommented to produce the final version of this child document.

%%%%%%%%%%%%%%%%%%%%%%%%%%%%%%%%%%%%%%%%%%%%%%%%%%%%%%%%%%%%%%%%%%%%%%%%%%%%%%%%
\subsection{Forwarding}
\label{sec:forward}

Different versions of the main or child documents
using compilation flags as described in \secref{sec:flags}
can be (permanently) stored in different files
for convenient compilation, viewing and distribution.
To this end, the package defines a command
to pass on compilation to a different file:

%%%%%%%%%%%%%%%%%%%%%%%%%%%%%%%%%%%%%%%%
\DescribeMacro{\childdocforward}
The command |\childdocforward| redirects processing to
another source file:
%
\begin{center}
\begin{tabular}{l}
|\input{childdoc.def}|\\
|\childdocforward[|\textit{main}|]{|\textit{dest}|}|\\
\end{tabular}
\end{center}
%
The argument \textit{dest} is the destination file
(without extension).
It should be the main file or one of the child files.
Note that further \textsf{childdoc} directives
such as |\childdocof| and |\childdocforward|
in the indicated file will be processed in this form.
The optional argument \textit{main}
passes on directly to the main file \textit{main}
while pretending to compile the child \textit{dest}.
This form behaves as if \textit{dest}
issues |\childdocof{|\textit{main}|}| right away,
and no further \textsf{childdoc} directives will be processed.

%%%%%%%%%%%%%%%%%%%%%%%%%%%%%%%%%%%%%%%%
\DescribeMacro{\...prefix}
In the alternative form |\childdocforwardprefix|,
%
\begin{center}
\begin{tabular}{l}
|\input{childdoc.def}|\\
|\childdocforwardprefix[|\textit{main}|]{|\textit{prefix}|}{|\textit{dest}|}|
\end{tabular}
\end{center}
%
the destination file is determined by a pattern
depending on the current file:
To make this work, the current file must be called
`{\textit{prefix}\hspace{0.2em}\textit{suffix}}'
with \textit{prefix} matching precisely the argument.
Processing is then passed on to the file
`{\textit{dest}\hspace{0.2em}\textit{suffix}}'.
Surely, the same effect is achieved by
directly specifying the
argument `{\textit{dest}\hspace{0.2em}\textit{suffix}}'
in the first form.
However, that requires to set up a different file
for each child. With the alternative form of the command
all these files can have exactly the same content
which simplifies setting them up and maintaining them.

For example, the following file |draft.tex|
with a compilation flag |\version| as described in \secref{sec:flags}
compiles the main document as a draft:
%
\begin{center}
\begin{tabular}{l}
|\def\version{draft}|\\
|\input{childdoc.def}|\\
|\childdocforward{|\textit{main}|}|
\end{tabular}
\end{center}
%
Likewise, the following files |final|\textit{nn}|.tex|
compile the final version of the child document
|child|\textit{nn}|.tex|:
%
\begin{center}
\begin{tabular}{l}
|\def\version{final}|\\
|\input{childdoc.def}|\\
|\childdocforwardprefix{final}{child}|
\end{tabular}
\end{center}
%

Note that when several versions of a main file and/or of each child file
are to be generated, it may be convenient to set up a |Makefile| or
shell script to automatise the process.

%%%%%%%%%%%%%%%%%%%%%%%%%%%%%%%%%%%%%%%%%%%%%%%%%%%%%%%%%%%%%%%%%%%%%%%%%%%%%%%%
\subsection{Command Line Processing}
\label{sec:commandline}

The effect of redirection files can also be achieved by invoking
the \LaTeX{} compiler with a more elaborate command line.
Most conveniently this should be done as part
of a shell script or a |Makefile|.

When using \textsf{childdoc} in the main file, the following
command lines effectively perform a redirection
(note that depending on the shell being used,
backslashes may have to be doubled: `|\|' $\to$ `|\\|'):
%
\begin{center}
|... -jobname "|\textit{target}|" |\\|"|[\textit{flags}]%
|\input{childdoc.def}\childdocforward[|\textit{main}|]{|\textit{dest}|}"|
\end{center}
%
Here \textit{target} is the name of the output file,
\textit{main} is the name of the main file
and \textit{dest} is the name of the main or child file to be processed
(all filenames without extensions).
The optional argument \textit{main} can be omitted
if \textit{main} matches \textit{dest}.
Optionally, compilation \textit{flags} can be defined via |\def| commands.
This command line makes the \TeX{} engine believe
it is compiling the file \textit{target}
whose content is specified as the latter parameter.
The provided code then forwards the processing to
\textit{main} or \textit{dest} as described in \secref{sec:forward}.

%%%%%%%%%%%%%%%%%%%%%%%%%%%%%%%%%%%%%%%%%%%%%%%%%%%%%%%%%%%%%%%%%%%%%%%%%%%%%%%%
\subsection{Include by Input}
\label{sec:input}

Including child documents by |\include| has some restrictions by design.
Most notably, the content of a child document always occupies
its own set of pages; pages cannot be shared between child documents.
Usually, this behaviour makes perfect sense
because each child document contain an essential part of the document.
However, in some situations it may be desirable to compose
a document from a collection of parts
without having mandatory page breaks between then.
For this case, the package
provides a mechanism to include parts
by |\input| which can also be processed individually.
However, by construction this mechanism
requires manual handling of the content to be output.

%%%%%%%%%%%%%%%%%%%%%%%%%%%%%%%%%%%%%%%%
\DescribeMacro{\ifchilddocmanual}
The main file should be prepared as usual, see \secref{sec:include}.
However, the document body must make a distinction
between processing of an individual part and of the main document, e.g.:
%
\begin{center}
\begin{tabular}{l}
|\ifchilddocmanual|\\
|\input{\childdocname}|\\
|\||else|\\
\textit{document body with }|\input{|\textit{part}|}|\\
|\||fi|
\end{tabular}
\end{center}
%
The conditional |\ifchilddocmanual| is true whenever
a part to be included by |\input| is being compiled,
and the name of the part is stored in |\childdocname|.

%%%%%%%%%%%%%%%%%%%%%%%%%%%%%%%%%%%%%%%%
\DescribeMacro{\childdocby}
Each part to be included by |\input| should start with:
%
\begin{center}
\begin{tabular}{l}
|\input{childdoc.def}|\\
|\childdocby{|\textit{main}|}|\\
\end{tabular}
\end{center}
%
The directive |\childdocby| is similar to |\childdocof|
described in \secref{sec:include},
but the subsequent selection of content must be done manually.
To that end, both |\ifchilddoc| and |\ifchilddocmanual|
will be true upon processing of a part,
and the name of the part is stored in |\childdocname|.
Note that |\jobname| will be set to the filename of the current part
so that each part receives an individual |.aux| file
that does not interfere with the |.aux| file(s) of the main document.
This behaviour can be altered by the alternative form
|\childdocby[*]{|\textit{main}|}| (with a non-empty optional argument)
which uses the |.aux| file of the main document
by setting |\jobname| to \textit{main}.

%%%%%%%%%%%%%%%%%%%%%%%%%%%%%%%%%%%%%%%%%%%%%%%%%%%%%%%%%%%%%%%%%%%%%%%%%%%%%%%%
\subsection{Driver Development}
\label{sec:driver}

The \textsf{childdoc} mechanism can also be use for the development
of definition files such as \LaTeX{} styles or classes.
This case differs from the above setup with multiple parts
included by |\include| in that no |\includeonly| should be invoked.
This can be achieved by starting the include file
(before |\ProvidesPackage|) with:
%
\begin{center}
\begin{tabular}{l}
|\input{childdoc.def}|\\
|\childdocforward{|\textit{main}|}|\\
\end{tabular}
\end{center}
%
or alternatively with:
%
\begin{center}
\begin{tabular}{l}
|\input{childdoc.def}|\\
|\childdocby{|\textit{main}|}|\\
\end{tabular}
\end{center}
%
Both forms have slightly different effects as described above.
The main file is prepared as usual, see \secref{sec:include}.

%%%%%%%%%%%%%%%%%%%%%%%%%%%%%%%%%%%%%%%%%%%%%%%%%%%%%%%%%%%%%%%%%%%%%%%%%%%%%%%%
\subsection{Legacy Detection}
\label{sec:detection}

The directive |\childdocmain| in the main file can detect
whether the complete document or merely a child is to be compiled
even without using the directive |\childdocof|.
This method is deprecated because it is less robust
and there is no compelling reason to use it;
it is merely provided for backward compatibility
and it may be removed in future versions.

If the detection mechanism is to be used,
it is mandatory to correctly specify
the filename of the main file as the argument of |\childdocmain|:
%
\begin{center}
\begin{tabular}{l}
|\input{childdoc.def}|\\
|\childdocmain{|\textit{main}|}|\\
\end{tabular}
\end{center}
%
If |\jobname| does not match the argument \textit{main} of |\childdocmain|,
it is assumed that |\jobname| points to the child file to be compiled.
When using |\childdocmain| with the main file specified as argument,
it suffices to start a child file
with just |\input{|\textit{main}|}|
without loading of the package and using |\childdocof|.
If instead all processing is done
with the appropriate \textsf{childdoc} directives,
the argument of \textit{main} of |\childdocmain| can be empty.

An alternative version of the command line processing described
in \secref{sec:commandline} using the detection mechanism reads:
%
\begin{center}
|... -jobname "|\textit{target}|" "|[\textit{flags}]%
[|\def\jobname{|\textit{dest}|}|]|\input{|\textit{main}|}"|
\end{center}

%%%%%%%%%%%%%%%%%%%%%%%%%%%%%%%%%%%%%%%%%%%%%%%%%%%%%%%%%%%%%%%%%%%%%%%%%%%%%%%%
\subsection{Manual Code}
\label{sec:manual}

In case one cannot be certain whether the definitions file |childdoc.def|
is installed on the target \TeX{} distribution
and one prefers not to ship it,
it is conceivable to paste a few relevant commands into the sources.

To that end, drop all statements |\input{childdoc.def}|
and perform the replacements as outlined below.
Instead of |\childdocmain{|\textit{main}|}| add the following code
to the top of the main file:
%
\begin{center}
\begin{tabular}{l}
|\||ifdefined\childdocname\endinput\||fi\newif\ifchilddoc|\\
|\edef\childdocname{\scantokens\expandafter{\jobname\noexpand}}|\\
|\def\childdocmain{|\textit{main}|}\||ifx\childdocmain\childdocname\||else|\\
|\childdoctrue\includeonly{\childdocname}\let\jobname\childdocmain\||fi|\\
\end{tabular}
\end{center}
%
Instead of |\childdocof{|\textit{main}|}| just include the main file
at the top of each child file:
%
\begin{center}
|\input{|\textit{main}|}|
\end{center}
%
A simple redirection |\childdocforward{|\textit{dest}|}| is achieved by:
%
\begin{center}
|\def\jobname{|\textit{dest}|}\input{\jobname}|
\end{center}
%
The redirection with prefix
|\childdocforwardprefix[|\textit{prefix}|]{|\textit{dest}|}|
is accomplished by:
%
\begin{center}
\begin{tabular}{l}
|{\edef\jobname{\scantokens\expandafter{\jobname\noexpand}}|\\
|\def\redirectjob |\textit{prefix}|#1~~~{\gdef\jobname{|\textit{dest}|#1}}|\\
|\expandafter\redirectjob\jobname~~~}\input{\jobname}|
\end{tabular}
\end{center}

In an alternative approach,
child documents can be compiled by a specific command line
without additional code or specific definitions:
%
\begin{center}
|... -jobname "|\textit{target}|" "|[\textit{flags}]%
|\includeonly{|\textit{dest}|}\input{|\textit{main}|}"|
\end{center}
%

%%%%%%%%%%%%%%%%%%%%%%%%%%%%%%%%%%%%%%%%%%%%%%%%%%%%%%%%%%%%%%%%%%%%%%%%%%%%%%%%
%%%%%%%%%%%%%%%%%%%%%%%%%%%%%%%%%%%%%%%%%%%%%%%%%%%%%%%%%%%%%%%%%%%%%%%%%%%%%%%%
\section{Information}

%%%%%%%%%%%%%%%%%%%%%%%%%%%%%%%%%%%%%%%%%%%%%%%%%%%%%%%%%%%%%%%%%%%%%%%%%%%%%%%%
\subsection{Copyright}

Copyright \copyright{} 2017--2018 Niklas Beisert

This work may be distributed and/or modified under the
conditions of the \LaTeX{} Project Public License, either version 1.3
of this license or (at your option) any later version.
The latest version of this license is in
  \url{http://www.latex-project.org/lppl.txt}
and version 1.3 or later is part of all distributions of \LaTeX{}
version 2005/12/01 or later.

This work has the LPPL maintenance status `maintained'.

The Current Maintainer of this work is Niklas Beisert.

This work consists of the files |README.txt|, |childdoc.ins| and |childdoc.dtx|
as well as the derived files |childdoc.def|, |cdocsamp.tex|
with |cdocsch1.tex|, |cdocsch2.tex|, |cdocspt3.tex|, |cdocspt4.tex|,
|cdocsdrf.tex|, |cdocsfn1.tex|, |cdocsfn2.tex|
as well as |childdoc.pdf|.

%%%%%%%%%%%%%%%%%%%%%%%%%%%%%%%%%%%%%%%%%%%%%%%%%%%%%%%%%%%%%%%%%%%%%%%%%%%%%%%%
\subsection{Files and Installation}

The package consists of the files:
%
\begin{center}
\begin{tabular}{ll}
    |README.txt|   & readme file \\
    |childdoc.ins| & installation file \\
    |childdoc.dtx| & source file \\
    |childdoc.def| & definition file \\
    |cdocsamp.tex| & sample main file \\
    |cdocsch1.tex| & sample include file \\
    |cdocsch2.tex| & sample include file \\
    |cdocspt3.tex| & sample part file \\
    |cdocspt4.tex| & sample part file \\
    |cdocsdrf.tex| & sample redirection file \\
    |cdocsfn1.tex| & sample redirection file \\
    |cdocsfn2.tex| & sample redirection file \\
    |childdoc.pdf| & manual
\end{tabular}
\end{center}
%
The distribution consists of the files
|README.txt|, |childdoc.ins| and |childdoc.dtx|.
%
\begin{itemize}
\item
Run (pdf)\LaTeX{} on |childdoc.dtx|
to compile the manual |childdoc.pdf| (this file).
\item
Run \LaTeX{} on |childdoc.ins| to create the definitions file |childdoc.def|
and the sample |cdocsamp.tex| with include files
|cdocsch1.tex|, |cdocsch2.tex|, |cdocspt3.tex|, |cdocspt4.tex|,
|cdocsdrf.tex|, |cdocsfn1.tex|, |cdocsfn2.tex|.
Then copy the file |childdoc.def| to an appropriate directory of your \LaTeX{}
distribution, e.g.\ \textit{texmf-root}|/tex/latex/childdoc|.
\end{itemize}

%%%%%%%%%%%%%%%%%%%%%%%%%%%%%%%%%%%%%%%%%%%%%%%%%%%%%%%%%%%%%%%%%%%%%%%%%%%%%%%%
\subsection{Related CTAN Packages}

There are several other packages which offer a similar functionality:
%
\begin{itemize}
\item
The packages
\href{http://ctan.org/pkg/docmute}{\textsf{docmute}},
\href{http://ctan.org/pkg/includex}{\textsf{includex}} and
\href{http://ctan.org/pkg/standalone}{\textsf{standalone}}
provide commands to include only the document body of
a child file thus allowing both files to be compiled individually.
\item
The packages \href{http://ctan.org/pkg/subdocs}{\textsf{subdocs}}
and \href{http://ctan.org/pkg/subfiles}{\textsf{subfiles}}
provide structures in which the main and child documents can be
encapsulated and allowing them to be compiled individually.
The inclusion mechanism is different from the conventional |\include|.
\item
The package \href{http://ctan.org/pkg/combine}{\textsf{combine}}
is an elaborate solution to combine several documents into one.
\end{itemize}
%
See also the CTAN topic \href{http://ctan.org/topic/subdocs}{\textsf{subdocs}}
for further related packages.
The present package differs from the above solutions in that
a document structure constructed with the conventional |\include| mechanism
just needs two extra commands at the top of every file
such that all constituent files can be compiled individually.

%%%%%%%%%%%%%%%%%%%%%%%%%%%%%%%%%%%%%%%%%%%%%%%%%%%%%%%%%%%%%%%%%%%%%%%%%%%%%%%%
%\subsection{Feature Suggestions}
%
%The following is a list of features which may be useful for future
%versions of this package:
%%
%\begin{itemize}
%\item
%\ldots
%\end{itemize}

%%%%%%%%%%%%%%%%%%%%%%%%%%%%%%%%%%%%%%%%%%%%%%%%%%%%%%%%%%%%%%%%%%%%%%%%%%%%%%%%
\subsection{Revision History}

%%%%%%%%%%%%%%%%%%%%%%%%%%%%%%%%%%%%%%%%
\paragraph{v2.0:} 2018/12/30

\begin{itemize}
\item
immediate forward processing
\item
added |\childdocby| mechanism
\item
manual restructured
\end{itemize}

%%%%%%%%%%%%%%%%%%%%%%%%%%%%%%%%%%%%%%%%
\paragraph{v1.6:} 2018/01/17

\begin{itemize}
\item
application for development of include files
\item
corrections to manual
\end{itemize}

%%%%%%%%%%%%%%%%%%%%%%%%%%%%%%%%%%%%%%%%
\paragraph{v1.5:} 2017/05/21

\begin{itemize}
\item
more complete structuring introduced
\item
|\childdocof| introduced
\item
|\childdoc| renamed to |\childdocmain|
\item
|\childredirect| renamed to |\childdocforward| and |\childdocforwardprefix|
and functionality expanded
\end{itemize}

%%%%%%%%%%%%%%%%%%%%%%%%%%%%%%%%%%%%%%%%
\paragraph{v1.0:} 2017/04/27

\begin{itemize}
\item
manual and install package
\item
first version published on CTAN
\end{itemize}

%%%%%%%%%%%%%%%%%%%%%%%%%%%%%%%%%%%%%%%%
\paragraph{v0.6:} 2017/04/26

\begin{itemize}
\item
redirection mechanism added
\end{itemize}

%%%%%%%%%%%%%%%%%%%%%%%%%%%%%%%%%%%%%%%%
\paragraph{v0.5:} 2017/04/26

\begin{itemize}
\item
functionality in definition file
\end{itemize}


%%%%%%%%%%%%%%%%%%%%%%%%%%%%%%%%%%%%%%%%%%%%%%%%%%%%%%%%%%%%%%%%%%%%%%%%%%%%%%%%
%%%%%%%%%%%%%%%%%%%%%%%%%%%%%%%%%%%%%%%%%%%%%%%%%%%%%%%%%%%%%%%%%%%%%%%%%%%%%%%%
%%%%%%%%%%%%%%%%%%%%%%%%%%%%%%%%%%%%%%%%%%%%%%%%%%%%%%%%%%%%%%%%%%%%%%%%%%%%%%%%
\appendix

\settowidth\MacroIndent{\rmfamily\scriptsize 000\ }

 \DocInput{childdoc.dtx}

\end{document}
%</driver>
% \fi
%
% %%%%%%%%%%%%%%%%%%%%%%%%%%%%%%%%%%%%%%%%%%%%%%%%%%%%%%%%%%%%%%%%%%%%%%%%%%%%%%
% %%%%%%%%%%%%%%%%%%%%%%%%%%%%%%%%%%%%%%%%%%%%%%%%%%%%%%%%%%%%%%%%%%%%%%%%%%%%%%
% \section{Sample}
%\iffalse
%<*samplemain>
%\fi
%
% The following presents a sample document
% with two chapters, two parts, a title page,
% a compile flag as well as three forwarding files to set the flag.
% It consists of eight |.tex| files:
% \begin{center}
% \begin{tabular}{ll}
% |cdocsamp.tex|&main file\\
% |cdocsch1.tex|&include file for chapter 1\\
% |cdocsch2.tex|&include file for chapter 2\\
% |cdocspt3.tex|&include file for part 3\\
% |cdocspt4.tex|&include file for part 4\\
% |cdocsdrf.tex|&forwarding file for main file in draft mode\\
% |cdocsfi1.tex|&forwarding file for final version of chapter 1\\
% |cdocsfi2.tex|&forwarding file for final version of chapter 2\\
% \end{tabular}
% \end{center}
% Each of the eight files can be compiled directly by the \LaTeX{} compiler.
%
% %%%%%%%%%%%%%%%%%%%%%%%%%%%%%%%%%%%%%%
% \paragraph{Main File.}
%
% The main file is called |cdocsamp.tex|.
%
% Load the \textsf{childdoc} definitions and
% declare the filename for the main document:
%    \begin{macrocode}
\input{childdoc.def}
\childdocmain{}
%    \end{macrocode}

% Optional override for |\version| flag:
%    \begin{macrocode}
%%\ifchilddoc\else\providecommand{\version}{draft}\fi
%    \end{macrocode}

% Define the default values for the |\version| flag
% (|final| for the main file and |draft| for childs):
%    \begin{macrocode}
\ifchilddoc
\providecommand{\version}{draft}
\else
\providecommand{\version}{final}
\fi
%    \end{macrocode}

% Load the standard document class:
%    \begin{macrocode}
\documentclass[12pt]{article}
%    \end{macrocode}

% Start the document body:
%    \begin{macrocode}
\begin{document}
%    \end{macrocode}

% Declare a title page.
% Print title, part of document being processed and version flag:
%    \begin{macrocode}
\addtocounter{page}{-1}
\begin{center}
{\LARGE\bfseries{}childdoc example\par}
\vspace{1cm}
\ifchilddoc
\ifchilddocmanual part\else chapter\fi:
`\childdocname' of `\childdocjob'\par
\else
main document: `\childdocjob'\par
\fi
version: \version\par
\end{center}
\newpage
%    \end{macrocode}

% Manually include selected file,
% otherwise process as usual:
%    \begin{macrocode}
\ifchilddocmanual
\section*{part `\childdocname'}
\input{\childdocname}
\else
%    \end{macrocode}

% Include the two chapters:
%    \begin{macrocode}
\include{cdocsch1}
\include{cdocsch2}
%    \end{macrocode}

% Include the two parts unless only chapters should be displayed:
%    \begin{macrocode}
\ifchilddoc\else
\section{part three}
\input{cdocspt3}
\section{part four}
\input{cdocspt4}
\fi
%    \end{macrocode}

% Process as usual until here:
%    \begin{macrocode}
\fi
%    \end{macrocode}

% End of document body:
%    \begin{macrocode}
\end{document}
%    \end{macrocode}
%\iffalse
%</samplemain>
%\fi
%
% %%%%%%%%%%%%%%%%%%%%%%%%%%%%%%%%%%%%%%
% \paragraph{Chapter Include Files.}
%
% The include files are called |cdocsch1.tex| and |cdocsch2.tex|.
%
%\iffalse
%<*samplechap1|samplechap2>
%\fi

% Optional override for |\version| flag:
%    \begin{macrocode}
%%\providecommand{\version}{final}
%    \end{macrocode}

% Include the main document:
%    \begin{macrocode}
\input{childdoc.def}
\childdocof{cdocsamp}
%    \end{macrocode}

%\iffalse
%</samplechap1|samplechap2>
%\fi
%
%\iffalse
%<*samplechap1>
%\fi
% Some text for chapter 1:
%    \begin{macrocode}
\section{one}
some text in chapter one
%    \end{macrocode}

%\iffalse
%</samplechap1>
%\fi
% Some text for chapter 2:
%\iffalse
%<*samplechap2>
%\fi
%    \begin{macrocode}
\section{two}
more text in chapter two
%    \end{macrocode}

%\iffalse
%</samplechap2>
%\fi
%
% %%%%%%%%%%%%%%%%%%%%%%%%%%%%%%%%%%%%%%
% \paragraph{Part Include Files.}
%
% The include files are called |cdocspt3.tex| and |cdocspt4.tex|.
%
%\iffalse
%<*samplepart3|samplepart4>
%\fi

% Optional override for |\version| flag:
%    \begin{macrocode}
%%\providecommand{\version}{final}
%    \end{macrocode}

% Include the main document:
%    \begin{macrocode}
\input{childdoc.def}
\childdocby{cdocsamp}
%    \end{macrocode}

%\iffalse
%</samplepart3|samplepart4>
%\fi
%
%\iffalse
%<*samplepart3>
%\fi
% Some text for part 3:
%    \begin{macrocode}
some text in part three
%    \end{macrocode}

%\iffalse
%</samplepart3>
%\fi
% Some text for part 4:
%\iffalse
%<*samplepart4>
%\fi
%    \begin{macrocode}
more text in part four
%    \end{macrocode}

%\iffalse
%</samplepart4>
%\fi
%
% %%%%%%%%%%%%%%%%%%%%%%%%%%%%%%%%%%%%%%
% \paragraph{Forwarding for a Complete Draft.}
%
% The following forwarding file |cdocsdrf.tex|
% compiles the main document in draft mode:
%\iffalse
%<*sampledraft>
%\fi
%    \begin{macrocode}
\def\version{draft}
\input{childdoc.def}
\childdocforward{cdocsamp}
%    \end{macrocode}

%\iffalse
%</sampledraft>
%\fi
%
% %%%%%%%%%%%%%%%%%%%%%%%%%%%%%%%%%%%%%%
% \paragraph{Forwarding for Final Version of the Chapters.}
%
% The following forwarding files |cdocsfn1.tex| and |cdocsfn2.tex|
% (with identical content)
% compile the final versions of the child documents
% |cdocsch1.tex| and |cdocsch2.tex|, respectively:
%\iffalse
%<*samplefinal>
%\fi
%    \begin{macrocode}
\def\version{final}
\input{childdoc.def}
\childdocforwardprefix[cdocsamp]{cdocsfn}{cdocsch}
%    \end{macrocode}

%\iffalse
%</samplefinal>
%\fi
%
% %%%%%%%%%%%%%%%%%%%%%%%%%%%%%%%%%%%%%%
% \paragraph{Command Line Processing.}
%
% The following three command lines generate the output files
% |cdocscld|, |cdocscl1| and |cdocscl2|
% which should be identical to
% |cdocsdrf|, |cdocsch1| and |cdocsfn2|, respectively:
% \begin{center}
% \begin{tabular}{l}
% |latex -jobname cdocscld \|\\
% |  "\def\version{draft}\input{childdoc.def}\childdocforward{cdocsamp}"|\\
% |latex -jobname cdocscl1 \|\\
% |  "\input{childdoc.def}\childdocforward[cdocsamp]{cdocsch1}"|\\
% |latex -jobname cdocscl2 \|\\
% |  "\def\version{final}\input{childdoc.def}\childdocforward{cdocsch2}"|
% \end{tabular}
% \end{center}
% Note that the trailing backslash on each first line
% merely continues the input to the second line
% (for convenient cut ant paste).
% Furthermore, the command |latex| can be replaced by any
% of its alternative versions such as |pdflatex|.
%
% %%%%%%%%%%%%%%%%%%%%%%%%%%%%%%%%%%%%%%%%%%%%%%%%%%%%%%%%%%%%%%%%%%%%%%%%%%%%%%
% %%%%%%%%%%%%%%%%%%%%%%%%%%%%%%%%%%%%%%%%%%%%%%%%%%%%%%%%%%%%%%%%%%%%%%%%%%%%%%
% \section{Implementation}
%\iffalse
%<*package>
%\fi
%
% This section describes the definitions file |childdoc.def|.

% The definitions cannot be loaded using |\usepackage| or |\RequirePackage|
% which has a mechanism to prevent loading a style file more than once.
% When loading the definitions by means of |\input|
% multiple instances have to be prevented manually:
%\iffalse
%This code needs to be before the `\ProvidesFile' directive
%which is defined at the beginning of this file.
%Therefore it is also placed there and commented out here.
%</package>
%<*discard>
%\fi
%    \begin{macrocode}
\ifdefined\childdocmain\endinput\fi
%    \end{macrocode}
%\iffalse
%</discard>
%<*package>
%\fi
%
% \macro{\ifchilddoc}
% \macro{\ifchilddocmanual}
% The conditional |\ifchilddoc| tells whether a
% child (true) or main (false) document is being compiled.
% The conditional |\ifchilddocmanual| tells whether
% the |\includeonly| mechanism is used (false) or
% the selection of child files must be performed manually (true).
% The definitions initialise to false:
%    \begin{macrocode}
\newif\ifchilddoc
\newif\ifchilddocmanual
%    \end{macrocode}

% \macro{\childdocname}
% \macro{\childdocjob}
% The macro |\childdocname| stores the name of the main document
% to be compiled. The macro |\childdocjob| stores the name of
% the document on which the \LaTeX{} compiler was originally invoked.
% The content of |\jobname| cannot be compared
% to filenames specified in the source due to different catcodes.
% The following code rescans |\jobname|, stores the result
% in |\childdocname| and saves a copy in |\childdocjob|:
%    \begin{macrocode}
\edef\childdocname{\scantokens\expandafter{\jobname\noexpand}}
\let\childdocjob\childdocname
%    \end{macrocode}

% \macro{\childdocdisable}
% The macro |\childdocdisable| prevents the main file
% from being processed more than once.
% At this stage, the main document command |\childdocmain|
% is assumed to be called once again where it should do nothing.
% Any subsequent call to it should prevent
% a secondary processing of the main document
% It overwrites the forwarding commands
% |\childdocof| and |\childdocforward|
% with empty macros to prevent further inclusions of the main document:
%    \begin{macrocode}
\newcommand{\childdocdisable}
{
  \renewcommand{\childdocmain}[1]{\renewcommand{\childdocmain}[1]{\endinput}}
  \renewcommand{\childdocof}[1]{}
  \renewcommand{\childdocby}[2][]{}
  \renewcommand{\childdocforward}[2][]{}
  \renewcommand{\childdocdisable}{}
}
%    \end{macrocode}

% \macro{\childdocmain}
% The macro |\childdocmain| is to be called at the top of the main file
% with nothing or the main filename (without extension) as argument.
% First, it breaks loops.
% If the argument is not empty and does not match |\childdocname|
% (which is set by the first inclusion of |childdoc.def|),
% |\ifchilddoc| is set to true, |\includeonly| is applied to the child file
% and |\jobname| is set to the main file
% (for proper handling of |.aux| files):
%    \begin{macrocode}
\newcommand{\childdocmain}[1]
{
  \childdocdisable\childdocmain{}
  \if?#1?\else
    \begingroup
      \def\childdoctmp{#1}
      \ifx\childdoctmp\childdocname
        \def\childdoctmp{}
      \else
        \def\childdoctmp
        {
          \childdoctrue
          \includeonly{\childdocname}
          \def\childdocjob{#1}
          \def\jobname{#1}
        }
      \fi
      \expandafter
    \endgroup
    \childdoctmp
  \fi
}
%    \end{macrocode}

% \macro{\childdocof}
% The command |\childdocof| redirects
% compilation to the main file |#1|.
%    \begin{macrocode}
\newcommand{\childdocof}[1]
{
  \childdocdisable
  \childdoctrue
  \includeonly{\childdocname}
  \def\jobname{#1}
  \def\childdocjob{#1}
  \input{#1}
}
%    \end{macrocode}

% \macro{\childdocby}
% The command |\childdocby| ....
%    \begin{macrocode}
\newcommand{\childdocby}[2][]
{
  \childdocdisable
  \childdoctrue
  \childdocmanualtrue
  \if?#1?\else
    \def\jobname{#2}
  \fi
  \def\childdocjob{#2}
  \input{#2}
  \endinput
}
%    \end{macrocode}

% \macro{\childdocforward}
% The command |\childdocforward| redirects
% compilation to the main file or
% (if the optional argument is given) a child file.
% Parameters are set as if the main file
% or a child file starting with |\childdocof| was compiled.
% Then compilation is handed over to the main file:
%    \begin{macrocode}
\newcommand{\childdocforward}[2][]
{
  \begingroup
    \if?#1?
      \def\childdoctmp
      {
        \def\childdocname{#2}
        \def\childdocjob{#2}
        \def\jobname{#2}
        \input{#2}
        \endinput
      }
    \else
      \def\childdoctmp
      {
        \childdocdisable
        \def\childdocname{#2}
        \childdoctrue
        \includeonly{#2}
        \def\childdocjob{#1}
        \def\jobname{#1}
        \input{#1}
        \endinput
      }
    \fi
    \expandafter
  \endgroup
  \childdoctmp
}
%    \end{macrocode}

% \macro{\childdocforwardprefix}
% The command |\childdocforwardprefix| redirects
% compilation to the main or a child file by means of a pattern.
% The prefix |#1| in the current filename is replaced by |#2|
% and the suffix of the current filename is kept
% (it is assumed that the filename does not contain the substring `|~~~|'
% which is used as a delimiter).
% Compilation is handed over to the new file by |\childdocforward|:
%    \begin{macrocode}
\newcommand{\childdocforwardprefix}[3][]
{
  \begingroup
    \def\childdocextract #2##1~~~{\def\childdoctmp{\childdocforward[#1]{#3##1}}}
    \expandafter\childdocextract\childdocname~~~
    \expandafter
  \endgroup
  \childdoctmp
}
%    \end{macrocode}

% \macro{\childdoc}
% The deprecated macro |\childdoc| is a legacy version of |\childdocmain|:
%    \begin{macrocode}
\newcommand{\childdoc}{\childdocmain}
%    \end{macrocode}

% \macro{\childdocredirect}
% The deprecated macro |\childdocredirect| is a legacy version
% of |\childdocforward| and |\childdocforwardprefix|:
%    \begin{macrocode}
\newcommand{\childdocredirect}[2][]
{
  \begingroup
    \if?#1?
      \def\childdoctmp{\childdocforward{#2}}
    \else
      \def\childdoctmp{\childdocforwardprefix{#1}{#2}}
    \fi
    \expandafter
  \endgroup
  \childdoctmp
}
%    \end{macrocode}

%\iffalse
%</package>
%\fi
%
\endinput
|\\
|\childdocmain{|\textit{main}|}|\\
\end{tabular}
\end{center}
%
If |\jobname| does not match the argument \textit{main} of |\childdocmain|,
it is assumed that |\jobname| points to the child file to be compiled.
When using |\childdocmain| with the main file specified as argument,
it suffices to start a child file
with just |\input{|\textit{main}|}|
without loading of the package and using |\childdocof|.
If instead all processing is done
with the appropriate \textsf{childdoc} directives,
the argument of \textit{main} of |\childdocmain| can be empty.

An alternative version of the command line processing described
in \secref{sec:commandline} using the detection mechanism reads:
%
\begin{center}
|... -jobname "|\textit{target}|" "|[\textit{flags}]%
[|\def\jobname{|\textit{dest}|}|]|\input{|\textit{main}|}"|
\end{center}

%%%%%%%%%%%%%%%%%%%%%%%%%%%%%%%%%%%%%%%%%%%%%%%%%%%%%%%%%%%%%%%%%%%%%%%%%%%%%%%%
\subsection{Manual Code}
\label{sec:manual}

In case one cannot be certain whether the definitions file |childdoc.def|
is installed on the target \TeX{} distribution
and one prefers not to ship it,
it is conceivable to paste a few relevant commands into the sources.

To that end, drop all statements |% \iffalse
%
% childdoc.dtx Copyright (C) 2017-2018 Niklas Beisert
%
% This work may be distributed and/or modified under the
% conditions of the LaTeX Project Public License, either version 1.3
% of this license or (at your option) any later version.
% The latest version of this license is in
%   http://www.latex-project.org/lppl.txt
% and version 1.3 or later is part of all distributions of LaTeX
% version 2005/12/01 or later.
%
% This work has the LPPL maintenance status `maintained'.
%
% The Current Maintainer of this work is Niklas Beisert.
%
% This work consists of the files childdoc.dtx and childdoc.ins
% and the derived files childdoc.def and cdocsamp.tex with
% cdocsch1.tex, cdocsch2.tex, cdocsdrf.tex, cdocsfn1.tex, cdocsfn2.tex.
%
%<package>\ifdefined\childdocmain\endinput\fi
%<package>\ProvidesFile{childdoc.def}[2018/12/30 v2.0 child document driver]
%<samplemain>\ProvidesFile{cdocsamp.tex}[2018/12/30 v2.0 sample for childdoc]
%<*driver>
%\ProvidesFile{childdoc.drv}[2018/12/30 v2.0 childdoc reference manual file]
\PassOptionsToClass{10pt,a4paper}{article}
\documentclass{ltxdoc}

\usepackage[margin=35mm]{geometry}
\usepackage{hyperref}
\usepackage{hyperxmp}
\usepackage[usenames]{color}

\hypersetup{colorlinks=true}
\hypersetup{pdfstartview=FitH}
\hypersetup{pdfpagemode=UseNone}
\hypersetup{pdfsource={}}
\hypersetup{pdflang={en-UK}}
\hypersetup{pdfcopyright={Copyright 2017-2018 Niklas Beisert.
  This work may be distributed and/or modified under the
  conditions of the LaTeX Project Public License, either version 1.3
  of this license or (at your option) any later version.}}
\hypersetup{pdflicenseurl={http://www.latex-project.org/lppl.txt}}
\hypersetup{pdfcontactaddress={ETH Zurich, ITP, HIT K,
  Wolfgang-Pauli-Strasse 27}}
\hypersetup{pdfcontactpostcode={8093}}
\hypersetup{pdfcontactcity={Zurich}}
\hypersetup{pdfcontactcountry={Switzerland}}
\hypersetup{pdfcontactemail={nbeisert@itp.phys.ethz.ch}}
\hypersetup{pdfcontacturl={http://people.phys.ethz.ch/\xmptilde nbeisert/}}

\newcommand{\secref}[1]{\hyperref[#1]{section \ref*{#1}}}

\parskip1ex
\parindent0pt
\let\olditemize\itemize
\def\itemize{\olditemize\parskip0pt}

\begin{document}

\title{The \textsf{childdoc} Package}
\hypersetup{pdftitle={The childdoc Package}}
\author{Niklas Beisert\\[2ex]
  Institut f\"ur Theoretische Physik\\
  Eidgen\"ossische Technische Hochschule Z\"urich\\
  Wolfgang-Pauli-Strasse 27, 8093 Z\"urich, Switzerland\\[1ex]
  \href{mailto:nbeisert@itp.phys.ethz.ch}
  {\texttt{nbeisert@itp.phys.ethz.ch}}}
\hypersetup{pdfauthor={Niklas Beisert}}
\hypersetup{pdfsubject={Manual for the LaTeX2e Package childdoc}}
\date{30 December 2018, \textsf{v2.0}}
\maketitle

\begin{abstract}\noindent
\textsf{childdoc} is a \LaTeXe{} package
that enables the direct compilation
of document sections included by |\include|
to individual files.
\end{abstract}

\begingroup
\parskip0ex
\tableofcontents
\endgroup

%%%%%%%%%%%%%%%%%%%%%%%%%%%%%%%%%%%%%%%%%%%%%%%%%%%%%%%%%%%%%%%%%%%%%%%%%%%%%%%%
%%%%%%%%%%%%%%%%%%%%%%%%%%%%%%%%%%%%%%%%%%%%%%%%%%%%%%%%%%%%%%%%%%%%%%%%%%%%%%%%
\section{Introduction}

\LaTeX{} provides a mechanism to structure a large document (such as a book)
into a main file and several child files (containing the chapters)
using the |\include| command.
This mechanism is beneficial for documents
which span hundreds of pages in order to
make the source file(s) more manageable.
Moreover, compilation can be restricted to
selected child files by means of the |\includeonly| command.
The latter feature can be used to reduce the compilation time while editing
(this was significantly more useful in the earlier days of \LaTeX{})
or to generate a smaller document which is easier to navigate.
Another application of |\includeonly| is to generate
documents consisting of selected parts of the complete document.

However, there are a few drawbacks of the plain |\include| mechanism:
\begin{itemize}
\item
The child files cannot be compiled on their own,
they can only be compiled via the main file.
A naive editing environment
(such as a text editor with an option
to have the current file processed by \LaTeX)
may require one to switch to the main file before compiling;
attempting to compile the child file produces errors.
\item
The main file must be modified (each time)
to adjust the |\includeonly| command
to the present needs. This easily leaves the main file in a messy state.
\item
The generated document will always carry the filename
of the main document. This is inconvenient if
several child files are to be compiled and
to be kept for distribution.
\end{itemize}

The present package provides a simple interface
to make child files individually compilable by \LaTeX{}.
Compiling a child file then has the same effect as compiling
the main file with an |\includeonly| command
to select the appropriate child.
Moreover the generated document will carry the name of the child
rather than the main file.
This resolves all three above issues.

This feature is meant to make the editing of books,
thesis documents and lecture notes somewhat more convenient.
However, the package can also be used efficiently for
composing a series of documents (such as exercise sheets)
which are typically distributed individually.
It then assists the author in generating the individual documents
(potentially in different versions)
as well as a document containing the collected series.
Another application is in developing style files
or other kinds of included material
where compilation of the style file could redirect
to a sample or test file.

%%%%%%%%%%%%%%%%%%%%%%%%%%%%%%%%%%%%%%%%%%%%%%%%%%%%%%%%%%%%%%%%%%%%%%%%%%%%%%%%
%%%%%%%%%%%%%%%%%%%%%%%%%%%%%%%%%%%%%%%%%%%%%%%%%%%%%%%%%%%%%%%%%%%%%%%%%%%%%%%%
\section{Usage}

First of all, the package \textsf{childdoc} is \emph{not} a standard
\LaTeXe{} |.sty| style file! Therefore it needs to be invoked in
a non-standard way.

%%%%%%%%%%%%%%%%%%%%%%%%%%%%%%%%%%%%%%%%%%%%%%%%%%%%%%%%%%%%%%%%%%%%%%%%%%%%%%%%
\subsection{Included Files}
\label{sec:include}

%%%%%%%%%%%%%%%%%%%%%%%%%%%%%%%%%%%%%%%%
\DescribeMacro{\childdocmain}
To use the package, add the commands
\begin{center}
\begin{tabular}{l}
|\input{childdoc.def}|\\
|\childdocmain{}|\\
\end{tabular}
\end{center}
at the very top of the main \LaTeX{} file,
in particular \emph{before} the |\documentclass| statement!
The argument of |\childdocmain| should be left empty
(but it must be present).

%%%%%%%%%%%%%%%%%%%%%%%%%%%%%%%%%%%%%%%%
\DescribeMacro{\childdocof}
Furthermore, add the commands
\begin{center}
\begin{tabular}{l}
|\input{childdoc.def}|\\
|\childdocof{|\textit{main}|}|\\
\end{tabular}
\end{center}
at the top of every child file \textit{child}
which is included by |\include{|\textit{child}|}|
from within the main file
(or at least for those files to be compiled individually).
The argument \textit{main} must be the filename of the main file.

There are a couple of
considerations in setting up the main and child documents:

%%%%%%%%%%%%%%%%%%%%%%%%%%%%%%%%%%%%%%%%
\paragraph{Restrictions.}

Please note the following restrictions:
\begin{itemize}
\item
|\childdocmain| must be called with one argument \textit{main}
to ensure compatibility with earlier version of the package.
It must either be empty (|\childdocmain{}|)
or precisely match the filename of the main file in which it is specified.
See \secref{sec:detection} for further information.
\item
The filename \textit{main} must be specified without the |.tex| extension.
\item
The filename \textit{main} is case sensitive
(even in case-insensitive file systems)
due to internal string comparison.
\item
The argument \textit{main} should be fully expanded, it cannot be a macro.
\item
Subdirectories and special characters should be avoided in filenames.
\item
The command |\childdocmain{|\textit{main}|}| must be followed by a whitespace.
It should not be followed immediately by another command
or by a comment mark `|%|'.
This is because the \TeX{} parser reads the token immediately following
the argument of |\childdocmain| and puts it
at the beginning of every child section;
however, a white\-space is ignored.
\end{itemize}

%%%%%%%%%%%%%%%%%%%%%%%%%%%%%%%%%%%%%%%%
\paragraph{Content of Main File.}

It is advisable to place all content in the child files included by |\include|.
Any output contained in the main file will appear in all child documents
unless suppressed manually;
it cannot be suppressed automatically by the |\includeonly| directive
and thus should normally be avoided.
A method to include some content in the main file
by means of conditional processing is described in \secref{sec:conditional}.

%%%%%%%%%%%%%%%%%%%%%%%%%%%%%%%%%%%%%%%%
\paragraph{Page Numbering.}

When only a part of the document is compiled,
the appropriate numbering of pages
(as well as other status parameters)
is determined from the |.aux| files.
The latter contain information from previous passes.
However this information needs to propagate through
all intermediate child documents.
Therefore the page numbering in child documents may well
be inconsistent until the complete document is compiled at least once.

A useful (if unconventional) way to always ensure a consistent
page numbering is to restart the numbering in each child document
and denote the pages by `\textit{child}|.|\textit{page}'
where \textit{child} represents the chapter/section number of the child file.
This can be achieved by the command
|\numberwithin{page}{|\textit{child}|}|
of the \textsf{amsmath} package
where \textit{child} can be |chapter| or |section|
depending on the chosen structuring.
Alternatively, one can modify the macro |\thepage| appropriately
and reset the counter |page| at the start of each child file.

%%%%%%%%%%%%%%%%%%%%%%%%%%%%%%%%%%%%%%%%%%%%%%%%%%%%%%%%%%%%%%%%%%%%%%%%%%%%%%%%
\subsection{Conditional Processing}
\label{sec:conditional}

The package provides a mechanism to compile different versions
of a document. To customise the versions further some conditional processing
can come in handy to distinguish which version is being compiled.
The package provides two macros to describe the compilation context:

%%%%%%%%%%%%%%%%%%%%%%%%%%%%%%%%%%%%%%%%
\DescribeMacro{\ifchilddoc}
The conditional |\ifchilddoc| distinguishes between the compilation of
child documents and the main document:
%
\begin{center}
|\ifchilddoc |\textit{child-code}| |[|\||else |\textit{main-code}]| \||fi|
\end{center}

%%%%%%%%%%%%%%%%%%%%%%%%%%%%%%%%%%%%%%%%
\DescribeMacro{\childdocname}
\DescribeMacro{\childdocjob}
The macro |\childdocname| contains the filename (without extension)
of the main or child file being processed.
Note that |\childdocjob| will always contain the name of the main file.

%%%%%%%%%%%%%%%%%%%%%%%%%%%%%%%%%%%%%%%%
\paragraph{Title Page.}

Conditional processing can be used to include a title or banner page
in the main document when proper precautions are taken.
Importantly, the code in the main file should ensure that the page counter
(as well as other status parameters which are stored in the |.aux| files)
takes the same value after the conditional processing.
Otherwise the page numbers may take divergent values
depending on which part is compiled.

For example, a title page could be declared by:
%
\begin{center}
\begin{tabular}{l}
|\ifchilddoc\||else|\\
|\addtocounter{page}{-1}|\\
\textit{code for title page}\\
|\newpage|\\
|\||fi|
\end{tabular}
\end{center}
%
A banner page for the child documents can be generated by:
%
\begin{center}
\begin{tabular}{l}
|\ifchilddoc|\\
|\addtocounter{page}{-1}|\\
\textit{code for banner page}\\
|\newpage|\\
|\||fi|
\end{tabular}
\end{center}
%
Here one could write a message such as:
\begin{center}
|This is the part \childdocname{} of \childdocjob{}.|
\end{center}

%%%%%%%%%%%%%%%%%%%%%%%%%%%%%%%%%%%%%%%%%%%%%%%%%%%%%%%%%%%%%%%%%%%%%%%%%%%%%%%%
\subsection{Flags}
\label{sec:flags}

The package makes it easy to generate different versions
of the main or child documents.
To this end compilation flags can be defined
and assigned different default values.
They will be particularly useful in conjunction
with the forwarding mechanism described in \secref{sec:forward}.

For example, it may be useful to have a flag |\version|
which can be set to |draft| or |final|.
The document source will contain some conditional code
depending on the value of |\version|.
Suppose further, the flag should default to |final| for the main file
and to |draft| for child files
which is a natural assignment for editing the document.
This is achieved by placing the following code
in the preamble of the main document
(below the |\childdocmain| directive):
%
\begin{center}
\begin{tabular}{l}
|\ifchilddoc|\\
|\providecommand{\version}{draft}|\\
|\||else|\\
|\providecommand{\version}{final}|\\
|\||fi|
\end{tabular}
\end{center}
%
The definition by |\providecommand| makes sure
that previous definitions are not overwritten.
Further statements |\providecommand{\version}{...}|
can thus be added before the above code to override it.

For the main file, one might add a line
(between |\childdocmain| and the above block)
%
\begin{center}
|%\ifchilddoc\||else\providecommand{\version}{draft}\||fi|
\end{center}
%
which can be uncommented to produce a draft version.
Likewise one can add a line to the very top of a child file
(above the |\childdocof{|\textit{main}|}| directive)
%
\begin{center}
|%\providecommand{\version}{final}|
\end{center}
%
which can be uncommented to produce the final version of this child document.

%%%%%%%%%%%%%%%%%%%%%%%%%%%%%%%%%%%%%%%%%%%%%%%%%%%%%%%%%%%%%%%%%%%%%%%%%%%%%%%%
\subsection{Forwarding}
\label{sec:forward}

Different versions of the main or child documents
using compilation flags as described in \secref{sec:flags}
can be (permanently) stored in different files
for convenient compilation, viewing and distribution.
To this end, the package defines a command
to pass on compilation to a different file:

%%%%%%%%%%%%%%%%%%%%%%%%%%%%%%%%%%%%%%%%
\DescribeMacro{\childdocforward}
The command |\childdocforward| redirects processing to
another source file:
%
\begin{center}
\begin{tabular}{l}
|\input{childdoc.def}|\\
|\childdocforward[|\textit{main}|]{|\textit{dest}|}|\\
\end{tabular}
\end{center}
%
The argument \textit{dest} is the destination file
(without extension).
It should be the main file or one of the child files.
Note that further \textsf{childdoc} directives
such as |\childdocof| and |\childdocforward|
in the indicated file will be processed in this form.
The optional argument \textit{main}
passes on directly to the main file \textit{main}
while pretending to compile the child \textit{dest}.
This form behaves as if \textit{dest}
issues |\childdocof{|\textit{main}|}| right away,
and no further \textsf{childdoc} directives will be processed.

%%%%%%%%%%%%%%%%%%%%%%%%%%%%%%%%%%%%%%%%
\DescribeMacro{\...prefix}
In the alternative form |\childdocforwardprefix|,
%
\begin{center}
\begin{tabular}{l}
|\input{childdoc.def}|\\
|\childdocforwardprefix[|\textit{main}|]{|\textit{prefix}|}{|\textit{dest}|}|
\end{tabular}
\end{center}
%
the destination file is determined by a pattern
depending on the current file:
To make this work, the current file must be called
`{\textit{prefix}\hspace{0.2em}\textit{suffix}}'
with \textit{prefix} matching precisely the argument.
Processing is then passed on to the file
`{\textit{dest}\hspace{0.2em}\textit{suffix}}'.
Surely, the same effect is achieved by
directly specifying the
argument `{\textit{dest}\hspace{0.2em}\textit{suffix}}'
in the first form.
However, that requires to set up a different file
for each child. With the alternative form of the command
all these files can have exactly the same content
which simplifies setting them up and maintaining them.

For example, the following file |draft.tex|
with a compilation flag |\version| as described in \secref{sec:flags}
compiles the main document as a draft:
%
\begin{center}
\begin{tabular}{l}
|\def\version{draft}|\\
|\input{childdoc.def}|\\
|\childdocforward{|\textit{main}|}|
\end{tabular}
\end{center}
%
Likewise, the following files |final|\textit{nn}|.tex|
compile the final version of the child document
|child|\textit{nn}|.tex|:
%
\begin{center}
\begin{tabular}{l}
|\def\version{final}|\\
|\input{childdoc.def}|\\
|\childdocforwardprefix{final}{child}|
\end{tabular}
\end{center}
%

Note that when several versions of a main file and/or of each child file
are to be generated, it may be convenient to set up a |Makefile| or
shell script to automatise the process.

%%%%%%%%%%%%%%%%%%%%%%%%%%%%%%%%%%%%%%%%%%%%%%%%%%%%%%%%%%%%%%%%%%%%%%%%%%%%%%%%
\subsection{Command Line Processing}
\label{sec:commandline}

The effect of redirection files can also be achieved by invoking
the \LaTeX{} compiler with a more elaborate command line.
Most conveniently this should be done as part
of a shell script or a |Makefile|.

When using \textsf{childdoc} in the main file, the following
command lines effectively perform a redirection
(note that depending on the shell being used,
backslashes may have to be doubled: `|\|' $\to$ `|\\|'):
%
\begin{center}
|... -jobname "|\textit{target}|" |\\|"|[\textit{flags}]%
|\input{childdoc.def}\childdocforward[|\textit{main}|]{|\textit{dest}|}"|
\end{center}
%
Here \textit{target} is the name of the output file,
\textit{main} is the name of the main file
and \textit{dest} is the name of the main or child file to be processed
(all filenames without extensions).
The optional argument \textit{main} can be omitted
if \textit{main} matches \textit{dest}.
Optionally, compilation \textit{flags} can be defined via |\def| commands.
This command line makes the \TeX{} engine believe
it is compiling the file \textit{target}
whose content is specified as the latter parameter.
The provided code then forwards the processing to
\textit{main} or \textit{dest} as described in \secref{sec:forward}.

%%%%%%%%%%%%%%%%%%%%%%%%%%%%%%%%%%%%%%%%%%%%%%%%%%%%%%%%%%%%%%%%%%%%%%%%%%%%%%%%
\subsection{Include by Input}
\label{sec:input}

Including child documents by |\include| has some restrictions by design.
Most notably, the content of a child document always occupies
its own set of pages; pages cannot be shared between child documents.
Usually, this behaviour makes perfect sense
because each child document contain an essential part of the document.
However, in some situations it may be desirable to compose
a document from a collection of parts
without having mandatory page breaks between then.
For this case, the package
provides a mechanism to include parts
by |\input| which can also be processed individually.
However, by construction this mechanism
requires manual handling of the content to be output.

%%%%%%%%%%%%%%%%%%%%%%%%%%%%%%%%%%%%%%%%
\DescribeMacro{\ifchilddocmanual}
The main file should be prepared as usual, see \secref{sec:include}.
However, the document body must make a distinction
between processing of an individual part and of the main document, e.g.:
%
\begin{center}
\begin{tabular}{l}
|\ifchilddocmanual|\\
|\input{\childdocname}|\\
|\||else|\\
\textit{document body with }|\input{|\textit{part}|}|\\
|\||fi|
\end{tabular}
\end{center}
%
The conditional |\ifchilddocmanual| is true whenever
a part to be included by |\input| is being compiled,
and the name of the part is stored in |\childdocname|.

%%%%%%%%%%%%%%%%%%%%%%%%%%%%%%%%%%%%%%%%
\DescribeMacro{\childdocby}
Each part to be included by |\input| should start with:
%
\begin{center}
\begin{tabular}{l}
|\input{childdoc.def}|\\
|\childdocby{|\textit{main}|}|\\
\end{tabular}
\end{center}
%
The directive |\childdocby| is similar to |\childdocof|
described in \secref{sec:include},
but the subsequent selection of content must be done manually.
To that end, both |\ifchilddoc| and |\ifchilddocmanual|
will be true upon processing of a part,
and the name of the part is stored in |\childdocname|.
Note that |\jobname| will be set to the filename of the current part
so that each part receives an individual |.aux| file
that does not interfere with the |.aux| file(s) of the main document.
This behaviour can be altered by the alternative form
|\childdocby[*]{|\textit{main}|}| (with a non-empty optional argument)
which uses the |.aux| file of the main document
by setting |\jobname| to \textit{main}.

%%%%%%%%%%%%%%%%%%%%%%%%%%%%%%%%%%%%%%%%%%%%%%%%%%%%%%%%%%%%%%%%%%%%%%%%%%%%%%%%
\subsection{Driver Development}
\label{sec:driver}

The \textsf{childdoc} mechanism can also be use for the development
of definition files such as \LaTeX{} styles or classes.
This case differs from the above setup with multiple parts
included by |\include| in that no |\includeonly| should be invoked.
This can be achieved by starting the include file
(before |\ProvidesPackage|) with:
%
\begin{center}
\begin{tabular}{l}
|\input{childdoc.def}|\\
|\childdocforward{|\textit{main}|}|\\
\end{tabular}
\end{center}
%
or alternatively with:
%
\begin{center}
\begin{tabular}{l}
|\input{childdoc.def}|\\
|\childdocby{|\textit{main}|}|\\
\end{tabular}
\end{center}
%
Both forms have slightly different effects as described above.
The main file is prepared as usual, see \secref{sec:include}.

%%%%%%%%%%%%%%%%%%%%%%%%%%%%%%%%%%%%%%%%%%%%%%%%%%%%%%%%%%%%%%%%%%%%%%%%%%%%%%%%
\subsection{Legacy Detection}
\label{sec:detection}

The directive |\childdocmain| in the main file can detect
whether the complete document or merely a child is to be compiled
even without using the directive |\childdocof|.
This method is deprecated because it is less robust
and there is no compelling reason to use it;
it is merely provided for backward compatibility
and it may be removed in future versions.

If the detection mechanism is to be used,
it is mandatory to correctly specify
the filename of the main file as the argument of |\childdocmain|:
%
\begin{center}
\begin{tabular}{l}
|\input{childdoc.def}|\\
|\childdocmain{|\textit{main}|}|\\
\end{tabular}
\end{center}
%
If |\jobname| does not match the argument \textit{main} of |\childdocmain|,
it is assumed that |\jobname| points to the child file to be compiled.
When using |\childdocmain| with the main file specified as argument,
it suffices to start a child file
with just |\input{|\textit{main}|}|
without loading of the package and using |\childdocof|.
If instead all processing is done
with the appropriate \textsf{childdoc} directives,
the argument of \textit{main} of |\childdocmain| can be empty.

An alternative version of the command line processing described
in \secref{sec:commandline} using the detection mechanism reads:
%
\begin{center}
|... -jobname "|\textit{target}|" "|[\textit{flags}]%
[|\def\jobname{|\textit{dest}|}|]|\input{|\textit{main}|}"|
\end{center}

%%%%%%%%%%%%%%%%%%%%%%%%%%%%%%%%%%%%%%%%%%%%%%%%%%%%%%%%%%%%%%%%%%%%%%%%%%%%%%%%
\subsection{Manual Code}
\label{sec:manual}

In case one cannot be certain whether the definitions file |childdoc.def|
is installed on the target \TeX{} distribution
and one prefers not to ship it,
it is conceivable to paste a few relevant commands into the sources.

To that end, drop all statements |\input{childdoc.def}|
and perform the replacements as outlined below.
Instead of |\childdocmain{|\textit{main}|}| add the following code
to the top of the main file:
%
\begin{center}
\begin{tabular}{l}
|\||ifdefined\childdocname\endinput\||fi\newif\ifchilddoc|\\
|\edef\childdocname{\scantokens\expandafter{\jobname\noexpand}}|\\
|\def\childdocmain{|\textit{main}|}\||ifx\childdocmain\childdocname\||else|\\
|\childdoctrue\includeonly{\childdocname}\let\jobname\childdocmain\||fi|\\
\end{tabular}
\end{center}
%
Instead of |\childdocof{|\textit{main}|}| just include the main file
at the top of each child file:
%
\begin{center}
|\input{|\textit{main}|}|
\end{center}
%
A simple redirection |\childdocforward{|\textit{dest}|}| is achieved by:
%
\begin{center}
|\def\jobname{|\textit{dest}|}\input{\jobname}|
\end{center}
%
The redirection with prefix
|\childdocforwardprefix[|\textit{prefix}|]{|\textit{dest}|}|
is accomplished by:
%
\begin{center}
\begin{tabular}{l}
|{\edef\jobname{\scantokens\expandafter{\jobname\noexpand}}|\\
|\def\redirectjob |\textit{prefix}|#1~~~{\gdef\jobname{|\textit{dest}|#1}}|\\
|\expandafter\redirectjob\jobname~~~}\input{\jobname}|
\end{tabular}
\end{center}

In an alternative approach,
child documents can be compiled by a specific command line
without additional code or specific definitions:
%
\begin{center}
|... -jobname "|\textit{target}|" "|[\textit{flags}]%
|\includeonly{|\textit{dest}|}\input{|\textit{main}|}"|
\end{center}
%

%%%%%%%%%%%%%%%%%%%%%%%%%%%%%%%%%%%%%%%%%%%%%%%%%%%%%%%%%%%%%%%%%%%%%%%%%%%%%%%%
%%%%%%%%%%%%%%%%%%%%%%%%%%%%%%%%%%%%%%%%%%%%%%%%%%%%%%%%%%%%%%%%%%%%%%%%%%%%%%%%
\section{Information}

%%%%%%%%%%%%%%%%%%%%%%%%%%%%%%%%%%%%%%%%%%%%%%%%%%%%%%%%%%%%%%%%%%%%%%%%%%%%%%%%
\subsection{Copyright}

Copyright \copyright{} 2017--2018 Niklas Beisert

This work may be distributed and/or modified under the
conditions of the \LaTeX{} Project Public License, either version 1.3
of this license or (at your option) any later version.
The latest version of this license is in
  \url{http://www.latex-project.org/lppl.txt}
and version 1.3 or later is part of all distributions of \LaTeX{}
version 2005/12/01 or later.

This work has the LPPL maintenance status `maintained'.

The Current Maintainer of this work is Niklas Beisert.

This work consists of the files |README.txt|, |childdoc.ins| and |childdoc.dtx|
as well as the derived files |childdoc.def|, |cdocsamp.tex|
with |cdocsch1.tex|, |cdocsch2.tex|, |cdocspt3.tex|, |cdocspt4.tex|,
|cdocsdrf.tex|, |cdocsfn1.tex|, |cdocsfn2.tex|
as well as |childdoc.pdf|.

%%%%%%%%%%%%%%%%%%%%%%%%%%%%%%%%%%%%%%%%%%%%%%%%%%%%%%%%%%%%%%%%%%%%%%%%%%%%%%%%
\subsection{Files and Installation}

The package consists of the files:
%
\begin{center}
\begin{tabular}{ll}
    |README.txt|   & readme file \\
    |childdoc.ins| & installation file \\
    |childdoc.dtx| & source file \\
    |childdoc.def| & definition file \\
    |cdocsamp.tex| & sample main file \\
    |cdocsch1.tex| & sample include file \\
    |cdocsch2.tex| & sample include file \\
    |cdocspt3.tex| & sample part file \\
    |cdocspt4.tex| & sample part file \\
    |cdocsdrf.tex| & sample redirection file \\
    |cdocsfn1.tex| & sample redirection file \\
    |cdocsfn2.tex| & sample redirection file \\
    |childdoc.pdf| & manual
\end{tabular}
\end{center}
%
The distribution consists of the files
|README.txt|, |childdoc.ins| and |childdoc.dtx|.
%
\begin{itemize}
\item
Run (pdf)\LaTeX{} on |childdoc.dtx|
to compile the manual |childdoc.pdf| (this file).
\item
Run \LaTeX{} on |childdoc.ins| to create the definitions file |childdoc.def|
and the sample |cdocsamp.tex| with include files
|cdocsch1.tex|, |cdocsch2.tex|, |cdocspt3.tex|, |cdocspt4.tex|,
|cdocsdrf.tex|, |cdocsfn1.tex|, |cdocsfn2.tex|.
Then copy the file |childdoc.def| to an appropriate directory of your \LaTeX{}
distribution, e.g.\ \textit{texmf-root}|/tex/latex/childdoc|.
\end{itemize}

%%%%%%%%%%%%%%%%%%%%%%%%%%%%%%%%%%%%%%%%%%%%%%%%%%%%%%%%%%%%%%%%%%%%%%%%%%%%%%%%
\subsection{Related CTAN Packages}

There are several other packages which offer a similar functionality:
%
\begin{itemize}
\item
The packages
\href{http://ctan.org/pkg/docmute}{\textsf{docmute}},
\href{http://ctan.org/pkg/includex}{\textsf{includex}} and
\href{http://ctan.org/pkg/standalone}{\textsf{standalone}}
provide commands to include only the document body of
a child file thus allowing both files to be compiled individually.
\item
The packages \href{http://ctan.org/pkg/subdocs}{\textsf{subdocs}}
and \href{http://ctan.org/pkg/subfiles}{\textsf{subfiles}}
provide structures in which the main and child documents can be
encapsulated and allowing them to be compiled individually.
The inclusion mechanism is different from the conventional |\include|.
\item
The package \href{http://ctan.org/pkg/combine}{\textsf{combine}}
is an elaborate solution to combine several documents into one.
\end{itemize}
%
See also the CTAN topic \href{http://ctan.org/topic/subdocs}{\textsf{subdocs}}
for further related packages.
The present package differs from the above solutions in that
a document structure constructed with the conventional |\include| mechanism
just needs two extra commands at the top of every file
such that all constituent files can be compiled individually.

%%%%%%%%%%%%%%%%%%%%%%%%%%%%%%%%%%%%%%%%%%%%%%%%%%%%%%%%%%%%%%%%%%%%%%%%%%%%%%%%
%\subsection{Feature Suggestions}
%
%The following is a list of features which may be useful for future
%versions of this package:
%%
%\begin{itemize}
%\item
%\ldots
%\end{itemize}

%%%%%%%%%%%%%%%%%%%%%%%%%%%%%%%%%%%%%%%%%%%%%%%%%%%%%%%%%%%%%%%%%%%%%%%%%%%%%%%%
\subsection{Revision History}

%%%%%%%%%%%%%%%%%%%%%%%%%%%%%%%%%%%%%%%%
\paragraph{v2.0:} 2018/12/30

\begin{itemize}
\item
immediate forward processing
\item
added |\childdocby| mechanism
\item
manual restructured
\end{itemize}

%%%%%%%%%%%%%%%%%%%%%%%%%%%%%%%%%%%%%%%%
\paragraph{v1.6:} 2018/01/17

\begin{itemize}
\item
application for development of include files
\item
corrections to manual
\end{itemize}

%%%%%%%%%%%%%%%%%%%%%%%%%%%%%%%%%%%%%%%%
\paragraph{v1.5:} 2017/05/21

\begin{itemize}
\item
more complete structuring introduced
\item
|\childdocof| introduced
\item
|\childdoc| renamed to |\childdocmain|
\item
|\childredirect| renamed to |\childdocforward| and |\childdocforwardprefix|
and functionality expanded
\end{itemize}

%%%%%%%%%%%%%%%%%%%%%%%%%%%%%%%%%%%%%%%%
\paragraph{v1.0:} 2017/04/27

\begin{itemize}
\item
manual and install package
\item
first version published on CTAN
\end{itemize}

%%%%%%%%%%%%%%%%%%%%%%%%%%%%%%%%%%%%%%%%
\paragraph{v0.6:} 2017/04/26

\begin{itemize}
\item
redirection mechanism added
\end{itemize}

%%%%%%%%%%%%%%%%%%%%%%%%%%%%%%%%%%%%%%%%
\paragraph{v0.5:} 2017/04/26

\begin{itemize}
\item
functionality in definition file
\end{itemize}


%%%%%%%%%%%%%%%%%%%%%%%%%%%%%%%%%%%%%%%%%%%%%%%%%%%%%%%%%%%%%%%%%%%%%%%%%%%%%%%%
%%%%%%%%%%%%%%%%%%%%%%%%%%%%%%%%%%%%%%%%%%%%%%%%%%%%%%%%%%%%%%%%%%%%%%%%%%%%%%%%
%%%%%%%%%%%%%%%%%%%%%%%%%%%%%%%%%%%%%%%%%%%%%%%%%%%%%%%%%%%%%%%%%%%%%%%%%%%%%%%%
\appendix

\settowidth\MacroIndent{\rmfamily\scriptsize 000\ }

 \DocInput{childdoc.dtx}

\end{document}
%</driver>
% \fi
%
% %%%%%%%%%%%%%%%%%%%%%%%%%%%%%%%%%%%%%%%%%%%%%%%%%%%%%%%%%%%%%%%%%%%%%%%%%%%%%%
% %%%%%%%%%%%%%%%%%%%%%%%%%%%%%%%%%%%%%%%%%%%%%%%%%%%%%%%%%%%%%%%%%%%%%%%%%%%%%%
% \section{Sample}
%\iffalse
%<*samplemain>
%\fi
%
% The following presents a sample document
% with two chapters, two parts, a title page,
% a compile flag as well as three forwarding files to set the flag.
% It consists of eight |.tex| files:
% \begin{center}
% \begin{tabular}{ll}
% |cdocsamp.tex|&main file\\
% |cdocsch1.tex|&include file for chapter 1\\
% |cdocsch2.tex|&include file for chapter 2\\
% |cdocspt3.tex|&include file for part 3\\
% |cdocspt4.tex|&include file for part 4\\
% |cdocsdrf.tex|&forwarding file for main file in draft mode\\
% |cdocsfi1.tex|&forwarding file for final version of chapter 1\\
% |cdocsfi2.tex|&forwarding file for final version of chapter 2\\
% \end{tabular}
% \end{center}
% Each of the eight files can be compiled directly by the \LaTeX{} compiler.
%
% %%%%%%%%%%%%%%%%%%%%%%%%%%%%%%%%%%%%%%
% \paragraph{Main File.}
%
% The main file is called |cdocsamp.tex|.
%
% Load the \textsf{childdoc} definitions and
% declare the filename for the main document:
%    \begin{macrocode}
\input{childdoc.def}
\childdocmain{}
%    \end{macrocode}

% Optional override for |\version| flag:
%    \begin{macrocode}
%%\ifchilddoc\else\providecommand{\version}{draft}\fi
%    \end{macrocode}

% Define the default values for the |\version| flag
% (|final| for the main file and |draft| for childs):
%    \begin{macrocode}
\ifchilddoc
\providecommand{\version}{draft}
\else
\providecommand{\version}{final}
\fi
%    \end{macrocode}

% Load the standard document class:
%    \begin{macrocode}
\documentclass[12pt]{article}
%    \end{macrocode}

% Start the document body:
%    \begin{macrocode}
\begin{document}
%    \end{macrocode}

% Declare a title page.
% Print title, part of document being processed and version flag:
%    \begin{macrocode}
\addtocounter{page}{-1}
\begin{center}
{\LARGE\bfseries{}childdoc example\par}
\vspace{1cm}
\ifchilddoc
\ifchilddocmanual part\else chapter\fi:
`\childdocname' of `\childdocjob'\par
\else
main document: `\childdocjob'\par
\fi
version: \version\par
\end{center}
\newpage
%    \end{macrocode}

% Manually include selected file,
% otherwise process as usual:
%    \begin{macrocode}
\ifchilddocmanual
\section*{part `\childdocname'}
\input{\childdocname}
\else
%    \end{macrocode}

% Include the two chapters:
%    \begin{macrocode}
\include{cdocsch1}
\include{cdocsch2}
%    \end{macrocode}

% Include the two parts unless only chapters should be displayed:
%    \begin{macrocode}
\ifchilddoc\else
\section{part three}
\input{cdocspt3}
\section{part four}
\input{cdocspt4}
\fi
%    \end{macrocode}

% Process as usual until here:
%    \begin{macrocode}
\fi
%    \end{macrocode}

% End of document body:
%    \begin{macrocode}
\end{document}
%    \end{macrocode}
%\iffalse
%</samplemain>
%\fi
%
% %%%%%%%%%%%%%%%%%%%%%%%%%%%%%%%%%%%%%%
% \paragraph{Chapter Include Files.}
%
% The include files are called |cdocsch1.tex| and |cdocsch2.tex|.
%
%\iffalse
%<*samplechap1|samplechap2>
%\fi

% Optional override for |\version| flag:
%    \begin{macrocode}
%%\providecommand{\version}{final}
%    \end{macrocode}

% Include the main document:
%    \begin{macrocode}
\input{childdoc.def}
\childdocof{cdocsamp}
%    \end{macrocode}

%\iffalse
%</samplechap1|samplechap2>
%\fi
%
%\iffalse
%<*samplechap1>
%\fi
% Some text for chapter 1:
%    \begin{macrocode}
\section{one}
some text in chapter one
%    \end{macrocode}

%\iffalse
%</samplechap1>
%\fi
% Some text for chapter 2:
%\iffalse
%<*samplechap2>
%\fi
%    \begin{macrocode}
\section{two}
more text in chapter two
%    \end{macrocode}

%\iffalse
%</samplechap2>
%\fi
%
% %%%%%%%%%%%%%%%%%%%%%%%%%%%%%%%%%%%%%%
% \paragraph{Part Include Files.}
%
% The include files are called |cdocspt3.tex| and |cdocspt4.tex|.
%
%\iffalse
%<*samplepart3|samplepart4>
%\fi

% Optional override for |\version| flag:
%    \begin{macrocode}
%%\providecommand{\version}{final}
%    \end{macrocode}

% Include the main document:
%    \begin{macrocode}
\input{childdoc.def}
\childdocby{cdocsamp}
%    \end{macrocode}

%\iffalse
%</samplepart3|samplepart4>
%\fi
%
%\iffalse
%<*samplepart3>
%\fi
% Some text for part 3:
%    \begin{macrocode}
some text in part three
%    \end{macrocode}

%\iffalse
%</samplepart3>
%\fi
% Some text for part 4:
%\iffalse
%<*samplepart4>
%\fi
%    \begin{macrocode}
more text in part four
%    \end{macrocode}

%\iffalse
%</samplepart4>
%\fi
%
% %%%%%%%%%%%%%%%%%%%%%%%%%%%%%%%%%%%%%%
% \paragraph{Forwarding for a Complete Draft.}
%
% The following forwarding file |cdocsdrf.tex|
% compiles the main document in draft mode:
%\iffalse
%<*sampledraft>
%\fi
%    \begin{macrocode}
\def\version{draft}
\input{childdoc.def}
\childdocforward{cdocsamp}
%    \end{macrocode}

%\iffalse
%</sampledraft>
%\fi
%
% %%%%%%%%%%%%%%%%%%%%%%%%%%%%%%%%%%%%%%
% \paragraph{Forwarding for Final Version of the Chapters.}
%
% The following forwarding files |cdocsfn1.tex| and |cdocsfn2.tex|
% (with identical content)
% compile the final versions of the child documents
% |cdocsch1.tex| and |cdocsch2.tex|, respectively:
%\iffalse
%<*samplefinal>
%\fi
%    \begin{macrocode}
\def\version{final}
\input{childdoc.def}
\childdocforwardprefix[cdocsamp]{cdocsfn}{cdocsch}
%    \end{macrocode}

%\iffalse
%</samplefinal>
%\fi
%
% %%%%%%%%%%%%%%%%%%%%%%%%%%%%%%%%%%%%%%
% \paragraph{Command Line Processing.}
%
% The following three command lines generate the output files
% |cdocscld|, |cdocscl1| and |cdocscl2|
% which should be identical to
% |cdocsdrf|, |cdocsch1| and |cdocsfn2|, respectively:
% \begin{center}
% \begin{tabular}{l}
% |latex -jobname cdocscld \|\\
% |  "\def\version{draft}\input{childdoc.def}\childdocforward{cdocsamp}"|\\
% |latex -jobname cdocscl1 \|\\
% |  "\input{childdoc.def}\childdocforward[cdocsamp]{cdocsch1}"|\\
% |latex -jobname cdocscl2 \|\\
% |  "\def\version{final}\input{childdoc.def}\childdocforward{cdocsch2}"|
% \end{tabular}
% \end{center}
% Note that the trailing backslash on each first line
% merely continues the input to the second line
% (for convenient cut ant paste).
% Furthermore, the command |latex| can be replaced by any
% of its alternative versions such as |pdflatex|.
%
% %%%%%%%%%%%%%%%%%%%%%%%%%%%%%%%%%%%%%%%%%%%%%%%%%%%%%%%%%%%%%%%%%%%%%%%%%%%%%%
% %%%%%%%%%%%%%%%%%%%%%%%%%%%%%%%%%%%%%%%%%%%%%%%%%%%%%%%%%%%%%%%%%%%%%%%%%%%%%%
% \section{Implementation}
%\iffalse
%<*package>
%\fi
%
% This section describes the definitions file |childdoc.def|.

% The definitions cannot be loaded using |\usepackage| or |\RequirePackage|
% which has a mechanism to prevent loading a style file more than once.
% When loading the definitions by means of |\input|
% multiple instances have to be prevented manually:
%\iffalse
%This code needs to be before the `\ProvidesFile' directive
%which is defined at the beginning of this file.
%Therefore it is also placed there and commented out here.
%</package>
%<*discard>
%\fi
%    \begin{macrocode}
\ifdefined\childdocmain\endinput\fi
%    \end{macrocode}
%\iffalse
%</discard>
%<*package>
%\fi
%
% \macro{\ifchilddoc}
% \macro{\ifchilddocmanual}
% The conditional |\ifchilddoc| tells whether a
% child (true) or main (false) document is being compiled.
% The conditional |\ifchilddocmanual| tells whether
% the |\includeonly| mechanism is used (false) or
% the selection of child files must be performed manually (true).
% The definitions initialise to false:
%    \begin{macrocode}
\newif\ifchilddoc
\newif\ifchilddocmanual
%    \end{macrocode}

% \macro{\childdocname}
% \macro{\childdocjob}
% The macro |\childdocname| stores the name of the main document
% to be compiled. The macro |\childdocjob| stores the name of
% the document on which the \LaTeX{} compiler was originally invoked.
% The content of |\jobname| cannot be compared
% to filenames specified in the source due to different catcodes.
% The following code rescans |\jobname|, stores the result
% in |\childdocname| and saves a copy in |\childdocjob|:
%    \begin{macrocode}
\edef\childdocname{\scantokens\expandafter{\jobname\noexpand}}
\let\childdocjob\childdocname
%    \end{macrocode}

% \macro{\childdocdisable}
% The macro |\childdocdisable| prevents the main file
% from being processed more than once.
% At this stage, the main document command |\childdocmain|
% is assumed to be called once again where it should do nothing.
% Any subsequent call to it should prevent
% a secondary processing of the main document
% It overwrites the forwarding commands
% |\childdocof| and |\childdocforward|
% with empty macros to prevent further inclusions of the main document:
%    \begin{macrocode}
\newcommand{\childdocdisable}
{
  \renewcommand{\childdocmain}[1]{\renewcommand{\childdocmain}[1]{\endinput}}
  \renewcommand{\childdocof}[1]{}
  \renewcommand{\childdocby}[2][]{}
  \renewcommand{\childdocforward}[2][]{}
  \renewcommand{\childdocdisable}{}
}
%    \end{macrocode}

% \macro{\childdocmain}
% The macro |\childdocmain| is to be called at the top of the main file
% with nothing or the main filename (without extension) as argument.
% First, it breaks loops.
% If the argument is not empty and does not match |\childdocname|
% (which is set by the first inclusion of |childdoc.def|),
% |\ifchilddoc| is set to true, |\includeonly| is applied to the child file
% and |\jobname| is set to the main file
% (for proper handling of |.aux| files):
%    \begin{macrocode}
\newcommand{\childdocmain}[1]
{
  \childdocdisable\childdocmain{}
  \if?#1?\else
    \begingroup
      \def\childdoctmp{#1}
      \ifx\childdoctmp\childdocname
        \def\childdoctmp{}
      \else
        \def\childdoctmp
        {
          \childdoctrue
          \includeonly{\childdocname}
          \def\childdocjob{#1}
          \def\jobname{#1}
        }
      \fi
      \expandafter
    \endgroup
    \childdoctmp
  \fi
}
%    \end{macrocode}

% \macro{\childdocof}
% The command |\childdocof| redirects
% compilation to the main file |#1|.
%    \begin{macrocode}
\newcommand{\childdocof}[1]
{
  \childdocdisable
  \childdoctrue
  \includeonly{\childdocname}
  \def\jobname{#1}
  \def\childdocjob{#1}
  \input{#1}
}
%    \end{macrocode}

% \macro{\childdocby}
% The command |\childdocby| ....
%    \begin{macrocode}
\newcommand{\childdocby}[2][]
{
  \childdocdisable
  \childdoctrue
  \childdocmanualtrue
  \if?#1?\else
    \def\jobname{#2}
  \fi
  \def\childdocjob{#2}
  \input{#2}
  \endinput
}
%    \end{macrocode}

% \macro{\childdocforward}
% The command |\childdocforward| redirects
% compilation to the main file or
% (if the optional argument is given) a child file.
% Parameters are set as if the main file
% or a child file starting with |\childdocof| was compiled.
% Then compilation is handed over to the main file:
%    \begin{macrocode}
\newcommand{\childdocforward}[2][]
{
  \begingroup
    \if?#1?
      \def\childdoctmp
      {
        \def\childdocname{#2}
        \def\childdocjob{#2}
        \def\jobname{#2}
        \input{#2}
        \endinput
      }
    \else
      \def\childdoctmp
      {
        \childdocdisable
        \def\childdocname{#2}
        \childdoctrue
        \includeonly{#2}
        \def\childdocjob{#1}
        \def\jobname{#1}
        \input{#1}
        \endinput
      }
    \fi
    \expandafter
  \endgroup
  \childdoctmp
}
%    \end{macrocode}

% \macro{\childdocforwardprefix}
% The command |\childdocforwardprefix| redirects
% compilation to the main or a child file by means of a pattern.
% The prefix |#1| in the current filename is replaced by |#2|
% and the suffix of the current filename is kept
% (it is assumed that the filename does not contain the substring `|~~~|'
% which is used as a delimiter).
% Compilation is handed over to the new file by |\childdocforward|:
%    \begin{macrocode}
\newcommand{\childdocforwardprefix}[3][]
{
  \begingroup
    \def\childdocextract #2##1~~~{\def\childdoctmp{\childdocforward[#1]{#3##1}}}
    \expandafter\childdocextract\childdocname~~~
    \expandafter
  \endgroup
  \childdoctmp
}
%    \end{macrocode}

% \macro{\childdoc}
% The deprecated macro |\childdoc| is a legacy version of |\childdocmain|:
%    \begin{macrocode}
\newcommand{\childdoc}{\childdocmain}
%    \end{macrocode}

% \macro{\childdocredirect}
% The deprecated macro |\childdocredirect| is a legacy version
% of |\childdocforward| and |\childdocforwardprefix|:
%    \begin{macrocode}
\newcommand{\childdocredirect}[2][]
{
  \begingroup
    \if?#1?
      \def\childdoctmp{\childdocforward{#2}}
    \else
      \def\childdoctmp{\childdocforwardprefix{#1}{#2}}
    \fi
    \expandafter
  \endgroup
  \childdoctmp
}
%    \end{macrocode}

%\iffalse
%</package>
%\fi
%
\endinput
|
and perform the replacements as outlined below.
Instead of |\childdocmain{|\textit{main}|}| add the following code
to the top of the main file:
%
\begin{center}
\begin{tabular}{l}
|\||ifdefined\childdocname\endinput\||fi\newif\ifchilddoc|\\
|\edef\childdocname{\scantokens\expandafter{\jobname\noexpand}}|\\
|\def\childdocmain{|\textit{main}|}\||ifx\childdocmain\childdocname\||else|\\
|\childdoctrue\includeonly{\childdocname}\let\jobname\childdocmain\||fi|\\
\end{tabular}
\end{center}
%
Instead of |\childdocof{|\textit{main}|}| just include the main file
at the top of each child file:
%
\begin{center}
|\input{|\textit{main}|}|
\end{center}
%
A simple redirection |\childdocforward{|\textit{dest}|}| is achieved by:
%
\begin{center}
|\def\jobname{|\textit{dest}|}\input{\jobname}|
\end{center}
%
The redirection with prefix
|\childdocforwardprefix[|\textit{prefix}|]{|\textit{dest}|}|
is accomplished by:
%
\begin{center}
\begin{tabular}{l}
|{\edef\jobname{\scantokens\expandafter{\jobname\noexpand}}|\\
|\def\redirectjob |\textit{prefix}|#1~~~{\gdef\jobname{|\textit{dest}|#1}}|\\
|\expandafter\redirectjob\jobname~~~}\input{\jobname}|
\end{tabular}
\end{center}

In an alternative approach,
child documents can be compiled by a specific command line
without additional code or specific definitions:
%
\begin{center}
|... -jobname "|\textit{target}|" "|[\textit{flags}]%
|\includeonly{|\textit{dest}|}\input{|\textit{main}|}"|
\end{center}
%

%%%%%%%%%%%%%%%%%%%%%%%%%%%%%%%%%%%%%%%%%%%%%%%%%%%%%%%%%%%%%%%%%%%%%%%%%%%%%%%%
%%%%%%%%%%%%%%%%%%%%%%%%%%%%%%%%%%%%%%%%%%%%%%%%%%%%%%%%%%%%%%%%%%%%%%%%%%%%%%%%
\section{Information}

%%%%%%%%%%%%%%%%%%%%%%%%%%%%%%%%%%%%%%%%%%%%%%%%%%%%%%%%%%%%%%%%%%%%%%%%%%%%%%%%
\subsection{Copyright}

Copyright \copyright{} 2017--2018 Niklas Beisert

This work may be distributed and/or modified under the
conditions of the \LaTeX{} Project Public License, either version 1.3
of this license or (at your option) any later version.
The latest version of this license is in
  \url{http://www.latex-project.org/lppl.txt}
and version 1.3 or later is part of all distributions of \LaTeX{}
version 2005/12/01 or later.

This work has the LPPL maintenance status `maintained'.

The Current Maintainer of this work is Niklas Beisert.

This work consists of the files |README.txt|, |childdoc.ins| and |childdoc.dtx|
as well as the derived files |childdoc.def|, |cdocsamp.tex|
with |cdocsch1.tex|, |cdocsch2.tex|, |cdocspt3.tex|, |cdocspt4.tex|,
|cdocsdrf.tex|, |cdocsfn1.tex|, |cdocsfn2.tex|
as well as |childdoc.pdf|.

%%%%%%%%%%%%%%%%%%%%%%%%%%%%%%%%%%%%%%%%%%%%%%%%%%%%%%%%%%%%%%%%%%%%%%%%%%%%%%%%
\subsection{Files and Installation}

The package consists of the files:
%
\begin{center}
\begin{tabular}{ll}
    |README.txt|   & readme file \\
    |childdoc.ins| & installation file \\
    |childdoc.dtx| & source file \\
    |childdoc.def| & definition file \\
    |cdocsamp.tex| & sample main file \\
    |cdocsch1.tex| & sample include file \\
    |cdocsch2.tex| & sample include file \\
    |cdocspt3.tex| & sample part file \\
    |cdocspt4.tex| & sample part file \\
    |cdocsdrf.tex| & sample redirection file \\
    |cdocsfn1.tex| & sample redirection file \\
    |cdocsfn2.tex| & sample redirection file \\
    |childdoc.pdf| & manual
\end{tabular}
\end{center}
%
The distribution consists of the files
|README.txt|, |childdoc.ins| and |childdoc.dtx|.
%
\begin{itemize}
\item
Run (pdf)\LaTeX{} on |childdoc.dtx|
to compile the manual |childdoc.pdf| (this file).
\item
Run \LaTeX{} on |childdoc.ins| to create the definitions file |childdoc.def|
and the sample |cdocsamp.tex| with include files
|cdocsch1.tex|, |cdocsch2.tex|, |cdocspt3.tex|, |cdocspt4.tex|,
|cdocsdrf.tex|, |cdocsfn1.tex|, |cdocsfn2.tex|.
Then copy the file |childdoc.def| to an appropriate directory of your \LaTeX{}
distribution, e.g.\ \textit{texmf-root}|/tex/latex/childdoc|.
\end{itemize}

%%%%%%%%%%%%%%%%%%%%%%%%%%%%%%%%%%%%%%%%%%%%%%%%%%%%%%%%%%%%%%%%%%%%%%%%%%%%%%%%
\subsection{Related CTAN Packages}

There are several other packages which offer a similar functionality:
%
\begin{itemize}
\item
The packages
\href{http://ctan.org/pkg/docmute}{\textsf{docmute}},
\href{http://ctan.org/pkg/includex}{\textsf{includex}} and
\href{http://ctan.org/pkg/standalone}{\textsf{standalone}}
provide commands to include only the document body of
a child file thus allowing both files to be compiled individually.
\item
The packages \href{http://ctan.org/pkg/subdocs}{\textsf{subdocs}}
and \href{http://ctan.org/pkg/subfiles}{\textsf{subfiles}}
provide structures in which the main and child documents can be
encapsulated and allowing them to be compiled individually.
The inclusion mechanism is different from the conventional |\include|.
\item
The package \href{http://ctan.org/pkg/combine}{\textsf{combine}}
is an elaborate solution to combine several documents into one.
\end{itemize}
%
See also the CTAN topic \href{http://ctan.org/topic/subdocs}{\textsf{subdocs}}
for further related packages.
The present package differs from the above solutions in that
a document structure constructed with the conventional |\include| mechanism
just needs two extra commands at the top of every file
such that all constituent files can be compiled individually.

%%%%%%%%%%%%%%%%%%%%%%%%%%%%%%%%%%%%%%%%%%%%%%%%%%%%%%%%%%%%%%%%%%%%%%%%%%%%%%%%
%\subsection{Feature Suggestions}
%
%The following is a list of features which may be useful for future
%versions of this package:
%%
%\begin{itemize}
%\item
%\ldots
%\end{itemize}

%%%%%%%%%%%%%%%%%%%%%%%%%%%%%%%%%%%%%%%%%%%%%%%%%%%%%%%%%%%%%%%%%%%%%%%%%%%%%%%%
\subsection{Revision History}

%%%%%%%%%%%%%%%%%%%%%%%%%%%%%%%%%%%%%%%%
\paragraph{v2.0:} 2018/12/30

\begin{itemize}
\item
immediate forward processing
\item
added |\childdocby| mechanism
\item
manual restructured
\end{itemize}

%%%%%%%%%%%%%%%%%%%%%%%%%%%%%%%%%%%%%%%%
\paragraph{v1.6:} 2018/01/17

\begin{itemize}
\item
application for development of include files
\item
corrections to manual
\end{itemize}

%%%%%%%%%%%%%%%%%%%%%%%%%%%%%%%%%%%%%%%%
\paragraph{v1.5:} 2017/05/21

\begin{itemize}
\item
more complete structuring introduced
\item
|\childdocof| introduced
\item
|\childdoc| renamed to |\childdocmain|
\item
|\childredirect| renamed to |\childdocforward| and |\childdocforwardprefix|
and functionality expanded
\end{itemize}

%%%%%%%%%%%%%%%%%%%%%%%%%%%%%%%%%%%%%%%%
\paragraph{v1.0:} 2017/04/27

\begin{itemize}
\item
manual and install package
\item
first version published on CTAN
\end{itemize}

%%%%%%%%%%%%%%%%%%%%%%%%%%%%%%%%%%%%%%%%
\paragraph{v0.6:} 2017/04/26

\begin{itemize}
\item
redirection mechanism added
\end{itemize}

%%%%%%%%%%%%%%%%%%%%%%%%%%%%%%%%%%%%%%%%
\paragraph{v0.5:} 2017/04/26

\begin{itemize}
\item
functionality in definition file
\end{itemize}


%%%%%%%%%%%%%%%%%%%%%%%%%%%%%%%%%%%%%%%%%%%%%%%%%%%%%%%%%%%%%%%%%%%%%%%%%%%%%%%%
%%%%%%%%%%%%%%%%%%%%%%%%%%%%%%%%%%%%%%%%%%%%%%%%%%%%%%%%%%%%%%%%%%%%%%%%%%%%%%%%
%%%%%%%%%%%%%%%%%%%%%%%%%%%%%%%%%%%%%%%%%%%%%%%%%%%%%%%%%%%%%%%%%%%%%%%%%%%%%%%%
\appendix

\settowidth\MacroIndent{\rmfamily\scriptsize 000\ }

 \DocInput{childdoc.dtx}

\end{document}
%</driver>
% \fi
%
% %%%%%%%%%%%%%%%%%%%%%%%%%%%%%%%%%%%%%%%%%%%%%%%%%%%%%%%%%%%%%%%%%%%%%%%%%%%%%%
% %%%%%%%%%%%%%%%%%%%%%%%%%%%%%%%%%%%%%%%%%%%%%%%%%%%%%%%%%%%%%%%%%%%%%%%%%%%%%%
% \section{Sample}
%\iffalse
%<*samplemain>
%\fi
%
% The following presents a sample document
% with two chapters, two parts, a title page,
% a compile flag as well as three forwarding files to set the flag.
% It consists of eight |.tex| files:
% \begin{center}
% \begin{tabular}{ll}
% |cdocsamp.tex|&main file\\
% |cdocsch1.tex|&include file for chapter 1\\
% |cdocsch2.tex|&include file for chapter 2\\
% |cdocspt3.tex|&include file for part 3\\
% |cdocspt4.tex|&include file for part 4\\
% |cdocsdrf.tex|&forwarding file for main file in draft mode\\
% |cdocsfi1.tex|&forwarding file for final version of chapter 1\\
% |cdocsfi2.tex|&forwarding file for final version of chapter 2\\
% \end{tabular}
% \end{center}
% Each of the eight files can be compiled directly by the \LaTeX{} compiler.
%
% %%%%%%%%%%%%%%%%%%%%%%%%%%%%%%%%%%%%%%
% \paragraph{Main File.}
%
% The main file is called |cdocsamp.tex|.
%
% Load the \textsf{childdoc} definitions and
% declare the filename for the main document:
%    \begin{macrocode}
% \iffalse
%
% childdoc.dtx Copyright (C) 2017-2018 Niklas Beisert
%
% This work may be distributed and/or modified under the
% conditions of the LaTeX Project Public License, either version 1.3
% of this license or (at your option) any later version.
% The latest version of this license is in
%   http://www.latex-project.org/lppl.txt
% and version 1.3 or later is part of all distributions of LaTeX
% version 2005/12/01 or later.
%
% This work has the LPPL maintenance status `maintained'.
%
% The Current Maintainer of this work is Niklas Beisert.
%
% This work consists of the files childdoc.dtx and childdoc.ins
% and the derived files childdoc.def and cdocsamp.tex with
% cdocsch1.tex, cdocsch2.tex, cdocsdrf.tex, cdocsfn1.tex, cdocsfn2.tex.
%
%<package>\ifdefined\childdocmain\endinput\fi
%<package>\ProvidesFile{childdoc.def}[2018/12/30 v2.0 child document driver]
%<samplemain>\ProvidesFile{cdocsamp.tex}[2018/12/30 v2.0 sample for childdoc]
%<*driver>
%\ProvidesFile{childdoc.drv}[2018/12/30 v2.0 childdoc reference manual file]
\PassOptionsToClass{10pt,a4paper}{article}
\documentclass{ltxdoc}

\usepackage[margin=35mm]{geometry}
\usepackage{hyperref}
\usepackage{hyperxmp}
\usepackage[usenames]{color}

\hypersetup{colorlinks=true}
\hypersetup{pdfstartview=FitH}
\hypersetup{pdfpagemode=UseNone}
\hypersetup{pdfsource={}}
\hypersetup{pdflang={en-UK}}
\hypersetup{pdfcopyright={Copyright 2017-2018 Niklas Beisert.
  This work may be distributed and/or modified under the
  conditions of the LaTeX Project Public License, either version 1.3
  of this license or (at your option) any later version.}}
\hypersetup{pdflicenseurl={http://www.latex-project.org/lppl.txt}}
\hypersetup{pdfcontactaddress={ETH Zurich, ITP, HIT K,
  Wolfgang-Pauli-Strasse 27}}
\hypersetup{pdfcontactpostcode={8093}}
\hypersetup{pdfcontactcity={Zurich}}
\hypersetup{pdfcontactcountry={Switzerland}}
\hypersetup{pdfcontactemail={nbeisert@itp.phys.ethz.ch}}
\hypersetup{pdfcontacturl={http://people.phys.ethz.ch/\xmptilde nbeisert/}}

\newcommand{\secref}[1]{\hyperref[#1]{section \ref*{#1}}}

\parskip1ex
\parindent0pt
\let\olditemize\itemize
\def\itemize{\olditemize\parskip0pt}

\begin{document}

\title{The \textsf{childdoc} Package}
\hypersetup{pdftitle={The childdoc Package}}
\author{Niklas Beisert\\[2ex]
  Institut f\"ur Theoretische Physik\\
  Eidgen\"ossische Technische Hochschule Z\"urich\\
  Wolfgang-Pauli-Strasse 27, 8093 Z\"urich, Switzerland\\[1ex]
  \href{mailto:nbeisert@itp.phys.ethz.ch}
  {\texttt{nbeisert@itp.phys.ethz.ch}}}
\hypersetup{pdfauthor={Niklas Beisert}}
\hypersetup{pdfsubject={Manual for the LaTeX2e Package childdoc}}
\date{30 December 2018, \textsf{v2.0}}
\maketitle

\begin{abstract}\noindent
\textsf{childdoc} is a \LaTeXe{} package
that enables the direct compilation
of document sections included by |\include|
to individual files.
\end{abstract}

\begingroup
\parskip0ex
\tableofcontents
\endgroup

%%%%%%%%%%%%%%%%%%%%%%%%%%%%%%%%%%%%%%%%%%%%%%%%%%%%%%%%%%%%%%%%%%%%%%%%%%%%%%%%
%%%%%%%%%%%%%%%%%%%%%%%%%%%%%%%%%%%%%%%%%%%%%%%%%%%%%%%%%%%%%%%%%%%%%%%%%%%%%%%%
\section{Introduction}

\LaTeX{} provides a mechanism to structure a large document (such as a book)
into a main file and several child files (containing the chapters)
using the |\include| command.
This mechanism is beneficial for documents
which span hundreds of pages in order to
make the source file(s) more manageable.
Moreover, compilation can be restricted to
selected child files by means of the |\includeonly| command.
The latter feature can be used to reduce the compilation time while editing
(this was significantly more useful in the earlier days of \LaTeX{})
or to generate a smaller document which is easier to navigate.
Another application of |\includeonly| is to generate
documents consisting of selected parts of the complete document.

However, there are a few drawbacks of the plain |\include| mechanism:
\begin{itemize}
\item
The child files cannot be compiled on their own,
they can only be compiled via the main file.
A naive editing environment
(such as a text editor with an option
to have the current file processed by \LaTeX)
may require one to switch to the main file before compiling;
attempting to compile the child file produces errors.
\item
The main file must be modified (each time)
to adjust the |\includeonly| command
to the present needs. This easily leaves the main file in a messy state.
\item
The generated document will always carry the filename
of the main document. This is inconvenient if
several child files are to be compiled and
to be kept for distribution.
\end{itemize}

The present package provides a simple interface
to make child files individually compilable by \LaTeX{}.
Compiling a child file then has the same effect as compiling
the main file with an |\includeonly| command
to select the appropriate child.
Moreover the generated document will carry the name of the child
rather than the main file.
This resolves all three above issues.

This feature is meant to make the editing of books,
thesis documents and lecture notes somewhat more convenient.
However, the package can also be used efficiently for
composing a series of documents (such as exercise sheets)
which are typically distributed individually.
It then assists the author in generating the individual documents
(potentially in different versions)
as well as a document containing the collected series.
Another application is in developing style files
or other kinds of included material
where compilation of the style file could redirect
to a sample or test file.

%%%%%%%%%%%%%%%%%%%%%%%%%%%%%%%%%%%%%%%%%%%%%%%%%%%%%%%%%%%%%%%%%%%%%%%%%%%%%%%%
%%%%%%%%%%%%%%%%%%%%%%%%%%%%%%%%%%%%%%%%%%%%%%%%%%%%%%%%%%%%%%%%%%%%%%%%%%%%%%%%
\section{Usage}

First of all, the package \textsf{childdoc} is \emph{not} a standard
\LaTeXe{} |.sty| style file! Therefore it needs to be invoked in
a non-standard way.

%%%%%%%%%%%%%%%%%%%%%%%%%%%%%%%%%%%%%%%%%%%%%%%%%%%%%%%%%%%%%%%%%%%%%%%%%%%%%%%%
\subsection{Included Files}
\label{sec:include}

%%%%%%%%%%%%%%%%%%%%%%%%%%%%%%%%%%%%%%%%
\DescribeMacro{\childdocmain}
To use the package, add the commands
\begin{center}
\begin{tabular}{l}
|\input{childdoc.def}|\\
|\childdocmain{}|\\
\end{tabular}
\end{center}
at the very top of the main \LaTeX{} file,
in particular \emph{before} the |\documentclass| statement!
The argument of |\childdocmain| should be left empty
(but it must be present).

%%%%%%%%%%%%%%%%%%%%%%%%%%%%%%%%%%%%%%%%
\DescribeMacro{\childdocof}
Furthermore, add the commands
\begin{center}
\begin{tabular}{l}
|\input{childdoc.def}|\\
|\childdocof{|\textit{main}|}|\\
\end{tabular}
\end{center}
at the top of every child file \textit{child}
which is included by |\include{|\textit{child}|}|
from within the main file
(or at least for those files to be compiled individually).
The argument \textit{main} must be the filename of the main file.

There are a couple of
considerations in setting up the main and child documents:

%%%%%%%%%%%%%%%%%%%%%%%%%%%%%%%%%%%%%%%%
\paragraph{Restrictions.}

Please note the following restrictions:
\begin{itemize}
\item
|\childdocmain| must be called with one argument \textit{main}
to ensure compatibility with earlier version of the package.
It must either be empty (|\childdocmain{}|)
or precisely match the filename of the main file in which it is specified.
See \secref{sec:detection} for further information.
\item
The filename \textit{main} must be specified without the |.tex| extension.
\item
The filename \textit{main} is case sensitive
(even in case-insensitive file systems)
due to internal string comparison.
\item
The argument \textit{main} should be fully expanded, it cannot be a macro.
\item
Subdirectories and special characters should be avoided in filenames.
\item
The command |\childdocmain{|\textit{main}|}| must be followed by a whitespace.
It should not be followed immediately by another command
or by a comment mark `|%|'.
This is because the \TeX{} parser reads the token immediately following
the argument of |\childdocmain| and puts it
at the beginning of every child section;
however, a white\-space is ignored.
\end{itemize}

%%%%%%%%%%%%%%%%%%%%%%%%%%%%%%%%%%%%%%%%
\paragraph{Content of Main File.}

It is advisable to place all content in the child files included by |\include|.
Any output contained in the main file will appear in all child documents
unless suppressed manually;
it cannot be suppressed automatically by the |\includeonly| directive
and thus should normally be avoided.
A method to include some content in the main file
by means of conditional processing is described in \secref{sec:conditional}.

%%%%%%%%%%%%%%%%%%%%%%%%%%%%%%%%%%%%%%%%
\paragraph{Page Numbering.}

When only a part of the document is compiled,
the appropriate numbering of pages
(as well as other status parameters)
is determined from the |.aux| files.
The latter contain information from previous passes.
However this information needs to propagate through
all intermediate child documents.
Therefore the page numbering in child documents may well
be inconsistent until the complete document is compiled at least once.

A useful (if unconventional) way to always ensure a consistent
page numbering is to restart the numbering in each child document
and denote the pages by `\textit{child}|.|\textit{page}'
where \textit{child} represents the chapter/section number of the child file.
This can be achieved by the command
|\numberwithin{page}{|\textit{child}|}|
of the \textsf{amsmath} package
where \textit{child} can be |chapter| or |section|
depending on the chosen structuring.
Alternatively, one can modify the macro |\thepage| appropriately
and reset the counter |page| at the start of each child file.

%%%%%%%%%%%%%%%%%%%%%%%%%%%%%%%%%%%%%%%%%%%%%%%%%%%%%%%%%%%%%%%%%%%%%%%%%%%%%%%%
\subsection{Conditional Processing}
\label{sec:conditional}

The package provides a mechanism to compile different versions
of a document. To customise the versions further some conditional processing
can come in handy to distinguish which version is being compiled.
The package provides two macros to describe the compilation context:

%%%%%%%%%%%%%%%%%%%%%%%%%%%%%%%%%%%%%%%%
\DescribeMacro{\ifchilddoc}
The conditional |\ifchilddoc| distinguishes between the compilation of
child documents and the main document:
%
\begin{center}
|\ifchilddoc |\textit{child-code}| |[|\||else |\textit{main-code}]| \||fi|
\end{center}

%%%%%%%%%%%%%%%%%%%%%%%%%%%%%%%%%%%%%%%%
\DescribeMacro{\childdocname}
\DescribeMacro{\childdocjob}
The macro |\childdocname| contains the filename (without extension)
of the main or child file being processed.
Note that |\childdocjob| will always contain the name of the main file.

%%%%%%%%%%%%%%%%%%%%%%%%%%%%%%%%%%%%%%%%
\paragraph{Title Page.}

Conditional processing can be used to include a title or banner page
in the main document when proper precautions are taken.
Importantly, the code in the main file should ensure that the page counter
(as well as other status parameters which are stored in the |.aux| files)
takes the same value after the conditional processing.
Otherwise the page numbers may take divergent values
depending on which part is compiled.

For example, a title page could be declared by:
%
\begin{center}
\begin{tabular}{l}
|\ifchilddoc\||else|\\
|\addtocounter{page}{-1}|\\
\textit{code for title page}\\
|\newpage|\\
|\||fi|
\end{tabular}
\end{center}
%
A banner page for the child documents can be generated by:
%
\begin{center}
\begin{tabular}{l}
|\ifchilddoc|\\
|\addtocounter{page}{-1}|\\
\textit{code for banner page}\\
|\newpage|\\
|\||fi|
\end{tabular}
\end{center}
%
Here one could write a message such as:
\begin{center}
|This is the part \childdocname{} of \childdocjob{}.|
\end{center}

%%%%%%%%%%%%%%%%%%%%%%%%%%%%%%%%%%%%%%%%%%%%%%%%%%%%%%%%%%%%%%%%%%%%%%%%%%%%%%%%
\subsection{Flags}
\label{sec:flags}

The package makes it easy to generate different versions
of the main or child documents.
To this end compilation flags can be defined
and assigned different default values.
They will be particularly useful in conjunction
with the forwarding mechanism described in \secref{sec:forward}.

For example, it may be useful to have a flag |\version|
which can be set to |draft| or |final|.
The document source will contain some conditional code
depending on the value of |\version|.
Suppose further, the flag should default to |final| for the main file
and to |draft| for child files
which is a natural assignment for editing the document.
This is achieved by placing the following code
in the preamble of the main document
(below the |\childdocmain| directive):
%
\begin{center}
\begin{tabular}{l}
|\ifchilddoc|\\
|\providecommand{\version}{draft}|\\
|\||else|\\
|\providecommand{\version}{final}|\\
|\||fi|
\end{tabular}
\end{center}
%
The definition by |\providecommand| makes sure
that previous definitions are not overwritten.
Further statements |\providecommand{\version}{...}|
can thus be added before the above code to override it.

For the main file, one might add a line
(between |\childdocmain| and the above block)
%
\begin{center}
|%\ifchilddoc\||else\providecommand{\version}{draft}\||fi|
\end{center}
%
which can be uncommented to produce a draft version.
Likewise one can add a line to the very top of a child file
(above the |\childdocof{|\textit{main}|}| directive)
%
\begin{center}
|%\providecommand{\version}{final}|
\end{center}
%
which can be uncommented to produce the final version of this child document.

%%%%%%%%%%%%%%%%%%%%%%%%%%%%%%%%%%%%%%%%%%%%%%%%%%%%%%%%%%%%%%%%%%%%%%%%%%%%%%%%
\subsection{Forwarding}
\label{sec:forward}

Different versions of the main or child documents
using compilation flags as described in \secref{sec:flags}
can be (permanently) stored in different files
for convenient compilation, viewing and distribution.
To this end, the package defines a command
to pass on compilation to a different file:

%%%%%%%%%%%%%%%%%%%%%%%%%%%%%%%%%%%%%%%%
\DescribeMacro{\childdocforward}
The command |\childdocforward| redirects processing to
another source file:
%
\begin{center}
\begin{tabular}{l}
|\input{childdoc.def}|\\
|\childdocforward[|\textit{main}|]{|\textit{dest}|}|\\
\end{tabular}
\end{center}
%
The argument \textit{dest} is the destination file
(without extension).
It should be the main file or one of the child files.
Note that further \textsf{childdoc} directives
such as |\childdocof| and |\childdocforward|
in the indicated file will be processed in this form.
The optional argument \textit{main}
passes on directly to the main file \textit{main}
while pretending to compile the child \textit{dest}.
This form behaves as if \textit{dest}
issues |\childdocof{|\textit{main}|}| right away,
and no further \textsf{childdoc} directives will be processed.

%%%%%%%%%%%%%%%%%%%%%%%%%%%%%%%%%%%%%%%%
\DescribeMacro{\...prefix}
In the alternative form |\childdocforwardprefix|,
%
\begin{center}
\begin{tabular}{l}
|\input{childdoc.def}|\\
|\childdocforwardprefix[|\textit{main}|]{|\textit{prefix}|}{|\textit{dest}|}|
\end{tabular}
\end{center}
%
the destination file is determined by a pattern
depending on the current file:
To make this work, the current file must be called
`{\textit{prefix}\hspace{0.2em}\textit{suffix}}'
with \textit{prefix} matching precisely the argument.
Processing is then passed on to the file
`{\textit{dest}\hspace{0.2em}\textit{suffix}}'.
Surely, the same effect is achieved by
directly specifying the
argument `{\textit{dest}\hspace{0.2em}\textit{suffix}}'
in the first form.
However, that requires to set up a different file
for each child. With the alternative form of the command
all these files can have exactly the same content
which simplifies setting them up and maintaining them.

For example, the following file |draft.tex|
with a compilation flag |\version| as described in \secref{sec:flags}
compiles the main document as a draft:
%
\begin{center}
\begin{tabular}{l}
|\def\version{draft}|\\
|\input{childdoc.def}|\\
|\childdocforward{|\textit{main}|}|
\end{tabular}
\end{center}
%
Likewise, the following files |final|\textit{nn}|.tex|
compile the final version of the child document
|child|\textit{nn}|.tex|:
%
\begin{center}
\begin{tabular}{l}
|\def\version{final}|\\
|\input{childdoc.def}|\\
|\childdocforwardprefix{final}{child}|
\end{tabular}
\end{center}
%

Note that when several versions of a main file and/or of each child file
are to be generated, it may be convenient to set up a |Makefile| or
shell script to automatise the process.

%%%%%%%%%%%%%%%%%%%%%%%%%%%%%%%%%%%%%%%%%%%%%%%%%%%%%%%%%%%%%%%%%%%%%%%%%%%%%%%%
\subsection{Command Line Processing}
\label{sec:commandline}

The effect of redirection files can also be achieved by invoking
the \LaTeX{} compiler with a more elaborate command line.
Most conveniently this should be done as part
of a shell script or a |Makefile|.

When using \textsf{childdoc} in the main file, the following
command lines effectively perform a redirection
(note that depending on the shell being used,
backslashes may have to be doubled: `|\|' $\to$ `|\\|'):
%
\begin{center}
|... -jobname "|\textit{target}|" |\\|"|[\textit{flags}]%
|\input{childdoc.def}\childdocforward[|\textit{main}|]{|\textit{dest}|}"|
\end{center}
%
Here \textit{target} is the name of the output file,
\textit{main} is the name of the main file
and \textit{dest} is the name of the main or child file to be processed
(all filenames without extensions).
The optional argument \textit{main} can be omitted
if \textit{main} matches \textit{dest}.
Optionally, compilation \textit{flags} can be defined via |\def| commands.
This command line makes the \TeX{} engine believe
it is compiling the file \textit{target}
whose content is specified as the latter parameter.
The provided code then forwards the processing to
\textit{main} or \textit{dest} as described in \secref{sec:forward}.

%%%%%%%%%%%%%%%%%%%%%%%%%%%%%%%%%%%%%%%%%%%%%%%%%%%%%%%%%%%%%%%%%%%%%%%%%%%%%%%%
\subsection{Include by Input}
\label{sec:input}

Including child documents by |\include| has some restrictions by design.
Most notably, the content of a child document always occupies
its own set of pages; pages cannot be shared between child documents.
Usually, this behaviour makes perfect sense
because each child document contain an essential part of the document.
However, in some situations it may be desirable to compose
a document from a collection of parts
without having mandatory page breaks between then.
For this case, the package
provides a mechanism to include parts
by |\input| which can also be processed individually.
However, by construction this mechanism
requires manual handling of the content to be output.

%%%%%%%%%%%%%%%%%%%%%%%%%%%%%%%%%%%%%%%%
\DescribeMacro{\ifchilddocmanual}
The main file should be prepared as usual, see \secref{sec:include}.
However, the document body must make a distinction
between processing of an individual part and of the main document, e.g.:
%
\begin{center}
\begin{tabular}{l}
|\ifchilddocmanual|\\
|\input{\childdocname}|\\
|\||else|\\
\textit{document body with }|\input{|\textit{part}|}|\\
|\||fi|
\end{tabular}
\end{center}
%
The conditional |\ifchilddocmanual| is true whenever
a part to be included by |\input| is being compiled,
and the name of the part is stored in |\childdocname|.

%%%%%%%%%%%%%%%%%%%%%%%%%%%%%%%%%%%%%%%%
\DescribeMacro{\childdocby}
Each part to be included by |\input| should start with:
%
\begin{center}
\begin{tabular}{l}
|\input{childdoc.def}|\\
|\childdocby{|\textit{main}|}|\\
\end{tabular}
\end{center}
%
The directive |\childdocby| is similar to |\childdocof|
described in \secref{sec:include},
but the subsequent selection of content must be done manually.
To that end, both |\ifchilddoc| and |\ifchilddocmanual|
will be true upon processing of a part,
and the name of the part is stored in |\childdocname|.
Note that |\jobname| will be set to the filename of the current part
so that each part receives an individual |.aux| file
that does not interfere with the |.aux| file(s) of the main document.
This behaviour can be altered by the alternative form
|\childdocby[*]{|\textit{main}|}| (with a non-empty optional argument)
which uses the |.aux| file of the main document
by setting |\jobname| to \textit{main}.

%%%%%%%%%%%%%%%%%%%%%%%%%%%%%%%%%%%%%%%%%%%%%%%%%%%%%%%%%%%%%%%%%%%%%%%%%%%%%%%%
\subsection{Driver Development}
\label{sec:driver}

The \textsf{childdoc} mechanism can also be use for the development
of definition files such as \LaTeX{} styles or classes.
This case differs from the above setup with multiple parts
included by |\include| in that no |\includeonly| should be invoked.
This can be achieved by starting the include file
(before |\ProvidesPackage|) with:
%
\begin{center}
\begin{tabular}{l}
|\input{childdoc.def}|\\
|\childdocforward{|\textit{main}|}|\\
\end{tabular}
\end{center}
%
or alternatively with:
%
\begin{center}
\begin{tabular}{l}
|\input{childdoc.def}|\\
|\childdocby{|\textit{main}|}|\\
\end{tabular}
\end{center}
%
Both forms have slightly different effects as described above.
The main file is prepared as usual, see \secref{sec:include}.

%%%%%%%%%%%%%%%%%%%%%%%%%%%%%%%%%%%%%%%%%%%%%%%%%%%%%%%%%%%%%%%%%%%%%%%%%%%%%%%%
\subsection{Legacy Detection}
\label{sec:detection}

The directive |\childdocmain| in the main file can detect
whether the complete document or merely a child is to be compiled
even without using the directive |\childdocof|.
This method is deprecated because it is less robust
and there is no compelling reason to use it;
it is merely provided for backward compatibility
and it may be removed in future versions.

If the detection mechanism is to be used,
it is mandatory to correctly specify
the filename of the main file as the argument of |\childdocmain|:
%
\begin{center}
\begin{tabular}{l}
|\input{childdoc.def}|\\
|\childdocmain{|\textit{main}|}|\\
\end{tabular}
\end{center}
%
If |\jobname| does not match the argument \textit{main} of |\childdocmain|,
it is assumed that |\jobname| points to the child file to be compiled.
When using |\childdocmain| with the main file specified as argument,
it suffices to start a child file
with just |\input{|\textit{main}|}|
without loading of the package and using |\childdocof|.
If instead all processing is done
with the appropriate \textsf{childdoc} directives,
the argument of \textit{main} of |\childdocmain| can be empty.

An alternative version of the command line processing described
in \secref{sec:commandline} using the detection mechanism reads:
%
\begin{center}
|... -jobname "|\textit{target}|" "|[\textit{flags}]%
[|\def\jobname{|\textit{dest}|}|]|\input{|\textit{main}|}"|
\end{center}

%%%%%%%%%%%%%%%%%%%%%%%%%%%%%%%%%%%%%%%%%%%%%%%%%%%%%%%%%%%%%%%%%%%%%%%%%%%%%%%%
\subsection{Manual Code}
\label{sec:manual}

In case one cannot be certain whether the definitions file |childdoc.def|
is installed on the target \TeX{} distribution
and one prefers not to ship it,
it is conceivable to paste a few relevant commands into the sources.

To that end, drop all statements |\input{childdoc.def}|
and perform the replacements as outlined below.
Instead of |\childdocmain{|\textit{main}|}| add the following code
to the top of the main file:
%
\begin{center}
\begin{tabular}{l}
|\||ifdefined\childdocname\endinput\||fi\newif\ifchilddoc|\\
|\edef\childdocname{\scantokens\expandafter{\jobname\noexpand}}|\\
|\def\childdocmain{|\textit{main}|}\||ifx\childdocmain\childdocname\||else|\\
|\childdoctrue\includeonly{\childdocname}\let\jobname\childdocmain\||fi|\\
\end{tabular}
\end{center}
%
Instead of |\childdocof{|\textit{main}|}| just include the main file
at the top of each child file:
%
\begin{center}
|\input{|\textit{main}|}|
\end{center}
%
A simple redirection |\childdocforward{|\textit{dest}|}| is achieved by:
%
\begin{center}
|\def\jobname{|\textit{dest}|}\input{\jobname}|
\end{center}
%
The redirection with prefix
|\childdocforwardprefix[|\textit{prefix}|]{|\textit{dest}|}|
is accomplished by:
%
\begin{center}
\begin{tabular}{l}
|{\edef\jobname{\scantokens\expandafter{\jobname\noexpand}}|\\
|\def\redirectjob |\textit{prefix}|#1~~~{\gdef\jobname{|\textit{dest}|#1}}|\\
|\expandafter\redirectjob\jobname~~~}\input{\jobname}|
\end{tabular}
\end{center}

In an alternative approach,
child documents can be compiled by a specific command line
without additional code or specific definitions:
%
\begin{center}
|... -jobname "|\textit{target}|" "|[\textit{flags}]%
|\includeonly{|\textit{dest}|}\input{|\textit{main}|}"|
\end{center}
%

%%%%%%%%%%%%%%%%%%%%%%%%%%%%%%%%%%%%%%%%%%%%%%%%%%%%%%%%%%%%%%%%%%%%%%%%%%%%%%%%
%%%%%%%%%%%%%%%%%%%%%%%%%%%%%%%%%%%%%%%%%%%%%%%%%%%%%%%%%%%%%%%%%%%%%%%%%%%%%%%%
\section{Information}

%%%%%%%%%%%%%%%%%%%%%%%%%%%%%%%%%%%%%%%%%%%%%%%%%%%%%%%%%%%%%%%%%%%%%%%%%%%%%%%%
\subsection{Copyright}

Copyright \copyright{} 2017--2018 Niklas Beisert

This work may be distributed and/or modified under the
conditions of the \LaTeX{} Project Public License, either version 1.3
of this license or (at your option) any later version.
The latest version of this license is in
  \url{http://www.latex-project.org/lppl.txt}
and version 1.3 or later is part of all distributions of \LaTeX{}
version 2005/12/01 or later.

This work has the LPPL maintenance status `maintained'.

The Current Maintainer of this work is Niklas Beisert.

This work consists of the files |README.txt|, |childdoc.ins| and |childdoc.dtx|
as well as the derived files |childdoc.def|, |cdocsamp.tex|
with |cdocsch1.tex|, |cdocsch2.tex|, |cdocspt3.tex|, |cdocspt4.tex|,
|cdocsdrf.tex|, |cdocsfn1.tex|, |cdocsfn2.tex|
as well as |childdoc.pdf|.

%%%%%%%%%%%%%%%%%%%%%%%%%%%%%%%%%%%%%%%%%%%%%%%%%%%%%%%%%%%%%%%%%%%%%%%%%%%%%%%%
\subsection{Files and Installation}

The package consists of the files:
%
\begin{center}
\begin{tabular}{ll}
    |README.txt|   & readme file \\
    |childdoc.ins| & installation file \\
    |childdoc.dtx| & source file \\
    |childdoc.def| & definition file \\
    |cdocsamp.tex| & sample main file \\
    |cdocsch1.tex| & sample include file \\
    |cdocsch2.tex| & sample include file \\
    |cdocspt3.tex| & sample part file \\
    |cdocspt4.tex| & sample part file \\
    |cdocsdrf.tex| & sample redirection file \\
    |cdocsfn1.tex| & sample redirection file \\
    |cdocsfn2.tex| & sample redirection file \\
    |childdoc.pdf| & manual
\end{tabular}
\end{center}
%
The distribution consists of the files
|README.txt|, |childdoc.ins| and |childdoc.dtx|.
%
\begin{itemize}
\item
Run (pdf)\LaTeX{} on |childdoc.dtx|
to compile the manual |childdoc.pdf| (this file).
\item
Run \LaTeX{} on |childdoc.ins| to create the definitions file |childdoc.def|
and the sample |cdocsamp.tex| with include files
|cdocsch1.tex|, |cdocsch2.tex|, |cdocspt3.tex|, |cdocspt4.tex|,
|cdocsdrf.tex|, |cdocsfn1.tex|, |cdocsfn2.tex|.
Then copy the file |childdoc.def| to an appropriate directory of your \LaTeX{}
distribution, e.g.\ \textit{texmf-root}|/tex/latex/childdoc|.
\end{itemize}

%%%%%%%%%%%%%%%%%%%%%%%%%%%%%%%%%%%%%%%%%%%%%%%%%%%%%%%%%%%%%%%%%%%%%%%%%%%%%%%%
\subsection{Related CTAN Packages}

There are several other packages which offer a similar functionality:
%
\begin{itemize}
\item
The packages
\href{http://ctan.org/pkg/docmute}{\textsf{docmute}},
\href{http://ctan.org/pkg/includex}{\textsf{includex}} and
\href{http://ctan.org/pkg/standalone}{\textsf{standalone}}
provide commands to include only the document body of
a child file thus allowing both files to be compiled individually.
\item
The packages \href{http://ctan.org/pkg/subdocs}{\textsf{subdocs}}
and \href{http://ctan.org/pkg/subfiles}{\textsf{subfiles}}
provide structures in which the main and child documents can be
encapsulated and allowing them to be compiled individually.
The inclusion mechanism is different from the conventional |\include|.
\item
The package \href{http://ctan.org/pkg/combine}{\textsf{combine}}
is an elaborate solution to combine several documents into one.
\end{itemize}
%
See also the CTAN topic \href{http://ctan.org/topic/subdocs}{\textsf{subdocs}}
for further related packages.
The present package differs from the above solutions in that
a document structure constructed with the conventional |\include| mechanism
just needs two extra commands at the top of every file
such that all constituent files can be compiled individually.

%%%%%%%%%%%%%%%%%%%%%%%%%%%%%%%%%%%%%%%%%%%%%%%%%%%%%%%%%%%%%%%%%%%%%%%%%%%%%%%%
%\subsection{Feature Suggestions}
%
%The following is a list of features which may be useful for future
%versions of this package:
%%
%\begin{itemize}
%\item
%\ldots
%\end{itemize}

%%%%%%%%%%%%%%%%%%%%%%%%%%%%%%%%%%%%%%%%%%%%%%%%%%%%%%%%%%%%%%%%%%%%%%%%%%%%%%%%
\subsection{Revision History}

%%%%%%%%%%%%%%%%%%%%%%%%%%%%%%%%%%%%%%%%
\paragraph{v2.0:} 2018/12/30

\begin{itemize}
\item
immediate forward processing
\item
added |\childdocby| mechanism
\item
manual restructured
\end{itemize}

%%%%%%%%%%%%%%%%%%%%%%%%%%%%%%%%%%%%%%%%
\paragraph{v1.6:} 2018/01/17

\begin{itemize}
\item
application for development of include files
\item
corrections to manual
\end{itemize}

%%%%%%%%%%%%%%%%%%%%%%%%%%%%%%%%%%%%%%%%
\paragraph{v1.5:} 2017/05/21

\begin{itemize}
\item
more complete structuring introduced
\item
|\childdocof| introduced
\item
|\childdoc| renamed to |\childdocmain|
\item
|\childredirect| renamed to |\childdocforward| and |\childdocforwardprefix|
and functionality expanded
\end{itemize}

%%%%%%%%%%%%%%%%%%%%%%%%%%%%%%%%%%%%%%%%
\paragraph{v1.0:} 2017/04/27

\begin{itemize}
\item
manual and install package
\item
first version published on CTAN
\end{itemize}

%%%%%%%%%%%%%%%%%%%%%%%%%%%%%%%%%%%%%%%%
\paragraph{v0.6:} 2017/04/26

\begin{itemize}
\item
redirection mechanism added
\end{itemize}

%%%%%%%%%%%%%%%%%%%%%%%%%%%%%%%%%%%%%%%%
\paragraph{v0.5:} 2017/04/26

\begin{itemize}
\item
functionality in definition file
\end{itemize}


%%%%%%%%%%%%%%%%%%%%%%%%%%%%%%%%%%%%%%%%%%%%%%%%%%%%%%%%%%%%%%%%%%%%%%%%%%%%%%%%
%%%%%%%%%%%%%%%%%%%%%%%%%%%%%%%%%%%%%%%%%%%%%%%%%%%%%%%%%%%%%%%%%%%%%%%%%%%%%%%%
%%%%%%%%%%%%%%%%%%%%%%%%%%%%%%%%%%%%%%%%%%%%%%%%%%%%%%%%%%%%%%%%%%%%%%%%%%%%%%%%
\appendix

\settowidth\MacroIndent{\rmfamily\scriptsize 000\ }

 \DocInput{childdoc.dtx}

\end{document}
%</driver>
% \fi
%
% %%%%%%%%%%%%%%%%%%%%%%%%%%%%%%%%%%%%%%%%%%%%%%%%%%%%%%%%%%%%%%%%%%%%%%%%%%%%%%
% %%%%%%%%%%%%%%%%%%%%%%%%%%%%%%%%%%%%%%%%%%%%%%%%%%%%%%%%%%%%%%%%%%%%%%%%%%%%%%
% \section{Sample}
%\iffalse
%<*samplemain>
%\fi
%
% The following presents a sample document
% with two chapters, two parts, a title page,
% a compile flag as well as three forwarding files to set the flag.
% It consists of eight |.tex| files:
% \begin{center}
% \begin{tabular}{ll}
% |cdocsamp.tex|&main file\\
% |cdocsch1.tex|&include file for chapter 1\\
% |cdocsch2.tex|&include file for chapter 2\\
% |cdocspt3.tex|&include file for part 3\\
% |cdocspt4.tex|&include file for part 4\\
% |cdocsdrf.tex|&forwarding file for main file in draft mode\\
% |cdocsfi1.tex|&forwarding file for final version of chapter 1\\
% |cdocsfi2.tex|&forwarding file for final version of chapter 2\\
% \end{tabular}
% \end{center}
% Each of the eight files can be compiled directly by the \LaTeX{} compiler.
%
% %%%%%%%%%%%%%%%%%%%%%%%%%%%%%%%%%%%%%%
% \paragraph{Main File.}
%
% The main file is called |cdocsamp.tex|.
%
% Load the \textsf{childdoc} definitions and
% declare the filename for the main document:
%    \begin{macrocode}
\input{childdoc.def}
\childdocmain{}
%    \end{macrocode}

% Optional override for |\version| flag:
%    \begin{macrocode}
%%\ifchilddoc\else\providecommand{\version}{draft}\fi
%    \end{macrocode}

% Define the default values for the |\version| flag
% (|final| for the main file and |draft| for childs):
%    \begin{macrocode}
\ifchilddoc
\providecommand{\version}{draft}
\else
\providecommand{\version}{final}
\fi
%    \end{macrocode}

% Load the standard document class:
%    \begin{macrocode}
\documentclass[12pt]{article}
%    \end{macrocode}

% Start the document body:
%    \begin{macrocode}
\begin{document}
%    \end{macrocode}

% Declare a title page.
% Print title, part of document being processed and version flag:
%    \begin{macrocode}
\addtocounter{page}{-1}
\begin{center}
{\LARGE\bfseries{}childdoc example\par}
\vspace{1cm}
\ifchilddoc
\ifchilddocmanual part\else chapter\fi:
`\childdocname' of `\childdocjob'\par
\else
main document: `\childdocjob'\par
\fi
version: \version\par
\end{center}
\newpage
%    \end{macrocode}

% Manually include selected file,
% otherwise process as usual:
%    \begin{macrocode}
\ifchilddocmanual
\section*{part `\childdocname'}
\input{\childdocname}
\else
%    \end{macrocode}

% Include the two chapters:
%    \begin{macrocode}
\include{cdocsch1}
\include{cdocsch2}
%    \end{macrocode}

% Include the two parts unless only chapters should be displayed:
%    \begin{macrocode}
\ifchilddoc\else
\section{part three}
\input{cdocspt3}
\section{part four}
\input{cdocspt4}
\fi
%    \end{macrocode}

% Process as usual until here:
%    \begin{macrocode}
\fi
%    \end{macrocode}

% End of document body:
%    \begin{macrocode}
\end{document}
%    \end{macrocode}
%\iffalse
%</samplemain>
%\fi
%
% %%%%%%%%%%%%%%%%%%%%%%%%%%%%%%%%%%%%%%
% \paragraph{Chapter Include Files.}
%
% The include files are called |cdocsch1.tex| and |cdocsch2.tex|.
%
%\iffalse
%<*samplechap1|samplechap2>
%\fi

% Optional override for |\version| flag:
%    \begin{macrocode}
%%\providecommand{\version}{final}
%    \end{macrocode}

% Include the main document:
%    \begin{macrocode}
\input{childdoc.def}
\childdocof{cdocsamp}
%    \end{macrocode}

%\iffalse
%</samplechap1|samplechap2>
%\fi
%
%\iffalse
%<*samplechap1>
%\fi
% Some text for chapter 1:
%    \begin{macrocode}
\section{one}
some text in chapter one
%    \end{macrocode}

%\iffalse
%</samplechap1>
%\fi
% Some text for chapter 2:
%\iffalse
%<*samplechap2>
%\fi
%    \begin{macrocode}
\section{two}
more text in chapter two
%    \end{macrocode}

%\iffalse
%</samplechap2>
%\fi
%
% %%%%%%%%%%%%%%%%%%%%%%%%%%%%%%%%%%%%%%
% \paragraph{Part Include Files.}
%
% The include files are called |cdocspt3.tex| and |cdocspt4.tex|.
%
%\iffalse
%<*samplepart3|samplepart4>
%\fi

% Optional override for |\version| flag:
%    \begin{macrocode}
%%\providecommand{\version}{final}
%    \end{macrocode}

% Include the main document:
%    \begin{macrocode}
\input{childdoc.def}
\childdocby{cdocsamp}
%    \end{macrocode}

%\iffalse
%</samplepart3|samplepart4>
%\fi
%
%\iffalse
%<*samplepart3>
%\fi
% Some text for part 3:
%    \begin{macrocode}
some text in part three
%    \end{macrocode}

%\iffalse
%</samplepart3>
%\fi
% Some text for part 4:
%\iffalse
%<*samplepart4>
%\fi
%    \begin{macrocode}
more text in part four
%    \end{macrocode}

%\iffalse
%</samplepart4>
%\fi
%
% %%%%%%%%%%%%%%%%%%%%%%%%%%%%%%%%%%%%%%
% \paragraph{Forwarding for a Complete Draft.}
%
% The following forwarding file |cdocsdrf.tex|
% compiles the main document in draft mode:
%\iffalse
%<*sampledraft>
%\fi
%    \begin{macrocode}
\def\version{draft}
\input{childdoc.def}
\childdocforward{cdocsamp}
%    \end{macrocode}

%\iffalse
%</sampledraft>
%\fi
%
% %%%%%%%%%%%%%%%%%%%%%%%%%%%%%%%%%%%%%%
% \paragraph{Forwarding for Final Version of the Chapters.}
%
% The following forwarding files |cdocsfn1.tex| and |cdocsfn2.tex|
% (with identical content)
% compile the final versions of the child documents
% |cdocsch1.tex| and |cdocsch2.tex|, respectively:
%\iffalse
%<*samplefinal>
%\fi
%    \begin{macrocode}
\def\version{final}
\input{childdoc.def}
\childdocforwardprefix[cdocsamp]{cdocsfn}{cdocsch}
%    \end{macrocode}

%\iffalse
%</samplefinal>
%\fi
%
% %%%%%%%%%%%%%%%%%%%%%%%%%%%%%%%%%%%%%%
% \paragraph{Command Line Processing.}
%
% The following three command lines generate the output files
% |cdocscld|, |cdocscl1| and |cdocscl2|
% which should be identical to
% |cdocsdrf|, |cdocsch1| and |cdocsfn2|, respectively:
% \begin{center}
% \begin{tabular}{l}
% |latex -jobname cdocscld \|\\
% |  "\def\version{draft}\input{childdoc.def}\childdocforward{cdocsamp}"|\\
% |latex -jobname cdocscl1 \|\\
% |  "\input{childdoc.def}\childdocforward[cdocsamp]{cdocsch1}"|\\
% |latex -jobname cdocscl2 \|\\
% |  "\def\version{final}\input{childdoc.def}\childdocforward{cdocsch2}"|
% \end{tabular}
% \end{center}
% Note that the trailing backslash on each first line
% merely continues the input to the second line
% (for convenient cut ant paste).
% Furthermore, the command |latex| can be replaced by any
% of its alternative versions such as |pdflatex|.
%
% %%%%%%%%%%%%%%%%%%%%%%%%%%%%%%%%%%%%%%%%%%%%%%%%%%%%%%%%%%%%%%%%%%%%%%%%%%%%%%
% %%%%%%%%%%%%%%%%%%%%%%%%%%%%%%%%%%%%%%%%%%%%%%%%%%%%%%%%%%%%%%%%%%%%%%%%%%%%%%
% \section{Implementation}
%\iffalse
%<*package>
%\fi
%
% This section describes the definitions file |childdoc.def|.

% The definitions cannot be loaded using |\usepackage| or |\RequirePackage|
% which has a mechanism to prevent loading a style file more than once.
% When loading the definitions by means of |\input|
% multiple instances have to be prevented manually:
%\iffalse
%This code needs to be before the `\ProvidesFile' directive
%which is defined at the beginning of this file.
%Therefore it is also placed there and commented out here.
%</package>
%<*discard>
%\fi
%    \begin{macrocode}
\ifdefined\childdocmain\endinput\fi
%    \end{macrocode}
%\iffalse
%</discard>
%<*package>
%\fi
%
% \macro{\ifchilddoc}
% \macro{\ifchilddocmanual}
% The conditional |\ifchilddoc| tells whether a
% child (true) or main (false) document is being compiled.
% The conditional |\ifchilddocmanual| tells whether
% the |\includeonly| mechanism is used (false) or
% the selection of child files must be performed manually (true).
% The definitions initialise to false:
%    \begin{macrocode}
\newif\ifchilddoc
\newif\ifchilddocmanual
%    \end{macrocode}

% \macro{\childdocname}
% \macro{\childdocjob}
% The macro |\childdocname| stores the name of the main document
% to be compiled. The macro |\childdocjob| stores the name of
% the document on which the \LaTeX{} compiler was originally invoked.
% The content of |\jobname| cannot be compared
% to filenames specified in the source due to different catcodes.
% The following code rescans |\jobname|, stores the result
% in |\childdocname| and saves a copy in |\childdocjob|:
%    \begin{macrocode}
\edef\childdocname{\scantokens\expandafter{\jobname\noexpand}}
\let\childdocjob\childdocname
%    \end{macrocode}

% \macro{\childdocdisable}
% The macro |\childdocdisable| prevents the main file
% from being processed more than once.
% At this stage, the main document command |\childdocmain|
% is assumed to be called once again where it should do nothing.
% Any subsequent call to it should prevent
% a secondary processing of the main document
% It overwrites the forwarding commands
% |\childdocof| and |\childdocforward|
% with empty macros to prevent further inclusions of the main document:
%    \begin{macrocode}
\newcommand{\childdocdisable}
{
  \renewcommand{\childdocmain}[1]{\renewcommand{\childdocmain}[1]{\endinput}}
  \renewcommand{\childdocof}[1]{}
  \renewcommand{\childdocby}[2][]{}
  \renewcommand{\childdocforward}[2][]{}
  \renewcommand{\childdocdisable}{}
}
%    \end{macrocode}

% \macro{\childdocmain}
% The macro |\childdocmain| is to be called at the top of the main file
% with nothing or the main filename (without extension) as argument.
% First, it breaks loops.
% If the argument is not empty and does not match |\childdocname|
% (which is set by the first inclusion of |childdoc.def|),
% |\ifchilddoc| is set to true, |\includeonly| is applied to the child file
% and |\jobname| is set to the main file
% (for proper handling of |.aux| files):
%    \begin{macrocode}
\newcommand{\childdocmain}[1]
{
  \childdocdisable\childdocmain{}
  \if?#1?\else
    \begingroup
      \def\childdoctmp{#1}
      \ifx\childdoctmp\childdocname
        \def\childdoctmp{}
      \else
        \def\childdoctmp
        {
          \childdoctrue
          \includeonly{\childdocname}
          \def\childdocjob{#1}
          \def\jobname{#1}
        }
      \fi
      \expandafter
    \endgroup
    \childdoctmp
  \fi
}
%    \end{macrocode}

% \macro{\childdocof}
% The command |\childdocof| redirects
% compilation to the main file |#1|.
%    \begin{macrocode}
\newcommand{\childdocof}[1]
{
  \childdocdisable
  \childdoctrue
  \includeonly{\childdocname}
  \def\jobname{#1}
  \def\childdocjob{#1}
  \input{#1}
}
%    \end{macrocode}

% \macro{\childdocby}
% The command |\childdocby| ....
%    \begin{macrocode}
\newcommand{\childdocby}[2][]
{
  \childdocdisable
  \childdoctrue
  \childdocmanualtrue
  \if?#1?\else
    \def\jobname{#2}
  \fi
  \def\childdocjob{#2}
  \input{#2}
  \endinput
}
%    \end{macrocode}

% \macro{\childdocforward}
% The command |\childdocforward| redirects
% compilation to the main file or
% (if the optional argument is given) a child file.
% Parameters are set as if the main file
% or a child file starting with |\childdocof| was compiled.
% Then compilation is handed over to the main file:
%    \begin{macrocode}
\newcommand{\childdocforward}[2][]
{
  \begingroup
    \if?#1?
      \def\childdoctmp
      {
        \def\childdocname{#2}
        \def\childdocjob{#2}
        \def\jobname{#2}
        \input{#2}
        \endinput
      }
    \else
      \def\childdoctmp
      {
        \childdocdisable
        \def\childdocname{#2}
        \childdoctrue
        \includeonly{#2}
        \def\childdocjob{#1}
        \def\jobname{#1}
        \input{#1}
        \endinput
      }
    \fi
    \expandafter
  \endgroup
  \childdoctmp
}
%    \end{macrocode}

% \macro{\childdocforwardprefix}
% The command |\childdocforwardprefix| redirects
% compilation to the main or a child file by means of a pattern.
% The prefix |#1| in the current filename is replaced by |#2|
% and the suffix of the current filename is kept
% (it is assumed that the filename does not contain the substring `|~~~|'
% which is used as a delimiter).
% Compilation is handed over to the new file by |\childdocforward|:
%    \begin{macrocode}
\newcommand{\childdocforwardprefix}[3][]
{
  \begingroup
    \def\childdocextract #2##1~~~{\def\childdoctmp{\childdocforward[#1]{#3##1}}}
    \expandafter\childdocextract\childdocname~~~
    \expandafter
  \endgroup
  \childdoctmp
}
%    \end{macrocode}

% \macro{\childdoc}
% The deprecated macro |\childdoc| is a legacy version of |\childdocmain|:
%    \begin{macrocode}
\newcommand{\childdoc}{\childdocmain}
%    \end{macrocode}

% \macro{\childdocredirect}
% The deprecated macro |\childdocredirect| is a legacy version
% of |\childdocforward| and |\childdocforwardprefix|:
%    \begin{macrocode}
\newcommand{\childdocredirect}[2][]
{
  \begingroup
    \if?#1?
      \def\childdoctmp{\childdocforward{#2}}
    \else
      \def\childdoctmp{\childdocforwardprefix{#1}{#2}}
    \fi
    \expandafter
  \endgroup
  \childdoctmp
}
%    \end{macrocode}

%\iffalse
%</package>
%\fi
%
\endinput

\childdocmain{}
%    \end{macrocode}

% Optional override for |\version| flag:
%    \begin{macrocode}
%%\ifchilddoc\else\providecommand{\version}{draft}\fi
%    \end{macrocode}

% Define the default values for the |\version| flag
% (|final| for the main file and |draft| for childs):
%    \begin{macrocode}
\ifchilddoc
\providecommand{\version}{draft}
\else
\providecommand{\version}{final}
\fi
%    \end{macrocode}

% Load the standard document class:
%    \begin{macrocode}
\documentclass[12pt]{article}
%    \end{macrocode}

% Start the document body:
%    \begin{macrocode}
\begin{document}
%    \end{macrocode}

% Declare a title page.
% Print title, part of document being processed and version flag:
%    \begin{macrocode}
\addtocounter{page}{-1}
\begin{center}
{\LARGE\bfseries{}childdoc example\par}
\vspace{1cm}
\ifchilddoc
\ifchilddocmanual part\else chapter\fi:
`\childdocname' of `\childdocjob'\par
\else
main document: `\childdocjob'\par
\fi
version: \version\par
\end{center}
\newpage
%    \end{macrocode}

% Manually include selected file,
% otherwise process as usual:
%    \begin{macrocode}
\ifchilddocmanual
\section*{part `\childdocname'}
\input{\childdocname}
\else
%    \end{macrocode}

% Include the two chapters:
%    \begin{macrocode}
\include{cdocsch1}
\include{cdocsch2}
%    \end{macrocode}

% Include the two parts unless only chapters should be displayed:
%    \begin{macrocode}
\ifchilddoc\else
\section{part three}
\input{cdocspt3}
\section{part four}
\input{cdocspt4}
\fi
%    \end{macrocode}

% Process as usual until here:
%    \begin{macrocode}
\fi
%    \end{macrocode}

% End of document body:
%    \begin{macrocode}
\end{document}
%    \end{macrocode}
%\iffalse
%</samplemain>
%\fi
%
% %%%%%%%%%%%%%%%%%%%%%%%%%%%%%%%%%%%%%%
% \paragraph{Chapter Include Files.}
%
% The include files are called |cdocsch1.tex| and |cdocsch2.tex|.
%
%\iffalse
%<*samplechap1|samplechap2>
%\fi

% Optional override for |\version| flag:
%    \begin{macrocode}
%%\providecommand{\version}{final}
%    \end{macrocode}

% Include the main document:
%    \begin{macrocode}
% \iffalse
%
% childdoc.dtx Copyright (C) 2017-2018 Niklas Beisert
%
% This work may be distributed and/or modified under the
% conditions of the LaTeX Project Public License, either version 1.3
% of this license or (at your option) any later version.
% The latest version of this license is in
%   http://www.latex-project.org/lppl.txt
% and version 1.3 or later is part of all distributions of LaTeX
% version 2005/12/01 or later.
%
% This work has the LPPL maintenance status `maintained'.
%
% The Current Maintainer of this work is Niklas Beisert.
%
% This work consists of the files childdoc.dtx and childdoc.ins
% and the derived files childdoc.def and cdocsamp.tex with
% cdocsch1.tex, cdocsch2.tex, cdocsdrf.tex, cdocsfn1.tex, cdocsfn2.tex.
%
%<package>\ifdefined\childdocmain\endinput\fi
%<package>\ProvidesFile{childdoc.def}[2018/12/30 v2.0 child document driver]
%<samplemain>\ProvidesFile{cdocsamp.tex}[2018/12/30 v2.0 sample for childdoc]
%<*driver>
%\ProvidesFile{childdoc.drv}[2018/12/30 v2.0 childdoc reference manual file]
\PassOptionsToClass{10pt,a4paper}{article}
\documentclass{ltxdoc}

\usepackage[margin=35mm]{geometry}
\usepackage{hyperref}
\usepackage{hyperxmp}
\usepackage[usenames]{color}

\hypersetup{colorlinks=true}
\hypersetup{pdfstartview=FitH}
\hypersetup{pdfpagemode=UseNone}
\hypersetup{pdfsource={}}
\hypersetup{pdflang={en-UK}}
\hypersetup{pdfcopyright={Copyright 2017-2018 Niklas Beisert.
  This work may be distributed and/or modified under the
  conditions of the LaTeX Project Public License, either version 1.3
  of this license or (at your option) any later version.}}
\hypersetup{pdflicenseurl={http://www.latex-project.org/lppl.txt}}
\hypersetup{pdfcontactaddress={ETH Zurich, ITP, HIT K,
  Wolfgang-Pauli-Strasse 27}}
\hypersetup{pdfcontactpostcode={8093}}
\hypersetup{pdfcontactcity={Zurich}}
\hypersetup{pdfcontactcountry={Switzerland}}
\hypersetup{pdfcontactemail={nbeisert@itp.phys.ethz.ch}}
\hypersetup{pdfcontacturl={http://people.phys.ethz.ch/\xmptilde nbeisert/}}

\newcommand{\secref}[1]{\hyperref[#1]{section \ref*{#1}}}

\parskip1ex
\parindent0pt
\let\olditemize\itemize
\def\itemize{\olditemize\parskip0pt}

\begin{document}

\title{The \textsf{childdoc} Package}
\hypersetup{pdftitle={The childdoc Package}}
\author{Niklas Beisert\\[2ex]
  Institut f\"ur Theoretische Physik\\
  Eidgen\"ossische Technische Hochschule Z\"urich\\
  Wolfgang-Pauli-Strasse 27, 8093 Z\"urich, Switzerland\\[1ex]
  \href{mailto:nbeisert@itp.phys.ethz.ch}
  {\texttt{nbeisert@itp.phys.ethz.ch}}}
\hypersetup{pdfauthor={Niklas Beisert}}
\hypersetup{pdfsubject={Manual for the LaTeX2e Package childdoc}}
\date{30 December 2018, \textsf{v2.0}}
\maketitle

\begin{abstract}\noindent
\textsf{childdoc} is a \LaTeXe{} package
that enables the direct compilation
of document sections included by |\include|
to individual files.
\end{abstract}

\begingroup
\parskip0ex
\tableofcontents
\endgroup

%%%%%%%%%%%%%%%%%%%%%%%%%%%%%%%%%%%%%%%%%%%%%%%%%%%%%%%%%%%%%%%%%%%%%%%%%%%%%%%%
%%%%%%%%%%%%%%%%%%%%%%%%%%%%%%%%%%%%%%%%%%%%%%%%%%%%%%%%%%%%%%%%%%%%%%%%%%%%%%%%
\section{Introduction}

\LaTeX{} provides a mechanism to structure a large document (such as a book)
into a main file and several child files (containing the chapters)
using the |\include| command.
This mechanism is beneficial for documents
which span hundreds of pages in order to
make the source file(s) more manageable.
Moreover, compilation can be restricted to
selected child files by means of the |\includeonly| command.
The latter feature can be used to reduce the compilation time while editing
(this was significantly more useful in the earlier days of \LaTeX{})
or to generate a smaller document which is easier to navigate.
Another application of |\includeonly| is to generate
documents consisting of selected parts of the complete document.

However, there are a few drawbacks of the plain |\include| mechanism:
\begin{itemize}
\item
The child files cannot be compiled on their own,
they can only be compiled via the main file.
A naive editing environment
(such as a text editor with an option
to have the current file processed by \LaTeX)
may require one to switch to the main file before compiling;
attempting to compile the child file produces errors.
\item
The main file must be modified (each time)
to adjust the |\includeonly| command
to the present needs. This easily leaves the main file in a messy state.
\item
The generated document will always carry the filename
of the main document. This is inconvenient if
several child files are to be compiled and
to be kept for distribution.
\end{itemize}

The present package provides a simple interface
to make child files individually compilable by \LaTeX{}.
Compiling a child file then has the same effect as compiling
the main file with an |\includeonly| command
to select the appropriate child.
Moreover the generated document will carry the name of the child
rather than the main file.
This resolves all three above issues.

This feature is meant to make the editing of books,
thesis documents and lecture notes somewhat more convenient.
However, the package can also be used efficiently for
composing a series of documents (such as exercise sheets)
which are typically distributed individually.
It then assists the author in generating the individual documents
(potentially in different versions)
as well as a document containing the collected series.
Another application is in developing style files
or other kinds of included material
where compilation of the style file could redirect
to a sample or test file.

%%%%%%%%%%%%%%%%%%%%%%%%%%%%%%%%%%%%%%%%%%%%%%%%%%%%%%%%%%%%%%%%%%%%%%%%%%%%%%%%
%%%%%%%%%%%%%%%%%%%%%%%%%%%%%%%%%%%%%%%%%%%%%%%%%%%%%%%%%%%%%%%%%%%%%%%%%%%%%%%%
\section{Usage}

First of all, the package \textsf{childdoc} is \emph{not} a standard
\LaTeXe{} |.sty| style file! Therefore it needs to be invoked in
a non-standard way.

%%%%%%%%%%%%%%%%%%%%%%%%%%%%%%%%%%%%%%%%%%%%%%%%%%%%%%%%%%%%%%%%%%%%%%%%%%%%%%%%
\subsection{Included Files}
\label{sec:include}

%%%%%%%%%%%%%%%%%%%%%%%%%%%%%%%%%%%%%%%%
\DescribeMacro{\childdocmain}
To use the package, add the commands
\begin{center}
\begin{tabular}{l}
|\input{childdoc.def}|\\
|\childdocmain{}|\\
\end{tabular}
\end{center}
at the very top of the main \LaTeX{} file,
in particular \emph{before} the |\documentclass| statement!
The argument of |\childdocmain| should be left empty
(but it must be present).

%%%%%%%%%%%%%%%%%%%%%%%%%%%%%%%%%%%%%%%%
\DescribeMacro{\childdocof}
Furthermore, add the commands
\begin{center}
\begin{tabular}{l}
|\input{childdoc.def}|\\
|\childdocof{|\textit{main}|}|\\
\end{tabular}
\end{center}
at the top of every child file \textit{child}
which is included by |\include{|\textit{child}|}|
from within the main file
(or at least for those files to be compiled individually).
The argument \textit{main} must be the filename of the main file.

There are a couple of
considerations in setting up the main and child documents:

%%%%%%%%%%%%%%%%%%%%%%%%%%%%%%%%%%%%%%%%
\paragraph{Restrictions.}

Please note the following restrictions:
\begin{itemize}
\item
|\childdocmain| must be called with one argument \textit{main}
to ensure compatibility with earlier version of the package.
It must either be empty (|\childdocmain{}|)
or precisely match the filename of the main file in which it is specified.
See \secref{sec:detection} for further information.
\item
The filename \textit{main} must be specified without the |.tex| extension.
\item
The filename \textit{main} is case sensitive
(even in case-insensitive file systems)
due to internal string comparison.
\item
The argument \textit{main} should be fully expanded, it cannot be a macro.
\item
Subdirectories and special characters should be avoided in filenames.
\item
The command |\childdocmain{|\textit{main}|}| must be followed by a whitespace.
It should not be followed immediately by another command
or by a comment mark `|%|'.
This is because the \TeX{} parser reads the token immediately following
the argument of |\childdocmain| and puts it
at the beginning of every child section;
however, a white\-space is ignored.
\end{itemize}

%%%%%%%%%%%%%%%%%%%%%%%%%%%%%%%%%%%%%%%%
\paragraph{Content of Main File.}

It is advisable to place all content in the child files included by |\include|.
Any output contained in the main file will appear in all child documents
unless suppressed manually;
it cannot be suppressed automatically by the |\includeonly| directive
and thus should normally be avoided.
A method to include some content in the main file
by means of conditional processing is described in \secref{sec:conditional}.

%%%%%%%%%%%%%%%%%%%%%%%%%%%%%%%%%%%%%%%%
\paragraph{Page Numbering.}

When only a part of the document is compiled,
the appropriate numbering of pages
(as well as other status parameters)
is determined from the |.aux| files.
The latter contain information from previous passes.
However this information needs to propagate through
all intermediate child documents.
Therefore the page numbering in child documents may well
be inconsistent until the complete document is compiled at least once.

A useful (if unconventional) way to always ensure a consistent
page numbering is to restart the numbering in each child document
and denote the pages by `\textit{child}|.|\textit{page}'
where \textit{child} represents the chapter/section number of the child file.
This can be achieved by the command
|\numberwithin{page}{|\textit{child}|}|
of the \textsf{amsmath} package
where \textit{child} can be |chapter| or |section|
depending on the chosen structuring.
Alternatively, one can modify the macro |\thepage| appropriately
and reset the counter |page| at the start of each child file.

%%%%%%%%%%%%%%%%%%%%%%%%%%%%%%%%%%%%%%%%%%%%%%%%%%%%%%%%%%%%%%%%%%%%%%%%%%%%%%%%
\subsection{Conditional Processing}
\label{sec:conditional}

The package provides a mechanism to compile different versions
of a document. To customise the versions further some conditional processing
can come in handy to distinguish which version is being compiled.
The package provides two macros to describe the compilation context:

%%%%%%%%%%%%%%%%%%%%%%%%%%%%%%%%%%%%%%%%
\DescribeMacro{\ifchilddoc}
The conditional |\ifchilddoc| distinguishes between the compilation of
child documents and the main document:
%
\begin{center}
|\ifchilddoc |\textit{child-code}| |[|\||else |\textit{main-code}]| \||fi|
\end{center}

%%%%%%%%%%%%%%%%%%%%%%%%%%%%%%%%%%%%%%%%
\DescribeMacro{\childdocname}
\DescribeMacro{\childdocjob}
The macro |\childdocname| contains the filename (without extension)
of the main or child file being processed.
Note that |\childdocjob| will always contain the name of the main file.

%%%%%%%%%%%%%%%%%%%%%%%%%%%%%%%%%%%%%%%%
\paragraph{Title Page.}

Conditional processing can be used to include a title or banner page
in the main document when proper precautions are taken.
Importantly, the code in the main file should ensure that the page counter
(as well as other status parameters which are stored in the |.aux| files)
takes the same value after the conditional processing.
Otherwise the page numbers may take divergent values
depending on which part is compiled.

For example, a title page could be declared by:
%
\begin{center}
\begin{tabular}{l}
|\ifchilddoc\||else|\\
|\addtocounter{page}{-1}|\\
\textit{code for title page}\\
|\newpage|\\
|\||fi|
\end{tabular}
\end{center}
%
A banner page for the child documents can be generated by:
%
\begin{center}
\begin{tabular}{l}
|\ifchilddoc|\\
|\addtocounter{page}{-1}|\\
\textit{code for banner page}\\
|\newpage|\\
|\||fi|
\end{tabular}
\end{center}
%
Here one could write a message such as:
\begin{center}
|This is the part \childdocname{} of \childdocjob{}.|
\end{center}

%%%%%%%%%%%%%%%%%%%%%%%%%%%%%%%%%%%%%%%%%%%%%%%%%%%%%%%%%%%%%%%%%%%%%%%%%%%%%%%%
\subsection{Flags}
\label{sec:flags}

The package makes it easy to generate different versions
of the main or child documents.
To this end compilation flags can be defined
and assigned different default values.
They will be particularly useful in conjunction
with the forwarding mechanism described in \secref{sec:forward}.

For example, it may be useful to have a flag |\version|
which can be set to |draft| or |final|.
The document source will contain some conditional code
depending on the value of |\version|.
Suppose further, the flag should default to |final| for the main file
and to |draft| for child files
which is a natural assignment for editing the document.
This is achieved by placing the following code
in the preamble of the main document
(below the |\childdocmain| directive):
%
\begin{center}
\begin{tabular}{l}
|\ifchilddoc|\\
|\providecommand{\version}{draft}|\\
|\||else|\\
|\providecommand{\version}{final}|\\
|\||fi|
\end{tabular}
\end{center}
%
The definition by |\providecommand| makes sure
that previous definitions are not overwritten.
Further statements |\providecommand{\version}{...}|
can thus be added before the above code to override it.

For the main file, one might add a line
(between |\childdocmain| and the above block)
%
\begin{center}
|%\ifchilddoc\||else\providecommand{\version}{draft}\||fi|
\end{center}
%
which can be uncommented to produce a draft version.
Likewise one can add a line to the very top of a child file
(above the |\childdocof{|\textit{main}|}| directive)
%
\begin{center}
|%\providecommand{\version}{final}|
\end{center}
%
which can be uncommented to produce the final version of this child document.

%%%%%%%%%%%%%%%%%%%%%%%%%%%%%%%%%%%%%%%%%%%%%%%%%%%%%%%%%%%%%%%%%%%%%%%%%%%%%%%%
\subsection{Forwarding}
\label{sec:forward}

Different versions of the main or child documents
using compilation flags as described in \secref{sec:flags}
can be (permanently) stored in different files
for convenient compilation, viewing and distribution.
To this end, the package defines a command
to pass on compilation to a different file:

%%%%%%%%%%%%%%%%%%%%%%%%%%%%%%%%%%%%%%%%
\DescribeMacro{\childdocforward}
The command |\childdocforward| redirects processing to
another source file:
%
\begin{center}
\begin{tabular}{l}
|\input{childdoc.def}|\\
|\childdocforward[|\textit{main}|]{|\textit{dest}|}|\\
\end{tabular}
\end{center}
%
The argument \textit{dest} is the destination file
(without extension).
It should be the main file or one of the child files.
Note that further \textsf{childdoc} directives
such as |\childdocof| and |\childdocforward|
in the indicated file will be processed in this form.
The optional argument \textit{main}
passes on directly to the main file \textit{main}
while pretending to compile the child \textit{dest}.
This form behaves as if \textit{dest}
issues |\childdocof{|\textit{main}|}| right away,
and no further \textsf{childdoc} directives will be processed.

%%%%%%%%%%%%%%%%%%%%%%%%%%%%%%%%%%%%%%%%
\DescribeMacro{\...prefix}
In the alternative form |\childdocforwardprefix|,
%
\begin{center}
\begin{tabular}{l}
|\input{childdoc.def}|\\
|\childdocforwardprefix[|\textit{main}|]{|\textit{prefix}|}{|\textit{dest}|}|
\end{tabular}
\end{center}
%
the destination file is determined by a pattern
depending on the current file:
To make this work, the current file must be called
`{\textit{prefix}\hspace{0.2em}\textit{suffix}}'
with \textit{prefix} matching precisely the argument.
Processing is then passed on to the file
`{\textit{dest}\hspace{0.2em}\textit{suffix}}'.
Surely, the same effect is achieved by
directly specifying the
argument `{\textit{dest}\hspace{0.2em}\textit{suffix}}'
in the first form.
However, that requires to set up a different file
for each child. With the alternative form of the command
all these files can have exactly the same content
which simplifies setting them up and maintaining them.

For example, the following file |draft.tex|
with a compilation flag |\version| as described in \secref{sec:flags}
compiles the main document as a draft:
%
\begin{center}
\begin{tabular}{l}
|\def\version{draft}|\\
|\input{childdoc.def}|\\
|\childdocforward{|\textit{main}|}|
\end{tabular}
\end{center}
%
Likewise, the following files |final|\textit{nn}|.tex|
compile the final version of the child document
|child|\textit{nn}|.tex|:
%
\begin{center}
\begin{tabular}{l}
|\def\version{final}|\\
|\input{childdoc.def}|\\
|\childdocforwardprefix{final}{child}|
\end{tabular}
\end{center}
%

Note that when several versions of a main file and/or of each child file
are to be generated, it may be convenient to set up a |Makefile| or
shell script to automatise the process.

%%%%%%%%%%%%%%%%%%%%%%%%%%%%%%%%%%%%%%%%%%%%%%%%%%%%%%%%%%%%%%%%%%%%%%%%%%%%%%%%
\subsection{Command Line Processing}
\label{sec:commandline}

The effect of redirection files can also be achieved by invoking
the \LaTeX{} compiler with a more elaborate command line.
Most conveniently this should be done as part
of a shell script or a |Makefile|.

When using \textsf{childdoc} in the main file, the following
command lines effectively perform a redirection
(note that depending on the shell being used,
backslashes may have to be doubled: `|\|' $\to$ `|\\|'):
%
\begin{center}
|... -jobname "|\textit{target}|" |\\|"|[\textit{flags}]%
|\input{childdoc.def}\childdocforward[|\textit{main}|]{|\textit{dest}|}"|
\end{center}
%
Here \textit{target} is the name of the output file,
\textit{main} is the name of the main file
and \textit{dest} is the name of the main or child file to be processed
(all filenames without extensions).
The optional argument \textit{main} can be omitted
if \textit{main} matches \textit{dest}.
Optionally, compilation \textit{flags} can be defined via |\def| commands.
This command line makes the \TeX{} engine believe
it is compiling the file \textit{target}
whose content is specified as the latter parameter.
The provided code then forwards the processing to
\textit{main} or \textit{dest} as described in \secref{sec:forward}.

%%%%%%%%%%%%%%%%%%%%%%%%%%%%%%%%%%%%%%%%%%%%%%%%%%%%%%%%%%%%%%%%%%%%%%%%%%%%%%%%
\subsection{Include by Input}
\label{sec:input}

Including child documents by |\include| has some restrictions by design.
Most notably, the content of a child document always occupies
its own set of pages; pages cannot be shared between child documents.
Usually, this behaviour makes perfect sense
because each child document contain an essential part of the document.
However, in some situations it may be desirable to compose
a document from a collection of parts
without having mandatory page breaks between then.
For this case, the package
provides a mechanism to include parts
by |\input| which can also be processed individually.
However, by construction this mechanism
requires manual handling of the content to be output.

%%%%%%%%%%%%%%%%%%%%%%%%%%%%%%%%%%%%%%%%
\DescribeMacro{\ifchilddocmanual}
The main file should be prepared as usual, see \secref{sec:include}.
However, the document body must make a distinction
between processing of an individual part and of the main document, e.g.:
%
\begin{center}
\begin{tabular}{l}
|\ifchilddocmanual|\\
|\input{\childdocname}|\\
|\||else|\\
\textit{document body with }|\input{|\textit{part}|}|\\
|\||fi|
\end{tabular}
\end{center}
%
The conditional |\ifchilddocmanual| is true whenever
a part to be included by |\input| is being compiled,
and the name of the part is stored in |\childdocname|.

%%%%%%%%%%%%%%%%%%%%%%%%%%%%%%%%%%%%%%%%
\DescribeMacro{\childdocby}
Each part to be included by |\input| should start with:
%
\begin{center}
\begin{tabular}{l}
|\input{childdoc.def}|\\
|\childdocby{|\textit{main}|}|\\
\end{tabular}
\end{center}
%
The directive |\childdocby| is similar to |\childdocof|
described in \secref{sec:include},
but the subsequent selection of content must be done manually.
To that end, both |\ifchilddoc| and |\ifchilddocmanual|
will be true upon processing of a part,
and the name of the part is stored in |\childdocname|.
Note that |\jobname| will be set to the filename of the current part
so that each part receives an individual |.aux| file
that does not interfere with the |.aux| file(s) of the main document.
This behaviour can be altered by the alternative form
|\childdocby[*]{|\textit{main}|}| (with a non-empty optional argument)
which uses the |.aux| file of the main document
by setting |\jobname| to \textit{main}.

%%%%%%%%%%%%%%%%%%%%%%%%%%%%%%%%%%%%%%%%%%%%%%%%%%%%%%%%%%%%%%%%%%%%%%%%%%%%%%%%
\subsection{Driver Development}
\label{sec:driver}

The \textsf{childdoc} mechanism can also be use for the development
of definition files such as \LaTeX{} styles or classes.
This case differs from the above setup with multiple parts
included by |\include| in that no |\includeonly| should be invoked.
This can be achieved by starting the include file
(before |\ProvidesPackage|) with:
%
\begin{center}
\begin{tabular}{l}
|\input{childdoc.def}|\\
|\childdocforward{|\textit{main}|}|\\
\end{tabular}
\end{center}
%
or alternatively with:
%
\begin{center}
\begin{tabular}{l}
|\input{childdoc.def}|\\
|\childdocby{|\textit{main}|}|\\
\end{tabular}
\end{center}
%
Both forms have slightly different effects as described above.
The main file is prepared as usual, see \secref{sec:include}.

%%%%%%%%%%%%%%%%%%%%%%%%%%%%%%%%%%%%%%%%%%%%%%%%%%%%%%%%%%%%%%%%%%%%%%%%%%%%%%%%
\subsection{Legacy Detection}
\label{sec:detection}

The directive |\childdocmain| in the main file can detect
whether the complete document or merely a child is to be compiled
even without using the directive |\childdocof|.
This method is deprecated because it is less robust
and there is no compelling reason to use it;
it is merely provided for backward compatibility
and it may be removed in future versions.

If the detection mechanism is to be used,
it is mandatory to correctly specify
the filename of the main file as the argument of |\childdocmain|:
%
\begin{center}
\begin{tabular}{l}
|\input{childdoc.def}|\\
|\childdocmain{|\textit{main}|}|\\
\end{tabular}
\end{center}
%
If |\jobname| does not match the argument \textit{main} of |\childdocmain|,
it is assumed that |\jobname| points to the child file to be compiled.
When using |\childdocmain| with the main file specified as argument,
it suffices to start a child file
with just |\input{|\textit{main}|}|
without loading of the package and using |\childdocof|.
If instead all processing is done
with the appropriate \textsf{childdoc} directives,
the argument of \textit{main} of |\childdocmain| can be empty.

An alternative version of the command line processing described
in \secref{sec:commandline} using the detection mechanism reads:
%
\begin{center}
|... -jobname "|\textit{target}|" "|[\textit{flags}]%
[|\def\jobname{|\textit{dest}|}|]|\input{|\textit{main}|}"|
\end{center}

%%%%%%%%%%%%%%%%%%%%%%%%%%%%%%%%%%%%%%%%%%%%%%%%%%%%%%%%%%%%%%%%%%%%%%%%%%%%%%%%
\subsection{Manual Code}
\label{sec:manual}

In case one cannot be certain whether the definitions file |childdoc.def|
is installed on the target \TeX{} distribution
and one prefers not to ship it,
it is conceivable to paste a few relevant commands into the sources.

To that end, drop all statements |\input{childdoc.def}|
and perform the replacements as outlined below.
Instead of |\childdocmain{|\textit{main}|}| add the following code
to the top of the main file:
%
\begin{center}
\begin{tabular}{l}
|\||ifdefined\childdocname\endinput\||fi\newif\ifchilddoc|\\
|\edef\childdocname{\scantokens\expandafter{\jobname\noexpand}}|\\
|\def\childdocmain{|\textit{main}|}\||ifx\childdocmain\childdocname\||else|\\
|\childdoctrue\includeonly{\childdocname}\let\jobname\childdocmain\||fi|\\
\end{tabular}
\end{center}
%
Instead of |\childdocof{|\textit{main}|}| just include the main file
at the top of each child file:
%
\begin{center}
|\input{|\textit{main}|}|
\end{center}
%
A simple redirection |\childdocforward{|\textit{dest}|}| is achieved by:
%
\begin{center}
|\def\jobname{|\textit{dest}|}\input{\jobname}|
\end{center}
%
The redirection with prefix
|\childdocforwardprefix[|\textit{prefix}|]{|\textit{dest}|}|
is accomplished by:
%
\begin{center}
\begin{tabular}{l}
|{\edef\jobname{\scantokens\expandafter{\jobname\noexpand}}|\\
|\def\redirectjob |\textit{prefix}|#1~~~{\gdef\jobname{|\textit{dest}|#1}}|\\
|\expandafter\redirectjob\jobname~~~}\input{\jobname}|
\end{tabular}
\end{center}

In an alternative approach,
child documents can be compiled by a specific command line
without additional code or specific definitions:
%
\begin{center}
|... -jobname "|\textit{target}|" "|[\textit{flags}]%
|\includeonly{|\textit{dest}|}\input{|\textit{main}|}"|
\end{center}
%

%%%%%%%%%%%%%%%%%%%%%%%%%%%%%%%%%%%%%%%%%%%%%%%%%%%%%%%%%%%%%%%%%%%%%%%%%%%%%%%%
%%%%%%%%%%%%%%%%%%%%%%%%%%%%%%%%%%%%%%%%%%%%%%%%%%%%%%%%%%%%%%%%%%%%%%%%%%%%%%%%
\section{Information}

%%%%%%%%%%%%%%%%%%%%%%%%%%%%%%%%%%%%%%%%%%%%%%%%%%%%%%%%%%%%%%%%%%%%%%%%%%%%%%%%
\subsection{Copyright}

Copyright \copyright{} 2017--2018 Niklas Beisert

This work may be distributed and/or modified under the
conditions of the \LaTeX{} Project Public License, either version 1.3
of this license or (at your option) any later version.
The latest version of this license is in
  \url{http://www.latex-project.org/lppl.txt}
and version 1.3 or later is part of all distributions of \LaTeX{}
version 2005/12/01 or later.

This work has the LPPL maintenance status `maintained'.

The Current Maintainer of this work is Niklas Beisert.

This work consists of the files |README.txt|, |childdoc.ins| and |childdoc.dtx|
as well as the derived files |childdoc.def|, |cdocsamp.tex|
with |cdocsch1.tex|, |cdocsch2.tex|, |cdocspt3.tex|, |cdocspt4.tex|,
|cdocsdrf.tex|, |cdocsfn1.tex|, |cdocsfn2.tex|
as well as |childdoc.pdf|.

%%%%%%%%%%%%%%%%%%%%%%%%%%%%%%%%%%%%%%%%%%%%%%%%%%%%%%%%%%%%%%%%%%%%%%%%%%%%%%%%
\subsection{Files and Installation}

The package consists of the files:
%
\begin{center}
\begin{tabular}{ll}
    |README.txt|   & readme file \\
    |childdoc.ins| & installation file \\
    |childdoc.dtx| & source file \\
    |childdoc.def| & definition file \\
    |cdocsamp.tex| & sample main file \\
    |cdocsch1.tex| & sample include file \\
    |cdocsch2.tex| & sample include file \\
    |cdocspt3.tex| & sample part file \\
    |cdocspt4.tex| & sample part file \\
    |cdocsdrf.tex| & sample redirection file \\
    |cdocsfn1.tex| & sample redirection file \\
    |cdocsfn2.tex| & sample redirection file \\
    |childdoc.pdf| & manual
\end{tabular}
\end{center}
%
The distribution consists of the files
|README.txt|, |childdoc.ins| and |childdoc.dtx|.
%
\begin{itemize}
\item
Run (pdf)\LaTeX{} on |childdoc.dtx|
to compile the manual |childdoc.pdf| (this file).
\item
Run \LaTeX{} on |childdoc.ins| to create the definitions file |childdoc.def|
and the sample |cdocsamp.tex| with include files
|cdocsch1.tex|, |cdocsch2.tex|, |cdocspt3.tex|, |cdocspt4.tex|,
|cdocsdrf.tex|, |cdocsfn1.tex|, |cdocsfn2.tex|.
Then copy the file |childdoc.def| to an appropriate directory of your \LaTeX{}
distribution, e.g.\ \textit{texmf-root}|/tex/latex/childdoc|.
\end{itemize}

%%%%%%%%%%%%%%%%%%%%%%%%%%%%%%%%%%%%%%%%%%%%%%%%%%%%%%%%%%%%%%%%%%%%%%%%%%%%%%%%
\subsection{Related CTAN Packages}

There are several other packages which offer a similar functionality:
%
\begin{itemize}
\item
The packages
\href{http://ctan.org/pkg/docmute}{\textsf{docmute}},
\href{http://ctan.org/pkg/includex}{\textsf{includex}} and
\href{http://ctan.org/pkg/standalone}{\textsf{standalone}}
provide commands to include only the document body of
a child file thus allowing both files to be compiled individually.
\item
The packages \href{http://ctan.org/pkg/subdocs}{\textsf{subdocs}}
and \href{http://ctan.org/pkg/subfiles}{\textsf{subfiles}}
provide structures in which the main and child documents can be
encapsulated and allowing them to be compiled individually.
The inclusion mechanism is different from the conventional |\include|.
\item
The package \href{http://ctan.org/pkg/combine}{\textsf{combine}}
is an elaborate solution to combine several documents into one.
\end{itemize}
%
See also the CTAN topic \href{http://ctan.org/topic/subdocs}{\textsf{subdocs}}
for further related packages.
The present package differs from the above solutions in that
a document structure constructed with the conventional |\include| mechanism
just needs two extra commands at the top of every file
such that all constituent files can be compiled individually.

%%%%%%%%%%%%%%%%%%%%%%%%%%%%%%%%%%%%%%%%%%%%%%%%%%%%%%%%%%%%%%%%%%%%%%%%%%%%%%%%
%\subsection{Feature Suggestions}
%
%The following is a list of features which may be useful for future
%versions of this package:
%%
%\begin{itemize}
%\item
%\ldots
%\end{itemize}

%%%%%%%%%%%%%%%%%%%%%%%%%%%%%%%%%%%%%%%%%%%%%%%%%%%%%%%%%%%%%%%%%%%%%%%%%%%%%%%%
\subsection{Revision History}

%%%%%%%%%%%%%%%%%%%%%%%%%%%%%%%%%%%%%%%%
\paragraph{v2.0:} 2018/12/30

\begin{itemize}
\item
immediate forward processing
\item
added |\childdocby| mechanism
\item
manual restructured
\end{itemize}

%%%%%%%%%%%%%%%%%%%%%%%%%%%%%%%%%%%%%%%%
\paragraph{v1.6:} 2018/01/17

\begin{itemize}
\item
application for development of include files
\item
corrections to manual
\end{itemize}

%%%%%%%%%%%%%%%%%%%%%%%%%%%%%%%%%%%%%%%%
\paragraph{v1.5:} 2017/05/21

\begin{itemize}
\item
more complete structuring introduced
\item
|\childdocof| introduced
\item
|\childdoc| renamed to |\childdocmain|
\item
|\childredirect| renamed to |\childdocforward| and |\childdocforwardprefix|
and functionality expanded
\end{itemize}

%%%%%%%%%%%%%%%%%%%%%%%%%%%%%%%%%%%%%%%%
\paragraph{v1.0:} 2017/04/27

\begin{itemize}
\item
manual and install package
\item
first version published on CTAN
\end{itemize}

%%%%%%%%%%%%%%%%%%%%%%%%%%%%%%%%%%%%%%%%
\paragraph{v0.6:} 2017/04/26

\begin{itemize}
\item
redirection mechanism added
\end{itemize}

%%%%%%%%%%%%%%%%%%%%%%%%%%%%%%%%%%%%%%%%
\paragraph{v0.5:} 2017/04/26

\begin{itemize}
\item
functionality in definition file
\end{itemize}


%%%%%%%%%%%%%%%%%%%%%%%%%%%%%%%%%%%%%%%%%%%%%%%%%%%%%%%%%%%%%%%%%%%%%%%%%%%%%%%%
%%%%%%%%%%%%%%%%%%%%%%%%%%%%%%%%%%%%%%%%%%%%%%%%%%%%%%%%%%%%%%%%%%%%%%%%%%%%%%%%
%%%%%%%%%%%%%%%%%%%%%%%%%%%%%%%%%%%%%%%%%%%%%%%%%%%%%%%%%%%%%%%%%%%%%%%%%%%%%%%%
\appendix

\settowidth\MacroIndent{\rmfamily\scriptsize 000\ }

 \DocInput{childdoc.dtx}

\end{document}
%</driver>
% \fi
%
% %%%%%%%%%%%%%%%%%%%%%%%%%%%%%%%%%%%%%%%%%%%%%%%%%%%%%%%%%%%%%%%%%%%%%%%%%%%%%%
% %%%%%%%%%%%%%%%%%%%%%%%%%%%%%%%%%%%%%%%%%%%%%%%%%%%%%%%%%%%%%%%%%%%%%%%%%%%%%%
% \section{Sample}
%\iffalse
%<*samplemain>
%\fi
%
% The following presents a sample document
% with two chapters, two parts, a title page,
% a compile flag as well as three forwarding files to set the flag.
% It consists of eight |.tex| files:
% \begin{center}
% \begin{tabular}{ll}
% |cdocsamp.tex|&main file\\
% |cdocsch1.tex|&include file for chapter 1\\
% |cdocsch2.tex|&include file for chapter 2\\
% |cdocspt3.tex|&include file for part 3\\
% |cdocspt4.tex|&include file for part 4\\
% |cdocsdrf.tex|&forwarding file for main file in draft mode\\
% |cdocsfi1.tex|&forwarding file for final version of chapter 1\\
% |cdocsfi2.tex|&forwarding file for final version of chapter 2\\
% \end{tabular}
% \end{center}
% Each of the eight files can be compiled directly by the \LaTeX{} compiler.
%
% %%%%%%%%%%%%%%%%%%%%%%%%%%%%%%%%%%%%%%
% \paragraph{Main File.}
%
% The main file is called |cdocsamp.tex|.
%
% Load the \textsf{childdoc} definitions and
% declare the filename for the main document:
%    \begin{macrocode}
\input{childdoc.def}
\childdocmain{}
%    \end{macrocode}

% Optional override for |\version| flag:
%    \begin{macrocode}
%%\ifchilddoc\else\providecommand{\version}{draft}\fi
%    \end{macrocode}

% Define the default values for the |\version| flag
% (|final| for the main file and |draft| for childs):
%    \begin{macrocode}
\ifchilddoc
\providecommand{\version}{draft}
\else
\providecommand{\version}{final}
\fi
%    \end{macrocode}

% Load the standard document class:
%    \begin{macrocode}
\documentclass[12pt]{article}
%    \end{macrocode}

% Start the document body:
%    \begin{macrocode}
\begin{document}
%    \end{macrocode}

% Declare a title page.
% Print title, part of document being processed and version flag:
%    \begin{macrocode}
\addtocounter{page}{-1}
\begin{center}
{\LARGE\bfseries{}childdoc example\par}
\vspace{1cm}
\ifchilddoc
\ifchilddocmanual part\else chapter\fi:
`\childdocname' of `\childdocjob'\par
\else
main document: `\childdocjob'\par
\fi
version: \version\par
\end{center}
\newpage
%    \end{macrocode}

% Manually include selected file,
% otherwise process as usual:
%    \begin{macrocode}
\ifchilddocmanual
\section*{part `\childdocname'}
\input{\childdocname}
\else
%    \end{macrocode}

% Include the two chapters:
%    \begin{macrocode}
\include{cdocsch1}
\include{cdocsch2}
%    \end{macrocode}

% Include the two parts unless only chapters should be displayed:
%    \begin{macrocode}
\ifchilddoc\else
\section{part three}
\input{cdocspt3}
\section{part four}
\input{cdocspt4}
\fi
%    \end{macrocode}

% Process as usual until here:
%    \begin{macrocode}
\fi
%    \end{macrocode}

% End of document body:
%    \begin{macrocode}
\end{document}
%    \end{macrocode}
%\iffalse
%</samplemain>
%\fi
%
% %%%%%%%%%%%%%%%%%%%%%%%%%%%%%%%%%%%%%%
% \paragraph{Chapter Include Files.}
%
% The include files are called |cdocsch1.tex| and |cdocsch2.tex|.
%
%\iffalse
%<*samplechap1|samplechap2>
%\fi

% Optional override for |\version| flag:
%    \begin{macrocode}
%%\providecommand{\version}{final}
%    \end{macrocode}

% Include the main document:
%    \begin{macrocode}
\input{childdoc.def}
\childdocof{cdocsamp}
%    \end{macrocode}

%\iffalse
%</samplechap1|samplechap2>
%\fi
%
%\iffalse
%<*samplechap1>
%\fi
% Some text for chapter 1:
%    \begin{macrocode}
\section{one}
some text in chapter one
%    \end{macrocode}

%\iffalse
%</samplechap1>
%\fi
% Some text for chapter 2:
%\iffalse
%<*samplechap2>
%\fi
%    \begin{macrocode}
\section{two}
more text in chapter two
%    \end{macrocode}

%\iffalse
%</samplechap2>
%\fi
%
% %%%%%%%%%%%%%%%%%%%%%%%%%%%%%%%%%%%%%%
% \paragraph{Part Include Files.}
%
% The include files are called |cdocspt3.tex| and |cdocspt4.tex|.
%
%\iffalse
%<*samplepart3|samplepart4>
%\fi

% Optional override for |\version| flag:
%    \begin{macrocode}
%%\providecommand{\version}{final}
%    \end{macrocode}

% Include the main document:
%    \begin{macrocode}
\input{childdoc.def}
\childdocby{cdocsamp}
%    \end{macrocode}

%\iffalse
%</samplepart3|samplepart4>
%\fi
%
%\iffalse
%<*samplepart3>
%\fi
% Some text for part 3:
%    \begin{macrocode}
some text in part three
%    \end{macrocode}

%\iffalse
%</samplepart3>
%\fi
% Some text for part 4:
%\iffalse
%<*samplepart4>
%\fi
%    \begin{macrocode}
more text in part four
%    \end{macrocode}

%\iffalse
%</samplepart4>
%\fi
%
% %%%%%%%%%%%%%%%%%%%%%%%%%%%%%%%%%%%%%%
% \paragraph{Forwarding for a Complete Draft.}
%
% The following forwarding file |cdocsdrf.tex|
% compiles the main document in draft mode:
%\iffalse
%<*sampledraft>
%\fi
%    \begin{macrocode}
\def\version{draft}
\input{childdoc.def}
\childdocforward{cdocsamp}
%    \end{macrocode}

%\iffalse
%</sampledraft>
%\fi
%
% %%%%%%%%%%%%%%%%%%%%%%%%%%%%%%%%%%%%%%
% \paragraph{Forwarding for Final Version of the Chapters.}
%
% The following forwarding files |cdocsfn1.tex| and |cdocsfn2.tex|
% (with identical content)
% compile the final versions of the child documents
% |cdocsch1.tex| and |cdocsch2.tex|, respectively:
%\iffalse
%<*samplefinal>
%\fi
%    \begin{macrocode}
\def\version{final}
\input{childdoc.def}
\childdocforwardprefix[cdocsamp]{cdocsfn}{cdocsch}
%    \end{macrocode}

%\iffalse
%</samplefinal>
%\fi
%
% %%%%%%%%%%%%%%%%%%%%%%%%%%%%%%%%%%%%%%
% \paragraph{Command Line Processing.}
%
% The following three command lines generate the output files
% |cdocscld|, |cdocscl1| and |cdocscl2|
% which should be identical to
% |cdocsdrf|, |cdocsch1| and |cdocsfn2|, respectively:
% \begin{center}
% \begin{tabular}{l}
% |latex -jobname cdocscld \|\\
% |  "\def\version{draft}\input{childdoc.def}\childdocforward{cdocsamp}"|\\
% |latex -jobname cdocscl1 \|\\
% |  "\input{childdoc.def}\childdocforward[cdocsamp]{cdocsch1}"|\\
% |latex -jobname cdocscl2 \|\\
% |  "\def\version{final}\input{childdoc.def}\childdocforward{cdocsch2}"|
% \end{tabular}
% \end{center}
% Note that the trailing backslash on each first line
% merely continues the input to the second line
% (for convenient cut ant paste).
% Furthermore, the command |latex| can be replaced by any
% of its alternative versions such as |pdflatex|.
%
% %%%%%%%%%%%%%%%%%%%%%%%%%%%%%%%%%%%%%%%%%%%%%%%%%%%%%%%%%%%%%%%%%%%%%%%%%%%%%%
% %%%%%%%%%%%%%%%%%%%%%%%%%%%%%%%%%%%%%%%%%%%%%%%%%%%%%%%%%%%%%%%%%%%%%%%%%%%%%%
% \section{Implementation}
%\iffalse
%<*package>
%\fi
%
% This section describes the definitions file |childdoc.def|.

% The definitions cannot be loaded using |\usepackage| or |\RequirePackage|
% which has a mechanism to prevent loading a style file more than once.
% When loading the definitions by means of |\input|
% multiple instances have to be prevented manually:
%\iffalse
%This code needs to be before the `\ProvidesFile' directive
%which is defined at the beginning of this file.
%Therefore it is also placed there and commented out here.
%</package>
%<*discard>
%\fi
%    \begin{macrocode}
\ifdefined\childdocmain\endinput\fi
%    \end{macrocode}
%\iffalse
%</discard>
%<*package>
%\fi
%
% \macro{\ifchilddoc}
% \macro{\ifchilddocmanual}
% The conditional |\ifchilddoc| tells whether a
% child (true) or main (false) document is being compiled.
% The conditional |\ifchilddocmanual| tells whether
% the |\includeonly| mechanism is used (false) or
% the selection of child files must be performed manually (true).
% The definitions initialise to false:
%    \begin{macrocode}
\newif\ifchilddoc
\newif\ifchilddocmanual
%    \end{macrocode}

% \macro{\childdocname}
% \macro{\childdocjob}
% The macro |\childdocname| stores the name of the main document
% to be compiled. The macro |\childdocjob| stores the name of
% the document on which the \LaTeX{} compiler was originally invoked.
% The content of |\jobname| cannot be compared
% to filenames specified in the source due to different catcodes.
% The following code rescans |\jobname|, stores the result
% in |\childdocname| and saves a copy in |\childdocjob|:
%    \begin{macrocode}
\edef\childdocname{\scantokens\expandafter{\jobname\noexpand}}
\let\childdocjob\childdocname
%    \end{macrocode}

% \macro{\childdocdisable}
% The macro |\childdocdisable| prevents the main file
% from being processed more than once.
% At this stage, the main document command |\childdocmain|
% is assumed to be called once again where it should do nothing.
% Any subsequent call to it should prevent
% a secondary processing of the main document
% It overwrites the forwarding commands
% |\childdocof| and |\childdocforward|
% with empty macros to prevent further inclusions of the main document:
%    \begin{macrocode}
\newcommand{\childdocdisable}
{
  \renewcommand{\childdocmain}[1]{\renewcommand{\childdocmain}[1]{\endinput}}
  \renewcommand{\childdocof}[1]{}
  \renewcommand{\childdocby}[2][]{}
  \renewcommand{\childdocforward}[2][]{}
  \renewcommand{\childdocdisable}{}
}
%    \end{macrocode}

% \macro{\childdocmain}
% The macro |\childdocmain| is to be called at the top of the main file
% with nothing or the main filename (without extension) as argument.
% First, it breaks loops.
% If the argument is not empty and does not match |\childdocname|
% (which is set by the first inclusion of |childdoc.def|),
% |\ifchilddoc| is set to true, |\includeonly| is applied to the child file
% and |\jobname| is set to the main file
% (for proper handling of |.aux| files):
%    \begin{macrocode}
\newcommand{\childdocmain}[1]
{
  \childdocdisable\childdocmain{}
  \if?#1?\else
    \begingroup
      \def\childdoctmp{#1}
      \ifx\childdoctmp\childdocname
        \def\childdoctmp{}
      \else
        \def\childdoctmp
        {
          \childdoctrue
          \includeonly{\childdocname}
          \def\childdocjob{#1}
          \def\jobname{#1}
        }
      \fi
      \expandafter
    \endgroup
    \childdoctmp
  \fi
}
%    \end{macrocode}

% \macro{\childdocof}
% The command |\childdocof| redirects
% compilation to the main file |#1|.
%    \begin{macrocode}
\newcommand{\childdocof}[1]
{
  \childdocdisable
  \childdoctrue
  \includeonly{\childdocname}
  \def\jobname{#1}
  \def\childdocjob{#1}
  \input{#1}
}
%    \end{macrocode}

% \macro{\childdocby}
% The command |\childdocby| ....
%    \begin{macrocode}
\newcommand{\childdocby}[2][]
{
  \childdocdisable
  \childdoctrue
  \childdocmanualtrue
  \if?#1?\else
    \def\jobname{#2}
  \fi
  \def\childdocjob{#2}
  \input{#2}
  \endinput
}
%    \end{macrocode}

% \macro{\childdocforward}
% The command |\childdocforward| redirects
% compilation to the main file or
% (if the optional argument is given) a child file.
% Parameters are set as if the main file
% or a child file starting with |\childdocof| was compiled.
% Then compilation is handed over to the main file:
%    \begin{macrocode}
\newcommand{\childdocforward}[2][]
{
  \begingroup
    \if?#1?
      \def\childdoctmp
      {
        \def\childdocname{#2}
        \def\childdocjob{#2}
        \def\jobname{#2}
        \input{#2}
        \endinput
      }
    \else
      \def\childdoctmp
      {
        \childdocdisable
        \def\childdocname{#2}
        \childdoctrue
        \includeonly{#2}
        \def\childdocjob{#1}
        \def\jobname{#1}
        \input{#1}
        \endinput
      }
    \fi
    \expandafter
  \endgroup
  \childdoctmp
}
%    \end{macrocode}

% \macro{\childdocforwardprefix}
% The command |\childdocforwardprefix| redirects
% compilation to the main or a child file by means of a pattern.
% The prefix |#1| in the current filename is replaced by |#2|
% and the suffix of the current filename is kept
% (it is assumed that the filename does not contain the substring `|~~~|'
% which is used as a delimiter).
% Compilation is handed over to the new file by |\childdocforward|:
%    \begin{macrocode}
\newcommand{\childdocforwardprefix}[3][]
{
  \begingroup
    \def\childdocextract #2##1~~~{\def\childdoctmp{\childdocforward[#1]{#3##1}}}
    \expandafter\childdocextract\childdocname~~~
    \expandafter
  \endgroup
  \childdoctmp
}
%    \end{macrocode}

% \macro{\childdoc}
% The deprecated macro |\childdoc| is a legacy version of |\childdocmain|:
%    \begin{macrocode}
\newcommand{\childdoc}{\childdocmain}
%    \end{macrocode}

% \macro{\childdocredirect}
% The deprecated macro |\childdocredirect| is a legacy version
% of |\childdocforward| and |\childdocforwardprefix|:
%    \begin{macrocode}
\newcommand{\childdocredirect}[2][]
{
  \begingroup
    \if?#1?
      \def\childdoctmp{\childdocforward{#2}}
    \else
      \def\childdoctmp{\childdocforwardprefix{#1}{#2}}
    \fi
    \expandafter
  \endgroup
  \childdoctmp
}
%    \end{macrocode}

%\iffalse
%</package>
%\fi
%
\endinput

\childdocof{cdocsamp}
%    \end{macrocode}

%\iffalse
%</samplechap1|samplechap2>
%\fi
%
%\iffalse
%<*samplechap1>
%\fi
% Some text for chapter 1:
%    \begin{macrocode}
\section{one}
some text in chapter one
%    \end{macrocode}

%\iffalse
%</samplechap1>
%\fi
% Some text for chapter 2:
%\iffalse
%<*samplechap2>
%\fi
%    \begin{macrocode}
\section{two}
more text in chapter two
%    \end{macrocode}

%\iffalse
%</samplechap2>
%\fi
%
% %%%%%%%%%%%%%%%%%%%%%%%%%%%%%%%%%%%%%%
% \paragraph{Part Include Files.}
%
% The include files are called |cdocspt3.tex| and |cdocspt4.tex|.
%
%\iffalse
%<*samplepart3|samplepart4>
%\fi

% Optional override for |\version| flag:
%    \begin{macrocode}
%%\providecommand{\version}{final}
%    \end{macrocode}

% Include the main document:
%    \begin{macrocode}
% \iffalse
%
% childdoc.dtx Copyright (C) 2017-2018 Niklas Beisert
%
% This work may be distributed and/or modified under the
% conditions of the LaTeX Project Public License, either version 1.3
% of this license or (at your option) any later version.
% The latest version of this license is in
%   http://www.latex-project.org/lppl.txt
% and version 1.3 or later is part of all distributions of LaTeX
% version 2005/12/01 or later.
%
% This work has the LPPL maintenance status `maintained'.
%
% The Current Maintainer of this work is Niklas Beisert.
%
% This work consists of the files childdoc.dtx and childdoc.ins
% and the derived files childdoc.def and cdocsamp.tex with
% cdocsch1.tex, cdocsch2.tex, cdocsdrf.tex, cdocsfn1.tex, cdocsfn2.tex.
%
%<package>\ifdefined\childdocmain\endinput\fi
%<package>\ProvidesFile{childdoc.def}[2018/12/30 v2.0 child document driver]
%<samplemain>\ProvidesFile{cdocsamp.tex}[2018/12/30 v2.0 sample for childdoc]
%<*driver>
%\ProvidesFile{childdoc.drv}[2018/12/30 v2.0 childdoc reference manual file]
\PassOptionsToClass{10pt,a4paper}{article}
\documentclass{ltxdoc}

\usepackage[margin=35mm]{geometry}
\usepackage{hyperref}
\usepackage{hyperxmp}
\usepackage[usenames]{color}

\hypersetup{colorlinks=true}
\hypersetup{pdfstartview=FitH}
\hypersetup{pdfpagemode=UseNone}
\hypersetup{pdfsource={}}
\hypersetup{pdflang={en-UK}}
\hypersetup{pdfcopyright={Copyright 2017-2018 Niklas Beisert.
  This work may be distributed and/or modified under the
  conditions of the LaTeX Project Public License, either version 1.3
  of this license or (at your option) any later version.}}
\hypersetup{pdflicenseurl={http://www.latex-project.org/lppl.txt}}
\hypersetup{pdfcontactaddress={ETH Zurich, ITP, HIT K,
  Wolfgang-Pauli-Strasse 27}}
\hypersetup{pdfcontactpostcode={8093}}
\hypersetup{pdfcontactcity={Zurich}}
\hypersetup{pdfcontactcountry={Switzerland}}
\hypersetup{pdfcontactemail={nbeisert@itp.phys.ethz.ch}}
\hypersetup{pdfcontacturl={http://people.phys.ethz.ch/\xmptilde nbeisert/}}

\newcommand{\secref}[1]{\hyperref[#1]{section \ref*{#1}}}

\parskip1ex
\parindent0pt
\let\olditemize\itemize
\def\itemize{\olditemize\parskip0pt}

\begin{document}

\title{The \textsf{childdoc} Package}
\hypersetup{pdftitle={The childdoc Package}}
\author{Niklas Beisert\\[2ex]
  Institut f\"ur Theoretische Physik\\
  Eidgen\"ossische Technische Hochschule Z\"urich\\
  Wolfgang-Pauli-Strasse 27, 8093 Z\"urich, Switzerland\\[1ex]
  \href{mailto:nbeisert@itp.phys.ethz.ch}
  {\texttt{nbeisert@itp.phys.ethz.ch}}}
\hypersetup{pdfauthor={Niklas Beisert}}
\hypersetup{pdfsubject={Manual for the LaTeX2e Package childdoc}}
\date{30 December 2018, \textsf{v2.0}}
\maketitle

\begin{abstract}\noindent
\textsf{childdoc} is a \LaTeXe{} package
that enables the direct compilation
of document sections included by |\include|
to individual files.
\end{abstract}

\begingroup
\parskip0ex
\tableofcontents
\endgroup

%%%%%%%%%%%%%%%%%%%%%%%%%%%%%%%%%%%%%%%%%%%%%%%%%%%%%%%%%%%%%%%%%%%%%%%%%%%%%%%%
%%%%%%%%%%%%%%%%%%%%%%%%%%%%%%%%%%%%%%%%%%%%%%%%%%%%%%%%%%%%%%%%%%%%%%%%%%%%%%%%
\section{Introduction}

\LaTeX{} provides a mechanism to structure a large document (such as a book)
into a main file and several child files (containing the chapters)
using the |\include| command.
This mechanism is beneficial for documents
which span hundreds of pages in order to
make the source file(s) more manageable.
Moreover, compilation can be restricted to
selected child files by means of the |\includeonly| command.
The latter feature can be used to reduce the compilation time while editing
(this was significantly more useful in the earlier days of \LaTeX{})
or to generate a smaller document which is easier to navigate.
Another application of |\includeonly| is to generate
documents consisting of selected parts of the complete document.

However, there are a few drawbacks of the plain |\include| mechanism:
\begin{itemize}
\item
The child files cannot be compiled on their own,
they can only be compiled via the main file.
A naive editing environment
(such as a text editor with an option
to have the current file processed by \LaTeX)
may require one to switch to the main file before compiling;
attempting to compile the child file produces errors.
\item
The main file must be modified (each time)
to adjust the |\includeonly| command
to the present needs. This easily leaves the main file in a messy state.
\item
The generated document will always carry the filename
of the main document. This is inconvenient if
several child files are to be compiled and
to be kept for distribution.
\end{itemize}

The present package provides a simple interface
to make child files individually compilable by \LaTeX{}.
Compiling a child file then has the same effect as compiling
the main file with an |\includeonly| command
to select the appropriate child.
Moreover the generated document will carry the name of the child
rather than the main file.
This resolves all three above issues.

This feature is meant to make the editing of books,
thesis documents and lecture notes somewhat more convenient.
However, the package can also be used efficiently for
composing a series of documents (such as exercise sheets)
which are typically distributed individually.
It then assists the author in generating the individual documents
(potentially in different versions)
as well as a document containing the collected series.
Another application is in developing style files
or other kinds of included material
where compilation of the style file could redirect
to a sample or test file.

%%%%%%%%%%%%%%%%%%%%%%%%%%%%%%%%%%%%%%%%%%%%%%%%%%%%%%%%%%%%%%%%%%%%%%%%%%%%%%%%
%%%%%%%%%%%%%%%%%%%%%%%%%%%%%%%%%%%%%%%%%%%%%%%%%%%%%%%%%%%%%%%%%%%%%%%%%%%%%%%%
\section{Usage}

First of all, the package \textsf{childdoc} is \emph{not} a standard
\LaTeXe{} |.sty| style file! Therefore it needs to be invoked in
a non-standard way.

%%%%%%%%%%%%%%%%%%%%%%%%%%%%%%%%%%%%%%%%%%%%%%%%%%%%%%%%%%%%%%%%%%%%%%%%%%%%%%%%
\subsection{Included Files}
\label{sec:include}

%%%%%%%%%%%%%%%%%%%%%%%%%%%%%%%%%%%%%%%%
\DescribeMacro{\childdocmain}
To use the package, add the commands
\begin{center}
\begin{tabular}{l}
|\input{childdoc.def}|\\
|\childdocmain{}|\\
\end{tabular}
\end{center}
at the very top of the main \LaTeX{} file,
in particular \emph{before} the |\documentclass| statement!
The argument of |\childdocmain| should be left empty
(but it must be present).

%%%%%%%%%%%%%%%%%%%%%%%%%%%%%%%%%%%%%%%%
\DescribeMacro{\childdocof}
Furthermore, add the commands
\begin{center}
\begin{tabular}{l}
|\input{childdoc.def}|\\
|\childdocof{|\textit{main}|}|\\
\end{tabular}
\end{center}
at the top of every child file \textit{child}
which is included by |\include{|\textit{child}|}|
from within the main file
(or at least for those files to be compiled individually).
The argument \textit{main} must be the filename of the main file.

There are a couple of
considerations in setting up the main and child documents:

%%%%%%%%%%%%%%%%%%%%%%%%%%%%%%%%%%%%%%%%
\paragraph{Restrictions.}

Please note the following restrictions:
\begin{itemize}
\item
|\childdocmain| must be called with one argument \textit{main}
to ensure compatibility with earlier version of the package.
It must either be empty (|\childdocmain{}|)
or precisely match the filename of the main file in which it is specified.
See \secref{sec:detection} for further information.
\item
The filename \textit{main} must be specified without the |.tex| extension.
\item
The filename \textit{main} is case sensitive
(even in case-insensitive file systems)
due to internal string comparison.
\item
The argument \textit{main} should be fully expanded, it cannot be a macro.
\item
Subdirectories and special characters should be avoided in filenames.
\item
The command |\childdocmain{|\textit{main}|}| must be followed by a whitespace.
It should not be followed immediately by another command
or by a comment mark `|%|'.
This is because the \TeX{} parser reads the token immediately following
the argument of |\childdocmain| and puts it
at the beginning of every child section;
however, a white\-space is ignored.
\end{itemize}

%%%%%%%%%%%%%%%%%%%%%%%%%%%%%%%%%%%%%%%%
\paragraph{Content of Main File.}

It is advisable to place all content in the child files included by |\include|.
Any output contained in the main file will appear in all child documents
unless suppressed manually;
it cannot be suppressed automatically by the |\includeonly| directive
and thus should normally be avoided.
A method to include some content in the main file
by means of conditional processing is described in \secref{sec:conditional}.

%%%%%%%%%%%%%%%%%%%%%%%%%%%%%%%%%%%%%%%%
\paragraph{Page Numbering.}

When only a part of the document is compiled,
the appropriate numbering of pages
(as well as other status parameters)
is determined from the |.aux| files.
The latter contain information from previous passes.
However this information needs to propagate through
all intermediate child documents.
Therefore the page numbering in child documents may well
be inconsistent until the complete document is compiled at least once.

A useful (if unconventional) way to always ensure a consistent
page numbering is to restart the numbering in each child document
and denote the pages by `\textit{child}|.|\textit{page}'
where \textit{child} represents the chapter/section number of the child file.
This can be achieved by the command
|\numberwithin{page}{|\textit{child}|}|
of the \textsf{amsmath} package
where \textit{child} can be |chapter| or |section|
depending on the chosen structuring.
Alternatively, one can modify the macro |\thepage| appropriately
and reset the counter |page| at the start of each child file.

%%%%%%%%%%%%%%%%%%%%%%%%%%%%%%%%%%%%%%%%%%%%%%%%%%%%%%%%%%%%%%%%%%%%%%%%%%%%%%%%
\subsection{Conditional Processing}
\label{sec:conditional}

The package provides a mechanism to compile different versions
of a document. To customise the versions further some conditional processing
can come in handy to distinguish which version is being compiled.
The package provides two macros to describe the compilation context:

%%%%%%%%%%%%%%%%%%%%%%%%%%%%%%%%%%%%%%%%
\DescribeMacro{\ifchilddoc}
The conditional |\ifchilddoc| distinguishes between the compilation of
child documents and the main document:
%
\begin{center}
|\ifchilddoc |\textit{child-code}| |[|\||else |\textit{main-code}]| \||fi|
\end{center}

%%%%%%%%%%%%%%%%%%%%%%%%%%%%%%%%%%%%%%%%
\DescribeMacro{\childdocname}
\DescribeMacro{\childdocjob}
The macro |\childdocname| contains the filename (without extension)
of the main or child file being processed.
Note that |\childdocjob| will always contain the name of the main file.

%%%%%%%%%%%%%%%%%%%%%%%%%%%%%%%%%%%%%%%%
\paragraph{Title Page.}

Conditional processing can be used to include a title or banner page
in the main document when proper precautions are taken.
Importantly, the code in the main file should ensure that the page counter
(as well as other status parameters which are stored in the |.aux| files)
takes the same value after the conditional processing.
Otherwise the page numbers may take divergent values
depending on which part is compiled.

For example, a title page could be declared by:
%
\begin{center}
\begin{tabular}{l}
|\ifchilddoc\||else|\\
|\addtocounter{page}{-1}|\\
\textit{code for title page}\\
|\newpage|\\
|\||fi|
\end{tabular}
\end{center}
%
A banner page for the child documents can be generated by:
%
\begin{center}
\begin{tabular}{l}
|\ifchilddoc|\\
|\addtocounter{page}{-1}|\\
\textit{code for banner page}\\
|\newpage|\\
|\||fi|
\end{tabular}
\end{center}
%
Here one could write a message such as:
\begin{center}
|This is the part \childdocname{} of \childdocjob{}.|
\end{center}

%%%%%%%%%%%%%%%%%%%%%%%%%%%%%%%%%%%%%%%%%%%%%%%%%%%%%%%%%%%%%%%%%%%%%%%%%%%%%%%%
\subsection{Flags}
\label{sec:flags}

The package makes it easy to generate different versions
of the main or child documents.
To this end compilation flags can be defined
and assigned different default values.
They will be particularly useful in conjunction
with the forwarding mechanism described in \secref{sec:forward}.

For example, it may be useful to have a flag |\version|
which can be set to |draft| or |final|.
The document source will contain some conditional code
depending on the value of |\version|.
Suppose further, the flag should default to |final| for the main file
and to |draft| for child files
which is a natural assignment for editing the document.
This is achieved by placing the following code
in the preamble of the main document
(below the |\childdocmain| directive):
%
\begin{center}
\begin{tabular}{l}
|\ifchilddoc|\\
|\providecommand{\version}{draft}|\\
|\||else|\\
|\providecommand{\version}{final}|\\
|\||fi|
\end{tabular}
\end{center}
%
The definition by |\providecommand| makes sure
that previous definitions are not overwritten.
Further statements |\providecommand{\version}{...}|
can thus be added before the above code to override it.

For the main file, one might add a line
(between |\childdocmain| and the above block)
%
\begin{center}
|%\ifchilddoc\||else\providecommand{\version}{draft}\||fi|
\end{center}
%
which can be uncommented to produce a draft version.
Likewise one can add a line to the very top of a child file
(above the |\childdocof{|\textit{main}|}| directive)
%
\begin{center}
|%\providecommand{\version}{final}|
\end{center}
%
which can be uncommented to produce the final version of this child document.

%%%%%%%%%%%%%%%%%%%%%%%%%%%%%%%%%%%%%%%%%%%%%%%%%%%%%%%%%%%%%%%%%%%%%%%%%%%%%%%%
\subsection{Forwarding}
\label{sec:forward}

Different versions of the main or child documents
using compilation flags as described in \secref{sec:flags}
can be (permanently) stored in different files
for convenient compilation, viewing and distribution.
To this end, the package defines a command
to pass on compilation to a different file:

%%%%%%%%%%%%%%%%%%%%%%%%%%%%%%%%%%%%%%%%
\DescribeMacro{\childdocforward}
The command |\childdocforward| redirects processing to
another source file:
%
\begin{center}
\begin{tabular}{l}
|\input{childdoc.def}|\\
|\childdocforward[|\textit{main}|]{|\textit{dest}|}|\\
\end{tabular}
\end{center}
%
The argument \textit{dest} is the destination file
(without extension).
It should be the main file or one of the child files.
Note that further \textsf{childdoc} directives
such as |\childdocof| and |\childdocforward|
in the indicated file will be processed in this form.
The optional argument \textit{main}
passes on directly to the main file \textit{main}
while pretending to compile the child \textit{dest}.
This form behaves as if \textit{dest}
issues |\childdocof{|\textit{main}|}| right away,
and no further \textsf{childdoc} directives will be processed.

%%%%%%%%%%%%%%%%%%%%%%%%%%%%%%%%%%%%%%%%
\DescribeMacro{\...prefix}
In the alternative form |\childdocforwardprefix|,
%
\begin{center}
\begin{tabular}{l}
|\input{childdoc.def}|\\
|\childdocforwardprefix[|\textit{main}|]{|\textit{prefix}|}{|\textit{dest}|}|
\end{tabular}
\end{center}
%
the destination file is determined by a pattern
depending on the current file:
To make this work, the current file must be called
`{\textit{prefix}\hspace{0.2em}\textit{suffix}}'
with \textit{prefix} matching precisely the argument.
Processing is then passed on to the file
`{\textit{dest}\hspace{0.2em}\textit{suffix}}'.
Surely, the same effect is achieved by
directly specifying the
argument `{\textit{dest}\hspace{0.2em}\textit{suffix}}'
in the first form.
However, that requires to set up a different file
for each child. With the alternative form of the command
all these files can have exactly the same content
which simplifies setting them up and maintaining them.

For example, the following file |draft.tex|
with a compilation flag |\version| as described in \secref{sec:flags}
compiles the main document as a draft:
%
\begin{center}
\begin{tabular}{l}
|\def\version{draft}|\\
|\input{childdoc.def}|\\
|\childdocforward{|\textit{main}|}|
\end{tabular}
\end{center}
%
Likewise, the following files |final|\textit{nn}|.tex|
compile the final version of the child document
|child|\textit{nn}|.tex|:
%
\begin{center}
\begin{tabular}{l}
|\def\version{final}|\\
|\input{childdoc.def}|\\
|\childdocforwardprefix{final}{child}|
\end{tabular}
\end{center}
%

Note that when several versions of a main file and/or of each child file
are to be generated, it may be convenient to set up a |Makefile| or
shell script to automatise the process.

%%%%%%%%%%%%%%%%%%%%%%%%%%%%%%%%%%%%%%%%%%%%%%%%%%%%%%%%%%%%%%%%%%%%%%%%%%%%%%%%
\subsection{Command Line Processing}
\label{sec:commandline}

The effect of redirection files can also be achieved by invoking
the \LaTeX{} compiler with a more elaborate command line.
Most conveniently this should be done as part
of a shell script or a |Makefile|.

When using \textsf{childdoc} in the main file, the following
command lines effectively perform a redirection
(note that depending on the shell being used,
backslashes may have to be doubled: `|\|' $\to$ `|\\|'):
%
\begin{center}
|... -jobname "|\textit{target}|" |\\|"|[\textit{flags}]%
|\input{childdoc.def}\childdocforward[|\textit{main}|]{|\textit{dest}|}"|
\end{center}
%
Here \textit{target} is the name of the output file,
\textit{main} is the name of the main file
and \textit{dest} is the name of the main or child file to be processed
(all filenames without extensions).
The optional argument \textit{main} can be omitted
if \textit{main} matches \textit{dest}.
Optionally, compilation \textit{flags} can be defined via |\def| commands.
This command line makes the \TeX{} engine believe
it is compiling the file \textit{target}
whose content is specified as the latter parameter.
The provided code then forwards the processing to
\textit{main} or \textit{dest} as described in \secref{sec:forward}.

%%%%%%%%%%%%%%%%%%%%%%%%%%%%%%%%%%%%%%%%%%%%%%%%%%%%%%%%%%%%%%%%%%%%%%%%%%%%%%%%
\subsection{Include by Input}
\label{sec:input}

Including child documents by |\include| has some restrictions by design.
Most notably, the content of a child document always occupies
its own set of pages; pages cannot be shared between child documents.
Usually, this behaviour makes perfect sense
because each child document contain an essential part of the document.
However, in some situations it may be desirable to compose
a document from a collection of parts
without having mandatory page breaks between then.
For this case, the package
provides a mechanism to include parts
by |\input| which can also be processed individually.
However, by construction this mechanism
requires manual handling of the content to be output.

%%%%%%%%%%%%%%%%%%%%%%%%%%%%%%%%%%%%%%%%
\DescribeMacro{\ifchilddocmanual}
The main file should be prepared as usual, see \secref{sec:include}.
However, the document body must make a distinction
between processing of an individual part and of the main document, e.g.:
%
\begin{center}
\begin{tabular}{l}
|\ifchilddocmanual|\\
|\input{\childdocname}|\\
|\||else|\\
\textit{document body with }|\input{|\textit{part}|}|\\
|\||fi|
\end{tabular}
\end{center}
%
The conditional |\ifchilddocmanual| is true whenever
a part to be included by |\input| is being compiled,
and the name of the part is stored in |\childdocname|.

%%%%%%%%%%%%%%%%%%%%%%%%%%%%%%%%%%%%%%%%
\DescribeMacro{\childdocby}
Each part to be included by |\input| should start with:
%
\begin{center}
\begin{tabular}{l}
|\input{childdoc.def}|\\
|\childdocby{|\textit{main}|}|\\
\end{tabular}
\end{center}
%
The directive |\childdocby| is similar to |\childdocof|
described in \secref{sec:include},
but the subsequent selection of content must be done manually.
To that end, both |\ifchilddoc| and |\ifchilddocmanual|
will be true upon processing of a part,
and the name of the part is stored in |\childdocname|.
Note that |\jobname| will be set to the filename of the current part
so that each part receives an individual |.aux| file
that does not interfere with the |.aux| file(s) of the main document.
This behaviour can be altered by the alternative form
|\childdocby[*]{|\textit{main}|}| (with a non-empty optional argument)
which uses the |.aux| file of the main document
by setting |\jobname| to \textit{main}.

%%%%%%%%%%%%%%%%%%%%%%%%%%%%%%%%%%%%%%%%%%%%%%%%%%%%%%%%%%%%%%%%%%%%%%%%%%%%%%%%
\subsection{Driver Development}
\label{sec:driver}

The \textsf{childdoc} mechanism can also be use for the development
of definition files such as \LaTeX{} styles or classes.
This case differs from the above setup with multiple parts
included by |\include| in that no |\includeonly| should be invoked.
This can be achieved by starting the include file
(before |\ProvidesPackage|) with:
%
\begin{center}
\begin{tabular}{l}
|\input{childdoc.def}|\\
|\childdocforward{|\textit{main}|}|\\
\end{tabular}
\end{center}
%
or alternatively with:
%
\begin{center}
\begin{tabular}{l}
|\input{childdoc.def}|\\
|\childdocby{|\textit{main}|}|\\
\end{tabular}
\end{center}
%
Both forms have slightly different effects as described above.
The main file is prepared as usual, see \secref{sec:include}.

%%%%%%%%%%%%%%%%%%%%%%%%%%%%%%%%%%%%%%%%%%%%%%%%%%%%%%%%%%%%%%%%%%%%%%%%%%%%%%%%
\subsection{Legacy Detection}
\label{sec:detection}

The directive |\childdocmain| in the main file can detect
whether the complete document or merely a child is to be compiled
even without using the directive |\childdocof|.
This method is deprecated because it is less robust
and there is no compelling reason to use it;
it is merely provided for backward compatibility
and it may be removed in future versions.

If the detection mechanism is to be used,
it is mandatory to correctly specify
the filename of the main file as the argument of |\childdocmain|:
%
\begin{center}
\begin{tabular}{l}
|\input{childdoc.def}|\\
|\childdocmain{|\textit{main}|}|\\
\end{tabular}
\end{center}
%
If |\jobname| does not match the argument \textit{main} of |\childdocmain|,
it is assumed that |\jobname| points to the child file to be compiled.
When using |\childdocmain| with the main file specified as argument,
it suffices to start a child file
with just |\input{|\textit{main}|}|
without loading of the package and using |\childdocof|.
If instead all processing is done
with the appropriate \textsf{childdoc} directives,
the argument of \textit{main} of |\childdocmain| can be empty.

An alternative version of the command line processing described
in \secref{sec:commandline} using the detection mechanism reads:
%
\begin{center}
|... -jobname "|\textit{target}|" "|[\textit{flags}]%
[|\def\jobname{|\textit{dest}|}|]|\input{|\textit{main}|}"|
\end{center}

%%%%%%%%%%%%%%%%%%%%%%%%%%%%%%%%%%%%%%%%%%%%%%%%%%%%%%%%%%%%%%%%%%%%%%%%%%%%%%%%
\subsection{Manual Code}
\label{sec:manual}

In case one cannot be certain whether the definitions file |childdoc.def|
is installed on the target \TeX{} distribution
and one prefers not to ship it,
it is conceivable to paste a few relevant commands into the sources.

To that end, drop all statements |\input{childdoc.def}|
and perform the replacements as outlined below.
Instead of |\childdocmain{|\textit{main}|}| add the following code
to the top of the main file:
%
\begin{center}
\begin{tabular}{l}
|\||ifdefined\childdocname\endinput\||fi\newif\ifchilddoc|\\
|\edef\childdocname{\scantokens\expandafter{\jobname\noexpand}}|\\
|\def\childdocmain{|\textit{main}|}\||ifx\childdocmain\childdocname\||else|\\
|\childdoctrue\includeonly{\childdocname}\let\jobname\childdocmain\||fi|\\
\end{tabular}
\end{center}
%
Instead of |\childdocof{|\textit{main}|}| just include the main file
at the top of each child file:
%
\begin{center}
|\input{|\textit{main}|}|
\end{center}
%
A simple redirection |\childdocforward{|\textit{dest}|}| is achieved by:
%
\begin{center}
|\def\jobname{|\textit{dest}|}\input{\jobname}|
\end{center}
%
The redirection with prefix
|\childdocforwardprefix[|\textit{prefix}|]{|\textit{dest}|}|
is accomplished by:
%
\begin{center}
\begin{tabular}{l}
|{\edef\jobname{\scantokens\expandafter{\jobname\noexpand}}|\\
|\def\redirectjob |\textit{prefix}|#1~~~{\gdef\jobname{|\textit{dest}|#1}}|\\
|\expandafter\redirectjob\jobname~~~}\input{\jobname}|
\end{tabular}
\end{center}

In an alternative approach,
child documents can be compiled by a specific command line
without additional code or specific definitions:
%
\begin{center}
|... -jobname "|\textit{target}|" "|[\textit{flags}]%
|\includeonly{|\textit{dest}|}\input{|\textit{main}|}"|
\end{center}
%

%%%%%%%%%%%%%%%%%%%%%%%%%%%%%%%%%%%%%%%%%%%%%%%%%%%%%%%%%%%%%%%%%%%%%%%%%%%%%%%%
%%%%%%%%%%%%%%%%%%%%%%%%%%%%%%%%%%%%%%%%%%%%%%%%%%%%%%%%%%%%%%%%%%%%%%%%%%%%%%%%
\section{Information}

%%%%%%%%%%%%%%%%%%%%%%%%%%%%%%%%%%%%%%%%%%%%%%%%%%%%%%%%%%%%%%%%%%%%%%%%%%%%%%%%
\subsection{Copyright}

Copyright \copyright{} 2017--2018 Niklas Beisert

This work may be distributed and/or modified under the
conditions of the \LaTeX{} Project Public License, either version 1.3
of this license or (at your option) any later version.
The latest version of this license is in
  \url{http://www.latex-project.org/lppl.txt}
and version 1.3 or later is part of all distributions of \LaTeX{}
version 2005/12/01 or later.

This work has the LPPL maintenance status `maintained'.

The Current Maintainer of this work is Niklas Beisert.

This work consists of the files |README.txt|, |childdoc.ins| and |childdoc.dtx|
as well as the derived files |childdoc.def|, |cdocsamp.tex|
with |cdocsch1.tex|, |cdocsch2.tex|, |cdocspt3.tex|, |cdocspt4.tex|,
|cdocsdrf.tex|, |cdocsfn1.tex|, |cdocsfn2.tex|
as well as |childdoc.pdf|.

%%%%%%%%%%%%%%%%%%%%%%%%%%%%%%%%%%%%%%%%%%%%%%%%%%%%%%%%%%%%%%%%%%%%%%%%%%%%%%%%
\subsection{Files and Installation}

The package consists of the files:
%
\begin{center}
\begin{tabular}{ll}
    |README.txt|   & readme file \\
    |childdoc.ins| & installation file \\
    |childdoc.dtx| & source file \\
    |childdoc.def| & definition file \\
    |cdocsamp.tex| & sample main file \\
    |cdocsch1.tex| & sample include file \\
    |cdocsch2.tex| & sample include file \\
    |cdocspt3.tex| & sample part file \\
    |cdocspt4.tex| & sample part file \\
    |cdocsdrf.tex| & sample redirection file \\
    |cdocsfn1.tex| & sample redirection file \\
    |cdocsfn2.tex| & sample redirection file \\
    |childdoc.pdf| & manual
\end{tabular}
\end{center}
%
The distribution consists of the files
|README.txt|, |childdoc.ins| and |childdoc.dtx|.
%
\begin{itemize}
\item
Run (pdf)\LaTeX{} on |childdoc.dtx|
to compile the manual |childdoc.pdf| (this file).
\item
Run \LaTeX{} on |childdoc.ins| to create the definitions file |childdoc.def|
and the sample |cdocsamp.tex| with include files
|cdocsch1.tex|, |cdocsch2.tex|, |cdocspt3.tex|, |cdocspt4.tex|,
|cdocsdrf.tex|, |cdocsfn1.tex|, |cdocsfn2.tex|.
Then copy the file |childdoc.def| to an appropriate directory of your \LaTeX{}
distribution, e.g.\ \textit{texmf-root}|/tex/latex/childdoc|.
\end{itemize}

%%%%%%%%%%%%%%%%%%%%%%%%%%%%%%%%%%%%%%%%%%%%%%%%%%%%%%%%%%%%%%%%%%%%%%%%%%%%%%%%
\subsection{Related CTAN Packages}

There are several other packages which offer a similar functionality:
%
\begin{itemize}
\item
The packages
\href{http://ctan.org/pkg/docmute}{\textsf{docmute}},
\href{http://ctan.org/pkg/includex}{\textsf{includex}} and
\href{http://ctan.org/pkg/standalone}{\textsf{standalone}}
provide commands to include only the document body of
a child file thus allowing both files to be compiled individually.
\item
The packages \href{http://ctan.org/pkg/subdocs}{\textsf{subdocs}}
and \href{http://ctan.org/pkg/subfiles}{\textsf{subfiles}}
provide structures in which the main and child documents can be
encapsulated and allowing them to be compiled individually.
The inclusion mechanism is different from the conventional |\include|.
\item
The package \href{http://ctan.org/pkg/combine}{\textsf{combine}}
is an elaborate solution to combine several documents into one.
\end{itemize}
%
See also the CTAN topic \href{http://ctan.org/topic/subdocs}{\textsf{subdocs}}
for further related packages.
The present package differs from the above solutions in that
a document structure constructed with the conventional |\include| mechanism
just needs two extra commands at the top of every file
such that all constituent files can be compiled individually.

%%%%%%%%%%%%%%%%%%%%%%%%%%%%%%%%%%%%%%%%%%%%%%%%%%%%%%%%%%%%%%%%%%%%%%%%%%%%%%%%
%\subsection{Feature Suggestions}
%
%The following is a list of features which may be useful for future
%versions of this package:
%%
%\begin{itemize}
%\item
%\ldots
%\end{itemize}

%%%%%%%%%%%%%%%%%%%%%%%%%%%%%%%%%%%%%%%%%%%%%%%%%%%%%%%%%%%%%%%%%%%%%%%%%%%%%%%%
\subsection{Revision History}

%%%%%%%%%%%%%%%%%%%%%%%%%%%%%%%%%%%%%%%%
\paragraph{v2.0:} 2018/12/30

\begin{itemize}
\item
immediate forward processing
\item
added |\childdocby| mechanism
\item
manual restructured
\end{itemize}

%%%%%%%%%%%%%%%%%%%%%%%%%%%%%%%%%%%%%%%%
\paragraph{v1.6:} 2018/01/17

\begin{itemize}
\item
application for development of include files
\item
corrections to manual
\end{itemize}

%%%%%%%%%%%%%%%%%%%%%%%%%%%%%%%%%%%%%%%%
\paragraph{v1.5:} 2017/05/21

\begin{itemize}
\item
more complete structuring introduced
\item
|\childdocof| introduced
\item
|\childdoc| renamed to |\childdocmain|
\item
|\childredirect| renamed to |\childdocforward| and |\childdocforwardprefix|
and functionality expanded
\end{itemize}

%%%%%%%%%%%%%%%%%%%%%%%%%%%%%%%%%%%%%%%%
\paragraph{v1.0:} 2017/04/27

\begin{itemize}
\item
manual and install package
\item
first version published on CTAN
\end{itemize}

%%%%%%%%%%%%%%%%%%%%%%%%%%%%%%%%%%%%%%%%
\paragraph{v0.6:} 2017/04/26

\begin{itemize}
\item
redirection mechanism added
\end{itemize}

%%%%%%%%%%%%%%%%%%%%%%%%%%%%%%%%%%%%%%%%
\paragraph{v0.5:} 2017/04/26

\begin{itemize}
\item
functionality in definition file
\end{itemize}


%%%%%%%%%%%%%%%%%%%%%%%%%%%%%%%%%%%%%%%%%%%%%%%%%%%%%%%%%%%%%%%%%%%%%%%%%%%%%%%%
%%%%%%%%%%%%%%%%%%%%%%%%%%%%%%%%%%%%%%%%%%%%%%%%%%%%%%%%%%%%%%%%%%%%%%%%%%%%%%%%
%%%%%%%%%%%%%%%%%%%%%%%%%%%%%%%%%%%%%%%%%%%%%%%%%%%%%%%%%%%%%%%%%%%%%%%%%%%%%%%%
\appendix

\settowidth\MacroIndent{\rmfamily\scriptsize 000\ }

 \DocInput{childdoc.dtx}

\end{document}
%</driver>
% \fi
%
% %%%%%%%%%%%%%%%%%%%%%%%%%%%%%%%%%%%%%%%%%%%%%%%%%%%%%%%%%%%%%%%%%%%%%%%%%%%%%%
% %%%%%%%%%%%%%%%%%%%%%%%%%%%%%%%%%%%%%%%%%%%%%%%%%%%%%%%%%%%%%%%%%%%%%%%%%%%%%%
% \section{Sample}
%\iffalse
%<*samplemain>
%\fi
%
% The following presents a sample document
% with two chapters, two parts, a title page,
% a compile flag as well as three forwarding files to set the flag.
% It consists of eight |.tex| files:
% \begin{center}
% \begin{tabular}{ll}
% |cdocsamp.tex|&main file\\
% |cdocsch1.tex|&include file for chapter 1\\
% |cdocsch2.tex|&include file for chapter 2\\
% |cdocspt3.tex|&include file for part 3\\
% |cdocspt4.tex|&include file for part 4\\
% |cdocsdrf.tex|&forwarding file for main file in draft mode\\
% |cdocsfi1.tex|&forwarding file for final version of chapter 1\\
% |cdocsfi2.tex|&forwarding file for final version of chapter 2\\
% \end{tabular}
% \end{center}
% Each of the eight files can be compiled directly by the \LaTeX{} compiler.
%
% %%%%%%%%%%%%%%%%%%%%%%%%%%%%%%%%%%%%%%
% \paragraph{Main File.}
%
% The main file is called |cdocsamp.tex|.
%
% Load the \textsf{childdoc} definitions and
% declare the filename for the main document:
%    \begin{macrocode}
\input{childdoc.def}
\childdocmain{}
%    \end{macrocode}

% Optional override for |\version| flag:
%    \begin{macrocode}
%%\ifchilddoc\else\providecommand{\version}{draft}\fi
%    \end{macrocode}

% Define the default values for the |\version| flag
% (|final| for the main file and |draft| for childs):
%    \begin{macrocode}
\ifchilddoc
\providecommand{\version}{draft}
\else
\providecommand{\version}{final}
\fi
%    \end{macrocode}

% Load the standard document class:
%    \begin{macrocode}
\documentclass[12pt]{article}
%    \end{macrocode}

% Start the document body:
%    \begin{macrocode}
\begin{document}
%    \end{macrocode}

% Declare a title page.
% Print title, part of document being processed and version flag:
%    \begin{macrocode}
\addtocounter{page}{-1}
\begin{center}
{\LARGE\bfseries{}childdoc example\par}
\vspace{1cm}
\ifchilddoc
\ifchilddocmanual part\else chapter\fi:
`\childdocname' of `\childdocjob'\par
\else
main document: `\childdocjob'\par
\fi
version: \version\par
\end{center}
\newpage
%    \end{macrocode}

% Manually include selected file,
% otherwise process as usual:
%    \begin{macrocode}
\ifchilddocmanual
\section*{part `\childdocname'}
\input{\childdocname}
\else
%    \end{macrocode}

% Include the two chapters:
%    \begin{macrocode}
\include{cdocsch1}
\include{cdocsch2}
%    \end{macrocode}

% Include the two parts unless only chapters should be displayed:
%    \begin{macrocode}
\ifchilddoc\else
\section{part three}
\input{cdocspt3}
\section{part four}
\input{cdocspt4}
\fi
%    \end{macrocode}

% Process as usual until here:
%    \begin{macrocode}
\fi
%    \end{macrocode}

% End of document body:
%    \begin{macrocode}
\end{document}
%    \end{macrocode}
%\iffalse
%</samplemain>
%\fi
%
% %%%%%%%%%%%%%%%%%%%%%%%%%%%%%%%%%%%%%%
% \paragraph{Chapter Include Files.}
%
% The include files are called |cdocsch1.tex| and |cdocsch2.tex|.
%
%\iffalse
%<*samplechap1|samplechap2>
%\fi

% Optional override for |\version| flag:
%    \begin{macrocode}
%%\providecommand{\version}{final}
%    \end{macrocode}

% Include the main document:
%    \begin{macrocode}
\input{childdoc.def}
\childdocof{cdocsamp}
%    \end{macrocode}

%\iffalse
%</samplechap1|samplechap2>
%\fi
%
%\iffalse
%<*samplechap1>
%\fi
% Some text for chapter 1:
%    \begin{macrocode}
\section{one}
some text in chapter one
%    \end{macrocode}

%\iffalse
%</samplechap1>
%\fi
% Some text for chapter 2:
%\iffalse
%<*samplechap2>
%\fi
%    \begin{macrocode}
\section{two}
more text in chapter two
%    \end{macrocode}

%\iffalse
%</samplechap2>
%\fi
%
% %%%%%%%%%%%%%%%%%%%%%%%%%%%%%%%%%%%%%%
% \paragraph{Part Include Files.}
%
% The include files are called |cdocspt3.tex| and |cdocspt4.tex|.
%
%\iffalse
%<*samplepart3|samplepart4>
%\fi

% Optional override for |\version| flag:
%    \begin{macrocode}
%%\providecommand{\version}{final}
%    \end{macrocode}

% Include the main document:
%    \begin{macrocode}
\input{childdoc.def}
\childdocby{cdocsamp}
%    \end{macrocode}

%\iffalse
%</samplepart3|samplepart4>
%\fi
%
%\iffalse
%<*samplepart3>
%\fi
% Some text for part 3:
%    \begin{macrocode}
some text in part three
%    \end{macrocode}

%\iffalse
%</samplepart3>
%\fi
% Some text for part 4:
%\iffalse
%<*samplepart4>
%\fi
%    \begin{macrocode}
more text in part four
%    \end{macrocode}

%\iffalse
%</samplepart4>
%\fi
%
% %%%%%%%%%%%%%%%%%%%%%%%%%%%%%%%%%%%%%%
% \paragraph{Forwarding for a Complete Draft.}
%
% The following forwarding file |cdocsdrf.tex|
% compiles the main document in draft mode:
%\iffalse
%<*sampledraft>
%\fi
%    \begin{macrocode}
\def\version{draft}
\input{childdoc.def}
\childdocforward{cdocsamp}
%    \end{macrocode}

%\iffalse
%</sampledraft>
%\fi
%
% %%%%%%%%%%%%%%%%%%%%%%%%%%%%%%%%%%%%%%
% \paragraph{Forwarding for Final Version of the Chapters.}
%
% The following forwarding files |cdocsfn1.tex| and |cdocsfn2.tex|
% (with identical content)
% compile the final versions of the child documents
% |cdocsch1.tex| and |cdocsch2.tex|, respectively:
%\iffalse
%<*samplefinal>
%\fi
%    \begin{macrocode}
\def\version{final}
\input{childdoc.def}
\childdocforwardprefix[cdocsamp]{cdocsfn}{cdocsch}
%    \end{macrocode}

%\iffalse
%</samplefinal>
%\fi
%
% %%%%%%%%%%%%%%%%%%%%%%%%%%%%%%%%%%%%%%
% \paragraph{Command Line Processing.}
%
% The following three command lines generate the output files
% |cdocscld|, |cdocscl1| and |cdocscl2|
% which should be identical to
% |cdocsdrf|, |cdocsch1| and |cdocsfn2|, respectively:
% \begin{center}
% \begin{tabular}{l}
% |latex -jobname cdocscld \|\\
% |  "\def\version{draft}\input{childdoc.def}\childdocforward{cdocsamp}"|\\
% |latex -jobname cdocscl1 \|\\
% |  "\input{childdoc.def}\childdocforward[cdocsamp]{cdocsch1}"|\\
% |latex -jobname cdocscl2 \|\\
% |  "\def\version{final}\input{childdoc.def}\childdocforward{cdocsch2}"|
% \end{tabular}
% \end{center}
% Note that the trailing backslash on each first line
% merely continues the input to the second line
% (for convenient cut ant paste).
% Furthermore, the command |latex| can be replaced by any
% of its alternative versions such as |pdflatex|.
%
% %%%%%%%%%%%%%%%%%%%%%%%%%%%%%%%%%%%%%%%%%%%%%%%%%%%%%%%%%%%%%%%%%%%%%%%%%%%%%%
% %%%%%%%%%%%%%%%%%%%%%%%%%%%%%%%%%%%%%%%%%%%%%%%%%%%%%%%%%%%%%%%%%%%%%%%%%%%%%%
% \section{Implementation}
%\iffalse
%<*package>
%\fi
%
% This section describes the definitions file |childdoc.def|.

% The definitions cannot be loaded using |\usepackage| or |\RequirePackage|
% which has a mechanism to prevent loading a style file more than once.
% When loading the definitions by means of |\input|
% multiple instances have to be prevented manually:
%\iffalse
%This code needs to be before the `\ProvidesFile' directive
%which is defined at the beginning of this file.
%Therefore it is also placed there and commented out here.
%</package>
%<*discard>
%\fi
%    \begin{macrocode}
\ifdefined\childdocmain\endinput\fi
%    \end{macrocode}
%\iffalse
%</discard>
%<*package>
%\fi
%
% \macro{\ifchilddoc}
% \macro{\ifchilddocmanual}
% The conditional |\ifchilddoc| tells whether a
% child (true) or main (false) document is being compiled.
% The conditional |\ifchilddocmanual| tells whether
% the |\includeonly| mechanism is used (false) or
% the selection of child files must be performed manually (true).
% The definitions initialise to false:
%    \begin{macrocode}
\newif\ifchilddoc
\newif\ifchilddocmanual
%    \end{macrocode}

% \macro{\childdocname}
% \macro{\childdocjob}
% The macro |\childdocname| stores the name of the main document
% to be compiled. The macro |\childdocjob| stores the name of
% the document on which the \LaTeX{} compiler was originally invoked.
% The content of |\jobname| cannot be compared
% to filenames specified in the source due to different catcodes.
% The following code rescans |\jobname|, stores the result
% in |\childdocname| and saves a copy in |\childdocjob|:
%    \begin{macrocode}
\edef\childdocname{\scantokens\expandafter{\jobname\noexpand}}
\let\childdocjob\childdocname
%    \end{macrocode}

% \macro{\childdocdisable}
% The macro |\childdocdisable| prevents the main file
% from being processed more than once.
% At this stage, the main document command |\childdocmain|
% is assumed to be called once again where it should do nothing.
% Any subsequent call to it should prevent
% a secondary processing of the main document
% It overwrites the forwarding commands
% |\childdocof| and |\childdocforward|
% with empty macros to prevent further inclusions of the main document:
%    \begin{macrocode}
\newcommand{\childdocdisable}
{
  \renewcommand{\childdocmain}[1]{\renewcommand{\childdocmain}[1]{\endinput}}
  \renewcommand{\childdocof}[1]{}
  \renewcommand{\childdocby}[2][]{}
  \renewcommand{\childdocforward}[2][]{}
  \renewcommand{\childdocdisable}{}
}
%    \end{macrocode}

% \macro{\childdocmain}
% The macro |\childdocmain| is to be called at the top of the main file
% with nothing or the main filename (without extension) as argument.
% First, it breaks loops.
% If the argument is not empty and does not match |\childdocname|
% (which is set by the first inclusion of |childdoc.def|),
% |\ifchilddoc| is set to true, |\includeonly| is applied to the child file
% and |\jobname| is set to the main file
% (for proper handling of |.aux| files):
%    \begin{macrocode}
\newcommand{\childdocmain}[1]
{
  \childdocdisable\childdocmain{}
  \if?#1?\else
    \begingroup
      \def\childdoctmp{#1}
      \ifx\childdoctmp\childdocname
        \def\childdoctmp{}
      \else
        \def\childdoctmp
        {
          \childdoctrue
          \includeonly{\childdocname}
          \def\childdocjob{#1}
          \def\jobname{#1}
        }
      \fi
      \expandafter
    \endgroup
    \childdoctmp
  \fi
}
%    \end{macrocode}

% \macro{\childdocof}
% The command |\childdocof| redirects
% compilation to the main file |#1|.
%    \begin{macrocode}
\newcommand{\childdocof}[1]
{
  \childdocdisable
  \childdoctrue
  \includeonly{\childdocname}
  \def\jobname{#1}
  \def\childdocjob{#1}
  \input{#1}
}
%    \end{macrocode}

% \macro{\childdocby}
% The command |\childdocby| ....
%    \begin{macrocode}
\newcommand{\childdocby}[2][]
{
  \childdocdisable
  \childdoctrue
  \childdocmanualtrue
  \if?#1?\else
    \def\jobname{#2}
  \fi
  \def\childdocjob{#2}
  \input{#2}
  \endinput
}
%    \end{macrocode}

% \macro{\childdocforward}
% The command |\childdocforward| redirects
% compilation to the main file or
% (if the optional argument is given) a child file.
% Parameters are set as if the main file
% or a child file starting with |\childdocof| was compiled.
% Then compilation is handed over to the main file:
%    \begin{macrocode}
\newcommand{\childdocforward}[2][]
{
  \begingroup
    \if?#1?
      \def\childdoctmp
      {
        \def\childdocname{#2}
        \def\childdocjob{#2}
        \def\jobname{#2}
        \input{#2}
        \endinput
      }
    \else
      \def\childdoctmp
      {
        \childdocdisable
        \def\childdocname{#2}
        \childdoctrue
        \includeonly{#2}
        \def\childdocjob{#1}
        \def\jobname{#1}
        \input{#1}
        \endinput
      }
    \fi
    \expandafter
  \endgroup
  \childdoctmp
}
%    \end{macrocode}

% \macro{\childdocforwardprefix}
% The command |\childdocforwardprefix| redirects
% compilation to the main or a child file by means of a pattern.
% The prefix |#1| in the current filename is replaced by |#2|
% and the suffix of the current filename is kept
% (it is assumed that the filename does not contain the substring `|~~~|'
% which is used as a delimiter).
% Compilation is handed over to the new file by |\childdocforward|:
%    \begin{macrocode}
\newcommand{\childdocforwardprefix}[3][]
{
  \begingroup
    \def\childdocextract #2##1~~~{\def\childdoctmp{\childdocforward[#1]{#3##1}}}
    \expandafter\childdocextract\childdocname~~~
    \expandafter
  \endgroup
  \childdoctmp
}
%    \end{macrocode}

% \macro{\childdoc}
% The deprecated macro |\childdoc| is a legacy version of |\childdocmain|:
%    \begin{macrocode}
\newcommand{\childdoc}{\childdocmain}
%    \end{macrocode}

% \macro{\childdocredirect}
% The deprecated macro |\childdocredirect| is a legacy version
% of |\childdocforward| and |\childdocforwardprefix|:
%    \begin{macrocode}
\newcommand{\childdocredirect}[2][]
{
  \begingroup
    \if?#1?
      \def\childdoctmp{\childdocforward{#2}}
    \else
      \def\childdoctmp{\childdocforwardprefix{#1}{#2}}
    \fi
    \expandafter
  \endgroup
  \childdoctmp
}
%    \end{macrocode}

%\iffalse
%</package>
%\fi
%
\endinput

\childdocby{cdocsamp}
%    \end{macrocode}

%\iffalse
%</samplepart3|samplepart4>
%\fi
%
%\iffalse
%<*samplepart3>
%\fi
% Some text for part 3:
%    \begin{macrocode}
some text in part three
%    \end{macrocode}

%\iffalse
%</samplepart3>
%\fi
% Some text for part 4:
%\iffalse
%<*samplepart4>
%\fi
%    \begin{macrocode}
more text in part four
%    \end{macrocode}

%\iffalse
%</samplepart4>
%\fi
%
% %%%%%%%%%%%%%%%%%%%%%%%%%%%%%%%%%%%%%%
% \paragraph{Forwarding for a Complete Draft.}
%
% The following forwarding file |cdocsdrf.tex|
% compiles the main document in draft mode:
%\iffalse
%<*sampledraft>
%\fi
%    \begin{macrocode}
\def\version{draft}
% \iffalse
%
% childdoc.dtx Copyright (C) 2017-2018 Niklas Beisert
%
% This work may be distributed and/or modified under the
% conditions of the LaTeX Project Public License, either version 1.3
% of this license or (at your option) any later version.
% The latest version of this license is in
%   http://www.latex-project.org/lppl.txt
% and version 1.3 or later is part of all distributions of LaTeX
% version 2005/12/01 or later.
%
% This work has the LPPL maintenance status `maintained'.
%
% The Current Maintainer of this work is Niklas Beisert.
%
% This work consists of the files childdoc.dtx and childdoc.ins
% and the derived files childdoc.def and cdocsamp.tex with
% cdocsch1.tex, cdocsch2.tex, cdocsdrf.tex, cdocsfn1.tex, cdocsfn2.tex.
%
%<package>\ifdefined\childdocmain\endinput\fi
%<package>\ProvidesFile{childdoc.def}[2018/12/30 v2.0 child document driver]
%<samplemain>\ProvidesFile{cdocsamp.tex}[2018/12/30 v2.0 sample for childdoc]
%<*driver>
%\ProvidesFile{childdoc.drv}[2018/12/30 v2.0 childdoc reference manual file]
\PassOptionsToClass{10pt,a4paper}{article}
\documentclass{ltxdoc}

\usepackage[margin=35mm]{geometry}
\usepackage{hyperref}
\usepackage{hyperxmp}
\usepackage[usenames]{color}

\hypersetup{colorlinks=true}
\hypersetup{pdfstartview=FitH}
\hypersetup{pdfpagemode=UseNone}
\hypersetup{pdfsource={}}
\hypersetup{pdflang={en-UK}}
\hypersetup{pdfcopyright={Copyright 2017-2018 Niklas Beisert.
  This work may be distributed and/or modified under the
  conditions of the LaTeX Project Public License, either version 1.3
  of this license or (at your option) any later version.}}
\hypersetup{pdflicenseurl={http://www.latex-project.org/lppl.txt}}
\hypersetup{pdfcontactaddress={ETH Zurich, ITP, HIT K,
  Wolfgang-Pauli-Strasse 27}}
\hypersetup{pdfcontactpostcode={8093}}
\hypersetup{pdfcontactcity={Zurich}}
\hypersetup{pdfcontactcountry={Switzerland}}
\hypersetup{pdfcontactemail={nbeisert@itp.phys.ethz.ch}}
\hypersetup{pdfcontacturl={http://people.phys.ethz.ch/\xmptilde nbeisert/}}

\newcommand{\secref}[1]{\hyperref[#1]{section \ref*{#1}}}

\parskip1ex
\parindent0pt
\let\olditemize\itemize
\def\itemize{\olditemize\parskip0pt}

\begin{document}

\title{The \textsf{childdoc} Package}
\hypersetup{pdftitle={The childdoc Package}}
\author{Niklas Beisert\\[2ex]
  Institut f\"ur Theoretische Physik\\
  Eidgen\"ossische Technische Hochschule Z\"urich\\
  Wolfgang-Pauli-Strasse 27, 8093 Z\"urich, Switzerland\\[1ex]
  \href{mailto:nbeisert@itp.phys.ethz.ch}
  {\texttt{nbeisert@itp.phys.ethz.ch}}}
\hypersetup{pdfauthor={Niklas Beisert}}
\hypersetup{pdfsubject={Manual for the LaTeX2e Package childdoc}}
\date{30 December 2018, \textsf{v2.0}}
\maketitle

\begin{abstract}\noindent
\textsf{childdoc} is a \LaTeXe{} package
that enables the direct compilation
of document sections included by |\include|
to individual files.
\end{abstract}

\begingroup
\parskip0ex
\tableofcontents
\endgroup

%%%%%%%%%%%%%%%%%%%%%%%%%%%%%%%%%%%%%%%%%%%%%%%%%%%%%%%%%%%%%%%%%%%%%%%%%%%%%%%%
%%%%%%%%%%%%%%%%%%%%%%%%%%%%%%%%%%%%%%%%%%%%%%%%%%%%%%%%%%%%%%%%%%%%%%%%%%%%%%%%
\section{Introduction}

\LaTeX{} provides a mechanism to structure a large document (such as a book)
into a main file and several child files (containing the chapters)
using the |\include| command.
This mechanism is beneficial for documents
which span hundreds of pages in order to
make the source file(s) more manageable.
Moreover, compilation can be restricted to
selected child files by means of the |\includeonly| command.
The latter feature can be used to reduce the compilation time while editing
(this was significantly more useful in the earlier days of \LaTeX{})
or to generate a smaller document which is easier to navigate.
Another application of |\includeonly| is to generate
documents consisting of selected parts of the complete document.

However, there are a few drawbacks of the plain |\include| mechanism:
\begin{itemize}
\item
The child files cannot be compiled on their own,
they can only be compiled via the main file.
A naive editing environment
(such as a text editor with an option
to have the current file processed by \LaTeX)
may require one to switch to the main file before compiling;
attempting to compile the child file produces errors.
\item
The main file must be modified (each time)
to adjust the |\includeonly| command
to the present needs. This easily leaves the main file in a messy state.
\item
The generated document will always carry the filename
of the main document. This is inconvenient if
several child files are to be compiled and
to be kept for distribution.
\end{itemize}

The present package provides a simple interface
to make child files individually compilable by \LaTeX{}.
Compiling a child file then has the same effect as compiling
the main file with an |\includeonly| command
to select the appropriate child.
Moreover the generated document will carry the name of the child
rather than the main file.
This resolves all three above issues.

This feature is meant to make the editing of books,
thesis documents and lecture notes somewhat more convenient.
However, the package can also be used efficiently for
composing a series of documents (such as exercise sheets)
which are typically distributed individually.
It then assists the author in generating the individual documents
(potentially in different versions)
as well as a document containing the collected series.
Another application is in developing style files
or other kinds of included material
where compilation of the style file could redirect
to a sample or test file.

%%%%%%%%%%%%%%%%%%%%%%%%%%%%%%%%%%%%%%%%%%%%%%%%%%%%%%%%%%%%%%%%%%%%%%%%%%%%%%%%
%%%%%%%%%%%%%%%%%%%%%%%%%%%%%%%%%%%%%%%%%%%%%%%%%%%%%%%%%%%%%%%%%%%%%%%%%%%%%%%%
\section{Usage}

First of all, the package \textsf{childdoc} is \emph{not} a standard
\LaTeXe{} |.sty| style file! Therefore it needs to be invoked in
a non-standard way.

%%%%%%%%%%%%%%%%%%%%%%%%%%%%%%%%%%%%%%%%%%%%%%%%%%%%%%%%%%%%%%%%%%%%%%%%%%%%%%%%
\subsection{Included Files}
\label{sec:include}

%%%%%%%%%%%%%%%%%%%%%%%%%%%%%%%%%%%%%%%%
\DescribeMacro{\childdocmain}
To use the package, add the commands
\begin{center}
\begin{tabular}{l}
|\input{childdoc.def}|\\
|\childdocmain{}|\\
\end{tabular}
\end{center}
at the very top of the main \LaTeX{} file,
in particular \emph{before} the |\documentclass| statement!
The argument of |\childdocmain| should be left empty
(but it must be present).

%%%%%%%%%%%%%%%%%%%%%%%%%%%%%%%%%%%%%%%%
\DescribeMacro{\childdocof}
Furthermore, add the commands
\begin{center}
\begin{tabular}{l}
|\input{childdoc.def}|\\
|\childdocof{|\textit{main}|}|\\
\end{tabular}
\end{center}
at the top of every child file \textit{child}
which is included by |\include{|\textit{child}|}|
from within the main file
(or at least for those files to be compiled individually).
The argument \textit{main} must be the filename of the main file.

There are a couple of
considerations in setting up the main and child documents:

%%%%%%%%%%%%%%%%%%%%%%%%%%%%%%%%%%%%%%%%
\paragraph{Restrictions.}

Please note the following restrictions:
\begin{itemize}
\item
|\childdocmain| must be called with one argument \textit{main}
to ensure compatibility with earlier version of the package.
It must either be empty (|\childdocmain{}|)
or precisely match the filename of the main file in which it is specified.
See \secref{sec:detection} for further information.
\item
The filename \textit{main} must be specified without the |.tex| extension.
\item
The filename \textit{main} is case sensitive
(even in case-insensitive file systems)
due to internal string comparison.
\item
The argument \textit{main} should be fully expanded, it cannot be a macro.
\item
Subdirectories and special characters should be avoided in filenames.
\item
The command |\childdocmain{|\textit{main}|}| must be followed by a whitespace.
It should not be followed immediately by another command
or by a comment mark `|%|'.
This is because the \TeX{} parser reads the token immediately following
the argument of |\childdocmain| and puts it
at the beginning of every child section;
however, a white\-space is ignored.
\end{itemize}

%%%%%%%%%%%%%%%%%%%%%%%%%%%%%%%%%%%%%%%%
\paragraph{Content of Main File.}

It is advisable to place all content in the child files included by |\include|.
Any output contained in the main file will appear in all child documents
unless suppressed manually;
it cannot be suppressed automatically by the |\includeonly| directive
and thus should normally be avoided.
A method to include some content in the main file
by means of conditional processing is described in \secref{sec:conditional}.

%%%%%%%%%%%%%%%%%%%%%%%%%%%%%%%%%%%%%%%%
\paragraph{Page Numbering.}

When only a part of the document is compiled,
the appropriate numbering of pages
(as well as other status parameters)
is determined from the |.aux| files.
The latter contain information from previous passes.
However this information needs to propagate through
all intermediate child documents.
Therefore the page numbering in child documents may well
be inconsistent until the complete document is compiled at least once.

A useful (if unconventional) way to always ensure a consistent
page numbering is to restart the numbering in each child document
and denote the pages by `\textit{child}|.|\textit{page}'
where \textit{child} represents the chapter/section number of the child file.
This can be achieved by the command
|\numberwithin{page}{|\textit{child}|}|
of the \textsf{amsmath} package
where \textit{child} can be |chapter| or |section|
depending on the chosen structuring.
Alternatively, one can modify the macro |\thepage| appropriately
and reset the counter |page| at the start of each child file.

%%%%%%%%%%%%%%%%%%%%%%%%%%%%%%%%%%%%%%%%%%%%%%%%%%%%%%%%%%%%%%%%%%%%%%%%%%%%%%%%
\subsection{Conditional Processing}
\label{sec:conditional}

The package provides a mechanism to compile different versions
of a document. To customise the versions further some conditional processing
can come in handy to distinguish which version is being compiled.
The package provides two macros to describe the compilation context:

%%%%%%%%%%%%%%%%%%%%%%%%%%%%%%%%%%%%%%%%
\DescribeMacro{\ifchilddoc}
The conditional |\ifchilddoc| distinguishes between the compilation of
child documents and the main document:
%
\begin{center}
|\ifchilddoc |\textit{child-code}| |[|\||else |\textit{main-code}]| \||fi|
\end{center}

%%%%%%%%%%%%%%%%%%%%%%%%%%%%%%%%%%%%%%%%
\DescribeMacro{\childdocname}
\DescribeMacro{\childdocjob}
The macro |\childdocname| contains the filename (without extension)
of the main or child file being processed.
Note that |\childdocjob| will always contain the name of the main file.

%%%%%%%%%%%%%%%%%%%%%%%%%%%%%%%%%%%%%%%%
\paragraph{Title Page.}

Conditional processing can be used to include a title or banner page
in the main document when proper precautions are taken.
Importantly, the code in the main file should ensure that the page counter
(as well as other status parameters which are stored in the |.aux| files)
takes the same value after the conditional processing.
Otherwise the page numbers may take divergent values
depending on which part is compiled.

For example, a title page could be declared by:
%
\begin{center}
\begin{tabular}{l}
|\ifchilddoc\||else|\\
|\addtocounter{page}{-1}|\\
\textit{code for title page}\\
|\newpage|\\
|\||fi|
\end{tabular}
\end{center}
%
A banner page for the child documents can be generated by:
%
\begin{center}
\begin{tabular}{l}
|\ifchilddoc|\\
|\addtocounter{page}{-1}|\\
\textit{code for banner page}\\
|\newpage|\\
|\||fi|
\end{tabular}
\end{center}
%
Here one could write a message such as:
\begin{center}
|This is the part \childdocname{} of \childdocjob{}.|
\end{center}

%%%%%%%%%%%%%%%%%%%%%%%%%%%%%%%%%%%%%%%%%%%%%%%%%%%%%%%%%%%%%%%%%%%%%%%%%%%%%%%%
\subsection{Flags}
\label{sec:flags}

The package makes it easy to generate different versions
of the main or child documents.
To this end compilation flags can be defined
and assigned different default values.
They will be particularly useful in conjunction
with the forwarding mechanism described in \secref{sec:forward}.

For example, it may be useful to have a flag |\version|
which can be set to |draft| or |final|.
The document source will contain some conditional code
depending on the value of |\version|.
Suppose further, the flag should default to |final| for the main file
and to |draft| for child files
which is a natural assignment for editing the document.
This is achieved by placing the following code
in the preamble of the main document
(below the |\childdocmain| directive):
%
\begin{center}
\begin{tabular}{l}
|\ifchilddoc|\\
|\providecommand{\version}{draft}|\\
|\||else|\\
|\providecommand{\version}{final}|\\
|\||fi|
\end{tabular}
\end{center}
%
The definition by |\providecommand| makes sure
that previous definitions are not overwritten.
Further statements |\providecommand{\version}{...}|
can thus be added before the above code to override it.

For the main file, one might add a line
(between |\childdocmain| and the above block)
%
\begin{center}
|%\ifchilddoc\||else\providecommand{\version}{draft}\||fi|
\end{center}
%
which can be uncommented to produce a draft version.
Likewise one can add a line to the very top of a child file
(above the |\childdocof{|\textit{main}|}| directive)
%
\begin{center}
|%\providecommand{\version}{final}|
\end{center}
%
which can be uncommented to produce the final version of this child document.

%%%%%%%%%%%%%%%%%%%%%%%%%%%%%%%%%%%%%%%%%%%%%%%%%%%%%%%%%%%%%%%%%%%%%%%%%%%%%%%%
\subsection{Forwarding}
\label{sec:forward}

Different versions of the main or child documents
using compilation flags as described in \secref{sec:flags}
can be (permanently) stored in different files
for convenient compilation, viewing and distribution.
To this end, the package defines a command
to pass on compilation to a different file:

%%%%%%%%%%%%%%%%%%%%%%%%%%%%%%%%%%%%%%%%
\DescribeMacro{\childdocforward}
The command |\childdocforward| redirects processing to
another source file:
%
\begin{center}
\begin{tabular}{l}
|\input{childdoc.def}|\\
|\childdocforward[|\textit{main}|]{|\textit{dest}|}|\\
\end{tabular}
\end{center}
%
The argument \textit{dest} is the destination file
(without extension).
It should be the main file or one of the child files.
Note that further \textsf{childdoc} directives
such as |\childdocof| and |\childdocforward|
in the indicated file will be processed in this form.
The optional argument \textit{main}
passes on directly to the main file \textit{main}
while pretending to compile the child \textit{dest}.
This form behaves as if \textit{dest}
issues |\childdocof{|\textit{main}|}| right away,
and no further \textsf{childdoc} directives will be processed.

%%%%%%%%%%%%%%%%%%%%%%%%%%%%%%%%%%%%%%%%
\DescribeMacro{\...prefix}
In the alternative form |\childdocforwardprefix|,
%
\begin{center}
\begin{tabular}{l}
|\input{childdoc.def}|\\
|\childdocforwardprefix[|\textit{main}|]{|\textit{prefix}|}{|\textit{dest}|}|
\end{tabular}
\end{center}
%
the destination file is determined by a pattern
depending on the current file:
To make this work, the current file must be called
`{\textit{prefix}\hspace{0.2em}\textit{suffix}}'
with \textit{prefix} matching precisely the argument.
Processing is then passed on to the file
`{\textit{dest}\hspace{0.2em}\textit{suffix}}'.
Surely, the same effect is achieved by
directly specifying the
argument `{\textit{dest}\hspace{0.2em}\textit{suffix}}'
in the first form.
However, that requires to set up a different file
for each child. With the alternative form of the command
all these files can have exactly the same content
which simplifies setting them up and maintaining them.

For example, the following file |draft.tex|
with a compilation flag |\version| as described in \secref{sec:flags}
compiles the main document as a draft:
%
\begin{center}
\begin{tabular}{l}
|\def\version{draft}|\\
|\input{childdoc.def}|\\
|\childdocforward{|\textit{main}|}|
\end{tabular}
\end{center}
%
Likewise, the following files |final|\textit{nn}|.tex|
compile the final version of the child document
|child|\textit{nn}|.tex|:
%
\begin{center}
\begin{tabular}{l}
|\def\version{final}|\\
|\input{childdoc.def}|\\
|\childdocforwardprefix{final}{child}|
\end{tabular}
\end{center}
%

Note that when several versions of a main file and/or of each child file
are to be generated, it may be convenient to set up a |Makefile| or
shell script to automatise the process.

%%%%%%%%%%%%%%%%%%%%%%%%%%%%%%%%%%%%%%%%%%%%%%%%%%%%%%%%%%%%%%%%%%%%%%%%%%%%%%%%
\subsection{Command Line Processing}
\label{sec:commandline}

The effect of redirection files can also be achieved by invoking
the \LaTeX{} compiler with a more elaborate command line.
Most conveniently this should be done as part
of a shell script or a |Makefile|.

When using \textsf{childdoc} in the main file, the following
command lines effectively perform a redirection
(note that depending on the shell being used,
backslashes may have to be doubled: `|\|' $\to$ `|\\|'):
%
\begin{center}
|... -jobname "|\textit{target}|" |\\|"|[\textit{flags}]%
|\input{childdoc.def}\childdocforward[|\textit{main}|]{|\textit{dest}|}"|
\end{center}
%
Here \textit{target} is the name of the output file,
\textit{main} is the name of the main file
and \textit{dest} is the name of the main or child file to be processed
(all filenames without extensions).
The optional argument \textit{main} can be omitted
if \textit{main} matches \textit{dest}.
Optionally, compilation \textit{flags} can be defined via |\def| commands.
This command line makes the \TeX{} engine believe
it is compiling the file \textit{target}
whose content is specified as the latter parameter.
The provided code then forwards the processing to
\textit{main} or \textit{dest} as described in \secref{sec:forward}.

%%%%%%%%%%%%%%%%%%%%%%%%%%%%%%%%%%%%%%%%%%%%%%%%%%%%%%%%%%%%%%%%%%%%%%%%%%%%%%%%
\subsection{Include by Input}
\label{sec:input}

Including child documents by |\include| has some restrictions by design.
Most notably, the content of a child document always occupies
its own set of pages; pages cannot be shared between child documents.
Usually, this behaviour makes perfect sense
because each child document contain an essential part of the document.
However, in some situations it may be desirable to compose
a document from a collection of parts
without having mandatory page breaks between then.
For this case, the package
provides a mechanism to include parts
by |\input| which can also be processed individually.
However, by construction this mechanism
requires manual handling of the content to be output.

%%%%%%%%%%%%%%%%%%%%%%%%%%%%%%%%%%%%%%%%
\DescribeMacro{\ifchilddocmanual}
The main file should be prepared as usual, see \secref{sec:include}.
However, the document body must make a distinction
between processing of an individual part and of the main document, e.g.:
%
\begin{center}
\begin{tabular}{l}
|\ifchilddocmanual|\\
|\input{\childdocname}|\\
|\||else|\\
\textit{document body with }|\input{|\textit{part}|}|\\
|\||fi|
\end{tabular}
\end{center}
%
The conditional |\ifchilddocmanual| is true whenever
a part to be included by |\input| is being compiled,
and the name of the part is stored in |\childdocname|.

%%%%%%%%%%%%%%%%%%%%%%%%%%%%%%%%%%%%%%%%
\DescribeMacro{\childdocby}
Each part to be included by |\input| should start with:
%
\begin{center}
\begin{tabular}{l}
|\input{childdoc.def}|\\
|\childdocby{|\textit{main}|}|\\
\end{tabular}
\end{center}
%
The directive |\childdocby| is similar to |\childdocof|
described in \secref{sec:include},
but the subsequent selection of content must be done manually.
To that end, both |\ifchilddoc| and |\ifchilddocmanual|
will be true upon processing of a part,
and the name of the part is stored in |\childdocname|.
Note that |\jobname| will be set to the filename of the current part
so that each part receives an individual |.aux| file
that does not interfere with the |.aux| file(s) of the main document.
This behaviour can be altered by the alternative form
|\childdocby[*]{|\textit{main}|}| (with a non-empty optional argument)
which uses the |.aux| file of the main document
by setting |\jobname| to \textit{main}.

%%%%%%%%%%%%%%%%%%%%%%%%%%%%%%%%%%%%%%%%%%%%%%%%%%%%%%%%%%%%%%%%%%%%%%%%%%%%%%%%
\subsection{Driver Development}
\label{sec:driver}

The \textsf{childdoc} mechanism can also be use for the development
of definition files such as \LaTeX{} styles or classes.
This case differs from the above setup with multiple parts
included by |\include| in that no |\includeonly| should be invoked.
This can be achieved by starting the include file
(before |\ProvidesPackage|) with:
%
\begin{center}
\begin{tabular}{l}
|\input{childdoc.def}|\\
|\childdocforward{|\textit{main}|}|\\
\end{tabular}
\end{center}
%
or alternatively with:
%
\begin{center}
\begin{tabular}{l}
|\input{childdoc.def}|\\
|\childdocby{|\textit{main}|}|\\
\end{tabular}
\end{center}
%
Both forms have slightly different effects as described above.
The main file is prepared as usual, see \secref{sec:include}.

%%%%%%%%%%%%%%%%%%%%%%%%%%%%%%%%%%%%%%%%%%%%%%%%%%%%%%%%%%%%%%%%%%%%%%%%%%%%%%%%
\subsection{Legacy Detection}
\label{sec:detection}

The directive |\childdocmain| in the main file can detect
whether the complete document or merely a child is to be compiled
even without using the directive |\childdocof|.
This method is deprecated because it is less robust
and there is no compelling reason to use it;
it is merely provided for backward compatibility
and it may be removed in future versions.

If the detection mechanism is to be used,
it is mandatory to correctly specify
the filename of the main file as the argument of |\childdocmain|:
%
\begin{center}
\begin{tabular}{l}
|\input{childdoc.def}|\\
|\childdocmain{|\textit{main}|}|\\
\end{tabular}
\end{center}
%
If |\jobname| does not match the argument \textit{main} of |\childdocmain|,
it is assumed that |\jobname| points to the child file to be compiled.
When using |\childdocmain| with the main file specified as argument,
it suffices to start a child file
with just |\input{|\textit{main}|}|
without loading of the package and using |\childdocof|.
If instead all processing is done
with the appropriate \textsf{childdoc} directives,
the argument of \textit{main} of |\childdocmain| can be empty.

An alternative version of the command line processing described
in \secref{sec:commandline} using the detection mechanism reads:
%
\begin{center}
|... -jobname "|\textit{target}|" "|[\textit{flags}]%
[|\def\jobname{|\textit{dest}|}|]|\input{|\textit{main}|}"|
\end{center}

%%%%%%%%%%%%%%%%%%%%%%%%%%%%%%%%%%%%%%%%%%%%%%%%%%%%%%%%%%%%%%%%%%%%%%%%%%%%%%%%
\subsection{Manual Code}
\label{sec:manual}

In case one cannot be certain whether the definitions file |childdoc.def|
is installed on the target \TeX{} distribution
and one prefers not to ship it,
it is conceivable to paste a few relevant commands into the sources.

To that end, drop all statements |\input{childdoc.def}|
and perform the replacements as outlined below.
Instead of |\childdocmain{|\textit{main}|}| add the following code
to the top of the main file:
%
\begin{center}
\begin{tabular}{l}
|\||ifdefined\childdocname\endinput\||fi\newif\ifchilddoc|\\
|\edef\childdocname{\scantokens\expandafter{\jobname\noexpand}}|\\
|\def\childdocmain{|\textit{main}|}\||ifx\childdocmain\childdocname\||else|\\
|\childdoctrue\includeonly{\childdocname}\let\jobname\childdocmain\||fi|\\
\end{tabular}
\end{center}
%
Instead of |\childdocof{|\textit{main}|}| just include the main file
at the top of each child file:
%
\begin{center}
|\input{|\textit{main}|}|
\end{center}
%
A simple redirection |\childdocforward{|\textit{dest}|}| is achieved by:
%
\begin{center}
|\def\jobname{|\textit{dest}|}\input{\jobname}|
\end{center}
%
The redirection with prefix
|\childdocforwardprefix[|\textit{prefix}|]{|\textit{dest}|}|
is accomplished by:
%
\begin{center}
\begin{tabular}{l}
|{\edef\jobname{\scantokens\expandafter{\jobname\noexpand}}|\\
|\def\redirectjob |\textit{prefix}|#1~~~{\gdef\jobname{|\textit{dest}|#1}}|\\
|\expandafter\redirectjob\jobname~~~}\input{\jobname}|
\end{tabular}
\end{center}

In an alternative approach,
child documents can be compiled by a specific command line
without additional code or specific definitions:
%
\begin{center}
|... -jobname "|\textit{target}|" "|[\textit{flags}]%
|\includeonly{|\textit{dest}|}\input{|\textit{main}|}"|
\end{center}
%

%%%%%%%%%%%%%%%%%%%%%%%%%%%%%%%%%%%%%%%%%%%%%%%%%%%%%%%%%%%%%%%%%%%%%%%%%%%%%%%%
%%%%%%%%%%%%%%%%%%%%%%%%%%%%%%%%%%%%%%%%%%%%%%%%%%%%%%%%%%%%%%%%%%%%%%%%%%%%%%%%
\section{Information}

%%%%%%%%%%%%%%%%%%%%%%%%%%%%%%%%%%%%%%%%%%%%%%%%%%%%%%%%%%%%%%%%%%%%%%%%%%%%%%%%
\subsection{Copyright}

Copyright \copyright{} 2017--2018 Niklas Beisert

This work may be distributed and/or modified under the
conditions of the \LaTeX{} Project Public License, either version 1.3
of this license or (at your option) any later version.
The latest version of this license is in
  \url{http://www.latex-project.org/lppl.txt}
and version 1.3 or later is part of all distributions of \LaTeX{}
version 2005/12/01 or later.

This work has the LPPL maintenance status `maintained'.

The Current Maintainer of this work is Niklas Beisert.

This work consists of the files |README.txt|, |childdoc.ins| and |childdoc.dtx|
as well as the derived files |childdoc.def|, |cdocsamp.tex|
with |cdocsch1.tex|, |cdocsch2.tex|, |cdocspt3.tex|, |cdocspt4.tex|,
|cdocsdrf.tex|, |cdocsfn1.tex|, |cdocsfn2.tex|
as well as |childdoc.pdf|.

%%%%%%%%%%%%%%%%%%%%%%%%%%%%%%%%%%%%%%%%%%%%%%%%%%%%%%%%%%%%%%%%%%%%%%%%%%%%%%%%
\subsection{Files and Installation}

The package consists of the files:
%
\begin{center}
\begin{tabular}{ll}
    |README.txt|   & readme file \\
    |childdoc.ins| & installation file \\
    |childdoc.dtx| & source file \\
    |childdoc.def| & definition file \\
    |cdocsamp.tex| & sample main file \\
    |cdocsch1.tex| & sample include file \\
    |cdocsch2.tex| & sample include file \\
    |cdocspt3.tex| & sample part file \\
    |cdocspt4.tex| & sample part file \\
    |cdocsdrf.tex| & sample redirection file \\
    |cdocsfn1.tex| & sample redirection file \\
    |cdocsfn2.tex| & sample redirection file \\
    |childdoc.pdf| & manual
\end{tabular}
\end{center}
%
The distribution consists of the files
|README.txt|, |childdoc.ins| and |childdoc.dtx|.
%
\begin{itemize}
\item
Run (pdf)\LaTeX{} on |childdoc.dtx|
to compile the manual |childdoc.pdf| (this file).
\item
Run \LaTeX{} on |childdoc.ins| to create the definitions file |childdoc.def|
and the sample |cdocsamp.tex| with include files
|cdocsch1.tex|, |cdocsch2.tex|, |cdocspt3.tex|, |cdocspt4.tex|,
|cdocsdrf.tex|, |cdocsfn1.tex|, |cdocsfn2.tex|.
Then copy the file |childdoc.def| to an appropriate directory of your \LaTeX{}
distribution, e.g.\ \textit{texmf-root}|/tex/latex/childdoc|.
\end{itemize}

%%%%%%%%%%%%%%%%%%%%%%%%%%%%%%%%%%%%%%%%%%%%%%%%%%%%%%%%%%%%%%%%%%%%%%%%%%%%%%%%
\subsection{Related CTAN Packages}

There are several other packages which offer a similar functionality:
%
\begin{itemize}
\item
The packages
\href{http://ctan.org/pkg/docmute}{\textsf{docmute}},
\href{http://ctan.org/pkg/includex}{\textsf{includex}} and
\href{http://ctan.org/pkg/standalone}{\textsf{standalone}}
provide commands to include only the document body of
a child file thus allowing both files to be compiled individually.
\item
The packages \href{http://ctan.org/pkg/subdocs}{\textsf{subdocs}}
and \href{http://ctan.org/pkg/subfiles}{\textsf{subfiles}}
provide structures in which the main and child documents can be
encapsulated and allowing them to be compiled individually.
The inclusion mechanism is different from the conventional |\include|.
\item
The package \href{http://ctan.org/pkg/combine}{\textsf{combine}}
is an elaborate solution to combine several documents into one.
\end{itemize}
%
See also the CTAN topic \href{http://ctan.org/topic/subdocs}{\textsf{subdocs}}
for further related packages.
The present package differs from the above solutions in that
a document structure constructed with the conventional |\include| mechanism
just needs two extra commands at the top of every file
such that all constituent files can be compiled individually.

%%%%%%%%%%%%%%%%%%%%%%%%%%%%%%%%%%%%%%%%%%%%%%%%%%%%%%%%%%%%%%%%%%%%%%%%%%%%%%%%
%\subsection{Feature Suggestions}
%
%The following is a list of features which may be useful for future
%versions of this package:
%%
%\begin{itemize}
%\item
%\ldots
%\end{itemize}

%%%%%%%%%%%%%%%%%%%%%%%%%%%%%%%%%%%%%%%%%%%%%%%%%%%%%%%%%%%%%%%%%%%%%%%%%%%%%%%%
\subsection{Revision History}

%%%%%%%%%%%%%%%%%%%%%%%%%%%%%%%%%%%%%%%%
\paragraph{v2.0:} 2018/12/30

\begin{itemize}
\item
immediate forward processing
\item
added |\childdocby| mechanism
\item
manual restructured
\end{itemize}

%%%%%%%%%%%%%%%%%%%%%%%%%%%%%%%%%%%%%%%%
\paragraph{v1.6:} 2018/01/17

\begin{itemize}
\item
application for development of include files
\item
corrections to manual
\end{itemize}

%%%%%%%%%%%%%%%%%%%%%%%%%%%%%%%%%%%%%%%%
\paragraph{v1.5:} 2017/05/21

\begin{itemize}
\item
more complete structuring introduced
\item
|\childdocof| introduced
\item
|\childdoc| renamed to |\childdocmain|
\item
|\childredirect| renamed to |\childdocforward| and |\childdocforwardprefix|
and functionality expanded
\end{itemize}

%%%%%%%%%%%%%%%%%%%%%%%%%%%%%%%%%%%%%%%%
\paragraph{v1.0:} 2017/04/27

\begin{itemize}
\item
manual and install package
\item
first version published on CTAN
\end{itemize}

%%%%%%%%%%%%%%%%%%%%%%%%%%%%%%%%%%%%%%%%
\paragraph{v0.6:} 2017/04/26

\begin{itemize}
\item
redirection mechanism added
\end{itemize}

%%%%%%%%%%%%%%%%%%%%%%%%%%%%%%%%%%%%%%%%
\paragraph{v0.5:} 2017/04/26

\begin{itemize}
\item
functionality in definition file
\end{itemize}


%%%%%%%%%%%%%%%%%%%%%%%%%%%%%%%%%%%%%%%%%%%%%%%%%%%%%%%%%%%%%%%%%%%%%%%%%%%%%%%%
%%%%%%%%%%%%%%%%%%%%%%%%%%%%%%%%%%%%%%%%%%%%%%%%%%%%%%%%%%%%%%%%%%%%%%%%%%%%%%%%
%%%%%%%%%%%%%%%%%%%%%%%%%%%%%%%%%%%%%%%%%%%%%%%%%%%%%%%%%%%%%%%%%%%%%%%%%%%%%%%%
\appendix

\settowidth\MacroIndent{\rmfamily\scriptsize 000\ }

 \DocInput{childdoc.dtx}

\end{document}
%</driver>
% \fi
%
% %%%%%%%%%%%%%%%%%%%%%%%%%%%%%%%%%%%%%%%%%%%%%%%%%%%%%%%%%%%%%%%%%%%%%%%%%%%%%%
% %%%%%%%%%%%%%%%%%%%%%%%%%%%%%%%%%%%%%%%%%%%%%%%%%%%%%%%%%%%%%%%%%%%%%%%%%%%%%%
% \section{Sample}
%\iffalse
%<*samplemain>
%\fi
%
% The following presents a sample document
% with two chapters, two parts, a title page,
% a compile flag as well as three forwarding files to set the flag.
% It consists of eight |.tex| files:
% \begin{center}
% \begin{tabular}{ll}
% |cdocsamp.tex|&main file\\
% |cdocsch1.tex|&include file for chapter 1\\
% |cdocsch2.tex|&include file for chapter 2\\
% |cdocspt3.tex|&include file for part 3\\
% |cdocspt4.tex|&include file for part 4\\
% |cdocsdrf.tex|&forwarding file for main file in draft mode\\
% |cdocsfi1.tex|&forwarding file for final version of chapter 1\\
% |cdocsfi2.tex|&forwarding file for final version of chapter 2\\
% \end{tabular}
% \end{center}
% Each of the eight files can be compiled directly by the \LaTeX{} compiler.
%
% %%%%%%%%%%%%%%%%%%%%%%%%%%%%%%%%%%%%%%
% \paragraph{Main File.}
%
% The main file is called |cdocsamp.tex|.
%
% Load the \textsf{childdoc} definitions and
% declare the filename for the main document:
%    \begin{macrocode}
\input{childdoc.def}
\childdocmain{}
%    \end{macrocode}

% Optional override for |\version| flag:
%    \begin{macrocode}
%%\ifchilddoc\else\providecommand{\version}{draft}\fi
%    \end{macrocode}

% Define the default values for the |\version| flag
% (|final| for the main file and |draft| for childs):
%    \begin{macrocode}
\ifchilddoc
\providecommand{\version}{draft}
\else
\providecommand{\version}{final}
\fi
%    \end{macrocode}

% Load the standard document class:
%    \begin{macrocode}
\documentclass[12pt]{article}
%    \end{macrocode}

% Start the document body:
%    \begin{macrocode}
\begin{document}
%    \end{macrocode}

% Declare a title page.
% Print title, part of document being processed and version flag:
%    \begin{macrocode}
\addtocounter{page}{-1}
\begin{center}
{\LARGE\bfseries{}childdoc example\par}
\vspace{1cm}
\ifchilddoc
\ifchilddocmanual part\else chapter\fi:
`\childdocname' of `\childdocjob'\par
\else
main document: `\childdocjob'\par
\fi
version: \version\par
\end{center}
\newpage
%    \end{macrocode}

% Manually include selected file,
% otherwise process as usual:
%    \begin{macrocode}
\ifchilddocmanual
\section*{part `\childdocname'}
\input{\childdocname}
\else
%    \end{macrocode}

% Include the two chapters:
%    \begin{macrocode}
\include{cdocsch1}
\include{cdocsch2}
%    \end{macrocode}

% Include the two parts unless only chapters should be displayed:
%    \begin{macrocode}
\ifchilddoc\else
\section{part three}
\input{cdocspt3}
\section{part four}
\input{cdocspt4}
\fi
%    \end{macrocode}

% Process as usual until here:
%    \begin{macrocode}
\fi
%    \end{macrocode}

% End of document body:
%    \begin{macrocode}
\end{document}
%    \end{macrocode}
%\iffalse
%</samplemain>
%\fi
%
% %%%%%%%%%%%%%%%%%%%%%%%%%%%%%%%%%%%%%%
% \paragraph{Chapter Include Files.}
%
% The include files are called |cdocsch1.tex| and |cdocsch2.tex|.
%
%\iffalse
%<*samplechap1|samplechap2>
%\fi

% Optional override for |\version| flag:
%    \begin{macrocode}
%%\providecommand{\version}{final}
%    \end{macrocode}

% Include the main document:
%    \begin{macrocode}
\input{childdoc.def}
\childdocof{cdocsamp}
%    \end{macrocode}

%\iffalse
%</samplechap1|samplechap2>
%\fi
%
%\iffalse
%<*samplechap1>
%\fi
% Some text for chapter 1:
%    \begin{macrocode}
\section{one}
some text in chapter one
%    \end{macrocode}

%\iffalse
%</samplechap1>
%\fi
% Some text for chapter 2:
%\iffalse
%<*samplechap2>
%\fi
%    \begin{macrocode}
\section{two}
more text in chapter two
%    \end{macrocode}

%\iffalse
%</samplechap2>
%\fi
%
% %%%%%%%%%%%%%%%%%%%%%%%%%%%%%%%%%%%%%%
% \paragraph{Part Include Files.}
%
% The include files are called |cdocspt3.tex| and |cdocspt4.tex|.
%
%\iffalse
%<*samplepart3|samplepart4>
%\fi

% Optional override for |\version| flag:
%    \begin{macrocode}
%%\providecommand{\version}{final}
%    \end{macrocode}

% Include the main document:
%    \begin{macrocode}
\input{childdoc.def}
\childdocby{cdocsamp}
%    \end{macrocode}

%\iffalse
%</samplepart3|samplepart4>
%\fi
%
%\iffalse
%<*samplepart3>
%\fi
% Some text for part 3:
%    \begin{macrocode}
some text in part three
%    \end{macrocode}

%\iffalse
%</samplepart3>
%\fi
% Some text for part 4:
%\iffalse
%<*samplepart4>
%\fi
%    \begin{macrocode}
more text in part four
%    \end{macrocode}

%\iffalse
%</samplepart4>
%\fi
%
% %%%%%%%%%%%%%%%%%%%%%%%%%%%%%%%%%%%%%%
% \paragraph{Forwarding for a Complete Draft.}
%
% The following forwarding file |cdocsdrf.tex|
% compiles the main document in draft mode:
%\iffalse
%<*sampledraft>
%\fi
%    \begin{macrocode}
\def\version{draft}
\input{childdoc.def}
\childdocforward{cdocsamp}
%    \end{macrocode}

%\iffalse
%</sampledraft>
%\fi
%
% %%%%%%%%%%%%%%%%%%%%%%%%%%%%%%%%%%%%%%
% \paragraph{Forwarding for Final Version of the Chapters.}
%
% The following forwarding files |cdocsfn1.tex| and |cdocsfn2.tex|
% (with identical content)
% compile the final versions of the child documents
% |cdocsch1.tex| and |cdocsch2.tex|, respectively:
%\iffalse
%<*samplefinal>
%\fi
%    \begin{macrocode}
\def\version{final}
\input{childdoc.def}
\childdocforwardprefix[cdocsamp]{cdocsfn}{cdocsch}
%    \end{macrocode}

%\iffalse
%</samplefinal>
%\fi
%
% %%%%%%%%%%%%%%%%%%%%%%%%%%%%%%%%%%%%%%
% \paragraph{Command Line Processing.}
%
% The following three command lines generate the output files
% |cdocscld|, |cdocscl1| and |cdocscl2|
% which should be identical to
% |cdocsdrf|, |cdocsch1| and |cdocsfn2|, respectively:
% \begin{center}
% \begin{tabular}{l}
% |latex -jobname cdocscld \|\\
% |  "\def\version{draft}\input{childdoc.def}\childdocforward{cdocsamp}"|\\
% |latex -jobname cdocscl1 \|\\
% |  "\input{childdoc.def}\childdocforward[cdocsamp]{cdocsch1}"|\\
% |latex -jobname cdocscl2 \|\\
% |  "\def\version{final}\input{childdoc.def}\childdocforward{cdocsch2}"|
% \end{tabular}
% \end{center}
% Note that the trailing backslash on each first line
% merely continues the input to the second line
% (for convenient cut ant paste).
% Furthermore, the command |latex| can be replaced by any
% of its alternative versions such as |pdflatex|.
%
% %%%%%%%%%%%%%%%%%%%%%%%%%%%%%%%%%%%%%%%%%%%%%%%%%%%%%%%%%%%%%%%%%%%%%%%%%%%%%%
% %%%%%%%%%%%%%%%%%%%%%%%%%%%%%%%%%%%%%%%%%%%%%%%%%%%%%%%%%%%%%%%%%%%%%%%%%%%%%%
% \section{Implementation}
%\iffalse
%<*package>
%\fi
%
% This section describes the definitions file |childdoc.def|.

% The definitions cannot be loaded using |\usepackage| or |\RequirePackage|
% which has a mechanism to prevent loading a style file more than once.
% When loading the definitions by means of |\input|
% multiple instances have to be prevented manually:
%\iffalse
%This code needs to be before the `\ProvidesFile' directive
%which is defined at the beginning of this file.
%Therefore it is also placed there and commented out here.
%</package>
%<*discard>
%\fi
%    \begin{macrocode}
\ifdefined\childdocmain\endinput\fi
%    \end{macrocode}
%\iffalse
%</discard>
%<*package>
%\fi
%
% \macro{\ifchilddoc}
% \macro{\ifchilddocmanual}
% The conditional |\ifchilddoc| tells whether a
% child (true) or main (false) document is being compiled.
% The conditional |\ifchilddocmanual| tells whether
% the |\includeonly| mechanism is used (false) or
% the selection of child files must be performed manually (true).
% The definitions initialise to false:
%    \begin{macrocode}
\newif\ifchilddoc
\newif\ifchilddocmanual
%    \end{macrocode}

% \macro{\childdocname}
% \macro{\childdocjob}
% The macro |\childdocname| stores the name of the main document
% to be compiled. The macro |\childdocjob| stores the name of
% the document on which the \LaTeX{} compiler was originally invoked.
% The content of |\jobname| cannot be compared
% to filenames specified in the source due to different catcodes.
% The following code rescans |\jobname|, stores the result
% in |\childdocname| and saves a copy in |\childdocjob|:
%    \begin{macrocode}
\edef\childdocname{\scantokens\expandafter{\jobname\noexpand}}
\let\childdocjob\childdocname
%    \end{macrocode}

% \macro{\childdocdisable}
% The macro |\childdocdisable| prevents the main file
% from being processed more than once.
% At this stage, the main document command |\childdocmain|
% is assumed to be called once again where it should do nothing.
% Any subsequent call to it should prevent
% a secondary processing of the main document
% It overwrites the forwarding commands
% |\childdocof| and |\childdocforward|
% with empty macros to prevent further inclusions of the main document:
%    \begin{macrocode}
\newcommand{\childdocdisable}
{
  \renewcommand{\childdocmain}[1]{\renewcommand{\childdocmain}[1]{\endinput}}
  \renewcommand{\childdocof}[1]{}
  \renewcommand{\childdocby}[2][]{}
  \renewcommand{\childdocforward}[2][]{}
  \renewcommand{\childdocdisable}{}
}
%    \end{macrocode}

% \macro{\childdocmain}
% The macro |\childdocmain| is to be called at the top of the main file
% with nothing or the main filename (without extension) as argument.
% First, it breaks loops.
% If the argument is not empty and does not match |\childdocname|
% (which is set by the first inclusion of |childdoc.def|),
% |\ifchilddoc| is set to true, |\includeonly| is applied to the child file
% and |\jobname| is set to the main file
% (for proper handling of |.aux| files):
%    \begin{macrocode}
\newcommand{\childdocmain}[1]
{
  \childdocdisable\childdocmain{}
  \if?#1?\else
    \begingroup
      \def\childdoctmp{#1}
      \ifx\childdoctmp\childdocname
        \def\childdoctmp{}
      \else
        \def\childdoctmp
        {
          \childdoctrue
          \includeonly{\childdocname}
          \def\childdocjob{#1}
          \def\jobname{#1}
        }
      \fi
      \expandafter
    \endgroup
    \childdoctmp
  \fi
}
%    \end{macrocode}

% \macro{\childdocof}
% The command |\childdocof| redirects
% compilation to the main file |#1|.
%    \begin{macrocode}
\newcommand{\childdocof}[1]
{
  \childdocdisable
  \childdoctrue
  \includeonly{\childdocname}
  \def\jobname{#1}
  \def\childdocjob{#1}
  \input{#1}
}
%    \end{macrocode}

% \macro{\childdocby}
% The command |\childdocby| ....
%    \begin{macrocode}
\newcommand{\childdocby}[2][]
{
  \childdocdisable
  \childdoctrue
  \childdocmanualtrue
  \if?#1?\else
    \def\jobname{#2}
  \fi
  \def\childdocjob{#2}
  \input{#2}
  \endinput
}
%    \end{macrocode}

% \macro{\childdocforward}
% The command |\childdocforward| redirects
% compilation to the main file or
% (if the optional argument is given) a child file.
% Parameters are set as if the main file
% or a child file starting with |\childdocof| was compiled.
% Then compilation is handed over to the main file:
%    \begin{macrocode}
\newcommand{\childdocforward}[2][]
{
  \begingroup
    \if?#1?
      \def\childdoctmp
      {
        \def\childdocname{#2}
        \def\childdocjob{#2}
        \def\jobname{#2}
        \input{#2}
        \endinput
      }
    \else
      \def\childdoctmp
      {
        \childdocdisable
        \def\childdocname{#2}
        \childdoctrue
        \includeonly{#2}
        \def\childdocjob{#1}
        \def\jobname{#1}
        \input{#1}
        \endinput
      }
    \fi
    \expandafter
  \endgroup
  \childdoctmp
}
%    \end{macrocode}

% \macro{\childdocforwardprefix}
% The command |\childdocforwardprefix| redirects
% compilation to the main or a child file by means of a pattern.
% The prefix |#1| in the current filename is replaced by |#2|
% and the suffix of the current filename is kept
% (it is assumed that the filename does not contain the substring `|~~~|'
% which is used as a delimiter).
% Compilation is handed over to the new file by |\childdocforward|:
%    \begin{macrocode}
\newcommand{\childdocforwardprefix}[3][]
{
  \begingroup
    \def\childdocextract #2##1~~~{\def\childdoctmp{\childdocforward[#1]{#3##1}}}
    \expandafter\childdocextract\childdocname~~~
    \expandafter
  \endgroup
  \childdoctmp
}
%    \end{macrocode}

% \macro{\childdoc}
% The deprecated macro |\childdoc| is a legacy version of |\childdocmain|:
%    \begin{macrocode}
\newcommand{\childdoc}{\childdocmain}
%    \end{macrocode}

% \macro{\childdocredirect}
% The deprecated macro |\childdocredirect| is a legacy version
% of |\childdocforward| and |\childdocforwardprefix|:
%    \begin{macrocode}
\newcommand{\childdocredirect}[2][]
{
  \begingroup
    \if?#1?
      \def\childdoctmp{\childdocforward{#2}}
    \else
      \def\childdoctmp{\childdocforwardprefix{#1}{#2}}
    \fi
    \expandafter
  \endgroup
  \childdoctmp
}
%    \end{macrocode}

%\iffalse
%</package>
%\fi
%
\endinput

\childdocforward{cdocsamp}
%    \end{macrocode}

%\iffalse
%</sampledraft>
%\fi
%
% %%%%%%%%%%%%%%%%%%%%%%%%%%%%%%%%%%%%%%
% \paragraph{Forwarding for Final Version of the Chapters.}
%
% The following forwarding files |cdocsfn1.tex| and |cdocsfn2.tex|
% (with identical content)
% compile the final versions of the child documents
% |cdocsch1.tex| and |cdocsch2.tex|, respectively:
%\iffalse
%<*samplefinal>
%\fi
%    \begin{macrocode}
\def\version{final}
% \iffalse
%
% childdoc.dtx Copyright (C) 2017-2018 Niklas Beisert
%
% This work may be distributed and/or modified under the
% conditions of the LaTeX Project Public License, either version 1.3
% of this license or (at your option) any later version.
% The latest version of this license is in
%   http://www.latex-project.org/lppl.txt
% and version 1.3 or later is part of all distributions of LaTeX
% version 2005/12/01 or later.
%
% This work has the LPPL maintenance status `maintained'.
%
% The Current Maintainer of this work is Niklas Beisert.
%
% This work consists of the files childdoc.dtx and childdoc.ins
% and the derived files childdoc.def and cdocsamp.tex with
% cdocsch1.tex, cdocsch2.tex, cdocsdrf.tex, cdocsfn1.tex, cdocsfn2.tex.
%
%<package>\ifdefined\childdocmain\endinput\fi
%<package>\ProvidesFile{childdoc.def}[2018/12/30 v2.0 child document driver]
%<samplemain>\ProvidesFile{cdocsamp.tex}[2018/12/30 v2.0 sample for childdoc]
%<*driver>
%\ProvidesFile{childdoc.drv}[2018/12/30 v2.0 childdoc reference manual file]
\PassOptionsToClass{10pt,a4paper}{article}
\documentclass{ltxdoc}

\usepackage[margin=35mm]{geometry}
\usepackage{hyperref}
\usepackage{hyperxmp}
\usepackage[usenames]{color}

\hypersetup{colorlinks=true}
\hypersetup{pdfstartview=FitH}
\hypersetup{pdfpagemode=UseNone}
\hypersetup{pdfsource={}}
\hypersetup{pdflang={en-UK}}
\hypersetup{pdfcopyright={Copyright 2017-2018 Niklas Beisert.
  This work may be distributed and/or modified under the
  conditions of the LaTeX Project Public License, either version 1.3
  of this license or (at your option) any later version.}}
\hypersetup{pdflicenseurl={http://www.latex-project.org/lppl.txt}}
\hypersetup{pdfcontactaddress={ETH Zurich, ITP, HIT K,
  Wolfgang-Pauli-Strasse 27}}
\hypersetup{pdfcontactpostcode={8093}}
\hypersetup{pdfcontactcity={Zurich}}
\hypersetup{pdfcontactcountry={Switzerland}}
\hypersetup{pdfcontactemail={nbeisert@itp.phys.ethz.ch}}
\hypersetup{pdfcontacturl={http://people.phys.ethz.ch/\xmptilde nbeisert/}}

\newcommand{\secref}[1]{\hyperref[#1]{section \ref*{#1}}}

\parskip1ex
\parindent0pt
\let\olditemize\itemize
\def\itemize{\olditemize\parskip0pt}

\begin{document}

\title{The \textsf{childdoc} Package}
\hypersetup{pdftitle={The childdoc Package}}
\author{Niklas Beisert\\[2ex]
  Institut f\"ur Theoretische Physik\\
  Eidgen\"ossische Technische Hochschule Z\"urich\\
  Wolfgang-Pauli-Strasse 27, 8093 Z\"urich, Switzerland\\[1ex]
  \href{mailto:nbeisert@itp.phys.ethz.ch}
  {\texttt{nbeisert@itp.phys.ethz.ch}}}
\hypersetup{pdfauthor={Niklas Beisert}}
\hypersetup{pdfsubject={Manual for the LaTeX2e Package childdoc}}
\date{30 December 2018, \textsf{v2.0}}
\maketitle

\begin{abstract}\noindent
\textsf{childdoc} is a \LaTeXe{} package
that enables the direct compilation
of document sections included by |\include|
to individual files.
\end{abstract}

\begingroup
\parskip0ex
\tableofcontents
\endgroup

%%%%%%%%%%%%%%%%%%%%%%%%%%%%%%%%%%%%%%%%%%%%%%%%%%%%%%%%%%%%%%%%%%%%%%%%%%%%%%%%
%%%%%%%%%%%%%%%%%%%%%%%%%%%%%%%%%%%%%%%%%%%%%%%%%%%%%%%%%%%%%%%%%%%%%%%%%%%%%%%%
\section{Introduction}

\LaTeX{} provides a mechanism to structure a large document (such as a book)
into a main file and several child files (containing the chapters)
using the |\include| command.
This mechanism is beneficial for documents
which span hundreds of pages in order to
make the source file(s) more manageable.
Moreover, compilation can be restricted to
selected child files by means of the |\includeonly| command.
The latter feature can be used to reduce the compilation time while editing
(this was significantly more useful in the earlier days of \LaTeX{})
or to generate a smaller document which is easier to navigate.
Another application of |\includeonly| is to generate
documents consisting of selected parts of the complete document.

However, there are a few drawbacks of the plain |\include| mechanism:
\begin{itemize}
\item
The child files cannot be compiled on their own,
they can only be compiled via the main file.
A naive editing environment
(such as a text editor with an option
to have the current file processed by \LaTeX)
may require one to switch to the main file before compiling;
attempting to compile the child file produces errors.
\item
The main file must be modified (each time)
to adjust the |\includeonly| command
to the present needs. This easily leaves the main file in a messy state.
\item
The generated document will always carry the filename
of the main document. This is inconvenient if
several child files are to be compiled and
to be kept for distribution.
\end{itemize}

The present package provides a simple interface
to make child files individually compilable by \LaTeX{}.
Compiling a child file then has the same effect as compiling
the main file with an |\includeonly| command
to select the appropriate child.
Moreover the generated document will carry the name of the child
rather than the main file.
This resolves all three above issues.

This feature is meant to make the editing of books,
thesis documents and lecture notes somewhat more convenient.
However, the package can also be used efficiently for
composing a series of documents (such as exercise sheets)
which are typically distributed individually.
It then assists the author in generating the individual documents
(potentially in different versions)
as well as a document containing the collected series.
Another application is in developing style files
or other kinds of included material
where compilation of the style file could redirect
to a sample or test file.

%%%%%%%%%%%%%%%%%%%%%%%%%%%%%%%%%%%%%%%%%%%%%%%%%%%%%%%%%%%%%%%%%%%%%%%%%%%%%%%%
%%%%%%%%%%%%%%%%%%%%%%%%%%%%%%%%%%%%%%%%%%%%%%%%%%%%%%%%%%%%%%%%%%%%%%%%%%%%%%%%
\section{Usage}

First of all, the package \textsf{childdoc} is \emph{not} a standard
\LaTeXe{} |.sty| style file! Therefore it needs to be invoked in
a non-standard way.

%%%%%%%%%%%%%%%%%%%%%%%%%%%%%%%%%%%%%%%%%%%%%%%%%%%%%%%%%%%%%%%%%%%%%%%%%%%%%%%%
\subsection{Included Files}
\label{sec:include}

%%%%%%%%%%%%%%%%%%%%%%%%%%%%%%%%%%%%%%%%
\DescribeMacro{\childdocmain}
To use the package, add the commands
\begin{center}
\begin{tabular}{l}
|\input{childdoc.def}|\\
|\childdocmain{}|\\
\end{tabular}
\end{center}
at the very top of the main \LaTeX{} file,
in particular \emph{before} the |\documentclass| statement!
The argument of |\childdocmain| should be left empty
(but it must be present).

%%%%%%%%%%%%%%%%%%%%%%%%%%%%%%%%%%%%%%%%
\DescribeMacro{\childdocof}
Furthermore, add the commands
\begin{center}
\begin{tabular}{l}
|\input{childdoc.def}|\\
|\childdocof{|\textit{main}|}|\\
\end{tabular}
\end{center}
at the top of every child file \textit{child}
which is included by |\include{|\textit{child}|}|
from within the main file
(or at least for those files to be compiled individually).
The argument \textit{main} must be the filename of the main file.

There are a couple of
considerations in setting up the main and child documents:

%%%%%%%%%%%%%%%%%%%%%%%%%%%%%%%%%%%%%%%%
\paragraph{Restrictions.}

Please note the following restrictions:
\begin{itemize}
\item
|\childdocmain| must be called with one argument \textit{main}
to ensure compatibility with earlier version of the package.
It must either be empty (|\childdocmain{}|)
or precisely match the filename of the main file in which it is specified.
See \secref{sec:detection} for further information.
\item
The filename \textit{main} must be specified without the |.tex| extension.
\item
The filename \textit{main} is case sensitive
(even in case-insensitive file systems)
due to internal string comparison.
\item
The argument \textit{main} should be fully expanded, it cannot be a macro.
\item
Subdirectories and special characters should be avoided in filenames.
\item
The command |\childdocmain{|\textit{main}|}| must be followed by a whitespace.
It should not be followed immediately by another command
or by a comment mark `|%|'.
This is because the \TeX{} parser reads the token immediately following
the argument of |\childdocmain| and puts it
at the beginning of every child section;
however, a white\-space is ignored.
\end{itemize}

%%%%%%%%%%%%%%%%%%%%%%%%%%%%%%%%%%%%%%%%
\paragraph{Content of Main File.}

It is advisable to place all content in the child files included by |\include|.
Any output contained in the main file will appear in all child documents
unless suppressed manually;
it cannot be suppressed automatically by the |\includeonly| directive
and thus should normally be avoided.
A method to include some content in the main file
by means of conditional processing is described in \secref{sec:conditional}.

%%%%%%%%%%%%%%%%%%%%%%%%%%%%%%%%%%%%%%%%
\paragraph{Page Numbering.}

When only a part of the document is compiled,
the appropriate numbering of pages
(as well as other status parameters)
is determined from the |.aux| files.
The latter contain information from previous passes.
However this information needs to propagate through
all intermediate child documents.
Therefore the page numbering in child documents may well
be inconsistent until the complete document is compiled at least once.

A useful (if unconventional) way to always ensure a consistent
page numbering is to restart the numbering in each child document
and denote the pages by `\textit{child}|.|\textit{page}'
where \textit{child} represents the chapter/section number of the child file.
This can be achieved by the command
|\numberwithin{page}{|\textit{child}|}|
of the \textsf{amsmath} package
where \textit{child} can be |chapter| or |section|
depending on the chosen structuring.
Alternatively, one can modify the macro |\thepage| appropriately
and reset the counter |page| at the start of each child file.

%%%%%%%%%%%%%%%%%%%%%%%%%%%%%%%%%%%%%%%%%%%%%%%%%%%%%%%%%%%%%%%%%%%%%%%%%%%%%%%%
\subsection{Conditional Processing}
\label{sec:conditional}

The package provides a mechanism to compile different versions
of a document. To customise the versions further some conditional processing
can come in handy to distinguish which version is being compiled.
The package provides two macros to describe the compilation context:

%%%%%%%%%%%%%%%%%%%%%%%%%%%%%%%%%%%%%%%%
\DescribeMacro{\ifchilddoc}
The conditional |\ifchilddoc| distinguishes between the compilation of
child documents and the main document:
%
\begin{center}
|\ifchilddoc |\textit{child-code}| |[|\||else |\textit{main-code}]| \||fi|
\end{center}

%%%%%%%%%%%%%%%%%%%%%%%%%%%%%%%%%%%%%%%%
\DescribeMacro{\childdocname}
\DescribeMacro{\childdocjob}
The macro |\childdocname| contains the filename (without extension)
of the main or child file being processed.
Note that |\childdocjob| will always contain the name of the main file.

%%%%%%%%%%%%%%%%%%%%%%%%%%%%%%%%%%%%%%%%
\paragraph{Title Page.}

Conditional processing can be used to include a title or banner page
in the main document when proper precautions are taken.
Importantly, the code in the main file should ensure that the page counter
(as well as other status parameters which are stored in the |.aux| files)
takes the same value after the conditional processing.
Otherwise the page numbers may take divergent values
depending on which part is compiled.

For example, a title page could be declared by:
%
\begin{center}
\begin{tabular}{l}
|\ifchilddoc\||else|\\
|\addtocounter{page}{-1}|\\
\textit{code for title page}\\
|\newpage|\\
|\||fi|
\end{tabular}
\end{center}
%
A banner page for the child documents can be generated by:
%
\begin{center}
\begin{tabular}{l}
|\ifchilddoc|\\
|\addtocounter{page}{-1}|\\
\textit{code for banner page}\\
|\newpage|\\
|\||fi|
\end{tabular}
\end{center}
%
Here one could write a message such as:
\begin{center}
|This is the part \childdocname{} of \childdocjob{}.|
\end{center}

%%%%%%%%%%%%%%%%%%%%%%%%%%%%%%%%%%%%%%%%%%%%%%%%%%%%%%%%%%%%%%%%%%%%%%%%%%%%%%%%
\subsection{Flags}
\label{sec:flags}

The package makes it easy to generate different versions
of the main or child documents.
To this end compilation flags can be defined
and assigned different default values.
They will be particularly useful in conjunction
with the forwarding mechanism described in \secref{sec:forward}.

For example, it may be useful to have a flag |\version|
which can be set to |draft| or |final|.
The document source will contain some conditional code
depending on the value of |\version|.
Suppose further, the flag should default to |final| for the main file
and to |draft| for child files
which is a natural assignment for editing the document.
This is achieved by placing the following code
in the preamble of the main document
(below the |\childdocmain| directive):
%
\begin{center}
\begin{tabular}{l}
|\ifchilddoc|\\
|\providecommand{\version}{draft}|\\
|\||else|\\
|\providecommand{\version}{final}|\\
|\||fi|
\end{tabular}
\end{center}
%
The definition by |\providecommand| makes sure
that previous definitions are not overwritten.
Further statements |\providecommand{\version}{...}|
can thus be added before the above code to override it.

For the main file, one might add a line
(between |\childdocmain| and the above block)
%
\begin{center}
|%\ifchilddoc\||else\providecommand{\version}{draft}\||fi|
\end{center}
%
which can be uncommented to produce a draft version.
Likewise one can add a line to the very top of a child file
(above the |\childdocof{|\textit{main}|}| directive)
%
\begin{center}
|%\providecommand{\version}{final}|
\end{center}
%
which can be uncommented to produce the final version of this child document.

%%%%%%%%%%%%%%%%%%%%%%%%%%%%%%%%%%%%%%%%%%%%%%%%%%%%%%%%%%%%%%%%%%%%%%%%%%%%%%%%
\subsection{Forwarding}
\label{sec:forward}

Different versions of the main or child documents
using compilation flags as described in \secref{sec:flags}
can be (permanently) stored in different files
for convenient compilation, viewing and distribution.
To this end, the package defines a command
to pass on compilation to a different file:

%%%%%%%%%%%%%%%%%%%%%%%%%%%%%%%%%%%%%%%%
\DescribeMacro{\childdocforward}
The command |\childdocforward| redirects processing to
another source file:
%
\begin{center}
\begin{tabular}{l}
|\input{childdoc.def}|\\
|\childdocforward[|\textit{main}|]{|\textit{dest}|}|\\
\end{tabular}
\end{center}
%
The argument \textit{dest} is the destination file
(without extension).
It should be the main file or one of the child files.
Note that further \textsf{childdoc} directives
such as |\childdocof| and |\childdocforward|
in the indicated file will be processed in this form.
The optional argument \textit{main}
passes on directly to the main file \textit{main}
while pretending to compile the child \textit{dest}.
This form behaves as if \textit{dest}
issues |\childdocof{|\textit{main}|}| right away,
and no further \textsf{childdoc} directives will be processed.

%%%%%%%%%%%%%%%%%%%%%%%%%%%%%%%%%%%%%%%%
\DescribeMacro{\...prefix}
In the alternative form |\childdocforwardprefix|,
%
\begin{center}
\begin{tabular}{l}
|\input{childdoc.def}|\\
|\childdocforwardprefix[|\textit{main}|]{|\textit{prefix}|}{|\textit{dest}|}|
\end{tabular}
\end{center}
%
the destination file is determined by a pattern
depending on the current file:
To make this work, the current file must be called
`{\textit{prefix}\hspace{0.2em}\textit{suffix}}'
with \textit{prefix} matching precisely the argument.
Processing is then passed on to the file
`{\textit{dest}\hspace{0.2em}\textit{suffix}}'.
Surely, the same effect is achieved by
directly specifying the
argument `{\textit{dest}\hspace{0.2em}\textit{suffix}}'
in the first form.
However, that requires to set up a different file
for each child. With the alternative form of the command
all these files can have exactly the same content
which simplifies setting them up and maintaining them.

For example, the following file |draft.tex|
with a compilation flag |\version| as described in \secref{sec:flags}
compiles the main document as a draft:
%
\begin{center}
\begin{tabular}{l}
|\def\version{draft}|\\
|\input{childdoc.def}|\\
|\childdocforward{|\textit{main}|}|
\end{tabular}
\end{center}
%
Likewise, the following files |final|\textit{nn}|.tex|
compile the final version of the child document
|child|\textit{nn}|.tex|:
%
\begin{center}
\begin{tabular}{l}
|\def\version{final}|\\
|\input{childdoc.def}|\\
|\childdocforwardprefix{final}{child}|
\end{tabular}
\end{center}
%

Note that when several versions of a main file and/or of each child file
are to be generated, it may be convenient to set up a |Makefile| or
shell script to automatise the process.

%%%%%%%%%%%%%%%%%%%%%%%%%%%%%%%%%%%%%%%%%%%%%%%%%%%%%%%%%%%%%%%%%%%%%%%%%%%%%%%%
\subsection{Command Line Processing}
\label{sec:commandline}

The effect of redirection files can also be achieved by invoking
the \LaTeX{} compiler with a more elaborate command line.
Most conveniently this should be done as part
of a shell script or a |Makefile|.

When using \textsf{childdoc} in the main file, the following
command lines effectively perform a redirection
(note that depending on the shell being used,
backslashes may have to be doubled: `|\|' $\to$ `|\\|'):
%
\begin{center}
|... -jobname "|\textit{target}|" |\\|"|[\textit{flags}]%
|\input{childdoc.def}\childdocforward[|\textit{main}|]{|\textit{dest}|}"|
\end{center}
%
Here \textit{target} is the name of the output file,
\textit{main} is the name of the main file
and \textit{dest} is the name of the main or child file to be processed
(all filenames without extensions).
The optional argument \textit{main} can be omitted
if \textit{main} matches \textit{dest}.
Optionally, compilation \textit{flags} can be defined via |\def| commands.
This command line makes the \TeX{} engine believe
it is compiling the file \textit{target}
whose content is specified as the latter parameter.
The provided code then forwards the processing to
\textit{main} or \textit{dest} as described in \secref{sec:forward}.

%%%%%%%%%%%%%%%%%%%%%%%%%%%%%%%%%%%%%%%%%%%%%%%%%%%%%%%%%%%%%%%%%%%%%%%%%%%%%%%%
\subsection{Include by Input}
\label{sec:input}

Including child documents by |\include| has some restrictions by design.
Most notably, the content of a child document always occupies
its own set of pages; pages cannot be shared between child documents.
Usually, this behaviour makes perfect sense
because each child document contain an essential part of the document.
However, in some situations it may be desirable to compose
a document from a collection of parts
without having mandatory page breaks between then.
For this case, the package
provides a mechanism to include parts
by |\input| which can also be processed individually.
However, by construction this mechanism
requires manual handling of the content to be output.

%%%%%%%%%%%%%%%%%%%%%%%%%%%%%%%%%%%%%%%%
\DescribeMacro{\ifchilddocmanual}
The main file should be prepared as usual, see \secref{sec:include}.
However, the document body must make a distinction
between processing of an individual part and of the main document, e.g.:
%
\begin{center}
\begin{tabular}{l}
|\ifchilddocmanual|\\
|\input{\childdocname}|\\
|\||else|\\
\textit{document body with }|\input{|\textit{part}|}|\\
|\||fi|
\end{tabular}
\end{center}
%
The conditional |\ifchilddocmanual| is true whenever
a part to be included by |\input| is being compiled,
and the name of the part is stored in |\childdocname|.

%%%%%%%%%%%%%%%%%%%%%%%%%%%%%%%%%%%%%%%%
\DescribeMacro{\childdocby}
Each part to be included by |\input| should start with:
%
\begin{center}
\begin{tabular}{l}
|\input{childdoc.def}|\\
|\childdocby{|\textit{main}|}|\\
\end{tabular}
\end{center}
%
The directive |\childdocby| is similar to |\childdocof|
described in \secref{sec:include},
but the subsequent selection of content must be done manually.
To that end, both |\ifchilddoc| and |\ifchilddocmanual|
will be true upon processing of a part,
and the name of the part is stored in |\childdocname|.
Note that |\jobname| will be set to the filename of the current part
so that each part receives an individual |.aux| file
that does not interfere with the |.aux| file(s) of the main document.
This behaviour can be altered by the alternative form
|\childdocby[*]{|\textit{main}|}| (with a non-empty optional argument)
which uses the |.aux| file of the main document
by setting |\jobname| to \textit{main}.

%%%%%%%%%%%%%%%%%%%%%%%%%%%%%%%%%%%%%%%%%%%%%%%%%%%%%%%%%%%%%%%%%%%%%%%%%%%%%%%%
\subsection{Driver Development}
\label{sec:driver}

The \textsf{childdoc} mechanism can also be use for the development
of definition files such as \LaTeX{} styles or classes.
This case differs from the above setup with multiple parts
included by |\include| in that no |\includeonly| should be invoked.
This can be achieved by starting the include file
(before |\ProvidesPackage|) with:
%
\begin{center}
\begin{tabular}{l}
|\input{childdoc.def}|\\
|\childdocforward{|\textit{main}|}|\\
\end{tabular}
\end{center}
%
or alternatively with:
%
\begin{center}
\begin{tabular}{l}
|\input{childdoc.def}|\\
|\childdocby{|\textit{main}|}|\\
\end{tabular}
\end{center}
%
Both forms have slightly different effects as described above.
The main file is prepared as usual, see \secref{sec:include}.

%%%%%%%%%%%%%%%%%%%%%%%%%%%%%%%%%%%%%%%%%%%%%%%%%%%%%%%%%%%%%%%%%%%%%%%%%%%%%%%%
\subsection{Legacy Detection}
\label{sec:detection}

The directive |\childdocmain| in the main file can detect
whether the complete document or merely a child is to be compiled
even without using the directive |\childdocof|.
This method is deprecated because it is less robust
and there is no compelling reason to use it;
it is merely provided for backward compatibility
and it may be removed in future versions.

If the detection mechanism is to be used,
it is mandatory to correctly specify
the filename of the main file as the argument of |\childdocmain|:
%
\begin{center}
\begin{tabular}{l}
|\input{childdoc.def}|\\
|\childdocmain{|\textit{main}|}|\\
\end{tabular}
\end{center}
%
If |\jobname| does not match the argument \textit{main} of |\childdocmain|,
it is assumed that |\jobname| points to the child file to be compiled.
When using |\childdocmain| with the main file specified as argument,
it suffices to start a child file
with just |\input{|\textit{main}|}|
without loading of the package and using |\childdocof|.
If instead all processing is done
with the appropriate \textsf{childdoc} directives,
the argument of \textit{main} of |\childdocmain| can be empty.

An alternative version of the command line processing described
in \secref{sec:commandline} using the detection mechanism reads:
%
\begin{center}
|... -jobname "|\textit{target}|" "|[\textit{flags}]%
[|\def\jobname{|\textit{dest}|}|]|\input{|\textit{main}|}"|
\end{center}

%%%%%%%%%%%%%%%%%%%%%%%%%%%%%%%%%%%%%%%%%%%%%%%%%%%%%%%%%%%%%%%%%%%%%%%%%%%%%%%%
\subsection{Manual Code}
\label{sec:manual}

In case one cannot be certain whether the definitions file |childdoc.def|
is installed on the target \TeX{} distribution
and one prefers not to ship it,
it is conceivable to paste a few relevant commands into the sources.

To that end, drop all statements |\input{childdoc.def}|
and perform the replacements as outlined below.
Instead of |\childdocmain{|\textit{main}|}| add the following code
to the top of the main file:
%
\begin{center}
\begin{tabular}{l}
|\||ifdefined\childdocname\endinput\||fi\newif\ifchilddoc|\\
|\edef\childdocname{\scantokens\expandafter{\jobname\noexpand}}|\\
|\def\childdocmain{|\textit{main}|}\||ifx\childdocmain\childdocname\||else|\\
|\childdoctrue\includeonly{\childdocname}\let\jobname\childdocmain\||fi|\\
\end{tabular}
\end{center}
%
Instead of |\childdocof{|\textit{main}|}| just include the main file
at the top of each child file:
%
\begin{center}
|\input{|\textit{main}|}|
\end{center}
%
A simple redirection |\childdocforward{|\textit{dest}|}| is achieved by:
%
\begin{center}
|\def\jobname{|\textit{dest}|}\input{\jobname}|
\end{center}
%
The redirection with prefix
|\childdocforwardprefix[|\textit{prefix}|]{|\textit{dest}|}|
is accomplished by:
%
\begin{center}
\begin{tabular}{l}
|{\edef\jobname{\scantokens\expandafter{\jobname\noexpand}}|\\
|\def\redirectjob |\textit{prefix}|#1~~~{\gdef\jobname{|\textit{dest}|#1}}|\\
|\expandafter\redirectjob\jobname~~~}\input{\jobname}|
\end{tabular}
\end{center}

In an alternative approach,
child documents can be compiled by a specific command line
without additional code or specific definitions:
%
\begin{center}
|... -jobname "|\textit{target}|" "|[\textit{flags}]%
|\includeonly{|\textit{dest}|}\input{|\textit{main}|}"|
\end{center}
%

%%%%%%%%%%%%%%%%%%%%%%%%%%%%%%%%%%%%%%%%%%%%%%%%%%%%%%%%%%%%%%%%%%%%%%%%%%%%%%%%
%%%%%%%%%%%%%%%%%%%%%%%%%%%%%%%%%%%%%%%%%%%%%%%%%%%%%%%%%%%%%%%%%%%%%%%%%%%%%%%%
\section{Information}

%%%%%%%%%%%%%%%%%%%%%%%%%%%%%%%%%%%%%%%%%%%%%%%%%%%%%%%%%%%%%%%%%%%%%%%%%%%%%%%%
\subsection{Copyright}

Copyright \copyright{} 2017--2018 Niklas Beisert

This work may be distributed and/or modified under the
conditions of the \LaTeX{} Project Public License, either version 1.3
of this license or (at your option) any later version.
The latest version of this license is in
  \url{http://www.latex-project.org/lppl.txt}
and version 1.3 or later is part of all distributions of \LaTeX{}
version 2005/12/01 or later.

This work has the LPPL maintenance status `maintained'.

The Current Maintainer of this work is Niklas Beisert.

This work consists of the files |README.txt|, |childdoc.ins| and |childdoc.dtx|
as well as the derived files |childdoc.def|, |cdocsamp.tex|
with |cdocsch1.tex|, |cdocsch2.tex|, |cdocspt3.tex|, |cdocspt4.tex|,
|cdocsdrf.tex|, |cdocsfn1.tex|, |cdocsfn2.tex|
as well as |childdoc.pdf|.

%%%%%%%%%%%%%%%%%%%%%%%%%%%%%%%%%%%%%%%%%%%%%%%%%%%%%%%%%%%%%%%%%%%%%%%%%%%%%%%%
\subsection{Files and Installation}

The package consists of the files:
%
\begin{center}
\begin{tabular}{ll}
    |README.txt|   & readme file \\
    |childdoc.ins| & installation file \\
    |childdoc.dtx| & source file \\
    |childdoc.def| & definition file \\
    |cdocsamp.tex| & sample main file \\
    |cdocsch1.tex| & sample include file \\
    |cdocsch2.tex| & sample include file \\
    |cdocspt3.tex| & sample part file \\
    |cdocspt4.tex| & sample part file \\
    |cdocsdrf.tex| & sample redirection file \\
    |cdocsfn1.tex| & sample redirection file \\
    |cdocsfn2.tex| & sample redirection file \\
    |childdoc.pdf| & manual
\end{tabular}
\end{center}
%
The distribution consists of the files
|README.txt|, |childdoc.ins| and |childdoc.dtx|.
%
\begin{itemize}
\item
Run (pdf)\LaTeX{} on |childdoc.dtx|
to compile the manual |childdoc.pdf| (this file).
\item
Run \LaTeX{} on |childdoc.ins| to create the definitions file |childdoc.def|
and the sample |cdocsamp.tex| with include files
|cdocsch1.tex|, |cdocsch2.tex|, |cdocspt3.tex|, |cdocspt4.tex|,
|cdocsdrf.tex|, |cdocsfn1.tex|, |cdocsfn2.tex|.
Then copy the file |childdoc.def| to an appropriate directory of your \LaTeX{}
distribution, e.g.\ \textit{texmf-root}|/tex/latex/childdoc|.
\end{itemize}

%%%%%%%%%%%%%%%%%%%%%%%%%%%%%%%%%%%%%%%%%%%%%%%%%%%%%%%%%%%%%%%%%%%%%%%%%%%%%%%%
\subsection{Related CTAN Packages}

There are several other packages which offer a similar functionality:
%
\begin{itemize}
\item
The packages
\href{http://ctan.org/pkg/docmute}{\textsf{docmute}},
\href{http://ctan.org/pkg/includex}{\textsf{includex}} and
\href{http://ctan.org/pkg/standalone}{\textsf{standalone}}
provide commands to include only the document body of
a child file thus allowing both files to be compiled individually.
\item
The packages \href{http://ctan.org/pkg/subdocs}{\textsf{subdocs}}
and \href{http://ctan.org/pkg/subfiles}{\textsf{subfiles}}
provide structures in which the main and child documents can be
encapsulated and allowing them to be compiled individually.
The inclusion mechanism is different from the conventional |\include|.
\item
The package \href{http://ctan.org/pkg/combine}{\textsf{combine}}
is an elaborate solution to combine several documents into one.
\end{itemize}
%
See also the CTAN topic \href{http://ctan.org/topic/subdocs}{\textsf{subdocs}}
for further related packages.
The present package differs from the above solutions in that
a document structure constructed with the conventional |\include| mechanism
just needs two extra commands at the top of every file
such that all constituent files can be compiled individually.

%%%%%%%%%%%%%%%%%%%%%%%%%%%%%%%%%%%%%%%%%%%%%%%%%%%%%%%%%%%%%%%%%%%%%%%%%%%%%%%%
%\subsection{Feature Suggestions}
%
%The following is a list of features which may be useful for future
%versions of this package:
%%
%\begin{itemize}
%\item
%\ldots
%\end{itemize}

%%%%%%%%%%%%%%%%%%%%%%%%%%%%%%%%%%%%%%%%%%%%%%%%%%%%%%%%%%%%%%%%%%%%%%%%%%%%%%%%
\subsection{Revision History}

%%%%%%%%%%%%%%%%%%%%%%%%%%%%%%%%%%%%%%%%
\paragraph{v2.0:} 2018/12/30

\begin{itemize}
\item
immediate forward processing
\item
added |\childdocby| mechanism
\item
manual restructured
\end{itemize}

%%%%%%%%%%%%%%%%%%%%%%%%%%%%%%%%%%%%%%%%
\paragraph{v1.6:} 2018/01/17

\begin{itemize}
\item
application for development of include files
\item
corrections to manual
\end{itemize}

%%%%%%%%%%%%%%%%%%%%%%%%%%%%%%%%%%%%%%%%
\paragraph{v1.5:} 2017/05/21

\begin{itemize}
\item
more complete structuring introduced
\item
|\childdocof| introduced
\item
|\childdoc| renamed to |\childdocmain|
\item
|\childredirect| renamed to |\childdocforward| and |\childdocforwardprefix|
and functionality expanded
\end{itemize}

%%%%%%%%%%%%%%%%%%%%%%%%%%%%%%%%%%%%%%%%
\paragraph{v1.0:} 2017/04/27

\begin{itemize}
\item
manual and install package
\item
first version published on CTAN
\end{itemize}

%%%%%%%%%%%%%%%%%%%%%%%%%%%%%%%%%%%%%%%%
\paragraph{v0.6:} 2017/04/26

\begin{itemize}
\item
redirection mechanism added
\end{itemize}

%%%%%%%%%%%%%%%%%%%%%%%%%%%%%%%%%%%%%%%%
\paragraph{v0.5:} 2017/04/26

\begin{itemize}
\item
functionality in definition file
\end{itemize}


%%%%%%%%%%%%%%%%%%%%%%%%%%%%%%%%%%%%%%%%%%%%%%%%%%%%%%%%%%%%%%%%%%%%%%%%%%%%%%%%
%%%%%%%%%%%%%%%%%%%%%%%%%%%%%%%%%%%%%%%%%%%%%%%%%%%%%%%%%%%%%%%%%%%%%%%%%%%%%%%%
%%%%%%%%%%%%%%%%%%%%%%%%%%%%%%%%%%%%%%%%%%%%%%%%%%%%%%%%%%%%%%%%%%%%%%%%%%%%%%%%
\appendix

\settowidth\MacroIndent{\rmfamily\scriptsize 000\ }

 \DocInput{childdoc.dtx}

\end{document}
%</driver>
% \fi
%
% %%%%%%%%%%%%%%%%%%%%%%%%%%%%%%%%%%%%%%%%%%%%%%%%%%%%%%%%%%%%%%%%%%%%%%%%%%%%%%
% %%%%%%%%%%%%%%%%%%%%%%%%%%%%%%%%%%%%%%%%%%%%%%%%%%%%%%%%%%%%%%%%%%%%%%%%%%%%%%
% \section{Sample}
%\iffalse
%<*samplemain>
%\fi
%
% The following presents a sample document
% with two chapters, two parts, a title page,
% a compile flag as well as three forwarding files to set the flag.
% It consists of eight |.tex| files:
% \begin{center}
% \begin{tabular}{ll}
% |cdocsamp.tex|&main file\\
% |cdocsch1.tex|&include file for chapter 1\\
% |cdocsch2.tex|&include file for chapter 2\\
% |cdocspt3.tex|&include file for part 3\\
% |cdocspt4.tex|&include file for part 4\\
% |cdocsdrf.tex|&forwarding file for main file in draft mode\\
% |cdocsfi1.tex|&forwarding file for final version of chapter 1\\
% |cdocsfi2.tex|&forwarding file for final version of chapter 2\\
% \end{tabular}
% \end{center}
% Each of the eight files can be compiled directly by the \LaTeX{} compiler.
%
% %%%%%%%%%%%%%%%%%%%%%%%%%%%%%%%%%%%%%%
% \paragraph{Main File.}
%
% The main file is called |cdocsamp.tex|.
%
% Load the \textsf{childdoc} definitions and
% declare the filename for the main document:
%    \begin{macrocode}
\input{childdoc.def}
\childdocmain{}
%    \end{macrocode}

% Optional override for |\version| flag:
%    \begin{macrocode}
%%\ifchilddoc\else\providecommand{\version}{draft}\fi
%    \end{macrocode}

% Define the default values for the |\version| flag
% (|final| for the main file and |draft| for childs):
%    \begin{macrocode}
\ifchilddoc
\providecommand{\version}{draft}
\else
\providecommand{\version}{final}
\fi
%    \end{macrocode}

% Load the standard document class:
%    \begin{macrocode}
\documentclass[12pt]{article}
%    \end{macrocode}

% Start the document body:
%    \begin{macrocode}
\begin{document}
%    \end{macrocode}

% Declare a title page.
% Print title, part of document being processed and version flag:
%    \begin{macrocode}
\addtocounter{page}{-1}
\begin{center}
{\LARGE\bfseries{}childdoc example\par}
\vspace{1cm}
\ifchilddoc
\ifchilddocmanual part\else chapter\fi:
`\childdocname' of `\childdocjob'\par
\else
main document: `\childdocjob'\par
\fi
version: \version\par
\end{center}
\newpage
%    \end{macrocode}

% Manually include selected file,
% otherwise process as usual:
%    \begin{macrocode}
\ifchilddocmanual
\section*{part `\childdocname'}
\input{\childdocname}
\else
%    \end{macrocode}

% Include the two chapters:
%    \begin{macrocode}
\include{cdocsch1}
\include{cdocsch2}
%    \end{macrocode}

% Include the two parts unless only chapters should be displayed:
%    \begin{macrocode}
\ifchilddoc\else
\section{part three}
\input{cdocspt3}
\section{part four}
\input{cdocspt4}
\fi
%    \end{macrocode}

% Process as usual until here:
%    \begin{macrocode}
\fi
%    \end{macrocode}

% End of document body:
%    \begin{macrocode}
\end{document}
%    \end{macrocode}
%\iffalse
%</samplemain>
%\fi
%
% %%%%%%%%%%%%%%%%%%%%%%%%%%%%%%%%%%%%%%
% \paragraph{Chapter Include Files.}
%
% The include files are called |cdocsch1.tex| and |cdocsch2.tex|.
%
%\iffalse
%<*samplechap1|samplechap2>
%\fi

% Optional override for |\version| flag:
%    \begin{macrocode}
%%\providecommand{\version}{final}
%    \end{macrocode}

% Include the main document:
%    \begin{macrocode}
\input{childdoc.def}
\childdocof{cdocsamp}
%    \end{macrocode}

%\iffalse
%</samplechap1|samplechap2>
%\fi
%
%\iffalse
%<*samplechap1>
%\fi
% Some text for chapter 1:
%    \begin{macrocode}
\section{one}
some text in chapter one
%    \end{macrocode}

%\iffalse
%</samplechap1>
%\fi
% Some text for chapter 2:
%\iffalse
%<*samplechap2>
%\fi
%    \begin{macrocode}
\section{two}
more text in chapter two
%    \end{macrocode}

%\iffalse
%</samplechap2>
%\fi
%
% %%%%%%%%%%%%%%%%%%%%%%%%%%%%%%%%%%%%%%
% \paragraph{Part Include Files.}
%
% The include files are called |cdocspt3.tex| and |cdocspt4.tex|.
%
%\iffalse
%<*samplepart3|samplepart4>
%\fi

% Optional override for |\version| flag:
%    \begin{macrocode}
%%\providecommand{\version}{final}
%    \end{macrocode}

% Include the main document:
%    \begin{macrocode}
\input{childdoc.def}
\childdocby{cdocsamp}
%    \end{macrocode}

%\iffalse
%</samplepart3|samplepart4>
%\fi
%
%\iffalse
%<*samplepart3>
%\fi
% Some text for part 3:
%    \begin{macrocode}
some text in part three
%    \end{macrocode}

%\iffalse
%</samplepart3>
%\fi
% Some text for part 4:
%\iffalse
%<*samplepart4>
%\fi
%    \begin{macrocode}
more text in part four
%    \end{macrocode}

%\iffalse
%</samplepart4>
%\fi
%
% %%%%%%%%%%%%%%%%%%%%%%%%%%%%%%%%%%%%%%
% \paragraph{Forwarding for a Complete Draft.}
%
% The following forwarding file |cdocsdrf.tex|
% compiles the main document in draft mode:
%\iffalse
%<*sampledraft>
%\fi
%    \begin{macrocode}
\def\version{draft}
\input{childdoc.def}
\childdocforward{cdocsamp}
%    \end{macrocode}

%\iffalse
%</sampledraft>
%\fi
%
% %%%%%%%%%%%%%%%%%%%%%%%%%%%%%%%%%%%%%%
% \paragraph{Forwarding for Final Version of the Chapters.}
%
% The following forwarding files |cdocsfn1.tex| and |cdocsfn2.tex|
% (with identical content)
% compile the final versions of the child documents
% |cdocsch1.tex| and |cdocsch2.tex|, respectively:
%\iffalse
%<*samplefinal>
%\fi
%    \begin{macrocode}
\def\version{final}
\input{childdoc.def}
\childdocforwardprefix[cdocsamp]{cdocsfn}{cdocsch}
%    \end{macrocode}

%\iffalse
%</samplefinal>
%\fi
%
% %%%%%%%%%%%%%%%%%%%%%%%%%%%%%%%%%%%%%%
% \paragraph{Command Line Processing.}
%
% The following three command lines generate the output files
% |cdocscld|, |cdocscl1| and |cdocscl2|
% which should be identical to
% |cdocsdrf|, |cdocsch1| and |cdocsfn2|, respectively:
% \begin{center}
% \begin{tabular}{l}
% |latex -jobname cdocscld \|\\
% |  "\def\version{draft}\input{childdoc.def}\childdocforward{cdocsamp}"|\\
% |latex -jobname cdocscl1 \|\\
% |  "\input{childdoc.def}\childdocforward[cdocsamp]{cdocsch1}"|\\
% |latex -jobname cdocscl2 \|\\
% |  "\def\version{final}\input{childdoc.def}\childdocforward{cdocsch2}"|
% \end{tabular}
% \end{center}
% Note that the trailing backslash on each first line
% merely continues the input to the second line
% (for convenient cut ant paste).
% Furthermore, the command |latex| can be replaced by any
% of its alternative versions such as |pdflatex|.
%
% %%%%%%%%%%%%%%%%%%%%%%%%%%%%%%%%%%%%%%%%%%%%%%%%%%%%%%%%%%%%%%%%%%%%%%%%%%%%%%
% %%%%%%%%%%%%%%%%%%%%%%%%%%%%%%%%%%%%%%%%%%%%%%%%%%%%%%%%%%%%%%%%%%%%%%%%%%%%%%
% \section{Implementation}
%\iffalse
%<*package>
%\fi
%
% This section describes the definitions file |childdoc.def|.

% The definitions cannot be loaded using |\usepackage| or |\RequirePackage|
% which has a mechanism to prevent loading a style file more than once.
% When loading the definitions by means of |\input|
% multiple instances have to be prevented manually:
%\iffalse
%This code needs to be before the `\ProvidesFile' directive
%which is defined at the beginning of this file.
%Therefore it is also placed there and commented out here.
%</package>
%<*discard>
%\fi
%    \begin{macrocode}
\ifdefined\childdocmain\endinput\fi
%    \end{macrocode}
%\iffalse
%</discard>
%<*package>
%\fi
%
% \macro{\ifchilddoc}
% \macro{\ifchilddocmanual}
% The conditional |\ifchilddoc| tells whether a
% child (true) or main (false) document is being compiled.
% The conditional |\ifchilddocmanual| tells whether
% the |\includeonly| mechanism is used (false) or
% the selection of child files must be performed manually (true).
% The definitions initialise to false:
%    \begin{macrocode}
\newif\ifchilddoc
\newif\ifchilddocmanual
%    \end{macrocode}

% \macro{\childdocname}
% \macro{\childdocjob}
% The macro |\childdocname| stores the name of the main document
% to be compiled. The macro |\childdocjob| stores the name of
% the document on which the \LaTeX{} compiler was originally invoked.
% The content of |\jobname| cannot be compared
% to filenames specified in the source due to different catcodes.
% The following code rescans |\jobname|, stores the result
% in |\childdocname| and saves a copy in |\childdocjob|:
%    \begin{macrocode}
\edef\childdocname{\scantokens\expandafter{\jobname\noexpand}}
\let\childdocjob\childdocname
%    \end{macrocode}

% \macro{\childdocdisable}
% The macro |\childdocdisable| prevents the main file
% from being processed more than once.
% At this stage, the main document command |\childdocmain|
% is assumed to be called once again where it should do nothing.
% Any subsequent call to it should prevent
% a secondary processing of the main document
% It overwrites the forwarding commands
% |\childdocof| and |\childdocforward|
% with empty macros to prevent further inclusions of the main document:
%    \begin{macrocode}
\newcommand{\childdocdisable}
{
  \renewcommand{\childdocmain}[1]{\renewcommand{\childdocmain}[1]{\endinput}}
  \renewcommand{\childdocof}[1]{}
  \renewcommand{\childdocby}[2][]{}
  \renewcommand{\childdocforward}[2][]{}
  \renewcommand{\childdocdisable}{}
}
%    \end{macrocode}

% \macro{\childdocmain}
% The macro |\childdocmain| is to be called at the top of the main file
% with nothing or the main filename (without extension) as argument.
% First, it breaks loops.
% If the argument is not empty and does not match |\childdocname|
% (which is set by the first inclusion of |childdoc.def|),
% |\ifchilddoc| is set to true, |\includeonly| is applied to the child file
% and |\jobname| is set to the main file
% (for proper handling of |.aux| files):
%    \begin{macrocode}
\newcommand{\childdocmain}[1]
{
  \childdocdisable\childdocmain{}
  \if?#1?\else
    \begingroup
      \def\childdoctmp{#1}
      \ifx\childdoctmp\childdocname
        \def\childdoctmp{}
      \else
        \def\childdoctmp
        {
          \childdoctrue
          \includeonly{\childdocname}
          \def\childdocjob{#1}
          \def\jobname{#1}
        }
      \fi
      \expandafter
    \endgroup
    \childdoctmp
  \fi
}
%    \end{macrocode}

% \macro{\childdocof}
% The command |\childdocof| redirects
% compilation to the main file |#1|.
%    \begin{macrocode}
\newcommand{\childdocof}[1]
{
  \childdocdisable
  \childdoctrue
  \includeonly{\childdocname}
  \def\jobname{#1}
  \def\childdocjob{#1}
  \input{#1}
}
%    \end{macrocode}

% \macro{\childdocby}
% The command |\childdocby| ....
%    \begin{macrocode}
\newcommand{\childdocby}[2][]
{
  \childdocdisable
  \childdoctrue
  \childdocmanualtrue
  \if?#1?\else
    \def\jobname{#2}
  \fi
  \def\childdocjob{#2}
  \input{#2}
  \endinput
}
%    \end{macrocode}

% \macro{\childdocforward}
% The command |\childdocforward| redirects
% compilation to the main file or
% (if the optional argument is given) a child file.
% Parameters are set as if the main file
% or a child file starting with |\childdocof| was compiled.
% Then compilation is handed over to the main file:
%    \begin{macrocode}
\newcommand{\childdocforward}[2][]
{
  \begingroup
    \if?#1?
      \def\childdoctmp
      {
        \def\childdocname{#2}
        \def\childdocjob{#2}
        \def\jobname{#2}
        \input{#2}
        \endinput
      }
    \else
      \def\childdoctmp
      {
        \childdocdisable
        \def\childdocname{#2}
        \childdoctrue
        \includeonly{#2}
        \def\childdocjob{#1}
        \def\jobname{#1}
        \input{#1}
        \endinput
      }
    \fi
    \expandafter
  \endgroup
  \childdoctmp
}
%    \end{macrocode}

% \macro{\childdocforwardprefix}
% The command |\childdocforwardprefix| redirects
% compilation to the main or a child file by means of a pattern.
% The prefix |#1| in the current filename is replaced by |#2|
% and the suffix of the current filename is kept
% (it is assumed that the filename does not contain the substring `|~~~|'
% which is used as a delimiter).
% Compilation is handed over to the new file by |\childdocforward|:
%    \begin{macrocode}
\newcommand{\childdocforwardprefix}[3][]
{
  \begingroup
    \def\childdocextract #2##1~~~{\def\childdoctmp{\childdocforward[#1]{#3##1}}}
    \expandafter\childdocextract\childdocname~~~
    \expandafter
  \endgroup
  \childdoctmp
}
%    \end{macrocode}

% \macro{\childdoc}
% The deprecated macro |\childdoc| is a legacy version of |\childdocmain|:
%    \begin{macrocode}
\newcommand{\childdoc}{\childdocmain}
%    \end{macrocode}

% \macro{\childdocredirect}
% The deprecated macro |\childdocredirect| is a legacy version
% of |\childdocforward| and |\childdocforwardprefix|:
%    \begin{macrocode}
\newcommand{\childdocredirect}[2][]
{
  \begingroup
    \if?#1?
      \def\childdoctmp{\childdocforward{#2}}
    \else
      \def\childdoctmp{\childdocforwardprefix{#1}{#2}}
    \fi
    \expandafter
  \endgroup
  \childdoctmp
}
%    \end{macrocode}

%\iffalse
%</package>
%\fi
%
\endinput

\childdocforwardprefix[cdocsamp]{cdocsfn}{cdocsch}
%    \end{macrocode}

%\iffalse
%</samplefinal>
%\fi
%
% %%%%%%%%%%%%%%%%%%%%%%%%%%%%%%%%%%%%%%
% \paragraph{Command Line Processing.}
%
% The following three command lines generate the output files
% |cdocscld|, |cdocscl1| and |cdocscl2|
% which should be identical to
% |cdocsdrf|, |cdocsch1| and |cdocsfn2|, respectively:
% \begin{center}
% \begin{tabular}{l}
% |latex -jobname cdocscld \|\\
% |  "\def\version{draft}% \iffalse
%
% childdoc.dtx Copyright (C) 2017-2018 Niklas Beisert
%
% This work may be distributed and/or modified under the
% conditions of the LaTeX Project Public License, either version 1.3
% of this license or (at your option) any later version.
% The latest version of this license is in
%   http://www.latex-project.org/lppl.txt
% and version 1.3 or later is part of all distributions of LaTeX
% version 2005/12/01 or later.
%
% This work has the LPPL maintenance status `maintained'.
%
% The Current Maintainer of this work is Niklas Beisert.
%
% This work consists of the files childdoc.dtx and childdoc.ins
% and the derived files childdoc.def and cdocsamp.tex with
% cdocsch1.tex, cdocsch2.tex, cdocsdrf.tex, cdocsfn1.tex, cdocsfn2.tex.
%
%<package>\ifdefined\childdocmain\endinput\fi
%<package>\ProvidesFile{childdoc.def}[2018/12/30 v2.0 child document driver]
%<samplemain>\ProvidesFile{cdocsamp.tex}[2018/12/30 v2.0 sample for childdoc]
%<*driver>
%\ProvidesFile{childdoc.drv}[2018/12/30 v2.0 childdoc reference manual file]
\PassOptionsToClass{10pt,a4paper}{article}
\documentclass{ltxdoc}

\usepackage[margin=35mm]{geometry}
\usepackage{hyperref}
\usepackage{hyperxmp}
\usepackage[usenames]{color}

\hypersetup{colorlinks=true}
\hypersetup{pdfstartview=FitH}
\hypersetup{pdfpagemode=UseNone}
\hypersetup{pdfsource={}}
\hypersetup{pdflang={en-UK}}
\hypersetup{pdfcopyright={Copyright 2017-2018 Niklas Beisert.
  This work may be distributed and/or modified under the
  conditions of the LaTeX Project Public License, either version 1.3
  of this license or (at your option) any later version.}}
\hypersetup{pdflicenseurl={http://www.latex-project.org/lppl.txt}}
\hypersetup{pdfcontactaddress={ETH Zurich, ITP, HIT K,
  Wolfgang-Pauli-Strasse 27}}
\hypersetup{pdfcontactpostcode={8093}}
\hypersetup{pdfcontactcity={Zurich}}
\hypersetup{pdfcontactcountry={Switzerland}}
\hypersetup{pdfcontactemail={nbeisert@itp.phys.ethz.ch}}
\hypersetup{pdfcontacturl={http://people.phys.ethz.ch/\xmptilde nbeisert/}}

\newcommand{\secref}[1]{\hyperref[#1]{section \ref*{#1}}}

\parskip1ex
\parindent0pt
\let\olditemize\itemize
\def\itemize{\olditemize\parskip0pt}

\begin{document}

\title{The \textsf{childdoc} Package}
\hypersetup{pdftitle={The childdoc Package}}
\author{Niklas Beisert\\[2ex]
  Institut f\"ur Theoretische Physik\\
  Eidgen\"ossische Technische Hochschule Z\"urich\\
  Wolfgang-Pauli-Strasse 27, 8093 Z\"urich, Switzerland\\[1ex]
  \href{mailto:nbeisert@itp.phys.ethz.ch}
  {\texttt{nbeisert@itp.phys.ethz.ch}}}
\hypersetup{pdfauthor={Niklas Beisert}}
\hypersetup{pdfsubject={Manual for the LaTeX2e Package childdoc}}
\date{30 December 2018, \textsf{v2.0}}
\maketitle

\begin{abstract}\noindent
\textsf{childdoc} is a \LaTeXe{} package
that enables the direct compilation
of document sections included by |\include|
to individual files.
\end{abstract}

\begingroup
\parskip0ex
\tableofcontents
\endgroup

%%%%%%%%%%%%%%%%%%%%%%%%%%%%%%%%%%%%%%%%%%%%%%%%%%%%%%%%%%%%%%%%%%%%%%%%%%%%%%%%
%%%%%%%%%%%%%%%%%%%%%%%%%%%%%%%%%%%%%%%%%%%%%%%%%%%%%%%%%%%%%%%%%%%%%%%%%%%%%%%%
\section{Introduction}

\LaTeX{} provides a mechanism to structure a large document (such as a book)
into a main file and several child files (containing the chapters)
using the |\include| command.
This mechanism is beneficial for documents
which span hundreds of pages in order to
make the source file(s) more manageable.
Moreover, compilation can be restricted to
selected child files by means of the |\includeonly| command.
The latter feature can be used to reduce the compilation time while editing
(this was significantly more useful in the earlier days of \LaTeX{})
or to generate a smaller document which is easier to navigate.
Another application of |\includeonly| is to generate
documents consisting of selected parts of the complete document.

However, there are a few drawbacks of the plain |\include| mechanism:
\begin{itemize}
\item
The child files cannot be compiled on their own,
they can only be compiled via the main file.
A naive editing environment
(such as a text editor with an option
to have the current file processed by \LaTeX)
may require one to switch to the main file before compiling;
attempting to compile the child file produces errors.
\item
The main file must be modified (each time)
to adjust the |\includeonly| command
to the present needs. This easily leaves the main file in a messy state.
\item
The generated document will always carry the filename
of the main document. This is inconvenient if
several child files are to be compiled and
to be kept for distribution.
\end{itemize}

The present package provides a simple interface
to make child files individually compilable by \LaTeX{}.
Compiling a child file then has the same effect as compiling
the main file with an |\includeonly| command
to select the appropriate child.
Moreover the generated document will carry the name of the child
rather than the main file.
This resolves all three above issues.

This feature is meant to make the editing of books,
thesis documents and lecture notes somewhat more convenient.
However, the package can also be used efficiently for
composing a series of documents (such as exercise sheets)
which are typically distributed individually.
It then assists the author in generating the individual documents
(potentially in different versions)
as well as a document containing the collected series.
Another application is in developing style files
or other kinds of included material
where compilation of the style file could redirect
to a sample or test file.

%%%%%%%%%%%%%%%%%%%%%%%%%%%%%%%%%%%%%%%%%%%%%%%%%%%%%%%%%%%%%%%%%%%%%%%%%%%%%%%%
%%%%%%%%%%%%%%%%%%%%%%%%%%%%%%%%%%%%%%%%%%%%%%%%%%%%%%%%%%%%%%%%%%%%%%%%%%%%%%%%
\section{Usage}

First of all, the package \textsf{childdoc} is \emph{not} a standard
\LaTeXe{} |.sty| style file! Therefore it needs to be invoked in
a non-standard way.

%%%%%%%%%%%%%%%%%%%%%%%%%%%%%%%%%%%%%%%%%%%%%%%%%%%%%%%%%%%%%%%%%%%%%%%%%%%%%%%%
\subsection{Included Files}
\label{sec:include}

%%%%%%%%%%%%%%%%%%%%%%%%%%%%%%%%%%%%%%%%
\DescribeMacro{\childdocmain}
To use the package, add the commands
\begin{center}
\begin{tabular}{l}
|\input{childdoc.def}|\\
|\childdocmain{}|\\
\end{tabular}
\end{center}
at the very top of the main \LaTeX{} file,
in particular \emph{before} the |\documentclass| statement!
The argument of |\childdocmain| should be left empty
(but it must be present).

%%%%%%%%%%%%%%%%%%%%%%%%%%%%%%%%%%%%%%%%
\DescribeMacro{\childdocof}
Furthermore, add the commands
\begin{center}
\begin{tabular}{l}
|\input{childdoc.def}|\\
|\childdocof{|\textit{main}|}|\\
\end{tabular}
\end{center}
at the top of every child file \textit{child}
which is included by |\include{|\textit{child}|}|
from within the main file
(or at least for those files to be compiled individually).
The argument \textit{main} must be the filename of the main file.

There are a couple of
considerations in setting up the main and child documents:

%%%%%%%%%%%%%%%%%%%%%%%%%%%%%%%%%%%%%%%%
\paragraph{Restrictions.}

Please note the following restrictions:
\begin{itemize}
\item
|\childdocmain| must be called with one argument \textit{main}
to ensure compatibility with earlier version of the package.
It must either be empty (|\childdocmain{}|)
or precisely match the filename of the main file in which it is specified.
See \secref{sec:detection} for further information.
\item
The filename \textit{main} must be specified without the |.tex| extension.
\item
The filename \textit{main} is case sensitive
(even in case-insensitive file systems)
due to internal string comparison.
\item
The argument \textit{main} should be fully expanded, it cannot be a macro.
\item
Subdirectories and special characters should be avoided in filenames.
\item
The command |\childdocmain{|\textit{main}|}| must be followed by a whitespace.
It should not be followed immediately by another command
or by a comment mark `|%|'.
This is because the \TeX{} parser reads the token immediately following
the argument of |\childdocmain| and puts it
at the beginning of every child section;
however, a white\-space is ignored.
\end{itemize}

%%%%%%%%%%%%%%%%%%%%%%%%%%%%%%%%%%%%%%%%
\paragraph{Content of Main File.}

It is advisable to place all content in the child files included by |\include|.
Any output contained in the main file will appear in all child documents
unless suppressed manually;
it cannot be suppressed automatically by the |\includeonly| directive
and thus should normally be avoided.
A method to include some content in the main file
by means of conditional processing is described in \secref{sec:conditional}.

%%%%%%%%%%%%%%%%%%%%%%%%%%%%%%%%%%%%%%%%
\paragraph{Page Numbering.}

When only a part of the document is compiled,
the appropriate numbering of pages
(as well as other status parameters)
is determined from the |.aux| files.
The latter contain information from previous passes.
However this information needs to propagate through
all intermediate child documents.
Therefore the page numbering in child documents may well
be inconsistent until the complete document is compiled at least once.

A useful (if unconventional) way to always ensure a consistent
page numbering is to restart the numbering in each child document
and denote the pages by `\textit{child}|.|\textit{page}'
where \textit{child} represents the chapter/section number of the child file.
This can be achieved by the command
|\numberwithin{page}{|\textit{child}|}|
of the \textsf{amsmath} package
where \textit{child} can be |chapter| or |section|
depending on the chosen structuring.
Alternatively, one can modify the macro |\thepage| appropriately
and reset the counter |page| at the start of each child file.

%%%%%%%%%%%%%%%%%%%%%%%%%%%%%%%%%%%%%%%%%%%%%%%%%%%%%%%%%%%%%%%%%%%%%%%%%%%%%%%%
\subsection{Conditional Processing}
\label{sec:conditional}

The package provides a mechanism to compile different versions
of a document. To customise the versions further some conditional processing
can come in handy to distinguish which version is being compiled.
The package provides two macros to describe the compilation context:

%%%%%%%%%%%%%%%%%%%%%%%%%%%%%%%%%%%%%%%%
\DescribeMacro{\ifchilddoc}
The conditional |\ifchilddoc| distinguishes between the compilation of
child documents and the main document:
%
\begin{center}
|\ifchilddoc |\textit{child-code}| |[|\||else |\textit{main-code}]| \||fi|
\end{center}

%%%%%%%%%%%%%%%%%%%%%%%%%%%%%%%%%%%%%%%%
\DescribeMacro{\childdocname}
\DescribeMacro{\childdocjob}
The macro |\childdocname| contains the filename (without extension)
of the main or child file being processed.
Note that |\childdocjob| will always contain the name of the main file.

%%%%%%%%%%%%%%%%%%%%%%%%%%%%%%%%%%%%%%%%
\paragraph{Title Page.}

Conditional processing can be used to include a title or banner page
in the main document when proper precautions are taken.
Importantly, the code in the main file should ensure that the page counter
(as well as other status parameters which are stored in the |.aux| files)
takes the same value after the conditional processing.
Otherwise the page numbers may take divergent values
depending on which part is compiled.

For example, a title page could be declared by:
%
\begin{center}
\begin{tabular}{l}
|\ifchilddoc\||else|\\
|\addtocounter{page}{-1}|\\
\textit{code for title page}\\
|\newpage|\\
|\||fi|
\end{tabular}
\end{center}
%
A banner page for the child documents can be generated by:
%
\begin{center}
\begin{tabular}{l}
|\ifchilddoc|\\
|\addtocounter{page}{-1}|\\
\textit{code for banner page}\\
|\newpage|\\
|\||fi|
\end{tabular}
\end{center}
%
Here one could write a message such as:
\begin{center}
|This is the part \childdocname{} of \childdocjob{}.|
\end{center}

%%%%%%%%%%%%%%%%%%%%%%%%%%%%%%%%%%%%%%%%%%%%%%%%%%%%%%%%%%%%%%%%%%%%%%%%%%%%%%%%
\subsection{Flags}
\label{sec:flags}

The package makes it easy to generate different versions
of the main or child documents.
To this end compilation flags can be defined
and assigned different default values.
They will be particularly useful in conjunction
with the forwarding mechanism described in \secref{sec:forward}.

For example, it may be useful to have a flag |\version|
which can be set to |draft| or |final|.
The document source will contain some conditional code
depending on the value of |\version|.
Suppose further, the flag should default to |final| for the main file
and to |draft| for child files
which is a natural assignment for editing the document.
This is achieved by placing the following code
in the preamble of the main document
(below the |\childdocmain| directive):
%
\begin{center}
\begin{tabular}{l}
|\ifchilddoc|\\
|\providecommand{\version}{draft}|\\
|\||else|\\
|\providecommand{\version}{final}|\\
|\||fi|
\end{tabular}
\end{center}
%
The definition by |\providecommand| makes sure
that previous definitions are not overwritten.
Further statements |\providecommand{\version}{...}|
can thus be added before the above code to override it.

For the main file, one might add a line
(between |\childdocmain| and the above block)
%
\begin{center}
|%\ifchilddoc\||else\providecommand{\version}{draft}\||fi|
\end{center}
%
which can be uncommented to produce a draft version.
Likewise one can add a line to the very top of a child file
(above the |\childdocof{|\textit{main}|}| directive)
%
\begin{center}
|%\providecommand{\version}{final}|
\end{center}
%
which can be uncommented to produce the final version of this child document.

%%%%%%%%%%%%%%%%%%%%%%%%%%%%%%%%%%%%%%%%%%%%%%%%%%%%%%%%%%%%%%%%%%%%%%%%%%%%%%%%
\subsection{Forwarding}
\label{sec:forward}

Different versions of the main or child documents
using compilation flags as described in \secref{sec:flags}
can be (permanently) stored in different files
for convenient compilation, viewing and distribution.
To this end, the package defines a command
to pass on compilation to a different file:

%%%%%%%%%%%%%%%%%%%%%%%%%%%%%%%%%%%%%%%%
\DescribeMacro{\childdocforward}
The command |\childdocforward| redirects processing to
another source file:
%
\begin{center}
\begin{tabular}{l}
|\input{childdoc.def}|\\
|\childdocforward[|\textit{main}|]{|\textit{dest}|}|\\
\end{tabular}
\end{center}
%
The argument \textit{dest} is the destination file
(without extension).
It should be the main file or one of the child files.
Note that further \textsf{childdoc} directives
such as |\childdocof| and |\childdocforward|
in the indicated file will be processed in this form.
The optional argument \textit{main}
passes on directly to the main file \textit{main}
while pretending to compile the child \textit{dest}.
This form behaves as if \textit{dest}
issues |\childdocof{|\textit{main}|}| right away,
and no further \textsf{childdoc} directives will be processed.

%%%%%%%%%%%%%%%%%%%%%%%%%%%%%%%%%%%%%%%%
\DescribeMacro{\...prefix}
In the alternative form |\childdocforwardprefix|,
%
\begin{center}
\begin{tabular}{l}
|\input{childdoc.def}|\\
|\childdocforwardprefix[|\textit{main}|]{|\textit{prefix}|}{|\textit{dest}|}|
\end{tabular}
\end{center}
%
the destination file is determined by a pattern
depending on the current file:
To make this work, the current file must be called
`{\textit{prefix}\hspace{0.2em}\textit{suffix}}'
with \textit{prefix} matching precisely the argument.
Processing is then passed on to the file
`{\textit{dest}\hspace{0.2em}\textit{suffix}}'.
Surely, the same effect is achieved by
directly specifying the
argument `{\textit{dest}\hspace{0.2em}\textit{suffix}}'
in the first form.
However, that requires to set up a different file
for each child. With the alternative form of the command
all these files can have exactly the same content
which simplifies setting them up and maintaining them.

For example, the following file |draft.tex|
with a compilation flag |\version| as described in \secref{sec:flags}
compiles the main document as a draft:
%
\begin{center}
\begin{tabular}{l}
|\def\version{draft}|\\
|\input{childdoc.def}|\\
|\childdocforward{|\textit{main}|}|
\end{tabular}
\end{center}
%
Likewise, the following files |final|\textit{nn}|.tex|
compile the final version of the child document
|child|\textit{nn}|.tex|:
%
\begin{center}
\begin{tabular}{l}
|\def\version{final}|\\
|\input{childdoc.def}|\\
|\childdocforwardprefix{final}{child}|
\end{tabular}
\end{center}
%

Note that when several versions of a main file and/or of each child file
are to be generated, it may be convenient to set up a |Makefile| or
shell script to automatise the process.

%%%%%%%%%%%%%%%%%%%%%%%%%%%%%%%%%%%%%%%%%%%%%%%%%%%%%%%%%%%%%%%%%%%%%%%%%%%%%%%%
\subsection{Command Line Processing}
\label{sec:commandline}

The effect of redirection files can also be achieved by invoking
the \LaTeX{} compiler with a more elaborate command line.
Most conveniently this should be done as part
of a shell script or a |Makefile|.

When using \textsf{childdoc} in the main file, the following
command lines effectively perform a redirection
(note that depending on the shell being used,
backslashes may have to be doubled: `|\|' $\to$ `|\\|'):
%
\begin{center}
|... -jobname "|\textit{target}|" |\\|"|[\textit{flags}]%
|\input{childdoc.def}\childdocforward[|\textit{main}|]{|\textit{dest}|}"|
\end{center}
%
Here \textit{target} is the name of the output file,
\textit{main} is the name of the main file
and \textit{dest} is the name of the main or child file to be processed
(all filenames without extensions).
The optional argument \textit{main} can be omitted
if \textit{main} matches \textit{dest}.
Optionally, compilation \textit{flags} can be defined via |\def| commands.
This command line makes the \TeX{} engine believe
it is compiling the file \textit{target}
whose content is specified as the latter parameter.
The provided code then forwards the processing to
\textit{main} or \textit{dest} as described in \secref{sec:forward}.

%%%%%%%%%%%%%%%%%%%%%%%%%%%%%%%%%%%%%%%%%%%%%%%%%%%%%%%%%%%%%%%%%%%%%%%%%%%%%%%%
\subsection{Include by Input}
\label{sec:input}

Including child documents by |\include| has some restrictions by design.
Most notably, the content of a child document always occupies
its own set of pages; pages cannot be shared between child documents.
Usually, this behaviour makes perfect sense
because each child document contain an essential part of the document.
However, in some situations it may be desirable to compose
a document from a collection of parts
without having mandatory page breaks between then.
For this case, the package
provides a mechanism to include parts
by |\input| which can also be processed individually.
However, by construction this mechanism
requires manual handling of the content to be output.

%%%%%%%%%%%%%%%%%%%%%%%%%%%%%%%%%%%%%%%%
\DescribeMacro{\ifchilddocmanual}
The main file should be prepared as usual, see \secref{sec:include}.
However, the document body must make a distinction
between processing of an individual part and of the main document, e.g.:
%
\begin{center}
\begin{tabular}{l}
|\ifchilddocmanual|\\
|\input{\childdocname}|\\
|\||else|\\
\textit{document body with }|\input{|\textit{part}|}|\\
|\||fi|
\end{tabular}
\end{center}
%
The conditional |\ifchilddocmanual| is true whenever
a part to be included by |\input| is being compiled,
and the name of the part is stored in |\childdocname|.

%%%%%%%%%%%%%%%%%%%%%%%%%%%%%%%%%%%%%%%%
\DescribeMacro{\childdocby}
Each part to be included by |\input| should start with:
%
\begin{center}
\begin{tabular}{l}
|\input{childdoc.def}|\\
|\childdocby{|\textit{main}|}|\\
\end{tabular}
\end{center}
%
The directive |\childdocby| is similar to |\childdocof|
described in \secref{sec:include},
but the subsequent selection of content must be done manually.
To that end, both |\ifchilddoc| and |\ifchilddocmanual|
will be true upon processing of a part,
and the name of the part is stored in |\childdocname|.
Note that |\jobname| will be set to the filename of the current part
so that each part receives an individual |.aux| file
that does not interfere with the |.aux| file(s) of the main document.
This behaviour can be altered by the alternative form
|\childdocby[*]{|\textit{main}|}| (with a non-empty optional argument)
which uses the |.aux| file of the main document
by setting |\jobname| to \textit{main}.

%%%%%%%%%%%%%%%%%%%%%%%%%%%%%%%%%%%%%%%%%%%%%%%%%%%%%%%%%%%%%%%%%%%%%%%%%%%%%%%%
\subsection{Driver Development}
\label{sec:driver}

The \textsf{childdoc} mechanism can also be use for the development
of definition files such as \LaTeX{} styles or classes.
This case differs from the above setup with multiple parts
included by |\include| in that no |\includeonly| should be invoked.
This can be achieved by starting the include file
(before |\ProvidesPackage|) with:
%
\begin{center}
\begin{tabular}{l}
|\input{childdoc.def}|\\
|\childdocforward{|\textit{main}|}|\\
\end{tabular}
\end{center}
%
or alternatively with:
%
\begin{center}
\begin{tabular}{l}
|\input{childdoc.def}|\\
|\childdocby{|\textit{main}|}|\\
\end{tabular}
\end{center}
%
Both forms have slightly different effects as described above.
The main file is prepared as usual, see \secref{sec:include}.

%%%%%%%%%%%%%%%%%%%%%%%%%%%%%%%%%%%%%%%%%%%%%%%%%%%%%%%%%%%%%%%%%%%%%%%%%%%%%%%%
\subsection{Legacy Detection}
\label{sec:detection}

The directive |\childdocmain| in the main file can detect
whether the complete document or merely a child is to be compiled
even without using the directive |\childdocof|.
This method is deprecated because it is less robust
and there is no compelling reason to use it;
it is merely provided for backward compatibility
and it may be removed in future versions.

If the detection mechanism is to be used,
it is mandatory to correctly specify
the filename of the main file as the argument of |\childdocmain|:
%
\begin{center}
\begin{tabular}{l}
|\input{childdoc.def}|\\
|\childdocmain{|\textit{main}|}|\\
\end{tabular}
\end{center}
%
If |\jobname| does not match the argument \textit{main} of |\childdocmain|,
it is assumed that |\jobname| points to the child file to be compiled.
When using |\childdocmain| with the main file specified as argument,
it suffices to start a child file
with just |\input{|\textit{main}|}|
without loading of the package and using |\childdocof|.
If instead all processing is done
with the appropriate \textsf{childdoc} directives,
the argument of \textit{main} of |\childdocmain| can be empty.

An alternative version of the command line processing described
in \secref{sec:commandline} using the detection mechanism reads:
%
\begin{center}
|... -jobname "|\textit{target}|" "|[\textit{flags}]%
[|\def\jobname{|\textit{dest}|}|]|\input{|\textit{main}|}"|
\end{center}

%%%%%%%%%%%%%%%%%%%%%%%%%%%%%%%%%%%%%%%%%%%%%%%%%%%%%%%%%%%%%%%%%%%%%%%%%%%%%%%%
\subsection{Manual Code}
\label{sec:manual}

In case one cannot be certain whether the definitions file |childdoc.def|
is installed on the target \TeX{} distribution
and one prefers not to ship it,
it is conceivable to paste a few relevant commands into the sources.

To that end, drop all statements |\input{childdoc.def}|
and perform the replacements as outlined below.
Instead of |\childdocmain{|\textit{main}|}| add the following code
to the top of the main file:
%
\begin{center}
\begin{tabular}{l}
|\||ifdefined\childdocname\endinput\||fi\newif\ifchilddoc|\\
|\edef\childdocname{\scantokens\expandafter{\jobname\noexpand}}|\\
|\def\childdocmain{|\textit{main}|}\||ifx\childdocmain\childdocname\||else|\\
|\childdoctrue\includeonly{\childdocname}\let\jobname\childdocmain\||fi|\\
\end{tabular}
\end{center}
%
Instead of |\childdocof{|\textit{main}|}| just include the main file
at the top of each child file:
%
\begin{center}
|\input{|\textit{main}|}|
\end{center}
%
A simple redirection |\childdocforward{|\textit{dest}|}| is achieved by:
%
\begin{center}
|\def\jobname{|\textit{dest}|}\input{\jobname}|
\end{center}
%
The redirection with prefix
|\childdocforwardprefix[|\textit{prefix}|]{|\textit{dest}|}|
is accomplished by:
%
\begin{center}
\begin{tabular}{l}
|{\edef\jobname{\scantokens\expandafter{\jobname\noexpand}}|\\
|\def\redirectjob |\textit{prefix}|#1~~~{\gdef\jobname{|\textit{dest}|#1}}|\\
|\expandafter\redirectjob\jobname~~~}\input{\jobname}|
\end{tabular}
\end{center}

In an alternative approach,
child documents can be compiled by a specific command line
without additional code or specific definitions:
%
\begin{center}
|... -jobname "|\textit{target}|" "|[\textit{flags}]%
|\includeonly{|\textit{dest}|}\input{|\textit{main}|}"|
\end{center}
%

%%%%%%%%%%%%%%%%%%%%%%%%%%%%%%%%%%%%%%%%%%%%%%%%%%%%%%%%%%%%%%%%%%%%%%%%%%%%%%%%
%%%%%%%%%%%%%%%%%%%%%%%%%%%%%%%%%%%%%%%%%%%%%%%%%%%%%%%%%%%%%%%%%%%%%%%%%%%%%%%%
\section{Information}

%%%%%%%%%%%%%%%%%%%%%%%%%%%%%%%%%%%%%%%%%%%%%%%%%%%%%%%%%%%%%%%%%%%%%%%%%%%%%%%%
\subsection{Copyright}

Copyright \copyright{} 2017--2018 Niklas Beisert

This work may be distributed and/or modified under the
conditions of the \LaTeX{} Project Public License, either version 1.3
of this license or (at your option) any later version.
The latest version of this license is in
  \url{http://www.latex-project.org/lppl.txt}
and version 1.3 or later is part of all distributions of \LaTeX{}
version 2005/12/01 or later.

This work has the LPPL maintenance status `maintained'.

The Current Maintainer of this work is Niklas Beisert.

This work consists of the files |README.txt|, |childdoc.ins| and |childdoc.dtx|
as well as the derived files |childdoc.def|, |cdocsamp.tex|
with |cdocsch1.tex|, |cdocsch2.tex|, |cdocspt3.tex|, |cdocspt4.tex|,
|cdocsdrf.tex|, |cdocsfn1.tex|, |cdocsfn2.tex|
as well as |childdoc.pdf|.

%%%%%%%%%%%%%%%%%%%%%%%%%%%%%%%%%%%%%%%%%%%%%%%%%%%%%%%%%%%%%%%%%%%%%%%%%%%%%%%%
\subsection{Files and Installation}

The package consists of the files:
%
\begin{center}
\begin{tabular}{ll}
    |README.txt|   & readme file \\
    |childdoc.ins| & installation file \\
    |childdoc.dtx| & source file \\
    |childdoc.def| & definition file \\
    |cdocsamp.tex| & sample main file \\
    |cdocsch1.tex| & sample include file \\
    |cdocsch2.tex| & sample include file \\
    |cdocspt3.tex| & sample part file \\
    |cdocspt4.tex| & sample part file \\
    |cdocsdrf.tex| & sample redirection file \\
    |cdocsfn1.tex| & sample redirection file \\
    |cdocsfn2.tex| & sample redirection file \\
    |childdoc.pdf| & manual
\end{tabular}
\end{center}
%
The distribution consists of the files
|README.txt|, |childdoc.ins| and |childdoc.dtx|.
%
\begin{itemize}
\item
Run (pdf)\LaTeX{} on |childdoc.dtx|
to compile the manual |childdoc.pdf| (this file).
\item
Run \LaTeX{} on |childdoc.ins| to create the definitions file |childdoc.def|
and the sample |cdocsamp.tex| with include files
|cdocsch1.tex|, |cdocsch2.tex|, |cdocspt3.tex|, |cdocspt4.tex|,
|cdocsdrf.tex|, |cdocsfn1.tex|, |cdocsfn2.tex|.
Then copy the file |childdoc.def| to an appropriate directory of your \LaTeX{}
distribution, e.g.\ \textit{texmf-root}|/tex/latex/childdoc|.
\end{itemize}

%%%%%%%%%%%%%%%%%%%%%%%%%%%%%%%%%%%%%%%%%%%%%%%%%%%%%%%%%%%%%%%%%%%%%%%%%%%%%%%%
\subsection{Related CTAN Packages}

There are several other packages which offer a similar functionality:
%
\begin{itemize}
\item
The packages
\href{http://ctan.org/pkg/docmute}{\textsf{docmute}},
\href{http://ctan.org/pkg/includex}{\textsf{includex}} and
\href{http://ctan.org/pkg/standalone}{\textsf{standalone}}
provide commands to include only the document body of
a child file thus allowing both files to be compiled individually.
\item
The packages \href{http://ctan.org/pkg/subdocs}{\textsf{subdocs}}
and \href{http://ctan.org/pkg/subfiles}{\textsf{subfiles}}
provide structures in which the main and child documents can be
encapsulated and allowing them to be compiled individually.
The inclusion mechanism is different from the conventional |\include|.
\item
The package \href{http://ctan.org/pkg/combine}{\textsf{combine}}
is an elaborate solution to combine several documents into one.
\end{itemize}
%
See also the CTAN topic \href{http://ctan.org/topic/subdocs}{\textsf{subdocs}}
for further related packages.
The present package differs from the above solutions in that
a document structure constructed with the conventional |\include| mechanism
just needs two extra commands at the top of every file
such that all constituent files can be compiled individually.

%%%%%%%%%%%%%%%%%%%%%%%%%%%%%%%%%%%%%%%%%%%%%%%%%%%%%%%%%%%%%%%%%%%%%%%%%%%%%%%%
%\subsection{Feature Suggestions}
%
%The following is a list of features which may be useful for future
%versions of this package:
%%
%\begin{itemize}
%\item
%\ldots
%\end{itemize}

%%%%%%%%%%%%%%%%%%%%%%%%%%%%%%%%%%%%%%%%%%%%%%%%%%%%%%%%%%%%%%%%%%%%%%%%%%%%%%%%
\subsection{Revision History}

%%%%%%%%%%%%%%%%%%%%%%%%%%%%%%%%%%%%%%%%
\paragraph{v2.0:} 2018/12/30

\begin{itemize}
\item
immediate forward processing
\item
added |\childdocby| mechanism
\item
manual restructured
\end{itemize}

%%%%%%%%%%%%%%%%%%%%%%%%%%%%%%%%%%%%%%%%
\paragraph{v1.6:} 2018/01/17

\begin{itemize}
\item
application for development of include files
\item
corrections to manual
\end{itemize}

%%%%%%%%%%%%%%%%%%%%%%%%%%%%%%%%%%%%%%%%
\paragraph{v1.5:} 2017/05/21

\begin{itemize}
\item
more complete structuring introduced
\item
|\childdocof| introduced
\item
|\childdoc| renamed to |\childdocmain|
\item
|\childredirect| renamed to |\childdocforward| and |\childdocforwardprefix|
and functionality expanded
\end{itemize}

%%%%%%%%%%%%%%%%%%%%%%%%%%%%%%%%%%%%%%%%
\paragraph{v1.0:} 2017/04/27

\begin{itemize}
\item
manual and install package
\item
first version published on CTAN
\end{itemize}

%%%%%%%%%%%%%%%%%%%%%%%%%%%%%%%%%%%%%%%%
\paragraph{v0.6:} 2017/04/26

\begin{itemize}
\item
redirection mechanism added
\end{itemize}

%%%%%%%%%%%%%%%%%%%%%%%%%%%%%%%%%%%%%%%%
\paragraph{v0.5:} 2017/04/26

\begin{itemize}
\item
functionality in definition file
\end{itemize}


%%%%%%%%%%%%%%%%%%%%%%%%%%%%%%%%%%%%%%%%%%%%%%%%%%%%%%%%%%%%%%%%%%%%%%%%%%%%%%%%
%%%%%%%%%%%%%%%%%%%%%%%%%%%%%%%%%%%%%%%%%%%%%%%%%%%%%%%%%%%%%%%%%%%%%%%%%%%%%%%%
%%%%%%%%%%%%%%%%%%%%%%%%%%%%%%%%%%%%%%%%%%%%%%%%%%%%%%%%%%%%%%%%%%%%%%%%%%%%%%%%
\appendix

\settowidth\MacroIndent{\rmfamily\scriptsize 000\ }

 \DocInput{childdoc.dtx}

\end{document}
%</driver>
% \fi
%
% %%%%%%%%%%%%%%%%%%%%%%%%%%%%%%%%%%%%%%%%%%%%%%%%%%%%%%%%%%%%%%%%%%%%%%%%%%%%%%
% %%%%%%%%%%%%%%%%%%%%%%%%%%%%%%%%%%%%%%%%%%%%%%%%%%%%%%%%%%%%%%%%%%%%%%%%%%%%%%
% \section{Sample}
%\iffalse
%<*samplemain>
%\fi
%
% The following presents a sample document
% with two chapters, two parts, a title page,
% a compile flag as well as three forwarding files to set the flag.
% It consists of eight |.tex| files:
% \begin{center}
% \begin{tabular}{ll}
% |cdocsamp.tex|&main file\\
% |cdocsch1.tex|&include file for chapter 1\\
% |cdocsch2.tex|&include file for chapter 2\\
% |cdocspt3.tex|&include file for part 3\\
% |cdocspt4.tex|&include file for part 4\\
% |cdocsdrf.tex|&forwarding file for main file in draft mode\\
% |cdocsfi1.tex|&forwarding file for final version of chapter 1\\
% |cdocsfi2.tex|&forwarding file for final version of chapter 2\\
% \end{tabular}
% \end{center}
% Each of the eight files can be compiled directly by the \LaTeX{} compiler.
%
% %%%%%%%%%%%%%%%%%%%%%%%%%%%%%%%%%%%%%%
% \paragraph{Main File.}
%
% The main file is called |cdocsamp.tex|.
%
% Load the \textsf{childdoc} definitions and
% declare the filename for the main document:
%    \begin{macrocode}
\input{childdoc.def}
\childdocmain{}
%    \end{macrocode}

% Optional override for |\version| flag:
%    \begin{macrocode}
%%\ifchilddoc\else\providecommand{\version}{draft}\fi
%    \end{macrocode}

% Define the default values for the |\version| flag
% (|final| for the main file and |draft| for childs):
%    \begin{macrocode}
\ifchilddoc
\providecommand{\version}{draft}
\else
\providecommand{\version}{final}
\fi
%    \end{macrocode}

% Load the standard document class:
%    \begin{macrocode}
\documentclass[12pt]{article}
%    \end{macrocode}

% Start the document body:
%    \begin{macrocode}
\begin{document}
%    \end{macrocode}

% Declare a title page.
% Print title, part of document being processed and version flag:
%    \begin{macrocode}
\addtocounter{page}{-1}
\begin{center}
{\LARGE\bfseries{}childdoc example\par}
\vspace{1cm}
\ifchilddoc
\ifchilddocmanual part\else chapter\fi:
`\childdocname' of `\childdocjob'\par
\else
main document: `\childdocjob'\par
\fi
version: \version\par
\end{center}
\newpage
%    \end{macrocode}

% Manually include selected file,
% otherwise process as usual:
%    \begin{macrocode}
\ifchilddocmanual
\section*{part `\childdocname'}
\input{\childdocname}
\else
%    \end{macrocode}

% Include the two chapters:
%    \begin{macrocode}
\include{cdocsch1}
\include{cdocsch2}
%    \end{macrocode}

% Include the two parts unless only chapters should be displayed:
%    \begin{macrocode}
\ifchilddoc\else
\section{part three}
\input{cdocspt3}
\section{part four}
\input{cdocspt4}
\fi
%    \end{macrocode}

% Process as usual until here:
%    \begin{macrocode}
\fi
%    \end{macrocode}

% End of document body:
%    \begin{macrocode}
\end{document}
%    \end{macrocode}
%\iffalse
%</samplemain>
%\fi
%
% %%%%%%%%%%%%%%%%%%%%%%%%%%%%%%%%%%%%%%
% \paragraph{Chapter Include Files.}
%
% The include files are called |cdocsch1.tex| and |cdocsch2.tex|.
%
%\iffalse
%<*samplechap1|samplechap2>
%\fi

% Optional override for |\version| flag:
%    \begin{macrocode}
%%\providecommand{\version}{final}
%    \end{macrocode}

% Include the main document:
%    \begin{macrocode}
\input{childdoc.def}
\childdocof{cdocsamp}
%    \end{macrocode}

%\iffalse
%</samplechap1|samplechap2>
%\fi
%
%\iffalse
%<*samplechap1>
%\fi
% Some text for chapter 1:
%    \begin{macrocode}
\section{one}
some text in chapter one
%    \end{macrocode}

%\iffalse
%</samplechap1>
%\fi
% Some text for chapter 2:
%\iffalse
%<*samplechap2>
%\fi
%    \begin{macrocode}
\section{two}
more text in chapter two
%    \end{macrocode}

%\iffalse
%</samplechap2>
%\fi
%
% %%%%%%%%%%%%%%%%%%%%%%%%%%%%%%%%%%%%%%
% \paragraph{Part Include Files.}
%
% The include files are called |cdocspt3.tex| and |cdocspt4.tex|.
%
%\iffalse
%<*samplepart3|samplepart4>
%\fi

% Optional override for |\version| flag:
%    \begin{macrocode}
%%\providecommand{\version}{final}
%    \end{macrocode}

% Include the main document:
%    \begin{macrocode}
\input{childdoc.def}
\childdocby{cdocsamp}
%    \end{macrocode}

%\iffalse
%</samplepart3|samplepart4>
%\fi
%
%\iffalse
%<*samplepart3>
%\fi
% Some text for part 3:
%    \begin{macrocode}
some text in part three
%    \end{macrocode}

%\iffalse
%</samplepart3>
%\fi
% Some text for part 4:
%\iffalse
%<*samplepart4>
%\fi
%    \begin{macrocode}
more text in part four
%    \end{macrocode}

%\iffalse
%</samplepart4>
%\fi
%
% %%%%%%%%%%%%%%%%%%%%%%%%%%%%%%%%%%%%%%
% \paragraph{Forwarding for a Complete Draft.}
%
% The following forwarding file |cdocsdrf.tex|
% compiles the main document in draft mode:
%\iffalse
%<*sampledraft>
%\fi
%    \begin{macrocode}
\def\version{draft}
\input{childdoc.def}
\childdocforward{cdocsamp}
%    \end{macrocode}

%\iffalse
%</sampledraft>
%\fi
%
% %%%%%%%%%%%%%%%%%%%%%%%%%%%%%%%%%%%%%%
% \paragraph{Forwarding for Final Version of the Chapters.}
%
% The following forwarding files |cdocsfn1.tex| and |cdocsfn2.tex|
% (with identical content)
% compile the final versions of the child documents
% |cdocsch1.tex| and |cdocsch2.tex|, respectively:
%\iffalse
%<*samplefinal>
%\fi
%    \begin{macrocode}
\def\version{final}
\input{childdoc.def}
\childdocforwardprefix[cdocsamp]{cdocsfn}{cdocsch}
%    \end{macrocode}

%\iffalse
%</samplefinal>
%\fi
%
% %%%%%%%%%%%%%%%%%%%%%%%%%%%%%%%%%%%%%%
% \paragraph{Command Line Processing.}
%
% The following three command lines generate the output files
% |cdocscld|, |cdocscl1| and |cdocscl2|
% which should be identical to
% |cdocsdrf|, |cdocsch1| and |cdocsfn2|, respectively:
% \begin{center}
% \begin{tabular}{l}
% |latex -jobname cdocscld \|\\
% |  "\def\version{draft}\input{childdoc.def}\childdocforward{cdocsamp}"|\\
% |latex -jobname cdocscl1 \|\\
% |  "\input{childdoc.def}\childdocforward[cdocsamp]{cdocsch1}"|\\
% |latex -jobname cdocscl2 \|\\
% |  "\def\version{final}\input{childdoc.def}\childdocforward{cdocsch2}"|
% \end{tabular}
% \end{center}
% Note that the trailing backslash on each first line
% merely continues the input to the second line
% (for convenient cut ant paste).
% Furthermore, the command |latex| can be replaced by any
% of its alternative versions such as |pdflatex|.
%
% %%%%%%%%%%%%%%%%%%%%%%%%%%%%%%%%%%%%%%%%%%%%%%%%%%%%%%%%%%%%%%%%%%%%%%%%%%%%%%
% %%%%%%%%%%%%%%%%%%%%%%%%%%%%%%%%%%%%%%%%%%%%%%%%%%%%%%%%%%%%%%%%%%%%%%%%%%%%%%
% \section{Implementation}
%\iffalse
%<*package>
%\fi
%
% This section describes the definitions file |childdoc.def|.

% The definitions cannot be loaded using |\usepackage| or |\RequirePackage|
% which has a mechanism to prevent loading a style file more than once.
% When loading the definitions by means of |\input|
% multiple instances have to be prevented manually:
%\iffalse
%This code needs to be before the `\ProvidesFile' directive
%which is defined at the beginning of this file.
%Therefore it is also placed there and commented out here.
%</package>
%<*discard>
%\fi
%    \begin{macrocode}
\ifdefined\childdocmain\endinput\fi
%    \end{macrocode}
%\iffalse
%</discard>
%<*package>
%\fi
%
% \macro{\ifchilddoc}
% \macro{\ifchilddocmanual}
% The conditional |\ifchilddoc| tells whether a
% child (true) or main (false) document is being compiled.
% The conditional |\ifchilddocmanual| tells whether
% the |\includeonly| mechanism is used (false) or
% the selection of child files must be performed manually (true).
% The definitions initialise to false:
%    \begin{macrocode}
\newif\ifchilddoc
\newif\ifchilddocmanual
%    \end{macrocode}

% \macro{\childdocname}
% \macro{\childdocjob}
% The macro |\childdocname| stores the name of the main document
% to be compiled. The macro |\childdocjob| stores the name of
% the document on which the \LaTeX{} compiler was originally invoked.
% The content of |\jobname| cannot be compared
% to filenames specified in the source due to different catcodes.
% The following code rescans |\jobname|, stores the result
% in |\childdocname| and saves a copy in |\childdocjob|:
%    \begin{macrocode}
\edef\childdocname{\scantokens\expandafter{\jobname\noexpand}}
\let\childdocjob\childdocname
%    \end{macrocode}

% \macro{\childdocdisable}
% The macro |\childdocdisable| prevents the main file
% from being processed more than once.
% At this stage, the main document command |\childdocmain|
% is assumed to be called once again where it should do nothing.
% Any subsequent call to it should prevent
% a secondary processing of the main document
% It overwrites the forwarding commands
% |\childdocof| and |\childdocforward|
% with empty macros to prevent further inclusions of the main document:
%    \begin{macrocode}
\newcommand{\childdocdisable}
{
  \renewcommand{\childdocmain}[1]{\renewcommand{\childdocmain}[1]{\endinput}}
  \renewcommand{\childdocof}[1]{}
  \renewcommand{\childdocby}[2][]{}
  \renewcommand{\childdocforward}[2][]{}
  \renewcommand{\childdocdisable}{}
}
%    \end{macrocode}

% \macro{\childdocmain}
% The macro |\childdocmain| is to be called at the top of the main file
% with nothing or the main filename (without extension) as argument.
% First, it breaks loops.
% If the argument is not empty and does not match |\childdocname|
% (which is set by the first inclusion of |childdoc.def|),
% |\ifchilddoc| is set to true, |\includeonly| is applied to the child file
% and |\jobname| is set to the main file
% (for proper handling of |.aux| files):
%    \begin{macrocode}
\newcommand{\childdocmain}[1]
{
  \childdocdisable\childdocmain{}
  \if?#1?\else
    \begingroup
      \def\childdoctmp{#1}
      \ifx\childdoctmp\childdocname
        \def\childdoctmp{}
      \else
        \def\childdoctmp
        {
          \childdoctrue
          \includeonly{\childdocname}
          \def\childdocjob{#1}
          \def\jobname{#1}
        }
      \fi
      \expandafter
    \endgroup
    \childdoctmp
  \fi
}
%    \end{macrocode}

% \macro{\childdocof}
% The command |\childdocof| redirects
% compilation to the main file |#1|.
%    \begin{macrocode}
\newcommand{\childdocof}[1]
{
  \childdocdisable
  \childdoctrue
  \includeonly{\childdocname}
  \def\jobname{#1}
  \def\childdocjob{#1}
  \input{#1}
}
%    \end{macrocode}

% \macro{\childdocby}
% The command |\childdocby| ....
%    \begin{macrocode}
\newcommand{\childdocby}[2][]
{
  \childdocdisable
  \childdoctrue
  \childdocmanualtrue
  \if?#1?\else
    \def\jobname{#2}
  \fi
  \def\childdocjob{#2}
  \input{#2}
  \endinput
}
%    \end{macrocode}

% \macro{\childdocforward}
% The command |\childdocforward| redirects
% compilation to the main file or
% (if the optional argument is given) a child file.
% Parameters are set as if the main file
% or a child file starting with |\childdocof| was compiled.
% Then compilation is handed over to the main file:
%    \begin{macrocode}
\newcommand{\childdocforward}[2][]
{
  \begingroup
    \if?#1?
      \def\childdoctmp
      {
        \def\childdocname{#2}
        \def\childdocjob{#2}
        \def\jobname{#2}
        \input{#2}
        \endinput
      }
    \else
      \def\childdoctmp
      {
        \childdocdisable
        \def\childdocname{#2}
        \childdoctrue
        \includeonly{#2}
        \def\childdocjob{#1}
        \def\jobname{#1}
        \input{#1}
        \endinput
      }
    \fi
    \expandafter
  \endgroup
  \childdoctmp
}
%    \end{macrocode}

% \macro{\childdocforwardprefix}
% The command |\childdocforwardprefix| redirects
% compilation to the main or a child file by means of a pattern.
% The prefix |#1| in the current filename is replaced by |#2|
% and the suffix of the current filename is kept
% (it is assumed that the filename does not contain the substring `|~~~|'
% which is used as a delimiter).
% Compilation is handed over to the new file by |\childdocforward|:
%    \begin{macrocode}
\newcommand{\childdocforwardprefix}[3][]
{
  \begingroup
    \def\childdocextract #2##1~~~{\def\childdoctmp{\childdocforward[#1]{#3##1}}}
    \expandafter\childdocextract\childdocname~~~
    \expandafter
  \endgroup
  \childdoctmp
}
%    \end{macrocode}

% \macro{\childdoc}
% The deprecated macro |\childdoc| is a legacy version of |\childdocmain|:
%    \begin{macrocode}
\newcommand{\childdoc}{\childdocmain}
%    \end{macrocode}

% \macro{\childdocredirect}
% The deprecated macro |\childdocredirect| is a legacy version
% of |\childdocforward| and |\childdocforwardprefix|:
%    \begin{macrocode}
\newcommand{\childdocredirect}[2][]
{
  \begingroup
    \if?#1?
      \def\childdoctmp{\childdocforward{#2}}
    \else
      \def\childdoctmp{\childdocforwardprefix{#1}{#2}}
    \fi
    \expandafter
  \endgroup
  \childdoctmp
}
%    \end{macrocode}

%\iffalse
%</package>
%\fi
%
\endinput
\childdocforward{cdocsamp}"|\\
% |latex -jobname cdocscl1 \|\\
% |  "% \iffalse
%
% childdoc.dtx Copyright (C) 2017-2018 Niklas Beisert
%
% This work may be distributed and/or modified under the
% conditions of the LaTeX Project Public License, either version 1.3
% of this license or (at your option) any later version.
% The latest version of this license is in
%   http://www.latex-project.org/lppl.txt
% and version 1.3 or later is part of all distributions of LaTeX
% version 2005/12/01 or later.
%
% This work has the LPPL maintenance status `maintained'.
%
% The Current Maintainer of this work is Niklas Beisert.
%
% This work consists of the files childdoc.dtx and childdoc.ins
% and the derived files childdoc.def and cdocsamp.tex with
% cdocsch1.tex, cdocsch2.tex, cdocsdrf.tex, cdocsfn1.tex, cdocsfn2.tex.
%
%<package>\ifdefined\childdocmain\endinput\fi
%<package>\ProvidesFile{childdoc.def}[2018/12/30 v2.0 child document driver]
%<samplemain>\ProvidesFile{cdocsamp.tex}[2018/12/30 v2.0 sample for childdoc]
%<*driver>
%\ProvidesFile{childdoc.drv}[2018/12/30 v2.0 childdoc reference manual file]
\PassOptionsToClass{10pt,a4paper}{article}
\documentclass{ltxdoc}

\usepackage[margin=35mm]{geometry}
\usepackage{hyperref}
\usepackage{hyperxmp}
\usepackage[usenames]{color}

\hypersetup{colorlinks=true}
\hypersetup{pdfstartview=FitH}
\hypersetup{pdfpagemode=UseNone}
\hypersetup{pdfsource={}}
\hypersetup{pdflang={en-UK}}
\hypersetup{pdfcopyright={Copyright 2017-2018 Niklas Beisert.
  This work may be distributed and/or modified under the
  conditions of the LaTeX Project Public License, either version 1.3
  of this license or (at your option) any later version.}}
\hypersetup{pdflicenseurl={http://www.latex-project.org/lppl.txt}}
\hypersetup{pdfcontactaddress={ETH Zurich, ITP, HIT K,
  Wolfgang-Pauli-Strasse 27}}
\hypersetup{pdfcontactpostcode={8093}}
\hypersetup{pdfcontactcity={Zurich}}
\hypersetup{pdfcontactcountry={Switzerland}}
\hypersetup{pdfcontactemail={nbeisert@itp.phys.ethz.ch}}
\hypersetup{pdfcontacturl={http://people.phys.ethz.ch/\xmptilde nbeisert/}}

\newcommand{\secref}[1]{\hyperref[#1]{section \ref*{#1}}}

\parskip1ex
\parindent0pt
\let\olditemize\itemize
\def\itemize{\olditemize\parskip0pt}

\begin{document}

\title{The \textsf{childdoc} Package}
\hypersetup{pdftitle={The childdoc Package}}
\author{Niklas Beisert\\[2ex]
  Institut f\"ur Theoretische Physik\\
  Eidgen\"ossische Technische Hochschule Z\"urich\\
  Wolfgang-Pauli-Strasse 27, 8093 Z\"urich, Switzerland\\[1ex]
  \href{mailto:nbeisert@itp.phys.ethz.ch}
  {\texttt{nbeisert@itp.phys.ethz.ch}}}
\hypersetup{pdfauthor={Niklas Beisert}}
\hypersetup{pdfsubject={Manual for the LaTeX2e Package childdoc}}
\date{30 December 2018, \textsf{v2.0}}
\maketitle

\begin{abstract}\noindent
\textsf{childdoc} is a \LaTeXe{} package
that enables the direct compilation
of document sections included by |\include|
to individual files.
\end{abstract}

\begingroup
\parskip0ex
\tableofcontents
\endgroup

%%%%%%%%%%%%%%%%%%%%%%%%%%%%%%%%%%%%%%%%%%%%%%%%%%%%%%%%%%%%%%%%%%%%%%%%%%%%%%%%
%%%%%%%%%%%%%%%%%%%%%%%%%%%%%%%%%%%%%%%%%%%%%%%%%%%%%%%%%%%%%%%%%%%%%%%%%%%%%%%%
\section{Introduction}

\LaTeX{} provides a mechanism to structure a large document (such as a book)
into a main file and several child files (containing the chapters)
using the |\include| command.
This mechanism is beneficial for documents
which span hundreds of pages in order to
make the source file(s) more manageable.
Moreover, compilation can be restricted to
selected child files by means of the |\includeonly| command.
The latter feature can be used to reduce the compilation time while editing
(this was significantly more useful in the earlier days of \LaTeX{})
or to generate a smaller document which is easier to navigate.
Another application of |\includeonly| is to generate
documents consisting of selected parts of the complete document.

However, there are a few drawbacks of the plain |\include| mechanism:
\begin{itemize}
\item
The child files cannot be compiled on their own,
they can only be compiled via the main file.
A naive editing environment
(such as a text editor with an option
to have the current file processed by \LaTeX)
may require one to switch to the main file before compiling;
attempting to compile the child file produces errors.
\item
The main file must be modified (each time)
to adjust the |\includeonly| command
to the present needs. This easily leaves the main file in a messy state.
\item
The generated document will always carry the filename
of the main document. This is inconvenient if
several child files are to be compiled and
to be kept for distribution.
\end{itemize}

The present package provides a simple interface
to make child files individually compilable by \LaTeX{}.
Compiling a child file then has the same effect as compiling
the main file with an |\includeonly| command
to select the appropriate child.
Moreover the generated document will carry the name of the child
rather than the main file.
This resolves all three above issues.

This feature is meant to make the editing of books,
thesis documents and lecture notes somewhat more convenient.
However, the package can also be used efficiently for
composing a series of documents (such as exercise sheets)
which are typically distributed individually.
It then assists the author in generating the individual documents
(potentially in different versions)
as well as a document containing the collected series.
Another application is in developing style files
or other kinds of included material
where compilation of the style file could redirect
to a sample or test file.

%%%%%%%%%%%%%%%%%%%%%%%%%%%%%%%%%%%%%%%%%%%%%%%%%%%%%%%%%%%%%%%%%%%%%%%%%%%%%%%%
%%%%%%%%%%%%%%%%%%%%%%%%%%%%%%%%%%%%%%%%%%%%%%%%%%%%%%%%%%%%%%%%%%%%%%%%%%%%%%%%
\section{Usage}

First of all, the package \textsf{childdoc} is \emph{not} a standard
\LaTeXe{} |.sty| style file! Therefore it needs to be invoked in
a non-standard way.

%%%%%%%%%%%%%%%%%%%%%%%%%%%%%%%%%%%%%%%%%%%%%%%%%%%%%%%%%%%%%%%%%%%%%%%%%%%%%%%%
\subsection{Included Files}
\label{sec:include}

%%%%%%%%%%%%%%%%%%%%%%%%%%%%%%%%%%%%%%%%
\DescribeMacro{\childdocmain}
To use the package, add the commands
\begin{center}
\begin{tabular}{l}
|\input{childdoc.def}|\\
|\childdocmain{}|\\
\end{tabular}
\end{center}
at the very top of the main \LaTeX{} file,
in particular \emph{before} the |\documentclass| statement!
The argument of |\childdocmain| should be left empty
(but it must be present).

%%%%%%%%%%%%%%%%%%%%%%%%%%%%%%%%%%%%%%%%
\DescribeMacro{\childdocof}
Furthermore, add the commands
\begin{center}
\begin{tabular}{l}
|\input{childdoc.def}|\\
|\childdocof{|\textit{main}|}|\\
\end{tabular}
\end{center}
at the top of every child file \textit{child}
which is included by |\include{|\textit{child}|}|
from within the main file
(or at least for those files to be compiled individually).
The argument \textit{main} must be the filename of the main file.

There are a couple of
considerations in setting up the main and child documents:

%%%%%%%%%%%%%%%%%%%%%%%%%%%%%%%%%%%%%%%%
\paragraph{Restrictions.}

Please note the following restrictions:
\begin{itemize}
\item
|\childdocmain| must be called with one argument \textit{main}
to ensure compatibility with earlier version of the package.
It must either be empty (|\childdocmain{}|)
or precisely match the filename of the main file in which it is specified.
See \secref{sec:detection} for further information.
\item
The filename \textit{main} must be specified without the |.tex| extension.
\item
The filename \textit{main} is case sensitive
(even in case-insensitive file systems)
due to internal string comparison.
\item
The argument \textit{main} should be fully expanded, it cannot be a macro.
\item
Subdirectories and special characters should be avoided in filenames.
\item
The command |\childdocmain{|\textit{main}|}| must be followed by a whitespace.
It should not be followed immediately by another command
or by a comment mark `|%|'.
This is because the \TeX{} parser reads the token immediately following
the argument of |\childdocmain| and puts it
at the beginning of every child section;
however, a white\-space is ignored.
\end{itemize}

%%%%%%%%%%%%%%%%%%%%%%%%%%%%%%%%%%%%%%%%
\paragraph{Content of Main File.}

It is advisable to place all content in the child files included by |\include|.
Any output contained in the main file will appear in all child documents
unless suppressed manually;
it cannot be suppressed automatically by the |\includeonly| directive
and thus should normally be avoided.
A method to include some content in the main file
by means of conditional processing is described in \secref{sec:conditional}.

%%%%%%%%%%%%%%%%%%%%%%%%%%%%%%%%%%%%%%%%
\paragraph{Page Numbering.}

When only a part of the document is compiled,
the appropriate numbering of pages
(as well as other status parameters)
is determined from the |.aux| files.
The latter contain information from previous passes.
However this information needs to propagate through
all intermediate child documents.
Therefore the page numbering in child documents may well
be inconsistent until the complete document is compiled at least once.

A useful (if unconventional) way to always ensure a consistent
page numbering is to restart the numbering in each child document
and denote the pages by `\textit{child}|.|\textit{page}'
where \textit{child} represents the chapter/section number of the child file.
This can be achieved by the command
|\numberwithin{page}{|\textit{child}|}|
of the \textsf{amsmath} package
where \textit{child} can be |chapter| or |section|
depending on the chosen structuring.
Alternatively, one can modify the macro |\thepage| appropriately
and reset the counter |page| at the start of each child file.

%%%%%%%%%%%%%%%%%%%%%%%%%%%%%%%%%%%%%%%%%%%%%%%%%%%%%%%%%%%%%%%%%%%%%%%%%%%%%%%%
\subsection{Conditional Processing}
\label{sec:conditional}

The package provides a mechanism to compile different versions
of a document. To customise the versions further some conditional processing
can come in handy to distinguish which version is being compiled.
The package provides two macros to describe the compilation context:

%%%%%%%%%%%%%%%%%%%%%%%%%%%%%%%%%%%%%%%%
\DescribeMacro{\ifchilddoc}
The conditional |\ifchilddoc| distinguishes between the compilation of
child documents and the main document:
%
\begin{center}
|\ifchilddoc |\textit{child-code}| |[|\||else |\textit{main-code}]| \||fi|
\end{center}

%%%%%%%%%%%%%%%%%%%%%%%%%%%%%%%%%%%%%%%%
\DescribeMacro{\childdocname}
\DescribeMacro{\childdocjob}
The macro |\childdocname| contains the filename (without extension)
of the main or child file being processed.
Note that |\childdocjob| will always contain the name of the main file.

%%%%%%%%%%%%%%%%%%%%%%%%%%%%%%%%%%%%%%%%
\paragraph{Title Page.}

Conditional processing can be used to include a title or banner page
in the main document when proper precautions are taken.
Importantly, the code in the main file should ensure that the page counter
(as well as other status parameters which are stored in the |.aux| files)
takes the same value after the conditional processing.
Otherwise the page numbers may take divergent values
depending on which part is compiled.

For example, a title page could be declared by:
%
\begin{center}
\begin{tabular}{l}
|\ifchilddoc\||else|\\
|\addtocounter{page}{-1}|\\
\textit{code for title page}\\
|\newpage|\\
|\||fi|
\end{tabular}
\end{center}
%
A banner page for the child documents can be generated by:
%
\begin{center}
\begin{tabular}{l}
|\ifchilddoc|\\
|\addtocounter{page}{-1}|\\
\textit{code for banner page}\\
|\newpage|\\
|\||fi|
\end{tabular}
\end{center}
%
Here one could write a message such as:
\begin{center}
|This is the part \childdocname{} of \childdocjob{}.|
\end{center}

%%%%%%%%%%%%%%%%%%%%%%%%%%%%%%%%%%%%%%%%%%%%%%%%%%%%%%%%%%%%%%%%%%%%%%%%%%%%%%%%
\subsection{Flags}
\label{sec:flags}

The package makes it easy to generate different versions
of the main or child documents.
To this end compilation flags can be defined
and assigned different default values.
They will be particularly useful in conjunction
with the forwarding mechanism described in \secref{sec:forward}.

For example, it may be useful to have a flag |\version|
which can be set to |draft| or |final|.
The document source will contain some conditional code
depending on the value of |\version|.
Suppose further, the flag should default to |final| for the main file
and to |draft| for child files
which is a natural assignment for editing the document.
This is achieved by placing the following code
in the preamble of the main document
(below the |\childdocmain| directive):
%
\begin{center}
\begin{tabular}{l}
|\ifchilddoc|\\
|\providecommand{\version}{draft}|\\
|\||else|\\
|\providecommand{\version}{final}|\\
|\||fi|
\end{tabular}
\end{center}
%
The definition by |\providecommand| makes sure
that previous definitions are not overwritten.
Further statements |\providecommand{\version}{...}|
can thus be added before the above code to override it.

For the main file, one might add a line
(between |\childdocmain| and the above block)
%
\begin{center}
|%\ifchilddoc\||else\providecommand{\version}{draft}\||fi|
\end{center}
%
which can be uncommented to produce a draft version.
Likewise one can add a line to the very top of a child file
(above the |\childdocof{|\textit{main}|}| directive)
%
\begin{center}
|%\providecommand{\version}{final}|
\end{center}
%
which can be uncommented to produce the final version of this child document.

%%%%%%%%%%%%%%%%%%%%%%%%%%%%%%%%%%%%%%%%%%%%%%%%%%%%%%%%%%%%%%%%%%%%%%%%%%%%%%%%
\subsection{Forwarding}
\label{sec:forward}

Different versions of the main or child documents
using compilation flags as described in \secref{sec:flags}
can be (permanently) stored in different files
for convenient compilation, viewing and distribution.
To this end, the package defines a command
to pass on compilation to a different file:

%%%%%%%%%%%%%%%%%%%%%%%%%%%%%%%%%%%%%%%%
\DescribeMacro{\childdocforward}
The command |\childdocforward| redirects processing to
another source file:
%
\begin{center}
\begin{tabular}{l}
|\input{childdoc.def}|\\
|\childdocforward[|\textit{main}|]{|\textit{dest}|}|\\
\end{tabular}
\end{center}
%
The argument \textit{dest} is the destination file
(without extension).
It should be the main file or one of the child files.
Note that further \textsf{childdoc} directives
such as |\childdocof| and |\childdocforward|
in the indicated file will be processed in this form.
The optional argument \textit{main}
passes on directly to the main file \textit{main}
while pretending to compile the child \textit{dest}.
This form behaves as if \textit{dest}
issues |\childdocof{|\textit{main}|}| right away,
and no further \textsf{childdoc} directives will be processed.

%%%%%%%%%%%%%%%%%%%%%%%%%%%%%%%%%%%%%%%%
\DescribeMacro{\...prefix}
In the alternative form |\childdocforwardprefix|,
%
\begin{center}
\begin{tabular}{l}
|\input{childdoc.def}|\\
|\childdocforwardprefix[|\textit{main}|]{|\textit{prefix}|}{|\textit{dest}|}|
\end{tabular}
\end{center}
%
the destination file is determined by a pattern
depending on the current file:
To make this work, the current file must be called
`{\textit{prefix}\hspace{0.2em}\textit{suffix}}'
with \textit{prefix} matching precisely the argument.
Processing is then passed on to the file
`{\textit{dest}\hspace{0.2em}\textit{suffix}}'.
Surely, the same effect is achieved by
directly specifying the
argument `{\textit{dest}\hspace{0.2em}\textit{suffix}}'
in the first form.
However, that requires to set up a different file
for each child. With the alternative form of the command
all these files can have exactly the same content
which simplifies setting them up and maintaining them.

For example, the following file |draft.tex|
with a compilation flag |\version| as described in \secref{sec:flags}
compiles the main document as a draft:
%
\begin{center}
\begin{tabular}{l}
|\def\version{draft}|\\
|\input{childdoc.def}|\\
|\childdocforward{|\textit{main}|}|
\end{tabular}
\end{center}
%
Likewise, the following files |final|\textit{nn}|.tex|
compile the final version of the child document
|child|\textit{nn}|.tex|:
%
\begin{center}
\begin{tabular}{l}
|\def\version{final}|\\
|\input{childdoc.def}|\\
|\childdocforwardprefix{final}{child}|
\end{tabular}
\end{center}
%

Note that when several versions of a main file and/or of each child file
are to be generated, it may be convenient to set up a |Makefile| or
shell script to automatise the process.

%%%%%%%%%%%%%%%%%%%%%%%%%%%%%%%%%%%%%%%%%%%%%%%%%%%%%%%%%%%%%%%%%%%%%%%%%%%%%%%%
\subsection{Command Line Processing}
\label{sec:commandline}

The effect of redirection files can also be achieved by invoking
the \LaTeX{} compiler with a more elaborate command line.
Most conveniently this should be done as part
of a shell script or a |Makefile|.

When using \textsf{childdoc} in the main file, the following
command lines effectively perform a redirection
(note that depending on the shell being used,
backslashes may have to be doubled: `|\|' $\to$ `|\\|'):
%
\begin{center}
|... -jobname "|\textit{target}|" |\\|"|[\textit{flags}]%
|\input{childdoc.def}\childdocforward[|\textit{main}|]{|\textit{dest}|}"|
\end{center}
%
Here \textit{target} is the name of the output file,
\textit{main} is the name of the main file
and \textit{dest} is the name of the main or child file to be processed
(all filenames without extensions).
The optional argument \textit{main} can be omitted
if \textit{main} matches \textit{dest}.
Optionally, compilation \textit{flags} can be defined via |\def| commands.
This command line makes the \TeX{} engine believe
it is compiling the file \textit{target}
whose content is specified as the latter parameter.
The provided code then forwards the processing to
\textit{main} or \textit{dest} as described in \secref{sec:forward}.

%%%%%%%%%%%%%%%%%%%%%%%%%%%%%%%%%%%%%%%%%%%%%%%%%%%%%%%%%%%%%%%%%%%%%%%%%%%%%%%%
\subsection{Include by Input}
\label{sec:input}

Including child documents by |\include| has some restrictions by design.
Most notably, the content of a child document always occupies
its own set of pages; pages cannot be shared between child documents.
Usually, this behaviour makes perfect sense
because each child document contain an essential part of the document.
However, in some situations it may be desirable to compose
a document from a collection of parts
without having mandatory page breaks between then.
For this case, the package
provides a mechanism to include parts
by |\input| which can also be processed individually.
However, by construction this mechanism
requires manual handling of the content to be output.

%%%%%%%%%%%%%%%%%%%%%%%%%%%%%%%%%%%%%%%%
\DescribeMacro{\ifchilddocmanual}
The main file should be prepared as usual, see \secref{sec:include}.
However, the document body must make a distinction
between processing of an individual part and of the main document, e.g.:
%
\begin{center}
\begin{tabular}{l}
|\ifchilddocmanual|\\
|\input{\childdocname}|\\
|\||else|\\
\textit{document body with }|\input{|\textit{part}|}|\\
|\||fi|
\end{tabular}
\end{center}
%
The conditional |\ifchilddocmanual| is true whenever
a part to be included by |\input| is being compiled,
and the name of the part is stored in |\childdocname|.

%%%%%%%%%%%%%%%%%%%%%%%%%%%%%%%%%%%%%%%%
\DescribeMacro{\childdocby}
Each part to be included by |\input| should start with:
%
\begin{center}
\begin{tabular}{l}
|\input{childdoc.def}|\\
|\childdocby{|\textit{main}|}|\\
\end{tabular}
\end{center}
%
The directive |\childdocby| is similar to |\childdocof|
described in \secref{sec:include},
but the subsequent selection of content must be done manually.
To that end, both |\ifchilddoc| and |\ifchilddocmanual|
will be true upon processing of a part,
and the name of the part is stored in |\childdocname|.
Note that |\jobname| will be set to the filename of the current part
so that each part receives an individual |.aux| file
that does not interfere with the |.aux| file(s) of the main document.
This behaviour can be altered by the alternative form
|\childdocby[*]{|\textit{main}|}| (with a non-empty optional argument)
which uses the |.aux| file of the main document
by setting |\jobname| to \textit{main}.

%%%%%%%%%%%%%%%%%%%%%%%%%%%%%%%%%%%%%%%%%%%%%%%%%%%%%%%%%%%%%%%%%%%%%%%%%%%%%%%%
\subsection{Driver Development}
\label{sec:driver}

The \textsf{childdoc} mechanism can also be use for the development
of definition files such as \LaTeX{} styles or classes.
This case differs from the above setup with multiple parts
included by |\include| in that no |\includeonly| should be invoked.
This can be achieved by starting the include file
(before |\ProvidesPackage|) with:
%
\begin{center}
\begin{tabular}{l}
|\input{childdoc.def}|\\
|\childdocforward{|\textit{main}|}|\\
\end{tabular}
\end{center}
%
or alternatively with:
%
\begin{center}
\begin{tabular}{l}
|\input{childdoc.def}|\\
|\childdocby{|\textit{main}|}|\\
\end{tabular}
\end{center}
%
Both forms have slightly different effects as described above.
The main file is prepared as usual, see \secref{sec:include}.

%%%%%%%%%%%%%%%%%%%%%%%%%%%%%%%%%%%%%%%%%%%%%%%%%%%%%%%%%%%%%%%%%%%%%%%%%%%%%%%%
\subsection{Legacy Detection}
\label{sec:detection}

The directive |\childdocmain| in the main file can detect
whether the complete document or merely a child is to be compiled
even without using the directive |\childdocof|.
This method is deprecated because it is less robust
and there is no compelling reason to use it;
it is merely provided for backward compatibility
and it may be removed in future versions.

If the detection mechanism is to be used,
it is mandatory to correctly specify
the filename of the main file as the argument of |\childdocmain|:
%
\begin{center}
\begin{tabular}{l}
|\input{childdoc.def}|\\
|\childdocmain{|\textit{main}|}|\\
\end{tabular}
\end{center}
%
If |\jobname| does not match the argument \textit{main} of |\childdocmain|,
it is assumed that |\jobname| points to the child file to be compiled.
When using |\childdocmain| with the main file specified as argument,
it suffices to start a child file
with just |\input{|\textit{main}|}|
without loading of the package and using |\childdocof|.
If instead all processing is done
with the appropriate \textsf{childdoc} directives,
the argument of \textit{main} of |\childdocmain| can be empty.

An alternative version of the command line processing described
in \secref{sec:commandline} using the detection mechanism reads:
%
\begin{center}
|... -jobname "|\textit{target}|" "|[\textit{flags}]%
[|\def\jobname{|\textit{dest}|}|]|\input{|\textit{main}|}"|
\end{center}

%%%%%%%%%%%%%%%%%%%%%%%%%%%%%%%%%%%%%%%%%%%%%%%%%%%%%%%%%%%%%%%%%%%%%%%%%%%%%%%%
\subsection{Manual Code}
\label{sec:manual}

In case one cannot be certain whether the definitions file |childdoc.def|
is installed on the target \TeX{} distribution
and one prefers not to ship it,
it is conceivable to paste a few relevant commands into the sources.

To that end, drop all statements |\input{childdoc.def}|
and perform the replacements as outlined below.
Instead of |\childdocmain{|\textit{main}|}| add the following code
to the top of the main file:
%
\begin{center}
\begin{tabular}{l}
|\||ifdefined\childdocname\endinput\||fi\newif\ifchilddoc|\\
|\edef\childdocname{\scantokens\expandafter{\jobname\noexpand}}|\\
|\def\childdocmain{|\textit{main}|}\||ifx\childdocmain\childdocname\||else|\\
|\childdoctrue\includeonly{\childdocname}\let\jobname\childdocmain\||fi|\\
\end{tabular}
\end{center}
%
Instead of |\childdocof{|\textit{main}|}| just include the main file
at the top of each child file:
%
\begin{center}
|\input{|\textit{main}|}|
\end{center}
%
A simple redirection |\childdocforward{|\textit{dest}|}| is achieved by:
%
\begin{center}
|\def\jobname{|\textit{dest}|}\input{\jobname}|
\end{center}
%
The redirection with prefix
|\childdocforwardprefix[|\textit{prefix}|]{|\textit{dest}|}|
is accomplished by:
%
\begin{center}
\begin{tabular}{l}
|{\edef\jobname{\scantokens\expandafter{\jobname\noexpand}}|\\
|\def\redirectjob |\textit{prefix}|#1~~~{\gdef\jobname{|\textit{dest}|#1}}|\\
|\expandafter\redirectjob\jobname~~~}\input{\jobname}|
\end{tabular}
\end{center}

In an alternative approach,
child documents can be compiled by a specific command line
without additional code or specific definitions:
%
\begin{center}
|... -jobname "|\textit{target}|" "|[\textit{flags}]%
|\includeonly{|\textit{dest}|}\input{|\textit{main}|}"|
\end{center}
%

%%%%%%%%%%%%%%%%%%%%%%%%%%%%%%%%%%%%%%%%%%%%%%%%%%%%%%%%%%%%%%%%%%%%%%%%%%%%%%%%
%%%%%%%%%%%%%%%%%%%%%%%%%%%%%%%%%%%%%%%%%%%%%%%%%%%%%%%%%%%%%%%%%%%%%%%%%%%%%%%%
\section{Information}

%%%%%%%%%%%%%%%%%%%%%%%%%%%%%%%%%%%%%%%%%%%%%%%%%%%%%%%%%%%%%%%%%%%%%%%%%%%%%%%%
\subsection{Copyright}

Copyright \copyright{} 2017--2018 Niklas Beisert

This work may be distributed and/or modified under the
conditions of the \LaTeX{} Project Public License, either version 1.3
of this license or (at your option) any later version.
The latest version of this license is in
  \url{http://www.latex-project.org/lppl.txt}
and version 1.3 or later is part of all distributions of \LaTeX{}
version 2005/12/01 or later.

This work has the LPPL maintenance status `maintained'.

The Current Maintainer of this work is Niklas Beisert.

This work consists of the files |README.txt|, |childdoc.ins| and |childdoc.dtx|
as well as the derived files |childdoc.def|, |cdocsamp.tex|
with |cdocsch1.tex|, |cdocsch2.tex|, |cdocspt3.tex|, |cdocspt4.tex|,
|cdocsdrf.tex|, |cdocsfn1.tex|, |cdocsfn2.tex|
as well as |childdoc.pdf|.

%%%%%%%%%%%%%%%%%%%%%%%%%%%%%%%%%%%%%%%%%%%%%%%%%%%%%%%%%%%%%%%%%%%%%%%%%%%%%%%%
\subsection{Files and Installation}

The package consists of the files:
%
\begin{center}
\begin{tabular}{ll}
    |README.txt|   & readme file \\
    |childdoc.ins| & installation file \\
    |childdoc.dtx| & source file \\
    |childdoc.def| & definition file \\
    |cdocsamp.tex| & sample main file \\
    |cdocsch1.tex| & sample include file \\
    |cdocsch2.tex| & sample include file \\
    |cdocspt3.tex| & sample part file \\
    |cdocspt4.tex| & sample part file \\
    |cdocsdrf.tex| & sample redirection file \\
    |cdocsfn1.tex| & sample redirection file \\
    |cdocsfn2.tex| & sample redirection file \\
    |childdoc.pdf| & manual
\end{tabular}
\end{center}
%
The distribution consists of the files
|README.txt|, |childdoc.ins| and |childdoc.dtx|.
%
\begin{itemize}
\item
Run (pdf)\LaTeX{} on |childdoc.dtx|
to compile the manual |childdoc.pdf| (this file).
\item
Run \LaTeX{} on |childdoc.ins| to create the definitions file |childdoc.def|
and the sample |cdocsamp.tex| with include files
|cdocsch1.tex|, |cdocsch2.tex|, |cdocspt3.tex|, |cdocspt4.tex|,
|cdocsdrf.tex|, |cdocsfn1.tex|, |cdocsfn2.tex|.
Then copy the file |childdoc.def| to an appropriate directory of your \LaTeX{}
distribution, e.g.\ \textit{texmf-root}|/tex/latex/childdoc|.
\end{itemize}

%%%%%%%%%%%%%%%%%%%%%%%%%%%%%%%%%%%%%%%%%%%%%%%%%%%%%%%%%%%%%%%%%%%%%%%%%%%%%%%%
\subsection{Related CTAN Packages}

There are several other packages which offer a similar functionality:
%
\begin{itemize}
\item
The packages
\href{http://ctan.org/pkg/docmute}{\textsf{docmute}},
\href{http://ctan.org/pkg/includex}{\textsf{includex}} and
\href{http://ctan.org/pkg/standalone}{\textsf{standalone}}
provide commands to include only the document body of
a child file thus allowing both files to be compiled individually.
\item
The packages \href{http://ctan.org/pkg/subdocs}{\textsf{subdocs}}
and \href{http://ctan.org/pkg/subfiles}{\textsf{subfiles}}
provide structures in which the main and child documents can be
encapsulated and allowing them to be compiled individually.
The inclusion mechanism is different from the conventional |\include|.
\item
The package \href{http://ctan.org/pkg/combine}{\textsf{combine}}
is an elaborate solution to combine several documents into one.
\end{itemize}
%
See also the CTAN topic \href{http://ctan.org/topic/subdocs}{\textsf{subdocs}}
for further related packages.
The present package differs from the above solutions in that
a document structure constructed with the conventional |\include| mechanism
just needs two extra commands at the top of every file
such that all constituent files can be compiled individually.

%%%%%%%%%%%%%%%%%%%%%%%%%%%%%%%%%%%%%%%%%%%%%%%%%%%%%%%%%%%%%%%%%%%%%%%%%%%%%%%%
%\subsection{Feature Suggestions}
%
%The following is a list of features which may be useful for future
%versions of this package:
%%
%\begin{itemize}
%\item
%\ldots
%\end{itemize}

%%%%%%%%%%%%%%%%%%%%%%%%%%%%%%%%%%%%%%%%%%%%%%%%%%%%%%%%%%%%%%%%%%%%%%%%%%%%%%%%
\subsection{Revision History}

%%%%%%%%%%%%%%%%%%%%%%%%%%%%%%%%%%%%%%%%
\paragraph{v2.0:} 2018/12/30

\begin{itemize}
\item
immediate forward processing
\item
added |\childdocby| mechanism
\item
manual restructured
\end{itemize}

%%%%%%%%%%%%%%%%%%%%%%%%%%%%%%%%%%%%%%%%
\paragraph{v1.6:} 2018/01/17

\begin{itemize}
\item
application for development of include files
\item
corrections to manual
\end{itemize}

%%%%%%%%%%%%%%%%%%%%%%%%%%%%%%%%%%%%%%%%
\paragraph{v1.5:} 2017/05/21

\begin{itemize}
\item
more complete structuring introduced
\item
|\childdocof| introduced
\item
|\childdoc| renamed to |\childdocmain|
\item
|\childredirect| renamed to |\childdocforward| and |\childdocforwardprefix|
and functionality expanded
\end{itemize}

%%%%%%%%%%%%%%%%%%%%%%%%%%%%%%%%%%%%%%%%
\paragraph{v1.0:} 2017/04/27

\begin{itemize}
\item
manual and install package
\item
first version published on CTAN
\end{itemize}

%%%%%%%%%%%%%%%%%%%%%%%%%%%%%%%%%%%%%%%%
\paragraph{v0.6:} 2017/04/26

\begin{itemize}
\item
redirection mechanism added
\end{itemize}

%%%%%%%%%%%%%%%%%%%%%%%%%%%%%%%%%%%%%%%%
\paragraph{v0.5:} 2017/04/26

\begin{itemize}
\item
functionality in definition file
\end{itemize}


%%%%%%%%%%%%%%%%%%%%%%%%%%%%%%%%%%%%%%%%%%%%%%%%%%%%%%%%%%%%%%%%%%%%%%%%%%%%%%%%
%%%%%%%%%%%%%%%%%%%%%%%%%%%%%%%%%%%%%%%%%%%%%%%%%%%%%%%%%%%%%%%%%%%%%%%%%%%%%%%%
%%%%%%%%%%%%%%%%%%%%%%%%%%%%%%%%%%%%%%%%%%%%%%%%%%%%%%%%%%%%%%%%%%%%%%%%%%%%%%%%
\appendix

\settowidth\MacroIndent{\rmfamily\scriptsize 000\ }

 \DocInput{childdoc.dtx}

\end{document}
%</driver>
% \fi
%
% %%%%%%%%%%%%%%%%%%%%%%%%%%%%%%%%%%%%%%%%%%%%%%%%%%%%%%%%%%%%%%%%%%%%%%%%%%%%%%
% %%%%%%%%%%%%%%%%%%%%%%%%%%%%%%%%%%%%%%%%%%%%%%%%%%%%%%%%%%%%%%%%%%%%%%%%%%%%%%
% \section{Sample}
%\iffalse
%<*samplemain>
%\fi
%
% The following presents a sample document
% with two chapters, two parts, a title page,
% a compile flag as well as three forwarding files to set the flag.
% It consists of eight |.tex| files:
% \begin{center}
% \begin{tabular}{ll}
% |cdocsamp.tex|&main file\\
% |cdocsch1.tex|&include file for chapter 1\\
% |cdocsch2.tex|&include file for chapter 2\\
% |cdocspt3.tex|&include file for part 3\\
% |cdocspt4.tex|&include file for part 4\\
% |cdocsdrf.tex|&forwarding file for main file in draft mode\\
% |cdocsfi1.tex|&forwarding file for final version of chapter 1\\
% |cdocsfi2.tex|&forwarding file for final version of chapter 2\\
% \end{tabular}
% \end{center}
% Each of the eight files can be compiled directly by the \LaTeX{} compiler.
%
% %%%%%%%%%%%%%%%%%%%%%%%%%%%%%%%%%%%%%%
% \paragraph{Main File.}
%
% The main file is called |cdocsamp.tex|.
%
% Load the \textsf{childdoc} definitions and
% declare the filename for the main document:
%    \begin{macrocode}
\input{childdoc.def}
\childdocmain{}
%    \end{macrocode}

% Optional override for |\version| flag:
%    \begin{macrocode}
%%\ifchilddoc\else\providecommand{\version}{draft}\fi
%    \end{macrocode}

% Define the default values for the |\version| flag
% (|final| for the main file and |draft| for childs):
%    \begin{macrocode}
\ifchilddoc
\providecommand{\version}{draft}
\else
\providecommand{\version}{final}
\fi
%    \end{macrocode}

% Load the standard document class:
%    \begin{macrocode}
\documentclass[12pt]{article}
%    \end{macrocode}

% Start the document body:
%    \begin{macrocode}
\begin{document}
%    \end{macrocode}

% Declare a title page.
% Print title, part of document being processed and version flag:
%    \begin{macrocode}
\addtocounter{page}{-1}
\begin{center}
{\LARGE\bfseries{}childdoc example\par}
\vspace{1cm}
\ifchilddoc
\ifchilddocmanual part\else chapter\fi:
`\childdocname' of `\childdocjob'\par
\else
main document: `\childdocjob'\par
\fi
version: \version\par
\end{center}
\newpage
%    \end{macrocode}

% Manually include selected file,
% otherwise process as usual:
%    \begin{macrocode}
\ifchilddocmanual
\section*{part `\childdocname'}
\input{\childdocname}
\else
%    \end{macrocode}

% Include the two chapters:
%    \begin{macrocode}
\include{cdocsch1}
\include{cdocsch2}
%    \end{macrocode}

% Include the two parts unless only chapters should be displayed:
%    \begin{macrocode}
\ifchilddoc\else
\section{part three}
\input{cdocspt3}
\section{part four}
\input{cdocspt4}
\fi
%    \end{macrocode}

% Process as usual until here:
%    \begin{macrocode}
\fi
%    \end{macrocode}

% End of document body:
%    \begin{macrocode}
\end{document}
%    \end{macrocode}
%\iffalse
%</samplemain>
%\fi
%
% %%%%%%%%%%%%%%%%%%%%%%%%%%%%%%%%%%%%%%
% \paragraph{Chapter Include Files.}
%
% The include files are called |cdocsch1.tex| and |cdocsch2.tex|.
%
%\iffalse
%<*samplechap1|samplechap2>
%\fi

% Optional override for |\version| flag:
%    \begin{macrocode}
%%\providecommand{\version}{final}
%    \end{macrocode}

% Include the main document:
%    \begin{macrocode}
\input{childdoc.def}
\childdocof{cdocsamp}
%    \end{macrocode}

%\iffalse
%</samplechap1|samplechap2>
%\fi
%
%\iffalse
%<*samplechap1>
%\fi
% Some text for chapter 1:
%    \begin{macrocode}
\section{one}
some text in chapter one
%    \end{macrocode}

%\iffalse
%</samplechap1>
%\fi
% Some text for chapter 2:
%\iffalse
%<*samplechap2>
%\fi
%    \begin{macrocode}
\section{two}
more text in chapter two
%    \end{macrocode}

%\iffalse
%</samplechap2>
%\fi
%
% %%%%%%%%%%%%%%%%%%%%%%%%%%%%%%%%%%%%%%
% \paragraph{Part Include Files.}
%
% The include files are called |cdocspt3.tex| and |cdocspt4.tex|.
%
%\iffalse
%<*samplepart3|samplepart4>
%\fi

% Optional override for |\version| flag:
%    \begin{macrocode}
%%\providecommand{\version}{final}
%    \end{macrocode}

% Include the main document:
%    \begin{macrocode}
\input{childdoc.def}
\childdocby{cdocsamp}
%    \end{macrocode}

%\iffalse
%</samplepart3|samplepart4>
%\fi
%
%\iffalse
%<*samplepart3>
%\fi
% Some text for part 3:
%    \begin{macrocode}
some text in part three
%    \end{macrocode}

%\iffalse
%</samplepart3>
%\fi
% Some text for part 4:
%\iffalse
%<*samplepart4>
%\fi
%    \begin{macrocode}
more text in part four
%    \end{macrocode}

%\iffalse
%</samplepart4>
%\fi
%
% %%%%%%%%%%%%%%%%%%%%%%%%%%%%%%%%%%%%%%
% \paragraph{Forwarding for a Complete Draft.}
%
% The following forwarding file |cdocsdrf.tex|
% compiles the main document in draft mode:
%\iffalse
%<*sampledraft>
%\fi
%    \begin{macrocode}
\def\version{draft}
\input{childdoc.def}
\childdocforward{cdocsamp}
%    \end{macrocode}

%\iffalse
%</sampledraft>
%\fi
%
% %%%%%%%%%%%%%%%%%%%%%%%%%%%%%%%%%%%%%%
% \paragraph{Forwarding for Final Version of the Chapters.}
%
% The following forwarding files |cdocsfn1.tex| and |cdocsfn2.tex|
% (with identical content)
% compile the final versions of the child documents
% |cdocsch1.tex| and |cdocsch2.tex|, respectively:
%\iffalse
%<*samplefinal>
%\fi
%    \begin{macrocode}
\def\version{final}
\input{childdoc.def}
\childdocforwardprefix[cdocsamp]{cdocsfn}{cdocsch}
%    \end{macrocode}

%\iffalse
%</samplefinal>
%\fi
%
% %%%%%%%%%%%%%%%%%%%%%%%%%%%%%%%%%%%%%%
% \paragraph{Command Line Processing.}
%
% The following three command lines generate the output files
% |cdocscld|, |cdocscl1| and |cdocscl2|
% which should be identical to
% |cdocsdrf|, |cdocsch1| and |cdocsfn2|, respectively:
% \begin{center}
% \begin{tabular}{l}
% |latex -jobname cdocscld \|\\
% |  "\def\version{draft}\input{childdoc.def}\childdocforward{cdocsamp}"|\\
% |latex -jobname cdocscl1 \|\\
% |  "\input{childdoc.def}\childdocforward[cdocsamp]{cdocsch1}"|\\
% |latex -jobname cdocscl2 \|\\
% |  "\def\version{final}\input{childdoc.def}\childdocforward{cdocsch2}"|
% \end{tabular}
% \end{center}
% Note that the trailing backslash on each first line
% merely continues the input to the second line
% (for convenient cut ant paste).
% Furthermore, the command |latex| can be replaced by any
% of its alternative versions such as |pdflatex|.
%
% %%%%%%%%%%%%%%%%%%%%%%%%%%%%%%%%%%%%%%%%%%%%%%%%%%%%%%%%%%%%%%%%%%%%%%%%%%%%%%
% %%%%%%%%%%%%%%%%%%%%%%%%%%%%%%%%%%%%%%%%%%%%%%%%%%%%%%%%%%%%%%%%%%%%%%%%%%%%%%
% \section{Implementation}
%\iffalse
%<*package>
%\fi
%
% This section describes the definitions file |childdoc.def|.

% The definitions cannot be loaded using |\usepackage| or |\RequirePackage|
% which has a mechanism to prevent loading a style file more than once.
% When loading the definitions by means of |\input|
% multiple instances have to be prevented manually:
%\iffalse
%This code needs to be before the `\ProvidesFile' directive
%which is defined at the beginning of this file.
%Therefore it is also placed there and commented out here.
%</package>
%<*discard>
%\fi
%    \begin{macrocode}
\ifdefined\childdocmain\endinput\fi
%    \end{macrocode}
%\iffalse
%</discard>
%<*package>
%\fi
%
% \macro{\ifchilddoc}
% \macro{\ifchilddocmanual}
% The conditional |\ifchilddoc| tells whether a
% child (true) or main (false) document is being compiled.
% The conditional |\ifchilddocmanual| tells whether
% the |\includeonly| mechanism is used (false) or
% the selection of child files must be performed manually (true).
% The definitions initialise to false:
%    \begin{macrocode}
\newif\ifchilddoc
\newif\ifchilddocmanual
%    \end{macrocode}

% \macro{\childdocname}
% \macro{\childdocjob}
% The macro |\childdocname| stores the name of the main document
% to be compiled. The macro |\childdocjob| stores the name of
% the document on which the \LaTeX{} compiler was originally invoked.
% The content of |\jobname| cannot be compared
% to filenames specified in the source due to different catcodes.
% The following code rescans |\jobname|, stores the result
% in |\childdocname| and saves a copy in |\childdocjob|:
%    \begin{macrocode}
\edef\childdocname{\scantokens\expandafter{\jobname\noexpand}}
\let\childdocjob\childdocname
%    \end{macrocode}

% \macro{\childdocdisable}
% The macro |\childdocdisable| prevents the main file
% from being processed more than once.
% At this stage, the main document command |\childdocmain|
% is assumed to be called once again where it should do nothing.
% Any subsequent call to it should prevent
% a secondary processing of the main document
% It overwrites the forwarding commands
% |\childdocof| and |\childdocforward|
% with empty macros to prevent further inclusions of the main document:
%    \begin{macrocode}
\newcommand{\childdocdisable}
{
  \renewcommand{\childdocmain}[1]{\renewcommand{\childdocmain}[1]{\endinput}}
  \renewcommand{\childdocof}[1]{}
  \renewcommand{\childdocby}[2][]{}
  \renewcommand{\childdocforward}[2][]{}
  \renewcommand{\childdocdisable}{}
}
%    \end{macrocode}

% \macro{\childdocmain}
% The macro |\childdocmain| is to be called at the top of the main file
% with nothing or the main filename (without extension) as argument.
% First, it breaks loops.
% If the argument is not empty and does not match |\childdocname|
% (which is set by the first inclusion of |childdoc.def|),
% |\ifchilddoc| is set to true, |\includeonly| is applied to the child file
% and |\jobname| is set to the main file
% (for proper handling of |.aux| files):
%    \begin{macrocode}
\newcommand{\childdocmain}[1]
{
  \childdocdisable\childdocmain{}
  \if?#1?\else
    \begingroup
      \def\childdoctmp{#1}
      \ifx\childdoctmp\childdocname
        \def\childdoctmp{}
      \else
        \def\childdoctmp
        {
          \childdoctrue
          \includeonly{\childdocname}
          \def\childdocjob{#1}
          \def\jobname{#1}
        }
      \fi
      \expandafter
    \endgroup
    \childdoctmp
  \fi
}
%    \end{macrocode}

% \macro{\childdocof}
% The command |\childdocof| redirects
% compilation to the main file |#1|.
%    \begin{macrocode}
\newcommand{\childdocof}[1]
{
  \childdocdisable
  \childdoctrue
  \includeonly{\childdocname}
  \def\jobname{#1}
  \def\childdocjob{#1}
  \input{#1}
}
%    \end{macrocode}

% \macro{\childdocby}
% The command |\childdocby| ....
%    \begin{macrocode}
\newcommand{\childdocby}[2][]
{
  \childdocdisable
  \childdoctrue
  \childdocmanualtrue
  \if?#1?\else
    \def\jobname{#2}
  \fi
  \def\childdocjob{#2}
  \input{#2}
  \endinput
}
%    \end{macrocode}

% \macro{\childdocforward}
% The command |\childdocforward| redirects
% compilation to the main file or
% (if the optional argument is given) a child file.
% Parameters are set as if the main file
% or a child file starting with |\childdocof| was compiled.
% Then compilation is handed over to the main file:
%    \begin{macrocode}
\newcommand{\childdocforward}[2][]
{
  \begingroup
    \if?#1?
      \def\childdoctmp
      {
        \def\childdocname{#2}
        \def\childdocjob{#2}
        \def\jobname{#2}
        \input{#2}
        \endinput
      }
    \else
      \def\childdoctmp
      {
        \childdocdisable
        \def\childdocname{#2}
        \childdoctrue
        \includeonly{#2}
        \def\childdocjob{#1}
        \def\jobname{#1}
        \input{#1}
        \endinput
      }
    \fi
    \expandafter
  \endgroup
  \childdoctmp
}
%    \end{macrocode}

% \macro{\childdocforwardprefix}
% The command |\childdocforwardprefix| redirects
% compilation to the main or a child file by means of a pattern.
% The prefix |#1| in the current filename is replaced by |#2|
% and the suffix of the current filename is kept
% (it is assumed that the filename does not contain the substring `|~~~|'
% which is used as a delimiter).
% Compilation is handed over to the new file by |\childdocforward|:
%    \begin{macrocode}
\newcommand{\childdocforwardprefix}[3][]
{
  \begingroup
    \def\childdocextract #2##1~~~{\def\childdoctmp{\childdocforward[#1]{#3##1}}}
    \expandafter\childdocextract\childdocname~~~
    \expandafter
  \endgroup
  \childdoctmp
}
%    \end{macrocode}

% \macro{\childdoc}
% The deprecated macro |\childdoc| is a legacy version of |\childdocmain|:
%    \begin{macrocode}
\newcommand{\childdoc}{\childdocmain}
%    \end{macrocode}

% \macro{\childdocredirect}
% The deprecated macro |\childdocredirect| is a legacy version
% of |\childdocforward| and |\childdocforwardprefix|:
%    \begin{macrocode}
\newcommand{\childdocredirect}[2][]
{
  \begingroup
    \if?#1?
      \def\childdoctmp{\childdocforward{#2}}
    \else
      \def\childdoctmp{\childdocforwardprefix{#1}{#2}}
    \fi
    \expandafter
  \endgroup
  \childdoctmp
}
%    \end{macrocode}

%\iffalse
%</package>
%\fi
%
\endinput
\childdocforward[cdocsamp]{cdocsch1}"|\\
% |latex -jobname cdocscl2 \|\\
% |  "\def\version{final}% \iffalse
%
% childdoc.dtx Copyright (C) 2017-2018 Niklas Beisert
%
% This work may be distributed and/or modified under the
% conditions of the LaTeX Project Public License, either version 1.3
% of this license or (at your option) any later version.
% The latest version of this license is in
%   http://www.latex-project.org/lppl.txt
% and version 1.3 or later is part of all distributions of LaTeX
% version 2005/12/01 or later.
%
% This work has the LPPL maintenance status `maintained'.
%
% The Current Maintainer of this work is Niklas Beisert.
%
% This work consists of the files childdoc.dtx and childdoc.ins
% and the derived files childdoc.def and cdocsamp.tex with
% cdocsch1.tex, cdocsch2.tex, cdocsdrf.tex, cdocsfn1.tex, cdocsfn2.tex.
%
%<package>\ifdefined\childdocmain\endinput\fi
%<package>\ProvidesFile{childdoc.def}[2018/12/30 v2.0 child document driver]
%<samplemain>\ProvidesFile{cdocsamp.tex}[2018/12/30 v2.0 sample for childdoc]
%<*driver>
%\ProvidesFile{childdoc.drv}[2018/12/30 v2.0 childdoc reference manual file]
\PassOptionsToClass{10pt,a4paper}{article}
\documentclass{ltxdoc}

\usepackage[margin=35mm]{geometry}
\usepackage{hyperref}
\usepackage{hyperxmp}
\usepackage[usenames]{color}

\hypersetup{colorlinks=true}
\hypersetup{pdfstartview=FitH}
\hypersetup{pdfpagemode=UseNone}
\hypersetup{pdfsource={}}
\hypersetup{pdflang={en-UK}}
\hypersetup{pdfcopyright={Copyright 2017-2018 Niklas Beisert.
  This work may be distributed and/or modified under the
  conditions of the LaTeX Project Public License, either version 1.3
  of this license or (at your option) any later version.}}
\hypersetup{pdflicenseurl={http://www.latex-project.org/lppl.txt}}
\hypersetup{pdfcontactaddress={ETH Zurich, ITP, HIT K,
  Wolfgang-Pauli-Strasse 27}}
\hypersetup{pdfcontactpostcode={8093}}
\hypersetup{pdfcontactcity={Zurich}}
\hypersetup{pdfcontactcountry={Switzerland}}
\hypersetup{pdfcontactemail={nbeisert@itp.phys.ethz.ch}}
\hypersetup{pdfcontacturl={http://people.phys.ethz.ch/\xmptilde nbeisert/}}

\newcommand{\secref}[1]{\hyperref[#1]{section \ref*{#1}}}

\parskip1ex
\parindent0pt
\let\olditemize\itemize
\def\itemize{\olditemize\parskip0pt}

\begin{document}

\title{The \textsf{childdoc} Package}
\hypersetup{pdftitle={The childdoc Package}}
\author{Niklas Beisert\\[2ex]
  Institut f\"ur Theoretische Physik\\
  Eidgen\"ossische Technische Hochschule Z\"urich\\
  Wolfgang-Pauli-Strasse 27, 8093 Z\"urich, Switzerland\\[1ex]
  \href{mailto:nbeisert@itp.phys.ethz.ch}
  {\texttt{nbeisert@itp.phys.ethz.ch}}}
\hypersetup{pdfauthor={Niklas Beisert}}
\hypersetup{pdfsubject={Manual for the LaTeX2e Package childdoc}}
\date{30 December 2018, \textsf{v2.0}}
\maketitle

\begin{abstract}\noindent
\textsf{childdoc} is a \LaTeXe{} package
that enables the direct compilation
of document sections included by |\include|
to individual files.
\end{abstract}

\begingroup
\parskip0ex
\tableofcontents
\endgroup

%%%%%%%%%%%%%%%%%%%%%%%%%%%%%%%%%%%%%%%%%%%%%%%%%%%%%%%%%%%%%%%%%%%%%%%%%%%%%%%%
%%%%%%%%%%%%%%%%%%%%%%%%%%%%%%%%%%%%%%%%%%%%%%%%%%%%%%%%%%%%%%%%%%%%%%%%%%%%%%%%
\section{Introduction}

\LaTeX{} provides a mechanism to structure a large document (such as a book)
into a main file and several child files (containing the chapters)
using the |\include| command.
This mechanism is beneficial for documents
which span hundreds of pages in order to
make the source file(s) more manageable.
Moreover, compilation can be restricted to
selected child files by means of the |\includeonly| command.
The latter feature can be used to reduce the compilation time while editing
(this was significantly more useful in the earlier days of \LaTeX{})
or to generate a smaller document which is easier to navigate.
Another application of |\includeonly| is to generate
documents consisting of selected parts of the complete document.

However, there are a few drawbacks of the plain |\include| mechanism:
\begin{itemize}
\item
The child files cannot be compiled on their own,
they can only be compiled via the main file.
A naive editing environment
(such as a text editor with an option
to have the current file processed by \LaTeX)
may require one to switch to the main file before compiling;
attempting to compile the child file produces errors.
\item
The main file must be modified (each time)
to adjust the |\includeonly| command
to the present needs. This easily leaves the main file in a messy state.
\item
The generated document will always carry the filename
of the main document. This is inconvenient if
several child files are to be compiled and
to be kept for distribution.
\end{itemize}

The present package provides a simple interface
to make child files individually compilable by \LaTeX{}.
Compiling a child file then has the same effect as compiling
the main file with an |\includeonly| command
to select the appropriate child.
Moreover the generated document will carry the name of the child
rather than the main file.
This resolves all three above issues.

This feature is meant to make the editing of books,
thesis documents and lecture notes somewhat more convenient.
However, the package can also be used efficiently for
composing a series of documents (such as exercise sheets)
which are typically distributed individually.
It then assists the author in generating the individual documents
(potentially in different versions)
as well as a document containing the collected series.
Another application is in developing style files
or other kinds of included material
where compilation of the style file could redirect
to a sample or test file.

%%%%%%%%%%%%%%%%%%%%%%%%%%%%%%%%%%%%%%%%%%%%%%%%%%%%%%%%%%%%%%%%%%%%%%%%%%%%%%%%
%%%%%%%%%%%%%%%%%%%%%%%%%%%%%%%%%%%%%%%%%%%%%%%%%%%%%%%%%%%%%%%%%%%%%%%%%%%%%%%%
\section{Usage}

First of all, the package \textsf{childdoc} is \emph{not} a standard
\LaTeXe{} |.sty| style file! Therefore it needs to be invoked in
a non-standard way.

%%%%%%%%%%%%%%%%%%%%%%%%%%%%%%%%%%%%%%%%%%%%%%%%%%%%%%%%%%%%%%%%%%%%%%%%%%%%%%%%
\subsection{Included Files}
\label{sec:include}

%%%%%%%%%%%%%%%%%%%%%%%%%%%%%%%%%%%%%%%%
\DescribeMacro{\childdocmain}
To use the package, add the commands
\begin{center}
\begin{tabular}{l}
|\input{childdoc.def}|\\
|\childdocmain{}|\\
\end{tabular}
\end{center}
at the very top of the main \LaTeX{} file,
in particular \emph{before} the |\documentclass| statement!
The argument of |\childdocmain| should be left empty
(but it must be present).

%%%%%%%%%%%%%%%%%%%%%%%%%%%%%%%%%%%%%%%%
\DescribeMacro{\childdocof}
Furthermore, add the commands
\begin{center}
\begin{tabular}{l}
|\input{childdoc.def}|\\
|\childdocof{|\textit{main}|}|\\
\end{tabular}
\end{center}
at the top of every child file \textit{child}
which is included by |\include{|\textit{child}|}|
from within the main file
(or at least for those files to be compiled individually).
The argument \textit{main} must be the filename of the main file.

There are a couple of
considerations in setting up the main and child documents:

%%%%%%%%%%%%%%%%%%%%%%%%%%%%%%%%%%%%%%%%
\paragraph{Restrictions.}

Please note the following restrictions:
\begin{itemize}
\item
|\childdocmain| must be called with one argument \textit{main}
to ensure compatibility with earlier version of the package.
It must either be empty (|\childdocmain{}|)
or precisely match the filename of the main file in which it is specified.
See \secref{sec:detection} for further information.
\item
The filename \textit{main} must be specified without the |.tex| extension.
\item
The filename \textit{main} is case sensitive
(even in case-insensitive file systems)
due to internal string comparison.
\item
The argument \textit{main} should be fully expanded, it cannot be a macro.
\item
Subdirectories and special characters should be avoided in filenames.
\item
The command |\childdocmain{|\textit{main}|}| must be followed by a whitespace.
It should not be followed immediately by another command
or by a comment mark `|%|'.
This is because the \TeX{} parser reads the token immediately following
the argument of |\childdocmain| and puts it
at the beginning of every child section;
however, a white\-space is ignored.
\end{itemize}

%%%%%%%%%%%%%%%%%%%%%%%%%%%%%%%%%%%%%%%%
\paragraph{Content of Main File.}

It is advisable to place all content in the child files included by |\include|.
Any output contained in the main file will appear in all child documents
unless suppressed manually;
it cannot be suppressed automatically by the |\includeonly| directive
and thus should normally be avoided.
A method to include some content in the main file
by means of conditional processing is described in \secref{sec:conditional}.

%%%%%%%%%%%%%%%%%%%%%%%%%%%%%%%%%%%%%%%%
\paragraph{Page Numbering.}

When only a part of the document is compiled,
the appropriate numbering of pages
(as well as other status parameters)
is determined from the |.aux| files.
The latter contain information from previous passes.
However this information needs to propagate through
all intermediate child documents.
Therefore the page numbering in child documents may well
be inconsistent until the complete document is compiled at least once.

A useful (if unconventional) way to always ensure a consistent
page numbering is to restart the numbering in each child document
and denote the pages by `\textit{child}|.|\textit{page}'
where \textit{child} represents the chapter/section number of the child file.
This can be achieved by the command
|\numberwithin{page}{|\textit{child}|}|
of the \textsf{amsmath} package
where \textit{child} can be |chapter| or |section|
depending on the chosen structuring.
Alternatively, one can modify the macro |\thepage| appropriately
and reset the counter |page| at the start of each child file.

%%%%%%%%%%%%%%%%%%%%%%%%%%%%%%%%%%%%%%%%%%%%%%%%%%%%%%%%%%%%%%%%%%%%%%%%%%%%%%%%
\subsection{Conditional Processing}
\label{sec:conditional}

The package provides a mechanism to compile different versions
of a document. To customise the versions further some conditional processing
can come in handy to distinguish which version is being compiled.
The package provides two macros to describe the compilation context:

%%%%%%%%%%%%%%%%%%%%%%%%%%%%%%%%%%%%%%%%
\DescribeMacro{\ifchilddoc}
The conditional |\ifchilddoc| distinguishes between the compilation of
child documents and the main document:
%
\begin{center}
|\ifchilddoc |\textit{child-code}| |[|\||else |\textit{main-code}]| \||fi|
\end{center}

%%%%%%%%%%%%%%%%%%%%%%%%%%%%%%%%%%%%%%%%
\DescribeMacro{\childdocname}
\DescribeMacro{\childdocjob}
The macro |\childdocname| contains the filename (without extension)
of the main or child file being processed.
Note that |\childdocjob| will always contain the name of the main file.

%%%%%%%%%%%%%%%%%%%%%%%%%%%%%%%%%%%%%%%%
\paragraph{Title Page.}

Conditional processing can be used to include a title or banner page
in the main document when proper precautions are taken.
Importantly, the code in the main file should ensure that the page counter
(as well as other status parameters which are stored in the |.aux| files)
takes the same value after the conditional processing.
Otherwise the page numbers may take divergent values
depending on which part is compiled.

For example, a title page could be declared by:
%
\begin{center}
\begin{tabular}{l}
|\ifchilddoc\||else|\\
|\addtocounter{page}{-1}|\\
\textit{code for title page}\\
|\newpage|\\
|\||fi|
\end{tabular}
\end{center}
%
A banner page for the child documents can be generated by:
%
\begin{center}
\begin{tabular}{l}
|\ifchilddoc|\\
|\addtocounter{page}{-1}|\\
\textit{code for banner page}\\
|\newpage|\\
|\||fi|
\end{tabular}
\end{center}
%
Here one could write a message such as:
\begin{center}
|This is the part \childdocname{} of \childdocjob{}.|
\end{center}

%%%%%%%%%%%%%%%%%%%%%%%%%%%%%%%%%%%%%%%%%%%%%%%%%%%%%%%%%%%%%%%%%%%%%%%%%%%%%%%%
\subsection{Flags}
\label{sec:flags}

The package makes it easy to generate different versions
of the main or child documents.
To this end compilation flags can be defined
and assigned different default values.
They will be particularly useful in conjunction
with the forwarding mechanism described in \secref{sec:forward}.

For example, it may be useful to have a flag |\version|
which can be set to |draft| or |final|.
The document source will contain some conditional code
depending on the value of |\version|.
Suppose further, the flag should default to |final| for the main file
and to |draft| for child files
which is a natural assignment for editing the document.
This is achieved by placing the following code
in the preamble of the main document
(below the |\childdocmain| directive):
%
\begin{center}
\begin{tabular}{l}
|\ifchilddoc|\\
|\providecommand{\version}{draft}|\\
|\||else|\\
|\providecommand{\version}{final}|\\
|\||fi|
\end{tabular}
\end{center}
%
The definition by |\providecommand| makes sure
that previous definitions are not overwritten.
Further statements |\providecommand{\version}{...}|
can thus be added before the above code to override it.

For the main file, one might add a line
(between |\childdocmain| and the above block)
%
\begin{center}
|%\ifchilddoc\||else\providecommand{\version}{draft}\||fi|
\end{center}
%
which can be uncommented to produce a draft version.
Likewise one can add a line to the very top of a child file
(above the |\childdocof{|\textit{main}|}| directive)
%
\begin{center}
|%\providecommand{\version}{final}|
\end{center}
%
which can be uncommented to produce the final version of this child document.

%%%%%%%%%%%%%%%%%%%%%%%%%%%%%%%%%%%%%%%%%%%%%%%%%%%%%%%%%%%%%%%%%%%%%%%%%%%%%%%%
\subsection{Forwarding}
\label{sec:forward}

Different versions of the main or child documents
using compilation flags as described in \secref{sec:flags}
can be (permanently) stored in different files
for convenient compilation, viewing and distribution.
To this end, the package defines a command
to pass on compilation to a different file:

%%%%%%%%%%%%%%%%%%%%%%%%%%%%%%%%%%%%%%%%
\DescribeMacro{\childdocforward}
The command |\childdocforward| redirects processing to
another source file:
%
\begin{center}
\begin{tabular}{l}
|\input{childdoc.def}|\\
|\childdocforward[|\textit{main}|]{|\textit{dest}|}|\\
\end{tabular}
\end{center}
%
The argument \textit{dest} is the destination file
(without extension).
It should be the main file or one of the child files.
Note that further \textsf{childdoc} directives
such as |\childdocof| and |\childdocforward|
in the indicated file will be processed in this form.
The optional argument \textit{main}
passes on directly to the main file \textit{main}
while pretending to compile the child \textit{dest}.
This form behaves as if \textit{dest}
issues |\childdocof{|\textit{main}|}| right away,
and no further \textsf{childdoc} directives will be processed.

%%%%%%%%%%%%%%%%%%%%%%%%%%%%%%%%%%%%%%%%
\DescribeMacro{\...prefix}
In the alternative form |\childdocforwardprefix|,
%
\begin{center}
\begin{tabular}{l}
|\input{childdoc.def}|\\
|\childdocforwardprefix[|\textit{main}|]{|\textit{prefix}|}{|\textit{dest}|}|
\end{tabular}
\end{center}
%
the destination file is determined by a pattern
depending on the current file:
To make this work, the current file must be called
`{\textit{prefix}\hspace{0.2em}\textit{suffix}}'
with \textit{prefix} matching precisely the argument.
Processing is then passed on to the file
`{\textit{dest}\hspace{0.2em}\textit{suffix}}'.
Surely, the same effect is achieved by
directly specifying the
argument `{\textit{dest}\hspace{0.2em}\textit{suffix}}'
in the first form.
However, that requires to set up a different file
for each child. With the alternative form of the command
all these files can have exactly the same content
which simplifies setting them up and maintaining them.

For example, the following file |draft.tex|
with a compilation flag |\version| as described in \secref{sec:flags}
compiles the main document as a draft:
%
\begin{center}
\begin{tabular}{l}
|\def\version{draft}|\\
|\input{childdoc.def}|\\
|\childdocforward{|\textit{main}|}|
\end{tabular}
\end{center}
%
Likewise, the following files |final|\textit{nn}|.tex|
compile the final version of the child document
|child|\textit{nn}|.tex|:
%
\begin{center}
\begin{tabular}{l}
|\def\version{final}|\\
|\input{childdoc.def}|\\
|\childdocforwardprefix{final}{child}|
\end{tabular}
\end{center}
%

Note that when several versions of a main file and/or of each child file
are to be generated, it may be convenient to set up a |Makefile| or
shell script to automatise the process.

%%%%%%%%%%%%%%%%%%%%%%%%%%%%%%%%%%%%%%%%%%%%%%%%%%%%%%%%%%%%%%%%%%%%%%%%%%%%%%%%
\subsection{Command Line Processing}
\label{sec:commandline}

The effect of redirection files can also be achieved by invoking
the \LaTeX{} compiler with a more elaborate command line.
Most conveniently this should be done as part
of a shell script or a |Makefile|.

When using \textsf{childdoc} in the main file, the following
command lines effectively perform a redirection
(note that depending on the shell being used,
backslashes may have to be doubled: `|\|' $\to$ `|\\|'):
%
\begin{center}
|... -jobname "|\textit{target}|" |\\|"|[\textit{flags}]%
|\input{childdoc.def}\childdocforward[|\textit{main}|]{|\textit{dest}|}"|
\end{center}
%
Here \textit{target} is the name of the output file,
\textit{main} is the name of the main file
and \textit{dest} is the name of the main or child file to be processed
(all filenames without extensions).
The optional argument \textit{main} can be omitted
if \textit{main} matches \textit{dest}.
Optionally, compilation \textit{flags} can be defined via |\def| commands.
This command line makes the \TeX{} engine believe
it is compiling the file \textit{target}
whose content is specified as the latter parameter.
The provided code then forwards the processing to
\textit{main} or \textit{dest} as described in \secref{sec:forward}.

%%%%%%%%%%%%%%%%%%%%%%%%%%%%%%%%%%%%%%%%%%%%%%%%%%%%%%%%%%%%%%%%%%%%%%%%%%%%%%%%
\subsection{Include by Input}
\label{sec:input}

Including child documents by |\include| has some restrictions by design.
Most notably, the content of a child document always occupies
its own set of pages; pages cannot be shared between child documents.
Usually, this behaviour makes perfect sense
because each child document contain an essential part of the document.
However, in some situations it may be desirable to compose
a document from a collection of parts
without having mandatory page breaks between then.
For this case, the package
provides a mechanism to include parts
by |\input| which can also be processed individually.
However, by construction this mechanism
requires manual handling of the content to be output.

%%%%%%%%%%%%%%%%%%%%%%%%%%%%%%%%%%%%%%%%
\DescribeMacro{\ifchilddocmanual}
The main file should be prepared as usual, see \secref{sec:include}.
However, the document body must make a distinction
between processing of an individual part and of the main document, e.g.:
%
\begin{center}
\begin{tabular}{l}
|\ifchilddocmanual|\\
|\input{\childdocname}|\\
|\||else|\\
\textit{document body with }|\input{|\textit{part}|}|\\
|\||fi|
\end{tabular}
\end{center}
%
The conditional |\ifchilddocmanual| is true whenever
a part to be included by |\input| is being compiled,
and the name of the part is stored in |\childdocname|.

%%%%%%%%%%%%%%%%%%%%%%%%%%%%%%%%%%%%%%%%
\DescribeMacro{\childdocby}
Each part to be included by |\input| should start with:
%
\begin{center}
\begin{tabular}{l}
|\input{childdoc.def}|\\
|\childdocby{|\textit{main}|}|\\
\end{tabular}
\end{center}
%
The directive |\childdocby| is similar to |\childdocof|
described in \secref{sec:include},
but the subsequent selection of content must be done manually.
To that end, both |\ifchilddoc| and |\ifchilddocmanual|
will be true upon processing of a part,
and the name of the part is stored in |\childdocname|.
Note that |\jobname| will be set to the filename of the current part
so that each part receives an individual |.aux| file
that does not interfere with the |.aux| file(s) of the main document.
This behaviour can be altered by the alternative form
|\childdocby[*]{|\textit{main}|}| (with a non-empty optional argument)
which uses the |.aux| file of the main document
by setting |\jobname| to \textit{main}.

%%%%%%%%%%%%%%%%%%%%%%%%%%%%%%%%%%%%%%%%%%%%%%%%%%%%%%%%%%%%%%%%%%%%%%%%%%%%%%%%
\subsection{Driver Development}
\label{sec:driver}

The \textsf{childdoc} mechanism can also be use for the development
of definition files such as \LaTeX{} styles or classes.
This case differs from the above setup with multiple parts
included by |\include| in that no |\includeonly| should be invoked.
This can be achieved by starting the include file
(before |\ProvidesPackage|) with:
%
\begin{center}
\begin{tabular}{l}
|\input{childdoc.def}|\\
|\childdocforward{|\textit{main}|}|\\
\end{tabular}
\end{center}
%
or alternatively with:
%
\begin{center}
\begin{tabular}{l}
|\input{childdoc.def}|\\
|\childdocby{|\textit{main}|}|\\
\end{tabular}
\end{center}
%
Both forms have slightly different effects as described above.
The main file is prepared as usual, see \secref{sec:include}.

%%%%%%%%%%%%%%%%%%%%%%%%%%%%%%%%%%%%%%%%%%%%%%%%%%%%%%%%%%%%%%%%%%%%%%%%%%%%%%%%
\subsection{Legacy Detection}
\label{sec:detection}

The directive |\childdocmain| in the main file can detect
whether the complete document or merely a child is to be compiled
even without using the directive |\childdocof|.
This method is deprecated because it is less robust
and there is no compelling reason to use it;
it is merely provided for backward compatibility
and it may be removed in future versions.

If the detection mechanism is to be used,
it is mandatory to correctly specify
the filename of the main file as the argument of |\childdocmain|:
%
\begin{center}
\begin{tabular}{l}
|\input{childdoc.def}|\\
|\childdocmain{|\textit{main}|}|\\
\end{tabular}
\end{center}
%
If |\jobname| does not match the argument \textit{main} of |\childdocmain|,
it is assumed that |\jobname| points to the child file to be compiled.
When using |\childdocmain| with the main file specified as argument,
it suffices to start a child file
with just |\input{|\textit{main}|}|
without loading of the package and using |\childdocof|.
If instead all processing is done
with the appropriate \textsf{childdoc} directives,
the argument of \textit{main} of |\childdocmain| can be empty.

An alternative version of the command line processing described
in \secref{sec:commandline} using the detection mechanism reads:
%
\begin{center}
|... -jobname "|\textit{target}|" "|[\textit{flags}]%
[|\def\jobname{|\textit{dest}|}|]|\input{|\textit{main}|}"|
\end{center}

%%%%%%%%%%%%%%%%%%%%%%%%%%%%%%%%%%%%%%%%%%%%%%%%%%%%%%%%%%%%%%%%%%%%%%%%%%%%%%%%
\subsection{Manual Code}
\label{sec:manual}

In case one cannot be certain whether the definitions file |childdoc.def|
is installed on the target \TeX{} distribution
and one prefers not to ship it,
it is conceivable to paste a few relevant commands into the sources.

To that end, drop all statements |\input{childdoc.def}|
and perform the replacements as outlined below.
Instead of |\childdocmain{|\textit{main}|}| add the following code
to the top of the main file:
%
\begin{center}
\begin{tabular}{l}
|\||ifdefined\childdocname\endinput\||fi\newif\ifchilddoc|\\
|\edef\childdocname{\scantokens\expandafter{\jobname\noexpand}}|\\
|\def\childdocmain{|\textit{main}|}\||ifx\childdocmain\childdocname\||else|\\
|\childdoctrue\includeonly{\childdocname}\let\jobname\childdocmain\||fi|\\
\end{tabular}
\end{center}
%
Instead of |\childdocof{|\textit{main}|}| just include the main file
at the top of each child file:
%
\begin{center}
|\input{|\textit{main}|}|
\end{center}
%
A simple redirection |\childdocforward{|\textit{dest}|}| is achieved by:
%
\begin{center}
|\def\jobname{|\textit{dest}|}\input{\jobname}|
\end{center}
%
The redirection with prefix
|\childdocforwardprefix[|\textit{prefix}|]{|\textit{dest}|}|
is accomplished by:
%
\begin{center}
\begin{tabular}{l}
|{\edef\jobname{\scantokens\expandafter{\jobname\noexpand}}|\\
|\def\redirectjob |\textit{prefix}|#1~~~{\gdef\jobname{|\textit{dest}|#1}}|\\
|\expandafter\redirectjob\jobname~~~}\input{\jobname}|
\end{tabular}
\end{center}

In an alternative approach,
child documents can be compiled by a specific command line
without additional code or specific definitions:
%
\begin{center}
|... -jobname "|\textit{target}|" "|[\textit{flags}]%
|\includeonly{|\textit{dest}|}\input{|\textit{main}|}"|
\end{center}
%

%%%%%%%%%%%%%%%%%%%%%%%%%%%%%%%%%%%%%%%%%%%%%%%%%%%%%%%%%%%%%%%%%%%%%%%%%%%%%%%%
%%%%%%%%%%%%%%%%%%%%%%%%%%%%%%%%%%%%%%%%%%%%%%%%%%%%%%%%%%%%%%%%%%%%%%%%%%%%%%%%
\section{Information}

%%%%%%%%%%%%%%%%%%%%%%%%%%%%%%%%%%%%%%%%%%%%%%%%%%%%%%%%%%%%%%%%%%%%%%%%%%%%%%%%
\subsection{Copyright}

Copyright \copyright{} 2017--2018 Niklas Beisert

This work may be distributed and/or modified under the
conditions of the \LaTeX{} Project Public License, either version 1.3
of this license or (at your option) any later version.
The latest version of this license is in
  \url{http://www.latex-project.org/lppl.txt}
and version 1.3 or later is part of all distributions of \LaTeX{}
version 2005/12/01 or later.

This work has the LPPL maintenance status `maintained'.

The Current Maintainer of this work is Niklas Beisert.

This work consists of the files |README.txt|, |childdoc.ins| and |childdoc.dtx|
as well as the derived files |childdoc.def|, |cdocsamp.tex|
with |cdocsch1.tex|, |cdocsch2.tex|, |cdocspt3.tex|, |cdocspt4.tex|,
|cdocsdrf.tex|, |cdocsfn1.tex|, |cdocsfn2.tex|
as well as |childdoc.pdf|.

%%%%%%%%%%%%%%%%%%%%%%%%%%%%%%%%%%%%%%%%%%%%%%%%%%%%%%%%%%%%%%%%%%%%%%%%%%%%%%%%
\subsection{Files and Installation}

The package consists of the files:
%
\begin{center}
\begin{tabular}{ll}
    |README.txt|   & readme file \\
    |childdoc.ins| & installation file \\
    |childdoc.dtx| & source file \\
    |childdoc.def| & definition file \\
    |cdocsamp.tex| & sample main file \\
    |cdocsch1.tex| & sample include file \\
    |cdocsch2.tex| & sample include file \\
    |cdocspt3.tex| & sample part file \\
    |cdocspt4.tex| & sample part file \\
    |cdocsdrf.tex| & sample redirection file \\
    |cdocsfn1.tex| & sample redirection file \\
    |cdocsfn2.tex| & sample redirection file \\
    |childdoc.pdf| & manual
\end{tabular}
\end{center}
%
The distribution consists of the files
|README.txt|, |childdoc.ins| and |childdoc.dtx|.
%
\begin{itemize}
\item
Run (pdf)\LaTeX{} on |childdoc.dtx|
to compile the manual |childdoc.pdf| (this file).
\item
Run \LaTeX{} on |childdoc.ins| to create the definitions file |childdoc.def|
and the sample |cdocsamp.tex| with include files
|cdocsch1.tex|, |cdocsch2.tex|, |cdocspt3.tex|, |cdocspt4.tex|,
|cdocsdrf.tex|, |cdocsfn1.tex|, |cdocsfn2.tex|.
Then copy the file |childdoc.def| to an appropriate directory of your \LaTeX{}
distribution, e.g.\ \textit{texmf-root}|/tex/latex/childdoc|.
\end{itemize}

%%%%%%%%%%%%%%%%%%%%%%%%%%%%%%%%%%%%%%%%%%%%%%%%%%%%%%%%%%%%%%%%%%%%%%%%%%%%%%%%
\subsection{Related CTAN Packages}

There are several other packages which offer a similar functionality:
%
\begin{itemize}
\item
The packages
\href{http://ctan.org/pkg/docmute}{\textsf{docmute}},
\href{http://ctan.org/pkg/includex}{\textsf{includex}} and
\href{http://ctan.org/pkg/standalone}{\textsf{standalone}}
provide commands to include only the document body of
a child file thus allowing both files to be compiled individually.
\item
The packages \href{http://ctan.org/pkg/subdocs}{\textsf{subdocs}}
and \href{http://ctan.org/pkg/subfiles}{\textsf{subfiles}}
provide structures in which the main and child documents can be
encapsulated and allowing them to be compiled individually.
The inclusion mechanism is different from the conventional |\include|.
\item
The package \href{http://ctan.org/pkg/combine}{\textsf{combine}}
is an elaborate solution to combine several documents into one.
\end{itemize}
%
See also the CTAN topic \href{http://ctan.org/topic/subdocs}{\textsf{subdocs}}
for further related packages.
The present package differs from the above solutions in that
a document structure constructed with the conventional |\include| mechanism
just needs two extra commands at the top of every file
such that all constituent files can be compiled individually.

%%%%%%%%%%%%%%%%%%%%%%%%%%%%%%%%%%%%%%%%%%%%%%%%%%%%%%%%%%%%%%%%%%%%%%%%%%%%%%%%
%\subsection{Feature Suggestions}
%
%The following is a list of features which may be useful for future
%versions of this package:
%%
%\begin{itemize}
%\item
%\ldots
%\end{itemize}

%%%%%%%%%%%%%%%%%%%%%%%%%%%%%%%%%%%%%%%%%%%%%%%%%%%%%%%%%%%%%%%%%%%%%%%%%%%%%%%%
\subsection{Revision History}

%%%%%%%%%%%%%%%%%%%%%%%%%%%%%%%%%%%%%%%%
\paragraph{v2.0:} 2018/12/30

\begin{itemize}
\item
immediate forward processing
\item
added |\childdocby| mechanism
\item
manual restructured
\end{itemize}

%%%%%%%%%%%%%%%%%%%%%%%%%%%%%%%%%%%%%%%%
\paragraph{v1.6:} 2018/01/17

\begin{itemize}
\item
application for development of include files
\item
corrections to manual
\end{itemize}

%%%%%%%%%%%%%%%%%%%%%%%%%%%%%%%%%%%%%%%%
\paragraph{v1.5:} 2017/05/21

\begin{itemize}
\item
more complete structuring introduced
\item
|\childdocof| introduced
\item
|\childdoc| renamed to |\childdocmain|
\item
|\childredirect| renamed to |\childdocforward| and |\childdocforwardprefix|
and functionality expanded
\end{itemize}

%%%%%%%%%%%%%%%%%%%%%%%%%%%%%%%%%%%%%%%%
\paragraph{v1.0:} 2017/04/27

\begin{itemize}
\item
manual and install package
\item
first version published on CTAN
\end{itemize}

%%%%%%%%%%%%%%%%%%%%%%%%%%%%%%%%%%%%%%%%
\paragraph{v0.6:} 2017/04/26

\begin{itemize}
\item
redirection mechanism added
\end{itemize}

%%%%%%%%%%%%%%%%%%%%%%%%%%%%%%%%%%%%%%%%
\paragraph{v0.5:} 2017/04/26

\begin{itemize}
\item
functionality in definition file
\end{itemize}


%%%%%%%%%%%%%%%%%%%%%%%%%%%%%%%%%%%%%%%%%%%%%%%%%%%%%%%%%%%%%%%%%%%%%%%%%%%%%%%%
%%%%%%%%%%%%%%%%%%%%%%%%%%%%%%%%%%%%%%%%%%%%%%%%%%%%%%%%%%%%%%%%%%%%%%%%%%%%%%%%
%%%%%%%%%%%%%%%%%%%%%%%%%%%%%%%%%%%%%%%%%%%%%%%%%%%%%%%%%%%%%%%%%%%%%%%%%%%%%%%%
\appendix

\settowidth\MacroIndent{\rmfamily\scriptsize 000\ }

 \DocInput{childdoc.dtx}

\end{document}
%</driver>
% \fi
%
% %%%%%%%%%%%%%%%%%%%%%%%%%%%%%%%%%%%%%%%%%%%%%%%%%%%%%%%%%%%%%%%%%%%%%%%%%%%%%%
% %%%%%%%%%%%%%%%%%%%%%%%%%%%%%%%%%%%%%%%%%%%%%%%%%%%%%%%%%%%%%%%%%%%%%%%%%%%%%%
% \section{Sample}
%\iffalse
%<*samplemain>
%\fi
%
% The following presents a sample document
% with two chapters, two parts, a title page,
% a compile flag as well as three forwarding files to set the flag.
% It consists of eight |.tex| files:
% \begin{center}
% \begin{tabular}{ll}
% |cdocsamp.tex|&main file\\
% |cdocsch1.tex|&include file for chapter 1\\
% |cdocsch2.tex|&include file for chapter 2\\
% |cdocspt3.tex|&include file for part 3\\
% |cdocspt4.tex|&include file for part 4\\
% |cdocsdrf.tex|&forwarding file for main file in draft mode\\
% |cdocsfi1.tex|&forwarding file for final version of chapter 1\\
% |cdocsfi2.tex|&forwarding file for final version of chapter 2\\
% \end{tabular}
% \end{center}
% Each of the eight files can be compiled directly by the \LaTeX{} compiler.
%
% %%%%%%%%%%%%%%%%%%%%%%%%%%%%%%%%%%%%%%
% \paragraph{Main File.}
%
% The main file is called |cdocsamp.tex|.
%
% Load the \textsf{childdoc} definitions and
% declare the filename for the main document:
%    \begin{macrocode}
\input{childdoc.def}
\childdocmain{}
%    \end{macrocode}

% Optional override for |\version| flag:
%    \begin{macrocode}
%%\ifchilddoc\else\providecommand{\version}{draft}\fi
%    \end{macrocode}

% Define the default values for the |\version| flag
% (|final| for the main file and |draft| for childs):
%    \begin{macrocode}
\ifchilddoc
\providecommand{\version}{draft}
\else
\providecommand{\version}{final}
\fi
%    \end{macrocode}

% Load the standard document class:
%    \begin{macrocode}
\documentclass[12pt]{article}
%    \end{macrocode}

% Start the document body:
%    \begin{macrocode}
\begin{document}
%    \end{macrocode}

% Declare a title page.
% Print title, part of document being processed and version flag:
%    \begin{macrocode}
\addtocounter{page}{-1}
\begin{center}
{\LARGE\bfseries{}childdoc example\par}
\vspace{1cm}
\ifchilddoc
\ifchilddocmanual part\else chapter\fi:
`\childdocname' of `\childdocjob'\par
\else
main document: `\childdocjob'\par
\fi
version: \version\par
\end{center}
\newpage
%    \end{macrocode}

% Manually include selected file,
% otherwise process as usual:
%    \begin{macrocode}
\ifchilddocmanual
\section*{part `\childdocname'}
\input{\childdocname}
\else
%    \end{macrocode}

% Include the two chapters:
%    \begin{macrocode}
\include{cdocsch1}
\include{cdocsch2}
%    \end{macrocode}

% Include the two parts unless only chapters should be displayed:
%    \begin{macrocode}
\ifchilddoc\else
\section{part three}
\input{cdocspt3}
\section{part four}
\input{cdocspt4}
\fi
%    \end{macrocode}

% Process as usual until here:
%    \begin{macrocode}
\fi
%    \end{macrocode}

% End of document body:
%    \begin{macrocode}
\end{document}
%    \end{macrocode}
%\iffalse
%</samplemain>
%\fi
%
% %%%%%%%%%%%%%%%%%%%%%%%%%%%%%%%%%%%%%%
% \paragraph{Chapter Include Files.}
%
% The include files are called |cdocsch1.tex| and |cdocsch2.tex|.
%
%\iffalse
%<*samplechap1|samplechap2>
%\fi

% Optional override for |\version| flag:
%    \begin{macrocode}
%%\providecommand{\version}{final}
%    \end{macrocode}

% Include the main document:
%    \begin{macrocode}
\input{childdoc.def}
\childdocof{cdocsamp}
%    \end{macrocode}

%\iffalse
%</samplechap1|samplechap2>
%\fi
%
%\iffalse
%<*samplechap1>
%\fi
% Some text for chapter 1:
%    \begin{macrocode}
\section{one}
some text in chapter one
%    \end{macrocode}

%\iffalse
%</samplechap1>
%\fi
% Some text for chapter 2:
%\iffalse
%<*samplechap2>
%\fi
%    \begin{macrocode}
\section{two}
more text in chapter two
%    \end{macrocode}

%\iffalse
%</samplechap2>
%\fi
%
% %%%%%%%%%%%%%%%%%%%%%%%%%%%%%%%%%%%%%%
% \paragraph{Part Include Files.}
%
% The include files are called |cdocspt3.tex| and |cdocspt4.tex|.
%
%\iffalse
%<*samplepart3|samplepart4>
%\fi

% Optional override for |\version| flag:
%    \begin{macrocode}
%%\providecommand{\version}{final}
%    \end{macrocode}

% Include the main document:
%    \begin{macrocode}
\input{childdoc.def}
\childdocby{cdocsamp}
%    \end{macrocode}

%\iffalse
%</samplepart3|samplepart4>
%\fi
%
%\iffalse
%<*samplepart3>
%\fi
% Some text for part 3:
%    \begin{macrocode}
some text in part three
%    \end{macrocode}

%\iffalse
%</samplepart3>
%\fi
% Some text for part 4:
%\iffalse
%<*samplepart4>
%\fi
%    \begin{macrocode}
more text in part four
%    \end{macrocode}

%\iffalse
%</samplepart4>
%\fi
%
% %%%%%%%%%%%%%%%%%%%%%%%%%%%%%%%%%%%%%%
% \paragraph{Forwarding for a Complete Draft.}
%
% The following forwarding file |cdocsdrf.tex|
% compiles the main document in draft mode:
%\iffalse
%<*sampledraft>
%\fi
%    \begin{macrocode}
\def\version{draft}
\input{childdoc.def}
\childdocforward{cdocsamp}
%    \end{macrocode}

%\iffalse
%</sampledraft>
%\fi
%
% %%%%%%%%%%%%%%%%%%%%%%%%%%%%%%%%%%%%%%
% \paragraph{Forwarding for Final Version of the Chapters.}
%
% The following forwarding files |cdocsfn1.tex| and |cdocsfn2.tex|
% (with identical content)
% compile the final versions of the child documents
% |cdocsch1.tex| and |cdocsch2.tex|, respectively:
%\iffalse
%<*samplefinal>
%\fi
%    \begin{macrocode}
\def\version{final}
\input{childdoc.def}
\childdocforwardprefix[cdocsamp]{cdocsfn}{cdocsch}
%    \end{macrocode}

%\iffalse
%</samplefinal>
%\fi
%
% %%%%%%%%%%%%%%%%%%%%%%%%%%%%%%%%%%%%%%
% \paragraph{Command Line Processing.}
%
% The following three command lines generate the output files
% |cdocscld|, |cdocscl1| and |cdocscl2|
% which should be identical to
% |cdocsdrf|, |cdocsch1| and |cdocsfn2|, respectively:
% \begin{center}
% \begin{tabular}{l}
% |latex -jobname cdocscld \|\\
% |  "\def\version{draft}\input{childdoc.def}\childdocforward{cdocsamp}"|\\
% |latex -jobname cdocscl1 \|\\
% |  "\input{childdoc.def}\childdocforward[cdocsamp]{cdocsch1}"|\\
% |latex -jobname cdocscl2 \|\\
% |  "\def\version{final}\input{childdoc.def}\childdocforward{cdocsch2}"|
% \end{tabular}
% \end{center}
% Note that the trailing backslash on each first line
% merely continues the input to the second line
% (for convenient cut ant paste).
% Furthermore, the command |latex| can be replaced by any
% of its alternative versions such as |pdflatex|.
%
% %%%%%%%%%%%%%%%%%%%%%%%%%%%%%%%%%%%%%%%%%%%%%%%%%%%%%%%%%%%%%%%%%%%%%%%%%%%%%%
% %%%%%%%%%%%%%%%%%%%%%%%%%%%%%%%%%%%%%%%%%%%%%%%%%%%%%%%%%%%%%%%%%%%%%%%%%%%%%%
% \section{Implementation}
%\iffalse
%<*package>
%\fi
%
% This section describes the definitions file |childdoc.def|.

% The definitions cannot be loaded using |\usepackage| or |\RequirePackage|
% which has a mechanism to prevent loading a style file more than once.
% When loading the definitions by means of |\input|
% multiple instances have to be prevented manually:
%\iffalse
%This code needs to be before the `\ProvidesFile' directive
%which is defined at the beginning of this file.
%Therefore it is also placed there and commented out here.
%</package>
%<*discard>
%\fi
%    \begin{macrocode}
\ifdefined\childdocmain\endinput\fi
%    \end{macrocode}
%\iffalse
%</discard>
%<*package>
%\fi
%
% \macro{\ifchilddoc}
% \macro{\ifchilddocmanual}
% The conditional |\ifchilddoc| tells whether a
% child (true) or main (false) document is being compiled.
% The conditional |\ifchilddocmanual| tells whether
% the |\includeonly| mechanism is used (false) or
% the selection of child files must be performed manually (true).
% The definitions initialise to false:
%    \begin{macrocode}
\newif\ifchilddoc
\newif\ifchilddocmanual
%    \end{macrocode}

% \macro{\childdocname}
% \macro{\childdocjob}
% The macro |\childdocname| stores the name of the main document
% to be compiled. The macro |\childdocjob| stores the name of
% the document on which the \LaTeX{} compiler was originally invoked.
% The content of |\jobname| cannot be compared
% to filenames specified in the source due to different catcodes.
% The following code rescans |\jobname|, stores the result
% in |\childdocname| and saves a copy in |\childdocjob|:
%    \begin{macrocode}
\edef\childdocname{\scantokens\expandafter{\jobname\noexpand}}
\let\childdocjob\childdocname
%    \end{macrocode}

% \macro{\childdocdisable}
% The macro |\childdocdisable| prevents the main file
% from being processed more than once.
% At this stage, the main document command |\childdocmain|
% is assumed to be called once again where it should do nothing.
% Any subsequent call to it should prevent
% a secondary processing of the main document
% It overwrites the forwarding commands
% |\childdocof| and |\childdocforward|
% with empty macros to prevent further inclusions of the main document:
%    \begin{macrocode}
\newcommand{\childdocdisable}
{
  \renewcommand{\childdocmain}[1]{\renewcommand{\childdocmain}[1]{\endinput}}
  \renewcommand{\childdocof}[1]{}
  \renewcommand{\childdocby}[2][]{}
  \renewcommand{\childdocforward}[2][]{}
  \renewcommand{\childdocdisable}{}
}
%    \end{macrocode}

% \macro{\childdocmain}
% The macro |\childdocmain| is to be called at the top of the main file
% with nothing or the main filename (without extension) as argument.
% First, it breaks loops.
% If the argument is not empty and does not match |\childdocname|
% (which is set by the first inclusion of |childdoc.def|),
% |\ifchilddoc| is set to true, |\includeonly| is applied to the child file
% and |\jobname| is set to the main file
% (for proper handling of |.aux| files):
%    \begin{macrocode}
\newcommand{\childdocmain}[1]
{
  \childdocdisable\childdocmain{}
  \if?#1?\else
    \begingroup
      \def\childdoctmp{#1}
      \ifx\childdoctmp\childdocname
        \def\childdoctmp{}
      \else
        \def\childdoctmp
        {
          \childdoctrue
          \includeonly{\childdocname}
          \def\childdocjob{#1}
          \def\jobname{#1}
        }
      \fi
      \expandafter
    \endgroup
    \childdoctmp
  \fi
}
%    \end{macrocode}

% \macro{\childdocof}
% The command |\childdocof| redirects
% compilation to the main file |#1|.
%    \begin{macrocode}
\newcommand{\childdocof}[1]
{
  \childdocdisable
  \childdoctrue
  \includeonly{\childdocname}
  \def\jobname{#1}
  \def\childdocjob{#1}
  \input{#1}
}
%    \end{macrocode}

% \macro{\childdocby}
% The command |\childdocby| ....
%    \begin{macrocode}
\newcommand{\childdocby}[2][]
{
  \childdocdisable
  \childdoctrue
  \childdocmanualtrue
  \if?#1?\else
    \def\jobname{#2}
  \fi
  \def\childdocjob{#2}
  \input{#2}
  \endinput
}
%    \end{macrocode}

% \macro{\childdocforward}
% The command |\childdocforward| redirects
% compilation to the main file or
% (if the optional argument is given) a child file.
% Parameters are set as if the main file
% or a child file starting with |\childdocof| was compiled.
% Then compilation is handed over to the main file:
%    \begin{macrocode}
\newcommand{\childdocforward}[2][]
{
  \begingroup
    \if?#1?
      \def\childdoctmp
      {
        \def\childdocname{#2}
        \def\childdocjob{#2}
        \def\jobname{#2}
        \input{#2}
        \endinput
      }
    \else
      \def\childdoctmp
      {
        \childdocdisable
        \def\childdocname{#2}
        \childdoctrue
        \includeonly{#2}
        \def\childdocjob{#1}
        \def\jobname{#1}
        \input{#1}
        \endinput
      }
    \fi
    \expandafter
  \endgroup
  \childdoctmp
}
%    \end{macrocode}

% \macro{\childdocforwardprefix}
% The command |\childdocforwardprefix| redirects
% compilation to the main or a child file by means of a pattern.
% The prefix |#1| in the current filename is replaced by |#2|
% and the suffix of the current filename is kept
% (it is assumed that the filename does not contain the substring `|~~~|'
% which is used as a delimiter).
% Compilation is handed over to the new file by |\childdocforward|:
%    \begin{macrocode}
\newcommand{\childdocforwardprefix}[3][]
{
  \begingroup
    \def\childdocextract #2##1~~~{\def\childdoctmp{\childdocforward[#1]{#3##1}}}
    \expandafter\childdocextract\childdocname~~~
    \expandafter
  \endgroup
  \childdoctmp
}
%    \end{macrocode}

% \macro{\childdoc}
% The deprecated macro |\childdoc| is a legacy version of |\childdocmain|:
%    \begin{macrocode}
\newcommand{\childdoc}{\childdocmain}
%    \end{macrocode}

% \macro{\childdocredirect}
% The deprecated macro |\childdocredirect| is a legacy version
% of |\childdocforward| and |\childdocforwardprefix|:
%    \begin{macrocode}
\newcommand{\childdocredirect}[2][]
{
  \begingroup
    \if?#1?
      \def\childdoctmp{\childdocforward{#2}}
    \else
      \def\childdoctmp{\childdocforwardprefix{#1}{#2}}
    \fi
    \expandafter
  \endgroup
  \childdoctmp
}
%    \end{macrocode}

%\iffalse
%</package>
%\fi
%
\endinput
\childdocforward{cdocsch2}"|
% \end{tabular}
% \end{center}
% Note that the trailing backslash on each first line
% merely continues the input to the second line
% (for convenient cut ant paste).
% Furthermore, the command |latex| can be replaced by any
% of its alternative versions such as |pdflatex|.
%
% %%%%%%%%%%%%%%%%%%%%%%%%%%%%%%%%%%%%%%%%%%%%%%%%%%%%%%%%%%%%%%%%%%%%%%%%%%%%%%
% %%%%%%%%%%%%%%%%%%%%%%%%%%%%%%%%%%%%%%%%%%%%%%%%%%%%%%%%%%%%%%%%%%%%%%%%%%%%%%
% \section{Implementation}
%\iffalse
%<*package>
%\fi
%
% This section describes the definitions file |childdoc.def|.

% The definitions cannot be loaded using |\usepackage| or |\RequirePackage|
% which has a mechanism to prevent loading a style file more than once.
% When loading the definitions by means of |\input|
% multiple instances have to be prevented manually:
%\iffalse
%This code needs to be before the `\ProvidesFile' directive
%which is defined at the beginning of this file.
%Therefore it is also placed there and commented out here.
%</package>
%<*discard>
%\fi
%    \begin{macrocode}
\ifdefined\childdocmain\endinput\fi
%    \end{macrocode}
%\iffalse
%</discard>
%<*package>
%\fi
%
% \macro{\ifchilddoc}
% \macro{\ifchilddocmanual}
% The conditional |\ifchilddoc| tells whether a
% child (true) or main (false) document is being compiled.
% The conditional |\ifchilddocmanual| tells whether
% the |\includeonly| mechanism is used (false) or
% the selection of child files must be performed manually (true).
% The definitions initialise to false:
%    \begin{macrocode}
\newif\ifchilddoc
\newif\ifchilddocmanual
%    \end{macrocode}

% \macro{\childdocname}
% \macro{\childdocjob}
% The macro |\childdocname| stores the name of the main document
% to be compiled. The macro |\childdocjob| stores the name of
% the document on which the \LaTeX{} compiler was originally invoked.
% The content of |\jobname| cannot be compared
% to filenames specified in the source due to different catcodes.
% The following code rescans |\jobname|, stores the result
% in |\childdocname| and saves a copy in |\childdocjob|:
%    \begin{macrocode}
\edef\childdocname{\scantokens\expandafter{\jobname\noexpand}}
\let\childdocjob\childdocname
%    \end{macrocode}

% \macro{\childdocdisable}
% The macro |\childdocdisable| prevents the main file
% from being processed more than once.
% At this stage, the main document command |\childdocmain|
% is assumed to be called once again where it should do nothing.
% Any subsequent call to it should prevent
% a secondary processing of the main document
% It overwrites the forwarding commands
% |\childdocof| and |\childdocforward|
% with empty macros to prevent further inclusions of the main document:
%    \begin{macrocode}
\newcommand{\childdocdisable}
{
  \renewcommand{\childdocmain}[1]{\renewcommand{\childdocmain}[1]{\endinput}}
  \renewcommand{\childdocof}[1]{}
  \renewcommand{\childdocby}[2][]{}
  \renewcommand{\childdocforward}[2][]{}
  \renewcommand{\childdocdisable}{}
}
%    \end{macrocode}

% \macro{\childdocmain}
% The macro |\childdocmain| is to be called at the top of the main file
% with nothing or the main filename (without extension) as argument.
% First, it breaks loops.
% If the argument is not empty and does not match |\childdocname|
% (which is set by the first inclusion of |childdoc.def|),
% |\ifchilddoc| is set to true, |\includeonly| is applied to the child file
% and |\jobname| is set to the main file
% (for proper handling of |.aux| files):
%    \begin{macrocode}
\newcommand{\childdocmain}[1]
{
  \childdocdisable\childdocmain{}
  \if?#1?\else
    \begingroup
      \def\childdoctmp{#1}
      \ifx\childdoctmp\childdocname
        \def\childdoctmp{}
      \else
        \def\childdoctmp
        {
          \childdoctrue
          \includeonly{\childdocname}
          \def\childdocjob{#1}
          \def\jobname{#1}
        }
      \fi
      \expandafter
    \endgroup
    \childdoctmp
  \fi
}
%    \end{macrocode}

% \macro{\childdocof}
% The command |\childdocof| redirects
% compilation to the main file |#1|.
%    \begin{macrocode}
\newcommand{\childdocof}[1]
{
  \childdocdisable
  \childdoctrue
  \includeonly{\childdocname}
  \def\jobname{#1}
  \def\childdocjob{#1}
  \input{#1}
}
%    \end{macrocode}

% \macro{\childdocby}
% The command |\childdocby| ....
%    \begin{macrocode}
\newcommand{\childdocby}[2][]
{
  \childdocdisable
  \childdoctrue
  \childdocmanualtrue
  \if?#1?\else
    \def\jobname{#2}
  \fi
  \def\childdocjob{#2}
  \input{#2}
  \endinput
}
%    \end{macrocode}

% \macro{\childdocforward}
% The command |\childdocforward| redirects
% compilation to the main file or
% (if the optional argument is given) a child file.
% Parameters are set as if the main file
% or a child file starting with |\childdocof| was compiled.
% Then compilation is handed over to the main file:
%    \begin{macrocode}
\newcommand{\childdocforward}[2][]
{
  \begingroup
    \if?#1?
      \def\childdoctmp
      {
        \def\childdocname{#2}
        \def\childdocjob{#2}
        \def\jobname{#2}
        \input{#2}
        \endinput
      }
    \else
      \def\childdoctmp
      {
        \childdocdisable
        \def\childdocname{#2}
        \childdoctrue
        \includeonly{#2}
        \def\childdocjob{#1}
        \def\jobname{#1}
        \input{#1}
        \endinput
      }
    \fi
    \expandafter
  \endgroup
  \childdoctmp
}
%    \end{macrocode}

% \macro{\childdocforwardprefix}
% The command |\childdocforwardprefix| redirects
% compilation to the main or a child file by means of a pattern.
% The prefix |#1| in the current filename is replaced by |#2|
% and the suffix of the current filename is kept
% (it is assumed that the filename does not contain the substring `|~~~|'
% which is used as a delimiter).
% Compilation is handed over to the new file by |\childdocforward|:
%    \begin{macrocode}
\newcommand{\childdocforwardprefix}[3][]
{
  \begingroup
    \def\childdocextract #2##1~~~{\def\childdoctmp{\childdocforward[#1]{#3##1}}}
    \expandafter\childdocextract\childdocname~~~
    \expandafter
  \endgroup
  \childdoctmp
}
%    \end{macrocode}

% \macro{\childdoc}
% The deprecated macro |\childdoc| is a legacy version of |\childdocmain|:
%    \begin{macrocode}
\newcommand{\childdoc}{\childdocmain}
%    \end{macrocode}

% \macro{\childdocredirect}
% The deprecated macro |\childdocredirect| is a legacy version
% of |\childdocforward| and |\childdocforwardprefix|:
%    \begin{macrocode}
\newcommand{\childdocredirect}[2][]
{
  \begingroup
    \if?#1?
      \def\childdoctmp{\childdocforward{#2}}
    \else
      \def\childdoctmp{\childdocforwardprefix{#1}{#2}}
    \fi
    \expandafter
  \endgroup
  \childdoctmp
}
%    \end{macrocode}

%\iffalse
%</package>
%\fi
%
\endinput
|\\
|\childdocmain{|\textit{main}|}|\\
\end{tabular}
\end{center}
%
If |\jobname| does not match the argument \textit{main} of |\childdocmain|,
it is assumed that |\jobname| points to the child file to be compiled.
When using |\childdocmain| with the main file specified as argument,
it suffices to start a child file
with just |\input{|\textit{main}|}|
without loading of the package and using |\childdocof|.
If instead all processing is done
with the appropriate \textsf{childdoc} directives,
the argument of \textit{main} of |\childdocmain| can be empty.

An alternative version of the command line processing described
in \secref{sec:commandline} using the detection mechanism reads:
%
\begin{center}
|... -jobname "|\textit{target}|" "|[\textit{flags}]%
[|\def\jobname{|\textit{dest}|}|]|\input{|\textit{main}|}"|
\end{center}

%%%%%%%%%%%%%%%%%%%%%%%%%%%%%%%%%%%%%%%%%%%%%%%%%%%%%%%%%%%%%%%%%%%%%%%%%%%%%%%%
\subsection{Manual Code}
\label{sec:manual}

In case one cannot be certain whether the definitions file |childdoc.def|
is installed on the target \TeX{} distribution
and one prefers not to ship it,
it is conceivable to paste a few relevant commands into the sources.

To that end, drop all statements |% \iffalse
%
% childdoc.dtx Copyright (C) 2017-2018 Niklas Beisert
%
% This work may be distributed and/or modified under the
% conditions of the LaTeX Project Public License, either version 1.3
% of this license or (at your option) any later version.
% The latest version of this license is in
%   http://www.latex-project.org/lppl.txt
% and version 1.3 or later is part of all distributions of LaTeX
% version 2005/12/01 or later.
%
% This work has the LPPL maintenance status `maintained'.
%
% The Current Maintainer of this work is Niklas Beisert.
%
% This work consists of the files childdoc.dtx and childdoc.ins
% and the derived files childdoc.def and cdocsamp.tex with
% cdocsch1.tex, cdocsch2.tex, cdocsdrf.tex, cdocsfn1.tex, cdocsfn2.tex.
%
%<package>\ifdefined\childdocmain\endinput\fi
%<package>\ProvidesFile{childdoc.def}[2018/12/30 v2.0 child document driver]
%<samplemain>\ProvidesFile{cdocsamp.tex}[2018/12/30 v2.0 sample for childdoc]
%<*driver>
%\ProvidesFile{childdoc.drv}[2018/12/30 v2.0 childdoc reference manual file]
\PassOptionsToClass{10pt,a4paper}{article}
\documentclass{ltxdoc}

\usepackage[margin=35mm]{geometry}
\usepackage{hyperref}
\usepackage{hyperxmp}
\usepackage[usenames]{color}

\hypersetup{colorlinks=true}
\hypersetup{pdfstartview=FitH}
\hypersetup{pdfpagemode=UseNone}
\hypersetup{pdfsource={}}
\hypersetup{pdflang={en-UK}}
\hypersetup{pdfcopyright={Copyright 2017-2018 Niklas Beisert.
  This work may be distributed and/or modified under the
  conditions of the LaTeX Project Public License, either version 1.3
  of this license or (at your option) any later version.}}
\hypersetup{pdflicenseurl={http://www.latex-project.org/lppl.txt}}
\hypersetup{pdfcontactaddress={ETH Zurich, ITP, HIT K,
  Wolfgang-Pauli-Strasse 27}}
\hypersetup{pdfcontactpostcode={8093}}
\hypersetup{pdfcontactcity={Zurich}}
\hypersetup{pdfcontactcountry={Switzerland}}
\hypersetup{pdfcontactemail={nbeisert@itp.phys.ethz.ch}}
\hypersetup{pdfcontacturl={http://people.phys.ethz.ch/\xmptilde nbeisert/}}

\newcommand{\secref}[1]{\hyperref[#1]{section \ref*{#1}}}

\parskip1ex
\parindent0pt
\let\olditemize\itemize
\def\itemize{\olditemize\parskip0pt}

\begin{document}

\title{The \textsf{childdoc} Package}
\hypersetup{pdftitle={The childdoc Package}}
\author{Niklas Beisert\\[2ex]
  Institut f\"ur Theoretische Physik\\
  Eidgen\"ossische Technische Hochschule Z\"urich\\
  Wolfgang-Pauli-Strasse 27, 8093 Z\"urich, Switzerland\\[1ex]
  \href{mailto:nbeisert@itp.phys.ethz.ch}
  {\texttt{nbeisert@itp.phys.ethz.ch}}}
\hypersetup{pdfauthor={Niklas Beisert}}
\hypersetup{pdfsubject={Manual for the LaTeX2e Package childdoc}}
\date{30 December 2018, \textsf{v2.0}}
\maketitle

\begin{abstract}\noindent
\textsf{childdoc} is a \LaTeXe{} package
that enables the direct compilation
of document sections included by |\include|
to individual files.
\end{abstract}

\begingroup
\parskip0ex
\tableofcontents
\endgroup

%%%%%%%%%%%%%%%%%%%%%%%%%%%%%%%%%%%%%%%%%%%%%%%%%%%%%%%%%%%%%%%%%%%%%%%%%%%%%%%%
%%%%%%%%%%%%%%%%%%%%%%%%%%%%%%%%%%%%%%%%%%%%%%%%%%%%%%%%%%%%%%%%%%%%%%%%%%%%%%%%
\section{Introduction}

\LaTeX{} provides a mechanism to structure a large document (such as a book)
into a main file and several child files (containing the chapters)
using the |\include| command.
This mechanism is beneficial for documents
which span hundreds of pages in order to
make the source file(s) more manageable.
Moreover, compilation can be restricted to
selected child files by means of the |\includeonly| command.
The latter feature can be used to reduce the compilation time while editing
(this was significantly more useful in the earlier days of \LaTeX{})
or to generate a smaller document which is easier to navigate.
Another application of |\includeonly| is to generate
documents consisting of selected parts of the complete document.

However, there are a few drawbacks of the plain |\include| mechanism:
\begin{itemize}
\item
The child files cannot be compiled on their own,
they can only be compiled via the main file.
A naive editing environment
(such as a text editor with an option
to have the current file processed by \LaTeX)
may require one to switch to the main file before compiling;
attempting to compile the child file produces errors.
\item
The main file must be modified (each time)
to adjust the |\includeonly| command
to the present needs. This easily leaves the main file in a messy state.
\item
The generated document will always carry the filename
of the main document. This is inconvenient if
several child files are to be compiled and
to be kept for distribution.
\end{itemize}

The present package provides a simple interface
to make child files individually compilable by \LaTeX{}.
Compiling a child file then has the same effect as compiling
the main file with an |\includeonly| command
to select the appropriate child.
Moreover the generated document will carry the name of the child
rather than the main file.
This resolves all three above issues.

This feature is meant to make the editing of books,
thesis documents and lecture notes somewhat more convenient.
However, the package can also be used efficiently for
composing a series of documents (such as exercise sheets)
which are typically distributed individually.
It then assists the author in generating the individual documents
(potentially in different versions)
as well as a document containing the collected series.
Another application is in developing style files
or other kinds of included material
where compilation of the style file could redirect
to a sample or test file.

%%%%%%%%%%%%%%%%%%%%%%%%%%%%%%%%%%%%%%%%%%%%%%%%%%%%%%%%%%%%%%%%%%%%%%%%%%%%%%%%
%%%%%%%%%%%%%%%%%%%%%%%%%%%%%%%%%%%%%%%%%%%%%%%%%%%%%%%%%%%%%%%%%%%%%%%%%%%%%%%%
\section{Usage}

First of all, the package \textsf{childdoc} is \emph{not} a standard
\LaTeXe{} |.sty| style file! Therefore it needs to be invoked in
a non-standard way.

%%%%%%%%%%%%%%%%%%%%%%%%%%%%%%%%%%%%%%%%%%%%%%%%%%%%%%%%%%%%%%%%%%%%%%%%%%%%%%%%
\subsection{Included Files}
\label{sec:include}

%%%%%%%%%%%%%%%%%%%%%%%%%%%%%%%%%%%%%%%%
\DescribeMacro{\childdocmain}
To use the package, add the commands
\begin{center}
\begin{tabular}{l}
|% \iffalse
%
% childdoc.dtx Copyright (C) 2017-2018 Niklas Beisert
%
% This work may be distributed and/or modified under the
% conditions of the LaTeX Project Public License, either version 1.3
% of this license or (at your option) any later version.
% The latest version of this license is in
%   http://www.latex-project.org/lppl.txt
% and version 1.3 or later is part of all distributions of LaTeX
% version 2005/12/01 or later.
%
% This work has the LPPL maintenance status `maintained'.
%
% The Current Maintainer of this work is Niklas Beisert.
%
% This work consists of the files childdoc.dtx and childdoc.ins
% and the derived files childdoc.def and cdocsamp.tex with
% cdocsch1.tex, cdocsch2.tex, cdocsdrf.tex, cdocsfn1.tex, cdocsfn2.tex.
%
%<package>\ifdefined\childdocmain\endinput\fi
%<package>\ProvidesFile{childdoc.def}[2018/12/30 v2.0 child document driver]
%<samplemain>\ProvidesFile{cdocsamp.tex}[2018/12/30 v2.0 sample for childdoc]
%<*driver>
%\ProvidesFile{childdoc.drv}[2018/12/30 v2.0 childdoc reference manual file]
\PassOptionsToClass{10pt,a4paper}{article}
\documentclass{ltxdoc}

\usepackage[margin=35mm]{geometry}
\usepackage{hyperref}
\usepackage{hyperxmp}
\usepackage[usenames]{color}

\hypersetup{colorlinks=true}
\hypersetup{pdfstartview=FitH}
\hypersetup{pdfpagemode=UseNone}
\hypersetup{pdfsource={}}
\hypersetup{pdflang={en-UK}}
\hypersetup{pdfcopyright={Copyright 2017-2018 Niklas Beisert.
  This work may be distributed and/or modified under the
  conditions of the LaTeX Project Public License, either version 1.3
  of this license or (at your option) any later version.}}
\hypersetup{pdflicenseurl={http://www.latex-project.org/lppl.txt}}
\hypersetup{pdfcontactaddress={ETH Zurich, ITP, HIT K,
  Wolfgang-Pauli-Strasse 27}}
\hypersetup{pdfcontactpostcode={8093}}
\hypersetup{pdfcontactcity={Zurich}}
\hypersetup{pdfcontactcountry={Switzerland}}
\hypersetup{pdfcontactemail={nbeisert@itp.phys.ethz.ch}}
\hypersetup{pdfcontacturl={http://people.phys.ethz.ch/\xmptilde nbeisert/}}

\newcommand{\secref}[1]{\hyperref[#1]{section \ref*{#1}}}

\parskip1ex
\parindent0pt
\let\olditemize\itemize
\def\itemize{\olditemize\parskip0pt}

\begin{document}

\title{The \textsf{childdoc} Package}
\hypersetup{pdftitle={The childdoc Package}}
\author{Niklas Beisert\\[2ex]
  Institut f\"ur Theoretische Physik\\
  Eidgen\"ossische Technische Hochschule Z\"urich\\
  Wolfgang-Pauli-Strasse 27, 8093 Z\"urich, Switzerland\\[1ex]
  \href{mailto:nbeisert@itp.phys.ethz.ch}
  {\texttt{nbeisert@itp.phys.ethz.ch}}}
\hypersetup{pdfauthor={Niklas Beisert}}
\hypersetup{pdfsubject={Manual for the LaTeX2e Package childdoc}}
\date{30 December 2018, \textsf{v2.0}}
\maketitle

\begin{abstract}\noindent
\textsf{childdoc} is a \LaTeXe{} package
that enables the direct compilation
of document sections included by |\include|
to individual files.
\end{abstract}

\begingroup
\parskip0ex
\tableofcontents
\endgroup

%%%%%%%%%%%%%%%%%%%%%%%%%%%%%%%%%%%%%%%%%%%%%%%%%%%%%%%%%%%%%%%%%%%%%%%%%%%%%%%%
%%%%%%%%%%%%%%%%%%%%%%%%%%%%%%%%%%%%%%%%%%%%%%%%%%%%%%%%%%%%%%%%%%%%%%%%%%%%%%%%
\section{Introduction}

\LaTeX{} provides a mechanism to structure a large document (such as a book)
into a main file and several child files (containing the chapters)
using the |\include| command.
This mechanism is beneficial for documents
which span hundreds of pages in order to
make the source file(s) more manageable.
Moreover, compilation can be restricted to
selected child files by means of the |\includeonly| command.
The latter feature can be used to reduce the compilation time while editing
(this was significantly more useful in the earlier days of \LaTeX{})
or to generate a smaller document which is easier to navigate.
Another application of |\includeonly| is to generate
documents consisting of selected parts of the complete document.

However, there are a few drawbacks of the plain |\include| mechanism:
\begin{itemize}
\item
The child files cannot be compiled on their own,
they can only be compiled via the main file.
A naive editing environment
(such as a text editor with an option
to have the current file processed by \LaTeX)
may require one to switch to the main file before compiling;
attempting to compile the child file produces errors.
\item
The main file must be modified (each time)
to adjust the |\includeonly| command
to the present needs. This easily leaves the main file in a messy state.
\item
The generated document will always carry the filename
of the main document. This is inconvenient if
several child files are to be compiled and
to be kept for distribution.
\end{itemize}

The present package provides a simple interface
to make child files individually compilable by \LaTeX{}.
Compiling a child file then has the same effect as compiling
the main file with an |\includeonly| command
to select the appropriate child.
Moreover the generated document will carry the name of the child
rather than the main file.
This resolves all three above issues.

This feature is meant to make the editing of books,
thesis documents and lecture notes somewhat more convenient.
However, the package can also be used efficiently for
composing a series of documents (such as exercise sheets)
which are typically distributed individually.
It then assists the author in generating the individual documents
(potentially in different versions)
as well as a document containing the collected series.
Another application is in developing style files
or other kinds of included material
where compilation of the style file could redirect
to a sample or test file.

%%%%%%%%%%%%%%%%%%%%%%%%%%%%%%%%%%%%%%%%%%%%%%%%%%%%%%%%%%%%%%%%%%%%%%%%%%%%%%%%
%%%%%%%%%%%%%%%%%%%%%%%%%%%%%%%%%%%%%%%%%%%%%%%%%%%%%%%%%%%%%%%%%%%%%%%%%%%%%%%%
\section{Usage}

First of all, the package \textsf{childdoc} is \emph{not} a standard
\LaTeXe{} |.sty| style file! Therefore it needs to be invoked in
a non-standard way.

%%%%%%%%%%%%%%%%%%%%%%%%%%%%%%%%%%%%%%%%%%%%%%%%%%%%%%%%%%%%%%%%%%%%%%%%%%%%%%%%
\subsection{Included Files}
\label{sec:include}

%%%%%%%%%%%%%%%%%%%%%%%%%%%%%%%%%%%%%%%%
\DescribeMacro{\childdocmain}
To use the package, add the commands
\begin{center}
\begin{tabular}{l}
|\input{childdoc.def}|\\
|\childdocmain{}|\\
\end{tabular}
\end{center}
at the very top of the main \LaTeX{} file,
in particular \emph{before} the |\documentclass| statement!
The argument of |\childdocmain| should be left empty
(but it must be present).

%%%%%%%%%%%%%%%%%%%%%%%%%%%%%%%%%%%%%%%%
\DescribeMacro{\childdocof}
Furthermore, add the commands
\begin{center}
\begin{tabular}{l}
|\input{childdoc.def}|\\
|\childdocof{|\textit{main}|}|\\
\end{tabular}
\end{center}
at the top of every child file \textit{child}
which is included by |\include{|\textit{child}|}|
from within the main file
(or at least for those files to be compiled individually).
The argument \textit{main} must be the filename of the main file.

There are a couple of
considerations in setting up the main and child documents:

%%%%%%%%%%%%%%%%%%%%%%%%%%%%%%%%%%%%%%%%
\paragraph{Restrictions.}

Please note the following restrictions:
\begin{itemize}
\item
|\childdocmain| must be called with one argument \textit{main}
to ensure compatibility with earlier version of the package.
It must either be empty (|\childdocmain{}|)
or precisely match the filename of the main file in which it is specified.
See \secref{sec:detection} for further information.
\item
The filename \textit{main} must be specified without the |.tex| extension.
\item
The filename \textit{main} is case sensitive
(even in case-insensitive file systems)
due to internal string comparison.
\item
The argument \textit{main} should be fully expanded, it cannot be a macro.
\item
Subdirectories and special characters should be avoided in filenames.
\item
The command |\childdocmain{|\textit{main}|}| must be followed by a whitespace.
It should not be followed immediately by another command
or by a comment mark `|%|'.
This is because the \TeX{} parser reads the token immediately following
the argument of |\childdocmain| and puts it
at the beginning of every child section;
however, a white\-space is ignored.
\end{itemize}

%%%%%%%%%%%%%%%%%%%%%%%%%%%%%%%%%%%%%%%%
\paragraph{Content of Main File.}

It is advisable to place all content in the child files included by |\include|.
Any output contained in the main file will appear in all child documents
unless suppressed manually;
it cannot be suppressed automatically by the |\includeonly| directive
and thus should normally be avoided.
A method to include some content in the main file
by means of conditional processing is described in \secref{sec:conditional}.

%%%%%%%%%%%%%%%%%%%%%%%%%%%%%%%%%%%%%%%%
\paragraph{Page Numbering.}

When only a part of the document is compiled,
the appropriate numbering of pages
(as well as other status parameters)
is determined from the |.aux| files.
The latter contain information from previous passes.
However this information needs to propagate through
all intermediate child documents.
Therefore the page numbering in child documents may well
be inconsistent until the complete document is compiled at least once.

A useful (if unconventional) way to always ensure a consistent
page numbering is to restart the numbering in each child document
and denote the pages by `\textit{child}|.|\textit{page}'
where \textit{child} represents the chapter/section number of the child file.
This can be achieved by the command
|\numberwithin{page}{|\textit{child}|}|
of the \textsf{amsmath} package
where \textit{child} can be |chapter| or |section|
depending on the chosen structuring.
Alternatively, one can modify the macro |\thepage| appropriately
and reset the counter |page| at the start of each child file.

%%%%%%%%%%%%%%%%%%%%%%%%%%%%%%%%%%%%%%%%%%%%%%%%%%%%%%%%%%%%%%%%%%%%%%%%%%%%%%%%
\subsection{Conditional Processing}
\label{sec:conditional}

The package provides a mechanism to compile different versions
of a document. To customise the versions further some conditional processing
can come in handy to distinguish which version is being compiled.
The package provides two macros to describe the compilation context:

%%%%%%%%%%%%%%%%%%%%%%%%%%%%%%%%%%%%%%%%
\DescribeMacro{\ifchilddoc}
The conditional |\ifchilddoc| distinguishes between the compilation of
child documents and the main document:
%
\begin{center}
|\ifchilddoc |\textit{child-code}| |[|\||else |\textit{main-code}]| \||fi|
\end{center}

%%%%%%%%%%%%%%%%%%%%%%%%%%%%%%%%%%%%%%%%
\DescribeMacro{\childdocname}
\DescribeMacro{\childdocjob}
The macro |\childdocname| contains the filename (without extension)
of the main or child file being processed.
Note that |\childdocjob| will always contain the name of the main file.

%%%%%%%%%%%%%%%%%%%%%%%%%%%%%%%%%%%%%%%%
\paragraph{Title Page.}

Conditional processing can be used to include a title or banner page
in the main document when proper precautions are taken.
Importantly, the code in the main file should ensure that the page counter
(as well as other status parameters which are stored in the |.aux| files)
takes the same value after the conditional processing.
Otherwise the page numbers may take divergent values
depending on which part is compiled.

For example, a title page could be declared by:
%
\begin{center}
\begin{tabular}{l}
|\ifchilddoc\||else|\\
|\addtocounter{page}{-1}|\\
\textit{code for title page}\\
|\newpage|\\
|\||fi|
\end{tabular}
\end{center}
%
A banner page for the child documents can be generated by:
%
\begin{center}
\begin{tabular}{l}
|\ifchilddoc|\\
|\addtocounter{page}{-1}|\\
\textit{code for banner page}\\
|\newpage|\\
|\||fi|
\end{tabular}
\end{center}
%
Here one could write a message such as:
\begin{center}
|This is the part \childdocname{} of \childdocjob{}.|
\end{center}

%%%%%%%%%%%%%%%%%%%%%%%%%%%%%%%%%%%%%%%%%%%%%%%%%%%%%%%%%%%%%%%%%%%%%%%%%%%%%%%%
\subsection{Flags}
\label{sec:flags}

The package makes it easy to generate different versions
of the main or child documents.
To this end compilation flags can be defined
and assigned different default values.
They will be particularly useful in conjunction
with the forwarding mechanism described in \secref{sec:forward}.

For example, it may be useful to have a flag |\version|
which can be set to |draft| or |final|.
The document source will contain some conditional code
depending on the value of |\version|.
Suppose further, the flag should default to |final| for the main file
and to |draft| for child files
which is a natural assignment for editing the document.
This is achieved by placing the following code
in the preamble of the main document
(below the |\childdocmain| directive):
%
\begin{center}
\begin{tabular}{l}
|\ifchilddoc|\\
|\providecommand{\version}{draft}|\\
|\||else|\\
|\providecommand{\version}{final}|\\
|\||fi|
\end{tabular}
\end{center}
%
The definition by |\providecommand| makes sure
that previous definitions are not overwritten.
Further statements |\providecommand{\version}{...}|
can thus be added before the above code to override it.

For the main file, one might add a line
(between |\childdocmain| and the above block)
%
\begin{center}
|%\ifchilddoc\||else\providecommand{\version}{draft}\||fi|
\end{center}
%
which can be uncommented to produce a draft version.
Likewise one can add a line to the very top of a child file
(above the |\childdocof{|\textit{main}|}| directive)
%
\begin{center}
|%\providecommand{\version}{final}|
\end{center}
%
which can be uncommented to produce the final version of this child document.

%%%%%%%%%%%%%%%%%%%%%%%%%%%%%%%%%%%%%%%%%%%%%%%%%%%%%%%%%%%%%%%%%%%%%%%%%%%%%%%%
\subsection{Forwarding}
\label{sec:forward}

Different versions of the main or child documents
using compilation flags as described in \secref{sec:flags}
can be (permanently) stored in different files
for convenient compilation, viewing and distribution.
To this end, the package defines a command
to pass on compilation to a different file:

%%%%%%%%%%%%%%%%%%%%%%%%%%%%%%%%%%%%%%%%
\DescribeMacro{\childdocforward}
The command |\childdocforward| redirects processing to
another source file:
%
\begin{center}
\begin{tabular}{l}
|\input{childdoc.def}|\\
|\childdocforward[|\textit{main}|]{|\textit{dest}|}|\\
\end{tabular}
\end{center}
%
The argument \textit{dest} is the destination file
(without extension).
It should be the main file or one of the child files.
Note that further \textsf{childdoc} directives
such as |\childdocof| and |\childdocforward|
in the indicated file will be processed in this form.
The optional argument \textit{main}
passes on directly to the main file \textit{main}
while pretending to compile the child \textit{dest}.
This form behaves as if \textit{dest}
issues |\childdocof{|\textit{main}|}| right away,
and no further \textsf{childdoc} directives will be processed.

%%%%%%%%%%%%%%%%%%%%%%%%%%%%%%%%%%%%%%%%
\DescribeMacro{\...prefix}
In the alternative form |\childdocforwardprefix|,
%
\begin{center}
\begin{tabular}{l}
|\input{childdoc.def}|\\
|\childdocforwardprefix[|\textit{main}|]{|\textit{prefix}|}{|\textit{dest}|}|
\end{tabular}
\end{center}
%
the destination file is determined by a pattern
depending on the current file:
To make this work, the current file must be called
`{\textit{prefix}\hspace{0.2em}\textit{suffix}}'
with \textit{prefix} matching precisely the argument.
Processing is then passed on to the file
`{\textit{dest}\hspace{0.2em}\textit{suffix}}'.
Surely, the same effect is achieved by
directly specifying the
argument `{\textit{dest}\hspace{0.2em}\textit{suffix}}'
in the first form.
However, that requires to set up a different file
for each child. With the alternative form of the command
all these files can have exactly the same content
which simplifies setting them up and maintaining them.

For example, the following file |draft.tex|
with a compilation flag |\version| as described in \secref{sec:flags}
compiles the main document as a draft:
%
\begin{center}
\begin{tabular}{l}
|\def\version{draft}|\\
|\input{childdoc.def}|\\
|\childdocforward{|\textit{main}|}|
\end{tabular}
\end{center}
%
Likewise, the following files |final|\textit{nn}|.tex|
compile the final version of the child document
|child|\textit{nn}|.tex|:
%
\begin{center}
\begin{tabular}{l}
|\def\version{final}|\\
|\input{childdoc.def}|\\
|\childdocforwardprefix{final}{child}|
\end{tabular}
\end{center}
%

Note that when several versions of a main file and/or of each child file
are to be generated, it may be convenient to set up a |Makefile| or
shell script to automatise the process.

%%%%%%%%%%%%%%%%%%%%%%%%%%%%%%%%%%%%%%%%%%%%%%%%%%%%%%%%%%%%%%%%%%%%%%%%%%%%%%%%
\subsection{Command Line Processing}
\label{sec:commandline}

The effect of redirection files can also be achieved by invoking
the \LaTeX{} compiler with a more elaborate command line.
Most conveniently this should be done as part
of a shell script or a |Makefile|.

When using \textsf{childdoc} in the main file, the following
command lines effectively perform a redirection
(note that depending on the shell being used,
backslashes may have to be doubled: `|\|' $\to$ `|\\|'):
%
\begin{center}
|... -jobname "|\textit{target}|" |\\|"|[\textit{flags}]%
|\input{childdoc.def}\childdocforward[|\textit{main}|]{|\textit{dest}|}"|
\end{center}
%
Here \textit{target} is the name of the output file,
\textit{main} is the name of the main file
and \textit{dest} is the name of the main or child file to be processed
(all filenames without extensions).
The optional argument \textit{main} can be omitted
if \textit{main} matches \textit{dest}.
Optionally, compilation \textit{flags} can be defined via |\def| commands.
This command line makes the \TeX{} engine believe
it is compiling the file \textit{target}
whose content is specified as the latter parameter.
The provided code then forwards the processing to
\textit{main} or \textit{dest} as described in \secref{sec:forward}.

%%%%%%%%%%%%%%%%%%%%%%%%%%%%%%%%%%%%%%%%%%%%%%%%%%%%%%%%%%%%%%%%%%%%%%%%%%%%%%%%
\subsection{Include by Input}
\label{sec:input}

Including child documents by |\include| has some restrictions by design.
Most notably, the content of a child document always occupies
its own set of pages; pages cannot be shared between child documents.
Usually, this behaviour makes perfect sense
because each child document contain an essential part of the document.
However, in some situations it may be desirable to compose
a document from a collection of parts
without having mandatory page breaks between then.
For this case, the package
provides a mechanism to include parts
by |\input| which can also be processed individually.
However, by construction this mechanism
requires manual handling of the content to be output.

%%%%%%%%%%%%%%%%%%%%%%%%%%%%%%%%%%%%%%%%
\DescribeMacro{\ifchilddocmanual}
The main file should be prepared as usual, see \secref{sec:include}.
However, the document body must make a distinction
between processing of an individual part and of the main document, e.g.:
%
\begin{center}
\begin{tabular}{l}
|\ifchilddocmanual|\\
|\input{\childdocname}|\\
|\||else|\\
\textit{document body with }|\input{|\textit{part}|}|\\
|\||fi|
\end{tabular}
\end{center}
%
The conditional |\ifchilddocmanual| is true whenever
a part to be included by |\input| is being compiled,
and the name of the part is stored in |\childdocname|.

%%%%%%%%%%%%%%%%%%%%%%%%%%%%%%%%%%%%%%%%
\DescribeMacro{\childdocby}
Each part to be included by |\input| should start with:
%
\begin{center}
\begin{tabular}{l}
|\input{childdoc.def}|\\
|\childdocby{|\textit{main}|}|\\
\end{tabular}
\end{center}
%
The directive |\childdocby| is similar to |\childdocof|
described in \secref{sec:include},
but the subsequent selection of content must be done manually.
To that end, both |\ifchilddoc| and |\ifchilddocmanual|
will be true upon processing of a part,
and the name of the part is stored in |\childdocname|.
Note that |\jobname| will be set to the filename of the current part
so that each part receives an individual |.aux| file
that does not interfere with the |.aux| file(s) of the main document.
This behaviour can be altered by the alternative form
|\childdocby[*]{|\textit{main}|}| (with a non-empty optional argument)
which uses the |.aux| file of the main document
by setting |\jobname| to \textit{main}.

%%%%%%%%%%%%%%%%%%%%%%%%%%%%%%%%%%%%%%%%%%%%%%%%%%%%%%%%%%%%%%%%%%%%%%%%%%%%%%%%
\subsection{Driver Development}
\label{sec:driver}

The \textsf{childdoc} mechanism can also be use for the development
of definition files such as \LaTeX{} styles or classes.
This case differs from the above setup with multiple parts
included by |\include| in that no |\includeonly| should be invoked.
This can be achieved by starting the include file
(before |\ProvidesPackage|) with:
%
\begin{center}
\begin{tabular}{l}
|\input{childdoc.def}|\\
|\childdocforward{|\textit{main}|}|\\
\end{tabular}
\end{center}
%
or alternatively with:
%
\begin{center}
\begin{tabular}{l}
|\input{childdoc.def}|\\
|\childdocby{|\textit{main}|}|\\
\end{tabular}
\end{center}
%
Both forms have slightly different effects as described above.
The main file is prepared as usual, see \secref{sec:include}.

%%%%%%%%%%%%%%%%%%%%%%%%%%%%%%%%%%%%%%%%%%%%%%%%%%%%%%%%%%%%%%%%%%%%%%%%%%%%%%%%
\subsection{Legacy Detection}
\label{sec:detection}

The directive |\childdocmain| in the main file can detect
whether the complete document or merely a child is to be compiled
even without using the directive |\childdocof|.
This method is deprecated because it is less robust
and there is no compelling reason to use it;
it is merely provided for backward compatibility
and it may be removed in future versions.

If the detection mechanism is to be used,
it is mandatory to correctly specify
the filename of the main file as the argument of |\childdocmain|:
%
\begin{center}
\begin{tabular}{l}
|\input{childdoc.def}|\\
|\childdocmain{|\textit{main}|}|\\
\end{tabular}
\end{center}
%
If |\jobname| does not match the argument \textit{main} of |\childdocmain|,
it is assumed that |\jobname| points to the child file to be compiled.
When using |\childdocmain| with the main file specified as argument,
it suffices to start a child file
with just |\input{|\textit{main}|}|
without loading of the package and using |\childdocof|.
If instead all processing is done
with the appropriate \textsf{childdoc} directives,
the argument of \textit{main} of |\childdocmain| can be empty.

An alternative version of the command line processing described
in \secref{sec:commandline} using the detection mechanism reads:
%
\begin{center}
|... -jobname "|\textit{target}|" "|[\textit{flags}]%
[|\def\jobname{|\textit{dest}|}|]|\input{|\textit{main}|}"|
\end{center}

%%%%%%%%%%%%%%%%%%%%%%%%%%%%%%%%%%%%%%%%%%%%%%%%%%%%%%%%%%%%%%%%%%%%%%%%%%%%%%%%
\subsection{Manual Code}
\label{sec:manual}

In case one cannot be certain whether the definitions file |childdoc.def|
is installed on the target \TeX{} distribution
and one prefers not to ship it,
it is conceivable to paste a few relevant commands into the sources.

To that end, drop all statements |\input{childdoc.def}|
and perform the replacements as outlined below.
Instead of |\childdocmain{|\textit{main}|}| add the following code
to the top of the main file:
%
\begin{center}
\begin{tabular}{l}
|\||ifdefined\childdocname\endinput\||fi\newif\ifchilddoc|\\
|\edef\childdocname{\scantokens\expandafter{\jobname\noexpand}}|\\
|\def\childdocmain{|\textit{main}|}\||ifx\childdocmain\childdocname\||else|\\
|\childdoctrue\includeonly{\childdocname}\let\jobname\childdocmain\||fi|\\
\end{tabular}
\end{center}
%
Instead of |\childdocof{|\textit{main}|}| just include the main file
at the top of each child file:
%
\begin{center}
|\input{|\textit{main}|}|
\end{center}
%
A simple redirection |\childdocforward{|\textit{dest}|}| is achieved by:
%
\begin{center}
|\def\jobname{|\textit{dest}|}\input{\jobname}|
\end{center}
%
The redirection with prefix
|\childdocforwardprefix[|\textit{prefix}|]{|\textit{dest}|}|
is accomplished by:
%
\begin{center}
\begin{tabular}{l}
|{\edef\jobname{\scantokens\expandafter{\jobname\noexpand}}|\\
|\def\redirectjob |\textit{prefix}|#1~~~{\gdef\jobname{|\textit{dest}|#1}}|\\
|\expandafter\redirectjob\jobname~~~}\input{\jobname}|
\end{tabular}
\end{center}

In an alternative approach,
child documents can be compiled by a specific command line
without additional code or specific definitions:
%
\begin{center}
|... -jobname "|\textit{target}|" "|[\textit{flags}]%
|\includeonly{|\textit{dest}|}\input{|\textit{main}|}"|
\end{center}
%

%%%%%%%%%%%%%%%%%%%%%%%%%%%%%%%%%%%%%%%%%%%%%%%%%%%%%%%%%%%%%%%%%%%%%%%%%%%%%%%%
%%%%%%%%%%%%%%%%%%%%%%%%%%%%%%%%%%%%%%%%%%%%%%%%%%%%%%%%%%%%%%%%%%%%%%%%%%%%%%%%
\section{Information}

%%%%%%%%%%%%%%%%%%%%%%%%%%%%%%%%%%%%%%%%%%%%%%%%%%%%%%%%%%%%%%%%%%%%%%%%%%%%%%%%
\subsection{Copyright}

Copyright \copyright{} 2017--2018 Niklas Beisert

This work may be distributed and/or modified under the
conditions of the \LaTeX{} Project Public License, either version 1.3
of this license or (at your option) any later version.
The latest version of this license is in
  \url{http://www.latex-project.org/lppl.txt}
and version 1.3 or later is part of all distributions of \LaTeX{}
version 2005/12/01 or later.

This work has the LPPL maintenance status `maintained'.

The Current Maintainer of this work is Niklas Beisert.

This work consists of the files |README.txt|, |childdoc.ins| and |childdoc.dtx|
as well as the derived files |childdoc.def|, |cdocsamp.tex|
with |cdocsch1.tex|, |cdocsch2.tex|, |cdocspt3.tex|, |cdocspt4.tex|,
|cdocsdrf.tex|, |cdocsfn1.tex|, |cdocsfn2.tex|
as well as |childdoc.pdf|.

%%%%%%%%%%%%%%%%%%%%%%%%%%%%%%%%%%%%%%%%%%%%%%%%%%%%%%%%%%%%%%%%%%%%%%%%%%%%%%%%
\subsection{Files and Installation}

The package consists of the files:
%
\begin{center}
\begin{tabular}{ll}
    |README.txt|   & readme file \\
    |childdoc.ins| & installation file \\
    |childdoc.dtx| & source file \\
    |childdoc.def| & definition file \\
    |cdocsamp.tex| & sample main file \\
    |cdocsch1.tex| & sample include file \\
    |cdocsch2.tex| & sample include file \\
    |cdocspt3.tex| & sample part file \\
    |cdocspt4.tex| & sample part file \\
    |cdocsdrf.tex| & sample redirection file \\
    |cdocsfn1.tex| & sample redirection file \\
    |cdocsfn2.tex| & sample redirection file \\
    |childdoc.pdf| & manual
\end{tabular}
\end{center}
%
The distribution consists of the files
|README.txt|, |childdoc.ins| and |childdoc.dtx|.
%
\begin{itemize}
\item
Run (pdf)\LaTeX{} on |childdoc.dtx|
to compile the manual |childdoc.pdf| (this file).
\item
Run \LaTeX{} on |childdoc.ins| to create the definitions file |childdoc.def|
and the sample |cdocsamp.tex| with include files
|cdocsch1.tex|, |cdocsch2.tex|, |cdocspt3.tex|, |cdocspt4.tex|,
|cdocsdrf.tex|, |cdocsfn1.tex|, |cdocsfn2.tex|.
Then copy the file |childdoc.def| to an appropriate directory of your \LaTeX{}
distribution, e.g.\ \textit{texmf-root}|/tex/latex/childdoc|.
\end{itemize}

%%%%%%%%%%%%%%%%%%%%%%%%%%%%%%%%%%%%%%%%%%%%%%%%%%%%%%%%%%%%%%%%%%%%%%%%%%%%%%%%
\subsection{Related CTAN Packages}

There are several other packages which offer a similar functionality:
%
\begin{itemize}
\item
The packages
\href{http://ctan.org/pkg/docmute}{\textsf{docmute}},
\href{http://ctan.org/pkg/includex}{\textsf{includex}} and
\href{http://ctan.org/pkg/standalone}{\textsf{standalone}}
provide commands to include only the document body of
a child file thus allowing both files to be compiled individually.
\item
The packages \href{http://ctan.org/pkg/subdocs}{\textsf{subdocs}}
and \href{http://ctan.org/pkg/subfiles}{\textsf{subfiles}}
provide structures in which the main and child documents can be
encapsulated and allowing them to be compiled individually.
The inclusion mechanism is different from the conventional |\include|.
\item
The package \href{http://ctan.org/pkg/combine}{\textsf{combine}}
is an elaborate solution to combine several documents into one.
\end{itemize}
%
See also the CTAN topic \href{http://ctan.org/topic/subdocs}{\textsf{subdocs}}
for further related packages.
The present package differs from the above solutions in that
a document structure constructed with the conventional |\include| mechanism
just needs two extra commands at the top of every file
such that all constituent files can be compiled individually.

%%%%%%%%%%%%%%%%%%%%%%%%%%%%%%%%%%%%%%%%%%%%%%%%%%%%%%%%%%%%%%%%%%%%%%%%%%%%%%%%
%\subsection{Feature Suggestions}
%
%The following is a list of features which may be useful for future
%versions of this package:
%%
%\begin{itemize}
%\item
%\ldots
%\end{itemize}

%%%%%%%%%%%%%%%%%%%%%%%%%%%%%%%%%%%%%%%%%%%%%%%%%%%%%%%%%%%%%%%%%%%%%%%%%%%%%%%%
\subsection{Revision History}

%%%%%%%%%%%%%%%%%%%%%%%%%%%%%%%%%%%%%%%%
\paragraph{v2.0:} 2018/12/30

\begin{itemize}
\item
immediate forward processing
\item
added |\childdocby| mechanism
\item
manual restructured
\end{itemize}

%%%%%%%%%%%%%%%%%%%%%%%%%%%%%%%%%%%%%%%%
\paragraph{v1.6:} 2018/01/17

\begin{itemize}
\item
application for development of include files
\item
corrections to manual
\end{itemize}

%%%%%%%%%%%%%%%%%%%%%%%%%%%%%%%%%%%%%%%%
\paragraph{v1.5:} 2017/05/21

\begin{itemize}
\item
more complete structuring introduced
\item
|\childdocof| introduced
\item
|\childdoc| renamed to |\childdocmain|
\item
|\childredirect| renamed to |\childdocforward| and |\childdocforwardprefix|
and functionality expanded
\end{itemize}

%%%%%%%%%%%%%%%%%%%%%%%%%%%%%%%%%%%%%%%%
\paragraph{v1.0:} 2017/04/27

\begin{itemize}
\item
manual and install package
\item
first version published on CTAN
\end{itemize}

%%%%%%%%%%%%%%%%%%%%%%%%%%%%%%%%%%%%%%%%
\paragraph{v0.6:} 2017/04/26

\begin{itemize}
\item
redirection mechanism added
\end{itemize}

%%%%%%%%%%%%%%%%%%%%%%%%%%%%%%%%%%%%%%%%
\paragraph{v0.5:} 2017/04/26

\begin{itemize}
\item
functionality in definition file
\end{itemize}


%%%%%%%%%%%%%%%%%%%%%%%%%%%%%%%%%%%%%%%%%%%%%%%%%%%%%%%%%%%%%%%%%%%%%%%%%%%%%%%%
%%%%%%%%%%%%%%%%%%%%%%%%%%%%%%%%%%%%%%%%%%%%%%%%%%%%%%%%%%%%%%%%%%%%%%%%%%%%%%%%
%%%%%%%%%%%%%%%%%%%%%%%%%%%%%%%%%%%%%%%%%%%%%%%%%%%%%%%%%%%%%%%%%%%%%%%%%%%%%%%%
\appendix

\settowidth\MacroIndent{\rmfamily\scriptsize 000\ }

 \DocInput{childdoc.dtx}

\end{document}
%</driver>
% \fi
%
% %%%%%%%%%%%%%%%%%%%%%%%%%%%%%%%%%%%%%%%%%%%%%%%%%%%%%%%%%%%%%%%%%%%%%%%%%%%%%%
% %%%%%%%%%%%%%%%%%%%%%%%%%%%%%%%%%%%%%%%%%%%%%%%%%%%%%%%%%%%%%%%%%%%%%%%%%%%%%%
% \section{Sample}
%\iffalse
%<*samplemain>
%\fi
%
% The following presents a sample document
% with two chapters, two parts, a title page,
% a compile flag as well as three forwarding files to set the flag.
% It consists of eight |.tex| files:
% \begin{center}
% \begin{tabular}{ll}
% |cdocsamp.tex|&main file\\
% |cdocsch1.tex|&include file for chapter 1\\
% |cdocsch2.tex|&include file for chapter 2\\
% |cdocspt3.tex|&include file for part 3\\
% |cdocspt4.tex|&include file for part 4\\
% |cdocsdrf.tex|&forwarding file for main file in draft mode\\
% |cdocsfi1.tex|&forwarding file for final version of chapter 1\\
% |cdocsfi2.tex|&forwarding file for final version of chapter 2\\
% \end{tabular}
% \end{center}
% Each of the eight files can be compiled directly by the \LaTeX{} compiler.
%
% %%%%%%%%%%%%%%%%%%%%%%%%%%%%%%%%%%%%%%
% \paragraph{Main File.}
%
% The main file is called |cdocsamp.tex|.
%
% Load the \textsf{childdoc} definitions and
% declare the filename for the main document:
%    \begin{macrocode}
\input{childdoc.def}
\childdocmain{}
%    \end{macrocode}

% Optional override for |\version| flag:
%    \begin{macrocode}
%%\ifchilddoc\else\providecommand{\version}{draft}\fi
%    \end{macrocode}

% Define the default values for the |\version| flag
% (|final| for the main file and |draft| for childs):
%    \begin{macrocode}
\ifchilddoc
\providecommand{\version}{draft}
\else
\providecommand{\version}{final}
\fi
%    \end{macrocode}

% Load the standard document class:
%    \begin{macrocode}
\documentclass[12pt]{article}
%    \end{macrocode}

% Start the document body:
%    \begin{macrocode}
\begin{document}
%    \end{macrocode}

% Declare a title page.
% Print title, part of document being processed and version flag:
%    \begin{macrocode}
\addtocounter{page}{-1}
\begin{center}
{\LARGE\bfseries{}childdoc example\par}
\vspace{1cm}
\ifchilddoc
\ifchilddocmanual part\else chapter\fi:
`\childdocname' of `\childdocjob'\par
\else
main document: `\childdocjob'\par
\fi
version: \version\par
\end{center}
\newpage
%    \end{macrocode}

% Manually include selected file,
% otherwise process as usual:
%    \begin{macrocode}
\ifchilddocmanual
\section*{part `\childdocname'}
\input{\childdocname}
\else
%    \end{macrocode}

% Include the two chapters:
%    \begin{macrocode}
\include{cdocsch1}
\include{cdocsch2}
%    \end{macrocode}

% Include the two parts unless only chapters should be displayed:
%    \begin{macrocode}
\ifchilddoc\else
\section{part three}
\input{cdocspt3}
\section{part four}
\input{cdocspt4}
\fi
%    \end{macrocode}

% Process as usual until here:
%    \begin{macrocode}
\fi
%    \end{macrocode}

% End of document body:
%    \begin{macrocode}
\end{document}
%    \end{macrocode}
%\iffalse
%</samplemain>
%\fi
%
% %%%%%%%%%%%%%%%%%%%%%%%%%%%%%%%%%%%%%%
% \paragraph{Chapter Include Files.}
%
% The include files are called |cdocsch1.tex| and |cdocsch2.tex|.
%
%\iffalse
%<*samplechap1|samplechap2>
%\fi

% Optional override for |\version| flag:
%    \begin{macrocode}
%%\providecommand{\version}{final}
%    \end{macrocode}

% Include the main document:
%    \begin{macrocode}
\input{childdoc.def}
\childdocof{cdocsamp}
%    \end{macrocode}

%\iffalse
%</samplechap1|samplechap2>
%\fi
%
%\iffalse
%<*samplechap1>
%\fi
% Some text for chapter 1:
%    \begin{macrocode}
\section{one}
some text in chapter one
%    \end{macrocode}

%\iffalse
%</samplechap1>
%\fi
% Some text for chapter 2:
%\iffalse
%<*samplechap2>
%\fi
%    \begin{macrocode}
\section{two}
more text in chapter two
%    \end{macrocode}

%\iffalse
%</samplechap2>
%\fi
%
% %%%%%%%%%%%%%%%%%%%%%%%%%%%%%%%%%%%%%%
% \paragraph{Part Include Files.}
%
% The include files are called |cdocspt3.tex| and |cdocspt4.tex|.
%
%\iffalse
%<*samplepart3|samplepart4>
%\fi

% Optional override for |\version| flag:
%    \begin{macrocode}
%%\providecommand{\version}{final}
%    \end{macrocode}

% Include the main document:
%    \begin{macrocode}
\input{childdoc.def}
\childdocby{cdocsamp}
%    \end{macrocode}

%\iffalse
%</samplepart3|samplepart4>
%\fi
%
%\iffalse
%<*samplepart3>
%\fi
% Some text for part 3:
%    \begin{macrocode}
some text in part three
%    \end{macrocode}

%\iffalse
%</samplepart3>
%\fi
% Some text for part 4:
%\iffalse
%<*samplepart4>
%\fi
%    \begin{macrocode}
more text in part four
%    \end{macrocode}

%\iffalse
%</samplepart4>
%\fi
%
% %%%%%%%%%%%%%%%%%%%%%%%%%%%%%%%%%%%%%%
% \paragraph{Forwarding for a Complete Draft.}
%
% The following forwarding file |cdocsdrf.tex|
% compiles the main document in draft mode:
%\iffalse
%<*sampledraft>
%\fi
%    \begin{macrocode}
\def\version{draft}
\input{childdoc.def}
\childdocforward{cdocsamp}
%    \end{macrocode}

%\iffalse
%</sampledraft>
%\fi
%
% %%%%%%%%%%%%%%%%%%%%%%%%%%%%%%%%%%%%%%
% \paragraph{Forwarding for Final Version of the Chapters.}
%
% The following forwarding files |cdocsfn1.tex| and |cdocsfn2.tex|
% (with identical content)
% compile the final versions of the child documents
% |cdocsch1.tex| and |cdocsch2.tex|, respectively:
%\iffalse
%<*samplefinal>
%\fi
%    \begin{macrocode}
\def\version{final}
\input{childdoc.def}
\childdocforwardprefix[cdocsamp]{cdocsfn}{cdocsch}
%    \end{macrocode}

%\iffalse
%</samplefinal>
%\fi
%
% %%%%%%%%%%%%%%%%%%%%%%%%%%%%%%%%%%%%%%
% \paragraph{Command Line Processing.}
%
% The following three command lines generate the output files
% |cdocscld|, |cdocscl1| and |cdocscl2|
% which should be identical to
% |cdocsdrf|, |cdocsch1| and |cdocsfn2|, respectively:
% \begin{center}
% \begin{tabular}{l}
% |latex -jobname cdocscld \|\\
% |  "\def\version{draft}\input{childdoc.def}\childdocforward{cdocsamp}"|\\
% |latex -jobname cdocscl1 \|\\
% |  "\input{childdoc.def}\childdocforward[cdocsamp]{cdocsch1}"|\\
% |latex -jobname cdocscl2 \|\\
% |  "\def\version{final}\input{childdoc.def}\childdocforward{cdocsch2}"|
% \end{tabular}
% \end{center}
% Note that the trailing backslash on each first line
% merely continues the input to the second line
% (for convenient cut ant paste).
% Furthermore, the command |latex| can be replaced by any
% of its alternative versions such as |pdflatex|.
%
% %%%%%%%%%%%%%%%%%%%%%%%%%%%%%%%%%%%%%%%%%%%%%%%%%%%%%%%%%%%%%%%%%%%%%%%%%%%%%%
% %%%%%%%%%%%%%%%%%%%%%%%%%%%%%%%%%%%%%%%%%%%%%%%%%%%%%%%%%%%%%%%%%%%%%%%%%%%%%%
% \section{Implementation}
%\iffalse
%<*package>
%\fi
%
% This section describes the definitions file |childdoc.def|.

% The definitions cannot be loaded using |\usepackage| or |\RequirePackage|
% which has a mechanism to prevent loading a style file more than once.
% When loading the definitions by means of |\input|
% multiple instances have to be prevented manually:
%\iffalse
%This code needs to be before the `\ProvidesFile' directive
%which is defined at the beginning of this file.
%Therefore it is also placed there and commented out here.
%</package>
%<*discard>
%\fi
%    \begin{macrocode}
\ifdefined\childdocmain\endinput\fi
%    \end{macrocode}
%\iffalse
%</discard>
%<*package>
%\fi
%
% \macro{\ifchilddoc}
% \macro{\ifchilddocmanual}
% The conditional |\ifchilddoc| tells whether a
% child (true) or main (false) document is being compiled.
% The conditional |\ifchilddocmanual| tells whether
% the |\includeonly| mechanism is used (false) or
% the selection of child files must be performed manually (true).
% The definitions initialise to false:
%    \begin{macrocode}
\newif\ifchilddoc
\newif\ifchilddocmanual
%    \end{macrocode}

% \macro{\childdocname}
% \macro{\childdocjob}
% The macro |\childdocname| stores the name of the main document
% to be compiled. The macro |\childdocjob| stores the name of
% the document on which the \LaTeX{} compiler was originally invoked.
% The content of |\jobname| cannot be compared
% to filenames specified in the source due to different catcodes.
% The following code rescans |\jobname|, stores the result
% in |\childdocname| and saves a copy in |\childdocjob|:
%    \begin{macrocode}
\edef\childdocname{\scantokens\expandafter{\jobname\noexpand}}
\let\childdocjob\childdocname
%    \end{macrocode}

% \macro{\childdocdisable}
% The macro |\childdocdisable| prevents the main file
% from being processed more than once.
% At this stage, the main document command |\childdocmain|
% is assumed to be called once again where it should do nothing.
% Any subsequent call to it should prevent
% a secondary processing of the main document
% It overwrites the forwarding commands
% |\childdocof| and |\childdocforward|
% with empty macros to prevent further inclusions of the main document:
%    \begin{macrocode}
\newcommand{\childdocdisable}
{
  \renewcommand{\childdocmain}[1]{\renewcommand{\childdocmain}[1]{\endinput}}
  \renewcommand{\childdocof}[1]{}
  \renewcommand{\childdocby}[2][]{}
  \renewcommand{\childdocforward}[2][]{}
  \renewcommand{\childdocdisable}{}
}
%    \end{macrocode}

% \macro{\childdocmain}
% The macro |\childdocmain| is to be called at the top of the main file
% with nothing or the main filename (without extension) as argument.
% First, it breaks loops.
% If the argument is not empty and does not match |\childdocname|
% (which is set by the first inclusion of |childdoc.def|),
% |\ifchilddoc| is set to true, |\includeonly| is applied to the child file
% and |\jobname| is set to the main file
% (for proper handling of |.aux| files):
%    \begin{macrocode}
\newcommand{\childdocmain}[1]
{
  \childdocdisable\childdocmain{}
  \if?#1?\else
    \begingroup
      \def\childdoctmp{#1}
      \ifx\childdoctmp\childdocname
        \def\childdoctmp{}
      \else
        \def\childdoctmp
        {
          \childdoctrue
          \includeonly{\childdocname}
          \def\childdocjob{#1}
          \def\jobname{#1}
        }
      \fi
      \expandafter
    \endgroup
    \childdoctmp
  \fi
}
%    \end{macrocode}

% \macro{\childdocof}
% The command |\childdocof| redirects
% compilation to the main file |#1|.
%    \begin{macrocode}
\newcommand{\childdocof}[1]
{
  \childdocdisable
  \childdoctrue
  \includeonly{\childdocname}
  \def\jobname{#1}
  \def\childdocjob{#1}
  \input{#1}
}
%    \end{macrocode}

% \macro{\childdocby}
% The command |\childdocby| ....
%    \begin{macrocode}
\newcommand{\childdocby}[2][]
{
  \childdocdisable
  \childdoctrue
  \childdocmanualtrue
  \if?#1?\else
    \def\jobname{#2}
  \fi
  \def\childdocjob{#2}
  \input{#2}
  \endinput
}
%    \end{macrocode}

% \macro{\childdocforward}
% The command |\childdocforward| redirects
% compilation to the main file or
% (if the optional argument is given) a child file.
% Parameters are set as if the main file
% or a child file starting with |\childdocof| was compiled.
% Then compilation is handed over to the main file:
%    \begin{macrocode}
\newcommand{\childdocforward}[2][]
{
  \begingroup
    \if?#1?
      \def\childdoctmp
      {
        \def\childdocname{#2}
        \def\childdocjob{#2}
        \def\jobname{#2}
        \input{#2}
        \endinput
      }
    \else
      \def\childdoctmp
      {
        \childdocdisable
        \def\childdocname{#2}
        \childdoctrue
        \includeonly{#2}
        \def\childdocjob{#1}
        \def\jobname{#1}
        \input{#1}
        \endinput
      }
    \fi
    \expandafter
  \endgroup
  \childdoctmp
}
%    \end{macrocode}

% \macro{\childdocforwardprefix}
% The command |\childdocforwardprefix| redirects
% compilation to the main or a child file by means of a pattern.
% The prefix |#1| in the current filename is replaced by |#2|
% and the suffix of the current filename is kept
% (it is assumed that the filename does not contain the substring `|~~~|'
% which is used as a delimiter).
% Compilation is handed over to the new file by |\childdocforward|:
%    \begin{macrocode}
\newcommand{\childdocforwardprefix}[3][]
{
  \begingroup
    \def\childdocextract #2##1~~~{\def\childdoctmp{\childdocforward[#1]{#3##1}}}
    \expandafter\childdocextract\childdocname~~~
    \expandafter
  \endgroup
  \childdoctmp
}
%    \end{macrocode}

% \macro{\childdoc}
% The deprecated macro |\childdoc| is a legacy version of |\childdocmain|:
%    \begin{macrocode}
\newcommand{\childdoc}{\childdocmain}
%    \end{macrocode}

% \macro{\childdocredirect}
% The deprecated macro |\childdocredirect| is a legacy version
% of |\childdocforward| and |\childdocforwardprefix|:
%    \begin{macrocode}
\newcommand{\childdocredirect}[2][]
{
  \begingroup
    \if?#1?
      \def\childdoctmp{\childdocforward{#2}}
    \else
      \def\childdoctmp{\childdocforwardprefix{#1}{#2}}
    \fi
    \expandafter
  \endgroup
  \childdoctmp
}
%    \end{macrocode}

%\iffalse
%</package>
%\fi
%
\endinput
|\\
|\childdocmain{}|\\
\end{tabular}
\end{center}
at the very top of the main \LaTeX{} file,
in particular \emph{before} the |\documentclass| statement!
The argument of |\childdocmain| should be left empty
(but it must be present).

%%%%%%%%%%%%%%%%%%%%%%%%%%%%%%%%%%%%%%%%
\DescribeMacro{\childdocof}
Furthermore, add the commands
\begin{center}
\begin{tabular}{l}
|% \iffalse
%
% childdoc.dtx Copyright (C) 2017-2018 Niklas Beisert
%
% This work may be distributed and/or modified under the
% conditions of the LaTeX Project Public License, either version 1.3
% of this license or (at your option) any later version.
% The latest version of this license is in
%   http://www.latex-project.org/lppl.txt
% and version 1.3 or later is part of all distributions of LaTeX
% version 2005/12/01 or later.
%
% This work has the LPPL maintenance status `maintained'.
%
% The Current Maintainer of this work is Niklas Beisert.
%
% This work consists of the files childdoc.dtx and childdoc.ins
% and the derived files childdoc.def and cdocsamp.tex with
% cdocsch1.tex, cdocsch2.tex, cdocsdrf.tex, cdocsfn1.tex, cdocsfn2.tex.
%
%<package>\ifdefined\childdocmain\endinput\fi
%<package>\ProvidesFile{childdoc.def}[2018/12/30 v2.0 child document driver]
%<samplemain>\ProvidesFile{cdocsamp.tex}[2018/12/30 v2.0 sample for childdoc]
%<*driver>
%\ProvidesFile{childdoc.drv}[2018/12/30 v2.0 childdoc reference manual file]
\PassOptionsToClass{10pt,a4paper}{article}
\documentclass{ltxdoc}

\usepackage[margin=35mm]{geometry}
\usepackage{hyperref}
\usepackage{hyperxmp}
\usepackage[usenames]{color}

\hypersetup{colorlinks=true}
\hypersetup{pdfstartview=FitH}
\hypersetup{pdfpagemode=UseNone}
\hypersetup{pdfsource={}}
\hypersetup{pdflang={en-UK}}
\hypersetup{pdfcopyright={Copyright 2017-2018 Niklas Beisert.
  This work may be distributed and/or modified under the
  conditions of the LaTeX Project Public License, either version 1.3
  of this license or (at your option) any later version.}}
\hypersetup{pdflicenseurl={http://www.latex-project.org/lppl.txt}}
\hypersetup{pdfcontactaddress={ETH Zurich, ITP, HIT K,
  Wolfgang-Pauli-Strasse 27}}
\hypersetup{pdfcontactpostcode={8093}}
\hypersetup{pdfcontactcity={Zurich}}
\hypersetup{pdfcontactcountry={Switzerland}}
\hypersetup{pdfcontactemail={nbeisert@itp.phys.ethz.ch}}
\hypersetup{pdfcontacturl={http://people.phys.ethz.ch/\xmptilde nbeisert/}}

\newcommand{\secref}[1]{\hyperref[#1]{section \ref*{#1}}}

\parskip1ex
\parindent0pt
\let\olditemize\itemize
\def\itemize{\olditemize\parskip0pt}

\begin{document}

\title{The \textsf{childdoc} Package}
\hypersetup{pdftitle={The childdoc Package}}
\author{Niklas Beisert\\[2ex]
  Institut f\"ur Theoretische Physik\\
  Eidgen\"ossische Technische Hochschule Z\"urich\\
  Wolfgang-Pauli-Strasse 27, 8093 Z\"urich, Switzerland\\[1ex]
  \href{mailto:nbeisert@itp.phys.ethz.ch}
  {\texttt{nbeisert@itp.phys.ethz.ch}}}
\hypersetup{pdfauthor={Niklas Beisert}}
\hypersetup{pdfsubject={Manual for the LaTeX2e Package childdoc}}
\date{30 December 2018, \textsf{v2.0}}
\maketitle

\begin{abstract}\noindent
\textsf{childdoc} is a \LaTeXe{} package
that enables the direct compilation
of document sections included by |\include|
to individual files.
\end{abstract}

\begingroup
\parskip0ex
\tableofcontents
\endgroup

%%%%%%%%%%%%%%%%%%%%%%%%%%%%%%%%%%%%%%%%%%%%%%%%%%%%%%%%%%%%%%%%%%%%%%%%%%%%%%%%
%%%%%%%%%%%%%%%%%%%%%%%%%%%%%%%%%%%%%%%%%%%%%%%%%%%%%%%%%%%%%%%%%%%%%%%%%%%%%%%%
\section{Introduction}

\LaTeX{} provides a mechanism to structure a large document (such as a book)
into a main file and several child files (containing the chapters)
using the |\include| command.
This mechanism is beneficial for documents
which span hundreds of pages in order to
make the source file(s) more manageable.
Moreover, compilation can be restricted to
selected child files by means of the |\includeonly| command.
The latter feature can be used to reduce the compilation time while editing
(this was significantly more useful in the earlier days of \LaTeX{})
or to generate a smaller document which is easier to navigate.
Another application of |\includeonly| is to generate
documents consisting of selected parts of the complete document.

However, there are a few drawbacks of the plain |\include| mechanism:
\begin{itemize}
\item
The child files cannot be compiled on their own,
they can only be compiled via the main file.
A naive editing environment
(such as a text editor with an option
to have the current file processed by \LaTeX)
may require one to switch to the main file before compiling;
attempting to compile the child file produces errors.
\item
The main file must be modified (each time)
to adjust the |\includeonly| command
to the present needs. This easily leaves the main file in a messy state.
\item
The generated document will always carry the filename
of the main document. This is inconvenient if
several child files are to be compiled and
to be kept for distribution.
\end{itemize}

The present package provides a simple interface
to make child files individually compilable by \LaTeX{}.
Compiling a child file then has the same effect as compiling
the main file with an |\includeonly| command
to select the appropriate child.
Moreover the generated document will carry the name of the child
rather than the main file.
This resolves all three above issues.

This feature is meant to make the editing of books,
thesis documents and lecture notes somewhat more convenient.
However, the package can also be used efficiently for
composing a series of documents (such as exercise sheets)
which are typically distributed individually.
It then assists the author in generating the individual documents
(potentially in different versions)
as well as a document containing the collected series.
Another application is in developing style files
or other kinds of included material
where compilation of the style file could redirect
to a sample or test file.

%%%%%%%%%%%%%%%%%%%%%%%%%%%%%%%%%%%%%%%%%%%%%%%%%%%%%%%%%%%%%%%%%%%%%%%%%%%%%%%%
%%%%%%%%%%%%%%%%%%%%%%%%%%%%%%%%%%%%%%%%%%%%%%%%%%%%%%%%%%%%%%%%%%%%%%%%%%%%%%%%
\section{Usage}

First of all, the package \textsf{childdoc} is \emph{not} a standard
\LaTeXe{} |.sty| style file! Therefore it needs to be invoked in
a non-standard way.

%%%%%%%%%%%%%%%%%%%%%%%%%%%%%%%%%%%%%%%%%%%%%%%%%%%%%%%%%%%%%%%%%%%%%%%%%%%%%%%%
\subsection{Included Files}
\label{sec:include}

%%%%%%%%%%%%%%%%%%%%%%%%%%%%%%%%%%%%%%%%
\DescribeMacro{\childdocmain}
To use the package, add the commands
\begin{center}
\begin{tabular}{l}
|\input{childdoc.def}|\\
|\childdocmain{}|\\
\end{tabular}
\end{center}
at the very top of the main \LaTeX{} file,
in particular \emph{before} the |\documentclass| statement!
The argument of |\childdocmain| should be left empty
(but it must be present).

%%%%%%%%%%%%%%%%%%%%%%%%%%%%%%%%%%%%%%%%
\DescribeMacro{\childdocof}
Furthermore, add the commands
\begin{center}
\begin{tabular}{l}
|\input{childdoc.def}|\\
|\childdocof{|\textit{main}|}|\\
\end{tabular}
\end{center}
at the top of every child file \textit{child}
which is included by |\include{|\textit{child}|}|
from within the main file
(or at least for those files to be compiled individually).
The argument \textit{main} must be the filename of the main file.

There are a couple of
considerations in setting up the main and child documents:

%%%%%%%%%%%%%%%%%%%%%%%%%%%%%%%%%%%%%%%%
\paragraph{Restrictions.}

Please note the following restrictions:
\begin{itemize}
\item
|\childdocmain| must be called with one argument \textit{main}
to ensure compatibility with earlier version of the package.
It must either be empty (|\childdocmain{}|)
or precisely match the filename of the main file in which it is specified.
See \secref{sec:detection} for further information.
\item
The filename \textit{main} must be specified without the |.tex| extension.
\item
The filename \textit{main} is case sensitive
(even in case-insensitive file systems)
due to internal string comparison.
\item
The argument \textit{main} should be fully expanded, it cannot be a macro.
\item
Subdirectories and special characters should be avoided in filenames.
\item
The command |\childdocmain{|\textit{main}|}| must be followed by a whitespace.
It should not be followed immediately by another command
or by a comment mark `|%|'.
This is because the \TeX{} parser reads the token immediately following
the argument of |\childdocmain| and puts it
at the beginning of every child section;
however, a white\-space is ignored.
\end{itemize}

%%%%%%%%%%%%%%%%%%%%%%%%%%%%%%%%%%%%%%%%
\paragraph{Content of Main File.}

It is advisable to place all content in the child files included by |\include|.
Any output contained in the main file will appear in all child documents
unless suppressed manually;
it cannot be suppressed automatically by the |\includeonly| directive
and thus should normally be avoided.
A method to include some content in the main file
by means of conditional processing is described in \secref{sec:conditional}.

%%%%%%%%%%%%%%%%%%%%%%%%%%%%%%%%%%%%%%%%
\paragraph{Page Numbering.}

When only a part of the document is compiled,
the appropriate numbering of pages
(as well as other status parameters)
is determined from the |.aux| files.
The latter contain information from previous passes.
However this information needs to propagate through
all intermediate child documents.
Therefore the page numbering in child documents may well
be inconsistent until the complete document is compiled at least once.

A useful (if unconventional) way to always ensure a consistent
page numbering is to restart the numbering in each child document
and denote the pages by `\textit{child}|.|\textit{page}'
where \textit{child} represents the chapter/section number of the child file.
This can be achieved by the command
|\numberwithin{page}{|\textit{child}|}|
of the \textsf{amsmath} package
where \textit{child} can be |chapter| or |section|
depending on the chosen structuring.
Alternatively, one can modify the macro |\thepage| appropriately
and reset the counter |page| at the start of each child file.

%%%%%%%%%%%%%%%%%%%%%%%%%%%%%%%%%%%%%%%%%%%%%%%%%%%%%%%%%%%%%%%%%%%%%%%%%%%%%%%%
\subsection{Conditional Processing}
\label{sec:conditional}

The package provides a mechanism to compile different versions
of a document. To customise the versions further some conditional processing
can come in handy to distinguish which version is being compiled.
The package provides two macros to describe the compilation context:

%%%%%%%%%%%%%%%%%%%%%%%%%%%%%%%%%%%%%%%%
\DescribeMacro{\ifchilddoc}
The conditional |\ifchilddoc| distinguishes between the compilation of
child documents and the main document:
%
\begin{center}
|\ifchilddoc |\textit{child-code}| |[|\||else |\textit{main-code}]| \||fi|
\end{center}

%%%%%%%%%%%%%%%%%%%%%%%%%%%%%%%%%%%%%%%%
\DescribeMacro{\childdocname}
\DescribeMacro{\childdocjob}
The macro |\childdocname| contains the filename (without extension)
of the main or child file being processed.
Note that |\childdocjob| will always contain the name of the main file.

%%%%%%%%%%%%%%%%%%%%%%%%%%%%%%%%%%%%%%%%
\paragraph{Title Page.}

Conditional processing can be used to include a title or banner page
in the main document when proper precautions are taken.
Importantly, the code in the main file should ensure that the page counter
(as well as other status parameters which are stored in the |.aux| files)
takes the same value after the conditional processing.
Otherwise the page numbers may take divergent values
depending on which part is compiled.

For example, a title page could be declared by:
%
\begin{center}
\begin{tabular}{l}
|\ifchilddoc\||else|\\
|\addtocounter{page}{-1}|\\
\textit{code for title page}\\
|\newpage|\\
|\||fi|
\end{tabular}
\end{center}
%
A banner page for the child documents can be generated by:
%
\begin{center}
\begin{tabular}{l}
|\ifchilddoc|\\
|\addtocounter{page}{-1}|\\
\textit{code for banner page}\\
|\newpage|\\
|\||fi|
\end{tabular}
\end{center}
%
Here one could write a message such as:
\begin{center}
|This is the part \childdocname{} of \childdocjob{}.|
\end{center}

%%%%%%%%%%%%%%%%%%%%%%%%%%%%%%%%%%%%%%%%%%%%%%%%%%%%%%%%%%%%%%%%%%%%%%%%%%%%%%%%
\subsection{Flags}
\label{sec:flags}

The package makes it easy to generate different versions
of the main or child documents.
To this end compilation flags can be defined
and assigned different default values.
They will be particularly useful in conjunction
with the forwarding mechanism described in \secref{sec:forward}.

For example, it may be useful to have a flag |\version|
which can be set to |draft| or |final|.
The document source will contain some conditional code
depending on the value of |\version|.
Suppose further, the flag should default to |final| for the main file
and to |draft| for child files
which is a natural assignment for editing the document.
This is achieved by placing the following code
in the preamble of the main document
(below the |\childdocmain| directive):
%
\begin{center}
\begin{tabular}{l}
|\ifchilddoc|\\
|\providecommand{\version}{draft}|\\
|\||else|\\
|\providecommand{\version}{final}|\\
|\||fi|
\end{tabular}
\end{center}
%
The definition by |\providecommand| makes sure
that previous definitions are not overwritten.
Further statements |\providecommand{\version}{...}|
can thus be added before the above code to override it.

For the main file, one might add a line
(between |\childdocmain| and the above block)
%
\begin{center}
|%\ifchilddoc\||else\providecommand{\version}{draft}\||fi|
\end{center}
%
which can be uncommented to produce a draft version.
Likewise one can add a line to the very top of a child file
(above the |\childdocof{|\textit{main}|}| directive)
%
\begin{center}
|%\providecommand{\version}{final}|
\end{center}
%
which can be uncommented to produce the final version of this child document.

%%%%%%%%%%%%%%%%%%%%%%%%%%%%%%%%%%%%%%%%%%%%%%%%%%%%%%%%%%%%%%%%%%%%%%%%%%%%%%%%
\subsection{Forwarding}
\label{sec:forward}

Different versions of the main or child documents
using compilation flags as described in \secref{sec:flags}
can be (permanently) stored in different files
for convenient compilation, viewing and distribution.
To this end, the package defines a command
to pass on compilation to a different file:

%%%%%%%%%%%%%%%%%%%%%%%%%%%%%%%%%%%%%%%%
\DescribeMacro{\childdocforward}
The command |\childdocforward| redirects processing to
another source file:
%
\begin{center}
\begin{tabular}{l}
|\input{childdoc.def}|\\
|\childdocforward[|\textit{main}|]{|\textit{dest}|}|\\
\end{tabular}
\end{center}
%
The argument \textit{dest} is the destination file
(without extension).
It should be the main file or one of the child files.
Note that further \textsf{childdoc} directives
such as |\childdocof| and |\childdocforward|
in the indicated file will be processed in this form.
The optional argument \textit{main}
passes on directly to the main file \textit{main}
while pretending to compile the child \textit{dest}.
This form behaves as if \textit{dest}
issues |\childdocof{|\textit{main}|}| right away,
and no further \textsf{childdoc} directives will be processed.

%%%%%%%%%%%%%%%%%%%%%%%%%%%%%%%%%%%%%%%%
\DescribeMacro{\...prefix}
In the alternative form |\childdocforwardprefix|,
%
\begin{center}
\begin{tabular}{l}
|\input{childdoc.def}|\\
|\childdocforwardprefix[|\textit{main}|]{|\textit{prefix}|}{|\textit{dest}|}|
\end{tabular}
\end{center}
%
the destination file is determined by a pattern
depending on the current file:
To make this work, the current file must be called
`{\textit{prefix}\hspace{0.2em}\textit{suffix}}'
with \textit{prefix} matching precisely the argument.
Processing is then passed on to the file
`{\textit{dest}\hspace{0.2em}\textit{suffix}}'.
Surely, the same effect is achieved by
directly specifying the
argument `{\textit{dest}\hspace{0.2em}\textit{suffix}}'
in the first form.
However, that requires to set up a different file
for each child. With the alternative form of the command
all these files can have exactly the same content
which simplifies setting them up and maintaining them.

For example, the following file |draft.tex|
with a compilation flag |\version| as described in \secref{sec:flags}
compiles the main document as a draft:
%
\begin{center}
\begin{tabular}{l}
|\def\version{draft}|\\
|\input{childdoc.def}|\\
|\childdocforward{|\textit{main}|}|
\end{tabular}
\end{center}
%
Likewise, the following files |final|\textit{nn}|.tex|
compile the final version of the child document
|child|\textit{nn}|.tex|:
%
\begin{center}
\begin{tabular}{l}
|\def\version{final}|\\
|\input{childdoc.def}|\\
|\childdocforwardprefix{final}{child}|
\end{tabular}
\end{center}
%

Note that when several versions of a main file and/or of each child file
are to be generated, it may be convenient to set up a |Makefile| or
shell script to automatise the process.

%%%%%%%%%%%%%%%%%%%%%%%%%%%%%%%%%%%%%%%%%%%%%%%%%%%%%%%%%%%%%%%%%%%%%%%%%%%%%%%%
\subsection{Command Line Processing}
\label{sec:commandline}

The effect of redirection files can also be achieved by invoking
the \LaTeX{} compiler with a more elaborate command line.
Most conveniently this should be done as part
of a shell script or a |Makefile|.

When using \textsf{childdoc} in the main file, the following
command lines effectively perform a redirection
(note that depending on the shell being used,
backslashes may have to be doubled: `|\|' $\to$ `|\\|'):
%
\begin{center}
|... -jobname "|\textit{target}|" |\\|"|[\textit{flags}]%
|\input{childdoc.def}\childdocforward[|\textit{main}|]{|\textit{dest}|}"|
\end{center}
%
Here \textit{target} is the name of the output file,
\textit{main} is the name of the main file
and \textit{dest} is the name of the main or child file to be processed
(all filenames without extensions).
The optional argument \textit{main} can be omitted
if \textit{main} matches \textit{dest}.
Optionally, compilation \textit{flags} can be defined via |\def| commands.
This command line makes the \TeX{} engine believe
it is compiling the file \textit{target}
whose content is specified as the latter parameter.
The provided code then forwards the processing to
\textit{main} or \textit{dest} as described in \secref{sec:forward}.

%%%%%%%%%%%%%%%%%%%%%%%%%%%%%%%%%%%%%%%%%%%%%%%%%%%%%%%%%%%%%%%%%%%%%%%%%%%%%%%%
\subsection{Include by Input}
\label{sec:input}

Including child documents by |\include| has some restrictions by design.
Most notably, the content of a child document always occupies
its own set of pages; pages cannot be shared between child documents.
Usually, this behaviour makes perfect sense
because each child document contain an essential part of the document.
However, in some situations it may be desirable to compose
a document from a collection of parts
without having mandatory page breaks between then.
For this case, the package
provides a mechanism to include parts
by |\input| which can also be processed individually.
However, by construction this mechanism
requires manual handling of the content to be output.

%%%%%%%%%%%%%%%%%%%%%%%%%%%%%%%%%%%%%%%%
\DescribeMacro{\ifchilddocmanual}
The main file should be prepared as usual, see \secref{sec:include}.
However, the document body must make a distinction
between processing of an individual part and of the main document, e.g.:
%
\begin{center}
\begin{tabular}{l}
|\ifchilddocmanual|\\
|\input{\childdocname}|\\
|\||else|\\
\textit{document body with }|\input{|\textit{part}|}|\\
|\||fi|
\end{tabular}
\end{center}
%
The conditional |\ifchilddocmanual| is true whenever
a part to be included by |\input| is being compiled,
and the name of the part is stored in |\childdocname|.

%%%%%%%%%%%%%%%%%%%%%%%%%%%%%%%%%%%%%%%%
\DescribeMacro{\childdocby}
Each part to be included by |\input| should start with:
%
\begin{center}
\begin{tabular}{l}
|\input{childdoc.def}|\\
|\childdocby{|\textit{main}|}|\\
\end{tabular}
\end{center}
%
The directive |\childdocby| is similar to |\childdocof|
described in \secref{sec:include},
but the subsequent selection of content must be done manually.
To that end, both |\ifchilddoc| and |\ifchilddocmanual|
will be true upon processing of a part,
and the name of the part is stored in |\childdocname|.
Note that |\jobname| will be set to the filename of the current part
so that each part receives an individual |.aux| file
that does not interfere with the |.aux| file(s) of the main document.
This behaviour can be altered by the alternative form
|\childdocby[*]{|\textit{main}|}| (with a non-empty optional argument)
which uses the |.aux| file of the main document
by setting |\jobname| to \textit{main}.

%%%%%%%%%%%%%%%%%%%%%%%%%%%%%%%%%%%%%%%%%%%%%%%%%%%%%%%%%%%%%%%%%%%%%%%%%%%%%%%%
\subsection{Driver Development}
\label{sec:driver}

The \textsf{childdoc} mechanism can also be use for the development
of definition files such as \LaTeX{} styles or classes.
This case differs from the above setup with multiple parts
included by |\include| in that no |\includeonly| should be invoked.
This can be achieved by starting the include file
(before |\ProvidesPackage|) with:
%
\begin{center}
\begin{tabular}{l}
|\input{childdoc.def}|\\
|\childdocforward{|\textit{main}|}|\\
\end{tabular}
\end{center}
%
or alternatively with:
%
\begin{center}
\begin{tabular}{l}
|\input{childdoc.def}|\\
|\childdocby{|\textit{main}|}|\\
\end{tabular}
\end{center}
%
Both forms have slightly different effects as described above.
The main file is prepared as usual, see \secref{sec:include}.

%%%%%%%%%%%%%%%%%%%%%%%%%%%%%%%%%%%%%%%%%%%%%%%%%%%%%%%%%%%%%%%%%%%%%%%%%%%%%%%%
\subsection{Legacy Detection}
\label{sec:detection}

The directive |\childdocmain| in the main file can detect
whether the complete document or merely a child is to be compiled
even without using the directive |\childdocof|.
This method is deprecated because it is less robust
and there is no compelling reason to use it;
it is merely provided for backward compatibility
and it may be removed in future versions.

If the detection mechanism is to be used,
it is mandatory to correctly specify
the filename of the main file as the argument of |\childdocmain|:
%
\begin{center}
\begin{tabular}{l}
|\input{childdoc.def}|\\
|\childdocmain{|\textit{main}|}|\\
\end{tabular}
\end{center}
%
If |\jobname| does not match the argument \textit{main} of |\childdocmain|,
it is assumed that |\jobname| points to the child file to be compiled.
When using |\childdocmain| with the main file specified as argument,
it suffices to start a child file
with just |\input{|\textit{main}|}|
without loading of the package and using |\childdocof|.
If instead all processing is done
with the appropriate \textsf{childdoc} directives,
the argument of \textit{main} of |\childdocmain| can be empty.

An alternative version of the command line processing described
in \secref{sec:commandline} using the detection mechanism reads:
%
\begin{center}
|... -jobname "|\textit{target}|" "|[\textit{flags}]%
[|\def\jobname{|\textit{dest}|}|]|\input{|\textit{main}|}"|
\end{center}

%%%%%%%%%%%%%%%%%%%%%%%%%%%%%%%%%%%%%%%%%%%%%%%%%%%%%%%%%%%%%%%%%%%%%%%%%%%%%%%%
\subsection{Manual Code}
\label{sec:manual}

In case one cannot be certain whether the definitions file |childdoc.def|
is installed on the target \TeX{} distribution
and one prefers not to ship it,
it is conceivable to paste a few relevant commands into the sources.

To that end, drop all statements |\input{childdoc.def}|
and perform the replacements as outlined below.
Instead of |\childdocmain{|\textit{main}|}| add the following code
to the top of the main file:
%
\begin{center}
\begin{tabular}{l}
|\||ifdefined\childdocname\endinput\||fi\newif\ifchilddoc|\\
|\edef\childdocname{\scantokens\expandafter{\jobname\noexpand}}|\\
|\def\childdocmain{|\textit{main}|}\||ifx\childdocmain\childdocname\||else|\\
|\childdoctrue\includeonly{\childdocname}\let\jobname\childdocmain\||fi|\\
\end{tabular}
\end{center}
%
Instead of |\childdocof{|\textit{main}|}| just include the main file
at the top of each child file:
%
\begin{center}
|\input{|\textit{main}|}|
\end{center}
%
A simple redirection |\childdocforward{|\textit{dest}|}| is achieved by:
%
\begin{center}
|\def\jobname{|\textit{dest}|}\input{\jobname}|
\end{center}
%
The redirection with prefix
|\childdocforwardprefix[|\textit{prefix}|]{|\textit{dest}|}|
is accomplished by:
%
\begin{center}
\begin{tabular}{l}
|{\edef\jobname{\scantokens\expandafter{\jobname\noexpand}}|\\
|\def\redirectjob |\textit{prefix}|#1~~~{\gdef\jobname{|\textit{dest}|#1}}|\\
|\expandafter\redirectjob\jobname~~~}\input{\jobname}|
\end{tabular}
\end{center}

In an alternative approach,
child documents can be compiled by a specific command line
without additional code or specific definitions:
%
\begin{center}
|... -jobname "|\textit{target}|" "|[\textit{flags}]%
|\includeonly{|\textit{dest}|}\input{|\textit{main}|}"|
\end{center}
%

%%%%%%%%%%%%%%%%%%%%%%%%%%%%%%%%%%%%%%%%%%%%%%%%%%%%%%%%%%%%%%%%%%%%%%%%%%%%%%%%
%%%%%%%%%%%%%%%%%%%%%%%%%%%%%%%%%%%%%%%%%%%%%%%%%%%%%%%%%%%%%%%%%%%%%%%%%%%%%%%%
\section{Information}

%%%%%%%%%%%%%%%%%%%%%%%%%%%%%%%%%%%%%%%%%%%%%%%%%%%%%%%%%%%%%%%%%%%%%%%%%%%%%%%%
\subsection{Copyright}

Copyright \copyright{} 2017--2018 Niklas Beisert

This work may be distributed and/or modified under the
conditions of the \LaTeX{} Project Public License, either version 1.3
of this license or (at your option) any later version.
The latest version of this license is in
  \url{http://www.latex-project.org/lppl.txt}
and version 1.3 or later is part of all distributions of \LaTeX{}
version 2005/12/01 or later.

This work has the LPPL maintenance status `maintained'.

The Current Maintainer of this work is Niklas Beisert.

This work consists of the files |README.txt|, |childdoc.ins| and |childdoc.dtx|
as well as the derived files |childdoc.def|, |cdocsamp.tex|
with |cdocsch1.tex|, |cdocsch2.tex|, |cdocspt3.tex|, |cdocspt4.tex|,
|cdocsdrf.tex|, |cdocsfn1.tex|, |cdocsfn2.tex|
as well as |childdoc.pdf|.

%%%%%%%%%%%%%%%%%%%%%%%%%%%%%%%%%%%%%%%%%%%%%%%%%%%%%%%%%%%%%%%%%%%%%%%%%%%%%%%%
\subsection{Files and Installation}

The package consists of the files:
%
\begin{center}
\begin{tabular}{ll}
    |README.txt|   & readme file \\
    |childdoc.ins| & installation file \\
    |childdoc.dtx| & source file \\
    |childdoc.def| & definition file \\
    |cdocsamp.tex| & sample main file \\
    |cdocsch1.tex| & sample include file \\
    |cdocsch2.tex| & sample include file \\
    |cdocspt3.tex| & sample part file \\
    |cdocspt4.tex| & sample part file \\
    |cdocsdrf.tex| & sample redirection file \\
    |cdocsfn1.tex| & sample redirection file \\
    |cdocsfn2.tex| & sample redirection file \\
    |childdoc.pdf| & manual
\end{tabular}
\end{center}
%
The distribution consists of the files
|README.txt|, |childdoc.ins| and |childdoc.dtx|.
%
\begin{itemize}
\item
Run (pdf)\LaTeX{} on |childdoc.dtx|
to compile the manual |childdoc.pdf| (this file).
\item
Run \LaTeX{} on |childdoc.ins| to create the definitions file |childdoc.def|
and the sample |cdocsamp.tex| with include files
|cdocsch1.tex|, |cdocsch2.tex|, |cdocspt3.tex|, |cdocspt4.tex|,
|cdocsdrf.tex|, |cdocsfn1.tex|, |cdocsfn2.tex|.
Then copy the file |childdoc.def| to an appropriate directory of your \LaTeX{}
distribution, e.g.\ \textit{texmf-root}|/tex/latex/childdoc|.
\end{itemize}

%%%%%%%%%%%%%%%%%%%%%%%%%%%%%%%%%%%%%%%%%%%%%%%%%%%%%%%%%%%%%%%%%%%%%%%%%%%%%%%%
\subsection{Related CTAN Packages}

There are several other packages which offer a similar functionality:
%
\begin{itemize}
\item
The packages
\href{http://ctan.org/pkg/docmute}{\textsf{docmute}},
\href{http://ctan.org/pkg/includex}{\textsf{includex}} and
\href{http://ctan.org/pkg/standalone}{\textsf{standalone}}
provide commands to include only the document body of
a child file thus allowing both files to be compiled individually.
\item
The packages \href{http://ctan.org/pkg/subdocs}{\textsf{subdocs}}
and \href{http://ctan.org/pkg/subfiles}{\textsf{subfiles}}
provide structures in which the main and child documents can be
encapsulated and allowing them to be compiled individually.
The inclusion mechanism is different from the conventional |\include|.
\item
The package \href{http://ctan.org/pkg/combine}{\textsf{combine}}
is an elaborate solution to combine several documents into one.
\end{itemize}
%
See also the CTAN topic \href{http://ctan.org/topic/subdocs}{\textsf{subdocs}}
for further related packages.
The present package differs from the above solutions in that
a document structure constructed with the conventional |\include| mechanism
just needs two extra commands at the top of every file
such that all constituent files can be compiled individually.

%%%%%%%%%%%%%%%%%%%%%%%%%%%%%%%%%%%%%%%%%%%%%%%%%%%%%%%%%%%%%%%%%%%%%%%%%%%%%%%%
%\subsection{Feature Suggestions}
%
%The following is a list of features which may be useful for future
%versions of this package:
%%
%\begin{itemize}
%\item
%\ldots
%\end{itemize}

%%%%%%%%%%%%%%%%%%%%%%%%%%%%%%%%%%%%%%%%%%%%%%%%%%%%%%%%%%%%%%%%%%%%%%%%%%%%%%%%
\subsection{Revision History}

%%%%%%%%%%%%%%%%%%%%%%%%%%%%%%%%%%%%%%%%
\paragraph{v2.0:} 2018/12/30

\begin{itemize}
\item
immediate forward processing
\item
added |\childdocby| mechanism
\item
manual restructured
\end{itemize}

%%%%%%%%%%%%%%%%%%%%%%%%%%%%%%%%%%%%%%%%
\paragraph{v1.6:} 2018/01/17

\begin{itemize}
\item
application for development of include files
\item
corrections to manual
\end{itemize}

%%%%%%%%%%%%%%%%%%%%%%%%%%%%%%%%%%%%%%%%
\paragraph{v1.5:} 2017/05/21

\begin{itemize}
\item
more complete structuring introduced
\item
|\childdocof| introduced
\item
|\childdoc| renamed to |\childdocmain|
\item
|\childredirect| renamed to |\childdocforward| and |\childdocforwardprefix|
and functionality expanded
\end{itemize}

%%%%%%%%%%%%%%%%%%%%%%%%%%%%%%%%%%%%%%%%
\paragraph{v1.0:} 2017/04/27

\begin{itemize}
\item
manual and install package
\item
first version published on CTAN
\end{itemize}

%%%%%%%%%%%%%%%%%%%%%%%%%%%%%%%%%%%%%%%%
\paragraph{v0.6:} 2017/04/26

\begin{itemize}
\item
redirection mechanism added
\end{itemize}

%%%%%%%%%%%%%%%%%%%%%%%%%%%%%%%%%%%%%%%%
\paragraph{v0.5:} 2017/04/26

\begin{itemize}
\item
functionality in definition file
\end{itemize}


%%%%%%%%%%%%%%%%%%%%%%%%%%%%%%%%%%%%%%%%%%%%%%%%%%%%%%%%%%%%%%%%%%%%%%%%%%%%%%%%
%%%%%%%%%%%%%%%%%%%%%%%%%%%%%%%%%%%%%%%%%%%%%%%%%%%%%%%%%%%%%%%%%%%%%%%%%%%%%%%%
%%%%%%%%%%%%%%%%%%%%%%%%%%%%%%%%%%%%%%%%%%%%%%%%%%%%%%%%%%%%%%%%%%%%%%%%%%%%%%%%
\appendix

\settowidth\MacroIndent{\rmfamily\scriptsize 000\ }

 \DocInput{childdoc.dtx}

\end{document}
%</driver>
% \fi
%
% %%%%%%%%%%%%%%%%%%%%%%%%%%%%%%%%%%%%%%%%%%%%%%%%%%%%%%%%%%%%%%%%%%%%%%%%%%%%%%
% %%%%%%%%%%%%%%%%%%%%%%%%%%%%%%%%%%%%%%%%%%%%%%%%%%%%%%%%%%%%%%%%%%%%%%%%%%%%%%
% \section{Sample}
%\iffalse
%<*samplemain>
%\fi
%
% The following presents a sample document
% with two chapters, two parts, a title page,
% a compile flag as well as three forwarding files to set the flag.
% It consists of eight |.tex| files:
% \begin{center}
% \begin{tabular}{ll}
% |cdocsamp.tex|&main file\\
% |cdocsch1.tex|&include file for chapter 1\\
% |cdocsch2.tex|&include file for chapter 2\\
% |cdocspt3.tex|&include file for part 3\\
% |cdocspt4.tex|&include file for part 4\\
% |cdocsdrf.tex|&forwarding file for main file in draft mode\\
% |cdocsfi1.tex|&forwarding file for final version of chapter 1\\
% |cdocsfi2.tex|&forwarding file for final version of chapter 2\\
% \end{tabular}
% \end{center}
% Each of the eight files can be compiled directly by the \LaTeX{} compiler.
%
% %%%%%%%%%%%%%%%%%%%%%%%%%%%%%%%%%%%%%%
% \paragraph{Main File.}
%
% The main file is called |cdocsamp.tex|.
%
% Load the \textsf{childdoc} definitions and
% declare the filename for the main document:
%    \begin{macrocode}
\input{childdoc.def}
\childdocmain{}
%    \end{macrocode}

% Optional override for |\version| flag:
%    \begin{macrocode}
%%\ifchilddoc\else\providecommand{\version}{draft}\fi
%    \end{macrocode}

% Define the default values for the |\version| flag
% (|final| for the main file and |draft| for childs):
%    \begin{macrocode}
\ifchilddoc
\providecommand{\version}{draft}
\else
\providecommand{\version}{final}
\fi
%    \end{macrocode}

% Load the standard document class:
%    \begin{macrocode}
\documentclass[12pt]{article}
%    \end{macrocode}

% Start the document body:
%    \begin{macrocode}
\begin{document}
%    \end{macrocode}

% Declare a title page.
% Print title, part of document being processed and version flag:
%    \begin{macrocode}
\addtocounter{page}{-1}
\begin{center}
{\LARGE\bfseries{}childdoc example\par}
\vspace{1cm}
\ifchilddoc
\ifchilddocmanual part\else chapter\fi:
`\childdocname' of `\childdocjob'\par
\else
main document: `\childdocjob'\par
\fi
version: \version\par
\end{center}
\newpage
%    \end{macrocode}

% Manually include selected file,
% otherwise process as usual:
%    \begin{macrocode}
\ifchilddocmanual
\section*{part `\childdocname'}
\input{\childdocname}
\else
%    \end{macrocode}

% Include the two chapters:
%    \begin{macrocode}
\include{cdocsch1}
\include{cdocsch2}
%    \end{macrocode}

% Include the two parts unless only chapters should be displayed:
%    \begin{macrocode}
\ifchilddoc\else
\section{part three}
\input{cdocspt3}
\section{part four}
\input{cdocspt4}
\fi
%    \end{macrocode}

% Process as usual until here:
%    \begin{macrocode}
\fi
%    \end{macrocode}

% End of document body:
%    \begin{macrocode}
\end{document}
%    \end{macrocode}
%\iffalse
%</samplemain>
%\fi
%
% %%%%%%%%%%%%%%%%%%%%%%%%%%%%%%%%%%%%%%
% \paragraph{Chapter Include Files.}
%
% The include files are called |cdocsch1.tex| and |cdocsch2.tex|.
%
%\iffalse
%<*samplechap1|samplechap2>
%\fi

% Optional override for |\version| flag:
%    \begin{macrocode}
%%\providecommand{\version}{final}
%    \end{macrocode}

% Include the main document:
%    \begin{macrocode}
\input{childdoc.def}
\childdocof{cdocsamp}
%    \end{macrocode}

%\iffalse
%</samplechap1|samplechap2>
%\fi
%
%\iffalse
%<*samplechap1>
%\fi
% Some text for chapter 1:
%    \begin{macrocode}
\section{one}
some text in chapter one
%    \end{macrocode}

%\iffalse
%</samplechap1>
%\fi
% Some text for chapter 2:
%\iffalse
%<*samplechap2>
%\fi
%    \begin{macrocode}
\section{two}
more text in chapter two
%    \end{macrocode}

%\iffalse
%</samplechap2>
%\fi
%
% %%%%%%%%%%%%%%%%%%%%%%%%%%%%%%%%%%%%%%
% \paragraph{Part Include Files.}
%
% The include files are called |cdocspt3.tex| and |cdocspt4.tex|.
%
%\iffalse
%<*samplepart3|samplepart4>
%\fi

% Optional override for |\version| flag:
%    \begin{macrocode}
%%\providecommand{\version}{final}
%    \end{macrocode}

% Include the main document:
%    \begin{macrocode}
\input{childdoc.def}
\childdocby{cdocsamp}
%    \end{macrocode}

%\iffalse
%</samplepart3|samplepart4>
%\fi
%
%\iffalse
%<*samplepart3>
%\fi
% Some text for part 3:
%    \begin{macrocode}
some text in part three
%    \end{macrocode}

%\iffalse
%</samplepart3>
%\fi
% Some text for part 4:
%\iffalse
%<*samplepart4>
%\fi
%    \begin{macrocode}
more text in part four
%    \end{macrocode}

%\iffalse
%</samplepart4>
%\fi
%
% %%%%%%%%%%%%%%%%%%%%%%%%%%%%%%%%%%%%%%
% \paragraph{Forwarding for a Complete Draft.}
%
% The following forwarding file |cdocsdrf.tex|
% compiles the main document in draft mode:
%\iffalse
%<*sampledraft>
%\fi
%    \begin{macrocode}
\def\version{draft}
\input{childdoc.def}
\childdocforward{cdocsamp}
%    \end{macrocode}

%\iffalse
%</sampledraft>
%\fi
%
% %%%%%%%%%%%%%%%%%%%%%%%%%%%%%%%%%%%%%%
% \paragraph{Forwarding for Final Version of the Chapters.}
%
% The following forwarding files |cdocsfn1.tex| and |cdocsfn2.tex|
% (with identical content)
% compile the final versions of the child documents
% |cdocsch1.tex| and |cdocsch2.tex|, respectively:
%\iffalse
%<*samplefinal>
%\fi
%    \begin{macrocode}
\def\version{final}
\input{childdoc.def}
\childdocforwardprefix[cdocsamp]{cdocsfn}{cdocsch}
%    \end{macrocode}

%\iffalse
%</samplefinal>
%\fi
%
% %%%%%%%%%%%%%%%%%%%%%%%%%%%%%%%%%%%%%%
% \paragraph{Command Line Processing.}
%
% The following three command lines generate the output files
% |cdocscld|, |cdocscl1| and |cdocscl2|
% which should be identical to
% |cdocsdrf|, |cdocsch1| and |cdocsfn2|, respectively:
% \begin{center}
% \begin{tabular}{l}
% |latex -jobname cdocscld \|\\
% |  "\def\version{draft}\input{childdoc.def}\childdocforward{cdocsamp}"|\\
% |latex -jobname cdocscl1 \|\\
% |  "\input{childdoc.def}\childdocforward[cdocsamp]{cdocsch1}"|\\
% |latex -jobname cdocscl2 \|\\
% |  "\def\version{final}\input{childdoc.def}\childdocforward{cdocsch2}"|
% \end{tabular}
% \end{center}
% Note that the trailing backslash on each first line
% merely continues the input to the second line
% (for convenient cut ant paste).
% Furthermore, the command |latex| can be replaced by any
% of its alternative versions such as |pdflatex|.
%
% %%%%%%%%%%%%%%%%%%%%%%%%%%%%%%%%%%%%%%%%%%%%%%%%%%%%%%%%%%%%%%%%%%%%%%%%%%%%%%
% %%%%%%%%%%%%%%%%%%%%%%%%%%%%%%%%%%%%%%%%%%%%%%%%%%%%%%%%%%%%%%%%%%%%%%%%%%%%%%
% \section{Implementation}
%\iffalse
%<*package>
%\fi
%
% This section describes the definitions file |childdoc.def|.

% The definitions cannot be loaded using |\usepackage| or |\RequirePackage|
% which has a mechanism to prevent loading a style file more than once.
% When loading the definitions by means of |\input|
% multiple instances have to be prevented manually:
%\iffalse
%This code needs to be before the `\ProvidesFile' directive
%which is defined at the beginning of this file.
%Therefore it is also placed there and commented out here.
%</package>
%<*discard>
%\fi
%    \begin{macrocode}
\ifdefined\childdocmain\endinput\fi
%    \end{macrocode}
%\iffalse
%</discard>
%<*package>
%\fi
%
% \macro{\ifchilddoc}
% \macro{\ifchilddocmanual}
% The conditional |\ifchilddoc| tells whether a
% child (true) or main (false) document is being compiled.
% The conditional |\ifchilddocmanual| tells whether
% the |\includeonly| mechanism is used (false) or
% the selection of child files must be performed manually (true).
% The definitions initialise to false:
%    \begin{macrocode}
\newif\ifchilddoc
\newif\ifchilddocmanual
%    \end{macrocode}

% \macro{\childdocname}
% \macro{\childdocjob}
% The macro |\childdocname| stores the name of the main document
% to be compiled. The macro |\childdocjob| stores the name of
% the document on which the \LaTeX{} compiler was originally invoked.
% The content of |\jobname| cannot be compared
% to filenames specified in the source due to different catcodes.
% The following code rescans |\jobname|, stores the result
% in |\childdocname| and saves a copy in |\childdocjob|:
%    \begin{macrocode}
\edef\childdocname{\scantokens\expandafter{\jobname\noexpand}}
\let\childdocjob\childdocname
%    \end{macrocode}

% \macro{\childdocdisable}
% The macro |\childdocdisable| prevents the main file
% from being processed more than once.
% At this stage, the main document command |\childdocmain|
% is assumed to be called once again where it should do nothing.
% Any subsequent call to it should prevent
% a secondary processing of the main document
% It overwrites the forwarding commands
% |\childdocof| and |\childdocforward|
% with empty macros to prevent further inclusions of the main document:
%    \begin{macrocode}
\newcommand{\childdocdisable}
{
  \renewcommand{\childdocmain}[1]{\renewcommand{\childdocmain}[1]{\endinput}}
  \renewcommand{\childdocof}[1]{}
  \renewcommand{\childdocby}[2][]{}
  \renewcommand{\childdocforward}[2][]{}
  \renewcommand{\childdocdisable}{}
}
%    \end{macrocode}

% \macro{\childdocmain}
% The macro |\childdocmain| is to be called at the top of the main file
% with nothing or the main filename (without extension) as argument.
% First, it breaks loops.
% If the argument is not empty and does not match |\childdocname|
% (which is set by the first inclusion of |childdoc.def|),
% |\ifchilddoc| is set to true, |\includeonly| is applied to the child file
% and |\jobname| is set to the main file
% (for proper handling of |.aux| files):
%    \begin{macrocode}
\newcommand{\childdocmain}[1]
{
  \childdocdisable\childdocmain{}
  \if?#1?\else
    \begingroup
      \def\childdoctmp{#1}
      \ifx\childdoctmp\childdocname
        \def\childdoctmp{}
      \else
        \def\childdoctmp
        {
          \childdoctrue
          \includeonly{\childdocname}
          \def\childdocjob{#1}
          \def\jobname{#1}
        }
      \fi
      \expandafter
    \endgroup
    \childdoctmp
  \fi
}
%    \end{macrocode}

% \macro{\childdocof}
% The command |\childdocof| redirects
% compilation to the main file |#1|.
%    \begin{macrocode}
\newcommand{\childdocof}[1]
{
  \childdocdisable
  \childdoctrue
  \includeonly{\childdocname}
  \def\jobname{#1}
  \def\childdocjob{#1}
  \input{#1}
}
%    \end{macrocode}

% \macro{\childdocby}
% The command |\childdocby| ....
%    \begin{macrocode}
\newcommand{\childdocby}[2][]
{
  \childdocdisable
  \childdoctrue
  \childdocmanualtrue
  \if?#1?\else
    \def\jobname{#2}
  \fi
  \def\childdocjob{#2}
  \input{#2}
  \endinput
}
%    \end{macrocode}

% \macro{\childdocforward}
% The command |\childdocforward| redirects
% compilation to the main file or
% (if the optional argument is given) a child file.
% Parameters are set as if the main file
% or a child file starting with |\childdocof| was compiled.
% Then compilation is handed over to the main file:
%    \begin{macrocode}
\newcommand{\childdocforward}[2][]
{
  \begingroup
    \if?#1?
      \def\childdoctmp
      {
        \def\childdocname{#2}
        \def\childdocjob{#2}
        \def\jobname{#2}
        \input{#2}
        \endinput
      }
    \else
      \def\childdoctmp
      {
        \childdocdisable
        \def\childdocname{#2}
        \childdoctrue
        \includeonly{#2}
        \def\childdocjob{#1}
        \def\jobname{#1}
        \input{#1}
        \endinput
      }
    \fi
    \expandafter
  \endgroup
  \childdoctmp
}
%    \end{macrocode}

% \macro{\childdocforwardprefix}
% The command |\childdocforwardprefix| redirects
% compilation to the main or a child file by means of a pattern.
% The prefix |#1| in the current filename is replaced by |#2|
% and the suffix of the current filename is kept
% (it is assumed that the filename does not contain the substring `|~~~|'
% which is used as a delimiter).
% Compilation is handed over to the new file by |\childdocforward|:
%    \begin{macrocode}
\newcommand{\childdocforwardprefix}[3][]
{
  \begingroup
    \def\childdocextract #2##1~~~{\def\childdoctmp{\childdocforward[#1]{#3##1}}}
    \expandafter\childdocextract\childdocname~~~
    \expandafter
  \endgroup
  \childdoctmp
}
%    \end{macrocode}

% \macro{\childdoc}
% The deprecated macro |\childdoc| is a legacy version of |\childdocmain|:
%    \begin{macrocode}
\newcommand{\childdoc}{\childdocmain}
%    \end{macrocode}

% \macro{\childdocredirect}
% The deprecated macro |\childdocredirect| is a legacy version
% of |\childdocforward| and |\childdocforwardprefix|:
%    \begin{macrocode}
\newcommand{\childdocredirect}[2][]
{
  \begingroup
    \if?#1?
      \def\childdoctmp{\childdocforward{#2}}
    \else
      \def\childdoctmp{\childdocforwardprefix{#1}{#2}}
    \fi
    \expandafter
  \endgroup
  \childdoctmp
}
%    \end{macrocode}

%\iffalse
%</package>
%\fi
%
\endinput
|\\
|\childdocof{|\textit{main}|}|\\
\end{tabular}
\end{center}
at the top of every child file \textit{child}
which is included by |\include{|\textit{child}|}|
from within the main file
(or at least for those files to be compiled individually).
The argument \textit{main} must be the filename of the main file.

There are a couple of
considerations in setting up the main and child documents:

%%%%%%%%%%%%%%%%%%%%%%%%%%%%%%%%%%%%%%%%
\paragraph{Restrictions.}

Please note the following restrictions:
\begin{itemize}
\item
|\childdocmain| must be called with one argument \textit{main}
to ensure compatibility with earlier version of the package.
It must either be empty (|\childdocmain{}|)
or precisely match the filename of the main file in which it is specified.
See \secref{sec:detection} for further information.
\item
The filename \textit{main} must be specified without the |.tex| extension.
\item
The filename \textit{main} is case sensitive
(even in case-insensitive file systems)
due to internal string comparison.
\item
The argument \textit{main} should be fully expanded, it cannot be a macro.
\item
Subdirectories and special characters should be avoided in filenames.
\item
The command |\childdocmain{|\textit{main}|}| must be followed by a whitespace.
It should not be followed immediately by another command
or by a comment mark `|%|'.
This is because the \TeX{} parser reads the token immediately following
the argument of |\childdocmain| and puts it
at the beginning of every child section;
however, a white\-space is ignored.
\end{itemize}

%%%%%%%%%%%%%%%%%%%%%%%%%%%%%%%%%%%%%%%%
\paragraph{Content of Main File.}

It is advisable to place all content in the child files included by |\include|.
Any output contained in the main file will appear in all child documents
unless suppressed manually;
it cannot be suppressed automatically by the |\includeonly| directive
and thus should normally be avoided.
A method to include some content in the main file
by means of conditional processing is described in \secref{sec:conditional}.

%%%%%%%%%%%%%%%%%%%%%%%%%%%%%%%%%%%%%%%%
\paragraph{Page Numbering.}

When only a part of the document is compiled,
the appropriate numbering of pages
(as well as other status parameters)
is determined from the |.aux| files.
The latter contain information from previous passes.
However this information needs to propagate through
all intermediate child documents.
Therefore the page numbering in child documents may well
be inconsistent until the complete document is compiled at least once.

A useful (if unconventional) way to always ensure a consistent
page numbering is to restart the numbering in each child document
and denote the pages by `\textit{child}|.|\textit{page}'
where \textit{child} represents the chapter/section number of the child file.
This can be achieved by the command
|\numberwithin{page}{|\textit{child}|}|
of the \textsf{amsmath} package
where \textit{child} can be |chapter| or |section|
depending on the chosen structuring.
Alternatively, one can modify the macro |\thepage| appropriately
and reset the counter |page| at the start of each child file.

%%%%%%%%%%%%%%%%%%%%%%%%%%%%%%%%%%%%%%%%%%%%%%%%%%%%%%%%%%%%%%%%%%%%%%%%%%%%%%%%
\subsection{Conditional Processing}
\label{sec:conditional}

The package provides a mechanism to compile different versions
of a document. To customise the versions further some conditional processing
can come in handy to distinguish which version is being compiled.
The package provides two macros to describe the compilation context:

%%%%%%%%%%%%%%%%%%%%%%%%%%%%%%%%%%%%%%%%
\DescribeMacro{\ifchilddoc}
The conditional |\ifchilddoc| distinguishes between the compilation of
child documents and the main document:
%
\begin{center}
|\ifchilddoc |\textit{child-code}| |[|\||else |\textit{main-code}]| \||fi|
\end{center}

%%%%%%%%%%%%%%%%%%%%%%%%%%%%%%%%%%%%%%%%
\DescribeMacro{\childdocname}
\DescribeMacro{\childdocjob}
The macro |\childdocname| contains the filename (without extension)
of the main or child file being processed.
Note that |\childdocjob| will always contain the name of the main file.

%%%%%%%%%%%%%%%%%%%%%%%%%%%%%%%%%%%%%%%%
\paragraph{Title Page.}

Conditional processing can be used to include a title or banner page
in the main document when proper precautions are taken.
Importantly, the code in the main file should ensure that the page counter
(as well as other status parameters which are stored in the |.aux| files)
takes the same value after the conditional processing.
Otherwise the page numbers may take divergent values
depending on which part is compiled.

For example, a title page could be declared by:
%
\begin{center}
\begin{tabular}{l}
|\ifchilddoc\||else|\\
|\addtocounter{page}{-1}|\\
\textit{code for title page}\\
|\newpage|\\
|\||fi|
\end{tabular}
\end{center}
%
A banner page for the child documents can be generated by:
%
\begin{center}
\begin{tabular}{l}
|\ifchilddoc|\\
|\addtocounter{page}{-1}|\\
\textit{code for banner page}\\
|\newpage|\\
|\||fi|
\end{tabular}
\end{center}
%
Here one could write a message such as:
\begin{center}
|This is the part \childdocname{} of \childdocjob{}.|
\end{center}

%%%%%%%%%%%%%%%%%%%%%%%%%%%%%%%%%%%%%%%%%%%%%%%%%%%%%%%%%%%%%%%%%%%%%%%%%%%%%%%%
\subsection{Flags}
\label{sec:flags}

The package makes it easy to generate different versions
of the main or child documents.
To this end compilation flags can be defined
and assigned different default values.
They will be particularly useful in conjunction
with the forwarding mechanism described in \secref{sec:forward}.

For example, it may be useful to have a flag |\version|
which can be set to |draft| or |final|.
The document source will contain some conditional code
depending on the value of |\version|.
Suppose further, the flag should default to |final| for the main file
and to |draft| for child files
which is a natural assignment for editing the document.
This is achieved by placing the following code
in the preamble of the main document
(below the |\childdocmain| directive):
%
\begin{center}
\begin{tabular}{l}
|\ifchilddoc|\\
|\providecommand{\version}{draft}|\\
|\||else|\\
|\providecommand{\version}{final}|\\
|\||fi|
\end{tabular}
\end{center}
%
The definition by |\providecommand| makes sure
that previous definitions are not overwritten.
Further statements |\providecommand{\version}{...}|
can thus be added before the above code to override it.

For the main file, one might add a line
(between |\childdocmain| and the above block)
%
\begin{center}
|%\ifchilddoc\||else\providecommand{\version}{draft}\||fi|
\end{center}
%
which can be uncommented to produce a draft version.
Likewise one can add a line to the very top of a child file
(above the |\childdocof{|\textit{main}|}| directive)
%
\begin{center}
|%\providecommand{\version}{final}|
\end{center}
%
which can be uncommented to produce the final version of this child document.

%%%%%%%%%%%%%%%%%%%%%%%%%%%%%%%%%%%%%%%%%%%%%%%%%%%%%%%%%%%%%%%%%%%%%%%%%%%%%%%%
\subsection{Forwarding}
\label{sec:forward}

Different versions of the main or child documents
using compilation flags as described in \secref{sec:flags}
can be (permanently) stored in different files
for convenient compilation, viewing and distribution.
To this end, the package defines a command
to pass on compilation to a different file:

%%%%%%%%%%%%%%%%%%%%%%%%%%%%%%%%%%%%%%%%
\DescribeMacro{\childdocforward}
The command |\childdocforward| redirects processing to
another source file:
%
\begin{center}
\begin{tabular}{l}
|% \iffalse
%
% childdoc.dtx Copyright (C) 2017-2018 Niklas Beisert
%
% This work may be distributed and/or modified under the
% conditions of the LaTeX Project Public License, either version 1.3
% of this license or (at your option) any later version.
% The latest version of this license is in
%   http://www.latex-project.org/lppl.txt
% and version 1.3 or later is part of all distributions of LaTeX
% version 2005/12/01 or later.
%
% This work has the LPPL maintenance status `maintained'.
%
% The Current Maintainer of this work is Niklas Beisert.
%
% This work consists of the files childdoc.dtx and childdoc.ins
% and the derived files childdoc.def and cdocsamp.tex with
% cdocsch1.tex, cdocsch2.tex, cdocsdrf.tex, cdocsfn1.tex, cdocsfn2.tex.
%
%<package>\ifdefined\childdocmain\endinput\fi
%<package>\ProvidesFile{childdoc.def}[2018/12/30 v2.0 child document driver]
%<samplemain>\ProvidesFile{cdocsamp.tex}[2018/12/30 v2.0 sample for childdoc]
%<*driver>
%\ProvidesFile{childdoc.drv}[2018/12/30 v2.0 childdoc reference manual file]
\PassOptionsToClass{10pt,a4paper}{article}
\documentclass{ltxdoc}

\usepackage[margin=35mm]{geometry}
\usepackage{hyperref}
\usepackage{hyperxmp}
\usepackage[usenames]{color}

\hypersetup{colorlinks=true}
\hypersetup{pdfstartview=FitH}
\hypersetup{pdfpagemode=UseNone}
\hypersetup{pdfsource={}}
\hypersetup{pdflang={en-UK}}
\hypersetup{pdfcopyright={Copyright 2017-2018 Niklas Beisert.
  This work may be distributed and/or modified under the
  conditions of the LaTeX Project Public License, either version 1.3
  of this license or (at your option) any later version.}}
\hypersetup{pdflicenseurl={http://www.latex-project.org/lppl.txt}}
\hypersetup{pdfcontactaddress={ETH Zurich, ITP, HIT K,
  Wolfgang-Pauli-Strasse 27}}
\hypersetup{pdfcontactpostcode={8093}}
\hypersetup{pdfcontactcity={Zurich}}
\hypersetup{pdfcontactcountry={Switzerland}}
\hypersetup{pdfcontactemail={nbeisert@itp.phys.ethz.ch}}
\hypersetup{pdfcontacturl={http://people.phys.ethz.ch/\xmptilde nbeisert/}}

\newcommand{\secref}[1]{\hyperref[#1]{section \ref*{#1}}}

\parskip1ex
\parindent0pt
\let\olditemize\itemize
\def\itemize{\olditemize\parskip0pt}

\begin{document}

\title{The \textsf{childdoc} Package}
\hypersetup{pdftitle={The childdoc Package}}
\author{Niklas Beisert\\[2ex]
  Institut f\"ur Theoretische Physik\\
  Eidgen\"ossische Technische Hochschule Z\"urich\\
  Wolfgang-Pauli-Strasse 27, 8093 Z\"urich, Switzerland\\[1ex]
  \href{mailto:nbeisert@itp.phys.ethz.ch}
  {\texttt{nbeisert@itp.phys.ethz.ch}}}
\hypersetup{pdfauthor={Niklas Beisert}}
\hypersetup{pdfsubject={Manual for the LaTeX2e Package childdoc}}
\date{30 December 2018, \textsf{v2.0}}
\maketitle

\begin{abstract}\noindent
\textsf{childdoc} is a \LaTeXe{} package
that enables the direct compilation
of document sections included by |\include|
to individual files.
\end{abstract}

\begingroup
\parskip0ex
\tableofcontents
\endgroup

%%%%%%%%%%%%%%%%%%%%%%%%%%%%%%%%%%%%%%%%%%%%%%%%%%%%%%%%%%%%%%%%%%%%%%%%%%%%%%%%
%%%%%%%%%%%%%%%%%%%%%%%%%%%%%%%%%%%%%%%%%%%%%%%%%%%%%%%%%%%%%%%%%%%%%%%%%%%%%%%%
\section{Introduction}

\LaTeX{} provides a mechanism to structure a large document (such as a book)
into a main file and several child files (containing the chapters)
using the |\include| command.
This mechanism is beneficial for documents
which span hundreds of pages in order to
make the source file(s) more manageable.
Moreover, compilation can be restricted to
selected child files by means of the |\includeonly| command.
The latter feature can be used to reduce the compilation time while editing
(this was significantly more useful in the earlier days of \LaTeX{})
or to generate a smaller document which is easier to navigate.
Another application of |\includeonly| is to generate
documents consisting of selected parts of the complete document.

However, there are a few drawbacks of the plain |\include| mechanism:
\begin{itemize}
\item
The child files cannot be compiled on their own,
they can only be compiled via the main file.
A naive editing environment
(such as a text editor with an option
to have the current file processed by \LaTeX)
may require one to switch to the main file before compiling;
attempting to compile the child file produces errors.
\item
The main file must be modified (each time)
to adjust the |\includeonly| command
to the present needs. This easily leaves the main file in a messy state.
\item
The generated document will always carry the filename
of the main document. This is inconvenient if
several child files are to be compiled and
to be kept for distribution.
\end{itemize}

The present package provides a simple interface
to make child files individually compilable by \LaTeX{}.
Compiling a child file then has the same effect as compiling
the main file with an |\includeonly| command
to select the appropriate child.
Moreover the generated document will carry the name of the child
rather than the main file.
This resolves all three above issues.

This feature is meant to make the editing of books,
thesis documents and lecture notes somewhat more convenient.
However, the package can also be used efficiently for
composing a series of documents (such as exercise sheets)
which are typically distributed individually.
It then assists the author in generating the individual documents
(potentially in different versions)
as well as a document containing the collected series.
Another application is in developing style files
or other kinds of included material
where compilation of the style file could redirect
to a sample or test file.

%%%%%%%%%%%%%%%%%%%%%%%%%%%%%%%%%%%%%%%%%%%%%%%%%%%%%%%%%%%%%%%%%%%%%%%%%%%%%%%%
%%%%%%%%%%%%%%%%%%%%%%%%%%%%%%%%%%%%%%%%%%%%%%%%%%%%%%%%%%%%%%%%%%%%%%%%%%%%%%%%
\section{Usage}

First of all, the package \textsf{childdoc} is \emph{not} a standard
\LaTeXe{} |.sty| style file! Therefore it needs to be invoked in
a non-standard way.

%%%%%%%%%%%%%%%%%%%%%%%%%%%%%%%%%%%%%%%%%%%%%%%%%%%%%%%%%%%%%%%%%%%%%%%%%%%%%%%%
\subsection{Included Files}
\label{sec:include}

%%%%%%%%%%%%%%%%%%%%%%%%%%%%%%%%%%%%%%%%
\DescribeMacro{\childdocmain}
To use the package, add the commands
\begin{center}
\begin{tabular}{l}
|\input{childdoc.def}|\\
|\childdocmain{}|\\
\end{tabular}
\end{center}
at the very top of the main \LaTeX{} file,
in particular \emph{before} the |\documentclass| statement!
The argument of |\childdocmain| should be left empty
(but it must be present).

%%%%%%%%%%%%%%%%%%%%%%%%%%%%%%%%%%%%%%%%
\DescribeMacro{\childdocof}
Furthermore, add the commands
\begin{center}
\begin{tabular}{l}
|\input{childdoc.def}|\\
|\childdocof{|\textit{main}|}|\\
\end{tabular}
\end{center}
at the top of every child file \textit{child}
which is included by |\include{|\textit{child}|}|
from within the main file
(or at least for those files to be compiled individually).
The argument \textit{main} must be the filename of the main file.

There are a couple of
considerations in setting up the main and child documents:

%%%%%%%%%%%%%%%%%%%%%%%%%%%%%%%%%%%%%%%%
\paragraph{Restrictions.}

Please note the following restrictions:
\begin{itemize}
\item
|\childdocmain| must be called with one argument \textit{main}
to ensure compatibility with earlier version of the package.
It must either be empty (|\childdocmain{}|)
or precisely match the filename of the main file in which it is specified.
See \secref{sec:detection} for further information.
\item
The filename \textit{main} must be specified without the |.tex| extension.
\item
The filename \textit{main} is case sensitive
(even in case-insensitive file systems)
due to internal string comparison.
\item
The argument \textit{main} should be fully expanded, it cannot be a macro.
\item
Subdirectories and special characters should be avoided in filenames.
\item
The command |\childdocmain{|\textit{main}|}| must be followed by a whitespace.
It should not be followed immediately by another command
or by a comment mark `|%|'.
This is because the \TeX{} parser reads the token immediately following
the argument of |\childdocmain| and puts it
at the beginning of every child section;
however, a white\-space is ignored.
\end{itemize}

%%%%%%%%%%%%%%%%%%%%%%%%%%%%%%%%%%%%%%%%
\paragraph{Content of Main File.}

It is advisable to place all content in the child files included by |\include|.
Any output contained in the main file will appear in all child documents
unless suppressed manually;
it cannot be suppressed automatically by the |\includeonly| directive
and thus should normally be avoided.
A method to include some content in the main file
by means of conditional processing is described in \secref{sec:conditional}.

%%%%%%%%%%%%%%%%%%%%%%%%%%%%%%%%%%%%%%%%
\paragraph{Page Numbering.}

When only a part of the document is compiled,
the appropriate numbering of pages
(as well as other status parameters)
is determined from the |.aux| files.
The latter contain information from previous passes.
However this information needs to propagate through
all intermediate child documents.
Therefore the page numbering in child documents may well
be inconsistent until the complete document is compiled at least once.

A useful (if unconventional) way to always ensure a consistent
page numbering is to restart the numbering in each child document
and denote the pages by `\textit{child}|.|\textit{page}'
where \textit{child} represents the chapter/section number of the child file.
This can be achieved by the command
|\numberwithin{page}{|\textit{child}|}|
of the \textsf{amsmath} package
where \textit{child} can be |chapter| or |section|
depending on the chosen structuring.
Alternatively, one can modify the macro |\thepage| appropriately
and reset the counter |page| at the start of each child file.

%%%%%%%%%%%%%%%%%%%%%%%%%%%%%%%%%%%%%%%%%%%%%%%%%%%%%%%%%%%%%%%%%%%%%%%%%%%%%%%%
\subsection{Conditional Processing}
\label{sec:conditional}

The package provides a mechanism to compile different versions
of a document. To customise the versions further some conditional processing
can come in handy to distinguish which version is being compiled.
The package provides two macros to describe the compilation context:

%%%%%%%%%%%%%%%%%%%%%%%%%%%%%%%%%%%%%%%%
\DescribeMacro{\ifchilddoc}
The conditional |\ifchilddoc| distinguishes between the compilation of
child documents and the main document:
%
\begin{center}
|\ifchilddoc |\textit{child-code}| |[|\||else |\textit{main-code}]| \||fi|
\end{center}

%%%%%%%%%%%%%%%%%%%%%%%%%%%%%%%%%%%%%%%%
\DescribeMacro{\childdocname}
\DescribeMacro{\childdocjob}
The macro |\childdocname| contains the filename (without extension)
of the main or child file being processed.
Note that |\childdocjob| will always contain the name of the main file.

%%%%%%%%%%%%%%%%%%%%%%%%%%%%%%%%%%%%%%%%
\paragraph{Title Page.}

Conditional processing can be used to include a title or banner page
in the main document when proper precautions are taken.
Importantly, the code in the main file should ensure that the page counter
(as well as other status parameters which are stored in the |.aux| files)
takes the same value after the conditional processing.
Otherwise the page numbers may take divergent values
depending on which part is compiled.

For example, a title page could be declared by:
%
\begin{center}
\begin{tabular}{l}
|\ifchilddoc\||else|\\
|\addtocounter{page}{-1}|\\
\textit{code for title page}\\
|\newpage|\\
|\||fi|
\end{tabular}
\end{center}
%
A banner page for the child documents can be generated by:
%
\begin{center}
\begin{tabular}{l}
|\ifchilddoc|\\
|\addtocounter{page}{-1}|\\
\textit{code for banner page}\\
|\newpage|\\
|\||fi|
\end{tabular}
\end{center}
%
Here one could write a message such as:
\begin{center}
|This is the part \childdocname{} of \childdocjob{}.|
\end{center}

%%%%%%%%%%%%%%%%%%%%%%%%%%%%%%%%%%%%%%%%%%%%%%%%%%%%%%%%%%%%%%%%%%%%%%%%%%%%%%%%
\subsection{Flags}
\label{sec:flags}

The package makes it easy to generate different versions
of the main or child documents.
To this end compilation flags can be defined
and assigned different default values.
They will be particularly useful in conjunction
with the forwarding mechanism described in \secref{sec:forward}.

For example, it may be useful to have a flag |\version|
which can be set to |draft| or |final|.
The document source will contain some conditional code
depending on the value of |\version|.
Suppose further, the flag should default to |final| for the main file
and to |draft| for child files
which is a natural assignment for editing the document.
This is achieved by placing the following code
in the preamble of the main document
(below the |\childdocmain| directive):
%
\begin{center}
\begin{tabular}{l}
|\ifchilddoc|\\
|\providecommand{\version}{draft}|\\
|\||else|\\
|\providecommand{\version}{final}|\\
|\||fi|
\end{tabular}
\end{center}
%
The definition by |\providecommand| makes sure
that previous definitions are not overwritten.
Further statements |\providecommand{\version}{...}|
can thus be added before the above code to override it.

For the main file, one might add a line
(between |\childdocmain| and the above block)
%
\begin{center}
|%\ifchilddoc\||else\providecommand{\version}{draft}\||fi|
\end{center}
%
which can be uncommented to produce a draft version.
Likewise one can add a line to the very top of a child file
(above the |\childdocof{|\textit{main}|}| directive)
%
\begin{center}
|%\providecommand{\version}{final}|
\end{center}
%
which can be uncommented to produce the final version of this child document.

%%%%%%%%%%%%%%%%%%%%%%%%%%%%%%%%%%%%%%%%%%%%%%%%%%%%%%%%%%%%%%%%%%%%%%%%%%%%%%%%
\subsection{Forwarding}
\label{sec:forward}

Different versions of the main or child documents
using compilation flags as described in \secref{sec:flags}
can be (permanently) stored in different files
for convenient compilation, viewing and distribution.
To this end, the package defines a command
to pass on compilation to a different file:

%%%%%%%%%%%%%%%%%%%%%%%%%%%%%%%%%%%%%%%%
\DescribeMacro{\childdocforward}
The command |\childdocforward| redirects processing to
another source file:
%
\begin{center}
\begin{tabular}{l}
|\input{childdoc.def}|\\
|\childdocforward[|\textit{main}|]{|\textit{dest}|}|\\
\end{tabular}
\end{center}
%
The argument \textit{dest} is the destination file
(without extension).
It should be the main file or one of the child files.
Note that further \textsf{childdoc} directives
such as |\childdocof| and |\childdocforward|
in the indicated file will be processed in this form.
The optional argument \textit{main}
passes on directly to the main file \textit{main}
while pretending to compile the child \textit{dest}.
This form behaves as if \textit{dest}
issues |\childdocof{|\textit{main}|}| right away,
and no further \textsf{childdoc} directives will be processed.

%%%%%%%%%%%%%%%%%%%%%%%%%%%%%%%%%%%%%%%%
\DescribeMacro{\...prefix}
In the alternative form |\childdocforwardprefix|,
%
\begin{center}
\begin{tabular}{l}
|\input{childdoc.def}|\\
|\childdocforwardprefix[|\textit{main}|]{|\textit{prefix}|}{|\textit{dest}|}|
\end{tabular}
\end{center}
%
the destination file is determined by a pattern
depending on the current file:
To make this work, the current file must be called
`{\textit{prefix}\hspace{0.2em}\textit{suffix}}'
with \textit{prefix} matching precisely the argument.
Processing is then passed on to the file
`{\textit{dest}\hspace{0.2em}\textit{suffix}}'.
Surely, the same effect is achieved by
directly specifying the
argument `{\textit{dest}\hspace{0.2em}\textit{suffix}}'
in the first form.
However, that requires to set up a different file
for each child. With the alternative form of the command
all these files can have exactly the same content
which simplifies setting them up and maintaining them.

For example, the following file |draft.tex|
with a compilation flag |\version| as described in \secref{sec:flags}
compiles the main document as a draft:
%
\begin{center}
\begin{tabular}{l}
|\def\version{draft}|\\
|\input{childdoc.def}|\\
|\childdocforward{|\textit{main}|}|
\end{tabular}
\end{center}
%
Likewise, the following files |final|\textit{nn}|.tex|
compile the final version of the child document
|child|\textit{nn}|.tex|:
%
\begin{center}
\begin{tabular}{l}
|\def\version{final}|\\
|\input{childdoc.def}|\\
|\childdocforwardprefix{final}{child}|
\end{tabular}
\end{center}
%

Note that when several versions of a main file and/or of each child file
are to be generated, it may be convenient to set up a |Makefile| or
shell script to automatise the process.

%%%%%%%%%%%%%%%%%%%%%%%%%%%%%%%%%%%%%%%%%%%%%%%%%%%%%%%%%%%%%%%%%%%%%%%%%%%%%%%%
\subsection{Command Line Processing}
\label{sec:commandline}

The effect of redirection files can also be achieved by invoking
the \LaTeX{} compiler with a more elaborate command line.
Most conveniently this should be done as part
of a shell script or a |Makefile|.

When using \textsf{childdoc} in the main file, the following
command lines effectively perform a redirection
(note that depending on the shell being used,
backslashes may have to be doubled: `|\|' $\to$ `|\\|'):
%
\begin{center}
|... -jobname "|\textit{target}|" |\\|"|[\textit{flags}]%
|\input{childdoc.def}\childdocforward[|\textit{main}|]{|\textit{dest}|}"|
\end{center}
%
Here \textit{target} is the name of the output file,
\textit{main} is the name of the main file
and \textit{dest} is the name of the main or child file to be processed
(all filenames without extensions).
The optional argument \textit{main} can be omitted
if \textit{main} matches \textit{dest}.
Optionally, compilation \textit{flags} can be defined via |\def| commands.
This command line makes the \TeX{} engine believe
it is compiling the file \textit{target}
whose content is specified as the latter parameter.
The provided code then forwards the processing to
\textit{main} or \textit{dest} as described in \secref{sec:forward}.

%%%%%%%%%%%%%%%%%%%%%%%%%%%%%%%%%%%%%%%%%%%%%%%%%%%%%%%%%%%%%%%%%%%%%%%%%%%%%%%%
\subsection{Include by Input}
\label{sec:input}

Including child documents by |\include| has some restrictions by design.
Most notably, the content of a child document always occupies
its own set of pages; pages cannot be shared between child documents.
Usually, this behaviour makes perfect sense
because each child document contain an essential part of the document.
However, in some situations it may be desirable to compose
a document from a collection of parts
without having mandatory page breaks between then.
For this case, the package
provides a mechanism to include parts
by |\input| which can also be processed individually.
However, by construction this mechanism
requires manual handling of the content to be output.

%%%%%%%%%%%%%%%%%%%%%%%%%%%%%%%%%%%%%%%%
\DescribeMacro{\ifchilddocmanual}
The main file should be prepared as usual, see \secref{sec:include}.
However, the document body must make a distinction
between processing of an individual part and of the main document, e.g.:
%
\begin{center}
\begin{tabular}{l}
|\ifchilddocmanual|\\
|\input{\childdocname}|\\
|\||else|\\
\textit{document body with }|\input{|\textit{part}|}|\\
|\||fi|
\end{tabular}
\end{center}
%
The conditional |\ifchilddocmanual| is true whenever
a part to be included by |\input| is being compiled,
and the name of the part is stored in |\childdocname|.

%%%%%%%%%%%%%%%%%%%%%%%%%%%%%%%%%%%%%%%%
\DescribeMacro{\childdocby}
Each part to be included by |\input| should start with:
%
\begin{center}
\begin{tabular}{l}
|\input{childdoc.def}|\\
|\childdocby{|\textit{main}|}|\\
\end{tabular}
\end{center}
%
The directive |\childdocby| is similar to |\childdocof|
described in \secref{sec:include},
but the subsequent selection of content must be done manually.
To that end, both |\ifchilddoc| and |\ifchilddocmanual|
will be true upon processing of a part,
and the name of the part is stored in |\childdocname|.
Note that |\jobname| will be set to the filename of the current part
so that each part receives an individual |.aux| file
that does not interfere with the |.aux| file(s) of the main document.
This behaviour can be altered by the alternative form
|\childdocby[*]{|\textit{main}|}| (with a non-empty optional argument)
which uses the |.aux| file of the main document
by setting |\jobname| to \textit{main}.

%%%%%%%%%%%%%%%%%%%%%%%%%%%%%%%%%%%%%%%%%%%%%%%%%%%%%%%%%%%%%%%%%%%%%%%%%%%%%%%%
\subsection{Driver Development}
\label{sec:driver}

The \textsf{childdoc} mechanism can also be use for the development
of definition files such as \LaTeX{} styles or classes.
This case differs from the above setup with multiple parts
included by |\include| in that no |\includeonly| should be invoked.
This can be achieved by starting the include file
(before |\ProvidesPackage|) with:
%
\begin{center}
\begin{tabular}{l}
|\input{childdoc.def}|\\
|\childdocforward{|\textit{main}|}|\\
\end{tabular}
\end{center}
%
or alternatively with:
%
\begin{center}
\begin{tabular}{l}
|\input{childdoc.def}|\\
|\childdocby{|\textit{main}|}|\\
\end{tabular}
\end{center}
%
Both forms have slightly different effects as described above.
The main file is prepared as usual, see \secref{sec:include}.

%%%%%%%%%%%%%%%%%%%%%%%%%%%%%%%%%%%%%%%%%%%%%%%%%%%%%%%%%%%%%%%%%%%%%%%%%%%%%%%%
\subsection{Legacy Detection}
\label{sec:detection}

The directive |\childdocmain| in the main file can detect
whether the complete document or merely a child is to be compiled
even without using the directive |\childdocof|.
This method is deprecated because it is less robust
and there is no compelling reason to use it;
it is merely provided for backward compatibility
and it may be removed in future versions.

If the detection mechanism is to be used,
it is mandatory to correctly specify
the filename of the main file as the argument of |\childdocmain|:
%
\begin{center}
\begin{tabular}{l}
|\input{childdoc.def}|\\
|\childdocmain{|\textit{main}|}|\\
\end{tabular}
\end{center}
%
If |\jobname| does not match the argument \textit{main} of |\childdocmain|,
it is assumed that |\jobname| points to the child file to be compiled.
When using |\childdocmain| with the main file specified as argument,
it suffices to start a child file
with just |\input{|\textit{main}|}|
without loading of the package and using |\childdocof|.
If instead all processing is done
with the appropriate \textsf{childdoc} directives,
the argument of \textit{main} of |\childdocmain| can be empty.

An alternative version of the command line processing described
in \secref{sec:commandline} using the detection mechanism reads:
%
\begin{center}
|... -jobname "|\textit{target}|" "|[\textit{flags}]%
[|\def\jobname{|\textit{dest}|}|]|\input{|\textit{main}|}"|
\end{center}

%%%%%%%%%%%%%%%%%%%%%%%%%%%%%%%%%%%%%%%%%%%%%%%%%%%%%%%%%%%%%%%%%%%%%%%%%%%%%%%%
\subsection{Manual Code}
\label{sec:manual}

In case one cannot be certain whether the definitions file |childdoc.def|
is installed on the target \TeX{} distribution
and one prefers not to ship it,
it is conceivable to paste a few relevant commands into the sources.

To that end, drop all statements |\input{childdoc.def}|
and perform the replacements as outlined below.
Instead of |\childdocmain{|\textit{main}|}| add the following code
to the top of the main file:
%
\begin{center}
\begin{tabular}{l}
|\||ifdefined\childdocname\endinput\||fi\newif\ifchilddoc|\\
|\edef\childdocname{\scantokens\expandafter{\jobname\noexpand}}|\\
|\def\childdocmain{|\textit{main}|}\||ifx\childdocmain\childdocname\||else|\\
|\childdoctrue\includeonly{\childdocname}\let\jobname\childdocmain\||fi|\\
\end{tabular}
\end{center}
%
Instead of |\childdocof{|\textit{main}|}| just include the main file
at the top of each child file:
%
\begin{center}
|\input{|\textit{main}|}|
\end{center}
%
A simple redirection |\childdocforward{|\textit{dest}|}| is achieved by:
%
\begin{center}
|\def\jobname{|\textit{dest}|}\input{\jobname}|
\end{center}
%
The redirection with prefix
|\childdocforwardprefix[|\textit{prefix}|]{|\textit{dest}|}|
is accomplished by:
%
\begin{center}
\begin{tabular}{l}
|{\edef\jobname{\scantokens\expandafter{\jobname\noexpand}}|\\
|\def\redirectjob |\textit{prefix}|#1~~~{\gdef\jobname{|\textit{dest}|#1}}|\\
|\expandafter\redirectjob\jobname~~~}\input{\jobname}|
\end{tabular}
\end{center}

In an alternative approach,
child documents can be compiled by a specific command line
without additional code or specific definitions:
%
\begin{center}
|... -jobname "|\textit{target}|" "|[\textit{flags}]%
|\includeonly{|\textit{dest}|}\input{|\textit{main}|}"|
\end{center}
%

%%%%%%%%%%%%%%%%%%%%%%%%%%%%%%%%%%%%%%%%%%%%%%%%%%%%%%%%%%%%%%%%%%%%%%%%%%%%%%%%
%%%%%%%%%%%%%%%%%%%%%%%%%%%%%%%%%%%%%%%%%%%%%%%%%%%%%%%%%%%%%%%%%%%%%%%%%%%%%%%%
\section{Information}

%%%%%%%%%%%%%%%%%%%%%%%%%%%%%%%%%%%%%%%%%%%%%%%%%%%%%%%%%%%%%%%%%%%%%%%%%%%%%%%%
\subsection{Copyright}

Copyright \copyright{} 2017--2018 Niklas Beisert

This work may be distributed and/or modified under the
conditions of the \LaTeX{} Project Public License, either version 1.3
of this license or (at your option) any later version.
The latest version of this license is in
  \url{http://www.latex-project.org/lppl.txt}
and version 1.3 or later is part of all distributions of \LaTeX{}
version 2005/12/01 or later.

This work has the LPPL maintenance status `maintained'.

The Current Maintainer of this work is Niklas Beisert.

This work consists of the files |README.txt|, |childdoc.ins| and |childdoc.dtx|
as well as the derived files |childdoc.def|, |cdocsamp.tex|
with |cdocsch1.tex|, |cdocsch2.tex|, |cdocspt3.tex|, |cdocspt4.tex|,
|cdocsdrf.tex|, |cdocsfn1.tex|, |cdocsfn2.tex|
as well as |childdoc.pdf|.

%%%%%%%%%%%%%%%%%%%%%%%%%%%%%%%%%%%%%%%%%%%%%%%%%%%%%%%%%%%%%%%%%%%%%%%%%%%%%%%%
\subsection{Files and Installation}

The package consists of the files:
%
\begin{center}
\begin{tabular}{ll}
    |README.txt|   & readme file \\
    |childdoc.ins| & installation file \\
    |childdoc.dtx| & source file \\
    |childdoc.def| & definition file \\
    |cdocsamp.tex| & sample main file \\
    |cdocsch1.tex| & sample include file \\
    |cdocsch2.tex| & sample include file \\
    |cdocspt3.tex| & sample part file \\
    |cdocspt4.tex| & sample part file \\
    |cdocsdrf.tex| & sample redirection file \\
    |cdocsfn1.tex| & sample redirection file \\
    |cdocsfn2.tex| & sample redirection file \\
    |childdoc.pdf| & manual
\end{tabular}
\end{center}
%
The distribution consists of the files
|README.txt|, |childdoc.ins| and |childdoc.dtx|.
%
\begin{itemize}
\item
Run (pdf)\LaTeX{} on |childdoc.dtx|
to compile the manual |childdoc.pdf| (this file).
\item
Run \LaTeX{} on |childdoc.ins| to create the definitions file |childdoc.def|
and the sample |cdocsamp.tex| with include files
|cdocsch1.tex|, |cdocsch2.tex|, |cdocspt3.tex|, |cdocspt4.tex|,
|cdocsdrf.tex|, |cdocsfn1.tex|, |cdocsfn2.tex|.
Then copy the file |childdoc.def| to an appropriate directory of your \LaTeX{}
distribution, e.g.\ \textit{texmf-root}|/tex/latex/childdoc|.
\end{itemize}

%%%%%%%%%%%%%%%%%%%%%%%%%%%%%%%%%%%%%%%%%%%%%%%%%%%%%%%%%%%%%%%%%%%%%%%%%%%%%%%%
\subsection{Related CTAN Packages}

There are several other packages which offer a similar functionality:
%
\begin{itemize}
\item
The packages
\href{http://ctan.org/pkg/docmute}{\textsf{docmute}},
\href{http://ctan.org/pkg/includex}{\textsf{includex}} and
\href{http://ctan.org/pkg/standalone}{\textsf{standalone}}
provide commands to include only the document body of
a child file thus allowing both files to be compiled individually.
\item
The packages \href{http://ctan.org/pkg/subdocs}{\textsf{subdocs}}
and \href{http://ctan.org/pkg/subfiles}{\textsf{subfiles}}
provide structures in which the main and child documents can be
encapsulated and allowing them to be compiled individually.
The inclusion mechanism is different from the conventional |\include|.
\item
The package \href{http://ctan.org/pkg/combine}{\textsf{combine}}
is an elaborate solution to combine several documents into one.
\end{itemize}
%
See also the CTAN topic \href{http://ctan.org/topic/subdocs}{\textsf{subdocs}}
for further related packages.
The present package differs from the above solutions in that
a document structure constructed with the conventional |\include| mechanism
just needs two extra commands at the top of every file
such that all constituent files can be compiled individually.

%%%%%%%%%%%%%%%%%%%%%%%%%%%%%%%%%%%%%%%%%%%%%%%%%%%%%%%%%%%%%%%%%%%%%%%%%%%%%%%%
%\subsection{Feature Suggestions}
%
%The following is a list of features which may be useful for future
%versions of this package:
%%
%\begin{itemize}
%\item
%\ldots
%\end{itemize}

%%%%%%%%%%%%%%%%%%%%%%%%%%%%%%%%%%%%%%%%%%%%%%%%%%%%%%%%%%%%%%%%%%%%%%%%%%%%%%%%
\subsection{Revision History}

%%%%%%%%%%%%%%%%%%%%%%%%%%%%%%%%%%%%%%%%
\paragraph{v2.0:} 2018/12/30

\begin{itemize}
\item
immediate forward processing
\item
added |\childdocby| mechanism
\item
manual restructured
\end{itemize}

%%%%%%%%%%%%%%%%%%%%%%%%%%%%%%%%%%%%%%%%
\paragraph{v1.6:} 2018/01/17

\begin{itemize}
\item
application for development of include files
\item
corrections to manual
\end{itemize}

%%%%%%%%%%%%%%%%%%%%%%%%%%%%%%%%%%%%%%%%
\paragraph{v1.5:} 2017/05/21

\begin{itemize}
\item
more complete structuring introduced
\item
|\childdocof| introduced
\item
|\childdoc| renamed to |\childdocmain|
\item
|\childredirect| renamed to |\childdocforward| and |\childdocforwardprefix|
and functionality expanded
\end{itemize}

%%%%%%%%%%%%%%%%%%%%%%%%%%%%%%%%%%%%%%%%
\paragraph{v1.0:} 2017/04/27

\begin{itemize}
\item
manual and install package
\item
first version published on CTAN
\end{itemize}

%%%%%%%%%%%%%%%%%%%%%%%%%%%%%%%%%%%%%%%%
\paragraph{v0.6:} 2017/04/26

\begin{itemize}
\item
redirection mechanism added
\end{itemize}

%%%%%%%%%%%%%%%%%%%%%%%%%%%%%%%%%%%%%%%%
\paragraph{v0.5:} 2017/04/26

\begin{itemize}
\item
functionality in definition file
\end{itemize}


%%%%%%%%%%%%%%%%%%%%%%%%%%%%%%%%%%%%%%%%%%%%%%%%%%%%%%%%%%%%%%%%%%%%%%%%%%%%%%%%
%%%%%%%%%%%%%%%%%%%%%%%%%%%%%%%%%%%%%%%%%%%%%%%%%%%%%%%%%%%%%%%%%%%%%%%%%%%%%%%%
%%%%%%%%%%%%%%%%%%%%%%%%%%%%%%%%%%%%%%%%%%%%%%%%%%%%%%%%%%%%%%%%%%%%%%%%%%%%%%%%
\appendix

\settowidth\MacroIndent{\rmfamily\scriptsize 000\ }

 \DocInput{childdoc.dtx}

\end{document}
%</driver>
% \fi
%
% %%%%%%%%%%%%%%%%%%%%%%%%%%%%%%%%%%%%%%%%%%%%%%%%%%%%%%%%%%%%%%%%%%%%%%%%%%%%%%
% %%%%%%%%%%%%%%%%%%%%%%%%%%%%%%%%%%%%%%%%%%%%%%%%%%%%%%%%%%%%%%%%%%%%%%%%%%%%%%
% \section{Sample}
%\iffalse
%<*samplemain>
%\fi
%
% The following presents a sample document
% with two chapters, two parts, a title page,
% a compile flag as well as three forwarding files to set the flag.
% It consists of eight |.tex| files:
% \begin{center}
% \begin{tabular}{ll}
% |cdocsamp.tex|&main file\\
% |cdocsch1.tex|&include file for chapter 1\\
% |cdocsch2.tex|&include file for chapter 2\\
% |cdocspt3.tex|&include file for part 3\\
% |cdocspt4.tex|&include file for part 4\\
% |cdocsdrf.tex|&forwarding file for main file in draft mode\\
% |cdocsfi1.tex|&forwarding file for final version of chapter 1\\
% |cdocsfi2.tex|&forwarding file for final version of chapter 2\\
% \end{tabular}
% \end{center}
% Each of the eight files can be compiled directly by the \LaTeX{} compiler.
%
% %%%%%%%%%%%%%%%%%%%%%%%%%%%%%%%%%%%%%%
% \paragraph{Main File.}
%
% The main file is called |cdocsamp.tex|.
%
% Load the \textsf{childdoc} definitions and
% declare the filename for the main document:
%    \begin{macrocode}
\input{childdoc.def}
\childdocmain{}
%    \end{macrocode}

% Optional override for |\version| flag:
%    \begin{macrocode}
%%\ifchilddoc\else\providecommand{\version}{draft}\fi
%    \end{macrocode}

% Define the default values for the |\version| flag
% (|final| for the main file and |draft| for childs):
%    \begin{macrocode}
\ifchilddoc
\providecommand{\version}{draft}
\else
\providecommand{\version}{final}
\fi
%    \end{macrocode}

% Load the standard document class:
%    \begin{macrocode}
\documentclass[12pt]{article}
%    \end{macrocode}

% Start the document body:
%    \begin{macrocode}
\begin{document}
%    \end{macrocode}

% Declare a title page.
% Print title, part of document being processed and version flag:
%    \begin{macrocode}
\addtocounter{page}{-1}
\begin{center}
{\LARGE\bfseries{}childdoc example\par}
\vspace{1cm}
\ifchilddoc
\ifchilddocmanual part\else chapter\fi:
`\childdocname' of `\childdocjob'\par
\else
main document: `\childdocjob'\par
\fi
version: \version\par
\end{center}
\newpage
%    \end{macrocode}

% Manually include selected file,
% otherwise process as usual:
%    \begin{macrocode}
\ifchilddocmanual
\section*{part `\childdocname'}
\input{\childdocname}
\else
%    \end{macrocode}

% Include the two chapters:
%    \begin{macrocode}
\include{cdocsch1}
\include{cdocsch2}
%    \end{macrocode}

% Include the two parts unless only chapters should be displayed:
%    \begin{macrocode}
\ifchilddoc\else
\section{part three}
\input{cdocspt3}
\section{part four}
\input{cdocspt4}
\fi
%    \end{macrocode}

% Process as usual until here:
%    \begin{macrocode}
\fi
%    \end{macrocode}

% End of document body:
%    \begin{macrocode}
\end{document}
%    \end{macrocode}
%\iffalse
%</samplemain>
%\fi
%
% %%%%%%%%%%%%%%%%%%%%%%%%%%%%%%%%%%%%%%
% \paragraph{Chapter Include Files.}
%
% The include files are called |cdocsch1.tex| and |cdocsch2.tex|.
%
%\iffalse
%<*samplechap1|samplechap2>
%\fi

% Optional override for |\version| flag:
%    \begin{macrocode}
%%\providecommand{\version}{final}
%    \end{macrocode}

% Include the main document:
%    \begin{macrocode}
\input{childdoc.def}
\childdocof{cdocsamp}
%    \end{macrocode}

%\iffalse
%</samplechap1|samplechap2>
%\fi
%
%\iffalse
%<*samplechap1>
%\fi
% Some text for chapter 1:
%    \begin{macrocode}
\section{one}
some text in chapter one
%    \end{macrocode}

%\iffalse
%</samplechap1>
%\fi
% Some text for chapter 2:
%\iffalse
%<*samplechap2>
%\fi
%    \begin{macrocode}
\section{two}
more text in chapter two
%    \end{macrocode}

%\iffalse
%</samplechap2>
%\fi
%
% %%%%%%%%%%%%%%%%%%%%%%%%%%%%%%%%%%%%%%
% \paragraph{Part Include Files.}
%
% The include files are called |cdocspt3.tex| and |cdocspt4.tex|.
%
%\iffalse
%<*samplepart3|samplepart4>
%\fi

% Optional override for |\version| flag:
%    \begin{macrocode}
%%\providecommand{\version}{final}
%    \end{macrocode}

% Include the main document:
%    \begin{macrocode}
\input{childdoc.def}
\childdocby{cdocsamp}
%    \end{macrocode}

%\iffalse
%</samplepart3|samplepart4>
%\fi
%
%\iffalse
%<*samplepart3>
%\fi
% Some text for part 3:
%    \begin{macrocode}
some text in part three
%    \end{macrocode}

%\iffalse
%</samplepart3>
%\fi
% Some text for part 4:
%\iffalse
%<*samplepart4>
%\fi
%    \begin{macrocode}
more text in part four
%    \end{macrocode}

%\iffalse
%</samplepart4>
%\fi
%
% %%%%%%%%%%%%%%%%%%%%%%%%%%%%%%%%%%%%%%
% \paragraph{Forwarding for a Complete Draft.}
%
% The following forwarding file |cdocsdrf.tex|
% compiles the main document in draft mode:
%\iffalse
%<*sampledraft>
%\fi
%    \begin{macrocode}
\def\version{draft}
\input{childdoc.def}
\childdocforward{cdocsamp}
%    \end{macrocode}

%\iffalse
%</sampledraft>
%\fi
%
% %%%%%%%%%%%%%%%%%%%%%%%%%%%%%%%%%%%%%%
% \paragraph{Forwarding for Final Version of the Chapters.}
%
% The following forwarding files |cdocsfn1.tex| and |cdocsfn2.tex|
% (with identical content)
% compile the final versions of the child documents
% |cdocsch1.tex| and |cdocsch2.tex|, respectively:
%\iffalse
%<*samplefinal>
%\fi
%    \begin{macrocode}
\def\version{final}
\input{childdoc.def}
\childdocforwardprefix[cdocsamp]{cdocsfn}{cdocsch}
%    \end{macrocode}

%\iffalse
%</samplefinal>
%\fi
%
% %%%%%%%%%%%%%%%%%%%%%%%%%%%%%%%%%%%%%%
% \paragraph{Command Line Processing.}
%
% The following three command lines generate the output files
% |cdocscld|, |cdocscl1| and |cdocscl2|
% which should be identical to
% |cdocsdrf|, |cdocsch1| and |cdocsfn2|, respectively:
% \begin{center}
% \begin{tabular}{l}
% |latex -jobname cdocscld \|\\
% |  "\def\version{draft}\input{childdoc.def}\childdocforward{cdocsamp}"|\\
% |latex -jobname cdocscl1 \|\\
% |  "\input{childdoc.def}\childdocforward[cdocsamp]{cdocsch1}"|\\
% |latex -jobname cdocscl2 \|\\
% |  "\def\version{final}\input{childdoc.def}\childdocforward{cdocsch2}"|
% \end{tabular}
% \end{center}
% Note that the trailing backslash on each first line
% merely continues the input to the second line
% (for convenient cut ant paste).
% Furthermore, the command |latex| can be replaced by any
% of its alternative versions such as |pdflatex|.
%
% %%%%%%%%%%%%%%%%%%%%%%%%%%%%%%%%%%%%%%%%%%%%%%%%%%%%%%%%%%%%%%%%%%%%%%%%%%%%%%
% %%%%%%%%%%%%%%%%%%%%%%%%%%%%%%%%%%%%%%%%%%%%%%%%%%%%%%%%%%%%%%%%%%%%%%%%%%%%%%
% \section{Implementation}
%\iffalse
%<*package>
%\fi
%
% This section describes the definitions file |childdoc.def|.

% The definitions cannot be loaded using |\usepackage| or |\RequirePackage|
% which has a mechanism to prevent loading a style file more than once.
% When loading the definitions by means of |\input|
% multiple instances have to be prevented manually:
%\iffalse
%This code needs to be before the `\ProvidesFile' directive
%which is defined at the beginning of this file.
%Therefore it is also placed there and commented out here.
%</package>
%<*discard>
%\fi
%    \begin{macrocode}
\ifdefined\childdocmain\endinput\fi
%    \end{macrocode}
%\iffalse
%</discard>
%<*package>
%\fi
%
% \macro{\ifchilddoc}
% \macro{\ifchilddocmanual}
% The conditional |\ifchilddoc| tells whether a
% child (true) or main (false) document is being compiled.
% The conditional |\ifchilddocmanual| tells whether
% the |\includeonly| mechanism is used (false) or
% the selection of child files must be performed manually (true).
% The definitions initialise to false:
%    \begin{macrocode}
\newif\ifchilddoc
\newif\ifchilddocmanual
%    \end{macrocode}

% \macro{\childdocname}
% \macro{\childdocjob}
% The macro |\childdocname| stores the name of the main document
% to be compiled. The macro |\childdocjob| stores the name of
% the document on which the \LaTeX{} compiler was originally invoked.
% The content of |\jobname| cannot be compared
% to filenames specified in the source due to different catcodes.
% The following code rescans |\jobname|, stores the result
% in |\childdocname| and saves a copy in |\childdocjob|:
%    \begin{macrocode}
\edef\childdocname{\scantokens\expandafter{\jobname\noexpand}}
\let\childdocjob\childdocname
%    \end{macrocode}

% \macro{\childdocdisable}
% The macro |\childdocdisable| prevents the main file
% from being processed more than once.
% At this stage, the main document command |\childdocmain|
% is assumed to be called once again where it should do nothing.
% Any subsequent call to it should prevent
% a secondary processing of the main document
% It overwrites the forwarding commands
% |\childdocof| and |\childdocforward|
% with empty macros to prevent further inclusions of the main document:
%    \begin{macrocode}
\newcommand{\childdocdisable}
{
  \renewcommand{\childdocmain}[1]{\renewcommand{\childdocmain}[1]{\endinput}}
  \renewcommand{\childdocof}[1]{}
  \renewcommand{\childdocby}[2][]{}
  \renewcommand{\childdocforward}[2][]{}
  \renewcommand{\childdocdisable}{}
}
%    \end{macrocode}

% \macro{\childdocmain}
% The macro |\childdocmain| is to be called at the top of the main file
% with nothing or the main filename (without extension) as argument.
% First, it breaks loops.
% If the argument is not empty and does not match |\childdocname|
% (which is set by the first inclusion of |childdoc.def|),
% |\ifchilddoc| is set to true, |\includeonly| is applied to the child file
% and |\jobname| is set to the main file
% (for proper handling of |.aux| files):
%    \begin{macrocode}
\newcommand{\childdocmain}[1]
{
  \childdocdisable\childdocmain{}
  \if?#1?\else
    \begingroup
      \def\childdoctmp{#1}
      \ifx\childdoctmp\childdocname
        \def\childdoctmp{}
      \else
        \def\childdoctmp
        {
          \childdoctrue
          \includeonly{\childdocname}
          \def\childdocjob{#1}
          \def\jobname{#1}
        }
      \fi
      \expandafter
    \endgroup
    \childdoctmp
  \fi
}
%    \end{macrocode}

% \macro{\childdocof}
% The command |\childdocof| redirects
% compilation to the main file |#1|.
%    \begin{macrocode}
\newcommand{\childdocof}[1]
{
  \childdocdisable
  \childdoctrue
  \includeonly{\childdocname}
  \def\jobname{#1}
  \def\childdocjob{#1}
  \input{#1}
}
%    \end{macrocode}

% \macro{\childdocby}
% The command |\childdocby| ....
%    \begin{macrocode}
\newcommand{\childdocby}[2][]
{
  \childdocdisable
  \childdoctrue
  \childdocmanualtrue
  \if?#1?\else
    \def\jobname{#2}
  \fi
  \def\childdocjob{#2}
  \input{#2}
  \endinput
}
%    \end{macrocode}

% \macro{\childdocforward}
% The command |\childdocforward| redirects
% compilation to the main file or
% (if the optional argument is given) a child file.
% Parameters are set as if the main file
% or a child file starting with |\childdocof| was compiled.
% Then compilation is handed over to the main file:
%    \begin{macrocode}
\newcommand{\childdocforward}[2][]
{
  \begingroup
    \if?#1?
      \def\childdoctmp
      {
        \def\childdocname{#2}
        \def\childdocjob{#2}
        \def\jobname{#2}
        \input{#2}
        \endinput
      }
    \else
      \def\childdoctmp
      {
        \childdocdisable
        \def\childdocname{#2}
        \childdoctrue
        \includeonly{#2}
        \def\childdocjob{#1}
        \def\jobname{#1}
        \input{#1}
        \endinput
      }
    \fi
    \expandafter
  \endgroup
  \childdoctmp
}
%    \end{macrocode}

% \macro{\childdocforwardprefix}
% The command |\childdocforwardprefix| redirects
% compilation to the main or a child file by means of a pattern.
% The prefix |#1| in the current filename is replaced by |#2|
% and the suffix of the current filename is kept
% (it is assumed that the filename does not contain the substring `|~~~|'
% which is used as a delimiter).
% Compilation is handed over to the new file by |\childdocforward|:
%    \begin{macrocode}
\newcommand{\childdocforwardprefix}[3][]
{
  \begingroup
    \def\childdocextract #2##1~~~{\def\childdoctmp{\childdocforward[#1]{#3##1}}}
    \expandafter\childdocextract\childdocname~~~
    \expandafter
  \endgroup
  \childdoctmp
}
%    \end{macrocode}

% \macro{\childdoc}
% The deprecated macro |\childdoc| is a legacy version of |\childdocmain|:
%    \begin{macrocode}
\newcommand{\childdoc}{\childdocmain}
%    \end{macrocode}

% \macro{\childdocredirect}
% The deprecated macro |\childdocredirect| is a legacy version
% of |\childdocforward| and |\childdocforwardprefix|:
%    \begin{macrocode}
\newcommand{\childdocredirect}[2][]
{
  \begingroup
    \if?#1?
      \def\childdoctmp{\childdocforward{#2}}
    \else
      \def\childdoctmp{\childdocforwardprefix{#1}{#2}}
    \fi
    \expandafter
  \endgroup
  \childdoctmp
}
%    \end{macrocode}

%\iffalse
%</package>
%\fi
%
\endinput
|\\
|\childdocforward[|\textit{main}|]{|\textit{dest}|}|\\
\end{tabular}
\end{center}
%
The argument \textit{dest} is the destination file
(without extension).
It should be the main file or one of the child files.
Note that further \textsf{childdoc} directives
such as |\childdocof| and |\childdocforward|
in the indicated file will be processed in this form.
The optional argument \textit{main}
passes on directly to the main file \textit{main}
while pretending to compile the child \textit{dest}.
This form behaves as if \textit{dest}
issues |\childdocof{|\textit{main}|}| right away,
and no further \textsf{childdoc} directives will be processed.

%%%%%%%%%%%%%%%%%%%%%%%%%%%%%%%%%%%%%%%%
\DescribeMacro{\...prefix}
In the alternative form |\childdocforwardprefix|,
%
\begin{center}
\begin{tabular}{l}
|% \iffalse
%
% childdoc.dtx Copyright (C) 2017-2018 Niklas Beisert
%
% This work may be distributed and/or modified under the
% conditions of the LaTeX Project Public License, either version 1.3
% of this license or (at your option) any later version.
% The latest version of this license is in
%   http://www.latex-project.org/lppl.txt
% and version 1.3 or later is part of all distributions of LaTeX
% version 2005/12/01 or later.
%
% This work has the LPPL maintenance status `maintained'.
%
% The Current Maintainer of this work is Niklas Beisert.
%
% This work consists of the files childdoc.dtx and childdoc.ins
% and the derived files childdoc.def and cdocsamp.tex with
% cdocsch1.tex, cdocsch2.tex, cdocsdrf.tex, cdocsfn1.tex, cdocsfn2.tex.
%
%<package>\ifdefined\childdocmain\endinput\fi
%<package>\ProvidesFile{childdoc.def}[2018/12/30 v2.0 child document driver]
%<samplemain>\ProvidesFile{cdocsamp.tex}[2018/12/30 v2.0 sample for childdoc]
%<*driver>
%\ProvidesFile{childdoc.drv}[2018/12/30 v2.0 childdoc reference manual file]
\PassOptionsToClass{10pt,a4paper}{article}
\documentclass{ltxdoc}

\usepackage[margin=35mm]{geometry}
\usepackage{hyperref}
\usepackage{hyperxmp}
\usepackage[usenames]{color}

\hypersetup{colorlinks=true}
\hypersetup{pdfstartview=FitH}
\hypersetup{pdfpagemode=UseNone}
\hypersetup{pdfsource={}}
\hypersetup{pdflang={en-UK}}
\hypersetup{pdfcopyright={Copyright 2017-2018 Niklas Beisert.
  This work may be distributed and/or modified under the
  conditions of the LaTeX Project Public License, either version 1.3
  of this license or (at your option) any later version.}}
\hypersetup{pdflicenseurl={http://www.latex-project.org/lppl.txt}}
\hypersetup{pdfcontactaddress={ETH Zurich, ITP, HIT K,
  Wolfgang-Pauli-Strasse 27}}
\hypersetup{pdfcontactpostcode={8093}}
\hypersetup{pdfcontactcity={Zurich}}
\hypersetup{pdfcontactcountry={Switzerland}}
\hypersetup{pdfcontactemail={nbeisert@itp.phys.ethz.ch}}
\hypersetup{pdfcontacturl={http://people.phys.ethz.ch/\xmptilde nbeisert/}}

\newcommand{\secref}[1]{\hyperref[#1]{section \ref*{#1}}}

\parskip1ex
\parindent0pt
\let\olditemize\itemize
\def\itemize{\olditemize\parskip0pt}

\begin{document}

\title{The \textsf{childdoc} Package}
\hypersetup{pdftitle={The childdoc Package}}
\author{Niklas Beisert\\[2ex]
  Institut f\"ur Theoretische Physik\\
  Eidgen\"ossische Technische Hochschule Z\"urich\\
  Wolfgang-Pauli-Strasse 27, 8093 Z\"urich, Switzerland\\[1ex]
  \href{mailto:nbeisert@itp.phys.ethz.ch}
  {\texttt{nbeisert@itp.phys.ethz.ch}}}
\hypersetup{pdfauthor={Niklas Beisert}}
\hypersetup{pdfsubject={Manual for the LaTeX2e Package childdoc}}
\date{30 December 2018, \textsf{v2.0}}
\maketitle

\begin{abstract}\noindent
\textsf{childdoc} is a \LaTeXe{} package
that enables the direct compilation
of document sections included by |\include|
to individual files.
\end{abstract}

\begingroup
\parskip0ex
\tableofcontents
\endgroup

%%%%%%%%%%%%%%%%%%%%%%%%%%%%%%%%%%%%%%%%%%%%%%%%%%%%%%%%%%%%%%%%%%%%%%%%%%%%%%%%
%%%%%%%%%%%%%%%%%%%%%%%%%%%%%%%%%%%%%%%%%%%%%%%%%%%%%%%%%%%%%%%%%%%%%%%%%%%%%%%%
\section{Introduction}

\LaTeX{} provides a mechanism to structure a large document (such as a book)
into a main file and several child files (containing the chapters)
using the |\include| command.
This mechanism is beneficial for documents
which span hundreds of pages in order to
make the source file(s) more manageable.
Moreover, compilation can be restricted to
selected child files by means of the |\includeonly| command.
The latter feature can be used to reduce the compilation time while editing
(this was significantly more useful in the earlier days of \LaTeX{})
or to generate a smaller document which is easier to navigate.
Another application of |\includeonly| is to generate
documents consisting of selected parts of the complete document.

However, there are a few drawbacks of the plain |\include| mechanism:
\begin{itemize}
\item
The child files cannot be compiled on their own,
they can only be compiled via the main file.
A naive editing environment
(such as a text editor with an option
to have the current file processed by \LaTeX)
may require one to switch to the main file before compiling;
attempting to compile the child file produces errors.
\item
The main file must be modified (each time)
to adjust the |\includeonly| command
to the present needs. This easily leaves the main file in a messy state.
\item
The generated document will always carry the filename
of the main document. This is inconvenient if
several child files are to be compiled and
to be kept for distribution.
\end{itemize}

The present package provides a simple interface
to make child files individually compilable by \LaTeX{}.
Compiling a child file then has the same effect as compiling
the main file with an |\includeonly| command
to select the appropriate child.
Moreover the generated document will carry the name of the child
rather than the main file.
This resolves all three above issues.

This feature is meant to make the editing of books,
thesis documents and lecture notes somewhat more convenient.
However, the package can also be used efficiently for
composing a series of documents (such as exercise sheets)
which are typically distributed individually.
It then assists the author in generating the individual documents
(potentially in different versions)
as well as a document containing the collected series.
Another application is in developing style files
or other kinds of included material
where compilation of the style file could redirect
to a sample or test file.

%%%%%%%%%%%%%%%%%%%%%%%%%%%%%%%%%%%%%%%%%%%%%%%%%%%%%%%%%%%%%%%%%%%%%%%%%%%%%%%%
%%%%%%%%%%%%%%%%%%%%%%%%%%%%%%%%%%%%%%%%%%%%%%%%%%%%%%%%%%%%%%%%%%%%%%%%%%%%%%%%
\section{Usage}

First of all, the package \textsf{childdoc} is \emph{not} a standard
\LaTeXe{} |.sty| style file! Therefore it needs to be invoked in
a non-standard way.

%%%%%%%%%%%%%%%%%%%%%%%%%%%%%%%%%%%%%%%%%%%%%%%%%%%%%%%%%%%%%%%%%%%%%%%%%%%%%%%%
\subsection{Included Files}
\label{sec:include}

%%%%%%%%%%%%%%%%%%%%%%%%%%%%%%%%%%%%%%%%
\DescribeMacro{\childdocmain}
To use the package, add the commands
\begin{center}
\begin{tabular}{l}
|\input{childdoc.def}|\\
|\childdocmain{}|\\
\end{tabular}
\end{center}
at the very top of the main \LaTeX{} file,
in particular \emph{before} the |\documentclass| statement!
The argument of |\childdocmain| should be left empty
(but it must be present).

%%%%%%%%%%%%%%%%%%%%%%%%%%%%%%%%%%%%%%%%
\DescribeMacro{\childdocof}
Furthermore, add the commands
\begin{center}
\begin{tabular}{l}
|\input{childdoc.def}|\\
|\childdocof{|\textit{main}|}|\\
\end{tabular}
\end{center}
at the top of every child file \textit{child}
which is included by |\include{|\textit{child}|}|
from within the main file
(or at least for those files to be compiled individually).
The argument \textit{main} must be the filename of the main file.

There are a couple of
considerations in setting up the main and child documents:

%%%%%%%%%%%%%%%%%%%%%%%%%%%%%%%%%%%%%%%%
\paragraph{Restrictions.}

Please note the following restrictions:
\begin{itemize}
\item
|\childdocmain| must be called with one argument \textit{main}
to ensure compatibility with earlier version of the package.
It must either be empty (|\childdocmain{}|)
or precisely match the filename of the main file in which it is specified.
See \secref{sec:detection} for further information.
\item
The filename \textit{main} must be specified without the |.tex| extension.
\item
The filename \textit{main} is case sensitive
(even in case-insensitive file systems)
due to internal string comparison.
\item
The argument \textit{main} should be fully expanded, it cannot be a macro.
\item
Subdirectories and special characters should be avoided in filenames.
\item
The command |\childdocmain{|\textit{main}|}| must be followed by a whitespace.
It should not be followed immediately by another command
or by a comment mark `|%|'.
This is because the \TeX{} parser reads the token immediately following
the argument of |\childdocmain| and puts it
at the beginning of every child section;
however, a white\-space is ignored.
\end{itemize}

%%%%%%%%%%%%%%%%%%%%%%%%%%%%%%%%%%%%%%%%
\paragraph{Content of Main File.}

It is advisable to place all content in the child files included by |\include|.
Any output contained in the main file will appear in all child documents
unless suppressed manually;
it cannot be suppressed automatically by the |\includeonly| directive
and thus should normally be avoided.
A method to include some content in the main file
by means of conditional processing is described in \secref{sec:conditional}.

%%%%%%%%%%%%%%%%%%%%%%%%%%%%%%%%%%%%%%%%
\paragraph{Page Numbering.}

When only a part of the document is compiled,
the appropriate numbering of pages
(as well as other status parameters)
is determined from the |.aux| files.
The latter contain information from previous passes.
However this information needs to propagate through
all intermediate child documents.
Therefore the page numbering in child documents may well
be inconsistent until the complete document is compiled at least once.

A useful (if unconventional) way to always ensure a consistent
page numbering is to restart the numbering in each child document
and denote the pages by `\textit{child}|.|\textit{page}'
where \textit{child} represents the chapter/section number of the child file.
This can be achieved by the command
|\numberwithin{page}{|\textit{child}|}|
of the \textsf{amsmath} package
where \textit{child} can be |chapter| or |section|
depending on the chosen structuring.
Alternatively, one can modify the macro |\thepage| appropriately
and reset the counter |page| at the start of each child file.

%%%%%%%%%%%%%%%%%%%%%%%%%%%%%%%%%%%%%%%%%%%%%%%%%%%%%%%%%%%%%%%%%%%%%%%%%%%%%%%%
\subsection{Conditional Processing}
\label{sec:conditional}

The package provides a mechanism to compile different versions
of a document. To customise the versions further some conditional processing
can come in handy to distinguish which version is being compiled.
The package provides two macros to describe the compilation context:

%%%%%%%%%%%%%%%%%%%%%%%%%%%%%%%%%%%%%%%%
\DescribeMacro{\ifchilddoc}
The conditional |\ifchilddoc| distinguishes between the compilation of
child documents and the main document:
%
\begin{center}
|\ifchilddoc |\textit{child-code}| |[|\||else |\textit{main-code}]| \||fi|
\end{center}

%%%%%%%%%%%%%%%%%%%%%%%%%%%%%%%%%%%%%%%%
\DescribeMacro{\childdocname}
\DescribeMacro{\childdocjob}
The macro |\childdocname| contains the filename (without extension)
of the main or child file being processed.
Note that |\childdocjob| will always contain the name of the main file.

%%%%%%%%%%%%%%%%%%%%%%%%%%%%%%%%%%%%%%%%
\paragraph{Title Page.}

Conditional processing can be used to include a title or banner page
in the main document when proper precautions are taken.
Importantly, the code in the main file should ensure that the page counter
(as well as other status parameters which are stored in the |.aux| files)
takes the same value after the conditional processing.
Otherwise the page numbers may take divergent values
depending on which part is compiled.

For example, a title page could be declared by:
%
\begin{center}
\begin{tabular}{l}
|\ifchilddoc\||else|\\
|\addtocounter{page}{-1}|\\
\textit{code for title page}\\
|\newpage|\\
|\||fi|
\end{tabular}
\end{center}
%
A banner page for the child documents can be generated by:
%
\begin{center}
\begin{tabular}{l}
|\ifchilddoc|\\
|\addtocounter{page}{-1}|\\
\textit{code for banner page}\\
|\newpage|\\
|\||fi|
\end{tabular}
\end{center}
%
Here one could write a message such as:
\begin{center}
|This is the part \childdocname{} of \childdocjob{}.|
\end{center}

%%%%%%%%%%%%%%%%%%%%%%%%%%%%%%%%%%%%%%%%%%%%%%%%%%%%%%%%%%%%%%%%%%%%%%%%%%%%%%%%
\subsection{Flags}
\label{sec:flags}

The package makes it easy to generate different versions
of the main or child documents.
To this end compilation flags can be defined
and assigned different default values.
They will be particularly useful in conjunction
with the forwarding mechanism described in \secref{sec:forward}.

For example, it may be useful to have a flag |\version|
which can be set to |draft| or |final|.
The document source will contain some conditional code
depending on the value of |\version|.
Suppose further, the flag should default to |final| for the main file
and to |draft| for child files
which is a natural assignment for editing the document.
This is achieved by placing the following code
in the preamble of the main document
(below the |\childdocmain| directive):
%
\begin{center}
\begin{tabular}{l}
|\ifchilddoc|\\
|\providecommand{\version}{draft}|\\
|\||else|\\
|\providecommand{\version}{final}|\\
|\||fi|
\end{tabular}
\end{center}
%
The definition by |\providecommand| makes sure
that previous definitions are not overwritten.
Further statements |\providecommand{\version}{...}|
can thus be added before the above code to override it.

For the main file, one might add a line
(between |\childdocmain| and the above block)
%
\begin{center}
|%\ifchilddoc\||else\providecommand{\version}{draft}\||fi|
\end{center}
%
which can be uncommented to produce a draft version.
Likewise one can add a line to the very top of a child file
(above the |\childdocof{|\textit{main}|}| directive)
%
\begin{center}
|%\providecommand{\version}{final}|
\end{center}
%
which can be uncommented to produce the final version of this child document.

%%%%%%%%%%%%%%%%%%%%%%%%%%%%%%%%%%%%%%%%%%%%%%%%%%%%%%%%%%%%%%%%%%%%%%%%%%%%%%%%
\subsection{Forwarding}
\label{sec:forward}

Different versions of the main or child documents
using compilation flags as described in \secref{sec:flags}
can be (permanently) stored in different files
for convenient compilation, viewing and distribution.
To this end, the package defines a command
to pass on compilation to a different file:

%%%%%%%%%%%%%%%%%%%%%%%%%%%%%%%%%%%%%%%%
\DescribeMacro{\childdocforward}
The command |\childdocforward| redirects processing to
another source file:
%
\begin{center}
\begin{tabular}{l}
|\input{childdoc.def}|\\
|\childdocforward[|\textit{main}|]{|\textit{dest}|}|\\
\end{tabular}
\end{center}
%
The argument \textit{dest} is the destination file
(without extension).
It should be the main file or one of the child files.
Note that further \textsf{childdoc} directives
such as |\childdocof| and |\childdocforward|
in the indicated file will be processed in this form.
The optional argument \textit{main}
passes on directly to the main file \textit{main}
while pretending to compile the child \textit{dest}.
This form behaves as if \textit{dest}
issues |\childdocof{|\textit{main}|}| right away,
and no further \textsf{childdoc} directives will be processed.

%%%%%%%%%%%%%%%%%%%%%%%%%%%%%%%%%%%%%%%%
\DescribeMacro{\...prefix}
In the alternative form |\childdocforwardprefix|,
%
\begin{center}
\begin{tabular}{l}
|\input{childdoc.def}|\\
|\childdocforwardprefix[|\textit{main}|]{|\textit{prefix}|}{|\textit{dest}|}|
\end{tabular}
\end{center}
%
the destination file is determined by a pattern
depending on the current file:
To make this work, the current file must be called
`{\textit{prefix}\hspace{0.2em}\textit{suffix}}'
with \textit{prefix} matching precisely the argument.
Processing is then passed on to the file
`{\textit{dest}\hspace{0.2em}\textit{suffix}}'.
Surely, the same effect is achieved by
directly specifying the
argument `{\textit{dest}\hspace{0.2em}\textit{suffix}}'
in the first form.
However, that requires to set up a different file
for each child. With the alternative form of the command
all these files can have exactly the same content
which simplifies setting them up and maintaining them.

For example, the following file |draft.tex|
with a compilation flag |\version| as described in \secref{sec:flags}
compiles the main document as a draft:
%
\begin{center}
\begin{tabular}{l}
|\def\version{draft}|\\
|\input{childdoc.def}|\\
|\childdocforward{|\textit{main}|}|
\end{tabular}
\end{center}
%
Likewise, the following files |final|\textit{nn}|.tex|
compile the final version of the child document
|child|\textit{nn}|.tex|:
%
\begin{center}
\begin{tabular}{l}
|\def\version{final}|\\
|\input{childdoc.def}|\\
|\childdocforwardprefix{final}{child}|
\end{tabular}
\end{center}
%

Note that when several versions of a main file and/or of each child file
are to be generated, it may be convenient to set up a |Makefile| or
shell script to automatise the process.

%%%%%%%%%%%%%%%%%%%%%%%%%%%%%%%%%%%%%%%%%%%%%%%%%%%%%%%%%%%%%%%%%%%%%%%%%%%%%%%%
\subsection{Command Line Processing}
\label{sec:commandline}

The effect of redirection files can also be achieved by invoking
the \LaTeX{} compiler with a more elaborate command line.
Most conveniently this should be done as part
of a shell script or a |Makefile|.

When using \textsf{childdoc} in the main file, the following
command lines effectively perform a redirection
(note that depending on the shell being used,
backslashes may have to be doubled: `|\|' $\to$ `|\\|'):
%
\begin{center}
|... -jobname "|\textit{target}|" |\\|"|[\textit{flags}]%
|\input{childdoc.def}\childdocforward[|\textit{main}|]{|\textit{dest}|}"|
\end{center}
%
Here \textit{target} is the name of the output file,
\textit{main} is the name of the main file
and \textit{dest} is the name of the main or child file to be processed
(all filenames without extensions).
The optional argument \textit{main} can be omitted
if \textit{main} matches \textit{dest}.
Optionally, compilation \textit{flags} can be defined via |\def| commands.
This command line makes the \TeX{} engine believe
it is compiling the file \textit{target}
whose content is specified as the latter parameter.
The provided code then forwards the processing to
\textit{main} or \textit{dest} as described in \secref{sec:forward}.

%%%%%%%%%%%%%%%%%%%%%%%%%%%%%%%%%%%%%%%%%%%%%%%%%%%%%%%%%%%%%%%%%%%%%%%%%%%%%%%%
\subsection{Include by Input}
\label{sec:input}

Including child documents by |\include| has some restrictions by design.
Most notably, the content of a child document always occupies
its own set of pages; pages cannot be shared between child documents.
Usually, this behaviour makes perfect sense
because each child document contain an essential part of the document.
However, in some situations it may be desirable to compose
a document from a collection of parts
without having mandatory page breaks between then.
For this case, the package
provides a mechanism to include parts
by |\input| which can also be processed individually.
However, by construction this mechanism
requires manual handling of the content to be output.

%%%%%%%%%%%%%%%%%%%%%%%%%%%%%%%%%%%%%%%%
\DescribeMacro{\ifchilddocmanual}
The main file should be prepared as usual, see \secref{sec:include}.
However, the document body must make a distinction
between processing of an individual part and of the main document, e.g.:
%
\begin{center}
\begin{tabular}{l}
|\ifchilddocmanual|\\
|\input{\childdocname}|\\
|\||else|\\
\textit{document body with }|\input{|\textit{part}|}|\\
|\||fi|
\end{tabular}
\end{center}
%
The conditional |\ifchilddocmanual| is true whenever
a part to be included by |\input| is being compiled,
and the name of the part is stored in |\childdocname|.

%%%%%%%%%%%%%%%%%%%%%%%%%%%%%%%%%%%%%%%%
\DescribeMacro{\childdocby}
Each part to be included by |\input| should start with:
%
\begin{center}
\begin{tabular}{l}
|\input{childdoc.def}|\\
|\childdocby{|\textit{main}|}|\\
\end{tabular}
\end{center}
%
The directive |\childdocby| is similar to |\childdocof|
described in \secref{sec:include},
but the subsequent selection of content must be done manually.
To that end, both |\ifchilddoc| and |\ifchilddocmanual|
will be true upon processing of a part,
and the name of the part is stored in |\childdocname|.
Note that |\jobname| will be set to the filename of the current part
so that each part receives an individual |.aux| file
that does not interfere with the |.aux| file(s) of the main document.
This behaviour can be altered by the alternative form
|\childdocby[*]{|\textit{main}|}| (with a non-empty optional argument)
which uses the |.aux| file of the main document
by setting |\jobname| to \textit{main}.

%%%%%%%%%%%%%%%%%%%%%%%%%%%%%%%%%%%%%%%%%%%%%%%%%%%%%%%%%%%%%%%%%%%%%%%%%%%%%%%%
\subsection{Driver Development}
\label{sec:driver}

The \textsf{childdoc} mechanism can also be use for the development
of definition files such as \LaTeX{} styles or classes.
This case differs from the above setup with multiple parts
included by |\include| in that no |\includeonly| should be invoked.
This can be achieved by starting the include file
(before |\ProvidesPackage|) with:
%
\begin{center}
\begin{tabular}{l}
|\input{childdoc.def}|\\
|\childdocforward{|\textit{main}|}|\\
\end{tabular}
\end{center}
%
or alternatively with:
%
\begin{center}
\begin{tabular}{l}
|\input{childdoc.def}|\\
|\childdocby{|\textit{main}|}|\\
\end{tabular}
\end{center}
%
Both forms have slightly different effects as described above.
The main file is prepared as usual, see \secref{sec:include}.

%%%%%%%%%%%%%%%%%%%%%%%%%%%%%%%%%%%%%%%%%%%%%%%%%%%%%%%%%%%%%%%%%%%%%%%%%%%%%%%%
\subsection{Legacy Detection}
\label{sec:detection}

The directive |\childdocmain| in the main file can detect
whether the complete document or merely a child is to be compiled
even without using the directive |\childdocof|.
This method is deprecated because it is less robust
and there is no compelling reason to use it;
it is merely provided for backward compatibility
and it may be removed in future versions.

If the detection mechanism is to be used,
it is mandatory to correctly specify
the filename of the main file as the argument of |\childdocmain|:
%
\begin{center}
\begin{tabular}{l}
|\input{childdoc.def}|\\
|\childdocmain{|\textit{main}|}|\\
\end{tabular}
\end{center}
%
If |\jobname| does not match the argument \textit{main} of |\childdocmain|,
it is assumed that |\jobname| points to the child file to be compiled.
When using |\childdocmain| with the main file specified as argument,
it suffices to start a child file
with just |\input{|\textit{main}|}|
without loading of the package and using |\childdocof|.
If instead all processing is done
with the appropriate \textsf{childdoc} directives,
the argument of \textit{main} of |\childdocmain| can be empty.

An alternative version of the command line processing described
in \secref{sec:commandline} using the detection mechanism reads:
%
\begin{center}
|... -jobname "|\textit{target}|" "|[\textit{flags}]%
[|\def\jobname{|\textit{dest}|}|]|\input{|\textit{main}|}"|
\end{center}

%%%%%%%%%%%%%%%%%%%%%%%%%%%%%%%%%%%%%%%%%%%%%%%%%%%%%%%%%%%%%%%%%%%%%%%%%%%%%%%%
\subsection{Manual Code}
\label{sec:manual}

In case one cannot be certain whether the definitions file |childdoc.def|
is installed on the target \TeX{} distribution
and one prefers not to ship it,
it is conceivable to paste a few relevant commands into the sources.

To that end, drop all statements |\input{childdoc.def}|
and perform the replacements as outlined below.
Instead of |\childdocmain{|\textit{main}|}| add the following code
to the top of the main file:
%
\begin{center}
\begin{tabular}{l}
|\||ifdefined\childdocname\endinput\||fi\newif\ifchilddoc|\\
|\edef\childdocname{\scantokens\expandafter{\jobname\noexpand}}|\\
|\def\childdocmain{|\textit{main}|}\||ifx\childdocmain\childdocname\||else|\\
|\childdoctrue\includeonly{\childdocname}\let\jobname\childdocmain\||fi|\\
\end{tabular}
\end{center}
%
Instead of |\childdocof{|\textit{main}|}| just include the main file
at the top of each child file:
%
\begin{center}
|\input{|\textit{main}|}|
\end{center}
%
A simple redirection |\childdocforward{|\textit{dest}|}| is achieved by:
%
\begin{center}
|\def\jobname{|\textit{dest}|}\input{\jobname}|
\end{center}
%
The redirection with prefix
|\childdocforwardprefix[|\textit{prefix}|]{|\textit{dest}|}|
is accomplished by:
%
\begin{center}
\begin{tabular}{l}
|{\edef\jobname{\scantokens\expandafter{\jobname\noexpand}}|\\
|\def\redirectjob |\textit{prefix}|#1~~~{\gdef\jobname{|\textit{dest}|#1}}|\\
|\expandafter\redirectjob\jobname~~~}\input{\jobname}|
\end{tabular}
\end{center}

In an alternative approach,
child documents can be compiled by a specific command line
without additional code or specific definitions:
%
\begin{center}
|... -jobname "|\textit{target}|" "|[\textit{flags}]%
|\includeonly{|\textit{dest}|}\input{|\textit{main}|}"|
\end{center}
%

%%%%%%%%%%%%%%%%%%%%%%%%%%%%%%%%%%%%%%%%%%%%%%%%%%%%%%%%%%%%%%%%%%%%%%%%%%%%%%%%
%%%%%%%%%%%%%%%%%%%%%%%%%%%%%%%%%%%%%%%%%%%%%%%%%%%%%%%%%%%%%%%%%%%%%%%%%%%%%%%%
\section{Information}

%%%%%%%%%%%%%%%%%%%%%%%%%%%%%%%%%%%%%%%%%%%%%%%%%%%%%%%%%%%%%%%%%%%%%%%%%%%%%%%%
\subsection{Copyright}

Copyright \copyright{} 2017--2018 Niklas Beisert

This work may be distributed and/or modified under the
conditions of the \LaTeX{} Project Public License, either version 1.3
of this license or (at your option) any later version.
The latest version of this license is in
  \url{http://www.latex-project.org/lppl.txt}
and version 1.3 or later is part of all distributions of \LaTeX{}
version 2005/12/01 or later.

This work has the LPPL maintenance status `maintained'.

The Current Maintainer of this work is Niklas Beisert.

This work consists of the files |README.txt|, |childdoc.ins| and |childdoc.dtx|
as well as the derived files |childdoc.def|, |cdocsamp.tex|
with |cdocsch1.tex|, |cdocsch2.tex|, |cdocspt3.tex|, |cdocspt4.tex|,
|cdocsdrf.tex|, |cdocsfn1.tex|, |cdocsfn2.tex|
as well as |childdoc.pdf|.

%%%%%%%%%%%%%%%%%%%%%%%%%%%%%%%%%%%%%%%%%%%%%%%%%%%%%%%%%%%%%%%%%%%%%%%%%%%%%%%%
\subsection{Files and Installation}

The package consists of the files:
%
\begin{center}
\begin{tabular}{ll}
    |README.txt|   & readme file \\
    |childdoc.ins| & installation file \\
    |childdoc.dtx| & source file \\
    |childdoc.def| & definition file \\
    |cdocsamp.tex| & sample main file \\
    |cdocsch1.tex| & sample include file \\
    |cdocsch2.tex| & sample include file \\
    |cdocspt3.tex| & sample part file \\
    |cdocspt4.tex| & sample part file \\
    |cdocsdrf.tex| & sample redirection file \\
    |cdocsfn1.tex| & sample redirection file \\
    |cdocsfn2.tex| & sample redirection file \\
    |childdoc.pdf| & manual
\end{tabular}
\end{center}
%
The distribution consists of the files
|README.txt|, |childdoc.ins| and |childdoc.dtx|.
%
\begin{itemize}
\item
Run (pdf)\LaTeX{} on |childdoc.dtx|
to compile the manual |childdoc.pdf| (this file).
\item
Run \LaTeX{} on |childdoc.ins| to create the definitions file |childdoc.def|
and the sample |cdocsamp.tex| with include files
|cdocsch1.tex|, |cdocsch2.tex|, |cdocspt3.tex|, |cdocspt4.tex|,
|cdocsdrf.tex|, |cdocsfn1.tex|, |cdocsfn2.tex|.
Then copy the file |childdoc.def| to an appropriate directory of your \LaTeX{}
distribution, e.g.\ \textit{texmf-root}|/tex/latex/childdoc|.
\end{itemize}

%%%%%%%%%%%%%%%%%%%%%%%%%%%%%%%%%%%%%%%%%%%%%%%%%%%%%%%%%%%%%%%%%%%%%%%%%%%%%%%%
\subsection{Related CTAN Packages}

There are several other packages which offer a similar functionality:
%
\begin{itemize}
\item
The packages
\href{http://ctan.org/pkg/docmute}{\textsf{docmute}},
\href{http://ctan.org/pkg/includex}{\textsf{includex}} and
\href{http://ctan.org/pkg/standalone}{\textsf{standalone}}
provide commands to include only the document body of
a child file thus allowing both files to be compiled individually.
\item
The packages \href{http://ctan.org/pkg/subdocs}{\textsf{subdocs}}
and \href{http://ctan.org/pkg/subfiles}{\textsf{subfiles}}
provide structures in which the main and child documents can be
encapsulated and allowing them to be compiled individually.
The inclusion mechanism is different from the conventional |\include|.
\item
The package \href{http://ctan.org/pkg/combine}{\textsf{combine}}
is an elaborate solution to combine several documents into one.
\end{itemize}
%
See also the CTAN topic \href{http://ctan.org/topic/subdocs}{\textsf{subdocs}}
for further related packages.
The present package differs from the above solutions in that
a document structure constructed with the conventional |\include| mechanism
just needs two extra commands at the top of every file
such that all constituent files can be compiled individually.

%%%%%%%%%%%%%%%%%%%%%%%%%%%%%%%%%%%%%%%%%%%%%%%%%%%%%%%%%%%%%%%%%%%%%%%%%%%%%%%%
%\subsection{Feature Suggestions}
%
%The following is a list of features which may be useful for future
%versions of this package:
%%
%\begin{itemize}
%\item
%\ldots
%\end{itemize}

%%%%%%%%%%%%%%%%%%%%%%%%%%%%%%%%%%%%%%%%%%%%%%%%%%%%%%%%%%%%%%%%%%%%%%%%%%%%%%%%
\subsection{Revision History}

%%%%%%%%%%%%%%%%%%%%%%%%%%%%%%%%%%%%%%%%
\paragraph{v2.0:} 2018/12/30

\begin{itemize}
\item
immediate forward processing
\item
added |\childdocby| mechanism
\item
manual restructured
\end{itemize}

%%%%%%%%%%%%%%%%%%%%%%%%%%%%%%%%%%%%%%%%
\paragraph{v1.6:} 2018/01/17

\begin{itemize}
\item
application for development of include files
\item
corrections to manual
\end{itemize}

%%%%%%%%%%%%%%%%%%%%%%%%%%%%%%%%%%%%%%%%
\paragraph{v1.5:} 2017/05/21

\begin{itemize}
\item
more complete structuring introduced
\item
|\childdocof| introduced
\item
|\childdoc| renamed to |\childdocmain|
\item
|\childredirect| renamed to |\childdocforward| and |\childdocforwardprefix|
and functionality expanded
\end{itemize}

%%%%%%%%%%%%%%%%%%%%%%%%%%%%%%%%%%%%%%%%
\paragraph{v1.0:} 2017/04/27

\begin{itemize}
\item
manual and install package
\item
first version published on CTAN
\end{itemize}

%%%%%%%%%%%%%%%%%%%%%%%%%%%%%%%%%%%%%%%%
\paragraph{v0.6:} 2017/04/26

\begin{itemize}
\item
redirection mechanism added
\end{itemize}

%%%%%%%%%%%%%%%%%%%%%%%%%%%%%%%%%%%%%%%%
\paragraph{v0.5:} 2017/04/26

\begin{itemize}
\item
functionality in definition file
\end{itemize}


%%%%%%%%%%%%%%%%%%%%%%%%%%%%%%%%%%%%%%%%%%%%%%%%%%%%%%%%%%%%%%%%%%%%%%%%%%%%%%%%
%%%%%%%%%%%%%%%%%%%%%%%%%%%%%%%%%%%%%%%%%%%%%%%%%%%%%%%%%%%%%%%%%%%%%%%%%%%%%%%%
%%%%%%%%%%%%%%%%%%%%%%%%%%%%%%%%%%%%%%%%%%%%%%%%%%%%%%%%%%%%%%%%%%%%%%%%%%%%%%%%
\appendix

\settowidth\MacroIndent{\rmfamily\scriptsize 000\ }

 \DocInput{childdoc.dtx}

\end{document}
%</driver>
% \fi
%
% %%%%%%%%%%%%%%%%%%%%%%%%%%%%%%%%%%%%%%%%%%%%%%%%%%%%%%%%%%%%%%%%%%%%%%%%%%%%%%
% %%%%%%%%%%%%%%%%%%%%%%%%%%%%%%%%%%%%%%%%%%%%%%%%%%%%%%%%%%%%%%%%%%%%%%%%%%%%%%
% \section{Sample}
%\iffalse
%<*samplemain>
%\fi
%
% The following presents a sample document
% with two chapters, two parts, a title page,
% a compile flag as well as three forwarding files to set the flag.
% It consists of eight |.tex| files:
% \begin{center}
% \begin{tabular}{ll}
% |cdocsamp.tex|&main file\\
% |cdocsch1.tex|&include file for chapter 1\\
% |cdocsch2.tex|&include file for chapter 2\\
% |cdocspt3.tex|&include file for part 3\\
% |cdocspt4.tex|&include file for part 4\\
% |cdocsdrf.tex|&forwarding file for main file in draft mode\\
% |cdocsfi1.tex|&forwarding file for final version of chapter 1\\
% |cdocsfi2.tex|&forwarding file for final version of chapter 2\\
% \end{tabular}
% \end{center}
% Each of the eight files can be compiled directly by the \LaTeX{} compiler.
%
% %%%%%%%%%%%%%%%%%%%%%%%%%%%%%%%%%%%%%%
% \paragraph{Main File.}
%
% The main file is called |cdocsamp.tex|.
%
% Load the \textsf{childdoc} definitions and
% declare the filename for the main document:
%    \begin{macrocode}
\input{childdoc.def}
\childdocmain{}
%    \end{macrocode}

% Optional override for |\version| flag:
%    \begin{macrocode}
%%\ifchilddoc\else\providecommand{\version}{draft}\fi
%    \end{macrocode}

% Define the default values for the |\version| flag
% (|final| for the main file and |draft| for childs):
%    \begin{macrocode}
\ifchilddoc
\providecommand{\version}{draft}
\else
\providecommand{\version}{final}
\fi
%    \end{macrocode}

% Load the standard document class:
%    \begin{macrocode}
\documentclass[12pt]{article}
%    \end{macrocode}

% Start the document body:
%    \begin{macrocode}
\begin{document}
%    \end{macrocode}

% Declare a title page.
% Print title, part of document being processed and version flag:
%    \begin{macrocode}
\addtocounter{page}{-1}
\begin{center}
{\LARGE\bfseries{}childdoc example\par}
\vspace{1cm}
\ifchilddoc
\ifchilddocmanual part\else chapter\fi:
`\childdocname' of `\childdocjob'\par
\else
main document: `\childdocjob'\par
\fi
version: \version\par
\end{center}
\newpage
%    \end{macrocode}

% Manually include selected file,
% otherwise process as usual:
%    \begin{macrocode}
\ifchilddocmanual
\section*{part `\childdocname'}
\input{\childdocname}
\else
%    \end{macrocode}

% Include the two chapters:
%    \begin{macrocode}
\include{cdocsch1}
\include{cdocsch2}
%    \end{macrocode}

% Include the two parts unless only chapters should be displayed:
%    \begin{macrocode}
\ifchilddoc\else
\section{part three}
\input{cdocspt3}
\section{part four}
\input{cdocspt4}
\fi
%    \end{macrocode}

% Process as usual until here:
%    \begin{macrocode}
\fi
%    \end{macrocode}

% End of document body:
%    \begin{macrocode}
\end{document}
%    \end{macrocode}
%\iffalse
%</samplemain>
%\fi
%
% %%%%%%%%%%%%%%%%%%%%%%%%%%%%%%%%%%%%%%
% \paragraph{Chapter Include Files.}
%
% The include files are called |cdocsch1.tex| and |cdocsch2.tex|.
%
%\iffalse
%<*samplechap1|samplechap2>
%\fi

% Optional override for |\version| flag:
%    \begin{macrocode}
%%\providecommand{\version}{final}
%    \end{macrocode}

% Include the main document:
%    \begin{macrocode}
\input{childdoc.def}
\childdocof{cdocsamp}
%    \end{macrocode}

%\iffalse
%</samplechap1|samplechap2>
%\fi
%
%\iffalse
%<*samplechap1>
%\fi
% Some text for chapter 1:
%    \begin{macrocode}
\section{one}
some text in chapter one
%    \end{macrocode}

%\iffalse
%</samplechap1>
%\fi
% Some text for chapter 2:
%\iffalse
%<*samplechap2>
%\fi
%    \begin{macrocode}
\section{two}
more text in chapter two
%    \end{macrocode}

%\iffalse
%</samplechap2>
%\fi
%
% %%%%%%%%%%%%%%%%%%%%%%%%%%%%%%%%%%%%%%
% \paragraph{Part Include Files.}
%
% The include files are called |cdocspt3.tex| and |cdocspt4.tex|.
%
%\iffalse
%<*samplepart3|samplepart4>
%\fi

% Optional override for |\version| flag:
%    \begin{macrocode}
%%\providecommand{\version}{final}
%    \end{macrocode}

% Include the main document:
%    \begin{macrocode}
\input{childdoc.def}
\childdocby{cdocsamp}
%    \end{macrocode}

%\iffalse
%</samplepart3|samplepart4>
%\fi
%
%\iffalse
%<*samplepart3>
%\fi
% Some text for part 3:
%    \begin{macrocode}
some text in part three
%    \end{macrocode}

%\iffalse
%</samplepart3>
%\fi
% Some text for part 4:
%\iffalse
%<*samplepart4>
%\fi
%    \begin{macrocode}
more text in part four
%    \end{macrocode}

%\iffalse
%</samplepart4>
%\fi
%
% %%%%%%%%%%%%%%%%%%%%%%%%%%%%%%%%%%%%%%
% \paragraph{Forwarding for a Complete Draft.}
%
% The following forwarding file |cdocsdrf.tex|
% compiles the main document in draft mode:
%\iffalse
%<*sampledraft>
%\fi
%    \begin{macrocode}
\def\version{draft}
\input{childdoc.def}
\childdocforward{cdocsamp}
%    \end{macrocode}

%\iffalse
%</sampledraft>
%\fi
%
% %%%%%%%%%%%%%%%%%%%%%%%%%%%%%%%%%%%%%%
% \paragraph{Forwarding for Final Version of the Chapters.}
%
% The following forwarding files |cdocsfn1.tex| and |cdocsfn2.tex|
% (with identical content)
% compile the final versions of the child documents
% |cdocsch1.tex| and |cdocsch2.tex|, respectively:
%\iffalse
%<*samplefinal>
%\fi
%    \begin{macrocode}
\def\version{final}
\input{childdoc.def}
\childdocforwardprefix[cdocsamp]{cdocsfn}{cdocsch}
%    \end{macrocode}

%\iffalse
%</samplefinal>
%\fi
%
% %%%%%%%%%%%%%%%%%%%%%%%%%%%%%%%%%%%%%%
% \paragraph{Command Line Processing.}
%
% The following three command lines generate the output files
% |cdocscld|, |cdocscl1| and |cdocscl2|
% which should be identical to
% |cdocsdrf|, |cdocsch1| and |cdocsfn2|, respectively:
% \begin{center}
% \begin{tabular}{l}
% |latex -jobname cdocscld \|\\
% |  "\def\version{draft}\input{childdoc.def}\childdocforward{cdocsamp}"|\\
% |latex -jobname cdocscl1 \|\\
% |  "\input{childdoc.def}\childdocforward[cdocsamp]{cdocsch1}"|\\
% |latex -jobname cdocscl2 \|\\
% |  "\def\version{final}\input{childdoc.def}\childdocforward{cdocsch2}"|
% \end{tabular}
% \end{center}
% Note that the trailing backslash on each first line
% merely continues the input to the second line
% (for convenient cut ant paste).
% Furthermore, the command |latex| can be replaced by any
% of its alternative versions such as |pdflatex|.
%
% %%%%%%%%%%%%%%%%%%%%%%%%%%%%%%%%%%%%%%%%%%%%%%%%%%%%%%%%%%%%%%%%%%%%%%%%%%%%%%
% %%%%%%%%%%%%%%%%%%%%%%%%%%%%%%%%%%%%%%%%%%%%%%%%%%%%%%%%%%%%%%%%%%%%%%%%%%%%%%
% \section{Implementation}
%\iffalse
%<*package>
%\fi
%
% This section describes the definitions file |childdoc.def|.

% The definitions cannot be loaded using |\usepackage| or |\RequirePackage|
% which has a mechanism to prevent loading a style file more than once.
% When loading the definitions by means of |\input|
% multiple instances have to be prevented manually:
%\iffalse
%This code needs to be before the `\ProvidesFile' directive
%which is defined at the beginning of this file.
%Therefore it is also placed there and commented out here.
%</package>
%<*discard>
%\fi
%    \begin{macrocode}
\ifdefined\childdocmain\endinput\fi
%    \end{macrocode}
%\iffalse
%</discard>
%<*package>
%\fi
%
% \macro{\ifchilddoc}
% \macro{\ifchilddocmanual}
% The conditional |\ifchilddoc| tells whether a
% child (true) or main (false) document is being compiled.
% The conditional |\ifchilddocmanual| tells whether
% the |\includeonly| mechanism is used (false) or
% the selection of child files must be performed manually (true).
% The definitions initialise to false:
%    \begin{macrocode}
\newif\ifchilddoc
\newif\ifchilddocmanual
%    \end{macrocode}

% \macro{\childdocname}
% \macro{\childdocjob}
% The macro |\childdocname| stores the name of the main document
% to be compiled. The macro |\childdocjob| stores the name of
% the document on which the \LaTeX{} compiler was originally invoked.
% The content of |\jobname| cannot be compared
% to filenames specified in the source due to different catcodes.
% The following code rescans |\jobname|, stores the result
% in |\childdocname| and saves a copy in |\childdocjob|:
%    \begin{macrocode}
\edef\childdocname{\scantokens\expandafter{\jobname\noexpand}}
\let\childdocjob\childdocname
%    \end{macrocode}

% \macro{\childdocdisable}
% The macro |\childdocdisable| prevents the main file
% from being processed more than once.
% At this stage, the main document command |\childdocmain|
% is assumed to be called once again where it should do nothing.
% Any subsequent call to it should prevent
% a secondary processing of the main document
% It overwrites the forwarding commands
% |\childdocof| and |\childdocforward|
% with empty macros to prevent further inclusions of the main document:
%    \begin{macrocode}
\newcommand{\childdocdisable}
{
  \renewcommand{\childdocmain}[1]{\renewcommand{\childdocmain}[1]{\endinput}}
  \renewcommand{\childdocof}[1]{}
  \renewcommand{\childdocby}[2][]{}
  \renewcommand{\childdocforward}[2][]{}
  \renewcommand{\childdocdisable}{}
}
%    \end{macrocode}

% \macro{\childdocmain}
% The macro |\childdocmain| is to be called at the top of the main file
% with nothing or the main filename (without extension) as argument.
% First, it breaks loops.
% If the argument is not empty and does not match |\childdocname|
% (which is set by the first inclusion of |childdoc.def|),
% |\ifchilddoc| is set to true, |\includeonly| is applied to the child file
% and |\jobname| is set to the main file
% (for proper handling of |.aux| files):
%    \begin{macrocode}
\newcommand{\childdocmain}[1]
{
  \childdocdisable\childdocmain{}
  \if?#1?\else
    \begingroup
      \def\childdoctmp{#1}
      \ifx\childdoctmp\childdocname
        \def\childdoctmp{}
      \else
        \def\childdoctmp
        {
          \childdoctrue
          \includeonly{\childdocname}
          \def\childdocjob{#1}
          \def\jobname{#1}
        }
      \fi
      \expandafter
    \endgroup
    \childdoctmp
  \fi
}
%    \end{macrocode}

% \macro{\childdocof}
% The command |\childdocof| redirects
% compilation to the main file |#1|.
%    \begin{macrocode}
\newcommand{\childdocof}[1]
{
  \childdocdisable
  \childdoctrue
  \includeonly{\childdocname}
  \def\jobname{#1}
  \def\childdocjob{#1}
  \input{#1}
}
%    \end{macrocode}

% \macro{\childdocby}
% The command |\childdocby| ....
%    \begin{macrocode}
\newcommand{\childdocby}[2][]
{
  \childdocdisable
  \childdoctrue
  \childdocmanualtrue
  \if?#1?\else
    \def\jobname{#2}
  \fi
  \def\childdocjob{#2}
  \input{#2}
  \endinput
}
%    \end{macrocode}

% \macro{\childdocforward}
% The command |\childdocforward| redirects
% compilation to the main file or
% (if the optional argument is given) a child file.
% Parameters are set as if the main file
% or a child file starting with |\childdocof| was compiled.
% Then compilation is handed over to the main file:
%    \begin{macrocode}
\newcommand{\childdocforward}[2][]
{
  \begingroup
    \if?#1?
      \def\childdoctmp
      {
        \def\childdocname{#2}
        \def\childdocjob{#2}
        \def\jobname{#2}
        \input{#2}
        \endinput
      }
    \else
      \def\childdoctmp
      {
        \childdocdisable
        \def\childdocname{#2}
        \childdoctrue
        \includeonly{#2}
        \def\childdocjob{#1}
        \def\jobname{#1}
        \input{#1}
        \endinput
      }
    \fi
    \expandafter
  \endgroup
  \childdoctmp
}
%    \end{macrocode}

% \macro{\childdocforwardprefix}
% The command |\childdocforwardprefix| redirects
% compilation to the main or a child file by means of a pattern.
% The prefix |#1| in the current filename is replaced by |#2|
% and the suffix of the current filename is kept
% (it is assumed that the filename does not contain the substring `|~~~|'
% which is used as a delimiter).
% Compilation is handed over to the new file by |\childdocforward|:
%    \begin{macrocode}
\newcommand{\childdocforwardprefix}[3][]
{
  \begingroup
    \def\childdocextract #2##1~~~{\def\childdoctmp{\childdocforward[#1]{#3##1}}}
    \expandafter\childdocextract\childdocname~~~
    \expandafter
  \endgroup
  \childdoctmp
}
%    \end{macrocode}

% \macro{\childdoc}
% The deprecated macro |\childdoc| is a legacy version of |\childdocmain|:
%    \begin{macrocode}
\newcommand{\childdoc}{\childdocmain}
%    \end{macrocode}

% \macro{\childdocredirect}
% The deprecated macro |\childdocredirect| is a legacy version
% of |\childdocforward| and |\childdocforwardprefix|:
%    \begin{macrocode}
\newcommand{\childdocredirect}[2][]
{
  \begingroup
    \if?#1?
      \def\childdoctmp{\childdocforward{#2}}
    \else
      \def\childdoctmp{\childdocforwardprefix{#1}{#2}}
    \fi
    \expandafter
  \endgroup
  \childdoctmp
}
%    \end{macrocode}

%\iffalse
%</package>
%\fi
%
\endinput
|\\
|\childdocforwardprefix[|\textit{main}|]{|\textit{prefix}|}{|\textit{dest}|}|
\end{tabular}
\end{center}
%
the destination file is determined by a pattern
depending on the current file:
To make this work, the current file must be called
`{\textit{prefix}\hspace{0.2em}\textit{suffix}}'
with \textit{prefix} matching precisely the argument.
Processing is then passed on to the file
`{\textit{dest}\hspace{0.2em}\textit{suffix}}'.
Surely, the same effect is achieved by
directly specifying the
argument `{\textit{dest}\hspace{0.2em}\textit{suffix}}'
in the first form.
However, that requires to set up a different file
for each child. With the alternative form of the command
all these files can have exactly the same content
which simplifies setting them up and maintaining them.

For example, the following file |draft.tex|
with a compilation flag |\version| as described in \secref{sec:flags}
compiles the main document as a draft:
%
\begin{center}
\begin{tabular}{l}
|\def\version{draft}|\\
|% \iffalse
%
% childdoc.dtx Copyright (C) 2017-2018 Niklas Beisert
%
% This work may be distributed and/or modified under the
% conditions of the LaTeX Project Public License, either version 1.3
% of this license or (at your option) any later version.
% The latest version of this license is in
%   http://www.latex-project.org/lppl.txt
% and version 1.3 or later is part of all distributions of LaTeX
% version 2005/12/01 or later.
%
% This work has the LPPL maintenance status `maintained'.
%
% The Current Maintainer of this work is Niklas Beisert.
%
% This work consists of the files childdoc.dtx and childdoc.ins
% and the derived files childdoc.def and cdocsamp.tex with
% cdocsch1.tex, cdocsch2.tex, cdocsdrf.tex, cdocsfn1.tex, cdocsfn2.tex.
%
%<package>\ifdefined\childdocmain\endinput\fi
%<package>\ProvidesFile{childdoc.def}[2018/12/30 v2.0 child document driver]
%<samplemain>\ProvidesFile{cdocsamp.tex}[2018/12/30 v2.0 sample for childdoc]
%<*driver>
%\ProvidesFile{childdoc.drv}[2018/12/30 v2.0 childdoc reference manual file]
\PassOptionsToClass{10pt,a4paper}{article}
\documentclass{ltxdoc}

\usepackage[margin=35mm]{geometry}
\usepackage{hyperref}
\usepackage{hyperxmp}
\usepackage[usenames]{color}

\hypersetup{colorlinks=true}
\hypersetup{pdfstartview=FitH}
\hypersetup{pdfpagemode=UseNone}
\hypersetup{pdfsource={}}
\hypersetup{pdflang={en-UK}}
\hypersetup{pdfcopyright={Copyright 2017-2018 Niklas Beisert.
  This work may be distributed and/or modified under the
  conditions of the LaTeX Project Public License, either version 1.3
  of this license or (at your option) any later version.}}
\hypersetup{pdflicenseurl={http://www.latex-project.org/lppl.txt}}
\hypersetup{pdfcontactaddress={ETH Zurich, ITP, HIT K,
  Wolfgang-Pauli-Strasse 27}}
\hypersetup{pdfcontactpostcode={8093}}
\hypersetup{pdfcontactcity={Zurich}}
\hypersetup{pdfcontactcountry={Switzerland}}
\hypersetup{pdfcontactemail={nbeisert@itp.phys.ethz.ch}}
\hypersetup{pdfcontacturl={http://people.phys.ethz.ch/\xmptilde nbeisert/}}

\newcommand{\secref}[1]{\hyperref[#1]{section \ref*{#1}}}

\parskip1ex
\parindent0pt
\let\olditemize\itemize
\def\itemize{\olditemize\parskip0pt}

\begin{document}

\title{The \textsf{childdoc} Package}
\hypersetup{pdftitle={The childdoc Package}}
\author{Niklas Beisert\\[2ex]
  Institut f\"ur Theoretische Physik\\
  Eidgen\"ossische Technische Hochschule Z\"urich\\
  Wolfgang-Pauli-Strasse 27, 8093 Z\"urich, Switzerland\\[1ex]
  \href{mailto:nbeisert@itp.phys.ethz.ch}
  {\texttt{nbeisert@itp.phys.ethz.ch}}}
\hypersetup{pdfauthor={Niklas Beisert}}
\hypersetup{pdfsubject={Manual for the LaTeX2e Package childdoc}}
\date{30 December 2018, \textsf{v2.0}}
\maketitle

\begin{abstract}\noindent
\textsf{childdoc} is a \LaTeXe{} package
that enables the direct compilation
of document sections included by |\include|
to individual files.
\end{abstract}

\begingroup
\parskip0ex
\tableofcontents
\endgroup

%%%%%%%%%%%%%%%%%%%%%%%%%%%%%%%%%%%%%%%%%%%%%%%%%%%%%%%%%%%%%%%%%%%%%%%%%%%%%%%%
%%%%%%%%%%%%%%%%%%%%%%%%%%%%%%%%%%%%%%%%%%%%%%%%%%%%%%%%%%%%%%%%%%%%%%%%%%%%%%%%
\section{Introduction}

\LaTeX{} provides a mechanism to structure a large document (such as a book)
into a main file and several child files (containing the chapters)
using the |\include| command.
This mechanism is beneficial for documents
which span hundreds of pages in order to
make the source file(s) more manageable.
Moreover, compilation can be restricted to
selected child files by means of the |\includeonly| command.
The latter feature can be used to reduce the compilation time while editing
(this was significantly more useful in the earlier days of \LaTeX{})
or to generate a smaller document which is easier to navigate.
Another application of |\includeonly| is to generate
documents consisting of selected parts of the complete document.

However, there are a few drawbacks of the plain |\include| mechanism:
\begin{itemize}
\item
The child files cannot be compiled on their own,
they can only be compiled via the main file.
A naive editing environment
(such as a text editor with an option
to have the current file processed by \LaTeX)
may require one to switch to the main file before compiling;
attempting to compile the child file produces errors.
\item
The main file must be modified (each time)
to adjust the |\includeonly| command
to the present needs. This easily leaves the main file in a messy state.
\item
The generated document will always carry the filename
of the main document. This is inconvenient if
several child files are to be compiled and
to be kept for distribution.
\end{itemize}

The present package provides a simple interface
to make child files individually compilable by \LaTeX{}.
Compiling a child file then has the same effect as compiling
the main file with an |\includeonly| command
to select the appropriate child.
Moreover the generated document will carry the name of the child
rather than the main file.
This resolves all three above issues.

This feature is meant to make the editing of books,
thesis documents and lecture notes somewhat more convenient.
However, the package can also be used efficiently for
composing a series of documents (such as exercise sheets)
which are typically distributed individually.
It then assists the author in generating the individual documents
(potentially in different versions)
as well as a document containing the collected series.
Another application is in developing style files
or other kinds of included material
where compilation of the style file could redirect
to a sample or test file.

%%%%%%%%%%%%%%%%%%%%%%%%%%%%%%%%%%%%%%%%%%%%%%%%%%%%%%%%%%%%%%%%%%%%%%%%%%%%%%%%
%%%%%%%%%%%%%%%%%%%%%%%%%%%%%%%%%%%%%%%%%%%%%%%%%%%%%%%%%%%%%%%%%%%%%%%%%%%%%%%%
\section{Usage}

First of all, the package \textsf{childdoc} is \emph{not} a standard
\LaTeXe{} |.sty| style file! Therefore it needs to be invoked in
a non-standard way.

%%%%%%%%%%%%%%%%%%%%%%%%%%%%%%%%%%%%%%%%%%%%%%%%%%%%%%%%%%%%%%%%%%%%%%%%%%%%%%%%
\subsection{Included Files}
\label{sec:include}

%%%%%%%%%%%%%%%%%%%%%%%%%%%%%%%%%%%%%%%%
\DescribeMacro{\childdocmain}
To use the package, add the commands
\begin{center}
\begin{tabular}{l}
|\input{childdoc.def}|\\
|\childdocmain{}|\\
\end{tabular}
\end{center}
at the very top of the main \LaTeX{} file,
in particular \emph{before} the |\documentclass| statement!
The argument of |\childdocmain| should be left empty
(but it must be present).

%%%%%%%%%%%%%%%%%%%%%%%%%%%%%%%%%%%%%%%%
\DescribeMacro{\childdocof}
Furthermore, add the commands
\begin{center}
\begin{tabular}{l}
|\input{childdoc.def}|\\
|\childdocof{|\textit{main}|}|\\
\end{tabular}
\end{center}
at the top of every child file \textit{child}
which is included by |\include{|\textit{child}|}|
from within the main file
(or at least for those files to be compiled individually).
The argument \textit{main} must be the filename of the main file.

There are a couple of
considerations in setting up the main and child documents:

%%%%%%%%%%%%%%%%%%%%%%%%%%%%%%%%%%%%%%%%
\paragraph{Restrictions.}

Please note the following restrictions:
\begin{itemize}
\item
|\childdocmain| must be called with one argument \textit{main}
to ensure compatibility with earlier version of the package.
It must either be empty (|\childdocmain{}|)
or precisely match the filename of the main file in which it is specified.
See \secref{sec:detection} for further information.
\item
The filename \textit{main} must be specified without the |.tex| extension.
\item
The filename \textit{main} is case sensitive
(even in case-insensitive file systems)
due to internal string comparison.
\item
The argument \textit{main} should be fully expanded, it cannot be a macro.
\item
Subdirectories and special characters should be avoided in filenames.
\item
The command |\childdocmain{|\textit{main}|}| must be followed by a whitespace.
It should not be followed immediately by another command
or by a comment mark `|%|'.
This is because the \TeX{} parser reads the token immediately following
the argument of |\childdocmain| and puts it
at the beginning of every child section;
however, a white\-space is ignored.
\end{itemize}

%%%%%%%%%%%%%%%%%%%%%%%%%%%%%%%%%%%%%%%%
\paragraph{Content of Main File.}

It is advisable to place all content in the child files included by |\include|.
Any output contained in the main file will appear in all child documents
unless suppressed manually;
it cannot be suppressed automatically by the |\includeonly| directive
and thus should normally be avoided.
A method to include some content in the main file
by means of conditional processing is described in \secref{sec:conditional}.

%%%%%%%%%%%%%%%%%%%%%%%%%%%%%%%%%%%%%%%%
\paragraph{Page Numbering.}

When only a part of the document is compiled,
the appropriate numbering of pages
(as well as other status parameters)
is determined from the |.aux| files.
The latter contain information from previous passes.
However this information needs to propagate through
all intermediate child documents.
Therefore the page numbering in child documents may well
be inconsistent until the complete document is compiled at least once.

A useful (if unconventional) way to always ensure a consistent
page numbering is to restart the numbering in each child document
and denote the pages by `\textit{child}|.|\textit{page}'
where \textit{child} represents the chapter/section number of the child file.
This can be achieved by the command
|\numberwithin{page}{|\textit{child}|}|
of the \textsf{amsmath} package
where \textit{child} can be |chapter| or |section|
depending on the chosen structuring.
Alternatively, one can modify the macro |\thepage| appropriately
and reset the counter |page| at the start of each child file.

%%%%%%%%%%%%%%%%%%%%%%%%%%%%%%%%%%%%%%%%%%%%%%%%%%%%%%%%%%%%%%%%%%%%%%%%%%%%%%%%
\subsection{Conditional Processing}
\label{sec:conditional}

The package provides a mechanism to compile different versions
of a document. To customise the versions further some conditional processing
can come in handy to distinguish which version is being compiled.
The package provides two macros to describe the compilation context:

%%%%%%%%%%%%%%%%%%%%%%%%%%%%%%%%%%%%%%%%
\DescribeMacro{\ifchilddoc}
The conditional |\ifchilddoc| distinguishes between the compilation of
child documents and the main document:
%
\begin{center}
|\ifchilddoc |\textit{child-code}| |[|\||else |\textit{main-code}]| \||fi|
\end{center}

%%%%%%%%%%%%%%%%%%%%%%%%%%%%%%%%%%%%%%%%
\DescribeMacro{\childdocname}
\DescribeMacro{\childdocjob}
The macro |\childdocname| contains the filename (without extension)
of the main or child file being processed.
Note that |\childdocjob| will always contain the name of the main file.

%%%%%%%%%%%%%%%%%%%%%%%%%%%%%%%%%%%%%%%%
\paragraph{Title Page.}

Conditional processing can be used to include a title or banner page
in the main document when proper precautions are taken.
Importantly, the code in the main file should ensure that the page counter
(as well as other status parameters which are stored in the |.aux| files)
takes the same value after the conditional processing.
Otherwise the page numbers may take divergent values
depending on which part is compiled.

For example, a title page could be declared by:
%
\begin{center}
\begin{tabular}{l}
|\ifchilddoc\||else|\\
|\addtocounter{page}{-1}|\\
\textit{code for title page}\\
|\newpage|\\
|\||fi|
\end{tabular}
\end{center}
%
A banner page for the child documents can be generated by:
%
\begin{center}
\begin{tabular}{l}
|\ifchilddoc|\\
|\addtocounter{page}{-1}|\\
\textit{code for banner page}\\
|\newpage|\\
|\||fi|
\end{tabular}
\end{center}
%
Here one could write a message such as:
\begin{center}
|This is the part \childdocname{} of \childdocjob{}.|
\end{center}

%%%%%%%%%%%%%%%%%%%%%%%%%%%%%%%%%%%%%%%%%%%%%%%%%%%%%%%%%%%%%%%%%%%%%%%%%%%%%%%%
\subsection{Flags}
\label{sec:flags}

The package makes it easy to generate different versions
of the main or child documents.
To this end compilation flags can be defined
and assigned different default values.
They will be particularly useful in conjunction
with the forwarding mechanism described in \secref{sec:forward}.

For example, it may be useful to have a flag |\version|
which can be set to |draft| or |final|.
The document source will contain some conditional code
depending on the value of |\version|.
Suppose further, the flag should default to |final| for the main file
and to |draft| for child files
which is a natural assignment for editing the document.
This is achieved by placing the following code
in the preamble of the main document
(below the |\childdocmain| directive):
%
\begin{center}
\begin{tabular}{l}
|\ifchilddoc|\\
|\providecommand{\version}{draft}|\\
|\||else|\\
|\providecommand{\version}{final}|\\
|\||fi|
\end{tabular}
\end{center}
%
The definition by |\providecommand| makes sure
that previous definitions are not overwritten.
Further statements |\providecommand{\version}{...}|
can thus be added before the above code to override it.

For the main file, one might add a line
(between |\childdocmain| and the above block)
%
\begin{center}
|%\ifchilddoc\||else\providecommand{\version}{draft}\||fi|
\end{center}
%
which can be uncommented to produce a draft version.
Likewise one can add a line to the very top of a child file
(above the |\childdocof{|\textit{main}|}| directive)
%
\begin{center}
|%\providecommand{\version}{final}|
\end{center}
%
which can be uncommented to produce the final version of this child document.

%%%%%%%%%%%%%%%%%%%%%%%%%%%%%%%%%%%%%%%%%%%%%%%%%%%%%%%%%%%%%%%%%%%%%%%%%%%%%%%%
\subsection{Forwarding}
\label{sec:forward}

Different versions of the main or child documents
using compilation flags as described in \secref{sec:flags}
can be (permanently) stored in different files
for convenient compilation, viewing and distribution.
To this end, the package defines a command
to pass on compilation to a different file:

%%%%%%%%%%%%%%%%%%%%%%%%%%%%%%%%%%%%%%%%
\DescribeMacro{\childdocforward}
The command |\childdocforward| redirects processing to
another source file:
%
\begin{center}
\begin{tabular}{l}
|\input{childdoc.def}|\\
|\childdocforward[|\textit{main}|]{|\textit{dest}|}|\\
\end{tabular}
\end{center}
%
The argument \textit{dest} is the destination file
(without extension).
It should be the main file or one of the child files.
Note that further \textsf{childdoc} directives
such as |\childdocof| and |\childdocforward|
in the indicated file will be processed in this form.
The optional argument \textit{main}
passes on directly to the main file \textit{main}
while pretending to compile the child \textit{dest}.
This form behaves as if \textit{dest}
issues |\childdocof{|\textit{main}|}| right away,
and no further \textsf{childdoc} directives will be processed.

%%%%%%%%%%%%%%%%%%%%%%%%%%%%%%%%%%%%%%%%
\DescribeMacro{\...prefix}
In the alternative form |\childdocforwardprefix|,
%
\begin{center}
\begin{tabular}{l}
|\input{childdoc.def}|\\
|\childdocforwardprefix[|\textit{main}|]{|\textit{prefix}|}{|\textit{dest}|}|
\end{tabular}
\end{center}
%
the destination file is determined by a pattern
depending on the current file:
To make this work, the current file must be called
`{\textit{prefix}\hspace{0.2em}\textit{suffix}}'
with \textit{prefix} matching precisely the argument.
Processing is then passed on to the file
`{\textit{dest}\hspace{0.2em}\textit{suffix}}'.
Surely, the same effect is achieved by
directly specifying the
argument `{\textit{dest}\hspace{0.2em}\textit{suffix}}'
in the first form.
However, that requires to set up a different file
for each child. With the alternative form of the command
all these files can have exactly the same content
which simplifies setting them up and maintaining them.

For example, the following file |draft.tex|
with a compilation flag |\version| as described in \secref{sec:flags}
compiles the main document as a draft:
%
\begin{center}
\begin{tabular}{l}
|\def\version{draft}|\\
|\input{childdoc.def}|\\
|\childdocforward{|\textit{main}|}|
\end{tabular}
\end{center}
%
Likewise, the following files |final|\textit{nn}|.tex|
compile the final version of the child document
|child|\textit{nn}|.tex|:
%
\begin{center}
\begin{tabular}{l}
|\def\version{final}|\\
|\input{childdoc.def}|\\
|\childdocforwardprefix{final}{child}|
\end{tabular}
\end{center}
%

Note that when several versions of a main file and/or of each child file
are to be generated, it may be convenient to set up a |Makefile| or
shell script to automatise the process.

%%%%%%%%%%%%%%%%%%%%%%%%%%%%%%%%%%%%%%%%%%%%%%%%%%%%%%%%%%%%%%%%%%%%%%%%%%%%%%%%
\subsection{Command Line Processing}
\label{sec:commandline}

The effect of redirection files can also be achieved by invoking
the \LaTeX{} compiler with a more elaborate command line.
Most conveniently this should be done as part
of a shell script or a |Makefile|.

When using \textsf{childdoc} in the main file, the following
command lines effectively perform a redirection
(note that depending on the shell being used,
backslashes may have to be doubled: `|\|' $\to$ `|\\|'):
%
\begin{center}
|... -jobname "|\textit{target}|" |\\|"|[\textit{flags}]%
|\input{childdoc.def}\childdocforward[|\textit{main}|]{|\textit{dest}|}"|
\end{center}
%
Here \textit{target} is the name of the output file,
\textit{main} is the name of the main file
and \textit{dest} is the name of the main or child file to be processed
(all filenames without extensions).
The optional argument \textit{main} can be omitted
if \textit{main} matches \textit{dest}.
Optionally, compilation \textit{flags} can be defined via |\def| commands.
This command line makes the \TeX{} engine believe
it is compiling the file \textit{target}
whose content is specified as the latter parameter.
The provided code then forwards the processing to
\textit{main} or \textit{dest} as described in \secref{sec:forward}.

%%%%%%%%%%%%%%%%%%%%%%%%%%%%%%%%%%%%%%%%%%%%%%%%%%%%%%%%%%%%%%%%%%%%%%%%%%%%%%%%
\subsection{Include by Input}
\label{sec:input}

Including child documents by |\include| has some restrictions by design.
Most notably, the content of a child document always occupies
its own set of pages; pages cannot be shared between child documents.
Usually, this behaviour makes perfect sense
because each child document contain an essential part of the document.
However, in some situations it may be desirable to compose
a document from a collection of parts
without having mandatory page breaks between then.
For this case, the package
provides a mechanism to include parts
by |\input| which can also be processed individually.
However, by construction this mechanism
requires manual handling of the content to be output.

%%%%%%%%%%%%%%%%%%%%%%%%%%%%%%%%%%%%%%%%
\DescribeMacro{\ifchilddocmanual}
The main file should be prepared as usual, see \secref{sec:include}.
However, the document body must make a distinction
between processing of an individual part and of the main document, e.g.:
%
\begin{center}
\begin{tabular}{l}
|\ifchilddocmanual|\\
|\input{\childdocname}|\\
|\||else|\\
\textit{document body with }|\input{|\textit{part}|}|\\
|\||fi|
\end{tabular}
\end{center}
%
The conditional |\ifchilddocmanual| is true whenever
a part to be included by |\input| is being compiled,
and the name of the part is stored in |\childdocname|.

%%%%%%%%%%%%%%%%%%%%%%%%%%%%%%%%%%%%%%%%
\DescribeMacro{\childdocby}
Each part to be included by |\input| should start with:
%
\begin{center}
\begin{tabular}{l}
|\input{childdoc.def}|\\
|\childdocby{|\textit{main}|}|\\
\end{tabular}
\end{center}
%
The directive |\childdocby| is similar to |\childdocof|
described in \secref{sec:include},
but the subsequent selection of content must be done manually.
To that end, both |\ifchilddoc| and |\ifchilddocmanual|
will be true upon processing of a part,
and the name of the part is stored in |\childdocname|.
Note that |\jobname| will be set to the filename of the current part
so that each part receives an individual |.aux| file
that does not interfere with the |.aux| file(s) of the main document.
This behaviour can be altered by the alternative form
|\childdocby[*]{|\textit{main}|}| (with a non-empty optional argument)
which uses the |.aux| file of the main document
by setting |\jobname| to \textit{main}.

%%%%%%%%%%%%%%%%%%%%%%%%%%%%%%%%%%%%%%%%%%%%%%%%%%%%%%%%%%%%%%%%%%%%%%%%%%%%%%%%
\subsection{Driver Development}
\label{sec:driver}

The \textsf{childdoc} mechanism can also be use for the development
of definition files such as \LaTeX{} styles or classes.
This case differs from the above setup with multiple parts
included by |\include| in that no |\includeonly| should be invoked.
This can be achieved by starting the include file
(before |\ProvidesPackage|) with:
%
\begin{center}
\begin{tabular}{l}
|\input{childdoc.def}|\\
|\childdocforward{|\textit{main}|}|\\
\end{tabular}
\end{center}
%
or alternatively with:
%
\begin{center}
\begin{tabular}{l}
|\input{childdoc.def}|\\
|\childdocby{|\textit{main}|}|\\
\end{tabular}
\end{center}
%
Both forms have slightly different effects as described above.
The main file is prepared as usual, see \secref{sec:include}.

%%%%%%%%%%%%%%%%%%%%%%%%%%%%%%%%%%%%%%%%%%%%%%%%%%%%%%%%%%%%%%%%%%%%%%%%%%%%%%%%
\subsection{Legacy Detection}
\label{sec:detection}

The directive |\childdocmain| in the main file can detect
whether the complete document or merely a child is to be compiled
even without using the directive |\childdocof|.
This method is deprecated because it is less robust
and there is no compelling reason to use it;
it is merely provided for backward compatibility
and it may be removed in future versions.

If the detection mechanism is to be used,
it is mandatory to correctly specify
the filename of the main file as the argument of |\childdocmain|:
%
\begin{center}
\begin{tabular}{l}
|\input{childdoc.def}|\\
|\childdocmain{|\textit{main}|}|\\
\end{tabular}
\end{center}
%
If |\jobname| does not match the argument \textit{main} of |\childdocmain|,
it is assumed that |\jobname| points to the child file to be compiled.
When using |\childdocmain| with the main file specified as argument,
it suffices to start a child file
with just |\input{|\textit{main}|}|
without loading of the package and using |\childdocof|.
If instead all processing is done
with the appropriate \textsf{childdoc} directives,
the argument of \textit{main} of |\childdocmain| can be empty.

An alternative version of the command line processing described
in \secref{sec:commandline} using the detection mechanism reads:
%
\begin{center}
|... -jobname "|\textit{target}|" "|[\textit{flags}]%
[|\def\jobname{|\textit{dest}|}|]|\input{|\textit{main}|}"|
\end{center}

%%%%%%%%%%%%%%%%%%%%%%%%%%%%%%%%%%%%%%%%%%%%%%%%%%%%%%%%%%%%%%%%%%%%%%%%%%%%%%%%
\subsection{Manual Code}
\label{sec:manual}

In case one cannot be certain whether the definitions file |childdoc.def|
is installed on the target \TeX{} distribution
and one prefers not to ship it,
it is conceivable to paste a few relevant commands into the sources.

To that end, drop all statements |\input{childdoc.def}|
and perform the replacements as outlined below.
Instead of |\childdocmain{|\textit{main}|}| add the following code
to the top of the main file:
%
\begin{center}
\begin{tabular}{l}
|\||ifdefined\childdocname\endinput\||fi\newif\ifchilddoc|\\
|\edef\childdocname{\scantokens\expandafter{\jobname\noexpand}}|\\
|\def\childdocmain{|\textit{main}|}\||ifx\childdocmain\childdocname\||else|\\
|\childdoctrue\includeonly{\childdocname}\let\jobname\childdocmain\||fi|\\
\end{tabular}
\end{center}
%
Instead of |\childdocof{|\textit{main}|}| just include the main file
at the top of each child file:
%
\begin{center}
|\input{|\textit{main}|}|
\end{center}
%
A simple redirection |\childdocforward{|\textit{dest}|}| is achieved by:
%
\begin{center}
|\def\jobname{|\textit{dest}|}\input{\jobname}|
\end{center}
%
The redirection with prefix
|\childdocforwardprefix[|\textit{prefix}|]{|\textit{dest}|}|
is accomplished by:
%
\begin{center}
\begin{tabular}{l}
|{\edef\jobname{\scantokens\expandafter{\jobname\noexpand}}|\\
|\def\redirectjob |\textit{prefix}|#1~~~{\gdef\jobname{|\textit{dest}|#1}}|\\
|\expandafter\redirectjob\jobname~~~}\input{\jobname}|
\end{tabular}
\end{center}

In an alternative approach,
child documents can be compiled by a specific command line
without additional code or specific definitions:
%
\begin{center}
|... -jobname "|\textit{target}|" "|[\textit{flags}]%
|\includeonly{|\textit{dest}|}\input{|\textit{main}|}"|
\end{center}
%

%%%%%%%%%%%%%%%%%%%%%%%%%%%%%%%%%%%%%%%%%%%%%%%%%%%%%%%%%%%%%%%%%%%%%%%%%%%%%%%%
%%%%%%%%%%%%%%%%%%%%%%%%%%%%%%%%%%%%%%%%%%%%%%%%%%%%%%%%%%%%%%%%%%%%%%%%%%%%%%%%
\section{Information}

%%%%%%%%%%%%%%%%%%%%%%%%%%%%%%%%%%%%%%%%%%%%%%%%%%%%%%%%%%%%%%%%%%%%%%%%%%%%%%%%
\subsection{Copyright}

Copyright \copyright{} 2017--2018 Niklas Beisert

This work may be distributed and/or modified under the
conditions of the \LaTeX{} Project Public License, either version 1.3
of this license or (at your option) any later version.
The latest version of this license is in
  \url{http://www.latex-project.org/lppl.txt}
and version 1.3 or later is part of all distributions of \LaTeX{}
version 2005/12/01 or later.

This work has the LPPL maintenance status `maintained'.

The Current Maintainer of this work is Niklas Beisert.

This work consists of the files |README.txt|, |childdoc.ins| and |childdoc.dtx|
as well as the derived files |childdoc.def|, |cdocsamp.tex|
with |cdocsch1.tex|, |cdocsch2.tex|, |cdocspt3.tex|, |cdocspt4.tex|,
|cdocsdrf.tex|, |cdocsfn1.tex|, |cdocsfn2.tex|
as well as |childdoc.pdf|.

%%%%%%%%%%%%%%%%%%%%%%%%%%%%%%%%%%%%%%%%%%%%%%%%%%%%%%%%%%%%%%%%%%%%%%%%%%%%%%%%
\subsection{Files and Installation}

The package consists of the files:
%
\begin{center}
\begin{tabular}{ll}
    |README.txt|   & readme file \\
    |childdoc.ins| & installation file \\
    |childdoc.dtx| & source file \\
    |childdoc.def| & definition file \\
    |cdocsamp.tex| & sample main file \\
    |cdocsch1.tex| & sample include file \\
    |cdocsch2.tex| & sample include file \\
    |cdocspt3.tex| & sample part file \\
    |cdocspt4.tex| & sample part file \\
    |cdocsdrf.tex| & sample redirection file \\
    |cdocsfn1.tex| & sample redirection file \\
    |cdocsfn2.tex| & sample redirection file \\
    |childdoc.pdf| & manual
\end{tabular}
\end{center}
%
The distribution consists of the files
|README.txt|, |childdoc.ins| and |childdoc.dtx|.
%
\begin{itemize}
\item
Run (pdf)\LaTeX{} on |childdoc.dtx|
to compile the manual |childdoc.pdf| (this file).
\item
Run \LaTeX{} on |childdoc.ins| to create the definitions file |childdoc.def|
and the sample |cdocsamp.tex| with include files
|cdocsch1.tex|, |cdocsch2.tex|, |cdocspt3.tex|, |cdocspt4.tex|,
|cdocsdrf.tex|, |cdocsfn1.tex|, |cdocsfn2.tex|.
Then copy the file |childdoc.def| to an appropriate directory of your \LaTeX{}
distribution, e.g.\ \textit{texmf-root}|/tex/latex/childdoc|.
\end{itemize}

%%%%%%%%%%%%%%%%%%%%%%%%%%%%%%%%%%%%%%%%%%%%%%%%%%%%%%%%%%%%%%%%%%%%%%%%%%%%%%%%
\subsection{Related CTAN Packages}

There are several other packages which offer a similar functionality:
%
\begin{itemize}
\item
The packages
\href{http://ctan.org/pkg/docmute}{\textsf{docmute}},
\href{http://ctan.org/pkg/includex}{\textsf{includex}} and
\href{http://ctan.org/pkg/standalone}{\textsf{standalone}}
provide commands to include only the document body of
a child file thus allowing both files to be compiled individually.
\item
The packages \href{http://ctan.org/pkg/subdocs}{\textsf{subdocs}}
and \href{http://ctan.org/pkg/subfiles}{\textsf{subfiles}}
provide structures in which the main and child documents can be
encapsulated and allowing them to be compiled individually.
The inclusion mechanism is different from the conventional |\include|.
\item
The package \href{http://ctan.org/pkg/combine}{\textsf{combine}}
is an elaborate solution to combine several documents into one.
\end{itemize}
%
See also the CTAN topic \href{http://ctan.org/topic/subdocs}{\textsf{subdocs}}
for further related packages.
The present package differs from the above solutions in that
a document structure constructed with the conventional |\include| mechanism
just needs two extra commands at the top of every file
such that all constituent files can be compiled individually.

%%%%%%%%%%%%%%%%%%%%%%%%%%%%%%%%%%%%%%%%%%%%%%%%%%%%%%%%%%%%%%%%%%%%%%%%%%%%%%%%
%\subsection{Feature Suggestions}
%
%The following is a list of features which may be useful for future
%versions of this package:
%%
%\begin{itemize}
%\item
%\ldots
%\end{itemize}

%%%%%%%%%%%%%%%%%%%%%%%%%%%%%%%%%%%%%%%%%%%%%%%%%%%%%%%%%%%%%%%%%%%%%%%%%%%%%%%%
\subsection{Revision History}

%%%%%%%%%%%%%%%%%%%%%%%%%%%%%%%%%%%%%%%%
\paragraph{v2.0:} 2018/12/30

\begin{itemize}
\item
immediate forward processing
\item
added |\childdocby| mechanism
\item
manual restructured
\end{itemize}

%%%%%%%%%%%%%%%%%%%%%%%%%%%%%%%%%%%%%%%%
\paragraph{v1.6:} 2018/01/17

\begin{itemize}
\item
application for development of include files
\item
corrections to manual
\end{itemize}

%%%%%%%%%%%%%%%%%%%%%%%%%%%%%%%%%%%%%%%%
\paragraph{v1.5:} 2017/05/21

\begin{itemize}
\item
more complete structuring introduced
\item
|\childdocof| introduced
\item
|\childdoc| renamed to |\childdocmain|
\item
|\childredirect| renamed to |\childdocforward| and |\childdocforwardprefix|
and functionality expanded
\end{itemize}

%%%%%%%%%%%%%%%%%%%%%%%%%%%%%%%%%%%%%%%%
\paragraph{v1.0:} 2017/04/27

\begin{itemize}
\item
manual and install package
\item
first version published on CTAN
\end{itemize}

%%%%%%%%%%%%%%%%%%%%%%%%%%%%%%%%%%%%%%%%
\paragraph{v0.6:} 2017/04/26

\begin{itemize}
\item
redirection mechanism added
\end{itemize}

%%%%%%%%%%%%%%%%%%%%%%%%%%%%%%%%%%%%%%%%
\paragraph{v0.5:} 2017/04/26

\begin{itemize}
\item
functionality in definition file
\end{itemize}


%%%%%%%%%%%%%%%%%%%%%%%%%%%%%%%%%%%%%%%%%%%%%%%%%%%%%%%%%%%%%%%%%%%%%%%%%%%%%%%%
%%%%%%%%%%%%%%%%%%%%%%%%%%%%%%%%%%%%%%%%%%%%%%%%%%%%%%%%%%%%%%%%%%%%%%%%%%%%%%%%
%%%%%%%%%%%%%%%%%%%%%%%%%%%%%%%%%%%%%%%%%%%%%%%%%%%%%%%%%%%%%%%%%%%%%%%%%%%%%%%%
\appendix

\settowidth\MacroIndent{\rmfamily\scriptsize 000\ }

 \DocInput{childdoc.dtx}

\end{document}
%</driver>
% \fi
%
% %%%%%%%%%%%%%%%%%%%%%%%%%%%%%%%%%%%%%%%%%%%%%%%%%%%%%%%%%%%%%%%%%%%%%%%%%%%%%%
% %%%%%%%%%%%%%%%%%%%%%%%%%%%%%%%%%%%%%%%%%%%%%%%%%%%%%%%%%%%%%%%%%%%%%%%%%%%%%%
% \section{Sample}
%\iffalse
%<*samplemain>
%\fi
%
% The following presents a sample document
% with two chapters, two parts, a title page,
% a compile flag as well as three forwarding files to set the flag.
% It consists of eight |.tex| files:
% \begin{center}
% \begin{tabular}{ll}
% |cdocsamp.tex|&main file\\
% |cdocsch1.tex|&include file for chapter 1\\
% |cdocsch2.tex|&include file for chapter 2\\
% |cdocspt3.tex|&include file for part 3\\
% |cdocspt4.tex|&include file for part 4\\
% |cdocsdrf.tex|&forwarding file for main file in draft mode\\
% |cdocsfi1.tex|&forwarding file for final version of chapter 1\\
% |cdocsfi2.tex|&forwarding file for final version of chapter 2\\
% \end{tabular}
% \end{center}
% Each of the eight files can be compiled directly by the \LaTeX{} compiler.
%
% %%%%%%%%%%%%%%%%%%%%%%%%%%%%%%%%%%%%%%
% \paragraph{Main File.}
%
% The main file is called |cdocsamp.tex|.
%
% Load the \textsf{childdoc} definitions and
% declare the filename for the main document:
%    \begin{macrocode}
\input{childdoc.def}
\childdocmain{}
%    \end{macrocode}

% Optional override for |\version| flag:
%    \begin{macrocode}
%%\ifchilddoc\else\providecommand{\version}{draft}\fi
%    \end{macrocode}

% Define the default values for the |\version| flag
% (|final| for the main file and |draft| for childs):
%    \begin{macrocode}
\ifchilddoc
\providecommand{\version}{draft}
\else
\providecommand{\version}{final}
\fi
%    \end{macrocode}

% Load the standard document class:
%    \begin{macrocode}
\documentclass[12pt]{article}
%    \end{macrocode}

% Start the document body:
%    \begin{macrocode}
\begin{document}
%    \end{macrocode}

% Declare a title page.
% Print title, part of document being processed and version flag:
%    \begin{macrocode}
\addtocounter{page}{-1}
\begin{center}
{\LARGE\bfseries{}childdoc example\par}
\vspace{1cm}
\ifchilddoc
\ifchilddocmanual part\else chapter\fi:
`\childdocname' of `\childdocjob'\par
\else
main document: `\childdocjob'\par
\fi
version: \version\par
\end{center}
\newpage
%    \end{macrocode}

% Manually include selected file,
% otherwise process as usual:
%    \begin{macrocode}
\ifchilddocmanual
\section*{part `\childdocname'}
\input{\childdocname}
\else
%    \end{macrocode}

% Include the two chapters:
%    \begin{macrocode}
\include{cdocsch1}
\include{cdocsch2}
%    \end{macrocode}

% Include the two parts unless only chapters should be displayed:
%    \begin{macrocode}
\ifchilddoc\else
\section{part three}
\input{cdocspt3}
\section{part four}
\input{cdocspt4}
\fi
%    \end{macrocode}

% Process as usual until here:
%    \begin{macrocode}
\fi
%    \end{macrocode}

% End of document body:
%    \begin{macrocode}
\end{document}
%    \end{macrocode}
%\iffalse
%</samplemain>
%\fi
%
% %%%%%%%%%%%%%%%%%%%%%%%%%%%%%%%%%%%%%%
% \paragraph{Chapter Include Files.}
%
% The include files are called |cdocsch1.tex| and |cdocsch2.tex|.
%
%\iffalse
%<*samplechap1|samplechap2>
%\fi

% Optional override for |\version| flag:
%    \begin{macrocode}
%%\providecommand{\version}{final}
%    \end{macrocode}

% Include the main document:
%    \begin{macrocode}
\input{childdoc.def}
\childdocof{cdocsamp}
%    \end{macrocode}

%\iffalse
%</samplechap1|samplechap2>
%\fi
%
%\iffalse
%<*samplechap1>
%\fi
% Some text for chapter 1:
%    \begin{macrocode}
\section{one}
some text in chapter one
%    \end{macrocode}

%\iffalse
%</samplechap1>
%\fi
% Some text for chapter 2:
%\iffalse
%<*samplechap2>
%\fi
%    \begin{macrocode}
\section{two}
more text in chapter two
%    \end{macrocode}

%\iffalse
%</samplechap2>
%\fi
%
% %%%%%%%%%%%%%%%%%%%%%%%%%%%%%%%%%%%%%%
% \paragraph{Part Include Files.}
%
% The include files are called |cdocspt3.tex| and |cdocspt4.tex|.
%
%\iffalse
%<*samplepart3|samplepart4>
%\fi

% Optional override for |\version| flag:
%    \begin{macrocode}
%%\providecommand{\version}{final}
%    \end{macrocode}

% Include the main document:
%    \begin{macrocode}
\input{childdoc.def}
\childdocby{cdocsamp}
%    \end{macrocode}

%\iffalse
%</samplepart3|samplepart4>
%\fi
%
%\iffalse
%<*samplepart3>
%\fi
% Some text for part 3:
%    \begin{macrocode}
some text in part three
%    \end{macrocode}

%\iffalse
%</samplepart3>
%\fi
% Some text for part 4:
%\iffalse
%<*samplepart4>
%\fi
%    \begin{macrocode}
more text in part four
%    \end{macrocode}

%\iffalse
%</samplepart4>
%\fi
%
% %%%%%%%%%%%%%%%%%%%%%%%%%%%%%%%%%%%%%%
% \paragraph{Forwarding for a Complete Draft.}
%
% The following forwarding file |cdocsdrf.tex|
% compiles the main document in draft mode:
%\iffalse
%<*sampledraft>
%\fi
%    \begin{macrocode}
\def\version{draft}
\input{childdoc.def}
\childdocforward{cdocsamp}
%    \end{macrocode}

%\iffalse
%</sampledraft>
%\fi
%
% %%%%%%%%%%%%%%%%%%%%%%%%%%%%%%%%%%%%%%
% \paragraph{Forwarding for Final Version of the Chapters.}
%
% The following forwarding files |cdocsfn1.tex| and |cdocsfn2.tex|
% (with identical content)
% compile the final versions of the child documents
% |cdocsch1.tex| and |cdocsch2.tex|, respectively:
%\iffalse
%<*samplefinal>
%\fi
%    \begin{macrocode}
\def\version{final}
\input{childdoc.def}
\childdocforwardprefix[cdocsamp]{cdocsfn}{cdocsch}
%    \end{macrocode}

%\iffalse
%</samplefinal>
%\fi
%
% %%%%%%%%%%%%%%%%%%%%%%%%%%%%%%%%%%%%%%
% \paragraph{Command Line Processing.}
%
% The following three command lines generate the output files
% |cdocscld|, |cdocscl1| and |cdocscl2|
% which should be identical to
% |cdocsdrf|, |cdocsch1| and |cdocsfn2|, respectively:
% \begin{center}
% \begin{tabular}{l}
% |latex -jobname cdocscld \|\\
% |  "\def\version{draft}\input{childdoc.def}\childdocforward{cdocsamp}"|\\
% |latex -jobname cdocscl1 \|\\
% |  "\input{childdoc.def}\childdocforward[cdocsamp]{cdocsch1}"|\\
% |latex -jobname cdocscl2 \|\\
% |  "\def\version{final}\input{childdoc.def}\childdocforward{cdocsch2}"|
% \end{tabular}
% \end{center}
% Note that the trailing backslash on each first line
% merely continues the input to the second line
% (for convenient cut ant paste).
% Furthermore, the command |latex| can be replaced by any
% of its alternative versions such as |pdflatex|.
%
% %%%%%%%%%%%%%%%%%%%%%%%%%%%%%%%%%%%%%%%%%%%%%%%%%%%%%%%%%%%%%%%%%%%%%%%%%%%%%%
% %%%%%%%%%%%%%%%%%%%%%%%%%%%%%%%%%%%%%%%%%%%%%%%%%%%%%%%%%%%%%%%%%%%%%%%%%%%%%%
% \section{Implementation}
%\iffalse
%<*package>
%\fi
%
% This section describes the definitions file |childdoc.def|.

% The definitions cannot be loaded using |\usepackage| or |\RequirePackage|
% which has a mechanism to prevent loading a style file more than once.
% When loading the definitions by means of |\input|
% multiple instances have to be prevented manually:
%\iffalse
%This code needs to be before the `\ProvidesFile' directive
%which is defined at the beginning of this file.
%Therefore it is also placed there and commented out here.
%</package>
%<*discard>
%\fi
%    \begin{macrocode}
\ifdefined\childdocmain\endinput\fi
%    \end{macrocode}
%\iffalse
%</discard>
%<*package>
%\fi
%
% \macro{\ifchilddoc}
% \macro{\ifchilddocmanual}
% The conditional |\ifchilddoc| tells whether a
% child (true) or main (false) document is being compiled.
% The conditional |\ifchilddocmanual| tells whether
% the |\includeonly| mechanism is used (false) or
% the selection of child files must be performed manually (true).
% The definitions initialise to false:
%    \begin{macrocode}
\newif\ifchilddoc
\newif\ifchilddocmanual
%    \end{macrocode}

% \macro{\childdocname}
% \macro{\childdocjob}
% The macro |\childdocname| stores the name of the main document
% to be compiled. The macro |\childdocjob| stores the name of
% the document on which the \LaTeX{} compiler was originally invoked.
% The content of |\jobname| cannot be compared
% to filenames specified in the source due to different catcodes.
% The following code rescans |\jobname|, stores the result
% in |\childdocname| and saves a copy in |\childdocjob|:
%    \begin{macrocode}
\edef\childdocname{\scantokens\expandafter{\jobname\noexpand}}
\let\childdocjob\childdocname
%    \end{macrocode}

% \macro{\childdocdisable}
% The macro |\childdocdisable| prevents the main file
% from being processed more than once.
% At this stage, the main document command |\childdocmain|
% is assumed to be called once again where it should do nothing.
% Any subsequent call to it should prevent
% a secondary processing of the main document
% It overwrites the forwarding commands
% |\childdocof| and |\childdocforward|
% with empty macros to prevent further inclusions of the main document:
%    \begin{macrocode}
\newcommand{\childdocdisable}
{
  \renewcommand{\childdocmain}[1]{\renewcommand{\childdocmain}[1]{\endinput}}
  \renewcommand{\childdocof}[1]{}
  \renewcommand{\childdocby}[2][]{}
  \renewcommand{\childdocforward}[2][]{}
  \renewcommand{\childdocdisable}{}
}
%    \end{macrocode}

% \macro{\childdocmain}
% The macro |\childdocmain| is to be called at the top of the main file
% with nothing or the main filename (without extension) as argument.
% First, it breaks loops.
% If the argument is not empty and does not match |\childdocname|
% (which is set by the first inclusion of |childdoc.def|),
% |\ifchilddoc| is set to true, |\includeonly| is applied to the child file
% and |\jobname| is set to the main file
% (for proper handling of |.aux| files):
%    \begin{macrocode}
\newcommand{\childdocmain}[1]
{
  \childdocdisable\childdocmain{}
  \if?#1?\else
    \begingroup
      \def\childdoctmp{#1}
      \ifx\childdoctmp\childdocname
        \def\childdoctmp{}
      \else
        \def\childdoctmp
        {
          \childdoctrue
          \includeonly{\childdocname}
          \def\childdocjob{#1}
          \def\jobname{#1}
        }
      \fi
      \expandafter
    \endgroup
    \childdoctmp
  \fi
}
%    \end{macrocode}

% \macro{\childdocof}
% The command |\childdocof| redirects
% compilation to the main file |#1|.
%    \begin{macrocode}
\newcommand{\childdocof}[1]
{
  \childdocdisable
  \childdoctrue
  \includeonly{\childdocname}
  \def\jobname{#1}
  \def\childdocjob{#1}
  \input{#1}
}
%    \end{macrocode}

% \macro{\childdocby}
% The command |\childdocby| ....
%    \begin{macrocode}
\newcommand{\childdocby}[2][]
{
  \childdocdisable
  \childdoctrue
  \childdocmanualtrue
  \if?#1?\else
    \def\jobname{#2}
  \fi
  \def\childdocjob{#2}
  \input{#2}
  \endinput
}
%    \end{macrocode}

% \macro{\childdocforward}
% The command |\childdocforward| redirects
% compilation to the main file or
% (if the optional argument is given) a child file.
% Parameters are set as if the main file
% or a child file starting with |\childdocof| was compiled.
% Then compilation is handed over to the main file:
%    \begin{macrocode}
\newcommand{\childdocforward}[2][]
{
  \begingroup
    \if?#1?
      \def\childdoctmp
      {
        \def\childdocname{#2}
        \def\childdocjob{#2}
        \def\jobname{#2}
        \input{#2}
        \endinput
      }
    \else
      \def\childdoctmp
      {
        \childdocdisable
        \def\childdocname{#2}
        \childdoctrue
        \includeonly{#2}
        \def\childdocjob{#1}
        \def\jobname{#1}
        \input{#1}
        \endinput
      }
    \fi
    \expandafter
  \endgroup
  \childdoctmp
}
%    \end{macrocode}

% \macro{\childdocforwardprefix}
% The command |\childdocforwardprefix| redirects
% compilation to the main or a child file by means of a pattern.
% The prefix |#1| in the current filename is replaced by |#2|
% and the suffix of the current filename is kept
% (it is assumed that the filename does not contain the substring `|~~~|'
% which is used as a delimiter).
% Compilation is handed over to the new file by |\childdocforward|:
%    \begin{macrocode}
\newcommand{\childdocforwardprefix}[3][]
{
  \begingroup
    \def\childdocextract #2##1~~~{\def\childdoctmp{\childdocforward[#1]{#3##1}}}
    \expandafter\childdocextract\childdocname~~~
    \expandafter
  \endgroup
  \childdoctmp
}
%    \end{macrocode}

% \macro{\childdoc}
% The deprecated macro |\childdoc| is a legacy version of |\childdocmain|:
%    \begin{macrocode}
\newcommand{\childdoc}{\childdocmain}
%    \end{macrocode}

% \macro{\childdocredirect}
% The deprecated macro |\childdocredirect| is a legacy version
% of |\childdocforward| and |\childdocforwardprefix|:
%    \begin{macrocode}
\newcommand{\childdocredirect}[2][]
{
  \begingroup
    \if?#1?
      \def\childdoctmp{\childdocforward{#2}}
    \else
      \def\childdoctmp{\childdocforwardprefix{#1}{#2}}
    \fi
    \expandafter
  \endgroup
  \childdoctmp
}
%    \end{macrocode}

%\iffalse
%</package>
%\fi
%
\endinput
|\\
|\childdocforward{|\textit{main}|}|
\end{tabular}
\end{center}
%
Likewise, the following files |final|\textit{nn}|.tex|
compile the final version of the child document
|child|\textit{nn}|.tex|:
%
\begin{center}
\begin{tabular}{l}
|\def\version{final}|\\
|% \iffalse
%
% childdoc.dtx Copyright (C) 2017-2018 Niklas Beisert
%
% This work may be distributed and/or modified under the
% conditions of the LaTeX Project Public License, either version 1.3
% of this license or (at your option) any later version.
% The latest version of this license is in
%   http://www.latex-project.org/lppl.txt
% and version 1.3 or later is part of all distributions of LaTeX
% version 2005/12/01 or later.
%
% This work has the LPPL maintenance status `maintained'.
%
% The Current Maintainer of this work is Niklas Beisert.
%
% This work consists of the files childdoc.dtx and childdoc.ins
% and the derived files childdoc.def and cdocsamp.tex with
% cdocsch1.tex, cdocsch2.tex, cdocsdrf.tex, cdocsfn1.tex, cdocsfn2.tex.
%
%<package>\ifdefined\childdocmain\endinput\fi
%<package>\ProvidesFile{childdoc.def}[2018/12/30 v2.0 child document driver]
%<samplemain>\ProvidesFile{cdocsamp.tex}[2018/12/30 v2.0 sample for childdoc]
%<*driver>
%\ProvidesFile{childdoc.drv}[2018/12/30 v2.0 childdoc reference manual file]
\PassOptionsToClass{10pt,a4paper}{article}
\documentclass{ltxdoc}

\usepackage[margin=35mm]{geometry}
\usepackage{hyperref}
\usepackage{hyperxmp}
\usepackage[usenames]{color}

\hypersetup{colorlinks=true}
\hypersetup{pdfstartview=FitH}
\hypersetup{pdfpagemode=UseNone}
\hypersetup{pdfsource={}}
\hypersetup{pdflang={en-UK}}
\hypersetup{pdfcopyright={Copyright 2017-2018 Niklas Beisert.
  This work may be distributed and/or modified under the
  conditions of the LaTeX Project Public License, either version 1.3
  of this license or (at your option) any later version.}}
\hypersetup{pdflicenseurl={http://www.latex-project.org/lppl.txt}}
\hypersetup{pdfcontactaddress={ETH Zurich, ITP, HIT K,
  Wolfgang-Pauli-Strasse 27}}
\hypersetup{pdfcontactpostcode={8093}}
\hypersetup{pdfcontactcity={Zurich}}
\hypersetup{pdfcontactcountry={Switzerland}}
\hypersetup{pdfcontactemail={nbeisert@itp.phys.ethz.ch}}
\hypersetup{pdfcontacturl={http://people.phys.ethz.ch/\xmptilde nbeisert/}}

\newcommand{\secref}[1]{\hyperref[#1]{section \ref*{#1}}}

\parskip1ex
\parindent0pt
\let\olditemize\itemize
\def\itemize{\olditemize\parskip0pt}

\begin{document}

\title{The \textsf{childdoc} Package}
\hypersetup{pdftitle={The childdoc Package}}
\author{Niklas Beisert\\[2ex]
  Institut f\"ur Theoretische Physik\\
  Eidgen\"ossische Technische Hochschule Z\"urich\\
  Wolfgang-Pauli-Strasse 27, 8093 Z\"urich, Switzerland\\[1ex]
  \href{mailto:nbeisert@itp.phys.ethz.ch}
  {\texttt{nbeisert@itp.phys.ethz.ch}}}
\hypersetup{pdfauthor={Niklas Beisert}}
\hypersetup{pdfsubject={Manual for the LaTeX2e Package childdoc}}
\date{30 December 2018, \textsf{v2.0}}
\maketitle

\begin{abstract}\noindent
\textsf{childdoc} is a \LaTeXe{} package
that enables the direct compilation
of document sections included by |\include|
to individual files.
\end{abstract}

\begingroup
\parskip0ex
\tableofcontents
\endgroup

%%%%%%%%%%%%%%%%%%%%%%%%%%%%%%%%%%%%%%%%%%%%%%%%%%%%%%%%%%%%%%%%%%%%%%%%%%%%%%%%
%%%%%%%%%%%%%%%%%%%%%%%%%%%%%%%%%%%%%%%%%%%%%%%%%%%%%%%%%%%%%%%%%%%%%%%%%%%%%%%%
\section{Introduction}

\LaTeX{} provides a mechanism to structure a large document (such as a book)
into a main file and several child files (containing the chapters)
using the |\include| command.
This mechanism is beneficial for documents
which span hundreds of pages in order to
make the source file(s) more manageable.
Moreover, compilation can be restricted to
selected child files by means of the |\includeonly| command.
The latter feature can be used to reduce the compilation time while editing
(this was significantly more useful in the earlier days of \LaTeX{})
or to generate a smaller document which is easier to navigate.
Another application of |\includeonly| is to generate
documents consisting of selected parts of the complete document.

However, there are a few drawbacks of the plain |\include| mechanism:
\begin{itemize}
\item
The child files cannot be compiled on their own,
they can only be compiled via the main file.
A naive editing environment
(such as a text editor with an option
to have the current file processed by \LaTeX)
may require one to switch to the main file before compiling;
attempting to compile the child file produces errors.
\item
The main file must be modified (each time)
to adjust the |\includeonly| command
to the present needs. This easily leaves the main file in a messy state.
\item
The generated document will always carry the filename
of the main document. This is inconvenient if
several child files are to be compiled and
to be kept for distribution.
\end{itemize}

The present package provides a simple interface
to make child files individually compilable by \LaTeX{}.
Compiling a child file then has the same effect as compiling
the main file with an |\includeonly| command
to select the appropriate child.
Moreover the generated document will carry the name of the child
rather than the main file.
This resolves all three above issues.

This feature is meant to make the editing of books,
thesis documents and lecture notes somewhat more convenient.
However, the package can also be used efficiently for
composing a series of documents (such as exercise sheets)
which are typically distributed individually.
It then assists the author in generating the individual documents
(potentially in different versions)
as well as a document containing the collected series.
Another application is in developing style files
or other kinds of included material
where compilation of the style file could redirect
to a sample or test file.

%%%%%%%%%%%%%%%%%%%%%%%%%%%%%%%%%%%%%%%%%%%%%%%%%%%%%%%%%%%%%%%%%%%%%%%%%%%%%%%%
%%%%%%%%%%%%%%%%%%%%%%%%%%%%%%%%%%%%%%%%%%%%%%%%%%%%%%%%%%%%%%%%%%%%%%%%%%%%%%%%
\section{Usage}

First of all, the package \textsf{childdoc} is \emph{not} a standard
\LaTeXe{} |.sty| style file! Therefore it needs to be invoked in
a non-standard way.

%%%%%%%%%%%%%%%%%%%%%%%%%%%%%%%%%%%%%%%%%%%%%%%%%%%%%%%%%%%%%%%%%%%%%%%%%%%%%%%%
\subsection{Included Files}
\label{sec:include}

%%%%%%%%%%%%%%%%%%%%%%%%%%%%%%%%%%%%%%%%
\DescribeMacro{\childdocmain}
To use the package, add the commands
\begin{center}
\begin{tabular}{l}
|\input{childdoc.def}|\\
|\childdocmain{}|\\
\end{tabular}
\end{center}
at the very top of the main \LaTeX{} file,
in particular \emph{before} the |\documentclass| statement!
The argument of |\childdocmain| should be left empty
(but it must be present).

%%%%%%%%%%%%%%%%%%%%%%%%%%%%%%%%%%%%%%%%
\DescribeMacro{\childdocof}
Furthermore, add the commands
\begin{center}
\begin{tabular}{l}
|\input{childdoc.def}|\\
|\childdocof{|\textit{main}|}|\\
\end{tabular}
\end{center}
at the top of every child file \textit{child}
which is included by |\include{|\textit{child}|}|
from within the main file
(or at least for those files to be compiled individually).
The argument \textit{main} must be the filename of the main file.

There are a couple of
considerations in setting up the main and child documents:

%%%%%%%%%%%%%%%%%%%%%%%%%%%%%%%%%%%%%%%%
\paragraph{Restrictions.}

Please note the following restrictions:
\begin{itemize}
\item
|\childdocmain| must be called with one argument \textit{main}
to ensure compatibility with earlier version of the package.
It must either be empty (|\childdocmain{}|)
or precisely match the filename of the main file in which it is specified.
See \secref{sec:detection} for further information.
\item
The filename \textit{main} must be specified without the |.tex| extension.
\item
The filename \textit{main} is case sensitive
(even in case-insensitive file systems)
due to internal string comparison.
\item
The argument \textit{main} should be fully expanded, it cannot be a macro.
\item
Subdirectories and special characters should be avoided in filenames.
\item
The command |\childdocmain{|\textit{main}|}| must be followed by a whitespace.
It should not be followed immediately by another command
or by a comment mark `|%|'.
This is because the \TeX{} parser reads the token immediately following
the argument of |\childdocmain| and puts it
at the beginning of every child section;
however, a white\-space is ignored.
\end{itemize}

%%%%%%%%%%%%%%%%%%%%%%%%%%%%%%%%%%%%%%%%
\paragraph{Content of Main File.}

It is advisable to place all content in the child files included by |\include|.
Any output contained in the main file will appear in all child documents
unless suppressed manually;
it cannot be suppressed automatically by the |\includeonly| directive
and thus should normally be avoided.
A method to include some content in the main file
by means of conditional processing is described in \secref{sec:conditional}.

%%%%%%%%%%%%%%%%%%%%%%%%%%%%%%%%%%%%%%%%
\paragraph{Page Numbering.}

When only a part of the document is compiled,
the appropriate numbering of pages
(as well as other status parameters)
is determined from the |.aux| files.
The latter contain information from previous passes.
However this information needs to propagate through
all intermediate child documents.
Therefore the page numbering in child documents may well
be inconsistent until the complete document is compiled at least once.

A useful (if unconventional) way to always ensure a consistent
page numbering is to restart the numbering in each child document
and denote the pages by `\textit{child}|.|\textit{page}'
where \textit{child} represents the chapter/section number of the child file.
This can be achieved by the command
|\numberwithin{page}{|\textit{child}|}|
of the \textsf{amsmath} package
where \textit{child} can be |chapter| or |section|
depending on the chosen structuring.
Alternatively, one can modify the macro |\thepage| appropriately
and reset the counter |page| at the start of each child file.

%%%%%%%%%%%%%%%%%%%%%%%%%%%%%%%%%%%%%%%%%%%%%%%%%%%%%%%%%%%%%%%%%%%%%%%%%%%%%%%%
\subsection{Conditional Processing}
\label{sec:conditional}

The package provides a mechanism to compile different versions
of a document. To customise the versions further some conditional processing
can come in handy to distinguish which version is being compiled.
The package provides two macros to describe the compilation context:

%%%%%%%%%%%%%%%%%%%%%%%%%%%%%%%%%%%%%%%%
\DescribeMacro{\ifchilddoc}
The conditional |\ifchilddoc| distinguishes between the compilation of
child documents and the main document:
%
\begin{center}
|\ifchilddoc |\textit{child-code}| |[|\||else |\textit{main-code}]| \||fi|
\end{center}

%%%%%%%%%%%%%%%%%%%%%%%%%%%%%%%%%%%%%%%%
\DescribeMacro{\childdocname}
\DescribeMacro{\childdocjob}
The macro |\childdocname| contains the filename (without extension)
of the main or child file being processed.
Note that |\childdocjob| will always contain the name of the main file.

%%%%%%%%%%%%%%%%%%%%%%%%%%%%%%%%%%%%%%%%
\paragraph{Title Page.}

Conditional processing can be used to include a title or banner page
in the main document when proper precautions are taken.
Importantly, the code in the main file should ensure that the page counter
(as well as other status parameters which are stored in the |.aux| files)
takes the same value after the conditional processing.
Otherwise the page numbers may take divergent values
depending on which part is compiled.

For example, a title page could be declared by:
%
\begin{center}
\begin{tabular}{l}
|\ifchilddoc\||else|\\
|\addtocounter{page}{-1}|\\
\textit{code for title page}\\
|\newpage|\\
|\||fi|
\end{tabular}
\end{center}
%
A banner page for the child documents can be generated by:
%
\begin{center}
\begin{tabular}{l}
|\ifchilddoc|\\
|\addtocounter{page}{-1}|\\
\textit{code for banner page}\\
|\newpage|\\
|\||fi|
\end{tabular}
\end{center}
%
Here one could write a message such as:
\begin{center}
|This is the part \childdocname{} of \childdocjob{}.|
\end{center}

%%%%%%%%%%%%%%%%%%%%%%%%%%%%%%%%%%%%%%%%%%%%%%%%%%%%%%%%%%%%%%%%%%%%%%%%%%%%%%%%
\subsection{Flags}
\label{sec:flags}

The package makes it easy to generate different versions
of the main or child documents.
To this end compilation flags can be defined
and assigned different default values.
They will be particularly useful in conjunction
with the forwarding mechanism described in \secref{sec:forward}.

For example, it may be useful to have a flag |\version|
which can be set to |draft| or |final|.
The document source will contain some conditional code
depending on the value of |\version|.
Suppose further, the flag should default to |final| for the main file
and to |draft| for child files
which is a natural assignment for editing the document.
This is achieved by placing the following code
in the preamble of the main document
(below the |\childdocmain| directive):
%
\begin{center}
\begin{tabular}{l}
|\ifchilddoc|\\
|\providecommand{\version}{draft}|\\
|\||else|\\
|\providecommand{\version}{final}|\\
|\||fi|
\end{tabular}
\end{center}
%
The definition by |\providecommand| makes sure
that previous definitions are not overwritten.
Further statements |\providecommand{\version}{...}|
can thus be added before the above code to override it.

For the main file, one might add a line
(between |\childdocmain| and the above block)
%
\begin{center}
|%\ifchilddoc\||else\providecommand{\version}{draft}\||fi|
\end{center}
%
which can be uncommented to produce a draft version.
Likewise one can add a line to the very top of a child file
(above the |\childdocof{|\textit{main}|}| directive)
%
\begin{center}
|%\providecommand{\version}{final}|
\end{center}
%
which can be uncommented to produce the final version of this child document.

%%%%%%%%%%%%%%%%%%%%%%%%%%%%%%%%%%%%%%%%%%%%%%%%%%%%%%%%%%%%%%%%%%%%%%%%%%%%%%%%
\subsection{Forwarding}
\label{sec:forward}

Different versions of the main or child documents
using compilation flags as described in \secref{sec:flags}
can be (permanently) stored in different files
for convenient compilation, viewing and distribution.
To this end, the package defines a command
to pass on compilation to a different file:

%%%%%%%%%%%%%%%%%%%%%%%%%%%%%%%%%%%%%%%%
\DescribeMacro{\childdocforward}
The command |\childdocforward| redirects processing to
another source file:
%
\begin{center}
\begin{tabular}{l}
|\input{childdoc.def}|\\
|\childdocforward[|\textit{main}|]{|\textit{dest}|}|\\
\end{tabular}
\end{center}
%
The argument \textit{dest} is the destination file
(without extension).
It should be the main file or one of the child files.
Note that further \textsf{childdoc} directives
such as |\childdocof| and |\childdocforward|
in the indicated file will be processed in this form.
The optional argument \textit{main}
passes on directly to the main file \textit{main}
while pretending to compile the child \textit{dest}.
This form behaves as if \textit{dest}
issues |\childdocof{|\textit{main}|}| right away,
and no further \textsf{childdoc} directives will be processed.

%%%%%%%%%%%%%%%%%%%%%%%%%%%%%%%%%%%%%%%%
\DescribeMacro{\...prefix}
In the alternative form |\childdocforwardprefix|,
%
\begin{center}
\begin{tabular}{l}
|\input{childdoc.def}|\\
|\childdocforwardprefix[|\textit{main}|]{|\textit{prefix}|}{|\textit{dest}|}|
\end{tabular}
\end{center}
%
the destination file is determined by a pattern
depending on the current file:
To make this work, the current file must be called
`{\textit{prefix}\hspace{0.2em}\textit{suffix}}'
with \textit{prefix} matching precisely the argument.
Processing is then passed on to the file
`{\textit{dest}\hspace{0.2em}\textit{suffix}}'.
Surely, the same effect is achieved by
directly specifying the
argument `{\textit{dest}\hspace{0.2em}\textit{suffix}}'
in the first form.
However, that requires to set up a different file
for each child. With the alternative form of the command
all these files can have exactly the same content
which simplifies setting them up and maintaining them.

For example, the following file |draft.tex|
with a compilation flag |\version| as described in \secref{sec:flags}
compiles the main document as a draft:
%
\begin{center}
\begin{tabular}{l}
|\def\version{draft}|\\
|\input{childdoc.def}|\\
|\childdocforward{|\textit{main}|}|
\end{tabular}
\end{center}
%
Likewise, the following files |final|\textit{nn}|.tex|
compile the final version of the child document
|child|\textit{nn}|.tex|:
%
\begin{center}
\begin{tabular}{l}
|\def\version{final}|\\
|\input{childdoc.def}|\\
|\childdocforwardprefix{final}{child}|
\end{tabular}
\end{center}
%

Note that when several versions of a main file and/or of each child file
are to be generated, it may be convenient to set up a |Makefile| or
shell script to automatise the process.

%%%%%%%%%%%%%%%%%%%%%%%%%%%%%%%%%%%%%%%%%%%%%%%%%%%%%%%%%%%%%%%%%%%%%%%%%%%%%%%%
\subsection{Command Line Processing}
\label{sec:commandline}

The effect of redirection files can also be achieved by invoking
the \LaTeX{} compiler with a more elaborate command line.
Most conveniently this should be done as part
of a shell script or a |Makefile|.

When using \textsf{childdoc} in the main file, the following
command lines effectively perform a redirection
(note that depending on the shell being used,
backslashes may have to be doubled: `|\|' $\to$ `|\\|'):
%
\begin{center}
|... -jobname "|\textit{target}|" |\\|"|[\textit{flags}]%
|\input{childdoc.def}\childdocforward[|\textit{main}|]{|\textit{dest}|}"|
\end{center}
%
Here \textit{target} is the name of the output file,
\textit{main} is the name of the main file
and \textit{dest} is the name of the main or child file to be processed
(all filenames without extensions).
The optional argument \textit{main} can be omitted
if \textit{main} matches \textit{dest}.
Optionally, compilation \textit{flags} can be defined via |\def| commands.
This command line makes the \TeX{} engine believe
it is compiling the file \textit{target}
whose content is specified as the latter parameter.
The provided code then forwards the processing to
\textit{main} or \textit{dest} as described in \secref{sec:forward}.

%%%%%%%%%%%%%%%%%%%%%%%%%%%%%%%%%%%%%%%%%%%%%%%%%%%%%%%%%%%%%%%%%%%%%%%%%%%%%%%%
\subsection{Include by Input}
\label{sec:input}

Including child documents by |\include| has some restrictions by design.
Most notably, the content of a child document always occupies
its own set of pages; pages cannot be shared between child documents.
Usually, this behaviour makes perfect sense
because each child document contain an essential part of the document.
However, in some situations it may be desirable to compose
a document from a collection of parts
without having mandatory page breaks between then.
For this case, the package
provides a mechanism to include parts
by |\input| which can also be processed individually.
However, by construction this mechanism
requires manual handling of the content to be output.

%%%%%%%%%%%%%%%%%%%%%%%%%%%%%%%%%%%%%%%%
\DescribeMacro{\ifchilddocmanual}
The main file should be prepared as usual, see \secref{sec:include}.
However, the document body must make a distinction
between processing of an individual part and of the main document, e.g.:
%
\begin{center}
\begin{tabular}{l}
|\ifchilddocmanual|\\
|\input{\childdocname}|\\
|\||else|\\
\textit{document body with }|\input{|\textit{part}|}|\\
|\||fi|
\end{tabular}
\end{center}
%
The conditional |\ifchilddocmanual| is true whenever
a part to be included by |\input| is being compiled,
and the name of the part is stored in |\childdocname|.

%%%%%%%%%%%%%%%%%%%%%%%%%%%%%%%%%%%%%%%%
\DescribeMacro{\childdocby}
Each part to be included by |\input| should start with:
%
\begin{center}
\begin{tabular}{l}
|\input{childdoc.def}|\\
|\childdocby{|\textit{main}|}|\\
\end{tabular}
\end{center}
%
The directive |\childdocby| is similar to |\childdocof|
described in \secref{sec:include},
but the subsequent selection of content must be done manually.
To that end, both |\ifchilddoc| and |\ifchilddocmanual|
will be true upon processing of a part,
and the name of the part is stored in |\childdocname|.
Note that |\jobname| will be set to the filename of the current part
so that each part receives an individual |.aux| file
that does not interfere with the |.aux| file(s) of the main document.
This behaviour can be altered by the alternative form
|\childdocby[*]{|\textit{main}|}| (with a non-empty optional argument)
which uses the |.aux| file of the main document
by setting |\jobname| to \textit{main}.

%%%%%%%%%%%%%%%%%%%%%%%%%%%%%%%%%%%%%%%%%%%%%%%%%%%%%%%%%%%%%%%%%%%%%%%%%%%%%%%%
\subsection{Driver Development}
\label{sec:driver}

The \textsf{childdoc} mechanism can also be use for the development
of definition files such as \LaTeX{} styles or classes.
This case differs from the above setup with multiple parts
included by |\include| in that no |\includeonly| should be invoked.
This can be achieved by starting the include file
(before |\ProvidesPackage|) with:
%
\begin{center}
\begin{tabular}{l}
|\input{childdoc.def}|\\
|\childdocforward{|\textit{main}|}|\\
\end{tabular}
\end{center}
%
or alternatively with:
%
\begin{center}
\begin{tabular}{l}
|\input{childdoc.def}|\\
|\childdocby{|\textit{main}|}|\\
\end{tabular}
\end{center}
%
Both forms have slightly different effects as described above.
The main file is prepared as usual, see \secref{sec:include}.

%%%%%%%%%%%%%%%%%%%%%%%%%%%%%%%%%%%%%%%%%%%%%%%%%%%%%%%%%%%%%%%%%%%%%%%%%%%%%%%%
\subsection{Legacy Detection}
\label{sec:detection}

The directive |\childdocmain| in the main file can detect
whether the complete document or merely a child is to be compiled
even without using the directive |\childdocof|.
This method is deprecated because it is less robust
and there is no compelling reason to use it;
it is merely provided for backward compatibility
and it may be removed in future versions.

If the detection mechanism is to be used,
it is mandatory to correctly specify
the filename of the main file as the argument of |\childdocmain|:
%
\begin{center}
\begin{tabular}{l}
|\input{childdoc.def}|\\
|\childdocmain{|\textit{main}|}|\\
\end{tabular}
\end{center}
%
If |\jobname| does not match the argument \textit{main} of |\childdocmain|,
it is assumed that |\jobname| points to the child file to be compiled.
When using |\childdocmain| with the main file specified as argument,
it suffices to start a child file
with just |\input{|\textit{main}|}|
without loading of the package and using |\childdocof|.
If instead all processing is done
with the appropriate \textsf{childdoc} directives,
the argument of \textit{main} of |\childdocmain| can be empty.

An alternative version of the command line processing described
in \secref{sec:commandline} using the detection mechanism reads:
%
\begin{center}
|... -jobname "|\textit{target}|" "|[\textit{flags}]%
[|\def\jobname{|\textit{dest}|}|]|\input{|\textit{main}|}"|
\end{center}

%%%%%%%%%%%%%%%%%%%%%%%%%%%%%%%%%%%%%%%%%%%%%%%%%%%%%%%%%%%%%%%%%%%%%%%%%%%%%%%%
\subsection{Manual Code}
\label{sec:manual}

In case one cannot be certain whether the definitions file |childdoc.def|
is installed on the target \TeX{} distribution
and one prefers not to ship it,
it is conceivable to paste a few relevant commands into the sources.

To that end, drop all statements |\input{childdoc.def}|
and perform the replacements as outlined below.
Instead of |\childdocmain{|\textit{main}|}| add the following code
to the top of the main file:
%
\begin{center}
\begin{tabular}{l}
|\||ifdefined\childdocname\endinput\||fi\newif\ifchilddoc|\\
|\edef\childdocname{\scantokens\expandafter{\jobname\noexpand}}|\\
|\def\childdocmain{|\textit{main}|}\||ifx\childdocmain\childdocname\||else|\\
|\childdoctrue\includeonly{\childdocname}\let\jobname\childdocmain\||fi|\\
\end{tabular}
\end{center}
%
Instead of |\childdocof{|\textit{main}|}| just include the main file
at the top of each child file:
%
\begin{center}
|\input{|\textit{main}|}|
\end{center}
%
A simple redirection |\childdocforward{|\textit{dest}|}| is achieved by:
%
\begin{center}
|\def\jobname{|\textit{dest}|}\input{\jobname}|
\end{center}
%
The redirection with prefix
|\childdocforwardprefix[|\textit{prefix}|]{|\textit{dest}|}|
is accomplished by:
%
\begin{center}
\begin{tabular}{l}
|{\edef\jobname{\scantokens\expandafter{\jobname\noexpand}}|\\
|\def\redirectjob |\textit{prefix}|#1~~~{\gdef\jobname{|\textit{dest}|#1}}|\\
|\expandafter\redirectjob\jobname~~~}\input{\jobname}|
\end{tabular}
\end{center}

In an alternative approach,
child documents can be compiled by a specific command line
without additional code or specific definitions:
%
\begin{center}
|... -jobname "|\textit{target}|" "|[\textit{flags}]%
|\includeonly{|\textit{dest}|}\input{|\textit{main}|}"|
\end{center}
%

%%%%%%%%%%%%%%%%%%%%%%%%%%%%%%%%%%%%%%%%%%%%%%%%%%%%%%%%%%%%%%%%%%%%%%%%%%%%%%%%
%%%%%%%%%%%%%%%%%%%%%%%%%%%%%%%%%%%%%%%%%%%%%%%%%%%%%%%%%%%%%%%%%%%%%%%%%%%%%%%%
\section{Information}

%%%%%%%%%%%%%%%%%%%%%%%%%%%%%%%%%%%%%%%%%%%%%%%%%%%%%%%%%%%%%%%%%%%%%%%%%%%%%%%%
\subsection{Copyright}

Copyright \copyright{} 2017--2018 Niklas Beisert

This work may be distributed and/or modified under the
conditions of the \LaTeX{} Project Public License, either version 1.3
of this license or (at your option) any later version.
The latest version of this license is in
  \url{http://www.latex-project.org/lppl.txt}
and version 1.3 or later is part of all distributions of \LaTeX{}
version 2005/12/01 or later.

This work has the LPPL maintenance status `maintained'.

The Current Maintainer of this work is Niklas Beisert.

This work consists of the files |README.txt|, |childdoc.ins| and |childdoc.dtx|
as well as the derived files |childdoc.def|, |cdocsamp.tex|
with |cdocsch1.tex|, |cdocsch2.tex|, |cdocspt3.tex|, |cdocspt4.tex|,
|cdocsdrf.tex|, |cdocsfn1.tex|, |cdocsfn2.tex|
as well as |childdoc.pdf|.

%%%%%%%%%%%%%%%%%%%%%%%%%%%%%%%%%%%%%%%%%%%%%%%%%%%%%%%%%%%%%%%%%%%%%%%%%%%%%%%%
\subsection{Files and Installation}

The package consists of the files:
%
\begin{center}
\begin{tabular}{ll}
    |README.txt|   & readme file \\
    |childdoc.ins| & installation file \\
    |childdoc.dtx| & source file \\
    |childdoc.def| & definition file \\
    |cdocsamp.tex| & sample main file \\
    |cdocsch1.tex| & sample include file \\
    |cdocsch2.tex| & sample include file \\
    |cdocspt3.tex| & sample part file \\
    |cdocspt4.tex| & sample part file \\
    |cdocsdrf.tex| & sample redirection file \\
    |cdocsfn1.tex| & sample redirection file \\
    |cdocsfn2.tex| & sample redirection file \\
    |childdoc.pdf| & manual
\end{tabular}
\end{center}
%
The distribution consists of the files
|README.txt|, |childdoc.ins| and |childdoc.dtx|.
%
\begin{itemize}
\item
Run (pdf)\LaTeX{} on |childdoc.dtx|
to compile the manual |childdoc.pdf| (this file).
\item
Run \LaTeX{} on |childdoc.ins| to create the definitions file |childdoc.def|
and the sample |cdocsamp.tex| with include files
|cdocsch1.tex|, |cdocsch2.tex|, |cdocspt3.tex|, |cdocspt4.tex|,
|cdocsdrf.tex|, |cdocsfn1.tex|, |cdocsfn2.tex|.
Then copy the file |childdoc.def| to an appropriate directory of your \LaTeX{}
distribution, e.g.\ \textit{texmf-root}|/tex/latex/childdoc|.
\end{itemize}

%%%%%%%%%%%%%%%%%%%%%%%%%%%%%%%%%%%%%%%%%%%%%%%%%%%%%%%%%%%%%%%%%%%%%%%%%%%%%%%%
\subsection{Related CTAN Packages}

There are several other packages which offer a similar functionality:
%
\begin{itemize}
\item
The packages
\href{http://ctan.org/pkg/docmute}{\textsf{docmute}},
\href{http://ctan.org/pkg/includex}{\textsf{includex}} and
\href{http://ctan.org/pkg/standalone}{\textsf{standalone}}
provide commands to include only the document body of
a child file thus allowing both files to be compiled individually.
\item
The packages \href{http://ctan.org/pkg/subdocs}{\textsf{subdocs}}
and \href{http://ctan.org/pkg/subfiles}{\textsf{subfiles}}
provide structures in which the main and child documents can be
encapsulated and allowing them to be compiled individually.
The inclusion mechanism is different from the conventional |\include|.
\item
The package \href{http://ctan.org/pkg/combine}{\textsf{combine}}
is an elaborate solution to combine several documents into one.
\end{itemize}
%
See also the CTAN topic \href{http://ctan.org/topic/subdocs}{\textsf{subdocs}}
for further related packages.
The present package differs from the above solutions in that
a document structure constructed with the conventional |\include| mechanism
just needs two extra commands at the top of every file
such that all constituent files can be compiled individually.

%%%%%%%%%%%%%%%%%%%%%%%%%%%%%%%%%%%%%%%%%%%%%%%%%%%%%%%%%%%%%%%%%%%%%%%%%%%%%%%%
%\subsection{Feature Suggestions}
%
%The following is a list of features which may be useful for future
%versions of this package:
%%
%\begin{itemize}
%\item
%\ldots
%\end{itemize}

%%%%%%%%%%%%%%%%%%%%%%%%%%%%%%%%%%%%%%%%%%%%%%%%%%%%%%%%%%%%%%%%%%%%%%%%%%%%%%%%
\subsection{Revision History}

%%%%%%%%%%%%%%%%%%%%%%%%%%%%%%%%%%%%%%%%
\paragraph{v2.0:} 2018/12/30

\begin{itemize}
\item
immediate forward processing
\item
added |\childdocby| mechanism
\item
manual restructured
\end{itemize}

%%%%%%%%%%%%%%%%%%%%%%%%%%%%%%%%%%%%%%%%
\paragraph{v1.6:} 2018/01/17

\begin{itemize}
\item
application for development of include files
\item
corrections to manual
\end{itemize}

%%%%%%%%%%%%%%%%%%%%%%%%%%%%%%%%%%%%%%%%
\paragraph{v1.5:} 2017/05/21

\begin{itemize}
\item
more complete structuring introduced
\item
|\childdocof| introduced
\item
|\childdoc| renamed to |\childdocmain|
\item
|\childredirect| renamed to |\childdocforward| and |\childdocforwardprefix|
and functionality expanded
\end{itemize}

%%%%%%%%%%%%%%%%%%%%%%%%%%%%%%%%%%%%%%%%
\paragraph{v1.0:} 2017/04/27

\begin{itemize}
\item
manual and install package
\item
first version published on CTAN
\end{itemize}

%%%%%%%%%%%%%%%%%%%%%%%%%%%%%%%%%%%%%%%%
\paragraph{v0.6:} 2017/04/26

\begin{itemize}
\item
redirection mechanism added
\end{itemize}

%%%%%%%%%%%%%%%%%%%%%%%%%%%%%%%%%%%%%%%%
\paragraph{v0.5:} 2017/04/26

\begin{itemize}
\item
functionality in definition file
\end{itemize}


%%%%%%%%%%%%%%%%%%%%%%%%%%%%%%%%%%%%%%%%%%%%%%%%%%%%%%%%%%%%%%%%%%%%%%%%%%%%%%%%
%%%%%%%%%%%%%%%%%%%%%%%%%%%%%%%%%%%%%%%%%%%%%%%%%%%%%%%%%%%%%%%%%%%%%%%%%%%%%%%%
%%%%%%%%%%%%%%%%%%%%%%%%%%%%%%%%%%%%%%%%%%%%%%%%%%%%%%%%%%%%%%%%%%%%%%%%%%%%%%%%
\appendix

\settowidth\MacroIndent{\rmfamily\scriptsize 000\ }

 \DocInput{childdoc.dtx}

\end{document}
%</driver>
% \fi
%
% %%%%%%%%%%%%%%%%%%%%%%%%%%%%%%%%%%%%%%%%%%%%%%%%%%%%%%%%%%%%%%%%%%%%%%%%%%%%%%
% %%%%%%%%%%%%%%%%%%%%%%%%%%%%%%%%%%%%%%%%%%%%%%%%%%%%%%%%%%%%%%%%%%%%%%%%%%%%%%
% \section{Sample}
%\iffalse
%<*samplemain>
%\fi
%
% The following presents a sample document
% with two chapters, two parts, a title page,
% a compile flag as well as three forwarding files to set the flag.
% It consists of eight |.tex| files:
% \begin{center}
% \begin{tabular}{ll}
% |cdocsamp.tex|&main file\\
% |cdocsch1.tex|&include file for chapter 1\\
% |cdocsch2.tex|&include file for chapter 2\\
% |cdocspt3.tex|&include file for part 3\\
% |cdocspt4.tex|&include file for part 4\\
% |cdocsdrf.tex|&forwarding file for main file in draft mode\\
% |cdocsfi1.tex|&forwarding file for final version of chapter 1\\
% |cdocsfi2.tex|&forwarding file for final version of chapter 2\\
% \end{tabular}
% \end{center}
% Each of the eight files can be compiled directly by the \LaTeX{} compiler.
%
% %%%%%%%%%%%%%%%%%%%%%%%%%%%%%%%%%%%%%%
% \paragraph{Main File.}
%
% The main file is called |cdocsamp.tex|.
%
% Load the \textsf{childdoc} definitions and
% declare the filename for the main document:
%    \begin{macrocode}
\input{childdoc.def}
\childdocmain{}
%    \end{macrocode}

% Optional override for |\version| flag:
%    \begin{macrocode}
%%\ifchilddoc\else\providecommand{\version}{draft}\fi
%    \end{macrocode}

% Define the default values for the |\version| flag
% (|final| for the main file and |draft| for childs):
%    \begin{macrocode}
\ifchilddoc
\providecommand{\version}{draft}
\else
\providecommand{\version}{final}
\fi
%    \end{macrocode}

% Load the standard document class:
%    \begin{macrocode}
\documentclass[12pt]{article}
%    \end{macrocode}

% Start the document body:
%    \begin{macrocode}
\begin{document}
%    \end{macrocode}

% Declare a title page.
% Print title, part of document being processed and version flag:
%    \begin{macrocode}
\addtocounter{page}{-1}
\begin{center}
{\LARGE\bfseries{}childdoc example\par}
\vspace{1cm}
\ifchilddoc
\ifchilddocmanual part\else chapter\fi:
`\childdocname' of `\childdocjob'\par
\else
main document: `\childdocjob'\par
\fi
version: \version\par
\end{center}
\newpage
%    \end{macrocode}

% Manually include selected file,
% otherwise process as usual:
%    \begin{macrocode}
\ifchilddocmanual
\section*{part `\childdocname'}
\input{\childdocname}
\else
%    \end{macrocode}

% Include the two chapters:
%    \begin{macrocode}
\include{cdocsch1}
\include{cdocsch2}
%    \end{macrocode}

% Include the two parts unless only chapters should be displayed:
%    \begin{macrocode}
\ifchilddoc\else
\section{part three}
\input{cdocspt3}
\section{part four}
\input{cdocspt4}
\fi
%    \end{macrocode}

% Process as usual until here:
%    \begin{macrocode}
\fi
%    \end{macrocode}

% End of document body:
%    \begin{macrocode}
\end{document}
%    \end{macrocode}
%\iffalse
%</samplemain>
%\fi
%
% %%%%%%%%%%%%%%%%%%%%%%%%%%%%%%%%%%%%%%
% \paragraph{Chapter Include Files.}
%
% The include files are called |cdocsch1.tex| and |cdocsch2.tex|.
%
%\iffalse
%<*samplechap1|samplechap2>
%\fi

% Optional override for |\version| flag:
%    \begin{macrocode}
%%\providecommand{\version}{final}
%    \end{macrocode}

% Include the main document:
%    \begin{macrocode}
\input{childdoc.def}
\childdocof{cdocsamp}
%    \end{macrocode}

%\iffalse
%</samplechap1|samplechap2>
%\fi
%
%\iffalse
%<*samplechap1>
%\fi
% Some text for chapter 1:
%    \begin{macrocode}
\section{one}
some text in chapter one
%    \end{macrocode}

%\iffalse
%</samplechap1>
%\fi
% Some text for chapter 2:
%\iffalse
%<*samplechap2>
%\fi
%    \begin{macrocode}
\section{two}
more text in chapter two
%    \end{macrocode}

%\iffalse
%</samplechap2>
%\fi
%
% %%%%%%%%%%%%%%%%%%%%%%%%%%%%%%%%%%%%%%
% \paragraph{Part Include Files.}
%
% The include files are called |cdocspt3.tex| and |cdocspt4.tex|.
%
%\iffalse
%<*samplepart3|samplepart4>
%\fi

% Optional override for |\version| flag:
%    \begin{macrocode}
%%\providecommand{\version}{final}
%    \end{macrocode}

% Include the main document:
%    \begin{macrocode}
\input{childdoc.def}
\childdocby{cdocsamp}
%    \end{macrocode}

%\iffalse
%</samplepart3|samplepart4>
%\fi
%
%\iffalse
%<*samplepart3>
%\fi
% Some text for part 3:
%    \begin{macrocode}
some text in part three
%    \end{macrocode}

%\iffalse
%</samplepart3>
%\fi
% Some text for part 4:
%\iffalse
%<*samplepart4>
%\fi
%    \begin{macrocode}
more text in part four
%    \end{macrocode}

%\iffalse
%</samplepart4>
%\fi
%
% %%%%%%%%%%%%%%%%%%%%%%%%%%%%%%%%%%%%%%
% \paragraph{Forwarding for a Complete Draft.}
%
% The following forwarding file |cdocsdrf.tex|
% compiles the main document in draft mode:
%\iffalse
%<*sampledraft>
%\fi
%    \begin{macrocode}
\def\version{draft}
\input{childdoc.def}
\childdocforward{cdocsamp}
%    \end{macrocode}

%\iffalse
%</sampledraft>
%\fi
%
% %%%%%%%%%%%%%%%%%%%%%%%%%%%%%%%%%%%%%%
% \paragraph{Forwarding for Final Version of the Chapters.}
%
% The following forwarding files |cdocsfn1.tex| and |cdocsfn2.tex|
% (with identical content)
% compile the final versions of the child documents
% |cdocsch1.tex| and |cdocsch2.tex|, respectively:
%\iffalse
%<*samplefinal>
%\fi
%    \begin{macrocode}
\def\version{final}
\input{childdoc.def}
\childdocforwardprefix[cdocsamp]{cdocsfn}{cdocsch}
%    \end{macrocode}

%\iffalse
%</samplefinal>
%\fi
%
% %%%%%%%%%%%%%%%%%%%%%%%%%%%%%%%%%%%%%%
% \paragraph{Command Line Processing.}
%
% The following three command lines generate the output files
% |cdocscld|, |cdocscl1| and |cdocscl2|
% which should be identical to
% |cdocsdrf|, |cdocsch1| and |cdocsfn2|, respectively:
% \begin{center}
% \begin{tabular}{l}
% |latex -jobname cdocscld \|\\
% |  "\def\version{draft}\input{childdoc.def}\childdocforward{cdocsamp}"|\\
% |latex -jobname cdocscl1 \|\\
% |  "\input{childdoc.def}\childdocforward[cdocsamp]{cdocsch1}"|\\
% |latex -jobname cdocscl2 \|\\
% |  "\def\version{final}\input{childdoc.def}\childdocforward{cdocsch2}"|
% \end{tabular}
% \end{center}
% Note that the trailing backslash on each first line
% merely continues the input to the second line
% (for convenient cut ant paste).
% Furthermore, the command |latex| can be replaced by any
% of its alternative versions such as |pdflatex|.
%
% %%%%%%%%%%%%%%%%%%%%%%%%%%%%%%%%%%%%%%%%%%%%%%%%%%%%%%%%%%%%%%%%%%%%%%%%%%%%%%
% %%%%%%%%%%%%%%%%%%%%%%%%%%%%%%%%%%%%%%%%%%%%%%%%%%%%%%%%%%%%%%%%%%%%%%%%%%%%%%
% \section{Implementation}
%\iffalse
%<*package>
%\fi
%
% This section describes the definitions file |childdoc.def|.

% The definitions cannot be loaded using |\usepackage| or |\RequirePackage|
% which has a mechanism to prevent loading a style file more than once.
% When loading the definitions by means of |\input|
% multiple instances have to be prevented manually:
%\iffalse
%This code needs to be before the `\ProvidesFile' directive
%which is defined at the beginning of this file.
%Therefore it is also placed there and commented out here.
%</package>
%<*discard>
%\fi
%    \begin{macrocode}
\ifdefined\childdocmain\endinput\fi
%    \end{macrocode}
%\iffalse
%</discard>
%<*package>
%\fi
%
% \macro{\ifchilddoc}
% \macro{\ifchilddocmanual}
% The conditional |\ifchilddoc| tells whether a
% child (true) or main (false) document is being compiled.
% The conditional |\ifchilddocmanual| tells whether
% the |\includeonly| mechanism is used (false) or
% the selection of child files must be performed manually (true).
% The definitions initialise to false:
%    \begin{macrocode}
\newif\ifchilddoc
\newif\ifchilddocmanual
%    \end{macrocode}

% \macro{\childdocname}
% \macro{\childdocjob}
% The macro |\childdocname| stores the name of the main document
% to be compiled. The macro |\childdocjob| stores the name of
% the document on which the \LaTeX{} compiler was originally invoked.
% The content of |\jobname| cannot be compared
% to filenames specified in the source due to different catcodes.
% The following code rescans |\jobname|, stores the result
% in |\childdocname| and saves a copy in |\childdocjob|:
%    \begin{macrocode}
\edef\childdocname{\scantokens\expandafter{\jobname\noexpand}}
\let\childdocjob\childdocname
%    \end{macrocode}

% \macro{\childdocdisable}
% The macro |\childdocdisable| prevents the main file
% from being processed more than once.
% At this stage, the main document command |\childdocmain|
% is assumed to be called once again where it should do nothing.
% Any subsequent call to it should prevent
% a secondary processing of the main document
% It overwrites the forwarding commands
% |\childdocof| and |\childdocforward|
% with empty macros to prevent further inclusions of the main document:
%    \begin{macrocode}
\newcommand{\childdocdisable}
{
  \renewcommand{\childdocmain}[1]{\renewcommand{\childdocmain}[1]{\endinput}}
  \renewcommand{\childdocof}[1]{}
  \renewcommand{\childdocby}[2][]{}
  \renewcommand{\childdocforward}[2][]{}
  \renewcommand{\childdocdisable}{}
}
%    \end{macrocode}

% \macro{\childdocmain}
% The macro |\childdocmain| is to be called at the top of the main file
% with nothing or the main filename (without extension) as argument.
% First, it breaks loops.
% If the argument is not empty and does not match |\childdocname|
% (which is set by the first inclusion of |childdoc.def|),
% |\ifchilddoc| is set to true, |\includeonly| is applied to the child file
% and |\jobname| is set to the main file
% (for proper handling of |.aux| files):
%    \begin{macrocode}
\newcommand{\childdocmain}[1]
{
  \childdocdisable\childdocmain{}
  \if?#1?\else
    \begingroup
      \def\childdoctmp{#1}
      \ifx\childdoctmp\childdocname
        \def\childdoctmp{}
      \else
        \def\childdoctmp
        {
          \childdoctrue
          \includeonly{\childdocname}
          \def\childdocjob{#1}
          \def\jobname{#1}
        }
      \fi
      \expandafter
    \endgroup
    \childdoctmp
  \fi
}
%    \end{macrocode}

% \macro{\childdocof}
% The command |\childdocof| redirects
% compilation to the main file |#1|.
%    \begin{macrocode}
\newcommand{\childdocof}[1]
{
  \childdocdisable
  \childdoctrue
  \includeonly{\childdocname}
  \def\jobname{#1}
  \def\childdocjob{#1}
  \input{#1}
}
%    \end{macrocode}

% \macro{\childdocby}
% The command |\childdocby| ....
%    \begin{macrocode}
\newcommand{\childdocby}[2][]
{
  \childdocdisable
  \childdoctrue
  \childdocmanualtrue
  \if?#1?\else
    \def\jobname{#2}
  \fi
  \def\childdocjob{#2}
  \input{#2}
  \endinput
}
%    \end{macrocode}

% \macro{\childdocforward}
% The command |\childdocforward| redirects
% compilation to the main file or
% (if the optional argument is given) a child file.
% Parameters are set as if the main file
% or a child file starting with |\childdocof| was compiled.
% Then compilation is handed over to the main file:
%    \begin{macrocode}
\newcommand{\childdocforward}[2][]
{
  \begingroup
    \if?#1?
      \def\childdoctmp
      {
        \def\childdocname{#2}
        \def\childdocjob{#2}
        \def\jobname{#2}
        \input{#2}
        \endinput
      }
    \else
      \def\childdoctmp
      {
        \childdocdisable
        \def\childdocname{#2}
        \childdoctrue
        \includeonly{#2}
        \def\childdocjob{#1}
        \def\jobname{#1}
        \input{#1}
        \endinput
      }
    \fi
    \expandafter
  \endgroup
  \childdoctmp
}
%    \end{macrocode}

% \macro{\childdocforwardprefix}
% The command |\childdocforwardprefix| redirects
% compilation to the main or a child file by means of a pattern.
% The prefix |#1| in the current filename is replaced by |#2|
% and the suffix of the current filename is kept
% (it is assumed that the filename does not contain the substring `|~~~|'
% which is used as a delimiter).
% Compilation is handed over to the new file by |\childdocforward|:
%    \begin{macrocode}
\newcommand{\childdocforwardprefix}[3][]
{
  \begingroup
    \def\childdocextract #2##1~~~{\def\childdoctmp{\childdocforward[#1]{#3##1}}}
    \expandafter\childdocextract\childdocname~~~
    \expandafter
  \endgroup
  \childdoctmp
}
%    \end{macrocode}

% \macro{\childdoc}
% The deprecated macro |\childdoc| is a legacy version of |\childdocmain|:
%    \begin{macrocode}
\newcommand{\childdoc}{\childdocmain}
%    \end{macrocode}

% \macro{\childdocredirect}
% The deprecated macro |\childdocredirect| is a legacy version
% of |\childdocforward| and |\childdocforwardprefix|:
%    \begin{macrocode}
\newcommand{\childdocredirect}[2][]
{
  \begingroup
    \if?#1?
      \def\childdoctmp{\childdocforward{#2}}
    \else
      \def\childdoctmp{\childdocforwardprefix{#1}{#2}}
    \fi
    \expandafter
  \endgroup
  \childdoctmp
}
%    \end{macrocode}

%\iffalse
%</package>
%\fi
%
\endinput
|\\
|\childdocforwardprefix{final}{child}|
\end{tabular}
\end{center}
%

Note that when several versions of a main file and/or of each child file
are to be generated, it may be convenient to set up a |Makefile| or
shell script to automatise the process.

%%%%%%%%%%%%%%%%%%%%%%%%%%%%%%%%%%%%%%%%%%%%%%%%%%%%%%%%%%%%%%%%%%%%%%%%%%%%%%%%
\subsection{Command Line Processing}
\label{sec:commandline}

The effect of redirection files can also be achieved by invoking
the \LaTeX{} compiler with a more elaborate command line.
Most conveniently this should be done as part
of a shell script or a |Makefile|.

When using \textsf{childdoc} in the main file, the following
command lines effectively perform a redirection
(note that depending on the shell being used,
backslashes may have to be doubled: `|\|' $\to$ `|\\|'):
%
\begin{center}
|... -jobname "|\textit{target}|" |\\|"|[\textit{flags}]%
|% \iffalse
%
% childdoc.dtx Copyright (C) 2017-2018 Niklas Beisert
%
% This work may be distributed and/or modified under the
% conditions of the LaTeX Project Public License, either version 1.3
% of this license or (at your option) any later version.
% The latest version of this license is in
%   http://www.latex-project.org/lppl.txt
% and version 1.3 or later is part of all distributions of LaTeX
% version 2005/12/01 or later.
%
% This work has the LPPL maintenance status `maintained'.
%
% The Current Maintainer of this work is Niklas Beisert.
%
% This work consists of the files childdoc.dtx and childdoc.ins
% and the derived files childdoc.def and cdocsamp.tex with
% cdocsch1.tex, cdocsch2.tex, cdocsdrf.tex, cdocsfn1.tex, cdocsfn2.tex.
%
%<package>\ifdefined\childdocmain\endinput\fi
%<package>\ProvidesFile{childdoc.def}[2018/12/30 v2.0 child document driver]
%<samplemain>\ProvidesFile{cdocsamp.tex}[2018/12/30 v2.0 sample for childdoc]
%<*driver>
%\ProvidesFile{childdoc.drv}[2018/12/30 v2.0 childdoc reference manual file]
\PassOptionsToClass{10pt,a4paper}{article}
\documentclass{ltxdoc}

\usepackage[margin=35mm]{geometry}
\usepackage{hyperref}
\usepackage{hyperxmp}
\usepackage[usenames]{color}

\hypersetup{colorlinks=true}
\hypersetup{pdfstartview=FitH}
\hypersetup{pdfpagemode=UseNone}
\hypersetup{pdfsource={}}
\hypersetup{pdflang={en-UK}}
\hypersetup{pdfcopyright={Copyright 2017-2018 Niklas Beisert.
  This work may be distributed and/or modified under the
  conditions of the LaTeX Project Public License, either version 1.3
  of this license or (at your option) any later version.}}
\hypersetup{pdflicenseurl={http://www.latex-project.org/lppl.txt}}
\hypersetup{pdfcontactaddress={ETH Zurich, ITP, HIT K,
  Wolfgang-Pauli-Strasse 27}}
\hypersetup{pdfcontactpostcode={8093}}
\hypersetup{pdfcontactcity={Zurich}}
\hypersetup{pdfcontactcountry={Switzerland}}
\hypersetup{pdfcontactemail={nbeisert@itp.phys.ethz.ch}}
\hypersetup{pdfcontacturl={http://people.phys.ethz.ch/\xmptilde nbeisert/}}

\newcommand{\secref}[1]{\hyperref[#1]{section \ref*{#1}}}

\parskip1ex
\parindent0pt
\let\olditemize\itemize
\def\itemize{\olditemize\parskip0pt}

\begin{document}

\title{The \textsf{childdoc} Package}
\hypersetup{pdftitle={The childdoc Package}}
\author{Niklas Beisert\\[2ex]
  Institut f\"ur Theoretische Physik\\
  Eidgen\"ossische Technische Hochschule Z\"urich\\
  Wolfgang-Pauli-Strasse 27, 8093 Z\"urich, Switzerland\\[1ex]
  \href{mailto:nbeisert@itp.phys.ethz.ch}
  {\texttt{nbeisert@itp.phys.ethz.ch}}}
\hypersetup{pdfauthor={Niklas Beisert}}
\hypersetup{pdfsubject={Manual for the LaTeX2e Package childdoc}}
\date{30 December 2018, \textsf{v2.0}}
\maketitle

\begin{abstract}\noindent
\textsf{childdoc} is a \LaTeXe{} package
that enables the direct compilation
of document sections included by |\include|
to individual files.
\end{abstract}

\begingroup
\parskip0ex
\tableofcontents
\endgroup

%%%%%%%%%%%%%%%%%%%%%%%%%%%%%%%%%%%%%%%%%%%%%%%%%%%%%%%%%%%%%%%%%%%%%%%%%%%%%%%%
%%%%%%%%%%%%%%%%%%%%%%%%%%%%%%%%%%%%%%%%%%%%%%%%%%%%%%%%%%%%%%%%%%%%%%%%%%%%%%%%
\section{Introduction}

\LaTeX{} provides a mechanism to structure a large document (such as a book)
into a main file and several child files (containing the chapters)
using the |\include| command.
This mechanism is beneficial for documents
which span hundreds of pages in order to
make the source file(s) more manageable.
Moreover, compilation can be restricted to
selected child files by means of the |\includeonly| command.
The latter feature can be used to reduce the compilation time while editing
(this was significantly more useful in the earlier days of \LaTeX{})
or to generate a smaller document which is easier to navigate.
Another application of |\includeonly| is to generate
documents consisting of selected parts of the complete document.

However, there are a few drawbacks of the plain |\include| mechanism:
\begin{itemize}
\item
The child files cannot be compiled on their own,
they can only be compiled via the main file.
A naive editing environment
(such as a text editor with an option
to have the current file processed by \LaTeX)
may require one to switch to the main file before compiling;
attempting to compile the child file produces errors.
\item
The main file must be modified (each time)
to adjust the |\includeonly| command
to the present needs. This easily leaves the main file in a messy state.
\item
The generated document will always carry the filename
of the main document. This is inconvenient if
several child files are to be compiled and
to be kept for distribution.
\end{itemize}

The present package provides a simple interface
to make child files individually compilable by \LaTeX{}.
Compiling a child file then has the same effect as compiling
the main file with an |\includeonly| command
to select the appropriate child.
Moreover the generated document will carry the name of the child
rather than the main file.
This resolves all three above issues.

This feature is meant to make the editing of books,
thesis documents and lecture notes somewhat more convenient.
However, the package can also be used efficiently for
composing a series of documents (such as exercise sheets)
which are typically distributed individually.
It then assists the author in generating the individual documents
(potentially in different versions)
as well as a document containing the collected series.
Another application is in developing style files
or other kinds of included material
where compilation of the style file could redirect
to a sample or test file.

%%%%%%%%%%%%%%%%%%%%%%%%%%%%%%%%%%%%%%%%%%%%%%%%%%%%%%%%%%%%%%%%%%%%%%%%%%%%%%%%
%%%%%%%%%%%%%%%%%%%%%%%%%%%%%%%%%%%%%%%%%%%%%%%%%%%%%%%%%%%%%%%%%%%%%%%%%%%%%%%%
\section{Usage}

First of all, the package \textsf{childdoc} is \emph{not} a standard
\LaTeXe{} |.sty| style file! Therefore it needs to be invoked in
a non-standard way.

%%%%%%%%%%%%%%%%%%%%%%%%%%%%%%%%%%%%%%%%%%%%%%%%%%%%%%%%%%%%%%%%%%%%%%%%%%%%%%%%
\subsection{Included Files}
\label{sec:include}

%%%%%%%%%%%%%%%%%%%%%%%%%%%%%%%%%%%%%%%%
\DescribeMacro{\childdocmain}
To use the package, add the commands
\begin{center}
\begin{tabular}{l}
|\input{childdoc.def}|\\
|\childdocmain{}|\\
\end{tabular}
\end{center}
at the very top of the main \LaTeX{} file,
in particular \emph{before} the |\documentclass| statement!
The argument of |\childdocmain| should be left empty
(but it must be present).

%%%%%%%%%%%%%%%%%%%%%%%%%%%%%%%%%%%%%%%%
\DescribeMacro{\childdocof}
Furthermore, add the commands
\begin{center}
\begin{tabular}{l}
|\input{childdoc.def}|\\
|\childdocof{|\textit{main}|}|\\
\end{tabular}
\end{center}
at the top of every child file \textit{child}
which is included by |\include{|\textit{child}|}|
from within the main file
(or at least for those files to be compiled individually).
The argument \textit{main} must be the filename of the main file.

There are a couple of
considerations in setting up the main and child documents:

%%%%%%%%%%%%%%%%%%%%%%%%%%%%%%%%%%%%%%%%
\paragraph{Restrictions.}

Please note the following restrictions:
\begin{itemize}
\item
|\childdocmain| must be called with one argument \textit{main}
to ensure compatibility with earlier version of the package.
It must either be empty (|\childdocmain{}|)
or precisely match the filename of the main file in which it is specified.
See \secref{sec:detection} for further information.
\item
The filename \textit{main} must be specified without the |.tex| extension.
\item
The filename \textit{main} is case sensitive
(even in case-insensitive file systems)
due to internal string comparison.
\item
The argument \textit{main} should be fully expanded, it cannot be a macro.
\item
Subdirectories and special characters should be avoided in filenames.
\item
The command |\childdocmain{|\textit{main}|}| must be followed by a whitespace.
It should not be followed immediately by another command
or by a comment mark `|%|'.
This is because the \TeX{} parser reads the token immediately following
the argument of |\childdocmain| and puts it
at the beginning of every child section;
however, a white\-space is ignored.
\end{itemize}

%%%%%%%%%%%%%%%%%%%%%%%%%%%%%%%%%%%%%%%%
\paragraph{Content of Main File.}

It is advisable to place all content in the child files included by |\include|.
Any output contained in the main file will appear in all child documents
unless suppressed manually;
it cannot be suppressed automatically by the |\includeonly| directive
and thus should normally be avoided.
A method to include some content in the main file
by means of conditional processing is described in \secref{sec:conditional}.

%%%%%%%%%%%%%%%%%%%%%%%%%%%%%%%%%%%%%%%%
\paragraph{Page Numbering.}

When only a part of the document is compiled,
the appropriate numbering of pages
(as well as other status parameters)
is determined from the |.aux| files.
The latter contain information from previous passes.
However this information needs to propagate through
all intermediate child documents.
Therefore the page numbering in child documents may well
be inconsistent until the complete document is compiled at least once.

A useful (if unconventional) way to always ensure a consistent
page numbering is to restart the numbering in each child document
and denote the pages by `\textit{child}|.|\textit{page}'
where \textit{child} represents the chapter/section number of the child file.
This can be achieved by the command
|\numberwithin{page}{|\textit{child}|}|
of the \textsf{amsmath} package
where \textit{child} can be |chapter| or |section|
depending on the chosen structuring.
Alternatively, one can modify the macro |\thepage| appropriately
and reset the counter |page| at the start of each child file.

%%%%%%%%%%%%%%%%%%%%%%%%%%%%%%%%%%%%%%%%%%%%%%%%%%%%%%%%%%%%%%%%%%%%%%%%%%%%%%%%
\subsection{Conditional Processing}
\label{sec:conditional}

The package provides a mechanism to compile different versions
of a document. To customise the versions further some conditional processing
can come in handy to distinguish which version is being compiled.
The package provides two macros to describe the compilation context:

%%%%%%%%%%%%%%%%%%%%%%%%%%%%%%%%%%%%%%%%
\DescribeMacro{\ifchilddoc}
The conditional |\ifchilddoc| distinguishes between the compilation of
child documents and the main document:
%
\begin{center}
|\ifchilddoc |\textit{child-code}| |[|\||else |\textit{main-code}]| \||fi|
\end{center}

%%%%%%%%%%%%%%%%%%%%%%%%%%%%%%%%%%%%%%%%
\DescribeMacro{\childdocname}
\DescribeMacro{\childdocjob}
The macro |\childdocname| contains the filename (without extension)
of the main or child file being processed.
Note that |\childdocjob| will always contain the name of the main file.

%%%%%%%%%%%%%%%%%%%%%%%%%%%%%%%%%%%%%%%%
\paragraph{Title Page.}

Conditional processing can be used to include a title or banner page
in the main document when proper precautions are taken.
Importantly, the code in the main file should ensure that the page counter
(as well as other status parameters which are stored in the |.aux| files)
takes the same value after the conditional processing.
Otherwise the page numbers may take divergent values
depending on which part is compiled.

For example, a title page could be declared by:
%
\begin{center}
\begin{tabular}{l}
|\ifchilddoc\||else|\\
|\addtocounter{page}{-1}|\\
\textit{code for title page}\\
|\newpage|\\
|\||fi|
\end{tabular}
\end{center}
%
A banner page for the child documents can be generated by:
%
\begin{center}
\begin{tabular}{l}
|\ifchilddoc|\\
|\addtocounter{page}{-1}|\\
\textit{code for banner page}\\
|\newpage|\\
|\||fi|
\end{tabular}
\end{center}
%
Here one could write a message such as:
\begin{center}
|This is the part \childdocname{} of \childdocjob{}.|
\end{center}

%%%%%%%%%%%%%%%%%%%%%%%%%%%%%%%%%%%%%%%%%%%%%%%%%%%%%%%%%%%%%%%%%%%%%%%%%%%%%%%%
\subsection{Flags}
\label{sec:flags}

The package makes it easy to generate different versions
of the main or child documents.
To this end compilation flags can be defined
and assigned different default values.
They will be particularly useful in conjunction
with the forwarding mechanism described in \secref{sec:forward}.

For example, it may be useful to have a flag |\version|
which can be set to |draft| or |final|.
The document source will contain some conditional code
depending on the value of |\version|.
Suppose further, the flag should default to |final| for the main file
and to |draft| for child files
which is a natural assignment for editing the document.
This is achieved by placing the following code
in the preamble of the main document
(below the |\childdocmain| directive):
%
\begin{center}
\begin{tabular}{l}
|\ifchilddoc|\\
|\providecommand{\version}{draft}|\\
|\||else|\\
|\providecommand{\version}{final}|\\
|\||fi|
\end{tabular}
\end{center}
%
The definition by |\providecommand| makes sure
that previous definitions are not overwritten.
Further statements |\providecommand{\version}{...}|
can thus be added before the above code to override it.

For the main file, one might add a line
(between |\childdocmain| and the above block)
%
\begin{center}
|%\ifchilddoc\||else\providecommand{\version}{draft}\||fi|
\end{center}
%
which can be uncommented to produce a draft version.
Likewise one can add a line to the very top of a child file
(above the |\childdocof{|\textit{main}|}| directive)
%
\begin{center}
|%\providecommand{\version}{final}|
\end{center}
%
which can be uncommented to produce the final version of this child document.

%%%%%%%%%%%%%%%%%%%%%%%%%%%%%%%%%%%%%%%%%%%%%%%%%%%%%%%%%%%%%%%%%%%%%%%%%%%%%%%%
\subsection{Forwarding}
\label{sec:forward}

Different versions of the main or child documents
using compilation flags as described in \secref{sec:flags}
can be (permanently) stored in different files
for convenient compilation, viewing and distribution.
To this end, the package defines a command
to pass on compilation to a different file:

%%%%%%%%%%%%%%%%%%%%%%%%%%%%%%%%%%%%%%%%
\DescribeMacro{\childdocforward}
The command |\childdocforward| redirects processing to
another source file:
%
\begin{center}
\begin{tabular}{l}
|\input{childdoc.def}|\\
|\childdocforward[|\textit{main}|]{|\textit{dest}|}|\\
\end{tabular}
\end{center}
%
The argument \textit{dest} is the destination file
(without extension).
It should be the main file or one of the child files.
Note that further \textsf{childdoc} directives
such as |\childdocof| and |\childdocforward|
in the indicated file will be processed in this form.
The optional argument \textit{main}
passes on directly to the main file \textit{main}
while pretending to compile the child \textit{dest}.
This form behaves as if \textit{dest}
issues |\childdocof{|\textit{main}|}| right away,
and no further \textsf{childdoc} directives will be processed.

%%%%%%%%%%%%%%%%%%%%%%%%%%%%%%%%%%%%%%%%
\DescribeMacro{\...prefix}
In the alternative form |\childdocforwardprefix|,
%
\begin{center}
\begin{tabular}{l}
|\input{childdoc.def}|\\
|\childdocforwardprefix[|\textit{main}|]{|\textit{prefix}|}{|\textit{dest}|}|
\end{tabular}
\end{center}
%
the destination file is determined by a pattern
depending on the current file:
To make this work, the current file must be called
`{\textit{prefix}\hspace{0.2em}\textit{suffix}}'
with \textit{prefix} matching precisely the argument.
Processing is then passed on to the file
`{\textit{dest}\hspace{0.2em}\textit{suffix}}'.
Surely, the same effect is achieved by
directly specifying the
argument `{\textit{dest}\hspace{0.2em}\textit{suffix}}'
in the first form.
However, that requires to set up a different file
for each child. With the alternative form of the command
all these files can have exactly the same content
which simplifies setting them up and maintaining them.

For example, the following file |draft.tex|
with a compilation flag |\version| as described in \secref{sec:flags}
compiles the main document as a draft:
%
\begin{center}
\begin{tabular}{l}
|\def\version{draft}|\\
|\input{childdoc.def}|\\
|\childdocforward{|\textit{main}|}|
\end{tabular}
\end{center}
%
Likewise, the following files |final|\textit{nn}|.tex|
compile the final version of the child document
|child|\textit{nn}|.tex|:
%
\begin{center}
\begin{tabular}{l}
|\def\version{final}|\\
|\input{childdoc.def}|\\
|\childdocforwardprefix{final}{child}|
\end{tabular}
\end{center}
%

Note that when several versions of a main file and/or of each child file
are to be generated, it may be convenient to set up a |Makefile| or
shell script to automatise the process.

%%%%%%%%%%%%%%%%%%%%%%%%%%%%%%%%%%%%%%%%%%%%%%%%%%%%%%%%%%%%%%%%%%%%%%%%%%%%%%%%
\subsection{Command Line Processing}
\label{sec:commandline}

The effect of redirection files can also be achieved by invoking
the \LaTeX{} compiler with a more elaborate command line.
Most conveniently this should be done as part
of a shell script or a |Makefile|.

When using \textsf{childdoc} in the main file, the following
command lines effectively perform a redirection
(note that depending on the shell being used,
backslashes may have to be doubled: `|\|' $\to$ `|\\|'):
%
\begin{center}
|... -jobname "|\textit{target}|" |\\|"|[\textit{flags}]%
|\input{childdoc.def}\childdocforward[|\textit{main}|]{|\textit{dest}|}"|
\end{center}
%
Here \textit{target} is the name of the output file,
\textit{main} is the name of the main file
and \textit{dest} is the name of the main or child file to be processed
(all filenames without extensions).
The optional argument \textit{main} can be omitted
if \textit{main} matches \textit{dest}.
Optionally, compilation \textit{flags} can be defined via |\def| commands.
This command line makes the \TeX{} engine believe
it is compiling the file \textit{target}
whose content is specified as the latter parameter.
The provided code then forwards the processing to
\textit{main} or \textit{dest} as described in \secref{sec:forward}.

%%%%%%%%%%%%%%%%%%%%%%%%%%%%%%%%%%%%%%%%%%%%%%%%%%%%%%%%%%%%%%%%%%%%%%%%%%%%%%%%
\subsection{Include by Input}
\label{sec:input}

Including child documents by |\include| has some restrictions by design.
Most notably, the content of a child document always occupies
its own set of pages; pages cannot be shared between child documents.
Usually, this behaviour makes perfect sense
because each child document contain an essential part of the document.
However, in some situations it may be desirable to compose
a document from a collection of parts
without having mandatory page breaks between then.
For this case, the package
provides a mechanism to include parts
by |\input| which can also be processed individually.
However, by construction this mechanism
requires manual handling of the content to be output.

%%%%%%%%%%%%%%%%%%%%%%%%%%%%%%%%%%%%%%%%
\DescribeMacro{\ifchilddocmanual}
The main file should be prepared as usual, see \secref{sec:include}.
However, the document body must make a distinction
between processing of an individual part and of the main document, e.g.:
%
\begin{center}
\begin{tabular}{l}
|\ifchilddocmanual|\\
|\input{\childdocname}|\\
|\||else|\\
\textit{document body with }|\input{|\textit{part}|}|\\
|\||fi|
\end{tabular}
\end{center}
%
The conditional |\ifchilddocmanual| is true whenever
a part to be included by |\input| is being compiled,
and the name of the part is stored in |\childdocname|.

%%%%%%%%%%%%%%%%%%%%%%%%%%%%%%%%%%%%%%%%
\DescribeMacro{\childdocby}
Each part to be included by |\input| should start with:
%
\begin{center}
\begin{tabular}{l}
|\input{childdoc.def}|\\
|\childdocby{|\textit{main}|}|\\
\end{tabular}
\end{center}
%
The directive |\childdocby| is similar to |\childdocof|
described in \secref{sec:include},
but the subsequent selection of content must be done manually.
To that end, both |\ifchilddoc| and |\ifchilddocmanual|
will be true upon processing of a part,
and the name of the part is stored in |\childdocname|.
Note that |\jobname| will be set to the filename of the current part
so that each part receives an individual |.aux| file
that does not interfere with the |.aux| file(s) of the main document.
This behaviour can be altered by the alternative form
|\childdocby[*]{|\textit{main}|}| (with a non-empty optional argument)
which uses the |.aux| file of the main document
by setting |\jobname| to \textit{main}.

%%%%%%%%%%%%%%%%%%%%%%%%%%%%%%%%%%%%%%%%%%%%%%%%%%%%%%%%%%%%%%%%%%%%%%%%%%%%%%%%
\subsection{Driver Development}
\label{sec:driver}

The \textsf{childdoc} mechanism can also be use for the development
of definition files such as \LaTeX{} styles or classes.
This case differs from the above setup with multiple parts
included by |\include| in that no |\includeonly| should be invoked.
This can be achieved by starting the include file
(before |\ProvidesPackage|) with:
%
\begin{center}
\begin{tabular}{l}
|\input{childdoc.def}|\\
|\childdocforward{|\textit{main}|}|\\
\end{tabular}
\end{center}
%
or alternatively with:
%
\begin{center}
\begin{tabular}{l}
|\input{childdoc.def}|\\
|\childdocby{|\textit{main}|}|\\
\end{tabular}
\end{center}
%
Both forms have slightly different effects as described above.
The main file is prepared as usual, see \secref{sec:include}.

%%%%%%%%%%%%%%%%%%%%%%%%%%%%%%%%%%%%%%%%%%%%%%%%%%%%%%%%%%%%%%%%%%%%%%%%%%%%%%%%
\subsection{Legacy Detection}
\label{sec:detection}

The directive |\childdocmain| in the main file can detect
whether the complete document or merely a child is to be compiled
even without using the directive |\childdocof|.
This method is deprecated because it is less robust
and there is no compelling reason to use it;
it is merely provided for backward compatibility
and it may be removed in future versions.

If the detection mechanism is to be used,
it is mandatory to correctly specify
the filename of the main file as the argument of |\childdocmain|:
%
\begin{center}
\begin{tabular}{l}
|\input{childdoc.def}|\\
|\childdocmain{|\textit{main}|}|\\
\end{tabular}
\end{center}
%
If |\jobname| does not match the argument \textit{main} of |\childdocmain|,
it is assumed that |\jobname| points to the child file to be compiled.
When using |\childdocmain| with the main file specified as argument,
it suffices to start a child file
with just |\input{|\textit{main}|}|
without loading of the package and using |\childdocof|.
If instead all processing is done
with the appropriate \textsf{childdoc} directives,
the argument of \textit{main} of |\childdocmain| can be empty.

An alternative version of the command line processing described
in \secref{sec:commandline} using the detection mechanism reads:
%
\begin{center}
|... -jobname "|\textit{target}|" "|[\textit{flags}]%
[|\def\jobname{|\textit{dest}|}|]|\input{|\textit{main}|}"|
\end{center}

%%%%%%%%%%%%%%%%%%%%%%%%%%%%%%%%%%%%%%%%%%%%%%%%%%%%%%%%%%%%%%%%%%%%%%%%%%%%%%%%
\subsection{Manual Code}
\label{sec:manual}

In case one cannot be certain whether the definitions file |childdoc.def|
is installed on the target \TeX{} distribution
and one prefers not to ship it,
it is conceivable to paste a few relevant commands into the sources.

To that end, drop all statements |\input{childdoc.def}|
and perform the replacements as outlined below.
Instead of |\childdocmain{|\textit{main}|}| add the following code
to the top of the main file:
%
\begin{center}
\begin{tabular}{l}
|\||ifdefined\childdocname\endinput\||fi\newif\ifchilddoc|\\
|\edef\childdocname{\scantokens\expandafter{\jobname\noexpand}}|\\
|\def\childdocmain{|\textit{main}|}\||ifx\childdocmain\childdocname\||else|\\
|\childdoctrue\includeonly{\childdocname}\let\jobname\childdocmain\||fi|\\
\end{tabular}
\end{center}
%
Instead of |\childdocof{|\textit{main}|}| just include the main file
at the top of each child file:
%
\begin{center}
|\input{|\textit{main}|}|
\end{center}
%
A simple redirection |\childdocforward{|\textit{dest}|}| is achieved by:
%
\begin{center}
|\def\jobname{|\textit{dest}|}\input{\jobname}|
\end{center}
%
The redirection with prefix
|\childdocforwardprefix[|\textit{prefix}|]{|\textit{dest}|}|
is accomplished by:
%
\begin{center}
\begin{tabular}{l}
|{\edef\jobname{\scantokens\expandafter{\jobname\noexpand}}|\\
|\def\redirectjob |\textit{prefix}|#1~~~{\gdef\jobname{|\textit{dest}|#1}}|\\
|\expandafter\redirectjob\jobname~~~}\input{\jobname}|
\end{tabular}
\end{center}

In an alternative approach,
child documents can be compiled by a specific command line
without additional code or specific definitions:
%
\begin{center}
|... -jobname "|\textit{target}|" "|[\textit{flags}]%
|\includeonly{|\textit{dest}|}\input{|\textit{main}|}"|
\end{center}
%

%%%%%%%%%%%%%%%%%%%%%%%%%%%%%%%%%%%%%%%%%%%%%%%%%%%%%%%%%%%%%%%%%%%%%%%%%%%%%%%%
%%%%%%%%%%%%%%%%%%%%%%%%%%%%%%%%%%%%%%%%%%%%%%%%%%%%%%%%%%%%%%%%%%%%%%%%%%%%%%%%
\section{Information}

%%%%%%%%%%%%%%%%%%%%%%%%%%%%%%%%%%%%%%%%%%%%%%%%%%%%%%%%%%%%%%%%%%%%%%%%%%%%%%%%
\subsection{Copyright}

Copyright \copyright{} 2017--2018 Niklas Beisert

This work may be distributed and/or modified under the
conditions of the \LaTeX{} Project Public License, either version 1.3
of this license or (at your option) any later version.
The latest version of this license is in
  \url{http://www.latex-project.org/lppl.txt}
and version 1.3 or later is part of all distributions of \LaTeX{}
version 2005/12/01 or later.

This work has the LPPL maintenance status `maintained'.

The Current Maintainer of this work is Niklas Beisert.

This work consists of the files |README.txt|, |childdoc.ins| and |childdoc.dtx|
as well as the derived files |childdoc.def|, |cdocsamp.tex|
with |cdocsch1.tex|, |cdocsch2.tex|, |cdocspt3.tex|, |cdocspt4.tex|,
|cdocsdrf.tex|, |cdocsfn1.tex|, |cdocsfn2.tex|
as well as |childdoc.pdf|.

%%%%%%%%%%%%%%%%%%%%%%%%%%%%%%%%%%%%%%%%%%%%%%%%%%%%%%%%%%%%%%%%%%%%%%%%%%%%%%%%
\subsection{Files and Installation}

The package consists of the files:
%
\begin{center}
\begin{tabular}{ll}
    |README.txt|   & readme file \\
    |childdoc.ins| & installation file \\
    |childdoc.dtx| & source file \\
    |childdoc.def| & definition file \\
    |cdocsamp.tex| & sample main file \\
    |cdocsch1.tex| & sample include file \\
    |cdocsch2.tex| & sample include file \\
    |cdocspt3.tex| & sample part file \\
    |cdocspt4.tex| & sample part file \\
    |cdocsdrf.tex| & sample redirection file \\
    |cdocsfn1.tex| & sample redirection file \\
    |cdocsfn2.tex| & sample redirection file \\
    |childdoc.pdf| & manual
\end{tabular}
\end{center}
%
The distribution consists of the files
|README.txt|, |childdoc.ins| and |childdoc.dtx|.
%
\begin{itemize}
\item
Run (pdf)\LaTeX{} on |childdoc.dtx|
to compile the manual |childdoc.pdf| (this file).
\item
Run \LaTeX{} on |childdoc.ins| to create the definitions file |childdoc.def|
and the sample |cdocsamp.tex| with include files
|cdocsch1.tex|, |cdocsch2.tex|, |cdocspt3.tex|, |cdocspt4.tex|,
|cdocsdrf.tex|, |cdocsfn1.tex|, |cdocsfn2.tex|.
Then copy the file |childdoc.def| to an appropriate directory of your \LaTeX{}
distribution, e.g.\ \textit{texmf-root}|/tex/latex/childdoc|.
\end{itemize}

%%%%%%%%%%%%%%%%%%%%%%%%%%%%%%%%%%%%%%%%%%%%%%%%%%%%%%%%%%%%%%%%%%%%%%%%%%%%%%%%
\subsection{Related CTAN Packages}

There are several other packages which offer a similar functionality:
%
\begin{itemize}
\item
The packages
\href{http://ctan.org/pkg/docmute}{\textsf{docmute}},
\href{http://ctan.org/pkg/includex}{\textsf{includex}} and
\href{http://ctan.org/pkg/standalone}{\textsf{standalone}}
provide commands to include only the document body of
a child file thus allowing both files to be compiled individually.
\item
The packages \href{http://ctan.org/pkg/subdocs}{\textsf{subdocs}}
and \href{http://ctan.org/pkg/subfiles}{\textsf{subfiles}}
provide structures in which the main and child documents can be
encapsulated and allowing them to be compiled individually.
The inclusion mechanism is different from the conventional |\include|.
\item
The package \href{http://ctan.org/pkg/combine}{\textsf{combine}}
is an elaborate solution to combine several documents into one.
\end{itemize}
%
See also the CTAN topic \href{http://ctan.org/topic/subdocs}{\textsf{subdocs}}
for further related packages.
The present package differs from the above solutions in that
a document structure constructed with the conventional |\include| mechanism
just needs two extra commands at the top of every file
such that all constituent files can be compiled individually.

%%%%%%%%%%%%%%%%%%%%%%%%%%%%%%%%%%%%%%%%%%%%%%%%%%%%%%%%%%%%%%%%%%%%%%%%%%%%%%%%
%\subsection{Feature Suggestions}
%
%The following is a list of features which may be useful for future
%versions of this package:
%%
%\begin{itemize}
%\item
%\ldots
%\end{itemize}

%%%%%%%%%%%%%%%%%%%%%%%%%%%%%%%%%%%%%%%%%%%%%%%%%%%%%%%%%%%%%%%%%%%%%%%%%%%%%%%%
\subsection{Revision History}

%%%%%%%%%%%%%%%%%%%%%%%%%%%%%%%%%%%%%%%%
\paragraph{v2.0:} 2018/12/30

\begin{itemize}
\item
immediate forward processing
\item
added |\childdocby| mechanism
\item
manual restructured
\end{itemize}

%%%%%%%%%%%%%%%%%%%%%%%%%%%%%%%%%%%%%%%%
\paragraph{v1.6:} 2018/01/17

\begin{itemize}
\item
application for development of include files
\item
corrections to manual
\end{itemize}

%%%%%%%%%%%%%%%%%%%%%%%%%%%%%%%%%%%%%%%%
\paragraph{v1.5:} 2017/05/21

\begin{itemize}
\item
more complete structuring introduced
\item
|\childdocof| introduced
\item
|\childdoc| renamed to |\childdocmain|
\item
|\childredirect| renamed to |\childdocforward| and |\childdocforwardprefix|
and functionality expanded
\end{itemize}

%%%%%%%%%%%%%%%%%%%%%%%%%%%%%%%%%%%%%%%%
\paragraph{v1.0:} 2017/04/27

\begin{itemize}
\item
manual and install package
\item
first version published on CTAN
\end{itemize}

%%%%%%%%%%%%%%%%%%%%%%%%%%%%%%%%%%%%%%%%
\paragraph{v0.6:} 2017/04/26

\begin{itemize}
\item
redirection mechanism added
\end{itemize}

%%%%%%%%%%%%%%%%%%%%%%%%%%%%%%%%%%%%%%%%
\paragraph{v0.5:} 2017/04/26

\begin{itemize}
\item
functionality in definition file
\end{itemize}


%%%%%%%%%%%%%%%%%%%%%%%%%%%%%%%%%%%%%%%%%%%%%%%%%%%%%%%%%%%%%%%%%%%%%%%%%%%%%%%%
%%%%%%%%%%%%%%%%%%%%%%%%%%%%%%%%%%%%%%%%%%%%%%%%%%%%%%%%%%%%%%%%%%%%%%%%%%%%%%%%
%%%%%%%%%%%%%%%%%%%%%%%%%%%%%%%%%%%%%%%%%%%%%%%%%%%%%%%%%%%%%%%%%%%%%%%%%%%%%%%%
\appendix

\settowidth\MacroIndent{\rmfamily\scriptsize 000\ }

 \DocInput{childdoc.dtx}

\end{document}
%</driver>
% \fi
%
% %%%%%%%%%%%%%%%%%%%%%%%%%%%%%%%%%%%%%%%%%%%%%%%%%%%%%%%%%%%%%%%%%%%%%%%%%%%%%%
% %%%%%%%%%%%%%%%%%%%%%%%%%%%%%%%%%%%%%%%%%%%%%%%%%%%%%%%%%%%%%%%%%%%%%%%%%%%%%%
% \section{Sample}
%\iffalse
%<*samplemain>
%\fi
%
% The following presents a sample document
% with two chapters, two parts, a title page,
% a compile flag as well as three forwarding files to set the flag.
% It consists of eight |.tex| files:
% \begin{center}
% \begin{tabular}{ll}
% |cdocsamp.tex|&main file\\
% |cdocsch1.tex|&include file for chapter 1\\
% |cdocsch2.tex|&include file for chapter 2\\
% |cdocspt3.tex|&include file for part 3\\
% |cdocspt4.tex|&include file for part 4\\
% |cdocsdrf.tex|&forwarding file for main file in draft mode\\
% |cdocsfi1.tex|&forwarding file for final version of chapter 1\\
% |cdocsfi2.tex|&forwarding file for final version of chapter 2\\
% \end{tabular}
% \end{center}
% Each of the eight files can be compiled directly by the \LaTeX{} compiler.
%
% %%%%%%%%%%%%%%%%%%%%%%%%%%%%%%%%%%%%%%
% \paragraph{Main File.}
%
% The main file is called |cdocsamp.tex|.
%
% Load the \textsf{childdoc} definitions and
% declare the filename for the main document:
%    \begin{macrocode}
\input{childdoc.def}
\childdocmain{}
%    \end{macrocode}

% Optional override for |\version| flag:
%    \begin{macrocode}
%%\ifchilddoc\else\providecommand{\version}{draft}\fi
%    \end{macrocode}

% Define the default values for the |\version| flag
% (|final| for the main file and |draft| for childs):
%    \begin{macrocode}
\ifchilddoc
\providecommand{\version}{draft}
\else
\providecommand{\version}{final}
\fi
%    \end{macrocode}

% Load the standard document class:
%    \begin{macrocode}
\documentclass[12pt]{article}
%    \end{macrocode}

% Start the document body:
%    \begin{macrocode}
\begin{document}
%    \end{macrocode}

% Declare a title page.
% Print title, part of document being processed and version flag:
%    \begin{macrocode}
\addtocounter{page}{-1}
\begin{center}
{\LARGE\bfseries{}childdoc example\par}
\vspace{1cm}
\ifchilddoc
\ifchilddocmanual part\else chapter\fi:
`\childdocname' of `\childdocjob'\par
\else
main document: `\childdocjob'\par
\fi
version: \version\par
\end{center}
\newpage
%    \end{macrocode}

% Manually include selected file,
% otherwise process as usual:
%    \begin{macrocode}
\ifchilddocmanual
\section*{part `\childdocname'}
\input{\childdocname}
\else
%    \end{macrocode}

% Include the two chapters:
%    \begin{macrocode}
\include{cdocsch1}
\include{cdocsch2}
%    \end{macrocode}

% Include the two parts unless only chapters should be displayed:
%    \begin{macrocode}
\ifchilddoc\else
\section{part three}
\input{cdocspt3}
\section{part four}
\input{cdocspt4}
\fi
%    \end{macrocode}

% Process as usual until here:
%    \begin{macrocode}
\fi
%    \end{macrocode}

% End of document body:
%    \begin{macrocode}
\end{document}
%    \end{macrocode}
%\iffalse
%</samplemain>
%\fi
%
% %%%%%%%%%%%%%%%%%%%%%%%%%%%%%%%%%%%%%%
% \paragraph{Chapter Include Files.}
%
% The include files are called |cdocsch1.tex| and |cdocsch2.tex|.
%
%\iffalse
%<*samplechap1|samplechap2>
%\fi

% Optional override for |\version| flag:
%    \begin{macrocode}
%%\providecommand{\version}{final}
%    \end{macrocode}

% Include the main document:
%    \begin{macrocode}
\input{childdoc.def}
\childdocof{cdocsamp}
%    \end{macrocode}

%\iffalse
%</samplechap1|samplechap2>
%\fi
%
%\iffalse
%<*samplechap1>
%\fi
% Some text for chapter 1:
%    \begin{macrocode}
\section{one}
some text in chapter one
%    \end{macrocode}

%\iffalse
%</samplechap1>
%\fi
% Some text for chapter 2:
%\iffalse
%<*samplechap2>
%\fi
%    \begin{macrocode}
\section{two}
more text in chapter two
%    \end{macrocode}

%\iffalse
%</samplechap2>
%\fi
%
% %%%%%%%%%%%%%%%%%%%%%%%%%%%%%%%%%%%%%%
% \paragraph{Part Include Files.}
%
% The include files are called |cdocspt3.tex| and |cdocspt4.tex|.
%
%\iffalse
%<*samplepart3|samplepart4>
%\fi

% Optional override for |\version| flag:
%    \begin{macrocode}
%%\providecommand{\version}{final}
%    \end{macrocode}

% Include the main document:
%    \begin{macrocode}
\input{childdoc.def}
\childdocby{cdocsamp}
%    \end{macrocode}

%\iffalse
%</samplepart3|samplepart4>
%\fi
%
%\iffalse
%<*samplepart3>
%\fi
% Some text for part 3:
%    \begin{macrocode}
some text in part three
%    \end{macrocode}

%\iffalse
%</samplepart3>
%\fi
% Some text for part 4:
%\iffalse
%<*samplepart4>
%\fi
%    \begin{macrocode}
more text in part four
%    \end{macrocode}

%\iffalse
%</samplepart4>
%\fi
%
% %%%%%%%%%%%%%%%%%%%%%%%%%%%%%%%%%%%%%%
% \paragraph{Forwarding for a Complete Draft.}
%
% The following forwarding file |cdocsdrf.tex|
% compiles the main document in draft mode:
%\iffalse
%<*sampledraft>
%\fi
%    \begin{macrocode}
\def\version{draft}
\input{childdoc.def}
\childdocforward{cdocsamp}
%    \end{macrocode}

%\iffalse
%</sampledraft>
%\fi
%
% %%%%%%%%%%%%%%%%%%%%%%%%%%%%%%%%%%%%%%
% \paragraph{Forwarding for Final Version of the Chapters.}
%
% The following forwarding files |cdocsfn1.tex| and |cdocsfn2.tex|
% (with identical content)
% compile the final versions of the child documents
% |cdocsch1.tex| and |cdocsch2.tex|, respectively:
%\iffalse
%<*samplefinal>
%\fi
%    \begin{macrocode}
\def\version{final}
\input{childdoc.def}
\childdocforwardprefix[cdocsamp]{cdocsfn}{cdocsch}
%    \end{macrocode}

%\iffalse
%</samplefinal>
%\fi
%
% %%%%%%%%%%%%%%%%%%%%%%%%%%%%%%%%%%%%%%
% \paragraph{Command Line Processing.}
%
% The following three command lines generate the output files
% |cdocscld|, |cdocscl1| and |cdocscl2|
% which should be identical to
% |cdocsdrf|, |cdocsch1| and |cdocsfn2|, respectively:
% \begin{center}
% \begin{tabular}{l}
% |latex -jobname cdocscld \|\\
% |  "\def\version{draft}\input{childdoc.def}\childdocforward{cdocsamp}"|\\
% |latex -jobname cdocscl1 \|\\
% |  "\input{childdoc.def}\childdocforward[cdocsamp]{cdocsch1}"|\\
% |latex -jobname cdocscl2 \|\\
% |  "\def\version{final}\input{childdoc.def}\childdocforward{cdocsch2}"|
% \end{tabular}
% \end{center}
% Note that the trailing backslash on each first line
% merely continues the input to the second line
% (for convenient cut ant paste).
% Furthermore, the command |latex| can be replaced by any
% of its alternative versions such as |pdflatex|.
%
% %%%%%%%%%%%%%%%%%%%%%%%%%%%%%%%%%%%%%%%%%%%%%%%%%%%%%%%%%%%%%%%%%%%%%%%%%%%%%%
% %%%%%%%%%%%%%%%%%%%%%%%%%%%%%%%%%%%%%%%%%%%%%%%%%%%%%%%%%%%%%%%%%%%%%%%%%%%%%%
% \section{Implementation}
%\iffalse
%<*package>
%\fi
%
% This section describes the definitions file |childdoc.def|.

% The definitions cannot be loaded using |\usepackage| or |\RequirePackage|
% which has a mechanism to prevent loading a style file more than once.
% When loading the definitions by means of |\input|
% multiple instances have to be prevented manually:
%\iffalse
%This code needs to be before the `\ProvidesFile' directive
%which is defined at the beginning of this file.
%Therefore it is also placed there and commented out here.
%</package>
%<*discard>
%\fi
%    \begin{macrocode}
\ifdefined\childdocmain\endinput\fi
%    \end{macrocode}
%\iffalse
%</discard>
%<*package>
%\fi
%
% \macro{\ifchilddoc}
% \macro{\ifchilddocmanual}
% The conditional |\ifchilddoc| tells whether a
% child (true) or main (false) document is being compiled.
% The conditional |\ifchilddocmanual| tells whether
% the |\includeonly| mechanism is used (false) or
% the selection of child files must be performed manually (true).
% The definitions initialise to false:
%    \begin{macrocode}
\newif\ifchilddoc
\newif\ifchilddocmanual
%    \end{macrocode}

% \macro{\childdocname}
% \macro{\childdocjob}
% The macro |\childdocname| stores the name of the main document
% to be compiled. The macro |\childdocjob| stores the name of
% the document on which the \LaTeX{} compiler was originally invoked.
% The content of |\jobname| cannot be compared
% to filenames specified in the source due to different catcodes.
% The following code rescans |\jobname|, stores the result
% in |\childdocname| and saves a copy in |\childdocjob|:
%    \begin{macrocode}
\edef\childdocname{\scantokens\expandafter{\jobname\noexpand}}
\let\childdocjob\childdocname
%    \end{macrocode}

% \macro{\childdocdisable}
% The macro |\childdocdisable| prevents the main file
% from being processed more than once.
% At this stage, the main document command |\childdocmain|
% is assumed to be called once again where it should do nothing.
% Any subsequent call to it should prevent
% a secondary processing of the main document
% It overwrites the forwarding commands
% |\childdocof| and |\childdocforward|
% with empty macros to prevent further inclusions of the main document:
%    \begin{macrocode}
\newcommand{\childdocdisable}
{
  \renewcommand{\childdocmain}[1]{\renewcommand{\childdocmain}[1]{\endinput}}
  \renewcommand{\childdocof}[1]{}
  \renewcommand{\childdocby}[2][]{}
  \renewcommand{\childdocforward}[2][]{}
  \renewcommand{\childdocdisable}{}
}
%    \end{macrocode}

% \macro{\childdocmain}
% The macro |\childdocmain| is to be called at the top of the main file
% with nothing or the main filename (without extension) as argument.
% First, it breaks loops.
% If the argument is not empty and does not match |\childdocname|
% (which is set by the first inclusion of |childdoc.def|),
% |\ifchilddoc| is set to true, |\includeonly| is applied to the child file
% and |\jobname| is set to the main file
% (for proper handling of |.aux| files):
%    \begin{macrocode}
\newcommand{\childdocmain}[1]
{
  \childdocdisable\childdocmain{}
  \if?#1?\else
    \begingroup
      \def\childdoctmp{#1}
      \ifx\childdoctmp\childdocname
        \def\childdoctmp{}
      \else
        \def\childdoctmp
        {
          \childdoctrue
          \includeonly{\childdocname}
          \def\childdocjob{#1}
          \def\jobname{#1}
        }
      \fi
      \expandafter
    \endgroup
    \childdoctmp
  \fi
}
%    \end{macrocode}

% \macro{\childdocof}
% The command |\childdocof| redirects
% compilation to the main file |#1|.
%    \begin{macrocode}
\newcommand{\childdocof}[1]
{
  \childdocdisable
  \childdoctrue
  \includeonly{\childdocname}
  \def\jobname{#1}
  \def\childdocjob{#1}
  \input{#1}
}
%    \end{macrocode}

% \macro{\childdocby}
% The command |\childdocby| ....
%    \begin{macrocode}
\newcommand{\childdocby}[2][]
{
  \childdocdisable
  \childdoctrue
  \childdocmanualtrue
  \if?#1?\else
    \def\jobname{#2}
  \fi
  \def\childdocjob{#2}
  \input{#2}
  \endinput
}
%    \end{macrocode}

% \macro{\childdocforward}
% The command |\childdocforward| redirects
% compilation to the main file or
% (if the optional argument is given) a child file.
% Parameters are set as if the main file
% or a child file starting with |\childdocof| was compiled.
% Then compilation is handed over to the main file:
%    \begin{macrocode}
\newcommand{\childdocforward}[2][]
{
  \begingroup
    \if?#1?
      \def\childdoctmp
      {
        \def\childdocname{#2}
        \def\childdocjob{#2}
        \def\jobname{#2}
        \input{#2}
        \endinput
      }
    \else
      \def\childdoctmp
      {
        \childdocdisable
        \def\childdocname{#2}
        \childdoctrue
        \includeonly{#2}
        \def\childdocjob{#1}
        \def\jobname{#1}
        \input{#1}
        \endinput
      }
    \fi
    \expandafter
  \endgroup
  \childdoctmp
}
%    \end{macrocode}

% \macro{\childdocforwardprefix}
% The command |\childdocforwardprefix| redirects
% compilation to the main or a child file by means of a pattern.
% The prefix |#1| in the current filename is replaced by |#2|
% and the suffix of the current filename is kept
% (it is assumed that the filename does not contain the substring `|~~~|'
% which is used as a delimiter).
% Compilation is handed over to the new file by |\childdocforward|:
%    \begin{macrocode}
\newcommand{\childdocforwardprefix}[3][]
{
  \begingroup
    \def\childdocextract #2##1~~~{\def\childdoctmp{\childdocforward[#1]{#3##1}}}
    \expandafter\childdocextract\childdocname~~~
    \expandafter
  \endgroup
  \childdoctmp
}
%    \end{macrocode}

% \macro{\childdoc}
% The deprecated macro |\childdoc| is a legacy version of |\childdocmain|:
%    \begin{macrocode}
\newcommand{\childdoc}{\childdocmain}
%    \end{macrocode}

% \macro{\childdocredirect}
% The deprecated macro |\childdocredirect| is a legacy version
% of |\childdocforward| and |\childdocforwardprefix|:
%    \begin{macrocode}
\newcommand{\childdocredirect}[2][]
{
  \begingroup
    \if?#1?
      \def\childdoctmp{\childdocforward{#2}}
    \else
      \def\childdoctmp{\childdocforwardprefix{#1}{#2}}
    \fi
    \expandafter
  \endgroup
  \childdoctmp
}
%    \end{macrocode}

%\iffalse
%</package>
%\fi
%
\endinput
\childdocforward[|\textit{main}|]{|\textit{dest}|}"|
\end{center}
%
Here \textit{target} is the name of the output file,
\textit{main} is the name of the main file
and \textit{dest} is the name of the main or child file to be processed
(all filenames without extensions).
The optional argument \textit{main} can be omitted
if \textit{main} matches \textit{dest}.
Optionally, compilation \textit{flags} can be defined via |\def| commands.
This command line makes the \TeX{} engine believe
it is compiling the file \textit{target}
whose content is specified as the latter parameter.
The provided code then forwards the processing to
\textit{main} or \textit{dest} as described in \secref{sec:forward}.

%%%%%%%%%%%%%%%%%%%%%%%%%%%%%%%%%%%%%%%%%%%%%%%%%%%%%%%%%%%%%%%%%%%%%%%%%%%%%%%%
\subsection{Include by Input}
\label{sec:input}

Including child documents by |\include| has some restrictions by design.
Most notably, the content of a child document always occupies
its own set of pages; pages cannot be shared between child documents.
Usually, this behaviour makes perfect sense
because each child document contain an essential part of the document.
However, in some situations it may be desirable to compose
a document from a collection of parts
without having mandatory page breaks between then.
For this case, the package
provides a mechanism to include parts
by |\input| which can also be processed individually.
However, by construction this mechanism
requires manual handling of the content to be output.

%%%%%%%%%%%%%%%%%%%%%%%%%%%%%%%%%%%%%%%%
\DescribeMacro{\ifchilddocmanual}
The main file should be prepared as usual, see \secref{sec:include}.
However, the document body must make a distinction
between processing of an individual part and of the main document, e.g.:
%
\begin{center}
\begin{tabular}{l}
|\ifchilddocmanual|\\
|\input{\childdocname}|\\
|\||else|\\
\textit{document body with }|\input{|\textit{part}|}|\\
|\||fi|
\end{tabular}
\end{center}
%
The conditional |\ifchilddocmanual| is true whenever
a part to be included by |\input| is being compiled,
and the name of the part is stored in |\childdocname|.

%%%%%%%%%%%%%%%%%%%%%%%%%%%%%%%%%%%%%%%%
\DescribeMacro{\childdocby}
Each part to be included by |\input| should start with:
%
\begin{center}
\begin{tabular}{l}
|% \iffalse
%
% childdoc.dtx Copyright (C) 2017-2018 Niklas Beisert
%
% This work may be distributed and/or modified under the
% conditions of the LaTeX Project Public License, either version 1.3
% of this license or (at your option) any later version.
% The latest version of this license is in
%   http://www.latex-project.org/lppl.txt
% and version 1.3 or later is part of all distributions of LaTeX
% version 2005/12/01 or later.
%
% This work has the LPPL maintenance status `maintained'.
%
% The Current Maintainer of this work is Niklas Beisert.
%
% This work consists of the files childdoc.dtx and childdoc.ins
% and the derived files childdoc.def and cdocsamp.tex with
% cdocsch1.tex, cdocsch2.tex, cdocsdrf.tex, cdocsfn1.tex, cdocsfn2.tex.
%
%<package>\ifdefined\childdocmain\endinput\fi
%<package>\ProvidesFile{childdoc.def}[2018/12/30 v2.0 child document driver]
%<samplemain>\ProvidesFile{cdocsamp.tex}[2018/12/30 v2.0 sample for childdoc]
%<*driver>
%\ProvidesFile{childdoc.drv}[2018/12/30 v2.0 childdoc reference manual file]
\PassOptionsToClass{10pt,a4paper}{article}
\documentclass{ltxdoc}

\usepackage[margin=35mm]{geometry}
\usepackage{hyperref}
\usepackage{hyperxmp}
\usepackage[usenames]{color}

\hypersetup{colorlinks=true}
\hypersetup{pdfstartview=FitH}
\hypersetup{pdfpagemode=UseNone}
\hypersetup{pdfsource={}}
\hypersetup{pdflang={en-UK}}
\hypersetup{pdfcopyright={Copyright 2017-2018 Niklas Beisert.
  This work may be distributed and/or modified under the
  conditions of the LaTeX Project Public License, either version 1.3
  of this license or (at your option) any later version.}}
\hypersetup{pdflicenseurl={http://www.latex-project.org/lppl.txt}}
\hypersetup{pdfcontactaddress={ETH Zurich, ITP, HIT K,
  Wolfgang-Pauli-Strasse 27}}
\hypersetup{pdfcontactpostcode={8093}}
\hypersetup{pdfcontactcity={Zurich}}
\hypersetup{pdfcontactcountry={Switzerland}}
\hypersetup{pdfcontactemail={nbeisert@itp.phys.ethz.ch}}
\hypersetup{pdfcontacturl={http://people.phys.ethz.ch/\xmptilde nbeisert/}}

\newcommand{\secref}[1]{\hyperref[#1]{section \ref*{#1}}}

\parskip1ex
\parindent0pt
\let\olditemize\itemize
\def\itemize{\olditemize\parskip0pt}

\begin{document}

\title{The \textsf{childdoc} Package}
\hypersetup{pdftitle={The childdoc Package}}
\author{Niklas Beisert\\[2ex]
  Institut f\"ur Theoretische Physik\\
  Eidgen\"ossische Technische Hochschule Z\"urich\\
  Wolfgang-Pauli-Strasse 27, 8093 Z\"urich, Switzerland\\[1ex]
  \href{mailto:nbeisert@itp.phys.ethz.ch}
  {\texttt{nbeisert@itp.phys.ethz.ch}}}
\hypersetup{pdfauthor={Niklas Beisert}}
\hypersetup{pdfsubject={Manual for the LaTeX2e Package childdoc}}
\date{30 December 2018, \textsf{v2.0}}
\maketitle

\begin{abstract}\noindent
\textsf{childdoc} is a \LaTeXe{} package
that enables the direct compilation
of document sections included by |\include|
to individual files.
\end{abstract}

\begingroup
\parskip0ex
\tableofcontents
\endgroup

%%%%%%%%%%%%%%%%%%%%%%%%%%%%%%%%%%%%%%%%%%%%%%%%%%%%%%%%%%%%%%%%%%%%%%%%%%%%%%%%
%%%%%%%%%%%%%%%%%%%%%%%%%%%%%%%%%%%%%%%%%%%%%%%%%%%%%%%%%%%%%%%%%%%%%%%%%%%%%%%%
\section{Introduction}

\LaTeX{} provides a mechanism to structure a large document (such as a book)
into a main file and several child files (containing the chapters)
using the |\include| command.
This mechanism is beneficial for documents
which span hundreds of pages in order to
make the source file(s) more manageable.
Moreover, compilation can be restricted to
selected child files by means of the |\includeonly| command.
The latter feature can be used to reduce the compilation time while editing
(this was significantly more useful in the earlier days of \LaTeX{})
or to generate a smaller document which is easier to navigate.
Another application of |\includeonly| is to generate
documents consisting of selected parts of the complete document.

However, there are a few drawbacks of the plain |\include| mechanism:
\begin{itemize}
\item
The child files cannot be compiled on their own,
they can only be compiled via the main file.
A naive editing environment
(such as a text editor with an option
to have the current file processed by \LaTeX)
may require one to switch to the main file before compiling;
attempting to compile the child file produces errors.
\item
The main file must be modified (each time)
to adjust the |\includeonly| command
to the present needs. This easily leaves the main file in a messy state.
\item
The generated document will always carry the filename
of the main document. This is inconvenient if
several child files are to be compiled and
to be kept for distribution.
\end{itemize}

The present package provides a simple interface
to make child files individually compilable by \LaTeX{}.
Compiling a child file then has the same effect as compiling
the main file with an |\includeonly| command
to select the appropriate child.
Moreover the generated document will carry the name of the child
rather than the main file.
This resolves all three above issues.

This feature is meant to make the editing of books,
thesis documents and lecture notes somewhat more convenient.
However, the package can also be used efficiently for
composing a series of documents (such as exercise sheets)
which are typically distributed individually.
It then assists the author in generating the individual documents
(potentially in different versions)
as well as a document containing the collected series.
Another application is in developing style files
or other kinds of included material
where compilation of the style file could redirect
to a sample or test file.

%%%%%%%%%%%%%%%%%%%%%%%%%%%%%%%%%%%%%%%%%%%%%%%%%%%%%%%%%%%%%%%%%%%%%%%%%%%%%%%%
%%%%%%%%%%%%%%%%%%%%%%%%%%%%%%%%%%%%%%%%%%%%%%%%%%%%%%%%%%%%%%%%%%%%%%%%%%%%%%%%
\section{Usage}

First of all, the package \textsf{childdoc} is \emph{not} a standard
\LaTeXe{} |.sty| style file! Therefore it needs to be invoked in
a non-standard way.

%%%%%%%%%%%%%%%%%%%%%%%%%%%%%%%%%%%%%%%%%%%%%%%%%%%%%%%%%%%%%%%%%%%%%%%%%%%%%%%%
\subsection{Included Files}
\label{sec:include}

%%%%%%%%%%%%%%%%%%%%%%%%%%%%%%%%%%%%%%%%
\DescribeMacro{\childdocmain}
To use the package, add the commands
\begin{center}
\begin{tabular}{l}
|\input{childdoc.def}|\\
|\childdocmain{}|\\
\end{tabular}
\end{center}
at the very top of the main \LaTeX{} file,
in particular \emph{before} the |\documentclass| statement!
The argument of |\childdocmain| should be left empty
(but it must be present).

%%%%%%%%%%%%%%%%%%%%%%%%%%%%%%%%%%%%%%%%
\DescribeMacro{\childdocof}
Furthermore, add the commands
\begin{center}
\begin{tabular}{l}
|\input{childdoc.def}|\\
|\childdocof{|\textit{main}|}|\\
\end{tabular}
\end{center}
at the top of every child file \textit{child}
which is included by |\include{|\textit{child}|}|
from within the main file
(or at least for those files to be compiled individually).
The argument \textit{main} must be the filename of the main file.

There are a couple of
considerations in setting up the main and child documents:

%%%%%%%%%%%%%%%%%%%%%%%%%%%%%%%%%%%%%%%%
\paragraph{Restrictions.}

Please note the following restrictions:
\begin{itemize}
\item
|\childdocmain| must be called with one argument \textit{main}
to ensure compatibility with earlier version of the package.
It must either be empty (|\childdocmain{}|)
or precisely match the filename of the main file in which it is specified.
See \secref{sec:detection} for further information.
\item
The filename \textit{main} must be specified without the |.tex| extension.
\item
The filename \textit{main} is case sensitive
(even in case-insensitive file systems)
due to internal string comparison.
\item
The argument \textit{main} should be fully expanded, it cannot be a macro.
\item
Subdirectories and special characters should be avoided in filenames.
\item
The command |\childdocmain{|\textit{main}|}| must be followed by a whitespace.
It should not be followed immediately by another command
or by a comment mark `|%|'.
This is because the \TeX{} parser reads the token immediately following
the argument of |\childdocmain| and puts it
at the beginning of every child section;
however, a white\-space is ignored.
\end{itemize}

%%%%%%%%%%%%%%%%%%%%%%%%%%%%%%%%%%%%%%%%
\paragraph{Content of Main File.}

It is advisable to place all content in the child files included by |\include|.
Any output contained in the main file will appear in all child documents
unless suppressed manually;
it cannot be suppressed automatically by the |\includeonly| directive
and thus should normally be avoided.
A method to include some content in the main file
by means of conditional processing is described in \secref{sec:conditional}.

%%%%%%%%%%%%%%%%%%%%%%%%%%%%%%%%%%%%%%%%
\paragraph{Page Numbering.}

When only a part of the document is compiled,
the appropriate numbering of pages
(as well as other status parameters)
is determined from the |.aux| files.
The latter contain information from previous passes.
However this information needs to propagate through
all intermediate child documents.
Therefore the page numbering in child documents may well
be inconsistent until the complete document is compiled at least once.

A useful (if unconventional) way to always ensure a consistent
page numbering is to restart the numbering in each child document
and denote the pages by `\textit{child}|.|\textit{page}'
where \textit{child} represents the chapter/section number of the child file.
This can be achieved by the command
|\numberwithin{page}{|\textit{child}|}|
of the \textsf{amsmath} package
where \textit{child} can be |chapter| or |section|
depending on the chosen structuring.
Alternatively, one can modify the macro |\thepage| appropriately
and reset the counter |page| at the start of each child file.

%%%%%%%%%%%%%%%%%%%%%%%%%%%%%%%%%%%%%%%%%%%%%%%%%%%%%%%%%%%%%%%%%%%%%%%%%%%%%%%%
\subsection{Conditional Processing}
\label{sec:conditional}

The package provides a mechanism to compile different versions
of a document. To customise the versions further some conditional processing
can come in handy to distinguish which version is being compiled.
The package provides two macros to describe the compilation context:

%%%%%%%%%%%%%%%%%%%%%%%%%%%%%%%%%%%%%%%%
\DescribeMacro{\ifchilddoc}
The conditional |\ifchilddoc| distinguishes between the compilation of
child documents and the main document:
%
\begin{center}
|\ifchilddoc |\textit{child-code}| |[|\||else |\textit{main-code}]| \||fi|
\end{center}

%%%%%%%%%%%%%%%%%%%%%%%%%%%%%%%%%%%%%%%%
\DescribeMacro{\childdocname}
\DescribeMacro{\childdocjob}
The macro |\childdocname| contains the filename (without extension)
of the main or child file being processed.
Note that |\childdocjob| will always contain the name of the main file.

%%%%%%%%%%%%%%%%%%%%%%%%%%%%%%%%%%%%%%%%
\paragraph{Title Page.}

Conditional processing can be used to include a title or banner page
in the main document when proper precautions are taken.
Importantly, the code in the main file should ensure that the page counter
(as well as other status parameters which are stored in the |.aux| files)
takes the same value after the conditional processing.
Otherwise the page numbers may take divergent values
depending on which part is compiled.

For example, a title page could be declared by:
%
\begin{center}
\begin{tabular}{l}
|\ifchilddoc\||else|\\
|\addtocounter{page}{-1}|\\
\textit{code for title page}\\
|\newpage|\\
|\||fi|
\end{tabular}
\end{center}
%
A banner page for the child documents can be generated by:
%
\begin{center}
\begin{tabular}{l}
|\ifchilddoc|\\
|\addtocounter{page}{-1}|\\
\textit{code for banner page}\\
|\newpage|\\
|\||fi|
\end{tabular}
\end{center}
%
Here one could write a message such as:
\begin{center}
|This is the part \childdocname{} of \childdocjob{}.|
\end{center}

%%%%%%%%%%%%%%%%%%%%%%%%%%%%%%%%%%%%%%%%%%%%%%%%%%%%%%%%%%%%%%%%%%%%%%%%%%%%%%%%
\subsection{Flags}
\label{sec:flags}

The package makes it easy to generate different versions
of the main or child documents.
To this end compilation flags can be defined
and assigned different default values.
They will be particularly useful in conjunction
with the forwarding mechanism described in \secref{sec:forward}.

For example, it may be useful to have a flag |\version|
which can be set to |draft| or |final|.
The document source will contain some conditional code
depending on the value of |\version|.
Suppose further, the flag should default to |final| for the main file
and to |draft| for child files
which is a natural assignment for editing the document.
This is achieved by placing the following code
in the preamble of the main document
(below the |\childdocmain| directive):
%
\begin{center}
\begin{tabular}{l}
|\ifchilddoc|\\
|\providecommand{\version}{draft}|\\
|\||else|\\
|\providecommand{\version}{final}|\\
|\||fi|
\end{tabular}
\end{center}
%
The definition by |\providecommand| makes sure
that previous definitions are not overwritten.
Further statements |\providecommand{\version}{...}|
can thus be added before the above code to override it.

For the main file, one might add a line
(between |\childdocmain| and the above block)
%
\begin{center}
|%\ifchilddoc\||else\providecommand{\version}{draft}\||fi|
\end{center}
%
which can be uncommented to produce a draft version.
Likewise one can add a line to the very top of a child file
(above the |\childdocof{|\textit{main}|}| directive)
%
\begin{center}
|%\providecommand{\version}{final}|
\end{center}
%
which can be uncommented to produce the final version of this child document.

%%%%%%%%%%%%%%%%%%%%%%%%%%%%%%%%%%%%%%%%%%%%%%%%%%%%%%%%%%%%%%%%%%%%%%%%%%%%%%%%
\subsection{Forwarding}
\label{sec:forward}

Different versions of the main or child documents
using compilation flags as described in \secref{sec:flags}
can be (permanently) stored in different files
for convenient compilation, viewing and distribution.
To this end, the package defines a command
to pass on compilation to a different file:

%%%%%%%%%%%%%%%%%%%%%%%%%%%%%%%%%%%%%%%%
\DescribeMacro{\childdocforward}
The command |\childdocforward| redirects processing to
another source file:
%
\begin{center}
\begin{tabular}{l}
|\input{childdoc.def}|\\
|\childdocforward[|\textit{main}|]{|\textit{dest}|}|\\
\end{tabular}
\end{center}
%
The argument \textit{dest} is the destination file
(without extension).
It should be the main file or one of the child files.
Note that further \textsf{childdoc} directives
such as |\childdocof| and |\childdocforward|
in the indicated file will be processed in this form.
The optional argument \textit{main}
passes on directly to the main file \textit{main}
while pretending to compile the child \textit{dest}.
This form behaves as if \textit{dest}
issues |\childdocof{|\textit{main}|}| right away,
and no further \textsf{childdoc} directives will be processed.

%%%%%%%%%%%%%%%%%%%%%%%%%%%%%%%%%%%%%%%%
\DescribeMacro{\...prefix}
In the alternative form |\childdocforwardprefix|,
%
\begin{center}
\begin{tabular}{l}
|\input{childdoc.def}|\\
|\childdocforwardprefix[|\textit{main}|]{|\textit{prefix}|}{|\textit{dest}|}|
\end{tabular}
\end{center}
%
the destination file is determined by a pattern
depending on the current file:
To make this work, the current file must be called
`{\textit{prefix}\hspace{0.2em}\textit{suffix}}'
with \textit{prefix} matching precisely the argument.
Processing is then passed on to the file
`{\textit{dest}\hspace{0.2em}\textit{suffix}}'.
Surely, the same effect is achieved by
directly specifying the
argument `{\textit{dest}\hspace{0.2em}\textit{suffix}}'
in the first form.
However, that requires to set up a different file
for each child. With the alternative form of the command
all these files can have exactly the same content
which simplifies setting them up and maintaining them.

For example, the following file |draft.tex|
with a compilation flag |\version| as described in \secref{sec:flags}
compiles the main document as a draft:
%
\begin{center}
\begin{tabular}{l}
|\def\version{draft}|\\
|\input{childdoc.def}|\\
|\childdocforward{|\textit{main}|}|
\end{tabular}
\end{center}
%
Likewise, the following files |final|\textit{nn}|.tex|
compile the final version of the child document
|child|\textit{nn}|.tex|:
%
\begin{center}
\begin{tabular}{l}
|\def\version{final}|\\
|\input{childdoc.def}|\\
|\childdocforwardprefix{final}{child}|
\end{tabular}
\end{center}
%

Note that when several versions of a main file and/or of each child file
are to be generated, it may be convenient to set up a |Makefile| or
shell script to automatise the process.

%%%%%%%%%%%%%%%%%%%%%%%%%%%%%%%%%%%%%%%%%%%%%%%%%%%%%%%%%%%%%%%%%%%%%%%%%%%%%%%%
\subsection{Command Line Processing}
\label{sec:commandline}

The effect of redirection files can also be achieved by invoking
the \LaTeX{} compiler with a more elaborate command line.
Most conveniently this should be done as part
of a shell script or a |Makefile|.

When using \textsf{childdoc} in the main file, the following
command lines effectively perform a redirection
(note that depending on the shell being used,
backslashes may have to be doubled: `|\|' $\to$ `|\\|'):
%
\begin{center}
|... -jobname "|\textit{target}|" |\\|"|[\textit{flags}]%
|\input{childdoc.def}\childdocforward[|\textit{main}|]{|\textit{dest}|}"|
\end{center}
%
Here \textit{target} is the name of the output file,
\textit{main} is the name of the main file
and \textit{dest} is the name of the main or child file to be processed
(all filenames without extensions).
The optional argument \textit{main} can be omitted
if \textit{main} matches \textit{dest}.
Optionally, compilation \textit{flags} can be defined via |\def| commands.
This command line makes the \TeX{} engine believe
it is compiling the file \textit{target}
whose content is specified as the latter parameter.
The provided code then forwards the processing to
\textit{main} or \textit{dest} as described in \secref{sec:forward}.

%%%%%%%%%%%%%%%%%%%%%%%%%%%%%%%%%%%%%%%%%%%%%%%%%%%%%%%%%%%%%%%%%%%%%%%%%%%%%%%%
\subsection{Include by Input}
\label{sec:input}

Including child documents by |\include| has some restrictions by design.
Most notably, the content of a child document always occupies
its own set of pages; pages cannot be shared between child documents.
Usually, this behaviour makes perfect sense
because each child document contain an essential part of the document.
However, in some situations it may be desirable to compose
a document from a collection of parts
without having mandatory page breaks between then.
For this case, the package
provides a mechanism to include parts
by |\input| which can also be processed individually.
However, by construction this mechanism
requires manual handling of the content to be output.

%%%%%%%%%%%%%%%%%%%%%%%%%%%%%%%%%%%%%%%%
\DescribeMacro{\ifchilddocmanual}
The main file should be prepared as usual, see \secref{sec:include}.
However, the document body must make a distinction
between processing of an individual part and of the main document, e.g.:
%
\begin{center}
\begin{tabular}{l}
|\ifchilddocmanual|\\
|\input{\childdocname}|\\
|\||else|\\
\textit{document body with }|\input{|\textit{part}|}|\\
|\||fi|
\end{tabular}
\end{center}
%
The conditional |\ifchilddocmanual| is true whenever
a part to be included by |\input| is being compiled,
and the name of the part is stored in |\childdocname|.

%%%%%%%%%%%%%%%%%%%%%%%%%%%%%%%%%%%%%%%%
\DescribeMacro{\childdocby}
Each part to be included by |\input| should start with:
%
\begin{center}
\begin{tabular}{l}
|\input{childdoc.def}|\\
|\childdocby{|\textit{main}|}|\\
\end{tabular}
\end{center}
%
The directive |\childdocby| is similar to |\childdocof|
described in \secref{sec:include},
but the subsequent selection of content must be done manually.
To that end, both |\ifchilddoc| and |\ifchilddocmanual|
will be true upon processing of a part,
and the name of the part is stored in |\childdocname|.
Note that |\jobname| will be set to the filename of the current part
so that each part receives an individual |.aux| file
that does not interfere with the |.aux| file(s) of the main document.
This behaviour can be altered by the alternative form
|\childdocby[*]{|\textit{main}|}| (with a non-empty optional argument)
which uses the |.aux| file of the main document
by setting |\jobname| to \textit{main}.

%%%%%%%%%%%%%%%%%%%%%%%%%%%%%%%%%%%%%%%%%%%%%%%%%%%%%%%%%%%%%%%%%%%%%%%%%%%%%%%%
\subsection{Driver Development}
\label{sec:driver}

The \textsf{childdoc} mechanism can also be use for the development
of definition files such as \LaTeX{} styles or classes.
This case differs from the above setup with multiple parts
included by |\include| in that no |\includeonly| should be invoked.
This can be achieved by starting the include file
(before |\ProvidesPackage|) with:
%
\begin{center}
\begin{tabular}{l}
|\input{childdoc.def}|\\
|\childdocforward{|\textit{main}|}|\\
\end{tabular}
\end{center}
%
or alternatively with:
%
\begin{center}
\begin{tabular}{l}
|\input{childdoc.def}|\\
|\childdocby{|\textit{main}|}|\\
\end{tabular}
\end{center}
%
Both forms have slightly different effects as described above.
The main file is prepared as usual, see \secref{sec:include}.

%%%%%%%%%%%%%%%%%%%%%%%%%%%%%%%%%%%%%%%%%%%%%%%%%%%%%%%%%%%%%%%%%%%%%%%%%%%%%%%%
\subsection{Legacy Detection}
\label{sec:detection}

The directive |\childdocmain| in the main file can detect
whether the complete document or merely a child is to be compiled
even without using the directive |\childdocof|.
This method is deprecated because it is less robust
and there is no compelling reason to use it;
it is merely provided for backward compatibility
and it may be removed in future versions.

If the detection mechanism is to be used,
it is mandatory to correctly specify
the filename of the main file as the argument of |\childdocmain|:
%
\begin{center}
\begin{tabular}{l}
|\input{childdoc.def}|\\
|\childdocmain{|\textit{main}|}|\\
\end{tabular}
\end{center}
%
If |\jobname| does not match the argument \textit{main} of |\childdocmain|,
it is assumed that |\jobname| points to the child file to be compiled.
When using |\childdocmain| with the main file specified as argument,
it suffices to start a child file
with just |\input{|\textit{main}|}|
without loading of the package and using |\childdocof|.
If instead all processing is done
with the appropriate \textsf{childdoc} directives,
the argument of \textit{main} of |\childdocmain| can be empty.

An alternative version of the command line processing described
in \secref{sec:commandline} using the detection mechanism reads:
%
\begin{center}
|... -jobname "|\textit{target}|" "|[\textit{flags}]%
[|\def\jobname{|\textit{dest}|}|]|\input{|\textit{main}|}"|
\end{center}

%%%%%%%%%%%%%%%%%%%%%%%%%%%%%%%%%%%%%%%%%%%%%%%%%%%%%%%%%%%%%%%%%%%%%%%%%%%%%%%%
\subsection{Manual Code}
\label{sec:manual}

In case one cannot be certain whether the definitions file |childdoc.def|
is installed on the target \TeX{} distribution
and one prefers not to ship it,
it is conceivable to paste a few relevant commands into the sources.

To that end, drop all statements |\input{childdoc.def}|
and perform the replacements as outlined below.
Instead of |\childdocmain{|\textit{main}|}| add the following code
to the top of the main file:
%
\begin{center}
\begin{tabular}{l}
|\||ifdefined\childdocname\endinput\||fi\newif\ifchilddoc|\\
|\edef\childdocname{\scantokens\expandafter{\jobname\noexpand}}|\\
|\def\childdocmain{|\textit{main}|}\||ifx\childdocmain\childdocname\||else|\\
|\childdoctrue\includeonly{\childdocname}\let\jobname\childdocmain\||fi|\\
\end{tabular}
\end{center}
%
Instead of |\childdocof{|\textit{main}|}| just include the main file
at the top of each child file:
%
\begin{center}
|\input{|\textit{main}|}|
\end{center}
%
A simple redirection |\childdocforward{|\textit{dest}|}| is achieved by:
%
\begin{center}
|\def\jobname{|\textit{dest}|}\input{\jobname}|
\end{center}
%
The redirection with prefix
|\childdocforwardprefix[|\textit{prefix}|]{|\textit{dest}|}|
is accomplished by:
%
\begin{center}
\begin{tabular}{l}
|{\edef\jobname{\scantokens\expandafter{\jobname\noexpand}}|\\
|\def\redirectjob |\textit{prefix}|#1~~~{\gdef\jobname{|\textit{dest}|#1}}|\\
|\expandafter\redirectjob\jobname~~~}\input{\jobname}|
\end{tabular}
\end{center}

In an alternative approach,
child documents can be compiled by a specific command line
without additional code or specific definitions:
%
\begin{center}
|... -jobname "|\textit{target}|" "|[\textit{flags}]%
|\includeonly{|\textit{dest}|}\input{|\textit{main}|}"|
\end{center}
%

%%%%%%%%%%%%%%%%%%%%%%%%%%%%%%%%%%%%%%%%%%%%%%%%%%%%%%%%%%%%%%%%%%%%%%%%%%%%%%%%
%%%%%%%%%%%%%%%%%%%%%%%%%%%%%%%%%%%%%%%%%%%%%%%%%%%%%%%%%%%%%%%%%%%%%%%%%%%%%%%%
\section{Information}

%%%%%%%%%%%%%%%%%%%%%%%%%%%%%%%%%%%%%%%%%%%%%%%%%%%%%%%%%%%%%%%%%%%%%%%%%%%%%%%%
\subsection{Copyright}

Copyright \copyright{} 2017--2018 Niklas Beisert

This work may be distributed and/or modified under the
conditions of the \LaTeX{} Project Public License, either version 1.3
of this license or (at your option) any later version.
The latest version of this license is in
  \url{http://www.latex-project.org/lppl.txt}
and version 1.3 or later is part of all distributions of \LaTeX{}
version 2005/12/01 or later.

This work has the LPPL maintenance status `maintained'.

The Current Maintainer of this work is Niklas Beisert.

This work consists of the files |README.txt|, |childdoc.ins| and |childdoc.dtx|
as well as the derived files |childdoc.def|, |cdocsamp.tex|
with |cdocsch1.tex|, |cdocsch2.tex|, |cdocspt3.tex|, |cdocspt4.tex|,
|cdocsdrf.tex|, |cdocsfn1.tex|, |cdocsfn2.tex|
as well as |childdoc.pdf|.

%%%%%%%%%%%%%%%%%%%%%%%%%%%%%%%%%%%%%%%%%%%%%%%%%%%%%%%%%%%%%%%%%%%%%%%%%%%%%%%%
\subsection{Files and Installation}

The package consists of the files:
%
\begin{center}
\begin{tabular}{ll}
    |README.txt|   & readme file \\
    |childdoc.ins| & installation file \\
    |childdoc.dtx| & source file \\
    |childdoc.def| & definition file \\
    |cdocsamp.tex| & sample main file \\
    |cdocsch1.tex| & sample include file \\
    |cdocsch2.tex| & sample include file \\
    |cdocspt3.tex| & sample part file \\
    |cdocspt4.tex| & sample part file \\
    |cdocsdrf.tex| & sample redirection file \\
    |cdocsfn1.tex| & sample redirection file \\
    |cdocsfn2.tex| & sample redirection file \\
    |childdoc.pdf| & manual
\end{tabular}
\end{center}
%
The distribution consists of the files
|README.txt|, |childdoc.ins| and |childdoc.dtx|.
%
\begin{itemize}
\item
Run (pdf)\LaTeX{} on |childdoc.dtx|
to compile the manual |childdoc.pdf| (this file).
\item
Run \LaTeX{} on |childdoc.ins| to create the definitions file |childdoc.def|
and the sample |cdocsamp.tex| with include files
|cdocsch1.tex|, |cdocsch2.tex|, |cdocspt3.tex|, |cdocspt4.tex|,
|cdocsdrf.tex|, |cdocsfn1.tex|, |cdocsfn2.tex|.
Then copy the file |childdoc.def| to an appropriate directory of your \LaTeX{}
distribution, e.g.\ \textit{texmf-root}|/tex/latex/childdoc|.
\end{itemize}

%%%%%%%%%%%%%%%%%%%%%%%%%%%%%%%%%%%%%%%%%%%%%%%%%%%%%%%%%%%%%%%%%%%%%%%%%%%%%%%%
\subsection{Related CTAN Packages}

There are several other packages which offer a similar functionality:
%
\begin{itemize}
\item
The packages
\href{http://ctan.org/pkg/docmute}{\textsf{docmute}},
\href{http://ctan.org/pkg/includex}{\textsf{includex}} and
\href{http://ctan.org/pkg/standalone}{\textsf{standalone}}
provide commands to include only the document body of
a child file thus allowing both files to be compiled individually.
\item
The packages \href{http://ctan.org/pkg/subdocs}{\textsf{subdocs}}
and \href{http://ctan.org/pkg/subfiles}{\textsf{subfiles}}
provide structures in which the main and child documents can be
encapsulated and allowing them to be compiled individually.
The inclusion mechanism is different from the conventional |\include|.
\item
The package \href{http://ctan.org/pkg/combine}{\textsf{combine}}
is an elaborate solution to combine several documents into one.
\end{itemize}
%
See also the CTAN topic \href{http://ctan.org/topic/subdocs}{\textsf{subdocs}}
for further related packages.
The present package differs from the above solutions in that
a document structure constructed with the conventional |\include| mechanism
just needs two extra commands at the top of every file
such that all constituent files can be compiled individually.

%%%%%%%%%%%%%%%%%%%%%%%%%%%%%%%%%%%%%%%%%%%%%%%%%%%%%%%%%%%%%%%%%%%%%%%%%%%%%%%%
%\subsection{Feature Suggestions}
%
%The following is a list of features which may be useful for future
%versions of this package:
%%
%\begin{itemize}
%\item
%\ldots
%\end{itemize}

%%%%%%%%%%%%%%%%%%%%%%%%%%%%%%%%%%%%%%%%%%%%%%%%%%%%%%%%%%%%%%%%%%%%%%%%%%%%%%%%
\subsection{Revision History}

%%%%%%%%%%%%%%%%%%%%%%%%%%%%%%%%%%%%%%%%
\paragraph{v2.0:} 2018/12/30

\begin{itemize}
\item
immediate forward processing
\item
added |\childdocby| mechanism
\item
manual restructured
\end{itemize}

%%%%%%%%%%%%%%%%%%%%%%%%%%%%%%%%%%%%%%%%
\paragraph{v1.6:} 2018/01/17

\begin{itemize}
\item
application for development of include files
\item
corrections to manual
\end{itemize}

%%%%%%%%%%%%%%%%%%%%%%%%%%%%%%%%%%%%%%%%
\paragraph{v1.5:} 2017/05/21

\begin{itemize}
\item
more complete structuring introduced
\item
|\childdocof| introduced
\item
|\childdoc| renamed to |\childdocmain|
\item
|\childredirect| renamed to |\childdocforward| and |\childdocforwardprefix|
and functionality expanded
\end{itemize}

%%%%%%%%%%%%%%%%%%%%%%%%%%%%%%%%%%%%%%%%
\paragraph{v1.0:} 2017/04/27

\begin{itemize}
\item
manual and install package
\item
first version published on CTAN
\end{itemize}

%%%%%%%%%%%%%%%%%%%%%%%%%%%%%%%%%%%%%%%%
\paragraph{v0.6:} 2017/04/26

\begin{itemize}
\item
redirection mechanism added
\end{itemize}

%%%%%%%%%%%%%%%%%%%%%%%%%%%%%%%%%%%%%%%%
\paragraph{v0.5:} 2017/04/26

\begin{itemize}
\item
functionality in definition file
\end{itemize}


%%%%%%%%%%%%%%%%%%%%%%%%%%%%%%%%%%%%%%%%%%%%%%%%%%%%%%%%%%%%%%%%%%%%%%%%%%%%%%%%
%%%%%%%%%%%%%%%%%%%%%%%%%%%%%%%%%%%%%%%%%%%%%%%%%%%%%%%%%%%%%%%%%%%%%%%%%%%%%%%%
%%%%%%%%%%%%%%%%%%%%%%%%%%%%%%%%%%%%%%%%%%%%%%%%%%%%%%%%%%%%%%%%%%%%%%%%%%%%%%%%
\appendix

\settowidth\MacroIndent{\rmfamily\scriptsize 000\ }

 \DocInput{childdoc.dtx}

\end{document}
%</driver>
% \fi
%
% %%%%%%%%%%%%%%%%%%%%%%%%%%%%%%%%%%%%%%%%%%%%%%%%%%%%%%%%%%%%%%%%%%%%%%%%%%%%%%
% %%%%%%%%%%%%%%%%%%%%%%%%%%%%%%%%%%%%%%%%%%%%%%%%%%%%%%%%%%%%%%%%%%%%%%%%%%%%%%
% \section{Sample}
%\iffalse
%<*samplemain>
%\fi
%
% The following presents a sample document
% with two chapters, two parts, a title page,
% a compile flag as well as three forwarding files to set the flag.
% It consists of eight |.tex| files:
% \begin{center}
% \begin{tabular}{ll}
% |cdocsamp.tex|&main file\\
% |cdocsch1.tex|&include file for chapter 1\\
% |cdocsch2.tex|&include file for chapter 2\\
% |cdocspt3.tex|&include file for part 3\\
% |cdocspt4.tex|&include file for part 4\\
% |cdocsdrf.tex|&forwarding file for main file in draft mode\\
% |cdocsfi1.tex|&forwarding file for final version of chapter 1\\
% |cdocsfi2.tex|&forwarding file for final version of chapter 2\\
% \end{tabular}
% \end{center}
% Each of the eight files can be compiled directly by the \LaTeX{} compiler.
%
% %%%%%%%%%%%%%%%%%%%%%%%%%%%%%%%%%%%%%%
% \paragraph{Main File.}
%
% The main file is called |cdocsamp.tex|.
%
% Load the \textsf{childdoc} definitions and
% declare the filename for the main document:
%    \begin{macrocode}
\input{childdoc.def}
\childdocmain{}
%    \end{macrocode}

% Optional override for |\version| flag:
%    \begin{macrocode}
%%\ifchilddoc\else\providecommand{\version}{draft}\fi
%    \end{macrocode}

% Define the default values for the |\version| flag
% (|final| for the main file and |draft| for childs):
%    \begin{macrocode}
\ifchilddoc
\providecommand{\version}{draft}
\else
\providecommand{\version}{final}
\fi
%    \end{macrocode}

% Load the standard document class:
%    \begin{macrocode}
\documentclass[12pt]{article}
%    \end{macrocode}

% Start the document body:
%    \begin{macrocode}
\begin{document}
%    \end{macrocode}

% Declare a title page.
% Print title, part of document being processed and version flag:
%    \begin{macrocode}
\addtocounter{page}{-1}
\begin{center}
{\LARGE\bfseries{}childdoc example\par}
\vspace{1cm}
\ifchilddoc
\ifchilddocmanual part\else chapter\fi:
`\childdocname' of `\childdocjob'\par
\else
main document: `\childdocjob'\par
\fi
version: \version\par
\end{center}
\newpage
%    \end{macrocode}

% Manually include selected file,
% otherwise process as usual:
%    \begin{macrocode}
\ifchilddocmanual
\section*{part `\childdocname'}
\input{\childdocname}
\else
%    \end{macrocode}

% Include the two chapters:
%    \begin{macrocode}
\include{cdocsch1}
\include{cdocsch2}
%    \end{macrocode}

% Include the two parts unless only chapters should be displayed:
%    \begin{macrocode}
\ifchilddoc\else
\section{part three}
\input{cdocspt3}
\section{part four}
\input{cdocspt4}
\fi
%    \end{macrocode}

% Process as usual until here:
%    \begin{macrocode}
\fi
%    \end{macrocode}

% End of document body:
%    \begin{macrocode}
\end{document}
%    \end{macrocode}
%\iffalse
%</samplemain>
%\fi
%
% %%%%%%%%%%%%%%%%%%%%%%%%%%%%%%%%%%%%%%
% \paragraph{Chapter Include Files.}
%
% The include files are called |cdocsch1.tex| and |cdocsch2.tex|.
%
%\iffalse
%<*samplechap1|samplechap2>
%\fi

% Optional override for |\version| flag:
%    \begin{macrocode}
%%\providecommand{\version}{final}
%    \end{macrocode}

% Include the main document:
%    \begin{macrocode}
\input{childdoc.def}
\childdocof{cdocsamp}
%    \end{macrocode}

%\iffalse
%</samplechap1|samplechap2>
%\fi
%
%\iffalse
%<*samplechap1>
%\fi
% Some text for chapter 1:
%    \begin{macrocode}
\section{one}
some text in chapter one
%    \end{macrocode}

%\iffalse
%</samplechap1>
%\fi
% Some text for chapter 2:
%\iffalse
%<*samplechap2>
%\fi
%    \begin{macrocode}
\section{two}
more text in chapter two
%    \end{macrocode}

%\iffalse
%</samplechap2>
%\fi
%
% %%%%%%%%%%%%%%%%%%%%%%%%%%%%%%%%%%%%%%
% \paragraph{Part Include Files.}
%
% The include files are called |cdocspt3.tex| and |cdocspt4.tex|.
%
%\iffalse
%<*samplepart3|samplepart4>
%\fi

% Optional override for |\version| flag:
%    \begin{macrocode}
%%\providecommand{\version}{final}
%    \end{macrocode}

% Include the main document:
%    \begin{macrocode}
\input{childdoc.def}
\childdocby{cdocsamp}
%    \end{macrocode}

%\iffalse
%</samplepart3|samplepart4>
%\fi
%
%\iffalse
%<*samplepart3>
%\fi
% Some text for part 3:
%    \begin{macrocode}
some text in part three
%    \end{macrocode}

%\iffalse
%</samplepart3>
%\fi
% Some text for part 4:
%\iffalse
%<*samplepart4>
%\fi
%    \begin{macrocode}
more text in part four
%    \end{macrocode}

%\iffalse
%</samplepart4>
%\fi
%
% %%%%%%%%%%%%%%%%%%%%%%%%%%%%%%%%%%%%%%
% \paragraph{Forwarding for a Complete Draft.}
%
% The following forwarding file |cdocsdrf.tex|
% compiles the main document in draft mode:
%\iffalse
%<*sampledraft>
%\fi
%    \begin{macrocode}
\def\version{draft}
\input{childdoc.def}
\childdocforward{cdocsamp}
%    \end{macrocode}

%\iffalse
%</sampledraft>
%\fi
%
% %%%%%%%%%%%%%%%%%%%%%%%%%%%%%%%%%%%%%%
% \paragraph{Forwarding for Final Version of the Chapters.}
%
% The following forwarding files |cdocsfn1.tex| and |cdocsfn2.tex|
% (with identical content)
% compile the final versions of the child documents
% |cdocsch1.tex| and |cdocsch2.tex|, respectively:
%\iffalse
%<*samplefinal>
%\fi
%    \begin{macrocode}
\def\version{final}
\input{childdoc.def}
\childdocforwardprefix[cdocsamp]{cdocsfn}{cdocsch}
%    \end{macrocode}

%\iffalse
%</samplefinal>
%\fi
%
% %%%%%%%%%%%%%%%%%%%%%%%%%%%%%%%%%%%%%%
% \paragraph{Command Line Processing.}
%
% The following three command lines generate the output files
% |cdocscld|, |cdocscl1| and |cdocscl2|
% which should be identical to
% |cdocsdrf|, |cdocsch1| and |cdocsfn2|, respectively:
% \begin{center}
% \begin{tabular}{l}
% |latex -jobname cdocscld \|\\
% |  "\def\version{draft}\input{childdoc.def}\childdocforward{cdocsamp}"|\\
% |latex -jobname cdocscl1 \|\\
% |  "\input{childdoc.def}\childdocforward[cdocsamp]{cdocsch1}"|\\
% |latex -jobname cdocscl2 \|\\
% |  "\def\version{final}\input{childdoc.def}\childdocforward{cdocsch2}"|
% \end{tabular}
% \end{center}
% Note that the trailing backslash on each first line
% merely continues the input to the second line
% (for convenient cut ant paste).
% Furthermore, the command |latex| can be replaced by any
% of its alternative versions such as |pdflatex|.
%
% %%%%%%%%%%%%%%%%%%%%%%%%%%%%%%%%%%%%%%%%%%%%%%%%%%%%%%%%%%%%%%%%%%%%%%%%%%%%%%
% %%%%%%%%%%%%%%%%%%%%%%%%%%%%%%%%%%%%%%%%%%%%%%%%%%%%%%%%%%%%%%%%%%%%%%%%%%%%%%
% \section{Implementation}
%\iffalse
%<*package>
%\fi
%
% This section describes the definitions file |childdoc.def|.

% The definitions cannot be loaded using |\usepackage| or |\RequirePackage|
% which has a mechanism to prevent loading a style file more than once.
% When loading the definitions by means of |\input|
% multiple instances have to be prevented manually:
%\iffalse
%This code needs to be before the `\ProvidesFile' directive
%which is defined at the beginning of this file.
%Therefore it is also placed there and commented out here.
%</package>
%<*discard>
%\fi
%    \begin{macrocode}
\ifdefined\childdocmain\endinput\fi
%    \end{macrocode}
%\iffalse
%</discard>
%<*package>
%\fi
%
% \macro{\ifchilddoc}
% \macro{\ifchilddocmanual}
% The conditional |\ifchilddoc| tells whether a
% child (true) or main (false) document is being compiled.
% The conditional |\ifchilddocmanual| tells whether
% the |\includeonly| mechanism is used (false) or
% the selection of child files must be performed manually (true).
% The definitions initialise to false:
%    \begin{macrocode}
\newif\ifchilddoc
\newif\ifchilddocmanual
%    \end{macrocode}

% \macro{\childdocname}
% \macro{\childdocjob}
% The macro |\childdocname| stores the name of the main document
% to be compiled. The macro |\childdocjob| stores the name of
% the document on which the \LaTeX{} compiler was originally invoked.
% The content of |\jobname| cannot be compared
% to filenames specified in the source due to different catcodes.
% The following code rescans |\jobname|, stores the result
% in |\childdocname| and saves a copy in |\childdocjob|:
%    \begin{macrocode}
\edef\childdocname{\scantokens\expandafter{\jobname\noexpand}}
\let\childdocjob\childdocname
%    \end{macrocode}

% \macro{\childdocdisable}
% The macro |\childdocdisable| prevents the main file
% from being processed more than once.
% At this stage, the main document command |\childdocmain|
% is assumed to be called once again where it should do nothing.
% Any subsequent call to it should prevent
% a secondary processing of the main document
% It overwrites the forwarding commands
% |\childdocof| and |\childdocforward|
% with empty macros to prevent further inclusions of the main document:
%    \begin{macrocode}
\newcommand{\childdocdisable}
{
  \renewcommand{\childdocmain}[1]{\renewcommand{\childdocmain}[1]{\endinput}}
  \renewcommand{\childdocof}[1]{}
  \renewcommand{\childdocby}[2][]{}
  \renewcommand{\childdocforward}[2][]{}
  \renewcommand{\childdocdisable}{}
}
%    \end{macrocode}

% \macro{\childdocmain}
% The macro |\childdocmain| is to be called at the top of the main file
% with nothing or the main filename (without extension) as argument.
% First, it breaks loops.
% If the argument is not empty and does not match |\childdocname|
% (which is set by the first inclusion of |childdoc.def|),
% |\ifchilddoc| is set to true, |\includeonly| is applied to the child file
% and |\jobname| is set to the main file
% (for proper handling of |.aux| files):
%    \begin{macrocode}
\newcommand{\childdocmain}[1]
{
  \childdocdisable\childdocmain{}
  \if?#1?\else
    \begingroup
      \def\childdoctmp{#1}
      \ifx\childdoctmp\childdocname
        \def\childdoctmp{}
      \else
        \def\childdoctmp
        {
          \childdoctrue
          \includeonly{\childdocname}
          \def\childdocjob{#1}
          \def\jobname{#1}
        }
      \fi
      \expandafter
    \endgroup
    \childdoctmp
  \fi
}
%    \end{macrocode}

% \macro{\childdocof}
% The command |\childdocof| redirects
% compilation to the main file |#1|.
%    \begin{macrocode}
\newcommand{\childdocof}[1]
{
  \childdocdisable
  \childdoctrue
  \includeonly{\childdocname}
  \def\jobname{#1}
  \def\childdocjob{#1}
  \input{#1}
}
%    \end{macrocode}

% \macro{\childdocby}
% The command |\childdocby| ....
%    \begin{macrocode}
\newcommand{\childdocby}[2][]
{
  \childdocdisable
  \childdoctrue
  \childdocmanualtrue
  \if?#1?\else
    \def\jobname{#2}
  \fi
  \def\childdocjob{#2}
  \input{#2}
  \endinput
}
%    \end{macrocode}

% \macro{\childdocforward}
% The command |\childdocforward| redirects
% compilation to the main file or
% (if the optional argument is given) a child file.
% Parameters are set as if the main file
% or a child file starting with |\childdocof| was compiled.
% Then compilation is handed over to the main file:
%    \begin{macrocode}
\newcommand{\childdocforward}[2][]
{
  \begingroup
    \if?#1?
      \def\childdoctmp
      {
        \def\childdocname{#2}
        \def\childdocjob{#2}
        \def\jobname{#2}
        \input{#2}
        \endinput
      }
    \else
      \def\childdoctmp
      {
        \childdocdisable
        \def\childdocname{#2}
        \childdoctrue
        \includeonly{#2}
        \def\childdocjob{#1}
        \def\jobname{#1}
        \input{#1}
        \endinput
      }
    \fi
    \expandafter
  \endgroup
  \childdoctmp
}
%    \end{macrocode}

% \macro{\childdocforwardprefix}
% The command |\childdocforwardprefix| redirects
% compilation to the main or a child file by means of a pattern.
% The prefix |#1| in the current filename is replaced by |#2|
% and the suffix of the current filename is kept
% (it is assumed that the filename does not contain the substring `|~~~|'
% which is used as a delimiter).
% Compilation is handed over to the new file by |\childdocforward|:
%    \begin{macrocode}
\newcommand{\childdocforwardprefix}[3][]
{
  \begingroup
    \def\childdocextract #2##1~~~{\def\childdoctmp{\childdocforward[#1]{#3##1}}}
    \expandafter\childdocextract\childdocname~~~
    \expandafter
  \endgroup
  \childdoctmp
}
%    \end{macrocode}

% \macro{\childdoc}
% The deprecated macro |\childdoc| is a legacy version of |\childdocmain|:
%    \begin{macrocode}
\newcommand{\childdoc}{\childdocmain}
%    \end{macrocode}

% \macro{\childdocredirect}
% The deprecated macro |\childdocredirect| is a legacy version
% of |\childdocforward| and |\childdocforwardprefix|:
%    \begin{macrocode}
\newcommand{\childdocredirect}[2][]
{
  \begingroup
    \if?#1?
      \def\childdoctmp{\childdocforward{#2}}
    \else
      \def\childdoctmp{\childdocforwardprefix{#1}{#2}}
    \fi
    \expandafter
  \endgroup
  \childdoctmp
}
%    \end{macrocode}

%\iffalse
%</package>
%\fi
%
\endinput
|\\
|\childdocby{|\textit{main}|}|\\
\end{tabular}
\end{center}
%
The directive |\childdocby| is similar to |\childdocof|
described in \secref{sec:include},
but the subsequent selection of content must be done manually.
To that end, both |\ifchilddoc| and |\ifchilddocmanual|
will be true upon processing of a part,
and the name of the part is stored in |\childdocname|.
Note that |\jobname| will be set to the filename of the current part
so that each part receives an individual |.aux| file
that does not interfere with the |.aux| file(s) of the main document.
This behaviour can be altered by the alternative form
|\childdocby[*]{|\textit{main}|}| (with a non-empty optional argument)
which uses the |.aux| file of the main document
by setting |\jobname| to \textit{main}.

%%%%%%%%%%%%%%%%%%%%%%%%%%%%%%%%%%%%%%%%%%%%%%%%%%%%%%%%%%%%%%%%%%%%%%%%%%%%%%%%
\subsection{Driver Development}
\label{sec:driver}

The \textsf{childdoc} mechanism can also be use for the development
of definition files such as \LaTeX{} styles or classes.
This case differs from the above setup with multiple parts
included by |\include| in that no |\includeonly| should be invoked.
This can be achieved by starting the include file
(before |\ProvidesPackage|) with:
%
\begin{center}
\begin{tabular}{l}
|% \iffalse
%
% childdoc.dtx Copyright (C) 2017-2018 Niklas Beisert
%
% This work may be distributed and/or modified under the
% conditions of the LaTeX Project Public License, either version 1.3
% of this license or (at your option) any later version.
% The latest version of this license is in
%   http://www.latex-project.org/lppl.txt
% and version 1.3 or later is part of all distributions of LaTeX
% version 2005/12/01 or later.
%
% This work has the LPPL maintenance status `maintained'.
%
% The Current Maintainer of this work is Niklas Beisert.
%
% This work consists of the files childdoc.dtx and childdoc.ins
% and the derived files childdoc.def and cdocsamp.tex with
% cdocsch1.tex, cdocsch2.tex, cdocsdrf.tex, cdocsfn1.tex, cdocsfn2.tex.
%
%<package>\ifdefined\childdocmain\endinput\fi
%<package>\ProvidesFile{childdoc.def}[2018/12/30 v2.0 child document driver]
%<samplemain>\ProvidesFile{cdocsamp.tex}[2018/12/30 v2.0 sample for childdoc]
%<*driver>
%\ProvidesFile{childdoc.drv}[2018/12/30 v2.0 childdoc reference manual file]
\PassOptionsToClass{10pt,a4paper}{article}
\documentclass{ltxdoc}

\usepackage[margin=35mm]{geometry}
\usepackage{hyperref}
\usepackage{hyperxmp}
\usepackage[usenames]{color}

\hypersetup{colorlinks=true}
\hypersetup{pdfstartview=FitH}
\hypersetup{pdfpagemode=UseNone}
\hypersetup{pdfsource={}}
\hypersetup{pdflang={en-UK}}
\hypersetup{pdfcopyright={Copyright 2017-2018 Niklas Beisert.
  This work may be distributed and/or modified under the
  conditions of the LaTeX Project Public License, either version 1.3
  of this license or (at your option) any later version.}}
\hypersetup{pdflicenseurl={http://www.latex-project.org/lppl.txt}}
\hypersetup{pdfcontactaddress={ETH Zurich, ITP, HIT K,
  Wolfgang-Pauli-Strasse 27}}
\hypersetup{pdfcontactpostcode={8093}}
\hypersetup{pdfcontactcity={Zurich}}
\hypersetup{pdfcontactcountry={Switzerland}}
\hypersetup{pdfcontactemail={nbeisert@itp.phys.ethz.ch}}
\hypersetup{pdfcontacturl={http://people.phys.ethz.ch/\xmptilde nbeisert/}}

\newcommand{\secref}[1]{\hyperref[#1]{section \ref*{#1}}}

\parskip1ex
\parindent0pt
\let\olditemize\itemize
\def\itemize{\olditemize\parskip0pt}

\begin{document}

\title{The \textsf{childdoc} Package}
\hypersetup{pdftitle={The childdoc Package}}
\author{Niklas Beisert\\[2ex]
  Institut f\"ur Theoretische Physik\\
  Eidgen\"ossische Technische Hochschule Z\"urich\\
  Wolfgang-Pauli-Strasse 27, 8093 Z\"urich, Switzerland\\[1ex]
  \href{mailto:nbeisert@itp.phys.ethz.ch}
  {\texttt{nbeisert@itp.phys.ethz.ch}}}
\hypersetup{pdfauthor={Niklas Beisert}}
\hypersetup{pdfsubject={Manual for the LaTeX2e Package childdoc}}
\date{30 December 2018, \textsf{v2.0}}
\maketitle

\begin{abstract}\noindent
\textsf{childdoc} is a \LaTeXe{} package
that enables the direct compilation
of document sections included by |\include|
to individual files.
\end{abstract}

\begingroup
\parskip0ex
\tableofcontents
\endgroup

%%%%%%%%%%%%%%%%%%%%%%%%%%%%%%%%%%%%%%%%%%%%%%%%%%%%%%%%%%%%%%%%%%%%%%%%%%%%%%%%
%%%%%%%%%%%%%%%%%%%%%%%%%%%%%%%%%%%%%%%%%%%%%%%%%%%%%%%%%%%%%%%%%%%%%%%%%%%%%%%%
\section{Introduction}

\LaTeX{} provides a mechanism to structure a large document (such as a book)
into a main file and several child files (containing the chapters)
using the |\include| command.
This mechanism is beneficial for documents
which span hundreds of pages in order to
make the source file(s) more manageable.
Moreover, compilation can be restricted to
selected child files by means of the |\includeonly| command.
The latter feature can be used to reduce the compilation time while editing
(this was significantly more useful in the earlier days of \LaTeX{})
or to generate a smaller document which is easier to navigate.
Another application of |\includeonly| is to generate
documents consisting of selected parts of the complete document.

However, there are a few drawbacks of the plain |\include| mechanism:
\begin{itemize}
\item
The child files cannot be compiled on their own,
they can only be compiled via the main file.
A naive editing environment
(such as a text editor with an option
to have the current file processed by \LaTeX)
may require one to switch to the main file before compiling;
attempting to compile the child file produces errors.
\item
The main file must be modified (each time)
to adjust the |\includeonly| command
to the present needs. This easily leaves the main file in a messy state.
\item
The generated document will always carry the filename
of the main document. This is inconvenient if
several child files are to be compiled and
to be kept for distribution.
\end{itemize}

The present package provides a simple interface
to make child files individually compilable by \LaTeX{}.
Compiling a child file then has the same effect as compiling
the main file with an |\includeonly| command
to select the appropriate child.
Moreover the generated document will carry the name of the child
rather than the main file.
This resolves all three above issues.

This feature is meant to make the editing of books,
thesis documents and lecture notes somewhat more convenient.
However, the package can also be used efficiently for
composing a series of documents (such as exercise sheets)
which are typically distributed individually.
It then assists the author in generating the individual documents
(potentially in different versions)
as well as a document containing the collected series.
Another application is in developing style files
or other kinds of included material
where compilation of the style file could redirect
to a sample or test file.

%%%%%%%%%%%%%%%%%%%%%%%%%%%%%%%%%%%%%%%%%%%%%%%%%%%%%%%%%%%%%%%%%%%%%%%%%%%%%%%%
%%%%%%%%%%%%%%%%%%%%%%%%%%%%%%%%%%%%%%%%%%%%%%%%%%%%%%%%%%%%%%%%%%%%%%%%%%%%%%%%
\section{Usage}

First of all, the package \textsf{childdoc} is \emph{not} a standard
\LaTeXe{} |.sty| style file! Therefore it needs to be invoked in
a non-standard way.

%%%%%%%%%%%%%%%%%%%%%%%%%%%%%%%%%%%%%%%%%%%%%%%%%%%%%%%%%%%%%%%%%%%%%%%%%%%%%%%%
\subsection{Included Files}
\label{sec:include}

%%%%%%%%%%%%%%%%%%%%%%%%%%%%%%%%%%%%%%%%
\DescribeMacro{\childdocmain}
To use the package, add the commands
\begin{center}
\begin{tabular}{l}
|\input{childdoc.def}|\\
|\childdocmain{}|\\
\end{tabular}
\end{center}
at the very top of the main \LaTeX{} file,
in particular \emph{before} the |\documentclass| statement!
The argument of |\childdocmain| should be left empty
(but it must be present).

%%%%%%%%%%%%%%%%%%%%%%%%%%%%%%%%%%%%%%%%
\DescribeMacro{\childdocof}
Furthermore, add the commands
\begin{center}
\begin{tabular}{l}
|\input{childdoc.def}|\\
|\childdocof{|\textit{main}|}|\\
\end{tabular}
\end{center}
at the top of every child file \textit{child}
which is included by |\include{|\textit{child}|}|
from within the main file
(or at least for those files to be compiled individually).
The argument \textit{main} must be the filename of the main file.

There are a couple of
considerations in setting up the main and child documents:

%%%%%%%%%%%%%%%%%%%%%%%%%%%%%%%%%%%%%%%%
\paragraph{Restrictions.}

Please note the following restrictions:
\begin{itemize}
\item
|\childdocmain| must be called with one argument \textit{main}
to ensure compatibility with earlier version of the package.
It must either be empty (|\childdocmain{}|)
or precisely match the filename of the main file in which it is specified.
See \secref{sec:detection} for further information.
\item
The filename \textit{main} must be specified without the |.tex| extension.
\item
The filename \textit{main} is case sensitive
(even in case-insensitive file systems)
due to internal string comparison.
\item
The argument \textit{main} should be fully expanded, it cannot be a macro.
\item
Subdirectories and special characters should be avoided in filenames.
\item
The command |\childdocmain{|\textit{main}|}| must be followed by a whitespace.
It should not be followed immediately by another command
or by a comment mark `|%|'.
This is because the \TeX{} parser reads the token immediately following
the argument of |\childdocmain| and puts it
at the beginning of every child section;
however, a white\-space is ignored.
\end{itemize}

%%%%%%%%%%%%%%%%%%%%%%%%%%%%%%%%%%%%%%%%
\paragraph{Content of Main File.}

It is advisable to place all content in the child files included by |\include|.
Any output contained in the main file will appear in all child documents
unless suppressed manually;
it cannot be suppressed automatically by the |\includeonly| directive
and thus should normally be avoided.
A method to include some content in the main file
by means of conditional processing is described in \secref{sec:conditional}.

%%%%%%%%%%%%%%%%%%%%%%%%%%%%%%%%%%%%%%%%
\paragraph{Page Numbering.}

When only a part of the document is compiled,
the appropriate numbering of pages
(as well as other status parameters)
is determined from the |.aux| files.
The latter contain information from previous passes.
However this information needs to propagate through
all intermediate child documents.
Therefore the page numbering in child documents may well
be inconsistent until the complete document is compiled at least once.

A useful (if unconventional) way to always ensure a consistent
page numbering is to restart the numbering in each child document
and denote the pages by `\textit{child}|.|\textit{page}'
where \textit{child} represents the chapter/section number of the child file.
This can be achieved by the command
|\numberwithin{page}{|\textit{child}|}|
of the \textsf{amsmath} package
where \textit{child} can be |chapter| or |section|
depending on the chosen structuring.
Alternatively, one can modify the macro |\thepage| appropriately
and reset the counter |page| at the start of each child file.

%%%%%%%%%%%%%%%%%%%%%%%%%%%%%%%%%%%%%%%%%%%%%%%%%%%%%%%%%%%%%%%%%%%%%%%%%%%%%%%%
\subsection{Conditional Processing}
\label{sec:conditional}

The package provides a mechanism to compile different versions
of a document. To customise the versions further some conditional processing
can come in handy to distinguish which version is being compiled.
The package provides two macros to describe the compilation context:

%%%%%%%%%%%%%%%%%%%%%%%%%%%%%%%%%%%%%%%%
\DescribeMacro{\ifchilddoc}
The conditional |\ifchilddoc| distinguishes between the compilation of
child documents and the main document:
%
\begin{center}
|\ifchilddoc |\textit{child-code}| |[|\||else |\textit{main-code}]| \||fi|
\end{center}

%%%%%%%%%%%%%%%%%%%%%%%%%%%%%%%%%%%%%%%%
\DescribeMacro{\childdocname}
\DescribeMacro{\childdocjob}
The macro |\childdocname| contains the filename (without extension)
of the main or child file being processed.
Note that |\childdocjob| will always contain the name of the main file.

%%%%%%%%%%%%%%%%%%%%%%%%%%%%%%%%%%%%%%%%
\paragraph{Title Page.}

Conditional processing can be used to include a title or banner page
in the main document when proper precautions are taken.
Importantly, the code in the main file should ensure that the page counter
(as well as other status parameters which are stored in the |.aux| files)
takes the same value after the conditional processing.
Otherwise the page numbers may take divergent values
depending on which part is compiled.

For example, a title page could be declared by:
%
\begin{center}
\begin{tabular}{l}
|\ifchilddoc\||else|\\
|\addtocounter{page}{-1}|\\
\textit{code for title page}\\
|\newpage|\\
|\||fi|
\end{tabular}
\end{center}
%
A banner page for the child documents can be generated by:
%
\begin{center}
\begin{tabular}{l}
|\ifchilddoc|\\
|\addtocounter{page}{-1}|\\
\textit{code for banner page}\\
|\newpage|\\
|\||fi|
\end{tabular}
\end{center}
%
Here one could write a message such as:
\begin{center}
|This is the part \childdocname{} of \childdocjob{}.|
\end{center}

%%%%%%%%%%%%%%%%%%%%%%%%%%%%%%%%%%%%%%%%%%%%%%%%%%%%%%%%%%%%%%%%%%%%%%%%%%%%%%%%
\subsection{Flags}
\label{sec:flags}

The package makes it easy to generate different versions
of the main or child documents.
To this end compilation flags can be defined
and assigned different default values.
They will be particularly useful in conjunction
with the forwarding mechanism described in \secref{sec:forward}.

For example, it may be useful to have a flag |\version|
which can be set to |draft| or |final|.
The document source will contain some conditional code
depending on the value of |\version|.
Suppose further, the flag should default to |final| for the main file
and to |draft| for child files
which is a natural assignment for editing the document.
This is achieved by placing the following code
in the preamble of the main document
(below the |\childdocmain| directive):
%
\begin{center}
\begin{tabular}{l}
|\ifchilddoc|\\
|\providecommand{\version}{draft}|\\
|\||else|\\
|\providecommand{\version}{final}|\\
|\||fi|
\end{tabular}
\end{center}
%
The definition by |\providecommand| makes sure
that previous definitions are not overwritten.
Further statements |\providecommand{\version}{...}|
can thus be added before the above code to override it.

For the main file, one might add a line
(between |\childdocmain| and the above block)
%
\begin{center}
|%\ifchilddoc\||else\providecommand{\version}{draft}\||fi|
\end{center}
%
which can be uncommented to produce a draft version.
Likewise one can add a line to the very top of a child file
(above the |\childdocof{|\textit{main}|}| directive)
%
\begin{center}
|%\providecommand{\version}{final}|
\end{center}
%
which can be uncommented to produce the final version of this child document.

%%%%%%%%%%%%%%%%%%%%%%%%%%%%%%%%%%%%%%%%%%%%%%%%%%%%%%%%%%%%%%%%%%%%%%%%%%%%%%%%
\subsection{Forwarding}
\label{sec:forward}

Different versions of the main or child documents
using compilation flags as described in \secref{sec:flags}
can be (permanently) stored in different files
for convenient compilation, viewing and distribution.
To this end, the package defines a command
to pass on compilation to a different file:

%%%%%%%%%%%%%%%%%%%%%%%%%%%%%%%%%%%%%%%%
\DescribeMacro{\childdocforward}
The command |\childdocforward| redirects processing to
another source file:
%
\begin{center}
\begin{tabular}{l}
|\input{childdoc.def}|\\
|\childdocforward[|\textit{main}|]{|\textit{dest}|}|\\
\end{tabular}
\end{center}
%
The argument \textit{dest} is the destination file
(without extension).
It should be the main file or one of the child files.
Note that further \textsf{childdoc} directives
such as |\childdocof| and |\childdocforward|
in the indicated file will be processed in this form.
The optional argument \textit{main}
passes on directly to the main file \textit{main}
while pretending to compile the child \textit{dest}.
This form behaves as if \textit{dest}
issues |\childdocof{|\textit{main}|}| right away,
and no further \textsf{childdoc} directives will be processed.

%%%%%%%%%%%%%%%%%%%%%%%%%%%%%%%%%%%%%%%%
\DescribeMacro{\...prefix}
In the alternative form |\childdocforwardprefix|,
%
\begin{center}
\begin{tabular}{l}
|\input{childdoc.def}|\\
|\childdocforwardprefix[|\textit{main}|]{|\textit{prefix}|}{|\textit{dest}|}|
\end{tabular}
\end{center}
%
the destination file is determined by a pattern
depending on the current file:
To make this work, the current file must be called
`{\textit{prefix}\hspace{0.2em}\textit{suffix}}'
with \textit{prefix} matching precisely the argument.
Processing is then passed on to the file
`{\textit{dest}\hspace{0.2em}\textit{suffix}}'.
Surely, the same effect is achieved by
directly specifying the
argument `{\textit{dest}\hspace{0.2em}\textit{suffix}}'
in the first form.
However, that requires to set up a different file
for each child. With the alternative form of the command
all these files can have exactly the same content
which simplifies setting them up and maintaining them.

For example, the following file |draft.tex|
with a compilation flag |\version| as described in \secref{sec:flags}
compiles the main document as a draft:
%
\begin{center}
\begin{tabular}{l}
|\def\version{draft}|\\
|\input{childdoc.def}|\\
|\childdocforward{|\textit{main}|}|
\end{tabular}
\end{center}
%
Likewise, the following files |final|\textit{nn}|.tex|
compile the final version of the child document
|child|\textit{nn}|.tex|:
%
\begin{center}
\begin{tabular}{l}
|\def\version{final}|\\
|\input{childdoc.def}|\\
|\childdocforwardprefix{final}{child}|
\end{tabular}
\end{center}
%

Note that when several versions of a main file and/or of each child file
are to be generated, it may be convenient to set up a |Makefile| or
shell script to automatise the process.

%%%%%%%%%%%%%%%%%%%%%%%%%%%%%%%%%%%%%%%%%%%%%%%%%%%%%%%%%%%%%%%%%%%%%%%%%%%%%%%%
\subsection{Command Line Processing}
\label{sec:commandline}

The effect of redirection files can also be achieved by invoking
the \LaTeX{} compiler with a more elaborate command line.
Most conveniently this should be done as part
of a shell script or a |Makefile|.

When using \textsf{childdoc} in the main file, the following
command lines effectively perform a redirection
(note that depending on the shell being used,
backslashes may have to be doubled: `|\|' $\to$ `|\\|'):
%
\begin{center}
|... -jobname "|\textit{target}|" |\\|"|[\textit{flags}]%
|\input{childdoc.def}\childdocforward[|\textit{main}|]{|\textit{dest}|}"|
\end{center}
%
Here \textit{target} is the name of the output file,
\textit{main} is the name of the main file
and \textit{dest} is the name of the main or child file to be processed
(all filenames without extensions).
The optional argument \textit{main} can be omitted
if \textit{main} matches \textit{dest}.
Optionally, compilation \textit{flags} can be defined via |\def| commands.
This command line makes the \TeX{} engine believe
it is compiling the file \textit{target}
whose content is specified as the latter parameter.
The provided code then forwards the processing to
\textit{main} or \textit{dest} as described in \secref{sec:forward}.

%%%%%%%%%%%%%%%%%%%%%%%%%%%%%%%%%%%%%%%%%%%%%%%%%%%%%%%%%%%%%%%%%%%%%%%%%%%%%%%%
\subsection{Include by Input}
\label{sec:input}

Including child documents by |\include| has some restrictions by design.
Most notably, the content of a child document always occupies
its own set of pages; pages cannot be shared between child documents.
Usually, this behaviour makes perfect sense
because each child document contain an essential part of the document.
However, in some situations it may be desirable to compose
a document from a collection of parts
without having mandatory page breaks between then.
For this case, the package
provides a mechanism to include parts
by |\input| which can also be processed individually.
However, by construction this mechanism
requires manual handling of the content to be output.

%%%%%%%%%%%%%%%%%%%%%%%%%%%%%%%%%%%%%%%%
\DescribeMacro{\ifchilddocmanual}
The main file should be prepared as usual, see \secref{sec:include}.
However, the document body must make a distinction
between processing of an individual part and of the main document, e.g.:
%
\begin{center}
\begin{tabular}{l}
|\ifchilddocmanual|\\
|\input{\childdocname}|\\
|\||else|\\
\textit{document body with }|\input{|\textit{part}|}|\\
|\||fi|
\end{tabular}
\end{center}
%
The conditional |\ifchilddocmanual| is true whenever
a part to be included by |\input| is being compiled,
and the name of the part is stored in |\childdocname|.

%%%%%%%%%%%%%%%%%%%%%%%%%%%%%%%%%%%%%%%%
\DescribeMacro{\childdocby}
Each part to be included by |\input| should start with:
%
\begin{center}
\begin{tabular}{l}
|\input{childdoc.def}|\\
|\childdocby{|\textit{main}|}|\\
\end{tabular}
\end{center}
%
The directive |\childdocby| is similar to |\childdocof|
described in \secref{sec:include},
but the subsequent selection of content must be done manually.
To that end, both |\ifchilddoc| and |\ifchilddocmanual|
will be true upon processing of a part,
and the name of the part is stored in |\childdocname|.
Note that |\jobname| will be set to the filename of the current part
so that each part receives an individual |.aux| file
that does not interfere with the |.aux| file(s) of the main document.
This behaviour can be altered by the alternative form
|\childdocby[*]{|\textit{main}|}| (with a non-empty optional argument)
which uses the |.aux| file of the main document
by setting |\jobname| to \textit{main}.

%%%%%%%%%%%%%%%%%%%%%%%%%%%%%%%%%%%%%%%%%%%%%%%%%%%%%%%%%%%%%%%%%%%%%%%%%%%%%%%%
\subsection{Driver Development}
\label{sec:driver}

The \textsf{childdoc} mechanism can also be use for the development
of definition files such as \LaTeX{} styles or classes.
This case differs from the above setup with multiple parts
included by |\include| in that no |\includeonly| should be invoked.
This can be achieved by starting the include file
(before |\ProvidesPackage|) with:
%
\begin{center}
\begin{tabular}{l}
|\input{childdoc.def}|\\
|\childdocforward{|\textit{main}|}|\\
\end{tabular}
\end{center}
%
or alternatively with:
%
\begin{center}
\begin{tabular}{l}
|\input{childdoc.def}|\\
|\childdocby{|\textit{main}|}|\\
\end{tabular}
\end{center}
%
Both forms have slightly different effects as described above.
The main file is prepared as usual, see \secref{sec:include}.

%%%%%%%%%%%%%%%%%%%%%%%%%%%%%%%%%%%%%%%%%%%%%%%%%%%%%%%%%%%%%%%%%%%%%%%%%%%%%%%%
\subsection{Legacy Detection}
\label{sec:detection}

The directive |\childdocmain| in the main file can detect
whether the complete document or merely a child is to be compiled
even without using the directive |\childdocof|.
This method is deprecated because it is less robust
and there is no compelling reason to use it;
it is merely provided for backward compatibility
and it may be removed in future versions.

If the detection mechanism is to be used,
it is mandatory to correctly specify
the filename of the main file as the argument of |\childdocmain|:
%
\begin{center}
\begin{tabular}{l}
|\input{childdoc.def}|\\
|\childdocmain{|\textit{main}|}|\\
\end{tabular}
\end{center}
%
If |\jobname| does not match the argument \textit{main} of |\childdocmain|,
it is assumed that |\jobname| points to the child file to be compiled.
When using |\childdocmain| with the main file specified as argument,
it suffices to start a child file
with just |\input{|\textit{main}|}|
without loading of the package and using |\childdocof|.
If instead all processing is done
with the appropriate \textsf{childdoc} directives,
the argument of \textit{main} of |\childdocmain| can be empty.

An alternative version of the command line processing described
in \secref{sec:commandline} using the detection mechanism reads:
%
\begin{center}
|... -jobname "|\textit{target}|" "|[\textit{flags}]%
[|\def\jobname{|\textit{dest}|}|]|\input{|\textit{main}|}"|
\end{center}

%%%%%%%%%%%%%%%%%%%%%%%%%%%%%%%%%%%%%%%%%%%%%%%%%%%%%%%%%%%%%%%%%%%%%%%%%%%%%%%%
\subsection{Manual Code}
\label{sec:manual}

In case one cannot be certain whether the definitions file |childdoc.def|
is installed on the target \TeX{} distribution
and one prefers not to ship it,
it is conceivable to paste a few relevant commands into the sources.

To that end, drop all statements |\input{childdoc.def}|
and perform the replacements as outlined below.
Instead of |\childdocmain{|\textit{main}|}| add the following code
to the top of the main file:
%
\begin{center}
\begin{tabular}{l}
|\||ifdefined\childdocname\endinput\||fi\newif\ifchilddoc|\\
|\edef\childdocname{\scantokens\expandafter{\jobname\noexpand}}|\\
|\def\childdocmain{|\textit{main}|}\||ifx\childdocmain\childdocname\||else|\\
|\childdoctrue\includeonly{\childdocname}\let\jobname\childdocmain\||fi|\\
\end{tabular}
\end{center}
%
Instead of |\childdocof{|\textit{main}|}| just include the main file
at the top of each child file:
%
\begin{center}
|\input{|\textit{main}|}|
\end{center}
%
A simple redirection |\childdocforward{|\textit{dest}|}| is achieved by:
%
\begin{center}
|\def\jobname{|\textit{dest}|}\input{\jobname}|
\end{center}
%
The redirection with prefix
|\childdocforwardprefix[|\textit{prefix}|]{|\textit{dest}|}|
is accomplished by:
%
\begin{center}
\begin{tabular}{l}
|{\edef\jobname{\scantokens\expandafter{\jobname\noexpand}}|\\
|\def\redirectjob |\textit{prefix}|#1~~~{\gdef\jobname{|\textit{dest}|#1}}|\\
|\expandafter\redirectjob\jobname~~~}\input{\jobname}|
\end{tabular}
\end{center}

In an alternative approach,
child documents can be compiled by a specific command line
without additional code or specific definitions:
%
\begin{center}
|... -jobname "|\textit{target}|" "|[\textit{flags}]%
|\includeonly{|\textit{dest}|}\input{|\textit{main}|}"|
\end{center}
%

%%%%%%%%%%%%%%%%%%%%%%%%%%%%%%%%%%%%%%%%%%%%%%%%%%%%%%%%%%%%%%%%%%%%%%%%%%%%%%%%
%%%%%%%%%%%%%%%%%%%%%%%%%%%%%%%%%%%%%%%%%%%%%%%%%%%%%%%%%%%%%%%%%%%%%%%%%%%%%%%%
\section{Information}

%%%%%%%%%%%%%%%%%%%%%%%%%%%%%%%%%%%%%%%%%%%%%%%%%%%%%%%%%%%%%%%%%%%%%%%%%%%%%%%%
\subsection{Copyright}

Copyright \copyright{} 2017--2018 Niklas Beisert

This work may be distributed and/or modified under the
conditions of the \LaTeX{} Project Public License, either version 1.3
of this license or (at your option) any later version.
The latest version of this license is in
  \url{http://www.latex-project.org/lppl.txt}
and version 1.3 or later is part of all distributions of \LaTeX{}
version 2005/12/01 or later.

This work has the LPPL maintenance status `maintained'.

The Current Maintainer of this work is Niklas Beisert.

This work consists of the files |README.txt|, |childdoc.ins| and |childdoc.dtx|
as well as the derived files |childdoc.def|, |cdocsamp.tex|
with |cdocsch1.tex|, |cdocsch2.tex|, |cdocspt3.tex|, |cdocspt4.tex|,
|cdocsdrf.tex|, |cdocsfn1.tex|, |cdocsfn2.tex|
as well as |childdoc.pdf|.

%%%%%%%%%%%%%%%%%%%%%%%%%%%%%%%%%%%%%%%%%%%%%%%%%%%%%%%%%%%%%%%%%%%%%%%%%%%%%%%%
\subsection{Files and Installation}

The package consists of the files:
%
\begin{center}
\begin{tabular}{ll}
    |README.txt|   & readme file \\
    |childdoc.ins| & installation file \\
    |childdoc.dtx| & source file \\
    |childdoc.def| & definition file \\
    |cdocsamp.tex| & sample main file \\
    |cdocsch1.tex| & sample include file \\
    |cdocsch2.tex| & sample include file \\
    |cdocspt3.tex| & sample part file \\
    |cdocspt4.tex| & sample part file \\
    |cdocsdrf.tex| & sample redirection file \\
    |cdocsfn1.tex| & sample redirection file \\
    |cdocsfn2.tex| & sample redirection file \\
    |childdoc.pdf| & manual
\end{tabular}
\end{center}
%
The distribution consists of the files
|README.txt|, |childdoc.ins| and |childdoc.dtx|.
%
\begin{itemize}
\item
Run (pdf)\LaTeX{} on |childdoc.dtx|
to compile the manual |childdoc.pdf| (this file).
\item
Run \LaTeX{} on |childdoc.ins| to create the definitions file |childdoc.def|
and the sample |cdocsamp.tex| with include files
|cdocsch1.tex|, |cdocsch2.tex|, |cdocspt3.tex|, |cdocspt4.tex|,
|cdocsdrf.tex|, |cdocsfn1.tex|, |cdocsfn2.tex|.
Then copy the file |childdoc.def| to an appropriate directory of your \LaTeX{}
distribution, e.g.\ \textit{texmf-root}|/tex/latex/childdoc|.
\end{itemize}

%%%%%%%%%%%%%%%%%%%%%%%%%%%%%%%%%%%%%%%%%%%%%%%%%%%%%%%%%%%%%%%%%%%%%%%%%%%%%%%%
\subsection{Related CTAN Packages}

There are several other packages which offer a similar functionality:
%
\begin{itemize}
\item
The packages
\href{http://ctan.org/pkg/docmute}{\textsf{docmute}},
\href{http://ctan.org/pkg/includex}{\textsf{includex}} and
\href{http://ctan.org/pkg/standalone}{\textsf{standalone}}
provide commands to include only the document body of
a child file thus allowing both files to be compiled individually.
\item
The packages \href{http://ctan.org/pkg/subdocs}{\textsf{subdocs}}
and \href{http://ctan.org/pkg/subfiles}{\textsf{subfiles}}
provide structures in which the main and child documents can be
encapsulated and allowing them to be compiled individually.
The inclusion mechanism is different from the conventional |\include|.
\item
The package \href{http://ctan.org/pkg/combine}{\textsf{combine}}
is an elaborate solution to combine several documents into one.
\end{itemize}
%
See also the CTAN topic \href{http://ctan.org/topic/subdocs}{\textsf{subdocs}}
for further related packages.
The present package differs from the above solutions in that
a document structure constructed with the conventional |\include| mechanism
just needs two extra commands at the top of every file
such that all constituent files can be compiled individually.

%%%%%%%%%%%%%%%%%%%%%%%%%%%%%%%%%%%%%%%%%%%%%%%%%%%%%%%%%%%%%%%%%%%%%%%%%%%%%%%%
%\subsection{Feature Suggestions}
%
%The following is a list of features which may be useful for future
%versions of this package:
%%
%\begin{itemize}
%\item
%\ldots
%\end{itemize}

%%%%%%%%%%%%%%%%%%%%%%%%%%%%%%%%%%%%%%%%%%%%%%%%%%%%%%%%%%%%%%%%%%%%%%%%%%%%%%%%
\subsection{Revision History}

%%%%%%%%%%%%%%%%%%%%%%%%%%%%%%%%%%%%%%%%
\paragraph{v2.0:} 2018/12/30

\begin{itemize}
\item
immediate forward processing
\item
added |\childdocby| mechanism
\item
manual restructured
\end{itemize}

%%%%%%%%%%%%%%%%%%%%%%%%%%%%%%%%%%%%%%%%
\paragraph{v1.6:} 2018/01/17

\begin{itemize}
\item
application for development of include files
\item
corrections to manual
\end{itemize}

%%%%%%%%%%%%%%%%%%%%%%%%%%%%%%%%%%%%%%%%
\paragraph{v1.5:} 2017/05/21

\begin{itemize}
\item
more complete structuring introduced
\item
|\childdocof| introduced
\item
|\childdoc| renamed to |\childdocmain|
\item
|\childredirect| renamed to |\childdocforward| and |\childdocforwardprefix|
and functionality expanded
\end{itemize}

%%%%%%%%%%%%%%%%%%%%%%%%%%%%%%%%%%%%%%%%
\paragraph{v1.0:} 2017/04/27

\begin{itemize}
\item
manual and install package
\item
first version published on CTAN
\end{itemize}

%%%%%%%%%%%%%%%%%%%%%%%%%%%%%%%%%%%%%%%%
\paragraph{v0.6:} 2017/04/26

\begin{itemize}
\item
redirection mechanism added
\end{itemize}

%%%%%%%%%%%%%%%%%%%%%%%%%%%%%%%%%%%%%%%%
\paragraph{v0.5:} 2017/04/26

\begin{itemize}
\item
functionality in definition file
\end{itemize}


%%%%%%%%%%%%%%%%%%%%%%%%%%%%%%%%%%%%%%%%%%%%%%%%%%%%%%%%%%%%%%%%%%%%%%%%%%%%%%%%
%%%%%%%%%%%%%%%%%%%%%%%%%%%%%%%%%%%%%%%%%%%%%%%%%%%%%%%%%%%%%%%%%%%%%%%%%%%%%%%%
%%%%%%%%%%%%%%%%%%%%%%%%%%%%%%%%%%%%%%%%%%%%%%%%%%%%%%%%%%%%%%%%%%%%%%%%%%%%%%%%
\appendix

\settowidth\MacroIndent{\rmfamily\scriptsize 000\ }

 \DocInput{childdoc.dtx}

\end{document}
%</driver>
% \fi
%
% %%%%%%%%%%%%%%%%%%%%%%%%%%%%%%%%%%%%%%%%%%%%%%%%%%%%%%%%%%%%%%%%%%%%%%%%%%%%%%
% %%%%%%%%%%%%%%%%%%%%%%%%%%%%%%%%%%%%%%%%%%%%%%%%%%%%%%%%%%%%%%%%%%%%%%%%%%%%%%
% \section{Sample}
%\iffalse
%<*samplemain>
%\fi
%
% The following presents a sample document
% with two chapters, two parts, a title page,
% a compile flag as well as three forwarding files to set the flag.
% It consists of eight |.tex| files:
% \begin{center}
% \begin{tabular}{ll}
% |cdocsamp.tex|&main file\\
% |cdocsch1.tex|&include file for chapter 1\\
% |cdocsch2.tex|&include file for chapter 2\\
% |cdocspt3.tex|&include file for part 3\\
% |cdocspt4.tex|&include file for part 4\\
% |cdocsdrf.tex|&forwarding file for main file in draft mode\\
% |cdocsfi1.tex|&forwarding file for final version of chapter 1\\
% |cdocsfi2.tex|&forwarding file for final version of chapter 2\\
% \end{tabular}
% \end{center}
% Each of the eight files can be compiled directly by the \LaTeX{} compiler.
%
% %%%%%%%%%%%%%%%%%%%%%%%%%%%%%%%%%%%%%%
% \paragraph{Main File.}
%
% The main file is called |cdocsamp.tex|.
%
% Load the \textsf{childdoc} definitions and
% declare the filename for the main document:
%    \begin{macrocode}
\input{childdoc.def}
\childdocmain{}
%    \end{macrocode}

% Optional override for |\version| flag:
%    \begin{macrocode}
%%\ifchilddoc\else\providecommand{\version}{draft}\fi
%    \end{macrocode}

% Define the default values for the |\version| flag
% (|final| for the main file and |draft| for childs):
%    \begin{macrocode}
\ifchilddoc
\providecommand{\version}{draft}
\else
\providecommand{\version}{final}
\fi
%    \end{macrocode}

% Load the standard document class:
%    \begin{macrocode}
\documentclass[12pt]{article}
%    \end{macrocode}

% Start the document body:
%    \begin{macrocode}
\begin{document}
%    \end{macrocode}

% Declare a title page.
% Print title, part of document being processed and version flag:
%    \begin{macrocode}
\addtocounter{page}{-1}
\begin{center}
{\LARGE\bfseries{}childdoc example\par}
\vspace{1cm}
\ifchilddoc
\ifchilddocmanual part\else chapter\fi:
`\childdocname' of `\childdocjob'\par
\else
main document: `\childdocjob'\par
\fi
version: \version\par
\end{center}
\newpage
%    \end{macrocode}

% Manually include selected file,
% otherwise process as usual:
%    \begin{macrocode}
\ifchilddocmanual
\section*{part `\childdocname'}
\input{\childdocname}
\else
%    \end{macrocode}

% Include the two chapters:
%    \begin{macrocode}
\include{cdocsch1}
\include{cdocsch2}
%    \end{macrocode}

% Include the two parts unless only chapters should be displayed:
%    \begin{macrocode}
\ifchilddoc\else
\section{part three}
\input{cdocspt3}
\section{part four}
\input{cdocspt4}
\fi
%    \end{macrocode}

% Process as usual until here:
%    \begin{macrocode}
\fi
%    \end{macrocode}

% End of document body:
%    \begin{macrocode}
\end{document}
%    \end{macrocode}
%\iffalse
%</samplemain>
%\fi
%
% %%%%%%%%%%%%%%%%%%%%%%%%%%%%%%%%%%%%%%
% \paragraph{Chapter Include Files.}
%
% The include files are called |cdocsch1.tex| and |cdocsch2.tex|.
%
%\iffalse
%<*samplechap1|samplechap2>
%\fi

% Optional override for |\version| flag:
%    \begin{macrocode}
%%\providecommand{\version}{final}
%    \end{macrocode}

% Include the main document:
%    \begin{macrocode}
\input{childdoc.def}
\childdocof{cdocsamp}
%    \end{macrocode}

%\iffalse
%</samplechap1|samplechap2>
%\fi
%
%\iffalse
%<*samplechap1>
%\fi
% Some text for chapter 1:
%    \begin{macrocode}
\section{one}
some text in chapter one
%    \end{macrocode}

%\iffalse
%</samplechap1>
%\fi
% Some text for chapter 2:
%\iffalse
%<*samplechap2>
%\fi
%    \begin{macrocode}
\section{two}
more text in chapter two
%    \end{macrocode}

%\iffalse
%</samplechap2>
%\fi
%
% %%%%%%%%%%%%%%%%%%%%%%%%%%%%%%%%%%%%%%
% \paragraph{Part Include Files.}
%
% The include files are called |cdocspt3.tex| and |cdocspt4.tex|.
%
%\iffalse
%<*samplepart3|samplepart4>
%\fi

% Optional override for |\version| flag:
%    \begin{macrocode}
%%\providecommand{\version}{final}
%    \end{macrocode}

% Include the main document:
%    \begin{macrocode}
\input{childdoc.def}
\childdocby{cdocsamp}
%    \end{macrocode}

%\iffalse
%</samplepart3|samplepart4>
%\fi
%
%\iffalse
%<*samplepart3>
%\fi
% Some text for part 3:
%    \begin{macrocode}
some text in part three
%    \end{macrocode}

%\iffalse
%</samplepart3>
%\fi
% Some text for part 4:
%\iffalse
%<*samplepart4>
%\fi
%    \begin{macrocode}
more text in part four
%    \end{macrocode}

%\iffalse
%</samplepart4>
%\fi
%
% %%%%%%%%%%%%%%%%%%%%%%%%%%%%%%%%%%%%%%
% \paragraph{Forwarding for a Complete Draft.}
%
% The following forwarding file |cdocsdrf.tex|
% compiles the main document in draft mode:
%\iffalse
%<*sampledraft>
%\fi
%    \begin{macrocode}
\def\version{draft}
\input{childdoc.def}
\childdocforward{cdocsamp}
%    \end{macrocode}

%\iffalse
%</sampledraft>
%\fi
%
% %%%%%%%%%%%%%%%%%%%%%%%%%%%%%%%%%%%%%%
% \paragraph{Forwarding for Final Version of the Chapters.}
%
% The following forwarding files |cdocsfn1.tex| and |cdocsfn2.tex|
% (with identical content)
% compile the final versions of the child documents
% |cdocsch1.tex| and |cdocsch2.tex|, respectively:
%\iffalse
%<*samplefinal>
%\fi
%    \begin{macrocode}
\def\version{final}
\input{childdoc.def}
\childdocforwardprefix[cdocsamp]{cdocsfn}{cdocsch}
%    \end{macrocode}

%\iffalse
%</samplefinal>
%\fi
%
% %%%%%%%%%%%%%%%%%%%%%%%%%%%%%%%%%%%%%%
% \paragraph{Command Line Processing.}
%
% The following three command lines generate the output files
% |cdocscld|, |cdocscl1| and |cdocscl2|
% which should be identical to
% |cdocsdrf|, |cdocsch1| and |cdocsfn2|, respectively:
% \begin{center}
% \begin{tabular}{l}
% |latex -jobname cdocscld \|\\
% |  "\def\version{draft}\input{childdoc.def}\childdocforward{cdocsamp}"|\\
% |latex -jobname cdocscl1 \|\\
% |  "\input{childdoc.def}\childdocforward[cdocsamp]{cdocsch1}"|\\
% |latex -jobname cdocscl2 \|\\
% |  "\def\version{final}\input{childdoc.def}\childdocforward{cdocsch2}"|
% \end{tabular}
% \end{center}
% Note that the trailing backslash on each first line
% merely continues the input to the second line
% (for convenient cut ant paste).
% Furthermore, the command |latex| can be replaced by any
% of its alternative versions such as |pdflatex|.
%
% %%%%%%%%%%%%%%%%%%%%%%%%%%%%%%%%%%%%%%%%%%%%%%%%%%%%%%%%%%%%%%%%%%%%%%%%%%%%%%
% %%%%%%%%%%%%%%%%%%%%%%%%%%%%%%%%%%%%%%%%%%%%%%%%%%%%%%%%%%%%%%%%%%%%%%%%%%%%%%
% \section{Implementation}
%\iffalse
%<*package>
%\fi
%
% This section describes the definitions file |childdoc.def|.

% The definitions cannot be loaded using |\usepackage| or |\RequirePackage|
% which has a mechanism to prevent loading a style file more than once.
% When loading the definitions by means of |\input|
% multiple instances have to be prevented manually:
%\iffalse
%This code needs to be before the `\ProvidesFile' directive
%which is defined at the beginning of this file.
%Therefore it is also placed there and commented out here.
%</package>
%<*discard>
%\fi
%    \begin{macrocode}
\ifdefined\childdocmain\endinput\fi
%    \end{macrocode}
%\iffalse
%</discard>
%<*package>
%\fi
%
% \macro{\ifchilddoc}
% \macro{\ifchilddocmanual}
% The conditional |\ifchilddoc| tells whether a
% child (true) or main (false) document is being compiled.
% The conditional |\ifchilddocmanual| tells whether
% the |\includeonly| mechanism is used (false) or
% the selection of child files must be performed manually (true).
% The definitions initialise to false:
%    \begin{macrocode}
\newif\ifchilddoc
\newif\ifchilddocmanual
%    \end{macrocode}

% \macro{\childdocname}
% \macro{\childdocjob}
% The macro |\childdocname| stores the name of the main document
% to be compiled. The macro |\childdocjob| stores the name of
% the document on which the \LaTeX{} compiler was originally invoked.
% The content of |\jobname| cannot be compared
% to filenames specified in the source due to different catcodes.
% The following code rescans |\jobname|, stores the result
% in |\childdocname| and saves a copy in |\childdocjob|:
%    \begin{macrocode}
\edef\childdocname{\scantokens\expandafter{\jobname\noexpand}}
\let\childdocjob\childdocname
%    \end{macrocode}

% \macro{\childdocdisable}
% The macro |\childdocdisable| prevents the main file
% from being processed more than once.
% At this stage, the main document command |\childdocmain|
% is assumed to be called once again where it should do nothing.
% Any subsequent call to it should prevent
% a secondary processing of the main document
% It overwrites the forwarding commands
% |\childdocof| and |\childdocforward|
% with empty macros to prevent further inclusions of the main document:
%    \begin{macrocode}
\newcommand{\childdocdisable}
{
  \renewcommand{\childdocmain}[1]{\renewcommand{\childdocmain}[1]{\endinput}}
  \renewcommand{\childdocof}[1]{}
  \renewcommand{\childdocby}[2][]{}
  \renewcommand{\childdocforward}[2][]{}
  \renewcommand{\childdocdisable}{}
}
%    \end{macrocode}

% \macro{\childdocmain}
% The macro |\childdocmain| is to be called at the top of the main file
% with nothing or the main filename (without extension) as argument.
% First, it breaks loops.
% If the argument is not empty and does not match |\childdocname|
% (which is set by the first inclusion of |childdoc.def|),
% |\ifchilddoc| is set to true, |\includeonly| is applied to the child file
% and |\jobname| is set to the main file
% (for proper handling of |.aux| files):
%    \begin{macrocode}
\newcommand{\childdocmain}[1]
{
  \childdocdisable\childdocmain{}
  \if?#1?\else
    \begingroup
      \def\childdoctmp{#1}
      \ifx\childdoctmp\childdocname
        \def\childdoctmp{}
      \else
        \def\childdoctmp
        {
          \childdoctrue
          \includeonly{\childdocname}
          \def\childdocjob{#1}
          \def\jobname{#1}
        }
      \fi
      \expandafter
    \endgroup
    \childdoctmp
  \fi
}
%    \end{macrocode}

% \macro{\childdocof}
% The command |\childdocof| redirects
% compilation to the main file |#1|.
%    \begin{macrocode}
\newcommand{\childdocof}[1]
{
  \childdocdisable
  \childdoctrue
  \includeonly{\childdocname}
  \def\jobname{#1}
  \def\childdocjob{#1}
  \input{#1}
}
%    \end{macrocode}

% \macro{\childdocby}
% The command |\childdocby| ....
%    \begin{macrocode}
\newcommand{\childdocby}[2][]
{
  \childdocdisable
  \childdoctrue
  \childdocmanualtrue
  \if?#1?\else
    \def\jobname{#2}
  \fi
  \def\childdocjob{#2}
  \input{#2}
  \endinput
}
%    \end{macrocode}

% \macro{\childdocforward}
% The command |\childdocforward| redirects
% compilation to the main file or
% (if the optional argument is given) a child file.
% Parameters are set as if the main file
% or a child file starting with |\childdocof| was compiled.
% Then compilation is handed over to the main file:
%    \begin{macrocode}
\newcommand{\childdocforward}[2][]
{
  \begingroup
    \if?#1?
      \def\childdoctmp
      {
        \def\childdocname{#2}
        \def\childdocjob{#2}
        \def\jobname{#2}
        \input{#2}
        \endinput
      }
    \else
      \def\childdoctmp
      {
        \childdocdisable
        \def\childdocname{#2}
        \childdoctrue
        \includeonly{#2}
        \def\childdocjob{#1}
        \def\jobname{#1}
        \input{#1}
        \endinput
      }
    \fi
    \expandafter
  \endgroup
  \childdoctmp
}
%    \end{macrocode}

% \macro{\childdocforwardprefix}
% The command |\childdocforwardprefix| redirects
% compilation to the main or a child file by means of a pattern.
% The prefix |#1| in the current filename is replaced by |#2|
% and the suffix of the current filename is kept
% (it is assumed that the filename does not contain the substring `|~~~|'
% which is used as a delimiter).
% Compilation is handed over to the new file by |\childdocforward|:
%    \begin{macrocode}
\newcommand{\childdocforwardprefix}[3][]
{
  \begingroup
    \def\childdocextract #2##1~~~{\def\childdoctmp{\childdocforward[#1]{#3##1}}}
    \expandafter\childdocextract\childdocname~~~
    \expandafter
  \endgroup
  \childdoctmp
}
%    \end{macrocode}

% \macro{\childdoc}
% The deprecated macro |\childdoc| is a legacy version of |\childdocmain|:
%    \begin{macrocode}
\newcommand{\childdoc}{\childdocmain}
%    \end{macrocode}

% \macro{\childdocredirect}
% The deprecated macro |\childdocredirect| is a legacy version
% of |\childdocforward| and |\childdocforwardprefix|:
%    \begin{macrocode}
\newcommand{\childdocredirect}[2][]
{
  \begingroup
    \if?#1?
      \def\childdoctmp{\childdocforward{#2}}
    \else
      \def\childdoctmp{\childdocforwardprefix{#1}{#2}}
    \fi
    \expandafter
  \endgroup
  \childdoctmp
}
%    \end{macrocode}

%\iffalse
%</package>
%\fi
%
\endinput
|\\
|\childdocforward{|\textit{main}|}|\\
\end{tabular}
\end{center}
%
or alternatively with:
%
\begin{center}
\begin{tabular}{l}
|% \iffalse
%
% childdoc.dtx Copyright (C) 2017-2018 Niklas Beisert
%
% This work may be distributed and/or modified under the
% conditions of the LaTeX Project Public License, either version 1.3
% of this license or (at your option) any later version.
% The latest version of this license is in
%   http://www.latex-project.org/lppl.txt
% and version 1.3 or later is part of all distributions of LaTeX
% version 2005/12/01 or later.
%
% This work has the LPPL maintenance status `maintained'.
%
% The Current Maintainer of this work is Niklas Beisert.
%
% This work consists of the files childdoc.dtx and childdoc.ins
% and the derived files childdoc.def and cdocsamp.tex with
% cdocsch1.tex, cdocsch2.tex, cdocsdrf.tex, cdocsfn1.tex, cdocsfn2.tex.
%
%<package>\ifdefined\childdocmain\endinput\fi
%<package>\ProvidesFile{childdoc.def}[2018/12/30 v2.0 child document driver]
%<samplemain>\ProvidesFile{cdocsamp.tex}[2018/12/30 v2.0 sample for childdoc]
%<*driver>
%\ProvidesFile{childdoc.drv}[2018/12/30 v2.0 childdoc reference manual file]
\PassOptionsToClass{10pt,a4paper}{article}
\documentclass{ltxdoc}

\usepackage[margin=35mm]{geometry}
\usepackage{hyperref}
\usepackage{hyperxmp}
\usepackage[usenames]{color}

\hypersetup{colorlinks=true}
\hypersetup{pdfstartview=FitH}
\hypersetup{pdfpagemode=UseNone}
\hypersetup{pdfsource={}}
\hypersetup{pdflang={en-UK}}
\hypersetup{pdfcopyright={Copyright 2017-2018 Niklas Beisert.
  This work may be distributed and/or modified under the
  conditions of the LaTeX Project Public License, either version 1.3
  of this license or (at your option) any later version.}}
\hypersetup{pdflicenseurl={http://www.latex-project.org/lppl.txt}}
\hypersetup{pdfcontactaddress={ETH Zurich, ITP, HIT K,
  Wolfgang-Pauli-Strasse 27}}
\hypersetup{pdfcontactpostcode={8093}}
\hypersetup{pdfcontactcity={Zurich}}
\hypersetup{pdfcontactcountry={Switzerland}}
\hypersetup{pdfcontactemail={nbeisert@itp.phys.ethz.ch}}
\hypersetup{pdfcontacturl={http://people.phys.ethz.ch/\xmptilde nbeisert/}}

\newcommand{\secref}[1]{\hyperref[#1]{section \ref*{#1}}}

\parskip1ex
\parindent0pt
\let\olditemize\itemize
\def\itemize{\olditemize\parskip0pt}

\begin{document}

\title{The \textsf{childdoc} Package}
\hypersetup{pdftitle={The childdoc Package}}
\author{Niklas Beisert\\[2ex]
  Institut f\"ur Theoretische Physik\\
  Eidgen\"ossische Technische Hochschule Z\"urich\\
  Wolfgang-Pauli-Strasse 27, 8093 Z\"urich, Switzerland\\[1ex]
  \href{mailto:nbeisert@itp.phys.ethz.ch}
  {\texttt{nbeisert@itp.phys.ethz.ch}}}
\hypersetup{pdfauthor={Niklas Beisert}}
\hypersetup{pdfsubject={Manual for the LaTeX2e Package childdoc}}
\date{30 December 2018, \textsf{v2.0}}
\maketitle

\begin{abstract}\noindent
\textsf{childdoc} is a \LaTeXe{} package
that enables the direct compilation
of document sections included by |\include|
to individual files.
\end{abstract}

\begingroup
\parskip0ex
\tableofcontents
\endgroup

%%%%%%%%%%%%%%%%%%%%%%%%%%%%%%%%%%%%%%%%%%%%%%%%%%%%%%%%%%%%%%%%%%%%%%%%%%%%%%%%
%%%%%%%%%%%%%%%%%%%%%%%%%%%%%%%%%%%%%%%%%%%%%%%%%%%%%%%%%%%%%%%%%%%%%%%%%%%%%%%%
\section{Introduction}

\LaTeX{} provides a mechanism to structure a large document (such as a book)
into a main file and several child files (containing the chapters)
using the |\include| command.
This mechanism is beneficial for documents
which span hundreds of pages in order to
make the source file(s) more manageable.
Moreover, compilation can be restricted to
selected child files by means of the |\includeonly| command.
The latter feature can be used to reduce the compilation time while editing
(this was significantly more useful in the earlier days of \LaTeX{})
or to generate a smaller document which is easier to navigate.
Another application of |\includeonly| is to generate
documents consisting of selected parts of the complete document.

However, there are a few drawbacks of the plain |\include| mechanism:
\begin{itemize}
\item
The child files cannot be compiled on their own,
they can only be compiled via the main file.
A naive editing environment
(such as a text editor with an option
to have the current file processed by \LaTeX)
may require one to switch to the main file before compiling;
attempting to compile the child file produces errors.
\item
The main file must be modified (each time)
to adjust the |\includeonly| command
to the present needs. This easily leaves the main file in a messy state.
\item
The generated document will always carry the filename
of the main document. This is inconvenient if
several child files are to be compiled and
to be kept for distribution.
\end{itemize}

The present package provides a simple interface
to make child files individually compilable by \LaTeX{}.
Compiling a child file then has the same effect as compiling
the main file with an |\includeonly| command
to select the appropriate child.
Moreover the generated document will carry the name of the child
rather than the main file.
This resolves all three above issues.

This feature is meant to make the editing of books,
thesis documents and lecture notes somewhat more convenient.
However, the package can also be used efficiently for
composing a series of documents (such as exercise sheets)
which are typically distributed individually.
It then assists the author in generating the individual documents
(potentially in different versions)
as well as a document containing the collected series.
Another application is in developing style files
or other kinds of included material
where compilation of the style file could redirect
to a sample or test file.

%%%%%%%%%%%%%%%%%%%%%%%%%%%%%%%%%%%%%%%%%%%%%%%%%%%%%%%%%%%%%%%%%%%%%%%%%%%%%%%%
%%%%%%%%%%%%%%%%%%%%%%%%%%%%%%%%%%%%%%%%%%%%%%%%%%%%%%%%%%%%%%%%%%%%%%%%%%%%%%%%
\section{Usage}

First of all, the package \textsf{childdoc} is \emph{not} a standard
\LaTeXe{} |.sty| style file! Therefore it needs to be invoked in
a non-standard way.

%%%%%%%%%%%%%%%%%%%%%%%%%%%%%%%%%%%%%%%%%%%%%%%%%%%%%%%%%%%%%%%%%%%%%%%%%%%%%%%%
\subsection{Included Files}
\label{sec:include}

%%%%%%%%%%%%%%%%%%%%%%%%%%%%%%%%%%%%%%%%
\DescribeMacro{\childdocmain}
To use the package, add the commands
\begin{center}
\begin{tabular}{l}
|\input{childdoc.def}|\\
|\childdocmain{}|\\
\end{tabular}
\end{center}
at the very top of the main \LaTeX{} file,
in particular \emph{before} the |\documentclass| statement!
The argument of |\childdocmain| should be left empty
(but it must be present).

%%%%%%%%%%%%%%%%%%%%%%%%%%%%%%%%%%%%%%%%
\DescribeMacro{\childdocof}
Furthermore, add the commands
\begin{center}
\begin{tabular}{l}
|\input{childdoc.def}|\\
|\childdocof{|\textit{main}|}|\\
\end{tabular}
\end{center}
at the top of every child file \textit{child}
which is included by |\include{|\textit{child}|}|
from within the main file
(or at least for those files to be compiled individually).
The argument \textit{main} must be the filename of the main file.

There are a couple of
considerations in setting up the main and child documents:

%%%%%%%%%%%%%%%%%%%%%%%%%%%%%%%%%%%%%%%%
\paragraph{Restrictions.}

Please note the following restrictions:
\begin{itemize}
\item
|\childdocmain| must be called with one argument \textit{main}
to ensure compatibility with earlier version of the package.
It must either be empty (|\childdocmain{}|)
or precisely match the filename of the main file in which it is specified.
See \secref{sec:detection} for further information.
\item
The filename \textit{main} must be specified without the |.tex| extension.
\item
The filename \textit{main} is case sensitive
(even in case-insensitive file systems)
due to internal string comparison.
\item
The argument \textit{main} should be fully expanded, it cannot be a macro.
\item
Subdirectories and special characters should be avoided in filenames.
\item
The command |\childdocmain{|\textit{main}|}| must be followed by a whitespace.
It should not be followed immediately by another command
or by a comment mark `|%|'.
This is because the \TeX{} parser reads the token immediately following
the argument of |\childdocmain| and puts it
at the beginning of every child section;
however, a white\-space is ignored.
\end{itemize}

%%%%%%%%%%%%%%%%%%%%%%%%%%%%%%%%%%%%%%%%
\paragraph{Content of Main File.}

It is advisable to place all content in the child files included by |\include|.
Any output contained in the main file will appear in all child documents
unless suppressed manually;
it cannot be suppressed automatically by the |\includeonly| directive
and thus should normally be avoided.
A method to include some content in the main file
by means of conditional processing is described in \secref{sec:conditional}.

%%%%%%%%%%%%%%%%%%%%%%%%%%%%%%%%%%%%%%%%
\paragraph{Page Numbering.}

When only a part of the document is compiled,
the appropriate numbering of pages
(as well as other status parameters)
is determined from the |.aux| files.
The latter contain information from previous passes.
However this information needs to propagate through
all intermediate child documents.
Therefore the page numbering in child documents may well
be inconsistent until the complete document is compiled at least once.

A useful (if unconventional) way to always ensure a consistent
page numbering is to restart the numbering in each child document
and denote the pages by `\textit{child}|.|\textit{page}'
where \textit{child} represents the chapter/section number of the child file.
This can be achieved by the command
|\numberwithin{page}{|\textit{child}|}|
of the \textsf{amsmath} package
where \textit{child} can be |chapter| or |section|
depending on the chosen structuring.
Alternatively, one can modify the macro |\thepage| appropriately
and reset the counter |page| at the start of each child file.

%%%%%%%%%%%%%%%%%%%%%%%%%%%%%%%%%%%%%%%%%%%%%%%%%%%%%%%%%%%%%%%%%%%%%%%%%%%%%%%%
\subsection{Conditional Processing}
\label{sec:conditional}

The package provides a mechanism to compile different versions
of a document. To customise the versions further some conditional processing
can come in handy to distinguish which version is being compiled.
The package provides two macros to describe the compilation context:

%%%%%%%%%%%%%%%%%%%%%%%%%%%%%%%%%%%%%%%%
\DescribeMacro{\ifchilddoc}
The conditional |\ifchilddoc| distinguishes between the compilation of
child documents and the main document:
%
\begin{center}
|\ifchilddoc |\textit{child-code}| |[|\||else |\textit{main-code}]| \||fi|
\end{center}

%%%%%%%%%%%%%%%%%%%%%%%%%%%%%%%%%%%%%%%%
\DescribeMacro{\childdocname}
\DescribeMacro{\childdocjob}
The macro |\childdocname| contains the filename (without extension)
of the main or child file being processed.
Note that |\childdocjob| will always contain the name of the main file.

%%%%%%%%%%%%%%%%%%%%%%%%%%%%%%%%%%%%%%%%
\paragraph{Title Page.}

Conditional processing can be used to include a title or banner page
in the main document when proper precautions are taken.
Importantly, the code in the main file should ensure that the page counter
(as well as other status parameters which are stored in the |.aux| files)
takes the same value after the conditional processing.
Otherwise the page numbers may take divergent values
depending on which part is compiled.

For example, a title page could be declared by:
%
\begin{center}
\begin{tabular}{l}
|\ifchilddoc\||else|\\
|\addtocounter{page}{-1}|\\
\textit{code for title page}\\
|\newpage|\\
|\||fi|
\end{tabular}
\end{center}
%
A banner page for the child documents can be generated by:
%
\begin{center}
\begin{tabular}{l}
|\ifchilddoc|\\
|\addtocounter{page}{-1}|\\
\textit{code for banner page}\\
|\newpage|\\
|\||fi|
\end{tabular}
\end{center}
%
Here one could write a message such as:
\begin{center}
|This is the part \childdocname{} of \childdocjob{}.|
\end{center}

%%%%%%%%%%%%%%%%%%%%%%%%%%%%%%%%%%%%%%%%%%%%%%%%%%%%%%%%%%%%%%%%%%%%%%%%%%%%%%%%
\subsection{Flags}
\label{sec:flags}

The package makes it easy to generate different versions
of the main or child documents.
To this end compilation flags can be defined
and assigned different default values.
They will be particularly useful in conjunction
with the forwarding mechanism described in \secref{sec:forward}.

For example, it may be useful to have a flag |\version|
which can be set to |draft| or |final|.
The document source will contain some conditional code
depending on the value of |\version|.
Suppose further, the flag should default to |final| for the main file
and to |draft| for child files
which is a natural assignment for editing the document.
This is achieved by placing the following code
in the preamble of the main document
(below the |\childdocmain| directive):
%
\begin{center}
\begin{tabular}{l}
|\ifchilddoc|\\
|\providecommand{\version}{draft}|\\
|\||else|\\
|\providecommand{\version}{final}|\\
|\||fi|
\end{tabular}
\end{center}
%
The definition by |\providecommand| makes sure
that previous definitions are not overwritten.
Further statements |\providecommand{\version}{...}|
can thus be added before the above code to override it.

For the main file, one might add a line
(between |\childdocmain| and the above block)
%
\begin{center}
|%\ifchilddoc\||else\providecommand{\version}{draft}\||fi|
\end{center}
%
which can be uncommented to produce a draft version.
Likewise one can add a line to the very top of a child file
(above the |\childdocof{|\textit{main}|}| directive)
%
\begin{center}
|%\providecommand{\version}{final}|
\end{center}
%
which can be uncommented to produce the final version of this child document.

%%%%%%%%%%%%%%%%%%%%%%%%%%%%%%%%%%%%%%%%%%%%%%%%%%%%%%%%%%%%%%%%%%%%%%%%%%%%%%%%
\subsection{Forwarding}
\label{sec:forward}

Different versions of the main or child documents
using compilation flags as described in \secref{sec:flags}
can be (permanently) stored in different files
for convenient compilation, viewing and distribution.
To this end, the package defines a command
to pass on compilation to a different file:

%%%%%%%%%%%%%%%%%%%%%%%%%%%%%%%%%%%%%%%%
\DescribeMacro{\childdocforward}
The command |\childdocforward| redirects processing to
another source file:
%
\begin{center}
\begin{tabular}{l}
|\input{childdoc.def}|\\
|\childdocforward[|\textit{main}|]{|\textit{dest}|}|\\
\end{tabular}
\end{center}
%
The argument \textit{dest} is the destination file
(without extension).
It should be the main file or one of the child files.
Note that further \textsf{childdoc} directives
such as |\childdocof| and |\childdocforward|
in the indicated file will be processed in this form.
The optional argument \textit{main}
passes on directly to the main file \textit{main}
while pretending to compile the child \textit{dest}.
This form behaves as if \textit{dest}
issues |\childdocof{|\textit{main}|}| right away,
and no further \textsf{childdoc} directives will be processed.

%%%%%%%%%%%%%%%%%%%%%%%%%%%%%%%%%%%%%%%%
\DescribeMacro{\...prefix}
In the alternative form |\childdocforwardprefix|,
%
\begin{center}
\begin{tabular}{l}
|\input{childdoc.def}|\\
|\childdocforwardprefix[|\textit{main}|]{|\textit{prefix}|}{|\textit{dest}|}|
\end{tabular}
\end{center}
%
the destination file is determined by a pattern
depending on the current file:
To make this work, the current file must be called
`{\textit{prefix}\hspace{0.2em}\textit{suffix}}'
with \textit{prefix} matching precisely the argument.
Processing is then passed on to the file
`{\textit{dest}\hspace{0.2em}\textit{suffix}}'.
Surely, the same effect is achieved by
directly specifying the
argument `{\textit{dest}\hspace{0.2em}\textit{suffix}}'
in the first form.
However, that requires to set up a different file
for each child. With the alternative form of the command
all these files can have exactly the same content
which simplifies setting them up and maintaining them.

For example, the following file |draft.tex|
with a compilation flag |\version| as described in \secref{sec:flags}
compiles the main document as a draft:
%
\begin{center}
\begin{tabular}{l}
|\def\version{draft}|\\
|\input{childdoc.def}|\\
|\childdocforward{|\textit{main}|}|
\end{tabular}
\end{center}
%
Likewise, the following files |final|\textit{nn}|.tex|
compile the final version of the child document
|child|\textit{nn}|.tex|:
%
\begin{center}
\begin{tabular}{l}
|\def\version{final}|\\
|\input{childdoc.def}|\\
|\childdocforwardprefix{final}{child}|
\end{tabular}
\end{center}
%

Note that when several versions of a main file and/or of each child file
are to be generated, it may be convenient to set up a |Makefile| or
shell script to automatise the process.

%%%%%%%%%%%%%%%%%%%%%%%%%%%%%%%%%%%%%%%%%%%%%%%%%%%%%%%%%%%%%%%%%%%%%%%%%%%%%%%%
\subsection{Command Line Processing}
\label{sec:commandline}

The effect of redirection files can also be achieved by invoking
the \LaTeX{} compiler with a more elaborate command line.
Most conveniently this should be done as part
of a shell script or a |Makefile|.

When using \textsf{childdoc} in the main file, the following
command lines effectively perform a redirection
(note that depending on the shell being used,
backslashes may have to be doubled: `|\|' $\to$ `|\\|'):
%
\begin{center}
|... -jobname "|\textit{target}|" |\\|"|[\textit{flags}]%
|\input{childdoc.def}\childdocforward[|\textit{main}|]{|\textit{dest}|}"|
\end{center}
%
Here \textit{target} is the name of the output file,
\textit{main} is the name of the main file
and \textit{dest} is the name of the main or child file to be processed
(all filenames without extensions).
The optional argument \textit{main} can be omitted
if \textit{main} matches \textit{dest}.
Optionally, compilation \textit{flags} can be defined via |\def| commands.
This command line makes the \TeX{} engine believe
it is compiling the file \textit{target}
whose content is specified as the latter parameter.
The provided code then forwards the processing to
\textit{main} or \textit{dest} as described in \secref{sec:forward}.

%%%%%%%%%%%%%%%%%%%%%%%%%%%%%%%%%%%%%%%%%%%%%%%%%%%%%%%%%%%%%%%%%%%%%%%%%%%%%%%%
\subsection{Include by Input}
\label{sec:input}

Including child documents by |\include| has some restrictions by design.
Most notably, the content of a child document always occupies
its own set of pages; pages cannot be shared between child documents.
Usually, this behaviour makes perfect sense
because each child document contain an essential part of the document.
However, in some situations it may be desirable to compose
a document from a collection of parts
without having mandatory page breaks between then.
For this case, the package
provides a mechanism to include parts
by |\input| which can also be processed individually.
However, by construction this mechanism
requires manual handling of the content to be output.

%%%%%%%%%%%%%%%%%%%%%%%%%%%%%%%%%%%%%%%%
\DescribeMacro{\ifchilddocmanual}
The main file should be prepared as usual, see \secref{sec:include}.
However, the document body must make a distinction
between processing of an individual part and of the main document, e.g.:
%
\begin{center}
\begin{tabular}{l}
|\ifchilddocmanual|\\
|\input{\childdocname}|\\
|\||else|\\
\textit{document body with }|\input{|\textit{part}|}|\\
|\||fi|
\end{tabular}
\end{center}
%
The conditional |\ifchilddocmanual| is true whenever
a part to be included by |\input| is being compiled,
and the name of the part is stored in |\childdocname|.

%%%%%%%%%%%%%%%%%%%%%%%%%%%%%%%%%%%%%%%%
\DescribeMacro{\childdocby}
Each part to be included by |\input| should start with:
%
\begin{center}
\begin{tabular}{l}
|\input{childdoc.def}|\\
|\childdocby{|\textit{main}|}|\\
\end{tabular}
\end{center}
%
The directive |\childdocby| is similar to |\childdocof|
described in \secref{sec:include},
but the subsequent selection of content must be done manually.
To that end, both |\ifchilddoc| and |\ifchilddocmanual|
will be true upon processing of a part,
and the name of the part is stored in |\childdocname|.
Note that |\jobname| will be set to the filename of the current part
so that each part receives an individual |.aux| file
that does not interfere with the |.aux| file(s) of the main document.
This behaviour can be altered by the alternative form
|\childdocby[*]{|\textit{main}|}| (with a non-empty optional argument)
which uses the |.aux| file of the main document
by setting |\jobname| to \textit{main}.

%%%%%%%%%%%%%%%%%%%%%%%%%%%%%%%%%%%%%%%%%%%%%%%%%%%%%%%%%%%%%%%%%%%%%%%%%%%%%%%%
\subsection{Driver Development}
\label{sec:driver}

The \textsf{childdoc} mechanism can also be use for the development
of definition files such as \LaTeX{} styles or classes.
This case differs from the above setup with multiple parts
included by |\include| in that no |\includeonly| should be invoked.
This can be achieved by starting the include file
(before |\ProvidesPackage|) with:
%
\begin{center}
\begin{tabular}{l}
|\input{childdoc.def}|\\
|\childdocforward{|\textit{main}|}|\\
\end{tabular}
\end{center}
%
or alternatively with:
%
\begin{center}
\begin{tabular}{l}
|\input{childdoc.def}|\\
|\childdocby{|\textit{main}|}|\\
\end{tabular}
\end{center}
%
Both forms have slightly different effects as described above.
The main file is prepared as usual, see \secref{sec:include}.

%%%%%%%%%%%%%%%%%%%%%%%%%%%%%%%%%%%%%%%%%%%%%%%%%%%%%%%%%%%%%%%%%%%%%%%%%%%%%%%%
\subsection{Legacy Detection}
\label{sec:detection}

The directive |\childdocmain| in the main file can detect
whether the complete document or merely a child is to be compiled
even without using the directive |\childdocof|.
This method is deprecated because it is less robust
and there is no compelling reason to use it;
it is merely provided for backward compatibility
and it may be removed in future versions.

If the detection mechanism is to be used,
it is mandatory to correctly specify
the filename of the main file as the argument of |\childdocmain|:
%
\begin{center}
\begin{tabular}{l}
|\input{childdoc.def}|\\
|\childdocmain{|\textit{main}|}|\\
\end{tabular}
\end{center}
%
If |\jobname| does not match the argument \textit{main} of |\childdocmain|,
it is assumed that |\jobname| points to the child file to be compiled.
When using |\childdocmain| with the main file specified as argument,
it suffices to start a child file
with just |\input{|\textit{main}|}|
without loading of the package and using |\childdocof|.
If instead all processing is done
with the appropriate \textsf{childdoc} directives,
the argument of \textit{main} of |\childdocmain| can be empty.

An alternative version of the command line processing described
in \secref{sec:commandline} using the detection mechanism reads:
%
\begin{center}
|... -jobname "|\textit{target}|" "|[\textit{flags}]%
[|\def\jobname{|\textit{dest}|}|]|\input{|\textit{main}|}"|
\end{center}

%%%%%%%%%%%%%%%%%%%%%%%%%%%%%%%%%%%%%%%%%%%%%%%%%%%%%%%%%%%%%%%%%%%%%%%%%%%%%%%%
\subsection{Manual Code}
\label{sec:manual}

In case one cannot be certain whether the definitions file |childdoc.def|
is installed on the target \TeX{} distribution
and one prefers not to ship it,
it is conceivable to paste a few relevant commands into the sources.

To that end, drop all statements |\input{childdoc.def}|
and perform the replacements as outlined below.
Instead of |\childdocmain{|\textit{main}|}| add the following code
to the top of the main file:
%
\begin{center}
\begin{tabular}{l}
|\||ifdefined\childdocname\endinput\||fi\newif\ifchilddoc|\\
|\edef\childdocname{\scantokens\expandafter{\jobname\noexpand}}|\\
|\def\childdocmain{|\textit{main}|}\||ifx\childdocmain\childdocname\||else|\\
|\childdoctrue\includeonly{\childdocname}\let\jobname\childdocmain\||fi|\\
\end{tabular}
\end{center}
%
Instead of |\childdocof{|\textit{main}|}| just include the main file
at the top of each child file:
%
\begin{center}
|\input{|\textit{main}|}|
\end{center}
%
A simple redirection |\childdocforward{|\textit{dest}|}| is achieved by:
%
\begin{center}
|\def\jobname{|\textit{dest}|}\input{\jobname}|
\end{center}
%
The redirection with prefix
|\childdocforwardprefix[|\textit{prefix}|]{|\textit{dest}|}|
is accomplished by:
%
\begin{center}
\begin{tabular}{l}
|{\edef\jobname{\scantokens\expandafter{\jobname\noexpand}}|\\
|\def\redirectjob |\textit{prefix}|#1~~~{\gdef\jobname{|\textit{dest}|#1}}|\\
|\expandafter\redirectjob\jobname~~~}\input{\jobname}|
\end{tabular}
\end{center}

In an alternative approach,
child documents can be compiled by a specific command line
without additional code or specific definitions:
%
\begin{center}
|... -jobname "|\textit{target}|" "|[\textit{flags}]%
|\includeonly{|\textit{dest}|}\input{|\textit{main}|}"|
\end{center}
%

%%%%%%%%%%%%%%%%%%%%%%%%%%%%%%%%%%%%%%%%%%%%%%%%%%%%%%%%%%%%%%%%%%%%%%%%%%%%%%%%
%%%%%%%%%%%%%%%%%%%%%%%%%%%%%%%%%%%%%%%%%%%%%%%%%%%%%%%%%%%%%%%%%%%%%%%%%%%%%%%%
\section{Information}

%%%%%%%%%%%%%%%%%%%%%%%%%%%%%%%%%%%%%%%%%%%%%%%%%%%%%%%%%%%%%%%%%%%%%%%%%%%%%%%%
\subsection{Copyright}

Copyright \copyright{} 2017--2018 Niklas Beisert

This work may be distributed and/or modified under the
conditions of the \LaTeX{} Project Public License, either version 1.3
of this license or (at your option) any later version.
The latest version of this license is in
  \url{http://www.latex-project.org/lppl.txt}
and version 1.3 or later is part of all distributions of \LaTeX{}
version 2005/12/01 or later.

This work has the LPPL maintenance status `maintained'.

The Current Maintainer of this work is Niklas Beisert.

This work consists of the files |README.txt|, |childdoc.ins| and |childdoc.dtx|
as well as the derived files |childdoc.def|, |cdocsamp.tex|
with |cdocsch1.tex|, |cdocsch2.tex|, |cdocspt3.tex|, |cdocspt4.tex|,
|cdocsdrf.tex|, |cdocsfn1.tex|, |cdocsfn2.tex|
as well as |childdoc.pdf|.

%%%%%%%%%%%%%%%%%%%%%%%%%%%%%%%%%%%%%%%%%%%%%%%%%%%%%%%%%%%%%%%%%%%%%%%%%%%%%%%%
\subsection{Files and Installation}

The package consists of the files:
%
\begin{center}
\begin{tabular}{ll}
    |README.txt|   & readme file \\
    |childdoc.ins| & installation file \\
    |childdoc.dtx| & source file \\
    |childdoc.def| & definition file \\
    |cdocsamp.tex| & sample main file \\
    |cdocsch1.tex| & sample include file \\
    |cdocsch2.tex| & sample include file \\
    |cdocspt3.tex| & sample part file \\
    |cdocspt4.tex| & sample part file \\
    |cdocsdrf.tex| & sample redirection file \\
    |cdocsfn1.tex| & sample redirection file \\
    |cdocsfn2.tex| & sample redirection file \\
    |childdoc.pdf| & manual
\end{tabular}
\end{center}
%
The distribution consists of the files
|README.txt|, |childdoc.ins| and |childdoc.dtx|.
%
\begin{itemize}
\item
Run (pdf)\LaTeX{} on |childdoc.dtx|
to compile the manual |childdoc.pdf| (this file).
\item
Run \LaTeX{} on |childdoc.ins| to create the definitions file |childdoc.def|
and the sample |cdocsamp.tex| with include files
|cdocsch1.tex|, |cdocsch2.tex|, |cdocspt3.tex|, |cdocspt4.tex|,
|cdocsdrf.tex|, |cdocsfn1.tex|, |cdocsfn2.tex|.
Then copy the file |childdoc.def| to an appropriate directory of your \LaTeX{}
distribution, e.g.\ \textit{texmf-root}|/tex/latex/childdoc|.
\end{itemize}

%%%%%%%%%%%%%%%%%%%%%%%%%%%%%%%%%%%%%%%%%%%%%%%%%%%%%%%%%%%%%%%%%%%%%%%%%%%%%%%%
\subsection{Related CTAN Packages}

There are several other packages which offer a similar functionality:
%
\begin{itemize}
\item
The packages
\href{http://ctan.org/pkg/docmute}{\textsf{docmute}},
\href{http://ctan.org/pkg/includex}{\textsf{includex}} and
\href{http://ctan.org/pkg/standalone}{\textsf{standalone}}
provide commands to include only the document body of
a child file thus allowing both files to be compiled individually.
\item
The packages \href{http://ctan.org/pkg/subdocs}{\textsf{subdocs}}
and \href{http://ctan.org/pkg/subfiles}{\textsf{subfiles}}
provide structures in which the main and child documents can be
encapsulated and allowing them to be compiled individually.
The inclusion mechanism is different from the conventional |\include|.
\item
The package \href{http://ctan.org/pkg/combine}{\textsf{combine}}
is an elaborate solution to combine several documents into one.
\end{itemize}
%
See also the CTAN topic \href{http://ctan.org/topic/subdocs}{\textsf{subdocs}}
for further related packages.
The present package differs from the above solutions in that
a document structure constructed with the conventional |\include| mechanism
just needs two extra commands at the top of every file
such that all constituent files can be compiled individually.

%%%%%%%%%%%%%%%%%%%%%%%%%%%%%%%%%%%%%%%%%%%%%%%%%%%%%%%%%%%%%%%%%%%%%%%%%%%%%%%%
%\subsection{Feature Suggestions}
%
%The following is a list of features which may be useful for future
%versions of this package:
%%
%\begin{itemize}
%\item
%\ldots
%\end{itemize}

%%%%%%%%%%%%%%%%%%%%%%%%%%%%%%%%%%%%%%%%%%%%%%%%%%%%%%%%%%%%%%%%%%%%%%%%%%%%%%%%
\subsection{Revision History}

%%%%%%%%%%%%%%%%%%%%%%%%%%%%%%%%%%%%%%%%
\paragraph{v2.0:} 2018/12/30

\begin{itemize}
\item
immediate forward processing
\item
added |\childdocby| mechanism
\item
manual restructured
\end{itemize}

%%%%%%%%%%%%%%%%%%%%%%%%%%%%%%%%%%%%%%%%
\paragraph{v1.6:} 2018/01/17

\begin{itemize}
\item
application for development of include files
\item
corrections to manual
\end{itemize}

%%%%%%%%%%%%%%%%%%%%%%%%%%%%%%%%%%%%%%%%
\paragraph{v1.5:} 2017/05/21

\begin{itemize}
\item
more complete structuring introduced
\item
|\childdocof| introduced
\item
|\childdoc| renamed to |\childdocmain|
\item
|\childredirect| renamed to |\childdocforward| and |\childdocforwardprefix|
and functionality expanded
\end{itemize}

%%%%%%%%%%%%%%%%%%%%%%%%%%%%%%%%%%%%%%%%
\paragraph{v1.0:} 2017/04/27

\begin{itemize}
\item
manual and install package
\item
first version published on CTAN
\end{itemize}

%%%%%%%%%%%%%%%%%%%%%%%%%%%%%%%%%%%%%%%%
\paragraph{v0.6:} 2017/04/26

\begin{itemize}
\item
redirection mechanism added
\end{itemize}

%%%%%%%%%%%%%%%%%%%%%%%%%%%%%%%%%%%%%%%%
\paragraph{v0.5:} 2017/04/26

\begin{itemize}
\item
functionality in definition file
\end{itemize}


%%%%%%%%%%%%%%%%%%%%%%%%%%%%%%%%%%%%%%%%%%%%%%%%%%%%%%%%%%%%%%%%%%%%%%%%%%%%%%%%
%%%%%%%%%%%%%%%%%%%%%%%%%%%%%%%%%%%%%%%%%%%%%%%%%%%%%%%%%%%%%%%%%%%%%%%%%%%%%%%%
%%%%%%%%%%%%%%%%%%%%%%%%%%%%%%%%%%%%%%%%%%%%%%%%%%%%%%%%%%%%%%%%%%%%%%%%%%%%%%%%
\appendix

\settowidth\MacroIndent{\rmfamily\scriptsize 000\ }

 \DocInput{childdoc.dtx}

\end{document}
%</driver>
% \fi
%
% %%%%%%%%%%%%%%%%%%%%%%%%%%%%%%%%%%%%%%%%%%%%%%%%%%%%%%%%%%%%%%%%%%%%%%%%%%%%%%
% %%%%%%%%%%%%%%%%%%%%%%%%%%%%%%%%%%%%%%%%%%%%%%%%%%%%%%%%%%%%%%%%%%%%%%%%%%%%%%
% \section{Sample}
%\iffalse
%<*samplemain>
%\fi
%
% The following presents a sample document
% with two chapters, two parts, a title page,
% a compile flag as well as three forwarding files to set the flag.
% It consists of eight |.tex| files:
% \begin{center}
% \begin{tabular}{ll}
% |cdocsamp.tex|&main file\\
% |cdocsch1.tex|&include file for chapter 1\\
% |cdocsch2.tex|&include file for chapter 2\\
% |cdocspt3.tex|&include file for part 3\\
% |cdocspt4.tex|&include file for part 4\\
% |cdocsdrf.tex|&forwarding file for main file in draft mode\\
% |cdocsfi1.tex|&forwarding file for final version of chapter 1\\
% |cdocsfi2.tex|&forwarding file for final version of chapter 2\\
% \end{tabular}
% \end{center}
% Each of the eight files can be compiled directly by the \LaTeX{} compiler.
%
% %%%%%%%%%%%%%%%%%%%%%%%%%%%%%%%%%%%%%%
% \paragraph{Main File.}
%
% The main file is called |cdocsamp.tex|.
%
% Load the \textsf{childdoc} definitions and
% declare the filename for the main document:
%    \begin{macrocode}
\input{childdoc.def}
\childdocmain{}
%    \end{macrocode}

% Optional override for |\version| flag:
%    \begin{macrocode}
%%\ifchilddoc\else\providecommand{\version}{draft}\fi
%    \end{macrocode}

% Define the default values for the |\version| flag
% (|final| for the main file and |draft| for childs):
%    \begin{macrocode}
\ifchilddoc
\providecommand{\version}{draft}
\else
\providecommand{\version}{final}
\fi
%    \end{macrocode}

% Load the standard document class:
%    \begin{macrocode}
\documentclass[12pt]{article}
%    \end{macrocode}

% Start the document body:
%    \begin{macrocode}
\begin{document}
%    \end{macrocode}

% Declare a title page.
% Print title, part of document being processed and version flag:
%    \begin{macrocode}
\addtocounter{page}{-1}
\begin{center}
{\LARGE\bfseries{}childdoc example\par}
\vspace{1cm}
\ifchilddoc
\ifchilddocmanual part\else chapter\fi:
`\childdocname' of `\childdocjob'\par
\else
main document: `\childdocjob'\par
\fi
version: \version\par
\end{center}
\newpage
%    \end{macrocode}

% Manually include selected file,
% otherwise process as usual:
%    \begin{macrocode}
\ifchilddocmanual
\section*{part `\childdocname'}
\input{\childdocname}
\else
%    \end{macrocode}

% Include the two chapters:
%    \begin{macrocode}
\include{cdocsch1}
\include{cdocsch2}
%    \end{macrocode}

% Include the two parts unless only chapters should be displayed:
%    \begin{macrocode}
\ifchilddoc\else
\section{part three}
\input{cdocspt3}
\section{part four}
\input{cdocspt4}
\fi
%    \end{macrocode}

% Process as usual until here:
%    \begin{macrocode}
\fi
%    \end{macrocode}

% End of document body:
%    \begin{macrocode}
\end{document}
%    \end{macrocode}
%\iffalse
%</samplemain>
%\fi
%
% %%%%%%%%%%%%%%%%%%%%%%%%%%%%%%%%%%%%%%
% \paragraph{Chapter Include Files.}
%
% The include files are called |cdocsch1.tex| and |cdocsch2.tex|.
%
%\iffalse
%<*samplechap1|samplechap2>
%\fi

% Optional override for |\version| flag:
%    \begin{macrocode}
%%\providecommand{\version}{final}
%    \end{macrocode}

% Include the main document:
%    \begin{macrocode}
\input{childdoc.def}
\childdocof{cdocsamp}
%    \end{macrocode}

%\iffalse
%</samplechap1|samplechap2>
%\fi
%
%\iffalse
%<*samplechap1>
%\fi
% Some text for chapter 1:
%    \begin{macrocode}
\section{one}
some text in chapter one
%    \end{macrocode}

%\iffalse
%</samplechap1>
%\fi
% Some text for chapter 2:
%\iffalse
%<*samplechap2>
%\fi
%    \begin{macrocode}
\section{two}
more text in chapter two
%    \end{macrocode}

%\iffalse
%</samplechap2>
%\fi
%
% %%%%%%%%%%%%%%%%%%%%%%%%%%%%%%%%%%%%%%
% \paragraph{Part Include Files.}
%
% The include files are called |cdocspt3.tex| and |cdocspt4.tex|.
%
%\iffalse
%<*samplepart3|samplepart4>
%\fi

% Optional override for |\version| flag:
%    \begin{macrocode}
%%\providecommand{\version}{final}
%    \end{macrocode}

% Include the main document:
%    \begin{macrocode}
\input{childdoc.def}
\childdocby{cdocsamp}
%    \end{macrocode}

%\iffalse
%</samplepart3|samplepart4>
%\fi
%
%\iffalse
%<*samplepart3>
%\fi
% Some text for part 3:
%    \begin{macrocode}
some text in part three
%    \end{macrocode}

%\iffalse
%</samplepart3>
%\fi
% Some text for part 4:
%\iffalse
%<*samplepart4>
%\fi
%    \begin{macrocode}
more text in part four
%    \end{macrocode}

%\iffalse
%</samplepart4>
%\fi
%
% %%%%%%%%%%%%%%%%%%%%%%%%%%%%%%%%%%%%%%
% \paragraph{Forwarding for a Complete Draft.}
%
% The following forwarding file |cdocsdrf.tex|
% compiles the main document in draft mode:
%\iffalse
%<*sampledraft>
%\fi
%    \begin{macrocode}
\def\version{draft}
\input{childdoc.def}
\childdocforward{cdocsamp}
%    \end{macrocode}

%\iffalse
%</sampledraft>
%\fi
%
% %%%%%%%%%%%%%%%%%%%%%%%%%%%%%%%%%%%%%%
% \paragraph{Forwarding for Final Version of the Chapters.}
%
% The following forwarding files |cdocsfn1.tex| and |cdocsfn2.tex|
% (with identical content)
% compile the final versions of the child documents
% |cdocsch1.tex| and |cdocsch2.tex|, respectively:
%\iffalse
%<*samplefinal>
%\fi
%    \begin{macrocode}
\def\version{final}
\input{childdoc.def}
\childdocforwardprefix[cdocsamp]{cdocsfn}{cdocsch}
%    \end{macrocode}

%\iffalse
%</samplefinal>
%\fi
%
% %%%%%%%%%%%%%%%%%%%%%%%%%%%%%%%%%%%%%%
% \paragraph{Command Line Processing.}
%
% The following three command lines generate the output files
% |cdocscld|, |cdocscl1| and |cdocscl2|
% which should be identical to
% |cdocsdrf|, |cdocsch1| and |cdocsfn2|, respectively:
% \begin{center}
% \begin{tabular}{l}
% |latex -jobname cdocscld \|\\
% |  "\def\version{draft}\input{childdoc.def}\childdocforward{cdocsamp}"|\\
% |latex -jobname cdocscl1 \|\\
% |  "\input{childdoc.def}\childdocforward[cdocsamp]{cdocsch1}"|\\
% |latex -jobname cdocscl2 \|\\
% |  "\def\version{final}\input{childdoc.def}\childdocforward{cdocsch2}"|
% \end{tabular}
% \end{center}
% Note that the trailing backslash on each first line
% merely continues the input to the second line
% (for convenient cut ant paste).
% Furthermore, the command |latex| can be replaced by any
% of its alternative versions such as |pdflatex|.
%
% %%%%%%%%%%%%%%%%%%%%%%%%%%%%%%%%%%%%%%%%%%%%%%%%%%%%%%%%%%%%%%%%%%%%%%%%%%%%%%
% %%%%%%%%%%%%%%%%%%%%%%%%%%%%%%%%%%%%%%%%%%%%%%%%%%%%%%%%%%%%%%%%%%%%%%%%%%%%%%
% \section{Implementation}
%\iffalse
%<*package>
%\fi
%
% This section describes the definitions file |childdoc.def|.

% The definitions cannot be loaded using |\usepackage| or |\RequirePackage|
% which has a mechanism to prevent loading a style file more than once.
% When loading the definitions by means of |\input|
% multiple instances have to be prevented manually:
%\iffalse
%This code needs to be before the `\ProvidesFile' directive
%which is defined at the beginning of this file.
%Therefore it is also placed there and commented out here.
%</package>
%<*discard>
%\fi
%    \begin{macrocode}
\ifdefined\childdocmain\endinput\fi
%    \end{macrocode}
%\iffalse
%</discard>
%<*package>
%\fi
%
% \macro{\ifchilddoc}
% \macro{\ifchilddocmanual}
% The conditional |\ifchilddoc| tells whether a
% child (true) or main (false) document is being compiled.
% The conditional |\ifchilddocmanual| tells whether
% the |\includeonly| mechanism is used (false) or
% the selection of child files must be performed manually (true).
% The definitions initialise to false:
%    \begin{macrocode}
\newif\ifchilddoc
\newif\ifchilddocmanual
%    \end{macrocode}

% \macro{\childdocname}
% \macro{\childdocjob}
% The macro |\childdocname| stores the name of the main document
% to be compiled. The macro |\childdocjob| stores the name of
% the document on which the \LaTeX{} compiler was originally invoked.
% The content of |\jobname| cannot be compared
% to filenames specified in the source due to different catcodes.
% The following code rescans |\jobname|, stores the result
% in |\childdocname| and saves a copy in |\childdocjob|:
%    \begin{macrocode}
\edef\childdocname{\scantokens\expandafter{\jobname\noexpand}}
\let\childdocjob\childdocname
%    \end{macrocode}

% \macro{\childdocdisable}
% The macro |\childdocdisable| prevents the main file
% from being processed more than once.
% At this stage, the main document command |\childdocmain|
% is assumed to be called once again where it should do nothing.
% Any subsequent call to it should prevent
% a secondary processing of the main document
% It overwrites the forwarding commands
% |\childdocof| and |\childdocforward|
% with empty macros to prevent further inclusions of the main document:
%    \begin{macrocode}
\newcommand{\childdocdisable}
{
  \renewcommand{\childdocmain}[1]{\renewcommand{\childdocmain}[1]{\endinput}}
  \renewcommand{\childdocof}[1]{}
  \renewcommand{\childdocby}[2][]{}
  \renewcommand{\childdocforward}[2][]{}
  \renewcommand{\childdocdisable}{}
}
%    \end{macrocode}

% \macro{\childdocmain}
% The macro |\childdocmain| is to be called at the top of the main file
% with nothing or the main filename (without extension) as argument.
% First, it breaks loops.
% If the argument is not empty and does not match |\childdocname|
% (which is set by the first inclusion of |childdoc.def|),
% |\ifchilddoc| is set to true, |\includeonly| is applied to the child file
% and |\jobname| is set to the main file
% (for proper handling of |.aux| files):
%    \begin{macrocode}
\newcommand{\childdocmain}[1]
{
  \childdocdisable\childdocmain{}
  \if?#1?\else
    \begingroup
      \def\childdoctmp{#1}
      \ifx\childdoctmp\childdocname
        \def\childdoctmp{}
      \else
        \def\childdoctmp
        {
          \childdoctrue
          \includeonly{\childdocname}
          \def\childdocjob{#1}
          \def\jobname{#1}
        }
      \fi
      \expandafter
    \endgroup
    \childdoctmp
  \fi
}
%    \end{macrocode}

% \macro{\childdocof}
% The command |\childdocof| redirects
% compilation to the main file |#1|.
%    \begin{macrocode}
\newcommand{\childdocof}[1]
{
  \childdocdisable
  \childdoctrue
  \includeonly{\childdocname}
  \def\jobname{#1}
  \def\childdocjob{#1}
  \input{#1}
}
%    \end{macrocode}

% \macro{\childdocby}
% The command |\childdocby| ....
%    \begin{macrocode}
\newcommand{\childdocby}[2][]
{
  \childdocdisable
  \childdoctrue
  \childdocmanualtrue
  \if?#1?\else
    \def\jobname{#2}
  \fi
  \def\childdocjob{#2}
  \input{#2}
  \endinput
}
%    \end{macrocode}

% \macro{\childdocforward}
% The command |\childdocforward| redirects
% compilation to the main file or
% (if the optional argument is given) a child file.
% Parameters are set as if the main file
% or a child file starting with |\childdocof| was compiled.
% Then compilation is handed over to the main file:
%    \begin{macrocode}
\newcommand{\childdocforward}[2][]
{
  \begingroup
    \if?#1?
      \def\childdoctmp
      {
        \def\childdocname{#2}
        \def\childdocjob{#2}
        \def\jobname{#2}
        \input{#2}
        \endinput
      }
    \else
      \def\childdoctmp
      {
        \childdocdisable
        \def\childdocname{#2}
        \childdoctrue
        \includeonly{#2}
        \def\childdocjob{#1}
        \def\jobname{#1}
        \input{#1}
        \endinput
      }
    \fi
    \expandafter
  \endgroup
  \childdoctmp
}
%    \end{macrocode}

% \macro{\childdocforwardprefix}
% The command |\childdocforwardprefix| redirects
% compilation to the main or a child file by means of a pattern.
% The prefix |#1| in the current filename is replaced by |#2|
% and the suffix of the current filename is kept
% (it is assumed that the filename does not contain the substring `|~~~|'
% which is used as a delimiter).
% Compilation is handed over to the new file by |\childdocforward|:
%    \begin{macrocode}
\newcommand{\childdocforwardprefix}[3][]
{
  \begingroup
    \def\childdocextract #2##1~~~{\def\childdoctmp{\childdocforward[#1]{#3##1}}}
    \expandafter\childdocextract\childdocname~~~
    \expandafter
  \endgroup
  \childdoctmp
}
%    \end{macrocode}

% \macro{\childdoc}
% The deprecated macro |\childdoc| is a legacy version of |\childdocmain|:
%    \begin{macrocode}
\newcommand{\childdoc}{\childdocmain}
%    \end{macrocode}

% \macro{\childdocredirect}
% The deprecated macro |\childdocredirect| is a legacy version
% of |\childdocforward| and |\childdocforwardprefix|:
%    \begin{macrocode}
\newcommand{\childdocredirect}[2][]
{
  \begingroup
    \if?#1?
      \def\childdoctmp{\childdocforward{#2}}
    \else
      \def\childdoctmp{\childdocforwardprefix{#1}{#2}}
    \fi
    \expandafter
  \endgroup
  \childdoctmp
}
%    \end{macrocode}

%\iffalse
%</package>
%\fi
%
\endinput
|\\
|\childdocby{|\textit{main}|}|\\
\end{tabular}
\end{center}
%
Both forms have slightly different effects as described above.
The main file is prepared as usual, see \secref{sec:include}.

%%%%%%%%%%%%%%%%%%%%%%%%%%%%%%%%%%%%%%%%%%%%%%%%%%%%%%%%%%%%%%%%%%%%%%%%%%%%%%%%
\subsection{Legacy Detection}
\label{sec:detection}

The directive |\childdocmain| in the main file can detect
whether the complete document or merely a child is to be compiled
even without using the directive |\childdocof|.
This method is deprecated because it is less robust
and there is no compelling reason to use it;
it is merely provided for backward compatibility
and it may be removed in future versions.

If the detection mechanism is to be used,
it is mandatory to correctly specify
the filename of the main file as the argument of |\childdocmain|:
%
\begin{center}
\begin{tabular}{l}
|% \iffalse
%
% childdoc.dtx Copyright (C) 2017-2018 Niklas Beisert
%
% This work may be distributed and/or modified under the
% conditions of the LaTeX Project Public License, either version 1.3
% of this license or (at your option) any later version.
% The latest version of this license is in
%   http://www.latex-project.org/lppl.txt
% and version 1.3 or later is part of all distributions of LaTeX
% version 2005/12/01 or later.
%
% This work has the LPPL maintenance status `maintained'.
%
% The Current Maintainer of this work is Niklas Beisert.
%
% This work consists of the files childdoc.dtx and childdoc.ins
% and the derived files childdoc.def and cdocsamp.tex with
% cdocsch1.tex, cdocsch2.tex, cdocsdrf.tex, cdocsfn1.tex, cdocsfn2.tex.
%
%<package>\ifdefined\childdocmain\endinput\fi
%<package>\ProvidesFile{childdoc.def}[2018/12/30 v2.0 child document driver]
%<samplemain>\ProvidesFile{cdocsamp.tex}[2018/12/30 v2.0 sample for childdoc]
%<*driver>
%\ProvidesFile{childdoc.drv}[2018/12/30 v2.0 childdoc reference manual file]
\PassOptionsToClass{10pt,a4paper}{article}
\documentclass{ltxdoc}

\usepackage[margin=35mm]{geometry}
\usepackage{hyperref}
\usepackage{hyperxmp}
\usepackage[usenames]{color}

\hypersetup{colorlinks=true}
\hypersetup{pdfstartview=FitH}
\hypersetup{pdfpagemode=UseNone}
\hypersetup{pdfsource={}}
\hypersetup{pdflang={en-UK}}
\hypersetup{pdfcopyright={Copyright 2017-2018 Niklas Beisert.
  This work may be distributed and/or modified under the
  conditions of the LaTeX Project Public License, either version 1.3
  of this license or (at your option) any later version.}}
\hypersetup{pdflicenseurl={http://www.latex-project.org/lppl.txt}}
\hypersetup{pdfcontactaddress={ETH Zurich, ITP, HIT K,
  Wolfgang-Pauli-Strasse 27}}
\hypersetup{pdfcontactpostcode={8093}}
\hypersetup{pdfcontactcity={Zurich}}
\hypersetup{pdfcontactcountry={Switzerland}}
\hypersetup{pdfcontactemail={nbeisert@itp.phys.ethz.ch}}
\hypersetup{pdfcontacturl={http://people.phys.ethz.ch/\xmptilde nbeisert/}}

\newcommand{\secref}[1]{\hyperref[#1]{section \ref*{#1}}}

\parskip1ex
\parindent0pt
\let\olditemize\itemize
\def\itemize{\olditemize\parskip0pt}

\begin{document}

\title{The \textsf{childdoc} Package}
\hypersetup{pdftitle={The childdoc Package}}
\author{Niklas Beisert\\[2ex]
  Institut f\"ur Theoretische Physik\\
  Eidgen\"ossische Technische Hochschule Z\"urich\\
  Wolfgang-Pauli-Strasse 27, 8093 Z\"urich, Switzerland\\[1ex]
  \href{mailto:nbeisert@itp.phys.ethz.ch}
  {\texttt{nbeisert@itp.phys.ethz.ch}}}
\hypersetup{pdfauthor={Niklas Beisert}}
\hypersetup{pdfsubject={Manual for the LaTeX2e Package childdoc}}
\date{30 December 2018, \textsf{v2.0}}
\maketitle

\begin{abstract}\noindent
\textsf{childdoc} is a \LaTeXe{} package
that enables the direct compilation
of document sections included by |\include|
to individual files.
\end{abstract}

\begingroup
\parskip0ex
\tableofcontents
\endgroup

%%%%%%%%%%%%%%%%%%%%%%%%%%%%%%%%%%%%%%%%%%%%%%%%%%%%%%%%%%%%%%%%%%%%%%%%%%%%%%%%
%%%%%%%%%%%%%%%%%%%%%%%%%%%%%%%%%%%%%%%%%%%%%%%%%%%%%%%%%%%%%%%%%%%%%%%%%%%%%%%%
\section{Introduction}

\LaTeX{} provides a mechanism to structure a large document (such as a book)
into a main file and several child files (containing the chapters)
using the |\include| command.
This mechanism is beneficial for documents
which span hundreds of pages in order to
make the source file(s) more manageable.
Moreover, compilation can be restricted to
selected child files by means of the |\includeonly| command.
The latter feature can be used to reduce the compilation time while editing
(this was significantly more useful in the earlier days of \LaTeX{})
or to generate a smaller document which is easier to navigate.
Another application of |\includeonly| is to generate
documents consisting of selected parts of the complete document.

However, there are a few drawbacks of the plain |\include| mechanism:
\begin{itemize}
\item
The child files cannot be compiled on their own,
they can only be compiled via the main file.
A naive editing environment
(such as a text editor with an option
to have the current file processed by \LaTeX)
may require one to switch to the main file before compiling;
attempting to compile the child file produces errors.
\item
The main file must be modified (each time)
to adjust the |\includeonly| command
to the present needs. This easily leaves the main file in a messy state.
\item
The generated document will always carry the filename
of the main document. This is inconvenient if
several child files are to be compiled and
to be kept for distribution.
\end{itemize}

The present package provides a simple interface
to make child files individually compilable by \LaTeX{}.
Compiling a child file then has the same effect as compiling
the main file with an |\includeonly| command
to select the appropriate child.
Moreover the generated document will carry the name of the child
rather than the main file.
This resolves all three above issues.

This feature is meant to make the editing of books,
thesis documents and lecture notes somewhat more convenient.
However, the package can also be used efficiently for
composing a series of documents (such as exercise sheets)
which are typically distributed individually.
It then assists the author in generating the individual documents
(potentially in different versions)
as well as a document containing the collected series.
Another application is in developing style files
or other kinds of included material
where compilation of the style file could redirect
to a sample or test file.

%%%%%%%%%%%%%%%%%%%%%%%%%%%%%%%%%%%%%%%%%%%%%%%%%%%%%%%%%%%%%%%%%%%%%%%%%%%%%%%%
%%%%%%%%%%%%%%%%%%%%%%%%%%%%%%%%%%%%%%%%%%%%%%%%%%%%%%%%%%%%%%%%%%%%%%%%%%%%%%%%
\section{Usage}

First of all, the package \textsf{childdoc} is \emph{not} a standard
\LaTeXe{} |.sty| style file! Therefore it needs to be invoked in
a non-standard way.

%%%%%%%%%%%%%%%%%%%%%%%%%%%%%%%%%%%%%%%%%%%%%%%%%%%%%%%%%%%%%%%%%%%%%%%%%%%%%%%%
\subsection{Included Files}
\label{sec:include}

%%%%%%%%%%%%%%%%%%%%%%%%%%%%%%%%%%%%%%%%
\DescribeMacro{\childdocmain}
To use the package, add the commands
\begin{center}
\begin{tabular}{l}
|\input{childdoc.def}|\\
|\childdocmain{}|\\
\end{tabular}
\end{center}
at the very top of the main \LaTeX{} file,
in particular \emph{before} the |\documentclass| statement!
The argument of |\childdocmain| should be left empty
(but it must be present).

%%%%%%%%%%%%%%%%%%%%%%%%%%%%%%%%%%%%%%%%
\DescribeMacro{\childdocof}
Furthermore, add the commands
\begin{center}
\begin{tabular}{l}
|\input{childdoc.def}|\\
|\childdocof{|\textit{main}|}|\\
\end{tabular}
\end{center}
at the top of every child file \textit{child}
which is included by |\include{|\textit{child}|}|
from within the main file
(or at least for those files to be compiled individually).
The argument \textit{main} must be the filename of the main file.

There are a couple of
considerations in setting up the main and child documents:

%%%%%%%%%%%%%%%%%%%%%%%%%%%%%%%%%%%%%%%%
\paragraph{Restrictions.}

Please note the following restrictions:
\begin{itemize}
\item
|\childdocmain| must be called with one argument \textit{main}
to ensure compatibility with earlier version of the package.
It must either be empty (|\childdocmain{}|)
or precisely match the filename of the main file in which it is specified.
See \secref{sec:detection} for further information.
\item
The filename \textit{main} must be specified without the |.tex| extension.
\item
The filename \textit{main} is case sensitive
(even in case-insensitive file systems)
due to internal string comparison.
\item
The argument \textit{main} should be fully expanded, it cannot be a macro.
\item
Subdirectories and special characters should be avoided in filenames.
\item
The command |\childdocmain{|\textit{main}|}| must be followed by a whitespace.
It should not be followed immediately by another command
or by a comment mark `|%|'.
This is because the \TeX{} parser reads the token immediately following
the argument of |\childdocmain| and puts it
at the beginning of every child section;
however, a white\-space is ignored.
\end{itemize}

%%%%%%%%%%%%%%%%%%%%%%%%%%%%%%%%%%%%%%%%
\paragraph{Content of Main File.}

It is advisable to place all content in the child files included by |\include|.
Any output contained in the main file will appear in all child documents
unless suppressed manually;
it cannot be suppressed automatically by the |\includeonly| directive
and thus should normally be avoided.
A method to include some content in the main file
by means of conditional processing is described in \secref{sec:conditional}.

%%%%%%%%%%%%%%%%%%%%%%%%%%%%%%%%%%%%%%%%
\paragraph{Page Numbering.}

When only a part of the document is compiled,
the appropriate numbering of pages
(as well as other status parameters)
is determined from the |.aux| files.
The latter contain information from previous passes.
However this information needs to propagate through
all intermediate child documents.
Therefore the page numbering in child documents may well
be inconsistent until the complete document is compiled at least once.

A useful (if unconventional) way to always ensure a consistent
page numbering is to restart the numbering in each child document
and denote the pages by `\textit{child}|.|\textit{page}'
where \textit{child} represents the chapter/section number of the child file.
This can be achieved by the command
|\numberwithin{page}{|\textit{child}|}|
of the \textsf{amsmath} package
where \textit{child} can be |chapter| or |section|
depending on the chosen structuring.
Alternatively, one can modify the macro |\thepage| appropriately
and reset the counter |page| at the start of each child file.

%%%%%%%%%%%%%%%%%%%%%%%%%%%%%%%%%%%%%%%%%%%%%%%%%%%%%%%%%%%%%%%%%%%%%%%%%%%%%%%%
\subsection{Conditional Processing}
\label{sec:conditional}

The package provides a mechanism to compile different versions
of a document. To customise the versions further some conditional processing
can come in handy to distinguish which version is being compiled.
The package provides two macros to describe the compilation context:

%%%%%%%%%%%%%%%%%%%%%%%%%%%%%%%%%%%%%%%%
\DescribeMacro{\ifchilddoc}
The conditional |\ifchilddoc| distinguishes between the compilation of
child documents and the main document:
%
\begin{center}
|\ifchilddoc |\textit{child-code}| |[|\||else |\textit{main-code}]| \||fi|
\end{center}

%%%%%%%%%%%%%%%%%%%%%%%%%%%%%%%%%%%%%%%%
\DescribeMacro{\childdocname}
\DescribeMacro{\childdocjob}
The macro |\childdocname| contains the filename (without extension)
of the main or child file being processed.
Note that |\childdocjob| will always contain the name of the main file.

%%%%%%%%%%%%%%%%%%%%%%%%%%%%%%%%%%%%%%%%
\paragraph{Title Page.}

Conditional processing can be used to include a title or banner page
in the main document when proper precautions are taken.
Importantly, the code in the main file should ensure that the page counter
(as well as other status parameters which are stored in the |.aux| files)
takes the same value after the conditional processing.
Otherwise the page numbers may take divergent values
depending on which part is compiled.

For example, a title page could be declared by:
%
\begin{center}
\begin{tabular}{l}
|\ifchilddoc\||else|\\
|\addtocounter{page}{-1}|\\
\textit{code for title page}\\
|\newpage|\\
|\||fi|
\end{tabular}
\end{center}
%
A banner page for the child documents can be generated by:
%
\begin{center}
\begin{tabular}{l}
|\ifchilddoc|\\
|\addtocounter{page}{-1}|\\
\textit{code for banner page}\\
|\newpage|\\
|\||fi|
\end{tabular}
\end{center}
%
Here one could write a message such as:
\begin{center}
|This is the part \childdocname{} of \childdocjob{}.|
\end{center}

%%%%%%%%%%%%%%%%%%%%%%%%%%%%%%%%%%%%%%%%%%%%%%%%%%%%%%%%%%%%%%%%%%%%%%%%%%%%%%%%
\subsection{Flags}
\label{sec:flags}

The package makes it easy to generate different versions
of the main or child documents.
To this end compilation flags can be defined
and assigned different default values.
They will be particularly useful in conjunction
with the forwarding mechanism described in \secref{sec:forward}.

For example, it may be useful to have a flag |\version|
which can be set to |draft| or |final|.
The document source will contain some conditional code
depending on the value of |\version|.
Suppose further, the flag should default to |final| for the main file
and to |draft| for child files
which is a natural assignment for editing the document.
This is achieved by placing the following code
in the preamble of the main document
(below the |\childdocmain| directive):
%
\begin{center}
\begin{tabular}{l}
|\ifchilddoc|\\
|\providecommand{\version}{draft}|\\
|\||else|\\
|\providecommand{\version}{final}|\\
|\||fi|
\end{tabular}
\end{center}
%
The definition by |\providecommand| makes sure
that previous definitions are not overwritten.
Further statements |\providecommand{\version}{...}|
can thus be added before the above code to override it.

For the main file, one might add a line
(between |\childdocmain| and the above block)
%
\begin{center}
|%\ifchilddoc\||else\providecommand{\version}{draft}\||fi|
\end{center}
%
which can be uncommented to produce a draft version.
Likewise one can add a line to the very top of a child file
(above the |\childdocof{|\textit{main}|}| directive)
%
\begin{center}
|%\providecommand{\version}{final}|
\end{center}
%
which can be uncommented to produce the final version of this child document.

%%%%%%%%%%%%%%%%%%%%%%%%%%%%%%%%%%%%%%%%%%%%%%%%%%%%%%%%%%%%%%%%%%%%%%%%%%%%%%%%
\subsection{Forwarding}
\label{sec:forward}

Different versions of the main or child documents
using compilation flags as described in \secref{sec:flags}
can be (permanently) stored in different files
for convenient compilation, viewing and distribution.
To this end, the package defines a command
to pass on compilation to a different file:

%%%%%%%%%%%%%%%%%%%%%%%%%%%%%%%%%%%%%%%%
\DescribeMacro{\childdocforward}
The command |\childdocforward| redirects processing to
another source file:
%
\begin{center}
\begin{tabular}{l}
|\input{childdoc.def}|\\
|\childdocforward[|\textit{main}|]{|\textit{dest}|}|\\
\end{tabular}
\end{center}
%
The argument \textit{dest} is the destination file
(without extension).
It should be the main file or one of the child files.
Note that further \textsf{childdoc} directives
such as |\childdocof| and |\childdocforward|
in the indicated file will be processed in this form.
The optional argument \textit{main}
passes on directly to the main file \textit{main}
while pretending to compile the child \textit{dest}.
This form behaves as if \textit{dest}
issues |\childdocof{|\textit{main}|}| right away,
and no further \textsf{childdoc} directives will be processed.

%%%%%%%%%%%%%%%%%%%%%%%%%%%%%%%%%%%%%%%%
\DescribeMacro{\...prefix}
In the alternative form |\childdocforwardprefix|,
%
\begin{center}
\begin{tabular}{l}
|\input{childdoc.def}|\\
|\childdocforwardprefix[|\textit{main}|]{|\textit{prefix}|}{|\textit{dest}|}|
\end{tabular}
\end{center}
%
the destination file is determined by a pattern
depending on the current file:
To make this work, the current file must be called
`{\textit{prefix}\hspace{0.2em}\textit{suffix}}'
with \textit{prefix} matching precisely the argument.
Processing is then passed on to the file
`{\textit{dest}\hspace{0.2em}\textit{suffix}}'.
Surely, the same effect is achieved by
directly specifying the
argument `{\textit{dest}\hspace{0.2em}\textit{suffix}}'
in the first form.
However, that requires to set up a different file
for each child. With the alternative form of the command
all these files can have exactly the same content
which simplifies setting them up and maintaining them.

For example, the following file |draft.tex|
with a compilation flag |\version| as described in \secref{sec:flags}
compiles the main document as a draft:
%
\begin{center}
\begin{tabular}{l}
|\def\version{draft}|\\
|\input{childdoc.def}|\\
|\childdocforward{|\textit{main}|}|
\end{tabular}
\end{center}
%
Likewise, the following files |final|\textit{nn}|.tex|
compile the final version of the child document
|child|\textit{nn}|.tex|:
%
\begin{center}
\begin{tabular}{l}
|\def\version{final}|\\
|\input{childdoc.def}|\\
|\childdocforwardprefix{final}{child}|
\end{tabular}
\end{center}
%

Note that when several versions of a main file and/or of each child file
are to be generated, it may be convenient to set up a |Makefile| or
shell script to automatise the process.

%%%%%%%%%%%%%%%%%%%%%%%%%%%%%%%%%%%%%%%%%%%%%%%%%%%%%%%%%%%%%%%%%%%%%%%%%%%%%%%%
\subsection{Command Line Processing}
\label{sec:commandline}

The effect of redirection files can also be achieved by invoking
the \LaTeX{} compiler with a more elaborate command line.
Most conveniently this should be done as part
of a shell script or a |Makefile|.

When using \textsf{childdoc} in the main file, the following
command lines effectively perform a redirection
(note that depending on the shell being used,
backslashes may have to be doubled: `|\|' $\to$ `|\\|'):
%
\begin{center}
|... -jobname "|\textit{target}|" |\\|"|[\textit{flags}]%
|\input{childdoc.def}\childdocforward[|\textit{main}|]{|\textit{dest}|}"|
\end{center}
%
Here \textit{target} is the name of the output file,
\textit{main} is the name of the main file
and \textit{dest} is the name of the main or child file to be processed
(all filenames without extensions).
The optional argument \textit{main} can be omitted
if \textit{main} matches \textit{dest}.
Optionally, compilation \textit{flags} can be defined via |\def| commands.
This command line makes the \TeX{} engine believe
it is compiling the file \textit{target}
whose content is specified as the latter parameter.
The provided code then forwards the processing to
\textit{main} or \textit{dest} as described in \secref{sec:forward}.

%%%%%%%%%%%%%%%%%%%%%%%%%%%%%%%%%%%%%%%%%%%%%%%%%%%%%%%%%%%%%%%%%%%%%%%%%%%%%%%%
\subsection{Include by Input}
\label{sec:input}

Including child documents by |\include| has some restrictions by design.
Most notably, the content of a child document always occupies
its own set of pages; pages cannot be shared between child documents.
Usually, this behaviour makes perfect sense
because each child document contain an essential part of the document.
However, in some situations it may be desirable to compose
a document from a collection of parts
without having mandatory page breaks between then.
For this case, the package
provides a mechanism to include parts
by |\input| which can also be processed individually.
However, by construction this mechanism
requires manual handling of the content to be output.

%%%%%%%%%%%%%%%%%%%%%%%%%%%%%%%%%%%%%%%%
\DescribeMacro{\ifchilddocmanual}
The main file should be prepared as usual, see \secref{sec:include}.
However, the document body must make a distinction
between processing of an individual part and of the main document, e.g.:
%
\begin{center}
\begin{tabular}{l}
|\ifchilddocmanual|\\
|\input{\childdocname}|\\
|\||else|\\
\textit{document body with }|\input{|\textit{part}|}|\\
|\||fi|
\end{tabular}
\end{center}
%
The conditional |\ifchilddocmanual| is true whenever
a part to be included by |\input| is being compiled,
and the name of the part is stored in |\childdocname|.

%%%%%%%%%%%%%%%%%%%%%%%%%%%%%%%%%%%%%%%%
\DescribeMacro{\childdocby}
Each part to be included by |\input| should start with:
%
\begin{center}
\begin{tabular}{l}
|\input{childdoc.def}|\\
|\childdocby{|\textit{main}|}|\\
\end{tabular}
\end{center}
%
The directive |\childdocby| is similar to |\childdocof|
described in \secref{sec:include},
but the subsequent selection of content must be done manually.
To that end, both |\ifchilddoc| and |\ifchilddocmanual|
will be true upon processing of a part,
and the name of the part is stored in |\childdocname|.
Note that |\jobname| will be set to the filename of the current part
so that each part receives an individual |.aux| file
that does not interfere with the |.aux| file(s) of the main document.
This behaviour can be altered by the alternative form
|\childdocby[*]{|\textit{main}|}| (with a non-empty optional argument)
which uses the |.aux| file of the main document
by setting |\jobname| to \textit{main}.

%%%%%%%%%%%%%%%%%%%%%%%%%%%%%%%%%%%%%%%%%%%%%%%%%%%%%%%%%%%%%%%%%%%%%%%%%%%%%%%%
\subsection{Driver Development}
\label{sec:driver}

The \textsf{childdoc} mechanism can also be use for the development
of definition files such as \LaTeX{} styles or classes.
This case differs from the above setup with multiple parts
included by |\include| in that no |\includeonly| should be invoked.
This can be achieved by starting the include file
(before |\ProvidesPackage|) with:
%
\begin{center}
\begin{tabular}{l}
|\input{childdoc.def}|\\
|\childdocforward{|\textit{main}|}|\\
\end{tabular}
\end{center}
%
or alternatively with:
%
\begin{center}
\begin{tabular}{l}
|\input{childdoc.def}|\\
|\childdocby{|\textit{main}|}|\\
\end{tabular}
\end{center}
%
Both forms have slightly different effects as described above.
The main file is prepared as usual, see \secref{sec:include}.

%%%%%%%%%%%%%%%%%%%%%%%%%%%%%%%%%%%%%%%%%%%%%%%%%%%%%%%%%%%%%%%%%%%%%%%%%%%%%%%%
\subsection{Legacy Detection}
\label{sec:detection}

The directive |\childdocmain| in the main file can detect
whether the complete document or merely a child is to be compiled
even without using the directive |\childdocof|.
This method is deprecated because it is less robust
and there is no compelling reason to use it;
it is merely provided for backward compatibility
and it may be removed in future versions.

If the detection mechanism is to be used,
it is mandatory to correctly specify
the filename of the main file as the argument of |\childdocmain|:
%
\begin{center}
\begin{tabular}{l}
|\input{childdoc.def}|\\
|\childdocmain{|\textit{main}|}|\\
\end{tabular}
\end{center}
%
If |\jobname| does not match the argument \textit{main} of |\childdocmain|,
it is assumed that |\jobname| points to the child file to be compiled.
When using |\childdocmain| with the main file specified as argument,
it suffices to start a child file
with just |\input{|\textit{main}|}|
without loading of the package and using |\childdocof|.
If instead all processing is done
with the appropriate \textsf{childdoc} directives,
the argument of \textit{main} of |\childdocmain| can be empty.

An alternative version of the command line processing described
in \secref{sec:commandline} using the detection mechanism reads:
%
\begin{center}
|... -jobname "|\textit{target}|" "|[\textit{flags}]%
[|\def\jobname{|\textit{dest}|}|]|\input{|\textit{main}|}"|
\end{center}

%%%%%%%%%%%%%%%%%%%%%%%%%%%%%%%%%%%%%%%%%%%%%%%%%%%%%%%%%%%%%%%%%%%%%%%%%%%%%%%%
\subsection{Manual Code}
\label{sec:manual}

In case one cannot be certain whether the definitions file |childdoc.def|
is installed on the target \TeX{} distribution
and one prefers not to ship it,
it is conceivable to paste a few relevant commands into the sources.

To that end, drop all statements |\input{childdoc.def}|
and perform the replacements as outlined below.
Instead of |\childdocmain{|\textit{main}|}| add the following code
to the top of the main file:
%
\begin{center}
\begin{tabular}{l}
|\||ifdefined\childdocname\endinput\||fi\newif\ifchilddoc|\\
|\edef\childdocname{\scantokens\expandafter{\jobname\noexpand}}|\\
|\def\childdocmain{|\textit{main}|}\||ifx\childdocmain\childdocname\||else|\\
|\childdoctrue\includeonly{\childdocname}\let\jobname\childdocmain\||fi|\\
\end{tabular}
\end{center}
%
Instead of |\childdocof{|\textit{main}|}| just include the main file
at the top of each child file:
%
\begin{center}
|\input{|\textit{main}|}|
\end{center}
%
A simple redirection |\childdocforward{|\textit{dest}|}| is achieved by:
%
\begin{center}
|\def\jobname{|\textit{dest}|}\input{\jobname}|
\end{center}
%
The redirection with prefix
|\childdocforwardprefix[|\textit{prefix}|]{|\textit{dest}|}|
is accomplished by:
%
\begin{center}
\begin{tabular}{l}
|{\edef\jobname{\scantokens\expandafter{\jobname\noexpand}}|\\
|\def\redirectjob |\textit{prefix}|#1~~~{\gdef\jobname{|\textit{dest}|#1}}|\\
|\expandafter\redirectjob\jobname~~~}\input{\jobname}|
\end{tabular}
\end{center}

In an alternative approach,
child documents can be compiled by a specific command line
without additional code or specific definitions:
%
\begin{center}
|... -jobname "|\textit{target}|" "|[\textit{flags}]%
|\includeonly{|\textit{dest}|}\input{|\textit{main}|}"|
\end{center}
%

%%%%%%%%%%%%%%%%%%%%%%%%%%%%%%%%%%%%%%%%%%%%%%%%%%%%%%%%%%%%%%%%%%%%%%%%%%%%%%%%
%%%%%%%%%%%%%%%%%%%%%%%%%%%%%%%%%%%%%%%%%%%%%%%%%%%%%%%%%%%%%%%%%%%%%%%%%%%%%%%%
\section{Information}

%%%%%%%%%%%%%%%%%%%%%%%%%%%%%%%%%%%%%%%%%%%%%%%%%%%%%%%%%%%%%%%%%%%%%%%%%%%%%%%%
\subsection{Copyright}

Copyright \copyright{} 2017--2018 Niklas Beisert

This work may be distributed and/or modified under the
conditions of the \LaTeX{} Project Public License, either version 1.3
of this license or (at your option) any later version.
The latest version of this license is in
  \url{http://www.latex-project.org/lppl.txt}
and version 1.3 or later is part of all distributions of \LaTeX{}
version 2005/12/01 or later.

This work has the LPPL maintenance status `maintained'.

The Current Maintainer of this work is Niklas Beisert.

This work consists of the files |README.txt|, |childdoc.ins| and |childdoc.dtx|
as well as the derived files |childdoc.def|, |cdocsamp.tex|
with |cdocsch1.tex|, |cdocsch2.tex|, |cdocspt3.tex|, |cdocspt4.tex|,
|cdocsdrf.tex|, |cdocsfn1.tex|, |cdocsfn2.tex|
as well as |childdoc.pdf|.

%%%%%%%%%%%%%%%%%%%%%%%%%%%%%%%%%%%%%%%%%%%%%%%%%%%%%%%%%%%%%%%%%%%%%%%%%%%%%%%%
\subsection{Files and Installation}

The package consists of the files:
%
\begin{center}
\begin{tabular}{ll}
    |README.txt|   & readme file \\
    |childdoc.ins| & installation file \\
    |childdoc.dtx| & source file \\
    |childdoc.def| & definition file \\
    |cdocsamp.tex| & sample main file \\
    |cdocsch1.tex| & sample include file \\
    |cdocsch2.tex| & sample include file \\
    |cdocspt3.tex| & sample part file \\
    |cdocspt4.tex| & sample part file \\
    |cdocsdrf.tex| & sample redirection file \\
    |cdocsfn1.tex| & sample redirection file \\
    |cdocsfn2.tex| & sample redirection file \\
    |childdoc.pdf| & manual
\end{tabular}
\end{center}
%
The distribution consists of the files
|README.txt|, |childdoc.ins| and |childdoc.dtx|.
%
\begin{itemize}
\item
Run (pdf)\LaTeX{} on |childdoc.dtx|
to compile the manual |childdoc.pdf| (this file).
\item
Run \LaTeX{} on |childdoc.ins| to create the definitions file |childdoc.def|
and the sample |cdocsamp.tex| with include files
|cdocsch1.tex|, |cdocsch2.tex|, |cdocspt3.tex|, |cdocspt4.tex|,
|cdocsdrf.tex|, |cdocsfn1.tex|, |cdocsfn2.tex|.
Then copy the file |childdoc.def| to an appropriate directory of your \LaTeX{}
distribution, e.g.\ \textit{texmf-root}|/tex/latex/childdoc|.
\end{itemize}

%%%%%%%%%%%%%%%%%%%%%%%%%%%%%%%%%%%%%%%%%%%%%%%%%%%%%%%%%%%%%%%%%%%%%%%%%%%%%%%%
\subsection{Related CTAN Packages}

There are several other packages which offer a similar functionality:
%
\begin{itemize}
\item
The packages
\href{http://ctan.org/pkg/docmute}{\textsf{docmute}},
\href{http://ctan.org/pkg/includex}{\textsf{includex}} and
\href{http://ctan.org/pkg/standalone}{\textsf{standalone}}
provide commands to include only the document body of
a child file thus allowing both files to be compiled individually.
\item
The packages \href{http://ctan.org/pkg/subdocs}{\textsf{subdocs}}
and \href{http://ctan.org/pkg/subfiles}{\textsf{subfiles}}
provide structures in which the main and child documents can be
encapsulated and allowing them to be compiled individually.
The inclusion mechanism is different from the conventional |\include|.
\item
The package \href{http://ctan.org/pkg/combine}{\textsf{combine}}
is an elaborate solution to combine several documents into one.
\end{itemize}
%
See also the CTAN topic \href{http://ctan.org/topic/subdocs}{\textsf{subdocs}}
for further related packages.
The present package differs from the above solutions in that
a document structure constructed with the conventional |\include| mechanism
just needs two extra commands at the top of every file
such that all constituent files can be compiled individually.

%%%%%%%%%%%%%%%%%%%%%%%%%%%%%%%%%%%%%%%%%%%%%%%%%%%%%%%%%%%%%%%%%%%%%%%%%%%%%%%%
%\subsection{Feature Suggestions}
%
%The following is a list of features which may be useful for future
%versions of this package:
%%
%\begin{itemize}
%\item
%\ldots
%\end{itemize}

%%%%%%%%%%%%%%%%%%%%%%%%%%%%%%%%%%%%%%%%%%%%%%%%%%%%%%%%%%%%%%%%%%%%%%%%%%%%%%%%
\subsection{Revision History}

%%%%%%%%%%%%%%%%%%%%%%%%%%%%%%%%%%%%%%%%
\paragraph{v2.0:} 2018/12/30

\begin{itemize}
\item
immediate forward processing
\item
added |\childdocby| mechanism
\item
manual restructured
\end{itemize}

%%%%%%%%%%%%%%%%%%%%%%%%%%%%%%%%%%%%%%%%
\paragraph{v1.6:} 2018/01/17

\begin{itemize}
\item
application for development of include files
\item
corrections to manual
\end{itemize}

%%%%%%%%%%%%%%%%%%%%%%%%%%%%%%%%%%%%%%%%
\paragraph{v1.5:} 2017/05/21

\begin{itemize}
\item
more complete structuring introduced
\item
|\childdocof| introduced
\item
|\childdoc| renamed to |\childdocmain|
\item
|\childredirect| renamed to |\childdocforward| and |\childdocforwardprefix|
and functionality expanded
\end{itemize}

%%%%%%%%%%%%%%%%%%%%%%%%%%%%%%%%%%%%%%%%
\paragraph{v1.0:} 2017/04/27

\begin{itemize}
\item
manual and install package
\item
first version published on CTAN
\end{itemize}

%%%%%%%%%%%%%%%%%%%%%%%%%%%%%%%%%%%%%%%%
\paragraph{v0.6:} 2017/04/26

\begin{itemize}
\item
redirection mechanism added
\end{itemize}

%%%%%%%%%%%%%%%%%%%%%%%%%%%%%%%%%%%%%%%%
\paragraph{v0.5:} 2017/04/26

\begin{itemize}
\item
functionality in definition file
\end{itemize}


%%%%%%%%%%%%%%%%%%%%%%%%%%%%%%%%%%%%%%%%%%%%%%%%%%%%%%%%%%%%%%%%%%%%%%%%%%%%%%%%
%%%%%%%%%%%%%%%%%%%%%%%%%%%%%%%%%%%%%%%%%%%%%%%%%%%%%%%%%%%%%%%%%%%%%%%%%%%%%%%%
%%%%%%%%%%%%%%%%%%%%%%%%%%%%%%%%%%%%%%%%%%%%%%%%%%%%%%%%%%%%%%%%%%%%%%%%%%%%%%%%
\appendix

\settowidth\MacroIndent{\rmfamily\scriptsize 000\ }

 \DocInput{childdoc.dtx}

\end{document}
%</driver>
% \fi
%
% %%%%%%%%%%%%%%%%%%%%%%%%%%%%%%%%%%%%%%%%%%%%%%%%%%%%%%%%%%%%%%%%%%%%%%%%%%%%%%
% %%%%%%%%%%%%%%%%%%%%%%%%%%%%%%%%%%%%%%%%%%%%%%%%%%%%%%%%%%%%%%%%%%%%%%%%%%%%%%
% \section{Sample}
%\iffalse
%<*samplemain>
%\fi
%
% The following presents a sample document
% with two chapters, two parts, a title page,
% a compile flag as well as three forwarding files to set the flag.
% It consists of eight |.tex| files:
% \begin{center}
% \begin{tabular}{ll}
% |cdocsamp.tex|&main file\\
% |cdocsch1.tex|&include file for chapter 1\\
% |cdocsch2.tex|&include file for chapter 2\\
% |cdocspt3.tex|&include file for part 3\\
% |cdocspt4.tex|&include file for part 4\\
% |cdocsdrf.tex|&forwarding file for main file in draft mode\\
% |cdocsfi1.tex|&forwarding file for final version of chapter 1\\
% |cdocsfi2.tex|&forwarding file for final version of chapter 2\\
% \end{tabular}
% \end{center}
% Each of the eight files can be compiled directly by the \LaTeX{} compiler.
%
% %%%%%%%%%%%%%%%%%%%%%%%%%%%%%%%%%%%%%%
% \paragraph{Main File.}
%
% The main file is called |cdocsamp.tex|.
%
% Load the \textsf{childdoc} definitions and
% declare the filename for the main document:
%    \begin{macrocode}
\input{childdoc.def}
\childdocmain{}
%    \end{macrocode}

% Optional override for |\version| flag:
%    \begin{macrocode}
%%\ifchilddoc\else\providecommand{\version}{draft}\fi
%    \end{macrocode}

% Define the default values for the |\version| flag
% (|final| for the main file and |draft| for childs):
%    \begin{macrocode}
\ifchilddoc
\providecommand{\version}{draft}
\else
\providecommand{\version}{final}
\fi
%    \end{macrocode}

% Load the standard document class:
%    \begin{macrocode}
\documentclass[12pt]{article}
%    \end{macrocode}

% Start the document body:
%    \begin{macrocode}
\begin{document}
%    \end{macrocode}

% Declare a title page.
% Print title, part of document being processed and version flag:
%    \begin{macrocode}
\addtocounter{page}{-1}
\begin{center}
{\LARGE\bfseries{}childdoc example\par}
\vspace{1cm}
\ifchilddoc
\ifchilddocmanual part\else chapter\fi:
`\childdocname' of `\childdocjob'\par
\else
main document: `\childdocjob'\par
\fi
version: \version\par
\end{center}
\newpage
%    \end{macrocode}

% Manually include selected file,
% otherwise process as usual:
%    \begin{macrocode}
\ifchilddocmanual
\section*{part `\childdocname'}
\input{\childdocname}
\else
%    \end{macrocode}

% Include the two chapters:
%    \begin{macrocode}
\include{cdocsch1}
\include{cdocsch2}
%    \end{macrocode}

% Include the two parts unless only chapters should be displayed:
%    \begin{macrocode}
\ifchilddoc\else
\section{part three}
\input{cdocspt3}
\section{part four}
\input{cdocspt4}
\fi
%    \end{macrocode}

% Process as usual until here:
%    \begin{macrocode}
\fi
%    \end{macrocode}

% End of document body:
%    \begin{macrocode}
\end{document}
%    \end{macrocode}
%\iffalse
%</samplemain>
%\fi
%
% %%%%%%%%%%%%%%%%%%%%%%%%%%%%%%%%%%%%%%
% \paragraph{Chapter Include Files.}
%
% The include files are called |cdocsch1.tex| and |cdocsch2.tex|.
%
%\iffalse
%<*samplechap1|samplechap2>
%\fi

% Optional override for |\version| flag:
%    \begin{macrocode}
%%\providecommand{\version}{final}
%    \end{macrocode}

% Include the main document:
%    \begin{macrocode}
\input{childdoc.def}
\childdocof{cdocsamp}
%    \end{macrocode}

%\iffalse
%</samplechap1|samplechap2>
%\fi
%
%\iffalse
%<*samplechap1>
%\fi
% Some text for chapter 1:
%    \begin{macrocode}
\section{one}
some text in chapter one
%    \end{macrocode}

%\iffalse
%</samplechap1>
%\fi
% Some text for chapter 2:
%\iffalse
%<*samplechap2>
%\fi
%    \begin{macrocode}
\section{two}
more text in chapter two
%    \end{macrocode}

%\iffalse
%</samplechap2>
%\fi
%
% %%%%%%%%%%%%%%%%%%%%%%%%%%%%%%%%%%%%%%
% \paragraph{Part Include Files.}
%
% The include files are called |cdocspt3.tex| and |cdocspt4.tex|.
%
%\iffalse
%<*samplepart3|samplepart4>
%\fi

% Optional override for |\version| flag:
%    \begin{macrocode}
%%\providecommand{\version}{final}
%    \end{macrocode}

% Include the main document:
%    \begin{macrocode}
\input{childdoc.def}
\childdocby{cdocsamp}
%    \end{macrocode}

%\iffalse
%</samplepart3|samplepart4>
%\fi
%
%\iffalse
%<*samplepart3>
%\fi
% Some text for part 3:
%    \begin{macrocode}
some text in part three
%    \end{macrocode}

%\iffalse
%</samplepart3>
%\fi
% Some text for part 4:
%\iffalse
%<*samplepart4>
%\fi
%    \begin{macrocode}
more text in part four
%    \end{macrocode}

%\iffalse
%</samplepart4>
%\fi
%
% %%%%%%%%%%%%%%%%%%%%%%%%%%%%%%%%%%%%%%
% \paragraph{Forwarding for a Complete Draft.}
%
% The following forwarding file |cdocsdrf.tex|
% compiles the main document in draft mode:
%\iffalse
%<*sampledraft>
%\fi
%    \begin{macrocode}
\def\version{draft}
\input{childdoc.def}
\childdocforward{cdocsamp}
%    \end{macrocode}

%\iffalse
%</sampledraft>
%\fi
%
% %%%%%%%%%%%%%%%%%%%%%%%%%%%%%%%%%%%%%%
% \paragraph{Forwarding for Final Version of the Chapters.}
%
% The following forwarding files |cdocsfn1.tex| and |cdocsfn2.tex|
% (with identical content)
% compile the final versions of the child documents
% |cdocsch1.tex| and |cdocsch2.tex|, respectively:
%\iffalse
%<*samplefinal>
%\fi
%    \begin{macrocode}
\def\version{final}
\input{childdoc.def}
\childdocforwardprefix[cdocsamp]{cdocsfn}{cdocsch}
%    \end{macrocode}

%\iffalse
%</samplefinal>
%\fi
%
% %%%%%%%%%%%%%%%%%%%%%%%%%%%%%%%%%%%%%%
% \paragraph{Command Line Processing.}
%
% The following three command lines generate the output files
% |cdocscld|, |cdocscl1| and |cdocscl2|
% which should be identical to
% |cdocsdrf|, |cdocsch1| and |cdocsfn2|, respectively:
% \begin{center}
% \begin{tabular}{l}
% |latex -jobname cdocscld \|\\
% |  "\def\version{draft}\input{childdoc.def}\childdocforward{cdocsamp}"|\\
% |latex -jobname cdocscl1 \|\\
% |  "\input{childdoc.def}\childdocforward[cdocsamp]{cdocsch1}"|\\
% |latex -jobname cdocscl2 \|\\
% |  "\def\version{final}\input{childdoc.def}\childdocforward{cdocsch2}"|
% \end{tabular}
% \end{center}
% Note that the trailing backslash on each first line
% merely continues the input to the second line
% (for convenient cut ant paste).
% Furthermore, the command |latex| can be replaced by any
% of its alternative versions such as |pdflatex|.
%
% %%%%%%%%%%%%%%%%%%%%%%%%%%%%%%%%%%%%%%%%%%%%%%%%%%%%%%%%%%%%%%%%%%%%%%%%%%%%%%
% %%%%%%%%%%%%%%%%%%%%%%%%%%%%%%%%%%%%%%%%%%%%%%%%%%%%%%%%%%%%%%%%%%%%%%%%%%%%%%
% \section{Implementation}
%\iffalse
%<*package>
%\fi
%
% This section describes the definitions file |childdoc.def|.

% The definitions cannot be loaded using |\usepackage| or |\RequirePackage|
% which has a mechanism to prevent loading a style file more than once.
% When loading the definitions by means of |\input|
% multiple instances have to be prevented manually:
%\iffalse
%This code needs to be before the `\ProvidesFile' directive
%which is defined at the beginning of this file.
%Therefore it is also placed there and commented out here.
%</package>
%<*discard>
%\fi
%    \begin{macrocode}
\ifdefined\childdocmain\endinput\fi
%    \end{macrocode}
%\iffalse
%</discard>
%<*package>
%\fi
%
% \macro{\ifchilddoc}
% \macro{\ifchilddocmanual}
% The conditional |\ifchilddoc| tells whether a
% child (true) or main (false) document is being compiled.
% The conditional |\ifchilddocmanual| tells whether
% the |\includeonly| mechanism is used (false) or
% the selection of child files must be performed manually (true).
% The definitions initialise to false:
%    \begin{macrocode}
\newif\ifchilddoc
\newif\ifchilddocmanual
%    \end{macrocode}

% \macro{\childdocname}
% \macro{\childdocjob}
% The macro |\childdocname| stores the name of the main document
% to be compiled. The macro |\childdocjob| stores the name of
% the document on which the \LaTeX{} compiler was originally invoked.
% The content of |\jobname| cannot be compared
% to filenames specified in the source due to different catcodes.
% The following code rescans |\jobname|, stores the result
% in |\childdocname| and saves a copy in |\childdocjob|:
%    \begin{macrocode}
\edef\childdocname{\scantokens\expandafter{\jobname\noexpand}}
\let\childdocjob\childdocname
%    \end{macrocode}

% \macro{\childdocdisable}
% The macro |\childdocdisable| prevents the main file
% from being processed more than once.
% At this stage, the main document command |\childdocmain|
% is assumed to be called once again where it should do nothing.
% Any subsequent call to it should prevent
% a secondary processing of the main document
% It overwrites the forwarding commands
% |\childdocof| and |\childdocforward|
% with empty macros to prevent further inclusions of the main document:
%    \begin{macrocode}
\newcommand{\childdocdisable}
{
  \renewcommand{\childdocmain}[1]{\renewcommand{\childdocmain}[1]{\endinput}}
  \renewcommand{\childdocof}[1]{}
  \renewcommand{\childdocby}[2][]{}
  \renewcommand{\childdocforward}[2][]{}
  \renewcommand{\childdocdisable}{}
}
%    \end{macrocode}

% \macro{\childdocmain}
% The macro |\childdocmain| is to be called at the top of the main file
% with nothing or the main filename (without extension) as argument.
% First, it breaks loops.
% If the argument is not empty and does not match |\childdocname|
% (which is set by the first inclusion of |childdoc.def|),
% |\ifchilddoc| is set to true, |\includeonly| is applied to the child file
% and |\jobname| is set to the main file
% (for proper handling of |.aux| files):
%    \begin{macrocode}
\newcommand{\childdocmain}[1]
{
  \childdocdisable\childdocmain{}
  \if?#1?\else
    \begingroup
      \def\childdoctmp{#1}
      \ifx\childdoctmp\childdocname
        \def\childdoctmp{}
      \else
        \def\childdoctmp
        {
          \childdoctrue
          \includeonly{\childdocname}
          \def\childdocjob{#1}
          \def\jobname{#1}
        }
      \fi
      \expandafter
    \endgroup
    \childdoctmp
  \fi
}
%    \end{macrocode}

% \macro{\childdocof}
% The command |\childdocof| redirects
% compilation to the main file |#1|.
%    \begin{macrocode}
\newcommand{\childdocof}[1]
{
  \childdocdisable
  \childdoctrue
  \includeonly{\childdocname}
  \def\jobname{#1}
  \def\childdocjob{#1}
  \input{#1}
}
%    \end{macrocode}

% \macro{\childdocby}
% The command |\childdocby| ....
%    \begin{macrocode}
\newcommand{\childdocby}[2][]
{
  \childdocdisable
  \childdoctrue
  \childdocmanualtrue
  \if?#1?\else
    \def\jobname{#2}
  \fi
  \def\childdocjob{#2}
  \input{#2}
  \endinput
}
%    \end{macrocode}

% \macro{\childdocforward}
% The command |\childdocforward| redirects
% compilation to the main file or
% (if the optional argument is given) a child file.
% Parameters are set as if the main file
% or a child file starting with |\childdocof| was compiled.
% Then compilation is handed over to the main file:
%    \begin{macrocode}
\newcommand{\childdocforward}[2][]
{
  \begingroup
    \if?#1?
      \def\childdoctmp
      {
        \def\childdocname{#2}
        \def\childdocjob{#2}
        \def\jobname{#2}
        \input{#2}
        \endinput
      }
    \else
      \def\childdoctmp
      {
        \childdocdisable
        \def\childdocname{#2}
        \childdoctrue
        \includeonly{#2}
        \def\childdocjob{#1}
        \def\jobname{#1}
        \input{#1}
        \endinput
      }
    \fi
    \expandafter
  \endgroup
  \childdoctmp
}
%    \end{macrocode}

% \macro{\childdocforwardprefix}
% The command |\childdocforwardprefix| redirects
% compilation to the main or a child file by means of a pattern.
% The prefix |#1| in the current filename is replaced by |#2|
% and the suffix of the current filename is kept
% (it is assumed that the filename does not contain the substring `|~~~|'
% which is used as a delimiter).
% Compilation is handed over to the new file by |\childdocforward|:
%    \begin{macrocode}
\newcommand{\childdocforwardprefix}[3][]
{
  \begingroup
    \def\childdocextract #2##1~~~{\def\childdoctmp{\childdocforward[#1]{#3##1}}}
    \expandafter\childdocextract\childdocname~~~
    \expandafter
  \endgroup
  \childdoctmp
}
%    \end{macrocode}

% \macro{\childdoc}
% The deprecated macro |\childdoc| is a legacy version of |\childdocmain|:
%    \begin{macrocode}
\newcommand{\childdoc}{\childdocmain}
%    \end{macrocode}

% \macro{\childdocredirect}
% The deprecated macro |\childdocredirect| is a legacy version
% of |\childdocforward| and |\childdocforwardprefix|:
%    \begin{macrocode}
\newcommand{\childdocredirect}[2][]
{
  \begingroup
    \if?#1?
      \def\childdoctmp{\childdocforward{#2}}
    \else
      \def\childdoctmp{\childdocforwardprefix{#1}{#2}}
    \fi
    \expandafter
  \endgroup
  \childdoctmp
}
%    \end{macrocode}

%\iffalse
%</package>
%\fi
%
\endinput
|\\
|\childdocmain{|\textit{main}|}|\\
\end{tabular}
\end{center}
%
If |\jobname| does not match the argument \textit{main} of |\childdocmain|,
it is assumed that |\jobname| points to the child file to be compiled.
When using |\childdocmain| with the main file specified as argument,
it suffices to start a child file
with just |\input{|\textit{main}|}|
without loading of the package and using |\childdocof|.
If instead all processing is done
with the appropriate \textsf{childdoc} directives,
the argument of \textit{main} of |\childdocmain| can be empty.

An alternative version of the command line processing described
in \secref{sec:commandline} using the detection mechanism reads:
%
\begin{center}
|... -jobname "|\textit{target}|" "|[\textit{flags}]%
[|\def\jobname{|\textit{dest}|}|]|\input{|\textit{main}|}"|
\end{center}

%%%%%%%%%%%%%%%%%%%%%%%%%%%%%%%%%%%%%%%%%%%%%%%%%%%%%%%%%%%%%%%%%%%%%%%%%%%%%%%%
\subsection{Manual Code}
\label{sec:manual}

In case one cannot be certain whether the definitions file |childdoc.def|
is installed on the target \TeX{} distribution
and one prefers not to ship it,
it is conceivable to paste a few relevant commands into the sources.

To that end, drop all statements |% \iffalse
%
% childdoc.dtx Copyright (C) 2017-2018 Niklas Beisert
%
% This work may be distributed and/or modified under the
% conditions of the LaTeX Project Public License, either version 1.3
% of this license or (at your option) any later version.
% The latest version of this license is in
%   http://www.latex-project.org/lppl.txt
% and version 1.3 or later is part of all distributions of LaTeX
% version 2005/12/01 or later.
%
% This work has the LPPL maintenance status `maintained'.
%
% The Current Maintainer of this work is Niklas Beisert.
%
% This work consists of the files childdoc.dtx and childdoc.ins
% and the derived files childdoc.def and cdocsamp.tex with
% cdocsch1.tex, cdocsch2.tex, cdocsdrf.tex, cdocsfn1.tex, cdocsfn2.tex.
%
%<package>\ifdefined\childdocmain\endinput\fi
%<package>\ProvidesFile{childdoc.def}[2018/12/30 v2.0 child document driver]
%<samplemain>\ProvidesFile{cdocsamp.tex}[2018/12/30 v2.0 sample for childdoc]
%<*driver>
%\ProvidesFile{childdoc.drv}[2018/12/30 v2.0 childdoc reference manual file]
\PassOptionsToClass{10pt,a4paper}{article}
\documentclass{ltxdoc}

\usepackage[margin=35mm]{geometry}
\usepackage{hyperref}
\usepackage{hyperxmp}
\usepackage[usenames]{color}

\hypersetup{colorlinks=true}
\hypersetup{pdfstartview=FitH}
\hypersetup{pdfpagemode=UseNone}
\hypersetup{pdfsource={}}
\hypersetup{pdflang={en-UK}}
\hypersetup{pdfcopyright={Copyright 2017-2018 Niklas Beisert.
  This work may be distributed and/or modified under the
  conditions of the LaTeX Project Public License, either version 1.3
  of this license or (at your option) any later version.}}
\hypersetup{pdflicenseurl={http://www.latex-project.org/lppl.txt}}
\hypersetup{pdfcontactaddress={ETH Zurich, ITP, HIT K,
  Wolfgang-Pauli-Strasse 27}}
\hypersetup{pdfcontactpostcode={8093}}
\hypersetup{pdfcontactcity={Zurich}}
\hypersetup{pdfcontactcountry={Switzerland}}
\hypersetup{pdfcontactemail={nbeisert@itp.phys.ethz.ch}}
\hypersetup{pdfcontacturl={http://people.phys.ethz.ch/\xmptilde nbeisert/}}

\newcommand{\secref}[1]{\hyperref[#1]{section \ref*{#1}}}

\parskip1ex
\parindent0pt
\let\olditemize\itemize
\def\itemize{\olditemize\parskip0pt}

\begin{document}

\title{The \textsf{childdoc} Package}
\hypersetup{pdftitle={The childdoc Package}}
\author{Niklas Beisert\\[2ex]
  Institut f\"ur Theoretische Physik\\
  Eidgen\"ossische Technische Hochschule Z\"urich\\
  Wolfgang-Pauli-Strasse 27, 8093 Z\"urich, Switzerland\\[1ex]
  \href{mailto:nbeisert@itp.phys.ethz.ch}
  {\texttt{nbeisert@itp.phys.ethz.ch}}}
\hypersetup{pdfauthor={Niklas Beisert}}
\hypersetup{pdfsubject={Manual for the LaTeX2e Package childdoc}}
\date{30 December 2018, \textsf{v2.0}}
\maketitle

\begin{abstract}\noindent
\textsf{childdoc} is a \LaTeXe{} package
that enables the direct compilation
of document sections included by |\include|
to individual files.
\end{abstract}

\begingroup
\parskip0ex
\tableofcontents
\endgroup

%%%%%%%%%%%%%%%%%%%%%%%%%%%%%%%%%%%%%%%%%%%%%%%%%%%%%%%%%%%%%%%%%%%%%%%%%%%%%%%%
%%%%%%%%%%%%%%%%%%%%%%%%%%%%%%%%%%%%%%%%%%%%%%%%%%%%%%%%%%%%%%%%%%%%%%%%%%%%%%%%
\section{Introduction}

\LaTeX{} provides a mechanism to structure a large document (such as a book)
into a main file and several child files (containing the chapters)
using the |\include| command.
This mechanism is beneficial for documents
which span hundreds of pages in order to
make the source file(s) more manageable.
Moreover, compilation can be restricted to
selected child files by means of the |\includeonly| command.
The latter feature can be used to reduce the compilation time while editing
(this was significantly more useful in the earlier days of \LaTeX{})
or to generate a smaller document which is easier to navigate.
Another application of |\includeonly| is to generate
documents consisting of selected parts of the complete document.

However, there are a few drawbacks of the plain |\include| mechanism:
\begin{itemize}
\item
The child files cannot be compiled on their own,
they can only be compiled via the main file.
A naive editing environment
(such as a text editor with an option
to have the current file processed by \LaTeX)
may require one to switch to the main file before compiling;
attempting to compile the child file produces errors.
\item
The main file must be modified (each time)
to adjust the |\includeonly| command
to the present needs. This easily leaves the main file in a messy state.
\item
The generated document will always carry the filename
of the main document. This is inconvenient if
several child files are to be compiled and
to be kept for distribution.
\end{itemize}

The present package provides a simple interface
to make child files individually compilable by \LaTeX{}.
Compiling a child file then has the same effect as compiling
the main file with an |\includeonly| command
to select the appropriate child.
Moreover the generated document will carry the name of the child
rather than the main file.
This resolves all three above issues.

This feature is meant to make the editing of books,
thesis documents and lecture notes somewhat more convenient.
However, the package can also be used efficiently for
composing a series of documents (such as exercise sheets)
which are typically distributed individually.
It then assists the author in generating the individual documents
(potentially in different versions)
as well as a document containing the collected series.
Another application is in developing style files
or other kinds of included material
where compilation of the style file could redirect
to a sample or test file.

%%%%%%%%%%%%%%%%%%%%%%%%%%%%%%%%%%%%%%%%%%%%%%%%%%%%%%%%%%%%%%%%%%%%%%%%%%%%%%%%
%%%%%%%%%%%%%%%%%%%%%%%%%%%%%%%%%%%%%%%%%%%%%%%%%%%%%%%%%%%%%%%%%%%%%%%%%%%%%%%%
\section{Usage}

First of all, the package \textsf{childdoc} is \emph{not} a standard
\LaTeXe{} |.sty| style file! Therefore it needs to be invoked in
a non-standard way.

%%%%%%%%%%%%%%%%%%%%%%%%%%%%%%%%%%%%%%%%%%%%%%%%%%%%%%%%%%%%%%%%%%%%%%%%%%%%%%%%
\subsection{Included Files}
\label{sec:include}

%%%%%%%%%%%%%%%%%%%%%%%%%%%%%%%%%%%%%%%%
\DescribeMacro{\childdocmain}
To use the package, add the commands
\begin{center}
\begin{tabular}{l}
|\input{childdoc.def}|\\
|\childdocmain{}|\\
\end{tabular}
\end{center}
at the very top of the main \LaTeX{} file,
in particular \emph{before} the |\documentclass| statement!
The argument of |\childdocmain| should be left empty
(but it must be present).

%%%%%%%%%%%%%%%%%%%%%%%%%%%%%%%%%%%%%%%%
\DescribeMacro{\childdocof}
Furthermore, add the commands
\begin{center}
\begin{tabular}{l}
|\input{childdoc.def}|\\
|\childdocof{|\textit{main}|}|\\
\end{tabular}
\end{center}
at the top of every child file \textit{child}
which is included by |\include{|\textit{child}|}|
from within the main file
(or at least for those files to be compiled individually).
The argument \textit{main} must be the filename of the main file.

There are a couple of
considerations in setting up the main and child documents:

%%%%%%%%%%%%%%%%%%%%%%%%%%%%%%%%%%%%%%%%
\paragraph{Restrictions.}

Please note the following restrictions:
\begin{itemize}
\item
|\childdocmain| must be called with one argument \textit{main}
to ensure compatibility with earlier version of the package.
It must either be empty (|\childdocmain{}|)
or precisely match the filename of the main file in which it is specified.
See \secref{sec:detection} for further information.
\item
The filename \textit{main} must be specified without the |.tex| extension.
\item
The filename \textit{main} is case sensitive
(even in case-insensitive file systems)
due to internal string comparison.
\item
The argument \textit{main} should be fully expanded, it cannot be a macro.
\item
Subdirectories and special characters should be avoided in filenames.
\item
The command |\childdocmain{|\textit{main}|}| must be followed by a whitespace.
It should not be followed immediately by another command
or by a comment mark `|%|'.
This is because the \TeX{} parser reads the token immediately following
the argument of |\childdocmain| and puts it
at the beginning of every child section;
however, a white\-space is ignored.
\end{itemize}

%%%%%%%%%%%%%%%%%%%%%%%%%%%%%%%%%%%%%%%%
\paragraph{Content of Main File.}

It is advisable to place all content in the child files included by |\include|.
Any output contained in the main file will appear in all child documents
unless suppressed manually;
it cannot be suppressed automatically by the |\includeonly| directive
and thus should normally be avoided.
A method to include some content in the main file
by means of conditional processing is described in \secref{sec:conditional}.

%%%%%%%%%%%%%%%%%%%%%%%%%%%%%%%%%%%%%%%%
\paragraph{Page Numbering.}

When only a part of the document is compiled,
the appropriate numbering of pages
(as well as other status parameters)
is determined from the |.aux| files.
The latter contain information from previous passes.
However this information needs to propagate through
all intermediate child documents.
Therefore the page numbering in child documents may well
be inconsistent until the complete document is compiled at least once.

A useful (if unconventional) way to always ensure a consistent
page numbering is to restart the numbering in each child document
and denote the pages by `\textit{child}|.|\textit{page}'
where \textit{child} represents the chapter/section number of the child file.
This can be achieved by the command
|\numberwithin{page}{|\textit{child}|}|
of the \textsf{amsmath} package
where \textit{child} can be |chapter| or |section|
depending on the chosen structuring.
Alternatively, one can modify the macro |\thepage| appropriately
and reset the counter |page| at the start of each child file.

%%%%%%%%%%%%%%%%%%%%%%%%%%%%%%%%%%%%%%%%%%%%%%%%%%%%%%%%%%%%%%%%%%%%%%%%%%%%%%%%
\subsection{Conditional Processing}
\label{sec:conditional}

The package provides a mechanism to compile different versions
of a document. To customise the versions further some conditional processing
can come in handy to distinguish which version is being compiled.
The package provides two macros to describe the compilation context:

%%%%%%%%%%%%%%%%%%%%%%%%%%%%%%%%%%%%%%%%
\DescribeMacro{\ifchilddoc}
The conditional |\ifchilddoc| distinguishes between the compilation of
child documents and the main document:
%
\begin{center}
|\ifchilddoc |\textit{child-code}| |[|\||else |\textit{main-code}]| \||fi|
\end{center}

%%%%%%%%%%%%%%%%%%%%%%%%%%%%%%%%%%%%%%%%
\DescribeMacro{\childdocname}
\DescribeMacro{\childdocjob}
The macro |\childdocname| contains the filename (without extension)
of the main or child file being processed.
Note that |\childdocjob| will always contain the name of the main file.

%%%%%%%%%%%%%%%%%%%%%%%%%%%%%%%%%%%%%%%%
\paragraph{Title Page.}

Conditional processing can be used to include a title or banner page
in the main document when proper precautions are taken.
Importantly, the code in the main file should ensure that the page counter
(as well as other status parameters which are stored in the |.aux| files)
takes the same value after the conditional processing.
Otherwise the page numbers may take divergent values
depending on which part is compiled.

For example, a title page could be declared by:
%
\begin{center}
\begin{tabular}{l}
|\ifchilddoc\||else|\\
|\addtocounter{page}{-1}|\\
\textit{code for title page}\\
|\newpage|\\
|\||fi|
\end{tabular}
\end{center}
%
A banner page for the child documents can be generated by:
%
\begin{center}
\begin{tabular}{l}
|\ifchilddoc|\\
|\addtocounter{page}{-1}|\\
\textit{code for banner page}\\
|\newpage|\\
|\||fi|
\end{tabular}
\end{center}
%
Here one could write a message such as:
\begin{center}
|This is the part \childdocname{} of \childdocjob{}.|
\end{center}

%%%%%%%%%%%%%%%%%%%%%%%%%%%%%%%%%%%%%%%%%%%%%%%%%%%%%%%%%%%%%%%%%%%%%%%%%%%%%%%%
\subsection{Flags}
\label{sec:flags}

The package makes it easy to generate different versions
of the main or child documents.
To this end compilation flags can be defined
and assigned different default values.
They will be particularly useful in conjunction
with the forwarding mechanism described in \secref{sec:forward}.

For example, it may be useful to have a flag |\version|
which can be set to |draft| or |final|.
The document source will contain some conditional code
depending on the value of |\version|.
Suppose further, the flag should default to |final| for the main file
and to |draft| for child files
which is a natural assignment for editing the document.
This is achieved by placing the following code
in the preamble of the main document
(below the |\childdocmain| directive):
%
\begin{center}
\begin{tabular}{l}
|\ifchilddoc|\\
|\providecommand{\version}{draft}|\\
|\||else|\\
|\providecommand{\version}{final}|\\
|\||fi|
\end{tabular}
\end{center}
%
The definition by |\providecommand| makes sure
that previous definitions are not overwritten.
Further statements |\providecommand{\version}{...}|
can thus be added before the above code to override it.

For the main file, one might add a line
(between |\childdocmain| and the above block)
%
\begin{center}
|%\ifchilddoc\||else\providecommand{\version}{draft}\||fi|
\end{center}
%
which can be uncommented to produce a draft version.
Likewise one can add a line to the very top of a child file
(above the |\childdocof{|\textit{main}|}| directive)
%
\begin{center}
|%\providecommand{\version}{final}|
\end{center}
%
which can be uncommented to produce the final version of this child document.

%%%%%%%%%%%%%%%%%%%%%%%%%%%%%%%%%%%%%%%%%%%%%%%%%%%%%%%%%%%%%%%%%%%%%%%%%%%%%%%%
\subsection{Forwarding}
\label{sec:forward}

Different versions of the main or child documents
using compilation flags as described in \secref{sec:flags}
can be (permanently) stored in different files
for convenient compilation, viewing and distribution.
To this end, the package defines a command
to pass on compilation to a different file:

%%%%%%%%%%%%%%%%%%%%%%%%%%%%%%%%%%%%%%%%
\DescribeMacro{\childdocforward}
The command |\childdocforward| redirects processing to
another source file:
%
\begin{center}
\begin{tabular}{l}
|\input{childdoc.def}|\\
|\childdocforward[|\textit{main}|]{|\textit{dest}|}|\\
\end{tabular}
\end{center}
%
The argument \textit{dest} is the destination file
(without extension).
It should be the main file or one of the child files.
Note that further \textsf{childdoc} directives
such as |\childdocof| and |\childdocforward|
in the indicated file will be processed in this form.
The optional argument \textit{main}
passes on directly to the main file \textit{main}
while pretending to compile the child \textit{dest}.
This form behaves as if \textit{dest}
issues |\childdocof{|\textit{main}|}| right away,
and no further \textsf{childdoc} directives will be processed.

%%%%%%%%%%%%%%%%%%%%%%%%%%%%%%%%%%%%%%%%
\DescribeMacro{\...prefix}
In the alternative form |\childdocforwardprefix|,
%
\begin{center}
\begin{tabular}{l}
|\input{childdoc.def}|\\
|\childdocforwardprefix[|\textit{main}|]{|\textit{prefix}|}{|\textit{dest}|}|
\end{tabular}
\end{center}
%
the destination file is determined by a pattern
depending on the current file:
To make this work, the current file must be called
`{\textit{prefix}\hspace{0.2em}\textit{suffix}}'
with \textit{prefix} matching precisely the argument.
Processing is then passed on to the file
`{\textit{dest}\hspace{0.2em}\textit{suffix}}'.
Surely, the same effect is achieved by
directly specifying the
argument `{\textit{dest}\hspace{0.2em}\textit{suffix}}'
in the first form.
However, that requires to set up a different file
for each child. With the alternative form of the command
all these files can have exactly the same content
which simplifies setting them up and maintaining them.

For example, the following file |draft.tex|
with a compilation flag |\version| as described in \secref{sec:flags}
compiles the main document as a draft:
%
\begin{center}
\begin{tabular}{l}
|\def\version{draft}|\\
|\input{childdoc.def}|\\
|\childdocforward{|\textit{main}|}|
\end{tabular}
\end{center}
%
Likewise, the following files |final|\textit{nn}|.tex|
compile the final version of the child document
|child|\textit{nn}|.tex|:
%
\begin{center}
\begin{tabular}{l}
|\def\version{final}|\\
|\input{childdoc.def}|\\
|\childdocforwardprefix{final}{child}|
\end{tabular}
\end{center}
%

Note that when several versions of a main file and/or of each child file
are to be generated, it may be convenient to set up a |Makefile| or
shell script to automatise the process.

%%%%%%%%%%%%%%%%%%%%%%%%%%%%%%%%%%%%%%%%%%%%%%%%%%%%%%%%%%%%%%%%%%%%%%%%%%%%%%%%
\subsection{Command Line Processing}
\label{sec:commandline}

The effect of redirection files can also be achieved by invoking
the \LaTeX{} compiler with a more elaborate command line.
Most conveniently this should be done as part
of a shell script or a |Makefile|.

When using \textsf{childdoc} in the main file, the following
command lines effectively perform a redirection
(note that depending on the shell being used,
backslashes may have to be doubled: `|\|' $\to$ `|\\|'):
%
\begin{center}
|... -jobname "|\textit{target}|" |\\|"|[\textit{flags}]%
|\input{childdoc.def}\childdocforward[|\textit{main}|]{|\textit{dest}|}"|
\end{center}
%
Here \textit{target} is the name of the output file,
\textit{main} is the name of the main file
and \textit{dest} is the name of the main or child file to be processed
(all filenames without extensions).
The optional argument \textit{main} can be omitted
if \textit{main} matches \textit{dest}.
Optionally, compilation \textit{flags} can be defined via |\def| commands.
This command line makes the \TeX{} engine believe
it is compiling the file \textit{target}
whose content is specified as the latter parameter.
The provided code then forwards the processing to
\textit{main} or \textit{dest} as described in \secref{sec:forward}.

%%%%%%%%%%%%%%%%%%%%%%%%%%%%%%%%%%%%%%%%%%%%%%%%%%%%%%%%%%%%%%%%%%%%%%%%%%%%%%%%
\subsection{Include by Input}
\label{sec:input}

Including child documents by |\include| has some restrictions by design.
Most notably, the content of a child document always occupies
its own set of pages; pages cannot be shared between child documents.
Usually, this behaviour makes perfect sense
because each child document contain an essential part of the document.
However, in some situations it may be desirable to compose
a document from a collection of parts
without having mandatory page breaks between then.
For this case, the package
provides a mechanism to include parts
by |\input| which can also be processed individually.
However, by construction this mechanism
requires manual handling of the content to be output.

%%%%%%%%%%%%%%%%%%%%%%%%%%%%%%%%%%%%%%%%
\DescribeMacro{\ifchilddocmanual}
The main file should be prepared as usual, see \secref{sec:include}.
However, the document body must make a distinction
between processing of an individual part and of the main document, e.g.:
%
\begin{center}
\begin{tabular}{l}
|\ifchilddocmanual|\\
|\input{\childdocname}|\\
|\||else|\\
\textit{document body with }|\input{|\textit{part}|}|\\
|\||fi|
\end{tabular}
\end{center}
%
The conditional |\ifchilddocmanual| is true whenever
a part to be included by |\input| is being compiled,
and the name of the part is stored in |\childdocname|.

%%%%%%%%%%%%%%%%%%%%%%%%%%%%%%%%%%%%%%%%
\DescribeMacro{\childdocby}
Each part to be included by |\input| should start with:
%
\begin{center}
\begin{tabular}{l}
|\input{childdoc.def}|\\
|\childdocby{|\textit{main}|}|\\
\end{tabular}
\end{center}
%
The directive |\childdocby| is similar to |\childdocof|
described in \secref{sec:include},
but the subsequent selection of content must be done manually.
To that end, both |\ifchilddoc| and |\ifchilddocmanual|
will be true upon processing of a part,
and the name of the part is stored in |\childdocname|.
Note that |\jobname| will be set to the filename of the current part
so that each part receives an individual |.aux| file
that does not interfere with the |.aux| file(s) of the main document.
This behaviour can be altered by the alternative form
|\childdocby[*]{|\textit{main}|}| (with a non-empty optional argument)
which uses the |.aux| file of the main document
by setting |\jobname| to \textit{main}.

%%%%%%%%%%%%%%%%%%%%%%%%%%%%%%%%%%%%%%%%%%%%%%%%%%%%%%%%%%%%%%%%%%%%%%%%%%%%%%%%
\subsection{Driver Development}
\label{sec:driver}

The \textsf{childdoc} mechanism can also be use for the development
of definition files such as \LaTeX{} styles or classes.
This case differs from the above setup with multiple parts
included by |\include| in that no |\includeonly| should be invoked.
This can be achieved by starting the include file
(before |\ProvidesPackage|) with:
%
\begin{center}
\begin{tabular}{l}
|\input{childdoc.def}|\\
|\childdocforward{|\textit{main}|}|\\
\end{tabular}
\end{center}
%
or alternatively with:
%
\begin{center}
\begin{tabular}{l}
|\input{childdoc.def}|\\
|\childdocby{|\textit{main}|}|\\
\end{tabular}
\end{center}
%
Both forms have slightly different effects as described above.
The main file is prepared as usual, see \secref{sec:include}.

%%%%%%%%%%%%%%%%%%%%%%%%%%%%%%%%%%%%%%%%%%%%%%%%%%%%%%%%%%%%%%%%%%%%%%%%%%%%%%%%
\subsection{Legacy Detection}
\label{sec:detection}

The directive |\childdocmain| in the main file can detect
whether the complete document or merely a child is to be compiled
even without using the directive |\childdocof|.
This method is deprecated because it is less robust
and there is no compelling reason to use it;
it is merely provided for backward compatibility
and it may be removed in future versions.

If the detection mechanism is to be used,
it is mandatory to correctly specify
the filename of the main file as the argument of |\childdocmain|:
%
\begin{center}
\begin{tabular}{l}
|\input{childdoc.def}|\\
|\childdocmain{|\textit{main}|}|\\
\end{tabular}
\end{center}
%
If |\jobname| does not match the argument \textit{main} of |\childdocmain|,
it is assumed that |\jobname| points to the child file to be compiled.
When using |\childdocmain| with the main file specified as argument,
it suffices to start a child file
with just |\input{|\textit{main}|}|
without loading of the package and using |\childdocof|.
If instead all processing is done
with the appropriate \textsf{childdoc} directives,
the argument of \textit{main} of |\childdocmain| can be empty.

An alternative version of the command line processing described
in \secref{sec:commandline} using the detection mechanism reads:
%
\begin{center}
|... -jobname "|\textit{target}|" "|[\textit{flags}]%
[|\def\jobname{|\textit{dest}|}|]|\input{|\textit{main}|}"|
\end{center}

%%%%%%%%%%%%%%%%%%%%%%%%%%%%%%%%%%%%%%%%%%%%%%%%%%%%%%%%%%%%%%%%%%%%%%%%%%%%%%%%
\subsection{Manual Code}
\label{sec:manual}

In case one cannot be certain whether the definitions file |childdoc.def|
is installed on the target \TeX{} distribution
and one prefers not to ship it,
it is conceivable to paste a few relevant commands into the sources.

To that end, drop all statements |\input{childdoc.def}|
and perform the replacements as outlined below.
Instead of |\childdocmain{|\textit{main}|}| add the following code
to the top of the main file:
%
\begin{center}
\begin{tabular}{l}
|\||ifdefined\childdocname\endinput\||fi\newif\ifchilddoc|\\
|\edef\childdocname{\scantokens\expandafter{\jobname\noexpand}}|\\
|\def\childdocmain{|\textit{main}|}\||ifx\childdocmain\childdocname\||else|\\
|\childdoctrue\includeonly{\childdocname}\let\jobname\childdocmain\||fi|\\
\end{tabular}
\end{center}
%
Instead of |\childdocof{|\textit{main}|}| just include the main file
at the top of each child file:
%
\begin{center}
|\input{|\textit{main}|}|
\end{center}
%
A simple redirection |\childdocforward{|\textit{dest}|}| is achieved by:
%
\begin{center}
|\def\jobname{|\textit{dest}|}\input{\jobname}|
\end{center}
%
The redirection with prefix
|\childdocforwardprefix[|\textit{prefix}|]{|\textit{dest}|}|
is accomplished by:
%
\begin{center}
\begin{tabular}{l}
|{\edef\jobname{\scantokens\expandafter{\jobname\noexpand}}|\\
|\def\redirectjob |\textit{prefix}|#1~~~{\gdef\jobname{|\textit{dest}|#1}}|\\
|\expandafter\redirectjob\jobname~~~}\input{\jobname}|
\end{tabular}
\end{center}

In an alternative approach,
child documents can be compiled by a specific command line
without additional code or specific definitions:
%
\begin{center}
|... -jobname "|\textit{target}|" "|[\textit{flags}]%
|\includeonly{|\textit{dest}|}\input{|\textit{main}|}"|
\end{center}
%

%%%%%%%%%%%%%%%%%%%%%%%%%%%%%%%%%%%%%%%%%%%%%%%%%%%%%%%%%%%%%%%%%%%%%%%%%%%%%%%%
%%%%%%%%%%%%%%%%%%%%%%%%%%%%%%%%%%%%%%%%%%%%%%%%%%%%%%%%%%%%%%%%%%%%%%%%%%%%%%%%
\section{Information}

%%%%%%%%%%%%%%%%%%%%%%%%%%%%%%%%%%%%%%%%%%%%%%%%%%%%%%%%%%%%%%%%%%%%%%%%%%%%%%%%
\subsection{Copyright}

Copyright \copyright{} 2017--2018 Niklas Beisert

This work may be distributed and/or modified under the
conditions of the \LaTeX{} Project Public License, either version 1.3
of this license or (at your option) any later version.
The latest version of this license is in
  \url{http://www.latex-project.org/lppl.txt}
and version 1.3 or later is part of all distributions of \LaTeX{}
version 2005/12/01 or later.

This work has the LPPL maintenance status `maintained'.

The Current Maintainer of this work is Niklas Beisert.

This work consists of the files |README.txt|, |childdoc.ins| and |childdoc.dtx|
as well as the derived files |childdoc.def|, |cdocsamp.tex|
with |cdocsch1.tex|, |cdocsch2.tex|, |cdocspt3.tex|, |cdocspt4.tex|,
|cdocsdrf.tex|, |cdocsfn1.tex|, |cdocsfn2.tex|
as well as |childdoc.pdf|.

%%%%%%%%%%%%%%%%%%%%%%%%%%%%%%%%%%%%%%%%%%%%%%%%%%%%%%%%%%%%%%%%%%%%%%%%%%%%%%%%
\subsection{Files and Installation}

The package consists of the files:
%
\begin{center}
\begin{tabular}{ll}
    |README.txt|   & readme file \\
    |childdoc.ins| & installation file \\
    |childdoc.dtx| & source file \\
    |childdoc.def| & definition file \\
    |cdocsamp.tex| & sample main file \\
    |cdocsch1.tex| & sample include file \\
    |cdocsch2.tex| & sample include file \\
    |cdocspt3.tex| & sample part file \\
    |cdocspt4.tex| & sample part file \\
    |cdocsdrf.tex| & sample redirection file \\
    |cdocsfn1.tex| & sample redirection file \\
    |cdocsfn2.tex| & sample redirection file \\
    |childdoc.pdf| & manual
\end{tabular}
\end{center}
%
The distribution consists of the files
|README.txt|, |childdoc.ins| and |childdoc.dtx|.
%
\begin{itemize}
\item
Run (pdf)\LaTeX{} on |childdoc.dtx|
to compile the manual |childdoc.pdf| (this file).
\item
Run \LaTeX{} on |childdoc.ins| to create the definitions file |childdoc.def|
and the sample |cdocsamp.tex| with include files
|cdocsch1.tex|, |cdocsch2.tex|, |cdocspt3.tex|, |cdocspt4.tex|,
|cdocsdrf.tex|, |cdocsfn1.tex|, |cdocsfn2.tex|.
Then copy the file |childdoc.def| to an appropriate directory of your \LaTeX{}
distribution, e.g.\ \textit{texmf-root}|/tex/latex/childdoc|.
\end{itemize}

%%%%%%%%%%%%%%%%%%%%%%%%%%%%%%%%%%%%%%%%%%%%%%%%%%%%%%%%%%%%%%%%%%%%%%%%%%%%%%%%
\subsection{Related CTAN Packages}

There are several other packages which offer a similar functionality:
%
\begin{itemize}
\item
The packages
\href{http://ctan.org/pkg/docmute}{\textsf{docmute}},
\href{http://ctan.org/pkg/includex}{\textsf{includex}} and
\href{http://ctan.org/pkg/standalone}{\textsf{standalone}}
provide commands to include only the document body of
a child file thus allowing both files to be compiled individually.
\item
The packages \href{http://ctan.org/pkg/subdocs}{\textsf{subdocs}}
and \href{http://ctan.org/pkg/subfiles}{\textsf{subfiles}}
provide structures in which the main and child documents can be
encapsulated and allowing them to be compiled individually.
The inclusion mechanism is different from the conventional |\include|.
\item
The package \href{http://ctan.org/pkg/combine}{\textsf{combine}}
is an elaborate solution to combine several documents into one.
\end{itemize}
%
See also the CTAN topic \href{http://ctan.org/topic/subdocs}{\textsf{subdocs}}
for further related packages.
The present package differs from the above solutions in that
a document structure constructed with the conventional |\include| mechanism
just needs two extra commands at the top of every file
such that all constituent files can be compiled individually.

%%%%%%%%%%%%%%%%%%%%%%%%%%%%%%%%%%%%%%%%%%%%%%%%%%%%%%%%%%%%%%%%%%%%%%%%%%%%%%%%
%\subsection{Feature Suggestions}
%
%The following is a list of features which may be useful for future
%versions of this package:
%%
%\begin{itemize}
%\item
%\ldots
%\end{itemize}

%%%%%%%%%%%%%%%%%%%%%%%%%%%%%%%%%%%%%%%%%%%%%%%%%%%%%%%%%%%%%%%%%%%%%%%%%%%%%%%%
\subsection{Revision History}

%%%%%%%%%%%%%%%%%%%%%%%%%%%%%%%%%%%%%%%%
\paragraph{v2.0:} 2018/12/30

\begin{itemize}
\item
immediate forward processing
\item
added |\childdocby| mechanism
\item
manual restructured
\end{itemize}

%%%%%%%%%%%%%%%%%%%%%%%%%%%%%%%%%%%%%%%%
\paragraph{v1.6:} 2018/01/17

\begin{itemize}
\item
application for development of include files
\item
corrections to manual
\end{itemize}

%%%%%%%%%%%%%%%%%%%%%%%%%%%%%%%%%%%%%%%%
\paragraph{v1.5:} 2017/05/21

\begin{itemize}
\item
more complete structuring introduced
\item
|\childdocof| introduced
\item
|\childdoc| renamed to |\childdocmain|
\item
|\childredirect| renamed to |\childdocforward| and |\childdocforwardprefix|
and functionality expanded
\end{itemize}

%%%%%%%%%%%%%%%%%%%%%%%%%%%%%%%%%%%%%%%%
\paragraph{v1.0:} 2017/04/27

\begin{itemize}
\item
manual and install package
\item
first version published on CTAN
\end{itemize}

%%%%%%%%%%%%%%%%%%%%%%%%%%%%%%%%%%%%%%%%
\paragraph{v0.6:} 2017/04/26

\begin{itemize}
\item
redirection mechanism added
\end{itemize}

%%%%%%%%%%%%%%%%%%%%%%%%%%%%%%%%%%%%%%%%
\paragraph{v0.5:} 2017/04/26

\begin{itemize}
\item
functionality in definition file
\end{itemize}


%%%%%%%%%%%%%%%%%%%%%%%%%%%%%%%%%%%%%%%%%%%%%%%%%%%%%%%%%%%%%%%%%%%%%%%%%%%%%%%%
%%%%%%%%%%%%%%%%%%%%%%%%%%%%%%%%%%%%%%%%%%%%%%%%%%%%%%%%%%%%%%%%%%%%%%%%%%%%%%%%
%%%%%%%%%%%%%%%%%%%%%%%%%%%%%%%%%%%%%%%%%%%%%%%%%%%%%%%%%%%%%%%%%%%%%%%%%%%%%%%%
\appendix

\settowidth\MacroIndent{\rmfamily\scriptsize 000\ }

 \DocInput{childdoc.dtx}

\end{document}
%</driver>
% \fi
%
% %%%%%%%%%%%%%%%%%%%%%%%%%%%%%%%%%%%%%%%%%%%%%%%%%%%%%%%%%%%%%%%%%%%%%%%%%%%%%%
% %%%%%%%%%%%%%%%%%%%%%%%%%%%%%%%%%%%%%%%%%%%%%%%%%%%%%%%%%%%%%%%%%%%%%%%%%%%%%%
% \section{Sample}
%\iffalse
%<*samplemain>
%\fi
%
% The following presents a sample document
% with two chapters, two parts, a title page,
% a compile flag as well as three forwarding files to set the flag.
% It consists of eight |.tex| files:
% \begin{center}
% \begin{tabular}{ll}
% |cdocsamp.tex|&main file\\
% |cdocsch1.tex|&include file for chapter 1\\
% |cdocsch2.tex|&include file for chapter 2\\
% |cdocspt3.tex|&include file for part 3\\
% |cdocspt4.tex|&include file for part 4\\
% |cdocsdrf.tex|&forwarding file for main file in draft mode\\
% |cdocsfi1.tex|&forwarding file for final version of chapter 1\\
% |cdocsfi2.tex|&forwarding file for final version of chapter 2\\
% \end{tabular}
% \end{center}
% Each of the eight files can be compiled directly by the \LaTeX{} compiler.
%
% %%%%%%%%%%%%%%%%%%%%%%%%%%%%%%%%%%%%%%
% \paragraph{Main File.}
%
% The main file is called |cdocsamp.tex|.
%
% Load the \textsf{childdoc} definitions and
% declare the filename for the main document:
%    \begin{macrocode}
\input{childdoc.def}
\childdocmain{}
%    \end{macrocode}

% Optional override for |\version| flag:
%    \begin{macrocode}
%%\ifchilddoc\else\providecommand{\version}{draft}\fi
%    \end{macrocode}

% Define the default values for the |\version| flag
% (|final| for the main file and |draft| for childs):
%    \begin{macrocode}
\ifchilddoc
\providecommand{\version}{draft}
\else
\providecommand{\version}{final}
\fi
%    \end{macrocode}

% Load the standard document class:
%    \begin{macrocode}
\documentclass[12pt]{article}
%    \end{macrocode}

% Start the document body:
%    \begin{macrocode}
\begin{document}
%    \end{macrocode}

% Declare a title page.
% Print title, part of document being processed and version flag:
%    \begin{macrocode}
\addtocounter{page}{-1}
\begin{center}
{\LARGE\bfseries{}childdoc example\par}
\vspace{1cm}
\ifchilddoc
\ifchilddocmanual part\else chapter\fi:
`\childdocname' of `\childdocjob'\par
\else
main document: `\childdocjob'\par
\fi
version: \version\par
\end{center}
\newpage
%    \end{macrocode}

% Manually include selected file,
% otherwise process as usual:
%    \begin{macrocode}
\ifchilddocmanual
\section*{part `\childdocname'}
\input{\childdocname}
\else
%    \end{macrocode}

% Include the two chapters:
%    \begin{macrocode}
\include{cdocsch1}
\include{cdocsch2}
%    \end{macrocode}

% Include the two parts unless only chapters should be displayed:
%    \begin{macrocode}
\ifchilddoc\else
\section{part three}
\input{cdocspt3}
\section{part four}
\input{cdocspt4}
\fi
%    \end{macrocode}

% Process as usual until here:
%    \begin{macrocode}
\fi
%    \end{macrocode}

% End of document body:
%    \begin{macrocode}
\end{document}
%    \end{macrocode}
%\iffalse
%</samplemain>
%\fi
%
% %%%%%%%%%%%%%%%%%%%%%%%%%%%%%%%%%%%%%%
% \paragraph{Chapter Include Files.}
%
% The include files are called |cdocsch1.tex| and |cdocsch2.tex|.
%
%\iffalse
%<*samplechap1|samplechap2>
%\fi

% Optional override for |\version| flag:
%    \begin{macrocode}
%%\providecommand{\version}{final}
%    \end{macrocode}

% Include the main document:
%    \begin{macrocode}
\input{childdoc.def}
\childdocof{cdocsamp}
%    \end{macrocode}

%\iffalse
%</samplechap1|samplechap2>
%\fi
%
%\iffalse
%<*samplechap1>
%\fi
% Some text for chapter 1:
%    \begin{macrocode}
\section{one}
some text in chapter one
%    \end{macrocode}

%\iffalse
%</samplechap1>
%\fi
% Some text for chapter 2:
%\iffalse
%<*samplechap2>
%\fi
%    \begin{macrocode}
\section{two}
more text in chapter two
%    \end{macrocode}

%\iffalse
%</samplechap2>
%\fi
%
% %%%%%%%%%%%%%%%%%%%%%%%%%%%%%%%%%%%%%%
% \paragraph{Part Include Files.}
%
% The include files are called |cdocspt3.tex| and |cdocspt4.tex|.
%
%\iffalse
%<*samplepart3|samplepart4>
%\fi

% Optional override for |\version| flag:
%    \begin{macrocode}
%%\providecommand{\version}{final}
%    \end{macrocode}

% Include the main document:
%    \begin{macrocode}
\input{childdoc.def}
\childdocby{cdocsamp}
%    \end{macrocode}

%\iffalse
%</samplepart3|samplepart4>
%\fi
%
%\iffalse
%<*samplepart3>
%\fi
% Some text for part 3:
%    \begin{macrocode}
some text in part three
%    \end{macrocode}

%\iffalse
%</samplepart3>
%\fi
% Some text for part 4:
%\iffalse
%<*samplepart4>
%\fi
%    \begin{macrocode}
more text in part four
%    \end{macrocode}

%\iffalse
%</samplepart4>
%\fi
%
% %%%%%%%%%%%%%%%%%%%%%%%%%%%%%%%%%%%%%%
% \paragraph{Forwarding for a Complete Draft.}
%
% The following forwarding file |cdocsdrf.tex|
% compiles the main document in draft mode:
%\iffalse
%<*sampledraft>
%\fi
%    \begin{macrocode}
\def\version{draft}
\input{childdoc.def}
\childdocforward{cdocsamp}
%    \end{macrocode}

%\iffalse
%</sampledraft>
%\fi
%
% %%%%%%%%%%%%%%%%%%%%%%%%%%%%%%%%%%%%%%
% \paragraph{Forwarding for Final Version of the Chapters.}
%
% The following forwarding files |cdocsfn1.tex| and |cdocsfn2.tex|
% (with identical content)
% compile the final versions of the child documents
% |cdocsch1.tex| and |cdocsch2.tex|, respectively:
%\iffalse
%<*samplefinal>
%\fi
%    \begin{macrocode}
\def\version{final}
\input{childdoc.def}
\childdocforwardprefix[cdocsamp]{cdocsfn}{cdocsch}
%    \end{macrocode}

%\iffalse
%</samplefinal>
%\fi
%
% %%%%%%%%%%%%%%%%%%%%%%%%%%%%%%%%%%%%%%
% \paragraph{Command Line Processing.}
%
% The following three command lines generate the output files
% |cdocscld|, |cdocscl1| and |cdocscl2|
% which should be identical to
% |cdocsdrf|, |cdocsch1| and |cdocsfn2|, respectively:
% \begin{center}
% \begin{tabular}{l}
% |latex -jobname cdocscld \|\\
% |  "\def\version{draft}\input{childdoc.def}\childdocforward{cdocsamp}"|\\
% |latex -jobname cdocscl1 \|\\
% |  "\input{childdoc.def}\childdocforward[cdocsamp]{cdocsch1}"|\\
% |latex -jobname cdocscl2 \|\\
% |  "\def\version{final}\input{childdoc.def}\childdocforward{cdocsch2}"|
% \end{tabular}
% \end{center}
% Note that the trailing backslash on each first line
% merely continues the input to the second line
% (for convenient cut ant paste).
% Furthermore, the command |latex| can be replaced by any
% of its alternative versions such as |pdflatex|.
%
% %%%%%%%%%%%%%%%%%%%%%%%%%%%%%%%%%%%%%%%%%%%%%%%%%%%%%%%%%%%%%%%%%%%%%%%%%%%%%%
% %%%%%%%%%%%%%%%%%%%%%%%%%%%%%%%%%%%%%%%%%%%%%%%%%%%%%%%%%%%%%%%%%%%%%%%%%%%%%%
% \section{Implementation}
%\iffalse
%<*package>
%\fi
%
% This section describes the definitions file |childdoc.def|.

% The definitions cannot be loaded using |\usepackage| or |\RequirePackage|
% which has a mechanism to prevent loading a style file more than once.
% When loading the definitions by means of |\input|
% multiple instances have to be prevented manually:
%\iffalse
%This code needs to be before the `\ProvidesFile' directive
%which is defined at the beginning of this file.
%Therefore it is also placed there and commented out here.
%</package>
%<*discard>
%\fi
%    \begin{macrocode}
\ifdefined\childdocmain\endinput\fi
%    \end{macrocode}
%\iffalse
%</discard>
%<*package>
%\fi
%
% \macro{\ifchilddoc}
% \macro{\ifchilddocmanual}
% The conditional |\ifchilddoc| tells whether a
% child (true) or main (false) document is being compiled.
% The conditional |\ifchilddocmanual| tells whether
% the |\includeonly| mechanism is used (false) or
% the selection of child files must be performed manually (true).
% The definitions initialise to false:
%    \begin{macrocode}
\newif\ifchilddoc
\newif\ifchilddocmanual
%    \end{macrocode}

% \macro{\childdocname}
% \macro{\childdocjob}
% The macro |\childdocname| stores the name of the main document
% to be compiled. The macro |\childdocjob| stores the name of
% the document on which the \LaTeX{} compiler was originally invoked.
% The content of |\jobname| cannot be compared
% to filenames specified in the source due to different catcodes.
% The following code rescans |\jobname|, stores the result
% in |\childdocname| and saves a copy in |\childdocjob|:
%    \begin{macrocode}
\edef\childdocname{\scantokens\expandafter{\jobname\noexpand}}
\let\childdocjob\childdocname
%    \end{macrocode}

% \macro{\childdocdisable}
% The macro |\childdocdisable| prevents the main file
% from being processed more than once.
% At this stage, the main document command |\childdocmain|
% is assumed to be called once again where it should do nothing.
% Any subsequent call to it should prevent
% a secondary processing of the main document
% It overwrites the forwarding commands
% |\childdocof| and |\childdocforward|
% with empty macros to prevent further inclusions of the main document:
%    \begin{macrocode}
\newcommand{\childdocdisable}
{
  \renewcommand{\childdocmain}[1]{\renewcommand{\childdocmain}[1]{\endinput}}
  \renewcommand{\childdocof}[1]{}
  \renewcommand{\childdocby}[2][]{}
  \renewcommand{\childdocforward}[2][]{}
  \renewcommand{\childdocdisable}{}
}
%    \end{macrocode}

% \macro{\childdocmain}
% The macro |\childdocmain| is to be called at the top of the main file
% with nothing or the main filename (without extension) as argument.
% First, it breaks loops.
% If the argument is not empty and does not match |\childdocname|
% (which is set by the first inclusion of |childdoc.def|),
% |\ifchilddoc| is set to true, |\includeonly| is applied to the child file
% and |\jobname| is set to the main file
% (for proper handling of |.aux| files):
%    \begin{macrocode}
\newcommand{\childdocmain}[1]
{
  \childdocdisable\childdocmain{}
  \if?#1?\else
    \begingroup
      \def\childdoctmp{#1}
      \ifx\childdoctmp\childdocname
        \def\childdoctmp{}
      \else
        \def\childdoctmp
        {
          \childdoctrue
          \includeonly{\childdocname}
          \def\childdocjob{#1}
          \def\jobname{#1}
        }
      \fi
      \expandafter
    \endgroup
    \childdoctmp
  \fi
}
%    \end{macrocode}

% \macro{\childdocof}
% The command |\childdocof| redirects
% compilation to the main file |#1|.
%    \begin{macrocode}
\newcommand{\childdocof}[1]
{
  \childdocdisable
  \childdoctrue
  \includeonly{\childdocname}
  \def\jobname{#1}
  \def\childdocjob{#1}
  \input{#1}
}
%    \end{macrocode}

% \macro{\childdocby}
% The command |\childdocby| ....
%    \begin{macrocode}
\newcommand{\childdocby}[2][]
{
  \childdocdisable
  \childdoctrue
  \childdocmanualtrue
  \if?#1?\else
    \def\jobname{#2}
  \fi
  \def\childdocjob{#2}
  \input{#2}
  \endinput
}
%    \end{macrocode}

% \macro{\childdocforward}
% The command |\childdocforward| redirects
% compilation to the main file or
% (if the optional argument is given) a child file.
% Parameters are set as if the main file
% or a child file starting with |\childdocof| was compiled.
% Then compilation is handed over to the main file:
%    \begin{macrocode}
\newcommand{\childdocforward}[2][]
{
  \begingroup
    \if?#1?
      \def\childdoctmp
      {
        \def\childdocname{#2}
        \def\childdocjob{#2}
        \def\jobname{#2}
        \input{#2}
        \endinput
      }
    \else
      \def\childdoctmp
      {
        \childdocdisable
        \def\childdocname{#2}
        \childdoctrue
        \includeonly{#2}
        \def\childdocjob{#1}
        \def\jobname{#1}
        \input{#1}
        \endinput
      }
    \fi
    \expandafter
  \endgroup
  \childdoctmp
}
%    \end{macrocode}

% \macro{\childdocforwardprefix}
% The command |\childdocforwardprefix| redirects
% compilation to the main or a child file by means of a pattern.
% The prefix |#1| in the current filename is replaced by |#2|
% and the suffix of the current filename is kept
% (it is assumed that the filename does not contain the substring `|~~~|'
% which is used as a delimiter).
% Compilation is handed over to the new file by |\childdocforward|:
%    \begin{macrocode}
\newcommand{\childdocforwardprefix}[3][]
{
  \begingroup
    \def\childdocextract #2##1~~~{\def\childdoctmp{\childdocforward[#1]{#3##1}}}
    \expandafter\childdocextract\childdocname~~~
    \expandafter
  \endgroup
  \childdoctmp
}
%    \end{macrocode}

% \macro{\childdoc}
% The deprecated macro |\childdoc| is a legacy version of |\childdocmain|:
%    \begin{macrocode}
\newcommand{\childdoc}{\childdocmain}
%    \end{macrocode}

% \macro{\childdocredirect}
% The deprecated macro |\childdocredirect| is a legacy version
% of |\childdocforward| and |\childdocforwardprefix|:
%    \begin{macrocode}
\newcommand{\childdocredirect}[2][]
{
  \begingroup
    \if?#1?
      \def\childdoctmp{\childdocforward{#2}}
    \else
      \def\childdoctmp{\childdocforwardprefix{#1}{#2}}
    \fi
    \expandafter
  \endgroup
  \childdoctmp
}
%    \end{macrocode}

%\iffalse
%</package>
%\fi
%
\endinput
|
and perform the replacements as outlined below.
Instead of |\childdocmain{|\textit{main}|}| add the following code
to the top of the main file:
%
\begin{center}
\begin{tabular}{l}
|\||ifdefined\childdocname\endinput\||fi\newif\ifchilddoc|\\
|\edef\childdocname{\scantokens\expandafter{\jobname\noexpand}}|\\
|\def\childdocmain{|\textit{main}|}\||ifx\childdocmain\childdocname\||else|\\
|\childdoctrue\includeonly{\childdocname}\let\jobname\childdocmain\||fi|\\
\end{tabular}
\end{center}
%
Instead of |\childdocof{|\textit{main}|}| just include the main file
at the top of each child file:
%
\begin{center}
|\input{|\textit{main}|}|
\end{center}
%
A simple redirection |\childdocforward{|\textit{dest}|}| is achieved by:
%
\begin{center}
|\def\jobname{|\textit{dest}|}\input{\jobname}|
\end{center}
%
The redirection with prefix
|\childdocforwardprefix[|\textit{prefix}|]{|\textit{dest}|}|
is accomplished by:
%
\begin{center}
\begin{tabular}{l}
|{\edef\jobname{\scantokens\expandafter{\jobname\noexpand}}|\\
|\def\redirectjob |\textit{prefix}|#1~~~{\gdef\jobname{|\textit{dest}|#1}}|\\
|\expandafter\redirectjob\jobname~~~}\input{\jobname}|
\end{tabular}
\end{center}

In an alternative approach,
child documents can be compiled by a specific command line
without additional code or specific definitions:
%
\begin{center}
|... -jobname "|\textit{target}|" "|[\textit{flags}]%
|\includeonly{|\textit{dest}|}\input{|\textit{main}|}"|
\end{center}
%

%%%%%%%%%%%%%%%%%%%%%%%%%%%%%%%%%%%%%%%%%%%%%%%%%%%%%%%%%%%%%%%%%%%%%%%%%%%%%%%%
%%%%%%%%%%%%%%%%%%%%%%%%%%%%%%%%%%%%%%%%%%%%%%%%%%%%%%%%%%%%%%%%%%%%%%%%%%%%%%%%
\section{Information}

%%%%%%%%%%%%%%%%%%%%%%%%%%%%%%%%%%%%%%%%%%%%%%%%%%%%%%%%%%%%%%%%%%%%%%%%%%%%%%%%
\subsection{Copyright}

Copyright \copyright{} 2017--2018 Niklas Beisert

This work may be distributed and/or modified under the
conditions of the \LaTeX{} Project Public License, either version 1.3
of this license or (at your option) any later version.
The latest version of this license is in
  \url{http://www.latex-project.org/lppl.txt}
and version 1.3 or later is part of all distributions of \LaTeX{}
version 2005/12/01 or later.

This work has the LPPL maintenance status `maintained'.

The Current Maintainer of this work is Niklas Beisert.

This work consists of the files |README.txt|, |childdoc.ins| and |childdoc.dtx|
as well as the derived files |childdoc.def|, |cdocsamp.tex|
with |cdocsch1.tex|, |cdocsch2.tex|, |cdocspt3.tex|, |cdocspt4.tex|,
|cdocsdrf.tex|, |cdocsfn1.tex|, |cdocsfn2.tex|
as well as |childdoc.pdf|.

%%%%%%%%%%%%%%%%%%%%%%%%%%%%%%%%%%%%%%%%%%%%%%%%%%%%%%%%%%%%%%%%%%%%%%%%%%%%%%%%
\subsection{Files and Installation}

The package consists of the files:
%
\begin{center}
\begin{tabular}{ll}
    |README.txt|   & readme file \\
    |childdoc.ins| & installation file \\
    |childdoc.dtx| & source file \\
    |childdoc.def| & definition file \\
    |cdocsamp.tex| & sample main file \\
    |cdocsch1.tex| & sample include file \\
    |cdocsch2.tex| & sample include file \\
    |cdocspt3.tex| & sample part file \\
    |cdocspt4.tex| & sample part file \\
    |cdocsdrf.tex| & sample redirection file \\
    |cdocsfn1.tex| & sample redirection file \\
    |cdocsfn2.tex| & sample redirection file \\
    |childdoc.pdf| & manual
\end{tabular}
\end{center}
%
The distribution consists of the files
|README.txt|, |childdoc.ins| and |childdoc.dtx|.
%
\begin{itemize}
\item
Run (pdf)\LaTeX{} on |childdoc.dtx|
to compile the manual |childdoc.pdf| (this file).
\item
Run \LaTeX{} on |childdoc.ins| to create the definitions file |childdoc.def|
and the sample |cdocsamp.tex| with include files
|cdocsch1.tex|, |cdocsch2.tex|, |cdocspt3.tex|, |cdocspt4.tex|,
|cdocsdrf.tex|, |cdocsfn1.tex|, |cdocsfn2.tex|.
Then copy the file |childdoc.def| to an appropriate directory of your \LaTeX{}
distribution, e.g.\ \textit{texmf-root}|/tex/latex/childdoc|.
\end{itemize}

%%%%%%%%%%%%%%%%%%%%%%%%%%%%%%%%%%%%%%%%%%%%%%%%%%%%%%%%%%%%%%%%%%%%%%%%%%%%%%%%
\subsection{Related CTAN Packages}

There are several other packages which offer a similar functionality:
%
\begin{itemize}
\item
The packages
\href{http://ctan.org/pkg/docmute}{\textsf{docmute}},
\href{http://ctan.org/pkg/includex}{\textsf{includex}} and
\href{http://ctan.org/pkg/standalone}{\textsf{standalone}}
provide commands to include only the document body of
a child file thus allowing both files to be compiled individually.
\item
The packages \href{http://ctan.org/pkg/subdocs}{\textsf{subdocs}}
and \href{http://ctan.org/pkg/subfiles}{\textsf{subfiles}}
provide structures in which the main and child documents can be
encapsulated and allowing them to be compiled individually.
The inclusion mechanism is different from the conventional |\include|.
\item
The package \href{http://ctan.org/pkg/combine}{\textsf{combine}}
is an elaborate solution to combine several documents into one.
\end{itemize}
%
See also the CTAN topic \href{http://ctan.org/topic/subdocs}{\textsf{subdocs}}
for further related packages.
The present package differs from the above solutions in that
a document structure constructed with the conventional |\include| mechanism
just needs two extra commands at the top of every file
such that all constituent files can be compiled individually.

%%%%%%%%%%%%%%%%%%%%%%%%%%%%%%%%%%%%%%%%%%%%%%%%%%%%%%%%%%%%%%%%%%%%%%%%%%%%%%%%
%\subsection{Feature Suggestions}
%
%The following is a list of features which may be useful for future
%versions of this package:
%%
%\begin{itemize}
%\item
%\ldots
%\end{itemize}

%%%%%%%%%%%%%%%%%%%%%%%%%%%%%%%%%%%%%%%%%%%%%%%%%%%%%%%%%%%%%%%%%%%%%%%%%%%%%%%%
\subsection{Revision History}

%%%%%%%%%%%%%%%%%%%%%%%%%%%%%%%%%%%%%%%%
\paragraph{v2.0:} 2018/12/30

\begin{itemize}
\item
immediate forward processing
\item
added |\childdocby| mechanism
\item
manual restructured
\end{itemize}

%%%%%%%%%%%%%%%%%%%%%%%%%%%%%%%%%%%%%%%%
\paragraph{v1.6:} 2018/01/17

\begin{itemize}
\item
application for development of include files
\item
corrections to manual
\end{itemize}

%%%%%%%%%%%%%%%%%%%%%%%%%%%%%%%%%%%%%%%%
\paragraph{v1.5:} 2017/05/21

\begin{itemize}
\item
more complete structuring introduced
\item
|\childdocof| introduced
\item
|\childdoc| renamed to |\childdocmain|
\item
|\childredirect| renamed to |\childdocforward| and |\childdocforwardprefix|
and functionality expanded
\end{itemize}

%%%%%%%%%%%%%%%%%%%%%%%%%%%%%%%%%%%%%%%%
\paragraph{v1.0:} 2017/04/27

\begin{itemize}
\item
manual and install package
\item
first version published on CTAN
\end{itemize}

%%%%%%%%%%%%%%%%%%%%%%%%%%%%%%%%%%%%%%%%
\paragraph{v0.6:} 2017/04/26

\begin{itemize}
\item
redirection mechanism added
\end{itemize}

%%%%%%%%%%%%%%%%%%%%%%%%%%%%%%%%%%%%%%%%
\paragraph{v0.5:} 2017/04/26

\begin{itemize}
\item
functionality in definition file
\end{itemize}


%%%%%%%%%%%%%%%%%%%%%%%%%%%%%%%%%%%%%%%%%%%%%%%%%%%%%%%%%%%%%%%%%%%%%%%%%%%%%%%%
%%%%%%%%%%%%%%%%%%%%%%%%%%%%%%%%%%%%%%%%%%%%%%%%%%%%%%%%%%%%%%%%%%%%%%%%%%%%%%%%
%%%%%%%%%%%%%%%%%%%%%%%%%%%%%%%%%%%%%%%%%%%%%%%%%%%%%%%%%%%%%%%%%%%%%%%%%%%%%%%%
\appendix

\settowidth\MacroIndent{\rmfamily\scriptsize 000\ }

 \DocInput{childdoc.dtx}

\end{document}
%</driver>
% \fi
%
% %%%%%%%%%%%%%%%%%%%%%%%%%%%%%%%%%%%%%%%%%%%%%%%%%%%%%%%%%%%%%%%%%%%%%%%%%%%%%%
% %%%%%%%%%%%%%%%%%%%%%%%%%%%%%%%%%%%%%%%%%%%%%%%%%%%%%%%%%%%%%%%%%%%%%%%%%%%%%%
% \section{Sample}
%\iffalse
%<*samplemain>
%\fi
%
% The following presents a sample document
% with two chapters, two parts, a title page,
% a compile flag as well as three forwarding files to set the flag.
% It consists of eight |.tex| files:
% \begin{center}
% \begin{tabular}{ll}
% |cdocsamp.tex|&main file\\
% |cdocsch1.tex|&include file for chapter 1\\
% |cdocsch2.tex|&include file for chapter 2\\
% |cdocspt3.tex|&include file for part 3\\
% |cdocspt4.tex|&include file for part 4\\
% |cdocsdrf.tex|&forwarding file for main file in draft mode\\
% |cdocsfi1.tex|&forwarding file for final version of chapter 1\\
% |cdocsfi2.tex|&forwarding file for final version of chapter 2\\
% \end{tabular}
% \end{center}
% Each of the eight files can be compiled directly by the \LaTeX{} compiler.
%
% %%%%%%%%%%%%%%%%%%%%%%%%%%%%%%%%%%%%%%
% \paragraph{Main File.}
%
% The main file is called |cdocsamp.tex|.
%
% Load the \textsf{childdoc} definitions and
% declare the filename for the main document:
%    \begin{macrocode}
% \iffalse
%
% childdoc.dtx Copyright (C) 2017-2018 Niklas Beisert
%
% This work may be distributed and/or modified under the
% conditions of the LaTeX Project Public License, either version 1.3
% of this license or (at your option) any later version.
% The latest version of this license is in
%   http://www.latex-project.org/lppl.txt
% and version 1.3 or later is part of all distributions of LaTeX
% version 2005/12/01 or later.
%
% This work has the LPPL maintenance status `maintained'.
%
% The Current Maintainer of this work is Niklas Beisert.
%
% This work consists of the files childdoc.dtx and childdoc.ins
% and the derived files childdoc.def and cdocsamp.tex with
% cdocsch1.tex, cdocsch2.tex, cdocsdrf.tex, cdocsfn1.tex, cdocsfn2.tex.
%
%<package>\ifdefined\childdocmain\endinput\fi
%<package>\ProvidesFile{childdoc.def}[2018/12/30 v2.0 child document driver]
%<samplemain>\ProvidesFile{cdocsamp.tex}[2018/12/30 v2.0 sample for childdoc]
%<*driver>
%\ProvidesFile{childdoc.drv}[2018/12/30 v2.0 childdoc reference manual file]
\PassOptionsToClass{10pt,a4paper}{article}
\documentclass{ltxdoc}

\usepackage[margin=35mm]{geometry}
\usepackage{hyperref}
\usepackage{hyperxmp}
\usepackage[usenames]{color}

\hypersetup{colorlinks=true}
\hypersetup{pdfstartview=FitH}
\hypersetup{pdfpagemode=UseNone}
\hypersetup{pdfsource={}}
\hypersetup{pdflang={en-UK}}
\hypersetup{pdfcopyright={Copyright 2017-2018 Niklas Beisert.
  This work may be distributed and/or modified under the
  conditions of the LaTeX Project Public License, either version 1.3
  of this license or (at your option) any later version.}}
\hypersetup{pdflicenseurl={http://www.latex-project.org/lppl.txt}}
\hypersetup{pdfcontactaddress={ETH Zurich, ITP, HIT K,
  Wolfgang-Pauli-Strasse 27}}
\hypersetup{pdfcontactpostcode={8093}}
\hypersetup{pdfcontactcity={Zurich}}
\hypersetup{pdfcontactcountry={Switzerland}}
\hypersetup{pdfcontactemail={nbeisert@itp.phys.ethz.ch}}
\hypersetup{pdfcontacturl={http://people.phys.ethz.ch/\xmptilde nbeisert/}}

\newcommand{\secref}[1]{\hyperref[#1]{section \ref*{#1}}}

\parskip1ex
\parindent0pt
\let\olditemize\itemize
\def\itemize{\olditemize\parskip0pt}

\begin{document}

\title{The \textsf{childdoc} Package}
\hypersetup{pdftitle={The childdoc Package}}
\author{Niklas Beisert\\[2ex]
  Institut f\"ur Theoretische Physik\\
  Eidgen\"ossische Technische Hochschule Z\"urich\\
  Wolfgang-Pauli-Strasse 27, 8093 Z\"urich, Switzerland\\[1ex]
  \href{mailto:nbeisert@itp.phys.ethz.ch}
  {\texttt{nbeisert@itp.phys.ethz.ch}}}
\hypersetup{pdfauthor={Niklas Beisert}}
\hypersetup{pdfsubject={Manual for the LaTeX2e Package childdoc}}
\date{30 December 2018, \textsf{v2.0}}
\maketitle

\begin{abstract}\noindent
\textsf{childdoc} is a \LaTeXe{} package
that enables the direct compilation
of document sections included by |\include|
to individual files.
\end{abstract}

\begingroup
\parskip0ex
\tableofcontents
\endgroup

%%%%%%%%%%%%%%%%%%%%%%%%%%%%%%%%%%%%%%%%%%%%%%%%%%%%%%%%%%%%%%%%%%%%%%%%%%%%%%%%
%%%%%%%%%%%%%%%%%%%%%%%%%%%%%%%%%%%%%%%%%%%%%%%%%%%%%%%%%%%%%%%%%%%%%%%%%%%%%%%%
\section{Introduction}

\LaTeX{} provides a mechanism to structure a large document (such as a book)
into a main file and several child files (containing the chapters)
using the |\include| command.
This mechanism is beneficial for documents
which span hundreds of pages in order to
make the source file(s) more manageable.
Moreover, compilation can be restricted to
selected child files by means of the |\includeonly| command.
The latter feature can be used to reduce the compilation time while editing
(this was significantly more useful in the earlier days of \LaTeX{})
or to generate a smaller document which is easier to navigate.
Another application of |\includeonly| is to generate
documents consisting of selected parts of the complete document.

However, there are a few drawbacks of the plain |\include| mechanism:
\begin{itemize}
\item
The child files cannot be compiled on their own,
they can only be compiled via the main file.
A naive editing environment
(such as a text editor with an option
to have the current file processed by \LaTeX)
may require one to switch to the main file before compiling;
attempting to compile the child file produces errors.
\item
The main file must be modified (each time)
to adjust the |\includeonly| command
to the present needs. This easily leaves the main file in a messy state.
\item
The generated document will always carry the filename
of the main document. This is inconvenient if
several child files are to be compiled and
to be kept for distribution.
\end{itemize}

The present package provides a simple interface
to make child files individually compilable by \LaTeX{}.
Compiling a child file then has the same effect as compiling
the main file with an |\includeonly| command
to select the appropriate child.
Moreover the generated document will carry the name of the child
rather than the main file.
This resolves all three above issues.

This feature is meant to make the editing of books,
thesis documents and lecture notes somewhat more convenient.
However, the package can also be used efficiently for
composing a series of documents (such as exercise sheets)
which are typically distributed individually.
It then assists the author in generating the individual documents
(potentially in different versions)
as well as a document containing the collected series.
Another application is in developing style files
or other kinds of included material
where compilation of the style file could redirect
to a sample or test file.

%%%%%%%%%%%%%%%%%%%%%%%%%%%%%%%%%%%%%%%%%%%%%%%%%%%%%%%%%%%%%%%%%%%%%%%%%%%%%%%%
%%%%%%%%%%%%%%%%%%%%%%%%%%%%%%%%%%%%%%%%%%%%%%%%%%%%%%%%%%%%%%%%%%%%%%%%%%%%%%%%
\section{Usage}

First of all, the package \textsf{childdoc} is \emph{not} a standard
\LaTeXe{} |.sty| style file! Therefore it needs to be invoked in
a non-standard way.

%%%%%%%%%%%%%%%%%%%%%%%%%%%%%%%%%%%%%%%%%%%%%%%%%%%%%%%%%%%%%%%%%%%%%%%%%%%%%%%%
\subsection{Included Files}
\label{sec:include}

%%%%%%%%%%%%%%%%%%%%%%%%%%%%%%%%%%%%%%%%
\DescribeMacro{\childdocmain}
To use the package, add the commands
\begin{center}
\begin{tabular}{l}
|\input{childdoc.def}|\\
|\childdocmain{}|\\
\end{tabular}
\end{center}
at the very top of the main \LaTeX{} file,
in particular \emph{before} the |\documentclass| statement!
The argument of |\childdocmain| should be left empty
(but it must be present).

%%%%%%%%%%%%%%%%%%%%%%%%%%%%%%%%%%%%%%%%
\DescribeMacro{\childdocof}
Furthermore, add the commands
\begin{center}
\begin{tabular}{l}
|\input{childdoc.def}|\\
|\childdocof{|\textit{main}|}|\\
\end{tabular}
\end{center}
at the top of every child file \textit{child}
which is included by |\include{|\textit{child}|}|
from within the main file
(or at least for those files to be compiled individually).
The argument \textit{main} must be the filename of the main file.

There are a couple of
considerations in setting up the main and child documents:

%%%%%%%%%%%%%%%%%%%%%%%%%%%%%%%%%%%%%%%%
\paragraph{Restrictions.}

Please note the following restrictions:
\begin{itemize}
\item
|\childdocmain| must be called with one argument \textit{main}
to ensure compatibility with earlier version of the package.
It must either be empty (|\childdocmain{}|)
or precisely match the filename of the main file in which it is specified.
See \secref{sec:detection} for further information.
\item
The filename \textit{main} must be specified without the |.tex| extension.
\item
The filename \textit{main} is case sensitive
(even in case-insensitive file systems)
due to internal string comparison.
\item
The argument \textit{main} should be fully expanded, it cannot be a macro.
\item
Subdirectories and special characters should be avoided in filenames.
\item
The command |\childdocmain{|\textit{main}|}| must be followed by a whitespace.
It should not be followed immediately by another command
or by a comment mark `|%|'.
This is because the \TeX{} parser reads the token immediately following
the argument of |\childdocmain| and puts it
at the beginning of every child section;
however, a white\-space is ignored.
\end{itemize}

%%%%%%%%%%%%%%%%%%%%%%%%%%%%%%%%%%%%%%%%
\paragraph{Content of Main File.}

It is advisable to place all content in the child files included by |\include|.
Any output contained in the main file will appear in all child documents
unless suppressed manually;
it cannot be suppressed automatically by the |\includeonly| directive
and thus should normally be avoided.
A method to include some content in the main file
by means of conditional processing is described in \secref{sec:conditional}.

%%%%%%%%%%%%%%%%%%%%%%%%%%%%%%%%%%%%%%%%
\paragraph{Page Numbering.}

When only a part of the document is compiled,
the appropriate numbering of pages
(as well as other status parameters)
is determined from the |.aux| files.
The latter contain information from previous passes.
However this information needs to propagate through
all intermediate child documents.
Therefore the page numbering in child documents may well
be inconsistent until the complete document is compiled at least once.

A useful (if unconventional) way to always ensure a consistent
page numbering is to restart the numbering in each child document
and denote the pages by `\textit{child}|.|\textit{page}'
where \textit{child} represents the chapter/section number of the child file.
This can be achieved by the command
|\numberwithin{page}{|\textit{child}|}|
of the \textsf{amsmath} package
where \textit{child} can be |chapter| or |section|
depending on the chosen structuring.
Alternatively, one can modify the macro |\thepage| appropriately
and reset the counter |page| at the start of each child file.

%%%%%%%%%%%%%%%%%%%%%%%%%%%%%%%%%%%%%%%%%%%%%%%%%%%%%%%%%%%%%%%%%%%%%%%%%%%%%%%%
\subsection{Conditional Processing}
\label{sec:conditional}

The package provides a mechanism to compile different versions
of a document. To customise the versions further some conditional processing
can come in handy to distinguish which version is being compiled.
The package provides two macros to describe the compilation context:

%%%%%%%%%%%%%%%%%%%%%%%%%%%%%%%%%%%%%%%%
\DescribeMacro{\ifchilddoc}
The conditional |\ifchilddoc| distinguishes between the compilation of
child documents and the main document:
%
\begin{center}
|\ifchilddoc |\textit{child-code}| |[|\||else |\textit{main-code}]| \||fi|
\end{center}

%%%%%%%%%%%%%%%%%%%%%%%%%%%%%%%%%%%%%%%%
\DescribeMacro{\childdocname}
\DescribeMacro{\childdocjob}
The macro |\childdocname| contains the filename (without extension)
of the main or child file being processed.
Note that |\childdocjob| will always contain the name of the main file.

%%%%%%%%%%%%%%%%%%%%%%%%%%%%%%%%%%%%%%%%
\paragraph{Title Page.}

Conditional processing can be used to include a title or banner page
in the main document when proper precautions are taken.
Importantly, the code in the main file should ensure that the page counter
(as well as other status parameters which are stored in the |.aux| files)
takes the same value after the conditional processing.
Otherwise the page numbers may take divergent values
depending on which part is compiled.

For example, a title page could be declared by:
%
\begin{center}
\begin{tabular}{l}
|\ifchilddoc\||else|\\
|\addtocounter{page}{-1}|\\
\textit{code for title page}\\
|\newpage|\\
|\||fi|
\end{tabular}
\end{center}
%
A banner page for the child documents can be generated by:
%
\begin{center}
\begin{tabular}{l}
|\ifchilddoc|\\
|\addtocounter{page}{-1}|\\
\textit{code for banner page}\\
|\newpage|\\
|\||fi|
\end{tabular}
\end{center}
%
Here one could write a message such as:
\begin{center}
|This is the part \childdocname{} of \childdocjob{}.|
\end{center}

%%%%%%%%%%%%%%%%%%%%%%%%%%%%%%%%%%%%%%%%%%%%%%%%%%%%%%%%%%%%%%%%%%%%%%%%%%%%%%%%
\subsection{Flags}
\label{sec:flags}

The package makes it easy to generate different versions
of the main or child documents.
To this end compilation flags can be defined
and assigned different default values.
They will be particularly useful in conjunction
with the forwarding mechanism described in \secref{sec:forward}.

For example, it may be useful to have a flag |\version|
which can be set to |draft| or |final|.
The document source will contain some conditional code
depending on the value of |\version|.
Suppose further, the flag should default to |final| for the main file
and to |draft| for child files
which is a natural assignment for editing the document.
This is achieved by placing the following code
in the preamble of the main document
(below the |\childdocmain| directive):
%
\begin{center}
\begin{tabular}{l}
|\ifchilddoc|\\
|\providecommand{\version}{draft}|\\
|\||else|\\
|\providecommand{\version}{final}|\\
|\||fi|
\end{tabular}
\end{center}
%
The definition by |\providecommand| makes sure
that previous definitions are not overwritten.
Further statements |\providecommand{\version}{...}|
can thus be added before the above code to override it.

For the main file, one might add a line
(between |\childdocmain| and the above block)
%
\begin{center}
|%\ifchilddoc\||else\providecommand{\version}{draft}\||fi|
\end{center}
%
which can be uncommented to produce a draft version.
Likewise one can add a line to the very top of a child file
(above the |\childdocof{|\textit{main}|}| directive)
%
\begin{center}
|%\providecommand{\version}{final}|
\end{center}
%
which can be uncommented to produce the final version of this child document.

%%%%%%%%%%%%%%%%%%%%%%%%%%%%%%%%%%%%%%%%%%%%%%%%%%%%%%%%%%%%%%%%%%%%%%%%%%%%%%%%
\subsection{Forwarding}
\label{sec:forward}

Different versions of the main or child documents
using compilation flags as described in \secref{sec:flags}
can be (permanently) stored in different files
for convenient compilation, viewing and distribution.
To this end, the package defines a command
to pass on compilation to a different file:

%%%%%%%%%%%%%%%%%%%%%%%%%%%%%%%%%%%%%%%%
\DescribeMacro{\childdocforward}
The command |\childdocforward| redirects processing to
another source file:
%
\begin{center}
\begin{tabular}{l}
|\input{childdoc.def}|\\
|\childdocforward[|\textit{main}|]{|\textit{dest}|}|\\
\end{tabular}
\end{center}
%
The argument \textit{dest} is the destination file
(without extension).
It should be the main file or one of the child files.
Note that further \textsf{childdoc} directives
such as |\childdocof| and |\childdocforward|
in the indicated file will be processed in this form.
The optional argument \textit{main}
passes on directly to the main file \textit{main}
while pretending to compile the child \textit{dest}.
This form behaves as if \textit{dest}
issues |\childdocof{|\textit{main}|}| right away,
and no further \textsf{childdoc} directives will be processed.

%%%%%%%%%%%%%%%%%%%%%%%%%%%%%%%%%%%%%%%%
\DescribeMacro{\...prefix}
In the alternative form |\childdocforwardprefix|,
%
\begin{center}
\begin{tabular}{l}
|\input{childdoc.def}|\\
|\childdocforwardprefix[|\textit{main}|]{|\textit{prefix}|}{|\textit{dest}|}|
\end{tabular}
\end{center}
%
the destination file is determined by a pattern
depending on the current file:
To make this work, the current file must be called
`{\textit{prefix}\hspace{0.2em}\textit{suffix}}'
with \textit{prefix} matching precisely the argument.
Processing is then passed on to the file
`{\textit{dest}\hspace{0.2em}\textit{suffix}}'.
Surely, the same effect is achieved by
directly specifying the
argument `{\textit{dest}\hspace{0.2em}\textit{suffix}}'
in the first form.
However, that requires to set up a different file
for each child. With the alternative form of the command
all these files can have exactly the same content
which simplifies setting them up and maintaining them.

For example, the following file |draft.tex|
with a compilation flag |\version| as described in \secref{sec:flags}
compiles the main document as a draft:
%
\begin{center}
\begin{tabular}{l}
|\def\version{draft}|\\
|\input{childdoc.def}|\\
|\childdocforward{|\textit{main}|}|
\end{tabular}
\end{center}
%
Likewise, the following files |final|\textit{nn}|.tex|
compile the final version of the child document
|child|\textit{nn}|.tex|:
%
\begin{center}
\begin{tabular}{l}
|\def\version{final}|\\
|\input{childdoc.def}|\\
|\childdocforwardprefix{final}{child}|
\end{tabular}
\end{center}
%

Note that when several versions of a main file and/or of each child file
are to be generated, it may be convenient to set up a |Makefile| or
shell script to automatise the process.

%%%%%%%%%%%%%%%%%%%%%%%%%%%%%%%%%%%%%%%%%%%%%%%%%%%%%%%%%%%%%%%%%%%%%%%%%%%%%%%%
\subsection{Command Line Processing}
\label{sec:commandline}

The effect of redirection files can also be achieved by invoking
the \LaTeX{} compiler with a more elaborate command line.
Most conveniently this should be done as part
of a shell script or a |Makefile|.

When using \textsf{childdoc} in the main file, the following
command lines effectively perform a redirection
(note that depending on the shell being used,
backslashes may have to be doubled: `|\|' $\to$ `|\\|'):
%
\begin{center}
|... -jobname "|\textit{target}|" |\\|"|[\textit{flags}]%
|\input{childdoc.def}\childdocforward[|\textit{main}|]{|\textit{dest}|}"|
\end{center}
%
Here \textit{target} is the name of the output file,
\textit{main} is the name of the main file
and \textit{dest} is the name of the main or child file to be processed
(all filenames without extensions).
The optional argument \textit{main} can be omitted
if \textit{main} matches \textit{dest}.
Optionally, compilation \textit{flags} can be defined via |\def| commands.
This command line makes the \TeX{} engine believe
it is compiling the file \textit{target}
whose content is specified as the latter parameter.
The provided code then forwards the processing to
\textit{main} or \textit{dest} as described in \secref{sec:forward}.

%%%%%%%%%%%%%%%%%%%%%%%%%%%%%%%%%%%%%%%%%%%%%%%%%%%%%%%%%%%%%%%%%%%%%%%%%%%%%%%%
\subsection{Include by Input}
\label{sec:input}

Including child documents by |\include| has some restrictions by design.
Most notably, the content of a child document always occupies
its own set of pages; pages cannot be shared between child documents.
Usually, this behaviour makes perfect sense
because each child document contain an essential part of the document.
However, in some situations it may be desirable to compose
a document from a collection of parts
without having mandatory page breaks between then.
For this case, the package
provides a mechanism to include parts
by |\input| which can also be processed individually.
However, by construction this mechanism
requires manual handling of the content to be output.

%%%%%%%%%%%%%%%%%%%%%%%%%%%%%%%%%%%%%%%%
\DescribeMacro{\ifchilddocmanual}
The main file should be prepared as usual, see \secref{sec:include}.
However, the document body must make a distinction
between processing of an individual part and of the main document, e.g.:
%
\begin{center}
\begin{tabular}{l}
|\ifchilddocmanual|\\
|\input{\childdocname}|\\
|\||else|\\
\textit{document body with }|\input{|\textit{part}|}|\\
|\||fi|
\end{tabular}
\end{center}
%
The conditional |\ifchilddocmanual| is true whenever
a part to be included by |\input| is being compiled,
and the name of the part is stored in |\childdocname|.

%%%%%%%%%%%%%%%%%%%%%%%%%%%%%%%%%%%%%%%%
\DescribeMacro{\childdocby}
Each part to be included by |\input| should start with:
%
\begin{center}
\begin{tabular}{l}
|\input{childdoc.def}|\\
|\childdocby{|\textit{main}|}|\\
\end{tabular}
\end{center}
%
The directive |\childdocby| is similar to |\childdocof|
described in \secref{sec:include},
but the subsequent selection of content must be done manually.
To that end, both |\ifchilddoc| and |\ifchilddocmanual|
will be true upon processing of a part,
and the name of the part is stored in |\childdocname|.
Note that |\jobname| will be set to the filename of the current part
so that each part receives an individual |.aux| file
that does not interfere with the |.aux| file(s) of the main document.
This behaviour can be altered by the alternative form
|\childdocby[*]{|\textit{main}|}| (with a non-empty optional argument)
which uses the |.aux| file of the main document
by setting |\jobname| to \textit{main}.

%%%%%%%%%%%%%%%%%%%%%%%%%%%%%%%%%%%%%%%%%%%%%%%%%%%%%%%%%%%%%%%%%%%%%%%%%%%%%%%%
\subsection{Driver Development}
\label{sec:driver}

The \textsf{childdoc} mechanism can also be use for the development
of definition files such as \LaTeX{} styles or classes.
This case differs from the above setup with multiple parts
included by |\include| in that no |\includeonly| should be invoked.
This can be achieved by starting the include file
(before |\ProvidesPackage|) with:
%
\begin{center}
\begin{tabular}{l}
|\input{childdoc.def}|\\
|\childdocforward{|\textit{main}|}|\\
\end{tabular}
\end{center}
%
or alternatively with:
%
\begin{center}
\begin{tabular}{l}
|\input{childdoc.def}|\\
|\childdocby{|\textit{main}|}|\\
\end{tabular}
\end{center}
%
Both forms have slightly different effects as described above.
The main file is prepared as usual, see \secref{sec:include}.

%%%%%%%%%%%%%%%%%%%%%%%%%%%%%%%%%%%%%%%%%%%%%%%%%%%%%%%%%%%%%%%%%%%%%%%%%%%%%%%%
\subsection{Legacy Detection}
\label{sec:detection}

The directive |\childdocmain| in the main file can detect
whether the complete document or merely a child is to be compiled
even without using the directive |\childdocof|.
This method is deprecated because it is less robust
and there is no compelling reason to use it;
it is merely provided for backward compatibility
and it may be removed in future versions.

If the detection mechanism is to be used,
it is mandatory to correctly specify
the filename of the main file as the argument of |\childdocmain|:
%
\begin{center}
\begin{tabular}{l}
|\input{childdoc.def}|\\
|\childdocmain{|\textit{main}|}|\\
\end{tabular}
\end{center}
%
If |\jobname| does not match the argument \textit{main} of |\childdocmain|,
it is assumed that |\jobname| points to the child file to be compiled.
When using |\childdocmain| with the main file specified as argument,
it suffices to start a child file
with just |\input{|\textit{main}|}|
without loading of the package and using |\childdocof|.
If instead all processing is done
with the appropriate \textsf{childdoc} directives,
the argument of \textit{main} of |\childdocmain| can be empty.

An alternative version of the command line processing described
in \secref{sec:commandline} using the detection mechanism reads:
%
\begin{center}
|... -jobname "|\textit{target}|" "|[\textit{flags}]%
[|\def\jobname{|\textit{dest}|}|]|\input{|\textit{main}|}"|
\end{center}

%%%%%%%%%%%%%%%%%%%%%%%%%%%%%%%%%%%%%%%%%%%%%%%%%%%%%%%%%%%%%%%%%%%%%%%%%%%%%%%%
\subsection{Manual Code}
\label{sec:manual}

In case one cannot be certain whether the definitions file |childdoc.def|
is installed on the target \TeX{} distribution
and one prefers not to ship it,
it is conceivable to paste a few relevant commands into the sources.

To that end, drop all statements |\input{childdoc.def}|
and perform the replacements as outlined below.
Instead of |\childdocmain{|\textit{main}|}| add the following code
to the top of the main file:
%
\begin{center}
\begin{tabular}{l}
|\||ifdefined\childdocname\endinput\||fi\newif\ifchilddoc|\\
|\edef\childdocname{\scantokens\expandafter{\jobname\noexpand}}|\\
|\def\childdocmain{|\textit{main}|}\||ifx\childdocmain\childdocname\||else|\\
|\childdoctrue\includeonly{\childdocname}\let\jobname\childdocmain\||fi|\\
\end{tabular}
\end{center}
%
Instead of |\childdocof{|\textit{main}|}| just include the main file
at the top of each child file:
%
\begin{center}
|\input{|\textit{main}|}|
\end{center}
%
A simple redirection |\childdocforward{|\textit{dest}|}| is achieved by:
%
\begin{center}
|\def\jobname{|\textit{dest}|}\input{\jobname}|
\end{center}
%
The redirection with prefix
|\childdocforwardprefix[|\textit{prefix}|]{|\textit{dest}|}|
is accomplished by:
%
\begin{center}
\begin{tabular}{l}
|{\edef\jobname{\scantokens\expandafter{\jobname\noexpand}}|\\
|\def\redirectjob |\textit{prefix}|#1~~~{\gdef\jobname{|\textit{dest}|#1}}|\\
|\expandafter\redirectjob\jobname~~~}\input{\jobname}|
\end{tabular}
\end{center}

In an alternative approach,
child documents can be compiled by a specific command line
without additional code or specific definitions:
%
\begin{center}
|... -jobname "|\textit{target}|" "|[\textit{flags}]%
|\includeonly{|\textit{dest}|}\input{|\textit{main}|}"|
\end{center}
%

%%%%%%%%%%%%%%%%%%%%%%%%%%%%%%%%%%%%%%%%%%%%%%%%%%%%%%%%%%%%%%%%%%%%%%%%%%%%%%%%
%%%%%%%%%%%%%%%%%%%%%%%%%%%%%%%%%%%%%%%%%%%%%%%%%%%%%%%%%%%%%%%%%%%%%%%%%%%%%%%%
\section{Information}

%%%%%%%%%%%%%%%%%%%%%%%%%%%%%%%%%%%%%%%%%%%%%%%%%%%%%%%%%%%%%%%%%%%%%%%%%%%%%%%%
\subsection{Copyright}

Copyright \copyright{} 2017--2018 Niklas Beisert

This work may be distributed and/or modified under the
conditions of the \LaTeX{} Project Public License, either version 1.3
of this license or (at your option) any later version.
The latest version of this license is in
  \url{http://www.latex-project.org/lppl.txt}
and version 1.3 or later is part of all distributions of \LaTeX{}
version 2005/12/01 or later.

This work has the LPPL maintenance status `maintained'.

The Current Maintainer of this work is Niklas Beisert.

This work consists of the files |README.txt|, |childdoc.ins| and |childdoc.dtx|
as well as the derived files |childdoc.def|, |cdocsamp.tex|
with |cdocsch1.tex|, |cdocsch2.tex|, |cdocspt3.tex|, |cdocspt4.tex|,
|cdocsdrf.tex|, |cdocsfn1.tex|, |cdocsfn2.tex|
as well as |childdoc.pdf|.

%%%%%%%%%%%%%%%%%%%%%%%%%%%%%%%%%%%%%%%%%%%%%%%%%%%%%%%%%%%%%%%%%%%%%%%%%%%%%%%%
\subsection{Files and Installation}

The package consists of the files:
%
\begin{center}
\begin{tabular}{ll}
    |README.txt|   & readme file \\
    |childdoc.ins| & installation file \\
    |childdoc.dtx| & source file \\
    |childdoc.def| & definition file \\
    |cdocsamp.tex| & sample main file \\
    |cdocsch1.tex| & sample include file \\
    |cdocsch2.tex| & sample include file \\
    |cdocspt3.tex| & sample part file \\
    |cdocspt4.tex| & sample part file \\
    |cdocsdrf.tex| & sample redirection file \\
    |cdocsfn1.tex| & sample redirection file \\
    |cdocsfn2.tex| & sample redirection file \\
    |childdoc.pdf| & manual
\end{tabular}
\end{center}
%
The distribution consists of the files
|README.txt|, |childdoc.ins| and |childdoc.dtx|.
%
\begin{itemize}
\item
Run (pdf)\LaTeX{} on |childdoc.dtx|
to compile the manual |childdoc.pdf| (this file).
\item
Run \LaTeX{} on |childdoc.ins| to create the definitions file |childdoc.def|
and the sample |cdocsamp.tex| with include files
|cdocsch1.tex|, |cdocsch2.tex|, |cdocspt3.tex|, |cdocspt4.tex|,
|cdocsdrf.tex|, |cdocsfn1.tex|, |cdocsfn2.tex|.
Then copy the file |childdoc.def| to an appropriate directory of your \LaTeX{}
distribution, e.g.\ \textit{texmf-root}|/tex/latex/childdoc|.
\end{itemize}

%%%%%%%%%%%%%%%%%%%%%%%%%%%%%%%%%%%%%%%%%%%%%%%%%%%%%%%%%%%%%%%%%%%%%%%%%%%%%%%%
\subsection{Related CTAN Packages}

There are several other packages which offer a similar functionality:
%
\begin{itemize}
\item
The packages
\href{http://ctan.org/pkg/docmute}{\textsf{docmute}},
\href{http://ctan.org/pkg/includex}{\textsf{includex}} and
\href{http://ctan.org/pkg/standalone}{\textsf{standalone}}
provide commands to include only the document body of
a child file thus allowing both files to be compiled individually.
\item
The packages \href{http://ctan.org/pkg/subdocs}{\textsf{subdocs}}
and \href{http://ctan.org/pkg/subfiles}{\textsf{subfiles}}
provide structures in which the main and child documents can be
encapsulated and allowing them to be compiled individually.
The inclusion mechanism is different from the conventional |\include|.
\item
The package \href{http://ctan.org/pkg/combine}{\textsf{combine}}
is an elaborate solution to combine several documents into one.
\end{itemize}
%
See also the CTAN topic \href{http://ctan.org/topic/subdocs}{\textsf{subdocs}}
for further related packages.
The present package differs from the above solutions in that
a document structure constructed with the conventional |\include| mechanism
just needs two extra commands at the top of every file
such that all constituent files can be compiled individually.

%%%%%%%%%%%%%%%%%%%%%%%%%%%%%%%%%%%%%%%%%%%%%%%%%%%%%%%%%%%%%%%%%%%%%%%%%%%%%%%%
%\subsection{Feature Suggestions}
%
%The following is a list of features which may be useful for future
%versions of this package:
%%
%\begin{itemize}
%\item
%\ldots
%\end{itemize}

%%%%%%%%%%%%%%%%%%%%%%%%%%%%%%%%%%%%%%%%%%%%%%%%%%%%%%%%%%%%%%%%%%%%%%%%%%%%%%%%
\subsection{Revision History}

%%%%%%%%%%%%%%%%%%%%%%%%%%%%%%%%%%%%%%%%
\paragraph{v2.0:} 2018/12/30

\begin{itemize}
\item
immediate forward processing
\item
added |\childdocby| mechanism
\item
manual restructured
\end{itemize}

%%%%%%%%%%%%%%%%%%%%%%%%%%%%%%%%%%%%%%%%
\paragraph{v1.6:} 2018/01/17

\begin{itemize}
\item
application for development of include files
\item
corrections to manual
\end{itemize}

%%%%%%%%%%%%%%%%%%%%%%%%%%%%%%%%%%%%%%%%
\paragraph{v1.5:} 2017/05/21

\begin{itemize}
\item
more complete structuring introduced
\item
|\childdocof| introduced
\item
|\childdoc| renamed to |\childdocmain|
\item
|\childredirect| renamed to |\childdocforward| and |\childdocforwardprefix|
and functionality expanded
\end{itemize}

%%%%%%%%%%%%%%%%%%%%%%%%%%%%%%%%%%%%%%%%
\paragraph{v1.0:} 2017/04/27

\begin{itemize}
\item
manual and install package
\item
first version published on CTAN
\end{itemize}

%%%%%%%%%%%%%%%%%%%%%%%%%%%%%%%%%%%%%%%%
\paragraph{v0.6:} 2017/04/26

\begin{itemize}
\item
redirection mechanism added
\end{itemize}

%%%%%%%%%%%%%%%%%%%%%%%%%%%%%%%%%%%%%%%%
\paragraph{v0.5:} 2017/04/26

\begin{itemize}
\item
functionality in definition file
\end{itemize}


%%%%%%%%%%%%%%%%%%%%%%%%%%%%%%%%%%%%%%%%%%%%%%%%%%%%%%%%%%%%%%%%%%%%%%%%%%%%%%%%
%%%%%%%%%%%%%%%%%%%%%%%%%%%%%%%%%%%%%%%%%%%%%%%%%%%%%%%%%%%%%%%%%%%%%%%%%%%%%%%%
%%%%%%%%%%%%%%%%%%%%%%%%%%%%%%%%%%%%%%%%%%%%%%%%%%%%%%%%%%%%%%%%%%%%%%%%%%%%%%%%
\appendix

\settowidth\MacroIndent{\rmfamily\scriptsize 000\ }

 \DocInput{childdoc.dtx}

\end{document}
%</driver>
% \fi
%
% %%%%%%%%%%%%%%%%%%%%%%%%%%%%%%%%%%%%%%%%%%%%%%%%%%%%%%%%%%%%%%%%%%%%%%%%%%%%%%
% %%%%%%%%%%%%%%%%%%%%%%%%%%%%%%%%%%%%%%%%%%%%%%%%%%%%%%%%%%%%%%%%%%%%%%%%%%%%%%
% \section{Sample}
%\iffalse
%<*samplemain>
%\fi
%
% The following presents a sample document
% with two chapters, two parts, a title page,
% a compile flag as well as three forwarding files to set the flag.
% It consists of eight |.tex| files:
% \begin{center}
% \begin{tabular}{ll}
% |cdocsamp.tex|&main file\\
% |cdocsch1.tex|&include file for chapter 1\\
% |cdocsch2.tex|&include file for chapter 2\\
% |cdocspt3.tex|&include file for part 3\\
% |cdocspt4.tex|&include file for part 4\\
% |cdocsdrf.tex|&forwarding file for main file in draft mode\\
% |cdocsfi1.tex|&forwarding file for final version of chapter 1\\
% |cdocsfi2.tex|&forwarding file for final version of chapter 2\\
% \end{tabular}
% \end{center}
% Each of the eight files can be compiled directly by the \LaTeX{} compiler.
%
% %%%%%%%%%%%%%%%%%%%%%%%%%%%%%%%%%%%%%%
% \paragraph{Main File.}
%
% The main file is called |cdocsamp.tex|.
%
% Load the \textsf{childdoc} definitions and
% declare the filename for the main document:
%    \begin{macrocode}
\input{childdoc.def}
\childdocmain{}
%    \end{macrocode}

% Optional override for |\version| flag:
%    \begin{macrocode}
%%\ifchilddoc\else\providecommand{\version}{draft}\fi
%    \end{macrocode}

% Define the default values for the |\version| flag
% (|final| for the main file and |draft| for childs):
%    \begin{macrocode}
\ifchilddoc
\providecommand{\version}{draft}
\else
\providecommand{\version}{final}
\fi
%    \end{macrocode}

% Load the standard document class:
%    \begin{macrocode}
\documentclass[12pt]{article}
%    \end{macrocode}

% Start the document body:
%    \begin{macrocode}
\begin{document}
%    \end{macrocode}

% Declare a title page.
% Print title, part of document being processed and version flag:
%    \begin{macrocode}
\addtocounter{page}{-1}
\begin{center}
{\LARGE\bfseries{}childdoc example\par}
\vspace{1cm}
\ifchilddoc
\ifchilddocmanual part\else chapter\fi:
`\childdocname' of `\childdocjob'\par
\else
main document: `\childdocjob'\par
\fi
version: \version\par
\end{center}
\newpage
%    \end{macrocode}

% Manually include selected file,
% otherwise process as usual:
%    \begin{macrocode}
\ifchilddocmanual
\section*{part `\childdocname'}
\input{\childdocname}
\else
%    \end{macrocode}

% Include the two chapters:
%    \begin{macrocode}
\include{cdocsch1}
\include{cdocsch2}
%    \end{macrocode}

% Include the two parts unless only chapters should be displayed:
%    \begin{macrocode}
\ifchilddoc\else
\section{part three}
\input{cdocspt3}
\section{part four}
\input{cdocspt4}
\fi
%    \end{macrocode}

% Process as usual until here:
%    \begin{macrocode}
\fi
%    \end{macrocode}

% End of document body:
%    \begin{macrocode}
\end{document}
%    \end{macrocode}
%\iffalse
%</samplemain>
%\fi
%
% %%%%%%%%%%%%%%%%%%%%%%%%%%%%%%%%%%%%%%
% \paragraph{Chapter Include Files.}
%
% The include files are called |cdocsch1.tex| and |cdocsch2.tex|.
%
%\iffalse
%<*samplechap1|samplechap2>
%\fi

% Optional override for |\version| flag:
%    \begin{macrocode}
%%\providecommand{\version}{final}
%    \end{macrocode}

% Include the main document:
%    \begin{macrocode}
\input{childdoc.def}
\childdocof{cdocsamp}
%    \end{macrocode}

%\iffalse
%</samplechap1|samplechap2>
%\fi
%
%\iffalse
%<*samplechap1>
%\fi
% Some text for chapter 1:
%    \begin{macrocode}
\section{one}
some text in chapter one
%    \end{macrocode}

%\iffalse
%</samplechap1>
%\fi
% Some text for chapter 2:
%\iffalse
%<*samplechap2>
%\fi
%    \begin{macrocode}
\section{two}
more text in chapter two
%    \end{macrocode}

%\iffalse
%</samplechap2>
%\fi
%
% %%%%%%%%%%%%%%%%%%%%%%%%%%%%%%%%%%%%%%
% \paragraph{Part Include Files.}
%
% The include files are called |cdocspt3.tex| and |cdocspt4.tex|.
%
%\iffalse
%<*samplepart3|samplepart4>
%\fi

% Optional override for |\version| flag:
%    \begin{macrocode}
%%\providecommand{\version}{final}
%    \end{macrocode}

% Include the main document:
%    \begin{macrocode}
\input{childdoc.def}
\childdocby{cdocsamp}
%    \end{macrocode}

%\iffalse
%</samplepart3|samplepart4>
%\fi
%
%\iffalse
%<*samplepart3>
%\fi
% Some text for part 3:
%    \begin{macrocode}
some text in part three
%    \end{macrocode}

%\iffalse
%</samplepart3>
%\fi
% Some text for part 4:
%\iffalse
%<*samplepart4>
%\fi
%    \begin{macrocode}
more text in part four
%    \end{macrocode}

%\iffalse
%</samplepart4>
%\fi
%
% %%%%%%%%%%%%%%%%%%%%%%%%%%%%%%%%%%%%%%
% \paragraph{Forwarding for a Complete Draft.}
%
% The following forwarding file |cdocsdrf.tex|
% compiles the main document in draft mode:
%\iffalse
%<*sampledraft>
%\fi
%    \begin{macrocode}
\def\version{draft}
\input{childdoc.def}
\childdocforward{cdocsamp}
%    \end{macrocode}

%\iffalse
%</sampledraft>
%\fi
%
% %%%%%%%%%%%%%%%%%%%%%%%%%%%%%%%%%%%%%%
% \paragraph{Forwarding for Final Version of the Chapters.}
%
% The following forwarding files |cdocsfn1.tex| and |cdocsfn2.tex|
% (with identical content)
% compile the final versions of the child documents
% |cdocsch1.tex| and |cdocsch2.tex|, respectively:
%\iffalse
%<*samplefinal>
%\fi
%    \begin{macrocode}
\def\version{final}
\input{childdoc.def}
\childdocforwardprefix[cdocsamp]{cdocsfn}{cdocsch}
%    \end{macrocode}

%\iffalse
%</samplefinal>
%\fi
%
% %%%%%%%%%%%%%%%%%%%%%%%%%%%%%%%%%%%%%%
% \paragraph{Command Line Processing.}
%
% The following three command lines generate the output files
% |cdocscld|, |cdocscl1| and |cdocscl2|
% which should be identical to
% |cdocsdrf|, |cdocsch1| and |cdocsfn2|, respectively:
% \begin{center}
% \begin{tabular}{l}
% |latex -jobname cdocscld \|\\
% |  "\def\version{draft}\input{childdoc.def}\childdocforward{cdocsamp}"|\\
% |latex -jobname cdocscl1 \|\\
% |  "\input{childdoc.def}\childdocforward[cdocsamp]{cdocsch1}"|\\
% |latex -jobname cdocscl2 \|\\
% |  "\def\version{final}\input{childdoc.def}\childdocforward{cdocsch2}"|
% \end{tabular}
% \end{center}
% Note that the trailing backslash on each first line
% merely continues the input to the second line
% (for convenient cut ant paste).
% Furthermore, the command |latex| can be replaced by any
% of its alternative versions such as |pdflatex|.
%
% %%%%%%%%%%%%%%%%%%%%%%%%%%%%%%%%%%%%%%%%%%%%%%%%%%%%%%%%%%%%%%%%%%%%%%%%%%%%%%
% %%%%%%%%%%%%%%%%%%%%%%%%%%%%%%%%%%%%%%%%%%%%%%%%%%%%%%%%%%%%%%%%%%%%%%%%%%%%%%
% \section{Implementation}
%\iffalse
%<*package>
%\fi
%
% This section describes the definitions file |childdoc.def|.

% The definitions cannot be loaded using |\usepackage| or |\RequirePackage|
% which has a mechanism to prevent loading a style file more than once.
% When loading the definitions by means of |\input|
% multiple instances have to be prevented manually:
%\iffalse
%This code needs to be before the `\ProvidesFile' directive
%which is defined at the beginning of this file.
%Therefore it is also placed there and commented out here.
%</package>
%<*discard>
%\fi
%    \begin{macrocode}
\ifdefined\childdocmain\endinput\fi
%    \end{macrocode}
%\iffalse
%</discard>
%<*package>
%\fi
%
% \macro{\ifchilddoc}
% \macro{\ifchilddocmanual}
% The conditional |\ifchilddoc| tells whether a
% child (true) or main (false) document is being compiled.
% The conditional |\ifchilddocmanual| tells whether
% the |\includeonly| mechanism is used (false) or
% the selection of child files must be performed manually (true).
% The definitions initialise to false:
%    \begin{macrocode}
\newif\ifchilddoc
\newif\ifchilddocmanual
%    \end{macrocode}

% \macro{\childdocname}
% \macro{\childdocjob}
% The macro |\childdocname| stores the name of the main document
% to be compiled. The macro |\childdocjob| stores the name of
% the document on which the \LaTeX{} compiler was originally invoked.
% The content of |\jobname| cannot be compared
% to filenames specified in the source due to different catcodes.
% The following code rescans |\jobname|, stores the result
% in |\childdocname| and saves a copy in |\childdocjob|:
%    \begin{macrocode}
\edef\childdocname{\scantokens\expandafter{\jobname\noexpand}}
\let\childdocjob\childdocname
%    \end{macrocode}

% \macro{\childdocdisable}
% The macro |\childdocdisable| prevents the main file
% from being processed more than once.
% At this stage, the main document command |\childdocmain|
% is assumed to be called once again where it should do nothing.
% Any subsequent call to it should prevent
% a secondary processing of the main document
% It overwrites the forwarding commands
% |\childdocof| and |\childdocforward|
% with empty macros to prevent further inclusions of the main document:
%    \begin{macrocode}
\newcommand{\childdocdisable}
{
  \renewcommand{\childdocmain}[1]{\renewcommand{\childdocmain}[1]{\endinput}}
  \renewcommand{\childdocof}[1]{}
  \renewcommand{\childdocby}[2][]{}
  \renewcommand{\childdocforward}[2][]{}
  \renewcommand{\childdocdisable}{}
}
%    \end{macrocode}

% \macro{\childdocmain}
% The macro |\childdocmain| is to be called at the top of the main file
% with nothing or the main filename (without extension) as argument.
% First, it breaks loops.
% If the argument is not empty and does not match |\childdocname|
% (which is set by the first inclusion of |childdoc.def|),
% |\ifchilddoc| is set to true, |\includeonly| is applied to the child file
% and |\jobname| is set to the main file
% (for proper handling of |.aux| files):
%    \begin{macrocode}
\newcommand{\childdocmain}[1]
{
  \childdocdisable\childdocmain{}
  \if?#1?\else
    \begingroup
      \def\childdoctmp{#1}
      \ifx\childdoctmp\childdocname
        \def\childdoctmp{}
      \else
        \def\childdoctmp
        {
          \childdoctrue
          \includeonly{\childdocname}
          \def\childdocjob{#1}
          \def\jobname{#1}
        }
      \fi
      \expandafter
    \endgroup
    \childdoctmp
  \fi
}
%    \end{macrocode}

% \macro{\childdocof}
% The command |\childdocof| redirects
% compilation to the main file |#1|.
%    \begin{macrocode}
\newcommand{\childdocof}[1]
{
  \childdocdisable
  \childdoctrue
  \includeonly{\childdocname}
  \def\jobname{#1}
  \def\childdocjob{#1}
  \input{#1}
}
%    \end{macrocode}

% \macro{\childdocby}
% The command |\childdocby| ....
%    \begin{macrocode}
\newcommand{\childdocby}[2][]
{
  \childdocdisable
  \childdoctrue
  \childdocmanualtrue
  \if?#1?\else
    \def\jobname{#2}
  \fi
  \def\childdocjob{#2}
  \input{#2}
  \endinput
}
%    \end{macrocode}

% \macro{\childdocforward}
% The command |\childdocforward| redirects
% compilation to the main file or
% (if the optional argument is given) a child file.
% Parameters are set as if the main file
% or a child file starting with |\childdocof| was compiled.
% Then compilation is handed over to the main file:
%    \begin{macrocode}
\newcommand{\childdocforward}[2][]
{
  \begingroup
    \if?#1?
      \def\childdoctmp
      {
        \def\childdocname{#2}
        \def\childdocjob{#2}
        \def\jobname{#2}
        \input{#2}
        \endinput
      }
    \else
      \def\childdoctmp
      {
        \childdocdisable
        \def\childdocname{#2}
        \childdoctrue
        \includeonly{#2}
        \def\childdocjob{#1}
        \def\jobname{#1}
        \input{#1}
        \endinput
      }
    \fi
    \expandafter
  \endgroup
  \childdoctmp
}
%    \end{macrocode}

% \macro{\childdocforwardprefix}
% The command |\childdocforwardprefix| redirects
% compilation to the main or a child file by means of a pattern.
% The prefix |#1| in the current filename is replaced by |#2|
% and the suffix of the current filename is kept
% (it is assumed that the filename does not contain the substring `|~~~|'
% which is used as a delimiter).
% Compilation is handed over to the new file by |\childdocforward|:
%    \begin{macrocode}
\newcommand{\childdocforwardprefix}[3][]
{
  \begingroup
    \def\childdocextract #2##1~~~{\def\childdoctmp{\childdocforward[#1]{#3##1}}}
    \expandafter\childdocextract\childdocname~~~
    \expandafter
  \endgroup
  \childdoctmp
}
%    \end{macrocode}

% \macro{\childdoc}
% The deprecated macro |\childdoc| is a legacy version of |\childdocmain|:
%    \begin{macrocode}
\newcommand{\childdoc}{\childdocmain}
%    \end{macrocode}

% \macro{\childdocredirect}
% The deprecated macro |\childdocredirect| is a legacy version
% of |\childdocforward| and |\childdocforwardprefix|:
%    \begin{macrocode}
\newcommand{\childdocredirect}[2][]
{
  \begingroup
    \if?#1?
      \def\childdoctmp{\childdocforward{#2}}
    \else
      \def\childdoctmp{\childdocforwardprefix{#1}{#2}}
    \fi
    \expandafter
  \endgroup
  \childdoctmp
}
%    \end{macrocode}

%\iffalse
%</package>
%\fi
%
\endinput

\childdocmain{}
%    \end{macrocode}

% Optional override for |\version| flag:
%    \begin{macrocode}
%%\ifchilddoc\else\providecommand{\version}{draft}\fi
%    \end{macrocode}

% Define the default values for the |\version| flag
% (|final| for the main file and |draft| for childs):
%    \begin{macrocode}
\ifchilddoc
\providecommand{\version}{draft}
\else
\providecommand{\version}{final}
\fi
%    \end{macrocode}

% Load the standard document class:
%    \begin{macrocode}
\documentclass[12pt]{article}
%    \end{macrocode}

% Start the document body:
%    \begin{macrocode}
\begin{document}
%    \end{macrocode}

% Declare a title page.
% Print title, part of document being processed and version flag:
%    \begin{macrocode}
\addtocounter{page}{-1}
\begin{center}
{\LARGE\bfseries{}childdoc example\par}
\vspace{1cm}
\ifchilddoc
\ifchilddocmanual part\else chapter\fi:
`\childdocname' of `\childdocjob'\par
\else
main document: `\childdocjob'\par
\fi
version: \version\par
\end{center}
\newpage
%    \end{macrocode}

% Manually include selected file,
% otherwise process as usual:
%    \begin{macrocode}
\ifchilddocmanual
\section*{part `\childdocname'}
\input{\childdocname}
\else
%    \end{macrocode}

% Include the two chapters:
%    \begin{macrocode}
\include{cdocsch1}
\include{cdocsch2}
%    \end{macrocode}

% Include the two parts unless only chapters should be displayed:
%    \begin{macrocode}
\ifchilddoc\else
\section{part three}
\input{cdocspt3}
\section{part four}
\input{cdocspt4}
\fi
%    \end{macrocode}

% Process as usual until here:
%    \begin{macrocode}
\fi
%    \end{macrocode}

% End of document body:
%    \begin{macrocode}
\end{document}
%    \end{macrocode}
%\iffalse
%</samplemain>
%\fi
%
% %%%%%%%%%%%%%%%%%%%%%%%%%%%%%%%%%%%%%%
% \paragraph{Chapter Include Files.}
%
% The include files are called |cdocsch1.tex| and |cdocsch2.tex|.
%
%\iffalse
%<*samplechap1|samplechap2>
%\fi

% Optional override for |\version| flag:
%    \begin{macrocode}
%%\providecommand{\version}{final}
%    \end{macrocode}

% Include the main document:
%    \begin{macrocode}
% \iffalse
%
% childdoc.dtx Copyright (C) 2017-2018 Niklas Beisert
%
% This work may be distributed and/or modified under the
% conditions of the LaTeX Project Public License, either version 1.3
% of this license or (at your option) any later version.
% The latest version of this license is in
%   http://www.latex-project.org/lppl.txt
% and version 1.3 or later is part of all distributions of LaTeX
% version 2005/12/01 or later.
%
% This work has the LPPL maintenance status `maintained'.
%
% The Current Maintainer of this work is Niklas Beisert.
%
% This work consists of the files childdoc.dtx and childdoc.ins
% and the derived files childdoc.def and cdocsamp.tex with
% cdocsch1.tex, cdocsch2.tex, cdocsdrf.tex, cdocsfn1.tex, cdocsfn2.tex.
%
%<package>\ifdefined\childdocmain\endinput\fi
%<package>\ProvidesFile{childdoc.def}[2018/12/30 v2.0 child document driver]
%<samplemain>\ProvidesFile{cdocsamp.tex}[2018/12/30 v2.0 sample for childdoc]
%<*driver>
%\ProvidesFile{childdoc.drv}[2018/12/30 v2.0 childdoc reference manual file]
\PassOptionsToClass{10pt,a4paper}{article}
\documentclass{ltxdoc}

\usepackage[margin=35mm]{geometry}
\usepackage{hyperref}
\usepackage{hyperxmp}
\usepackage[usenames]{color}

\hypersetup{colorlinks=true}
\hypersetup{pdfstartview=FitH}
\hypersetup{pdfpagemode=UseNone}
\hypersetup{pdfsource={}}
\hypersetup{pdflang={en-UK}}
\hypersetup{pdfcopyright={Copyright 2017-2018 Niklas Beisert.
  This work may be distributed and/or modified under the
  conditions of the LaTeX Project Public License, either version 1.3
  of this license or (at your option) any later version.}}
\hypersetup{pdflicenseurl={http://www.latex-project.org/lppl.txt}}
\hypersetup{pdfcontactaddress={ETH Zurich, ITP, HIT K,
  Wolfgang-Pauli-Strasse 27}}
\hypersetup{pdfcontactpostcode={8093}}
\hypersetup{pdfcontactcity={Zurich}}
\hypersetup{pdfcontactcountry={Switzerland}}
\hypersetup{pdfcontactemail={nbeisert@itp.phys.ethz.ch}}
\hypersetup{pdfcontacturl={http://people.phys.ethz.ch/\xmptilde nbeisert/}}

\newcommand{\secref}[1]{\hyperref[#1]{section \ref*{#1}}}

\parskip1ex
\parindent0pt
\let\olditemize\itemize
\def\itemize{\olditemize\parskip0pt}

\begin{document}

\title{The \textsf{childdoc} Package}
\hypersetup{pdftitle={The childdoc Package}}
\author{Niklas Beisert\\[2ex]
  Institut f\"ur Theoretische Physik\\
  Eidgen\"ossische Technische Hochschule Z\"urich\\
  Wolfgang-Pauli-Strasse 27, 8093 Z\"urich, Switzerland\\[1ex]
  \href{mailto:nbeisert@itp.phys.ethz.ch}
  {\texttt{nbeisert@itp.phys.ethz.ch}}}
\hypersetup{pdfauthor={Niklas Beisert}}
\hypersetup{pdfsubject={Manual for the LaTeX2e Package childdoc}}
\date{30 December 2018, \textsf{v2.0}}
\maketitle

\begin{abstract}\noindent
\textsf{childdoc} is a \LaTeXe{} package
that enables the direct compilation
of document sections included by |\include|
to individual files.
\end{abstract}

\begingroup
\parskip0ex
\tableofcontents
\endgroup

%%%%%%%%%%%%%%%%%%%%%%%%%%%%%%%%%%%%%%%%%%%%%%%%%%%%%%%%%%%%%%%%%%%%%%%%%%%%%%%%
%%%%%%%%%%%%%%%%%%%%%%%%%%%%%%%%%%%%%%%%%%%%%%%%%%%%%%%%%%%%%%%%%%%%%%%%%%%%%%%%
\section{Introduction}

\LaTeX{} provides a mechanism to structure a large document (such as a book)
into a main file and several child files (containing the chapters)
using the |\include| command.
This mechanism is beneficial for documents
which span hundreds of pages in order to
make the source file(s) more manageable.
Moreover, compilation can be restricted to
selected child files by means of the |\includeonly| command.
The latter feature can be used to reduce the compilation time while editing
(this was significantly more useful in the earlier days of \LaTeX{})
or to generate a smaller document which is easier to navigate.
Another application of |\includeonly| is to generate
documents consisting of selected parts of the complete document.

However, there are a few drawbacks of the plain |\include| mechanism:
\begin{itemize}
\item
The child files cannot be compiled on their own,
they can only be compiled via the main file.
A naive editing environment
(such as a text editor with an option
to have the current file processed by \LaTeX)
may require one to switch to the main file before compiling;
attempting to compile the child file produces errors.
\item
The main file must be modified (each time)
to adjust the |\includeonly| command
to the present needs. This easily leaves the main file in a messy state.
\item
The generated document will always carry the filename
of the main document. This is inconvenient if
several child files are to be compiled and
to be kept for distribution.
\end{itemize}

The present package provides a simple interface
to make child files individually compilable by \LaTeX{}.
Compiling a child file then has the same effect as compiling
the main file with an |\includeonly| command
to select the appropriate child.
Moreover the generated document will carry the name of the child
rather than the main file.
This resolves all three above issues.

This feature is meant to make the editing of books,
thesis documents and lecture notes somewhat more convenient.
However, the package can also be used efficiently for
composing a series of documents (such as exercise sheets)
which are typically distributed individually.
It then assists the author in generating the individual documents
(potentially in different versions)
as well as a document containing the collected series.
Another application is in developing style files
or other kinds of included material
where compilation of the style file could redirect
to a sample or test file.

%%%%%%%%%%%%%%%%%%%%%%%%%%%%%%%%%%%%%%%%%%%%%%%%%%%%%%%%%%%%%%%%%%%%%%%%%%%%%%%%
%%%%%%%%%%%%%%%%%%%%%%%%%%%%%%%%%%%%%%%%%%%%%%%%%%%%%%%%%%%%%%%%%%%%%%%%%%%%%%%%
\section{Usage}

First of all, the package \textsf{childdoc} is \emph{not} a standard
\LaTeXe{} |.sty| style file! Therefore it needs to be invoked in
a non-standard way.

%%%%%%%%%%%%%%%%%%%%%%%%%%%%%%%%%%%%%%%%%%%%%%%%%%%%%%%%%%%%%%%%%%%%%%%%%%%%%%%%
\subsection{Included Files}
\label{sec:include}

%%%%%%%%%%%%%%%%%%%%%%%%%%%%%%%%%%%%%%%%
\DescribeMacro{\childdocmain}
To use the package, add the commands
\begin{center}
\begin{tabular}{l}
|\input{childdoc.def}|\\
|\childdocmain{}|\\
\end{tabular}
\end{center}
at the very top of the main \LaTeX{} file,
in particular \emph{before} the |\documentclass| statement!
The argument of |\childdocmain| should be left empty
(but it must be present).

%%%%%%%%%%%%%%%%%%%%%%%%%%%%%%%%%%%%%%%%
\DescribeMacro{\childdocof}
Furthermore, add the commands
\begin{center}
\begin{tabular}{l}
|\input{childdoc.def}|\\
|\childdocof{|\textit{main}|}|\\
\end{tabular}
\end{center}
at the top of every child file \textit{child}
which is included by |\include{|\textit{child}|}|
from within the main file
(or at least for those files to be compiled individually).
The argument \textit{main} must be the filename of the main file.

There are a couple of
considerations in setting up the main and child documents:

%%%%%%%%%%%%%%%%%%%%%%%%%%%%%%%%%%%%%%%%
\paragraph{Restrictions.}

Please note the following restrictions:
\begin{itemize}
\item
|\childdocmain| must be called with one argument \textit{main}
to ensure compatibility with earlier version of the package.
It must either be empty (|\childdocmain{}|)
or precisely match the filename of the main file in which it is specified.
See \secref{sec:detection} for further information.
\item
The filename \textit{main} must be specified without the |.tex| extension.
\item
The filename \textit{main} is case sensitive
(even in case-insensitive file systems)
due to internal string comparison.
\item
The argument \textit{main} should be fully expanded, it cannot be a macro.
\item
Subdirectories and special characters should be avoided in filenames.
\item
The command |\childdocmain{|\textit{main}|}| must be followed by a whitespace.
It should not be followed immediately by another command
or by a comment mark `|%|'.
This is because the \TeX{} parser reads the token immediately following
the argument of |\childdocmain| and puts it
at the beginning of every child section;
however, a white\-space is ignored.
\end{itemize}

%%%%%%%%%%%%%%%%%%%%%%%%%%%%%%%%%%%%%%%%
\paragraph{Content of Main File.}

It is advisable to place all content in the child files included by |\include|.
Any output contained in the main file will appear in all child documents
unless suppressed manually;
it cannot be suppressed automatically by the |\includeonly| directive
and thus should normally be avoided.
A method to include some content in the main file
by means of conditional processing is described in \secref{sec:conditional}.

%%%%%%%%%%%%%%%%%%%%%%%%%%%%%%%%%%%%%%%%
\paragraph{Page Numbering.}

When only a part of the document is compiled,
the appropriate numbering of pages
(as well as other status parameters)
is determined from the |.aux| files.
The latter contain information from previous passes.
However this information needs to propagate through
all intermediate child documents.
Therefore the page numbering in child documents may well
be inconsistent until the complete document is compiled at least once.

A useful (if unconventional) way to always ensure a consistent
page numbering is to restart the numbering in each child document
and denote the pages by `\textit{child}|.|\textit{page}'
where \textit{child} represents the chapter/section number of the child file.
This can be achieved by the command
|\numberwithin{page}{|\textit{child}|}|
of the \textsf{amsmath} package
where \textit{child} can be |chapter| or |section|
depending on the chosen structuring.
Alternatively, one can modify the macro |\thepage| appropriately
and reset the counter |page| at the start of each child file.

%%%%%%%%%%%%%%%%%%%%%%%%%%%%%%%%%%%%%%%%%%%%%%%%%%%%%%%%%%%%%%%%%%%%%%%%%%%%%%%%
\subsection{Conditional Processing}
\label{sec:conditional}

The package provides a mechanism to compile different versions
of a document. To customise the versions further some conditional processing
can come in handy to distinguish which version is being compiled.
The package provides two macros to describe the compilation context:

%%%%%%%%%%%%%%%%%%%%%%%%%%%%%%%%%%%%%%%%
\DescribeMacro{\ifchilddoc}
The conditional |\ifchilddoc| distinguishes between the compilation of
child documents and the main document:
%
\begin{center}
|\ifchilddoc |\textit{child-code}| |[|\||else |\textit{main-code}]| \||fi|
\end{center}

%%%%%%%%%%%%%%%%%%%%%%%%%%%%%%%%%%%%%%%%
\DescribeMacro{\childdocname}
\DescribeMacro{\childdocjob}
The macro |\childdocname| contains the filename (without extension)
of the main or child file being processed.
Note that |\childdocjob| will always contain the name of the main file.

%%%%%%%%%%%%%%%%%%%%%%%%%%%%%%%%%%%%%%%%
\paragraph{Title Page.}

Conditional processing can be used to include a title or banner page
in the main document when proper precautions are taken.
Importantly, the code in the main file should ensure that the page counter
(as well as other status parameters which are stored in the |.aux| files)
takes the same value after the conditional processing.
Otherwise the page numbers may take divergent values
depending on which part is compiled.

For example, a title page could be declared by:
%
\begin{center}
\begin{tabular}{l}
|\ifchilddoc\||else|\\
|\addtocounter{page}{-1}|\\
\textit{code for title page}\\
|\newpage|\\
|\||fi|
\end{tabular}
\end{center}
%
A banner page for the child documents can be generated by:
%
\begin{center}
\begin{tabular}{l}
|\ifchilddoc|\\
|\addtocounter{page}{-1}|\\
\textit{code for banner page}\\
|\newpage|\\
|\||fi|
\end{tabular}
\end{center}
%
Here one could write a message such as:
\begin{center}
|This is the part \childdocname{} of \childdocjob{}.|
\end{center}

%%%%%%%%%%%%%%%%%%%%%%%%%%%%%%%%%%%%%%%%%%%%%%%%%%%%%%%%%%%%%%%%%%%%%%%%%%%%%%%%
\subsection{Flags}
\label{sec:flags}

The package makes it easy to generate different versions
of the main or child documents.
To this end compilation flags can be defined
and assigned different default values.
They will be particularly useful in conjunction
with the forwarding mechanism described in \secref{sec:forward}.

For example, it may be useful to have a flag |\version|
which can be set to |draft| or |final|.
The document source will contain some conditional code
depending on the value of |\version|.
Suppose further, the flag should default to |final| for the main file
and to |draft| for child files
which is a natural assignment for editing the document.
This is achieved by placing the following code
in the preamble of the main document
(below the |\childdocmain| directive):
%
\begin{center}
\begin{tabular}{l}
|\ifchilddoc|\\
|\providecommand{\version}{draft}|\\
|\||else|\\
|\providecommand{\version}{final}|\\
|\||fi|
\end{tabular}
\end{center}
%
The definition by |\providecommand| makes sure
that previous definitions are not overwritten.
Further statements |\providecommand{\version}{...}|
can thus be added before the above code to override it.

For the main file, one might add a line
(between |\childdocmain| and the above block)
%
\begin{center}
|%\ifchilddoc\||else\providecommand{\version}{draft}\||fi|
\end{center}
%
which can be uncommented to produce a draft version.
Likewise one can add a line to the very top of a child file
(above the |\childdocof{|\textit{main}|}| directive)
%
\begin{center}
|%\providecommand{\version}{final}|
\end{center}
%
which can be uncommented to produce the final version of this child document.

%%%%%%%%%%%%%%%%%%%%%%%%%%%%%%%%%%%%%%%%%%%%%%%%%%%%%%%%%%%%%%%%%%%%%%%%%%%%%%%%
\subsection{Forwarding}
\label{sec:forward}

Different versions of the main or child documents
using compilation flags as described in \secref{sec:flags}
can be (permanently) stored in different files
for convenient compilation, viewing and distribution.
To this end, the package defines a command
to pass on compilation to a different file:

%%%%%%%%%%%%%%%%%%%%%%%%%%%%%%%%%%%%%%%%
\DescribeMacro{\childdocforward}
The command |\childdocforward| redirects processing to
another source file:
%
\begin{center}
\begin{tabular}{l}
|\input{childdoc.def}|\\
|\childdocforward[|\textit{main}|]{|\textit{dest}|}|\\
\end{tabular}
\end{center}
%
The argument \textit{dest} is the destination file
(without extension).
It should be the main file or one of the child files.
Note that further \textsf{childdoc} directives
such as |\childdocof| and |\childdocforward|
in the indicated file will be processed in this form.
The optional argument \textit{main}
passes on directly to the main file \textit{main}
while pretending to compile the child \textit{dest}.
This form behaves as if \textit{dest}
issues |\childdocof{|\textit{main}|}| right away,
and no further \textsf{childdoc} directives will be processed.

%%%%%%%%%%%%%%%%%%%%%%%%%%%%%%%%%%%%%%%%
\DescribeMacro{\...prefix}
In the alternative form |\childdocforwardprefix|,
%
\begin{center}
\begin{tabular}{l}
|\input{childdoc.def}|\\
|\childdocforwardprefix[|\textit{main}|]{|\textit{prefix}|}{|\textit{dest}|}|
\end{tabular}
\end{center}
%
the destination file is determined by a pattern
depending on the current file:
To make this work, the current file must be called
`{\textit{prefix}\hspace{0.2em}\textit{suffix}}'
with \textit{prefix} matching precisely the argument.
Processing is then passed on to the file
`{\textit{dest}\hspace{0.2em}\textit{suffix}}'.
Surely, the same effect is achieved by
directly specifying the
argument `{\textit{dest}\hspace{0.2em}\textit{suffix}}'
in the first form.
However, that requires to set up a different file
for each child. With the alternative form of the command
all these files can have exactly the same content
which simplifies setting them up and maintaining them.

For example, the following file |draft.tex|
with a compilation flag |\version| as described in \secref{sec:flags}
compiles the main document as a draft:
%
\begin{center}
\begin{tabular}{l}
|\def\version{draft}|\\
|\input{childdoc.def}|\\
|\childdocforward{|\textit{main}|}|
\end{tabular}
\end{center}
%
Likewise, the following files |final|\textit{nn}|.tex|
compile the final version of the child document
|child|\textit{nn}|.tex|:
%
\begin{center}
\begin{tabular}{l}
|\def\version{final}|\\
|\input{childdoc.def}|\\
|\childdocforwardprefix{final}{child}|
\end{tabular}
\end{center}
%

Note that when several versions of a main file and/or of each child file
are to be generated, it may be convenient to set up a |Makefile| or
shell script to automatise the process.

%%%%%%%%%%%%%%%%%%%%%%%%%%%%%%%%%%%%%%%%%%%%%%%%%%%%%%%%%%%%%%%%%%%%%%%%%%%%%%%%
\subsection{Command Line Processing}
\label{sec:commandline}

The effect of redirection files can also be achieved by invoking
the \LaTeX{} compiler with a more elaborate command line.
Most conveniently this should be done as part
of a shell script or a |Makefile|.

When using \textsf{childdoc} in the main file, the following
command lines effectively perform a redirection
(note that depending on the shell being used,
backslashes may have to be doubled: `|\|' $\to$ `|\\|'):
%
\begin{center}
|... -jobname "|\textit{target}|" |\\|"|[\textit{flags}]%
|\input{childdoc.def}\childdocforward[|\textit{main}|]{|\textit{dest}|}"|
\end{center}
%
Here \textit{target} is the name of the output file,
\textit{main} is the name of the main file
and \textit{dest} is the name of the main or child file to be processed
(all filenames without extensions).
The optional argument \textit{main} can be omitted
if \textit{main} matches \textit{dest}.
Optionally, compilation \textit{flags} can be defined via |\def| commands.
This command line makes the \TeX{} engine believe
it is compiling the file \textit{target}
whose content is specified as the latter parameter.
The provided code then forwards the processing to
\textit{main} or \textit{dest} as described in \secref{sec:forward}.

%%%%%%%%%%%%%%%%%%%%%%%%%%%%%%%%%%%%%%%%%%%%%%%%%%%%%%%%%%%%%%%%%%%%%%%%%%%%%%%%
\subsection{Include by Input}
\label{sec:input}

Including child documents by |\include| has some restrictions by design.
Most notably, the content of a child document always occupies
its own set of pages; pages cannot be shared between child documents.
Usually, this behaviour makes perfect sense
because each child document contain an essential part of the document.
However, in some situations it may be desirable to compose
a document from a collection of parts
without having mandatory page breaks between then.
For this case, the package
provides a mechanism to include parts
by |\input| which can also be processed individually.
However, by construction this mechanism
requires manual handling of the content to be output.

%%%%%%%%%%%%%%%%%%%%%%%%%%%%%%%%%%%%%%%%
\DescribeMacro{\ifchilddocmanual}
The main file should be prepared as usual, see \secref{sec:include}.
However, the document body must make a distinction
between processing of an individual part and of the main document, e.g.:
%
\begin{center}
\begin{tabular}{l}
|\ifchilddocmanual|\\
|\input{\childdocname}|\\
|\||else|\\
\textit{document body with }|\input{|\textit{part}|}|\\
|\||fi|
\end{tabular}
\end{center}
%
The conditional |\ifchilddocmanual| is true whenever
a part to be included by |\input| is being compiled,
and the name of the part is stored in |\childdocname|.

%%%%%%%%%%%%%%%%%%%%%%%%%%%%%%%%%%%%%%%%
\DescribeMacro{\childdocby}
Each part to be included by |\input| should start with:
%
\begin{center}
\begin{tabular}{l}
|\input{childdoc.def}|\\
|\childdocby{|\textit{main}|}|\\
\end{tabular}
\end{center}
%
The directive |\childdocby| is similar to |\childdocof|
described in \secref{sec:include},
but the subsequent selection of content must be done manually.
To that end, both |\ifchilddoc| and |\ifchilddocmanual|
will be true upon processing of a part,
and the name of the part is stored in |\childdocname|.
Note that |\jobname| will be set to the filename of the current part
so that each part receives an individual |.aux| file
that does not interfere with the |.aux| file(s) of the main document.
This behaviour can be altered by the alternative form
|\childdocby[*]{|\textit{main}|}| (with a non-empty optional argument)
which uses the |.aux| file of the main document
by setting |\jobname| to \textit{main}.

%%%%%%%%%%%%%%%%%%%%%%%%%%%%%%%%%%%%%%%%%%%%%%%%%%%%%%%%%%%%%%%%%%%%%%%%%%%%%%%%
\subsection{Driver Development}
\label{sec:driver}

The \textsf{childdoc} mechanism can also be use for the development
of definition files such as \LaTeX{} styles or classes.
This case differs from the above setup with multiple parts
included by |\include| in that no |\includeonly| should be invoked.
This can be achieved by starting the include file
(before |\ProvidesPackage|) with:
%
\begin{center}
\begin{tabular}{l}
|\input{childdoc.def}|\\
|\childdocforward{|\textit{main}|}|\\
\end{tabular}
\end{center}
%
or alternatively with:
%
\begin{center}
\begin{tabular}{l}
|\input{childdoc.def}|\\
|\childdocby{|\textit{main}|}|\\
\end{tabular}
\end{center}
%
Both forms have slightly different effects as described above.
The main file is prepared as usual, see \secref{sec:include}.

%%%%%%%%%%%%%%%%%%%%%%%%%%%%%%%%%%%%%%%%%%%%%%%%%%%%%%%%%%%%%%%%%%%%%%%%%%%%%%%%
\subsection{Legacy Detection}
\label{sec:detection}

The directive |\childdocmain| in the main file can detect
whether the complete document or merely a child is to be compiled
even without using the directive |\childdocof|.
This method is deprecated because it is less robust
and there is no compelling reason to use it;
it is merely provided for backward compatibility
and it may be removed in future versions.

If the detection mechanism is to be used,
it is mandatory to correctly specify
the filename of the main file as the argument of |\childdocmain|:
%
\begin{center}
\begin{tabular}{l}
|\input{childdoc.def}|\\
|\childdocmain{|\textit{main}|}|\\
\end{tabular}
\end{center}
%
If |\jobname| does not match the argument \textit{main} of |\childdocmain|,
it is assumed that |\jobname| points to the child file to be compiled.
When using |\childdocmain| with the main file specified as argument,
it suffices to start a child file
with just |\input{|\textit{main}|}|
without loading of the package and using |\childdocof|.
If instead all processing is done
with the appropriate \textsf{childdoc} directives,
the argument of \textit{main} of |\childdocmain| can be empty.

An alternative version of the command line processing described
in \secref{sec:commandline} using the detection mechanism reads:
%
\begin{center}
|... -jobname "|\textit{target}|" "|[\textit{flags}]%
[|\def\jobname{|\textit{dest}|}|]|\input{|\textit{main}|}"|
\end{center}

%%%%%%%%%%%%%%%%%%%%%%%%%%%%%%%%%%%%%%%%%%%%%%%%%%%%%%%%%%%%%%%%%%%%%%%%%%%%%%%%
\subsection{Manual Code}
\label{sec:manual}

In case one cannot be certain whether the definitions file |childdoc.def|
is installed on the target \TeX{} distribution
and one prefers not to ship it,
it is conceivable to paste a few relevant commands into the sources.

To that end, drop all statements |\input{childdoc.def}|
and perform the replacements as outlined below.
Instead of |\childdocmain{|\textit{main}|}| add the following code
to the top of the main file:
%
\begin{center}
\begin{tabular}{l}
|\||ifdefined\childdocname\endinput\||fi\newif\ifchilddoc|\\
|\edef\childdocname{\scantokens\expandafter{\jobname\noexpand}}|\\
|\def\childdocmain{|\textit{main}|}\||ifx\childdocmain\childdocname\||else|\\
|\childdoctrue\includeonly{\childdocname}\let\jobname\childdocmain\||fi|\\
\end{tabular}
\end{center}
%
Instead of |\childdocof{|\textit{main}|}| just include the main file
at the top of each child file:
%
\begin{center}
|\input{|\textit{main}|}|
\end{center}
%
A simple redirection |\childdocforward{|\textit{dest}|}| is achieved by:
%
\begin{center}
|\def\jobname{|\textit{dest}|}\input{\jobname}|
\end{center}
%
The redirection with prefix
|\childdocforwardprefix[|\textit{prefix}|]{|\textit{dest}|}|
is accomplished by:
%
\begin{center}
\begin{tabular}{l}
|{\edef\jobname{\scantokens\expandafter{\jobname\noexpand}}|\\
|\def\redirectjob |\textit{prefix}|#1~~~{\gdef\jobname{|\textit{dest}|#1}}|\\
|\expandafter\redirectjob\jobname~~~}\input{\jobname}|
\end{tabular}
\end{center}

In an alternative approach,
child documents can be compiled by a specific command line
without additional code or specific definitions:
%
\begin{center}
|... -jobname "|\textit{target}|" "|[\textit{flags}]%
|\includeonly{|\textit{dest}|}\input{|\textit{main}|}"|
\end{center}
%

%%%%%%%%%%%%%%%%%%%%%%%%%%%%%%%%%%%%%%%%%%%%%%%%%%%%%%%%%%%%%%%%%%%%%%%%%%%%%%%%
%%%%%%%%%%%%%%%%%%%%%%%%%%%%%%%%%%%%%%%%%%%%%%%%%%%%%%%%%%%%%%%%%%%%%%%%%%%%%%%%
\section{Information}

%%%%%%%%%%%%%%%%%%%%%%%%%%%%%%%%%%%%%%%%%%%%%%%%%%%%%%%%%%%%%%%%%%%%%%%%%%%%%%%%
\subsection{Copyright}

Copyright \copyright{} 2017--2018 Niklas Beisert

This work may be distributed and/or modified under the
conditions of the \LaTeX{} Project Public License, either version 1.3
of this license or (at your option) any later version.
The latest version of this license is in
  \url{http://www.latex-project.org/lppl.txt}
and version 1.3 or later is part of all distributions of \LaTeX{}
version 2005/12/01 or later.

This work has the LPPL maintenance status `maintained'.

The Current Maintainer of this work is Niklas Beisert.

This work consists of the files |README.txt|, |childdoc.ins| and |childdoc.dtx|
as well as the derived files |childdoc.def|, |cdocsamp.tex|
with |cdocsch1.tex|, |cdocsch2.tex|, |cdocspt3.tex|, |cdocspt4.tex|,
|cdocsdrf.tex|, |cdocsfn1.tex|, |cdocsfn2.tex|
as well as |childdoc.pdf|.

%%%%%%%%%%%%%%%%%%%%%%%%%%%%%%%%%%%%%%%%%%%%%%%%%%%%%%%%%%%%%%%%%%%%%%%%%%%%%%%%
\subsection{Files and Installation}

The package consists of the files:
%
\begin{center}
\begin{tabular}{ll}
    |README.txt|   & readme file \\
    |childdoc.ins| & installation file \\
    |childdoc.dtx| & source file \\
    |childdoc.def| & definition file \\
    |cdocsamp.tex| & sample main file \\
    |cdocsch1.tex| & sample include file \\
    |cdocsch2.tex| & sample include file \\
    |cdocspt3.tex| & sample part file \\
    |cdocspt4.tex| & sample part file \\
    |cdocsdrf.tex| & sample redirection file \\
    |cdocsfn1.tex| & sample redirection file \\
    |cdocsfn2.tex| & sample redirection file \\
    |childdoc.pdf| & manual
\end{tabular}
\end{center}
%
The distribution consists of the files
|README.txt|, |childdoc.ins| and |childdoc.dtx|.
%
\begin{itemize}
\item
Run (pdf)\LaTeX{} on |childdoc.dtx|
to compile the manual |childdoc.pdf| (this file).
\item
Run \LaTeX{} on |childdoc.ins| to create the definitions file |childdoc.def|
and the sample |cdocsamp.tex| with include files
|cdocsch1.tex|, |cdocsch2.tex|, |cdocspt3.tex|, |cdocspt4.tex|,
|cdocsdrf.tex|, |cdocsfn1.tex|, |cdocsfn2.tex|.
Then copy the file |childdoc.def| to an appropriate directory of your \LaTeX{}
distribution, e.g.\ \textit{texmf-root}|/tex/latex/childdoc|.
\end{itemize}

%%%%%%%%%%%%%%%%%%%%%%%%%%%%%%%%%%%%%%%%%%%%%%%%%%%%%%%%%%%%%%%%%%%%%%%%%%%%%%%%
\subsection{Related CTAN Packages}

There are several other packages which offer a similar functionality:
%
\begin{itemize}
\item
The packages
\href{http://ctan.org/pkg/docmute}{\textsf{docmute}},
\href{http://ctan.org/pkg/includex}{\textsf{includex}} and
\href{http://ctan.org/pkg/standalone}{\textsf{standalone}}
provide commands to include only the document body of
a child file thus allowing both files to be compiled individually.
\item
The packages \href{http://ctan.org/pkg/subdocs}{\textsf{subdocs}}
and \href{http://ctan.org/pkg/subfiles}{\textsf{subfiles}}
provide structures in which the main and child documents can be
encapsulated and allowing them to be compiled individually.
The inclusion mechanism is different from the conventional |\include|.
\item
The package \href{http://ctan.org/pkg/combine}{\textsf{combine}}
is an elaborate solution to combine several documents into one.
\end{itemize}
%
See also the CTAN topic \href{http://ctan.org/topic/subdocs}{\textsf{subdocs}}
for further related packages.
The present package differs from the above solutions in that
a document structure constructed with the conventional |\include| mechanism
just needs two extra commands at the top of every file
such that all constituent files can be compiled individually.

%%%%%%%%%%%%%%%%%%%%%%%%%%%%%%%%%%%%%%%%%%%%%%%%%%%%%%%%%%%%%%%%%%%%%%%%%%%%%%%%
%\subsection{Feature Suggestions}
%
%The following is a list of features which may be useful for future
%versions of this package:
%%
%\begin{itemize}
%\item
%\ldots
%\end{itemize}

%%%%%%%%%%%%%%%%%%%%%%%%%%%%%%%%%%%%%%%%%%%%%%%%%%%%%%%%%%%%%%%%%%%%%%%%%%%%%%%%
\subsection{Revision History}

%%%%%%%%%%%%%%%%%%%%%%%%%%%%%%%%%%%%%%%%
\paragraph{v2.0:} 2018/12/30

\begin{itemize}
\item
immediate forward processing
\item
added |\childdocby| mechanism
\item
manual restructured
\end{itemize}

%%%%%%%%%%%%%%%%%%%%%%%%%%%%%%%%%%%%%%%%
\paragraph{v1.6:} 2018/01/17

\begin{itemize}
\item
application for development of include files
\item
corrections to manual
\end{itemize}

%%%%%%%%%%%%%%%%%%%%%%%%%%%%%%%%%%%%%%%%
\paragraph{v1.5:} 2017/05/21

\begin{itemize}
\item
more complete structuring introduced
\item
|\childdocof| introduced
\item
|\childdoc| renamed to |\childdocmain|
\item
|\childredirect| renamed to |\childdocforward| and |\childdocforwardprefix|
and functionality expanded
\end{itemize}

%%%%%%%%%%%%%%%%%%%%%%%%%%%%%%%%%%%%%%%%
\paragraph{v1.0:} 2017/04/27

\begin{itemize}
\item
manual and install package
\item
first version published on CTAN
\end{itemize}

%%%%%%%%%%%%%%%%%%%%%%%%%%%%%%%%%%%%%%%%
\paragraph{v0.6:} 2017/04/26

\begin{itemize}
\item
redirection mechanism added
\end{itemize}

%%%%%%%%%%%%%%%%%%%%%%%%%%%%%%%%%%%%%%%%
\paragraph{v0.5:} 2017/04/26

\begin{itemize}
\item
functionality in definition file
\end{itemize}


%%%%%%%%%%%%%%%%%%%%%%%%%%%%%%%%%%%%%%%%%%%%%%%%%%%%%%%%%%%%%%%%%%%%%%%%%%%%%%%%
%%%%%%%%%%%%%%%%%%%%%%%%%%%%%%%%%%%%%%%%%%%%%%%%%%%%%%%%%%%%%%%%%%%%%%%%%%%%%%%%
%%%%%%%%%%%%%%%%%%%%%%%%%%%%%%%%%%%%%%%%%%%%%%%%%%%%%%%%%%%%%%%%%%%%%%%%%%%%%%%%
\appendix

\settowidth\MacroIndent{\rmfamily\scriptsize 000\ }

 \DocInput{childdoc.dtx}

\end{document}
%</driver>
% \fi
%
% %%%%%%%%%%%%%%%%%%%%%%%%%%%%%%%%%%%%%%%%%%%%%%%%%%%%%%%%%%%%%%%%%%%%%%%%%%%%%%
% %%%%%%%%%%%%%%%%%%%%%%%%%%%%%%%%%%%%%%%%%%%%%%%%%%%%%%%%%%%%%%%%%%%%%%%%%%%%%%
% \section{Sample}
%\iffalse
%<*samplemain>
%\fi
%
% The following presents a sample document
% with two chapters, two parts, a title page,
% a compile flag as well as three forwarding files to set the flag.
% It consists of eight |.tex| files:
% \begin{center}
% \begin{tabular}{ll}
% |cdocsamp.tex|&main file\\
% |cdocsch1.tex|&include file for chapter 1\\
% |cdocsch2.tex|&include file for chapter 2\\
% |cdocspt3.tex|&include file for part 3\\
% |cdocspt4.tex|&include file for part 4\\
% |cdocsdrf.tex|&forwarding file for main file in draft mode\\
% |cdocsfi1.tex|&forwarding file for final version of chapter 1\\
% |cdocsfi2.tex|&forwarding file for final version of chapter 2\\
% \end{tabular}
% \end{center}
% Each of the eight files can be compiled directly by the \LaTeX{} compiler.
%
% %%%%%%%%%%%%%%%%%%%%%%%%%%%%%%%%%%%%%%
% \paragraph{Main File.}
%
% The main file is called |cdocsamp.tex|.
%
% Load the \textsf{childdoc} definitions and
% declare the filename for the main document:
%    \begin{macrocode}
\input{childdoc.def}
\childdocmain{}
%    \end{macrocode}

% Optional override for |\version| flag:
%    \begin{macrocode}
%%\ifchilddoc\else\providecommand{\version}{draft}\fi
%    \end{macrocode}

% Define the default values for the |\version| flag
% (|final| for the main file and |draft| for childs):
%    \begin{macrocode}
\ifchilddoc
\providecommand{\version}{draft}
\else
\providecommand{\version}{final}
\fi
%    \end{macrocode}

% Load the standard document class:
%    \begin{macrocode}
\documentclass[12pt]{article}
%    \end{macrocode}

% Start the document body:
%    \begin{macrocode}
\begin{document}
%    \end{macrocode}

% Declare a title page.
% Print title, part of document being processed and version flag:
%    \begin{macrocode}
\addtocounter{page}{-1}
\begin{center}
{\LARGE\bfseries{}childdoc example\par}
\vspace{1cm}
\ifchilddoc
\ifchilddocmanual part\else chapter\fi:
`\childdocname' of `\childdocjob'\par
\else
main document: `\childdocjob'\par
\fi
version: \version\par
\end{center}
\newpage
%    \end{macrocode}

% Manually include selected file,
% otherwise process as usual:
%    \begin{macrocode}
\ifchilddocmanual
\section*{part `\childdocname'}
\input{\childdocname}
\else
%    \end{macrocode}

% Include the two chapters:
%    \begin{macrocode}
\include{cdocsch1}
\include{cdocsch2}
%    \end{macrocode}

% Include the two parts unless only chapters should be displayed:
%    \begin{macrocode}
\ifchilddoc\else
\section{part three}
\input{cdocspt3}
\section{part four}
\input{cdocspt4}
\fi
%    \end{macrocode}

% Process as usual until here:
%    \begin{macrocode}
\fi
%    \end{macrocode}

% End of document body:
%    \begin{macrocode}
\end{document}
%    \end{macrocode}
%\iffalse
%</samplemain>
%\fi
%
% %%%%%%%%%%%%%%%%%%%%%%%%%%%%%%%%%%%%%%
% \paragraph{Chapter Include Files.}
%
% The include files are called |cdocsch1.tex| and |cdocsch2.tex|.
%
%\iffalse
%<*samplechap1|samplechap2>
%\fi

% Optional override for |\version| flag:
%    \begin{macrocode}
%%\providecommand{\version}{final}
%    \end{macrocode}

% Include the main document:
%    \begin{macrocode}
\input{childdoc.def}
\childdocof{cdocsamp}
%    \end{macrocode}

%\iffalse
%</samplechap1|samplechap2>
%\fi
%
%\iffalse
%<*samplechap1>
%\fi
% Some text for chapter 1:
%    \begin{macrocode}
\section{one}
some text in chapter one
%    \end{macrocode}

%\iffalse
%</samplechap1>
%\fi
% Some text for chapter 2:
%\iffalse
%<*samplechap2>
%\fi
%    \begin{macrocode}
\section{two}
more text in chapter two
%    \end{macrocode}

%\iffalse
%</samplechap2>
%\fi
%
% %%%%%%%%%%%%%%%%%%%%%%%%%%%%%%%%%%%%%%
% \paragraph{Part Include Files.}
%
% The include files are called |cdocspt3.tex| and |cdocspt4.tex|.
%
%\iffalse
%<*samplepart3|samplepart4>
%\fi

% Optional override for |\version| flag:
%    \begin{macrocode}
%%\providecommand{\version}{final}
%    \end{macrocode}

% Include the main document:
%    \begin{macrocode}
\input{childdoc.def}
\childdocby{cdocsamp}
%    \end{macrocode}

%\iffalse
%</samplepart3|samplepart4>
%\fi
%
%\iffalse
%<*samplepart3>
%\fi
% Some text for part 3:
%    \begin{macrocode}
some text in part three
%    \end{macrocode}

%\iffalse
%</samplepart3>
%\fi
% Some text for part 4:
%\iffalse
%<*samplepart4>
%\fi
%    \begin{macrocode}
more text in part four
%    \end{macrocode}

%\iffalse
%</samplepart4>
%\fi
%
% %%%%%%%%%%%%%%%%%%%%%%%%%%%%%%%%%%%%%%
% \paragraph{Forwarding for a Complete Draft.}
%
% The following forwarding file |cdocsdrf.tex|
% compiles the main document in draft mode:
%\iffalse
%<*sampledraft>
%\fi
%    \begin{macrocode}
\def\version{draft}
\input{childdoc.def}
\childdocforward{cdocsamp}
%    \end{macrocode}

%\iffalse
%</sampledraft>
%\fi
%
% %%%%%%%%%%%%%%%%%%%%%%%%%%%%%%%%%%%%%%
% \paragraph{Forwarding for Final Version of the Chapters.}
%
% The following forwarding files |cdocsfn1.tex| and |cdocsfn2.tex|
% (with identical content)
% compile the final versions of the child documents
% |cdocsch1.tex| and |cdocsch2.tex|, respectively:
%\iffalse
%<*samplefinal>
%\fi
%    \begin{macrocode}
\def\version{final}
\input{childdoc.def}
\childdocforwardprefix[cdocsamp]{cdocsfn}{cdocsch}
%    \end{macrocode}

%\iffalse
%</samplefinal>
%\fi
%
% %%%%%%%%%%%%%%%%%%%%%%%%%%%%%%%%%%%%%%
% \paragraph{Command Line Processing.}
%
% The following three command lines generate the output files
% |cdocscld|, |cdocscl1| and |cdocscl2|
% which should be identical to
% |cdocsdrf|, |cdocsch1| and |cdocsfn2|, respectively:
% \begin{center}
% \begin{tabular}{l}
% |latex -jobname cdocscld \|\\
% |  "\def\version{draft}\input{childdoc.def}\childdocforward{cdocsamp}"|\\
% |latex -jobname cdocscl1 \|\\
% |  "\input{childdoc.def}\childdocforward[cdocsamp]{cdocsch1}"|\\
% |latex -jobname cdocscl2 \|\\
% |  "\def\version{final}\input{childdoc.def}\childdocforward{cdocsch2}"|
% \end{tabular}
% \end{center}
% Note that the trailing backslash on each first line
% merely continues the input to the second line
% (for convenient cut ant paste).
% Furthermore, the command |latex| can be replaced by any
% of its alternative versions such as |pdflatex|.
%
% %%%%%%%%%%%%%%%%%%%%%%%%%%%%%%%%%%%%%%%%%%%%%%%%%%%%%%%%%%%%%%%%%%%%%%%%%%%%%%
% %%%%%%%%%%%%%%%%%%%%%%%%%%%%%%%%%%%%%%%%%%%%%%%%%%%%%%%%%%%%%%%%%%%%%%%%%%%%%%
% \section{Implementation}
%\iffalse
%<*package>
%\fi
%
% This section describes the definitions file |childdoc.def|.

% The definitions cannot be loaded using |\usepackage| or |\RequirePackage|
% which has a mechanism to prevent loading a style file more than once.
% When loading the definitions by means of |\input|
% multiple instances have to be prevented manually:
%\iffalse
%This code needs to be before the `\ProvidesFile' directive
%which is defined at the beginning of this file.
%Therefore it is also placed there and commented out here.
%</package>
%<*discard>
%\fi
%    \begin{macrocode}
\ifdefined\childdocmain\endinput\fi
%    \end{macrocode}
%\iffalse
%</discard>
%<*package>
%\fi
%
% \macro{\ifchilddoc}
% \macro{\ifchilddocmanual}
% The conditional |\ifchilddoc| tells whether a
% child (true) or main (false) document is being compiled.
% The conditional |\ifchilddocmanual| tells whether
% the |\includeonly| mechanism is used (false) or
% the selection of child files must be performed manually (true).
% The definitions initialise to false:
%    \begin{macrocode}
\newif\ifchilddoc
\newif\ifchilddocmanual
%    \end{macrocode}

% \macro{\childdocname}
% \macro{\childdocjob}
% The macro |\childdocname| stores the name of the main document
% to be compiled. The macro |\childdocjob| stores the name of
% the document on which the \LaTeX{} compiler was originally invoked.
% The content of |\jobname| cannot be compared
% to filenames specified in the source due to different catcodes.
% The following code rescans |\jobname|, stores the result
% in |\childdocname| and saves a copy in |\childdocjob|:
%    \begin{macrocode}
\edef\childdocname{\scantokens\expandafter{\jobname\noexpand}}
\let\childdocjob\childdocname
%    \end{macrocode}

% \macro{\childdocdisable}
% The macro |\childdocdisable| prevents the main file
% from being processed more than once.
% At this stage, the main document command |\childdocmain|
% is assumed to be called once again where it should do nothing.
% Any subsequent call to it should prevent
% a secondary processing of the main document
% It overwrites the forwarding commands
% |\childdocof| and |\childdocforward|
% with empty macros to prevent further inclusions of the main document:
%    \begin{macrocode}
\newcommand{\childdocdisable}
{
  \renewcommand{\childdocmain}[1]{\renewcommand{\childdocmain}[1]{\endinput}}
  \renewcommand{\childdocof}[1]{}
  \renewcommand{\childdocby}[2][]{}
  \renewcommand{\childdocforward}[2][]{}
  \renewcommand{\childdocdisable}{}
}
%    \end{macrocode}

% \macro{\childdocmain}
% The macro |\childdocmain| is to be called at the top of the main file
% with nothing or the main filename (without extension) as argument.
% First, it breaks loops.
% If the argument is not empty and does not match |\childdocname|
% (which is set by the first inclusion of |childdoc.def|),
% |\ifchilddoc| is set to true, |\includeonly| is applied to the child file
% and |\jobname| is set to the main file
% (for proper handling of |.aux| files):
%    \begin{macrocode}
\newcommand{\childdocmain}[1]
{
  \childdocdisable\childdocmain{}
  \if?#1?\else
    \begingroup
      \def\childdoctmp{#1}
      \ifx\childdoctmp\childdocname
        \def\childdoctmp{}
      \else
        \def\childdoctmp
        {
          \childdoctrue
          \includeonly{\childdocname}
          \def\childdocjob{#1}
          \def\jobname{#1}
        }
      \fi
      \expandafter
    \endgroup
    \childdoctmp
  \fi
}
%    \end{macrocode}

% \macro{\childdocof}
% The command |\childdocof| redirects
% compilation to the main file |#1|.
%    \begin{macrocode}
\newcommand{\childdocof}[1]
{
  \childdocdisable
  \childdoctrue
  \includeonly{\childdocname}
  \def\jobname{#1}
  \def\childdocjob{#1}
  \input{#1}
}
%    \end{macrocode}

% \macro{\childdocby}
% The command |\childdocby| ....
%    \begin{macrocode}
\newcommand{\childdocby}[2][]
{
  \childdocdisable
  \childdoctrue
  \childdocmanualtrue
  \if?#1?\else
    \def\jobname{#2}
  \fi
  \def\childdocjob{#2}
  \input{#2}
  \endinput
}
%    \end{macrocode}

% \macro{\childdocforward}
% The command |\childdocforward| redirects
% compilation to the main file or
% (if the optional argument is given) a child file.
% Parameters are set as if the main file
% or a child file starting with |\childdocof| was compiled.
% Then compilation is handed over to the main file:
%    \begin{macrocode}
\newcommand{\childdocforward}[2][]
{
  \begingroup
    \if?#1?
      \def\childdoctmp
      {
        \def\childdocname{#2}
        \def\childdocjob{#2}
        \def\jobname{#2}
        \input{#2}
        \endinput
      }
    \else
      \def\childdoctmp
      {
        \childdocdisable
        \def\childdocname{#2}
        \childdoctrue
        \includeonly{#2}
        \def\childdocjob{#1}
        \def\jobname{#1}
        \input{#1}
        \endinput
      }
    \fi
    \expandafter
  \endgroup
  \childdoctmp
}
%    \end{macrocode}

% \macro{\childdocforwardprefix}
% The command |\childdocforwardprefix| redirects
% compilation to the main or a child file by means of a pattern.
% The prefix |#1| in the current filename is replaced by |#2|
% and the suffix of the current filename is kept
% (it is assumed that the filename does not contain the substring `|~~~|'
% which is used as a delimiter).
% Compilation is handed over to the new file by |\childdocforward|:
%    \begin{macrocode}
\newcommand{\childdocforwardprefix}[3][]
{
  \begingroup
    \def\childdocextract #2##1~~~{\def\childdoctmp{\childdocforward[#1]{#3##1}}}
    \expandafter\childdocextract\childdocname~~~
    \expandafter
  \endgroup
  \childdoctmp
}
%    \end{macrocode}

% \macro{\childdoc}
% The deprecated macro |\childdoc| is a legacy version of |\childdocmain|:
%    \begin{macrocode}
\newcommand{\childdoc}{\childdocmain}
%    \end{macrocode}

% \macro{\childdocredirect}
% The deprecated macro |\childdocredirect| is a legacy version
% of |\childdocforward| and |\childdocforwardprefix|:
%    \begin{macrocode}
\newcommand{\childdocredirect}[2][]
{
  \begingroup
    \if?#1?
      \def\childdoctmp{\childdocforward{#2}}
    \else
      \def\childdoctmp{\childdocforwardprefix{#1}{#2}}
    \fi
    \expandafter
  \endgroup
  \childdoctmp
}
%    \end{macrocode}

%\iffalse
%</package>
%\fi
%
\endinput

\childdocof{cdocsamp}
%    \end{macrocode}

%\iffalse
%</samplechap1|samplechap2>
%\fi
%
%\iffalse
%<*samplechap1>
%\fi
% Some text for chapter 1:
%    \begin{macrocode}
\section{one}
some text in chapter one
%    \end{macrocode}

%\iffalse
%</samplechap1>
%\fi
% Some text for chapter 2:
%\iffalse
%<*samplechap2>
%\fi
%    \begin{macrocode}
\section{two}
more text in chapter two
%    \end{macrocode}

%\iffalse
%</samplechap2>
%\fi
%
% %%%%%%%%%%%%%%%%%%%%%%%%%%%%%%%%%%%%%%
% \paragraph{Part Include Files.}
%
% The include files are called |cdocspt3.tex| and |cdocspt4.tex|.
%
%\iffalse
%<*samplepart3|samplepart4>
%\fi

% Optional override for |\version| flag:
%    \begin{macrocode}
%%\providecommand{\version}{final}
%    \end{macrocode}

% Include the main document:
%    \begin{macrocode}
% \iffalse
%
% childdoc.dtx Copyright (C) 2017-2018 Niklas Beisert
%
% This work may be distributed and/or modified under the
% conditions of the LaTeX Project Public License, either version 1.3
% of this license or (at your option) any later version.
% The latest version of this license is in
%   http://www.latex-project.org/lppl.txt
% and version 1.3 or later is part of all distributions of LaTeX
% version 2005/12/01 or later.
%
% This work has the LPPL maintenance status `maintained'.
%
% The Current Maintainer of this work is Niklas Beisert.
%
% This work consists of the files childdoc.dtx and childdoc.ins
% and the derived files childdoc.def and cdocsamp.tex with
% cdocsch1.tex, cdocsch2.tex, cdocsdrf.tex, cdocsfn1.tex, cdocsfn2.tex.
%
%<package>\ifdefined\childdocmain\endinput\fi
%<package>\ProvidesFile{childdoc.def}[2018/12/30 v2.0 child document driver]
%<samplemain>\ProvidesFile{cdocsamp.tex}[2018/12/30 v2.0 sample for childdoc]
%<*driver>
%\ProvidesFile{childdoc.drv}[2018/12/30 v2.0 childdoc reference manual file]
\PassOptionsToClass{10pt,a4paper}{article}
\documentclass{ltxdoc}

\usepackage[margin=35mm]{geometry}
\usepackage{hyperref}
\usepackage{hyperxmp}
\usepackage[usenames]{color}

\hypersetup{colorlinks=true}
\hypersetup{pdfstartview=FitH}
\hypersetup{pdfpagemode=UseNone}
\hypersetup{pdfsource={}}
\hypersetup{pdflang={en-UK}}
\hypersetup{pdfcopyright={Copyright 2017-2018 Niklas Beisert.
  This work may be distributed and/or modified under the
  conditions of the LaTeX Project Public License, either version 1.3
  of this license or (at your option) any later version.}}
\hypersetup{pdflicenseurl={http://www.latex-project.org/lppl.txt}}
\hypersetup{pdfcontactaddress={ETH Zurich, ITP, HIT K,
  Wolfgang-Pauli-Strasse 27}}
\hypersetup{pdfcontactpostcode={8093}}
\hypersetup{pdfcontactcity={Zurich}}
\hypersetup{pdfcontactcountry={Switzerland}}
\hypersetup{pdfcontactemail={nbeisert@itp.phys.ethz.ch}}
\hypersetup{pdfcontacturl={http://people.phys.ethz.ch/\xmptilde nbeisert/}}

\newcommand{\secref}[1]{\hyperref[#1]{section \ref*{#1}}}

\parskip1ex
\parindent0pt
\let\olditemize\itemize
\def\itemize{\olditemize\parskip0pt}

\begin{document}

\title{The \textsf{childdoc} Package}
\hypersetup{pdftitle={The childdoc Package}}
\author{Niklas Beisert\\[2ex]
  Institut f\"ur Theoretische Physik\\
  Eidgen\"ossische Technische Hochschule Z\"urich\\
  Wolfgang-Pauli-Strasse 27, 8093 Z\"urich, Switzerland\\[1ex]
  \href{mailto:nbeisert@itp.phys.ethz.ch}
  {\texttt{nbeisert@itp.phys.ethz.ch}}}
\hypersetup{pdfauthor={Niklas Beisert}}
\hypersetup{pdfsubject={Manual for the LaTeX2e Package childdoc}}
\date{30 December 2018, \textsf{v2.0}}
\maketitle

\begin{abstract}\noindent
\textsf{childdoc} is a \LaTeXe{} package
that enables the direct compilation
of document sections included by |\include|
to individual files.
\end{abstract}

\begingroup
\parskip0ex
\tableofcontents
\endgroup

%%%%%%%%%%%%%%%%%%%%%%%%%%%%%%%%%%%%%%%%%%%%%%%%%%%%%%%%%%%%%%%%%%%%%%%%%%%%%%%%
%%%%%%%%%%%%%%%%%%%%%%%%%%%%%%%%%%%%%%%%%%%%%%%%%%%%%%%%%%%%%%%%%%%%%%%%%%%%%%%%
\section{Introduction}

\LaTeX{} provides a mechanism to structure a large document (such as a book)
into a main file and several child files (containing the chapters)
using the |\include| command.
This mechanism is beneficial for documents
which span hundreds of pages in order to
make the source file(s) more manageable.
Moreover, compilation can be restricted to
selected child files by means of the |\includeonly| command.
The latter feature can be used to reduce the compilation time while editing
(this was significantly more useful in the earlier days of \LaTeX{})
or to generate a smaller document which is easier to navigate.
Another application of |\includeonly| is to generate
documents consisting of selected parts of the complete document.

However, there are a few drawbacks of the plain |\include| mechanism:
\begin{itemize}
\item
The child files cannot be compiled on their own,
they can only be compiled via the main file.
A naive editing environment
(such as a text editor with an option
to have the current file processed by \LaTeX)
may require one to switch to the main file before compiling;
attempting to compile the child file produces errors.
\item
The main file must be modified (each time)
to adjust the |\includeonly| command
to the present needs. This easily leaves the main file in a messy state.
\item
The generated document will always carry the filename
of the main document. This is inconvenient if
several child files are to be compiled and
to be kept for distribution.
\end{itemize}

The present package provides a simple interface
to make child files individually compilable by \LaTeX{}.
Compiling a child file then has the same effect as compiling
the main file with an |\includeonly| command
to select the appropriate child.
Moreover the generated document will carry the name of the child
rather than the main file.
This resolves all three above issues.

This feature is meant to make the editing of books,
thesis documents and lecture notes somewhat more convenient.
However, the package can also be used efficiently for
composing a series of documents (such as exercise sheets)
which are typically distributed individually.
It then assists the author in generating the individual documents
(potentially in different versions)
as well as a document containing the collected series.
Another application is in developing style files
or other kinds of included material
where compilation of the style file could redirect
to a sample or test file.

%%%%%%%%%%%%%%%%%%%%%%%%%%%%%%%%%%%%%%%%%%%%%%%%%%%%%%%%%%%%%%%%%%%%%%%%%%%%%%%%
%%%%%%%%%%%%%%%%%%%%%%%%%%%%%%%%%%%%%%%%%%%%%%%%%%%%%%%%%%%%%%%%%%%%%%%%%%%%%%%%
\section{Usage}

First of all, the package \textsf{childdoc} is \emph{not} a standard
\LaTeXe{} |.sty| style file! Therefore it needs to be invoked in
a non-standard way.

%%%%%%%%%%%%%%%%%%%%%%%%%%%%%%%%%%%%%%%%%%%%%%%%%%%%%%%%%%%%%%%%%%%%%%%%%%%%%%%%
\subsection{Included Files}
\label{sec:include}

%%%%%%%%%%%%%%%%%%%%%%%%%%%%%%%%%%%%%%%%
\DescribeMacro{\childdocmain}
To use the package, add the commands
\begin{center}
\begin{tabular}{l}
|\input{childdoc.def}|\\
|\childdocmain{}|\\
\end{tabular}
\end{center}
at the very top of the main \LaTeX{} file,
in particular \emph{before} the |\documentclass| statement!
The argument of |\childdocmain| should be left empty
(but it must be present).

%%%%%%%%%%%%%%%%%%%%%%%%%%%%%%%%%%%%%%%%
\DescribeMacro{\childdocof}
Furthermore, add the commands
\begin{center}
\begin{tabular}{l}
|\input{childdoc.def}|\\
|\childdocof{|\textit{main}|}|\\
\end{tabular}
\end{center}
at the top of every child file \textit{child}
which is included by |\include{|\textit{child}|}|
from within the main file
(or at least for those files to be compiled individually).
The argument \textit{main} must be the filename of the main file.

There are a couple of
considerations in setting up the main and child documents:

%%%%%%%%%%%%%%%%%%%%%%%%%%%%%%%%%%%%%%%%
\paragraph{Restrictions.}

Please note the following restrictions:
\begin{itemize}
\item
|\childdocmain| must be called with one argument \textit{main}
to ensure compatibility with earlier version of the package.
It must either be empty (|\childdocmain{}|)
or precisely match the filename of the main file in which it is specified.
See \secref{sec:detection} for further information.
\item
The filename \textit{main} must be specified without the |.tex| extension.
\item
The filename \textit{main} is case sensitive
(even in case-insensitive file systems)
due to internal string comparison.
\item
The argument \textit{main} should be fully expanded, it cannot be a macro.
\item
Subdirectories and special characters should be avoided in filenames.
\item
The command |\childdocmain{|\textit{main}|}| must be followed by a whitespace.
It should not be followed immediately by another command
or by a comment mark `|%|'.
This is because the \TeX{} parser reads the token immediately following
the argument of |\childdocmain| and puts it
at the beginning of every child section;
however, a white\-space is ignored.
\end{itemize}

%%%%%%%%%%%%%%%%%%%%%%%%%%%%%%%%%%%%%%%%
\paragraph{Content of Main File.}

It is advisable to place all content in the child files included by |\include|.
Any output contained in the main file will appear in all child documents
unless suppressed manually;
it cannot be suppressed automatically by the |\includeonly| directive
and thus should normally be avoided.
A method to include some content in the main file
by means of conditional processing is described in \secref{sec:conditional}.

%%%%%%%%%%%%%%%%%%%%%%%%%%%%%%%%%%%%%%%%
\paragraph{Page Numbering.}

When only a part of the document is compiled,
the appropriate numbering of pages
(as well as other status parameters)
is determined from the |.aux| files.
The latter contain information from previous passes.
However this information needs to propagate through
all intermediate child documents.
Therefore the page numbering in child documents may well
be inconsistent until the complete document is compiled at least once.

A useful (if unconventional) way to always ensure a consistent
page numbering is to restart the numbering in each child document
and denote the pages by `\textit{child}|.|\textit{page}'
where \textit{child} represents the chapter/section number of the child file.
This can be achieved by the command
|\numberwithin{page}{|\textit{child}|}|
of the \textsf{amsmath} package
where \textit{child} can be |chapter| or |section|
depending on the chosen structuring.
Alternatively, one can modify the macro |\thepage| appropriately
and reset the counter |page| at the start of each child file.

%%%%%%%%%%%%%%%%%%%%%%%%%%%%%%%%%%%%%%%%%%%%%%%%%%%%%%%%%%%%%%%%%%%%%%%%%%%%%%%%
\subsection{Conditional Processing}
\label{sec:conditional}

The package provides a mechanism to compile different versions
of a document. To customise the versions further some conditional processing
can come in handy to distinguish which version is being compiled.
The package provides two macros to describe the compilation context:

%%%%%%%%%%%%%%%%%%%%%%%%%%%%%%%%%%%%%%%%
\DescribeMacro{\ifchilddoc}
The conditional |\ifchilddoc| distinguishes between the compilation of
child documents and the main document:
%
\begin{center}
|\ifchilddoc |\textit{child-code}| |[|\||else |\textit{main-code}]| \||fi|
\end{center}

%%%%%%%%%%%%%%%%%%%%%%%%%%%%%%%%%%%%%%%%
\DescribeMacro{\childdocname}
\DescribeMacro{\childdocjob}
The macro |\childdocname| contains the filename (without extension)
of the main or child file being processed.
Note that |\childdocjob| will always contain the name of the main file.

%%%%%%%%%%%%%%%%%%%%%%%%%%%%%%%%%%%%%%%%
\paragraph{Title Page.}

Conditional processing can be used to include a title or banner page
in the main document when proper precautions are taken.
Importantly, the code in the main file should ensure that the page counter
(as well as other status parameters which are stored in the |.aux| files)
takes the same value after the conditional processing.
Otherwise the page numbers may take divergent values
depending on which part is compiled.

For example, a title page could be declared by:
%
\begin{center}
\begin{tabular}{l}
|\ifchilddoc\||else|\\
|\addtocounter{page}{-1}|\\
\textit{code for title page}\\
|\newpage|\\
|\||fi|
\end{tabular}
\end{center}
%
A banner page for the child documents can be generated by:
%
\begin{center}
\begin{tabular}{l}
|\ifchilddoc|\\
|\addtocounter{page}{-1}|\\
\textit{code for banner page}\\
|\newpage|\\
|\||fi|
\end{tabular}
\end{center}
%
Here one could write a message such as:
\begin{center}
|This is the part \childdocname{} of \childdocjob{}.|
\end{center}

%%%%%%%%%%%%%%%%%%%%%%%%%%%%%%%%%%%%%%%%%%%%%%%%%%%%%%%%%%%%%%%%%%%%%%%%%%%%%%%%
\subsection{Flags}
\label{sec:flags}

The package makes it easy to generate different versions
of the main or child documents.
To this end compilation flags can be defined
and assigned different default values.
They will be particularly useful in conjunction
with the forwarding mechanism described in \secref{sec:forward}.

For example, it may be useful to have a flag |\version|
which can be set to |draft| or |final|.
The document source will contain some conditional code
depending on the value of |\version|.
Suppose further, the flag should default to |final| for the main file
and to |draft| for child files
which is a natural assignment for editing the document.
This is achieved by placing the following code
in the preamble of the main document
(below the |\childdocmain| directive):
%
\begin{center}
\begin{tabular}{l}
|\ifchilddoc|\\
|\providecommand{\version}{draft}|\\
|\||else|\\
|\providecommand{\version}{final}|\\
|\||fi|
\end{tabular}
\end{center}
%
The definition by |\providecommand| makes sure
that previous definitions are not overwritten.
Further statements |\providecommand{\version}{...}|
can thus be added before the above code to override it.

For the main file, one might add a line
(between |\childdocmain| and the above block)
%
\begin{center}
|%\ifchilddoc\||else\providecommand{\version}{draft}\||fi|
\end{center}
%
which can be uncommented to produce a draft version.
Likewise one can add a line to the very top of a child file
(above the |\childdocof{|\textit{main}|}| directive)
%
\begin{center}
|%\providecommand{\version}{final}|
\end{center}
%
which can be uncommented to produce the final version of this child document.

%%%%%%%%%%%%%%%%%%%%%%%%%%%%%%%%%%%%%%%%%%%%%%%%%%%%%%%%%%%%%%%%%%%%%%%%%%%%%%%%
\subsection{Forwarding}
\label{sec:forward}

Different versions of the main or child documents
using compilation flags as described in \secref{sec:flags}
can be (permanently) stored in different files
for convenient compilation, viewing and distribution.
To this end, the package defines a command
to pass on compilation to a different file:

%%%%%%%%%%%%%%%%%%%%%%%%%%%%%%%%%%%%%%%%
\DescribeMacro{\childdocforward}
The command |\childdocforward| redirects processing to
another source file:
%
\begin{center}
\begin{tabular}{l}
|\input{childdoc.def}|\\
|\childdocforward[|\textit{main}|]{|\textit{dest}|}|\\
\end{tabular}
\end{center}
%
The argument \textit{dest} is the destination file
(without extension).
It should be the main file or one of the child files.
Note that further \textsf{childdoc} directives
such as |\childdocof| and |\childdocforward|
in the indicated file will be processed in this form.
The optional argument \textit{main}
passes on directly to the main file \textit{main}
while pretending to compile the child \textit{dest}.
This form behaves as if \textit{dest}
issues |\childdocof{|\textit{main}|}| right away,
and no further \textsf{childdoc} directives will be processed.

%%%%%%%%%%%%%%%%%%%%%%%%%%%%%%%%%%%%%%%%
\DescribeMacro{\...prefix}
In the alternative form |\childdocforwardprefix|,
%
\begin{center}
\begin{tabular}{l}
|\input{childdoc.def}|\\
|\childdocforwardprefix[|\textit{main}|]{|\textit{prefix}|}{|\textit{dest}|}|
\end{tabular}
\end{center}
%
the destination file is determined by a pattern
depending on the current file:
To make this work, the current file must be called
`{\textit{prefix}\hspace{0.2em}\textit{suffix}}'
with \textit{prefix} matching precisely the argument.
Processing is then passed on to the file
`{\textit{dest}\hspace{0.2em}\textit{suffix}}'.
Surely, the same effect is achieved by
directly specifying the
argument `{\textit{dest}\hspace{0.2em}\textit{suffix}}'
in the first form.
However, that requires to set up a different file
for each child. With the alternative form of the command
all these files can have exactly the same content
which simplifies setting them up and maintaining them.

For example, the following file |draft.tex|
with a compilation flag |\version| as described in \secref{sec:flags}
compiles the main document as a draft:
%
\begin{center}
\begin{tabular}{l}
|\def\version{draft}|\\
|\input{childdoc.def}|\\
|\childdocforward{|\textit{main}|}|
\end{tabular}
\end{center}
%
Likewise, the following files |final|\textit{nn}|.tex|
compile the final version of the child document
|child|\textit{nn}|.tex|:
%
\begin{center}
\begin{tabular}{l}
|\def\version{final}|\\
|\input{childdoc.def}|\\
|\childdocforwardprefix{final}{child}|
\end{tabular}
\end{center}
%

Note that when several versions of a main file and/or of each child file
are to be generated, it may be convenient to set up a |Makefile| or
shell script to automatise the process.

%%%%%%%%%%%%%%%%%%%%%%%%%%%%%%%%%%%%%%%%%%%%%%%%%%%%%%%%%%%%%%%%%%%%%%%%%%%%%%%%
\subsection{Command Line Processing}
\label{sec:commandline}

The effect of redirection files can also be achieved by invoking
the \LaTeX{} compiler with a more elaborate command line.
Most conveniently this should be done as part
of a shell script or a |Makefile|.

When using \textsf{childdoc} in the main file, the following
command lines effectively perform a redirection
(note that depending on the shell being used,
backslashes may have to be doubled: `|\|' $\to$ `|\\|'):
%
\begin{center}
|... -jobname "|\textit{target}|" |\\|"|[\textit{flags}]%
|\input{childdoc.def}\childdocforward[|\textit{main}|]{|\textit{dest}|}"|
\end{center}
%
Here \textit{target} is the name of the output file,
\textit{main} is the name of the main file
and \textit{dest} is the name of the main or child file to be processed
(all filenames without extensions).
The optional argument \textit{main} can be omitted
if \textit{main} matches \textit{dest}.
Optionally, compilation \textit{flags} can be defined via |\def| commands.
This command line makes the \TeX{} engine believe
it is compiling the file \textit{target}
whose content is specified as the latter parameter.
The provided code then forwards the processing to
\textit{main} or \textit{dest} as described in \secref{sec:forward}.

%%%%%%%%%%%%%%%%%%%%%%%%%%%%%%%%%%%%%%%%%%%%%%%%%%%%%%%%%%%%%%%%%%%%%%%%%%%%%%%%
\subsection{Include by Input}
\label{sec:input}

Including child documents by |\include| has some restrictions by design.
Most notably, the content of a child document always occupies
its own set of pages; pages cannot be shared between child documents.
Usually, this behaviour makes perfect sense
because each child document contain an essential part of the document.
However, in some situations it may be desirable to compose
a document from a collection of parts
without having mandatory page breaks between then.
For this case, the package
provides a mechanism to include parts
by |\input| which can also be processed individually.
However, by construction this mechanism
requires manual handling of the content to be output.

%%%%%%%%%%%%%%%%%%%%%%%%%%%%%%%%%%%%%%%%
\DescribeMacro{\ifchilddocmanual}
The main file should be prepared as usual, see \secref{sec:include}.
However, the document body must make a distinction
between processing of an individual part and of the main document, e.g.:
%
\begin{center}
\begin{tabular}{l}
|\ifchilddocmanual|\\
|\input{\childdocname}|\\
|\||else|\\
\textit{document body with }|\input{|\textit{part}|}|\\
|\||fi|
\end{tabular}
\end{center}
%
The conditional |\ifchilddocmanual| is true whenever
a part to be included by |\input| is being compiled,
and the name of the part is stored in |\childdocname|.

%%%%%%%%%%%%%%%%%%%%%%%%%%%%%%%%%%%%%%%%
\DescribeMacro{\childdocby}
Each part to be included by |\input| should start with:
%
\begin{center}
\begin{tabular}{l}
|\input{childdoc.def}|\\
|\childdocby{|\textit{main}|}|\\
\end{tabular}
\end{center}
%
The directive |\childdocby| is similar to |\childdocof|
described in \secref{sec:include},
but the subsequent selection of content must be done manually.
To that end, both |\ifchilddoc| and |\ifchilddocmanual|
will be true upon processing of a part,
and the name of the part is stored in |\childdocname|.
Note that |\jobname| will be set to the filename of the current part
so that each part receives an individual |.aux| file
that does not interfere with the |.aux| file(s) of the main document.
This behaviour can be altered by the alternative form
|\childdocby[*]{|\textit{main}|}| (with a non-empty optional argument)
which uses the |.aux| file of the main document
by setting |\jobname| to \textit{main}.

%%%%%%%%%%%%%%%%%%%%%%%%%%%%%%%%%%%%%%%%%%%%%%%%%%%%%%%%%%%%%%%%%%%%%%%%%%%%%%%%
\subsection{Driver Development}
\label{sec:driver}

The \textsf{childdoc} mechanism can also be use for the development
of definition files such as \LaTeX{} styles or classes.
This case differs from the above setup with multiple parts
included by |\include| in that no |\includeonly| should be invoked.
This can be achieved by starting the include file
(before |\ProvidesPackage|) with:
%
\begin{center}
\begin{tabular}{l}
|\input{childdoc.def}|\\
|\childdocforward{|\textit{main}|}|\\
\end{tabular}
\end{center}
%
or alternatively with:
%
\begin{center}
\begin{tabular}{l}
|\input{childdoc.def}|\\
|\childdocby{|\textit{main}|}|\\
\end{tabular}
\end{center}
%
Both forms have slightly different effects as described above.
The main file is prepared as usual, see \secref{sec:include}.

%%%%%%%%%%%%%%%%%%%%%%%%%%%%%%%%%%%%%%%%%%%%%%%%%%%%%%%%%%%%%%%%%%%%%%%%%%%%%%%%
\subsection{Legacy Detection}
\label{sec:detection}

The directive |\childdocmain| in the main file can detect
whether the complete document or merely a child is to be compiled
even without using the directive |\childdocof|.
This method is deprecated because it is less robust
and there is no compelling reason to use it;
it is merely provided for backward compatibility
and it may be removed in future versions.

If the detection mechanism is to be used,
it is mandatory to correctly specify
the filename of the main file as the argument of |\childdocmain|:
%
\begin{center}
\begin{tabular}{l}
|\input{childdoc.def}|\\
|\childdocmain{|\textit{main}|}|\\
\end{tabular}
\end{center}
%
If |\jobname| does not match the argument \textit{main} of |\childdocmain|,
it is assumed that |\jobname| points to the child file to be compiled.
When using |\childdocmain| with the main file specified as argument,
it suffices to start a child file
with just |\input{|\textit{main}|}|
without loading of the package and using |\childdocof|.
If instead all processing is done
with the appropriate \textsf{childdoc} directives,
the argument of \textit{main} of |\childdocmain| can be empty.

An alternative version of the command line processing described
in \secref{sec:commandline} using the detection mechanism reads:
%
\begin{center}
|... -jobname "|\textit{target}|" "|[\textit{flags}]%
[|\def\jobname{|\textit{dest}|}|]|\input{|\textit{main}|}"|
\end{center}

%%%%%%%%%%%%%%%%%%%%%%%%%%%%%%%%%%%%%%%%%%%%%%%%%%%%%%%%%%%%%%%%%%%%%%%%%%%%%%%%
\subsection{Manual Code}
\label{sec:manual}

In case one cannot be certain whether the definitions file |childdoc.def|
is installed on the target \TeX{} distribution
and one prefers not to ship it,
it is conceivable to paste a few relevant commands into the sources.

To that end, drop all statements |\input{childdoc.def}|
and perform the replacements as outlined below.
Instead of |\childdocmain{|\textit{main}|}| add the following code
to the top of the main file:
%
\begin{center}
\begin{tabular}{l}
|\||ifdefined\childdocname\endinput\||fi\newif\ifchilddoc|\\
|\edef\childdocname{\scantokens\expandafter{\jobname\noexpand}}|\\
|\def\childdocmain{|\textit{main}|}\||ifx\childdocmain\childdocname\||else|\\
|\childdoctrue\includeonly{\childdocname}\let\jobname\childdocmain\||fi|\\
\end{tabular}
\end{center}
%
Instead of |\childdocof{|\textit{main}|}| just include the main file
at the top of each child file:
%
\begin{center}
|\input{|\textit{main}|}|
\end{center}
%
A simple redirection |\childdocforward{|\textit{dest}|}| is achieved by:
%
\begin{center}
|\def\jobname{|\textit{dest}|}\input{\jobname}|
\end{center}
%
The redirection with prefix
|\childdocforwardprefix[|\textit{prefix}|]{|\textit{dest}|}|
is accomplished by:
%
\begin{center}
\begin{tabular}{l}
|{\edef\jobname{\scantokens\expandafter{\jobname\noexpand}}|\\
|\def\redirectjob |\textit{prefix}|#1~~~{\gdef\jobname{|\textit{dest}|#1}}|\\
|\expandafter\redirectjob\jobname~~~}\input{\jobname}|
\end{tabular}
\end{center}

In an alternative approach,
child documents can be compiled by a specific command line
without additional code or specific definitions:
%
\begin{center}
|... -jobname "|\textit{target}|" "|[\textit{flags}]%
|\includeonly{|\textit{dest}|}\input{|\textit{main}|}"|
\end{center}
%

%%%%%%%%%%%%%%%%%%%%%%%%%%%%%%%%%%%%%%%%%%%%%%%%%%%%%%%%%%%%%%%%%%%%%%%%%%%%%%%%
%%%%%%%%%%%%%%%%%%%%%%%%%%%%%%%%%%%%%%%%%%%%%%%%%%%%%%%%%%%%%%%%%%%%%%%%%%%%%%%%
\section{Information}

%%%%%%%%%%%%%%%%%%%%%%%%%%%%%%%%%%%%%%%%%%%%%%%%%%%%%%%%%%%%%%%%%%%%%%%%%%%%%%%%
\subsection{Copyright}

Copyright \copyright{} 2017--2018 Niklas Beisert

This work may be distributed and/or modified under the
conditions of the \LaTeX{} Project Public License, either version 1.3
of this license or (at your option) any later version.
The latest version of this license is in
  \url{http://www.latex-project.org/lppl.txt}
and version 1.3 or later is part of all distributions of \LaTeX{}
version 2005/12/01 or later.

This work has the LPPL maintenance status `maintained'.

The Current Maintainer of this work is Niklas Beisert.

This work consists of the files |README.txt|, |childdoc.ins| and |childdoc.dtx|
as well as the derived files |childdoc.def|, |cdocsamp.tex|
with |cdocsch1.tex|, |cdocsch2.tex|, |cdocspt3.tex|, |cdocspt4.tex|,
|cdocsdrf.tex|, |cdocsfn1.tex|, |cdocsfn2.tex|
as well as |childdoc.pdf|.

%%%%%%%%%%%%%%%%%%%%%%%%%%%%%%%%%%%%%%%%%%%%%%%%%%%%%%%%%%%%%%%%%%%%%%%%%%%%%%%%
\subsection{Files and Installation}

The package consists of the files:
%
\begin{center}
\begin{tabular}{ll}
    |README.txt|   & readme file \\
    |childdoc.ins| & installation file \\
    |childdoc.dtx| & source file \\
    |childdoc.def| & definition file \\
    |cdocsamp.tex| & sample main file \\
    |cdocsch1.tex| & sample include file \\
    |cdocsch2.tex| & sample include file \\
    |cdocspt3.tex| & sample part file \\
    |cdocspt4.tex| & sample part file \\
    |cdocsdrf.tex| & sample redirection file \\
    |cdocsfn1.tex| & sample redirection file \\
    |cdocsfn2.tex| & sample redirection file \\
    |childdoc.pdf| & manual
\end{tabular}
\end{center}
%
The distribution consists of the files
|README.txt|, |childdoc.ins| and |childdoc.dtx|.
%
\begin{itemize}
\item
Run (pdf)\LaTeX{} on |childdoc.dtx|
to compile the manual |childdoc.pdf| (this file).
\item
Run \LaTeX{} on |childdoc.ins| to create the definitions file |childdoc.def|
and the sample |cdocsamp.tex| with include files
|cdocsch1.tex|, |cdocsch2.tex|, |cdocspt3.tex|, |cdocspt4.tex|,
|cdocsdrf.tex|, |cdocsfn1.tex|, |cdocsfn2.tex|.
Then copy the file |childdoc.def| to an appropriate directory of your \LaTeX{}
distribution, e.g.\ \textit{texmf-root}|/tex/latex/childdoc|.
\end{itemize}

%%%%%%%%%%%%%%%%%%%%%%%%%%%%%%%%%%%%%%%%%%%%%%%%%%%%%%%%%%%%%%%%%%%%%%%%%%%%%%%%
\subsection{Related CTAN Packages}

There are several other packages which offer a similar functionality:
%
\begin{itemize}
\item
The packages
\href{http://ctan.org/pkg/docmute}{\textsf{docmute}},
\href{http://ctan.org/pkg/includex}{\textsf{includex}} and
\href{http://ctan.org/pkg/standalone}{\textsf{standalone}}
provide commands to include only the document body of
a child file thus allowing both files to be compiled individually.
\item
The packages \href{http://ctan.org/pkg/subdocs}{\textsf{subdocs}}
and \href{http://ctan.org/pkg/subfiles}{\textsf{subfiles}}
provide structures in which the main and child documents can be
encapsulated and allowing them to be compiled individually.
The inclusion mechanism is different from the conventional |\include|.
\item
The package \href{http://ctan.org/pkg/combine}{\textsf{combine}}
is an elaborate solution to combine several documents into one.
\end{itemize}
%
See also the CTAN topic \href{http://ctan.org/topic/subdocs}{\textsf{subdocs}}
for further related packages.
The present package differs from the above solutions in that
a document structure constructed with the conventional |\include| mechanism
just needs two extra commands at the top of every file
such that all constituent files can be compiled individually.

%%%%%%%%%%%%%%%%%%%%%%%%%%%%%%%%%%%%%%%%%%%%%%%%%%%%%%%%%%%%%%%%%%%%%%%%%%%%%%%%
%\subsection{Feature Suggestions}
%
%The following is a list of features which may be useful for future
%versions of this package:
%%
%\begin{itemize}
%\item
%\ldots
%\end{itemize}

%%%%%%%%%%%%%%%%%%%%%%%%%%%%%%%%%%%%%%%%%%%%%%%%%%%%%%%%%%%%%%%%%%%%%%%%%%%%%%%%
\subsection{Revision History}

%%%%%%%%%%%%%%%%%%%%%%%%%%%%%%%%%%%%%%%%
\paragraph{v2.0:} 2018/12/30

\begin{itemize}
\item
immediate forward processing
\item
added |\childdocby| mechanism
\item
manual restructured
\end{itemize}

%%%%%%%%%%%%%%%%%%%%%%%%%%%%%%%%%%%%%%%%
\paragraph{v1.6:} 2018/01/17

\begin{itemize}
\item
application for development of include files
\item
corrections to manual
\end{itemize}

%%%%%%%%%%%%%%%%%%%%%%%%%%%%%%%%%%%%%%%%
\paragraph{v1.5:} 2017/05/21

\begin{itemize}
\item
more complete structuring introduced
\item
|\childdocof| introduced
\item
|\childdoc| renamed to |\childdocmain|
\item
|\childredirect| renamed to |\childdocforward| and |\childdocforwardprefix|
and functionality expanded
\end{itemize}

%%%%%%%%%%%%%%%%%%%%%%%%%%%%%%%%%%%%%%%%
\paragraph{v1.0:} 2017/04/27

\begin{itemize}
\item
manual and install package
\item
first version published on CTAN
\end{itemize}

%%%%%%%%%%%%%%%%%%%%%%%%%%%%%%%%%%%%%%%%
\paragraph{v0.6:} 2017/04/26

\begin{itemize}
\item
redirection mechanism added
\end{itemize}

%%%%%%%%%%%%%%%%%%%%%%%%%%%%%%%%%%%%%%%%
\paragraph{v0.5:} 2017/04/26

\begin{itemize}
\item
functionality in definition file
\end{itemize}


%%%%%%%%%%%%%%%%%%%%%%%%%%%%%%%%%%%%%%%%%%%%%%%%%%%%%%%%%%%%%%%%%%%%%%%%%%%%%%%%
%%%%%%%%%%%%%%%%%%%%%%%%%%%%%%%%%%%%%%%%%%%%%%%%%%%%%%%%%%%%%%%%%%%%%%%%%%%%%%%%
%%%%%%%%%%%%%%%%%%%%%%%%%%%%%%%%%%%%%%%%%%%%%%%%%%%%%%%%%%%%%%%%%%%%%%%%%%%%%%%%
\appendix

\settowidth\MacroIndent{\rmfamily\scriptsize 000\ }

 \DocInput{childdoc.dtx}

\end{document}
%</driver>
% \fi
%
% %%%%%%%%%%%%%%%%%%%%%%%%%%%%%%%%%%%%%%%%%%%%%%%%%%%%%%%%%%%%%%%%%%%%%%%%%%%%%%
% %%%%%%%%%%%%%%%%%%%%%%%%%%%%%%%%%%%%%%%%%%%%%%%%%%%%%%%%%%%%%%%%%%%%%%%%%%%%%%
% \section{Sample}
%\iffalse
%<*samplemain>
%\fi
%
% The following presents a sample document
% with two chapters, two parts, a title page,
% a compile flag as well as three forwarding files to set the flag.
% It consists of eight |.tex| files:
% \begin{center}
% \begin{tabular}{ll}
% |cdocsamp.tex|&main file\\
% |cdocsch1.tex|&include file for chapter 1\\
% |cdocsch2.tex|&include file for chapter 2\\
% |cdocspt3.tex|&include file for part 3\\
% |cdocspt4.tex|&include file for part 4\\
% |cdocsdrf.tex|&forwarding file for main file in draft mode\\
% |cdocsfi1.tex|&forwarding file for final version of chapter 1\\
% |cdocsfi2.tex|&forwarding file for final version of chapter 2\\
% \end{tabular}
% \end{center}
% Each of the eight files can be compiled directly by the \LaTeX{} compiler.
%
% %%%%%%%%%%%%%%%%%%%%%%%%%%%%%%%%%%%%%%
% \paragraph{Main File.}
%
% The main file is called |cdocsamp.tex|.
%
% Load the \textsf{childdoc} definitions and
% declare the filename for the main document:
%    \begin{macrocode}
\input{childdoc.def}
\childdocmain{}
%    \end{macrocode}

% Optional override for |\version| flag:
%    \begin{macrocode}
%%\ifchilddoc\else\providecommand{\version}{draft}\fi
%    \end{macrocode}

% Define the default values for the |\version| flag
% (|final| for the main file and |draft| for childs):
%    \begin{macrocode}
\ifchilddoc
\providecommand{\version}{draft}
\else
\providecommand{\version}{final}
\fi
%    \end{macrocode}

% Load the standard document class:
%    \begin{macrocode}
\documentclass[12pt]{article}
%    \end{macrocode}

% Start the document body:
%    \begin{macrocode}
\begin{document}
%    \end{macrocode}

% Declare a title page.
% Print title, part of document being processed and version flag:
%    \begin{macrocode}
\addtocounter{page}{-1}
\begin{center}
{\LARGE\bfseries{}childdoc example\par}
\vspace{1cm}
\ifchilddoc
\ifchilddocmanual part\else chapter\fi:
`\childdocname' of `\childdocjob'\par
\else
main document: `\childdocjob'\par
\fi
version: \version\par
\end{center}
\newpage
%    \end{macrocode}

% Manually include selected file,
% otherwise process as usual:
%    \begin{macrocode}
\ifchilddocmanual
\section*{part `\childdocname'}
\input{\childdocname}
\else
%    \end{macrocode}

% Include the two chapters:
%    \begin{macrocode}
\include{cdocsch1}
\include{cdocsch2}
%    \end{macrocode}

% Include the two parts unless only chapters should be displayed:
%    \begin{macrocode}
\ifchilddoc\else
\section{part three}
\input{cdocspt3}
\section{part four}
\input{cdocspt4}
\fi
%    \end{macrocode}

% Process as usual until here:
%    \begin{macrocode}
\fi
%    \end{macrocode}

% End of document body:
%    \begin{macrocode}
\end{document}
%    \end{macrocode}
%\iffalse
%</samplemain>
%\fi
%
% %%%%%%%%%%%%%%%%%%%%%%%%%%%%%%%%%%%%%%
% \paragraph{Chapter Include Files.}
%
% The include files are called |cdocsch1.tex| and |cdocsch2.tex|.
%
%\iffalse
%<*samplechap1|samplechap2>
%\fi

% Optional override for |\version| flag:
%    \begin{macrocode}
%%\providecommand{\version}{final}
%    \end{macrocode}

% Include the main document:
%    \begin{macrocode}
\input{childdoc.def}
\childdocof{cdocsamp}
%    \end{macrocode}

%\iffalse
%</samplechap1|samplechap2>
%\fi
%
%\iffalse
%<*samplechap1>
%\fi
% Some text for chapter 1:
%    \begin{macrocode}
\section{one}
some text in chapter one
%    \end{macrocode}

%\iffalse
%</samplechap1>
%\fi
% Some text for chapter 2:
%\iffalse
%<*samplechap2>
%\fi
%    \begin{macrocode}
\section{two}
more text in chapter two
%    \end{macrocode}

%\iffalse
%</samplechap2>
%\fi
%
% %%%%%%%%%%%%%%%%%%%%%%%%%%%%%%%%%%%%%%
% \paragraph{Part Include Files.}
%
% The include files are called |cdocspt3.tex| and |cdocspt4.tex|.
%
%\iffalse
%<*samplepart3|samplepart4>
%\fi

% Optional override for |\version| flag:
%    \begin{macrocode}
%%\providecommand{\version}{final}
%    \end{macrocode}

% Include the main document:
%    \begin{macrocode}
\input{childdoc.def}
\childdocby{cdocsamp}
%    \end{macrocode}

%\iffalse
%</samplepart3|samplepart4>
%\fi
%
%\iffalse
%<*samplepart3>
%\fi
% Some text for part 3:
%    \begin{macrocode}
some text in part three
%    \end{macrocode}

%\iffalse
%</samplepart3>
%\fi
% Some text for part 4:
%\iffalse
%<*samplepart4>
%\fi
%    \begin{macrocode}
more text in part four
%    \end{macrocode}

%\iffalse
%</samplepart4>
%\fi
%
% %%%%%%%%%%%%%%%%%%%%%%%%%%%%%%%%%%%%%%
% \paragraph{Forwarding for a Complete Draft.}
%
% The following forwarding file |cdocsdrf.tex|
% compiles the main document in draft mode:
%\iffalse
%<*sampledraft>
%\fi
%    \begin{macrocode}
\def\version{draft}
\input{childdoc.def}
\childdocforward{cdocsamp}
%    \end{macrocode}

%\iffalse
%</sampledraft>
%\fi
%
% %%%%%%%%%%%%%%%%%%%%%%%%%%%%%%%%%%%%%%
% \paragraph{Forwarding for Final Version of the Chapters.}
%
% The following forwarding files |cdocsfn1.tex| and |cdocsfn2.tex|
% (with identical content)
% compile the final versions of the child documents
% |cdocsch1.tex| and |cdocsch2.tex|, respectively:
%\iffalse
%<*samplefinal>
%\fi
%    \begin{macrocode}
\def\version{final}
\input{childdoc.def}
\childdocforwardprefix[cdocsamp]{cdocsfn}{cdocsch}
%    \end{macrocode}

%\iffalse
%</samplefinal>
%\fi
%
% %%%%%%%%%%%%%%%%%%%%%%%%%%%%%%%%%%%%%%
% \paragraph{Command Line Processing.}
%
% The following three command lines generate the output files
% |cdocscld|, |cdocscl1| and |cdocscl2|
% which should be identical to
% |cdocsdrf|, |cdocsch1| and |cdocsfn2|, respectively:
% \begin{center}
% \begin{tabular}{l}
% |latex -jobname cdocscld \|\\
% |  "\def\version{draft}\input{childdoc.def}\childdocforward{cdocsamp}"|\\
% |latex -jobname cdocscl1 \|\\
% |  "\input{childdoc.def}\childdocforward[cdocsamp]{cdocsch1}"|\\
% |latex -jobname cdocscl2 \|\\
% |  "\def\version{final}\input{childdoc.def}\childdocforward{cdocsch2}"|
% \end{tabular}
% \end{center}
% Note that the trailing backslash on each first line
% merely continues the input to the second line
% (for convenient cut ant paste).
% Furthermore, the command |latex| can be replaced by any
% of its alternative versions such as |pdflatex|.
%
% %%%%%%%%%%%%%%%%%%%%%%%%%%%%%%%%%%%%%%%%%%%%%%%%%%%%%%%%%%%%%%%%%%%%%%%%%%%%%%
% %%%%%%%%%%%%%%%%%%%%%%%%%%%%%%%%%%%%%%%%%%%%%%%%%%%%%%%%%%%%%%%%%%%%%%%%%%%%%%
% \section{Implementation}
%\iffalse
%<*package>
%\fi
%
% This section describes the definitions file |childdoc.def|.

% The definitions cannot be loaded using |\usepackage| or |\RequirePackage|
% which has a mechanism to prevent loading a style file more than once.
% When loading the definitions by means of |\input|
% multiple instances have to be prevented manually:
%\iffalse
%This code needs to be before the `\ProvidesFile' directive
%which is defined at the beginning of this file.
%Therefore it is also placed there and commented out here.
%</package>
%<*discard>
%\fi
%    \begin{macrocode}
\ifdefined\childdocmain\endinput\fi
%    \end{macrocode}
%\iffalse
%</discard>
%<*package>
%\fi
%
% \macro{\ifchilddoc}
% \macro{\ifchilddocmanual}
% The conditional |\ifchilddoc| tells whether a
% child (true) or main (false) document is being compiled.
% The conditional |\ifchilddocmanual| tells whether
% the |\includeonly| mechanism is used (false) or
% the selection of child files must be performed manually (true).
% The definitions initialise to false:
%    \begin{macrocode}
\newif\ifchilddoc
\newif\ifchilddocmanual
%    \end{macrocode}

% \macro{\childdocname}
% \macro{\childdocjob}
% The macro |\childdocname| stores the name of the main document
% to be compiled. The macro |\childdocjob| stores the name of
% the document on which the \LaTeX{} compiler was originally invoked.
% The content of |\jobname| cannot be compared
% to filenames specified in the source due to different catcodes.
% The following code rescans |\jobname|, stores the result
% in |\childdocname| and saves a copy in |\childdocjob|:
%    \begin{macrocode}
\edef\childdocname{\scantokens\expandafter{\jobname\noexpand}}
\let\childdocjob\childdocname
%    \end{macrocode}

% \macro{\childdocdisable}
% The macro |\childdocdisable| prevents the main file
% from being processed more than once.
% At this stage, the main document command |\childdocmain|
% is assumed to be called once again where it should do nothing.
% Any subsequent call to it should prevent
% a secondary processing of the main document
% It overwrites the forwarding commands
% |\childdocof| and |\childdocforward|
% with empty macros to prevent further inclusions of the main document:
%    \begin{macrocode}
\newcommand{\childdocdisable}
{
  \renewcommand{\childdocmain}[1]{\renewcommand{\childdocmain}[1]{\endinput}}
  \renewcommand{\childdocof}[1]{}
  \renewcommand{\childdocby}[2][]{}
  \renewcommand{\childdocforward}[2][]{}
  \renewcommand{\childdocdisable}{}
}
%    \end{macrocode}

% \macro{\childdocmain}
% The macro |\childdocmain| is to be called at the top of the main file
% with nothing or the main filename (without extension) as argument.
% First, it breaks loops.
% If the argument is not empty and does not match |\childdocname|
% (which is set by the first inclusion of |childdoc.def|),
% |\ifchilddoc| is set to true, |\includeonly| is applied to the child file
% and |\jobname| is set to the main file
% (for proper handling of |.aux| files):
%    \begin{macrocode}
\newcommand{\childdocmain}[1]
{
  \childdocdisable\childdocmain{}
  \if?#1?\else
    \begingroup
      \def\childdoctmp{#1}
      \ifx\childdoctmp\childdocname
        \def\childdoctmp{}
      \else
        \def\childdoctmp
        {
          \childdoctrue
          \includeonly{\childdocname}
          \def\childdocjob{#1}
          \def\jobname{#1}
        }
      \fi
      \expandafter
    \endgroup
    \childdoctmp
  \fi
}
%    \end{macrocode}

% \macro{\childdocof}
% The command |\childdocof| redirects
% compilation to the main file |#1|.
%    \begin{macrocode}
\newcommand{\childdocof}[1]
{
  \childdocdisable
  \childdoctrue
  \includeonly{\childdocname}
  \def\jobname{#1}
  \def\childdocjob{#1}
  \input{#1}
}
%    \end{macrocode}

% \macro{\childdocby}
% The command |\childdocby| ....
%    \begin{macrocode}
\newcommand{\childdocby}[2][]
{
  \childdocdisable
  \childdoctrue
  \childdocmanualtrue
  \if?#1?\else
    \def\jobname{#2}
  \fi
  \def\childdocjob{#2}
  \input{#2}
  \endinput
}
%    \end{macrocode}

% \macro{\childdocforward}
% The command |\childdocforward| redirects
% compilation to the main file or
% (if the optional argument is given) a child file.
% Parameters are set as if the main file
% or a child file starting with |\childdocof| was compiled.
% Then compilation is handed over to the main file:
%    \begin{macrocode}
\newcommand{\childdocforward}[2][]
{
  \begingroup
    \if?#1?
      \def\childdoctmp
      {
        \def\childdocname{#2}
        \def\childdocjob{#2}
        \def\jobname{#2}
        \input{#2}
        \endinput
      }
    \else
      \def\childdoctmp
      {
        \childdocdisable
        \def\childdocname{#2}
        \childdoctrue
        \includeonly{#2}
        \def\childdocjob{#1}
        \def\jobname{#1}
        \input{#1}
        \endinput
      }
    \fi
    \expandafter
  \endgroup
  \childdoctmp
}
%    \end{macrocode}

% \macro{\childdocforwardprefix}
% The command |\childdocforwardprefix| redirects
% compilation to the main or a child file by means of a pattern.
% The prefix |#1| in the current filename is replaced by |#2|
% and the suffix of the current filename is kept
% (it is assumed that the filename does not contain the substring `|~~~|'
% which is used as a delimiter).
% Compilation is handed over to the new file by |\childdocforward|:
%    \begin{macrocode}
\newcommand{\childdocforwardprefix}[3][]
{
  \begingroup
    \def\childdocextract #2##1~~~{\def\childdoctmp{\childdocforward[#1]{#3##1}}}
    \expandafter\childdocextract\childdocname~~~
    \expandafter
  \endgroup
  \childdoctmp
}
%    \end{macrocode}

% \macro{\childdoc}
% The deprecated macro |\childdoc| is a legacy version of |\childdocmain|:
%    \begin{macrocode}
\newcommand{\childdoc}{\childdocmain}
%    \end{macrocode}

% \macro{\childdocredirect}
% The deprecated macro |\childdocredirect| is a legacy version
% of |\childdocforward| and |\childdocforwardprefix|:
%    \begin{macrocode}
\newcommand{\childdocredirect}[2][]
{
  \begingroup
    \if?#1?
      \def\childdoctmp{\childdocforward{#2}}
    \else
      \def\childdoctmp{\childdocforwardprefix{#1}{#2}}
    \fi
    \expandafter
  \endgroup
  \childdoctmp
}
%    \end{macrocode}

%\iffalse
%</package>
%\fi
%
\endinput

\childdocby{cdocsamp}
%    \end{macrocode}

%\iffalse
%</samplepart3|samplepart4>
%\fi
%
%\iffalse
%<*samplepart3>
%\fi
% Some text for part 3:
%    \begin{macrocode}
some text in part three
%    \end{macrocode}

%\iffalse
%</samplepart3>
%\fi
% Some text for part 4:
%\iffalse
%<*samplepart4>
%\fi
%    \begin{macrocode}
more text in part four
%    \end{macrocode}

%\iffalse
%</samplepart4>
%\fi
%
% %%%%%%%%%%%%%%%%%%%%%%%%%%%%%%%%%%%%%%
% \paragraph{Forwarding for a Complete Draft.}
%
% The following forwarding file |cdocsdrf.tex|
% compiles the main document in draft mode:
%\iffalse
%<*sampledraft>
%\fi
%    \begin{macrocode}
\def\version{draft}
% \iffalse
%
% childdoc.dtx Copyright (C) 2017-2018 Niklas Beisert
%
% This work may be distributed and/or modified under the
% conditions of the LaTeX Project Public License, either version 1.3
% of this license or (at your option) any later version.
% The latest version of this license is in
%   http://www.latex-project.org/lppl.txt
% and version 1.3 or later is part of all distributions of LaTeX
% version 2005/12/01 or later.
%
% This work has the LPPL maintenance status `maintained'.
%
% The Current Maintainer of this work is Niklas Beisert.
%
% This work consists of the files childdoc.dtx and childdoc.ins
% and the derived files childdoc.def and cdocsamp.tex with
% cdocsch1.tex, cdocsch2.tex, cdocsdrf.tex, cdocsfn1.tex, cdocsfn2.tex.
%
%<package>\ifdefined\childdocmain\endinput\fi
%<package>\ProvidesFile{childdoc.def}[2018/12/30 v2.0 child document driver]
%<samplemain>\ProvidesFile{cdocsamp.tex}[2018/12/30 v2.0 sample for childdoc]
%<*driver>
%\ProvidesFile{childdoc.drv}[2018/12/30 v2.0 childdoc reference manual file]
\PassOptionsToClass{10pt,a4paper}{article}
\documentclass{ltxdoc}

\usepackage[margin=35mm]{geometry}
\usepackage{hyperref}
\usepackage{hyperxmp}
\usepackage[usenames]{color}

\hypersetup{colorlinks=true}
\hypersetup{pdfstartview=FitH}
\hypersetup{pdfpagemode=UseNone}
\hypersetup{pdfsource={}}
\hypersetup{pdflang={en-UK}}
\hypersetup{pdfcopyright={Copyright 2017-2018 Niklas Beisert.
  This work may be distributed and/or modified under the
  conditions of the LaTeX Project Public License, either version 1.3
  of this license or (at your option) any later version.}}
\hypersetup{pdflicenseurl={http://www.latex-project.org/lppl.txt}}
\hypersetup{pdfcontactaddress={ETH Zurich, ITP, HIT K,
  Wolfgang-Pauli-Strasse 27}}
\hypersetup{pdfcontactpostcode={8093}}
\hypersetup{pdfcontactcity={Zurich}}
\hypersetup{pdfcontactcountry={Switzerland}}
\hypersetup{pdfcontactemail={nbeisert@itp.phys.ethz.ch}}
\hypersetup{pdfcontacturl={http://people.phys.ethz.ch/\xmptilde nbeisert/}}

\newcommand{\secref}[1]{\hyperref[#1]{section \ref*{#1}}}

\parskip1ex
\parindent0pt
\let\olditemize\itemize
\def\itemize{\olditemize\parskip0pt}

\begin{document}

\title{The \textsf{childdoc} Package}
\hypersetup{pdftitle={The childdoc Package}}
\author{Niklas Beisert\\[2ex]
  Institut f\"ur Theoretische Physik\\
  Eidgen\"ossische Technische Hochschule Z\"urich\\
  Wolfgang-Pauli-Strasse 27, 8093 Z\"urich, Switzerland\\[1ex]
  \href{mailto:nbeisert@itp.phys.ethz.ch}
  {\texttt{nbeisert@itp.phys.ethz.ch}}}
\hypersetup{pdfauthor={Niklas Beisert}}
\hypersetup{pdfsubject={Manual for the LaTeX2e Package childdoc}}
\date{30 December 2018, \textsf{v2.0}}
\maketitle

\begin{abstract}\noindent
\textsf{childdoc} is a \LaTeXe{} package
that enables the direct compilation
of document sections included by |\include|
to individual files.
\end{abstract}

\begingroup
\parskip0ex
\tableofcontents
\endgroup

%%%%%%%%%%%%%%%%%%%%%%%%%%%%%%%%%%%%%%%%%%%%%%%%%%%%%%%%%%%%%%%%%%%%%%%%%%%%%%%%
%%%%%%%%%%%%%%%%%%%%%%%%%%%%%%%%%%%%%%%%%%%%%%%%%%%%%%%%%%%%%%%%%%%%%%%%%%%%%%%%
\section{Introduction}

\LaTeX{} provides a mechanism to structure a large document (such as a book)
into a main file and several child files (containing the chapters)
using the |\include| command.
This mechanism is beneficial for documents
which span hundreds of pages in order to
make the source file(s) more manageable.
Moreover, compilation can be restricted to
selected child files by means of the |\includeonly| command.
The latter feature can be used to reduce the compilation time while editing
(this was significantly more useful in the earlier days of \LaTeX{})
or to generate a smaller document which is easier to navigate.
Another application of |\includeonly| is to generate
documents consisting of selected parts of the complete document.

However, there are a few drawbacks of the plain |\include| mechanism:
\begin{itemize}
\item
The child files cannot be compiled on their own,
they can only be compiled via the main file.
A naive editing environment
(such as a text editor with an option
to have the current file processed by \LaTeX)
may require one to switch to the main file before compiling;
attempting to compile the child file produces errors.
\item
The main file must be modified (each time)
to adjust the |\includeonly| command
to the present needs. This easily leaves the main file in a messy state.
\item
The generated document will always carry the filename
of the main document. This is inconvenient if
several child files are to be compiled and
to be kept for distribution.
\end{itemize}

The present package provides a simple interface
to make child files individually compilable by \LaTeX{}.
Compiling a child file then has the same effect as compiling
the main file with an |\includeonly| command
to select the appropriate child.
Moreover the generated document will carry the name of the child
rather than the main file.
This resolves all three above issues.

This feature is meant to make the editing of books,
thesis documents and lecture notes somewhat more convenient.
However, the package can also be used efficiently for
composing a series of documents (such as exercise sheets)
which are typically distributed individually.
It then assists the author in generating the individual documents
(potentially in different versions)
as well as a document containing the collected series.
Another application is in developing style files
or other kinds of included material
where compilation of the style file could redirect
to a sample or test file.

%%%%%%%%%%%%%%%%%%%%%%%%%%%%%%%%%%%%%%%%%%%%%%%%%%%%%%%%%%%%%%%%%%%%%%%%%%%%%%%%
%%%%%%%%%%%%%%%%%%%%%%%%%%%%%%%%%%%%%%%%%%%%%%%%%%%%%%%%%%%%%%%%%%%%%%%%%%%%%%%%
\section{Usage}

First of all, the package \textsf{childdoc} is \emph{not} a standard
\LaTeXe{} |.sty| style file! Therefore it needs to be invoked in
a non-standard way.

%%%%%%%%%%%%%%%%%%%%%%%%%%%%%%%%%%%%%%%%%%%%%%%%%%%%%%%%%%%%%%%%%%%%%%%%%%%%%%%%
\subsection{Included Files}
\label{sec:include}

%%%%%%%%%%%%%%%%%%%%%%%%%%%%%%%%%%%%%%%%
\DescribeMacro{\childdocmain}
To use the package, add the commands
\begin{center}
\begin{tabular}{l}
|\input{childdoc.def}|\\
|\childdocmain{}|\\
\end{tabular}
\end{center}
at the very top of the main \LaTeX{} file,
in particular \emph{before} the |\documentclass| statement!
The argument of |\childdocmain| should be left empty
(but it must be present).

%%%%%%%%%%%%%%%%%%%%%%%%%%%%%%%%%%%%%%%%
\DescribeMacro{\childdocof}
Furthermore, add the commands
\begin{center}
\begin{tabular}{l}
|\input{childdoc.def}|\\
|\childdocof{|\textit{main}|}|\\
\end{tabular}
\end{center}
at the top of every child file \textit{child}
which is included by |\include{|\textit{child}|}|
from within the main file
(or at least for those files to be compiled individually).
The argument \textit{main} must be the filename of the main file.

There are a couple of
considerations in setting up the main and child documents:

%%%%%%%%%%%%%%%%%%%%%%%%%%%%%%%%%%%%%%%%
\paragraph{Restrictions.}

Please note the following restrictions:
\begin{itemize}
\item
|\childdocmain| must be called with one argument \textit{main}
to ensure compatibility with earlier version of the package.
It must either be empty (|\childdocmain{}|)
or precisely match the filename of the main file in which it is specified.
See \secref{sec:detection} for further information.
\item
The filename \textit{main} must be specified without the |.tex| extension.
\item
The filename \textit{main} is case sensitive
(even in case-insensitive file systems)
due to internal string comparison.
\item
The argument \textit{main} should be fully expanded, it cannot be a macro.
\item
Subdirectories and special characters should be avoided in filenames.
\item
The command |\childdocmain{|\textit{main}|}| must be followed by a whitespace.
It should not be followed immediately by another command
or by a comment mark `|%|'.
This is because the \TeX{} parser reads the token immediately following
the argument of |\childdocmain| and puts it
at the beginning of every child section;
however, a white\-space is ignored.
\end{itemize}

%%%%%%%%%%%%%%%%%%%%%%%%%%%%%%%%%%%%%%%%
\paragraph{Content of Main File.}

It is advisable to place all content in the child files included by |\include|.
Any output contained in the main file will appear in all child documents
unless suppressed manually;
it cannot be suppressed automatically by the |\includeonly| directive
and thus should normally be avoided.
A method to include some content in the main file
by means of conditional processing is described in \secref{sec:conditional}.

%%%%%%%%%%%%%%%%%%%%%%%%%%%%%%%%%%%%%%%%
\paragraph{Page Numbering.}

When only a part of the document is compiled,
the appropriate numbering of pages
(as well as other status parameters)
is determined from the |.aux| files.
The latter contain information from previous passes.
However this information needs to propagate through
all intermediate child documents.
Therefore the page numbering in child documents may well
be inconsistent until the complete document is compiled at least once.

A useful (if unconventional) way to always ensure a consistent
page numbering is to restart the numbering in each child document
and denote the pages by `\textit{child}|.|\textit{page}'
where \textit{child} represents the chapter/section number of the child file.
This can be achieved by the command
|\numberwithin{page}{|\textit{child}|}|
of the \textsf{amsmath} package
where \textit{child} can be |chapter| or |section|
depending on the chosen structuring.
Alternatively, one can modify the macro |\thepage| appropriately
and reset the counter |page| at the start of each child file.

%%%%%%%%%%%%%%%%%%%%%%%%%%%%%%%%%%%%%%%%%%%%%%%%%%%%%%%%%%%%%%%%%%%%%%%%%%%%%%%%
\subsection{Conditional Processing}
\label{sec:conditional}

The package provides a mechanism to compile different versions
of a document. To customise the versions further some conditional processing
can come in handy to distinguish which version is being compiled.
The package provides two macros to describe the compilation context:

%%%%%%%%%%%%%%%%%%%%%%%%%%%%%%%%%%%%%%%%
\DescribeMacro{\ifchilddoc}
The conditional |\ifchilddoc| distinguishes between the compilation of
child documents and the main document:
%
\begin{center}
|\ifchilddoc |\textit{child-code}| |[|\||else |\textit{main-code}]| \||fi|
\end{center}

%%%%%%%%%%%%%%%%%%%%%%%%%%%%%%%%%%%%%%%%
\DescribeMacro{\childdocname}
\DescribeMacro{\childdocjob}
The macro |\childdocname| contains the filename (without extension)
of the main or child file being processed.
Note that |\childdocjob| will always contain the name of the main file.

%%%%%%%%%%%%%%%%%%%%%%%%%%%%%%%%%%%%%%%%
\paragraph{Title Page.}

Conditional processing can be used to include a title or banner page
in the main document when proper precautions are taken.
Importantly, the code in the main file should ensure that the page counter
(as well as other status parameters which are stored in the |.aux| files)
takes the same value after the conditional processing.
Otherwise the page numbers may take divergent values
depending on which part is compiled.

For example, a title page could be declared by:
%
\begin{center}
\begin{tabular}{l}
|\ifchilddoc\||else|\\
|\addtocounter{page}{-1}|\\
\textit{code for title page}\\
|\newpage|\\
|\||fi|
\end{tabular}
\end{center}
%
A banner page for the child documents can be generated by:
%
\begin{center}
\begin{tabular}{l}
|\ifchilddoc|\\
|\addtocounter{page}{-1}|\\
\textit{code for banner page}\\
|\newpage|\\
|\||fi|
\end{tabular}
\end{center}
%
Here one could write a message such as:
\begin{center}
|This is the part \childdocname{} of \childdocjob{}.|
\end{center}

%%%%%%%%%%%%%%%%%%%%%%%%%%%%%%%%%%%%%%%%%%%%%%%%%%%%%%%%%%%%%%%%%%%%%%%%%%%%%%%%
\subsection{Flags}
\label{sec:flags}

The package makes it easy to generate different versions
of the main or child documents.
To this end compilation flags can be defined
and assigned different default values.
They will be particularly useful in conjunction
with the forwarding mechanism described in \secref{sec:forward}.

For example, it may be useful to have a flag |\version|
which can be set to |draft| or |final|.
The document source will contain some conditional code
depending on the value of |\version|.
Suppose further, the flag should default to |final| for the main file
and to |draft| for child files
which is a natural assignment for editing the document.
This is achieved by placing the following code
in the preamble of the main document
(below the |\childdocmain| directive):
%
\begin{center}
\begin{tabular}{l}
|\ifchilddoc|\\
|\providecommand{\version}{draft}|\\
|\||else|\\
|\providecommand{\version}{final}|\\
|\||fi|
\end{tabular}
\end{center}
%
The definition by |\providecommand| makes sure
that previous definitions are not overwritten.
Further statements |\providecommand{\version}{...}|
can thus be added before the above code to override it.

For the main file, one might add a line
(between |\childdocmain| and the above block)
%
\begin{center}
|%\ifchilddoc\||else\providecommand{\version}{draft}\||fi|
\end{center}
%
which can be uncommented to produce a draft version.
Likewise one can add a line to the very top of a child file
(above the |\childdocof{|\textit{main}|}| directive)
%
\begin{center}
|%\providecommand{\version}{final}|
\end{center}
%
which can be uncommented to produce the final version of this child document.

%%%%%%%%%%%%%%%%%%%%%%%%%%%%%%%%%%%%%%%%%%%%%%%%%%%%%%%%%%%%%%%%%%%%%%%%%%%%%%%%
\subsection{Forwarding}
\label{sec:forward}

Different versions of the main or child documents
using compilation flags as described in \secref{sec:flags}
can be (permanently) stored in different files
for convenient compilation, viewing and distribution.
To this end, the package defines a command
to pass on compilation to a different file:

%%%%%%%%%%%%%%%%%%%%%%%%%%%%%%%%%%%%%%%%
\DescribeMacro{\childdocforward}
The command |\childdocforward| redirects processing to
another source file:
%
\begin{center}
\begin{tabular}{l}
|\input{childdoc.def}|\\
|\childdocforward[|\textit{main}|]{|\textit{dest}|}|\\
\end{tabular}
\end{center}
%
The argument \textit{dest} is the destination file
(without extension).
It should be the main file or one of the child files.
Note that further \textsf{childdoc} directives
such as |\childdocof| and |\childdocforward|
in the indicated file will be processed in this form.
The optional argument \textit{main}
passes on directly to the main file \textit{main}
while pretending to compile the child \textit{dest}.
This form behaves as if \textit{dest}
issues |\childdocof{|\textit{main}|}| right away,
and no further \textsf{childdoc} directives will be processed.

%%%%%%%%%%%%%%%%%%%%%%%%%%%%%%%%%%%%%%%%
\DescribeMacro{\...prefix}
In the alternative form |\childdocforwardprefix|,
%
\begin{center}
\begin{tabular}{l}
|\input{childdoc.def}|\\
|\childdocforwardprefix[|\textit{main}|]{|\textit{prefix}|}{|\textit{dest}|}|
\end{tabular}
\end{center}
%
the destination file is determined by a pattern
depending on the current file:
To make this work, the current file must be called
`{\textit{prefix}\hspace{0.2em}\textit{suffix}}'
with \textit{prefix} matching precisely the argument.
Processing is then passed on to the file
`{\textit{dest}\hspace{0.2em}\textit{suffix}}'.
Surely, the same effect is achieved by
directly specifying the
argument `{\textit{dest}\hspace{0.2em}\textit{suffix}}'
in the first form.
However, that requires to set up a different file
for each child. With the alternative form of the command
all these files can have exactly the same content
which simplifies setting them up and maintaining them.

For example, the following file |draft.tex|
with a compilation flag |\version| as described in \secref{sec:flags}
compiles the main document as a draft:
%
\begin{center}
\begin{tabular}{l}
|\def\version{draft}|\\
|\input{childdoc.def}|\\
|\childdocforward{|\textit{main}|}|
\end{tabular}
\end{center}
%
Likewise, the following files |final|\textit{nn}|.tex|
compile the final version of the child document
|child|\textit{nn}|.tex|:
%
\begin{center}
\begin{tabular}{l}
|\def\version{final}|\\
|\input{childdoc.def}|\\
|\childdocforwardprefix{final}{child}|
\end{tabular}
\end{center}
%

Note that when several versions of a main file and/or of each child file
are to be generated, it may be convenient to set up a |Makefile| or
shell script to automatise the process.

%%%%%%%%%%%%%%%%%%%%%%%%%%%%%%%%%%%%%%%%%%%%%%%%%%%%%%%%%%%%%%%%%%%%%%%%%%%%%%%%
\subsection{Command Line Processing}
\label{sec:commandline}

The effect of redirection files can also be achieved by invoking
the \LaTeX{} compiler with a more elaborate command line.
Most conveniently this should be done as part
of a shell script or a |Makefile|.

When using \textsf{childdoc} in the main file, the following
command lines effectively perform a redirection
(note that depending on the shell being used,
backslashes may have to be doubled: `|\|' $\to$ `|\\|'):
%
\begin{center}
|... -jobname "|\textit{target}|" |\\|"|[\textit{flags}]%
|\input{childdoc.def}\childdocforward[|\textit{main}|]{|\textit{dest}|}"|
\end{center}
%
Here \textit{target} is the name of the output file,
\textit{main} is the name of the main file
and \textit{dest} is the name of the main or child file to be processed
(all filenames without extensions).
The optional argument \textit{main} can be omitted
if \textit{main} matches \textit{dest}.
Optionally, compilation \textit{flags} can be defined via |\def| commands.
This command line makes the \TeX{} engine believe
it is compiling the file \textit{target}
whose content is specified as the latter parameter.
The provided code then forwards the processing to
\textit{main} or \textit{dest} as described in \secref{sec:forward}.

%%%%%%%%%%%%%%%%%%%%%%%%%%%%%%%%%%%%%%%%%%%%%%%%%%%%%%%%%%%%%%%%%%%%%%%%%%%%%%%%
\subsection{Include by Input}
\label{sec:input}

Including child documents by |\include| has some restrictions by design.
Most notably, the content of a child document always occupies
its own set of pages; pages cannot be shared between child documents.
Usually, this behaviour makes perfect sense
because each child document contain an essential part of the document.
However, in some situations it may be desirable to compose
a document from a collection of parts
without having mandatory page breaks between then.
For this case, the package
provides a mechanism to include parts
by |\input| which can also be processed individually.
However, by construction this mechanism
requires manual handling of the content to be output.

%%%%%%%%%%%%%%%%%%%%%%%%%%%%%%%%%%%%%%%%
\DescribeMacro{\ifchilddocmanual}
The main file should be prepared as usual, see \secref{sec:include}.
However, the document body must make a distinction
between processing of an individual part and of the main document, e.g.:
%
\begin{center}
\begin{tabular}{l}
|\ifchilddocmanual|\\
|\input{\childdocname}|\\
|\||else|\\
\textit{document body with }|\input{|\textit{part}|}|\\
|\||fi|
\end{tabular}
\end{center}
%
The conditional |\ifchilddocmanual| is true whenever
a part to be included by |\input| is being compiled,
and the name of the part is stored in |\childdocname|.

%%%%%%%%%%%%%%%%%%%%%%%%%%%%%%%%%%%%%%%%
\DescribeMacro{\childdocby}
Each part to be included by |\input| should start with:
%
\begin{center}
\begin{tabular}{l}
|\input{childdoc.def}|\\
|\childdocby{|\textit{main}|}|\\
\end{tabular}
\end{center}
%
The directive |\childdocby| is similar to |\childdocof|
described in \secref{sec:include},
but the subsequent selection of content must be done manually.
To that end, both |\ifchilddoc| and |\ifchilddocmanual|
will be true upon processing of a part,
and the name of the part is stored in |\childdocname|.
Note that |\jobname| will be set to the filename of the current part
so that each part receives an individual |.aux| file
that does not interfere with the |.aux| file(s) of the main document.
This behaviour can be altered by the alternative form
|\childdocby[*]{|\textit{main}|}| (with a non-empty optional argument)
which uses the |.aux| file of the main document
by setting |\jobname| to \textit{main}.

%%%%%%%%%%%%%%%%%%%%%%%%%%%%%%%%%%%%%%%%%%%%%%%%%%%%%%%%%%%%%%%%%%%%%%%%%%%%%%%%
\subsection{Driver Development}
\label{sec:driver}

The \textsf{childdoc} mechanism can also be use for the development
of definition files such as \LaTeX{} styles or classes.
This case differs from the above setup with multiple parts
included by |\include| in that no |\includeonly| should be invoked.
This can be achieved by starting the include file
(before |\ProvidesPackage|) with:
%
\begin{center}
\begin{tabular}{l}
|\input{childdoc.def}|\\
|\childdocforward{|\textit{main}|}|\\
\end{tabular}
\end{center}
%
or alternatively with:
%
\begin{center}
\begin{tabular}{l}
|\input{childdoc.def}|\\
|\childdocby{|\textit{main}|}|\\
\end{tabular}
\end{center}
%
Both forms have slightly different effects as described above.
The main file is prepared as usual, see \secref{sec:include}.

%%%%%%%%%%%%%%%%%%%%%%%%%%%%%%%%%%%%%%%%%%%%%%%%%%%%%%%%%%%%%%%%%%%%%%%%%%%%%%%%
\subsection{Legacy Detection}
\label{sec:detection}

The directive |\childdocmain| in the main file can detect
whether the complete document or merely a child is to be compiled
even without using the directive |\childdocof|.
This method is deprecated because it is less robust
and there is no compelling reason to use it;
it is merely provided for backward compatibility
and it may be removed in future versions.

If the detection mechanism is to be used,
it is mandatory to correctly specify
the filename of the main file as the argument of |\childdocmain|:
%
\begin{center}
\begin{tabular}{l}
|\input{childdoc.def}|\\
|\childdocmain{|\textit{main}|}|\\
\end{tabular}
\end{center}
%
If |\jobname| does not match the argument \textit{main} of |\childdocmain|,
it is assumed that |\jobname| points to the child file to be compiled.
When using |\childdocmain| with the main file specified as argument,
it suffices to start a child file
with just |\input{|\textit{main}|}|
without loading of the package and using |\childdocof|.
If instead all processing is done
with the appropriate \textsf{childdoc} directives,
the argument of \textit{main} of |\childdocmain| can be empty.

An alternative version of the command line processing described
in \secref{sec:commandline} using the detection mechanism reads:
%
\begin{center}
|... -jobname "|\textit{target}|" "|[\textit{flags}]%
[|\def\jobname{|\textit{dest}|}|]|\input{|\textit{main}|}"|
\end{center}

%%%%%%%%%%%%%%%%%%%%%%%%%%%%%%%%%%%%%%%%%%%%%%%%%%%%%%%%%%%%%%%%%%%%%%%%%%%%%%%%
\subsection{Manual Code}
\label{sec:manual}

In case one cannot be certain whether the definitions file |childdoc.def|
is installed on the target \TeX{} distribution
and one prefers not to ship it,
it is conceivable to paste a few relevant commands into the sources.

To that end, drop all statements |\input{childdoc.def}|
and perform the replacements as outlined below.
Instead of |\childdocmain{|\textit{main}|}| add the following code
to the top of the main file:
%
\begin{center}
\begin{tabular}{l}
|\||ifdefined\childdocname\endinput\||fi\newif\ifchilddoc|\\
|\edef\childdocname{\scantokens\expandafter{\jobname\noexpand}}|\\
|\def\childdocmain{|\textit{main}|}\||ifx\childdocmain\childdocname\||else|\\
|\childdoctrue\includeonly{\childdocname}\let\jobname\childdocmain\||fi|\\
\end{tabular}
\end{center}
%
Instead of |\childdocof{|\textit{main}|}| just include the main file
at the top of each child file:
%
\begin{center}
|\input{|\textit{main}|}|
\end{center}
%
A simple redirection |\childdocforward{|\textit{dest}|}| is achieved by:
%
\begin{center}
|\def\jobname{|\textit{dest}|}\input{\jobname}|
\end{center}
%
The redirection with prefix
|\childdocforwardprefix[|\textit{prefix}|]{|\textit{dest}|}|
is accomplished by:
%
\begin{center}
\begin{tabular}{l}
|{\edef\jobname{\scantokens\expandafter{\jobname\noexpand}}|\\
|\def\redirectjob |\textit{prefix}|#1~~~{\gdef\jobname{|\textit{dest}|#1}}|\\
|\expandafter\redirectjob\jobname~~~}\input{\jobname}|
\end{tabular}
\end{center}

In an alternative approach,
child documents can be compiled by a specific command line
without additional code or specific definitions:
%
\begin{center}
|... -jobname "|\textit{target}|" "|[\textit{flags}]%
|\includeonly{|\textit{dest}|}\input{|\textit{main}|}"|
\end{center}
%

%%%%%%%%%%%%%%%%%%%%%%%%%%%%%%%%%%%%%%%%%%%%%%%%%%%%%%%%%%%%%%%%%%%%%%%%%%%%%%%%
%%%%%%%%%%%%%%%%%%%%%%%%%%%%%%%%%%%%%%%%%%%%%%%%%%%%%%%%%%%%%%%%%%%%%%%%%%%%%%%%
\section{Information}

%%%%%%%%%%%%%%%%%%%%%%%%%%%%%%%%%%%%%%%%%%%%%%%%%%%%%%%%%%%%%%%%%%%%%%%%%%%%%%%%
\subsection{Copyright}

Copyright \copyright{} 2017--2018 Niklas Beisert

This work may be distributed and/or modified under the
conditions of the \LaTeX{} Project Public License, either version 1.3
of this license or (at your option) any later version.
The latest version of this license is in
  \url{http://www.latex-project.org/lppl.txt}
and version 1.3 or later is part of all distributions of \LaTeX{}
version 2005/12/01 or later.

This work has the LPPL maintenance status `maintained'.

The Current Maintainer of this work is Niklas Beisert.

This work consists of the files |README.txt|, |childdoc.ins| and |childdoc.dtx|
as well as the derived files |childdoc.def|, |cdocsamp.tex|
with |cdocsch1.tex|, |cdocsch2.tex|, |cdocspt3.tex|, |cdocspt4.tex|,
|cdocsdrf.tex|, |cdocsfn1.tex|, |cdocsfn2.tex|
as well as |childdoc.pdf|.

%%%%%%%%%%%%%%%%%%%%%%%%%%%%%%%%%%%%%%%%%%%%%%%%%%%%%%%%%%%%%%%%%%%%%%%%%%%%%%%%
\subsection{Files and Installation}

The package consists of the files:
%
\begin{center}
\begin{tabular}{ll}
    |README.txt|   & readme file \\
    |childdoc.ins| & installation file \\
    |childdoc.dtx| & source file \\
    |childdoc.def| & definition file \\
    |cdocsamp.tex| & sample main file \\
    |cdocsch1.tex| & sample include file \\
    |cdocsch2.tex| & sample include file \\
    |cdocspt3.tex| & sample part file \\
    |cdocspt4.tex| & sample part file \\
    |cdocsdrf.tex| & sample redirection file \\
    |cdocsfn1.tex| & sample redirection file \\
    |cdocsfn2.tex| & sample redirection file \\
    |childdoc.pdf| & manual
\end{tabular}
\end{center}
%
The distribution consists of the files
|README.txt|, |childdoc.ins| and |childdoc.dtx|.
%
\begin{itemize}
\item
Run (pdf)\LaTeX{} on |childdoc.dtx|
to compile the manual |childdoc.pdf| (this file).
\item
Run \LaTeX{} on |childdoc.ins| to create the definitions file |childdoc.def|
and the sample |cdocsamp.tex| with include files
|cdocsch1.tex|, |cdocsch2.tex|, |cdocspt3.tex|, |cdocspt4.tex|,
|cdocsdrf.tex|, |cdocsfn1.tex|, |cdocsfn2.tex|.
Then copy the file |childdoc.def| to an appropriate directory of your \LaTeX{}
distribution, e.g.\ \textit{texmf-root}|/tex/latex/childdoc|.
\end{itemize}

%%%%%%%%%%%%%%%%%%%%%%%%%%%%%%%%%%%%%%%%%%%%%%%%%%%%%%%%%%%%%%%%%%%%%%%%%%%%%%%%
\subsection{Related CTAN Packages}

There are several other packages which offer a similar functionality:
%
\begin{itemize}
\item
The packages
\href{http://ctan.org/pkg/docmute}{\textsf{docmute}},
\href{http://ctan.org/pkg/includex}{\textsf{includex}} and
\href{http://ctan.org/pkg/standalone}{\textsf{standalone}}
provide commands to include only the document body of
a child file thus allowing both files to be compiled individually.
\item
The packages \href{http://ctan.org/pkg/subdocs}{\textsf{subdocs}}
and \href{http://ctan.org/pkg/subfiles}{\textsf{subfiles}}
provide structures in which the main and child documents can be
encapsulated and allowing them to be compiled individually.
The inclusion mechanism is different from the conventional |\include|.
\item
The package \href{http://ctan.org/pkg/combine}{\textsf{combine}}
is an elaborate solution to combine several documents into one.
\end{itemize}
%
See also the CTAN topic \href{http://ctan.org/topic/subdocs}{\textsf{subdocs}}
for further related packages.
The present package differs from the above solutions in that
a document structure constructed with the conventional |\include| mechanism
just needs two extra commands at the top of every file
such that all constituent files can be compiled individually.

%%%%%%%%%%%%%%%%%%%%%%%%%%%%%%%%%%%%%%%%%%%%%%%%%%%%%%%%%%%%%%%%%%%%%%%%%%%%%%%%
%\subsection{Feature Suggestions}
%
%The following is a list of features which may be useful for future
%versions of this package:
%%
%\begin{itemize}
%\item
%\ldots
%\end{itemize}

%%%%%%%%%%%%%%%%%%%%%%%%%%%%%%%%%%%%%%%%%%%%%%%%%%%%%%%%%%%%%%%%%%%%%%%%%%%%%%%%
\subsection{Revision History}

%%%%%%%%%%%%%%%%%%%%%%%%%%%%%%%%%%%%%%%%
\paragraph{v2.0:} 2018/12/30

\begin{itemize}
\item
immediate forward processing
\item
added |\childdocby| mechanism
\item
manual restructured
\end{itemize}

%%%%%%%%%%%%%%%%%%%%%%%%%%%%%%%%%%%%%%%%
\paragraph{v1.6:} 2018/01/17

\begin{itemize}
\item
application for development of include files
\item
corrections to manual
\end{itemize}

%%%%%%%%%%%%%%%%%%%%%%%%%%%%%%%%%%%%%%%%
\paragraph{v1.5:} 2017/05/21

\begin{itemize}
\item
more complete structuring introduced
\item
|\childdocof| introduced
\item
|\childdoc| renamed to |\childdocmain|
\item
|\childredirect| renamed to |\childdocforward| and |\childdocforwardprefix|
and functionality expanded
\end{itemize}

%%%%%%%%%%%%%%%%%%%%%%%%%%%%%%%%%%%%%%%%
\paragraph{v1.0:} 2017/04/27

\begin{itemize}
\item
manual and install package
\item
first version published on CTAN
\end{itemize}

%%%%%%%%%%%%%%%%%%%%%%%%%%%%%%%%%%%%%%%%
\paragraph{v0.6:} 2017/04/26

\begin{itemize}
\item
redirection mechanism added
\end{itemize}

%%%%%%%%%%%%%%%%%%%%%%%%%%%%%%%%%%%%%%%%
\paragraph{v0.5:} 2017/04/26

\begin{itemize}
\item
functionality in definition file
\end{itemize}


%%%%%%%%%%%%%%%%%%%%%%%%%%%%%%%%%%%%%%%%%%%%%%%%%%%%%%%%%%%%%%%%%%%%%%%%%%%%%%%%
%%%%%%%%%%%%%%%%%%%%%%%%%%%%%%%%%%%%%%%%%%%%%%%%%%%%%%%%%%%%%%%%%%%%%%%%%%%%%%%%
%%%%%%%%%%%%%%%%%%%%%%%%%%%%%%%%%%%%%%%%%%%%%%%%%%%%%%%%%%%%%%%%%%%%%%%%%%%%%%%%
\appendix

\settowidth\MacroIndent{\rmfamily\scriptsize 000\ }

 \DocInput{childdoc.dtx}

\end{document}
%</driver>
% \fi
%
% %%%%%%%%%%%%%%%%%%%%%%%%%%%%%%%%%%%%%%%%%%%%%%%%%%%%%%%%%%%%%%%%%%%%%%%%%%%%%%
% %%%%%%%%%%%%%%%%%%%%%%%%%%%%%%%%%%%%%%%%%%%%%%%%%%%%%%%%%%%%%%%%%%%%%%%%%%%%%%
% \section{Sample}
%\iffalse
%<*samplemain>
%\fi
%
% The following presents a sample document
% with two chapters, two parts, a title page,
% a compile flag as well as three forwarding files to set the flag.
% It consists of eight |.tex| files:
% \begin{center}
% \begin{tabular}{ll}
% |cdocsamp.tex|&main file\\
% |cdocsch1.tex|&include file for chapter 1\\
% |cdocsch2.tex|&include file for chapter 2\\
% |cdocspt3.tex|&include file for part 3\\
% |cdocspt4.tex|&include file for part 4\\
% |cdocsdrf.tex|&forwarding file for main file in draft mode\\
% |cdocsfi1.tex|&forwarding file for final version of chapter 1\\
% |cdocsfi2.tex|&forwarding file for final version of chapter 2\\
% \end{tabular}
% \end{center}
% Each of the eight files can be compiled directly by the \LaTeX{} compiler.
%
% %%%%%%%%%%%%%%%%%%%%%%%%%%%%%%%%%%%%%%
% \paragraph{Main File.}
%
% The main file is called |cdocsamp.tex|.
%
% Load the \textsf{childdoc} definitions and
% declare the filename for the main document:
%    \begin{macrocode}
\input{childdoc.def}
\childdocmain{}
%    \end{macrocode}

% Optional override for |\version| flag:
%    \begin{macrocode}
%%\ifchilddoc\else\providecommand{\version}{draft}\fi
%    \end{macrocode}

% Define the default values for the |\version| flag
% (|final| for the main file and |draft| for childs):
%    \begin{macrocode}
\ifchilddoc
\providecommand{\version}{draft}
\else
\providecommand{\version}{final}
\fi
%    \end{macrocode}

% Load the standard document class:
%    \begin{macrocode}
\documentclass[12pt]{article}
%    \end{macrocode}

% Start the document body:
%    \begin{macrocode}
\begin{document}
%    \end{macrocode}

% Declare a title page.
% Print title, part of document being processed and version flag:
%    \begin{macrocode}
\addtocounter{page}{-1}
\begin{center}
{\LARGE\bfseries{}childdoc example\par}
\vspace{1cm}
\ifchilddoc
\ifchilddocmanual part\else chapter\fi:
`\childdocname' of `\childdocjob'\par
\else
main document: `\childdocjob'\par
\fi
version: \version\par
\end{center}
\newpage
%    \end{macrocode}

% Manually include selected file,
% otherwise process as usual:
%    \begin{macrocode}
\ifchilddocmanual
\section*{part `\childdocname'}
\input{\childdocname}
\else
%    \end{macrocode}

% Include the two chapters:
%    \begin{macrocode}
\include{cdocsch1}
\include{cdocsch2}
%    \end{macrocode}

% Include the two parts unless only chapters should be displayed:
%    \begin{macrocode}
\ifchilddoc\else
\section{part three}
\input{cdocspt3}
\section{part four}
\input{cdocspt4}
\fi
%    \end{macrocode}

% Process as usual until here:
%    \begin{macrocode}
\fi
%    \end{macrocode}

% End of document body:
%    \begin{macrocode}
\end{document}
%    \end{macrocode}
%\iffalse
%</samplemain>
%\fi
%
% %%%%%%%%%%%%%%%%%%%%%%%%%%%%%%%%%%%%%%
% \paragraph{Chapter Include Files.}
%
% The include files are called |cdocsch1.tex| and |cdocsch2.tex|.
%
%\iffalse
%<*samplechap1|samplechap2>
%\fi

% Optional override for |\version| flag:
%    \begin{macrocode}
%%\providecommand{\version}{final}
%    \end{macrocode}

% Include the main document:
%    \begin{macrocode}
\input{childdoc.def}
\childdocof{cdocsamp}
%    \end{macrocode}

%\iffalse
%</samplechap1|samplechap2>
%\fi
%
%\iffalse
%<*samplechap1>
%\fi
% Some text for chapter 1:
%    \begin{macrocode}
\section{one}
some text in chapter one
%    \end{macrocode}

%\iffalse
%</samplechap1>
%\fi
% Some text for chapter 2:
%\iffalse
%<*samplechap2>
%\fi
%    \begin{macrocode}
\section{two}
more text in chapter two
%    \end{macrocode}

%\iffalse
%</samplechap2>
%\fi
%
% %%%%%%%%%%%%%%%%%%%%%%%%%%%%%%%%%%%%%%
% \paragraph{Part Include Files.}
%
% The include files are called |cdocspt3.tex| and |cdocspt4.tex|.
%
%\iffalse
%<*samplepart3|samplepart4>
%\fi

% Optional override for |\version| flag:
%    \begin{macrocode}
%%\providecommand{\version}{final}
%    \end{macrocode}

% Include the main document:
%    \begin{macrocode}
\input{childdoc.def}
\childdocby{cdocsamp}
%    \end{macrocode}

%\iffalse
%</samplepart3|samplepart4>
%\fi
%
%\iffalse
%<*samplepart3>
%\fi
% Some text for part 3:
%    \begin{macrocode}
some text in part three
%    \end{macrocode}

%\iffalse
%</samplepart3>
%\fi
% Some text for part 4:
%\iffalse
%<*samplepart4>
%\fi
%    \begin{macrocode}
more text in part four
%    \end{macrocode}

%\iffalse
%</samplepart4>
%\fi
%
% %%%%%%%%%%%%%%%%%%%%%%%%%%%%%%%%%%%%%%
% \paragraph{Forwarding for a Complete Draft.}
%
% The following forwarding file |cdocsdrf.tex|
% compiles the main document in draft mode:
%\iffalse
%<*sampledraft>
%\fi
%    \begin{macrocode}
\def\version{draft}
\input{childdoc.def}
\childdocforward{cdocsamp}
%    \end{macrocode}

%\iffalse
%</sampledraft>
%\fi
%
% %%%%%%%%%%%%%%%%%%%%%%%%%%%%%%%%%%%%%%
% \paragraph{Forwarding for Final Version of the Chapters.}
%
% The following forwarding files |cdocsfn1.tex| and |cdocsfn2.tex|
% (with identical content)
% compile the final versions of the child documents
% |cdocsch1.tex| and |cdocsch2.tex|, respectively:
%\iffalse
%<*samplefinal>
%\fi
%    \begin{macrocode}
\def\version{final}
\input{childdoc.def}
\childdocforwardprefix[cdocsamp]{cdocsfn}{cdocsch}
%    \end{macrocode}

%\iffalse
%</samplefinal>
%\fi
%
% %%%%%%%%%%%%%%%%%%%%%%%%%%%%%%%%%%%%%%
% \paragraph{Command Line Processing.}
%
% The following three command lines generate the output files
% |cdocscld|, |cdocscl1| and |cdocscl2|
% which should be identical to
% |cdocsdrf|, |cdocsch1| and |cdocsfn2|, respectively:
% \begin{center}
% \begin{tabular}{l}
% |latex -jobname cdocscld \|\\
% |  "\def\version{draft}\input{childdoc.def}\childdocforward{cdocsamp}"|\\
% |latex -jobname cdocscl1 \|\\
% |  "\input{childdoc.def}\childdocforward[cdocsamp]{cdocsch1}"|\\
% |latex -jobname cdocscl2 \|\\
% |  "\def\version{final}\input{childdoc.def}\childdocforward{cdocsch2}"|
% \end{tabular}
% \end{center}
% Note that the trailing backslash on each first line
% merely continues the input to the second line
% (for convenient cut ant paste).
% Furthermore, the command |latex| can be replaced by any
% of its alternative versions such as |pdflatex|.
%
% %%%%%%%%%%%%%%%%%%%%%%%%%%%%%%%%%%%%%%%%%%%%%%%%%%%%%%%%%%%%%%%%%%%%%%%%%%%%%%
% %%%%%%%%%%%%%%%%%%%%%%%%%%%%%%%%%%%%%%%%%%%%%%%%%%%%%%%%%%%%%%%%%%%%%%%%%%%%%%
% \section{Implementation}
%\iffalse
%<*package>
%\fi
%
% This section describes the definitions file |childdoc.def|.

% The definitions cannot be loaded using |\usepackage| or |\RequirePackage|
% which has a mechanism to prevent loading a style file more than once.
% When loading the definitions by means of |\input|
% multiple instances have to be prevented manually:
%\iffalse
%This code needs to be before the `\ProvidesFile' directive
%which is defined at the beginning of this file.
%Therefore it is also placed there and commented out here.
%</package>
%<*discard>
%\fi
%    \begin{macrocode}
\ifdefined\childdocmain\endinput\fi
%    \end{macrocode}
%\iffalse
%</discard>
%<*package>
%\fi
%
% \macro{\ifchilddoc}
% \macro{\ifchilddocmanual}
% The conditional |\ifchilddoc| tells whether a
% child (true) or main (false) document is being compiled.
% The conditional |\ifchilddocmanual| tells whether
% the |\includeonly| mechanism is used (false) or
% the selection of child files must be performed manually (true).
% The definitions initialise to false:
%    \begin{macrocode}
\newif\ifchilddoc
\newif\ifchilddocmanual
%    \end{macrocode}

% \macro{\childdocname}
% \macro{\childdocjob}
% The macro |\childdocname| stores the name of the main document
% to be compiled. The macro |\childdocjob| stores the name of
% the document on which the \LaTeX{} compiler was originally invoked.
% The content of |\jobname| cannot be compared
% to filenames specified in the source due to different catcodes.
% The following code rescans |\jobname|, stores the result
% in |\childdocname| and saves a copy in |\childdocjob|:
%    \begin{macrocode}
\edef\childdocname{\scantokens\expandafter{\jobname\noexpand}}
\let\childdocjob\childdocname
%    \end{macrocode}

% \macro{\childdocdisable}
% The macro |\childdocdisable| prevents the main file
% from being processed more than once.
% At this stage, the main document command |\childdocmain|
% is assumed to be called once again where it should do nothing.
% Any subsequent call to it should prevent
% a secondary processing of the main document
% It overwrites the forwarding commands
% |\childdocof| and |\childdocforward|
% with empty macros to prevent further inclusions of the main document:
%    \begin{macrocode}
\newcommand{\childdocdisable}
{
  \renewcommand{\childdocmain}[1]{\renewcommand{\childdocmain}[1]{\endinput}}
  \renewcommand{\childdocof}[1]{}
  \renewcommand{\childdocby}[2][]{}
  \renewcommand{\childdocforward}[2][]{}
  \renewcommand{\childdocdisable}{}
}
%    \end{macrocode}

% \macro{\childdocmain}
% The macro |\childdocmain| is to be called at the top of the main file
% with nothing or the main filename (without extension) as argument.
% First, it breaks loops.
% If the argument is not empty and does not match |\childdocname|
% (which is set by the first inclusion of |childdoc.def|),
% |\ifchilddoc| is set to true, |\includeonly| is applied to the child file
% and |\jobname| is set to the main file
% (for proper handling of |.aux| files):
%    \begin{macrocode}
\newcommand{\childdocmain}[1]
{
  \childdocdisable\childdocmain{}
  \if?#1?\else
    \begingroup
      \def\childdoctmp{#1}
      \ifx\childdoctmp\childdocname
        \def\childdoctmp{}
      \else
        \def\childdoctmp
        {
          \childdoctrue
          \includeonly{\childdocname}
          \def\childdocjob{#1}
          \def\jobname{#1}
        }
      \fi
      \expandafter
    \endgroup
    \childdoctmp
  \fi
}
%    \end{macrocode}

% \macro{\childdocof}
% The command |\childdocof| redirects
% compilation to the main file |#1|.
%    \begin{macrocode}
\newcommand{\childdocof}[1]
{
  \childdocdisable
  \childdoctrue
  \includeonly{\childdocname}
  \def\jobname{#1}
  \def\childdocjob{#1}
  \input{#1}
}
%    \end{macrocode}

% \macro{\childdocby}
% The command |\childdocby| ....
%    \begin{macrocode}
\newcommand{\childdocby}[2][]
{
  \childdocdisable
  \childdoctrue
  \childdocmanualtrue
  \if?#1?\else
    \def\jobname{#2}
  \fi
  \def\childdocjob{#2}
  \input{#2}
  \endinput
}
%    \end{macrocode}

% \macro{\childdocforward}
% The command |\childdocforward| redirects
% compilation to the main file or
% (if the optional argument is given) a child file.
% Parameters are set as if the main file
% or a child file starting with |\childdocof| was compiled.
% Then compilation is handed over to the main file:
%    \begin{macrocode}
\newcommand{\childdocforward}[2][]
{
  \begingroup
    \if?#1?
      \def\childdoctmp
      {
        \def\childdocname{#2}
        \def\childdocjob{#2}
        \def\jobname{#2}
        \input{#2}
        \endinput
      }
    \else
      \def\childdoctmp
      {
        \childdocdisable
        \def\childdocname{#2}
        \childdoctrue
        \includeonly{#2}
        \def\childdocjob{#1}
        \def\jobname{#1}
        \input{#1}
        \endinput
      }
    \fi
    \expandafter
  \endgroup
  \childdoctmp
}
%    \end{macrocode}

% \macro{\childdocforwardprefix}
% The command |\childdocforwardprefix| redirects
% compilation to the main or a child file by means of a pattern.
% The prefix |#1| in the current filename is replaced by |#2|
% and the suffix of the current filename is kept
% (it is assumed that the filename does not contain the substring `|~~~|'
% which is used as a delimiter).
% Compilation is handed over to the new file by |\childdocforward|:
%    \begin{macrocode}
\newcommand{\childdocforwardprefix}[3][]
{
  \begingroup
    \def\childdocextract #2##1~~~{\def\childdoctmp{\childdocforward[#1]{#3##1}}}
    \expandafter\childdocextract\childdocname~~~
    \expandafter
  \endgroup
  \childdoctmp
}
%    \end{macrocode}

% \macro{\childdoc}
% The deprecated macro |\childdoc| is a legacy version of |\childdocmain|:
%    \begin{macrocode}
\newcommand{\childdoc}{\childdocmain}
%    \end{macrocode}

% \macro{\childdocredirect}
% The deprecated macro |\childdocredirect| is a legacy version
% of |\childdocforward| and |\childdocforwardprefix|:
%    \begin{macrocode}
\newcommand{\childdocredirect}[2][]
{
  \begingroup
    \if?#1?
      \def\childdoctmp{\childdocforward{#2}}
    \else
      \def\childdoctmp{\childdocforwardprefix{#1}{#2}}
    \fi
    \expandafter
  \endgroup
  \childdoctmp
}
%    \end{macrocode}

%\iffalse
%</package>
%\fi
%
\endinput

\childdocforward{cdocsamp}
%    \end{macrocode}

%\iffalse
%</sampledraft>
%\fi
%
% %%%%%%%%%%%%%%%%%%%%%%%%%%%%%%%%%%%%%%
% \paragraph{Forwarding for Final Version of the Chapters.}
%
% The following forwarding files |cdocsfn1.tex| and |cdocsfn2.tex|
% (with identical content)
% compile the final versions of the child documents
% |cdocsch1.tex| and |cdocsch2.tex|, respectively:
%\iffalse
%<*samplefinal>
%\fi
%    \begin{macrocode}
\def\version{final}
% \iffalse
%
% childdoc.dtx Copyright (C) 2017-2018 Niklas Beisert
%
% This work may be distributed and/or modified under the
% conditions of the LaTeX Project Public License, either version 1.3
% of this license or (at your option) any later version.
% The latest version of this license is in
%   http://www.latex-project.org/lppl.txt
% and version 1.3 or later is part of all distributions of LaTeX
% version 2005/12/01 or later.
%
% This work has the LPPL maintenance status `maintained'.
%
% The Current Maintainer of this work is Niklas Beisert.
%
% This work consists of the files childdoc.dtx and childdoc.ins
% and the derived files childdoc.def and cdocsamp.tex with
% cdocsch1.tex, cdocsch2.tex, cdocsdrf.tex, cdocsfn1.tex, cdocsfn2.tex.
%
%<package>\ifdefined\childdocmain\endinput\fi
%<package>\ProvidesFile{childdoc.def}[2018/12/30 v2.0 child document driver]
%<samplemain>\ProvidesFile{cdocsamp.tex}[2018/12/30 v2.0 sample for childdoc]
%<*driver>
%\ProvidesFile{childdoc.drv}[2018/12/30 v2.0 childdoc reference manual file]
\PassOptionsToClass{10pt,a4paper}{article}
\documentclass{ltxdoc}

\usepackage[margin=35mm]{geometry}
\usepackage{hyperref}
\usepackage{hyperxmp}
\usepackage[usenames]{color}

\hypersetup{colorlinks=true}
\hypersetup{pdfstartview=FitH}
\hypersetup{pdfpagemode=UseNone}
\hypersetup{pdfsource={}}
\hypersetup{pdflang={en-UK}}
\hypersetup{pdfcopyright={Copyright 2017-2018 Niklas Beisert.
  This work may be distributed and/or modified under the
  conditions of the LaTeX Project Public License, either version 1.3
  of this license or (at your option) any later version.}}
\hypersetup{pdflicenseurl={http://www.latex-project.org/lppl.txt}}
\hypersetup{pdfcontactaddress={ETH Zurich, ITP, HIT K,
  Wolfgang-Pauli-Strasse 27}}
\hypersetup{pdfcontactpostcode={8093}}
\hypersetup{pdfcontactcity={Zurich}}
\hypersetup{pdfcontactcountry={Switzerland}}
\hypersetup{pdfcontactemail={nbeisert@itp.phys.ethz.ch}}
\hypersetup{pdfcontacturl={http://people.phys.ethz.ch/\xmptilde nbeisert/}}

\newcommand{\secref}[1]{\hyperref[#1]{section \ref*{#1}}}

\parskip1ex
\parindent0pt
\let\olditemize\itemize
\def\itemize{\olditemize\parskip0pt}

\begin{document}

\title{The \textsf{childdoc} Package}
\hypersetup{pdftitle={The childdoc Package}}
\author{Niklas Beisert\\[2ex]
  Institut f\"ur Theoretische Physik\\
  Eidgen\"ossische Technische Hochschule Z\"urich\\
  Wolfgang-Pauli-Strasse 27, 8093 Z\"urich, Switzerland\\[1ex]
  \href{mailto:nbeisert@itp.phys.ethz.ch}
  {\texttt{nbeisert@itp.phys.ethz.ch}}}
\hypersetup{pdfauthor={Niklas Beisert}}
\hypersetup{pdfsubject={Manual for the LaTeX2e Package childdoc}}
\date{30 December 2018, \textsf{v2.0}}
\maketitle

\begin{abstract}\noindent
\textsf{childdoc} is a \LaTeXe{} package
that enables the direct compilation
of document sections included by |\include|
to individual files.
\end{abstract}

\begingroup
\parskip0ex
\tableofcontents
\endgroup

%%%%%%%%%%%%%%%%%%%%%%%%%%%%%%%%%%%%%%%%%%%%%%%%%%%%%%%%%%%%%%%%%%%%%%%%%%%%%%%%
%%%%%%%%%%%%%%%%%%%%%%%%%%%%%%%%%%%%%%%%%%%%%%%%%%%%%%%%%%%%%%%%%%%%%%%%%%%%%%%%
\section{Introduction}

\LaTeX{} provides a mechanism to structure a large document (such as a book)
into a main file and several child files (containing the chapters)
using the |\include| command.
This mechanism is beneficial for documents
which span hundreds of pages in order to
make the source file(s) more manageable.
Moreover, compilation can be restricted to
selected child files by means of the |\includeonly| command.
The latter feature can be used to reduce the compilation time while editing
(this was significantly more useful in the earlier days of \LaTeX{})
or to generate a smaller document which is easier to navigate.
Another application of |\includeonly| is to generate
documents consisting of selected parts of the complete document.

However, there are a few drawbacks of the plain |\include| mechanism:
\begin{itemize}
\item
The child files cannot be compiled on their own,
they can only be compiled via the main file.
A naive editing environment
(such as a text editor with an option
to have the current file processed by \LaTeX)
may require one to switch to the main file before compiling;
attempting to compile the child file produces errors.
\item
The main file must be modified (each time)
to adjust the |\includeonly| command
to the present needs. This easily leaves the main file in a messy state.
\item
The generated document will always carry the filename
of the main document. This is inconvenient if
several child files are to be compiled and
to be kept for distribution.
\end{itemize}

The present package provides a simple interface
to make child files individually compilable by \LaTeX{}.
Compiling a child file then has the same effect as compiling
the main file with an |\includeonly| command
to select the appropriate child.
Moreover the generated document will carry the name of the child
rather than the main file.
This resolves all three above issues.

This feature is meant to make the editing of books,
thesis documents and lecture notes somewhat more convenient.
However, the package can also be used efficiently for
composing a series of documents (such as exercise sheets)
which are typically distributed individually.
It then assists the author in generating the individual documents
(potentially in different versions)
as well as a document containing the collected series.
Another application is in developing style files
or other kinds of included material
where compilation of the style file could redirect
to a sample or test file.

%%%%%%%%%%%%%%%%%%%%%%%%%%%%%%%%%%%%%%%%%%%%%%%%%%%%%%%%%%%%%%%%%%%%%%%%%%%%%%%%
%%%%%%%%%%%%%%%%%%%%%%%%%%%%%%%%%%%%%%%%%%%%%%%%%%%%%%%%%%%%%%%%%%%%%%%%%%%%%%%%
\section{Usage}

First of all, the package \textsf{childdoc} is \emph{not} a standard
\LaTeXe{} |.sty| style file! Therefore it needs to be invoked in
a non-standard way.

%%%%%%%%%%%%%%%%%%%%%%%%%%%%%%%%%%%%%%%%%%%%%%%%%%%%%%%%%%%%%%%%%%%%%%%%%%%%%%%%
\subsection{Included Files}
\label{sec:include}

%%%%%%%%%%%%%%%%%%%%%%%%%%%%%%%%%%%%%%%%
\DescribeMacro{\childdocmain}
To use the package, add the commands
\begin{center}
\begin{tabular}{l}
|\input{childdoc.def}|\\
|\childdocmain{}|\\
\end{tabular}
\end{center}
at the very top of the main \LaTeX{} file,
in particular \emph{before} the |\documentclass| statement!
The argument of |\childdocmain| should be left empty
(but it must be present).

%%%%%%%%%%%%%%%%%%%%%%%%%%%%%%%%%%%%%%%%
\DescribeMacro{\childdocof}
Furthermore, add the commands
\begin{center}
\begin{tabular}{l}
|\input{childdoc.def}|\\
|\childdocof{|\textit{main}|}|\\
\end{tabular}
\end{center}
at the top of every child file \textit{child}
which is included by |\include{|\textit{child}|}|
from within the main file
(or at least for those files to be compiled individually).
The argument \textit{main} must be the filename of the main file.

There are a couple of
considerations in setting up the main and child documents:

%%%%%%%%%%%%%%%%%%%%%%%%%%%%%%%%%%%%%%%%
\paragraph{Restrictions.}

Please note the following restrictions:
\begin{itemize}
\item
|\childdocmain| must be called with one argument \textit{main}
to ensure compatibility with earlier version of the package.
It must either be empty (|\childdocmain{}|)
or precisely match the filename of the main file in which it is specified.
See \secref{sec:detection} for further information.
\item
The filename \textit{main} must be specified without the |.tex| extension.
\item
The filename \textit{main} is case sensitive
(even in case-insensitive file systems)
due to internal string comparison.
\item
The argument \textit{main} should be fully expanded, it cannot be a macro.
\item
Subdirectories and special characters should be avoided in filenames.
\item
The command |\childdocmain{|\textit{main}|}| must be followed by a whitespace.
It should not be followed immediately by another command
or by a comment mark `|%|'.
This is because the \TeX{} parser reads the token immediately following
the argument of |\childdocmain| and puts it
at the beginning of every child section;
however, a white\-space is ignored.
\end{itemize}

%%%%%%%%%%%%%%%%%%%%%%%%%%%%%%%%%%%%%%%%
\paragraph{Content of Main File.}

It is advisable to place all content in the child files included by |\include|.
Any output contained in the main file will appear in all child documents
unless suppressed manually;
it cannot be suppressed automatically by the |\includeonly| directive
and thus should normally be avoided.
A method to include some content in the main file
by means of conditional processing is described in \secref{sec:conditional}.

%%%%%%%%%%%%%%%%%%%%%%%%%%%%%%%%%%%%%%%%
\paragraph{Page Numbering.}

When only a part of the document is compiled,
the appropriate numbering of pages
(as well as other status parameters)
is determined from the |.aux| files.
The latter contain information from previous passes.
However this information needs to propagate through
all intermediate child documents.
Therefore the page numbering in child documents may well
be inconsistent until the complete document is compiled at least once.

A useful (if unconventional) way to always ensure a consistent
page numbering is to restart the numbering in each child document
and denote the pages by `\textit{child}|.|\textit{page}'
where \textit{child} represents the chapter/section number of the child file.
This can be achieved by the command
|\numberwithin{page}{|\textit{child}|}|
of the \textsf{amsmath} package
where \textit{child} can be |chapter| or |section|
depending on the chosen structuring.
Alternatively, one can modify the macro |\thepage| appropriately
and reset the counter |page| at the start of each child file.

%%%%%%%%%%%%%%%%%%%%%%%%%%%%%%%%%%%%%%%%%%%%%%%%%%%%%%%%%%%%%%%%%%%%%%%%%%%%%%%%
\subsection{Conditional Processing}
\label{sec:conditional}

The package provides a mechanism to compile different versions
of a document. To customise the versions further some conditional processing
can come in handy to distinguish which version is being compiled.
The package provides two macros to describe the compilation context:

%%%%%%%%%%%%%%%%%%%%%%%%%%%%%%%%%%%%%%%%
\DescribeMacro{\ifchilddoc}
The conditional |\ifchilddoc| distinguishes between the compilation of
child documents and the main document:
%
\begin{center}
|\ifchilddoc |\textit{child-code}| |[|\||else |\textit{main-code}]| \||fi|
\end{center}

%%%%%%%%%%%%%%%%%%%%%%%%%%%%%%%%%%%%%%%%
\DescribeMacro{\childdocname}
\DescribeMacro{\childdocjob}
The macro |\childdocname| contains the filename (without extension)
of the main or child file being processed.
Note that |\childdocjob| will always contain the name of the main file.

%%%%%%%%%%%%%%%%%%%%%%%%%%%%%%%%%%%%%%%%
\paragraph{Title Page.}

Conditional processing can be used to include a title or banner page
in the main document when proper precautions are taken.
Importantly, the code in the main file should ensure that the page counter
(as well as other status parameters which are stored in the |.aux| files)
takes the same value after the conditional processing.
Otherwise the page numbers may take divergent values
depending on which part is compiled.

For example, a title page could be declared by:
%
\begin{center}
\begin{tabular}{l}
|\ifchilddoc\||else|\\
|\addtocounter{page}{-1}|\\
\textit{code for title page}\\
|\newpage|\\
|\||fi|
\end{tabular}
\end{center}
%
A banner page for the child documents can be generated by:
%
\begin{center}
\begin{tabular}{l}
|\ifchilddoc|\\
|\addtocounter{page}{-1}|\\
\textit{code for banner page}\\
|\newpage|\\
|\||fi|
\end{tabular}
\end{center}
%
Here one could write a message such as:
\begin{center}
|This is the part \childdocname{} of \childdocjob{}.|
\end{center}

%%%%%%%%%%%%%%%%%%%%%%%%%%%%%%%%%%%%%%%%%%%%%%%%%%%%%%%%%%%%%%%%%%%%%%%%%%%%%%%%
\subsection{Flags}
\label{sec:flags}

The package makes it easy to generate different versions
of the main or child documents.
To this end compilation flags can be defined
and assigned different default values.
They will be particularly useful in conjunction
with the forwarding mechanism described in \secref{sec:forward}.

For example, it may be useful to have a flag |\version|
which can be set to |draft| or |final|.
The document source will contain some conditional code
depending on the value of |\version|.
Suppose further, the flag should default to |final| for the main file
and to |draft| for child files
which is a natural assignment for editing the document.
This is achieved by placing the following code
in the preamble of the main document
(below the |\childdocmain| directive):
%
\begin{center}
\begin{tabular}{l}
|\ifchilddoc|\\
|\providecommand{\version}{draft}|\\
|\||else|\\
|\providecommand{\version}{final}|\\
|\||fi|
\end{tabular}
\end{center}
%
The definition by |\providecommand| makes sure
that previous definitions are not overwritten.
Further statements |\providecommand{\version}{...}|
can thus be added before the above code to override it.

For the main file, one might add a line
(between |\childdocmain| and the above block)
%
\begin{center}
|%\ifchilddoc\||else\providecommand{\version}{draft}\||fi|
\end{center}
%
which can be uncommented to produce a draft version.
Likewise one can add a line to the very top of a child file
(above the |\childdocof{|\textit{main}|}| directive)
%
\begin{center}
|%\providecommand{\version}{final}|
\end{center}
%
which can be uncommented to produce the final version of this child document.

%%%%%%%%%%%%%%%%%%%%%%%%%%%%%%%%%%%%%%%%%%%%%%%%%%%%%%%%%%%%%%%%%%%%%%%%%%%%%%%%
\subsection{Forwarding}
\label{sec:forward}

Different versions of the main or child documents
using compilation flags as described in \secref{sec:flags}
can be (permanently) stored in different files
for convenient compilation, viewing and distribution.
To this end, the package defines a command
to pass on compilation to a different file:

%%%%%%%%%%%%%%%%%%%%%%%%%%%%%%%%%%%%%%%%
\DescribeMacro{\childdocforward}
The command |\childdocforward| redirects processing to
another source file:
%
\begin{center}
\begin{tabular}{l}
|\input{childdoc.def}|\\
|\childdocforward[|\textit{main}|]{|\textit{dest}|}|\\
\end{tabular}
\end{center}
%
The argument \textit{dest} is the destination file
(without extension).
It should be the main file or one of the child files.
Note that further \textsf{childdoc} directives
such as |\childdocof| and |\childdocforward|
in the indicated file will be processed in this form.
The optional argument \textit{main}
passes on directly to the main file \textit{main}
while pretending to compile the child \textit{dest}.
This form behaves as if \textit{dest}
issues |\childdocof{|\textit{main}|}| right away,
and no further \textsf{childdoc} directives will be processed.

%%%%%%%%%%%%%%%%%%%%%%%%%%%%%%%%%%%%%%%%
\DescribeMacro{\...prefix}
In the alternative form |\childdocforwardprefix|,
%
\begin{center}
\begin{tabular}{l}
|\input{childdoc.def}|\\
|\childdocforwardprefix[|\textit{main}|]{|\textit{prefix}|}{|\textit{dest}|}|
\end{tabular}
\end{center}
%
the destination file is determined by a pattern
depending on the current file:
To make this work, the current file must be called
`{\textit{prefix}\hspace{0.2em}\textit{suffix}}'
with \textit{prefix} matching precisely the argument.
Processing is then passed on to the file
`{\textit{dest}\hspace{0.2em}\textit{suffix}}'.
Surely, the same effect is achieved by
directly specifying the
argument `{\textit{dest}\hspace{0.2em}\textit{suffix}}'
in the first form.
However, that requires to set up a different file
for each child. With the alternative form of the command
all these files can have exactly the same content
which simplifies setting them up and maintaining them.

For example, the following file |draft.tex|
with a compilation flag |\version| as described in \secref{sec:flags}
compiles the main document as a draft:
%
\begin{center}
\begin{tabular}{l}
|\def\version{draft}|\\
|\input{childdoc.def}|\\
|\childdocforward{|\textit{main}|}|
\end{tabular}
\end{center}
%
Likewise, the following files |final|\textit{nn}|.tex|
compile the final version of the child document
|child|\textit{nn}|.tex|:
%
\begin{center}
\begin{tabular}{l}
|\def\version{final}|\\
|\input{childdoc.def}|\\
|\childdocforwardprefix{final}{child}|
\end{tabular}
\end{center}
%

Note that when several versions of a main file and/or of each child file
are to be generated, it may be convenient to set up a |Makefile| or
shell script to automatise the process.

%%%%%%%%%%%%%%%%%%%%%%%%%%%%%%%%%%%%%%%%%%%%%%%%%%%%%%%%%%%%%%%%%%%%%%%%%%%%%%%%
\subsection{Command Line Processing}
\label{sec:commandline}

The effect of redirection files can also be achieved by invoking
the \LaTeX{} compiler with a more elaborate command line.
Most conveniently this should be done as part
of a shell script or a |Makefile|.

When using \textsf{childdoc} in the main file, the following
command lines effectively perform a redirection
(note that depending on the shell being used,
backslashes may have to be doubled: `|\|' $\to$ `|\\|'):
%
\begin{center}
|... -jobname "|\textit{target}|" |\\|"|[\textit{flags}]%
|\input{childdoc.def}\childdocforward[|\textit{main}|]{|\textit{dest}|}"|
\end{center}
%
Here \textit{target} is the name of the output file,
\textit{main} is the name of the main file
and \textit{dest} is the name of the main or child file to be processed
(all filenames without extensions).
The optional argument \textit{main} can be omitted
if \textit{main} matches \textit{dest}.
Optionally, compilation \textit{flags} can be defined via |\def| commands.
This command line makes the \TeX{} engine believe
it is compiling the file \textit{target}
whose content is specified as the latter parameter.
The provided code then forwards the processing to
\textit{main} or \textit{dest} as described in \secref{sec:forward}.

%%%%%%%%%%%%%%%%%%%%%%%%%%%%%%%%%%%%%%%%%%%%%%%%%%%%%%%%%%%%%%%%%%%%%%%%%%%%%%%%
\subsection{Include by Input}
\label{sec:input}

Including child documents by |\include| has some restrictions by design.
Most notably, the content of a child document always occupies
its own set of pages; pages cannot be shared between child documents.
Usually, this behaviour makes perfect sense
because each child document contain an essential part of the document.
However, in some situations it may be desirable to compose
a document from a collection of parts
without having mandatory page breaks between then.
For this case, the package
provides a mechanism to include parts
by |\input| which can also be processed individually.
However, by construction this mechanism
requires manual handling of the content to be output.

%%%%%%%%%%%%%%%%%%%%%%%%%%%%%%%%%%%%%%%%
\DescribeMacro{\ifchilddocmanual}
The main file should be prepared as usual, see \secref{sec:include}.
However, the document body must make a distinction
between processing of an individual part and of the main document, e.g.:
%
\begin{center}
\begin{tabular}{l}
|\ifchilddocmanual|\\
|\input{\childdocname}|\\
|\||else|\\
\textit{document body with }|\input{|\textit{part}|}|\\
|\||fi|
\end{tabular}
\end{center}
%
The conditional |\ifchilddocmanual| is true whenever
a part to be included by |\input| is being compiled,
and the name of the part is stored in |\childdocname|.

%%%%%%%%%%%%%%%%%%%%%%%%%%%%%%%%%%%%%%%%
\DescribeMacro{\childdocby}
Each part to be included by |\input| should start with:
%
\begin{center}
\begin{tabular}{l}
|\input{childdoc.def}|\\
|\childdocby{|\textit{main}|}|\\
\end{tabular}
\end{center}
%
The directive |\childdocby| is similar to |\childdocof|
described in \secref{sec:include},
but the subsequent selection of content must be done manually.
To that end, both |\ifchilddoc| and |\ifchilddocmanual|
will be true upon processing of a part,
and the name of the part is stored in |\childdocname|.
Note that |\jobname| will be set to the filename of the current part
so that each part receives an individual |.aux| file
that does not interfere with the |.aux| file(s) of the main document.
This behaviour can be altered by the alternative form
|\childdocby[*]{|\textit{main}|}| (with a non-empty optional argument)
which uses the |.aux| file of the main document
by setting |\jobname| to \textit{main}.

%%%%%%%%%%%%%%%%%%%%%%%%%%%%%%%%%%%%%%%%%%%%%%%%%%%%%%%%%%%%%%%%%%%%%%%%%%%%%%%%
\subsection{Driver Development}
\label{sec:driver}

The \textsf{childdoc} mechanism can also be use for the development
of definition files such as \LaTeX{} styles or classes.
This case differs from the above setup with multiple parts
included by |\include| in that no |\includeonly| should be invoked.
This can be achieved by starting the include file
(before |\ProvidesPackage|) with:
%
\begin{center}
\begin{tabular}{l}
|\input{childdoc.def}|\\
|\childdocforward{|\textit{main}|}|\\
\end{tabular}
\end{center}
%
or alternatively with:
%
\begin{center}
\begin{tabular}{l}
|\input{childdoc.def}|\\
|\childdocby{|\textit{main}|}|\\
\end{tabular}
\end{center}
%
Both forms have slightly different effects as described above.
The main file is prepared as usual, see \secref{sec:include}.

%%%%%%%%%%%%%%%%%%%%%%%%%%%%%%%%%%%%%%%%%%%%%%%%%%%%%%%%%%%%%%%%%%%%%%%%%%%%%%%%
\subsection{Legacy Detection}
\label{sec:detection}

The directive |\childdocmain| in the main file can detect
whether the complete document or merely a child is to be compiled
even without using the directive |\childdocof|.
This method is deprecated because it is less robust
and there is no compelling reason to use it;
it is merely provided for backward compatibility
and it may be removed in future versions.

If the detection mechanism is to be used,
it is mandatory to correctly specify
the filename of the main file as the argument of |\childdocmain|:
%
\begin{center}
\begin{tabular}{l}
|\input{childdoc.def}|\\
|\childdocmain{|\textit{main}|}|\\
\end{tabular}
\end{center}
%
If |\jobname| does not match the argument \textit{main} of |\childdocmain|,
it is assumed that |\jobname| points to the child file to be compiled.
When using |\childdocmain| with the main file specified as argument,
it suffices to start a child file
with just |\input{|\textit{main}|}|
without loading of the package and using |\childdocof|.
If instead all processing is done
with the appropriate \textsf{childdoc} directives,
the argument of \textit{main} of |\childdocmain| can be empty.

An alternative version of the command line processing described
in \secref{sec:commandline} using the detection mechanism reads:
%
\begin{center}
|... -jobname "|\textit{target}|" "|[\textit{flags}]%
[|\def\jobname{|\textit{dest}|}|]|\input{|\textit{main}|}"|
\end{center}

%%%%%%%%%%%%%%%%%%%%%%%%%%%%%%%%%%%%%%%%%%%%%%%%%%%%%%%%%%%%%%%%%%%%%%%%%%%%%%%%
\subsection{Manual Code}
\label{sec:manual}

In case one cannot be certain whether the definitions file |childdoc.def|
is installed on the target \TeX{} distribution
and one prefers not to ship it,
it is conceivable to paste a few relevant commands into the sources.

To that end, drop all statements |\input{childdoc.def}|
and perform the replacements as outlined below.
Instead of |\childdocmain{|\textit{main}|}| add the following code
to the top of the main file:
%
\begin{center}
\begin{tabular}{l}
|\||ifdefined\childdocname\endinput\||fi\newif\ifchilddoc|\\
|\edef\childdocname{\scantokens\expandafter{\jobname\noexpand}}|\\
|\def\childdocmain{|\textit{main}|}\||ifx\childdocmain\childdocname\||else|\\
|\childdoctrue\includeonly{\childdocname}\let\jobname\childdocmain\||fi|\\
\end{tabular}
\end{center}
%
Instead of |\childdocof{|\textit{main}|}| just include the main file
at the top of each child file:
%
\begin{center}
|\input{|\textit{main}|}|
\end{center}
%
A simple redirection |\childdocforward{|\textit{dest}|}| is achieved by:
%
\begin{center}
|\def\jobname{|\textit{dest}|}\input{\jobname}|
\end{center}
%
The redirection with prefix
|\childdocforwardprefix[|\textit{prefix}|]{|\textit{dest}|}|
is accomplished by:
%
\begin{center}
\begin{tabular}{l}
|{\edef\jobname{\scantokens\expandafter{\jobname\noexpand}}|\\
|\def\redirectjob |\textit{prefix}|#1~~~{\gdef\jobname{|\textit{dest}|#1}}|\\
|\expandafter\redirectjob\jobname~~~}\input{\jobname}|
\end{tabular}
\end{center}

In an alternative approach,
child documents can be compiled by a specific command line
without additional code or specific definitions:
%
\begin{center}
|... -jobname "|\textit{target}|" "|[\textit{flags}]%
|\includeonly{|\textit{dest}|}\input{|\textit{main}|}"|
\end{center}
%

%%%%%%%%%%%%%%%%%%%%%%%%%%%%%%%%%%%%%%%%%%%%%%%%%%%%%%%%%%%%%%%%%%%%%%%%%%%%%%%%
%%%%%%%%%%%%%%%%%%%%%%%%%%%%%%%%%%%%%%%%%%%%%%%%%%%%%%%%%%%%%%%%%%%%%%%%%%%%%%%%
\section{Information}

%%%%%%%%%%%%%%%%%%%%%%%%%%%%%%%%%%%%%%%%%%%%%%%%%%%%%%%%%%%%%%%%%%%%%%%%%%%%%%%%
\subsection{Copyright}

Copyright \copyright{} 2017--2018 Niklas Beisert

This work may be distributed and/or modified under the
conditions of the \LaTeX{} Project Public License, either version 1.3
of this license or (at your option) any later version.
The latest version of this license is in
  \url{http://www.latex-project.org/lppl.txt}
and version 1.3 or later is part of all distributions of \LaTeX{}
version 2005/12/01 or later.

This work has the LPPL maintenance status `maintained'.

The Current Maintainer of this work is Niklas Beisert.

This work consists of the files |README.txt|, |childdoc.ins| and |childdoc.dtx|
as well as the derived files |childdoc.def|, |cdocsamp.tex|
with |cdocsch1.tex|, |cdocsch2.tex|, |cdocspt3.tex|, |cdocspt4.tex|,
|cdocsdrf.tex|, |cdocsfn1.tex|, |cdocsfn2.tex|
as well as |childdoc.pdf|.

%%%%%%%%%%%%%%%%%%%%%%%%%%%%%%%%%%%%%%%%%%%%%%%%%%%%%%%%%%%%%%%%%%%%%%%%%%%%%%%%
\subsection{Files and Installation}

The package consists of the files:
%
\begin{center}
\begin{tabular}{ll}
    |README.txt|   & readme file \\
    |childdoc.ins| & installation file \\
    |childdoc.dtx| & source file \\
    |childdoc.def| & definition file \\
    |cdocsamp.tex| & sample main file \\
    |cdocsch1.tex| & sample include file \\
    |cdocsch2.tex| & sample include file \\
    |cdocspt3.tex| & sample part file \\
    |cdocspt4.tex| & sample part file \\
    |cdocsdrf.tex| & sample redirection file \\
    |cdocsfn1.tex| & sample redirection file \\
    |cdocsfn2.tex| & sample redirection file \\
    |childdoc.pdf| & manual
\end{tabular}
\end{center}
%
The distribution consists of the files
|README.txt|, |childdoc.ins| and |childdoc.dtx|.
%
\begin{itemize}
\item
Run (pdf)\LaTeX{} on |childdoc.dtx|
to compile the manual |childdoc.pdf| (this file).
\item
Run \LaTeX{} on |childdoc.ins| to create the definitions file |childdoc.def|
and the sample |cdocsamp.tex| with include files
|cdocsch1.tex|, |cdocsch2.tex|, |cdocspt3.tex|, |cdocspt4.tex|,
|cdocsdrf.tex|, |cdocsfn1.tex|, |cdocsfn2.tex|.
Then copy the file |childdoc.def| to an appropriate directory of your \LaTeX{}
distribution, e.g.\ \textit{texmf-root}|/tex/latex/childdoc|.
\end{itemize}

%%%%%%%%%%%%%%%%%%%%%%%%%%%%%%%%%%%%%%%%%%%%%%%%%%%%%%%%%%%%%%%%%%%%%%%%%%%%%%%%
\subsection{Related CTAN Packages}

There are several other packages which offer a similar functionality:
%
\begin{itemize}
\item
The packages
\href{http://ctan.org/pkg/docmute}{\textsf{docmute}},
\href{http://ctan.org/pkg/includex}{\textsf{includex}} and
\href{http://ctan.org/pkg/standalone}{\textsf{standalone}}
provide commands to include only the document body of
a child file thus allowing both files to be compiled individually.
\item
The packages \href{http://ctan.org/pkg/subdocs}{\textsf{subdocs}}
and \href{http://ctan.org/pkg/subfiles}{\textsf{subfiles}}
provide structures in which the main and child documents can be
encapsulated and allowing them to be compiled individually.
The inclusion mechanism is different from the conventional |\include|.
\item
The package \href{http://ctan.org/pkg/combine}{\textsf{combine}}
is an elaborate solution to combine several documents into one.
\end{itemize}
%
See also the CTAN topic \href{http://ctan.org/topic/subdocs}{\textsf{subdocs}}
for further related packages.
The present package differs from the above solutions in that
a document structure constructed with the conventional |\include| mechanism
just needs two extra commands at the top of every file
such that all constituent files can be compiled individually.

%%%%%%%%%%%%%%%%%%%%%%%%%%%%%%%%%%%%%%%%%%%%%%%%%%%%%%%%%%%%%%%%%%%%%%%%%%%%%%%%
%\subsection{Feature Suggestions}
%
%The following is a list of features which may be useful for future
%versions of this package:
%%
%\begin{itemize}
%\item
%\ldots
%\end{itemize}

%%%%%%%%%%%%%%%%%%%%%%%%%%%%%%%%%%%%%%%%%%%%%%%%%%%%%%%%%%%%%%%%%%%%%%%%%%%%%%%%
\subsection{Revision History}

%%%%%%%%%%%%%%%%%%%%%%%%%%%%%%%%%%%%%%%%
\paragraph{v2.0:} 2018/12/30

\begin{itemize}
\item
immediate forward processing
\item
added |\childdocby| mechanism
\item
manual restructured
\end{itemize}

%%%%%%%%%%%%%%%%%%%%%%%%%%%%%%%%%%%%%%%%
\paragraph{v1.6:} 2018/01/17

\begin{itemize}
\item
application for development of include files
\item
corrections to manual
\end{itemize}

%%%%%%%%%%%%%%%%%%%%%%%%%%%%%%%%%%%%%%%%
\paragraph{v1.5:} 2017/05/21

\begin{itemize}
\item
more complete structuring introduced
\item
|\childdocof| introduced
\item
|\childdoc| renamed to |\childdocmain|
\item
|\childredirect| renamed to |\childdocforward| and |\childdocforwardprefix|
and functionality expanded
\end{itemize}

%%%%%%%%%%%%%%%%%%%%%%%%%%%%%%%%%%%%%%%%
\paragraph{v1.0:} 2017/04/27

\begin{itemize}
\item
manual and install package
\item
first version published on CTAN
\end{itemize}

%%%%%%%%%%%%%%%%%%%%%%%%%%%%%%%%%%%%%%%%
\paragraph{v0.6:} 2017/04/26

\begin{itemize}
\item
redirection mechanism added
\end{itemize}

%%%%%%%%%%%%%%%%%%%%%%%%%%%%%%%%%%%%%%%%
\paragraph{v0.5:} 2017/04/26

\begin{itemize}
\item
functionality in definition file
\end{itemize}


%%%%%%%%%%%%%%%%%%%%%%%%%%%%%%%%%%%%%%%%%%%%%%%%%%%%%%%%%%%%%%%%%%%%%%%%%%%%%%%%
%%%%%%%%%%%%%%%%%%%%%%%%%%%%%%%%%%%%%%%%%%%%%%%%%%%%%%%%%%%%%%%%%%%%%%%%%%%%%%%%
%%%%%%%%%%%%%%%%%%%%%%%%%%%%%%%%%%%%%%%%%%%%%%%%%%%%%%%%%%%%%%%%%%%%%%%%%%%%%%%%
\appendix

\settowidth\MacroIndent{\rmfamily\scriptsize 000\ }

 \DocInput{childdoc.dtx}

\end{document}
%</driver>
% \fi
%
% %%%%%%%%%%%%%%%%%%%%%%%%%%%%%%%%%%%%%%%%%%%%%%%%%%%%%%%%%%%%%%%%%%%%%%%%%%%%%%
% %%%%%%%%%%%%%%%%%%%%%%%%%%%%%%%%%%%%%%%%%%%%%%%%%%%%%%%%%%%%%%%%%%%%%%%%%%%%%%
% \section{Sample}
%\iffalse
%<*samplemain>
%\fi
%
% The following presents a sample document
% with two chapters, two parts, a title page,
% a compile flag as well as three forwarding files to set the flag.
% It consists of eight |.tex| files:
% \begin{center}
% \begin{tabular}{ll}
% |cdocsamp.tex|&main file\\
% |cdocsch1.tex|&include file for chapter 1\\
% |cdocsch2.tex|&include file for chapter 2\\
% |cdocspt3.tex|&include file for part 3\\
% |cdocspt4.tex|&include file for part 4\\
% |cdocsdrf.tex|&forwarding file for main file in draft mode\\
% |cdocsfi1.tex|&forwarding file for final version of chapter 1\\
% |cdocsfi2.tex|&forwarding file for final version of chapter 2\\
% \end{tabular}
% \end{center}
% Each of the eight files can be compiled directly by the \LaTeX{} compiler.
%
% %%%%%%%%%%%%%%%%%%%%%%%%%%%%%%%%%%%%%%
% \paragraph{Main File.}
%
% The main file is called |cdocsamp.tex|.
%
% Load the \textsf{childdoc} definitions and
% declare the filename for the main document:
%    \begin{macrocode}
\input{childdoc.def}
\childdocmain{}
%    \end{macrocode}

% Optional override for |\version| flag:
%    \begin{macrocode}
%%\ifchilddoc\else\providecommand{\version}{draft}\fi
%    \end{macrocode}

% Define the default values for the |\version| flag
% (|final| for the main file and |draft| for childs):
%    \begin{macrocode}
\ifchilddoc
\providecommand{\version}{draft}
\else
\providecommand{\version}{final}
\fi
%    \end{macrocode}

% Load the standard document class:
%    \begin{macrocode}
\documentclass[12pt]{article}
%    \end{macrocode}

% Start the document body:
%    \begin{macrocode}
\begin{document}
%    \end{macrocode}

% Declare a title page.
% Print title, part of document being processed and version flag:
%    \begin{macrocode}
\addtocounter{page}{-1}
\begin{center}
{\LARGE\bfseries{}childdoc example\par}
\vspace{1cm}
\ifchilddoc
\ifchilddocmanual part\else chapter\fi:
`\childdocname' of `\childdocjob'\par
\else
main document: `\childdocjob'\par
\fi
version: \version\par
\end{center}
\newpage
%    \end{macrocode}

% Manually include selected file,
% otherwise process as usual:
%    \begin{macrocode}
\ifchilddocmanual
\section*{part `\childdocname'}
\input{\childdocname}
\else
%    \end{macrocode}

% Include the two chapters:
%    \begin{macrocode}
\include{cdocsch1}
\include{cdocsch2}
%    \end{macrocode}

% Include the two parts unless only chapters should be displayed:
%    \begin{macrocode}
\ifchilddoc\else
\section{part three}
\input{cdocspt3}
\section{part four}
\input{cdocspt4}
\fi
%    \end{macrocode}

% Process as usual until here:
%    \begin{macrocode}
\fi
%    \end{macrocode}

% End of document body:
%    \begin{macrocode}
\end{document}
%    \end{macrocode}
%\iffalse
%</samplemain>
%\fi
%
% %%%%%%%%%%%%%%%%%%%%%%%%%%%%%%%%%%%%%%
% \paragraph{Chapter Include Files.}
%
% The include files are called |cdocsch1.tex| and |cdocsch2.tex|.
%
%\iffalse
%<*samplechap1|samplechap2>
%\fi

% Optional override for |\version| flag:
%    \begin{macrocode}
%%\providecommand{\version}{final}
%    \end{macrocode}

% Include the main document:
%    \begin{macrocode}
\input{childdoc.def}
\childdocof{cdocsamp}
%    \end{macrocode}

%\iffalse
%</samplechap1|samplechap2>
%\fi
%
%\iffalse
%<*samplechap1>
%\fi
% Some text for chapter 1:
%    \begin{macrocode}
\section{one}
some text in chapter one
%    \end{macrocode}

%\iffalse
%</samplechap1>
%\fi
% Some text for chapter 2:
%\iffalse
%<*samplechap2>
%\fi
%    \begin{macrocode}
\section{two}
more text in chapter two
%    \end{macrocode}

%\iffalse
%</samplechap2>
%\fi
%
% %%%%%%%%%%%%%%%%%%%%%%%%%%%%%%%%%%%%%%
% \paragraph{Part Include Files.}
%
% The include files are called |cdocspt3.tex| and |cdocspt4.tex|.
%
%\iffalse
%<*samplepart3|samplepart4>
%\fi

% Optional override for |\version| flag:
%    \begin{macrocode}
%%\providecommand{\version}{final}
%    \end{macrocode}

% Include the main document:
%    \begin{macrocode}
\input{childdoc.def}
\childdocby{cdocsamp}
%    \end{macrocode}

%\iffalse
%</samplepart3|samplepart4>
%\fi
%
%\iffalse
%<*samplepart3>
%\fi
% Some text for part 3:
%    \begin{macrocode}
some text in part three
%    \end{macrocode}

%\iffalse
%</samplepart3>
%\fi
% Some text for part 4:
%\iffalse
%<*samplepart4>
%\fi
%    \begin{macrocode}
more text in part four
%    \end{macrocode}

%\iffalse
%</samplepart4>
%\fi
%
% %%%%%%%%%%%%%%%%%%%%%%%%%%%%%%%%%%%%%%
% \paragraph{Forwarding for a Complete Draft.}
%
% The following forwarding file |cdocsdrf.tex|
% compiles the main document in draft mode:
%\iffalse
%<*sampledraft>
%\fi
%    \begin{macrocode}
\def\version{draft}
\input{childdoc.def}
\childdocforward{cdocsamp}
%    \end{macrocode}

%\iffalse
%</sampledraft>
%\fi
%
% %%%%%%%%%%%%%%%%%%%%%%%%%%%%%%%%%%%%%%
% \paragraph{Forwarding for Final Version of the Chapters.}
%
% The following forwarding files |cdocsfn1.tex| and |cdocsfn2.tex|
% (with identical content)
% compile the final versions of the child documents
% |cdocsch1.tex| and |cdocsch2.tex|, respectively:
%\iffalse
%<*samplefinal>
%\fi
%    \begin{macrocode}
\def\version{final}
\input{childdoc.def}
\childdocforwardprefix[cdocsamp]{cdocsfn}{cdocsch}
%    \end{macrocode}

%\iffalse
%</samplefinal>
%\fi
%
% %%%%%%%%%%%%%%%%%%%%%%%%%%%%%%%%%%%%%%
% \paragraph{Command Line Processing.}
%
% The following three command lines generate the output files
% |cdocscld|, |cdocscl1| and |cdocscl2|
% which should be identical to
% |cdocsdrf|, |cdocsch1| and |cdocsfn2|, respectively:
% \begin{center}
% \begin{tabular}{l}
% |latex -jobname cdocscld \|\\
% |  "\def\version{draft}\input{childdoc.def}\childdocforward{cdocsamp}"|\\
% |latex -jobname cdocscl1 \|\\
% |  "\input{childdoc.def}\childdocforward[cdocsamp]{cdocsch1}"|\\
% |latex -jobname cdocscl2 \|\\
% |  "\def\version{final}\input{childdoc.def}\childdocforward{cdocsch2}"|
% \end{tabular}
% \end{center}
% Note that the trailing backslash on each first line
% merely continues the input to the second line
% (for convenient cut ant paste).
% Furthermore, the command |latex| can be replaced by any
% of its alternative versions such as |pdflatex|.
%
% %%%%%%%%%%%%%%%%%%%%%%%%%%%%%%%%%%%%%%%%%%%%%%%%%%%%%%%%%%%%%%%%%%%%%%%%%%%%%%
% %%%%%%%%%%%%%%%%%%%%%%%%%%%%%%%%%%%%%%%%%%%%%%%%%%%%%%%%%%%%%%%%%%%%%%%%%%%%%%
% \section{Implementation}
%\iffalse
%<*package>
%\fi
%
% This section describes the definitions file |childdoc.def|.

% The definitions cannot be loaded using |\usepackage| or |\RequirePackage|
% which has a mechanism to prevent loading a style file more than once.
% When loading the definitions by means of |\input|
% multiple instances have to be prevented manually:
%\iffalse
%This code needs to be before the `\ProvidesFile' directive
%which is defined at the beginning of this file.
%Therefore it is also placed there and commented out here.
%</package>
%<*discard>
%\fi
%    \begin{macrocode}
\ifdefined\childdocmain\endinput\fi
%    \end{macrocode}
%\iffalse
%</discard>
%<*package>
%\fi
%
% \macro{\ifchilddoc}
% \macro{\ifchilddocmanual}
% The conditional |\ifchilddoc| tells whether a
% child (true) or main (false) document is being compiled.
% The conditional |\ifchilddocmanual| tells whether
% the |\includeonly| mechanism is used (false) or
% the selection of child files must be performed manually (true).
% The definitions initialise to false:
%    \begin{macrocode}
\newif\ifchilddoc
\newif\ifchilddocmanual
%    \end{macrocode}

% \macro{\childdocname}
% \macro{\childdocjob}
% The macro |\childdocname| stores the name of the main document
% to be compiled. The macro |\childdocjob| stores the name of
% the document on which the \LaTeX{} compiler was originally invoked.
% The content of |\jobname| cannot be compared
% to filenames specified in the source due to different catcodes.
% The following code rescans |\jobname|, stores the result
% in |\childdocname| and saves a copy in |\childdocjob|:
%    \begin{macrocode}
\edef\childdocname{\scantokens\expandafter{\jobname\noexpand}}
\let\childdocjob\childdocname
%    \end{macrocode}

% \macro{\childdocdisable}
% The macro |\childdocdisable| prevents the main file
% from being processed more than once.
% At this stage, the main document command |\childdocmain|
% is assumed to be called once again where it should do nothing.
% Any subsequent call to it should prevent
% a secondary processing of the main document
% It overwrites the forwarding commands
% |\childdocof| and |\childdocforward|
% with empty macros to prevent further inclusions of the main document:
%    \begin{macrocode}
\newcommand{\childdocdisable}
{
  \renewcommand{\childdocmain}[1]{\renewcommand{\childdocmain}[1]{\endinput}}
  \renewcommand{\childdocof}[1]{}
  \renewcommand{\childdocby}[2][]{}
  \renewcommand{\childdocforward}[2][]{}
  \renewcommand{\childdocdisable}{}
}
%    \end{macrocode}

% \macro{\childdocmain}
% The macro |\childdocmain| is to be called at the top of the main file
% with nothing or the main filename (without extension) as argument.
% First, it breaks loops.
% If the argument is not empty and does not match |\childdocname|
% (which is set by the first inclusion of |childdoc.def|),
% |\ifchilddoc| is set to true, |\includeonly| is applied to the child file
% and |\jobname| is set to the main file
% (for proper handling of |.aux| files):
%    \begin{macrocode}
\newcommand{\childdocmain}[1]
{
  \childdocdisable\childdocmain{}
  \if?#1?\else
    \begingroup
      \def\childdoctmp{#1}
      \ifx\childdoctmp\childdocname
        \def\childdoctmp{}
      \else
        \def\childdoctmp
        {
          \childdoctrue
          \includeonly{\childdocname}
          \def\childdocjob{#1}
          \def\jobname{#1}
        }
      \fi
      \expandafter
    \endgroup
    \childdoctmp
  \fi
}
%    \end{macrocode}

% \macro{\childdocof}
% The command |\childdocof| redirects
% compilation to the main file |#1|.
%    \begin{macrocode}
\newcommand{\childdocof}[1]
{
  \childdocdisable
  \childdoctrue
  \includeonly{\childdocname}
  \def\jobname{#1}
  \def\childdocjob{#1}
  \input{#1}
}
%    \end{macrocode}

% \macro{\childdocby}
% The command |\childdocby| ....
%    \begin{macrocode}
\newcommand{\childdocby}[2][]
{
  \childdocdisable
  \childdoctrue
  \childdocmanualtrue
  \if?#1?\else
    \def\jobname{#2}
  \fi
  \def\childdocjob{#2}
  \input{#2}
  \endinput
}
%    \end{macrocode}

% \macro{\childdocforward}
% The command |\childdocforward| redirects
% compilation to the main file or
% (if the optional argument is given) a child file.
% Parameters are set as if the main file
% or a child file starting with |\childdocof| was compiled.
% Then compilation is handed over to the main file:
%    \begin{macrocode}
\newcommand{\childdocforward}[2][]
{
  \begingroup
    \if?#1?
      \def\childdoctmp
      {
        \def\childdocname{#2}
        \def\childdocjob{#2}
        \def\jobname{#2}
        \input{#2}
        \endinput
      }
    \else
      \def\childdoctmp
      {
        \childdocdisable
        \def\childdocname{#2}
        \childdoctrue
        \includeonly{#2}
        \def\childdocjob{#1}
        \def\jobname{#1}
        \input{#1}
        \endinput
      }
    \fi
    \expandafter
  \endgroup
  \childdoctmp
}
%    \end{macrocode}

% \macro{\childdocforwardprefix}
% The command |\childdocforwardprefix| redirects
% compilation to the main or a child file by means of a pattern.
% The prefix |#1| in the current filename is replaced by |#2|
% and the suffix of the current filename is kept
% (it is assumed that the filename does not contain the substring `|~~~|'
% which is used as a delimiter).
% Compilation is handed over to the new file by |\childdocforward|:
%    \begin{macrocode}
\newcommand{\childdocforwardprefix}[3][]
{
  \begingroup
    \def\childdocextract #2##1~~~{\def\childdoctmp{\childdocforward[#1]{#3##1}}}
    \expandafter\childdocextract\childdocname~~~
    \expandafter
  \endgroup
  \childdoctmp
}
%    \end{macrocode}

% \macro{\childdoc}
% The deprecated macro |\childdoc| is a legacy version of |\childdocmain|:
%    \begin{macrocode}
\newcommand{\childdoc}{\childdocmain}
%    \end{macrocode}

% \macro{\childdocredirect}
% The deprecated macro |\childdocredirect| is a legacy version
% of |\childdocforward| and |\childdocforwardprefix|:
%    \begin{macrocode}
\newcommand{\childdocredirect}[2][]
{
  \begingroup
    \if?#1?
      \def\childdoctmp{\childdocforward{#2}}
    \else
      \def\childdoctmp{\childdocforwardprefix{#1}{#2}}
    \fi
    \expandafter
  \endgroup
  \childdoctmp
}
%    \end{macrocode}

%\iffalse
%</package>
%\fi
%
\endinput

\childdocforwardprefix[cdocsamp]{cdocsfn}{cdocsch}
%    \end{macrocode}

%\iffalse
%</samplefinal>
%\fi
%
% %%%%%%%%%%%%%%%%%%%%%%%%%%%%%%%%%%%%%%
% \paragraph{Command Line Processing.}
%
% The following three command lines generate the output files
% |cdocscld|, |cdocscl1| and |cdocscl2|
% which should be identical to
% |cdocsdrf|, |cdocsch1| and |cdocsfn2|, respectively:
% \begin{center}
% \begin{tabular}{l}
% |latex -jobname cdocscld \|\\
% |  "\def\version{draft}% \iffalse
%
% childdoc.dtx Copyright (C) 2017-2018 Niklas Beisert
%
% This work may be distributed and/or modified under the
% conditions of the LaTeX Project Public License, either version 1.3
% of this license or (at your option) any later version.
% The latest version of this license is in
%   http://www.latex-project.org/lppl.txt
% and version 1.3 or later is part of all distributions of LaTeX
% version 2005/12/01 or later.
%
% This work has the LPPL maintenance status `maintained'.
%
% The Current Maintainer of this work is Niklas Beisert.
%
% This work consists of the files childdoc.dtx and childdoc.ins
% and the derived files childdoc.def and cdocsamp.tex with
% cdocsch1.tex, cdocsch2.tex, cdocsdrf.tex, cdocsfn1.tex, cdocsfn2.tex.
%
%<package>\ifdefined\childdocmain\endinput\fi
%<package>\ProvidesFile{childdoc.def}[2018/12/30 v2.0 child document driver]
%<samplemain>\ProvidesFile{cdocsamp.tex}[2018/12/30 v2.0 sample for childdoc]
%<*driver>
%\ProvidesFile{childdoc.drv}[2018/12/30 v2.0 childdoc reference manual file]
\PassOptionsToClass{10pt,a4paper}{article}
\documentclass{ltxdoc}

\usepackage[margin=35mm]{geometry}
\usepackage{hyperref}
\usepackage{hyperxmp}
\usepackage[usenames]{color}

\hypersetup{colorlinks=true}
\hypersetup{pdfstartview=FitH}
\hypersetup{pdfpagemode=UseNone}
\hypersetup{pdfsource={}}
\hypersetup{pdflang={en-UK}}
\hypersetup{pdfcopyright={Copyright 2017-2018 Niklas Beisert.
  This work may be distributed and/or modified under the
  conditions of the LaTeX Project Public License, either version 1.3
  of this license or (at your option) any later version.}}
\hypersetup{pdflicenseurl={http://www.latex-project.org/lppl.txt}}
\hypersetup{pdfcontactaddress={ETH Zurich, ITP, HIT K,
  Wolfgang-Pauli-Strasse 27}}
\hypersetup{pdfcontactpostcode={8093}}
\hypersetup{pdfcontactcity={Zurich}}
\hypersetup{pdfcontactcountry={Switzerland}}
\hypersetup{pdfcontactemail={nbeisert@itp.phys.ethz.ch}}
\hypersetup{pdfcontacturl={http://people.phys.ethz.ch/\xmptilde nbeisert/}}

\newcommand{\secref}[1]{\hyperref[#1]{section \ref*{#1}}}

\parskip1ex
\parindent0pt
\let\olditemize\itemize
\def\itemize{\olditemize\parskip0pt}

\begin{document}

\title{The \textsf{childdoc} Package}
\hypersetup{pdftitle={The childdoc Package}}
\author{Niklas Beisert\\[2ex]
  Institut f\"ur Theoretische Physik\\
  Eidgen\"ossische Technische Hochschule Z\"urich\\
  Wolfgang-Pauli-Strasse 27, 8093 Z\"urich, Switzerland\\[1ex]
  \href{mailto:nbeisert@itp.phys.ethz.ch}
  {\texttt{nbeisert@itp.phys.ethz.ch}}}
\hypersetup{pdfauthor={Niklas Beisert}}
\hypersetup{pdfsubject={Manual for the LaTeX2e Package childdoc}}
\date{30 December 2018, \textsf{v2.0}}
\maketitle

\begin{abstract}\noindent
\textsf{childdoc} is a \LaTeXe{} package
that enables the direct compilation
of document sections included by |\include|
to individual files.
\end{abstract}

\begingroup
\parskip0ex
\tableofcontents
\endgroup

%%%%%%%%%%%%%%%%%%%%%%%%%%%%%%%%%%%%%%%%%%%%%%%%%%%%%%%%%%%%%%%%%%%%%%%%%%%%%%%%
%%%%%%%%%%%%%%%%%%%%%%%%%%%%%%%%%%%%%%%%%%%%%%%%%%%%%%%%%%%%%%%%%%%%%%%%%%%%%%%%
\section{Introduction}

\LaTeX{} provides a mechanism to structure a large document (such as a book)
into a main file and several child files (containing the chapters)
using the |\include| command.
This mechanism is beneficial for documents
which span hundreds of pages in order to
make the source file(s) more manageable.
Moreover, compilation can be restricted to
selected child files by means of the |\includeonly| command.
The latter feature can be used to reduce the compilation time while editing
(this was significantly more useful in the earlier days of \LaTeX{})
or to generate a smaller document which is easier to navigate.
Another application of |\includeonly| is to generate
documents consisting of selected parts of the complete document.

However, there are a few drawbacks of the plain |\include| mechanism:
\begin{itemize}
\item
The child files cannot be compiled on their own,
they can only be compiled via the main file.
A naive editing environment
(such as a text editor with an option
to have the current file processed by \LaTeX)
may require one to switch to the main file before compiling;
attempting to compile the child file produces errors.
\item
The main file must be modified (each time)
to adjust the |\includeonly| command
to the present needs. This easily leaves the main file in a messy state.
\item
The generated document will always carry the filename
of the main document. This is inconvenient if
several child files are to be compiled and
to be kept for distribution.
\end{itemize}

The present package provides a simple interface
to make child files individually compilable by \LaTeX{}.
Compiling a child file then has the same effect as compiling
the main file with an |\includeonly| command
to select the appropriate child.
Moreover the generated document will carry the name of the child
rather than the main file.
This resolves all three above issues.

This feature is meant to make the editing of books,
thesis documents and lecture notes somewhat more convenient.
However, the package can also be used efficiently for
composing a series of documents (such as exercise sheets)
which are typically distributed individually.
It then assists the author in generating the individual documents
(potentially in different versions)
as well as a document containing the collected series.
Another application is in developing style files
or other kinds of included material
where compilation of the style file could redirect
to a sample or test file.

%%%%%%%%%%%%%%%%%%%%%%%%%%%%%%%%%%%%%%%%%%%%%%%%%%%%%%%%%%%%%%%%%%%%%%%%%%%%%%%%
%%%%%%%%%%%%%%%%%%%%%%%%%%%%%%%%%%%%%%%%%%%%%%%%%%%%%%%%%%%%%%%%%%%%%%%%%%%%%%%%
\section{Usage}

First of all, the package \textsf{childdoc} is \emph{not} a standard
\LaTeXe{} |.sty| style file! Therefore it needs to be invoked in
a non-standard way.

%%%%%%%%%%%%%%%%%%%%%%%%%%%%%%%%%%%%%%%%%%%%%%%%%%%%%%%%%%%%%%%%%%%%%%%%%%%%%%%%
\subsection{Included Files}
\label{sec:include}

%%%%%%%%%%%%%%%%%%%%%%%%%%%%%%%%%%%%%%%%
\DescribeMacro{\childdocmain}
To use the package, add the commands
\begin{center}
\begin{tabular}{l}
|\input{childdoc.def}|\\
|\childdocmain{}|\\
\end{tabular}
\end{center}
at the very top of the main \LaTeX{} file,
in particular \emph{before} the |\documentclass| statement!
The argument of |\childdocmain| should be left empty
(but it must be present).

%%%%%%%%%%%%%%%%%%%%%%%%%%%%%%%%%%%%%%%%
\DescribeMacro{\childdocof}
Furthermore, add the commands
\begin{center}
\begin{tabular}{l}
|\input{childdoc.def}|\\
|\childdocof{|\textit{main}|}|\\
\end{tabular}
\end{center}
at the top of every child file \textit{child}
which is included by |\include{|\textit{child}|}|
from within the main file
(or at least for those files to be compiled individually).
The argument \textit{main} must be the filename of the main file.

There are a couple of
considerations in setting up the main and child documents:

%%%%%%%%%%%%%%%%%%%%%%%%%%%%%%%%%%%%%%%%
\paragraph{Restrictions.}

Please note the following restrictions:
\begin{itemize}
\item
|\childdocmain| must be called with one argument \textit{main}
to ensure compatibility with earlier version of the package.
It must either be empty (|\childdocmain{}|)
or precisely match the filename of the main file in which it is specified.
See \secref{sec:detection} for further information.
\item
The filename \textit{main} must be specified without the |.tex| extension.
\item
The filename \textit{main} is case sensitive
(even in case-insensitive file systems)
due to internal string comparison.
\item
The argument \textit{main} should be fully expanded, it cannot be a macro.
\item
Subdirectories and special characters should be avoided in filenames.
\item
The command |\childdocmain{|\textit{main}|}| must be followed by a whitespace.
It should not be followed immediately by another command
or by a comment mark `|%|'.
This is because the \TeX{} parser reads the token immediately following
the argument of |\childdocmain| and puts it
at the beginning of every child section;
however, a white\-space is ignored.
\end{itemize}

%%%%%%%%%%%%%%%%%%%%%%%%%%%%%%%%%%%%%%%%
\paragraph{Content of Main File.}

It is advisable to place all content in the child files included by |\include|.
Any output contained in the main file will appear in all child documents
unless suppressed manually;
it cannot be suppressed automatically by the |\includeonly| directive
and thus should normally be avoided.
A method to include some content in the main file
by means of conditional processing is described in \secref{sec:conditional}.

%%%%%%%%%%%%%%%%%%%%%%%%%%%%%%%%%%%%%%%%
\paragraph{Page Numbering.}

When only a part of the document is compiled,
the appropriate numbering of pages
(as well as other status parameters)
is determined from the |.aux| files.
The latter contain information from previous passes.
However this information needs to propagate through
all intermediate child documents.
Therefore the page numbering in child documents may well
be inconsistent until the complete document is compiled at least once.

A useful (if unconventional) way to always ensure a consistent
page numbering is to restart the numbering in each child document
and denote the pages by `\textit{child}|.|\textit{page}'
where \textit{child} represents the chapter/section number of the child file.
This can be achieved by the command
|\numberwithin{page}{|\textit{child}|}|
of the \textsf{amsmath} package
where \textit{child} can be |chapter| or |section|
depending on the chosen structuring.
Alternatively, one can modify the macro |\thepage| appropriately
and reset the counter |page| at the start of each child file.

%%%%%%%%%%%%%%%%%%%%%%%%%%%%%%%%%%%%%%%%%%%%%%%%%%%%%%%%%%%%%%%%%%%%%%%%%%%%%%%%
\subsection{Conditional Processing}
\label{sec:conditional}

The package provides a mechanism to compile different versions
of a document. To customise the versions further some conditional processing
can come in handy to distinguish which version is being compiled.
The package provides two macros to describe the compilation context:

%%%%%%%%%%%%%%%%%%%%%%%%%%%%%%%%%%%%%%%%
\DescribeMacro{\ifchilddoc}
The conditional |\ifchilddoc| distinguishes between the compilation of
child documents and the main document:
%
\begin{center}
|\ifchilddoc |\textit{child-code}| |[|\||else |\textit{main-code}]| \||fi|
\end{center}

%%%%%%%%%%%%%%%%%%%%%%%%%%%%%%%%%%%%%%%%
\DescribeMacro{\childdocname}
\DescribeMacro{\childdocjob}
The macro |\childdocname| contains the filename (without extension)
of the main or child file being processed.
Note that |\childdocjob| will always contain the name of the main file.

%%%%%%%%%%%%%%%%%%%%%%%%%%%%%%%%%%%%%%%%
\paragraph{Title Page.}

Conditional processing can be used to include a title or banner page
in the main document when proper precautions are taken.
Importantly, the code in the main file should ensure that the page counter
(as well as other status parameters which are stored in the |.aux| files)
takes the same value after the conditional processing.
Otherwise the page numbers may take divergent values
depending on which part is compiled.

For example, a title page could be declared by:
%
\begin{center}
\begin{tabular}{l}
|\ifchilddoc\||else|\\
|\addtocounter{page}{-1}|\\
\textit{code for title page}\\
|\newpage|\\
|\||fi|
\end{tabular}
\end{center}
%
A banner page for the child documents can be generated by:
%
\begin{center}
\begin{tabular}{l}
|\ifchilddoc|\\
|\addtocounter{page}{-1}|\\
\textit{code for banner page}\\
|\newpage|\\
|\||fi|
\end{tabular}
\end{center}
%
Here one could write a message such as:
\begin{center}
|This is the part \childdocname{} of \childdocjob{}.|
\end{center}

%%%%%%%%%%%%%%%%%%%%%%%%%%%%%%%%%%%%%%%%%%%%%%%%%%%%%%%%%%%%%%%%%%%%%%%%%%%%%%%%
\subsection{Flags}
\label{sec:flags}

The package makes it easy to generate different versions
of the main or child documents.
To this end compilation flags can be defined
and assigned different default values.
They will be particularly useful in conjunction
with the forwarding mechanism described in \secref{sec:forward}.

For example, it may be useful to have a flag |\version|
which can be set to |draft| or |final|.
The document source will contain some conditional code
depending on the value of |\version|.
Suppose further, the flag should default to |final| for the main file
and to |draft| for child files
which is a natural assignment for editing the document.
This is achieved by placing the following code
in the preamble of the main document
(below the |\childdocmain| directive):
%
\begin{center}
\begin{tabular}{l}
|\ifchilddoc|\\
|\providecommand{\version}{draft}|\\
|\||else|\\
|\providecommand{\version}{final}|\\
|\||fi|
\end{tabular}
\end{center}
%
The definition by |\providecommand| makes sure
that previous definitions are not overwritten.
Further statements |\providecommand{\version}{...}|
can thus be added before the above code to override it.

For the main file, one might add a line
(between |\childdocmain| and the above block)
%
\begin{center}
|%\ifchilddoc\||else\providecommand{\version}{draft}\||fi|
\end{center}
%
which can be uncommented to produce a draft version.
Likewise one can add a line to the very top of a child file
(above the |\childdocof{|\textit{main}|}| directive)
%
\begin{center}
|%\providecommand{\version}{final}|
\end{center}
%
which can be uncommented to produce the final version of this child document.

%%%%%%%%%%%%%%%%%%%%%%%%%%%%%%%%%%%%%%%%%%%%%%%%%%%%%%%%%%%%%%%%%%%%%%%%%%%%%%%%
\subsection{Forwarding}
\label{sec:forward}

Different versions of the main or child documents
using compilation flags as described in \secref{sec:flags}
can be (permanently) stored in different files
for convenient compilation, viewing and distribution.
To this end, the package defines a command
to pass on compilation to a different file:

%%%%%%%%%%%%%%%%%%%%%%%%%%%%%%%%%%%%%%%%
\DescribeMacro{\childdocforward}
The command |\childdocforward| redirects processing to
another source file:
%
\begin{center}
\begin{tabular}{l}
|\input{childdoc.def}|\\
|\childdocforward[|\textit{main}|]{|\textit{dest}|}|\\
\end{tabular}
\end{center}
%
The argument \textit{dest} is the destination file
(without extension).
It should be the main file or one of the child files.
Note that further \textsf{childdoc} directives
such as |\childdocof| and |\childdocforward|
in the indicated file will be processed in this form.
The optional argument \textit{main}
passes on directly to the main file \textit{main}
while pretending to compile the child \textit{dest}.
This form behaves as if \textit{dest}
issues |\childdocof{|\textit{main}|}| right away,
and no further \textsf{childdoc} directives will be processed.

%%%%%%%%%%%%%%%%%%%%%%%%%%%%%%%%%%%%%%%%
\DescribeMacro{\...prefix}
In the alternative form |\childdocforwardprefix|,
%
\begin{center}
\begin{tabular}{l}
|\input{childdoc.def}|\\
|\childdocforwardprefix[|\textit{main}|]{|\textit{prefix}|}{|\textit{dest}|}|
\end{tabular}
\end{center}
%
the destination file is determined by a pattern
depending on the current file:
To make this work, the current file must be called
`{\textit{prefix}\hspace{0.2em}\textit{suffix}}'
with \textit{prefix} matching precisely the argument.
Processing is then passed on to the file
`{\textit{dest}\hspace{0.2em}\textit{suffix}}'.
Surely, the same effect is achieved by
directly specifying the
argument `{\textit{dest}\hspace{0.2em}\textit{suffix}}'
in the first form.
However, that requires to set up a different file
for each child. With the alternative form of the command
all these files can have exactly the same content
which simplifies setting them up and maintaining them.

For example, the following file |draft.tex|
with a compilation flag |\version| as described in \secref{sec:flags}
compiles the main document as a draft:
%
\begin{center}
\begin{tabular}{l}
|\def\version{draft}|\\
|\input{childdoc.def}|\\
|\childdocforward{|\textit{main}|}|
\end{tabular}
\end{center}
%
Likewise, the following files |final|\textit{nn}|.tex|
compile the final version of the child document
|child|\textit{nn}|.tex|:
%
\begin{center}
\begin{tabular}{l}
|\def\version{final}|\\
|\input{childdoc.def}|\\
|\childdocforwardprefix{final}{child}|
\end{tabular}
\end{center}
%

Note that when several versions of a main file and/or of each child file
are to be generated, it may be convenient to set up a |Makefile| or
shell script to automatise the process.

%%%%%%%%%%%%%%%%%%%%%%%%%%%%%%%%%%%%%%%%%%%%%%%%%%%%%%%%%%%%%%%%%%%%%%%%%%%%%%%%
\subsection{Command Line Processing}
\label{sec:commandline}

The effect of redirection files can also be achieved by invoking
the \LaTeX{} compiler with a more elaborate command line.
Most conveniently this should be done as part
of a shell script or a |Makefile|.

When using \textsf{childdoc} in the main file, the following
command lines effectively perform a redirection
(note that depending on the shell being used,
backslashes may have to be doubled: `|\|' $\to$ `|\\|'):
%
\begin{center}
|... -jobname "|\textit{target}|" |\\|"|[\textit{flags}]%
|\input{childdoc.def}\childdocforward[|\textit{main}|]{|\textit{dest}|}"|
\end{center}
%
Here \textit{target} is the name of the output file,
\textit{main} is the name of the main file
and \textit{dest} is the name of the main or child file to be processed
(all filenames without extensions).
The optional argument \textit{main} can be omitted
if \textit{main} matches \textit{dest}.
Optionally, compilation \textit{flags} can be defined via |\def| commands.
This command line makes the \TeX{} engine believe
it is compiling the file \textit{target}
whose content is specified as the latter parameter.
The provided code then forwards the processing to
\textit{main} or \textit{dest} as described in \secref{sec:forward}.

%%%%%%%%%%%%%%%%%%%%%%%%%%%%%%%%%%%%%%%%%%%%%%%%%%%%%%%%%%%%%%%%%%%%%%%%%%%%%%%%
\subsection{Include by Input}
\label{sec:input}

Including child documents by |\include| has some restrictions by design.
Most notably, the content of a child document always occupies
its own set of pages; pages cannot be shared between child documents.
Usually, this behaviour makes perfect sense
because each child document contain an essential part of the document.
However, in some situations it may be desirable to compose
a document from a collection of parts
without having mandatory page breaks between then.
For this case, the package
provides a mechanism to include parts
by |\input| which can also be processed individually.
However, by construction this mechanism
requires manual handling of the content to be output.

%%%%%%%%%%%%%%%%%%%%%%%%%%%%%%%%%%%%%%%%
\DescribeMacro{\ifchilddocmanual}
The main file should be prepared as usual, see \secref{sec:include}.
However, the document body must make a distinction
between processing of an individual part and of the main document, e.g.:
%
\begin{center}
\begin{tabular}{l}
|\ifchilddocmanual|\\
|\input{\childdocname}|\\
|\||else|\\
\textit{document body with }|\input{|\textit{part}|}|\\
|\||fi|
\end{tabular}
\end{center}
%
The conditional |\ifchilddocmanual| is true whenever
a part to be included by |\input| is being compiled,
and the name of the part is stored in |\childdocname|.

%%%%%%%%%%%%%%%%%%%%%%%%%%%%%%%%%%%%%%%%
\DescribeMacro{\childdocby}
Each part to be included by |\input| should start with:
%
\begin{center}
\begin{tabular}{l}
|\input{childdoc.def}|\\
|\childdocby{|\textit{main}|}|\\
\end{tabular}
\end{center}
%
The directive |\childdocby| is similar to |\childdocof|
described in \secref{sec:include},
but the subsequent selection of content must be done manually.
To that end, both |\ifchilddoc| and |\ifchilddocmanual|
will be true upon processing of a part,
and the name of the part is stored in |\childdocname|.
Note that |\jobname| will be set to the filename of the current part
so that each part receives an individual |.aux| file
that does not interfere with the |.aux| file(s) of the main document.
This behaviour can be altered by the alternative form
|\childdocby[*]{|\textit{main}|}| (with a non-empty optional argument)
which uses the |.aux| file of the main document
by setting |\jobname| to \textit{main}.

%%%%%%%%%%%%%%%%%%%%%%%%%%%%%%%%%%%%%%%%%%%%%%%%%%%%%%%%%%%%%%%%%%%%%%%%%%%%%%%%
\subsection{Driver Development}
\label{sec:driver}

The \textsf{childdoc} mechanism can also be use for the development
of definition files such as \LaTeX{} styles or classes.
This case differs from the above setup with multiple parts
included by |\include| in that no |\includeonly| should be invoked.
This can be achieved by starting the include file
(before |\ProvidesPackage|) with:
%
\begin{center}
\begin{tabular}{l}
|\input{childdoc.def}|\\
|\childdocforward{|\textit{main}|}|\\
\end{tabular}
\end{center}
%
or alternatively with:
%
\begin{center}
\begin{tabular}{l}
|\input{childdoc.def}|\\
|\childdocby{|\textit{main}|}|\\
\end{tabular}
\end{center}
%
Both forms have slightly different effects as described above.
The main file is prepared as usual, see \secref{sec:include}.

%%%%%%%%%%%%%%%%%%%%%%%%%%%%%%%%%%%%%%%%%%%%%%%%%%%%%%%%%%%%%%%%%%%%%%%%%%%%%%%%
\subsection{Legacy Detection}
\label{sec:detection}

The directive |\childdocmain| in the main file can detect
whether the complete document or merely a child is to be compiled
even without using the directive |\childdocof|.
This method is deprecated because it is less robust
and there is no compelling reason to use it;
it is merely provided for backward compatibility
and it may be removed in future versions.

If the detection mechanism is to be used,
it is mandatory to correctly specify
the filename of the main file as the argument of |\childdocmain|:
%
\begin{center}
\begin{tabular}{l}
|\input{childdoc.def}|\\
|\childdocmain{|\textit{main}|}|\\
\end{tabular}
\end{center}
%
If |\jobname| does not match the argument \textit{main} of |\childdocmain|,
it is assumed that |\jobname| points to the child file to be compiled.
When using |\childdocmain| with the main file specified as argument,
it suffices to start a child file
with just |\input{|\textit{main}|}|
without loading of the package and using |\childdocof|.
If instead all processing is done
with the appropriate \textsf{childdoc} directives,
the argument of \textit{main} of |\childdocmain| can be empty.

An alternative version of the command line processing described
in \secref{sec:commandline} using the detection mechanism reads:
%
\begin{center}
|... -jobname "|\textit{target}|" "|[\textit{flags}]%
[|\def\jobname{|\textit{dest}|}|]|\input{|\textit{main}|}"|
\end{center}

%%%%%%%%%%%%%%%%%%%%%%%%%%%%%%%%%%%%%%%%%%%%%%%%%%%%%%%%%%%%%%%%%%%%%%%%%%%%%%%%
\subsection{Manual Code}
\label{sec:manual}

In case one cannot be certain whether the definitions file |childdoc.def|
is installed on the target \TeX{} distribution
and one prefers not to ship it,
it is conceivable to paste a few relevant commands into the sources.

To that end, drop all statements |\input{childdoc.def}|
and perform the replacements as outlined below.
Instead of |\childdocmain{|\textit{main}|}| add the following code
to the top of the main file:
%
\begin{center}
\begin{tabular}{l}
|\||ifdefined\childdocname\endinput\||fi\newif\ifchilddoc|\\
|\edef\childdocname{\scantokens\expandafter{\jobname\noexpand}}|\\
|\def\childdocmain{|\textit{main}|}\||ifx\childdocmain\childdocname\||else|\\
|\childdoctrue\includeonly{\childdocname}\let\jobname\childdocmain\||fi|\\
\end{tabular}
\end{center}
%
Instead of |\childdocof{|\textit{main}|}| just include the main file
at the top of each child file:
%
\begin{center}
|\input{|\textit{main}|}|
\end{center}
%
A simple redirection |\childdocforward{|\textit{dest}|}| is achieved by:
%
\begin{center}
|\def\jobname{|\textit{dest}|}\input{\jobname}|
\end{center}
%
The redirection with prefix
|\childdocforwardprefix[|\textit{prefix}|]{|\textit{dest}|}|
is accomplished by:
%
\begin{center}
\begin{tabular}{l}
|{\edef\jobname{\scantokens\expandafter{\jobname\noexpand}}|\\
|\def\redirectjob |\textit{prefix}|#1~~~{\gdef\jobname{|\textit{dest}|#1}}|\\
|\expandafter\redirectjob\jobname~~~}\input{\jobname}|
\end{tabular}
\end{center}

In an alternative approach,
child documents can be compiled by a specific command line
without additional code or specific definitions:
%
\begin{center}
|... -jobname "|\textit{target}|" "|[\textit{flags}]%
|\includeonly{|\textit{dest}|}\input{|\textit{main}|}"|
\end{center}
%

%%%%%%%%%%%%%%%%%%%%%%%%%%%%%%%%%%%%%%%%%%%%%%%%%%%%%%%%%%%%%%%%%%%%%%%%%%%%%%%%
%%%%%%%%%%%%%%%%%%%%%%%%%%%%%%%%%%%%%%%%%%%%%%%%%%%%%%%%%%%%%%%%%%%%%%%%%%%%%%%%
\section{Information}

%%%%%%%%%%%%%%%%%%%%%%%%%%%%%%%%%%%%%%%%%%%%%%%%%%%%%%%%%%%%%%%%%%%%%%%%%%%%%%%%
\subsection{Copyright}

Copyright \copyright{} 2017--2018 Niklas Beisert

This work may be distributed and/or modified under the
conditions of the \LaTeX{} Project Public License, either version 1.3
of this license or (at your option) any later version.
The latest version of this license is in
  \url{http://www.latex-project.org/lppl.txt}
and version 1.3 or later is part of all distributions of \LaTeX{}
version 2005/12/01 or later.

This work has the LPPL maintenance status `maintained'.

The Current Maintainer of this work is Niklas Beisert.

This work consists of the files |README.txt|, |childdoc.ins| and |childdoc.dtx|
as well as the derived files |childdoc.def|, |cdocsamp.tex|
with |cdocsch1.tex|, |cdocsch2.tex|, |cdocspt3.tex|, |cdocspt4.tex|,
|cdocsdrf.tex|, |cdocsfn1.tex|, |cdocsfn2.tex|
as well as |childdoc.pdf|.

%%%%%%%%%%%%%%%%%%%%%%%%%%%%%%%%%%%%%%%%%%%%%%%%%%%%%%%%%%%%%%%%%%%%%%%%%%%%%%%%
\subsection{Files and Installation}

The package consists of the files:
%
\begin{center}
\begin{tabular}{ll}
    |README.txt|   & readme file \\
    |childdoc.ins| & installation file \\
    |childdoc.dtx| & source file \\
    |childdoc.def| & definition file \\
    |cdocsamp.tex| & sample main file \\
    |cdocsch1.tex| & sample include file \\
    |cdocsch2.tex| & sample include file \\
    |cdocspt3.tex| & sample part file \\
    |cdocspt4.tex| & sample part file \\
    |cdocsdrf.tex| & sample redirection file \\
    |cdocsfn1.tex| & sample redirection file \\
    |cdocsfn2.tex| & sample redirection file \\
    |childdoc.pdf| & manual
\end{tabular}
\end{center}
%
The distribution consists of the files
|README.txt|, |childdoc.ins| and |childdoc.dtx|.
%
\begin{itemize}
\item
Run (pdf)\LaTeX{} on |childdoc.dtx|
to compile the manual |childdoc.pdf| (this file).
\item
Run \LaTeX{} on |childdoc.ins| to create the definitions file |childdoc.def|
and the sample |cdocsamp.tex| with include files
|cdocsch1.tex|, |cdocsch2.tex|, |cdocspt3.tex|, |cdocspt4.tex|,
|cdocsdrf.tex|, |cdocsfn1.tex|, |cdocsfn2.tex|.
Then copy the file |childdoc.def| to an appropriate directory of your \LaTeX{}
distribution, e.g.\ \textit{texmf-root}|/tex/latex/childdoc|.
\end{itemize}

%%%%%%%%%%%%%%%%%%%%%%%%%%%%%%%%%%%%%%%%%%%%%%%%%%%%%%%%%%%%%%%%%%%%%%%%%%%%%%%%
\subsection{Related CTAN Packages}

There are several other packages which offer a similar functionality:
%
\begin{itemize}
\item
The packages
\href{http://ctan.org/pkg/docmute}{\textsf{docmute}},
\href{http://ctan.org/pkg/includex}{\textsf{includex}} and
\href{http://ctan.org/pkg/standalone}{\textsf{standalone}}
provide commands to include only the document body of
a child file thus allowing both files to be compiled individually.
\item
The packages \href{http://ctan.org/pkg/subdocs}{\textsf{subdocs}}
and \href{http://ctan.org/pkg/subfiles}{\textsf{subfiles}}
provide structures in which the main and child documents can be
encapsulated and allowing them to be compiled individually.
The inclusion mechanism is different from the conventional |\include|.
\item
The package \href{http://ctan.org/pkg/combine}{\textsf{combine}}
is an elaborate solution to combine several documents into one.
\end{itemize}
%
See also the CTAN topic \href{http://ctan.org/topic/subdocs}{\textsf{subdocs}}
for further related packages.
The present package differs from the above solutions in that
a document structure constructed with the conventional |\include| mechanism
just needs two extra commands at the top of every file
such that all constituent files can be compiled individually.

%%%%%%%%%%%%%%%%%%%%%%%%%%%%%%%%%%%%%%%%%%%%%%%%%%%%%%%%%%%%%%%%%%%%%%%%%%%%%%%%
%\subsection{Feature Suggestions}
%
%The following is a list of features which may be useful for future
%versions of this package:
%%
%\begin{itemize}
%\item
%\ldots
%\end{itemize}

%%%%%%%%%%%%%%%%%%%%%%%%%%%%%%%%%%%%%%%%%%%%%%%%%%%%%%%%%%%%%%%%%%%%%%%%%%%%%%%%
\subsection{Revision History}

%%%%%%%%%%%%%%%%%%%%%%%%%%%%%%%%%%%%%%%%
\paragraph{v2.0:} 2018/12/30

\begin{itemize}
\item
immediate forward processing
\item
added |\childdocby| mechanism
\item
manual restructured
\end{itemize}

%%%%%%%%%%%%%%%%%%%%%%%%%%%%%%%%%%%%%%%%
\paragraph{v1.6:} 2018/01/17

\begin{itemize}
\item
application for development of include files
\item
corrections to manual
\end{itemize}

%%%%%%%%%%%%%%%%%%%%%%%%%%%%%%%%%%%%%%%%
\paragraph{v1.5:} 2017/05/21

\begin{itemize}
\item
more complete structuring introduced
\item
|\childdocof| introduced
\item
|\childdoc| renamed to |\childdocmain|
\item
|\childredirect| renamed to |\childdocforward| and |\childdocforwardprefix|
and functionality expanded
\end{itemize}

%%%%%%%%%%%%%%%%%%%%%%%%%%%%%%%%%%%%%%%%
\paragraph{v1.0:} 2017/04/27

\begin{itemize}
\item
manual and install package
\item
first version published on CTAN
\end{itemize}

%%%%%%%%%%%%%%%%%%%%%%%%%%%%%%%%%%%%%%%%
\paragraph{v0.6:} 2017/04/26

\begin{itemize}
\item
redirection mechanism added
\end{itemize}

%%%%%%%%%%%%%%%%%%%%%%%%%%%%%%%%%%%%%%%%
\paragraph{v0.5:} 2017/04/26

\begin{itemize}
\item
functionality in definition file
\end{itemize}


%%%%%%%%%%%%%%%%%%%%%%%%%%%%%%%%%%%%%%%%%%%%%%%%%%%%%%%%%%%%%%%%%%%%%%%%%%%%%%%%
%%%%%%%%%%%%%%%%%%%%%%%%%%%%%%%%%%%%%%%%%%%%%%%%%%%%%%%%%%%%%%%%%%%%%%%%%%%%%%%%
%%%%%%%%%%%%%%%%%%%%%%%%%%%%%%%%%%%%%%%%%%%%%%%%%%%%%%%%%%%%%%%%%%%%%%%%%%%%%%%%
\appendix

\settowidth\MacroIndent{\rmfamily\scriptsize 000\ }

 \DocInput{childdoc.dtx}

\end{document}
%</driver>
% \fi
%
% %%%%%%%%%%%%%%%%%%%%%%%%%%%%%%%%%%%%%%%%%%%%%%%%%%%%%%%%%%%%%%%%%%%%%%%%%%%%%%
% %%%%%%%%%%%%%%%%%%%%%%%%%%%%%%%%%%%%%%%%%%%%%%%%%%%%%%%%%%%%%%%%%%%%%%%%%%%%%%
% \section{Sample}
%\iffalse
%<*samplemain>
%\fi
%
% The following presents a sample document
% with two chapters, two parts, a title page,
% a compile flag as well as three forwarding files to set the flag.
% It consists of eight |.tex| files:
% \begin{center}
% \begin{tabular}{ll}
% |cdocsamp.tex|&main file\\
% |cdocsch1.tex|&include file for chapter 1\\
% |cdocsch2.tex|&include file for chapter 2\\
% |cdocspt3.tex|&include file for part 3\\
% |cdocspt4.tex|&include file for part 4\\
% |cdocsdrf.tex|&forwarding file for main file in draft mode\\
% |cdocsfi1.tex|&forwarding file for final version of chapter 1\\
% |cdocsfi2.tex|&forwarding file for final version of chapter 2\\
% \end{tabular}
% \end{center}
% Each of the eight files can be compiled directly by the \LaTeX{} compiler.
%
% %%%%%%%%%%%%%%%%%%%%%%%%%%%%%%%%%%%%%%
% \paragraph{Main File.}
%
% The main file is called |cdocsamp.tex|.
%
% Load the \textsf{childdoc} definitions and
% declare the filename for the main document:
%    \begin{macrocode}
\input{childdoc.def}
\childdocmain{}
%    \end{macrocode}

% Optional override for |\version| flag:
%    \begin{macrocode}
%%\ifchilddoc\else\providecommand{\version}{draft}\fi
%    \end{macrocode}

% Define the default values for the |\version| flag
% (|final| for the main file and |draft| for childs):
%    \begin{macrocode}
\ifchilddoc
\providecommand{\version}{draft}
\else
\providecommand{\version}{final}
\fi
%    \end{macrocode}

% Load the standard document class:
%    \begin{macrocode}
\documentclass[12pt]{article}
%    \end{macrocode}

% Start the document body:
%    \begin{macrocode}
\begin{document}
%    \end{macrocode}

% Declare a title page.
% Print title, part of document being processed and version flag:
%    \begin{macrocode}
\addtocounter{page}{-1}
\begin{center}
{\LARGE\bfseries{}childdoc example\par}
\vspace{1cm}
\ifchilddoc
\ifchilddocmanual part\else chapter\fi:
`\childdocname' of `\childdocjob'\par
\else
main document: `\childdocjob'\par
\fi
version: \version\par
\end{center}
\newpage
%    \end{macrocode}

% Manually include selected file,
% otherwise process as usual:
%    \begin{macrocode}
\ifchilddocmanual
\section*{part `\childdocname'}
\input{\childdocname}
\else
%    \end{macrocode}

% Include the two chapters:
%    \begin{macrocode}
\include{cdocsch1}
\include{cdocsch2}
%    \end{macrocode}

% Include the two parts unless only chapters should be displayed:
%    \begin{macrocode}
\ifchilddoc\else
\section{part three}
\input{cdocspt3}
\section{part four}
\input{cdocspt4}
\fi
%    \end{macrocode}

% Process as usual until here:
%    \begin{macrocode}
\fi
%    \end{macrocode}

% End of document body:
%    \begin{macrocode}
\end{document}
%    \end{macrocode}
%\iffalse
%</samplemain>
%\fi
%
% %%%%%%%%%%%%%%%%%%%%%%%%%%%%%%%%%%%%%%
% \paragraph{Chapter Include Files.}
%
% The include files are called |cdocsch1.tex| and |cdocsch2.tex|.
%
%\iffalse
%<*samplechap1|samplechap2>
%\fi

% Optional override for |\version| flag:
%    \begin{macrocode}
%%\providecommand{\version}{final}
%    \end{macrocode}

% Include the main document:
%    \begin{macrocode}
\input{childdoc.def}
\childdocof{cdocsamp}
%    \end{macrocode}

%\iffalse
%</samplechap1|samplechap2>
%\fi
%
%\iffalse
%<*samplechap1>
%\fi
% Some text for chapter 1:
%    \begin{macrocode}
\section{one}
some text in chapter one
%    \end{macrocode}

%\iffalse
%</samplechap1>
%\fi
% Some text for chapter 2:
%\iffalse
%<*samplechap2>
%\fi
%    \begin{macrocode}
\section{two}
more text in chapter two
%    \end{macrocode}

%\iffalse
%</samplechap2>
%\fi
%
% %%%%%%%%%%%%%%%%%%%%%%%%%%%%%%%%%%%%%%
% \paragraph{Part Include Files.}
%
% The include files are called |cdocspt3.tex| and |cdocspt4.tex|.
%
%\iffalse
%<*samplepart3|samplepart4>
%\fi

% Optional override for |\version| flag:
%    \begin{macrocode}
%%\providecommand{\version}{final}
%    \end{macrocode}

% Include the main document:
%    \begin{macrocode}
\input{childdoc.def}
\childdocby{cdocsamp}
%    \end{macrocode}

%\iffalse
%</samplepart3|samplepart4>
%\fi
%
%\iffalse
%<*samplepart3>
%\fi
% Some text for part 3:
%    \begin{macrocode}
some text in part three
%    \end{macrocode}

%\iffalse
%</samplepart3>
%\fi
% Some text for part 4:
%\iffalse
%<*samplepart4>
%\fi
%    \begin{macrocode}
more text in part four
%    \end{macrocode}

%\iffalse
%</samplepart4>
%\fi
%
% %%%%%%%%%%%%%%%%%%%%%%%%%%%%%%%%%%%%%%
% \paragraph{Forwarding for a Complete Draft.}
%
% The following forwarding file |cdocsdrf.tex|
% compiles the main document in draft mode:
%\iffalse
%<*sampledraft>
%\fi
%    \begin{macrocode}
\def\version{draft}
\input{childdoc.def}
\childdocforward{cdocsamp}
%    \end{macrocode}

%\iffalse
%</sampledraft>
%\fi
%
% %%%%%%%%%%%%%%%%%%%%%%%%%%%%%%%%%%%%%%
% \paragraph{Forwarding for Final Version of the Chapters.}
%
% The following forwarding files |cdocsfn1.tex| and |cdocsfn2.tex|
% (with identical content)
% compile the final versions of the child documents
% |cdocsch1.tex| and |cdocsch2.tex|, respectively:
%\iffalse
%<*samplefinal>
%\fi
%    \begin{macrocode}
\def\version{final}
\input{childdoc.def}
\childdocforwardprefix[cdocsamp]{cdocsfn}{cdocsch}
%    \end{macrocode}

%\iffalse
%</samplefinal>
%\fi
%
% %%%%%%%%%%%%%%%%%%%%%%%%%%%%%%%%%%%%%%
% \paragraph{Command Line Processing.}
%
% The following three command lines generate the output files
% |cdocscld|, |cdocscl1| and |cdocscl2|
% which should be identical to
% |cdocsdrf|, |cdocsch1| and |cdocsfn2|, respectively:
% \begin{center}
% \begin{tabular}{l}
% |latex -jobname cdocscld \|\\
% |  "\def\version{draft}\input{childdoc.def}\childdocforward{cdocsamp}"|\\
% |latex -jobname cdocscl1 \|\\
% |  "\input{childdoc.def}\childdocforward[cdocsamp]{cdocsch1}"|\\
% |latex -jobname cdocscl2 \|\\
% |  "\def\version{final}\input{childdoc.def}\childdocforward{cdocsch2}"|
% \end{tabular}
% \end{center}
% Note that the trailing backslash on each first line
% merely continues the input to the second line
% (for convenient cut ant paste).
% Furthermore, the command |latex| can be replaced by any
% of its alternative versions such as |pdflatex|.
%
% %%%%%%%%%%%%%%%%%%%%%%%%%%%%%%%%%%%%%%%%%%%%%%%%%%%%%%%%%%%%%%%%%%%%%%%%%%%%%%
% %%%%%%%%%%%%%%%%%%%%%%%%%%%%%%%%%%%%%%%%%%%%%%%%%%%%%%%%%%%%%%%%%%%%%%%%%%%%%%
% \section{Implementation}
%\iffalse
%<*package>
%\fi
%
% This section describes the definitions file |childdoc.def|.

% The definitions cannot be loaded using |\usepackage| or |\RequirePackage|
% which has a mechanism to prevent loading a style file more than once.
% When loading the definitions by means of |\input|
% multiple instances have to be prevented manually:
%\iffalse
%This code needs to be before the `\ProvidesFile' directive
%which is defined at the beginning of this file.
%Therefore it is also placed there and commented out here.
%</package>
%<*discard>
%\fi
%    \begin{macrocode}
\ifdefined\childdocmain\endinput\fi
%    \end{macrocode}
%\iffalse
%</discard>
%<*package>
%\fi
%
% \macro{\ifchilddoc}
% \macro{\ifchilddocmanual}
% The conditional |\ifchilddoc| tells whether a
% child (true) or main (false) document is being compiled.
% The conditional |\ifchilddocmanual| tells whether
% the |\includeonly| mechanism is used (false) or
% the selection of child files must be performed manually (true).
% The definitions initialise to false:
%    \begin{macrocode}
\newif\ifchilddoc
\newif\ifchilddocmanual
%    \end{macrocode}

% \macro{\childdocname}
% \macro{\childdocjob}
% The macro |\childdocname| stores the name of the main document
% to be compiled. The macro |\childdocjob| stores the name of
% the document on which the \LaTeX{} compiler was originally invoked.
% The content of |\jobname| cannot be compared
% to filenames specified in the source due to different catcodes.
% The following code rescans |\jobname|, stores the result
% in |\childdocname| and saves a copy in |\childdocjob|:
%    \begin{macrocode}
\edef\childdocname{\scantokens\expandafter{\jobname\noexpand}}
\let\childdocjob\childdocname
%    \end{macrocode}

% \macro{\childdocdisable}
% The macro |\childdocdisable| prevents the main file
% from being processed more than once.
% At this stage, the main document command |\childdocmain|
% is assumed to be called once again where it should do nothing.
% Any subsequent call to it should prevent
% a secondary processing of the main document
% It overwrites the forwarding commands
% |\childdocof| and |\childdocforward|
% with empty macros to prevent further inclusions of the main document:
%    \begin{macrocode}
\newcommand{\childdocdisable}
{
  \renewcommand{\childdocmain}[1]{\renewcommand{\childdocmain}[1]{\endinput}}
  \renewcommand{\childdocof}[1]{}
  \renewcommand{\childdocby}[2][]{}
  \renewcommand{\childdocforward}[2][]{}
  \renewcommand{\childdocdisable}{}
}
%    \end{macrocode}

% \macro{\childdocmain}
% The macro |\childdocmain| is to be called at the top of the main file
% with nothing or the main filename (without extension) as argument.
% First, it breaks loops.
% If the argument is not empty and does not match |\childdocname|
% (which is set by the first inclusion of |childdoc.def|),
% |\ifchilddoc| is set to true, |\includeonly| is applied to the child file
% and |\jobname| is set to the main file
% (for proper handling of |.aux| files):
%    \begin{macrocode}
\newcommand{\childdocmain}[1]
{
  \childdocdisable\childdocmain{}
  \if?#1?\else
    \begingroup
      \def\childdoctmp{#1}
      \ifx\childdoctmp\childdocname
        \def\childdoctmp{}
      \else
        \def\childdoctmp
        {
          \childdoctrue
          \includeonly{\childdocname}
          \def\childdocjob{#1}
          \def\jobname{#1}
        }
      \fi
      \expandafter
    \endgroup
    \childdoctmp
  \fi
}
%    \end{macrocode}

% \macro{\childdocof}
% The command |\childdocof| redirects
% compilation to the main file |#1|.
%    \begin{macrocode}
\newcommand{\childdocof}[1]
{
  \childdocdisable
  \childdoctrue
  \includeonly{\childdocname}
  \def\jobname{#1}
  \def\childdocjob{#1}
  \input{#1}
}
%    \end{macrocode}

% \macro{\childdocby}
% The command |\childdocby| ....
%    \begin{macrocode}
\newcommand{\childdocby}[2][]
{
  \childdocdisable
  \childdoctrue
  \childdocmanualtrue
  \if?#1?\else
    \def\jobname{#2}
  \fi
  \def\childdocjob{#2}
  \input{#2}
  \endinput
}
%    \end{macrocode}

% \macro{\childdocforward}
% The command |\childdocforward| redirects
% compilation to the main file or
% (if the optional argument is given) a child file.
% Parameters are set as if the main file
% or a child file starting with |\childdocof| was compiled.
% Then compilation is handed over to the main file:
%    \begin{macrocode}
\newcommand{\childdocforward}[2][]
{
  \begingroup
    \if?#1?
      \def\childdoctmp
      {
        \def\childdocname{#2}
        \def\childdocjob{#2}
        \def\jobname{#2}
        \input{#2}
        \endinput
      }
    \else
      \def\childdoctmp
      {
        \childdocdisable
        \def\childdocname{#2}
        \childdoctrue
        \includeonly{#2}
        \def\childdocjob{#1}
        \def\jobname{#1}
        \input{#1}
        \endinput
      }
    \fi
    \expandafter
  \endgroup
  \childdoctmp
}
%    \end{macrocode}

% \macro{\childdocforwardprefix}
% The command |\childdocforwardprefix| redirects
% compilation to the main or a child file by means of a pattern.
% The prefix |#1| in the current filename is replaced by |#2|
% and the suffix of the current filename is kept
% (it is assumed that the filename does not contain the substring `|~~~|'
% which is used as a delimiter).
% Compilation is handed over to the new file by |\childdocforward|:
%    \begin{macrocode}
\newcommand{\childdocforwardprefix}[3][]
{
  \begingroup
    \def\childdocextract #2##1~~~{\def\childdoctmp{\childdocforward[#1]{#3##1}}}
    \expandafter\childdocextract\childdocname~~~
    \expandafter
  \endgroup
  \childdoctmp
}
%    \end{macrocode}

% \macro{\childdoc}
% The deprecated macro |\childdoc| is a legacy version of |\childdocmain|:
%    \begin{macrocode}
\newcommand{\childdoc}{\childdocmain}
%    \end{macrocode}

% \macro{\childdocredirect}
% The deprecated macro |\childdocredirect| is a legacy version
% of |\childdocforward| and |\childdocforwardprefix|:
%    \begin{macrocode}
\newcommand{\childdocredirect}[2][]
{
  \begingroup
    \if?#1?
      \def\childdoctmp{\childdocforward{#2}}
    \else
      \def\childdoctmp{\childdocforwardprefix{#1}{#2}}
    \fi
    \expandafter
  \endgroup
  \childdoctmp
}
%    \end{macrocode}

%\iffalse
%</package>
%\fi
%
\endinput
\childdocforward{cdocsamp}"|\\
% |latex -jobname cdocscl1 \|\\
% |  "% \iffalse
%
% childdoc.dtx Copyright (C) 2017-2018 Niklas Beisert
%
% This work may be distributed and/or modified under the
% conditions of the LaTeX Project Public License, either version 1.3
% of this license or (at your option) any later version.
% The latest version of this license is in
%   http://www.latex-project.org/lppl.txt
% and version 1.3 or later is part of all distributions of LaTeX
% version 2005/12/01 or later.
%
% This work has the LPPL maintenance status `maintained'.
%
% The Current Maintainer of this work is Niklas Beisert.
%
% This work consists of the files childdoc.dtx and childdoc.ins
% and the derived files childdoc.def and cdocsamp.tex with
% cdocsch1.tex, cdocsch2.tex, cdocsdrf.tex, cdocsfn1.tex, cdocsfn2.tex.
%
%<package>\ifdefined\childdocmain\endinput\fi
%<package>\ProvidesFile{childdoc.def}[2018/12/30 v2.0 child document driver]
%<samplemain>\ProvidesFile{cdocsamp.tex}[2018/12/30 v2.0 sample for childdoc]
%<*driver>
%\ProvidesFile{childdoc.drv}[2018/12/30 v2.0 childdoc reference manual file]
\PassOptionsToClass{10pt,a4paper}{article}
\documentclass{ltxdoc}

\usepackage[margin=35mm]{geometry}
\usepackage{hyperref}
\usepackage{hyperxmp}
\usepackage[usenames]{color}

\hypersetup{colorlinks=true}
\hypersetup{pdfstartview=FitH}
\hypersetup{pdfpagemode=UseNone}
\hypersetup{pdfsource={}}
\hypersetup{pdflang={en-UK}}
\hypersetup{pdfcopyright={Copyright 2017-2018 Niklas Beisert.
  This work may be distributed and/or modified under the
  conditions of the LaTeX Project Public License, either version 1.3
  of this license or (at your option) any later version.}}
\hypersetup{pdflicenseurl={http://www.latex-project.org/lppl.txt}}
\hypersetup{pdfcontactaddress={ETH Zurich, ITP, HIT K,
  Wolfgang-Pauli-Strasse 27}}
\hypersetup{pdfcontactpostcode={8093}}
\hypersetup{pdfcontactcity={Zurich}}
\hypersetup{pdfcontactcountry={Switzerland}}
\hypersetup{pdfcontactemail={nbeisert@itp.phys.ethz.ch}}
\hypersetup{pdfcontacturl={http://people.phys.ethz.ch/\xmptilde nbeisert/}}

\newcommand{\secref}[1]{\hyperref[#1]{section \ref*{#1}}}

\parskip1ex
\parindent0pt
\let\olditemize\itemize
\def\itemize{\olditemize\parskip0pt}

\begin{document}

\title{The \textsf{childdoc} Package}
\hypersetup{pdftitle={The childdoc Package}}
\author{Niklas Beisert\\[2ex]
  Institut f\"ur Theoretische Physik\\
  Eidgen\"ossische Technische Hochschule Z\"urich\\
  Wolfgang-Pauli-Strasse 27, 8093 Z\"urich, Switzerland\\[1ex]
  \href{mailto:nbeisert@itp.phys.ethz.ch}
  {\texttt{nbeisert@itp.phys.ethz.ch}}}
\hypersetup{pdfauthor={Niklas Beisert}}
\hypersetup{pdfsubject={Manual for the LaTeX2e Package childdoc}}
\date{30 December 2018, \textsf{v2.0}}
\maketitle

\begin{abstract}\noindent
\textsf{childdoc} is a \LaTeXe{} package
that enables the direct compilation
of document sections included by |\include|
to individual files.
\end{abstract}

\begingroup
\parskip0ex
\tableofcontents
\endgroup

%%%%%%%%%%%%%%%%%%%%%%%%%%%%%%%%%%%%%%%%%%%%%%%%%%%%%%%%%%%%%%%%%%%%%%%%%%%%%%%%
%%%%%%%%%%%%%%%%%%%%%%%%%%%%%%%%%%%%%%%%%%%%%%%%%%%%%%%%%%%%%%%%%%%%%%%%%%%%%%%%
\section{Introduction}

\LaTeX{} provides a mechanism to structure a large document (such as a book)
into a main file and several child files (containing the chapters)
using the |\include| command.
This mechanism is beneficial for documents
which span hundreds of pages in order to
make the source file(s) more manageable.
Moreover, compilation can be restricted to
selected child files by means of the |\includeonly| command.
The latter feature can be used to reduce the compilation time while editing
(this was significantly more useful in the earlier days of \LaTeX{})
or to generate a smaller document which is easier to navigate.
Another application of |\includeonly| is to generate
documents consisting of selected parts of the complete document.

However, there are a few drawbacks of the plain |\include| mechanism:
\begin{itemize}
\item
The child files cannot be compiled on their own,
they can only be compiled via the main file.
A naive editing environment
(such as a text editor with an option
to have the current file processed by \LaTeX)
may require one to switch to the main file before compiling;
attempting to compile the child file produces errors.
\item
The main file must be modified (each time)
to adjust the |\includeonly| command
to the present needs. This easily leaves the main file in a messy state.
\item
The generated document will always carry the filename
of the main document. This is inconvenient if
several child files are to be compiled and
to be kept for distribution.
\end{itemize}

The present package provides a simple interface
to make child files individually compilable by \LaTeX{}.
Compiling a child file then has the same effect as compiling
the main file with an |\includeonly| command
to select the appropriate child.
Moreover the generated document will carry the name of the child
rather than the main file.
This resolves all three above issues.

This feature is meant to make the editing of books,
thesis documents and lecture notes somewhat more convenient.
However, the package can also be used efficiently for
composing a series of documents (such as exercise sheets)
which are typically distributed individually.
It then assists the author in generating the individual documents
(potentially in different versions)
as well as a document containing the collected series.
Another application is in developing style files
or other kinds of included material
where compilation of the style file could redirect
to a sample or test file.

%%%%%%%%%%%%%%%%%%%%%%%%%%%%%%%%%%%%%%%%%%%%%%%%%%%%%%%%%%%%%%%%%%%%%%%%%%%%%%%%
%%%%%%%%%%%%%%%%%%%%%%%%%%%%%%%%%%%%%%%%%%%%%%%%%%%%%%%%%%%%%%%%%%%%%%%%%%%%%%%%
\section{Usage}

First of all, the package \textsf{childdoc} is \emph{not} a standard
\LaTeXe{} |.sty| style file! Therefore it needs to be invoked in
a non-standard way.

%%%%%%%%%%%%%%%%%%%%%%%%%%%%%%%%%%%%%%%%%%%%%%%%%%%%%%%%%%%%%%%%%%%%%%%%%%%%%%%%
\subsection{Included Files}
\label{sec:include}

%%%%%%%%%%%%%%%%%%%%%%%%%%%%%%%%%%%%%%%%
\DescribeMacro{\childdocmain}
To use the package, add the commands
\begin{center}
\begin{tabular}{l}
|\input{childdoc.def}|\\
|\childdocmain{}|\\
\end{tabular}
\end{center}
at the very top of the main \LaTeX{} file,
in particular \emph{before} the |\documentclass| statement!
The argument of |\childdocmain| should be left empty
(but it must be present).

%%%%%%%%%%%%%%%%%%%%%%%%%%%%%%%%%%%%%%%%
\DescribeMacro{\childdocof}
Furthermore, add the commands
\begin{center}
\begin{tabular}{l}
|\input{childdoc.def}|\\
|\childdocof{|\textit{main}|}|\\
\end{tabular}
\end{center}
at the top of every child file \textit{child}
which is included by |\include{|\textit{child}|}|
from within the main file
(or at least for those files to be compiled individually).
The argument \textit{main} must be the filename of the main file.

There are a couple of
considerations in setting up the main and child documents:

%%%%%%%%%%%%%%%%%%%%%%%%%%%%%%%%%%%%%%%%
\paragraph{Restrictions.}

Please note the following restrictions:
\begin{itemize}
\item
|\childdocmain| must be called with one argument \textit{main}
to ensure compatibility with earlier version of the package.
It must either be empty (|\childdocmain{}|)
or precisely match the filename of the main file in which it is specified.
See \secref{sec:detection} for further information.
\item
The filename \textit{main} must be specified without the |.tex| extension.
\item
The filename \textit{main} is case sensitive
(even in case-insensitive file systems)
due to internal string comparison.
\item
The argument \textit{main} should be fully expanded, it cannot be a macro.
\item
Subdirectories and special characters should be avoided in filenames.
\item
The command |\childdocmain{|\textit{main}|}| must be followed by a whitespace.
It should not be followed immediately by another command
or by a comment mark `|%|'.
This is because the \TeX{} parser reads the token immediately following
the argument of |\childdocmain| and puts it
at the beginning of every child section;
however, a white\-space is ignored.
\end{itemize}

%%%%%%%%%%%%%%%%%%%%%%%%%%%%%%%%%%%%%%%%
\paragraph{Content of Main File.}

It is advisable to place all content in the child files included by |\include|.
Any output contained in the main file will appear in all child documents
unless suppressed manually;
it cannot be suppressed automatically by the |\includeonly| directive
and thus should normally be avoided.
A method to include some content in the main file
by means of conditional processing is described in \secref{sec:conditional}.

%%%%%%%%%%%%%%%%%%%%%%%%%%%%%%%%%%%%%%%%
\paragraph{Page Numbering.}

When only a part of the document is compiled,
the appropriate numbering of pages
(as well as other status parameters)
is determined from the |.aux| files.
The latter contain information from previous passes.
However this information needs to propagate through
all intermediate child documents.
Therefore the page numbering in child documents may well
be inconsistent until the complete document is compiled at least once.

A useful (if unconventional) way to always ensure a consistent
page numbering is to restart the numbering in each child document
and denote the pages by `\textit{child}|.|\textit{page}'
where \textit{child} represents the chapter/section number of the child file.
This can be achieved by the command
|\numberwithin{page}{|\textit{child}|}|
of the \textsf{amsmath} package
where \textit{child} can be |chapter| or |section|
depending on the chosen structuring.
Alternatively, one can modify the macro |\thepage| appropriately
and reset the counter |page| at the start of each child file.

%%%%%%%%%%%%%%%%%%%%%%%%%%%%%%%%%%%%%%%%%%%%%%%%%%%%%%%%%%%%%%%%%%%%%%%%%%%%%%%%
\subsection{Conditional Processing}
\label{sec:conditional}

The package provides a mechanism to compile different versions
of a document. To customise the versions further some conditional processing
can come in handy to distinguish which version is being compiled.
The package provides two macros to describe the compilation context:

%%%%%%%%%%%%%%%%%%%%%%%%%%%%%%%%%%%%%%%%
\DescribeMacro{\ifchilddoc}
The conditional |\ifchilddoc| distinguishes between the compilation of
child documents and the main document:
%
\begin{center}
|\ifchilddoc |\textit{child-code}| |[|\||else |\textit{main-code}]| \||fi|
\end{center}

%%%%%%%%%%%%%%%%%%%%%%%%%%%%%%%%%%%%%%%%
\DescribeMacro{\childdocname}
\DescribeMacro{\childdocjob}
The macro |\childdocname| contains the filename (without extension)
of the main or child file being processed.
Note that |\childdocjob| will always contain the name of the main file.

%%%%%%%%%%%%%%%%%%%%%%%%%%%%%%%%%%%%%%%%
\paragraph{Title Page.}

Conditional processing can be used to include a title or banner page
in the main document when proper precautions are taken.
Importantly, the code in the main file should ensure that the page counter
(as well as other status parameters which are stored in the |.aux| files)
takes the same value after the conditional processing.
Otherwise the page numbers may take divergent values
depending on which part is compiled.

For example, a title page could be declared by:
%
\begin{center}
\begin{tabular}{l}
|\ifchilddoc\||else|\\
|\addtocounter{page}{-1}|\\
\textit{code for title page}\\
|\newpage|\\
|\||fi|
\end{tabular}
\end{center}
%
A banner page for the child documents can be generated by:
%
\begin{center}
\begin{tabular}{l}
|\ifchilddoc|\\
|\addtocounter{page}{-1}|\\
\textit{code for banner page}\\
|\newpage|\\
|\||fi|
\end{tabular}
\end{center}
%
Here one could write a message such as:
\begin{center}
|This is the part \childdocname{} of \childdocjob{}.|
\end{center}

%%%%%%%%%%%%%%%%%%%%%%%%%%%%%%%%%%%%%%%%%%%%%%%%%%%%%%%%%%%%%%%%%%%%%%%%%%%%%%%%
\subsection{Flags}
\label{sec:flags}

The package makes it easy to generate different versions
of the main or child documents.
To this end compilation flags can be defined
and assigned different default values.
They will be particularly useful in conjunction
with the forwarding mechanism described in \secref{sec:forward}.

For example, it may be useful to have a flag |\version|
which can be set to |draft| or |final|.
The document source will contain some conditional code
depending on the value of |\version|.
Suppose further, the flag should default to |final| for the main file
and to |draft| for child files
which is a natural assignment for editing the document.
This is achieved by placing the following code
in the preamble of the main document
(below the |\childdocmain| directive):
%
\begin{center}
\begin{tabular}{l}
|\ifchilddoc|\\
|\providecommand{\version}{draft}|\\
|\||else|\\
|\providecommand{\version}{final}|\\
|\||fi|
\end{tabular}
\end{center}
%
The definition by |\providecommand| makes sure
that previous definitions are not overwritten.
Further statements |\providecommand{\version}{...}|
can thus be added before the above code to override it.

For the main file, one might add a line
(between |\childdocmain| and the above block)
%
\begin{center}
|%\ifchilddoc\||else\providecommand{\version}{draft}\||fi|
\end{center}
%
which can be uncommented to produce a draft version.
Likewise one can add a line to the very top of a child file
(above the |\childdocof{|\textit{main}|}| directive)
%
\begin{center}
|%\providecommand{\version}{final}|
\end{center}
%
which can be uncommented to produce the final version of this child document.

%%%%%%%%%%%%%%%%%%%%%%%%%%%%%%%%%%%%%%%%%%%%%%%%%%%%%%%%%%%%%%%%%%%%%%%%%%%%%%%%
\subsection{Forwarding}
\label{sec:forward}

Different versions of the main or child documents
using compilation flags as described in \secref{sec:flags}
can be (permanently) stored in different files
for convenient compilation, viewing and distribution.
To this end, the package defines a command
to pass on compilation to a different file:

%%%%%%%%%%%%%%%%%%%%%%%%%%%%%%%%%%%%%%%%
\DescribeMacro{\childdocforward}
The command |\childdocforward| redirects processing to
another source file:
%
\begin{center}
\begin{tabular}{l}
|\input{childdoc.def}|\\
|\childdocforward[|\textit{main}|]{|\textit{dest}|}|\\
\end{tabular}
\end{center}
%
The argument \textit{dest} is the destination file
(without extension).
It should be the main file or one of the child files.
Note that further \textsf{childdoc} directives
such as |\childdocof| and |\childdocforward|
in the indicated file will be processed in this form.
The optional argument \textit{main}
passes on directly to the main file \textit{main}
while pretending to compile the child \textit{dest}.
This form behaves as if \textit{dest}
issues |\childdocof{|\textit{main}|}| right away,
and no further \textsf{childdoc} directives will be processed.

%%%%%%%%%%%%%%%%%%%%%%%%%%%%%%%%%%%%%%%%
\DescribeMacro{\...prefix}
In the alternative form |\childdocforwardprefix|,
%
\begin{center}
\begin{tabular}{l}
|\input{childdoc.def}|\\
|\childdocforwardprefix[|\textit{main}|]{|\textit{prefix}|}{|\textit{dest}|}|
\end{tabular}
\end{center}
%
the destination file is determined by a pattern
depending on the current file:
To make this work, the current file must be called
`{\textit{prefix}\hspace{0.2em}\textit{suffix}}'
with \textit{prefix} matching precisely the argument.
Processing is then passed on to the file
`{\textit{dest}\hspace{0.2em}\textit{suffix}}'.
Surely, the same effect is achieved by
directly specifying the
argument `{\textit{dest}\hspace{0.2em}\textit{suffix}}'
in the first form.
However, that requires to set up a different file
for each child. With the alternative form of the command
all these files can have exactly the same content
which simplifies setting them up and maintaining them.

For example, the following file |draft.tex|
with a compilation flag |\version| as described in \secref{sec:flags}
compiles the main document as a draft:
%
\begin{center}
\begin{tabular}{l}
|\def\version{draft}|\\
|\input{childdoc.def}|\\
|\childdocforward{|\textit{main}|}|
\end{tabular}
\end{center}
%
Likewise, the following files |final|\textit{nn}|.tex|
compile the final version of the child document
|child|\textit{nn}|.tex|:
%
\begin{center}
\begin{tabular}{l}
|\def\version{final}|\\
|\input{childdoc.def}|\\
|\childdocforwardprefix{final}{child}|
\end{tabular}
\end{center}
%

Note that when several versions of a main file and/or of each child file
are to be generated, it may be convenient to set up a |Makefile| or
shell script to automatise the process.

%%%%%%%%%%%%%%%%%%%%%%%%%%%%%%%%%%%%%%%%%%%%%%%%%%%%%%%%%%%%%%%%%%%%%%%%%%%%%%%%
\subsection{Command Line Processing}
\label{sec:commandline}

The effect of redirection files can also be achieved by invoking
the \LaTeX{} compiler with a more elaborate command line.
Most conveniently this should be done as part
of a shell script or a |Makefile|.

When using \textsf{childdoc} in the main file, the following
command lines effectively perform a redirection
(note that depending on the shell being used,
backslashes may have to be doubled: `|\|' $\to$ `|\\|'):
%
\begin{center}
|... -jobname "|\textit{target}|" |\\|"|[\textit{flags}]%
|\input{childdoc.def}\childdocforward[|\textit{main}|]{|\textit{dest}|}"|
\end{center}
%
Here \textit{target} is the name of the output file,
\textit{main} is the name of the main file
and \textit{dest} is the name of the main or child file to be processed
(all filenames without extensions).
The optional argument \textit{main} can be omitted
if \textit{main} matches \textit{dest}.
Optionally, compilation \textit{flags} can be defined via |\def| commands.
This command line makes the \TeX{} engine believe
it is compiling the file \textit{target}
whose content is specified as the latter parameter.
The provided code then forwards the processing to
\textit{main} or \textit{dest} as described in \secref{sec:forward}.

%%%%%%%%%%%%%%%%%%%%%%%%%%%%%%%%%%%%%%%%%%%%%%%%%%%%%%%%%%%%%%%%%%%%%%%%%%%%%%%%
\subsection{Include by Input}
\label{sec:input}

Including child documents by |\include| has some restrictions by design.
Most notably, the content of a child document always occupies
its own set of pages; pages cannot be shared between child documents.
Usually, this behaviour makes perfect sense
because each child document contain an essential part of the document.
However, in some situations it may be desirable to compose
a document from a collection of parts
without having mandatory page breaks between then.
For this case, the package
provides a mechanism to include parts
by |\input| which can also be processed individually.
However, by construction this mechanism
requires manual handling of the content to be output.

%%%%%%%%%%%%%%%%%%%%%%%%%%%%%%%%%%%%%%%%
\DescribeMacro{\ifchilddocmanual}
The main file should be prepared as usual, see \secref{sec:include}.
However, the document body must make a distinction
between processing of an individual part and of the main document, e.g.:
%
\begin{center}
\begin{tabular}{l}
|\ifchilddocmanual|\\
|\input{\childdocname}|\\
|\||else|\\
\textit{document body with }|\input{|\textit{part}|}|\\
|\||fi|
\end{tabular}
\end{center}
%
The conditional |\ifchilddocmanual| is true whenever
a part to be included by |\input| is being compiled,
and the name of the part is stored in |\childdocname|.

%%%%%%%%%%%%%%%%%%%%%%%%%%%%%%%%%%%%%%%%
\DescribeMacro{\childdocby}
Each part to be included by |\input| should start with:
%
\begin{center}
\begin{tabular}{l}
|\input{childdoc.def}|\\
|\childdocby{|\textit{main}|}|\\
\end{tabular}
\end{center}
%
The directive |\childdocby| is similar to |\childdocof|
described in \secref{sec:include},
but the subsequent selection of content must be done manually.
To that end, both |\ifchilddoc| and |\ifchilddocmanual|
will be true upon processing of a part,
and the name of the part is stored in |\childdocname|.
Note that |\jobname| will be set to the filename of the current part
so that each part receives an individual |.aux| file
that does not interfere with the |.aux| file(s) of the main document.
This behaviour can be altered by the alternative form
|\childdocby[*]{|\textit{main}|}| (with a non-empty optional argument)
which uses the |.aux| file of the main document
by setting |\jobname| to \textit{main}.

%%%%%%%%%%%%%%%%%%%%%%%%%%%%%%%%%%%%%%%%%%%%%%%%%%%%%%%%%%%%%%%%%%%%%%%%%%%%%%%%
\subsection{Driver Development}
\label{sec:driver}

The \textsf{childdoc} mechanism can also be use for the development
of definition files such as \LaTeX{} styles or classes.
This case differs from the above setup with multiple parts
included by |\include| in that no |\includeonly| should be invoked.
This can be achieved by starting the include file
(before |\ProvidesPackage|) with:
%
\begin{center}
\begin{tabular}{l}
|\input{childdoc.def}|\\
|\childdocforward{|\textit{main}|}|\\
\end{tabular}
\end{center}
%
or alternatively with:
%
\begin{center}
\begin{tabular}{l}
|\input{childdoc.def}|\\
|\childdocby{|\textit{main}|}|\\
\end{tabular}
\end{center}
%
Both forms have slightly different effects as described above.
The main file is prepared as usual, see \secref{sec:include}.

%%%%%%%%%%%%%%%%%%%%%%%%%%%%%%%%%%%%%%%%%%%%%%%%%%%%%%%%%%%%%%%%%%%%%%%%%%%%%%%%
\subsection{Legacy Detection}
\label{sec:detection}

The directive |\childdocmain| in the main file can detect
whether the complete document or merely a child is to be compiled
even without using the directive |\childdocof|.
This method is deprecated because it is less robust
and there is no compelling reason to use it;
it is merely provided for backward compatibility
and it may be removed in future versions.

If the detection mechanism is to be used,
it is mandatory to correctly specify
the filename of the main file as the argument of |\childdocmain|:
%
\begin{center}
\begin{tabular}{l}
|\input{childdoc.def}|\\
|\childdocmain{|\textit{main}|}|\\
\end{tabular}
\end{center}
%
If |\jobname| does not match the argument \textit{main} of |\childdocmain|,
it is assumed that |\jobname| points to the child file to be compiled.
When using |\childdocmain| with the main file specified as argument,
it suffices to start a child file
with just |\input{|\textit{main}|}|
without loading of the package and using |\childdocof|.
If instead all processing is done
with the appropriate \textsf{childdoc} directives,
the argument of \textit{main} of |\childdocmain| can be empty.

An alternative version of the command line processing described
in \secref{sec:commandline} using the detection mechanism reads:
%
\begin{center}
|... -jobname "|\textit{target}|" "|[\textit{flags}]%
[|\def\jobname{|\textit{dest}|}|]|\input{|\textit{main}|}"|
\end{center}

%%%%%%%%%%%%%%%%%%%%%%%%%%%%%%%%%%%%%%%%%%%%%%%%%%%%%%%%%%%%%%%%%%%%%%%%%%%%%%%%
\subsection{Manual Code}
\label{sec:manual}

In case one cannot be certain whether the definitions file |childdoc.def|
is installed on the target \TeX{} distribution
and one prefers not to ship it,
it is conceivable to paste a few relevant commands into the sources.

To that end, drop all statements |\input{childdoc.def}|
and perform the replacements as outlined below.
Instead of |\childdocmain{|\textit{main}|}| add the following code
to the top of the main file:
%
\begin{center}
\begin{tabular}{l}
|\||ifdefined\childdocname\endinput\||fi\newif\ifchilddoc|\\
|\edef\childdocname{\scantokens\expandafter{\jobname\noexpand}}|\\
|\def\childdocmain{|\textit{main}|}\||ifx\childdocmain\childdocname\||else|\\
|\childdoctrue\includeonly{\childdocname}\let\jobname\childdocmain\||fi|\\
\end{tabular}
\end{center}
%
Instead of |\childdocof{|\textit{main}|}| just include the main file
at the top of each child file:
%
\begin{center}
|\input{|\textit{main}|}|
\end{center}
%
A simple redirection |\childdocforward{|\textit{dest}|}| is achieved by:
%
\begin{center}
|\def\jobname{|\textit{dest}|}\input{\jobname}|
\end{center}
%
The redirection with prefix
|\childdocforwardprefix[|\textit{prefix}|]{|\textit{dest}|}|
is accomplished by:
%
\begin{center}
\begin{tabular}{l}
|{\edef\jobname{\scantokens\expandafter{\jobname\noexpand}}|\\
|\def\redirectjob |\textit{prefix}|#1~~~{\gdef\jobname{|\textit{dest}|#1}}|\\
|\expandafter\redirectjob\jobname~~~}\input{\jobname}|
\end{tabular}
\end{center}

In an alternative approach,
child documents can be compiled by a specific command line
without additional code or specific definitions:
%
\begin{center}
|... -jobname "|\textit{target}|" "|[\textit{flags}]%
|\includeonly{|\textit{dest}|}\input{|\textit{main}|}"|
\end{center}
%

%%%%%%%%%%%%%%%%%%%%%%%%%%%%%%%%%%%%%%%%%%%%%%%%%%%%%%%%%%%%%%%%%%%%%%%%%%%%%%%%
%%%%%%%%%%%%%%%%%%%%%%%%%%%%%%%%%%%%%%%%%%%%%%%%%%%%%%%%%%%%%%%%%%%%%%%%%%%%%%%%
\section{Information}

%%%%%%%%%%%%%%%%%%%%%%%%%%%%%%%%%%%%%%%%%%%%%%%%%%%%%%%%%%%%%%%%%%%%%%%%%%%%%%%%
\subsection{Copyright}

Copyright \copyright{} 2017--2018 Niklas Beisert

This work may be distributed and/or modified under the
conditions of the \LaTeX{} Project Public License, either version 1.3
of this license or (at your option) any later version.
The latest version of this license is in
  \url{http://www.latex-project.org/lppl.txt}
and version 1.3 or later is part of all distributions of \LaTeX{}
version 2005/12/01 or later.

This work has the LPPL maintenance status `maintained'.

The Current Maintainer of this work is Niklas Beisert.

This work consists of the files |README.txt|, |childdoc.ins| and |childdoc.dtx|
as well as the derived files |childdoc.def|, |cdocsamp.tex|
with |cdocsch1.tex|, |cdocsch2.tex|, |cdocspt3.tex|, |cdocspt4.tex|,
|cdocsdrf.tex|, |cdocsfn1.tex|, |cdocsfn2.tex|
as well as |childdoc.pdf|.

%%%%%%%%%%%%%%%%%%%%%%%%%%%%%%%%%%%%%%%%%%%%%%%%%%%%%%%%%%%%%%%%%%%%%%%%%%%%%%%%
\subsection{Files and Installation}

The package consists of the files:
%
\begin{center}
\begin{tabular}{ll}
    |README.txt|   & readme file \\
    |childdoc.ins| & installation file \\
    |childdoc.dtx| & source file \\
    |childdoc.def| & definition file \\
    |cdocsamp.tex| & sample main file \\
    |cdocsch1.tex| & sample include file \\
    |cdocsch2.tex| & sample include file \\
    |cdocspt3.tex| & sample part file \\
    |cdocspt4.tex| & sample part file \\
    |cdocsdrf.tex| & sample redirection file \\
    |cdocsfn1.tex| & sample redirection file \\
    |cdocsfn2.tex| & sample redirection file \\
    |childdoc.pdf| & manual
\end{tabular}
\end{center}
%
The distribution consists of the files
|README.txt|, |childdoc.ins| and |childdoc.dtx|.
%
\begin{itemize}
\item
Run (pdf)\LaTeX{} on |childdoc.dtx|
to compile the manual |childdoc.pdf| (this file).
\item
Run \LaTeX{} on |childdoc.ins| to create the definitions file |childdoc.def|
and the sample |cdocsamp.tex| with include files
|cdocsch1.tex|, |cdocsch2.tex|, |cdocspt3.tex|, |cdocspt4.tex|,
|cdocsdrf.tex|, |cdocsfn1.tex|, |cdocsfn2.tex|.
Then copy the file |childdoc.def| to an appropriate directory of your \LaTeX{}
distribution, e.g.\ \textit{texmf-root}|/tex/latex/childdoc|.
\end{itemize}

%%%%%%%%%%%%%%%%%%%%%%%%%%%%%%%%%%%%%%%%%%%%%%%%%%%%%%%%%%%%%%%%%%%%%%%%%%%%%%%%
\subsection{Related CTAN Packages}

There are several other packages which offer a similar functionality:
%
\begin{itemize}
\item
The packages
\href{http://ctan.org/pkg/docmute}{\textsf{docmute}},
\href{http://ctan.org/pkg/includex}{\textsf{includex}} and
\href{http://ctan.org/pkg/standalone}{\textsf{standalone}}
provide commands to include only the document body of
a child file thus allowing both files to be compiled individually.
\item
The packages \href{http://ctan.org/pkg/subdocs}{\textsf{subdocs}}
and \href{http://ctan.org/pkg/subfiles}{\textsf{subfiles}}
provide structures in which the main and child documents can be
encapsulated and allowing them to be compiled individually.
The inclusion mechanism is different from the conventional |\include|.
\item
The package \href{http://ctan.org/pkg/combine}{\textsf{combine}}
is an elaborate solution to combine several documents into one.
\end{itemize}
%
See also the CTAN topic \href{http://ctan.org/topic/subdocs}{\textsf{subdocs}}
for further related packages.
The present package differs from the above solutions in that
a document structure constructed with the conventional |\include| mechanism
just needs two extra commands at the top of every file
such that all constituent files can be compiled individually.

%%%%%%%%%%%%%%%%%%%%%%%%%%%%%%%%%%%%%%%%%%%%%%%%%%%%%%%%%%%%%%%%%%%%%%%%%%%%%%%%
%\subsection{Feature Suggestions}
%
%The following is a list of features which may be useful for future
%versions of this package:
%%
%\begin{itemize}
%\item
%\ldots
%\end{itemize}

%%%%%%%%%%%%%%%%%%%%%%%%%%%%%%%%%%%%%%%%%%%%%%%%%%%%%%%%%%%%%%%%%%%%%%%%%%%%%%%%
\subsection{Revision History}

%%%%%%%%%%%%%%%%%%%%%%%%%%%%%%%%%%%%%%%%
\paragraph{v2.0:} 2018/12/30

\begin{itemize}
\item
immediate forward processing
\item
added |\childdocby| mechanism
\item
manual restructured
\end{itemize}

%%%%%%%%%%%%%%%%%%%%%%%%%%%%%%%%%%%%%%%%
\paragraph{v1.6:} 2018/01/17

\begin{itemize}
\item
application for development of include files
\item
corrections to manual
\end{itemize}

%%%%%%%%%%%%%%%%%%%%%%%%%%%%%%%%%%%%%%%%
\paragraph{v1.5:} 2017/05/21

\begin{itemize}
\item
more complete structuring introduced
\item
|\childdocof| introduced
\item
|\childdoc| renamed to |\childdocmain|
\item
|\childredirect| renamed to |\childdocforward| and |\childdocforwardprefix|
and functionality expanded
\end{itemize}

%%%%%%%%%%%%%%%%%%%%%%%%%%%%%%%%%%%%%%%%
\paragraph{v1.0:} 2017/04/27

\begin{itemize}
\item
manual and install package
\item
first version published on CTAN
\end{itemize}

%%%%%%%%%%%%%%%%%%%%%%%%%%%%%%%%%%%%%%%%
\paragraph{v0.6:} 2017/04/26

\begin{itemize}
\item
redirection mechanism added
\end{itemize}

%%%%%%%%%%%%%%%%%%%%%%%%%%%%%%%%%%%%%%%%
\paragraph{v0.5:} 2017/04/26

\begin{itemize}
\item
functionality in definition file
\end{itemize}


%%%%%%%%%%%%%%%%%%%%%%%%%%%%%%%%%%%%%%%%%%%%%%%%%%%%%%%%%%%%%%%%%%%%%%%%%%%%%%%%
%%%%%%%%%%%%%%%%%%%%%%%%%%%%%%%%%%%%%%%%%%%%%%%%%%%%%%%%%%%%%%%%%%%%%%%%%%%%%%%%
%%%%%%%%%%%%%%%%%%%%%%%%%%%%%%%%%%%%%%%%%%%%%%%%%%%%%%%%%%%%%%%%%%%%%%%%%%%%%%%%
\appendix

\settowidth\MacroIndent{\rmfamily\scriptsize 000\ }

 \DocInput{childdoc.dtx}

\end{document}
%</driver>
% \fi
%
% %%%%%%%%%%%%%%%%%%%%%%%%%%%%%%%%%%%%%%%%%%%%%%%%%%%%%%%%%%%%%%%%%%%%%%%%%%%%%%
% %%%%%%%%%%%%%%%%%%%%%%%%%%%%%%%%%%%%%%%%%%%%%%%%%%%%%%%%%%%%%%%%%%%%%%%%%%%%%%
% \section{Sample}
%\iffalse
%<*samplemain>
%\fi
%
% The following presents a sample document
% with two chapters, two parts, a title page,
% a compile flag as well as three forwarding files to set the flag.
% It consists of eight |.tex| files:
% \begin{center}
% \begin{tabular}{ll}
% |cdocsamp.tex|&main file\\
% |cdocsch1.tex|&include file for chapter 1\\
% |cdocsch2.tex|&include file for chapter 2\\
% |cdocspt3.tex|&include file for part 3\\
% |cdocspt4.tex|&include file for part 4\\
% |cdocsdrf.tex|&forwarding file for main file in draft mode\\
% |cdocsfi1.tex|&forwarding file for final version of chapter 1\\
% |cdocsfi2.tex|&forwarding file for final version of chapter 2\\
% \end{tabular}
% \end{center}
% Each of the eight files can be compiled directly by the \LaTeX{} compiler.
%
% %%%%%%%%%%%%%%%%%%%%%%%%%%%%%%%%%%%%%%
% \paragraph{Main File.}
%
% The main file is called |cdocsamp.tex|.
%
% Load the \textsf{childdoc} definitions and
% declare the filename for the main document:
%    \begin{macrocode}
\input{childdoc.def}
\childdocmain{}
%    \end{macrocode}

% Optional override for |\version| flag:
%    \begin{macrocode}
%%\ifchilddoc\else\providecommand{\version}{draft}\fi
%    \end{macrocode}

% Define the default values for the |\version| flag
% (|final| for the main file and |draft| for childs):
%    \begin{macrocode}
\ifchilddoc
\providecommand{\version}{draft}
\else
\providecommand{\version}{final}
\fi
%    \end{macrocode}

% Load the standard document class:
%    \begin{macrocode}
\documentclass[12pt]{article}
%    \end{macrocode}

% Start the document body:
%    \begin{macrocode}
\begin{document}
%    \end{macrocode}

% Declare a title page.
% Print title, part of document being processed and version flag:
%    \begin{macrocode}
\addtocounter{page}{-1}
\begin{center}
{\LARGE\bfseries{}childdoc example\par}
\vspace{1cm}
\ifchilddoc
\ifchilddocmanual part\else chapter\fi:
`\childdocname' of `\childdocjob'\par
\else
main document: `\childdocjob'\par
\fi
version: \version\par
\end{center}
\newpage
%    \end{macrocode}

% Manually include selected file,
% otherwise process as usual:
%    \begin{macrocode}
\ifchilddocmanual
\section*{part `\childdocname'}
\input{\childdocname}
\else
%    \end{macrocode}

% Include the two chapters:
%    \begin{macrocode}
\include{cdocsch1}
\include{cdocsch2}
%    \end{macrocode}

% Include the two parts unless only chapters should be displayed:
%    \begin{macrocode}
\ifchilddoc\else
\section{part three}
\input{cdocspt3}
\section{part four}
\input{cdocspt4}
\fi
%    \end{macrocode}

% Process as usual until here:
%    \begin{macrocode}
\fi
%    \end{macrocode}

% End of document body:
%    \begin{macrocode}
\end{document}
%    \end{macrocode}
%\iffalse
%</samplemain>
%\fi
%
% %%%%%%%%%%%%%%%%%%%%%%%%%%%%%%%%%%%%%%
% \paragraph{Chapter Include Files.}
%
% The include files are called |cdocsch1.tex| and |cdocsch2.tex|.
%
%\iffalse
%<*samplechap1|samplechap2>
%\fi

% Optional override for |\version| flag:
%    \begin{macrocode}
%%\providecommand{\version}{final}
%    \end{macrocode}

% Include the main document:
%    \begin{macrocode}
\input{childdoc.def}
\childdocof{cdocsamp}
%    \end{macrocode}

%\iffalse
%</samplechap1|samplechap2>
%\fi
%
%\iffalse
%<*samplechap1>
%\fi
% Some text for chapter 1:
%    \begin{macrocode}
\section{one}
some text in chapter one
%    \end{macrocode}

%\iffalse
%</samplechap1>
%\fi
% Some text for chapter 2:
%\iffalse
%<*samplechap2>
%\fi
%    \begin{macrocode}
\section{two}
more text in chapter two
%    \end{macrocode}

%\iffalse
%</samplechap2>
%\fi
%
% %%%%%%%%%%%%%%%%%%%%%%%%%%%%%%%%%%%%%%
% \paragraph{Part Include Files.}
%
% The include files are called |cdocspt3.tex| and |cdocspt4.tex|.
%
%\iffalse
%<*samplepart3|samplepart4>
%\fi

% Optional override for |\version| flag:
%    \begin{macrocode}
%%\providecommand{\version}{final}
%    \end{macrocode}

% Include the main document:
%    \begin{macrocode}
\input{childdoc.def}
\childdocby{cdocsamp}
%    \end{macrocode}

%\iffalse
%</samplepart3|samplepart4>
%\fi
%
%\iffalse
%<*samplepart3>
%\fi
% Some text for part 3:
%    \begin{macrocode}
some text in part three
%    \end{macrocode}

%\iffalse
%</samplepart3>
%\fi
% Some text for part 4:
%\iffalse
%<*samplepart4>
%\fi
%    \begin{macrocode}
more text in part four
%    \end{macrocode}

%\iffalse
%</samplepart4>
%\fi
%
% %%%%%%%%%%%%%%%%%%%%%%%%%%%%%%%%%%%%%%
% \paragraph{Forwarding for a Complete Draft.}
%
% The following forwarding file |cdocsdrf.tex|
% compiles the main document in draft mode:
%\iffalse
%<*sampledraft>
%\fi
%    \begin{macrocode}
\def\version{draft}
\input{childdoc.def}
\childdocforward{cdocsamp}
%    \end{macrocode}

%\iffalse
%</sampledraft>
%\fi
%
% %%%%%%%%%%%%%%%%%%%%%%%%%%%%%%%%%%%%%%
% \paragraph{Forwarding for Final Version of the Chapters.}
%
% The following forwarding files |cdocsfn1.tex| and |cdocsfn2.tex|
% (with identical content)
% compile the final versions of the child documents
% |cdocsch1.tex| and |cdocsch2.tex|, respectively:
%\iffalse
%<*samplefinal>
%\fi
%    \begin{macrocode}
\def\version{final}
\input{childdoc.def}
\childdocforwardprefix[cdocsamp]{cdocsfn}{cdocsch}
%    \end{macrocode}

%\iffalse
%</samplefinal>
%\fi
%
% %%%%%%%%%%%%%%%%%%%%%%%%%%%%%%%%%%%%%%
% \paragraph{Command Line Processing.}
%
% The following three command lines generate the output files
% |cdocscld|, |cdocscl1| and |cdocscl2|
% which should be identical to
% |cdocsdrf|, |cdocsch1| and |cdocsfn2|, respectively:
% \begin{center}
% \begin{tabular}{l}
% |latex -jobname cdocscld \|\\
% |  "\def\version{draft}\input{childdoc.def}\childdocforward{cdocsamp}"|\\
% |latex -jobname cdocscl1 \|\\
% |  "\input{childdoc.def}\childdocforward[cdocsamp]{cdocsch1}"|\\
% |latex -jobname cdocscl2 \|\\
% |  "\def\version{final}\input{childdoc.def}\childdocforward{cdocsch2}"|
% \end{tabular}
% \end{center}
% Note that the trailing backslash on each first line
% merely continues the input to the second line
% (for convenient cut ant paste).
% Furthermore, the command |latex| can be replaced by any
% of its alternative versions such as |pdflatex|.
%
% %%%%%%%%%%%%%%%%%%%%%%%%%%%%%%%%%%%%%%%%%%%%%%%%%%%%%%%%%%%%%%%%%%%%%%%%%%%%%%
% %%%%%%%%%%%%%%%%%%%%%%%%%%%%%%%%%%%%%%%%%%%%%%%%%%%%%%%%%%%%%%%%%%%%%%%%%%%%%%
% \section{Implementation}
%\iffalse
%<*package>
%\fi
%
% This section describes the definitions file |childdoc.def|.

% The definitions cannot be loaded using |\usepackage| or |\RequirePackage|
% which has a mechanism to prevent loading a style file more than once.
% When loading the definitions by means of |\input|
% multiple instances have to be prevented manually:
%\iffalse
%This code needs to be before the `\ProvidesFile' directive
%which is defined at the beginning of this file.
%Therefore it is also placed there and commented out here.
%</package>
%<*discard>
%\fi
%    \begin{macrocode}
\ifdefined\childdocmain\endinput\fi
%    \end{macrocode}
%\iffalse
%</discard>
%<*package>
%\fi
%
% \macro{\ifchilddoc}
% \macro{\ifchilddocmanual}
% The conditional |\ifchilddoc| tells whether a
% child (true) or main (false) document is being compiled.
% The conditional |\ifchilddocmanual| tells whether
% the |\includeonly| mechanism is used (false) or
% the selection of child files must be performed manually (true).
% The definitions initialise to false:
%    \begin{macrocode}
\newif\ifchilddoc
\newif\ifchilddocmanual
%    \end{macrocode}

% \macro{\childdocname}
% \macro{\childdocjob}
% The macro |\childdocname| stores the name of the main document
% to be compiled. The macro |\childdocjob| stores the name of
% the document on which the \LaTeX{} compiler was originally invoked.
% The content of |\jobname| cannot be compared
% to filenames specified in the source due to different catcodes.
% The following code rescans |\jobname|, stores the result
% in |\childdocname| and saves a copy in |\childdocjob|:
%    \begin{macrocode}
\edef\childdocname{\scantokens\expandafter{\jobname\noexpand}}
\let\childdocjob\childdocname
%    \end{macrocode}

% \macro{\childdocdisable}
% The macro |\childdocdisable| prevents the main file
% from being processed more than once.
% At this stage, the main document command |\childdocmain|
% is assumed to be called once again where it should do nothing.
% Any subsequent call to it should prevent
% a secondary processing of the main document
% It overwrites the forwarding commands
% |\childdocof| and |\childdocforward|
% with empty macros to prevent further inclusions of the main document:
%    \begin{macrocode}
\newcommand{\childdocdisable}
{
  \renewcommand{\childdocmain}[1]{\renewcommand{\childdocmain}[1]{\endinput}}
  \renewcommand{\childdocof}[1]{}
  \renewcommand{\childdocby}[2][]{}
  \renewcommand{\childdocforward}[2][]{}
  \renewcommand{\childdocdisable}{}
}
%    \end{macrocode}

% \macro{\childdocmain}
% The macro |\childdocmain| is to be called at the top of the main file
% with nothing or the main filename (without extension) as argument.
% First, it breaks loops.
% If the argument is not empty and does not match |\childdocname|
% (which is set by the first inclusion of |childdoc.def|),
% |\ifchilddoc| is set to true, |\includeonly| is applied to the child file
% and |\jobname| is set to the main file
% (for proper handling of |.aux| files):
%    \begin{macrocode}
\newcommand{\childdocmain}[1]
{
  \childdocdisable\childdocmain{}
  \if?#1?\else
    \begingroup
      \def\childdoctmp{#1}
      \ifx\childdoctmp\childdocname
        \def\childdoctmp{}
      \else
        \def\childdoctmp
        {
          \childdoctrue
          \includeonly{\childdocname}
          \def\childdocjob{#1}
          \def\jobname{#1}
        }
      \fi
      \expandafter
    \endgroup
    \childdoctmp
  \fi
}
%    \end{macrocode}

% \macro{\childdocof}
% The command |\childdocof| redirects
% compilation to the main file |#1|.
%    \begin{macrocode}
\newcommand{\childdocof}[1]
{
  \childdocdisable
  \childdoctrue
  \includeonly{\childdocname}
  \def\jobname{#1}
  \def\childdocjob{#1}
  \input{#1}
}
%    \end{macrocode}

% \macro{\childdocby}
% The command |\childdocby| ....
%    \begin{macrocode}
\newcommand{\childdocby}[2][]
{
  \childdocdisable
  \childdoctrue
  \childdocmanualtrue
  \if?#1?\else
    \def\jobname{#2}
  \fi
  \def\childdocjob{#2}
  \input{#2}
  \endinput
}
%    \end{macrocode}

% \macro{\childdocforward}
% The command |\childdocforward| redirects
% compilation to the main file or
% (if the optional argument is given) a child file.
% Parameters are set as if the main file
% or a child file starting with |\childdocof| was compiled.
% Then compilation is handed over to the main file:
%    \begin{macrocode}
\newcommand{\childdocforward}[2][]
{
  \begingroup
    \if?#1?
      \def\childdoctmp
      {
        \def\childdocname{#2}
        \def\childdocjob{#2}
        \def\jobname{#2}
        \input{#2}
        \endinput
      }
    \else
      \def\childdoctmp
      {
        \childdocdisable
        \def\childdocname{#2}
        \childdoctrue
        \includeonly{#2}
        \def\childdocjob{#1}
        \def\jobname{#1}
        \input{#1}
        \endinput
      }
    \fi
    \expandafter
  \endgroup
  \childdoctmp
}
%    \end{macrocode}

% \macro{\childdocforwardprefix}
% The command |\childdocforwardprefix| redirects
% compilation to the main or a child file by means of a pattern.
% The prefix |#1| in the current filename is replaced by |#2|
% and the suffix of the current filename is kept
% (it is assumed that the filename does not contain the substring `|~~~|'
% which is used as a delimiter).
% Compilation is handed over to the new file by |\childdocforward|:
%    \begin{macrocode}
\newcommand{\childdocforwardprefix}[3][]
{
  \begingroup
    \def\childdocextract #2##1~~~{\def\childdoctmp{\childdocforward[#1]{#3##1}}}
    \expandafter\childdocextract\childdocname~~~
    \expandafter
  \endgroup
  \childdoctmp
}
%    \end{macrocode}

% \macro{\childdoc}
% The deprecated macro |\childdoc| is a legacy version of |\childdocmain|:
%    \begin{macrocode}
\newcommand{\childdoc}{\childdocmain}
%    \end{macrocode}

% \macro{\childdocredirect}
% The deprecated macro |\childdocredirect| is a legacy version
% of |\childdocforward| and |\childdocforwardprefix|:
%    \begin{macrocode}
\newcommand{\childdocredirect}[2][]
{
  \begingroup
    \if?#1?
      \def\childdoctmp{\childdocforward{#2}}
    \else
      \def\childdoctmp{\childdocforwardprefix{#1}{#2}}
    \fi
    \expandafter
  \endgroup
  \childdoctmp
}
%    \end{macrocode}

%\iffalse
%</package>
%\fi
%
\endinput
\childdocforward[cdocsamp]{cdocsch1}"|\\
% |latex -jobname cdocscl2 \|\\
% |  "\def\version{final}% \iffalse
%
% childdoc.dtx Copyright (C) 2017-2018 Niklas Beisert
%
% This work may be distributed and/or modified under the
% conditions of the LaTeX Project Public License, either version 1.3
% of this license or (at your option) any later version.
% The latest version of this license is in
%   http://www.latex-project.org/lppl.txt
% and version 1.3 or later is part of all distributions of LaTeX
% version 2005/12/01 or later.
%
% This work has the LPPL maintenance status `maintained'.
%
% The Current Maintainer of this work is Niklas Beisert.
%
% This work consists of the files childdoc.dtx and childdoc.ins
% and the derived files childdoc.def and cdocsamp.tex with
% cdocsch1.tex, cdocsch2.tex, cdocsdrf.tex, cdocsfn1.tex, cdocsfn2.tex.
%
%<package>\ifdefined\childdocmain\endinput\fi
%<package>\ProvidesFile{childdoc.def}[2018/12/30 v2.0 child document driver]
%<samplemain>\ProvidesFile{cdocsamp.tex}[2018/12/30 v2.0 sample for childdoc]
%<*driver>
%\ProvidesFile{childdoc.drv}[2018/12/30 v2.0 childdoc reference manual file]
\PassOptionsToClass{10pt,a4paper}{article}
\documentclass{ltxdoc}

\usepackage[margin=35mm]{geometry}
\usepackage{hyperref}
\usepackage{hyperxmp}
\usepackage[usenames]{color}

\hypersetup{colorlinks=true}
\hypersetup{pdfstartview=FitH}
\hypersetup{pdfpagemode=UseNone}
\hypersetup{pdfsource={}}
\hypersetup{pdflang={en-UK}}
\hypersetup{pdfcopyright={Copyright 2017-2018 Niklas Beisert.
  This work may be distributed and/or modified under the
  conditions of the LaTeX Project Public License, either version 1.3
  of this license or (at your option) any later version.}}
\hypersetup{pdflicenseurl={http://www.latex-project.org/lppl.txt}}
\hypersetup{pdfcontactaddress={ETH Zurich, ITP, HIT K,
  Wolfgang-Pauli-Strasse 27}}
\hypersetup{pdfcontactpostcode={8093}}
\hypersetup{pdfcontactcity={Zurich}}
\hypersetup{pdfcontactcountry={Switzerland}}
\hypersetup{pdfcontactemail={nbeisert@itp.phys.ethz.ch}}
\hypersetup{pdfcontacturl={http://people.phys.ethz.ch/\xmptilde nbeisert/}}

\newcommand{\secref}[1]{\hyperref[#1]{section \ref*{#1}}}

\parskip1ex
\parindent0pt
\let\olditemize\itemize
\def\itemize{\olditemize\parskip0pt}

\begin{document}

\title{The \textsf{childdoc} Package}
\hypersetup{pdftitle={The childdoc Package}}
\author{Niklas Beisert\\[2ex]
  Institut f\"ur Theoretische Physik\\
  Eidgen\"ossische Technische Hochschule Z\"urich\\
  Wolfgang-Pauli-Strasse 27, 8093 Z\"urich, Switzerland\\[1ex]
  \href{mailto:nbeisert@itp.phys.ethz.ch}
  {\texttt{nbeisert@itp.phys.ethz.ch}}}
\hypersetup{pdfauthor={Niklas Beisert}}
\hypersetup{pdfsubject={Manual for the LaTeX2e Package childdoc}}
\date{30 December 2018, \textsf{v2.0}}
\maketitle

\begin{abstract}\noindent
\textsf{childdoc} is a \LaTeXe{} package
that enables the direct compilation
of document sections included by |\include|
to individual files.
\end{abstract}

\begingroup
\parskip0ex
\tableofcontents
\endgroup

%%%%%%%%%%%%%%%%%%%%%%%%%%%%%%%%%%%%%%%%%%%%%%%%%%%%%%%%%%%%%%%%%%%%%%%%%%%%%%%%
%%%%%%%%%%%%%%%%%%%%%%%%%%%%%%%%%%%%%%%%%%%%%%%%%%%%%%%%%%%%%%%%%%%%%%%%%%%%%%%%
\section{Introduction}

\LaTeX{} provides a mechanism to structure a large document (such as a book)
into a main file and several child files (containing the chapters)
using the |\include| command.
This mechanism is beneficial for documents
which span hundreds of pages in order to
make the source file(s) more manageable.
Moreover, compilation can be restricted to
selected child files by means of the |\includeonly| command.
The latter feature can be used to reduce the compilation time while editing
(this was significantly more useful in the earlier days of \LaTeX{})
or to generate a smaller document which is easier to navigate.
Another application of |\includeonly| is to generate
documents consisting of selected parts of the complete document.

However, there are a few drawbacks of the plain |\include| mechanism:
\begin{itemize}
\item
The child files cannot be compiled on their own,
they can only be compiled via the main file.
A naive editing environment
(such as a text editor with an option
to have the current file processed by \LaTeX)
may require one to switch to the main file before compiling;
attempting to compile the child file produces errors.
\item
The main file must be modified (each time)
to adjust the |\includeonly| command
to the present needs. This easily leaves the main file in a messy state.
\item
The generated document will always carry the filename
of the main document. This is inconvenient if
several child files are to be compiled and
to be kept for distribution.
\end{itemize}

The present package provides a simple interface
to make child files individually compilable by \LaTeX{}.
Compiling a child file then has the same effect as compiling
the main file with an |\includeonly| command
to select the appropriate child.
Moreover the generated document will carry the name of the child
rather than the main file.
This resolves all three above issues.

This feature is meant to make the editing of books,
thesis documents and lecture notes somewhat more convenient.
However, the package can also be used efficiently for
composing a series of documents (such as exercise sheets)
which are typically distributed individually.
It then assists the author in generating the individual documents
(potentially in different versions)
as well as a document containing the collected series.
Another application is in developing style files
or other kinds of included material
where compilation of the style file could redirect
to a sample or test file.

%%%%%%%%%%%%%%%%%%%%%%%%%%%%%%%%%%%%%%%%%%%%%%%%%%%%%%%%%%%%%%%%%%%%%%%%%%%%%%%%
%%%%%%%%%%%%%%%%%%%%%%%%%%%%%%%%%%%%%%%%%%%%%%%%%%%%%%%%%%%%%%%%%%%%%%%%%%%%%%%%
\section{Usage}

First of all, the package \textsf{childdoc} is \emph{not} a standard
\LaTeXe{} |.sty| style file! Therefore it needs to be invoked in
a non-standard way.

%%%%%%%%%%%%%%%%%%%%%%%%%%%%%%%%%%%%%%%%%%%%%%%%%%%%%%%%%%%%%%%%%%%%%%%%%%%%%%%%
\subsection{Included Files}
\label{sec:include}

%%%%%%%%%%%%%%%%%%%%%%%%%%%%%%%%%%%%%%%%
\DescribeMacro{\childdocmain}
To use the package, add the commands
\begin{center}
\begin{tabular}{l}
|\input{childdoc.def}|\\
|\childdocmain{}|\\
\end{tabular}
\end{center}
at the very top of the main \LaTeX{} file,
in particular \emph{before} the |\documentclass| statement!
The argument of |\childdocmain| should be left empty
(but it must be present).

%%%%%%%%%%%%%%%%%%%%%%%%%%%%%%%%%%%%%%%%
\DescribeMacro{\childdocof}
Furthermore, add the commands
\begin{center}
\begin{tabular}{l}
|\input{childdoc.def}|\\
|\childdocof{|\textit{main}|}|\\
\end{tabular}
\end{center}
at the top of every child file \textit{child}
which is included by |\include{|\textit{child}|}|
from within the main file
(or at least for those files to be compiled individually).
The argument \textit{main} must be the filename of the main file.

There are a couple of
considerations in setting up the main and child documents:

%%%%%%%%%%%%%%%%%%%%%%%%%%%%%%%%%%%%%%%%
\paragraph{Restrictions.}

Please note the following restrictions:
\begin{itemize}
\item
|\childdocmain| must be called with one argument \textit{main}
to ensure compatibility with earlier version of the package.
It must either be empty (|\childdocmain{}|)
or precisely match the filename of the main file in which it is specified.
See \secref{sec:detection} for further information.
\item
The filename \textit{main} must be specified without the |.tex| extension.
\item
The filename \textit{main} is case sensitive
(even in case-insensitive file systems)
due to internal string comparison.
\item
The argument \textit{main} should be fully expanded, it cannot be a macro.
\item
Subdirectories and special characters should be avoided in filenames.
\item
The command |\childdocmain{|\textit{main}|}| must be followed by a whitespace.
It should not be followed immediately by another command
or by a comment mark `|%|'.
This is because the \TeX{} parser reads the token immediately following
the argument of |\childdocmain| and puts it
at the beginning of every child section;
however, a white\-space is ignored.
\end{itemize}

%%%%%%%%%%%%%%%%%%%%%%%%%%%%%%%%%%%%%%%%
\paragraph{Content of Main File.}

It is advisable to place all content in the child files included by |\include|.
Any output contained in the main file will appear in all child documents
unless suppressed manually;
it cannot be suppressed automatically by the |\includeonly| directive
and thus should normally be avoided.
A method to include some content in the main file
by means of conditional processing is described in \secref{sec:conditional}.

%%%%%%%%%%%%%%%%%%%%%%%%%%%%%%%%%%%%%%%%
\paragraph{Page Numbering.}

When only a part of the document is compiled,
the appropriate numbering of pages
(as well as other status parameters)
is determined from the |.aux| files.
The latter contain information from previous passes.
However this information needs to propagate through
all intermediate child documents.
Therefore the page numbering in child documents may well
be inconsistent until the complete document is compiled at least once.

A useful (if unconventional) way to always ensure a consistent
page numbering is to restart the numbering in each child document
and denote the pages by `\textit{child}|.|\textit{page}'
where \textit{child} represents the chapter/section number of the child file.
This can be achieved by the command
|\numberwithin{page}{|\textit{child}|}|
of the \textsf{amsmath} package
where \textit{child} can be |chapter| or |section|
depending on the chosen structuring.
Alternatively, one can modify the macro |\thepage| appropriately
and reset the counter |page| at the start of each child file.

%%%%%%%%%%%%%%%%%%%%%%%%%%%%%%%%%%%%%%%%%%%%%%%%%%%%%%%%%%%%%%%%%%%%%%%%%%%%%%%%
\subsection{Conditional Processing}
\label{sec:conditional}

The package provides a mechanism to compile different versions
of a document. To customise the versions further some conditional processing
can come in handy to distinguish which version is being compiled.
The package provides two macros to describe the compilation context:

%%%%%%%%%%%%%%%%%%%%%%%%%%%%%%%%%%%%%%%%
\DescribeMacro{\ifchilddoc}
The conditional |\ifchilddoc| distinguishes between the compilation of
child documents and the main document:
%
\begin{center}
|\ifchilddoc |\textit{child-code}| |[|\||else |\textit{main-code}]| \||fi|
\end{center}

%%%%%%%%%%%%%%%%%%%%%%%%%%%%%%%%%%%%%%%%
\DescribeMacro{\childdocname}
\DescribeMacro{\childdocjob}
The macro |\childdocname| contains the filename (without extension)
of the main or child file being processed.
Note that |\childdocjob| will always contain the name of the main file.

%%%%%%%%%%%%%%%%%%%%%%%%%%%%%%%%%%%%%%%%
\paragraph{Title Page.}

Conditional processing can be used to include a title or banner page
in the main document when proper precautions are taken.
Importantly, the code in the main file should ensure that the page counter
(as well as other status parameters which are stored in the |.aux| files)
takes the same value after the conditional processing.
Otherwise the page numbers may take divergent values
depending on which part is compiled.

For example, a title page could be declared by:
%
\begin{center}
\begin{tabular}{l}
|\ifchilddoc\||else|\\
|\addtocounter{page}{-1}|\\
\textit{code for title page}\\
|\newpage|\\
|\||fi|
\end{tabular}
\end{center}
%
A banner page for the child documents can be generated by:
%
\begin{center}
\begin{tabular}{l}
|\ifchilddoc|\\
|\addtocounter{page}{-1}|\\
\textit{code for banner page}\\
|\newpage|\\
|\||fi|
\end{tabular}
\end{center}
%
Here one could write a message such as:
\begin{center}
|This is the part \childdocname{} of \childdocjob{}.|
\end{center}

%%%%%%%%%%%%%%%%%%%%%%%%%%%%%%%%%%%%%%%%%%%%%%%%%%%%%%%%%%%%%%%%%%%%%%%%%%%%%%%%
\subsection{Flags}
\label{sec:flags}

The package makes it easy to generate different versions
of the main or child documents.
To this end compilation flags can be defined
and assigned different default values.
They will be particularly useful in conjunction
with the forwarding mechanism described in \secref{sec:forward}.

For example, it may be useful to have a flag |\version|
which can be set to |draft| or |final|.
The document source will contain some conditional code
depending on the value of |\version|.
Suppose further, the flag should default to |final| for the main file
and to |draft| for child files
which is a natural assignment for editing the document.
This is achieved by placing the following code
in the preamble of the main document
(below the |\childdocmain| directive):
%
\begin{center}
\begin{tabular}{l}
|\ifchilddoc|\\
|\providecommand{\version}{draft}|\\
|\||else|\\
|\providecommand{\version}{final}|\\
|\||fi|
\end{tabular}
\end{center}
%
The definition by |\providecommand| makes sure
that previous definitions are not overwritten.
Further statements |\providecommand{\version}{...}|
can thus be added before the above code to override it.

For the main file, one might add a line
(between |\childdocmain| and the above block)
%
\begin{center}
|%\ifchilddoc\||else\providecommand{\version}{draft}\||fi|
\end{center}
%
which can be uncommented to produce a draft version.
Likewise one can add a line to the very top of a child file
(above the |\childdocof{|\textit{main}|}| directive)
%
\begin{center}
|%\providecommand{\version}{final}|
\end{center}
%
which can be uncommented to produce the final version of this child document.

%%%%%%%%%%%%%%%%%%%%%%%%%%%%%%%%%%%%%%%%%%%%%%%%%%%%%%%%%%%%%%%%%%%%%%%%%%%%%%%%
\subsection{Forwarding}
\label{sec:forward}

Different versions of the main or child documents
using compilation flags as described in \secref{sec:flags}
can be (permanently) stored in different files
for convenient compilation, viewing and distribution.
To this end, the package defines a command
to pass on compilation to a different file:

%%%%%%%%%%%%%%%%%%%%%%%%%%%%%%%%%%%%%%%%
\DescribeMacro{\childdocforward}
The command |\childdocforward| redirects processing to
another source file:
%
\begin{center}
\begin{tabular}{l}
|\input{childdoc.def}|\\
|\childdocforward[|\textit{main}|]{|\textit{dest}|}|\\
\end{tabular}
\end{center}
%
The argument \textit{dest} is the destination file
(without extension).
It should be the main file or one of the child files.
Note that further \textsf{childdoc} directives
such as |\childdocof| and |\childdocforward|
in the indicated file will be processed in this form.
The optional argument \textit{main}
passes on directly to the main file \textit{main}
while pretending to compile the child \textit{dest}.
This form behaves as if \textit{dest}
issues |\childdocof{|\textit{main}|}| right away,
and no further \textsf{childdoc} directives will be processed.

%%%%%%%%%%%%%%%%%%%%%%%%%%%%%%%%%%%%%%%%
\DescribeMacro{\...prefix}
In the alternative form |\childdocforwardprefix|,
%
\begin{center}
\begin{tabular}{l}
|\input{childdoc.def}|\\
|\childdocforwardprefix[|\textit{main}|]{|\textit{prefix}|}{|\textit{dest}|}|
\end{tabular}
\end{center}
%
the destination file is determined by a pattern
depending on the current file:
To make this work, the current file must be called
`{\textit{prefix}\hspace{0.2em}\textit{suffix}}'
with \textit{prefix} matching precisely the argument.
Processing is then passed on to the file
`{\textit{dest}\hspace{0.2em}\textit{suffix}}'.
Surely, the same effect is achieved by
directly specifying the
argument `{\textit{dest}\hspace{0.2em}\textit{suffix}}'
in the first form.
However, that requires to set up a different file
for each child. With the alternative form of the command
all these files can have exactly the same content
which simplifies setting them up and maintaining them.

For example, the following file |draft.tex|
with a compilation flag |\version| as described in \secref{sec:flags}
compiles the main document as a draft:
%
\begin{center}
\begin{tabular}{l}
|\def\version{draft}|\\
|\input{childdoc.def}|\\
|\childdocforward{|\textit{main}|}|
\end{tabular}
\end{center}
%
Likewise, the following files |final|\textit{nn}|.tex|
compile the final version of the child document
|child|\textit{nn}|.tex|:
%
\begin{center}
\begin{tabular}{l}
|\def\version{final}|\\
|\input{childdoc.def}|\\
|\childdocforwardprefix{final}{child}|
\end{tabular}
\end{center}
%

Note that when several versions of a main file and/or of each child file
are to be generated, it may be convenient to set up a |Makefile| or
shell script to automatise the process.

%%%%%%%%%%%%%%%%%%%%%%%%%%%%%%%%%%%%%%%%%%%%%%%%%%%%%%%%%%%%%%%%%%%%%%%%%%%%%%%%
\subsection{Command Line Processing}
\label{sec:commandline}

The effect of redirection files can also be achieved by invoking
the \LaTeX{} compiler with a more elaborate command line.
Most conveniently this should be done as part
of a shell script or a |Makefile|.

When using \textsf{childdoc} in the main file, the following
command lines effectively perform a redirection
(note that depending on the shell being used,
backslashes may have to be doubled: `|\|' $\to$ `|\\|'):
%
\begin{center}
|... -jobname "|\textit{target}|" |\\|"|[\textit{flags}]%
|\input{childdoc.def}\childdocforward[|\textit{main}|]{|\textit{dest}|}"|
\end{center}
%
Here \textit{target} is the name of the output file,
\textit{main} is the name of the main file
and \textit{dest} is the name of the main or child file to be processed
(all filenames without extensions).
The optional argument \textit{main} can be omitted
if \textit{main} matches \textit{dest}.
Optionally, compilation \textit{flags} can be defined via |\def| commands.
This command line makes the \TeX{} engine believe
it is compiling the file \textit{target}
whose content is specified as the latter parameter.
The provided code then forwards the processing to
\textit{main} or \textit{dest} as described in \secref{sec:forward}.

%%%%%%%%%%%%%%%%%%%%%%%%%%%%%%%%%%%%%%%%%%%%%%%%%%%%%%%%%%%%%%%%%%%%%%%%%%%%%%%%
\subsection{Include by Input}
\label{sec:input}

Including child documents by |\include| has some restrictions by design.
Most notably, the content of a child document always occupies
its own set of pages; pages cannot be shared between child documents.
Usually, this behaviour makes perfect sense
because each child document contain an essential part of the document.
However, in some situations it may be desirable to compose
a document from a collection of parts
without having mandatory page breaks between then.
For this case, the package
provides a mechanism to include parts
by |\input| which can also be processed individually.
However, by construction this mechanism
requires manual handling of the content to be output.

%%%%%%%%%%%%%%%%%%%%%%%%%%%%%%%%%%%%%%%%
\DescribeMacro{\ifchilddocmanual}
The main file should be prepared as usual, see \secref{sec:include}.
However, the document body must make a distinction
between processing of an individual part and of the main document, e.g.:
%
\begin{center}
\begin{tabular}{l}
|\ifchilddocmanual|\\
|\input{\childdocname}|\\
|\||else|\\
\textit{document body with }|\input{|\textit{part}|}|\\
|\||fi|
\end{tabular}
\end{center}
%
The conditional |\ifchilddocmanual| is true whenever
a part to be included by |\input| is being compiled,
and the name of the part is stored in |\childdocname|.

%%%%%%%%%%%%%%%%%%%%%%%%%%%%%%%%%%%%%%%%
\DescribeMacro{\childdocby}
Each part to be included by |\input| should start with:
%
\begin{center}
\begin{tabular}{l}
|\input{childdoc.def}|\\
|\childdocby{|\textit{main}|}|\\
\end{tabular}
\end{center}
%
The directive |\childdocby| is similar to |\childdocof|
described in \secref{sec:include},
but the subsequent selection of content must be done manually.
To that end, both |\ifchilddoc| and |\ifchilddocmanual|
will be true upon processing of a part,
and the name of the part is stored in |\childdocname|.
Note that |\jobname| will be set to the filename of the current part
so that each part receives an individual |.aux| file
that does not interfere with the |.aux| file(s) of the main document.
This behaviour can be altered by the alternative form
|\childdocby[*]{|\textit{main}|}| (with a non-empty optional argument)
which uses the |.aux| file of the main document
by setting |\jobname| to \textit{main}.

%%%%%%%%%%%%%%%%%%%%%%%%%%%%%%%%%%%%%%%%%%%%%%%%%%%%%%%%%%%%%%%%%%%%%%%%%%%%%%%%
\subsection{Driver Development}
\label{sec:driver}

The \textsf{childdoc} mechanism can also be use for the development
of definition files such as \LaTeX{} styles or classes.
This case differs from the above setup with multiple parts
included by |\include| in that no |\includeonly| should be invoked.
This can be achieved by starting the include file
(before |\ProvidesPackage|) with:
%
\begin{center}
\begin{tabular}{l}
|\input{childdoc.def}|\\
|\childdocforward{|\textit{main}|}|\\
\end{tabular}
\end{center}
%
or alternatively with:
%
\begin{center}
\begin{tabular}{l}
|\input{childdoc.def}|\\
|\childdocby{|\textit{main}|}|\\
\end{tabular}
\end{center}
%
Both forms have slightly different effects as described above.
The main file is prepared as usual, see \secref{sec:include}.

%%%%%%%%%%%%%%%%%%%%%%%%%%%%%%%%%%%%%%%%%%%%%%%%%%%%%%%%%%%%%%%%%%%%%%%%%%%%%%%%
\subsection{Legacy Detection}
\label{sec:detection}

The directive |\childdocmain| in the main file can detect
whether the complete document or merely a child is to be compiled
even without using the directive |\childdocof|.
This method is deprecated because it is less robust
and there is no compelling reason to use it;
it is merely provided for backward compatibility
and it may be removed in future versions.

If the detection mechanism is to be used,
it is mandatory to correctly specify
the filename of the main file as the argument of |\childdocmain|:
%
\begin{center}
\begin{tabular}{l}
|\input{childdoc.def}|\\
|\childdocmain{|\textit{main}|}|\\
\end{tabular}
\end{center}
%
If |\jobname| does not match the argument \textit{main} of |\childdocmain|,
it is assumed that |\jobname| points to the child file to be compiled.
When using |\childdocmain| with the main file specified as argument,
it suffices to start a child file
with just |\input{|\textit{main}|}|
without loading of the package and using |\childdocof|.
If instead all processing is done
with the appropriate \textsf{childdoc} directives,
the argument of \textit{main} of |\childdocmain| can be empty.

An alternative version of the command line processing described
in \secref{sec:commandline} using the detection mechanism reads:
%
\begin{center}
|... -jobname "|\textit{target}|" "|[\textit{flags}]%
[|\def\jobname{|\textit{dest}|}|]|\input{|\textit{main}|}"|
\end{center}

%%%%%%%%%%%%%%%%%%%%%%%%%%%%%%%%%%%%%%%%%%%%%%%%%%%%%%%%%%%%%%%%%%%%%%%%%%%%%%%%
\subsection{Manual Code}
\label{sec:manual}

In case one cannot be certain whether the definitions file |childdoc.def|
is installed on the target \TeX{} distribution
and one prefers not to ship it,
it is conceivable to paste a few relevant commands into the sources.

To that end, drop all statements |\input{childdoc.def}|
and perform the replacements as outlined below.
Instead of |\childdocmain{|\textit{main}|}| add the following code
to the top of the main file:
%
\begin{center}
\begin{tabular}{l}
|\||ifdefined\childdocname\endinput\||fi\newif\ifchilddoc|\\
|\edef\childdocname{\scantokens\expandafter{\jobname\noexpand}}|\\
|\def\childdocmain{|\textit{main}|}\||ifx\childdocmain\childdocname\||else|\\
|\childdoctrue\includeonly{\childdocname}\let\jobname\childdocmain\||fi|\\
\end{tabular}
\end{center}
%
Instead of |\childdocof{|\textit{main}|}| just include the main file
at the top of each child file:
%
\begin{center}
|\input{|\textit{main}|}|
\end{center}
%
A simple redirection |\childdocforward{|\textit{dest}|}| is achieved by:
%
\begin{center}
|\def\jobname{|\textit{dest}|}\input{\jobname}|
\end{center}
%
The redirection with prefix
|\childdocforwardprefix[|\textit{prefix}|]{|\textit{dest}|}|
is accomplished by:
%
\begin{center}
\begin{tabular}{l}
|{\edef\jobname{\scantokens\expandafter{\jobname\noexpand}}|\\
|\def\redirectjob |\textit{prefix}|#1~~~{\gdef\jobname{|\textit{dest}|#1}}|\\
|\expandafter\redirectjob\jobname~~~}\input{\jobname}|
\end{tabular}
\end{center}

In an alternative approach,
child documents can be compiled by a specific command line
without additional code or specific definitions:
%
\begin{center}
|... -jobname "|\textit{target}|" "|[\textit{flags}]%
|\includeonly{|\textit{dest}|}\input{|\textit{main}|}"|
\end{center}
%

%%%%%%%%%%%%%%%%%%%%%%%%%%%%%%%%%%%%%%%%%%%%%%%%%%%%%%%%%%%%%%%%%%%%%%%%%%%%%%%%
%%%%%%%%%%%%%%%%%%%%%%%%%%%%%%%%%%%%%%%%%%%%%%%%%%%%%%%%%%%%%%%%%%%%%%%%%%%%%%%%
\section{Information}

%%%%%%%%%%%%%%%%%%%%%%%%%%%%%%%%%%%%%%%%%%%%%%%%%%%%%%%%%%%%%%%%%%%%%%%%%%%%%%%%
\subsection{Copyright}

Copyright \copyright{} 2017--2018 Niklas Beisert

This work may be distributed and/or modified under the
conditions of the \LaTeX{} Project Public License, either version 1.3
of this license or (at your option) any later version.
The latest version of this license is in
  \url{http://www.latex-project.org/lppl.txt}
and version 1.3 or later is part of all distributions of \LaTeX{}
version 2005/12/01 or later.

This work has the LPPL maintenance status `maintained'.

The Current Maintainer of this work is Niklas Beisert.

This work consists of the files |README.txt|, |childdoc.ins| and |childdoc.dtx|
as well as the derived files |childdoc.def|, |cdocsamp.tex|
with |cdocsch1.tex|, |cdocsch2.tex|, |cdocspt3.tex|, |cdocspt4.tex|,
|cdocsdrf.tex|, |cdocsfn1.tex|, |cdocsfn2.tex|
as well as |childdoc.pdf|.

%%%%%%%%%%%%%%%%%%%%%%%%%%%%%%%%%%%%%%%%%%%%%%%%%%%%%%%%%%%%%%%%%%%%%%%%%%%%%%%%
\subsection{Files and Installation}

The package consists of the files:
%
\begin{center}
\begin{tabular}{ll}
    |README.txt|   & readme file \\
    |childdoc.ins| & installation file \\
    |childdoc.dtx| & source file \\
    |childdoc.def| & definition file \\
    |cdocsamp.tex| & sample main file \\
    |cdocsch1.tex| & sample include file \\
    |cdocsch2.tex| & sample include file \\
    |cdocspt3.tex| & sample part file \\
    |cdocspt4.tex| & sample part file \\
    |cdocsdrf.tex| & sample redirection file \\
    |cdocsfn1.tex| & sample redirection file \\
    |cdocsfn2.tex| & sample redirection file \\
    |childdoc.pdf| & manual
\end{tabular}
\end{center}
%
The distribution consists of the files
|README.txt|, |childdoc.ins| and |childdoc.dtx|.
%
\begin{itemize}
\item
Run (pdf)\LaTeX{} on |childdoc.dtx|
to compile the manual |childdoc.pdf| (this file).
\item
Run \LaTeX{} on |childdoc.ins| to create the definitions file |childdoc.def|
and the sample |cdocsamp.tex| with include files
|cdocsch1.tex|, |cdocsch2.tex|, |cdocspt3.tex|, |cdocspt4.tex|,
|cdocsdrf.tex|, |cdocsfn1.tex|, |cdocsfn2.tex|.
Then copy the file |childdoc.def| to an appropriate directory of your \LaTeX{}
distribution, e.g.\ \textit{texmf-root}|/tex/latex/childdoc|.
\end{itemize}

%%%%%%%%%%%%%%%%%%%%%%%%%%%%%%%%%%%%%%%%%%%%%%%%%%%%%%%%%%%%%%%%%%%%%%%%%%%%%%%%
\subsection{Related CTAN Packages}

There are several other packages which offer a similar functionality:
%
\begin{itemize}
\item
The packages
\href{http://ctan.org/pkg/docmute}{\textsf{docmute}},
\href{http://ctan.org/pkg/includex}{\textsf{includex}} and
\href{http://ctan.org/pkg/standalone}{\textsf{standalone}}
provide commands to include only the document body of
a child file thus allowing both files to be compiled individually.
\item
The packages \href{http://ctan.org/pkg/subdocs}{\textsf{subdocs}}
and \href{http://ctan.org/pkg/subfiles}{\textsf{subfiles}}
provide structures in which the main and child documents can be
encapsulated and allowing them to be compiled individually.
The inclusion mechanism is different from the conventional |\include|.
\item
The package \href{http://ctan.org/pkg/combine}{\textsf{combine}}
is an elaborate solution to combine several documents into one.
\end{itemize}
%
See also the CTAN topic \href{http://ctan.org/topic/subdocs}{\textsf{subdocs}}
for further related packages.
The present package differs from the above solutions in that
a document structure constructed with the conventional |\include| mechanism
just needs two extra commands at the top of every file
such that all constituent files can be compiled individually.

%%%%%%%%%%%%%%%%%%%%%%%%%%%%%%%%%%%%%%%%%%%%%%%%%%%%%%%%%%%%%%%%%%%%%%%%%%%%%%%%
%\subsection{Feature Suggestions}
%
%The following is a list of features which may be useful for future
%versions of this package:
%%
%\begin{itemize}
%\item
%\ldots
%\end{itemize}

%%%%%%%%%%%%%%%%%%%%%%%%%%%%%%%%%%%%%%%%%%%%%%%%%%%%%%%%%%%%%%%%%%%%%%%%%%%%%%%%
\subsection{Revision History}

%%%%%%%%%%%%%%%%%%%%%%%%%%%%%%%%%%%%%%%%
\paragraph{v2.0:} 2018/12/30

\begin{itemize}
\item
immediate forward processing
\item
added |\childdocby| mechanism
\item
manual restructured
\end{itemize}

%%%%%%%%%%%%%%%%%%%%%%%%%%%%%%%%%%%%%%%%
\paragraph{v1.6:} 2018/01/17

\begin{itemize}
\item
application for development of include files
\item
corrections to manual
\end{itemize}

%%%%%%%%%%%%%%%%%%%%%%%%%%%%%%%%%%%%%%%%
\paragraph{v1.5:} 2017/05/21

\begin{itemize}
\item
more complete structuring introduced
\item
|\childdocof| introduced
\item
|\childdoc| renamed to |\childdocmain|
\item
|\childredirect| renamed to |\childdocforward| and |\childdocforwardprefix|
and functionality expanded
\end{itemize}

%%%%%%%%%%%%%%%%%%%%%%%%%%%%%%%%%%%%%%%%
\paragraph{v1.0:} 2017/04/27

\begin{itemize}
\item
manual and install package
\item
first version published on CTAN
\end{itemize}

%%%%%%%%%%%%%%%%%%%%%%%%%%%%%%%%%%%%%%%%
\paragraph{v0.6:} 2017/04/26

\begin{itemize}
\item
redirection mechanism added
\end{itemize}

%%%%%%%%%%%%%%%%%%%%%%%%%%%%%%%%%%%%%%%%
\paragraph{v0.5:} 2017/04/26

\begin{itemize}
\item
functionality in definition file
\end{itemize}


%%%%%%%%%%%%%%%%%%%%%%%%%%%%%%%%%%%%%%%%%%%%%%%%%%%%%%%%%%%%%%%%%%%%%%%%%%%%%%%%
%%%%%%%%%%%%%%%%%%%%%%%%%%%%%%%%%%%%%%%%%%%%%%%%%%%%%%%%%%%%%%%%%%%%%%%%%%%%%%%%
%%%%%%%%%%%%%%%%%%%%%%%%%%%%%%%%%%%%%%%%%%%%%%%%%%%%%%%%%%%%%%%%%%%%%%%%%%%%%%%%
\appendix

\settowidth\MacroIndent{\rmfamily\scriptsize 000\ }

 \DocInput{childdoc.dtx}

\end{document}
%</driver>
% \fi
%
% %%%%%%%%%%%%%%%%%%%%%%%%%%%%%%%%%%%%%%%%%%%%%%%%%%%%%%%%%%%%%%%%%%%%%%%%%%%%%%
% %%%%%%%%%%%%%%%%%%%%%%%%%%%%%%%%%%%%%%%%%%%%%%%%%%%%%%%%%%%%%%%%%%%%%%%%%%%%%%
% \section{Sample}
%\iffalse
%<*samplemain>
%\fi
%
% The following presents a sample document
% with two chapters, two parts, a title page,
% a compile flag as well as three forwarding files to set the flag.
% It consists of eight |.tex| files:
% \begin{center}
% \begin{tabular}{ll}
% |cdocsamp.tex|&main file\\
% |cdocsch1.tex|&include file for chapter 1\\
% |cdocsch2.tex|&include file for chapter 2\\
% |cdocspt3.tex|&include file for part 3\\
% |cdocspt4.tex|&include file for part 4\\
% |cdocsdrf.tex|&forwarding file for main file in draft mode\\
% |cdocsfi1.tex|&forwarding file for final version of chapter 1\\
% |cdocsfi2.tex|&forwarding file for final version of chapter 2\\
% \end{tabular}
% \end{center}
% Each of the eight files can be compiled directly by the \LaTeX{} compiler.
%
% %%%%%%%%%%%%%%%%%%%%%%%%%%%%%%%%%%%%%%
% \paragraph{Main File.}
%
% The main file is called |cdocsamp.tex|.
%
% Load the \textsf{childdoc} definitions and
% declare the filename for the main document:
%    \begin{macrocode}
\input{childdoc.def}
\childdocmain{}
%    \end{macrocode}

% Optional override for |\version| flag:
%    \begin{macrocode}
%%\ifchilddoc\else\providecommand{\version}{draft}\fi
%    \end{macrocode}

% Define the default values for the |\version| flag
% (|final| for the main file and |draft| for childs):
%    \begin{macrocode}
\ifchilddoc
\providecommand{\version}{draft}
\else
\providecommand{\version}{final}
\fi
%    \end{macrocode}

% Load the standard document class:
%    \begin{macrocode}
\documentclass[12pt]{article}
%    \end{macrocode}

% Start the document body:
%    \begin{macrocode}
\begin{document}
%    \end{macrocode}

% Declare a title page.
% Print title, part of document being processed and version flag:
%    \begin{macrocode}
\addtocounter{page}{-1}
\begin{center}
{\LARGE\bfseries{}childdoc example\par}
\vspace{1cm}
\ifchilddoc
\ifchilddocmanual part\else chapter\fi:
`\childdocname' of `\childdocjob'\par
\else
main document: `\childdocjob'\par
\fi
version: \version\par
\end{center}
\newpage
%    \end{macrocode}

% Manually include selected file,
% otherwise process as usual:
%    \begin{macrocode}
\ifchilddocmanual
\section*{part `\childdocname'}
\input{\childdocname}
\else
%    \end{macrocode}

% Include the two chapters:
%    \begin{macrocode}
\include{cdocsch1}
\include{cdocsch2}
%    \end{macrocode}

% Include the two parts unless only chapters should be displayed:
%    \begin{macrocode}
\ifchilddoc\else
\section{part three}
\input{cdocspt3}
\section{part four}
\input{cdocspt4}
\fi
%    \end{macrocode}

% Process as usual until here:
%    \begin{macrocode}
\fi
%    \end{macrocode}

% End of document body:
%    \begin{macrocode}
\end{document}
%    \end{macrocode}
%\iffalse
%</samplemain>
%\fi
%
% %%%%%%%%%%%%%%%%%%%%%%%%%%%%%%%%%%%%%%
% \paragraph{Chapter Include Files.}
%
% The include files are called |cdocsch1.tex| and |cdocsch2.tex|.
%
%\iffalse
%<*samplechap1|samplechap2>
%\fi

% Optional override for |\version| flag:
%    \begin{macrocode}
%%\providecommand{\version}{final}
%    \end{macrocode}

% Include the main document:
%    \begin{macrocode}
\input{childdoc.def}
\childdocof{cdocsamp}
%    \end{macrocode}

%\iffalse
%</samplechap1|samplechap2>
%\fi
%
%\iffalse
%<*samplechap1>
%\fi
% Some text for chapter 1:
%    \begin{macrocode}
\section{one}
some text in chapter one
%    \end{macrocode}

%\iffalse
%</samplechap1>
%\fi
% Some text for chapter 2:
%\iffalse
%<*samplechap2>
%\fi
%    \begin{macrocode}
\section{two}
more text in chapter two
%    \end{macrocode}

%\iffalse
%</samplechap2>
%\fi
%
% %%%%%%%%%%%%%%%%%%%%%%%%%%%%%%%%%%%%%%
% \paragraph{Part Include Files.}
%
% The include files are called |cdocspt3.tex| and |cdocspt4.tex|.
%
%\iffalse
%<*samplepart3|samplepart4>
%\fi

% Optional override for |\version| flag:
%    \begin{macrocode}
%%\providecommand{\version}{final}
%    \end{macrocode}

% Include the main document:
%    \begin{macrocode}
\input{childdoc.def}
\childdocby{cdocsamp}
%    \end{macrocode}

%\iffalse
%</samplepart3|samplepart4>
%\fi
%
%\iffalse
%<*samplepart3>
%\fi
% Some text for part 3:
%    \begin{macrocode}
some text in part three
%    \end{macrocode}

%\iffalse
%</samplepart3>
%\fi
% Some text for part 4:
%\iffalse
%<*samplepart4>
%\fi
%    \begin{macrocode}
more text in part four
%    \end{macrocode}

%\iffalse
%</samplepart4>
%\fi
%
% %%%%%%%%%%%%%%%%%%%%%%%%%%%%%%%%%%%%%%
% \paragraph{Forwarding for a Complete Draft.}
%
% The following forwarding file |cdocsdrf.tex|
% compiles the main document in draft mode:
%\iffalse
%<*sampledraft>
%\fi
%    \begin{macrocode}
\def\version{draft}
\input{childdoc.def}
\childdocforward{cdocsamp}
%    \end{macrocode}

%\iffalse
%</sampledraft>
%\fi
%
% %%%%%%%%%%%%%%%%%%%%%%%%%%%%%%%%%%%%%%
% \paragraph{Forwarding for Final Version of the Chapters.}
%
% The following forwarding files |cdocsfn1.tex| and |cdocsfn2.tex|
% (with identical content)
% compile the final versions of the child documents
% |cdocsch1.tex| and |cdocsch2.tex|, respectively:
%\iffalse
%<*samplefinal>
%\fi
%    \begin{macrocode}
\def\version{final}
\input{childdoc.def}
\childdocforwardprefix[cdocsamp]{cdocsfn}{cdocsch}
%    \end{macrocode}

%\iffalse
%</samplefinal>
%\fi
%
% %%%%%%%%%%%%%%%%%%%%%%%%%%%%%%%%%%%%%%
% \paragraph{Command Line Processing.}
%
% The following three command lines generate the output files
% |cdocscld|, |cdocscl1| and |cdocscl2|
% which should be identical to
% |cdocsdrf|, |cdocsch1| and |cdocsfn2|, respectively:
% \begin{center}
% \begin{tabular}{l}
% |latex -jobname cdocscld \|\\
% |  "\def\version{draft}\input{childdoc.def}\childdocforward{cdocsamp}"|\\
% |latex -jobname cdocscl1 \|\\
% |  "\input{childdoc.def}\childdocforward[cdocsamp]{cdocsch1}"|\\
% |latex -jobname cdocscl2 \|\\
% |  "\def\version{final}\input{childdoc.def}\childdocforward{cdocsch2}"|
% \end{tabular}
% \end{center}
% Note that the trailing backslash on each first line
% merely continues the input to the second line
% (for convenient cut ant paste).
% Furthermore, the command |latex| can be replaced by any
% of its alternative versions such as |pdflatex|.
%
% %%%%%%%%%%%%%%%%%%%%%%%%%%%%%%%%%%%%%%%%%%%%%%%%%%%%%%%%%%%%%%%%%%%%%%%%%%%%%%
% %%%%%%%%%%%%%%%%%%%%%%%%%%%%%%%%%%%%%%%%%%%%%%%%%%%%%%%%%%%%%%%%%%%%%%%%%%%%%%
% \section{Implementation}
%\iffalse
%<*package>
%\fi
%
% This section describes the definitions file |childdoc.def|.

% The definitions cannot be loaded using |\usepackage| or |\RequirePackage|
% which has a mechanism to prevent loading a style file more than once.
% When loading the definitions by means of |\input|
% multiple instances have to be prevented manually:
%\iffalse
%This code needs to be before the `\ProvidesFile' directive
%which is defined at the beginning of this file.
%Therefore it is also placed there and commented out here.
%</package>
%<*discard>
%\fi
%    \begin{macrocode}
\ifdefined\childdocmain\endinput\fi
%    \end{macrocode}
%\iffalse
%</discard>
%<*package>
%\fi
%
% \macro{\ifchilddoc}
% \macro{\ifchilddocmanual}
% The conditional |\ifchilddoc| tells whether a
% child (true) or main (false) document is being compiled.
% The conditional |\ifchilddocmanual| tells whether
% the |\includeonly| mechanism is used (false) or
% the selection of child files must be performed manually (true).
% The definitions initialise to false:
%    \begin{macrocode}
\newif\ifchilddoc
\newif\ifchilddocmanual
%    \end{macrocode}

% \macro{\childdocname}
% \macro{\childdocjob}
% The macro |\childdocname| stores the name of the main document
% to be compiled. The macro |\childdocjob| stores the name of
% the document on which the \LaTeX{} compiler was originally invoked.
% The content of |\jobname| cannot be compared
% to filenames specified in the source due to different catcodes.
% The following code rescans |\jobname|, stores the result
% in |\childdocname| and saves a copy in |\childdocjob|:
%    \begin{macrocode}
\edef\childdocname{\scantokens\expandafter{\jobname\noexpand}}
\let\childdocjob\childdocname
%    \end{macrocode}

% \macro{\childdocdisable}
% The macro |\childdocdisable| prevents the main file
% from being processed more than once.
% At this stage, the main document command |\childdocmain|
% is assumed to be called once again where it should do nothing.
% Any subsequent call to it should prevent
% a secondary processing of the main document
% It overwrites the forwarding commands
% |\childdocof| and |\childdocforward|
% with empty macros to prevent further inclusions of the main document:
%    \begin{macrocode}
\newcommand{\childdocdisable}
{
  \renewcommand{\childdocmain}[1]{\renewcommand{\childdocmain}[1]{\endinput}}
  \renewcommand{\childdocof}[1]{}
  \renewcommand{\childdocby}[2][]{}
  \renewcommand{\childdocforward}[2][]{}
  \renewcommand{\childdocdisable}{}
}
%    \end{macrocode}

% \macro{\childdocmain}
% The macro |\childdocmain| is to be called at the top of the main file
% with nothing or the main filename (without extension) as argument.
% First, it breaks loops.
% If the argument is not empty and does not match |\childdocname|
% (which is set by the first inclusion of |childdoc.def|),
% |\ifchilddoc| is set to true, |\includeonly| is applied to the child file
% and |\jobname| is set to the main file
% (for proper handling of |.aux| files):
%    \begin{macrocode}
\newcommand{\childdocmain}[1]
{
  \childdocdisable\childdocmain{}
  \if?#1?\else
    \begingroup
      \def\childdoctmp{#1}
      \ifx\childdoctmp\childdocname
        \def\childdoctmp{}
      \else
        \def\childdoctmp
        {
          \childdoctrue
          \includeonly{\childdocname}
          \def\childdocjob{#1}
          \def\jobname{#1}
        }
      \fi
      \expandafter
    \endgroup
    \childdoctmp
  \fi
}
%    \end{macrocode}

% \macro{\childdocof}
% The command |\childdocof| redirects
% compilation to the main file |#1|.
%    \begin{macrocode}
\newcommand{\childdocof}[1]
{
  \childdocdisable
  \childdoctrue
  \includeonly{\childdocname}
  \def\jobname{#1}
  \def\childdocjob{#1}
  \input{#1}
}
%    \end{macrocode}

% \macro{\childdocby}
% The command |\childdocby| ....
%    \begin{macrocode}
\newcommand{\childdocby}[2][]
{
  \childdocdisable
  \childdoctrue
  \childdocmanualtrue
  \if?#1?\else
    \def\jobname{#2}
  \fi
  \def\childdocjob{#2}
  \input{#2}
  \endinput
}
%    \end{macrocode}

% \macro{\childdocforward}
% The command |\childdocforward| redirects
% compilation to the main file or
% (if the optional argument is given) a child file.
% Parameters are set as if the main file
% or a child file starting with |\childdocof| was compiled.
% Then compilation is handed over to the main file:
%    \begin{macrocode}
\newcommand{\childdocforward}[2][]
{
  \begingroup
    \if?#1?
      \def\childdoctmp
      {
        \def\childdocname{#2}
        \def\childdocjob{#2}
        \def\jobname{#2}
        \input{#2}
        \endinput
      }
    \else
      \def\childdoctmp
      {
        \childdocdisable
        \def\childdocname{#2}
        \childdoctrue
        \includeonly{#2}
        \def\childdocjob{#1}
        \def\jobname{#1}
        \input{#1}
        \endinput
      }
    \fi
    \expandafter
  \endgroup
  \childdoctmp
}
%    \end{macrocode}

% \macro{\childdocforwardprefix}
% The command |\childdocforwardprefix| redirects
% compilation to the main or a child file by means of a pattern.
% The prefix |#1| in the current filename is replaced by |#2|
% and the suffix of the current filename is kept
% (it is assumed that the filename does not contain the substring `|~~~|'
% which is used as a delimiter).
% Compilation is handed over to the new file by |\childdocforward|:
%    \begin{macrocode}
\newcommand{\childdocforwardprefix}[3][]
{
  \begingroup
    \def\childdocextract #2##1~~~{\def\childdoctmp{\childdocforward[#1]{#3##1}}}
    \expandafter\childdocextract\childdocname~~~
    \expandafter
  \endgroup
  \childdoctmp
}
%    \end{macrocode}

% \macro{\childdoc}
% The deprecated macro |\childdoc| is a legacy version of |\childdocmain|:
%    \begin{macrocode}
\newcommand{\childdoc}{\childdocmain}
%    \end{macrocode}

% \macro{\childdocredirect}
% The deprecated macro |\childdocredirect| is a legacy version
% of |\childdocforward| and |\childdocforwardprefix|:
%    \begin{macrocode}
\newcommand{\childdocredirect}[2][]
{
  \begingroup
    \if?#1?
      \def\childdoctmp{\childdocforward{#2}}
    \else
      \def\childdoctmp{\childdocforwardprefix{#1}{#2}}
    \fi
    \expandafter
  \endgroup
  \childdoctmp
}
%    \end{macrocode}

%\iffalse
%</package>
%\fi
%
\endinput
\childdocforward{cdocsch2}"|
% \end{tabular}
% \end{center}
% Note that the trailing backslash on each first line
% merely continues the input to the second line
% (for convenient cut ant paste).
% Furthermore, the command |latex| can be replaced by any
% of its alternative versions such as |pdflatex|.
%
% %%%%%%%%%%%%%%%%%%%%%%%%%%%%%%%%%%%%%%%%%%%%%%%%%%%%%%%%%%%%%%%%%%%%%%%%%%%%%%
% %%%%%%%%%%%%%%%%%%%%%%%%%%%%%%%%%%%%%%%%%%%%%%%%%%%%%%%%%%%%%%%%%%%%%%%%%%%%%%
% \section{Implementation}
%\iffalse
%<*package>
%\fi
%
% This section describes the definitions file |childdoc.def|.

% The definitions cannot be loaded using |\usepackage| or |\RequirePackage|
% which has a mechanism to prevent loading a style file more than once.
% When loading the definitions by means of |\input|
% multiple instances have to be prevented manually:
%\iffalse
%This code needs to be before the `\ProvidesFile' directive
%which is defined at the beginning of this file.
%Therefore it is also placed there and commented out here.
%</package>
%<*discard>
%\fi
%    \begin{macrocode}
\ifdefined\childdocmain\endinput\fi
%    \end{macrocode}
%\iffalse
%</discard>
%<*package>
%\fi
%
% \macro{\ifchilddoc}
% \macro{\ifchilddocmanual}
% The conditional |\ifchilddoc| tells whether a
% child (true) or main (false) document is being compiled.
% The conditional |\ifchilddocmanual| tells whether
% the |\includeonly| mechanism is used (false) or
% the selection of child files must be performed manually (true).
% The definitions initialise to false:
%    \begin{macrocode}
\newif\ifchilddoc
\newif\ifchilddocmanual
%    \end{macrocode}

% \macro{\childdocname}
% \macro{\childdocjob}
% The macro |\childdocname| stores the name of the main document
% to be compiled. The macro |\childdocjob| stores the name of
% the document on which the \LaTeX{} compiler was originally invoked.
% The content of |\jobname| cannot be compared
% to filenames specified in the source due to different catcodes.
% The following code rescans |\jobname|, stores the result
% in |\childdocname| and saves a copy in |\childdocjob|:
%    \begin{macrocode}
\edef\childdocname{\scantokens\expandafter{\jobname\noexpand}}
\let\childdocjob\childdocname
%    \end{macrocode}

% \macro{\childdocdisable}
% The macro |\childdocdisable| prevents the main file
% from being processed more than once.
% At this stage, the main document command |\childdocmain|
% is assumed to be called once again where it should do nothing.
% Any subsequent call to it should prevent
% a secondary processing of the main document
% It overwrites the forwarding commands
% |\childdocof| and |\childdocforward|
% with empty macros to prevent further inclusions of the main document:
%    \begin{macrocode}
\newcommand{\childdocdisable}
{
  \renewcommand{\childdocmain}[1]{\renewcommand{\childdocmain}[1]{\endinput}}
  \renewcommand{\childdocof}[1]{}
  \renewcommand{\childdocby}[2][]{}
  \renewcommand{\childdocforward}[2][]{}
  \renewcommand{\childdocdisable}{}
}
%    \end{macrocode}

% \macro{\childdocmain}
% The macro |\childdocmain| is to be called at the top of the main file
% with nothing or the main filename (without extension) as argument.
% First, it breaks loops.
% If the argument is not empty and does not match |\childdocname|
% (which is set by the first inclusion of |childdoc.def|),
% |\ifchilddoc| is set to true, |\includeonly| is applied to the child file
% and |\jobname| is set to the main file
% (for proper handling of |.aux| files):
%    \begin{macrocode}
\newcommand{\childdocmain}[1]
{
  \childdocdisable\childdocmain{}
  \if?#1?\else
    \begingroup
      \def\childdoctmp{#1}
      \ifx\childdoctmp\childdocname
        \def\childdoctmp{}
      \else
        \def\childdoctmp
        {
          \childdoctrue
          \includeonly{\childdocname}
          \def\childdocjob{#1}
          \def\jobname{#1}
        }
      \fi
      \expandafter
    \endgroup
    \childdoctmp
  \fi
}
%    \end{macrocode}

% \macro{\childdocof}
% The command |\childdocof| redirects
% compilation to the main file |#1|.
%    \begin{macrocode}
\newcommand{\childdocof}[1]
{
  \childdocdisable
  \childdoctrue
  \includeonly{\childdocname}
  \def\jobname{#1}
  \def\childdocjob{#1}
  \input{#1}
}
%    \end{macrocode}

% \macro{\childdocby}
% The command |\childdocby| ....
%    \begin{macrocode}
\newcommand{\childdocby}[2][]
{
  \childdocdisable
  \childdoctrue
  \childdocmanualtrue
  \if?#1?\else
    \def\jobname{#2}
  \fi
  \def\childdocjob{#2}
  \input{#2}
  \endinput
}
%    \end{macrocode}

% \macro{\childdocforward}
% The command |\childdocforward| redirects
% compilation to the main file or
% (if the optional argument is given) a child file.
% Parameters are set as if the main file
% or a child file starting with |\childdocof| was compiled.
% Then compilation is handed over to the main file:
%    \begin{macrocode}
\newcommand{\childdocforward}[2][]
{
  \begingroup
    \if?#1?
      \def\childdoctmp
      {
        \def\childdocname{#2}
        \def\childdocjob{#2}
        \def\jobname{#2}
        \input{#2}
        \endinput
      }
    \else
      \def\childdoctmp
      {
        \childdocdisable
        \def\childdocname{#2}
        \childdoctrue
        \includeonly{#2}
        \def\childdocjob{#1}
        \def\jobname{#1}
        \input{#1}
        \endinput
      }
    \fi
    \expandafter
  \endgroup
  \childdoctmp
}
%    \end{macrocode}

% \macro{\childdocforwardprefix}
% The command |\childdocforwardprefix| redirects
% compilation to the main or a child file by means of a pattern.
% The prefix |#1| in the current filename is replaced by |#2|
% and the suffix of the current filename is kept
% (it is assumed that the filename does not contain the substring `|~~~|'
% which is used as a delimiter).
% Compilation is handed over to the new file by |\childdocforward|:
%    \begin{macrocode}
\newcommand{\childdocforwardprefix}[3][]
{
  \begingroup
    \def\childdocextract #2##1~~~{\def\childdoctmp{\childdocforward[#1]{#3##1}}}
    \expandafter\childdocextract\childdocname~~~
    \expandafter
  \endgroup
  \childdoctmp
}
%    \end{macrocode}

% \macro{\childdoc}
% The deprecated macro |\childdoc| is a legacy version of |\childdocmain|:
%    \begin{macrocode}
\newcommand{\childdoc}{\childdocmain}
%    \end{macrocode}

% \macro{\childdocredirect}
% The deprecated macro |\childdocredirect| is a legacy version
% of |\childdocforward| and |\childdocforwardprefix|:
%    \begin{macrocode}
\newcommand{\childdocredirect}[2][]
{
  \begingroup
    \if?#1?
      \def\childdoctmp{\childdocforward{#2}}
    \else
      \def\childdoctmp{\childdocforwardprefix{#1}{#2}}
    \fi
    \expandafter
  \endgroup
  \childdoctmp
}
%    \end{macrocode}

%\iffalse
%</package>
%\fi
%
\endinput
|
and perform the replacements as outlined below.
Instead of |\childdocmain{|\textit{main}|}| add the following code
to the top of the main file:
%
\begin{center}
\begin{tabular}{l}
|\||ifdefined\childdocname\endinput\||fi\newif\ifchilddoc|\\
|\edef\childdocname{\scantokens\expandafter{\jobname\noexpand}}|\\
|\def\childdocmain{|\textit{main}|}\||ifx\childdocmain\childdocname\||else|\\
|\childdoctrue\includeonly{\childdocname}\let\jobname\childdocmain\||fi|\\
\end{tabular}
\end{center}
%
Instead of |\childdocof{|\textit{main}|}| just include the main file
at the top of each child file:
%
\begin{center}
|\input{|\textit{main}|}|
\end{center}
%
A simple redirection |\childdocforward{|\textit{dest}|}| is achieved by:
%
\begin{center}
|\def\jobname{|\textit{dest}|}\input{\jobname}|
\end{center}
%
The redirection with prefix
|\childdocforwardprefix[|\textit{prefix}|]{|\textit{dest}|}|
is accomplished by:
%
\begin{center}
\begin{tabular}{l}
|{\edef\jobname{\scantokens\expandafter{\jobname\noexpand}}|\\
|\def\redirectjob |\textit{prefix}|#1~~~{\gdef\jobname{|\textit{dest}|#1}}|\\
|\expandafter\redirectjob\jobname~~~}\input{\jobname}|
\end{tabular}
\end{center}

In an alternative approach,
child documents can be compiled by a specific command line
without additional code or specific definitions:
%
\begin{center}
|... -jobname "|\textit{target}|" "|[\textit{flags}]%
|\includeonly{|\textit{dest}|}\input{|\textit{main}|}"|
\end{center}
%

%%%%%%%%%%%%%%%%%%%%%%%%%%%%%%%%%%%%%%%%%%%%%%%%%%%%%%%%%%%%%%%%%%%%%%%%%%%%%%%%
%%%%%%%%%%%%%%%%%%%%%%%%%%%%%%%%%%%%%%%%%%%%%%%%%%%%%%%%%%%%%%%%%%%%%%%%%%%%%%%%
\section{Information}

%%%%%%%%%%%%%%%%%%%%%%%%%%%%%%%%%%%%%%%%%%%%%%%%%%%%%%%%%%%%%%%%%%%%%%%%%%%%%%%%
\subsection{Copyright}

Copyright \copyright{} 2017--2018 Niklas Beisert

This work may be distributed and/or modified under the
conditions of the \LaTeX{} Project Public License, either version 1.3
of this license or (at your option) any later version.
The latest version of this license is in
  \url{http://www.latex-project.org/lppl.txt}
and version 1.3 or later is part of all distributions of \LaTeX{}
version 2005/12/01 or later.

This work has the LPPL maintenance status `maintained'.

The Current Maintainer of this work is Niklas Beisert.

This work consists of the files |README.txt|, |childdoc.ins| and |childdoc.dtx|
as well as the derived files |childdoc.def|, |cdocsamp.tex|
with |cdocsch1.tex|, |cdocsch2.tex|, |cdocspt3.tex|, |cdocspt4.tex|,
|cdocsdrf.tex|, |cdocsfn1.tex|, |cdocsfn2.tex|
as well as |childdoc.pdf|.

%%%%%%%%%%%%%%%%%%%%%%%%%%%%%%%%%%%%%%%%%%%%%%%%%%%%%%%%%%%%%%%%%%%%%%%%%%%%%%%%
\subsection{Files and Installation}

The package consists of the files:
%
\begin{center}
\begin{tabular}{ll}
    |README.txt|   & readme file \\
    |childdoc.ins| & installation file \\
    |childdoc.dtx| & source file \\
    |childdoc.def| & definition file \\
    |cdocsamp.tex| & sample main file \\
    |cdocsch1.tex| & sample include file \\
    |cdocsch2.tex| & sample include file \\
    |cdocspt3.tex| & sample part file \\
    |cdocspt4.tex| & sample part file \\
    |cdocsdrf.tex| & sample redirection file \\
    |cdocsfn1.tex| & sample redirection file \\
    |cdocsfn2.tex| & sample redirection file \\
    |childdoc.pdf| & manual
\end{tabular}
\end{center}
%
The distribution consists of the files
|README.txt|, |childdoc.ins| and |childdoc.dtx|.
%
\begin{itemize}
\item
Run (pdf)\LaTeX{} on |childdoc.dtx|
to compile the manual |childdoc.pdf| (this file).
\item
Run \LaTeX{} on |childdoc.ins| to create the definitions file |childdoc.def|
and the sample |cdocsamp.tex| with include files
|cdocsch1.tex|, |cdocsch2.tex|, |cdocspt3.tex|, |cdocspt4.tex|,
|cdocsdrf.tex|, |cdocsfn1.tex|, |cdocsfn2.tex|.
Then copy the file |childdoc.def| to an appropriate directory of your \LaTeX{}
distribution, e.g.\ \textit{texmf-root}|/tex/latex/childdoc|.
\end{itemize}

%%%%%%%%%%%%%%%%%%%%%%%%%%%%%%%%%%%%%%%%%%%%%%%%%%%%%%%%%%%%%%%%%%%%%%%%%%%%%%%%
\subsection{Related CTAN Packages}

There are several other packages which offer a similar functionality:
%
\begin{itemize}
\item
The packages
\href{http://ctan.org/pkg/docmute}{\textsf{docmute}},
\href{http://ctan.org/pkg/includex}{\textsf{includex}} and
\href{http://ctan.org/pkg/standalone}{\textsf{standalone}}
provide commands to include only the document body of
a child file thus allowing both files to be compiled individually.
\item
The packages \href{http://ctan.org/pkg/subdocs}{\textsf{subdocs}}
and \href{http://ctan.org/pkg/subfiles}{\textsf{subfiles}}
provide structures in which the main and child documents can be
encapsulated and allowing them to be compiled individually.
The inclusion mechanism is different from the conventional |\include|.
\item
The package \href{http://ctan.org/pkg/combine}{\textsf{combine}}
is an elaborate solution to combine several documents into one.
\end{itemize}
%
See also the CTAN topic \href{http://ctan.org/topic/subdocs}{\textsf{subdocs}}
for further related packages.
The present package differs from the above solutions in that
a document structure constructed with the conventional |\include| mechanism
just needs two extra commands at the top of every file
such that all constituent files can be compiled individually.

%%%%%%%%%%%%%%%%%%%%%%%%%%%%%%%%%%%%%%%%%%%%%%%%%%%%%%%%%%%%%%%%%%%%%%%%%%%%%%%%
%\subsection{Feature Suggestions}
%
%The following is a list of features which may be useful for future
%versions of this package:
%%
%\begin{itemize}
%\item
%\ldots
%\end{itemize}

%%%%%%%%%%%%%%%%%%%%%%%%%%%%%%%%%%%%%%%%%%%%%%%%%%%%%%%%%%%%%%%%%%%%%%%%%%%%%%%%
\subsection{Revision History}

%%%%%%%%%%%%%%%%%%%%%%%%%%%%%%%%%%%%%%%%
\paragraph{v2.0:} 2018/12/30

\begin{itemize}
\item
immediate forward processing
\item
added |\childdocby| mechanism
\item
manual restructured
\end{itemize}

%%%%%%%%%%%%%%%%%%%%%%%%%%%%%%%%%%%%%%%%
\paragraph{v1.6:} 2018/01/17

\begin{itemize}
\item
application for development of include files
\item
corrections to manual
\end{itemize}

%%%%%%%%%%%%%%%%%%%%%%%%%%%%%%%%%%%%%%%%
\paragraph{v1.5:} 2017/05/21

\begin{itemize}
\item
more complete structuring introduced
\item
|\childdocof| introduced
\item
|\childdoc| renamed to |\childdocmain|
\item
|\childredirect| renamed to |\childdocforward| and |\childdocforwardprefix|
and functionality expanded
\end{itemize}

%%%%%%%%%%%%%%%%%%%%%%%%%%%%%%%%%%%%%%%%
\paragraph{v1.0:} 2017/04/27

\begin{itemize}
\item
manual and install package
\item
first version published on CTAN
\end{itemize}

%%%%%%%%%%%%%%%%%%%%%%%%%%%%%%%%%%%%%%%%
\paragraph{v0.6:} 2017/04/26

\begin{itemize}
\item
redirection mechanism added
\end{itemize}

%%%%%%%%%%%%%%%%%%%%%%%%%%%%%%%%%%%%%%%%
\paragraph{v0.5:} 2017/04/26

\begin{itemize}
\item
functionality in definition file
\end{itemize}


%%%%%%%%%%%%%%%%%%%%%%%%%%%%%%%%%%%%%%%%%%%%%%%%%%%%%%%%%%%%%%%%%%%%%%%%%%%%%%%%
%%%%%%%%%%%%%%%%%%%%%%%%%%%%%%%%%%%%%%%%%%%%%%%%%%%%%%%%%%%%%%%%%%%%%%%%%%%%%%%%
%%%%%%%%%%%%%%%%%%%%%%%%%%%%%%%%%%%%%%%%%%%%%%%%%%%%%%%%%%%%%%%%%%%%%%%%%%%%%%%%
\appendix

\settowidth\MacroIndent{\rmfamily\scriptsize 000\ }

 \DocInput{childdoc.dtx}

\end{document}
%</driver>
% \fi
%
% %%%%%%%%%%%%%%%%%%%%%%%%%%%%%%%%%%%%%%%%%%%%%%%%%%%%%%%%%%%%%%%%%%%%%%%%%%%%%%
% %%%%%%%%%%%%%%%%%%%%%%%%%%%%%%%%%%%%%%%%%%%%%%%%%%%%%%%%%%%%%%%%%%%%%%%%%%%%%%
% \section{Sample}
%\iffalse
%<*samplemain>
%\fi
%
% The following presents a sample document
% with two chapters, two parts, a title page,
% a compile flag as well as three forwarding files to set the flag.
% It consists of eight |.tex| files:
% \begin{center}
% \begin{tabular}{ll}
% |cdocsamp.tex|&main file\\
% |cdocsch1.tex|&include file for chapter 1\\
% |cdocsch2.tex|&include file for chapter 2\\
% |cdocspt3.tex|&include file for part 3\\
% |cdocspt4.tex|&include file for part 4\\
% |cdocsdrf.tex|&forwarding file for main file in draft mode\\
% |cdocsfi1.tex|&forwarding file for final version of chapter 1\\
% |cdocsfi2.tex|&forwarding file for final version of chapter 2\\
% \end{tabular}
% \end{center}
% Each of the eight files can be compiled directly by the \LaTeX{} compiler.
%
% %%%%%%%%%%%%%%%%%%%%%%%%%%%%%%%%%%%%%%
% \paragraph{Main File.}
%
% The main file is called |cdocsamp.tex|.
%
% Load the \textsf{childdoc} definitions and
% declare the filename for the main document:
%    \begin{macrocode}
% \iffalse
%
% childdoc.dtx Copyright (C) 2017-2018 Niklas Beisert
%
% This work may be distributed and/or modified under the
% conditions of the LaTeX Project Public License, either version 1.3
% of this license or (at your option) any later version.
% The latest version of this license is in
%   http://www.latex-project.org/lppl.txt
% and version 1.3 or later is part of all distributions of LaTeX
% version 2005/12/01 or later.
%
% This work has the LPPL maintenance status `maintained'.
%
% The Current Maintainer of this work is Niklas Beisert.
%
% This work consists of the files childdoc.dtx and childdoc.ins
% and the derived files childdoc.def and cdocsamp.tex with
% cdocsch1.tex, cdocsch2.tex, cdocsdrf.tex, cdocsfn1.tex, cdocsfn2.tex.
%
%<package>\ifdefined\childdocmain\endinput\fi
%<package>\ProvidesFile{childdoc.def}[2018/12/30 v2.0 child document driver]
%<samplemain>\ProvidesFile{cdocsamp.tex}[2018/12/30 v2.0 sample for childdoc]
%<*driver>
%\ProvidesFile{childdoc.drv}[2018/12/30 v2.0 childdoc reference manual file]
\PassOptionsToClass{10pt,a4paper}{article}
\documentclass{ltxdoc}

\usepackage[margin=35mm]{geometry}
\usepackage{hyperref}
\usepackage{hyperxmp}
\usepackage[usenames]{color}

\hypersetup{colorlinks=true}
\hypersetup{pdfstartview=FitH}
\hypersetup{pdfpagemode=UseNone}
\hypersetup{pdfsource={}}
\hypersetup{pdflang={en-UK}}
\hypersetup{pdfcopyright={Copyright 2017-2018 Niklas Beisert.
  This work may be distributed and/or modified under the
  conditions of the LaTeX Project Public License, either version 1.3
  of this license or (at your option) any later version.}}
\hypersetup{pdflicenseurl={http://www.latex-project.org/lppl.txt}}
\hypersetup{pdfcontactaddress={ETH Zurich, ITP, HIT K,
  Wolfgang-Pauli-Strasse 27}}
\hypersetup{pdfcontactpostcode={8093}}
\hypersetup{pdfcontactcity={Zurich}}
\hypersetup{pdfcontactcountry={Switzerland}}
\hypersetup{pdfcontactemail={nbeisert@itp.phys.ethz.ch}}
\hypersetup{pdfcontacturl={http://people.phys.ethz.ch/\xmptilde nbeisert/}}

\newcommand{\secref}[1]{\hyperref[#1]{section \ref*{#1}}}

\parskip1ex
\parindent0pt
\let\olditemize\itemize
\def\itemize{\olditemize\parskip0pt}

\begin{document}

\title{The \textsf{childdoc} Package}
\hypersetup{pdftitle={The childdoc Package}}
\author{Niklas Beisert\\[2ex]
  Institut f\"ur Theoretische Physik\\
  Eidgen\"ossische Technische Hochschule Z\"urich\\
  Wolfgang-Pauli-Strasse 27, 8093 Z\"urich, Switzerland\\[1ex]
  \href{mailto:nbeisert@itp.phys.ethz.ch}
  {\texttt{nbeisert@itp.phys.ethz.ch}}}
\hypersetup{pdfauthor={Niklas Beisert}}
\hypersetup{pdfsubject={Manual for the LaTeX2e Package childdoc}}
\date{30 December 2018, \textsf{v2.0}}
\maketitle

\begin{abstract}\noindent
\textsf{childdoc} is a \LaTeXe{} package
that enables the direct compilation
of document sections included by |\include|
to individual files.
\end{abstract}

\begingroup
\parskip0ex
\tableofcontents
\endgroup

%%%%%%%%%%%%%%%%%%%%%%%%%%%%%%%%%%%%%%%%%%%%%%%%%%%%%%%%%%%%%%%%%%%%%%%%%%%%%%%%
%%%%%%%%%%%%%%%%%%%%%%%%%%%%%%%%%%%%%%%%%%%%%%%%%%%%%%%%%%%%%%%%%%%%%%%%%%%%%%%%
\section{Introduction}

\LaTeX{} provides a mechanism to structure a large document (such as a book)
into a main file and several child files (containing the chapters)
using the |\include| command.
This mechanism is beneficial for documents
which span hundreds of pages in order to
make the source file(s) more manageable.
Moreover, compilation can be restricted to
selected child files by means of the |\includeonly| command.
The latter feature can be used to reduce the compilation time while editing
(this was significantly more useful in the earlier days of \LaTeX{})
or to generate a smaller document which is easier to navigate.
Another application of |\includeonly| is to generate
documents consisting of selected parts of the complete document.

However, there are a few drawbacks of the plain |\include| mechanism:
\begin{itemize}
\item
The child files cannot be compiled on their own,
they can only be compiled via the main file.
A naive editing environment
(such as a text editor with an option
to have the current file processed by \LaTeX)
may require one to switch to the main file before compiling;
attempting to compile the child file produces errors.
\item
The main file must be modified (each time)
to adjust the |\includeonly| command
to the present needs. This easily leaves the main file in a messy state.
\item
The generated document will always carry the filename
of the main document. This is inconvenient if
several child files are to be compiled and
to be kept for distribution.
\end{itemize}

The present package provides a simple interface
to make child files individually compilable by \LaTeX{}.
Compiling a child file then has the same effect as compiling
the main file with an |\includeonly| command
to select the appropriate child.
Moreover the generated document will carry the name of the child
rather than the main file.
This resolves all three above issues.

This feature is meant to make the editing of books,
thesis documents and lecture notes somewhat more convenient.
However, the package can also be used efficiently for
composing a series of documents (such as exercise sheets)
which are typically distributed individually.
It then assists the author in generating the individual documents
(potentially in different versions)
as well as a document containing the collected series.
Another application is in developing style files
or other kinds of included material
where compilation of the style file could redirect
to a sample or test file.

%%%%%%%%%%%%%%%%%%%%%%%%%%%%%%%%%%%%%%%%%%%%%%%%%%%%%%%%%%%%%%%%%%%%%%%%%%%%%%%%
%%%%%%%%%%%%%%%%%%%%%%%%%%%%%%%%%%%%%%%%%%%%%%%%%%%%%%%%%%%%%%%%%%%%%%%%%%%%%%%%
\section{Usage}

First of all, the package \textsf{childdoc} is \emph{not} a standard
\LaTeXe{} |.sty| style file! Therefore it needs to be invoked in
a non-standard way.

%%%%%%%%%%%%%%%%%%%%%%%%%%%%%%%%%%%%%%%%%%%%%%%%%%%%%%%%%%%%%%%%%%%%%%%%%%%%%%%%
\subsection{Included Files}
\label{sec:include}

%%%%%%%%%%%%%%%%%%%%%%%%%%%%%%%%%%%%%%%%
\DescribeMacro{\childdocmain}
To use the package, add the commands
\begin{center}
\begin{tabular}{l}
|% \iffalse
%
% childdoc.dtx Copyright (C) 2017-2018 Niklas Beisert
%
% This work may be distributed and/or modified under the
% conditions of the LaTeX Project Public License, either version 1.3
% of this license or (at your option) any later version.
% The latest version of this license is in
%   http://www.latex-project.org/lppl.txt
% and version 1.3 or later is part of all distributions of LaTeX
% version 2005/12/01 or later.
%
% This work has the LPPL maintenance status `maintained'.
%
% The Current Maintainer of this work is Niklas Beisert.
%
% This work consists of the files childdoc.dtx and childdoc.ins
% and the derived files childdoc.def and cdocsamp.tex with
% cdocsch1.tex, cdocsch2.tex, cdocsdrf.tex, cdocsfn1.tex, cdocsfn2.tex.
%
%<package>\ifdefined\childdocmain\endinput\fi
%<package>\ProvidesFile{childdoc.def}[2018/12/30 v2.0 child document driver]
%<samplemain>\ProvidesFile{cdocsamp.tex}[2018/12/30 v2.0 sample for childdoc]
%<*driver>
%\ProvidesFile{childdoc.drv}[2018/12/30 v2.0 childdoc reference manual file]
\PassOptionsToClass{10pt,a4paper}{article}
\documentclass{ltxdoc}

\usepackage[margin=35mm]{geometry}
\usepackage{hyperref}
\usepackage{hyperxmp}
\usepackage[usenames]{color}

\hypersetup{colorlinks=true}
\hypersetup{pdfstartview=FitH}
\hypersetup{pdfpagemode=UseNone}
\hypersetup{pdfsource={}}
\hypersetup{pdflang={en-UK}}
\hypersetup{pdfcopyright={Copyright 2017-2018 Niklas Beisert.
  This work may be distributed and/or modified under the
  conditions of the LaTeX Project Public License, either version 1.3
  of this license or (at your option) any later version.}}
\hypersetup{pdflicenseurl={http://www.latex-project.org/lppl.txt}}
\hypersetup{pdfcontactaddress={ETH Zurich, ITP, HIT K,
  Wolfgang-Pauli-Strasse 27}}
\hypersetup{pdfcontactpostcode={8093}}
\hypersetup{pdfcontactcity={Zurich}}
\hypersetup{pdfcontactcountry={Switzerland}}
\hypersetup{pdfcontactemail={nbeisert@itp.phys.ethz.ch}}
\hypersetup{pdfcontacturl={http://people.phys.ethz.ch/\xmptilde nbeisert/}}

\newcommand{\secref}[1]{\hyperref[#1]{section \ref*{#1}}}

\parskip1ex
\parindent0pt
\let\olditemize\itemize
\def\itemize{\olditemize\parskip0pt}

\begin{document}

\title{The \textsf{childdoc} Package}
\hypersetup{pdftitle={The childdoc Package}}
\author{Niklas Beisert\\[2ex]
  Institut f\"ur Theoretische Physik\\
  Eidgen\"ossische Technische Hochschule Z\"urich\\
  Wolfgang-Pauli-Strasse 27, 8093 Z\"urich, Switzerland\\[1ex]
  \href{mailto:nbeisert@itp.phys.ethz.ch}
  {\texttt{nbeisert@itp.phys.ethz.ch}}}
\hypersetup{pdfauthor={Niklas Beisert}}
\hypersetup{pdfsubject={Manual for the LaTeX2e Package childdoc}}
\date{30 December 2018, \textsf{v2.0}}
\maketitle

\begin{abstract}\noindent
\textsf{childdoc} is a \LaTeXe{} package
that enables the direct compilation
of document sections included by |\include|
to individual files.
\end{abstract}

\begingroup
\parskip0ex
\tableofcontents
\endgroup

%%%%%%%%%%%%%%%%%%%%%%%%%%%%%%%%%%%%%%%%%%%%%%%%%%%%%%%%%%%%%%%%%%%%%%%%%%%%%%%%
%%%%%%%%%%%%%%%%%%%%%%%%%%%%%%%%%%%%%%%%%%%%%%%%%%%%%%%%%%%%%%%%%%%%%%%%%%%%%%%%
\section{Introduction}

\LaTeX{} provides a mechanism to structure a large document (such as a book)
into a main file and several child files (containing the chapters)
using the |\include| command.
This mechanism is beneficial for documents
which span hundreds of pages in order to
make the source file(s) more manageable.
Moreover, compilation can be restricted to
selected child files by means of the |\includeonly| command.
The latter feature can be used to reduce the compilation time while editing
(this was significantly more useful in the earlier days of \LaTeX{})
or to generate a smaller document which is easier to navigate.
Another application of |\includeonly| is to generate
documents consisting of selected parts of the complete document.

However, there are a few drawbacks of the plain |\include| mechanism:
\begin{itemize}
\item
The child files cannot be compiled on their own,
they can only be compiled via the main file.
A naive editing environment
(such as a text editor with an option
to have the current file processed by \LaTeX)
may require one to switch to the main file before compiling;
attempting to compile the child file produces errors.
\item
The main file must be modified (each time)
to adjust the |\includeonly| command
to the present needs. This easily leaves the main file in a messy state.
\item
The generated document will always carry the filename
of the main document. This is inconvenient if
several child files are to be compiled and
to be kept for distribution.
\end{itemize}

The present package provides a simple interface
to make child files individually compilable by \LaTeX{}.
Compiling a child file then has the same effect as compiling
the main file with an |\includeonly| command
to select the appropriate child.
Moreover the generated document will carry the name of the child
rather than the main file.
This resolves all three above issues.

This feature is meant to make the editing of books,
thesis documents and lecture notes somewhat more convenient.
However, the package can also be used efficiently for
composing a series of documents (such as exercise sheets)
which are typically distributed individually.
It then assists the author in generating the individual documents
(potentially in different versions)
as well as a document containing the collected series.
Another application is in developing style files
or other kinds of included material
where compilation of the style file could redirect
to a sample or test file.

%%%%%%%%%%%%%%%%%%%%%%%%%%%%%%%%%%%%%%%%%%%%%%%%%%%%%%%%%%%%%%%%%%%%%%%%%%%%%%%%
%%%%%%%%%%%%%%%%%%%%%%%%%%%%%%%%%%%%%%%%%%%%%%%%%%%%%%%%%%%%%%%%%%%%%%%%%%%%%%%%
\section{Usage}

First of all, the package \textsf{childdoc} is \emph{not} a standard
\LaTeXe{} |.sty| style file! Therefore it needs to be invoked in
a non-standard way.

%%%%%%%%%%%%%%%%%%%%%%%%%%%%%%%%%%%%%%%%%%%%%%%%%%%%%%%%%%%%%%%%%%%%%%%%%%%%%%%%
\subsection{Included Files}
\label{sec:include}

%%%%%%%%%%%%%%%%%%%%%%%%%%%%%%%%%%%%%%%%
\DescribeMacro{\childdocmain}
To use the package, add the commands
\begin{center}
\begin{tabular}{l}
|\input{childdoc.def}|\\
|\childdocmain{}|\\
\end{tabular}
\end{center}
at the very top of the main \LaTeX{} file,
in particular \emph{before} the |\documentclass| statement!
The argument of |\childdocmain| should be left empty
(but it must be present).

%%%%%%%%%%%%%%%%%%%%%%%%%%%%%%%%%%%%%%%%
\DescribeMacro{\childdocof}
Furthermore, add the commands
\begin{center}
\begin{tabular}{l}
|\input{childdoc.def}|\\
|\childdocof{|\textit{main}|}|\\
\end{tabular}
\end{center}
at the top of every child file \textit{child}
which is included by |\include{|\textit{child}|}|
from within the main file
(or at least for those files to be compiled individually).
The argument \textit{main} must be the filename of the main file.

There are a couple of
considerations in setting up the main and child documents:

%%%%%%%%%%%%%%%%%%%%%%%%%%%%%%%%%%%%%%%%
\paragraph{Restrictions.}

Please note the following restrictions:
\begin{itemize}
\item
|\childdocmain| must be called with one argument \textit{main}
to ensure compatibility with earlier version of the package.
It must either be empty (|\childdocmain{}|)
or precisely match the filename of the main file in which it is specified.
See \secref{sec:detection} for further information.
\item
The filename \textit{main} must be specified without the |.tex| extension.
\item
The filename \textit{main} is case sensitive
(even in case-insensitive file systems)
due to internal string comparison.
\item
The argument \textit{main} should be fully expanded, it cannot be a macro.
\item
Subdirectories and special characters should be avoided in filenames.
\item
The command |\childdocmain{|\textit{main}|}| must be followed by a whitespace.
It should not be followed immediately by another command
or by a comment mark `|%|'.
This is because the \TeX{} parser reads the token immediately following
the argument of |\childdocmain| and puts it
at the beginning of every child section;
however, a white\-space is ignored.
\end{itemize}

%%%%%%%%%%%%%%%%%%%%%%%%%%%%%%%%%%%%%%%%
\paragraph{Content of Main File.}

It is advisable to place all content in the child files included by |\include|.
Any output contained in the main file will appear in all child documents
unless suppressed manually;
it cannot be suppressed automatically by the |\includeonly| directive
and thus should normally be avoided.
A method to include some content in the main file
by means of conditional processing is described in \secref{sec:conditional}.

%%%%%%%%%%%%%%%%%%%%%%%%%%%%%%%%%%%%%%%%
\paragraph{Page Numbering.}

When only a part of the document is compiled,
the appropriate numbering of pages
(as well as other status parameters)
is determined from the |.aux| files.
The latter contain information from previous passes.
However this information needs to propagate through
all intermediate child documents.
Therefore the page numbering in child documents may well
be inconsistent until the complete document is compiled at least once.

A useful (if unconventional) way to always ensure a consistent
page numbering is to restart the numbering in each child document
and denote the pages by `\textit{child}|.|\textit{page}'
where \textit{child} represents the chapter/section number of the child file.
This can be achieved by the command
|\numberwithin{page}{|\textit{child}|}|
of the \textsf{amsmath} package
where \textit{child} can be |chapter| or |section|
depending on the chosen structuring.
Alternatively, one can modify the macro |\thepage| appropriately
and reset the counter |page| at the start of each child file.

%%%%%%%%%%%%%%%%%%%%%%%%%%%%%%%%%%%%%%%%%%%%%%%%%%%%%%%%%%%%%%%%%%%%%%%%%%%%%%%%
\subsection{Conditional Processing}
\label{sec:conditional}

The package provides a mechanism to compile different versions
of a document. To customise the versions further some conditional processing
can come in handy to distinguish which version is being compiled.
The package provides two macros to describe the compilation context:

%%%%%%%%%%%%%%%%%%%%%%%%%%%%%%%%%%%%%%%%
\DescribeMacro{\ifchilddoc}
The conditional |\ifchilddoc| distinguishes between the compilation of
child documents and the main document:
%
\begin{center}
|\ifchilddoc |\textit{child-code}| |[|\||else |\textit{main-code}]| \||fi|
\end{center}

%%%%%%%%%%%%%%%%%%%%%%%%%%%%%%%%%%%%%%%%
\DescribeMacro{\childdocname}
\DescribeMacro{\childdocjob}
The macro |\childdocname| contains the filename (without extension)
of the main or child file being processed.
Note that |\childdocjob| will always contain the name of the main file.

%%%%%%%%%%%%%%%%%%%%%%%%%%%%%%%%%%%%%%%%
\paragraph{Title Page.}

Conditional processing can be used to include a title or banner page
in the main document when proper precautions are taken.
Importantly, the code in the main file should ensure that the page counter
(as well as other status parameters which are stored in the |.aux| files)
takes the same value after the conditional processing.
Otherwise the page numbers may take divergent values
depending on which part is compiled.

For example, a title page could be declared by:
%
\begin{center}
\begin{tabular}{l}
|\ifchilddoc\||else|\\
|\addtocounter{page}{-1}|\\
\textit{code for title page}\\
|\newpage|\\
|\||fi|
\end{tabular}
\end{center}
%
A banner page for the child documents can be generated by:
%
\begin{center}
\begin{tabular}{l}
|\ifchilddoc|\\
|\addtocounter{page}{-1}|\\
\textit{code for banner page}\\
|\newpage|\\
|\||fi|
\end{tabular}
\end{center}
%
Here one could write a message such as:
\begin{center}
|This is the part \childdocname{} of \childdocjob{}.|
\end{center}

%%%%%%%%%%%%%%%%%%%%%%%%%%%%%%%%%%%%%%%%%%%%%%%%%%%%%%%%%%%%%%%%%%%%%%%%%%%%%%%%
\subsection{Flags}
\label{sec:flags}

The package makes it easy to generate different versions
of the main or child documents.
To this end compilation flags can be defined
and assigned different default values.
They will be particularly useful in conjunction
with the forwarding mechanism described in \secref{sec:forward}.

For example, it may be useful to have a flag |\version|
which can be set to |draft| or |final|.
The document source will contain some conditional code
depending on the value of |\version|.
Suppose further, the flag should default to |final| for the main file
and to |draft| for child files
which is a natural assignment for editing the document.
This is achieved by placing the following code
in the preamble of the main document
(below the |\childdocmain| directive):
%
\begin{center}
\begin{tabular}{l}
|\ifchilddoc|\\
|\providecommand{\version}{draft}|\\
|\||else|\\
|\providecommand{\version}{final}|\\
|\||fi|
\end{tabular}
\end{center}
%
The definition by |\providecommand| makes sure
that previous definitions are not overwritten.
Further statements |\providecommand{\version}{...}|
can thus be added before the above code to override it.

For the main file, one might add a line
(between |\childdocmain| and the above block)
%
\begin{center}
|%\ifchilddoc\||else\providecommand{\version}{draft}\||fi|
\end{center}
%
which can be uncommented to produce a draft version.
Likewise one can add a line to the very top of a child file
(above the |\childdocof{|\textit{main}|}| directive)
%
\begin{center}
|%\providecommand{\version}{final}|
\end{center}
%
which can be uncommented to produce the final version of this child document.

%%%%%%%%%%%%%%%%%%%%%%%%%%%%%%%%%%%%%%%%%%%%%%%%%%%%%%%%%%%%%%%%%%%%%%%%%%%%%%%%
\subsection{Forwarding}
\label{sec:forward}

Different versions of the main or child documents
using compilation flags as described in \secref{sec:flags}
can be (permanently) stored in different files
for convenient compilation, viewing and distribution.
To this end, the package defines a command
to pass on compilation to a different file:

%%%%%%%%%%%%%%%%%%%%%%%%%%%%%%%%%%%%%%%%
\DescribeMacro{\childdocforward}
The command |\childdocforward| redirects processing to
another source file:
%
\begin{center}
\begin{tabular}{l}
|\input{childdoc.def}|\\
|\childdocforward[|\textit{main}|]{|\textit{dest}|}|\\
\end{tabular}
\end{center}
%
The argument \textit{dest} is the destination file
(without extension).
It should be the main file or one of the child files.
Note that further \textsf{childdoc} directives
such as |\childdocof| and |\childdocforward|
in the indicated file will be processed in this form.
The optional argument \textit{main}
passes on directly to the main file \textit{main}
while pretending to compile the child \textit{dest}.
This form behaves as if \textit{dest}
issues |\childdocof{|\textit{main}|}| right away,
and no further \textsf{childdoc} directives will be processed.

%%%%%%%%%%%%%%%%%%%%%%%%%%%%%%%%%%%%%%%%
\DescribeMacro{\...prefix}
In the alternative form |\childdocforwardprefix|,
%
\begin{center}
\begin{tabular}{l}
|\input{childdoc.def}|\\
|\childdocforwardprefix[|\textit{main}|]{|\textit{prefix}|}{|\textit{dest}|}|
\end{tabular}
\end{center}
%
the destination file is determined by a pattern
depending on the current file:
To make this work, the current file must be called
`{\textit{prefix}\hspace{0.2em}\textit{suffix}}'
with \textit{prefix} matching precisely the argument.
Processing is then passed on to the file
`{\textit{dest}\hspace{0.2em}\textit{suffix}}'.
Surely, the same effect is achieved by
directly specifying the
argument `{\textit{dest}\hspace{0.2em}\textit{suffix}}'
in the first form.
However, that requires to set up a different file
for each child. With the alternative form of the command
all these files can have exactly the same content
which simplifies setting them up and maintaining them.

For example, the following file |draft.tex|
with a compilation flag |\version| as described in \secref{sec:flags}
compiles the main document as a draft:
%
\begin{center}
\begin{tabular}{l}
|\def\version{draft}|\\
|\input{childdoc.def}|\\
|\childdocforward{|\textit{main}|}|
\end{tabular}
\end{center}
%
Likewise, the following files |final|\textit{nn}|.tex|
compile the final version of the child document
|child|\textit{nn}|.tex|:
%
\begin{center}
\begin{tabular}{l}
|\def\version{final}|\\
|\input{childdoc.def}|\\
|\childdocforwardprefix{final}{child}|
\end{tabular}
\end{center}
%

Note that when several versions of a main file and/or of each child file
are to be generated, it may be convenient to set up a |Makefile| or
shell script to automatise the process.

%%%%%%%%%%%%%%%%%%%%%%%%%%%%%%%%%%%%%%%%%%%%%%%%%%%%%%%%%%%%%%%%%%%%%%%%%%%%%%%%
\subsection{Command Line Processing}
\label{sec:commandline}

The effect of redirection files can also be achieved by invoking
the \LaTeX{} compiler with a more elaborate command line.
Most conveniently this should be done as part
of a shell script or a |Makefile|.

When using \textsf{childdoc} in the main file, the following
command lines effectively perform a redirection
(note that depending on the shell being used,
backslashes may have to be doubled: `|\|' $\to$ `|\\|'):
%
\begin{center}
|... -jobname "|\textit{target}|" |\\|"|[\textit{flags}]%
|\input{childdoc.def}\childdocforward[|\textit{main}|]{|\textit{dest}|}"|
\end{center}
%
Here \textit{target} is the name of the output file,
\textit{main} is the name of the main file
and \textit{dest} is the name of the main or child file to be processed
(all filenames without extensions).
The optional argument \textit{main} can be omitted
if \textit{main} matches \textit{dest}.
Optionally, compilation \textit{flags} can be defined via |\def| commands.
This command line makes the \TeX{} engine believe
it is compiling the file \textit{target}
whose content is specified as the latter parameter.
The provided code then forwards the processing to
\textit{main} or \textit{dest} as described in \secref{sec:forward}.

%%%%%%%%%%%%%%%%%%%%%%%%%%%%%%%%%%%%%%%%%%%%%%%%%%%%%%%%%%%%%%%%%%%%%%%%%%%%%%%%
\subsection{Include by Input}
\label{sec:input}

Including child documents by |\include| has some restrictions by design.
Most notably, the content of a child document always occupies
its own set of pages; pages cannot be shared between child documents.
Usually, this behaviour makes perfect sense
because each child document contain an essential part of the document.
However, in some situations it may be desirable to compose
a document from a collection of parts
without having mandatory page breaks between then.
For this case, the package
provides a mechanism to include parts
by |\input| which can also be processed individually.
However, by construction this mechanism
requires manual handling of the content to be output.

%%%%%%%%%%%%%%%%%%%%%%%%%%%%%%%%%%%%%%%%
\DescribeMacro{\ifchilddocmanual}
The main file should be prepared as usual, see \secref{sec:include}.
However, the document body must make a distinction
between processing of an individual part and of the main document, e.g.:
%
\begin{center}
\begin{tabular}{l}
|\ifchilddocmanual|\\
|\input{\childdocname}|\\
|\||else|\\
\textit{document body with }|\input{|\textit{part}|}|\\
|\||fi|
\end{tabular}
\end{center}
%
The conditional |\ifchilddocmanual| is true whenever
a part to be included by |\input| is being compiled,
and the name of the part is stored in |\childdocname|.

%%%%%%%%%%%%%%%%%%%%%%%%%%%%%%%%%%%%%%%%
\DescribeMacro{\childdocby}
Each part to be included by |\input| should start with:
%
\begin{center}
\begin{tabular}{l}
|\input{childdoc.def}|\\
|\childdocby{|\textit{main}|}|\\
\end{tabular}
\end{center}
%
The directive |\childdocby| is similar to |\childdocof|
described in \secref{sec:include},
but the subsequent selection of content must be done manually.
To that end, both |\ifchilddoc| and |\ifchilddocmanual|
will be true upon processing of a part,
and the name of the part is stored in |\childdocname|.
Note that |\jobname| will be set to the filename of the current part
so that each part receives an individual |.aux| file
that does not interfere with the |.aux| file(s) of the main document.
This behaviour can be altered by the alternative form
|\childdocby[*]{|\textit{main}|}| (with a non-empty optional argument)
which uses the |.aux| file of the main document
by setting |\jobname| to \textit{main}.

%%%%%%%%%%%%%%%%%%%%%%%%%%%%%%%%%%%%%%%%%%%%%%%%%%%%%%%%%%%%%%%%%%%%%%%%%%%%%%%%
\subsection{Driver Development}
\label{sec:driver}

The \textsf{childdoc} mechanism can also be use for the development
of definition files such as \LaTeX{} styles or classes.
This case differs from the above setup with multiple parts
included by |\include| in that no |\includeonly| should be invoked.
This can be achieved by starting the include file
(before |\ProvidesPackage|) with:
%
\begin{center}
\begin{tabular}{l}
|\input{childdoc.def}|\\
|\childdocforward{|\textit{main}|}|\\
\end{tabular}
\end{center}
%
or alternatively with:
%
\begin{center}
\begin{tabular}{l}
|\input{childdoc.def}|\\
|\childdocby{|\textit{main}|}|\\
\end{tabular}
\end{center}
%
Both forms have slightly different effects as described above.
The main file is prepared as usual, see \secref{sec:include}.

%%%%%%%%%%%%%%%%%%%%%%%%%%%%%%%%%%%%%%%%%%%%%%%%%%%%%%%%%%%%%%%%%%%%%%%%%%%%%%%%
\subsection{Legacy Detection}
\label{sec:detection}

The directive |\childdocmain| in the main file can detect
whether the complete document or merely a child is to be compiled
even without using the directive |\childdocof|.
This method is deprecated because it is less robust
and there is no compelling reason to use it;
it is merely provided for backward compatibility
and it may be removed in future versions.

If the detection mechanism is to be used,
it is mandatory to correctly specify
the filename of the main file as the argument of |\childdocmain|:
%
\begin{center}
\begin{tabular}{l}
|\input{childdoc.def}|\\
|\childdocmain{|\textit{main}|}|\\
\end{tabular}
\end{center}
%
If |\jobname| does not match the argument \textit{main} of |\childdocmain|,
it is assumed that |\jobname| points to the child file to be compiled.
When using |\childdocmain| with the main file specified as argument,
it suffices to start a child file
with just |\input{|\textit{main}|}|
without loading of the package and using |\childdocof|.
If instead all processing is done
with the appropriate \textsf{childdoc} directives,
the argument of \textit{main} of |\childdocmain| can be empty.

An alternative version of the command line processing described
in \secref{sec:commandline} using the detection mechanism reads:
%
\begin{center}
|... -jobname "|\textit{target}|" "|[\textit{flags}]%
[|\def\jobname{|\textit{dest}|}|]|\input{|\textit{main}|}"|
\end{center}

%%%%%%%%%%%%%%%%%%%%%%%%%%%%%%%%%%%%%%%%%%%%%%%%%%%%%%%%%%%%%%%%%%%%%%%%%%%%%%%%
\subsection{Manual Code}
\label{sec:manual}

In case one cannot be certain whether the definitions file |childdoc.def|
is installed on the target \TeX{} distribution
and one prefers not to ship it,
it is conceivable to paste a few relevant commands into the sources.

To that end, drop all statements |\input{childdoc.def}|
and perform the replacements as outlined below.
Instead of |\childdocmain{|\textit{main}|}| add the following code
to the top of the main file:
%
\begin{center}
\begin{tabular}{l}
|\||ifdefined\childdocname\endinput\||fi\newif\ifchilddoc|\\
|\edef\childdocname{\scantokens\expandafter{\jobname\noexpand}}|\\
|\def\childdocmain{|\textit{main}|}\||ifx\childdocmain\childdocname\||else|\\
|\childdoctrue\includeonly{\childdocname}\let\jobname\childdocmain\||fi|\\
\end{tabular}
\end{center}
%
Instead of |\childdocof{|\textit{main}|}| just include the main file
at the top of each child file:
%
\begin{center}
|\input{|\textit{main}|}|
\end{center}
%
A simple redirection |\childdocforward{|\textit{dest}|}| is achieved by:
%
\begin{center}
|\def\jobname{|\textit{dest}|}\input{\jobname}|
\end{center}
%
The redirection with prefix
|\childdocforwardprefix[|\textit{prefix}|]{|\textit{dest}|}|
is accomplished by:
%
\begin{center}
\begin{tabular}{l}
|{\edef\jobname{\scantokens\expandafter{\jobname\noexpand}}|\\
|\def\redirectjob |\textit{prefix}|#1~~~{\gdef\jobname{|\textit{dest}|#1}}|\\
|\expandafter\redirectjob\jobname~~~}\input{\jobname}|
\end{tabular}
\end{center}

In an alternative approach,
child documents can be compiled by a specific command line
without additional code or specific definitions:
%
\begin{center}
|... -jobname "|\textit{target}|" "|[\textit{flags}]%
|\includeonly{|\textit{dest}|}\input{|\textit{main}|}"|
\end{center}
%

%%%%%%%%%%%%%%%%%%%%%%%%%%%%%%%%%%%%%%%%%%%%%%%%%%%%%%%%%%%%%%%%%%%%%%%%%%%%%%%%
%%%%%%%%%%%%%%%%%%%%%%%%%%%%%%%%%%%%%%%%%%%%%%%%%%%%%%%%%%%%%%%%%%%%%%%%%%%%%%%%
\section{Information}

%%%%%%%%%%%%%%%%%%%%%%%%%%%%%%%%%%%%%%%%%%%%%%%%%%%%%%%%%%%%%%%%%%%%%%%%%%%%%%%%
\subsection{Copyright}

Copyright \copyright{} 2017--2018 Niklas Beisert

This work may be distributed and/or modified under the
conditions of the \LaTeX{} Project Public License, either version 1.3
of this license or (at your option) any later version.
The latest version of this license is in
  \url{http://www.latex-project.org/lppl.txt}
and version 1.3 or later is part of all distributions of \LaTeX{}
version 2005/12/01 or later.

This work has the LPPL maintenance status `maintained'.

The Current Maintainer of this work is Niklas Beisert.

This work consists of the files |README.txt|, |childdoc.ins| and |childdoc.dtx|
as well as the derived files |childdoc.def|, |cdocsamp.tex|
with |cdocsch1.tex|, |cdocsch2.tex|, |cdocspt3.tex|, |cdocspt4.tex|,
|cdocsdrf.tex|, |cdocsfn1.tex|, |cdocsfn2.tex|
as well as |childdoc.pdf|.

%%%%%%%%%%%%%%%%%%%%%%%%%%%%%%%%%%%%%%%%%%%%%%%%%%%%%%%%%%%%%%%%%%%%%%%%%%%%%%%%
\subsection{Files and Installation}

The package consists of the files:
%
\begin{center}
\begin{tabular}{ll}
    |README.txt|   & readme file \\
    |childdoc.ins| & installation file \\
    |childdoc.dtx| & source file \\
    |childdoc.def| & definition file \\
    |cdocsamp.tex| & sample main file \\
    |cdocsch1.tex| & sample include file \\
    |cdocsch2.tex| & sample include file \\
    |cdocspt3.tex| & sample part file \\
    |cdocspt4.tex| & sample part file \\
    |cdocsdrf.tex| & sample redirection file \\
    |cdocsfn1.tex| & sample redirection file \\
    |cdocsfn2.tex| & sample redirection file \\
    |childdoc.pdf| & manual
\end{tabular}
\end{center}
%
The distribution consists of the files
|README.txt|, |childdoc.ins| and |childdoc.dtx|.
%
\begin{itemize}
\item
Run (pdf)\LaTeX{} on |childdoc.dtx|
to compile the manual |childdoc.pdf| (this file).
\item
Run \LaTeX{} on |childdoc.ins| to create the definitions file |childdoc.def|
and the sample |cdocsamp.tex| with include files
|cdocsch1.tex|, |cdocsch2.tex|, |cdocspt3.tex|, |cdocspt4.tex|,
|cdocsdrf.tex|, |cdocsfn1.tex|, |cdocsfn2.tex|.
Then copy the file |childdoc.def| to an appropriate directory of your \LaTeX{}
distribution, e.g.\ \textit{texmf-root}|/tex/latex/childdoc|.
\end{itemize}

%%%%%%%%%%%%%%%%%%%%%%%%%%%%%%%%%%%%%%%%%%%%%%%%%%%%%%%%%%%%%%%%%%%%%%%%%%%%%%%%
\subsection{Related CTAN Packages}

There are several other packages which offer a similar functionality:
%
\begin{itemize}
\item
The packages
\href{http://ctan.org/pkg/docmute}{\textsf{docmute}},
\href{http://ctan.org/pkg/includex}{\textsf{includex}} and
\href{http://ctan.org/pkg/standalone}{\textsf{standalone}}
provide commands to include only the document body of
a child file thus allowing both files to be compiled individually.
\item
The packages \href{http://ctan.org/pkg/subdocs}{\textsf{subdocs}}
and \href{http://ctan.org/pkg/subfiles}{\textsf{subfiles}}
provide structures in which the main and child documents can be
encapsulated and allowing them to be compiled individually.
The inclusion mechanism is different from the conventional |\include|.
\item
The package \href{http://ctan.org/pkg/combine}{\textsf{combine}}
is an elaborate solution to combine several documents into one.
\end{itemize}
%
See also the CTAN topic \href{http://ctan.org/topic/subdocs}{\textsf{subdocs}}
for further related packages.
The present package differs from the above solutions in that
a document structure constructed with the conventional |\include| mechanism
just needs two extra commands at the top of every file
such that all constituent files can be compiled individually.

%%%%%%%%%%%%%%%%%%%%%%%%%%%%%%%%%%%%%%%%%%%%%%%%%%%%%%%%%%%%%%%%%%%%%%%%%%%%%%%%
%\subsection{Feature Suggestions}
%
%The following is a list of features which may be useful for future
%versions of this package:
%%
%\begin{itemize}
%\item
%\ldots
%\end{itemize}

%%%%%%%%%%%%%%%%%%%%%%%%%%%%%%%%%%%%%%%%%%%%%%%%%%%%%%%%%%%%%%%%%%%%%%%%%%%%%%%%
\subsection{Revision History}

%%%%%%%%%%%%%%%%%%%%%%%%%%%%%%%%%%%%%%%%
\paragraph{v2.0:} 2018/12/30

\begin{itemize}
\item
immediate forward processing
\item
added |\childdocby| mechanism
\item
manual restructured
\end{itemize}

%%%%%%%%%%%%%%%%%%%%%%%%%%%%%%%%%%%%%%%%
\paragraph{v1.6:} 2018/01/17

\begin{itemize}
\item
application for development of include files
\item
corrections to manual
\end{itemize}

%%%%%%%%%%%%%%%%%%%%%%%%%%%%%%%%%%%%%%%%
\paragraph{v1.5:} 2017/05/21

\begin{itemize}
\item
more complete structuring introduced
\item
|\childdocof| introduced
\item
|\childdoc| renamed to |\childdocmain|
\item
|\childredirect| renamed to |\childdocforward| and |\childdocforwardprefix|
and functionality expanded
\end{itemize}

%%%%%%%%%%%%%%%%%%%%%%%%%%%%%%%%%%%%%%%%
\paragraph{v1.0:} 2017/04/27

\begin{itemize}
\item
manual and install package
\item
first version published on CTAN
\end{itemize}

%%%%%%%%%%%%%%%%%%%%%%%%%%%%%%%%%%%%%%%%
\paragraph{v0.6:} 2017/04/26

\begin{itemize}
\item
redirection mechanism added
\end{itemize}

%%%%%%%%%%%%%%%%%%%%%%%%%%%%%%%%%%%%%%%%
\paragraph{v0.5:} 2017/04/26

\begin{itemize}
\item
functionality in definition file
\end{itemize}


%%%%%%%%%%%%%%%%%%%%%%%%%%%%%%%%%%%%%%%%%%%%%%%%%%%%%%%%%%%%%%%%%%%%%%%%%%%%%%%%
%%%%%%%%%%%%%%%%%%%%%%%%%%%%%%%%%%%%%%%%%%%%%%%%%%%%%%%%%%%%%%%%%%%%%%%%%%%%%%%%
%%%%%%%%%%%%%%%%%%%%%%%%%%%%%%%%%%%%%%%%%%%%%%%%%%%%%%%%%%%%%%%%%%%%%%%%%%%%%%%%
\appendix

\settowidth\MacroIndent{\rmfamily\scriptsize 000\ }

 \DocInput{childdoc.dtx}

\end{document}
%</driver>
% \fi
%
% %%%%%%%%%%%%%%%%%%%%%%%%%%%%%%%%%%%%%%%%%%%%%%%%%%%%%%%%%%%%%%%%%%%%%%%%%%%%%%
% %%%%%%%%%%%%%%%%%%%%%%%%%%%%%%%%%%%%%%%%%%%%%%%%%%%%%%%%%%%%%%%%%%%%%%%%%%%%%%
% \section{Sample}
%\iffalse
%<*samplemain>
%\fi
%
% The following presents a sample document
% with two chapters, two parts, a title page,
% a compile flag as well as three forwarding files to set the flag.
% It consists of eight |.tex| files:
% \begin{center}
% \begin{tabular}{ll}
% |cdocsamp.tex|&main file\\
% |cdocsch1.tex|&include file for chapter 1\\
% |cdocsch2.tex|&include file for chapter 2\\
% |cdocspt3.tex|&include file for part 3\\
% |cdocspt4.tex|&include file for part 4\\
% |cdocsdrf.tex|&forwarding file for main file in draft mode\\
% |cdocsfi1.tex|&forwarding file for final version of chapter 1\\
% |cdocsfi2.tex|&forwarding file for final version of chapter 2\\
% \end{tabular}
% \end{center}
% Each of the eight files can be compiled directly by the \LaTeX{} compiler.
%
% %%%%%%%%%%%%%%%%%%%%%%%%%%%%%%%%%%%%%%
% \paragraph{Main File.}
%
% The main file is called |cdocsamp.tex|.
%
% Load the \textsf{childdoc} definitions and
% declare the filename for the main document:
%    \begin{macrocode}
\input{childdoc.def}
\childdocmain{}
%    \end{macrocode}

% Optional override for |\version| flag:
%    \begin{macrocode}
%%\ifchilddoc\else\providecommand{\version}{draft}\fi
%    \end{macrocode}

% Define the default values for the |\version| flag
% (|final| for the main file and |draft| for childs):
%    \begin{macrocode}
\ifchilddoc
\providecommand{\version}{draft}
\else
\providecommand{\version}{final}
\fi
%    \end{macrocode}

% Load the standard document class:
%    \begin{macrocode}
\documentclass[12pt]{article}
%    \end{macrocode}

% Start the document body:
%    \begin{macrocode}
\begin{document}
%    \end{macrocode}

% Declare a title page.
% Print title, part of document being processed and version flag:
%    \begin{macrocode}
\addtocounter{page}{-1}
\begin{center}
{\LARGE\bfseries{}childdoc example\par}
\vspace{1cm}
\ifchilddoc
\ifchilddocmanual part\else chapter\fi:
`\childdocname' of `\childdocjob'\par
\else
main document: `\childdocjob'\par
\fi
version: \version\par
\end{center}
\newpage
%    \end{macrocode}

% Manually include selected file,
% otherwise process as usual:
%    \begin{macrocode}
\ifchilddocmanual
\section*{part `\childdocname'}
\input{\childdocname}
\else
%    \end{macrocode}

% Include the two chapters:
%    \begin{macrocode}
\include{cdocsch1}
\include{cdocsch2}
%    \end{macrocode}

% Include the two parts unless only chapters should be displayed:
%    \begin{macrocode}
\ifchilddoc\else
\section{part three}
\input{cdocspt3}
\section{part four}
\input{cdocspt4}
\fi
%    \end{macrocode}

% Process as usual until here:
%    \begin{macrocode}
\fi
%    \end{macrocode}

% End of document body:
%    \begin{macrocode}
\end{document}
%    \end{macrocode}
%\iffalse
%</samplemain>
%\fi
%
% %%%%%%%%%%%%%%%%%%%%%%%%%%%%%%%%%%%%%%
% \paragraph{Chapter Include Files.}
%
% The include files are called |cdocsch1.tex| and |cdocsch2.tex|.
%
%\iffalse
%<*samplechap1|samplechap2>
%\fi

% Optional override for |\version| flag:
%    \begin{macrocode}
%%\providecommand{\version}{final}
%    \end{macrocode}

% Include the main document:
%    \begin{macrocode}
\input{childdoc.def}
\childdocof{cdocsamp}
%    \end{macrocode}

%\iffalse
%</samplechap1|samplechap2>
%\fi
%
%\iffalse
%<*samplechap1>
%\fi
% Some text for chapter 1:
%    \begin{macrocode}
\section{one}
some text in chapter one
%    \end{macrocode}

%\iffalse
%</samplechap1>
%\fi
% Some text for chapter 2:
%\iffalse
%<*samplechap2>
%\fi
%    \begin{macrocode}
\section{two}
more text in chapter two
%    \end{macrocode}

%\iffalse
%</samplechap2>
%\fi
%
% %%%%%%%%%%%%%%%%%%%%%%%%%%%%%%%%%%%%%%
% \paragraph{Part Include Files.}
%
% The include files are called |cdocspt3.tex| and |cdocspt4.tex|.
%
%\iffalse
%<*samplepart3|samplepart4>
%\fi

% Optional override for |\version| flag:
%    \begin{macrocode}
%%\providecommand{\version}{final}
%    \end{macrocode}

% Include the main document:
%    \begin{macrocode}
\input{childdoc.def}
\childdocby{cdocsamp}
%    \end{macrocode}

%\iffalse
%</samplepart3|samplepart4>
%\fi
%
%\iffalse
%<*samplepart3>
%\fi
% Some text for part 3:
%    \begin{macrocode}
some text in part three
%    \end{macrocode}

%\iffalse
%</samplepart3>
%\fi
% Some text for part 4:
%\iffalse
%<*samplepart4>
%\fi
%    \begin{macrocode}
more text in part four
%    \end{macrocode}

%\iffalse
%</samplepart4>
%\fi
%
% %%%%%%%%%%%%%%%%%%%%%%%%%%%%%%%%%%%%%%
% \paragraph{Forwarding for a Complete Draft.}
%
% The following forwarding file |cdocsdrf.tex|
% compiles the main document in draft mode:
%\iffalse
%<*sampledraft>
%\fi
%    \begin{macrocode}
\def\version{draft}
\input{childdoc.def}
\childdocforward{cdocsamp}
%    \end{macrocode}

%\iffalse
%</sampledraft>
%\fi
%
% %%%%%%%%%%%%%%%%%%%%%%%%%%%%%%%%%%%%%%
% \paragraph{Forwarding for Final Version of the Chapters.}
%
% The following forwarding files |cdocsfn1.tex| and |cdocsfn2.tex|
% (with identical content)
% compile the final versions of the child documents
% |cdocsch1.tex| and |cdocsch2.tex|, respectively:
%\iffalse
%<*samplefinal>
%\fi
%    \begin{macrocode}
\def\version{final}
\input{childdoc.def}
\childdocforwardprefix[cdocsamp]{cdocsfn}{cdocsch}
%    \end{macrocode}

%\iffalse
%</samplefinal>
%\fi
%
% %%%%%%%%%%%%%%%%%%%%%%%%%%%%%%%%%%%%%%
% \paragraph{Command Line Processing.}
%
% The following three command lines generate the output files
% |cdocscld|, |cdocscl1| and |cdocscl2|
% which should be identical to
% |cdocsdrf|, |cdocsch1| and |cdocsfn2|, respectively:
% \begin{center}
% \begin{tabular}{l}
% |latex -jobname cdocscld \|\\
% |  "\def\version{draft}\input{childdoc.def}\childdocforward{cdocsamp}"|\\
% |latex -jobname cdocscl1 \|\\
% |  "\input{childdoc.def}\childdocforward[cdocsamp]{cdocsch1}"|\\
% |latex -jobname cdocscl2 \|\\
% |  "\def\version{final}\input{childdoc.def}\childdocforward{cdocsch2}"|
% \end{tabular}
% \end{center}
% Note that the trailing backslash on each first line
% merely continues the input to the second line
% (for convenient cut ant paste).
% Furthermore, the command |latex| can be replaced by any
% of its alternative versions such as |pdflatex|.
%
% %%%%%%%%%%%%%%%%%%%%%%%%%%%%%%%%%%%%%%%%%%%%%%%%%%%%%%%%%%%%%%%%%%%%%%%%%%%%%%
% %%%%%%%%%%%%%%%%%%%%%%%%%%%%%%%%%%%%%%%%%%%%%%%%%%%%%%%%%%%%%%%%%%%%%%%%%%%%%%
% \section{Implementation}
%\iffalse
%<*package>
%\fi
%
% This section describes the definitions file |childdoc.def|.

% The definitions cannot be loaded using |\usepackage| or |\RequirePackage|
% which has a mechanism to prevent loading a style file more than once.
% When loading the definitions by means of |\input|
% multiple instances have to be prevented manually:
%\iffalse
%This code needs to be before the `\ProvidesFile' directive
%which is defined at the beginning of this file.
%Therefore it is also placed there and commented out here.
%</package>
%<*discard>
%\fi
%    \begin{macrocode}
\ifdefined\childdocmain\endinput\fi
%    \end{macrocode}
%\iffalse
%</discard>
%<*package>
%\fi
%
% \macro{\ifchilddoc}
% \macro{\ifchilddocmanual}
% The conditional |\ifchilddoc| tells whether a
% child (true) or main (false) document is being compiled.
% The conditional |\ifchilddocmanual| tells whether
% the |\includeonly| mechanism is used (false) or
% the selection of child files must be performed manually (true).
% The definitions initialise to false:
%    \begin{macrocode}
\newif\ifchilddoc
\newif\ifchilddocmanual
%    \end{macrocode}

% \macro{\childdocname}
% \macro{\childdocjob}
% The macro |\childdocname| stores the name of the main document
% to be compiled. The macro |\childdocjob| stores the name of
% the document on which the \LaTeX{} compiler was originally invoked.
% The content of |\jobname| cannot be compared
% to filenames specified in the source due to different catcodes.
% The following code rescans |\jobname|, stores the result
% in |\childdocname| and saves a copy in |\childdocjob|:
%    \begin{macrocode}
\edef\childdocname{\scantokens\expandafter{\jobname\noexpand}}
\let\childdocjob\childdocname
%    \end{macrocode}

% \macro{\childdocdisable}
% The macro |\childdocdisable| prevents the main file
% from being processed more than once.
% At this stage, the main document command |\childdocmain|
% is assumed to be called once again where it should do nothing.
% Any subsequent call to it should prevent
% a secondary processing of the main document
% It overwrites the forwarding commands
% |\childdocof| and |\childdocforward|
% with empty macros to prevent further inclusions of the main document:
%    \begin{macrocode}
\newcommand{\childdocdisable}
{
  \renewcommand{\childdocmain}[1]{\renewcommand{\childdocmain}[1]{\endinput}}
  \renewcommand{\childdocof}[1]{}
  \renewcommand{\childdocby}[2][]{}
  \renewcommand{\childdocforward}[2][]{}
  \renewcommand{\childdocdisable}{}
}
%    \end{macrocode}

% \macro{\childdocmain}
% The macro |\childdocmain| is to be called at the top of the main file
% with nothing or the main filename (without extension) as argument.
% First, it breaks loops.
% If the argument is not empty and does not match |\childdocname|
% (which is set by the first inclusion of |childdoc.def|),
% |\ifchilddoc| is set to true, |\includeonly| is applied to the child file
% and |\jobname| is set to the main file
% (for proper handling of |.aux| files):
%    \begin{macrocode}
\newcommand{\childdocmain}[1]
{
  \childdocdisable\childdocmain{}
  \if?#1?\else
    \begingroup
      \def\childdoctmp{#1}
      \ifx\childdoctmp\childdocname
        \def\childdoctmp{}
      \else
        \def\childdoctmp
        {
          \childdoctrue
          \includeonly{\childdocname}
          \def\childdocjob{#1}
          \def\jobname{#1}
        }
      \fi
      \expandafter
    \endgroup
    \childdoctmp
  \fi
}
%    \end{macrocode}

% \macro{\childdocof}
% The command |\childdocof| redirects
% compilation to the main file |#1|.
%    \begin{macrocode}
\newcommand{\childdocof}[1]
{
  \childdocdisable
  \childdoctrue
  \includeonly{\childdocname}
  \def\jobname{#1}
  \def\childdocjob{#1}
  \input{#1}
}
%    \end{macrocode}

% \macro{\childdocby}
% The command |\childdocby| ....
%    \begin{macrocode}
\newcommand{\childdocby}[2][]
{
  \childdocdisable
  \childdoctrue
  \childdocmanualtrue
  \if?#1?\else
    \def\jobname{#2}
  \fi
  \def\childdocjob{#2}
  \input{#2}
  \endinput
}
%    \end{macrocode}

% \macro{\childdocforward}
% The command |\childdocforward| redirects
% compilation to the main file or
% (if the optional argument is given) a child file.
% Parameters are set as if the main file
% or a child file starting with |\childdocof| was compiled.
% Then compilation is handed over to the main file:
%    \begin{macrocode}
\newcommand{\childdocforward}[2][]
{
  \begingroup
    \if?#1?
      \def\childdoctmp
      {
        \def\childdocname{#2}
        \def\childdocjob{#2}
        \def\jobname{#2}
        \input{#2}
        \endinput
      }
    \else
      \def\childdoctmp
      {
        \childdocdisable
        \def\childdocname{#2}
        \childdoctrue
        \includeonly{#2}
        \def\childdocjob{#1}
        \def\jobname{#1}
        \input{#1}
        \endinput
      }
    \fi
    \expandafter
  \endgroup
  \childdoctmp
}
%    \end{macrocode}

% \macro{\childdocforwardprefix}
% The command |\childdocforwardprefix| redirects
% compilation to the main or a child file by means of a pattern.
% The prefix |#1| in the current filename is replaced by |#2|
% and the suffix of the current filename is kept
% (it is assumed that the filename does not contain the substring `|~~~|'
% which is used as a delimiter).
% Compilation is handed over to the new file by |\childdocforward|:
%    \begin{macrocode}
\newcommand{\childdocforwardprefix}[3][]
{
  \begingroup
    \def\childdocextract #2##1~~~{\def\childdoctmp{\childdocforward[#1]{#3##1}}}
    \expandafter\childdocextract\childdocname~~~
    \expandafter
  \endgroup
  \childdoctmp
}
%    \end{macrocode}

% \macro{\childdoc}
% The deprecated macro |\childdoc| is a legacy version of |\childdocmain|:
%    \begin{macrocode}
\newcommand{\childdoc}{\childdocmain}
%    \end{macrocode}

% \macro{\childdocredirect}
% The deprecated macro |\childdocredirect| is a legacy version
% of |\childdocforward| and |\childdocforwardprefix|:
%    \begin{macrocode}
\newcommand{\childdocredirect}[2][]
{
  \begingroup
    \if?#1?
      \def\childdoctmp{\childdocforward{#2}}
    \else
      \def\childdoctmp{\childdocforwardprefix{#1}{#2}}
    \fi
    \expandafter
  \endgroup
  \childdoctmp
}
%    \end{macrocode}

%\iffalse
%</package>
%\fi
%
\endinput
|\\
|\childdocmain{}|\\
\end{tabular}
\end{center}
at the very top of the main \LaTeX{} file,
in particular \emph{before} the |\documentclass| statement!
The argument of |\childdocmain| should be left empty
(but it must be present).

%%%%%%%%%%%%%%%%%%%%%%%%%%%%%%%%%%%%%%%%
\DescribeMacro{\childdocof}
Furthermore, add the commands
\begin{center}
\begin{tabular}{l}
|% \iffalse
%
% childdoc.dtx Copyright (C) 2017-2018 Niklas Beisert
%
% This work may be distributed and/or modified under the
% conditions of the LaTeX Project Public License, either version 1.3
% of this license or (at your option) any later version.
% The latest version of this license is in
%   http://www.latex-project.org/lppl.txt
% and version 1.3 or later is part of all distributions of LaTeX
% version 2005/12/01 or later.
%
% This work has the LPPL maintenance status `maintained'.
%
% The Current Maintainer of this work is Niklas Beisert.
%
% This work consists of the files childdoc.dtx and childdoc.ins
% and the derived files childdoc.def and cdocsamp.tex with
% cdocsch1.tex, cdocsch2.tex, cdocsdrf.tex, cdocsfn1.tex, cdocsfn2.tex.
%
%<package>\ifdefined\childdocmain\endinput\fi
%<package>\ProvidesFile{childdoc.def}[2018/12/30 v2.0 child document driver]
%<samplemain>\ProvidesFile{cdocsamp.tex}[2018/12/30 v2.0 sample for childdoc]
%<*driver>
%\ProvidesFile{childdoc.drv}[2018/12/30 v2.0 childdoc reference manual file]
\PassOptionsToClass{10pt,a4paper}{article}
\documentclass{ltxdoc}

\usepackage[margin=35mm]{geometry}
\usepackage{hyperref}
\usepackage{hyperxmp}
\usepackage[usenames]{color}

\hypersetup{colorlinks=true}
\hypersetup{pdfstartview=FitH}
\hypersetup{pdfpagemode=UseNone}
\hypersetup{pdfsource={}}
\hypersetup{pdflang={en-UK}}
\hypersetup{pdfcopyright={Copyright 2017-2018 Niklas Beisert.
  This work may be distributed and/or modified under the
  conditions of the LaTeX Project Public License, either version 1.3
  of this license or (at your option) any later version.}}
\hypersetup{pdflicenseurl={http://www.latex-project.org/lppl.txt}}
\hypersetup{pdfcontactaddress={ETH Zurich, ITP, HIT K,
  Wolfgang-Pauli-Strasse 27}}
\hypersetup{pdfcontactpostcode={8093}}
\hypersetup{pdfcontactcity={Zurich}}
\hypersetup{pdfcontactcountry={Switzerland}}
\hypersetup{pdfcontactemail={nbeisert@itp.phys.ethz.ch}}
\hypersetup{pdfcontacturl={http://people.phys.ethz.ch/\xmptilde nbeisert/}}

\newcommand{\secref}[1]{\hyperref[#1]{section \ref*{#1}}}

\parskip1ex
\parindent0pt
\let\olditemize\itemize
\def\itemize{\olditemize\parskip0pt}

\begin{document}

\title{The \textsf{childdoc} Package}
\hypersetup{pdftitle={The childdoc Package}}
\author{Niklas Beisert\\[2ex]
  Institut f\"ur Theoretische Physik\\
  Eidgen\"ossische Technische Hochschule Z\"urich\\
  Wolfgang-Pauli-Strasse 27, 8093 Z\"urich, Switzerland\\[1ex]
  \href{mailto:nbeisert@itp.phys.ethz.ch}
  {\texttt{nbeisert@itp.phys.ethz.ch}}}
\hypersetup{pdfauthor={Niklas Beisert}}
\hypersetup{pdfsubject={Manual for the LaTeX2e Package childdoc}}
\date{30 December 2018, \textsf{v2.0}}
\maketitle

\begin{abstract}\noindent
\textsf{childdoc} is a \LaTeXe{} package
that enables the direct compilation
of document sections included by |\include|
to individual files.
\end{abstract}

\begingroup
\parskip0ex
\tableofcontents
\endgroup

%%%%%%%%%%%%%%%%%%%%%%%%%%%%%%%%%%%%%%%%%%%%%%%%%%%%%%%%%%%%%%%%%%%%%%%%%%%%%%%%
%%%%%%%%%%%%%%%%%%%%%%%%%%%%%%%%%%%%%%%%%%%%%%%%%%%%%%%%%%%%%%%%%%%%%%%%%%%%%%%%
\section{Introduction}

\LaTeX{} provides a mechanism to structure a large document (such as a book)
into a main file and several child files (containing the chapters)
using the |\include| command.
This mechanism is beneficial for documents
which span hundreds of pages in order to
make the source file(s) more manageable.
Moreover, compilation can be restricted to
selected child files by means of the |\includeonly| command.
The latter feature can be used to reduce the compilation time while editing
(this was significantly more useful in the earlier days of \LaTeX{})
or to generate a smaller document which is easier to navigate.
Another application of |\includeonly| is to generate
documents consisting of selected parts of the complete document.

However, there are a few drawbacks of the plain |\include| mechanism:
\begin{itemize}
\item
The child files cannot be compiled on their own,
they can only be compiled via the main file.
A naive editing environment
(such as a text editor with an option
to have the current file processed by \LaTeX)
may require one to switch to the main file before compiling;
attempting to compile the child file produces errors.
\item
The main file must be modified (each time)
to adjust the |\includeonly| command
to the present needs. This easily leaves the main file in a messy state.
\item
The generated document will always carry the filename
of the main document. This is inconvenient if
several child files are to be compiled and
to be kept for distribution.
\end{itemize}

The present package provides a simple interface
to make child files individually compilable by \LaTeX{}.
Compiling a child file then has the same effect as compiling
the main file with an |\includeonly| command
to select the appropriate child.
Moreover the generated document will carry the name of the child
rather than the main file.
This resolves all three above issues.

This feature is meant to make the editing of books,
thesis documents and lecture notes somewhat more convenient.
However, the package can also be used efficiently for
composing a series of documents (such as exercise sheets)
which are typically distributed individually.
It then assists the author in generating the individual documents
(potentially in different versions)
as well as a document containing the collected series.
Another application is in developing style files
or other kinds of included material
where compilation of the style file could redirect
to a sample or test file.

%%%%%%%%%%%%%%%%%%%%%%%%%%%%%%%%%%%%%%%%%%%%%%%%%%%%%%%%%%%%%%%%%%%%%%%%%%%%%%%%
%%%%%%%%%%%%%%%%%%%%%%%%%%%%%%%%%%%%%%%%%%%%%%%%%%%%%%%%%%%%%%%%%%%%%%%%%%%%%%%%
\section{Usage}

First of all, the package \textsf{childdoc} is \emph{not} a standard
\LaTeXe{} |.sty| style file! Therefore it needs to be invoked in
a non-standard way.

%%%%%%%%%%%%%%%%%%%%%%%%%%%%%%%%%%%%%%%%%%%%%%%%%%%%%%%%%%%%%%%%%%%%%%%%%%%%%%%%
\subsection{Included Files}
\label{sec:include}

%%%%%%%%%%%%%%%%%%%%%%%%%%%%%%%%%%%%%%%%
\DescribeMacro{\childdocmain}
To use the package, add the commands
\begin{center}
\begin{tabular}{l}
|\input{childdoc.def}|\\
|\childdocmain{}|\\
\end{tabular}
\end{center}
at the very top of the main \LaTeX{} file,
in particular \emph{before} the |\documentclass| statement!
The argument of |\childdocmain| should be left empty
(but it must be present).

%%%%%%%%%%%%%%%%%%%%%%%%%%%%%%%%%%%%%%%%
\DescribeMacro{\childdocof}
Furthermore, add the commands
\begin{center}
\begin{tabular}{l}
|\input{childdoc.def}|\\
|\childdocof{|\textit{main}|}|\\
\end{tabular}
\end{center}
at the top of every child file \textit{child}
which is included by |\include{|\textit{child}|}|
from within the main file
(or at least for those files to be compiled individually).
The argument \textit{main} must be the filename of the main file.

There are a couple of
considerations in setting up the main and child documents:

%%%%%%%%%%%%%%%%%%%%%%%%%%%%%%%%%%%%%%%%
\paragraph{Restrictions.}

Please note the following restrictions:
\begin{itemize}
\item
|\childdocmain| must be called with one argument \textit{main}
to ensure compatibility with earlier version of the package.
It must either be empty (|\childdocmain{}|)
or precisely match the filename of the main file in which it is specified.
See \secref{sec:detection} for further information.
\item
The filename \textit{main} must be specified without the |.tex| extension.
\item
The filename \textit{main} is case sensitive
(even in case-insensitive file systems)
due to internal string comparison.
\item
The argument \textit{main} should be fully expanded, it cannot be a macro.
\item
Subdirectories and special characters should be avoided in filenames.
\item
The command |\childdocmain{|\textit{main}|}| must be followed by a whitespace.
It should not be followed immediately by another command
or by a comment mark `|%|'.
This is because the \TeX{} parser reads the token immediately following
the argument of |\childdocmain| and puts it
at the beginning of every child section;
however, a white\-space is ignored.
\end{itemize}

%%%%%%%%%%%%%%%%%%%%%%%%%%%%%%%%%%%%%%%%
\paragraph{Content of Main File.}

It is advisable to place all content in the child files included by |\include|.
Any output contained in the main file will appear in all child documents
unless suppressed manually;
it cannot be suppressed automatically by the |\includeonly| directive
and thus should normally be avoided.
A method to include some content in the main file
by means of conditional processing is described in \secref{sec:conditional}.

%%%%%%%%%%%%%%%%%%%%%%%%%%%%%%%%%%%%%%%%
\paragraph{Page Numbering.}

When only a part of the document is compiled,
the appropriate numbering of pages
(as well as other status parameters)
is determined from the |.aux| files.
The latter contain information from previous passes.
However this information needs to propagate through
all intermediate child documents.
Therefore the page numbering in child documents may well
be inconsistent until the complete document is compiled at least once.

A useful (if unconventional) way to always ensure a consistent
page numbering is to restart the numbering in each child document
and denote the pages by `\textit{child}|.|\textit{page}'
where \textit{child} represents the chapter/section number of the child file.
This can be achieved by the command
|\numberwithin{page}{|\textit{child}|}|
of the \textsf{amsmath} package
where \textit{child} can be |chapter| or |section|
depending on the chosen structuring.
Alternatively, one can modify the macro |\thepage| appropriately
and reset the counter |page| at the start of each child file.

%%%%%%%%%%%%%%%%%%%%%%%%%%%%%%%%%%%%%%%%%%%%%%%%%%%%%%%%%%%%%%%%%%%%%%%%%%%%%%%%
\subsection{Conditional Processing}
\label{sec:conditional}

The package provides a mechanism to compile different versions
of a document. To customise the versions further some conditional processing
can come in handy to distinguish which version is being compiled.
The package provides two macros to describe the compilation context:

%%%%%%%%%%%%%%%%%%%%%%%%%%%%%%%%%%%%%%%%
\DescribeMacro{\ifchilddoc}
The conditional |\ifchilddoc| distinguishes between the compilation of
child documents and the main document:
%
\begin{center}
|\ifchilddoc |\textit{child-code}| |[|\||else |\textit{main-code}]| \||fi|
\end{center}

%%%%%%%%%%%%%%%%%%%%%%%%%%%%%%%%%%%%%%%%
\DescribeMacro{\childdocname}
\DescribeMacro{\childdocjob}
The macro |\childdocname| contains the filename (without extension)
of the main or child file being processed.
Note that |\childdocjob| will always contain the name of the main file.

%%%%%%%%%%%%%%%%%%%%%%%%%%%%%%%%%%%%%%%%
\paragraph{Title Page.}

Conditional processing can be used to include a title or banner page
in the main document when proper precautions are taken.
Importantly, the code in the main file should ensure that the page counter
(as well as other status parameters which are stored in the |.aux| files)
takes the same value after the conditional processing.
Otherwise the page numbers may take divergent values
depending on which part is compiled.

For example, a title page could be declared by:
%
\begin{center}
\begin{tabular}{l}
|\ifchilddoc\||else|\\
|\addtocounter{page}{-1}|\\
\textit{code for title page}\\
|\newpage|\\
|\||fi|
\end{tabular}
\end{center}
%
A banner page for the child documents can be generated by:
%
\begin{center}
\begin{tabular}{l}
|\ifchilddoc|\\
|\addtocounter{page}{-1}|\\
\textit{code for banner page}\\
|\newpage|\\
|\||fi|
\end{tabular}
\end{center}
%
Here one could write a message such as:
\begin{center}
|This is the part \childdocname{} of \childdocjob{}.|
\end{center}

%%%%%%%%%%%%%%%%%%%%%%%%%%%%%%%%%%%%%%%%%%%%%%%%%%%%%%%%%%%%%%%%%%%%%%%%%%%%%%%%
\subsection{Flags}
\label{sec:flags}

The package makes it easy to generate different versions
of the main or child documents.
To this end compilation flags can be defined
and assigned different default values.
They will be particularly useful in conjunction
with the forwarding mechanism described in \secref{sec:forward}.

For example, it may be useful to have a flag |\version|
which can be set to |draft| or |final|.
The document source will contain some conditional code
depending on the value of |\version|.
Suppose further, the flag should default to |final| for the main file
and to |draft| for child files
which is a natural assignment for editing the document.
This is achieved by placing the following code
in the preamble of the main document
(below the |\childdocmain| directive):
%
\begin{center}
\begin{tabular}{l}
|\ifchilddoc|\\
|\providecommand{\version}{draft}|\\
|\||else|\\
|\providecommand{\version}{final}|\\
|\||fi|
\end{tabular}
\end{center}
%
The definition by |\providecommand| makes sure
that previous definitions are not overwritten.
Further statements |\providecommand{\version}{...}|
can thus be added before the above code to override it.

For the main file, one might add a line
(between |\childdocmain| and the above block)
%
\begin{center}
|%\ifchilddoc\||else\providecommand{\version}{draft}\||fi|
\end{center}
%
which can be uncommented to produce a draft version.
Likewise one can add a line to the very top of a child file
(above the |\childdocof{|\textit{main}|}| directive)
%
\begin{center}
|%\providecommand{\version}{final}|
\end{center}
%
which can be uncommented to produce the final version of this child document.

%%%%%%%%%%%%%%%%%%%%%%%%%%%%%%%%%%%%%%%%%%%%%%%%%%%%%%%%%%%%%%%%%%%%%%%%%%%%%%%%
\subsection{Forwarding}
\label{sec:forward}

Different versions of the main or child documents
using compilation flags as described in \secref{sec:flags}
can be (permanently) stored in different files
for convenient compilation, viewing and distribution.
To this end, the package defines a command
to pass on compilation to a different file:

%%%%%%%%%%%%%%%%%%%%%%%%%%%%%%%%%%%%%%%%
\DescribeMacro{\childdocforward}
The command |\childdocforward| redirects processing to
another source file:
%
\begin{center}
\begin{tabular}{l}
|\input{childdoc.def}|\\
|\childdocforward[|\textit{main}|]{|\textit{dest}|}|\\
\end{tabular}
\end{center}
%
The argument \textit{dest} is the destination file
(without extension).
It should be the main file or one of the child files.
Note that further \textsf{childdoc} directives
such as |\childdocof| and |\childdocforward|
in the indicated file will be processed in this form.
The optional argument \textit{main}
passes on directly to the main file \textit{main}
while pretending to compile the child \textit{dest}.
This form behaves as if \textit{dest}
issues |\childdocof{|\textit{main}|}| right away,
and no further \textsf{childdoc} directives will be processed.

%%%%%%%%%%%%%%%%%%%%%%%%%%%%%%%%%%%%%%%%
\DescribeMacro{\...prefix}
In the alternative form |\childdocforwardprefix|,
%
\begin{center}
\begin{tabular}{l}
|\input{childdoc.def}|\\
|\childdocforwardprefix[|\textit{main}|]{|\textit{prefix}|}{|\textit{dest}|}|
\end{tabular}
\end{center}
%
the destination file is determined by a pattern
depending on the current file:
To make this work, the current file must be called
`{\textit{prefix}\hspace{0.2em}\textit{suffix}}'
with \textit{prefix} matching precisely the argument.
Processing is then passed on to the file
`{\textit{dest}\hspace{0.2em}\textit{suffix}}'.
Surely, the same effect is achieved by
directly specifying the
argument `{\textit{dest}\hspace{0.2em}\textit{suffix}}'
in the first form.
However, that requires to set up a different file
for each child. With the alternative form of the command
all these files can have exactly the same content
which simplifies setting them up and maintaining them.

For example, the following file |draft.tex|
with a compilation flag |\version| as described in \secref{sec:flags}
compiles the main document as a draft:
%
\begin{center}
\begin{tabular}{l}
|\def\version{draft}|\\
|\input{childdoc.def}|\\
|\childdocforward{|\textit{main}|}|
\end{tabular}
\end{center}
%
Likewise, the following files |final|\textit{nn}|.tex|
compile the final version of the child document
|child|\textit{nn}|.tex|:
%
\begin{center}
\begin{tabular}{l}
|\def\version{final}|\\
|\input{childdoc.def}|\\
|\childdocforwardprefix{final}{child}|
\end{tabular}
\end{center}
%

Note that when several versions of a main file and/or of each child file
are to be generated, it may be convenient to set up a |Makefile| or
shell script to automatise the process.

%%%%%%%%%%%%%%%%%%%%%%%%%%%%%%%%%%%%%%%%%%%%%%%%%%%%%%%%%%%%%%%%%%%%%%%%%%%%%%%%
\subsection{Command Line Processing}
\label{sec:commandline}

The effect of redirection files can also be achieved by invoking
the \LaTeX{} compiler with a more elaborate command line.
Most conveniently this should be done as part
of a shell script or a |Makefile|.

When using \textsf{childdoc} in the main file, the following
command lines effectively perform a redirection
(note that depending on the shell being used,
backslashes may have to be doubled: `|\|' $\to$ `|\\|'):
%
\begin{center}
|... -jobname "|\textit{target}|" |\\|"|[\textit{flags}]%
|\input{childdoc.def}\childdocforward[|\textit{main}|]{|\textit{dest}|}"|
\end{center}
%
Here \textit{target} is the name of the output file,
\textit{main} is the name of the main file
and \textit{dest} is the name of the main or child file to be processed
(all filenames without extensions).
The optional argument \textit{main} can be omitted
if \textit{main} matches \textit{dest}.
Optionally, compilation \textit{flags} can be defined via |\def| commands.
This command line makes the \TeX{} engine believe
it is compiling the file \textit{target}
whose content is specified as the latter parameter.
The provided code then forwards the processing to
\textit{main} or \textit{dest} as described in \secref{sec:forward}.

%%%%%%%%%%%%%%%%%%%%%%%%%%%%%%%%%%%%%%%%%%%%%%%%%%%%%%%%%%%%%%%%%%%%%%%%%%%%%%%%
\subsection{Include by Input}
\label{sec:input}

Including child documents by |\include| has some restrictions by design.
Most notably, the content of a child document always occupies
its own set of pages; pages cannot be shared between child documents.
Usually, this behaviour makes perfect sense
because each child document contain an essential part of the document.
However, in some situations it may be desirable to compose
a document from a collection of parts
without having mandatory page breaks between then.
For this case, the package
provides a mechanism to include parts
by |\input| which can also be processed individually.
However, by construction this mechanism
requires manual handling of the content to be output.

%%%%%%%%%%%%%%%%%%%%%%%%%%%%%%%%%%%%%%%%
\DescribeMacro{\ifchilddocmanual}
The main file should be prepared as usual, see \secref{sec:include}.
However, the document body must make a distinction
between processing of an individual part and of the main document, e.g.:
%
\begin{center}
\begin{tabular}{l}
|\ifchilddocmanual|\\
|\input{\childdocname}|\\
|\||else|\\
\textit{document body with }|\input{|\textit{part}|}|\\
|\||fi|
\end{tabular}
\end{center}
%
The conditional |\ifchilddocmanual| is true whenever
a part to be included by |\input| is being compiled,
and the name of the part is stored in |\childdocname|.

%%%%%%%%%%%%%%%%%%%%%%%%%%%%%%%%%%%%%%%%
\DescribeMacro{\childdocby}
Each part to be included by |\input| should start with:
%
\begin{center}
\begin{tabular}{l}
|\input{childdoc.def}|\\
|\childdocby{|\textit{main}|}|\\
\end{tabular}
\end{center}
%
The directive |\childdocby| is similar to |\childdocof|
described in \secref{sec:include},
but the subsequent selection of content must be done manually.
To that end, both |\ifchilddoc| and |\ifchilddocmanual|
will be true upon processing of a part,
and the name of the part is stored in |\childdocname|.
Note that |\jobname| will be set to the filename of the current part
so that each part receives an individual |.aux| file
that does not interfere with the |.aux| file(s) of the main document.
This behaviour can be altered by the alternative form
|\childdocby[*]{|\textit{main}|}| (with a non-empty optional argument)
which uses the |.aux| file of the main document
by setting |\jobname| to \textit{main}.

%%%%%%%%%%%%%%%%%%%%%%%%%%%%%%%%%%%%%%%%%%%%%%%%%%%%%%%%%%%%%%%%%%%%%%%%%%%%%%%%
\subsection{Driver Development}
\label{sec:driver}

The \textsf{childdoc} mechanism can also be use for the development
of definition files such as \LaTeX{} styles or classes.
This case differs from the above setup with multiple parts
included by |\include| in that no |\includeonly| should be invoked.
This can be achieved by starting the include file
(before |\ProvidesPackage|) with:
%
\begin{center}
\begin{tabular}{l}
|\input{childdoc.def}|\\
|\childdocforward{|\textit{main}|}|\\
\end{tabular}
\end{center}
%
or alternatively with:
%
\begin{center}
\begin{tabular}{l}
|\input{childdoc.def}|\\
|\childdocby{|\textit{main}|}|\\
\end{tabular}
\end{center}
%
Both forms have slightly different effects as described above.
The main file is prepared as usual, see \secref{sec:include}.

%%%%%%%%%%%%%%%%%%%%%%%%%%%%%%%%%%%%%%%%%%%%%%%%%%%%%%%%%%%%%%%%%%%%%%%%%%%%%%%%
\subsection{Legacy Detection}
\label{sec:detection}

The directive |\childdocmain| in the main file can detect
whether the complete document or merely a child is to be compiled
even without using the directive |\childdocof|.
This method is deprecated because it is less robust
and there is no compelling reason to use it;
it is merely provided for backward compatibility
and it may be removed in future versions.

If the detection mechanism is to be used,
it is mandatory to correctly specify
the filename of the main file as the argument of |\childdocmain|:
%
\begin{center}
\begin{tabular}{l}
|\input{childdoc.def}|\\
|\childdocmain{|\textit{main}|}|\\
\end{tabular}
\end{center}
%
If |\jobname| does not match the argument \textit{main} of |\childdocmain|,
it is assumed that |\jobname| points to the child file to be compiled.
When using |\childdocmain| with the main file specified as argument,
it suffices to start a child file
with just |\input{|\textit{main}|}|
without loading of the package and using |\childdocof|.
If instead all processing is done
with the appropriate \textsf{childdoc} directives,
the argument of \textit{main} of |\childdocmain| can be empty.

An alternative version of the command line processing described
in \secref{sec:commandline} using the detection mechanism reads:
%
\begin{center}
|... -jobname "|\textit{target}|" "|[\textit{flags}]%
[|\def\jobname{|\textit{dest}|}|]|\input{|\textit{main}|}"|
\end{center}

%%%%%%%%%%%%%%%%%%%%%%%%%%%%%%%%%%%%%%%%%%%%%%%%%%%%%%%%%%%%%%%%%%%%%%%%%%%%%%%%
\subsection{Manual Code}
\label{sec:manual}

In case one cannot be certain whether the definitions file |childdoc.def|
is installed on the target \TeX{} distribution
and one prefers not to ship it,
it is conceivable to paste a few relevant commands into the sources.

To that end, drop all statements |\input{childdoc.def}|
and perform the replacements as outlined below.
Instead of |\childdocmain{|\textit{main}|}| add the following code
to the top of the main file:
%
\begin{center}
\begin{tabular}{l}
|\||ifdefined\childdocname\endinput\||fi\newif\ifchilddoc|\\
|\edef\childdocname{\scantokens\expandafter{\jobname\noexpand}}|\\
|\def\childdocmain{|\textit{main}|}\||ifx\childdocmain\childdocname\||else|\\
|\childdoctrue\includeonly{\childdocname}\let\jobname\childdocmain\||fi|\\
\end{tabular}
\end{center}
%
Instead of |\childdocof{|\textit{main}|}| just include the main file
at the top of each child file:
%
\begin{center}
|\input{|\textit{main}|}|
\end{center}
%
A simple redirection |\childdocforward{|\textit{dest}|}| is achieved by:
%
\begin{center}
|\def\jobname{|\textit{dest}|}\input{\jobname}|
\end{center}
%
The redirection with prefix
|\childdocforwardprefix[|\textit{prefix}|]{|\textit{dest}|}|
is accomplished by:
%
\begin{center}
\begin{tabular}{l}
|{\edef\jobname{\scantokens\expandafter{\jobname\noexpand}}|\\
|\def\redirectjob |\textit{prefix}|#1~~~{\gdef\jobname{|\textit{dest}|#1}}|\\
|\expandafter\redirectjob\jobname~~~}\input{\jobname}|
\end{tabular}
\end{center}

In an alternative approach,
child documents can be compiled by a specific command line
without additional code or specific definitions:
%
\begin{center}
|... -jobname "|\textit{target}|" "|[\textit{flags}]%
|\includeonly{|\textit{dest}|}\input{|\textit{main}|}"|
\end{center}
%

%%%%%%%%%%%%%%%%%%%%%%%%%%%%%%%%%%%%%%%%%%%%%%%%%%%%%%%%%%%%%%%%%%%%%%%%%%%%%%%%
%%%%%%%%%%%%%%%%%%%%%%%%%%%%%%%%%%%%%%%%%%%%%%%%%%%%%%%%%%%%%%%%%%%%%%%%%%%%%%%%
\section{Information}

%%%%%%%%%%%%%%%%%%%%%%%%%%%%%%%%%%%%%%%%%%%%%%%%%%%%%%%%%%%%%%%%%%%%%%%%%%%%%%%%
\subsection{Copyright}

Copyright \copyright{} 2017--2018 Niklas Beisert

This work may be distributed and/or modified under the
conditions of the \LaTeX{} Project Public License, either version 1.3
of this license or (at your option) any later version.
The latest version of this license is in
  \url{http://www.latex-project.org/lppl.txt}
and version 1.3 or later is part of all distributions of \LaTeX{}
version 2005/12/01 or later.

This work has the LPPL maintenance status `maintained'.

The Current Maintainer of this work is Niklas Beisert.

This work consists of the files |README.txt|, |childdoc.ins| and |childdoc.dtx|
as well as the derived files |childdoc.def|, |cdocsamp.tex|
with |cdocsch1.tex|, |cdocsch2.tex|, |cdocspt3.tex|, |cdocspt4.tex|,
|cdocsdrf.tex|, |cdocsfn1.tex|, |cdocsfn2.tex|
as well as |childdoc.pdf|.

%%%%%%%%%%%%%%%%%%%%%%%%%%%%%%%%%%%%%%%%%%%%%%%%%%%%%%%%%%%%%%%%%%%%%%%%%%%%%%%%
\subsection{Files and Installation}

The package consists of the files:
%
\begin{center}
\begin{tabular}{ll}
    |README.txt|   & readme file \\
    |childdoc.ins| & installation file \\
    |childdoc.dtx| & source file \\
    |childdoc.def| & definition file \\
    |cdocsamp.tex| & sample main file \\
    |cdocsch1.tex| & sample include file \\
    |cdocsch2.tex| & sample include file \\
    |cdocspt3.tex| & sample part file \\
    |cdocspt4.tex| & sample part file \\
    |cdocsdrf.tex| & sample redirection file \\
    |cdocsfn1.tex| & sample redirection file \\
    |cdocsfn2.tex| & sample redirection file \\
    |childdoc.pdf| & manual
\end{tabular}
\end{center}
%
The distribution consists of the files
|README.txt|, |childdoc.ins| and |childdoc.dtx|.
%
\begin{itemize}
\item
Run (pdf)\LaTeX{} on |childdoc.dtx|
to compile the manual |childdoc.pdf| (this file).
\item
Run \LaTeX{} on |childdoc.ins| to create the definitions file |childdoc.def|
and the sample |cdocsamp.tex| with include files
|cdocsch1.tex|, |cdocsch2.tex|, |cdocspt3.tex|, |cdocspt4.tex|,
|cdocsdrf.tex|, |cdocsfn1.tex|, |cdocsfn2.tex|.
Then copy the file |childdoc.def| to an appropriate directory of your \LaTeX{}
distribution, e.g.\ \textit{texmf-root}|/tex/latex/childdoc|.
\end{itemize}

%%%%%%%%%%%%%%%%%%%%%%%%%%%%%%%%%%%%%%%%%%%%%%%%%%%%%%%%%%%%%%%%%%%%%%%%%%%%%%%%
\subsection{Related CTAN Packages}

There are several other packages which offer a similar functionality:
%
\begin{itemize}
\item
The packages
\href{http://ctan.org/pkg/docmute}{\textsf{docmute}},
\href{http://ctan.org/pkg/includex}{\textsf{includex}} and
\href{http://ctan.org/pkg/standalone}{\textsf{standalone}}
provide commands to include only the document body of
a child file thus allowing both files to be compiled individually.
\item
The packages \href{http://ctan.org/pkg/subdocs}{\textsf{subdocs}}
and \href{http://ctan.org/pkg/subfiles}{\textsf{subfiles}}
provide structures in which the main and child documents can be
encapsulated and allowing them to be compiled individually.
The inclusion mechanism is different from the conventional |\include|.
\item
The package \href{http://ctan.org/pkg/combine}{\textsf{combine}}
is an elaborate solution to combine several documents into one.
\end{itemize}
%
See also the CTAN topic \href{http://ctan.org/topic/subdocs}{\textsf{subdocs}}
for further related packages.
The present package differs from the above solutions in that
a document structure constructed with the conventional |\include| mechanism
just needs two extra commands at the top of every file
such that all constituent files can be compiled individually.

%%%%%%%%%%%%%%%%%%%%%%%%%%%%%%%%%%%%%%%%%%%%%%%%%%%%%%%%%%%%%%%%%%%%%%%%%%%%%%%%
%\subsection{Feature Suggestions}
%
%The following is a list of features which may be useful for future
%versions of this package:
%%
%\begin{itemize}
%\item
%\ldots
%\end{itemize}

%%%%%%%%%%%%%%%%%%%%%%%%%%%%%%%%%%%%%%%%%%%%%%%%%%%%%%%%%%%%%%%%%%%%%%%%%%%%%%%%
\subsection{Revision History}

%%%%%%%%%%%%%%%%%%%%%%%%%%%%%%%%%%%%%%%%
\paragraph{v2.0:} 2018/12/30

\begin{itemize}
\item
immediate forward processing
\item
added |\childdocby| mechanism
\item
manual restructured
\end{itemize}

%%%%%%%%%%%%%%%%%%%%%%%%%%%%%%%%%%%%%%%%
\paragraph{v1.6:} 2018/01/17

\begin{itemize}
\item
application for development of include files
\item
corrections to manual
\end{itemize}

%%%%%%%%%%%%%%%%%%%%%%%%%%%%%%%%%%%%%%%%
\paragraph{v1.5:} 2017/05/21

\begin{itemize}
\item
more complete structuring introduced
\item
|\childdocof| introduced
\item
|\childdoc| renamed to |\childdocmain|
\item
|\childredirect| renamed to |\childdocforward| and |\childdocforwardprefix|
and functionality expanded
\end{itemize}

%%%%%%%%%%%%%%%%%%%%%%%%%%%%%%%%%%%%%%%%
\paragraph{v1.0:} 2017/04/27

\begin{itemize}
\item
manual and install package
\item
first version published on CTAN
\end{itemize}

%%%%%%%%%%%%%%%%%%%%%%%%%%%%%%%%%%%%%%%%
\paragraph{v0.6:} 2017/04/26

\begin{itemize}
\item
redirection mechanism added
\end{itemize}

%%%%%%%%%%%%%%%%%%%%%%%%%%%%%%%%%%%%%%%%
\paragraph{v0.5:} 2017/04/26

\begin{itemize}
\item
functionality in definition file
\end{itemize}


%%%%%%%%%%%%%%%%%%%%%%%%%%%%%%%%%%%%%%%%%%%%%%%%%%%%%%%%%%%%%%%%%%%%%%%%%%%%%%%%
%%%%%%%%%%%%%%%%%%%%%%%%%%%%%%%%%%%%%%%%%%%%%%%%%%%%%%%%%%%%%%%%%%%%%%%%%%%%%%%%
%%%%%%%%%%%%%%%%%%%%%%%%%%%%%%%%%%%%%%%%%%%%%%%%%%%%%%%%%%%%%%%%%%%%%%%%%%%%%%%%
\appendix

\settowidth\MacroIndent{\rmfamily\scriptsize 000\ }

 \DocInput{childdoc.dtx}

\end{document}
%</driver>
% \fi
%
% %%%%%%%%%%%%%%%%%%%%%%%%%%%%%%%%%%%%%%%%%%%%%%%%%%%%%%%%%%%%%%%%%%%%%%%%%%%%%%
% %%%%%%%%%%%%%%%%%%%%%%%%%%%%%%%%%%%%%%%%%%%%%%%%%%%%%%%%%%%%%%%%%%%%%%%%%%%%%%
% \section{Sample}
%\iffalse
%<*samplemain>
%\fi
%
% The following presents a sample document
% with two chapters, two parts, a title page,
% a compile flag as well as three forwarding files to set the flag.
% It consists of eight |.tex| files:
% \begin{center}
% \begin{tabular}{ll}
% |cdocsamp.tex|&main file\\
% |cdocsch1.tex|&include file for chapter 1\\
% |cdocsch2.tex|&include file for chapter 2\\
% |cdocspt3.tex|&include file for part 3\\
% |cdocspt4.tex|&include file for part 4\\
% |cdocsdrf.tex|&forwarding file for main file in draft mode\\
% |cdocsfi1.tex|&forwarding file for final version of chapter 1\\
% |cdocsfi2.tex|&forwarding file for final version of chapter 2\\
% \end{tabular}
% \end{center}
% Each of the eight files can be compiled directly by the \LaTeX{} compiler.
%
% %%%%%%%%%%%%%%%%%%%%%%%%%%%%%%%%%%%%%%
% \paragraph{Main File.}
%
% The main file is called |cdocsamp.tex|.
%
% Load the \textsf{childdoc} definitions and
% declare the filename for the main document:
%    \begin{macrocode}
\input{childdoc.def}
\childdocmain{}
%    \end{macrocode}

% Optional override for |\version| flag:
%    \begin{macrocode}
%%\ifchilddoc\else\providecommand{\version}{draft}\fi
%    \end{macrocode}

% Define the default values for the |\version| flag
% (|final| for the main file and |draft| for childs):
%    \begin{macrocode}
\ifchilddoc
\providecommand{\version}{draft}
\else
\providecommand{\version}{final}
\fi
%    \end{macrocode}

% Load the standard document class:
%    \begin{macrocode}
\documentclass[12pt]{article}
%    \end{macrocode}

% Start the document body:
%    \begin{macrocode}
\begin{document}
%    \end{macrocode}

% Declare a title page.
% Print title, part of document being processed and version flag:
%    \begin{macrocode}
\addtocounter{page}{-1}
\begin{center}
{\LARGE\bfseries{}childdoc example\par}
\vspace{1cm}
\ifchilddoc
\ifchilddocmanual part\else chapter\fi:
`\childdocname' of `\childdocjob'\par
\else
main document: `\childdocjob'\par
\fi
version: \version\par
\end{center}
\newpage
%    \end{macrocode}

% Manually include selected file,
% otherwise process as usual:
%    \begin{macrocode}
\ifchilddocmanual
\section*{part `\childdocname'}
\input{\childdocname}
\else
%    \end{macrocode}

% Include the two chapters:
%    \begin{macrocode}
\include{cdocsch1}
\include{cdocsch2}
%    \end{macrocode}

% Include the two parts unless only chapters should be displayed:
%    \begin{macrocode}
\ifchilddoc\else
\section{part three}
\input{cdocspt3}
\section{part four}
\input{cdocspt4}
\fi
%    \end{macrocode}

% Process as usual until here:
%    \begin{macrocode}
\fi
%    \end{macrocode}

% End of document body:
%    \begin{macrocode}
\end{document}
%    \end{macrocode}
%\iffalse
%</samplemain>
%\fi
%
% %%%%%%%%%%%%%%%%%%%%%%%%%%%%%%%%%%%%%%
% \paragraph{Chapter Include Files.}
%
% The include files are called |cdocsch1.tex| and |cdocsch2.tex|.
%
%\iffalse
%<*samplechap1|samplechap2>
%\fi

% Optional override for |\version| flag:
%    \begin{macrocode}
%%\providecommand{\version}{final}
%    \end{macrocode}

% Include the main document:
%    \begin{macrocode}
\input{childdoc.def}
\childdocof{cdocsamp}
%    \end{macrocode}

%\iffalse
%</samplechap1|samplechap2>
%\fi
%
%\iffalse
%<*samplechap1>
%\fi
% Some text for chapter 1:
%    \begin{macrocode}
\section{one}
some text in chapter one
%    \end{macrocode}

%\iffalse
%</samplechap1>
%\fi
% Some text for chapter 2:
%\iffalse
%<*samplechap2>
%\fi
%    \begin{macrocode}
\section{two}
more text in chapter two
%    \end{macrocode}

%\iffalse
%</samplechap2>
%\fi
%
% %%%%%%%%%%%%%%%%%%%%%%%%%%%%%%%%%%%%%%
% \paragraph{Part Include Files.}
%
% The include files are called |cdocspt3.tex| and |cdocspt4.tex|.
%
%\iffalse
%<*samplepart3|samplepart4>
%\fi

% Optional override for |\version| flag:
%    \begin{macrocode}
%%\providecommand{\version}{final}
%    \end{macrocode}

% Include the main document:
%    \begin{macrocode}
\input{childdoc.def}
\childdocby{cdocsamp}
%    \end{macrocode}

%\iffalse
%</samplepart3|samplepart4>
%\fi
%
%\iffalse
%<*samplepart3>
%\fi
% Some text for part 3:
%    \begin{macrocode}
some text in part three
%    \end{macrocode}

%\iffalse
%</samplepart3>
%\fi
% Some text for part 4:
%\iffalse
%<*samplepart4>
%\fi
%    \begin{macrocode}
more text in part four
%    \end{macrocode}

%\iffalse
%</samplepart4>
%\fi
%
% %%%%%%%%%%%%%%%%%%%%%%%%%%%%%%%%%%%%%%
% \paragraph{Forwarding for a Complete Draft.}
%
% The following forwarding file |cdocsdrf.tex|
% compiles the main document in draft mode:
%\iffalse
%<*sampledraft>
%\fi
%    \begin{macrocode}
\def\version{draft}
\input{childdoc.def}
\childdocforward{cdocsamp}
%    \end{macrocode}

%\iffalse
%</sampledraft>
%\fi
%
% %%%%%%%%%%%%%%%%%%%%%%%%%%%%%%%%%%%%%%
% \paragraph{Forwarding for Final Version of the Chapters.}
%
% The following forwarding files |cdocsfn1.tex| and |cdocsfn2.tex|
% (with identical content)
% compile the final versions of the child documents
% |cdocsch1.tex| and |cdocsch2.tex|, respectively:
%\iffalse
%<*samplefinal>
%\fi
%    \begin{macrocode}
\def\version{final}
\input{childdoc.def}
\childdocforwardprefix[cdocsamp]{cdocsfn}{cdocsch}
%    \end{macrocode}

%\iffalse
%</samplefinal>
%\fi
%
% %%%%%%%%%%%%%%%%%%%%%%%%%%%%%%%%%%%%%%
% \paragraph{Command Line Processing.}
%
% The following three command lines generate the output files
% |cdocscld|, |cdocscl1| and |cdocscl2|
% which should be identical to
% |cdocsdrf|, |cdocsch1| and |cdocsfn2|, respectively:
% \begin{center}
% \begin{tabular}{l}
% |latex -jobname cdocscld \|\\
% |  "\def\version{draft}\input{childdoc.def}\childdocforward{cdocsamp}"|\\
% |latex -jobname cdocscl1 \|\\
% |  "\input{childdoc.def}\childdocforward[cdocsamp]{cdocsch1}"|\\
% |latex -jobname cdocscl2 \|\\
% |  "\def\version{final}\input{childdoc.def}\childdocforward{cdocsch2}"|
% \end{tabular}
% \end{center}
% Note that the trailing backslash on each first line
% merely continues the input to the second line
% (for convenient cut ant paste).
% Furthermore, the command |latex| can be replaced by any
% of its alternative versions such as |pdflatex|.
%
% %%%%%%%%%%%%%%%%%%%%%%%%%%%%%%%%%%%%%%%%%%%%%%%%%%%%%%%%%%%%%%%%%%%%%%%%%%%%%%
% %%%%%%%%%%%%%%%%%%%%%%%%%%%%%%%%%%%%%%%%%%%%%%%%%%%%%%%%%%%%%%%%%%%%%%%%%%%%%%
% \section{Implementation}
%\iffalse
%<*package>
%\fi
%
% This section describes the definitions file |childdoc.def|.

% The definitions cannot be loaded using |\usepackage| or |\RequirePackage|
% which has a mechanism to prevent loading a style file more than once.
% When loading the definitions by means of |\input|
% multiple instances have to be prevented manually:
%\iffalse
%This code needs to be before the `\ProvidesFile' directive
%which is defined at the beginning of this file.
%Therefore it is also placed there and commented out here.
%</package>
%<*discard>
%\fi
%    \begin{macrocode}
\ifdefined\childdocmain\endinput\fi
%    \end{macrocode}
%\iffalse
%</discard>
%<*package>
%\fi
%
% \macro{\ifchilddoc}
% \macro{\ifchilddocmanual}
% The conditional |\ifchilddoc| tells whether a
% child (true) or main (false) document is being compiled.
% The conditional |\ifchilddocmanual| tells whether
% the |\includeonly| mechanism is used (false) or
% the selection of child files must be performed manually (true).
% The definitions initialise to false:
%    \begin{macrocode}
\newif\ifchilddoc
\newif\ifchilddocmanual
%    \end{macrocode}

% \macro{\childdocname}
% \macro{\childdocjob}
% The macro |\childdocname| stores the name of the main document
% to be compiled. The macro |\childdocjob| stores the name of
% the document on which the \LaTeX{} compiler was originally invoked.
% The content of |\jobname| cannot be compared
% to filenames specified in the source due to different catcodes.
% The following code rescans |\jobname|, stores the result
% in |\childdocname| and saves a copy in |\childdocjob|:
%    \begin{macrocode}
\edef\childdocname{\scantokens\expandafter{\jobname\noexpand}}
\let\childdocjob\childdocname
%    \end{macrocode}

% \macro{\childdocdisable}
% The macro |\childdocdisable| prevents the main file
% from being processed more than once.
% At this stage, the main document command |\childdocmain|
% is assumed to be called once again where it should do nothing.
% Any subsequent call to it should prevent
% a secondary processing of the main document
% It overwrites the forwarding commands
% |\childdocof| and |\childdocforward|
% with empty macros to prevent further inclusions of the main document:
%    \begin{macrocode}
\newcommand{\childdocdisable}
{
  \renewcommand{\childdocmain}[1]{\renewcommand{\childdocmain}[1]{\endinput}}
  \renewcommand{\childdocof}[1]{}
  \renewcommand{\childdocby}[2][]{}
  \renewcommand{\childdocforward}[2][]{}
  \renewcommand{\childdocdisable}{}
}
%    \end{macrocode}

% \macro{\childdocmain}
% The macro |\childdocmain| is to be called at the top of the main file
% with nothing or the main filename (without extension) as argument.
% First, it breaks loops.
% If the argument is not empty and does not match |\childdocname|
% (which is set by the first inclusion of |childdoc.def|),
% |\ifchilddoc| is set to true, |\includeonly| is applied to the child file
% and |\jobname| is set to the main file
% (for proper handling of |.aux| files):
%    \begin{macrocode}
\newcommand{\childdocmain}[1]
{
  \childdocdisable\childdocmain{}
  \if?#1?\else
    \begingroup
      \def\childdoctmp{#1}
      \ifx\childdoctmp\childdocname
        \def\childdoctmp{}
      \else
        \def\childdoctmp
        {
          \childdoctrue
          \includeonly{\childdocname}
          \def\childdocjob{#1}
          \def\jobname{#1}
        }
      \fi
      \expandafter
    \endgroup
    \childdoctmp
  \fi
}
%    \end{macrocode}

% \macro{\childdocof}
% The command |\childdocof| redirects
% compilation to the main file |#1|.
%    \begin{macrocode}
\newcommand{\childdocof}[1]
{
  \childdocdisable
  \childdoctrue
  \includeonly{\childdocname}
  \def\jobname{#1}
  \def\childdocjob{#1}
  \input{#1}
}
%    \end{macrocode}

% \macro{\childdocby}
% The command |\childdocby| ....
%    \begin{macrocode}
\newcommand{\childdocby}[2][]
{
  \childdocdisable
  \childdoctrue
  \childdocmanualtrue
  \if?#1?\else
    \def\jobname{#2}
  \fi
  \def\childdocjob{#2}
  \input{#2}
  \endinput
}
%    \end{macrocode}

% \macro{\childdocforward}
% The command |\childdocforward| redirects
% compilation to the main file or
% (if the optional argument is given) a child file.
% Parameters are set as if the main file
% or a child file starting with |\childdocof| was compiled.
% Then compilation is handed over to the main file:
%    \begin{macrocode}
\newcommand{\childdocforward}[2][]
{
  \begingroup
    \if?#1?
      \def\childdoctmp
      {
        \def\childdocname{#2}
        \def\childdocjob{#2}
        \def\jobname{#2}
        \input{#2}
        \endinput
      }
    \else
      \def\childdoctmp
      {
        \childdocdisable
        \def\childdocname{#2}
        \childdoctrue
        \includeonly{#2}
        \def\childdocjob{#1}
        \def\jobname{#1}
        \input{#1}
        \endinput
      }
    \fi
    \expandafter
  \endgroup
  \childdoctmp
}
%    \end{macrocode}

% \macro{\childdocforwardprefix}
% The command |\childdocforwardprefix| redirects
% compilation to the main or a child file by means of a pattern.
% The prefix |#1| in the current filename is replaced by |#2|
% and the suffix of the current filename is kept
% (it is assumed that the filename does not contain the substring `|~~~|'
% which is used as a delimiter).
% Compilation is handed over to the new file by |\childdocforward|:
%    \begin{macrocode}
\newcommand{\childdocforwardprefix}[3][]
{
  \begingroup
    \def\childdocextract #2##1~~~{\def\childdoctmp{\childdocforward[#1]{#3##1}}}
    \expandafter\childdocextract\childdocname~~~
    \expandafter
  \endgroup
  \childdoctmp
}
%    \end{macrocode}

% \macro{\childdoc}
% The deprecated macro |\childdoc| is a legacy version of |\childdocmain|:
%    \begin{macrocode}
\newcommand{\childdoc}{\childdocmain}
%    \end{macrocode}

% \macro{\childdocredirect}
% The deprecated macro |\childdocredirect| is a legacy version
% of |\childdocforward| and |\childdocforwardprefix|:
%    \begin{macrocode}
\newcommand{\childdocredirect}[2][]
{
  \begingroup
    \if?#1?
      \def\childdoctmp{\childdocforward{#2}}
    \else
      \def\childdoctmp{\childdocforwardprefix{#1}{#2}}
    \fi
    \expandafter
  \endgroup
  \childdoctmp
}
%    \end{macrocode}

%\iffalse
%</package>
%\fi
%
\endinput
|\\
|\childdocof{|\textit{main}|}|\\
\end{tabular}
\end{center}
at the top of every child file \textit{child}
which is included by |\include{|\textit{child}|}|
from within the main file
(or at least for those files to be compiled individually).
The argument \textit{main} must be the filename of the main file.

There are a couple of
considerations in setting up the main and child documents:

%%%%%%%%%%%%%%%%%%%%%%%%%%%%%%%%%%%%%%%%
\paragraph{Restrictions.}

Please note the following restrictions:
\begin{itemize}
\item
|\childdocmain| must be called with one argument \textit{main}
to ensure compatibility with earlier version of the package.
It must either be empty (|\childdocmain{}|)
or precisely match the filename of the main file in which it is specified.
See \secref{sec:detection} for further information.
\item
The filename \textit{main} must be specified without the |.tex| extension.
\item
The filename \textit{main} is case sensitive
(even in case-insensitive file systems)
due to internal string comparison.
\item
The argument \textit{main} should be fully expanded, it cannot be a macro.
\item
Subdirectories and special characters should be avoided in filenames.
\item
The command |\childdocmain{|\textit{main}|}| must be followed by a whitespace.
It should not be followed immediately by another command
or by a comment mark `|%|'.
This is because the \TeX{} parser reads the token immediately following
the argument of |\childdocmain| and puts it
at the beginning of every child section;
however, a white\-space is ignored.
\end{itemize}

%%%%%%%%%%%%%%%%%%%%%%%%%%%%%%%%%%%%%%%%
\paragraph{Content of Main File.}

It is advisable to place all content in the child files included by |\include|.
Any output contained in the main file will appear in all child documents
unless suppressed manually;
it cannot be suppressed automatically by the |\includeonly| directive
and thus should normally be avoided.
A method to include some content in the main file
by means of conditional processing is described in \secref{sec:conditional}.

%%%%%%%%%%%%%%%%%%%%%%%%%%%%%%%%%%%%%%%%
\paragraph{Page Numbering.}

When only a part of the document is compiled,
the appropriate numbering of pages
(as well as other status parameters)
is determined from the |.aux| files.
The latter contain information from previous passes.
However this information needs to propagate through
all intermediate child documents.
Therefore the page numbering in child documents may well
be inconsistent until the complete document is compiled at least once.

A useful (if unconventional) way to always ensure a consistent
page numbering is to restart the numbering in each child document
and denote the pages by `\textit{child}|.|\textit{page}'
where \textit{child} represents the chapter/section number of the child file.
This can be achieved by the command
|\numberwithin{page}{|\textit{child}|}|
of the \textsf{amsmath} package
where \textit{child} can be |chapter| or |section|
depending on the chosen structuring.
Alternatively, one can modify the macro |\thepage| appropriately
and reset the counter |page| at the start of each child file.

%%%%%%%%%%%%%%%%%%%%%%%%%%%%%%%%%%%%%%%%%%%%%%%%%%%%%%%%%%%%%%%%%%%%%%%%%%%%%%%%
\subsection{Conditional Processing}
\label{sec:conditional}

The package provides a mechanism to compile different versions
of a document. To customise the versions further some conditional processing
can come in handy to distinguish which version is being compiled.
The package provides two macros to describe the compilation context:

%%%%%%%%%%%%%%%%%%%%%%%%%%%%%%%%%%%%%%%%
\DescribeMacro{\ifchilddoc}
The conditional |\ifchilddoc| distinguishes between the compilation of
child documents and the main document:
%
\begin{center}
|\ifchilddoc |\textit{child-code}| |[|\||else |\textit{main-code}]| \||fi|
\end{center}

%%%%%%%%%%%%%%%%%%%%%%%%%%%%%%%%%%%%%%%%
\DescribeMacro{\childdocname}
\DescribeMacro{\childdocjob}
The macro |\childdocname| contains the filename (without extension)
of the main or child file being processed.
Note that |\childdocjob| will always contain the name of the main file.

%%%%%%%%%%%%%%%%%%%%%%%%%%%%%%%%%%%%%%%%
\paragraph{Title Page.}

Conditional processing can be used to include a title or banner page
in the main document when proper precautions are taken.
Importantly, the code in the main file should ensure that the page counter
(as well as other status parameters which are stored in the |.aux| files)
takes the same value after the conditional processing.
Otherwise the page numbers may take divergent values
depending on which part is compiled.

For example, a title page could be declared by:
%
\begin{center}
\begin{tabular}{l}
|\ifchilddoc\||else|\\
|\addtocounter{page}{-1}|\\
\textit{code for title page}\\
|\newpage|\\
|\||fi|
\end{tabular}
\end{center}
%
A banner page for the child documents can be generated by:
%
\begin{center}
\begin{tabular}{l}
|\ifchilddoc|\\
|\addtocounter{page}{-1}|\\
\textit{code for banner page}\\
|\newpage|\\
|\||fi|
\end{tabular}
\end{center}
%
Here one could write a message such as:
\begin{center}
|This is the part \childdocname{} of \childdocjob{}.|
\end{center}

%%%%%%%%%%%%%%%%%%%%%%%%%%%%%%%%%%%%%%%%%%%%%%%%%%%%%%%%%%%%%%%%%%%%%%%%%%%%%%%%
\subsection{Flags}
\label{sec:flags}

The package makes it easy to generate different versions
of the main or child documents.
To this end compilation flags can be defined
and assigned different default values.
They will be particularly useful in conjunction
with the forwarding mechanism described in \secref{sec:forward}.

For example, it may be useful to have a flag |\version|
which can be set to |draft| or |final|.
The document source will contain some conditional code
depending on the value of |\version|.
Suppose further, the flag should default to |final| for the main file
and to |draft| for child files
which is a natural assignment for editing the document.
This is achieved by placing the following code
in the preamble of the main document
(below the |\childdocmain| directive):
%
\begin{center}
\begin{tabular}{l}
|\ifchilddoc|\\
|\providecommand{\version}{draft}|\\
|\||else|\\
|\providecommand{\version}{final}|\\
|\||fi|
\end{tabular}
\end{center}
%
The definition by |\providecommand| makes sure
that previous definitions are not overwritten.
Further statements |\providecommand{\version}{...}|
can thus be added before the above code to override it.

For the main file, one might add a line
(between |\childdocmain| and the above block)
%
\begin{center}
|%\ifchilddoc\||else\providecommand{\version}{draft}\||fi|
\end{center}
%
which can be uncommented to produce a draft version.
Likewise one can add a line to the very top of a child file
(above the |\childdocof{|\textit{main}|}| directive)
%
\begin{center}
|%\providecommand{\version}{final}|
\end{center}
%
which can be uncommented to produce the final version of this child document.

%%%%%%%%%%%%%%%%%%%%%%%%%%%%%%%%%%%%%%%%%%%%%%%%%%%%%%%%%%%%%%%%%%%%%%%%%%%%%%%%
\subsection{Forwarding}
\label{sec:forward}

Different versions of the main or child documents
using compilation flags as described in \secref{sec:flags}
can be (permanently) stored in different files
for convenient compilation, viewing and distribution.
To this end, the package defines a command
to pass on compilation to a different file:

%%%%%%%%%%%%%%%%%%%%%%%%%%%%%%%%%%%%%%%%
\DescribeMacro{\childdocforward}
The command |\childdocforward| redirects processing to
another source file:
%
\begin{center}
\begin{tabular}{l}
|% \iffalse
%
% childdoc.dtx Copyright (C) 2017-2018 Niklas Beisert
%
% This work may be distributed and/or modified under the
% conditions of the LaTeX Project Public License, either version 1.3
% of this license or (at your option) any later version.
% The latest version of this license is in
%   http://www.latex-project.org/lppl.txt
% and version 1.3 or later is part of all distributions of LaTeX
% version 2005/12/01 or later.
%
% This work has the LPPL maintenance status `maintained'.
%
% The Current Maintainer of this work is Niklas Beisert.
%
% This work consists of the files childdoc.dtx and childdoc.ins
% and the derived files childdoc.def and cdocsamp.tex with
% cdocsch1.tex, cdocsch2.tex, cdocsdrf.tex, cdocsfn1.tex, cdocsfn2.tex.
%
%<package>\ifdefined\childdocmain\endinput\fi
%<package>\ProvidesFile{childdoc.def}[2018/12/30 v2.0 child document driver]
%<samplemain>\ProvidesFile{cdocsamp.tex}[2018/12/30 v2.0 sample for childdoc]
%<*driver>
%\ProvidesFile{childdoc.drv}[2018/12/30 v2.0 childdoc reference manual file]
\PassOptionsToClass{10pt,a4paper}{article}
\documentclass{ltxdoc}

\usepackage[margin=35mm]{geometry}
\usepackage{hyperref}
\usepackage{hyperxmp}
\usepackage[usenames]{color}

\hypersetup{colorlinks=true}
\hypersetup{pdfstartview=FitH}
\hypersetup{pdfpagemode=UseNone}
\hypersetup{pdfsource={}}
\hypersetup{pdflang={en-UK}}
\hypersetup{pdfcopyright={Copyright 2017-2018 Niklas Beisert.
  This work may be distributed and/or modified under the
  conditions of the LaTeX Project Public License, either version 1.3
  of this license or (at your option) any later version.}}
\hypersetup{pdflicenseurl={http://www.latex-project.org/lppl.txt}}
\hypersetup{pdfcontactaddress={ETH Zurich, ITP, HIT K,
  Wolfgang-Pauli-Strasse 27}}
\hypersetup{pdfcontactpostcode={8093}}
\hypersetup{pdfcontactcity={Zurich}}
\hypersetup{pdfcontactcountry={Switzerland}}
\hypersetup{pdfcontactemail={nbeisert@itp.phys.ethz.ch}}
\hypersetup{pdfcontacturl={http://people.phys.ethz.ch/\xmptilde nbeisert/}}

\newcommand{\secref}[1]{\hyperref[#1]{section \ref*{#1}}}

\parskip1ex
\parindent0pt
\let\olditemize\itemize
\def\itemize{\olditemize\parskip0pt}

\begin{document}

\title{The \textsf{childdoc} Package}
\hypersetup{pdftitle={The childdoc Package}}
\author{Niklas Beisert\\[2ex]
  Institut f\"ur Theoretische Physik\\
  Eidgen\"ossische Technische Hochschule Z\"urich\\
  Wolfgang-Pauli-Strasse 27, 8093 Z\"urich, Switzerland\\[1ex]
  \href{mailto:nbeisert@itp.phys.ethz.ch}
  {\texttt{nbeisert@itp.phys.ethz.ch}}}
\hypersetup{pdfauthor={Niklas Beisert}}
\hypersetup{pdfsubject={Manual for the LaTeX2e Package childdoc}}
\date{30 December 2018, \textsf{v2.0}}
\maketitle

\begin{abstract}\noindent
\textsf{childdoc} is a \LaTeXe{} package
that enables the direct compilation
of document sections included by |\include|
to individual files.
\end{abstract}

\begingroup
\parskip0ex
\tableofcontents
\endgroup

%%%%%%%%%%%%%%%%%%%%%%%%%%%%%%%%%%%%%%%%%%%%%%%%%%%%%%%%%%%%%%%%%%%%%%%%%%%%%%%%
%%%%%%%%%%%%%%%%%%%%%%%%%%%%%%%%%%%%%%%%%%%%%%%%%%%%%%%%%%%%%%%%%%%%%%%%%%%%%%%%
\section{Introduction}

\LaTeX{} provides a mechanism to structure a large document (such as a book)
into a main file and several child files (containing the chapters)
using the |\include| command.
This mechanism is beneficial for documents
which span hundreds of pages in order to
make the source file(s) more manageable.
Moreover, compilation can be restricted to
selected child files by means of the |\includeonly| command.
The latter feature can be used to reduce the compilation time while editing
(this was significantly more useful in the earlier days of \LaTeX{})
or to generate a smaller document which is easier to navigate.
Another application of |\includeonly| is to generate
documents consisting of selected parts of the complete document.

However, there are a few drawbacks of the plain |\include| mechanism:
\begin{itemize}
\item
The child files cannot be compiled on their own,
they can only be compiled via the main file.
A naive editing environment
(such as a text editor with an option
to have the current file processed by \LaTeX)
may require one to switch to the main file before compiling;
attempting to compile the child file produces errors.
\item
The main file must be modified (each time)
to adjust the |\includeonly| command
to the present needs. This easily leaves the main file in a messy state.
\item
The generated document will always carry the filename
of the main document. This is inconvenient if
several child files are to be compiled and
to be kept for distribution.
\end{itemize}

The present package provides a simple interface
to make child files individually compilable by \LaTeX{}.
Compiling a child file then has the same effect as compiling
the main file with an |\includeonly| command
to select the appropriate child.
Moreover the generated document will carry the name of the child
rather than the main file.
This resolves all three above issues.

This feature is meant to make the editing of books,
thesis documents and lecture notes somewhat more convenient.
However, the package can also be used efficiently for
composing a series of documents (such as exercise sheets)
which are typically distributed individually.
It then assists the author in generating the individual documents
(potentially in different versions)
as well as a document containing the collected series.
Another application is in developing style files
or other kinds of included material
where compilation of the style file could redirect
to a sample or test file.

%%%%%%%%%%%%%%%%%%%%%%%%%%%%%%%%%%%%%%%%%%%%%%%%%%%%%%%%%%%%%%%%%%%%%%%%%%%%%%%%
%%%%%%%%%%%%%%%%%%%%%%%%%%%%%%%%%%%%%%%%%%%%%%%%%%%%%%%%%%%%%%%%%%%%%%%%%%%%%%%%
\section{Usage}

First of all, the package \textsf{childdoc} is \emph{not} a standard
\LaTeXe{} |.sty| style file! Therefore it needs to be invoked in
a non-standard way.

%%%%%%%%%%%%%%%%%%%%%%%%%%%%%%%%%%%%%%%%%%%%%%%%%%%%%%%%%%%%%%%%%%%%%%%%%%%%%%%%
\subsection{Included Files}
\label{sec:include}

%%%%%%%%%%%%%%%%%%%%%%%%%%%%%%%%%%%%%%%%
\DescribeMacro{\childdocmain}
To use the package, add the commands
\begin{center}
\begin{tabular}{l}
|\input{childdoc.def}|\\
|\childdocmain{}|\\
\end{tabular}
\end{center}
at the very top of the main \LaTeX{} file,
in particular \emph{before} the |\documentclass| statement!
The argument of |\childdocmain| should be left empty
(but it must be present).

%%%%%%%%%%%%%%%%%%%%%%%%%%%%%%%%%%%%%%%%
\DescribeMacro{\childdocof}
Furthermore, add the commands
\begin{center}
\begin{tabular}{l}
|\input{childdoc.def}|\\
|\childdocof{|\textit{main}|}|\\
\end{tabular}
\end{center}
at the top of every child file \textit{child}
which is included by |\include{|\textit{child}|}|
from within the main file
(or at least for those files to be compiled individually).
The argument \textit{main} must be the filename of the main file.

There are a couple of
considerations in setting up the main and child documents:

%%%%%%%%%%%%%%%%%%%%%%%%%%%%%%%%%%%%%%%%
\paragraph{Restrictions.}

Please note the following restrictions:
\begin{itemize}
\item
|\childdocmain| must be called with one argument \textit{main}
to ensure compatibility with earlier version of the package.
It must either be empty (|\childdocmain{}|)
or precisely match the filename of the main file in which it is specified.
See \secref{sec:detection} for further information.
\item
The filename \textit{main} must be specified without the |.tex| extension.
\item
The filename \textit{main} is case sensitive
(even in case-insensitive file systems)
due to internal string comparison.
\item
The argument \textit{main} should be fully expanded, it cannot be a macro.
\item
Subdirectories and special characters should be avoided in filenames.
\item
The command |\childdocmain{|\textit{main}|}| must be followed by a whitespace.
It should not be followed immediately by another command
or by a comment mark `|%|'.
This is because the \TeX{} parser reads the token immediately following
the argument of |\childdocmain| and puts it
at the beginning of every child section;
however, a white\-space is ignored.
\end{itemize}

%%%%%%%%%%%%%%%%%%%%%%%%%%%%%%%%%%%%%%%%
\paragraph{Content of Main File.}

It is advisable to place all content in the child files included by |\include|.
Any output contained in the main file will appear in all child documents
unless suppressed manually;
it cannot be suppressed automatically by the |\includeonly| directive
and thus should normally be avoided.
A method to include some content in the main file
by means of conditional processing is described in \secref{sec:conditional}.

%%%%%%%%%%%%%%%%%%%%%%%%%%%%%%%%%%%%%%%%
\paragraph{Page Numbering.}

When only a part of the document is compiled,
the appropriate numbering of pages
(as well as other status parameters)
is determined from the |.aux| files.
The latter contain information from previous passes.
However this information needs to propagate through
all intermediate child documents.
Therefore the page numbering in child documents may well
be inconsistent until the complete document is compiled at least once.

A useful (if unconventional) way to always ensure a consistent
page numbering is to restart the numbering in each child document
and denote the pages by `\textit{child}|.|\textit{page}'
where \textit{child} represents the chapter/section number of the child file.
This can be achieved by the command
|\numberwithin{page}{|\textit{child}|}|
of the \textsf{amsmath} package
where \textit{child} can be |chapter| or |section|
depending on the chosen structuring.
Alternatively, one can modify the macro |\thepage| appropriately
and reset the counter |page| at the start of each child file.

%%%%%%%%%%%%%%%%%%%%%%%%%%%%%%%%%%%%%%%%%%%%%%%%%%%%%%%%%%%%%%%%%%%%%%%%%%%%%%%%
\subsection{Conditional Processing}
\label{sec:conditional}

The package provides a mechanism to compile different versions
of a document. To customise the versions further some conditional processing
can come in handy to distinguish which version is being compiled.
The package provides two macros to describe the compilation context:

%%%%%%%%%%%%%%%%%%%%%%%%%%%%%%%%%%%%%%%%
\DescribeMacro{\ifchilddoc}
The conditional |\ifchilddoc| distinguishes between the compilation of
child documents and the main document:
%
\begin{center}
|\ifchilddoc |\textit{child-code}| |[|\||else |\textit{main-code}]| \||fi|
\end{center}

%%%%%%%%%%%%%%%%%%%%%%%%%%%%%%%%%%%%%%%%
\DescribeMacro{\childdocname}
\DescribeMacro{\childdocjob}
The macro |\childdocname| contains the filename (without extension)
of the main or child file being processed.
Note that |\childdocjob| will always contain the name of the main file.

%%%%%%%%%%%%%%%%%%%%%%%%%%%%%%%%%%%%%%%%
\paragraph{Title Page.}

Conditional processing can be used to include a title or banner page
in the main document when proper precautions are taken.
Importantly, the code in the main file should ensure that the page counter
(as well as other status parameters which are stored in the |.aux| files)
takes the same value after the conditional processing.
Otherwise the page numbers may take divergent values
depending on which part is compiled.

For example, a title page could be declared by:
%
\begin{center}
\begin{tabular}{l}
|\ifchilddoc\||else|\\
|\addtocounter{page}{-1}|\\
\textit{code for title page}\\
|\newpage|\\
|\||fi|
\end{tabular}
\end{center}
%
A banner page for the child documents can be generated by:
%
\begin{center}
\begin{tabular}{l}
|\ifchilddoc|\\
|\addtocounter{page}{-1}|\\
\textit{code for banner page}\\
|\newpage|\\
|\||fi|
\end{tabular}
\end{center}
%
Here one could write a message such as:
\begin{center}
|This is the part \childdocname{} of \childdocjob{}.|
\end{center}

%%%%%%%%%%%%%%%%%%%%%%%%%%%%%%%%%%%%%%%%%%%%%%%%%%%%%%%%%%%%%%%%%%%%%%%%%%%%%%%%
\subsection{Flags}
\label{sec:flags}

The package makes it easy to generate different versions
of the main or child documents.
To this end compilation flags can be defined
and assigned different default values.
They will be particularly useful in conjunction
with the forwarding mechanism described in \secref{sec:forward}.

For example, it may be useful to have a flag |\version|
which can be set to |draft| or |final|.
The document source will contain some conditional code
depending on the value of |\version|.
Suppose further, the flag should default to |final| for the main file
and to |draft| for child files
which is a natural assignment for editing the document.
This is achieved by placing the following code
in the preamble of the main document
(below the |\childdocmain| directive):
%
\begin{center}
\begin{tabular}{l}
|\ifchilddoc|\\
|\providecommand{\version}{draft}|\\
|\||else|\\
|\providecommand{\version}{final}|\\
|\||fi|
\end{tabular}
\end{center}
%
The definition by |\providecommand| makes sure
that previous definitions are not overwritten.
Further statements |\providecommand{\version}{...}|
can thus be added before the above code to override it.

For the main file, one might add a line
(between |\childdocmain| and the above block)
%
\begin{center}
|%\ifchilddoc\||else\providecommand{\version}{draft}\||fi|
\end{center}
%
which can be uncommented to produce a draft version.
Likewise one can add a line to the very top of a child file
(above the |\childdocof{|\textit{main}|}| directive)
%
\begin{center}
|%\providecommand{\version}{final}|
\end{center}
%
which can be uncommented to produce the final version of this child document.

%%%%%%%%%%%%%%%%%%%%%%%%%%%%%%%%%%%%%%%%%%%%%%%%%%%%%%%%%%%%%%%%%%%%%%%%%%%%%%%%
\subsection{Forwarding}
\label{sec:forward}

Different versions of the main or child documents
using compilation flags as described in \secref{sec:flags}
can be (permanently) stored in different files
for convenient compilation, viewing and distribution.
To this end, the package defines a command
to pass on compilation to a different file:

%%%%%%%%%%%%%%%%%%%%%%%%%%%%%%%%%%%%%%%%
\DescribeMacro{\childdocforward}
The command |\childdocforward| redirects processing to
another source file:
%
\begin{center}
\begin{tabular}{l}
|\input{childdoc.def}|\\
|\childdocforward[|\textit{main}|]{|\textit{dest}|}|\\
\end{tabular}
\end{center}
%
The argument \textit{dest} is the destination file
(without extension).
It should be the main file or one of the child files.
Note that further \textsf{childdoc} directives
such as |\childdocof| and |\childdocforward|
in the indicated file will be processed in this form.
The optional argument \textit{main}
passes on directly to the main file \textit{main}
while pretending to compile the child \textit{dest}.
This form behaves as if \textit{dest}
issues |\childdocof{|\textit{main}|}| right away,
and no further \textsf{childdoc} directives will be processed.

%%%%%%%%%%%%%%%%%%%%%%%%%%%%%%%%%%%%%%%%
\DescribeMacro{\...prefix}
In the alternative form |\childdocforwardprefix|,
%
\begin{center}
\begin{tabular}{l}
|\input{childdoc.def}|\\
|\childdocforwardprefix[|\textit{main}|]{|\textit{prefix}|}{|\textit{dest}|}|
\end{tabular}
\end{center}
%
the destination file is determined by a pattern
depending on the current file:
To make this work, the current file must be called
`{\textit{prefix}\hspace{0.2em}\textit{suffix}}'
with \textit{prefix} matching precisely the argument.
Processing is then passed on to the file
`{\textit{dest}\hspace{0.2em}\textit{suffix}}'.
Surely, the same effect is achieved by
directly specifying the
argument `{\textit{dest}\hspace{0.2em}\textit{suffix}}'
in the first form.
However, that requires to set up a different file
for each child. With the alternative form of the command
all these files can have exactly the same content
which simplifies setting them up and maintaining them.

For example, the following file |draft.tex|
with a compilation flag |\version| as described in \secref{sec:flags}
compiles the main document as a draft:
%
\begin{center}
\begin{tabular}{l}
|\def\version{draft}|\\
|\input{childdoc.def}|\\
|\childdocforward{|\textit{main}|}|
\end{tabular}
\end{center}
%
Likewise, the following files |final|\textit{nn}|.tex|
compile the final version of the child document
|child|\textit{nn}|.tex|:
%
\begin{center}
\begin{tabular}{l}
|\def\version{final}|\\
|\input{childdoc.def}|\\
|\childdocforwardprefix{final}{child}|
\end{tabular}
\end{center}
%

Note that when several versions of a main file and/or of each child file
are to be generated, it may be convenient to set up a |Makefile| or
shell script to automatise the process.

%%%%%%%%%%%%%%%%%%%%%%%%%%%%%%%%%%%%%%%%%%%%%%%%%%%%%%%%%%%%%%%%%%%%%%%%%%%%%%%%
\subsection{Command Line Processing}
\label{sec:commandline}

The effect of redirection files can also be achieved by invoking
the \LaTeX{} compiler with a more elaborate command line.
Most conveniently this should be done as part
of a shell script or a |Makefile|.

When using \textsf{childdoc} in the main file, the following
command lines effectively perform a redirection
(note that depending on the shell being used,
backslashes may have to be doubled: `|\|' $\to$ `|\\|'):
%
\begin{center}
|... -jobname "|\textit{target}|" |\\|"|[\textit{flags}]%
|\input{childdoc.def}\childdocforward[|\textit{main}|]{|\textit{dest}|}"|
\end{center}
%
Here \textit{target} is the name of the output file,
\textit{main} is the name of the main file
and \textit{dest} is the name of the main or child file to be processed
(all filenames without extensions).
The optional argument \textit{main} can be omitted
if \textit{main} matches \textit{dest}.
Optionally, compilation \textit{flags} can be defined via |\def| commands.
This command line makes the \TeX{} engine believe
it is compiling the file \textit{target}
whose content is specified as the latter parameter.
The provided code then forwards the processing to
\textit{main} or \textit{dest} as described in \secref{sec:forward}.

%%%%%%%%%%%%%%%%%%%%%%%%%%%%%%%%%%%%%%%%%%%%%%%%%%%%%%%%%%%%%%%%%%%%%%%%%%%%%%%%
\subsection{Include by Input}
\label{sec:input}

Including child documents by |\include| has some restrictions by design.
Most notably, the content of a child document always occupies
its own set of pages; pages cannot be shared between child documents.
Usually, this behaviour makes perfect sense
because each child document contain an essential part of the document.
However, in some situations it may be desirable to compose
a document from a collection of parts
without having mandatory page breaks between then.
For this case, the package
provides a mechanism to include parts
by |\input| which can also be processed individually.
However, by construction this mechanism
requires manual handling of the content to be output.

%%%%%%%%%%%%%%%%%%%%%%%%%%%%%%%%%%%%%%%%
\DescribeMacro{\ifchilddocmanual}
The main file should be prepared as usual, see \secref{sec:include}.
However, the document body must make a distinction
between processing of an individual part and of the main document, e.g.:
%
\begin{center}
\begin{tabular}{l}
|\ifchilddocmanual|\\
|\input{\childdocname}|\\
|\||else|\\
\textit{document body with }|\input{|\textit{part}|}|\\
|\||fi|
\end{tabular}
\end{center}
%
The conditional |\ifchilddocmanual| is true whenever
a part to be included by |\input| is being compiled,
and the name of the part is stored in |\childdocname|.

%%%%%%%%%%%%%%%%%%%%%%%%%%%%%%%%%%%%%%%%
\DescribeMacro{\childdocby}
Each part to be included by |\input| should start with:
%
\begin{center}
\begin{tabular}{l}
|\input{childdoc.def}|\\
|\childdocby{|\textit{main}|}|\\
\end{tabular}
\end{center}
%
The directive |\childdocby| is similar to |\childdocof|
described in \secref{sec:include},
but the subsequent selection of content must be done manually.
To that end, both |\ifchilddoc| and |\ifchilddocmanual|
will be true upon processing of a part,
and the name of the part is stored in |\childdocname|.
Note that |\jobname| will be set to the filename of the current part
so that each part receives an individual |.aux| file
that does not interfere with the |.aux| file(s) of the main document.
This behaviour can be altered by the alternative form
|\childdocby[*]{|\textit{main}|}| (with a non-empty optional argument)
which uses the |.aux| file of the main document
by setting |\jobname| to \textit{main}.

%%%%%%%%%%%%%%%%%%%%%%%%%%%%%%%%%%%%%%%%%%%%%%%%%%%%%%%%%%%%%%%%%%%%%%%%%%%%%%%%
\subsection{Driver Development}
\label{sec:driver}

The \textsf{childdoc} mechanism can also be use for the development
of definition files such as \LaTeX{} styles or classes.
This case differs from the above setup with multiple parts
included by |\include| in that no |\includeonly| should be invoked.
This can be achieved by starting the include file
(before |\ProvidesPackage|) with:
%
\begin{center}
\begin{tabular}{l}
|\input{childdoc.def}|\\
|\childdocforward{|\textit{main}|}|\\
\end{tabular}
\end{center}
%
or alternatively with:
%
\begin{center}
\begin{tabular}{l}
|\input{childdoc.def}|\\
|\childdocby{|\textit{main}|}|\\
\end{tabular}
\end{center}
%
Both forms have slightly different effects as described above.
The main file is prepared as usual, see \secref{sec:include}.

%%%%%%%%%%%%%%%%%%%%%%%%%%%%%%%%%%%%%%%%%%%%%%%%%%%%%%%%%%%%%%%%%%%%%%%%%%%%%%%%
\subsection{Legacy Detection}
\label{sec:detection}

The directive |\childdocmain| in the main file can detect
whether the complete document or merely a child is to be compiled
even without using the directive |\childdocof|.
This method is deprecated because it is less robust
and there is no compelling reason to use it;
it is merely provided for backward compatibility
and it may be removed in future versions.

If the detection mechanism is to be used,
it is mandatory to correctly specify
the filename of the main file as the argument of |\childdocmain|:
%
\begin{center}
\begin{tabular}{l}
|\input{childdoc.def}|\\
|\childdocmain{|\textit{main}|}|\\
\end{tabular}
\end{center}
%
If |\jobname| does not match the argument \textit{main} of |\childdocmain|,
it is assumed that |\jobname| points to the child file to be compiled.
When using |\childdocmain| with the main file specified as argument,
it suffices to start a child file
with just |\input{|\textit{main}|}|
without loading of the package and using |\childdocof|.
If instead all processing is done
with the appropriate \textsf{childdoc} directives,
the argument of \textit{main} of |\childdocmain| can be empty.

An alternative version of the command line processing described
in \secref{sec:commandline} using the detection mechanism reads:
%
\begin{center}
|... -jobname "|\textit{target}|" "|[\textit{flags}]%
[|\def\jobname{|\textit{dest}|}|]|\input{|\textit{main}|}"|
\end{center}

%%%%%%%%%%%%%%%%%%%%%%%%%%%%%%%%%%%%%%%%%%%%%%%%%%%%%%%%%%%%%%%%%%%%%%%%%%%%%%%%
\subsection{Manual Code}
\label{sec:manual}

In case one cannot be certain whether the definitions file |childdoc.def|
is installed on the target \TeX{} distribution
and one prefers not to ship it,
it is conceivable to paste a few relevant commands into the sources.

To that end, drop all statements |\input{childdoc.def}|
and perform the replacements as outlined below.
Instead of |\childdocmain{|\textit{main}|}| add the following code
to the top of the main file:
%
\begin{center}
\begin{tabular}{l}
|\||ifdefined\childdocname\endinput\||fi\newif\ifchilddoc|\\
|\edef\childdocname{\scantokens\expandafter{\jobname\noexpand}}|\\
|\def\childdocmain{|\textit{main}|}\||ifx\childdocmain\childdocname\||else|\\
|\childdoctrue\includeonly{\childdocname}\let\jobname\childdocmain\||fi|\\
\end{tabular}
\end{center}
%
Instead of |\childdocof{|\textit{main}|}| just include the main file
at the top of each child file:
%
\begin{center}
|\input{|\textit{main}|}|
\end{center}
%
A simple redirection |\childdocforward{|\textit{dest}|}| is achieved by:
%
\begin{center}
|\def\jobname{|\textit{dest}|}\input{\jobname}|
\end{center}
%
The redirection with prefix
|\childdocforwardprefix[|\textit{prefix}|]{|\textit{dest}|}|
is accomplished by:
%
\begin{center}
\begin{tabular}{l}
|{\edef\jobname{\scantokens\expandafter{\jobname\noexpand}}|\\
|\def\redirectjob |\textit{prefix}|#1~~~{\gdef\jobname{|\textit{dest}|#1}}|\\
|\expandafter\redirectjob\jobname~~~}\input{\jobname}|
\end{tabular}
\end{center}

In an alternative approach,
child documents can be compiled by a specific command line
without additional code or specific definitions:
%
\begin{center}
|... -jobname "|\textit{target}|" "|[\textit{flags}]%
|\includeonly{|\textit{dest}|}\input{|\textit{main}|}"|
\end{center}
%

%%%%%%%%%%%%%%%%%%%%%%%%%%%%%%%%%%%%%%%%%%%%%%%%%%%%%%%%%%%%%%%%%%%%%%%%%%%%%%%%
%%%%%%%%%%%%%%%%%%%%%%%%%%%%%%%%%%%%%%%%%%%%%%%%%%%%%%%%%%%%%%%%%%%%%%%%%%%%%%%%
\section{Information}

%%%%%%%%%%%%%%%%%%%%%%%%%%%%%%%%%%%%%%%%%%%%%%%%%%%%%%%%%%%%%%%%%%%%%%%%%%%%%%%%
\subsection{Copyright}

Copyright \copyright{} 2017--2018 Niklas Beisert

This work may be distributed and/or modified under the
conditions of the \LaTeX{} Project Public License, either version 1.3
of this license or (at your option) any later version.
The latest version of this license is in
  \url{http://www.latex-project.org/lppl.txt}
and version 1.3 or later is part of all distributions of \LaTeX{}
version 2005/12/01 or later.

This work has the LPPL maintenance status `maintained'.

The Current Maintainer of this work is Niklas Beisert.

This work consists of the files |README.txt|, |childdoc.ins| and |childdoc.dtx|
as well as the derived files |childdoc.def|, |cdocsamp.tex|
with |cdocsch1.tex|, |cdocsch2.tex|, |cdocspt3.tex|, |cdocspt4.tex|,
|cdocsdrf.tex|, |cdocsfn1.tex|, |cdocsfn2.tex|
as well as |childdoc.pdf|.

%%%%%%%%%%%%%%%%%%%%%%%%%%%%%%%%%%%%%%%%%%%%%%%%%%%%%%%%%%%%%%%%%%%%%%%%%%%%%%%%
\subsection{Files and Installation}

The package consists of the files:
%
\begin{center}
\begin{tabular}{ll}
    |README.txt|   & readme file \\
    |childdoc.ins| & installation file \\
    |childdoc.dtx| & source file \\
    |childdoc.def| & definition file \\
    |cdocsamp.tex| & sample main file \\
    |cdocsch1.tex| & sample include file \\
    |cdocsch2.tex| & sample include file \\
    |cdocspt3.tex| & sample part file \\
    |cdocspt4.tex| & sample part file \\
    |cdocsdrf.tex| & sample redirection file \\
    |cdocsfn1.tex| & sample redirection file \\
    |cdocsfn2.tex| & sample redirection file \\
    |childdoc.pdf| & manual
\end{tabular}
\end{center}
%
The distribution consists of the files
|README.txt|, |childdoc.ins| and |childdoc.dtx|.
%
\begin{itemize}
\item
Run (pdf)\LaTeX{} on |childdoc.dtx|
to compile the manual |childdoc.pdf| (this file).
\item
Run \LaTeX{} on |childdoc.ins| to create the definitions file |childdoc.def|
and the sample |cdocsamp.tex| with include files
|cdocsch1.tex|, |cdocsch2.tex|, |cdocspt3.tex|, |cdocspt4.tex|,
|cdocsdrf.tex|, |cdocsfn1.tex|, |cdocsfn2.tex|.
Then copy the file |childdoc.def| to an appropriate directory of your \LaTeX{}
distribution, e.g.\ \textit{texmf-root}|/tex/latex/childdoc|.
\end{itemize}

%%%%%%%%%%%%%%%%%%%%%%%%%%%%%%%%%%%%%%%%%%%%%%%%%%%%%%%%%%%%%%%%%%%%%%%%%%%%%%%%
\subsection{Related CTAN Packages}

There are several other packages which offer a similar functionality:
%
\begin{itemize}
\item
The packages
\href{http://ctan.org/pkg/docmute}{\textsf{docmute}},
\href{http://ctan.org/pkg/includex}{\textsf{includex}} and
\href{http://ctan.org/pkg/standalone}{\textsf{standalone}}
provide commands to include only the document body of
a child file thus allowing both files to be compiled individually.
\item
The packages \href{http://ctan.org/pkg/subdocs}{\textsf{subdocs}}
and \href{http://ctan.org/pkg/subfiles}{\textsf{subfiles}}
provide structures in which the main and child documents can be
encapsulated and allowing them to be compiled individually.
The inclusion mechanism is different from the conventional |\include|.
\item
The package \href{http://ctan.org/pkg/combine}{\textsf{combine}}
is an elaborate solution to combine several documents into one.
\end{itemize}
%
See also the CTAN topic \href{http://ctan.org/topic/subdocs}{\textsf{subdocs}}
for further related packages.
The present package differs from the above solutions in that
a document structure constructed with the conventional |\include| mechanism
just needs two extra commands at the top of every file
such that all constituent files can be compiled individually.

%%%%%%%%%%%%%%%%%%%%%%%%%%%%%%%%%%%%%%%%%%%%%%%%%%%%%%%%%%%%%%%%%%%%%%%%%%%%%%%%
%\subsection{Feature Suggestions}
%
%The following is a list of features which may be useful for future
%versions of this package:
%%
%\begin{itemize}
%\item
%\ldots
%\end{itemize}

%%%%%%%%%%%%%%%%%%%%%%%%%%%%%%%%%%%%%%%%%%%%%%%%%%%%%%%%%%%%%%%%%%%%%%%%%%%%%%%%
\subsection{Revision History}

%%%%%%%%%%%%%%%%%%%%%%%%%%%%%%%%%%%%%%%%
\paragraph{v2.0:} 2018/12/30

\begin{itemize}
\item
immediate forward processing
\item
added |\childdocby| mechanism
\item
manual restructured
\end{itemize}

%%%%%%%%%%%%%%%%%%%%%%%%%%%%%%%%%%%%%%%%
\paragraph{v1.6:} 2018/01/17

\begin{itemize}
\item
application for development of include files
\item
corrections to manual
\end{itemize}

%%%%%%%%%%%%%%%%%%%%%%%%%%%%%%%%%%%%%%%%
\paragraph{v1.5:} 2017/05/21

\begin{itemize}
\item
more complete structuring introduced
\item
|\childdocof| introduced
\item
|\childdoc| renamed to |\childdocmain|
\item
|\childredirect| renamed to |\childdocforward| and |\childdocforwardprefix|
and functionality expanded
\end{itemize}

%%%%%%%%%%%%%%%%%%%%%%%%%%%%%%%%%%%%%%%%
\paragraph{v1.0:} 2017/04/27

\begin{itemize}
\item
manual and install package
\item
first version published on CTAN
\end{itemize}

%%%%%%%%%%%%%%%%%%%%%%%%%%%%%%%%%%%%%%%%
\paragraph{v0.6:} 2017/04/26

\begin{itemize}
\item
redirection mechanism added
\end{itemize}

%%%%%%%%%%%%%%%%%%%%%%%%%%%%%%%%%%%%%%%%
\paragraph{v0.5:} 2017/04/26

\begin{itemize}
\item
functionality in definition file
\end{itemize}


%%%%%%%%%%%%%%%%%%%%%%%%%%%%%%%%%%%%%%%%%%%%%%%%%%%%%%%%%%%%%%%%%%%%%%%%%%%%%%%%
%%%%%%%%%%%%%%%%%%%%%%%%%%%%%%%%%%%%%%%%%%%%%%%%%%%%%%%%%%%%%%%%%%%%%%%%%%%%%%%%
%%%%%%%%%%%%%%%%%%%%%%%%%%%%%%%%%%%%%%%%%%%%%%%%%%%%%%%%%%%%%%%%%%%%%%%%%%%%%%%%
\appendix

\settowidth\MacroIndent{\rmfamily\scriptsize 000\ }

 \DocInput{childdoc.dtx}

\end{document}
%</driver>
% \fi
%
% %%%%%%%%%%%%%%%%%%%%%%%%%%%%%%%%%%%%%%%%%%%%%%%%%%%%%%%%%%%%%%%%%%%%%%%%%%%%%%
% %%%%%%%%%%%%%%%%%%%%%%%%%%%%%%%%%%%%%%%%%%%%%%%%%%%%%%%%%%%%%%%%%%%%%%%%%%%%%%
% \section{Sample}
%\iffalse
%<*samplemain>
%\fi
%
% The following presents a sample document
% with two chapters, two parts, a title page,
% a compile flag as well as three forwarding files to set the flag.
% It consists of eight |.tex| files:
% \begin{center}
% \begin{tabular}{ll}
% |cdocsamp.tex|&main file\\
% |cdocsch1.tex|&include file for chapter 1\\
% |cdocsch2.tex|&include file for chapter 2\\
% |cdocspt3.tex|&include file for part 3\\
% |cdocspt4.tex|&include file for part 4\\
% |cdocsdrf.tex|&forwarding file for main file in draft mode\\
% |cdocsfi1.tex|&forwarding file for final version of chapter 1\\
% |cdocsfi2.tex|&forwarding file for final version of chapter 2\\
% \end{tabular}
% \end{center}
% Each of the eight files can be compiled directly by the \LaTeX{} compiler.
%
% %%%%%%%%%%%%%%%%%%%%%%%%%%%%%%%%%%%%%%
% \paragraph{Main File.}
%
% The main file is called |cdocsamp.tex|.
%
% Load the \textsf{childdoc} definitions and
% declare the filename for the main document:
%    \begin{macrocode}
\input{childdoc.def}
\childdocmain{}
%    \end{macrocode}

% Optional override for |\version| flag:
%    \begin{macrocode}
%%\ifchilddoc\else\providecommand{\version}{draft}\fi
%    \end{macrocode}

% Define the default values for the |\version| flag
% (|final| for the main file and |draft| for childs):
%    \begin{macrocode}
\ifchilddoc
\providecommand{\version}{draft}
\else
\providecommand{\version}{final}
\fi
%    \end{macrocode}

% Load the standard document class:
%    \begin{macrocode}
\documentclass[12pt]{article}
%    \end{macrocode}

% Start the document body:
%    \begin{macrocode}
\begin{document}
%    \end{macrocode}

% Declare a title page.
% Print title, part of document being processed and version flag:
%    \begin{macrocode}
\addtocounter{page}{-1}
\begin{center}
{\LARGE\bfseries{}childdoc example\par}
\vspace{1cm}
\ifchilddoc
\ifchilddocmanual part\else chapter\fi:
`\childdocname' of `\childdocjob'\par
\else
main document: `\childdocjob'\par
\fi
version: \version\par
\end{center}
\newpage
%    \end{macrocode}

% Manually include selected file,
% otherwise process as usual:
%    \begin{macrocode}
\ifchilddocmanual
\section*{part `\childdocname'}
\input{\childdocname}
\else
%    \end{macrocode}

% Include the two chapters:
%    \begin{macrocode}
\include{cdocsch1}
\include{cdocsch2}
%    \end{macrocode}

% Include the two parts unless only chapters should be displayed:
%    \begin{macrocode}
\ifchilddoc\else
\section{part three}
\input{cdocspt3}
\section{part four}
\input{cdocspt4}
\fi
%    \end{macrocode}

% Process as usual until here:
%    \begin{macrocode}
\fi
%    \end{macrocode}

% End of document body:
%    \begin{macrocode}
\end{document}
%    \end{macrocode}
%\iffalse
%</samplemain>
%\fi
%
% %%%%%%%%%%%%%%%%%%%%%%%%%%%%%%%%%%%%%%
% \paragraph{Chapter Include Files.}
%
% The include files are called |cdocsch1.tex| and |cdocsch2.tex|.
%
%\iffalse
%<*samplechap1|samplechap2>
%\fi

% Optional override for |\version| flag:
%    \begin{macrocode}
%%\providecommand{\version}{final}
%    \end{macrocode}

% Include the main document:
%    \begin{macrocode}
\input{childdoc.def}
\childdocof{cdocsamp}
%    \end{macrocode}

%\iffalse
%</samplechap1|samplechap2>
%\fi
%
%\iffalse
%<*samplechap1>
%\fi
% Some text for chapter 1:
%    \begin{macrocode}
\section{one}
some text in chapter one
%    \end{macrocode}

%\iffalse
%</samplechap1>
%\fi
% Some text for chapter 2:
%\iffalse
%<*samplechap2>
%\fi
%    \begin{macrocode}
\section{two}
more text in chapter two
%    \end{macrocode}

%\iffalse
%</samplechap2>
%\fi
%
% %%%%%%%%%%%%%%%%%%%%%%%%%%%%%%%%%%%%%%
% \paragraph{Part Include Files.}
%
% The include files are called |cdocspt3.tex| and |cdocspt4.tex|.
%
%\iffalse
%<*samplepart3|samplepart4>
%\fi

% Optional override for |\version| flag:
%    \begin{macrocode}
%%\providecommand{\version}{final}
%    \end{macrocode}

% Include the main document:
%    \begin{macrocode}
\input{childdoc.def}
\childdocby{cdocsamp}
%    \end{macrocode}

%\iffalse
%</samplepart3|samplepart4>
%\fi
%
%\iffalse
%<*samplepart3>
%\fi
% Some text for part 3:
%    \begin{macrocode}
some text in part three
%    \end{macrocode}

%\iffalse
%</samplepart3>
%\fi
% Some text for part 4:
%\iffalse
%<*samplepart4>
%\fi
%    \begin{macrocode}
more text in part four
%    \end{macrocode}

%\iffalse
%</samplepart4>
%\fi
%
% %%%%%%%%%%%%%%%%%%%%%%%%%%%%%%%%%%%%%%
% \paragraph{Forwarding for a Complete Draft.}
%
% The following forwarding file |cdocsdrf.tex|
% compiles the main document in draft mode:
%\iffalse
%<*sampledraft>
%\fi
%    \begin{macrocode}
\def\version{draft}
\input{childdoc.def}
\childdocforward{cdocsamp}
%    \end{macrocode}

%\iffalse
%</sampledraft>
%\fi
%
% %%%%%%%%%%%%%%%%%%%%%%%%%%%%%%%%%%%%%%
% \paragraph{Forwarding for Final Version of the Chapters.}
%
% The following forwarding files |cdocsfn1.tex| and |cdocsfn2.tex|
% (with identical content)
% compile the final versions of the child documents
% |cdocsch1.tex| and |cdocsch2.tex|, respectively:
%\iffalse
%<*samplefinal>
%\fi
%    \begin{macrocode}
\def\version{final}
\input{childdoc.def}
\childdocforwardprefix[cdocsamp]{cdocsfn}{cdocsch}
%    \end{macrocode}

%\iffalse
%</samplefinal>
%\fi
%
% %%%%%%%%%%%%%%%%%%%%%%%%%%%%%%%%%%%%%%
% \paragraph{Command Line Processing.}
%
% The following three command lines generate the output files
% |cdocscld|, |cdocscl1| and |cdocscl2|
% which should be identical to
% |cdocsdrf|, |cdocsch1| and |cdocsfn2|, respectively:
% \begin{center}
% \begin{tabular}{l}
% |latex -jobname cdocscld \|\\
% |  "\def\version{draft}\input{childdoc.def}\childdocforward{cdocsamp}"|\\
% |latex -jobname cdocscl1 \|\\
% |  "\input{childdoc.def}\childdocforward[cdocsamp]{cdocsch1}"|\\
% |latex -jobname cdocscl2 \|\\
% |  "\def\version{final}\input{childdoc.def}\childdocforward{cdocsch2}"|
% \end{tabular}
% \end{center}
% Note that the trailing backslash on each first line
% merely continues the input to the second line
% (for convenient cut ant paste).
% Furthermore, the command |latex| can be replaced by any
% of its alternative versions such as |pdflatex|.
%
% %%%%%%%%%%%%%%%%%%%%%%%%%%%%%%%%%%%%%%%%%%%%%%%%%%%%%%%%%%%%%%%%%%%%%%%%%%%%%%
% %%%%%%%%%%%%%%%%%%%%%%%%%%%%%%%%%%%%%%%%%%%%%%%%%%%%%%%%%%%%%%%%%%%%%%%%%%%%%%
% \section{Implementation}
%\iffalse
%<*package>
%\fi
%
% This section describes the definitions file |childdoc.def|.

% The definitions cannot be loaded using |\usepackage| or |\RequirePackage|
% which has a mechanism to prevent loading a style file more than once.
% When loading the definitions by means of |\input|
% multiple instances have to be prevented manually:
%\iffalse
%This code needs to be before the `\ProvidesFile' directive
%which is defined at the beginning of this file.
%Therefore it is also placed there and commented out here.
%</package>
%<*discard>
%\fi
%    \begin{macrocode}
\ifdefined\childdocmain\endinput\fi
%    \end{macrocode}
%\iffalse
%</discard>
%<*package>
%\fi
%
% \macro{\ifchilddoc}
% \macro{\ifchilddocmanual}
% The conditional |\ifchilddoc| tells whether a
% child (true) or main (false) document is being compiled.
% The conditional |\ifchilddocmanual| tells whether
% the |\includeonly| mechanism is used (false) or
% the selection of child files must be performed manually (true).
% The definitions initialise to false:
%    \begin{macrocode}
\newif\ifchilddoc
\newif\ifchilddocmanual
%    \end{macrocode}

% \macro{\childdocname}
% \macro{\childdocjob}
% The macro |\childdocname| stores the name of the main document
% to be compiled. The macro |\childdocjob| stores the name of
% the document on which the \LaTeX{} compiler was originally invoked.
% The content of |\jobname| cannot be compared
% to filenames specified in the source due to different catcodes.
% The following code rescans |\jobname|, stores the result
% in |\childdocname| and saves a copy in |\childdocjob|:
%    \begin{macrocode}
\edef\childdocname{\scantokens\expandafter{\jobname\noexpand}}
\let\childdocjob\childdocname
%    \end{macrocode}

% \macro{\childdocdisable}
% The macro |\childdocdisable| prevents the main file
% from being processed more than once.
% At this stage, the main document command |\childdocmain|
% is assumed to be called once again where it should do nothing.
% Any subsequent call to it should prevent
% a secondary processing of the main document
% It overwrites the forwarding commands
% |\childdocof| and |\childdocforward|
% with empty macros to prevent further inclusions of the main document:
%    \begin{macrocode}
\newcommand{\childdocdisable}
{
  \renewcommand{\childdocmain}[1]{\renewcommand{\childdocmain}[1]{\endinput}}
  \renewcommand{\childdocof}[1]{}
  \renewcommand{\childdocby}[2][]{}
  \renewcommand{\childdocforward}[2][]{}
  \renewcommand{\childdocdisable}{}
}
%    \end{macrocode}

% \macro{\childdocmain}
% The macro |\childdocmain| is to be called at the top of the main file
% with nothing or the main filename (without extension) as argument.
% First, it breaks loops.
% If the argument is not empty and does not match |\childdocname|
% (which is set by the first inclusion of |childdoc.def|),
% |\ifchilddoc| is set to true, |\includeonly| is applied to the child file
% and |\jobname| is set to the main file
% (for proper handling of |.aux| files):
%    \begin{macrocode}
\newcommand{\childdocmain}[1]
{
  \childdocdisable\childdocmain{}
  \if?#1?\else
    \begingroup
      \def\childdoctmp{#1}
      \ifx\childdoctmp\childdocname
        \def\childdoctmp{}
      \else
        \def\childdoctmp
        {
          \childdoctrue
          \includeonly{\childdocname}
          \def\childdocjob{#1}
          \def\jobname{#1}
        }
      \fi
      \expandafter
    \endgroup
    \childdoctmp
  \fi
}
%    \end{macrocode}

% \macro{\childdocof}
% The command |\childdocof| redirects
% compilation to the main file |#1|.
%    \begin{macrocode}
\newcommand{\childdocof}[1]
{
  \childdocdisable
  \childdoctrue
  \includeonly{\childdocname}
  \def\jobname{#1}
  \def\childdocjob{#1}
  \input{#1}
}
%    \end{macrocode}

% \macro{\childdocby}
% The command |\childdocby| ....
%    \begin{macrocode}
\newcommand{\childdocby}[2][]
{
  \childdocdisable
  \childdoctrue
  \childdocmanualtrue
  \if?#1?\else
    \def\jobname{#2}
  \fi
  \def\childdocjob{#2}
  \input{#2}
  \endinput
}
%    \end{macrocode}

% \macro{\childdocforward}
% The command |\childdocforward| redirects
% compilation to the main file or
% (if the optional argument is given) a child file.
% Parameters are set as if the main file
% or a child file starting with |\childdocof| was compiled.
% Then compilation is handed over to the main file:
%    \begin{macrocode}
\newcommand{\childdocforward}[2][]
{
  \begingroup
    \if?#1?
      \def\childdoctmp
      {
        \def\childdocname{#2}
        \def\childdocjob{#2}
        \def\jobname{#2}
        \input{#2}
        \endinput
      }
    \else
      \def\childdoctmp
      {
        \childdocdisable
        \def\childdocname{#2}
        \childdoctrue
        \includeonly{#2}
        \def\childdocjob{#1}
        \def\jobname{#1}
        \input{#1}
        \endinput
      }
    \fi
    \expandafter
  \endgroup
  \childdoctmp
}
%    \end{macrocode}

% \macro{\childdocforwardprefix}
% The command |\childdocforwardprefix| redirects
% compilation to the main or a child file by means of a pattern.
% The prefix |#1| in the current filename is replaced by |#2|
% and the suffix of the current filename is kept
% (it is assumed that the filename does not contain the substring `|~~~|'
% which is used as a delimiter).
% Compilation is handed over to the new file by |\childdocforward|:
%    \begin{macrocode}
\newcommand{\childdocforwardprefix}[3][]
{
  \begingroup
    \def\childdocextract #2##1~~~{\def\childdoctmp{\childdocforward[#1]{#3##1}}}
    \expandafter\childdocextract\childdocname~~~
    \expandafter
  \endgroup
  \childdoctmp
}
%    \end{macrocode}

% \macro{\childdoc}
% The deprecated macro |\childdoc| is a legacy version of |\childdocmain|:
%    \begin{macrocode}
\newcommand{\childdoc}{\childdocmain}
%    \end{macrocode}

% \macro{\childdocredirect}
% The deprecated macro |\childdocredirect| is a legacy version
% of |\childdocforward| and |\childdocforwardprefix|:
%    \begin{macrocode}
\newcommand{\childdocredirect}[2][]
{
  \begingroup
    \if?#1?
      \def\childdoctmp{\childdocforward{#2}}
    \else
      \def\childdoctmp{\childdocforwardprefix{#1}{#2}}
    \fi
    \expandafter
  \endgroup
  \childdoctmp
}
%    \end{macrocode}

%\iffalse
%</package>
%\fi
%
\endinput
|\\
|\childdocforward[|\textit{main}|]{|\textit{dest}|}|\\
\end{tabular}
\end{center}
%
The argument \textit{dest} is the destination file
(without extension).
It should be the main file or one of the child files.
Note that further \textsf{childdoc} directives
such as |\childdocof| and |\childdocforward|
in the indicated file will be processed in this form.
The optional argument \textit{main}
passes on directly to the main file \textit{main}
while pretending to compile the child \textit{dest}.
This form behaves as if \textit{dest}
issues |\childdocof{|\textit{main}|}| right away,
and no further \textsf{childdoc} directives will be processed.

%%%%%%%%%%%%%%%%%%%%%%%%%%%%%%%%%%%%%%%%
\DescribeMacro{\...prefix}
In the alternative form |\childdocforwardprefix|,
%
\begin{center}
\begin{tabular}{l}
|% \iffalse
%
% childdoc.dtx Copyright (C) 2017-2018 Niklas Beisert
%
% This work may be distributed and/or modified under the
% conditions of the LaTeX Project Public License, either version 1.3
% of this license or (at your option) any later version.
% The latest version of this license is in
%   http://www.latex-project.org/lppl.txt
% and version 1.3 or later is part of all distributions of LaTeX
% version 2005/12/01 or later.
%
% This work has the LPPL maintenance status `maintained'.
%
% The Current Maintainer of this work is Niklas Beisert.
%
% This work consists of the files childdoc.dtx and childdoc.ins
% and the derived files childdoc.def and cdocsamp.tex with
% cdocsch1.tex, cdocsch2.tex, cdocsdrf.tex, cdocsfn1.tex, cdocsfn2.tex.
%
%<package>\ifdefined\childdocmain\endinput\fi
%<package>\ProvidesFile{childdoc.def}[2018/12/30 v2.0 child document driver]
%<samplemain>\ProvidesFile{cdocsamp.tex}[2018/12/30 v2.0 sample for childdoc]
%<*driver>
%\ProvidesFile{childdoc.drv}[2018/12/30 v2.0 childdoc reference manual file]
\PassOptionsToClass{10pt,a4paper}{article}
\documentclass{ltxdoc}

\usepackage[margin=35mm]{geometry}
\usepackage{hyperref}
\usepackage{hyperxmp}
\usepackage[usenames]{color}

\hypersetup{colorlinks=true}
\hypersetup{pdfstartview=FitH}
\hypersetup{pdfpagemode=UseNone}
\hypersetup{pdfsource={}}
\hypersetup{pdflang={en-UK}}
\hypersetup{pdfcopyright={Copyright 2017-2018 Niklas Beisert.
  This work may be distributed and/or modified under the
  conditions of the LaTeX Project Public License, either version 1.3
  of this license or (at your option) any later version.}}
\hypersetup{pdflicenseurl={http://www.latex-project.org/lppl.txt}}
\hypersetup{pdfcontactaddress={ETH Zurich, ITP, HIT K,
  Wolfgang-Pauli-Strasse 27}}
\hypersetup{pdfcontactpostcode={8093}}
\hypersetup{pdfcontactcity={Zurich}}
\hypersetup{pdfcontactcountry={Switzerland}}
\hypersetup{pdfcontactemail={nbeisert@itp.phys.ethz.ch}}
\hypersetup{pdfcontacturl={http://people.phys.ethz.ch/\xmptilde nbeisert/}}

\newcommand{\secref}[1]{\hyperref[#1]{section \ref*{#1}}}

\parskip1ex
\parindent0pt
\let\olditemize\itemize
\def\itemize{\olditemize\parskip0pt}

\begin{document}

\title{The \textsf{childdoc} Package}
\hypersetup{pdftitle={The childdoc Package}}
\author{Niklas Beisert\\[2ex]
  Institut f\"ur Theoretische Physik\\
  Eidgen\"ossische Technische Hochschule Z\"urich\\
  Wolfgang-Pauli-Strasse 27, 8093 Z\"urich, Switzerland\\[1ex]
  \href{mailto:nbeisert@itp.phys.ethz.ch}
  {\texttt{nbeisert@itp.phys.ethz.ch}}}
\hypersetup{pdfauthor={Niklas Beisert}}
\hypersetup{pdfsubject={Manual for the LaTeX2e Package childdoc}}
\date{30 December 2018, \textsf{v2.0}}
\maketitle

\begin{abstract}\noindent
\textsf{childdoc} is a \LaTeXe{} package
that enables the direct compilation
of document sections included by |\include|
to individual files.
\end{abstract}

\begingroup
\parskip0ex
\tableofcontents
\endgroup

%%%%%%%%%%%%%%%%%%%%%%%%%%%%%%%%%%%%%%%%%%%%%%%%%%%%%%%%%%%%%%%%%%%%%%%%%%%%%%%%
%%%%%%%%%%%%%%%%%%%%%%%%%%%%%%%%%%%%%%%%%%%%%%%%%%%%%%%%%%%%%%%%%%%%%%%%%%%%%%%%
\section{Introduction}

\LaTeX{} provides a mechanism to structure a large document (such as a book)
into a main file and several child files (containing the chapters)
using the |\include| command.
This mechanism is beneficial for documents
which span hundreds of pages in order to
make the source file(s) more manageable.
Moreover, compilation can be restricted to
selected child files by means of the |\includeonly| command.
The latter feature can be used to reduce the compilation time while editing
(this was significantly more useful in the earlier days of \LaTeX{})
or to generate a smaller document which is easier to navigate.
Another application of |\includeonly| is to generate
documents consisting of selected parts of the complete document.

However, there are a few drawbacks of the plain |\include| mechanism:
\begin{itemize}
\item
The child files cannot be compiled on their own,
they can only be compiled via the main file.
A naive editing environment
(such as a text editor with an option
to have the current file processed by \LaTeX)
may require one to switch to the main file before compiling;
attempting to compile the child file produces errors.
\item
The main file must be modified (each time)
to adjust the |\includeonly| command
to the present needs. This easily leaves the main file in a messy state.
\item
The generated document will always carry the filename
of the main document. This is inconvenient if
several child files are to be compiled and
to be kept for distribution.
\end{itemize}

The present package provides a simple interface
to make child files individually compilable by \LaTeX{}.
Compiling a child file then has the same effect as compiling
the main file with an |\includeonly| command
to select the appropriate child.
Moreover the generated document will carry the name of the child
rather than the main file.
This resolves all three above issues.

This feature is meant to make the editing of books,
thesis documents and lecture notes somewhat more convenient.
However, the package can also be used efficiently for
composing a series of documents (such as exercise sheets)
which are typically distributed individually.
It then assists the author in generating the individual documents
(potentially in different versions)
as well as a document containing the collected series.
Another application is in developing style files
or other kinds of included material
where compilation of the style file could redirect
to a sample or test file.

%%%%%%%%%%%%%%%%%%%%%%%%%%%%%%%%%%%%%%%%%%%%%%%%%%%%%%%%%%%%%%%%%%%%%%%%%%%%%%%%
%%%%%%%%%%%%%%%%%%%%%%%%%%%%%%%%%%%%%%%%%%%%%%%%%%%%%%%%%%%%%%%%%%%%%%%%%%%%%%%%
\section{Usage}

First of all, the package \textsf{childdoc} is \emph{not} a standard
\LaTeXe{} |.sty| style file! Therefore it needs to be invoked in
a non-standard way.

%%%%%%%%%%%%%%%%%%%%%%%%%%%%%%%%%%%%%%%%%%%%%%%%%%%%%%%%%%%%%%%%%%%%%%%%%%%%%%%%
\subsection{Included Files}
\label{sec:include}

%%%%%%%%%%%%%%%%%%%%%%%%%%%%%%%%%%%%%%%%
\DescribeMacro{\childdocmain}
To use the package, add the commands
\begin{center}
\begin{tabular}{l}
|\input{childdoc.def}|\\
|\childdocmain{}|\\
\end{tabular}
\end{center}
at the very top of the main \LaTeX{} file,
in particular \emph{before} the |\documentclass| statement!
The argument of |\childdocmain| should be left empty
(but it must be present).

%%%%%%%%%%%%%%%%%%%%%%%%%%%%%%%%%%%%%%%%
\DescribeMacro{\childdocof}
Furthermore, add the commands
\begin{center}
\begin{tabular}{l}
|\input{childdoc.def}|\\
|\childdocof{|\textit{main}|}|\\
\end{tabular}
\end{center}
at the top of every child file \textit{child}
which is included by |\include{|\textit{child}|}|
from within the main file
(or at least for those files to be compiled individually).
The argument \textit{main} must be the filename of the main file.

There are a couple of
considerations in setting up the main and child documents:

%%%%%%%%%%%%%%%%%%%%%%%%%%%%%%%%%%%%%%%%
\paragraph{Restrictions.}

Please note the following restrictions:
\begin{itemize}
\item
|\childdocmain| must be called with one argument \textit{main}
to ensure compatibility with earlier version of the package.
It must either be empty (|\childdocmain{}|)
or precisely match the filename of the main file in which it is specified.
See \secref{sec:detection} for further information.
\item
The filename \textit{main} must be specified without the |.tex| extension.
\item
The filename \textit{main} is case sensitive
(even in case-insensitive file systems)
due to internal string comparison.
\item
The argument \textit{main} should be fully expanded, it cannot be a macro.
\item
Subdirectories and special characters should be avoided in filenames.
\item
The command |\childdocmain{|\textit{main}|}| must be followed by a whitespace.
It should not be followed immediately by another command
or by a comment mark `|%|'.
This is because the \TeX{} parser reads the token immediately following
the argument of |\childdocmain| and puts it
at the beginning of every child section;
however, a white\-space is ignored.
\end{itemize}

%%%%%%%%%%%%%%%%%%%%%%%%%%%%%%%%%%%%%%%%
\paragraph{Content of Main File.}

It is advisable to place all content in the child files included by |\include|.
Any output contained in the main file will appear in all child documents
unless suppressed manually;
it cannot be suppressed automatically by the |\includeonly| directive
and thus should normally be avoided.
A method to include some content in the main file
by means of conditional processing is described in \secref{sec:conditional}.

%%%%%%%%%%%%%%%%%%%%%%%%%%%%%%%%%%%%%%%%
\paragraph{Page Numbering.}

When only a part of the document is compiled,
the appropriate numbering of pages
(as well as other status parameters)
is determined from the |.aux| files.
The latter contain information from previous passes.
However this information needs to propagate through
all intermediate child documents.
Therefore the page numbering in child documents may well
be inconsistent until the complete document is compiled at least once.

A useful (if unconventional) way to always ensure a consistent
page numbering is to restart the numbering in each child document
and denote the pages by `\textit{child}|.|\textit{page}'
where \textit{child} represents the chapter/section number of the child file.
This can be achieved by the command
|\numberwithin{page}{|\textit{child}|}|
of the \textsf{amsmath} package
where \textit{child} can be |chapter| or |section|
depending on the chosen structuring.
Alternatively, one can modify the macro |\thepage| appropriately
and reset the counter |page| at the start of each child file.

%%%%%%%%%%%%%%%%%%%%%%%%%%%%%%%%%%%%%%%%%%%%%%%%%%%%%%%%%%%%%%%%%%%%%%%%%%%%%%%%
\subsection{Conditional Processing}
\label{sec:conditional}

The package provides a mechanism to compile different versions
of a document. To customise the versions further some conditional processing
can come in handy to distinguish which version is being compiled.
The package provides two macros to describe the compilation context:

%%%%%%%%%%%%%%%%%%%%%%%%%%%%%%%%%%%%%%%%
\DescribeMacro{\ifchilddoc}
The conditional |\ifchilddoc| distinguishes between the compilation of
child documents and the main document:
%
\begin{center}
|\ifchilddoc |\textit{child-code}| |[|\||else |\textit{main-code}]| \||fi|
\end{center}

%%%%%%%%%%%%%%%%%%%%%%%%%%%%%%%%%%%%%%%%
\DescribeMacro{\childdocname}
\DescribeMacro{\childdocjob}
The macro |\childdocname| contains the filename (without extension)
of the main or child file being processed.
Note that |\childdocjob| will always contain the name of the main file.

%%%%%%%%%%%%%%%%%%%%%%%%%%%%%%%%%%%%%%%%
\paragraph{Title Page.}

Conditional processing can be used to include a title or banner page
in the main document when proper precautions are taken.
Importantly, the code in the main file should ensure that the page counter
(as well as other status parameters which are stored in the |.aux| files)
takes the same value after the conditional processing.
Otherwise the page numbers may take divergent values
depending on which part is compiled.

For example, a title page could be declared by:
%
\begin{center}
\begin{tabular}{l}
|\ifchilddoc\||else|\\
|\addtocounter{page}{-1}|\\
\textit{code for title page}\\
|\newpage|\\
|\||fi|
\end{tabular}
\end{center}
%
A banner page for the child documents can be generated by:
%
\begin{center}
\begin{tabular}{l}
|\ifchilddoc|\\
|\addtocounter{page}{-1}|\\
\textit{code for banner page}\\
|\newpage|\\
|\||fi|
\end{tabular}
\end{center}
%
Here one could write a message such as:
\begin{center}
|This is the part \childdocname{} of \childdocjob{}.|
\end{center}

%%%%%%%%%%%%%%%%%%%%%%%%%%%%%%%%%%%%%%%%%%%%%%%%%%%%%%%%%%%%%%%%%%%%%%%%%%%%%%%%
\subsection{Flags}
\label{sec:flags}

The package makes it easy to generate different versions
of the main or child documents.
To this end compilation flags can be defined
and assigned different default values.
They will be particularly useful in conjunction
with the forwarding mechanism described in \secref{sec:forward}.

For example, it may be useful to have a flag |\version|
which can be set to |draft| or |final|.
The document source will contain some conditional code
depending on the value of |\version|.
Suppose further, the flag should default to |final| for the main file
and to |draft| for child files
which is a natural assignment for editing the document.
This is achieved by placing the following code
in the preamble of the main document
(below the |\childdocmain| directive):
%
\begin{center}
\begin{tabular}{l}
|\ifchilddoc|\\
|\providecommand{\version}{draft}|\\
|\||else|\\
|\providecommand{\version}{final}|\\
|\||fi|
\end{tabular}
\end{center}
%
The definition by |\providecommand| makes sure
that previous definitions are not overwritten.
Further statements |\providecommand{\version}{...}|
can thus be added before the above code to override it.

For the main file, one might add a line
(between |\childdocmain| and the above block)
%
\begin{center}
|%\ifchilddoc\||else\providecommand{\version}{draft}\||fi|
\end{center}
%
which can be uncommented to produce a draft version.
Likewise one can add a line to the very top of a child file
(above the |\childdocof{|\textit{main}|}| directive)
%
\begin{center}
|%\providecommand{\version}{final}|
\end{center}
%
which can be uncommented to produce the final version of this child document.

%%%%%%%%%%%%%%%%%%%%%%%%%%%%%%%%%%%%%%%%%%%%%%%%%%%%%%%%%%%%%%%%%%%%%%%%%%%%%%%%
\subsection{Forwarding}
\label{sec:forward}

Different versions of the main or child documents
using compilation flags as described in \secref{sec:flags}
can be (permanently) stored in different files
for convenient compilation, viewing and distribution.
To this end, the package defines a command
to pass on compilation to a different file:

%%%%%%%%%%%%%%%%%%%%%%%%%%%%%%%%%%%%%%%%
\DescribeMacro{\childdocforward}
The command |\childdocforward| redirects processing to
another source file:
%
\begin{center}
\begin{tabular}{l}
|\input{childdoc.def}|\\
|\childdocforward[|\textit{main}|]{|\textit{dest}|}|\\
\end{tabular}
\end{center}
%
The argument \textit{dest} is the destination file
(without extension).
It should be the main file or one of the child files.
Note that further \textsf{childdoc} directives
such as |\childdocof| and |\childdocforward|
in the indicated file will be processed in this form.
The optional argument \textit{main}
passes on directly to the main file \textit{main}
while pretending to compile the child \textit{dest}.
This form behaves as if \textit{dest}
issues |\childdocof{|\textit{main}|}| right away,
and no further \textsf{childdoc} directives will be processed.

%%%%%%%%%%%%%%%%%%%%%%%%%%%%%%%%%%%%%%%%
\DescribeMacro{\...prefix}
In the alternative form |\childdocforwardprefix|,
%
\begin{center}
\begin{tabular}{l}
|\input{childdoc.def}|\\
|\childdocforwardprefix[|\textit{main}|]{|\textit{prefix}|}{|\textit{dest}|}|
\end{tabular}
\end{center}
%
the destination file is determined by a pattern
depending on the current file:
To make this work, the current file must be called
`{\textit{prefix}\hspace{0.2em}\textit{suffix}}'
with \textit{prefix} matching precisely the argument.
Processing is then passed on to the file
`{\textit{dest}\hspace{0.2em}\textit{suffix}}'.
Surely, the same effect is achieved by
directly specifying the
argument `{\textit{dest}\hspace{0.2em}\textit{suffix}}'
in the first form.
However, that requires to set up a different file
for each child. With the alternative form of the command
all these files can have exactly the same content
which simplifies setting them up and maintaining them.

For example, the following file |draft.tex|
with a compilation flag |\version| as described in \secref{sec:flags}
compiles the main document as a draft:
%
\begin{center}
\begin{tabular}{l}
|\def\version{draft}|\\
|\input{childdoc.def}|\\
|\childdocforward{|\textit{main}|}|
\end{tabular}
\end{center}
%
Likewise, the following files |final|\textit{nn}|.tex|
compile the final version of the child document
|child|\textit{nn}|.tex|:
%
\begin{center}
\begin{tabular}{l}
|\def\version{final}|\\
|\input{childdoc.def}|\\
|\childdocforwardprefix{final}{child}|
\end{tabular}
\end{center}
%

Note that when several versions of a main file and/or of each child file
are to be generated, it may be convenient to set up a |Makefile| or
shell script to automatise the process.

%%%%%%%%%%%%%%%%%%%%%%%%%%%%%%%%%%%%%%%%%%%%%%%%%%%%%%%%%%%%%%%%%%%%%%%%%%%%%%%%
\subsection{Command Line Processing}
\label{sec:commandline}

The effect of redirection files can also be achieved by invoking
the \LaTeX{} compiler with a more elaborate command line.
Most conveniently this should be done as part
of a shell script or a |Makefile|.

When using \textsf{childdoc} in the main file, the following
command lines effectively perform a redirection
(note that depending on the shell being used,
backslashes may have to be doubled: `|\|' $\to$ `|\\|'):
%
\begin{center}
|... -jobname "|\textit{target}|" |\\|"|[\textit{flags}]%
|\input{childdoc.def}\childdocforward[|\textit{main}|]{|\textit{dest}|}"|
\end{center}
%
Here \textit{target} is the name of the output file,
\textit{main} is the name of the main file
and \textit{dest} is the name of the main or child file to be processed
(all filenames without extensions).
The optional argument \textit{main} can be omitted
if \textit{main} matches \textit{dest}.
Optionally, compilation \textit{flags} can be defined via |\def| commands.
This command line makes the \TeX{} engine believe
it is compiling the file \textit{target}
whose content is specified as the latter parameter.
The provided code then forwards the processing to
\textit{main} or \textit{dest} as described in \secref{sec:forward}.

%%%%%%%%%%%%%%%%%%%%%%%%%%%%%%%%%%%%%%%%%%%%%%%%%%%%%%%%%%%%%%%%%%%%%%%%%%%%%%%%
\subsection{Include by Input}
\label{sec:input}

Including child documents by |\include| has some restrictions by design.
Most notably, the content of a child document always occupies
its own set of pages; pages cannot be shared between child documents.
Usually, this behaviour makes perfect sense
because each child document contain an essential part of the document.
However, in some situations it may be desirable to compose
a document from a collection of parts
without having mandatory page breaks between then.
For this case, the package
provides a mechanism to include parts
by |\input| which can also be processed individually.
However, by construction this mechanism
requires manual handling of the content to be output.

%%%%%%%%%%%%%%%%%%%%%%%%%%%%%%%%%%%%%%%%
\DescribeMacro{\ifchilddocmanual}
The main file should be prepared as usual, see \secref{sec:include}.
However, the document body must make a distinction
between processing of an individual part and of the main document, e.g.:
%
\begin{center}
\begin{tabular}{l}
|\ifchilddocmanual|\\
|\input{\childdocname}|\\
|\||else|\\
\textit{document body with }|\input{|\textit{part}|}|\\
|\||fi|
\end{tabular}
\end{center}
%
The conditional |\ifchilddocmanual| is true whenever
a part to be included by |\input| is being compiled,
and the name of the part is stored in |\childdocname|.

%%%%%%%%%%%%%%%%%%%%%%%%%%%%%%%%%%%%%%%%
\DescribeMacro{\childdocby}
Each part to be included by |\input| should start with:
%
\begin{center}
\begin{tabular}{l}
|\input{childdoc.def}|\\
|\childdocby{|\textit{main}|}|\\
\end{tabular}
\end{center}
%
The directive |\childdocby| is similar to |\childdocof|
described in \secref{sec:include},
but the subsequent selection of content must be done manually.
To that end, both |\ifchilddoc| and |\ifchilddocmanual|
will be true upon processing of a part,
and the name of the part is stored in |\childdocname|.
Note that |\jobname| will be set to the filename of the current part
so that each part receives an individual |.aux| file
that does not interfere with the |.aux| file(s) of the main document.
This behaviour can be altered by the alternative form
|\childdocby[*]{|\textit{main}|}| (with a non-empty optional argument)
which uses the |.aux| file of the main document
by setting |\jobname| to \textit{main}.

%%%%%%%%%%%%%%%%%%%%%%%%%%%%%%%%%%%%%%%%%%%%%%%%%%%%%%%%%%%%%%%%%%%%%%%%%%%%%%%%
\subsection{Driver Development}
\label{sec:driver}

The \textsf{childdoc} mechanism can also be use for the development
of definition files such as \LaTeX{} styles or classes.
This case differs from the above setup with multiple parts
included by |\include| in that no |\includeonly| should be invoked.
This can be achieved by starting the include file
(before |\ProvidesPackage|) with:
%
\begin{center}
\begin{tabular}{l}
|\input{childdoc.def}|\\
|\childdocforward{|\textit{main}|}|\\
\end{tabular}
\end{center}
%
or alternatively with:
%
\begin{center}
\begin{tabular}{l}
|\input{childdoc.def}|\\
|\childdocby{|\textit{main}|}|\\
\end{tabular}
\end{center}
%
Both forms have slightly different effects as described above.
The main file is prepared as usual, see \secref{sec:include}.

%%%%%%%%%%%%%%%%%%%%%%%%%%%%%%%%%%%%%%%%%%%%%%%%%%%%%%%%%%%%%%%%%%%%%%%%%%%%%%%%
\subsection{Legacy Detection}
\label{sec:detection}

The directive |\childdocmain| in the main file can detect
whether the complete document or merely a child is to be compiled
even without using the directive |\childdocof|.
This method is deprecated because it is less robust
and there is no compelling reason to use it;
it is merely provided for backward compatibility
and it may be removed in future versions.

If the detection mechanism is to be used,
it is mandatory to correctly specify
the filename of the main file as the argument of |\childdocmain|:
%
\begin{center}
\begin{tabular}{l}
|\input{childdoc.def}|\\
|\childdocmain{|\textit{main}|}|\\
\end{tabular}
\end{center}
%
If |\jobname| does not match the argument \textit{main} of |\childdocmain|,
it is assumed that |\jobname| points to the child file to be compiled.
When using |\childdocmain| with the main file specified as argument,
it suffices to start a child file
with just |\input{|\textit{main}|}|
without loading of the package and using |\childdocof|.
If instead all processing is done
with the appropriate \textsf{childdoc} directives,
the argument of \textit{main} of |\childdocmain| can be empty.

An alternative version of the command line processing described
in \secref{sec:commandline} using the detection mechanism reads:
%
\begin{center}
|... -jobname "|\textit{target}|" "|[\textit{flags}]%
[|\def\jobname{|\textit{dest}|}|]|\input{|\textit{main}|}"|
\end{center}

%%%%%%%%%%%%%%%%%%%%%%%%%%%%%%%%%%%%%%%%%%%%%%%%%%%%%%%%%%%%%%%%%%%%%%%%%%%%%%%%
\subsection{Manual Code}
\label{sec:manual}

In case one cannot be certain whether the definitions file |childdoc.def|
is installed on the target \TeX{} distribution
and one prefers not to ship it,
it is conceivable to paste a few relevant commands into the sources.

To that end, drop all statements |\input{childdoc.def}|
and perform the replacements as outlined below.
Instead of |\childdocmain{|\textit{main}|}| add the following code
to the top of the main file:
%
\begin{center}
\begin{tabular}{l}
|\||ifdefined\childdocname\endinput\||fi\newif\ifchilddoc|\\
|\edef\childdocname{\scantokens\expandafter{\jobname\noexpand}}|\\
|\def\childdocmain{|\textit{main}|}\||ifx\childdocmain\childdocname\||else|\\
|\childdoctrue\includeonly{\childdocname}\let\jobname\childdocmain\||fi|\\
\end{tabular}
\end{center}
%
Instead of |\childdocof{|\textit{main}|}| just include the main file
at the top of each child file:
%
\begin{center}
|\input{|\textit{main}|}|
\end{center}
%
A simple redirection |\childdocforward{|\textit{dest}|}| is achieved by:
%
\begin{center}
|\def\jobname{|\textit{dest}|}\input{\jobname}|
\end{center}
%
The redirection with prefix
|\childdocforwardprefix[|\textit{prefix}|]{|\textit{dest}|}|
is accomplished by:
%
\begin{center}
\begin{tabular}{l}
|{\edef\jobname{\scantokens\expandafter{\jobname\noexpand}}|\\
|\def\redirectjob |\textit{prefix}|#1~~~{\gdef\jobname{|\textit{dest}|#1}}|\\
|\expandafter\redirectjob\jobname~~~}\input{\jobname}|
\end{tabular}
\end{center}

In an alternative approach,
child documents can be compiled by a specific command line
without additional code or specific definitions:
%
\begin{center}
|... -jobname "|\textit{target}|" "|[\textit{flags}]%
|\includeonly{|\textit{dest}|}\input{|\textit{main}|}"|
\end{center}
%

%%%%%%%%%%%%%%%%%%%%%%%%%%%%%%%%%%%%%%%%%%%%%%%%%%%%%%%%%%%%%%%%%%%%%%%%%%%%%%%%
%%%%%%%%%%%%%%%%%%%%%%%%%%%%%%%%%%%%%%%%%%%%%%%%%%%%%%%%%%%%%%%%%%%%%%%%%%%%%%%%
\section{Information}

%%%%%%%%%%%%%%%%%%%%%%%%%%%%%%%%%%%%%%%%%%%%%%%%%%%%%%%%%%%%%%%%%%%%%%%%%%%%%%%%
\subsection{Copyright}

Copyright \copyright{} 2017--2018 Niklas Beisert

This work may be distributed and/or modified under the
conditions of the \LaTeX{} Project Public License, either version 1.3
of this license or (at your option) any later version.
The latest version of this license is in
  \url{http://www.latex-project.org/lppl.txt}
and version 1.3 or later is part of all distributions of \LaTeX{}
version 2005/12/01 or later.

This work has the LPPL maintenance status `maintained'.

The Current Maintainer of this work is Niklas Beisert.

This work consists of the files |README.txt|, |childdoc.ins| and |childdoc.dtx|
as well as the derived files |childdoc.def|, |cdocsamp.tex|
with |cdocsch1.tex|, |cdocsch2.tex|, |cdocspt3.tex|, |cdocspt4.tex|,
|cdocsdrf.tex|, |cdocsfn1.tex|, |cdocsfn2.tex|
as well as |childdoc.pdf|.

%%%%%%%%%%%%%%%%%%%%%%%%%%%%%%%%%%%%%%%%%%%%%%%%%%%%%%%%%%%%%%%%%%%%%%%%%%%%%%%%
\subsection{Files and Installation}

The package consists of the files:
%
\begin{center}
\begin{tabular}{ll}
    |README.txt|   & readme file \\
    |childdoc.ins| & installation file \\
    |childdoc.dtx| & source file \\
    |childdoc.def| & definition file \\
    |cdocsamp.tex| & sample main file \\
    |cdocsch1.tex| & sample include file \\
    |cdocsch2.tex| & sample include file \\
    |cdocspt3.tex| & sample part file \\
    |cdocspt4.tex| & sample part file \\
    |cdocsdrf.tex| & sample redirection file \\
    |cdocsfn1.tex| & sample redirection file \\
    |cdocsfn2.tex| & sample redirection file \\
    |childdoc.pdf| & manual
\end{tabular}
\end{center}
%
The distribution consists of the files
|README.txt|, |childdoc.ins| and |childdoc.dtx|.
%
\begin{itemize}
\item
Run (pdf)\LaTeX{} on |childdoc.dtx|
to compile the manual |childdoc.pdf| (this file).
\item
Run \LaTeX{} on |childdoc.ins| to create the definitions file |childdoc.def|
and the sample |cdocsamp.tex| with include files
|cdocsch1.tex|, |cdocsch2.tex|, |cdocspt3.tex|, |cdocspt4.tex|,
|cdocsdrf.tex|, |cdocsfn1.tex|, |cdocsfn2.tex|.
Then copy the file |childdoc.def| to an appropriate directory of your \LaTeX{}
distribution, e.g.\ \textit{texmf-root}|/tex/latex/childdoc|.
\end{itemize}

%%%%%%%%%%%%%%%%%%%%%%%%%%%%%%%%%%%%%%%%%%%%%%%%%%%%%%%%%%%%%%%%%%%%%%%%%%%%%%%%
\subsection{Related CTAN Packages}

There are several other packages which offer a similar functionality:
%
\begin{itemize}
\item
The packages
\href{http://ctan.org/pkg/docmute}{\textsf{docmute}},
\href{http://ctan.org/pkg/includex}{\textsf{includex}} and
\href{http://ctan.org/pkg/standalone}{\textsf{standalone}}
provide commands to include only the document body of
a child file thus allowing both files to be compiled individually.
\item
The packages \href{http://ctan.org/pkg/subdocs}{\textsf{subdocs}}
and \href{http://ctan.org/pkg/subfiles}{\textsf{subfiles}}
provide structures in which the main and child documents can be
encapsulated and allowing them to be compiled individually.
The inclusion mechanism is different from the conventional |\include|.
\item
The package \href{http://ctan.org/pkg/combine}{\textsf{combine}}
is an elaborate solution to combine several documents into one.
\end{itemize}
%
See also the CTAN topic \href{http://ctan.org/topic/subdocs}{\textsf{subdocs}}
for further related packages.
The present package differs from the above solutions in that
a document structure constructed with the conventional |\include| mechanism
just needs two extra commands at the top of every file
such that all constituent files can be compiled individually.

%%%%%%%%%%%%%%%%%%%%%%%%%%%%%%%%%%%%%%%%%%%%%%%%%%%%%%%%%%%%%%%%%%%%%%%%%%%%%%%%
%\subsection{Feature Suggestions}
%
%The following is a list of features which may be useful for future
%versions of this package:
%%
%\begin{itemize}
%\item
%\ldots
%\end{itemize}

%%%%%%%%%%%%%%%%%%%%%%%%%%%%%%%%%%%%%%%%%%%%%%%%%%%%%%%%%%%%%%%%%%%%%%%%%%%%%%%%
\subsection{Revision History}

%%%%%%%%%%%%%%%%%%%%%%%%%%%%%%%%%%%%%%%%
\paragraph{v2.0:} 2018/12/30

\begin{itemize}
\item
immediate forward processing
\item
added |\childdocby| mechanism
\item
manual restructured
\end{itemize}

%%%%%%%%%%%%%%%%%%%%%%%%%%%%%%%%%%%%%%%%
\paragraph{v1.6:} 2018/01/17

\begin{itemize}
\item
application for development of include files
\item
corrections to manual
\end{itemize}

%%%%%%%%%%%%%%%%%%%%%%%%%%%%%%%%%%%%%%%%
\paragraph{v1.5:} 2017/05/21

\begin{itemize}
\item
more complete structuring introduced
\item
|\childdocof| introduced
\item
|\childdoc| renamed to |\childdocmain|
\item
|\childredirect| renamed to |\childdocforward| and |\childdocforwardprefix|
and functionality expanded
\end{itemize}

%%%%%%%%%%%%%%%%%%%%%%%%%%%%%%%%%%%%%%%%
\paragraph{v1.0:} 2017/04/27

\begin{itemize}
\item
manual and install package
\item
first version published on CTAN
\end{itemize}

%%%%%%%%%%%%%%%%%%%%%%%%%%%%%%%%%%%%%%%%
\paragraph{v0.6:} 2017/04/26

\begin{itemize}
\item
redirection mechanism added
\end{itemize}

%%%%%%%%%%%%%%%%%%%%%%%%%%%%%%%%%%%%%%%%
\paragraph{v0.5:} 2017/04/26

\begin{itemize}
\item
functionality in definition file
\end{itemize}


%%%%%%%%%%%%%%%%%%%%%%%%%%%%%%%%%%%%%%%%%%%%%%%%%%%%%%%%%%%%%%%%%%%%%%%%%%%%%%%%
%%%%%%%%%%%%%%%%%%%%%%%%%%%%%%%%%%%%%%%%%%%%%%%%%%%%%%%%%%%%%%%%%%%%%%%%%%%%%%%%
%%%%%%%%%%%%%%%%%%%%%%%%%%%%%%%%%%%%%%%%%%%%%%%%%%%%%%%%%%%%%%%%%%%%%%%%%%%%%%%%
\appendix

\settowidth\MacroIndent{\rmfamily\scriptsize 000\ }

 \DocInput{childdoc.dtx}

\end{document}
%</driver>
% \fi
%
% %%%%%%%%%%%%%%%%%%%%%%%%%%%%%%%%%%%%%%%%%%%%%%%%%%%%%%%%%%%%%%%%%%%%%%%%%%%%%%
% %%%%%%%%%%%%%%%%%%%%%%%%%%%%%%%%%%%%%%%%%%%%%%%%%%%%%%%%%%%%%%%%%%%%%%%%%%%%%%
% \section{Sample}
%\iffalse
%<*samplemain>
%\fi
%
% The following presents a sample document
% with two chapters, two parts, a title page,
% a compile flag as well as three forwarding files to set the flag.
% It consists of eight |.tex| files:
% \begin{center}
% \begin{tabular}{ll}
% |cdocsamp.tex|&main file\\
% |cdocsch1.tex|&include file for chapter 1\\
% |cdocsch2.tex|&include file for chapter 2\\
% |cdocspt3.tex|&include file for part 3\\
% |cdocspt4.tex|&include file for part 4\\
% |cdocsdrf.tex|&forwarding file for main file in draft mode\\
% |cdocsfi1.tex|&forwarding file for final version of chapter 1\\
% |cdocsfi2.tex|&forwarding file for final version of chapter 2\\
% \end{tabular}
% \end{center}
% Each of the eight files can be compiled directly by the \LaTeX{} compiler.
%
% %%%%%%%%%%%%%%%%%%%%%%%%%%%%%%%%%%%%%%
% \paragraph{Main File.}
%
% The main file is called |cdocsamp.tex|.
%
% Load the \textsf{childdoc} definitions and
% declare the filename for the main document:
%    \begin{macrocode}
\input{childdoc.def}
\childdocmain{}
%    \end{macrocode}

% Optional override for |\version| flag:
%    \begin{macrocode}
%%\ifchilddoc\else\providecommand{\version}{draft}\fi
%    \end{macrocode}

% Define the default values for the |\version| flag
% (|final| for the main file and |draft| for childs):
%    \begin{macrocode}
\ifchilddoc
\providecommand{\version}{draft}
\else
\providecommand{\version}{final}
\fi
%    \end{macrocode}

% Load the standard document class:
%    \begin{macrocode}
\documentclass[12pt]{article}
%    \end{macrocode}

% Start the document body:
%    \begin{macrocode}
\begin{document}
%    \end{macrocode}

% Declare a title page.
% Print title, part of document being processed and version flag:
%    \begin{macrocode}
\addtocounter{page}{-1}
\begin{center}
{\LARGE\bfseries{}childdoc example\par}
\vspace{1cm}
\ifchilddoc
\ifchilddocmanual part\else chapter\fi:
`\childdocname' of `\childdocjob'\par
\else
main document: `\childdocjob'\par
\fi
version: \version\par
\end{center}
\newpage
%    \end{macrocode}

% Manually include selected file,
% otherwise process as usual:
%    \begin{macrocode}
\ifchilddocmanual
\section*{part `\childdocname'}
\input{\childdocname}
\else
%    \end{macrocode}

% Include the two chapters:
%    \begin{macrocode}
\include{cdocsch1}
\include{cdocsch2}
%    \end{macrocode}

% Include the two parts unless only chapters should be displayed:
%    \begin{macrocode}
\ifchilddoc\else
\section{part three}
\input{cdocspt3}
\section{part four}
\input{cdocspt4}
\fi
%    \end{macrocode}

% Process as usual until here:
%    \begin{macrocode}
\fi
%    \end{macrocode}

% End of document body:
%    \begin{macrocode}
\end{document}
%    \end{macrocode}
%\iffalse
%</samplemain>
%\fi
%
% %%%%%%%%%%%%%%%%%%%%%%%%%%%%%%%%%%%%%%
% \paragraph{Chapter Include Files.}
%
% The include files are called |cdocsch1.tex| and |cdocsch2.tex|.
%
%\iffalse
%<*samplechap1|samplechap2>
%\fi

% Optional override for |\version| flag:
%    \begin{macrocode}
%%\providecommand{\version}{final}
%    \end{macrocode}

% Include the main document:
%    \begin{macrocode}
\input{childdoc.def}
\childdocof{cdocsamp}
%    \end{macrocode}

%\iffalse
%</samplechap1|samplechap2>
%\fi
%
%\iffalse
%<*samplechap1>
%\fi
% Some text for chapter 1:
%    \begin{macrocode}
\section{one}
some text in chapter one
%    \end{macrocode}

%\iffalse
%</samplechap1>
%\fi
% Some text for chapter 2:
%\iffalse
%<*samplechap2>
%\fi
%    \begin{macrocode}
\section{two}
more text in chapter two
%    \end{macrocode}

%\iffalse
%</samplechap2>
%\fi
%
% %%%%%%%%%%%%%%%%%%%%%%%%%%%%%%%%%%%%%%
% \paragraph{Part Include Files.}
%
% The include files are called |cdocspt3.tex| and |cdocspt4.tex|.
%
%\iffalse
%<*samplepart3|samplepart4>
%\fi

% Optional override for |\version| flag:
%    \begin{macrocode}
%%\providecommand{\version}{final}
%    \end{macrocode}

% Include the main document:
%    \begin{macrocode}
\input{childdoc.def}
\childdocby{cdocsamp}
%    \end{macrocode}

%\iffalse
%</samplepart3|samplepart4>
%\fi
%
%\iffalse
%<*samplepart3>
%\fi
% Some text for part 3:
%    \begin{macrocode}
some text in part three
%    \end{macrocode}

%\iffalse
%</samplepart3>
%\fi
% Some text for part 4:
%\iffalse
%<*samplepart4>
%\fi
%    \begin{macrocode}
more text in part four
%    \end{macrocode}

%\iffalse
%</samplepart4>
%\fi
%
% %%%%%%%%%%%%%%%%%%%%%%%%%%%%%%%%%%%%%%
% \paragraph{Forwarding for a Complete Draft.}
%
% The following forwarding file |cdocsdrf.tex|
% compiles the main document in draft mode:
%\iffalse
%<*sampledraft>
%\fi
%    \begin{macrocode}
\def\version{draft}
\input{childdoc.def}
\childdocforward{cdocsamp}
%    \end{macrocode}

%\iffalse
%</sampledraft>
%\fi
%
% %%%%%%%%%%%%%%%%%%%%%%%%%%%%%%%%%%%%%%
% \paragraph{Forwarding for Final Version of the Chapters.}
%
% The following forwarding files |cdocsfn1.tex| and |cdocsfn2.tex|
% (with identical content)
% compile the final versions of the child documents
% |cdocsch1.tex| and |cdocsch2.tex|, respectively:
%\iffalse
%<*samplefinal>
%\fi
%    \begin{macrocode}
\def\version{final}
\input{childdoc.def}
\childdocforwardprefix[cdocsamp]{cdocsfn}{cdocsch}
%    \end{macrocode}

%\iffalse
%</samplefinal>
%\fi
%
% %%%%%%%%%%%%%%%%%%%%%%%%%%%%%%%%%%%%%%
% \paragraph{Command Line Processing.}
%
% The following three command lines generate the output files
% |cdocscld|, |cdocscl1| and |cdocscl2|
% which should be identical to
% |cdocsdrf|, |cdocsch1| and |cdocsfn2|, respectively:
% \begin{center}
% \begin{tabular}{l}
% |latex -jobname cdocscld \|\\
% |  "\def\version{draft}\input{childdoc.def}\childdocforward{cdocsamp}"|\\
% |latex -jobname cdocscl1 \|\\
% |  "\input{childdoc.def}\childdocforward[cdocsamp]{cdocsch1}"|\\
% |latex -jobname cdocscl2 \|\\
% |  "\def\version{final}\input{childdoc.def}\childdocforward{cdocsch2}"|
% \end{tabular}
% \end{center}
% Note that the trailing backslash on each first line
% merely continues the input to the second line
% (for convenient cut ant paste).
% Furthermore, the command |latex| can be replaced by any
% of its alternative versions such as |pdflatex|.
%
% %%%%%%%%%%%%%%%%%%%%%%%%%%%%%%%%%%%%%%%%%%%%%%%%%%%%%%%%%%%%%%%%%%%%%%%%%%%%%%
% %%%%%%%%%%%%%%%%%%%%%%%%%%%%%%%%%%%%%%%%%%%%%%%%%%%%%%%%%%%%%%%%%%%%%%%%%%%%%%
% \section{Implementation}
%\iffalse
%<*package>
%\fi
%
% This section describes the definitions file |childdoc.def|.

% The definitions cannot be loaded using |\usepackage| or |\RequirePackage|
% which has a mechanism to prevent loading a style file more than once.
% When loading the definitions by means of |\input|
% multiple instances have to be prevented manually:
%\iffalse
%This code needs to be before the `\ProvidesFile' directive
%which is defined at the beginning of this file.
%Therefore it is also placed there and commented out here.
%</package>
%<*discard>
%\fi
%    \begin{macrocode}
\ifdefined\childdocmain\endinput\fi
%    \end{macrocode}
%\iffalse
%</discard>
%<*package>
%\fi
%
% \macro{\ifchilddoc}
% \macro{\ifchilddocmanual}
% The conditional |\ifchilddoc| tells whether a
% child (true) or main (false) document is being compiled.
% The conditional |\ifchilddocmanual| tells whether
% the |\includeonly| mechanism is used (false) or
% the selection of child files must be performed manually (true).
% The definitions initialise to false:
%    \begin{macrocode}
\newif\ifchilddoc
\newif\ifchilddocmanual
%    \end{macrocode}

% \macro{\childdocname}
% \macro{\childdocjob}
% The macro |\childdocname| stores the name of the main document
% to be compiled. The macro |\childdocjob| stores the name of
% the document on which the \LaTeX{} compiler was originally invoked.
% The content of |\jobname| cannot be compared
% to filenames specified in the source due to different catcodes.
% The following code rescans |\jobname|, stores the result
% in |\childdocname| and saves a copy in |\childdocjob|:
%    \begin{macrocode}
\edef\childdocname{\scantokens\expandafter{\jobname\noexpand}}
\let\childdocjob\childdocname
%    \end{macrocode}

% \macro{\childdocdisable}
% The macro |\childdocdisable| prevents the main file
% from being processed more than once.
% At this stage, the main document command |\childdocmain|
% is assumed to be called once again where it should do nothing.
% Any subsequent call to it should prevent
% a secondary processing of the main document
% It overwrites the forwarding commands
% |\childdocof| and |\childdocforward|
% with empty macros to prevent further inclusions of the main document:
%    \begin{macrocode}
\newcommand{\childdocdisable}
{
  \renewcommand{\childdocmain}[1]{\renewcommand{\childdocmain}[1]{\endinput}}
  \renewcommand{\childdocof}[1]{}
  \renewcommand{\childdocby}[2][]{}
  \renewcommand{\childdocforward}[2][]{}
  \renewcommand{\childdocdisable}{}
}
%    \end{macrocode}

% \macro{\childdocmain}
% The macro |\childdocmain| is to be called at the top of the main file
% with nothing or the main filename (without extension) as argument.
% First, it breaks loops.
% If the argument is not empty and does not match |\childdocname|
% (which is set by the first inclusion of |childdoc.def|),
% |\ifchilddoc| is set to true, |\includeonly| is applied to the child file
% and |\jobname| is set to the main file
% (for proper handling of |.aux| files):
%    \begin{macrocode}
\newcommand{\childdocmain}[1]
{
  \childdocdisable\childdocmain{}
  \if?#1?\else
    \begingroup
      \def\childdoctmp{#1}
      \ifx\childdoctmp\childdocname
        \def\childdoctmp{}
      \else
        \def\childdoctmp
        {
          \childdoctrue
          \includeonly{\childdocname}
          \def\childdocjob{#1}
          \def\jobname{#1}
        }
      \fi
      \expandafter
    \endgroup
    \childdoctmp
  \fi
}
%    \end{macrocode}

% \macro{\childdocof}
% The command |\childdocof| redirects
% compilation to the main file |#1|.
%    \begin{macrocode}
\newcommand{\childdocof}[1]
{
  \childdocdisable
  \childdoctrue
  \includeonly{\childdocname}
  \def\jobname{#1}
  \def\childdocjob{#1}
  \input{#1}
}
%    \end{macrocode}

% \macro{\childdocby}
% The command |\childdocby| ....
%    \begin{macrocode}
\newcommand{\childdocby}[2][]
{
  \childdocdisable
  \childdoctrue
  \childdocmanualtrue
  \if?#1?\else
    \def\jobname{#2}
  \fi
  \def\childdocjob{#2}
  \input{#2}
  \endinput
}
%    \end{macrocode}

% \macro{\childdocforward}
% The command |\childdocforward| redirects
% compilation to the main file or
% (if the optional argument is given) a child file.
% Parameters are set as if the main file
% or a child file starting with |\childdocof| was compiled.
% Then compilation is handed over to the main file:
%    \begin{macrocode}
\newcommand{\childdocforward}[2][]
{
  \begingroup
    \if?#1?
      \def\childdoctmp
      {
        \def\childdocname{#2}
        \def\childdocjob{#2}
        \def\jobname{#2}
        \input{#2}
        \endinput
      }
    \else
      \def\childdoctmp
      {
        \childdocdisable
        \def\childdocname{#2}
        \childdoctrue
        \includeonly{#2}
        \def\childdocjob{#1}
        \def\jobname{#1}
        \input{#1}
        \endinput
      }
    \fi
    \expandafter
  \endgroup
  \childdoctmp
}
%    \end{macrocode}

% \macro{\childdocforwardprefix}
% The command |\childdocforwardprefix| redirects
% compilation to the main or a child file by means of a pattern.
% The prefix |#1| in the current filename is replaced by |#2|
% and the suffix of the current filename is kept
% (it is assumed that the filename does not contain the substring `|~~~|'
% which is used as a delimiter).
% Compilation is handed over to the new file by |\childdocforward|:
%    \begin{macrocode}
\newcommand{\childdocforwardprefix}[3][]
{
  \begingroup
    \def\childdocextract #2##1~~~{\def\childdoctmp{\childdocforward[#1]{#3##1}}}
    \expandafter\childdocextract\childdocname~~~
    \expandafter
  \endgroup
  \childdoctmp
}
%    \end{macrocode}

% \macro{\childdoc}
% The deprecated macro |\childdoc| is a legacy version of |\childdocmain|:
%    \begin{macrocode}
\newcommand{\childdoc}{\childdocmain}
%    \end{macrocode}

% \macro{\childdocredirect}
% The deprecated macro |\childdocredirect| is a legacy version
% of |\childdocforward| and |\childdocforwardprefix|:
%    \begin{macrocode}
\newcommand{\childdocredirect}[2][]
{
  \begingroup
    \if?#1?
      \def\childdoctmp{\childdocforward{#2}}
    \else
      \def\childdoctmp{\childdocforwardprefix{#1}{#2}}
    \fi
    \expandafter
  \endgroup
  \childdoctmp
}
%    \end{macrocode}

%\iffalse
%</package>
%\fi
%
\endinput
|\\
|\childdocforwardprefix[|\textit{main}|]{|\textit{prefix}|}{|\textit{dest}|}|
\end{tabular}
\end{center}
%
the destination file is determined by a pattern
depending on the current file:
To make this work, the current file must be called
`{\textit{prefix}\hspace{0.2em}\textit{suffix}}'
with \textit{prefix} matching precisely the argument.
Processing is then passed on to the file
`{\textit{dest}\hspace{0.2em}\textit{suffix}}'.
Surely, the same effect is achieved by
directly specifying the
argument `{\textit{dest}\hspace{0.2em}\textit{suffix}}'
in the first form.
However, that requires to set up a different file
for each child. With the alternative form of the command
all these files can have exactly the same content
which simplifies setting them up and maintaining them.

For example, the following file |draft.tex|
with a compilation flag |\version| as described in \secref{sec:flags}
compiles the main document as a draft:
%
\begin{center}
\begin{tabular}{l}
|\def\version{draft}|\\
|% \iffalse
%
% childdoc.dtx Copyright (C) 2017-2018 Niklas Beisert
%
% This work may be distributed and/or modified under the
% conditions of the LaTeX Project Public License, either version 1.3
% of this license or (at your option) any later version.
% The latest version of this license is in
%   http://www.latex-project.org/lppl.txt
% and version 1.3 or later is part of all distributions of LaTeX
% version 2005/12/01 or later.
%
% This work has the LPPL maintenance status `maintained'.
%
% The Current Maintainer of this work is Niklas Beisert.
%
% This work consists of the files childdoc.dtx and childdoc.ins
% and the derived files childdoc.def and cdocsamp.tex with
% cdocsch1.tex, cdocsch2.tex, cdocsdrf.tex, cdocsfn1.tex, cdocsfn2.tex.
%
%<package>\ifdefined\childdocmain\endinput\fi
%<package>\ProvidesFile{childdoc.def}[2018/12/30 v2.0 child document driver]
%<samplemain>\ProvidesFile{cdocsamp.tex}[2018/12/30 v2.0 sample for childdoc]
%<*driver>
%\ProvidesFile{childdoc.drv}[2018/12/30 v2.0 childdoc reference manual file]
\PassOptionsToClass{10pt,a4paper}{article}
\documentclass{ltxdoc}

\usepackage[margin=35mm]{geometry}
\usepackage{hyperref}
\usepackage{hyperxmp}
\usepackage[usenames]{color}

\hypersetup{colorlinks=true}
\hypersetup{pdfstartview=FitH}
\hypersetup{pdfpagemode=UseNone}
\hypersetup{pdfsource={}}
\hypersetup{pdflang={en-UK}}
\hypersetup{pdfcopyright={Copyright 2017-2018 Niklas Beisert.
  This work may be distributed and/or modified under the
  conditions of the LaTeX Project Public License, either version 1.3
  of this license or (at your option) any later version.}}
\hypersetup{pdflicenseurl={http://www.latex-project.org/lppl.txt}}
\hypersetup{pdfcontactaddress={ETH Zurich, ITP, HIT K,
  Wolfgang-Pauli-Strasse 27}}
\hypersetup{pdfcontactpostcode={8093}}
\hypersetup{pdfcontactcity={Zurich}}
\hypersetup{pdfcontactcountry={Switzerland}}
\hypersetup{pdfcontactemail={nbeisert@itp.phys.ethz.ch}}
\hypersetup{pdfcontacturl={http://people.phys.ethz.ch/\xmptilde nbeisert/}}

\newcommand{\secref}[1]{\hyperref[#1]{section \ref*{#1}}}

\parskip1ex
\parindent0pt
\let\olditemize\itemize
\def\itemize{\olditemize\parskip0pt}

\begin{document}

\title{The \textsf{childdoc} Package}
\hypersetup{pdftitle={The childdoc Package}}
\author{Niklas Beisert\\[2ex]
  Institut f\"ur Theoretische Physik\\
  Eidgen\"ossische Technische Hochschule Z\"urich\\
  Wolfgang-Pauli-Strasse 27, 8093 Z\"urich, Switzerland\\[1ex]
  \href{mailto:nbeisert@itp.phys.ethz.ch}
  {\texttt{nbeisert@itp.phys.ethz.ch}}}
\hypersetup{pdfauthor={Niklas Beisert}}
\hypersetup{pdfsubject={Manual for the LaTeX2e Package childdoc}}
\date{30 December 2018, \textsf{v2.0}}
\maketitle

\begin{abstract}\noindent
\textsf{childdoc} is a \LaTeXe{} package
that enables the direct compilation
of document sections included by |\include|
to individual files.
\end{abstract}

\begingroup
\parskip0ex
\tableofcontents
\endgroup

%%%%%%%%%%%%%%%%%%%%%%%%%%%%%%%%%%%%%%%%%%%%%%%%%%%%%%%%%%%%%%%%%%%%%%%%%%%%%%%%
%%%%%%%%%%%%%%%%%%%%%%%%%%%%%%%%%%%%%%%%%%%%%%%%%%%%%%%%%%%%%%%%%%%%%%%%%%%%%%%%
\section{Introduction}

\LaTeX{} provides a mechanism to structure a large document (such as a book)
into a main file and several child files (containing the chapters)
using the |\include| command.
This mechanism is beneficial for documents
which span hundreds of pages in order to
make the source file(s) more manageable.
Moreover, compilation can be restricted to
selected child files by means of the |\includeonly| command.
The latter feature can be used to reduce the compilation time while editing
(this was significantly more useful in the earlier days of \LaTeX{})
or to generate a smaller document which is easier to navigate.
Another application of |\includeonly| is to generate
documents consisting of selected parts of the complete document.

However, there are a few drawbacks of the plain |\include| mechanism:
\begin{itemize}
\item
The child files cannot be compiled on their own,
they can only be compiled via the main file.
A naive editing environment
(such as a text editor with an option
to have the current file processed by \LaTeX)
may require one to switch to the main file before compiling;
attempting to compile the child file produces errors.
\item
The main file must be modified (each time)
to adjust the |\includeonly| command
to the present needs. This easily leaves the main file in a messy state.
\item
The generated document will always carry the filename
of the main document. This is inconvenient if
several child files are to be compiled and
to be kept for distribution.
\end{itemize}

The present package provides a simple interface
to make child files individually compilable by \LaTeX{}.
Compiling a child file then has the same effect as compiling
the main file with an |\includeonly| command
to select the appropriate child.
Moreover the generated document will carry the name of the child
rather than the main file.
This resolves all three above issues.

This feature is meant to make the editing of books,
thesis documents and lecture notes somewhat more convenient.
However, the package can also be used efficiently for
composing a series of documents (such as exercise sheets)
which are typically distributed individually.
It then assists the author in generating the individual documents
(potentially in different versions)
as well as a document containing the collected series.
Another application is in developing style files
or other kinds of included material
where compilation of the style file could redirect
to a sample or test file.

%%%%%%%%%%%%%%%%%%%%%%%%%%%%%%%%%%%%%%%%%%%%%%%%%%%%%%%%%%%%%%%%%%%%%%%%%%%%%%%%
%%%%%%%%%%%%%%%%%%%%%%%%%%%%%%%%%%%%%%%%%%%%%%%%%%%%%%%%%%%%%%%%%%%%%%%%%%%%%%%%
\section{Usage}

First of all, the package \textsf{childdoc} is \emph{not} a standard
\LaTeXe{} |.sty| style file! Therefore it needs to be invoked in
a non-standard way.

%%%%%%%%%%%%%%%%%%%%%%%%%%%%%%%%%%%%%%%%%%%%%%%%%%%%%%%%%%%%%%%%%%%%%%%%%%%%%%%%
\subsection{Included Files}
\label{sec:include}

%%%%%%%%%%%%%%%%%%%%%%%%%%%%%%%%%%%%%%%%
\DescribeMacro{\childdocmain}
To use the package, add the commands
\begin{center}
\begin{tabular}{l}
|\input{childdoc.def}|\\
|\childdocmain{}|\\
\end{tabular}
\end{center}
at the very top of the main \LaTeX{} file,
in particular \emph{before} the |\documentclass| statement!
The argument of |\childdocmain| should be left empty
(but it must be present).

%%%%%%%%%%%%%%%%%%%%%%%%%%%%%%%%%%%%%%%%
\DescribeMacro{\childdocof}
Furthermore, add the commands
\begin{center}
\begin{tabular}{l}
|\input{childdoc.def}|\\
|\childdocof{|\textit{main}|}|\\
\end{tabular}
\end{center}
at the top of every child file \textit{child}
which is included by |\include{|\textit{child}|}|
from within the main file
(or at least for those files to be compiled individually).
The argument \textit{main} must be the filename of the main file.

There are a couple of
considerations in setting up the main and child documents:

%%%%%%%%%%%%%%%%%%%%%%%%%%%%%%%%%%%%%%%%
\paragraph{Restrictions.}

Please note the following restrictions:
\begin{itemize}
\item
|\childdocmain| must be called with one argument \textit{main}
to ensure compatibility with earlier version of the package.
It must either be empty (|\childdocmain{}|)
or precisely match the filename of the main file in which it is specified.
See \secref{sec:detection} for further information.
\item
The filename \textit{main} must be specified without the |.tex| extension.
\item
The filename \textit{main} is case sensitive
(even in case-insensitive file systems)
due to internal string comparison.
\item
The argument \textit{main} should be fully expanded, it cannot be a macro.
\item
Subdirectories and special characters should be avoided in filenames.
\item
The command |\childdocmain{|\textit{main}|}| must be followed by a whitespace.
It should not be followed immediately by another command
or by a comment mark `|%|'.
This is because the \TeX{} parser reads the token immediately following
the argument of |\childdocmain| and puts it
at the beginning of every child section;
however, a white\-space is ignored.
\end{itemize}

%%%%%%%%%%%%%%%%%%%%%%%%%%%%%%%%%%%%%%%%
\paragraph{Content of Main File.}

It is advisable to place all content in the child files included by |\include|.
Any output contained in the main file will appear in all child documents
unless suppressed manually;
it cannot be suppressed automatically by the |\includeonly| directive
and thus should normally be avoided.
A method to include some content in the main file
by means of conditional processing is described in \secref{sec:conditional}.

%%%%%%%%%%%%%%%%%%%%%%%%%%%%%%%%%%%%%%%%
\paragraph{Page Numbering.}

When only a part of the document is compiled,
the appropriate numbering of pages
(as well as other status parameters)
is determined from the |.aux| files.
The latter contain information from previous passes.
However this information needs to propagate through
all intermediate child documents.
Therefore the page numbering in child documents may well
be inconsistent until the complete document is compiled at least once.

A useful (if unconventional) way to always ensure a consistent
page numbering is to restart the numbering in each child document
and denote the pages by `\textit{child}|.|\textit{page}'
where \textit{child} represents the chapter/section number of the child file.
This can be achieved by the command
|\numberwithin{page}{|\textit{child}|}|
of the \textsf{amsmath} package
where \textit{child} can be |chapter| or |section|
depending on the chosen structuring.
Alternatively, one can modify the macro |\thepage| appropriately
and reset the counter |page| at the start of each child file.

%%%%%%%%%%%%%%%%%%%%%%%%%%%%%%%%%%%%%%%%%%%%%%%%%%%%%%%%%%%%%%%%%%%%%%%%%%%%%%%%
\subsection{Conditional Processing}
\label{sec:conditional}

The package provides a mechanism to compile different versions
of a document. To customise the versions further some conditional processing
can come in handy to distinguish which version is being compiled.
The package provides two macros to describe the compilation context:

%%%%%%%%%%%%%%%%%%%%%%%%%%%%%%%%%%%%%%%%
\DescribeMacro{\ifchilddoc}
The conditional |\ifchilddoc| distinguishes between the compilation of
child documents and the main document:
%
\begin{center}
|\ifchilddoc |\textit{child-code}| |[|\||else |\textit{main-code}]| \||fi|
\end{center}

%%%%%%%%%%%%%%%%%%%%%%%%%%%%%%%%%%%%%%%%
\DescribeMacro{\childdocname}
\DescribeMacro{\childdocjob}
The macro |\childdocname| contains the filename (without extension)
of the main or child file being processed.
Note that |\childdocjob| will always contain the name of the main file.

%%%%%%%%%%%%%%%%%%%%%%%%%%%%%%%%%%%%%%%%
\paragraph{Title Page.}

Conditional processing can be used to include a title or banner page
in the main document when proper precautions are taken.
Importantly, the code in the main file should ensure that the page counter
(as well as other status parameters which are stored in the |.aux| files)
takes the same value after the conditional processing.
Otherwise the page numbers may take divergent values
depending on which part is compiled.

For example, a title page could be declared by:
%
\begin{center}
\begin{tabular}{l}
|\ifchilddoc\||else|\\
|\addtocounter{page}{-1}|\\
\textit{code for title page}\\
|\newpage|\\
|\||fi|
\end{tabular}
\end{center}
%
A banner page for the child documents can be generated by:
%
\begin{center}
\begin{tabular}{l}
|\ifchilddoc|\\
|\addtocounter{page}{-1}|\\
\textit{code for banner page}\\
|\newpage|\\
|\||fi|
\end{tabular}
\end{center}
%
Here one could write a message such as:
\begin{center}
|This is the part \childdocname{} of \childdocjob{}.|
\end{center}

%%%%%%%%%%%%%%%%%%%%%%%%%%%%%%%%%%%%%%%%%%%%%%%%%%%%%%%%%%%%%%%%%%%%%%%%%%%%%%%%
\subsection{Flags}
\label{sec:flags}

The package makes it easy to generate different versions
of the main or child documents.
To this end compilation flags can be defined
and assigned different default values.
They will be particularly useful in conjunction
with the forwarding mechanism described in \secref{sec:forward}.

For example, it may be useful to have a flag |\version|
which can be set to |draft| or |final|.
The document source will contain some conditional code
depending on the value of |\version|.
Suppose further, the flag should default to |final| for the main file
and to |draft| for child files
which is a natural assignment for editing the document.
This is achieved by placing the following code
in the preamble of the main document
(below the |\childdocmain| directive):
%
\begin{center}
\begin{tabular}{l}
|\ifchilddoc|\\
|\providecommand{\version}{draft}|\\
|\||else|\\
|\providecommand{\version}{final}|\\
|\||fi|
\end{tabular}
\end{center}
%
The definition by |\providecommand| makes sure
that previous definitions are not overwritten.
Further statements |\providecommand{\version}{...}|
can thus be added before the above code to override it.

For the main file, one might add a line
(between |\childdocmain| and the above block)
%
\begin{center}
|%\ifchilddoc\||else\providecommand{\version}{draft}\||fi|
\end{center}
%
which can be uncommented to produce a draft version.
Likewise one can add a line to the very top of a child file
(above the |\childdocof{|\textit{main}|}| directive)
%
\begin{center}
|%\providecommand{\version}{final}|
\end{center}
%
which can be uncommented to produce the final version of this child document.

%%%%%%%%%%%%%%%%%%%%%%%%%%%%%%%%%%%%%%%%%%%%%%%%%%%%%%%%%%%%%%%%%%%%%%%%%%%%%%%%
\subsection{Forwarding}
\label{sec:forward}

Different versions of the main or child documents
using compilation flags as described in \secref{sec:flags}
can be (permanently) stored in different files
for convenient compilation, viewing and distribution.
To this end, the package defines a command
to pass on compilation to a different file:

%%%%%%%%%%%%%%%%%%%%%%%%%%%%%%%%%%%%%%%%
\DescribeMacro{\childdocforward}
The command |\childdocforward| redirects processing to
another source file:
%
\begin{center}
\begin{tabular}{l}
|\input{childdoc.def}|\\
|\childdocforward[|\textit{main}|]{|\textit{dest}|}|\\
\end{tabular}
\end{center}
%
The argument \textit{dest} is the destination file
(without extension).
It should be the main file or one of the child files.
Note that further \textsf{childdoc} directives
such as |\childdocof| and |\childdocforward|
in the indicated file will be processed in this form.
The optional argument \textit{main}
passes on directly to the main file \textit{main}
while pretending to compile the child \textit{dest}.
This form behaves as if \textit{dest}
issues |\childdocof{|\textit{main}|}| right away,
and no further \textsf{childdoc} directives will be processed.

%%%%%%%%%%%%%%%%%%%%%%%%%%%%%%%%%%%%%%%%
\DescribeMacro{\...prefix}
In the alternative form |\childdocforwardprefix|,
%
\begin{center}
\begin{tabular}{l}
|\input{childdoc.def}|\\
|\childdocforwardprefix[|\textit{main}|]{|\textit{prefix}|}{|\textit{dest}|}|
\end{tabular}
\end{center}
%
the destination file is determined by a pattern
depending on the current file:
To make this work, the current file must be called
`{\textit{prefix}\hspace{0.2em}\textit{suffix}}'
with \textit{prefix} matching precisely the argument.
Processing is then passed on to the file
`{\textit{dest}\hspace{0.2em}\textit{suffix}}'.
Surely, the same effect is achieved by
directly specifying the
argument `{\textit{dest}\hspace{0.2em}\textit{suffix}}'
in the first form.
However, that requires to set up a different file
for each child. With the alternative form of the command
all these files can have exactly the same content
which simplifies setting them up and maintaining them.

For example, the following file |draft.tex|
with a compilation flag |\version| as described in \secref{sec:flags}
compiles the main document as a draft:
%
\begin{center}
\begin{tabular}{l}
|\def\version{draft}|\\
|\input{childdoc.def}|\\
|\childdocforward{|\textit{main}|}|
\end{tabular}
\end{center}
%
Likewise, the following files |final|\textit{nn}|.tex|
compile the final version of the child document
|child|\textit{nn}|.tex|:
%
\begin{center}
\begin{tabular}{l}
|\def\version{final}|\\
|\input{childdoc.def}|\\
|\childdocforwardprefix{final}{child}|
\end{tabular}
\end{center}
%

Note that when several versions of a main file and/or of each child file
are to be generated, it may be convenient to set up a |Makefile| or
shell script to automatise the process.

%%%%%%%%%%%%%%%%%%%%%%%%%%%%%%%%%%%%%%%%%%%%%%%%%%%%%%%%%%%%%%%%%%%%%%%%%%%%%%%%
\subsection{Command Line Processing}
\label{sec:commandline}

The effect of redirection files can also be achieved by invoking
the \LaTeX{} compiler with a more elaborate command line.
Most conveniently this should be done as part
of a shell script or a |Makefile|.

When using \textsf{childdoc} in the main file, the following
command lines effectively perform a redirection
(note that depending on the shell being used,
backslashes may have to be doubled: `|\|' $\to$ `|\\|'):
%
\begin{center}
|... -jobname "|\textit{target}|" |\\|"|[\textit{flags}]%
|\input{childdoc.def}\childdocforward[|\textit{main}|]{|\textit{dest}|}"|
\end{center}
%
Here \textit{target} is the name of the output file,
\textit{main} is the name of the main file
and \textit{dest} is the name of the main or child file to be processed
(all filenames without extensions).
The optional argument \textit{main} can be omitted
if \textit{main} matches \textit{dest}.
Optionally, compilation \textit{flags} can be defined via |\def| commands.
This command line makes the \TeX{} engine believe
it is compiling the file \textit{target}
whose content is specified as the latter parameter.
The provided code then forwards the processing to
\textit{main} or \textit{dest} as described in \secref{sec:forward}.

%%%%%%%%%%%%%%%%%%%%%%%%%%%%%%%%%%%%%%%%%%%%%%%%%%%%%%%%%%%%%%%%%%%%%%%%%%%%%%%%
\subsection{Include by Input}
\label{sec:input}

Including child documents by |\include| has some restrictions by design.
Most notably, the content of a child document always occupies
its own set of pages; pages cannot be shared between child documents.
Usually, this behaviour makes perfect sense
because each child document contain an essential part of the document.
However, in some situations it may be desirable to compose
a document from a collection of parts
without having mandatory page breaks between then.
For this case, the package
provides a mechanism to include parts
by |\input| which can also be processed individually.
However, by construction this mechanism
requires manual handling of the content to be output.

%%%%%%%%%%%%%%%%%%%%%%%%%%%%%%%%%%%%%%%%
\DescribeMacro{\ifchilddocmanual}
The main file should be prepared as usual, see \secref{sec:include}.
However, the document body must make a distinction
between processing of an individual part and of the main document, e.g.:
%
\begin{center}
\begin{tabular}{l}
|\ifchilddocmanual|\\
|\input{\childdocname}|\\
|\||else|\\
\textit{document body with }|\input{|\textit{part}|}|\\
|\||fi|
\end{tabular}
\end{center}
%
The conditional |\ifchilddocmanual| is true whenever
a part to be included by |\input| is being compiled,
and the name of the part is stored in |\childdocname|.

%%%%%%%%%%%%%%%%%%%%%%%%%%%%%%%%%%%%%%%%
\DescribeMacro{\childdocby}
Each part to be included by |\input| should start with:
%
\begin{center}
\begin{tabular}{l}
|\input{childdoc.def}|\\
|\childdocby{|\textit{main}|}|\\
\end{tabular}
\end{center}
%
The directive |\childdocby| is similar to |\childdocof|
described in \secref{sec:include},
but the subsequent selection of content must be done manually.
To that end, both |\ifchilddoc| and |\ifchilddocmanual|
will be true upon processing of a part,
and the name of the part is stored in |\childdocname|.
Note that |\jobname| will be set to the filename of the current part
so that each part receives an individual |.aux| file
that does not interfere with the |.aux| file(s) of the main document.
This behaviour can be altered by the alternative form
|\childdocby[*]{|\textit{main}|}| (with a non-empty optional argument)
which uses the |.aux| file of the main document
by setting |\jobname| to \textit{main}.

%%%%%%%%%%%%%%%%%%%%%%%%%%%%%%%%%%%%%%%%%%%%%%%%%%%%%%%%%%%%%%%%%%%%%%%%%%%%%%%%
\subsection{Driver Development}
\label{sec:driver}

The \textsf{childdoc} mechanism can also be use for the development
of definition files such as \LaTeX{} styles or classes.
This case differs from the above setup with multiple parts
included by |\include| in that no |\includeonly| should be invoked.
This can be achieved by starting the include file
(before |\ProvidesPackage|) with:
%
\begin{center}
\begin{tabular}{l}
|\input{childdoc.def}|\\
|\childdocforward{|\textit{main}|}|\\
\end{tabular}
\end{center}
%
or alternatively with:
%
\begin{center}
\begin{tabular}{l}
|\input{childdoc.def}|\\
|\childdocby{|\textit{main}|}|\\
\end{tabular}
\end{center}
%
Both forms have slightly different effects as described above.
The main file is prepared as usual, see \secref{sec:include}.

%%%%%%%%%%%%%%%%%%%%%%%%%%%%%%%%%%%%%%%%%%%%%%%%%%%%%%%%%%%%%%%%%%%%%%%%%%%%%%%%
\subsection{Legacy Detection}
\label{sec:detection}

The directive |\childdocmain| in the main file can detect
whether the complete document or merely a child is to be compiled
even without using the directive |\childdocof|.
This method is deprecated because it is less robust
and there is no compelling reason to use it;
it is merely provided for backward compatibility
and it may be removed in future versions.

If the detection mechanism is to be used,
it is mandatory to correctly specify
the filename of the main file as the argument of |\childdocmain|:
%
\begin{center}
\begin{tabular}{l}
|\input{childdoc.def}|\\
|\childdocmain{|\textit{main}|}|\\
\end{tabular}
\end{center}
%
If |\jobname| does not match the argument \textit{main} of |\childdocmain|,
it is assumed that |\jobname| points to the child file to be compiled.
When using |\childdocmain| with the main file specified as argument,
it suffices to start a child file
with just |\input{|\textit{main}|}|
without loading of the package and using |\childdocof|.
If instead all processing is done
with the appropriate \textsf{childdoc} directives,
the argument of \textit{main} of |\childdocmain| can be empty.

An alternative version of the command line processing described
in \secref{sec:commandline} using the detection mechanism reads:
%
\begin{center}
|... -jobname "|\textit{target}|" "|[\textit{flags}]%
[|\def\jobname{|\textit{dest}|}|]|\input{|\textit{main}|}"|
\end{center}

%%%%%%%%%%%%%%%%%%%%%%%%%%%%%%%%%%%%%%%%%%%%%%%%%%%%%%%%%%%%%%%%%%%%%%%%%%%%%%%%
\subsection{Manual Code}
\label{sec:manual}

In case one cannot be certain whether the definitions file |childdoc.def|
is installed on the target \TeX{} distribution
and one prefers not to ship it,
it is conceivable to paste a few relevant commands into the sources.

To that end, drop all statements |\input{childdoc.def}|
and perform the replacements as outlined below.
Instead of |\childdocmain{|\textit{main}|}| add the following code
to the top of the main file:
%
\begin{center}
\begin{tabular}{l}
|\||ifdefined\childdocname\endinput\||fi\newif\ifchilddoc|\\
|\edef\childdocname{\scantokens\expandafter{\jobname\noexpand}}|\\
|\def\childdocmain{|\textit{main}|}\||ifx\childdocmain\childdocname\||else|\\
|\childdoctrue\includeonly{\childdocname}\let\jobname\childdocmain\||fi|\\
\end{tabular}
\end{center}
%
Instead of |\childdocof{|\textit{main}|}| just include the main file
at the top of each child file:
%
\begin{center}
|\input{|\textit{main}|}|
\end{center}
%
A simple redirection |\childdocforward{|\textit{dest}|}| is achieved by:
%
\begin{center}
|\def\jobname{|\textit{dest}|}\input{\jobname}|
\end{center}
%
The redirection with prefix
|\childdocforwardprefix[|\textit{prefix}|]{|\textit{dest}|}|
is accomplished by:
%
\begin{center}
\begin{tabular}{l}
|{\edef\jobname{\scantokens\expandafter{\jobname\noexpand}}|\\
|\def\redirectjob |\textit{prefix}|#1~~~{\gdef\jobname{|\textit{dest}|#1}}|\\
|\expandafter\redirectjob\jobname~~~}\input{\jobname}|
\end{tabular}
\end{center}

In an alternative approach,
child documents can be compiled by a specific command line
without additional code or specific definitions:
%
\begin{center}
|... -jobname "|\textit{target}|" "|[\textit{flags}]%
|\includeonly{|\textit{dest}|}\input{|\textit{main}|}"|
\end{center}
%

%%%%%%%%%%%%%%%%%%%%%%%%%%%%%%%%%%%%%%%%%%%%%%%%%%%%%%%%%%%%%%%%%%%%%%%%%%%%%%%%
%%%%%%%%%%%%%%%%%%%%%%%%%%%%%%%%%%%%%%%%%%%%%%%%%%%%%%%%%%%%%%%%%%%%%%%%%%%%%%%%
\section{Information}

%%%%%%%%%%%%%%%%%%%%%%%%%%%%%%%%%%%%%%%%%%%%%%%%%%%%%%%%%%%%%%%%%%%%%%%%%%%%%%%%
\subsection{Copyright}

Copyright \copyright{} 2017--2018 Niklas Beisert

This work may be distributed and/or modified under the
conditions of the \LaTeX{} Project Public License, either version 1.3
of this license or (at your option) any later version.
The latest version of this license is in
  \url{http://www.latex-project.org/lppl.txt}
and version 1.3 or later is part of all distributions of \LaTeX{}
version 2005/12/01 or later.

This work has the LPPL maintenance status `maintained'.

The Current Maintainer of this work is Niklas Beisert.

This work consists of the files |README.txt|, |childdoc.ins| and |childdoc.dtx|
as well as the derived files |childdoc.def|, |cdocsamp.tex|
with |cdocsch1.tex|, |cdocsch2.tex|, |cdocspt3.tex|, |cdocspt4.tex|,
|cdocsdrf.tex|, |cdocsfn1.tex|, |cdocsfn2.tex|
as well as |childdoc.pdf|.

%%%%%%%%%%%%%%%%%%%%%%%%%%%%%%%%%%%%%%%%%%%%%%%%%%%%%%%%%%%%%%%%%%%%%%%%%%%%%%%%
\subsection{Files and Installation}

The package consists of the files:
%
\begin{center}
\begin{tabular}{ll}
    |README.txt|   & readme file \\
    |childdoc.ins| & installation file \\
    |childdoc.dtx| & source file \\
    |childdoc.def| & definition file \\
    |cdocsamp.tex| & sample main file \\
    |cdocsch1.tex| & sample include file \\
    |cdocsch2.tex| & sample include file \\
    |cdocspt3.tex| & sample part file \\
    |cdocspt4.tex| & sample part file \\
    |cdocsdrf.tex| & sample redirection file \\
    |cdocsfn1.tex| & sample redirection file \\
    |cdocsfn2.tex| & sample redirection file \\
    |childdoc.pdf| & manual
\end{tabular}
\end{center}
%
The distribution consists of the files
|README.txt|, |childdoc.ins| and |childdoc.dtx|.
%
\begin{itemize}
\item
Run (pdf)\LaTeX{} on |childdoc.dtx|
to compile the manual |childdoc.pdf| (this file).
\item
Run \LaTeX{} on |childdoc.ins| to create the definitions file |childdoc.def|
and the sample |cdocsamp.tex| with include files
|cdocsch1.tex|, |cdocsch2.tex|, |cdocspt3.tex|, |cdocspt4.tex|,
|cdocsdrf.tex|, |cdocsfn1.tex|, |cdocsfn2.tex|.
Then copy the file |childdoc.def| to an appropriate directory of your \LaTeX{}
distribution, e.g.\ \textit{texmf-root}|/tex/latex/childdoc|.
\end{itemize}

%%%%%%%%%%%%%%%%%%%%%%%%%%%%%%%%%%%%%%%%%%%%%%%%%%%%%%%%%%%%%%%%%%%%%%%%%%%%%%%%
\subsection{Related CTAN Packages}

There are several other packages which offer a similar functionality:
%
\begin{itemize}
\item
The packages
\href{http://ctan.org/pkg/docmute}{\textsf{docmute}},
\href{http://ctan.org/pkg/includex}{\textsf{includex}} and
\href{http://ctan.org/pkg/standalone}{\textsf{standalone}}
provide commands to include only the document body of
a child file thus allowing both files to be compiled individually.
\item
The packages \href{http://ctan.org/pkg/subdocs}{\textsf{subdocs}}
and \href{http://ctan.org/pkg/subfiles}{\textsf{subfiles}}
provide structures in which the main and child documents can be
encapsulated and allowing them to be compiled individually.
The inclusion mechanism is different from the conventional |\include|.
\item
The package \href{http://ctan.org/pkg/combine}{\textsf{combine}}
is an elaborate solution to combine several documents into one.
\end{itemize}
%
See also the CTAN topic \href{http://ctan.org/topic/subdocs}{\textsf{subdocs}}
for further related packages.
The present package differs from the above solutions in that
a document structure constructed with the conventional |\include| mechanism
just needs two extra commands at the top of every file
such that all constituent files can be compiled individually.

%%%%%%%%%%%%%%%%%%%%%%%%%%%%%%%%%%%%%%%%%%%%%%%%%%%%%%%%%%%%%%%%%%%%%%%%%%%%%%%%
%\subsection{Feature Suggestions}
%
%The following is a list of features which may be useful for future
%versions of this package:
%%
%\begin{itemize}
%\item
%\ldots
%\end{itemize}

%%%%%%%%%%%%%%%%%%%%%%%%%%%%%%%%%%%%%%%%%%%%%%%%%%%%%%%%%%%%%%%%%%%%%%%%%%%%%%%%
\subsection{Revision History}

%%%%%%%%%%%%%%%%%%%%%%%%%%%%%%%%%%%%%%%%
\paragraph{v2.0:} 2018/12/30

\begin{itemize}
\item
immediate forward processing
\item
added |\childdocby| mechanism
\item
manual restructured
\end{itemize}

%%%%%%%%%%%%%%%%%%%%%%%%%%%%%%%%%%%%%%%%
\paragraph{v1.6:} 2018/01/17

\begin{itemize}
\item
application for development of include files
\item
corrections to manual
\end{itemize}

%%%%%%%%%%%%%%%%%%%%%%%%%%%%%%%%%%%%%%%%
\paragraph{v1.5:} 2017/05/21

\begin{itemize}
\item
more complete structuring introduced
\item
|\childdocof| introduced
\item
|\childdoc| renamed to |\childdocmain|
\item
|\childredirect| renamed to |\childdocforward| and |\childdocforwardprefix|
and functionality expanded
\end{itemize}

%%%%%%%%%%%%%%%%%%%%%%%%%%%%%%%%%%%%%%%%
\paragraph{v1.0:} 2017/04/27

\begin{itemize}
\item
manual and install package
\item
first version published on CTAN
\end{itemize}

%%%%%%%%%%%%%%%%%%%%%%%%%%%%%%%%%%%%%%%%
\paragraph{v0.6:} 2017/04/26

\begin{itemize}
\item
redirection mechanism added
\end{itemize}

%%%%%%%%%%%%%%%%%%%%%%%%%%%%%%%%%%%%%%%%
\paragraph{v0.5:} 2017/04/26

\begin{itemize}
\item
functionality in definition file
\end{itemize}


%%%%%%%%%%%%%%%%%%%%%%%%%%%%%%%%%%%%%%%%%%%%%%%%%%%%%%%%%%%%%%%%%%%%%%%%%%%%%%%%
%%%%%%%%%%%%%%%%%%%%%%%%%%%%%%%%%%%%%%%%%%%%%%%%%%%%%%%%%%%%%%%%%%%%%%%%%%%%%%%%
%%%%%%%%%%%%%%%%%%%%%%%%%%%%%%%%%%%%%%%%%%%%%%%%%%%%%%%%%%%%%%%%%%%%%%%%%%%%%%%%
\appendix

\settowidth\MacroIndent{\rmfamily\scriptsize 000\ }

 \DocInput{childdoc.dtx}

\end{document}
%</driver>
% \fi
%
% %%%%%%%%%%%%%%%%%%%%%%%%%%%%%%%%%%%%%%%%%%%%%%%%%%%%%%%%%%%%%%%%%%%%%%%%%%%%%%
% %%%%%%%%%%%%%%%%%%%%%%%%%%%%%%%%%%%%%%%%%%%%%%%%%%%%%%%%%%%%%%%%%%%%%%%%%%%%%%
% \section{Sample}
%\iffalse
%<*samplemain>
%\fi
%
% The following presents a sample document
% with two chapters, two parts, a title page,
% a compile flag as well as three forwarding files to set the flag.
% It consists of eight |.tex| files:
% \begin{center}
% \begin{tabular}{ll}
% |cdocsamp.tex|&main file\\
% |cdocsch1.tex|&include file for chapter 1\\
% |cdocsch2.tex|&include file for chapter 2\\
% |cdocspt3.tex|&include file for part 3\\
% |cdocspt4.tex|&include file for part 4\\
% |cdocsdrf.tex|&forwarding file for main file in draft mode\\
% |cdocsfi1.tex|&forwarding file for final version of chapter 1\\
% |cdocsfi2.tex|&forwarding file for final version of chapter 2\\
% \end{tabular}
% \end{center}
% Each of the eight files can be compiled directly by the \LaTeX{} compiler.
%
% %%%%%%%%%%%%%%%%%%%%%%%%%%%%%%%%%%%%%%
% \paragraph{Main File.}
%
% The main file is called |cdocsamp.tex|.
%
% Load the \textsf{childdoc} definitions and
% declare the filename for the main document:
%    \begin{macrocode}
\input{childdoc.def}
\childdocmain{}
%    \end{macrocode}

% Optional override for |\version| flag:
%    \begin{macrocode}
%%\ifchilddoc\else\providecommand{\version}{draft}\fi
%    \end{macrocode}

% Define the default values for the |\version| flag
% (|final| for the main file and |draft| for childs):
%    \begin{macrocode}
\ifchilddoc
\providecommand{\version}{draft}
\else
\providecommand{\version}{final}
\fi
%    \end{macrocode}

% Load the standard document class:
%    \begin{macrocode}
\documentclass[12pt]{article}
%    \end{macrocode}

% Start the document body:
%    \begin{macrocode}
\begin{document}
%    \end{macrocode}

% Declare a title page.
% Print title, part of document being processed and version flag:
%    \begin{macrocode}
\addtocounter{page}{-1}
\begin{center}
{\LARGE\bfseries{}childdoc example\par}
\vspace{1cm}
\ifchilddoc
\ifchilddocmanual part\else chapter\fi:
`\childdocname' of `\childdocjob'\par
\else
main document: `\childdocjob'\par
\fi
version: \version\par
\end{center}
\newpage
%    \end{macrocode}

% Manually include selected file,
% otherwise process as usual:
%    \begin{macrocode}
\ifchilddocmanual
\section*{part `\childdocname'}
\input{\childdocname}
\else
%    \end{macrocode}

% Include the two chapters:
%    \begin{macrocode}
\include{cdocsch1}
\include{cdocsch2}
%    \end{macrocode}

% Include the two parts unless only chapters should be displayed:
%    \begin{macrocode}
\ifchilddoc\else
\section{part three}
\input{cdocspt3}
\section{part four}
\input{cdocspt4}
\fi
%    \end{macrocode}

% Process as usual until here:
%    \begin{macrocode}
\fi
%    \end{macrocode}

% End of document body:
%    \begin{macrocode}
\end{document}
%    \end{macrocode}
%\iffalse
%</samplemain>
%\fi
%
% %%%%%%%%%%%%%%%%%%%%%%%%%%%%%%%%%%%%%%
% \paragraph{Chapter Include Files.}
%
% The include files are called |cdocsch1.tex| and |cdocsch2.tex|.
%
%\iffalse
%<*samplechap1|samplechap2>
%\fi

% Optional override for |\version| flag:
%    \begin{macrocode}
%%\providecommand{\version}{final}
%    \end{macrocode}

% Include the main document:
%    \begin{macrocode}
\input{childdoc.def}
\childdocof{cdocsamp}
%    \end{macrocode}

%\iffalse
%</samplechap1|samplechap2>
%\fi
%
%\iffalse
%<*samplechap1>
%\fi
% Some text for chapter 1:
%    \begin{macrocode}
\section{one}
some text in chapter one
%    \end{macrocode}

%\iffalse
%</samplechap1>
%\fi
% Some text for chapter 2:
%\iffalse
%<*samplechap2>
%\fi
%    \begin{macrocode}
\section{two}
more text in chapter two
%    \end{macrocode}

%\iffalse
%</samplechap2>
%\fi
%
% %%%%%%%%%%%%%%%%%%%%%%%%%%%%%%%%%%%%%%
% \paragraph{Part Include Files.}
%
% The include files are called |cdocspt3.tex| and |cdocspt4.tex|.
%
%\iffalse
%<*samplepart3|samplepart4>
%\fi

% Optional override for |\version| flag:
%    \begin{macrocode}
%%\providecommand{\version}{final}
%    \end{macrocode}

% Include the main document:
%    \begin{macrocode}
\input{childdoc.def}
\childdocby{cdocsamp}
%    \end{macrocode}

%\iffalse
%</samplepart3|samplepart4>
%\fi
%
%\iffalse
%<*samplepart3>
%\fi
% Some text for part 3:
%    \begin{macrocode}
some text in part three
%    \end{macrocode}

%\iffalse
%</samplepart3>
%\fi
% Some text for part 4:
%\iffalse
%<*samplepart4>
%\fi
%    \begin{macrocode}
more text in part four
%    \end{macrocode}

%\iffalse
%</samplepart4>
%\fi
%
% %%%%%%%%%%%%%%%%%%%%%%%%%%%%%%%%%%%%%%
% \paragraph{Forwarding for a Complete Draft.}
%
% The following forwarding file |cdocsdrf.tex|
% compiles the main document in draft mode:
%\iffalse
%<*sampledraft>
%\fi
%    \begin{macrocode}
\def\version{draft}
\input{childdoc.def}
\childdocforward{cdocsamp}
%    \end{macrocode}

%\iffalse
%</sampledraft>
%\fi
%
% %%%%%%%%%%%%%%%%%%%%%%%%%%%%%%%%%%%%%%
% \paragraph{Forwarding for Final Version of the Chapters.}
%
% The following forwarding files |cdocsfn1.tex| and |cdocsfn2.tex|
% (with identical content)
% compile the final versions of the child documents
% |cdocsch1.tex| and |cdocsch2.tex|, respectively:
%\iffalse
%<*samplefinal>
%\fi
%    \begin{macrocode}
\def\version{final}
\input{childdoc.def}
\childdocforwardprefix[cdocsamp]{cdocsfn}{cdocsch}
%    \end{macrocode}

%\iffalse
%</samplefinal>
%\fi
%
% %%%%%%%%%%%%%%%%%%%%%%%%%%%%%%%%%%%%%%
% \paragraph{Command Line Processing.}
%
% The following three command lines generate the output files
% |cdocscld|, |cdocscl1| and |cdocscl2|
% which should be identical to
% |cdocsdrf|, |cdocsch1| and |cdocsfn2|, respectively:
% \begin{center}
% \begin{tabular}{l}
% |latex -jobname cdocscld \|\\
% |  "\def\version{draft}\input{childdoc.def}\childdocforward{cdocsamp}"|\\
% |latex -jobname cdocscl1 \|\\
% |  "\input{childdoc.def}\childdocforward[cdocsamp]{cdocsch1}"|\\
% |latex -jobname cdocscl2 \|\\
% |  "\def\version{final}\input{childdoc.def}\childdocforward{cdocsch2}"|
% \end{tabular}
% \end{center}
% Note that the trailing backslash on each first line
% merely continues the input to the second line
% (for convenient cut ant paste).
% Furthermore, the command |latex| can be replaced by any
% of its alternative versions such as |pdflatex|.
%
% %%%%%%%%%%%%%%%%%%%%%%%%%%%%%%%%%%%%%%%%%%%%%%%%%%%%%%%%%%%%%%%%%%%%%%%%%%%%%%
% %%%%%%%%%%%%%%%%%%%%%%%%%%%%%%%%%%%%%%%%%%%%%%%%%%%%%%%%%%%%%%%%%%%%%%%%%%%%%%
% \section{Implementation}
%\iffalse
%<*package>
%\fi
%
% This section describes the definitions file |childdoc.def|.

% The definitions cannot be loaded using |\usepackage| or |\RequirePackage|
% which has a mechanism to prevent loading a style file more than once.
% When loading the definitions by means of |\input|
% multiple instances have to be prevented manually:
%\iffalse
%This code needs to be before the `\ProvidesFile' directive
%which is defined at the beginning of this file.
%Therefore it is also placed there and commented out here.
%</package>
%<*discard>
%\fi
%    \begin{macrocode}
\ifdefined\childdocmain\endinput\fi
%    \end{macrocode}
%\iffalse
%</discard>
%<*package>
%\fi
%
% \macro{\ifchilddoc}
% \macro{\ifchilddocmanual}
% The conditional |\ifchilddoc| tells whether a
% child (true) or main (false) document is being compiled.
% The conditional |\ifchilddocmanual| tells whether
% the |\includeonly| mechanism is used (false) or
% the selection of child files must be performed manually (true).
% The definitions initialise to false:
%    \begin{macrocode}
\newif\ifchilddoc
\newif\ifchilddocmanual
%    \end{macrocode}

% \macro{\childdocname}
% \macro{\childdocjob}
% The macro |\childdocname| stores the name of the main document
% to be compiled. The macro |\childdocjob| stores the name of
% the document on which the \LaTeX{} compiler was originally invoked.
% The content of |\jobname| cannot be compared
% to filenames specified in the source due to different catcodes.
% The following code rescans |\jobname|, stores the result
% in |\childdocname| and saves a copy in |\childdocjob|:
%    \begin{macrocode}
\edef\childdocname{\scantokens\expandafter{\jobname\noexpand}}
\let\childdocjob\childdocname
%    \end{macrocode}

% \macro{\childdocdisable}
% The macro |\childdocdisable| prevents the main file
% from being processed more than once.
% At this stage, the main document command |\childdocmain|
% is assumed to be called once again where it should do nothing.
% Any subsequent call to it should prevent
% a secondary processing of the main document
% It overwrites the forwarding commands
% |\childdocof| and |\childdocforward|
% with empty macros to prevent further inclusions of the main document:
%    \begin{macrocode}
\newcommand{\childdocdisable}
{
  \renewcommand{\childdocmain}[1]{\renewcommand{\childdocmain}[1]{\endinput}}
  \renewcommand{\childdocof}[1]{}
  \renewcommand{\childdocby}[2][]{}
  \renewcommand{\childdocforward}[2][]{}
  \renewcommand{\childdocdisable}{}
}
%    \end{macrocode}

% \macro{\childdocmain}
% The macro |\childdocmain| is to be called at the top of the main file
% with nothing or the main filename (without extension) as argument.
% First, it breaks loops.
% If the argument is not empty and does not match |\childdocname|
% (which is set by the first inclusion of |childdoc.def|),
% |\ifchilddoc| is set to true, |\includeonly| is applied to the child file
% and |\jobname| is set to the main file
% (for proper handling of |.aux| files):
%    \begin{macrocode}
\newcommand{\childdocmain}[1]
{
  \childdocdisable\childdocmain{}
  \if?#1?\else
    \begingroup
      \def\childdoctmp{#1}
      \ifx\childdoctmp\childdocname
        \def\childdoctmp{}
      \else
        \def\childdoctmp
        {
          \childdoctrue
          \includeonly{\childdocname}
          \def\childdocjob{#1}
          \def\jobname{#1}
        }
      \fi
      \expandafter
    \endgroup
    \childdoctmp
  \fi
}
%    \end{macrocode}

% \macro{\childdocof}
% The command |\childdocof| redirects
% compilation to the main file |#1|.
%    \begin{macrocode}
\newcommand{\childdocof}[1]
{
  \childdocdisable
  \childdoctrue
  \includeonly{\childdocname}
  \def\jobname{#1}
  \def\childdocjob{#1}
  \input{#1}
}
%    \end{macrocode}

% \macro{\childdocby}
% The command |\childdocby| ....
%    \begin{macrocode}
\newcommand{\childdocby}[2][]
{
  \childdocdisable
  \childdoctrue
  \childdocmanualtrue
  \if?#1?\else
    \def\jobname{#2}
  \fi
  \def\childdocjob{#2}
  \input{#2}
  \endinput
}
%    \end{macrocode}

% \macro{\childdocforward}
% The command |\childdocforward| redirects
% compilation to the main file or
% (if the optional argument is given) a child file.
% Parameters are set as if the main file
% or a child file starting with |\childdocof| was compiled.
% Then compilation is handed over to the main file:
%    \begin{macrocode}
\newcommand{\childdocforward}[2][]
{
  \begingroup
    \if?#1?
      \def\childdoctmp
      {
        \def\childdocname{#2}
        \def\childdocjob{#2}
        \def\jobname{#2}
        \input{#2}
        \endinput
      }
    \else
      \def\childdoctmp
      {
        \childdocdisable
        \def\childdocname{#2}
        \childdoctrue
        \includeonly{#2}
        \def\childdocjob{#1}
        \def\jobname{#1}
        \input{#1}
        \endinput
      }
    \fi
    \expandafter
  \endgroup
  \childdoctmp
}
%    \end{macrocode}

% \macro{\childdocforwardprefix}
% The command |\childdocforwardprefix| redirects
% compilation to the main or a child file by means of a pattern.
% The prefix |#1| in the current filename is replaced by |#2|
% and the suffix of the current filename is kept
% (it is assumed that the filename does not contain the substring `|~~~|'
% which is used as a delimiter).
% Compilation is handed over to the new file by |\childdocforward|:
%    \begin{macrocode}
\newcommand{\childdocforwardprefix}[3][]
{
  \begingroup
    \def\childdocextract #2##1~~~{\def\childdoctmp{\childdocforward[#1]{#3##1}}}
    \expandafter\childdocextract\childdocname~~~
    \expandafter
  \endgroup
  \childdoctmp
}
%    \end{macrocode}

% \macro{\childdoc}
% The deprecated macro |\childdoc| is a legacy version of |\childdocmain|:
%    \begin{macrocode}
\newcommand{\childdoc}{\childdocmain}
%    \end{macrocode}

% \macro{\childdocredirect}
% The deprecated macro |\childdocredirect| is a legacy version
% of |\childdocforward| and |\childdocforwardprefix|:
%    \begin{macrocode}
\newcommand{\childdocredirect}[2][]
{
  \begingroup
    \if?#1?
      \def\childdoctmp{\childdocforward{#2}}
    \else
      \def\childdoctmp{\childdocforwardprefix{#1}{#2}}
    \fi
    \expandafter
  \endgroup
  \childdoctmp
}
%    \end{macrocode}

%\iffalse
%</package>
%\fi
%
\endinput
|\\
|\childdocforward{|\textit{main}|}|
\end{tabular}
\end{center}
%
Likewise, the following files |final|\textit{nn}|.tex|
compile the final version of the child document
|child|\textit{nn}|.tex|:
%
\begin{center}
\begin{tabular}{l}
|\def\version{final}|\\
|% \iffalse
%
% childdoc.dtx Copyright (C) 2017-2018 Niklas Beisert
%
% This work may be distributed and/or modified under the
% conditions of the LaTeX Project Public License, either version 1.3
% of this license or (at your option) any later version.
% The latest version of this license is in
%   http://www.latex-project.org/lppl.txt
% and version 1.3 or later is part of all distributions of LaTeX
% version 2005/12/01 or later.
%
% This work has the LPPL maintenance status `maintained'.
%
% The Current Maintainer of this work is Niklas Beisert.
%
% This work consists of the files childdoc.dtx and childdoc.ins
% and the derived files childdoc.def and cdocsamp.tex with
% cdocsch1.tex, cdocsch2.tex, cdocsdrf.tex, cdocsfn1.tex, cdocsfn2.tex.
%
%<package>\ifdefined\childdocmain\endinput\fi
%<package>\ProvidesFile{childdoc.def}[2018/12/30 v2.0 child document driver]
%<samplemain>\ProvidesFile{cdocsamp.tex}[2018/12/30 v2.0 sample for childdoc]
%<*driver>
%\ProvidesFile{childdoc.drv}[2018/12/30 v2.0 childdoc reference manual file]
\PassOptionsToClass{10pt,a4paper}{article}
\documentclass{ltxdoc}

\usepackage[margin=35mm]{geometry}
\usepackage{hyperref}
\usepackage{hyperxmp}
\usepackage[usenames]{color}

\hypersetup{colorlinks=true}
\hypersetup{pdfstartview=FitH}
\hypersetup{pdfpagemode=UseNone}
\hypersetup{pdfsource={}}
\hypersetup{pdflang={en-UK}}
\hypersetup{pdfcopyright={Copyright 2017-2018 Niklas Beisert.
  This work may be distributed and/or modified under the
  conditions of the LaTeX Project Public License, either version 1.3
  of this license or (at your option) any later version.}}
\hypersetup{pdflicenseurl={http://www.latex-project.org/lppl.txt}}
\hypersetup{pdfcontactaddress={ETH Zurich, ITP, HIT K,
  Wolfgang-Pauli-Strasse 27}}
\hypersetup{pdfcontactpostcode={8093}}
\hypersetup{pdfcontactcity={Zurich}}
\hypersetup{pdfcontactcountry={Switzerland}}
\hypersetup{pdfcontactemail={nbeisert@itp.phys.ethz.ch}}
\hypersetup{pdfcontacturl={http://people.phys.ethz.ch/\xmptilde nbeisert/}}

\newcommand{\secref}[1]{\hyperref[#1]{section \ref*{#1}}}

\parskip1ex
\parindent0pt
\let\olditemize\itemize
\def\itemize{\olditemize\parskip0pt}

\begin{document}

\title{The \textsf{childdoc} Package}
\hypersetup{pdftitle={The childdoc Package}}
\author{Niklas Beisert\\[2ex]
  Institut f\"ur Theoretische Physik\\
  Eidgen\"ossische Technische Hochschule Z\"urich\\
  Wolfgang-Pauli-Strasse 27, 8093 Z\"urich, Switzerland\\[1ex]
  \href{mailto:nbeisert@itp.phys.ethz.ch}
  {\texttt{nbeisert@itp.phys.ethz.ch}}}
\hypersetup{pdfauthor={Niklas Beisert}}
\hypersetup{pdfsubject={Manual for the LaTeX2e Package childdoc}}
\date{30 December 2018, \textsf{v2.0}}
\maketitle

\begin{abstract}\noindent
\textsf{childdoc} is a \LaTeXe{} package
that enables the direct compilation
of document sections included by |\include|
to individual files.
\end{abstract}

\begingroup
\parskip0ex
\tableofcontents
\endgroup

%%%%%%%%%%%%%%%%%%%%%%%%%%%%%%%%%%%%%%%%%%%%%%%%%%%%%%%%%%%%%%%%%%%%%%%%%%%%%%%%
%%%%%%%%%%%%%%%%%%%%%%%%%%%%%%%%%%%%%%%%%%%%%%%%%%%%%%%%%%%%%%%%%%%%%%%%%%%%%%%%
\section{Introduction}

\LaTeX{} provides a mechanism to structure a large document (such as a book)
into a main file and several child files (containing the chapters)
using the |\include| command.
This mechanism is beneficial for documents
which span hundreds of pages in order to
make the source file(s) more manageable.
Moreover, compilation can be restricted to
selected child files by means of the |\includeonly| command.
The latter feature can be used to reduce the compilation time while editing
(this was significantly more useful in the earlier days of \LaTeX{})
or to generate a smaller document which is easier to navigate.
Another application of |\includeonly| is to generate
documents consisting of selected parts of the complete document.

However, there are a few drawbacks of the plain |\include| mechanism:
\begin{itemize}
\item
The child files cannot be compiled on their own,
they can only be compiled via the main file.
A naive editing environment
(such as a text editor with an option
to have the current file processed by \LaTeX)
may require one to switch to the main file before compiling;
attempting to compile the child file produces errors.
\item
The main file must be modified (each time)
to adjust the |\includeonly| command
to the present needs. This easily leaves the main file in a messy state.
\item
The generated document will always carry the filename
of the main document. This is inconvenient if
several child files are to be compiled and
to be kept for distribution.
\end{itemize}

The present package provides a simple interface
to make child files individually compilable by \LaTeX{}.
Compiling a child file then has the same effect as compiling
the main file with an |\includeonly| command
to select the appropriate child.
Moreover the generated document will carry the name of the child
rather than the main file.
This resolves all three above issues.

This feature is meant to make the editing of books,
thesis documents and lecture notes somewhat more convenient.
However, the package can also be used efficiently for
composing a series of documents (such as exercise sheets)
which are typically distributed individually.
It then assists the author in generating the individual documents
(potentially in different versions)
as well as a document containing the collected series.
Another application is in developing style files
or other kinds of included material
where compilation of the style file could redirect
to a sample or test file.

%%%%%%%%%%%%%%%%%%%%%%%%%%%%%%%%%%%%%%%%%%%%%%%%%%%%%%%%%%%%%%%%%%%%%%%%%%%%%%%%
%%%%%%%%%%%%%%%%%%%%%%%%%%%%%%%%%%%%%%%%%%%%%%%%%%%%%%%%%%%%%%%%%%%%%%%%%%%%%%%%
\section{Usage}

First of all, the package \textsf{childdoc} is \emph{not} a standard
\LaTeXe{} |.sty| style file! Therefore it needs to be invoked in
a non-standard way.

%%%%%%%%%%%%%%%%%%%%%%%%%%%%%%%%%%%%%%%%%%%%%%%%%%%%%%%%%%%%%%%%%%%%%%%%%%%%%%%%
\subsection{Included Files}
\label{sec:include}

%%%%%%%%%%%%%%%%%%%%%%%%%%%%%%%%%%%%%%%%
\DescribeMacro{\childdocmain}
To use the package, add the commands
\begin{center}
\begin{tabular}{l}
|\input{childdoc.def}|\\
|\childdocmain{}|\\
\end{tabular}
\end{center}
at the very top of the main \LaTeX{} file,
in particular \emph{before} the |\documentclass| statement!
The argument of |\childdocmain| should be left empty
(but it must be present).

%%%%%%%%%%%%%%%%%%%%%%%%%%%%%%%%%%%%%%%%
\DescribeMacro{\childdocof}
Furthermore, add the commands
\begin{center}
\begin{tabular}{l}
|\input{childdoc.def}|\\
|\childdocof{|\textit{main}|}|\\
\end{tabular}
\end{center}
at the top of every child file \textit{child}
which is included by |\include{|\textit{child}|}|
from within the main file
(or at least for those files to be compiled individually).
The argument \textit{main} must be the filename of the main file.

There are a couple of
considerations in setting up the main and child documents:

%%%%%%%%%%%%%%%%%%%%%%%%%%%%%%%%%%%%%%%%
\paragraph{Restrictions.}

Please note the following restrictions:
\begin{itemize}
\item
|\childdocmain| must be called with one argument \textit{main}
to ensure compatibility with earlier version of the package.
It must either be empty (|\childdocmain{}|)
or precisely match the filename of the main file in which it is specified.
See \secref{sec:detection} for further information.
\item
The filename \textit{main} must be specified without the |.tex| extension.
\item
The filename \textit{main} is case sensitive
(even in case-insensitive file systems)
due to internal string comparison.
\item
The argument \textit{main} should be fully expanded, it cannot be a macro.
\item
Subdirectories and special characters should be avoided in filenames.
\item
The command |\childdocmain{|\textit{main}|}| must be followed by a whitespace.
It should not be followed immediately by another command
or by a comment mark `|%|'.
This is because the \TeX{} parser reads the token immediately following
the argument of |\childdocmain| and puts it
at the beginning of every child section;
however, a white\-space is ignored.
\end{itemize}

%%%%%%%%%%%%%%%%%%%%%%%%%%%%%%%%%%%%%%%%
\paragraph{Content of Main File.}

It is advisable to place all content in the child files included by |\include|.
Any output contained in the main file will appear in all child documents
unless suppressed manually;
it cannot be suppressed automatically by the |\includeonly| directive
and thus should normally be avoided.
A method to include some content in the main file
by means of conditional processing is described in \secref{sec:conditional}.

%%%%%%%%%%%%%%%%%%%%%%%%%%%%%%%%%%%%%%%%
\paragraph{Page Numbering.}

When only a part of the document is compiled,
the appropriate numbering of pages
(as well as other status parameters)
is determined from the |.aux| files.
The latter contain information from previous passes.
However this information needs to propagate through
all intermediate child documents.
Therefore the page numbering in child documents may well
be inconsistent until the complete document is compiled at least once.

A useful (if unconventional) way to always ensure a consistent
page numbering is to restart the numbering in each child document
and denote the pages by `\textit{child}|.|\textit{page}'
where \textit{child} represents the chapter/section number of the child file.
This can be achieved by the command
|\numberwithin{page}{|\textit{child}|}|
of the \textsf{amsmath} package
where \textit{child} can be |chapter| or |section|
depending on the chosen structuring.
Alternatively, one can modify the macro |\thepage| appropriately
and reset the counter |page| at the start of each child file.

%%%%%%%%%%%%%%%%%%%%%%%%%%%%%%%%%%%%%%%%%%%%%%%%%%%%%%%%%%%%%%%%%%%%%%%%%%%%%%%%
\subsection{Conditional Processing}
\label{sec:conditional}

The package provides a mechanism to compile different versions
of a document. To customise the versions further some conditional processing
can come in handy to distinguish which version is being compiled.
The package provides two macros to describe the compilation context:

%%%%%%%%%%%%%%%%%%%%%%%%%%%%%%%%%%%%%%%%
\DescribeMacro{\ifchilddoc}
The conditional |\ifchilddoc| distinguishes between the compilation of
child documents and the main document:
%
\begin{center}
|\ifchilddoc |\textit{child-code}| |[|\||else |\textit{main-code}]| \||fi|
\end{center}

%%%%%%%%%%%%%%%%%%%%%%%%%%%%%%%%%%%%%%%%
\DescribeMacro{\childdocname}
\DescribeMacro{\childdocjob}
The macro |\childdocname| contains the filename (without extension)
of the main or child file being processed.
Note that |\childdocjob| will always contain the name of the main file.

%%%%%%%%%%%%%%%%%%%%%%%%%%%%%%%%%%%%%%%%
\paragraph{Title Page.}

Conditional processing can be used to include a title or banner page
in the main document when proper precautions are taken.
Importantly, the code in the main file should ensure that the page counter
(as well as other status parameters which are stored in the |.aux| files)
takes the same value after the conditional processing.
Otherwise the page numbers may take divergent values
depending on which part is compiled.

For example, a title page could be declared by:
%
\begin{center}
\begin{tabular}{l}
|\ifchilddoc\||else|\\
|\addtocounter{page}{-1}|\\
\textit{code for title page}\\
|\newpage|\\
|\||fi|
\end{tabular}
\end{center}
%
A banner page for the child documents can be generated by:
%
\begin{center}
\begin{tabular}{l}
|\ifchilddoc|\\
|\addtocounter{page}{-1}|\\
\textit{code for banner page}\\
|\newpage|\\
|\||fi|
\end{tabular}
\end{center}
%
Here one could write a message such as:
\begin{center}
|This is the part \childdocname{} of \childdocjob{}.|
\end{center}

%%%%%%%%%%%%%%%%%%%%%%%%%%%%%%%%%%%%%%%%%%%%%%%%%%%%%%%%%%%%%%%%%%%%%%%%%%%%%%%%
\subsection{Flags}
\label{sec:flags}

The package makes it easy to generate different versions
of the main or child documents.
To this end compilation flags can be defined
and assigned different default values.
They will be particularly useful in conjunction
with the forwarding mechanism described in \secref{sec:forward}.

For example, it may be useful to have a flag |\version|
which can be set to |draft| or |final|.
The document source will contain some conditional code
depending on the value of |\version|.
Suppose further, the flag should default to |final| for the main file
and to |draft| for child files
which is a natural assignment for editing the document.
This is achieved by placing the following code
in the preamble of the main document
(below the |\childdocmain| directive):
%
\begin{center}
\begin{tabular}{l}
|\ifchilddoc|\\
|\providecommand{\version}{draft}|\\
|\||else|\\
|\providecommand{\version}{final}|\\
|\||fi|
\end{tabular}
\end{center}
%
The definition by |\providecommand| makes sure
that previous definitions are not overwritten.
Further statements |\providecommand{\version}{...}|
can thus be added before the above code to override it.

For the main file, one might add a line
(between |\childdocmain| and the above block)
%
\begin{center}
|%\ifchilddoc\||else\providecommand{\version}{draft}\||fi|
\end{center}
%
which can be uncommented to produce a draft version.
Likewise one can add a line to the very top of a child file
(above the |\childdocof{|\textit{main}|}| directive)
%
\begin{center}
|%\providecommand{\version}{final}|
\end{center}
%
which can be uncommented to produce the final version of this child document.

%%%%%%%%%%%%%%%%%%%%%%%%%%%%%%%%%%%%%%%%%%%%%%%%%%%%%%%%%%%%%%%%%%%%%%%%%%%%%%%%
\subsection{Forwarding}
\label{sec:forward}

Different versions of the main or child documents
using compilation flags as described in \secref{sec:flags}
can be (permanently) stored in different files
for convenient compilation, viewing and distribution.
To this end, the package defines a command
to pass on compilation to a different file:

%%%%%%%%%%%%%%%%%%%%%%%%%%%%%%%%%%%%%%%%
\DescribeMacro{\childdocforward}
The command |\childdocforward| redirects processing to
another source file:
%
\begin{center}
\begin{tabular}{l}
|\input{childdoc.def}|\\
|\childdocforward[|\textit{main}|]{|\textit{dest}|}|\\
\end{tabular}
\end{center}
%
The argument \textit{dest} is the destination file
(without extension).
It should be the main file or one of the child files.
Note that further \textsf{childdoc} directives
such as |\childdocof| and |\childdocforward|
in the indicated file will be processed in this form.
The optional argument \textit{main}
passes on directly to the main file \textit{main}
while pretending to compile the child \textit{dest}.
This form behaves as if \textit{dest}
issues |\childdocof{|\textit{main}|}| right away,
and no further \textsf{childdoc} directives will be processed.

%%%%%%%%%%%%%%%%%%%%%%%%%%%%%%%%%%%%%%%%
\DescribeMacro{\...prefix}
In the alternative form |\childdocforwardprefix|,
%
\begin{center}
\begin{tabular}{l}
|\input{childdoc.def}|\\
|\childdocforwardprefix[|\textit{main}|]{|\textit{prefix}|}{|\textit{dest}|}|
\end{tabular}
\end{center}
%
the destination file is determined by a pattern
depending on the current file:
To make this work, the current file must be called
`{\textit{prefix}\hspace{0.2em}\textit{suffix}}'
with \textit{prefix} matching precisely the argument.
Processing is then passed on to the file
`{\textit{dest}\hspace{0.2em}\textit{suffix}}'.
Surely, the same effect is achieved by
directly specifying the
argument `{\textit{dest}\hspace{0.2em}\textit{suffix}}'
in the first form.
However, that requires to set up a different file
for each child. With the alternative form of the command
all these files can have exactly the same content
which simplifies setting them up and maintaining them.

For example, the following file |draft.tex|
with a compilation flag |\version| as described in \secref{sec:flags}
compiles the main document as a draft:
%
\begin{center}
\begin{tabular}{l}
|\def\version{draft}|\\
|\input{childdoc.def}|\\
|\childdocforward{|\textit{main}|}|
\end{tabular}
\end{center}
%
Likewise, the following files |final|\textit{nn}|.tex|
compile the final version of the child document
|child|\textit{nn}|.tex|:
%
\begin{center}
\begin{tabular}{l}
|\def\version{final}|\\
|\input{childdoc.def}|\\
|\childdocforwardprefix{final}{child}|
\end{tabular}
\end{center}
%

Note that when several versions of a main file and/or of each child file
are to be generated, it may be convenient to set up a |Makefile| or
shell script to automatise the process.

%%%%%%%%%%%%%%%%%%%%%%%%%%%%%%%%%%%%%%%%%%%%%%%%%%%%%%%%%%%%%%%%%%%%%%%%%%%%%%%%
\subsection{Command Line Processing}
\label{sec:commandline}

The effect of redirection files can also be achieved by invoking
the \LaTeX{} compiler with a more elaborate command line.
Most conveniently this should be done as part
of a shell script or a |Makefile|.

When using \textsf{childdoc} in the main file, the following
command lines effectively perform a redirection
(note that depending on the shell being used,
backslashes may have to be doubled: `|\|' $\to$ `|\\|'):
%
\begin{center}
|... -jobname "|\textit{target}|" |\\|"|[\textit{flags}]%
|\input{childdoc.def}\childdocforward[|\textit{main}|]{|\textit{dest}|}"|
\end{center}
%
Here \textit{target} is the name of the output file,
\textit{main} is the name of the main file
and \textit{dest} is the name of the main or child file to be processed
(all filenames without extensions).
The optional argument \textit{main} can be omitted
if \textit{main} matches \textit{dest}.
Optionally, compilation \textit{flags} can be defined via |\def| commands.
This command line makes the \TeX{} engine believe
it is compiling the file \textit{target}
whose content is specified as the latter parameter.
The provided code then forwards the processing to
\textit{main} or \textit{dest} as described in \secref{sec:forward}.

%%%%%%%%%%%%%%%%%%%%%%%%%%%%%%%%%%%%%%%%%%%%%%%%%%%%%%%%%%%%%%%%%%%%%%%%%%%%%%%%
\subsection{Include by Input}
\label{sec:input}

Including child documents by |\include| has some restrictions by design.
Most notably, the content of a child document always occupies
its own set of pages; pages cannot be shared between child documents.
Usually, this behaviour makes perfect sense
because each child document contain an essential part of the document.
However, in some situations it may be desirable to compose
a document from a collection of parts
without having mandatory page breaks between then.
For this case, the package
provides a mechanism to include parts
by |\input| which can also be processed individually.
However, by construction this mechanism
requires manual handling of the content to be output.

%%%%%%%%%%%%%%%%%%%%%%%%%%%%%%%%%%%%%%%%
\DescribeMacro{\ifchilddocmanual}
The main file should be prepared as usual, see \secref{sec:include}.
However, the document body must make a distinction
between processing of an individual part and of the main document, e.g.:
%
\begin{center}
\begin{tabular}{l}
|\ifchilddocmanual|\\
|\input{\childdocname}|\\
|\||else|\\
\textit{document body with }|\input{|\textit{part}|}|\\
|\||fi|
\end{tabular}
\end{center}
%
The conditional |\ifchilddocmanual| is true whenever
a part to be included by |\input| is being compiled,
and the name of the part is stored in |\childdocname|.

%%%%%%%%%%%%%%%%%%%%%%%%%%%%%%%%%%%%%%%%
\DescribeMacro{\childdocby}
Each part to be included by |\input| should start with:
%
\begin{center}
\begin{tabular}{l}
|\input{childdoc.def}|\\
|\childdocby{|\textit{main}|}|\\
\end{tabular}
\end{center}
%
The directive |\childdocby| is similar to |\childdocof|
described in \secref{sec:include},
but the subsequent selection of content must be done manually.
To that end, both |\ifchilddoc| and |\ifchilddocmanual|
will be true upon processing of a part,
and the name of the part is stored in |\childdocname|.
Note that |\jobname| will be set to the filename of the current part
so that each part receives an individual |.aux| file
that does not interfere with the |.aux| file(s) of the main document.
This behaviour can be altered by the alternative form
|\childdocby[*]{|\textit{main}|}| (with a non-empty optional argument)
which uses the |.aux| file of the main document
by setting |\jobname| to \textit{main}.

%%%%%%%%%%%%%%%%%%%%%%%%%%%%%%%%%%%%%%%%%%%%%%%%%%%%%%%%%%%%%%%%%%%%%%%%%%%%%%%%
\subsection{Driver Development}
\label{sec:driver}

The \textsf{childdoc} mechanism can also be use for the development
of definition files such as \LaTeX{} styles or classes.
This case differs from the above setup with multiple parts
included by |\include| in that no |\includeonly| should be invoked.
This can be achieved by starting the include file
(before |\ProvidesPackage|) with:
%
\begin{center}
\begin{tabular}{l}
|\input{childdoc.def}|\\
|\childdocforward{|\textit{main}|}|\\
\end{tabular}
\end{center}
%
or alternatively with:
%
\begin{center}
\begin{tabular}{l}
|\input{childdoc.def}|\\
|\childdocby{|\textit{main}|}|\\
\end{tabular}
\end{center}
%
Both forms have slightly different effects as described above.
The main file is prepared as usual, see \secref{sec:include}.

%%%%%%%%%%%%%%%%%%%%%%%%%%%%%%%%%%%%%%%%%%%%%%%%%%%%%%%%%%%%%%%%%%%%%%%%%%%%%%%%
\subsection{Legacy Detection}
\label{sec:detection}

The directive |\childdocmain| in the main file can detect
whether the complete document or merely a child is to be compiled
even without using the directive |\childdocof|.
This method is deprecated because it is less robust
and there is no compelling reason to use it;
it is merely provided for backward compatibility
and it may be removed in future versions.

If the detection mechanism is to be used,
it is mandatory to correctly specify
the filename of the main file as the argument of |\childdocmain|:
%
\begin{center}
\begin{tabular}{l}
|\input{childdoc.def}|\\
|\childdocmain{|\textit{main}|}|\\
\end{tabular}
\end{center}
%
If |\jobname| does not match the argument \textit{main} of |\childdocmain|,
it is assumed that |\jobname| points to the child file to be compiled.
When using |\childdocmain| with the main file specified as argument,
it suffices to start a child file
with just |\input{|\textit{main}|}|
without loading of the package and using |\childdocof|.
If instead all processing is done
with the appropriate \textsf{childdoc} directives,
the argument of \textit{main} of |\childdocmain| can be empty.

An alternative version of the command line processing described
in \secref{sec:commandline} using the detection mechanism reads:
%
\begin{center}
|... -jobname "|\textit{target}|" "|[\textit{flags}]%
[|\def\jobname{|\textit{dest}|}|]|\input{|\textit{main}|}"|
\end{center}

%%%%%%%%%%%%%%%%%%%%%%%%%%%%%%%%%%%%%%%%%%%%%%%%%%%%%%%%%%%%%%%%%%%%%%%%%%%%%%%%
\subsection{Manual Code}
\label{sec:manual}

In case one cannot be certain whether the definitions file |childdoc.def|
is installed on the target \TeX{} distribution
and one prefers not to ship it,
it is conceivable to paste a few relevant commands into the sources.

To that end, drop all statements |\input{childdoc.def}|
and perform the replacements as outlined below.
Instead of |\childdocmain{|\textit{main}|}| add the following code
to the top of the main file:
%
\begin{center}
\begin{tabular}{l}
|\||ifdefined\childdocname\endinput\||fi\newif\ifchilddoc|\\
|\edef\childdocname{\scantokens\expandafter{\jobname\noexpand}}|\\
|\def\childdocmain{|\textit{main}|}\||ifx\childdocmain\childdocname\||else|\\
|\childdoctrue\includeonly{\childdocname}\let\jobname\childdocmain\||fi|\\
\end{tabular}
\end{center}
%
Instead of |\childdocof{|\textit{main}|}| just include the main file
at the top of each child file:
%
\begin{center}
|\input{|\textit{main}|}|
\end{center}
%
A simple redirection |\childdocforward{|\textit{dest}|}| is achieved by:
%
\begin{center}
|\def\jobname{|\textit{dest}|}\input{\jobname}|
\end{center}
%
The redirection with prefix
|\childdocforwardprefix[|\textit{prefix}|]{|\textit{dest}|}|
is accomplished by:
%
\begin{center}
\begin{tabular}{l}
|{\edef\jobname{\scantokens\expandafter{\jobname\noexpand}}|\\
|\def\redirectjob |\textit{prefix}|#1~~~{\gdef\jobname{|\textit{dest}|#1}}|\\
|\expandafter\redirectjob\jobname~~~}\input{\jobname}|
\end{tabular}
\end{center}

In an alternative approach,
child documents can be compiled by a specific command line
without additional code or specific definitions:
%
\begin{center}
|... -jobname "|\textit{target}|" "|[\textit{flags}]%
|\includeonly{|\textit{dest}|}\input{|\textit{main}|}"|
\end{center}
%

%%%%%%%%%%%%%%%%%%%%%%%%%%%%%%%%%%%%%%%%%%%%%%%%%%%%%%%%%%%%%%%%%%%%%%%%%%%%%%%%
%%%%%%%%%%%%%%%%%%%%%%%%%%%%%%%%%%%%%%%%%%%%%%%%%%%%%%%%%%%%%%%%%%%%%%%%%%%%%%%%
\section{Information}

%%%%%%%%%%%%%%%%%%%%%%%%%%%%%%%%%%%%%%%%%%%%%%%%%%%%%%%%%%%%%%%%%%%%%%%%%%%%%%%%
\subsection{Copyright}

Copyright \copyright{} 2017--2018 Niklas Beisert

This work may be distributed and/or modified under the
conditions of the \LaTeX{} Project Public License, either version 1.3
of this license or (at your option) any later version.
The latest version of this license is in
  \url{http://www.latex-project.org/lppl.txt}
and version 1.3 or later is part of all distributions of \LaTeX{}
version 2005/12/01 or later.

This work has the LPPL maintenance status `maintained'.

The Current Maintainer of this work is Niklas Beisert.

This work consists of the files |README.txt|, |childdoc.ins| and |childdoc.dtx|
as well as the derived files |childdoc.def|, |cdocsamp.tex|
with |cdocsch1.tex|, |cdocsch2.tex|, |cdocspt3.tex|, |cdocspt4.tex|,
|cdocsdrf.tex|, |cdocsfn1.tex|, |cdocsfn2.tex|
as well as |childdoc.pdf|.

%%%%%%%%%%%%%%%%%%%%%%%%%%%%%%%%%%%%%%%%%%%%%%%%%%%%%%%%%%%%%%%%%%%%%%%%%%%%%%%%
\subsection{Files and Installation}

The package consists of the files:
%
\begin{center}
\begin{tabular}{ll}
    |README.txt|   & readme file \\
    |childdoc.ins| & installation file \\
    |childdoc.dtx| & source file \\
    |childdoc.def| & definition file \\
    |cdocsamp.tex| & sample main file \\
    |cdocsch1.tex| & sample include file \\
    |cdocsch2.tex| & sample include file \\
    |cdocspt3.tex| & sample part file \\
    |cdocspt4.tex| & sample part file \\
    |cdocsdrf.tex| & sample redirection file \\
    |cdocsfn1.tex| & sample redirection file \\
    |cdocsfn2.tex| & sample redirection file \\
    |childdoc.pdf| & manual
\end{tabular}
\end{center}
%
The distribution consists of the files
|README.txt|, |childdoc.ins| and |childdoc.dtx|.
%
\begin{itemize}
\item
Run (pdf)\LaTeX{} on |childdoc.dtx|
to compile the manual |childdoc.pdf| (this file).
\item
Run \LaTeX{} on |childdoc.ins| to create the definitions file |childdoc.def|
and the sample |cdocsamp.tex| with include files
|cdocsch1.tex|, |cdocsch2.tex|, |cdocspt3.tex|, |cdocspt4.tex|,
|cdocsdrf.tex|, |cdocsfn1.tex|, |cdocsfn2.tex|.
Then copy the file |childdoc.def| to an appropriate directory of your \LaTeX{}
distribution, e.g.\ \textit{texmf-root}|/tex/latex/childdoc|.
\end{itemize}

%%%%%%%%%%%%%%%%%%%%%%%%%%%%%%%%%%%%%%%%%%%%%%%%%%%%%%%%%%%%%%%%%%%%%%%%%%%%%%%%
\subsection{Related CTAN Packages}

There are several other packages which offer a similar functionality:
%
\begin{itemize}
\item
The packages
\href{http://ctan.org/pkg/docmute}{\textsf{docmute}},
\href{http://ctan.org/pkg/includex}{\textsf{includex}} and
\href{http://ctan.org/pkg/standalone}{\textsf{standalone}}
provide commands to include only the document body of
a child file thus allowing both files to be compiled individually.
\item
The packages \href{http://ctan.org/pkg/subdocs}{\textsf{subdocs}}
and \href{http://ctan.org/pkg/subfiles}{\textsf{subfiles}}
provide structures in which the main and child documents can be
encapsulated and allowing them to be compiled individually.
The inclusion mechanism is different from the conventional |\include|.
\item
The package \href{http://ctan.org/pkg/combine}{\textsf{combine}}
is an elaborate solution to combine several documents into one.
\end{itemize}
%
See also the CTAN topic \href{http://ctan.org/topic/subdocs}{\textsf{subdocs}}
for further related packages.
The present package differs from the above solutions in that
a document structure constructed with the conventional |\include| mechanism
just needs two extra commands at the top of every file
such that all constituent files can be compiled individually.

%%%%%%%%%%%%%%%%%%%%%%%%%%%%%%%%%%%%%%%%%%%%%%%%%%%%%%%%%%%%%%%%%%%%%%%%%%%%%%%%
%\subsection{Feature Suggestions}
%
%The following is a list of features which may be useful for future
%versions of this package:
%%
%\begin{itemize}
%\item
%\ldots
%\end{itemize}

%%%%%%%%%%%%%%%%%%%%%%%%%%%%%%%%%%%%%%%%%%%%%%%%%%%%%%%%%%%%%%%%%%%%%%%%%%%%%%%%
\subsection{Revision History}

%%%%%%%%%%%%%%%%%%%%%%%%%%%%%%%%%%%%%%%%
\paragraph{v2.0:} 2018/12/30

\begin{itemize}
\item
immediate forward processing
\item
added |\childdocby| mechanism
\item
manual restructured
\end{itemize}

%%%%%%%%%%%%%%%%%%%%%%%%%%%%%%%%%%%%%%%%
\paragraph{v1.6:} 2018/01/17

\begin{itemize}
\item
application for development of include files
\item
corrections to manual
\end{itemize}

%%%%%%%%%%%%%%%%%%%%%%%%%%%%%%%%%%%%%%%%
\paragraph{v1.5:} 2017/05/21

\begin{itemize}
\item
more complete structuring introduced
\item
|\childdocof| introduced
\item
|\childdoc| renamed to |\childdocmain|
\item
|\childredirect| renamed to |\childdocforward| and |\childdocforwardprefix|
and functionality expanded
\end{itemize}

%%%%%%%%%%%%%%%%%%%%%%%%%%%%%%%%%%%%%%%%
\paragraph{v1.0:} 2017/04/27

\begin{itemize}
\item
manual and install package
\item
first version published on CTAN
\end{itemize}

%%%%%%%%%%%%%%%%%%%%%%%%%%%%%%%%%%%%%%%%
\paragraph{v0.6:} 2017/04/26

\begin{itemize}
\item
redirection mechanism added
\end{itemize}

%%%%%%%%%%%%%%%%%%%%%%%%%%%%%%%%%%%%%%%%
\paragraph{v0.5:} 2017/04/26

\begin{itemize}
\item
functionality in definition file
\end{itemize}


%%%%%%%%%%%%%%%%%%%%%%%%%%%%%%%%%%%%%%%%%%%%%%%%%%%%%%%%%%%%%%%%%%%%%%%%%%%%%%%%
%%%%%%%%%%%%%%%%%%%%%%%%%%%%%%%%%%%%%%%%%%%%%%%%%%%%%%%%%%%%%%%%%%%%%%%%%%%%%%%%
%%%%%%%%%%%%%%%%%%%%%%%%%%%%%%%%%%%%%%%%%%%%%%%%%%%%%%%%%%%%%%%%%%%%%%%%%%%%%%%%
\appendix

\settowidth\MacroIndent{\rmfamily\scriptsize 000\ }

 \DocInput{childdoc.dtx}

\end{document}
%</driver>
% \fi
%
% %%%%%%%%%%%%%%%%%%%%%%%%%%%%%%%%%%%%%%%%%%%%%%%%%%%%%%%%%%%%%%%%%%%%%%%%%%%%%%
% %%%%%%%%%%%%%%%%%%%%%%%%%%%%%%%%%%%%%%%%%%%%%%%%%%%%%%%%%%%%%%%%%%%%%%%%%%%%%%
% \section{Sample}
%\iffalse
%<*samplemain>
%\fi
%
% The following presents a sample document
% with two chapters, two parts, a title page,
% a compile flag as well as three forwarding files to set the flag.
% It consists of eight |.tex| files:
% \begin{center}
% \begin{tabular}{ll}
% |cdocsamp.tex|&main file\\
% |cdocsch1.tex|&include file for chapter 1\\
% |cdocsch2.tex|&include file for chapter 2\\
% |cdocspt3.tex|&include file for part 3\\
% |cdocspt4.tex|&include file for part 4\\
% |cdocsdrf.tex|&forwarding file for main file in draft mode\\
% |cdocsfi1.tex|&forwarding file for final version of chapter 1\\
% |cdocsfi2.tex|&forwarding file for final version of chapter 2\\
% \end{tabular}
% \end{center}
% Each of the eight files can be compiled directly by the \LaTeX{} compiler.
%
% %%%%%%%%%%%%%%%%%%%%%%%%%%%%%%%%%%%%%%
% \paragraph{Main File.}
%
% The main file is called |cdocsamp.tex|.
%
% Load the \textsf{childdoc} definitions and
% declare the filename for the main document:
%    \begin{macrocode}
\input{childdoc.def}
\childdocmain{}
%    \end{macrocode}

% Optional override for |\version| flag:
%    \begin{macrocode}
%%\ifchilddoc\else\providecommand{\version}{draft}\fi
%    \end{macrocode}

% Define the default values for the |\version| flag
% (|final| for the main file and |draft| for childs):
%    \begin{macrocode}
\ifchilddoc
\providecommand{\version}{draft}
\else
\providecommand{\version}{final}
\fi
%    \end{macrocode}

% Load the standard document class:
%    \begin{macrocode}
\documentclass[12pt]{article}
%    \end{macrocode}

% Start the document body:
%    \begin{macrocode}
\begin{document}
%    \end{macrocode}

% Declare a title page.
% Print title, part of document being processed and version flag:
%    \begin{macrocode}
\addtocounter{page}{-1}
\begin{center}
{\LARGE\bfseries{}childdoc example\par}
\vspace{1cm}
\ifchilddoc
\ifchilddocmanual part\else chapter\fi:
`\childdocname' of `\childdocjob'\par
\else
main document: `\childdocjob'\par
\fi
version: \version\par
\end{center}
\newpage
%    \end{macrocode}

% Manually include selected file,
% otherwise process as usual:
%    \begin{macrocode}
\ifchilddocmanual
\section*{part `\childdocname'}
\input{\childdocname}
\else
%    \end{macrocode}

% Include the two chapters:
%    \begin{macrocode}
\include{cdocsch1}
\include{cdocsch2}
%    \end{macrocode}

% Include the two parts unless only chapters should be displayed:
%    \begin{macrocode}
\ifchilddoc\else
\section{part three}
\input{cdocspt3}
\section{part four}
\input{cdocspt4}
\fi
%    \end{macrocode}

% Process as usual until here:
%    \begin{macrocode}
\fi
%    \end{macrocode}

% End of document body:
%    \begin{macrocode}
\end{document}
%    \end{macrocode}
%\iffalse
%</samplemain>
%\fi
%
% %%%%%%%%%%%%%%%%%%%%%%%%%%%%%%%%%%%%%%
% \paragraph{Chapter Include Files.}
%
% The include files are called |cdocsch1.tex| and |cdocsch2.tex|.
%
%\iffalse
%<*samplechap1|samplechap2>
%\fi

% Optional override for |\version| flag:
%    \begin{macrocode}
%%\providecommand{\version}{final}
%    \end{macrocode}

% Include the main document:
%    \begin{macrocode}
\input{childdoc.def}
\childdocof{cdocsamp}
%    \end{macrocode}

%\iffalse
%</samplechap1|samplechap2>
%\fi
%
%\iffalse
%<*samplechap1>
%\fi
% Some text for chapter 1:
%    \begin{macrocode}
\section{one}
some text in chapter one
%    \end{macrocode}

%\iffalse
%</samplechap1>
%\fi
% Some text for chapter 2:
%\iffalse
%<*samplechap2>
%\fi
%    \begin{macrocode}
\section{two}
more text in chapter two
%    \end{macrocode}

%\iffalse
%</samplechap2>
%\fi
%
% %%%%%%%%%%%%%%%%%%%%%%%%%%%%%%%%%%%%%%
% \paragraph{Part Include Files.}
%
% The include files are called |cdocspt3.tex| and |cdocspt4.tex|.
%
%\iffalse
%<*samplepart3|samplepart4>
%\fi

% Optional override for |\version| flag:
%    \begin{macrocode}
%%\providecommand{\version}{final}
%    \end{macrocode}

% Include the main document:
%    \begin{macrocode}
\input{childdoc.def}
\childdocby{cdocsamp}
%    \end{macrocode}

%\iffalse
%</samplepart3|samplepart4>
%\fi
%
%\iffalse
%<*samplepart3>
%\fi
% Some text for part 3:
%    \begin{macrocode}
some text in part three
%    \end{macrocode}

%\iffalse
%</samplepart3>
%\fi
% Some text for part 4:
%\iffalse
%<*samplepart4>
%\fi
%    \begin{macrocode}
more text in part four
%    \end{macrocode}

%\iffalse
%</samplepart4>
%\fi
%
% %%%%%%%%%%%%%%%%%%%%%%%%%%%%%%%%%%%%%%
% \paragraph{Forwarding for a Complete Draft.}
%
% The following forwarding file |cdocsdrf.tex|
% compiles the main document in draft mode:
%\iffalse
%<*sampledraft>
%\fi
%    \begin{macrocode}
\def\version{draft}
\input{childdoc.def}
\childdocforward{cdocsamp}
%    \end{macrocode}

%\iffalse
%</sampledraft>
%\fi
%
% %%%%%%%%%%%%%%%%%%%%%%%%%%%%%%%%%%%%%%
% \paragraph{Forwarding for Final Version of the Chapters.}
%
% The following forwarding files |cdocsfn1.tex| and |cdocsfn2.tex|
% (with identical content)
% compile the final versions of the child documents
% |cdocsch1.tex| and |cdocsch2.tex|, respectively:
%\iffalse
%<*samplefinal>
%\fi
%    \begin{macrocode}
\def\version{final}
\input{childdoc.def}
\childdocforwardprefix[cdocsamp]{cdocsfn}{cdocsch}
%    \end{macrocode}

%\iffalse
%</samplefinal>
%\fi
%
% %%%%%%%%%%%%%%%%%%%%%%%%%%%%%%%%%%%%%%
% \paragraph{Command Line Processing.}
%
% The following three command lines generate the output files
% |cdocscld|, |cdocscl1| and |cdocscl2|
% which should be identical to
% |cdocsdrf|, |cdocsch1| and |cdocsfn2|, respectively:
% \begin{center}
% \begin{tabular}{l}
% |latex -jobname cdocscld \|\\
% |  "\def\version{draft}\input{childdoc.def}\childdocforward{cdocsamp}"|\\
% |latex -jobname cdocscl1 \|\\
% |  "\input{childdoc.def}\childdocforward[cdocsamp]{cdocsch1}"|\\
% |latex -jobname cdocscl2 \|\\
% |  "\def\version{final}\input{childdoc.def}\childdocforward{cdocsch2}"|
% \end{tabular}
% \end{center}
% Note that the trailing backslash on each first line
% merely continues the input to the second line
% (for convenient cut ant paste).
% Furthermore, the command |latex| can be replaced by any
% of its alternative versions such as |pdflatex|.
%
% %%%%%%%%%%%%%%%%%%%%%%%%%%%%%%%%%%%%%%%%%%%%%%%%%%%%%%%%%%%%%%%%%%%%%%%%%%%%%%
% %%%%%%%%%%%%%%%%%%%%%%%%%%%%%%%%%%%%%%%%%%%%%%%%%%%%%%%%%%%%%%%%%%%%%%%%%%%%%%
% \section{Implementation}
%\iffalse
%<*package>
%\fi
%
% This section describes the definitions file |childdoc.def|.

% The definitions cannot be loaded using |\usepackage| or |\RequirePackage|
% which has a mechanism to prevent loading a style file more than once.
% When loading the definitions by means of |\input|
% multiple instances have to be prevented manually:
%\iffalse
%This code needs to be before the `\ProvidesFile' directive
%which is defined at the beginning of this file.
%Therefore it is also placed there and commented out here.
%</package>
%<*discard>
%\fi
%    \begin{macrocode}
\ifdefined\childdocmain\endinput\fi
%    \end{macrocode}
%\iffalse
%</discard>
%<*package>
%\fi
%
% \macro{\ifchilddoc}
% \macro{\ifchilddocmanual}
% The conditional |\ifchilddoc| tells whether a
% child (true) or main (false) document is being compiled.
% The conditional |\ifchilddocmanual| tells whether
% the |\includeonly| mechanism is used (false) or
% the selection of child files must be performed manually (true).
% The definitions initialise to false:
%    \begin{macrocode}
\newif\ifchilddoc
\newif\ifchilddocmanual
%    \end{macrocode}

% \macro{\childdocname}
% \macro{\childdocjob}
% The macro |\childdocname| stores the name of the main document
% to be compiled. The macro |\childdocjob| stores the name of
% the document on which the \LaTeX{} compiler was originally invoked.
% The content of |\jobname| cannot be compared
% to filenames specified in the source due to different catcodes.
% The following code rescans |\jobname|, stores the result
% in |\childdocname| and saves a copy in |\childdocjob|:
%    \begin{macrocode}
\edef\childdocname{\scantokens\expandafter{\jobname\noexpand}}
\let\childdocjob\childdocname
%    \end{macrocode}

% \macro{\childdocdisable}
% The macro |\childdocdisable| prevents the main file
% from being processed more than once.
% At this stage, the main document command |\childdocmain|
% is assumed to be called once again where it should do nothing.
% Any subsequent call to it should prevent
% a secondary processing of the main document
% It overwrites the forwarding commands
% |\childdocof| and |\childdocforward|
% with empty macros to prevent further inclusions of the main document:
%    \begin{macrocode}
\newcommand{\childdocdisable}
{
  \renewcommand{\childdocmain}[1]{\renewcommand{\childdocmain}[1]{\endinput}}
  \renewcommand{\childdocof}[1]{}
  \renewcommand{\childdocby}[2][]{}
  \renewcommand{\childdocforward}[2][]{}
  \renewcommand{\childdocdisable}{}
}
%    \end{macrocode}

% \macro{\childdocmain}
% The macro |\childdocmain| is to be called at the top of the main file
% with nothing or the main filename (without extension) as argument.
% First, it breaks loops.
% If the argument is not empty and does not match |\childdocname|
% (which is set by the first inclusion of |childdoc.def|),
% |\ifchilddoc| is set to true, |\includeonly| is applied to the child file
% and |\jobname| is set to the main file
% (for proper handling of |.aux| files):
%    \begin{macrocode}
\newcommand{\childdocmain}[1]
{
  \childdocdisable\childdocmain{}
  \if?#1?\else
    \begingroup
      \def\childdoctmp{#1}
      \ifx\childdoctmp\childdocname
        \def\childdoctmp{}
      \else
        \def\childdoctmp
        {
          \childdoctrue
          \includeonly{\childdocname}
          \def\childdocjob{#1}
          \def\jobname{#1}
        }
      \fi
      \expandafter
    \endgroup
    \childdoctmp
  \fi
}
%    \end{macrocode}

% \macro{\childdocof}
% The command |\childdocof| redirects
% compilation to the main file |#1|.
%    \begin{macrocode}
\newcommand{\childdocof}[1]
{
  \childdocdisable
  \childdoctrue
  \includeonly{\childdocname}
  \def\jobname{#1}
  \def\childdocjob{#1}
  \input{#1}
}
%    \end{macrocode}

% \macro{\childdocby}
% The command |\childdocby| ....
%    \begin{macrocode}
\newcommand{\childdocby}[2][]
{
  \childdocdisable
  \childdoctrue
  \childdocmanualtrue
  \if?#1?\else
    \def\jobname{#2}
  \fi
  \def\childdocjob{#2}
  \input{#2}
  \endinput
}
%    \end{macrocode}

% \macro{\childdocforward}
% The command |\childdocforward| redirects
% compilation to the main file or
% (if the optional argument is given) a child file.
% Parameters are set as if the main file
% or a child file starting with |\childdocof| was compiled.
% Then compilation is handed over to the main file:
%    \begin{macrocode}
\newcommand{\childdocforward}[2][]
{
  \begingroup
    \if?#1?
      \def\childdoctmp
      {
        \def\childdocname{#2}
        \def\childdocjob{#2}
        \def\jobname{#2}
        \input{#2}
        \endinput
      }
    \else
      \def\childdoctmp
      {
        \childdocdisable
        \def\childdocname{#2}
        \childdoctrue
        \includeonly{#2}
        \def\childdocjob{#1}
        \def\jobname{#1}
        \input{#1}
        \endinput
      }
    \fi
    \expandafter
  \endgroup
  \childdoctmp
}
%    \end{macrocode}

% \macro{\childdocforwardprefix}
% The command |\childdocforwardprefix| redirects
% compilation to the main or a child file by means of a pattern.
% The prefix |#1| in the current filename is replaced by |#2|
% and the suffix of the current filename is kept
% (it is assumed that the filename does not contain the substring `|~~~|'
% which is used as a delimiter).
% Compilation is handed over to the new file by |\childdocforward|:
%    \begin{macrocode}
\newcommand{\childdocforwardprefix}[3][]
{
  \begingroup
    \def\childdocextract #2##1~~~{\def\childdoctmp{\childdocforward[#1]{#3##1}}}
    \expandafter\childdocextract\childdocname~~~
    \expandafter
  \endgroup
  \childdoctmp
}
%    \end{macrocode}

% \macro{\childdoc}
% The deprecated macro |\childdoc| is a legacy version of |\childdocmain|:
%    \begin{macrocode}
\newcommand{\childdoc}{\childdocmain}
%    \end{macrocode}

% \macro{\childdocredirect}
% The deprecated macro |\childdocredirect| is a legacy version
% of |\childdocforward| and |\childdocforwardprefix|:
%    \begin{macrocode}
\newcommand{\childdocredirect}[2][]
{
  \begingroup
    \if?#1?
      \def\childdoctmp{\childdocforward{#2}}
    \else
      \def\childdoctmp{\childdocforwardprefix{#1}{#2}}
    \fi
    \expandafter
  \endgroup
  \childdoctmp
}
%    \end{macrocode}

%\iffalse
%</package>
%\fi
%
\endinput
|\\
|\childdocforwardprefix{final}{child}|
\end{tabular}
\end{center}
%

Note that when several versions of a main file and/or of each child file
are to be generated, it may be convenient to set up a |Makefile| or
shell script to automatise the process.

%%%%%%%%%%%%%%%%%%%%%%%%%%%%%%%%%%%%%%%%%%%%%%%%%%%%%%%%%%%%%%%%%%%%%%%%%%%%%%%%
\subsection{Command Line Processing}
\label{sec:commandline}

The effect of redirection files can also be achieved by invoking
the \LaTeX{} compiler with a more elaborate command line.
Most conveniently this should be done as part
of a shell script or a |Makefile|.

When using \textsf{childdoc} in the main file, the following
command lines effectively perform a redirection
(note that depending on the shell being used,
backslashes may have to be doubled: `|\|' $\to$ `|\\|'):
%
\begin{center}
|... -jobname "|\textit{target}|" |\\|"|[\textit{flags}]%
|% \iffalse
%
% childdoc.dtx Copyright (C) 2017-2018 Niklas Beisert
%
% This work may be distributed and/or modified under the
% conditions of the LaTeX Project Public License, either version 1.3
% of this license or (at your option) any later version.
% The latest version of this license is in
%   http://www.latex-project.org/lppl.txt
% and version 1.3 or later is part of all distributions of LaTeX
% version 2005/12/01 or later.
%
% This work has the LPPL maintenance status `maintained'.
%
% The Current Maintainer of this work is Niklas Beisert.
%
% This work consists of the files childdoc.dtx and childdoc.ins
% and the derived files childdoc.def and cdocsamp.tex with
% cdocsch1.tex, cdocsch2.tex, cdocsdrf.tex, cdocsfn1.tex, cdocsfn2.tex.
%
%<package>\ifdefined\childdocmain\endinput\fi
%<package>\ProvidesFile{childdoc.def}[2018/12/30 v2.0 child document driver]
%<samplemain>\ProvidesFile{cdocsamp.tex}[2018/12/30 v2.0 sample for childdoc]
%<*driver>
%\ProvidesFile{childdoc.drv}[2018/12/30 v2.0 childdoc reference manual file]
\PassOptionsToClass{10pt,a4paper}{article}
\documentclass{ltxdoc}

\usepackage[margin=35mm]{geometry}
\usepackage{hyperref}
\usepackage{hyperxmp}
\usepackage[usenames]{color}

\hypersetup{colorlinks=true}
\hypersetup{pdfstartview=FitH}
\hypersetup{pdfpagemode=UseNone}
\hypersetup{pdfsource={}}
\hypersetup{pdflang={en-UK}}
\hypersetup{pdfcopyright={Copyright 2017-2018 Niklas Beisert.
  This work may be distributed and/or modified under the
  conditions of the LaTeX Project Public License, either version 1.3
  of this license or (at your option) any later version.}}
\hypersetup{pdflicenseurl={http://www.latex-project.org/lppl.txt}}
\hypersetup{pdfcontactaddress={ETH Zurich, ITP, HIT K,
  Wolfgang-Pauli-Strasse 27}}
\hypersetup{pdfcontactpostcode={8093}}
\hypersetup{pdfcontactcity={Zurich}}
\hypersetup{pdfcontactcountry={Switzerland}}
\hypersetup{pdfcontactemail={nbeisert@itp.phys.ethz.ch}}
\hypersetup{pdfcontacturl={http://people.phys.ethz.ch/\xmptilde nbeisert/}}

\newcommand{\secref}[1]{\hyperref[#1]{section \ref*{#1}}}

\parskip1ex
\parindent0pt
\let\olditemize\itemize
\def\itemize{\olditemize\parskip0pt}

\begin{document}

\title{The \textsf{childdoc} Package}
\hypersetup{pdftitle={The childdoc Package}}
\author{Niklas Beisert\\[2ex]
  Institut f\"ur Theoretische Physik\\
  Eidgen\"ossische Technische Hochschule Z\"urich\\
  Wolfgang-Pauli-Strasse 27, 8093 Z\"urich, Switzerland\\[1ex]
  \href{mailto:nbeisert@itp.phys.ethz.ch}
  {\texttt{nbeisert@itp.phys.ethz.ch}}}
\hypersetup{pdfauthor={Niklas Beisert}}
\hypersetup{pdfsubject={Manual for the LaTeX2e Package childdoc}}
\date{30 December 2018, \textsf{v2.0}}
\maketitle

\begin{abstract}\noindent
\textsf{childdoc} is a \LaTeXe{} package
that enables the direct compilation
of document sections included by |\include|
to individual files.
\end{abstract}

\begingroup
\parskip0ex
\tableofcontents
\endgroup

%%%%%%%%%%%%%%%%%%%%%%%%%%%%%%%%%%%%%%%%%%%%%%%%%%%%%%%%%%%%%%%%%%%%%%%%%%%%%%%%
%%%%%%%%%%%%%%%%%%%%%%%%%%%%%%%%%%%%%%%%%%%%%%%%%%%%%%%%%%%%%%%%%%%%%%%%%%%%%%%%
\section{Introduction}

\LaTeX{} provides a mechanism to structure a large document (such as a book)
into a main file and several child files (containing the chapters)
using the |\include| command.
This mechanism is beneficial for documents
which span hundreds of pages in order to
make the source file(s) more manageable.
Moreover, compilation can be restricted to
selected child files by means of the |\includeonly| command.
The latter feature can be used to reduce the compilation time while editing
(this was significantly more useful in the earlier days of \LaTeX{})
or to generate a smaller document which is easier to navigate.
Another application of |\includeonly| is to generate
documents consisting of selected parts of the complete document.

However, there are a few drawbacks of the plain |\include| mechanism:
\begin{itemize}
\item
The child files cannot be compiled on their own,
they can only be compiled via the main file.
A naive editing environment
(such as a text editor with an option
to have the current file processed by \LaTeX)
may require one to switch to the main file before compiling;
attempting to compile the child file produces errors.
\item
The main file must be modified (each time)
to adjust the |\includeonly| command
to the present needs. This easily leaves the main file in a messy state.
\item
The generated document will always carry the filename
of the main document. This is inconvenient if
several child files are to be compiled and
to be kept for distribution.
\end{itemize}

The present package provides a simple interface
to make child files individually compilable by \LaTeX{}.
Compiling a child file then has the same effect as compiling
the main file with an |\includeonly| command
to select the appropriate child.
Moreover the generated document will carry the name of the child
rather than the main file.
This resolves all three above issues.

This feature is meant to make the editing of books,
thesis documents and lecture notes somewhat more convenient.
However, the package can also be used efficiently for
composing a series of documents (such as exercise sheets)
which are typically distributed individually.
It then assists the author in generating the individual documents
(potentially in different versions)
as well as a document containing the collected series.
Another application is in developing style files
or other kinds of included material
where compilation of the style file could redirect
to a sample or test file.

%%%%%%%%%%%%%%%%%%%%%%%%%%%%%%%%%%%%%%%%%%%%%%%%%%%%%%%%%%%%%%%%%%%%%%%%%%%%%%%%
%%%%%%%%%%%%%%%%%%%%%%%%%%%%%%%%%%%%%%%%%%%%%%%%%%%%%%%%%%%%%%%%%%%%%%%%%%%%%%%%
\section{Usage}

First of all, the package \textsf{childdoc} is \emph{not} a standard
\LaTeXe{} |.sty| style file! Therefore it needs to be invoked in
a non-standard way.

%%%%%%%%%%%%%%%%%%%%%%%%%%%%%%%%%%%%%%%%%%%%%%%%%%%%%%%%%%%%%%%%%%%%%%%%%%%%%%%%
\subsection{Included Files}
\label{sec:include}

%%%%%%%%%%%%%%%%%%%%%%%%%%%%%%%%%%%%%%%%
\DescribeMacro{\childdocmain}
To use the package, add the commands
\begin{center}
\begin{tabular}{l}
|\input{childdoc.def}|\\
|\childdocmain{}|\\
\end{tabular}
\end{center}
at the very top of the main \LaTeX{} file,
in particular \emph{before} the |\documentclass| statement!
The argument of |\childdocmain| should be left empty
(but it must be present).

%%%%%%%%%%%%%%%%%%%%%%%%%%%%%%%%%%%%%%%%
\DescribeMacro{\childdocof}
Furthermore, add the commands
\begin{center}
\begin{tabular}{l}
|\input{childdoc.def}|\\
|\childdocof{|\textit{main}|}|\\
\end{tabular}
\end{center}
at the top of every child file \textit{child}
which is included by |\include{|\textit{child}|}|
from within the main file
(or at least for those files to be compiled individually).
The argument \textit{main} must be the filename of the main file.

There are a couple of
considerations in setting up the main and child documents:

%%%%%%%%%%%%%%%%%%%%%%%%%%%%%%%%%%%%%%%%
\paragraph{Restrictions.}

Please note the following restrictions:
\begin{itemize}
\item
|\childdocmain| must be called with one argument \textit{main}
to ensure compatibility with earlier version of the package.
It must either be empty (|\childdocmain{}|)
or precisely match the filename of the main file in which it is specified.
See \secref{sec:detection} for further information.
\item
The filename \textit{main} must be specified without the |.tex| extension.
\item
The filename \textit{main} is case sensitive
(even in case-insensitive file systems)
due to internal string comparison.
\item
The argument \textit{main} should be fully expanded, it cannot be a macro.
\item
Subdirectories and special characters should be avoided in filenames.
\item
The command |\childdocmain{|\textit{main}|}| must be followed by a whitespace.
It should not be followed immediately by another command
or by a comment mark `|%|'.
This is because the \TeX{} parser reads the token immediately following
the argument of |\childdocmain| and puts it
at the beginning of every child section;
however, a white\-space is ignored.
\end{itemize}

%%%%%%%%%%%%%%%%%%%%%%%%%%%%%%%%%%%%%%%%
\paragraph{Content of Main File.}

It is advisable to place all content in the child files included by |\include|.
Any output contained in the main file will appear in all child documents
unless suppressed manually;
it cannot be suppressed automatically by the |\includeonly| directive
and thus should normally be avoided.
A method to include some content in the main file
by means of conditional processing is described in \secref{sec:conditional}.

%%%%%%%%%%%%%%%%%%%%%%%%%%%%%%%%%%%%%%%%
\paragraph{Page Numbering.}

When only a part of the document is compiled,
the appropriate numbering of pages
(as well as other status parameters)
is determined from the |.aux| files.
The latter contain information from previous passes.
However this information needs to propagate through
all intermediate child documents.
Therefore the page numbering in child documents may well
be inconsistent until the complete document is compiled at least once.

A useful (if unconventional) way to always ensure a consistent
page numbering is to restart the numbering in each child document
and denote the pages by `\textit{child}|.|\textit{page}'
where \textit{child} represents the chapter/section number of the child file.
This can be achieved by the command
|\numberwithin{page}{|\textit{child}|}|
of the \textsf{amsmath} package
where \textit{child} can be |chapter| or |section|
depending on the chosen structuring.
Alternatively, one can modify the macro |\thepage| appropriately
and reset the counter |page| at the start of each child file.

%%%%%%%%%%%%%%%%%%%%%%%%%%%%%%%%%%%%%%%%%%%%%%%%%%%%%%%%%%%%%%%%%%%%%%%%%%%%%%%%
\subsection{Conditional Processing}
\label{sec:conditional}

The package provides a mechanism to compile different versions
of a document. To customise the versions further some conditional processing
can come in handy to distinguish which version is being compiled.
The package provides two macros to describe the compilation context:

%%%%%%%%%%%%%%%%%%%%%%%%%%%%%%%%%%%%%%%%
\DescribeMacro{\ifchilddoc}
The conditional |\ifchilddoc| distinguishes between the compilation of
child documents and the main document:
%
\begin{center}
|\ifchilddoc |\textit{child-code}| |[|\||else |\textit{main-code}]| \||fi|
\end{center}

%%%%%%%%%%%%%%%%%%%%%%%%%%%%%%%%%%%%%%%%
\DescribeMacro{\childdocname}
\DescribeMacro{\childdocjob}
The macro |\childdocname| contains the filename (without extension)
of the main or child file being processed.
Note that |\childdocjob| will always contain the name of the main file.

%%%%%%%%%%%%%%%%%%%%%%%%%%%%%%%%%%%%%%%%
\paragraph{Title Page.}

Conditional processing can be used to include a title or banner page
in the main document when proper precautions are taken.
Importantly, the code in the main file should ensure that the page counter
(as well as other status parameters which are stored in the |.aux| files)
takes the same value after the conditional processing.
Otherwise the page numbers may take divergent values
depending on which part is compiled.

For example, a title page could be declared by:
%
\begin{center}
\begin{tabular}{l}
|\ifchilddoc\||else|\\
|\addtocounter{page}{-1}|\\
\textit{code for title page}\\
|\newpage|\\
|\||fi|
\end{tabular}
\end{center}
%
A banner page for the child documents can be generated by:
%
\begin{center}
\begin{tabular}{l}
|\ifchilddoc|\\
|\addtocounter{page}{-1}|\\
\textit{code for banner page}\\
|\newpage|\\
|\||fi|
\end{tabular}
\end{center}
%
Here one could write a message such as:
\begin{center}
|This is the part \childdocname{} of \childdocjob{}.|
\end{center}

%%%%%%%%%%%%%%%%%%%%%%%%%%%%%%%%%%%%%%%%%%%%%%%%%%%%%%%%%%%%%%%%%%%%%%%%%%%%%%%%
\subsection{Flags}
\label{sec:flags}

The package makes it easy to generate different versions
of the main or child documents.
To this end compilation flags can be defined
and assigned different default values.
They will be particularly useful in conjunction
with the forwarding mechanism described in \secref{sec:forward}.

For example, it may be useful to have a flag |\version|
which can be set to |draft| or |final|.
The document source will contain some conditional code
depending on the value of |\version|.
Suppose further, the flag should default to |final| for the main file
and to |draft| for child files
which is a natural assignment for editing the document.
This is achieved by placing the following code
in the preamble of the main document
(below the |\childdocmain| directive):
%
\begin{center}
\begin{tabular}{l}
|\ifchilddoc|\\
|\providecommand{\version}{draft}|\\
|\||else|\\
|\providecommand{\version}{final}|\\
|\||fi|
\end{tabular}
\end{center}
%
The definition by |\providecommand| makes sure
that previous definitions are not overwritten.
Further statements |\providecommand{\version}{...}|
can thus be added before the above code to override it.

For the main file, one might add a line
(between |\childdocmain| and the above block)
%
\begin{center}
|%\ifchilddoc\||else\providecommand{\version}{draft}\||fi|
\end{center}
%
which can be uncommented to produce a draft version.
Likewise one can add a line to the very top of a child file
(above the |\childdocof{|\textit{main}|}| directive)
%
\begin{center}
|%\providecommand{\version}{final}|
\end{center}
%
which can be uncommented to produce the final version of this child document.

%%%%%%%%%%%%%%%%%%%%%%%%%%%%%%%%%%%%%%%%%%%%%%%%%%%%%%%%%%%%%%%%%%%%%%%%%%%%%%%%
\subsection{Forwarding}
\label{sec:forward}

Different versions of the main or child documents
using compilation flags as described in \secref{sec:flags}
can be (permanently) stored in different files
for convenient compilation, viewing and distribution.
To this end, the package defines a command
to pass on compilation to a different file:

%%%%%%%%%%%%%%%%%%%%%%%%%%%%%%%%%%%%%%%%
\DescribeMacro{\childdocforward}
The command |\childdocforward| redirects processing to
another source file:
%
\begin{center}
\begin{tabular}{l}
|\input{childdoc.def}|\\
|\childdocforward[|\textit{main}|]{|\textit{dest}|}|\\
\end{tabular}
\end{center}
%
The argument \textit{dest} is the destination file
(without extension).
It should be the main file or one of the child files.
Note that further \textsf{childdoc} directives
such as |\childdocof| and |\childdocforward|
in the indicated file will be processed in this form.
The optional argument \textit{main}
passes on directly to the main file \textit{main}
while pretending to compile the child \textit{dest}.
This form behaves as if \textit{dest}
issues |\childdocof{|\textit{main}|}| right away,
and no further \textsf{childdoc} directives will be processed.

%%%%%%%%%%%%%%%%%%%%%%%%%%%%%%%%%%%%%%%%
\DescribeMacro{\...prefix}
In the alternative form |\childdocforwardprefix|,
%
\begin{center}
\begin{tabular}{l}
|\input{childdoc.def}|\\
|\childdocforwardprefix[|\textit{main}|]{|\textit{prefix}|}{|\textit{dest}|}|
\end{tabular}
\end{center}
%
the destination file is determined by a pattern
depending on the current file:
To make this work, the current file must be called
`{\textit{prefix}\hspace{0.2em}\textit{suffix}}'
with \textit{prefix} matching precisely the argument.
Processing is then passed on to the file
`{\textit{dest}\hspace{0.2em}\textit{suffix}}'.
Surely, the same effect is achieved by
directly specifying the
argument `{\textit{dest}\hspace{0.2em}\textit{suffix}}'
in the first form.
However, that requires to set up a different file
for each child. With the alternative form of the command
all these files can have exactly the same content
which simplifies setting them up and maintaining them.

For example, the following file |draft.tex|
with a compilation flag |\version| as described in \secref{sec:flags}
compiles the main document as a draft:
%
\begin{center}
\begin{tabular}{l}
|\def\version{draft}|\\
|\input{childdoc.def}|\\
|\childdocforward{|\textit{main}|}|
\end{tabular}
\end{center}
%
Likewise, the following files |final|\textit{nn}|.tex|
compile the final version of the child document
|child|\textit{nn}|.tex|:
%
\begin{center}
\begin{tabular}{l}
|\def\version{final}|\\
|\input{childdoc.def}|\\
|\childdocforwardprefix{final}{child}|
\end{tabular}
\end{center}
%

Note that when several versions of a main file and/or of each child file
are to be generated, it may be convenient to set up a |Makefile| or
shell script to automatise the process.

%%%%%%%%%%%%%%%%%%%%%%%%%%%%%%%%%%%%%%%%%%%%%%%%%%%%%%%%%%%%%%%%%%%%%%%%%%%%%%%%
\subsection{Command Line Processing}
\label{sec:commandline}

The effect of redirection files can also be achieved by invoking
the \LaTeX{} compiler with a more elaborate command line.
Most conveniently this should be done as part
of a shell script or a |Makefile|.

When using \textsf{childdoc} in the main file, the following
command lines effectively perform a redirection
(note that depending on the shell being used,
backslashes may have to be doubled: `|\|' $\to$ `|\\|'):
%
\begin{center}
|... -jobname "|\textit{target}|" |\\|"|[\textit{flags}]%
|\input{childdoc.def}\childdocforward[|\textit{main}|]{|\textit{dest}|}"|
\end{center}
%
Here \textit{target} is the name of the output file,
\textit{main} is the name of the main file
and \textit{dest} is the name of the main or child file to be processed
(all filenames without extensions).
The optional argument \textit{main} can be omitted
if \textit{main} matches \textit{dest}.
Optionally, compilation \textit{flags} can be defined via |\def| commands.
This command line makes the \TeX{} engine believe
it is compiling the file \textit{target}
whose content is specified as the latter parameter.
The provided code then forwards the processing to
\textit{main} or \textit{dest} as described in \secref{sec:forward}.

%%%%%%%%%%%%%%%%%%%%%%%%%%%%%%%%%%%%%%%%%%%%%%%%%%%%%%%%%%%%%%%%%%%%%%%%%%%%%%%%
\subsection{Include by Input}
\label{sec:input}

Including child documents by |\include| has some restrictions by design.
Most notably, the content of a child document always occupies
its own set of pages; pages cannot be shared between child documents.
Usually, this behaviour makes perfect sense
because each child document contain an essential part of the document.
However, in some situations it may be desirable to compose
a document from a collection of parts
without having mandatory page breaks between then.
For this case, the package
provides a mechanism to include parts
by |\input| which can also be processed individually.
However, by construction this mechanism
requires manual handling of the content to be output.

%%%%%%%%%%%%%%%%%%%%%%%%%%%%%%%%%%%%%%%%
\DescribeMacro{\ifchilddocmanual}
The main file should be prepared as usual, see \secref{sec:include}.
However, the document body must make a distinction
between processing of an individual part and of the main document, e.g.:
%
\begin{center}
\begin{tabular}{l}
|\ifchilddocmanual|\\
|\input{\childdocname}|\\
|\||else|\\
\textit{document body with }|\input{|\textit{part}|}|\\
|\||fi|
\end{tabular}
\end{center}
%
The conditional |\ifchilddocmanual| is true whenever
a part to be included by |\input| is being compiled,
and the name of the part is stored in |\childdocname|.

%%%%%%%%%%%%%%%%%%%%%%%%%%%%%%%%%%%%%%%%
\DescribeMacro{\childdocby}
Each part to be included by |\input| should start with:
%
\begin{center}
\begin{tabular}{l}
|\input{childdoc.def}|\\
|\childdocby{|\textit{main}|}|\\
\end{tabular}
\end{center}
%
The directive |\childdocby| is similar to |\childdocof|
described in \secref{sec:include},
but the subsequent selection of content must be done manually.
To that end, both |\ifchilddoc| and |\ifchilddocmanual|
will be true upon processing of a part,
and the name of the part is stored in |\childdocname|.
Note that |\jobname| will be set to the filename of the current part
so that each part receives an individual |.aux| file
that does not interfere with the |.aux| file(s) of the main document.
This behaviour can be altered by the alternative form
|\childdocby[*]{|\textit{main}|}| (with a non-empty optional argument)
which uses the |.aux| file of the main document
by setting |\jobname| to \textit{main}.

%%%%%%%%%%%%%%%%%%%%%%%%%%%%%%%%%%%%%%%%%%%%%%%%%%%%%%%%%%%%%%%%%%%%%%%%%%%%%%%%
\subsection{Driver Development}
\label{sec:driver}

The \textsf{childdoc} mechanism can also be use for the development
of definition files such as \LaTeX{} styles or classes.
This case differs from the above setup with multiple parts
included by |\include| in that no |\includeonly| should be invoked.
This can be achieved by starting the include file
(before |\ProvidesPackage|) with:
%
\begin{center}
\begin{tabular}{l}
|\input{childdoc.def}|\\
|\childdocforward{|\textit{main}|}|\\
\end{tabular}
\end{center}
%
or alternatively with:
%
\begin{center}
\begin{tabular}{l}
|\input{childdoc.def}|\\
|\childdocby{|\textit{main}|}|\\
\end{tabular}
\end{center}
%
Both forms have slightly different effects as described above.
The main file is prepared as usual, see \secref{sec:include}.

%%%%%%%%%%%%%%%%%%%%%%%%%%%%%%%%%%%%%%%%%%%%%%%%%%%%%%%%%%%%%%%%%%%%%%%%%%%%%%%%
\subsection{Legacy Detection}
\label{sec:detection}

The directive |\childdocmain| in the main file can detect
whether the complete document or merely a child is to be compiled
even without using the directive |\childdocof|.
This method is deprecated because it is less robust
and there is no compelling reason to use it;
it is merely provided for backward compatibility
and it may be removed in future versions.

If the detection mechanism is to be used,
it is mandatory to correctly specify
the filename of the main file as the argument of |\childdocmain|:
%
\begin{center}
\begin{tabular}{l}
|\input{childdoc.def}|\\
|\childdocmain{|\textit{main}|}|\\
\end{tabular}
\end{center}
%
If |\jobname| does not match the argument \textit{main} of |\childdocmain|,
it is assumed that |\jobname| points to the child file to be compiled.
When using |\childdocmain| with the main file specified as argument,
it suffices to start a child file
with just |\input{|\textit{main}|}|
without loading of the package and using |\childdocof|.
If instead all processing is done
with the appropriate \textsf{childdoc} directives,
the argument of \textit{main} of |\childdocmain| can be empty.

An alternative version of the command line processing described
in \secref{sec:commandline} using the detection mechanism reads:
%
\begin{center}
|... -jobname "|\textit{target}|" "|[\textit{flags}]%
[|\def\jobname{|\textit{dest}|}|]|\input{|\textit{main}|}"|
\end{center}

%%%%%%%%%%%%%%%%%%%%%%%%%%%%%%%%%%%%%%%%%%%%%%%%%%%%%%%%%%%%%%%%%%%%%%%%%%%%%%%%
\subsection{Manual Code}
\label{sec:manual}

In case one cannot be certain whether the definitions file |childdoc.def|
is installed on the target \TeX{} distribution
and one prefers not to ship it,
it is conceivable to paste a few relevant commands into the sources.

To that end, drop all statements |\input{childdoc.def}|
and perform the replacements as outlined below.
Instead of |\childdocmain{|\textit{main}|}| add the following code
to the top of the main file:
%
\begin{center}
\begin{tabular}{l}
|\||ifdefined\childdocname\endinput\||fi\newif\ifchilddoc|\\
|\edef\childdocname{\scantokens\expandafter{\jobname\noexpand}}|\\
|\def\childdocmain{|\textit{main}|}\||ifx\childdocmain\childdocname\||else|\\
|\childdoctrue\includeonly{\childdocname}\let\jobname\childdocmain\||fi|\\
\end{tabular}
\end{center}
%
Instead of |\childdocof{|\textit{main}|}| just include the main file
at the top of each child file:
%
\begin{center}
|\input{|\textit{main}|}|
\end{center}
%
A simple redirection |\childdocforward{|\textit{dest}|}| is achieved by:
%
\begin{center}
|\def\jobname{|\textit{dest}|}\input{\jobname}|
\end{center}
%
The redirection with prefix
|\childdocforwardprefix[|\textit{prefix}|]{|\textit{dest}|}|
is accomplished by:
%
\begin{center}
\begin{tabular}{l}
|{\edef\jobname{\scantokens\expandafter{\jobname\noexpand}}|\\
|\def\redirectjob |\textit{prefix}|#1~~~{\gdef\jobname{|\textit{dest}|#1}}|\\
|\expandafter\redirectjob\jobname~~~}\input{\jobname}|
\end{tabular}
\end{center}

In an alternative approach,
child documents can be compiled by a specific command line
without additional code or specific definitions:
%
\begin{center}
|... -jobname "|\textit{target}|" "|[\textit{flags}]%
|\includeonly{|\textit{dest}|}\input{|\textit{main}|}"|
\end{center}
%

%%%%%%%%%%%%%%%%%%%%%%%%%%%%%%%%%%%%%%%%%%%%%%%%%%%%%%%%%%%%%%%%%%%%%%%%%%%%%%%%
%%%%%%%%%%%%%%%%%%%%%%%%%%%%%%%%%%%%%%%%%%%%%%%%%%%%%%%%%%%%%%%%%%%%%%%%%%%%%%%%
\section{Information}

%%%%%%%%%%%%%%%%%%%%%%%%%%%%%%%%%%%%%%%%%%%%%%%%%%%%%%%%%%%%%%%%%%%%%%%%%%%%%%%%
\subsection{Copyright}

Copyright \copyright{} 2017--2018 Niklas Beisert

This work may be distributed and/or modified under the
conditions of the \LaTeX{} Project Public License, either version 1.3
of this license or (at your option) any later version.
The latest version of this license is in
  \url{http://www.latex-project.org/lppl.txt}
and version 1.3 or later is part of all distributions of \LaTeX{}
version 2005/12/01 or later.

This work has the LPPL maintenance status `maintained'.

The Current Maintainer of this work is Niklas Beisert.

This work consists of the files |README.txt|, |childdoc.ins| and |childdoc.dtx|
as well as the derived files |childdoc.def|, |cdocsamp.tex|
with |cdocsch1.tex|, |cdocsch2.tex|, |cdocspt3.tex|, |cdocspt4.tex|,
|cdocsdrf.tex|, |cdocsfn1.tex|, |cdocsfn2.tex|
as well as |childdoc.pdf|.

%%%%%%%%%%%%%%%%%%%%%%%%%%%%%%%%%%%%%%%%%%%%%%%%%%%%%%%%%%%%%%%%%%%%%%%%%%%%%%%%
\subsection{Files and Installation}

The package consists of the files:
%
\begin{center}
\begin{tabular}{ll}
    |README.txt|   & readme file \\
    |childdoc.ins| & installation file \\
    |childdoc.dtx| & source file \\
    |childdoc.def| & definition file \\
    |cdocsamp.tex| & sample main file \\
    |cdocsch1.tex| & sample include file \\
    |cdocsch2.tex| & sample include file \\
    |cdocspt3.tex| & sample part file \\
    |cdocspt4.tex| & sample part file \\
    |cdocsdrf.tex| & sample redirection file \\
    |cdocsfn1.tex| & sample redirection file \\
    |cdocsfn2.tex| & sample redirection file \\
    |childdoc.pdf| & manual
\end{tabular}
\end{center}
%
The distribution consists of the files
|README.txt|, |childdoc.ins| and |childdoc.dtx|.
%
\begin{itemize}
\item
Run (pdf)\LaTeX{} on |childdoc.dtx|
to compile the manual |childdoc.pdf| (this file).
\item
Run \LaTeX{} on |childdoc.ins| to create the definitions file |childdoc.def|
and the sample |cdocsamp.tex| with include files
|cdocsch1.tex|, |cdocsch2.tex|, |cdocspt3.tex|, |cdocspt4.tex|,
|cdocsdrf.tex|, |cdocsfn1.tex|, |cdocsfn2.tex|.
Then copy the file |childdoc.def| to an appropriate directory of your \LaTeX{}
distribution, e.g.\ \textit{texmf-root}|/tex/latex/childdoc|.
\end{itemize}

%%%%%%%%%%%%%%%%%%%%%%%%%%%%%%%%%%%%%%%%%%%%%%%%%%%%%%%%%%%%%%%%%%%%%%%%%%%%%%%%
\subsection{Related CTAN Packages}

There are several other packages which offer a similar functionality:
%
\begin{itemize}
\item
The packages
\href{http://ctan.org/pkg/docmute}{\textsf{docmute}},
\href{http://ctan.org/pkg/includex}{\textsf{includex}} and
\href{http://ctan.org/pkg/standalone}{\textsf{standalone}}
provide commands to include only the document body of
a child file thus allowing both files to be compiled individually.
\item
The packages \href{http://ctan.org/pkg/subdocs}{\textsf{subdocs}}
and \href{http://ctan.org/pkg/subfiles}{\textsf{subfiles}}
provide structures in which the main and child documents can be
encapsulated and allowing them to be compiled individually.
The inclusion mechanism is different from the conventional |\include|.
\item
The package \href{http://ctan.org/pkg/combine}{\textsf{combine}}
is an elaborate solution to combine several documents into one.
\end{itemize}
%
See also the CTAN topic \href{http://ctan.org/topic/subdocs}{\textsf{subdocs}}
for further related packages.
The present package differs from the above solutions in that
a document structure constructed with the conventional |\include| mechanism
just needs two extra commands at the top of every file
such that all constituent files can be compiled individually.

%%%%%%%%%%%%%%%%%%%%%%%%%%%%%%%%%%%%%%%%%%%%%%%%%%%%%%%%%%%%%%%%%%%%%%%%%%%%%%%%
%\subsection{Feature Suggestions}
%
%The following is a list of features which may be useful for future
%versions of this package:
%%
%\begin{itemize}
%\item
%\ldots
%\end{itemize}

%%%%%%%%%%%%%%%%%%%%%%%%%%%%%%%%%%%%%%%%%%%%%%%%%%%%%%%%%%%%%%%%%%%%%%%%%%%%%%%%
\subsection{Revision History}

%%%%%%%%%%%%%%%%%%%%%%%%%%%%%%%%%%%%%%%%
\paragraph{v2.0:} 2018/12/30

\begin{itemize}
\item
immediate forward processing
\item
added |\childdocby| mechanism
\item
manual restructured
\end{itemize}

%%%%%%%%%%%%%%%%%%%%%%%%%%%%%%%%%%%%%%%%
\paragraph{v1.6:} 2018/01/17

\begin{itemize}
\item
application for development of include files
\item
corrections to manual
\end{itemize}

%%%%%%%%%%%%%%%%%%%%%%%%%%%%%%%%%%%%%%%%
\paragraph{v1.5:} 2017/05/21

\begin{itemize}
\item
more complete structuring introduced
\item
|\childdocof| introduced
\item
|\childdoc| renamed to |\childdocmain|
\item
|\childredirect| renamed to |\childdocforward| and |\childdocforwardprefix|
and functionality expanded
\end{itemize}

%%%%%%%%%%%%%%%%%%%%%%%%%%%%%%%%%%%%%%%%
\paragraph{v1.0:} 2017/04/27

\begin{itemize}
\item
manual and install package
\item
first version published on CTAN
\end{itemize}

%%%%%%%%%%%%%%%%%%%%%%%%%%%%%%%%%%%%%%%%
\paragraph{v0.6:} 2017/04/26

\begin{itemize}
\item
redirection mechanism added
\end{itemize}

%%%%%%%%%%%%%%%%%%%%%%%%%%%%%%%%%%%%%%%%
\paragraph{v0.5:} 2017/04/26

\begin{itemize}
\item
functionality in definition file
\end{itemize}


%%%%%%%%%%%%%%%%%%%%%%%%%%%%%%%%%%%%%%%%%%%%%%%%%%%%%%%%%%%%%%%%%%%%%%%%%%%%%%%%
%%%%%%%%%%%%%%%%%%%%%%%%%%%%%%%%%%%%%%%%%%%%%%%%%%%%%%%%%%%%%%%%%%%%%%%%%%%%%%%%
%%%%%%%%%%%%%%%%%%%%%%%%%%%%%%%%%%%%%%%%%%%%%%%%%%%%%%%%%%%%%%%%%%%%%%%%%%%%%%%%
\appendix

\settowidth\MacroIndent{\rmfamily\scriptsize 000\ }

 \DocInput{childdoc.dtx}

\end{document}
%</driver>
% \fi
%
% %%%%%%%%%%%%%%%%%%%%%%%%%%%%%%%%%%%%%%%%%%%%%%%%%%%%%%%%%%%%%%%%%%%%%%%%%%%%%%
% %%%%%%%%%%%%%%%%%%%%%%%%%%%%%%%%%%%%%%%%%%%%%%%%%%%%%%%%%%%%%%%%%%%%%%%%%%%%%%
% \section{Sample}
%\iffalse
%<*samplemain>
%\fi
%
% The following presents a sample document
% with two chapters, two parts, a title page,
% a compile flag as well as three forwarding files to set the flag.
% It consists of eight |.tex| files:
% \begin{center}
% \begin{tabular}{ll}
% |cdocsamp.tex|&main file\\
% |cdocsch1.tex|&include file for chapter 1\\
% |cdocsch2.tex|&include file for chapter 2\\
% |cdocspt3.tex|&include file for part 3\\
% |cdocspt4.tex|&include file for part 4\\
% |cdocsdrf.tex|&forwarding file for main file in draft mode\\
% |cdocsfi1.tex|&forwarding file for final version of chapter 1\\
% |cdocsfi2.tex|&forwarding file for final version of chapter 2\\
% \end{tabular}
% \end{center}
% Each of the eight files can be compiled directly by the \LaTeX{} compiler.
%
% %%%%%%%%%%%%%%%%%%%%%%%%%%%%%%%%%%%%%%
% \paragraph{Main File.}
%
% The main file is called |cdocsamp.tex|.
%
% Load the \textsf{childdoc} definitions and
% declare the filename for the main document:
%    \begin{macrocode}
\input{childdoc.def}
\childdocmain{}
%    \end{macrocode}

% Optional override for |\version| flag:
%    \begin{macrocode}
%%\ifchilddoc\else\providecommand{\version}{draft}\fi
%    \end{macrocode}

% Define the default values for the |\version| flag
% (|final| for the main file and |draft| for childs):
%    \begin{macrocode}
\ifchilddoc
\providecommand{\version}{draft}
\else
\providecommand{\version}{final}
\fi
%    \end{macrocode}

% Load the standard document class:
%    \begin{macrocode}
\documentclass[12pt]{article}
%    \end{macrocode}

% Start the document body:
%    \begin{macrocode}
\begin{document}
%    \end{macrocode}

% Declare a title page.
% Print title, part of document being processed and version flag:
%    \begin{macrocode}
\addtocounter{page}{-1}
\begin{center}
{\LARGE\bfseries{}childdoc example\par}
\vspace{1cm}
\ifchilddoc
\ifchilddocmanual part\else chapter\fi:
`\childdocname' of `\childdocjob'\par
\else
main document: `\childdocjob'\par
\fi
version: \version\par
\end{center}
\newpage
%    \end{macrocode}

% Manually include selected file,
% otherwise process as usual:
%    \begin{macrocode}
\ifchilddocmanual
\section*{part `\childdocname'}
\input{\childdocname}
\else
%    \end{macrocode}

% Include the two chapters:
%    \begin{macrocode}
\include{cdocsch1}
\include{cdocsch2}
%    \end{macrocode}

% Include the two parts unless only chapters should be displayed:
%    \begin{macrocode}
\ifchilddoc\else
\section{part three}
\input{cdocspt3}
\section{part four}
\input{cdocspt4}
\fi
%    \end{macrocode}

% Process as usual until here:
%    \begin{macrocode}
\fi
%    \end{macrocode}

% End of document body:
%    \begin{macrocode}
\end{document}
%    \end{macrocode}
%\iffalse
%</samplemain>
%\fi
%
% %%%%%%%%%%%%%%%%%%%%%%%%%%%%%%%%%%%%%%
% \paragraph{Chapter Include Files.}
%
% The include files are called |cdocsch1.tex| and |cdocsch2.tex|.
%
%\iffalse
%<*samplechap1|samplechap2>
%\fi

% Optional override for |\version| flag:
%    \begin{macrocode}
%%\providecommand{\version}{final}
%    \end{macrocode}

% Include the main document:
%    \begin{macrocode}
\input{childdoc.def}
\childdocof{cdocsamp}
%    \end{macrocode}

%\iffalse
%</samplechap1|samplechap2>
%\fi
%
%\iffalse
%<*samplechap1>
%\fi
% Some text for chapter 1:
%    \begin{macrocode}
\section{one}
some text in chapter one
%    \end{macrocode}

%\iffalse
%</samplechap1>
%\fi
% Some text for chapter 2:
%\iffalse
%<*samplechap2>
%\fi
%    \begin{macrocode}
\section{two}
more text in chapter two
%    \end{macrocode}

%\iffalse
%</samplechap2>
%\fi
%
% %%%%%%%%%%%%%%%%%%%%%%%%%%%%%%%%%%%%%%
% \paragraph{Part Include Files.}
%
% The include files are called |cdocspt3.tex| and |cdocspt4.tex|.
%
%\iffalse
%<*samplepart3|samplepart4>
%\fi

% Optional override for |\version| flag:
%    \begin{macrocode}
%%\providecommand{\version}{final}
%    \end{macrocode}

% Include the main document:
%    \begin{macrocode}
\input{childdoc.def}
\childdocby{cdocsamp}
%    \end{macrocode}

%\iffalse
%</samplepart3|samplepart4>
%\fi
%
%\iffalse
%<*samplepart3>
%\fi
% Some text for part 3:
%    \begin{macrocode}
some text in part three
%    \end{macrocode}

%\iffalse
%</samplepart3>
%\fi
% Some text for part 4:
%\iffalse
%<*samplepart4>
%\fi
%    \begin{macrocode}
more text in part four
%    \end{macrocode}

%\iffalse
%</samplepart4>
%\fi
%
% %%%%%%%%%%%%%%%%%%%%%%%%%%%%%%%%%%%%%%
% \paragraph{Forwarding for a Complete Draft.}
%
% The following forwarding file |cdocsdrf.tex|
% compiles the main document in draft mode:
%\iffalse
%<*sampledraft>
%\fi
%    \begin{macrocode}
\def\version{draft}
\input{childdoc.def}
\childdocforward{cdocsamp}
%    \end{macrocode}

%\iffalse
%</sampledraft>
%\fi
%
% %%%%%%%%%%%%%%%%%%%%%%%%%%%%%%%%%%%%%%
% \paragraph{Forwarding for Final Version of the Chapters.}
%
% The following forwarding files |cdocsfn1.tex| and |cdocsfn2.tex|
% (with identical content)
% compile the final versions of the child documents
% |cdocsch1.tex| and |cdocsch2.tex|, respectively:
%\iffalse
%<*samplefinal>
%\fi
%    \begin{macrocode}
\def\version{final}
\input{childdoc.def}
\childdocforwardprefix[cdocsamp]{cdocsfn}{cdocsch}
%    \end{macrocode}

%\iffalse
%</samplefinal>
%\fi
%
% %%%%%%%%%%%%%%%%%%%%%%%%%%%%%%%%%%%%%%
% \paragraph{Command Line Processing.}
%
% The following three command lines generate the output files
% |cdocscld|, |cdocscl1| and |cdocscl2|
% which should be identical to
% |cdocsdrf|, |cdocsch1| and |cdocsfn2|, respectively:
% \begin{center}
% \begin{tabular}{l}
% |latex -jobname cdocscld \|\\
% |  "\def\version{draft}\input{childdoc.def}\childdocforward{cdocsamp}"|\\
% |latex -jobname cdocscl1 \|\\
% |  "\input{childdoc.def}\childdocforward[cdocsamp]{cdocsch1}"|\\
% |latex -jobname cdocscl2 \|\\
% |  "\def\version{final}\input{childdoc.def}\childdocforward{cdocsch2}"|
% \end{tabular}
% \end{center}
% Note that the trailing backslash on each first line
% merely continues the input to the second line
% (for convenient cut ant paste).
% Furthermore, the command |latex| can be replaced by any
% of its alternative versions such as |pdflatex|.
%
% %%%%%%%%%%%%%%%%%%%%%%%%%%%%%%%%%%%%%%%%%%%%%%%%%%%%%%%%%%%%%%%%%%%%%%%%%%%%%%
% %%%%%%%%%%%%%%%%%%%%%%%%%%%%%%%%%%%%%%%%%%%%%%%%%%%%%%%%%%%%%%%%%%%%%%%%%%%%%%
% \section{Implementation}
%\iffalse
%<*package>
%\fi
%
% This section describes the definitions file |childdoc.def|.

% The definitions cannot be loaded using |\usepackage| or |\RequirePackage|
% which has a mechanism to prevent loading a style file more than once.
% When loading the definitions by means of |\input|
% multiple instances have to be prevented manually:
%\iffalse
%This code needs to be before the `\ProvidesFile' directive
%which is defined at the beginning of this file.
%Therefore it is also placed there and commented out here.
%</package>
%<*discard>
%\fi
%    \begin{macrocode}
\ifdefined\childdocmain\endinput\fi
%    \end{macrocode}
%\iffalse
%</discard>
%<*package>
%\fi
%
% \macro{\ifchilddoc}
% \macro{\ifchilddocmanual}
% The conditional |\ifchilddoc| tells whether a
% child (true) or main (false) document is being compiled.
% The conditional |\ifchilddocmanual| tells whether
% the |\includeonly| mechanism is used (false) or
% the selection of child files must be performed manually (true).
% The definitions initialise to false:
%    \begin{macrocode}
\newif\ifchilddoc
\newif\ifchilddocmanual
%    \end{macrocode}

% \macro{\childdocname}
% \macro{\childdocjob}
% The macro |\childdocname| stores the name of the main document
% to be compiled. The macro |\childdocjob| stores the name of
% the document on which the \LaTeX{} compiler was originally invoked.
% The content of |\jobname| cannot be compared
% to filenames specified in the source due to different catcodes.
% The following code rescans |\jobname|, stores the result
% in |\childdocname| and saves a copy in |\childdocjob|:
%    \begin{macrocode}
\edef\childdocname{\scantokens\expandafter{\jobname\noexpand}}
\let\childdocjob\childdocname
%    \end{macrocode}

% \macro{\childdocdisable}
% The macro |\childdocdisable| prevents the main file
% from being processed more than once.
% At this stage, the main document command |\childdocmain|
% is assumed to be called once again where it should do nothing.
% Any subsequent call to it should prevent
% a secondary processing of the main document
% It overwrites the forwarding commands
% |\childdocof| and |\childdocforward|
% with empty macros to prevent further inclusions of the main document:
%    \begin{macrocode}
\newcommand{\childdocdisable}
{
  \renewcommand{\childdocmain}[1]{\renewcommand{\childdocmain}[1]{\endinput}}
  \renewcommand{\childdocof}[1]{}
  \renewcommand{\childdocby}[2][]{}
  \renewcommand{\childdocforward}[2][]{}
  \renewcommand{\childdocdisable}{}
}
%    \end{macrocode}

% \macro{\childdocmain}
% The macro |\childdocmain| is to be called at the top of the main file
% with nothing or the main filename (without extension) as argument.
% First, it breaks loops.
% If the argument is not empty and does not match |\childdocname|
% (which is set by the first inclusion of |childdoc.def|),
% |\ifchilddoc| is set to true, |\includeonly| is applied to the child file
% and |\jobname| is set to the main file
% (for proper handling of |.aux| files):
%    \begin{macrocode}
\newcommand{\childdocmain}[1]
{
  \childdocdisable\childdocmain{}
  \if?#1?\else
    \begingroup
      \def\childdoctmp{#1}
      \ifx\childdoctmp\childdocname
        \def\childdoctmp{}
      \else
        \def\childdoctmp
        {
          \childdoctrue
          \includeonly{\childdocname}
          \def\childdocjob{#1}
          \def\jobname{#1}
        }
      \fi
      \expandafter
    \endgroup
    \childdoctmp
  \fi
}
%    \end{macrocode}

% \macro{\childdocof}
% The command |\childdocof| redirects
% compilation to the main file |#1|.
%    \begin{macrocode}
\newcommand{\childdocof}[1]
{
  \childdocdisable
  \childdoctrue
  \includeonly{\childdocname}
  \def\jobname{#1}
  \def\childdocjob{#1}
  \input{#1}
}
%    \end{macrocode}

% \macro{\childdocby}
% The command |\childdocby| ....
%    \begin{macrocode}
\newcommand{\childdocby}[2][]
{
  \childdocdisable
  \childdoctrue
  \childdocmanualtrue
  \if?#1?\else
    \def\jobname{#2}
  \fi
  \def\childdocjob{#2}
  \input{#2}
  \endinput
}
%    \end{macrocode}

% \macro{\childdocforward}
% The command |\childdocforward| redirects
% compilation to the main file or
% (if the optional argument is given) a child file.
% Parameters are set as if the main file
% or a child file starting with |\childdocof| was compiled.
% Then compilation is handed over to the main file:
%    \begin{macrocode}
\newcommand{\childdocforward}[2][]
{
  \begingroup
    \if?#1?
      \def\childdoctmp
      {
        \def\childdocname{#2}
        \def\childdocjob{#2}
        \def\jobname{#2}
        \input{#2}
        \endinput
      }
    \else
      \def\childdoctmp
      {
        \childdocdisable
        \def\childdocname{#2}
        \childdoctrue
        \includeonly{#2}
        \def\childdocjob{#1}
        \def\jobname{#1}
        \input{#1}
        \endinput
      }
    \fi
    \expandafter
  \endgroup
  \childdoctmp
}
%    \end{macrocode}

% \macro{\childdocforwardprefix}
% The command |\childdocforwardprefix| redirects
% compilation to the main or a child file by means of a pattern.
% The prefix |#1| in the current filename is replaced by |#2|
% and the suffix of the current filename is kept
% (it is assumed that the filename does not contain the substring `|~~~|'
% which is used as a delimiter).
% Compilation is handed over to the new file by |\childdocforward|:
%    \begin{macrocode}
\newcommand{\childdocforwardprefix}[3][]
{
  \begingroup
    \def\childdocextract #2##1~~~{\def\childdoctmp{\childdocforward[#1]{#3##1}}}
    \expandafter\childdocextract\childdocname~~~
    \expandafter
  \endgroup
  \childdoctmp
}
%    \end{macrocode}

% \macro{\childdoc}
% The deprecated macro |\childdoc| is a legacy version of |\childdocmain|:
%    \begin{macrocode}
\newcommand{\childdoc}{\childdocmain}
%    \end{macrocode}

% \macro{\childdocredirect}
% The deprecated macro |\childdocredirect| is a legacy version
% of |\childdocforward| and |\childdocforwardprefix|:
%    \begin{macrocode}
\newcommand{\childdocredirect}[2][]
{
  \begingroup
    \if?#1?
      \def\childdoctmp{\childdocforward{#2}}
    \else
      \def\childdoctmp{\childdocforwardprefix{#1}{#2}}
    \fi
    \expandafter
  \endgroup
  \childdoctmp
}
%    \end{macrocode}

%\iffalse
%</package>
%\fi
%
\endinput
\childdocforward[|\textit{main}|]{|\textit{dest}|}"|
\end{center}
%
Here \textit{target} is the name of the output file,
\textit{main} is the name of the main file
and \textit{dest} is the name of the main or child file to be processed
(all filenames without extensions).
The optional argument \textit{main} can be omitted
if \textit{main} matches \textit{dest}.
Optionally, compilation \textit{flags} can be defined via |\def| commands.
This command line makes the \TeX{} engine believe
it is compiling the file \textit{target}
whose content is specified as the latter parameter.
The provided code then forwards the processing to
\textit{main} or \textit{dest} as described in \secref{sec:forward}.

%%%%%%%%%%%%%%%%%%%%%%%%%%%%%%%%%%%%%%%%%%%%%%%%%%%%%%%%%%%%%%%%%%%%%%%%%%%%%%%%
\subsection{Include by Input}
\label{sec:input}

Including child documents by |\include| has some restrictions by design.
Most notably, the content of a child document always occupies
its own set of pages; pages cannot be shared between child documents.
Usually, this behaviour makes perfect sense
because each child document contain an essential part of the document.
However, in some situations it may be desirable to compose
a document from a collection of parts
without having mandatory page breaks between then.
For this case, the package
provides a mechanism to include parts
by |\input| which can also be processed individually.
However, by construction this mechanism
requires manual handling of the content to be output.

%%%%%%%%%%%%%%%%%%%%%%%%%%%%%%%%%%%%%%%%
\DescribeMacro{\ifchilddocmanual}
The main file should be prepared as usual, see \secref{sec:include}.
However, the document body must make a distinction
between processing of an individual part and of the main document, e.g.:
%
\begin{center}
\begin{tabular}{l}
|\ifchilddocmanual|\\
|\input{\childdocname}|\\
|\||else|\\
\textit{document body with }|\input{|\textit{part}|}|\\
|\||fi|
\end{tabular}
\end{center}
%
The conditional |\ifchilddocmanual| is true whenever
a part to be included by |\input| is being compiled,
and the name of the part is stored in |\childdocname|.

%%%%%%%%%%%%%%%%%%%%%%%%%%%%%%%%%%%%%%%%
\DescribeMacro{\childdocby}
Each part to be included by |\input| should start with:
%
\begin{center}
\begin{tabular}{l}
|% \iffalse
%
% childdoc.dtx Copyright (C) 2017-2018 Niklas Beisert
%
% This work may be distributed and/or modified under the
% conditions of the LaTeX Project Public License, either version 1.3
% of this license or (at your option) any later version.
% The latest version of this license is in
%   http://www.latex-project.org/lppl.txt
% and version 1.3 or later is part of all distributions of LaTeX
% version 2005/12/01 or later.
%
% This work has the LPPL maintenance status `maintained'.
%
% The Current Maintainer of this work is Niklas Beisert.
%
% This work consists of the files childdoc.dtx and childdoc.ins
% and the derived files childdoc.def and cdocsamp.tex with
% cdocsch1.tex, cdocsch2.tex, cdocsdrf.tex, cdocsfn1.tex, cdocsfn2.tex.
%
%<package>\ifdefined\childdocmain\endinput\fi
%<package>\ProvidesFile{childdoc.def}[2018/12/30 v2.0 child document driver]
%<samplemain>\ProvidesFile{cdocsamp.tex}[2018/12/30 v2.0 sample for childdoc]
%<*driver>
%\ProvidesFile{childdoc.drv}[2018/12/30 v2.0 childdoc reference manual file]
\PassOptionsToClass{10pt,a4paper}{article}
\documentclass{ltxdoc}

\usepackage[margin=35mm]{geometry}
\usepackage{hyperref}
\usepackage{hyperxmp}
\usepackage[usenames]{color}

\hypersetup{colorlinks=true}
\hypersetup{pdfstartview=FitH}
\hypersetup{pdfpagemode=UseNone}
\hypersetup{pdfsource={}}
\hypersetup{pdflang={en-UK}}
\hypersetup{pdfcopyright={Copyright 2017-2018 Niklas Beisert.
  This work may be distributed and/or modified under the
  conditions of the LaTeX Project Public License, either version 1.3
  of this license or (at your option) any later version.}}
\hypersetup{pdflicenseurl={http://www.latex-project.org/lppl.txt}}
\hypersetup{pdfcontactaddress={ETH Zurich, ITP, HIT K,
  Wolfgang-Pauli-Strasse 27}}
\hypersetup{pdfcontactpostcode={8093}}
\hypersetup{pdfcontactcity={Zurich}}
\hypersetup{pdfcontactcountry={Switzerland}}
\hypersetup{pdfcontactemail={nbeisert@itp.phys.ethz.ch}}
\hypersetup{pdfcontacturl={http://people.phys.ethz.ch/\xmptilde nbeisert/}}

\newcommand{\secref}[1]{\hyperref[#1]{section \ref*{#1}}}

\parskip1ex
\parindent0pt
\let\olditemize\itemize
\def\itemize{\olditemize\parskip0pt}

\begin{document}

\title{The \textsf{childdoc} Package}
\hypersetup{pdftitle={The childdoc Package}}
\author{Niklas Beisert\\[2ex]
  Institut f\"ur Theoretische Physik\\
  Eidgen\"ossische Technische Hochschule Z\"urich\\
  Wolfgang-Pauli-Strasse 27, 8093 Z\"urich, Switzerland\\[1ex]
  \href{mailto:nbeisert@itp.phys.ethz.ch}
  {\texttt{nbeisert@itp.phys.ethz.ch}}}
\hypersetup{pdfauthor={Niklas Beisert}}
\hypersetup{pdfsubject={Manual for the LaTeX2e Package childdoc}}
\date{30 December 2018, \textsf{v2.0}}
\maketitle

\begin{abstract}\noindent
\textsf{childdoc} is a \LaTeXe{} package
that enables the direct compilation
of document sections included by |\include|
to individual files.
\end{abstract}

\begingroup
\parskip0ex
\tableofcontents
\endgroup

%%%%%%%%%%%%%%%%%%%%%%%%%%%%%%%%%%%%%%%%%%%%%%%%%%%%%%%%%%%%%%%%%%%%%%%%%%%%%%%%
%%%%%%%%%%%%%%%%%%%%%%%%%%%%%%%%%%%%%%%%%%%%%%%%%%%%%%%%%%%%%%%%%%%%%%%%%%%%%%%%
\section{Introduction}

\LaTeX{} provides a mechanism to structure a large document (such as a book)
into a main file and several child files (containing the chapters)
using the |\include| command.
This mechanism is beneficial for documents
which span hundreds of pages in order to
make the source file(s) more manageable.
Moreover, compilation can be restricted to
selected child files by means of the |\includeonly| command.
The latter feature can be used to reduce the compilation time while editing
(this was significantly more useful in the earlier days of \LaTeX{})
or to generate a smaller document which is easier to navigate.
Another application of |\includeonly| is to generate
documents consisting of selected parts of the complete document.

However, there are a few drawbacks of the plain |\include| mechanism:
\begin{itemize}
\item
The child files cannot be compiled on their own,
they can only be compiled via the main file.
A naive editing environment
(such as a text editor with an option
to have the current file processed by \LaTeX)
may require one to switch to the main file before compiling;
attempting to compile the child file produces errors.
\item
The main file must be modified (each time)
to adjust the |\includeonly| command
to the present needs. This easily leaves the main file in a messy state.
\item
The generated document will always carry the filename
of the main document. This is inconvenient if
several child files are to be compiled and
to be kept for distribution.
\end{itemize}

The present package provides a simple interface
to make child files individually compilable by \LaTeX{}.
Compiling a child file then has the same effect as compiling
the main file with an |\includeonly| command
to select the appropriate child.
Moreover the generated document will carry the name of the child
rather than the main file.
This resolves all three above issues.

This feature is meant to make the editing of books,
thesis documents and lecture notes somewhat more convenient.
However, the package can also be used efficiently for
composing a series of documents (such as exercise sheets)
which are typically distributed individually.
It then assists the author in generating the individual documents
(potentially in different versions)
as well as a document containing the collected series.
Another application is in developing style files
or other kinds of included material
where compilation of the style file could redirect
to a sample or test file.

%%%%%%%%%%%%%%%%%%%%%%%%%%%%%%%%%%%%%%%%%%%%%%%%%%%%%%%%%%%%%%%%%%%%%%%%%%%%%%%%
%%%%%%%%%%%%%%%%%%%%%%%%%%%%%%%%%%%%%%%%%%%%%%%%%%%%%%%%%%%%%%%%%%%%%%%%%%%%%%%%
\section{Usage}

First of all, the package \textsf{childdoc} is \emph{not} a standard
\LaTeXe{} |.sty| style file! Therefore it needs to be invoked in
a non-standard way.

%%%%%%%%%%%%%%%%%%%%%%%%%%%%%%%%%%%%%%%%%%%%%%%%%%%%%%%%%%%%%%%%%%%%%%%%%%%%%%%%
\subsection{Included Files}
\label{sec:include}

%%%%%%%%%%%%%%%%%%%%%%%%%%%%%%%%%%%%%%%%
\DescribeMacro{\childdocmain}
To use the package, add the commands
\begin{center}
\begin{tabular}{l}
|\input{childdoc.def}|\\
|\childdocmain{}|\\
\end{tabular}
\end{center}
at the very top of the main \LaTeX{} file,
in particular \emph{before} the |\documentclass| statement!
The argument of |\childdocmain| should be left empty
(but it must be present).

%%%%%%%%%%%%%%%%%%%%%%%%%%%%%%%%%%%%%%%%
\DescribeMacro{\childdocof}
Furthermore, add the commands
\begin{center}
\begin{tabular}{l}
|\input{childdoc.def}|\\
|\childdocof{|\textit{main}|}|\\
\end{tabular}
\end{center}
at the top of every child file \textit{child}
which is included by |\include{|\textit{child}|}|
from within the main file
(or at least for those files to be compiled individually).
The argument \textit{main} must be the filename of the main file.

There are a couple of
considerations in setting up the main and child documents:

%%%%%%%%%%%%%%%%%%%%%%%%%%%%%%%%%%%%%%%%
\paragraph{Restrictions.}

Please note the following restrictions:
\begin{itemize}
\item
|\childdocmain| must be called with one argument \textit{main}
to ensure compatibility with earlier version of the package.
It must either be empty (|\childdocmain{}|)
or precisely match the filename of the main file in which it is specified.
See \secref{sec:detection} for further information.
\item
The filename \textit{main} must be specified without the |.tex| extension.
\item
The filename \textit{main} is case sensitive
(even in case-insensitive file systems)
due to internal string comparison.
\item
The argument \textit{main} should be fully expanded, it cannot be a macro.
\item
Subdirectories and special characters should be avoided in filenames.
\item
The command |\childdocmain{|\textit{main}|}| must be followed by a whitespace.
It should not be followed immediately by another command
or by a comment mark `|%|'.
This is because the \TeX{} parser reads the token immediately following
the argument of |\childdocmain| and puts it
at the beginning of every child section;
however, a white\-space is ignored.
\end{itemize}

%%%%%%%%%%%%%%%%%%%%%%%%%%%%%%%%%%%%%%%%
\paragraph{Content of Main File.}

It is advisable to place all content in the child files included by |\include|.
Any output contained in the main file will appear in all child documents
unless suppressed manually;
it cannot be suppressed automatically by the |\includeonly| directive
and thus should normally be avoided.
A method to include some content in the main file
by means of conditional processing is described in \secref{sec:conditional}.

%%%%%%%%%%%%%%%%%%%%%%%%%%%%%%%%%%%%%%%%
\paragraph{Page Numbering.}

When only a part of the document is compiled,
the appropriate numbering of pages
(as well as other status parameters)
is determined from the |.aux| files.
The latter contain information from previous passes.
However this information needs to propagate through
all intermediate child documents.
Therefore the page numbering in child documents may well
be inconsistent until the complete document is compiled at least once.

A useful (if unconventional) way to always ensure a consistent
page numbering is to restart the numbering in each child document
and denote the pages by `\textit{child}|.|\textit{page}'
where \textit{child} represents the chapter/section number of the child file.
This can be achieved by the command
|\numberwithin{page}{|\textit{child}|}|
of the \textsf{amsmath} package
where \textit{child} can be |chapter| or |section|
depending on the chosen structuring.
Alternatively, one can modify the macro |\thepage| appropriately
and reset the counter |page| at the start of each child file.

%%%%%%%%%%%%%%%%%%%%%%%%%%%%%%%%%%%%%%%%%%%%%%%%%%%%%%%%%%%%%%%%%%%%%%%%%%%%%%%%
\subsection{Conditional Processing}
\label{sec:conditional}

The package provides a mechanism to compile different versions
of a document. To customise the versions further some conditional processing
can come in handy to distinguish which version is being compiled.
The package provides two macros to describe the compilation context:

%%%%%%%%%%%%%%%%%%%%%%%%%%%%%%%%%%%%%%%%
\DescribeMacro{\ifchilddoc}
The conditional |\ifchilddoc| distinguishes between the compilation of
child documents and the main document:
%
\begin{center}
|\ifchilddoc |\textit{child-code}| |[|\||else |\textit{main-code}]| \||fi|
\end{center}

%%%%%%%%%%%%%%%%%%%%%%%%%%%%%%%%%%%%%%%%
\DescribeMacro{\childdocname}
\DescribeMacro{\childdocjob}
The macro |\childdocname| contains the filename (without extension)
of the main or child file being processed.
Note that |\childdocjob| will always contain the name of the main file.

%%%%%%%%%%%%%%%%%%%%%%%%%%%%%%%%%%%%%%%%
\paragraph{Title Page.}

Conditional processing can be used to include a title or banner page
in the main document when proper precautions are taken.
Importantly, the code in the main file should ensure that the page counter
(as well as other status parameters which are stored in the |.aux| files)
takes the same value after the conditional processing.
Otherwise the page numbers may take divergent values
depending on which part is compiled.

For example, a title page could be declared by:
%
\begin{center}
\begin{tabular}{l}
|\ifchilddoc\||else|\\
|\addtocounter{page}{-1}|\\
\textit{code for title page}\\
|\newpage|\\
|\||fi|
\end{tabular}
\end{center}
%
A banner page for the child documents can be generated by:
%
\begin{center}
\begin{tabular}{l}
|\ifchilddoc|\\
|\addtocounter{page}{-1}|\\
\textit{code for banner page}\\
|\newpage|\\
|\||fi|
\end{tabular}
\end{center}
%
Here one could write a message such as:
\begin{center}
|This is the part \childdocname{} of \childdocjob{}.|
\end{center}

%%%%%%%%%%%%%%%%%%%%%%%%%%%%%%%%%%%%%%%%%%%%%%%%%%%%%%%%%%%%%%%%%%%%%%%%%%%%%%%%
\subsection{Flags}
\label{sec:flags}

The package makes it easy to generate different versions
of the main or child documents.
To this end compilation flags can be defined
and assigned different default values.
They will be particularly useful in conjunction
with the forwarding mechanism described in \secref{sec:forward}.

For example, it may be useful to have a flag |\version|
which can be set to |draft| or |final|.
The document source will contain some conditional code
depending on the value of |\version|.
Suppose further, the flag should default to |final| for the main file
and to |draft| for child files
which is a natural assignment for editing the document.
This is achieved by placing the following code
in the preamble of the main document
(below the |\childdocmain| directive):
%
\begin{center}
\begin{tabular}{l}
|\ifchilddoc|\\
|\providecommand{\version}{draft}|\\
|\||else|\\
|\providecommand{\version}{final}|\\
|\||fi|
\end{tabular}
\end{center}
%
The definition by |\providecommand| makes sure
that previous definitions are not overwritten.
Further statements |\providecommand{\version}{...}|
can thus be added before the above code to override it.

For the main file, one might add a line
(between |\childdocmain| and the above block)
%
\begin{center}
|%\ifchilddoc\||else\providecommand{\version}{draft}\||fi|
\end{center}
%
which can be uncommented to produce a draft version.
Likewise one can add a line to the very top of a child file
(above the |\childdocof{|\textit{main}|}| directive)
%
\begin{center}
|%\providecommand{\version}{final}|
\end{center}
%
which can be uncommented to produce the final version of this child document.

%%%%%%%%%%%%%%%%%%%%%%%%%%%%%%%%%%%%%%%%%%%%%%%%%%%%%%%%%%%%%%%%%%%%%%%%%%%%%%%%
\subsection{Forwarding}
\label{sec:forward}

Different versions of the main or child documents
using compilation flags as described in \secref{sec:flags}
can be (permanently) stored in different files
for convenient compilation, viewing and distribution.
To this end, the package defines a command
to pass on compilation to a different file:

%%%%%%%%%%%%%%%%%%%%%%%%%%%%%%%%%%%%%%%%
\DescribeMacro{\childdocforward}
The command |\childdocforward| redirects processing to
another source file:
%
\begin{center}
\begin{tabular}{l}
|\input{childdoc.def}|\\
|\childdocforward[|\textit{main}|]{|\textit{dest}|}|\\
\end{tabular}
\end{center}
%
The argument \textit{dest} is the destination file
(without extension).
It should be the main file or one of the child files.
Note that further \textsf{childdoc} directives
such as |\childdocof| and |\childdocforward|
in the indicated file will be processed in this form.
The optional argument \textit{main}
passes on directly to the main file \textit{main}
while pretending to compile the child \textit{dest}.
This form behaves as if \textit{dest}
issues |\childdocof{|\textit{main}|}| right away,
and no further \textsf{childdoc} directives will be processed.

%%%%%%%%%%%%%%%%%%%%%%%%%%%%%%%%%%%%%%%%
\DescribeMacro{\...prefix}
In the alternative form |\childdocforwardprefix|,
%
\begin{center}
\begin{tabular}{l}
|\input{childdoc.def}|\\
|\childdocforwardprefix[|\textit{main}|]{|\textit{prefix}|}{|\textit{dest}|}|
\end{tabular}
\end{center}
%
the destination file is determined by a pattern
depending on the current file:
To make this work, the current file must be called
`{\textit{prefix}\hspace{0.2em}\textit{suffix}}'
with \textit{prefix} matching precisely the argument.
Processing is then passed on to the file
`{\textit{dest}\hspace{0.2em}\textit{suffix}}'.
Surely, the same effect is achieved by
directly specifying the
argument `{\textit{dest}\hspace{0.2em}\textit{suffix}}'
in the first form.
However, that requires to set up a different file
for each child. With the alternative form of the command
all these files can have exactly the same content
which simplifies setting them up and maintaining them.

For example, the following file |draft.tex|
with a compilation flag |\version| as described in \secref{sec:flags}
compiles the main document as a draft:
%
\begin{center}
\begin{tabular}{l}
|\def\version{draft}|\\
|\input{childdoc.def}|\\
|\childdocforward{|\textit{main}|}|
\end{tabular}
\end{center}
%
Likewise, the following files |final|\textit{nn}|.tex|
compile the final version of the child document
|child|\textit{nn}|.tex|:
%
\begin{center}
\begin{tabular}{l}
|\def\version{final}|\\
|\input{childdoc.def}|\\
|\childdocforwardprefix{final}{child}|
\end{tabular}
\end{center}
%

Note that when several versions of a main file and/or of each child file
are to be generated, it may be convenient to set up a |Makefile| or
shell script to automatise the process.

%%%%%%%%%%%%%%%%%%%%%%%%%%%%%%%%%%%%%%%%%%%%%%%%%%%%%%%%%%%%%%%%%%%%%%%%%%%%%%%%
\subsection{Command Line Processing}
\label{sec:commandline}

The effect of redirection files can also be achieved by invoking
the \LaTeX{} compiler with a more elaborate command line.
Most conveniently this should be done as part
of a shell script or a |Makefile|.

When using \textsf{childdoc} in the main file, the following
command lines effectively perform a redirection
(note that depending on the shell being used,
backslashes may have to be doubled: `|\|' $\to$ `|\\|'):
%
\begin{center}
|... -jobname "|\textit{target}|" |\\|"|[\textit{flags}]%
|\input{childdoc.def}\childdocforward[|\textit{main}|]{|\textit{dest}|}"|
\end{center}
%
Here \textit{target} is the name of the output file,
\textit{main} is the name of the main file
and \textit{dest} is the name of the main or child file to be processed
(all filenames without extensions).
The optional argument \textit{main} can be omitted
if \textit{main} matches \textit{dest}.
Optionally, compilation \textit{flags} can be defined via |\def| commands.
This command line makes the \TeX{} engine believe
it is compiling the file \textit{target}
whose content is specified as the latter parameter.
The provided code then forwards the processing to
\textit{main} or \textit{dest} as described in \secref{sec:forward}.

%%%%%%%%%%%%%%%%%%%%%%%%%%%%%%%%%%%%%%%%%%%%%%%%%%%%%%%%%%%%%%%%%%%%%%%%%%%%%%%%
\subsection{Include by Input}
\label{sec:input}

Including child documents by |\include| has some restrictions by design.
Most notably, the content of a child document always occupies
its own set of pages; pages cannot be shared between child documents.
Usually, this behaviour makes perfect sense
because each child document contain an essential part of the document.
However, in some situations it may be desirable to compose
a document from a collection of parts
without having mandatory page breaks between then.
For this case, the package
provides a mechanism to include parts
by |\input| which can also be processed individually.
However, by construction this mechanism
requires manual handling of the content to be output.

%%%%%%%%%%%%%%%%%%%%%%%%%%%%%%%%%%%%%%%%
\DescribeMacro{\ifchilddocmanual}
The main file should be prepared as usual, see \secref{sec:include}.
However, the document body must make a distinction
between processing of an individual part and of the main document, e.g.:
%
\begin{center}
\begin{tabular}{l}
|\ifchilddocmanual|\\
|\input{\childdocname}|\\
|\||else|\\
\textit{document body with }|\input{|\textit{part}|}|\\
|\||fi|
\end{tabular}
\end{center}
%
The conditional |\ifchilddocmanual| is true whenever
a part to be included by |\input| is being compiled,
and the name of the part is stored in |\childdocname|.

%%%%%%%%%%%%%%%%%%%%%%%%%%%%%%%%%%%%%%%%
\DescribeMacro{\childdocby}
Each part to be included by |\input| should start with:
%
\begin{center}
\begin{tabular}{l}
|\input{childdoc.def}|\\
|\childdocby{|\textit{main}|}|\\
\end{tabular}
\end{center}
%
The directive |\childdocby| is similar to |\childdocof|
described in \secref{sec:include},
but the subsequent selection of content must be done manually.
To that end, both |\ifchilddoc| and |\ifchilddocmanual|
will be true upon processing of a part,
and the name of the part is stored in |\childdocname|.
Note that |\jobname| will be set to the filename of the current part
so that each part receives an individual |.aux| file
that does not interfere with the |.aux| file(s) of the main document.
This behaviour can be altered by the alternative form
|\childdocby[*]{|\textit{main}|}| (with a non-empty optional argument)
which uses the |.aux| file of the main document
by setting |\jobname| to \textit{main}.

%%%%%%%%%%%%%%%%%%%%%%%%%%%%%%%%%%%%%%%%%%%%%%%%%%%%%%%%%%%%%%%%%%%%%%%%%%%%%%%%
\subsection{Driver Development}
\label{sec:driver}

The \textsf{childdoc} mechanism can also be use for the development
of definition files such as \LaTeX{} styles or classes.
This case differs from the above setup with multiple parts
included by |\include| in that no |\includeonly| should be invoked.
This can be achieved by starting the include file
(before |\ProvidesPackage|) with:
%
\begin{center}
\begin{tabular}{l}
|\input{childdoc.def}|\\
|\childdocforward{|\textit{main}|}|\\
\end{tabular}
\end{center}
%
or alternatively with:
%
\begin{center}
\begin{tabular}{l}
|\input{childdoc.def}|\\
|\childdocby{|\textit{main}|}|\\
\end{tabular}
\end{center}
%
Both forms have slightly different effects as described above.
The main file is prepared as usual, see \secref{sec:include}.

%%%%%%%%%%%%%%%%%%%%%%%%%%%%%%%%%%%%%%%%%%%%%%%%%%%%%%%%%%%%%%%%%%%%%%%%%%%%%%%%
\subsection{Legacy Detection}
\label{sec:detection}

The directive |\childdocmain| in the main file can detect
whether the complete document or merely a child is to be compiled
even without using the directive |\childdocof|.
This method is deprecated because it is less robust
and there is no compelling reason to use it;
it is merely provided for backward compatibility
and it may be removed in future versions.

If the detection mechanism is to be used,
it is mandatory to correctly specify
the filename of the main file as the argument of |\childdocmain|:
%
\begin{center}
\begin{tabular}{l}
|\input{childdoc.def}|\\
|\childdocmain{|\textit{main}|}|\\
\end{tabular}
\end{center}
%
If |\jobname| does not match the argument \textit{main} of |\childdocmain|,
it is assumed that |\jobname| points to the child file to be compiled.
When using |\childdocmain| with the main file specified as argument,
it suffices to start a child file
with just |\input{|\textit{main}|}|
without loading of the package and using |\childdocof|.
If instead all processing is done
with the appropriate \textsf{childdoc} directives,
the argument of \textit{main} of |\childdocmain| can be empty.

An alternative version of the command line processing described
in \secref{sec:commandline} using the detection mechanism reads:
%
\begin{center}
|... -jobname "|\textit{target}|" "|[\textit{flags}]%
[|\def\jobname{|\textit{dest}|}|]|\input{|\textit{main}|}"|
\end{center}

%%%%%%%%%%%%%%%%%%%%%%%%%%%%%%%%%%%%%%%%%%%%%%%%%%%%%%%%%%%%%%%%%%%%%%%%%%%%%%%%
\subsection{Manual Code}
\label{sec:manual}

In case one cannot be certain whether the definitions file |childdoc.def|
is installed on the target \TeX{} distribution
and one prefers not to ship it,
it is conceivable to paste a few relevant commands into the sources.

To that end, drop all statements |\input{childdoc.def}|
and perform the replacements as outlined below.
Instead of |\childdocmain{|\textit{main}|}| add the following code
to the top of the main file:
%
\begin{center}
\begin{tabular}{l}
|\||ifdefined\childdocname\endinput\||fi\newif\ifchilddoc|\\
|\edef\childdocname{\scantokens\expandafter{\jobname\noexpand}}|\\
|\def\childdocmain{|\textit{main}|}\||ifx\childdocmain\childdocname\||else|\\
|\childdoctrue\includeonly{\childdocname}\let\jobname\childdocmain\||fi|\\
\end{tabular}
\end{center}
%
Instead of |\childdocof{|\textit{main}|}| just include the main file
at the top of each child file:
%
\begin{center}
|\input{|\textit{main}|}|
\end{center}
%
A simple redirection |\childdocforward{|\textit{dest}|}| is achieved by:
%
\begin{center}
|\def\jobname{|\textit{dest}|}\input{\jobname}|
\end{center}
%
The redirection with prefix
|\childdocforwardprefix[|\textit{prefix}|]{|\textit{dest}|}|
is accomplished by:
%
\begin{center}
\begin{tabular}{l}
|{\edef\jobname{\scantokens\expandafter{\jobname\noexpand}}|\\
|\def\redirectjob |\textit{prefix}|#1~~~{\gdef\jobname{|\textit{dest}|#1}}|\\
|\expandafter\redirectjob\jobname~~~}\input{\jobname}|
\end{tabular}
\end{center}

In an alternative approach,
child documents can be compiled by a specific command line
without additional code or specific definitions:
%
\begin{center}
|... -jobname "|\textit{target}|" "|[\textit{flags}]%
|\includeonly{|\textit{dest}|}\input{|\textit{main}|}"|
\end{center}
%

%%%%%%%%%%%%%%%%%%%%%%%%%%%%%%%%%%%%%%%%%%%%%%%%%%%%%%%%%%%%%%%%%%%%%%%%%%%%%%%%
%%%%%%%%%%%%%%%%%%%%%%%%%%%%%%%%%%%%%%%%%%%%%%%%%%%%%%%%%%%%%%%%%%%%%%%%%%%%%%%%
\section{Information}

%%%%%%%%%%%%%%%%%%%%%%%%%%%%%%%%%%%%%%%%%%%%%%%%%%%%%%%%%%%%%%%%%%%%%%%%%%%%%%%%
\subsection{Copyright}

Copyright \copyright{} 2017--2018 Niklas Beisert

This work may be distributed and/or modified under the
conditions of the \LaTeX{} Project Public License, either version 1.3
of this license or (at your option) any later version.
The latest version of this license is in
  \url{http://www.latex-project.org/lppl.txt}
and version 1.3 or later is part of all distributions of \LaTeX{}
version 2005/12/01 or later.

This work has the LPPL maintenance status `maintained'.

The Current Maintainer of this work is Niklas Beisert.

This work consists of the files |README.txt|, |childdoc.ins| and |childdoc.dtx|
as well as the derived files |childdoc.def|, |cdocsamp.tex|
with |cdocsch1.tex|, |cdocsch2.tex|, |cdocspt3.tex|, |cdocspt4.tex|,
|cdocsdrf.tex|, |cdocsfn1.tex|, |cdocsfn2.tex|
as well as |childdoc.pdf|.

%%%%%%%%%%%%%%%%%%%%%%%%%%%%%%%%%%%%%%%%%%%%%%%%%%%%%%%%%%%%%%%%%%%%%%%%%%%%%%%%
\subsection{Files and Installation}

The package consists of the files:
%
\begin{center}
\begin{tabular}{ll}
    |README.txt|   & readme file \\
    |childdoc.ins| & installation file \\
    |childdoc.dtx| & source file \\
    |childdoc.def| & definition file \\
    |cdocsamp.tex| & sample main file \\
    |cdocsch1.tex| & sample include file \\
    |cdocsch2.tex| & sample include file \\
    |cdocspt3.tex| & sample part file \\
    |cdocspt4.tex| & sample part file \\
    |cdocsdrf.tex| & sample redirection file \\
    |cdocsfn1.tex| & sample redirection file \\
    |cdocsfn2.tex| & sample redirection file \\
    |childdoc.pdf| & manual
\end{tabular}
\end{center}
%
The distribution consists of the files
|README.txt|, |childdoc.ins| and |childdoc.dtx|.
%
\begin{itemize}
\item
Run (pdf)\LaTeX{} on |childdoc.dtx|
to compile the manual |childdoc.pdf| (this file).
\item
Run \LaTeX{} on |childdoc.ins| to create the definitions file |childdoc.def|
and the sample |cdocsamp.tex| with include files
|cdocsch1.tex|, |cdocsch2.tex|, |cdocspt3.tex|, |cdocspt4.tex|,
|cdocsdrf.tex|, |cdocsfn1.tex|, |cdocsfn2.tex|.
Then copy the file |childdoc.def| to an appropriate directory of your \LaTeX{}
distribution, e.g.\ \textit{texmf-root}|/tex/latex/childdoc|.
\end{itemize}

%%%%%%%%%%%%%%%%%%%%%%%%%%%%%%%%%%%%%%%%%%%%%%%%%%%%%%%%%%%%%%%%%%%%%%%%%%%%%%%%
\subsection{Related CTAN Packages}

There are several other packages which offer a similar functionality:
%
\begin{itemize}
\item
The packages
\href{http://ctan.org/pkg/docmute}{\textsf{docmute}},
\href{http://ctan.org/pkg/includex}{\textsf{includex}} and
\href{http://ctan.org/pkg/standalone}{\textsf{standalone}}
provide commands to include only the document body of
a child file thus allowing both files to be compiled individually.
\item
The packages \href{http://ctan.org/pkg/subdocs}{\textsf{subdocs}}
and \href{http://ctan.org/pkg/subfiles}{\textsf{subfiles}}
provide structures in which the main and child documents can be
encapsulated and allowing them to be compiled individually.
The inclusion mechanism is different from the conventional |\include|.
\item
The package \href{http://ctan.org/pkg/combine}{\textsf{combine}}
is an elaborate solution to combine several documents into one.
\end{itemize}
%
See also the CTAN topic \href{http://ctan.org/topic/subdocs}{\textsf{subdocs}}
for further related packages.
The present package differs from the above solutions in that
a document structure constructed with the conventional |\include| mechanism
just needs two extra commands at the top of every file
such that all constituent files can be compiled individually.

%%%%%%%%%%%%%%%%%%%%%%%%%%%%%%%%%%%%%%%%%%%%%%%%%%%%%%%%%%%%%%%%%%%%%%%%%%%%%%%%
%\subsection{Feature Suggestions}
%
%The following is a list of features which may be useful for future
%versions of this package:
%%
%\begin{itemize}
%\item
%\ldots
%\end{itemize}

%%%%%%%%%%%%%%%%%%%%%%%%%%%%%%%%%%%%%%%%%%%%%%%%%%%%%%%%%%%%%%%%%%%%%%%%%%%%%%%%
\subsection{Revision History}

%%%%%%%%%%%%%%%%%%%%%%%%%%%%%%%%%%%%%%%%
\paragraph{v2.0:} 2018/12/30

\begin{itemize}
\item
immediate forward processing
\item
added |\childdocby| mechanism
\item
manual restructured
\end{itemize}

%%%%%%%%%%%%%%%%%%%%%%%%%%%%%%%%%%%%%%%%
\paragraph{v1.6:} 2018/01/17

\begin{itemize}
\item
application for development of include files
\item
corrections to manual
\end{itemize}

%%%%%%%%%%%%%%%%%%%%%%%%%%%%%%%%%%%%%%%%
\paragraph{v1.5:} 2017/05/21

\begin{itemize}
\item
more complete structuring introduced
\item
|\childdocof| introduced
\item
|\childdoc| renamed to |\childdocmain|
\item
|\childredirect| renamed to |\childdocforward| and |\childdocforwardprefix|
and functionality expanded
\end{itemize}

%%%%%%%%%%%%%%%%%%%%%%%%%%%%%%%%%%%%%%%%
\paragraph{v1.0:} 2017/04/27

\begin{itemize}
\item
manual and install package
\item
first version published on CTAN
\end{itemize}

%%%%%%%%%%%%%%%%%%%%%%%%%%%%%%%%%%%%%%%%
\paragraph{v0.6:} 2017/04/26

\begin{itemize}
\item
redirection mechanism added
\end{itemize}

%%%%%%%%%%%%%%%%%%%%%%%%%%%%%%%%%%%%%%%%
\paragraph{v0.5:} 2017/04/26

\begin{itemize}
\item
functionality in definition file
\end{itemize}


%%%%%%%%%%%%%%%%%%%%%%%%%%%%%%%%%%%%%%%%%%%%%%%%%%%%%%%%%%%%%%%%%%%%%%%%%%%%%%%%
%%%%%%%%%%%%%%%%%%%%%%%%%%%%%%%%%%%%%%%%%%%%%%%%%%%%%%%%%%%%%%%%%%%%%%%%%%%%%%%%
%%%%%%%%%%%%%%%%%%%%%%%%%%%%%%%%%%%%%%%%%%%%%%%%%%%%%%%%%%%%%%%%%%%%%%%%%%%%%%%%
\appendix

\settowidth\MacroIndent{\rmfamily\scriptsize 000\ }

 \DocInput{childdoc.dtx}

\end{document}
%</driver>
% \fi
%
% %%%%%%%%%%%%%%%%%%%%%%%%%%%%%%%%%%%%%%%%%%%%%%%%%%%%%%%%%%%%%%%%%%%%%%%%%%%%%%
% %%%%%%%%%%%%%%%%%%%%%%%%%%%%%%%%%%%%%%%%%%%%%%%%%%%%%%%%%%%%%%%%%%%%%%%%%%%%%%
% \section{Sample}
%\iffalse
%<*samplemain>
%\fi
%
% The following presents a sample document
% with two chapters, two parts, a title page,
% a compile flag as well as three forwarding files to set the flag.
% It consists of eight |.tex| files:
% \begin{center}
% \begin{tabular}{ll}
% |cdocsamp.tex|&main file\\
% |cdocsch1.tex|&include file for chapter 1\\
% |cdocsch2.tex|&include file for chapter 2\\
% |cdocspt3.tex|&include file for part 3\\
% |cdocspt4.tex|&include file for part 4\\
% |cdocsdrf.tex|&forwarding file for main file in draft mode\\
% |cdocsfi1.tex|&forwarding file for final version of chapter 1\\
% |cdocsfi2.tex|&forwarding file for final version of chapter 2\\
% \end{tabular}
% \end{center}
% Each of the eight files can be compiled directly by the \LaTeX{} compiler.
%
% %%%%%%%%%%%%%%%%%%%%%%%%%%%%%%%%%%%%%%
% \paragraph{Main File.}
%
% The main file is called |cdocsamp.tex|.
%
% Load the \textsf{childdoc} definitions and
% declare the filename for the main document:
%    \begin{macrocode}
\input{childdoc.def}
\childdocmain{}
%    \end{macrocode}

% Optional override for |\version| flag:
%    \begin{macrocode}
%%\ifchilddoc\else\providecommand{\version}{draft}\fi
%    \end{macrocode}

% Define the default values for the |\version| flag
% (|final| for the main file and |draft| for childs):
%    \begin{macrocode}
\ifchilddoc
\providecommand{\version}{draft}
\else
\providecommand{\version}{final}
\fi
%    \end{macrocode}

% Load the standard document class:
%    \begin{macrocode}
\documentclass[12pt]{article}
%    \end{macrocode}

% Start the document body:
%    \begin{macrocode}
\begin{document}
%    \end{macrocode}

% Declare a title page.
% Print title, part of document being processed and version flag:
%    \begin{macrocode}
\addtocounter{page}{-1}
\begin{center}
{\LARGE\bfseries{}childdoc example\par}
\vspace{1cm}
\ifchilddoc
\ifchilddocmanual part\else chapter\fi:
`\childdocname' of `\childdocjob'\par
\else
main document: `\childdocjob'\par
\fi
version: \version\par
\end{center}
\newpage
%    \end{macrocode}

% Manually include selected file,
% otherwise process as usual:
%    \begin{macrocode}
\ifchilddocmanual
\section*{part `\childdocname'}
\input{\childdocname}
\else
%    \end{macrocode}

% Include the two chapters:
%    \begin{macrocode}
\include{cdocsch1}
\include{cdocsch2}
%    \end{macrocode}

% Include the two parts unless only chapters should be displayed:
%    \begin{macrocode}
\ifchilddoc\else
\section{part three}
\input{cdocspt3}
\section{part four}
\input{cdocspt4}
\fi
%    \end{macrocode}

% Process as usual until here:
%    \begin{macrocode}
\fi
%    \end{macrocode}

% End of document body:
%    \begin{macrocode}
\end{document}
%    \end{macrocode}
%\iffalse
%</samplemain>
%\fi
%
% %%%%%%%%%%%%%%%%%%%%%%%%%%%%%%%%%%%%%%
% \paragraph{Chapter Include Files.}
%
% The include files are called |cdocsch1.tex| and |cdocsch2.tex|.
%
%\iffalse
%<*samplechap1|samplechap2>
%\fi

% Optional override for |\version| flag:
%    \begin{macrocode}
%%\providecommand{\version}{final}
%    \end{macrocode}

% Include the main document:
%    \begin{macrocode}
\input{childdoc.def}
\childdocof{cdocsamp}
%    \end{macrocode}

%\iffalse
%</samplechap1|samplechap2>
%\fi
%
%\iffalse
%<*samplechap1>
%\fi
% Some text for chapter 1:
%    \begin{macrocode}
\section{one}
some text in chapter one
%    \end{macrocode}

%\iffalse
%</samplechap1>
%\fi
% Some text for chapter 2:
%\iffalse
%<*samplechap2>
%\fi
%    \begin{macrocode}
\section{two}
more text in chapter two
%    \end{macrocode}

%\iffalse
%</samplechap2>
%\fi
%
% %%%%%%%%%%%%%%%%%%%%%%%%%%%%%%%%%%%%%%
% \paragraph{Part Include Files.}
%
% The include files are called |cdocspt3.tex| and |cdocspt4.tex|.
%
%\iffalse
%<*samplepart3|samplepart4>
%\fi

% Optional override for |\version| flag:
%    \begin{macrocode}
%%\providecommand{\version}{final}
%    \end{macrocode}

% Include the main document:
%    \begin{macrocode}
\input{childdoc.def}
\childdocby{cdocsamp}
%    \end{macrocode}

%\iffalse
%</samplepart3|samplepart4>
%\fi
%
%\iffalse
%<*samplepart3>
%\fi
% Some text for part 3:
%    \begin{macrocode}
some text in part three
%    \end{macrocode}

%\iffalse
%</samplepart3>
%\fi
% Some text for part 4:
%\iffalse
%<*samplepart4>
%\fi
%    \begin{macrocode}
more text in part four
%    \end{macrocode}

%\iffalse
%</samplepart4>
%\fi
%
% %%%%%%%%%%%%%%%%%%%%%%%%%%%%%%%%%%%%%%
% \paragraph{Forwarding for a Complete Draft.}
%
% The following forwarding file |cdocsdrf.tex|
% compiles the main document in draft mode:
%\iffalse
%<*sampledraft>
%\fi
%    \begin{macrocode}
\def\version{draft}
\input{childdoc.def}
\childdocforward{cdocsamp}
%    \end{macrocode}

%\iffalse
%</sampledraft>
%\fi
%
% %%%%%%%%%%%%%%%%%%%%%%%%%%%%%%%%%%%%%%
% \paragraph{Forwarding for Final Version of the Chapters.}
%
% The following forwarding files |cdocsfn1.tex| and |cdocsfn2.tex|
% (with identical content)
% compile the final versions of the child documents
% |cdocsch1.tex| and |cdocsch2.tex|, respectively:
%\iffalse
%<*samplefinal>
%\fi
%    \begin{macrocode}
\def\version{final}
\input{childdoc.def}
\childdocforwardprefix[cdocsamp]{cdocsfn}{cdocsch}
%    \end{macrocode}

%\iffalse
%</samplefinal>
%\fi
%
% %%%%%%%%%%%%%%%%%%%%%%%%%%%%%%%%%%%%%%
% \paragraph{Command Line Processing.}
%
% The following three command lines generate the output files
% |cdocscld|, |cdocscl1| and |cdocscl2|
% which should be identical to
% |cdocsdrf|, |cdocsch1| and |cdocsfn2|, respectively:
% \begin{center}
% \begin{tabular}{l}
% |latex -jobname cdocscld \|\\
% |  "\def\version{draft}\input{childdoc.def}\childdocforward{cdocsamp}"|\\
% |latex -jobname cdocscl1 \|\\
% |  "\input{childdoc.def}\childdocforward[cdocsamp]{cdocsch1}"|\\
% |latex -jobname cdocscl2 \|\\
% |  "\def\version{final}\input{childdoc.def}\childdocforward{cdocsch2}"|
% \end{tabular}
% \end{center}
% Note that the trailing backslash on each first line
% merely continues the input to the second line
% (for convenient cut ant paste).
% Furthermore, the command |latex| can be replaced by any
% of its alternative versions such as |pdflatex|.
%
% %%%%%%%%%%%%%%%%%%%%%%%%%%%%%%%%%%%%%%%%%%%%%%%%%%%%%%%%%%%%%%%%%%%%%%%%%%%%%%
% %%%%%%%%%%%%%%%%%%%%%%%%%%%%%%%%%%%%%%%%%%%%%%%%%%%%%%%%%%%%%%%%%%%%%%%%%%%%%%
% \section{Implementation}
%\iffalse
%<*package>
%\fi
%
% This section describes the definitions file |childdoc.def|.

% The definitions cannot be loaded using |\usepackage| or |\RequirePackage|
% which has a mechanism to prevent loading a style file more than once.
% When loading the definitions by means of |\input|
% multiple instances have to be prevented manually:
%\iffalse
%This code needs to be before the `\ProvidesFile' directive
%which is defined at the beginning of this file.
%Therefore it is also placed there and commented out here.
%</package>
%<*discard>
%\fi
%    \begin{macrocode}
\ifdefined\childdocmain\endinput\fi
%    \end{macrocode}
%\iffalse
%</discard>
%<*package>
%\fi
%
% \macro{\ifchilddoc}
% \macro{\ifchilddocmanual}
% The conditional |\ifchilddoc| tells whether a
% child (true) or main (false) document is being compiled.
% The conditional |\ifchilddocmanual| tells whether
% the |\includeonly| mechanism is used (false) or
% the selection of child files must be performed manually (true).
% The definitions initialise to false:
%    \begin{macrocode}
\newif\ifchilddoc
\newif\ifchilddocmanual
%    \end{macrocode}

% \macro{\childdocname}
% \macro{\childdocjob}
% The macro |\childdocname| stores the name of the main document
% to be compiled. The macro |\childdocjob| stores the name of
% the document on which the \LaTeX{} compiler was originally invoked.
% The content of |\jobname| cannot be compared
% to filenames specified in the source due to different catcodes.
% The following code rescans |\jobname|, stores the result
% in |\childdocname| and saves a copy in |\childdocjob|:
%    \begin{macrocode}
\edef\childdocname{\scantokens\expandafter{\jobname\noexpand}}
\let\childdocjob\childdocname
%    \end{macrocode}

% \macro{\childdocdisable}
% The macro |\childdocdisable| prevents the main file
% from being processed more than once.
% At this stage, the main document command |\childdocmain|
% is assumed to be called once again where it should do nothing.
% Any subsequent call to it should prevent
% a secondary processing of the main document
% It overwrites the forwarding commands
% |\childdocof| and |\childdocforward|
% with empty macros to prevent further inclusions of the main document:
%    \begin{macrocode}
\newcommand{\childdocdisable}
{
  \renewcommand{\childdocmain}[1]{\renewcommand{\childdocmain}[1]{\endinput}}
  \renewcommand{\childdocof}[1]{}
  \renewcommand{\childdocby}[2][]{}
  \renewcommand{\childdocforward}[2][]{}
  \renewcommand{\childdocdisable}{}
}
%    \end{macrocode}

% \macro{\childdocmain}
% The macro |\childdocmain| is to be called at the top of the main file
% with nothing or the main filename (without extension) as argument.
% First, it breaks loops.
% If the argument is not empty and does not match |\childdocname|
% (which is set by the first inclusion of |childdoc.def|),
% |\ifchilddoc| is set to true, |\includeonly| is applied to the child file
% and |\jobname| is set to the main file
% (for proper handling of |.aux| files):
%    \begin{macrocode}
\newcommand{\childdocmain}[1]
{
  \childdocdisable\childdocmain{}
  \if?#1?\else
    \begingroup
      \def\childdoctmp{#1}
      \ifx\childdoctmp\childdocname
        \def\childdoctmp{}
      \else
        \def\childdoctmp
        {
          \childdoctrue
          \includeonly{\childdocname}
          \def\childdocjob{#1}
          \def\jobname{#1}
        }
      \fi
      \expandafter
    \endgroup
    \childdoctmp
  \fi
}
%    \end{macrocode}

% \macro{\childdocof}
% The command |\childdocof| redirects
% compilation to the main file |#1|.
%    \begin{macrocode}
\newcommand{\childdocof}[1]
{
  \childdocdisable
  \childdoctrue
  \includeonly{\childdocname}
  \def\jobname{#1}
  \def\childdocjob{#1}
  \input{#1}
}
%    \end{macrocode}

% \macro{\childdocby}
% The command |\childdocby| ....
%    \begin{macrocode}
\newcommand{\childdocby}[2][]
{
  \childdocdisable
  \childdoctrue
  \childdocmanualtrue
  \if?#1?\else
    \def\jobname{#2}
  \fi
  \def\childdocjob{#2}
  \input{#2}
  \endinput
}
%    \end{macrocode}

% \macro{\childdocforward}
% The command |\childdocforward| redirects
% compilation to the main file or
% (if the optional argument is given) a child file.
% Parameters are set as if the main file
% or a child file starting with |\childdocof| was compiled.
% Then compilation is handed over to the main file:
%    \begin{macrocode}
\newcommand{\childdocforward}[2][]
{
  \begingroup
    \if?#1?
      \def\childdoctmp
      {
        \def\childdocname{#2}
        \def\childdocjob{#2}
        \def\jobname{#2}
        \input{#2}
        \endinput
      }
    \else
      \def\childdoctmp
      {
        \childdocdisable
        \def\childdocname{#2}
        \childdoctrue
        \includeonly{#2}
        \def\childdocjob{#1}
        \def\jobname{#1}
        \input{#1}
        \endinput
      }
    \fi
    \expandafter
  \endgroup
  \childdoctmp
}
%    \end{macrocode}

% \macro{\childdocforwardprefix}
% The command |\childdocforwardprefix| redirects
% compilation to the main or a child file by means of a pattern.
% The prefix |#1| in the current filename is replaced by |#2|
% and the suffix of the current filename is kept
% (it is assumed that the filename does not contain the substring `|~~~|'
% which is used as a delimiter).
% Compilation is handed over to the new file by |\childdocforward|:
%    \begin{macrocode}
\newcommand{\childdocforwardprefix}[3][]
{
  \begingroup
    \def\childdocextract #2##1~~~{\def\childdoctmp{\childdocforward[#1]{#3##1}}}
    \expandafter\childdocextract\childdocname~~~
    \expandafter
  \endgroup
  \childdoctmp
}
%    \end{macrocode}

% \macro{\childdoc}
% The deprecated macro |\childdoc| is a legacy version of |\childdocmain|:
%    \begin{macrocode}
\newcommand{\childdoc}{\childdocmain}
%    \end{macrocode}

% \macro{\childdocredirect}
% The deprecated macro |\childdocredirect| is a legacy version
% of |\childdocforward| and |\childdocforwardprefix|:
%    \begin{macrocode}
\newcommand{\childdocredirect}[2][]
{
  \begingroup
    \if?#1?
      \def\childdoctmp{\childdocforward{#2}}
    \else
      \def\childdoctmp{\childdocforwardprefix{#1}{#2}}
    \fi
    \expandafter
  \endgroup
  \childdoctmp
}
%    \end{macrocode}

%\iffalse
%</package>
%\fi
%
\endinput
|\\
|\childdocby{|\textit{main}|}|\\
\end{tabular}
\end{center}
%
The directive |\childdocby| is similar to |\childdocof|
described in \secref{sec:include},
but the subsequent selection of content must be done manually.
To that end, both |\ifchilddoc| and |\ifchilddocmanual|
will be true upon processing of a part,
and the name of the part is stored in |\childdocname|.
Note that |\jobname| will be set to the filename of the current part
so that each part receives an individual |.aux| file
that does not interfere with the |.aux| file(s) of the main document.
This behaviour can be altered by the alternative form
|\childdocby[*]{|\textit{main}|}| (with a non-empty optional argument)
which uses the |.aux| file of the main document
by setting |\jobname| to \textit{main}.

%%%%%%%%%%%%%%%%%%%%%%%%%%%%%%%%%%%%%%%%%%%%%%%%%%%%%%%%%%%%%%%%%%%%%%%%%%%%%%%%
\subsection{Driver Development}
\label{sec:driver}

The \textsf{childdoc} mechanism can also be use for the development
of definition files such as \LaTeX{} styles or classes.
This case differs from the above setup with multiple parts
included by |\include| in that no |\includeonly| should be invoked.
This can be achieved by starting the include file
(before |\ProvidesPackage|) with:
%
\begin{center}
\begin{tabular}{l}
|% \iffalse
%
% childdoc.dtx Copyright (C) 2017-2018 Niklas Beisert
%
% This work may be distributed and/or modified under the
% conditions of the LaTeX Project Public License, either version 1.3
% of this license or (at your option) any later version.
% The latest version of this license is in
%   http://www.latex-project.org/lppl.txt
% and version 1.3 or later is part of all distributions of LaTeX
% version 2005/12/01 or later.
%
% This work has the LPPL maintenance status `maintained'.
%
% The Current Maintainer of this work is Niklas Beisert.
%
% This work consists of the files childdoc.dtx and childdoc.ins
% and the derived files childdoc.def and cdocsamp.tex with
% cdocsch1.tex, cdocsch2.tex, cdocsdrf.tex, cdocsfn1.tex, cdocsfn2.tex.
%
%<package>\ifdefined\childdocmain\endinput\fi
%<package>\ProvidesFile{childdoc.def}[2018/12/30 v2.0 child document driver]
%<samplemain>\ProvidesFile{cdocsamp.tex}[2018/12/30 v2.0 sample for childdoc]
%<*driver>
%\ProvidesFile{childdoc.drv}[2018/12/30 v2.0 childdoc reference manual file]
\PassOptionsToClass{10pt,a4paper}{article}
\documentclass{ltxdoc}

\usepackage[margin=35mm]{geometry}
\usepackage{hyperref}
\usepackage{hyperxmp}
\usepackage[usenames]{color}

\hypersetup{colorlinks=true}
\hypersetup{pdfstartview=FitH}
\hypersetup{pdfpagemode=UseNone}
\hypersetup{pdfsource={}}
\hypersetup{pdflang={en-UK}}
\hypersetup{pdfcopyright={Copyright 2017-2018 Niklas Beisert.
  This work may be distributed and/or modified under the
  conditions of the LaTeX Project Public License, either version 1.3
  of this license or (at your option) any later version.}}
\hypersetup{pdflicenseurl={http://www.latex-project.org/lppl.txt}}
\hypersetup{pdfcontactaddress={ETH Zurich, ITP, HIT K,
  Wolfgang-Pauli-Strasse 27}}
\hypersetup{pdfcontactpostcode={8093}}
\hypersetup{pdfcontactcity={Zurich}}
\hypersetup{pdfcontactcountry={Switzerland}}
\hypersetup{pdfcontactemail={nbeisert@itp.phys.ethz.ch}}
\hypersetup{pdfcontacturl={http://people.phys.ethz.ch/\xmptilde nbeisert/}}

\newcommand{\secref}[1]{\hyperref[#1]{section \ref*{#1}}}

\parskip1ex
\parindent0pt
\let\olditemize\itemize
\def\itemize{\olditemize\parskip0pt}

\begin{document}

\title{The \textsf{childdoc} Package}
\hypersetup{pdftitle={The childdoc Package}}
\author{Niklas Beisert\\[2ex]
  Institut f\"ur Theoretische Physik\\
  Eidgen\"ossische Technische Hochschule Z\"urich\\
  Wolfgang-Pauli-Strasse 27, 8093 Z\"urich, Switzerland\\[1ex]
  \href{mailto:nbeisert@itp.phys.ethz.ch}
  {\texttt{nbeisert@itp.phys.ethz.ch}}}
\hypersetup{pdfauthor={Niklas Beisert}}
\hypersetup{pdfsubject={Manual for the LaTeX2e Package childdoc}}
\date{30 December 2018, \textsf{v2.0}}
\maketitle

\begin{abstract}\noindent
\textsf{childdoc} is a \LaTeXe{} package
that enables the direct compilation
of document sections included by |\include|
to individual files.
\end{abstract}

\begingroup
\parskip0ex
\tableofcontents
\endgroup

%%%%%%%%%%%%%%%%%%%%%%%%%%%%%%%%%%%%%%%%%%%%%%%%%%%%%%%%%%%%%%%%%%%%%%%%%%%%%%%%
%%%%%%%%%%%%%%%%%%%%%%%%%%%%%%%%%%%%%%%%%%%%%%%%%%%%%%%%%%%%%%%%%%%%%%%%%%%%%%%%
\section{Introduction}

\LaTeX{} provides a mechanism to structure a large document (such as a book)
into a main file and several child files (containing the chapters)
using the |\include| command.
This mechanism is beneficial for documents
which span hundreds of pages in order to
make the source file(s) more manageable.
Moreover, compilation can be restricted to
selected child files by means of the |\includeonly| command.
The latter feature can be used to reduce the compilation time while editing
(this was significantly more useful in the earlier days of \LaTeX{})
or to generate a smaller document which is easier to navigate.
Another application of |\includeonly| is to generate
documents consisting of selected parts of the complete document.

However, there are a few drawbacks of the plain |\include| mechanism:
\begin{itemize}
\item
The child files cannot be compiled on their own,
they can only be compiled via the main file.
A naive editing environment
(such as a text editor with an option
to have the current file processed by \LaTeX)
may require one to switch to the main file before compiling;
attempting to compile the child file produces errors.
\item
The main file must be modified (each time)
to adjust the |\includeonly| command
to the present needs. This easily leaves the main file in a messy state.
\item
The generated document will always carry the filename
of the main document. This is inconvenient if
several child files are to be compiled and
to be kept for distribution.
\end{itemize}

The present package provides a simple interface
to make child files individually compilable by \LaTeX{}.
Compiling a child file then has the same effect as compiling
the main file with an |\includeonly| command
to select the appropriate child.
Moreover the generated document will carry the name of the child
rather than the main file.
This resolves all three above issues.

This feature is meant to make the editing of books,
thesis documents and lecture notes somewhat more convenient.
However, the package can also be used efficiently for
composing a series of documents (such as exercise sheets)
which are typically distributed individually.
It then assists the author in generating the individual documents
(potentially in different versions)
as well as a document containing the collected series.
Another application is in developing style files
or other kinds of included material
where compilation of the style file could redirect
to a sample or test file.

%%%%%%%%%%%%%%%%%%%%%%%%%%%%%%%%%%%%%%%%%%%%%%%%%%%%%%%%%%%%%%%%%%%%%%%%%%%%%%%%
%%%%%%%%%%%%%%%%%%%%%%%%%%%%%%%%%%%%%%%%%%%%%%%%%%%%%%%%%%%%%%%%%%%%%%%%%%%%%%%%
\section{Usage}

First of all, the package \textsf{childdoc} is \emph{not} a standard
\LaTeXe{} |.sty| style file! Therefore it needs to be invoked in
a non-standard way.

%%%%%%%%%%%%%%%%%%%%%%%%%%%%%%%%%%%%%%%%%%%%%%%%%%%%%%%%%%%%%%%%%%%%%%%%%%%%%%%%
\subsection{Included Files}
\label{sec:include}

%%%%%%%%%%%%%%%%%%%%%%%%%%%%%%%%%%%%%%%%
\DescribeMacro{\childdocmain}
To use the package, add the commands
\begin{center}
\begin{tabular}{l}
|\input{childdoc.def}|\\
|\childdocmain{}|\\
\end{tabular}
\end{center}
at the very top of the main \LaTeX{} file,
in particular \emph{before} the |\documentclass| statement!
The argument of |\childdocmain| should be left empty
(but it must be present).

%%%%%%%%%%%%%%%%%%%%%%%%%%%%%%%%%%%%%%%%
\DescribeMacro{\childdocof}
Furthermore, add the commands
\begin{center}
\begin{tabular}{l}
|\input{childdoc.def}|\\
|\childdocof{|\textit{main}|}|\\
\end{tabular}
\end{center}
at the top of every child file \textit{child}
which is included by |\include{|\textit{child}|}|
from within the main file
(or at least for those files to be compiled individually).
The argument \textit{main} must be the filename of the main file.

There are a couple of
considerations in setting up the main and child documents:

%%%%%%%%%%%%%%%%%%%%%%%%%%%%%%%%%%%%%%%%
\paragraph{Restrictions.}

Please note the following restrictions:
\begin{itemize}
\item
|\childdocmain| must be called with one argument \textit{main}
to ensure compatibility with earlier version of the package.
It must either be empty (|\childdocmain{}|)
or precisely match the filename of the main file in which it is specified.
See \secref{sec:detection} for further information.
\item
The filename \textit{main} must be specified without the |.tex| extension.
\item
The filename \textit{main} is case sensitive
(even in case-insensitive file systems)
due to internal string comparison.
\item
The argument \textit{main} should be fully expanded, it cannot be a macro.
\item
Subdirectories and special characters should be avoided in filenames.
\item
The command |\childdocmain{|\textit{main}|}| must be followed by a whitespace.
It should not be followed immediately by another command
or by a comment mark `|%|'.
This is because the \TeX{} parser reads the token immediately following
the argument of |\childdocmain| and puts it
at the beginning of every child section;
however, a white\-space is ignored.
\end{itemize}

%%%%%%%%%%%%%%%%%%%%%%%%%%%%%%%%%%%%%%%%
\paragraph{Content of Main File.}

It is advisable to place all content in the child files included by |\include|.
Any output contained in the main file will appear in all child documents
unless suppressed manually;
it cannot be suppressed automatically by the |\includeonly| directive
and thus should normally be avoided.
A method to include some content in the main file
by means of conditional processing is described in \secref{sec:conditional}.

%%%%%%%%%%%%%%%%%%%%%%%%%%%%%%%%%%%%%%%%
\paragraph{Page Numbering.}

When only a part of the document is compiled,
the appropriate numbering of pages
(as well as other status parameters)
is determined from the |.aux| files.
The latter contain information from previous passes.
However this information needs to propagate through
all intermediate child documents.
Therefore the page numbering in child documents may well
be inconsistent until the complete document is compiled at least once.

A useful (if unconventional) way to always ensure a consistent
page numbering is to restart the numbering in each child document
and denote the pages by `\textit{child}|.|\textit{page}'
where \textit{child} represents the chapter/section number of the child file.
This can be achieved by the command
|\numberwithin{page}{|\textit{child}|}|
of the \textsf{amsmath} package
where \textit{child} can be |chapter| or |section|
depending on the chosen structuring.
Alternatively, one can modify the macro |\thepage| appropriately
and reset the counter |page| at the start of each child file.

%%%%%%%%%%%%%%%%%%%%%%%%%%%%%%%%%%%%%%%%%%%%%%%%%%%%%%%%%%%%%%%%%%%%%%%%%%%%%%%%
\subsection{Conditional Processing}
\label{sec:conditional}

The package provides a mechanism to compile different versions
of a document. To customise the versions further some conditional processing
can come in handy to distinguish which version is being compiled.
The package provides two macros to describe the compilation context:

%%%%%%%%%%%%%%%%%%%%%%%%%%%%%%%%%%%%%%%%
\DescribeMacro{\ifchilddoc}
The conditional |\ifchilddoc| distinguishes between the compilation of
child documents and the main document:
%
\begin{center}
|\ifchilddoc |\textit{child-code}| |[|\||else |\textit{main-code}]| \||fi|
\end{center}

%%%%%%%%%%%%%%%%%%%%%%%%%%%%%%%%%%%%%%%%
\DescribeMacro{\childdocname}
\DescribeMacro{\childdocjob}
The macro |\childdocname| contains the filename (without extension)
of the main or child file being processed.
Note that |\childdocjob| will always contain the name of the main file.

%%%%%%%%%%%%%%%%%%%%%%%%%%%%%%%%%%%%%%%%
\paragraph{Title Page.}

Conditional processing can be used to include a title or banner page
in the main document when proper precautions are taken.
Importantly, the code in the main file should ensure that the page counter
(as well as other status parameters which are stored in the |.aux| files)
takes the same value after the conditional processing.
Otherwise the page numbers may take divergent values
depending on which part is compiled.

For example, a title page could be declared by:
%
\begin{center}
\begin{tabular}{l}
|\ifchilddoc\||else|\\
|\addtocounter{page}{-1}|\\
\textit{code for title page}\\
|\newpage|\\
|\||fi|
\end{tabular}
\end{center}
%
A banner page for the child documents can be generated by:
%
\begin{center}
\begin{tabular}{l}
|\ifchilddoc|\\
|\addtocounter{page}{-1}|\\
\textit{code for banner page}\\
|\newpage|\\
|\||fi|
\end{tabular}
\end{center}
%
Here one could write a message such as:
\begin{center}
|This is the part \childdocname{} of \childdocjob{}.|
\end{center}

%%%%%%%%%%%%%%%%%%%%%%%%%%%%%%%%%%%%%%%%%%%%%%%%%%%%%%%%%%%%%%%%%%%%%%%%%%%%%%%%
\subsection{Flags}
\label{sec:flags}

The package makes it easy to generate different versions
of the main or child documents.
To this end compilation flags can be defined
and assigned different default values.
They will be particularly useful in conjunction
with the forwarding mechanism described in \secref{sec:forward}.

For example, it may be useful to have a flag |\version|
which can be set to |draft| or |final|.
The document source will contain some conditional code
depending on the value of |\version|.
Suppose further, the flag should default to |final| for the main file
and to |draft| for child files
which is a natural assignment for editing the document.
This is achieved by placing the following code
in the preamble of the main document
(below the |\childdocmain| directive):
%
\begin{center}
\begin{tabular}{l}
|\ifchilddoc|\\
|\providecommand{\version}{draft}|\\
|\||else|\\
|\providecommand{\version}{final}|\\
|\||fi|
\end{tabular}
\end{center}
%
The definition by |\providecommand| makes sure
that previous definitions are not overwritten.
Further statements |\providecommand{\version}{...}|
can thus be added before the above code to override it.

For the main file, one might add a line
(between |\childdocmain| and the above block)
%
\begin{center}
|%\ifchilddoc\||else\providecommand{\version}{draft}\||fi|
\end{center}
%
which can be uncommented to produce a draft version.
Likewise one can add a line to the very top of a child file
(above the |\childdocof{|\textit{main}|}| directive)
%
\begin{center}
|%\providecommand{\version}{final}|
\end{center}
%
which can be uncommented to produce the final version of this child document.

%%%%%%%%%%%%%%%%%%%%%%%%%%%%%%%%%%%%%%%%%%%%%%%%%%%%%%%%%%%%%%%%%%%%%%%%%%%%%%%%
\subsection{Forwarding}
\label{sec:forward}

Different versions of the main or child documents
using compilation flags as described in \secref{sec:flags}
can be (permanently) stored in different files
for convenient compilation, viewing and distribution.
To this end, the package defines a command
to pass on compilation to a different file:

%%%%%%%%%%%%%%%%%%%%%%%%%%%%%%%%%%%%%%%%
\DescribeMacro{\childdocforward}
The command |\childdocforward| redirects processing to
another source file:
%
\begin{center}
\begin{tabular}{l}
|\input{childdoc.def}|\\
|\childdocforward[|\textit{main}|]{|\textit{dest}|}|\\
\end{tabular}
\end{center}
%
The argument \textit{dest} is the destination file
(without extension).
It should be the main file or one of the child files.
Note that further \textsf{childdoc} directives
such as |\childdocof| and |\childdocforward|
in the indicated file will be processed in this form.
The optional argument \textit{main}
passes on directly to the main file \textit{main}
while pretending to compile the child \textit{dest}.
This form behaves as if \textit{dest}
issues |\childdocof{|\textit{main}|}| right away,
and no further \textsf{childdoc} directives will be processed.

%%%%%%%%%%%%%%%%%%%%%%%%%%%%%%%%%%%%%%%%
\DescribeMacro{\...prefix}
In the alternative form |\childdocforwardprefix|,
%
\begin{center}
\begin{tabular}{l}
|\input{childdoc.def}|\\
|\childdocforwardprefix[|\textit{main}|]{|\textit{prefix}|}{|\textit{dest}|}|
\end{tabular}
\end{center}
%
the destination file is determined by a pattern
depending on the current file:
To make this work, the current file must be called
`{\textit{prefix}\hspace{0.2em}\textit{suffix}}'
with \textit{prefix} matching precisely the argument.
Processing is then passed on to the file
`{\textit{dest}\hspace{0.2em}\textit{suffix}}'.
Surely, the same effect is achieved by
directly specifying the
argument `{\textit{dest}\hspace{0.2em}\textit{suffix}}'
in the first form.
However, that requires to set up a different file
for each child. With the alternative form of the command
all these files can have exactly the same content
which simplifies setting them up and maintaining them.

For example, the following file |draft.tex|
with a compilation flag |\version| as described in \secref{sec:flags}
compiles the main document as a draft:
%
\begin{center}
\begin{tabular}{l}
|\def\version{draft}|\\
|\input{childdoc.def}|\\
|\childdocforward{|\textit{main}|}|
\end{tabular}
\end{center}
%
Likewise, the following files |final|\textit{nn}|.tex|
compile the final version of the child document
|child|\textit{nn}|.tex|:
%
\begin{center}
\begin{tabular}{l}
|\def\version{final}|\\
|\input{childdoc.def}|\\
|\childdocforwardprefix{final}{child}|
\end{tabular}
\end{center}
%

Note that when several versions of a main file and/or of each child file
are to be generated, it may be convenient to set up a |Makefile| or
shell script to automatise the process.

%%%%%%%%%%%%%%%%%%%%%%%%%%%%%%%%%%%%%%%%%%%%%%%%%%%%%%%%%%%%%%%%%%%%%%%%%%%%%%%%
\subsection{Command Line Processing}
\label{sec:commandline}

The effect of redirection files can also be achieved by invoking
the \LaTeX{} compiler with a more elaborate command line.
Most conveniently this should be done as part
of a shell script or a |Makefile|.

When using \textsf{childdoc} in the main file, the following
command lines effectively perform a redirection
(note that depending on the shell being used,
backslashes may have to be doubled: `|\|' $\to$ `|\\|'):
%
\begin{center}
|... -jobname "|\textit{target}|" |\\|"|[\textit{flags}]%
|\input{childdoc.def}\childdocforward[|\textit{main}|]{|\textit{dest}|}"|
\end{center}
%
Here \textit{target} is the name of the output file,
\textit{main} is the name of the main file
and \textit{dest} is the name of the main or child file to be processed
(all filenames without extensions).
The optional argument \textit{main} can be omitted
if \textit{main} matches \textit{dest}.
Optionally, compilation \textit{flags} can be defined via |\def| commands.
This command line makes the \TeX{} engine believe
it is compiling the file \textit{target}
whose content is specified as the latter parameter.
The provided code then forwards the processing to
\textit{main} or \textit{dest} as described in \secref{sec:forward}.

%%%%%%%%%%%%%%%%%%%%%%%%%%%%%%%%%%%%%%%%%%%%%%%%%%%%%%%%%%%%%%%%%%%%%%%%%%%%%%%%
\subsection{Include by Input}
\label{sec:input}

Including child documents by |\include| has some restrictions by design.
Most notably, the content of a child document always occupies
its own set of pages; pages cannot be shared between child documents.
Usually, this behaviour makes perfect sense
because each child document contain an essential part of the document.
However, in some situations it may be desirable to compose
a document from a collection of parts
without having mandatory page breaks between then.
For this case, the package
provides a mechanism to include parts
by |\input| which can also be processed individually.
However, by construction this mechanism
requires manual handling of the content to be output.

%%%%%%%%%%%%%%%%%%%%%%%%%%%%%%%%%%%%%%%%
\DescribeMacro{\ifchilddocmanual}
The main file should be prepared as usual, see \secref{sec:include}.
However, the document body must make a distinction
between processing of an individual part and of the main document, e.g.:
%
\begin{center}
\begin{tabular}{l}
|\ifchilddocmanual|\\
|\input{\childdocname}|\\
|\||else|\\
\textit{document body with }|\input{|\textit{part}|}|\\
|\||fi|
\end{tabular}
\end{center}
%
The conditional |\ifchilddocmanual| is true whenever
a part to be included by |\input| is being compiled,
and the name of the part is stored in |\childdocname|.

%%%%%%%%%%%%%%%%%%%%%%%%%%%%%%%%%%%%%%%%
\DescribeMacro{\childdocby}
Each part to be included by |\input| should start with:
%
\begin{center}
\begin{tabular}{l}
|\input{childdoc.def}|\\
|\childdocby{|\textit{main}|}|\\
\end{tabular}
\end{center}
%
The directive |\childdocby| is similar to |\childdocof|
described in \secref{sec:include},
but the subsequent selection of content must be done manually.
To that end, both |\ifchilddoc| and |\ifchilddocmanual|
will be true upon processing of a part,
and the name of the part is stored in |\childdocname|.
Note that |\jobname| will be set to the filename of the current part
so that each part receives an individual |.aux| file
that does not interfere with the |.aux| file(s) of the main document.
This behaviour can be altered by the alternative form
|\childdocby[*]{|\textit{main}|}| (with a non-empty optional argument)
which uses the |.aux| file of the main document
by setting |\jobname| to \textit{main}.

%%%%%%%%%%%%%%%%%%%%%%%%%%%%%%%%%%%%%%%%%%%%%%%%%%%%%%%%%%%%%%%%%%%%%%%%%%%%%%%%
\subsection{Driver Development}
\label{sec:driver}

The \textsf{childdoc} mechanism can also be use for the development
of definition files such as \LaTeX{} styles or classes.
This case differs from the above setup with multiple parts
included by |\include| in that no |\includeonly| should be invoked.
This can be achieved by starting the include file
(before |\ProvidesPackage|) with:
%
\begin{center}
\begin{tabular}{l}
|\input{childdoc.def}|\\
|\childdocforward{|\textit{main}|}|\\
\end{tabular}
\end{center}
%
or alternatively with:
%
\begin{center}
\begin{tabular}{l}
|\input{childdoc.def}|\\
|\childdocby{|\textit{main}|}|\\
\end{tabular}
\end{center}
%
Both forms have slightly different effects as described above.
The main file is prepared as usual, see \secref{sec:include}.

%%%%%%%%%%%%%%%%%%%%%%%%%%%%%%%%%%%%%%%%%%%%%%%%%%%%%%%%%%%%%%%%%%%%%%%%%%%%%%%%
\subsection{Legacy Detection}
\label{sec:detection}

The directive |\childdocmain| in the main file can detect
whether the complete document or merely a child is to be compiled
even without using the directive |\childdocof|.
This method is deprecated because it is less robust
and there is no compelling reason to use it;
it is merely provided for backward compatibility
and it may be removed in future versions.

If the detection mechanism is to be used,
it is mandatory to correctly specify
the filename of the main file as the argument of |\childdocmain|:
%
\begin{center}
\begin{tabular}{l}
|\input{childdoc.def}|\\
|\childdocmain{|\textit{main}|}|\\
\end{tabular}
\end{center}
%
If |\jobname| does not match the argument \textit{main} of |\childdocmain|,
it is assumed that |\jobname| points to the child file to be compiled.
When using |\childdocmain| with the main file specified as argument,
it suffices to start a child file
with just |\input{|\textit{main}|}|
without loading of the package and using |\childdocof|.
If instead all processing is done
with the appropriate \textsf{childdoc} directives,
the argument of \textit{main} of |\childdocmain| can be empty.

An alternative version of the command line processing described
in \secref{sec:commandline} using the detection mechanism reads:
%
\begin{center}
|... -jobname "|\textit{target}|" "|[\textit{flags}]%
[|\def\jobname{|\textit{dest}|}|]|\input{|\textit{main}|}"|
\end{center}

%%%%%%%%%%%%%%%%%%%%%%%%%%%%%%%%%%%%%%%%%%%%%%%%%%%%%%%%%%%%%%%%%%%%%%%%%%%%%%%%
\subsection{Manual Code}
\label{sec:manual}

In case one cannot be certain whether the definitions file |childdoc.def|
is installed on the target \TeX{} distribution
and one prefers not to ship it,
it is conceivable to paste a few relevant commands into the sources.

To that end, drop all statements |\input{childdoc.def}|
and perform the replacements as outlined below.
Instead of |\childdocmain{|\textit{main}|}| add the following code
to the top of the main file:
%
\begin{center}
\begin{tabular}{l}
|\||ifdefined\childdocname\endinput\||fi\newif\ifchilddoc|\\
|\edef\childdocname{\scantokens\expandafter{\jobname\noexpand}}|\\
|\def\childdocmain{|\textit{main}|}\||ifx\childdocmain\childdocname\||else|\\
|\childdoctrue\includeonly{\childdocname}\let\jobname\childdocmain\||fi|\\
\end{tabular}
\end{center}
%
Instead of |\childdocof{|\textit{main}|}| just include the main file
at the top of each child file:
%
\begin{center}
|\input{|\textit{main}|}|
\end{center}
%
A simple redirection |\childdocforward{|\textit{dest}|}| is achieved by:
%
\begin{center}
|\def\jobname{|\textit{dest}|}\input{\jobname}|
\end{center}
%
The redirection with prefix
|\childdocforwardprefix[|\textit{prefix}|]{|\textit{dest}|}|
is accomplished by:
%
\begin{center}
\begin{tabular}{l}
|{\edef\jobname{\scantokens\expandafter{\jobname\noexpand}}|\\
|\def\redirectjob |\textit{prefix}|#1~~~{\gdef\jobname{|\textit{dest}|#1}}|\\
|\expandafter\redirectjob\jobname~~~}\input{\jobname}|
\end{tabular}
\end{center}

In an alternative approach,
child documents can be compiled by a specific command line
without additional code or specific definitions:
%
\begin{center}
|... -jobname "|\textit{target}|" "|[\textit{flags}]%
|\includeonly{|\textit{dest}|}\input{|\textit{main}|}"|
\end{center}
%

%%%%%%%%%%%%%%%%%%%%%%%%%%%%%%%%%%%%%%%%%%%%%%%%%%%%%%%%%%%%%%%%%%%%%%%%%%%%%%%%
%%%%%%%%%%%%%%%%%%%%%%%%%%%%%%%%%%%%%%%%%%%%%%%%%%%%%%%%%%%%%%%%%%%%%%%%%%%%%%%%
\section{Information}

%%%%%%%%%%%%%%%%%%%%%%%%%%%%%%%%%%%%%%%%%%%%%%%%%%%%%%%%%%%%%%%%%%%%%%%%%%%%%%%%
\subsection{Copyright}

Copyright \copyright{} 2017--2018 Niklas Beisert

This work may be distributed and/or modified under the
conditions of the \LaTeX{} Project Public License, either version 1.3
of this license or (at your option) any later version.
The latest version of this license is in
  \url{http://www.latex-project.org/lppl.txt}
and version 1.3 or later is part of all distributions of \LaTeX{}
version 2005/12/01 or later.

This work has the LPPL maintenance status `maintained'.

The Current Maintainer of this work is Niklas Beisert.

This work consists of the files |README.txt|, |childdoc.ins| and |childdoc.dtx|
as well as the derived files |childdoc.def|, |cdocsamp.tex|
with |cdocsch1.tex|, |cdocsch2.tex|, |cdocspt3.tex|, |cdocspt4.tex|,
|cdocsdrf.tex|, |cdocsfn1.tex|, |cdocsfn2.tex|
as well as |childdoc.pdf|.

%%%%%%%%%%%%%%%%%%%%%%%%%%%%%%%%%%%%%%%%%%%%%%%%%%%%%%%%%%%%%%%%%%%%%%%%%%%%%%%%
\subsection{Files and Installation}

The package consists of the files:
%
\begin{center}
\begin{tabular}{ll}
    |README.txt|   & readme file \\
    |childdoc.ins| & installation file \\
    |childdoc.dtx| & source file \\
    |childdoc.def| & definition file \\
    |cdocsamp.tex| & sample main file \\
    |cdocsch1.tex| & sample include file \\
    |cdocsch2.tex| & sample include file \\
    |cdocspt3.tex| & sample part file \\
    |cdocspt4.tex| & sample part file \\
    |cdocsdrf.tex| & sample redirection file \\
    |cdocsfn1.tex| & sample redirection file \\
    |cdocsfn2.tex| & sample redirection file \\
    |childdoc.pdf| & manual
\end{tabular}
\end{center}
%
The distribution consists of the files
|README.txt|, |childdoc.ins| and |childdoc.dtx|.
%
\begin{itemize}
\item
Run (pdf)\LaTeX{} on |childdoc.dtx|
to compile the manual |childdoc.pdf| (this file).
\item
Run \LaTeX{} on |childdoc.ins| to create the definitions file |childdoc.def|
and the sample |cdocsamp.tex| with include files
|cdocsch1.tex|, |cdocsch2.tex|, |cdocspt3.tex|, |cdocspt4.tex|,
|cdocsdrf.tex|, |cdocsfn1.tex|, |cdocsfn2.tex|.
Then copy the file |childdoc.def| to an appropriate directory of your \LaTeX{}
distribution, e.g.\ \textit{texmf-root}|/tex/latex/childdoc|.
\end{itemize}

%%%%%%%%%%%%%%%%%%%%%%%%%%%%%%%%%%%%%%%%%%%%%%%%%%%%%%%%%%%%%%%%%%%%%%%%%%%%%%%%
\subsection{Related CTAN Packages}

There are several other packages which offer a similar functionality:
%
\begin{itemize}
\item
The packages
\href{http://ctan.org/pkg/docmute}{\textsf{docmute}},
\href{http://ctan.org/pkg/includex}{\textsf{includex}} and
\href{http://ctan.org/pkg/standalone}{\textsf{standalone}}
provide commands to include only the document body of
a child file thus allowing both files to be compiled individually.
\item
The packages \href{http://ctan.org/pkg/subdocs}{\textsf{subdocs}}
and \href{http://ctan.org/pkg/subfiles}{\textsf{subfiles}}
provide structures in which the main and child documents can be
encapsulated and allowing them to be compiled individually.
The inclusion mechanism is different from the conventional |\include|.
\item
The package \href{http://ctan.org/pkg/combine}{\textsf{combine}}
is an elaborate solution to combine several documents into one.
\end{itemize}
%
See also the CTAN topic \href{http://ctan.org/topic/subdocs}{\textsf{subdocs}}
for further related packages.
The present package differs from the above solutions in that
a document structure constructed with the conventional |\include| mechanism
just needs two extra commands at the top of every file
such that all constituent files can be compiled individually.

%%%%%%%%%%%%%%%%%%%%%%%%%%%%%%%%%%%%%%%%%%%%%%%%%%%%%%%%%%%%%%%%%%%%%%%%%%%%%%%%
%\subsection{Feature Suggestions}
%
%The following is a list of features which may be useful for future
%versions of this package:
%%
%\begin{itemize}
%\item
%\ldots
%\end{itemize}

%%%%%%%%%%%%%%%%%%%%%%%%%%%%%%%%%%%%%%%%%%%%%%%%%%%%%%%%%%%%%%%%%%%%%%%%%%%%%%%%
\subsection{Revision History}

%%%%%%%%%%%%%%%%%%%%%%%%%%%%%%%%%%%%%%%%
\paragraph{v2.0:} 2018/12/30

\begin{itemize}
\item
immediate forward processing
\item
added |\childdocby| mechanism
\item
manual restructured
\end{itemize}

%%%%%%%%%%%%%%%%%%%%%%%%%%%%%%%%%%%%%%%%
\paragraph{v1.6:} 2018/01/17

\begin{itemize}
\item
application for development of include files
\item
corrections to manual
\end{itemize}

%%%%%%%%%%%%%%%%%%%%%%%%%%%%%%%%%%%%%%%%
\paragraph{v1.5:} 2017/05/21

\begin{itemize}
\item
more complete structuring introduced
\item
|\childdocof| introduced
\item
|\childdoc| renamed to |\childdocmain|
\item
|\childredirect| renamed to |\childdocforward| and |\childdocforwardprefix|
and functionality expanded
\end{itemize}

%%%%%%%%%%%%%%%%%%%%%%%%%%%%%%%%%%%%%%%%
\paragraph{v1.0:} 2017/04/27

\begin{itemize}
\item
manual and install package
\item
first version published on CTAN
\end{itemize}

%%%%%%%%%%%%%%%%%%%%%%%%%%%%%%%%%%%%%%%%
\paragraph{v0.6:} 2017/04/26

\begin{itemize}
\item
redirection mechanism added
\end{itemize}

%%%%%%%%%%%%%%%%%%%%%%%%%%%%%%%%%%%%%%%%
\paragraph{v0.5:} 2017/04/26

\begin{itemize}
\item
functionality in definition file
\end{itemize}


%%%%%%%%%%%%%%%%%%%%%%%%%%%%%%%%%%%%%%%%%%%%%%%%%%%%%%%%%%%%%%%%%%%%%%%%%%%%%%%%
%%%%%%%%%%%%%%%%%%%%%%%%%%%%%%%%%%%%%%%%%%%%%%%%%%%%%%%%%%%%%%%%%%%%%%%%%%%%%%%%
%%%%%%%%%%%%%%%%%%%%%%%%%%%%%%%%%%%%%%%%%%%%%%%%%%%%%%%%%%%%%%%%%%%%%%%%%%%%%%%%
\appendix

\settowidth\MacroIndent{\rmfamily\scriptsize 000\ }

 \DocInput{childdoc.dtx}

\end{document}
%</driver>
% \fi
%
% %%%%%%%%%%%%%%%%%%%%%%%%%%%%%%%%%%%%%%%%%%%%%%%%%%%%%%%%%%%%%%%%%%%%%%%%%%%%%%
% %%%%%%%%%%%%%%%%%%%%%%%%%%%%%%%%%%%%%%%%%%%%%%%%%%%%%%%%%%%%%%%%%%%%%%%%%%%%%%
% \section{Sample}
%\iffalse
%<*samplemain>
%\fi
%
% The following presents a sample document
% with two chapters, two parts, a title page,
% a compile flag as well as three forwarding files to set the flag.
% It consists of eight |.tex| files:
% \begin{center}
% \begin{tabular}{ll}
% |cdocsamp.tex|&main file\\
% |cdocsch1.tex|&include file for chapter 1\\
% |cdocsch2.tex|&include file for chapter 2\\
% |cdocspt3.tex|&include file for part 3\\
% |cdocspt4.tex|&include file for part 4\\
% |cdocsdrf.tex|&forwarding file for main file in draft mode\\
% |cdocsfi1.tex|&forwarding file for final version of chapter 1\\
% |cdocsfi2.tex|&forwarding file for final version of chapter 2\\
% \end{tabular}
% \end{center}
% Each of the eight files can be compiled directly by the \LaTeX{} compiler.
%
% %%%%%%%%%%%%%%%%%%%%%%%%%%%%%%%%%%%%%%
% \paragraph{Main File.}
%
% The main file is called |cdocsamp.tex|.
%
% Load the \textsf{childdoc} definitions and
% declare the filename for the main document:
%    \begin{macrocode}
\input{childdoc.def}
\childdocmain{}
%    \end{macrocode}

% Optional override for |\version| flag:
%    \begin{macrocode}
%%\ifchilddoc\else\providecommand{\version}{draft}\fi
%    \end{macrocode}

% Define the default values for the |\version| flag
% (|final| for the main file and |draft| for childs):
%    \begin{macrocode}
\ifchilddoc
\providecommand{\version}{draft}
\else
\providecommand{\version}{final}
\fi
%    \end{macrocode}

% Load the standard document class:
%    \begin{macrocode}
\documentclass[12pt]{article}
%    \end{macrocode}

% Start the document body:
%    \begin{macrocode}
\begin{document}
%    \end{macrocode}

% Declare a title page.
% Print title, part of document being processed and version flag:
%    \begin{macrocode}
\addtocounter{page}{-1}
\begin{center}
{\LARGE\bfseries{}childdoc example\par}
\vspace{1cm}
\ifchilddoc
\ifchilddocmanual part\else chapter\fi:
`\childdocname' of `\childdocjob'\par
\else
main document: `\childdocjob'\par
\fi
version: \version\par
\end{center}
\newpage
%    \end{macrocode}

% Manually include selected file,
% otherwise process as usual:
%    \begin{macrocode}
\ifchilddocmanual
\section*{part `\childdocname'}
\input{\childdocname}
\else
%    \end{macrocode}

% Include the two chapters:
%    \begin{macrocode}
\include{cdocsch1}
\include{cdocsch2}
%    \end{macrocode}

% Include the two parts unless only chapters should be displayed:
%    \begin{macrocode}
\ifchilddoc\else
\section{part three}
\input{cdocspt3}
\section{part four}
\input{cdocspt4}
\fi
%    \end{macrocode}

% Process as usual until here:
%    \begin{macrocode}
\fi
%    \end{macrocode}

% End of document body:
%    \begin{macrocode}
\end{document}
%    \end{macrocode}
%\iffalse
%</samplemain>
%\fi
%
% %%%%%%%%%%%%%%%%%%%%%%%%%%%%%%%%%%%%%%
% \paragraph{Chapter Include Files.}
%
% The include files are called |cdocsch1.tex| and |cdocsch2.tex|.
%
%\iffalse
%<*samplechap1|samplechap2>
%\fi

% Optional override for |\version| flag:
%    \begin{macrocode}
%%\providecommand{\version}{final}
%    \end{macrocode}

% Include the main document:
%    \begin{macrocode}
\input{childdoc.def}
\childdocof{cdocsamp}
%    \end{macrocode}

%\iffalse
%</samplechap1|samplechap2>
%\fi
%
%\iffalse
%<*samplechap1>
%\fi
% Some text for chapter 1:
%    \begin{macrocode}
\section{one}
some text in chapter one
%    \end{macrocode}

%\iffalse
%</samplechap1>
%\fi
% Some text for chapter 2:
%\iffalse
%<*samplechap2>
%\fi
%    \begin{macrocode}
\section{two}
more text in chapter two
%    \end{macrocode}

%\iffalse
%</samplechap2>
%\fi
%
% %%%%%%%%%%%%%%%%%%%%%%%%%%%%%%%%%%%%%%
% \paragraph{Part Include Files.}
%
% The include files are called |cdocspt3.tex| and |cdocspt4.tex|.
%
%\iffalse
%<*samplepart3|samplepart4>
%\fi

% Optional override for |\version| flag:
%    \begin{macrocode}
%%\providecommand{\version}{final}
%    \end{macrocode}

% Include the main document:
%    \begin{macrocode}
\input{childdoc.def}
\childdocby{cdocsamp}
%    \end{macrocode}

%\iffalse
%</samplepart3|samplepart4>
%\fi
%
%\iffalse
%<*samplepart3>
%\fi
% Some text for part 3:
%    \begin{macrocode}
some text in part three
%    \end{macrocode}

%\iffalse
%</samplepart3>
%\fi
% Some text for part 4:
%\iffalse
%<*samplepart4>
%\fi
%    \begin{macrocode}
more text in part four
%    \end{macrocode}

%\iffalse
%</samplepart4>
%\fi
%
% %%%%%%%%%%%%%%%%%%%%%%%%%%%%%%%%%%%%%%
% \paragraph{Forwarding for a Complete Draft.}
%
% The following forwarding file |cdocsdrf.tex|
% compiles the main document in draft mode:
%\iffalse
%<*sampledraft>
%\fi
%    \begin{macrocode}
\def\version{draft}
\input{childdoc.def}
\childdocforward{cdocsamp}
%    \end{macrocode}

%\iffalse
%</sampledraft>
%\fi
%
% %%%%%%%%%%%%%%%%%%%%%%%%%%%%%%%%%%%%%%
% \paragraph{Forwarding for Final Version of the Chapters.}
%
% The following forwarding files |cdocsfn1.tex| and |cdocsfn2.tex|
% (with identical content)
% compile the final versions of the child documents
% |cdocsch1.tex| and |cdocsch2.tex|, respectively:
%\iffalse
%<*samplefinal>
%\fi
%    \begin{macrocode}
\def\version{final}
\input{childdoc.def}
\childdocforwardprefix[cdocsamp]{cdocsfn}{cdocsch}
%    \end{macrocode}

%\iffalse
%</samplefinal>
%\fi
%
% %%%%%%%%%%%%%%%%%%%%%%%%%%%%%%%%%%%%%%
% \paragraph{Command Line Processing.}
%
% The following three command lines generate the output files
% |cdocscld|, |cdocscl1| and |cdocscl2|
% which should be identical to
% |cdocsdrf|, |cdocsch1| and |cdocsfn2|, respectively:
% \begin{center}
% \begin{tabular}{l}
% |latex -jobname cdocscld \|\\
% |  "\def\version{draft}\input{childdoc.def}\childdocforward{cdocsamp}"|\\
% |latex -jobname cdocscl1 \|\\
% |  "\input{childdoc.def}\childdocforward[cdocsamp]{cdocsch1}"|\\
% |latex -jobname cdocscl2 \|\\
% |  "\def\version{final}\input{childdoc.def}\childdocforward{cdocsch2}"|
% \end{tabular}
% \end{center}
% Note that the trailing backslash on each first line
% merely continues the input to the second line
% (for convenient cut ant paste).
% Furthermore, the command |latex| can be replaced by any
% of its alternative versions such as |pdflatex|.
%
% %%%%%%%%%%%%%%%%%%%%%%%%%%%%%%%%%%%%%%%%%%%%%%%%%%%%%%%%%%%%%%%%%%%%%%%%%%%%%%
% %%%%%%%%%%%%%%%%%%%%%%%%%%%%%%%%%%%%%%%%%%%%%%%%%%%%%%%%%%%%%%%%%%%%%%%%%%%%%%
% \section{Implementation}
%\iffalse
%<*package>
%\fi
%
% This section describes the definitions file |childdoc.def|.

% The definitions cannot be loaded using |\usepackage| or |\RequirePackage|
% which has a mechanism to prevent loading a style file more than once.
% When loading the definitions by means of |\input|
% multiple instances have to be prevented manually:
%\iffalse
%This code needs to be before the `\ProvidesFile' directive
%which is defined at the beginning of this file.
%Therefore it is also placed there and commented out here.
%</package>
%<*discard>
%\fi
%    \begin{macrocode}
\ifdefined\childdocmain\endinput\fi
%    \end{macrocode}
%\iffalse
%</discard>
%<*package>
%\fi
%
% \macro{\ifchilddoc}
% \macro{\ifchilddocmanual}
% The conditional |\ifchilddoc| tells whether a
% child (true) or main (false) document is being compiled.
% The conditional |\ifchilddocmanual| tells whether
% the |\includeonly| mechanism is used (false) or
% the selection of child files must be performed manually (true).
% The definitions initialise to false:
%    \begin{macrocode}
\newif\ifchilddoc
\newif\ifchilddocmanual
%    \end{macrocode}

% \macro{\childdocname}
% \macro{\childdocjob}
% The macro |\childdocname| stores the name of the main document
% to be compiled. The macro |\childdocjob| stores the name of
% the document on which the \LaTeX{} compiler was originally invoked.
% The content of |\jobname| cannot be compared
% to filenames specified in the source due to different catcodes.
% The following code rescans |\jobname|, stores the result
% in |\childdocname| and saves a copy in |\childdocjob|:
%    \begin{macrocode}
\edef\childdocname{\scantokens\expandafter{\jobname\noexpand}}
\let\childdocjob\childdocname
%    \end{macrocode}

% \macro{\childdocdisable}
% The macro |\childdocdisable| prevents the main file
% from being processed more than once.
% At this stage, the main document command |\childdocmain|
% is assumed to be called once again where it should do nothing.
% Any subsequent call to it should prevent
% a secondary processing of the main document
% It overwrites the forwarding commands
% |\childdocof| and |\childdocforward|
% with empty macros to prevent further inclusions of the main document:
%    \begin{macrocode}
\newcommand{\childdocdisable}
{
  \renewcommand{\childdocmain}[1]{\renewcommand{\childdocmain}[1]{\endinput}}
  \renewcommand{\childdocof}[1]{}
  \renewcommand{\childdocby}[2][]{}
  \renewcommand{\childdocforward}[2][]{}
  \renewcommand{\childdocdisable}{}
}
%    \end{macrocode}

% \macro{\childdocmain}
% The macro |\childdocmain| is to be called at the top of the main file
% with nothing or the main filename (without extension) as argument.
% First, it breaks loops.
% If the argument is not empty and does not match |\childdocname|
% (which is set by the first inclusion of |childdoc.def|),
% |\ifchilddoc| is set to true, |\includeonly| is applied to the child file
% and |\jobname| is set to the main file
% (for proper handling of |.aux| files):
%    \begin{macrocode}
\newcommand{\childdocmain}[1]
{
  \childdocdisable\childdocmain{}
  \if?#1?\else
    \begingroup
      \def\childdoctmp{#1}
      \ifx\childdoctmp\childdocname
        \def\childdoctmp{}
      \else
        \def\childdoctmp
        {
          \childdoctrue
          \includeonly{\childdocname}
          \def\childdocjob{#1}
          \def\jobname{#1}
        }
      \fi
      \expandafter
    \endgroup
    \childdoctmp
  \fi
}
%    \end{macrocode}

% \macro{\childdocof}
% The command |\childdocof| redirects
% compilation to the main file |#1|.
%    \begin{macrocode}
\newcommand{\childdocof}[1]
{
  \childdocdisable
  \childdoctrue
  \includeonly{\childdocname}
  \def\jobname{#1}
  \def\childdocjob{#1}
  \input{#1}
}
%    \end{macrocode}

% \macro{\childdocby}
% The command |\childdocby| ....
%    \begin{macrocode}
\newcommand{\childdocby}[2][]
{
  \childdocdisable
  \childdoctrue
  \childdocmanualtrue
  \if?#1?\else
    \def\jobname{#2}
  \fi
  \def\childdocjob{#2}
  \input{#2}
  \endinput
}
%    \end{macrocode}

% \macro{\childdocforward}
% The command |\childdocforward| redirects
% compilation to the main file or
% (if the optional argument is given) a child file.
% Parameters are set as if the main file
% or a child file starting with |\childdocof| was compiled.
% Then compilation is handed over to the main file:
%    \begin{macrocode}
\newcommand{\childdocforward}[2][]
{
  \begingroup
    \if?#1?
      \def\childdoctmp
      {
        \def\childdocname{#2}
        \def\childdocjob{#2}
        \def\jobname{#2}
        \input{#2}
        \endinput
      }
    \else
      \def\childdoctmp
      {
        \childdocdisable
        \def\childdocname{#2}
        \childdoctrue
        \includeonly{#2}
        \def\childdocjob{#1}
        \def\jobname{#1}
        \input{#1}
        \endinput
      }
    \fi
    \expandafter
  \endgroup
  \childdoctmp
}
%    \end{macrocode}

% \macro{\childdocforwardprefix}
% The command |\childdocforwardprefix| redirects
% compilation to the main or a child file by means of a pattern.
% The prefix |#1| in the current filename is replaced by |#2|
% and the suffix of the current filename is kept
% (it is assumed that the filename does not contain the substring `|~~~|'
% which is used as a delimiter).
% Compilation is handed over to the new file by |\childdocforward|:
%    \begin{macrocode}
\newcommand{\childdocforwardprefix}[3][]
{
  \begingroup
    \def\childdocextract #2##1~~~{\def\childdoctmp{\childdocforward[#1]{#3##1}}}
    \expandafter\childdocextract\childdocname~~~
    \expandafter
  \endgroup
  \childdoctmp
}
%    \end{macrocode}

% \macro{\childdoc}
% The deprecated macro |\childdoc| is a legacy version of |\childdocmain|:
%    \begin{macrocode}
\newcommand{\childdoc}{\childdocmain}
%    \end{macrocode}

% \macro{\childdocredirect}
% The deprecated macro |\childdocredirect| is a legacy version
% of |\childdocforward| and |\childdocforwardprefix|:
%    \begin{macrocode}
\newcommand{\childdocredirect}[2][]
{
  \begingroup
    \if?#1?
      \def\childdoctmp{\childdocforward{#2}}
    \else
      \def\childdoctmp{\childdocforwardprefix{#1}{#2}}
    \fi
    \expandafter
  \endgroup
  \childdoctmp
}
%    \end{macrocode}

%\iffalse
%</package>
%\fi
%
\endinput
|\\
|\childdocforward{|\textit{main}|}|\\
\end{tabular}
\end{center}
%
or alternatively with:
%
\begin{center}
\begin{tabular}{l}
|% \iffalse
%
% childdoc.dtx Copyright (C) 2017-2018 Niklas Beisert
%
% This work may be distributed and/or modified under the
% conditions of the LaTeX Project Public License, either version 1.3
% of this license or (at your option) any later version.
% The latest version of this license is in
%   http://www.latex-project.org/lppl.txt
% and version 1.3 or later is part of all distributions of LaTeX
% version 2005/12/01 or later.
%
% This work has the LPPL maintenance status `maintained'.
%
% The Current Maintainer of this work is Niklas Beisert.
%
% This work consists of the files childdoc.dtx and childdoc.ins
% and the derived files childdoc.def and cdocsamp.tex with
% cdocsch1.tex, cdocsch2.tex, cdocsdrf.tex, cdocsfn1.tex, cdocsfn2.tex.
%
%<package>\ifdefined\childdocmain\endinput\fi
%<package>\ProvidesFile{childdoc.def}[2018/12/30 v2.0 child document driver]
%<samplemain>\ProvidesFile{cdocsamp.tex}[2018/12/30 v2.0 sample for childdoc]
%<*driver>
%\ProvidesFile{childdoc.drv}[2018/12/30 v2.0 childdoc reference manual file]
\PassOptionsToClass{10pt,a4paper}{article}
\documentclass{ltxdoc}

\usepackage[margin=35mm]{geometry}
\usepackage{hyperref}
\usepackage{hyperxmp}
\usepackage[usenames]{color}

\hypersetup{colorlinks=true}
\hypersetup{pdfstartview=FitH}
\hypersetup{pdfpagemode=UseNone}
\hypersetup{pdfsource={}}
\hypersetup{pdflang={en-UK}}
\hypersetup{pdfcopyright={Copyright 2017-2018 Niklas Beisert.
  This work may be distributed and/or modified under the
  conditions of the LaTeX Project Public License, either version 1.3
  of this license or (at your option) any later version.}}
\hypersetup{pdflicenseurl={http://www.latex-project.org/lppl.txt}}
\hypersetup{pdfcontactaddress={ETH Zurich, ITP, HIT K,
  Wolfgang-Pauli-Strasse 27}}
\hypersetup{pdfcontactpostcode={8093}}
\hypersetup{pdfcontactcity={Zurich}}
\hypersetup{pdfcontactcountry={Switzerland}}
\hypersetup{pdfcontactemail={nbeisert@itp.phys.ethz.ch}}
\hypersetup{pdfcontacturl={http://people.phys.ethz.ch/\xmptilde nbeisert/}}

\newcommand{\secref}[1]{\hyperref[#1]{section \ref*{#1}}}

\parskip1ex
\parindent0pt
\let\olditemize\itemize
\def\itemize{\olditemize\parskip0pt}

\begin{document}

\title{The \textsf{childdoc} Package}
\hypersetup{pdftitle={The childdoc Package}}
\author{Niklas Beisert\\[2ex]
  Institut f\"ur Theoretische Physik\\
  Eidgen\"ossische Technische Hochschule Z\"urich\\
  Wolfgang-Pauli-Strasse 27, 8093 Z\"urich, Switzerland\\[1ex]
  \href{mailto:nbeisert@itp.phys.ethz.ch}
  {\texttt{nbeisert@itp.phys.ethz.ch}}}
\hypersetup{pdfauthor={Niklas Beisert}}
\hypersetup{pdfsubject={Manual for the LaTeX2e Package childdoc}}
\date{30 December 2018, \textsf{v2.0}}
\maketitle

\begin{abstract}\noindent
\textsf{childdoc} is a \LaTeXe{} package
that enables the direct compilation
of document sections included by |\include|
to individual files.
\end{abstract}

\begingroup
\parskip0ex
\tableofcontents
\endgroup

%%%%%%%%%%%%%%%%%%%%%%%%%%%%%%%%%%%%%%%%%%%%%%%%%%%%%%%%%%%%%%%%%%%%%%%%%%%%%%%%
%%%%%%%%%%%%%%%%%%%%%%%%%%%%%%%%%%%%%%%%%%%%%%%%%%%%%%%%%%%%%%%%%%%%%%%%%%%%%%%%
\section{Introduction}

\LaTeX{} provides a mechanism to structure a large document (such as a book)
into a main file and several child files (containing the chapters)
using the |\include| command.
This mechanism is beneficial for documents
which span hundreds of pages in order to
make the source file(s) more manageable.
Moreover, compilation can be restricted to
selected child files by means of the |\includeonly| command.
The latter feature can be used to reduce the compilation time while editing
(this was significantly more useful in the earlier days of \LaTeX{})
or to generate a smaller document which is easier to navigate.
Another application of |\includeonly| is to generate
documents consisting of selected parts of the complete document.

However, there are a few drawbacks of the plain |\include| mechanism:
\begin{itemize}
\item
The child files cannot be compiled on their own,
they can only be compiled via the main file.
A naive editing environment
(such as a text editor with an option
to have the current file processed by \LaTeX)
may require one to switch to the main file before compiling;
attempting to compile the child file produces errors.
\item
The main file must be modified (each time)
to adjust the |\includeonly| command
to the present needs. This easily leaves the main file in a messy state.
\item
The generated document will always carry the filename
of the main document. This is inconvenient if
several child files are to be compiled and
to be kept for distribution.
\end{itemize}

The present package provides a simple interface
to make child files individually compilable by \LaTeX{}.
Compiling a child file then has the same effect as compiling
the main file with an |\includeonly| command
to select the appropriate child.
Moreover the generated document will carry the name of the child
rather than the main file.
This resolves all three above issues.

This feature is meant to make the editing of books,
thesis documents and lecture notes somewhat more convenient.
However, the package can also be used efficiently for
composing a series of documents (such as exercise sheets)
which are typically distributed individually.
It then assists the author in generating the individual documents
(potentially in different versions)
as well as a document containing the collected series.
Another application is in developing style files
or other kinds of included material
where compilation of the style file could redirect
to a sample or test file.

%%%%%%%%%%%%%%%%%%%%%%%%%%%%%%%%%%%%%%%%%%%%%%%%%%%%%%%%%%%%%%%%%%%%%%%%%%%%%%%%
%%%%%%%%%%%%%%%%%%%%%%%%%%%%%%%%%%%%%%%%%%%%%%%%%%%%%%%%%%%%%%%%%%%%%%%%%%%%%%%%
\section{Usage}

First of all, the package \textsf{childdoc} is \emph{not} a standard
\LaTeXe{} |.sty| style file! Therefore it needs to be invoked in
a non-standard way.

%%%%%%%%%%%%%%%%%%%%%%%%%%%%%%%%%%%%%%%%%%%%%%%%%%%%%%%%%%%%%%%%%%%%%%%%%%%%%%%%
\subsection{Included Files}
\label{sec:include}

%%%%%%%%%%%%%%%%%%%%%%%%%%%%%%%%%%%%%%%%
\DescribeMacro{\childdocmain}
To use the package, add the commands
\begin{center}
\begin{tabular}{l}
|\input{childdoc.def}|\\
|\childdocmain{}|\\
\end{tabular}
\end{center}
at the very top of the main \LaTeX{} file,
in particular \emph{before} the |\documentclass| statement!
The argument of |\childdocmain| should be left empty
(but it must be present).

%%%%%%%%%%%%%%%%%%%%%%%%%%%%%%%%%%%%%%%%
\DescribeMacro{\childdocof}
Furthermore, add the commands
\begin{center}
\begin{tabular}{l}
|\input{childdoc.def}|\\
|\childdocof{|\textit{main}|}|\\
\end{tabular}
\end{center}
at the top of every child file \textit{child}
which is included by |\include{|\textit{child}|}|
from within the main file
(or at least for those files to be compiled individually).
The argument \textit{main} must be the filename of the main file.

There are a couple of
considerations in setting up the main and child documents:

%%%%%%%%%%%%%%%%%%%%%%%%%%%%%%%%%%%%%%%%
\paragraph{Restrictions.}

Please note the following restrictions:
\begin{itemize}
\item
|\childdocmain| must be called with one argument \textit{main}
to ensure compatibility with earlier version of the package.
It must either be empty (|\childdocmain{}|)
or precisely match the filename of the main file in which it is specified.
See \secref{sec:detection} for further information.
\item
The filename \textit{main} must be specified without the |.tex| extension.
\item
The filename \textit{main} is case sensitive
(even in case-insensitive file systems)
due to internal string comparison.
\item
The argument \textit{main} should be fully expanded, it cannot be a macro.
\item
Subdirectories and special characters should be avoided in filenames.
\item
The command |\childdocmain{|\textit{main}|}| must be followed by a whitespace.
It should not be followed immediately by another command
or by a comment mark `|%|'.
This is because the \TeX{} parser reads the token immediately following
the argument of |\childdocmain| and puts it
at the beginning of every child section;
however, a white\-space is ignored.
\end{itemize}

%%%%%%%%%%%%%%%%%%%%%%%%%%%%%%%%%%%%%%%%
\paragraph{Content of Main File.}

It is advisable to place all content in the child files included by |\include|.
Any output contained in the main file will appear in all child documents
unless suppressed manually;
it cannot be suppressed automatically by the |\includeonly| directive
and thus should normally be avoided.
A method to include some content in the main file
by means of conditional processing is described in \secref{sec:conditional}.

%%%%%%%%%%%%%%%%%%%%%%%%%%%%%%%%%%%%%%%%
\paragraph{Page Numbering.}

When only a part of the document is compiled,
the appropriate numbering of pages
(as well as other status parameters)
is determined from the |.aux| files.
The latter contain information from previous passes.
However this information needs to propagate through
all intermediate child documents.
Therefore the page numbering in child documents may well
be inconsistent until the complete document is compiled at least once.

A useful (if unconventional) way to always ensure a consistent
page numbering is to restart the numbering in each child document
and denote the pages by `\textit{child}|.|\textit{page}'
where \textit{child} represents the chapter/section number of the child file.
This can be achieved by the command
|\numberwithin{page}{|\textit{child}|}|
of the \textsf{amsmath} package
where \textit{child} can be |chapter| or |section|
depending on the chosen structuring.
Alternatively, one can modify the macro |\thepage| appropriately
and reset the counter |page| at the start of each child file.

%%%%%%%%%%%%%%%%%%%%%%%%%%%%%%%%%%%%%%%%%%%%%%%%%%%%%%%%%%%%%%%%%%%%%%%%%%%%%%%%
\subsection{Conditional Processing}
\label{sec:conditional}

The package provides a mechanism to compile different versions
of a document. To customise the versions further some conditional processing
can come in handy to distinguish which version is being compiled.
The package provides two macros to describe the compilation context:

%%%%%%%%%%%%%%%%%%%%%%%%%%%%%%%%%%%%%%%%
\DescribeMacro{\ifchilddoc}
The conditional |\ifchilddoc| distinguishes between the compilation of
child documents and the main document:
%
\begin{center}
|\ifchilddoc |\textit{child-code}| |[|\||else |\textit{main-code}]| \||fi|
\end{center}

%%%%%%%%%%%%%%%%%%%%%%%%%%%%%%%%%%%%%%%%
\DescribeMacro{\childdocname}
\DescribeMacro{\childdocjob}
The macro |\childdocname| contains the filename (without extension)
of the main or child file being processed.
Note that |\childdocjob| will always contain the name of the main file.

%%%%%%%%%%%%%%%%%%%%%%%%%%%%%%%%%%%%%%%%
\paragraph{Title Page.}

Conditional processing can be used to include a title or banner page
in the main document when proper precautions are taken.
Importantly, the code in the main file should ensure that the page counter
(as well as other status parameters which are stored in the |.aux| files)
takes the same value after the conditional processing.
Otherwise the page numbers may take divergent values
depending on which part is compiled.

For example, a title page could be declared by:
%
\begin{center}
\begin{tabular}{l}
|\ifchilddoc\||else|\\
|\addtocounter{page}{-1}|\\
\textit{code for title page}\\
|\newpage|\\
|\||fi|
\end{tabular}
\end{center}
%
A banner page for the child documents can be generated by:
%
\begin{center}
\begin{tabular}{l}
|\ifchilddoc|\\
|\addtocounter{page}{-1}|\\
\textit{code for banner page}\\
|\newpage|\\
|\||fi|
\end{tabular}
\end{center}
%
Here one could write a message such as:
\begin{center}
|This is the part \childdocname{} of \childdocjob{}.|
\end{center}

%%%%%%%%%%%%%%%%%%%%%%%%%%%%%%%%%%%%%%%%%%%%%%%%%%%%%%%%%%%%%%%%%%%%%%%%%%%%%%%%
\subsection{Flags}
\label{sec:flags}

The package makes it easy to generate different versions
of the main or child documents.
To this end compilation flags can be defined
and assigned different default values.
They will be particularly useful in conjunction
with the forwarding mechanism described in \secref{sec:forward}.

For example, it may be useful to have a flag |\version|
which can be set to |draft| or |final|.
The document source will contain some conditional code
depending on the value of |\version|.
Suppose further, the flag should default to |final| for the main file
and to |draft| for child files
which is a natural assignment for editing the document.
This is achieved by placing the following code
in the preamble of the main document
(below the |\childdocmain| directive):
%
\begin{center}
\begin{tabular}{l}
|\ifchilddoc|\\
|\providecommand{\version}{draft}|\\
|\||else|\\
|\providecommand{\version}{final}|\\
|\||fi|
\end{tabular}
\end{center}
%
The definition by |\providecommand| makes sure
that previous definitions are not overwritten.
Further statements |\providecommand{\version}{...}|
can thus be added before the above code to override it.

For the main file, one might add a line
(between |\childdocmain| and the above block)
%
\begin{center}
|%\ifchilddoc\||else\providecommand{\version}{draft}\||fi|
\end{center}
%
which can be uncommented to produce a draft version.
Likewise one can add a line to the very top of a child file
(above the |\childdocof{|\textit{main}|}| directive)
%
\begin{center}
|%\providecommand{\version}{final}|
\end{center}
%
which can be uncommented to produce the final version of this child document.

%%%%%%%%%%%%%%%%%%%%%%%%%%%%%%%%%%%%%%%%%%%%%%%%%%%%%%%%%%%%%%%%%%%%%%%%%%%%%%%%
\subsection{Forwarding}
\label{sec:forward}

Different versions of the main or child documents
using compilation flags as described in \secref{sec:flags}
can be (permanently) stored in different files
for convenient compilation, viewing and distribution.
To this end, the package defines a command
to pass on compilation to a different file:

%%%%%%%%%%%%%%%%%%%%%%%%%%%%%%%%%%%%%%%%
\DescribeMacro{\childdocforward}
The command |\childdocforward| redirects processing to
another source file:
%
\begin{center}
\begin{tabular}{l}
|\input{childdoc.def}|\\
|\childdocforward[|\textit{main}|]{|\textit{dest}|}|\\
\end{tabular}
\end{center}
%
The argument \textit{dest} is the destination file
(without extension).
It should be the main file or one of the child files.
Note that further \textsf{childdoc} directives
such as |\childdocof| and |\childdocforward|
in the indicated file will be processed in this form.
The optional argument \textit{main}
passes on directly to the main file \textit{main}
while pretending to compile the child \textit{dest}.
This form behaves as if \textit{dest}
issues |\childdocof{|\textit{main}|}| right away,
and no further \textsf{childdoc} directives will be processed.

%%%%%%%%%%%%%%%%%%%%%%%%%%%%%%%%%%%%%%%%
\DescribeMacro{\...prefix}
In the alternative form |\childdocforwardprefix|,
%
\begin{center}
\begin{tabular}{l}
|\input{childdoc.def}|\\
|\childdocforwardprefix[|\textit{main}|]{|\textit{prefix}|}{|\textit{dest}|}|
\end{tabular}
\end{center}
%
the destination file is determined by a pattern
depending on the current file:
To make this work, the current file must be called
`{\textit{prefix}\hspace{0.2em}\textit{suffix}}'
with \textit{prefix} matching precisely the argument.
Processing is then passed on to the file
`{\textit{dest}\hspace{0.2em}\textit{suffix}}'.
Surely, the same effect is achieved by
directly specifying the
argument `{\textit{dest}\hspace{0.2em}\textit{suffix}}'
in the first form.
However, that requires to set up a different file
for each child. With the alternative form of the command
all these files can have exactly the same content
which simplifies setting them up and maintaining them.

For example, the following file |draft.tex|
with a compilation flag |\version| as described in \secref{sec:flags}
compiles the main document as a draft:
%
\begin{center}
\begin{tabular}{l}
|\def\version{draft}|\\
|\input{childdoc.def}|\\
|\childdocforward{|\textit{main}|}|
\end{tabular}
\end{center}
%
Likewise, the following files |final|\textit{nn}|.tex|
compile the final version of the child document
|child|\textit{nn}|.tex|:
%
\begin{center}
\begin{tabular}{l}
|\def\version{final}|\\
|\input{childdoc.def}|\\
|\childdocforwardprefix{final}{child}|
\end{tabular}
\end{center}
%

Note that when several versions of a main file and/or of each child file
are to be generated, it may be convenient to set up a |Makefile| or
shell script to automatise the process.

%%%%%%%%%%%%%%%%%%%%%%%%%%%%%%%%%%%%%%%%%%%%%%%%%%%%%%%%%%%%%%%%%%%%%%%%%%%%%%%%
\subsection{Command Line Processing}
\label{sec:commandline}

The effect of redirection files can also be achieved by invoking
the \LaTeX{} compiler with a more elaborate command line.
Most conveniently this should be done as part
of a shell script or a |Makefile|.

When using \textsf{childdoc} in the main file, the following
command lines effectively perform a redirection
(note that depending on the shell being used,
backslashes may have to be doubled: `|\|' $\to$ `|\\|'):
%
\begin{center}
|... -jobname "|\textit{target}|" |\\|"|[\textit{flags}]%
|\input{childdoc.def}\childdocforward[|\textit{main}|]{|\textit{dest}|}"|
\end{center}
%
Here \textit{target} is the name of the output file,
\textit{main} is the name of the main file
and \textit{dest} is the name of the main or child file to be processed
(all filenames without extensions).
The optional argument \textit{main} can be omitted
if \textit{main} matches \textit{dest}.
Optionally, compilation \textit{flags} can be defined via |\def| commands.
This command line makes the \TeX{} engine believe
it is compiling the file \textit{target}
whose content is specified as the latter parameter.
The provided code then forwards the processing to
\textit{main} or \textit{dest} as described in \secref{sec:forward}.

%%%%%%%%%%%%%%%%%%%%%%%%%%%%%%%%%%%%%%%%%%%%%%%%%%%%%%%%%%%%%%%%%%%%%%%%%%%%%%%%
\subsection{Include by Input}
\label{sec:input}

Including child documents by |\include| has some restrictions by design.
Most notably, the content of a child document always occupies
its own set of pages; pages cannot be shared between child documents.
Usually, this behaviour makes perfect sense
because each child document contain an essential part of the document.
However, in some situations it may be desirable to compose
a document from a collection of parts
without having mandatory page breaks between then.
For this case, the package
provides a mechanism to include parts
by |\input| which can also be processed individually.
However, by construction this mechanism
requires manual handling of the content to be output.

%%%%%%%%%%%%%%%%%%%%%%%%%%%%%%%%%%%%%%%%
\DescribeMacro{\ifchilddocmanual}
The main file should be prepared as usual, see \secref{sec:include}.
However, the document body must make a distinction
between processing of an individual part and of the main document, e.g.:
%
\begin{center}
\begin{tabular}{l}
|\ifchilddocmanual|\\
|\input{\childdocname}|\\
|\||else|\\
\textit{document body with }|\input{|\textit{part}|}|\\
|\||fi|
\end{tabular}
\end{center}
%
The conditional |\ifchilddocmanual| is true whenever
a part to be included by |\input| is being compiled,
and the name of the part is stored in |\childdocname|.

%%%%%%%%%%%%%%%%%%%%%%%%%%%%%%%%%%%%%%%%
\DescribeMacro{\childdocby}
Each part to be included by |\input| should start with:
%
\begin{center}
\begin{tabular}{l}
|\input{childdoc.def}|\\
|\childdocby{|\textit{main}|}|\\
\end{tabular}
\end{center}
%
The directive |\childdocby| is similar to |\childdocof|
described in \secref{sec:include},
but the subsequent selection of content must be done manually.
To that end, both |\ifchilddoc| and |\ifchilddocmanual|
will be true upon processing of a part,
and the name of the part is stored in |\childdocname|.
Note that |\jobname| will be set to the filename of the current part
so that each part receives an individual |.aux| file
that does not interfere with the |.aux| file(s) of the main document.
This behaviour can be altered by the alternative form
|\childdocby[*]{|\textit{main}|}| (with a non-empty optional argument)
which uses the |.aux| file of the main document
by setting |\jobname| to \textit{main}.

%%%%%%%%%%%%%%%%%%%%%%%%%%%%%%%%%%%%%%%%%%%%%%%%%%%%%%%%%%%%%%%%%%%%%%%%%%%%%%%%
\subsection{Driver Development}
\label{sec:driver}

The \textsf{childdoc} mechanism can also be use for the development
of definition files such as \LaTeX{} styles or classes.
This case differs from the above setup with multiple parts
included by |\include| in that no |\includeonly| should be invoked.
This can be achieved by starting the include file
(before |\ProvidesPackage|) with:
%
\begin{center}
\begin{tabular}{l}
|\input{childdoc.def}|\\
|\childdocforward{|\textit{main}|}|\\
\end{tabular}
\end{center}
%
or alternatively with:
%
\begin{center}
\begin{tabular}{l}
|\input{childdoc.def}|\\
|\childdocby{|\textit{main}|}|\\
\end{tabular}
\end{center}
%
Both forms have slightly different effects as described above.
The main file is prepared as usual, see \secref{sec:include}.

%%%%%%%%%%%%%%%%%%%%%%%%%%%%%%%%%%%%%%%%%%%%%%%%%%%%%%%%%%%%%%%%%%%%%%%%%%%%%%%%
\subsection{Legacy Detection}
\label{sec:detection}

The directive |\childdocmain| in the main file can detect
whether the complete document or merely a child is to be compiled
even without using the directive |\childdocof|.
This method is deprecated because it is less robust
and there is no compelling reason to use it;
it is merely provided for backward compatibility
and it may be removed in future versions.

If the detection mechanism is to be used,
it is mandatory to correctly specify
the filename of the main file as the argument of |\childdocmain|:
%
\begin{center}
\begin{tabular}{l}
|\input{childdoc.def}|\\
|\childdocmain{|\textit{main}|}|\\
\end{tabular}
\end{center}
%
If |\jobname| does not match the argument \textit{main} of |\childdocmain|,
it is assumed that |\jobname| points to the child file to be compiled.
When using |\childdocmain| with the main file specified as argument,
it suffices to start a child file
with just |\input{|\textit{main}|}|
without loading of the package and using |\childdocof|.
If instead all processing is done
with the appropriate \textsf{childdoc} directives,
the argument of \textit{main} of |\childdocmain| can be empty.

An alternative version of the command line processing described
in \secref{sec:commandline} using the detection mechanism reads:
%
\begin{center}
|... -jobname "|\textit{target}|" "|[\textit{flags}]%
[|\def\jobname{|\textit{dest}|}|]|\input{|\textit{main}|}"|
\end{center}

%%%%%%%%%%%%%%%%%%%%%%%%%%%%%%%%%%%%%%%%%%%%%%%%%%%%%%%%%%%%%%%%%%%%%%%%%%%%%%%%
\subsection{Manual Code}
\label{sec:manual}

In case one cannot be certain whether the definitions file |childdoc.def|
is installed on the target \TeX{} distribution
and one prefers not to ship it,
it is conceivable to paste a few relevant commands into the sources.

To that end, drop all statements |\input{childdoc.def}|
and perform the replacements as outlined below.
Instead of |\childdocmain{|\textit{main}|}| add the following code
to the top of the main file:
%
\begin{center}
\begin{tabular}{l}
|\||ifdefined\childdocname\endinput\||fi\newif\ifchilddoc|\\
|\edef\childdocname{\scantokens\expandafter{\jobname\noexpand}}|\\
|\def\childdocmain{|\textit{main}|}\||ifx\childdocmain\childdocname\||else|\\
|\childdoctrue\includeonly{\childdocname}\let\jobname\childdocmain\||fi|\\
\end{tabular}
\end{center}
%
Instead of |\childdocof{|\textit{main}|}| just include the main file
at the top of each child file:
%
\begin{center}
|\input{|\textit{main}|}|
\end{center}
%
A simple redirection |\childdocforward{|\textit{dest}|}| is achieved by:
%
\begin{center}
|\def\jobname{|\textit{dest}|}\input{\jobname}|
\end{center}
%
The redirection with prefix
|\childdocforwardprefix[|\textit{prefix}|]{|\textit{dest}|}|
is accomplished by:
%
\begin{center}
\begin{tabular}{l}
|{\edef\jobname{\scantokens\expandafter{\jobname\noexpand}}|\\
|\def\redirectjob |\textit{prefix}|#1~~~{\gdef\jobname{|\textit{dest}|#1}}|\\
|\expandafter\redirectjob\jobname~~~}\input{\jobname}|
\end{tabular}
\end{center}

In an alternative approach,
child documents can be compiled by a specific command line
without additional code or specific definitions:
%
\begin{center}
|... -jobname "|\textit{target}|" "|[\textit{flags}]%
|\includeonly{|\textit{dest}|}\input{|\textit{main}|}"|
\end{center}
%

%%%%%%%%%%%%%%%%%%%%%%%%%%%%%%%%%%%%%%%%%%%%%%%%%%%%%%%%%%%%%%%%%%%%%%%%%%%%%%%%
%%%%%%%%%%%%%%%%%%%%%%%%%%%%%%%%%%%%%%%%%%%%%%%%%%%%%%%%%%%%%%%%%%%%%%%%%%%%%%%%
\section{Information}

%%%%%%%%%%%%%%%%%%%%%%%%%%%%%%%%%%%%%%%%%%%%%%%%%%%%%%%%%%%%%%%%%%%%%%%%%%%%%%%%
\subsection{Copyright}

Copyright \copyright{} 2017--2018 Niklas Beisert

This work may be distributed and/or modified under the
conditions of the \LaTeX{} Project Public License, either version 1.3
of this license or (at your option) any later version.
The latest version of this license is in
  \url{http://www.latex-project.org/lppl.txt}
and version 1.3 or later is part of all distributions of \LaTeX{}
version 2005/12/01 or later.

This work has the LPPL maintenance status `maintained'.

The Current Maintainer of this work is Niklas Beisert.

This work consists of the files |README.txt|, |childdoc.ins| and |childdoc.dtx|
as well as the derived files |childdoc.def|, |cdocsamp.tex|
with |cdocsch1.tex|, |cdocsch2.tex|, |cdocspt3.tex|, |cdocspt4.tex|,
|cdocsdrf.tex|, |cdocsfn1.tex|, |cdocsfn2.tex|
as well as |childdoc.pdf|.

%%%%%%%%%%%%%%%%%%%%%%%%%%%%%%%%%%%%%%%%%%%%%%%%%%%%%%%%%%%%%%%%%%%%%%%%%%%%%%%%
\subsection{Files and Installation}

The package consists of the files:
%
\begin{center}
\begin{tabular}{ll}
    |README.txt|   & readme file \\
    |childdoc.ins| & installation file \\
    |childdoc.dtx| & source file \\
    |childdoc.def| & definition file \\
    |cdocsamp.tex| & sample main file \\
    |cdocsch1.tex| & sample include file \\
    |cdocsch2.tex| & sample include file \\
    |cdocspt3.tex| & sample part file \\
    |cdocspt4.tex| & sample part file \\
    |cdocsdrf.tex| & sample redirection file \\
    |cdocsfn1.tex| & sample redirection file \\
    |cdocsfn2.tex| & sample redirection file \\
    |childdoc.pdf| & manual
\end{tabular}
\end{center}
%
The distribution consists of the files
|README.txt|, |childdoc.ins| and |childdoc.dtx|.
%
\begin{itemize}
\item
Run (pdf)\LaTeX{} on |childdoc.dtx|
to compile the manual |childdoc.pdf| (this file).
\item
Run \LaTeX{} on |childdoc.ins| to create the definitions file |childdoc.def|
and the sample |cdocsamp.tex| with include files
|cdocsch1.tex|, |cdocsch2.tex|, |cdocspt3.tex|, |cdocspt4.tex|,
|cdocsdrf.tex|, |cdocsfn1.tex|, |cdocsfn2.tex|.
Then copy the file |childdoc.def| to an appropriate directory of your \LaTeX{}
distribution, e.g.\ \textit{texmf-root}|/tex/latex/childdoc|.
\end{itemize}

%%%%%%%%%%%%%%%%%%%%%%%%%%%%%%%%%%%%%%%%%%%%%%%%%%%%%%%%%%%%%%%%%%%%%%%%%%%%%%%%
\subsection{Related CTAN Packages}

There are several other packages which offer a similar functionality:
%
\begin{itemize}
\item
The packages
\href{http://ctan.org/pkg/docmute}{\textsf{docmute}},
\href{http://ctan.org/pkg/includex}{\textsf{includex}} and
\href{http://ctan.org/pkg/standalone}{\textsf{standalone}}
provide commands to include only the document body of
a child file thus allowing both files to be compiled individually.
\item
The packages \href{http://ctan.org/pkg/subdocs}{\textsf{subdocs}}
and \href{http://ctan.org/pkg/subfiles}{\textsf{subfiles}}
provide structures in which the main and child documents can be
encapsulated and allowing them to be compiled individually.
The inclusion mechanism is different from the conventional |\include|.
\item
The package \href{http://ctan.org/pkg/combine}{\textsf{combine}}
is an elaborate solution to combine several documents into one.
\end{itemize}
%
See also the CTAN topic \href{http://ctan.org/topic/subdocs}{\textsf{subdocs}}
for further related packages.
The present package differs from the above solutions in that
a document structure constructed with the conventional |\include| mechanism
just needs two extra commands at the top of every file
such that all constituent files can be compiled individually.

%%%%%%%%%%%%%%%%%%%%%%%%%%%%%%%%%%%%%%%%%%%%%%%%%%%%%%%%%%%%%%%%%%%%%%%%%%%%%%%%
%\subsection{Feature Suggestions}
%
%The following is a list of features which may be useful for future
%versions of this package:
%%
%\begin{itemize}
%\item
%\ldots
%\end{itemize}

%%%%%%%%%%%%%%%%%%%%%%%%%%%%%%%%%%%%%%%%%%%%%%%%%%%%%%%%%%%%%%%%%%%%%%%%%%%%%%%%
\subsection{Revision History}

%%%%%%%%%%%%%%%%%%%%%%%%%%%%%%%%%%%%%%%%
\paragraph{v2.0:} 2018/12/30

\begin{itemize}
\item
immediate forward processing
\item
added |\childdocby| mechanism
\item
manual restructured
\end{itemize}

%%%%%%%%%%%%%%%%%%%%%%%%%%%%%%%%%%%%%%%%
\paragraph{v1.6:} 2018/01/17

\begin{itemize}
\item
application for development of include files
\item
corrections to manual
\end{itemize}

%%%%%%%%%%%%%%%%%%%%%%%%%%%%%%%%%%%%%%%%
\paragraph{v1.5:} 2017/05/21

\begin{itemize}
\item
more complete structuring introduced
\item
|\childdocof| introduced
\item
|\childdoc| renamed to |\childdocmain|
\item
|\childredirect| renamed to |\childdocforward| and |\childdocforwardprefix|
and functionality expanded
\end{itemize}

%%%%%%%%%%%%%%%%%%%%%%%%%%%%%%%%%%%%%%%%
\paragraph{v1.0:} 2017/04/27

\begin{itemize}
\item
manual and install package
\item
first version published on CTAN
\end{itemize}

%%%%%%%%%%%%%%%%%%%%%%%%%%%%%%%%%%%%%%%%
\paragraph{v0.6:} 2017/04/26

\begin{itemize}
\item
redirection mechanism added
\end{itemize}

%%%%%%%%%%%%%%%%%%%%%%%%%%%%%%%%%%%%%%%%
\paragraph{v0.5:} 2017/04/26

\begin{itemize}
\item
functionality in definition file
\end{itemize}


%%%%%%%%%%%%%%%%%%%%%%%%%%%%%%%%%%%%%%%%%%%%%%%%%%%%%%%%%%%%%%%%%%%%%%%%%%%%%%%%
%%%%%%%%%%%%%%%%%%%%%%%%%%%%%%%%%%%%%%%%%%%%%%%%%%%%%%%%%%%%%%%%%%%%%%%%%%%%%%%%
%%%%%%%%%%%%%%%%%%%%%%%%%%%%%%%%%%%%%%%%%%%%%%%%%%%%%%%%%%%%%%%%%%%%%%%%%%%%%%%%
\appendix

\settowidth\MacroIndent{\rmfamily\scriptsize 000\ }

 \DocInput{childdoc.dtx}

\end{document}
%</driver>
% \fi
%
% %%%%%%%%%%%%%%%%%%%%%%%%%%%%%%%%%%%%%%%%%%%%%%%%%%%%%%%%%%%%%%%%%%%%%%%%%%%%%%
% %%%%%%%%%%%%%%%%%%%%%%%%%%%%%%%%%%%%%%%%%%%%%%%%%%%%%%%%%%%%%%%%%%%%%%%%%%%%%%
% \section{Sample}
%\iffalse
%<*samplemain>
%\fi
%
% The following presents a sample document
% with two chapters, two parts, a title page,
% a compile flag as well as three forwarding files to set the flag.
% It consists of eight |.tex| files:
% \begin{center}
% \begin{tabular}{ll}
% |cdocsamp.tex|&main file\\
% |cdocsch1.tex|&include file for chapter 1\\
% |cdocsch2.tex|&include file for chapter 2\\
% |cdocspt3.tex|&include file for part 3\\
% |cdocspt4.tex|&include file for part 4\\
% |cdocsdrf.tex|&forwarding file for main file in draft mode\\
% |cdocsfi1.tex|&forwarding file for final version of chapter 1\\
% |cdocsfi2.tex|&forwarding file for final version of chapter 2\\
% \end{tabular}
% \end{center}
% Each of the eight files can be compiled directly by the \LaTeX{} compiler.
%
% %%%%%%%%%%%%%%%%%%%%%%%%%%%%%%%%%%%%%%
% \paragraph{Main File.}
%
% The main file is called |cdocsamp.tex|.
%
% Load the \textsf{childdoc} definitions and
% declare the filename for the main document:
%    \begin{macrocode}
\input{childdoc.def}
\childdocmain{}
%    \end{macrocode}

% Optional override for |\version| flag:
%    \begin{macrocode}
%%\ifchilddoc\else\providecommand{\version}{draft}\fi
%    \end{macrocode}

% Define the default values for the |\version| flag
% (|final| for the main file and |draft| for childs):
%    \begin{macrocode}
\ifchilddoc
\providecommand{\version}{draft}
\else
\providecommand{\version}{final}
\fi
%    \end{macrocode}

% Load the standard document class:
%    \begin{macrocode}
\documentclass[12pt]{article}
%    \end{macrocode}

% Start the document body:
%    \begin{macrocode}
\begin{document}
%    \end{macrocode}

% Declare a title page.
% Print title, part of document being processed and version flag:
%    \begin{macrocode}
\addtocounter{page}{-1}
\begin{center}
{\LARGE\bfseries{}childdoc example\par}
\vspace{1cm}
\ifchilddoc
\ifchilddocmanual part\else chapter\fi:
`\childdocname' of `\childdocjob'\par
\else
main document: `\childdocjob'\par
\fi
version: \version\par
\end{center}
\newpage
%    \end{macrocode}

% Manually include selected file,
% otherwise process as usual:
%    \begin{macrocode}
\ifchilddocmanual
\section*{part `\childdocname'}
\input{\childdocname}
\else
%    \end{macrocode}

% Include the two chapters:
%    \begin{macrocode}
\include{cdocsch1}
\include{cdocsch2}
%    \end{macrocode}

% Include the two parts unless only chapters should be displayed:
%    \begin{macrocode}
\ifchilddoc\else
\section{part three}
\input{cdocspt3}
\section{part four}
\input{cdocspt4}
\fi
%    \end{macrocode}

% Process as usual until here:
%    \begin{macrocode}
\fi
%    \end{macrocode}

% End of document body:
%    \begin{macrocode}
\end{document}
%    \end{macrocode}
%\iffalse
%</samplemain>
%\fi
%
% %%%%%%%%%%%%%%%%%%%%%%%%%%%%%%%%%%%%%%
% \paragraph{Chapter Include Files.}
%
% The include files are called |cdocsch1.tex| and |cdocsch2.tex|.
%
%\iffalse
%<*samplechap1|samplechap2>
%\fi

% Optional override for |\version| flag:
%    \begin{macrocode}
%%\providecommand{\version}{final}
%    \end{macrocode}

% Include the main document:
%    \begin{macrocode}
\input{childdoc.def}
\childdocof{cdocsamp}
%    \end{macrocode}

%\iffalse
%</samplechap1|samplechap2>
%\fi
%
%\iffalse
%<*samplechap1>
%\fi
% Some text for chapter 1:
%    \begin{macrocode}
\section{one}
some text in chapter one
%    \end{macrocode}

%\iffalse
%</samplechap1>
%\fi
% Some text for chapter 2:
%\iffalse
%<*samplechap2>
%\fi
%    \begin{macrocode}
\section{two}
more text in chapter two
%    \end{macrocode}

%\iffalse
%</samplechap2>
%\fi
%
% %%%%%%%%%%%%%%%%%%%%%%%%%%%%%%%%%%%%%%
% \paragraph{Part Include Files.}
%
% The include files are called |cdocspt3.tex| and |cdocspt4.tex|.
%
%\iffalse
%<*samplepart3|samplepart4>
%\fi

% Optional override for |\version| flag:
%    \begin{macrocode}
%%\providecommand{\version}{final}
%    \end{macrocode}

% Include the main document:
%    \begin{macrocode}
\input{childdoc.def}
\childdocby{cdocsamp}
%    \end{macrocode}

%\iffalse
%</samplepart3|samplepart4>
%\fi
%
%\iffalse
%<*samplepart3>
%\fi
% Some text for part 3:
%    \begin{macrocode}
some text in part three
%    \end{macrocode}

%\iffalse
%</samplepart3>
%\fi
% Some text for part 4:
%\iffalse
%<*samplepart4>
%\fi
%    \begin{macrocode}
more text in part four
%    \end{macrocode}

%\iffalse
%</samplepart4>
%\fi
%
% %%%%%%%%%%%%%%%%%%%%%%%%%%%%%%%%%%%%%%
% \paragraph{Forwarding for a Complete Draft.}
%
% The following forwarding file |cdocsdrf.tex|
% compiles the main document in draft mode:
%\iffalse
%<*sampledraft>
%\fi
%    \begin{macrocode}
\def\version{draft}
\input{childdoc.def}
\childdocforward{cdocsamp}
%    \end{macrocode}

%\iffalse
%</sampledraft>
%\fi
%
% %%%%%%%%%%%%%%%%%%%%%%%%%%%%%%%%%%%%%%
% \paragraph{Forwarding for Final Version of the Chapters.}
%
% The following forwarding files |cdocsfn1.tex| and |cdocsfn2.tex|
% (with identical content)
% compile the final versions of the child documents
% |cdocsch1.tex| and |cdocsch2.tex|, respectively:
%\iffalse
%<*samplefinal>
%\fi
%    \begin{macrocode}
\def\version{final}
\input{childdoc.def}
\childdocforwardprefix[cdocsamp]{cdocsfn}{cdocsch}
%    \end{macrocode}

%\iffalse
%</samplefinal>
%\fi
%
% %%%%%%%%%%%%%%%%%%%%%%%%%%%%%%%%%%%%%%
% \paragraph{Command Line Processing.}
%
% The following three command lines generate the output files
% |cdocscld|, |cdocscl1| and |cdocscl2|
% which should be identical to
% |cdocsdrf|, |cdocsch1| and |cdocsfn2|, respectively:
% \begin{center}
% \begin{tabular}{l}
% |latex -jobname cdocscld \|\\
% |  "\def\version{draft}\input{childdoc.def}\childdocforward{cdocsamp}"|\\
% |latex -jobname cdocscl1 \|\\
% |  "\input{childdoc.def}\childdocforward[cdocsamp]{cdocsch1}"|\\
% |latex -jobname cdocscl2 \|\\
% |  "\def\version{final}\input{childdoc.def}\childdocforward{cdocsch2}"|
% \end{tabular}
% \end{center}
% Note that the trailing backslash on each first line
% merely continues the input to the second line
% (for convenient cut ant paste).
% Furthermore, the command |latex| can be replaced by any
% of its alternative versions such as |pdflatex|.
%
% %%%%%%%%%%%%%%%%%%%%%%%%%%%%%%%%%%%%%%%%%%%%%%%%%%%%%%%%%%%%%%%%%%%%%%%%%%%%%%
% %%%%%%%%%%%%%%%%%%%%%%%%%%%%%%%%%%%%%%%%%%%%%%%%%%%%%%%%%%%%%%%%%%%%%%%%%%%%%%
% \section{Implementation}
%\iffalse
%<*package>
%\fi
%
% This section describes the definitions file |childdoc.def|.

% The definitions cannot be loaded using |\usepackage| or |\RequirePackage|
% which has a mechanism to prevent loading a style file more than once.
% When loading the definitions by means of |\input|
% multiple instances have to be prevented manually:
%\iffalse
%This code needs to be before the `\ProvidesFile' directive
%which is defined at the beginning of this file.
%Therefore it is also placed there and commented out here.
%</package>
%<*discard>
%\fi
%    \begin{macrocode}
\ifdefined\childdocmain\endinput\fi
%    \end{macrocode}
%\iffalse
%</discard>
%<*package>
%\fi
%
% \macro{\ifchilddoc}
% \macro{\ifchilddocmanual}
% The conditional |\ifchilddoc| tells whether a
% child (true) or main (false) document is being compiled.
% The conditional |\ifchilddocmanual| tells whether
% the |\includeonly| mechanism is used (false) or
% the selection of child files must be performed manually (true).
% The definitions initialise to false:
%    \begin{macrocode}
\newif\ifchilddoc
\newif\ifchilddocmanual
%    \end{macrocode}

% \macro{\childdocname}
% \macro{\childdocjob}
% The macro |\childdocname| stores the name of the main document
% to be compiled. The macro |\childdocjob| stores the name of
% the document on which the \LaTeX{} compiler was originally invoked.
% The content of |\jobname| cannot be compared
% to filenames specified in the source due to different catcodes.
% The following code rescans |\jobname|, stores the result
% in |\childdocname| and saves a copy in |\childdocjob|:
%    \begin{macrocode}
\edef\childdocname{\scantokens\expandafter{\jobname\noexpand}}
\let\childdocjob\childdocname
%    \end{macrocode}

% \macro{\childdocdisable}
% The macro |\childdocdisable| prevents the main file
% from being processed more than once.
% At this stage, the main document command |\childdocmain|
% is assumed to be called once again where it should do nothing.
% Any subsequent call to it should prevent
% a secondary processing of the main document
% It overwrites the forwarding commands
% |\childdocof| and |\childdocforward|
% with empty macros to prevent further inclusions of the main document:
%    \begin{macrocode}
\newcommand{\childdocdisable}
{
  \renewcommand{\childdocmain}[1]{\renewcommand{\childdocmain}[1]{\endinput}}
  \renewcommand{\childdocof}[1]{}
  \renewcommand{\childdocby}[2][]{}
  \renewcommand{\childdocforward}[2][]{}
  \renewcommand{\childdocdisable}{}
}
%    \end{macrocode}

% \macro{\childdocmain}
% The macro |\childdocmain| is to be called at the top of the main file
% with nothing or the main filename (without extension) as argument.
% First, it breaks loops.
% If the argument is not empty and does not match |\childdocname|
% (which is set by the first inclusion of |childdoc.def|),
% |\ifchilddoc| is set to true, |\includeonly| is applied to the child file
% and |\jobname| is set to the main file
% (for proper handling of |.aux| files):
%    \begin{macrocode}
\newcommand{\childdocmain}[1]
{
  \childdocdisable\childdocmain{}
  \if?#1?\else
    \begingroup
      \def\childdoctmp{#1}
      \ifx\childdoctmp\childdocname
        \def\childdoctmp{}
      \else
        \def\childdoctmp
        {
          \childdoctrue
          \includeonly{\childdocname}
          \def\childdocjob{#1}
          \def\jobname{#1}
        }
      \fi
      \expandafter
    \endgroup
    \childdoctmp
  \fi
}
%    \end{macrocode}

% \macro{\childdocof}
% The command |\childdocof| redirects
% compilation to the main file |#1|.
%    \begin{macrocode}
\newcommand{\childdocof}[1]
{
  \childdocdisable
  \childdoctrue
  \includeonly{\childdocname}
  \def\jobname{#1}
  \def\childdocjob{#1}
  \input{#1}
}
%    \end{macrocode}

% \macro{\childdocby}
% The command |\childdocby| ....
%    \begin{macrocode}
\newcommand{\childdocby}[2][]
{
  \childdocdisable
  \childdoctrue
  \childdocmanualtrue
  \if?#1?\else
    \def\jobname{#2}
  \fi
  \def\childdocjob{#2}
  \input{#2}
  \endinput
}
%    \end{macrocode}

% \macro{\childdocforward}
% The command |\childdocforward| redirects
% compilation to the main file or
% (if the optional argument is given) a child file.
% Parameters are set as if the main file
% or a child file starting with |\childdocof| was compiled.
% Then compilation is handed over to the main file:
%    \begin{macrocode}
\newcommand{\childdocforward}[2][]
{
  \begingroup
    \if?#1?
      \def\childdoctmp
      {
        \def\childdocname{#2}
        \def\childdocjob{#2}
        \def\jobname{#2}
        \input{#2}
        \endinput
      }
    \else
      \def\childdoctmp
      {
        \childdocdisable
        \def\childdocname{#2}
        \childdoctrue
        \includeonly{#2}
        \def\childdocjob{#1}
        \def\jobname{#1}
        \input{#1}
        \endinput
      }
    \fi
    \expandafter
  \endgroup
  \childdoctmp
}
%    \end{macrocode}

% \macro{\childdocforwardprefix}
% The command |\childdocforwardprefix| redirects
% compilation to the main or a child file by means of a pattern.
% The prefix |#1| in the current filename is replaced by |#2|
% and the suffix of the current filename is kept
% (it is assumed that the filename does not contain the substring `|~~~|'
% which is used as a delimiter).
% Compilation is handed over to the new file by |\childdocforward|:
%    \begin{macrocode}
\newcommand{\childdocforwardprefix}[3][]
{
  \begingroup
    \def\childdocextract #2##1~~~{\def\childdoctmp{\childdocforward[#1]{#3##1}}}
    \expandafter\childdocextract\childdocname~~~
    \expandafter
  \endgroup
  \childdoctmp
}
%    \end{macrocode}

% \macro{\childdoc}
% The deprecated macro |\childdoc| is a legacy version of |\childdocmain|:
%    \begin{macrocode}
\newcommand{\childdoc}{\childdocmain}
%    \end{macrocode}

% \macro{\childdocredirect}
% The deprecated macro |\childdocredirect| is a legacy version
% of |\childdocforward| and |\childdocforwardprefix|:
%    \begin{macrocode}
\newcommand{\childdocredirect}[2][]
{
  \begingroup
    \if?#1?
      \def\childdoctmp{\childdocforward{#2}}
    \else
      \def\childdoctmp{\childdocforwardprefix{#1}{#2}}
    \fi
    \expandafter
  \endgroup
  \childdoctmp
}
%    \end{macrocode}

%\iffalse
%</package>
%\fi
%
\endinput
|\\
|\childdocby{|\textit{main}|}|\\
\end{tabular}
\end{center}
%
Both forms have slightly different effects as described above.
The main file is prepared as usual, see \secref{sec:include}.

%%%%%%%%%%%%%%%%%%%%%%%%%%%%%%%%%%%%%%%%%%%%%%%%%%%%%%%%%%%%%%%%%%%%%%%%%%%%%%%%
\subsection{Legacy Detection}
\label{sec:detection}

The directive |\childdocmain| in the main file can detect
whether the complete document or merely a child is to be compiled
even without using the directive |\childdocof|.
This method is deprecated because it is less robust
and there is no compelling reason to use it;
it is merely provided for backward compatibility
and it may be removed in future versions.

If the detection mechanism is to be used,
it is mandatory to correctly specify
the filename of the main file as the argument of |\childdocmain|:
%
\begin{center}
\begin{tabular}{l}
|% \iffalse
%
% childdoc.dtx Copyright (C) 2017-2018 Niklas Beisert
%
% This work may be distributed and/or modified under the
% conditions of the LaTeX Project Public License, either version 1.3
% of this license or (at your option) any later version.
% The latest version of this license is in
%   http://www.latex-project.org/lppl.txt
% and version 1.3 or later is part of all distributions of LaTeX
% version 2005/12/01 or later.
%
% This work has the LPPL maintenance status `maintained'.
%
% The Current Maintainer of this work is Niklas Beisert.
%
% This work consists of the files childdoc.dtx and childdoc.ins
% and the derived files childdoc.def and cdocsamp.tex with
% cdocsch1.tex, cdocsch2.tex, cdocsdrf.tex, cdocsfn1.tex, cdocsfn2.tex.
%
%<package>\ifdefined\childdocmain\endinput\fi
%<package>\ProvidesFile{childdoc.def}[2018/12/30 v2.0 child document driver]
%<samplemain>\ProvidesFile{cdocsamp.tex}[2018/12/30 v2.0 sample for childdoc]
%<*driver>
%\ProvidesFile{childdoc.drv}[2018/12/30 v2.0 childdoc reference manual file]
\PassOptionsToClass{10pt,a4paper}{article}
\documentclass{ltxdoc}

\usepackage[margin=35mm]{geometry}
\usepackage{hyperref}
\usepackage{hyperxmp}
\usepackage[usenames]{color}

\hypersetup{colorlinks=true}
\hypersetup{pdfstartview=FitH}
\hypersetup{pdfpagemode=UseNone}
\hypersetup{pdfsource={}}
\hypersetup{pdflang={en-UK}}
\hypersetup{pdfcopyright={Copyright 2017-2018 Niklas Beisert.
  This work may be distributed and/or modified under the
  conditions of the LaTeX Project Public License, either version 1.3
  of this license or (at your option) any later version.}}
\hypersetup{pdflicenseurl={http://www.latex-project.org/lppl.txt}}
\hypersetup{pdfcontactaddress={ETH Zurich, ITP, HIT K,
  Wolfgang-Pauli-Strasse 27}}
\hypersetup{pdfcontactpostcode={8093}}
\hypersetup{pdfcontactcity={Zurich}}
\hypersetup{pdfcontactcountry={Switzerland}}
\hypersetup{pdfcontactemail={nbeisert@itp.phys.ethz.ch}}
\hypersetup{pdfcontacturl={http://people.phys.ethz.ch/\xmptilde nbeisert/}}

\newcommand{\secref}[1]{\hyperref[#1]{section \ref*{#1}}}

\parskip1ex
\parindent0pt
\let\olditemize\itemize
\def\itemize{\olditemize\parskip0pt}

\begin{document}

\title{The \textsf{childdoc} Package}
\hypersetup{pdftitle={The childdoc Package}}
\author{Niklas Beisert\\[2ex]
  Institut f\"ur Theoretische Physik\\
  Eidgen\"ossische Technische Hochschule Z\"urich\\
  Wolfgang-Pauli-Strasse 27, 8093 Z\"urich, Switzerland\\[1ex]
  \href{mailto:nbeisert@itp.phys.ethz.ch}
  {\texttt{nbeisert@itp.phys.ethz.ch}}}
\hypersetup{pdfauthor={Niklas Beisert}}
\hypersetup{pdfsubject={Manual for the LaTeX2e Package childdoc}}
\date{30 December 2018, \textsf{v2.0}}
\maketitle

\begin{abstract}\noindent
\textsf{childdoc} is a \LaTeXe{} package
that enables the direct compilation
of document sections included by |\include|
to individual files.
\end{abstract}

\begingroup
\parskip0ex
\tableofcontents
\endgroup

%%%%%%%%%%%%%%%%%%%%%%%%%%%%%%%%%%%%%%%%%%%%%%%%%%%%%%%%%%%%%%%%%%%%%%%%%%%%%%%%
%%%%%%%%%%%%%%%%%%%%%%%%%%%%%%%%%%%%%%%%%%%%%%%%%%%%%%%%%%%%%%%%%%%%%%%%%%%%%%%%
\section{Introduction}

\LaTeX{} provides a mechanism to structure a large document (such as a book)
into a main file and several child files (containing the chapters)
using the |\include| command.
This mechanism is beneficial for documents
which span hundreds of pages in order to
make the source file(s) more manageable.
Moreover, compilation can be restricted to
selected child files by means of the |\includeonly| command.
The latter feature can be used to reduce the compilation time while editing
(this was significantly more useful in the earlier days of \LaTeX{})
or to generate a smaller document which is easier to navigate.
Another application of |\includeonly| is to generate
documents consisting of selected parts of the complete document.

However, there are a few drawbacks of the plain |\include| mechanism:
\begin{itemize}
\item
The child files cannot be compiled on their own,
they can only be compiled via the main file.
A naive editing environment
(such as a text editor with an option
to have the current file processed by \LaTeX)
may require one to switch to the main file before compiling;
attempting to compile the child file produces errors.
\item
The main file must be modified (each time)
to adjust the |\includeonly| command
to the present needs. This easily leaves the main file in a messy state.
\item
The generated document will always carry the filename
of the main document. This is inconvenient if
several child files are to be compiled and
to be kept for distribution.
\end{itemize}

The present package provides a simple interface
to make child files individually compilable by \LaTeX{}.
Compiling a child file then has the same effect as compiling
the main file with an |\includeonly| command
to select the appropriate child.
Moreover the generated document will carry the name of the child
rather than the main file.
This resolves all three above issues.

This feature is meant to make the editing of books,
thesis documents and lecture notes somewhat more convenient.
However, the package can also be used efficiently for
composing a series of documents (such as exercise sheets)
which are typically distributed individually.
It then assists the author in generating the individual documents
(potentially in different versions)
as well as a document containing the collected series.
Another application is in developing style files
or other kinds of included material
where compilation of the style file could redirect
to a sample or test file.

%%%%%%%%%%%%%%%%%%%%%%%%%%%%%%%%%%%%%%%%%%%%%%%%%%%%%%%%%%%%%%%%%%%%%%%%%%%%%%%%
%%%%%%%%%%%%%%%%%%%%%%%%%%%%%%%%%%%%%%%%%%%%%%%%%%%%%%%%%%%%%%%%%%%%%%%%%%%%%%%%
\section{Usage}

First of all, the package \textsf{childdoc} is \emph{not} a standard
\LaTeXe{} |.sty| style file! Therefore it needs to be invoked in
a non-standard way.

%%%%%%%%%%%%%%%%%%%%%%%%%%%%%%%%%%%%%%%%%%%%%%%%%%%%%%%%%%%%%%%%%%%%%%%%%%%%%%%%
\subsection{Included Files}
\label{sec:include}

%%%%%%%%%%%%%%%%%%%%%%%%%%%%%%%%%%%%%%%%
\DescribeMacro{\childdocmain}
To use the package, add the commands
\begin{center}
\begin{tabular}{l}
|\input{childdoc.def}|\\
|\childdocmain{}|\\
\end{tabular}
\end{center}
at the very top of the main \LaTeX{} file,
in particular \emph{before} the |\documentclass| statement!
The argument of |\childdocmain| should be left empty
(but it must be present).

%%%%%%%%%%%%%%%%%%%%%%%%%%%%%%%%%%%%%%%%
\DescribeMacro{\childdocof}
Furthermore, add the commands
\begin{center}
\begin{tabular}{l}
|\input{childdoc.def}|\\
|\childdocof{|\textit{main}|}|\\
\end{tabular}
\end{center}
at the top of every child file \textit{child}
which is included by |\include{|\textit{child}|}|
from within the main file
(or at least for those files to be compiled individually).
The argument \textit{main} must be the filename of the main file.

There are a couple of
considerations in setting up the main and child documents:

%%%%%%%%%%%%%%%%%%%%%%%%%%%%%%%%%%%%%%%%
\paragraph{Restrictions.}

Please note the following restrictions:
\begin{itemize}
\item
|\childdocmain| must be called with one argument \textit{main}
to ensure compatibility with earlier version of the package.
It must either be empty (|\childdocmain{}|)
or precisely match the filename of the main file in which it is specified.
See \secref{sec:detection} for further information.
\item
The filename \textit{main} must be specified without the |.tex| extension.
\item
The filename \textit{main} is case sensitive
(even in case-insensitive file systems)
due to internal string comparison.
\item
The argument \textit{main} should be fully expanded, it cannot be a macro.
\item
Subdirectories and special characters should be avoided in filenames.
\item
The command |\childdocmain{|\textit{main}|}| must be followed by a whitespace.
It should not be followed immediately by another command
or by a comment mark `|%|'.
This is because the \TeX{} parser reads the token immediately following
the argument of |\childdocmain| and puts it
at the beginning of every child section;
however, a white\-space is ignored.
\end{itemize}

%%%%%%%%%%%%%%%%%%%%%%%%%%%%%%%%%%%%%%%%
\paragraph{Content of Main File.}

It is advisable to place all content in the child files included by |\include|.
Any output contained in the main file will appear in all child documents
unless suppressed manually;
it cannot be suppressed automatically by the |\includeonly| directive
and thus should normally be avoided.
A method to include some content in the main file
by means of conditional processing is described in \secref{sec:conditional}.

%%%%%%%%%%%%%%%%%%%%%%%%%%%%%%%%%%%%%%%%
\paragraph{Page Numbering.}

When only a part of the document is compiled,
the appropriate numbering of pages
(as well as other status parameters)
is determined from the |.aux| files.
The latter contain information from previous passes.
However this information needs to propagate through
all intermediate child documents.
Therefore the page numbering in child documents may well
be inconsistent until the complete document is compiled at least once.

A useful (if unconventional) way to always ensure a consistent
page numbering is to restart the numbering in each child document
and denote the pages by `\textit{child}|.|\textit{page}'
where \textit{child} represents the chapter/section number of the child file.
This can be achieved by the command
|\numberwithin{page}{|\textit{child}|}|
of the \textsf{amsmath} package
where \textit{child} can be |chapter| or |section|
depending on the chosen structuring.
Alternatively, one can modify the macro |\thepage| appropriately
and reset the counter |page| at the start of each child file.

%%%%%%%%%%%%%%%%%%%%%%%%%%%%%%%%%%%%%%%%%%%%%%%%%%%%%%%%%%%%%%%%%%%%%%%%%%%%%%%%
\subsection{Conditional Processing}
\label{sec:conditional}

The package provides a mechanism to compile different versions
of a document. To customise the versions further some conditional processing
can come in handy to distinguish which version is being compiled.
The package provides two macros to describe the compilation context:

%%%%%%%%%%%%%%%%%%%%%%%%%%%%%%%%%%%%%%%%
\DescribeMacro{\ifchilddoc}
The conditional |\ifchilddoc| distinguishes between the compilation of
child documents and the main document:
%
\begin{center}
|\ifchilddoc |\textit{child-code}| |[|\||else |\textit{main-code}]| \||fi|
\end{center}

%%%%%%%%%%%%%%%%%%%%%%%%%%%%%%%%%%%%%%%%
\DescribeMacro{\childdocname}
\DescribeMacro{\childdocjob}
The macro |\childdocname| contains the filename (without extension)
of the main or child file being processed.
Note that |\childdocjob| will always contain the name of the main file.

%%%%%%%%%%%%%%%%%%%%%%%%%%%%%%%%%%%%%%%%
\paragraph{Title Page.}

Conditional processing can be used to include a title or banner page
in the main document when proper precautions are taken.
Importantly, the code in the main file should ensure that the page counter
(as well as other status parameters which are stored in the |.aux| files)
takes the same value after the conditional processing.
Otherwise the page numbers may take divergent values
depending on which part is compiled.

For example, a title page could be declared by:
%
\begin{center}
\begin{tabular}{l}
|\ifchilddoc\||else|\\
|\addtocounter{page}{-1}|\\
\textit{code for title page}\\
|\newpage|\\
|\||fi|
\end{tabular}
\end{center}
%
A banner page for the child documents can be generated by:
%
\begin{center}
\begin{tabular}{l}
|\ifchilddoc|\\
|\addtocounter{page}{-1}|\\
\textit{code for banner page}\\
|\newpage|\\
|\||fi|
\end{tabular}
\end{center}
%
Here one could write a message such as:
\begin{center}
|This is the part \childdocname{} of \childdocjob{}.|
\end{center}

%%%%%%%%%%%%%%%%%%%%%%%%%%%%%%%%%%%%%%%%%%%%%%%%%%%%%%%%%%%%%%%%%%%%%%%%%%%%%%%%
\subsection{Flags}
\label{sec:flags}

The package makes it easy to generate different versions
of the main or child documents.
To this end compilation flags can be defined
and assigned different default values.
They will be particularly useful in conjunction
with the forwarding mechanism described in \secref{sec:forward}.

For example, it may be useful to have a flag |\version|
which can be set to |draft| or |final|.
The document source will contain some conditional code
depending on the value of |\version|.
Suppose further, the flag should default to |final| for the main file
and to |draft| for child files
which is a natural assignment for editing the document.
This is achieved by placing the following code
in the preamble of the main document
(below the |\childdocmain| directive):
%
\begin{center}
\begin{tabular}{l}
|\ifchilddoc|\\
|\providecommand{\version}{draft}|\\
|\||else|\\
|\providecommand{\version}{final}|\\
|\||fi|
\end{tabular}
\end{center}
%
The definition by |\providecommand| makes sure
that previous definitions are not overwritten.
Further statements |\providecommand{\version}{...}|
can thus be added before the above code to override it.

For the main file, one might add a line
(between |\childdocmain| and the above block)
%
\begin{center}
|%\ifchilddoc\||else\providecommand{\version}{draft}\||fi|
\end{center}
%
which can be uncommented to produce a draft version.
Likewise one can add a line to the very top of a child file
(above the |\childdocof{|\textit{main}|}| directive)
%
\begin{center}
|%\providecommand{\version}{final}|
\end{center}
%
which can be uncommented to produce the final version of this child document.

%%%%%%%%%%%%%%%%%%%%%%%%%%%%%%%%%%%%%%%%%%%%%%%%%%%%%%%%%%%%%%%%%%%%%%%%%%%%%%%%
\subsection{Forwarding}
\label{sec:forward}

Different versions of the main or child documents
using compilation flags as described in \secref{sec:flags}
can be (permanently) stored in different files
for convenient compilation, viewing and distribution.
To this end, the package defines a command
to pass on compilation to a different file:

%%%%%%%%%%%%%%%%%%%%%%%%%%%%%%%%%%%%%%%%
\DescribeMacro{\childdocforward}
The command |\childdocforward| redirects processing to
another source file:
%
\begin{center}
\begin{tabular}{l}
|\input{childdoc.def}|\\
|\childdocforward[|\textit{main}|]{|\textit{dest}|}|\\
\end{tabular}
\end{center}
%
The argument \textit{dest} is the destination file
(without extension).
It should be the main file or one of the child files.
Note that further \textsf{childdoc} directives
such as |\childdocof| and |\childdocforward|
in the indicated file will be processed in this form.
The optional argument \textit{main}
passes on directly to the main file \textit{main}
while pretending to compile the child \textit{dest}.
This form behaves as if \textit{dest}
issues |\childdocof{|\textit{main}|}| right away,
and no further \textsf{childdoc} directives will be processed.

%%%%%%%%%%%%%%%%%%%%%%%%%%%%%%%%%%%%%%%%
\DescribeMacro{\...prefix}
In the alternative form |\childdocforwardprefix|,
%
\begin{center}
\begin{tabular}{l}
|\input{childdoc.def}|\\
|\childdocforwardprefix[|\textit{main}|]{|\textit{prefix}|}{|\textit{dest}|}|
\end{tabular}
\end{center}
%
the destination file is determined by a pattern
depending on the current file:
To make this work, the current file must be called
`{\textit{prefix}\hspace{0.2em}\textit{suffix}}'
with \textit{prefix} matching precisely the argument.
Processing is then passed on to the file
`{\textit{dest}\hspace{0.2em}\textit{suffix}}'.
Surely, the same effect is achieved by
directly specifying the
argument `{\textit{dest}\hspace{0.2em}\textit{suffix}}'
in the first form.
However, that requires to set up a different file
for each child. With the alternative form of the command
all these files can have exactly the same content
which simplifies setting them up and maintaining them.

For example, the following file |draft.tex|
with a compilation flag |\version| as described in \secref{sec:flags}
compiles the main document as a draft:
%
\begin{center}
\begin{tabular}{l}
|\def\version{draft}|\\
|\input{childdoc.def}|\\
|\childdocforward{|\textit{main}|}|
\end{tabular}
\end{center}
%
Likewise, the following files |final|\textit{nn}|.tex|
compile the final version of the child document
|child|\textit{nn}|.tex|:
%
\begin{center}
\begin{tabular}{l}
|\def\version{final}|\\
|\input{childdoc.def}|\\
|\childdocforwardprefix{final}{child}|
\end{tabular}
\end{center}
%

Note that when several versions of a main file and/or of each child file
are to be generated, it may be convenient to set up a |Makefile| or
shell script to automatise the process.

%%%%%%%%%%%%%%%%%%%%%%%%%%%%%%%%%%%%%%%%%%%%%%%%%%%%%%%%%%%%%%%%%%%%%%%%%%%%%%%%
\subsection{Command Line Processing}
\label{sec:commandline}

The effect of redirection files can also be achieved by invoking
the \LaTeX{} compiler with a more elaborate command line.
Most conveniently this should be done as part
of a shell script or a |Makefile|.

When using \textsf{childdoc} in the main file, the following
command lines effectively perform a redirection
(note that depending on the shell being used,
backslashes may have to be doubled: `|\|' $\to$ `|\\|'):
%
\begin{center}
|... -jobname "|\textit{target}|" |\\|"|[\textit{flags}]%
|\input{childdoc.def}\childdocforward[|\textit{main}|]{|\textit{dest}|}"|
\end{center}
%
Here \textit{target} is the name of the output file,
\textit{main} is the name of the main file
and \textit{dest} is the name of the main or child file to be processed
(all filenames without extensions).
The optional argument \textit{main} can be omitted
if \textit{main} matches \textit{dest}.
Optionally, compilation \textit{flags} can be defined via |\def| commands.
This command line makes the \TeX{} engine believe
it is compiling the file \textit{target}
whose content is specified as the latter parameter.
The provided code then forwards the processing to
\textit{main} or \textit{dest} as described in \secref{sec:forward}.

%%%%%%%%%%%%%%%%%%%%%%%%%%%%%%%%%%%%%%%%%%%%%%%%%%%%%%%%%%%%%%%%%%%%%%%%%%%%%%%%
\subsection{Include by Input}
\label{sec:input}

Including child documents by |\include| has some restrictions by design.
Most notably, the content of a child document always occupies
its own set of pages; pages cannot be shared between child documents.
Usually, this behaviour makes perfect sense
because each child document contain an essential part of the document.
However, in some situations it may be desirable to compose
a document from a collection of parts
without having mandatory page breaks between then.
For this case, the package
provides a mechanism to include parts
by |\input| which can also be processed individually.
However, by construction this mechanism
requires manual handling of the content to be output.

%%%%%%%%%%%%%%%%%%%%%%%%%%%%%%%%%%%%%%%%
\DescribeMacro{\ifchilddocmanual}
The main file should be prepared as usual, see \secref{sec:include}.
However, the document body must make a distinction
between processing of an individual part and of the main document, e.g.:
%
\begin{center}
\begin{tabular}{l}
|\ifchilddocmanual|\\
|\input{\childdocname}|\\
|\||else|\\
\textit{document body with }|\input{|\textit{part}|}|\\
|\||fi|
\end{tabular}
\end{center}
%
The conditional |\ifchilddocmanual| is true whenever
a part to be included by |\input| is being compiled,
and the name of the part is stored in |\childdocname|.

%%%%%%%%%%%%%%%%%%%%%%%%%%%%%%%%%%%%%%%%
\DescribeMacro{\childdocby}
Each part to be included by |\input| should start with:
%
\begin{center}
\begin{tabular}{l}
|\input{childdoc.def}|\\
|\childdocby{|\textit{main}|}|\\
\end{tabular}
\end{center}
%
The directive |\childdocby| is similar to |\childdocof|
described in \secref{sec:include},
but the subsequent selection of content must be done manually.
To that end, both |\ifchilddoc| and |\ifchilddocmanual|
will be true upon processing of a part,
and the name of the part is stored in |\childdocname|.
Note that |\jobname| will be set to the filename of the current part
so that each part receives an individual |.aux| file
that does not interfere with the |.aux| file(s) of the main document.
This behaviour can be altered by the alternative form
|\childdocby[*]{|\textit{main}|}| (with a non-empty optional argument)
which uses the |.aux| file of the main document
by setting |\jobname| to \textit{main}.

%%%%%%%%%%%%%%%%%%%%%%%%%%%%%%%%%%%%%%%%%%%%%%%%%%%%%%%%%%%%%%%%%%%%%%%%%%%%%%%%
\subsection{Driver Development}
\label{sec:driver}

The \textsf{childdoc} mechanism can also be use for the development
of definition files such as \LaTeX{} styles or classes.
This case differs from the above setup with multiple parts
included by |\include| in that no |\includeonly| should be invoked.
This can be achieved by starting the include file
(before |\ProvidesPackage|) with:
%
\begin{center}
\begin{tabular}{l}
|\input{childdoc.def}|\\
|\childdocforward{|\textit{main}|}|\\
\end{tabular}
\end{center}
%
or alternatively with:
%
\begin{center}
\begin{tabular}{l}
|\input{childdoc.def}|\\
|\childdocby{|\textit{main}|}|\\
\end{tabular}
\end{center}
%
Both forms have slightly different effects as described above.
The main file is prepared as usual, see \secref{sec:include}.

%%%%%%%%%%%%%%%%%%%%%%%%%%%%%%%%%%%%%%%%%%%%%%%%%%%%%%%%%%%%%%%%%%%%%%%%%%%%%%%%
\subsection{Legacy Detection}
\label{sec:detection}

The directive |\childdocmain| in the main file can detect
whether the complete document or merely a child is to be compiled
even without using the directive |\childdocof|.
This method is deprecated because it is less robust
and there is no compelling reason to use it;
it is merely provided for backward compatibility
and it may be removed in future versions.

If the detection mechanism is to be used,
it is mandatory to correctly specify
the filename of the main file as the argument of |\childdocmain|:
%
\begin{center}
\begin{tabular}{l}
|\input{childdoc.def}|\\
|\childdocmain{|\textit{main}|}|\\
\end{tabular}
\end{center}
%
If |\jobname| does not match the argument \textit{main} of |\childdocmain|,
it is assumed that |\jobname| points to the child file to be compiled.
When using |\childdocmain| with the main file specified as argument,
it suffices to start a child file
with just |\input{|\textit{main}|}|
without loading of the package and using |\childdocof|.
If instead all processing is done
with the appropriate \textsf{childdoc} directives,
the argument of \textit{main} of |\childdocmain| can be empty.

An alternative version of the command line processing described
in \secref{sec:commandline} using the detection mechanism reads:
%
\begin{center}
|... -jobname "|\textit{target}|" "|[\textit{flags}]%
[|\def\jobname{|\textit{dest}|}|]|\input{|\textit{main}|}"|
\end{center}

%%%%%%%%%%%%%%%%%%%%%%%%%%%%%%%%%%%%%%%%%%%%%%%%%%%%%%%%%%%%%%%%%%%%%%%%%%%%%%%%
\subsection{Manual Code}
\label{sec:manual}

In case one cannot be certain whether the definitions file |childdoc.def|
is installed on the target \TeX{} distribution
and one prefers not to ship it,
it is conceivable to paste a few relevant commands into the sources.

To that end, drop all statements |\input{childdoc.def}|
and perform the replacements as outlined below.
Instead of |\childdocmain{|\textit{main}|}| add the following code
to the top of the main file:
%
\begin{center}
\begin{tabular}{l}
|\||ifdefined\childdocname\endinput\||fi\newif\ifchilddoc|\\
|\edef\childdocname{\scantokens\expandafter{\jobname\noexpand}}|\\
|\def\childdocmain{|\textit{main}|}\||ifx\childdocmain\childdocname\||else|\\
|\childdoctrue\includeonly{\childdocname}\let\jobname\childdocmain\||fi|\\
\end{tabular}
\end{center}
%
Instead of |\childdocof{|\textit{main}|}| just include the main file
at the top of each child file:
%
\begin{center}
|\input{|\textit{main}|}|
\end{center}
%
A simple redirection |\childdocforward{|\textit{dest}|}| is achieved by:
%
\begin{center}
|\def\jobname{|\textit{dest}|}\input{\jobname}|
\end{center}
%
The redirection with prefix
|\childdocforwardprefix[|\textit{prefix}|]{|\textit{dest}|}|
is accomplished by:
%
\begin{center}
\begin{tabular}{l}
|{\edef\jobname{\scantokens\expandafter{\jobname\noexpand}}|\\
|\def\redirectjob |\textit{prefix}|#1~~~{\gdef\jobname{|\textit{dest}|#1}}|\\
|\expandafter\redirectjob\jobname~~~}\input{\jobname}|
\end{tabular}
\end{center}

In an alternative approach,
child documents can be compiled by a specific command line
without additional code or specific definitions:
%
\begin{center}
|... -jobname "|\textit{target}|" "|[\textit{flags}]%
|\includeonly{|\textit{dest}|}\input{|\textit{main}|}"|
\end{center}
%

%%%%%%%%%%%%%%%%%%%%%%%%%%%%%%%%%%%%%%%%%%%%%%%%%%%%%%%%%%%%%%%%%%%%%%%%%%%%%%%%
%%%%%%%%%%%%%%%%%%%%%%%%%%%%%%%%%%%%%%%%%%%%%%%%%%%%%%%%%%%%%%%%%%%%%%%%%%%%%%%%
\section{Information}

%%%%%%%%%%%%%%%%%%%%%%%%%%%%%%%%%%%%%%%%%%%%%%%%%%%%%%%%%%%%%%%%%%%%%%%%%%%%%%%%
\subsection{Copyright}

Copyright \copyright{} 2017--2018 Niklas Beisert

This work may be distributed and/or modified under the
conditions of the \LaTeX{} Project Public License, either version 1.3
of this license or (at your option) any later version.
The latest version of this license is in
  \url{http://www.latex-project.org/lppl.txt}
and version 1.3 or later is part of all distributions of \LaTeX{}
version 2005/12/01 or later.

This work has the LPPL maintenance status `maintained'.

The Current Maintainer of this work is Niklas Beisert.

This work consists of the files |README.txt|, |childdoc.ins| and |childdoc.dtx|
as well as the derived files |childdoc.def|, |cdocsamp.tex|
with |cdocsch1.tex|, |cdocsch2.tex|, |cdocspt3.tex|, |cdocspt4.tex|,
|cdocsdrf.tex|, |cdocsfn1.tex|, |cdocsfn2.tex|
as well as |childdoc.pdf|.

%%%%%%%%%%%%%%%%%%%%%%%%%%%%%%%%%%%%%%%%%%%%%%%%%%%%%%%%%%%%%%%%%%%%%%%%%%%%%%%%
\subsection{Files and Installation}

The package consists of the files:
%
\begin{center}
\begin{tabular}{ll}
    |README.txt|   & readme file \\
    |childdoc.ins| & installation file \\
    |childdoc.dtx| & source file \\
    |childdoc.def| & definition file \\
    |cdocsamp.tex| & sample main file \\
    |cdocsch1.tex| & sample include file \\
    |cdocsch2.tex| & sample include file \\
    |cdocspt3.tex| & sample part file \\
    |cdocspt4.tex| & sample part file \\
    |cdocsdrf.tex| & sample redirection file \\
    |cdocsfn1.tex| & sample redirection file \\
    |cdocsfn2.tex| & sample redirection file \\
    |childdoc.pdf| & manual
\end{tabular}
\end{center}
%
The distribution consists of the files
|README.txt|, |childdoc.ins| and |childdoc.dtx|.
%
\begin{itemize}
\item
Run (pdf)\LaTeX{} on |childdoc.dtx|
to compile the manual |childdoc.pdf| (this file).
\item
Run \LaTeX{} on |childdoc.ins| to create the definitions file |childdoc.def|
and the sample |cdocsamp.tex| with include files
|cdocsch1.tex|, |cdocsch2.tex|, |cdocspt3.tex|, |cdocspt4.tex|,
|cdocsdrf.tex|, |cdocsfn1.tex|, |cdocsfn2.tex|.
Then copy the file |childdoc.def| to an appropriate directory of your \LaTeX{}
distribution, e.g.\ \textit{texmf-root}|/tex/latex/childdoc|.
\end{itemize}

%%%%%%%%%%%%%%%%%%%%%%%%%%%%%%%%%%%%%%%%%%%%%%%%%%%%%%%%%%%%%%%%%%%%%%%%%%%%%%%%
\subsection{Related CTAN Packages}

There are several other packages which offer a similar functionality:
%
\begin{itemize}
\item
The packages
\href{http://ctan.org/pkg/docmute}{\textsf{docmute}},
\href{http://ctan.org/pkg/includex}{\textsf{includex}} and
\href{http://ctan.org/pkg/standalone}{\textsf{standalone}}
provide commands to include only the document body of
a child file thus allowing both files to be compiled individually.
\item
The packages \href{http://ctan.org/pkg/subdocs}{\textsf{subdocs}}
and \href{http://ctan.org/pkg/subfiles}{\textsf{subfiles}}
provide structures in which the main and child documents can be
encapsulated and allowing them to be compiled individually.
The inclusion mechanism is different from the conventional |\include|.
\item
The package \href{http://ctan.org/pkg/combine}{\textsf{combine}}
is an elaborate solution to combine several documents into one.
\end{itemize}
%
See also the CTAN topic \href{http://ctan.org/topic/subdocs}{\textsf{subdocs}}
for further related packages.
The present package differs from the above solutions in that
a document structure constructed with the conventional |\include| mechanism
just needs two extra commands at the top of every file
such that all constituent files can be compiled individually.

%%%%%%%%%%%%%%%%%%%%%%%%%%%%%%%%%%%%%%%%%%%%%%%%%%%%%%%%%%%%%%%%%%%%%%%%%%%%%%%%
%\subsection{Feature Suggestions}
%
%The following is a list of features which may be useful for future
%versions of this package:
%%
%\begin{itemize}
%\item
%\ldots
%\end{itemize}

%%%%%%%%%%%%%%%%%%%%%%%%%%%%%%%%%%%%%%%%%%%%%%%%%%%%%%%%%%%%%%%%%%%%%%%%%%%%%%%%
\subsection{Revision History}

%%%%%%%%%%%%%%%%%%%%%%%%%%%%%%%%%%%%%%%%
\paragraph{v2.0:} 2018/12/30

\begin{itemize}
\item
immediate forward processing
\item
added |\childdocby| mechanism
\item
manual restructured
\end{itemize}

%%%%%%%%%%%%%%%%%%%%%%%%%%%%%%%%%%%%%%%%
\paragraph{v1.6:} 2018/01/17

\begin{itemize}
\item
application for development of include files
\item
corrections to manual
\end{itemize}

%%%%%%%%%%%%%%%%%%%%%%%%%%%%%%%%%%%%%%%%
\paragraph{v1.5:} 2017/05/21

\begin{itemize}
\item
more complete structuring introduced
\item
|\childdocof| introduced
\item
|\childdoc| renamed to |\childdocmain|
\item
|\childredirect| renamed to |\childdocforward| and |\childdocforwardprefix|
and functionality expanded
\end{itemize}

%%%%%%%%%%%%%%%%%%%%%%%%%%%%%%%%%%%%%%%%
\paragraph{v1.0:} 2017/04/27

\begin{itemize}
\item
manual and install package
\item
first version published on CTAN
\end{itemize}

%%%%%%%%%%%%%%%%%%%%%%%%%%%%%%%%%%%%%%%%
\paragraph{v0.6:} 2017/04/26

\begin{itemize}
\item
redirection mechanism added
\end{itemize}

%%%%%%%%%%%%%%%%%%%%%%%%%%%%%%%%%%%%%%%%
\paragraph{v0.5:} 2017/04/26

\begin{itemize}
\item
functionality in definition file
\end{itemize}


%%%%%%%%%%%%%%%%%%%%%%%%%%%%%%%%%%%%%%%%%%%%%%%%%%%%%%%%%%%%%%%%%%%%%%%%%%%%%%%%
%%%%%%%%%%%%%%%%%%%%%%%%%%%%%%%%%%%%%%%%%%%%%%%%%%%%%%%%%%%%%%%%%%%%%%%%%%%%%%%%
%%%%%%%%%%%%%%%%%%%%%%%%%%%%%%%%%%%%%%%%%%%%%%%%%%%%%%%%%%%%%%%%%%%%%%%%%%%%%%%%
\appendix

\settowidth\MacroIndent{\rmfamily\scriptsize 000\ }

 \DocInput{childdoc.dtx}

\end{document}
%</driver>
% \fi
%
% %%%%%%%%%%%%%%%%%%%%%%%%%%%%%%%%%%%%%%%%%%%%%%%%%%%%%%%%%%%%%%%%%%%%%%%%%%%%%%
% %%%%%%%%%%%%%%%%%%%%%%%%%%%%%%%%%%%%%%%%%%%%%%%%%%%%%%%%%%%%%%%%%%%%%%%%%%%%%%
% \section{Sample}
%\iffalse
%<*samplemain>
%\fi
%
% The following presents a sample document
% with two chapters, two parts, a title page,
% a compile flag as well as three forwarding files to set the flag.
% It consists of eight |.tex| files:
% \begin{center}
% \begin{tabular}{ll}
% |cdocsamp.tex|&main file\\
% |cdocsch1.tex|&include file for chapter 1\\
% |cdocsch2.tex|&include file for chapter 2\\
% |cdocspt3.tex|&include file for part 3\\
% |cdocspt4.tex|&include file for part 4\\
% |cdocsdrf.tex|&forwarding file for main file in draft mode\\
% |cdocsfi1.tex|&forwarding file for final version of chapter 1\\
% |cdocsfi2.tex|&forwarding file for final version of chapter 2\\
% \end{tabular}
% \end{center}
% Each of the eight files can be compiled directly by the \LaTeX{} compiler.
%
% %%%%%%%%%%%%%%%%%%%%%%%%%%%%%%%%%%%%%%
% \paragraph{Main File.}
%
% The main file is called |cdocsamp.tex|.
%
% Load the \textsf{childdoc} definitions and
% declare the filename for the main document:
%    \begin{macrocode}
\input{childdoc.def}
\childdocmain{}
%    \end{macrocode}

% Optional override for |\version| flag:
%    \begin{macrocode}
%%\ifchilddoc\else\providecommand{\version}{draft}\fi
%    \end{macrocode}

% Define the default values for the |\version| flag
% (|final| for the main file and |draft| for childs):
%    \begin{macrocode}
\ifchilddoc
\providecommand{\version}{draft}
\else
\providecommand{\version}{final}
\fi
%    \end{macrocode}

% Load the standard document class:
%    \begin{macrocode}
\documentclass[12pt]{article}
%    \end{macrocode}

% Start the document body:
%    \begin{macrocode}
\begin{document}
%    \end{macrocode}

% Declare a title page.
% Print title, part of document being processed and version flag:
%    \begin{macrocode}
\addtocounter{page}{-1}
\begin{center}
{\LARGE\bfseries{}childdoc example\par}
\vspace{1cm}
\ifchilddoc
\ifchilddocmanual part\else chapter\fi:
`\childdocname' of `\childdocjob'\par
\else
main document: `\childdocjob'\par
\fi
version: \version\par
\end{center}
\newpage
%    \end{macrocode}

% Manually include selected file,
% otherwise process as usual:
%    \begin{macrocode}
\ifchilddocmanual
\section*{part `\childdocname'}
\input{\childdocname}
\else
%    \end{macrocode}

% Include the two chapters:
%    \begin{macrocode}
\include{cdocsch1}
\include{cdocsch2}
%    \end{macrocode}

% Include the two parts unless only chapters should be displayed:
%    \begin{macrocode}
\ifchilddoc\else
\section{part three}
\input{cdocspt3}
\section{part four}
\input{cdocspt4}
\fi
%    \end{macrocode}

% Process as usual until here:
%    \begin{macrocode}
\fi
%    \end{macrocode}

% End of document body:
%    \begin{macrocode}
\end{document}
%    \end{macrocode}
%\iffalse
%</samplemain>
%\fi
%
% %%%%%%%%%%%%%%%%%%%%%%%%%%%%%%%%%%%%%%
% \paragraph{Chapter Include Files.}
%
% The include files are called |cdocsch1.tex| and |cdocsch2.tex|.
%
%\iffalse
%<*samplechap1|samplechap2>
%\fi

% Optional override for |\version| flag:
%    \begin{macrocode}
%%\providecommand{\version}{final}
%    \end{macrocode}

% Include the main document:
%    \begin{macrocode}
\input{childdoc.def}
\childdocof{cdocsamp}
%    \end{macrocode}

%\iffalse
%</samplechap1|samplechap2>
%\fi
%
%\iffalse
%<*samplechap1>
%\fi
% Some text for chapter 1:
%    \begin{macrocode}
\section{one}
some text in chapter one
%    \end{macrocode}

%\iffalse
%</samplechap1>
%\fi
% Some text for chapter 2:
%\iffalse
%<*samplechap2>
%\fi
%    \begin{macrocode}
\section{two}
more text in chapter two
%    \end{macrocode}

%\iffalse
%</samplechap2>
%\fi
%
% %%%%%%%%%%%%%%%%%%%%%%%%%%%%%%%%%%%%%%
% \paragraph{Part Include Files.}
%
% The include files are called |cdocspt3.tex| and |cdocspt4.tex|.
%
%\iffalse
%<*samplepart3|samplepart4>
%\fi

% Optional override for |\version| flag:
%    \begin{macrocode}
%%\providecommand{\version}{final}
%    \end{macrocode}

% Include the main document:
%    \begin{macrocode}
\input{childdoc.def}
\childdocby{cdocsamp}
%    \end{macrocode}

%\iffalse
%</samplepart3|samplepart4>
%\fi
%
%\iffalse
%<*samplepart3>
%\fi
% Some text for part 3:
%    \begin{macrocode}
some text in part three
%    \end{macrocode}

%\iffalse
%</samplepart3>
%\fi
% Some text for part 4:
%\iffalse
%<*samplepart4>
%\fi
%    \begin{macrocode}
more text in part four
%    \end{macrocode}

%\iffalse
%</samplepart4>
%\fi
%
% %%%%%%%%%%%%%%%%%%%%%%%%%%%%%%%%%%%%%%
% \paragraph{Forwarding for a Complete Draft.}
%
% The following forwarding file |cdocsdrf.tex|
% compiles the main document in draft mode:
%\iffalse
%<*sampledraft>
%\fi
%    \begin{macrocode}
\def\version{draft}
\input{childdoc.def}
\childdocforward{cdocsamp}
%    \end{macrocode}

%\iffalse
%</sampledraft>
%\fi
%
% %%%%%%%%%%%%%%%%%%%%%%%%%%%%%%%%%%%%%%
% \paragraph{Forwarding for Final Version of the Chapters.}
%
% The following forwarding files |cdocsfn1.tex| and |cdocsfn2.tex|
% (with identical content)
% compile the final versions of the child documents
% |cdocsch1.tex| and |cdocsch2.tex|, respectively:
%\iffalse
%<*samplefinal>
%\fi
%    \begin{macrocode}
\def\version{final}
\input{childdoc.def}
\childdocforwardprefix[cdocsamp]{cdocsfn}{cdocsch}
%    \end{macrocode}

%\iffalse
%</samplefinal>
%\fi
%
% %%%%%%%%%%%%%%%%%%%%%%%%%%%%%%%%%%%%%%
% \paragraph{Command Line Processing.}
%
% The following three command lines generate the output files
% |cdocscld|, |cdocscl1| and |cdocscl2|
% which should be identical to
% |cdocsdrf|, |cdocsch1| and |cdocsfn2|, respectively:
% \begin{center}
% \begin{tabular}{l}
% |latex -jobname cdocscld \|\\
% |  "\def\version{draft}\input{childdoc.def}\childdocforward{cdocsamp}"|\\
% |latex -jobname cdocscl1 \|\\
% |  "\input{childdoc.def}\childdocforward[cdocsamp]{cdocsch1}"|\\
% |latex -jobname cdocscl2 \|\\
% |  "\def\version{final}\input{childdoc.def}\childdocforward{cdocsch2}"|
% \end{tabular}
% \end{center}
% Note that the trailing backslash on each first line
% merely continues the input to the second line
% (for convenient cut ant paste).
% Furthermore, the command |latex| can be replaced by any
% of its alternative versions such as |pdflatex|.
%
% %%%%%%%%%%%%%%%%%%%%%%%%%%%%%%%%%%%%%%%%%%%%%%%%%%%%%%%%%%%%%%%%%%%%%%%%%%%%%%
% %%%%%%%%%%%%%%%%%%%%%%%%%%%%%%%%%%%%%%%%%%%%%%%%%%%%%%%%%%%%%%%%%%%%%%%%%%%%%%
% \section{Implementation}
%\iffalse
%<*package>
%\fi
%
% This section describes the definitions file |childdoc.def|.

% The definitions cannot be loaded using |\usepackage| or |\RequirePackage|
% which has a mechanism to prevent loading a style file more than once.
% When loading the definitions by means of |\input|
% multiple instances have to be prevented manually:
%\iffalse
%This code needs to be before the `\ProvidesFile' directive
%which is defined at the beginning of this file.
%Therefore it is also placed there and commented out here.
%</package>
%<*discard>
%\fi
%    \begin{macrocode}
\ifdefined\childdocmain\endinput\fi
%    \end{macrocode}
%\iffalse
%</discard>
%<*package>
%\fi
%
% \macro{\ifchilddoc}
% \macro{\ifchilddocmanual}
% The conditional |\ifchilddoc| tells whether a
% child (true) or main (false) document is being compiled.
% The conditional |\ifchilddocmanual| tells whether
% the |\includeonly| mechanism is used (false) or
% the selection of child files must be performed manually (true).
% The definitions initialise to false:
%    \begin{macrocode}
\newif\ifchilddoc
\newif\ifchilddocmanual
%    \end{macrocode}

% \macro{\childdocname}
% \macro{\childdocjob}
% The macro |\childdocname| stores the name of the main document
% to be compiled. The macro |\childdocjob| stores the name of
% the document on which the \LaTeX{} compiler was originally invoked.
% The content of |\jobname| cannot be compared
% to filenames specified in the source due to different catcodes.
% The following code rescans |\jobname|, stores the result
% in |\childdocname| and saves a copy in |\childdocjob|:
%    \begin{macrocode}
\edef\childdocname{\scantokens\expandafter{\jobname\noexpand}}
\let\childdocjob\childdocname
%    \end{macrocode}

% \macro{\childdocdisable}
% The macro |\childdocdisable| prevents the main file
% from being processed more than once.
% At this stage, the main document command |\childdocmain|
% is assumed to be called once again where it should do nothing.
% Any subsequent call to it should prevent
% a secondary processing of the main document
% It overwrites the forwarding commands
% |\childdocof| and |\childdocforward|
% with empty macros to prevent further inclusions of the main document:
%    \begin{macrocode}
\newcommand{\childdocdisable}
{
  \renewcommand{\childdocmain}[1]{\renewcommand{\childdocmain}[1]{\endinput}}
  \renewcommand{\childdocof}[1]{}
  \renewcommand{\childdocby}[2][]{}
  \renewcommand{\childdocforward}[2][]{}
  \renewcommand{\childdocdisable}{}
}
%    \end{macrocode}

% \macro{\childdocmain}
% The macro |\childdocmain| is to be called at the top of the main file
% with nothing or the main filename (without extension) as argument.
% First, it breaks loops.
% If the argument is not empty and does not match |\childdocname|
% (which is set by the first inclusion of |childdoc.def|),
% |\ifchilddoc| is set to true, |\includeonly| is applied to the child file
% and |\jobname| is set to the main file
% (for proper handling of |.aux| files):
%    \begin{macrocode}
\newcommand{\childdocmain}[1]
{
  \childdocdisable\childdocmain{}
  \if?#1?\else
    \begingroup
      \def\childdoctmp{#1}
      \ifx\childdoctmp\childdocname
        \def\childdoctmp{}
      \else
        \def\childdoctmp
        {
          \childdoctrue
          \includeonly{\childdocname}
          \def\childdocjob{#1}
          \def\jobname{#1}
        }
      \fi
      \expandafter
    \endgroup
    \childdoctmp
  \fi
}
%    \end{macrocode}

% \macro{\childdocof}
% The command |\childdocof| redirects
% compilation to the main file |#1|.
%    \begin{macrocode}
\newcommand{\childdocof}[1]
{
  \childdocdisable
  \childdoctrue
  \includeonly{\childdocname}
  \def\jobname{#1}
  \def\childdocjob{#1}
  \input{#1}
}
%    \end{macrocode}

% \macro{\childdocby}
% The command |\childdocby| ....
%    \begin{macrocode}
\newcommand{\childdocby}[2][]
{
  \childdocdisable
  \childdoctrue
  \childdocmanualtrue
  \if?#1?\else
    \def\jobname{#2}
  \fi
  \def\childdocjob{#2}
  \input{#2}
  \endinput
}
%    \end{macrocode}

% \macro{\childdocforward}
% The command |\childdocforward| redirects
% compilation to the main file or
% (if the optional argument is given) a child file.
% Parameters are set as if the main file
% or a child file starting with |\childdocof| was compiled.
% Then compilation is handed over to the main file:
%    \begin{macrocode}
\newcommand{\childdocforward}[2][]
{
  \begingroup
    \if?#1?
      \def\childdoctmp
      {
        \def\childdocname{#2}
        \def\childdocjob{#2}
        \def\jobname{#2}
        \input{#2}
        \endinput
      }
    \else
      \def\childdoctmp
      {
        \childdocdisable
        \def\childdocname{#2}
        \childdoctrue
        \includeonly{#2}
        \def\childdocjob{#1}
        \def\jobname{#1}
        \input{#1}
        \endinput
      }
    \fi
    \expandafter
  \endgroup
  \childdoctmp
}
%    \end{macrocode}

% \macro{\childdocforwardprefix}
% The command |\childdocforwardprefix| redirects
% compilation to the main or a child file by means of a pattern.
% The prefix |#1| in the current filename is replaced by |#2|
% and the suffix of the current filename is kept
% (it is assumed that the filename does not contain the substring `|~~~|'
% which is used as a delimiter).
% Compilation is handed over to the new file by |\childdocforward|:
%    \begin{macrocode}
\newcommand{\childdocforwardprefix}[3][]
{
  \begingroup
    \def\childdocextract #2##1~~~{\def\childdoctmp{\childdocforward[#1]{#3##1}}}
    \expandafter\childdocextract\childdocname~~~
    \expandafter
  \endgroup
  \childdoctmp
}
%    \end{macrocode}

% \macro{\childdoc}
% The deprecated macro |\childdoc| is a legacy version of |\childdocmain|:
%    \begin{macrocode}
\newcommand{\childdoc}{\childdocmain}
%    \end{macrocode}

% \macro{\childdocredirect}
% The deprecated macro |\childdocredirect| is a legacy version
% of |\childdocforward| and |\childdocforwardprefix|:
%    \begin{macrocode}
\newcommand{\childdocredirect}[2][]
{
  \begingroup
    \if?#1?
      \def\childdoctmp{\childdocforward{#2}}
    \else
      \def\childdoctmp{\childdocforwardprefix{#1}{#2}}
    \fi
    \expandafter
  \endgroup
  \childdoctmp
}
%    \end{macrocode}

%\iffalse
%</package>
%\fi
%
\endinput
|\\
|\childdocmain{|\textit{main}|}|\\
\end{tabular}
\end{center}
%
If |\jobname| does not match the argument \textit{main} of |\childdocmain|,
it is assumed that |\jobname| points to the child file to be compiled.
When using |\childdocmain| with the main file specified as argument,
it suffices to start a child file
with just |\input{|\textit{main}|}|
without loading of the package and using |\childdocof|.
If instead all processing is done
with the appropriate \textsf{childdoc} directives,
the argument of \textit{main} of |\childdocmain| can be empty.

An alternative version of the command line processing described
in \secref{sec:commandline} using the detection mechanism reads:
%
\begin{center}
|... -jobname "|\textit{target}|" "|[\textit{flags}]%
[|\def\jobname{|\textit{dest}|}|]|\input{|\textit{main}|}"|
\end{center}

%%%%%%%%%%%%%%%%%%%%%%%%%%%%%%%%%%%%%%%%%%%%%%%%%%%%%%%%%%%%%%%%%%%%%%%%%%%%%%%%
\subsection{Manual Code}
\label{sec:manual}

In case one cannot be certain whether the definitions file |childdoc.def|
is installed on the target \TeX{} distribution
and one prefers not to ship it,
it is conceivable to paste a few relevant commands into the sources.

To that end, drop all statements |% \iffalse
%
% childdoc.dtx Copyright (C) 2017-2018 Niklas Beisert
%
% This work may be distributed and/or modified under the
% conditions of the LaTeX Project Public License, either version 1.3
% of this license or (at your option) any later version.
% The latest version of this license is in
%   http://www.latex-project.org/lppl.txt
% and version 1.3 or later is part of all distributions of LaTeX
% version 2005/12/01 or later.
%
% This work has the LPPL maintenance status `maintained'.
%
% The Current Maintainer of this work is Niklas Beisert.
%
% This work consists of the files childdoc.dtx and childdoc.ins
% and the derived files childdoc.def and cdocsamp.tex with
% cdocsch1.tex, cdocsch2.tex, cdocsdrf.tex, cdocsfn1.tex, cdocsfn2.tex.
%
%<package>\ifdefined\childdocmain\endinput\fi
%<package>\ProvidesFile{childdoc.def}[2018/12/30 v2.0 child document driver]
%<samplemain>\ProvidesFile{cdocsamp.tex}[2018/12/30 v2.0 sample for childdoc]
%<*driver>
%\ProvidesFile{childdoc.drv}[2018/12/30 v2.0 childdoc reference manual file]
\PassOptionsToClass{10pt,a4paper}{article}
\documentclass{ltxdoc}

\usepackage[margin=35mm]{geometry}
\usepackage{hyperref}
\usepackage{hyperxmp}
\usepackage[usenames]{color}

\hypersetup{colorlinks=true}
\hypersetup{pdfstartview=FitH}
\hypersetup{pdfpagemode=UseNone}
\hypersetup{pdfsource={}}
\hypersetup{pdflang={en-UK}}
\hypersetup{pdfcopyright={Copyright 2017-2018 Niklas Beisert.
  This work may be distributed and/or modified under the
  conditions of the LaTeX Project Public License, either version 1.3
  of this license or (at your option) any later version.}}
\hypersetup{pdflicenseurl={http://www.latex-project.org/lppl.txt}}
\hypersetup{pdfcontactaddress={ETH Zurich, ITP, HIT K,
  Wolfgang-Pauli-Strasse 27}}
\hypersetup{pdfcontactpostcode={8093}}
\hypersetup{pdfcontactcity={Zurich}}
\hypersetup{pdfcontactcountry={Switzerland}}
\hypersetup{pdfcontactemail={nbeisert@itp.phys.ethz.ch}}
\hypersetup{pdfcontacturl={http://people.phys.ethz.ch/\xmptilde nbeisert/}}

\newcommand{\secref}[1]{\hyperref[#1]{section \ref*{#1}}}

\parskip1ex
\parindent0pt
\let\olditemize\itemize
\def\itemize{\olditemize\parskip0pt}

\begin{document}

\title{The \textsf{childdoc} Package}
\hypersetup{pdftitle={The childdoc Package}}
\author{Niklas Beisert\\[2ex]
  Institut f\"ur Theoretische Physik\\
  Eidgen\"ossische Technische Hochschule Z\"urich\\
  Wolfgang-Pauli-Strasse 27, 8093 Z\"urich, Switzerland\\[1ex]
  \href{mailto:nbeisert@itp.phys.ethz.ch}
  {\texttt{nbeisert@itp.phys.ethz.ch}}}
\hypersetup{pdfauthor={Niklas Beisert}}
\hypersetup{pdfsubject={Manual for the LaTeX2e Package childdoc}}
\date{30 December 2018, \textsf{v2.0}}
\maketitle

\begin{abstract}\noindent
\textsf{childdoc} is a \LaTeXe{} package
that enables the direct compilation
of document sections included by |\include|
to individual files.
\end{abstract}

\begingroup
\parskip0ex
\tableofcontents
\endgroup

%%%%%%%%%%%%%%%%%%%%%%%%%%%%%%%%%%%%%%%%%%%%%%%%%%%%%%%%%%%%%%%%%%%%%%%%%%%%%%%%
%%%%%%%%%%%%%%%%%%%%%%%%%%%%%%%%%%%%%%%%%%%%%%%%%%%%%%%%%%%%%%%%%%%%%%%%%%%%%%%%
\section{Introduction}

\LaTeX{} provides a mechanism to structure a large document (such as a book)
into a main file and several child files (containing the chapters)
using the |\include| command.
This mechanism is beneficial for documents
which span hundreds of pages in order to
make the source file(s) more manageable.
Moreover, compilation can be restricted to
selected child files by means of the |\includeonly| command.
The latter feature can be used to reduce the compilation time while editing
(this was significantly more useful in the earlier days of \LaTeX{})
or to generate a smaller document which is easier to navigate.
Another application of |\includeonly| is to generate
documents consisting of selected parts of the complete document.

However, there are a few drawbacks of the plain |\include| mechanism:
\begin{itemize}
\item
The child files cannot be compiled on their own,
they can only be compiled via the main file.
A naive editing environment
(such as a text editor with an option
to have the current file processed by \LaTeX)
may require one to switch to the main file before compiling;
attempting to compile the child file produces errors.
\item
The main file must be modified (each time)
to adjust the |\includeonly| command
to the present needs. This easily leaves the main file in a messy state.
\item
The generated document will always carry the filename
of the main document. This is inconvenient if
several child files are to be compiled and
to be kept for distribution.
\end{itemize}

The present package provides a simple interface
to make child files individually compilable by \LaTeX{}.
Compiling a child file then has the same effect as compiling
the main file with an |\includeonly| command
to select the appropriate child.
Moreover the generated document will carry the name of the child
rather than the main file.
This resolves all three above issues.

This feature is meant to make the editing of books,
thesis documents and lecture notes somewhat more convenient.
However, the package can also be used efficiently for
composing a series of documents (such as exercise sheets)
which are typically distributed individually.
It then assists the author in generating the individual documents
(potentially in different versions)
as well as a document containing the collected series.
Another application is in developing style files
or other kinds of included material
where compilation of the style file could redirect
to a sample or test file.

%%%%%%%%%%%%%%%%%%%%%%%%%%%%%%%%%%%%%%%%%%%%%%%%%%%%%%%%%%%%%%%%%%%%%%%%%%%%%%%%
%%%%%%%%%%%%%%%%%%%%%%%%%%%%%%%%%%%%%%%%%%%%%%%%%%%%%%%%%%%%%%%%%%%%%%%%%%%%%%%%
\section{Usage}

First of all, the package \textsf{childdoc} is \emph{not} a standard
\LaTeXe{} |.sty| style file! Therefore it needs to be invoked in
a non-standard way.

%%%%%%%%%%%%%%%%%%%%%%%%%%%%%%%%%%%%%%%%%%%%%%%%%%%%%%%%%%%%%%%%%%%%%%%%%%%%%%%%
\subsection{Included Files}
\label{sec:include}

%%%%%%%%%%%%%%%%%%%%%%%%%%%%%%%%%%%%%%%%
\DescribeMacro{\childdocmain}
To use the package, add the commands
\begin{center}
\begin{tabular}{l}
|\input{childdoc.def}|\\
|\childdocmain{}|\\
\end{tabular}
\end{center}
at the very top of the main \LaTeX{} file,
in particular \emph{before} the |\documentclass| statement!
The argument of |\childdocmain| should be left empty
(but it must be present).

%%%%%%%%%%%%%%%%%%%%%%%%%%%%%%%%%%%%%%%%
\DescribeMacro{\childdocof}
Furthermore, add the commands
\begin{center}
\begin{tabular}{l}
|\input{childdoc.def}|\\
|\childdocof{|\textit{main}|}|\\
\end{tabular}
\end{center}
at the top of every child file \textit{child}
which is included by |\include{|\textit{child}|}|
from within the main file
(or at least for those files to be compiled individually).
The argument \textit{main} must be the filename of the main file.

There are a couple of
considerations in setting up the main and child documents:

%%%%%%%%%%%%%%%%%%%%%%%%%%%%%%%%%%%%%%%%
\paragraph{Restrictions.}

Please note the following restrictions:
\begin{itemize}
\item
|\childdocmain| must be called with one argument \textit{main}
to ensure compatibility with earlier version of the package.
It must either be empty (|\childdocmain{}|)
or precisely match the filename of the main file in which it is specified.
See \secref{sec:detection} for further information.
\item
The filename \textit{main} must be specified without the |.tex| extension.
\item
The filename \textit{main} is case sensitive
(even in case-insensitive file systems)
due to internal string comparison.
\item
The argument \textit{main} should be fully expanded, it cannot be a macro.
\item
Subdirectories and special characters should be avoided in filenames.
\item
The command |\childdocmain{|\textit{main}|}| must be followed by a whitespace.
It should not be followed immediately by another command
or by a comment mark `|%|'.
This is because the \TeX{} parser reads the token immediately following
the argument of |\childdocmain| and puts it
at the beginning of every child section;
however, a white\-space is ignored.
\end{itemize}

%%%%%%%%%%%%%%%%%%%%%%%%%%%%%%%%%%%%%%%%
\paragraph{Content of Main File.}

It is advisable to place all content in the child files included by |\include|.
Any output contained in the main file will appear in all child documents
unless suppressed manually;
it cannot be suppressed automatically by the |\includeonly| directive
and thus should normally be avoided.
A method to include some content in the main file
by means of conditional processing is described in \secref{sec:conditional}.

%%%%%%%%%%%%%%%%%%%%%%%%%%%%%%%%%%%%%%%%
\paragraph{Page Numbering.}

When only a part of the document is compiled,
the appropriate numbering of pages
(as well as other status parameters)
is determined from the |.aux| files.
The latter contain information from previous passes.
However this information needs to propagate through
all intermediate child documents.
Therefore the page numbering in child documents may well
be inconsistent until the complete document is compiled at least once.

A useful (if unconventional) way to always ensure a consistent
page numbering is to restart the numbering in each child document
and denote the pages by `\textit{child}|.|\textit{page}'
where \textit{child} represents the chapter/section number of the child file.
This can be achieved by the command
|\numberwithin{page}{|\textit{child}|}|
of the \textsf{amsmath} package
where \textit{child} can be |chapter| or |section|
depending on the chosen structuring.
Alternatively, one can modify the macro |\thepage| appropriately
and reset the counter |page| at the start of each child file.

%%%%%%%%%%%%%%%%%%%%%%%%%%%%%%%%%%%%%%%%%%%%%%%%%%%%%%%%%%%%%%%%%%%%%%%%%%%%%%%%
\subsection{Conditional Processing}
\label{sec:conditional}

The package provides a mechanism to compile different versions
of a document. To customise the versions further some conditional processing
can come in handy to distinguish which version is being compiled.
The package provides two macros to describe the compilation context:

%%%%%%%%%%%%%%%%%%%%%%%%%%%%%%%%%%%%%%%%
\DescribeMacro{\ifchilddoc}
The conditional |\ifchilddoc| distinguishes between the compilation of
child documents and the main document:
%
\begin{center}
|\ifchilddoc |\textit{child-code}| |[|\||else |\textit{main-code}]| \||fi|
\end{center}

%%%%%%%%%%%%%%%%%%%%%%%%%%%%%%%%%%%%%%%%
\DescribeMacro{\childdocname}
\DescribeMacro{\childdocjob}
The macro |\childdocname| contains the filename (without extension)
of the main or child file being processed.
Note that |\childdocjob| will always contain the name of the main file.

%%%%%%%%%%%%%%%%%%%%%%%%%%%%%%%%%%%%%%%%
\paragraph{Title Page.}

Conditional processing can be used to include a title or banner page
in the main document when proper precautions are taken.
Importantly, the code in the main file should ensure that the page counter
(as well as other status parameters which are stored in the |.aux| files)
takes the same value after the conditional processing.
Otherwise the page numbers may take divergent values
depending on which part is compiled.

For example, a title page could be declared by:
%
\begin{center}
\begin{tabular}{l}
|\ifchilddoc\||else|\\
|\addtocounter{page}{-1}|\\
\textit{code for title page}\\
|\newpage|\\
|\||fi|
\end{tabular}
\end{center}
%
A banner page for the child documents can be generated by:
%
\begin{center}
\begin{tabular}{l}
|\ifchilddoc|\\
|\addtocounter{page}{-1}|\\
\textit{code for banner page}\\
|\newpage|\\
|\||fi|
\end{tabular}
\end{center}
%
Here one could write a message such as:
\begin{center}
|This is the part \childdocname{} of \childdocjob{}.|
\end{center}

%%%%%%%%%%%%%%%%%%%%%%%%%%%%%%%%%%%%%%%%%%%%%%%%%%%%%%%%%%%%%%%%%%%%%%%%%%%%%%%%
\subsection{Flags}
\label{sec:flags}

The package makes it easy to generate different versions
of the main or child documents.
To this end compilation flags can be defined
and assigned different default values.
They will be particularly useful in conjunction
with the forwarding mechanism described in \secref{sec:forward}.

For example, it may be useful to have a flag |\version|
which can be set to |draft| or |final|.
The document source will contain some conditional code
depending on the value of |\version|.
Suppose further, the flag should default to |final| for the main file
and to |draft| for child files
which is a natural assignment for editing the document.
This is achieved by placing the following code
in the preamble of the main document
(below the |\childdocmain| directive):
%
\begin{center}
\begin{tabular}{l}
|\ifchilddoc|\\
|\providecommand{\version}{draft}|\\
|\||else|\\
|\providecommand{\version}{final}|\\
|\||fi|
\end{tabular}
\end{center}
%
The definition by |\providecommand| makes sure
that previous definitions are not overwritten.
Further statements |\providecommand{\version}{...}|
can thus be added before the above code to override it.

For the main file, one might add a line
(between |\childdocmain| and the above block)
%
\begin{center}
|%\ifchilddoc\||else\providecommand{\version}{draft}\||fi|
\end{center}
%
which can be uncommented to produce a draft version.
Likewise one can add a line to the very top of a child file
(above the |\childdocof{|\textit{main}|}| directive)
%
\begin{center}
|%\providecommand{\version}{final}|
\end{center}
%
which can be uncommented to produce the final version of this child document.

%%%%%%%%%%%%%%%%%%%%%%%%%%%%%%%%%%%%%%%%%%%%%%%%%%%%%%%%%%%%%%%%%%%%%%%%%%%%%%%%
\subsection{Forwarding}
\label{sec:forward}

Different versions of the main or child documents
using compilation flags as described in \secref{sec:flags}
can be (permanently) stored in different files
for convenient compilation, viewing and distribution.
To this end, the package defines a command
to pass on compilation to a different file:

%%%%%%%%%%%%%%%%%%%%%%%%%%%%%%%%%%%%%%%%
\DescribeMacro{\childdocforward}
The command |\childdocforward| redirects processing to
another source file:
%
\begin{center}
\begin{tabular}{l}
|\input{childdoc.def}|\\
|\childdocforward[|\textit{main}|]{|\textit{dest}|}|\\
\end{tabular}
\end{center}
%
The argument \textit{dest} is the destination file
(without extension).
It should be the main file or one of the child files.
Note that further \textsf{childdoc} directives
such as |\childdocof| and |\childdocforward|
in the indicated file will be processed in this form.
The optional argument \textit{main}
passes on directly to the main file \textit{main}
while pretending to compile the child \textit{dest}.
This form behaves as if \textit{dest}
issues |\childdocof{|\textit{main}|}| right away,
and no further \textsf{childdoc} directives will be processed.

%%%%%%%%%%%%%%%%%%%%%%%%%%%%%%%%%%%%%%%%
\DescribeMacro{\...prefix}
In the alternative form |\childdocforwardprefix|,
%
\begin{center}
\begin{tabular}{l}
|\input{childdoc.def}|\\
|\childdocforwardprefix[|\textit{main}|]{|\textit{prefix}|}{|\textit{dest}|}|
\end{tabular}
\end{center}
%
the destination file is determined by a pattern
depending on the current file:
To make this work, the current file must be called
`{\textit{prefix}\hspace{0.2em}\textit{suffix}}'
with \textit{prefix} matching precisely the argument.
Processing is then passed on to the file
`{\textit{dest}\hspace{0.2em}\textit{suffix}}'.
Surely, the same effect is achieved by
directly specifying the
argument `{\textit{dest}\hspace{0.2em}\textit{suffix}}'
in the first form.
However, that requires to set up a different file
for each child. With the alternative form of the command
all these files can have exactly the same content
which simplifies setting them up and maintaining them.

For example, the following file |draft.tex|
with a compilation flag |\version| as described in \secref{sec:flags}
compiles the main document as a draft:
%
\begin{center}
\begin{tabular}{l}
|\def\version{draft}|\\
|\input{childdoc.def}|\\
|\childdocforward{|\textit{main}|}|
\end{tabular}
\end{center}
%
Likewise, the following files |final|\textit{nn}|.tex|
compile the final version of the child document
|child|\textit{nn}|.tex|:
%
\begin{center}
\begin{tabular}{l}
|\def\version{final}|\\
|\input{childdoc.def}|\\
|\childdocforwardprefix{final}{child}|
\end{tabular}
\end{center}
%

Note that when several versions of a main file and/or of each child file
are to be generated, it may be convenient to set up a |Makefile| or
shell script to automatise the process.

%%%%%%%%%%%%%%%%%%%%%%%%%%%%%%%%%%%%%%%%%%%%%%%%%%%%%%%%%%%%%%%%%%%%%%%%%%%%%%%%
\subsection{Command Line Processing}
\label{sec:commandline}

The effect of redirection files can also be achieved by invoking
the \LaTeX{} compiler with a more elaborate command line.
Most conveniently this should be done as part
of a shell script or a |Makefile|.

When using \textsf{childdoc} in the main file, the following
command lines effectively perform a redirection
(note that depending on the shell being used,
backslashes may have to be doubled: `|\|' $\to$ `|\\|'):
%
\begin{center}
|... -jobname "|\textit{target}|" |\\|"|[\textit{flags}]%
|\input{childdoc.def}\childdocforward[|\textit{main}|]{|\textit{dest}|}"|
\end{center}
%
Here \textit{target} is the name of the output file,
\textit{main} is the name of the main file
and \textit{dest} is the name of the main or child file to be processed
(all filenames without extensions).
The optional argument \textit{main} can be omitted
if \textit{main} matches \textit{dest}.
Optionally, compilation \textit{flags} can be defined via |\def| commands.
This command line makes the \TeX{} engine believe
it is compiling the file \textit{target}
whose content is specified as the latter parameter.
The provided code then forwards the processing to
\textit{main} or \textit{dest} as described in \secref{sec:forward}.

%%%%%%%%%%%%%%%%%%%%%%%%%%%%%%%%%%%%%%%%%%%%%%%%%%%%%%%%%%%%%%%%%%%%%%%%%%%%%%%%
\subsection{Include by Input}
\label{sec:input}

Including child documents by |\include| has some restrictions by design.
Most notably, the content of a child document always occupies
its own set of pages; pages cannot be shared between child documents.
Usually, this behaviour makes perfect sense
because each child document contain an essential part of the document.
However, in some situations it may be desirable to compose
a document from a collection of parts
without having mandatory page breaks between then.
For this case, the package
provides a mechanism to include parts
by |\input| which can also be processed individually.
However, by construction this mechanism
requires manual handling of the content to be output.

%%%%%%%%%%%%%%%%%%%%%%%%%%%%%%%%%%%%%%%%
\DescribeMacro{\ifchilddocmanual}
The main file should be prepared as usual, see \secref{sec:include}.
However, the document body must make a distinction
between processing of an individual part and of the main document, e.g.:
%
\begin{center}
\begin{tabular}{l}
|\ifchilddocmanual|\\
|\input{\childdocname}|\\
|\||else|\\
\textit{document body with }|\input{|\textit{part}|}|\\
|\||fi|
\end{tabular}
\end{center}
%
The conditional |\ifchilddocmanual| is true whenever
a part to be included by |\input| is being compiled,
and the name of the part is stored in |\childdocname|.

%%%%%%%%%%%%%%%%%%%%%%%%%%%%%%%%%%%%%%%%
\DescribeMacro{\childdocby}
Each part to be included by |\input| should start with:
%
\begin{center}
\begin{tabular}{l}
|\input{childdoc.def}|\\
|\childdocby{|\textit{main}|}|\\
\end{tabular}
\end{center}
%
The directive |\childdocby| is similar to |\childdocof|
described in \secref{sec:include},
but the subsequent selection of content must be done manually.
To that end, both |\ifchilddoc| and |\ifchilddocmanual|
will be true upon processing of a part,
and the name of the part is stored in |\childdocname|.
Note that |\jobname| will be set to the filename of the current part
so that each part receives an individual |.aux| file
that does not interfere with the |.aux| file(s) of the main document.
This behaviour can be altered by the alternative form
|\childdocby[*]{|\textit{main}|}| (with a non-empty optional argument)
which uses the |.aux| file of the main document
by setting |\jobname| to \textit{main}.

%%%%%%%%%%%%%%%%%%%%%%%%%%%%%%%%%%%%%%%%%%%%%%%%%%%%%%%%%%%%%%%%%%%%%%%%%%%%%%%%
\subsection{Driver Development}
\label{sec:driver}

The \textsf{childdoc} mechanism can also be use for the development
of definition files such as \LaTeX{} styles or classes.
This case differs from the above setup with multiple parts
included by |\include| in that no |\includeonly| should be invoked.
This can be achieved by starting the include file
(before |\ProvidesPackage|) with:
%
\begin{center}
\begin{tabular}{l}
|\input{childdoc.def}|\\
|\childdocforward{|\textit{main}|}|\\
\end{tabular}
\end{center}
%
or alternatively with:
%
\begin{center}
\begin{tabular}{l}
|\input{childdoc.def}|\\
|\childdocby{|\textit{main}|}|\\
\end{tabular}
\end{center}
%
Both forms have slightly different effects as described above.
The main file is prepared as usual, see \secref{sec:include}.

%%%%%%%%%%%%%%%%%%%%%%%%%%%%%%%%%%%%%%%%%%%%%%%%%%%%%%%%%%%%%%%%%%%%%%%%%%%%%%%%
\subsection{Legacy Detection}
\label{sec:detection}

The directive |\childdocmain| in the main file can detect
whether the complete document or merely a child is to be compiled
even without using the directive |\childdocof|.
This method is deprecated because it is less robust
and there is no compelling reason to use it;
it is merely provided for backward compatibility
and it may be removed in future versions.

If the detection mechanism is to be used,
it is mandatory to correctly specify
the filename of the main file as the argument of |\childdocmain|:
%
\begin{center}
\begin{tabular}{l}
|\input{childdoc.def}|\\
|\childdocmain{|\textit{main}|}|\\
\end{tabular}
\end{center}
%
If |\jobname| does not match the argument \textit{main} of |\childdocmain|,
it is assumed that |\jobname| points to the child file to be compiled.
When using |\childdocmain| with the main file specified as argument,
it suffices to start a child file
with just |\input{|\textit{main}|}|
without loading of the package and using |\childdocof|.
If instead all processing is done
with the appropriate \textsf{childdoc} directives,
the argument of \textit{main} of |\childdocmain| can be empty.

An alternative version of the command line processing described
in \secref{sec:commandline} using the detection mechanism reads:
%
\begin{center}
|... -jobname "|\textit{target}|" "|[\textit{flags}]%
[|\def\jobname{|\textit{dest}|}|]|\input{|\textit{main}|}"|
\end{center}

%%%%%%%%%%%%%%%%%%%%%%%%%%%%%%%%%%%%%%%%%%%%%%%%%%%%%%%%%%%%%%%%%%%%%%%%%%%%%%%%
\subsection{Manual Code}
\label{sec:manual}

In case one cannot be certain whether the definitions file |childdoc.def|
is installed on the target \TeX{} distribution
and one prefers not to ship it,
it is conceivable to paste a few relevant commands into the sources.

To that end, drop all statements |\input{childdoc.def}|
and perform the replacements as outlined below.
Instead of |\childdocmain{|\textit{main}|}| add the following code
to the top of the main file:
%
\begin{center}
\begin{tabular}{l}
|\||ifdefined\childdocname\endinput\||fi\newif\ifchilddoc|\\
|\edef\childdocname{\scantokens\expandafter{\jobname\noexpand}}|\\
|\def\childdocmain{|\textit{main}|}\||ifx\childdocmain\childdocname\||else|\\
|\childdoctrue\includeonly{\childdocname}\let\jobname\childdocmain\||fi|\\
\end{tabular}
\end{center}
%
Instead of |\childdocof{|\textit{main}|}| just include the main file
at the top of each child file:
%
\begin{center}
|\input{|\textit{main}|}|
\end{center}
%
A simple redirection |\childdocforward{|\textit{dest}|}| is achieved by:
%
\begin{center}
|\def\jobname{|\textit{dest}|}\input{\jobname}|
\end{center}
%
The redirection with prefix
|\childdocforwardprefix[|\textit{prefix}|]{|\textit{dest}|}|
is accomplished by:
%
\begin{center}
\begin{tabular}{l}
|{\edef\jobname{\scantokens\expandafter{\jobname\noexpand}}|\\
|\def\redirectjob |\textit{prefix}|#1~~~{\gdef\jobname{|\textit{dest}|#1}}|\\
|\expandafter\redirectjob\jobname~~~}\input{\jobname}|
\end{tabular}
\end{center}

In an alternative approach,
child documents can be compiled by a specific command line
without additional code or specific definitions:
%
\begin{center}
|... -jobname "|\textit{target}|" "|[\textit{flags}]%
|\includeonly{|\textit{dest}|}\input{|\textit{main}|}"|
\end{center}
%

%%%%%%%%%%%%%%%%%%%%%%%%%%%%%%%%%%%%%%%%%%%%%%%%%%%%%%%%%%%%%%%%%%%%%%%%%%%%%%%%
%%%%%%%%%%%%%%%%%%%%%%%%%%%%%%%%%%%%%%%%%%%%%%%%%%%%%%%%%%%%%%%%%%%%%%%%%%%%%%%%
\section{Information}

%%%%%%%%%%%%%%%%%%%%%%%%%%%%%%%%%%%%%%%%%%%%%%%%%%%%%%%%%%%%%%%%%%%%%%%%%%%%%%%%
\subsection{Copyright}

Copyright \copyright{} 2017--2018 Niklas Beisert

This work may be distributed and/or modified under the
conditions of the \LaTeX{} Project Public License, either version 1.3
of this license or (at your option) any later version.
The latest version of this license is in
  \url{http://www.latex-project.org/lppl.txt}
and version 1.3 or later is part of all distributions of \LaTeX{}
version 2005/12/01 or later.

This work has the LPPL maintenance status `maintained'.

The Current Maintainer of this work is Niklas Beisert.

This work consists of the files |README.txt|, |childdoc.ins| and |childdoc.dtx|
as well as the derived files |childdoc.def|, |cdocsamp.tex|
with |cdocsch1.tex|, |cdocsch2.tex|, |cdocspt3.tex|, |cdocspt4.tex|,
|cdocsdrf.tex|, |cdocsfn1.tex|, |cdocsfn2.tex|
as well as |childdoc.pdf|.

%%%%%%%%%%%%%%%%%%%%%%%%%%%%%%%%%%%%%%%%%%%%%%%%%%%%%%%%%%%%%%%%%%%%%%%%%%%%%%%%
\subsection{Files and Installation}

The package consists of the files:
%
\begin{center}
\begin{tabular}{ll}
    |README.txt|   & readme file \\
    |childdoc.ins| & installation file \\
    |childdoc.dtx| & source file \\
    |childdoc.def| & definition file \\
    |cdocsamp.tex| & sample main file \\
    |cdocsch1.tex| & sample include file \\
    |cdocsch2.tex| & sample include file \\
    |cdocspt3.tex| & sample part file \\
    |cdocspt4.tex| & sample part file \\
    |cdocsdrf.tex| & sample redirection file \\
    |cdocsfn1.tex| & sample redirection file \\
    |cdocsfn2.tex| & sample redirection file \\
    |childdoc.pdf| & manual
\end{tabular}
\end{center}
%
The distribution consists of the files
|README.txt|, |childdoc.ins| and |childdoc.dtx|.
%
\begin{itemize}
\item
Run (pdf)\LaTeX{} on |childdoc.dtx|
to compile the manual |childdoc.pdf| (this file).
\item
Run \LaTeX{} on |childdoc.ins| to create the definitions file |childdoc.def|
and the sample |cdocsamp.tex| with include files
|cdocsch1.tex|, |cdocsch2.tex|, |cdocspt3.tex|, |cdocspt4.tex|,
|cdocsdrf.tex|, |cdocsfn1.tex|, |cdocsfn2.tex|.
Then copy the file |childdoc.def| to an appropriate directory of your \LaTeX{}
distribution, e.g.\ \textit{texmf-root}|/tex/latex/childdoc|.
\end{itemize}

%%%%%%%%%%%%%%%%%%%%%%%%%%%%%%%%%%%%%%%%%%%%%%%%%%%%%%%%%%%%%%%%%%%%%%%%%%%%%%%%
\subsection{Related CTAN Packages}

There are several other packages which offer a similar functionality:
%
\begin{itemize}
\item
The packages
\href{http://ctan.org/pkg/docmute}{\textsf{docmute}},
\href{http://ctan.org/pkg/includex}{\textsf{includex}} and
\href{http://ctan.org/pkg/standalone}{\textsf{standalone}}
provide commands to include only the document body of
a child file thus allowing both files to be compiled individually.
\item
The packages \href{http://ctan.org/pkg/subdocs}{\textsf{subdocs}}
and \href{http://ctan.org/pkg/subfiles}{\textsf{subfiles}}
provide structures in which the main and child documents can be
encapsulated and allowing them to be compiled individually.
The inclusion mechanism is different from the conventional |\include|.
\item
The package \href{http://ctan.org/pkg/combine}{\textsf{combine}}
is an elaborate solution to combine several documents into one.
\end{itemize}
%
See also the CTAN topic \href{http://ctan.org/topic/subdocs}{\textsf{subdocs}}
for further related packages.
The present package differs from the above solutions in that
a document structure constructed with the conventional |\include| mechanism
just needs two extra commands at the top of every file
such that all constituent files can be compiled individually.

%%%%%%%%%%%%%%%%%%%%%%%%%%%%%%%%%%%%%%%%%%%%%%%%%%%%%%%%%%%%%%%%%%%%%%%%%%%%%%%%
%\subsection{Feature Suggestions}
%
%The following is a list of features which may be useful for future
%versions of this package:
%%
%\begin{itemize}
%\item
%\ldots
%\end{itemize}

%%%%%%%%%%%%%%%%%%%%%%%%%%%%%%%%%%%%%%%%%%%%%%%%%%%%%%%%%%%%%%%%%%%%%%%%%%%%%%%%
\subsection{Revision History}

%%%%%%%%%%%%%%%%%%%%%%%%%%%%%%%%%%%%%%%%
\paragraph{v2.0:} 2018/12/30

\begin{itemize}
\item
immediate forward processing
\item
added |\childdocby| mechanism
\item
manual restructured
\end{itemize}

%%%%%%%%%%%%%%%%%%%%%%%%%%%%%%%%%%%%%%%%
\paragraph{v1.6:} 2018/01/17

\begin{itemize}
\item
application for development of include files
\item
corrections to manual
\end{itemize}

%%%%%%%%%%%%%%%%%%%%%%%%%%%%%%%%%%%%%%%%
\paragraph{v1.5:} 2017/05/21

\begin{itemize}
\item
more complete structuring introduced
\item
|\childdocof| introduced
\item
|\childdoc| renamed to |\childdocmain|
\item
|\childredirect| renamed to |\childdocforward| and |\childdocforwardprefix|
and functionality expanded
\end{itemize}

%%%%%%%%%%%%%%%%%%%%%%%%%%%%%%%%%%%%%%%%
\paragraph{v1.0:} 2017/04/27

\begin{itemize}
\item
manual and install package
\item
first version published on CTAN
\end{itemize}

%%%%%%%%%%%%%%%%%%%%%%%%%%%%%%%%%%%%%%%%
\paragraph{v0.6:} 2017/04/26

\begin{itemize}
\item
redirection mechanism added
\end{itemize}

%%%%%%%%%%%%%%%%%%%%%%%%%%%%%%%%%%%%%%%%
\paragraph{v0.5:} 2017/04/26

\begin{itemize}
\item
functionality in definition file
\end{itemize}


%%%%%%%%%%%%%%%%%%%%%%%%%%%%%%%%%%%%%%%%%%%%%%%%%%%%%%%%%%%%%%%%%%%%%%%%%%%%%%%%
%%%%%%%%%%%%%%%%%%%%%%%%%%%%%%%%%%%%%%%%%%%%%%%%%%%%%%%%%%%%%%%%%%%%%%%%%%%%%%%%
%%%%%%%%%%%%%%%%%%%%%%%%%%%%%%%%%%%%%%%%%%%%%%%%%%%%%%%%%%%%%%%%%%%%%%%%%%%%%%%%
\appendix

\settowidth\MacroIndent{\rmfamily\scriptsize 000\ }

 \DocInput{childdoc.dtx}

\end{document}
%</driver>
% \fi
%
% %%%%%%%%%%%%%%%%%%%%%%%%%%%%%%%%%%%%%%%%%%%%%%%%%%%%%%%%%%%%%%%%%%%%%%%%%%%%%%
% %%%%%%%%%%%%%%%%%%%%%%%%%%%%%%%%%%%%%%%%%%%%%%%%%%%%%%%%%%%%%%%%%%%%%%%%%%%%%%
% \section{Sample}
%\iffalse
%<*samplemain>
%\fi
%
% The following presents a sample document
% with two chapters, two parts, a title page,
% a compile flag as well as three forwarding files to set the flag.
% It consists of eight |.tex| files:
% \begin{center}
% \begin{tabular}{ll}
% |cdocsamp.tex|&main file\\
% |cdocsch1.tex|&include file for chapter 1\\
% |cdocsch2.tex|&include file for chapter 2\\
% |cdocspt3.tex|&include file for part 3\\
% |cdocspt4.tex|&include file for part 4\\
% |cdocsdrf.tex|&forwarding file for main file in draft mode\\
% |cdocsfi1.tex|&forwarding file for final version of chapter 1\\
% |cdocsfi2.tex|&forwarding file for final version of chapter 2\\
% \end{tabular}
% \end{center}
% Each of the eight files can be compiled directly by the \LaTeX{} compiler.
%
% %%%%%%%%%%%%%%%%%%%%%%%%%%%%%%%%%%%%%%
% \paragraph{Main File.}
%
% The main file is called |cdocsamp.tex|.
%
% Load the \textsf{childdoc} definitions and
% declare the filename for the main document:
%    \begin{macrocode}
\input{childdoc.def}
\childdocmain{}
%    \end{macrocode}

% Optional override for |\version| flag:
%    \begin{macrocode}
%%\ifchilddoc\else\providecommand{\version}{draft}\fi
%    \end{macrocode}

% Define the default values for the |\version| flag
% (|final| for the main file and |draft| for childs):
%    \begin{macrocode}
\ifchilddoc
\providecommand{\version}{draft}
\else
\providecommand{\version}{final}
\fi
%    \end{macrocode}

% Load the standard document class:
%    \begin{macrocode}
\documentclass[12pt]{article}
%    \end{macrocode}

% Start the document body:
%    \begin{macrocode}
\begin{document}
%    \end{macrocode}

% Declare a title page.
% Print title, part of document being processed and version flag:
%    \begin{macrocode}
\addtocounter{page}{-1}
\begin{center}
{\LARGE\bfseries{}childdoc example\par}
\vspace{1cm}
\ifchilddoc
\ifchilddocmanual part\else chapter\fi:
`\childdocname' of `\childdocjob'\par
\else
main document: `\childdocjob'\par
\fi
version: \version\par
\end{center}
\newpage
%    \end{macrocode}

% Manually include selected file,
% otherwise process as usual:
%    \begin{macrocode}
\ifchilddocmanual
\section*{part `\childdocname'}
\input{\childdocname}
\else
%    \end{macrocode}

% Include the two chapters:
%    \begin{macrocode}
\include{cdocsch1}
\include{cdocsch2}
%    \end{macrocode}

% Include the two parts unless only chapters should be displayed:
%    \begin{macrocode}
\ifchilddoc\else
\section{part three}
\input{cdocspt3}
\section{part four}
\input{cdocspt4}
\fi
%    \end{macrocode}

% Process as usual until here:
%    \begin{macrocode}
\fi
%    \end{macrocode}

% End of document body:
%    \begin{macrocode}
\end{document}
%    \end{macrocode}
%\iffalse
%</samplemain>
%\fi
%
% %%%%%%%%%%%%%%%%%%%%%%%%%%%%%%%%%%%%%%
% \paragraph{Chapter Include Files.}
%
% The include files are called |cdocsch1.tex| and |cdocsch2.tex|.
%
%\iffalse
%<*samplechap1|samplechap2>
%\fi

% Optional override for |\version| flag:
%    \begin{macrocode}
%%\providecommand{\version}{final}
%    \end{macrocode}

% Include the main document:
%    \begin{macrocode}
\input{childdoc.def}
\childdocof{cdocsamp}
%    \end{macrocode}

%\iffalse
%</samplechap1|samplechap2>
%\fi
%
%\iffalse
%<*samplechap1>
%\fi
% Some text for chapter 1:
%    \begin{macrocode}
\section{one}
some text in chapter one
%    \end{macrocode}

%\iffalse
%</samplechap1>
%\fi
% Some text for chapter 2:
%\iffalse
%<*samplechap2>
%\fi
%    \begin{macrocode}
\section{two}
more text in chapter two
%    \end{macrocode}

%\iffalse
%</samplechap2>
%\fi
%
% %%%%%%%%%%%%%%%%%%%%%%%%%%%%%%%%%%%%%%
% \paragraph{Part Include Files.}
%
% The include files are called |cdocspt3.tex| and |cdocspt4.tex|.
%
%\iffalse
%<*samplepart3|samplepart4>
%\fi

% Optional override for |\version| flag:
%    \begin{macrocode}
%%\providecommand{\version}{final}
%    \end{macrocode}

% Include the main document:
%    \begin{macrocode}
\input{childdoc.def}
\childdocby{cdocsamp}
%    \end{macrocode}

%\iffalse
%</samplepart3|samplepart4>
%\fi
%
%\iffalse
%<*samplepart3>
%\fi
% Some text for part 3:
%    \begin{macrocode}
some text in part three
%    \end{macrocode}

%\iffalse
%</samplepart3>
%\fi
% Some text for part 4:
%\iffalse
%<*samplepart4>
%\fi
%    \begin{macrocode}
more text in part four
%    \end{macrocode}

%\iffalse
%</samplepart4>
%\fi
%
% %%%%%%%%%%%%%%%%%%%%%%%%%%%%%%%%%%%%%%
% \paragraph{Forwarding for a Complete Draft.}
%
% The following forwarding file |cdocsdrf.tex|
% compiles the main document in draft mode:
%\iffalse
%<*sampledraft>
%\fi
%    \begin{macrocode}
\def\version{draft}
\input{childdoc.def}
\childdocforward{cdocsamp}
%    \end{macrocode}

%\iffalse
%</sampledraft>
%\fi
%
% %%%%%%%%%%%%%%%%%%%%%%%%%%%%%%%%%%%%%%
% \paragraph{Forwarding for Final Version of the Chapters.}
%
% The following forwarding files |cdocsfn1.tex| and |cdocsfn2.tex|
% (with identical content)
% compile the final versions of the child documents
% |cdocsch1.tex| and |cdocsch2.tex|, respectively:
%\iffalse
%<*samplefinal>
%\fi
%    \begin{macrocode}
\def\version{final}
\input{childdoc.def}
\childdocforwardprefix[cdocsamp]{cdocsfn}{cdocsch}
%    \end{macrocode}

%\iffalse
%</samplefinal>
%\fi
%
% %%%%%%%%%%%%%%%%%%%%%%%%%%%%%%%%%%%%%%
% \paragraph{Command Line Processing.}
%
% The following three command lines generate the output files
% |cdocscld|, |cdocscl1| and |cdocscl2|
% which should be identical to
% |cdocsdrf|, |cdocsch1| and |cdocsfn2|, respectively:
% \begin{center}
% \begin{tabular}{l}
% |latex -jobname cdocscld \|\\
% |  "\def\version{draft}\input{childdoc.def}\childdocforward{cdocsamp}"|\\
% |latex -jobname cdocscl1 \|\\
% |  "\input{childdoc.def}\childdocforward[cdocsamp]{cdocsch1}"|\\
% |latex -jobname cdocscl2 \|\\
% |  "\def\version{final}\input{childdoc.def}\childdocforward{cdocsch2}"|
% \end{tabular}
% \end{center}
% Note that the trailing backslash on each first line
% merely continues the input to the second line
% (for convenient cut ant paste).
% Furthermore, the command |latex| can be replaced by any
% of its alternative versions such as |pdflatex|.
%
% %%%%%%%%%%%%%%%%%%%%%%%%%%%%%%%%%%%%%%%%%%%%%%%%%%%%%%%%%%%%%%%%%%%%%%%%%%%%%%
% %%%%%%%%%%%%%%%%%%%%%%%%%%%%%%%%%%%%%%%%%%%%%%%%%%%%%%%%%%%%%%%%%%%%%%%%%%%%%%
% \section{Implementation}
%\iffalse
%<*package>
%\fi
%
% This section describes the definitions file |childdoc.def|.

% The definitions cannot be loaded using |\usepackage| or |\RequirePackage|
% which has a mechanism to prevent loading a style file more than once.
% When loading the definitions by means of |\input|
% multiple instances have to be prevented manually:
%\iffalse
%This code needs to be before the `\ProvidesFile' directive
%which is defined at the beginning of this file.
%Therefore it is also placed there and commented out here.
%</package>
%<*discard>
%\fi
%    \begin{macrocode}
\ifdefined\childdocmain\endinput\fi
%    \end{macrocode}
%\iffalse
%</discard>
%<*package>
%\fi
%
% \macro{\ifchilddoc}
% \macro{\ifchilddocmanual}
% The conditional |\ifchilddoc| tells whether a
% child (true) or main (false) document is being compiled.
% The conditional |\ifchilddocmanual| tells whether
% the |\includeonly| mechanism is used (false) or
% the selection of child files must be performed manually (true).
% The definitions initialise to false:
%    \begin{macrocode}
\newif\ifchilddoc
\newif\ifchilddocmanual
%    \end{macrocode}

% \macro{\childdocname}
% \macro{\childdocjob}
% The macro |\childdocname| stores the name of the main document
% to be compiled. The macro |\childdocjob| stores the name of
% the document on which the \LaTeX{} compiler was originally invoked.
% The content of |\jobname| cannot be compared
% to filenames specified in the source due to different catcodes.
% The following code rescans |\jobname|, stores the result
% in |\childdocname| and saves a copy in |\childdocjob|:
%    \begin{macrocode}
\edef\childdocname{\scantokens\expandafter{\jobname\noexpand}}
\let\childdocjob\childdocname
%    \end{macrocode}

% \macro{\childdocdisable}
% The macro |\childdocdisable| prevents the main file
% from being processed more than once.
% At this stage, the main document command |\childdocmain|
% is assumed to be called once again where it should do nothing.
% Any subsequent call to it should prevent
% a secondary processing of the main document
% It overwrites the forwarding commands
% |\childdocof| and |\childdocforward|
% with empty macros to prevent further inclusions of the main document:
%    \begin{macrocode}
\newcommand{\childdocdisable}
{
  \renewcommand{\childdocmain}[1]{\renewcommand{\childdocmain}[1]{\endinput}}
  \renewcommand{\childdocof}[1]{}
  \renewcommand{\childdocby}[2][]{}
  \renewcommand{\childdocforward}[2][]{}
  \renewcommand{\childdocdisable}{}
}
%    \end{macrocode}

% \macro{\childdocmain}
% The macro |\childdocmain| is to be called at the top of the main file
% with nothing or the main filename (without extension) as argument.
% First, it breaks loops.
% If the argument is not empty and does not match |\childdocname|
% (which is set by the first inclusion of |childdoc.def|),
% |\ifchilddoc| is set to true, |\includeonly| is applied to the child file
% and |\jobname| is set to the main file
% (for proper handling of |.aux| files):
%    \begin{macrocode}
\newcommand{\childdocmain}[1]
{
  \childdocdisable\childdocmain{}
  \if?#1?\else
    \begingroup
      \def\childdoctmp{#1}
      \ifx\childdoctmp\childdocname
        \def\childdoctmp{}
      \else
        \def\childdoctmp
        {
          \childdoctrue
          \includeonly{\childdocname}
          \def\childdocjob{#1}
          \def\jobname{#1}
        }
      \fi
      \expandafter
    \endgroup
    \childdoctmp
  \fi
}
%    \end{macrocode}

% \macro{\childdocof}
% The command |\childdocof| redirects
% compilation to the main file |#1|.
%    \begin{macrocode}
\newcommand{\childdocof}[1]
{
  \childdocdisable
  \childdoctrue
  \includeonly{\childdocname}
  \def\jobname{#1}
  \def\childdocjob{#1}
  \input{#1}
}
%    \end{macrocode}

% \macro{\childdocby}
% The command |\childdocby| ....
%    \begin{macrocode}
\newcommand{\childdocby}[2][]
{
  \childdocdisable
  \childdoctrue
  \childdocmanualtrue
  \if?#1?\else
    \def\jobname{#2}
  \fi
  \def\childdocjob{#2}
  \input{#2}
  \endinput
}
%    \end{macrocode}

% \macro{\childdocforward}
% The command |\childdocforward| redirects
% compilation to the main file or
% (if the optional argument is given) a child file.
% Parameters are set as if the main file
% or a child file starting with |\childdocof| was compiled.
% Then compilation is handed over to the main file:
%    \begin{macrocode}
\newcommand{\childdocforward}[2][]
{
  \begingroup
    \if?#1?
      \def\childdoctmp
      {
        \def\childdocname{#2}
        \def\childdocjob{#2}
        \def\jobname{#2}
        \input{#2}
        \endinput
      }
    \else
      \def\childdoctmp
      {
        \childdocdisable
        \def\childdocname{#2}
        \childdoctrue
        \includeonly{#2}
        \def\childdocjob{#1}
        \def\jobname{#1}
        \input{#1}
        \endinput
      }
    \fi
    \expandafter
  \endgroup
  \childdoctmp
}
%    \end{macrocode}

% \macro{\childdocforwardprefix}
% The command |\childdocforwardprefix| redirects
% compilation to the main or a child file by means of a pattern.
% The prefix |#1| in the current filename is replaced by |#2|
% and the suffix of the current filename is kept
% (it is assumed that the filename does not contain the substring `|~~~|'
% which is used as a delimiter).
% Compilation is handed over to the new file by |\childdocforward|:
%    \begin{macrocode}
\newcommand{\childdocforwardprefix}[3][]
{
  \begingroup
    \def\childdocextract #2##1~~~{\def\childdoctmp{\childdocforward[#1]{#3##1}}}
    \expandafter\childdocextract\childdocname~~~
    \expandafter
  \endgroup
  \childdoctmp
}
%    \end{macrocode}

% \macro{\childdoc}
% The deprecated macro |\childdoc| is a legacy version of |\childdocmain|:
%    \begin{macrocode}
\newcommand{\childdoc}{\childdocmain}
%    \end{macrocode}

% \macro{\childdocredirect}
% The deprecated macro |\childdocredirect| is a legacy version
% of |\childdocforward| and |\childdocforwardprefix|:
%    \begin{macrocode}
\newcommand{\childdocredirect}[2][]
{
  \begingroup
    \if?#1?
      \def\childdoctmp{\childdocforward{#2}}
    \else
      \def\childdoctmp{\childdocforwardprefix{#1}{#2}}
    \fi
    \expandafter
  \endgroup
  \childdoctmp
}
%    \end{macrocode}

%\iffalse
%</package>
%\fi
%
\endinput
|
and perform the replacements as outlined below.
Instead of |\childdocmain{|\textit{main}|}| add the following code
to the top of the main file:
%
\begin{center}
\begin{tabular}{l}
|\||ifdefined\childdocname\endinput\||fi\newif\ifchilddoc|\\
|\edef\childdocname{\scantokens\expandafter{\jobname\noexpand}}|\\
|\def\childdocmain{|\textit{main}|}\||ifx\childdocmain\childdocname\||else|\\
|\childdoctrue\includeonly{\childdocname}\let\jobname\childdocmain\||fi|\\
\end{tabular}
\end{center}
%
Instead of |\childdocof{|\textit{main}|}| just include the main file
at the top of each child file:
%
\begin{center}
|\input{|\textit{main}|}|
\end{center}
%
A simple redirection |\childdocforward{|\textit{dest}|}| is achieved by:
%
\begin{center}
|\def\jobname{|\textit{dest}|}\input{\jobname}|
\end{center}
%
The redirection with prefix
|\childdocforwardprefix[|\textit{prefix}|]{|\textit{dest}|}|
is accomplished by:
%
\begin{center}
\begin{tabular}{l}
|{\edef\jobname{\scantokens\expandafter{\jobname\noexpand}}|\\
|\def\redirectjob |\textit{prefix}|#1~~~{\gdef\jobname{|\textit{dest}|#1}}|\\
|\expandafter\redirectjob\jobname~~~}\input{\jobname}|
\end{tabular}
\end{center}

In an alternative approach,
child documents can be compiled by a specific command line
without additional code or specific definitions:
%
\begin{center}
|... -jobname "|\textit{target}|" "|[\textit{flags}]%
|\includeonly{|\textit{dest}|}\input{|\textit{main}|}"|
\end{center}
%

%%%%%%%%%%%%%%%%%%%%%%%%%%%%%%%%%%%%%%%%%%%%%%%%%%%%%%%%%%%%%%%%%%%%%%%%%%%%%%%%
%%%%%%%%%%%%%%%%%%%%%%%%%%%%%%%%%%%%%%%%%%%%%%%%%%%%%%%%%%%%%%%%%%%%%%%%%%%%%%%%
\section{Information}

%%%%%%%%%%%%%%%%%%%%%%%%%%%%%%%%%%%%%%%%%%%%%%%%%%%%%%%%%%%%%%%%%%%%%%%%%%%%%%%%
\subsection{Copyright}

Copyright \copyright{} 2017--2018 Niklas Beisert

This work may be distributed and/or modified under the
conditions of the \LaTeX{} Project Public License, either version 1.3
of this license or (at your option) any later version.
The latest version of this license is in
  \url{http://www.latex-project.org/lppl.txt}
and version 1.3 or later is part of all distributions of \LaTeX{}
version 2005/12/01 or later.

This work has the LPPL maintenance status `maintained'.

The Current Maintainer of this work is Niklas Beisert.

This work consists of the files |README.txt|, |childdoc.ins| and |childdoc.dtx|
as well as the derived files |childdoc.def|, |cdocsamp.tex|
with |cdocsch1.tex|, |cdocsch2.tex|, |cdocspt3.tex|, |cdocspt4.tex|,
|cdocsdrf.tex|, |cdocsfn1.tex|, |cdocsfn2.tex|
as well as |childdoc.pdf|.

%%%%%%%%%%%%%%%%%%%%%%%%%%%%%%%%%%%%%%%%%%%%%%%%%%%%%%%%%%%%%%%%%%%%%%%%%%%%%%%%
\subsection{Files and Installation}

The package consists of the files:
%
\begin{center}
\begin{tabular}{ll}
    |README.txt|   & readme file \\
    |childdoc.ins| & installation file \\
    |childdoc.dtx| & source file \\
    |childdoc.def| & definition file \\
    |cdocsamp.tex| & sample main file \\
    |cdocsch1.tex| & sample include file \\
    |cdocsch2.tex| & sample include file \\
    |cdocspt3.tex| & sample part file \\
    |cdocspt4.tex| & sample part file \\
    |cdocsdrf.tex| & sample redirection file \\
    |cdocsfn1.tex| & sample redirection file \\
    |cdocsfn2.tex| & sample redirection file \\
    |childdoc.pdf| & manual
\end{tabular}
\end{center}
%
The distribution consists of the files
|README.txt|, |childdoc.ins| and |childdoc.dtx|.
%
\begin{itemize}
\item
Run (pdf)\LaTeX{} on |childdoc.dtx|
to compile the manual |childdoc.pdf| (this file).
\item
Run \LaTeX{} on |childdoc.ins| to create the definitions file |childdoc.def|
and the sample |cdocsamp.tex| with include files
|cdocsch1.tex|, |cdocsch2.tex|, |cdocspt3.tex|, |cdocspt4.tex|,
|cdocsdrf.tex|, |cdocsfn1.tex|, |cdocsfn2.tex|.
Then copy the file |childdoc.def| to an appropriate directory of your \LaTeX{}
distribution, e.g.\ \textit{texmf-root}|/tex/latex/childdoc|.
\end{itemize}

%%%%%%%%%%%%%%%%%%%%%%%%%%%%%%%%%%%%%%%%%%%%%%%%%%%%%%%%%%%%%%%%%%%%%%%%%%%%%%%%
\subsection{Related CTAN Packages}

There are several other packages which offer a similar functionality:
%
\begin{itemize}
\item
The packages
\href{http://ctan.org/pkg/docmute}{\textsf{docmute}},
\href{http://ctan.org/pkg/includex}{\textsf{includex}} and
\href{http://ctan.org/pkg/standalone}{\textsf{standalone}}
provide commands to include only the document body of
a child file thus allowing both files to be compiled individually.
\item
The packages \href{http://ctan.org/pkg/subdocs}{\textsf{subdocs}}
and \href{http://ctan.org/pkg/subfiles}{\textsf{subfiles}}
provide structures in which the main and child documents can be
encapsulated and allowing them to be compiled individually.
The inclusion mechanism is different from the conventional |\include|.
\item
The package \href{http://ctan.org/pkg/combine}{\textsf{combine}}
is an elaborate solution to combine several documents into one.
\end{itemize}
%
See also the CTAN topic \href{http://ctan.org/topic/subdocs}{\textsf{subdocs}}
for further related packages.
The present package differs from the above solutions in that
a document structure constructed with the conventional |\include| mechanism
just needs two extra commands at the top of every file
such that all constituent files can be compiled individually.

%%%%%%%%%%%%%%%%%%%%%%%%%%%%%%%%%%%%%%%%%%%%%%%%%%%%%%%%%%%%%%%%%%%%%%%%%%%%%%%%
%\subsection{Feature Suggestions}
%
%The following is a list of features which may be useful for future
%versions of this package:
%%
%\begin{itemize}
%\item
%\ldots
%\end{itemize}

%%%%%%%%%%%%%%%%%%%%%%%%%%%%%%%%%%%%%%%%%%%%%%%%%%%%%%%%%%%%%%%%%%%%%%%%%%%%%%%%
\subsection{Revision History}

%%%%%%%%%%%%%%%%%%%%%%%%%%%%%%%%%%%%%%%%
\paragraph{v2.0:} 2018/12/30

\begin{itemize}
\item
immediate forward processing
\item
added |\childdocby| mechanism
\item
manual restructured
\end{itemize}

%%%%%%%%%%%%%%%%%%%%%%%%%%%%%%%%%%%%%%%%
\paragraph{v1.6:} 2018/01/17

\begin{itemize}
\item
application for development of include files
\item
corrections to manual
\end{itemize}

%%%%%%%%%%%%%%%%%%%%%%%%%%%%%%%%%%%%%%%%
\paragraph{v1.5:} 2017/05/21

\begin{itemize}
\item
more complete structuring introduced
\item
|\childdocof| introduced
\item
|\childdoc| renamed to |\childdocmain|
\item
|\childredirect| renamed to |\childdocforward| and |\childdocforwardprefix|
and functionality expanded
\end{itemize}

%%%%%%%%%%%%%%%%%%%%%%%%%%%%%%%%%%%%%%%%
\paragraph{v1.0:} 2017/04/27

\begin{itemize}
\item
manual and install package
\item
first version published on CTAN
\end{itemize}

%%%%%%%%%%%%%%%%%%%%%%%%%%%%%%%%%%%%%%%%
\paragraph{v0.6:} 2017/04/26

\begin{itemize}
\item
redirection mechanism added
\end{itemize}

%%%%%%%%%%%%%%%%%%%%%%%%%%%%%%%%%%%%%%%%
\paragraph{v0.5:} 2017/04/26

\begin{itemize}
\item
functionality in definition file
\end{itemize}


%%%%%%%%%%%%%%%%%%%%%%%%%%%%%%%%%%%%%%%%%%%%%%%%%%%%%%%%%%%%%%%%%%%%%%%%%%%%%%%%
%%%%%%%%%%%%%%%%%%%%%%%%%%%%%%%%%%%%%%%%%%%%%%%%%%%%%%%%%%%%%%%%%%%%%%%%%%%%%%%%
%%%%%%%%%%%%%%%%%%%%%%%%%%%%%%%%%%%%%%%%%%%%%%%%%%%%%%%%%%%%%%%%%%%%%%%%%%%%%%%%
\appendix

\settowidth\MacroIndent{\rmfamily\scriptsize 000\ }

 \DocInput{childdoc.dtx}

\end{document}
%</driver>
% \fi
%
% %%%%%%%%%%%%%%%%%%%%%%%%%%%%%%%%%%%%%%%%%%%%%%%%%%%%%%%%%%%%%%%%%%%%%%%%%%%%%%
% %%%%%%%%%%%%%%%%%%%%%%%%%%%%%%%%%%%%%%%%%%%%%%%%%%%%%%%%%%%%%%%%%%%%%%%%%%%%%%
% \section{Sample}
%\iffalse
%<*samplemain>
%\fi
%
% The following presents a sample document
% with two chapters, two parts, a title page,
% a compile flag as well as three forwarding files to set the flag.
% It consists of eight |.tex| files:
% \begin{center}
% \begin{tabular}{ll}
% |cdocsamp.tex|&main file\\
% |cdocsch1.tex|&include file for chapter 1\\
% |cdocsch2.tex|&include file for chapter 2\\
% |cdocspt3.tex|&include file for part 3\\
% |cdocspt4.tex|&include file for part 4\\
% |cdocsdrf.tex|&forwarding file for main file in draft mode\\
% |cdocsfi1.tex|&forwarding file for final version of chapter 1\\
% |cdocsfi2.tex|&forwarding file for final version of chapter 2\\
% \end{tabular}
% \end{center}
% Each of the eight files can be compiled directly by the \LaTeX{} compiler.
%
% %%%%%%%%%%%%%%%%%%%%%%%%%%%%%%%%%%%%%%
% \paragraph{Main File.}
%
% The main file is called |cdocsamp.tex|.
%
% Load the \textsf{childdoc} definitions and
% declare the filename for the main document:
%    \begin{macrocode}
% \iffalse
%
% childdoc.dtx Copyright (C) 2017-2018 Niklas Beisert
%
% This work may be distributed and/or modified under the
% conditions of the LaTeX Project Public License, either version 1.3
% of this license or (at your option) any later version.
% The latest version of this license is in
%   http://www.latex-project.org/lppl.txt
% and version 1.3 or later is part of all distributions of LaTeX
% version 2005/12/01 or later.
%
% This work has the LPPL maintenance status `maintained'.
%
% The Current Maintainer of this work is Niklas Beisert.
%
% This work consists of the files childdoc.dtx and childdoc.ins
% and the derived files childdoc.def and cdocsamp.tex with
% cdocsch1.tex, cdocsch2.tex, cdocsdrf.tex, cdocsfn1.tex, cdocsfn2.tex.
%
%<package>\ifdefined\childdocmain\endinput\fi
%<package>\ProvidesFile{childdoc.def}[2018/12/30 v2.0 child document driver]
%<samplemain>\ProvidesFile{cdocsamp.tex}[2018/12/30 v2.0 sample for childdoc]
%<*driver>
%\ProvidesFile{childdoc.drv}[2018/12/30 v2.0 childdoc reference manual file]
\PassOptionsToClass{10pt,a4paper}{article}
\documentclass{ltxdoc}

\usepackage[margin=35mm]{geometry}
\usepackage{hyperref}
\usepackage{hyperxmp}
\usepackage[usenames]{color}

\hypersetup{colorlinks=true}
\hypersetup{pdfstartview=FitH}
\hypersetup{pdfpagemode=UseNone}
\hypersetup{pdfsource={}}
\hypersetup{pdflang={en-UK}}
\hypersetup{pdfcopyright={Copyright 2017-2018 Niklas Beisert.
  This work may be distributed and/or modified under the
  conditions of the LaTeX Project Public License, either version 1.3
  of this license or (at your option) any later version.}}
\hypersetup{pdflicenseurl={http://www.latex-project.org/lppl.txt}}
\hypersetup{pdfcontactaddress={ETH Zurich, ITP, HIT K,
  Wolfgang-Pauli-Strasse 27}}
\hypersetup{pdfcontactpostcode={8093}}
\hypersetup{pdfcontactcity={Zurich}}
\hypersetup{pdfcontactcountry={Switzerland}}
\hypersetup{pdfcontactemail={nbeisert@itp.phys.ethz.ch}}
\hypersetup{pdfcontacturl={http://people.phys.ethz.ch/\xmptilde nbeisert/}}

\newcommand{\secref}[1]{\hyperref[#1]{section \ref*{#1}}}

\parskip1ex
\parindent0pt
\let\olditemize\itemize
\def\itemize{\olditemize\parskip0pt}

\begin{document}

\title{The \textsf{childdoc} Package}
\hypersetup{pdftitle={The childdoc Package}}
\author{Niklas Beisert\\[2ex]
  Institut f\"ur Theoretische Physik\\
  Eidgen\"ossische Technische Hochschule Z\"urich\\
  Wolfgang-Pauli-Strasse 27, 8093 Z\"urich, Switzerland\\[1ex]
  \href{mailto:nbeisert@itp.phys.ethz.ch}
  {\texttt{nbeisert@itp.phys.ethz.ch}}}
\hypersetup{pdfauthor={Niklas Beisert}}
\hypersetup{pdfsubject={Manual for the LaTeX2e Package childdoc}}
\date{30 December 2018, \textsf{v2.0}}
\maketitle

\begin{abstract}\noindent
\textsf{childdoc} is a \LaTeXe{} package
that enables the direct compilation
of document sections included by |\include|
to individual files.
\end{abstract}

\begingroup
\parskip0ex
\tableofcontents
\endgroup

%%%%%%%%%%%%%%%%%%%%%%%%%%%%%%%%%%%%%%%%%%%%%%%%%%%%%%%%%%%%%%%%%%%%%%%%%%%%%%%%
%%%%%%%%%%%%%%%%%%%%%%%%%%%%%%%%%%%%%%%%%%%%%%%%%%%%%%%%%%%%%%%%%%%%%%%%%%%%%%%%
\section{Introduction}

\LaTeX{} provides a mechanism to structure a large document (such as a book)
into a main file and several child files (containing the chapters)
using the |\include| command.
This mechanism is beneficial for documents
which span hundreds of pages in order to
make the source file(s) more manageable.
Moreover, compilation can be restricted to
selected child files by means of the |\includeonly| command.
The latter feature can be used to reduce the compilation time while editing
(this was significantly more useful in the earlier days of \LaTeX{})
or to generate a smaller document which is easier to navigate.
Another application of |\includeonly| is to generate
documents consisting of selected parts of the complete document.

However, there are a few drawbacks of the plain |\include| mechanism:
\begin{itemize}
\item
The child files cannot be compiled on their own,
they can only be compiled via the main file.
A naive editing environment
(such as a text editor with an option
to have the current file processed by \LaTeX)
may require one to switch to the main file before compiling;
attempting to compile the child file produces errors.
\item
The main file must be modified (each time)
to adjust the |\includeonly| command
to the present needs. This easily leaves the main file in a messy state.
\item
The generated document will always carry the filename
of the main document. This is inconvenient if
several child files are to be compiled and
to be kept for distribution.
\end{itemize}

The present package provides a simple interface
to make child files individually compilable by \LaTeX{}.
Compiling a child file then has the same effect as compiling
the main file with an |\includeonly| command
to select the appropriate child.
Moreover the generated document will carry the name of the child
rather than the main file.
This resolves all three above issues.

This feature is meant to make the editing of books,
thesis documents and lecture notes somewhat more convenient.
However, the package can also be used efficiently for
composing a series of documents (such as exercise sheets)
which are typically distributed individually.
It then assists the author in generating the individual documents
(potentially in different versions)
as well as a document containing the collected series.
Another application is in developing style files
or other kinds of included material
where compilation of the style file could redirect
to a sample or test file.

%%%%%%%%%%%%%%%%%%%%%%%%%%%%%%%%%%%%%%%%%%%%%%%%%%%%%%%%%%%%%%%%%%%%%%%%%%%%%%%%
%%%%%%%%%%%%%%%%%%%%%%%%%%%%%%%%%%%%%%%%%%%%%%%%%%%%%%%%%%%%%%%%%%%%%%%%%%%%%%%%
\section{Usage}

First of all, the package \textsf{childdoc} is \emph{not} a standard
\LaTeXe{} |.sty| style file! Therefore it needs to be invoked in
a non-standard way.

%%%%%%%%%%%%%%%%%%%%%%%%%%%%%%%%%%%%%%%%%%%%%%%%%%%%%%%%%%%%%%%%%%%%%%%%%%%%%%%%
\subsection{Included Files}
\label{sec:include}

%%%%%%%%%%%%%%%%%%%%%%%%%%%%%%%%%%%%%%%%
\DescribeMacro{\childdocmain}
To use the package, add the commands
\begin{center}
\begin{tabular}{l}
|\input{childdoc.def}|\\
|\childdocmain{}|\\
\end{tabular}
\end{center}
at the very top of the main \LaTeX{} file,
in particular \emph{before} the |\documentclass| statement!
The argument of |\childdocmain| should be left empty
(but it must be present).

%%%%%%%%%%%%%%%%%%%%%%%%%%%%%%%%%%%%%%%%
\DescribeMacro{\childdocof}
Furthermore, add the commands
\begin{center}
\begin{tabular}{l}
|\input{childdoc.def}|\\
|\childdocof{|\textit{main}|}|\\
\end{tabular}
\end{center}
at the top of every child file \textit{child}
which is included by |\include{|\textit{child}|}|
from within the main file
(or at least for those files to be compiled individually).
The argument \textit{main} must be the filename of the main file.

There are a couple of
considerations in setting up the main and child documents:

%%%%%%%%%%%%%%%%%%%%%%%%%%%%%%%%%%%%%%%%
\paragraph{Restrictions.}

Please note the following restrictions:
\begin{itemize}
\item
|\childdocmain| must be called with one argument \textit{main}
to ensure compatibility with earlier version of the package.
It must either be empty (|\childdocmain{}|)
or precisely match the filename of the main file in which it is specified.
See \secref{sec:detection} for further information.
\item
The filename \textit{main} must be specified without the |.tex| extension.
\item
The filename \textit{main} is case sensitive
(even in case-insensitive file systems)
due to internal string comparison.
\item
The argument \textit{main} should be fully expanded, it cannot be a macro.
\item
Subdirectories and special characters should be avoided in filenames.
\item
The command |\childdocmain{|\textit{main}|}| must be followed by a whitespace.
It should not be followed immediately by another command
or by a comment mark `|%|'.
This is because the \TeX{} parser reads the token immediately following
the argument of |\childdocmain| and puts it
at the beginning of every child section;
however, a white\-space is ignored.
\end{itemize}

%%%%%%%%%%%%%%%%%%%%%%%%%%%%%%%%%%%%%%%%
\paragraph{Content of Main File.}

It is advisable to place all content in the child files included by |\include|.
Any output contained in the main file will appear in all child documents
unless suppressed manually;
it cannot be suppressed automatically by the |\includeonly| directive
and thus should normally be avoided.
A method to include some content in the main file
by means of conditional processing is described in \secref{sec:conditional}.

%%%%%%%%%%%%%%%%%%%%%%%%%%%%%%%%%%%%%%%%
\paragraph{Page Numbering.}

When only a part of the document is compiled,
the appropriate numbering of pages
(as well as other status parameters)
is determined from the |.aux| files.
The latter contain information from previous passes.
However this information needs to propagate through
all intermediate child documents.
Therefore the page numbering in child documents may well
be inconsistent until the complete document is compiled at least once.

A useful (if unconventional) way to always ensure a consistent
page numbering is to restart the numbering in each child document
and denote the pages by `\textit{child}|.|\textit{page}'
where \textit{child} represents the chapter/section number of the child file.
This can be achieved by the command
|\numberwithin{page}{|\textit{child}|}|
of the \textsf{amsmath} package
where \textit{child} can be |chapter| or |section|
depending on the chosen structuring.
Alternatively, one can modify the macro |\thepage| appropriately
and reset the counter |page| at the start of each child file.

%%%%%%%%%%%%%%%%%%%%%%%%%%%%%%%%%%%%%%%%%%%%%%%%%%%%%%%%%%%%%%%%%%%%%%%%%%%%%%%%
\subsection{Conditional Processing}
\label{sec:conditional}

The package provides a mechanism to compile different versions
of a document. To customise the versions further some conditional processing
can come in handy to distinguish which version is being compiled.
The package provides two macros to describe the compilation context:

%%%%%%%%%%%%%%%%%%%%%%%%%%%%%%%%%%%%%%%%
\DescribeMacro{\ifchilddoc}
The conditional |\ifchilddoc| distinguishes between the compilation of
child documents and the main document:
%
\begin{center}
|\ifchilddoc |\textit{child-code}| |[|\||else |\textit{main-code}]| \||fi|
\end{center}

%%%%%%%%%%%%%%%%%%%%%%%%%%%%%%%%%%%%%%%%
\DescribeMacro{\childdocname}
\DescribeMacro{\childdocjob}
The macro |\childdocname| contains the filename (without extension)
of the main or child file being processed.
Note that |\childdocjob| will always contain the name of the main file.

%%%%%%%%%%%%%%%%%%%%%%%%%%%%%%%%%%%%%%%%
\paragraph{Title Page.}

Conditional processing can be used to include a title or banner page
in the main document when proper precautions are taken.
Importantly, the code in the main file should ensure that the page counter
(as well as other status parameters which are stored in the |.aux| files)
takes the same value after the conditional processing.
Otherwise the page numbers may take divergent values
depending on which part is compiled.

For example, a title page could be declared by:
%
\begin{center}
\begin{tabular}{l}
|\ifchilddoc\||else|\\
|\addtocounter{page}{-1}|\\
\textit{code for title page}\\
|\newpage|\\
|\||fi|
\end{tabular}
\end{center}
%
A banner page for the child documents can be generated by:
%
\begin{center}
\begin{tabular}{l}
|\ifchilddoc|\\
|\addtocounter{page}{-1}|\\
\textit{code for banner page}\\
|\newpage|\\
|\||fi|
\end{tabular}
\end{center}
%
Here one could write a message such as:
\begin{center}
|This is the part \childdocname{} of \childdocjob{}.|
\end{center}

%%%%%%%%%%%%%%%%%%%%%%%%%%%%%%%%%%%%%%%%%%%%%%%%%%%%%%%%%%%%%%%%%%%%%%%%%%%%%%%%
\subsection{Flags}
\label{sec:flags}

The package makes it easy to generate different versions
of the main or child documents.
To this end compilation flags can be defined
and assigned different default values.
They will be particularly useful in conjunction
with the forwarding mechanism described in \secref{sec:forward}.

For example, it may be useful to have a flag |\version|
which can be set to |draft| or |final|.
The document source will contain some conditional code
depending on the value of |\version|.
Suppose further, the flag should default to |final| for the main file
and to |draft| for child files
which is a natural assignment for editing the document.
This is achieved by placing the following code
in the preamble of the main document
(below the |\childdocmain| directive):
%
\begin{center}
\begin{tabular}{l}
|\ifchilddoc|\\
|\providecommand{\version}{draft}|\\
|\||else|\\
|\providecommand{\version}{final}|\\
|\||fi|
\end{tabular}
\end{center}
%
The definition by |\providecommand| makes sure
that previous definitions are not overwritten.
Further statements |\providecommand{\version}{...}|
can thus be added before the above code to override it.

For the main file, one might add a line
(between |\childdocmain| and the above block)
%
\begin{center}
|%\ifchilddoc\||else\providecommand{\version}{draft}\||fi|
\end{center}
%
which can be uncommented to produce a draft version.
Likewise one can add a line to the very top of a child file
(above the |\childdocof{|\textit{main}|}| directive)
%
\begin{center}
|%\providecommand{\version}{final}|
\end{center}
%
which can be uncommented to produce the final version of this child document.

%%%%%%%%%%%%%%%%%%%%%%%%%%%%%%%%%%%%%%%%%%%%%%%%%%%%%%%%%%%%%%%%%%%%%%%%%%%%%%%%
\subsection{Forwarding}
\label{sec:forward}

Different versions of the main or child documents
using compilation flags as described in \secref{sec:flags}
can be (permanently) stored in different files
for convenient compilation, viewing and distribution.
To this end, the package defines a command
to pass on compilation to a different file:

%%%%%%%%%%%%%%%%%%%%%%%%%%%%%%%%%%%%%%%%
\DescribeMacro{\childdocforward}
The command |\childdocforward| redirects processing to
another source file:
%
\begin{center}
\begin{tabular}{l}
|\input{childdoc.def}|\\
|\childdocforward[|\textit{main}|]{|\textit{dest}|}|\\
\end{tabular}
\end{center}
%
The argument \textit{dest} is the destination file
(without extension).
It should be the main file or one of the child files.
Note that further \textsf{childdoc} directives
such as |\childdocof| and |\childdocforward|
in the indicated file will be processed in this form.
The optional argument \textit{main}
passes on directly to the main file \textit{main}
while pretending to compile the child \textit{dest}.
This form behaves as if \textit{dest}
issues |\childdocof{|\textit{main}|}| right away,
and no further \textsf{childdoc} directives will be processed.

%%%%%%%%%%%%%%%%%%%%%%%%%%%%%%%%%%%%%%%%
\DescribeMacro{\...prefix}
In the alternative form |\childdocforwardprefix|,
%
\begin{center}
\begin{tabular}{l}
|\input{childdoc.def}|\\
|\childdocforwardprefix[|\textit{main}|]{|\textit{prefix}|}{|\textit{dest}|}|
\end{tabular}
\end{center}
%
the destination file is determined by a pattern
depending on the current file:
To make this work, the current file must be called
`{\textit{prefix}\hspace{0.2em}\textit{suffix}}'
with \textit{prefix} matching precisely the argument.
Processing is then passed on to the file
`{\textit{dest}\hspace{0.2em}\textit{suffix}}'.
Surely, the same effect is achieved by
directly specifying the
argument `{\textit{dest}\hspace{0.2em}\textit{suffix}}'
in the first form.
However, that requires to set up a different file
for each child. With the alternative form of the command
all these files can have exactly the same content
which simplifies setting them up and maintaining them.

For example, the following file |draft.tex|
with a compilation flag |\version| as described in \secref{sec:flags}
compiles the main document as a draft:
%
\begin{center}
\begin{tabular}{l}
|\def\version{draft}|\\
|\input{childdoc.def}|\\
|\childdocforward{|\textit{main}|}|
\end{tabular}
\end{center}
%
Likewise, the following files |final|\textit{nn}|.tex|
compile the final version of the child document
|child|\textit{nn}|.tex|:
%
\begin{center}
\begin{tabular}{l}
|\def\version{final}|\\
|\input{childdoc.def}|\\
|\childdocforwardprefix{final}{child}|
\end{tabular}
\end{center}
%

Note that when several versions of a main file and/or of each child file
are to be generated, it may be convenient to set up a |Makefile| or
shell script to automatise the process.

%%%%%%%%%%%%%%%%%%%%%%%%%%%%%%%%%%%%%%%%%%%%%%%%%%%%%%%%%%%%%%%%%%%%%%%%%%%%%%%%
\subsection{Command Line Processing}
\label{sec:commandline}

The effect of redirection files can also be achieved by invoking
the \LaTeX{} compiler with a more elaborate command line.
Most conveniently this should be done as part
of a shell script or a |Makefile|.

When using \textsf{childdoc} in the main file, the following
command lines effectively perform a redirection
(note that depending on the shell being used,
backslashes may have to be doubled: `|\|' $\to$ `|\\|'):
%
\begin{center}
|... -jobname "|\textit{target}|" |\\|"|[\textit{flags}]%
|\input{childdoc.def}\childdocforward[|\textit{main}|]{|\textit{dest}|}"|
\end{center}
%
Here \textit{target} is the name of the output file,
\textit{main} is the name of the main file
and \textit{dest} is the name of the main or child file to be processed
(all filenames without extensions).
The optional argument \textit{main} can be omitted
if \textit{main} matches \textit{dest}.
Optionally, compilation \textit{flags} can be defined via |\def| commands.
This command line makes the \TeX{} engine believe
it is compiling the file \textit{target}
whose content is specified as the latter parameter.
The provided code then forwards the processing to
\textit{main} or \textit{dest} as described in \secref{sec:forward}.

%%%%%%%%%%%%%%%%%%%%%%%%%%%%%%%%%%%%%%%%%%%%%%%%%%%%%%%%%%%%%%%%%%%%%%%%%%%%%%%%
\subsection{Include by Input}
\label{sec:input}

Including child documents by |\include| has some restrictions by design.
Most notably, the content of a child document always occupies
its own set of pages; pages cannot be shared between child documents.
Usually, this behaviour makes perfect sense
because each child document contain an essential part of the document.
However, in some situations it may be desirable to compose
a document from a collection of parts
without having mandatory page breaks between then.
For this case, the package
provides a mechanism to include parts
by |\input| which can also be processed individually.
However, by construction this mechanism
requires manual handling of the content to be output.

%%%%%%%%%%%%%%%%%%%%%%%%%%%%%%%%%%%%%%%%
\DescribeMacro{\ifchilddocmanual}
The main file should be prepared as usual, see \secref{sec:include}.
However, the document body must make a distinction
between processing of an individual part and of the main document, e.g.:
%
\begin{center}
\begin{tabular}{l}
|\ifchilddocmanual|\\
|\input{\childdocname}|\\
|\||else|\\
\textit{document body with }|\input{|\textit{part}|}|\\
|\||fi|
\end{tabular}
\end{center}
%
The conditional |\ifchilddocmanual| is true whenever
a part to be included by |\input| is being compiled,
and the name of the part is stored in |\childdocname|.

%%%%%%%%%%%%%%%%%%%%%%%%%%%%%%%%%%%%%%%%
\DescribeMacro{\childdocby}
Each part to be included by |\input| should start with:
%
\begin{center}
\begin{tabular}{l}
|\input{childdoc.def}|\\
|\childdocby{|\textit{main}|}|\\
\end{tabular}
\end{center}
%
The directive |\childdocby| is similar to |\childdocof|
described in \secref{sec:include},
but the subsequent selection of content must be done manually.
To that end, both |\ifchilddoc| and |\ifchilddocmanual|
will be true upon processing of a part,
and the name of the part is stored in |\childdocname|.
Note that |\jobname| will be set to the filename of the current part
so that each part receives an individual |.aux| file
that does not interfere with the |.aux| file(s) of the main document.
This behaviour can be altered by the alternative form
|\childdocby[*]{|\textit{main}|}| (with a non-empty optional argument)
which uses the |.aux| file of the main document
by setting |\jobname| to \textit{main}.

%%%%%%%%%%%%%%%%%%%%%%%%%%%%%%%%%%%%%%%%%%%%%%%%%%%%%%%%%%%%%%%%%%%%%%%%%%%%%%%%
\subsection{Driver Development}
\label{sec:driver}

The \textsf{childdoc} mechanism can also be use for the development
of definition files such as \LaTeX{} styles or classes.
This case differs from the above setup with multiple parts
included by |\include| in that no |\includeonly| should be invoked.
This can be achieved by starting the include file
(before |\ProvidesPackage|) with:
%
\begin{center}
\begin{tabular}{l}
|\input{childdoc.def}|\\
|\childdocforward{|\textit{main}|}|\\
\end{tabular}
\end{center}
%
or alternatively with:
%
\begin{center}
\begin{tabular}{l}
|\input{childdoc.def}|\\
|\childdocby{|\textit{main}|}|\\
\end{tabular}
\end{center}
%
Both forms have slightly different effects as described above.
The main file is prepared as usual, see \secref{sec:include}.

%%%%%%%%%%%%%%%%%%%%%%%%%%%%%%%%%%%%%%%%%%%%%%%%%%%%%%%%%%%%%%%%%%%%%%%%%%%%%%%%
\subsection{Legacy Detection}
\label{sec:detection}

The directive |\childdocmain| in the main file can detect
whether the complete document or merely a child is to be compiled
even without using the directive |\childdocof|.
This method is deprecated because it is less robust
and there is no compelling reason to use it;
it is merely provided for backward compatibility
and it may be removed in future versions.

If the detection mechanism is to be used,
it is mandatory to correctly specify
the filename of the main file as the argument of |\childdocmain|:
%
\begin{center}
\begin{tabular}{l}
|\input{childdoc.def}|\\
|\childdocmain{|\textit{main}|}|\\
\end{tabular}
\end{center}
%
If |\jobname| does not match the argument \textit{main} of |\childdocmain|,
it is assumed that |\jobname| points to the child file to be compiled.
When using |\childdocmain| with the main file specified as argument,
it suffices to start a child file
with just |\input{|\textit{main}|}|
without loading of the package and using |\childdocof|.
If instead all processing is done
with the appropriate \textsf{childdoc} directives,
the argument of \textit{main} of |\childdocmain| can be empty.

An alternative version of the command line processing described
in \secref{sec:commandline} using the detection mechanism reads:
%
\begin{center}
|... -jobname "|\textit{target}|" "|[\textit{flags}]%
[|\def\jobname{|\textit{dest}|}|]|\input{|\textit{main}|}"|
\end{center}

%%%%%%%%%%%%%%%%%%%%%%%%%%%%%%%%%%%%%%%%%%%%%%%%%%%%%%%%%%%%%%%%%%%%%%%%%%%%%%%%
\subsection{Manual Code}
\label{sec:manual}

In case one cannot be certain whether the definitions file |childdoc.def|
is installed on the target \TeX{} distribution
and one prefers not to ship it,
it is conceivable to paste a few relevant commands into the sources.

To that end, drop all statements |\input{childdoc.def}|
and perform the replacements as outlined below.
Instead of |\childdocmain{|\textit{main}|}| add the following code
to the top of the main file:
%
\begin{center}
\begin{tabular}{l}
|\||ifdefined\childdocname\endinput\||fi\newif\ifchilddoc|\\
|\edef\childdocname{\scantokens\expandafter{\jobname\noexpand}}|\\
|\def\childdocmain{|\textit{main}|}\||ifx\childdocmain\childdocname\||else|\\
|\childdoctrue\includeonly{\childdocname}\let\jobname\childdocmain\||fi|\\
\end{tabular}
\end{center}
%
Instead of |\childdocof{|\textit{main}|}| just include the main file
at the top of each child file:
%
\begin{center}
|\input{|\textit{main}|}|
\end{center}
%
A simple redirection |\childdocforward{|\textit{dest}|}| is achieved by:
%
\begin{center}
|\def\jobname{|\textit{dest}|}\input{\jobname}|
\end{center}
%
The redirection with prefix
|\childdocforwardprefix[|\textit{prefix}|]{|\textit{dest}|}|
is accomplished by:
%
\begin{center}
\begin{tabular}{l}
|{\edef\jobname{\scantokens\expandafter{\jobname\noexpand}}|\\
|\def\redirectjob |\textit{prefix}|#1~~~{\gdef\jobname{|\textit{dest}|#1}}|\\
|\expandafter\redirectjob\jobname~~~}\input{\jobname}|
\end{tabular}
\end{center}

In an alternative approach,
child documents can be compiled by a specific command line
without additional code or specific definitions:
%
\begin{center}
|... -jobname "|\textit{target}|" "|[\textit{flags}]%
|\includeonly{|\textit{dest}|}\input{|\textit{main}|}"|
\end{center}
%

%%%%%%%%%%%%%%%%%%%%%%%%%%%%%%%%%%%%%%%%%%%%%%%%%%%%%%%%%%%%%%%%%%%%%%%%%%%%%%%%
%%%%%%%%%%%%%%%%%%%%%%%%%%%%%%%%%%%%%%%%%%%%%%%%%%%%%%%%%%%%%%%%%%%%%%%%%%%%%%%%
\section{Information}

%%%%%%%%%%%%%%%%%%%%%%%%%%%%%%%%%%%%%%%%%%%%%%%%%%%%%%%%%%%%%%%%%%%%%%%%%%%%%%%%
\subsection{Copyright}

Copyright \copyright{} 2017--2018 Niklas Beisert

This work may be distributed and/or modified under the
conditions of the \LaTeX{} Project Public License, either version 1.3
of this license or (at your option) any later version.
The latest version of this license is in
  \url{http://www.latex-project.org/lppl.txt}
and version 1.3 or later is part of all distributions of \LaTeX{}
version 2005/12/01 or later.

This work has the LPPL maintenance status `maintained'.

The Current Maintainer of this work is Niklas Beisert.

This work consists of the files |README.txt|, |childdoc.ins| and |childdoc.dtx|
as well as the derived files |childdoc.def|, |cdocsamp.tex|
with |cdocsch1.tex|, |cdocsch2.tex|, |cdocspt3.tex|, |cdocspt4.tex|,
|cdocsdrf.tex|, |cdocsfn1.tex|, |cdocsfn2.tex|
as well as |childdoc.pdf|.

%%%%%%%%%%%%%%%%%%%%%%%%%%%%%%%%%%%%%%%%%%%%%%%%%%%%%%%%%%%%%%%%%%%%%%%%%%%%%%%%
\subsection{Files and Installation}

The package consists of the files:
%
\begin{center}
\begin{tabular}{ll}
    |README.txt|   & readme file \\
    |childdoc.ins| & installation file \\
    |childdoc.dtx| & source file \\
    |childdoc.def| & definition file \\
    |cdocsamp.tex| & sample main file \\
    |cdocsch1.tex| & sample include file \\
    |cdocsch2.tex| & sample include file \\
    |cdocspt3.tex| & sample part file \\
    |cdocspt4.tex| & sample part file \\
    |cdocsdrf.tex| & sample redirection file \\
    |cdocsfn1.tex| & sample redirection file \\
    |cdocsfn2.tex| & sample redirection file \\
    |childdoc.pdf| & manual
\end{tabular}
\end{center}
%
The distribution consists of the files
|README.txt|, |childdoc.ins| and |childdoc.dtx|.
%
\begin{itemize}
\item
Run (pdf)\LaTeX{} on |childdoc.dtx|
to compile the manual |childdoc.pdf| (this file).
\item
Run \LaTeX{} on |childdoc.ins| to create the definitions file |childdoc.def|
and the sample |cdocsamp.tex| with include files
|cdocsch1.tex|, |cdocsch2.tex|, |cdocspt3.tex|, |cdocspt4.tex|,
|cdocsdrf.tex|, |cdocsfn1.tex|, |cdocsfn2.tex|.
Then copy the file |childdoc.def| to an appropriate directory of your \LaTeX{}
distribution, e.g.\ \textit{texmf-root}|/tex/latex/childdoc|.
\end{itemize}

%%%%%%%%%%%%%%%%%%%%%%%%%%%%%%%%%%%%%%%%%%%%%%%%%%%%%%%%%%%%%%%%%%%%%%%%%%%%%%%%
\subsection{Related CTAN Packages}

There are several other packages which offer a similar functionality:
%
\begin{itemize}
\item
The packages
\href{http://ctan.org/pkg/docmute}{\textsf{docmute}},
\href{http://ctan.org/pkg/includex}{\textsf{includex}} and
\href{http://ctan.org/pkg/standalone}{\textsf{standalone}}
provide commands to include only the document body of
a child file thus allowing both files to be compiled individually.
\item
The packages \href{http://ctan.org/pkg/subdocs}{\textsf{subdocs}}
and \href{http://ctan.org/pkg/subfiles}{\textsf{subfiles}}
provide structures in which the main and child documents can be
encapsulated and allowing them to be compiled individually.
The inclusion mechanism is different from the conventional |\include|.
\item
The package \href{http://ctan.org/pkg/combine}{\textsf{combine}}
is an elaborate solution to combine several documents into one.
\end{itemize}
%
See also the CTAN topic \href{http://ctan.org/topic/subdocs}{\textsf{subdocs}}
for further related packages.
The present package differs from the above solutions in that
a document structure constructed with the conventional |\include| mechanism
just needs two extra commands at the top of every file
such that all constituent files can be compiled individually.

%%%%%%%%%%%%%%%%%%%%%%%%%%%%%%%%%%%%%%%%%%%%%%%%%%%%%%%%%%%%%%%%%%%%%%%%%%%%%%%%
%\subsection{Feature Suggestions}
%
%The following is a list of features which may be useful for future
%versions of this package:
%%
%\begin{itemize}
%\item
%\ldots
%\end{itemize}

%%%%%%%%%%%%%%%%%%%%%%%%%%%%%%%%%%%%%%%%%%%%%%%%%%%%%%%%%%%%%%%%%%%%%%%%%%%%%%%%
\subsection{Revision History}

%%%%%%%%%%%%%%%%%%%%%%%%%%%%%%%%%%%%%%%%
\paragraph{v2.0:} 2018/12/30

\begin{itemize}
\item
immediate forward processing
\item
added |\childdocby| mechanism
\item
manual restructured
\end{itemize}

%%%%%%%%%%%%%%%%%%%%%%%%%%%%%%%%%%%%%%%%
\paragraph{v1.6:} 2018/01/17

\begin{itemize}
\item
application for development of include files
\item
corrections to manual
\end{itemize}

%%%%%%%%%%%%%%%%%%%%%%%%%%%%%%%%%%%%%%%%
\paragraph{v1.5:} 2017/05/21

\begin{itemize}
\item
more complete structuring introduced
\item
|\childdocof| introduced
\item
|\childdoc| renamed to |\childdocmain|
\item
|\childredirect| renamed to |\childdocforward| and |\childdocforwardprefix|
and functionality expanded
\end{itemize}

%%%%%%%%%%%%%%%%%%%%%%%%%%%%%%%%%%%%%%%%
\paragraph{v1.0:} 2017/04/27

\begin{itemize}
\item
manual and install package
\item
first version published on CTAN
\end{itemize}

%%%%%%%%%%%%%%%%%%%%%%%%%%%%%%%%%%%%%%%%
\paragraph{v0.6:} 2017/04/26

\begin{itemize}
\item
redirection mechanism added
\end{itemize}

%%%%%%%%%%%%%%%%%%%%%%%%%%%%%%%%%%%%%%%%
\paragraph{v0.5:} 2017/04/26

\begin{itemize}
\item
functionality in definition file
\end{itemize}


%%%%%%%%%%%%%%%%%%%%%%%%%%%%%%%%%%%%%%%%%%%%%%%%%%%%%%%%%%%%%%%%%%%%%%%%%%%%%%%%
%%%%%%%%%%%%%%%%%%%%%%%%%%%%%%%%%%%%%%%%%%%%%%%%%%%%%%%%%%%%%%%%%%%%%%%%%%%%%%%%
%%%%%%%%%%%%%%%%%%%%%%%%%%%%%%%%%%%%%%%%%%%%%%%%%%%%%%%%%%%%%%%%%%%%%%%%%%%%%%%%
\appendix

\settowidth\MacroIndent{\rmfamily\scriptsize 000\ }

 \DocInput{childdoc.dtx}

\end{document}
%</driver>
% \fi
%
% %%%%%%%%%%%%%%%%%%%%%%%%%%%%%%%%%%%%%%%%%%%%%%%%%%%%%%%%%%%%%%%%%%%%%%%%%%%%%%
% %%%%%%%%%%%%%%%%%%%%%%%%%%%%%%%%%%%%%%%%%%%%%%%%%%%%%%%%%%%%%%%%%%%%%%%%%%%%%%
% \section{Sample}
%\iffalse
%<*samplemain>
%\fi
%
% The following presents a sample document
% with two chapters, two parts, a title page,
% a compile flag as well as three forwarding files to set the flag.
% It consists of eight |.tex| files:
% \begin{center}
% \begin{tabular}{ll}
% |cdocsamp.tex|&main file\\
% |cdocsch1.tex|&include file for chapter 1\\
% |cdocsch2.tex|&include file for chapter 2\\
% |cdocspt3.tex|&include file for part 3\\
% |cdocspt4.tex|&include file for part 4\\
% |cdocsdrf.tex|&forwarding file for main file in draft mode\\
% |cdocsfi1.tex|&forwarding file for final version of chapter 1\\
% |cdocsfi2.tex|&forwarding file for final version of chapter 2\\
% \end{tabular}
% \end{center}
% Each of the eight files can be compiled directly by the \LaTeX{} compiler.
%
% %%%%%%%%%%%%%%%%%%%%%%%%%%%%%%%%%%%%%%
% \paragraph{Main File.}
%
% The main file is called |cdocsamp.tex|.
%
% Load the \textsf{childdoc} definitions and
% declare the filename for the main document:
%    \begin{macrocode}
\input{childdoc.def}
\childdocmain{}
%    \end{macrocode}

% Optional override for |\version| flag:
%    \begin{macrocode}
%%\ifchilddoc\else\providecommand{\version}{draft}\fi
%    \end{macrocode}

% Define the default values for the |\version| flag
% (|final| for the main file and |draft| for childs):
%    \begin{macrocode}
\ifchilddoc
\providecommand{\version}{draft}
\else
\providecommand{\version}{final}
\fi
%    \end{macrocode}

% Load the standard document class:
%    \begin{macrocode}
\documentclass[12pt]{article}
%    \end{macrocode}

% Start the document body:
%    \begin{macrocode}
\begin{document}
%    \end{macrocode}

% Declare a title page.
% Print title, part of document being processed and version flag:
%    \begin{macrocode}
\addtocounter{page}{-1}
\begin{center}
{\LARGE\bfseries{}childdoc example\par}
\vspace{1cm}
\ifchilddoc
\ifchilddocmanual part\else chapter\fi:
`\childdocname' of `\childdocjob'\par
\else
main document: `\childdocjob'\par
\fi
version: \version\par
\end{center}
\newpage
%    \end{macrocode}

% Manually include selected file,
% otherwise process as usual:
%    \begin{macrocode}
\ifchilddocmanual
\section*{part `\childdocname'}
\input{\childdocname}
\else
%    \end{macrocode}

% Include the two chapters:
%    \begin{macrocode}
\include{cdocsch1}
\include{cdocsch2}
%    \end{macrocode}

% Include the two parts unless only chapters should be displayed:
%    \begin{macrocode}
\ifchilddoc\else
\section{part three}
\input{cdocspt3}
\section{part four}
\input{cdocspt4}
\fi
%    \end{macrocode}

% Process as usual until here:
%    \begin{macrocode}
\fi
%    \end{macrocode}

% End of document body:
%    \begin{macrocode}
\end{document}
%    \end{macrocode}
%\iffalse
%</samplemain>
%\fi
%
% %%%%%%%%%%%%%%%%%%%%%%%%%%%%%%%%%%%%%%
% \paragraph{Chapter Include Files.}
%
% The include files are called |cdocsch1.tex| and |cdocsch2.tex|.
%
%\iffalse
%<*samplechap1|samplechap2>
%\fi

% Optional override for |\version| flag:
%    \begin{macrocode}
%%\providecommand{\version}{final}
%    \end{macrocode}

% Include the main document:
%    \begin{macrocode}
\input{childdoc.def}
\childdocof{cdocsamp}
%    \end{macrocode}

%\iffalse
%</samplechap1|samplechap2>
%\fi
%
%\iffalse
%<*samplechap1>
%\fi
% Some text for chapter 1:
%    \begin{macrocode}
\section{one}
some text in chapter one
%    \end{macrocode}

%\iffalse
%</samplechap1>
%\fi
% Some text for chapter 2:
%\iffalse
%<*samplechap2>
%\fi
%    \begin{macrocode}
\section{two}
more text in chapter two
%    \end{macrocode}

%\iffalse
%</samplechap2>
%\fi
%
% %%%%%%%%%%%%%%%%%%%%%%%%%%%%%%%%%%%%%%
% \paragraph{Part Include Files.}
%
% The include files are called |cdocspt3.tex| and |cdocspt4.tex|.
%
%\iffalse
%<*samplepart3|samplepart4>
%\fi

% Optional override for |\version| flag:
%    \begin{macrocode}
%%\providecommand{\version}{final}
%    \end{macrocode}

% Include the main document:
%    \begin{macrocode}
\input{childdoc.def}
\childdocby{cdocsamp}
%    \end{macrocode}

%\iffalse
%</samplepart3|samplepart4>
%\fi
%
%\iffalse
%<*samplepart3>
%\fi
% Some text for part 3:
%    \begin{macrocode}
some text in part three
%    \end{macrocode}

%\iffalse
%</samplepart3>
%\fi
% Some text for part 4:
%\iffalse
%<*samplepart4>
%\fi
%    \begin{macrocode}
more text in part four
%    \end{macrocode}

%\iffalse
%</samplepart4>
%\fi
%
% %%%%%%%%%%%%%%%%%%%%%%%%%%%%%%%%%%%%%%
% \paragraph{Forwarding for a Complete Draft.}
%
% The following forwarding file |cdocsdrf.tex|
% compiles the main document in draft mode:
%\iffalse
%<*sampledraft>
%\fi
%    \begin{macrocode}
\def\version{draft}
\input{childdoc.def}
\childdocforward{cdocsamp}
%    \end{macrocode}

%\iffalse
%</sampledraft>
%\fi
%
% %%%%%%%%%%%%%%%%%%%%%%%%%%%%%%%%%%%%%%
% \paragraph{Forwarding for Final Version of the Chapters.}
%
% The following forwarding files |cdocsfn1.tex| and |cdocsfn2.tex|
% (with identical content)
% compile the final versions of the child documents
% |cdocsch1.tex| and |cdocsch2.tex|, respectively:
%\iffalse
%<*samplefinal>
%\fi
%    \begin{macrocode}
\def\version{final}
\input{childdoc.def}
\childdocforwardprefix[cdocsamp]{cdocsfn}{cdocsch}
%    \end{macrocode}

%\iffalse
%</samplefinal>
%\fi
%
% %%%%%%%%%%%%%%%%%%%%%%%%%%%%%%%%%%%%%%
% \paragraph{Command Line Processing.}
%
% The following three command lines generate the output files
% |cdocscld|, |cdocscl1| and |cdocscl2|
% which should be identical to
% |cdocsdrf|, |cdocsch1| and |cdocsfn2|, respectively:
% \begin{center}
% \begin{tabular}{l}
% |latex -jobname cdocscld \|\\
% |  "\def\version{draft}\input{childdoc.def}\childdocforward{cdocsamp}"|\\
% |latex -jobname cdocscl1 \|\\
% |  "\input{childdoc.def}\childdocforward[cdocsamp]{cdocsch1}"|\\
% |latex -jobname cdocscl2 \|\\
% |  "\def\version{final}\input{childdoc.def}\childdocforward{cdocsch2}"|
% \end{tabular}
% \end{center}
% Note that the trailing backslash on each first line
% merely continues the input to the second line
% (for convenient cut ant paste).
% Furthermore, the command |latex| can be replaced by any
% of its alternative versions such as |pdflatex|.
%
% %%%%%%%%%%%%%%%%%%%%%%%%%%%%%%%%%%%%%%%%%%%%%%%%%%%%%%%%%%%%%%%%%%%%%%%%%%%%%%
% %%%%%%%%%%%%%%%%%%%%%%%%%%%%%%%%%%%%%%%%%%%%%%%%%%%%%%%%%%%%%%%%%%%%%%%%%%%%%%
% \section{Implementation}
%\iffalse
%<*package>
%\fi
%
% This section describes the definitions file |childdoc.def|.

% The definitions cannot be loaded using |\usepackage| or |\RequirePackage|
% which has a mechanism to prevent loading a style file more than once.
% When loading the definitions by means of |\input|
% multiple instances have to be prevented manually:
%\iffalse
%This code needs to be before the `\ProvidesFile' directive
%which is defined at the beginning of this file.
%Therefore it is also placed there and commented out here.
%</package>
%<*discard>
%\fi
%    \begin{macrocode}
\ifdefined\childdocmain\endinput\fi
%    \end{macrocode}
%\iffalse
%</discard>
%<*package>
%\fi
%
% \macro{\ifchilddoc}
% \macro{\ifchilddocmanual}
% The conditional |\ifchilddoc| tells whether a
% child (true) or main (false) document is being compiled.
% The conditional |\ifchilddocmanual| tells whether
% the |\includeonly| mechanism is used (false) or
% the selection of child files must be performed manually (true).
% The definitions initialise to false:
%    \begin{macrocode}
\newif\ifchilddoc
\newif\ifchilddocmanual
%    \end{macrocode}

% \macro{\childdocname}
% \macro{\childdocjob}
% The macro |\childdocname| stores the name of the main document
% to be compiled. The macro |\childdocjob| stores the name of
% the document on which the \LaTeX{} compiler was originally invoked.
% The content of |\jobname| cannot be compared
% to filenames specified in the source due to different catcodes.
% The following code rescans |\jobname|, stores the result
% in |\childdocname| and saves a copy in |\childdocjob|:
%    \begin{macrocode}
\edef\childdocname{\scantokens\expandafter{\jobname\noexpand}}
\let\childdocjob\childdocname
%    \end{macrocode}

% \macro{\childdocdisable}
% The macro |\childdocdisable| prevents the main file
% from being processed more than once.
% At this stage, the main document command |\childdocmain|
% is assumed to be called once again where it should do nothing.
% Any subsequent call to it should prevent
% a secondary processing of the main document
% It overwrites the forwarding commands
% |\childdocof| and |\childdocforward|
% with empty macros to prevent further inclusions of the main document:
%    \begin{macrocode}
\newcommand{\childdocdisable}
{
  \renewcommand{\childdocmain}[1]{\renewcommand{\childdocmain}[1]{\endinput}}
  \renewcommand{\childdocof}[1]{}
  \renewcommand{\childdocby}[2][]{}
  \renewcommand{\childdocforward}[2][]{}
  \renewcommand{\childdocdisable}{}
}
%    \end{macrocode}

% \macro{\childdocmain}
% The macro |\childdocmain| is to be called at the top of the main file
% with nothing or the main filename (without extension) as argument.
% First, it breaks loops.
% If the argument is not empty and does not match |\childdocname|
% (which is set by the first inclusion of |childdoc.def|),
% |\ifchilddoc| is set to true, |\includeonly| is applied to the child file
% and |\jobname| is set to the main file
% (for proper handling of |.aux| files):
%    \begin{macrocode}
\newcommand{\childdocmain}[1]
{
  \childdocdisable\childdocmain{}
  \if?#1?\else
    \begingroup
      \def\childdoctmp{#1}
      \ifx\childdoctmp\childdocname
        \def\childdoctmp{}
      \else
        \def\childdoctmp
        {
          \childdoctrue
          \includeonly{\childdocname}
          \def\childdocjob{#1}
          \def\jobname{#1}
        }
      \fi
      \expandafter
    \endgroup
    \childdoctmp
  \fi
}
%    \end{macrocode}

% \macro{\childdocof}
% The command |\childdocof| redirects
% compilation to the main file |#1|.
%    \begin{macrocode}
\newcommand{\childdocof}[1]
{
  \childdocdisable
  \childdoctrue
  \includeonly{\childdocname}
  \def\jobname{#1}
  \def\childdocjob{#1}
  \input{#1}
}
%    \end{macrocode}

% \macro{\childdocby}
% The command |\childdocby| ....
%    \begin{macrocode}
\newcommand{\childdocby}[2][]
{
  \childdocdisable
  \childdoctrue
  \childdocmanualtrue
  \if?#1?\else
    \def\jobname{#2}
  \fi
  \def\childdocjob{#2}
  \input{#2}
  \endinput
}
%    \end{macrocode}

% \macro{\childdocforward}
% The command |\childdocforward| redirects
% compilation to the main file or
% (if the optional argument is given) a child file.
% Parameters are set as if the main file
% or a child file starting with |\childdocof| was compiled.
% Then compilation is handed over to the main file:
%    \begin{macrocode}
\newcommand{\childdocforward}[2][]
{
  \begingroup
    \if?#1?
      \def\childdoctmp
      {
        \def\childdocname{#2}
        \def\childdocjob{#2}
        \def\jobname{#2}
        \input{#2}
        \endinput
      }
    \else
      \def\childdoctmp
      {
        \childdocdisable
        \def\childdocname{#2}
        \childdoctrue
        \includeonly{#2}
        \def\childdocjob{#1}
        \def\jobname{#1}
        \input{#1}
        \endinput
      }
    \fi
    \expandafter
  \endgroup
  \childdoctmp
}
%    \end{macrocode}

% \macro{\childdocforwardprefix}
% The command |\childdocforwardprefix| redirects
% compilation to the main or a child file by means of a pattern.
% The prefix |#1| in the current filename is replaced by |#2|
% and the suffix of the current filename is kept
% (it is assumed that the filename does not contain the substring `|~~~|'
% which is used as a delimiter).
% Compilation is handed over to the new file by |\childdocforward|:
%    \begin{macrocode}
\newcommand{\childdocforwardprefix}[3][]
{
  \begingroup
    \def\childdocextract #2##1~~~{\def\childdoctmp{\childdocforward[#1]{#3##1}}}
    \expandafter\childdocextract\childdocname~~~
    \expandafter
  \endgroup
  \childdoctmp
}
%    \end{macrocode}

% \macro{\childdoc}
% The deprecated macro |\childdoc| is a legacy version of |\childdocmain|:
%    \begin{macrocode}
\newcommand{\childdoc}{\childdocmain}
%    \end{macrocode}

% \macro{\childdocredirect}
% The deprecated macro |\childdocredirect| is a legacy version
% of |\childdocforward| and |\childdocforwardprefix|:
%    \begin{macrocode}
\newcommand{\childdocredirect}[2][]
{
  \begingroup
    \if?#1?
      \def\childdoctmp{\childdocforward{#2}}
    \else
      \def\childdoctmp{\childdocforwardprefix{#1}{#2}}
    \fi
    \expandafter
  \endgroup
  \childdoctmp
}
%    \end{macrocode}

%\iffalse
%</package>
%\fi
%
\endinput

\childdocmain{}
%    \end{macrocode}

% Optional override for |\version| flag:
%    \begin{macrocode}
%%\ifchilddoc\else\providecommand{\version}{draft}\fi
%    \end{macrocode}

% Define the default values for the |\version| flag
% (|final| for the main file and |draft| for childs):
%    \begin{macrocode}
\ifchilddoc
\providecommand{\version}{draft}
\else
\providecommand{\version}{final}
\fi
%    \end{macrocode}

% Load the standard document class:
%    \begin{macrocode}
\documentclass[12pt]{article}
%    \end{macrocode}

% Start the document body:
%    \begin{macrocode}
\begin{document}
%    \end{macrocode}

% Declare a title page.
% Print title, part of document being processed and version flag:
%    \begin{macrocode}
\addtocounter{page}{-1}
\begin{center}
{\LARGE\bfseries{}childdoc example\par}
\vspace{1cm}
\ifchilddoc
\ifchilddocmanual part\else chapter\fi:
`\childdocname' of `\childdocjob'\par
\else
main document: `\childdocjob'\par
\fi
version: \version\par
\end{center}
\newpage
%    \end{macrocode}

% Manually include selected file,
% otherwise process as usual:
%    \begin{macrocode}
\ifchilddocmanual
\section*{part `\childdocname'}
\input{\childdocname}
\else
%    \end{macrocode}

% Include the two chapters:
%    \begin{macrocode}
\include{cdocsch1}
\include{cdocsch2}
%    \end{macrocode}

% Include the two parts unless only chapters should be displayed:
%    \begin{macrocode}
\ifchilddoc\else
\section{part three}
\input{cdocspt3}
\section{part four}
\input{cdocspt4}
\fi
%    \end{macrocode}

% Process as usual until here:
%    \begin{macrocode}
\fi
%    \end{macrocode}

% End of document body:
%    \begin{macrocode}
\end{document}
%    \end{macrocode}
%\iffalse
%</samplemain>
%\fi
%
% %%%%%%%%%%%%%%%%%%%%%%%%%%%%%%%%%%%%%%
% \paragraph{Chapter Include Files.}
%
% The include files are called |cdocsch1.tex| and |cdocsch2.tex|.
%
%\iffalse
%<*samplechap1|samplechap2>
%\fi

% Optional override for |\version| flag:
%    \begin{macrocode}
%%\providecommand{\version}{final}
%    \end{macrocode}

% Include the main document:
%    \begin{macrocode}
% \iffalse
%
% childdoc.dtx Copyright (C) 2017-2018 Niklas Beisert
%
% This work may be distributed and/or modified under the
% conditions of the LaTeX Project Public License, either version 1.3
% of this license or (at your option) any later version.
% The latest version of this license is in
%   http://www.latex-project.org/lppl.txt
% and version 1.3 or later is part of all distributions of LaTeX
% version 2005/12/01 or later.
%
% This work has the LPPL maintenance status `maintained'.
%
% The Current Maintainer of this work is Niklas Beisert.
%
% This work consists of the files childdoc.dtx and childdoc.ins
% and the derived files childdoc.def and cdocsamp.tex with
% cdocsch1.tex, cdocsch2.tex, cdocsdrf.tex, cdocsfn1.tex, cdocsfn2.tex.
%
%<package>\ifdefined\childdocmain\endinput\fi
%<package>\ProvidesFile{childdoc.def}[2018/12/30 v2.0 child document driver]
%<samplemain>\ProvidesFile{cdocsamp.tex}[2018/12/30 v2.0 sample for childdoc]
%<*driver>
%\ProvidesFile{childdoc.drv}[2018/12/30 v2.0 childdoc reference manual file]
\PassOptionsToClass{10pt,a4paper}{article}
\documentclass{ltxdoc}

\usepackage[margin=35mm]{geometry}
\usepackage{hyperref}
\usepackage{hyperxmp}
\usepackage[usenames]{color}

\hypersetup{colorlinks=true}
\hypersetup{pdfstartview=FitH}
\hypersetup{pdfpagemode=UseNone}
\hypersetup{pdfsource={}}
\hypersetup{pdflang={en-UK}}
\hypersetup{pdfcopyright={Copyright 2017-2018 Niklas Beisert.
  This work may be distributed and/or modified under the
  conditions of the LaTeX Project Public License, either version 1.3
  of this license or (at your option) any later version.}}
\hypersetup{pdflicenseurl={http://www.latex-project.org/lppl.txt}}
\hypersetup{pdfcontactaddress={ETH Zurich, ITP, HIT K,
  Wolfgang-Pauli-Strasse 27}}
\hypersetup{pdfcontactpostcode={8093}}
\hypersetup{pdfcontactcity={Zurich}}
\hypersetup{pdfcontactcountry={Switzerland}}
\hypersetup{pdfcontactemail={nbeisert@itp.phys.ethz.ch}}
\hypersetup{pdfcontacturl={http://people.phys.ethz.ch/\xmptilde nbeisert/}}

\newcommand{\secref}[1]{\hyperref[#1]{section \ref*{#1}}}

\parskip1ex
\parindent0pt
\let\olditemize\itemize
\def\itemize{\olditemize\parskip0pt}

\begin{document}

\title{The \textsf{childdoc} Package}
\hypersetup{pdftitle={The childdoc Package}}
\author{Niklas Beisert\\[2ex]
  Institut f\"ur Theoretische Physik\\
  Eidgen\"ossische Technische Hochschule Z\"urich\\
  Wolfgang-Pauli-Strasse 27, 8093 Z\"urich, Switzerland\\[1ex]
  \href{mailto:nbeisert@itp.phys.ethz.ch}
  {\texttt{nbeisert@itp.phys.ethz.ch}}}
\hypersetup{pdfauthor={Niklas Beisert}}
\hypersetup{pdfsubject={Manual for the LaTeX2e Package childdoc}}
\date{30 December 2018, \textsf{v2.0}}
\maketitle

\begin{abstract}\noindent
\textsf{childdoc} is a \LaTeXe{} package
that enables the direct compilation
of document sections included by |\include|
to individual files.
\end{abstract}

\begingroup
\parskip0ex
\tableofcontents
\endgroup

%%%%%%%%%%%%%%%%%%%%%%%%%%%%%%%%%%%%%%%%%%%%%%%%%%%%%%%%%%%%%%%%%%%%%%%%%%%%%%%%
%%%%%%%%%%%%%%%%%%%%%%%%%%%%%%%%%%%%%%%%%%%%%%%%%%%%%%%%%%%%%%%%%%%%%%%%%%%%%%%%
\section{Introduction}

\LaTeX{} provides a mechanism to structure a large document (such as a book)
into a main file and several child files (containing the chapters)
using the |\include| command.
This mechanism is beneficial for documents
which span hundreds of pages in order to
make the source file(s) more manageable.
Moreover, compilation can be restricted to
selected child files by means of the |\includeonly| command.
The latter feature can be used to reduce the compilation time while editing
(this was significantly more useful in the earlier days of \LaTeX{})
or to generate a smaller document which is easier to navigate.
Another application of |\includeonly| is to generate
documents consisting of selected parts of the complete document.

However, there are a few drawbacks of the plain |\include| mechanism:
\begin{itemize}
\item
The child files cannot be compiled on their own,
they can only be compiled via the main file.
A naive editing environment
(such as a text editor with an option
to have the current file processed by \LaTeX)
may require one to switch to the main file before compiling;
attempting to compile the child file produces errors.
\item
The main file must be modified (each time)
to adjust the |\includeonly| command
to the present needs. This easily leaves the main file in a messy state.
\item
The generated document will always carry the filename
of the main document. This is inconvenient if
several child files are to be compiled and
to be kept for distribution.
\end{itemize}

The present package provides a simple interface
to make child files individually compilable by \LaTeX{}.
Compiling a child file then has the same effect as compiling
the main file with an |\includeonly| command
to select the appropriate child.
Moreover the generated document will carry the name of the child
rather than the main file.
This resolves all three above issues.

This feature is meant to make the editing of books,
thesis documents and lecture notes somewhat more convenient.
However, the package can also be used efficiently for
composing a series of documents (such as exercise sheets)
which are typically distributed individually.
It then assists the author in generating the individual documents
(potentially in different versions)
as well as a document containing the collected series.
Another application is in developing style files
or other kinds of included material
where compilation of the style file could redirect
to a sample or test file.

%%%%%%%%%%%%%%%%%%%%%%%%%%%%%%%%%%%%%%%%%%%%%%%%%%%%%%%%%%%%%%%%%%%%%%%%%%%%%%%%
%%%%%%%%%%%%%%%%%%%%%%%%%%%%%%%%%%%%%%%%%%%%%%%%%%%%%%%%%%%%%%%%%%%%%%%%%%%%%%%%
\section{Usage}

First of all, the package \textsf{childdoc} is \emph{not} a standard
\LaTeXe{} |.sty| style file! Therefore it needs to be invoked in
a non-standard way.

%%%%%%%%%%%%%%%%%%%%%%%%%%%%%%%%%%%%%%%%%%%%%%%%%%%%%%%%%%%%%%%%%%%%%%%%%%%%%%%%
\subsection{Included Files}
\label{sec:include}

%%%%%%%%%%%%%%%%%%%%%%%%%%%%%%%%%%%%%%%%
\DescribeMacro{\childdocmain}
To use the package, add the commands
\begin{center}
\begin{tabular}{l}
|\input{childdoc.def}|\\
|\childdocmain{}|\\
\end{tabular}
\end{center}
at the very top of the main \LaTeX{} file,
in particular \emph{before} the |\documentclass| statement!
The argument of |\childdocmain| should be left empty
(but it must be present).

%%%%%%%%%%%%%%%%%%%%%%%%%%%%%%%%%%%%%%%%
\DescribeMacro{\childdocof}
Furthermore, add the commands
\begin{center}
\begin{tabular}{l}
|\input{childdoc.def}|\\
|\childdocof{|\textit{main}|}|\\
\end{tabular}
\end{center}
at the top of every child file \textit{child}
which is included by |\include{|\textit{child}|}|
from within the main file
(or at least for those files to be compiled individually).
The argument \textit{main} must be the filename of the main file.

There are a couple of
considerations in setting up the main and child documents:

%%%%%%%%%%%%%%%%%%%%%%%%%%%%%%%%%%%%%%%%
\paragraph{Restrictions.}

Please note the following restrictions:
\begin{itemize}
\item
|\childdocmain| must be called with one argument \textit{main}
to ensure compatibility with earlier version of the package.
It must either be empty (|\childdocmain{}|)
or precisely match the filename of the main file in which it is specified.
See \secref{sec:detection} for further information.
\item
The filename \textit{main} must be specified without the |.tex| extension.
\item
The filename \textit{main} is case sensitive
(even in case-insensitive file systems)
due to internal string comparison.
\item
The argument \textit{main} should be fully expanded, it cannot be a macro.
\item
Subdirectories and special characters should be avoided in filenames.
\item
The command |\childdocmain{|\textit{main}|}| must be followed by a whitespace.
It should not be followed immediately by another command
or by a comment mark `|%|'.
This is because the \TeX{} parser reads the token immediately following
the argument of |\childdocmain| and puts it
at the beginning of every child section;
however, a white\-space is ignored.
\end{itemize}

%%%%%%%%%%%%%%%%%%%%%%%%%%%%%%%%%%%%%%%%
\paragraph{Content of Main File.}

It is advisable to place all content in the child files included by |\include|.
Any output contained in the main file will appear in all child documents
unless suppressed manually;
it cannot be suppressed automatically by the |\includeonly| directive
and thus should normally be avoided.
A method to include some content in the main file
by means of conditional processing is described in \secref{sec:conditional}.

%%%%%%%%%%%%%%%%%%%%%%%%%%%%%%%%%%%%%%%%
\paragraph{Page Numbering.}

When only a part of the document is compiled,
the appropriate numbering of pages
(as well as other status parameters)
is determined from the |.aux| files.
The latter contain information from previous passes.
However this information needs to propagate through
all intermediate child documents.
Therefore the page numbering in child documents may well
be inconsistent until the complete document is compiled at least once.

A useful (if unconventional) way to always ensure a consistent
page numbering is to restart the numbering in each child document
and denote the pages by `\textit{child}|.|\textit{page}'
where \textit{child} represents the chapter/section number of the child file.
This can be achieved by the command
|\numberwithin{page}{|\textit{child}|}|
of the \textsf{amsmath} package
where \textit{child} can be |chapter| or |section|
depending on the chosen structuring.
Alternatively, one can modify the macro |\thepage| appropriately
and reset the counter |page| at the start of each child file.

%%%%%%%%%%%%%%%%%%%%%%%%%%%%%%%%%%%%%%%%%%%%%%%%%%%%%%%%%%%%%%%%%%%%%%%%%%%%%%%%
\subsection{Conditional Processing}
\label{sec:conditional}

The package provides a mechanism to compile different versions
of a document. To customise the versions further some conditional processing
can come in handy to distinguish which version is being compiled.
The package provides two macros to describe the compilation context:

%%%%%%%%%%%%%%%%%%%%%%%%%%%%%%%%%%%%%%%%
\DescribeMacro{\ifchilddoc}
The conditional |\ifchilddoc| distinguishes between the compilation of
child documents and the main document:
%
\begin{center}
|\ifchilddoc |\textit{child-code}| |[|\||else |\textit{main-code}]| \||fi|
\end{center}

%%%%%%%%%%%%%%%%%%%%%%%%%%%%%%%%%%%%%%%%
\DescribeMacro{\childdocname}
\DescribeMacro{\childdocjob}
The macro |\childdocname| contains the filename (without extension)
of the main or child file being processed.
Note that |\childdocjob| will always contain the name of the main file.

%%%%%%%%%%%%%%%%%%%%%%%%%%%%%%%%%%%%%%%%
\paragraph{Title Page.}

Conditional processing can be used to include a title or banner page
in the main document when proper precautions are taken.
Importantly, the code in the main file should ensure that the page counter
(as well as other status parameters which are stored in the |.aux| files)
takes the same value after the conditional processing.
Otherwise the page numbers may take divergent values
depending on which part is compiled.

For example, a title page could be declared by:
%
\begin{center}
\begin{tabular}{l}
|\ifchilddoc\||else|\\
|\addtocounter{page}{-1}|\\
\textit{code for title page}\\
|\newpage|\\
|\||fi|
\end{tabular}
\end{center}
%
A banner page for the child documents can be generated by:
%
\begin{center}
\begin{tabular}{l}
|\ifchilddoc|\\
|\addtocounter{page}{-1}|\\
\textit{code for banner page}\\
|\newpage|\\
|\||fi|
\end{tabular}
\end{center}
%
Here one could write a message such as:
\begin{center}
|This is the part \childdocname{} of \childdocjob{}.|
\end{center}

%%%%%%%%%%%%%%%%%%%%%%%%%%%%%%%%%%%%%%%%%%%%%%%%%%%%%%%%%%%%%%%%%%%%%%%%%%%%%%%%
\subsection{Flags}
\label{sec:flags}

The package makes it easy to generate different versions
of the main or child documents.
To this end compilation flags can be defined
and assigned different default values.
They will be particularly useful in conjunction
with the forwarding mechanism described in \secref{sec:forward}.

For example, it may be useful to have a flag |\version|
which can be set to |draft| or |final|.
The document source will contain some conditional code
depending on the value of |\version|.
Suppose further, the flag should default to |final| for the main file
and to |draft| for child files
which is a natural assignment for editing the document.
This is achieved by placing the following code
in the preamble of the main document
(below the |\childdocmain| directive):
%
\begin{center}
\begin{tabular}{l}
|\ifchilddoc|\\
|\providecommand{\version}{draft}|\\
|\||else|\\
|\providecommand{\version}{final}|\\
|\||fi|
\end{tabular}
\end{center}
%
The definition by |\providecommand| makes sure
that previous definitions are not overwritten.
Further statements |\providecommand{\version}{...}|
can thus be added before the above code to override it.

For the main file, one might add a line
(between |\childdocmain| and the above block)
%
\begin{center}
|%\ifchilddoc\||else\providecommand{\version}{draft}\||fi|
\end{center}
%
which can be uncommented to produce a draft version.
Likewise one can add a line to the very top of a child file
(above the |\childdocof{|\textit{main}|}| directive)
%
\begin{center}
|%\providecommand{\version}{final}|
\end{center}
%
which can be uncommented to produce the final version of this child document.

%%%%%%%%%%%%%%%%%%%%%%%%%%%%%%%%%%%%%%%%%%%%%%%%%%%%%%%%%%%%%%%%%%%%%%%%%%%%%%%%
\subsection{Forwarding}
\label{sec:forward}

Different versions of the main or child documents
using compilation flags as described in \secref{sec:flags}
can be (permanently) stored in different files
for convenient compilation, viewing and distribution.
To this end, the package defines a command
to pass on compilation to a different file:

%%%%%%%%%%%%%%%%%%%%%%%%%%%%%%%%%%%%%%%%
\DescribeMacro{\childdocforward}
The command |\childdocforward| redirects processing to
another source file:
%
\begin{center}
\begin{tabular}{l}
|\input{childdoc.def}|\\
|\childdocforward[|\textit{main}|]{|\textit{dest}|}|\\
\end{tabular}
\end{center}
%
The argument \textit{dest} is the destination file
(without extension).
It should be the main file or one of the child files.
Note that further \textsf{childdoc} directives
such as |\childdocof| and |\childdocforward|
in the indicated file will be processed in this form.
The optional argument \textit{main}
passes on directly to the main file \textit{main}
while pretending to compile the child \textit{dest}.
This form behaves as if \textit{dest}
issues |\childdocof{|\textit{main}|}| right away,
and no further \textsf{childdoc} directives will be processed.

%%%%%%%%%%%%%%%%%%%%%%%%%%%%%%%%%%%%%%%%
\DescribeMacro{\...prefix}
In the alternative form |\childdocforwardprefix|,
%
\begin{center}
\begin{tabular}{l}
|\input{childdoc.def}|\\
|\childdocforwardprefix[|\textit{main}|]{|\textit{prefix}|}{|\textit{dest}|}|
\end{tabular}
\end{center}
%
the destination file is determined by a pattern
depending on the current file:
To make this work, the current file must be called
`{\textit{prefix}\hspace{0.2em}\textit{suffix}}'
with \textit{prefix} matching precisely the argument.
Processing is then passed on to the file
`{\textit{dest}\hspace{0.2em}\textit{suffix}}'.
Surely, the same effect is achieved by
directly specifying the
argument `{\textit{dest}\hspace{0.2em}\textit{suffix}}'
in the first form.
However, that requires to set up a different file
for each child. With the alternative form of the command
all these files can have exactly the same content
which simplifies setting them up and maintaining them.

For example, the following file |draft.tex|
with a compilation flag |\version| as described in \secref{sec:flags}
compiles the main document as a draft:
%
\begin{center}
\begin{tabular}{l}
|\def\version{draft}|\\
|\input{childdoc.def}|\\
|\childdocforward{|\textit{main}|}|
\end{tabular}
\end{center}
%
Likewise, the following files |final|\textit{nn}|.tex|
compile the final version of the child document
|child|\textit{nn}|.tex|:
%
\begin{center}
\begin{tabular}{l}
|\def\version{final}|\\
|\input{childdoc.def}|\\
|\childdocforwardprefix{final}{child}|
\end{tabular}
\end{center}
%

Note that when several versions of a main file and/or of each child file
are to be generated, it may be convenient to set up a |Makefile| or
shell script to automatise the process.

%%%%%%%%%%%%%%%%%%%%%%%%%%%%%%%%%%%%%%%%%%%%%%%%%%%%%%%%%%%%%%%%%%%%%%%%%%%%%%%%
\subsection{Command Line Processing}
\label{sec:commandline}

The effect of redirection files can also be achieved by invoking
the \LaTeX{} compiler with a more elaborate command line.
Most conveniently this should be done as part
of a shell script or a |Makefile|.

When using \textsf{childdoc} in the main file, the following
command lines effectively perform a redirection
(note that depending on the shell being used,
backslashes may have to be doubled: `|\|' $\to$ `|\\|'):
%
\begin{center}
|... -jobname "|\textit{target}|" |\\|"|[\textit{flags}]%
|\input{childdoc.def}\childdocforward[|\textit{main}|]{|\textit{dest}|}"|
\end{center}
%
Here \textit{target} is the name of the output file,
\textit{main} is the name of the main file
and \textit{dest} is the name of the main or child file to be processed
(all filenames without extensions).
The optional argument \textit{main} can be omitted
if \textit{main} matches \textit{dest}.
Optionally, compilation \textit{flags} can be defined via |\def| commands.
This command line makes the \TeX{} engine believe
it is compiling the file \textit{target}
whose content is specified as the latter parameter.
The provided code then forwards the processing to
\textit{main} or \textit{dest} as described in \secref{sec:forward}.

%%%%%%%%%%%%%%%%%%%%%%%%%%%%%%%%%%%%%%%%%%%%%%%%%%%%%%%%%%%%%%%%%%%%%%%%%%%%%%%%
\subsection{Include by Input}
\label{sec:input}

Including child documents by |\include| has some restrictions by design.
Most notably, the content of a child document always occupies
its own set of pages; pages cannot be shared between child documents.
Usually, this behaviour makes perfect sense
because each child document contain an essential part of the document.
However, in some situations it may be desirable to compose
a document from a collection of parts
without having mandatory page breaks between then.
For this case, the package
provides a mechanism to include parts
by |\input| which can also be processed individually.
However, by construction this mechanism
requires manual handling of the content to be output.

%%%%%%%%%%%%%%%%%%%%%%%%%%%%%%%%%%%%%%%%
\DescribeMacro{\ifchilddocmanual}
The main file should be prepared as usual, see \secref{sec:include}.
However, the document body must make a distinction
between processing of an individual part and of the main document, e.g.:
%
\begin{center}
\begin{tabular}{l}
|\ifchilddocmanual|\\
|\input{\childdocname}|\\
|\||else|\\
\textit{document body with }|\input{|\textit{part}|}|\\
|\||fi|
\end{tabular}
\end{center}
%
The conditional |\ifchilddocmanual| is true whenever
a part to be included by |\input| is being compiled,
and the name of the part is stored in |\childdocname|.

%%%%%%%%%%%%%%%%%%%%%%%%%%%%%%%%%%%%%%%%
\DescribeMacro{\childdocby}
Each part to be included by |\input| should start with:
%
\begin{center}
\begin{tabular}{l}
|\input{childdoc.def}|\\
|\childdocby{|\textit{main}|}|\\
\end{tabular}
\end{center}
%
The directive |\childdocby| is similar to |\childdocof|
described in \secref{sec:include},
but the subsequent selection of content must be done manually.
To that end, both |\ifchilddoc| and |\ifchilddocmanual|
will be true upon processing of a part,
and the name of the part is stored in |\childdocname|.
Note that |\jobname| will be set to the filename of the current part
so that each part receives an individual |.aux| file
that does not interfere with the |.aux| file(s) of the main document.
This behaviour can be altered by the alternative form
|\childdocby[*]{|\textit{main}|}| (with a non-empty optional argument)
which uses the |.aux| file of the main document
by setting |\jobname| to \textit{main}.

%%%%%%%%%%%%%%%%%%%%%%%%%%%%%%%%%%%%%%%%%%%%%%%%%%%%%%%%%%%%%%%%%%%%%%%%%%%%%%%%
\subsection{Driver Development}
\label{sec:driver}

The \textsf{childdoc} mechanism can also be use for the development
of definition files such as \LaTeX{} styles or classes.
This case differs from the above setup with multiple parts
included by |\include| in that no |\includeonly| should be invoked.
This can be achieved by starting the include file
(before |\ProvidesPackage|) with:
%
\begin{center}
\begin{tabular}{l}
|\input{childdoc.def}|\\
|\childdocforward{|\textit{main}|}|\\
\end{tabular}
\end{center}
%
or alternatively with:
%
\begin{center}
\begin{tabular}{l}
|\input{childdoc.def}|\\
|\childdocby{|\textit{main}|}|\\
\end{tabular}
\end{center}
%
Both forms have slightly different effects as described above.
The main file is prepared as usual, see \secref{sec:include}.

%%%%%%%%%%%%%%%%%%%%%%%%%%%%%%%%%%%%%%%%%%%%%%%%%%%%%%%%%%%%%%%%%%%%%%%%%%%%%%%%
\subsection{Legacy Detection}
\label{sec:detection}

The directive |\childdocmain| in the main file can detect
whether the complete document or merely a child is to be compiled
even without using the directive |\childdocof|.
This method is deprecated because it is less robust
and there is no compelling reason to use it;
it is merely provided for backward compatibility
and it may be removed in future versions.

If the detection mechanism is to be used,
it is mandatory to correctly specify
the filename of the main file as the argument of |\childdocmain|:
%
\begin{center}
\begin{tabular}{l}
|\input{childdoc.def}|\\
|\childdocmain{|\textit{main}|}|\\
\end{tabular}
\end{center}
%
If |\jobname| does not match the argument \textit{main} of |\childdocmain|,
it is assumed that |\jobname| points to the child file to be compiled.
When using |\childdocmain| with the main file specified as argument,
it suffices to start a child file
with just |\input{|\textit{main}|}|
without loading of the package and using |\childdocof|.
If instead all processing is done
with the appropriate \textsf{childdoc} directives,
the argument of \textit{main} of |\childdocmain| can be empty.

An alternative version of the command line processing described
in \secref{sec:commandline} using the detection mechanism reads:
%
\begin{center}
|... -jobname "|\textit{target}|" "|[\textit{flags}]%
[|\def\jobname{|\textit{dest}|}|]|\input{|\textit{main}|}"|
\end{center}

%%%%%%%%%%%%%%%%%%%%%%%%%%%%%%%%%%%%%%%%%%%%%%%%%%%%%%%%%%%%%%%%%%%%%%%%%%%%%%%%
\subsection{Manual Code}
\label{sec:manual}

In case one cannot be certain whether the definitions file |childdoc.def|
is installed on the target \TeX{} distribution
and one prefers not to ship it,
it is conceivable to paste a few relevant commands into the sources.

To that end, drop all statements |\input{childdoc.def}|
and perform the replacements as outlined below.
Instead of |\childdocmain{|\textit{main}|}| add the following code
to the top of the main file:
%
\begin{center}
\begin{tabular}{l}
|\||ifdefined\childdocname\endinput\||fi\newif\ifchilddoc|\\
|\edef\childdocname{\scantokens\expandafter{\jobname\noexpand}}|\\
|\def\childdocmain{|\textit{main}|}\||ifx\childdocmain\childdocname\||else|\\
|\childdoctrue\includeonly{\childdocname}\let\jobname\childdocmain\||fi|\\
\end{tabular}
\end{center}
%
Instead of |\childdocof{|\textit{main}|}| just include the main file
at the top of each child file:
%
\begin{center}
|\input{|\textit{main}|}|
\end{center}
%
A simple redirection |\childdocforward{|\textit{dest}|}| is achieved by:
%
\begin{center}
|\def\jobname{|\textit{dest}|}\input{\jobname}|
\end{center}
%
The redirection with prefix
|\childdocforwardprefix[|\textit{prefix}|]{|\textit{dest}|}|
is accomplished by:
%
\begin{center}
\begin{tabular}{l}
|{\edef\jobname{\scantokens\expandafter{\jobname\noexpand}}|\\
|\def\redirectjob |\textit{prefix}|#1~~~{\gdef\jobname{|\textit{dest}|#1}}|\\
|\expandafter\redirectjob\jobname~~~}\input{\jobname}|
\end{tabular}
\end{center}

In an alternative approach,
child documents can be compiled by a specific command line
without additional code or specific definitions:
%
\begin{center}
|... -jobname "|\textit{target}|" "|[\textit{flags}]%
|\includeonly{|\textit{dest}|}\input{|\textit{main}|}"|
\end{center}
%

%%%%%%%%%%%%%%%%%%%%%%%%%%%%%%%%%%%%%%%%%%%%%%%%%%%%%%%%%%%%%%%%%%%%%%%%%%%%%%%%
%%%%%%%%%%%%%%%%%%%%%%%%%%%%%%%%%%%%%%%%%%%%%%%%%%%%%%%%%%%%%%%%%%%%%%%%%%%%%%%%
\section{Information}

%%%%%%%%%%%%%%%%%%%%%%%%%%%%%%%%%%%%%%%%%%%%%%%%%%%%%%%%%%%%%%%%%%%%%%%%%%%%%%%%
\subsection{Copyright}

Copyright \copyright{} 2017--2018 Niklas Beisert

This work may be distributed and/or modified under the
conditions of the \LaTeX{} Project Public License, either version 1.3
of this license or (at your option) any later version.
The latest version of this license is in
  \url{http://www.latex-project.org/lppl.txt}
and version 1.3 or later is part of all distributions of \LaTeX{}
version 2005/12/01 or later.

This work has the LPPL maintenance status `maintained'.

The Current Maintainer of this work is Niklas Beisert.

This work consists of the files |README.txt|, |childdoc.ins| and |childdoc.dtx|
as well as the derived files |childdoc.def|, |cdocsamp.tex|
with |cdocsch1.tex|, |cdocsch2.tex|, |cdocspt3.tex|, |cdocspt4.tex|,
|cdocsdrf.tex|, |cdocsfn1.tex|, |cdocsfn2.tex|
as well as |childdoc.pdf|.

%%%%%%%%%%%%%%%%%%%%%%%%%%%%%%%%%%%%%%%%%%%%%%%%%%%%%%%%%%%%%%%%%%%%%%%%%%%%%%%%
\subsection{Files and Installation}

The package consists of the files:
%
\begin{center}
\begin{tabular}{ll}
    |README.txt|   & readme file \\
    |childdoc.ins| & installation file \\
    |childdoc.dtx| & source file \\
    |childdoc.def| & definition file \\
    |cdocsamp.tex| & sample main file \\
    |cdocsch1.tex| & sample include file \\
    |cdocsch2.tex| & sample include file \\
    |cdocspt3.tex| & sample part file \\
    |cdocspt4.tex| & sample part file \\
    |cdocsdrf.tex| & sample redirection file \\
    |cdocsfn1.tex| & sample redirection file \\
    |cdocsfn2.tex| & sample redirection file \\
    |childdoc.pdf| & manual
\end{tabular}
\end{center}
%
The distribution consists of the files
|README.txt|, |childdoc.ins| and |childdoc.dtx|.
%
\begin{itemize}
\item
Run (pdf)\LaTeX{} on |childdoc.dtx|
to compile the manual |childdoc.pdf| (this file).
\item
Run \LaTeX{} on |childdoc.ins| to create the definitions file |childdoc.def|
and the sample |cdocsamp.tex| with include files
|cdocsch1.tex|, |cdocsch2.tex|, |cdocspt3.tex|, |cdocspt4.tex|,
|cdocsdrf.tex|, |cdocsfn1.tex|, |cdocsfn2.tex|.
Then copy the file |childdoc.def| to an appropriate directory of your \LaTeX{}
distribution, e.g.\ \textit{texmf-root}|/tex/latex/childdoc|.
\end{itemize}

%%%%%%%%%%%%%%%%%%%%%%%%%%%%%%%%%%%%%%%%%%%%%%%%%%%%%%%%%%%%%%%%%%%%%%%%%%%%%%%%
\subsection{Related CTAN Packages}

There are several other packages which offer a similar functionality:
%
\begin{itemize}
\item
The packages
\href{http://ctan.org/pkg/docmute}{\textsf{docmute}},
\href{http://ctan.org/pkg/includex}{\textsf{includex}} and
\href{http://ctan.org/pkg/standalone}{\textsf{standalone}}
provide commands to include only the document body of
a child file thus allowing both files to be compiled individually.
\item
The packages \href{http://ctan.org/pkg/subdocs}{\textsf{subdocs}}
and \href{http://ctan.org/pkg/subfiles}{\textsf{subfiles}}
provide structures in which the main and child documents can be
encapsulated and allowing them to be compiled individually.
The inclusion mechanism is different from the conventional |\include|.
\item
The package \href{http://ctan.org/pkg/combine}{\textsf{combine}}
is an elaborate solution to combine several documents into one.
\end{itemize}
%
See also the CTAN topic \href{http://ctan.org/topic/subdocs}{\textsf{subdocs}}
for further related packages.
The present package differs from the above solutions in that
a document structure constructed with the conventional |\include| mechanism
just needs two extra commands at the top of every file
such that all constituent files can be compiled individually.

%%%%%%%%%%%%%%%%%%%%%%%%%%%%%%%%%%%%%%%%%%%%%%%%%%%%%%%%%%%%%%%%%%%%%%%%%%%%%%%%
%\subsection{Feature Suggestions}
%
%The following is a list of features which may be useful for future
%versions of this package:
%%
%\begin{itemize}
%\item
%\ldots
%\end{itemize}

%%%%%%%%%%%%%%%%%%%%%%%%%%%%%%%%%%%%%%%%%%%%%%%%%%%%%%%%%%%%%%%%%%%%%%%%%%%%%%%%
\subsection{Revision History}

%%%%%%%%%%%%%%%%%%%%%%%%%%%%%%%%%%%%%%%%
\paragraph{v2.0:} 2018/12/30

\begin{itemize}
\item
immediate forward processing
\item
added |\childdocby| mechanism
\item
manual restructured
\end{itemize}

%%%%%%%%%%%%%%%%%%%%%%%%%%%%%%%%%%%%%%%%
\paragraph{v1.6:} 2018/01/17

\begin{itemize}
\item
application for development of include files
\item
corrections to manual
\end{itemize}

%%%%%%%%%%%%%%%%%%%%%%%%%%%%%%%%%%%%%%%%
\paragraph{v1.5:} 2017/05/21

\begin{itemize}
\item
more complete structuring introduced
\item
|\childdocof| introduced
\item
|\childdoc| renamed to |\childdocmain|
\item
|\childredirect| renamed to |\childdocforward| and |\childdocforwardprefix|
and functionality expanded
\end{itemize}

%%%%%%%%%%%%%%%%%%%%%%%%%%%%%%%%%%%%%%%%
\paragraph{v1.0:} 2017/04/27

\begin{itemize}
\item
manual and install package
\item
first version published on CTAN
\end{itemize}

%%%%%%%%%%%%%%%%%%%%%%%%%%%%%%%%%%%%%%%%
\paragraph{v0.6:} 2017/04/26

\begin{itemize}
\item
redirection mechanism added
\end{itemize}

%%%%%%%%%%%%%%%%%%%%%%%%%%%%%%%%%%%%%%%%
\paragraph{v0.5:} 2017/04/26

\begin{itemize}
\item
functionality in definition file
\end{itemize}


%%%%%%%%%%%%%%%%%%%%%%%%%%%%%%%%%%%%%%%%%%%%%%%%%%%%%%%%%%%%%%%%%%%%%%%%%%%%%%%%
%%%%%%%%%%%%%%%%%%%%%%%%%%%%%%%%%%%%%%%%%%%%%%%%%%%%%%%%%%%%%%%%%%%%%%%%%%%%%%%%
%%%%%%%%%%%%%%%%%%%%%%%%%%%%%%%%%%%%%%%%%%%%%%%%%%%%%%%%%%%%%%%%%%%%%%%%%%%%%%%%
\appendix

\settowidth\MacroIndent{\rmfamily\scriptsize 000\ }

 \DocInput{childdoc.dtx}

\end{document}
%</driver>
% \fi
%
% %%%%%%%%%%%%%%%%%%%%%%%%%%%%%%%%%%%%%%%%%%%%%%%%%%%%%%%%%%%%%%%%%%%%%%%%%%%%%%
% %%%%%%%%%%%%%%%%%%%%%%%%%%%%%%%%%%%%%%%%%%%%%%%%%%%%%%%%%%%%%%%%%%%%%%%%%%%%%%
% \section{Sample}
%\iffalse
%<*samplemain>
%\fi
%
% The following presents a sample document
% with two chapters, two parts, a title page,
% a compile flag as well as three forwarding files to set the flag.
% It consists of eight |.tex| files:
% \begin{center}
% \begin{tabular}{ll}
% |cdocsamp.tex|&main file\\
% |cdocsch1.tex|&include file for chapter 1\\
% |cdocsch2.tex|&include file for chapter 2\\
% |cdocspt3.tex|&include file for part 3\\
% |cdocspt4.tex|&include file for part 4\\
% |cdocsdrf.tex|&forwarding file for main file in draft mode\\
% |cdocsfi1.tex|&forwarding file for final version of chapter 1\\
% |cdocsfi2.tex|&forwarding file for final version of chapter 2\\
% \end{tabular}
% \end{center}
% Each of the eight files can be compiled directly by the \LaTeX{} compiler.
%
% %%%%%%%%%%%%%%%%%%%%%%%%%%%%%%%%%%%%%%
% \paragraph{Main File.}
%
% The main file is called |cdocsamp.tex|.
%
% Load the \textsf{childdoc} definitions and
% declare the filename for the main document:
%    \begin{macrocode}
\input{childdoc.def}
\childdocmain{}
%    \end{macrocode}

% Optional override for |\version| flag:
%    \begin{macrocode}
%%\ifchilddoc\else\providecommand{\version}{draft}\fi
%    \end{macrocode}

% Define the default values for the |\version| flag
% (|final| for the main file and |draft| for childs):
%    \begin{macrocode}
\ifchilddoc
\providecommand{\version}{draft}
\else
\providecommand{\version}{final}
\fi
%    \end{macrocode}

% Load the standard document class:
%    \begin{macrocode}
\documentclass[12pt]{article}
%    \end{macrocode}

% Start the document body:
%    \begin{macrocode}
\begin{document}
%    \end{macrocode}

% Declare a title page.
% Print title, part of document being processed and version flag:
%    \begin{macrocode}
\addtocounter{page}{-1}
\begin{center}
{\LARGE\bfseries{}childdoc example\par}
\vspace{1cm}
\ifchilddoc
\ifchilddocmanual part\else chapter\fi:
`\childdocname' of `\childdocjob'\par
\else
main document: `\childdocjob'\par
\fi
version: \version\par
\end{center}
\newpage
%    \end{macrocode}

% Manually include selected file,
% otherwise process as usual:
%    \begin{macrocode}
\ifchilddocmanual
\section*{part `\childdocname'}
\input{\childdocname}
\else
%    \end{macrocode}

% Include the two chapters:
%    \begin{macrocode}
\include{cdocsch1}
\include{cdocsch2}
%    \end{macrocode}

% Include the two parts unless only chapters should be displayed:
%    \begin{macrocode}
\ifchilddoc\else
\section{part three}
\input{cdocspt3}
\section{part four}
\input{cdocspt4}
\fi
%    \end{macrocode}

% Process as usual until here:
%    \begin{macrocode}
\fi
%    \end{macrocode}

% End of document body:
%    \begin{macrocode}
\end{document}
%    \end{macrocode}
%\iffalse
%</samplemain>
%\fi
%
% %%%%%%%%%%%%%%%%%%%%%%%%%%%%%%%%%%%%%%
% \paragraph{Chapter Include Files.}
%
% The include files are called |cdocsch1.tex| and |cdocsch2.tex|.
%
%\iffalse
%<*samplechap1|samplechap2>
%\fi

% Optional override for |\version| flag:
%    \begin{macrocode}
%%\providecommand{\version}{final}
%    \end{macrocode}

% Include the main document:
%    \begin{macrocode}
\input{childdoc.def}
\childdocof{cdocsamp}
%    \end{macrocode}

%\iffalse
%</samplechap1|samplechap2>
%\fi
%
%\iffalse
%<*samplechap1>
%\fi
% Some text for chapter 1:
%    \begin{macrocode}
\section{one}
some text in chapter one
%    \end{macrocode}

%\iffalse
%</samplechap1>
%\fi
% Some text for chapter 2:
%\iffalse
%<*samplechap2>
%\fi
%    \begin{macrocode}
\section{two}
more text in chapter two
%    \end{macrocode}

%\iffalse
%</samplechap2>
%\fi
%
% %%%%%%%%%%%%%%%%%%%%%%%%%%%%%%%%%%%%%%
% \paragraph{Part Include Files.}
%
% The include files are called |cdocspt3.tex| and |cdocspt4.tex|.
%
%\iffalse
%<*samplepart3|samplepart4>
%\fi

% Optional override for |\version| flag:
%    \begin{macrocode}
%%\providecommand{\version}{final}
%    \end{macrocode}

% Include the main document:
%    \begin{macrocode}
\input{childdoc.def}
\childdocby{cdocsamp}
%    \end{macrocode}

%\iffalse
%</samplepart3|samplepart4>
%\fi
%
%\iffalse
%<*samplepart3>
%\fi
% Some text for part 3:
%    \begin{macrocode}
some text in part three
%    \end{macrocode}

%\iffalse
%</samplepart3>
%\fi
% Some text for part 4:
%\iffalse
%<*samplepart4>
%\fi
%    \begin{macrocode}
more text in part four
%    \end{macrocode}

%\iffalse
%</samplepart4>
%\fi
%
% %%%%%%%%%%%%%%%%%%%%%%%%%%%%%%%%%%%%%%
% \paragraph{Forwarding for a Complete Draft.}
%
% The following forwarding file |cdocsdrf.tex|
% compiles the main document in draft mode:
%\iffalse
%<*sampledraft>
%\fi
%    \begin{macrocode}
\def\version{draft}
\input{childdoc.def}
\childdocforward{cdocsamp}
%    \end{macrocode}

%\iffalse
%</sampledraft>
%\fi
%
% %%%%%%%%%%%%%%%%%%%%%%%%%%%%%%%%%%%%%%
% \paragraph{Forwarding for Final Version of the Chapters.}
%
% The following forwarding files |cdocsfn1.tex| and |cdocsfn2.tex|
% (with identical content)
% compile the final versions of the child documents
% |cdocsch1.tex| and |cdocsch2.tex|, respectively:
%\iffalse
%<*samplefinal>
%\fi
%    \begin{macrocode}
\def\version{final}
\input{childdoc.def}
\childdocforwardprefix[cdocsamp]{cdocsfn}{cdocsch}
%    \end{macrocode}

%\iffalse
%</samplefinal>
%\fi
%
% %%%%%%%%%%%%%%%%%%%%%%%%%%%%%%%%%%%%%%
% \paragraph{Command Line Processing.}
%
% The following three command lines generate the output files
% |cdocscld|, |cdocscl1| and |cdocscl2|
% which should be identical to
% |cdocsdrf|, |cdocsch1| and |cdocsfn2|, respectively:
% \begin{center}
% \begin{tabular}{l}
% |latex -jobname cdocscld \|\\
% |  "\def\version{draft}\input{childdoc.def}\childdocforward{cdocsamp}"|\\
% |latex -jobname cdocscl1 \|\\
% |  "\input{childdoc.def}\childdocforward[cdocsamp]{cdocsch1}"|\\
% |latex -jobname cdocscl2 \|\\
% |  "\def\version{final}\input{childdoc.def}\childdocforward{cdocsch2}"|
% \end{tabular}
% \end{center}
% Note that the trailing backslash on each first line
% merely continues the input to the second line
% (for convenient cut ant paste).
% Furthermore, the command |latex| can be replaced by any
% of its alternative versions such as |pdflatex|.
%
% %%%%%%%%%%%%%%%%%%%%%%%%%%%%%%%%%%%%%%%%%%%%%%%%%%%%%%%%%%%%%%%%%%%%%%%%%%%%%%
% %%%%%%%%%%%%%%%%%%%%%%%%%%%%%%%%%%%%%%%%%%%%%%%%%%%%%%%%%%%%%%%%%%%%%%%%%%%%%%
% \section{Implementation}
%\iffalse
%<*package>
%\fi
%
% This section describes the definitions file |childdoc.def|.

% The definitions cannot be loaded using |\usepackage| or |\RequirePackage|
% which has a mechanism to prevent loading a style file more than once.
% When loading the definitions by means of |\input|
% multiple instances have to be prevented manually:
%\iffalse
%This code needs to be before the `\ProvidesFile' directive
%which is defined at the beginning of this file.
%Therefore it is also placed there and commented out here.
%</package>
%<*discard>
%\fi
%    \begin{macrocode}
\ifdefined\childdocmain\endinput\fi
%    \end{macrocode}
%\iffalse
%</discard>
%<*package>
%\fi
%
% \macro{\ifchilddoc}
% \macro{\ifchilddocmanual}
% The conditional |\ifchilddoc| tells whether a
% child (true) or main (false) document is being compiled.
% The conditional |\ifchilddocmanual| tells whether
% the |\includeonly| mechanism is used (false) or
% the selection of child files must be performed manually (true).
% The definitions initialise to false:
%    \begin{macrocode}
\newif\ifchilddoc
\newif\ifchilddocmanual
%    \end{macrocode}

% \macro{\childdocname}
% \macro{\childdocjob}
% The macro |\childdocname| stores the name of the main document
% to be compiled. The macro |\childdocjob| stores the name of
% the document on which the \LaTeX{} compiler was originally invoked.
% The content of |\jobname| cannot be compared
% to filenames specified in the source due to different catcodes.
% The following code rescans |\jobname|, stores the result
% in |\childdocname| and saves a copy in |\childdocjob|:
%    \begin{macrocode}
\edef\childdocname{\scantokens\expandafter{\jobname\noexpand}}
\let\childdocjob\childdocname
%    \end{macrocode}

% \macro{\childdocdisable}
% The macro |\childdocdisable| prevents the main file
% from being processed more than once.
% At this stage, the main document command |\childdocmain|
% is assumed to be called once again where it should do nothing.
% Any subsequent call to it should prevent
% a secondary processing of the main document
% It overwrites the forwarding commands
% |\childdocof| and |\childdocforward|
% with empty macros to prevent further inclusions of the main document:
%    \begin{macrocode}
\newcommand{\childdocdisable}
{
  \renewcommand{\childdocmain}[1]{\renewcommand{\childdocmain}[1]{\endinput}}
  \renewcommand{\childdocof}[1]{}
  \renewcommand{\childdocby}[2][]{}
  \renewcommand{\childdocforward}[2][]{}
  \renewcommand{\childdocdisable}{}
}
%    \end{macrocode}

% \macro{\childdocmain}
% The macro |\childdocmain| is to be called at the top of the main file
% with nothing or the main filename (without extension) as argument.
% First, it breaks loops.
% If the argument is not empty and does not match |\childdocname|
% (which is set by the first inclusion of |childdoc.def|),
% |\ifchilddoc| is set to true, |\includeonly| is applied to the child file
% and |\jobname| is set to the main file
% (for proper handling of |.aux| files):
%    \begin{macrocode}
\newcommand{\childdocmain}[1]
{
  \childdocdisable\childdocmain{}
  \if?#1?\else
    \begingroup
      \def\childdoctmp{#1}
      \ifx\childdoctmp\childdocname
        \def\childdoctmp{}
      \else
        \def\childdoctmp
        {
          \childdoctrue
          \includeonly{\childdocname}
          \def\childdocjob{#1}
          \def\jobname{#1}
        }
      \fi
      \expandafter
    \endgroup
    \childdoctmp
  \fi
}
%    \end{macrocode}

% \macro{\childdocof}
% The command |\childdocof| redirects
% compilation to the main file |#1|.
%    \begin{macrocode}
\newcommand{\childdocof}[1]
{
  \childdocdisable
  \childdoctrue
  \includeonly{\childdocname}
  \def\jobname{#1}
  \def\childdocjob{#1}
  \input{#1}
}
%    \end{macrocode}

% \macro{\childdocby}
% The command |\childdocby| ....
%    \begin{macrocode}
\newcommand{\childdocby}[2][]
{
  \childdocdisable
  \childdoctrue
  \childdocmanualtrue
  \if?#1?\else
    \def\jobname{#2}
  \fi
  \def\childdocjob{#2}
  \input{#2}
  \endinput
}
%    \end{macrocode}

% \macro{\childdocforward}
% The command |\childdocforward| redirects
% compilation to the main file or
% (if the optional argument is given) a child file.
% Parameters are set as if the main file
% or a child file starting with |\childdocof| was compiled.
% Then compilation is handed over to the main file:
%    \begin{macrocode}
\newcommand{\childdocforward}[2][]
{
  \begingroup
    \if?#1?
      \def\childdoctmp
      {
        \def\childdocname{#2}
        \def\childdocjob{#2}
        \def\jobname{#2}
        \input{#2}
        \endinput
      }
    \else
      \def\childdoctmp
      {
        \childdocdisable
        \def\childdocname{#2}
        \childdoctrue
        \includeonly{#2}
        \def\childdocjob{#1}
        \def\jobname{#1}
        \input{#1}
        \endinput
      }
    \fi
    \expandafter
  \endgroup
  \childdoctmp
}
%    \end{macrocode}

% \macro{\childdocforwardprefix}
% The command |\childdocforwardprefix| redirects
% compilation to the main or a child file by means of a pattern.
% The prefix |#1| in the current filename is replaced by |#2|
% and the suffix of the current filename is kept
% (it is assumed that the filename does not contain the substring `|~~~|'
% which is used as a delimiter).
% Compilation is handed over to the new file by |\childdocforward|:
%    \begin{macrocode}
\newcommand{\childdocforwardprefix}[3][]
{
  \begingroup
    \def\childdocextract #2##1~~~{\def\childdoctmp{\childdocforward[#1]{#3##1}}}
    \expandafter\childdocextract\childdocname~~~
    \expandafter
  \endgroup
  \childdoctmp
}
%    \end{macrocode}

% \macro{\childdoc}
% The deprecated macro |\childdoc| is a legacy version of |\childdocmain|:
%    \begin{macrocode}
\newcommand{\childdoc}{\childdocmain}
%    \end{macrocode}

% \macro{\childdocredirect}
% The deprecated macro |\childdocredirect| is a legacy version
% of |\childdocforward| and |\childdocforwardprefix|:
%    \begin{macrocode}
\newcommand{\childdocredirect}[2][]
{
  \begingroup
    \if?#1?
      \def\childdoctmp{\childdocforward{#2}}
    \else
      \def\childdoctmp{\childdocforwardprefix{#1}{#2}}
    \fi
    \expandafter
  \endgroup
  \childdoctmp
}
%    \end{macrocode}

%\iffalse
%</package>
%\fi
%
\endinput

\childdocof{cdocsamp}
%    \end{macrocode}

%\iffalse
%</samplechap1|samplechap2>
%\fi
%
%\iffalse
%<*samplechap1>
%\fi
% Some text for chapter 1:
%    \begin{macrocode}
\section{one}
some text in chapter one
%    \end{macrocode}

%\iffalse
%</samplechap1>
%\fi
% Some text for chapter 2:
%\iffalse
%<*samplechap2>
%\fi
%    \begin{macrocode}
\section{two}
more text in chapter two
%    \end{macrocode}

%\iffalse
%</samplechap2>
%\fi
%
% %%%%%%%%%%%%%%%%%%%%%%%%%%%%%%%%%%%%%%
% \paragraph{Part Include Files.}
%
% The include files are called |cdocspt3.tex| and |cdocspt4.tex|.
%
%\iffalse
%<*samplepart3|samplepart4>
%\fi

% Optional override for |\version| flag:
%    \begin{macrocode}
%%\providecommand{\version}{final}
%    \end{macrocode}

% Include the main document:
%    \begin{macrocode}
% \iffalse
%
% childdoc.dtx Copyright (C) 2017-2018 Niklas Beisert
%
% This work may be distributed and/or modified under the
% conditions of the LaTeX Project Public License, either version 1.3
% of this license or (at your option) any later version.
% The latest version of this license is in
%   http://www.latex-project.org/lppl.txt
% and version 1.3 or later is part of all distributions of LaTeX
% version 2005/12/01 or later.
%
% This work has the LPPL maintenance status `maintained'.
%
% The Current Maintainer of this work is Niklas Beisert.
%
% This work consists of the files childdoc.dtx and childdoc.ins
% and the derived files childdoc.def and cdocsamp.tex with
% cdocsch1.tex, cdocsch2.tex, cdocsdrf.tex, cdocsfn1.tex, cdocsfn2.tex.
%
%<package>\ifdefined\childdocmain\endinput\fi
%<package>\ProvidesFile{childdoc.def}[2018/12/30 v2.0 child document driver]
%<samplemain>\ProvidesFile{cdocsamp.tex}[2018/12/30 v2.0 sample for childdoc]
%<*driver>
%\ProvidesFile{childdoc.drv}[2018/12/30 v2.0 childdoc reference manual file]
\PassOptionsToClass{10pt,a4paper}{article}
\documentclass{ltxdoc}

\usepackage[margin=35mm]{geometry}
\usepackage{hyperref}
\usepackage{hyperxmp}
\usepackage[usenames]{color}

\hypersetup{colorlinks=true}
\hypersetup{pdfstartview=FitH}
\hypersetup{pdfpagemode=UseNone}
\hypersetup{pdfsource={}}
\hypersetup{pdflang={en-UK}}
\hypersetup{pdfcopyright={Copyright 2017-2018 Niklas Beisert.
  This work may be distributed and/or modified under the
  conditions of the LaTeX Project Public License, either version 1.3
  of this license or (at your option) any later version.}}
\hypersetup{pdflicenseurl={http://www.latex-project.org/lppl.txt}}
\hypersetup{pdfcontactaddress={ETH Zurich, ITP, HIT K,
  Wolfgang-Pauli-Strasse 27}}
\hypersetup{pdfcontactpostcode={8093}}
\hypersetup{pdfcontactcity={Zurich}}
\hypersetup{pdfcontactcountry={Switzerland}}
\hypersetup{pdfcontactemail={nbeisert@itp.phys.ethz.ch}}
\hypersetup{pdfcontacturl={http://people.phys.ethz.ch/\xmptilde nbeisert/}}

\newcommand{\secref}[1]{\hyperref[#1]{section \ref*{#1}}}

\parskip1ex
\parindent0pt
\let\olditemize\itemize
\def\itemize{\olditemize\parskip0pt}

\begin{document}

\title{The \textsf{childdoc} Package}
\hypersetup{pdftitle={The childdoc Package}}
\author{Niklas Beisert\\[2ex]
  Institut f\"ur Theoretische Physik\\
  Eidgen\"ossische Technische Hochschule Z\"urich\\
  Wolfgang-Pauli-Strasse 27, 8093 Z\"urich, Switzerland\\[1ex]
  \href{mailto:nbeisert@itp.phys.ethz.ch}
  {\texttt{nbeisert@itp.phys.ethz.ch}}}
\hypersetup{pdfauthor={Niklas Beisert}}
\hypersetup{pdfsubject={Manual for the LaTeX2e Package childdoc}}
\date{30 December 2018, \textsf{v2.0}}
\maketitle

\begin{abstract}\noindent
\textsf{childdoc} is a \LaTeXe{} package
that enables the direct compilation
of document sections included by |\include|
to individual files.
\end{abstract}

\begingroup
\parskip0ex
\tableofcontents
\endgroup

%%%%%%%%%%%%%%%%%%%%%%%%%%%%%%%%%%%%%%%%%%%%%%%%%%%%%%%%%%%%%%%%%%%%%%%%%%%%%%%%
%%%%%%%%%%%%%%%%%%%%%%%%%%%%%%%%%%%%%%%%%%%%%%%%%%%%%%%%%%%%%%%%%%%%%%%%%%%%%%%%
\section{Introduction}

\LaTeX{} provides a mechanism to structure a large document (such as a book)
into a main file and several child files (containing the chapters)
using the |\include| command.
This mechanism is beneficial for documents
which span hundreds of pages in order to
make the source file(s) more manageable.
Moreover, compilation can be restricted to
selected child files by means of the |\includeonly| command.
The latter feature can be used to reduce the compilation time while editing
(this was significantly more useful in the earlier days of \LaTeX{})
or to generate a smaller document which is easier to navigate.
Another application of |\includeonly| is to generate
documents consisting of selected parts of the complete document.

However, there are a few drawbacks of the plain |\include| mechanism:
\begin{itemize}
\item
The child files cannot be compiled on their own,
they can only be compiled via the main file.
A naive editing environment
(such as a text editor with an option
to have the current file processed by \LaTeX)
may require one to switch to the main file before compiling;
attempting to compile the child file produces errors.
\item
The main file must be modified (each time)
to adjust the |\includeonly| command
to the present needs. This easily leaves the main file in a messy state.
\item
The generated document will always carry the filename
of the main document. This is inconvenient if
several child files are to be compiled and
to be kept for distribution.
\end{itemize}

The present package provides a simple interface
to make child files individually compilable by \LaTeX{}.
Compiling a child file then has the same effect as compiling
the main file with an |\includeonly| command
to select the appropriate child.
Moreover the generated document will carry the name of the child
rather than the main file.
This resolves all three above issues.

This feature is meant to make the editing of books,
thesis documents and lecture notes somewhat more convenient.
However, the package can also be used efficiently for
composing a series of documents (such as exercise sheets)
which are typically distributed individually.
It then assists the author in generating the individual documents
(potentially in different versions)
as well as a document containing the collected series.
Another application is in developing style files
or other kinds of included material
where compilation of the style file could redirect
to a sample or test file.

%%%%%%%%%%%%%%%%%%%%%%%%%%%%%%%%%%%%%%%%%%%%%%%%%%%%%%%%%%%%%%%%%%%%%%%%%%%%%%%%
%%%%%%%%%%%%%%%%%%%%%%%%%%%%%%%%%%%%%%%%%%%%%%%%%%%%%%%%%%%%%%%%%%%%%%%%%%%%%%%%
\section{Usage}

First of all, the package \textsf{childdoc} is \emph{not} a standard
\LaTeXe{} |.sty| style file! Therefore it needs to be invoked in
a non-standard way.

%%%%%%%%%%%%%%%%%%%%%%%%%%%%%%%%%%%%%%%%%%%%%%%%%%%%%%%%%%%%%%%%%%%%%%%%%%%%%%%%
\subsection{Included Files}
\label{sec:include}

%%%%%%%%%%%%%%%%%%%%%%%%%%%%%%%%%%%%%%%%
\DescribeMacro{\childdocmain}
To use the package, add the commands
\begin{center}
\begin{tabular}{l}
|\input{childdoc.def}|\\
|\childdocmain{}|\\
\end{tabular}
\end{center}
at the very top of the main \LaTeX{} file,
in particular \emph{before} the |\documentclass| statement!
The argument of |\childdocmain| should be left empty
(but it must be present).

%%%%%%%%%%%%%%%%%%%%%%%%%%%%%%%%%%%%%%%%
\DescribeMacro{\childdocof}
Furthermore, add the commands
\begin{center}
\begin{tabular}{l}
|\input{childdoc.def}|\\
|\childdocof{|\textit{main}|}|\\
\end{tabular}
\end{center}
at the top of every child file \textit{child}
which is included by |\include{|\textit{child}|}|
from within the main file
(or at least for those files to be compiled individually).
The argument \textit{main} must be the filename of the main file.

There are a couple of
considerations in setting up the main and child documents:

%%%%%%%%%%%%%%%%%%%%%%%%%%%%%%%%%%%%%%%%
\paragraph{Restrictions.}

Please note the following restrictions:
\begin{itemize}
\item
|\childdocmain| must be called with one argument \textit{main}
to ensure compatibility with earlier version of the package.
It must either be empty (|\childdocmain{}|)
or precisely match the filename of the main file in which it is specified.
See \secref{sec:detection} for further information.
\item
The filename \textit{main} must be specified without the |.tex| extension.
\item
The filename \textit{main} is case sensitive
(even in case-insensitive file systems)
due to internal string comparison.
\item
The argument \textit{main} should be fully expanded, it cannot be a macro.
\item
Subdirectories and special characters should be avoided in filenames.
\item
The command |\childdocmain{|\textit{main}|}| must be followed by a whitespace.
It should not be followed immediately by another command
or by a comment mark `|%|'.
This is because the \TeX{} parser reads the token immediately following
the argument of |\childdocmain| and puts it
at the beginning of every child section;
however, a white\-space is ignored.
\end{itemize}

%%%%%%%%%%%%%%%%%%%%%%%%%%%%%%%%%%%%%%%%
\paragraph{Content of Main File.}

It is advisable to place all content in the child files included by |\include|.
Any output contained in the main file will appear in all child documents
unless suppressed manually;
it cannot be suppressed automatically by the |\includeonly| directive
and thus should normally be avoided.
A method to include some content in the main file
by means of conditional processing is described in \secref{sec:conditional}.

%%%%%%%%%%%%%%%%%%%%%%%%%%%%%%%%%%%%%%%%
\paragraph{Page Numbering.}

When only a part of the document is compiled,
the appropriate numbering of pages
(as well as other status parameters)
is determined from the |.aux| files.
The latter contain information from previous passes.
However this information needs to propagate through
all intermediate child documents.
Therefore the page numbering in child documents may well
be inconsistent until the complete document is compiled at least once.

A useful (if unconventional) way to always ensure a consistent
page numbering is to restart the numbering in each child document
and denote the pages by `\textit{child}|.|\textit{page}'
where \textit{child} represents the chapter/section number of the child file.
This can be achieved by the command
|\numberwithin{page}{|\textit{child}|}|
of the \textsf{amsmath} package
where \textit{child} can be |chapter| or |section|
depending on the chosen structuring.
Alternatively, one can modify the macro |\thepage| appropriately
and reset the counter |page| at the start of each child file.

%%%%%%%%%%%%%%%%%%%%%%%%%%%%%%%%%%%%%%%%%%%%%%%%%%%%%%%%%%%%%%%%%%%%%%%%%%%%%%%%
\subsection{Conditional Processing}
\label{sec:conditional}

The package provides a mechanism to compile different versions
of a document. To customise the versions further some conditional processing
can come in handy to distinguish which version is being compiled.
The package provides two macros to describe the compilation context:

%%%%%%%%%%%%%%%%%%%%%%%%%%%%%%%%%%%%%%%%
\DescribeMacro{\ifchilddoc}
The conditional |\ifchilddoc| distinguishes between the compilation of
child documents and the main document:
%
\begin{center}
|\ifchilddoc |\textit{child-code}| |[|\||else |\textit{main-code}]| \||fi|
\end{center}

%%%%%%%%%%%%%%%%%%%%%%%%%%%%%%%%%%%%%%%%
\DescribeMacro{\childdocname}
\DescribeMacro{\childdocjob}
The macro |\childdocname| contains the filename (without extension)
of the main or child file being processed.
Note that |\childdocjob| will always contain the name of the main file.

%%%%%%%%%%%%%%%%%%%%%%%%%%%%%%%%%%%%%%%%
\paragraph{Title Page.}

Conditional processing can be used to include a title or banner page
in the main document when proper precautions are taken.
Importantly, the code in the main file should ensure that the page counter
(as well as other status parameters which are stored in the |.aux| files)
takes the same value after the conditional processing.
Otherwise the page numbers may take divergent values
depending on which part is compiled.

For example, a title page could be declared by:
%
\begin{center}
\begin{tabular}{l}
|\ifchilddoc\||else|\\
|\addtocounter{page}{-1}|\\
\textit{code for title page}\\
|\newpage|\\
|\||fi|
\end{tabular}
\end{center}
%
A banner page for the child documents can be generated by:
%
\begin{center}
\begin{tabular}{l}
|\ifchilddoc|\\
|\addtocounter{page}{-1}|\\
\textit{code for banner page}\\
|\newpage|\\
|\||fi|
\end{tabular}
\end{center}
%
Here one could write a message such as:
\begin{center}
|This is the part \childdocname{} of \childdocjob{}.|
\end{center}

%%%%%%%%%%%%%%%%%%%%%%%%%%%%%%%%%%%%%%%%%%%%%%%%%%%%%%%%%%%%%%%%%%%%%%%%%%%%%%%%
\subsection{Flags}
\label{sec:flags}

The package makes it easy to generate different versions
of the main or child documents.
To this end compilation flags can be defined
and assigned different default values.
They will be particularly useful in conjunction
with the forwarding mechanism described in \secref{sec:forward}.

For example, it may be useful to have a flag |\version|
which can be set to |draft| or |final|.
The document source will contain some conditional code
depending on the value of |\version|.
Suppose further, the flag should default to |final| for the main file
and to |draft| for child files
which is a natural assignment for editing the document.
This is achieved by placing the following code
in the preamble of the main document
(below the |\childdocmain| directive):
%
\begin{center}
\begin{tabular}{l}
|\ifchilddoc|\\
|\providecommand{\version}{draft}|\\
|\||else|\\
|\providecommand{\version}{final}|\\
|\||fi|
\end{tabular}
\end{center}
%
The definition by |\providecommand| makes sure
that previous definitions are not overwritten.
Further statements |\providecommand{\version}{...}|
can thus be added before the above code to override it.

For the main file, one might add a line
(between |\childdocmain| and the above block)
%
\begin{center}
|%\ifchilddoc\||else\providecommand{\version}{draft}\||fi|
\end{center}
%
which can be uncommented to produce a draft version.
Likewise one can add a line to the very top of a child file
(above the |\childdocof{|\textit{main}|}| directive)
%
\begin{center}
|%\providecommand{\version}{final}|
\end{center}
%
which can be uncommented to produce the final version of this child document.

%%%%%%%%%%%%%%%%%%%%%%%%%%%%%%%%%%%%%%%%%%%%%%%%%%%%%%%%%%%%%%%%%%%%%%%%%%%%%%%%
\subsection{Forwarding}
\label{sec:forward}

Different versions of the main or child documents
using compilation flags as described in \secref{sec:flags}
can be (permanently) stored in different files
for convenient compilation, viewing and distribution.
To this end, the package defines a command
to pass on compilation to a different file:

%%%%%%%%%%%%%%%%%%%%%%%%%%%%%%%%%%%%%%%%
\DescribeMacro{\childdocforward}
The command |\childdocforward| redirects processing to
another source file:
%
\begin{center}
\begin{tabular}{l}
|\input{childdoc.def}|\\
|\childdocforward[|\textit{main}|]{|\textit{dest}|}|\\
\end{tabular}
\end{center}
%
The argument \textit{dest} is the destination file
(without extension).
It should be the main file or one of the child files.
Note that further \textsf{childdoc} directives
such as |\childdocof| and |\childdocforward|
in the indicated file will be processed in this form.
The optional argument \textit{main}
passes on directly to the main file \textit{main}
while pretending to compile the child \textit{dest}.
This form behaves as if \textit{dest}
issues |\childdocof{|\textit{main}|}| right away,
and no further \textsf{childdoc} directives will be processed.

%%%%%%%%%%%%%%%%%%%%%%%%%%%%%%%%%%%%%%%%
\DescribeMacro{\...prefix}
In the alternative form |\childdocforwardprefix|,
%
\begin{center}
\begin{tabular}{l}
|\input{childdoc.def}|\\
|\childdocforwardprefix[|\textit{main}|]{|\textit{prefix}|}{|\textit{dest}|}|
\end{tabular}
\end{center}
%
the destination file is determined by a pattern
depending on the current file:
To make this work, the current file must be called
`{\textit{prefix}\hspace{0.2em}\textit{suffix}}'
with \textit{prefix} matching precisely the argument.
Processing is then passed on to the file
`{\textit{dest}\hspace{0.2em}\textit{suffix}}'.
Surely, the same effect is achieved by
directly specifying the
argument `{\textit{dest}\hspace{0.2em}\textit{suffix}}'
in the first form.
However, that requires to set up a different file
for each child. With the alternative form of the command
all these files can have exactly the same content
which simplifies setting them up and maintaining them.

For example, the following file |draft.tex|
with a compilation flag |\version| as described in \secref{sec:flags}
compiles the main document as a draft:
%
\begin{center}
\begin{tabular}{l}
|\def\version{draft}|\\
|\input{childdoc.def}|\\
|\childdocforward{|\textit{main}|}|
\end{tabular}
\end{center}
%
Likewise, the following files |final|\textit{nn}|.tex|
compile the final version of the child document
|child|\textit{nn}|.tex|:
%
\begin{center}
\begin{tabular}{l}
|\def\version{final}|\\
|\input{childdoc.def}|\\
|\childdocforwardprefix{final}{child}|
\end{tabular}
\end{center}
%

Note that when several versions of a main file and/or of each child file
are to be generated, it may be convenient to set up a |Makefile| or
shell script to automatise the process.

%%%%%%%%%%%%%%%%%%%%%%%%%%%%%%%%%%%%%%%%%%%%%%%%%%%%%%%%%%%%%%%%%%%%%%%%%%%%%%%%
\subsection{Command Line Processing}
\label{sec:commandline}

The effect of redirection files can also be achieved by invoking
the \LaTeX{} compiler with a more elaborate command line.
Most conveniently this should be done as part
of a shell script or a |Makefile|.

When using \textsf{childdoc} in the main file, the following
command lines effectively perform a redirection
(note that depending on the shell being used,
backslashes may have to be doubled: `|\|' $\to$ `|\\|'):
%
\begin{center}
|... -jobname "|\textit{target}|" |\\|"|[\textit{flags}]%
|\input{childdoc.def}\childdocforward[|\textit{main}|]{|\textit{dest}|}"|
\end{center}
%
Here \textit{target} is the name of the output file,
\textit{main} is the name of the main file
and \textit{dest} is the name of the main or child file to be processed
(all filenames without extensions).
The optional argument \textit{main} can be omitted
if \textit{main} matches \textit{dest}.
Optionally, compilation \textit{flags} can be defined via |\def| commands.
This command line makes the \TeX{} engine believe
it is compiling the file \textit{target}
whose content is specified as the latter parameter.
The provided code then forwards the processing to
\textit{main} or \textit{dest} as described in \secref{sec:forward}.

%%%%%%%%%%%%%%%%%%%%%%%%%%%%%%%%%%%%%%%%%%%%%%%%%%%%%%%%%%%%%%%%%%%%%%%%%%%%%%%%
\subsection{Include by Input}
\label{sec:input}

Including child documents by |\include| has some restrictions by design.
Most notably, the content of a child document always occupies
its own set of pages; pages cannot be shared between child documents.
Usually, this behaviour makes perfect sense
because each child document contain an essential part of the document.
However, in some situations it may be desirable to compose
a document from a collection of parts
without having mandatory page breaks between then.
For this case, the package
provides a mechanism to include parts
by |\input| which can also be processed individually.
However, by construction this mechanism
requires manual handling of the content to be output.

%%%%%%%%%%%%%%%%%%%%%%%%%%%%%%%%%%%%%%%%
\DescribeMacro{\ifchilddocmanual}
The main file should be prepared as usual, see \secref{sec:include}.
However, the document body must make a distinction
between processing of an individual part and of the main document, e.g.:
%
\begin{center}
\begin{tabular}{l}
|\ifchilddocmanual|\\
|\input{\childdocname}|\\
|\||else|\\
\textit{document body with }|\input{|\textit{part}|}|\\
|\||fi|
\end{tabular}
\end{center}
%
The conditional |\ifchilddocmanual| is true whenever
a part to be included by |\input| is being compiled,
and the name of the part is stored in |\childdocname|.

%%%%%%%%%%%%%%%%%%%%%%%%%%%%%%%%%%%%%%%%
\DescribeMacro{\childdocby}
Each part to be included by |\input| should start with:
%
\begin{center}
\begin{tabular}{l}
|\input{childdoc.def}|\\
|\childdocby{|\textit{main}|}|\\
\end{tabular}
\end{center}
%
The directive |\childdocby| is similar to |\childdocof|
described in \secref{sec:include},
but the subsequent selection of content must be done manually.
To that end, both |\ifchilddoc| and |\ifchilddocmanual|
will be true upon processing of a part,
and the name of the part is stored in |\childdocname|.
Note that |\jobname| will be set to the filename of the current part
so that each part receives an individual |.aux| file
that does not interfere with the |.aux| file(s) of the main document.
This behaviour can be altered by the alternative form
|\childdocby[*]{|\textit{main}|}| (with a non-empty optional argument)
which uses the |.aux| file of the main document
by setting |\jobname| to \textit{main}.

%%%%%%%%%%%%%%%%%%%%%%%%%%%%%%%%%%%%%%%%%%%%%%%%%%%%%%%%%%%%%%%%%%%%%%%%%%%%%%%%
\subsection{Driver Development}
\label{sec:driver}

The \textsf{childdoc} mechanism can also be use for the development
of definition files such as \LaTeX{} styles or classes.
This case differs from the above setup with multiple parts
included by |\include| in that no |\includeonly| should be invoked.
This can be achieved by starting the include file
(before |\ProvidesPackage|) with:
%
\begin{center}
\begin{tabular}{l}
|\input{childdoc.def}|\\
|\childdocforward{|\textit{main}|}|\\
\end{tabular}
\end{center}
%
or alternatively with:
%
\begin{center}
\begin{tabular}{l}
|\input{childdoc.def}|\\
|\childdocby{|\textit{main}|}|\\
\end{tabular}
\end{center}
%
Both forms have slightly different effects as described above.
The main file is prepared as usual, see \secref{sec:include}.

%%%%%%%%%%%%%%%%%%%%%%%%%%%%%%%%%%%%%%%%%%%%%%%%%%%%%%%%%%%%%%%%%%%%%%%%%%%%%%%%
\subsection{Legacy Detection}
\label{sec:detection}

The directive |\childdocmain| in the main file can detect
whether the complete document or merely a child is to be compiled
even without using the directive |\childdocof|.
This method is deprecated because it is less robust
and there is no compelling reason to use it;
it is merely provided for backward compatibility
and it may be removed in future versions.

If the detection mechanism is to be used,
it is mandatory to correctly specify
the filename of the main file as the argument of |\childdocmain|:
%
\begin{center}
\begin{tabular}{l}
|\input{childdoc.def}|\\
|\childdocmain{|\textit{main}|}|\\
\end{tabular}
\end{center}
%
If |\jobname| does not match the argument \textit{main} of |\childdocmain|,
it is assumed that |\jobname| points to the child file to be compiled.
When using |\childdocmain| with the main file specified as argument,
it suffices to start a child file
with just |\input{|\textit{main}|}|
without loading of the package and using |\childdocof|.
If instead all processing is done
with the appropriate \textsf{childdoc} directives,
the argument of \textit{main} of |\childdocmain| can be empty.

An alternative version of the command line processing described
in \secref{sec:commandline} using the detection mechanism reads:
%
\begin{center}
|... -jobname "|\textit{target}|" "|[\textit{flags}]%
[|\def\jobname{|\textit{dest}|}|]|\input{|\textit{main}|}"|
\end{center}

%%%%%%%%%%%%%%%%%%%%%%%%%%%%%%%%%%%%%%%%%%%%%%%%%%%%%%%%%%%%%%%%%%%%%%%%%%%%%%%%
\subsection{Manual Code}
\label{sec:manual}

In case one cannot be certain whether the definitions file |childdoc.def|
is installed on the target \TeX{} distribution
and one prefers not to ship it,
it is conceivable to paste a few relevant commands into the sources.

To that end, drop all statements |\input{childdoc.def}|
and perform the replacements as outlined below.
Instead of |\childdocmain{|\textit{main}|}| add the following code
to the top of the main file:
%
\begin{center}
\begin{tabular}{l}
|\||ifdefined\childdocname\endinput\||fi\newif\ifchilddoc|\\
|\edef\childdocname{\scantokens\expandafter{\jobname\noexpand}}|\\
|\def\childdocmain{|\textit{main}|}\||ifx\childdocmain\childdocname\||else|\\
|\childdoctrue\includeonly{\childdocname}\let\jobname\childdocmain\||fi|\\
\end{tabular}
\end{center}
%
Instead of |\childdocof{|\textit{main}|}| just include the main file
at the top of each child file:
%
\begin{center}
|\input{|\textit{main}|}|
\end{center}
%
A simple redirection |\childdocforward{|\textit{dest}|}| is achieved by:
%
\begin{center}
|\def\jobname{|\textit{dest}|}\input{\jobname}|
\end{center}
%
The redirection with prefix
|\childdocforwardprefix[|\textit{prefix}|]{|\textit{dest}|}|
is accomplished by:
%
\begin{center}
\begin{tabular}{l}
|{\edef\jobname{\scantokens\expandafter{\jobname\noexpand}}|\\
|\def\redirectjob |\textit{prefix}|#1~~~{\gdef\jobname{|\textit{dest}|#1}}|\\
|\expandafter\redirectjob\jobname~~~}\input{\jobname}|
\end{tabular}
\end{center}

In an alternative approach,
child documents can be compiled by a specific command line
without additional code or specific definitions:
%
\begin{center}
|... -jobname "|\textit{target}|" "|[\textit{flags}]%
|\includeonly{|\textit{dest}|}\input{|\textit{main}|}"|
\end{center}
%

%%%%%%%%%%%%%%%%%%%%%%%%%%%%%%%%%%%%%%%%%%%%%%%%%%%%%%%%%%%%%%%%%%%%%%%%%%%%%%%%
%%%%%%%%%%%%%%%%%%%%%%%%%%%%%%%%%%%%%%%%%%%%%%%%%%%%%%%%%%%%%%%%%%%%%%%%%%%%%%%%
\section{Information}

%%%%%%%%%%%%%%%%%%%%%%%%%%%%%%%%%%%%%%%%%%%%%%%%%%%%%%%%%%%%%%%%%%%%%%%%%%%%%%%%
\subsection{Copyright}

Copyright \copyright{} 2017--2018 Niklas Beisert

This work may be distributed and/or modified under the
conditions of the \LaTeX{} Project Public License, either version 1.3
of this license or (at your option) any later version.
The latest version of this license is in
  \url{http://www.latex-project.org/lppl.txt}
and version 1.3 or later is part of all distributions of \LaTeX{}
version 2005/12/01 or later.

This work has the LPPL maintenance status `maintained'.

The Current Maintainer of this work is Niklas Beisert.

This work consists of the files |README.txt|, |childdoc.ins| and |childdoc.dtx|
as well as the derived files |childdoc.def|, |cdocsamp.tex|
with |cdocsch1.tex|, |cdocsch2.tex|, |cdocspt3.tex|, |cdocspt4.tex|,
|cdocsdrf.tex|, |cdocsfn1.tex|, |cdocsfn2.tex|
as well as |childdoc.pdf|.

%%%%%%%%%%%%%%%%%%%%%%%%%%%%%%%%%%%%%%%%%%%%%%%%%%%%%%%%%%%%%%%%%%%%%%%%%%%%%%%%
\subsection{Files and Installation}

The package consists of the files:
%
\begin{center}
\begin{tabular}{ll}
    |README.txt|   & readme file \\
    |childdoc.ins| & installation file \\
    |childdoc.dtx| & source file \\
    |childdoc.def| & definition file \\
    |cdocsamp.tex| & sample main file \\
    |cdocsch1.tex| & sample include file \\
    |cdocsch2.tex| & sample include file \\
    |cdocspt3.tex| & sample part file \\
    |cdocspt4.tex| & sample part file \\
    |cdocsdrf.tex| & sample redirection file \\
    |cdocsfn1.tex| & sample redirection file \\
    |cdocsfn2.tex| & sample redirection file \\
    |childdoc.pdf| & manual
\end{tabular}
\end{center}
%
The distribution consists of the files
|README.txt|, |childdoc.ins| and |childdoc.dtx|.
%
\begin{itemize}
\item
Run (pdf)\LaTeX{} on |childdoc.dtx|
to compile the manual |childdoc.pdf| (this file).
\item
Run \LaTeX{} on |childdoc.ins| to create the definitions file |childdoc.def|
and the sample |cdocsamp.tex| with include files
|cdocsch1.tex|, |cdocsch2.tex|, |cdocspt3.tex|, |cdocspt4.tex|,
|cdocsdrf.tex|, |cdocsfn1.tex|, |cdocsfn2.tex|.
Then copy the file |childdoc.def| to an appropriate directory of your \LaTeX{}
distribution, e.g.\ \textit{texmf-root}|/tex/latex/childdoc|.
\end{itemize}

%%%%%%%%%%%%%%%%%%%%%%%%%%%%%%%%%%%%%%%%%%%%%%%%%%%%%%%%%%%%%%%%%%%%%%%%%%%%%%%%
\subsection{Related CTAN Packages}

There are several other packages which offer a similar functionality:
%
\begin{itemize}
\item
The packages
\href{http://ctan.org/pkg/docmute}{\textsf{docmute}},
\href{http://ctan.org/pkg/includex}{\textsf{includex}} and
\href{http://ctan.org/pkg/standalone}{\textsf{standalone}}
provide commands to include only the document body of
a child file thus allowing both files to be compiled individually.
\item
The packages \href{http://ctan.org/pkg/subdocs}{\textsf{subdocs}}
and \href{http://ctan.org/pkg/subfiles}{\textsf{subfiles}}
provide structures in which the main and child documents can be
encapsulated and allowing them to be compiled individually.
The inclusion mechanism is different from the conventional |\include|.
\item
The package \href{http://ctan.org/pkg/combine}{\textsf{combine}}
is an elaborate solution to combine several documents into one.
\end{itemize}
%
See also the CTAN topic \href{http://ctan.org/topic/subdocs}{\textsf{subdocs}}
for further related packages.
The present package differs from the above solutions in that
a document structure constructed with the conventional |\include| mechanism
just needs two extra commands at the top of every file
such that all constituent files can be compiled individually.

%%%%%%%%%%%%%%%%%%%%%%%%%%%%%%%%%%%%%%%%%%%%%%%%%%%%%%%%%%%%%%%%%%%%%%%%%%%%%%%%
%\subsection{Feature Suggestions}
%
%The following is a list of features which may be useful for future
%versions of this package:
%%
%\begin{itemize}
%\item
%\ldots
%\end{itemize}

%%%%%%%%%%%%%%%%%%%%%%%%%%%%%%%%%%%%%%%%%%%%%%%%%%%%%%%%%%%%%%%%%%%%%%%%%%%%%%%%
\subsection{Revision History}

%%%%%%%%%%%%%%%%%%%%%%%%%%%%%%%%%%%%%%%%
\paragraph{v2.0:} 2018/12/30

\begin{itemize}
\item
immediate forward processing
\item
added |\childdocby| mechanism
\item
manual restructured
\end{itemize}

%%%%%%%%%%%%%%%%%%%%%%%%%%%%%%%%%%%%%%%%
\paragraph{v1.6:} 2018/01/17

\begin{itemize}
\item
application for development of include files
\item
corrections to manual
\end{itemize}

%%%%%%%%%%%%%%%%%%%%%%%%%%%%%%%%%%%%%%%%
\paragraph{v1.5:} 2017/05/21

\begin{itemize}
\item
more complete structuring introduced
\item
|\childdocof| introduced
\item
|\childdoc| renamed to |\childdocmain|
\item
|\childredirect| renamed to |\childdocforward| and |\childdocforwardprefix|
and functionality expanded
\end{itemize}

%%%%%%%%%%%%%%%%%%%%%%%%%%%%%%%%%%%%%%%%
\paragraph{v1.0:} 2017/04/27

\begin{itemize}
\item
manual and install package
\item
first version published on CTAN
\end{itemize}

%%%%%%%%%%%%%%%%%%%%%%%%%%%%%%%%%%%%%%%%
\paragraph{v0.6:} 2017/04/26

\begin{itemize}
\item
redirection mechanism added
\end{itemize}

%%%%%%%%%%%%%%%%%%%%%%%%%%%%%%%%%%%%%%%%
\paragraph{v0.5:} 2017/04/26

\begin{itemize}
\item
functionality in definition file
\end{itemize}


%%%%%%%%%%%%%%%%%%%%%%%%%%%%%%%%%%%%%%%%%%%%%%%%%%%%%%%%%%%%%%%%%%%%%%%%%%%%%%%%
%%%%%%%%%%%%%%%%%%%%%%%%%%%%%%%%%%%%%%%%%%%%%%%%%%%%%%%%%%%%%%%%%%%%%%%%%%%%%%%%
%%%%%%%%%%%%%%%%%%%%%%%%%%%%%%%%%%%%%%%%%%%%%%%%%%%%%%%%%%%%%%%%%%%%%%%%%%%%%%%%
\appendix

\settowidth\MacroIndent{\rmfamily\scriptsize 000\ }

 \DocInput{childdoc.dtx}

\end{document}
%</driver>
% \fi
%
% %%%%%%%%%%%%%%%%%%%%%%%%%%%%%%%%%%%%%%%%%%%%%%%%%%%%%%%%%%%%%%%%%%%%%%%%%%%%%%
% %%%%%%%%%%%%%%%%%%%%%%%%%%%%%%%%%%%%%%%%%%%%%%%%%%%%%%%%%%%%%%%%%%%%%%%%%%%%%%
% \section{Sample}
%\iffalse
%<*samplemain>
%\fi
%
% The following presents a sample document
% with two chapters, two parts, a title page,
% a compile flag as well as three forwarding files to set the flag.
% It consists of eight |.tex| files:
% \begin{center}
% \begin{tabular}{ll}
% |cdocsamp.tex|&main file\\
% |cdocsch1.tex|&include file for chapter 1\\
% |cdocsch2.tex|&include file for chapter 2\\
% |cdocspt3.tex|&include file for part 3\\
% |cdocspt4.tex|&include file for part 4\\
% |cdocsdrf.tex|&forwarding file for main file in draft mode\\
% |cdocsfi1.tex|&forwarding file for final version of chapter 1\\
% |cdocsfi2.tex|&forwarding file for final version of chapter 2\\
% \end{tabular}
% \end{center}
% Each of the eight files can be compiled directly by the \LaTeX{} compiler.
%
% %%%%%%%%%%%%%%%%%%%%%%%%%%%%%%%%%%%%%%
% \paragraph{Main File.}
%
% The main file is called |cdocsamp.tex|.
%
% Load the \textsf{childdoc} definitions and
% declare the filename for the main document:
%    \begin{macrocode}
\input{childdoc.def}
\childdocmain{}
%    \end{macrocode}

% Optional override for |\version| flag:
%    \begin{macrocode}
%%\ifchilddoc\else\providecommand{\version}{draft}\fi
%    \end{macrocode}

% Define the default values for the |\version| flag
% (|final| for the main file and |draft| for childs):
%    \begin{macrocode}
\ifchilddoc
\providecommand{\version}{draft}
\else
\providecommand{\version}{final}
\fi
%    \end{macrocode}

% Load the standard document class:
%    \begin{macrocode}
\documentclass[12pt]{article}
%    \end{macrocode}

% Start the document body:
%    \begin{macrocode}
\begin{document}
%    \end{macrocode}

% Declare a title page.
% Print title, part of document being processed and version flag:
%    \begin{macrocode}
\addtocounter{page}{-1}
\begin{center}
{\LARGE\bfseries{}childdoc example\par}
\vspace{1cm}
\ifchilddoc
\ifchilddocmanual part\else chapter\fi:
`\childdocname' of `\childdocjob'\par
\else
main document: `\childdocjob'\par
\fi
version: \version\par
\end{center}
\newpage
%    \end{macrocode}

% Manually include selected file,
% otherwise process as usual:
%    \begin{macrocode}
\ifchilddocmanual
\section*{part `\childdocname'}
\input{\childdocname}
\else
%    \end{macrocode}

% Include the two chapters:
%    \begin{macrocode}
\include{cdocsch1}
\include{cdocsch2}
%    \end{macrocode}

% Include the two parts unless only chapters should be displayed:
%    \begin{macrocode}
\ifchilddoc\else
\section{part three}
\input{cdocspt3}
\section{part four}
\input{cdocspt4}
\fi
%    \end{macrocode}

% Process as usual until here:
%    \begin{macrocode}
\fi
%    \end{macrocode}

% End of document body:
%    \begin{macrocode}
\end{document}
%    \end{macrocode}
%\iffalse
%</samplemain>
%\fi
%
% %%%%%%%%%%%%%%%%%%%%%%%%%%%%%%%%%%%%%%
% \paragraph{Chapter Include Files.}
%
% The include files are called |cdocsch1.tex| and |cdocsch2.tex|.
%
%\iffalse
%<*samplechap1|samplechap2>
%\fi

% Optional override for |\version| flag:
%    \begin{macrocode}
%%\providecommand{\version}{final}
%    \end{macrocode}

% Include the main document:
%    \begin{macrocode}
\input{childdoc.def}
\childdocof{cdocsamp}
%    \end{macrocode}

%\iffalse
%</samplechap1|samplechap2>
%\fi
%
%\iffalse
%<*samplechap1>
%\fi
% Some text for chapter 1:
%    \begin{macrocode}
\section{one}
some text in chapter one
%    \end{macrocode}

%\iffalse
%</samplechap1>
%\fi
% Some text for chapter 2:
%\iffalse
%<*samplechap2>
%\fi
%    \begin{macrocode}
\section{two}
more text in chapter two
%    \end{macrocode}

%\iffalse
%</samplechap2>
%\fi
%
% %%%%%%%%%%%%%%%%%%%%%%%%%%%%%%%%%%%%%%
% \paragraph{Part Include Files.}
%
% The include files are called |cdocspt3.tex| and |cdocspt4.tex|.
%
%\iffalse
%<*samplepart3|samplepart4>
%\fi

% Optional override for |\version| flag:
%    \begin{macrocode}
%%\providecommand{\version}{final}
%    \end{macrocode}

% Include the main document:
%    \begin{macrocode}
\input{childdoc.def}
\childdocby{cdocsamp}
%    \end{macrocode}

%\iffalse
%</samplepart3|samplepart4>
%\fi
%
%\iffalse
%<*samplepart3>
%\fi
% Some text for part 3:
%    \begin{macrocode}
some text in part three
%    \end{macrocode}

%\iffalse
%</samplepart3>
%\fi
% Some text for part 4:
%\iffalse
%<*samplepart4>
%\fi
%    \begin{macrocode}
more text in part four
%    \end{macrocode}

%\iffalse
%</samplepart4>
%\fi
%
% %%%%%%%%%%%%%%%%%%%%%%%%%%%%%%%%%%%%%%
% \paragraph{Forwarding for a Complete Draft.}
%
% The following forwarding file |cdocsdrf.tex|
% compiles the main document in draft mode:
%\iffalse
%<*sampledraft>
%\fi
%    \begin{macrocode}
\def\version{draft}
\input{childdoc.def}
\childdocforward{cdocsamp}
%    \end{macrocode}

%\iffalse
%</sampledraft>
%\fi
%
% %%%%%%%%%%%%%%%%%%%%%%%%%%%%%%%%%%%%%%
% \paragraph{Forwarding for Final Version of the Chapters.}
%
% The following forwarding files |cdocsfn1.tex| and |cdocsfn2.tex|
% (with identical content)
% compile the final versions of the child documents
% |cdocsch1.tex| and |cdocsch2.tex|, respectively:
%\iffalse
%<*samplefinal>
%\fi
%    \begin{macrocode}
\def\version{final}
\input{childdoc.def}
\childdocforwardprefix[cdocsamp]{cdocsfn}{cdocsch}
%    \end{macrocode}

%\iffalse
%</samplefinal>
%\fi
%
% %%%%%%%%%%%%%%%%%%%%%%%%%%%%%%%%%%%%%%
% \paragraph{Command Line Processing.}
%
% The following three command lines generate the output files
% |cdocscld|, |cdocscl1| and |cdocscl2|
% which should be identical to
% |cdocsdrf|, |cdocsch1| and |cdocsfn2|, respectively:
% \begin{center}
% \begin{tabular}{l}
% |latex -jobname cdocscld \|\\
% |  "\def\version{draft}\input{childdoc.def}\childdocforward{cdocsamp}"|\\
% |latex -jobname cdocscl1 \|\\
% |  "\input{childdoc.def}\childdocforward[cdocsamp]{cdocsch1}"|\\
% |latex -jobname cdocscl2 \|\\
% |  "\def\version{final}\input{childdoc.def}\childdocforward{cdocsch2}"|
% \end{tabular}
% \end{center}
% Note that the trailing backslash on each first line
% merely continues the input to the second line
% (for convenient cut ant paste).
% Furthermore, the command |latex| can be replaced by any
% of its alternative versions such as |pdflatex|.
%
% %%%%%%%%%%%%%%%%%%%%%%%%%%%%%%%%%%%%%%%%%%%%%%%%%%%%%%%%%%%%%%%%%%%%%%%%%%%%%%
% %%%%%%%%%%%%%%%%%%%%%%%%%%%%%%%%%%%%%%%%%%%%%%%%%%%%%%%%%%%%%%%%%%%%%%%%%%%%%%
% \section{Implementation}
%\iffalse
%<*package>
%\fi
%
% This section describes the definitions file |childdoc.def|.

% The definitions cannot be loaded using |\usepackage| or |\RequirePackage|
% which has a mechanism to prevent loading a style file more than once.
% When loading the definitions by means of |\input|
% multiple instances have to be prevented manually:
%\iffalse
%This code needs to be before the `\ProvidesFile' directive
%which is defined at the beginning of this file.
%Therefore it is also placed there and commented out here.
%</package>
%<*discard>
%\fi
%    \begin{macrocode}
\ifdefined\childdocmain\endinput\fi
%    \end{macrocode}
%\iffalse
%</discard>
%<*package>
%\fi
%
% \macro{\ifchilddoc}
% \macro{\ifchilddocmanual}
% The conditional |\ifchilddoc| tells whether a
% child (true) or main (false) document is being compiled.
% The conditional |\ifchilddocmanual| tells whether
% the |\includeonly| mechanism is used (false) or
% the selection of child files must be performed manually (true).
% The definitions initialise to false:
%    \begin{macrocode}
\newif\ifchilddoc
\newif\ifchilddocmanual
%    \end{macrocode}

% \macro{\childdocname}
% \macro{\childdocjob}
% The macro |\childdocname| stores the name of the main document
% to be compiled. The macro |\childdocjob| stores the name of
% the document on which the \LaTeX{} compiler was originally invoked.
% The content of |\jobname| cannot be compared
% to filenames specified in the source due to different catcodes.
% The following code rescans |\jobname|, stores the result
% in |\childdocname| and saves a copy in |\childdocjob|:
%    \begin{macrocode}
\edef\childdocname{\scantokens\expandafter{\jobname\noexpand}}
\let\childdocjob\childdocname
%    \end{macrocode}

% \macro{\childdocdisable}
% The macro |\childdocdisable| prevents the main file
% from being processed more than once.
% At this stage, the main document command |\childdocmain|
% is assumed to be called once again where it should do nothing.
% Any subsequent call to it should prevent
% a secondary processing of the main document
% It overwrites the forwarding commands
% |\childdocof| and |\childdocforward|
% with empty macros to prevent further inclusions of the main document:
%    \begin{macrocode}
\newcommand{\childdocdisable}
{
  \renewcommand{\childdocmain}[1]{\renewcommand{\childdocmain}[1]{\endinput}}
  \renewcommand{\childdocof}[1]{}
  \renewcommand{\childdocby}[2][]{}
  \renewcommand{\childdocforward}[2][]{}
  \renewcommand{\childdocdisable}{}
}
%    \end{macrocode}

% \macro{\childdocmain}
% The macro |\childdocmain| is to be called at the top of the main file
% with nothing or the main filename (without extension) as argument.
% First, it breaks loops.
% If the argument is not empty and does not match |\childdocname|
% (which is set by the first inclusion of |childdoc.def|),
% |\ifchilddoc| is set to true, |\includeonly| is applied to the child file
% and |\jobname| is set to the main file
% (for proper handling of |.aux| files):
%    \begin{macrocode}
\newcommand{\childdocmain}[1]
{
  \childdocdisable\childdocmain{}
  \if?#1?\else
    \begingroup
      \def\childdoctmp{#1}
      \ifx\childdoctmp\childdocname
        \def\childdoctmp{}
      \else
        \def\childdoctmp
        {
          \childdoctrue
          \includeonly{\childdocname}
          \def\childdocjob{#1}
          \def\jobname{#1}
        }
      \fi
      \expandafter
    \endgroup
    \childdoctmp
  \fi
}
%    \end{macrocode}

% \macro{\childdocof}
% The command |\childdocof| redirects
% compilation to the main file |#1|.
%    \begin{macrocode}
\newcommand{\childdocof}[1]
{
  \childdocdisable
  \childdoctrue
  \includeonly{\childdocname}
  \def\jobname{#1}
  \def\childdocjob{#1}
  \input{#1}
}
%    \end{macrocode}

% \macro{\childdocby}
% The command |\childdocby| ....
%    \begin{macrocode}
\newcommand{\childdocby}[2][]
{
  \childdocdisable
  \childdoctrue
  \childdocmanualtrue
  \if?#1?\else
    \def\jobname{#2}
  \fi
  \def\childdocjob{#2}
  \input{#2}
  \endinput
}
%    \end{macrocode}

% \macro{\childdocforward}
% The command |\childdocforward| redirects
% compilation to the main file or
% (if the optional argument is given) a child file.
% Parameters are set as if the main file
% or a child file starting with |\childdocof| was compiled.
% Then compilation is handed over to the main file:
%    \begin{macrocode}
\newcommand{\childdocforward}[2][]
{
  \begingroup
    \if?#1?
      \def\childdoctmp
      {
        \def\childdocname{#2}
        \def\childdocjob{#2}
        \def\jobname{#2}
        \input{#2}
        \endinput
      }
    \else
      \def\childdoctmp
      {
        \childdocdisable
        \def\childdocname{#2}
        \childdoctrue
        \includeonly{#2}
        \def\childdocjob{#1}
        \def\jobname{#1}
        \input{#1}
        \endinput
      }
    \fi
    \expandafter
  \endgroup
  \childdoctmp
}
%    \end{macrocode}

% \macro{\childdocforwardprefix}
% The command |\childdocforwardprefix| redirects
% compilation to the main or a child file by means of a pattern.
% The prefix |#1| in the current filename is replaced by |#2|
% and the suffix of the current filename is kept
% (it is assumed that the filename does not contain the substring `|~~~|'
% which is used as a delimiter).
% Compilation is handed over to the new file by |\childdocforward|:
%    \begin{macrocode}
\newcommand{\childdocforwardprefix}[3][]
{
  \begingroup
    \def\childdocextract #2##1~~~{\def\childdoctmp{\childdocforward[#1]{#3##1}}}
    \expandafter\childdocextract\childdocname~~~
    \expandafter
  \endgroup
  \childdoctmp
}
%    \end{macrocode}

% \macro{\childdoc}
% The deprecated macro |\childdoc| is a legacy version of |\childdocmain|:
%    \begin{macrocode}
\newcommand{\childdoc}{\childdocmain}
%    \end{macrocode}

% \macro{\childdocredirect}
% The deprecated macro |\childdocredirect| is a legacy version
% of |\childdocforward| and |\childdocforwardprefix|:
%    \begin{macrocode}
\newcommand{\childdocredirect}[2][]
{
  \begingroup
    \if?#1?
      \def\childdoctmp{\childdocforward{#2}}
    \else
      \def\childdoctmp{\childdocforwardprefix{#1}{#2}}
    \fi
    \expandafter
  \endgroup
  \childdoctmp
}
%    \end{macrocode}

%\iffalse
%</package>
%\fi
%
\endinput

\childdocby{cdocsamp}
%    \end{macrocode}

%\iffalse
%</samplepart3|samplepart4>
%\fi
%
%\iffalse
%<*samplepart3>
%\fi
% Some text for part 3:
%    \begin{macrocode}
some text in part three
%    \end{macrocode}

%\iffalse
%</samplepart3>
%\fi
% Some text for part 4:
%\iffalse
%<*samplepart4>
%\fi
%    \begin{macrocode}
more text in part four
%    \end{macrocode}

%\iffalse
%</samplepart4>
%\fi
%
% %%%%%%%%%%%%%%%%%%%%%%%%%%%%%%%%%%%%%%
% \paragraph{Forwarding for a Complete Draft.}
%
% The following forwarding file |cdocsdrf.tex|
% compiles the main document in draft mode:
%\iffalse
%<*sampledraft>
%\fi
%    \begin{macrocode}
\def\version{draft}
% \iffalse
%
% childdoc.dtx Copyright (C) 2017-2018 Niklas Beisert
%
% This work may be distributed and/or modified under the
% conditions of the LaTeX Project Public License, either version 1.3
% of this license or (at your option) any later version.
% The latest version of this license is in
%   http://www.latex-project.org/lppl.txt
% and version 1.3 or later is part of all distributions of LaTeX
% version 2005/12/01 or later.
%
% This work has the LPPL maintenance status `maintained'.
%
% The Current Maintainer of this work is Niklas Beisert.
%
% This work consists of the files childdoc.dtx and childdoc.ins
% and the derived files childdoc.def and cdocsamp.tex with
% cdocsch1.tex, cdocsch2.tex, cdocsdrf.tex, cdocsfn1.tex, cdocsfn2.tex.
%
%<package>\ifdefined\childdocmain\endinput\fi
%<package>\ProvidesFile{childdoc.def}[2018/12/30 v2.0 child document driver]
%<samplemain>\ProvidesFile{cdocsamp.tex}[2018/12/30 v2.0 sample for childdoc]
%<*driver>
%\ProvidesFile{childdoc.drv}[2018/12/30 v2.0 childdoc reference manual file]
\PassOptionsToClass{10pt,a4paper}{article}
\documentclass{ltxdoc}

\usepackage[margin=35mm]{geometry}
\usepackage{hyperref}
\usepackage{hyperxmp}
\usepackage[usenames]{color}

\hypersetup{colorlinks=true}
\hypersetup{pdfstartview=FitH}
\hypersetup{pdfpagemode=UseNone}
\hypersetup{pdfsource={}}
\hypersetup{pdflang={en-UK}}
\hypersetup{pdfcopyright={Copyright 2017-2018 Niklas Beisert.
  This work may be distributed and/or modified under the
  conditions of the LaTeX Project Public License, either version 1.3
  of this license or (at your option) any later version.}}
\hypersetup{pdflicenseurl={http://www.latex-project.org/lppl.txt}}
\hypersetup{pdfcontactaddress={ETH Zurich, ITP, HIT K,
  Wolfgang-Pauli-Strasse 27}}
\hypersetup{pdfcontactpostcode={8093}}
\hypersetup{pdfcontactcity={Zurich}}
\hypersetup{pdfcontactcountry={Switzerland}}
\hypersetup{pdfcontactemail={nbeisert@itp.phys.ethz.ch}}
\hypersetup{pdfcontacturl={http://people.phys.ethz.ch/\xmptilde nbeisert/}}

\newcommand{\secref}[1]{\hyperref[#1]{section \ref*{#1}}}

\parskip1ex
\parindent0pt
\let\olditemize\itemize
\def\itemize{\olditemize\parskip0pt}

\begin{document}

\title{The \textsf{childdoc} Package}
\hypersetup{pdftitle={The childdoc Package}}
\author{Niklas Beisert\\[2ex]
  Institut f\"ur Theoretische Physik\\
  Eidgen\"ossische Technische Hochschule Z\"urich\\
  Wolfgang-Pauli-Strasse 27, 8093 Z\"urich, Switzerland\\[1ex]
  \href{mailto:nbeisert@itp.phys.ethz.ch}
  {\texttt{nbeisert@itp.phys.ethz.ch}}}
\hypersetup{pdfauthor={Niklas Beisert}}
\hypersetup{pdfsubject={Manual for the LaTeX2e Package childdoc}}
\date{30 December 2018, \textsf{v2.0}}
\maketitle

\begin{abstract}\noindent
\textsf{childdoc} is a \LaTeXe{} package
that enables the direct compilation
of document sections included by |\include|
to individual files.
\end{abstract}

\begingroup
\parskip0ex
\tableofcontents
\endgroup

%%%%%%%%%%%%%%%%%%%%%%%%%%%%%%%%%%%%%%%%%%%%%%%%%%%%%%%%%%%%%%%%%%%%%%%%%%%%%%%%
%%%%%%%%%%%%%%%%%%%%%%%%%%%%%%%%%%%%%%%%%%%%%%%%%%%%%%%%%%%%%%%%%%%%%%%%%%%%%%%%
\section{Introduction}

\LaTeX{} provides a mechanism to structure a large document (such as a book)
into a main file and several child files (containing the chapters)
using the |\include| command.
This mechanism is beneficial for documents
which span hundreds of pages in order to
make the source file(s) more manageable.
Moreover, compilation can be restricted to
selected child files by means of the |\includeonly| command.
The latter feature can be used to reduce the compilation time while editing
(this was significantly more useful in the earlier days of \LaTeX{})
or to generate a smaller document which is easier to navigate.
Another application of |\includeonly| is to generate
documents consisting of selected parts of the complete document.

However, there are a few drawbacks of the plain |\include| mechanism:
\begin{itemize}
\item
The child files cannot be compiled on their own,
they can only be compiled via the main file.
A naive editing environment
(such as a text editor with an option
to have the current file processed by \LaTeX)
may require one to switch to the main file before compiling;
attempting to compile the child file produces errors.
\item
The main file must be modified (each time)
to adjust the |\includeonly| command
to the present needs. This easily leaves the main file in a messy state.
\item
The generated document will always carry the filename
of the main document. This is inconvenient if
several child files are to be compiled and
to be kept for distribution.
\end{itemize}

The present package provides a simple interface
to make child files individually compilable by \LaTeX{}.
Compiling a child file then has the same effect as compiling
the main file with an |\includeonly| command
to select the appropriate child.
Moreover the generated document will carry the name of the child
rather than the main file.
This resolves all three above issues.

This feature is meant to make the editing of books,
thesis documents and lecture notes somewhat more convenient.
However, the package can also be used efficiently for
composing a series of documents (such as exercise sheets)
which are typically distributed individually.
It then assists the author in generating the individual documents
(potentially in different versions)
as well as a document containing the collected series.
Another application is in developing style files
or other kinds of included material
where compilation of the style file could redirect
to a sample or test file.

%%%%%%%%%%%%%%%%%%%%%%%%%%%%%%%%%%%%%%%%%%%%%%%%%%%%%%%%%%%%%%%%%%%%%%%%%%%%%%%%
%%%%%%%%%%%%%%%%%%%%%%%%%%%%%%%%%%%%%%%%%%%%%%%%%%%%%%%%%%%%%%%%%%%%%%%%%%%%%%%%
\section{Usage}

First of all, the package \textsf{childdoc} is \emph{not} a standard
\LaTeXe{} |.sty| style file! Therefore it needs to be invoked in
a non-standard way.

%%%%%%%%%%%%%%%%%%%%%%%%%%%%%%%%%%%%%%%%%%%%%%%%%%%%%%%%%%%%%%%%%%%%%%%%%%%%%%%%
\subsection{Included Files}
\label{sec:include}

%%%%%%%%%%%%%%%%%%%%%%%%%%%%%%%%%%%%%%%%
\DescribeMacro{\childdocmain}
To use the package, add the commands
\begin{center}
\begin{tabular}{l}
|\input{childdoc.def}|\\
|\childdocmain{}|\\
\end{tabular}
\end{center}
at the very top of the main \LaTeX{} file,
in particular \emph{before} the |\documentclass| statement!
The argument of |\childdocmain| should be left empty
(but it must be present).

%%%%%%%%%%%%%%%%%%%%%%%%%%%%%%%%%%%%%%%%
\DescribeMacro{\childdocof}
Furthermore, add the commands
\begin{center}
\begin{tabular}{l}
|\input{childdoc.def}|\\
|\childdocof{|\textit{main}|}|\\
\end{tabular}
\end{center}
at the top of every child file \textit{child}
which is included by |\include{|\textit{child}|}|
from within the main file
(or at least for those files to be compiled individually).
The argument \textit{main} must be the filename of the main file.

There are a couple of
considerations in setting up the main and child documents:

%%%%%%%%%%%%%%%%%%%%%%%%%%%%%%%%%%%%%%%%
\paragraph{Restrictions.}

Please note the following restrictions:
\begin{itemize}
\item
|\childdocmain| must be called with one argument \textit{main}
to ensure compatibility with earlier version of the package.
It must either be empty (|\childdocmain{}|)
or precisely match the filename of the main file in which it is specified.
See \secref{sec:detection} for further information.
\item
The filename \textit{main} must be specified without the |.tex| extension.
\item
The filename \textit{main} is case sensitive
(even in case-insensitive file systems)
due to internal string comparison.
\item
The argument \textit{main} should be fully expanded, it cannot be a macro.
\item
Subdirectories and special characters should be avoided in filenames.
\item
The command |\childdocmain{|\textit{main}|}| must be followed by a whitespace.
It should not be followed immediately by another command
or by a comment mark `|%|'.
This is because the \TeX{} parser reads the token immediately following
the argument of |\childdocmain| and puts it
at the beginning of every child section;
however, a white\-space is ignored.
\end{itemize}

%%%%%%%%%%%%%%%%%%%%%%%%%%%%%%%%%%%%%%%%
\paragraph{Content of Main File.}

It is advisable to place all content in the child files included by |\include|.
Any output contained in the main file will appear in all child documents
unless suppressed manually;
it cannot be suppressed automatically by the |\includeonly| directive
and thus should normally be avoided.
A method to include some content in the main file
by means of conditional processing is described in \secref{sec:conditional}.

%%%%%%%%%%%%%%%%%%%%%%%%%%%%%%%%%%%%%%%%
\paragraph{Page Numbering.}

When only a part of the document is compiled,
the appropriate numbering of pages
(as well as other status parameters)
is determined from the |.aux| files.
The latter contain information from previous passes.
However this information needs to propagate through
all intermediate child documents.
Therefore the page numbering in child documents may well
be inconsistent until the complete document is compiled at least once.

A useful (if unconventional) way to always ensure a consistent
page numbering is to restart the numbering in each child document
and denote the pages by `\textit{child}|.|\textit{page}'
where \textit{child} represents the chapter/section number of the child file.
This can be achieved by the command
|\numberwithin{page}{|\textit{child}|}|
of the \textsf{amsmath} package
where \textit{child} can be |chapter| or |section|
depending on the chosen structuring.
Alternatively, one can modify the macro |\thepage| appropriately
and reset the counter |page| at the start of each child file.

%%%%%%%%%%%%%%%%%%%%%%%%%%%%%%%%%%%%%%%%%%%%%%%%%%%%%%%%%%%%%%%%%%%%%%%%%%%%%%%%
\subsection{Conditional Processing}
\label{sec:conditional}

The package provides a mechanism to compile different versions
of a document. To customise the versions further some conditional processing
can come in handy to distinguish which version is being compiled.
The package provides two macros to describe the compilation context:

%%%%%%%%%%%%%%%%%%%%%%%%%%%%%%%%%%%%%%%%
\DescribeMacro{\ifchilddoc}
The conditional |\ifchilddoc| distinguishes between the compilation of
child documents and the main document:
%
\begin{center}
|\ifchilddoc |\textit{child-code}| |[|\||else |\textit{main-code}]| \||fi|
\end{center}

%%%%%%%%%%%%%%%%%%%%%%%%%%%%%%%%%%%%%%%%
\DescribeMacro{\childdocname}
\DescribeMacro{\childdocjob}
The macro |\childdocname| contains the filename (without extension)
of the main or child file being processed.
Note that |\childdocjob| will always contain the name of the main file.

%%%%%%%%%%%%%%%%%%%%%%%%%%%%%%%%%%%%%%%%
\paragraph{Title Page.}

Conditional processing can be used to include a title or banner page
in the main document when proper precautions are taken.
Importantly, the code in the main file should ensure that the page counter
(as well as other status parameters which are stored in the |.aux| files)
takes the same value after the conditional processing.
Otherwise the page numbers may take divergent values
depending on which part is compiled.

For example, a title page could be declared by:
%
\begin{center}
\begin{tabular}{l}
|\ifchilddoc\||else|\\
|\addtocounter{page}{-1}|\\
\textit{code for title page}\\
|\newpage|\\
|\||fi|
\end{tabular}
\end{center}
%
A banner page for the child documents can be generated by:
%
\begin{center}
\begin{tabular}{l}
|\ifchilddoc|\\
|\addtocounter{page}{-1}|\\
\textit{code for banner page}\\
|\newpage|\\
|\||fi|
\end{tabular}
\end{center}
%
Here one could write a message such as:
\begin{center}
|This is the part \childdocname{} of \childdocjob{}.|
\end{center}

%%%%%%%%%%%%%%%%%%%%%%%%%%%%%%%%%%%%%%%%%%%%%%%%%%%%%%%%%%%%%%%%%%%%%%%%%%%%%%%%
\subsection{Flags}
\label{sec:flags}

The package makes it easy to generate different versions
of the main or child documents.
To this end compilation flags can be defined
and assigned different default values.
They will be particularly useful in conjunction
with the forwarding mechanism described in \secref{sec:forward}.

For example, it may be useful to have a flag |\version|
which can be set to |draft| or |final|.
The document source will contain some conditional code
depending on the value of |\version|.
Suppose further, the flag should default to |final| for the main file
and to |draft| for child files
which is a natural assignment for editing the document.
This is achieved by placing the following code
in the preamble of the main document
(below the |\childdocmain| directive):
%
\begin{center}
\begin{tabular}{l}
|\ifchilddoc|\\
|\providecommand{\version}{draft}|\\
|\||else|\\
|\providecommand{\version}{final}|\\
|\||fi|
\end{tabular}
\end{center}
%
The definition by |\providecommand| makes sure
that previous definitions are not overwritten.
Further statements |\providecommand{\version}{...}|
can thus be added before the above code to override it.

For the main file, one might add a line
(between |\childdocmain| and the above block)
%
\begin{center}
|%\ifchilddoc\||else\providecommand{\version}{draft}\||fi|
\end{center}
%
which can be uncommented to produce a draft version.
Likewise one can add a line to the very top of a child file
(above the |\childdocof{|\textit{main}|}| directive)
%
\begin{center}
|%\providecommand{\version}{final}|
\end{center}
%
which can be uncommented to produce the final version of this child document.

%%%%%%%%%%%%%%%%%%%%%%%%%%%%%%%%%%%%%%%%%%%%%%%%%%%%%%%%%%%%%%%%%%%%%%%%%%%%%%%%
\subsection{Forwarding}
\label{sec:forward}

Different versions of the main or child documents
using compilation flags as described in \secref{sec:flags}
can be (permanently) stored in different files
for convenient compilation, viewing and distribution.
To this end, the package defines a command
to pass on compilation to a different file:

%%%%%%%%%%%%%%%%%%%%%%%%%%%%%%%%%%%%%%%%
\DescribeMacro{\childdocforward}
The command |\childdocforward| redirects processing to
another source file:
%
\begin{center}
\begin{tabular}{l}
|\input{childdoc.def}|\\
|\childdocforward[|\textit{main}|]{|\textit{dest}|}|\\
\end{tabular}
\end{center}
%
The argument \textit{dest} is the destination file
(without extension).
It should be the main file or one of the child files.
Note that further \textsf{childdoc} directives
such as |\childdocof| and |\childdocforward|
in the indicated file will be processed in this form.
The optional argument \textit{main}
passes on directly to the main file \textit{main}
while pretending to compile the child \textit{dest}.
This form behaves as if \textit{dest}
issues |\childdocof{|\textit{main}|}| right away,
and no further \textsf{childdoc} directives will be processed.

%%%%%%%%%%%%%%%%%%%%%%%%%%%%%%%%%%%%%%%%
\DescribeMacro{\...prefix}
In the alternative form |\childdocforwardprefix|,
%
\begin{center}
\begin{tabular}{l}
|\input{childdoc.def}|\\
|\childdocforwardprefix[|\textit{main}|]{|\textit{prefix}|}{|\textit{dest}|}|
\end{tabular}
\end{center}
%
the destination file is determined by a pattern
depending on the current file:
To make this work, the current file must be called
`{\textit{prefix}\hspace{0.2em}\textit{suffix}}'
with \textit{prefix} matching precisely the argument.
Processing is then passed on to the file
`{\textit{dest}\hspace{0.2em}\textit{suffix}}'.
Surely, the same effect is achieved by
directly specifying the
argument `{\textit{dest}\hspace{0.2em}\textit{suffix}}'
in the first form.
However, that requires to set up a different file
for each child. With the alternative form of the command
all these files can have exactly the same content
which simplifies setting them up and maintaining them.

For example, the following file |draft.tex|
with a compilation flag |\version| as described in \secref{sec:flags}
compiles the main document as a draft:
%
\begin{center}
\begin{tabular}{l}
|\def\version{draft}|\\
|\input{childdoc.def}|\\
|\childdocforward{|\textit{main}|}|
\end{tabular}
\end{center}
%
Likewise, the following files |final|\textit{nn}|.tex|
compile the final version of the child document
|child|\textit{nn}|.tex|:
%
\begin{center}
\begin{tabular}{l}
|\def\version{final}|\\
|\input{childdoc.def}|\\
|\childdocforwardprefix{final}{child}|
\end{tabular}
\end{center}
%

Note that when several versions of a main file and/or of each child file
are to be generated, it may be convenient to set up a |Makefile| or
shell script to automatise the process.

%%%%%%%%%%%%%%%%%%%%%%%%%%%%%%%%%%%%%%%%%%%%%%%%%%%%%%%%%%%%%%%%%%%%%%%%%%%%%%%%
\subsection{Command Line Processing}
\label{sec:commandline}

The effect of redirection files can also be achieved by invoking
the \LaTeX{} compiler with a more elaborate command line.
Most conveniently this should be done as part
of a shell script or a |Makefile|.

When using \textsf{childdoc} in the main file, the following
command lines effectively perform a redirection
(note that depending on the shell being used,
backslashes may have to be doubled: `|\|' $\to$ `|\\|'):
%
\begin{center}
|... -jobname "|\textit{target}|" |\\|"|[\textit{flags}]%
|\input{childdoc.def}\childdocforward[|\textit{main}|]{|\textit{dest}|}"|
\end{center}
%
Here \textit{target} is the name of the output file,
\textit{main} is the name of the main file
and \textit{dest} is the name of the main or child file to be processed
(all filenames without extensions).
The optional argument \textit{main} can be omitted
if \textit{main} matches \textit{dest}.
Optionally, compilation \textit{flags} can be defined via |\def| commands.
This command line makes the \TeX{} engine believe
it is compiling the file \textit{target}
whose content is specified as the latter parameter.
The provided code then forwards the processing to
\textit{main} or \textit{dest} as described in \secref{sec:forward}.

%%%%%%%%%%%%%%%%%%%%%%%%%%%%%%%%%%%%%%%%%%%%%%%%%%%%%%%%%%%%%%%%%%%%%%%%%%%%%%%%
\subsection{Include by Input}
\label{sec:input}

Including child documents by |\include| has some restrictions by design.
Most notably, the content of a child document always occupies
its own set of pages; pages cannot be shared between child documents.
Usually, this behaviour makes perfect sense
because each child document contain an essential part of the document.
However, in some situations it may be desirable to compose
a document from a collection of parts
without having mandatory page breaks between then.
For this case, the package
provides a mechanism to include parts
by |\input| which can also be processed individually.
However, by construction this mechanism
requires manual handling of the content to be output.

%%%%%%%%%%%%%%%%%%%%%%%%%%%%%%%%%%%%%%%%
\DescribeMacro{\ifchilddocmanual}
The main file should be prepared as usual, see \secref{sec:include}.
However, the document body must make a distinction
between processing of an individual part and of the main document, e.g.:
%
\begin{center}
\begin{tabular}{l}
|\ifchilddocmanual|\\
|\input{\childdocname}|\\
|\||else|\\
\textit{document body with }|\input{|\textit{part}|}|\\
|\||fi|
\end{tabular}
\end{center}
%
The conditional |\ifchilddocmanual| is true whenever
a part to be included by |\input| is being compiled,
and the name of the part is stored in |\childdocname|.

%%%%%%%%%%%%%%%%%%%%%%%%%%%%%%%%%%%%%%%%
\DescribeMacro{\childdocby}
Each part to be included by |\input| should start with:
%
\begin{center}
\begin{tabular}{l}
|\input{childdoc.def}|\\
|\childdocby{|\textit{main}|}|\\
\end{tabular}
\end{center}
%
The directive |\childdocby| is similar to |\childdocof|
described in \secref{sec:include},
but the subsequent selection of content must be done manually.
To that end, both |\ifchilddoc| and |\ifchilddocmanual|
will be true upon processing of a part,
and the name of the part is stored in |\childdocname|.
Note that |\jobname| will be set to the filename of the current part
so that each part receives an individual |.aux| file
that does not interfere with the |.aux| file(s) of the main document.
This behaviour can be altered by the alternative form
|\childdocby[*]{|\textit{main}|}| (with a non-empty optional argument)
which uses the |.aux| file of the main document
by setting |\jobname| to \textit{main}.

%%%%%%%%%%%%%%%%%%%%%%%%%%%%%%%%%%%%%%%%%%%%%%%%%%%%%%%%%%%%%%%%%%%%%%%%%%%%%%%%
\subsection{Driver Development}
\label{sec:driver}

The \textsf{childdoc} mechanism can also be use for the development
of definition files such as \LaTeX{} styles or classes.
This case differs from the above setup with multiple parts
included by |\include| in that no |\includeonly| should be invoked.
This can be achieved by starting the include file
(before |\ProvidesPackage|) with:
%
\begin{center}
\begin{tabular}{l}
|\input{childdoc.def}|\\
|\childdocforward{|\textit{main}|}|\\
\end{tabular}
\end{center}
%
or alternatively with:
%
\begin{center}
\begin{tabular}{l}
|\input{childdoc.def}|\\
|\childdocby{|\textit{main}|}|\\
\end{tabular}
\end{center}
%
Both forms have slightly different effects as described above.
The main file is prepared as usual, see \secref{sec:include}.

%%%%%%%%%%%%%%%%%%%%%%%%%%%%%%%%%%%%%%%%%%%%%%%%%%%%%%%%%%%%%%%%%%%%%%%%%%%%%%%%
\subsection{Legacy Detection}
\label{sec:detection}

The directive |\childdocmain| in the main file can detect
whether the complete document or merely a child is to be compiled
even without using the directive |\childdocof|.
This method is deprecated because it is less robust
and there is no compelling reason to use it;
it is merely provided for backward compatibility
and it may be removed in future versions.

If the detection mechanism is to be used,
it is mandatory to correctly specify
the filename of the main file as the argument of |\childdocmain|:
%
\begin{center}
\begin{tabular}{l}
|\input{childdoc.def}|\\
|\childdocmain{|\textit{main}|}|\\
\end{tabular}
\end{center}
%
If |\jobname| does not match the argument \textit{main} of |\childdocmain|,
it is assumed that |\jobname| points to the child file to be compiled.
When using |\childdocmain| with the main file specified as argument,
it suffices to start a child file
with just |\input{|\textit{main}|}|
without loading of the package and using |\childdocof|.
If instead all processing is done
with the appropriate \textsf{childdoc} directives,
the argument of \textit{main} of |\childdocmain| can be empty.

An alternative version of the command line processing described
in \secref{sec:commandline} using the detection mechanism reads:
%
\begin{center}
|... -jobname "|\textit{target}|" "|[\textit{flags}]%
[|\def\jobname{|\textit{dest}|}|]|\input{|\textit{main}|}"|
\end{center}

%%%%%%%%%%%%%%%%%%%%%%%%%%%%%%%%%%%%%%%%%%%%%%%%%%%%%%%%%%%%%%%%%%%%%%%%%%%%%%%%
\subsection{Manual Code}
\label{sec:manual}

In case one cannot be certain whether the definitions file |childdoc.def|
is installed on the target \TeX{} distribution
and one prefers not to ship it,
it is conceivable to paste a few relevant commands into the sources.

To that end, drop all statements |\input{childdoc.def}|
and perform the replacements as outlined below.
Instead of |\childdocmain{|\textit{main}|}| add the following code
to the top of the main file:
%
\begin{center}
\begin{tabular}{l}
|\||ifdefined\childdocname\endinput\||fi\newif\ifchilddoc|\\
|\edef\childdocname{\scantokens\expandafter{\jobname\noexpand}}|\\
|\def\childdocmain{|\textit{main}|}\||ifx\childdocmain\childdocname\||else|\\
|\childdoctrue\includeonly{\childdocname}\let\jobname\childdocmain\||fi|\\
\end{tabular}
\end{center}
%
Instead of |\childdocof{|\textit{main}|}| just include the main file
at the top of each child file:
%
\begin{center}
|\input{|\textit{main}|}|
\end{center}
%
A simple redirection |\childdocforward{|\textit{dest}|}| is achieved by:
%
\begin{center}
|\def\jobname{|\textit{dest}|}\input{\jobname}|
\end{center}
%
The redirection with prefix
|\childdocforwardprefix[|\textit{prefix}|]{|\textit{dest}|}|
is accomplished by:
%
\begin{center}
\begin{tabular}{l}
|{\edef\jobname{\scantokens\expandafter{\jobname\noexpand}}|\\
|\def\redirectjob |\textit{prefix}|#1~~~{\gdef\jobname{|\textit{dest}|#1}}|\\
|\expandafter\redirectjob\jobname~~~}\input{\jobname}|
\end{tabular}
\end{center}

In an alternative approach,
child documents can be compiled by a specific command line
without additional code or specific definitions:
%
\begin{center}
|... -jobname "|\textit{target}|" "|[\textit{flags}]%
|\includeonly{|\textit{dest}|}\input{|\textit{main}|}"|
\end{center}
%

%%%%%%%%%%%%%%%%%%%%%%%%%%%%%%%%%%%%%%%%%%%%%%%%%%%%%%%%%%%%%%%%%%%%%%%%%%%%%%%%
%%%%%%%%%%%%%%%%%%%%%%%%%%%%%%%%%%%%%%%%%%%%%%%%%%%%%%%%%%%%%%%%%%%%%%%%%%%%%%%%
\section{Information}

%%%%%%%%%%%%%%%%%%%%%%%%%%%%%%%%%%%%%%%%%%%%%%%%%%%%%%%%%%%%%%%%%%%%%%%%%%%%%%%%
\subsection{Copyright}

Copyright \copyright{} 2017--2018 Niklas Beisert

This work may be distributed and/or modified under the
conditions of the \LaTeX{} Project Public License, either version 1.3
of this license or (at your option) any later version.
The latest version of this license is in
  \url{http://www.latex-project.org/lppl.txt}
and version 1.3 or later is part of all distributions of \LaTeX{}
version 2005/12/01 or later.

This work has the LPPL maintenance status `maintained'.

The Current Maintainer of this work is Niklas Beisert.

This work consists of the files |README.txt|, |childdoc.ins| and |childdoc.dtx|
as well as the derived files |childdoc.def|, |cdocsamp.tex|
with |cdocsch1.tex|, |cdocsch2.tex|, |cdocspt3.tex|, |cdocspt4.tex|,
|cdocsdrf.tex|, |cdocsfn1.tex|, |cdocsfn2.tex|
as well as |childdoc.pdf|.

%%%%%%%%%%%%%%%%%%%%%%%%%%%%%%%%%%%%%%%%%%%%%%%%%%%%%%%%%%%%%%%%%%%%%%%%%%%%%%%%
\subsection{Files and Installation}

The package consists of the files:
%
\begin{center}
\begin{tabular}{ll}
    |README.txt|   & readme file \\
    |childdoc.ins| & installation file \\
    |childdoc.dtx| & source file \\
    |childdoc.def| & definition file \\
    |cdocsamp.tex| & sample main file \\
    |cdocsch1.tex| & sample include file \\
    |cdocsch2.tex| & sample include file \\
    |cdocspt3.tex| & sample part file \\
    |cdocspt4.tex| & sample part file \\
    |cdocsdrf.tex| & sample redirection file \\
    |cdocsfn1.tex| & sample redirection file \\
    |cdocsfn2.tex| & sample redirection file \\
    |childdoc.pdf| & manual
\end{tabular}
\end{center}
%
The distribution consists of the files
|README.txt|, |childdoc.ins| and |childdoc.dtx|.
%
\begin{itemize}
\item
Run (pdf)\LaTeX{} on |childdoc.dtx|
to compile the manual |childdoc.pdf| (this file).
\item
Run \LaTeX{} on |childdoc.ins| to create the definitions file |childdoc.def|
and the sample |cdocsamp.tex| with include files
|cdocsch1.tex|, |cdocsch2.tex|, |cdocspt3.tex|, |cdocspt4.tex|,
|cdocsdrf.tex|, |cdocsfn1.tex|, |cdocsfn2.tex|.
Then copy the file |childdoc.def| to an appropriate directory of your \LaTeX{}
distribution, e.g.\ \textit{texmf-root}|/tex/latex/childdoc|.
\end{itemize}

%%%%%%%%%%%%%%%%%%%%%%%%%%%%%%%%%%%%%%%%%%%%%%%%%%%%%%%%%%%%%%%%%%%%%%%%%%%%%%%%
\subsection{Related CTAN Packages}

There are several other packages which offer a similar functionality:
%
\begin{itemize}
\item
The packages
\href{http://ctan.org/pkg/docmute}{\textsf{docmute}},
\href{http://ctan.org/pkg/includex}{\textsf{includex}} and
\href{http://ctan.org/pkg/standalone}{\textsf{standalone}}
provide commands to include only the document body of
a child file thus allowing both files to be compiled individually.
\item
The packages \href{http://ctan.org/pkg/subdocs}{\textsf{subdocs}}
and \href{http://ctan.org/pkg/subfiles}{\textsf{subfiles}}
provide structures in which the main and child documents can be
encapsulated and allowing them to be compiled individually.
The inclusion mechanism is different from the conventional |\include|.
\item
The package \href{http://ctan.org/pkg/combine}{\textsf{combine}}
is an elaborate solution to combine several documents into one.
\end{itemize}
%
See also the CTAN topic \href{http://ctan.org/topic/subdocs}{\textsf{subdocs}}
for further related packages.
The present package differs from the above solutions in that
a document structure constructed with the conventional |\include| mechanism
just needs two extra commands at the top of every file
such that all constituent files can be compiled individually.

%%%%%%%%%%%%%%%%%%%%%%%%%%%%%%%%%%%%%%%%%%%%%%%%%%%%%%%%%%%%%%%%%%%%%%%%%%%%%%%%
%\subsection{Feature Suggestions}
%
%The following is a list of features which may be useful for future
%versions of this package:
%%
%\begin{itemize}
%\item
%\ldots
%\end{itemize}

%%%%%%%%%%%%%%%%%%%%%%%%%%%%%%%%%%%%%%%%%%%%%%%%%%%%%%%%%%%%%%%%%%%%%%%%%%%%%%%%
\subsection{Revision History}

%%%%%%%%%%%%%%%%%%%%%%%%%%%%%%%%%%%%%%%%
\paragraph{v2.0:} 2018/12/30

\begin{itemize}
\item
immediate forward processing
\item
added |\childdocby| mechanism
\item
manual restructured
\end{itemize}

%%%%%%%%%%%%%%%%%%%%%%%%%%%%%%%%%%%%%%%%
\paragraph{v1.6:} 2018/01/17

\begin{itemize}
\item
application for development of include files
\item
corrections to manual
\end{itemize}

%%%%%%%%%%%%%%%%%%%%%%%%%%%%%%%%%%%%%%%%
\paragraph{v1.5:} 2017/05/21

\begin{itemize}
\item
more complete structuring introduced
\item
|\childdocof| introduced
\item
|\childdoc| renamed to |\childdocmain|
\item
|\childredirect| renamed to |\childdocforward| and |\childdocforwardprefix|
and functionality expanded
\end{itemize}

%%%%%%%%%%%%%%%%%%%%%%%%%%%%%%%%%%%%%%%%
\paragraph{v1.0:} 2017/04/27

\begin{itemize}
\item
manual and install package
\item
first version published on CTAN
\end{itemize}

%%%%%%%%%%%%%%%%%%%%%%%%%%%%%%%%%%%%%%%%
\paragraph{v0.6:} 2017/04/26

\begin{itemize}
\item
redirection mechanism added
\end{itemize}

%%%%%%%%%%%%%%%%%%%%%%%%%%%%%%%%%%%%%%%%
\paragraph{v0.5:} 2017/04/26

\begin{itemize}
\item
functionality in definition file
\end{itemize}


%%%%%%%%%%%%%%%%%%%%%%%%%%%%%%%%%%%%%%%%%%%%%%%%%%%%%%%%%%%%%%%%%%%%%%%%%%%%%%%%
%%%%%%%%%%%%%%%%%%%%%%%%%%%%%%%%%%%%%%%%%%%%%%%%%%%%%%%%%%%%%%%%%%%%%%%%%%%%%%%%
%%%%%%%%%%%%%%%%%%%%%%%%%%%%%%%%%%%%%%%%%%%%%%%%%%%%%%%%%%%%%%%%%%%%%%%%%%%%%%%%
\appendix

\settowidth\MacroIndent{\rmfamily\scriptsize 000\ }

 \DocInput{childdoc.dtx}

\end{document}
%</driver>
% \fi
%
% %%%%%%%%%%%%%%%%%%%%%%%%%%%%%%%%%%%%%%%%%%%%%%%%%%%%%%%%%%%%%%%%%%%%%%%%%%%%%%
% %%%%%%%%%%%%%%%%%%%%%%%%%%%%%%%%%%%%%%%%%%%%%%%%%%%%%%%%%%%%%%%%%%%%%%%%%%%%%%
% \section{Sample}
%\iffalse
%<*samplemain>
%\fi
%
% The following presents a sample document
% with two chapters, two parts, a title page,
% a compile flag as well as three forwarding files to set the flag.
% It consists of eight |.tex| files:
% \begin{center}
% \begin{tabular}{ll}
% |cdocsamp.tex|&main file\\
% |cdocsch1.tex|&include file for chapter 1\\
% |cdocsch2.tex|&include file for chapter 2\\
% |cdocspt3.tex|&include file for part 3\\
% |cdocspt4.tex|&include file for part 4\\
% |cdocsdrf.tex|&forwarding file for main file in draft mode\\
% |cdocsfi1.tex|&forwarding file for final version of chapter 1\\
% |cdocsfi2.tex|&forwarding file for final version of chapter 2\\
% \end{tabular}
% \end{center}
% Each of the eight files can be compiled directly by the \LaTeX{} compiler.
%
% %%%%%%%%%%%%%%%%%%%%%%%%%%%%%%%%%%%%%%
% \paragraph{Main File.}
%
% The main file is called |cdocsamp.tex|.
%
% Load the \textsf{childdoc} definitions and
% declare the filename for the main document:
%    \begin{macrocode}
\input{childdoc.def}
\childdocmain{}
%    \end{macrocode}

% Optional override for |\version| flag:
%    \begin{macrocode}
%%\ifchilddoc\else\providecommand{\version}{draft}\fi
%    \end{macrocode}

% Define the default values for the |\version| flag
% (|final| for the main file and |draft| for childs):
%    \begin{macrocode}
\ifchilddoc
\providecommand{\version}{draft}
\else
\providecommand{\version}{final}
\fi
%    \end{macrocode}

% Load the standard document class:
%    \begin{macrocode}
\documentclass[12pt]{article}
%    \end{macrocode}

% Start the document body:
%    \begin{macrocode}
\begin{document}
%    \end{macrocode}

% Declare a title page.
% Print title, part of document being processed and version flag:
%    \begin{macrocode}
\addtocounter{page}{-1}
\begin{center}
{\LARGE\bfseries{}childdoc example\par}
\vspace{1cm}
\ifchilddoc
\ifchilddocmanual part\else chapter\fi:
`\childdocname' of `\childdocjob'\par
\else
main document: `\childdocjob'\par
\fi
version: \version\par
\end{center}
\newpage
%    \end{macrocode}

% Manually include selected file,
% otherwise process as usual:
%    \begin{macrocode}
\ifchilddocmanual
\section*{part `\childdocname'}
\input{\childdocname}
\else
%    \end{macrocode}

% Include the two chapters:
%    \begin{macrocode}
\include{cdocsch1}
\include{cdocsch2}
%    \end{macrocode}

% Include the two parts unless only chapters should be displayed:
%    \begin{macrocode}
\ifchilddoc\else
\section{part three}
\input{cdocspt3}
\section{part four}
\input{cdocspt4}
\fi
%    \end{macrocode}

% Process as usual until here:
%    \begin{macrocode}
\fi
%    \end{macrocode}

% End of document body:
%    \begin{macrocode}
\end{document}
%    \end{macrocode}
%\iffalse
%</samplemain>
%\fi
%
% %%%%%%%%%%%%%%%%%%%%%%%%%%%%%%%%%%%%%%
% \paragraph{Chapter Include Files.}
%
% The include files are called |cdocsch1.tex| and |cdocsch2.tex|.
%
%\iffalse
%<*samplechap1|samplechap2>
%\fi

% Optional override for |\version| flag:
%    \begin{macrocode}
%%\providecommand{\version}{final}
%    \end{macrocode}

% Include the main document:
%    \begin{macrocode}
\input{childdoc.def}
\childdocof{cdocsamp}
%    \end{macrocode}

%\iffalse
%</samplechap1|samplechap2>
%\fi
%
%\iffalse
%<*samplechap1>
%\fi
% Some text for chapter 1:
%    \begin{macrocode}
\section{one}
some text in chapter one
%    \end{macrocode}

%\iffalse
%</samplechap1>
%\fi
% Some text for chapter 2:
%\iffalse
%<*samplechap2>
%\fi
%    \begin{macrocode}
\section{two}
more text in chapter two
%    \end{macrocode}

%\iffalse
%</samplechap2>
%\fi
%
% %%%%%%%%%%%%%%%%%%%%%%%%%%%%%%%%%%%%%%
% \paragraph{Part Include Files.}
%
% The include files are called |cdocspt3.tex| and |cdocspt4.tex|.
%
%\iffalse
%<*samplepart3|samplepart4>
%\fi

% Optional override for |\version| flag:
%    \begin{macrocode}
%%\providecommand{\version}{final}
%    \end{macrocode}

% Include the main document:
%    \begin{macrocode}
\input{childdoc.def}
\childdocby{cdocsamp}
%    \end{macrocode}

%\iffalse
%</samplepart3|samplepart4>
%\fi
%
%\iffalse
%<*samplepart3>
%\fi
% Some text for part 3:
%    \begin{macrocode}
some text in part three
%    \end{macrocode}

%\iffalse
%</samplepart3>
%\fi
% Some text for part 4:
%\iffalse
%<*samplepart4>
%\fi
%    \begin{macrocode}
more text in part four
%    \end{macrocode}

%\iffalse
%</samplepart4>
%\fi
%
% %%%%%%%%%%%%%%%%%%%%%%%%%%%%%%%%%%%%%%
% \paragraph{Forwarding for a Complete Draft.}
%
% The following forwarding file |cdocsdrf.tex|
% compiles the main document in draft mode:
%\iffalse
%<*sampledraft>
%\fi
%    \begin{macrocode}
\def\version{draft}
\input{childdoc.def}
\childdocforward{cdocsamp}
%    \end{macrocode}

%\iffalse
%</sampledraft>
%\fi
%
% %%%%%%%%%%%%%%%%%%%%%%%%%%%%%%%%%%%%%%
% \paragraph{Forwarding for Final Version of the Chapters.}
%
% The following forwarding files |cdocsfn1.tex| and |cdocsfn2.tex|
% (with identical content)
% compile the final versions of the child documents
% |cdocsch1.tex| and |cdocsch2.tex|, respectively:
%\iffalse
%<*samplefinal>
%\fi
%    \begin{macrocode}
\def\version{final}
\input{childdoc.def}
\childdocforwardprefix[cdocsamp]{cdocsfn}{cdocsch}
%    \end{macrocode}

%\iffalse
%</samplefinal>
%\fi
%
% %%%%%%%%%%%%%%%%%%%%%%%%%%%%%%%%%%%%%%
% \paragraph{Command Line Processing.}
%
% The following three command lines generate the output files
% |cdocscld|, |cdocscl1| and |cdocscl2|
% which should be identical to
% |cdocsdrf|, |cdocsch1| and |cdocsfn2|, respectively:
% \begin{center}
% \begin{tabular}{l}
% |latex -jobname cdocscld \|\\
% |  "\def\version{draft}\input{childdoc.def}\childdocforward{cdocsamp}"|\\
% |latex -jobname cdocscl1 \|\\
% |  "\input{childdoc.def}\childdocforward[cdocsamp]{cdocsch1}"|\\
% |latex -jobname cdocscl2 \|\\
% |  "\def\version{final}\input{childdoc.def}\childdocforward{cdocsch2}"|
% \end{tabular}
% \end{center}
% Note that the trailing backslash on each first line
% merely continues the input to the second line
% (for convenient cut ant paste).
% Furthermore, the command |latex| can be replaced by any
% of its alternative versions such as |pdflatex|.
%
% %%%%%%%%%%%%%%%%%%%%%%%%%%%%%%%%%%%%%%%%%%%%%%%%%%%%%%%%%%%%%%%%%%%%%%%%%%%%%%
% %%%%%%%%%%%%%%%%%%%%%%%%%%%%%%%%%%%%%%%%%%%%%%%%%%%%%%%%%%%%%%%%%%%%%%%%%%%%%%
% \section{Implementation}
%\iffalse
%<*package>
%\fi
%
% This section describes the definitions file |childdoc.def|.

% The definitions cannot be loaded using |\usepackage| or |\RequirePackage|
% which has a mechanism to prevent loading a style file more than once.
% When loading the definitions by means of |\input|
% multiple instances have to be prevented manually:
%\iffalse
%This code needs to be before the `\ProvidesFile' directive
%which is defined at the beginning of this file.
%Therefore it is also placed there and commented out here.
%</package>
%<*discard>
%\fi
%    \begin{macrocode}
\ifdefined\childdocmain\endinput\fi
%    \end{macrocode}
%\iffalse
%</discard>
%<*package>
%\fi
%
% \macro{\ifchilddoc}
% \macro{\ifchilddocmanual}
% The conditional |\ifchilddoc| tells whether a
% child (true) or main (false) document is being compiled.
% The conditional |\ifchilddocmanual| tells whether
% the |\includeonly| mechanism is used (false) or
% the selection of child files must be performed manually (true).
% The definitions initialise to false:
%    \begin{macrocode}
\newif\ifchilddoc
\newif\ifchilddocmanual
%    \end{macrocode}

% \macro{\childdocname}
% \macro{\childdocjob}
% The macro |\childdocname| stores the name of the main document
% to be compiled. The macro |\childdocjob| stores the name of
% the document on which the \LaTeX{} compiler was originally invoked.
% The content of |\jobname| cannot be compared
% to filenames specified in the source due to different catcodes.
% The following code rescans |\jobname|, stores the result
% in |\childdocname| and saves a copy in |\childdocjob|:
%    \begin{macrocode}
\edef\childdocname{\scantokens\expandafter{\jobname\noexpand}}
\let\childdocjob\childdocname
%    \end{macrocode}

% \macro{\childdocdisable}
% The macro |\childdocdisable| prevents the main file
% from being processed more than once.
% At this stage, the main document command |\childdocmain|
% is assumed to be called once again where it should do nothing.
% Any subsequent call to it should prevent
% a secondary processing of the main document
% It overwrites the forwarding commands
% |\childdocof| and |\childdocforward|
% with empty macros to prevent further inclusions of the main document:
%    \begin{macrocode}
\newcommand{\childdocdisable}
{
  \renewcommand{\childdocmain}[1]{\renewcommand{\childdocmain}[1]{\endinput}}
  \renewcommand{\childdocof}[1]{}
  \renewcommand{\childdocby}[2][]{}
  \renewcommand{\childdocforward}[2][]{}
  \renewcommand{\childdocdisable}{}
}
%    \end{macrocode}

% \macro{\childdocmain}
% The macro |\childdocmain| is to be called at the top of the main file
% with nothing or the main filename (without extension) as argument.
% First, it breaks loops.
% If the argument is not empty and does not match |\childdocname|
% (which is set by the first inclusion of |childdoc.def|),
% |\ifchilddoc| is set to true, |\includeonly| is applied to the child file
% and |\jobname| is set to the main file
% (for proper handling of |.aux| files):
%    \begin{macrocode}
\newcommand{\childdocmain}[1]
{
  \childdocdisable\childdocmain{}
  \if?#1?\else
    \begingroup
      \def\childdoctmp{#1}
      \ifx\childdoctmp\childdocname
        \def\childdoctmp{}
      \else
        \def\childdoctmp
        {
          \childdoctrue
          \includeonly{\childdocname}
          \def\childdocjob{#1}
          \def\jobname{#1}
        }
      \fi
      \expandafter
    \endgroup
    \childdoctmp
  \fi
}
%    \end{macrocode}

% \macro{\childdocof}
% The command |\childdocof| redirects
% compilation to the main file |#1|.
%    \begin{macrocode}
\newcommand{\childdocof}[1]
{
  \childdocdisable
  \childdoctrue
  \includeonly{\childdocname}
  \def\jobname{#1}
  \def\childdocjob{#1}
  \input{#1}
}
%    \end{macrocode}

% \macro{\childdocby}
% The command |\childdocby| ....
%    \begin{macrocode}
\newcommand{\childdocby}[2][]
{
  \childdocdisable
  \childdoctrue
  \childdocmanualtrue
  \if?#1?\else
    \def\jobname{#2}
  \fi
  \def\childdocjob{#2}
  \input{#2}
  \endinput
}
%    \end{macrocode}

% \macro{\childdocforward}
% The command |\childdocforward| redirects
% compilation to the main file or
% (if the optional argument is given) a child file.
% Parameters are set as if the main file
% or a child file starting with |\childdocof| was compiled.
% Then compilation is handed over to the main file:
%    \begin{macrocode}
\newcommand{\childdocforward}[2][]
{
  \begingroup
    \if?#1?
      \def\childdoctmp
      {
        \def\childdocname{#2}
        \def\childdocjob{#2}
        \def\jobname{#2}
        \input{#2}
        \endinput
      }
    \else
      \def\childdoctmp
      {
        \childdocdisable
        \def\childdocname{#2}
        \childdoctrue
        \includeonly{#2}
        \def\childdocjob{#1}
        \def\jobname{#1}
        \input{#1}
        \endinput
      }
    \fi
    \expandafter
  \endgroup
  \childdoctmp
}
%    \end{macrocode}

% \macro{\childdocforwardprefix}
% The command |\childdocforwardprefix| redirects
% compilation to the main or a child file by means of a pattern.
% The prefix |#1| in the current filename is replaced by |#2|
% and the suffix of the current filename is kept
% (it is assumed that the filename does not contain the substring `|~~~|'
% which is used as a delimiter).
% Compilation is handed over to the new file by |\childdocforward|:
%    \begin{macrocode}
\newcommand{\childdocforwardprefix}[3][]
{
  \begingroup
    \def\childdocextract #2##1~~~{\def\childdoctmp{\childdocforward[#1]{#3##1}}}
    \expandafter\childdocextract\childdocname~~~
    \expandafter
  \endgroup
  \childdoctmp
}
%    \end{macrocode}

% \macro{\childdoc}
% The deprecated macro |\childdoc| is a legacy version of |\childdocmain|:
%    \begin{macrocode}
\newcommand{\childdoc}{\childdocmain}
%    \end{macrocode}

% \macro{\childdocredirect}
% The deprecated macro |\childdocredirect| is a legacy version
% of |\childdocforward| and |\childdocforwardprefix|:
%    \begin{macrocode}
\newcommand{\childdocredirect}[2][]
{
  \begingroup
    \if?#1?
      \def\childdoctmp{\childdocforward{#2}}
    \else
      \def\childdoctmp{\childdocforwardprefix{#1}{#2}}
    \fi
    \expandafter
  \endgroup
  \childdoctmp
}
%    \end{macrocode}

%\iffalse
%</package>
%\fi
%
\endinput

\childdocforward{cdocsamp}
%    \end{macrocode}

%\iffalse
%</sampledraft>
%\fi
%
% %%%%%%%%%%%%%%%%%%%%%%%%%%%%%%%%%%%%%%
% \paragraph{Forwarding for Final Version of the Chapters.}
%
% The following forwarding files |cdocsfn1.tex| and |cdocsfn2.tex|
% (with identical content)
% compile the final versions of the child documents
% |cdocsch1.tex| and |cdocsch2.tex|, respectively:
%\iffalse
%<*samplefinal>
%\fi
%    \begin{macrocode}
\def\version{final}
% \iffalse
%
% childdoc.dtx Copyright (C) 2017-2018 Niklas Beisert
%
% This work may be distributed and/or modified under the
% conditions of the LaTeX Project Public License, either version 1.3
% of this license or (at your option) any later version.
% The latest version of this license is in
%   http://www.latex-project.org/lppl.txt
% and version 1.3 or later is part of all distributions of LaTeX
% version 2005/12/01 or later.
%
% This work has the LPPL maintenance status `maintained'.
%
% The Current Maintainer of this work is Niklas Beisert.
%
% This work consists of the files childdoc.dtx and childdoc.ins
% and the derived files childdoc.def and cdocsamp.tex with
% cdocsch1.tex, cdocsch2.tex, cdocsdrf.tex, cdocsfn1.tex, cdocsfn2.tex.
%
%<package>\ifdefined\childdocmain\endinput\fi
%<package>\ProvidesFile{childdoc.def}[2018/12/30 v2.0 child document driver]
%<samplemain>\ProvidesFile{cdocsamp.tex}[2018/12/30 v2.0 sample for childdoc]
%<*driver>
%\ProvidesFile{childdoc.drv}[2018/12/30 v2.0 childdoc reference manual file]
\PassOptionsToClass{10pt,a4paper}{article}
\documentclass{ltxdoc}

\usepackage[margin=35mm]{geometry}
\usepackage{hyperref}
\usepackage{hyperxmp}
\usepackage[usenames]{color}

\hypersetup{colorlinks=true}
\hypersetup{pdfstartview=FitH}
\hypersetup{pdfpagemode=UseNone}
\hypersetup{pdfsource={}}
\hypersetup{pdflang={en-UK}}
\hypersetup{pdfcopyright={Copyright 2017-2018 Niklas Beisert.
  This work may be distributed and/or modified under the
  conditions of the LaTeX Project Public License, either version 1.3
  of this license or (at your option) any later version.}}
\hypersetup{pdflicenseurl={http://www.latex-project.org/lppl.txt}}
\hypersetup{pdfcontactaddress={ETH Zurich, ITP, HIT K,
  Wolfgang-Pauli-Strasse 27}}
\hypersetup{pdfcontactpostcode={8093}}
\hypersetup{pdfcontactcity={Zurich}}
\hypersetup{pdfcontactcountry={Switzerland}}
\hypersetup{pdfcontactemail={nbeisert@itp.phys.ethz.ch}}
\hypersetup{pdfcontacturl={http://people.phys.ethz.ch/\xmptilde nbeisert/}}

\newcommand{\secref}[1]{\hyperref[#1]{section \ref*{#1}}}

\parskip1ex
\parindent0pt
\let\olditemize\itemize
\def\itemize{\olditemize\parskip0pt}

\begin{document}

\title{The \textsf{childdoc} Package}
\hypersetup{pdftitle={The childdoc Package}}
\author{Niklas Beisert\\[2ex]
  Institut f\"ur Theoretische Physik\\
  Eidgen\"ossische Technische Hochschule Z\"urich\\
  Wolfgang-Pauli-Strasse 27, 8093 Z\"urich, Switzerland\\[1ex]
  \href{mailto:nbeisert@itp.phys.ethz.ch}
  {\texttt{nbeisert@itp.phys.ethz.ch}}}
\hypersetup{pdfauthor={Niklas Beisert}}
\hypersetup{pdfsubject={Manual for the LaTeX2e Package childdoc}}
\date{30 December 2018, \textsf{v2.0}}
\maketitle

\begin{abstract}\noindent
\textsf{childdoc} is a \LaTeXe{} package
that enables the direct compilation
of document sections included by |\include|
to individual files.
\end{abstract}

\begingroup
\parskip0ex
\tableofcontents
\endgroup

%%%%%%%%%%%%%%%%%%%%%%%%%%%%%%%%%%%%%%%%%%%%%%%%%%%%%%%%%%%%%%%%%%%%%%%%%%%%%%%%
%%%%%%%%%%%%%%%%%%%%%%%%%%%%%%%%%%%%%%%%%%%%%%%%%%%%%%%%%%%%%%%%%%%%%%%%%%%%%%%%
\section{Introduction}

\LaTeX{} provides a mechanism to structure a large document (such as a book)
into a main file and several child files (containing the chapters)
using the |\include| command.
This mechanism is beneficial for documents
which span hundreds of pages in order to
make the source file(s) more manageable.
Moreover, compilation can be restricted to
selected child files by means of the |\includeonly| command.
The latter feature can be used to reduce the compilation time while editing
(this was significantly more useful in the earlier days of \LaTeX{})
or to generate a smaller document which is easier to navigate.
Another application of |\includeonly| is to generate
documents consisting of selected parts of the complete document.

However, there are a few drawbacks of the plain |\include| mechanism:
\begin{itemize}
\item
The child files cannot be compiled on their own,
they can only be compiled via the main file.
A naive editing environment
(such as a text editor with an option
to have the current file processed by \LaTeX)
may require one to switch to the main file before compiling;
attempting to compile the child file produces errors.
\item
The main file must be modified (each time)
to adjust the |\includeonly| command
to the present needs. This easily leaves the main file in a messy state.
\item
The generated document will always carry the filename
of the main document. This is inconvenient if
several child files are to be compiled and
to be kept for distribution.
\end{itemize}

The present package provides a simple interface
to make child files individually compilable by \LaTeX{}.
Compiling a child file then has the same effect as compiling
the main file with an |\includeonly| command
to select the appropriate child.
Moreover the generated document will carry the name of the child
rather than the main file.
This resolves all three above issues.

This feature is meant to make the editing of books,
thesis documents and lecture notes somewhat more convenient.
However, the package can also be used efficiently for
composing a series of documents (such as exercise sheets)
which are typically distributed individually.
It then assists the author in generating the individual documents
(potentially in different versions)
as well as a document containing the collected series.
Another application is in developing style files
or other kinds of included material
where compilation of the style file could redirect
to a sample or test file.

%%%%%%%%%%%%%%%%%%%%%%%%%%%%%%%%%%%%%%%%%%%%%%%%%%%%%%%%%%%%%%%%%%%%%%%%%%%%%%%%
%%%%%%%%%%%%%%%%%%%%%%%%%%%%%%%%%%%%%%%%%%%%%%%%%%%%%%%%%%%%%%%%%%%%%%%%%%%%%%%%
\section{Usage}

First of all, the package \textsf{childdoc} is \emph{not} a standard
\LaTeXe{} |.sty| style file! Therefore it needs to be invoked in
a non-standard way.

%%%%%%%%%%%%%%%%%%%%%%%%%%%%%%%%%%%%%%%%%%%%%%%%%%%%%%%%%%%%%%%%%%%%%%%%%%%%%%%%
\subsection{Included Files}
\label{sec:include}

%%%%%%%%%%%%%%%%%%%%%%%%%%%%%%%%%%%%%%%%
\DescribeMacro{\childdocmain}
To use the package, add the commands
\begin{center}
\begin{tabular}{l}
|\input{childdoc.def}|\\
|\childdocmain{}|\\
\end{tabular}
\end{center}
at the very top of the main \LaTeX{} file,
in particular \emph{before} the |\documentclass| statement!
The argument of |\childdocmain| should be left empty
(but it must be present).

%%%%%%%%%%%%%%%%%%%%%%%%%%%%%%%%%%%%%%%%
\DescribeMacro{\childdocof}
Furthermore, add the commands
\begin{center}
\begin{tabular}{l}
|\input{childdoc.def}|\\
|\childdocof{|\textit{main}|}|\\
\end{tabular}
\end{center}
at the top of every child file \textit{child}
which is included by |\include{|\textit{child}|}|
from within the main file
(or at least for those files to be compiled individually).
The argument \textit{main} must be the filename of the main file.

There are a couple of
considerations in setting up the main and child documents:

%%%%%%%%%%%%%%%%%%%%%%%%%%%%%%%%%%%%%%%%
\paragraph{Restrictions.}

Please note the following restrictions:
\begin{itemize}
\item
|\childdocmain| must be called with one argument \textit{main}
to ensure compatibility with earlier version of the package.
It must either be empty (|\childdocmain{}|)
or precisely match the filename of the main file in which it is specified.
See \secref{sec:detection} for further information.
\item
The filename \textit{main} must be specified without the |.tex| extension.
\item
The filename \textit{main} is case sensitive
(even in case-insensitive file systems)
due to internal string comparison.
\item
The argument \textit{main} should be fully expanded, it cannot be a macro.
\item
Subdirectories and special characters should be avoided in filenames.
\item
The command |\childdocmain{|\textit{main}|}| must be followed by a whitespace.
It should not be followed immediately by another command
or by a comment mark `|%|'.
This is because the \TeX{} parser reads the token immediately following
the argument of |\childdocmain| and puts it
at the beginning of every child section;
however, a white\-space is ignored.
\end{itemize}

%%%%%%%%%%%%%%%%%%%%%%%%%%%%%%%%%%%%%%%%
\paragraph{Content of Main File.}

It is advisable to place all content in the child files included by |\include|.
Any output contained in the main file will appear in all child documents
unless suppressed manually;
it cannot be suppressed automatically by the |\includeonly| directive
and thus should normally be avoided.
A method to include some content in the main file
by means of conditional processing is described in \secref{sec:conditional}.

%%%%%%%%%%%%%%%%%%%%%%%%%%%%%%%%%%%%%%%%
\paragraph{Page Numbering.}

When only a part of the document is compiled,
the appropriate numbering of pages
(as well as other status parameters)
is determined from the |.aux| files.
The latter contain information from previous passes.
However this information needs to propagate through
all intermediate child documents.
Therefore the page numbering in child documents may well
be inconsistent until the complete document is compiled at least once.

A useful (if unconventional) way to always ensure a consistent
page numbering is to restart the numbering in each child document
and denote the pages by `\textit{child}|.|\textit{page}'
where \textit{child} represents the chapter/section number of the child file.
This can be achieved by the command
|\numberwithin{page}{|\textit{child}|}|
of the \textsf{amsmath} package
where \textit{child} can be |chapter| or |section|
depending on the chosen structuring.
Alternatively, one can modify the macro |\thepage| appropriately
and reset the counter |page| at the start of each child file.

%%%%%%%%%%%%%%%%%%%%%%%%%%%%%%%%%%%%%%%%%%%%%%%%%%%%%%%%%%%%%%%%%%%%%%%%%%%%%%%%
\subsection{Conditional Processing}
\label{sec:conditional}

The package provides a mechanism to compile different versions
of a document. To customise the versions further some conditional processing
can come in handy to distinguish which version is being compiled.
The package provides two macros to describe the compilation context:

%%%%%%%%%%%%%%%%%%%%%%%%%%%%%%%%%%%%%%%%
\DescribeMacro{\ifchilddoc}
The conditional |\ifchilddoc| distinguishes between the compilation of
child documents and the main document:
%
\begin{center}
|\ifchilddoc |\textit{child-code}| |[|\||else |\textit{main-code}]| \||fi|
\end{center}

%%%%%%%%%%%%%%%%%%%%%%%%%%%%%%%%%%%%%%%%
\DescribeMacro{\childdocname}
\DescribeMacro{\childdocjob}
The macro |\childdocname| contains the filename (without extension)
of the main or child file being processed.
Note that |\childdocjob| will always contain the name of the main file.

%%%%%%%%%%%%%%%%%%%%%%%%%%%%%%%%%%%%%%%%
\paragraph{Title Page.}

Conditional processing can be used to include a title or banner page
in the main document when proper precautions are taken.
Importantly, the code in the main file should ensure that the page counter
(as well as other status parameters which are stored in the |.aux| files)
takes the same value after the conditional processing.
Otherwise the page numbers may take divergent values
depending on which part is compiled.

For example, a title page could be declared by:
%
\begin{center}
\begin{tabular}{l}
|\ifchilddoc\||else|\\
|\addtocounter{page}{-1}|\\
\textit{code for title page}\\
|\newpage|\\
|\||fi|
\end{tabular}
\end{center}
%
A banner page for the child documents can be generated by:
%
\begin{center}
\begin{tabular}{l}
|\ifchilddoc|\\
|\addtocounter{page}{-1}|\\
\textit{code for banner page}\\
|\newpage|\\
|\||fi|
\end{tabular}
\end{center}
%
Here one could write a message such as:
\begin{center}
|This is the part \childdocname{} of \childdocjob{}.|
\end{center}

%%%%%%%%%%%%%%%%%%%%%%%%%%%%%%%%%%%%%%%%%%%%%%%%%%%%%%%%%%%%%%%%%%%%%%%%%%%%%%%%
\subsection{Flags}
\label{sec:flags}

The package makes it easy to generate different versions
of the main or child documents.
To this end compilation flags can be defined
and assigned different default values.
They will be particularly useful in conjunction
with the forwarding mechanism described in \secref{sec:forward}.

For example, it may be useful to have a flag |\version|
which can be set to |draft| or |final|.
The document source will contain some conditional code
depending on the value of |\version|.
Suppose further, the flag should default to |final| for the main file
and to |draft| for child files
which is a natural assignment for editing the document.
This is achieved by placing the following code
in the preamble of the main document
(below the |\childdocmain| directive):
%
\begin{center}
\begin{tabular}{l}
|\ifchilddoc|\\
|\providecommand{\version}{draft}|\\
|\||else|\\
|\providecommand{\version}{final}|\\
|\||fi|
\end{tabular}
\end{center}
%
The definition by |\providecommand| makes sure
that previous definitions are not overwritten.
Further statements |\providecommand{\version}{...}|
can thus be added before the above code to override it.

For the main file, one might add a line
(between |\childdocmain| and the above block)
%
\begin{center}
|%\ifchilddoc\||else\providecommand{\version}{draft}\||fi|
\end{center}
%
which can be uncommented to produce a draft version.
Likewise one can add a line to the very top of a child file
(above the |\childdocof{|\textit{main}|}| directive)
%
\begin{center}
|%\providecommand{\version}{final}|
\end{center}
%
which can be uncommented to produce the final version of this child document.

%%%%%%%%%%%%%%%%%%%%%%%%%%%%%%%%%%%%%%%%%%%%%%%%%%%%%%%%%%%%%%%%%%%%%%%%%%%%%%%%
\subsection{Forwarding}
\label{sec:forward}

Different versions of the main or child documents
using compilation flags as described in \secref{sec:flags}
can be (permanently) stored in different files
for convenient compilation, viewing and distribution.
To this end, the package defines a command
to pass on compilation to a different file:

%%%%%%%%%%%%%%%%%%%%%%%%%%%%%%%%%%%%%%%%
\DescribeMacro{\childdocforward}
The command |\childdocforward| redirects processing to
another source file:
%
\begin{center}
\begin{tabular}{l}
|\input{childdoc.def}|\\
|\childdocforward[|\textit{main}|]{|\textit{dest}|}|\\
\end{tabular}
\end{center}
%
The argument \textit{dest} is the destination file
(without extension).
It should be the main file or one of the child files.
Note that further \textsf{childdoc} directives
such as |\childdocof| and |\childdocforward|
in the indicated file will be processed in this form.
The optional argument \textit{main}
passes on directly to the main file \textit{main}
while pretending to compile the child \textit{dest}.
This form behaves as if \textit{dest}
issues |\childdocof{|\textit{main}|}| right away,
and no further \textsf{childdoc} directives will be processed.

%%%%%%%%%%%%%%%%%%%%%%%%%%%%%%%%%%%%%%%%
\DescribeMacro{\...prefix}
In the alternative form |\childdocforwardprefix|,
%
\begin{center}
\begin{tabular}{l}
|\input{childdoc.def}|\\
|\childdocforwardprefix[|\textit{main}|]{|\textit{prefix}|}{|\textit{dest}|}|
\end{tabular}
\end{center}
%
the destination file is determined by a pattern
depending on the current file:
To make this work, the current file must be called
`{\textit{prefix}\hspace{0.2em}\textit{suffix}}'
with \textit{prefix} matching precisely the argument.
Processing is then passed on to the file
`{\textit{dest}\hspace{0.2em}\textit{suffix}}'.
Surely, the same effect is achieved by
directly specifying the
argument `{\textit{dest}\hspace{0.2em}\textit{suffix}}'
in the first form.
However, that requires to set up a different file
for each child. With the alternative form of the command
all these files can have exactly the same content
which simplifies setting them up and maintaining them.

For example, the following file |draft.tex|
with a compilation flag |\version| as described in \secref{sec:flags}
compiles the main document as a draft:
%
\begin{center}
\begin{tabular}{l}
|\def\version{draft}|\\
|\input{childdoc.def}|\\
|\childdocforward{|\textit{main}|}|
\end{tabular}
\end{center}
%
Likewise, the following files |final|\textit{nn}|.tex|
compile the final version of the child document
|child|\textit{nn}|.tex|:
%
\begin{center}
\begin{tabular}{l}
|\def\version{final}|\\
|\input{childdoc.def}|\\
|\childdocforwardprefix{final}{child}|
\end{tabular}
\end{center}
%

Note that when several versions of a main file and/or of each child file
are to be generated, it may be convenient to set up a |Makefile| or
shell script to automatise the process.

%%%%%%%%%%%%%%%%%%%%%%%%%%%%%%%%%%%%%%%%%%%%%%%%%%%%%%%%%%%%%%%%%%%%%%%%%%%%%%%%
\subsection{Command Line Processing}
\label{sec:commandline}

The effect of redirection files can also be achieved by invoking
the \LaTeX{} compiler with a more elaborate command line.
Most conveniently this should be done as part
of a shell script or a |Makefile|.

When using \textsf{childdoc} in the main file, the following
command lines effectively perform a redirection
(note that depending on the shell being used,
backslashes may have to be doubled: `|\|' $\to$ `|\\|'):
%
\begin{center}
|... -jobname "|\textit{target}|" |\\|"|[\textit{flags}]%
|\input{childdoc.def}\childdocforward[|\textit{main}|]{|\textit{dest}|}"|
\end{center}
%
Here \textit{target} is the name of the output file,
\textit{main} is the name of the main file
and \textit{dest} is the name of the main or child file to be processed
(all filenames without extensions).
The optional argument \textit{main} can be omitted
if \textit{main} matches \textit{dest}.
Optionally, compilation \textit{flags} can be defined via |\def| commands.
This command line makes the \TeX{} engine believe
it is compiling the file \textit{target}
whose content is specified as the latter parameter.
The provided code then forwards the processing to
\textit{main} or \textit{dest} as described in \secref{sec:forward}.

%%%%%%%%%%%%%%%%%%%%%%%%%%%%%%%%%%%%%%%%%%%%%%%%%%%%%%%%%%%%%%%%%%%%%%%%%%%%%%%%
\subsection{Include by Input}
\label{sec:input}

Including child documents by |\include| has some restrictions by design.
Most notably, the content of a child document always occupies
its own set of pages; pages cannot be shared between child documents.
Usually, this behaviour makes perfect sense
because each child document contain an essential part of the document.
However, in some situations it may be desirable to compose
a document from a collection of parts
without having mandatory page breaks between then.
For this case, the package
provides a mechanism to include parts
by |\input| which can also be processed individually.
However, by construction this mechanism
requires manual handling of the content to be output.

%%%%%%%%%%%%%%%%%%%%%%%%%%%%%%%%%%%%%%%%
\DescribeMacro{\ifchilddocmanual}
The main file should be prepared as usual, see \secref{sec:include}.
However, the document body must make a distinction
between processing of an individual part and of the main document, e.g.:
%
\begin{center}
\begin{tabular}{l}
|\ifchilddocmanual|\\
|\input{\childdocname}|\\
|\||else|\\
\textit{document body with }|\input{|\textit{part}|}|\\
|\||fi|
\end{tabular}
\end{center}
%
The conditional |\ifchilddocmanual| is true whenever
a part to be included by |\input| is being compiled,
and the name of the part is stored in |\childdocname|.

%%%%%%%%%%%%%%%%%%%%%%%%%%%%%%%%%%%%%%%%
\DescribeMacro{\childdocby}
Each part to be included by |\input| should start with:
%
\begin{center}
\begin{tabular}{l}
|\input{childdoc.def}|\\
|\childdocby{|\textit{main}|}|\\
\end{tabular}
\end{center}
%
The directive |\childdocby| is similar to |\childdocof|
described in \secref{sec:include},
but the subsequent selection of content must be done manually.
To that end, both |\ifchilddoc| and |\ifchilddocmanual|
will be true upon processing of a part,
and the name of the part is stored in |\childdocname|.
Note that |\jobname| will be set to the filename of the current part
so that each part receives an individual |.aux| file
that does not interfere with the |.aux| file(s) of the main document.
This behaviour can be altered by the alternative form
|\childdocby[*]{|\textit{main}|}| (with a non-empty optional argument)
which uses the |.aux| file of the main document
by setting |\jobname| to \textit{main}.

%%%%%%%%%%%%%%%%%%%%%%%%%%%%%%%%%%%%%%%%%%%%%%%%%%%%%%%%%%%%%%%%%%%%%%%%%%%%%%%%
\subsection{Driver Development}
\label{sec:driver}

The \textsf{childdoc} mechanism can also be use for the development
of definition files such as \LaTeX{} styles or classes.
This case differs from the above setup with multiple parts
included by |\include| in that no |\includeonly| should be invoked.
This can be achieved by starting the include file
(before |\ProvidesPackage|) with:
%
\begin{center}
\begin{tabular}{l}
|\input{childdoc.def}|\\
|\childdocforward{|\textit{main}|}|\\
\end{tabular}
\end{center}
%
or alternatively with:
%
\begin{center}
\begin{tabular}{l}
|\input{childdoc.def}|\\
|\childdocby{|\textit{main}|}|\\
\end{tabular}
\end{center}
%
Both forms have slightly different effects as described above.
The main file is prepared as usual, see \secref{sec:include}.

%%%%%%%%%%%%%%%%%%%%%%%%%%%%%%%%%%%%%%%%%%%%%%%%%%%%%%%%%%%%%%%%%%%%%%%%%%%%%%%%
\subsection{Legacy Detection}
\label{sec:detection}

The directive |\childdocmain| in the main file can detect
whether the complete document or merely a child is to be compiled
even without using the directive |\childdocof|.
This method is deprecated because it is less robust
and there is no compelling reason to use it;
it is merely provided for backward compatibility
and it may be removed in future versions.

If the detection mechanism is to be used,
it is mandatory to correctly specify
the filename of the main file as the argument of |\childdocmain|:
%
\begin{center}
\begin{tabular}{l}
|\input{childdoc.def}|\\
|\childdocmain{|\textit{main}|}|\\
\end{tabular}
\end{center}
%
If |\jobname| does not match the argument \textit{main} of |\childdocmain|,
it is assumed that |\jobname| points to the child file to be compiled.
When using |\childdocmain| with the main file specified as argument,
it suffices to start a child file
with just |\input{|\textit{main}|}|
without loading of the package and using |\childdocof|.
If instead all processing is done
with the appropriate \textsf{childdoc} directives,
the argument of \textit{main} of |\childdocmain| can be empty.

An alternative version of the command line processing described
in \secref{sec:commandline} using the detection mechanism reads:
%
\begin{center}
|... -jobname "|\textit{target}|" "|[\textit{flags}]%
[|\def\jobname{|\textit{dest}|}|]|\input{|\textit{main}|}"|
\end{center}

%%%%%%%%%%%%%%%%%%%%%%%%%%%%%%%%%%%%%%%%%%%%%%%%%%%%%%%%%%%%%%%%%%%%%%%%%%%%%%%%
\subsection{Manual Code}
\label{sec:manual}

In case one cannot be certain whether the definitions file |childdoc.def|
is installed on the target \TeX{} distribution
and one prefers not to ship it,
it is conceivable to paste a few relevant commands into the sources.

To that end, drop all statements |\input{childdoc.def}|
and perform the replacements as outlined below.
Instead of |\childdocmain{|\textit{main}|}| add the following code
to the top of the main file:
%
\begin{center}
\begin{tabular}{l}
|\||ifdefined\childdocname\endinput\||fi\newif\ifchilddoc|\\
|\edef\childdocname{\scantokens\expandafter{\jobname\noexpand}}|\\
|\def\childdocmain{|\textit{main}|}\||ifx\childdocmain\childdocname\||else|\\
|\childdoctrue\includeonly{\childdocname}\let\jobname\childdocmain\||fi|\\
\end{tabular}
\end{center}
%
Instead of |\childdocof{|\textit{main}|}| just include the main file
at the top of each child file:
%
\begin{center}
|\input{|\textit{main}|}|
\end{center}
%
A simple redirection |\childdocforward{|\textit{dest}|}| is achieved by:
%
\begin{center}
|\def\jobname{|\textit{dest}|}\input{\jobname}|
\end{center}
%
The redirection with prefix
|\childdocforwardprefix[|\textit{prefix}|]{|\textit{dest}|}|
is accomplished by:
%
\begin{center}
\begin{tabular}{l}
|{\edef\jobname{\scantokens\expandafter{\jobname\noexpand}}|\\
|\def\redirectjob |\textit{prefix}|#1~~~{\gdef\jobname{|\textit{dest}|#1}}|\\
|\expandafter\redirectjob\jobname~~~}\input{\jobname}|
\end{tabular}
\end{center}

In an alternative approach,
child documents can be compiled by a specific command line
without additional code or specific definitions:
%
\begin{center}
|... -jobname "|\textit{target}|" "|[\textit{flags}]%
|\includeonly{|\textit{dest}|}\input{|\textit{main}|}"|
\end{center}
%

%%%%%%%%%%%%%%%%%%%%%%%%%%%%%%%%%%%%%%%%%%%%%%%%%%%%%%%%%%%%%%%%%%%%%%%%%%%%%%%%
%%%%%%%%%%%%%%%%%%%%%%%%%%%%%%%%%%%%%%%%%%%%%%%%%%%%%%%%%%%%%%%%%%%%%%%%%%%%%%%%
\section{Information}

%%%%%%%%%%%%%%%%%%%%%%%%%%%%%%%%%%%%%%%%%%%%%%%%%%%%%%%%%%%%%%%%%%%%%%%%%%%%%%%%
\subsection{Copyright}

Copyright \copyright{} 2017--2018 Niklas Beisert

This work may be distributed and/or modified under the
conditions of the \LaTeX{} Project Public License, either version 1.3
of this license or (at your option) any later version.
The latest version of this license is in
  \url{http://www.latex-project.org/lppl.txt}
and version 1.3 or later is part of all distributions of \LaTeX{}
version 2005/12/01 or later.

This work has the LPPL maintenance status `maintained'.

The Current Maintainer of this work is Niklas Beisert.

This work consists of the files |README.txt|, |childdoc.ins| and |childdoc.dtx|
as well as the derived files |childdoc.def|, |cdocsamp.tex|
with |cdocsch1.tex|, |cdocsch2.tex|, |cdocspt3.tex|, |cdocspt4.tex|,
|cdocsdrf.tex|, |cdocsfn1.tex|, |cdocsfn2.tex|
as well as |childdoc.pdf|.

%%%%%%%%%%%%%%%%%%%%%%%%%%%%%%%%%%%%%%%%%%%%%%%%%%%%%%%%%%%%%%%%%%%%%%%%%%%%%%%%
\subsection{Files and Installation}

The package consists of the files:
%
\begin{center}
\begin{tabular}{ll}
    |README.txt|   & readme file \\
    |childdoc.ins| & installation file \\
    |childdoc.dtx| & source file \\
    |childdoc.def| & definition file \\
    |cdocsamp.tex| & sample main file \\
    |cdocsch1.tex| & sample include file \\
    |cdocsch2.tex| & sample include file \\
    |cdocspt3.tex| & sample part file \\
    |cdocspt4.tex| & sample part file \\
    |cdocsdrf.tex| & sample redirection file \\
    |cdocsfn1.tex| & sample redirection file \\
    |cdocsfn2.tex| & sample redirection file \\
    |childdoc.pdf| & manual
\end{tabular}
\end{center}
%
The distribution consists of the files
|README.txt|, |childdoc.ins| and |childdoc.dtx|.
%
\begin{itemize}
\item
Run (pdf)\LaTeX{} on |childdoc.dtx|
to compile the manual |childdoc.pdf| (this file).
\item
Run \LaTeX{} on |childdoc.ins| to create the definitions file |childdoc.def|
and the sample |cdocsamp.tex| with include files
|cdocsch1.tex|, |cdocsch2.tex|, |cdocspt3.tex|, |cdocspt4.tex|,
|cdocsdrf.tex|, |cdocsfn1.tex|, |cdocsfn2.tex|.
Then copy the file |childdoc.def| to an appropriate directory of your \LaTeX{}
distribution, e.g.\ \textit{texmf-root}|/tex/latex/childdoc|.
\end{itemize}

%%%%%%%%%%%%%%%%%%%%%%%%%%%%%%%%%%%%%%%%%%%%%%%%%%%%%%%%%%%%%%%%%%%%%%%%%%%%%%%%
\subsection{Related CTAN Packages}

There are several other packages which offer a similar functionality:
%
\begin{itemize}
\item
The packages
\href{http://ctan.org/pkg/docmute}{\textsf{docmute}},
\href{http://ctan.org/pkg/includex}{\textsf{includex}} and
\href{http://ctan.org/pkg/standalone}{\textsf{standalone}}
provide commands to include only the document body of
a child file thus allowing both files to be compiled individually.
\item
The packages \href{http://ctan.org/pkg/subdocs}{\textsf{subdocs}}
and \href{http://ctan.org/pkg/subfiles}{\textsf{subfiles}}
provide structures in which the main and child documents can be
encapsulated and allowing them to be compiled individually.
The inclusion mechanism is different from the conventional |\include|.
\item
The package \href{http://ctan.org/pkg/combine}{\textsf{combine}}
is an elaborate solution to combine several documents into one.
\end{itemize}
%
See also the CTAN topic \href{http://ctan.org/topic/subdocs}{\textsf{subdocs}}
for further related packages.
The present package differs from the above solutions in that
a document structure constructed with the conventional |\include| mechanism
just needs two extra commands at the top of every file
such that all constituent files can be compiled individually.

%%%%%%%%%%%%%%%%%%%%%%%%%%%%%%%%%%%%%%%%%%%%%%%%%%%%%%%%%%%%%%%%%%%%%%%%%%%%%%%%
%\subsection{Feature Suggestions}
%
%The following is a list of features which may be useful for future
%versions of this package:
%%
%\begin{itemize}
%\item
%\ldots
%\end{itemize}

%%%%%%%%%%%%%%%%%%%%%%%%%%%%%%%%%%%%%%%%%%%%%%%%%%%%%%%%%%%%%%%%%%%%%%%%%%%%%%%%
\subsection{Revision History}

%%%%%%%%%%%%%%%%%%%%%%%%%%%%%%%%%%%%%%%%
\paragraph{v2.0:} 2018/12/30

\begin{itemize}
\item
immediate forward processing
\item
added |\childdocby| mechanism
\item
manual restructured
\end{itemize}

%%%%%%%%%%%%%%%%%%%%%%%%%%%%%%%%%%%%%%%%
\paragraph{v1.6:} 2018/01/17

\begin{itemize}
\item
application for development of include files
\item
corrections to manual
\end{itemize}

%%%%%%%%%%%%%%%%%%%%%%%%%%%%%%%%%%%%%%%%
\paragraph{v1.5:} 2017/05/21

\begin{itemize}
\item
more complete structuring introduced
\item
|\childdocof| introduced
\item
|\childdoc| renamed to |\childdocmain|
\item
|\childredirect| renamed to |\childdocforward| and |\childdocforwardprefix|
and functionality expanded
\end{itemize}

%%%%%%%%%%%%%%%%%%%%%%%%%%%%%%%%%%%%%%%%
\paragraph{v1.0:} 2017/04/27

\begin{itemize}
\item
manual and install package
\item
first version published on CTAN
\end{itemize}

%%%%%%%%%%%%%%%%%%%%%%%%%%%%%%%%%%%%%%%%
\paragraph{v0.6:} 2017/04/26

\begin{itemize}
\item
redirection mechanism added
\end{itemize}

%%%%%%%%%%%%%%%%%%%%%%%%%%%%%%%%%%%%%%%%
\paragraph{v0.5:} 2017/04/26

\begin{itemize}
\item
functionality in definition file
\end{itemize}


%%%%%%%%%%%%%%%%%%%%%%%%%%%%%%%%%%%%%%%%%%%%%%%%%%%%%%%%%%%%%%%%%%%%%%%%%%%%%%%%
%%%%%%%%%%%%%%%%%%%%%%%%%%%%%%%%%%%%%%%%%%%%%%%%%%%%%%%%%%%%%%%%%%%%%%%%%%%%%%%%
%%%%%%%%%%%%%%%%%%%%%%%%%%%%%%%%%%%%%%%%%%%%%%%%%%%%%%%%%%%%%%%%%%%%%%%%%%%%%%%%
\appendix

\settowidth\MacroIndent{\rmfamily\scriptsize 000\ }

 \DocInput{childdoc.dtx}

\end{document}
%</driver>
% \fi
%
% %%%%%%%%%%%%%%%%%%%%%%%%%%%%%%%%%%%%%%%%%%%%%%%%%%%%%%%%%%%%%%%%%%%%%%%%%%%%%%
% %%%%%%%%%%%%%%%%%%%%%%%%%%%%%%%%%%%%%%%%%%%%%%%%%%%%%%%%%%%%%%%%%%%%%%%%%%%%%%
% \section{Sample}
%\iffalse
%<*samplemain>
%\fi
%
% The following presents a sample document
% with two chapters, two parts, a title page,
% a compile flag as well as three forwarding files to set the flag.
% It consists of eight |.tex| files:
% \begin{center}
% \begin{tabular}{ll}
% |cdocsamp.tex|&main file\\
% |cdocsch1.tex|&include file for chapter 1\\
% |cdocsch2.tex|&include file for chapter 2\\
% |cdocspt3.tex|&include file for part 3\\
% |cdocspt4.tex|&include file for part 4\\
% |cdocsdrf.tex|&forwarding file for main file in draft mode\\
% |cdocsfi1.tex|&forwarding file for final version of chapter 1\\
% |cdocsfi2.tex|&forwarding file for final version of chapter 2\\
% \end{tabular}
% \end{center}
% Each of the eight files can be compiled directly by the \LaTeX{} compiler.
%
% %%%%%%%%%%%%%%%%%%%%%%%%%%%%%%%%%%%%%%
% \paragraph{Main File.}
%
% The main file is called |cdocsamp.tex|.
%
% Load the \textsf{childdoc} definitions and
% declare the filename for the main document:
%    \begin{macrocode}
\input{childdoc.def}
\childdocmain{}
%    \end{macrocode}

% Optional override for |\version| flag:
%    \begin{macrocode}
%%\ifchilddoc\else\providecommand{\version}{draft}\fi
%    \end{macrocode}

% Define the default values for the |\version| flag
% (|final| for the main file and |draft| for childs):
%    \begin{macrocode}
\ifchilddoc
\providecommand{\version}{draft}
\else
\providecommand{\version}{final}
\fi
%    \end{macrocode}

% Load the standard document class:
%    \begin{macrocode}
\documentclass[12pt]{article}
%    \end{macrocode}

% Start the document body:
%    \begin{macrocode}
\begin{document}
%    \end{macrocode}

% Declare a title page.
% Print title, part of document being processed and version flag:
%    \begin{macrocode}
\addtocounter{page}{-1}
\begin{center}
{\LARGE\bfseries{}childdoc example\par}
\vspace{1cm}
\ifchilddoc
\ifchilddocmanual part\else chapter\fi:
`\childdocname' of `\childdocjob'\par
\else
main document: `\childdocjob'\par
\fi
version: \version\par
\end{center}
\newpage
%    \end{macrocode}

% Manually include selected file,
% otherwise process as usual:
%    \begin{macrocode}
\ifchilddocmanual
\section*{part `\childdocname'}
\input{\childdocname}
\else
%    \end{macrocode}

% Include the two chapters:
%    \begin{macrocode}
\include{cdocsch1}
\include{cdocsch2}
%    \end{macrocode}

% Include the two parts unless only chapters should be displayed:
%    \begin{macrocode}
\ifchilddoc\else
\section{part three}
\input{cdocspt3}
\section{part four}
\input{cdocspt4}
\fi
%    \end{macrocode}

% Process as usual until here:
%    \begin{macrocode}
\fi
%    \end{macrocode}

% End of document body:
%    \begin{macrocode}
\end{document}
%    \end{macrocode}
%\iffalse
%</samplemain>
%\fi
%
% %%%%%%%%%%%%%%%%%%%%%%%%%%%%%%%%%%%%%%
% \paragraph{Chapter Include Files.}
%
% The include files are called |cdocsch1.tex| and |cdocsch2.tex|.
%
%\iffalse
%<*samplechap1|samplechap2>
%\fi

% Optional override for |\version| flag:
%    \begin{macrocode}
%%\providecommand{\version}{final}
%    \end{macrocode}

% Include the main document:
%    \begin{macrocode}
\input{childdoc.def}
\childdocof{cdocsamp}
%    \end{macrocode}

%\iffalse
%</samplechap1|samplechap2>
%\fi
%
%\iffalse
%<*samplechap1>
%\fi
% Some text for chapter 1:
%    \begin{macrocode}
\section{one}
some text in chapter one
%    \end{macrocode}

%\iffalse
%</samplechap1>
%\fi
% Some text for chapter 2:
%\iffalse
%<*samplechap2>
%\fi
%    \begin{macrocode}
\section{two}
more text in chapter two
%    \end{macrocode}

%\iffalse
%</samplechap2>
%\fi
%
% %%%%%%%%%%%%%%%%%%%%%%%%%%%%%%%%%%%%%%
% \paragraph{Part Include Files.}
%
% The include files are called |cdocspt3.tex| and |cdocspt4.tex|.
%
%\iffalse
%<*samplepart3|samplepart4>
%\fi

% Optional override for |\version| flag:
%    \begin{macrocode}
%%\providecommand{\version}{final}
%    \end{macrocode}

% Include the main document:
%    \begin{macrocode}
\input{childdoc.def}
\childdocby{cdocsamp}
%    \end{macrocode}

%\iffalse
%</samplepart3|samplepart4>
%\fi
%
%\iffalse
%<*samplepart3>
%\fi
% Some text for part 3:
%    \begin{macrocode}
some text in part three
%    \end{macrocode}

%\iffalse
%</samplepart3>
%\fi
% Some text for part 4:
%\iffalse
%<*samplepart4>
%\fi
%    \begin{macrocode}
more text in part four
%    \end{macrocode}

%\iffalse
%</samplepart4>
%\fi
%
% %%%%%%%%%%%%%%%%%%%%%%%%%%%%%%%%%%%%%%
% \paragraph{Forwarding for a Complete Draft.}
%
% The following forwarding file |cdocsdrf.tex|
% compiles the main document in draft mode:
%\iffalse
%<*sampledraft>
%\fi
%    \begin{macrocode}
\def\version{draft}
\input{childdoc.def}
\childdocforward{cdocsamp}
%    \end{macrocode}

%\iffalse
%</sampledraft>
%\fi
%
% %%%%%%%%%%%%%%%%%%%%%%%%%%%%%%%%%%%%%%
% \paragraph{Forwarding for Final Version of the Chapters.}
%
% The following forwarding files |cdocsfn1.tex| and |cdocsfn2.tex|
% (with identical content)
% compile the final versions of the child documents
% |cdocsch1.tex| and |cdocsch2.tex|, respectively:
%\iffalse
%<*samplefinal>
%\fi
%    \begin{macrocode}
\def\version{final}
\input{childdoc.def}
\childdocforwardprefix[cdocsamp]{cdocsfn}{cdocsch}
%    \end{macrocode}

%\iffalse
%</samplefinal>
%\fi
%
% %%%%%%%%%%%%%%%%%%%%%%%%%%%%%%%%%%%%%%
% \paragraph{Command Line Processing.}
%
% The following three command lines generate the output files
% |cdocscld|, |cdocscl1| and |cdocscl2|
% which should be identical to
% |cdocsdrf|, |cdocsch1| and |cdocsfn2|, respectively:
% \begin{center}
% \begin{tabular}{l}
% |latex -jobname cdocscld \|\\
% |  "\def\version{draft}\input{childdoc.def}\childdocforward{cdocsamp}"|\\
% |latex -jobname cdocscl1 \|\\
% |  "\input{childdoc.def}\childdocforward[cdocsamp]{cdocsch1}"|\\
% |latex -jobname cdocscl2 \|\\
% |  "\def\version{final}\input{childdoc.def}\childdocforward{cdocsch2}"|
% \end{tabular}
% \end{center}
% Note that the trailing backslash on each first line
% merely continues the input to the second line
% (for convenient cut ant paste).
% Furthermore, the command |latex| can be replaced by any
% of its alternative versions such as |pdflatex|.
%
% %%%%%%%%%%%%%%%%%%%%%%%%%%%%%%%%%%%%%%%%%%%%%%%%%%%%%%%%%%%%%%%%%%%%%%%%%%%%%%
% %%%%%%%%%%%%%%%%%%%%%%%%%%%%%%%%%%%%%%%%%%%%%%%%%%%%%%%%%%%%%%%%%%%%%%%%%%%%%%
% \section{Implementation}
%\iffalse
%<*package>
%\fi
%
% This section describes the definitions file |childdoc.def|.

% The definitions cannot be loaded using |\usepackage| or |\RequirePackage|
% which has a mechanism to prevent loading a style file more than once.
% When loading the definitions by means of |\input|
% multiple instances have to be prevented manually:
%\iffalse
%This code needs to be before the `\ProvidesFile' directive
%which is defined at the beginning of this file.
%Therefore it is also placed there and commented out here.
%</package>
%<*discard>
%\fi
%    \begin{macrocode}
\ifdefined\childdocmain\endinput\fi
%    \end{macrocode}
%\iffalse
%</discard>
%<*package>
%\fi
%
% \macro{\ifchilddoc}
% \macro{\ifchilddocmanual}
% The conditional |\ifchilddoc| tells whether a
% child (true) or main (false) document is being compiled.
% The conditional |\ifchilddocmanual| tells whether
% the |\includeonly| mechanism is used (false) or
% the selection of child files must be performed manually (true).
% The definitions initialise to false:
%    \begin{macrocode}
\newif\ifchilddoc
\newif\ifchilddocmanual
%    \end{macrocode}

% \macro{\childdocname}
% \macro{\childdocjob}
% The macro |\childdocname| stores the name of the main document
% to be compiled. The macro |\childdocjob| stores the name of
% the document on which the \LaTeX{} compiler was originally invoked.
% The content of |\jobname| cannot be compared
% to filenames specified in the source due to different catcodes.
% The following code rescans |\jobname|, stores the result
% in |\childdocname| and saves a copy in |\childdocjob|:
%    \begin{macrocode}
\edef\childdocname{\scantokens\expandafter{\jobname\noexpand}}
\let\childdocjob\childdocname
%    \end{macrocode}

% \macro{\childdocdisable}
% The macro |\childdocdisable| prevents the main file
% from being processed more than once.
% At this stage, the main document command |\childdocmain|
% is assumed to be called once again where it should do nothing.
% Any subsequent call to it should prevent
% a secondary processing of the main document
% It overwrites the forwarding commands
% |\childdocof| and |\childdocforward|
% with empty macros to prevent further inclusions of the main document:
%    \begin{macrocode}
\newcommand{\childdocdisable}
{
  \renewcommand{\childdocmain}[1]{\renewcommand{\childdocmain}[1]{\endinput}}
  \renewcommand{\childdocof}[1]{}
  \renewcommand{\childdocby}[2][]{}
  \renewcommand{\childdocforward}[2][]{}
  \renewcommand{\childdocdisable}{}
}
%    \end{macrocode}

% \macro{\childdocmain}
% The macro |\childdocmain| is to be called at the top of the main file
% with nothing or the main filename (without extension) as argument.
% First, it breaks loops.
% If the argument is not empty and does not match |\childdocname|
% (which is set by the first inclusion of |childdoc.def|),
% |\ifchilddoc| is set to true, |\includeonly| is applied to the child file
% and |\jobname| is set to the main file
% (for proper handling of |.aux| files):
%    \begin{macrocode}
\newcommand{\childdocmain}[1]
{
  \childdocdisable\childdocmain{}
  \if?#1?\else
    \begingroup
      \def\childdoctmp{#1}
      \ifx\childdoctmp\childdocname
        \def\childdoctmp{}
      \else
        \def\childdoctmp
        {
          \childdoctrue
          \includeonly{\childdocname}
          \def\childdocjob{#1}
          \def\jobname{#1}
        }
      \fi
      \expandafter
    \endgroup
    \childdoctmp
  \fi
}
%    \end{macrocode}

% \macro{\childdocof}
% The command |\childdocof| redirects
% compilation to the main file |#1|.
%    \begin{macrocode}
\newcommand{\childdocof}[1]
{
  \childdocdisable
  \childdoctrue
  \includeonly{\childdocname}
  \def\jobname{#1}
  \def\childdocjob{#1}
  \input{#1}
}
%    \end{macrocode}

% \macro{\childdocby}
% The command |\childdocby| ....
%    \begin{macrocode}
\newcommand{\childdocby}[2][]
{
  \childdocdisable
  \childdoctrue
  \childdocmanualtrue
  \if?#1?\else
    \def\jobname{#2}
  \fi
  \def\childdocjob{#2}
  \input{#2}
  \endinput
}
%    \end{macrocode}

% \macro{\childdocforward}
% The command |\childdocforward| redirects
% compilation to the main file or
% (if the optional argument is given) a child file.
% Parameters are set as if the main file
% or a child file starting with |\childdocof| was compiled.
% Then compilation is handed over to the main file:
%    \begin{macrocode}
\newcommand{\childdocforward}[2][]
{
  \begingroup
    \if?#1?
      \def\childdoctmp
      {
        \def\childdocname{#2}
        \def\childdocjob{#2}
        \def\jobname{#2}
        \input{#2}
        \endinput
      }
    \else
      \def\childdoctmp
      {
        \childdocdisable
        \def\childdocname{#2}
        \childdoctrue
        \includeonly{#2}
        \def\childdocjob{#1}
        \def\jobname{#1}
        \input{#1}
        \endinput
      }
    \fi
    \expandafter
  \endgroup
  \childdoctmp
}
%    \end{macrocode}

% \macro{\childdocforwardprefix}
% The command |\childdocforwardprefix| redirects
% compilation to the main or a child file by means of a pattern.
% The prefix |#1| in the current filename is replaced by |#2|
% and the suffix of the current filename is kept
% (it is assumed that the filename does not contain the substring `|~~~|'
% which is used as a delimiter).
% Compilation is handed over to the new file by |\childdocforward|:
%    \begin{macrocode}
\newcommand{\childdocforwardprefix}[3][]
{
  \begingroup
    \def\childdocextract #2##1~~~{\def\childdoctmp{\childdocforward[#1]{#3##1}}}
    \expandafter\childdocextract\childdocname~~~
    \expandafter
  \endgroup
  \childdoctmp
}
%    \end{macrocode}

% \macro{\childdoc}
% The deprecated macro |\childdoc| is a legacy version of |\childdocmain|:
%    \begin{macrocode}
\newcommand{\childdoc}{\childdocmain}
%    \end{macrocode}

% \macro{\childdocredirect}
% The deprecated macro |\childdocredirect| is a legacy version
% of |\childdocforward| and |\childdocforwardprefix|:
%    \begin{macrocode}
\newcommand{\childdocredirect}[2][]
{
  \begingroup
    \if?#1?
      \def\childdoctmp{\childdocforward{#2}}
    \else
      \def\childdoctmp{\childdocforwardprefix{#1}{#2}}
    \fi
    \expandafter
  \endgroup
  \childdoctmp
}
%    \end{macrocode}

%\iffalse
%</package>
%\fi
%
\endinput

\childdocforwardprefix[cdocsamp]{cdocsfn}{cdocsch}
%    \end{macrocode}

%\iffalse
%</samplefinal>
%\fi
%
% %%%%%%%%%%%%%%%%%%%%%%%%%%%%%%%%%%%%%%
% \paragraph{Command Line Processing.}
%
% The following three command lines generate the output files
% |cdocscld|, |cdocscl1| and |cdocscl2|
% which should be identical to
% |cdocsdrf|, |cdocsch1| and |cdocsfn2|, respectively:
% \begin{center}
% \begin{tabular}{l}
% |latex -jobname cdocscld \|\\
% |  "\def\version{draft}% \iffalse
%
% childdoc.dtx Copyright (C) 2017-2018 Niklas Beisert
%
% This work may be distributed and/or modified under the
% conditions of the LaTeX Project Public License, either version 1.3
% of this license or (at your option) any later version.
% The latest version of this license is in
%   http://www.latex-project.org/lppl.txt
% and version 1.3 or later is part of all distributions of LaTeX
% version 2005/12/01 or later.
%
% This work has the LPPL maintenance status `maintained'.
%
% The Current Maintainer of this work is Niklas Beisert.
%
% This work consists of the files childdoc.dtx and childdoc.ins
% and the derived files childdoc.def and cdocsamp.tex with
% cdocsch1.tex, cdocsch2.tex, cdocsdrf.tex, cdocsfn1.tex, cdocsfn2.tex.
%
%<package>\ifdefined\childdocmain\endinput\fi
%<package>\ProvidesFile{childdoc.def}[2018/12/30 v2.0 child document driver]
%<samplemain>\ProvidesFile{cdocsamp.tex}[2018/12/30 v2.0 sample for childdoc]
%<*driver>
%\ProvidesFile{childdoc.drv}[2018/12/30 v2.0 childdoc reference manual file]
\PassOptionsToClass{10pt,a4paper}{article}
\documentclass{ltxdoc}

\usepackage[margin=35mm]{geometry}
\usepackage{hyperref}
\usepackage{hyperxmp}
\usepackage[usenames]{color}

\hypersetup{colorlinks=true}
\hypersetup{pdfstartview=FitH}
\hypersetup{pdfpagemode=UseNone}
\hypersetup{pdfsource={}}
\hypersetup{pdflang={en-UK}}
\hypersetup{pdfcopyright={Copyright 2017-2018 Niklas Beisert.
  This work may be distributed and/or modified under the
  conditions of the LaTeX Project Public License, either version 1.3
  of this license or (at your option) any later version.}}
\hypersetup{pdflicenseurl={http://www.latex-project.org/lppl.txt}}
\hypersetup{pdfcontactaddress={ETH Zurich, ITP, HIT K,
  Wolfgang-Pauli-Strasse 27}}
\hypersetup{pdfcontactpostcode={8093}}
\hypersetup{pdfcontactcity={Zurich}}
\hypersetup{pdfcontactcountry={Switzerland}}
\hypersetup{pdfcontactemail={nbeisert@itp.phys.ethz.ch}}
\hypersetup{pdfcontacturl={http://people.phys.ethz.ch/\xmptilde nbeisert/}}

\newcommand{\secref}[1]{\hyperref[#1]{section \ref*{#1}}}

\parskip1ex
\parindent0pt
\let\olditemize\itemize
\def\itemize{\olditemize\parskip0pt}

\begin{document}

\title{The \textsf{childdoc} Package}
\hypersetup{pdftitle={The childdoc Package}}
\author{Niklas Beisert\\[2ex]
  Institut f\"ur Theoretische Physik\\
  Eidgen\"ossische Technische Hochschule Z\"urich\\
  Wolfgang-Pauli-Strasse 27, 8093 Z\"urich, Switzerland\\[1ex]
  \href{mailto:nbeisert@itp.phys.ethz.ch}
  {\texttt{nbeisert@itp.phys.ethz.ch}}}
\hypersetup{pdfauthor={Niklas Beisert}}
\hypersetup{pdfsubject={Manual for the LaTeX2e Package childdoc}}
\date{30 December 2018, \textsf{v2.0}}
\maketitle

\begin{abstract}\noindent
\textsf{childdoc} is a \LaTeXe{} package
that enables the direct compilation
of document sections included by |\include|
to individual files.
\end{abstract}

\begingroup
\parskip0ex
\tableofcontents
\endgroup

%%%%%%%%%%%%%%%%%%%%%%%%%%%%%%%%%%%%%%%%%%%%%%%%%%%%%%%%%%%%%%%%%%%%%%%%%%%%%%%%
%%%%%%%%%%%%%%%%%%%%%%%%%%%%%%%%%%%%%%%%%%%%%%%%%%%%%%%%%%%%%%%%%%%%%%%%%%%%%%%%
\section{Introduction}

\LaTeX{} provides a mechanism to structure a large document (such as a book)
into a main file and several child files (containing the chapters)
using the |\include| command.
This mechanism is beneficial for documents
which span hundreds of pages in order to
make the source file(s) more manageable.
Moreover, compilation can be restricted to
selected child files by means of the |\includeonly| command.
The latter feature can be used to reduce the compilation time while editing
(this was significantly more useful in the earlier days of \LaTeX{})
or to generate a smaller document which is easier to navigate.
Another application of |\includeonly| is to generate
documents consisting of selected parts of the complete document.

However, there are a few drawbacks of the plain |\include| mechanism:
\begin{itemize}
\item
The child files cannot be compiled on their own,
they can only be compiled via the main file.
A naive editing environment
(such as a text editor with an option
to have the current file processed by \LaTeX)
may require one to switch to the main file before compiling;
attempting to compile the child file produces errors.
\item
The main file must be modified (each time)
to adjust the |\includeonly| command
to the present needs. This easily leaves the main file in a messy state.
\item
The generated document will always carry the filename
of the main document. This is inconvenient if
several child files are to be compiled and
to be kept for distribution.
\end{itemize}

The present package provides a simple interface
to make child files individually compilable by \LaTeX{}.
Compiling a child file then has the same effect as compiling
the main file with an |\includeonly| command
to select the appropriate child.
Moreover the generated document will carry the name of the child
rather than the main file.
This resolves all three above issues.

This feature is meant to make the editing of books,
thesis documents and lecture notes somewhat more convenient.
However, the package can also be used efficiently for
composing a series of documents (such as exercise sheets)
which are typically distributed individually.
It then assists the author in generating the individual documents
(potentially in different versions)
as well as a document containing the collected series.
Another application is in developing style files
or other kinds of included material
where compilation of the style file could redirect
to a sample or test file.

%%%%%%%%%%%%%%%%%%%%%%%%%%%%%%%%%%%%%%%%%%%%%%%%%%%%%%%%%%%%%%%%%%%%%%%%%%%%%%%%
%%%%%%%%%%%%%%%%%%%%%%%%%%%%%%%%%%%%%%%%%%%%%%%%%%%%%%%%%%%%%%%%%%%%%%%%%%%%%%%%
\section{Usage}

First of all, the package \textsf{childdoc} is \emph{not} a standard
\LaTeXe{} |.sty| style file! Therefore it needs to be invoked in
a non-standard way.

%%%%%%%%%%%%%%%%%%%%%%%%%%%%%%%%%%%%%%%%%%%%%%%%%%%%%%%%%%%%%%%%%%%%%%%%%%%%%%%%
\subsection{Included Files}
\label{sec:include}

%%%%%%%%%%%%%%%%%%%%%%%%%%%%%%%%%%%%%%%%
\DescribeMacro{\childdocmain}
To use the package, add the commands
\begin{center}
\begin{tabular}{l}
|\input{childdoc.def}|\\
|\childdocmain{}|\\
\end{tabular}
\end{center}
at the very top of the main \LaTeX{} file,
in particular \emph{before} the |\documentclass| statement!
The argument of |\childdocmain| should be left empty
(but it must be present).

%%%%%%%%%%%%%%%%%%%%%%%%%%%%%%%%%%%%%%%%
\DescribeMacro{\childdocof}
Furthermore, add the commands
\begin{center}
\begin{tabular}{l}
|\input{childdoc.def}|\\
|\childdocof{|\textit{main}|}|\\
\end{tabular}
\end{center}
at the top of every child file \textit{child}
which is included by |\include{|\textit{child}|}|
from within the main file
(or at least for those files to be compiled individually).
The argument \textit{main} must be the filename of the main file.

There are a couple of
considerations in setting up the main and child documents:

%%%%%%%%%%%%%%%%%%%%%%%%%%%%%%%%%%%%%%%%
\paragraph{Restrictions.}

Please note the following restrictions:
\begin{itemize}
\item
|\childdocmain| must be called with one argument \textit{main}
to ensure compatibility with earlier version of the package.
It must either be empty (|\childdocmain{}|)
or precisely match the filename of the main file in which it is specified.
See \secref{sec:detection} for further information.
\item
The filename \textit{main} must be specified without the |.tex| extension.
\item
The filename \textit{main} is case sensitive
(even in case-insensitive file systems)
due to internal string comparison.
\item
The argument \textit{main} should be fully expanded, it cannot be a macro.
\item
Subdirectories and special characters should be avoided in filenames.
\item
The command |\childdocmain{|\textit{main}|}| must be followed by a whitespace.
It should not be followed immediately by another command
or by a comment mark `|%|'.
This is because the \TeX{} parser reads the token immediately following
the argument of |\childdocmain| and puts it
at the beginning of every child section;
however, a white\-space is ignored.
\end{itemize}

%%%%%%%%%%%%%%%%%%%%%%%%%%%%%%%%%%%%%%%%
\paragraph{Content of Main File.}

It is advisable to place all content in the child files included by |\include|.
Any output contained in the main file will appear in all child documents
unless suppressed manually;
it cannot be suppressed automatically by the |\includeonly| directive
and thus should normally be avoided.
A method to include some content in the main file
by means of conditional processing is described in \secref{sec:conditional}.

%%%%%%%%%%%%%%%%%%%%%%%%%%%%%%%%%%%%%%%%
\paragraph{Page Numbering.}

When only a part of the document is compiled,
the appropriate numbering of pages
(as well as other status parameters)
is determined from the |.aux| files.
The latter contain information from previous passes.
However this information needs to propagate through
all intermediate child documents.
Therefore the page numbering in child documents may well
be inconsistent until the complete document is compiled at least once.

A useful (if unconventional) way to always ensure a consistent
page numbering is to restart the numbering in each child document
and denote the pages by `\textit{child}|.|\textit{page}'
where \textit{child} represents the chapter/section number of the child file.
This can be achieved by the command
|\numberwithin{page}{|\textit{child}|}|
of the \textsf{amsmath} package
where \textit{child} can be |chapter| or |section|
depending on the chosen structuring.
Alternatively, one can modify the macro |\thepage| appropriately
and reset the counter |page| at the start of each child file.

%%%%%%%%%%%%%%%%%%%%%%%%%%%%%%%%%%%%%%%%%%%%%%%%%%%%%%%%%%%%%%%%%%%%%%%%%%%%%%%%
\subsection{Conditional Processing}
\label{sec:conditional}

The package provides a mechanism to compile different versions
of a document. To customise the versions further some conditional processing
can come in handy to distinguish which version is being compiled.
The package provides two macros to describe the compilation context:

%%%%%%%%%%%%%%%%%%%%%%%%%%%%%%%%%%%%%%%%
\DescribeMacro{\ifchilddoc}
The conditional |\ifchilddoc| distinguishes between the compilation of
child documents and the main document:
%
\begin{center}
|\ifchilddoc |\textit{child-code}| |[|\||else |\textit{main-code}]| \||fi|
\end{center}

%%%%%%%%%%%%%%%%%%%%%%%%%%%%%%%%%%%%%%%%
\DescribeMacro{\childdocname}
\DescribeMacro{\childdocjob}
The macro |\childdocname| contains the filename (without extension)
of the main or child file being processed.
Note that |\childdocjob| will always contain the name of the main file.

%%%%%%%%%%%%%%%%%%%%%%%%%%%%%%%%%%%%%%%%
\paragraph{Title Page.}

Conditional processing can be used to include a title or banner page
in the main document when proper precautions are taken.
Importantly, the code in the main file should ensure that the page counter
(as well as other status parameters which are stored in the |.aux| files)
takes the same value after the conditional processing.
Otherwise the page numbers may take divergent values
depending on which part is compiled.

For example, a title page could be declared by:
%
\begin{center}
\begin{tabular}{l}
|\ifchilddoc\||else|\\
|\addtocounter{page}{-1}|\\
\textit{code for title page}\\
|\newpage|\\
|\||fi|
\end{tabular}
\end{center}
%
A banner page for the child documents can be generated by:
%
\begin{center}
\begin{tabular}{l}
|\ifchilddoc|\\
|\addtocounter{page}{-1}|\\
\textit{code for banner page}\\
|\newpage|\\
|\||fi|
\end{tabular}
\end{center}
%
Here one could write a message such as:
\begin{center}
|This is the part \childdocname{} of \childdocjob{}.|
\end{center}

%%%%%%%%%%%%%%%%%%%%%%%%%%%%%%%%%%%%%%%%%%%%%%%%%%%%%%%%%%%%%%%%%%%%%%%%%%%%%%%%
\subsection{Flags}
\label{sec:flags}

The package makes it easy to generate different versions
of the main or child documents.
To this end compilation flags can be defined
and assigned different default values.
They will be particularly useful in conjunction
with the forwarding mechanism described in \secref{sec:forward}.

For example, it may be useful to have a flag |\version|
which can be set to |draft| or |final|.
The document source will contain some conditional code
depending on the value of |\version|.
Suppose further, the flag should default to |final| for the main file
and to |draft| for child files
which is a natural assignment for editing the document.
This is achieved by placing the following code
in the preamble of the main document
(below the |\childdocmain| directive):
%
\begin{center}
\begin{tabular}{l}
|\ifchilddoc|\\
|\providecommand{\version}{draft}|\\
|\||else|\\
|\providecommand{\version}{final}|\\
|\||fi|
\end{tabular}
\end{center}
%
The definition by |\providecommand| makes sure
that previous definitions are not overwritten.
Further statements |\providecommand{\version}{...}|
can thus be added before the above code to override it.

For the main file, one might add a line
(between |\childdocmain| and the above block)
%
\begin{center}
|%\ifchilddoc\||else\providecommand{\version}{draft}\||fi|
\end{center}
%
which can be uncommented to produce a draft version.
Likewise one can add a line to the very top of a child file
(above the |\childdocof{|\textit{main}|}| directive)
%
\begin{center}
|%\providecommand{\version}{final}|
\end{center}
%
which can be uncommented to produce the final version of this child document.

%%%%%%%%%%%%%%%%%%%%%%%%%%%%%%%%%%%%%%%%%%%%%%%%%%%%%%%%%%%%%%%%%%%%%%%%%%%%%%%%
\subsection{Forwarding}
\label{sec:forward}

Different versions of the main or child documents
using compilation flags as described in \secref{sec:flags}
can be (permanently) stored in different files
for convenient compilation, viewing and distribution.
To this end, the package defines a command
to pass on compilation to a different file:

%%%%%%%%%%%%%%%%%%%%%%%%%%%%%%%%%%%%%%%%
\DescribeMacro{\childdocforward}
The command |\childdocforward| redirects processing to
another source file:
%
\begin{center}
\begin{tabular}{l}
|\input{childdoc.def}|\\
|\childdocforward[|\textit{main}|]{|\textit{dest}|}|\\
\end{tabular}
\end{center}
%
The argument \textit{dest} is the destination file
(without extension).
It should be the main file or one of the child files.
Note that further \textsf{childdoc} directives
such as |\childdocof| and |\childdocforward|
in the indicated file will be processed in this form.
The optional argument \textit{main}
passes on directly to the main file \textit{main}
while pretending to compile the child \textit{dest}.
This form behaves as if \textit{dest}
issues |\childdocof{|\textit{main}|}| right away,
and no further \textsf{childdoc} directives will be processed.

%%%%%%%%%%%%%%%%%%%%%%%%%%%%%%%%%%%%%%%%
\DescribeMacro{\...prefix}
In the alternative form |\childdocforwardprefix|,
%
\begin{center}
\begin{tabular}{l}
|\input{childdoc.def}|\\
|\childdocforwardprefix[|\textit{main}|]{|\textit{prefix}|}{|\textit{dest}|}|
\end{tabular}
\end{center}
%
the destination file is determined by a pattern
depending on the current file:
To make this work, the current file must be called
`{\textit{prefix}\hspace{0.2em}\textit{suffix}}'
with \textit{prefix} matching precisely the argument.
Processing is then passed on to the file
`{\textit{dest}\hspace{0.2em}\textit{suffix}}'.
Surely, the same effect is achieved by
directly specifying the
argument `{\textit{dest}\hspace{0.2em}\textit{suffix}}'
in the first form.
However, that requires to set up a different file
for each child. With the alternative form of the command
all these files can have exactly the same content
which simplifies setting them up and maintaining them.

For example, the following file |draft.tex|
with a compilation flag |\version| as described in \secref{sec:flags}
compiles the main document as a draft:
%
\begin{center}
\begin{tabular}{l}
|\def\version{draft}|\\
|\input{childdoc.def}|\\
|\childdocforward{|\textit{main}|}|
\end{tabular}
\end{center}
%
Likewise, the following files |final|\textit{nn}|.tex|
compile the final version of the child document
|child|\textit{nn}|.tex|:
%
\begin{center}
\begin{tabular}{l}
|\def\version{final}|\\
|\input{childdoc.def}|\\
|\childdocforwardprefix{final}{child}|
\end{tabular}
\end{center}
%

Note that when several versions of a main file and/or of each child file
are to be generated, it may be convenient to set up a |Makefile| or
shell script to automatise the process.

%%%%%%%%%%%%%%%%%%%%%%%%%%%%%%%%%%%%%%%%%%%%%%%%%%%%%%%%%%%%%%%%%%%%%%%%%%%%%%%%
\subsection{Command Line Processing}
\label{sec:commandline}

The effect of redirection files can also be achieved by invoking
the \LaTeX{} compiler with a more elaborate command line.
Most conveniently this should be done as part
of a shell script or a |Makefile|.

When using \textsf{childdoc} in the main file, the following
command lines effectively perform a redirection
(note that depending on the shell being used,
backslashes may have to be doubled: `|\|' $\to$ `|\\|'):
%
\begin{center}
|... -jobname "|\textit{target}|" |\\|"|[\textit{flags}]%
|\input{childdoc.def}\childdocforward[|\textit{main}|]{|\textit{dest}|}"|
\end{center}
%
Here \textit{target} is the name of the output file,
\textit{main} is the name of the main file
and \textit{dest} is the name of the main or child file to be processed
(all filenames without extensions).
The optional argument \textit{main} can be omitted
if \textit{main} matches \textit{dest}.
Optionally, compilation \textit{flags} can be defined via |\def| commands.
This command line makes the \TeX{} engine believe
it is compiling the file \textit{target}
whose content is specified as the latter parameter.
The provided code then forwards the processing to
\textit{main} or \textit{dest} as described in \secref{sec:forward}.

%%%%%%%%%%%%%%%%%%%%%%%%%%%%%%%%%%%%%%%%%%%%%%%%%%%%%%%%%%%%%%%%%%%%%%%%%%%%%%%%
\subsection{Include by Input}
\label{sec:input}

Including child documents by |\include| has some restrictions by design.
Most notably, the content of a child document always occupies
its own set of pages; pages cannot be shared between child documents.
Usually, this behaviour makes perfect sense
because each child document contain an essential part of the document.
However, in some situations it may be desirable to compose
a document from a collection of parts
without having mandatory page breaks between then.
For this case, the package
provides a mechanism to include parts
by |\input| which can also be processed individually.
However, by construction this mechanism
requires manual handling of the content to be output.

%%%%%%%%%%%%%%%%%%%%%%%%%%%%%%%%%%%%%%%%
\DescribeMacro{\ifchilddocmanual}
The main file should be prepared as usual, see \secref{sec:include}.
However, the document body must make a distinction
between processing of an individual part and of the main document, e.g.:
%
\begin{center}
\begin{tabular}{l}
|\ifchilddocmanual|\\
|\input{\childdocname}|\\
|\||else|\\
\textit{document body with }|\input{|\textit{part}|}|\\
|\||fi|
\end{tabular}
\end{center}
%
The conditional |\ifchilddocmanual| is true whenever
a part to be included by |\input| is being compiled,
and the name of the part is stored in |\childdocname|.

%%%%%%%%%%%%%%%%%%%%%%%%%%%%%%%%%%%%%%%%
\DescribeMacro{\childdocby}
Each part to be included by |\input| should start with:
%
\begin{center}
\begin{tabular}{l}
|\input{childdoc.def}|\\
|\childdocby{|\textit{main}|}|\\
\end{tabular}
\end{center}
%
The directive |\childdocby| is similar to |\childdocof|
described in \secref{sec:include},
but the subsequent selection of content must be done manually.
To that end, both |\ifchilddoc| and |\ifchilddocmanual|
will be true upon processing of a part,
and the name of the part is stored in |\childdocname|.
Note that |\jobname| will be set to the filename of the current part
so that each part receives an individual |.aux| file
that does not interfere with the |.aux| file(s) of the main document.
This behaviour can be altered by the alternative form
|\childdocby[*]{|\textit{main}|}| (with a non-empty optional argument)
which uses the |.aux| file of the main document
by setting |\jobname| to \textit{main}.

%%%%%%%%%%%%%%%%%%%%%%%%%%%%%%%%%%%%%%%%%%%%%%%%%%%%%%%%%%%%%%%%%%%%%%%%%%%%%%%%
\subsection{Driver Development}
\label{sec:driver}

The \textsf{childdoc} mechanism can also be use for the development
of definition files such as \LaTeX{} styles or classes.
This case differs from the above setup with multiple parts
included by |\include| in that no |\includeonly| should be invoked.
This can be achieved by starting the include file
(before |\ProvidesPackage|) with:
%
\begin{center}
\begin{tabular}{l}
|\input{childdoc.def}|\\
|\childdocforward{|\textit{main}|}|\\
\end{tabular}
\end{center}
%
or alternatively with:
%
\begin{center}
\begin{tabular}{l}
|\input{childdoc.def}|\\
|\childdocby{|\textit{main}|}|\\
\end{tabular}
\end{center}
%
Both forms have slightly different effects as described above.
The main file is prepared as usual, see \secref{sec:include}.

%%%%%%%%%%%%%%%%%%%%%%%%%%%%%%%%%%%%%%%%%%%%%%%%%%%%%%%%%%%%%%%%%%%%%%%%%%%%%%%%
\subsection{Legacy Detection}
\label{sec:detection}

The directive |\childdocmain| in the main file can detect
whether the complete document or merely a child is to be compiled
even without using the directive |\childdocof|.
This method is deprecated because it is less robust
and there is no compelling reason to use it;
it is merely provided for backward compatibility
and it may be removed in future versions.

If the detection mechanism is to be used,
it is mandatory to correctly specify
the filename of the main file as the argument of |\childdocmain|:
%
\begin{center}
\begin{tabular}{l}
|\input{childdoc.def}|\\
|\childdocmain{|\textit{main}|}|\\
\end{tabular}
\end{center}
%
If |\jobname| does not match the argument \textit{main} of |\childdocmain|,
it is assumed that |\jobname| points to the child file to be compiled.
When using |\childdocmain| with the main file specified as argument,
it suffices to start a child file
with just |\input{|\textit{main}|}|
without loading of the package and using |\childdocof|.
If instead all processing is done
with the appropriate \textsf{childdoc} directives,
the argument of \textit{main} of |\childdocmain| can be empty.

An alternative version of the command line processing described
in \secref{sec:commandline} using the detection mechanism reads:
%
\begin{center}
|... -jobname "|\textit{target}|" "|[\textit{flags}]%
[|\def\jobname{|\textit{dest}|}|]|\input{|\textit{main}|}"|
\end{center}

%%%%%%%%%%%%%%%%%%%%%%%%%%%%%%%%%%%%%%%%%%%%%%%%%%%%%%%%%%%%%%%%%%%%%%%%%%%%%%%%
\subsection{Manual Code}
\label{sec:manual}

In case one cannot be certain whether the definitions file |childdoc.def|
is installed on the target \TeX{} distribution
and one prefers not to ship it,
it is conceivable to paste a few relevant commands into the sources.

To that end, drop all statements |\input{childdoc.def}|
and perform the replacements as outlined below.
Instead of |\childdocmain{|\textit{main}|}| add the following code
to the top of the main file:
%
\begin{center}
\begin{tabular}{l}
|\||ifdefined\childdocname\endinput\||fi\newif\ifchilddoc|\\
|\edef\childdocname{\scantokens\expandafter{\jobname\noexpand}}|\\
|\def\childdocmain{|\textit{main}|}\||ifx\childdocmain\childdocname\||else|\\
|\childdoctrue\includeonly{\childdocname}\let\jobname\childdocmain\||fi|\\
\end{tabular}
\end{center}
%
Instead of |\childdocof{|\textit{main}|}| just include the main file
at the top of each child file:
%
\begin{center}
|\input{|\textit{main}|}|
\end{center}
%
A simple redirection |\childdocforward{|\textit{dest}|}| is achieved by:
%
\begin{center}
|\def\jobname{|\textit{dest}|}\input{\jobname}|
\end{center}
%
The redirection with prefix
|\childdocforwardprefix[|\textit{prefix}|]{|\textit{dest}|}|
is accomplished by:
%
\begin{center}
\begin{tabular}{l}
|{\edef\jobname{\scantokens\expandafter{\jobname\noexpand}}|\\
|\def\redirectjob |\textit{prefix}|#1~~~{\gdef\jobname{|\textit{dest}|#1}}|\\
|\expandafter\redirectjob\jobname~~~}\input{\jobname}|
\end{tabular}
\end{center}

In an alternative approach,
child documents can be compiled by a specific command line
without additional code or specific definitions:
%
\begin{center}
|... -jobname "|\textit{target}|" "|[\textit{flags}]%
|\includeonly{|\textit{dest}|}\input{|\textit{main}|}"|
\end{center}
%

%%%%%%%%%%%%%%%%%%%%%%%%%%%%%%%%%%%%%%%%%%%%%%%%%%%%%%%%%%%%%%%%%%%%%%%%%%%%%%%%
%%%%%%%%%%%%%%%%%%%%%%%%%%%%%%%%%%%%%%%%%%%%%%%%%%%%%%%%%%%%%%%%%%%%%%%%%%%%%%%%
\section{Information}

%%%%%%%%%%%%%%%%%%%%%%%%%%%%%%%%%%%%%%%%%%%%%%%%%%%%%%%%%%%%%%%%%%%%%%%%%%%%%%%%
\subsection{Copyright}

Copyright \copyright{} 2017--2018 Niklas Beisert

This work may be distributed and/or modified under the
conditions of the \LaTeX{} Project Public License, either version 1.3
of this license or (at your option) any later version.
The latest version of this license is in
  \url{http://www.latex-project.org/lppl.txt}
and version 1.3 or later is part of all distributions of \LaTeX{}
version 2005/12/01 or later.

This work has the LPPL maintenance status `maintained'.

The Current Maintainer of this work is Niklas Beisert.

This work consists of the files |README.txt|, |childdoc.ins| and |childdoc.dtx|
as well as the derived files |childdoc.def|, |cdocsamp.tex|
with |cdocsch1.tex|, |cdocsch2.tex|, |cdocspt3.tex|, |cdocspt4.tex|,
|cdocsdrf.tex|, |cdocsfn1.tex|, |cdocsfn2.tex|
as well as |childdoc.pdf|.

%%%%%%%%%%%%%%%%%%%%%%%%%%%%%%%%%%%%%%%%%%%%%%%%%%%%%%%%%%%%%%%%%%%%%%%%%%%%%%%%
\subsection{Files and Installation}

The package consists of the files:
%
\begin{center}
\begin{tabular}{ll}
    |README.txt|   & readme file \\
    |childdoc.ins| & installation file \\
    |childdoc.dtx| & source file \\
    |childdoc.def| & definition file \\
    |cdocsamp.tex| & sample main file \\
    |cdocsch1.tex| & sample include file \\
    |cdocsch2.tex| & sample include file \\
    |cdocspt3.tex| & sample part file \\
    |cdocspt4.tex| & sample part file \\
    |cdocsdrf.tex| & sample redirection file \\
    |cdocsfn1.tex| & sample redirection file \\
    |cdocsfn2.tex| & sample redirection file \\
    |childdoc.pdf| & manual
\end{tabular}
\end{center}
%
The distribution consists of the files
|README.txt|, |childdoc.ins| and |childdoc.dtx|.
%
\begin{itemize}
\item
Run (pdf)\LaTeX{} on |childdoc.dtx|
to compile the manual |childdoc.pdf| (this file).
\item
Run \LaTeX{} on |childdoc.ins| to create the definitions file |childdoc.def|
and the sample |cdocsamp.tex| with include files
|cdocsch1.tex|, |cdocsch2.tex|, |cdocspt3.tex|, |cdocspt4.tex|,
|cdocsdrf.tex|, |cdocsfn1.tex|, |cdocsfn2.tex|.
Then copy the file |childdoc.def| to an appropriate directory of your \LaTeX{}
distribution, e.g.\ \textit{texmf-root}|/tex/latex/childdoc|.
\end{itemize}

%%%%%%%%%%%%%%%%%%%%%%%%%%%%%%%%%%%%%%%%%%%%%%%%%%%%%%%%%%%%%%%%%%%%%%%%%%%%%%%%
\subsection{Related CTAN Packages}

There are several other packages which offer a similar functionality:
%
\begin{itemize}
\item
The packages
\href{http://ctan.org/pkg/docmute}{\textsf{docmute}},
\href{http://ctan.org/pkg/includex}{\textsf{includex}} and
\href{http://ctan.org/pkg/standalone}{\textsf{standalone}}
provide commands to include only the document body of
a child file thus allowing both files to be compiled individually.
\item
The packages \href{http://ctan.org/pkg/subdocs}{\textsf{subdocs}}
and \href{http://ctan.org/pkg/subfiles}{\textsf{subfiles}}
provide structures in which the main and child documents can be
encapsulated and allowing them to be compiled individually.
The inclusion mechanism is different from the conventional |\include|.
\item
The package \href{http://ctan.org/pkg/combine}{\textsf{combine}}
is an elaborate solution to combine several documents into one.
\end{itemize}
%
See also the CTAN topic \href{http://ctan.org/topic/subdocs}{\textsf{subdocs}}
for further related packages.
The present package differs from the above solutions in that
a document structure constructed with the conventional |\include| mechanism
just needs two extra commands at the top of every file
such that all constituent files can be compiled individually.

%%%%%%%%%%%%%%%%%%%%%%%%%%%%%%%%%%%%%%%%%%%%%%%%%%%%%%%%%%%%%%%%%%%%%%%%%%%%%%%%
%\subsection{Feature Suggestions}
%
%The following is a list of features which may be useful for future
%versions of this package:
%%
%\begin{itemize}
%\item
%\ldots
%\end{itemize}

%%%%%%%%%%%%%%%%%%%%%%%%%%%%%%%%%%%%%%%%%%%%%%%%%%%%%%%%%%%%%%%%%%%%%%%%%%%%%%%%
\subsection{Revision History}

%%%%%%%%%%%%%%%%%%%%%%%%%%%%%%%%%%%%%%%%
\paragraph{v2.0:} 2018/12/30

\begin{itemize}
\item
immediate forward processing
\item
added |\childdocby| mechanism
\item
manual restructured
\end{itemize}

%%%%%%%%%%%%%%%%%%%%%%%%%%%%%%%%%%%%%%%%
\paragraph{v1.6:} 2018/01/17

\begin{itemize}
\item
application for development of include files
\item
corrections to manual
\end{itemize}

%%%%%%%%%%%%%%%%%%%%%%%%%%%%%%%%%%%%%%%%
\paragraph{v1.5:} 2017/05/21

\begin{itemize}
\item
more complete structuring introduced
\item
|\childdocof| introduced
\item
|\childdoc| renamed to |\childdocmain|
\item
|\childredirect| renamed to |\childdocforward| and |\childdocforwardprefix|
and functionality expanded
\end{itemize}

%%%%%%%%%%%%%%%%%%%%%%%%%%%%%%%%%%%%%%%%
\paragraph{v1.0:} 2017/04/27

\begin{itemize}
\item
manual and install package
\item
first version published on CTAN
\end{itemize}

%%%%%%%%%%%%%%%%%%%%%%%%%%%%%%%%%%%%%%%%
\paragraph{v0.6:} 2017/04/26

\begin{itemize}
\item
redirection mechanism added
\end{itemize}

%%%%%%%%%%%%%%%%%%%%%%%%%%%%%%%%%%%%%%%%
\paragraph{v0.5:} 2017/04/26

\begin{itemize}
\item
functionality in definition file
\end{itemize}


%%%%%%%%%%%%%%%%%%%%%%%%%%%%%%%%%%%%%%%%%%%%%%%%%%%%%%%%%%%%%%%%%%%%%%%%%%%%%%%%
%%%%%%%%%%%%%%%%%%%%%%%%%%%%%%%%%%%%%%%%%%%%%%%%%%%%%%%%%%%%%%%%%%%%%%%%%%%%%%%%
%%%%%%%%%%%%%%%%%%%%%%%%%%%%%%%%%%%%%%%%%%%%%%%%%%%%%%%%%%%%%%%%%%%%%%%%%%%%%%%%
\appendix

\settowidth\MacroIndent{\rmfamily\scriptsize 000\ }

 \DocInput{childdoc.dtx}

\end{document}
%</driver>
% \fi
%
% %%%%%%%%%%%%%%%%%%%%%%%%%%%%%%%%%%%%%%%%%%%%%%%%%%%%%%%%%%%%%%%%%%%%%%%%%%%%%%
% %%%%%%%%%%%%%%%%%%%%%%%%%%%%%%%%%%%%%%%%%%%%%%%%%%%%%%%%%%%%%%%%%%%%%%%%%%%%%%
% \section{Sample}
%\iffalse
%<*samplemain>
%\fi
%
% The following presents a sample document
% with two chapters, two parts, a title page,
% a compile flag as well as three forwarding files to set the flag.
% It consists of eight |.tex| files:
% \begin{center}
% \begin{tabular}{ll}
% |cdocsamp.tex|&main file\\
% |cdocsch1.tex|&include file for chapter 1\\
% |cdocsch2.tex|&include file for chapter 2\\
% |cdocspt3.tex|&include file for part 3\\
% |cdocspt4.tex|&include file for part 4\\
% |cdocsdrf.tex|&forwarding file for main file in draft mode\\
% |cdocsfi1.tex|&forwarding file for final version of chapter 1\\
% |cdocsfi2.tex|&forwarding file for final version of chapter 2\\
% \end{tabular}
% \end{center}
% Each of the eight files can be compiled directly by the \LaTeX{} compiler.
%
% %%%%%%%%%%%%%%%%%%%%%%%%%%%%%%%%%%%%%%
% \paragraph{Main File.}
%
% The main file is called |cdocsamp.tex|.
%
% Load the \textsf{childdoc} definitions and
% declare the filename for the main document:
%    \begin{macrocode}
\input{childdoc.def}
\childdocmain{}
%    \end{macrocode}

% Optional override for |\version| flag:
%    \begin{macrocode}
%%\ifchilddoc\else\providecommand{\version}{draft}\fi
%    \end{macrocode}

% Define the default values for the |\version| flag
% (|final| for the main file and |draft| for childs):
%    \begin{macrocode}
\ifchilddoc
\providecommand{\version}{draft}
\else
\providecommand{\version}{final}
\fi
%    \end{macrocode}

% Load the standard document class:
%    \begin{macrocode}
\documentclass[12pt]{article}
%    \end{macrocode}

% Start the document body:
%    \begin{macrocode}
\begin{document}
%    \end{macrocode}

% Declare a title page.
% Print title, part of document being processed and version flag:
%    \begin{macrocode}
\addtocounter{page}{-1}
\begin{center}
{\LARGE\bfseries{}childdoc example\par}
\vspace{1cm}
\ifchilddoc
\ifchilddocmanual part\else chapter\fi:
`\childdocname' of `\childdocjob'\par
\else
main document: `\childdocjob'\par
\fi
version: \version\par
\end{center}
\newpage
%    \end{macrocode}

% Manually include selected file,
% otherwise process as usual:
%    \begin{macrocode}
\ifchilddocmanual
\section*{part `\childdocname'}
\input{\childdocname}
\else
%    \end{macrocode}

% Include the two chapters:
%    \begin{macrocode}
\include{cdocsch1}
\include{cdocsch2}
%    \end{macrocode}

% Include the two parts unless only chapters should be displayed:
%    \begin{macrocode}
\ifchilddoc\else
\section{part three}
\input{cdocspt3}
\section{part four}
\input{cdocspt4}
\fi
%    \end{macrocode}

% Process as usual until here:
%    \begin{macrocode}
\fi
%    \end{macrocode}

% End of document body:
%    \begin{macrocode}
\end{document}
%    \end{macrocode}
%\iffalse
%</samplemain>
%\fi
%
% %%%%%%%%%%%%%%%%%%%%%%%%%%%%%%%%%%%%%%
% \paragraph{Chapter Include Files.}
%
% The include files are called |cdocsch1.tex| and |cdocsch2.tex|.
%
%\iffalse
%<*samplechap1|samplechap2>
%\fi

% Optional override for |\version| flag:
%    \begin{macrocode}
%%\providecommand{\version}{final}
%    \end{macrocode}

% Include the main document:
%    \begin{macrocode}
\input{childdoc.def}
\childdocof{cdocsamp}
%    \end{macrocode}

%\iffalse
%</samplechap1|samplechap2>
%\fi
%
%\iffalse
%<*samplechap1>
%\fi
% Some text for chapter 1:
%    \begin{macrocode}
\section{one}
some text in chapter one
%    \end{macrocode}

%\iffalse
%</samplechap1>
%\fi
% Some text for chapter 2:
%\iffalse
%<*samplechap2>
%\fi
%    \begin{macrocode}
\section{two}
more text in chapter two
%    \end{macrocode}

%\iffalse
%</samplechap2>
%\fi
%
% %%%%%%%%%%%%%%%%%%%%%%%%%%%%%%%%%%%%%%
% \paragraph{Part Include Files.}
%
% The include files are called |cdocspt3.tex| and |cdocspt4.tex|.
%
%\iffalse
%<*samplepart3|samplepart4>
%\fi

% Optional override for |\version| flag:
%    \begin{macrocode}
%%\providecommand{\version}{final}
%    \end{macrocode}

% Include the main document:
%    \begin{macrocode}
\input{childdoc.def}
\childdocby{cdocsamp}
%    \end{macrocode}

%\iffalse
%</samplepart3|samplepart4>
%\fi
%
%\iffalse
%<*samplepart3>
%\fi
% Some text for part 3:
%    \begin{macrocode}
some text in part three
%    \end{macrocode}

%\iffalse
%</samplepart3>
%\fi
% Some text for part 4:
%\iffalse
%<*samplepart4>
%\fi
%    \begin{macrocode}
more text in part four
%    \end{macrocode}

%\iffalse
%</samplepart4>
%\fi
%
% %%%%%%%%%%%%%%%%%%%%%%%%%%%%%%%%%%%%%%
% \paragraph{Forwarding for a Complete Draft.}
%
% The following forwarding file |cdocsdrf.tex|
% compiles the main document in draft mode:
%\iffalse
%<*sampledraft>
%\fi
%    \begin{macrocode}
\def\version{draft}
\input{childdoc.def}
\childdocforward{cdocsamp}
%    \end{macrocode}

%\iffalse
%</sampledraft>
%\fi
%
% %%%%%%%%%%%%%%%%%%%%%%%%%%%%%%%%%%%%%%
% \paragraph{Forwarding for Final Version of the Chapters.}
%
% The following forwarding files |cdocsfn1.tex| and |cdocsfn2.tex|
% (with identical content)
% compile the final versions of the child documents
% |cdocsch1.tex| and |cdocsch2.tex|, respectively:
%\iffalse
%<*samplefinal>
%\fi
%    \begin{macrocode}
\def\version{final}
\input{childdoc.def}
\childdocforwardprefix[cdocsamp]{cdocsfn}{cdocsch}
%    \end{macrocode}

%\iffalse
%</samplefinal>
%\fi
%
% %%%%%%%%%%%%%%%%%%%%%%%%%%%%%%%%%%%%%%
% \paragraph{Command Line Processing.}
%
% The following three command lines generate the output files
% |cdocscld|, |cdocscl1| and |cdocscl2|
% which should be identical to
% |cdocsdrf|, |cdocsch1| and |cdocsfn2|, respectively:
% \begin{center}
% \begin{tabular}{l}
% |latex -jobname cdocscld \|\\
% |  "\def\version{draft}\input{childdoc.def}\childdocforward{cdocsamp}"|\\
% |latex -jobname cdocscl1 \|\\
% |  "\input{childdoc.def}\childdocforward[cdocsamp]{cdocsch1}"|\\
% |latex -jobname cdocscl2 \|\\
% |  "\def\version{final}\input{childdoc.def}\childdocforward{cdocsch2}"|
% \end{tabular}
% \end{center}
% Note that the trailing backslash on each first line
% merely continues the input to the second line
% (for convenient cut ant paste).
% Furthermore, the command |latex| can be replaced by any
% of its alternative versions such as |pdflatex|.
%
% %%%%%%%%%%%%%%%%%%%%%%%%%%%%%%%%%%%%%%%%%%%%%%%%%%%%%%%%%%%%%%%%%%%%%%%%%%%%%%
% %%%%%%%%%%%%%%%%%%%%%%%%%%%%%%%%%%%%%%%%%%%%%%%%%%%%%%%%%%%%%%%%%%%%%%%%%%%%%%
% \section{Implementation}
%\iffalse
%<*package>
%\fi
%
% This section describes the definitions file |childdoc.def|.

% The definitions cannot be loaded using |\usepackage| or |\RequirePackage|
% which has a mechanism to prevent loading a style file more than once.
% When loading the definitions by means of |\input|
% multiple instances have to be prevented manually:
%\iffalse
%This code needs to be before the `\ProvidesFile' directive
%which is defined at the beginning of this file.
%Therefore it is also placed there and commented out here.
%</package>
%<*discard>
%\fi
%    \begin{macrocode}
\ifdefined\childdocmain\endinput\fi
%    \end{macrocode}
%\iffalse
%</discard>
%<*package>
%\fi
%
% \macro{\ifchilddoc}
% \macro{\ifchilddocmanual}
% The conditional |\ifchilddoc| tells whether a
% child (true) or main (false) document is being compiled.
% The conditional |\ifchilddocmanual| tells whether
% the |\includeonly| mechanism is used (false) or
% the selection of child files must be performed manually (true).
% The definitions initialise to false:
%    \begin{macrocode}
\newif\ifchilddoc
\newif\ifchilddocmanual
%    \end{macrocode}

% \macro{\childdocname}
% \macro{\childdocjob}
% The macro |\childdocname| stores the name of the main document
% to be compiled. The macro |\childdocjob| stores the name of
% the document on which the \LaTeX{} compiler was originally invoked.
% The content of |\jobname| cannot be compared
% to filenames specified in the source due to different catcodes.
% The following code rescans |\jobname|, stores the result
% in |\childdocname| and saves a copy in |\childdocjob|:
%    \begin{macrocode}
\edef\childdocname{\scantokens\expandafter{\jobname\noexpand}}
\let\childdocjob\childdocname
%    \end{macrocode}

% \macro{\childdocdisable}
% The macro |\childdocdisable| prevents the main file
% from being processed more than once.
% At this stage, the main document command |\childdocmain|
% is assumed to be called once again where it should do nothing.
% Any subsequent call to it should prevent
% a secondary processing of the main document
% It overwrites the forwarding commands
% |\childdocof| and |\childdocforward|
% with empty macros to prevent further inclusions of the main document:
%    \begin{macrocode}
\newcommand{\childdocdisable}
{
  \renewcommand{\childdocmain}[1]{\renewcommand{\childdocmain}[1]{\endinput}}
  \renewcommand{\childdocof}[1]{}
  \renewcommand{\childdocby}[2][]{}
  \renewcommand{\childdocforward}[2][]{}
  \renewcommand{\childdocdisable}{}
}
%    \end{macrocode}

% \macro{\childdocmain}
% The macro |\childdocmain| is to be called at the top of the main file
% with nothing or the main filename (without extension) as argument.
% First, it breaks loops.
% If the argument is not empty and does not match |\childdocname|
% (which is set by the first inclusion of |childdoc.def|),
% |\ifchilddoc| is set to true, |\includeonly| is applied to the child file
% and |\jobname| is set to the main file
% (for proper handling of |.aux| files):
%    \begin{macrocode}
\newcommand{\childdocmain}[1]
{
  \childdocdisable\childdocmain{}
  \if?#1?\else
    \begingroup
      \def\childdoctmp{#1}
      \ifx\childdoctmp\childdocname
        \def\childdoctmp{}
      \else
        \def\childdoctmp
        {
          \childdoctrue
          \includeonly{\childdocname}
          \def\childdocjob{#1}
          \def\jobname{#1}
        }
      \fi
      \expandafter
    \endgroup
    \childdoctmp
  \fi
}
%    \end{macrocode}

% \macro{\childdocof}
% The command |\childdocof| redirects
% compilation to the main file |#1|.
%    \begin{macrocode}
\newcommand{\childdocof}[1]
{
  \childdocdisable
  \childdoctrue
  \includeonly{\childdocname}
  \def\jobname{#1}
  \def\childdocjob{#1}
  \input{#1}
}
%    \end{macrocode}

% \macro{\childdocby}
% The command |\childdocby| ....
%    \begin{macrocode}
\newcommand{\childdocby}[2][]
{
  \childdocdisable
  \childdoctrue
  \childdocmanualtrue
  \if?#1?\else
    \def\jobname{#2}
  \fi
  \def\childdocjob{#2}
  \input{#2}
  \endinput
}
%    \end{macrocode}

% \macro{\childdocforward}
% The command |\childdocforward| redirects
% compilation to the main file or
% (if the optional argument is given) a child file.
% Parameters are set as if the main file
% or a child file starting with |\childdocof| was compiled.
% Then compilation is handed over to the main file:
%    \begin{macrocode}
\newcommand{\childdocforward}[2][]
{
  \begingroup
    \if?#1?
      \def\childdoctmp
      {
        \def\childdocname{#2}
        \def\childdocjob{#2}
        \def\jobname{#2}
        \input{#2}
        \endinput
      }
    \else
      \def\childdoctmp
      {
        \childdocdisable
        \def\childdocname{#2}
        \childdoctrue
        \includeonly{#2}
        \def\childdocjob{#1}
        \def\jobname{#1}
        \input{#1}
        \endinput
      }
    \fi
    \expandafter
  \endgroup
  \childdoctmp
}
%    \end{macrocode}

% \macro{\childdocforwardprefix}
% The command |\childdocforwardprefix| redirects
% compilation to the main or a child file by means of a pattern.
% The prefix |#1| in the current filename is replaced by |#2|
% and the suffix of the current filename is kept
% (it is assumed that the filename does not contain the substring `|~~~|'
% which is used as a delimiter).
% Compilation is handed over to the new file by |\childdocforward|:
%    \begin{macrocode}
\newcommand{\childdocforwardprefix}[3][]
{
  \begingroup
    \def\childdocextract #2##1~~~{\def\childdoctmp{\childdocforward[#1]{#3##1}}}
    \expandafter\childdocextract\childdocname~~~
    \expandafter
  \endgroup
  \childdoctmp
}
%    \end{macrocode}

% \macro{\childdoc}
% The deprecated macro |\childdoc| is a legacy version of |\childdocmain|:
%    \begin{macrocode}
\newcommand{\childdoc}{\childdocmain}
%    \end{macrocode}

% \macro{\childdocredirect}
% The deprecated macro |\childdocredirect| is a legacy version
% of |\childdocforward| and |\childdocforwardprefix|:
%    \begin{macrocode}
\newcommand{\childdocredirect}[2][]
{
  \begingroup
    \if?#1?
      \def\childdoctmp{\childdocforward{#2}}
    \else
      \def\childdoctmp{\childdocforwardprefix{#1}{#2}}
    \fi
    \expandafter
  \endgroup
  \childdoctmp
}
%    \end{macrocode}

%\iffalse
%</package>
%\fi
%
\endinput
\childdocforward{cdocsamp}"|\\
% |latex -jobname cdocscl1 \|\\
% |  "% \iffalse
%
% childdoc.dtx Copyright (C) 2017-2018 Niklas Beisert
%
% This work may be distributed and/or modified under the
% conditions of the LaTeX Project Public License, either version 1.3
% of this license or (at your option) any later version.
% The latest version of this license is in
%   http://www.latex-project.org/lppl.txt
% and version 1.3 or later is part of all distributions of LaTeX
% version 2005/12/01 or later.
%
% This work has the LPPL maintenance status `maintained'.
%
% The Current Maintainer of this work is Niklas Beisert.
%
% This work consists of the files childdoc.dtx and childdoc.ins
% and the derived files childdoc.def and cdocsamp.tex with
% cdocsch1.tex, cdocsch2.tex, cdocsdrf.tex, cdocsfn1.tex, cdocsfn2.tex.
%
%<package>\ifdefined\childdocmain\endinput\fi
%<package>\ProvidesFile{childdoc.def}[2018/12/30 v2.0 child document driver]
%<samplemain>\ProvidesFile{cdocsamp.tex}[2018/12/30 v2.0 sample for childdoc]
%<*driver>
%\ProvidesFile{childdoc.drv}[2018/12/30 v2.0 childdoc reference manual file]
\PassOptionsToClass{10pt,a4paper}{article}
\documentclass{ltxdoc}

\usepackage[margin=35mm]{geometry}
\usepackage{hyperref}
\usepackage{hyperxmp}
\usepackage[usenames]{color}

\hypersetup{colorlinks=true}
\hypersetup{pdfstartview=FitH}
\hypersetup{pdfpagemode=UseNone}
\hypersetup{pdfsource={}}
\hypersetup{pdflang={en-UK}}
\hypersetup{pdfcopyright={Copyright 2017-2018 Niklas Beisert.
  This work may be distributed and/or modified under the
  conditions of the LaTeX Project Public License, either version 1.3
  of this license or (at your option) any later version.}}
\hypersetup{pdflicenseurl={http://www.latex-project.org/lppl.txt}}
\hypersetup{pdfcontactaddress={ETH Zurich, ITP, HIT K,
  Wolfgang-Pauli-Strasse 27}}
\hypersetup{pdfcontactpostcode={8093}}
\hypersetup{pdfcontactcity={Zurich}}
\hypersetup{pdfcontactcountry={Switzerland}}
\hypersetup{pdfcontactemail={nbeisert@itp.phys.ethz.ch}}
\hypersetup{pdfcontacturl={http://people.phys.ethz.ch/\xmptilde nbeisert/}}

\newcommand{\secref}[1]{\hyperref[#1]{section \ref*{#1}}}

\parskip1ex
\parindent0pt
\let\olditemize\itemize
\def\itemize{\olditemize\parskip0pt}

\begin{document}

\title{The \textsf{childdoc} Package}
\hypersetup{pdftitle={The childdoc Package}}
\author{Niklas Beisert\\[2ex]
  Institut f\"ur Theoretische Physik\\
  Eidgen\"ossische Technische Hochschule Z\"urich\\
  Wolfgang-Pauli-Strasse 27, 8093 Z\"urich, Switzerland\\[1ex]
  \href{mailto:nbeisert@itp.phys.ethz.ch}
  {\texttt{nbeisert@itp.phys.ethz.ch}}}
\hypersetup{pdfauthor={Niklas Beisert}}
\hypersetup{pdfsubject={Manual for the LaTeX2e Package childdoc}}
\date{30 December 2018, \textsf{v2.0}}
\maketitle

\begin{abstract}\noindent
\textsf{childdoc} is a \LaTeXe{} package
that enables the direct compilation
of document sections included by |\include|
to individual files.
\end{abstract}

\begingroup
\parskip0ex
\tableofcontents
\endgroup

%%%%%%%%%%%%%%%%%%%%%%%%%%%%%%%%%%%%%%%%%%%%%%%%%%%%%%%%%%%%%%%%%%%%%%%%%%%%%%%%
%%%%%%%%%%%%%%%%%%%%%%%%%%%%%%%%%%%%%%%%%%%%%%%%%%%%%%%%%%%%%%%%%%%%%%%%%%%%%%%%
\section{Introduction}

\LaTeX{} provides a mechanism to structure a large document (such as a book)
into a main file and several child files (containing the chapters)
using the |\include| command.
This mechanism is beneficial for documents
which span hundreds of pages in order to
make the source file(s) more manageable.
Moreover, compilation can be restricted to
selected child files by means of the |\includeonly| command.
The latter feature can be used to reduce the compilation time while editing
(this was significantly more useful in the earlier days of \LaTeX{})
or to generate a smaller document which is easier to navigate.
Another application of |\includeonly| is to generate
documents consisting of selected parts of the complete document.

However, there are a few drawbacks of the plain |\include| mechanism:
\begin{itemize}
\item
The child files cannot be compiled on their own,
they can only be compiled via the main file.
A naive editing environment
(such as a text editor with an option
to have the current file processed by \LaTeX)
may require one to switch to the main file before compiling;
attempting to compile the child file produces errors.
\item
The main file must be modified (each time)
to adjust the |\includeonly| command
to the present needs. This easily leaves the main file in a messy state.
\item
The generated document will always carry the filename
of the main document. This is inconvenient if
several child files are to be compiled and
to be kept for distribution.
\end{itemize}

The present package provides a simple interface
to make child files individually compilable by \LaTeX{}.
Compiling a child file then has the same effect as compiling
the main file with an |\includeonly| command
to select the appropriate child.
Moreover the generated document will carry the name of the child
rather than the main file.
This resolves all three above issues.

This feature is meant to make the editing of books,
thesis documents and lecture notes somewhat more convenient.
However, the package can also be used efficiently for
composing a series of documents (such as exercise sheets)
which are typically distributed individually.
It then assists the author in generating the individual documents
(potentially in different versions)
as well as a document containing the collected series.
Another application is in developing style files
or other kinds of included material
where compilation of the style file could redirect
to a sample or test file.

%%%%%%%%%%%%%%%%%%%%%%%%%%%%%%%%%%%%%%%%%%%%%%%%%%%%%%%%%%%%%%%%%%%%%%%%%%%%%%%%
%%%%%%%%%%%%%%%%%%%%%%%%%%%%%%%%%%%%%%%%%%%%%%%%%%%%%%%%%%%%%%%%%%%%%%%%%%%%%%%%
\section{Usage}

First of all, the package \textsf{childdoc} is \emph{not} a standard
\LaTeXe{} |.sty| style file! Therefore it needs to be invoked in
a non-standard way.

%%%%%%%%%%%%%%%%%%%%%%%%%%%%%%%%%%%%%%%%%%%%%%%%%%%%%%%%%%%%%%%%%%%%%%%%%%%%%%%%
\subsection{Included Files}
\label{sec:include}

%%%%%%%%%%%%%%%%%%%%%%%%%%%%%%%%%%%%%%%%
\DescribeMacro{\childdocmain}
To use the package, add the commands
\begin{center}
\begin{tabular}{l}
|\input{childdoc.def}|\\
|\childdocmain{}|\\
\end{tabular}
\end{center}
at the very top of the main \LaTeX{} file,
in particular \emph{before} the |\documentclass| statement!
The argument of |\childdocmain| should be left empty
(but it must be present).

%%%%%%%%%%%%%%%%%%%%%%%%%%%%%%%%%%%%%%%%
\DescribeMacro{\childdocof}
Furthermore, add the commands
\begin{center}
\begin{tabular}{l}
|\input{childdoc.def}|\\
|\childdocof{|\textit{main}|}|\\
\end{tabular}
\end{center}
at the top of every child file \textit{child}
which is included by |\include{|\textit{child}|}|
from within the main file
(or at least for those files to be compiled individually).
The argument \textit{main} must be the filename of the main file.

There are a couple of
considerations in setting up the main and child documents:

%%%%%%%%%%%%%%%%%%%%%%%%%%%%%%%%%%%%%%%%
\paragraph{Restrictions.}

Please note the following restrictions:
\begin{itemize}
\item
|\childdocmain| must be called with one argument \textit{main}
to ensure compatibility with earlier version of the package.
It must either be empty (|\childdocmain{}|)
or precisely match the filename of the main file in which it is specified.
See \secref{sec:detection} for further information.
\item
The filename \textit{main} must be specified without the |.tex| extension.
\item
The filename \textit{main} is case sensitive
(even in case-insensitive file systems)
due to internal string comparison.
\item
The argument \textit{main} should be fully expanded, it cannot be a macro.
\item
Subdirectories and special characters should be avoided in filenames.
\item
The command |\childdocmain{|\textit{main}|}| must be followed by a whitespace.
It should not be followed immediately by another command
or by a comment mark `|%|'.
This is because the \TeX{} parser reads the token immediately following
the argument of |\childdocmain| and puts it
at the beginning of every child section;
however, a white\-space is ignored.
\end{itemize}

%%%%%%%%%%%%%%%%%%%%%%%%%%%%%%%%%%%%%%%%
\paragraph{Content of Main File.}

It is advisable to place all content in the child files included by |\include|.
Any output contained in the main file will appear in all child documents
unless suppressed manually;
it cannot be suppressed automatically by the |\includeonly| directive
and thus should normally be avoided.
A method to include some content in the main file
by means of conditional processing is described in \secref{sec:conditional}.

%%%%%%%%%%%%%%%%%%%%%%%%%%%%%%%%%%%%%%%%
\paragraph{Page Numbering.}

When only a part of the document is compiled,
the appropriate numbering of pages
(as well as other status parameters)
is determined from the |.aux| files.
The latter contain information from previous passes.
However this information needs to propagate through
all intermediate child documents.
Therefore the page numbering in child documents may well
be inconsistent until the complete document is compiled at least once.

A useful (if unconventional) way to always ensure a consistent
page numbering is to restart the numbering in each child document
and denote the pages by `\textit{child}|.|\textit{page}'
where \textit{child} represents the chapter/section number of the child file.
This can be achieved by the command
|\numberwithin{page}{|\textit{child}|}|
of the \textsf{amsmath} package
where \textit{child} can be |chapter| or |section|
depending on the chosen structuring.
Alternatively, one can modify the macro |\thepage| appropriately
and reset the counter |page| at the start of each child file.

%%%%%%%%%%%%%%%%%%%%%%%%%%%%%%%%%%%%%%%%%%%%%%%%%%%%%%%%%%%%%%%%%%%%%%%%%%%%%%%%
\subsection{Conditional Processing}
\label{sec:conditional}

The package provides a mechanism to compile different versions
of a document. To customise the versions further some conditional processing
can come in handy to distinguish which version is being compiled.
The package provides two macros to describe the compilation context:

%%%%%%%%%%%%%%%%%%%%%%%%%%%%%%%%%%%%%%%%
\DescribeMacro{\ifchilddoc}
The conditional |\ifchilddoc| distinguishes between the compilation of
child documents and the main document:
%
\begin{center}
|\ifchilddoc |\textit{child-code}| |[|\||else |\textit{main-code}]| \||fi|
\end{center}

%%%%%%%%%%%%%%%%%%%%%%%%%%%%%%%%%%%%%%%%
\DescribeMacro{\childdocname}
\DescribeMacro{\childdocjob}
The macro |\childdocname| contains the filename (without extension)
of the main or child file being processed.
Note that |\childdocjob| will always contain the name of the main file.

%%%%%%%%%%%%%%%%%%%%%%%%%%%%%%%%%%%%%%%%
\paragraph{Title Page.}

Conditional processing can be used to include a title or banner page
in the main document when proper precautions are taken.
Importantly, the code in the main file should ensure that the page counter
(as well as other status parameters which are stored in the |.aux| files)
takes the same value after the conditional processing.
Otherwise the page numbers may take divergent values
depending on which part is compiled.

For example, a title page could be declared by:
%
\begin{center}
\begin{tabular}{l}
|\ifchilddoc\||else|\\
|\addtocounter{page}{-1}|\\
\textit{code for title page}\\
|\newpage|\\
|\||fi|
\end{tabular}
\end{center}
%
A banner page for the child documents can be generated by:
%
\begin{center}
\begin{tabular}{l}
|\ifchilddoc|\\
|\addtocounter{page}{-1}|\\
\textit{code for banner page}\\
|\newpage|\\
|\||fi|
\end{tabular}
\end{center}
%
Here one could write a message such as:
\begin{center}
|This is the part \childdocname{} of \childdocjob{}.|
\end{center}

%%%%%%%%%%%%%%%%%%%%%%%%%%%%%%%%%%%%%%%%%%%%%%%%%%%%%%%%%%%%%%%%%%%%%%%%%%%%%%%%
\subsection{Flags}
\label{sec:flags}

The package makes it easy to generate different versions
of the main or child documents.
To this end compilation flags can be defined
and assigned different default values.
They will be particularly useful in conjunction
with the forwarding mechanism described in \secref{sec:forward}.

For example, it may be useful to have a flag |\version|
which can be set to |draft| or |final|.
The document source will contain some conditional code
depending on the value of |\version|.
Suppose further, the flag should default to |final| for the main file
and to |draft| for child files
which is a natural assignment for editing the document.
This is achieved by placing the following code
in the preamble of the main document
(below the |\childdocmain| directive):
%
\begin{center}
\begin{tabular}{l}
|\ifchilddoc|\\
|\providecommand{\version}{draft}|\\
|\||else|\\
|\providecommand{\version}{final}|\\
|\||fi|
\end{tabular}
\end{center}
%
The definition by |\providecommand| makes sure
that previous definitions are not overwritten.
Further statements |\providecommand{\version}{...}|
can thus be added before the above code to override it.

For the main file, one might add a line
(between |\childdocmain| and the above block)
%
\begin{center}
|%\ifchilddoc\||else\providecommand{\version}{draft}\||fi|
\end{center}
%
which can be uncommented to produce a draft version.
Likewise one can add a line to the very top of a child file
(above the |\childdocof{|\textit{main}|}| directive)
%
\begin{center}
|%\providecommand{\version}{final}|
\end{center}
%
which can be uncommented to produce the final version of this child document.

%%%%%%%%%%%%%%%%%%%%%%%%%%%%%%%%%%%%%%%%%%%%%%%%%%%%%%%%%%%%%%%%%%%%%%%%%%%%%%%%
\subsection{Forwarding}
\label{sec:forward}

Different versions of the main or child documents
using compilation flags as described in \secref{sec:flags}
can be (permanently) stored in different files
for convenient compilation, viewing and distribution.
To this end, the package defines a command
to pass on compilation to a different file:

%%%%%%%%%%%%%%%%%%%%%%%%%%%%%%%%%%%%%%%%
\DescribeMacro{\childdocforward}
The command |\childdocforward| redirects processing to
another source file:
%
\begin{center}
\begin{tabular}{l}
|\input{childdoc.def}|\\
|\childdocforward[|\textit{main}|]{|\textit{dest}|}|\\
\end{tabular}
\end{center}
%
The argument \textit{dest} is the destination file
(without extension).
It should be the main file or one of the child files.
Note that further \textsf{childdoc} directives
such as |\childdocof| and |\childdocforward|
in the indicated file will be processed in this form.
The optional argument \textit{main}
passes on directly to the main file \textit{main}
while pretending to compile the child \textit{dest}.
This form behaves as if \textit{dest}
issues |\childdocof{|\textit{main}|}| right away,
and no further \textsf{childdoc} directives will be processed.

%%%%%%%%%%%%%%%%%%%%%%%%%%%%%%%%%%%%%%%%
\DescribeMacro{\...prefix}
In the alternative form |\childdocforwardprefix|,
%
\begin{center}
\begin{tabular}{l}
|\input{childdoc.def}|\\
|\childdocforwardprefix[|\textit{main}|]{|\textit{prefix}|}{|\textit{dest}|}|
\end{tabular}
\end{center}
%
the destination file is determined by a pattern
depending on the current file:
To make this work, the current file must be called
`{\textit{prefix}\hspace{0.2em}\textit{suffix}}'
with \textit{prefix} matching precisely the argument.
Processing is then passed on to the file
`{\textit{dest}\hspace{0.2em}\textit{suffix}}'.
Surely, the same effect is achieved by
directly specifying the
argument `{\textit{dest}\hspace{0.2em}\textit{suffix}}'
in the first form.
However, that requires to set up a different file
for each child. With the alternative form of the command
all these files can have exactly the same content
which simplifies setting them up and maintaining them.

For example, the following file |draft.tex|
with a compilation flag |\version| as described in \secref{sec:flags}
compiles the main document as a draft:
%
\begin{center}
\begin{tabular}{l}
|\def\version{draft}|\\
|\input{childdoc.def}|\\
|\childdocforward{|\textit{main}|}|
\end{tabular}
\end{center}
%
Likewise, the following files |final|\textit{nn}|.tex|
compile the final version of the child document
|child|\textit{nn}|.tex|:
%
\begin{center}
\begin{tabular}{l}
|\def\version{final}|\\
|\input{childdoc.def}|\\
|\childdocforwardprefix{final}{child}|
\end{tabular}
\end{center}
%

Note that when several versions of a main file and/or of each child file
are to be generated, it may be convenient to set up a |Makefile| or
shell script to automatise the process.

%%%%%%%%%%%%%%%%%%%%%%%%%%%%%%%%%%%%%%%%%%%%%%%%%%%%%%%%%%%%%%%%%%%%%%%%%%%%%%%%
\subsection{Command Line Processing}
\label{sec:commandline}

The effect of redirection files can also be achieved by invoking
the \LaTeX{} compiler with a more elaborate command line.
Most conveniently this should be done as part
of a shell script or a |Makefile|.

When using \textsf{childdoc} in the main file, the following
command lines effectively perform a redirection
(note that depending on the shell being used,
backslashes may have to be doubled: `|\|' $\to$ `|\\|'):
%
\begin{center}
|... -jobname "|\textit{target}|" |\\|"|[\textit{flags}]%
|\input{childdoc.def}\childdocforward[|\textit{main}|]{|\textit{dest}|}"|
\end{center}
%
Here \textit{target} is the name of the output file,
\textit{main} is the name of the main file
and \textit{dest} is the name of the main or child file to be processed
(all filenames without extensions).
The optional argument \textit{main} can be omitted
if \textit{main} matches \textit{dest}.
Optionally, compilation \textit{flags} can be defined via |\def| commands.
This command line makes the \TeX{} engine believe
it is compiling the file \textit{target}
whose content is specified as the latter parameter.
The provided code then forwards the processing to
\textit{main} or \textit{dest} as described in \secref{sec:forward}.

%%%%%%%%%%%%%%%%%%%%%%%%%%%%%%%%%%%%%%%%%%%%%%%%%%%%%%%%%%%%%%%%%%%%%%%%%%%%%%%%
\subsection{Include by Input}
\label{sec:input}

Including child documents by |\include| has some restrictions by design.
Most notably, the content of a child document always occupies
its own set of pages; pages cannot be shared between child documents.
Usually, this behaviour makes perfect sense
because each child document contain an essential part of the document.
However, in some situations it may be desirable to compose
a document from a collection of parts
without having mandatory page breaks between then.
For this case, the package
provides a mechanism to include parts
by |\input| which can also be processed individually.
However, by construction this mechanism
requires manual handling of the content to be output.

%%%%%%%%%%%%%%%%%%%%%%%%%%%%%%%%%%%%%%%%
\DescribeMacro{\ifchilddocmanual}
The main file should be prepared as usual, see \secref{sec:include}.
However, the document body must make a distinction
between processing of an individual part and of the main document, e.g.:
%
\begin{center}
\begin{tabular}{l}
|\ifchilddocmanual|\\
|\input{\childdocname}|\\
|\||else|\\
\textit{document body with }|\input{|\textit{part}|}|\\
|\||fi|
\end{tabular}
\end{center}
%
The conditional |\ifchilddocmanual| is true whenever
a part to be included by |\input| is being compiled,
and the name of the part is stored in |\childdocname|.

%%%%%%%%%%%%%%%%%%%%%%%%%%%%%%%%%%%%%%%%
\DescribeMacro{\childdocby}
Each part to be included by |\input| should start with:
%
\begin{center}
\begin{tabular}{l}
|\input{childdoc.def}|\\
|\childdocby{|\textit{main}|}|\\
\end{tabular}
\end{center}
%
The directive |\childdocby| is similar to |\childdocof|
described in \secref{sec:include},
but the subsequent selection of content must be done manually.
To that end, both |\ifchilddoc| and |\ifchilddocmanual|
will be true upon processing of a part,
and the name of the part is stored in |\childdocname|.
Note that |\jobname| will be set to the filename of the current part
so that each part receives an individual |.aux| file
that does not interfere with the |.aux| file(s) of the main document.
This behaviour can be altered by the alternative form
|\childdocby[*]{|\textit{main}|}| (with a non-empty optional argument)
which uses the |.aux| file of the main document
by setting |\jobname| to \textit{main}.

%%%%%%%%%%%%%%%%%%%%%%%%%%%%%%%%%%%%%%%%%%%%%%%%%%%%%%%%%%%%%%%%%%%%%%%%%%%%%%%%
\subsection{Driver Development}
\label{sec:driver}

The \textsf{childdoc} mechanism can also be use for the development
of definition files such as \LaTeX{} styles or classes.
This case differs from the above setup with multiple parts
included by |\include| in that no |\includeonly| should be invoked.
This can be achieved by starting the include file
(before |\ProvidesPackage|) with:
%
\begin{center}
\begin{tabular}{l}
|\input{childdoc.def}|\\
|\childdocforward{|\textit{main}|}|\\
\end{tabular}
\end{center}
%
or alternatively with:
%
\begin{center}
\begin{tabular}{l}
|\input{childdoc.def}|\\
|\childdocby{|\textit{main}|}|\\
\end{tabular}
\end{center}
%
Both forms have slightly different effects as described above.
The main file is prepared as usual, see \secref{sec:include}.

%%%%%%%%%%%%%%%%%%%%%%%%%%%%%%%%%%%%%%%%%%%%%%%%%%%%%%%%%%%%%%%%%%%%%%%%%%%%%%%%
\subsection{Legacy Detection}
\label{sec:detection}

The directive |\childdocmain| in the main file can detect
whether the complete document or merely a child is to be compiled
even without using the directive |\childdocof|.
This method is deprecated because it is less robust
and there is no compelling reason to use it;
it is merely provided for backward compatibility
and it may be removed in future versions.

If the detection mechanism is to be used,
it is mandatory to correctly specify
the filename of the main file as the argument of |\childdocmain|:
%
\begin{center}
\begin{tabular}{l}
|\input{childdoc.def}|\\
|\childdocmain{|\textit{main}|}|\\
\end{tabular}
\end{center}
%
If |\jobname| does not match the argument \textit{main} of |\childdocmain|,
it is assumed that |\jobname| points to the child file to be compiled.
When using |\childdocmain| with the main file specified as argument,
it suffices to start a child file
with just |\input{|\textit{main}|}|
without loading of the package and using |\childdocof|.
If instead all processing is done
with the appropriate \textsf{childdoc} directives,
the argument of \textit{main} of |\childdocmain| can be empty.

An alternative version of the command line processing described
in \secref{sec:commandline} using the detection mechanism reads:
%
\begin{center}
|... -jobname "|\textit{target}|" "|[\textit{flags}]%
[|\def\jobname{|\textit{dest}|}|]|\input{|\textit{main}|}"|
\end{center}

%%%%%%%%%%%%%%%%%%%%%%%%%%%%%%%%%%%%%%%%%%%%%%%%%%%%%%%%%%%%%%%%%%%%%%%%%%%%%%%%
\subsection{Manual Code}
\label{sec:manual}

In case one cannot be certain whether the definitions file |childdoc.def|
is installed on the target \TeX{} distribution
and one prefers not to ship it,
it is conceivable to paste a few relevant commands into the sources.

To that end, drop all statements |\input{childdoc.def}|
and perform the replacements as outlined below.
Instead of |\childdocmain{|\textit{main}|}| add the following code
to the top of the main file:
%
\begin{center}
\begin{tabular}{l}
|\||ifdefined\childdocname\endinput\||fi\newif\ifchilddoc|\\
|\edef\childdocname{\scantokens\expandafter{\jobname\noexpand}}|\\
|\def\childdocmain{|\textit{main}|}\||ifx\childdocmain\childdocname\||else|\\
|\childdoctrue\includeonly{\childdocname}\let\jobname\childdocmain\||fi|\\
\end{tabular}
\end{center}
%
Instead of |\childdocof{|\textit{main}|}| just include the main file
at the top of each child file:
%
\begin{center}
|\input{|\textit{main}|}|
\end{center}
%
A simple redirection |\childdocforward{|\textit{dest}|}| is achieved by:
%
\begin{center}
|\def\jobname{|\textit{dest}|}\input{\jobname}|
\end{center}
%
The redirection with prefix
|\childdocforwardprefix[|\textit{prefix}|]{|\textit{dest}|}|
is accomplished by:
%
\begin{center}
\begin{tabular}{l}
|{\edef\jobname{\scantokens\expandafter{\jobname\noexpand}}|\\
|\def\redirectjob |\textit{prefix}|#1~~~{\gdef\jobname{|\textit{dest}|#1}}|\\
|\expandafter\redirectjob\jobname~~~}\input{\jobname}|
\end{tabular}
\end{center}

In an alternative approach,
child documents can be compiled by a specific command line
without additional code or specific definitions:
%
\begin{center}
|... -jobname "|\textit{target}|" "|[\textit{flags}]%
|\includeonly{|\textit{dest}|}\input{|\textit{main}|}"|
\end{center}
%

%%%%%%%%%%%%%%%%%%%%%%%%%%%%%%%%%%%%%%%%%%%%%%%%%%%%%%%%%%%%%%%%%%%%%%%%%%%%%%%%
%%%%%%%%%%%%%%%%%%%%%%%%%%%%%%%%%%%%%%%%%%%%%%%%%%%%%%%%%%%%%%%%%%%%%%%%%%%%%%%%
\section{Information}

%%%%%%%%%%%%%%%%%%%%%%%%%%%%%%%%%%%%%%%%%%%%%%%%%%%%%%%%%%%%%%%%%%%%%%%%%%%%%%%%
\subsection{Copyright}

Copyright \copyright{} 2017--2018 Niklas Beisert

This work may be distributed and/or modified under the
conditions of the \LaTeX{} Project Public License, either version 1.3
of this license or (at your option) any later version.
The latest version of this license is in
  \url{http://www.latex-project.org/lppl.txt}
and version 1.3 or later is part of all distributions of \LaTeX{}
version 2005/12/01 or later.

This work has the LPPL maintenance status `maintained'.

The Current Maintainer of this work is Niklas Beisert.

This work consists of the files |README.txt|, |childdoc.ins| and |childdoc.dtx|
as well as the derived files |childdoc.def|, |cdocsamp.tex|
with |cdocsch1.tex|, |cdocsch2.tex|, |cdocspt3.tex|, |cdocspt4.tex|,
|cdocsdrf.tex|, |cdocsfn1.tex|, |cdocsfn2.tex|
as well as |childdoc.pdf|.

%%%%%%%%%%%%%%%%%%%%%%%%%%%%%%%%%%%%%%%%%%%%%%%%%%%%%%%%%%%%%%%%%%%%%%%%%%%%%%%%
\subsection{Files and Installation}

The package consists of the files:
%
\begin{center}
\begin{tabular}{ll}
    |README.txt|   & readme file \\
    |childdoc.ins| & installation file \\
    |childdoc.dtx| & source file \\
    |childdoc.def| & definition file \\
    |cdocsamp.tex| & sample main file \\
    |cdocsch1.tex| & sample include file \\
    |cdocsch2.tex| & sample include file \\
    |cdocspt3.tex| & sample part file \\
    |cdocspt4.tex| & sample part file \\
    |cdocsdrf.tex| & sample redirection file \\
    |cdocsfn1.tex| & sample redirection file \\
    |cdocsfn2.tex| & sample redirection file \\
    |childdoc.pdf| & manual
\end{tabular}
\end{center}
%
The distribution consists of the files
|README.txt|, |childdoc.ins| and |childdoc.dtx|.
%
\begin{itemize}
\item
Run (pdf)\LaTeX{} on |childdoc.dtx|
to compile the manual |childdoc.pdf| (this file).
\item
Run \LaTeX{} on |childdoc.ins| to create the definitions file |childdoc.def|
and the sample |cdocsamp.tex| with include files
|cdocsch1.tex|, |cdocsch2.tex|, |cdocspt3.tex|, |cdocspt4.tex|,
|cdocsdrf.tex|, |cdocsfn1.tex|, |cdocsfn2.tex|.
Then copy the file |childdoc.def| to an appropriate directory of your \LaTeX{}
distribution, e.g.\ \textit{texmf-root}|/tex/latex/childdoc|.
\end{itemize}

%%%%%%%%%%%%%%%%%%%%%%%%%%%%%%%%%%%%%%%%%%%%%%%%%%%%%%%%%%%%%%%%%%%%%%%%%%%%%%%%
\subsection{Related CTAN Packages}

There are several other packages which offer a similar functionality:
%
\begin{itemize}
\item
The packages
\href{http://ctan.org/pkg/docmute}{\textsf{docmute}},
\href{http://ctan.org/pkg/includex}{\textsf{includex}} and
\href{http://ctan.org/pkg/standalone}{\textsf{standalone}}
provide commands to include only the document body of
a child file thus allowing both files to be compiled individually.
\item
The packages \href{http://ctan.org/pkg/subdocs}{\textsf{subdocs}}
and \href{http://ctan.org/pkg/subfiles}{\textsf{subfiles}}
provide structures in which the main and child documents can be
encapsulated and allowing them to be compiled individually.
The inclusion mechanism is different from the conventional |\include|.
\item
The package \href{http://ctan.org/pkg/combine}{\textsf{combine}}
is an elaborate solution to combine several documents into one.
\end{itemize}
%
See also the CTAN topic \href{http://ctan.org/topic/subdocs}{\textsf{subdocs}}
for further related packages.
The present package differs from the above solutions in that
a document structure constructed with the conventional |\include| mechanism
just needs two extra commands at the top of every file
such that all constituent files can be compiled individually.

%%%%%%%%%%%%%%%%%%%%%%%%%%%%%%%%%%%%%%%%%%%%%%%%%%%%%%%%%%%%%%%%%%%%%%%%%%%%%%%%
%\subsection{Feature Suggestions}
%
%The following is a list of features which may be useful for future
%versions of this package:
%%
%\begin{itemize}
%\item
%\ldots
%\end{itemize}

%%%%%%%%%%%%%%%%%%%%%%%%%%%%%%%%%%%%%%%%%%%%%%%%%%%%%%%%%%%%%%%%%%%%%%%%%%%%%%%%
\subsection{Revision History}

%%%%%%%%%%%%%%%%%%%%%%%%%%%%%%%%%%%%%%%%
\paragraph{v2.0:} 2018/12/30

\begin{itemize}
\item
immediate forward processing
\item
added |\childdocby| mechanism
\item
manual restructured
\end{itemize}

%%%%%%%%%%%%%%%%%%%%%%%%%%%%%%%%%%%%%%%%
\paragraph{v1.6:} 2018/01/17

\begin{itemize}
\item
application for development of include files
\item
corrections to manual
\end{itemize}

%%%%%%%%%%%%%%%%%%%%%%%%%%%%%%%%%%%%%%%%
\paragraph{v1.5:} 2017/05/21

\begin{itemize}
\item
more complete structuring introduced
\item
|\childdocof| introduced
\item
|\childdoc| renamed to |\childdocmain|
\item
|\childredirect| renamed to |\childdocforward| and |\childdocforwardprefix|
and functionality expanded
\end{itemize}

%%%%%%%%%%%%%%%%%%%%%%%%%%%%%%%%%%%%%%%%
\paragraph{v1.0:} 2017/04/27

\begin{itemize}
\item
manual and install package
\item
first version published on CTAN
\end{itemize}

%%%%%%%%%%%%%%%%%%%%%%%%%%%%%%%%%%%%%%%%
\paragraph{v0.6:} 2017/04/26

\begin{itemize}
\item
redirection mechanism added
\end{itemize}

%%%%%%%%%%%%%%%%%%%%%%%%%%%%%%%%%%%%%%%%
\paragraph{v0.5:} 2017/04/26

\begin{itemize}
\item
functionality in definition file
\end{itemize}


%%%%%%%%%%%%%%%%%%%%%%%%%%%%%%%%%%%%%%%%%%%%%%%%%%%%%%%%%%%%%%%%%%%%%%%%%%%%%%%%
%%%%%%%%%%%%%%%%%%%%%%%%%%%%%%%%%%%%%%%%%%%%%%%%%%%%%%%%%%%%%%%%%%%%%%%%%%%%%%%%
%%%%%%%%%%%%%%%%%%%%%%%%%%%%%%%%%%%%%%%%%%%%%%%%%%%%%%%%%%%%%%%%%%%%%%%%%%%%%%%%
\appendix

\settowidth\MacroIndent{\rmfamily\scriptsize 000\ }

 \DocInput{childdoc.dtx}

\end{document}
%</driver>
% \fi
%
% %%%%%%%%%%%%%%%%%%%%%%%%%%%%%%%%%%%%%%%%%%%%%%%%%%%%%%%%%%%%%%%%%%%%%%%%%%%%%%
% %%%%%%%%%%%%%%%%%%%%%%%%%%%%%%%%%%%%%%%%%%%%%%%%%%%%%%%%%%%%%%%%%%%%%%%%%%%%%%
% \section{Sample}
%\iffalse
%<*samplemain>
%\fi
%
% The following presents a sample document
% with two chapters, two parts, a title page,
% a compile flag as well as three forwarding files to set the flag.
% It consists of eight |.tex| files:
% \begin{center}
% \begin{tabular}{ll}
% |cdocsamp.tex|&main file\\
% |cdocsch1.tex|&include file for chapter 1\\
% |cdocsch2.tex|&include file for chapter 2\\
% |cdocspt3.tex|&include file for part 3\\
% |cdocspt4.tex|&include file for part 4\\
% |cdocsdrf.tex|&forwarding file for main file in draft mode\\
% |cdocsfi1.tex|&forwarding file for final version of chapter 1\\
% |cdocsfi2.tex|&forwarding file for final version of chapter 2\\
% \end{tabular}
% \end{center}
% Each of the eight files can be compiled directly by the \LaTeX{} compiler.
%
% %%%%%%%%%%%%%%%%%%%%%%%%%%%%%%%%%%%%%%
% \paragraph{Main File.}
%
% The main file is called |cdocsamp.tex|.
%
% Load the \textsf{childdoc} definitions and
% declare the filename for the main document:
%    \begin{macrocode}
\input{childdoc.def}
\childdocmain{}
%    \end{macrocode}

% Optional override for |\version| flag:
%    \begin{macrocode}
%%\ifchilddoc\else\providecommand{\version}{draft}\fi
%    \end{macrocode}

% Define the default values for the |\version| flag
% (|final| for the main file and |draft| for childs):
%    \begin{macrocode}
\ifchilddoc
\providecommand{\version}{draft}
\else
\providecommand{\version}{final}
\fi
%    \end{macrocode}

% Load the standard document class:
%    \begin{macrocode}
\documentclass[12pt]{article}
%    \end{macrocode}

% Start the document body:
%    \begin{macrocode}
\begin{document}
%    \end{macrocode}

% Declare a title page.
% Print title, part of document being processed and version flag:
%    \begin{macrocode}
\addtocounter{page}{-1}
\begin{center}
{\LARGE\bfseries{}childdoc example\par}
\vspace{1cm}
\ifchilddoc
\ifchilddocmanual part\else chapter\fi:
`\childdocname' of `\childdocjob'\par
\else
main document: `\childdocjob'\par
\fi
version: \version\par
\end{center}
\newpage
%    \end{macrocode}

% Manually include selected file,
% otherwise process as usual:
%    \begin{macrocode}
\ifchilddocmanual
\section*{part `\childdocname'}
\input{\childdocname}
\else
%    \end{macrocode}

% Include the two chapters:
%    \begin{macrocode}
\include{cdocsch1}
\include{cdocsch2}
%    \end{macrocode}

% Include the two parts unless only chapters should be displayed:
%    \begin{macrocode}
\ifchilddoc\else
\section{part three}
\input{cdocspt3}
\section{part four}
\input{cdocspt4}
\fi
%    \end{macrocode}

% Process as usual until here:
%    \begin{macrocode}
\fi
%    \end{macrocode}

% End of document body:
%    \begin{macrocode}
\end{document}
%    \end{macrocode}
%\iffalse
%</samplemain>
%\fi
%
% %%%%%%%%%%%%%%%%%%%%%%%%%%%%%%%%%%%%%%
% \paragraph{Chapter Include Files.}
%
% The include files are called |cdocsch1.tex| and |cdocsch2.tex|.
%
%\iffalse
%<*samplechap1|samplechap2>
%\fi

% Optional override for |\version| flag:
%    \begin{macrocode}
%%\providecommand{\version}{final}
%    \end{macrocode}

% Include the main document:
%    \begin{macrocode}
\input{childdoc.def}
\childdocof{cdocsamp}
%    \end{macrocode}

%\iffalse
%</samplechap1|samplechap2>
%\fi
%
%\iffalse
%<*samplechap1>
%\fi
% Some text for chapter 1:
%    \begin{macrocode}
\section{one}
some text in chapter one
%    \end{macrocode}

%\iffalse
%</samplechap1>
%\fi
% Some text for chapter 2:
%\iffalse
%<*samplechap2>
%\fi
%    \begin{macrocode}
\section{two}
more text in chapter two
%    \end{macrocode}

%\iffalse
%</samplechap2>
%\fi
%
% %%%%%%%%%%%%%%%%%%%%%%%%%%%%%%%%%%%%%%
% \paragraph{Part Include Files.}
%
% The include files are called |cdocspt3.tex| and |cdocspt4.tex|.
%
%\iffalse
%<*samplepart3|samplepart4>
%\fi

% Optional override for |\version| flag:
%    \begin{macrocode}
%%\providecommand{\version}{final}
%    \end{macrocode}

% Include the main document:
%    \begin{macrocode}
\input{childdoc.def}
\childdocby{cdocsamp}
%    \end{macrocode}

%\iffalse
%</samplepart3|samplepart4>
%\fi
%
%\iffalse
%<*samplepart3>
%\fi
% Some text for part 3:
%    \begin{macrocode}
some text in part three
%    \end{macrocode}

%\iffalse
%</samplepart3>
%\fi
% Some text for part 4:
%\iffalse
%<*samplepart4>
%\fi
%    \begin{macrocode}
more text in part four
%    \end{macrocode}

%\iffalse
%</samplepart4>
%\fi
%
% %%%%%%%%%%%%%%%%%%%%%%%%%%%%%%%%%%%%%%
% \paragraph{Forwarding for a Complete Draft.}
%
% The following forwarding file |cdocsdrf.tex|
% compiles the main document in draft mode:
%\iffalse
%<*sampledraft>
%\fi
%    \begin{macrocode}
\def\version{draft}
\input{childdoc.def}
\childdocforward{cdocsamp}
%    \end{macrocode}

%\iffalse
%</sampledraft>
%\fi
%
% %%%%%%%%%%%%%%%%%%%%%%%%%%%%%%%%%%%%%%
% \paragraph{Forwarding for Final Version of the Chapters.}
%
% The following forwarding files |cdocsfn1.tex| and |cdocsfn2.tex|
% (with identical content)
% compile the final versions of the child documents
% |cdocsch1.tex| and |cdocsch2.tex|, respectively:
%\iffalse
%<*samplefinal>
%\fi
%    \begin{macrocode}
\def\version{final}
\input{childdoc.def}
\childdocforwardprefix[cdocsamp]{cdocsfn}{cdocsch}
%    \end{macrocode}

%\iffalse
%</samplefinal>
%\fi
%
% %%%%%%%%%%%%%%%%%%%%%%%%%%%%%%%%%%%%%%
% \paragraph{Command Line Processing.}
%
% The following three command lines generate the output files
% |cdocscld|, |cdocscl1| and |cdocscl2|
% which should be identical to
% |cdocsdrf|, |cdocsch1| and |cdocsfn2|, respectively:
% \begin{center}
% \begin{tabular}{l}
% |latex -jobname cdocscld \|\\
% |  "\def\version{draft}\input{childdoc.def}\childdocforward{cdocsamp}"|\\
% |latex -jobname cdocscl1 \|\\
% |  "\input{childdoc.def}\childdocforward[cdocsamp]{cdocsch1}"|\\
% |latex -jobname cdocscl2 \|\\
% |  "\def\version{final}\input{childdoc.def}\childdocforward{cdocsch2}"|
% \end{tabular}
% \end{center}
% Note that the trailing backslash on each first line
% merely continues the input to the second line
% (for convenient cut ant paste).
% Furthermore, the command |latex| can be replaced by any
% of its alternative versions such as |pdflatex|.
%
% %%%%%%%%%%%%%%%%%%%%%%%%%%%%%%%%%%%%%%%%%%%%%%%%%%%%%%%%%%%%%%%%%%%%%%%%%%%%%%
% %%%%%%%%%%%%%%%%%%%%%%%%%%%%%%%%%%%%%%%%%%%%%%%%%%%%%%%%%%%%%%%%%%%%%%%%%%%%%%
% \section{Implementation}
%\iffalse
%<*package>
%\fi
%
% This section describes the definitions file |childdoc.def|.

% The definitions cannot be loaded using |\usepackage| or |\RequirePackage|
% which has a mechanism to prevent loading a style file more than once.
% When loading the definitions by means of |\input|
% multiple instances have to be prevented manually:
%\iffalse
%This code needs to be before the `\ProvidesFile' directive
%which is defined at the beginning of this file.
%Therefore it is also placed there and commented out here.
%</package>
%<*discard>
%\fi
%    \begin{macrocode}
\ifdefined\childdocmain\endinput\fi
%    \end{macrocode}
%\iffalse
%</discard>
%<*package>
%\fi
%
% \macro{\ifchilddoc}
% \macro{\ifchilddocmanual}
% The conditional |\ifchilddoc| tells whether a
% child (true) or main (false) document is being compiled.
% The conditional |\ifchilddocmanual| tells whether
% the |\includeonly| mechanism is used (false) or
% the selection of child files must be performed manually (true).
% The definitions initialise to false:
%    \begin{macrocode}
\newif\ifchilddoc
\newif\ifchilddocmanual
%    \end{macrocode}

% \macro{\childdocname}
% \macro{\childdocjob}
% The macro |\childdocname| stores the name of the main document
% to be compiled. The macro |\childdocjob| stores the name of
% the document on which the \LaTeX{} compiler was originally invoked.
% The content of |\jobname| cannot be compared
% to filenames specified in the source due to different catcodes.
% The following code rescans |\jobname|, stores the result
% in |\childdocname| and saves a copy in |\childdocjob|:
%    \begin{macrocode}
\edef\childdocname{\scantokens\expandafter{\jobname\noexpand}}
\let\childdocjob\childdocname
%    \end{macrocode}

% \macro{\childdocdisable}
% The macro |\childdocdisable| prevents the main file
% from being processed more than once.
% At this stage, the main document command |\childdocmain|
% is assumed to be called once again where it should do nothing.
% Any subsequent call to it should prevent
% a secondary processing of the main document
% It overwrites the forwarding commands
% |\childdocof| and |\childdocforward|
% with empty macros to prevent further inclusions of the main document:
%    \begin{macrocode}
\newcommand{\childdocdisable}
{
  \renewcommand{\childdocmain}[1]{\renewcommand{\childdocmain}[1]{\endinput}}
  \renewcommand{\childdocof}[1]{}
  \renewcommand{\childdocby}[2][]{}
  \renewcommand{\childdocforward}[2][]{}
  \renewcommand{\childdocdisable}{}
}
%    \end{macrocode}

% \macro{\childdocmain}
% The macro |\childdocmain| is to be called at the top of the main file
% with nothing or the main filename (without extension) as argument.
% First, it breaks loops.
% If the argument is not empty and does not match |\childdocname|
% (which is set by the first inclusion of |childdoc.def|),
% |\ifchilddoc| is set to true, |\includeonly| is applied to the child file
% and |\jobname| is set to the main file
% (for proper handling of |.aux| files):
%    \begin{macrocode}
\newcommand{\childdocmain}[1]
{
  \childdocdisable\childdocmain{}
  \if?#1?\else
    \begingroup
      \def\childdoctmp{#1}
      \ifx\childdoctmp\childdocname
        \def\childdoctmp{}
      \else
        \def\childdoctmp
        {
          \childdoctrue
          \includeonly{\childdocname}
          \def\childdocjob{#1}
          \def\jobname{#1}
        }
      \fi
      \expandafter
    \endgroup
    \childdoctmp
  \fi
}
%    \end{macrocode}

% \macro{\childdocof}
% The command |\childdocof| redirects
% compilation to the main file |#1|.
%    \begin{macrocode}
\newcommand{\childdocof}[1]
{
  \childdocdisable
  \childdoctrue
  \includeonly{\childdocname}
  \def\jobname{#1}
  \def\childdocjob{#1}
  \input{#1}
}
%    \end{macrocode}

% \macro{\childdocby}
% The command |\childdocby| ....
%    \begin{macrocode}
\newcommand{\childdocby}[2][]
{
  \childdocdisable
  \childdoctrue
  \childdocmanualtrue
  \if?#1?\else
    \def\jobname{#2}
  \fi
  \def\childdocjob{#2}
  \input{#2}
  \endinput
}
%    \end{macrocode}

% \macro{\childdocforward}
% The command |\childdocforward| redirects
% compilation to the main file or
% (if the optional argument is given) a child file.
% Parameters are set as if the main file
% or a child file starting with |\childdocof| was compiled.
% Then compilation is handed over to the main file:
%    \begin{macrocode}
\newcommand{\childdocforward}[2][]
{
  \begingroup
    \if?#1?
      \def\childdoctmp
      {
        \def\childdocname{#2}
        \def\childdocjob{#2}
        \def\jobname{#2}
        \input{#2}
        \endinput
      }
    \else
      \def\childdoctmp
      {
        \childdocdisable
        \def\childdocname{#2}
        \childdoctrue
        \includeonly{#2}
        \def\childdocjob{#1}
        \def\jobname{#1}
        \input{#1}
        \endinput
      }
    \fi
    \expandafter
  \endgroup
  \childdoctmp
}
%    \end{macrocode}

% \macro{\childdocforwardprefix}
% The command |\childdocforwardprefix| redirects
% compilation to the main or a child file by means of a pattern.
% The prefix |#1| in the current filename is replaced by |#2|
% and the suffix of the current filename is kept
% (it is assumed that the filename does not contain the substring `|~~~|'
% which is used as a delimiter).
% Compilation is handed over to the new file by |\childdocforward|:
%    \begin{macrocode}
\newcommand{\childdocforwardprefix}[3][]
{
  \begingroup
    \def\childdocextract #2##1~~~{\def\childdoctmp{\childdocforward[#1]{#3##1}}}
    \expandafter\childdocextract\childdocname~~~
    \expandafter
  \endgroup
  \childdoctmp
}
%    \end{macrocode}

% \macro{\childdoc}
% The deprecated macro |\childdoc| is a legacy version of |\childdocmain|:
%    \begin{macrocode}
\newcommand{\childdoc}{\childdocmain}
%    \end{macrocode}

% \macro{\childdocredirect}
% The deprecated macro |\childdocredirect| is a legacy version
% of |\childdocforward| and |\childdocforwardprefix|:
%    \begin{macrocode}
\newcommand{\childdocredirect}[2][]
{
  \begingroup
    \if?#1?
      \def\childdoctmp{\childdocforward{#2}}
    \else
      \def\childdoctmp{\childdocforwardprefix{#1}{#2}}
    \fi
    \expandafter
  \endgroup
  \childdoctmp
}
%    \end{macrocode}

%\iffalse
%</package>
%\fi
%
\endinput
\childdocforward[cdocsamp]{cdocsch1}"|\\
% |latex -jobname cdocscl2 \|\\
% |  "\def\version{final}% \iffalse
%
% childdoc.dtx Copyright (C) 2017-2018 Niklas Beisert
%
% This work may be distributed and/or modified under the
% conditions of the LaTeX Project Public License, either version 1.3
% of this license or (at your option) any later version.
% The latest version of this license is in
%   http://www.latex-project.org/lppl.txt
% and version 1.3 or later is part of all distributions of LaTeX
% version 2005/12/01 or later.
%
% This work has the LPPL maintenance status `maintained'.
%
% The Current Maintainer of this work is Niklas Beisert.
%
% This work consists of the files childdoc.dtx and childdoc.ins
% and the derived files childdoc.def and cdocsamp.tex with
% cdocsch1.tex, cdocsch2.tex, cdocsdrf.tex, cdocsfn1.tex, cdocsfn2.tex.
%
%<package>\ifdefined\childdocmain\endinput\fi
%<package>\ProvidesFile{childdoc.def}[2018/12/30 v2.0 child document driver]
%<samplemain>\ProvidesFile{cdocsamp.tex}[2018/12/30 v2.0 sample for childdoc]
%<*driver>
%\ProvidesFile{childdoc.drv}[2018/12/30 v2.0 childdoc reference manual file]
\PassOptionsToClass{10pt,a4paper}{article}
\documentclass{ltxdoc}

\usepackage[margin=35mm]{geometry}
\usepackage{hyperref}
\usepackage{hyperxmp}
\usepackage[usenames]{color}

\hypersetup{colorlinks=true}
\hypersetup{pdfstartview=FitH}
\hypersetup{pdfpagemode=UseNone}
\hypersetup{pdfsource={}}
\hypersetup{pdflang={en-UK}}
\hypersetup{pdfcopyright={Copyright 2017-2018 Niklas Beisert.
  This work may be distributed and/or modified under the
  conditions of the LaTeX Project Public License, either version 1.3
  of this license or (at your option) any later version.}}
\hypersetup{pdflicenseurl={http://www.latex-project.org/lppl.txt}}
\hypersetup{pdfcontactaddress={ETH Zurich, ITP, HIT K,
  Wolfgang-Pauli-Strasse 27}}
\hypersetup{pdfcontactpostcode={8093}}
\hypersetup{pdfcontactcity={Zurich}}
\hypersetup{pdfcontactcountry={Switzerland}}
\hypersetup{pdfcontactemail={nbeisert@itp.phys.ethz.ch}}
\hypersetup{pdfcontacturl={http://people.phys.ethz.ch/\xmptilde nbeisert/}}

\newcommand{\secref}[1]{\hyperref[#1]{section \ref*{#1}}}

\parskip1ex
\parindent0pt
\let\olditemize\itemize
\def\itemize{\olditemize\parskip0pt}

\begin{document}

\title{The \textsf{childdoc} Package}
\hypersetup{pdftitle={The childdoc Package}}
\author{Niklas Beisert\\[2ex]
  Institut f\"ur Theoretische Physik\\
  Eidgen\"ossische Technische Hochschule Z\"urich\\
  Wolfgang-Pauli-Strasse 27, 8093 Z\"urich, Switzerland\\[1ex]
  \href{mailto:nbeisert@itp.phys.ethz.ch}
  {\texttt{nbeisert@itp.phys.ethz.ch}}}
\hypersetup{pdfauthor={Niklas Beisert}}
\hypersetup{pdfsubject={Manual for the LaTeX2e Package childdoc}}
\date{30 December 2018, \textsf{v2.0}}
\maketitle

\begin{abstract}\noindent
\textsf{childdoc} is a \LaTeXe{} package
that enables the direct compilation
of document sections included by |\include|
to individual files.
\end{abstract}

\begingroup
\parskip0ex
\tableofcontents
\endgroup

%%%%%%%%%%%%%%%%%%%%%%%%%%%%%%%%%%%%%%%%%%%%%%%%%%%%%%%%%%%%%%%%%%%%%%%%%%%%%%%%
%%%%%%%%%%%%%%%%%%%%%%%%%%%%%%%%%%%%%%%%%%%%%%%%%%%%%%%%%%%%%%%%%%%%%%%%%%%%%%%%
\section{Introduction}

\LaTeX{} provides a mechanism to structure a large document (such as a book)
into a main file and several child files (containing the chapters)
using the |\include| command.
This mechanism is beneficial for documents
which span hundreds of pages in order to
make the source file(s) more manageable.
Moreover, compilation can be restricted to
selected child files by means of the |\includeonly| command.
The latter feature can be used to reduce the compilation time while editing
(this was significantly more useful in the earlier days of \LaTeX{})
or to generate a smaller document which is easier to navigate.
Another application of |\includeonly| is to generate
documents consisting of selected parts of the complete document.

However, there are a few drawbacks of the plain |\include| mechanism:
\begin{itemize}
\item
The child files cannot be compiled on their own,
they can only be compiled via the main file.
A naive editing environment
(such as a text editor with an option
to have the current file processed by \LaTeX)
may require one to switch to the main file before compiling;
attempting to compile the child file produces errors.
\item
The main file must be modified (each time)
to adjust the |\includeonly| command
to the present needs. This easily leaves the main file in a messy state.
\item
The generated document will always carry the filename
of the main document. This is inconvenient if
several child files are to be compiled and
to be kept for distribution.
\end{itemize}

The present package provides a simple interface
to make child files individually compilable by \LaTeX{}.
Compiling a child file then has the same effect as compiling
the main file with an |\includeonly| command
to select the appropriate child.
Moreover the generated document will carry the name of the child
rather than the main file.
This resolves all three above issues.

This feature is meant to make the editing of books,
thesis documents and lecture notes somewhat more convenient.
However, the package can also be used efficiently for
composing a series of documents (such as exercise sheets)
which are typically distributed individually.
It then assists the author in generating the individual documents
(potentially in different versions)
as well as a document containing the collected series.
Another application is in developing style files
or other kinds of included material
where compilation of the style file could redirect
to a sample or test file.

%%%%%%%%%%%%%%%%%%%%%%%%%%%%%%%%%%%%%%%%%%%%%%%%%%%%%%%%%%%%%%%%%%%%%%%%%%%%%%%%
%%%%%%%%%%%%%%%%%%%%%%%%%%%%%%%%%%%%%%%%%%%%%%%%%%%%%%%%%%%%%%%%%%%%%%%%%%%%%%%%
\section{Usage}

First of all, the package \textsf{childdoc} is \emph{not} a standard
\LaTeXe{} |.sty| style file! Therefore it needs to be invoked in
a non-standard way.

%%%%%%%%%%%%%%%%%%%%%%%%%%%%%%%%%%%%%%%%%%%%%%%%%%%%%%%%%%%%%%%%%%%%%%%%%%%%%%%%
\subsection{Included Files}
\label{sec:include}

%%%%%%%%%%%%%%%%%%%%%%%%%%%%%%%%%%%%%%%%
\DescribeMacro{\childdocmain}
To use the package, add the commands
\begin{center}
\begin{tabular}{l}
|\input{childdoc.def}|\\
|\childdocmain{}|\\
\end{tabular}
\end{center}
at the very top of the main \LaTeX{} file,
in particular \emph{before} the |\documentclass| statement!
The argument of |\childdocmain| should be left empty
(but it must be present).

%%%%%%%%%%%%%%%%%%%%%%%%%%%%%%%%%%%%%%%%
\DescribeMacro{\childdocof}
Furthermore, add the commands
\begin{center}
\begin{tabular}{l}
|\input{childdoc.def}|\\
|\childdocof{|\textit{main}|}|\\
\end{tabular}
\end{center}
at the top of every child file \textit{child}
which is included by |\include{|\textit{child}|}|
from within the main file
(or at least for those files to be compiled individually).
The argument \textit{main} must be the filename of the main file.

There are a couple of
considerations in setting up the main and child documents:

%%%%%%%%%%%%%%%%%%%%%%%%%%%%%%%%%%%%%%%%
\paragraph{Restrictions.}

Please note the following restrictions:
\begin{itemize}
\item
|\childdocmain| must be called with one argument \textit{main}
to ensure compatibility with earlier version of the package.
It must either be empty (|\childdocmain{}|)
or precisely match the filename of the main file in which it is specified.
See \secref{sec:detection} for further information.
\item
The filename \textit{main} must be specified without the |.tex| extension.
\item
The filename \textit{main} is case sensitive
(even in case-insensitive file systems)
due to internal string comparison.
\item
The argument \textit{main} should be fully expanded, it cannot be a macro.
\item
Subdirectories and special characters should be avoided in filenames.
\item
The command |\childdocmain{|\textit{main}|}| must be followed by a whitespace.
It should not be followed immediately by another command
or by a comment mark `|%|'.
This is because the \TeX{} parser reads the token immediately following
the argument of |\childdocmain| and puts it
at the beginning of every child section;
however, a white\-space is ignored.
\end{itemize}

%%%%%%%%%%%%%%%%%%%%%%%%%%%%%%%%%%%%%%%%
\paragraph{Content of Main File.}

It is advisable to place all content in the child files included by |\include|.
Any output contained in the main file will appear in all child documents
unless suppressed manually;
it cannot be suppressed automatically by the |\includeonly| directive
and thus should normally be avoided.
A method to include some content in the main file
by means of conditional processing is described in \secref{sec:conditional}.

%%%%%%%%%%%%%%%%%%%%%%%%%%%%%%%%%%%%%%%%
\paragraph{Page Numbering.}

When only a part of the document is compiled,
the appropriate numbering of pages
(as well as other status parameters)
is determined from the |.aux| files.
The latter contain information from previous passes.
However this information needs to propagate through
all intermediate child documents.
Therefore the page numbering in child documents may well
be inconsistent until the complete document is compiled at least once.

A useful (if unconventional) way to always ensure a consistent
page numbering is to restart the numbering in each child document
and denote the pages by `\textit{child}|.|\textit{page}'
where \textit{child} represents the chapter/section number of the child file.
This can be achieved by the command
|\numberwithin{page}{|\textit{child}|}|
of the \textsf{amsmath} package
where \textit{child} can be |chapter| or |section|
depending on the chosen structuring.
Alternatively, one can modify the macro |\thepage| appropriately
and reset the counter |page| at the start of each child file.

%%%%%%%%%%%%%%%%%%%%%%%%%%%%%%%%%%%%%%%%%%%%%%%%%%%%%%%%%%%%%%%%%%%%%%%%%%%%%%%%
\subsection{Conditional Processing}
\label{sec:conditional}

The package provides a mechanism to compile different versions
of a document. To customise the versions further some conditional processing
can come in handy to distinguish which version is being compiled.
The package provides two macros to describe the compilation context:

%%%%%%%%%%%%%%%%%%%%%%%%%%%%%%%%%%%%%%%%
\DescribeMacro{\ifchilddoc}
The conditional |\ifchilddoc| distinguishes between the compilation of
child documents and the main document:
%
\begin{center}
|\ifchilddoc |\textit{child-code}| |[|\||else |\textit{main-code}]| \||fi|
\end{center}

%%%%%%%%%%%%%%%%%%%%%%%%%%%%%%%%%%%%%%%%
\DescribeMacro{\childdocname}
\DescribeMacro{\childdocjob}
The macro |\childdocname| contains the filename (without extension)
of the main or child file being processed.
Note that |\childdocjob| will always contain the name of the main file.

%%%%%%%%%%%%%%%%%%%%%%%%%%%%%%%%%%%%%%%%
\paragraph{Title Page.}

Conditional processing can be used to include a title or banner page
in the main document when proper precautions are taken.
Importantly, the code in the main file should ensure that the page counter
(as well as other status parameters which are stored in the |.aux| files)
takes the same value after the conditional processing.
Otherwise the page numbers may take divergent values
depending on which part is compiled.

For example, a title page could be declared by:
%
\begin{center}
\begin{tabular}{l}
|\ifchilddoc\||else|\\
|\addtocounter{page}{-1}|\\
\textit{code for title page}\\
|\newpage|\\
|\||fi|
\end{tabular}
\end{center}
%
A banner page for the child documents can be generated by:
%
\begin{center}
\begin{tabular}{l}
|\ifchilddoc|\\
|\addtocounter{page}{-1}|\\
\textit{code for banner page}\\
|\newpage|\\
|\||fi|
\end{tabular}
\end{center}
%
Here one could write a message such as:
\begin{center}
|This is the part \childdocname{} of \childdocjob{}.|
\end{center}

%%%%%%%%%%%%%%%%%%%%%%%%%%%%%%%%%%%%%%%%%%%%%%%%%%%%%%%%%%%%%%%%%%%%%%%%%%%%%%%%
\subsection{Flags}
\label{sec:flags}

The package makes it easy to generate different versions
of the main or child documents.
To this end compilation flags can be defined
and assigned different default values.
They will be particularly useful in conjunction
with the forwarding mechanism described in \secref{sec:forward}.

For example, it may be useful to have a flag |\version|
which can be set to |draft| or |final|.
The document source will contain some conditional code
depending on the value of |\version|.
Suppose further, the flag should default to |final| for the main file
and to |draft| for child files
which is a natural assignment for editing the document.
This is achieved by placing the following code
in the preamble of the main document
(below the |\childdocmain| directive):
%
\begin{center}
\begin{tabular}{l}
|\ifchilddoc|\\
|\providecommand{\version}{draft}|\\
|\||else|\\
|\providecommand{\version}{final}|\\
|\||fi|
\end{tabular}
\end{center}
%
The definition by |\providecommand| makes sure
that previous definitions are not overwritten.
Further statements |\providecommand{\version}{...}|
can thus be added before the above code to override it.

For the main file, one might add a line
(between |\childdocmain| and the above block)
%
\begin{center}
|%\ifchilddoc\||else\providecommand{\version}{draft}\||fi|
\end{center}
%
which can be uncommented to produce a draft version.
Likewise one can add a line to the very top of a child file
(above the |\childdocof{|\textit{main}|}| directive)
%
\begin{center}
|%\providecommand{\version}{final}|
\end{center}
%
which can be uncommented to produce the final version of this child document.

%%%%%%%%%%%%%%%%%%%%%%%%%%%%%%%%%%%%%%%%%%%%%%%%%%%%%%%%%%%%%%%%%%%%%%%%%%%%%%%%
\subsection{Forwarding}
\label{sec:forward}

Different versions of the main or child documents
using compilation flags as described in \secref{sec:flags}
can be (permanently) stored in different files
for convenient compilation, viewing and distribution.
To this end, the package defines a command
to pass on compilation to a different file:

%%%%%%%%%%%%%%%%%%%%%%%%%%%%%%%%%%%%%%%%
\DescribeMacro{\childdocforward}
The command |\childdocforward| redirects processing to
another source file:
%
\begin{center}
\begin{tabular}{l}
|\input{childdoc.def}|\\
|\childdocforward[|\textit{main}|]{|\textit{dest}|}|\\
\end{tabular}
\end{center}
%
The argument \textit{dest} is the destination file
(without extension).
It should be the main file or one of the child files.
Note that further \textsf{childdoc} directives
such as |\childdocof| and |\childdocforward|
in the indicated file will be processed in this form.
The optional argument \textit{main}
passes on directly to the main file \textit{main}
while pretending to compile the child \textit{dest}.
This form behaves as if \textit{dest}
issues |\childdocof{|\textit{main}|}| right away,
and no further \textsf{childdoc} directives will be processed.

%%%%%%%%%%%%%%%%%%%%%%%%%%%%%%%%%%%%%%%%
\DescribeMacro{\...prefix}
In the alternative form |\childdocforwardprefix|,
%
\begin{center}
\begin{tabular}{l}
|\input{childdoc.def}|\\
|\childdocforwardprefix[|\textit{main}|]{|\textit{prefix}|}{|\textit{dest}|}|
\end{tabular}
\end{center}
%
the destination file is determined by a pattern
depending on the current file:
To make this work, the current file must be called
`{\textit{prefix}\hspace{0.2em}\textit{suffix}}'
with \textit{prefix} matching precisely the argument.
Processing is then passed on to the file
`{\textit{dest}\hspace{0.2em}\textit{suffix}}'.
Surely, the same effect is achieved by
directly specifying the
argument `{\textit{dest}\hspace{0.2em}\textit{suffix}}'
in the first form.
However, that requires to set up a different file
for each child. With the alternative form of the command
all these files can have exactly the same content
which simplifies setting them up and maintaining them.

For example, the following file |draft.tex|
with a compilation flag |\version| as described in \secref{sec:flags}
compiles the main document as a draft:
%
\begin{center}
\begin{tabular}{l}
|\def\version{draft}|\\
|\input{childdoc.def}|\\
|\childdocforward{|\textit{main}|}|
\end{tabular}
\end{center}
%
Likewise, the following files |final|\textit{nn}|.tex|
compile the final version of the child document
|child|\textit{nn}|.tex|:
%
\begin{center}
\begin{tabular}{l}
|\def\version{final}|\\
|\input{childdoc.def}|\\
|\childdocforwardprefix{final}{child}|
\end{tabular}
\end{center}
%

Note that when several versions of a main file and/or of each child file
are to be generated, it may be convenient to set up a |Makefile| or
shell script to automatise the process.

%%%%%%%%%%%%%%%%%%%%%%%%%%%%%%%%%%%%%%%%%%%%%%%%%%%%%%%%%%%%%%%%%%%%%%%%%%%%%%%%
\subsection{Command Line Processing}
\label{sec:commandline}

The effect of redirection files can also be achieved by invoking
the \LaTeX{} compiler with a more elaborate command line.
Most conveniently this should be done as part
of a shell script or a |Makefile|.

When using \textsf{childdoc} in the main file, the following
command lines effectively perform a redirection
(note that depending on the shell being used,
backslashes may have to be doubled: `|\|' $\to$ `|\\|'):
%
\begin{center}
|... -jobname "|\textit{target}|" |\\|"|[\textit{flags}]%
|\input{childdoc.def}\childdocforward[|\textit{main}|]{|\textit{dest}|}"|
\end{center}
%
Here \textit{target} is the name of the output file,
\textit{main} is the name of the main file
and \textit{dest} is the name of the main or child file to be processed
(all filenames without extensions).
The optional argument \textit{main} can be omitted
if \textit{main} matches \textit{dest}.
Optionally, compilation \textit{flags} can be defined via |\def| commands.
This command line makes the \TeX{} engine believe
it is compiling the file \textit{target}
whose content is specified as the latter parameter.
The provided code then forwards the processing to
\textit{main} or \textit{dest} as described in \secref{sec:forward}.

%%%%%%%%%%%%%%%%%%%%%%%%%%%%%%%%%%%%%%%%%%%%%%%%%%%%%%%%%%%%%%%%%%%%%%%%%%%%%%%%
\subsection{Include by Input}
\label{sec:input}

Including child documents by |\include| has some restrictions by design.
Most notably, the content of a child document always occupies
its own set of pages; pages cannot be shared between child documents.
Usually, this behaviour makes perfect sense
because each child document contain an essential part of the document.
However, in some situations it may be desirable to compose
a document from a collection of parts
without having mandatory page breaks between then.
For this case, the package
provides a mechanism to include parts
by |\input| which can also be processed individually.
However, by construction this mechanism
requires manual handling of the content to be output.

%%%%%%%%%%%%%%%%%%%%%%%%%%%%%%%%%%%%%%%%
\DescribeMacro{\ifchilddocmanual}
The main file should be prepared as usual, see \secref{sec:include}.
However, the document body must make a distinction
between processing of an individual part and of the main document, e.g.:
%
\begin{center}
\begin{tabular}{l}
|\ifchilddocmanual|\\
|\input{\childdocname}|\\
|\||else|\\
\textit{document body with }|\input{|\textit{part}|}|\\
|\||fi|
\end{tabular}
\end{center}
%
The conditional |\ifchilddocmanual| is true whenever
a part to be included by |\input| is being compiled,
and the name of the part is stored in |\childdocname|.

%%%%%%%%%%%%%%%%%%%%%%%%%%%%%%%%%%%%%%%%
\DescribeMacro{\childdocby}
Each part to be included by |\input| should start with:
%
\begin{center}
\begin{tabular}{l}
|\input{childdoc.def}|\\
|\childdocby{|\textit{main}|}|\\
\end{tabular}
\end{center}
%
The directive |\childdocby| is similar to |\childdocof|
described in \secref{sec:include},
but the subsequent selection of content must be done manually.
To that end, both |\ifchilddoc| and |\ifchilddocmanual|
will be true upon processing of a part,
and the name of the part is stored in |\childdocname|.
Note that |\jobname| will be set to the filename of the current part
so that each part receives an individual |.aux| file
that does not interfere with the |.aux| file(s) of the main document.
This behaviour can be altered by the alternative form
|\childdocby[*]{|\textit{main}|}| (with a non-empty optional argument)
which uses the |.aux| file of the main document
by setting |\jobname| to \textit{main}.

%%%%%%%%%%%%%%%%%%%%%%%%%%%%%%%%%%%%%%%%%%%%%%%%%%%%%%%%%%%%%%%%%%%%%%%%%%%%%%%%
\subsection{Driver Development}
\label{sec:driver}

The \textsf{childdoc} mechanism can also be use for the development
of definition files such as \LaTeX{} styles or classes.
This case differs from the above setup with multiple parts
included by |\include| in that no |\includeonly| should be invoked.
This can be achieved by starting the include file
(before |\ProvidesPackage|) with:
%
\begin{center}
\begin{tabular}{l}
|\input{childdoc.def}|\\
|\childdocforward{|\textit{main}|}|\\
\end{tabular}
\end{center}
%
or alternatively with:
%
\begin{center}
\begin{tabular}{l}
|\input{childdoc.def}|\\
|\childdocby{|\textit{main}|}|\\
\end{tabular}
\end{center}
%
Both forms have slightly different effects as described above.
The main file is prepared as usual, see \secref{sec:include}.

%%%%%%%%%%%%%%%%%%%%%%%%%%%%%%%%%%%%%%%%%%%%%%%%%%%%%%%%%%%%%%%%%%%%%%%%%%%%%%%%
\subsection{Legacy Detection}
\label{sec:detection}

The directive |\childdocmain| in the main file can detect
whether the complete document or merely a child is to be compiled
even without using the directive |\childdocof|.
This method is deprecated because it is less robust
and there is no compelling reason to use it;
it is merely provided for backward compatibility
and it may be removed in future versions.

If the detection mechanism is to be used,
it is mandatory to correctly specify
the filename of the main file as the argument of |\childdocmain|:
%
\begin{center}
\begin{tabular}{l}
|\input{childdoc.def}|\\
|\childdocmain{|\textit{main}|}|\\
\end{tabular}
\end{center}
%
If |\jobname| does not match the argument \textit{main} of |\childdocmain|,
it is assumed that |\jobname| points to the child file to be compiled.
When using |\childdocmain| with the main file specified as argument,
it suffices to start a child file
with just |\input{|\textit{main}|}|
without loading of the package and using |\childdocof|.
If instead all processing is done
with the appropriate \textsf{childdoc} directives,
the argument of \textit{main} of |\childdocmain| can be empty.

An alternative version of the command line processing described
in \secref{sec:commandline} using the detection mechanism reads:
%
\begin{center}
|... -jobname "|\textit{target}|" "|[\textit{flags}]%
[|\def\jobname{|\textit{dest}|}|]|\input{|\textit{main}|}"|
\end{center}

%%%%%%%%%%%%%%%%%%%%%%%%%%%%%%%%%%%%%%%%%%%%%%%%%%%%%%%%%%%%%%%%%%%%%%%%%%%%%%%%
\subsection{Manual Code}
\label{sec:manual}

In case one cannot be certain whether the definitions file |childdoc.def|
is installed on the target \TeX{} distribution
and one prefers not to ship it,
it is conceivable to paste a few relevant commands into the sources.

To that end, drop all statements |\input{childdoc.def}|
and perform the replacements as outlined below.
Instead of |\childdocmain{|\textit{main}|}| add the following code
to the top of the main file:
%
\begin{center}
\begin{tabular}{l}
|\||ifdefined\childdocname\endinput\||fi\newif\ifchilddoc|\\
|\edef\childdocname{\scantokens\expandafter{\jobname\noexpand}}|\\
|\def\childdocmain{|\textit{main}|}\||ifx\childdocmain\childdocname\||else|\\
|\childdoctrue\includeonly{\childdocname}\let\jobname\childdocmain\||fi|\\
\end{tabular}
\end{center}
%
Instead of |\childdocof{|\textit{main}|}| just include the main file
at the top of each child file:
%
\begin{center}
|\input{|\textit{main}|}|
\end{center}
%
A simple redirection |\childdocforward{|\textit{dest}|}| is achieved by:
%
\begin{center}
|\def\jobname{|\textit{dest}|}\input{\jobname}|
\end{center}
%
The redirection with prefix
|\childdocforwardprefix[|\textit{prefix}|]{|\textit{dest}|}|
is accomplished by:
%
\begin{center}
\begin{tabular}{l}
|{\edef\jobname{\scantokens\expandafter{\jobname\noexpand}}|\\
|\def\redirectjob |\textit{prefix}|#1~~~{\gdef\jobname{|\textit{dest}|#1}}|\\
|\expandafter\redirectjob\jobname~~~}\input{\jobname}|
\end{tabular}
\end{center}

In an alternative approach,
child documents can be compiled by a specific command line
without additional code or specific definitions:
%
\begin{center}
|... -jobname "|\textit{target}|" "|[\textit{flags}]%
|\includeonly{|\textit{dest}|}\input{|\textit{main}|}"|
\end{center}
%

%%%%%%%%%%%%%%%%%%%%%%%%%%%%%%%%%%%%%%%%%%%%%%%%%%%%%%%%%%%%%%%%%%%%%%%%%%%%%%%%
%%%%%%%%%%%%%%%%%%%%%%%%%%%%%%%%%%%%%%%%%%%%%%%%%%%%%%%%%%%%%%%%%%%%%%%%%%%%%%%%
\section{Information}

%%%%%%%%%%%%%%%%%%%%%%%%%%%%%%%%%%%%%%%%%%%%%%%%%%%%%%%%%%%%%%%%%%%%%%%%%%%%%%%%
\subsection{Copyright}

Copyright \copyright{} 2017--2018 Niklas Beisert

This work may be distributed and/or modified under the
conditions of the \LaTeX{} Project Public License, either version 1.3
of this license or (at your option) any later version.
The latest version of this license is in
  \url{http://www.latex-project.org/lppl.txt}
and version 1.3 or later is part of all distributions of \LaTeX{}
version 2005/12/01 or later.

This work has the LPPL maintenance status `maintained'.

The Current Maintainer of this work is Niklas Beisert.

This work consists of the files |README.txt|, |childdoc.ins| and |childdoc.dtx|
as well as the derived files |childdoc.def|, |cdocsamp.tex|
with |cdocsch1.tex|, |cdocsch2.tex|, |cdocspt3.tex|, |cdocspt4.tex|,
|cdocsdrf.tex|, |cdocsfn1.tex|, |cdocsfn2.tex|
as well as |childdoc.pdf|.

%%%%%%%%%%%%%%%%%%%%%%%%%%%%%%%%%%%%%%%%%%%%%%%%%%%%%%%%%%%%%%%%%%%%%%%%%%%%%%%%
\subsection{Files and Installation}

The package consists of the files:
%
\begin{center}
\begin{tabular}{ll}
    |README.txt|   & readme file \\
    |childdoc.ins| & installation file \\
    |childdoc.dtx| & source file \\
    |childdoc.def| & definition file \\
    |cdocsamp.tex| & sample main file \\
    |cdocsch1.tex| & sample include file \\
    |cdocsch2.tex| & sample include file \\
    |cdocspt3.tex| & sample part file \\
    |cdocspt4.tex| & sample part file \\
    |cdocsdrf.tex| & sample redirection file \\
    |cdocsfn1.tex| & sample redirection file \\
    |cdocsfn2.tex| & sample redirection file \\
    |childdoc.pdf| & manual
\end{tabular}
\end{center}
%
The distribution consists of the files
|README.txt|, |childdoc.ins| and |childdoc.dtx|.
%
\begin{itemize}
\item
Run (pdf)\LaTeX{} on |childdoc.dtx|
to compile the manual |childdoc.pdf| (this file).
\item
Run \LaTeX{} on |childdoc.ins| to create the definitions file |childdoc.def|
and the sample |cdocsamp.tex| with include files
|cdocsch1.tex|, |cdocsch2.tex|, |cdocspt3.tex|, |cdocspt4.tex|,
|cdocsdrf.tex|, |cdocsfn1.tex|, |cdocsfn2.tex|.
Then copy the file |childdoc.def| to an appropriate directory of your \LaTeX{}
distribution, e.g.\ \textit{texmf-root}|/tex/latex/childdoc|.
\end{itemize}

%%%%%%%%%%%%%%%%%%%%%%%%%%%%%%%%%%%%%%%%%%%%%%%%%%%%%%%%%%%%%%%%%%%%%%%%%%%%%%%%
\subsection{Related CTAN Packages}

There are several other packages which offer a similar functionality:
%
\begin{itemize}
\item
The packages
\href{http://ctan.org/pkg/docmute}{\textsf{docmute}},
\href{http://ctan.org/pkg/includex}{\textsf{includex}} and
\href{http://ctan.org/pkg/standalone}{\textsf{standalone}}
provide commands to include only the document body of
a child file thus allowing both files to be compiled individually.
\item
The packages \href{http://ctan.org/pkg/subdocs}{\textsf{subdocs}}
and \href{http://ctan.org/pkg/subfiles}{\textsf{subfiles}}
provide structures in which the main and child documents can be
encapsulated and allowing them to be compiled individually.
The inclusion mechanism is different from the conventional |\include|.
\item
The package \href{http://ctan.org/pkg/combine}{\textsf{combine}}
is an elaborate solution to combine several documents into one.
\end{itemize}
%
See also the CTAN topic \href{http://ctan.org/topic/subdocs}{\textsf{subdocs}}
for further related packages.
The present package differs from the above solutions in that
a document structure constructed with the conventional |\include| mechanism
just needs two extra commands at the top of every file
such that all constituent files can be compiled individually.

%%%%%%%%%%%%%%%%%%%%%%%%%%%%%%%%%%%%%%%%%%%%%%%%%%%%%%%%%%%%%%%%%%%%%%%%%%%%%%%%
%\subsection{Feature Suggestions}
%
%The following is a list of features which may be useful for future
%versions of this package:
%%
%\begin{itemize}
%\item
%\ldots
%\end{itemize}

%%%%%%%%%%%%%%%%%%%%%%%%%%%%%%%%%%%%%%%%%%%%%%%%%%%%%%%%%%%%%%%%%%%%%%%%%%%%%%%%
\subsection{Revision History}

%%%%%%%%%%%%%%%%%%%%%%%%%%%%%%%%%%%%%%%%
\paragraph{v2.0:} 2018/12/30

\begin{itemize}
\item
immediate forward processing
\item
added |\childdocby| mechanism
\item
manual restructured
\end{itemize}

%%%%%%%%%%%%%%%%%%%%%%%%%%%%%%%%%%%%%%%%
\paragraph{v1.6:} 2018/01/17

\begin{itemize}
\item
application for development of include files
\item
corrections to manual
\end{itemize}

%%%%%%%%%%%%%%%%%%%%%%%%%%%%%%%%%%%%%%%%
\paragraph{v1.5:} 2017/05/21

\begin{itemize}
\item
more complete structuring introduced
\item
|\childdocof| introduced
\item
|\childdoc| renamed to |\childdocmain|
\item
|\childredirect| renamed to |\childdocforward| and |\childdocforwardprefix|
and functionality expanded
\end{itemize}

%%%%%%%%%%%%%%%%%%%%%%%%%%%%%%%%%%%%%%%%
\paragraph{v1.0:} 2017/04/27

\begin{itemize}
\item
manual and install package
\item
first version published on CTAN
\end{itemize}

%%%%%%%%%%%%%%%%%%%%%%%%%%%%%%%%%%%%%%%%
\paragraph{v0.6:} 2017/04/26

\begin{itemize}
\item
redirection mechanism added
\end{itemize}

%%%%%%%%%%%%%%%%%%%%%%%%%%%%%%%%%%%%%%%%
\paragraph{v0.5:} 2017/04/26

\begin{itemize}
\item
functionality in definition file
\end{itemize}


%%%%%%%%%%%%%%%%%%%%%%%%%%%%%%%%%%%%%%%%%%%%%%%%%%%%%%%%%%%%%%%%%%%%%%%%%%%%%%%%
%%%%%%%%%%%%%%%%%%%%%%%%%%%%%%%%%%%%%%%%%%%%%%%%%%%%%%%%%%%%%%%%%%%%%%%%%%%%%%%%
%%%%%%%%%%%%%%%%%%%%%%%%%%%%%%%%%%%%%%%%%%%%%%%%%%%%%%%%%%%%%%%%%%%%%%%%%%%%%%%%
\appendix

\settowidth\MacroIndent{\rmfamily\scriptsize 000\ }

 \DocInput{childdoc.dtx}

\end{document}
%</driver>
% \fi
%
% %%%%%%%%%%%%%%%%%%%%%%%%%%%%%%%%%%%%%%%%%%%%%%%%%%%%%%%%%%%%%%%%%%%%%%%%%%%%%%
% %%%%%%%%%%%%%%%%%%%%%%%%%%%%%%%%%%%%%%%%%%%%%%%%%%%%%%%%%%%%%%%%%%%%%%%%%%%%%%
% \section{Sample}
%\iffalse
%<*samplemain>
%\fi
%
% The following presents a sample document
% with two chapters, two parts, a title page,
% a compile flag as well as three forwarding files to set the flag.
% It consists of eight |.tex| files:
% \begin{center}
% \begin{tabular}{ll}
% |cdocsamp.tex|&main file\\
% |cdocsch1.tex|&include file for chapter 1\\
% |cdocsch2.tex|&include file for chapter 2\\
% |cdocspt3.tex|&include file for part 3\\
% |cdocspt4.tex|&include file for part 4\\
% |cdocsdrf.tex|&forwarding file for main file in draft mode\\
% |cdocsfi1.tex|&forwarding file for final version of chapter 1\\
% |cdocsfi2.tex|&forwarding file for final version of chapter 2\\
% \end{tabular}
% \end{center}
% Each of the eight files can be compiled directly by the \LaTeX{} compiler.
%
% %%%%%%%%%%%%%%%%%%%%%%%%%%%%%%%%%%%%%%
% \paragraph{Main File.}
%
% The main file is called |cdocsamp.tex|.
%
% Load the \textsf{childdoc} definitions and
% declare the filename for the main document:
%    \begin{macrocode}
\input{childdoc.def}
\childdocmain{}
%    \end{macrocode}

% Optional override for |\version| flag:
%    \begin{macrocode}
%%\ifchilddoc\else\providecommand{\version}{draft}\fi
%    \end{macrocode}

% Define the default values for the |\version| flag
% (|final| for the main file and |draft| for childs):
%    \begin{macrocode}
\ifchilddoc
\providecommand{\version}{draft}
\else
\providecommand{\version}{final}
\fi
%    \end{macrocode}

% Load the standard document class:
%    \begin{macrocode}
\documentclass[12pt]{article}
%    \end{macrocode}

% Start the document body:
%    \begin{macrocode}
\begin{document}
%    \end{macrocode}

% Declare a title page.
% Print title, part of document being processed and version flag:
%    \begin{macrocode}
\addtocounter{page}{-1}
\begin{center}
{\LARGE\bfseries{}childdoc example\par}
\vspace{1cm}
\ifchilddoc
\ifchilddocmanual part\else chapter\fi:
`\childdocname' of `\childdocjob'\par
\else
main document: `\childdocjob'\par
\fi
version: \version\par
\end{center}
\newpage
%    \end{macrocode}

% Manually include selected file,
% otherwise process as usual:
%    \begin{macrocode}
\ifchilddocmanual
\section*{part `\childdocname'}
\input{\childdocname}
\else
%    \end{macrocode}

% Include the two chapters:
%    \begin{macrocode}
\include{cdocsch1}
\include{cdocsch2}
%    \end{macrocode}

% Include the two parts unless only chapters should be displayed:
%    \begin{macrocode}
\ifchilddoc\else
\section{part three}
\input{cdocspt3}
\section{part four}
\input{cdocspt4}
\fi
%    \end{macrocode}

% Process as usual until here:
%    \begin{macrocode}
\fi
%    \end{macrocode}

% End of document body:
%    \begin{macrocode}
\end{document}
%    \end{macrocode}
%\iffalse
%</samplemain>
%\fi
%
% %%%%%%%%%%%%%%%%%%%%%%%%%%%%%%%%%%%%%%
% \paragraph{Chapter Include Files.}
%
% The include files are called |cdocsch1.tex| and |cdocsch2.tex|.
%
%\iffalse
%<*samplechap1|samplechap2>
%\fi

% Optional override for |\version| flag:
%    \begin{macrocode}
%%\providecommand{\version}{final}
%    \end{macrocode}

% Include the main document:
%    \begin{macrocode}
\input{childdoc.def}
\childdocof{cdocsamp}
%    \end{macrocode}

%\iffalse
%</samplechap1|samplechap2>
%\fi
%
%\iffalse
%<*samplechap1>
%\fi
% Some text for chapter 1:
%    \begin{macrocode}
\section{one}
some text in chapter one
%    \end{macrocode}

%\iffalse
%</samplechap1>
%\fi
% Some text for chapter 2:
%\iffalse
%<*samplechap2>
%\fi
%    \begin{macrocode}
\section{two}
more text in chapter two
%    \end{macrocode}

%\iffalse
%</samplechap2>
%\fi
%
% %%%%%%%%%%%%%%%%%%%%%%%%%%%%%%%%%%%%%%
% \paragraph{Part Include Files.}
%
% The include files are called |cdocspt3.tex| and |cdocspt4.tex|.
%
%\iffalse
%<*samplepart3|samplepart4>
%\fi

% Optional override for |\version| flag:
%    \begin{macrocode}
%%\providecommand{\version}{final}
%    \end{macrocode}

% Include the main document:
%    \begin{macrocode}
\input{childdoc.def}
\childdocby{cdocsamp}
%    \end{macrocode}

%\iffalse
%</samplepart3|samplepart4>
%\fi
%
%\iffalse
%<*samplepart3>
%\fi
% Some text for part 3:
%    \begin{macrocode}
some text in part three
%    \end{macrocode}

%\iffalse
%</samplepart3>
%\fi
% Some text for part 4:
%\iffalse
%<*samplepart4>
%\fi
%    \begin{macrocode}
more text in part four
%    \end{macrocode}

%\iffalse
%</samplepart4>
%\fi
%
% %%%%%%%%%%%%%%%%%%%%%%%%%%%%%%%%%%%%%%
% \paragraph{Forwarding for a Complete Draft.}
%
% The following forwarding file |cdocsdrf.tex|
% compiles the main document in draft mode:
%\iffalse
%<*sampledraft>
%\fi
%    \begin{macrocode}
\def\version{draft}
\input{childdoc.def}
\childdocforward{cdocsamp}
%    \end{macrocode}

%\iffalse
%</sampledraft>
%\fi
%
% %%%%%%%%%%%%%%%%%%%%%%%%%%%%%%%%%%%%%%
% \paragraph{Forwarding for Final Version of the Chapters.}
%
% The following forwarding files |cdocsfn1.tex| and |cdocsfn2.tex|
% (with identical content)
% compile the final versions of the child documents
% |cdocsch1.tex| and |cdocsch2.tex|, respectively:
%\iffalse
%<*samplefinal>
%\fi
%    \begin{macrocode}
\def\version{final}
\input{childdoc.def}
\childdocforwardprefix[cdocsamp]{cdocsfn}{cdocsch}
%    \end{macrocode}

%\iffalse
%</samplefinal>
%\fi
%
% %%%%%%%%%%%%%%%%%%%%%%%%%%%%%%%%%%%%%%
% \paragraph{Command Line Processing.}
%
% The following three command lines generate the output files
% |cdocscld|, |cdocscl1| and |cdocscl2|
% which should be identical to
% |cdocsdrf|, |cdocsch1| and |cdocsfn2|, respectively:
% \begin{center}
% \begin{tabular}{l}
% |latex -jobname cdocscld \|\\
% |  "\def\version{draft}\input{childdoc.def}\childdocforward{cdocsamp}"|\\
% |latex -jobname cdocscl1 \|\\
% |  "\input{childdoc.def}\childdocforward[cdocsamp]{cdocsch1}"|\\
% |latex -jobname cdocscl2 \|\\
% |  "\def\version{final}\input{childdoc.def}\childdocforward{cdocsch2}"|
% \end{tabular}
% \end{center}
% Note that the trailing backslash on each first line
% merely continues the input to the second line
% (for convenient cut ant paste).
% Furthermore, the command |latex| can be replaced by any
% of its alternative versions such as |pdflatex|.
%
% %%%%%%%%%%%%%%%%%%%%%%%%%%%%%%%%%%%%%%%%%%%%%%%%%%%%%%%%%%%%%%%%%%%%%%%%%%%%%%
% %%%%%%%%%%%%%%%%%%%%%%%%%%%%%%%%%%%%%%%%%%%%%%%%%%%%%%%%%%%%%%%%%%%%%%%%%%%%%%
% \section{Implementation}
%\iffalse
%<*package>
%\fi
%
% This section describes the definitions file |childdoc.def|.

% The definitions cannot be loaded using |\usepackage| or |\RequirePackage|
% which has a mechanism to prevent loading a style file more than once.
% When loading the definitions by means of |\input|
% multiple instances have to be prevented manually:
%\iffalse
%This code needs to be before the `\ProvidesFile' directive
%which is defined at the beginning of this file.
%Therefore it is also placed there and commented out here.
%</package>
%<*discard>
%\fi
%    \begin{macrocode}
\ifdefined\childdocmain\endinput\fi
%    \end{macrocode}
%\iffalse
%</discard>
%<*package>
%\fi
%
% \macro{\ifchilddoc}
% \macro{\ifchilddocmanual}
% The conditional |\ifchilddoc| tells whether a
% child (true) or main (false) document is being compiled.
% The conditional |\ifchilddocmanual| tells whether
% the |\includeonly| mechanism is used (false) or
% the selection of child files must be performed manually (true).
% The definitions initialise to false:
%    \begin{macrocode}
\newif\ifchilddoc
\newif\ifchilddocmanual
%    \end{macrocode}

% \macro{\childdocname}
% \macro{\childdocjob}
% The macro |\childdocname| stores the name of the main document
% to be compiled. The macro |\childdocjob| stores the name of
% the document on which the \LaTeX{} compiler was originally invoked.
% The content of |\jobname| cannot be compared
% to filenames specified in the source due to different catcodes.
% The following code rescans |\jobname|, stores the result
% in |\childdocname| and saves a copy in |\childdocjob|:
%    \begin{macrocode}
\edef\childdocname{\scantokens\expandafter{\jobname\noexpand}}
\let\childdocjob\childdocname
%    \end{macrocode}

% \macro{\childdocdisable}
% The macro |\childdocdisable| prevents the main file
% from being processed more than once.
% At this stage, the main document command |\childdocmain|
% is assumed to be called once again where it should do nothing.
% Any subsequent call to it should prevent
% a secondary processing of the main document
% It overwrites the forwarding commands
% |\childdocof| and |\childdocforward|
% with empty macros to prevent further inclusions of the main document:
%    \begin{macrocode}
\newcommand{\childdocdisable}
{
  \renewcommand{\childdocmain}[1]{\renewcommand{\childdocmain}[1]{\endinput}}
  \renewcommand{\childdocof}[1]{}
  \renewcommand{\childdocby}[2][]{}
  \renewcommand{\childdocforward}[2][]{}
  \renewcommand{\childdocdisable}{}
}
%    \end{macrocode}

% \macro{\childdocmain}
% The macro |\childdocmain| is to be called at the top of the main file
% with nothing or the main filename (without extension) as argument.
% First, it breaks loops.
% If the argument is not empty and does not match |\childdocname|
% (which is set by the first inclusion of |childdoc.def|),
% |\ifchilddoc| is set to true, |\includeonly| is applied to the child file
% and |\jobname| is set to the main file
% (for proper handling of |.aux| files):
%    \begin{macrocode}
\newcommand{\childdocmain}[1]
{
  \childdocdisable\childdocmain{}
  \if?#1?\else
    \begingroup
      \def\childdoctmp{#1}
      \ifx\childdoctmp\childdocname
        \def\childdoctmp{}
      \else
        \def\childdoctmp
        {
          \childdoctrue
          \includeonly{\childdocname}
          \def\childdocjob{#1}
          \def\jobname{#1}
        }
      \fi
      \expandafter
    \endgroup
    \childdoctmp
  \fi
}
%    \end{macrocode}

% \macro{\childdocof}
% The command |\childdocof| redirects
% compilation to the main file |#1|.
%    \begin{macrocode}
\newcommand{\childdocof}[1]
{
  \childdocdisable
  \childdoctrue
  \includeonly{\childdocname}
  \def\jobname{#1}
  \def\childdocjob{#1}
  \input{#1}
}
%    \end{macrocode}

% \macro{\childdocby}
% The command |\childdocby| ....
%    \begin{macrocode}
\newcommand{\childdocby}[2][]
{
  \childdocdisable
  \childdoctrue
  \childdocmanualtrue
  \if?#1?\else
    \def\jobname{#2}
  \fi
  \def\childdocjob{#2}
  \input{#2}
  \endinput
}
%    \end{macrocode}

% \macro{\childdocforward}
% The command |\childdocforward| redirects
% compilation to the main file or
% (if the optional argument is given) a child file.
% Parameters are set as if the main file
% or a child file starting with |\childdocof| was compiled.
% Then compilation is handed over to the main file:
%    \begin{macrocode}
\newcommand{\childdocforward}[2][]
{
  \begingroup
    \if?#1?
      \def\childdoctmp
      {
        \def\childdocname{#2}
        \def\childdocjob{#2}
        \def\jobname{#2}
        \input{#2}
        \endinput
      }
    \else
      \def\childdoctmp
      {
        \childdocdisable
        \def\childdocname{#2}
        \childdoctrue
        \includeonly{#2}
        \def\childdocjob{#1}
        \def\jobname{#1}
        \input{#1}
        \endinput
      }
    \fi
    \expandafter
  \endgroup
  \childdoctmp
}
%    \end{macrocode}

% \macro{\childdocforwardprefix}
% The command |\childdocforwardprefix| redirects
% compilation to the main or a child file by means of a pattern.
% The prefix |#1| in the current filename is replaced by |#2|
% and the suffix of the current filename is kept
% (it is assumed that the filename does not contain the substring `|~~~|'
% which is used as a delimiter).
% Compilation is handed over to the new file by |\childdocforward|:
%    \begin{macrocode}
\newcommand{\childdocforwardprefix}[3][]
{
  \begingroup
    \def\childdocextract #2##1~~~{\def\childdoctmp{\childdocforward[#1]{#3##1}}}
    \expandafter\childdocextract\childdocname~~~
    \expandafter
  \endgroup
  \childdoctmp
}
%    \end{macrocode}

% \macro{\childdoc}
% The deprecated macro |\childdoc| is a legacy version of |\childdocmain|:
%    \begin{macrocode}
\newcommand{\childdoc}{\childdocmain}
%    \end{macrocode}

% \macro{\childdocredirect}
% The deprecated macro |\childdocredirect| is a legacy version
% of |\childdocforward| and |\childdocforwardprefix|:
%    \begin{macrocode}
\newcommand{\childdocredirect}[2][]
{
  \begingroup
    \if?#1?
      \def\childdoctmp{\childdocforward{#2}}
    \else
      \def\childdoctmp{\childdocforwardprefix{#1}{#2}}
    \fi
    \expandafter
  \endgroup
  \childdoctmp
}
%    \end{macrocode}

%\iffalse
%</package>
%\fi
%
\endinput
\childdocforward{cdocsch2}"|
% \end{tabular}
% \end{center}
% Note that the trailing backslash on each first line
% merely continues the input to the second line
% (for convenient cut ant paste).
% Furthermore, the command |latex| can be replaced by any
% of its alternative versions such as |pdflatex|.
%
% %%%%%%%%%%%%%%%%%%%%%%%%%%%%%%%%%%%%%%%%%%%%%%%%%%%%%%%%%%%%%%%%%%%%%%%%%%%%%%
% %%%%%%%%%%%%%%%%%%%%%%%%%%%%%%%%%%%%%%%%%%%%%%%%%%%%%%%%%%%%%%%%%%%%%%%%%%%%%%
% \section{Implementation}
%\iffalse
%<*package>
%\fi
%
% This section describes the definitions file |childdoc.def|.

% The definitions cannot be loaded using |\usepackage| or |\RequirePackage|
% which has a mechanism to prevent loading a style file more than once.
% When loading the definitions by means of |\input|
% multiple instances have to be prevented manually:
%\iffalse
%This code needs to be before the `\ProvidesFile' directive
%which is defined at the beginning of this file.
%Therefore it is also placed there and commented out here.
%</package>
%<*discard>
%\fi
%    \begin{macrocode}
\ifdefined\childdocmain\endinput\fi
%    \end{macrocode}
%\iffalse
%</discard>
%<*package>
%\fi
%
% \macro{\ifchilddoc}
% \macro{\ifchilddocmanual}
% The conditional |\ifchilddoc| tells whether a
% child (true) or main (false) document is being compiled.
% The conditional |\ifchilddocmanual| tells whether
% the |\includeonly| mechanism is used (false) or
% the selection of child files must be performed manually (true).
% The definitions initialise to false:
%    \begin{macrocode}
\newif\ifchilddoc
\newif\ifchilddocmanual
%    \end{macrocode}

% \macro{\childdocname}
% \macro{\childdocjob}
% The macro |\childdocname| stores the name of the main document
% to be compiled. The macro |\childdocjob| stores the name of
% the document on which the \LaTeX{} compiler was originally invoked.
% The content of |\jobname| cannot be compared
% to filenames specified in the source due to different catcodes.
% The following code rescans |\jobname|, stores the result
% in |\childdocname| and saves a copy in |\childdocjob|:
%    \begin{macrocode}
\edef\childdocname{\scantokens\expandafter{\jobname\noexpand}}
\let\childdocjob\childdocname
%    \end{macrocode}

% \macro{\childdocdisable}
% The macro |\childdocdisable| prevents the main file
% from being processed more than once.
% At this stage, the main document command |\childdocmain|
% is assumed to be called once again where it should do nothing.
% Any subsequent call to it should prevent
% a secondary processing of the main document
% It overwrites the forwarding commands
% |\childdocof| and |\childdocforward|
% with empty macros to prevent further inclusions of the main document:
%    \begin{macrocode}
\newcommand{\childdocdisable}
{
  \renewcommand{\childdocmain}[1]{\renewcommand{\childdocmain}[1]{\endinput}}
  \renewcommand{\childdocof}[1]{}
  \renewcommand{\childdocby}[2][]{}
  \renewcommand{\childdocforward}[2][]{}
  \renewcommand{\childdocdisable}{}
}
%    \end{macrocode}

% \macro{\childdocmain}
% The macro |\childdocmain| is to be called at the top of the main file
% with nothing or the main filename (without extension) as argument.
% First, it breaks loops.
% If the argument is not empty and does not match |\childdocname|
% (which is set by the first inclusion of |childdoc.def|),
% |\ifchilddoc| is set to true, |\includeonly| is applied to the child file
% and |\jobname| is set to the main file
% (for proper handling of |.aux| files):
%    \begin{macrocode}
\newcommand{\childdocmain}[1]
{
  \childdocdisable\childdocmain{}
  \if?#1?\else
    \begingroup
      \def\childdoctmp{#1}
      \ifx\childdoctmp\childdocname
        \def\childdoctmp{}
      \else
        \def\childdoctmp
        {
          \childdoctrue
          \includeonly{\childdocname}
          \def\childdocjob{#1}
          \def\jobname{#1}
        }
      \fi
      \expandafter
    \endgroup
    \childdoctmp
  \fi
}
%    \end{macrocode}

% \macro{\childdocof}
% The command |\childdocof| redirects
% compilation to the main file |#1|.
%    \begin{macrocode}
\newcommand{\childdocof}[1]
{
  \childdocdisable
  \childdoctrue
  \includeonly{\childdocname}
  \def\jobname{#1}
  \def\childdocjob{#1}
  \input{#1}
}
%    \end{macrocode}

% \macro{\childdocby}
% The command |\childdocby| ....
%    \begin{macrocode}
\newcommand{\childdocby}[2][]
{
  \childdocdisable
  \childdoctrue
  \childdocmanualtrue
  \if?#1?\else
    \def\jobname{#2}
  \fi
  \def\childdocjob{#2}
  \input{#2}
  \endinput
}
%    \end{macrocode}

% \macro{\childdocforward}
% The command |\childdocforward| redirects
% compilation to the main file or
% (if the optional argument is given) a child file.
% Parameters are set as if the main file
% or a child file starting with |\childdocof| was compiled.
% Then compilation is handed over to the main file:
%    \begin{macrocode}
\newcommand{\childdocforward}[2][]
{
  \begingroup
    \if?#1?
      \def\childdoctmp
      {
        \def\childdocname{#2}
        \def\childdocjob{#2}
        \def\jobname{#2}
        \input{#2}
        \endinput
      }
    \else
      \def\childdoctmp
      {
        \childdocdisable
        \def\childdocname{#2}
        \childdoctrue
        \includeonly{#2}
        \def\childdocjob{#1}
        \def\jobname{#1}
        \input{#1}
        \endinput
      }
    \fi
    \expandafter
  \endgroup
  \childdoctmp
}
%    \end{macrocode}

% \macro{\childdocforwardprefix}
% The command |\childdocforwardprefix| redirects
% compilation to the main or a child file by means of a pattern.
% The prefix |#1| in the current filename is replaced by |#2|
% and the suffix of the current filename is kept
% (it is assumed that the filename does not contain the substring `|~~~|'
% which is used as a delimiter).
% Compilation is handed over to the new file by |\childdocforward|:
%    \begin{macrocode}
\newcommand{\childdocforwardprefix}[3][]
{
  \begingroup
    \def\childdocextract #2##1~~~{\def\childdoctmp{\childdocforward[#1]{#3##1}}}
    \expandafter\childdocextract\childdocname~~~
    \expandafter
  \endgroup
  \childdoctmp
}
%    \end{macrocode}

% \macro{\childdoc}
% The deprecated macro |\childdoc| is a legacy version of |\childdocmain|:
%    \begin{macrocode}
\newcommand{\childdoc}{\childdocmain}
%    \end{macrocode}

% \macro{\childdocredirect}
% The deprecated macro |\childdocredirect| is a legacy version
% of |\childdocforward| and |\childdocforwardprefix|:
%    \begin{macrocode}
\newcommand{\childdocredirect}[2][]
{
  \begingroup
    \if?#1?
      \def\childdoctmp{\childdocforward{#2}}
    \else
      \def\childdoctmp{\childdocforwardprefix{#1}{#2}}
    \fi
    \expandafter
  \endgroup
  \childdoctmp
}
%    \end{macrocode}

%\iffalse
%</package>
%\fi
%
\endinput

\childdocmain{}
%    \end{macrocode}

% Optional override for |\version| flag:
%    \begin{macrocode}
%%\ifchilddoc\else\providecommand{\version}{draft}\fi
%    \end{macrocode}

% Define the default values for the |\version| flag
% (|final| for the main file and |draft| for childs):
%    \begin{macrocode}
\ifchilddoc
\providecommand{\version}{draft}
\else
\providecommand{\version}{final}
\fi
%    \end{macrocode}

% Load the standard document class:
%    \begin{macrocode}
\documentclass[12pt]{article}
%    \end{macrocode}

% Start the document body:
%    \begin{macrocode}
\begin{document}
%    \end{macrocode}

% Declare a title page.
% Print title, part of document being processed and version flag:
%    \begin{macrocode}
\addtocounter{page}{-1}
\begin{center}
{\LARGE\bfseries{}childdoc example\par}
\vspace{1cm}
\ifchilddoc
\ifchilddocmanual part\else chapter\fi:
`\childdocname' of `\childdocjob'\par
\else
main document: `\childdocjob'\par
\fi
version: \version\par
\end{center}
\newpage
%    \end{macrocode}

% Manually include selected file,
% otherwise process as usual:
%    \begin{macrocode}
\ifchilddocmanual
\section*{part `\childdocname'}
\input{\childdocname}
\else
%    \end{macrocode}

% Include the two chapters:
%    \begin{macrocode}
\include{cdocsch1}
\include{cdocsch2}
%    \end{macrocode}

% Include the two parts unless only chapters should be displayed:
%    \begin{macrocode}
\ifchilddoc\else
\section{part three}
\input{cdocspt3}
\section{part four}
\input{cdocspt4}
\fi
%    \end{macrocode}

% Process as usual until here:
%    \begin{macrocode}
\fi
%    \end{macrocode}

% End of document body:
%    \begin{macrocode}
\end{document}
%    \end{macrocode}
%\iffalse
%</samplemain>
%\fi
%
% %%%%%%%%%%%%%%%%%%%%%%%%%%%%%%%%%%%%%%
% \paragraph{Chapter Include Files.}
%
% The include files are called |cdocsch1.tex| and |cdocsch2.tex|.
%
%\iffalse
%<*samplechap1|samplechap2>
%\fi

% Optional override for |\version| flag:
%    \begin{macrocode}
%%\providecommand{\version}{final}
%    \end{macrocode}

% Include the main document:
%    \begin{macrocode}
% \iffalse
%
% childdoc.dtx Copyright (C) 2017-2018 Niklas Beisert
%
% This work may be distributed and/or modified under the
% conditions of the LaTeX Project Public License, either version 1.3
% of this license or (at your option) any later version.
% The latest version of this license is in
%   http://www.latex-project.org/lppl.txt
% and version 1.3 or later is part of all distributions of LaTeX
% version 2005/12/01 or later.
%
% This work has the LPPL maintenance status `maintained'.
%
% The Current Maintainer of this work is Niklas Beisert.
%
% This work consists of the files childdoc.dtx and childdoc.ins
% and the derived files childdoc.def and cdocsamp.tex with
% cdocsch1.tex, cdocsch2.tex, cdocsdrf.tex, cdocsfn1.tex, cdocsfn2.tex.
%
%<package>\ifdefined\childdocmain\endinput\fi
%<package>\ProvidesFile{childdoc.def}[2018/12/30 v2.0 child document driver]
%<samplemain>\ProvidesFile{cdocsamp.tex}[2018/12/30 v2.0 sample for childdoc]
%<*driver>
%\ProvidesFile{childdoc.drv}[2018/12/30 v2.0 childdoc reference manual file]
\PassOptionsToClass{10pt,a4paper}{article}
\documentclass{ltxdoc}

\usepackage[margin=35mm]{geometry}
\usepackage{hyperref}
\usepackage{hyperxmp}
\usepackage[usenames]{color}

\hypersetup{colorlinks=true}
\hypersetup{pdfstartview=FitH}
\hypersetup{pdfpagemode=UseNone}
\hypersetup{pdfsource={}}
\hypersetup{pdflang={en-UK}}
\hypersetup{pdfcopyright={Copyright 2017-2018 Niklas Beisert.
  This work may be distributed and/or modified under the
  conditions of the LaTeX Project Public License, either version 1.3
  of this license or (at your option) any later version.}}
\hypersetup{pdflicenseurl={http://www.latex-project.org/lppl.txt}}
\hypersetup{pdfcontactaddress={ETH Zurich, ITP, HIT K,
  Wolfgang-Pauli-Strasse 27}}
\hypersetup{pdfcontactpostcode={8093}}
\hypersetup{pdfcontactcity={Zurich}}
\hypersetup{pdfcontactcountry={Switzerland}}
\hypersetup{pdfcontactemail={nbeisert@itp.phys.ethz.ch}}
\hypersetup{pdfcontacturl={http://people.phys.ethz.ch/\xmptilde nbeisert/}}

\newcommand{\secref}[1]{\hyperref[#1]{section \ref*{#1}}}

\parskip1ex
\parindent0pt
\let\olditemize\itemize
\def\itemize{\olditemize\parskip0pt}

\begin{document}

\title{The \textsf{childdoc} Package}
\hypersetup{pdftitle={The childdoc Package}}
\author{Niklas Beisert\\[2ex]
  Institut f\"ur Theoretische Physik\\
  Eidgen\"ossische Technische Hochschule Z\"urich\\
  Wolfgang-Pauli-Strasse 27, 8093 Z\"urich, Switzerland\\[1ex]
  \href{mailto:nbeisert@itp.phys.ethz.ch}
  {\texttt{nbeisert@itp.phys.ethz.ch}}}
\hypersetup{pdfauthor={Niklas Beisert}}
\hypersetup{pdfsubject={Manual for the LaTeX2e Package childdoc}}
\date{30 December 2018, \textsf{v2.0}}
\maketitle

\begin{abstract}\noindent
\textsf{childdoc} is a \LaTeXe{} package
that enables the direct compilation
of document sections included by |\include|
to individual files.
\end{abstract}

\begingroup
\parskip0ex
\tableofcontents
\endgroup

%%%%%%%%%%%%%%%%%%%%%%%%%%%%%%%%%%%%%%%%%%%%%%%%%%%%%%%%%%%%%%%%%%%%%%%%%%%%%%%%
%%%%%%%%%%%%%%%%%%%%%%%%%%%%%%%%%%%%%%%%%%%%%%%%%%%%%%%%%%%%%%%%%%%%%%%%%%%%%%%%
\section{Introduction}

\LaTeX{} provides a mechanism to structure a large document (such as a book)
into a main file and several child files (containing the chapters)
using the |\include| command.
This mechanism is beneficial for documents
which span hundreds of pages in order to
make the source file(s) more manageable.
Moreover, compilation can be restricted to
selected child files by means of the |\includeonly| command.
The latter feature can be used to reduce the compilation time while editing
(this was significantly more useful in the earlier days of \LaTeX{})
or to generate a smaller document which is easier to navigate.
Another application of |\includeonly| is to generate
documents consisting of selected parts of the complete document.

However, there are a few drawbacks of the plain |\include| mechanism:
\begin{itemize}
\item
The child files cannot be compiled on their own,
they can only be compiled via the main file.
A naive editing environment
(such as a text editor with an option
to have the current file processed by \LaTeX)
may require one to switch to the main file before compiling;
attempting to compile the child file produces errors.
\item
The main file must be modified (each time)
to adjust the |\includeonly| command
to the present needs. This easily leaves the main file in a messy state.
\item
The generated document will always carry the filename
of the main document. This is inconvenient if
several child files are to be compiled and
to be kept for distribution.
\end{itemize}

The present package provides a simple interface
to make child files individually compilable by \LaTeX{}.
Compiling a child file then has the same effect as compiling
the main file with an |\includeonly| command
to select the appropriate child.
Moreover the generated document will carry the name of the child
rather than the main file.
This resolves all three above issues.

This feature is meant to make the editing of books,
thesis documents and lecture notes somewhat more convenient.
However, the package can also be used efficiently for
composing a series of documents (such as exercise sheets)
which are typically distributed individually.
It then assists the author in generating the individual documents
(potentially in different versions)
as well as a document containing the collected series.
Another application is in developing style files
or other kinds of included material
where compilation of the style file could redirect
to a sample or test file.

%%%%%%%%%%%%%%%%%%%%%%%%%%%%%%%%%%%%%%%%%%%%%%%%%%%%%%%%%%%%%%%%%%%%%%%%%%%%%%%%
%%%%%%%%%%%%%%%%%%%%%%%%%%%%%%%%%%%%%%%%%%%%%%%%%%%%%%%%%%%%%%%%%%%%%%%%%%%%%%%%
\section{Usage}

First of all, the package \textsf{childdoc} is \emph{not} a standard
\LaTeXe{} |.sty| style file! Therefore it needs to be invoked in
a non-standard way.

%%%%%%%%%%%%%%%%%%%%%%%%%%%%%%%%%%%%%%%%%%%%%%%%%%%%%%%%%%%%%%%%%%%%%%%%%%%%%%%%
\subsection{Included Files}
\label{sec:include}

%%%%%%%%%%%%%%%%%%%%%%%%%%%%%%%%%%%%%%%%
\DescribeMacro{\childdocmain}
To use the package, add the commands
\begin{center}
\begin{tabular}{l}
|% \iffalse
%
% childdoc.dtx Copyright (C) 2017-2018 Niklas Beisert
%
% This work may be distributed and/or modified under the
% conditions of the LaTeX Project Public License, either version 1.3
% of this license or (at your option) any later version.
% The latest version of this license is in
%   http://www.latex-project.org/lppl.txt
% and version 1.3 or later is part of all distributions of LaTeX
% version 2005/12/01 or later.
%
% This work has the LPPL maintenance status `maintained'.
%
% The Current Maintainer of this work is Niklas Beisert.
%
% This work consists of the files childdoc.dtx and childdoc.ins
% and the derived files childdoc.def and cdocsamp.tex with
% cdocsch1.tex, cdocsch2.tex, cdocsdrf.tex, cdocsfn1.tex, cdocsfn2.tex.
%
%<package>\ifdefined\childdocmain\endinput\fi
%<package>\ProvidesFile{childdoc.def}[2018/12/30 v2.0 child document driver]
%<samplemain>\ProvidesFile{cdocsamp.tex}[2018/12/30 v2.0 sample for childdoc]
%<*driver>
%\ProvidesFile{childdoc.drv}[2018/12/30 v2.0 childdoc reference manual file]
\PassOptionsToClass{10pt,a4paper}{article}
\documentclass{ltxdoc}

\usepackage[margin=35mm]{geometry}
\usepackage{hyperref}
\usepackage{hyperxmp}
\usepackage[usenames]{color}

\hypersetup{colorlinks=true}
\hypersetup{pdfstartview=FitH}
\hypersetup{pdfpagemode=UseNone}
\hypersetup{pdfsource={}}
\hypersetup{pdflang={en-UK}}
\hypersetup{pdfcopyright={Copyright 2017-2018 Niklas Beisert.
  This work may be distributed and/or modified under the
  conditions of the LaTeX Project Public License, either version 1.3
  of this license or (at your option) any later version.}}
\hypersetup{pdflicenseurl={http://www.latex-project.org/lppl.txt}}
\hypersetup{pdfcontactaddress={ETH Zurich, ITP, HIT K,
  Wolfgang-Pauli-Strasse 27}}
\hypersetup{pdfcontactpostcode={8093}}
\hypersetup{pdfcontactcity={Zurich}}
\hypersetup{pdfcontactcountry={Switzerland}}
\hypersetup{pdfcontactemail={nbeisert@itp.phys.ethz.ch}}
\hypersetup{pdfcontacturl={http://people.phys.ethz.ch/\xmptilde nbeisert/}}

\newcommand{\secref}[1]{\hyperref[#1]{section \ref*{#1}}}

\parskip1ex
\parindent0pt
\let\olditemize\itemize
\def\itemize{\olditemize\parskip0pt}

\begin{document}

\title{The \textsf{childdoc} Package}
\hypersetup{pdftitle={The childdoc Package}}
\author{Niklas Beisert\\[2ex]
  Institut f\"ur Theoretische Physik\\
  Eidgen\"ossische Technische Hochschule Z\"urich\\
  Wolfgang-Pauli-Strasse 27, 8093 Z\"urich, Switzerland\\[1ex]
  \href{mailto:nbeisert@itp.phys.ethz.ch}
  {\texttt{nbeisert@itp.phys.ethz.ch}}}
\hypersetup{pdfauthor={Niklas Beisert}}
\hypersetup{pdfsubject={Manual for the LaTeX2e Package childdoc}}
\date{30 December 2018, \textsf{v2.0}}
\maketitle

\begin{abstract}\noindent
\textsf{childdoc} is a \LaTeXe{} package
that enables the direct compilation
of document sections included by |\include|
to individual files.
\end{abstract}

\begingroup
\parskip0ex
\tableofcontents
\endgroup

%%%%%%%%%%%%%%%%%%%%%%%%%%%%%%%%%%%%%%%%%%%%%%%%%%%%%%%%%%%%%%%%%%%%%%%%%%%%%%%%
%%%%%%%%%%%%%%%%%%%%%%%%%%%%%%%%%%%%%%%%%%%%%%%%%%%%%%%%%%%%%%%%%%%%%%%%%%%%%%%%
\section{Introduction}

\LaTeX{} provides a mechanism to structure a large document (such as a book)
into a main file and several child files (containing the chapters)
using the |\include| command.
This mechanism is beneficial for documents
which span hundreds of pages in order to
make the source file(s) more manageable.
Moreover, compilation can be restricted to
selected child files by means of the |\includeonly| command.
The latter feature can be used to reduce the compilation time while editing
(this was significantly more useful in the earlier days of \LaTeX{})
or to generate a smaller document which is easier to navigate.
Another application of |\includeonly| is to generate
documents consisting of selected parts of the complete document.

However, there are a few drawbacks of the plain |\include| mechanism:
\begin{itemize}
\item
The child files cannot be compiled on their own,
they can only be compiled via the main file.
A naive editing environment
(such as a text editor with an option
to have the current file processed by \LaTeX)
may require one to switch to the main file before compiling;
attempting to compile the child file produces errors.
\item
The main file must be modified (each time)
to adjust the |\includeonly| command
to the present needs. This easily leaves the main file in a messy state.
\item
The generated document will always carry the filename
of the main document. This is inconvenient if
several child files are to be compiled and
to be kept for distribution.
\end{itemize}

The present package provides a simple interface
to make child files individually compilable by \LaTeX{}.
Compiling a child file then has the same effect as compiling
the main file with an |\includeonly| command
to select the appropriate child.
Moreover the generated document will carry the name of the child
rather than the main file.
This resolves all three above issues.

This feature is meant to make the editing of books,
thesis documents and lecture notes somewhat more convenient.
However, the package can also be used efficiently for
composing a series of documents (such as exercise sheets)
which are typically distributed individually.
It then assists the author in generating the individual documents
(potentially in different versions)
as well as a document containing the collected series.
Another application is in developing style files
or other kinds of included material
where compilation of the style file could redirect
to a sample or test file.

%%%%%%%%%%%%%%%%%%%%%%%%%%%%%%%%%%%%%%%%%%%%%%%%%%%%%%%%%%%%%%%%%%%%%%%%%%%%%%%%
%%%%%%%%%%%%%%%%%%%%%%%%%%%%%%%%%%%%%%%%%%%%%%%%%%%%%%%%%%%%%%%%%%%%%%%%%%%%%%%%
\section{Usage}

First of all, the package \textsf{childdoc} is \emph{not} a standard
\LaTeXe{} |.sty| style file! Therefore it needs to be invoked in
a non-standard way.

%%%%%%%%%%%%%%%%%%%%%%%%%%%%%%%%%%%%%%%%%%%%%%%%%%%%%%%%%%%%%%%%%%%%%%%%%%%%%%%%
\subsection{Included Files}
\label{sec:include}

%%%%%%%%%%%%%%%%%%%%%%%%%%%%%%%%%%%%%%%%
\DescribeMacro{\childdocmain}
To use the package, add the commands
\begin{center}
\begin{tabular}{l}
|\input{childdoc.def}|\\
|\childdocmain{}|\\
\end{tabular}
\end{center}
at the very top of the main \LaTeX{} file,
in particular \emph{before} the |\documentclass| statement!
The argument of |\childdocmain| should be left empty
(but it must be present).

%%%%%%%%%%%%%%%%%%%%%%%%%%%%%%%%%%%%%%%%
\DescribeMacro{\childdocof}
Furthermore, add the commands
\begin{center}
\begin{tabular}{l}
|\input{childdoc.def}|\\
|\childdocof{|\textit{main}|}|\\
\end{tabular}
\end{center}
at the top of every child file \textit{child}
which is included by |\include{|\textit{child}|}|
from within the main file
(or at least for those files to be compiled individually).
The argument \textit{main} must be the filename of the main file.

There are a couple of
considerations in setting up the main and child documents:

%%%%%%%%%%%%%%%%%%%%%%%%%%%%%%%%%%%%%%%%
\paragraph{Restrictions.}

Please note the following restrictions:
\begin{itemize}
\item
|\childdocmain| must be called with one argument \textit{main}
to ensure compatibility with earlier version of the package.
It must either be empty (|\childdocmain{}|)
or precisely match the filename of the main file in which it is specified.
See \secref{sec:detection} for further information.
\item
The filename \textit{main} must be specified without the |.tex| extension.
\item
The filename \textit{main} is case sensitive
(even in case-insensitive file systems)
due to internal string comparison.
\item
The argument \textit{main} should be fully expanded, it cannot be a macro.
\item
Subdirectories and special characters should be avoided in filenames.
\item
The command |\childdocmain{|\textit{main}|}| must be followed by a whitespace.
It should not be followed immediately by another command
or by a comment mark `|%|'.
This is because the \TeX{} parser reads the token immediately following
the argument of |\childdocmain| and puts it
at the beginning of every child section;
however, a white\-space is ignored.
\end{itemize}

%%%%%%%%%%%%%%%%%%%%%%%%%%%%%%%%%%%%%%%%
\paragraph{Content of Main File.}

It is advisable to place all content in the child files included by |\include|.
Any output contained in the main file will appear in all child documents
unless suppressed manually;
it cannot be suppressed automatically by the |\includeonly| directive
and thus should normally be avoided.
A method to include some content in the main file
by means of conditional processing is described in \secref{sec:conditional}.

%%%%%%%%%%%%%%%%%%%%%%%%%%%%%%%%%%%%%%%%
\paragraph{Page Numbering.}

When only a part of the document is compiled,
the appropriate numbering of pages
(as well as other status parameters)
is determined from the |.aux| files.
The latter contain information from previous passes.
However this information needs to propagate through
all intermediate child documents.
Therefore the page numbering in child documents may well
be inconsistent until the complete document is compiled at least once.

A useful (if unconventional) way to always ensure a consistent
page numbering is to restart the numbering in each child document
and denote the pages by `\textit{child}|.|\textit{page}'
where \textit{child} represents the chapter/section number of the child file.
This can be achieved by the command
|\numberwithin{page}{|\textit{child}|}|
of the \textsf{amsmath} package
where \textit{child} can be |chapter| or |section|
depending on the chosen structuring.
Alternatively, one can modify the macro |\thepage| appropriately
and reset the counter |page| at the start of each child file.

%%%%%%%%%%%%%%%%%%%%%%%%%%%%%%%%%%%%%%%%%%%%%%%%%%%%%%%%%%%%%%%%%%%%%%%%%%%%%%%%
\subsection{Conditional Processing}
\label{sec:conditional}

The package provides a mechanism to compile different versions
of a document. To customise the versions further some conditional processing
can come in handy to distinguish which version is being compiled.
The package provides two macros to describe the compilation context:

%%%%%%%%%%%%%%%%%%%%%%%%%%%%%%%%%%%%%%%%
\DescribeMacro{\ifchilddoc}
The conditional |\ifchilddoc| distinguishes between the compilation of
child documents and the main document:
%
\begin{center}
|\ifchilddoc |\textit{child-code}| |[|\||else |\textit{main-code}]| \||fi|
\end{center}

%%%%%%%%%%%%%%%%%%%%%%%%%%%%%%%%%%%%%%%%
\DescribeMacro{\childdocname}
\DescribeMacro{\childdocjob}
The macro |\childdocname| contains the filename (without extension)
of the main or child file being processed.
Note that |\childdocjob| will always contain the name of the main file.

%%%%%%%%%%%%%%%%%%%%%%%%%%%%%%%%%%%%%%%%
\paragraph{Title Page.}

Conditional processing can be used to include a title or banner page
in the main document when proper precautions are taken.
Importantly, the code in the main file should ensure that the page counter
(as well as other status parameters which are stored in the |.aux| files)
takes the same value after the conditional processing.
Otherwise the page numbers may take divergent values
depending on which part is compiled.

For example, a title page could be declared by:
%
\begin{center}
\begin{tabular}{l}
|\ifchilddoc\||else|\\
|\addtocounter{page}{-1}|\\
\textit{code for title page}\\
|\newpage|\\
|\||fi|
\end{tabular}
\end{center}
%
A banner page for the child documents can be generated by:
%
\begin{center}
\begin{tabular}{l}
|\ifchilddoc|\\
|\addtocounter{page}{-1}|\\
\textit{code for banner page}\\
|\newpage|\\
|\||fi|
\end{tabular}
\end{center}
%
Here one could write a message such as:
\begin{center}
|This is the part \childdocname{} of \childdocjob{}.|
\end{center}

%%%%%%%%%%%%%%%%%%%%%%%%%%%%%%%%%%%%%%%%%%%%%%%%%%%%%%%%%%%%%%%%%%%%%%%%%%%%%%%%
\subsection{Flags}
\label{sec:flags}

The package makes it easy to generate different versions
of the main or child documents.
To this end compilation flags can be defined
and assigned different default values.
They will be particularly useful in conjunction
with the forwarding mechanism described in \secref{sec:forward}.

For example, it may be useful to have a flag |\version|
which can be set to |draft| or |final|.
The document source will contain some conditional code
depending on the value of |\version|.
Suppose further, the flag should default to |final| for the main file
and to |draft| for child files
which is a natural assignment for editing the document.
This is achieved by placing the following code
in the preamble of the main document
(below the |\childdocmain| directive):
%
\begin{center}
\begin{tabular}{l}
|\ifchilddoc|\\
|\providecommand{\version}{draft}|\\
|\||else|\\
|\providecommand{\version}{final}|\\
|\||fi|
\end{tabular}
\end{center}
%
The definition by |\providecommand| makes sure
that previous definitions are not overwritten.
Further statements |\providecommand{\version}{...}|
can thus be added before the above code to override it.

For the main file, one might add a line
(between |\childdocmain| and the above block)
%
\begin{center}
|%\ifchilddoc\||else\providecommand{\version}{draft}\||fi|
\end{center}
%
which can be uncommented to produce a draft version.
Likewise one can add a line to the very top of a child file
(above the |\childdocof{|\textit{main}|}| directive)
%
\begin{center}
|%\providecommand{\version}{final}|
\end{center}
%
which can be uncommented to produce the final version of this child document.

%%%%%%%%%%%%%%%%%%%%%%%%%%%%%%%%%%%%%%%%%%%%%%%%%%%%%%%%%%%%%%%%%%%%%%%%%%%%%%%%
\subsection{Forwarding}
\label{sec:forward}

Different versions of the main or child documents
using compilation flags as described in \secref{sec:flags}
can be (permanently) stored in different files
for convenient compilation, viewing and distribution.
To this end, the package defines a command
to pass on compilation to a different file:

%%%%%%%%%%%%%%%%%%%%%%%%%%%%%%%%%%%%%%%%
\DescribeMacro{\childdocforward}
The command |\childdocforward| redirects processing to
another source file:
%
\begin{center}
\begin{tabular}{l}
|\input{childdoc.def}|\\
|\childdocforward[|\textit{main}|]{|\textit{dest}|}|\\
\end{tabular}
\end{center}
%
The argument \textit{dest} is the destination file
(without extension).
It should be the main file or one of the child files.
Note that further \textsf{childdoc} directives
such as |\childdocof| and |\childdocforward|
in the indicated file will be processed in this form.
The optional argument \textit{main}
passes on directly to the main file \textit{main}
while pretending to compile the child \textit{dest}.
This form behaves as if \textit{dest}
issues |\childdocof{|\textit{main}|}| right away,
and no further \textsf{childdoc} directives will be processed.

%%%%%%%%%%%%%%%%%%%%%%%%%%%%%%%%%%%%%%%%
\DescribeMacro{\...prefix}
In the alternative form |\childdocforwardprefix|,
%
\begin{center}
\begin{tabular}{l}
|\input{childdoc.def}|\\
|\childdocforwardprefix[|\textit{main}|]{|\textit{prefix}|}{|\textit{dest}|}|
\end{tabular}
\end{center}
%
the destination file is determined by a pattern
depending on the current file:
To make this work, the current file must be called
`{\textit{prefix}\hspace{0.2em}\textit{suffix}}'
with \textit{prefix} matching precisely the argument.
Processing is then passed on to the file
`{\textit{dest}\hspace{0.2em}\textit{suffix}}'.
Surely, the same effect is achieved by
directly specifying the
argument `{\textit{dest}\hspace{0.2em}\textit{suffix}}'
in the first form.
However, that requires to set up a different file
for each child. With the alternative form of the command
all these files can have exactly the same content
which simplifies setting them up and maintaining them.

For example, the following file |draft.tex|
with a compilation flag |\version| as described in \secref{sec:flags}
compiles the main document as a draft:
%
\begin{center}
\begin{tabular}{l}
|\def\version{draft}|\\
|\input{childdoc.def}|\\
|\childdocforward{|\textit{main}|}|
\end{tabular}
\end{center}
%
Likewise, the following files |final|\textit{nn}|.tex|
compile the final version of the child document
|child|\textit{nn}|.tex|:
%
\begin{center}
\begin{tabular}{l}
|\def\version{final}|\\
|\input{childdoc.def}|\\
|\childdocforwardprefix{final}{child}|
\end{tabular}
\end{center}
%

Note that when several versions of a main file and/or of each child file
are to be generated, it may be convenient to set up a |Makefile| or
shell script to automatise the process.

%%%%%%%%%%%%%%%%%%%%%%%%%%%%%%%%%%%%%%%%%%%%%%%%%%%%%%%%%%%%%%%%%%%%%%%%%%%%%%%%
\subsection{Command Line Processing}
\label{sec:commandline}

The effect of redirection files can also be achieved by invoking
the \LaTeX{} compiler with a more elaborate command line.
Most conveniently this should be done as part
of a shell script or a |Makefile|.

When using \textsf{childdoc} in the main file, the following
command lines effectively perform a redirection
(note that depending on the shell being used,
backslashes may have to be doubled: `|\|' $\to$ `|\\|'):
%
\begin{center}
|... -jobname "|\textit{target}|" |\\|"|[\textit{flags}]%
|\input{childdoc.def}\childdocforward[|\textit{main}|]{|\textit{dest}|}"|
\end{center}
%
Here \textit{target} is the name of the output file,
\textit{main} is the name of the main file
and \textit{dest} is the name of the main or child file to be processed
(all filenames without extensions).
The optional argument \textit{main} can be omitted
if \textit{main} matches \textit{dest}.
Optionally, compilation \textit{flags} can be defined via |\def| commands.
This command line makes the \TeX{} engine believe
it is compiling the file \textit{target}
whose content is specified as the latter parameter.
The provided code then forwards the processing to
\textit{main} or \textit{dest} as described in \secref{sec:forward}.

%%%%%%%%%%%%%%%%%%%%%%%%%%%%%%%%%%%%%%%%%%%%%%%%%%%%%%%%%%%%%%%%%%%%%%%%%%%%%%%%
\subsection{Include by Input}
\label{sec:input}

Including child documents by |\include| has some restrictions by design.
Most notably, the content of a child document always occupies
its own set of pages; pages cannot be shared between child documents.
Usually, this behaviour makes perfect sense
because each child document contain an essential part of the document.
However, in some situations it may be desirable to compose
a document from a collection of parts
without having mandatory page breaks between then.
For this case, the package
provides a mechanism to include parts
by |\input| which can also be processed individually.
However, by construction this mechanism
requires manual handling of the content to be output.

%%%%%%%%%%%%%%%%%%%%%%%%%%%%%%%%%%%%%%%%
\DescribeMacro{\ifchilddocmanual}
The main file should be prepared as usual, see \secref{sec:include}.
However, the document body must make a distinction
between processing of an individual part and of the main document, e.g.:
%
\begin{center}
\begin{tabular}{l}
|\ifchilddocmanual|\\
|\input{\childdocname}|\\
|\||else|\\
\textit{document body with }|\input{|\textit{part}|}|\\
|\||fi|
\end{tabular}
\end{center}
%
The conditional |\ifchilddocmanual| is true whenever
a part to be included by |\input| is being compiled,
and the name of the part is stored in |\childdocname|.

%%%%%%%%%%%%%%%%%%%%%%%%%%%%%%%%%%%%%%%%
\DescribeMacro{\childdocby}
Each part to be included by |\input| should start with:
%
\begin{center}
\begin{tabular}{l}
|\input{childdoc.def}|\\
|\childdocby{|\textit{main}|}|\\
\end{tabular}
\end{center}
%
The directive |\childdocby| is similar to |\childdocof|
described in \secref{sec:include},
but the subsequent selection of content must be done manually.
To that end, both |\ifchilddoc| and |\ifchilddocmanual|
will be true upon processing of a part,
and the name of the part is stored in |\childdocname|.
Note that |\jobname| will be set to the filename of the current part
so that each part receives an individual |.aux| file
that does not interfere with the |.aux| file(s) of the main document.
This behaviour can be altered by the alternative form
|\childdocby[*]{|\textit{main}|}| (with a non-empty optional argument)
which uses the |.aux| file of the main document
by setting |\jobname| to \textit{main}.

%%%%%%%%%%%%%%%%%%%%%%%%%%%%%%%%%%%%%%%%%%%%%%%%%%%%%%%%%%%%%%%%%%%%%%%%%%%%%%%%
\subsection{Driver Development}
\label{sec:driver}

The \textsf{childdoc} mechanism can also be use for the development
of definition files such as \LaTeX{} styles or classes.
This case differs from the above setup with multiple parts
included by |\include| in that no |\includeonly| should be invoked.
This can be achieved by starting the include file
(before |\ProvidesPackage|) with:
%
\begin{center}
\begin{tabular}{l}
|\input{childdoc.def}|\\
|\childdocforward{|\textit{main}|}|\\
\end{tabular}
\end{center}
%
or alternatively with:
%
\begin{center}
\begin{tabular}{l}
|\input{childdoc.def}|\\
|\childdocby{|\textit{main}|}|\\
\end{tabular}
\end{center}
%
Both forms have slightly different effects as described above.
The main file is prepared as usual, see \secref{sec:include}.

%%%%%%%%%%%%%%%%%%%%%%%%%%%%%%%%%%%%%%%%%%%%%%%%%%%%%%%%%%%%%%%%%%%%%%%%%%%%%%%%
\subsection{Legacy Detection}
\label{sec:detection}

The directive |\childdocmain| in the main file can detect
whether the complete document or merely a child is to be compiled
even without using the directive |\childdocof|.
This method is deprecated because it is less robust
and there is no compelling reason to use it;
it is merely provided for backward compatibility
and it may be removed in future versions.

If the detection mechanism is to be used,
it is mandatory to correctly specify
the filename of the main file as the argument of |\childdocmain|:
%
\begin{center}
\begin{tabular}{l}
|\input{childdoc.def}|\\
|\childdocmain{|\textit{main}|}|\\
\end{tabular}
\end{center}
%
If |\jobname| does not match the argument \textit{main} of |\childdocmain|,
it is assumed that |\jobname| points to the child file to be compiled.
When using |\childdocmain| with the main file specified as argument,
it suffices to start a child file
with just |\input{|\textit{main}|}|
without loading of the package and using |\childdocof|.
If instead all processing is done
with the appropriate \textsf{childdoc} directives,
the argument of \textit{main} of |\childdocmain| can be empty.

An alternative version of the command line processing described
in \secref{sec:commandline} using the detection mechanism reads:
%
\begin{center}
|... -jobname "|\textit{target}|" "|[\textit{flags}]%
[|\def\jobname{|\textit{dest}|}|]|\input{|\textit{main}|}"|
\end{center}

%%%%%%%%%%%%%%%%%%%%%%%%%%%%%%%%%%%%%%%%%%%%%%%%%%%%%%%%%%%%%%%%%%%%%%%%%%%%%%%%
\subsection{Manual Code}
\label{sec:manual}

In case one cannot be certain whether the definitions file |childdoc.def|
is installed on the target \TeX{} distribution
and one prefers not to ship it,
it is conceivable to paste a few relevant commands into the sources.

To that end, drop all statements |\input{childdoc.def}|
and perform the replacements as outlined below.
Instead of |\childdocmain{|\textit{main}|}| add the following code
to the top of the main file:
%
\begin{center}
\begin{tabular}{l}
|\||ifdefined\childdocname\endinput\||fi\newif\ifchilddoc|\\
|\edef\childdocname{\scantokens\expandafter{\jobname\noexpand}}|\\
|\def\childdocmain{|\textit{main}|}\||ifx\childdocmain\childdocname\||else|\\
|\childdoctrue\includeonly{\childdocname}\let\jobname\childdocmain\||fi|\\
\end{tabular}
\end{center}
%
Instead of |\childdocof{|\textit{main}|}| just include the main file
at the top of each child file:
%
\begin{center}
|\input{|\textit{main}|}|
\end{center}
%
A simple redirection |\childdocforward{|\textit{dest}|}| is achieved by:
%
\begin{center}
|\def\jobname{|\textit{dest}|}\input{\jobname}|
\end{center}
%
The redirection with prefix
|\childdocforwardprefix[|\textit{prefix}|]{|\textit{dest}|}|
is accomplished by:
%
\begin{center}
\begin{tabular}{l}
|{\edef\jobname{\scantokens\expandafter{\jobname\noexpand}}|\\
|\def\redirectjob |\textit{prefix}|#1~~~{\gdef\jobname{|\textit{dest}|#1}}|\\
|\expandafter\redirectjob\jobname~~~}\input{\jobname}|
\end{tabular}
\end{center}

In an alternative approach,
child documents can be compiled by a specific command line
without additional code or specific definitions:
%
\begin{center}
|... -jobname "|\textit{target}|" "|[\textit{flags}]%
|\includeonly{|\textit{dest}|}\input{|\textit{main}|}"|
\end{center}
%

%%%%%%%%%%%%%%%%%%%%%%%%%%%%%%%%%%%%%%%%%%%%%%%%%%%%%%%%%%%%%%%%%%%%%%%%%%%%%%%%
%%%%%%%%%%%%%%%%%%%%%%%%%%%%%%%%%%%%%%%%%%%%%%%%%%%%%%%%%%%%%%%%%%%%%%%%%%%%%%%%
\section{Information}

%%%%%%%%%%%%%%%%%%%%%%%%%%%%%%%%%%%%%%%%%%%%%%%%%%%%%%%%%%%%%%%%%%%%%%%%%%%%%%%%
\subsection{Copyright}

Copyright \copyright{} 2017--2018 Niklas Beisert

This work may be distributed and/or modified under the
conditions of the \LaTeX{} Project Public License, either version 1.3
of this license or (at your option) any later version.
The latest version of this license is in
  \url{http://www.latex-project.org/lppl.txt}
and version 1.3 or later is part of all distributions of \LaTeX{}
version 2005/12/01 or later.

This work has the LPPL maintenance status `maintained'.

The Current Maintainer of this work is Niklas Beisert.

This work consists of the files |README.txt|, |childdoc.ins| and |childdoc.dtx|
as well as the derived files |childdoc.def|, |cdocsamp.tex|
with |cdocsch1.tex|, |cdocsch2.tex|, |cdocspt3.tex|, |cdocspt4.tex|,
|cdocsdrf.tex|, |cdocsfn1.tex|, |cdocsfn2.tex|
as well as |childdoc.pdf|.

%%%%%%%%%%%%%%%%%%%%%%%%%%%%%%%%%%%%%%%%%%%%%%%%%%%%%%%%%%%%%%%%%%%%%%%%%%%%%%%%
\subsection{Files and Installation}

The package consists of the files:
%
\begin{center}
\begin{tabular}{ll}
    |README.txt|   & readme file \\
    |childdoc.ins| & installation file \\
    |childdoc.dtx| & source file \\
    |childdoc.def| & definition file \\
    |cdocsamp.tex| & sample main file \\
    |cdocsch1.tex| & sample include file \\
    |cdocsch2.tex| & sample include file \\
    |cdocspt3.tex| & sample part file \\
    |cdocspt4.tex| & sample part file \\
    |cdocsdrf.tex| & sample redirection file \\
    |cdocsfn1.tex| & sample redirection file \\
    |cdocsfn2.tex| & sample redirection file \\
    |childdoc.pdf| & manual
\end{tabular}
\end{center}
%
The distribution consists of the files
|README.txt|, |childdoc.ins| and |childdoc.dtx|.
%
\begin{itemize}
\item
Run (pdf)\LaTeX{} on |childdoc.dtx|
to compile the manual |childdoc.pdf| (this file).
\item
Run \LaTeX{} on |childdoc.ins| to create the definitions file |childdoc.def|
and the sample |cdocsamp.tex| with include files
|cdocsch1.tex|, |cdocsch2.tex|, |cdocspt3.tex|, |cdocspt4.tex|,
|cdocsdrf.tex|, |cdocsfn1.tex|, |cdocsfn2.tex|.
Then copy the file |childdoc.def| to an appropriate directory of your \LaTeX{}
distribution, e.g.\ \textit{texmf-root}|/tex/latex/childdoc|.
\end{itemize}

%%%%%%%%%%%%%%%%%%%%%%%%%%%%%%%%%%%%%%%%%%%%%%%%%%%%%%%%%%%%%%%%%%%%%%%%%%%%%%%%
\subsection{Related CTAN Packages}

There are several other packages which offer a similar functionality:
%
\begin{itemize}
\item
The packages
\href{http://ctan.org/pkg/docmute}{\textsf{docmute}},
\href{http://ctan.org/pkg/includex}{\textsf{includex}} and
\href{http://ctan.org/pkg/standalone}{\textsf{standalone}}
provide commands to include only the document body of
a child file thus allowing both files to be compiled individually.
\item
The packages \href{http://ctan.org/pkg/subdocs}{\textsf{subdocs}}
and \href{http://ctan.org/pkg/subfiles}{\textsf{subfiles}}
provide structures in which the main and child documents can be
encapsulated and allowing them to be compiled individually.
The inclusion mechanism is different from the conventional |\include|.
\item
The package \href{http://ctan.org/pkg/combine}{\textsf{combine}}
is an elaborate solution to combine several documents into one.
\end{itemize}
%
See also the CTAN topic \href{http://ctan.org/topic/subdocs}{\textsf{subdocs}}
for further related packages.
The present package differs from the above solutions in that
a document structure constructed with the conventional |\include| mechanism
just needs two extra commands at the top of every file
such that all constituent files can be compiled individually.

%%%%%%%%%%%%%%%%%%%%%%%%%%%%%%%%%%%%%%%%%%%%%%%%%%%%%%%%%%%%%%%%%%%%%%%%%%%%%%%%
%\subsection{Feature Suggestions}
%
%The following is a list of features which may be useful for future
%versions of this package:
%%
%\begin{itemize}
%\item
%\ldots
%\end{itemize}

%%%%%%%%%%%%%%%%%%%%%%%%%%%%%%%%%%%%%%%%%%%%%%%%%%%%%%%%%%%%%%%%%%%%%%%%%%%%%%%%
\subsection{Revision History}

%%%%%%%%%%%%%%%%%%%%%%%%%%%%%%%%%%%%%%%%
\paragraph{v2.0:} 2018/12/30

\begin{itemize}
\item
immediate forward processing
\item
added |\childdocby| mechanism
\item
manual restructured
\end{itemize}

%%%%%%%%%%%%%%%%%%%%%%%%%%%%%%%%%%%%%%%%
\paragraph{v1.6:} 2018/01/17

\begin{itemize}
\item
application for development of include files
\item
corrections to manual
\end{itemize}

%%%%%%%%%%%%%%%%%%%%%%%%%%%%%%%%%%%%%%%%
\paragraph{v1.5:} 2017/05/21

\begin{itemize}
\item
more complete structuring introduced
\item
|\childdocof| introduced
\item
|\childdoc| renamed to |\childdocmain|
\item
|\childredirect| renamed to |\childdocforward| and |\childdocforwardprefix|
and functionality expanded
\end{itemize}

%%%%%%%%%%%%%%%%%%%%%%%%%%%%%%%%%%%%%%%%
\paragraph{v1.0:} 2017/04/27

\begin{itemize}
\item
manual and install package
\item
first version published on CTAN
\end{itemize}

%%%%%%%%%%%%%%%%%%%%%%%%%%%%%%%%%%%%%%%%
\paragraph{v0.6:} 2017/04/26

\begin{itemize}
\item
redirection mechanism added
\end{itemize}

%%%%%%%%%%%%%%%%%%%%%%%%%%%%%%%%%%%%%%%%
\paragraph{v0.5:} 2017/04/26

\begin{itemize}
\item
functionality in definition file
\end{itemize}


%%%%%%%%%%%%%%%%%%%%%%%%%%%%%%%%%%%%%%%%%%%%%%%%%%%%%%%%%%%%%%%%%%%%%%%%%%%%%%%%
%%%%%%%%%%%%%%%%%%%%%%%%%%%%%%%%%%%%%%%%%%%%%%%%%%%%%%%%%%%%%%%%%%%%%%%%%%%%%%%%
%%%%%%%%%%%%%%%%%%%%%%%%%%%%%%%%%%%%%%%%%%%%%%%%%%%%%%%%%%%%%%%%%%%%%%%%%%%%%%%%
\appendix

\settowidth\MacroIndent{\rmfamily\scriptsize 000\ }

 \DocInput{childdoc.dtx}

\end{document}
%</driver>
% \fi
%
% %%%%%%%%%%%%%%%%%%%%%%%%%%%%%%%%%%%%%%%%%%%%%%%%%%%%%%%%%%%%%%%%%%%%%%%%%%%%%%
% %%%%%%%%%%%%%%%%%%%%%%%%%%%%%%%%%%%%%%%%%%%%%%%%%%%%%%%%%%%%%%%%%%%%%%%%%%%%%%
% \section{Sample}
%\iffalse
%<*samplemain>
%\fi
%
% The following presents a sample document
% with two chapters, two parts, a title page,
% a compile flag as well as three forwarding files to set the flag.
% It consists of eight |.tex| files:
% \begin{center}
% \begin{tabular}{ll}
% |cdocsamp.tex|&main file\\
% |cdocsch1.tex|&include file for chapter 1\\
% |cdocsch2.tex|&include file for chapter 2\\
% |cdocspt3.tex|&include file for part 3\\
% |cdocspt4.tex|&include file for part 4\\
% |cdocsdrf.tex|&forwarding file for main file in draft mode\\
% |cdocsfi1.tex|&forwarding file for final version of chapter 1\\
% |cdocsfi2.tex|&forwarding file for final version of chapter 2\\
% \end{tabular}
% \end{center}
% Each of the eight files can be compiled directly by the \LaTeX{} compiler.
%
% %%%%%%%%%%%%%%%%%%%%%%%%%%%%%%%%%%%%%%
% \paragraph{Main File.}
%
% The main file is called |cdocsamp.tex|.
%
% Load the \textsf{childdoc} definitions and
% declare the filename for the main document:
%    \begin{macrocode}
\input{childdoc.def}
\childdocmain{}
%    \end{macrocode}

% Optional override for |\version| flag:
%    \begin{macrocode}
%%\ifchilddoc\else\providecommand{\version}{draft}\fi
%    \end{macrocode}

% Define the default values for the |\version| flag
% (|final| for the main file and |draft| for childs):
%    \begin{macrocode}
\ifchilddoc
\providecommand{\version}{draft}
\else
\providecommand{\version}{final}
\fi
%    \end{macrocode}

% Load the standard document class:
%    \begin{macrocode}
\documentclass[12pt]{article}
%    \end{macrocode}

% Start the document body:
%    \begin{macrocode}
\begin{document}
%    \end{macrocode}

% Declare a title page.
% Print title, part of document being processed and version flag:
%    \begin{macrocode}
\addtocounter{page}{-1}
\begin{center}
{\LARGE\bfseries{}childdoc example\par}
\vspace{1cm}
\ifchilddoc
\ifchilddocmanual part\else chapter\fi:
`\childdocname' of `\childdocjob'\par
\else
main document: `\childdocjob'\par
\fi
version: \version\par
\end{center}
\newpage
%    \end{macrocode}

% Manually include selected file,
% otherwise process as usual:
%    \begin{macrocode}
\ifchilddocmanual
\section*{part `\childdocname'}
\input{\childdocname}
\else
%    \end{macrocode}

% Include the two chapters:
%    \begin{macrocode}
\include{cdocsch1}
\include{cdocsch2}
%    \end{macrocode}

% Include the two parts unless only chapters should be displayed:
%    \begin{macrocode}
\ifchilddoc\else
\section{part three}
\input{cdocspt3}
\section{part four}
\input{cdocspt4}
\fi
%    \end{macrocode}

% Process as usual until here:
%    \begin{macrocode}
\fi
%    \end{macrocode}

% End of document body:
%    \begin{macrocode}
\end{document}
%    \end{macrocode}
%\iffalse
%</samplemain>
%\fi
%
% %%%%%%%%%%%%%%%%%%%%%%%%%%%%%%%%%%%%%%
% \paragraph{Chapter Include Files.}
%
% The include files are called |cdocsch1.tex| and |cdocsch2.tex|.
%
%\iffalse
%<*samplechap1|samplechap2>
%\fi

% Optional override for |\version| flag:
%    \begin{macrocode}
%%\providecommand{\version}{final}
%    \end{macrocode}

% Include the main document:
%    \begin{macrocode}
\input{childdoc.def}
\childdocof{cdocsamp}
%    \end{macrocode}

%\iffalse
%</samplechap1|samplechap2>
%\fi
%
%\iffalse
%<*samplechap1>
%\fi
% Some text for chapter 1:
%    \begin{macrocode}
\section{one}
some text in chapter one
%    \end{macrocode}

%\iffalse
%</samplechap1>
%\fi
% Some text for chapter 2:
%\iffalse
%<*samplechap2>
%\fi
%    \begin{macrocode}
\section{two}
more text in chapter two
%    \end{macrocode}

%\iffalse
%</samplechap2>
%\fi
%
% %%%%%%%%%%%%%%%%%%%%%%%%%%%%%%%%%%%%%%
% \paragraph{Part Include Files.}
%
% The include files are called |cdocspt3.tex| and |cdocspt4.tex|.
%
%\iffalse
%<*samplepart3|samplepart4>
%\fi

% Optional override for |\version| flag:
%    \begin{macrocode}
%%\providecommand{\version}{final}
%    \end{macrocode}

% Include the main document:
%    \begin{macrocode}
\input{childdoc.def}
\childdocby{cdocsamp}
%    \end{macrocode}

%\iffalse
%</samplepart3|samplepart4>
%\fi
%
%\iffalse
%<*samplepart3>
%\fi
% Some text for part 3:
%    \begin{macrocode}
some text in part three
%    \end{macrocode}

%\iffalse
%</samplepart3>
%\fi
% Some text for part 4:
%\iffalse
%<*samplepart4>
%\fi
%    \begin{macrocode}
more text in part four
%    \end{macrocode}

%\iffalse
%</samplepart4>
%\fi
%
% %%%%%%%%%%%%%%%%%%%%%%%%%%%%%%%%%%%%%%
% \paragraph{Forwarding for a Complete Draft.}
%
% The following forwarding file |cdocsdrf.tex|
% compiles the main document in draft mode:
%\iffalse
%<*sampledraft>
%\fi
%    \begin{macrocode}
\def\version{draft}
\input{childdoc.def}
\childdocforward{cdocsamp}
%    \end{macrocode}

%\iffalse
%</sampledraft>
%\fi
%
% %%%%%%%%%%%%%%%%%%%%%%%%%%%%%%%%%%%%%%
% \paragraph{Forwarding for Final Version of the Chapters.}
%
% The following forwarding files |cdocsfn1.tex| and |cdocsfn2.tex|
% (with identical content)
% compile the final versions of the child documents
% |cdocsch1.tex| and |cdocsch2.tex|, respectively:
%\iffalse
%<*samplefinal>
%\fi
%    \begin{macrocode}
\def\version{final}
\input{childdoc.def}
\childdocforwardprefix[cdocsamp]{cdocsfn}{cdocsch}
%    \end{macrocode}

%\iffalse
%</samplefinal>
%\fi
%
% %%%%%%%%%%%%%%%%%%%%%%%%%%%%%%%%%%%%%%
% \paragraph{Command Line Processing.}
%
% The following three command lines generate the output files
% |cdocscld|, |cdocscl1| and |cdocscl2|
% which should be identical to
% |cdocsdrf|, |cdocsch1| and |cdocsfn2|, respectively:
% \begin{center}
% \begin{tabular}{l}
% |latex -jobname cdocscld \|\\
% |  "\def\version{draft}\input{childdoc.def}\childdocforward{cdocsamp}"|\\
% |latex -jobname cdocscl1 \|\\
% |  "\input{childdoc.def}\childdocforward[cdocsamp]{cdocsch1}"|\\
% |latex -jobname cdocscl2 \|\\
% |  "\def\version{final}\input{childdoc.def}\childdocforward{cdocsch2}"|
% \end{tabular}
% \end{center}
% Note that the trailing backslash on each first line
% merely continues the input to the second line
% (for convenient cut ant paste).
% Furthermore, the command |latex| can be replaced by any
% of its alternative versions such as |pdflatex|.
%
% %%%%%%%%%%%%%%%%%%%%%%%%%%%%%%%%%%%%%%%%%%%%%%%%%%%%%%%%%%%%%%%%%%%%%%%%%%%%%%
% %%%%%%%%%%%%%%%%%%%%%%%%%%%%%%%%%%%%%%%%%%%%%%%%%%%%%%%%%%%%%%%%%%%%%%%%%%%%%%
% \section{Implementation}
%\iffalse
%<*package>
%\fi
%
% This section describes the definitions file |childdoc.def|.

% The definitions cannot be loaded using |\usepackage| or |\RequirePackage|
% which has a mechanism to prevent loading a style file more than once.
% When loading the definitions by means of |\input|
% multiple instances have to be prevented manually:
%\iffalse
%This code needs to be before the `\ProvidesFile' directive
%which is defined at the beginning of this file.
%Therefore it is also placed there and commented out here.
%</package>
%<*discard>
%\fi
%    \begin{macrocode}
\ifdefined\childdocmain\endinput\fi
%    \end{macrocode}
%\iffalse
%</discard>
%<*package>
%\fi
%
% \macro{\ifchilddoc}
% \macro{\ifchilddocmanual}
% The conditional |\ifchilddoc| tells whether a
% child (true) or main (false) document is being compiled.
% The conditional |\ifchilddocmanual| tells whether
% the |\includeonly| mechanism is used (false) or
% the selection of child files must be performed manually (true).
% The definitions initialise to false:
%    \begin{macrocode}
\newif\ifchilddoc
\newif\ifchilddocmanual
%    \end{macrocode}

% \macro{\childdocname}
% \macro{\childdocjob}
% The macro |\childdocname| stores the name of the main document
% to be compiled. The macro |\childdocjob| stores the name of
% the document on which the \LaTeX{} compiler was originally invoked.
% The content of |\jobname| cannot be compared
% to filenames specified in the source due to different catcodes.
% The following code rescans |\jobname|, stores the result
% in |\childdocname| and saves a copy in |\childdocjob|:
%    \begin{macrocode}
\edef\childdocname{\scantokens\expandafter{\jobname\noexpand}}
\let\childdocjob\childdocname
%    \end{macrocode}

% \macro{\childdocdisable}
% The macro |\childdocdisable| prevents the main file
% from being processed more than once.
% At this stage, the main document command |\childdocmain|
% is assumed to be called once again where it should do nothing.
% Any subsequent call to it should prevent
% a secondary processing of the main document
% It overwrites the forwarding commands
% |\childdocof| and |\childdocforward|
% with empty macros to prevent further inclusions of the main document:
%    \begin{macrocode}
\newcommand{\childdocdisable}
{
  \renewcommand{\childdocmain}[1]{\renewcommand{\childdocmain}[1]{\endinput}}
  \renewcommand{\childdocof}[1]{}
  \renewcommand{\childdocby}[2][]{}
  \renewcommand{\childdocforward}[2][]{}
  \renewcommand{\childdocdisable}{}
}
%    \end{macrocode}

% \macro{\childdocmain}
% The macro |\childdocmain| is to be called at the top of the main file
% with nothing or the main filename (without extension) as argument.
% First, it breaks loops.
% If the argument is not empty and does not match |\childdocname|
% (which is set by the first inclusion of |childdoc.def|),
% |\ifchilddoc| is set to true, |\includeonly| is applied to the child file
% and |\jobname| is set to the main file
% (for proper handling of |.aux| files):
%    \begin{macrocode}
\newcommand{\childdocmain}[1]
{
  \childdocdisable\childdocmain{}
  \if?#1?\else
    \begingroup
      \def\childdoctmp{#1}
      \ifx\childdoctmp\childdocname
        \def\childdoctmp{}
      \else
        \def\childdoctmp
        {
          \childdoctrue
          \includeonly{\childdocname}
          \def\childdocjob{#1}
          \def\jobname{#1}
        }
      \fi
      \expandafter
    \endgroup
    \childdoctmp
  \fi
}
%    \end{macrocode}

% \macro{\childdocof}
% The command |\childdocof| redirects
% compilation to the main file |#1|.
%    \begin{macrocode}
\newcommand{\childdocof}[1]
{
  \childdocdisable
  \childdoctrue
  \includeonly{\childdocname}
  \def\jobname{#1}
  \def\childdocjob{#1}
  \input{#1}
}
%    \end{macrocode}

% \macro{\childdocby}
% The command |\childdocby| ....
%    \begin{macrocode}
\newcommand{\childdocby}[2][]
{
  \childdocdisable
  \childdoctrue
  \childdocmanualtrue
  \if?#1?\else
    \def\jobname{#2}
  \fi
  \def\childdocjob{#2}
  \input{#2}
  \endinput
}
%    \end{macrocode}

% \macro{\childdocforward}
% The command |\childdocforward| redirects
% compilation to the main file or
% (if the optional argument is given) a child file.
% Parameters are set as if the main file
% or a child file starting with |\childdocof| was compiled.
% Then compilation is handed over to the main file:
%    \begin{macrocode}
\newcommand{\childdocforward}[2][]
{
  \begingroup
    \if?#1?
      \def\childdoctmp
      {
        \def\childdocname{#2}
        \def\childdocjob{#2}
        \def\jobname{#2}
        \input{#2}
        \endinput
      }
    \else
      \def\childdoctmp
      {
        \childdocdisable
        \def\childdocname{#2}
        \childdoctrue
        \includeonly{#2}
        \def\childdocjob{#1}
        \def\jobname{#1}
        \input{#1}
        \endinput
      }
    \fi
    \expandafter
  \endgroup
  \childdoctmp
}
%    \end{macrocode}

% \macro{\childdocforwardprefix}
% The command |\childdocforwardprefix| redirects
% compilation to the main or a child file by means of a pattern.
% The prefix |#1| in the current filename is replaced by |#2|
% and the suffix of the current filename is kept
% (it is assumed that the filename does not contain the substring `|~~~|'
% which is used as a delimiter).
% Compilation is handed over to the new file by |\childdocforward|:
%    \begin{macrocode}
\newcommand{\childdocforwardprefix}[3][]
{
  \begingroup
    \def\childdocextract #2##1~~~{\def\childdoctmp{\childdocforward[#1]{#3##1}}}
    \expandafter\childdocextract\childdocname~~~
    \expandafter
  \endgroup
  \childdoctmp
}
%    \end{macrocode}

% \macro{\childdoc}
% The deprecated macro |\childdoc| is a legacy version of |\childdocmain|:
%    \begin{macrocode}
\newcommand{\childdoc}{\childdocmain}
%    \end{macrocode}

% \macro{\childdocredirect}
% The deprecated macro |\childdocredirect| is a legacy version
% of |\childdocforward| and |\childdocforwardprefix|:
%    \begin{macrocode}
\newcommand{\childdocredirect}[2][]
{
  \begingroup
    \if?#1?
      \def\childdoctmp{\childdocforward{#2}}
    \else
      \def\childdoctmp{\childdocforwardprefix{#1}{#2}}
    \fi
    \expandafter
  \endgroup
  \childdoctmp
}
%    \end{macrocode}

%\iffalse
%</package>
%\fi
%
\endinput
|\\
|\childdocmain{}|\\
\end{tabular}
\end{center}
at the very top of the main \LaTeX{} file,
in particular \emph{before} the |\documentclass| statement!
The argument of |\childdocmain| should be left empty
(but it must be present).

%%%%%%%%%%%%%%%%%%%%%%%%%%%%%%%%%%%%%%%%
\DescribeMacro{\childdocof}
Furthermore, add the commands
\begin{center}
\begin{tabular}{l}
|% \iffalse
%
% childdoc.dtx Copyright (C) 2017-2018 Niklas Beisert
%
% This work may be distributed and/or modified under the
% conditions of the LaTeX Project Public License, either version 1.3
% of this license or (at your option) any later version.
% The latest version of this license is in
%   http://www.latex-project.org/lppl.txt
% and version 1.3 or later is part of all distributions of LaTeX
% version 2005/12/01 or later.
%
% This work has the LPPL maintenance status `maintained'.
%
% The Current Maintainer of this work is Niklas Beisert.
%
% This work consists of the files childdoc.dtx and childdoc.ins
% and the derived files childdoc.def and cdocsamp.tex with
% cdocsch1.tex, cdocsch2.tex, cdocsdrf.tex, cdocsfn1.tex, cdocsfn2.tex.
%
%<package>\ifdefined\childdocmain\endinput\fi
%<package>\ProvidesFile{childdoc.def}[2018/12/30 v2.0 child document driver]
%<samplemain>\ProvidesFile{cdocsamp.tex}[2018/12/30 v2.0 sample for childdoc]
%<*driver>
%\ProvidesFile{childdoc.drv}[2018/12/30 v2.0 childdoc reference manual file]
\PassOptionsToClass{10pt,a4paper}{article}
\documentclass{ltxdoc}

\usepackage[margin=35mm]{geometry}
\usepackage{hyperref}
\usepackage{hyperxmp}
\usepackage[usenames]{color}

\hypersetup{colorlinks=true}
\hypersetup{pdfstartview=FitH}
\hypersetup{pdfpagemode=UseNone}
\hypersetup{pdfsource={}}
\hypersetup{pdflang={en-UK}}
\hypersetup{pdfcopyright={Copyright 2017-2018 Niklas Beisert.
  This work may be distributed and/or modified under the
  conditions of the LaTeX Project Public License, either version 1.3
  of this license or (at your option) any later version.}}
\hypersetup{pdflicenseurl={http://www.latex-project.org/lppl.txt}}
\hypersetup{pdfcontactaddress={ETH Zurich, ITP, HIT K,
  Wolfgang-Pauli-Strasse 27}}
\hypersetup{pdfcontactpostcode={8093}}
\hypersetup{pdfcontactcity={Zurich}}
\hypersetup{pdfcontactcountry={Switzerland}}
\hypersetup{pdfcontactemail={nbeisert@itp.phys.ethz.ch}}
\hypersetup{pdfcontacturl={http://people.phys.ethz.ch/\xmptilde nbeisert/}}

\newcommand{\secref}[1]{\hyperref[#1]{section \ref*{#1}}}

\parskip1ex
\parindent0pt
\let\olditemize\itemize
\def\itemize{\olditemize\parskip0pt}

\begin{document}

\title{The \textsf{childdoc} Package}
\hypersetup{pdftitle={The childdoc Package}}
\author{Niklas Beisert\\[2ex]
  Institut f\"ur Theoretische Physik\\
  Eidgen\"ossische Technische Hochschule Z\"urich\\
  Wolfgang-Pauli-Strasse 27, 8093 Z\"urich, Switzerland\\[1ex]
  \href{mailto:nbeisert@itp.phys.ethz.ch}
  {\texttt{nbeisert@itp.phys.ethz.ch}}}
\hypersetup{pdfauthor={Niklas Beisert}}
\hypersetup{pdfsubject={Manual for the LaTeX2e Package childdoc}}
\date{30 December 2018, \textsf{v2.0}}
\maketitle

\begin{abstract}\noindent
\textsf{childdoc} is a \LaTeXe{} package
that enables the direct compilation
of document sections included by |\include|
to individual files.
\end{abstract}

\begingroup
\parskip0ex
\tableofcontents
\endgroup

%%%%%%%%%%%%%%%%%%%%%%%%%%%%%%%%%%%%%%%%%%%%%%%%%%%%%%%%%%%%%%%%%%%%%%%%%%%%%%%%
%%%%%%%%%%%%%%%%%%%%%%%%%%%%%%%%%%%%%%%%%%%%%%%%%%%%%%%%%%%%%%%%%%%%%%%%%%%%%%%%
\section{Introduction}

\LaTeX{} provides a mechanism to structure a large document (such as a book)
into a main file and several child files (containing the chapters)
using the |\include| command.
This mechanism is beneficial for documents
which span hundreds of pages in order to
make the source file(s) more manageable.
Moreover, compilation can be restricted to
selected child files by means of the |\includeonly| command.
The latter feature can be used to reduce the compilation time while editing
(this was significantly more useful in the earlier days of \LaTeX{})
or to generate a smaller document which is easier to navigate.
Another application of |\includeonly| is to generate
documents consisting of selected parts of the complete document.

However, there are a few drawbacks of the plain |\include| mechanism:
\begin{itemize}
\item
The child files cannot be compiled on their own,
they can only be compiled via the main file.
A naive editing environment
(such as a text editor with an option
to have the current file processed by \LaTeX)
may require one to switch to the main file before compiling;
attempting to compile the child file produces errors.
\item
The main file must be modified (each time)
to adjust the |\includeonly| command
to the present needs. This easily leaves the main file in a messy state.
\item
The generated document will always carry the filename
of the main document. This is inconvenient if
several child files are to be compiled and
to be kept for distribution.
\end{itemize}

The present package provides a simple interface
to make child files individually compilable by \LaTeX{}.
Compiling a child file then has the same effect as compiling
the main file with an |\includeonly| command
to select the appropriate child.
Moreover the generated document will carry the name of the child
rather than the main file.
This resolves all three above issues.

This feature is meant to make the editing of books,
thesis documents and lecture notes somewhat more convenient.
However, the package can also be used efficiently for
composing a series of documents (such as exercise sheets)
which are typically distributed individually.
It then assists the author in generating the individual documents
(potentially in different versions)
as well as a document containing the collected series.
Another application is in developing style files
or other kinds of included material
where compilation of the style file could redirect
to a sample or test file.

%%%%%%%%%%%%%%%%%%%%%%%%%%%%%%%%%%%%%%%%%%%%%%%%%%%%%%%%%%%%%%%%%%%%%%%%%%%%%%%%
%%%%%%%%%%%%%%%%%%%%%%%%%%%%%%%%%%%%%%%%%%%%%%%%%%%%%%%%%%%%%%%%%%%%%%%%%%%%%%%%
\section{Usage}

First of all, the package \textsf{childdoc} is \emph{not} a standard
\LaTeXe{} |.sty| style file! Therefore it needs to be invoked in
a non-standard way.

%%%%%%%%%%%%%%%%%%%%%%%%%%%%%%%%%%%%%%%%%%%%%%%%%%%%%%%%%%%%%%%%%%%%%%%%%%%%%%%%
\subsection{Included Files}
\label{sec:include}

%%%%%%%%%%%%%%%%%%%%%%%%%%%%%%%%%%%%%%%%
\DescribeMacro{\childdocmain}
To use the package, add the commands
\begin{center}
\begin{tabular}{l}
|\input{childdoc.def}|\\
|\childdocmain{}|\\
\end{tabular}
\end{center}
at the very top of the main \LaTeX{} file,
in particular \emph{before} the |\documentclass| statement!
The argument of |\childdocmain| should be left empty
(but it must be present).

%%%%%%%%%%%%%%%%%%%%%%%%%%%%%%%%%%%%%%%%
\DescribeMacro{\childdocof}
Furthermore, add the commands
\begin{center}
\begin{tabular}{l}
|\input{childdoc.def}|\\
|\childdocof{|\textit{main}|}|\\
\end{tabular}
\end{center}
at the top of every child file \textit{child}
which is included by |\include{|\textit{child}|}|
from within the main file
(or at least for those files to be compiled individually).
The argument \textit{main} must be the filename of the main file.

There are a couple of
considerations in setting up the main and child documents:

%%%%%%%%%%%%%%%%%%%%%%%%%%%%%%%%%%%%%%%%
\paragraph{Restrictions.}

Please note the following restrictions:
\begin{itemize}
\item
|\childdocmain| must be called with one argument \textit{main}
to ensure compatibility with earlier version of the package.
It must either be empty (|\childdocmain{}|)
or precisely match the filename of the main file in which it is specified.
See \secref{sec:detection} for further information.
\item
The filename \textit{main} must be specified without the |.tex| extension.
\item
The filename \textit{main} is case sensitive
(even in case-insensitive file systems)
due to internal string comparison.
\item
The argument \textit{main} should be fully expanded, it cannot be a macro.
\item
Subdirectories and special characters should be avoided in filenames.
\item
The command |\childdocmain{|\textit{main}|}| must be followed by a whitespace.
It should not be followed immediately by another command
or by a comment mark `|%|'.
This is because the \TeX{} parser reads the token immediately following
the argument of |\childdocmain| and puts it
at the beginning of every child section;
however, a white\-space is ignored.
\end{itemize}

%%%%%%%%%%%%%%%%%%%%%%%%%%%%%%%%%%%%%%%%
\paragraph{Content of Main File.}

It is advisable to place all content in the child files included by |\include|.
Any output contained in the main file will appear in all child documents
unless suppressed manually;
it cannot be suppressed automatically by the |\includeonly| directive
and thus should normally be avoided.
A method to include some content in the main file
by means of conditional processing is described in \secref{sec:conditional}.

%%%%%%%%%%%%%%%%%%%%%%%%%%%%%%%%%%%%%%%%
\paragraph{Page Numbering.}

When only a part of the document is compiled,
the appropriate numbering of pages
(as well as other status parameters)
is determined from the |.aux| files.
The latter contain information from previous passes.
However this information needs to propagate through
all intermediate child documents.
Therefore the page numbering in child documents may well
be inconsistent until the complete document is compiled at least once.

A useful (if unconventional) way to always ensure a consistent
page numbering is to restart the numbering in each child document
and denote the pages by `\textit{child}|.|\textit{page}'
where \textit{child} represents the chapter/section number of the child file.
This can be achieved by the command
|\numberwithin{page}{|\textit{child}|}|
of the \textsf{amsmath} package
where \textit{child} can be |chapter| or |section|
depending on the chosen structuring.
Alternatively, one can modify the macro |\thepage| appropriately
and reset the counter |page| at the start of each child file.

%%%%%%%%%%%%%%%%%%%%%%%%%%%%%%%%%%%%%%%%%%%%%%%%%%%%%%%%%%%%%%%%%%%%%%%%%%%%%%%%
\subsection{Conditional Processing}
\label{sec:conditional}

The package provides a mechanism to compile different versions
of a document. To customise the versions further some conditional processing
can come in handy to distinguish which version is being compiled.
The package provides two macros to describe the compilation context:

%%%%%%%%%%%%%%%%%%%%%%%%%%%%%%%%%%%%%%%%
\DescribeMacro{\ifchilddoc}
The conditional |\ifchilddoc| distinguishes between the compilation of
child documents and the main document:
%
\begin{center}
|\ifchilddoc |\textit{child-code}| |[|\||else |\textit{main-code}]| \||fi|
\end{center}

%%%%%%%%%%%%%%%%%%%%%%%%%%%%%%%%%%%%%%%%
\DescribeMacro{\childdocname}
\DescribeMacro{\childdocjob}
The macro |\childdocname| contains the filename (without extension)
of the main or child file being processed.
Note that |\childdocjob| will always contain the name of the main file.

%%%%%%%%%%%%%%%%%%%%%%%%%%%%%%%%%%%%%%%%
\paragraph{Title Page.}

Conditional processing can be used to include a title or banner page
in the main document when proper precautions are taken.
Importantly, the code in the main file should ensure that the page counter
(as well as other status parameters which are stored in the |.aux| files)
takes the same value after the conditional processing.
Otherwise the page numbers may take divergent values
depending on which part is compiled.

For example, a title page could be declared by:
%
\begin{center}
\begin{tabular}{l}
|\ifchilddoc\||else|\\
|\addtocounter{page}{-1}|\\
\textit{code for title page}\\
|\newpage|\\
|\||fi|
\end{tabular}
\end{center}
%
A banner page for the child documents can be generated by:
%
\begin{center}
\begin{tabular}{l}
|\ifchilddoc|\\
|\addtocounter{page}{-1}|\\
\textit{code for banner page}\\
|\newpage|\\
|\||fi|
\end{tabular}
\end{center}
%
Here one could write a message such as:
\begin{center}
|This is the part \childdocname{} of \childdocjob{}.|
\end{center}

%%%%%%%%%%%%%%%%%%%%%%%%%%%%%%%%%%%%%%%%%%%%%%%%%%%%%%%%%%%%%%%%%%%%%%%%%%%%%%%%
\subsection{Flags}
\label{sec:flags}

The package makes it easy to generate different versions
of the main or child documents.
To this end compilation flags can be defined
and assigned different default values.
They will be particularly useful in conjunction
with the forwarding mechanism described in \secref{sec:forward}.

For example, it may be useful to have a flag |\version|
which can be set to |draft| or |final|.
The document source will contain some conditional code
depending on the value of |\version|.
Suppose further, the flag should default to |final| for the main file
and to |draft| for child files
which is a natural assignment for editing the document.
This is achieved by placing the following code
in the preamble of the main document
(below the |\childdocmain| directive):
%
\begin{center}
\begin{tabular}{l}
|\ifchilddoc|\\
|\providecommand{\version}{draft}|\\
|\||else|\\
|\providecommand{\version}{final}|\\
|\||fi|
\end{tabular}
\end{center}
%
The definition by |\providecommand| makes sure
that previous definitions are not overwritten.
Further statements |\providecommand{\version}{...}|
can thus be added before the above code to override it.

For the main file, one might add a line
(between |\childdocmain| and the above block)
%
\begin{center}
|%\ifchilddoc\||else\providecommand{\version}{draft}\||fi|
\end{center}
%
which can be uncommented to produce a draft version.
Likewise one can add a line to the very top of a child file
(above the |\childdocof{|\textit{main}|}| directive)
%
\begin{center}
|%\providecommand{\version}{final}|
\end{center}
%
which can be uncommented to produce the final version of this child document.

%%%%%%%%%%%%%%%%%%%%%%%%%%%%%%%%%%%%%%%%%%%%%%%%%%%%%%%%%%%%%%%%%%%%%%%%%%%%%%%%
\subsection{Forwarding}
\label{sec:forward}

Different versions of the main or child documents
using compilation flags as described in \secref{sec:flags}
can be (permanently) stored in different files
for convenient compilation, viewing and distribution.
To this end, the package defines a command
to pass on compilation to a different file:

%%%%%%%%%%%%%%%%%%%%%%%%%%%%%%%%%%%%%%%%
\DescribeMacro{\childdocforward}
The command |\childdocforward| redirects processing to
another source file:
%
\begin{center}
\begin{tabular}{l}
|\input{childdoc.def}|\\
|\childdocforward[|\textit{main}|]{|\textit{dest}|}|\\
\end{tabular}
\end{center}
%
The argument \textit{dest} is the destination file
(without extension).
It should be the main file or one of the child files.
Note that further \textsf{childdoc} directives
such as |\childdocof| and |\childdocforward|
in the indicated file will be processed in this form.
The optional argument \textit{main}
passes on directly to the main file \textit{main}
while pretending to compile the child \textit{dest}.
This form behaves as if \textit{dest}
issues |\childdocof{|\textit{main}|}| right away,
and no further \textsf{childdoc} directives will be processed.

%%%%%%%%%%%%%%%%%%%%%%%%%%%%%%%%%%%%%%%%
\DescribeMacro{\...prefix}
In the alternative form |\childdocforwardprefix|,
%
\begin{center}
\begin{tabular}{l}
|\input{childdoc.def}|\\
|\childdocforwardprefix[|\textit{main}|]{|\textit{prefix}|}{|\textit{dest}|}|
\end{tabular}
\end{center}
%
the destination file is determined by a pattern
depending on the current file:
To make this work, the current file must be called
`{\textit{prefix}\hspace{0.2em}\textit{suffix}}'
with \textit{prefix} matching precisely the argument.
Processing is then passed on to the file
`{\textit{dest}\hspace{0.2em}\textit{suffix}}'.
Surely, the same effect is achieved by
directly specifying the
argument `{\textit{dest}\hspace{0.2em}\textit{suffix}}'
in the first form.
However, that requires to set up a different file
for each child. With the alternative form of the command
all these files can have exactly the same content
which simplifies setting them up and maintaining them.

For example, the following file |draft.tex|
with a compilation flag |\version| as described in \secref{sec:flags}
compiles the main document as a draft:
%
\begin{center}
\begin{tabular}{l}
|\def\version{draft}|\\
|\input{childdoc.def}|\\
|\childdocforward{|\textit{main}|}|
\end{tabular}
\end{center}
%
Likewise, the following files |final|\textit{nn}|.tex|
compile the final version of the child document
|child|\textit{nn}|.tex|:
%
\begin{center}
\begin{tabular}{l}
|\def\version{final}|\\
|\input{childdoc.def}|\\
|\childdocforwardprefix{final}{child}|
\end{tabular}
\end{center}
%

Note that when several versions of a main file and/or of each child file
are to be generated, it may be convenient to set up a |Makefile| or
shell script to automatise the process.

%%%%%%%%%%%%%%%%%%%%%%%%%%%%%%%%%%%%%%%%%%%%%%%%%%%%%%%%%%%%%%%%%%%%%%%%%%%%%%%%
\subsection{Command Line Processing}
\label{sec:commandline}

The effect of redirection files can also be achieved by invoking
the \LaTeX{} compiler with a more elaborate command line.
Most conveniently this should be done as part
of a shell script or a |Makefile|.

When using \textsf{childdoc} in the main file, the following
command lines effectively perform a redirection
(note that depending on the shell being used,
backslashes may have to be doubled: `|\|' $\to$ `|\\|'):
%
\begin{center}
|... -jobname "|\textit{target}|" |\\|"|[\textit{flags}]%
|\input{childdoc.def}\childdocforward[|\textit{main}|]{|\textit{dest}|}"|
\end{center}
%
Here \textit{target} is the name of the output file,
\textit{main} is the name of the main file
and \textit{dest} is the name of the main or child file to be processed
(all filenames without extensions).
The optional argument \textit{main} can be omitted
if \textit{main} matches \textit{dest}.
Optionally, compilation \textit{flags} can be defined via |\def| commands.
This command line makes the \TeX{} engine believe
it is compiling the file \textit{target}
whose content is specified as the latter parameter.
The provided code then forwards the processing to
\textit{main} or \textit{dest} as described in \secref{sec:forward}.

%%%%%%%%%%%%%%%%%%%%%%%%%%%%%%%%%%%%%%%%%%%%%%%%%%%%%%%%%%%%%%%%%%%%%%%%%%%%%%%%
\subsection{Include by Input}
\label{sec:input}

Including child documents by |\include| has some restrictions by design.
Most notably, the content of a child document always occupies
its own set of pages; pages cannot be shared between child documents.
Usually, this behaviour makes perfect sense
because each child document contain an essential part of the document.
However, in some situations it may be desirable to compose
a document from a collection of parts
without having mandatory page breaks between then.
For this case, the package
provides a mechanism to include parts
by |\input| which can also be processed individually.
However, by construction this mechanism
requires manual handling of the content to be output.

%%%%%%%%%%%%%%%%%%%%%%%%%%%%%%%%%%%%%%%%
\DescribeMacro{\ifchilddocmanual}
The main file should be prepared as usual, see \secref{sec:include}.
However, the document body must make a distinction
between processing of an individual part and of the main document, e.g.:
%
\begin{center}
\begin{tabular}{l}
|\ifchilddocmanual|\\
|\input{\childdocname}|\\
|\||else|\\
\textit{document body with }|\input{|\textit{part}|}|\\
|\||fi|
\end{tabular}
\end{center}
%
The conditional |\ifchilddocmanual| is true whenever
a part to be included by |\input| is being compiled,
and the name of the part is stored in |\childdocname|.

%%%%%%%%%%%%%%%%%%%%%%%%%%%%%%%%%%%%%%%%
\DescribeMacro{\childdocby}
Each part to be included by |\input| should start with:
%
\begin{center}
\begin{tabular}{l}
|\input{childdoc.def}|\\
|\childdocby{|\textit{main}|}|\\
\end{tabular}
\end{center}
%
The directive |\childdocby| is similar to |\childdocof|
described in \secref{sec:include},
but the subsequent selection of content must be done manually.
To that end, both |\ifchilddoc| and |\ifchilddocmanual|
will be true upon processing of a part,
and the name of the part is stored in |\childdocname|.
Note that |\jobname| will be set to the filename of the current part
so that each part receives an individual |.aux| file
that does not interfere with the |.aux| file(s) of the main document.
This behaviour can be altered by the alternative form
|\childdocby[*]{|\textit{main}|}| (with a non-empty optional argument)
which uses the |.aux| file of the main document
by setting |\jobname| to \textit{main}.

%%%%%%%%%%%%%%%%%%%%%%%%%%%%%%%%%%%%%%%%%%%%%%%%%%%%%%%%%%%%%%%%%%%%%%%%%%%%%%%%
\subsection{Driver Development}
\label{sec:driver}

The \textsf{childdoc} mechanism can also be use for the development
of definition files such as \LaTeX{} styles or classes.
This case differs from the above setup with multiple parts
included by |\include| in that no |\includeonly| should be invoked.
This can be achieved by starting the include file
(before |\ProvidesPackage|) with:
%
\begin{center}
\begin{tabular}{l}
|\input{childdoc.def}|\\
|\childdocforward{|\textit{main}|}|\\
\end{tabular}
\end{center}
%
or alternatively with:
%
\begin{center}
\begin{tabular}{l}
|\input{childdoc.def}|\\
|\childdocby{|\textit{main}|}|\\
\end{tabular}
\end{center}
%
Both forms have slightly different effects as described above.
The main file is prepared as usual, see \secref{sec:include}.

%%%%%%%%%%%%%%%%%%%%%%%%%%%%%%%%%%%%%%%%%%%%%%%%%%%%%%%%%%%%%%%%%%%%%%%%%%%%%%%%
\subsection{Legacy Detection}
\label{sec:detection}

The directive |\childdocmain| in the main file can detect
whether the complete document or merely a child is to be compiled
even without using the directive |\childdocof|.
This method is deprecated because it is less robust
and there is no compelling reason to use it;
it is merely provided for backward compatibility
and it may be removed in future versions.

If the detection mechanism is to be used,
it is mandatory to correctly specify
the filename of the main file as the argument of |\childdocmain|:
%
\begin{center}
\begin{tabular}{l}
|\input{childdoc.def}|\\
|\childdocmain{|\textit{main}|}|\\
\end{tabular}
\end{center}
%
If |\jobname| does not match the argument \textit{main} of |\childdocmain|,
it is assumed that |\jobname| points to the child file to be compiled.
When using |\childdocmain| with the main file specified as argument,
it suffices to start a child file
with just |\input{|\textit{main}|}|
without loading of the package and using |\childdocof|.
If instead all processing is done
with the appropriate \textsf{childdoc} directives,
the argument of \textit{main} of |\childdocmain| can be empty.

An alternative version of the command line processing described
in \secref{sec:commandline} using the detection mechanism reads:
%
\begin{center}
|... -jobname "|\textit{target}|" "|[\textit{flags}]%
[|\def\jobname{|\textit{dest}|}|]|\input{|\textit{main}|}"|
\end{center}

%%%%%%%%%%%%%%%%%%%%%%%%%%%%%%%%%%%%%%%%%%%%%%%%%%%%%%%%%%%%%%%%%%%%%%%%%%%%%%%%
\subsection{Manual Code}
\label{sec:manual}

In case one cannot be certain whether the definitions file |childdoc.def|
is installed on the target \TeX{} distribution
and one prefers not to ship it,
it is conceivable to paste a few relevant commands into the sources.

To that end, drop all statements |\input{childdoc.def}|
and perform the replacements as outlined below.
Instead of |\childdocmain{|\textit{main}|}| add the following code
to the top of the main file:
%
\begin{center}
\begin{tabular}{l}
|\||ifdefined\childdocname\endinput\||fi\newif\ifchilddoc|\\
|\edef\childdocname{\scantokens\expandafter{\jobname\noexpand}}|\\
|\def\childdocmain{|\textit{main}|}\||ifx\childdocmain\childdocname\||else|\\
|\childdoctrue\includeonly{\childdocname}\let\jobname\childdocmain\||fi|\\
\end{tabular}
\end{center}
%
Instead of |\childdocof{|\textit{main}|}| just include the main file
at the top of each child file:
%
\begin{center}
|\input{|\textit{main}|}|
\end{center}
%
A simple redirection |\childdocforward{|\textit{dest}|}| is achieved by:
%
\begin{center}
|\def\jobname{|\textit{dest}|}\input{\jobname}|
\end{center}
%
The redirection with prefix
|\childdocforwardprefix[|\textit{prefix}|]{|\textit{dest}|}|
is accomplished by:
%
\begin{center}
\begin{tabular}{l}
|{\edef\jobname{\scantokens\expandafter{\jobname\noexpand}}|\\
|\def\redirectjob |\textit{prefix}|#1~~~{\gdef\jobname{|\textit{dest}|#1}}|\\
|\expandafter\redirectjob\jobname~~~}\input{\jobname}|
\end{tabular}
\end{center}

In an alternative approach,
child documents can be compiled by a specific command line
without additional code or specific definitions:
%
\begin{center}
|... -jobname "|\textit{target}|" "|[\textit{flags}]%
|\includeonly{|\textit{dest}|}\input{|\textit{main}|}"|
\end{center}
%

%%%%%%%%%%%%%%%%%%%%%%%%%%%%%%%%%%%%%%%%%%%%%%%%%%%%%%%%%%%%%%%%%%%%%%%%%%%%%%%%
%%%%%%%%%%%%%%%%%%%%%%%%%%%%%%%%%%%%%%%%%%%%%%%%%%%%%%%%%%%%%%%%%%%%%%%%%%%%%%%%
\section{Information}

%%%%%%%%%%%%%%%%%%%%%%%%%%%%%%%%%%%%%%%%%%%%%%%%%%%%%%%%%%%%%%%%%%%%%%%%%%%%%%%%
\subsection{Copyright}

Copyright \copyright{} 2017--2018 Niklas Beisert

This work may be distributed and/or modified under the
conditions of the \LaTeX{} Project Public License, either version 1.3
of this license or (at your option) any later version.
The latest version of this license is in
  \url{http://www.latex-project.org/lppl.txt}
and version 1.3 or later is part of all distributions of \LaTeX{}
version 2005/12/01 or later.

This work has the LPPL maintenance status `maintained'.

The Current Maintainer of this work is Niklas Beisert.

This work consists of the files |README.txt|, |childdoc.ins| and |childdoc.dtx|
as well as the derived files |childdoc.def|, |cdocsamp.tex|
with |cdocsch1.tex|, |cdocsch2.tex|, |cdocspt3.tex|, |cdocspt4.tex|,
|cdocsdrf.tex|, |cdocsfn1.tex|, |cdocsfn2.tex|
as well as |childdoc.pdf|.

%%%%%%%%%%%%%%%%%%%%%%%%%%%%%%%%%%%%%%%%%%%%%%%%%%%%%%%%%%%%%%%%%%%%%%%%%%%%%%%%
\subsection{Files and Installation}

The package consists of the files:
%
\begin{center}
\begin{tabular}{ll}
    |README.txt|   & readme file \\
    |childdoc.ins| & installation file \\
    |childdoc.dtx| & source file \\
    |childdoc.def| & definition file \\
    |cdocsamp.tex| & sample main file \\
    |cdocsch1.tex| & sample include file \\
    |cdocsch2.tex| & sample include file \\
    |cdocspt3.tex| & sample part file \\
    |cdocspt4.tex| & sample part file \\
    |cdocsdrf.tex| & sample redirection file \\
    |cdocsfn1.tex| & sample redirection file \\
    |cdocsfn2.tex| & sample redirection file \\
    |childdoc.pdf| & manual
\end{tabular}
\end{center}
%
The distribution consists of the files
|README.txt|, |childdoc.ins| and |childdoc.dtx|.
%
\begin{itemize}
\item
Run (pdf)\LaTeX{} on |childdoc.dtx|
to compile the manual |childdoc.pdf| (this file).
\item
Run \LaTeX{} on |childdoc.ins| to create the definitions file |childdoc.def|
and the sample |cdocsamp.tex| with include files
|cdocsch1.tex|, |cdocsch2.tex|, |cdocspt3.tex|, |cdocspt4.tex|,
|cdocsdrf.tex|, |cdocsfn1.tex|, |cdocsfn2.tex|.
Then copy the file |childdoc.def| to an appropriate directory of your \LaTeX{}
distribution, e.g.\ \textit{texmf-root}|/tex/latex/childdoc|.
\end{itemize}

%%%%%%%%%%%%%%%%%%%%%%%%%%%%%%%%%%%%%%%%%%%%%%%%%%%%%%%%%%%%%%%%%%%%%%%%%%%%%%%%
\subsection{Related CTAN Packages}

There are several other packages which offer a similar functionality:
%
\begin{itemize}
\item
The packages
\href{http://ctan.org/pkg/docmute}{\textsf{docmute}},
\href{http://ctan.org/pkg/includex}{\textsf{includex}} and
\href{http://ctan.org/pkg/standalone}{\textsf{standalone}}
provide commands to include only the document body of
a child file thus allowing both files to be compiled individually.
\item
The packages \href{http://ctan.org/pkg/subdocs}{\textsf{subdocs}}
and \href{http://ctan.org/pkg/subfiles}{\textsf{subfiles}}
provide structures in which the main and child documents can be
encapsulated and allowing them to be compiled individually.
The inclusion mechanism is different from the conventional |\include|.
\item
The package \href{http://ctan.org/pkg/combine}{\textsf{combine}}
is an elaborate solution to combine several documents into one.
\end{itemize}
%
See also the CTAN topic \href{http://ctan.org/topic/subdocs}{\textsf{subdocs}}
for further related packages.
The present package differs from the above solutions in that
a document structure constructed with the conventional |\include| mechanism
just needs two extra commands at the top of every file
such that all constituent files can be compiled individually.

%%%%%%%%%%%%%%%%%%%%%%%%%%%%%%%%%%%%%%%%%%%%%%%%%%%%%%%%%%%%%%%%%%%%%%%%%%%%%%%%
%\subsection{Feature Suggestions}
%
%The following is a list of features which may be useful for future
%versions of this package:
%%
%\begin{itemize}
%\item
%\ldots
%\end{itemize}

%%%%%%%%%%%%%%%%%%%%%%%%%%%%%%%%%%%%%%%%%%%%%%%%%%%%%%%%%%%%%%%%%%%%%%%%%%%%%%%%
\subsection{Revision History}

%%%%%%%%%%%%%%%%%%%%%%%%%%%%%%%%%%%%%%%%
\paragraph{v2.0:} 2018/12/30

\begin{itemize}
\item
immediate forward processing
\item
added |\childdocby| mechanism
\item
manual restructured
\end{itemize}

%%%%%%%%%%%%%%%%%%%%%%%%%%%%%%%%%%%%%%%%
\paragraph{v1.6:} 2018/01/17

\begin{itemize}
\item
application for development of include files
\item
corrections to manual
\end{itemize}

%%%%%%%%%%%%%%%%%%%%%%%%%%%%%%%%%%%%%%%%
\paragraph{v1.5:} 2017/05/21

\begin{itemize}
\item
more complete structuring introduced
\item
|\childdocof| introduced
\item
|\childdoc| renamed to |\childdocmain|
\item
|\childredirect| renamed to |\childdocforward| and |\childdocforwardprefix|
and functionality expanded
\end{itemize}

%%%%%%%%%%%%%%%%%%%%%%%%%%%%%%%%%%%%%%%%
\paragraph{v1.0:} 2017/04/27

\begin{itemize}
\item
manual and install package
\item
first version published on CTAN
\end{itemize}

%%%%%%%%%%%%%%%%%%%%%%%%%%%%%%%%%%%%%%%%
\paragraph{v0.6:} 2017/04/26

\begin{itemize}
\item
redirection mechanism added
\end{itemize}

%%%%%%%%%%%%%%%%%%%%%%%%%%%%%%%%%%%%%%%%
\paragraph{v0.5:} 2017/04/26

\begin{itemize}
\item
functionality in definition file
\end{itemize}


%%%%%%%%%%%%%%%%%%%%%%%%%%%%%%%%%%%%%%%%%%%%%%%%%%%%%%%%%%%%%%%%%%%%%%%%%%%%%%%%
%%%%%%%%%%%%%%%%%%%%%%%%%%%%%%%%%%%%%%%%%%%%%%%%%%%%%%%%%%%%%%%%%%%%%%%%%%%%%%%%
%%%%%%%%%%%%%%%%%%%%%%%%%%%%%%%%%%%%%%%%%%%%%%%%%%%%%%%%%%%%%%%%%%%%%%%%%%%%%%%%
\appendix

\settowidth\MacroIndent{\rmfamily\scriptsize 000\ }

 \DocInput{childdoc.dtx}

\end{document}
%</driver>
% \fi
%
% %%%%%%%%%%%%%%%%%%%%%%%%%%%%%%%%%%%%%%%%%%%%%%%%%%%%%%%%%%%%%%%%%%%%%%%%%%%%%%
% %%%%%%%%%%%%%%%%%%%%%%%%%%%%%%%%%%%%%%%%%%%%%%%%%%%%%%%%%%%%%%%%%%%%%%%%%%%%%%
% \section{Sample}
%\iffalse
%<*samplemain>
%\fi
%
% The following presents a sample document
% with two chapters, two parts, a title page,
% a compile flag as well as three forwarding files to set the flag.
% It consists of eight |.tex| files:
% \begin{center}
% \begin{tabular}{ll}
% |cdocsamp.tex|&main file\\
% |cdocsch1.tex|&include file for chapter 1\\
% |cdocsch2.tex|&include file for chapter 2\\
% |cdocspt3.tex|&include file for part 3\\
% |cdocspt4.tex|&include file for part 4\\
% |cdocsdrf.tex|&forwarding file for main file in draft mode\\
% |cdocsfi1.tex|&forwarding file for final version of chapter 1\\
% |cdocsfi2.tex|&forwarding file for final version of chapter 2\\
% \end{tabular}
% \end{center}
% Each of the eight files can be compiled directly by the \LaTeX{} compiler.
%
% %%%%%%%%%%%%%%%%%%%%%%%%%%%%%%%%%%%%%%
% \paragraph{Main File.}
%
% The main file is called |cdocsamp.tex|.
%
% Load the \textsf{childdoc} definitions and
% declare the filename for the main document:
%    \begin{macrocode}
\input{childdoc.def}
\childdocmain{}
%    \end{macrocode}

% Optional override for |\version| flag:
%    \begin{macrocode}
%%\ifchilddoc\else\providecommand{\version}{draft}\fi
%    \end{macrocode}

% Define the default values for the |\version| flag
% (|final| for the main file and |draft| for childs):
%    \begin{macrocode}
\ifchilddoc
\providecommand{\version}{draft}
\else
\providecommand{\version}{final}
\fi
%    \end{macrocode}

% Load the standard document class:
%    \begin{macrocode}
\documentclass[12pt]{article}
%    \end{macrocode}

% Start the document body:
%    \begin{macrocode}
\begin{document}
%    \end{macrocode}

% Declare a title page.
% Print title, part of document being processed and version flag:
%    \begin{macrocode}
\addtocounter{page}{-1}
\begin{center}
{\LARGE\bfseries{}childdoc example\par}
\vspace{1cm}
\ifchilddoc
\ifchilddocmanual part\else chapter\fi:
`\childdocname' of `\childdocjob'\par
\else
main document: `\childdocjob'\par
\fi
version: \version\par
\end{center}
\newpage
%    \end{macrocode}

% Manually include selected file,
% otherwise process as usual:
%    \begin{macrocode}
\ifchilddocmanual
\section*{part `\childdocname'}
\input{\childdocname}
\else
%    \end{macrocode}

% Include the two chapters:
%    \begin{macrocode}
\include{cdocsch1}
\include{cdocsch2}
%    \end{macrocode}

% Include the two parts unless only chapters should be displayed:
%    \begin{macrocode}
\ifchilddoc\else
\section{part three}
\input{cdocspt3}
\section{part four}
\input{cdocspt4}
\fi
%    \end{macrocode}

% Process as usual until here:
%    \begin{macrocode}
\fi
%    \end{macrocode}

% End of document body:
%    \begin{macrocode}
\end{document}
%    \end{macrocode}
%\iffalse
%</samplemain>
%\fi
%
% %%%%%%%%%%%%%%%%%%%%%%%%%%%%%%%%%%%%%%
% \paragraph{Chapter Include Files.}
%
% The include files are called |cdocsch1.tex| and |cdocsch2.tex|.
%
%\iffalse
%<*samplechap1|samplechap2>
%\fi

% Optional override for |\version| flag:
%    \begin{macrocode}
%%\providecommand{\version}{final}
%    \end{macrocode}

% Include the main document:
%    \begin{macrocode}
\input{childdoc.def}
\childdocof{cdocsamp}
%    \end{macrocode}

%\iffalse
%</samplechap1|samplechap2>
%\fi
%
%\iffalse
%<*samplechap1>
%\fi
% Some text for chapter 1:
%    \begin{macrocode}
\section{one}
some text in chapter one
%    \end{macrocode}

%\iffalse
%</samplechap1>
%\fi
% Some text for chapter 2:
%\iffalse
%<*samplechap2>
%\fi
%    \begin{macrocode}
\section{two}
more text in chapter two
%    \end{macrocode}

%\iffalse
%</samplechap2>
%\fi
%
% %%%%%%%%%%%%%%%%%%%%%%%%%%%%%%%%%%%%%%
% \paragraph{Part Include Files.}
%
% The include files are called |cdocspt3.tex| and |cdocspt4.tex|.
%
%\iffalse
%<*samplepart3|samplepart4>
%\fi

% Optional override for |\version| flag:
%    \begin{macrocode}
%%\providecommand{\version}{final}
%    \end{macrocode}

% Include the main document:
%    \begin{macrocode}
\input{childdoc.def}
\childdocby{cdocsamp}
%    \end{macrocode}

%\iffalse
%</samplepart3|samplepart4>
%\fi
%
%\iffalse
%<*samplepart3>
%\fi
% Some text for part 3:
%    \begin{macrocode}
some text in part three
%    \end{macrocode}

%\iffalse
%</samplepart3>
%\fi
% Some text for part 4:
%\iffalse
%<*samplepart4>
%\fi
%    \begin{macrocode}
more text in part four
%    \end{macrocode}

%\iffalse
%</samplepart4>
%\fi
%
% %%%%%%%%%%%%%%%%%%%%%%%%%%%%%%%%%%%%%%
% \paragraph{Forwarding for a Complete Draft.}
%
% The following forwarding file |cdocsdrf.tex|
% compiles the main document in draft mode:
%\iffalse
%<*sampledraft>
%\fi
%    \begin{macrocode}
\def\version{draft}
\input{childdoc.def}
\childdocforward{cdocsamp}
%    \end{macrocode}

%\iffalse
%</sampledraft>
%\fi
%
% %%%%%%%%%%%%%%%%%%%%%%%%%%%%%%%%%%%%%%
% \paragraph{Forwarding for Final Version of the Chapters.}
%
% The following forwarding files |cdocsfn1.tex| and |cdocsfn2.tex|
% (with identical content)
% compile the final versions of the child documents
% |cdocsch1.tex| and |cdocsch2.tex|, respectively:
%\iffalse
%<*samplefinal>
%\fi
%    \begin{macrocode}
\def\version{final}
\input{childdoc.def}
\childdocforwardprefix[cdocsamp]{cdocsfn}{cdocsch}
%    \end{macrocode}

%\iffalse
%</samplefinal>
%\fi
%
% %%%%%%%%%%%%%%%%%%%%%%%%%%%%%%%%%%%%%%
% \paragraph{Command Line Processing.}
%
% The following three command lines generate the output files
% |cdocscld|, |cdocscl1| and |cdocscl2|
% which should be identical to
% |cdocsdrf|, |cdocsch1| and |cdocsfn2|, respectively:
% \begin{center}
% \begin{tabular}{l}
% |latex -jobname cdocscld \|\\
% |  "\def\version{draft}\input{childdoc.def}\childdocforward{cdocsamp}"|\\
% |latex -jobname cdocscl1 \|\\
% |  "\input{childdoc.def}\childdocforward[cdocsamp]{cdocsch1}"|\\
% |latex -jobname cdocscl2 \|\\
% |  "\def\version{final}\input{childdoc.def}\childdocforward{cdocsch2}"|
% \end{tabular}
% \end{center}
% Note that the trailing backslash on each first line
% merely continues the input to the second line
% (for convenient cut ant paste).
% Furthermore, the command |latex| can be replaced by any
% of its alternative versions such as |pdflatex|.
%
% %%%%%%%%%%%%%%%%%%%%%%%%%%%%%%%%%%%%%%%%%%%%%%%%%%%%%%%%%%%%%%%%%%%%%%%%%%%%%%
% %%%%%%%%%%%%%%%%%%%%%%%%%%%%%%%%%%%%%%%%%%%%%%%%%%%%%%%%%%%%%%%%%%%%%%%%%%%%%%
% \section{Implementation}
%\iffalse
%<*package>
%\fi
%
% This section describes the definitions file |childdoc.def|.

% The definitions cannot be loaded using |\usepackage| or |\RequirePackage|
% which has a mechanism to prevent loading a style file more than once.
% When loading the definitions by means of |\input|
% multiple instances have to be prevented manually:
%\iffalse
%This code needs to be before the `\ProvidesFile' directive
%which is defined at the beginning of this file.
%Therefore it is also placed there and commented out here.
%</package>
%<*discard>
%\fi
%    \begin{macrocode}
\ifdefined\childdocmain\endinput\fi
%    \end{macrocode}
%\iffalse
%</discard>
%<*package>
%\fi
%
% \macro{\ifchilddoc}
% \macro{\ifchilddocmanual}
% The conditional |\ifchilddoc| tells whether a
% child (true) or main (false) document is being compiled.
% The conditional |\ifchilddocmanual| tells whether
% the |\includeonly| mechanism is used (false) or
% the selection of child files must be performed manually (true).
% The definitions initialise to false:
%    \begin{macrocode}
\newif\ifchilddoc
\newif\ifchilddocmanual
%    \end{macrocode}

% \macro{\childdocname}
% \macro{\childdocjob}
% The macro |\childdocname| stores the name of the main document
% to be compiled. The macro |\childdocjob| stores the name of
% the document on which the \LaTeX{} compiler was originally invoked.
% The content of |\jobname| cannot be compared
% to filenames specified in the source due to different catcodes.
% The following code rescans |\jobname|, stores the result
% in |\childdocname| and saves a copy in |\childdocjob|:
%    \begin{macrocode}
\edef\childdocname{\scantokens\expandafter{\jobname\noexpand}}
\let\childdocjob\childdocname
%    \end{macrocode}

% \macro{\childdocdisable}
% The macro |\childdocdisable| prevents the main file
% from being processed more than once.
% At this stage, the main document command |\childdocmain|
% is assumed to be called once again where it should do nothing.
% Any subsequent call to it should prevent
% a secondary processing of the main document
% It overwrites the forwarding commands
% |\childdocof| and |\childdocforward|
% with empty macros to prevent further inclusions of the main document:
%    \begin{macrocode}
\newcommand{\childdocdisable}
{
  \renewcommand{\childdocmain}[1]{\renewcommand{\childdocmain}[1]{\endinput}}
  \renewcommand{\childdocof}[1]{}
  \renewcommand{\childdocby}[2][]{}
  \renewcommand{\childdocforward}[2][]{}
  \renewcommand{\childdocdisable}{}
}
%    \end{macrocode}

% \macro{\childdocmain}
% The macro |\childdocmain| is to be called at the top of the main file
% with nothing or the main filename (without extension) as argument.
% First, it breaks loops.
% If the argument is not empty and does not match |\childdocname|
% (which is set by the first inclusion of |childdoc.def|),
% |\ifchilddoc| is set to true, |\includeonly| is applied to the child file
% and |\jobname| is set to the main file
% (for proper handling of |.aux| files):
%    \begin{macrocode}
\newcommand{\childdocmain}[1]
{
  \childdocdisable\childdocmain{}
  \if?#1?\else
    \begingroup
      \def\childdoctmp{#1}
      \ifx\childdoctmp\childdocname
        \def\childdoctmp{}
      \else
        \def\childdoctmp
        {
          \childdoctrue
          \includeonly{\childdocname}
          \def\childdocjob{#1}
          \def\jobname{#1}
        }
      \fi
      \expandafter
    \endgroup
    \childdoctmp
  \fi
}
%    \end{macrocode}

% \macro{\childdocof}
% The command |\childdocof| redirects
% compilation to the main file |#1|.
%    \begin{macrocode}
\newcommand{\childdocof}[1]
{
  \childdocdisable
  \childdoctrue
  \includeonly{\childdocname}
  \def\jobname{#1}
  \def\childdocjob{#1}
  \input{#1}
}
%    \end{macrocode}

% \macro{\childdocby}
% The command |\childdocby| ....
%    \begin{macrocode}
\newcommand{\childdocby}[2][]
{
  \childdocdisable
  \childdoctrue
  \childdocmanualtrue
  \if?#1?\else
    \def\jobname{#2}
  \fi
  \def\childdocjob{#2}
  \input{#2}
  \endinput
}
%    \end{macrocode}

% \macro{\childdocforward}
% The command |\childdocforward| redirects
% compilation to the main file or
% (if the optional argument is given) a child file.
% Parameters are set as if the main file
% or a child file starting with |\childdocof| was compiled.
% Then compilation is handed over to the main file:
%    \begin{macrocode}
\newcommand{\childdocforward}[2][]
{
  \begingroup
    \if?#1?
      \def\childdoctmp
      {
        \def\childdocname{#2}
        \def\childdocjob{#2}
        \def\jobname{#2}
        \input{#2}
        \endinput
      }
    \else
      \def\childdoctmp
      {
        \childdocdisable
        \def\childdocname{#2}
        \childdoctrue
        \includeonly{#2}
        \def\childdocjob{#1}
        \def\jobname{#1}
        \input{#1}
        \endinput
      }
    \fi
    \expandafter
  \endgroup
  \childdoctmp
}
%    \end{macrocode}

% \macro{\childdocforwardprefix}
% The command |\childdocforwardprefix| redirects
% compilation to the main or a child file by means of a pattern.
% The prefix |#1| in the current filename is replaced by |#2|
% and the suffix of the current filename is kept
% (it is assumed that the filename does not contain the substring `|~~~|'
% which is used as a delimiter).
% Compilation is handed over to the new file by |\childdocforward|:
%    \begin{macrocode}
\newcommand{\childdocforwardprefix}[3][]
{
  \begingroup
    \def\childdocextract #2##1~~~{\def\childdoctmp{\childdocforward[#1]{#3##1}}}
    \expandafter\childdocextract\childdocname~~~
    \expandafter
  \endgroup
  \childdoctmp
}
%    \end{macrocode}

% \macro{\childdoc}
% The deprecated macro |\childdoc| is a legacy version of |\childdocmain|:
%    \begin{macrocode}
\newcommand{\childdoc}{\childdocmain}
%    \end{macrocode}

% \macro{\childdocredirect}
% The deprecated macro |\childdocredirect| is a legacy version
% of |\childdocforward| and |\childdocforwardprefix|:
%    \begin{macrocode}
\newcommand{\childdocredirect}[2][]
{
  \begingroup
    \if?#1?
      \def\childdoctmp{\childdocforward{#2}}
    \else
      \def\childdoctmp{\childdocforwardprefix{#1}{#2}}
    \fi
    \expandafter
  \endgroup
  \childdoctmp
}
%    \end{macrocode}

%\iffalse
%</package>
%\fi
%
\endinput
|\\
|\childdocof{|\textit{main}|}|\\
\end{tabular}
\end{center}
at the top of every child file \textit{child}
which is included by |\include{|\textit{child}|}|
from within the main file
(or at least for those files to be compiled individually).
The argument \textit{main} must be the filename of the main file.

There are a couple of
considerations in setting up the main and child documents:

%%%%%%%%%%%%%%%%%%%%%%%%%%%%%%%%%%%%%%%%
\paragraph{Restrictions.}

Please note the following restrictions:
\begin{itemize}
\item
|\childdocmain| must be called with one argument \textit{main}
to ensure compatibility with earlier version of the package.
It must either be empty (|\childdocmain{}|)
or precisely match the filename of the main file in which it is specified.
See \secref{sec:detection} for further information.
\item
The filename \textit{main} must be specified without the |.tex| extension.
\item
The filename \textit{main} is case sensitive
(even in case-insensitive file systems)
due to internal string comparison.
\item
The argument \textit{main} should be fully expanded, it cannot be a macro.
\item
Subdirectories and special characters should be avoided in filenames.
\item
The command |\childdocmain{|\textit{main}|}| must be followed by a whitespace.
It should not be followed immediately by another command
or by a comment mark `|%|'.
This is because the \TeX{} parser reads the token immediately following
the argument of |\childdocmain| and puts it
at the beginning of every child section;
however, a white\-space is ignored.
\end{itemize}

%%%%%%%%%%%%%%%%%%%%%%%%%%%%%%%%%%%%%%%%
\paragraph{Content of Main File.}

It is advisable to place all content in the child files included by |\include|.
Any output contained in the main file will appear in all child documents
unless suppressed manually;
it cannot be suppressed automatically by the |\includeonly| directive
and thus should normally be avoided.
A method to include some content in the main file
by means of conditional processing is described in \secref{sec:conditional}.

%%%%%%%%%%%%%%%%%%%%%%%%%%%%%%%%%%%%%%%%
\paragraph{Page Numbering.}

When only a part of the document is compiled,
the appropriate numbering of pages
(as well as other status parameters)
is determined from the |.aux| files.
The latter contain information from previous passes.
However this information needs to propagate through
all intermediate child documents.
Therefore the page numbering in child documents may well
be inconsistent until the complete document is compiled at least once.

A useful (if unconventional) way to always ensure a consistent
page numbering is to restart the numbering in each child document
and denote the pages by `\textit{child}|.|\textit{page}'
where \textit{child} represents the chapter/section number of the child file.
This can be achieved by the command
|\numberwithin{page}{|\textit{child}|}|
of the \textsf{amsmath} package
where \textit{child} can be |chapter| or |section|
depending on the chosen structuring.
Alternatively, one can modify the macro |\thepage| appropriately
and reset the counter |page| at the start of each child file.

%%%%%%%%%%%%%%%%%%%%%%%%%%%%%%%%%%%%%%%%%%%%%%%%%%%%%%%%%%%%%%%%%%%%%%%%%%%%%%%%
\subsection{Conditional Processing}
\label{sec:conditional}

The package provides a mechanism to compile different versions
of a document. To customise the versions further some conditional processing
can come in handy to distinguish which version is being compiled.
The package provides two macros to describe the compilation context:

%%%%%%%%%%%%%%%%%%%%%%%%%%%%%%%%%%%%%%%%
\DescribeMacro{\ifchilddoc}
The conditional |\ifchilddoc| distinguishes between the compilation of
child documents and the main document:
%
\begin{center}
|\ifchilddoc |\textit{child-code}| |[|\||else |\textit{main-code}]| \||fi|
\end{center}

%%%%%%%%%%%%%%%%%%%%%%%%%%%%%%%%%%%%%%%%
\DescribeMacro{\childdocname}
\DescribeMacro{\childdocjob}
The macro |\childdocname| contains the filename (without extension)
of the main or child file being processed.
Note that |\childdocjob| will always contain the name of the main file.

%%%%%%%%%%%%%%%%%%%%%%%%%%%%%%%%%%%%%%%%
\paragraph{Title Page.}

Conditional processing can be used to include a title or banner page
in the main document when proper precautions are taken.
Importantly, the code in the main file should ensure that the page counter
(as well as other status parameters which are stored in the |.aux| files)
takes the same value after the conditional processing.
Otherwise the page numbers may take divergent values
depending on which part is compiled.

For example, a title page could be declared by:
%
\begin{center}
\begin{tabular}{l}
|\ifchilddoc\||else|\\
|\addtocounter{page}{-1}|\\
\textit{code for title page}\\
|\newpage|\\
|\||fi|
\end{tabular}
\end{center}
%
A banner page for the child documents can be generated by:
%
\begin{center}
\begin{tabular}{l}
|\ifchilddoc|\\
|\addtocounter{page}{-1}|\\
\textit{code for banner page}\\
|\newpage|\\
|\||fi|
\end{tabular}
\end{center}
%
Here one could write a message such as:
\begin{center}
|This is the part \childdocname{} of \childdocjob{}.|
\end{center}

%%%%%%%%%%%%%%%%%%%%%%%%%%%%%%%%%%%%%%%%%%%%%%%%%%%%%%%%%%%%%%%%%%%%%%%%%%%%%%%%
\subsection{Flags}
\label{sec:flags}

The package makes it easy to generate different versions
of the main or child documents.
To this end compilation flags can be defined
and assigned different default values.
They will be particularly useful in conjunction
with the forwarding mechanism described in \secref{sec:forward}.

For example, it may be useful to have a flag |\version|
which can be set to |draft| or |final|.
The document source will contain some conditional code
depending on the value of |\version|.
Suppose further, the flag should default to |final| for the main file
and to |draft| for child files
which is a natural assignment for editing the document.
This is achieved by placing the following code
in the preamble of the main document
(below the |\childdocmain| directive):
%
\begin{center}
\begin{tabular}{l}
|\ifchilddoc|\\
|\providecommand{\version}{draft}|\\
|\||else|\\
|\providecommand{\version}{final}|\\
|\||fi|
\end{tabular}
\end{center}
%
The definition by |\providecommand| makes sure
that previous definitions are not overwritten.
Further statements |\providecommand{\version}{...}|
can thus be added before the above code to override it.

For the main file, one might add a line
(between |\childdocmain| and the above block)
%
\begin{center}
|%\ifchilddoc\||else\providecommand{\version}{draft}\||fi|
\end{center}
%
which can be uncommented to produce a draft version.
Likewise one can add a line to the very top of a child file
(above the |\childdocof{|\textit{main}|}| directive)
%
\begin{center}
|%\providecommand{\version}{final}|
\end{center}
%
which can be uncommented to produce the final version of this child document.

%%%%%%%%%%%%%%%%%%%%%%%%%%%%%%%%%%%%%%%%%%%%%%%%%%%%%%%%%%%%%%%%%%%%%%%%%%%%%%%%
\subsection{Forwarding}
\label{sec:forward}

Different versions of the main or child documents
using compilation flags as described in \secref{sec:flags}
can be (permanently) stored in different files
for convenient compilation, viewing and distribution.
To this end, the package defines a command
to pass on compilation to a different file:

%%%%%%%%%%%%%%%%%%%%%%%%%%%%%%%%%%%%%%%%
\DescribeMacro{\childdocforward}
The command |\childdocforward| redirects processing to
another source file:
%
\begin{center}
\begin{tabular}{l}
|% \iffalse
%
% childdoc.dtx Copyright (C) 2017-2018 Niklas Beisert
%
% This work may be distributed and/or modified under the
% conditions of the LaTeX Project Public License, either version 1.3
% of this license or (at your option) any later version.
% The latest version of this license is in
%   http://www.latex-project.org/lppl.txt
% and version 1.3 or later is part of all distributions of LaTeX
% version 2005/12/01 or later.
%
% This work has the LPPL maintenance status `maintained'.
%
% The Current Maintainer of this work is Niklas Beisert.
%
% This work consists of the files childdoc.dtx and childdoc.ins
% and the derived files childdoc.def and cdocsamp.tex with
% cdocsch1.tex, cdocsch2.tex, cdocsdrf.tex, cdocsfn1.tex, cdocsfn2.tex.
%
%<package>\ifdefined\childdocmain\endinput\fi
%<package>\ProvidesFile{childdoc.def}[2018/12/30 v2.0 child document driver]
%<samplemain>\ProvidesFile{cdocsamp.tex}[2018/12/30 v2.0 sample for childdoc]
%<*driver>
%\ProvidesFile{childdoc.drv}[2018/12/30 v2.0 childdoc reference manual file]
\PassOptionsToClass{10pt,a4paper}{article}
\documentclass{ltxdoc}

\usepackage[margin=35mm]{geometry}
\usepackage{hyperref}
\usepackage{hyperxmp}
\usepackage[usenames]{color}

\hypersetup{colorlinks=true}
\hypersetup{pdfstartview=FitH}
\hypersetup{pdfpagemode=UseNone}
\hypersetup{pdfsource={}}
\hypersetup{pdflang={en-UK}}
\hypersetup{pdfcopyright={Copyright 2017-2018 Niklas Beisert.
  This work may be distributed and/or modified under the
  conditions of the LaTeX Project Public License, either version 1.3
  of this license or (at your option) any later version.}}
\hypersetup{pdflicenseurl={http://www.latex-project.org/lppl.txt}}
\hypersetup{pdfcontactaddress={ETH Zurich, ITP, HIT K,
  Wolfgang-Pauli-Strasse 27}}
\hypersetup{pdfcontactpostcode={8093}}
\hypersetup{pdfcontactcity={Zurich}}
\hypersetup{pdfcontactcountry={Switzerland}}
\hypersetup{pdfcontactemail={nbeisert@itp.phys.ethz.ch}}
\hypersetup{pdfcontacturl={http://people.phys.ethz.ch/\xmptilde nbeisert/}}

\newcommand{\secref}[1]{\hyperref[#1]{section \ref*{#1}}}

\parskip1ex
\parindent0pt
\let\olditemize\itemize
\def\itemize{\olditemize\parskip0pt}

\begin{document}

\title{The \textsf{childdoc} Package}
\hypersetup{pdftitle={The childdoc Package}}
\author{Niklas Beisert\\[2ex]
  Institut f\"ur Theoretische Physik\\
  Eidgen\"ossische Technische Hochschule Z\"urich\\
  Wolfgang-Pauli-Strasse 27, 8093 Z\"urich, Switzerland\\[1ex]
  \href{mailto:nbeisert@itp.phys.ethz.ch}
  {\texttt{nbeisert@itp.phys.ethz.ch}}}
\hypersetup{pdfauthor={Niklas Beisert}}
\hypersetup{pdfsubject={Manual for the LaTeX2e Package childdoc}}
\date{30 December 2018, \textsf{v2.0}}
\maketitle

\begin{abstract}\noindent
\textsf{childdoc} is a \LaTeXe{} package
that enables the direct compilation
of document sections included by |\include|
to individual files.
\end{abstract}

\begingroup
\parskip0ex
\tableofcontents
\endgroup

%%%%%%%%%%%%%%%%%%%%%%%%%%%%%%%%%%%%%%%%%%%%%%%%%%%%%%%%%%%%%%%%%%%%%%%%%%%%%%%%
%%%%%%%%%%%%%%%%%%%%%%%%%%%%%%%%%%%%%%%%%%%%%%%%%%%%%%%%%%%%%%%%%%%%%%%%%%%%%%%%
\section{Introduction}

\LaTeX{} provides a mechanism to structure a large document (such as a book)
into a main file and several child files (containing the chapters)
using the |\include| command.
This mechanism is beneficial for documents
which span hundreds of pages in order to
make the source file(s) more manageable.
Moreover, compilation can be restricted to
selected child files by means of the |\includeonly| command.
The latter feature can be used to reduce the compilation time while editing
(this was significantly more useful in the earlier days of \LaTeX{})
or to generate a smaller document which is easier to navigate.
Another application of |\includeonly| is to generate
documents consisting of selected parts of the complete document.

However, there are a few drawbacks of the plain |\include| mechanism:
\begin{itemize}
\item
The child files cannot be compiled on their own,
they can only be compiled via the main file.
A naive editing environment
(such as a text editor with an option
to have the current file processed by \LaTeX)
may require one to switch to the main file before compiling;
attempting to compile the child file produces errors.
\item
The main file must be modified (each time)
to adjust the |\includeonly| command
to the present needs. This easily leaves the main file in a messy state.
\item
The generated document will always carry the filename
of the main document. This is inconvenient if
several child files are to be compiled and
to be kept for distribution.
\end{itemize}

The present package provides a simple interface
to make child files individually compilable by \LaTeX{}.
Compiling a child file then has the same effect as compiling
the main file with an |\includeonly| command
to select the appropriate child.
Moreover the generated document will carry the name of the child
rather than the main file.
This resolves all three above issues.

This feature is meant to make the editing of books,
thesis documents and lecture notes somewhat more convenient.
However, the package can also be used efficiently for
composing a series of documents (such as exercise sheets)
which are typically distributed individually.
It then assists the author in generating the individual documents
(potentially in different versions)
as well as a document containing the collected series.
Another application is in developing style files
or other kinds of included material
where compilation of the style file could redirect
to a sample or test file.

%%%%%%%%%%%%%%%%%%%%%%%%%%%%%%%%%%%%%%%%%%%%%%%%%%%%%%%%%%%%%%%%%%%%%%%%%%%%%%%%
%%%%%%%%%%%%%%%%%%%%%%%%%%%%%%%%%%%%%%%%%%%%%%%%%%%%%%%%%%%%%%%%%%%%%%%%%%%%%%%%
\section{Usage}

First of all, the package \textsf{childdoc} is \emph{not} a standard
\LaTeXe{} |.sty| style file! Therefore it needs to be invoked in
a non-standard way.

%%%%%%%%%%%%%%%%%%%%%%%%%%%%%%%%%%%%%%%%%%%%%%%%%%%%%%%%%%%%%%%%%%%%%%%%%%%%%%%%
\subsection{Included Files}
\label{sec:include}

%%%%%%%%%%%%%%%%%%%%%%%%%%%%%%%%%%%%%%%%
\DescribeMacro{\childdocmain}
To use the package, add the commands
\begin{center}
\begin{tabular}{l}
|\input{childdoc.def}|\\
|\childdocmain{}|\\
\end{tabular}
\end{center}
at the very top of the main \LaTeX{} file,
in particular \emph{before} the |\documentclass| statement!
The argument of |\childdocmain| should be left empty
(but it must be present).

%%%%%%%%%%%%%%%%%%%%%%%%%%%%%%%%%%%%%%%%
\DescribeMacro{\childdocof}
Furthermore, add the commands
\begin{center}
\begin{tabular}{l}
|\input{childdoc.def}|\\
|\childdocof{|\textit{main}|}|\\
\end{tabular}
\end{center}
at the top of every child file \textit{child}
which is included by |\include{|\textit{child}|}|
from within the main file
(or at least for those files to be compiled individually).
The argument \textit{main} must be the filename of the main file.

There are a couple of
considerations in setting up the main and child documents:

%%%%%%%%%%%%%%%%%%%%%%%%%%%%%%%%%%%%%%%%
\paragraph{Restrictions.}

Please note the following restrictions:
\begin{itemize}
\item
|\childdocmain| must be called with one argument \textit{main}
to ensure compatibility with earlier version of the package.
It must either be empty (|\childdocmain{}|)
or precisely match the filename of the main file in which it is specified.
See \secref{sec:detection} for further information.
\item
The filename \textit{main} must be specified without the |.tex| extension.
\item
The filename \textit{main} is case sensitive
(even in case-insensitive file systems)
due to internal string comparison.
\item
The argument \textit{main} should be fully expanded, it cannot be a macro.
\item
Subdirectories and special characters should be avoided in filenames.
\item
The command |\childdocmain{|\textit{main}|}| must be followed by a whitespace.
It should not be followed immediately by another command
or by a comment mark `|%|'.
This is because the \TeX{} parser reads the token immediately following
the argument of |\childdocmain| and puts it
at the beginning of every child section;
however, a white\-space is ignored.
\end{itemize}

%%%%%%%%%%%%%%%%%%%%%%%%%%%%%%%%%%%%%%%%
\paragraph{Content of Main File.}

It is advisable to place all content in the child files included by |\include|.
Any output contained in the main file will appear in all child documents
unless suppressed manually;
it cannot be suppressed automatically by the |\includeonly| directive
and thus should normally be avoided.
A method to include some content in the main file
by means of conditional processing is described in \secref{sec:conditional}.

%%%%%%%%%%%%%%%%%%%%%%%%%%%%%%%%%%%%%%%%
\paragraph{Page Numbering.}

When only a part of the document is compiled,
the appropriate numbering of pages
(as well as other status parameters)
is determined from the |.aux| files.
The latter contain information from previous passes.
However this information needs to propagate through
all intermediate child documents.
Therefore the page numbering in child documents may well
be inconsistent until the complete document is compiled at least once.

A useful (if unconventional) way to always ensure a consistent
page numbering is to restart the numbering in each child document
and denote the pages by `\textit{child}|.|\textit{page}'
where \textit{child} represents the chapter/section number of the child file.
This can be achieved by the command
|\numberwithin{page}{|\textit{child}|}|
of the \textsf{amsmath} package
where \textit{child} can be |chapter| or |section|
depending on the chosen structuring.
Alternatively, one can modify the macro |\thepage| appropriately
and reset the counter |page| at the start of each child file.

%%%%%%%%%%%%%%%%%%%%%%%%%%%%%%%%%%%%%%%%%%%%%%%%%%%%%%%%%%%%%%%%%%%%%%%%%%%%%%%%
\subsection{Conditional Processing}
\label{sec:conditional}

The package provides a mechanism to compile different versions
of a document. To customise the versions further some conditional processing
can come in handy to distinguish which version is being compiled.
The package provides two macros to describe the compilation context:

%%%%%%%%%%%%%%%%%%%%%%%%%%%%%%%%%%%%%%%%
\DescribeMacro{\ifchilddoc}
The conditional |\ifchilddoc| distinguishes between the compilation of
child documents and the main document:
%
\begin{center}
|\ifchilddoc |\textit{child-code}| |[|\||else |\textit{main-code}]| \||fi|
\end{center}

%%%%%%%%%%%%%%%%%%%%%%%%%%%%%%%%%%%%%%%%
\DescribeMacro{\childdocname}
\DescribeMacro{\childdocjob}
The macro |\childdocname| contains the filename (without extension)
of the main or child file being processed.
Note that |\childdocjob| will always contain the name of the main file.

%%%%%%%%%%%%%%%%%%%%%%%%%%%%%%%%%%%%%%%%
\paragraph{Title Page.}

Conditional processing can be used to include a title or banner page
in the main document when proper precautions are taken.
Importantly, the code in the main file should ensure that the page counter
(as well as other status parameters which are stored in the |.aux| files)
takes the same value after the conditional processing.
Otherwise the page numbers may take divergent values
depending on which part is compiled.

For example, a title page could be declared by:
%
\begin{center}
\begin{tabular}{l}
|\ifchilddoc\||else|\\
|\addtocounter{page}{-1}|\\
\textit{code for title page}\\
|\newpage|\\
|\||fi|
\end{tabular}
\end{center}
%
A banner page for the child documents can be generated by:
%
\begin{center}
\begin{tabular}{l}
|\ifchilddoc|\\
|\addtocounter{page}{-1}|\\
\textit{code for banner page}\\
|\newpage|\\
|\||fi|
\end{tabular}
\end{center}
%
Here one could write a message such as:
\begin{center}
|This is the part \childdocname{} of \childdocjob{}.|
\end{center}

%%%%%%%%%%%%%%%%%%%%%%%%%%%%%%%%%%%%%%%%%%%%%%%%%%%%%%%%%%%%%%%%%%%%%%%%%%%%%%%%
\subsection{Flags}
\label{sec:flags}

The package makes it easy to generate different versions
of the main or child documents.
To this end compilation flags can be defined
and assigned different default values.
They will be particularly useful in conjunction
with the forwarding mechanism described in \secref{sec:forward}.

For example, it may be useful to have a flag |\version|
which can be set to |draft| or |final|.
The document source will contain some conditional code
depending on the value of |\version|.
Suppose further, the flag should default to |final| for the main file
and to |draft| for child files
which is a natural assignment for editing the document.
This is achieved by placing the following code
in the preamble of the main document
(below the |\childdocmain| directive):
%
\begin{center}
\begin{tabular}{l}
|\ifchilddoc|\\
|\providecommand{\version}{draft}|\\
|\||else|\\
|\providecommand{\version}{final}|\\
|\||fi|
\end{tabular}
\end{center}
%
The definition by |\providecommand| makes sure
that previous definitions are not overwritten.
Further statements |\providecommand{\version}{...}|
can thus be added before the above code to override it.

For the main file, one might add a line
(between |\childdocmain| and the above block)
%
\begin{center}
|%\ifchilddoc\||else\providecommand{\version}{draft}\||fi|
\end{center}
%
which can be uncommented to produce a draft version.
Likewise one can add a line to the very top of a child file
(above the |\childdocof{|\textit{main}|}| directive)
%
\begin{center}
|%\providecommand{\version}{final}|
\end{center}
%
which can be uncommented to produce the final version of this child document.

%%%%%%%%%%%%%%%%%%%%%%%%%%%%%%%%%%%%%%%%%%%%%%%%%%%%%%%%%%%%%%%%%%%%%%%%%%%%%%%%
\subsection{Forwarding}
\label{sec:forward}

Different versions of the main or child documents
using compilation flags as described in \secref{sec:flags}
can be (permanently) stored in different files
for convenient compilation, viewing and distribution.
To this end, the package defines a command
to pass on compilation to a different file:

%%%%%%%%%%%%%%%%%%%%%%%%%%%%%%%%%%%%%%%%
\DescribeMacro{\childdocforward}
The command |\childdocforward| redirects processing to
another source file:
%
\begin{center}
\begin{tabular}{l}
|\input{childdoc.def}|\\
|\childdocforward[|\textit{main}|]{|\textit{dest}|}|\\
\end{tabular}
\end{center}
%
The argument \textit{dest} is the destination file
(without extension).
It should be the main file or one of the child files.
Note that further \textsf{childdoc} directives
such as |\childdocof| and |\childdocforward|
in the indicated file will be processed in this form.
The optional argument \textit{main}
passes on directly to the main file \textit{main}
while pretending to compile the child \textit{dest}.
This form behaves as if \textit{dest}
issues |\childdocof{|\textit{main}|}| right away,
and no further \textsf{childdoc} directives will be processed.

%%%%%%%%%%%%%%%%%%%%%%%%%%%%%%%%%%%%%%%%
\DescribeMacro{\...prefix}
In the alternative form |\childdocforwardprefix|,
%
\begin{center}
\begin{tabular}{l}
|\input{childdoc.def}|\\
|\childdocforwardprefix[|\textit{main}|]{|\textit{prefix}|}{|\textit{dest}|}|
\end{tabular}
\end{center}
%
the destination file is determined by a pattern
depending on the current file:
To make this work, the current file must be called
`{\textit{prefix}\hspace{0.2em}\textit{suffix}}'
with \textit{prefix} matching precisely the argument.
Processing is then passed on to the file
`{\textit{dest}\hspace{0.2em}\textit{suffix}}'.
Surely, the same effect is achieved by
directly specifying the
argument `{\textit{dest}\hspace{0.2em}\textit{suffix}}'
in the first form.
However, that requires to set up a different file
for each child. With the alternative form of the command
all these files can have exactly the same content
which simplifies setting them up and maintaining them.

For example, the following file |draft.tex|
with a compilation flag |\version| as described in \secref{sec:flags}
compiles the main document as a draft:
%
\begin{center}
\begin{tabular}{l}
|\def\version{draft}|\\
|\input{childdoc.def}|\\
|\childdocforward{|\textit{main}|}|
\end{tabular}
\end{center}
%
Likewise, the following files |final|\textit{nn}|.tex|
compile the final version of the child document
|child|\textit{nn}|.tex|:
%
\begin{center}
\begin{tabular}{l}
|\def\version{final}|\\
|\input{childdoc.def}|\\
|\childdocforwardprefix{final}{child}|
\end{tabular}
\end{center}
%

Note that when several versions of a main file and/or of each child file
are to be generated, it may be convenient to set up a |Makefile| or
shell script to automatise the process.

%%%%%%%%%%%%%%%%%%%%%%%%%%%%%%%%%%%%%%%%%%%%%%%%%%%%%%%%%%%%%%%%%%%%%%%%%%%%%%%%
\subsection{Command Line Processing}
\label{sec:commandline}

The effect of redirection files can also be achieved by invoking
the \LaTeX{} compiler with a more elaborate command line.
Most conveniently this should be done as part
of a shell script or a |Makefile|.

When using \textsf{childdoc} in the main file, the following
command lines effectively perform a redirection
(note that depending on the shell being used,
backslashes may have to be doubled: `|\|' $\to$ `|\\|'):
%
\begin{center}
|... -jobname "|\textit{target}|" |\\|"|[\textit{flags}]%
|\input{childdoc.def}\childdocforward[|\textit{main}|]{|\textit{dest}|}"|
\end{center}
%
Here \textit{target} is the name of the output file,
\textit{main} is the name of the main file
and \textit{dest} is the name of the main or child file to be processed
(all filenames without extensions).
The optional argument \textit{main} can be omitted
if \textit{main} matches \textit{dest}.
Optionally, compilation \textit{flags} can be defined via |\def| commands.
This command line makes the \TeX{} engine believe
it is compiling the file \textit{target}
whose content is specified as the latter parameter.
The provided code then forwards the processing to
\textit{main} or \textit{dest} as described in \secref{sec:forward}.

%%%%%%%%%%%%%%%%%%%%%%%%%%%%%%%%%%%%%%%%%%%%%%%%%%%%%%%%%%%%%%%%%%%%%%%%%%%%%%%%
\subsection{Include by Input}
\label{sec:input}

Including child documents by |\include| has some restrictions by design.
Most notably, the content of a child document always occupies
its own set of pages; pages cannot be shared between child documents.
Usually, this behaviour makes perfect sense
because each child document contain an essential part of the document.
However, in some situations it may be desirable to compose
a document from a collection of parts
without having mandatory page breaks between then.
For this case, the package
provides a mechanism to include parts
by |\input| which can also be processed individually.
However, by construction this mechanism
requires manual handling of the content to be output.

%%%%%%%%%%%%%%%%%%%%%%%%%%%%%%%%%%%%%%%%
\DescribeMacro{\ifchilddocmanual}
The main file should be prepared as usual, see \secref{sec:include}.
However, the document body must make a distinction
between processing of an individual part and of the main document, e.g.:
%
\begin{center}
\begin{tabular}{l}
|\ifchilddocmanual|\\
|\input{\childdocname}|\\
|\||else|\\
\textit{document body with }|\input{|\textit{part}|}|\\
|\||fi|
\end{tabular}
\end{center}
%
The conditional |\ifchilddocmanual| is true whenever
a part to be included by |\input| is being compiled,
and the name of the part is stored in |\childdocname|.

%%%%%%%%%%%%%%%%%%%%%%%%%%%%%%%%%%%%%%%%
\DescribeMacro{\childdocby}
Each part to be included by |\input| should start with:
%
\begin{center}
\begin{tabular}{l}
|\input{childdoc.def}|\\
|\childdocby{|\textit{main}|}|\\
\end{tabular}
\end{center}
%
The directive |\childdocby| is similar to |\childdocof|
described in \secref{sec:include},
but the subsequent selection of content must be done manually.
To that end, both |\ifchilddoc| and |\ifchilddocmanual|
will be true upon processing of a part,
and the name of the part is stored in |\childdocname|.
Note that |\jobname| will be set to the filename of the current part
so that each part receives an individual |.aux| file
that does not interfere with the |.aux| file(s) of the main document.
This behaviour can be altered by the alternative form
|\childdocby[*]{|\textit{main}|}| (with a non-empty optional argument)
which uses the |.aux| file of the main document
by setting |\jobname| to \textit{main}.

%%%%%%%%%%%%%%%%%%%%%%%%%%%%%%%%%%%%%%%%%%%%%%%%%%%%%%%%%%%%%%%%%%%%%%%%%%%%%%%%
\subsection{Driver Development}
\label{sec:driver}

The \textsf{childdoc} mechanism can also be use for the development
of definition files such as \LaTeX{} styles or classes.
This case differs from the above setup with multiple parts
included by |\include| in that no |\includeonly| should be invoked.
This can be achieved by starting the include file
(before |\ProvidesPackage|) with:
%
\begin{center}
\begin{tabular}{l}
|\input{childdoc.def}|\\
|\childdocforward{|\textit{main}|}|\\
\end{tabular}
\end{center}
%
or alternatively with:
%
\begin{center}
\begin{tabular}{l}
|\input{childdoc.def}|\\
|\childdocby{|\textit{main}|}|\\
\end{tabular}
\end{center}
%
Both forms have slightly different effects as described above.
The main file is prepared as usual, see \secref{sec:include}.

%%%%%%%%%%%%%%%%%%%%%%%%%%%%%%%%%%%%%%%%%%%%%%%%%%%%%%%%%%%%%%%%%%%%%%%%%%%%%%%%
\subsection{Legacy Detection}
\label{sec:detection}

The directive |\childdocmain| in the main file can detect
whether the complete document or merely a child is to be compiled
even without using the directive |\childdocof|.
This method is deprecated because it is less robust
and there is no compelling reason to use it;
it is merely provided for backward compatibility
and it may be removed in future versions.

If the detection mechanism is to be used,
it is mandatory to correctly specify
the filename of the main file as the argument of |\childdocmain|:
%
\begin{center}
\begin{tabular}{l}
|\input{childdoc.def}|\\
|\childdocmain{|\textit{main}|}|\\
\end{tabular}
\end{center}
%
If |\jobname| does not match the argument \textit{main} of |\childdocmain|,
it is assumed that |\jobname| points to the child file to be compiled.
When using |\childdocmain| with the main file specified as argument,
it suffices to start a child file
with just |\input{|\textit{main}|}|
without loading of the package and using |\childdocof|.
If instead all processing is done
with the appropriate \textsf{childdoc} directives,
the argument of \textit{main} of |\childdocmain| can be empty.

An alternative version of the command line processing described
in \secref{sec:commandline} using the detection mechanism reads:
%
\begin{center}
|... -jobname "|\textit{target}|" "|[\textit{flags}]%
[|\def\jobname{|\textit{dest}|}|]|\input{|\textit{main}|}"|
\end{center}

%%%%%%%%%%%%%%%%%%%%%%%%%%%%%%%%%%%%%%%%%%%%%%%%%%%%%%%%%%%%%%%%%%%%%%%%%%%%%%%%
\subsection{Manual Code}
\label{sec:manual}

In case one cannot be certain whether the definitions file |childdoc.def|
is installed on the target \TeX{} distribution
and one prefers not to ship it,
it is conceivable to paste a few relevant commands into the sources.

To that end, drop all statements |\input{childdoc.def}|
and perform the replacements as outlined below.
Instead of |\childdocmain{|\textit{main}|}| add the following code
to the top of the main file:
%
\begin{center}
\begin{tabular}{l}
|\||ifdefined\childdocname\endinput\||fi\newif\ifchilddoc|\\
|\edef\childdocname{\scantokens\expandafter{\jobname\noexpand}}|\\
|\def\childdocmain{|\textit{main}|}\||ifx\childdocmain\childdocname\||else|\\
|\childdoctrue\includeonly{\childdocname}\let\jobname\childdocmain\||fi|\\
\end{tabular}
\end{center}
%
Instead of |\childdocof{|\textit{main}|}| just include the main file
at the top of each child file:
%
\begin{center}
|\input{|\textit{main}|}|
\end{center}
%
A simple redirection |\childdocforward{|\textit{dest}|}| is achieved by:
%
\begin{center}
|\def\jobname{|\textit{dest}|}\input{\jobname}|
\end{center}
%
The redirection with prefix
|\childdocforwardprefix[|\textit{prefix}|]{|\textit{dest}|}|
is accomplished by:
%
\begin{center}
\begin{tabular}{l}
|{\edef\jobname{\scantokens\expandafter{\jobname\noexpand}}|\\
|\def\redirectjob |\textit{prefix}|#1~~~{\gdef\jobname{|\textit{dest}|#1}}|\\
|\expandafter\redirectjob\jobname~~~}\input{\jobname}|
\end{tabular}
\end{center}

In an alternative approach,
child documents can be compiled by a specific command line
without additional code or specific definitions:
%
\begin{center}
|... -jobname "|\textit{target}|" "|[\textit{flags}]%
|\includeonly{|\textit{dest}|}\input{|\textit{main}|}"|
\end{center}
%

%%%%%%%%%%%%%%%%%%%%%%%%%%%%%%%%%%%%%%%%%%%%%%%%%%%%%%%%%%%%%%%%%%%%%%%%%%%%%%%%
%%%%%%%%%%%%%%%%%%%%%%%%%%%%%%%%%%%%%%%%%%%%%%%%%%%%%%%%%%%%%%%%%%%%%%%%%%%%%%%%
\section{Information}

%%%%%%%%%%%%%%%%%%%%%%%%%%%%%%%%%%%%%%%%%%%%%%%%%%%%%%%%%%%%%%%%%%%%%%%%%%%%%%%%
\subsection{Copyright}

Copyright \copyright{} 2017--2018 Niklas Beisert

This work may be distributed and/or modified under the
conditions of the \LaTeX{} Project Public License, either version 1.3
of this license or (at your option) any later version.
The latest version of this license is in
  \url{http://www.latex-project.org/lppl.txt}
and version 1.3 or later is part of all distributions of \LaTeX{}
version 2005/12/01 or later.

This work has the LPPL maintenance status `maintained'.

The Current Maintainer of this work is Niklas Beisert.

This work consists of the files |README.txt|, |childdoc.ins| and |childdoc.dtx|
as well as the derived files |childdoc.def|, |cdocsamp.tex|
with |cdocsch1.tex|, |cdocsch2.tex|, |cdocspt3.tex|, |cdocspt4.tex|,
|cdocsdrf.tex|, |cdocsfn1.tex|, |cdocsfn2.tex|
as well as |childdoc.pdf|.

%%%%%%%%%%%%%%%%%%%%%%%%%%%%%%%%%%%%%%%%%%%%%%%%%%%%%%%%%%%%%%%%%%%%%%%%%%%%%%%%
\subsection{Files and Installation}

The package consists of the files:
%
\begin{center}
\begin{tabular}{ll}
    |README.txt|   & readme file \\
    |childdoc.ins| & installation file \\
    |childdoc.dtx| & source file \\
    |childdoc.def| & definition file \\
    |cdocsamp.tex| & sample main file \\
    |cdocsch1.tex| & sample include file \\
    |cdocsch2.tex| & sample include file \\
    |cdocspt3.tex| & sample part file \\
    |cdocspt4.tex| & sample part file \\
    |cdocsdrf.tex| & sample redirection file \\
    |cdocsfn1.tex| & sample redirection file \\
    |cdocsfn2.tex| & sample redirection file \\
    |childdoc.pdf| & manual
\end{tabular}
\end{center}
%
The distribution consists of the files
|README.txt|, |childdoc.ins| and |childdoc.dtx|.
%
\begin{itemize}
\item
Run (pdf)\LaTeX{} on |childdoc.dtx|
to compile the manual |childdoc.pdf| (this file).
\item
Run \LaTeX{} on |childdoc.ins| to create the definitions file |childdoc.def|
and the sample |cdocsamp.tex| with include files
|cdocsch1.tex|, |cdocsch2.tex|, |cdocspt3.tex|, |cdocspt4.tex|,
|cdocsdrf.tex|, |cdocsfn1.tex|, |cdocsfn2.tex|.
Then copy the file |childdoc.def| to an appropriate directory of your \LaTeX{}
distribution, e.g.\ \textit{texmf-root}|/tex/latex/childdoc|.
\end{itemize}

%%%%%%%%%%%%%%%%%%%%%%%%%%%%%%%%%%%%%%%%%%%%%%%%%%%%%%%%%%%%%%%%%%%%%%%%%%%%%%%%
\subsection{Related CTAN Packages}

There are several other packages which offer a similar functionality:
%
\begin{itemize}
\item
The packages
\href{http://ctan.org/pkg/docmute}{\textsf{docmute}},
\href{http://ctan.org/pkg/includex}{\textsf{includex}} and
\href{http://ctan.org/pkg/standalone}{\textsf{standalone}}
provide commands to include only the document body of
a child file thus allowing both files to be compiled individually.
\item
The packages \href{http://ctan.org/pkg/subdocs}{\textsf{subdocs}}
and \href{http://ctan.org/pkg/subfiles}{\textsf{subfiles}}
provide structures in which the main and child documents can be
encapsulated and allowing them to be compiled individually.
The inclusion mechanism is different from the conventional |\include|.
\item
The package \href{http://ctan.org/pkg/combine}{\textsf{combine}}
is an elaborate solution to combine several documents into one.
\end{itemize}
%
See also the CTAN topic \href{http://ctan.org/topic/subdocs}{\textsf{subdocs}}
for further related packages.
The present package differs from the above solutions in that
a document structure constructed with the conventional |\include| mechanism
just needs two extra commands at the top of every file
such that all constituent files can be compiled individually.

%%%%%%%%%%%%%%%%%%%%%%%%%%%%%%%%%%%%%%%%%%%%%%%%%%%%%%%%%%%%%%%%%%%%%%%%%%%%%%%%
%\subsection{Feature Suggestions}
%
%The following is a list of features which may be useful for future
%versions of this package:
%%
%\begin{itemize}
%\item
%\ldots
%\end{itemize}

%%%%%%%%%%%%%%%%%%%%%%%%%%%%%%%%%%%%%%%%%%%%%%%%%%%%%%%%%%%%%%%%%%%%%%%%%%%%%%%%
\subsection{Revision History}

%%%%%%%%%%%%%%%%%%%%%%%%%%%%%%%%%%%%%%%%
\paragraph{v2.0:} 2018/12/30

\begin{itemize}
\item
immediate forward processing
\item
added |\childdocby| mechanism
\item
manual restructured
\end{itemize}

%%%%%%%%%%%%%%%%%%%%%%%%%%%%%%%%%%%%%%%%
\paragraph{v1.6:} 2018/01/17

\begin{itemize}
\item
application for development of include files
\item
corrections to manual
\end{itemize}

%%%%%%%%%%%%%%%%%%%%%%%%%%%%%%%%%%%%%%%%
\paragraph{v1.5:} 2017/05/21

\begin{itemize}
\item
more complete structuring introduced
\item
|\childdocof| introduced
\item
|\childdoc| renamed to |\childdocmain|
\item
|\childredirect| renamed to |\childdocforward| and |\childdocforwardprefix|
and functionality expanded
\end{itemize}

%%%%%%%%%%%%%%%%%%%%%%%%%%%%%%%%%%%%%%%%
\paragraph{v1.0:} 2017/04/27

\begin{itemize}
\item
manual and install package
\item
first version published on CTAN
\end{itemize}

%%%%%%%%%%%%%%%%%%%%%%%%%%%%%%%%%%%%%%%%
\paragraph{v0.6:} 2017/04/26

\begin{itemize}
\item
redirection mechanism added
\end{itemize}

%%%%%%%%%%%%%%%%%%%%%%%%%%%%%%%%%%%%%%%%
\paragraph{v0.5:} 2017/04/26

\begin{itemize}
\item
functionality in definition file
\end{itemize}


%%%%%%%%%%%%%%%%%%%%%%%%%%%%%%%%%%%%%%%%%%%%%%%%%%%%%%%%%%%%%%%%%%%%%%%%%%%%%%%%
%%%%%%%%%%%%%%%%%%%%%%%%%%%%%%%%%%%%%%%%%%%%%%%%%%%%%%%%%%%%%%%%%%%%%%%%%%%%%%%%
%%%%%%%%%%%%%%%%%%%%%%%%%%%%%%%%%%%%%%%%%%%%%%%%%%%%%%%%%%%%%%%%%%%%%%%%%%%%%%%%
\appendix

\settowidth\MacroIndent{\rmfamily\scriptsize 000\ }

 \DocInput{childdoc.dtx}

\end{document}
%</driver>
% \fi
%
% %%%%%%%%%%%%%%%%%%%%%%%%%%%%%%%%%%%%%%%%%%%%%%%%%%%%%%%%%%%%%%%%%%%%%%%%%%%%%%
% %%%%%%%%%%%%%%%%%%%%%%%%%%%%%%%%%%%%%%%%%%%%%%%%%%%%%%%%%%%%%%%%%%%%%%%%%%%%%%
% \section{Sample}
%\iffalse
%<*samplemain>
%\fi
%
% The following presents a sample document
% with two chapters, two parts, a title page,
% a compile flag as well as three forwarding files to set the flag.
% It consists of eight |.tex| files:
% \begin{center}
% \begin{tabular}{ll}
% |cdocsamp.tex|&main file\\
% |cdocsch1.tex|&include file for chapter 1\\
% |cdocsch2.tex|&include file for chapter 2\\
% |cdocspt3.tex|&include file for part 3\\
% |cdocspt4.tex|&include file for part 4\\
% |cdocsdrf.tex|&forwarding file for main file in draft mode\\
% |cdocsfi1.tex|&forwarding file for final version of chapter 1\\
% |cdocsfi2.tex|&forwarding file for final version of chapter 2\\
% \end{tabular}
% \end{center}
% Each of the eight files can be compiled directly by the \LaTeX{} compiler.
%
% %%%%%%%%%%%%%%%%%%%%%%%%%%%%%%%%%%%%%%
% \paragraph{Main File.}
%
% The main file is called |cdocsamp.tex|.
%
% Load the \textsf{childdoc} definitions and
% declare the filename for the main document:
%    \begin{macrocode}
\input{childdoc.def}
\childdocmain{}
%    \end{macrocode}

% Optional override for |\version| flag:
%    \begin{macrocode}
%%\ifchilddoc\else\providecommand{\version}{draft}\fi
%    \end{macrocode}

% Define the default values for the |\version| flag
% (|final| for the main file and |draft| for childs):
%    \begin{macrocode}
\ifchilddoc
\providecommand{\version}{draft}
\else
\providecommand{\version}{final}
\fi
%    \end{macrocode}

% Load the standard document class:
%    \begin{macrocode}
\documentclass[12pt]{article}
%    \end{macrocode}

% Start the document body:
%    \begin{macrocode}
\begin{document}
%    \end{macrocode}

% Declare a title page.
% Print title, part of document being processed and version flag:
%    \begin{macrocode}
\addtocounter{page}{-1}
\begin{center}
{\LARGE\bfseries{}childdoc example\par}
\vspace{1cm}
\ifchilddoc
\ifchilddocmanual part\else chapter\fi:
`\childdocname' of `\childdocjob'\par
\else
main document: `\childdocjob'\par
\fi
version: \version\par
\end{center}
\newpage
%    \end{macrocode}

% Manually include selected file,
% otherwise process as usual:
%    \begin{macrocode}
\ifchilddocmanual
\section*{part `\childdocname'}
\input{\childdocname}
\else
%    \end{macrocode}

% Include the two chapters:
%    \begin{macrocode}
\include{cdocsch1}
\include{cdocsch2}
%    \end{macrocode}

% Include the two parts unless only chapters should be displayed:
%    \begin{macrocode}
\ifchilddoc\else
\section{part three}
\input{cdocspt3}
\section{part four}
\input{cdocspt4}
\fi
%    \end{macrocode}

% Process as usual until here:
%    \begin{macrocode}
\fi
%    \end{macrocode}

% End of document body:
%    \begin{macrocode}
\end{document}
%    \end{macrocode}
%\iffalse
%</samplemain>
%\fi
%
% %%%%%%%%%%%%%%%%%%%%%%%%%%%%%%%%%%%%%%
% \paragraph{Chapter Include Files.}
%
% The include files are called |cdocsch1.tex| and |cdocsch2.tex|.
%
%\iffalse
%<*samplechap1|samplechap2>
%\fi

% Optional override for |\version| flag:
%    \begin{macrocode}
%%\providecommand{\version}{final}
%    \end{macrocode}

% Include the main document:
%    \begin{macrocode}
\input{childdoc.def}
\childdocof{cdocsamp}
%    \end{macrocode}

%\iffalse
%</samplechap1|samplechap2>
%\fi
%
%\iffalse
%<*samplechap1>
%\fi
% Some text for chapter 1:
%    \begin{macrocode}
\section{one}
some text in chapter one
%    \end{macrocode}

%\iffalse
%</samplechap1>
%\fi
% Some text for chapter 2:
%\iffalse
%<*samplechap2>
%\fi
%    \begin{macrocode}
\section{two}
more text in chapter two
%    \end{macrocode}

%\iffalse
%</samplechap2>
%\fi
%
% %%%%%%%%%%%%%%%%%%%%%%%%%%%%%%%%%%%%%%
% \paragraph{Part Include Files.}
%
% The include files are called |cdocspt3.tex| and |cdocspt4.tex|.
%
%\iffalse
%<*samplepart3|samplepart4>
%\fi

% Optional override for |\version| flag:
%    \begin{macrocode}
%%\providecommand{\version}{final}
%    \end{macrocode}

% Include the main document:
%    \begin{macrocode}
\input{childdoc.def}
\childdocby{cdocsamp}
%    \end{macrocode}

%\iffalse
%</samplepart3|samplepart4>
%\fi
%
%\iffalse
%<*samplepart3>
%\fi
% Some text for part 3:
%    \begin{macrocode}
some text in part three
%    \end{macrocode}

%\iffalse
%</samplepart3>
%\fi
% Some text for part 4:
%\iffalse
%<*samplepart4>
%\fi
%    \begin{macrocode}
more text in part four
%    \end{macrocode}

%\iffalse
%</samplepart4>
%\fi
%
% %%%%%%%%%%%%%%%%%%%%%%%%%%%%%%%%%%%%%%
% \paragraph{Forwarding for a Complete Draft.}
%
% The following forwarding file |cdocsdrf.tex|
% compiles the main document in draft mode:
%\iffalse
%<*sampledraft>
%\fi
%    \begin{macrocode}
\def\version{draft}
\input{childdoc.def}
\childdocforward{cdocsamp}
%    \end{macrocode}

%\iffalse
%</sampledraft>
%\fi
%
% %%%%%%%%%%%%%%%%%%%%%%%%%%%%%%%%%%%%%%
% \paragraph{Forwarding for Final Version of the Chapters.}
%
% The following forwarding files |cdocsfn1.tex| and |cdocsfn2.tex|
% (with identical content)
% compile the final versions of the child documents
% |cdocsch1.tex| and |cdocsch2.tex|, respectively:
%\iffalse
%<*samplefinal>
%\fi
%    \begin{macrocode}
\def\version{final}
\input{childdoc.def}
\childdocforwardprefix[cdocsamp]{cdocsfn}{cdocsch}
%    \end{macrocode}

%\iffalse
%</samplefinal>
%\fi
%
% %%%%%%%%%%%%%%%%%%%%%%%%%%%%%%%%%%%%%%
% \paragraph{Command Line Processing.}
%
% The following three command lines generate the output files
% |cdocscld|, |cdocscl1| and |cdocscl2|
% which should be identical to
% |cdocsdrf|, |cdocsch1| and |cdocsfn2|, respectively:
% \begin{center}
% \begin{tabular}{l}
% |latex -jobname cdocscld \|\\
% |  "\def\version{draft}\input{childdoc.def}\childdocforward{cdocsamp}"|\\
% |latex -jobname cdocscl1 \|\\
% |  "\input{childdoc.def}\childdocforward[cdocsamp]{cdocsch1}"|\\
% |latex -jobname cdocscl2 \|\\
% |  "\def\version{final}\input{childdoc.def}\childdocforward{cdocsch2}"|
% \end{tabular}
% \end{center}
% Note that the trailing backslash on each first line
% merely continues the input to the second line
% (for convenient cut ant paste).
% Furthermore, the command |latex| can be replaced by any
% of its alternative versions such as |pdflatex|.
%
% %%%%%%%%%%%%%%%%%%%%%%%%%%%%%%%%%%%%%%%%%%%%%%%%%%%%%%%%%%%%%%%%%%%%%%%%%%%%%%
% %%%%%%%%%%%%%%%%%%%%%%%%%%%%%%%%%%%%%%%%%%%%%%%%%%%%%%%%%%%%%%%%%%%%%%%%%%%%%%
% \section{Implementation}
%\iffalse
%<*package>
%\fi
%
% This section describes the definitions file |childdoc.def|.

% The definitions cannot be loaded using |\usepackage| or |\RequirePackage|
% which has a mechanism to prevent loading a style file more than once.
% When loading the definitions by means of |\input|
% multiple instances have to be prevented manually:
%\iffalse
%This code needs to be before the `\ProvidesFile' directive
%which is defined at the beginning of this file.
%Therefore it is also placed there and commented out here.
%</package>
%<*discard>
%\fi
%    \begin{macrocode}
\ifdefined\childdocmain\endinput\fi
%    \end{macrocode}
%\iffalse
%</discard>
%<*package>
%\fi
%
% \macro{\ifchilddoc}
% \macro{\ifchilddocmanual}
% The conditional |\ifchilddoc| tells whether a
% child (true) or main (false) document is being compiled.
% The conditional |\ifchilddocmanual| tells whether
% the |\includeonly| mechanism is used (false) or
% the selection of child files must be performed manually (true).
% The definitions initialise to false:
%    \begin{macrocode}
\newif\ifchilddoc
\newif\ifchilddocmanual
%    \end{macrocode}

% \macro{\childdocname}
% \macro{\childdocjob}
% The macro |\childdocname| stores the name of the main document
% to be compiled. The macro |\childdocjob| stores the name of
% the document on which the \LaTeX{} compiler was originally invoked.
% The content of |\jobname| cannot be compared
% to filenames specified in the source due to different catcodes.
% The following code rescans |\jobname|, stores the result
% in |\childdocname| and saves a copy in |\childdocjob|:
%    \begin{macrocode}
\edef\childdocname{\scantokens\expandafter{\jobname\noexpand}}
\let\childdocjob\childdocname
%    \end{macrocode}

% \macro{\childdocdisable}
% The macro |\childdocdisable| prevents the main file
% from being processed more than once.
% At this stage, the main document command |\childdocmain|
% is assumed to be called once again where it should do nothing.
% Any subsequent call to it should prevent
% a secondary processing of the main document
% It overwrites the forwarding commands
% |\childdocof| and |\childdocforward|
% with empty macros to prevent further inclusions of the main document:
%    \begin{macrocode}
\newcommand{\childdocdisable}
{
  \renewcommand{\childdocmain}[1]{\renewcommand{\childdocmain}[1]{\endinput}}
  \renewcommand{\childdocof}[1]{}
  \renewcommand{\childdocby}[2][]{}
  \renewcommand{\childdocforward}[2][]{}
  \renewcommand{\childdocdisable}{}
}
%    \end{macrocode}

% \macro{\childdocmain}
% The macro |\childdocmain| is to be called at the top of the main file
% with nothing or the main filename (without extension) as argument.
% First, it breaks loops.
% If the argument is not empty and does not match |\childdocname|
% (which is set by the first inclusion of |childdoc.def|),
% |\ifchilddoc| is set to true, |\includeonly| is applied to the child file
% and |\jobname| is set to the main file
% (for proper handling of |.aux| files):
%    \begin{macrocode}
\newcommand{\childdocmain}[1]
{
  \childdocdisable\childdocmain{}
  \if?#1?\else
    \begingroup
      \def\childdoctmp{#1}
      \ifx\childdoctmp\childdocname
        \def\childdoctmp{}
      \else
        \def\childdoctmp
        {
          \childdoctrue
          \includeonly{\childdocname}
          \def\childdocjob{#1}
          \def\jobname{#1}
        }
      \fi
      \expandafter
    \endgroup
    \childdoctmp
  \fi
}
%    \end{macrocode}

% \macro{\childdocof}
% The command |\childdocof| redirects
% compilation to the main file |#1|.
%    \begin{macrocode}
\newcommand{\childdocof}[1]
{
  \childdocdisable
  \childdoctrue
  \includeonly{\childdocname}
  \def\jobname{#1}
  \def\childdocjob{#1}
  \input{#1}
}
%    \end{macrocode}

% \macro{\childdocby}
% The command |\childdocby| ....
%    \begin{macrocode}
\newcommand{\childdocby}[2][]
{
  \childdocdisable
  \childdoctrue
  \childdocmanualtrue
  \if?#1?\else
    \def\jobname{#2}
  \fi
  \def\childdocjob{#2}
  \input{#2}
  \endinput
}
%    \end{macrocode}

% \macro{\childdocforward}
% The command |\childdocforward| redirects
% compilation to the main file or
% (if the optional argument is given) a child file.
% Parameters are set as if the main file
% or a child file starting with |\childdocof| was compiled.
% Then compilation is handed over to the main file:
%    \begin{macrocode}
\newcommand{\childdocforward}[2][]
{
  \begingroup
    \if?#1?
      \def\childdoctmp
      {
        \def\childdocname{#2}
        \def\childdocjob{#2}
        \def\jobname{#2}
        \input{#2}
        \endinput
      }
    \else
      \def\childdoctmp
      {
        \childdocdisable
        \def\childdocname{#2}
        \childdoctrue
        \includeonly{#2}
        \def\childdocjob{#1}
        \def\jobname{#1}
        \input{#1}
        \endinput
      }
    \fi
    \expandafter
  \endgroup
  \childdoctmp
}
%    \end{macrocode}

% \macro{\childdocforwardprefix}
% The command |\childdocforwardprefix| redirects
% compilation to the main or a child file by means of a pattern.
% The prefix |#1| in the current filename is replaced by |#2|
% and the suffix of the current filename is kept
% (it is assumed that the filename does not contain the substring `|~~~|'
% which is used as a delimiter).
% Compilation is handed over to the new file by |\childdocforward|:
%    \begin{macrocode}
\newcommand{\childdocforwardprefix}[3][]
{
  \begingroup
    \def\childdocextract #2##1~~~{\def\childdoctmp{\childdocforward[#1]{#3##1}}}
    \expandafter\childdocextract\childdocname~~~
    \expandafter
  \endgroup
  \childdoctmp
}
%    \end{macrocode}

% \macro{\childdoc}
% The deprecated macro |\childdoc| is a legacy version of |\childdocmain|:
%    \begin{macrocode}
\newcommand{\childdoc}{\childdocmain}
%    \end{macrocode}

% \macro{\childdocredirect}
% The deprecated macro |\childdocredirect| is a legacy version
% of |\childdocforward| and |\childdocforwardprefix|:
%    \begin{macrocode}
\newcommand{\childdocredirect}[2][]
{
  \begingroup
    \if?#1?
      \def\childdoctmp{\childdocforward{#2}}
    \else
      \def\childdoctmp{\childdocforwardprefix{#1}{#2}}
    \fi
    \expandafter
  \endgroup
  \childdoctmp
}
%    \end{macrocode}

%\iffalse
%</package>
%\fi
%
\endinput
|\\
|\childdocforward[|\textit{main}|]{|\textit{dest}|}|\\
\end{tabular}
\end{center}
%
The argument \textit{dest} is the destination file
(without extension).
It should be the main file or one of the child files.
Note that further \textsf{childdoc} directives
such as |\childdocof| and |\childdocforward|
in the indicated file will be processed in this form.
The optional argument \textit{main}
passes on directly to the main file \textit{main}
while pretending to compile the child \textit{dest}.
This form behaves as if \textit{dest}
issues |\childdocof{|\textit{main}|}| right away,
and no further \textsf{childdoc} directives will be processed.

%%%%%%%%%%%%%%%%%%%%%%%%%%%%%%%%%%%%%%%%
\DescribeMacro{\...prefix}
In the alternative form |\childdocforwardprefix|,
%
\begin{center}
\begin{tabular}{l}
|% \iffalse
%
% childdoc.dtx Copyright (C) 2017-2018 Niklas Beisert
%
% This work may be distributed and/or modified under the
% conditions of the LaTeX Project Public License, either version 1.3
% of this license or (at your option) any later version.
% The latest version of this license is in
%   http://www.latex-project.org/lppl.txt
% and version 1.3 or later is part of all distributions of LaTeX
% version 2005/12/01 or later.
%
% This work has the LPPL maintenance status `maintained'.
%
% The Current Maintainer of this work is Niklas Beisert.
%
% This work consists of the files childdoc.dtx and childdoc.ins
% and the derived files childdoc.def and cdocsamp.tex with
% cdocsch1.tex, cdocsch2.tex, cdocsdrf.tex, cdocsfn1.tex, cdocsfn2.tex.
%
%<package>\ifdefined\childdocmain\endinput\fi
%<package>\ProvidesFile{childdoc.def}[2018/12/30 v2.0 child document driver]
%<samplemain>\ProvidesFile{cdocsamp.tex}[2018/12/30 v2.0 sample for childdoc]
%<*driver>
%\ProvidesFile{childdoc.drv}[2018/12/30 v2.0 childdoc reference manual file]
\PassOptionsToClass{10pt,a4paper}{article}
\documentclass{ltxdoc}

\usepackage[margin=35mm]{geometry}
\usepackage{hyperref}
\usepackage{hyperxmp}
\usepackage[usenames]{color}

\hypersetup{colorlinks=true}
\hypersetup{pdfstartview=FitH}
\hypersetup{pdfpagemode=UseNone}
\hypersetup{pdfsource={}}
\hypersetup{pdflang={en-UK}}
\hypersetup{pdfcopyright={Copyright 2017-2018 Niklas Beisert.
  This work may be distributed and/or modified under the
  conditions of the LaTeX Project Public License, either version 1.3
  of this license or (at your option) any later version.}}
\hypersetup{pdflicenseurl={http://www.latex-project.org/lppl.txt}}
\hypersetup{pdfcontactaddress={ETH Zurich, ITP, HIT K,
  Wolfgang-Pauli-Strasse 27}}
\hypersetup{pdfcontactpostcode={8093}}
\hypersetup{pdfcontactcity={Zurich}}
\hypersetup{pdfcontactcountry={Switzerland}}
\hypersetup{pdfcontactemail={nbeisert@itp.phys.ethz.ch}}
\hypersetup{pdfcontacturl={http://people.phys.ethz.ch/\xmptilde nbeisert/}}

\newcommand{\secref}[1]{\hyperref[#1]{section \ref*{#1}}}

\parskip1ex
\parindent0pt
\let\olditemize\itemize
\def\itemize{\olditemize\parskip0pt}

\begin{document}

\title{The \textsf{childdoc} Package}
\hypersetup{pdftitle={The childdoc Package}}
\author{Niklas Beisert\\[2ex]
  Institut f\"ur Theoretische Physik\\
  Eidgen\"ossische Technische Hochschule Z\"urich\\
  Wolfgang-Pauli-Strasse 27, 8093 Z\"urich, Switzerland\\[1ex]
  \href{mailto:nbeisert@itp.phys.ethz.ch}
  {\texttt{nbeisert@itp.phys.ethz.ch}}}
\hypersetup{pdfauthor={Niklas Beisert}}
\hypersetup{pdfsubject={Manual for the LaTeX2e Package childdoc}}
\date{30 December 2018, \textsf{v2.0}}
\maketitle

\begin{abstract}\noindent
\textsf{childdoc} is a \LaTeXe{} package
that enables the direct compilation
of document sections included by |\include|
to individual files.
\end{abstract}

\begingroup
\parskip0ex
\tableofcontents
\endgroup

%%%%%%%%%%%%%%%%%%%%%%%%%%%%%%%%%%%%%%%%%%%%%%%%%%%%%%%%%%%%%%%%%%%%%%%%%%%%%%%%
%%%%%%%%%%%%%%%%%%%%%%%%%%%%%%%%%%%%%%%%%%%%%%%%%%%%%%%%%%%%%%%%%%%%%%%%%%%%%%%%
\section{Introduction}

\LaTeX{} provides a mechanism to structure a large document (such as a book)
into a main file and several child files (containing the chapters)
using the |\include| command.
This mechanism is beneficial for documents
which span hundreds of pages in order to
make the source file(s) more manageable.
Moreover, compilation can be restricted to
selected child files by means of the |\includeonly| command.
The latter feature can be used to reduce the compilation time while editing
(this was significantly more useful in the earlier days of \LaTeX{})
or to generate a smaller document which is easier to navigate.
Another application of |\includeonly| is to generate
documents consisting of selected parts of the complete document.

However, there are a few drawbacks of the plain |\include| mechanism:
\begin{itemize}
\item
The child files cannot be compiled on their own,
they can only be compiled via the main file.
A naive editing environment
(such as a text editor with an option
to have the current file processed by \LaTeX)
may require one to switch to the main file before compiling;
attempting to compile the child file produces errors.
\item
The main file must be modified (each time)
to adjust the |\includeonly| command
to the present needs. This easily leaves the main file in a messy state.
\item
The generated document will always carry the filename
of the main document. This is inconvenient if
several child files are to be compiled and
to be kept for distribution.
\end{itemize}

The present package provides a simple interface
to make child files individually compilable by \LaTeX{}.
Compiling a child file then has the same effect as compiling
the main file with an |\includeonly| command
to select the appropriate child.
Moreover the generated document will carry the name of the child
rather than the main file.
This resolves all three above issues.

This feature is meant to make the editing of books,
thesis documents and lecture notes somewhat more convenient.
However, the package can also be used efficiently for
composing a series of documents (such as exercise sheets)
which are typically distributed individually.
It then assists the author in generating the individual documents
(potentially in different versions)
as well as a document containing the collected series.
Another application is in developing style files
or other kinds of included material
where compilation of the style file could redirect
to a sample or test file.

%%%%%%%%%%%%%%%%%%%%%%%%%%%%%%%%%%%%%%%%%%%%%%%%%%%%%%%%%%%%%%%%%%%%%%%%%%%%%%%%
%%%%%%%%%%%%%%%%%%%%%%%%%%%%%%%%%%%%%%%%%%%%%%%%%%%%%%%%%%%%%%%%%%%%%%%%%%%%%%%%
\section{Usage}

First of all, the package \textsf{childdoc} is \emph{not} a standard
\LaTeXe{} |.sty| style file! Therefore it needs to be invoked in
a non-standard way.

%%%%%%%%%%%%%%%%%%%%%%%%%%%%%%%%%%%%%%%%%%%%%%%%%%%%%%%%%%%%%%%%%%%%%%%%%%%%%%%%
\subsection{Included Files}
\label{sec:include}

%%%%%%%%%%%%%%%%%%%%%%%%%%%%%%%%%%%%%%%%
\DescribeMacro{\childdocmain}
To use the package, add the commands
\begin{center}
\begin{tabular}{l}
|\input{childdoc.def}|\\
|\childdocmain{}|\\
\end{tabular}
\end{center}
at the very top of the main \LaTeX{} file,
in particular \emph{before} the |\documentclass| statement!
The argument of |\childdocmain| should be left empty
(but it must be present).

%%%%%%%%%%%%%%%%%%%%%%%%%%%%%%%%%%%%%%%%
\DescribeMacro{\childdocof}
Furthermore, add the commands
\begin{center}
\begin{tabular}{l}
|\input{childdoc.def}|\\
|\childdocof{|\textit{main}|}|\\
\end{tabular}
\end{center}
at the top of every child file \textit{child}
which is included by |\include{|\textit{child}|}|
from within the main file
(or at least for those files to be compiled individually).
The argument \textit{main} must be the filename of the main file.

There are a couple of
considerations in setting up the main and child documents:

%%%%%%%%%%%%%%%%%%%%%%%%%%%%%%%%%%%%%%%%
\paragraph{Restrictions.}

Please note the following restrictions:
\begin{itemize}
\item
|\childdocmain| must be called with one argument \textit{main}
to ensure compatibility with earlier version of the package.
It must either be empty (|\childdocmain{}|)
or precisely match the filename of the main file in which it is specified.
See \secref{sec:detection} for further information.
\item
The filename \textit{main} must be specified without the |.tex| extension.
\item
The filename \textit{main} is case sensitive
(even in case-insensitive file systems)
due to internal string comparison.
\item
The argument \textit{main} should be fully expanded, it cannot be a macro.
\item
Subdirectories and special characters should be avoided in filenames.
\item
The command |\childdocmain{|\textit{main}|}| must be followed by a whitespace.
It should not be followed immediately by another command
or by a comment mark `|%|'.
This is because the \TeX{} parser reads the token immediately following
the argument of |\childdocmain| and puts it
at the beginning of every child section;
however, a white\-space is ignored.
\end{itemize}

%%%%%%%%%%%%%%%%%%%%%%%%%%%%%%%%%%%%%%%%
\paragraph{Content of Main File.}

It is advisable to place all content in the child files included by |\include|.
Any output contained in the main file will appear in all child documents
unless suppressed manually;
it cannot be suppressed automatically by the |\includeonly| directive
and thus should normally be avoided.
A method to include some content in the main file
by means of conditional processing is described in \secref{sec:conditional}.

%%%%%%%%%%%%%%%%%%%%%%%%%%%%%%%%%%%%%%%%
\paragraph{Page Numbering.}

When only a part of the document is compiled,
the appropriate numbering of pages
(as well as other status parameters)
is determined from the |.aux| files.
The latter contain information from previous passes.
However this information needs to propagate through
all intermediate child documents.
Therefore the page numbering in child documents may well
be inconsistent until the complete document is compiled at least once.

A useful (if unconventional) way to always ensure a consistent
page numbering is to restart the numbering in each child document
and denote the pages by `\textit{child}|.|\textit{page}'
where \textit{child} represents the chapter/section number of the child file.
This can be achieved by the command
|\numberwithin{page}{|\textit{child}|}|
of the \textsf{amsmath} package
where \textit{child} can be |chapter| or |section|
depending on the chosen structuring.
Alternatively, one can modify the macro |\thepage| appropriately
and reset the counter |page| at the start of each child file.

%%%%%%%%%%%%%%%%%%%%%%%%%%%%%%%%%%%%%%%%%%%%%%%%%%%%%%%%%%%%%%%%%%%%%%%%%%%%%%%%
\subsection{Conditional Processing}
\label{sec:conditional}

The package provides a mechanism to compile different versions
of a document. To customise the versions further some conditional processing
can come in handy to distinguish which version is being compiled.
The package provides two macros to describe the compilation context:

%%%%%%%%%%%%%%%%%%%%%%%%%%%%%%%%%%%%%%%%
\DescribeMacro{\ifchilddoc}
The conditional |\ifchilddoc| distinguishes between the compilation of
child documents and the main document:
%
\begin{center}
|\ifchilddoc |\textit{child-code}| |[|\||else |\textit{main-code}]| \||fi|
\end{center}

%%%%%%%%%%%%%%%%%%%%%%%%%%%%%%%%%%%%%%%%
\DescribeMacro{\childdocname}
\DescribeMacro{\childdocjob}
The macro |\childdocname| contains the filename (without extension)
of the main or child file being processed.
Note that |\childdocjob| will always contain the name of the main file.

%%%%%%%%%%%%%%%%%%%%%%%%%%%%%%%%%%%%%%%%
\paragraph{Title Page.}

Conditional processing can be used to include a title or banner page
in the main document when proper precautions are taken.
Importantly, the code in the main file should ensure that the page counter
(as well as other status parameters which are stored in the |.aux| files)
takes the same value after the conditional processing.
Otherwise the page numbers may take divergent values
depending on which part is compiled.

For example, a title page could be declared by:
%
\begin{center}
\begin{tabular}{l}
|\ifchilddoc\||else|\\
|\addtocounter{page}{-1}|\\
\textit{code for title page}\\
|\newpage|\\
|\||fi|
\end{tabular}
\end{center}
%
A banner page for the child documents can be generated by:
%
\begin{center}
\begin{tabular}{l}
|\ifchilddoc|\\
|\addtocounter{page}{-1}|\\
\textit{code for banner page}\\
|\newpage|\\
|\||fi|
\end{tabular}
\end{center}
%
Here one could write a message such as:
\begin{center}
|This is the part \childdocname{} of \childdocjob{}.|
\end{center}

%%%%%%%%%%%%%%%%%%%%%%%%%%%%%%%%%%%%%%%%%%%%%%%%%%%%%%%%%%%%%%%%%%%%%%%%%%%%%%%%
\subsection{Flags}
\label{sec:flags}

The package makes it easy to generate different versions
of the main or child documents.
To this end compilation flags can be defined
and assigned different default values.
They will be particularly useful in conjunction
with the forwarding mechanism described in \secref{sec:forward}.

For example, it may be useful to have a flag |\version|
which can be set to |draft| or |final|.
The document source will contain some conditional code
depending on the value of |\version|.
Suppose further, the flag should default to |final| for the main file
and to |draft| for child files
which is a natural assignment for editing the document.
This is achieved by placing the following code
in the preamble of the main document
(below the |\childdocmain| directive):
%
\begin{center}
\begin{tabular}{l}
|\ifchilddoc|\\
|\providecommand{\version}{draft}|\\
|\||else|\\
|\providecommand{\version}{final}|\\
|\||fi|
\end{tabular}
\end{center}
%
The definition by |\providecommand| makes sure
that previous definitions are not overwritten.
Further statements |\providecommand{\version}{...}|
can thus be added before the above code to override it.

For the main file, one might add a line
(between |\childdocmain| and the above block)
%
\begin{center}
|%\ifchilddoc\||else\providecommand{\version}{draft}\||fi|
\end{center}
%
which can be uncommented to produce a draft version.
Likewise one can add a line to the very top of a child file
(above the |\childdocof{|\textit{main}|}| directive)
%
\begin{center}
|%\providecommand{\version}{final}|
\end{center}
%
which can be uncommented to produce the final version of this child document.

%%%%%%%%%%%%%%%%%%%%%%%%%%%%%%%%%%%%%%%%%%%%%%%%%%%%%%%%%%%%%%%%%%%%%%%%%%%%%%%%
\subsection{Forwarding}
\label{sec:forward}

Different versions of the main or child documents
using compilation flags as described in \secref{sec:flags}
can be (permanently) stored in different files
for convenient compilation, viewing and distribution.
To this end, the package defines a command
to pass on compilation to a different file:

%%%%%%%%%%%%%%%%%%%%%%%%%%%%%%%%%%%%%%%%
\DescribeMacro{\childdocforward}
The command |\childdocforward| redirects processing to
another source file:
%
\begin{center}
\begin{tabular}{l}
|\input{childdoc.def}|\\
|\childdocforward[|\textit{main}|]{|\textit{dest}|}|\\
\end{tabular}
\end{center}
%
The argument \textit{dest} is the destination file
(without extension).
It should be the main file or one of the child files.
Note that further \textsf{childdoc} directives
such as |\childdocof| and |\childdocforward|
in the indicated file will be processed in this form.
The optional argument \textit{main}
passes on directly to the main file \textit{main}
while pretending to compile the child \textit{dest}.
This form behaves as if \textit{dest}
issues |\childdocof{|\textit{main}|}| right away,
and no further \textsf{childdoc} directives will be processed.

%%%%%%%%%%%%%%%%%%%%%%%%%%%%%%%%%%%%%%%%
\DescribeMacro{\...prefix}
In the alternative form |\childdocforwardprefix|,
%
\begin{center}
\begin{tabular}{l}
|\input{childdoc.def}|\\
|\childdocforwardprefix[|\textit{main}|]{|\textit{prefix}|}{|\textit{dest}|}|
\end{tabular}
\end{center}
%
the destination file is determined by a pattern
depending on the current file:
To make this work, the current file must be called
`{\textit{prefix}\hspace{0.2em}\textit{suffix}}'
with \textit{prefix} matching precisely the argument.
Processing is then passed on to the file
`{\textit{dest}\hspace{0.2em}\textit{suffix}}'.
Surely, the same effect is achieved by
directly specifying the
argument `{\textit{dest}\hspace{0.2em}\textit{suffix}}'
in the first form.
However, that requires to set up a different file
for each child. With the alternative form of the command
all these files can have exactly the same content
which simplifies setting them up and maintaining them.

For example, the following file |draft.tex|
with a compilation flag |\version| as described in \secref{sec:flags}
compiles the main document as a draft:
%
\begin{center}
\begin{tabular}{l}
|\def\version{draft}|\\
|\input{childdoc.def}|\\
|\childdocforward{|\textit{main}|}|
\end{tabular}
\end{center}
%
Likewise, the following files |final|\textit{nn}|.tex|
compile the final version of the child document
|child|\textit{nn}|.tex|:
%
\begin{center}
\begin{tabular}{l}
|\def\version{final}|\\
|\input{childdoc.def}|\\
|\childdocforwardprefix{final}{child}|
\end{tabular}
\end{center}
%

Note that when several versions of a main file and/or of each child file
are to be generated, it may be convenient to set up a |Makefile| or
shell script to automatise the process.

%%%%%%%%%%%%%%%%%%%%%%%%%%%%%%%%%%%%%%%%%%%%%%%%%%%%%%%%%%%%%%%%%%%%%%%%%%%%%%%%
\subsection{Command Line Processing}
\label{sec:commandline}

The effect of redirection files can also be achieved by invoking
the \LaTeX{} compiler with a more elaborate command line.
Most conveniently this should be done as part
of a shell script or a |Makefile|.

When using \textsf{childdoc} in the main file, the following
command lines effectively perform a redirection
(note that depending on the shell being used,
backslashes may have to be doubled: `|\|' $\to$ `|\\|'):
%
\begin{center}
|... -jobname "|\textit{target}|" |\\|"|[\textit{flags}]%
|\input{childdoc.def}\childdocforward[|\textit{main}|]{|\textit{dest}|}"|
\end{center}
%
Here \textit{target} is the name of the output file,
\textit{main} is the name of the main file
and \textit{dest} is the name of the main or child file to be processed
(all filenames without extensions).
The optional argument \textit{main} can be omitted
if \textit{main} matches \textit{dest}.
Optionally, compilation \textit{flags} can be defined via |\def| commands.
This command line makes the \TeX{} engine believe
it is compiling the file \textit{target}
whose content is specified as the latter parameter.
The provided code then forwards the processing to
\textit{main} or \textit{dest} as described in \secref{sec:forward}.

%%%%%%%%%%%%%%%%%%%%%%%%%%%%%%%%%%%%%%%%%%%%%%%%%%%%%%%%%%%%%%%%%%%%%%%%%%%%%%%%
\subsection{Include by Input}
\label{sec:input}

Including child documents by |\include| has some restrictions by design.
Most notably, the content of a child document always occupies
its own set of pages; pages cannot be shared between child documents.
Usually, this behaviour makes perfect sense
because each child document contain an essential part of the document.
However, in some situations it may be desirable to compose
a document from a collection of parts
without having mandatory page breaks between then.
For this case, the package
provides a mechanism to include parts
by |\input| which can also be processed individually.
However, by construction this mechanism
requires manual handling of the content to be output.

%%%%%%%%%%%%%%%%%%%%%%%%%%%%%%%%%%%%%%%%
\DescribeMacro{\ifchilddocmanual}
The main file should be prepared as usual, see \secref{sec:include}.
However, the document body must make a distinction
between processing of an individual part and of the main document, e.g.:
%
\begin{center}
\begin{tabular}{l}
|\ifchilddocmanual|\\
|\input{\childdocname}|\\
|\||else|\\
\textit{document body with }|\input{|\textit{part}|}|\\
|\||fi|
\end{tabular}
\end{center}
%
The conditional |\ifchilddocmanual| is true whenever
a part to be included by |\input| is being compiled,
and the name of the part is stored in |\childdocname|.

%%%%%%%%%%%%%%%%%%%%%%%%%%%%%%%%%%%%%%%%
\DescribeMacro{\childdocby}
Each part to be included by |\input| should start with:
%
\begin{center}
\begin{tabular}{l}
|\input{childdoc.def}|\\
|\childdocby{|\textit{main}|}|\\
\end{tabular}
\end{center}
%
The directive |\childdocby| is similar to |\childdocof|
described in \secref{sec:include},
but the subsequent selection of content must be done manually.
To that end, both |\ifchilddoc| and |\ifchilddocmanual|
will be true upon processing of a part,
and the name of the part is stored in |\childdocname|.
Note that |\jobname| will be set to the filename of the current part
so that each part receives an individual |.aux| file
that does not interfere with the |.aux| file(s) of the main document.
This behaviour can be altered by the alternative form
|\childdocby[*]{|\textit{main}|}| (with a non-empty optional argument)
which uses the |.aux| file of the main document
by setting |\jobname| to \textit{main}.

%%%%%%%%%%%%%%%%%%%%%%%%%%%%%%%%%%%%%%%%%%%%%%%%%%%%%%%%%%%%%%%%%%%%%%%%%%%%%%%%
\subsection{Driver Development}
\label{sec:driver}

The \textsf{childdoc} mechanism can also be use for the development
of definition files such as \LaTeX{} styles or classes.
This case differs from the above setup with multiple parts
included by |\include| in that no |\includeonly| should be invoked.
This can be achieved by starting the include file
(before |\ProvidesPackage|) with:
%
\begin{center}
\begin{tabular}{l}
|\input{childdoc.def}|\\
|\childdocforward{|\textit{main}|}|\\
\end{tabular}
\end{center}
%
or alternatively with:
%
\begin{center}
\begin{tabular}{l}
|\input{childdoc.def}|\\
|\childdocby{|\textit{main}|}|\\
\end{tabular}
\end{center}
%
Both forms have slightly different effects as described above.
The main file is prepared as usual, see \secref{sec:include}.

%%%%%%%%%%%%%%%%%%%%%%%%%%%%%%%%%%%%%%%%%%%%%%%%%%%%%%%%%%%%%%%%%%%%%%%%%%%%%%%%
\subsection{Legacy Detection}
\label{sec:detection}

The directive |\childdocmain| in the main file can detect
whether the complete document or merely a child is to be compiled
even without using the directive |\childdocof|.
This method is deprecated because it is less robust
and there is no compelling reason to use it;
it is merely provided for backward compatibility
and it may be removed in future versions.

If the detection mechanism is to be used,
it is mandatory to correctly specify
the filename of the main file as the argument of |\childdocmain|:
%
\begin{center}
\begin{tabular}{l}
|\input{childdoc.def}|\\
|\childdocmain{|\textit{main}|}|\\
\end{tabular}
\end{center}
%
If |\jobname| does not match the argument \textit{main} of |\childdocmain|,
it is assumed that |\jobname| points to the child file to be compiled.
When using |\childdocmain| with the main file specified as argument,
it suffices to start a child file
with just |\input{|\textit{main}|}|
without loading of the package and using |\childdocof|.
If instead all processing is done
with the appropriate \textsf{childdoc} directives,
the argument of \textit{main} of |\childdocmain| can be empty.

An alternative version of the command line processing described
in \secref{sec:commandline} using the detection mechanism reads:
%
\begin{center}
|... -jobname "|\textit{target}|" "|[\textit{flags}]%
[|\def\jobname{|\textit{dest}|}|]|\input{|\textit{main}|}"|
\end{center}

%%%%%%%%%%%%%%%%%%%%%%%%%%%%%%%%%%%%%%%%%%%%%%%%%%%%%%%%%%%%%%%%%%%%%%%%%%%%%%%%
\subsection{Manual Code}
\label{sec:manual}

In case one cannot be certain whether the definitions file |childdoc.def|
is installed on the target \TeX{} distribution
and one prefers not to ship it,
it is conceivable to paste a few relevant commands into the sources.

To that end, drop all statements |\input{childdoc.def}|
and perform the replacements as outlined below.
Instead of |\childdocmain{|\textit{main}|}| add the following code
to the top of the main file:
%
\begin{center}
\begin{tabular}{l}
|\||ifdefined\childdocname\endinput\||fi\newif\ifchilddoc|\\
|\edef\childdocname{\scantokens\expandafter{\jobname\noexpand}}|\\
|\def\childdocmain{|\textit{main}|}\||ifx\childdocmain\childdocname\||else|\\
|\childdoctrue\includeonly{\childdocname}\let\jobname\childdocmain\||fi|\\
\end{tabular}
\end{center}
%
Instead of |\childdocof{|\textit{main}|}| just include the main file
at the top of each child file:
%
\begin{center}
|\input{|\textit{main}|}|
\end{center}
%
A simple redirection |\childdocforward{|\textit{dest}|}| is achieved by:
%
\begin{center}
|\def\jobname{|\textit{dest}|}\input{\jobname}|
\end{center}
%
The redirection with prefix
|\childdocforwardprefix[|\textit{prefix}|]{|\textit{dest}|}|
is accomplished by:
%
\begin{center}
\begin{tabular}{l}
|{\edef\jobname{\scantokens\expandafter{\jobname\noexpand}}|\\
|\def\redirectjob |\textit{prefix}|#1~~~{\gdef\jobname{|\textit{dest}|#1}}|\\
|\expandafter\redirectjob\jobname~~~}\input{\jobname}|
\end{tabular}
\end{center}

In an alternative approach,
child documents can be compiled by a specific command line
without additional code or specific definitions:
%
\begin{center}
|... -jobname "|\textit{target}|" "|[\textit{flags}]%
|\includeonly{|\textit{dest}|}\input{|\textit{main}|}"|
\end{center}
%

%%%%%%%%%%%%%%%%%%%%%%%%%%%%%%%%%%%%%%%%%%%%%%%%%%%%%%%%%%%%%%%%%%%%%%%%%%%%%%%%
%%%%%%%%%%%%%%%%%%%%%%%%%%%%%%%%%%%%%%%%%%%%%%%%%%%%%%%%%%%%%%%%%%%%%%%%%%%%%%%%
\section{Information}

%%%%%%%%%%%%%%%%%%%%%%%%%%%%%%%%%%%%%%%%%%%%%%%%%%%%%%%%%%%%%%%%%%%%%%%%%%%%%%%%
\subsection{Copyright}

Copyright \copyright{} 2017--2018 Niklas Beisert

This work may be distributed and/or modified under the
conditions of the \LaTeX{} Project Public License, either version 1.3
of this license or (at your option) any later version.
The latest version of this license is in
  \url{http://www.latex-project.org/lppl.txt}
and version 1.3 or later is part of all distributions of \LaTeX{}
version 2005/12/01 or later.

This work has the LPPL maintenance status `maintained'.

The Current Maintainer of this work is Niklas Beisert.

This work consists of the files |README.txt|, |childdoc.ins| and |childdoc.dtx|
as well as the derived files |childdoc.def|, |cdocsamp.tex|
with |cdocsch1.tex|, |cdocsch2.tex|, |cdocspt3.tex|, |cdocspt4.tex|,
|cdocsdrf.tex|, |cdocsfn1.tex|, |cdocsfn2.tex|
as well as |childdoc.pdf|.

%%%%%%%%%%%%%%%%%%%%%%%%%%%%%%%%%%%%%%%%%%%%%%%%%%%%%%%%%%%%%%%%%%%%%%%%%%%%%%%%
\subsection{Files and Installation}

The package consists of the files:
%
\begin{center}
\begin{tabular}{ll}
    |README.txt|   & readme file \\
    |childdoc.ins| & installation file \\
    |childdoc.dtx| & source file \\
    |childdoc.def| & definition file \\
    |cdocsamp.tex| & sample main file \\
    |cdocsch1.tex| & sample include file \\
    |cdocsch2.tex| & sample include file \\
    |cdocspt3.tex| & sample part file \\
    |cdocspt4.tex| & sample part file \\
    |cdocsdrf.tex| & sample redirection file \\
    |cdocsfn1.tex| & sample redirection file \\
    |cdocsfn2.tex| & sample redirection file \\
    |childdoc.pdf| & manual
\end{tabular}
\end{center}
%
The distribution consists of the files
|README.txt|, |childdoc.ins| and |childdoc.dtx|.
%
\begin{itemize}
\item
Run (pdf)\LaTeX{} on |childdoc.dtx|
to compile the manual |childdoc.pdf| (this file).
\item
Run \LaTeX{} on |childdoc.ins| to create the definitions file |childdoc.def|
and the sample |cdocsamp.tex| with include files
|cdocsch1.tex|, |cdocsch2.tex|, |cdocspt3.tex|, |cdocspt4.tex|,
|cdocsdrf.tex|, |cdocsfn1.tex|, |cdocsfn2.tex|.
Then copy the file |childdoc.def| to an appropriate directory of your \LaTeX{}
distribution, e.g.\ \textit{texmf-root}|/tex/latex/childdoc|.
\end{itemize}

%%%%%%%%%%%%%%%%%%%%%%%%%%%%%%%%%%%%%%%%%%%%%%%%%%%%%%%%%%%%%%%%%%%%%%%%%%%%%%%%
\subsection{Related CTAN Packages}

There are several other packages which offer a similar functionality:
%
\begin{itemize}
\item
The packages
\href{http://ctan.org/pkg/docmute}{\textsf{docmute}},
\href{http://ctan.org/pkg/includex}{\textsf{includex}} and
\href{http://ctan.org/pkg/standalone}{\textsf{standalone}}
provide commands to include only the document body of
a child file thus allowing both files to be compiled individually.
\item
The packages \href{http://ctan.org/pkg/subdocs}{\textsf{subdocs}}
and \href{http://ctan.org/pkg/subfiles}{\textsf{subfiles}}
provide structures in which the main and child documents can be
encapsulated and allowing them to be compiled individually.
The inclusion mechanism is different from the conventional |\include|.
\item
The package \href{http://ctan.org/pkg/combine}{\textsf{combine}}
is an elaborate solution to combine several documents into one.
\end{itemize}
%
See also the CTAN topic \href{http://ctan.org/topic/subdocs}{\textsf{subdocs}}
for further related packages.
The present package differs from the above solutions in that
a document structure constructed with the conventional |\include| mechanism
just needs two extra commands at the top of every file
such that all constituent files can be compiled individually.

%%%%%%%%%%%%%%%%%%%%%%%%%%%%%%%%%%%%%%%%%%%%%%%%%%%%%%%%%%%%%%%%%%%%%%%%%%%%%%%%
%\subsection{Feature Suggestions}
%
%The following is a list of features which may be useful for future
%versions of this package:
%%
%\begin{itemize}
%\item
%\ldots
%\end{itemize}

%%%%%%%%%%%%%%%%%%%%%%%%%%%%%%%%%%%%%%%%%%%%%%%%%%%%%%%%%%%%%%%%%%%%%%%%%%%%%%%%
\subsection{Revision History}

%%%%%%%%%%%%%%%%%%%%%%%%%%%%%%%%%%%%%%%%
\paragraph{v2.0:} 2018/12/30

\begin{itemize}
\item
immediate forward processing
\item
added |\childdocby| mechanism
\item
manual restructured
\end{itemize}

%%%%%%%%%%%%%%%%%%%%%%%%%%%%%%%%%%%%%%%%
\paragraph{v1.6:} 2018/01/17

\begin{itemize}
\item
application for development of include files
\item
corrections to manual
\end{itemize}

%%%%%%%%%%%%%%%%%%%%%%%%%%%%%%%%%%%%%%%%
\paragraph{v1.5:} 2017/05/21

\begin{itemize}
\item
more complete structuring introduced
\item
|\childdocof| introduced
\item
|\childdoc| renamed to |\childdocmain|
\item
|\childredirect| renamed to |\childdocforward| and |\childdocforwardprefix|
and functionality expanded
\end{itemize}

%%%%%%%%%%%%%%%%%%%%%%%%%%%%%%%%%%%%%%%%
\paragraph{v1.0:} 2017/04/27

\begin{itemize}
\item
manual and install package
\item
first version published on CTAN
\end{itemize}

%%%%%%%%%%%%%%%%%%%%%%%%%%%%%%%%%%%%%%%%
\paragraph{v0.6:} 2017/04/26

\begin{itemize}
\item
redirection mechanism added
\end{itemize}

%%%%%%%%%%%%%%%%%%%%%%%%%%%%%%%%%%%%%%%%
\paragraph{v0.5:} 2017/04/26

\begin{itemize}
\item
functionality in definition file
\end{itemize}


%%%%%%%%%%%%%%%%%%%%%%%%%%%%%%%%%%%%%%%%%%%%%%%%%%%%%%%%%%%%%%%%%%%%%%%%%%%%%%%%
%%%%%%%%%%%%%%%%%%%%%%%%%%%%%%%%%%%%%%%%%%%%%%%%%%%%%%%%%%%%%%%%%%%%%%%%%%%%%%%%
%%%%%%%%%%%%%%%%%%%%%%%%%%%%%%%%%%%%%%%%%%%%%%%%%%%%%%%%%%%%%%%%%%%%%%%%%%%%%%%%
\appendix

\settowidth\MacroIndent{\rmfamily\scriptsize 000\ }

 \DocInput{childdoc.dtx}

\end{document}
%</driver>
% \fi
%
% %%%%%%%%%%%%%%%%%%%%%%%%%%%%%%%%%%%%%%%%%%%%%%%%%%%%%%%%%%%%%%%%%%%%%%%%%%%%%%
% %%%%%%%%%%%%%%%%%%%%%%%%%%%%%%%%%%%%%%%%%%%%%%%%%%%%%%%%%%%%%%%%%%%%%%%%%%%%%%
% \section{Sample}
%\iffalse
%<*samplemain>
%\fi
%
% The following presents a sample document
% with two chapters, two parts, a title page,
% a compile flag as well as three forwarding files to set the flag.
% It consists of eight |.tex| files:
% \begin{center}
% \begin{tabular}{ll}
% |cdocsamp.tex|&main file\\
% |cdocsch1.tex|&include file for chapter 1\\
% |cdocsch2.tex|&include file for chapter 2\\
% |cdocspt3.tex|&include file for part 3\\
% |cdocspt4.tex|&include file for part 4\\
% |cdocsdrf.tex|&forwarding file for main file in draft mode\\
% |cdocsfi1.tex|&forwarding file for final version of chapter 1\\
% |cdocsfi2.tex|&forwarding file for final version of chapter 2\\
% \end{tabular}
% \end{center}
% Each of the eight files can be compiled directly by the \LaTeX{} compiler.
%
% %%%%%%%%%%%%%%%%%%%%%%%%%%%%%%%%%%%%%%
% \paragraph{Main File.}
%
% The main file is called |cdocsamp.tex|.
%
% Load the \textsf{childdoc} definitions and
% declare the filename for the main document:
%    \begin{macrocode}
\input{childdoc.def}
\childdocmain{}
%    \end{macrocode}

% Optional override for |\version| flag:
%    \begin{macrocode}
%%\ifchilddoc\else\providecommand{\version}{draft}\fi
%    \end{macrocode}

% Define the default values for the |\version| flag
% (|final| for the main file and |draft| for childs):
%    \begin{macrocode}
\ifchilddoc
\providecommand{\version}{draft}
\else
\providecommand{\version}{final}
\fi
%    \end{macrocode}

% Load the standard document class:
%    \begin{macrocode}
\documentclass[12pt]{article}
%    \end{macrocode}

% Start the document body:
%    \begin{macrocode}
\begin{document}
%    \end{macrocode}

% Declare a title page.
% Print title, part of document being processed and version flag:
%    \begin{macrocode}
\addtocounter{page}{-1}
\begin{center}
{\LARGE\bfseries{}childdoc example\par}
\vspace{1cm}
\ifchilddoc
\ifchilddocmanual part\else chapter\fi:
`\childdocname' of `\childdocjob'\par
\else
main document: `\childdocjob'\par
\fi
version: \version\par
\end{center}
\newpage
%    \end{macrocode}

% Manually include selected file,
% otherwise process as usual:
%    \begin{macrocode}
\ifchilddocmanual
\section*{part `\childdocname'}
\input{\childdocname}
\else
%    \end{macrocode}

% Include the two chapters:
%    \begin{macrocode}
\include{cdocsch1}
\include{cdocsch2}
%    \end{macrocode}

% Include the two parts unless only chapters should be displayed:
%    \begin{macrocode}
\ifchilddoc\else
\section{part three}
\input{cdocspt3}
\section{part four}
\input{cdocspt4}
\fi
%    \end{macrocode}

% Process as usual until here:
%    \begin{macrocode}
\fi
%    \end{macrocode}

% End of document body:
%    \begin{macrocode}
\end{document}
%    \end{macrocode}
%\iffalse
%</samplemain>
%\fi
%
% %%%%%%%%%%%%%%%%%%%%%%%%%%%%%%%%%%%%%%
% \paragraph{Chapter Include Files.}
%
% The include files are called |cdocsch1.tex| and |cdocsch2.tex|.
%
%\iffalse
%<*samplechap1|samplechap2>
%\fi

% Optional override for |\version| flag:
%    \begin{macrocode}
%%\providecommand{\version}{final}
%    \end{macrocode}

% Include the main document:
%    \begin{macrocode}
\input{childdoc.def}
\childdocof{cdocsamp}
%    \end{macrocode}

%\iffalse
%</samplechap1|samplechap2>
%\fi
%
%\iffalse
%<*samplechap1>
%\fi
% Some text for chapter 1:
%    \begin{macrocode}
\section{one}
some text in chapter one
%    \end{macrocode}

%\iffalse
%</samplechap1>
%\fi
% Some text for chapter 2:
%\iffalse
%<*samplechap2>
%\fi
%    \begin{macrocode}
\section{two}
more text in chapter two
%    \end{macrocode}

%\iffalse
%</samplechap2>
%\fi
%
% %%%%%%%%%%%%%%%%%%%%%%%%%%%%%%%%%%%%%%
% \paragraph{Part Include Files.}
%
% The include files are called |cdocspt3.tex| and |cdocspt4.tex|.
%
%\iffalse
%<*samplepart3|samplepart4>
%\fi

% Optional override for |\version| flag:
%    \begin{macrocode}
%%\providecommand{\version}{final}
%    \end{macrocode}

% Include the main document:
%    \begin{macrocode}
\input{childdoc.def}
\childdocby{cdocsamp}
%    \end{macrocode}

%\iffalse
%</samplepart3|samplepart4>
%\fi
%
%\iffalse
%<*samplepart3>
%\fi
% Some text for part 3:
%    \begin{macrocode}
some text in part three
%    \end{macrocode}

%\iffalse
%</samplepart3>
%\fi
% Some text for part 4:
%\iffalse
%<*samplepart4>
%\fi
%    \begin{macrocode}
more text in part four
%    \end{macrocode}

%\iffalse
%</samplepart4>
%\fi
%
% %%%%%%%%%%%%%%%%%%%%%%%%%%%%%%%%%%%%%%
% \paragraph{Forwarding for a Complete Draft.}
%
% The following forwarding file |cdocsdrf.tex|
% compiles the main document in draft mode:
%\iffalse
%<*sampledraft>
%\fi
%    \begin{macrocode}
\def\version{draft}
\input{childdoc.def}
\childdocforward{cdocsamp}
%    \end{macrocode}

%\iffalse
%</sampledraft>
%\fi
%
% %%%%%%%%%%%%%%%%%%%%%%%%%%%%%%%%%%%%%%
% \paragraph{Forwarding for Final Version of the Chapters.}
%
% The following forwarding files |cdocsfn1.tex| and |cdocsfn2.tex|
% (with identical content)
% compile the final versions of the child documents
% |cdocsch1.tex| and |cdocsch2.tex|, respectively:
%\iffalse
%<*samplefinal>
%\fi
%    \begin{macrocode}
\def\version{final}
\input{childdoc.def}
\childdocforwardprefix[cdocsamp]{cdocsfn}{cdocsch}
%    \end{macrocode}

%\iffalse
%</samplefinal>
%\fi
%
% %%%%%%%%%%%%%%%%%%%%%%%%%%%%%%%%%%%%%%
% \paragraph{Command Line Processing.}
%
% The following three command lines generate the output files
% |cdocscld|, |cdocscl1| and |cdocscl2|
% which should be identical to
% |cdocsdrf|, |cdocsch1| and |cdocsfn2|, respectively:
% \begin{center}
% \begin{tabular}{l}
% |latex -jobname cdocscld \|\\
% |  "\def\version{draft}\input{childdoc.def}\childdocforward{cdocsamp}"|\\
% |latex -jobname cdocscl1 \|\\
% |  "\input{childdoc.def}\childdocforward[cdocsamp]{cdocsch1}"|\\
% |latex -jobname cdocscl2 \|\\
% |  "\def\version{final}\input{childdoc.def}\childdocforward{cdocsch2}"|
% \end{tabular}
% \end{center}
% Note that the trailing backslash on each first line
% merely continues the input to the second line
% (for convenient cut ant paste).
% Furthermore, the command |latex| can be replaced by any
% of its alternative versions such as |pdflatex|.
%
% %%%%%%%%%%%%%%%%%%%%%%%%%%%%%%%%%%%%%%%%%%%%%%%%%%%%%%%%%%%%%%%%%%%%%%%%%%%%%%
% %%%%%%%%%%%%%%%%%%%%%%%%%%%%%%%%%%%%%%%%%%%%%%%%%%%%%%%%%%%%%%%%%%%%%%%%%%%%%%
% \section{Implementation}
%\iffalse
%<*package>
%\fi
%
% This section describes the definitions file |childdoc.def|.

% The definitions cannot be loaded using |\usepackage| or |\RequirePackage|
% which has a mechanism to prevent loading a style file more than once.
% When loading the definitions by means of |\input|
% multiple instances have to be prevented manually:
%\iffalse
%This code needs to be before the `\ProvidesFile' directive
%which is defined at the beginning of this file.
%Therefore it is also placed there and commented out here.
%</package>
%<*discard>
%\fi
%    \begin{macrocode}
\ifdefined\childdocmain\endinput\fi
%    \end{macrocode}
%\iffalse
%</discard>
%<*package>
%\fi
%
% \macro{\ifchilddoc}
% \macro{\ifchilddocmanual}
% The conditional |\ifchilddoc| tells whether a
% child (true) or main (false) document is being compiled.
% The conditional |\ifchilddocmanual| tells whether
% the |\includeonly| mechanism is used (false) or
% the selection of child files must be performed manually (true).
% The definitions initialise to false:
%    \begin{macrocode}
\newif\ifchilddoc
\newif\ifchilddocmanual
%    \end{macrocode}

% \macro{\childdocname}
% \macro{\childdocjob}
% The macro |\childdocname| stores the name of the main document
% to be compiled. The macro |\childdocjob| stores the name of
% the document on which the \LaTeX{} compiler was originally invoked.
% The content of |\jobname| cannot be compared
% to filenames specified in the source due to different catcodes.
% The following code rescans |\jobname|, stores the result
% in |\childdocname| and saves a copy in |\childdocjob|:
%    \begin{macrocode}
\edef\childdocname{\scantokens\expandafter{\jobname\noexpand}}
\let\childdocjob\childdocname
%    \end{macrocode}

% \macro{\childdocdisable}
% The macro |\childdocdisable| prevents the main file
% from being processed more than once.
% At this stage, the main document command |\childdocmain|
% is assumed to be called once again where it should do nothing.
% Any subsequent call to it should prevent
% a secondary processing of the main document
% It overwrites the forwarding commands
% |\childdocof| and |\childdocforward|
% with empty macros to prevent further inclusions of the main document:
%    \begin{macrocode}
\newcommand{\childdocdisable}
{
  \renewcommand{\childdocmain}[1]{\renewcommand{\childdocmain}[1]{\endinput}}
  \renewcommand{\childdocof}[1]{}
  \renewcommand{\childdocby}[2][]{}
  \renewcommand{\childdocforward}[2][]{}
  \renewcommand{\childdocdisable}{}
}
%    \end{macrocode}

% \macro{\childdocmain}
% The macro |\childdocmain| is to be called at the top of the main file
% with nothing or the main filename (without extension) as argument.
% First, it breaks loops.
% If the argument is not empty and does not match |\childdocname|
% (which is set by the first inclusion of |childdoc.def|),
% |\ifchilddoc| is set to true, |\includeonly| is applied to the child file
% and |\jobname| is set to the main file
% (for proper handling of |.aux| files):
%    \begin{macrocode}
\newcommand{\childdocmain}[1]
{
  \childdocdisable\childdocmain{}
  \if?#1?\else
    \begingroup
      \def\childdoctmp{#1}
      \ifx\childdoctmp\childdocname
        \def\childdoctmp{}
      \else
        \def\childdoctmp
        {
          \childdoctrue
          \includeonly{\childdocname}
          \def\childdocjob{#1}
          \def\jobname{#1}
        }
      \fi
      \expandafter
    \endgroup
    \childdoctmp
  \fi
}
%    \end{macrocode}

% \macro{\childdocof}
% The command |\childdocof| redirects
% compilation to the main file |#1|.
%    \begin{macrocode}
\newcommand{\childdocof}[1]
{
  \childdocdisable
  \childdoctrue
  \includeonly{\childdocname}
  \def\jobname{#1}
  \def\childdocjob{#1}
  \input{#1}
}
%    \end{macrocode}

% \macro{\childdocby}
% The command |\childdocby| ....
%    \begin{macrocode}
\newcommand{\childdocby}[2][]
{
  \childdocdisable
  \childdoctrue
  \childdocmanualtrue
  \if?#1?\else
    \def\jobname{#2}
  \fi
  \def\childdocjob{#2}
  \input{#2}
  \endinput
}
%    \end{macrocode}

% \macro{\childdocforward}
% The command |\childdocforward| redirects
% compilation to the main file or
% (if the optional argument is given) a child file.
% Parameters are set as if the main file
% or a child file starting with |\childdocof| was compiled.
% Then compilation is handed over to the main file:
%    \begin{macrocode}
\newcommand{\childdocforward}[2][]
{
  \begingroup
    \if?#1?
      \def\childdoctmp
      {
        \def\childdocname{#2}
        \def\childdocjob{#2}
        \def\jobname{#2}
        \input{#2}
        \endinput
      }
    \else
      \def\childdoctmp
      {
        \childdocdisable
        \def\childdocname{#2}
        \childdoctrue
        \includeonly{#2}
        \def\childdocjob{#1}
        \def\jobname{#1}
        \input{#1}
        \endinput
      }
    \fi
    \expandafter
  \endgroup
  \childdoctmp
}
%    \end{macrocode}

% \macro{\childdocforwardprefix}
% The command |\childdocforwardprefix| redirects
% compilation to the main or a child file by means of a pattern.
% The prefix |#1| in the current filename is replaced by |#2|
% and the suffix of the current filename is kept
% (it is assumed that the filename does not contain the substring `|~~~|'
% which is used as a delimiter).
% Compilation is handed over to the new file by |\childdocforward|:
%    \begin{macrocode}
\newcommand{\childdocforwardprefix}[3][]
{
  \begingroup
    \def\childdocextract #2##1~~~{\def\childdoctmp{\childdocforward[#1]{#3##1}}}
    \expandafter\childdocextract\childdocname~~~
    \expandafter
  \endgroup
  \childdoctmp
}
%    \end{macrocode}

% \macro{\childdoc}
% The deprecated macro |\childdoc| is a legacy version of |\childdocmain|:
%    \begin{macrocode}
\newcommand{\childdoc}{\childdocmain}
%    \end{macrocode}

% \macro{\childdocredirect}
% The deprecated macro |\childdocredirect| is a legacy version
% of |\childdocforward| and |\childdocforwardprefix|:
%    \begin{macrocode}
\newcommand{\childdocredirect}[2][]
{
  \begingroup
    \if?#1?
      \def\childdoctmp{\childdocforward{#2}}
    \else
      \def\childdoctmp{\childdocforwardprefix{#1}{#2}}
    \fi
    \expandafter
  \endgroup
  \childdoctmp
}
%    \end{macrocode}

%\iffalse
%</package>
%\fi
%
\endinput
|\\
|\childdocforwardprefix[|\textit{main}|]{|\textit{prefix}|}{|\textit{dest}|}|
\end{tabular}
\end{center}
%
the destination file is determined by a pattern
depending on the current file:
To make this work, the current file must be called
`{\textit{prefix}\hspace{0.2em}\textit{suffix}}'
with \textit{prefix} matching precisely the argument.
Processing is then passed on to the file
`{\textit{dest}\hspace{0.2em}\textit{suffix}}'.
Surely, the same effect is achieved by
directly specifying the
argument `{\textit{dest}\hspace{0.2em}\textit{suffix}}'
in the first form.
However, that requires to set up a different file
for each child. With the alternative form of the command
all these files can have exactly the same content
which simplifies setting them up and maintaining them.

For example, the following file |draft.tex|
with a compilation flag |\version| as described in \secref{sec:flags}
compiles the main document as a draft:
%
\begin{center}
\begin{tabular}{l}
|\def\version{draft}|\\
|% \iffalse
%
% childdoc.dtx Copyright (C) 2017-2018 Niklas Beisert
%
% This work may be distributed and/or modified under the
% conditions of the LaTeX Project Public License, either version 1.3
% of this license or (at your option) any later version.
% The latest version of this license is in
%   http://www.latex-project.org/lppl.txt
% and version 1.3 or later is part of all distributions of LaTeX
% version 2005/12/01 or later.
%
% This work has the LPPL maintenance status `maintained'.
%
% The Current Maintainer of this work is Niklas Beisert.
%
% This work consists of the files childdoc.dtx and childdoc.ins
% and the derived files childdoc.def and cdocsamp.tex with
% cdocsch1.tex, cdocsch2.tex, cdocsdrf.tex, cdocsfn1.tex, cdocsfn2.tex.
%
%<package>\ifdefined\childdocmain\endinput\fi
%<package>\ProvidesFile{childdoc.def}[2018/12/30 v2.0 child document driver]
%<samplemain>\ProvidesFile{cdocsamp.tex}[2018/12/30 v2.0 sample for childdoc]
%<*driver>
%\ProvidesFile{childdoc.drv}[2018/12/30 v2.0 childdoc reference manual file]
\PassOptionsToClass{10pt,a4paper}{article}
\documentclass{ltxdoc}

\usepackage[margin=35mm]{geometry}
\usepackage{hyperref}
\usepackage{hyperxmp}
\usepackage[usenames]{color}

\hypersetup{colorlinks=true}
\hypersetup{pdfstartview=FitH}
\hypersetup{pdfpagemode=UseNone}
\hypersetup{pdfsource={}}
\hypersetup{pdflang={en-UK}}
\hypersetup{pdfcopyright={Copyright 2017-2018 Niklas Beisert.
  This work may be distributed and/or modified under the
  conditions of the LaTeX Project Public License, either version 1.3
  of this license or (at your option) any later version.}}
\hypersetup{pdflicenseurl={http://www.latex-project.org/lppl.txt}}
\hypersetup{pdfcontactaddress={ETH Zurich, ITP, HIT K,
  Wolfgang-Pauli-Strasse 27}}
\hypersetup{pdfcontactpostcode={8093}}
\hypersetup{pdfcontactcity={Zurich}}
\hypersetup{pdfcontactcountry={Switzerland}}
\hypersetup{pdfcontactemail={nbeisert@itp.phys.ethz.ch}}
\hypersetup{pdfcontacturl={http://people.phys.ethz.ch/\xmptilde nbeisert/}}

\newcommand{\secref}[1]{\hyperref[#1]{section \ref*{#1}}}

\parskip1ex
\parindent0pt
\let\olditemize\itemize
\def\itemize{\olditemize\parskip0pt}

\begin{document}

\title{The \textsf{childdoc} Package}
\hypersetup{pdftitle={The childdoc Package}}
\author{Niklas Beisert\\[2ex]
  Institut f\"ur Theoretische Physik\\
  Eidgen\"ossische Technische Hochschule Z\"urich\\
  Wolfgang-Pauli-Strasse 27, 8093 Z\"urich, Switzerland\\[1ex]
  \href{mailto:nbeisert@itp.phys.ethz.ch}
  {\texttt{nbeisert@itp.phys.ethz.ch}}}
\hypersetup{pdfauthor={Niklas Beisert}}
\hypersetup{pdfsubject={Manual for the LaTeX2e Package childdoc}}
\date{30 December 2018, \textsf{v2.0}}
\maketitle

\begin{abstract}\noindent
\textsf{childdoc} is a \LaTeXe{} package
that enables the direct compilation
of document sections included by |\include|
to individual files.
\end{abstract}

\begingroup
\parskip0ex
\tableofcontents
\endgroup

%%%%%%%%%%%%%%%%%%%%%%%%%%%%%%%%%%%%%%%%%%%%%%%%%%%%%%%%%%%%%%%%%%%%%%%%%%%%%%%%
%%%%%%%%%%%%%%%%%%%%%%%%%%%%%%%%%%%%%%%%%%%%%%%%%%%%%%%%%%%%%%%%%%%%%%%%%%%%%%%%
\section{Introduction}

\LaTeX{} provides a mechanism to structure a large document (such as a book)
into a main file and several child files (containing the chapters)
using the |\include| command.
This mechanism is beneficial for documents
which span hundreds of pages in order to
make the source file(s) more manageable.
Moreover, compilation can be restricted to
selected child files by means of the |\includeonly| command.
The latter feature can be used to reduce the compilation time while editing
(this was significantly more useful in the earlier days of \LaTeX{})
or to generate a smaller document which is easier to navigate.
Another application of |\includeonly| is to generate
documents consisting of selected parts of the complete document.

However, there are a few drawbacks of the plain |\include| mechanism:
\begin{itemize}
\item
The child files cannot be compiled on their own,
they can only be compiled via the main file.
A naive editing environment
(such as a text editor with an option
to have the current file processed by \LaTeX)
may require one to switch to the main file before compiling;
attempting to compile the child file produces errors.
\item
The main file must be modified (each time)
to adjust the |\includeonly| command
to the present needs. This easily leaves the main file in a messy state.
\item
The generated document will always carry the filename
of the main document. This is inconvenient if
several child files are to be compiled and
to be kept for distribution.
\end{itemize}

The present package provides a simple interface
to make child files individually compilable by \LaTeX{}.
Compiling a child file then has the same effect as compiling
the main file with an |\includeonly| command
to select the appropriate child.
Moreover the generated document will carry the name of the child
rather than the main file.
This resolves all three above issues.

This feature is meant to make the editing of books,
thesis documents and lecture notes somewhat more convenient.
However, the package can also be used efficiently for
composing a series of documents (such as exercise sheets)
which are typically distributed individually.
It then assists the author in generating the individual documents
(potentially in different versions)
as well as a document containing the collected series.
Another application is in developing style files
or other kinds of included material
where compilation of the style file could redirect
to a sample or test file.

%%%%%%%%%%%%%%%%%%%%%%%%%%%%%%%%%%%%%%%%%%%%%%%%%%%%%%%%%%%%%%%%%%%%%%%%%%%%%%%%
%%%%%%%%%%%%%%%%%%%%%%%%%%%%%%%%%%%%%%%%%%%%%%%%%%%%%%%%%%%%%%%%%%%%%%%%%%%%%%%%
\section{Usage}

First of all, the package \textsf{childdoc} is \emph{not} a standard
\LaTeXe{} |.sty| style file! Therefore it needs to be invoked in
a non-standard way.

%%%%%%%%%%%%%%%%%%%%%%%%%%%%%%%%%%%%%%%%%%%%%%%%%%%%%%%%%%%%%%%%%%%%%%%%%%%%%%%%
\subsection{Included Files}
\label{sec:include}

%%%%%%%%%%%%%%%%%%%%%%%%%%%%%%%%%%%%%%%%
\DescribeMacro{\childdocmain}
To use the package, add the commands
\begin{center}
\begin{tabular}{l}
|\input{childdoc.def}|\\
|\childdocmain{}|\\
\end{tabular}
\end{center}
at the very top of the main \LaTeX{} file,
in particular \emph{before} the |\documentclass| statement!
The argument of |\childdocmain| should be left empty
(but it must be present).

%%%%%%%%%%%%%%%%%%%%%%%%%%%%%%%%%%%%%%%%
\DescribeMacro{\childdocof}
Furthermore, add the commands
\begin{center}
\begin{tabular}{l}
|\input{childdoc.def}|\\
|\childdocof{|\textit{main}|}|\\
\end{tabular}
\end{center}
at the top of every child file \textit{child}
which is included by |\include{|\textit{child}|}|
from within the main file
(or at least for those files to be compiled individually).
The argument \textit{main} must be the filename of the main file.

There are a couple of
considerations in setting up the main and child documents:

%%%%%%%%%%%%%%%%%%%%%%%%%%%%%%%%%%%%%%%%
\paragraph{Restrictions.}

Please note the following restrictions:
\begin{itemize}
\item
|\childdocmain| must be called with one argument \textit{main}
to ensure compatibility with earlier version of the package.
It must either be empty (|\childdocmain{}|)
or precisely match the filename of the main file in which it is specified.
See \secref{sec:detection} for further information.
\item
The filename \textit{main} must be specified without the |.tex| extension.
\item
The filename \textit{main} is case sensitive
(even in case-insensitive file systems)
due to internal string comparison.
\item
The argument \textit{main} should be fully expanded, it cannot be a macro.
\item
Subdirectories and special characters should be avoided in filenames.
\item
The command |\childdocmain{|\textit{main}|}| must be followed by a whitespace.
It should not be followed immediately by another command
or by a comment mark `|%|'.
This is because the \TeX{} parser reads the token immediately following
the argument of |\childdocmain| and puts it
at the beginning of every child section;
however, a white\-space is ignored.
\end{itemize}

%%%%%%%%%%%%%%%%%%%%%%%%%%%%%%%%%%%%%%%%
\paragraph{Content of Main File.}

It is advisable to place all content in the child files included by |\include|.
Any output contained in the main file will appear in all child documents
unless suppressed manually;
it cannot be suppressed automatically by the |\includeonly| directive
and thus should normally be avoided.
A method to include some content in the main file
by means of conditional processing is described in \secref{sec:conditional}.

%%%%%%%%%%%%%%%%%%%%%%%%%%%%%%%%%%%%%%%%
\paragraph{Page Numbering.}

When only a part of the document is compiled,
the appropriate numbering of pages
(as well as other status parameters)
is determined from the |.aux| files.
The latter contain information from previous passes.
However this information needs to propagate through
all intermediate child documents.
Therefore the page numbering in child documents may well
be inconsistent until the complete document is compiled at least once.

A useful (if unconventional) way to always ensure a consistent
page numbering is to restart the numbering in each child document
and denote the pages by `\textit{child}|.|\textit{page}'
where \textit{child} represents the chapter/section number of the child file.
This can be achieved by the command
|\numberwithin{page}{|\textit{child}|}|
of the \textsf{amsmath} package
where \textit{child} can be |chapter| or |section|
depending on the chosen structuring.
Alternatively, one can modify the macro |\thepage| appropriately
and reset the counter |page| at the start of each child file.

%%%%%%%%%%%%%%%%%%%%%%%%%%%%%%%%%%%%%%%%%%%%%%%%%%%%%%%%%%%%%%%%%%%%%%%%%%%%%%%%
\subsection{Conditional Processing}
\label{sec:conditional}

The package provides a mechanism to compile different versions
of a document. To customise the versions further some conditional processing
can come in handy to distinguish which version is being compiled.
The package provides two macros to describe the compilation context:

%%%%%%%%%%%%%%%%%%%%%%%%%%%%%%%%%%%%%%%%
\DescribeMacro{\ifchilddoc}
The conditional |\ifchilddoc| distinguishes between the compilation of
child documents and the main document:
%
\begin{center}
|\ifchilddoc |\textit{child-code}| |[|\||else |\textit{main-code}]| \||fi|
\end{center}

%%%%%%%%%%%%%%%%%%%%%%%%%%%%%%%%%%%%%%%%
\DescribeMacro{\childdocname}
\DescribeMacro{\childdocjob}
The macro |\childdocname| contains the filename (without extension)
of the main or child file being processed.
Note that |\childdocjob| will always contain the name of the main file.

%%%%%%%%%%%%%%%%%%%%%%%%%%%%%%%%%%%%%%%%
\paragraph{Title Page.}

Conditional processing can be used to include a title or banner page
in the main document when proper precautions are taken.
Importantly, the code in the main file should ensure that the page counter
(as well as other status parameters which are stored in the |.aux| files)
takes the same value after the conditional processing.
Otherwise the page numbers may take divergent values
depending on which part is compiled.

For example, a title page could be declared by:
%
\begin{center}
\begin{tabular}{l}
|\ifchilddoc\||else|\\
|\addtocounter{page}{-1}|\\
\textit{code for title page}\\
|\newpage|\\
|\||fi|
\end{tabular}
\end{center}
%
A banner page for the child documents can be generated by:
%
\begin{center}
\begin{tabular}{l}
|\ifchilddoc|\\
|\addtocounter{page}{-1}|\\
\textit{code for banner page}\\
|\newpage|\\
|\||fi|
\end{tabular}
\end{center}
%
Here one could write a message such as:
\begin{center}
|This is the part \childdocname{} of \childdocjob{}.|
\end{center}

%%%%%%%%%%%%%%%%%%%%%%%%%%%%%%%%%%%%%%%%%%%%%%%%%%%%%%%%%%%%%%%%%%%%%%%%%%%%%%%%
\subsection{Flags}
\label{sec:flags}

The package makes it easy to generate different versions
of the main or child documents.
To this end compilation flags can be defined
and assigned different default values.
They will be particularly useful in conjunction
with the forwarding mechanism described in \secref{sec:forward}.

For example, it may be useful to have a flag |\version|
which can be set to |draft| or |final|.
The document source will contain some conditional code
depending on the value of |\version|.
Suppose further, the flag should default to |final| for the main file
and to |draft| for child files
which is a natural assignment for editing the document.
This is achieved by placing the following code
in the preamble of the main document
(below the |\childdocmain| directive):
%
\begin{center}
\begin{tabular}{l}
|\ifchilddoc|\\
|\providecommand{\version}{draft}|\\
|\||else|\\
|\providecommand{\version}{final}|\\
|\||fi|
\end{tabular}
\end{center}
%
The definition by |\providecommand| makes sure
that previous definitions are not overwritten.
Further statements |\providecommand{\version}{...}|
can thus be added before the above code to override it.

For the main file, one might add a line
(between |\childdocmain| and the above block)
%
\begin{center}
|%\ifchilddoc\||else\providecommand{\version}{draft}\||fi|
\end{center}
%
which can be uncommented to produce a draft version.
Likewise one can add a line to the very top of a child file
(above the |\childdocof{|\textit{main}|}| directive)
%
\begin{center}
|%\providecommand{\version}{final}|
\end{center}
%
which can be uncommented to produce the final version of this child document.

%%%%%%%%%%%%%%%%%%%%%%%%%%%%%%%%%%%%%%%%%%%%%%%%%%%%%%%%%%%%%%%%%%%%%%%%%%%%%%%%
\subsection{Forwarding}
\label{sec:forward}

Different versions of the main or child documents
using compilation flags as described in \secref{sec:flags}
can be (permanently) stored in different files
for convenient compilation, viewing and distribution.
To this end, the package defines a command
to pass on compilation to a different file:

%%%%%%%%%%%%%%%%%%%%%%%%%%%%%%%%%%%%%%%%
\DescribeMacro{\childdocforward}
The command |\childdocforward| redirects processing to
another source file:
%
\begin{center}
\begin{tabular}{l}
|\input{childdoc.def}|\\
|\childdocforward[|\textit{main}|]{|\textit{dest}|}|\\
\end{tabular}
\end{center}
%
The argument \textit{dest} is the destination file
(without extension).
It should be the main file or one of the child files.
Note that further \textsf{childdoc} directives
such as |\childdocof| and |\childdocforward|
in the indicated file will be processed in this form.
The optional argument \textit{main}
passes on directly to the main file \textit{main}
while pretending to compile the child \textit{dest}.
This form behaves as if \textit{dest}
issues |\childdocof{|\textit{main}|}| right away,
and no further \textsf{childdoc} directives will be processed.

%%%%%%%%%%%%%%%%%%%%%%%%%%%%%%%%%%%%%%%%
\DescribeMacro{\...prefix}
In the alternative form |\childdocforwardprefix|,
%
\begin{center}
\begin{tabular}{l}
|\input{childdoc.def}|\\
|\childdocforwardprefix[|\textit{main}|]{|\textit{prefix}|}{|\textit{dest}|}|
\end{tabular}
\end{center}
%
the destination file is determined by a pattern
depending on the current file:
To make this work, the current file must be called
`{\textit{prefix}\hspace{0.2em}\textit{suffix}}'
with \textit{prefix} matching precisely the argument.
Processing is then passed on to the file
`{\textit{dest}\hspace{0.2em}\textit{suffix}}'.
Surely, the same effect is achieved by
directly specifying the
argument `{\textit{dest}\hspace{0.2em}\textit{suffix}}'
in the first form.
However, that requires to set up a different file
for each child. With the alternative form of the command
all these files can have exactly the same content
which simplifies setting them up and maintaining them.

For example, the following file |draft.tex|
with a compilation flag |\version| as described in \secref{sec:flags}
compiles the main document as a draft:
%
\begin{center}
\begin{tabular}{l}
|\def\version{draft}|\\
|\input{childdoc.def}|\\
|\childdocforward{|\textit{main}|}|
\end{tabular}
\end{center}
%
Likewise, the following files |final|\textit{nn}|.tex|
compile the final version of the child document
|child|\textit{nn}|.tex|:
%
\begin{center}
\begin{tabular}{l}
|\def\version{final}|\\
|\input{childdoc.def}|\\
|\childdocforwardprefix{final}{child}|
\end{tabular}
\end{center}
%

Note that when several versions of a main file and/or of each child file
are to be generated, it may be convenient to set up a |Makefile| or
shell script to automatise the process.

%%%%%%%%%%%%%%%%%%%%%%%%%%%%%%%%%%%%%%%%%%%%%%%%%%%%%%%%%%%%%%%%%%%%%%%%%%%%%%%%
\subsection{Command Line Processing}
\label{sec:commandline}

The effect of redirection files can also be achieved by invoking
the \LaTeX{} compiler with a more elaborate command line.
Most conveniently this should be done as part
of a shell script or a |Makefile|.

When using \textsf{childdoc} in the main file, the following
command lines effectively perform a redirection
(note that depending on the shell being used,
backslashes may have to be doubled: `|\|' $\to$ `|\\|'):
%
\begin{center}
|... -jobname "|\textit{target}|" |\\|"|[\textit{flags}]%
|\input{childdoc.def}\childdocforward[|\textit{main}|]{|\textit{dest}|}"|
\end{center}
%
Here \textit{target} is the name of the output file,
\textit{main} is the name of the main file
and \textit{dest} is the name of the main or child file to be processed
(all filenames without extensions).
The optional argument \textit{main} can be omitted
if \textit{main} matches \textit{dest}.
Optionally, compilation \textit{flags} can be defined via |\def| commands.
This command line makes the \TeX{} engine believe
it is compiling the file \textit{target}
whose content is specified as the latter parameter.
The provided code then forwards the processing to
\textit{main} or \textit{dest} as described in \secref{sec:forward}.

%%%%%%%%%%%%%%%%%%%%%%%%%%%%%%%%%%%%%%%%%%%%%%%%%%%%%%%%%%%%%%%%%%%%%%%%%%%%%%%%
\subsection{Include by Input}
\label{sec:input}

Including child documents by |\include| has some restrictions by design.
Most notably, the content of a child document always occupies
its own set of pages; pages cannot be shared between child documents.
Usually, this behaviour makes perfect sense
because each child document contain an essential part of the document.
However, in some situations it may be desirable to compose
a document from a collection of parts
without having mandatory page breaks between then.
For this case, the package
provides a mechanism to include parts
by |\input| which can also be processed individually.
However, by construction this mechanism
requires manual handling of the content to be output.

%%%%%%%%%%%%%%%%%%%%%%%%%%%%%%%%%%%%%%%%
\DescribeMacro{\ifchilddocmanual}
The main file should be prepared as usual, see \secref{sec:include}.
However, the document body must make a distinction
between processing of an individual part and of the main document, e.g.:
%
\begin{center}
\begin{tabular}{l}
|\ifchilddocmanual|\\
|\input{\childdocname}|\\
|\||else|\\
\textit{document body with }|\input{|\textit{part}|}|\\
|\||fi|
\end{tabular}
\end{center}
%
The conditional |\ifchilddocmanual| is true whenever
a part to be included by |\input| is being compiled,
and the name of the part is stored in |\childdocname|.

%%%%%%%%%%%%%%%%%%%%%%%%%%%%%%%%%%%%%%%%
\DescribeMacro{\childdocby}
Each part to be included by |\input| should start with:
%
\begin{center}
\begin{tabular}{l}
|\input{childdoc.def}|\\
|\childdocby{|\textit{main}|}|\\
\end{tabular}
\end{center}
%
The directive |\childdocby| is similar to |\childdocof|
described in \secref{sec:include},
but the subsequent selection of content must be done manually.
To that end, both |\ifchilddoc| and |\ifchilddocmanual|
will be true upon processing of a part,
and the name of the part is stored in |\childdocname|.
Note that |\jobname| will be set to the filename of the current part
so that each part receives an individual |.aux| file
that does not interfere with the |.aux| file(s) of the main document.
This behaviour can be altered by the alternative form
|\childdocby[*]{|\textit{main}|}| (with a non-empty optional argument)
which uses the |.aux| file of the main document
by setting |\jobname| to \textit{main}.

%%%%%%%%%%%%%%%%%%%%%%%%%%%%%%%%%%%%%%%%%%%%%%%%%%%%%%%%%%%%%%%%%%%%%%%%%%%%%%%%
\subsection{Driver Development}
\label{sec:driver}

The \textsf{childdoc} mechanism can also be use for the development
of definition files such as \LaTeX{} styles or classes.
This case differs from the above setup with multiple parts
included by |\include| in that no |\includeonly| should be invoked.
This can be achieved by starting the include file
(before |\ProvidesPackage|) with:
%
\begin{center}
\begin{tabular}{l}
|\input{childdoc.def}|\\
|\childdocforward{|\textit{main}|}|\\
\end{tabular}
\end{center}
%
or alternatively with:
%
\begin{center}
\begin{tabular}{l}
|\input{childdoc.def}|\\
|\childdocby{|\textit{main}|}|\\
\end{tabular}
\end{center}
%
Both forms have slightly different effects as described above.
The main file is prepared as usual, see \secref{sec:include}.

%%%%%%%%%%%%%%%%%%%%%%%%%%%%%%%%%%%%%%%%%%%%%%%%%%%%%%%%%%%%%%%%%%%%%%%%%%%%%%%%
\subsection{Legacy Detection}
\label{sec:detection}

The directive |\childdocmain| in the main file can detect
whether the complete document or merely a child is to be compiled
even without using the directive |\childdocof|.
This method is deprecated because it is less robust
and there is no compelling reason to use it;
it is merely provided for backward compatibility
and it may be removed in future versions.

If the detection mechanism is to be used,
it is mandatory to correctly specify
the filename of the main file as the argument of |\childdocmain|:
%
\begin{center}
\begin{tabular}{l}
|\input{childdoc.def}|\\
|\childdocmain{|\textit{main}|}|\\
\end{tabular}
\end{center}
%
If |\jobname| does not match the argument \textit{main} of |\childdocmain|,
it is assumed that |\jobname| points to the child file to be compiled.
When using |\childdocmain| with the main file specified as argument,
it suffices to start a child file
with just |\input{|\textit{main}|}|
without loading of the package and using |\childdocof|.
If instead all processing is done
with the appropriate \textsf{childdoc} directives,
the argument of \textit{main} of |\childdocmain| can be empty.

An alternative version of the command line processing described
in \secref{sec:commandline} using the detection mechanism reads:
%
\begin{center}
|... -jobname "|\textit{target}|" "|[\textit{flags}]%
[|\def\jobname{|\textit{dest}|}|]|\input{|\textit{main}|}"|
\end{center}

%%%%%%%%%%%%%%%%%%%%%%%%%%%%%%%%%%%%%%%%%%%%%%%%%%%%%%%%%%%%%%%%%%%%%%%%%%%%%%%%
\subsection{Manual Code}
\label{sec:manual}

In case one cannot be certain whether the definitions file |childdoc.def|
is installed on the target \TeX{} distribution
and one prefers not to ship it,
it is conceivable to paste a few relevant commands into the sources.

To that end, drop all statements |\input{childdoc.def}|
and perform the replacements as outlined below.
Instead of |\childdocmain{|\textit{main}|}| add the following code
to the top of the main file:
%
\begin{center}
\begin{tabular}{l}
|\||ifdefined\childdocname\endinput\||fi\newif\ifchilddoc|\\
|\edef\childdocname{\scantokens\expandafter{\jobname\noexpand}}|\\
|\def\childdocmain{|\textit{main}|}\||ifx\childdocmain\childdocname\||else|\\
|\childdoctrue\includeonly{\childdocname}\let\jobname\childdocmain\||fi|\\
\end{tabular}
\end{center}
%
Instead of |\childdocof{|\textit{main}|}| just include the main file
at the top of each child file:
%
\begin{center}
|\input{|\textit{main}|}|
\end{center}
%
A simple redirection |\childdocforward{|\textit{dest}|}| is achieved by:
%
\begin{center}
|\def\jobname{|\textit{dest}|}\input{\jobname}|
\end{center}
%
The redirection with prefix
|\childdocforwardprefix[|\textit{prefix}|]{|\textit{dest}|}|
is accomplished by:
%
\begin{center}
\begin{tabular}{l}
|{\edef\jobname{\scantokens\expandafter{\jobname\noexpand}}|\\
|\def\redirectjob |\textit{prefix}|#1~~~{\gdef\jobname{|\textit{dest}|#1}}|\\
|\expandafter\redirectjob\jobname~~~}\input{\jobname}|
\end{tabular}
\end{center}

In an alternative approach,
child documents can be compiled by a specific command line
without additional code or specific definitions:
%
\begin{center}
|... -jobname "|\textit{target}|" "|[\textit{flags}]%
|\includeonly{|\textit{dest}|}\input{|\textit{main}|}"|
\end{center}
%

%%%%%%%%%%%%%%%%%%%%%%%%%%%%%%%%%%%%%%%%%%%%%%%%%%%%%%%%%%%%%%%%%%%%%%%%%%%%%%%%
%%%%%%%%%%%%%%%%%%%%%%%%%%%%%%%%%%%%%%%%%%%%%%%%%%%%%%%%%%%%%%%%%%%%%%%%%%%%%%%%
\section{Information}

%%%%%%%%%%%%%%%%%%%%%%%%%%%%%%%%%%%%%%%%%%%%%%%%%%%%%%%%%%%%%%%%%%%%%%%%%%%%%%%%
\subsection{Copyright}

Copyright \copyright{} 2017--2018 Niklas Beisert

This work may be distributed and/or modified under the
conditions of the \LaTeX{} Project Public License, either version 1.3
of this license or (at your option) any later version.
The latest version of this license is in
  \url{http://www.latex-project.org/lppl.txt}
and version 1.3 or later is part of all distributions of \LaTeX{}
version 2005/12/01 or later.

This work has the LPPL maintenance status `maintained'.

The Current Maintainer of this work is Niklas Beisert.

This work consists of the files |README.txt|, |childdoc.ins| and |childdoc.dtx|
as well as the derived files |childdoc.def|, |cdocsamp.tex|
with |cdocsch1.tex|, |cdocsch2.tex|, |cdocspt3.tex|, |cdocspt4.tex|,
|cdocsdrf.tex|, |cdocsfn1.tex|, |cdocsfn2.tex|
as well as |childdoc.pdf|.

%%%%%%%%%%%%%%%%%%%%%%%%%%%%%%%%%%%%%%%%%%%%%%%%%%%%%%%%%%%%%%%%%%%%%%%%%%%%%%%%
\subsection{Files and Installation}

The package consists of the files:
%
\begin{center}
\begin{tabular}{ll}
    |README.txt|   & readme file \\
    |childdoc.ins| & installation file \\
    |childdoc.dtx| & source file \\
    |childdoc.def| & definition file \\
    |cdocsamp.tex| & sample main file \\
    |cdocsch1.tex| & sample include file \\
    |cdocsch2.tex| & sample include file \\
    |cdocspt3.tex| & sample part file \\
    |cdocspt4.tex| & sample part file \\
    |cdocsdrf.tex| & sample redirection file \\
    |cdocsfn1.tex| & sample redirection file \\
    |cdocsfn2.tex| & sample redirection file \\
    |childdoc.pdf| & manual
\end{tabular}
\end{center}
%
The distribution consists of the files
|README.txt|, |childdoc.ins| and |childdoc.dtx|.
%
\begin{itemize}
\item
Run (pdf)\LaTeX{} on |childdoc.dtx|
to compile the manual |childdoc.pdf| (this file).
\item
Run \LaTeX{} on |childdoc.ins| to create the definitions file |childdoc.def|
and the sample |cdocsamp.tex| with include files
|cdocsch1.tex|, |cdocsch2.tex|, |cdocspt3.tex|, |cdocspt4.tex|,
|cdocsdrf.tex|, |cdocsfn1.tex|, |cdocsfn2.tex|.
Then copy the file |childdoc.def| to an appropriate directory of your \LaTeX{}
distribution, e.g.\ \textit{texmf-root}|/tex/latex/childdoc|.
\end{itemize}

%%%%%%%%%%%%%%%%%%%%%%%%%%%%%%%%%%%%%%%%%%%%%%%%%%%%%%%%%%%%%%%%%%%%%%%%%%%%%%%%
\subsection{Related CTAN Packages}

There are several other packages which offer a similar functionality:
%
\begin{itemize}
\item
The packages
\href{http://ctan.org/pkg/docmute}{\textsf{docmute}},
\href{http://ctan.org/pkg/includex}{\textsf{includex}} and
\href{http://ctan.org/pkg/standalone}{\textsf{standalone}}
provide commands to include only the document body of
a child file thus allowing both files to be compiled individually.
\item
The packages \href{http://ctan.org/pkg/subdocs}{\textsf{subdocs}}
and \href{http://ctan.org/pkg/subfiles}{\textsf{subfiles}}
provide structures in which the main and child documents can be
encapsulated and allowing them to be compiled individually.
The inclusion mechanism is different from the conventional |\include|.
\item
The package \href{http://ctan.org/pkg/combine}{\textsf{combine}}
is an elaborate solution to combine several documents into one.
\end{itemize}
%
See also the CTAN topic \href{http://ctan.org/topic/subdocs}{\textsf{subdocs}}
for further related packages.
The present package differs from the above solutions in that
a document structure constructed with the conventional |\include| mechanism
just needs two extra commands at the top of every file
such that all constituent files can be compiled individually.

%%%%%%%%%%%%%%%%%%%%%%%%%%%%%%%%%%%%%%%%%%%%%%%%%%%%%%%%%%%%%%%%%%%%%%%%%%%%%%%%
%\subsection{Feature Suggestions}
%
%The following is a list of features which may be useful for future
%versions of this package:
%%
%\begin{itemize}
%\item
%\ldots
%\end{itemize}

%%%%%%%%%%%%%%%%%%%%%%%%%%%%%%%%%%%%%%%%%%%%%%%%%%%%%%%%%%%%%%%%%%%%%%%%%%%%%%%%
\subsection{Revision History}

%%%%%%%%%%%%%%%%%%%%%%%%%%%%%%%%%%%%%%%%
\paragraph{v2.0:} 2018/12/30

\begin{itemize}
\item
immediate forward processing
\item
added |\childdocby| mechanism
\item
manual restructured
\end{itemize}

%%%%%%%%%%%%%%%%%%%%%%%%%%%%%%%%%%%%%%%%
\paragraph{v1.6:} 2018/01/17

\begin{itemize}
\item
application for development of include files
\item
corrections to manual
\end{itemize}

%%%%%%%%%%%%%%%%%%%%%%%%%%%%%%%%%%%%%%%%
\paragraph{v1.5:} 2017/05/21

\begin{itemize}
\item
more complete structuring introduced
\item
|\childdocof| introduced
\item
|\childdoc| renamed to |\childdocmain|
\item
|\childredirect| renamed to |\childdocforward| and |\childdocforwardprefix|
and functionality expanded
\end{itemize}

%%%%%%%%%%%%%%%%%%%%%%%%%%%%%%%%%%%%%%%%
\paragraph{v1.0:} 2017/04/27

\begin{itemize}
\item
manual and install package
\item
first version published on CTAN
\end{itemize}

%%%%%%%%%%%%%%%%%%%%%%%%%%%%%%%%%%%%%%%%
\paragraph{v0.6:} 2017/04/26

\begin{itemize}
\item
redirection mechanism added
\end{itemize}

%%%%%%%%%%%%%%%%%%%%%%%%%%%%%%%%%%%%%%%%
\paragraph{v0.5:} 2017/04/26

\begin{itemize}
\item
functionality in definition file
\end{itemize}


%%%%%%%%%%%%%%%%%%%%%%%%%%%%%%%%%%%%%%%%%%%%%%%%%%%%%%%%%%%%%%%%%%%%%%%%%%%%%%%%
%%%%%%%%%%%%%%%%%%%%%%%%%%%%%%%%%%%%%%%%%%%%%%%%%%%%%%%%%%%%%%%%%%%%%%%%%%%%%%%%
%%%%%%%%%%%%%%%%%%%%%%%%%%%%%%%%%%%%%%%%%%%%%%%%%%%%%%%%%%%%%%%%%%%%%%%%%%%%%%%%
\appendix

\settowidth\MacroIndent{\rmfamily\scriptsize 000\ }

 \DocInput{childdoc.dtx}

\end{document}
%</driver>
% \fi
%
% %%%%%%%%%%%%%%%%%%%%%%%%%%%%%%%%%%%%%%%%%%%%%%%%%%%%%%%%%%%%%%%%%%%%%%%%%%%%%%
% %%%%%%%%%%%%%%%%%%%%%%%%%%%%%%%%%%%%%%%%%%%%%%%%%%%%%%%%%%%%%%%%%%%%%%%%%%%%%%
% \section{Sample}
%\iffalse
%<*samplemain>
%\fi
%
% The following presents a sample document
% with two chapters, two parts, a title page,
% a compile flag as well as three forwarding files to set the flag.
% It consists of eight |.tex| files:
% \begin{center}
% \begin{tabular}{ll}
% |cdocsamp.tex|&main file\\
% |cdocsch1.tex|&include file for chapter 1\\
% |cdocsch2.tex|&include file for chapter 2\\
% |cdocspt3.tex|&include file for part 3\\
% |cdocspt4.tex|&include file for part 4\\
% |cdocsdrf.tex|&forwarding file for main file in draft mode\\
% |cdocsfi1.tex|&forwarding file for final version of chapter 1\\
% |cdocsfi2.tex|&forwarding file for final version of chapter 2\\
% \end{tabular}
% \end{center}
% Each of the eight files can be compiled directly by the \LaTeX{} compiler.
%
% %%%%%%%%%%%%%%%%%%%%%%%%%%%%%%%%%%%%%%
% \paragraph{Main File.}
%
% The main file is called |cdocsamp.tex|.
%
% Load the \textsf{childdoc} definitions and
% declare the filename for the main document:
%    \begin{macrocode}
\input{childdoc.def}
\childdocmain{}
%    \end{macrocode}

% Optional override for |\version| flag:
%    \begin{macrocode}
%%\ifchilddoc\else\providecommand{\version}{draft}\fi
%    \end{macrocode}

% Define the default values for the |\version| flag
% (|final| for the main file and |draft| for childs):
%    \begin{macrocode}
\ifchilddoc
\providecommand{\version}{draft}
\else
\providecommand{\version}{final}
\fi
%    \end{macrocode}

% Load the standard document class:
%    \begin{macrocode}
\documentclass[12pt]{article}
%    \end{macrocode}

% Start the document body:
%    \begin{macrocode}
\begin{document}
%    \end{macrocode}

% Declare a title page.
% Print title, part of document being processed and version flag:
%    \begin{macrocode}
\addtocounter{page}{-1}
\begin{center}
{\LARGE\bfseries{}childdoc example\par}
\vspace{1cm}
\ifchilddoc
\ifchilddocmanual part\else chapter\fi:
`\childdocname' of `\childdocjob'\par
\else
main document: `\childdocjob'\par
\fi
version: \version\par
\end{center}
\newpage
%    \end{macrocode}

% Manually include selected file,
% otherwise process as usual:
%    \begin{macrocode}
\ifchilddocmanual
\section*{part `\childdocname'}
\input{\childdocname}
\else
%    \end{macrocode}

% Include the two chapters:
%    \begin{macrocode}
\include{cdocsch1}
\include{cdocsch2}
%    \end{macrocode}

% Include the two parts unless only chapters should be displayed:
%    \begin{macrocode}
\ifchilddoc\else
\section{part three}
\input{cdocspt3}
\section{part four}
\input{cdocspt4}
\fi
%    \end{macrocode}

% Process as usual until here:
%    \begin{macrocode}
\fi
%    \end{macrocode}

% End of document body:
%    \begin{macrocode}
\end{document}
%    \end{macrocode}
%\iffalse
%</samplemain>
%\fi
%
% %%%%%%%%%%%%%%%%%%%%%%%%%%%%%%%%%%%%%%
% \paragraph{Chapter Include Files.}
%
% The include files are called |cdocsch1.tex| and |cdocsch2.tex|.
%
%\iffalse
%<*samplechap1|samplechap2>
%\fi

% Optional override for |\version| flag:
%    \begin{macrocode}
%%\providecommand{\version}{final}
%    \end{macrocode}

% Include the main document:
%    \begin{macrocode}
\input{childdoc.def}
\childdocof{cdocsamp}
%    \end{macrocode}

%\iffalse
%</samplechap1|samplechap2>
%\fi
%
%\iffalse
%<*samplechap1>
%\fi
% Some text for chapter 1:
%    \begin{macrocode}
\section{one}
some text in chapter one
%    \end{macrocode}

%\iffalse
%</samplechap1>
%\fi
% Some text for chapter 2:
%\iffalse
%<*samplechap2>
%\fi
%    \begin{macrocode}
\section{two}
more text in chapter two
%    \end{macrocode}

%\iffalse
%</samplechap2>
%\fi
%
% %%%%%%%%%%%%%%%%%%%%%%%%%%%%%%%%%%%%%%
% \paragraph{Part Include Files.}
%
% The include files are called |cdocspt3.tex| and |cdocspt4.tex|.
%
%\iffalse
%<*samplepart3|samplepart4>
%\fi

% Optional override for |\version| flag:
%    \begin{macrocode}
%%\providecommand{\version}{final}
%    \end{macrocode}

% Include the main document:
%    \begin{macrocode}
\input{childdoc.def}
\childdocby{cdocsamp}
%    \end{macrocode}

%\iffalse
%</samplepart3|samplepart4>
%\fi
%
%\iffalse
%<*samplepart3>
%\fi
% Some text for part 3:
%    \begin{macrocode}
some text in part three
%    \end{macrocode}

%\iffalse
%</samplepart3>
%\fi
% Some text for part 4:
%\iffalse
%<*samplepart4>
%\fi
%    \begin{macrocode}
more text in part four
%    \end{macrocode}

%\iffalse
%</samplepart4>
%\fi
%
% %%%%%%%%%%%%%%%%%%%%%%%%%%%%%%%%%%%%%%
% \paragraph{Forwarding for a Complete Draft.}
%
% The following forwarding file |cdocsdrf.tex|
% compiles the main document in draft mode:
%\iffalse
%<*sampledraft>
%\fi
%    \begin{macrocode}
\def\version{draft}
\input{childdoc.def}
\childdocforward{cdocsamp}
%    \end{macrocode}

%\iffalse
%</sampledraft>
%\fi
%
% %%%%%%%%%%%%%%%%%%%%%%%%%%%%%%%%%%%%%%
% \paragraph{Forwarding for Final Version of the Chapters.}
%
% The following forwarding files |cdocsfn1.tex| and |cdocsfn2.tex|
% (with identical content)
% compile the final versions of the child documents
% |cdocsch1.tex| and |cdocsch2.tex|, respectively:
%\iffalse
%<*samplefinal>
%\fi
%    \begin{macrocode}
\def\version{final}
\input{childdoc.def}
\childdocforwardprefix[cdocsamp]{cdocsfn}{cdocsch}
%    \end{macrocode}

%\iffalse
%</samplefinal>
%\fi
%
% %%%%%%%%%%%%%%%%%%%%%%%%%%%%%%%%%%%%%%
% \paragraph{Command Line Processing.}
%
% The following three command lines generate the output files
% |cdocscld|, |cdocscl1| and |cdocscl2|
% which should be identical to
% |cdocsdrf|, |cdocsch1| and |cdocsfn2|, respectively:
% \begin{center}
% \begin{tabular}{l}
% |latex -jobname cdocscld \|\\
% |  "\def\version{draft}\input{childdoc.def}\childdocforward{cdocsamp}"|\\
% |latex -jobname cdocscl1 \|\\
% |  "\input{childdoc.def}\childdocforward[cdocsamp]{cdocsch1}"|\\
% |latex -jobname cdocscl2 \|\\
% |  "\def\version{final}\input{childdoc.def}\childdocforward{cdocsch2}"|
% \end{tabular}
% \end{center}
% Note that the trailing backslash on each first line
% merely continues the input to the second line
% (for convenient cut ant paste).
% Furthermore, the command |latex| can be replaced by any
% of its alternative versions such as |pdflatex|.
%
% %%%%%%%%%%%%%%%%%%%%%%%%%%%%%%%%%%%%%%%%%%%%%%%%%%%%%%%%%%%%%%%%%%%%%%%%%%%%%%
% %%%%%%%%%%%%%%%%%%%%%%%%%%%%%%%%%%%%%%%%%%%%%%%%%%%%%%%%%%%%%%%%%%%%%%%%%%%%%%
% \section{Implementation}
%\iffalse
%<*package>
%\fi
%
% This section describes the definitions file |childdoc.def|.

% The definitions cannot be loaded using |\usepackage| or |\RequirePackage|
% which has a mechanism to prevent loading a style file more than once.
% When loading the definitions by means of |\input|
% multiple instances have to be prevented manually:
%\iffalse
%This code needs to be before the `\ProvidesFile' directive
%which is defined at the beginning of this file.
%Therefore it is also placed there and commented out here.
%</package>
%<*discard>
%\fi
%    \begin{macrocode}
\ifdefined\childdocmain\endinput\fi
%    \end{macrocode}
%\iffalse
%</discard>
%<*package>
%\fi
%
% \macro{\ifchilddoc}
% \macro{\ifchilddocmanual}
% The conditional |\ifchilddoc| tells whether a
% child (true) or main (false) document is being compiled.
% The conditional |\ifchilddocmanual| tells whether
% the |\includeonly| mechanism is used (false) or
% the selection of child files must be performed manually (true).
% The definitions initialise to false:
%    \begin{macrocode}
\newif\ifchilddoc
\newif\ifchilddocmanual
%    \end{macrocode}

% \macro{\childdocname}
% \macro{\childdocjob}
% The macro |\childdocname| stores the name of the main document
% to be compiled. The macro |\childdocjob| stores the name of
% the document on which the \LaTeX{} compiler was originally invoked.
% The content of |\jobname| cannot be compared
% to filenames specified in the source due to different catcodes.
% The following code rescans |\jobname|, stores the result
% in |\childdocname| and saves a copy in |\childdocjob|:
%    \begin{macrocode}
\edef\childdocname{\scantokens\expandafter{\jobname\noexpand}}
\let\childdocjob\childdocname
%    \end{macrocode}

% \macro{\childdocdisable}
% The macro |\childdocdisable| prevents the main file
% from being processed more than once.
% At this stage, the main document command |\childdocmain|
% is assumed to be called once again where it should do nothing.
% Any subsequent call to it should prevent
% a secondary processing of the main document
% It overwrites the forwarding commands
% |\childdocof| and |\childdocforward|
% with empty macros to prevent further inclusions of the main document:
%    \begin{macrocode}
\newcommand{\childdocdisable}
{
  \renewcommand{\childdocmain}[1]{\renewcommand{\childdocmain}[1]{\endinput}}
  \renewcommand{\childdocof}[1]{}
  \renewcommand{\childdocby}[2][]{}
  \renewcommand{\childdocforward}[2][]{}
  \renewcommand{\childdocdisable}{}
}
%    \end{macrocode}

% \macro{\childdocmain}
% The macro |\childdocmain| is to be called at the top of the main file
% with nothing or the main filename (without extension) as argument.
% First, it breaks loops.
% If the argument is not empty and does not match |\childdocname|
% (which is set by the first inclusion of |childdoc.def|),
% |\ifchilddoc| is set to true, |\includeonly| is applied to the child file
% and |\jobname| is set to the main file
% (for proper handling of |.aux| files):
%    \begin{macrocode}
\newcommand{\childdocmain}[1]
{
  \childdocdisable\childdocmain{}
  \if?#1?\else
    \begingroup
      \def\childdoctmp{#1}
      \ifx\childdoctmp\childdocname
        \def\childdoctmp{}
      \else
        \def\childdoctmp
        {
          \childdoctrue
          \includeonly{\childdocname}
          \def\childdocjob{#1}
          \def\jobname{#1}
        }
      \fi
      \expandafter
    \endgroup
    \childdoctmp
  \fi
}
%    \end{macrocode}

% \macro{\childdocof}
% The command |\childdocof| redirects
% compilation to the main file |#1|.
%    \begin{macrocode}
\newcommand{\childdocof}[1]
{
  \childdocdisable
  \childdoctrue
  \includeonly{\childdocname}
  \def\jobname{#1}
  \def\childdocjob{#1}
  \input{#1}
}
%    \end{macrocode}

% \macro{\childdocby}
% The command |\childdocby| ....
%    \begin{macrocode}
\newcommand{\childdocby}[2][]
{
  \childdocdisable
  \childdoctrue
  \childdocmanualtrue
  \if?#1?\else
    \def\jobname{#2}
  \fi
  \def\childdocjob{#2}
  \input{#2}
  \endinput
}
%    \end{macrocode}

% \macro{\childdocforward}
% The command |\childdocforward| redirects
% compilation to the main file or
% (if the optional argument is given) a child file.
% Parameters are set as if the main file
% or a child file starting with |\childdocof| was compiled.
% Then compilation is handed over to the main file:
%    \begin{macrocode}
\newcommand{\childdocforward}[2][]
{
  \begingroup
    \if?#1?
      \def\childdoctmp
      {
        \def\childdocname{#2}
        \def\childdocjob{#2}
        \def\jobname{#2}
        \input{#2}
        \endinput
      }
    \else
      \def\childdoctmp
      {
        \childdocdisable
        \def\childdocname{#2}
        \childdoctrue
        \includeonly{#2}
        \def\childdocjob{#1}
        \def\jobname{#1}
        \input{#1}
        \endinput
      }
    \fi
    \expandafter
  \endgroup
  \childdoctmp
}
%    \end{macrocode}

% \macro{\childdocforwardprefix}
% The command |\childdocforwardprefix| redirects
% compilation to the main or a child file by means of a pattern.
% The prefix |#1| in the current filename is replaced by |#2|
% and the suffix of the current filename is kept
% (it is assumed that the filename does not contain the substring `|~~~|'
% which is used as a delimiter).
% Compilation is handed over to the new file by |\childdocforward|:
%    \begin{macrocode}
\newcommand{\childdocforwardprefix}[3][]
{
  \begingroup
    \def\childdocextract #2##1~~~{\def\childdoctmp{\childdocforward[#1]{#3##1}}}
    \expandafter\childdocextract\childdocname~~~
    \expandafter
  \endgroup
  \childdoctmp
}
%    \end{macrocode}

% \macro{\childdoc}
% The deprecated macro |\childdoc| is a legacy version of |\childdocmain|:
%    \begin{macrocode}
\newcommand{\childdoc}{\childdocmain}
%    \end{macrocode}

% \macro{\childdocredirect}
% The deprecated macro |\childdocredirect| is a legacy version
% of |\childdocforward| and |\childdocforwardprefix|:
%    \begin{macrocode}
\newcommand{\childdocredirect}[2][]
{
  \begingroup
    \if?#1?
      \def\childdoctmp{\childdocforward{#2}}
    \else
      \def\childdoctmp{\childdocforwardprefix{#1}{#2}}
    \fi
    \expandafter
  \endgroup
  \childdoctmp
}
%    \end{macrocode}

%\iffalse
%</package>
%\fi
%
\endinput
|\\
|\childdocforward{|\textit{main}|}|
\end{tabular}
\end{center}
%
Likewise, the following files |final|\textit{nn}|.tex|
compile the final version of the child document
|child|\textit{nn}|.tex|:
%
\begin{center}
\begin{tabular}{l}
|\def\version{final}|\\
|% \iffalse
%
% childdoc.dtx Copyright (C) 2017-2018 Niklas Beisert
%
% This work may be distributed and/or modified under the
% conditions of the LaTeX Project Public License, either version 1.3
% of this license or (at your option) any later version.
% The latest version of this license is in
%   http://www.latex-project.org/lppl.txt
% and version 1.3 or later is part of all distributions of LaTeX
% version 2005/12/01 or later.
%
% This work has the LPPL maintenance status `maintained'.
%
% The Current Maintainer of this work is Niklas Beisert.
%
% This work consists of the files childdoc.dtx and childdoc.ins
% and the derived files childdoc.def and cdocsamp.tex with
% cdocsch1.tex, cdocsch2.tex, cdocsdrf.tex, cdocsfn1.tex, cdocsfn2.tex.
%
%<package>\ifdefined\childdocmain\endinput\fi
%<package>\ProvidesFile{childdoc.def}[2018/12/30 v2.0 child document driver]
%<samplemain>\ProvidesFile{cdocsamp.tex}[2018/12/30 v2.0 sample for childdoc]
%<*driver>
%\ProvidesFile{childdoc.drv}[2018/12/30 v2.0 childdoc reference manual file]
\PassOptionsToClass{10pt,a4paper}{article}
\documentclass{ltxdoc}

\usepackage[margin=35mm]{geometry}
\usepackage{hyperref}
\usepackage{hyperxmp}
\usepackage[usenames]{color}

\hypersetup{colorlinks=true}
\hypersetup{pdfstartview=FitH}
\hypersetup{pdfpagemode=UseNone}
\hypersetup{pdfsource={}}
\hypersetup{pdflang={en-UK}}
\hypersetup{pdfcopyright={Copyright 2017-2018 Niklas Beisert.
  This work may be distributed and/or modified under the
  conditions of the LaTeX Project Public License, either version 1.3
  of this license or (at your option) any later version.}}
\hypersetup{pdflicenseurl={http://www.latex-project.org/lppl.txt}}
\hypersetup{pdfcontactaddress={ETH Zurich, ITP, HIT K,
  Wolfgang-Pauli-Strasse 27}}
\hypersetup{pdfcontactpostcode={8093}}
\hypersetup{pdfcontactcity={Zurich}}
\hypersetup{pdfcontactcountry={Switzerland}}
\hypersetup{pdfcontactemail={nbeisert@itp.phys.ethz.ch}}
\hypersetup{pdfcontacturl={http://people.phys.ethz.ch/\xmptilde nbeisert/}}

\newcommand{\secref}[1]{\hyperref[#1]{section \ref*{#1}}}

\parskip1ex
\parindent0pt
\let\olditemize\itemize
\def\itemize{\olditemize\parskip0pt}

\begin{document}

\title{The \textsf{childdoc} Package}
\hypersetup{pdftitle={The childdoc Package}}
\author{Niklas Beisert\\[2ex]
  Institut f\"ur Theoretische Physik\\
  Eidgen\"ossische Technische Hochschule Z\"urich\\
  Wolfgang-Pauli-Strasse 27, 8093 Z\"urich, Switzerland\\[1ex]
  \href{mailto:nbeisert@itp.phys.ethz.ch}
  {\texttt{nbeisert@itp.phys.ethz.ch}}}
\hypersetup{pdfauthor={Niklas Beisert}}
\hypersetup{pdfsubject={Manual for the LaTeX2e Package childdoc}}
\date{30 December 2018, \textsf{v2.0}}
\maketitle

\begin{abstract}\noindent
\textsf{childdoc} is a \LaTeXe{} package
that enables the direct compilation
of document sections included by |\include|
to individual files.
\end{abstract}

\begingroup
\parskip0ex
\tableofcontents
\endgroup

%%%%%%%%%%%%%%%%%%%%%%%%%%%%%%%%%%%%%%%%%%%%%%%%%%%%%%%%%%%%%%%%%%%%%%%%%%%%%%%%
%%%%%%%%%%%%%%%%%%%%%%%%%%%%%%%%%%%%%%%%%%%%%%%%%%%%%%%%%%%%%%%%%%%%%%%%%%%%%%%%
\section{Introduction}

\LaTeX{} provides a mechanism to structure a large document (such as a book)
into a main file and several child files (containing the chapters)
using the |\include| command.
This mechanism is beneficial for documents
which span hundreds of pages in order to
make the source file(s) more manageable.
Moreover, compilation can be restricted to
selected child files by means of the |\includeonly| command.
The latter feature can be used to reduce the compilation time while editing
(this was significantly more useful in the earlier days of \LaTeX{})
or to generate a smaller document which is easier to navigate.
Another application of |\includeonly| is to generate
documents consisting of selected parts of the complete document.

However, there are a few drawbacks of the plain |\include| mechanism:
\begin{itemize}
\item
The child files cannot be compiled on their own,
they can only be compiled via the main file.
A naive editing environment
(such as a text editor with an option
to have the current file processed by \LaTeX)
may require one to switch to the main file before compiling;
attempting to compile the child file produces errors.
\item
The main file must be modified (each time)
to adjust the |\includeonly| command
to the present needs. This easily leaves the main file in a messy state.
\item
The generated document will always carry the filename
of the main document. This is inconvenient if
several child files are to be compiled and
to be kept for distribution.
\end{itemize}

The present package provides a simple interface
to make child files individually compilable by \LaTeX{}.
Compiling a child file then has the same effect as compiling
the main file with an |\includeonly| command
to select the appropriate child.
Moreover the generated document will carry the name of the child
rather than the main file.
This resolves all three above issues.

This feature is meant to make the editing of books,
thesis documents and lecture notes somewhat more convenient.
However, the package can also be used efficiently for
composing a series of documents (such as exercise sheets)
which are typically distributed individually.
It then assists the author in generating the individual documents
(potentially in different versions)
as well as a document containing the collected series.
Another application is in developing style files
or other kinds of included material
where compilation of the style file could redirect
to a sample or test file.

%%%%%%%%%%%%%%%%%%%%%%%%%%%%%%%%%%%%%%%%%%%%%%%%%%%%%%%%%%%%%%%%%%%%%%%%%%%%%%%%
%%%%%%%%%%%%%%%%%%%%%%%%%%%%%%%%%%%%%%%%%%%%%%%%%%%%%%%%%%%%%%%%%%%%%%%%%%%%%%%%
\section{Usage}

First of all, the package \textsf{childdoc} is \emph{not} a standard
\LaTeXe{} |.sty| style file! Therefore it needs to be invoked in
a non-standard way.

%%%%%%%%%%%%%%%%%%%%%%%%%%%%%%%%%%%%%%%%%%%%%%%%%%%%%%%%%%%%%%%%%%%%%%%%%%%%%%%%
\subsection{Included Files}
\label{sec:include}

%%%%%%%%%%%%%%%%%%%%%%%%%%%%%%%%%%%%%%%%
\DescribeMacro{\childdocmain}
To use the package, add the commands
\begin{center}
\begin{tabular}{l}
|\input{childdoc.def}|\\
|\childdocmain{}|\\
\end{tabular}
\end{center}
at the very top of the main \LaTeX{} file,
in particular \emph{before} the |\documentclass| statement!
The argument of |\childdocmain| should be left empty
(but it must be present).

%%%%%%%%%%%%%%%%%%%%%%%%%%%%%%%%%%%%%%%%
\DescribeMacro{\childdocof}
Furthermore, add the commands
\begin{center}
\begin{tabular}{l}
|\input{childdoc.def}|\\
|\childdocof{|\textit{main}|}|\\
\end{tabular}
\end{center}
at the top of every child file \textit{child}
which is included by |\include{|\textit{child}|}|
from within the main file
(or at least for those files to be compiled individually).
The argument \textit{main} must be the filename of the main file.

There are a couple of
considerations in setting up the main and child documents:

%%%%%%%%%%%%%%%%%%%%%%%%%%%%%%%%%%%%%%%%
\paragraph{Restrictions.}

Please note the following restrictions:
\begin{itemize}
\item
|\childdocmain| must be called with one argument \textit{main}
to ensure compatibility with earlier version of the package.
It must either be empty (|\childdocmain{}|)
or precisely match the filename of the main file in which it is specified.
See \secref{sec:detection} for further information.
\item
The filename \textit{main} must be specified without the |.tex| extension.
\item
The filename \textit{main} is case sensitive
(even in case-insensitive file systems)
due to internal string comparison.
\item
The argument \textit{main} should be fully expanded, it cannot be a macro.
\item
Subdirectories and special characters should be avoided in filenames.
\item
The command |\childdocmain{|\textit{main}|}| must be followed by a whitespace.
It should not be followed immediately by another command
or by a comment mark `|%|'.
This is because the \TeX{} parser reads the token immediately following
the argument of |\childdocmain| and puts it
at the beginning of every child section;
however, a white\-space is ignored.
\end{itemize}

%%%%%%%%%%%%%%%%%%%%%%%%%%%%%%%%%%%%%%%%
\paragraph{Content of Main File.}

It is advisable to place all content in the child files included by |\include|.
Any output contained in the main file will appear in all child documents
unless suppressed manually;
it cannot be suppressed automatically by the |\includeonly| directive
and thus should normally be avoided.
A method to include some content in the main file
by means of conditional processing is described in \secref{sec:conditional}.

%%%%%%%%%%%%%%%%%%%%%%%%%%%%%%%%%%%%%%%%
\paragraph{Page Numbering.}

When only a part of the document is compiled,
the appropriate numbering of pages
(as well as other status parameters)
is determined from the |.aux| files.
The latter contain information from previous passes.
However this information needs to propagate through
all intermediate child documents.
Therefore the page numbering in child documents may well
be inconsistent until the complete document is compiled at least once.

A useful (if unconventional) way to always ensure a consistent
page numbering is to restart the numbering in each child document
and denote the pages by `\textit{child}|.|\textit{page}'
where \textit{child} represents the chapter/section number of the child file.
This can be achieved by the command
|\numberwithin{page}{|\textit{child}|}|
of the \textsf{amsmath} package
where \textit{child} can be |chapter| or |section|
depending on the chosen structuring.
Alternatively, one can modify the macro |\thepage| appropriately
and reset the counter |page| at the start of each child file.

%%%%%%%%%%%%%%%%%%%%%%%%%%%%%%%%%%%%%%%%%%%%%%%%%%%%%%%%%%%%%%%%%%%%%%%%%%%%%%%%
\subsection{Conditional Processing}
\label{sec:conditional}

The package provides a mechanism to compile different versions
of a document. To customise the versions further some conditional processing
can come in handy to distinguish which version is being compiled.
The package provides two macros to describe the compilation context:

%%%%%%%%%%%%%%%%%%%%%%%%%%%%%%%%%%%%%%%%
\DescribeMacro{\ifchilddoc}
The conditional |\ifchilddoc| distinguishes between the compilation of
child documents and the main document:
%
\begin{center}
|\ifchilddoc |\textit{child-code}| |[|\||else |\textit{main-code}]| \||fi|
\end{center}

%%%%%%%%%%%%%%%%%%%%%%%%%%%%%%%%%%%%%%%%
\DescribeMacro{\childdocname}
\DescribeMacro{\childdocjob}
The macro |\childdocname| contains the filename (without extension)
of the main or child file being processed.
Note that |\childdocjob| will always contain the name of the main file.

%%%%%%%%%%%%%%%%%%%%%%%%%%%%%%%%%%%%%%%%
\paragraph{Title Page.}

Conditional processing can be used to include a title or banner page
in the main document when proper precautions are taken.
Importantly, the code in the main file should ensure that the page counter
(as well as other status parameters which are stored in the |.aux| files)
takes the same value after the conditional processing.
Otherwise the page numbers may take divergent values
depending on which part is compiled.

For example, a title page could be declared by:
%
\begin{center}
\begin{tabular}{l}
|\ifchilddoc\||else|\\
|\addtocounter{page}{-1}|\\
\textit{code for title page}\\
|\newpage|\\
|\||fi|
\end{tabular}
\end{center}
%
A banner page for the child documents can be generated by:
%
\begin{center}
\begin{tabular}{l}
|\ifchilddoc|\\
|\addtocounter{page}{-1}|\\
\textit{code for banner page}\\
|\newpage|\\
|\||fi|
\end{tabular}
\end{center}
%
Here one could write a message such as:
\begin{center}
|This is the part \childdocname{} of \childdocjob{}.|
\end{center}

%%%%%%%%%%%%%%%%%%%%%%%%%%%%%%%%%%%%%%%%%%%%%%%%%%%%%%%%%%%%%%%%%%%%%%%%%%%%%%%%
\subsection{Flags}
\label{sec:flags}

The package makes it easy to generate different versions
of the main or child documents.
To this end compilation flags can be defined
and assigned different default values.
They will be particularly useful in conjunction
with the forwarding mechanism described in \secref{sec:forward}.

For example, it may be useful to have a flag |\version|
which can be set to |draft| or |final|.
The document source will contain some conditional code
depending on the value of |\version|.
Suppose further, the flag should default to |final| for the main file
and to |draft| for child files
which is a natural assignment for editing the document.
This is achieved by placing the following code
in the preamble of the main document
(below the |\childdocmain| directive):
%
\begin{center}
\begin{tabular}{l}
|\ifchilddoc|\\
|\providecommand{\version}{draft}|\\
|\||else|\\
|\providecommand{\version}{final}|\\
|\||fi|
\end{tabular}
\end{center}
%
The definition by |\providecommand| makes sure
that previous definitions are not overwritten.
Further statements |\providecommand{\version}{...}|
can thus be added before the above code to override it.

For the main file, one might add a line
(between |\childdocmain| and the above block)
%
\begin{center}
|%\ifchilddoc\||else\providecommand{\version}{draft}\||fi|
\end{center}
%
which can be uncommented to produce a draft version.
Likewise one can add a line to the very top of a child file
(above the |\childdocof{|\textit{main}|}| directive)
%
\begin{center}
|%\providecommand{\version}{final}|
\end{center}
%
which can be uncommented to produce the final version of this child document.

%%%%%%%%%%%%%%%%%%%%%%%%%%%%%%%%%%%%%%%%%%%%%%%%%%%%%%%%%%%%%%%%%%%%%%%%%%%%%%%%
\subsection{Forwarding}
\label{sec:forward}

Different versions of the main or child documents
using compilation flags as described in \secref{sec:flags}
can be (permanently) stored in different files
for convenient compilation, viewing and distribution.
To this end, the package defines a command
to pass on compilation to a different file:

%%%%%%%%%%%%%%%%%%%%%%%%%%%%%%%%%%%%%%%%
\DescribeMacro{\childdocforward}
The command |\childdocforward| redirects processing to
another source file:
%
\begin{center}
\begin{tabular}{l}
|\input{childdoc.def}|\\
|\childdocforward[|\textit{main}|]{|\textit{dest}|}|\\
\end{tabular}
\end{center}
%
The argument \textit{dest} is the destination file
(without extension).
It should be the main file or one of the child files.
Note that further \textsf{childdoc} directives
such as |\childdocof| and |\childdocforward|
in the indicated file will be processed in this form.
The optional argument \textit{main}
passes on directly to the main file \textit{main}
while pretending to compile the child \textit{dest}.
This form behaves as if \textit{dest}
issues |\childdocof{|\textit{main}|}| right away,
and no further \textsf{childdoc} directives will be processed.

%%%%%%%%%%%%%%%%%%%%%%%%%%%%%%%%%%%%%%%%
\DescribeMacro{\...prefix}
In the alternative form |\childdocforwardprefix|,
%
\begin{center}
\begin{tabular}{l}
|\input{childdoc.def}|\\
|\childdocforwardprefix[|\textit{main}|]{|\textit{prefix}|}{|\textit{dest}|}|
\end{tabular}
\end{center}
%
the destination file is determined by a pattern
depending on the current file:
To make this work, the current file must be called
`{\textit{prefix}\hspace{0.2em}\textit{suffix}}'
with \textit{prefix} matching precisely the argument.
Processing is then passed on to the file
`{\textit{dest}\hspace{0.2em}\textit{suffix}}'.
Surely, the same effect is achieved by
directly specifying the
argument `{\textit{dest}\hspace{0.2em}\textit{suffix}}'
in the first form.
However, that requires to set up a different file
for each child. With the alternative form of the command
all these files can have exactly the same content
which simplifies setting them up and maintaining them.

For example, the following file |draft.tex|
with a compilation flag |\version| as described in \secref{sec:flags}
compiles the main document as a draft:
%
\begin{center}
\begin{tabular}{l}
|\def\version{draft}|\\
|\input{childdoc.def}|\\
|\childdocforward{|\textit{main}|}|
\end{tabular}
\end{center}
%
Likewise, the following files |final|\textit{nn}|.tex|
compile the final version of the child document
|child|\textit{nn}|.tex|:
%
\begin{center}
\begin{tabular}{l}
|\def\version{final}|\\
|\input{childdoc.def}|\\
|\childdocforwardprefix{final}{child}|
\end{tabular}
\end{center}
%

Note that when several versions of a main file and/or of each child file
are to be generated, it may be convenient to set up a |Makefile| or
shell script to automatise the process.

%%%%%%%%%%%%%%%%%%%%%%%%%%%%%%%%%%%%%%%%%%%%%%%%%%%%%%%%%%%%%%%%%%%%%%%%%%%%%%%%
\subsection{Command Line Processing}
\label{sec:commandline}

The effect of redirection files can also be achieved by invoking
the \LaTeX{} compiler with a more elaborate command line.
Most conveniently this should be done as part
of a shell script or a |Makefile|.

When using \textsf{childdoc} in the main file, the following
command lines effectively perform a redirection
(note that depending on the shell being used,
backslashes may have to be doubled: `|\|' $\to$ `|\\|'):
%
\begin{center}
|... -jobname "|\textit{target}|" |\\|"|[\textit{flags}]%
|\input{childdoc.def}\childdocforward[|\textit{main}|]{|\textit{dest}|}"|
\end{center}
%
Here \textit{target} is the name of the output file,
\textit{main} is the name of the main file
and \textit{dest} is the name of the main or child file to be processed
(all filenames without extensions).
The optional argument \textit{main} can be omitted
if \textit{main} matches \textit{dest}.
Optionally, compilation \textit{flags} can be defined via |\def| commands.
This command line makes the \TeX{} engine believe
it is compiling the file \textit{target}
whose content is specified as the latter parameter.
The provided code then forwards the processing to
\textit{main} or \textit{dest} as described in \secref{sec:forward}.

%%%%%%%%%%%%%%%%%%%%%%%%%%%%%%%%%%%%%%%%%%%%%%%%%%%%%%%%%%%%%%%%%%%%%%%%%%%%%%%%
\subsection{Include by Input}
\label{sec:input}

Including child documents by |\include| has some restrictions by design.
Most notably, the content of a child document always occupies
its own set of pages; pages cannot be shared between child documents.
Usually, this behaviour makes perfect sense
because each child document contain an essential part of the document.
However, in some situations it may be desirable to compose
a document from a collection of parts
without having mandatory page breaks between then.
For this case, the package
provides a mechanism to include parts
by |\input| which can also be processed individually.
However, by construction this mechanism
requires manual handling of the content to be output.

%%%%%%%%%%%%%%%%%%%%%%%%%%%%%%%%%%%%%%%%
\DescribeMacro{\ifchilddocmanual}
The main file should be prepared as usual, see \secref{sec:include}.
However, the document body must make a distinction
between processing of an individual part and of the main document, e.g.:
%
\begin{center}
\begin{tabular}{l}
|\ifchilddocmanual|\\
|\input{\childdocname}|\\
|\||else|\\
\textit{document body with }|\input{|\textit{part}|}|\\
|\||fi|
\end{tabular}
\end{center}
%
The conditional |\ifchilddocmanual| is true whenever
a part to be included by |\input| is being compiled,
and the name of the part is stored in |\childdocname|.

%%%%%%%%%%%%%%%%%%%%%%%%%%%%%%%%%%%%%%%%
\DescribeMacro{\childdocby}
Each part to be included by |\input| should start with:
%
\begin{center}
\begin{tabular}{l}
|\input{childdoc.def}|\\
|\childdocby{|\textit{main}|}|\\
\end{tabular}
\end{center}
%
The directive |\childdocby| is similar to |\childdocof|
described in \secref{sec:include},
but the subsequent selection of content must be done manually.
To that end, both |\ifchilddoc| and |\ifchilddocmanual|
will be true upon processing of a part,
and the name of the part is stored in |\childdocname|.
Note that |\jobname| will be set to the filename of the current part
so that each part receives an individual |.aux| file
that does not interfere with the |.aux| file(s) of the main document.
This behaviour can be altered by the alternative form
|\childdocby[*]{|\textit{main}|}| (with a non-empty optional argument)
which uses the |.aux| file of the main document
by setting |\jobname| to \textit{main}.

%%%%%%%%%%%%%%%%%%%%%%%%%%%%%%%%%%%%%%%%%%%%%%%%%%%%%%%%%%%%%%%%%%%%%%%%%%%%%%%%
\subsection{Driver Development}
\label{sec:driver}

The \textsf{childdoc} mechanism can also be use for the development
of definition files such as \LaTeX{} styles or classes.
This case differs from the above setup with multiple parts
included by |\include| in that no |\includeonly| should be invoked.
This can be achieved by starting the include file
(before |\ProvidesPackage|) with:
%
\begin{center}
\begin{tabular}{l}
|\input{childdoc.def}|\\
|\childdocforward{|\textit{main}|}|\\
\end{tabular}
\end{center}
%
or alternatively with:
%
\begin{center}
\begin{tabular}{l}
|\input{childdoc.def}|\\
|\childdocby{|\textit{main}|}|\\
\end{tabular}
\end{center}
%
Both forms have slightly different effects as described above.
The main file is prepared as usual, see \secref{sec:include}.

%%%%%%%%%%%%%%%%%%%%%%%%%%%%%%%%%%%%%%%%%%%%%%%%%%%%%%%%%%%%%%%%%%%%%%%%%%%%%%%%
\subsection{Legacy Detection}
\label{sec:detection}

The directive |\childdocmain| in the main file can detect
whether the complete document or merely a child is to be compiled
even without using the directive |\childdocof|.
This method is deprecated because it is less robust
and there is no compelling reason to use it;
it is merely provided for backward compatibility
and it may be removed in future versions.

If the detection mechanism is to be used,
it is mandatory to correctly specify
the filename of the main file as the argument of |\childdocmain|:
%
\begin{center}
\begin{tabular}{l}
|\input{childdoc.def}|\\
|\childdocmain{|\textit{main}|}|\\
\end{tabular}
\end{center}
%
If |\jobname| does not match the argument \textit{main} of |\childdocmain|,
it is assumed that |\jobname| points to the child file to be compiled.
When using |\childdocmain| with the main file specified as argument,
it suffices to start a child file
with just |\input{|\textit{main}|}|
without loading of the package and using |\childdocof|.
If instead all processing is done
with the appropriate \textsf{childdoc} directives,
the argument of \textit{main} of |\childdocmain| can be empty.

An alternative version of the command line processing described
in \secref{sec:commandline} using the detection mechanism reads:
%
\begin{center}
|... -jobname "|\textit{target}|" "|[\textit{flags}]%
[|\def\jobname{|\textit{dest}|}|]|\input{|\textit{main}|}"|
\end{center}

%%%%%%%%%%%%%%%%%%%%%%%%%%%%%%%%%%%%%%%%%%%%%%%%%%%%%%%%%%%%%%%%%%%%%%%%%%%%%%%%
\subsection{Manual Code}
\label{sec:manual}

In case one cannot be certain whether the definitions file |childdoc.def|
is installed on the target \TeX{} distribution
and one prefers not to ship it,
it is conceivable to paste a few relevant commands into the sources.

To that end, drop all statements |\input{childdoc.def}|
and perform the replacements as outlined below.
Instead of |\childdocmain{|\textit{main}|}| add the following code
to the top of the main file:
%
\begin{center}
\begin{tabular}{l}
|\||ifdefined\childdocname\endinput\||fi\newif\ifchilddoc|\\
|\edef\childdocname{\scantokens\expandafter{\jobname\noexpand}}|\\
|\def\childdocmain{|\textit{main}|}\||ifx\childdocmain\childdocname\||else|\\
|\childdoctrue\includeonly{\childdocname}\let\jobname\childdocmain\||fi|\\
\end{tabular}
\end{center}
%
Instead of |\childdocof{|\textit{main}|}| just include the main file
at the top of each child file:
%
\begin{center}
|\input{|\textit{main}|}|
\end{center}
%
A simple redirection |\childdocforward{|\textit{dest}|}| is achieved by:
%
\begin{center}
|\def\jobname{|\textit{dest}|}\input{\jobname}|
\end{center}
%
The redirection with prefix
|\childdocforwardprefix[|\textit{prefix}|]{|\textit{dest}|}|
is accomplished by:
%
\begin{center}
\begin{tabular}{l}
|{\edef\jobname{\scantokens\expandafter{\jobname\noexpand}}|\\
|\def\redirectjob |\textit{prefix}|#1~~~{\gdef\jobname{|\textit{dest}|#1}}|\\
|\expandafter\redirectjob\jobname~~~}\input{\jobname}|
\end{tabular}
\end{center}

In an alternative approach,
child documents can be compiled by a specific command line
without additional code or specific definitions:
%
\begin{center}
|... -jobname "|\textit{target}|" "|[\textit{flags}]%
|\includeonly{|\textit{dest}|}\input{|\textit{main}|}"|
\end{center}
%

%%%%%%%%%%%%%%%%%%%%%%%%%%%%%%%%%%%%%%%%%%%%%%%%%%%%%%%%%%%%%%%%%%%%%%%%%%%%%%%%
%%%%%%%%%%%%%%%%%%%%%%%%%%%%%%%%%%%%%%%%%%%%%%%%%%%%%%%%%%%%%%%%%%%%%%%%%%%%%%%%
\section{Information}

%%%%%%%%%%%%%%%%%%%%%%%%%%%%%%%%%%%%%%%%%%%%%%%%%%%%%%%%%%%%%%%%%%%%%%%%%%%%%%%%
\subsection{Copyright}

Copyright \copyright{} 2017--2018 Niklas Beisert

This work may be distributed and/or modified under the
conditions of the \LaTeX{} Project Public License, either version 1.3
of this license or (at your option) any later version.
The latest version of this license is in
  \url{http://www.latex-project.org/lppl.txt}
and version 1.3 or later is part of all distributions of \LaTeX{}
version 2005/12/01 or later.

This work has the LPPL maintenance status `maintained'.

The Current Maintainer of this work is Niklas Beisert.

This work consists of the files |README.txt|, |childdoc.ins| and |childdoc.dtx|
as well as the derived files |childdoc.def|, |cdocsamp.tex|
with |cdocsch1.tex|, |cdocsch2.tex|, |cdocspt3.tex|, |cdocspt4.tex|,
|cdocsdrf.tex|, |cdocsfn1.tex|, |cdocsfn2.tex|
as well as |childdoc.pdf|.

%%%%%%%%%%%%%%%%%%%%%%%%%%%%%%%%%%%%%%%%%%%%%%%%%%%%%%%%%%%%%%%%%%%%%%%%%%%%%%%%
\subsection{Files and Installation}

The package consists of the files:
%
\begin{center}
\begin{tabular}{ll}
    |README.txt|   & readme file \\
    |childdoc.ins| & installation file \\
    |childdoc.dtx| & source file \\
    |childdoc.def| & definition file \\
    |cdocsamp.tex| & sample main file \\
    |cdocsch1.tex| & sample include file \\
    |cdocsch2.tex| & sample include file \\
    |cdocspt3.tex| & sample part file \\
    |cdocspt4.tex| & sample part file \\
    |cdocsdrf.tex| & sample redirection file \\
    |cdocsfn1.tex| & sample redirection file \\
    |cdocsfn2.tex| & sample redirection file \\
    |childdoc.pdf| & manual
\end{tabular}
\end{center}
%
The distribution consists of the files
|README.txt|, |childdoc.ins| and |childdoc.dtx|.
%
\begin{itemize}
\item
Run (pdf)\LaTeX{} on |childdoc.dtx|
to compile the manual |childdoc.pdf| (this file).
\item
Run \LaTeX{} on |childdoc.ins| to create the definitions file |childdoc.def|
and the sample |cdocsamp.tex| with include files
|cdocsch1.tex|, |cdocsch2.tex|, |cdocspt3.tex|, |cdocspt4.tex|,
|cdocsdrf.tex|, |cdocsfn1.tex|, |cdocsfn2.tex|.
Then copy the file |childdoc.def| to an appropriate directory of your \LaTeX{}
distribution, e.g.\ \textit{texmf-root}|/tex/latex/childdoc|.
\end{itemize}

%%%%%%%%%%%%%%%%%%%%%%%%%%%%%%%%%%%%%%%%%%%%%%%%%%%%%%%%%%%%%%%%%%%%%%%%%%%%%%%%
\subsection{Related CTAN Packages}

There are several other packages which offer a similar functionality:
%
\begin{itemize}
\item
The packages
\href{http://ctan.org/pkg/docmute}{\textsf{docmute}},
\href{http://ctan.org/pkg/includex}{\textsf{includex}} and
\href{http://ctan.org/pkg/standalone}{\textsf{standalone}}
provide commands to include only the document body of
a child file thus allowing both files to be compiled individually.
\item
The packages \href{http://ctan.org/pkg/subdocs}{\textsf{subdocs}}
and \href{http://ctan.org/pkg/subfiles}{\textsf{subfiles}}
provide structures in which the main and child documents can be
encapsulated and allowing them to be compiled individually.
The inclusion mechanism is different from the conventional |\include|.
\item
The package \href{http://ctan.org/pkg/combine}{\textsf{combine}}
is an elaborate solution to combine several documents into one.
\end{itemize}
%
See also the CTAN topic \href{http://ctan.org/topic/subdocs}{\textsf{subdocs}}
for further related packages.
The present package differs from the above solutions in that
a document structure constructed with the conventional |\include| mechanism
just needs two extra commands at the top of every file
such that all constituent files can be compiled individually.

%%%%%%%%%%%%%%%%%%%%%%%%%%%%%%%%%%%%%%%%%%%%%%%%%%%%%%%%%%%%%%%%%%%%%%%%%%%%%%%%
%\subsection{Feature Suggestions}
%
%The following is a list of features which may be useful for future
%versions of this package:
%%
%\begin{itemize}
%\item
%\ldots
%\end{itemize}

%%%%%%%%%%%%%%%%%%%%%%%%%%%%%%%%%%%%%%%%%%%%%%%%%%%%%%%%%%%%%%%%%%%%%%%%%%%%%%%%
\subsection{Revision History}

%%%%%%%%%%%%%%%%%%%%%%%%%%%%%%%%%%%%%%%%
\paragraph{v2.0:} 2018/12/30

\begin{itemize}
\item
immediate forward processing
\item
added |\childdocby| mechanism
\item
manual restructured
\end{itemize}

%%%%%%%%%%%%%%%%%%%%%%%%%%%%%%%%%%%%%%%%
\paragraph{v1.6:} 2018/01/17

\begin{itemize}
\item
application for development of include files
\item
corrections to manual
\end{itemize}

%%%%%%%%%%%%%%%%%%%%%%%%%%%%%%%%%%%%%%%%
\paragraph{v1.5:} 2017/05/21

\begin{itemize}
\item
more complete structuring introduced
\item
|\childdocof| introduced
\item
|\childdoc| renamed to |\childdocmain|
\item
|\childredirect| renamed to |\childdocforward| and |\childdocforwardprefix|
and functionality expanded
\end{itemize}

%%%%%%%%%%%%%%%%%%%%%%%%%%%%%%%%%%%%%%%%
\paragraph{v1.0:} 2017/04/27

\begin{itemize}
\item
manual and install package
\item
first version published on CTAN
\end{itemize}

%%%%%%%%%%%%%%%%%%%%%%%%%%%%%%%%%%%%%%%%
\paragraph{v0.6:} 2017/04/26

\begin{itemize}
\item
redirection mechanism added
\end{itemize}

%%%%%%%%%%%%%%%%%%%%%%%%%%%%%%%%%%%%%%%%
\paragraph{v0.5:} 2017/04/26

\begin{itemize}
\item
functionality in definition file
\end{itemize}


%%%%%%%%%%%%%%%%%%%%%%%%%%%%%%%%%%%%%%%%%%%%%%%%%%%%%%%%%%%%%%%%%%%%%%%%%%%%%%%%
%%%%%%%%%%%%%%%%%%%%%%%%%%%%%%%%%%%%%%%%%%%%%%%%%%%%%%%%%%%%%%%%%%%%%%%%%%%%%%%%
%%%%%%%%%%%%%%%%%%%%%%%%%%%%%%%%%%%%%%%%%%%%%%%%%%%%%%%%%%%%%%%%%%%%%%%%%%%%%%%%
\appendix

\settowidth\MacroIndent{\rmfamily\scriptsize 000\ }

 \DocInput{childdoc.dtx}

\end{document}
%</driver>
% \fi
%
% %%%%%%%%%%%%%%%%%%%%%%%%%%%%%%%%%%%%%%%%%%%%%%%%%%%%%%%%%%%%%%%%%%%%%%%%%%%%%%
% %%%%%%%%%%%%%%%%%%%%%%%%%%%%%%%%%%%%%%%%%%%%%%%%%%%%%%%%%%%%%%%%%%%%%%%%%%%%%%
% \section{Sample}
%\iffalse
%<*samplemain>
%\fi
%
% The following presents a sample document
% with two chapters, two parts, a title page,
% a compile flag as well as three forwarding files to set the flag.
% It consists of eight |.tex| files:
% \begin{center}
% \begin{tabular}{ll}
% |cdocsamp.tex|&main file\\
% |cdocsch1.tex|&include file for chapter 1\\
% |cdocsch2.tex|&include file for chapter 2\\
% |cdocspt3.tex|&include file for part 3\\
% |cdocspt4.tex|&include file for part 4\\
% |cdocsdrf.tex|&forwarding file for main file in draft mode\\
% |cdocsfi1.tex|&forwarding file for final version of chapter 1\\
% |cdocsfi2.tex|&forwarding file for final version of chapter 2\\
% \end{tabular}
% \end{center}
% Each of the eight files can be compiled directly by the \LaTeX{} compiler.
%
% %%%%%%%%%%%%%%%%%%%%%%%%%%%%%%%%%%%%%%
% \paragraph{Main File.}
%
% The main file is called |cdocsamp.tex|.
%
% Load the \textsf{childdoc} definitions and
% declare the filename for the main document:
%    \begin{macrocode}
\input{childdoc.def}
\childdocmain{}
%    \end{macrocode}

% Optional override for |\version| flag:
%    \begin{macrocode}
%%\ifchilddoc\else\providecommand{\version}{draft}\fi
%    \end{macrocode}

% Define the default values for the |\version| flag
% (|final| for the main file and |draft| for childs):
%    \begin{macrocode}
\ifchilddoc
\providecommand{\version}{draft}
\else
\providecommand{\version}{final}
\fi
%    \end{macrocode}

% Load the standard document class:
%    \begin{macrocode}
\documentclass[12pt]{article}
%    \end{macrocode}

% Start the document body:
%    \begin{macrocode}
\begin{document}
%    \end{macrocode}

% Declare a title page.
% Print title, part of document being processed and version flag:
%    \begin{macrocode}
\addtocounter{page}{-1}
\begin{center}
{\LARGE\bfseries{}childdoc example\par}
\vspace{1cm}
\ifchilddoc
\ifchilddocmanual part\else chapter\fi:
`\childdocname' of `\childdocjob'\par
\else
main document: `\childdocjob'\par
\fi
version: \version\par
\end{center}
\newpage
%    \end{macrocode}

% Manually include selected file,
% otherwise process as usual:
%    \begin{macrocode}
\ifchilddocmanual
\section*{part `\childdocname'}
\input{\childdocname}
\else
%    \end{macrocode}

% Include the two chapters:
%    \begin{macrocode}
\include{cdocsch1}
\include{cdocsch2}
%    \end{macrocode}

% Include the two parts unless only chapters should be displayed:
%    \begin{macrocode}
\ifchilddoc\else
\section{part three}
\input{cdocspt3}
\section{part four}
\input{cdocspt4}
\fi
%    \end{macrocode}

% Process as usual until here:
%    \begin{macrocode}
\fi
%    \end{macrocode}

% End of document body:
%    \begin{macrocode}
\end{document}
%    \end{macrocode}
%\iffalse
%</samplemain>
%\fi
%
% %%%%%%%%%%%%%%%%%%%%%%%%%%%%%%%%%%%%%%
% \paragraph{Chapter Include Files.}
%
% The include files are called |cdocsch1.tex| and |cdocsch2.tex|.
%
%\iffalse
%<*samplechap1|samplechap2>
%\fi

% Optional override for |\version| flag:
%    \begin{macrocode}
%%\providecommand{\version}{final}
%    \end{macrocode}

% Include the main document:
%    \begin{macrocode}
\input{childdoc.def}
\childdocof{cdocsamp}
%    \end{macrocode}

%\iffalse
%</samplechap1|samplechap2>
%\fi
%
%\iffalse
%<*samplechap1>
%\fi
% Some text for chapter 1:
%    \begin{macrocode}
\section{one}
some text in chapter one
%    \end{macrocode}

%\iffalse
%</samplechap1>
%\fi
% Some text for chapter 2:
%\iffalse
%<*samplechap2>
%\fi
%    \begin{macrocode}
\section{two}
more text in chapter two
%    \end{macrocode}

%\iffalse
%</samplechap2>
%\fi
%
% %%%%%%%%%%%%%%%%%%%%%%%%%%%%%%%%%%%%%%
% \paragraph{Part Include Files.}
%
% The include files are called |cdocspt3.tex| and |cdocspt4.tex|.
%
%\iffalse
%<*samplepart3|samplepart4>
%\fi

% Optional override for |\version| flag:
%    \begin{macrocode}
%%\providecommand{\version}{final}
%    \end{macrocode}

% Include the main document:
%    \begin{macrocode}
\input{childdoc.def}
\childdocby{cdocsamp}
%    \end{macrocode}

%\iffalse
%</samplepart3|samplepart4>
%\fi
%
%\iffalse
%<*samplepart3>
%\fi
% Some text for part 3:
%    \begin{macrocode}
some text in part three
%    \end{macrocode}

%\iffalse
%</samplepart3>
%\fi
% Some text for part 4:
%\iffalse
%<*samplepart4>
%\fi
%    \begin{macrocode}
more text in part four
%    \end{macrocode}

%\iffalse
%</samplepart4>
%\fi
%
% %%%%%%%%%%%%%%%%%%%%%%%%%%%%%%%%%%%%%%
% \paragraph{Forwarding for a Complete Draft.}
%
% The following forwarding file |cdocsdrf.tex|
% compiles the main document in draft mode:
%\iffalse
%<*sampledraft>
%\fi
%    \begin{macrocode}
\def\version{draft}
\input{childdoc.def}
\childdocforward{cdocsamp}
%    \end{macrocode}

%\iffalse
%</sampledraft>
%\fi
%
% %%%%%%%%%%%%%%%%%%%%%%%%%%%%%%%%%%%%%%
% \paragraph{Forwarding for Final Version of the Chapters.}
%
% The following forwarding files |cdocsfn1.tex| and |cdocsfn2.tex|
% (with identical content)
% compile the final versions of the child documents
% |cdocsch1.tex| and |cdocsch2.tex|, respectively:
%\iffalse
%<*samplefinal>
%\fi
%    \begin{macrocode}
\def\version{final}
\input{childdoc.def}
\childdocforwardprefix[cdocsamp]{cdocsfn}{cdocsch}
%    \end{macrocode}

%\iffalse
%</samplefinal>
%\fi
%
% %%%%%%%%%%%%%%%%%%%%%%%%%%%%%%%%%%%%%%
% \paragraph{Command Line Processing.}
%
% The following three command lines generate the output files
% |cdocscld|, |cdocscl1| and |cdocscl2|
% which should be identical to
% |cdocsdrf|, |cdocsch1| and |cdocsfn2|, respectively:
% \begin{center}
% \begin{tabular}{l}
% |latex -jobname cdocscld \|\\
% |  "\def\version{draft}\input{childdoc.def}\childdocforward{cdocsamp}"|\\
% |latex -jobname cdocscl1 \|\\
% |  "\input{childdoc.def}\childdocforward[cdocsamp]{cdocsch1}"|\\
% |latex -jobname cdocscl2 \|\\
% |  "\def\version{final}\input{childdoc.def}\childdocforward{cdocsch2}"|
% \end{tabular}
% \end{center}
% Note that the trailing backslash on each first line
% merely continues the input to the second line
% (for convenient cut ant paste).
% Furthermore, the command |latex| can be replaced by any
% of its alternative versions such as |pdflatex|.
%
% %%%%%%%%%%%%%%%%%%%%%%%%%%%%%%%%%%%%%%%%%%%%%%%%%%%%%%%%%%%%%%%%%%%%%%%%%%%%%%
% %%%%%%%%%%%%%%%%%%%%%%%%%%%%%%%%%%%%%%%%%%%%%%%%%%%%%%%%%%%%%%%%%%%%%%%%%%%%%%
% \section{Implementation}
%\iffalse
%<*package>
%\fi
%
% This section describes the definitions file |childdoc.def|.

% The definitions cannot be loaded using |\usepackage| or |\RequirePackage|
% which has a mechanism to prevent loading a style file more than once.
% When loading the definitions by means of |\input|
% multiple instances have to be prevented manually:
%\iffalse
%This code needs to be before the `\ProvidesFile' directive
%which is defined at the beginning of this file.
%Therefore it is also placed there and commented out here.
%</package>
%<*discard>
%\fi
%    \begin{macrocode}
\ifdefined\childdocmain\endinput\fi
%    \end{macrocode}
%\iffalse
%</discard>
%<*package>
%\fi
%
% \macro{\ifchilddoc}
% \macro{\ifchilddocmanual}
% The conditional |\ifchilddoc| tells whether a
% child (true) or main (false) document is being compiled.
% The conditional |\ifchilddocmanual| tells whether
% the |\includeonly| mechanism is used (false) or
% the selection of child files must be performed manually (true).
% The definitions initialise to false:
%    \begin{macrocode}
\newif\ifchilddoc
\newif\ifchilddocmanual
%    \end{macrocode}

% \macro{\childdocname}
% \macro{\childdocjob}
% The macro |\childdocname| stores the name of the main document
% to be compiled. The macro |\childdocjob| stores the name of
% the document on which the \LaTeX{} compiler was originally invoked.
% The content of |\jobname| cannot be compared
% to filenames specified in the source due to different catcodes.
% The following code rescans |\jobname|, stores the result
% in |\childdocname| and saves a copy in |\childdocjob|:
%    \begin{macrocode}
\edef\childdocname{\scantokens\expandafter{\jobname\noexpand}}
\let\childdocjob\childdocname
%    \end{macrocode}

% \macro{\childdocdisable}
% The macro |\childdocdisable| prevents the main file
% from being processed more than once.
% At this stage, the main document command |\childdocmain|
% is assumed to be called once again where it should do nothing.
% Any subsequent call to it should prevent
% a secondary processing of the main document
% It overwrites the forwarding commands
% |\childdocof| and |\childdocforward|
% with empty macros to prevent further inclusions of the main document:
%    \begin{macrocode}
\newcommand{\childdocdisable}
{
  \renewcommand{\childdocmain}[1]{\renewcommand{\childdocmain}[1]{\endinput}}
  \renewcommand{\childdocof}[1]{}
  \renewcommand{\childdocby}[2][]{}
  \renewcommand{\childdocforward}[2][]{}
  \renewcommand{\childdocdisable}{}
}
%    \end{macrocode}

% \macro{\childdocmain}
% The macro |\childdocmain| is to be called at the top of the main file
% with nothing or the main filename (without extension) as argument.
% First, it breaks loops.
% If the argument is not empty and does not match |\childdocname|
% (which is set by the first inclusion of |childdoc.def|),
% |\ifchilddoc| is set to true, |\includeonly| is applied to the child file
% and |\jobname| is set to the main file
% (for proper handling of |.aux| files):
%    \begin{macrocode}
\newcommand{\childdocmain}[1]
{
  \childdocdisable\childdocmain{}
  \if?#1?\else
    \begingroup
      \def\childdoctmp{#1}
      \ifx\childdoctmp\childdocname
        \def\childdoctmp{}
      \else
        \def\childdoctmp
        {
          \childdoctrue
          \includeonly{\childdocname}
          \def\childdocjob{#1}
          \def\jobname{#1}
        }
      \fi
      \expandafter
    \endgroup
    \childdoctmp
  \fi
}
%    \end{macrocode}

% \macro{\childdocof}
% The command |\childdocof| redirects
% compilation to the main file |#1|.
%    \begin{macrocode}
\newcommand{\childdocof}[1]
{
  \childdocdisable
  \childdoctrue
  \includeonly{\childdocname}
  \def\jobname{#1}
  \def\childdocjob{#1}
  \input{#1}
}
%    \end{macrocode}

% \macro{\childdocby}
% The command |\childdocby| ....
%    \begin{macrocode}
\newcommand{\childdocby}[2][]
{
  \childdocdisable
  \childdoctrue
  \childdocmanualtrue
  \if?#1?\else
    \def\jobname{#2}
  \fi
  \def\childdocjob{#2}
  \input{#2}
  \endinput
}
%    \end{macrocode}

% \macro{\childdocforward}
% The command |\childdocforward| redirects
% compilation to the main file or
% (if the optional argument is given) a child file.
% Parameters are set as if the main file
% or a child file starting with |\childdocof| was compiled.
% Then compilation is handed over to the main file:
%    \begin{macrocode}
\newcommand{\childdocforward}[2][]
{
  \begingroup
    \if?#1?
      \def\childdoctmp
      {
        \def\childdocname{#2}
        \def\childdocjob{#2}
        \def\jobname{#2}
        \input{#2}
        \endinput
      }
    \else
      \def\childdoctmp
      {
        \childdocdisable
        \def\childdocname{#2}
        \childdoctrue
        \includeonly{#2}
        \def\childdocjob{#1}
        \def\jobname{#1}
        \input{#1}
        \endinput
      }
    \fi
    \expandafter
  \endgroup
  \childdoctmp
}
%    \end{macrocode}

% \macro{\childdocforwardprefix}
% The command |\childdocforwardprefix| redirects
% compilation to the main or a child file by means of a pattern.
% The prefix |#1| in the current filename is replaced by |#2|
% and the suffix of the current filename is kept
% (it is assumed that the filename does not contain the substring `|~~~|'
% which is used as a delimiter).
% Compilation is handed over to the new file by |\childdocforward|:
%    \begin{macrocode}
\newcommand{\childdocforwardprefix}[3][]
{
  \begingroup
    \def\childdocextract #2##1~~~{\def\childdoctmp{\childdocforward[#1]{#3##1}}}
    \expandafter\childdocextract\childdocname~~~
    \expandafter
  \endgroup
  \childdoctmp
}
%    \end{macrocode}

% \macro{\childdoc}
% The deprecated macro |\childdoc| is a legacy version of |\childdocmain|:
%    \begin{macrocode}
\newcommand{\childdoc}{\childdocmain}
%    \end{macrocode}

% \macro{\childdocredirect}
% The deprecated macro |\childdocredirect| is a legacy version
% of |\childdocforward| and |\childdocforwardprefix|:
%    \begin{macrocode}
\newcommand{\childdocredirect}[2][]
{
  \begingroup
    \if?#1?
      \def\childdoctmp{\childdocforward{#2}}
    \else
      \def\childdoctmp{\childdocforwardprefix{#1}{#2}}
    \fi
    \expandafter
  \endgroup
  \childdoctmp
}
%    \end{macrocode}

%\iffalse
%</package>
%\fi
%
\endinput
|\\
|\childdocforwardprefix{final}{child}|
\end{tabular}
\end{center}
%

Note that when several versions of a main file and/or of each child file
are to be generated, it may be convenient to set up a |Makefile| or
shell script to automatise the process.

%%%%%%%%%%%%%%%%%%%%%%%%%%%%%%%%%%%%%%%%%%%%%%%%%%%%%%%%%%%%%%%%%%%%%%%%%%%%%%%%
\subsection{Command Line Processing}
\label{sec:commandline}

The effect of redirection files can also be achieved by invoking
the \LaTeX{} compiler with a more elaborate command line.
Most conveniently this should be done as part
of a shell script or a |Makefile|.

When using \textsf{childdoc} in the main file, the following
command lines effectively perform a redirection
(note that depending on the shell being used,
backslashes may have to be doubled: `|\|' $\to$ `|\\|'):
%
\begin{center}
|... -jobname "|\textit{target}|" |\\|"|[\textit{flags}]%
|% \iffalse
%
% childdoc.dtx Copyright (C) 2017-2018 Niklas Beisert
%
% This work may be distributed and/or modified under the
% conditions of the LaTeX Project Public License, either version 1.3
% of this license or (at your option) any later version.
% The latest version of this license is in
%   http://www.latex-project.org/lppl.txt
% and version 1.3 or later is part of all distributions of LaTeX
% version 2005/12/01 or later.
%
% This work has the LPPL maintenance status `maintained'.
%
% The Current Maintainer of this work is Niklas Beisert.
%
% This work consists of the files childdoc.dtx and childdoc.ins
% and the derived files childdoc.def and cdocsamp.tex with
% cdocsch1.tex, cdocsch2.tex, cdocsdrf.tex, cdocsfn1.tex, cdocsfn2.tex.
%
%<package>\ifdefined\childdocmain\endinput\fi
%<package>\ProvidesFile{childdoc.def}[2018/12/30 v2.0 child document driver]
%<samplemain>\ProvidesFile{cdocsamp.tex}[2018/12/30 v2.0 sample for childdoc]
%<*driver>
%\ProvidesFile{childdoc.drv}[2018/12/30 v2.0 childdoc reference manual file]
\PassOptionsToClass{10pt,a4paper}{article}
\documentclass{ltxdoc}

\usepackage[margin=35mm]{geometry}
\usepackage{hyperref}
\usepackage{hyperxmp}
\usepackage[usenames]{color}

\hypersetup{colorlinks=true}
\hypersetup{pdfstartview=FitH}
\hypersetup{pdfpagemode=UseNone}
\hypersetup{pdfsource={}}
\hypersetup{pdflang={en-UK}}
\hypersetup{pdfcopyright={Copyright 2017-2018 Niklas Beisert.
  This work may be distributed and/or modified under the
  conditions of the LaTeX Project Public License, either version 1.3
  of this license or (at your option) any later version.}}
\hypersetup{pdflicenseurl={http://www.latex-project.org/lppl.txt}}
\hypersetup{pdfcontactaddress={ETH Zurich, ITP, HIT K,
  Wolfgang-Pauli-Strasse 27}}
\hypersetup{pdfcontactpostcode={8093}}
\hypersetup{pdfcontactcity={Zurich}}
\hypersetup{pdfcontactcountry={Switzerland}}
\hypersetup{pdfcontactemail={nbeisert@itp.phys.ethz.ch}}
\hypersetup{pdfcontacturl={http://people.phys.ethz.ch/\xmptilde nbeisert/}}

\newcommand{\secref}[1]{\hyperref[#1]{section \ref*{#1}}}

\parskip1ex
\parindent0pt
\let\olditemize\itemize
\def\itemize{\olditemize\parskip0pt}

\begin{document}

\title{The \textsf{childdoc} Package}
\hypersetup{pdftitle={The childdoc Package}}
\author{Niklas Beisert\\[2ex]
  Institut f\"ur Theoretische Physik\\
  Eidgen\"ossische Technische Hochschule Z\"urich\\
  Wolfgang-Pauli-Strasse 27, 8093 Z\"urich, Switzerland\\[1ex]
  \href{mailto:nbeisert@itp.phys.ethz.ch}
  {\texttt{nbeisert@itp.phys.ethz.ch}}}
\hypersetup{pdfauthor={Niklas Beisert}}
\hypersetup{pdfsubject={Manual for the LaTeX2e Package childdoc}}
\date{30 December 2018, \textsf{v2.0}}
\maketitle

\begin{abstract}\noindent
\textsf{childdoc} is a \LaTeXe{} package
that enables the direct compilation
of document sections included by |\include|
to individual files.
\end{abstract}

\begingroup
\parskip0ex
\tableofcontents
\endgroup

%%%%%%%%%%%%%%%%%%%%%%%%%%%%%%%%%%%%%%%%%%%%%%%%%%%%%%%%%%%%%%%%%%%%%%%%%%%%%%%%
%%%%%%%%%%%%%%%%%%%%%%%%%%%%%%%%%%%%%%%%%%%%%%%%%%%%%%%%%%%%%%%%%%%%%%%%%%%%%%%%
\section{Introduction}

\LaTeX{} provides a mechanism to structure a large document (such as a book)
into a main file and several child files (containing the chapters)
using the |\include| command.
This mechanism is beneficial for documents
which span hundreds of pages in order to
make the source file(s) more manageable.
Moreover, compilation can be restricted to
selected child files by means of the |\includeonly| command.
The latter feature can be used to reduce the compilation time while editing
(this was significantly more useful in the earlier days of \LaTeX{})
or to generate a smaller document which is easier to navigate.
Another application of |\includeonly| is to generate
documents consisting of selected parts of the complete document.

However, there are a few drawbacks of the plain |\include| mechanism:
\begin{itemize}
\item
The child files cannot be compiled on their own,
they can only be compiled via the main file.
A naive editing environment
(such as a text editor with an option
to have the current file processed by \LaTeX)
may require one to switch to the main file before compiling;
attempting to compile the child file produces errors.
\item
The main file must be modified (each time)
to adjust the |\includeonly| command
to the present needs. This easily leaves the main file in a messy state.
\item
The generated document will always carry the filename
of the main document. This is inconvenient if
several child files are to be compiled and
to be kept for distribution.
\end{itemize}

The present package provides a simple interface
to make child files individually compilable by \LaTeX{}.
Compiling a child file then has the same effect as compiling
the main file with an |\includeonly| command
to select the appropriate child.
Moreover the generated document will carry the name of the child
rather than the main file.
This resolves all three above issues.

This feature is meant to make the editing of books,
thesis documents and lecture notes somewhat more convenient.
However, the package can also be used efficiently for
composing a series of documents (such as exercise sheets)
which are typically distributed individually.
It then assists the author in generating the individual documents
(potentially in different versions)
as well as a document containing the collected series.
Another application is in developing style files
or other kinds of included material
where compilation of the style file could redirect
to a sample or test file.

%%%%%%%%%%%%%%%%%%%%%%%%%%%%%%%%%%%%%%%%%%%%%%%%%%%%%%%%%%%%%%%%%%%%%%%%%%%%%%%%
%%%%%%%%%%%%%%%%%%%%%%%%%%%%%%%%%%%%%%%%%%%%%%%%%%%%%%%%%%%%%%%%%%%%%%%%%%%%%%%%
\section{Usage}

First of all, the package \textsf{childdoc} is \emph{not} a standard
\LaTeXe{} |.sty| style file! Therefore it needs to be invoked in
a non-standard way.

%%%%%%%%%%%%%%%%%%%%%%%%%%%%%%%%%%%%%%%%%%%%%%%%%%%%%%%%%%%%%%%%%%%%%%%%%%%%%%%%
\subsection{Included Files}
\label{sec:include}

%%%%%%%%%%%%%%%%%%%%%%%%%%%%%%%%%%%%%%%%
\DescribeMacro{\childdocmain}
To use the package, add the commands
\begin{center}
\begin{tabular}{l}
|\input{childdoc.def}|\\
|\childdocmain{}|\\
\end{tabular}
\end{center}
at the very top of the main \LaTeX{} file,
in particular \emph{before} the |\documentclass| statement!
The argument of |\childdocmain| should be left empty
(but it must be present).

%%%%%%%%%%%%%%%%%%%%%%%%%%%%%%%%%%%%%%%%
\DescribeMacro{\childdocof}
Furthermore, add the commands
\begin{center}
\begin{tabular}{l}
|\input{childdoc.def}|\\
|\childdocof{|\textit{main}|}|\\
\end{tabular}
\end{center}
at the top of every child file \textit{child}
which is included by |\include{|\textit{child}|}|
from within the main file
(or at least for those files to be compiled individually).
The argument \textit{main} must be the filename of the main file.

There are a couple of
considerations in setting up the main and child documents:

%%%%%%%%%%%%%%%%%%%%%%%%%%%%%%%%%%%%%%%%
\paragraph{Restrictions.}

Please note the following restrictions:
\begin{itemize}
\item
|\childdocmain| must be called with one argument \textit{main}
to ensure compatibility with earlier version of the package.
It must either be empty (|\childdocmain{}|)
or precisely match the filename of the main file in which it is specified.
See \secref{sec:detection} for further information.
\item
The filename \textit{main} must be specified without the |.tex| extension.
\item
The filename \textit{main} is case sensitive
(even in case-insensitive file systems)
due to internal string comparison.
\item
The argument \textit{main} should be fully expanded, it cannot be a macro.
\item
Subdirectories and special characters should be avoided in filenames.
\item
The command |\childdocmain{|\textit{main}|}| must be followed by a whitespace.
It should not be followed immediately by another command
or by a comment mark `|%|'.
This is because the \TeX{} parser reads the token immediately following
the argument of |\childdocmain| and puts it
at the beginning of every child section;
however, a white\-space is ignored.
\end{itemize}

%%%%%%%%%%%%%%%%%%%%%%%%%%%%%%%%%%%%%%%%
\paragraph{Content of Main File.}

It is advisable to place all content in the child files included by |\include|.
Any output contained in the main file will appear in all child documents
unless suppressed manually;
it cannot be suppressed automatically by the |\includeonly| directive
and thus should normally be avoided.
A method to include some content in the main file
by means of conditional processing is described in \secref{sec:conditional}.

%%%%%%%%%%%%%%%%%%%%%%%%%%%%%%%%%%%%%%%%
\paragraph{Page Numbering.}

When only a part of the document is compiled,
the appropriate numbering of pages
(as well as other status parameters)
is determined from the |.aux| files.
The latter contain information from previous passes.
However this information needs to propagate through
all intermediate child documents.
Therefore the page numbering in child documents may well
be inconsistent until the complete document is compiled at least once.

A useful (if unconventional) way to always ensure a consistent
page numbering is to restart the numbering in each child document
and denote the pages by `\textit{child}|.|\textit{page}'
where \textit{child} represents the chapter/section number of the child file.
This can be achieved by the command
|\numberwithin{page}{|\textit{child}|}|
of the \textsf{amsmath} package
where \textit{child} can be |chapter| or |section|
depending on the chosen structuring.
Alternatively, one can modify the macro |\thepage| appropriately
and reset the counter |page| at the start of each child file.

%%%%%%%%%%%%%%%%%%%%%%%%%%%%%%%%%%%%%%%%%%%%%%%%%%%%%%%%%%%%%%%%%%%%%%%%%%%%%%%%
\subsection{Conditional Processing}
\label{sec:conditional}

The package provides a mechanism to compile different versions
of a document. To customise the versions further some conditional processing
can come in handy to distinguish which version is being compiled.
The package provides two macros to describe the compilation context:

%%%%%%%%%%%%%%%%%%%%%%%%%%%%%%%%%%%%%%%%
\DescribeMacro{\ifchilddoc}
The conditional |\ifchilddoc| distinguishes between the compilation of
child documents and the main document:
%
\begin{center}
|\ifchilddoc |\textit{child-code}| |[|\||else |\textit{main-code}]| \||fi|
\end{center}

%%%%%%%%%%%%%%%%%%%%%%%%%%%%%%%%%%%%%%%%
\DescribeMacro{\childdocname}
\DescribeMacro{\childdocjob}
The macro |\childdocname| contains the filename (without extension)
of the main or child file being processed.
Note that |\childdocjob| will always contain the name of the main file.

%%%%%%%%%%%%%%%%%%%%%%%%%%%%%%%%%%%%%%%%
\paragraph{Title Page.}

Conditional processing can be used to include a title or banner page
in the main document when proper precautions are taken.
Importantly, the code in the main file should ensure that the page counter
(as well as other status parameters which are stored in the |.aux| files)
takes the same value after the conditional processing.
Otherwise the page numbers may take divergent values
depending on which part is compiled.

For example, a title page could be declared by:
%
\begin{center}
\begin{tabular}{l}
|\ifchilddoc\||else|\\
|\addtocounter{page}{-1}|\\
\textit{code for title page}\\
|\newpage|\\
|\||fi|
\end{tabular}
\end{center}
%
A banner page for the child documents can be generated by:
%
\begin{center}
\begin{tabular}{l}
|\ifchilddoc|\\
|\addtocounter{page}{-1}|\\
\textit{code for banner page}\\
|\newpage|\\
|\||fi|
\end{tabular}
\end{center}
%
Here one could write a message such as:
\begin{center}
|This is the part \childdocname{} of \childdocjob{}.|
\end{center}

%%%%%%%%%%%%%%%%%%%%%%%%%%%%%%%%%%%%%%%%%%%%%%%%%%%%%%%%%%%%%%%%%%%%%%%%%%%%%%%%
\subsection{Flags}
\label{sec:flags}

The package makes it easy to generate different versions
of the main or child documents.
To this end compilation flags can be defined
and assigned different default values.
They will be particularly useful in conjunction
with the forwarding mechanism described in \secref{sec:forward}.

For example, it may be useful to have a flag |\version|
which can be set to |draft| or |final|.
The document source will contain some conditional code
depending on the value of |\version|.
Suppose further, the flag should default to |final| for the main file
and to |draft| for child files
which is a natural assignment for editing the document.
This is achieved by placing the following code
in the preamble of the main document
(below the |\childdocmain| directive):
%
\begin{center}
\begin{tabular}{l}
|\ifchilddoc|\\
|\providecommand{\version}{draft}|\\
|\||else|\\
|\providecommand{\version}{final}|\\
|\||fi|
\end{tabular}
\end{center}
%
The definition by |\providecommand| makes sure
that previous definitions are not overwritten.
Further statements |\providecommand{\version}{...}|
can thus be added before the above code to override it.

For the main file, one might add a line
(between |\childdocmain| and the above block)
%
\begin{center}
|%\ifchilddoc\||else\providecommand{\version}{draft}\||fi|
\end{center}
%
which can be uncommented to produce a draft version.
Likewise one can add a line to the very top of a child file
(above the |\childdocof{|\textit{main}|}| directive)
%
\begin{center}
|%\providecommand{\version}{final}|
\end{center}
%
which can be uncommented to produce the final version of this child document.

%%%%%%%%%%%%%%%%%%%%%%%%%%%%%%%%%%%%%%%%%%%%%%%%%%%%%%%%%%%%%%%%%%%%%%%%%%%%%%%%
\subsection{Forwarding}
\label{sec:forward}

Different versions of the main or child documents
using compilation flags as described in \secref{sec:flags}
can be (permanently) stored in different files
for convenient compilation, viewing and distribution.
To this end, the package defines a command
to pass on compilation to a different file:

%%%%%%%%%%%%%%%%%%%%%%%%%%%%%%%%%%%%%%%%
\DescribeMacro{\childdocforward}
The command |\childdocforward| redirects processing to
another source file:
%
\begin{center}
\begin{tabular}{l}
|\input{childdoc.def}|\\
|\childdocforward[|\textit{main}|]{|\textit{dest}|}|\\
\end{tabular}
\end{center}
%
The argument \textit{dest} is the destination file
(without extension).
It should be the main file or one of the child files.
Note that further \textsf{childdoc} directives
such as |\childdocof| and |\childdocforward|
in the indicated file will be processed in this form.
The optional argument \textit{main}
passes on directly to the main file \textit{main}
while pretending to compile the child \textit{dest}.
This form behaves as if \textit{dest}
issues |\childdocof{|\textit{main}|}| right away,
and no further \textsf{childdoc} directives will be processed.

%%%%%%%%%%%%%%%%%%%%%%%%%%%%%%%%%%%%%%%%
\DescribeMacro{\...prefix}
In the alternative form |\childdocforwardprefix|,
%
\begin{center}
\begin{tabular}{l}
|\input{childdoc.def}|\\
|\childdocforwardprefix[|\textit{main}|]{|\textit{prefix}|}{|\textit{dest}|}|
\end{tabular}
\end{center}
%
the destination file is determined by a pattern
depending on the current file:
To make this work, the current file must be called
`{\textit{prefix}\hspace{0.2em}\textit{suffix}}'
with \textit{prefix} matching precisely the argument.
Processing is then passed on to the file
`{\textit{dest}\hspace{0.2em}\textit{suffix}}'.
Surely, the same effect is achieved by
directly specifying the
argument `{\textit{dest}\hspace{0.2em}\textit{suffix}}'
in the first form.
However, that requires to set up a different file
for each child. With the alternative form of the command
all these files can have exactly the same content
which simplifies setting them up and maintaining them.

For example, the following file |draft.tex|
with a compilation flag |\version| as described in \secref{sec:flags}
compiles the main document as a draft:
%
\begin{center}
\begin{tabular}{l}
|\def\version{draft}|\\
|\input{childdoc.def}|\\
|\childdocforward{|\textit{main}|}|
\end{tabular}
\end{center}
%
Likewise, the following files |final|\textit{nn}|.tex|
compile the final version of the child document
|child|\textit{nn}|.tex|:
%
\begin{center}
\begin{tabular}{l}
|\def\version{final}|\\
|\input{childdoc.def}|\\
|\childdocforwardprefix{final}{child}|
\end{tabular}
\end{center}
%

Note that when several versions of a main file and/or of each child file
are to be generated, it may be convenient to set up a |Makefile| or
shell script to automatise the process.

%%%%%%%%%%%%%%%%%%%%%%%%%%%%%%%%%%%%%%%%%%%%%%%%%%%%%%%%%%%%%%%%%%%%%%%%%%%%%%%%
\subsection{Command Line Processing}
\label{sec:commandline}

The effect of redirection files can also be achieved by invoking
the \LaTeX{} compiler with a more elaborate command line.
Most conveniently this should be done as part
of a shell script or a |Makefile|.

When using \textsf{childdoc} in the main file, the following
command lines effectively perform a redirection
(note that depending on the shell being used,
backslashes may have to be doubled: `|\|' $\to$ `|\\|'):
%
\begin{center}
|... -jobname "|\textit{target}|" |\\|"|[\textit{flags}]%
|\input{childdoc.def}\childdocforward[|\textit{main}|]{|\textit{dest}|}"|
\end{center}
%
Here \textit{target} is the name of the output file,
\textit{main} is the name of the main file
and \textit{dest} is the name of the main or child file to be processed
(all filenames without extensions).
The optional argument \textit{main} can be omitted
if \textit{main} matches \textit{dest}.
Optionally, compilation \textit{flags} can be defined via |\def| commands.
This command line makes the \TeX{} engine believe
it is compiling the file \textit{target}
whose content is specified as the latter parameter.
The provided code then forwards the processing to
\textit{main} or \textit{dest} as described in \secref{sec:forward}.

%%%%%%%%%%%%%%%%%%%%%%%%%%%%%%%%%%%%%%%%%%%%%%%%%%%%%%%%%%%%%%%%%%%%%%%%%%%%%%%%
\subsection{Include by Input}
\label{sec:input}

Including child documents by |\include| has some restrictions by design.
Most notably, the content of a child document always occupies
its own set of pages; pages cannot be shared between child documents.
Usually, this behaviour makes perfect sense
because each child document contain an essential part of the document.
However, in some situations it may be desirable to compose
a document from a collection of parts
without having mandatory page breaks between then.
For this case, the package
provides a mechanism to include parts
by |\input| which can also be processed individually.
However, by construction this mechanism
requires manual handling of the content to be output.

%%%%%%%%%%%%%%%%%%%%%%%%%%%%%%%%%%%%%%%%
\DescribeMacro{\ifchilddocmanual}
The main file should be prepared as usual, see \secref{sec:include}.
However, the document body must make a distinction
between processing of an individual part and of the main document, e.g.:
%
\begin{center}
\begin{tabular}{l}
|\ifchilddocmanual|\\
|\input{\childdocname}|\\
|\||else|\\
\textit{document body with }|\input{|\textit{part}|}|\\
|\||fi|
\end{tabular}
\end{center}
%
The conditional |\ifchilddocmanual| is true whenever
a part to be included by |\input| is being compiled,
and the name of the part is stored in |\childdocname|.

%%%%%%%%%%%%%%%%%%%%%%%%%%%%%%%%%%%%%%%%
\DescribeMacro{\childdocby}
Each part to be included by |\input| should start with:
%
\begin{center}
\begin{tabular}{l}
|\input{childdoc.def}|\\
|\childdocby{|\textit{main}|}|\\
\end{tabular}
\end{center}
%
The directive |\childdocby| is similar to |\childdocof|
described in \secref{sec:include},
but the subsequent selection of content must be done manually.
To that end, both |\ifchilddoc| and |\ifchilddocmanual|
will be true upon processing of a part,
and the name of the part is stored in |\childdocname|.
Note that |\jobname| will be set to the filename of the current part
so that each part receives an individual |.aux| file
that does not interfere with the |.aux| file(s) of the main document.
This behaviour can be altered by the alternative form
|\childdocby[*]{|\textit{main}|}| (with a non-empty optional argument)
which uses the |.aux| file of the main document
by setting |\jobname| to \textit{main}.

%%%%%%%%%%%%%%%%%%%%%%%%%%%%%%%%%%%%%%%%%%%%%%%%%%%%%%%%%%%%%%%%%%%%%%%%%%%%%%%%
\subsection{Driver Development}
\label{sec:driver}

The \textsf{childdoc} mechanism can also be use for the development
of definition files such as \LaTeX{} styles or classes.
This case differs from the above setup with multiple parts
included by |\include| in that no |\includeonly| should be invoked.
This can be achieved by starting the include file
(before |\ProvidesPackage|) with:
%
\begin{center}
\begin{tabular}{l}
|\input{childdoc.def}|\\
|\childdocforward{|\textit{main}|}|\\
\end{tabular}
\end{center}
%
or alternatively with:
%
\begin{center}
\begin{tabular}{l}
|\input{childdoc.def}|\\
|\childdocby{|\textit{main}|}|\\
\end{tabular}
\end{center}
%
Both forms have slightly different effects as described above.
The main file is prepared as usual, see \secref{sec:include}.

%%%%%%%%%%%%%%%%%%%%%%%%%%%%%%%%%%%%%%%%%%%%%%%%%%%%%%%%%%%%%%%%%%%%%%%%%%%%%%%%
\subsection{Legacy Detection}
\label{sec:detection}

The directive |\childdocmain| in the main file can detect
whether the complete document or merely a child is to be compiled
even without using the directive |\childdocof|.
This method is deprecated because it is less robust
and there is no compelling reason to use it;
it is merely provided for backward compatibility
and it may be removed in future versions.

If the detection mechanism is to be used,
it is mandatory to correctly specify
the filename of the main file as the argument of |\childdocmain|:
%
\begin{center}
\begin{tabular}{l}
|\input{childdoc.def}|\\
|\childdocmain{|\textit{main}|}|\\
\end{tabular}
\end{center}
%
If |\jobname| does not match the argument \textit{main} of |\childdocmain|,
it is assumed that |\jobname| points to the child file to be compiled.
When using |\childdocmain| with the main file specified as argument,
it suffices to start a child file
with just |\input{|\textit{main}|}|
without loading of the package and using |\childdocof|.
If instead all processing is done
with the appropriate \textsf{childdoc} directives,
the argument of \textit{main} of |\childdocmain| can be empty.

An alternative version of the command line processing described
in \secref{sec:commandline} using the detection mechanism reads:
%
\begin{center}
|... -jobname "|\textit{target}|" "|[\textit{flags}]%
[|\def\jobname{|\textit{dest}|}|]|\input{|\textit{main}|}"|
\end{center}

%%%%%%%%%%%%%%%%%%%%%%%%%%%%%%%%%%%%%%%%%%%%%%%%%%%%%%%%%%%%%%%%%%%%%%%%%%%%%%%%
\subsection{Manual Code}
\label{sec:manual}

In case one cannot be certain whether the definitions file |childdoc.def|
is installed on the target \TeX{} distribution
and one prefers not to ship it,
it is conceivable to paste a few relevant commands into the sources.

To that end, drop all statements |\input{childdoc.def}|
and perform the replacements as outlined below.
Instead of |\childdocmain{|\textit{main}|}| add the following code
to the top of the main file:
%
\begin{center}
\begin{tabular}{l}
|\||ifdefined\childdocname\endinput\||fi\newif\ifchilddoc|\\
|\edef\childdocname{\scantokens\expandafter{\jobname\noexpand}}|\\
|\def\childdocmain{|\textit{main}|}\||ifx\childdocmain\childdocname\||else|\\
|\childdoctrue\includeonly{\childdocname}\let\jobname\childdocmain\||fi|\\
\end{tabular}
\end{center}
%
Instead of |\childdocof{|\textit{main}|}| just include the main file
at the top of each child file:
%
\begin{center}
|\input{|\textit{main}|}|
\end{center}
%
A simple redirection |\childdocforward{|\textit{dest}|}| is achieved by:
%
\begin{center}
|\def\jobname{|\textit{dest}|}\input{\jobname}|
\end{center}
%
The redirection with prefix
|\childdocforwardprefix[|\textit{prefix}|]{|\textit{dest}|}|
is accomplished by:
%
\begin{center}
\begin{tabular}{l}
|{\edef\jobname{\scantokens\expandafter{\jobname\noexpand}}|\\
|\def\redirectjob |\textit{prefix}|#1~~~{\gdef\jobname{|\textit{dest}|#1}}|\\
|\expandafter\redirectjob\jobname~~~}\input{\jobname}|
\end{tabular}
\end{center}

In an alternative approach,
child documents can be compiled by a specific command line
without additional code or specific definitions:
%
\begin{center}
|... -jobname "|\textit{target}|" "|[\textit{flags}]%
|\includeonly{|\textit{dest}|}\input{|\textit{main}|}"|
\end{center}
%

%%%%%%%%%%%%%%%%%%%%%%%%%%%%%%%%%%%%%%%%%%%%%%%%%%%%%%%%%%%%%%%%%%%%%%%%%%%%%%%%
%%%%%%%%%%%%%%%%%%%%%%%%%%%%%%%%%%%%%%%%%%%%%%%%%%%%%%%%%%%%%%%%%%%%%%%%%%%%%%%%
\section{Information}

%%%%%%%%%%%%%%%%%%%%%%%%%%%%%%%%%%%%%%%%%%%%%%%%%%%%%%%%%%%%%%%%%%%%%%%%%%%%%%%%
\subsection{Copyright}

Copyright \copyright{} 2017--2018 Niklas Beisert

This work may be distributed and/or modified under the
conditions of the \LaTeX{} Project Public License, either version 1.3
of this license or (at your option) any later version.
The latest version of this license is in
  \url{http://www.latex-project.org/lppl.txt}
and version 1.3 or later is part of all distributions of \LaTeX{}
version 2005/12/01 or later.

This work has the LPPL maintenance status `maintained'.

The Current Maintainer of this work is Niklas Beisert.

This work consists of the files |README.txt|, |childdoc.ins| and |childdoc.dtx|
as well as the derived files |childdoc.def|, |cdocsamp.tex|
with |cdocsch1.tex|, |cdocsch2.tex|, |cdocspt3.tex|, |cdocspt4.tex|,
|cdocsdrf.tex|, |cdocsfn1.tex|, |cdocsfn2.tex|
as well as |childdoc.pdf|.

%%%%%%%%%%%%%%%%%%%%%%%%%%%%%%%%%%%%%%%%%%%%%%%%%%%%%%%%%%%%%%%%%%%%%%%%%%%%%%%%
\subsection{Files and Installation}

The package consists of the files:
%
\begin{center}
\begin{tabular}{ll}
    |README.txt|   & readme file \\
    |childdoc.ins| & installation file \\
    |childdoc.dtx| & source file \\
    |childdoc.def| & definition file \\
    |cdocsamp.tex| & sample main file \\
    |cdocsch1.tex| & sample include file \\
    |cdocsch2.tex| & sample include file \\
    |cdocspt3.tex| & sample part file \\
    |cdocspt4.tex| & sample part file \\
    |cdocsdrf.tex| & sample redirection file \\
    |cdocsfn1.tex| & sample redirection file \\
    |cdocsfn2.tex| & sample redirection file \\
    |childdoc.pdf| & manual
\end{tabular}
\end{center}
%
The distribution consists of the files
|README.txt|, |childdoc.ins| and |childdoc.dtx|.
%
\begin{itemize}
\item
Run (pdf)\LaTeX{} on |childdoc.dtx|
to compile the manual |childdoc.pdf| (this file).
\item
Run \LaTeX{} on |childdoc.ins| to create the definitions file |childdoc.def|
and the sample |cdocsamp.tex| with include files
|cdocsch1.tex|, |cdocsch2.tex|, |cdocspt3.tex|, |cdocspt4.tex|,
|cdocsdrf.tex|, |cdocsfn1.tex|, |cdocsfn2.tex|.
Then copy the file |childdoc.def| to an appropriate directory of your \LaTeX{}
distribution, e.g.\ \textit{texmf-root}|/tex/latex/childdoc|.
\end{itemize}

%%%%%%%%%%%%%%%%%%%%%%%%%%%%%%%%%%%%%%%%%%%%%%%%%%%%%%%%%%%%%%%%%%%%%%%%%%%%%%%%
\subsection{Related CTAN Packages}

There are several other packages which offer a similar functionality:
%
\begin{itemize}
\item
The packages
\href{http://ctan.org/pkg/docmute}{\textsf{docmute}},
\href{http://ctan.org/pkg/includex}{\textsf{includex}} and
\href{http://ctan.org/pkg/standalone}{\textsf{standalone}}
provide commands to include only the document body of
a child file thus allowing both files to be compiled individually.
\item
The packages \href{http://ctan.org/pkg/subdocs}{\textsf{subdocs}}
and \href{http://ctan.org/pkg/subfiles}{\textsf{subfiles}}
provide structures in which the main and child documents can be
encapsulated and allowing them to be compiled individually.
The inclusion mechanism is different from the conventional |\include|.
\item
The package \href{http://ctan.org/pkg/combine}{\textsf{combine}}
is an elaborate solution to combine several documents into one.
\end{itemize}
%
See also the CTAN topic \href{http://ctan.org/topic/subdocs}{\textsf{subdocs}}
for further related packages.
The present package differs from the above solutions in that
a document structure constructed with the conventional |\include| mechanism
just needs two extra commands at the top of every file
such that all constituent files can be compiled individually.

%%%%%%%%%%%%%%%%%%%%%%%%%%%%%%%%%%%%%%%%%%%%%%%%%%%%%%%%%%%%%%%%%%%%%%%%%%%%%%%%
%\subsection{Feature Suggestions}
%
%The following is a list of features which may be useful for future
%versions of this package:
%%
%\begin{itemize}
%\item
%\ldots
%\end{itemize}

%%%%%%%%%%%%%%%%%%%%%%%%%%%%%%%%%%%%%%%%%%%%%%%%%%%%%%%%%%%%%%%%%%%%%%%%%%%%%%%%
\subsection{Revision History}

%%%%%%%%%%%%%%%%%%%%%%%%%%%%%%%%%%%%%%%%
\paragraph{v2.0:} 2018/12/30

\begin{itemize}
\item
immediate forward processing
\item
added |\childdocby| mechanism
\item
manual restructured
\end{itemize}

%%%%%%%%%%%%%%%%%%%%%%%%%%%%%%%%%%%%%%%%
\paragraph{v1.6:} 2018/01/17

\begin{itemize}
\item
application for development of include files
\item
corrections to manual
\end{itemize}

%%%%%%%%%%%%%%%%%%%%%%%%%%%%%%%%%%%%%%%%
\paragraph{v1.5:} 2017/05/21

\begin{itemize}
\item
more complete structuring introduced
\item
|\childdocof| introduced
\item
|\childdoc| renamed to |\childdocmain|
\item
|\childredirect| renamed to |\childdocforward| and |\childdocforwardprefix|
and functionality expanded
\end{itemize}

%%%%%%%%%%%%%%%%%%%%%%%%%%%%%%%%%%%%%%%%
\paragraph{v1.0:} 2017/04/27

\begin{itemize}
\item
manual and install package
\item
first version published on CTAN
\end{itemize}

%%%%%%%%%%%%%%%%%%%%%%%%%%%%%%%%%%%%%%%%
\paragraph{v0.6:} 2017/04/26

\begin{itemize}
\item
redirection mechanism added
\end{itemize}

%%%%%%%%%%%%%%%%%%%%%%%%%%%%%%%%%%%%%%%%
\paragraph{v0.5:} 2017/04/26

\begin{itemize}
\item
functionality in definition file
\end{itemize}


%%%%%%%%%%%%%%%%%%%%%%%%%%%%%%%%%%%%%%%%%%%%%%%%%%%%%%%%%%%%%%%%%%%%%%%%%%%%%%%%
%%%%%%%%%%%%%%%%%%%%%%%%%%%%%%%%%%%%%%%%%%%%%%%%%%%%%%%%%%%%%%%%%%%%%%%%%%%%%%%%
%%%%%%%%%%%%%%%%%%%%%%%%%%%%%%%%%%%%%%%%%%%%%%%%%%%%%%%%%%%%%%%%%%%%%%%%%%%%%%%%
\appendix

\settowidth\MacroIndent{\rmfamily\scriptsize 000\ }

 \DocInput{childdoc.dtx}

\end{document}
%</driver>
% \fi
%
% %%%%%%%%%%%%%%%%%%%%%%%%%%%%%%%%%%%%%%%%%%%%%%%%%%%%%%%%%%%%%%%%%%%%%%%%%%%%%%
% %%%%%%%%%%%%%%%%%%%%%%%%%%%%%%%%%%%%%%%%%%%%%%%%%%%%%%%%%%%%%%%%%%%%%%%%%%%%%%
% \section{Sample}
%\iffalse
%<*samplemain>
%\fi
%
% The following presents a sample document
% with two chapters, two parts, a title page,
% a compile flag as well as three forwarding files to set the flag.
% It consists of eight |.tex| files:
% \begin{center}
% \begin{tabular}{ll}
% |cdocsamp.tex|&main file\\
% |cdocsch1.tex|&include file for chapter 1\\
% |cdocsch2.tex|&include file for chapter 2\\
% |cdocspt3.tex|&include file for part 3\\
% |cdocspt4.tex|&include file for part 4\\
% |cdocsdrf.tex|&forwarding file for main file in draft mode\\
% |cdocsfi1.tex|&forwarding file for final version of chapter 1\\
% |cdocsfi2.tex|&forwarding file for final version of chapter 2\\
% \end{tabular}
% \end{center}
% Each of the eight files can be compiled directly by the \LaTeX{} compiler.
%
% %%%%%%%%%%%%%%%%%%%%%%%%%%%%%%%%%%%%%%
% \paragraph{Main File.}
%
% The main file is called |cdocsamp.tex|.
%
% Load the \textsf{childdoc} definitions and
% declare the filename for the main document:
%    \begin{macrocode}
\input{childdoc.def}
\childdocmain{}
%    \end{macrocode}

% Optional override for |\version| flag:
%    \begin{macrocode}
%%\ifchilddoc\else\providecommand{\version}{draft}\fi
%    \end{macrocode}

% Define the default values for the |\version| flag
% (|final| for the main file and |draft| for childs):
%    \begin{macrocode}
\ifchilddoc
\providecommand{\version}{draft}
\else
\providecommand{\version}{final}
\fi
%    \end{macrocode}

% Load the standard document class:
%    \begin{macrocode}
\documentclass[12pt]{article}
%    \end{macrocode}

% Start the document body:
%    \begin{macrocode}
\begin{document}
%    \end{macrocode}

% Declare a title page.
% Print title, part of document being processed and version flag:
%    \begin{macrocode}
\addtocounter{page}{-1}
\begin{center}
{\LARGE\bfseries{}childdoc example\par}
\vspace{1cm}
\ifchilddoc
\ifchilddocmanual part\else chapter\fi:
`\childdocname' of `\childdocjob'\par
\else
main document: `\childdocjob'\par
\fi
version: \version\par
\end{center}
\newpage
%    \end{macrocode}

% Manually include selected file,
% otherwise process as usual:
%    \begin{macrocode}
\ifchilddocmanual
\section*{part `\childdocname'}
\input{\childdocname}
\else
%    \end{macrocode}

% Include the two chapters:
%    \begin{macrocode}
\include{cdocsch1}
\include{cdocsch2}
%    \end{macrocode}

% Include the two parts unless only chapters should be displayed:
%    \begin{macrocode}
\ifchilddoc\else
\section{part three}
\input{cdocspt3}
\section{part four}
\input{cdocspt4}
\fi
%    \end{macrocode}

% Process as usual until here:
%    \begin{macrocode}
\fi
%    \end{macrocode}

% End of document body:
%    \begin{macrocode}
\end{document}
%    \end{macrocode}
%\iffalse
%</samplemain>
%\fi
%
% %%%%%%%%%%%%%%%%%%%%%%%%%%%%%%%%%%%%%%
% \paragraph{Chapter Include Files.}
%
% The include files are called |cdocsch1.tex| and |cdocsch2.tex|.
%
%\iffalse
%<*samplechap1|samplechap2>
%\fi

% Optional override for |\version| flag:
%    \begin{macrocode}
%%\providecommand{\version}{final}
%    \end{macrocode}

% Include the main document:
%    \begin{macrocode}
\input{childdoc.def}
\childdocof{cdocsamp}
%    \end{macrocode}

%\iffalse
%</samplechap1|samplechap2>
%\fi
%
%\iffalse
%<*samplechap1>
%\fi
% Some text for chapter 1:
%    \begin{macrocode}
\section{one}
some text in chapter one
%    \end{macrocode}

%\iffalse
%</samplechap1>
%\fi
% Some text for chapter 2:
%\iffalse
%<*samplechap2>
%\fi
%    \begin{macrocode}
\section{two}
more text in chapter two
%    \end{macrocode}

%\iffalse
%</samplechap2>
%\fi
%
% %%%%%%%%%%%%%%%%%%%%%%%%%%%%%%%%%%%%%%
% \paragraph{Part Include Files.}
%
% The include files are called |cdocspt3.tex| and |cdocspt4.tex|.
%
%\iffalse
%<*samplepart3|samplepart4>
%\fi

% Optional override for |\version| flag:
%    \begin{macrocode}
%%\providecommand{\version}{final}
%    \end{macrocode}

% Include the main document:
%    \begin{macrocode}
\input{childdoc.def}
\childdocby{cdocsamp}
%    \end{macrocode}

%\iffalse
%</samplepart3|samplepart4>
%\fi
%
%\iffalse
%<*samplepart3>
%\fi
% Some text for part 3:
%    \begin{macrocode}
some text in part three
%    \end{macrocode}

%\iffalse
%</samplepart3>
%\fi
% Some text for part 4:
%\iffalse
%<*samplepart4>
%\fi
%    \begin{macrocode}
more text in part four
%    \end{macrocode}

%\iffalse
%</samplepart4>
%\fi
%
% %%%%%%%%%%%%%%%%%%%%%%%%%%%%%%%%%%%%%%
% \paragraph{Forwarding for a Complete Draft.}
%
% The following forwarding file |cdocsdrf.tex|
% compiles the main document in draft mode:
%\iffalse
%<*sampledraft>
%\fi
%    \begin{macrocode}
\def\version{draft}
\input{childdoc.def}
\childdocforward{cdocsamp}
%    \end{macrocode}

%\iffalse
%</sampledraft>
%\fi
%
% %%%%%%%%%%%%%%%%%%%%%%%%%%%%%%%%%%%%%%
% \paragraph{Forwarding for Final Version of the Chapters.}
%
% The following forwarding files |cdocsfn1.tex| and |cdocsfn2.tex|
% (with identical content)
% compile the final versions of the child documents
% |cdocsch1.tex| and |cdocsch2.tex|, respectively:
%\iffalse
%<*samplefinal>
%\fi
%    \begin{macrocode}
\def\version{final}
\input{childdoc.def}
\childdocforwardprefix[cdocsamp]{cdocsfn}{cdocsch}
%    \end{macrocode}

%\iffalse
%</samplefinal>
%\fi
%
% %%%%%%%%%%%%%%%%%%%%%%%%%%%%%%%%%%%%%%
% \paragraph{Command Line Processing.}
%
% The following three command lines generate the output files
% |cdocscld|, |cdocscl1| and |cdocscl2|
% which should be identical to
% |cdocsdrf|, |cdocsch1| and |cdocsfn2|, respectively:
% \begin{center}
% \begin{tabular}{l}
% |latex -jobname cdocscld \|\\
% |  "\def\version{draft}\input{childdoc.def}\childdocforward{cdocsamp}"|\\
% |latex -jobname cdocscl1 \|\\
% |  "\input{childdoc.def}\childdocforward[cdocsamp]{cdocsch1}"|\\
% |latex -jobname cdocscl2 \|\\
% |  "\def\version{final}\input{childdoc.def}\childdocforward{cdocsch2}"|
% \end{tabular}
% \end{center}
% Note that the trailing backslash on each first line
% merely continues the input to the second line
% (for convenient cut ant paste).
% Furthermore, the command |latex| can be replaced by any
% of its alternative versions such as |pdflatex|.
%
% %%%%%%%%%%%%%%%%%%%%%%%%%%%%%%%%%%%%%%%%%%%%%%%%%%%%%%%%%%%%%%%%%%%%%%%%%%%%%%
% %%%%%%%%%%%%%%%%%%%%%%%%%%%%%%%%%%%%%%%%%%%%%%%%%%%%%%%%%%%%%%%%%%%%%%%%%%%%%%
% \section{Implementation}
%\iffalse
%<*package>
%\fi
%
% This section describes the definitions file |childdoc.def|.

% The definitions cannot be loaded using |\usepackage| or |\RequirePackage|
% which has a mechanism to prevent loading a style file more than once.
% When loading the definitions by means of |\input|
% multiple instances have to be prevented manually:
%\iffalse
%This code needs to be before the `\ProvidesFile' directive
%which is defined at the beginning of this file.
%Therefore it is also placed there and commented out here.
%</package>
%<*discard>
%\fi
%    \begin{macrocode}
\ifdefined\childdocmain\endinput\fi
%    \end{macrocode}
%\iffalse
%</discard>
%<*package>
%\fi
%
% \macro{\ifchilddoc}
% \macro{\ifchilddocmanual}
% The conditional |\ifchilddoc| tells whether a
% child (true) or main (false) document is being compiled.
% The conditional |\ifchilddocmanual| tells whether
% the |\includeonly| mechanism is used (false) or
% the selection of child files must be performed manually (true).
% The definitions initialise to false:
%    \begin{macrocode}
\newif\ifchilddoc
\newif\ifchilddocmanual
%    \end{macrocode}

% \macro{\childdocname}
% \macro{\childdocjob}
% The macro |\childdocname| stores the name of the main document
% to be compiled. The macro |\childdocjob| stores the name of
% the document on which the \LaTeX{} compiler was originally invoked.
% The content of |\jobname| cannot be compared
% to filenames specified in the source due to different catcodes.
% The following code rescans |\jobname|, stores the result
% in |\childdocname| and saves a copy in |\childdocjob|:
%    \begin{macrocode}
\edef\childdocname{\scantokens\expandafter{\jobname\noexpand}}
\let\childdocjob\childdocname
%    \end{macrocode}

% \macro{\childdocdisable}
% The macro |\childdocdisable| prevents the main file
% from being processed more than once.
% At this stage, the main document command |\childdocmain|
% is assumed to be called once again where it should do nothing.
% Any subsequent call to it should prevent
% a secondary processing of the main document
% It overwrites the forwarding commands
% |\childdocof| and |\childdocforward|
% with empty macros to prevent further inclusions of the main document:
%    \begin{macrocode}
\newcommand{\childdocdisable}
{
  \renewcommand{\childdocmain}[1]{\renewcommand{\childdocmain}[1]{\endinput}}
  \renewcommand{\childdocof}[1]{}
  \renewcommand{\childdocby}[2][]{}
  \renewcommand{\childdocforward}[2][]{}
  \renewcommand{\childdocdisable}{}
}
%    \end{macrocode}

% \macro{\childdocmain}
% The macro |\childdocmain| is to be called at the top of the main file
% with nothing or the main filename (without extension) as argument.
% First, it breaks loops.
% If the argument is not empty and does not match |\childdocname|
% (which is set by the first inclusion of |childdoc.def|),
% |\ifchilddoc| is set to true, |\includeonly| is applied to the child file
% and |\jobname| is set to the main file
% (for proper handling of |.aux| files):
%    \begin{macrocode}
\newcommand{\childdocmain}[1]
{
  \childdocdisable\childdocmain{}
  \if?#1?\else
    \begingroup
      \def\childdoctmp{#1}
      \ifx\childdoctmp\childdocname
        \def\childdoctmp{}
      \else
        \def\childdoctmp
        {
          \childdoctrue
          \includeonly{\childdocname}
          \def\childdocjob{#1}
          \def\jobname{#1}
        }
      \fi
      \expandafter
    \endgroup
    \childdoctmp
  \fi
}
%    \end{macrocode}

% \macro{\childdocof}
% The command |\childdocof| redirects
% compilation to the main file |#1|.
%    \begin{macrocode}
\newcommand{\childdocof}[1]
{
  \childdocdisable
  \childdoctrue
  \includeonly{\childdocname}
  \def\jobname{#1}
  \def\childdocjob{#1}
  \input{#1}
}
%    \end{macrocode}

% \macro{\childdocby}
% The command |\childdocby| ....
%    \begin{macrocode}
\newcommand{\childdocby}[2][]
{
  \childdocdisable
  \childdoctrue
  \childdocmanualtrue
  \if?#1?\else
    \def\jobname{#2}
  \fi
  \def\childdocjob{#2}
  \input{#2}
  \endinput
}
%    \end{macrocode}

% \macro{\childdocforward}
% The command |\childdocforward| redirects
% compilation to the main file or
% (if the optional argument is given) a child file.
% Parameters are set as if the main file
% or a child file starting with |\childdocof| was compiled.
% Then compilation is handed over to the main file:
%    \begin{macrocode}
\newcommand{\childdocforward}[2][]
{
  \begingroup
    \if?#1?
      \def\childdoctmp
      {
        \def\childdocname{#2}
        \def\childdocjob{#2}
        \def\jobname{#2}
        \input{#2}
        \endinput
      }
    \else
      \def\childdoctmp
      {
        \childdocdisable
        \def\childdocname{#2}
        \childdoctrue
        \includeonly{#2}
        \def\childdocjob{#1}
        \def\jobname{#1}
        \input{#1}
        \endinput
      }
    \fi
    \expandafter
  \endgroup
  \childdoctmp
}
%    \end{macrocode}

% \macro{\childdocforwardprefix}
% The command |\childdocforwardprefix| redirects
% compilation to the main or a child file by means of a pattern.
% The prefix |#1| in the current filename is replaced by |#2|
% and the suffix of the current filename is kept
% (it is assumed that the filename does not contain the substring `|~~~|'
% which is used as a delimiter).
% Compilation is handed over to the new file by |\childdocforward|:
%    \begin{macrocode}
\newcommand{\childdocforwardprefix}[3][]
{
  \begingroup
    \def\childdocextract #2##1~~~{\def\childdoctmp{\childdocforward[#1]{#3##1}}}
    \expandafter\childdocextract\childdocname~~~
    \expandafter
  \endgroup
  \childdoctmp
}
%    \end{macrocode}

% \macro{\childdoc}
% The deprecated macro |\childdoc| is a legacy version of |\childdocmain|:
%    \begin{macrocode}
\newcommand{\childdoc}{\childdocmain}
%    \end{macrocode}

% \macro{\childdocredirect}
% The deprecated macro |\childdocredirect| is a legacy version
% of |\childdocforward| and |\childdocforwardprefix|:
%    \begin{macrocode}
\newcommand{\childdocredirect}[2][]
{
  \begingroup
    \if?#1?
      \def\childdoctmp{\childdocforward{#2}}
    \else
      \def\childdoctmp{\childdocforwardprefix{#1}{#2}}
    \fi
    \expandafter
  \endgroup
  \childdoctmp
}
%    \end{macrocode}

%\iffalse
%</package>
%\fi
%
\endinput
\childdocforward[|\textit{main}|]{|\textit{dest}|}"|
\end{center}
%
Here \textit{target} is the name of the output file,
\textit{main} is the name of the main file
and \textit{dest} is the name of the main or child file to be processed
(all filenames without extensions).
The optional argument \textit{main} can be omitted
if \textit{main} matches \textit{dest}.
Optionally, compilation \textit{flags} can be defined via |\def| commands.
This command line makes the \TeX{} engine believe
it is compiling the file \textit{target}
whose content is specified as the latter parameter.
The provided code then forwards the processing to
\textit{main} or \textit{dest} as described in \secref{sec:forward}.

%%%%%%%%%%%%%%%%%%%%%%%%%%%%%%%%%%%%%%%%%%%%%%%%%%%%%%%%%%%%%%%%%%%%%%%%%%%%%%%%
\subsection{Include by Input}
\label{sec:input}

Including child documents by |\include| has some restrictions by design.
Most notably, the content of a child document always occupies
its own set of pages; pages cannot be shared between child documents.
Usually, this behaviour makes perfect sense
because each child document contain an essential part of the document.
However, in some situations it may be desirable to compose
a document from a collection of parts
without having mandatory page breaks between then.
For this case, the package
provides a mechanism to include parts
by |\input| which can also be processed individually.
However, by construction this mechanism
requires manual handling of the content to be output.

%%%%%%%%%%%%%%%%%%%%%%%%%%%%%%%%%%%%%%%%
\DescribeMacro{\ifchilddocmanual}
The main file should be prepared as usual, see \secref{sec:include}.
However, the document body must make a distinction
between processing of an individual part and of the main document, e.g.:
%
\begin{center}
\begin{tabular}{l}
|\ifchilddocmanual|\\
|\input{\childdocname}|\\
|\||else|\\
\textit{document body with }|\input{|\textit{part}|}|\\
|\||fi|
\end{tabular}
\end{center}
%
The conditional |\ifchilddocmanual| is true whenever
a part to be included by |\input| is being compiled,
and the name of the part is stored in |\childdocname|.

%%%%%%%%%%%%%%%%%%%%%%%%%%%%%%%%%%%%%%%%
\DescribeMacro{\childdocby}
Each part to be included by |\input| should start with:
%
\begin{center}
\begin{tabular}{l}
|% \iffalse
%
% childdoc.dtx Copyright (C) 2017-2018 Niklas Beisert
%
% This work may be distributed and/or modified under the
% conditions of the LaTeX Project Public License, either version 1.3
% of this license or (at your option) any later version.
% The latest version of this license is in
%   http://www.latex-project.org/lppl.txt
% and version 1.3 or later is part of all distributions of LaTeX
% version 2005/12/01 or later.
%
% This work has the LPPL maintenance status `maintained'.
%
% The Current Maintainer of this work is Niklas Beisert.
%
% This work consists of the files childdoc.dtx and childdoc.ins
% and the derived files childdoc.def and cdocsamp.tex with
% cdocsch1.tex, cdocsch2.tex, cdocsdrf.tex, cdocsfn1.tex, cdocsfn2.tex.
%
%<package>\ifdefined\childdocmain\endinput\fi
%<package>\ProvidesFile{childdoc.def}[2018/12/30 v2.0 child document driver]
%<samplemain>\ProvidesFile{cdocsamp.tex}[2018/12/30 v2.0 sample for childdoc]
%<*driver>
%\ProvidesFile{childdoc.drv}[2018/12/30 v2.0 childdoc reference manual file]
\PassOptionsToClass{10pt,a4paper}{article}
\documentclass{ltxdoc}

\usepackage[margin=35mm]{geometry}
\usepackage{hyperref}
\usepackage{hyperxmp}
\usepackage[usenames]{color}

\hypersetup{colorlinks=true}
\hypersetup{pdfstartview=FitH}
\hypersetup{pdfpagemode=UseNone}
\hypersetup{pdfsource={}}
\hypersetup{pdflang={en-UK}}
\hypersetup{pdfcopyright={Copyright 2017-2018 Niklas Beisert.
  This work may be distributed and/or modified under the
  conditions of the LaTeX Project Public License, either version 1.3
  of this license or (at your option) any later version.}}
\hypersetup{pdflicenseurl={http://www.latex-project.org/lppl.txt}}
\hypersetup{pdfcontactaddress={ETH Zurich, ITP, HIT K,
  Wolfgang-Pauli-Strasse 27}}
\hypersetup{pdfcontactpostcode={8093}}
\hypersetup{pdfcontactcity={Zurich}}
\hypersetup{pdfcontactcountry={Switzerland}}
\hypersetup{pdfcontactemail={nbeisert@itp.phys.ethz.ch}}
\hypersetup{pdfcontacturl={http://people.phys.ethz.ch/\xmptilde nbeisert/}}

\newcommand{\secref}[1]{\hyperref[#1]{section \ref*{#1}}}

\parskip1ex
\parindent0pt
\let\olditemize\itemize
\def\itemize{\olditemize\parskip0pt}

\begin{document}

\title{The \textsf{childdoc} Package}
\hypersetup{pdftitle={The childdoc Package}}
\author{Niklas Beisert\\[2ex]
  Institut f\"ur Theoretische Physik\\
  Eidgen\"ossische Technische Hochschule Z\"urich\\
  Wolfgang-Pauli-Strasse 27, 8093 Z\"urich, Switzerland\\[1ex]
  \href{mailto:nbeisert@itp.phys.ethz.ch}
  {\texttt{nbeisert@itp.phys.ethz.ch}}}
\hypersetup{pdfauthor={Niklas Beisert}}
\hypersetup{pdfsubject={Manual for the LaTeX2e Package childdoc}}
\date{30 December 2018, \textsf{v2.0}}
\maketitle

\begin{abstract}\noindent
\textsf{childdoc} is a \LaTeXe{} package
that enables the direct compilation
of document sections included by |\include|
to individual files.
\end{abstract}

\begingroup
\parskip0ex
\tableofcontents
\endgroup

%%%%%%%%%%%%%%%%%%%%%%%%%%%%%%%%%%%%%%%%%%%%%%%%%%%%%%%%%%%%%%%%%%%%%%%%%%%%%%%%
%%%%%%%%%%%%%%%%%%%%%%%%%%%%%%%%%%%%%%%%%%%%%%%%%%%%%%%%%%%%%%%%%%%%%%%%%%%%%%%%
\section{Introduction}

\LaTeX{} provides a mechanism to structure a large document (such as a book)
into a main file and several child files (containing the chapters)
using the |\include| command.
This mechanism is beneficial for documents
which span hundreds of pages in order to
make the source file(s) more manageable.
Moreover, compilation can be restricted to
selected child files by means of the |\includeonly| command.
The latter feature can be used to reduce the compilation time while editing
(this was significantly more useful in the earlier days of \LaTeX{})
or to generate a smaller document which is easier to navigate.
Another application of |\includeonly| is to generate
documents consisting of selected parts of the complete document.

However, there are a few drawbacks of the plain |\include| mechanism:
\begin{itemize}
\item
The child files cannot be compiled on their own,
they can only be compiled via the main file.
A naive editing environment
(such as a text editor with an option
to have the current file processed by \LaTeX)
may require one to switch to the main file before compiling;
attempting to compile the child file produces errors.
\item
The main file must be modified (each time)
to adjust the |\includeonly| command
to the present needs. This easily leaves the main file in a messy state.
\item
The generated document will always carry the filename
of the main document. This is inconvenient if
several child files are to be compiled and
to be kept for distribution.
\end{itemize}

The present package provides a simple interface
to make child files individually compilable by \LaTeX{}.
Compiling a child file then has the same effect as compiling
the main file with an |\includeonly| command
to select the appropriate child.
Moreover the generated document will carry the name of the child
rather than the main file.
This resolves all three above issues.

This feature is meant to make the editing of books,
thesis documents and lecture notes somewhat more convenient.
However, the package can also be used efficiently for
composing a series of documents (such as exercise sheets)
which are typically distributed individually.
It then assists the author in generating the individual documents
(potentially in different versions)
as well as a document containing the collected series.
Another application is in developing style files
or other kinds of included material
where compilation of the style file could redirect
to a sample or test file.

%%%%%%%%%%%%%%%%%%%%%%%%%%%%%%%%%%%%%%%%%%%%%%%%%%%%%%%%%%%%%%%%%%%%%%%%%%%%%%%%
%%%%%%%%%%%%%%%%%%%%%%%%%%%%%%%%%%%%%%%%%%%%%%%%%%%%%%%%%%%%%%%%%%%%%%%%%%%%%%%%
\section{Usage}

First of all, the package \textsf{childdoc} is \emph{not} a standard
\LaTeXe{} |.sty| style file! Therefore it needs to be invoked in
a non-standard way.

%%%%%%%%%%%%%%%%%%%%%%%%%%%%%%%%%%%%%%%%%%%%%%%%%%%%%%%%%%%%%%%%%%%%%%%%%%%%%%%%
\subsection{Included Files}
\label{sec:include}

%%%%%%%%%%%%%%%%%%%%%%%%%%%%%%%%%%%%%%%%
\DescribeMacro{\childdocmain}
To use the package, add the commands
\begin{center}
\begin{tabular}{l}
|\input{childdoc.def}|\\
|\childdocmain{}|\\
\end{tabular}
\end{center}
at the very top of the main \LaTeX{} file,
in particular \emph{before} the |\documentclass| statement!
The argument of |\childdocmain| should be left empty
(but it must be present).

%%%%%%%%%%%%%%%%%%%%%%%%%%%%%%%%%%%%%%%%
\DescribeMacro{\childdocof}
Furthermore, add the commands
\begin{center}
\begin{tabular}{l}
|\input{childdoc.def}|\\
|\childdocof{|\textit{main}|}|\\
\end{tabular}
\end{center}
at the top of every child file \textit{child}
which is included by |\include{|\textit{child}|}|
from within the main file
(or at least for those files to be compiled individually).
The argument \textit{main} must be the filename of the main file.

There are a couple of
considerations in setting up the main and child documents:

%%%%%%%%%%%%%%%%%%%%%%%%%%%%%%%%%%%%%%%%
\paragraph{Restrictions.}

Please note the following restrictions:
\begin{itemize}
\item
|\childdocmain| must be called with one argument \textit{main}
to ensure compatibility with earlier version of the package.
It must either be empty (|\childdocmain{}|)
or precisely match the filename of the main file in which it is specified.
See \secref{sec:detection} for further information.
\item
The filename \textit{main} must be specified without the |.tex| extension.
\item
The filename \textit{main} is case sensitive
(even in case-insensitive file systems)
due to internal string comparison.
\item
The argument \textit{main} should be fully expanded, it cannot be a macro.
\item
Subdirectories and special characters should be avoided in filenames.
\item
The command |\childdocmain{|\textit{main}|}| must be followed by a whitespace.
It should not be followed immediately by another command
or by a comment mark `|%|'.
This is because the \TeX{} parser reads the token immediately following
the argument of |\childdocmain| and puts it
at the beginning of every child section;
however, a white\-space is ignored.
\end{itemize}

%%%%%%%%%%%%%%%%%%%%%%%%%%%%%%%%%%%%%%%%
\paragraph{Content of Main File.}

It is advisable to place all content in the child files included by |\include|.
Any output contained in the main file will appear in all child documents
unless suppressed manually;
it cannot be suppressed automatically by the |\includeonly| directive
and thus should normally be avoided.
A method to include some content in the main file
by means of conditional processing is described in \secref{sec:conditional}.

%%%%%%%%%%%%%%%%%%%%%%%%%%%%%%%%%%%%%%%%
\paragraph{Page Numbering.}

When only a part of the document is compiled,
the appropriate numbering of pages
(as well as other status parameters)
is determined from the |.aux| files.
The latter contain information from previous passes.
However this information needs to propagate through
all intermediate child documents.
Therefore the page numbering in child documents may well
be inconsistent until the complete document is compiled at least once.

A useful (if unconventional) way to always ensure a consistent
page numbering is to restart the numbering in each child document
and denote the pages by `\textit{child}|.|\textit{page}'
where \textit{child} represents the chapter/section number of the child file.
This can be achieved by the command
|\numberwithin{page}{|\textit{child}|}|
of the \textsf{amsmath} package
where \textit{child} can be |chapter| or |section|
depending on the chosen structuring.
Alternatively, one can modify the macro |\thepage| appropriately
and reset the counter |page| at the start of each child file.

%%%%%%%%%%%%%%%%%%%%%%%%%%%%%%%%%%%%%%%%%%%%%%%%%%%%%%%%%%%%%%%%%%%%%%%%%%%%%%%%
\subsection{Conditional Processing}
\label{sec:conditional}

The package provides a mechanism to compile different versions
of a document. To customise the versions further some conditional processing
can come in handy to distinguish which version is being compiled.
The package provides two macros to describe the compilation context:

%%%%%%%%%%%%%%%%%%%%%%%%%%%%%%%%%%%%%%%%
\DescribeMacro{\ifchilddoc}
The conditional |\ifchilddoc| distinguishes between the compilation of
child documents and the main document:
%
\begin{center}
|\ifchilddoc |\textit{child-code}| |[|\||else |\textit{main-code}]| \||fi|
\end{center}

%%%%%%%%%%%%%%%%%%%%%%%%%%%%%%%%%%%%%%%%
\DescribeMacro{\childdocname}
\DescribeMacro{\childdocjob}
The macro |\childdocname| contains the filename (without extension)
of the main or child file being processed.
Note that |\childdocjob| will always contain the name of the main file.

%%%%%%%%%%%%%%%%%%%%%%%%%%%%%%%%%%%%%%%%
\paragraph{Title Page.}

Conditional processing can be used to include a title or banner page
in the main document when proper precautions are taken.
Importantly, the code in the main file should ensure that the page counter
(as well as other status parameters which are stored in the |.aux| files)
takes the same value after the conditional processing.
Otherwise the page numbers may take divergent values
depending on which part is compiled.

For example, a title page could be declared by:
%
\begin{center}
\begin{tabular}{l}
|\ifchilddoc\||else|\\
|\addtocounter{page}{-1}|\\
\textit{code for title page}\\
|\newpage|\\
|\||fi|
\end{tabular}
\end{center}
%
A banner page for the child documents can be generated by:
%
\begin{center}
\begin{tabular}{l}
|\ifchilddoc|\\
|\addtocounter{page}{-1}|\\
\textit{code for banner page}\\
|\newpage|\\
|\||fi|
\end{tabular}
\end{center}
%
Here one could write a message such as:
\begin{center}
|This is the part \childdocname{} of \childdocjob{}.|
\end{center}

%%%%%%%%%%%%%%%%%%%%%%%%%%%%%%%%%%%%%%%%%%%%%%%%%%%%%%%%%%%%%%%%%%%%%%%%%%%%%%%%
\subsection{Flags}
\label{sec:flags}

The package makes it easy to generate different versions
of the main or child documents.
To this end compilation flags can be defined
and assigned different default values.
They will be particularly useful in conjunction
with the forwarding mechanism described in \secref{sec:forward}.

For example, it may be useful to have a flag |\version|
which can be set to |draft| or |final|.
The document source will contain some conditional code
depending on the value of |\version|.
Suppose further, the flag should default to |final| for the main file
and to |draft| for child files
which is a natural assignment for editing the document.
This is achieved by placing the following code
in the preamble of the main document
(below the |\childdocmain| directive):
%
\begin{center}
\begin{tabular}{l}
|\ifchilddoc|\\
|\providecommand{\version}{draft}|\\
|\||else|\\
|\providecommand{\version}{final}|\\
|\||fi|
\end{tabular}
\end{center}
%
The definition by |\providecommand| makes sure
that previous definitions are not overwritten.
Further statements |\providecommand{\version}{...}|
can thus be added before the above code to override it.

For the main file, one might add a line
(between |\childdocmain| and the above block)
%
\begin{center}
|%\ifchilddoc\||else\providecommand{\version}{draft}\||fi|
\end{center}
%
which can be uncommented to produce a draft version.
Likewise one can add a line to the very top of a child file
(above the |\childdocof{|\textit{main}|}| directive)
%
\begin{center}
|%\providecommand{\version}{final}|
\end{center}
%
which can be uncommented to produce the final version of this child document.

%%%%%%%%%%%%%%%%%%%%%%%%%%%%%%%%%%%%%%%%%%%%%%%%%%%%%%%%%%%%%%%%%%%%%%%%%%%%%%%%
\subsection{Forwarding}
\label{sec:forward}

Different versions of the main or child documents
using compilation flags as described in \secref{sec:flags}
can be (permanently) stored in different files
for convenient compilation, viewing and distribution.
To this end, the package defines a command
to pass on compilation to a different file:

%%%%%%%%%%%%%%%%%%%%%%%%%%%%%%%%%%%%%%%%
\DescribeMacro{\childdocforward}
The command |\childdocforward| redirects processing to
another source file:
%
\begin{center}
\begin{tabular}{l}
|\input{childdoc.def}|\\
|\childdocforward[|\textit{main}|]{|\textit{dest}|}|\\
\end{tabular}
\end{center}
%
The argument \textit{dest} is the destination file
(without extension).
It should be the main file or one of the child files.
Note that further \textsf{childdoc} directives
such as |\childdocof| and |\childdocforward|
in the indicated file will be processed in this form.
The optional argument \textit{main}
passes on directly to the main file \textit{main}
while pretending to compile the child \textit{dest}.
This form behaves as if \textit{dest}
issues |\childdocof{|\textit{main}|}| right away,
and no further \textsf{childdoc} directives will be processed.

%%%%%%%%%%%%%%%%%%%%%%%%%%%%%%%%%%%%%%%%
\DescribeMacro{\...prefix}
In the alternative form |\childdocforwardprefix|,
%
\begin{center}
\begin{tabular}{l}
|\input{childdoc.def}|\\
|\childdocforwardprefix[|\textit{main}|]{|\textit{prefix}|}{|\textit{dest}|}|
\end{tabular}
\end{center}
%
the destination file is determined by a pattern
depending on the current file:
To make this work, the current file must be called
`{\textit{prefix}\hspace{0.2em}\textit{suffix}}'
with \textit{prefix} matching precisely the argument.
Processing is then passed on to the file
`{\textit{dest}\hspace{0.2em}\textit{suffix}}'.
Surely, the same effect is achieved by
directly specifying the
argument `{\textit{dest}\hspace{0.2em}\textit{suffix}}'
in the first form.
However, that requires to set up a different file
for each child. With the alternative form of the command
all these files can have exactly the same content
which simplifies setting them up and maintaining them.

For example, the following file |draft.tex|
with a compilation flag |\version| as described in \secref{sec:flags}
compiles the main document as a draft:
%
\begin{center}
\begin{tabular}{l}
|\def\version{draft}|\\
|\input{childdoc.def}|\\
|\childdocforward{|\textit{main}|}|
\end{tabular}
\end{center}
%
Likewise, the following files |final|\textit{nn}|.tex|
compile the final version of the child document
|child|\textit{nn}|.tex|:
%
\begin{center}
\begin{tabular}{l}
|\def\version{final}|\\
|\input{childdoc.def}|\\
|\childdocforwardprefix{final}{child}|
\end{tabular}
\end{center}
%

Note that when several versions of a main file and/or of each child file
are to be generated, it may be convenient to set up a |Makefile| or
shell script to automatise the process.

%%%%%%%%%%%%%%%%%%%%%%%%%%%%%%%%%%%%%%%%%%%%%%%%%%%%%%%%%%%%%%%%%%%%%%%%%%%%%%%%
\subsection{Command Line Processing}
\label{sec:commandline}

The effect of redirection files can also be achieved by invoking
the \LaTeX{} compiler with a more elaborate command line.
Most conveniently this should be done as part
of a shell script or a |Makefile|.

When using \textsf{childdoc} in the main file, the following
command lines effectively perform a redirection
(note that depending on the shell being used,
backslashes may have to be doubled: `|\|' $\to$ `|\\|'):
%
\begin{center}
|... -jobname "|\textit{target}|" |\\|"|[\textit{flags}]%
|\input{childdoc.def}\childdocforward[|\textit{main}|]{|\textit{dest}|}"|
\end{center}
%
Here \textit{target} is the name of the output file,
\textit{main} is the name of the main file
and \textit{dest} is the name of the main or child file to be processed
(all filenames without extensions).
The optional argument \textit{main} can be omitted
if \textit{main} matches \textit{dest}.
Optionally, compilation \textit{flags} can be defined via |\def| commands.
This command line makes the \TeX{} engine believe
it is compiling the file \textit{target}
whose content is specified as the latter parameter.
The provided code then forwards the processing to
\textit{main} or \textit{dest} as described in \secref{sec:forward}.

%%%%%%%%%%%%%%%%%%%%%%%%%%%%%%%%%%%%%%%%%%%%%%%%%%%%%%%%%%%%%%%%%%%%%%%%%%%%%%%%
\subsection{Include by Input}
\label{sec:input}

Including child documents by |\include| has some restrictions by design.
Most notably, the content of a child document always occupies
its own set of pages; pages cannot be shared between child documents.
Usually, this behaviour makes perfect sense
because each child document contain an essential part of the document.
However, in some situations it may be desirable to compose
a document from a collection of parts
without having mandatory page breaks between then.
For this case, the package
provides a mechanism to include parts
by |\input| which can also be processed individually.
However, by construction this mechanism
requires manual handling of the content to be output.

%%%%%%%%%%%%%%%%%%%%%%%%%%%%%%%%%%%%%%%%
\DescribeMacro{\ifchilddocmanual}
The main file should be prepared as usual, see \secref{sec:include}.
However, the document body must make a distinction
between processing of an individual part and of the main document, e.g.:
%
\begin{center}
\begin{tabular}{l}
|\ifchilddocmanual|\\
|\input{\childdocname}|\\
|\||else|\\
\textit{document body with }|\input{|\textit{part}|}|\\
|\||fi|
\end{tabular}
\end{center}
%
The conditional |\ifchilddocmanual| is true whenever
a part to be included by |\input| is being compiled,
and the name of the part is stored in |\childdocname|.

%%%%%%%%%%%%%%%%%%%%%%%%%%%%%%%%%%%%%%%%
\DescribeMacro{\childdocby}
Each part to be included by |\input| should start with:
%
\begin{center}
\begin{tabular}{l}
|\input{childdoc.def}|\\
|\childdocby{|\textit{main}|}|\\
\end{tabular}
\end{center}
%
The directive |\childdocby| is similar to |\childdocof|
described in \secref{sec:include},
but the subsequent selection of content must be done manually.
To that end, both |\ifchilddoc| and |\ifchilddocmanual|
will be true upon processing of a part,
and the name of the part is stored in |\childdocname|.
Note that |\jobname| will be set to the filename of the current part
so that each part receives an individual |.aux| file
that does not interfere with the |.aux| file(s) of the main document.
This behaviour can be altered by the alternative form
|\childdocby[*]{|\textit{main}|}| (with a non-empty optional argument)
which uses the |.aux| file of the main document
by setting |\jobname| to \textit{main}.

%%%%%%%%%%%%%%%%%%%%%%%%%%%%%%%%%%%%%%%%%%%%%%%%%%%%%%%%%%%%%%%%%%%%%%%%%%%%%%%%
\subsection{Driver Development}
\label{sec:driver}

The \textsf{childdoc} mechanism can also be use for the development
of definition files such as \LaTeX{} styles or classes.
This case differs from the above setup with multiple parts
included by |\include| in that no |\includeonly| should be invoked.
This can be achieved by starting the include file
(before |\ProvidesPackage|) with:
%
\begin{center}
\begin{tabular}{l}
|\input{childdoc.def}|\\
|\childdocforward{|\textit{main}|}|\\
\end{tabular}
\end{center}
%
or alternatively with:
%
\begin{center}
\begin{tabular}{l}
|\input{childdoc.def}|\\
|\childdocby{|\textit{main}|}|\\
\end{tabular}
\end{center}
%
Both forms have slightly different effects as described above.
The main file is prepared as usual, see \secref{sec:include}.

%%%%%%%%%%%%%%%%%%%%%%%%%%%%%%%%%%%%%%%%%%%%%%%%%%%%%%%%%%%%%%%%%%%%%%%%%%%%%%%%
\subsection{Legacy Detection}
\label{sec:detection}

The directive |\childdocmain| in the main file can detect
whether the complete document or merely a child is to be compiled
even without using the directive |\childdocof|.
This method is deprecated because it is less robust
and there is no compelling reason to use it;
it is merely provided for backward compatibility
and it may be removed in future versions.

If the detection mechanism is to be used,
it is mandatory to correctly specify
the filename of the main file as the argument of |\childdocmain|:
%
\begin{center}
\begin{tabular}{l}
|\input{childdoc.def}|\\
|\childdocmain{|\textit{main}|}|\\
\end{tabular}
\end{center}
%
If |\jobname| does not match the argument \textit{main} of |\childdocmain|,
it is assumed that |\jobname| points to the child file to be compiled.
When using |\childdocmain| with the main file specified as argument,
it suffices to start a child file
with just |\input{|\textit{main}|}|
without loading of the package and using |\childdocof|.
If instead all processing is done
with the appropriate \textsf{childdoc} directives,
the argument of \textit{main} of |\childdocmain| can be empty.

An alternative version of the command line processing described
in \secref{sec:commandline} using the detection mechanism reads:
%
\begin{center}
|... -jobname "|\textit{target}|" "|[\textit{flags}]%
[|\def\jobname{|\textit{dest}|}|]|\input{|\textit{main}|}"|
\end{center}

%%%%%%%%%%%%%%%%%%%%%%%%%%%%%%%%%%%%%%%%%%%%%%%%%%%%%%%%%%%%%%%%%%%%%%%%%%%%%%%%
\subsection{Manual Code}
\label{sec:manual}

In case one cannot be certain whether the definitions file |childdoc.def|
is installed on the target \TeX{} distribution
and one prefers not to ship it,
it is conceivable to paste a few relevant commands into the sources.

To that end, drop all statements |\input{childdoc.def}|
and perform the replacements as outlined below.
Instead of |\childdocmain{|\textit{main}|}| add the following code
to the top of the main file:
%
\begin{center}
\begin{tabular}{l}
|\||ifdefined\childdocname\endinput\||fi\newif\ifchilddoc|\\
|\edef\childdocname{\scantokens\expandafter{\jobname\noexpand}}|\\
|\def\childdocmain{|\textit{main}|}\||ifx\childdocmain\childdocname\||else|\\
|\childdoctrue\includeonly{\childdocname}\let\jobname\childdocmain\||fi|\\
\end{tabular}
\end{center}
%
Instead of |\childdocof{|\textit{main}|}| just include the main file
at the top of each child file:
%
\begin{center}
|\input{|\textit{main}|}|
\end{center}
%
A simple redirection |\childdocforward{|\textit{dest}|}| is achieved by:
%
\begin{center}
|\def\jobname{|\textit{dest}|}\input{\jobname}|
\end{center}
%
The redirection with prefix
|\childdocforwardprefix[|\textit{prefix}|]{|\textit{dest}|}|
is accomplished by:
%
\begin{center}
\begin{tabular}{l}
|{\edef\jobname{\scantokens\expandafter{\jobname\noexpand}}|\\
|\def\redirectjob |\textit{prefix}|#1~~~{\gdef\jobname{|\textit{dest}|#1}}|\\
|\expandafter\redirectjob\jobname~~~}\input{\jobname}|
\end{tabular}
\end{center}

In an alternative approach,
child documents can be compiled by a specific command line
without additional code or specific definitions:
%
\begin{center}
|... -jobname "|\textit{target}|" "|[\textit{flags}]%
|\includeonly{|\textit{dest}|}\input{|\textit{main}|}"|
\end{center}
%

%%%%%%%%%%%%%%%%%%%%%%%%%%%%%%%%%%%%%%%%%%%%%%%%%%%%%%%%%%%%%%%%%%%%%%%%%%%%%%%%
%%%%%%%%%%%%%%%%%%%%%%%%%%%%%%%%%%%%%%%%%%%%%%%%%%%%%%%%%%%%%%%%%%%%%%%%%%%%%%%%
\section{Information}

%%%%%%%%%%%%%%%%%%%%%%%%%%%%%%%%%%%%%%%%%%%%%%%%%%%%%%%%%%%%%%%%%%%%%%%%%%%%%%%%
\subsection{Copyright}

Copyright \copyright{} 2017--2018 Niklas Beisert

This work may be distributed and/or modified under the
conditions of the \LaTeX{} Project Public License, either version 1.3
of this license or (at your option) any later version.
The latest version of this license is in
  \url{http://www.latex-project.org/lppl.txt}
and version 1.3 or later is part of all distributions of \LaTeX{}
version 2005/12/01 or later.

This work has the LPPL maintenance status `maintained'.

The Current Maintainer of this work is Niklas Beisert.

This work consists of the files |README.txt|, |childdoc.ins| and |childdoc.dtx|
as well as the derived files |childdoc.def|, |cdocsamp.tex|
with |cdocsch1.tex|, |cdocsch2.tex|, |cdocspt3.tex|, |cdocspt4.tex|,
|cdocsdrf.tex|, |cdocsfn1.tex|, |cdocsfn2.tex|
as well as |childdoc.pdf|.

%%%%%%%%%%%%%%%%%%%%%%%%%%%%%%%%%%%%%%%%%%%%%%%%%%%%%%%%%%%%%%%%%%%%%%%%%%%%%%%%
\subsection{Files and Installation}

The package consists of the files:
%
\begin{center}
\begin{tabular}{ll}
    |README.txt|   & readme file \\
    |childdoc.ins| & installation file \\
    |childdoc.dtx| & source file \\
    |childdoc.def| & definition file \\
    |cdocsamp.tex| & sample main file \\
    |cdocsch1.tex| & sample include file \\
    |cdocsch2.tex| & sample include file \\
    |cdocspt3.tex| & sample part file \\
    |cdocspt4.tex| & sample part file \\
    |cdocsdrf.tex| & sample redirection file \\
    |cdocsfn1.tex| & sample redirection file \\
    |cdocsfn2.tex| & sample redirection file \\
    |childdoc.pdf| & manual
\end{tabular}
\end{center}
%
The distribution consists of the files
|README.txt|, |childdoc.ins| and |childdoc.dtx|.
%
\begin{itemize}
\item
Run (pdf)\LaTeX{} on |childdoc.dtx|
to compile the manual |childdoc.pdf| (this file).
\item
Run \LaTeX{} on |childdoc.ins| to create the definitions file |childdoc.def|
and the sample |cdocsamp.tex| with include files
|cdocsch1.tex|, |cdocsch2.tex|, |cdocspt3.tex|, |cdocspt4.tex|,
|cdocsdrf.tex|, |cdocsfn1.tex|, |cdocsfn2.tex|.
Then copy the file |childdoc.def| to an appropriate directory of your \LaTeX{}
distribution, e.g.\ \textit{texmf-root}|/tex/latex/childdoc|.
\end{itemize}

%%%%%%%%%%%%%%%%%%%%%%%%%%%%%%%%%%%%%%%%%%%%%%%%%%%%%%%%%%%%%%%%%%%%%%%%%%%%%%%%
\subsection{Related CTAN Packages}

There are several other packages which offer a similar functionality:
%
\begin{itemize}
\item
The packages
\href{http://ctan.org/pkg/docmute}{\textsf{docmute}},
\href{http://ctan.org/pkg/includex}{\textsf{includex}} and
\href{http://ctan.org/pkg/standalone}{\textsf{standalone}}
provide commands to include only the document body of
a child file thus allowing both files to be compiled individually.
\item
The packages \href{http://ctan.org/pkg/subdocs}{\textsf{subdocs}}
and \href{http://ctan.org/pkg/subfiles}{\textsf{subfiles}}
provide structures in which the main and child documents can be
encapsulated and allowing them to be compiled individually.
The inclusion mechanism is different from the conventional |\include|.
\item
The package \href{http://ctan.org/pkg/combine}{\textsf{combine}}
is an elaborate solution to combine several documents into one.
\end{itemize}
%
See also the CTAN topic \href{http://ctan.org/topic/subdocs}{\textsf{subdocs}}
for further related packages.
The present package differs from the above solutions in that
a document structure constructed with the conventional |\include| mechanism
just needs two extra commands at the top of every file
such that all constituent files can be compiled individually.

%%%%%%%%%%%%%%%%%%%%%%%%%%%%%%%%%%%%%%%%%%%%%%%%%%%%%%%%%%%%%%%%%%%%%%%%%%%%%%%%
%\subsection{Feature Suggestions}
%
%The following is a list of features which may be useful for future
%versions of this package:
%%
%\begin{itemize}
%\item
%\ldots
%\end{itemize}

%%%%%%%%%%%%%%%%%%%%%%%%%%%%%%%%%%%%%%%%%%%%%%%%%%%%%%%%%%%%%%%%%%%%%%%%%%%%%%%%
\subsection{Revision History}

%%%%%%%%%%%%%%%%%%%%%%%%%%%%%%%%%%%%%%%%
\paragraph{v2.0:} 2018/12/30

\begin{itemize}
\item
immediate forward processing
\item
added |\childdocby| mechanism
\item
manual restructured
\end{itemize}

%%%%%%%%%%%%%%%%%%%%%%%%%%%%%%%%%%%%%%%%
\paragraph{v1.6:} 2018/01/17

\begin{itemize}
\item
application for development of include files
\item
corrections to manual
\end{itemize}

%%%%%%%%%%%%%%%%%%%%%%%%%%%%%%%%%%%%%%%%
\paragraph{v1.5:} 2017/05/21

\begin{itemize}
\item
more complete structuring introduced
\item
|\childdocof| introduced
\item
|\childdoc| renamed to |\childdocmain|
\item
|\childredirect| renamed to |\childdocforward| and |\childdocforwardprefix|
and functionality expanded
\end{itemize}

%%%%%%%%%%%%%%%%%%%%%%%%%%%%%%%%%%%%%%%%
\paragraph{v1.0:} 2017/04/27

\begin{itemize}
\item
manual and install package
\item
first version published on CTAN
\end{itemize}

%%%%%%%%%%%%%%%%%%%%%%%%%%%%%%%%%%%%%%%%
\paragraph{v0.6:} 2017/04/26

\begin{itemize}
\item
redirection mechanism added
\end{itemize}

%%%%%%%%%%%%%%%%%%%%%%%%%%%%%%%%%%%%%%%%
\paragraph{v0.5:} 2017/04/26

\begin{itemize}
\item
functionality in definition file
\end{itemize}


%%%%%%%%%%%%%%%%%%%%%%%%%%%%%%%%%%%%%%%%%%%%%%%%%%%%%%%%%%%%%%%%%%%%%%%%%%%%%%%%
%%%%%%%%%%%%%%%%%%%%%%%%%%%%%%%%%%%%%%%%%%%%%%%%%%%%%%%%%%%%%%%%%%%%%%%%%%%%%%%%
%%%%%%%%%%%%%%%%%%%%%%%%%%%%%%%%%%%%%%%%%%%%%%%%%%%%%%%%%%%%%%%%%%%%%%%%%%%%%%%%
\appendix

\settowidth\MacroIndent{\rmfamily\scriptsize 000\ }

 \DocInput{childdoc.dtx}

\end{document}
%</driver>
% \fi
%
% %%%%%%%%%%%%%%%%%%%%%%%%%%%%%%%%%%%%%%%%%%%%%%%%%%%%%%%%%%%%%%%%%%%%%%%%%%%%%%
% %%%%%%%%%%%%%%%%%%%%%%%%%%%%%%%%%%%%%%%%%%%%%%%%%%%%%%%%%%%%%%%%%%%%%%%%%%%%%%
% \section{Sample}
%\iffalse
%<*samplemain>
%\fi
%
% The following presents a sample document
% with two chapters, two parts, a title page,
% a compile flag as well as three forwarding files to set the flag.
% It consists of eight |.tex| files:
% \begin{center}
% \begin{tabular}{ll}
% |cdocsamp.tex|&main file\\
% |cdocsch1.tex|&include file for chapter 1\\
% |cdocsch2.tex|&include file for chapter 2\\
% |cdocspt3.tex|&include file for part 3\\
% |cdocspt4.tex|&include file for part 4\\
% |cdocsdrf.tex|&forwarding file for main file in draft mode\\
% |cdocsfi1.tex|&forwarding file for final version of chapter 1\\
% |cdocsfi2.tex|&forwarding file for final version of chapter 2\\
% \end{tabular}
% \end{center}
% Each of the eight files can be compiled directly by the \LaTeX{} compiler.
%
% %%%%%%%%%%%%%%%%%%%%%%%%%%%%%%%%%%%%%%
% \paragraph{Main File.}
%
% The main file is called |cdocsamp.tex|.
%
% Load the \textsf{childdoc} definitions and
% declare the filename for the main document:
%    \begin{macrocode}
\input{childdoc.def}
\childdocmain{}
%    \end{macrocode}

% Optional override for |\version| flag:
%    \begin{macrocode}
%%\ifchilddoc\else\providecommand{\version}{draft}\fi
%    \end{macrocode}

% Define the default values for the |\version| flag
% (|final| for the main file and |draft| for childs):
%    \begin{macrocode}
\ifchilddoc
\providecommand{\version}{draft}
\else
\providecommand{\version}{final}
\fi
%    \end{macrocode}

% Load the standard document class:
%    \begin{macrocode}
\documentclass[12pt]{article}
%    \end{macrocode}

% Start the document body:
%    \begin{macrocode}
\begin{document}
%    \end{macrocode}

% Declare a title page.
% Print title, part of document being processed and version flag:
%    \begin{macrocode}
\addtocounter{page}{-1}
\begin{center}
{\LARGE\bfseries{}childdoc example\par}
\vspace{1cm}
\ifchilddoc
\ifchilddocmanual part\else chapter\fi:
`\childdocname' of `\childdocjob'\par
\else
main document: `\childdocjob'\par
\fi
version: \version\par
\end{center}
\newpage
%    \end{macrocode}

% Manually include selected file,
% otherwise process as usual:
%    \begin{macrocode}
\ifchilddocmanual
\section*{part `\childdocname'}
\input{\childdocname}
\else
%    \end{macrocode}

% Include the two chapters:
%    \begin{macrocode}
\include{cdocsch1}
\include{cdocsch2}
%    \end{macrocode}

% Include the two parts unless only chapters should be displayed:
%    \begin{macrocode}
\ifchilddoc\else
\section{part three}
\input{cdocspt3}
\section{part four}
\input{cdocspt4}
\fi
%    \end{macrocode}

% Process as usual until here:
%    \begin{macrocode}
\fi
%    \end{macrocode}

% End of document body:
%    \begin{macrocode}
\end{document}
%    \end{macrocode}
%\iffalse
%</samplemain>
%\fi
%
% %%%%%%%%%%%%%%%%%%%%%%%%%%%%%%%%%%%%%%
% \paragraph{Chapter Include Files.}
%
% The include files are called |cdocsch1.tex| and |cdocsch2.tex|.
%
%\iffalse
%<*samplechap1|samplechap2>
%\fi

% Optional override for |\version| flag:
%    \begin{macrocode}
%%\providecommand{\version}{final}
%    \end{macrocode}

% Include the main document:
%    \begin{macrocode}
\input{childdoc.def}
\childdocof{cdocsamp}
%    \end{macrocode}

%\iffalse
%</samplechap1|samplechap2>
%\fi
%
%\iffalse
%<*samplechap1>
%\fi
% Some text for chapter 1:
%    \begin{macrocode}
\section{one}
some text in chapter one
%    \end{macrocode}

%\iffalse
%</samplechap1>
%\fi
% Some text for chapter 2:
%\iffalse
%<*samplechap2>
%\fi
%    \begin{macrocode}
\section{two}
more text in chapter two
%    \end{macrocode}

%\iffalse
%</samplechap2>
%\fi
%
% %%%%%%%%%%%%%%%%%%%%%%%%%%%%%%%%%%%%%%
% \paragraph{Part Include Files.}
%
% The include files are called |cdocspt3.tex| and |cdocspt4.tex|.
%
%\iffalse
%<*samplepart3|samplepart4>
%\fi

% Optional override for |\version| flag:
%    \begin{macrocode}
%%\providecommand{\version}{final}
%    \end{macrocode}

% Include the main document:
%    \begin{macrocode}
\input{childdoc.def}
\childdocby{cdocsamp}
%    \end{macrocode}

%\iffalse
%</samplepart3|samplepart4>
%\fi
%
%\iffalse
%<*samplepart3>
%\fi
% Some text for part 3:
%    \begin{macrocode}
some text in part three
%    \end{macrocode}

%\iffalse
%</samplepart3>
%\fi
% Some text for part 4:
%\iffalse
%<*samplepart4>
%\fi
%    \begin{macrocode}
more text in part four
%    \end{macrocode}

%\iffalse
%</samplepart4>
%\fi
%
% %%%%%%%%%%%%%%%%%%%%%%%%%%%%%%%%%%%%%%
% \paragraph{Forwarding for a Complete Draft.}
%
% The following forwarding file |cdocsdrf.tex|
% compiles the main document in draft mode:
%\iffalse
%<*sampledraft>
%\fi
%    \begin{macrocode}
\def\version{draft}
\input{childdoc.def}
\childdocforward{cdocsamp}
%    \end{macrocode}

%\iffalse
%</sampledraft>
%\fi
%
% %%%%%%%%%%%%%%%%%%%%%%%%%%%%%%%%%%%%%%
% \paragraph{Forwarding for Final Version of the Chapters.}
%
% The following forwarding files |cdocsfn1.tex| and |cdocsfn2.tex|
% (with identical content)
% compile the final versions of the child documents
% |cdocsch1.tex| and |cdocsch2.tex|, respectively:
%\iffalse
%<*samplefinal>
%\fi
%    \begin{macrocode}
\def\version{final}
\input{childdoc.def}
\childdocforwardprefix[cdocsamp]{cdocsfn}{cdocsch}
%    \end{macrocode}

%\iffalse
%</samplefinal>
%\fi
%
% %%%%%%%%%%%%%%%%%%%%%%%%%%%%%%%%%%%%%%
% \paragraph{Command Line Processing.}
%
% The following three command lines generate the output files
% |cdocscld|, |cdocscl1| and |cdocscl2|
% which should be identical to
% |cdocsdrf|, |cdocsch1| and |cdocsfn2|, respectively:
% \begin{center}
% \begin{tabular}{l}
% |latex -jobname cdocscld \|\\
% |  "\def\version{draft}\input{childdoc.def}\childdocforward{cdocsamp}"|\\
% |latex -jobname cdocscl1 \|\\
% |  "\input{childdoc.def}\childdocforward[cdocsamp]{cdocsch1}"|\\
% |latex -jobname cdocscl2 \|\\
% |  "\def\version{final}\input{childdoc.def}\childdocforward{cdocsch2}"|
% \end{tabular}
% \end{center}
% Note that the trailing backslash on each first line
% merely continues the input to the second line
% (for convenient cut ant paste).
% Furthermore, the command |latex| can be replaced by any
% of its alternative versions such as |pdflatex|.
%
% %%%%%%%%%%%%%%%%%%%%%%%%%%%%%%%%%%%%%%%%%%%%%%%%%%%%%%%%%%%%%%%%%%%%%%%%%%%%%%
% %%%%%%%%%%%%%%%%%%%%%%%%%%%%%%%%%%%%%%%%%%%%%%%%%%%%%%%%%%%%%%%%%%%%%%%%%%%%%%
% \section{Implementation}
%\iffalse
%<*package>
%\fi
%
% This section describes the definitions file |childdoc.def|.

% The definitions cannot be loaded using |\usepackage| or |\RequirePackage|
% which has a mechanism to prevent loading a style file more than once.
% When loading the definitions by means of |\input|
% multiple instances have to be prevented manually:
%\iffalse
%This code needs to be before the `\ProvidesFile' directive
%which is defined at the beginning of this file.
%Therefore it is also placed there and commented out here.
%</package>
%<*discard>
%\fi
%    \begin{macrocode}
\ifdefined\childdocmain\endinput\fi
%    \end{macrocode}
%\iffalse
%</discard>
%<*package>
%\fi
%
% \macro{\ifchilddoc}
% \macro{\ifchilddocmanual}
% The conditional |\ifchilddoc| tells whether a
% child (true) or main (false) document is being compiled.
% The conditional |\ifchilddocmanual| tells whether
% the |\includeonly| mechanism is used (false) or
% the selection of child files must be performed manually (true).
% The definitions initialise to false:
%    \begin{macrocode}
\newif\ifchilddoc
\newif\ifchilddocmanual
%    \end{macrocode}

% \macro{\childdocname}
% \macro{\childdocjob}
% The macro |\childdocname| stores the name of the main document
% to be compiled. The macro |\childdocjob| stores the name of
% the document on which the \LaTeX{} compiler was originally invoked.
% The content of |\jobname| cannot be compared
% to filenames specified in the source due to different catcodes.
% The following code rescans |\jobname|, stores the result
% in |\childdocname| and saves a copy in |\childdocjob|:
%    \begin{macrocode}
\edef\childdocname{\scantokens\expandafter{\jobname\noexpand}}
\let\childdocjob\childdocname
%    \end{macrocode}

% \macro{\childdocdisable}
% The macro |\childdocdisable| prevents the main file
% from being processed more than once.
% At this stage, the main document command |\childdocmain|
% is assumed to be called once again where it should do nothing.
% Any subsequent call to it should prevent
% a secondary processing of the main document
% It overwrites the forwarding commands
% |\childdocof| and |\childdocforward|
% with empty macros to prevent further inclusions of the main document:
%    \begin{macrocode}
\newcommand{\childdocdisable}
{
  \renewcommand{\childdocmain}[1]{\renewcommand{\childdocmain}[1]{\endinput}}
  \renewcommand{\childdocof}[1]{}
  \renewcommand{\childdocby}[2][]{}
  \renewcommand{\childdocforward}[2][]{}
  \renewcommand{\childdocdisable}{}
}
%    \end{macrocode}

% \macro{\childdocmain}
% The macro |\childdocmain| is to be called at the top of the main file
% with nothing or the main filename (without extension) as argument.
% First, it breaks loops.
% If the argument is not empty and does not match |\childdocname|
% (which is set by the first inclusion of |childdoc.def|),
% |\ifchilddoc| is set to true, |\includeonly| is applied to the child file
% and |\jobname| is set to the main file
% (for proper handling of |.aux| files):
%    \begin{macrocode}
\newcommand{\childdocmain}[1]
{
  \childdocdisable\childdocmain{}
  \if?#1?\else
    \begingroup
      \def\childdoctmp{#1}
      \ifx\childdoctmp\childdocname
        \def\childdoctmp{}
      \else
        \def\childdoctmp
        {
          \childdoctrue
          \includeonly{\childdocname}
          \def\childdocjob{#1}
          \def\jobname{#1}
        }
      \fi
      \expandafter
    \endgroup
    \childdoctmp
  \fi
}
%    \end{macrocode}

% \macro{\childdocof}
% The command |\childdocof| redirects
% compilation to the main file |#1|.
%    \begin{macrocode}
\newcommand{\childdocof}[1]
{
  \childdocdisable
  \childdoctrue
  \includeonly{\childdocname}
  \def\jobname{#1}
  \def\childdocjob{#1}
  \input{#1}
}
%    \end{macrocode}

% \macro{\childdocby}
% The command |\childdocby| ....
%    \begin{macrocode}
\newcommand{\childdocby}[2][]
{
  \childdocdisable
  \childdoctrue
  \childdocmanualtrue
  \if?#1?\else
    \def\jobname{#2}
  \fi
  \def\childdocjob{#2}
  \input{#2}
  \endinput
}
%    \end{macrocode}

% \macro{\childdocforward}
% The command |\childdocforward| redirects
% compilation to the main file or
% (if the optional argument is given) a child file.
% Parameters are set as if the main file
% or a child file starting with |\childdocof| was compiled.
% Then compilation is handed over to the main file:
%    \begin{macrocode}
\newcommand{\childdocforward}[2][]
{
  \begingroup
    \if?#1?
      \def\childdoctmp
      {
        \def\childdocname{#2}
        \def\childdocjob{#2}
        \def\jobname{#2}
        \input{#2}
        \endinput
      }
    \else
      \def\childdoctmp
      {
        \childdocdisable
        \def\childdocname{#2}
        \childdoctrue
        \includeonly{#2}
        \def\childdocjob{#1}
        \def\jobname{#1}
        \input{#1}
        \endinput
      }
    \fi
    \expandafter
  \endgroup
  \childdoctmp
}
%    \end{macrocode}

% \macro{\childdocforwardprefix}
% The command |\childdocforwardprefix| redirects
% compilation to the main or a child file by means of a pattern.
% The prefix |#1| in the current filename is replaced by |#2|
% and the suffix of the current filename is kept
% (it is assumed that the filename does not contain the substring `|~~~|'
% which is used as a delimiter).
% Compilation is handed over to the new file by |\childdocforward|:
%    \begin{macrocode}
\newcommand{\childdocforwardprefix}[3][]
{
  \begingroup
    \def\childdocextract #2##1~~~{\def\childdoctmp{\childdocforward[#1]{#3##1}}}
    \expandafter\childdocextract\childdocname~~~
    \expandafter
  \endgroup
  \childdoctmp
}
%    \end{macrocode}

% \macro{\childdoc}
% The deprecated macro |\childdoc| is a legacy version of |\childdocmain|:
%    \begin{macrocode}
\newcommand{\childdoc}{\childdocmain}
%    \end{macrocode}

% \macro{\childdocredirect}
% The deprecated macro |\childdocredirect| is a legacy version
% of |\childdocforward| and |\childdocforwardprefix|:
%    \begin{macrocode}
\newcommand{\childdocredirect}[2][]
{
  \begingroup
    \if?#1?
      \def\childdoctmp{\childdocforward{#2}}
    \else
      \def\childdoctmp{\childdocforwardprefix{#1}{#2}}
    \fi
    \expandafter
  \endgroup
  \childdoctmp
}
%    \end{macrocode}

%\iffalse
%</package>
%\fi
%
\endinput
|\\
|\childdocby{|\textit{main}|}|\\
\end{tabular}
\end{center}
%
The directive |\childdocby| is similar to |\childdocof|
described in \secref{sec:include},
but the subsequent selection of content must be done manually.
To that end, both |\ifchilddoc| and |\ifchilddocmanual|
will be true upon processing of a part,
and the name of the part is stored in |\childdocname|.
Note that |\jobname| will be set to the filename of the current part
so that each part receives an individual |.aux| file
that does not interfere with the |.aux| file(s) of the main document.
This behaviour can be altered by the alternative form
|\childdocby[*]{|\textit{main}|}| (with a non-empty optional argument)
which uses the |.aux| file of the main document
by setting |\jobname| to \textit{main}.

%%%%%%%%%%%%%%%%%%%%%%%%%%%%%%%%%%%%%%%%%%%%%%%%%%%%%%%%%%%%%%%%%%%%%%%%%%%%%%%%
\subsection{Driver Development}
\label{sec:driver}

The \textsf{childdoc} mechanism can also be use for the development
of definition files such as \LaTeX{} styles or classes.
This case differs from the above setup with multiple parts
included by |\include| in that no |\includeonly| should be invoked.
This can be achieved by starting the include file
(before |\ProvidesPackage|) with:
%
\begin{center}
\begin{tabular}{l}
|% \iffalse
%
% childdoc.dtx Copyright (C) 2017-2018 Niklas Beisert
%
% This work may be distributed and/or modified under the
% conditions of the LaTeX Project Public License, either version 1.3
% of this license or (at your option) any later version.
% The latest version of this license is in
%   http://www.latex-project.org/lppl.txt
% and version 1.3 or later is part of all distributions of LaTeX
% version 2005/12/01 or later.
%
% This work has the LPPL maintenance status `maintained'.
%
% The Current Maintainer of this work is Niklas Beisert.
%
% This work consists of the files childdoc.dtx and childdoc.ins
% and the derived files childdoc.def and cdocsamp.tex with
% cdocsch1.tex, cdocsch2.tex, cdocsdrf.tex, cdocsfn1.tex, cdocsfn2.tex.
%
%<package>\ifdefined\childdocmain\endinput\fi
%<package>\ProvidesFile{childdoc.def}[2018/12/30 v2.0 child document driver]
%<samplemain>\ProvidesFile{cdocsamp.tex}[2018/12/30 v2.0 sample for childdoc]
%<*driver>
%\ProvidesFile{childdoc.drv}[2018/12/30 v2.0 childdoc reference manual file]
\PassOptionsToClass{10pt,a4paper}{article}
\documentclass{ltxdoc}

\usepackage[margin=35mm]{geometry}
\usepackage{hyperref}
\usepackage{hyperxmp}
\usepackage[usenames]{color}

\hypersetup{colorlinks=true}
\hypersetup{pdfstartview=FitH}
\hypersetup{pdfpagemode=UseNone}
\hypersetup{pdfsource={}}
\hypersetup{pdflang={en-UK}}
\hypersetup{pdfcopyright={Copyright 2017-2018 Niklas Beisert.
  This work may be distributed and/or modified under the
  conditions of the LaTeX Project Public License, either version 1.3
  of this license or (at your option) any later version.}}
\hypersetup{pdflicenseurl={http://www.latex-project.org/lppl.txt}}
\hypersetup{pdfcontactaddress={ETH Zurich, ITP, HIT K,
  Wolfgang-Pauli-Strasse 27}}
\hypersetup{pdfcontactpostcode={8093}}
\hypersetup{pdfcontactcity={Zurich}}
\hypersetup{pdfcontactcountry={Switzerland}}
\hypersetup{pdfcontactemail={nbeisert@itp.phys.ethz.ch}}
\hypersetup{pdfcontacturl={http://people.phys.ethz.ch/\xmptilde nbeisert/}}

\newcommand{\secref}[1]{\hyperref[#1]{section \ref*{#1}}}

\parskip1ex
\parindent0pt
\let\olditemize\itemize
\def\itemize{\olditemize\parskip0pt}

\begin{document}

\title{The \textsf{childdoc} Package}
\hypersetup{pdftitle={The childdoc Package}}
\author{Niklas Beisert\\[2ex]
  Institut f\"ur Theoretische Physik\\
  Eidgen\"ossische Technische Hochschule Z\"urich\\
  Wolfgang-Pauli-Strasse 27, 8093 Z\"urich, Switzerland\\[1ex]
  \href{mailto:nbeisert@itp.phys.ethz.ch}
  {\texttt{nbeisert@itp.phys.ethz.ch}}}
\hypersetup{pdfauthor={Niklas Beisert}}
\hypersetup{pdfsubject={Manual for the LaTeX2e Package childdoc}}
\date{30 December 2018, \textsf{v2.0}}
\maketitle

\begin{abstract}\noindent
\textsf{childdoc} is a \LaTeXe{} package
that enables the direct compilation
of document sections included by |\include|
to individual files.
\end{abstract}

\begingroup
\parskip0ex
\tableofcontents
\endgroup

%%%%%%%%%%%%%%%%%%%%%%%%%%%%%%%%%%%%%%%%%%%%%%%%%%%%%%%%%%%%%%%%%%%%%%%%%%%%%%%%
%%%%%%%%%%%%%%%%%%%%%%%%%%%%%%%%%%%%%%%%%%%%%%%%%%%%%%%%%%%%%%%%%%%%%%%%%%%%%%%%
\section{Introduction}

\LaTeX{} provides a mechanism to structure a large document (such as a book)
into a main file and several child files (containing the chapters)
using the |\include| command.
This mechanism is beneficial for documents
which span hundreds of pages in order to
make the source file(s) more manageable.
Moreover, compilation can be restricted to
selected child files by means of the |\includeonly| command.
The latter feature can be used to reduce the compilation time while editing
(this was significantly more useful in the earlier days of \LaTeX{})
or to generate a smaller document which is easier to navigate.
Another application of |\includeonly| is to generate
documents consisting of selected parts of the complete document.

However, there are a few drawbacks of the plain |\include| mechanism:
\begin{itemize}
\item
The child files cannot be compiled on their own,
they can only be compiled via the main file.
A naive editing environment
(such as a text editor with an option
to have the current file processed by \LaTeX)
may require one to switch to the main file before compiling;
attempting to compile the child file produces errors.
\item
The main file must be modified (each time)
to adjust the |\includeonly| command
to the present needs. This easily leaves the main file in a messy state.
\item
The generated document will always carry the filename
of the main document. This is inconvenient if
several child files are to be compiled and
to be kept for distribution.
\end{itemize}

The present package provides a simple interface
to make child files individually compilable by \LaTeX{}.
Compiling a child file then has the same effect as compiling
the main file with an |\includeonly| command
to select the appropriate child.
Moreover the generated document will carry the name of the child
rather than the main file.
This resolves all three above issues.

This feature is meant to make the editing of books,
thesis documents and lecture notes somewhat more convenient.
However, the package can also be used efficiently for
composing a series of documents (such as exercise sheets)
which are typically distributed individually.
It then assists the author in generating the individual documents
(potentially in different versions)
as well as a document containing the collected series.
Another application is in developing style files
or other kinds of included material
where compilation of the style file could redirect
to a sample or test file.

%%%%%%%%%%%%%%%%%%%%%%%%%%%%%%%%%%%%%%%%%%%%%%%%%%%%%%%%%%%%%%%%%%%%%%%%%%%%%%%%
%%%%%%%%%%%%%%%%%%%%%%%%%%%%%%%%%%%%%%%%%%%%%%%%%%%%%%%%%%%%%%%%%%%%%%%%%%%%%%%%
\section{Usage}

First of all, the package \textsf{childdoc} is \emph{not} a standard
\LaTeXe{} |.sty| style file! Therefore it needs to be invoked in
a non-standard way.

%%%%%%%%%%%%%%%%%%%%%%%%%%%%%%%%%%%%%%%%%%%%%%%%%%%%%%%%%%%%%%%%%%%%%%%%%%%%%%%%
\subsection{Included Files}
\label{sec:include}

%%%%%%%%%%%%%%%%%%%%%%%%%%%%%%%%%%%%%%%%
\DescribeMacro{\childdocmain}
To use the package, add the commands
\begin{center}
\begin{tabular}{l}
|\input{childdoc.def}|\\
|\childdocmain{}|\\
\end{tabular}
\end{center}
at the very top of the main \LaTeX{} file,
in particular \emph{before} the |\documentclass| statement!
The argument of |\childdocmain| should be left empty
(but it must be present).

%%%%%%%%%%%%%%%%%%%%%%%%%%%%%%%%%%%%%%%%
\DescribeMacro{\childdocof}
Furthermore, add the commands
\begin{center}
\begin{tabular}{l}
|\input{childdoc.def}|\\
|\childdocof{|\textit{main}|}|\\
\end{tabular}
\end{center}
at the top of every child file \textit{child}
which is included by |\include{|\textit{child}|}|
from within the main file
(or at least for those files to be compiled individually).
The argument \textit{main} must be the filename of the main file.

There are a couple of
considerations in setting up the main and child documents:

%%%%%%%%%%%%%%%%%%%%%%%%%%%%%%%%%%%%%%%%
\paragraph{Restrictions.}

Please note the following restrictions:
\begin{itemize}
\item
|\childdocmain| must be called with one argument \textit{main}
to ensure compatibility with earlier version of the package.
It must either be empty (|\childdocmain{}|)
or precisely match the filename of the main file in which it is specified.
See \secref{sec:detection} for further information.
\item
The filename \textit{main} must be specified without the |.tex| extension.
\item
The filename \textit{main} is case sensitive
(even in case-insensitive file systems)
due to internal string comparison.
\item
The argument \textit{main} should be fully expanded, it cannot be a macro.
\item
Subdirectories and special characters should be avoided in filenames.
\item
The command |\childdocmain{|\textit{main}|}| must be followed by a whitespace.
It should not be followed immediately by another command
or by a comment mark `|%|'.
This is because the \TeX{} parser reads the token immediately following
the argument of |\childdocmain| and puts it
at the beginning of every child section;
however, a white\-space is ignored.
\end{itemize}

%%%%%%%%%%%%%%%%%%%%%%%%%%%%%%%%%%%%%%%%
\paragraph{Content of Main File.}

It is advisable to place all content in the child files included by |\include|.
Any output contained in the main file will appear in all child documents
unless suppressed manually;
it cannot be suppressed automatically by the |\includeonly| directive
and thus should normally be avoided.
A method to include some content in the main file
by means of conditional processing is described in \secref{sec:conditional}.

%%%%%%%%%%%%%%%%%%%%%%%%%%%%%%%%%%%%%%%%
\paragraph{Page Numbering.}

When only a part of the document is compiled,
the appropriate numbering of pages
(as well as other status parameters)
is determined from the |.aux| files.
The latter contain information from previous passes.
However this information needs to propagate through
all intermediate child documents.
Therefore the page numbering in child documents may well
be inconsistent until the complete document is compiled at least once.

A useful (if unconventional) way to always ensure a consistent
page numbering is to restart the numbering in each child document
and denote the pages by `\textit{child}|.|\textit{page}'
where \textit{child} represents the chapter/section number of the child file.
This can be achieved by the command
|\numberwithin{page}{|\textit{child}|}|
of the \textsf{amsmath} package
where \textit{child} can be |chapter| or |section|
depending on the chosen structuring.
Alternatively, one can modify the macro |\thepage| appropriately
and reset the counter |page| at the start of each child file.

%%%%%%%%%%%%%%%%%%%%%%%%%%%%%%%%%%%%%%%%%%%%%%%%%%%%%%%%%%%%%%%%%%%%%%%%%%%%%%%%
\subsection{Conditional Processing}
\label{sec:conditional}

The package provides a mechanism to compile different versions
of a document. To customise the versions further some conditional processing
can come in handy to distinguish which version is being compiled.
The package provides two macros to describe the compilation context:

%%%%%%%%%%%%%%%%%%%%%%%%%%%%%%%%%%%%%%%%
\DescribeMacro{\ifchilddoc}
The conditional |\ifchilddoc| distinguishes between the compilation of
child documents and the main document:
%
\begin{center}
|\ifchilddoc |\textit{child-code}| |[|\||else |\textit{main-code}]| \||fi|
\end{center}

%%%%%%%%%%%%%%%%%%%%%%%%%%%%%%%%%%%%%%%%
\DescribeMacro{\childdocname}
\DescribeMacro{\childdocjob}
The macro |\childdocname| contains the filename (without extension)
of the main or child file being processed.
Note that |\childdocjob| will always contain the name of the main file.

%%%%%%%%%%%%%%%%%%%%%%%%%%%%%%%%%%%%%%%%
\paragraph{Title Page.}

Conditional processing can be used to include a title or banner page
in the main document when proper precautions are taken.
Importantly, the code in the main file should ensure that the page counter
(as well as other status parameters which are stored in the |.aux| files)
takes the same value after the conditional processing.
Otherwise the page numbers may take divergent values
depending on which part is compiled.

For example, a title page could be declared by:
%
\begin{center}
\begin{tabular}{l}
|\ifchilddoc\||else|\\
|\addtocounter{page}{-1}|\\
\textit{code for title page}\\
|\newpage|\\
|\||fi|
\end{tabular}
\end{center}
%
A banner page for the child documents can be generated by:
%
\begin{center}
\begin{tabular}{l}
|\ifchilddoc|\\
|\addtocounter{page}{-1}|\\
\textit{code for banner page}\\
|\newpage|\\
|\||fi|
\end{tabular}
\end{center}
%
Here one could write a message such as:
\begin{center}
|This is the part \childdocname{} of \childdocjob{}.|
\end{center}

%%%%%%%%%%%%%%%%%%%%%%%%%%%%%%%%%%%%%%%%%%%%%%%%%%%%%%%%%%%%%%%%%%%%%%%%%%%%%%%%
\subsection{Flags}
\label{sec:flags}

The package makes it easy to generate different versions
of the main or child documents.
To this end compilation flags can be defined
and assigned different default values.
They will be particularly useful in conjunction
with the forwarding mechanism described in \secref{sec:forward}.

For example, it may be useful to have a flag |\version|
which can be set to |draft| or |final|.
The document source will contain some conditional code
depending on the value of |\version|.
Suppose further, the flag should default to |final| for the main file
and to |draft| for child files
which is a natural assignment for editing the document.
This is achieved by placing the following code
in the preamble of the main document
(below the |\childdocmain| directive):
%
\begin{center}
\begin{tabular}{l}
|\ifchilddoc|\\
|\providecommand{\version}{draft}|\\
|\||else|\\
|\providecommand{\version}{final}|\\
|\||fi|
\end{tabular}
\end{center}
%
The definition by |\providecommand| makes sure
that previous definitions are not overwritten.
Further statements |\providecommand{\version}{...}|
can thus be added before the above code to override it.

For the main file, one might add a line
(between |\childdocmain| and the above block)
%
\begin{center}
|%\ifchilddoc\||else\providecommand{\version}{draft}\||fi|
\end{center}
%
which can be uncommented to produce a draft version.
Likewise one can add a line to the very top of a child file
(above the |\childdocof{|\textit{main}|}| directive)
%
\begin{center}
|%\providecommand{\version}{final}|
\end{center}
%
which can be uncommented to produce the final version of this child document.

%%%%%%%%%%%%%%%%%%%%%%%%%%%%%%%%%%%%%%%%%%%%%%%%%%%%%%%%%%%%%%%%%%%%%%%%%%%%%%%%
\subsection{Forwarding}
\label{sec:forward}

Different versions of the main or child documents
using compilation flags as described in \secref{sec:flags}
can be (permanently) stored in different files
for convenient compilation, viewing and distribution.
To this end, the package defines a command
to pass on compilation to a different file:

%%%%%%%%%%%%%%%%%%%%%%%%%%%%%%%%%%%%%%%%
\DescribeMacro{\childdocforward}
The command |\childdocforward| redirects processing to
another source file:
%
\begin{center}
\begin{tabular}{l}
|\input{childdoc.def}|\\
|\childdocforward[|\textit{main}|]{|\textit{dest}|}|\\
\end{tabular}
\end{center}
%
The argument \textit{dest} is the destination file
(without extension).
It should be the main file or one of the child files.
Note that further \textsf{childdoc} directives
such as |\childdocof| and |\childdocforward|
in the indicated file will be processed in this form.
The optional argument \textit{main}
passes on directly to the main file \textit{main}
while pretending to compile the child \textit{dest}.
This form behaves as if \textit{dest}
issues |\childdocof{|\textit{main}|}| right away,
and no further \textsf{childdoc} directives will be processed.

%%%%%%%%%%%%%%%%%%%%%%%%%%%%%%%%%%%%%%%%
\DescribeMacro{\...prefix}
In the alternative form |\childdocforwardprefix|,
%
\begin{center}
\begin{tabular}{l}
|\input{childdoc.def}|\\
|\childdocforwardprefix[|\textit{main}|]{|\textit{prefix}|}{|\textit{dest}|}|
\end{tabular}
\end{center}
%
the destination file is determined by a pattern
depending on the current file:
To make this work, the current file must be called
`{\textit{prefix}\hspace{0.2em}\textit{suffix}}'
with \textit{prefix} matching precisely the argument.
Processing is then passed on to the file
`{\textit{dest}\hspace{0.2em}\textit{suffix}}'.
Surely, the same effect is achieved by
directly specifying the
argument `{\textit{dest}\hspace{0.2em}\textit{suffix}}'
in the first form.
However, that requires to set up a different file
for each child. With the alternative form of the command
all these files can have exactly the same content
which simplifies setting them up and maintaining them.

For example, the following file |draft.tex|
with a compilation flag |\version| as described in \secref{sec:flags}
compiles the main document as a draft:
%
\begin{center}
\begin{tabular}{l}
|\def\version{draft}|\\
|\input{childdoc.def}|\\
|\childdocforward{|\textit{main}|}|
\end{tabular}
\end{center}
%
Likewise, the following files |final|\textit{nn}|.tex|
compile the final version of the child document
|child|\textit{nn}|.tex|:
%
\begin{center}
\begin{tabular}{l}
|\def\version{final}|\\
|\input{childdoc.def}|\\
|\childdocforwardprefix{final}{child}|
\end{tabular}
\end{center}
%

Note that when several versions of a main file and/or of each child file
are to be generated, it may be convenient to set up a |Makefile| or
shell script to automatise the process.

%%%%%%%%%%%%%%%%%%%%%%%%%%%%%%%%%%%%%%%%%%%%%%%%%%%%%%%%%%%%%%%%%%%%%%%%%%%%%%%%
\subsection{Command Line Processing}
\label{sec:commandline}

The effect of redirection files can also be achieved by invoking
the \LaTeX{} compiler with a more elaborate command line.
Most conveniently this should be done as part
of a shell script or a |Makefile|.

When using \textsf{childdoc} in the main file, the following
command lines effectively perform a redirection
(note that depending on the shell being used,
backslashes may have to be doubled: `|\|' $\to$ `|\\|'):
%
\begin{center}
|... -jobname "|\textit{target}|" |\\|"|[\textit{flags}]%
|\input{childdoc.def}\childdocforward[|\textit{main}|]{|\textit{dest}|}"|
\end{center}
%
Here \textit{target} is the name of the output file,
\textit{main} is the name of the main file
and \textit{dest} is the name of the main or child file to be processed
(all filenames without extensions).
The optional argument \textit{main} can be omitted
if \textit{main} matches \textit{dest}.
Optionally, compilation \textit{flags} can be defined via |\def| commands.
This command line makes the \TeX{} engine believe
it is compiling the file \textit{target}
whose content is specified as the latter parameter.
The provided code then forwards the processing to
\textit{main} or \textit{dest} as described in \secref{sec:forward}.

%%%%%%%%%%%%%%%%%%%%%%%%%%%%%%%%%%%%%%%%%%%%%%%%%%%%%%%%%%%%%%%%%%%%%%%%%%%%%%%%
\subsection{Include by Input}
\label{sec:input}

Including child documents by |\include| has some restrictions by design.
Most notably, the content of a child document always occupies
its own set of pages; pages cannot be shared between child documents.
Usually, this behaviour makes perfect sense
because each child document contain an essential part of the document.
However, in some situations it may be desirable to compose
a document from a collection of parts
without having mandatory page breaks between then.
For this case, the package
provides a mechanism to include parts
by |\input| which can also be processed individually.
However, by construction this mechanism
requires manual handling of the content to be output.

%%%%%%%%%%%%%%%%%%%%%%%%%%%%%%%%%%%%%%%%
\DescribeMacro{\ifchilddocmanual}
The main file should be prepared as usual, see \secref{sec:include}.
However, the document body must make a distinction
between processing of an individual part and of the main document, e.g.:
%
\begin{center}
\begin{tabular}{l}
|\ifchilddocmanual|\\
|\input{\childdocname}|\\
|\||else|\\
\textit{document body with }|\input{|\textit{part}|}|\\
|\||fi|
\end{tabular}
\end{center}
%
The conditional |\ifchilddocmanual| is true whenever
a part to be included by |\input| is being compiled,
and the name of the part is stored in |\childdocname|.

%%%%%%%%%%%%%%%%%%%%%%%%%%%%%%%%%%%%%%%%
\DescribeMacro{\childdocby}
Each part to be included by |\input| should start with:
%
\begin{center}
\begin{tabular}{l}
|\input{childdoc.def}|\\
|\childdocby{|\textit{main}|}|\\
\end{tabular}
\end{center}
%
The directive |\childdocby| is similar to |\childdocof|
described in \secref{sec:include},
but the subsequent selection of content must be done manually.
To that end, both |\ifchilddoc| and |\ifchilddocmanual|
will be true upon processing of a part,
and the name of the part is stored in |\childdocname|.
Note that |\jobname| will be set to the filename of the current part
so that each part receives an individual |.aux| file
that does not interfere with the |.aux| file(s) of the main document.
This behaviour can be altered by the alternative form
|\childdocby[*]{|\textit{main}|}| (with a non-empty optional argument)
which uses the |.aux| file of the main document
by setting |\jobname| to \textit{main}.

%%%%%%%%%%%%%%%%%%%%%%%%%%%%%%%%%%%%%%%%%%%%%%%%%%%%%%%%%%%%%%%%%%%%%%%%%%%%%%%%
\subsection{Driver Development}
\label{sec:driver}

The \textsf{childdoc} mechanism can also be use for the development
of definition files such as \LaTeX{} styles or classes.
This case differs from the above setup with multiple parts
included by |\include| in that no |\includeonly| should be invoked.
This can be achieved by starting the include file
(before |\ProvidesPackage|) with:
%
\begin{center}
\begin{tabular}{l}
|\input{childdoc.def}|\\
|\childdocforward{|\textit{main}|}|\\
\end{tabular}
\end{center}
%
or alternatively with:
%
\begin{center}
\begin{tabular}{l}
|\input{childdoc.def}|\\
|\childdocby{|\textit{main}|}|\\
\end{tabular}
\end{center}
%
Both forms have slightly different effects as described above.
The main file is prepared as usual, see \secref{sec:include}.

%%%%%%%%%%%%%%%%%%%%%%%%%%%%%%%%%%%%%%%%%%%%%%%%%%%%%%%%%%%%%%%%%%%%%%%%%%%%%%%%
\subsection{Legacy Detection}
\label{sec:detection}

The directive |\childdocmain| in the main file can detect
whether the complete document or merely a child is to be compiled
even without using the directive |\childdocof|.
This method is deprecated because it is less robust
and there is no compelling reason to use it;
it is merely provided for backward compatibility
and it may be removed in future versions.

If the detection mechanism is to be used,
it is mandatory to correctly specify
the filename of the main file as the argument of |\childdocmain|:
%
\begin{center}
\begin{tabular}{l}
|\input{childdoc.def}|\\
|\childdocmain{|\textit{main}|}|\\
\end{tabular}
\end{center}
%
If |\jobname| does not match the argument \textit{main} of |\childdocmain|,
it is assumed that |\jobname| points to the child file to be compiled.
When using |\childdocmain| with the main file specified as argument,
it suffices to start a child file
with just |\input{|\textit{main}|}|
without loading of the package and using |\childdocof|.
If instead all processing is done
with the appropriate \textsf{childdoc} directives,
the argument of \textit{main} of |\childdocmain| can be empty.

An alternative version of the command line processing described
in \secref{sec:commandline} using the detection mechanism reads:
%
\begin{center}
|... -jobname "|\textit{target}|" "|[\textit{flags}]%
[|\def\jobname{|\textit{dest}|}|]|\input{|\textit{main}|}"|
\end{center}

%%%%%%%%%%%%%%%%%%%%%%%%%%%%%%%%%%%%%%%%%%%%%%%%%%%%%%%%%%%%%%%%%%%%%%%%%%%%%%%%
\subsection{Manual Code}
\label{sec:manual}

In case one cannot be certain whether the definitions file |childdoc.def|
is installed on the target \TeX{} distribution
and one prefers not to ship it,
it is conceivable to paste a few relevant commands into the sources.

To that end, drop all statements |\input{childdoc.def}|
and perform the replacements as outlined below.
Instead of |\childdocmain{|\textit{main}|}| add the following code
to the top of the main file:
%
\begin{center}
\begin{tabular}{l}
|\||ifdefined\childdocname\endinput\||fi\newif\ifchilddoc|\\
|\edef\childdocname{\scantokens\expandafter{\jobname\noexpand}}|\\
|\def\childdocmain{|\textit{main}|}\||ifx\childdocmain\childdocname\||else|\\
|\childdoctrue\includeonly{\childdocname}\let\jobname\childdocmain\||fi|\\
\end{tabular}
\end{center}
%
Instead of |\childdocof{|\textit{main}|}| just include the main file
at the top of each child file:
%
\begin{center}
|\input{|\textit{main}|}|
\end{center}
%
A simple redirection |\childdocforward{|\textit{dest}|}| is achieved by:
%
\begin{center}
|\def\jobname{|\textit{dest}|}\input{\jobname}|
\end{center}
%
The redirection with prefix
|\childdocforwardprefix[|\textit{prefix}|]{|\textit{dest}|}|
is accomplished by:
%
\begin{center}
\begin{tabular}{l}
|{\edef\jobname{\scantokens\expandafter{\jobname\noexpand}}|\\
|\def\redirectjob |\textit{prefix}|#1~~~{\gdef\jobname{|\textit{dest}|#1}}|\\
|\expandafter\redirectjob\jobname~~~}\input{\jobname}|
\end{tabular}
\end{center}

In an alternative approach,
child documents can be compiled by a specific command line
without additional code or specific definitions:
%
\begin{center}
|... -jobname "|\textit{target}|" "|[\textit{flags}]%
|\includeonly{|\textit{dest}|}\input{|\textit{main}|}"|
\end{center}
%

%%%%%%%%%%%%%%%%%%%%%%%%%%%%%%%%%%%%%%%%%%%%%%%%%%%%%%%%%%%%%%%%%%%%%%%%%%%%%%%%
%%%%%%%%%%%%%%%%%%%%%%%%%%%%%%%%%%%%%%%%%%%%%%%%%%%%%%%%%%%%%%%%%%%%%%%%%%%%%%%%
\section{Information}

%%%%%%%%%%%%%%%%%%%%%%%%%%%%%%%%%%%%%%%%%%%%%%%%%%%%%%%%%%%%%%%%%%%%%%%%%%%%%%%%
\subsection{Copyright}

Copyright \copyright{} 2017--2018 Niklas Beisert

This work may be distributed and/or modified under the
conditions of the \LaTeX{} Project Public License, either version 1.3
of this license or (at your option) any later version.
The latest version of this license is in
  \url{http://www.latex-project.org/lppl.txt}
and version 1.3 or later is part of all distributions of \LaTeX{}
version 2005/12/01 or later.

This work has the LPPL maintenance status `maintained'.

The Current Maintainer of this work is Niklas Beisert.

This work consists of the files |README.txt|, |childdoc.ins| and |childdoc.dtx|
as well as the derived files |childdoc.def|, |cdocsamp.tex|
with |cdocsch1.tex|, |cdocsch2.tex|, |cdocspt3.tex|, |cdocspt4.tex|,
|cdocsdrf.tex|, |cdocsfn1.tex|, |cdocsfn2.tex|
as well as |childdoc.pdf|.

%%%%%%%%%%%%%%%%%%%%%%%%%%%%%%%%%%%%%%%%%%%%%%%%%%%%%%%%%%%%%%%%%%%%%%%%%%%%%%%%
\subsection{Files and Installation}

The package consists of the files:
%
\begin{center}
\begin{tabular}{ll}
    |README.txt|   & readme file \\
    |childdoc.ins| & installation file \\
    |childdoc.dtx| & source file \\
    |childdoc.def| & definition file \\
    |cdocsamp.tex| & sample main file \\
    |cdocsch1.tex| & sample include file \\
    |cdocsch2.tex| & sample include file \\
    |cdocspt3.tex| & sample part file \\
    |cdocspt4.tex| & sample part file \\
    |cdocsdrf.tex| & sample redirection file \\
    |cdocsfn1.tex| & sample redirection file \\
    |cdocsfn2.tex| & sample redirection file \\
    |childdoc.pdf| & manual
\end{tabular}
\end{center}
%
The distribution consists of the files
|README.txt|, |childdoc.ins| and |childdoc.dtx|.
%
\begin{itemize}
\item
Run (pdf)\LaTeX{} on |childdoc.dtx|
to compile the manual |childdoc.pdf| (this file).
\item
Run \LaTeX{} on |childdoc.ins| to create the definitions file |childdoc.def|
and the sample |cdocsamp.tex| with include files
|cdocsch1.tex|, |cdocsch2.tex|, |cdocspt3.tex|, |cdocspt4.tex|,
|cdocsdrf.tex|, |cdocsfn1.tex|, |cdocsfn2.tex|.
Then copy the file |childdoc.def| to an appropriate directory of your \LaTeX{}
distribution, e.g.\ \textit{texmf-root}|/tex/latex/childdoc|.
\end{itemize}

%%%%%%%%%%%%%%%%%%%%%%%%%%%%%%%%%%%%%%%%%%%%%%%%%%%%%%%%%%%%%%%%%%%%%%%%%%%%%%%%
\subsection{Related CTAN Packages}

There are several other packages which offer a similar functionality:
%
\begin{itemize}
\item
The packages
\href{http://ctan.org/pkg/docmute}{\textsf{docmute}},
\href{http://ctan.org/pkg/includex}{\textsf{includex}} and
\href{http://ctan.org/pkg/standalone}{\textsf{standalone}}
provide commands to include only the document body of
a child file thus allowing both files to be compiled individually.
\item
The packages \href{http://ctan.org/pkg/subdocs}{\textsf{subdocs}}
and \href{http://ctan.org/pkg/subfiles}{\textsf{subfiles}}
provide structures in which the main and child documents can be
encapsulated and allowing them to be compiled individually.
The inclusion mechanism is different from the conventional |\include|.
\item
The package \href{http://ctan.org/pkg/combine}{\textsf{combine}}
is an elaborate solution to combine several documents into one.
\end{itemize}
%
See also the CTAN topic \href{http://ctan.org/topic/subdocs}{\textsf{subdocs}}
for further related packages.
The present package differs from the above solutions in that
a document structure constructed with the conventional |\include| mechanism
just needs two extra commands at the top of every file
such that all constituent files can be compiled individually.

%%%%%%%%%%%%%%%%%%%%%%%%%%%%%%%%%%%%%%%%%%%%%%%%%%%%%%%%%%%%%%%%%%%%%%%%%%%%%%%%
%\subsection{Feature Suggestions}
%
%The following is a list of features which may be useful for future
%versions of this package:
%%
%\begin{itemize}
%\item
%\ldots
%\end{itemize}

%%%%%%%%%%%%%%%%%%%%%%%%%%%%%%%%%%%%%%%%%%%%%%%%%%%%%%%%%%%%%%%%%%%%%%%%%%%%%%%%
\subsection{Revision History}

%%%%%%%%%%%%%%%%%%%%%%%%%%%%%%%%%%%%%%%%
\paragraph{v2.0:} 2018/12/30

\begin{itemize}
\item
immediate forward processing
\item
added |\childdocby| mechanism
\item
manual restructured
\end{itemize}

%%%%%%%%%%%%%%%%%%%%%%%%%%%%%%%%%%%%%%%%
\paragraph{v1.6:} 2018/01/17

\begin{itemize}
\item
application for development of include files
\item
corrections to manual
\end{itemize}

%%%%%%%%%%%%%%%%%%%%%%%%%%%%%%%%%%%%%%%%
\paragraph{v1.5:} 2017/05/21

\begin{itemize}
\item
more complete structuring introduced
\item
|\childdocof| introduced
\item
|\childdoc| renamed to |\childdocmain|
\item
|\childredirect| renamed to |\childdocforward| and |\childdocforwardprefix|
and functionality expanded
\end{itemize}

%%%%%%%%%%%%%%%%%%%%%%%%%%%%%%%%%%%%%%%%
\paragraph{v1.0:} 2017/04/27

\begin{itemize}
\item
manual and install package
\item
first version published on CTAN
\end{itemize}

%%%%%%%%%%%%%%%%%%%%%%%%%%%%%%%%%%%%%%%%
\paragraph{v0.6:} 2017/04/26

\begin{itemize}
\item
redirection mechanism added
\end{itemize}

%%%%%%%%%%%%%%%%%%%%%%%%%%%%%%%%%%%%%%%%
\paragraph{v0.5:} 2017/04/26

\begin{itemize}
\item
functionality in definition file
\end{itemize}


%%%%%%%%%%%%%%%%%%%%%%%%%%%%%%%%%%%%%%%%%%%%%%%%%%%%%%%%%%%%%%%%%%%%%%%%%%%%%%%%
%%%%%%%%%%%%%%%%%%%%%%%%%%%%%%%%%%%%%%%%%%%%%%%%%%%%%%%%%%%%%%%%%%%%%%%%%%%%%%%%
%%%%%%%%%%%%%%%%%%%%%%%%%%%%%%%%%%%%%%%%%%%%%%%%%%%%%%%%%%%%%%%%%%%%%%%%%%%%%%%%
\appendix

\settowidth\MacroIndent{\rmfamily\scriptsize 000\ }

 \DocInput{childdoc.dtx}

\end{document}
%</driver>
% \fi
%
% %%%%%%%%%%%%%%%%%%%%%%%%%%%%%%%%%%%%%%%%%%%%%%%%%%%%%%%%%%%%%%%%%%%%%%%%%%%%%%
% %%%%%%%%%%%%%%%%%%%%%%%%%%%%%%%%%%%%%%%%%%%%%%%%%%%%%%%%%%%%%%%%%%%%%%%%%%%%%%
% \section{Sample}
%\iffalse
%<*samplemain>
%\fi
%
% The following presents a sample document
% with two chapters, two parts, a title page,
% a compile flag as well as three forwarding files to set the flag.
% It consists of eight |.tex| files:
% \begin{center}
% \begin{tabular}{ll}
% |cdocsamp.tex|&main file\\
% |cdocsch1.tex|&include file for chapter 1\\
% |cdocsch2.tex|&include file for chapter 2\\
% |cdocspt3.tex|&include file for part 3\\
% |cdocspt4.tex|&include file for part 4\\
% |cdocsdrf.tex|&forwarding file for main file in draft mode\\
% |cdocsfi1.tex|&forwarding file for final version of chapter 1\\
% |cdocsfi2.tex|&forwarding file for final version of chapter 2\\
% \end{tabular}
% \end{center}
% Each of the eight files can be compiled directly by the \LaTeX{} compiler.
%
% %%%%%%%%%%%%%%%%%%%%%%%%%%%%%%%%%%%%%%
% \paragraph{Main File.}
%
% The main file is called |cdocsamp.tex|.
%
% Load the \textsf{childdoc} definitions and
% declare the filename for the main document:
%    \begin{macrocode}
\input{childdoc.def}
\childdocmain{}
%    \end{macrocode}

% Optional override for |\version| flag:
%    \begin{macrocode}
%%\ifchilddoc\else\providecommand{\version}{draft}\fi
%    \end{macrocode}

% Define the default values for the |\version| flag
% (|final| for the main file and |draft| for childs):
%    \begin{macrocode}
\ifchilddoc
\providecommand{\version}{draft}
\else
\providecommand{\version}{final}
\fi
%    \end{macrocode}

% Load the standard document class:
%    \begin{macrocode}
\documentclass[12pt]{article}
%    \end{macrocode}

% Start the document body:
%    \begin{macrocode}
\begin{document}
%    \end{macrocode}

% Declare a title page.
% Print title, part of document being processed and version flag:
%    \begin{macrocode}
\addtocounter{page}{-1}
\begin{center}
{\LARGE\bfseries{}childdoc example\par}
\vspace{1cm}
\ifchilddoc
\ifchilddocmanual part\else chapter\fi:
`\childdocname' of `\childdocjob'\par
\else
main document: `\childdocjob'\par
\fi
version: \version\par
\end{center}
\newpage
%    \end{macrocode}

% Manually include selected file,
% otherwise process as usual:
%    \begin{macrocode}
\ifchilddocmanual
\section*{part `\childdocname'}
\input{\childdocname}
\else
%    \end{macrocode}

% Include the two chapters:
%    \begin{macrocode}
\include{cdocsch1}
\include{cdocsch2}
%    \end{macrocode}

% Include the two parts unless only chapters should be displayed:
%    \begin{macrocode}
\ifchilddoc\else
\section{part three}
\input{cdocspt3}
\section{part four}
\input{cdocspt4}
\fi
%    \end{macrocode}

% Process as usual until here:
%    \begin{macrocode}
\fi
%    \end{macrocode}

% End of document body:
%    \begin{macrocode}
\end{document}
%    \end{macrocode}
%\iffalse
%</samplemain>
%\fi
%
% %%%%%%%%%%%%%%%%%%%%%%%%%%%%%%%%%%%%%%
% \paragraph{Chapter Include Files.}
%
% The include files are called |cdocsch1.tex| and |cdocsch2.tex|.
%
%\iffalse
%<*samplechap1|samplechap2>
%\fi

% Optional override for |\version| flag:
%    \begin{macrocode}
%%\providecommand{\version}{final}
%    \end{macrocode}

% Include the main document:
%    \begin{macrocode}
\input{childdoc.def}
\childdocof{cdocsamp}
%    \end{macrocode}

%\iffalse
%</samplechap1|samplechap2>
%\fi
%
%\iffalse
%<*samplechap1>
%\fi
% Some text for chapter 1:
%    \begin{macrocode}
\section{one}
some text in chapter one
%    \end{macrocode}

%\iffalse
%</samplechap1>
%\fi
% Some text for chapter 2:
%\iffalse
%<*samplechap2>
%\fi
%    \begin{macrocode}
\section{two}
more text in chapter two
%    \end{macrocode}

%\iffalse
%</samplechap2>
%\fi
%
% %%%%%%%%%%%%%%%%%%%%%%%%%%%%%%%%%%%%%%
% \paragraph{Part Include Files.}
%
% The include files are called |cdocspt3.tex| and |cdocspt4.tex|.
%
%\iffalse
%<*samplepart3|samplepart4>
%\fi

% Optional override for |\version| flag:
%    \begin{macrocode}
%%\providecommand{\version}{final}
%    \end{macrocode}

% Include the main document:
%    \begin{macrocode}
\input{childdoc.def}
\childdocby{cdocsamp}
%    \end{macrocode}

%\iffalse
%</samplepart3|samplepart4>
%\fi
%
%\iffalse
%<*samplepart3>
%\fi
% Some text for part 3:
%    \begin{macrocode}
some text in part three
%    \end{macrocode}

%\iffalse
%</samplepart3>
%\fi
% Some text for part 4:
%\iffalse
%<*samplepart4>
%\fi
%    \begin{macrocode}
more text in part four
%    \end{macrocode}

%\iffalse
%</samplepart4>
%\fi
%
% %%%%%%%%%%%%%%%%%%%%%%%%%%%%%%%%%%%%%%
% \paragraph{Forwarding for a Complete Draft.}
%
% The following forwarding file |cdocsdrf.tex|
% compiles the main document in draft mode:
%\iffalse
%<*sampledraft>
%\fi
%    \begin{macrocode}
\def\version{draft}
\input{childdoc.def}
\childdocforward{cdocsamp}
%    \end{macrocode}

%\iffalse
%</sampledraft>
%\fi
%
% %%%%%%%%%%%%%%%%%%%%%%%%%%%%%%%%%%%%%%
% \paragraph{Forwarding for Final Version of the Chapters.}
%
% The following forwarding files |cdocsfn1.tex| and |cdocsfn2.tex|
% (with identical content)
% compile the final versions of the child documents
% |cdocsch1.tex| and |cdocsch2.tex|, respectively:
%\iffalse
%<*samplefinal>
%\fi
%    \begin{macrocode}
\def\version{final}
\input{childdoc.def}
\childdocforwardprefix[cdocsamp]{cdocsfn}{cdocsch}
%    \end{macrocode}

%\iffalse
%</samplefinal>
%\fi
%
% %%%%%%%%%%%%%%%%%%%%%%%%%%%%%%%%%%%%%%
% \paragraph{Command Line Processing.}
%
% The following three command lines generate the output files
% |cdocscld|, |cdocscl1| and |cdocscl2|
% which should be identical to
% |cdocsdrf|, |cdocsch1| and |cdocsfn2|, respectively:
% \begin{center}
% \begin{tabular}{l}
% |latex -jobname cdocscld \|\\
% |  "\def\version{draft}\input{childdoc.def}\childdocforward{cdocsamp}"|\\
% |latex -jobname cdocscl1 \|\\
% |  "\input{childdoc.def}\childdocforward[cdocsamp]{cdocsch1}"|\\
% |latex -jobname cdocscl2 \|\\
% |  "\def\version{final}\input{childdoc.def}\childdocforward{cdocsch2}"|
% \end{tabular}
% \end{center}
% Note that the trailing backslash on each first line
% merely continues the input to the second line
% (for convenient cut ant paste).
% Furthermore, the command |latex| can be replaced by any
% of its alternative versions such as |pdflatex|.
%
% %%%%%%%%%%%%%%%%%%%%%%%%%%%%%%%%%%%%%%%%%%%%%%%%%%%%%%%%%%%%%%%%%%%%%%%%%%%%%%
% %%%%%%%%%%%%%%%%%%%%%%%%%%%%%%%%%%%%%%%%%%%%%%%%%%%%%%%%%%%%%%%%%%%%%%%%%%%%%%
% \section{Implementation}
%\iffalse
%<*package>
%\fi
%
% This section describes the definitions file |childdoc.def|.

% The definitions cannot be loaded using |\usepackage| or |\RequirePackage|
% which has a mechanism to prevent loading a style file more than once.
% When loading the definitions by means of |\input|
% multiple instances have to be prevented manually:
%\iffalse
%This code needs to be before the `\ProvidesFile' directive
%which is defined at the beginning of this file.
%Therefore it is also placed there and commented out here.
%</package>
%<*discard>
%\fi
%    \begin{macrocode}
\ifdefined\childdocmain\endinput\fi
%    \end{macrocode}
%\iffalse
%</discard>
%<*package>
%\fi
%
% \macro{\ifchilddoc}
% \macro{\ifchilddocmanual}
% The conditional |\ifchilddoc| tells whether a
% child (true) or main (false) document is being compiled.
% The conditional |\ifchilddocmanual| tells whether
% the |\includeonly| mechanism is used (false) or
% the selection of child files must be performed manually (true).
% The definitions initialise to false:
%    \begin{macrocode}
\newif\ifchilddoc
\newif\ifchilddocmanual
%    \end{macrocode}

% \macro{\childdocname}
% \macro{\childdocjob}
% The macro |\childdocname| stores the name of the main document
% to be compiled. The macro |\childdocjob| stores the name of
% the document on which the \LaTeX{} compiler was originally invoked.
% The content of |\jobname| cannot be compared
% to filenames specified in the source due to different catcodes.
% The following code rescans |\jobname|, stores the result
% in |\childdocname| and saves a copy in |\childdocjob|:
%    \begin{macrocode}
\edef\childdocname{\scantokens\expandafter{\jobname\noexpand}}
\let\childdocjob\childdocname
%    \end{macrocode}

% \macro{\childdocdisable}
% The macro |\childdocdisable| prevents the main file
% from being processed more than once.
% At this stage, the main document command |\childdocmain|
% is assumed to be called once again where it should do nothing.
% Any subsequent call to it should prevent
% a secondary processing of the main document
% It overwrites the forwarding commands
% |\childdocof| and |\childdocforward|
% with empty macros to prevent further inclusions of the main document:
%    \begin{macrocode}
\newcommand{\childdocdisable}
{
  \renewcommand{\childdocmain}[1]{\renewcommand{\childdocmain}[1]{\endinput}}
  \renewcommand{\childdocof}[1]{}
  \renewcommand{\childdocby}[2][]{}
  \renewcommand{\childdocforward}[2][]{}
  \renewcommand{\childdocdisable}{}
}
%    \end{macrocode}

% \macro{\childdocmain}
% The macro |\childdocmain| is to be called at the top of the main file
% with nothing or the main filename (without extension) as argument.
% First, it breaks loops.
% If the argument is not empty and does not match |\childdocname|
% (which is set by the first inclusion of |childdoc.def|),
% |\ifchilddoc| is set to true, |\includeonly| is applied to the child file
% and |\jobname| is set to the main file
% (for proper handling of |.aux| files):
%    \begin{macrocode}
\newcommand{\childdocmain}[1]
{
  \childdocdisable\childdocmain{}
  \if?#1?\else
    \begingroup
      \def\childdoctmp{#1}
      \ifx\childdoctmp\childdocname
        \def\childdoctmp{}
      \else
        \def\childdoctmp
        {
          \childdoctrue
          \includeonly{\childdocname}
          \def\childdocjob{#1}
          \def\jobname{#1}
        }
      \fi
      \expandafter
    \endgroup
    \childdoctmp
  \fi
}
%    \end{macrocode}

% \macro{\childdocof}
% The command |\childdocof| redirects
% compilation to the main file |#1|.
%    \begin{macrocode}
\newcommand{\childdocof}[1]
{
  \childdocdisable
  \childdoctrue
  \includeonly{\childdocname}
  \def\jobname{#1}
  \def\childdocjob{#1}
  \input{#1}
}
%    \end{macrocode}

% \macro{\childdocby}
% The command |\childdocby| ....
%    \begin{macrocode}
\newcommand{\childdocby}[2][]
{
  \childdocdisable
  \childdoctrue
  \childdocmanualtrue
  \if?#1?\else
    \def\jobname{#2}
  \fi
  \def\childdocjob{#2}
  \input{#2}
  \endinput
}
%    \end{macrocode}

% \macro{\childdocforward}
% The command |\childdocforward| redirects
% compilation to the main file or
% (if the optional argument is given) a child file.
% Parameters are set as if the main file
% or a child file starting with |\childdocof| was compiled.
% Then compilation is handed over to the main file:
%    \begin{macrocode}
\newcommand{\childdocforward}[2][]
{
  \begingroup
    \if?#1?
      \def\childdoctmp
      {
        \def\childdocname{#2}
        \def\childdocjob{#2}
        \def\jobname{#2}
        \input{#2}
        \endinput
      }
    \else
      \def\childdoctmp
      {
        \childdocdisable
        \def\childdocname{#2}
        \childdoctrue
        \includeonly{#2}
        \def\childdocjob{#1}
        \def\jobname{#1}
        \input{#1}
        \endinput
      }
    \fi
    \expandafter
  \endgroup
  \childdoctmp
}
%    \end{macrocode}

% \macro{\childdocforwardprefix}
% The command |\childdocforwardprefix| redirects
% compilation to the main or a child file by means of a pattern.
% The prefix |#1| in the current filename is replaced by |#2|
% and the suffix of the current filename is kept
% (it is assumed that the filename does not contain the substring `|~~~|'
% which is used as a delimiter).
% Compilation is handed over to the new file by |\childdocforward|:
%    \begin{macrocode}
\newcommand{\childdocforwardprefix}[3][]
{
  \begingroup
    \def\childdocextract #2##1~~~{\def\childdoctmp{\childdocforward[#1]{#3##1}}}
    \expandafter\childdocextract\childdocname~~~
    \expandafter
  \endgroup
  \childdoctmp
}
%    \end{macrocode}

% \macro{\childdoc}
% The deprecated macro |\childdoc| is a legacy version of |\childdocmain|:
%    \begin{macrocode}
\newcommand{\childdoc}{\childdocmain}
%    \end{macrocode}

% \macro{\childdocredirect}
% The deprecated macro |\childdocredirect| is a legacy version
% of |\childdocforward| and |\childdocforwardprefix|:
%    \begin{macrocode}
\newcommand{\childdocredirect}[2][]
{
  \begingroup
    \if?#1?
      \def\childdoctmp{\childdocforward{#2}}
    \else
      \def\childdoctmp{\childdocforwardprefix{#1}{#2}}
    \fi
    \expandafter
  \endgroup
  \childdoctmp
}
%    \end{macrocode}

%\iffalse
%</package>
%\fi
%
\endinput
|\\
|\childdocforward{|\textit{main}|}|\\
\end{tabular}
\end{center}
%
or alternatively with:
%
\begin{center}
\begin{tabular}{l}
|% \iffalse
%
% childdoc.dtx Copyright (C) 2017-2018 Niklas Beisert
%
% This work may be distributed and/or modified under the
% conditions of the LaTeX Project Public License, either version 1.3
% of this license or (at your option) any later version.
% The latest version of this license is in
%   http://www.latex-project.org/lppl.txt
% and version 1.3 or later is part of all distributions of LaTeX
% version 2005/12/01 or later.
%
% This work has the LPPL maintenance status `maintained'.
%
% The Current Maintainer of this work is Niklas Beisert.
%
% This work consists of the files childdoc.dtx and childdoc.ins
% and the derived files childdoc.def and cdocsamp.tex with
% cdocsch1.tex, cdocsch2.tex, cdocsdrf.tex, cdocsfn1.tex, cdocsfn2.tex.
%
%<package>\ifdefined\childdocmain\endinput\fi
%<package>\ProvidesFile{childdoc.def}[2018/12/30 v2.0 child document driver]
%<samplemain>\ProvidesFile{cdocsamp.tex}[2018/12/30 v2.0 sample for childdoc]
%<*driver>
%\ProvidesFile{childdoc.drv}[2018/12/30 v2.0 childdoc reference manual file]
\PassOptionsToClass{10pt,a4paper}{article}
\documentclass{ltxdoc}

\usepackage[margin=35mm]{geometry}
\usepackage{hyperref}
\usepackage{hyperxmp}
\usepackage[usenames]{color}

\hypersetup{colorlinks=true}
\hypersetup{pdfstartview=FitH}
\hypersetup{pdfpagemode=UseNone}
\hypersetup{pdfsource={}}
\hypersetup{pdflang={en-UK}}
\hypersetup{pdfcopyright={Copyright 2017-2018 Niklas Beisert.
  This work may be distributed and/or modified under the
  conditions of the LaTeX Project Public License, either version 1.3
  of this license or (at your option) any later version.}}
\hypersetup{pdflicenseurl={http://www.latex-project.org/lppl.txt}}
\hypersetup{pdfcontactaddress={ETH Zurich, ITP, HIT K,
  Wolfgang-Pauli-Strasse 27}}
\hypersetup{pdfcontactpostcode={8093}}
\hypersetup{pdfcontactcity={Zurich}}
\hypersetup{pdfcontactcountry={Switzerland}}
\hypersetup{pdfcontactemail={nbeisert@itp.phys.ethz.ch}}
\hypersetup{pdfcontacturl={http://people.phys.ethz.ch/\xmptilde nbeisert/}}

\newcommand{\secref}[1]{\hyperref[#1]{section \ref*{#1}}}

\parskip1ex
\parindent0pt
\let\olditemize\itemize
\def\itemize{\olditemize\parskip0pt}

\begin{document}

\title{The \textsf{childdoc} Package}
\hypersetup{pdftitle={The childdoc Package}}
\author{Niklas Beisert\\[2ex]
  Institut f\"ur Theoretische Physik\\
  Eidgen\"ossische Technische Hochschule Z\"urich\\
  Wolfgang-Pauli-Strasse 27, 8093 Z\"urich, Switzerland\\[1ex]
  \href{mailto:nbeisert@itp.phys.ethz.ch}
  {\texttt{nbeisert@itp.phys.ethz.ch}}}
\hypersetup{pdfauthor={Niklas Beisert}}
\hypersetup{pdfsubject={Manual for the LaTeX2e Package childdoc}}
\date{30 December 2018, \textsf{v2.0}}
\maketitle

\begin{abstract}\noindent
\textsf{childdoc} is a \LaTeXe{} package
that enables the direct compilation
of document sections included by |\include|
to individual files.
\end{abstract}

\begingroup
\parskip0ex
\tableofcontents
\endgroup

%%%%%%%%%%%%%%%%%%%%%%%%%%%%%%%%%%%%%%%%%%%%%%%%%%%%%%%%%%%%%%%%%%%%%%%%%%%%%%%%
%%%%%%%%%%%%%%%%%%%%%%%%%%%%%%%%%%%%%%%%%%%%%%%%%%%%%%%%%%%%%%%%%%%%%%%%%%%%%%%%
\section{Introduction}

\LaTeX{} provides a mechanism to structure a large document (such as a book)
into a main file and several child files (containing the chapters)
using the |\include| command.
This mechanism is beneficial for documents
which span hundreds of pages in order to
make the source file(s) more manageable.
Moreover, compilation can be restricted to
selected child files by means of the |\includeonly| command.
The latter feature can be used to reduce the compilation time while editing
(this was significantly more useful in the earlier days of \LaTeX{})
or to generate a smaller document which is easier to navigate.
Another application of |\includeonly| is to generate
documents consisting of selected parts of the complete document.

However, there are a few drawbacks of the plain |\include| mechanism:
\begin{itemize}
\item
The child files cannot be compiled on their own,
they can only be compiled via the main file.
A naive editing environment
(such as a text editor with an option
to have the current file processed by \LaTeX)
may require one to switch to the main file before compiling;
attempting to compile the child file produces errors.
\item
The main file must be modified (each time)
to adjust the |\includeonly| command
to the present needs. This easily leaves the main file in a messy state.
\item
The generated document will always carry the filename
of the main document. This is inconvenient if
several child files are to be compiled and
to be kept for distribution.
\end{itemize}

The present package provides a simple interface
to make child files individually compilable by \LaTeX{}.
Compiling a child file then has the same effect as compiling
the main file with an |\includeonly| command
to select the appropriate child.
Moreover the generated document will carry the name of the child
rather than the main file.
This resolves all three above issues.

This feature is meant to make the editing of books,
thesis documents and lecture notes somewhat more convenient.
However, the package can also be used efficiently for
composing a series of documents (such as exercise sheets)
which are typically distributed individually.
It then assists the author in generating the individual documents
(potentially in different versions)
as well as a document containing the collected series.
Another application is in developing style files
or other kinds of included material
where compilation of the style file could redirect
to a sample or test file.

%%%%%%%%%%%%%%%%%%%%%%%%%%%%%%%%%%%%%%%%%%%%%%%%%%%%%%%%%%%%%%%%%%%%%%%%%%%%%%%%
%%%%%%%%%%%%%%%%%%%%%%%%%%%%%%%%%%%%%%%%%%%%%%%%%%%%%%%%%%%%%%%%%%%%%%%%%%%%%%%%
\section{Usage}

First of all, the package \textsf{childdoc} is \emph{not} a standard
\LaTeXe{} |.sty| style file! Therefore it needs to be invoked in
a non-standard way.

%%%%%%%%%%%%%%%%%%%%%%%%%%%%%%%%%%%%%%%%%%%%%%%%%%%%%%%%%%%%%%%%%%%%%%%%%%%%%%%%
\subsection{Included Files}
\label{sec:include}

%%%%%%%%%%%%%%%%%%%%%%%%%%%%%%%%%%%%%%%%
\DescribeMacro{\childdocmain}
To use the package, add the commands
\begin{center}
\begin{tabular}{l}
|\input{childdoc.def}|\\
|\childdocmain{}|\\
\end{tabular}
\end{center}
at the very top of the main \LaTeX{} file,
in particular \emph{before} the |\documentclass| statement!
The argument of |\childdocmain| should be left empty
(but it must be present).

%%%%%%%%%%%%%%%%%%%%%%%%%%%%%%%%%%%%%%%%
\DescribeMacro{\childdocof}
Furthermore, add the commands
\begin{center}
\begin{tabular}{l}
|\input{childdoc.def}|\\
|\childdocof{|\textit{main}|}|\\
\end{tabular}
\end{center}
at the top of every child file \textit{child}
which is included by |\include{|\textit{child}|}|
from within the main file
(or at least for those files to be compiled individually).
The argument \textit{main} must be the filename of the main file.

There are a couple of
considerations in setting up the main and child documents:

%%%%%%%%%%%%%%%%%%%%%%%%%%%%%%%%%%%%%%%%
\paragraph{Restrictions.}

Please note the following restrictions:
\begin{itemize}
\item
|\childdocmain| must be called with one argument \textit{main}
to ensure compatibility with earlier version of the package.
It must either be empty (|\childdocmain{}|)
or precisely match the filename of the main file in which it is specified.
See \secref{sec:detection} for further information.
\item
The filename \textit{main} must be specified without the |.tex| extension.
\item
The filename \textit{main} is case sensitive
(even in case-insensitive file systems)
due to internal string comparison.
\item
The argument \textit{main} should be fully expanded, it cannot be a macro.
\item
Subdirectories and special characters should be avoided in filenames.
\item
The command |\childdocmain{|\textit{main}|}| must be followed by a whitespace.
It should not be followed immediately by another command
or by a comment mark `|%|'.
This is because the \TeX{} parser reads the token immediately following
the argument of |\childdocmain| and puts it
at the beginning of every child section;
however, a white\-space is ignored.
\end{itemize}

%%%%%%%%%%%%%%%%%%%%%%%%%%%%%%%%%%%%%%%%
\paragraph{Content of Main File.}

It is advisable to place all content in the child files included by |\include|.
Any output contained in the main file will appear in all child documents
unless suppressed manually;
it cannot be suppressed automatically by the |\includeonly| directive
and thus should normally be avoided.
A method to include some content in the main file
by means of conditional processing is described in \secref{sec:conditional}.

%%%%%%%%%%%%%%%%%%%%%%%%%%%%%%%%%%%%%%%%
\paragraph{Page Numbering.}

When only a part of the document is compiled,
the appropriate numbering of pages
(as well as other status parameters)
is determined from the |.aux| files.
The latter contain information from previous passes.
However this information needs to propagate through
all intermediate child documents.
Therefore the page numbering in child documents may well
be inconsistent until the complete document is compiled at least once.

A useful (if unconventional) way to always ensure a consistent
page numbering is to restart the numbering in each child document
and denote the pages by `\textit{child}|.|\textit{page}'
where \textit{child} represents the chapter/section number of the child file.
This can be achieved by the command
|\numberwithin{page}{|\textit{child}|}|
of the \textsf{amsmath} package
where \textit{child} can be |chapter| or |section|
depending on the chosen structuring.
Alternatively, one can modify the macro |\thepage| appropriately
and reset the counter |page| at the start of each child file.

%%%%%%%%%%%%%%%%%%%%%%%%%%%%%%%%%%%%%%%%%%%%%%%%%%%%%%%%%%%%%%%%%%%%%%%%%%%%%%%%
\subsection{Conditional Processing}
\label{sec:conditional}

The package provides a mechanism to compile different versions
of a document. To customise the versions further some conditional processing
can come in handy to distinguish which version is being compiled.
The package provides two macros to describe the compilation context:

%%%%%%%%%%%%%%%%%%%%%%%%%%%%%%%%%%%%%%%%
\DescribeMacro{\ifchilddoc}
The conditional |\ifchilddoc| distinguishes between the compilation of
child documents and the main document:
%
\begin{center}
|\ifchilddoc |\textit{child-code}| |[|\||else |\textit{main-code}]| \||fi|
\end{center}

%%%%%%%%%%%%%%%%%%%%%%%%%%%%%%%%%%%%%%%%
\DescribeMacro{\childdocname}
\DescribeMacro{\childdocjob}
The macro |\childdocname| contains the filename (without extension)
of the main or child file being processed.
Note that |\childdocjob| will always contain the name of the main file.

%%%%%%%%%%%%%%%%%%%%%%%%%%%%%%%%%%%%%%%%
\paragraph{Title Page.}

Conditional processing can be used to include a title or banner page
in the main document when proper precautions are taken.
Importantly, the code in the main file should ensure that the page counter
(as well as other status parameters which are stored in the |.aux| files)
takes the same value after the conditional processing.
Otherwise the page numbers may take divergent values
depending on which part is compiled.

For example, a title page could be declared by:
%
\begin{center}
\begin{tabular}{l}
|\ifchilddoc\||else|\\
|\addtocounter{page}{-1}|\\
\textit{code for title page}\\
|\newpage|\\
|\||fi|
\end{tabular}
\end{center}
%
A banner page for the child documents can be generated by:
%
\begin{center}
\begin{tabular}{l}
|\ifchilddoc|\\
|\addtocounter{page}{-1}|\\
\textit{code for banner page}\\
|\newpage|\\
|\||fi|
\end{tabular}
\end{center}
%
Here one could write a message such as:
\begin{center}
|This is the part \childdocname{} of \childdocjob{}.|
\end{center}

%%%%%%%%%%%%%%%%%%%%%%%%%%%%%%%%%%%%%%%%%%%%%%%%%%%%%%%%%%%%%%%%%%%%%%%%%%%%%%%%
\subsection{Flags}
\label{sec:flags}

The package makes it easy to generate different versions
of the main or child documents.
To this end compilation flags can be defined
and assigned different default values.
They will be particularly useful in conjunction
with the forwarding mechanism described in \secref{sec:forward}.

For example, it may be useful to have a flag |\version|
which can be set to |draft| or |final|.
The document source will contain some conditional code
depending on the value of |\version|.
Suppose further, the flag should default to |final| for the main file
and to |draft| for child files
which is a natural assignment for editing the document.
This is achieved by placing the following code
in the preamble of the main document
(below the |\childdocmain| directive):
%
\begin{center}
\begin{tabular}{l}
|\ifchilddoc|\\
|\providecommand{\version}{draft}|\\
|\||else|\\
|\providecommand{\version}{final}|\\
|\||fi|
\end{tabular}
\end{center}
%
The definition by |\providecommand| makes sure
that previous definitions are not overwritten.
Further statements |\providecommand{\version}{...}|
can thus be added before the above code to override it.

For the main file, one might add a line
(between |\childdocmain| and the above block)
%
\begin{center}
|%\ifchilddoc\||else\providecommand{\version}{draft}\||fi|
\end{center}
%
which can be uncommented to produce a draft version.
Likewise one can add a line to the very top of a child file
(above the |\childdocof{|\textit{main}|}| directive)
%
\begin{center}
|%\providecommand{\version}{final}|
\end{center}
%
which can be uncommented to produce the final version of this child document.

%%%%%%%%%%%%%%%%%%%%%%%%%%%%%%%%%%%%%%%%%%%%%%%%%%%%%%%%%%%%%%%%%%%%%%%%%%%%%%%%
\subsection{Forwarding}
\label{sec:forward}

Different versions of the main or child documents
using compilation flags as described in \secref{sec:flags}
can be (permanently) stored in different files
for convenient compilation, viewing and distribution.
To this end, the package defines a command
to pass on compilation to a different file:

%%%%%%%%%%%%%%%%%%%%%%%%%%%%%%%%%%%%%%%%
\DescribeMacro{\childdocforward}
The command |\childdocforward| redirects processing to
another source file:
%
\begin{center}
\begin{tabular}{l}
|\input{childdoc.def}|\\
|\childdocforward[|\textit{main}|]{|\textit{dest}|}|\\
\end{tabular}
\end{center}
%
The argument \textit{dest} is the destination file
(without extension).
It should be the main file or one of the child files.
Note that further \textsf{childdoc} directives
such as |\childdocof| and |\childdocforward|
in the indicated file will be processed in this form.
The optional argument \textit{main}
passes on directly to the main file \textit{main}
while pretending to compile the child \textit{dest}.
This form behaves as if \textit{dest}
issues |\childdocof{|\textit{main}|}| right away,
and no further \textsf{childdoc} directives will be processed.

%%%%%%%%%%%%%%%%%%%%%%%%%%%%%%%%%%%%%%%%
\DescribeMacro{\...prefix}
In the alternative form |\childdocforwardprefix|,
%
\begin{center}
\begin{tabular}{l}
|\input{childdoc.def}|\\
|\childdocforwardprefix[|\textit{main}|]{|\textit{prefix}|}{|\textit{dest}|}|
\end{tabular}
\end{center}
%
the destination file is determined by a pattern
depending on the current file:
To make this work, the current file must be called
`{\textit{prefix}\hspace{0.2em}\textit{suffix}}'
with \textit{prefix} matching precisely the argument.
Processing is then passed on to the file
`{\textit{dest}\hspace{0.2em}\textit{suffix}}'.
Surely, the same effect is achieved by
directly specifying the
argument `{\textit{dest}\hspace{0.2em}\textit{suffix}}'
in the first form.
However, that requires to set up a different file
for each child. With the alternative form of the command
all these files can have exactly the same content
which simplifies setting them up and maintaining them.

For example, the following file |draft.tex|
with a compilation flag |\version| as described in \secref{sec:flags}
compiles the main document as a draft:
%
\begin{center}
\begin{tabular}{l}
|\def\version{draft}|\\
|\input{childdoc.def}|\\
|\childdocforward{|\textit{main}|}|
\end{tabular}
\end{center}
%
Likewise, the following files |final|\textit{nn}|.tex|
compile the final version of the child document
|child|\textit{nn}|.tex|:
%
\begin{center}
\begin{tabular}{l}
|\def\version{final}|\\
|\input{childdoc.def}|\\
|\childdocforwardprefix{final}{child}|
\end{tabular}
\end{center}
%

Note that when several versions of a main file and/or of each child file
are to be generated, it may be convenient to set up a |Makefile| or
shell script to automatise the process.

%%%%%%%%%%%%%%%%%%%%%%%%%%%%%%%%%%%%%%%%%%%%%%%%%%%%%%%%%%%%%%%%%%%%%%%%%%%%%%%%
\subsection{Command Line Processing}
\label{sec:commandline}

The effect of redirection files can also be achieved by invoking
the \LaTeX{} compiler with a more elaborate command line.
Most conveniently this should be done as part
of a shell script or a |Makefile|.

When using \textsf{childdoc} in the main file, the following
command lines effectively perform a redirection
(note that depending on the shell being used,
backslashes may have to be doubled: `|\|' $\to$ `|\\|'):
%
\begin{center}
|... -jobname "|\textit{target}|" |\\|"|[\textit{flags}]%
|\input{childdoc.def}\childdocforward[|\textit{main}|]{|\textit{dest}|}"|
\end{center}
%
Here \textit{target} is the name of the output file,
\textit{main} is the name of the main file
and \textit{dest} is the name of the main or child file to be processed
(all filenames without extensions).
The optional argument \textit{main} can be omitted
if \textit{main} matches \textit{dest}.
Optionally, compilation \textit{flags} can be defined via |\def| commands.
This command line makes the \TeX{} engine believe
it is compiling the file \textit{target}
whose content is specified as the latter parameter.
The provided code then forwards the processing to
\textit{main} or \textit{dest} as described in \secref{sec:forward}.

%%%%%%%%%%%%%%%%%%%%%%%%%%%%%%%%%%%%%%%%%%%%%%%%%%%%%%%%%%%%%%%%%%%%%%%%%%%%%%%%
\subsection{Include by Input}
\label{sec:input}

Including child documents by |\include| has some restrictions by design.
Most notably, the content of a child document always occupies
its own set of pages; pages cannot be shared between child documents.
Usually, this behaviour makes perfect sense
because each child document contain an essential part of the document.
However, in some situations it may be desirable to compose
a document from a collection of parts
without having mandatory page breaks between then.
For this case, the package
provides a mechanism to include parts
by |\input| which can also be processed individually.
However, by construction this mechanism
requires manual handling of the content to be output.

%%%%%%%%%%%%%%%%%%%%%%%%%%%%%%%%%%%%%%%%
\DescribeMacro{\ifchilddocmanual}
The main file should be prepared as usual, see \secref{sec:include}.
However, the document body must make a distinction
between processing of an individual part and of the main document, e.g.:
%
\begin{center}
\begin{tabular}{l}
|\ifchilddocmanual|\\
|\input{\childdocname}|\\
|\||else|\\
\textit{document body with }|\input{|\textit{part}|}|\\
|\||fi|
\end{tabular}
\end{center}
%
The conditional |\ifchilddocmanual| is true whenever
a part to be included by |\input| is being compiled,
and the name of the part is stored in |\childdocname|.

%%%%%%%%%%%%%%%%%%%%%%%%%%%%%%%%%%%%%%%%
\DescribeMacro{\childdocby}
Each part to be included by |\input| should start with:
%
\begin{center}
\begin{tabular}{l}
|\input{childdoc.def}|\\
|\childdocby{|\textit{main}|}|\\
\end{tabular}
\end{center}
%
The directive |\childdocby| is similar to |\childdocof|
described in \secref{sec:include},
but the subsequent selection of content must be done manually.
To that end, both |\ifchilddoc| and |\ifchilddocmanual|
will be true upon processing of a part,
and the name of the part is stored in |\childdocname|.
Note that |\jobname| will be set to the filename of the current part
so that each part receives an individual |.aux| file
that does not interfere with the |.aux| file(s) of the main document.
This behaviour can be altered by the alternative form
|\childdocby[*]{|\textit{main}|}| (with a non-empty optional argument)
which uses the |.aux| file of the main document
by setting |\jobname| to \textit{main}.

%%%%%%%%%%%%%%%%%%%%%%%%%%%%%%%%%%%%%%%%%%%%%%%%%%%%%%%%%%%%%%%%%%%%%%%%%%%%%%%%
\subsection{Driver Development}
\label{sec:driver}

The \textsf{childdoc} mechanism can also be use for the development
of definition files such as \LaTeX{} styles or classes.
This case differs from the above setup with multiple parts
included by |\include| in that no |\includeonly| should be invoked.
This can be achieved by starting the include file
(before |\ProvidesPackage|) with:
%
\begin{center}
\begin{tabular}{l}
|\input{childdoc.def}|\\
|\childdocforward{|\textit{main}|}|\\
\end{tabular}
\end{center}
%
or alternatively with:
%
\begin{center}
\begin{tabular}{l}
|\input{childdoc.def}|\\
|\childdocby{|\textit{main}|}|\\
\end{tabular}
\end{center}
%
Both forms have slightly different effects as described above.
The main file is prepared as usual, see \secref{sec:include}.

%%%%%%%%%%%%%%%%%%%%%%%%%%%%%%%%%%%%%%%%%%%%%%%%%%%%%%%%%%%%%%%%%%%%%%%%%%%%%%%%
\subsection{Legacy Detection}
\label{sec:detection}

The directive |\childdocmain| in the main file can detect
whether the complete document or merely a child is to be compiled
even without using the directive |\childdocof|.
This method is deprecated because it is less robust
and there is no compelling reason to use it;
it is merely provided for backward compatibility
and it may be removed in future versions.

If the detection mechanism is to be used,
it is mandatory to correctly specify
the filename of the main file as the argument of |\childdocmain|:
%
\begin{center}
\begin{tabular}{l}
|\input{childdoc.def}|\\
|\childdocmain{|\textit{main}|}|\\
\end{tabular}
\end{center}
%
If |\jobname| does not match the argument \textit{main} of |\childdocmain|,
it is assumed that |\jobname| points to the child file to be compiled.
When using |\childdocmain| with the main file specified as argument,
it suffices to start a child file
with just |\input{|\textit{main}|}|
without loading of the package and using |\childdocof|.
If instead all processing is done
with the appropriate \textsf{childdoc} directives,
the argument of \textit{main} of |\childdocmain| can be empty.

An alternative version of the command line processing described
in \secref{sec:commandline} using the detection mechanism reads:
%
\begin{center}
|... -jobname "|\textit{target}|" "|[\textit{flags}]%
[|\def\jobname{|\textit{dest}|}|]|\input{|\textit{main}|}"|
\end{center}

%%%%%%%%%%%%%%%%%%%%%%%%%%%%%%%%%%%%%%%%%%%%%%%%%%%%%%%%%%%%%%%%%%%%%%%%%%%%%%%%
\subsection{Manual Code}
\label{sec:manual}

In case one cannot be certain whether the definitions file |childdoc.def|
is installed on the target \TeX{} distribution
and one prefers not to ship it,
it is conceivable to paste a few relevant commands into the sources.

To that end, drop all statements |\input{childdoc.def}|
and perform the replacements as outlined below.
Instead of |\childdocmain{|\textit{main}|}| add the following code
to the top of the main file:
%
\begin{center}
\begin{tabular}{l}
|\||ifdefined\childdocname\endinput\||fi\newif\ifchilddoc|\\
|\edef\childdocname{\scantokens\expandafter{\jobname\noexpand}}|\\
|\def\childdocmain{|\textit{main}|}\||ifx\childdocmain\childdocname\||else|\\
|\childdoctrue\includeonly{\childdocname}\let\jobname\childdocmain\||fi|\\
\end{tabular}
\end{center}
%
Instead of |\childdocof{|\textit{main}|}| just include the main file
at the top of each child file:
%
\begin{center}
|\input{|\textit{main}|}|
\end{center}
%
A simple redirection |\childdocforward{|\textit{dest}|}| is achieved by:
%
\begin{center}
|\def\jobname{|\textit{dest}|}\input{\jobname}|
\end{center}
%
The redirection with prefix
|\childdocforwardprefix[|\textit{prefix}|]{|\textit{dest}|}|
is accomplished by:
%
\begin{center}
\begin{tabular}{l}
|{\edef\jobname{\scantokens\expandafter{\jobname\noexpand}}|\\
|\def\redirectjob |\textit{prefix}|#1~~~{\gdef\jobname{|\textit{dest}|#1}}|\\
|\expandafter\redirectjob\jobname~~~}\input{\jobname}|
\end{tabular}
\end{center}

In an alternative approach,
child documents can be compiled by a specific command line
without additional code or specific definitions:
%
\begin{center}
|... -jobname "|\textit{target}|" "|[\textit{flags}]%
|\includeonly{|\textit{dest}|}\input{|\textit{main}|}"|
\end{center}
%

%%%%%%%%%%%%%%%%%%%%%%%%%%%%%%%%%%%%%%%%%%%%%%%%%%%%%%%%%%%%%%%%%%%%%%%%%%%%%%%%
%%%%%%%%%%%%%%%%%%%%%%%%%%%%%%%%%%%%%%%%%%%%%%%%%%%%%%%%%%%%%%%%%%%%%%%%%%%%%%%%
\section{Information}

%%%%%%%%%%%%%%%%%%%%%%%%%%%%%%%%%%%%%%%%%%%%%%%%%%%%%%%%%%%%%%%%%%%%%%%%%%%%%%%%
\subsection{Copyright}

Copyright \copyright{} 2017--2018 Niklas Beisert

This work may be distributed and/or modified under the
conditions of the \LaTeX{} Project Public License, either version 1.3
of this license or (at your option) any later version.
The latest version of this license is in
  \url{http://www.latex-project.org/lppl.txt}
and version 1.3 or later is part of all distributions of \LaTeX{}
version 2005/12/01 or later.

This work has the LPPL maintenance status `maintained'.

The Current Maintainer of this work is Niklas Beisert.

This work consists of the files |README.txt|, |childdoc.ins| and |childdoc.dtx|
as well as the derived files |childdoc.def|, |cdocsamp.tex|
with |cdocsch1.tex|, |cdocsch2.tex|, |cdocspt3.tex|, |cdocspt4.tex|,
|cdocsdrf.tex|, |cdocsfn1.tex|, |cdocsfn2.tex|
as well as |childdoc.pdf|.

%%%%%%%%%%%%%%%%%%%%%%%%%%%%%%%%%%%%%%%%%%%%%%%%%%%%%%%%%%%%%%%%%%%%%%%%%%%%%%%%
\subsection{Files and Installation}

The package consists of the files:
%
\begin{center}
\begin{tabular}{ll}
    |README.txt|   & readme file \\
    |childdoc.ins| & installation file \\
    |childdoc.dtx| & source file \\
    |childdoc.def| & definition file \\
    |cdocsamp.tex| & sample main file \\
    |cdocsch1.tex| & sample include file \\
    |cdocsch2.tex| & sample include file \\
    |cdocspt3.tex| & sample part file \\
    |cdocspt4.tex| & sample part file \\
    |cdocsdrf.tex| & sample redirection file \\
    |cdocsfn1.tex| & sample redirection file \\
    |cdocsfn2.tex| & sample redirection file \\
    |childdoc.pdf| & manual
\end{tabular}
\end{center}
%
The distribution consists of the files
|README.txt|, |childdoc.ins| and |childdoc.dtx|.
%
\begin{itemize}
\item
Run (pdf)\LaTeX{} on |childdoc.dtx|
to compile the manual |childdoc.pdf| (this file).
\item
Run \LaTeX{} on |childdoc.ins| to create the definitions file |childdoc.def|
and the sample |cdocsamp.tex| with include files
|cdocsch1.tex|, |cdocsch2.tex|, |cdocspt3.tex|, |cdocspt4.tex|,
|cdocsdrf.tex|, |cdocsfn1.tex|, |cdocsfn2.tex|.
Then copy the file |childdoc.def| to an appropriate directory of your \LaTeX{}
distribution, e.g.\ \textit{texmf-root}|/tex/latex/childdoc|.
\end{itemize}

%%%%%%%%%%%%%%%%%%%%%%%%%%%%%%%%%%%%%%%%%%%%%%%%%%%%%%%%%%%%%%%%%%%%%%%%%%%%%%%%
\subsection{Related CTAN Packages}

There are several other packages which offer a similar functionality:
%
\begin{itemize}
\item
The packages
\href{http://ctan.org/pkg/docmute}{\textsf{docmute}},
\href{http://ctan.org/pkg/includex}{\textsf{includex}} and
\href{http://ctan.org/pkg/standalone}{\textsf{standalone}}
provide commands to include only the document body of
a child file thus allowing both files to be compiled individually.
\item
The packages \href{http://ctan.org/pkg/subdocs}{\textsf{subdocs}}
and \href{http://ctan.org/pkg/subfiles}{\textsf{subfiles}}
provide structures in which the main and child documents can be
encapsulated and allowing them to be compiled individually.
The inclusion mechanism is different from the conventional |\include|.
\item
The package \href{http://ctan.org/pkg/combine}{\textsf{combine}}
is an elaborate solution to combine several documents into one.
\end{itemize}
%
See also the CTAN topic \href{http://ctan.org/topic/subdocs}{\textsf{subdocs}}
for further related packages.
The present package differs from the above solutions in that
a document structure constructed with the conventional |\include| mechanism
just needs two extra commands at the top of every file
such that all constituent files can be compiled individually.

%%%%%%%%%%%%%%%%%%%%%%%%%%%%%%%%%%%%%%%%%%%%%%%%%%%%%%%%%%%%%%%%%%%%%%%%%%%%%%%%
%\subsection{Feature Suggestions}
%
%The following is a list of features which may be useful for future
%versions of this package:
%%
%\begin{itemize}
%\item
%\ldots
%\end{itemize}

%%%%%%%%%%%%%%%%%%%%%%%%%%%%%%%%%%%%%%%%%%%%%%%%%%%%%%%%%%%%%%%%%%%%%%%%%%%%%%%%
\subsection{Revision History}

%%%%%%%%%%%%%%%%%%%%%%%%%%%%%%%%%%%%%%%%
\paragraph{v2.0:} 2018/12/30

\begin{itemize}
\item
immediate forward processing
\item
added |\childdocby| mechanism
\item
manual restructured
\end{itemize}

%%%%%%%%%%%%%%%%%%%%%%%%%%%%%%%%%%%%%%%%
\paragraph{v1.6:} 2018/01/17

\begin{itemize}
\item
application for development of include files
\item
corrections to manual
\end{itemize}

%%%%%%%%%%%%%%%%%%%%%%%%%%%%%%%%%%%%%%%%
\paragraph{v1.5:} 2017/05/21

\begin{itemize}
\item
more complete structuring introduced
\item
|\childdocof| introduced
\item
|\childdoc| renamed to |\childdocmain|
\item
|\childredirect| renamed to |\childdocforward| and |\childdocforwardprefix|
and functionality expanded
\end{itemize}

%%%%%%%%%%%%%%%%%%%%%%%%%%%%%%%%%%%%%%%%
\paragraph{v1.0:} 2017/04/27

\begin{itemize}
\item
manual and install package
\item
first version published on CTAN
\end{itemize}

%%%%%%%%%%%%%%%%%%%%%%%%%%%%%%%%%%%%%%%%
\paragraph{v0.6:} 2017/04/26

\begin{itemize}
\item
redirection mechanism added
\end{itemize}

%%%%%%%%%%%%%%%%%%%%%%%%%%%%%%%%%%%%%%%%
\paragraph{v0.5:} 2017/04/26

\begin{itemize}
\item
functionality in definition file
\end{itemize}


%%%%%%%%%%%%%%%%%%%%%%%%%%%%%%%%%%%%%%%%%%%%%%%%%%%%%%%%%%%%%%%%%%%%%%%%%%%%%%%%
%%%%%%%%%%%%%%%%%%%%%%%%%%%%%%%%%%%%%%%%%%%%%%%%%%%%%%%%%%%%%%%%%%%%%%%%%%%%%%%%
%%%%%%%%%%%%%%%%%%%%%%%%%%%%%%%%%%%%%%%%%%%%%%%%%%%%%%%%%%%%%%%%%%%%%%%%%%%%%%%%
\appendix

\settowidth\MacroIndent{\rmfamily\scriptsize 000\ }

 \DocInput{childdoc.dtx}

\end{document}
%</driver>
% \fi
%
% %%%%%%%%%%%%%%%%%%%%%%%%%%%%%%%%%%%%%%%%%%%%%%%%%%%%%%%%%%%%%%%%%%%%%%%%%%%%%%
% %%%%%%%%%%%%%%%%%%%%%%%%%%%%%%%%%%%%%%%%%%%%%%%%%%%%%%%%%%%%%%%%%%%%%%%%%%%%%%
% \section{Sample}
%\iffalse
%<*samplemain>
%\fi
%
% The following presents a sample document
% with two chapters, two parts, a title page,
% a compile flag as well as three forwarding files to set the flag.
% It consists of eight |.tex| files:
% \begin{center}
% \begin{tabular}{ll}
% |cdocsamp.tex|&main file\\
% |cdocsch1.tex|&include file for chapter 1\\
% |cdocsch2.tex|&include file for chapter 2\\
% |cdocspt3.tex|&include file for part 3\\
% |cdocspt4.tex|&include file for part 4\\
% |cdocsdrf.tex|&forwarding file for main file in draft mode\\
% |cdocsfi1.tex|&forwarding file for final version of chapter 1\\
% |cdocsfi2.tex|&forwarding file for final version of chapter 2\\
% \end{tabular}
% \end{center}
% Each of the eight files can be compiled directly by the \LaTeX{} compiler.
%
% %%%%%%%%%%%%%%%%%%%%%%%%%%%%%%%%%%%%%%
% \paragraph{Main File.}
%
% The main file is called |cdocsamp.tex|.
%
% Load the \textsf{childdoc} definitions and
% declare the filename for the main document:
%    \begin{macrocode}
\input{childdoc.def}
\childdocmain{}
%    \end{macrocode}

% Optional override for |\version| flag:
%    \begin{macrocode}
%%\ifchilddoc\else\providecommand{\version}{draft}\fi
%    \end{macrocode}

% Define the default values for the |\version| flag
% (|final| for the main file and |draft| for childs):
%    \begin{macrocode}
\ifchilddoc
\providecommand{\version}{draft}
\else
\providecommand{\version}{final}
\fi
%    \end{macrocode}

% Load the standard document class:
%    \begin{macrocode}
\documentclass[12pt]{article}
%    \end{macrocode}

% Start the document body:
%    \begin{macrocode}
\begin{document}
%    \end{macrocode}

% Declare a title page.
% Print title, part of document being processed and version flag:
%    \begin{macrocode}
\addtocounter{page}{-1}
\begin{center}
{\LARGE\bfseries{}childdoc example\par}
\vspace{1cm}
\ifchilddoc
\ifchilddocmanual part\else chapter\fi:
`\childdocname' of `\childdocjob'\par
\else
main document: `\childdocjob'\par
\fi
version: \version\par
\end{center}
\newpage
%    \end{macrocode}

% Manually include selected file,
% otherwise process as usual:
%    \begin{macrocode}
\ifchilddocmanual
\section*{part `\childdocname'}
\input{\childdocname}
\else
%    \end{macrocode}

% Include the two chapters:
%    \begin{macrocode}
\include{cdocsch1}
\include{cdocsch2}
%    \end{macrocode}

% Include the two parts unless only chapters should be displayed:
%    \begin{macrocode}
\ifchilddoc\else
\section{part three}
\input{cdocspt3}
\section{part four}
\input{cdocspt4}
\fi
%    \end{macrocode}

% Process as usual until here:
%    \begin{macrocode}
\fi
%    \end{macrocode}

% End of document body:
%    \begin{macrocode}
\end{document}
%    \end{macrocode}
%\iffalse
%</samplemain>
%\fi
%
% %%%%%%%%%%%%%%%%%%%%%%%%%%%%%%%%%%%%%%
% \paragraph{Chapter Include Files.}
%
% The include files are called |cdocsch1.tex| and |cdocsch2.tex|.
%
%\iffalse
%<*samplechap1|samplechap2>
%\fi

% Optional override for |\version| flag:
%    \begin{macrocode}
%%\providecommand{\version}{final}
%    \end{macrocode}

% Include the main document:
%    \begin{macrocode}
\input{childdoc.def}
\childdocof{cdocsamp}
%    \end{macrocode}

%\iffalse
%</samplechap1|samplechap2>
%\fi
%
%\iffalse
%<*samplechap1>
%\fi
% Some text for chapter 1:
%    \begin{macrocode}
\section{one}
some text in chapter one
%    \end{macrocode}

%\iffalse
%</samplechap1>
%\fi
% Some text for chapter 2:
%\iffalse
%<*samplechap2>
%\fi
%    \begin{macrocode}
\section{two}
more text in chapter two
%    \end{macrocode}

%\iffalse
%</samplechap2>
%\fi
%
% %%%%%%%%%%%%%%%%%%%%%%%%%%%%%%%%%%%%%%
% \paragraph{Part Include Files.}
%
% The include files are called |cdocspt3.tex| and |cdocspt4.tex|.
%
%\iffalse
%<*samplepart3|samplepart4>
%\fi

% Optional override for |\version| flag:
%    \begin{macrocode}
%%\providecommand{\version}{final}
%    \end{macrocode}

% Include the main document:
%    \begin{macrocode}
\input{childdoc.def}
\childdocby{cdocsamp}
%    \end{macrocode}

%\iffalse
%</samplepart3|samplepart4>
%\fi
%
%\iffalse
%<*samplepart3>
%\fi
% Some text for part 3:
%    \begin{macrocode}
some text in part three
%    \end{macrocode}

%\iffalse
%</samplepart3>
%\fi
% Some text for part 4:
%\iffalse
%<*samplepart4>
%\fi
%    \begin{macrocode}
more text in part four
%    \end{macrocode}

%\iffalse
%</samplepart4>
%\fi
%
% %%%%%%%%%%%%%%%%%%%%%%%%%%%%%%%%%%%%%%
% \paragraph{Forwarding for a Complete Draft.}
%
% The following forwarding file |cdocsdrf.tex|
% compiles the main document in draft mode:
%\iffalse
%<*sampledraft>
%\fi
%    \begin{macrocode}
\def\version{draft}
\input{childdoc.def}
\childdocforward{cdocsamp}
%    \end{macrocode}

%\iffalse
%</sampledraft>
%\fi
%
% %%%%%%%%%%%%%%%%%%%%%%%%%%%%%%%%%%%%%%
% \paragraph{Forwarding for Final Version of the Chapters.}
%
% The following forwarding files |cdocsfn1.tex| and |cdocsfn2.tex|
% (with identical content)
% compile the final versions of the child documents
% |cdocsch1.tex| and |cdocsch2.tex|, respectively:
%\iffalse
%<*samplefinal>
%\fi
%    \begin{macrocode}
\def\version{final}
\input{childdoc.def}
\childdocforwardprefix[cdocsamp]{cdocsfn}{cdocsch}
%    \end{macrocode}

%\iffalse
%</samplefinal>
%\fi
%
% %%%%%%%%%%%%%%%%%%%%%%%%%%%%%%%%%%%%%%
% \paragraph{Command Line Processing.}
%
% The following three command lines generate the output files
% |cdocscld|, |cdocscl1| and |cdocscl2|
% which should be identical to
% |cdocsdrf|, |cdocsch1| and |cdocsfn2|, respectively:
% \begin{center}
% \begin{tabular}{l}
% |latex -jobname cdocscld \|\\
% |  "\def\version{draft}\input{childdoc.def}\childdocforward{cdocsamp}"|\\
% |latex -jobname cdocscl1 \|\\
% |  "\input{childdoc.def}\childdocforward[cdocsamp]{cdocsch1}"|\\
% |latex -jobname cdocscl2 \|\\
% |  "\def\version{final}\input{childdoc.def}\childdocforward{cdocsch2}"|
% \end{tabular}
% \end{center}
% Note that the trailing backslash on each first line
% merely continues the input to the second line
% (for convenient cut ant paste).
% Furthermore, the command |latex| can be replaced by any
% of its alternative versions such as |pdflatex|.
%
% %%%%%%%%%%%%%%%%%%%%%%%%%%%%%%%%%%%%%%%%%%%%%%%%%%%%%%%%%%%%%%%%%%%%%%%%%%%%%%
% %%%%%%%%%%%%%%%%%%%%%%%%%%%%%%%%%%%%%%%%%%%%%%%%%%%%%%%%%%%%%%%%%%%%%%%%%%%%%%
% \section{Implementation}
%\iffalse
%<*package>
%\fi
%
% This section describes the definitions file |childdoc.def|.

% The definitions cannot be loaded using |\usepackage| or |\RequirePackage|
% which has a mechanism to prevent loading a style file more than once.
% When loading the definitions by means of |\input|
% multiple instances have to be prevented manually:
%\iffalse
%This code needs to be before the `\ProvidesFile' directive
%which is defined at the beginning of this file.
%Therefore it is also placed there and commented out here.
%</package>
%<*discard>
%\fi
%    \begin{macrocode}
\ifdefined\childdocmain\endinput\fi
%    \end{macrocode}
%\iffalse
%</discard>
%<*package>
%\fi
%
% \macro{\ifchilddoc}
% \macro{\ifchilddocmanual}
% The conditional |\ifchilddoc| tells whether a
% child (true) or main (false) document is being compiled.
% The conditional |\ifchilddocmanual| tells whether
% the |\includeonly| mechanism is used (false) or
% the selection of child files must be performed manually (true).
% The definitions initialise to false:
%    \begin{macrocode}
\newif\ifchilddoc
\newif\ifchilddocmanual
%    \end{macrocode}

% \macro{\childdocname}
% \macro{\childdocjob}
% The macro |\childdocname| stores the name of the main document
% to be compiled. The macro |\childdocjob| stores the name of
% the document on which the \LaTeX{} compiler was originally invoked.
% The content of |\jobname| cannot be compared
% to filenames specified in the source due to different catcodes.
% The following code rescans |\jobname|, stores the result
% in |\childdocname| and saves a copy in |\childdocjob|:
%    \begin{macrocode}
\edef\childdocname{\scantokens\expandafter{\jobname\noexpand}}
\let\childdocjob\childdocname
%    \end{macrocode}

% \macro{\childdocdisable}
% The macro |\childdocdisable| prevents the main file
% from being processed more than once.
% At this stage, the main document command |\childdocmain|
% is assumed to be called once again where it should do nothing.
% Any subsequent call to it should prevent
% a secondary processing of the main document
% It overwrites the forwarding commands
% |\childdocof| and |\childdocforward|
% with empty macros to prevent further inclusions of the main document:
%    \begin{macrocode}
\newcommand{\childdocdisable}
{
  \renewcommand{\childdocmain}[1]{\renewcommand{\childdocmain}[1]{\endinput}}
  \renewcommand{\childdocof}[1]{}
  \renewcommand{\childdocby}[2][]{}
  \renewcommand{\childdocforward}[2][]{}
  \renewcommand{\childdocdisable}{}
}
%    \end{macrocode}

% \macro{\childdocmain}
% The macro |\childdocmain| is to be called at the top of the main file
% with nothing or the main filename (without extension) as argument.
% First, it breaks loops.
% If the argument is not empty and does not match |\childdocname|
% (which is set by the first inclusion of |childdoc.def|),
% |\ifchilddoc| is set to true, |\includeonly| is applied to the child file
% and |\jobname| is set to the main file
% (for proper handling of |.aux| files):
%    \begin{macrocode}
\newcommand{\childdocmain}[1]
{
  \childdocdisable\childdocmain{}
  \if?#1?\else
    \begingroup
      \def\childdoctmp{#1}
      \ifx\childdoctmp\childdocname
        \def\childdoctmp{}
      \else
        \def\childdoctmp
        {
          \childdoctrue
          \includeonly{\childdocname}
          \def\childdocjob{#1}
          \def\jobname{#1}
        }
      \fi
      \expandafter
    \endgroup
    \childdoctmp
  \fi
}
%    \end{macrocode}

% \macro{\childdocof}
% The command |\childdocof| redirects
% compilation to the main file |#1|.
%    \begin{macrocode}
\newcommand{\childdocof}[1]
{
  \childdocdisable
  \childdoctrue
  \includeonly{\childdocname}
  \def\jobname{#1}
  \def\childdocjob{#1}
  \input{#1}
}
%    \end{macrocode}

% \macro{\childdocby}
% The command |\childdocby| ....
%    \begin{macrocode}
\newcommand{\childdocby}[2][]
{
  \childdocdisable
  \childdoctrue
  \childdocmanualtrue
  \if?#1?\else
    \def\jobname{#2}
  \fi
  \def\childdocjob{#2}
  \input{#2}
  \endinput
}
%    \end{macrocode}

% \macro{\childdocforward}
% The command |\childdocforward| redirects
% compilation to the main file or
% (if the optional argument is given) a child file.
% Parameters are set as if the main file
% or a child file starting with |\childdocof| was compiled.
% Then compilation is handed over to the main file:
%    \begin{macrocode}
\newcommand{\childdocforward}[2][]
{
  \begingroup
    \if?#1?
      \def\childdoctmp
      {
        \def\childdocname{#2}
        \def\childdocjob{#2}
        \def\jobname{#2}
        \input{#2}
        \endinput
      }
    \else
      \def\childdoctmp
      {
        \childdocdisable
        \def\childdocname{#2}
        \childdoctrue
        \includeonly{#2}
        \def\childdocjob{#1}
        \def\jobname{#1}
        \input{#1}
        \endinput
      }
    \fi
    \expandafter
  \endgroup
  \childdoctmp
}
%    \end{macrocode}

% \macro{\childdocforwardprefix}
% The command |\childdocforwardprefix| redirects
% compilation to the main or a child file by means of a pattern.
% The prefix |#1| in the current filename is replaced by |#2|
% and the suffix of the current filename is kept
% (it is assumed that the filename does not contain the substring `|~~~|'
% which is used as a delimiter).
% Compilation is handed over to the new file by |\childdocforward|:
%    \begin{macrocode}
\newcommand{\childdocforwardprefix}[3][]
{
  \begingroup
    \def\childdocextract #2##1~~~{\def\childdoctmp{\childdocforward[#1]{#3##1}}}
    \expandafter\childdocextract\childdocname~~~
    \expandafter
  \endgroup
  \childdoctmp
}
%    \end{macrocode}

% \macro{\childdoc}
% The deprecated macro |\childdoc| is a legacy version of |\childdocmain|:
%    \begin{macrocode}
\newcommand{\childdoc}{\childdocmain}
%    \end{macrocode}

% \macro{\childdocredirect}
% The deprecated macro |\childdocredirect| is a legacy version
% of |\childdocforward| and |\childdocforwardprefix|:
%    \begin{macrocode}
\newcommand{\childdocredirect}[2][]
{
  \begingroup
    \if?#1?
      \def\childdoctmp{\childdocforward{#2}}
    \else
      \def\childdoctmp{\childdocforwardprefix{#1}{#2}}
    \fi
    \expandafter
  \endgroup
  \childdoctmp
}
%    \end{macrocode}

%\iffalse
%</package>
%\fi
%
\endinput
|\\
|\childdocby{|\textit{main}|}|\\
\end{tabular}
\end{center}
%
Both forms have slightly different effects as described above.
The main file is prepared as usual, see \secref{sec:include}.

%%%%%%%%%%%%%%%%%%%%%%%%%%%%%%%%%%%%%%%%%%%%%%%%%%%%%%%%%%%%%%%%%%%%%%%%%%%%%%%%
\subsection{Legacy Detection}
\label{sec:detection}

The directive |\childdocmain| in the main file can detect
whether the complete document or merely a child is to be compiled
even without using the directive |\childdocof|.
This method is deprecated because it is less robust
and there is no compelling reason to use it;
it is merely provided for backward compatibility
and it may be removed in future versions.

If the detection mechanism is to be used,
it is mandatory to correctly specify
the filename of the main file as the argument of |\childdocmain|:
%
\begin{center}
\begin{tabular}{l}
|% \iffalse
%
% childdoc.dtx Copyright (C) 2017-2018 Niklas Beisert
%
% This work may be distributed and/or modified under the
% conditions of the LaTeX Project Public License, either version 1.3
% of this license or (at your option) any later version.
% The latest version of this license is in
%   http://www.latex-project.org/lppl.txt
% and version 1.3 or later is part of all distributions of LaTeX
% version 2005/12/01 or later.
%
% This work has the LPPL maintenance status `maintained'.
%
% The Current Maintainer of this work is Niklas Beisert.
%
% This work consists of the files childdoc.dtx and childdoc.ins
% and the derived files childdoc.def and cdocsamp.tex with
% cdocsch1.tex, cdocsch2.tex, cdocsdrf.tex, cdocsfn1.tex, cdocsfn2.tex.
%
%<package>\ifdefined\childdocmain\endinput\fi
%<package>\ProvidesFile{childdoc.def}[2018/12/30 v2.0 child document driver]
%<samplemain>\ProvidesFile{cdocsamp.tex}[2018/12/30 v2.0 sample for childdoc]
%<*driver>
%\ProvidesFile{childdoc.drv}[2018/12/30 v2.0 childdoc reference manual file]
\PassOptionsToClass{10pt,a4paper}{article}
\documentclass{ltxdoc}

\usepackage[margin=35mm]{geometry}
\usepackage{hyperref}
\usepackage{hyperxmp}
\usepackage[usenames]{color}

\hypersetup{colorlinks=true}
\hypersetup{pdfstartview=FitH}
\hypersetup{pdfpagemode=UseNone}
\hypersetup{pdfsource={}}
\hypersetup{pdflang={en-UK}}
\hypersetup{pdfcopyright={Copyright 2017-2018 Niklas Beisert.
  This work may be distributed and/or modified under the
  conditions of the LaTeX Project Public License, either version 1.3
  of this license or (at your option) any later version.}}
\hypersetup{pdflicenseurl={http://www.latex-project.org/lppl.txt}}
\hypersetup{pdfcontactaddress={ETH Zurich, ITP, HIT K,
  Wolfgang-Pauli-Strasse 27}}
\hypersetup{pdfcontactpostcode={8093}}
\hypersetup{pdfcontactcity={Zurich}}
\hypersetup{pdfcontactcountry={Switzerland}}
\hypersetup{pdfcontactemail={nbeisert@itp.phys.ethz.ch}}
\hypersetup{pdfcontacturl={http://people.phys.ethz.ch/\xmptilde nbeisert/}}

\newcommand{\secref}[1]{\hyperref[#1]{section \ref*{#1}}}

\parskip1ex
\parindent0pt
\let\olditemize\itemize
\def\itemize{\olditemize\parskip0pt}

\begin{document}

\title{The \textsf{childdoc} Package}
\hypersetup{pdftitle={The childdoc Package}}
\author{Niklas Beisert\\[2ex]
  Institut f\"ur Theoretische Physik\\
  Eidgen\"ossische Technische Hochschule Z\"urich\\
  Wolfgang-Pauli-Strasse 27, 8093 Z\"urich, Switzerland\\[1ex]
  \href{mailto:nbeisert@itp.phys.ethz.ch}
  {\texttt{nbeisert@itp.phys.ethz.ch}}}
\hypersetup{pdfauthor={Niklas Beisert}}
\hypersetup{pdfsubject={Manual for the LaTeX2e Package childdoc}}
\date{30 December 2018, \textsf{v2.0}}
\maketitle

\begin{abstract}\noindent
\textsf{childdoc} is a \LaTeXe{} package
that enables the direct compilation
of document sections included by |\include|
to individual files.
\end{abstract}

\begingroup
\parskip0ex
\tableofcontents
\endgroup

%%%%%%%%%%%%%%%%%%%%%%%%%%%%%%%%%%%%%%%%%%%%%%%%%%%%%%%%%%%%%%%%%%%%%%%%%%%%%%%%
%%%%%%%%%%%%%%%%%%%%%%%%%%%%%%%%%%%%%%%%%%%%%%%%%%%%%%%%%%%%%%%%%%%%%%%%%%%%%%%%
\section{Introduction}

\LaTeX{} provides a mechanism to structure a large document (such as a book)
into a main file and several child files (containing the chapters)
using the |\include| command.
This mechanism is beneficial for documents
which span hundreds of pages in order to
make the source file(s) more manageable.
Moreover, compilation can be restricted to
selected child files by means of the |\includeonly| command.
The latter feature can be used to reduce the compilation time while editing
(this was significantly more useful in the earlier days of \LaTeX{})
or to generate a smaller document which is easier to navigate.
Another application of |\includeonly| is to generate
documents consisting of selected parts of the complete document.

However, there are a few drawbacks of the plain |\include| mechanism:
\begin{itemize}
\item
The child files cannot be compiled on their own,
they can only be compiled via the main file.
A naive editing environment
(such as a text editor with an option
to have the current file processed by \LaTeX)
may require one to switch to the main file before compiling;
attempting to compile the child file produces errors.
\item
The main file must be modified (each time)
to adjust the |\includeonly| command
to the present needs. This easily leaves the main file in a messy state.
\item
The generated document will always carry the filename
of the main document. This is inconvenient if
several child files are to be compiled and
to be kept for distribution.
\end{itemize}

The present package provides a simple interface
to make child files individually compilable by \LaTeX{}.
Compiling a child file then has the same effect as compiling
the main file with an |\includeonly| command
to select the appropriate child.
Moreover the generated document will carry the name of the child
rather than the main file.
This resolves all three above issues.

This feature is meant to make the editing of books,
thesis documents and lecture notes somewhat more convenient.
However, the package can also be used efficiently for
composing a series of documents (such as exercise sheets)
which are typically distributed individually.
It then assists the author in generating the individual documents
(potentially in different versions)
as well as a document containing the collected series.
Another application is in developing style files
or other kinds of included material
where compilation of the style file could redirect
to a sample or test file.

%%%%%%%%%%%%%%%%%%%%%%%%%%%%%%%%%%%%%%%%%%%%%%%%%%%%%%%%%%%%%%%%%%%%%%%%%%%%%%%%
%%%%%%%%%%%%%%%%%%%%%%%%%%%%%%%%%%%%%%%%%%%%%%%%%%%%%%%%%%%%%%%%%%%%%%%%%%%%%%%%
\section{Usage}

First of all, the package \textsf{childdoc} is \emph{not} a standard
\LaTeXe{} |.sty| style file! Therefore it needs to be invoked in
a non-standard way.

%%%%%%%%%%%%%%%%%%%%%%%%%%%%%%%%%%%%%%%%%%%%%%%%%%%%%%%%%%%%%%%%%%%%%%%%%%%%%%%%
\subsection{Included Files}
\label{sec:include}

%%%%%%%%%%%%%%%%%%%%%%%%%%%%%%%%%%%%%%%%
\DescribeMacro{\childdocmain}
To use the package, add the commands
\begin{center}
\begin{tabular}{l}
|\input{childdoc.def}|\\
|\childdocmain{}|\\
\end{tabular}
\end{center}
at the very top of the main \LaTeX{} file,
in particular \emph{before} the |\documentclass| statement!
The argument of |\childdocmain| should be left empty
(but it must be present).

%%%%%%%%%%%%%%%%%%%%%%%%%%%%%%%%%%%%%%%%
\DescribeMacro{\childdocof}
Furthermore, add the commands
\begin{center}
\begin{tabular}{l}
|\input{childdoc.def}|\\
|\childdocof{|\textit{main}|}|\\
\end{tabular}
\end{center}
at the top of every child file \textit{child}
which is included by |\include{|\textit{child}|}|
from within the main file
(or at least for those files to be compiled individually).
The argument \textit{main} must be the filename of the main file.

There are a couple of
considerations in setting up the main and child documents:

%%%%%%%%%%%%%%%%%%%%%%%%%%%%%%%%%%%%%%%%
\paragraph{Restrictions.}

Please note the following restrictions:
\begin{itemize}
\item
|\childdocmain| must be called with one argument \textit{main}
to ensure compatibility with earlier version of the package.
It must either be empty (|\childdocmain{}|)
or precisely match the filename of the main file in which it is specified.
See \secref{sec:detection} for further information.
\item
The filename \textit{main} must be specified without the |.tex| extension.
\item
The filename \textit{main} is case sensitive
(even in case-insensitive file systems)
due to internal string comparison.
\item
The argument \textit{main} should be fully expanded, it cannot be a macro.
\item
Subdirectories and special characters should be avoided in filenames.
\item
The command |\childdocmain{|\textit{main}|}| must be followed by a whitespace.
It should not be followed immediately by another command
or by a comment mark `|%|'.
This is because the \TeX{} parser reads the token immediately following
the argument of |\childdocmain| and puts it
at the beginning of every child section;
however, a white\-space is ignored.
\end{itemize}

%%%%%%%%%%%%%%%%%%%%%%%%%%%%%%%%%%%%%%%%
\paragraph{Content of Main File.}

It is advisable to place all content in the child files included by |\include|.
Any output contained in the main file will appear in all child documents
unless suppressed manually;
it cannot be suppressed automatically by the |\includeonly| directive
and thus should normally be avoided.
A method to include some content in the main file
by means of conditional processing is described in \secref{sec:conditional}.

%%%%%%%%%%%%%%%%%%%%%%%%%%%%%%%%%%%%%%%%
\paragraph{Page Numbering.}

When only a part of the document is compiled,
the appropriate numbering of pages
(as well as other status parameters)
is determined from the |.aux| files.
The latter contain information from previous passes.
However this information needs to propagate through
all intermediate child documents.
Therefore the page numbering in child documents may well
be inconsistent until the complete document is compiled at least once.

A useful (if unconventional) way to always ensure a consistent
page numbering is to restart the numbering in each child document
and denote the pages by `\textit{child}|.|\textit{page}'
where \textit{child} represents the chapter/section number of the child file.
This can be achieved by the command
|\numberwithin{page}{|\textit{child}|}|
of the \textsf{amsmath} package
where \textit{child} can be |chapter| or |section|
depending on the chosen structuring.
Alternatively, one can modify the macro |\thepage| appropriately
and reset the counter |page| at the start of each child file.

%%%%%%%%%%%%%%%%%%%%%%%%%%%%%%%%%%%%%%%%%%%%%%%%%%%%%%%%%%%%%%%%%%%%%%%%%%%%%%%%
\subsection{Conditional Processing}
\label{sec:conditional}

The package provides a mechanism to compile different versions
of a document. To customise the versions further some conditional processing
can come in handy to distinguish which version is being compiled.
The package provides two macros to describe the compilation context:

%%%%%%%%%%%%%%%%%%%%%%%%%%%%%%%%%%%%%%%%
\DescribeMacro{\ifchilddoc}
The conditional |\ifchilddoc| distinguishes between the compilation of
child documents and the main document:
%
\begin{center}
|\ifchilddoc |\textit{child-code}| |[|\||else |\textit{main-code}]| \||fi|
\end{center}

%%%%%%%%%%%%%%%%%%%%%%%%%%%%%%%%%%%%%%%%
\DescribeMacro{\childdocname}
\DescribeMacro{\childdocjob}
The macro |\childdocname| contains the filename (without extension)
of the main or child file being processed.
Note that |\childdocjob| will always contain the name of the main file.

%%%%%%%%%%%%%%%%%%%%%%%%%%%%%%%%%%%%%%%%
\paragraph{Title Page.}

Conditional processing can be used to include a title or banner page
in the main document when proper precautions are taken.
Importantly, the code in the main file should ensure that the page counter
(as well as other status parameters which are stored in the |.aux| files)
takes the same value after the conditional processing.
Otherwise the page numbers may take divergent values
depending on which part is compiled.

For example, a title page could be declared by:
%
\begin{center}
\begin{tabular}{l}
|\ifchilddoc\||else|\\
|\addtocounter{page}{-1}|\\
\textit{code for title page}\\
|\newpage|\\
|\||fi|
\end{tabular}
\end{center}
%
A banner page for the child documents can be generated by:
%
\begin{center}
\begin{tabular}{l}
|\ifchilddoc|\\
|\addtocounter{page}{-1}|\\
\textit{code for banner page}\\
|\newpage|\\
|\||fi|
\end{tabular}
\end{center}
%
Here one could write a message such as:
\begin{center}
|This is the part \childdocname{} of \childdocjob{}.|
\end{center}

%%%%%%%%%%%%%%%%%%%%%%%%%%%%%%%%%%%%%%%%%%%%%%%%%%%%%%%%%%%%%%%%%%%%%%%%%%%%%%%%
\subsection{Flags}
\label{sec:flags}

The package makes it easy to generate different versions
of the main or child documents.
To this end compilation flags can be defined
and assigned different default values.
They will be particularly useful in conjunction
with the forwarding mechanism described in \secref{sec:forward}.

For example, it may be useful to have a flag |\version|
which can be set to |draft| or |final|.
The document source will contain some conditional code
depending on the value of |\version|.
Suppose further, the flag should default to |final| for the main file
and to |draft| for child files
which is a natural assignment for editing the document.
This is achieved by placing the following code
in the preamble of the main document
(below the |\childdocmain| directive):
%
\begin{center}
\begin{tabular}{l}
|\ifchilddoc|\\
|\providecommand{\version}{draft}|\\
|\||else|\\
|\providecommand{\version}{final}|\\
|\||fi|
\end{tabular}
\end{center}
%
The definition by |\providecommand| makes sure
that previous definitions are not overwritten.
Further statements |\providecommand{\version}{...}|
can thus be added before the above code to override it.

For the main file, one might add a line
(between |\childdocmain| and the above block)
%
\begin{center}
|%\ifchilddoc\||else\providecommand{\version}{draft}\||fi|
\end{center}
%
which can be uncommented to produce a draft version.
Likewise one can add a line to the very top of a child file
(above the |\childdocof{|\textit{main}|}| directive)
%
\begin{center}
|%\providecommand{\version}{final}|
\end{center}
%
which can be uncommented to produce the final version of this child document.

%%%%%%%%%%%%%%%%%%%%%%%%%%%%%%%%%%%%%%%%%%%%%%%%%%%%%%%%%%%%%%%%%%%%%%%%%%%%%%%%
\subsection{Forwarding}
\label{sec:forward}

Different versions of the main or child documents
using compilation flags as described in \secref{sec:flags}
can be (permanently) stored in different files
for convenient compilation, viewing and distribution.
To this end, the package defines a command
to pass on compilation to a different file:

%%%%%%%%%%%%%%%%%%%%%%%%%%%%%%%%%%%%%%%%
\DescribeMacro{\childdocforward}
The command |\childdocforward| redirects processing to
another source file:
%
\begin{center}
\begin{tabular}{l}
|\input{childdoc.def}|\\
|\childdocforward[|\textit{main}|]{|\textit{dest}|}|\\
\end{tabular}
\end{center}
%
The argument \textit{dest} is the destination file
(without extension).
It should be the main file or one of the child files.
Note that further \textsf{childdoc} directives
such as |\childdocof| and |\childdocforward|
in the indicated file will be processed in this form.
The optional argument \textit{main}
passes on directly to the main file \textit{main}
while pretending to compile the child \textit{dest}.
This form behaves as if \textit{dest}
issues |\childdocof{|\textit{main}|}| right away,
and no further \textsf{childdoc} directives will be processed.

%%%%%%%%%%%%%%%%%%%%%%%%%%%%%%%%%%%%%%%%
\DescribeMacro{\...prefix}
In the alternative form |\childdocforwardprefix|,
%
\begin{center}
\begin{tabular}{l}
|\input{childdoc.def}|\\
|\childdocforwardprefix[|\textit{main}|]{|\textit{prefix}|}{|\textit{dest}|}|
\end{tabular}
\end{center}
%
the destination file is determined by a pattern
depending on the current file:
To make this work, the current file must be called
`{\textit{prefix}\hspace{0.2em}\textit{suffix}}'
with \textit{prefix} matching precisely the argument.
Processing is then passed on to the file
`{\textit{dest}\hspace{0.2em}\textit{suffix}}'.
Surely, the same effect is achieved by
directly specifying the
argument `{\textit{dest}\hspace{0.2em}\textit{suffix}}'
in the first form.
However, that requires to set up a different file
for each child. With the alternative form of the command
all these files can have exactly the same content
which simplifies setting them up and maintaining them.

For example, the following file |draft.tex|
with a compilation flag |\version| as described in \secref{sec:flags}
compiles the main document as a draft:
%
\begin{center}
\begin{tabular}{l}
|\def\version{draft}|\\
|\input{childdoc.def}|\\
|\childdocforward{|\textit{main}|}|
\end{tabular}
\end{center}
%
Likewise, the following files |final|\textit{nn}|.tex|
compile the final version of the child document
|child|\textit{nn}|.tex|:
%
\begin{center}
\begin{tabular}{l}
|\def\version{final}|\\
|\input{childdoc.def}|\\
|\childdocforwardprefix{final}{child}|
\end{tabular}
\end{center}
%

Note that when several versions of a main file and/or of each child file
are to be generated, it may be convenient to set up a |Makefile| or
shell script to automatise the process.

%%%%%%%%%%%%%%%%%%%%%%%%%%%%%%%%%%%%%%%%%%%%%%%%%%%%%%%%%%%%%%%%%%%%%%%%%%%%%%%%
\subsection{Command Line Processing}
\label{sec:commandline}

The effect of redirection files can also be achieved by invoking
the \LaTeX{} compiler with a more elaborate command line.
Most conveniently this should be done as part
of a shell script or a |Makefile|.

When using \textsf{childdoc} in the main file, the following
command lines effectively perform a redirection
(note that depending on the shell being used,
backslashes may have to be doubled: `|\|' $\to$ `|\\|'):
%
\begin{center}
|... -jobname "|\textit{target}|" |\\|"|[\textit{flags}]%
|\input{childdoc.def}\childdocforward[|\textit{main}|]{|\textit{dest}|}"|
\end{center}
%
Here \textit{target} is the name of the output file,
\textit{main} is the name of the main file
and \textit{dest} is the name of the main or child file to be processed
(all filenames without extensions).
The optional argument \textit{main} can be omitted
if \textit{main} matches \textit{dest}.
Optionally, compilation \textit{flags} can be defined via |\def| commands.
This command line makes the \TeX{} engine believe
it is compiling the file \textit{target}
whose content is specified as the latter parameter.
The provided code then forwards the processing to
\textit{main} or \textit{dest} as described in \secref{sec:forward}.

%%%%%%%%%%%%%%%%%%%%%%%%%%%%%%%%%%%%%%%%%%%%%%%%%%%%%%%%%%%%%%%%%%%%%%%%%%%%%%%%
\subsection{Include by Input}
\label{sec:input}

Including child documents by |\include| has some restrictions by design.
Most notably, the content of a child document always occupies
its own set of pages; pages cannot be shared between child documents.
Usually, this behaviour makes perfect sense
because each child document contain an essential part of the document.
However, in some situations it may be desirable to compose
a document from a collection of parts
without having mandatory page breaks between then.
For this case, the package
provides a mechanism to include parts
by |\input| which can also be processed individually.
However, by construction this mechanism
requires manual handling of the content to be output.

%%%%%%%%%%%%%%%%%%%%%%%%%%%%%%%%%%%%%%%%
\DescribeMacro{\ifchilddocmanual}
The main file should be prepared as usual, see \secref{sec:include}.
However, the document body must make a distinction
between processing of an individual part and of the main document, e.g.:
%
\begin{center}
\begin{tabular}{l}
|\ifchilddocmanual|\\
|\input{\childdocname}|\\
|\||else|\\
\textit{document body with }|\input{|\textit{part}|}|\\
|\||fi|
\end{tabular}
\end{center}
%
The conditional |\ifchilddocmanual| is true whenever
a part to be included by |\input| is being compiled,
and the name of the part is stored in |\childdocname|.

%%%%%%%%%%%%%%%%%%%%%%%%%%%%%%%%%%%%%%%%
\DescribeMacro{\childdocby}
Each part to be included by |\input| should start with:
%
\begin{center}
\begin{tabular}{l}
|\input{childdoc.def}|\\
|\childdocby{|\textit{main}|}|\\
\end{tabular}
\end{center}
%
The directive |\childdocby| is similar to |\childdocof|
described in \secref{sec:include},
but the subsequent selection of content must be done manually.
To that end, both |\ifchilddoc| and |\ifchilddocmanual|
will be true upon processing of a part,
and the name of the part is stored in |\childdocname|.
Note that |\jobname| will be set to the filename of the current part
so that each part receives an individual |.aux| file
that does not interfere with the |.aux| file(s) of the main document.
This behaviour can be altered by the alternative form
|\childdocby[*]{|\textit{main}|}| (with a non-empty optional argument)
which uses the |.aux| file of the main document
by setting |\jobname| to \textit{main}.

%%%%%%%%%%%%%%%%%%%%%%%%%%%%%%%%%%%%%%%%%%%%%%%%%%%%%%%%%%%%%%%%%%%%%%%%%%%%%%%%
\subsection{Driver Development}
\label{sec:driver}

The \textsf{childdoc} mechanism can also be use for the development
of definition files such as \LaTeX{} styles or classes.
This case differs from the above setup with multiple parts
included by |\include| in that no |\includeonly| should be invoked.
This can be achieved by starting the include file
(before |\ProvidesPackage|) with:
%
\begin{center}
\begin{tabular}{l}
|\input{childdoc.def}|\\
|\childdocforward{|\textit{main}|}|\\
\end{tabular}
\end{center}
%
or alternatively with:
%
\begin{center}
\begin{tabular}{l}
|\input{childdoc.def}|\\
|\childdocby{|\textit{main}|}|\\
\end{tabular}
\end{center}
%
Both forms have slightly different effects as described above.
The main file is prepared as usual, see \secref{sec:include}.

%%%%%%%%%%%%%%%%%%%%%%%%%%%%%%%%%%%%%%%%%%%%%%%%%%%%%%%%%%%%%%%%%%%%%%%%%%%%%%%%
\subsection{Legacy Detection}
\label{sec:detection}

The directive |\childdocmain| in the main file can detect
whether the complete document or merely a child is to be compiled
even without using the directive |\childdocof|.
This method is deprecated because it is less robust
and there is no compelling reason to use it;
it is merely provided for backward compatibility
and it may be removed in future versions.

If the detection mechanism is to be used,
it is mandatory to correctly specify
the filename of the main file as the argument of |\childdocmain|:
%
\begin{center}
\begin{tabular}{l}
|\input{childdoc.def}|\\
|\childdocmain{|\textit{main}|}|\\
\end{tabular}
\end{center}
%
If |\jobname| does not match the argument \textit{main} of |\childdocmain|,
it is assumed that |\jobname| points to the child file to be compiled.
When using |\childdocmain| with the main file specified as argument,
it suffices to start a child file
with just |\input{|\textit{main}|}|
without loading of the package and using |\childdocof|.
If instead all processing is done
with the appropriate \textsf{childdoc} directives,
the argument of \textit{main} of |\childdocmain| can be empty.

An alternative version of the command line processing described
in \secref{sec:commandline} using the detection mechanism reads:
%
\begin{center}
|... -jobname "|\textit{target}|" "|[\textit{flags}]%
[|\def\jobname{|\textit{dest}|}|]|\input{|\textit{main}|}"|
\end{center}

%%%%%%%%%%%%%%%%%%%%%%%%%%%%%%%%%%%%%%%%%%%%%%%%%%%%%%%%%%%%%%%%%%%%%%%%%%%%%%%%
\subsection{Manual Code}
\label{sec:manual}

In case one cannot be certain whether the definitions file |childdoc.def|
is installed on the target \TeX{} distribution
and one prefers not to ship it,
it is conceivable to paste a few relevant commands into the sources.

To that end, drop all statements |\input{childdoc.def}|
and perform the replacements as outlined below.
Instead of |\childdocmain{|\textit{main}|}| add the following code
to the top of the main file:
%
\begin{center}
\begin{tabular}{l}
|\||ifdefined\childdocname\endinput\||fi\newif\ifchilddoc|\\
|\edef\childdocname{\scantokens\expandafter{\jobname\noexpand}}|\\
|\def\childdocmain{|\textit{main}|}\||ifx\childdocmain\childdocname\||else|\\
|\childdoctrue\includeonly{\childdocname}\let\jobname\childdocmain\||fi|\\
\end{tabular}
\end{center}
%
Instead of |\childdocof{|\textit{main}|}| just include the main file
at the top of each child file:
%
\begin{center}
|\input{|\textit{main}|}|
\end{center}
%
A simple redirection |\childdocforward{|\textit{dest}|}| is achieved by:
%
\begin{center}
|\def\jobname{|\textit{dest}|}\input{\jobname}|
\end{center}
%
The redirection with prefix
|\childdocforwardprefix[|\textit{prefix}|]{|\textit{dest}|}|
is accomplished by:
%
\begin{center}
\begin{tabular}{l}
|{\edef\jobname{\scantokens\expandafter{\jobname\noexpand}}|\\
|\def\redirectjob |\textit{prefix}|#1~~~{\gdef\jobname{|\textit{dest}|#1}}|\\
|\expandafter\redirectjob\jobname~~~}\input{\jobname}|
\end{tabular}
\end{center}

In an alternative approach,
child documents can be compiled by a specific command line
without additional code or specific definitions:
%
\begin{center}
|... -jobname "|\textit{target}|" "|[\textit{flags}]%
|\includeonly{|\textit{dest}|}\input{|\textit{main}|}"|
\end{center}
%

%%%%%%%%%%%%%%%%%%%%%%%%%%%%%%%%%%%%%%%%%%%%%%%%%%%%%%%%%%%%%%%%%%%%%%%%%%%%%%%%
%%%%%%%%%%%%%%%%%%%%%%%%%%%%%%%%%%%%%%%%%%%%%%%%%%%%%%%%%%%%%%%%%%%%%%%%%%%%%%%%
\section{Information}

%%%%%%%%%%%%%%%%%%%%%%%%%%%%%%%%%%%%%%%%%%%%%%%%%%%%%%%%%%%%%%%%%%%%%%%%%%%%%%%%
\subsection{Copyright}

Copyright \copyright{} 2017--2018 Niklas Beisert

This work may be distributed and/or modified under the
conditions of the \LaTeX{} Project Public License, either version 1.3
of this license or (at your option) any later version.
The latest version of this license is in
  \url{http://www.latex-project.org/lppl.txt}
and version 1.3 or later is part of all distributions of \LaTeX{}
version 2005/12/01 or later.

This work has the LPPL maintenance status `maintained'.

The Current Maintainer of this work is Niklas Beisert.

This work consists of the files |README.txt|, |childdoc.ins| and |childdoc.dtx|
as well as the derived files |childdoc.def|, |cdocsamp.tex|
with |cdocsch1.tex|, |cdocsch2.tex|, |cdocspt3.tex|, |cdocspt4.tex|,
|cdocsdrf.tex|, |cdocsfn1.tex|, |cdocsfn2.tex|
as well as |childdoc.pdf|.

%%%%%%%%%%%%%%%%%%%%%%%%%%%%%%%%%%%%%%%%%%%%%%%%%%%%%%%%%%%%%%%%%%%%%%%%%%%%%%%%
\subsection{Files and Installation}

The package consists of the files:
%
\begin{center}
\begin{tabular}{ll}
    |README.txt|   & readme file \\
    |childdoc.ins| & installation file \\
    |childdoc.dtx| & source file \\
    |childdoc.def| & definition file \\
    |cdocsamp.tex| & sample main file \\
    |cdocsch1.tex| & sample include file \\
    |cdocsch2.tex| & sample include file \\
    |cdocspt3.tex| & sample part file \\
    |cdocspt4.tex| & sample part file \\
    |cdocsdrf.tex| & sample redirection file \\
    |cdocsfn1.tex| & sample redirection file \\
    |cdocsfn2.tex| & sample redirection file \\
    |childdoc.pdf| & manual
\end{tabular}
\end{center}
%
The distribution consists of the files
|README.txt|, |childdoc.ins| and |childdoc.dtx|.
%
\begin{itemize}
\item
Run (pdf)\LaTeX{} on |childdoc.dtx|
to compile the manual |childdoc.pdf| (this file).
\item
Run \LaTeX{} on |childdoc.ins| to create the definitions file |childdoc.def|
and the sample |cdocsamp.tex| with include files
|cdocsch1.tex|, |cdocsch2.tex|, |cdocspt3.tex|, |cdocspt4.tex|,
|cdocsdrf.tex|, |cdocsfn1.tex|, |cdocsfn2.tex|.
Then copy the file |childdoc.def| to an appropriate directory of your \LaTeX{}
distribution, e.g.\ \textit{texmf-root}|/tex/latex/childdoc|.
\end{itemize}

%%%%%%%%%%%%%%%%%%%%%%%%%%%%%%%%%%%%%%%%%%%%%%%%%%%%%%%%%%%%%%%%%%%%%%%%%%%%%%%%
\subsection{Related CTAN Packages}

There are several other packages which offer a similar functionality:
%
\begin{itemize}
\item
The packages
\href{http://ctan.org/pkg/docmute}{\textsf{docmute}},
\href{http://ctan.org/pkg/includex}{\textsf{includex}} and
\href{http://ctan.org/pkg/standalone}{\textsf{standalone}}
provide commands to include only the document body of
a child file thus allowing both files to be compiled individually.
\item
The packages \href{http://ctan.org/pkg/subdocs}{\textsf{subdocs}}
and \href{http://ctan.org/pkg/subfiles}{\textsf{subfiles}}
provide structures in which the main and child documents can be
encapsulated and allowing them to be compiled individually.
The inclusion mechanism is different from the conventional |\include|.
\item
The package \href{http://ctan.org/pkg/combine}{\textsf{combine}}
is an elaborate solution to combine several documents into one.
\end{itemize}
%
See also the CTAN topic \href{http://ctan.org/topic/subdocs}{\textsf{subdocs}}
for further related packages.
The present package differs from the above solutions in that
a document structure constructed with the conventional |\include| mechanism
just needs two extra commands at the top of every file
such that all constituent files can be compiled individually.

%%%%%%%%%%%%%%%%%%%%%%%%%%%%%%%%%%%%%%%%%%%%%%%%%%%%%%%%%%%%%%%%%%%%%%%%%%%%%%%%
%\subsection{Feature Suggestions}
%
%The following is a list of features which may be useful for future
%versions of this package:
%%
%\begin{itemize}
%\item
%\ldots
%\end{itemize}

%%%%%%%%%%%%%%%%%%%%%%%%%%%%%%%%%%%%%%%%%%%%%%%%%%%%%%%%%%%%%%%%%%%%%%%%%%%%%%%%
\subsection{Revision History}

%%%%%%%%%%%%%%%%%%%%%%%%%%%%%%%%%%%%%%%%
\paragraph{v2.0:} 2018/12/30

\begin{itemize}
\item
immediate forward processing
\item
added |\childdocby| mechanism
\item
manual restructured
\end{itemize}

%%%%%%%%%%%%%%%%%%%%%%%%%%%%%%%%%%%%%%%%
\paragraph{v1.6:} 2018/01/17

\begin{itemize}
\item
application for development of include files
\item
corrections to manual
\end{itemize}

%%%%%%%%%%%%%%%%%%%%%%%%%%%%%%%%%%%%%%%%
\paragraph{v1.5:} 2017/05/21

\begin{itemize}
\item
more complete structuring introduced
\item
|\childdocof| introduced
\item
|\childdoc| renamed to |\childdocmain|
\item
|\childredirect| renamed to |\childdocforward| and |\childdocforwardprefix|
and functionality expanded
\end{itemize}

%%%%%%%%%%%%%%%%%%%%%%%%%%%%%%%%%%%%%%%%
\paragraph{v1.0:} 2017/04/27

\begin{itemize}
\item
manual and install package
\item
first version published on CTAN
\end{itemize}

%%%%%%%%%%%%%%%%%%%%%%%%%%%%%%%%%%%%%%%%
\paragraph{v0.6:} 2017/04/26

\begin{itemize}
\item
redirection mechanism added
\end{itemize}

%%%%%%%%%%%%%%%%%%%%%%%%%%%%%%%%%%%%%%%%
\paragraph{v0.5:} 2017/04/26

\begin{itemize}
\item
functionality in definition file
\end{itemize}


%%%%%%%%%%%%%%%%%%%%%%%%%%%%%%%%%%%%%%%%%%%%%%%%%%%%%%%%%%%%%%%%%%%%%%%%%%%%%%%%
%%%%%%%%%%%%%%%%%%%%%%%%%%%%%%%%%%%%%%%%%%%%%%%%%%%%%%%%%%%%%%%%%%%%%%%%%%%%%%%%
%%%%%%%%%%%%%%%%%%%%%%%%%%%%%%%%%%%%%%%%%%%%%%%%%%%%%%%%%%%%%%%%%%%%%%%%%%%%%%%%
\appendix

\settowidth\MacroIndent{\rmfamily\scriptsize 000\ }

 \DocInput{childdoc.dtx}

\end{document}
%</driver>
% \fi
%
% %%%%%%%%%%%%%%%%%%%%%%%%%%%%%%%%%%%%%%%%%%%%%%%%%%%%%%%%%%%%%%%%%%%%%%%%%%%%%%
% %%%%%%%%%%%%%%%%%%%%%%%%%%%%%%%%%%%%%%%%%%%%%%%%%%%%%%%%%%%%%%%%%%%%%%%%%%%%%%
% \section{Sample}
%\iffalse
%<*samplemain>
%\fi
%
% The following presents a sample document
% with two chapters, two parts, a title page,
% a compile flag as well as three forwarding files to set the flag.
% It consists of eight |.tex| files:
% \begin{center}
% \begin{tabular}{ll}
% |cdocsamp.tex|&main file\\
% |cdocsch1.tex|&include file for chapter 1\\
% |cdocsch2.tex|&include file for chapter 2\\
% |cdocspt3.tex|&include file for part 3\\
% |cdocspt4.tex|&include file for part 4\\
% |cdocsdrf.tex|&forwarding file for main file in draft mode\\
% |cdocsfi1.tex|&forwarding file for final version of chapter 1\\
% |cdocsfi2.tex|&forwarding file for final version of chapter 2\\
% \end{tabular}
% \end{center}
% Each of the eight files can be compiled directly by the \LaTeX{} compiler.
%
% %%%%%%%%%%%%%%%%%%%%%%%%%%%%%%%%%%%%%%
% \paragraph{Main File.}
%
% The main file is called |cdocsamp.tex|.
%
% Load the \textsf{childdoc} definitions and
% declare the filename for the main document:
%    \begin{macrocode}
\input{childdoc.def}
\childdocmain{}
%    \end{macrocode}

% Optional override for |\version| flag:
%    \begin{macrocode}
%%\ifchilddoc\else\providecommand{\version}{draft}\fi
%    \end{macrocode}

% Define the default values for the |\version| flag
% (|final| for the main file and |draft| for childs):
%    \begin{macrocode}
\ifchilddoc
\providecommand{\version}{draft}
\else
\providecommand{\version}{final}
\fi
%    \end{macrocode}

% Load the standard document class:
%    \begin{macrocode}
\documentclass[12pt]{article}
%    \end{macrocode}

% Start the document body:
%    \begin{macrocode}
\begin{document}
%    \end{macrocode}

% Declare a title page.
% Print title, part of document being processed and version flag:
%    \begin{macrocode}
\addtocounter{page}{-1}
\begin{center}
{\LARGE\bfseries{}childdoc example\par}
\vspace{1cm}
\ifchilddoc
\ifchilddocmanual part\else chapter\fi:
`\childdocname' of `\childdocjob'\par
\else
main document: `\childdocjob'\par
\fi
version: \version\par
\end{center}
\newpage
%    \end{macrocode}

% Manually include selected file,
% otherwise process as usual:
%    \begin{macrocode}
\ifchilddocmanual
\section*{part `\childdocname'}
\input{\childdocname}
\else
%    \end{macrocode}

% Include the two chapters:
%    \begin{macrocode}
\include{cdocsch1}
\include{cdocsch2}
%    \end{macrocode}

% Include the two parts unless only chapters should be displayed:
%    \begin{macrocode}
\ifchilddoc\else
\section{part three}
\input{cdocspt3}
\section{part four}
\input{cdocspt4}
\fi
%    \end{macrocode}

% Process as usual until here:
%    \begin{macrocode}
\fi
%    \end{macrocode}

% End of document body:
%    \begin{macrocode}
\end{document}
%    \end{macrocode}
%\iffalse
%</samplemain>
%\fi
%
% %%%%%%%%%%%%%%%%%%%%%%%%%%%%%%%%%%%%%%
% \paragraph{Chapter Include Files.}
%
% The include files are called |cdocsch1.tex| and |cdocsch2.tex|.
%
%\iffalse
%<*samplechap1|samplechap2>
%\fi

% Optional override for |\version| flag:
%    \begin{macrocode}
%%\providecommand{\version}{final}
%    \end{macrocode}

% Include the main document:
%    \begin{macrocode}
\input{childdoc.def}
\childdocof{cdocsamp}
%    \end{macrocode}

%\iffalse
%</samplechap1|samplechap2>
%\fi
%
%\iffalse
%<*samplechap1>
%\fi
% Some text for chapter 1:
%    \begin{macrocode}
\section{one}
some text in chapter one
%    \end{macrocode}

%\iffalse
%</samplechap1>
%\fi
% Some text for chapter 2:
%\iffalse
%<*samplechap2>
%\fi
%    \begin{macrocode}
\section{two}
more text in chapter two
%    \end{macrocode}

%\iffalse
%</samplechap2>
%\fi
%
% %%%%%%%%%%%%%%%%%%%%%%%%%%%%%%%%%%%%%%
% \paragraph{Part Include Files.}
%
% The include files are called |cdocspt3.tex| and |cdocspt4.tex|.
%
%\iffalse
%<*samplepart3|samplepart4>
%\fi

% Optional override for |\version| flag:
%    \begin{macrocode}
%%\providecommand{\version}{final}
%    \end{macrocode}

% Include the main document:
%    \begin{macrocode}
\input{childdoc.def}
\childdocby{cdocsamp}
%    \end{macrocode}

%\iffalse
%</samplepart3|samplepart4>
%\fi
%
%\iffalse
%<*samplepart3>
%\fi
% Some text for part 3:
%    \begin{macrocode}
some text in part three
%    \end{macrocode}

%\iffalse
%</samplepart3>
%\fi
% Some text for part 4:
%\iffalse
%<*samplepart4>
%\fi
%    \begin{macrocode}
more text in part four
%    \end{macrocode}

%\iffalse
%</samplepart4>
%\fi
%
% %%%%%%%%%%%%%%%%%%%%%%%%%%%%%%%%%%%%%%
% \paragraph{Forwarding for a Complete Draft.}
%
% The following forwarding file |cdocsdrf.tex|
% compiles the main document in draft mode:
%\iffalse
%<*sampledraft>
%\fi
%    \begin{macrocode}
\def\version{draft}
\input{childdoc.def}
\childdocforward{cdocsamp}
%    \end{macrocode}

%\iffalse
%</sampledraft>
%\fi
%
% %%%%%%%%%%%%%%%%%%%%%%%%%%%%%%%%%%%%%%
% \paragraph{Forwarding for Final Version of the Chapters.}
%
% The following forwarding files |cdocsfn1.tex| and |cdocsfn2.tex|
% (with identical content)
% compile the final versions of the child documents
% |cdocsch1.tex| and |cdocsch2.tex|, respectively:
%\iffalse
%<*samplefinal>
%\fi
%    \begin{macrocode}
\def\version{final}
\input{childdoc.def}
\childdocforwardprefix[cdocsamp]{cdocsfn}{cdocsch}
%    \end{macrocode}

%\iffalse
%</samplefinal>
%\fi
%
% %%%%%%%%%%%%%%%%%%%%%%%%%%%%%%%%%%%%%%
% \paragraph{Command Line Processing.}
%
% The following three command lines generate the output files
% |cdocscld|, |cdocscl1| and |cdocscl2|
% which should be identical to
% |cdocsdrf|, |cdocsch1| and |cdocsfn2|, respectively:
% \begin{center}
% \begin{tabular}{l}
% |latex -jobname cdocscld \|\\
% |  "\def\version{draft}\input{childdoc.def}\childdocforward{cdocsamp}"|\\
% |latex -jobname cdocscl1 \|\\
% |  "\input{childdoc.def}\childdocforward[cdocsamp]{cdocsch1}"|\\
% |latex -jobname cdocscl2 \|\\
% |  "\def\version{final}\input{childdoc.def}\childdocforward{cdocsch2}"|
% \end{tabular}
% \end{center}
% Note that the trailing backslash on each first line
% merely continues the input to the second line
% (for convenient cut ant paste).
% Furthermore, the command |latex| can be replaced by any
% of its alternative versions such as |pdflatex|.
%
% %%%%%%%%%%%%%%%%%%%%%%%%%%%%%%%%%%%%%%%%%%%%%%%%%%%%%%%%%%%%%%%%%%%%%%%%%%%%%%
% %%%%%%%%%%%%%%%%%%%%%%%%%%%%%%%%%%%%%%%%%%%%%%%%%%%%%%%%%%%%%%%%%%%%%%%%%%%%%%
% \section{Implementation}
%\iffalse
%<*package>
%\fi
%
% This section describes the definitions file |childdoc.def|.

% The definitions cannot be loaded using |\usepackage| or |\RequirePackage|
% which has a mechanism to prevent loading a style file more than once.
% When loading the definitions by means of |\input|
% multiple instances have to be prevented manually:
%\iffalse
%This code needs to be before the `\ProvidesFile' directive
%which is defined at the beginning of this file.
%Therefore it is also placed there and commented out here.
%</package>
%<*discard>
%\fi
%    \begin{macrocode}
\ifdefined\childdocmain\endinput\fi
%    \end{macrocode}
%\iffalse
%</discard>
%<*package>
%\fi
%
% \macro{\ifchilddoc}
% \macro{\ifchilddocmanual}
% The conditional |\ifchilddoc| tells whether a
% child (true) or main (false) document is being compiled.
% The conditional |\ifchilddocmanual| tells whether
% the |\includeonly| mechanism is used (false) or
% the selection of child files must be performed manually (true).
% The definitions initialise to false:
%    \begin{macrocode}
\newif\ifchilddoc
\newif\ifchilddocmanual
%    \end{macrocode}

% \macro{\childdocname}
% \macro{\childdocjob}
% The macro |\childdocname| stores the name of the main document
% to be compiled. The macro |\childdocjob| stores the name of
% the document on which the \LaTeX{} compiler was originally invoked.
% The content of |\jobname| cannot be compared
% to filenames specified in the source due to different catcodes.
% The following code rescans |\jobname|, stores the result
% in |\childdocname| and saves a copy in |\childdocjob|:
%    \begin{macrocode}
\edef\childdocname{\scantokens\expandafter{\jobname\noexpand}}
\let\childdocjob\childdocname
%    \end{macrocode}

% \macro{\childdocdisable}
% The macro |\childdocdisable| prevents the main file
% from being processed more than once.
% At this stage, the main document command |\childdocmain|
% is assumed to be called once again where it should do nothing.
% Any subsequent call to it should prevent
% a secondary processing of the main document
% It overwrites the forwarding commands
% |\childdocof| and |\childdocforward|
% with empty macros to prevent further inclusions of the main document:
%    \begin{macrocode}
\newcommand{\childdocdisable}
{
  \renewcommand{\childdocmain}[1]{\renewcommand{\childdocmain}[1]{\endinput}}
  \renewcommand{\childdocof}[1]{}
  \renewcommand{\childdocby}[2][]{}
  \renewcommand{\childdocforward}[2][]{}
  \renewcommand{\childdocdisable}{}
}
%    \end{macrocode}

% \macro{\childdocmain}
% The macro |\childdocmain| is to be called at the top of the main file
% with nothing or the main filename (without extension) as argument.
% First, it breaks loops.
% If the argument is not empty and does not match |\childdocname|
% (which is set by the first inclusion of |childdoc.def|),
% |\ifchilddoc| is set to true, |\includeonly| is applied to the child file
% and |\jobname| is set to the main file
% (for proper handling of |.aux| files):
%    \begin{macrocode}
\newcommand{\childdocmain}[1]
{
  \childdocdisable\childdocmain{}
  \if?#1?\else
    \begingroup
      \def\childdoctmp{#1}
      \ifx\childdoctmp\childdocname
        \def\childdoctmp{}
      \else
        \def\childdoctmp
        {
          \childdoctrue
          \includeonly{\childdocname}
          \def\childdocjob{#1}
          \def\jobname{#1}
        }
      \fi
      \expandafter
    \endgroup
    \childdoctmp
  \fi
}
%    \end{macrocode}

% \macro{\childdocof}
% The command |\childdocof| redirects
% compilation to the main file |#1|.
%    \begin{macrocode}
\newcommand{\childdocof}[1]
{
  \childdocdisable
  \childdoctrue
  \includeonly{\childdocname}
  \def\jobname{#1}
  \def\childdocjob{#1}
  \input{#1}
}
%    \end{macrocode}

% \macro{\childdocby}
% The command |\childdocby| ....
%    \begin{macrocode}
\newcommand{\childdocby}[2][]
{
  \childdocdisable
  \childdoctrue
  \childdocmanualtrue
  \if?#1?\else
    \def\jobname{#2}
  \fi
  \def\childdocjob{#2}
  \input{#2}
  \endinput
}
%    \end{macrocode}

% \macro{\childdocforward}
% The command |\childdocforward| redirects
% compilation to the main file or
% (if the optional argument is given) a child file.
% Parameters are set as if the main file
% or a child file starting with |\childdocof| was compiled.
% Then compilation is handed over to the main file:
%    \begin{macrocode}
\newcommand{\childdocforward}[2][]
{
  \begingroup
    \if?#1?
      \def\childdoctmp
      {
        \def\childdocname{#2}
        \def\childdocjob{#2}
        \def\jobname{#2}
        \input{#2}
        \endinput
      }
    \else
      \def\childdoctmp
      {
        \childdocdisable
        \def\childdocname{#2}
        \childdoctrue
        \includeonly{#2}
        \def\childdocjob{#1}
        \def\jobname{#1}
        \input{#1}
        \endinput
      }
    \fi
    \expandafter
  \endgroup
  \childdoctmp
}
%    \end{macrocode}

% \macro{\childdocforwardprefix}
% The command |\childdocforwardprefix| redirects
% compilation to the main or a child file by means of a pattern.
% The prefix |#1| in the current filename is replaced by |#2|
% and the suffix of the current filename is kept
% (it is assumed that the filename does not contain the substring `|~~~|'
% which is used as a delimiter).
% Compilation is handed over to the new file by |\childdocforward|:
%    \begin{macrocode}
\newcommand{\childdocforwardprefix}[3][]
{
  \begingroup
    \def\childdocextract #2##1~~~{\def\childdoctmp{\childdocforward[#1]{#3##1}}}
    \expandafter\childdocextract\childdocname~~~
    \expandafter
  \endgroup
  \childdoctmp
}
%    \end{macrocode}

% \macro{\childdoc}
% The deprecated macro |\childdoc| is a legacy version of |\childdocmain|:
%    \begin{macrocode}
\newcommand{\childdoc}{\childdocmain}
%    \end{macrocode}

% \macro{\childdocredirect}
% The deprecated macro |\childdocredirect| is a legacy version
% of |\childdocforward| and |\childdocforwardprefix|:
%    \begin{macrocode}
\newcommand{\childdocredirect}[2][]
{
  \begingroup
    \if?#1?
      \def\childdoctmp{\childdocforward{#2}}
    \else
      \def\childdoctmp{\childdocforwardprefix{#1}{#2}}
    \fi
    \expandafter
  \endgroup
  \childdoctmp
}
%    \end{macrocode}

%\iffalse
%</package>
%\fi
%
\endinput
|\\
|\childdocmain{|\textit{main}|}|\\
\end{tabular}
\end{center}
%
If |\jobname| does not match the argument \textit{main} of |\childdocmain|,
it is assumed that |\jobname| points to the child file to be compiled.
When using |\childdocmain| with the main file specified as argument,
it suffices to start a child file
with just |\input{|\textit{main}|}|
without loading of the package and using |\childdocof|.
If instead all processing is done
with the appropriate \textsf{childdoc} directives,
the argument of \textit{main} of |\childdocmain| can be empty.

An alternative version of the command line processing described
in \secref{sec:commandline} using the detection mechanism reads:
%
\begin{center}
|... -jobname "|\textit{target}|" "|[\textit{flags}]%
[|\def\jobname{|\textit{dest}|}|]|\input{|\textit{main}|}"|
\end{center}

%%%%%%%%%%%%%%%%%%%%%%%%%%%%%%%%%%%%%%%%%%%%%%%%%%%%%%%%%%%%%%%%%%%%%%%%%%%%%%%%
\subsection{Manual Code}
\label{sec:manual}

In case one cannot be certain whether the definitions file |childdoc.def|
is installed on the target \TeX{} distribution
and one prefers not to ship it,
it is conceivable to paste a few relevant commands into the sources.

To that end, drop all statements |% \iffalse
%
% childdoc.dtx Copyright (C) 2017-2018 Niklas Beisert
%
% This work may be distributed and/or modified under the
% conditions of the LaTeX Project Public License, either version 1.3
% of this license or (at your option) any later version.
% The latest version of this license is in
%   http://www.latex-project.org/lppl.txt
% and version 1.3 or later is part of all distributions of LaTeX
% version 2005/12/01 or later.
%
% This work has the LPPL maintenance status `maintained'.
%
% The Current Maintainer of this work is Niklas Beisert.
%
% This work consists of the files childdoc.dtx and childdoc.ins
% and the derived files childdoc.def and cdocsamp.tex with
% cdocsch1.tex, cdocsch2.tex, cdocsdrf.tex, cdocsfn1.tex, cdocsfn2.tex.
%
%<package>\ifdefined\childdocmain\endinput\fi
%<package>\ProvidesFile{childdoc.def}[2018/12/30 v2.0 child document driver]
%<samplemain>\ProvidesFile{cdocsamp.tex}[2018/12/30 v2.0 sample for childdoc]
%<*driver>
%\ProvidesFile{childdoc.drv}[2018/12/30 v2.0 childdoc reference manual file]
\PassOptionsToClass{10pt,a4paper}{article}
\documentclass{ltxdoc}

\usepackage[margin=35mm]{geometry}
\usepackage{hyperref}
\usepackage{hyperxmp}
\usepackage[usenames]{color}

\hypersetup{colorlinks=true}
\hypersetup{pdfstartview=FitH}
\hypersetup{pdfpagemode=UseNone}
\hypersetup{pdfsource={}}
\hypersetup{pdflang={en-UK}}
\hypersetup{pdfcopyright={Copyright 2017-2018 Niklas Beisert.
  This work may be distributed and/or modified under the
  conditions of the LaTeX Project Public License, either version 1.3
  of this license or (at your option) any later version.}}
\hypersetup{pdflicenseurl={http://www.latex-project.org/lppl.txt}}
\hypersetup{pdfcontactaddress={ETH Zurich, ITP, HIT K,
  Wolfgang-Pauli-Strasse 27}}
\hypersetup{pdfcontactpostcode={8093}}
\hypersetup{pdfcontactcity={Zurich}}
\hypersetup{pdfcontactcountry={Switzerland}}
\hypersetup{pdfcontactemail={nbeisert@itp.phys.ethz.ch}}
\hypersetup{pdfcontacturl={http://people.phys.ethz.ch/\xmptilde nbeisert/}}

\newcommand{\secref}[1]{\hyperref[#1]{section \ref*{#1}}}

\parskip1ex
\parindent0pt
\let\olditemize\itemize
\def\itemize{\olditemize\parskip0pt}

\begin{document}

\title{The \textsf{childdoc} Package}
\hypersetup{pdftitle={The childdoc Package}}
\author{Niklas Beisert\\[2ex]
  Institut f\"ur Theoretische Physik\\
  Eidgen\"ossische Technische Hochschule Z\"urich\\
  Wolfgang-Pauli-Strasse 27, 8093 Z\"urich, Switzerland\\[1ex]
  \href{mailto:nbeisert@itp.phys.ethz.ch}
  {\texttt{nbeisert@itp.phys.ethz.ch}}}
\hypersetup{pdfauthor={Niklas Beisert}}
\hypersetup{pdfsubject={Manual for the LaTeX2e Package childdoc}}
\date{30 December 2018, \textsf{v2.0}}
\maketitle

\begin{abstract}\noindent
\textsf{childdoc} is a \LaTeXe{} package
that enables the direct compilation
of document sections included by |\include|
to individual files.
\end{abstract}

\begingroup
\parskip0ex
\tableofcontents
\endgroup

%%%%%%%%%%%%%%%%%%%%%%%%%%%%%%%%%%%%%%%%%%%%%%%%%%%%%%%%%%%%%%%%%%%%%%%%%%%%%%%%
%%%%%%%%%%%%%%%%%%%%%%%%%%%%%%%%%%%%%%%%%%%%%%%%%%%%%%%%%%%%%%%%%%%%%%%%%%%%%%%%
\section{Introduction}

\LaTeX{} provides a mechanism to structure a large document (such as a book)
into a main file and several child files (containing the chapters)
using the |\include| command.
This mechanism is beneficial for documents
which span hundreds of pages in order to
make the source file(s) more manageable.
Moreover, compilation can be restricted to
selected child files by means of the |\includeonly| command.
The latter feature can be used to reduce the compilation time while editing
(this was significantly more useful in the earlier days of \LaTeX{})
or to generate a smaller document which is easier to navigate.
Another application of |\includeonly| is to generate
documents consisting of selected parts of the complete document.

However, there are a few drawbacks of the plain |\include| mechanism:
\begin{itemize}
\item
The child files cannot be compiled on their own,
they can only be compiled via the main file.
A naive editing environment
(such as a text editor with an option
to have the current file processed by \LaTeX)
may require one to switch to the main file before compiling;
attempting to compile the child file produces errors.
\item
The main file must be modified (each time)
to adjust the |\includeonly| command
to the present needs. This easily leaves the main file in a messy state.
\item
The generated document will always carry the filename
of the main document. This is inconvenient if
several child files are to be compiled and
to be kept for distribution.
\end{itemize}

The present package provides a simple interface
to make child files individually compilable by \LaTeX{}.
Compiling a child file then has the same effect as compiling
the main file with an |\includeonly| command
to select the appropriate child.
Moreover the generated document will carry the name of the child
rather than the main file.
This resolves all three above issues.

This feature is meant to make the editing of books,
thesis documents and lecture notes somewhat more convenient.
However, the package can also be used efficiently for
composing a series of documents (such as exercise sheets)
which are typically distributed individually.
It then assists the author in generating the individual documents
(potentially in different versions)
as well as a document containing the collected series.
Another application is in developing style files
or other kinds of included material
where compilation of the style file could redirect
to a sample or test file.

%%%%%%%%%%%%%%%%%%%%%%%%%%%%%%%%%%%%%%%%%%%%%%%%%%%%%%%%%%%%%%%%%%%%%%%%%%%%%%%%
%%%%%%%%%%%%%%%%%%%%%%%%%%%%%%%%%%%%%%%%%%%%%%%%%%%%%%%%%%%%%%%%%%%%%%%%%%%%%%%%
\section{Usage}

First of all, the package \textsf{childdoc} is \emph{not} a standard
\LaTeXe{} |.sty| style file! Therefore it needs to be invoked in
a non-standard way.

%%%%%%%%%%%%%%%%%%%%%%%%%%%%%%%%%%%%%%%%%%%%%%%%%%%%%%%%%%%%%%%%%%%%%%%%%%%%%%%%
\subsection{Included Files}
\label{sec:include}

%%%%%%%%%%%%%%%%%%%%%%%%%%%%%%%%%%%%%%%%
\DescribeMacro{\childdocmain}
To use the package, add the commands
\begin{center}
\begin{tabular}{l}
|\input{childdoc.def}|\\
|\childdocmain{}|\\
\end{tabular}
\end{center}
at the very top of the main \LaTeX{} file,
in particular \emph{before} the |\documentclass| statement!
The argument of |\childdocmain| should be left empty
(but it must be present).

%%%%%%%%%%%%%%%%%%%%%%%%%%%%%%%%%%%%%%%%
\DescribeMacro{\childdocof}
Furthermore, add the commands
\begin{center}
\begin{tabular}{l}
|\input{childdoc.def}|\\
|\childdocof{|\textit{main}|}|\\
\end{tabular}
\end{center}
at the top of every child file \textit{child}
which is included by |\include{|\textit{child}|}|
from within the main file
(or at least for those files to be compiled individually).
The argument \textit{main} must be the filename of the main file.

There are a couple of
considerations in setting up the main and child documents:

%%%%%%%%%%%%%%%%%%%%%%%%%%%%%%%%%%%%%%%%
\paragraph{Restrictions.}

Please note the following restrictions:
\begin{itemize}
\item
|\childdocmain| must be called with one argument \textit{main}
to ensure compatibility with earlier version of the package.
It must either be empty (|\childdocmain{}|)
or precisely match the filename of the main file in which it is specified.
See \secref{sec:detection} for further information.
\item
The filename \textit{main} must be specified without the |.tex| extension.
\item
The filename \textit{main} is case sensitive
(even in case-insensitive file systems)
due to internal string comparison.
\item
The argument \textit{main} should be fully expanded, it cannot be a macro.
\item
Subdirectories and special characters should be avoided in filenames.
\item
The command |\childdocmain{|\textit{main}|}| must be followed by a whitespace.
It should not be followed immediately by another command
or by a comment mark `|%|'.
This is because the \TeX{} parser reads the token immediately following
the argument of |\childdocmain| and puts it
at the beginning of every child section;
however, a white\-space is ignored.
\end{itemize}

%%%%%%%%%%%%%%%%%%%%%%%%%%%%%%%%%%%%%%%%
\paragraph{Content of Main File.}

It is advisable to place all content in the child files included by |\include|.
Any output contained in the main file will appear in all child documents
unless suppressed manually;
it cannot be suppressed automatically by the |\includeonly| directive
and thus should normally be avoided.
A method to include some content in the main file
by means of conditional processing is described in \secref{sec:conditional}.

%%%%%%%%%%%%%%%%%%%%%%%%%%%%%%%%%%%%%%%%
\paragraph{Page Numbering.}

When only a part of the document is compiled,
the appropriate numbering of pages
(as well as other status parameters)
is determined from the |.aux| files.
The latter contain information from previous passes.
However this information needs to propagate through
all intermediate child documents.
Therefore the page numbering in child documents may well
be inconsistent until the complete document is compiled at least once.

A useful (if unconventional) way to always ensure a consistent
page numbering is to restart the numbering in each child document
and denote the pages by `\textit{child}|.|\textit{page}'
where \textit{child} represents the chapter/section number of the child file.
This can be achieved by the command
|\numberwithin{page}{|\textit{child}|}|
of the \textsf{amsmath} package
where \textit{child} can be |chapter| or |section|
depending on the chosen structuring.
Alternatively, one can modify the macro |\thepage| appropriately
and reset the counter |page| at the start of each child file.

%%%%%%%%%%%%%%%%%%%%%%%%%%%%%%%%%%%%%%%%%%%%%%%%%%%%%%%%%%%%%%%%%%%%%%%%%%%%%%%%
\subsection{Conditional Processing}
\label{sec:conditional}

The package provides a mechanism to compile different versions
of a document. To customise the versions further some conditional processing
can come in handy to distinguish which version is being compiled.
The package provides two macros to describe the compilation context:

%%%%%%%%%%%%%%%%%%%%%%%%%%%%%%%%%%%%%%%%
\DescribeMacro{\ifchilddoc}
The conditional |\ifchilddoc| distinguishes between the compilation of
child documents and the main document:
%
\begin{center}
|\ifchilddoc |\textit{child-code}| |[|\||else |\textit{main-code}]| \||fi|
\end{center}

%%%%%%%%%%%%%%%%%%%%%%%%%%%%%%%%%%%%%%%%
\DescribeMacro{\childdocname}
\DescribeMacro{\childdocjob}
The macro |\childdocname| contains the filename (without extension)
of the main or child file being processed.
Note that |\childdocjob| will always contain the name of the main file.

%%%%%%%%%%%%%%%%%%%%%%%%%%%%%%%%%%%%%%%%
\paragraph{Title Page.}

Conditional processing can be used to include a title or banner page
in the main document when proper precautions are taken.
Importantly, the code in the main file should ensure that the page counter
(as well as other status parameters which are stored in the |.aux| files)
takes the same value after the conditional processing.
Otherwise the page numbers may take divergent values
depending on which part is compiled.

For example, a title page could be declared by:
%
\begin{center}
\begin{tabular}{l}
|\ifchilddoc\||else|\\
|\addtocounter{page}{-1}|\\
\textit{code for title page}\\
|\newpage|\\
|\||fi|
\end{tabular}
\end{center}
%
A banner page for the child documents can be generated by:
%
\begin{center}
\begin{tabular}{l}
|\ifchilddoc|\\
|\addtocounter{page}{-1}|\\
\textit{code for banner page}\\
|\newpage|\\
|\||fi|
\end{tabular}
\end{center}
%
Here one could write a message such as:
\begin{center}
|This is the part \childdocname{} of \childdocjob{}.|
\end{center}

%%%%%%%%%%%%%%%%%%%%%%%%%%%%%%%%%%%%%%%%%%%%%%%%%%%%%%%%%%%%%%%%%%%%%%%%%%%%%%%%
\subsection{Flags}
\label{sec:flags}

The package makes it easy to generate different versions
of the main or child documents.
To this end compilation flags can be defined
and assigned different default values.
They will be particularly useful in conjunction
with the forwarding mechanism described in \secref{sec:forward}.

For example, it may be useful to have a flag |\version|
which can be set to |draft| or |final|.
The document source will contain some conditional code
depending on the value of |\version|.
Suppose further, the flag should default to |final| for the main file
and to |draft| for child files
which is a natural assignment for editing the document.
This is achieved by placing the following code
in the preamble of the main document
(below the |\childdocmain| directive):
%
\begin{center}
\begin{tabular}{l}
|\ifchilddoc|\\
|\providecommand{\version}{draft}|\\
|\||else|\\
|\providecommand{\version}{final}|\\
|\||fi|
\end{tabular}
\end{center}
%
The definition by |\providecommand| makes sure
that previous definitions are not overwritten.
Further statements |\providecommand{\version}{...}|
can thus be added before the above code to override it.

For the main file, one might add a line
(between |\childdocmain| and the above block)
%
\begin{center}
|%\ifchilddoc\||else\providecommand{\version}{draft}\||fi|
\end{center}
%
which can be uncommented to produce a draft version.
Likewise one can add a line to the very top of a child file
(above the |\childdocof{|\textit{main}|}| directive)
%
\begin{center}
|%\providecommand{\version}{final}|
\end{center}
%
which can be uncommented to produce the final version of this child document.

%%%%%%%%%%%%%%%%%%%%%%%%%%%%%%%%%%%%%%%%%%%%%%%%%%%%%%%%%%%%%%%%%%%%%%%%%%%%%%%%
\subsection{Forwarding}
\label{sec:forward}

Different versions of the main or child documents
using compilation flags as described in \secref{sec:flags}
can be (permanently) stored in different files
for convenient compilation, viewing and distribution.
To this end, the package defines a command
to pass on compilation to a different file:

%%%%%%%%%%%%%%%%%%%%%%%%%%%%%%%%%%%%%%%%
\DescribeMacro{\childdocforward}
The command |\childdocforward| redirects processing to
another source file:
%
\begin{center}
\begin{tabular}{l}
|\input{childdoc.def}|\\
|\childdocforward[|\textit{main}|]{|\textit{dest}|}|\\
\end{tabular}
\end{center}
%
The argument \textit{dest} is the destination file
(without extension).
It should be the main file or one of the child files.
Note that further \textsf{childdoc} directives
such as |\childdocof| and |\childdocforward|
in the indicated file will be processed in this form.
The optional argument \textit{main}
passes on directly to the main file \textit{main}
while pretending to compile the child \textit{dest}.
This form behaves as if \textit{dest}
issues |\childdocof{|\textit{main}|}| right away,
and no further \textsf{childdoc} directives will be processed.

%%%%%%%%%%%%%%%%%%%%%%%%%%%%%%%%%%%%%%%%
\DescribeMacro{\...prefix}
In the alternative form |\childdocforwardprefix|,
%
\begin{center}
\begin{tabular}{l}
|\input{childdoc.def}|\\
|\childdocforwardprefix[|\textit{main}|]{|\textit{prefix}|}{|\textit{dest}|}|
\end{tabular}
\end{center}
%
the destination file is determined by a pattern
depending on the current file:
To make this work, the current file must be called
`{\textit{prefix}\hspace{0.2em}\textit{suffix}}'
with \textit{prefix} matching precisely the argument.
Processing is then passed on to the file
`{\textit{dest}\hspace{0.2em}\textit{suffix}}'.
Surely, the same effect is achieved by
directly specifying the
argument `{\textit{dest}\hspace{0.2em}\textit{suffix}}'
in the first form.
However, that requires to set up a different file
for each child. With the alternative form of the command
all these files can have exactly the same content
which simplifies setting them up and maintaining them.

For example, the following file |draft.tex|
with a compilation flag |\version| as described in \secref{sec:flags}
compiles the main document as a draft:
%
\begin{center}
\begin{tabular}{l}
|\def\version{draft}|\\
|\input{childdoc.def}|\\
|\childdocforward{|\textit{main}|}|
\end{tabular}
\end{center}
%
Likewise, the following files |final|\textit{nn}|.tex|
compile the final version of the child document
|child|\textit{nn}|.tex|:
%
\begin{center}
\begin{tabular}{l}
|\def\version{final}|\\
|\input{childdoc.def}|\\
|\childdocforwardprefix{final}{child}|
\end{tabular}
\end{center}
%

Note that when several versions of a main file and/or of each child file
are to be generated, it may be convenient to set up a |Makefile| or
shell script to automatise the process.

%%%%%%%%%%%%%%%%%%%%%%%%%%%%%%%%%%%%%%%%%%%%%%%%%%%%%%%%%%%%%%%%%%%%%%%%%%%%%%%%
\subsection{Command Line Processing}
\label{sec:commandline}

The effect of redirection files can also be achieved by invoking
the \LaTeX{} compiler with a more elaborate command line.
Most conveniently this should be done as part
of a shell script or a |Makefile|.

When using \textsf{childdoc} in the main file, the following
command lines effectively perform a redirection
(note that depending on the shell being used,
backslashes may have to be doubled: `|\|' $\to$ `|\\|'):
%
\begin{center}
|... -jobname "|\textit{target}|" |\\|"|[\textit{flags}]%
|\input{childdoc.def}\childdocforward[|\textit{main}|]{|\textit{dest}|}"|
\end{center}
%
Here \textit{target} is the name of the output file,
\textit{main} is the name of the main file
and \textit{dest} is the name of the main or child file to be processed
(all filenames without extensions).
The optional argument \textit{main} can be omitted
if \textit{main} matches \textit{dest}.
Optionally, compilation \textit{flags} can be defined via |\def| commands.
This command line makes the \TeX{} engine believe
it is compiling the file \textit{target}
whose content is specified as the latter parameter.
The provided code then forwards the processing to
\textit{main} or \textit{dest} as described in \secref{sec:forward}.

%%%%%%%%%%%%%%%%%%%%%%%%%%%%%%%%%%%%%%%%%%%%%%%%%%%%%%%%%%%%%%%%%%%%%%%%%%%%%%%%
\subsection{Include by Input}
\label{sec:input}

Including child documents by |\include| has some restrictions by design.
Most notably, the content of a child document always occupies
its own set of pages; pages cannot be shared between child documents.
Usually, this behaviour makes perfect sense
because each child document contain an essential part of the document.
However, in some situations it may be desirable to compose
a document from a collection of parts
without having mandatory page breaks between then.
For this case, the package
provides a mechanism to include parts
by |\input| which can also be processed individually.
However, by construction this mechanism
requires manual handling of the content to be output.

%%%%%%%%%%%%%%%%%%%%%%%%%%%%%%%%%%%%%%%%
\DescribeMacro{\ifchilddocmanual}
The main file should be prepared as usual, see \secref{sec:include}.
However, the document body must make a distinction
between processing of an individual part and of the main document, e.g.:
%
\begin{center}
\begin{tabular}{l}
|\ifchilddocmanual|\\
|\input{\childdocname}|\\
|\||else|\\
\textit{document body with }|\input{|\textit{part}|}|\\
|\||fi|
\end{tabular}
\end{center}
%
The conditional |\ifchilddocmanual| is true whenever
a part to be included by |\input| is being compiled,
and the name of the part is stored in |\childdocname|.

%%%%%%%%%%%%%%%%%%%%%%%%%%%%%%%%%%%%%%%%
\DescribeMacro{\childdocby}
Each part to be included by |\input| should start with:
%
\begin{center}
\begin{tabular}{l}
|\input{childdoc.def}|\\
|\childdocby{|\textit{main}|}|\\
\end{tabular}
\end{center}
%
The directive |\childdocby| is similar to |\childdocof|
described in \secref{sec:include},
but the subsequent selection of content must be done manually.
To that end, both |\ifchilddoc| and |\ifchilddocmanual|
will be true upon processing of a part,
and the name of the part is stored in |\childdocname|.
Note that |\jobname| will be set to the filename of the current part
so that each part receives an individual |.aux| file
that does not interfere with the |.aux| file(s) of the main document.
This behaviour can be altered by the alternative form
|\childdocby[*]{|\textit{main}|}| (with a non-empty optional argument)
which uses the |.aux| file of the main document
by setting |\jobname| to \textit{main}.

%%%%%%%%%%%%%%%%%%%%%%%%%%%%%%%%%%%%%%%%%%%%%%%%%%%%%%%%%%%%%%%%%%%%%%%%%%%%%%%%
\subsection{Driver Development}
\label{sec:driver}

The \textsf{childdoc} mechanism can also be use for the development
of definition files such as \LaTeX{} styles or classes.
This case differs from the above setup with multiple parts
included by |\include| in that no |\includeonly| should be invoked.
This can be achieved by starting the include file
(before |\ProvidesPackage|) with:
%
\begin{center}
\begin{tabular}{l}
|\input{childdoc.def}|\\
|\childdocforward{|\textit{main}|}|\\
\end{tabular}
\end{center}
%
or alternatively with:
%
\begin{center}
\begin{tabular}{l}
|\input{childdoc.def}|\\
|\childdocby{|\textit{main}|}|\\
\end{tabular}
\end{center}
%
Both forms have slightly different effects as described above.
The main file is prepared as usual, see \secref{sec:include}.

%%%%%%%%%%%%%%%%%%%%%%%%%%%%%%%%%%%%%%%%%%%%%%%%%%%%%%%%%%%%%%%%%%%%%%%%%%%%%%%%
\subsection{Legacy Detection}
\label{sec:detection}

The directive |\childdocmain| in the main file can detect
whether the complete document or merely a child is to be compiled
even without using the directive |\childdocof|.
This method is deprecated because it is less robust
and there is no compelling reason to use it;
it is merely provided for backward compatibility
and it may be removed in future versions.

If the detection mechanism is to be used,
it is mandatory to correctly specify
the filename of the main file as the argument of |\childdocmain|:
%
\begin{center}
\begin{tabular}{l}
|\input{childdoc.def}|\\
|\childdocmain{|\textit{main}|}|\\
\end{tabular}
\end{center}
%
If |\jobname| does not match the argument \textit{main} of |\childdocmain|,
it is assumed that |\jobname| points to the child file to be compiled.
When using |\childdocmain| with the main file specified as argument,
it suffices to start a child file
with just |\input{|\textit{main}|}|
without loading of the package and using |\childdocof|.
If instead all processing is done
with the appropriate \textsf{childdoc} directives,
the argument of \textit{main} of |\childdocmain| can be empty.

An alternative version of the command line processing described
in \secref{sec:commandline} using the detection mechanism reads:
%
\begin{center}
|... -jobname "|\textit{target}|" "|[\textit{flags}]%
[|\def\jobname{|\textit{dest}|}|]|\input{|\textit{main}|}"|
\end{center}

%%%%%%%%%%%%%%%%%%%%%%%%%%%%%%%%%%%%%%%%%%%%%%%%%%%%%%%%%%%%%%%%%%%%%%%%%%%%%%%%
\subsection{Manual Code}
\label{sec:manual}

In case one cannot be certain whether the definitions file |childdoc.def|
is installed on the target \TeX{} distribution
and one prefers not to ship it,
it is conceivable to paste a few relevant commands into the sources.

To that end, drop all statements |\input{childdoc.def}|
and perform the replacements as outlined below.
Instead of |\childdocmain{|\textit{main}|}| add the following code
to the top of the main file:
%
\begin{center}
\begin{tabular}{l}
|\||ifdefined\childdocname\endinput\||fi\newif\ifchilddoc|\\
|\edef\childdocname{\scantokens\expandafter{\jobname\noexpand}}|\\
|\def\childdocmain{|\textit{main}|}\||ifx\childdocmain\childdocname\||else|\\
|\childdoctrue\includeonly{\childdocname}\let\jobname\childdocmain\||fi|\\
\end{tabular}
\end{center}
%
Instead of |\childdocof{|\textit{main}|}| just include the main file
at the top of each child file:
%
\begin{center}
|\input{|\textit{main}|}|
\end{center}
%
A simple redirection |\childdocforward{|\textit{dest}|}| is achieved by:
%
\begin{center}
|\def\jobname{|\textit{dest}|}\input{\jobname}|
\end{center}
%
The redirection with prefix
|\childdocforwardprefix[|\textit{prefix}|]{|\textit{dest}|}|
is accomplished by:
%
\begin{center}
\begin{tabular}{l}
|{\edef\jobname{\scantokens\expandafter{\jobname\noexpand}}|\\
|\def\redirectjob |\textit{prefix}|#1~~~{\gdef\jobname{|\textit{dest}|#1}}|\\
|\expandafter\redirectjob\jobname~~~}\input{\jobname}|
\end{tabular}
\end{center}

In an alternative approach,
child documents can be compiled by a specific command line
without additional code or specific definitions:
%
\begin{center}
|... -jobname "|\textit{target}|" "|[\textit{flags}]%
|\includeonly{|\textit{dest}|}\input{|\textit{main}|}"|
\end{center}
%

%%%%%%%%%%%%%%%%%%%%%%%%%%%%%%%%%%%%%%%%%%%%%%%%%%%%%%%%%%%%%%%%%%%%%%%%%%%%%%%%
%%%%%%%%%%%%%%%%%%%%%%%%%%%%%%%%%%%%%%%%%%%%%%%%%%%%%%%%%%%%%%%%%%%%%%%%%%%%%%%%
\section{Information}

%%%%%%%%%%%%%%%%%%%%%%%%%%%%%%%%%%%%%%%%%%%%%%%%%%%%%%%%%%%%%%%%%%%%%%%%%%%%%%%%
\subsection{Copyright}

Copyright \copyright{} 2017--2018 Niklas Beisert

This work may be distributed and/or modified under the
conditions of the \LaTeX{} Project Public License, either version 1.3
of this license or (at your option) any later version.
The latest version of this license is in
  \url{http://www.latex-project.org/lppl.txt}
and version 1.3 or later is part of all distributions of \LaTeX{}
version 2005/12/01 or later.

This work has the LPPL maintenance status `maintained'.

The Current Maintainer of this work is Niklas Beisert.

This work consists of the files |README.txt|, |childdoc.ins| and |childdoc.dtx|
as well as the derived files |childdoc.def|, |cdocsamp.tex|
with |cdocsch1.tex|, |cdocsch2.tex|, |cdocspt3.tex|, |cdocspt4.tex|,
|cdocsdrf.tex|, |cdocsfn1.tex|, |cdocsfn2.tex|
as well as |childdoc.pdf|.

%%%%%%%%%%%%%%%%%%%%%%%%%%%%%%%%%%%%%%%%%%%%%%%%%%%%%%%%%%%%%%%%%%%%%%%%%%%%%%%%
\subsection{Files and Installation}

The package consists of the files:
%
\begin{center}
\begin{tabular}{ll}
    |README.txt|   & readme file \\
    |childdoc.ins| & installation file \\
    |childdoc.dtx| & source file \\
    |childdoc.def| & definition file \\
    |cdocsamp.tex| & sample main file \\
    |cdocsch1.tex| & sample include file \\
    |cdocsch2.tex| & sample include file \\
    |cdocspt3.tex| & sample part file \\
    |cdocspt4.tex| & sample part file \\
    |cdocsdrf.tex| & sample redirection file \\
    |cdocsfn1.tex| & sample redirection file \\
    |cdocsfn2.tex| & sample redirection file \\
    |childdoc.pdf| & manual
\end{tabular}
\end{center}
%
The distribution consists of the files
|README.txt|, |childdoc.ins| and |childdoc.dtx|.
%
\begin{itemize}
\item
Run (pdf)\LaTeX{} on |childdoc.dtx|
to compile the manual |childdoc.pdf| (this file).
\item
Run \LaTeX{} on |childdoc.ins| to create the definitions file |childdoc.def|
and the sample |cdocsamp.tex| with include files
|cdocsch1.tex|, |cdocsch2.tex|, |cdocspt3.tex|, |cdocspt4.tex|,
|cdocsdrf.tex|, |cdocsfn1.tex|, |cdocsfn2.tex|.
Then copy the file |childdoc.def| to an appropriate directory of your \LaTeX{}
distribution, e.g.\ \textit{texmf-root}|/tex/latex/childdoc|.
\end{itemize}

%%%%%%%%%%%%%%%%%%%%%%%%%%%%%%%%%%%%%%%%%%%%%%%%%%%%%%%%%%%%%%%%%%%%%%%%%%%%%%%%
\subsection{Related CTAN Packages}

There are several other packages which offer a similar functionality:
%
\begin{itemize}
\item
The packages
\href{http://ctan.org/pkg/docmute}{\textsf{docmute}},
\href{http://ctan.org/pkg/includex}{\textsf{includex}} and
\href{http://ctan.org/pkg/standalone}{\textsf{standalone}}
provide commands to include only the document body of
a child file thus allowing both files to be compiled individually.
\item
The packages \href{http://ctan.org/pkg/subdocs}{\textsf{subdocs}}
and \href{http://ctan.org/pkg/subfiles}{\textsf{subfiles}}
provide structures in which the main and child documents can be
encapsulated and allowing them to be compiled individually.
The inclusion mechanism is different from the conventional |\include|.
\item
The package \href{http://ctan.org/pkg/combine}{\textsf{combine}}
is an elaborate solution to combine several documents into one.
\end{itemize}
%
See also the CTAN topic \href{http://ctan.org/topic/subdocs}{\textsf{subdocs}}
for further related packages.
The present package differs from the above solutions in that
a document structure constructed with the conventional |\include| mechanism
just needs two extra commands at the top of every file
such that all constituent files can be compiled individually.

%%%%%%%%%%%%%%%%%%%%%%%%%%%%%%%%%%%%%%%%%%%%%%%%%%%%%%%%%%%%%%%%%%%%%%%%%%%%%%%%
%\subsection{Feature Suggestions}
%
%The following is a list of features which may be useful for future
%versions of this package:
%%
%\begin{itemize}
%\item
%\ldots
%\end{itemize}

%%%%%%%%%%%%%%%%%%%%%%%%%%%%%%%%%%%%%%%%%%%%%%%%%%%%%%%%%%%%%%%%%%%%%%%%%%%%%%%%
\subsection{Revision History}

%%%%%%%%%%%%%%%%%%%%%%%%%%%%%%%%%%%%%%%%
\paragraph{v2.0:} 2018/12/30

\begin{itemize}
\item
immediate forward processing
\item
added |\childdocby| mechanism
\item
manual restructured
\end{itemize}

%%%%%%%%%%%%%%%%%%%%%%%%%%%%%%%%%%%%%%%%
\paragraph{v1.6:} 2018/01/17

\begin{itemize}
\item
application for development of include files
\item
corrections to manual
\end{itemize}

%%%%%%%%%%%%%%%%%%%%%%%%%%%%%%%%%%%%%%%%
\paragraph{v1.5:} 2017/05/21

\begin{itemize}
\item
more complete structuring introduced
\item
|\childdocof| introduced
\item
|\childdoc| renamed to |\childdocmain|
\item
|\childredirect| renamed to |\childdocforward| and |\childdocforwardprefix|
and functionality expanded
\end{itemize}

%%%%%%%%%%%%%%%%%%%%%%%%%%%%%%%%%%%%%%%%
\paragraph{v1.0:} 2017/04/27

\begin{itemize}
\item
manual and install package
\item
first version published on CTAN
\end{itemize}

%%%%%%%%%%%%%%%%%%%%%%%%%%%%%%%%%%%%%%%%
\paragraph{v0.6:} 2017/04/26

\begin{itemize}
\item
redirection mechanism added
\end{itemize}

%%%%%%%%%%%%%%%%%%%%%%%%%%%%%%%%%%%%%%%%
\paragraph{v0.5:} 2017/04/26

\begin{itemize}
\item
functionality in definition file
\end{itemize}


%%%%%%%%%%%%%%%%%%%%%%%%%%%%%%%%%%%%%%%%%%%%%%%%%%%%%%%%%%%%%%%%%%%%%%%%%%%%%%%%
%%%%%%%%%%%%%%%%%%%%%%%%%%%%%%%%%%%%%%%%%%%%%%%%%%%%%%%%%%%%%%%%%%%%%%%%%%%%%%%%
%%%%%%%%%%%%%%%%%%%%%%%%%%%%%%%%%%%%%%%%%%%%%%%%%%%%%%%%%%%%%%%%%%%%%%%%%%%%%%%%
\appendix

\settowidth\MacroIndent{\rmfamily\scriptsize 000\ }

 \DocInput{childdoc.dtx}

\end{document}
%</driver>
% \fi
%
% %%%%%%%%%%%%%%%%%%%%%%%%%%%%%%%%%%%%%%%%%%%%%%%%%%%%%%%%%%%%%%%%%%%%%%%%%%%%%%
% %%%%%%%%%%%%%%%%%%%%%%%%%%%%%%%%%%%%%%%%%%%%%%%%%%%%%%%%%%%%%%%%%%%%%%%%%%%%%%
% \section{Sample}
%\iffalse
%<*samplemain>
%\fi
%
% The following presents a sample document
% with two chapters, two parts, a title page,
% a compile flag as well as three forwarding files to set the flag.
% It consists of eight |.tex| files:
% \begin{center}
% \begin{tabular}{ll}
% |cdocsamp.tex|&main file\\
% |cdocsch1.tex|&include file for chapter 1\\
% |cdocsch2.tex|&include file for chapter 2\\
% |cdocspt3.tex|&include file for part 3\\
% |cdocspt4.tex|&include file for part 4\\
% |cdocsdrf.tex|&forwarding file for main file in draft mode\\
% |cdocsfi1.tex|&forwarding file for final version of chapter 1\\
% |cdocsfi2.tex|&forwarding file for final version of chapter 2\\
% \end{tabular}
% \end{center}
% Each of the eight files can be compiled directly by the \LaTeX{} compiler.
%
% %%%%%%%%%%%%%%%%%%%%%%%%%%%%%%%%%%%%%%
% \paragraph{Main File.}
%
% The main file is called |cdocsamp.tex|.
%
% Load the \textsf{childdoc} definitions and
% declare the filename for the main document:
%    \begin{macrocode}
\input{childdoc.def}
\childdocmain{}
%    \end{macrocode}

% Optional override for |\version| flag:
%    \begin{macrocode}
%%\ifchilddoc\else\providecommand{\version}{draft}\fi
%    \end{macrocode}

% Define the default values for the |\version| flag
% (|final| for the main file and |draft| for childs):
%    \begin{macrocode}
\ifchilddoc
\providecommand{\version}{draft}
\else
\providecommand{\version}{final}
\fi
%    \end{macrocode}

% Load the standard document class:
%    \begin{macrocode}
\documentclass[12pt]{article}
%    \end{macrocode}

% Start the document body:
%    \begin{macrocode}
\begin{document}
%    \end{macrocode}

% Declare a title page.
% Print title, part of document being processed and version flag:
%    \begin{macrocode}
\addtocounter{page}{-1}
\begin{center}
{\LARGE\bfseries{}childdoc example\par}
\vspace{1cm}
\ifchilddoc
\ifchilddocmanual part\else chapter\fi:
`\childdocname' of `\childdocjob'\par
\else
main document: `\childdocjob'\par
\fi
version: \version\par
\end{center}
\newpage
%    \end{macrocode}

% Manually include selected file,
% otherwise process as usual:
%    \begin{macrocode}
\ifchilddocmanual
\section*{part `\childdocname'}
\input{\childdocname}
\else
%    \end{macrocode}

% Include the two chapters:
%    \begin{macrocode}
\include{cdocsch1}
\include{cdocsch2}
%    \end{macrocode}

% Include the two parts unless only chapters should be displayed:
%    \begin{macrocode}
\ifchilddoc\else
\section{part three}
\input{cdocspt3}
\section{part four}
\input{cdocspt4}
\fi
%    \end{macrocode}

% Process as usual until here:
%    \begin{macrocode}
\fi
%    \end{macrocode}

% End of document body:
%    \begin{macrocode}
\end{document}
%    \end{macrocode}
%\iffalse
%</samplemain>
%\fi
%
% %%%%%%%%%%%%%%%%%%%%%%%%%%%%%%%%%%%%%%
% \paragraph{Chapter Include Files.}
%
% The include files are called |cdocsch1.tex| and |cdocsch2.tex|.
%
%\iffalse
%<*samplechap1|samplechap2>
%\fi

% Optional override for |\version| flag:
%    \begin{macrocode}
%%\providecommand{\version}{final}
%    \end{macrocode}

% Include the main document:
%    \begin{macrocode}
\input{childdoc.def}
\childdocof{cdocsamp}
%    \end{macrocode}

%\iffalse
%</samplechap1|samplechap2>
%\fi
%
%\iffalse
%<*samplechap1>
%\fi
% Some text for chapter 1:
%    \begin{macrocode}
\section{one}
some text in chapter one
%    \end{macrocode}

%\iffalse
%</samplechap1>
%\fi
% Some text for chapter 2:
%\iffalse
%<*samplechap2>
%\fi
%    \begin{macrocode}
\section{two}
more text in chapter two
%    \end{macrocode}

%\iffalse
%</samplechap2>
%\fi
%
% %%%%%%%%%%%%%%%%%%%%%%%%%%%%%%%%%%%%%%
% \paragraph{Part Include Files.}
%
% The include files are called |cdocspt3.tex| and |cdocspt4.tex|.
%
%\iffalse
%<*samplepart3|samplepart4>
%\fi

% Optional override for |\version| flag:
%    \begin{macrocode}
%%\providecommand{\version}{final}
%    \end{macrocode}

% Include the main document:
%    \begin{macrocode}
\input{childdoc.def}
\childdocby{cdocsamp}
%    \end{macrocode}

%\iffalse
%</samplepart3|samplepart4>
%\fi
%
%\iffalse
%<*samplepart3>
%\fi
% Some text for part 3:
%    \begin{macrocode}
some text in part three
%    \end{macrocode}

%\iffalse
%</samplepart3>
%\fi
% Some text for part 4:
%\iffalse
%<*samplepart4>
%\fi
%    \begin{macrocode}
more text in part four
%    \end{macrocode}

%\iffalse
%</samplepart4>
%\fi
%
% %%%%%%%%%%%%%%%%%%%%%%%%%%%%%%%%%%%%%%
% \paragraph{Forwarding for a Complete Draft.}
%
% The following forwarding file |cdocsdrf.tex|
% compiles the main document in draft mode:
%\iffalse
%<*sampledraft>
%\fi
%    \begin{macrocode}
\def\version{draft}
\input{childdoc.def}
\childdocforward{cdocsamp}
%    \end{macrocode}

%\iffalse
%</sampledraft>
%\fi
%
% %%%%%%%%%%%%%%%%%%%%%%%%%%%%%%%%%%%%%%
% \paragraph{Forwarding for Final Version of the Chapters.}
%
% The following forwarding files |cdocsfn1.tex| and |cdocsfn2.tex|
% (with identical content)
% compile the final versions of the child documents
% |cdocsch1.tex| and |cdocsch2.tex|, respectively:
%\iffalse
%<*samplefinal>
%\fi
%    \begin{macrocode}
\def\version{final}
\input{childdoc.def}
\childdocforwardprefix[cdocsamp]{cdocsfn}{cdocsch}
%    \end{macrocode}

%\iffalse
%</samplefinal>
%\fi
%
% %%%%%%%%%%%%%%%%%%%%%%%%%%%%%%%%%%%%%%
% \paragraph{Command Line Processing.}
%
% The following three command lines generate the output files
% |cdocscld|, |cdocscl1| and |cdocscl2|
% which should be identical to
% |cdocsdrf|, |cdocsch1| and |cdocsfn2|, respectively:
% \begin{center}
% \begin{tabular}{l}
% |latex -jobname cdocscld \|\\
% |  "\def\version{draft}\input{childdoc.def}\childdocforward{cdocsamp}"|\\
% |latex -jobname cdocscl1 \|\\
% |  "\input{childdoc.def}\childdocforward[cdocsamp]{cdocsch1}"|\\
% |latex -jobname cdocscl2 \|\\
% |  "\def\version{final}\input{childdoc.def}\childdocforward{cdocsch2}"|
% \end{tabular}
% \end{center}
% Note that the trailing backslash on each first line
% merely continues the input to the second line
% (for convenient cut ant paste).
% Furthermore, the command |latex| can be replaced by any
% of its alternative versions such as |pdflatex|.
%
% %%%%%%%%%%%%%%%%%%%%%%%%%%%%%%%%%%%%%%%%%%%%%%%%%%%%%%%%%%%%%%%%%%%%%%%%%%%%%%
% %%%%%%%%%%%%%%%%%%%%%%%%%%%%%%%%%%%%%%%%%%%%%%%%%%%%%%%%%%%%%%%%%%%%%%%%%%%%%%
% \section{Implementation}
%\iffalse
%<*package>
%\fi
%
% This section describes the definitions file |childdoc.def|.

% The definitions cannot be loaded using |\usepackage| or |\RequirePackage|
% which has a mechanism to prevent loading a style file more than once.
% When loading the definitions by means of |\input|
% multiple instances have to be prevented manually:
%\iffalse
%This code needs to be before the `\ProvidesFile' directive
%which is defined at the beginning of this file.
%Therefore it is also placed there and commented out here.
%</package>
%<*discard>
%\fi
%    \begin{macrocode}
\ifdefined\childdocmain\endinput\fi
%    \end{macrocode}
%\iffalse
%</discard>
%<*package>
%\fi
%
% \macro{\ifchilddoc}
% \macro{\ifchilddocmanual}
% The conditional |\ifchilddoc| tells whether a
% child (true) or main (false) document is being compiled.
% The conditional |\ifchilddocmanual| tells whether
% the |\includeonly| mechanism is used (false) or
% the selection of child files must be performed manually (true).
% The definitions initialise to false:
%    \begin{macrocode}
\newif\ifchilddoc
\newif\ifchilddocmanual
%    \end{macrocode}

% \macro{\childdocname}
% \macro{\childdocjob}
% The macro |\childdocname| stores the name of the main document
% to be compiled. The macro |\childdocjob| stores the name of
% the document on which the \LaTeX{} compiler was originally invoked.
% The content of |\jobname| cannot be compared
% to filenames specified in the source due to different catcodes.
% The following code rescans |\jobname|, stores the result
% in |\childdocname| and saves a copy in |\childdocjob|:
%    \begin{macrocode}
\edef\childdocname{\scantokens\expandafter{\jobname\noexpand}}
\let\childdocjob\childdocname
%    \end{macrocode}

% \macro{\childdocdisable}
% The macro |\childdocdisable| prevents the main file
% from being processed more than once.
% At this stage, the main document command |\childdocmain|
% is assumed to be called once again where it should do nothing.
% Any subsequent call to it should prevent
% a secondary processing of the main document
% It overwrites the forwarding commands
% |\childdocof| and |\childdocforward|
% with empty macros to prevent further inclusions of the main document:
%    \begin{macrocode}
\newcommand{\childdocdisable}
{
  \renewcommand{\childdocmain}[1]{\renewcommand{\childdocmain}[1]{\endinput}}
  \renewcommand{\childdocof}[1]{}
  \renewcommand{\childdocby}[2][]{}
  \renewcommand{\childdocforward}[2][]{}
  \renewcommand{\childdocdisable}{}
}
%    \end{macrocode}

% \macro{\childdocmain}
% The macro |\childdocmain| is to be called at the top of the main file
% with nothing or the main filename (without extension) as argument.
% First, it breaks loops.
% If the argument is not empty and does not match |\childdocname|
% (which is set by the first inclusion of |childdoc.def|),
% |\ifchilddoc| is set to true, |\includeonly| is applied to the child file
% and |\jobname| is set to the main file
% (for proper handling of |.aux| files):
%    \begin{macrocode}
\newcommand{\childdocmain}[1]
{
  \childdocdisable\childdocmain{}
  \if?#1?\else
    \begingroup
      \def\childdoctmp{#1}
      \ifx\childdoctmp\childdocname
        \def\childdoctmp{}
      \else
        \def\childdoctmp
        {
          \childdoctrue
          \includeonly{\childdocname}
          \def\childdocjob{#1}
          \def\jobname{#1}
        }
      \fi
      \expandafter
    \endgroup
    \childdoctmp
  \fi
}
%    \end{macrocode}

% \macro{\childdocof}
% The command |\childdocof| redirects
% compilation to the main file |#1|.
%    \begin{macrocode}
\newcommand{\childdocof}[1]
{
  \childdocdisable
  \childdoctrue
  \includeonly{\childdocname}
  \def\jobname{#1}
  \def\childdocjob{#1}
  \input{#1}
}
%    \end{macrocode}

% \macro{\childdocby}
% The command |\childdocby| ....
%    \begin{macrocode}
\newcommand{\childdocby}[2][]
{
  \childdocdisable
  \childdoctrue
  \childdocmanualtrue
  \if?#1?\else
    \def\jobname{#2}
  \fi
  \def\childdocjob{#2}
  \input{#2}
  \endinput
}
%    \end{macrocode}

% \macro{\childdocforward}
% The command |\childdocforward| redirects
% compilation to the main file or
% (if the optional argument is given) a child file.
% Parameters are set as if the main file
% or a child file starting with |\childdocof| was compiled.
% Then compilation is handed over to the main file:
%    \begin{macrocode}
\newcommand{\childdocforward}[2][]
{
  \begingroup
    \if?#1?
      \def\childdoctmp
      {
        \def\childdocname{#2}
        \def\childdocjob{#2}
        \def\jobname{#2}
        \input{#2}
        \endinput
      }
    \else
      \def\childdoctmp
      {
        \childdocdisable
        \def\childdocname{#2}
        \childdoctrue
        \includeonly{#2}
        \def\childdocjob{#1}
        \def\jobname{#1}
        \input{#1}
        \endinput
      }
    \fi
    \expandafter
  \endgroup
  \childdoctmp
}
%    \end{macrocode}

% \macro{\childdocforwardprefix}
% The command |\childdocforwardprefix| redirects
% compilation to the main or a child file by means of a pattern.
% The prefix |#1| in the current filename is replaced by |#2|
% and the suffix of the current filename is kept
% (it is assumed that the filename does not contain the substring `|~~~|'
% which is used as a delimiter).
% Compilation is handed over to the new file by |\childdocforward|:
%    \begin{macrocode}
\newcommand{\childdocforwardprefix}[3][]
{
  \begingroup
    \def\childdocextract #2##1~~~{\def\childdoctmp{\childdocforward[#1]{#3##1}}}
    \expandafter\childdocextract\childdocname~~~
    \expandafter
  \endgroup
  \childdoctmp
}
%    \end{macrocode}

% \macro{\childdoc}
% The deprecated macro |\childdoc| is a legacy version of |\childdocmain|:
%    \begin{macrocode}
\newcommand{\childdoc}{\childdocmain}
%    \end{macrocode}

% \macro{\childdocredirect}
% The deprecated macro |\childdocredirect| is a legacy version
% of |\childdocforward| and |\childdocforwardprefix|:
%    \begin{macrocode}
\newcommand{\childdocredirect}[2][]
{
  \begingroup
    \if?#1?
      \def\childdoctmp{\childdocforward{#2}}
    \else
      \def\childdoctmp{\childdocforwardprefix{#1}{#2}}
    \fi
    \expandafter
  \endgroup
  \childdoctmp
}
%    \end{macrocode}

%\iffalse
%</package>
%\fi
%
\endinput
|
and perform the replacements as outlined below.
Instead of |\childdocmain{|\textit{main}|}| add the following code
to the top of the main file:
%
\begin{center}
\begin{tabular}{l}
|\||ifdefined\childdocname\endinput\||fi\newif\ifchilddoc|\\
|\edef\childdocname{\scantokens\expandafter{\jobname\noexpand}}|\\
|\def\childdocmain{|\textit{main}|}\||ifx\childdocmain\childdocname\||else|\\
|\childdoctrue\includeonly{\childdocname}\let\jobname\childdocmain\||fi|\\
\end{tabular}
\end{center}
%
Instead of |\childdocof{|\textit{main}|}| just include the main file
at the top of each child file:
%
\begin{center}
|\input{|\textit{main}|}|
\end{center}
%
A simple redirection |\childdocforward{|\textit{dest}|}| is achieved by:
%
\begin{center}
|\def\jobname{|\textit{dest}|}\input{\jobname}|
\end{center}
%
The redirection with prefix
|\childdocforwardprefix[|\textit{prefix}|]{|\textit{dest}|}|
is accomplished by:
%
\begin{center}
\begin{tabular}{l}
|{\edef\jobname{\scantokens\expandafter{\jobname\noexpand}}|\\
|\def\redirectjob |\textit{prefix}|#1~~~{\gdef\jobname{|\textit{dest}|#1}}|\\
|\expandafter\redirectjob\jobname~~~}\input{\jobname}|
\end{tabular}
\end{center}

In an alternative approach,
child documents can be compiled by a specific command line
without additional code or specific definitions:
%
\begin{center}
|... -jobname "|\textit{target}|" "|[\textit{flags}]%
|\includeonly{|\textit{dest}|}\input{|\textit{main}|}"|
\end{center}
%

%%%%%%%%%%%%%%%%%%%%%%%%%%%%%%%%%%%%%%%%%%%%%%%%%%%%%%%%%%%%%%%%%%%%%%%%%%%%%%%%
%%%%%%%%%%%%%%%%%%%%%%%%%%%%%%%%%%%%%%%%%%%%%%%%%%%%%%%%%%%%%%%%%%%%%%%%%%%%%%%%
\section{Information}

%%%%%%%%%%%%%%%%%%%%%%%%%%%%%%%%%%%%%%%%%%%%%%%%%%%%%%%%%%%%%%%%%%%%%%%%%%%%%%%%
\subsection{Copyright}

Copyright \copyright{} 2017--2018 Niklas Beisert

This work may be distributed and/or modified under the
conditions of the \LaTeX{} Project Public License, either version 1.3
of this license or (at your option) any later version.
The latest version of this license is in
  \url{http://www.latex-project.org/lppl.txt}
and version 1.3 or later is part of all distributions of \LaTeX{}
version 2005/12/01 or later.

This work has the LPPL maintenance status `maintained'.

The Current Maintainer of this work is Niklas Beisert.

This work consists of the files |README.txt|, |childdoc.ins| and |childdoc.dtx|
as well as the derived files |childdoc.def|, |cdocsamp.tex|
with |cdocsch1.tex|, |cdocsch2.tex|, |cdocspt3.tex|, |cdocspt4.tex|,
|cdocsdrf.tex|, |cdocsfn1.tex|, |cdocsfn2.tex|
as well as |childdoc.pdf|.

%%%%%%%%%%%%%%%%%%%%%%%%%%%%%%%%%%%%%%%%%%%%%%%%%%%%%%%%%%%%%%%%%%%%%%%%%%%%%%%%
\subsection{Files and Installation}

The package consists of the files:
%
\begin{center}
\begin{tabular}{ll}
    |README.txt|   & readme file \\
    |childdoc.ins| & installation file \\
    |childdoc.dtx| & source file \\
    |childdoc.def| & definition file \\
    |cdocsamp.tex| & sample main file \\
    |cdocsch1.tex| & sample include file \\
    |cdocsch2.tex| & sample include file \\
    |cdocspt3.tex| & sample part file \\
    |cdocspt4.tex| & sample part file \\
    |cdocsdrf.tex| & sample redirection file \\
    |cdocsfn1.tex| & sample redirection file \\
    |cdocsfn2.tex| & sample redirection file \\
    |childdoc.pdf| & manual
\end{tabular}
\end{center}
%
The distribution consists of the files
|README.txt|, |childdoc.ins| and |childdoc.dtx|.
%
\begin{itemize}
\item
Run (pdf)\LaTeX{} on |childdoc.dtx|
to compile the manual |childdoc.pdf| (this file).
\item
Run \LaTeX{} on |childdoc.ins| to create the definitions file |childdoc.def|
and the sample |cdocsamp.tex| with include files
|cdocsch1.tex|, |cdocsch2.tex|, |cdocspt3.tex|, |cdocspt4.tex|,
|cdocsdrf.tex|, |cdocsfn1.tex|, |cdocsfn2.tex|.
Then copy the file |childdoc.def| to an appropriate directory of your \LaTeX{}
distribution, e.g.\ \textit{texmf-root}|/tex/latex/childdoc|.
\end{itemize}

%%%%%%%%%%%%%%%%%%%%%%%%%%%%%%%%%%%%%%%%%%%%%%%%%%%%%%%%%%%%%%%%%%%%%%%%%%%%%%%%
\subsection{Related CTAN Packages}

There are several other packages which offer a similar functionality:
%
\begin{itemize}
\item
The packages
\href{http://ctan.org/pkg/docmute}{\textsf{docmute}},
\href{http://ctan.org/pkg/includex}{\textsf{includex}} and
\href{http://ctan.org/pkg/standalone}{\textsf{standalone}}
provide commands to include only the document body of
a child file thus allowing both files to be compiled individually.
\item
The packages \href{http://ctan.org/pkg/subdocs}{\textsf{subdocs}}
and \href{http://ctan.org/pkg/subfiles}{\textsf{subfiles}}
provide structures in which the main and child documents can be
encapsulated and allowing them to be compiled individually.
The inclusion mechanism is different from the conventional |\include|.
\item
The package \href{http://ctan.org/pkg/combine}{\textsf{combine}}
is an elaborate solution to combine several documents into one.
\end{itemize}
%
See also the CTAN topic \href{http://ctan.org/topic/subdocs}{\textsf{subdocs}}
for further related packages.
The present package differs from the above solutions in that
a document structure constructed with the conventional |\include| mechanism
just needs two extra commands at the top of every file
such that all constituent files can be compiled individually.

%%%%%%%%%%%%%%%%%%%%%%%%%%%%%%%%%%%%%%%%%%%%%%%%%%%%%%%%%%%%%%%%%%%%%%%%%%%%%%%%
%\subsection{Feature Suggestions}
%
%The following is a list of features which may be useful for future
%versions of this package:
%%
%\begin{itemize}
%\item
%\ldots
%\end{itemize}

%%%%%%%%%%%%%%%%%%%%%%%%%%%%%%%%%%%%%%%%%%%%%%%%%%%%%%%%%%%%%%%%%%%%%%%%%%%%%%%%
\subsection{Revision History}

%%%%%%%%%%%%%%%%%%%%%%%%%%%%%%%%%%%%%%%%
\paragraph{v2.0:} 2018/12/30

\begin{itemize}
\item
immediate forward processing
\item
added |\childdocby| mechanism
\item
manual restructured
\end{itemize}

%%%%%%%%%%%%%%%%%%%%%%%%%%%%%%%%%%%%%%%%
\paragraph{v1.6:} 2018/01/17

\begin{itemize}
\item
application for development of include files
\item
corrections to manual
\end{itemize}

%%%%%%%%%%%%%%%%%%%%%%%%%%%%%%%%%%%%%%%%
\paragraph{v1.5:} 2017/05/21

\begin{itemize}
\item
more complete structuring introduced
\item
|\childdocof| introduced
\item
|\childdoc| renamed to |\childdocmain|
\item
|\childredirect| renamed to |\childdocforward| and |\childdocforwardprefix|
and functionality expanded
\end{itemize}

%%%%%%%%%%%%%%%%%%%%%%%%%%%%%%%%%%%%%%%%
\paragraph{v1.0:} 2017/04/27

\begin{itemize}
\item
manual and install package
\item
first version published on CTAN
\end{itemize}

%%%%%%%%%%%%%%%%%%%%%%%%%%%%%%%%%%%%%%%%
\paragraph{v0.6:} 2017/04/26

\begin{itemize}
\item
redirection mechanism added
\end{itemize}

%%%%%%%%%%%%%%%%%%%%%%%%%%%%%%%%%%%%%%%%
\paragraph{v0.5:} 2017/04/26

\begin{itemize}
\item
functionality in definition file
\end{itemize}


%%%%%%%%%%%%%%%%%%%%%%%%%%%%%%%%%%%%%%%%%%%%%%%%%%%%%%%%%%%%%%%%%%%%%%%%%%%%%%%%
%%%%%%%%%%%%%%%%%%%%%%%%%%%%%%%%%%%%%%%%%%%%%%%%%%%%%%%%%%%%%%%%%%%%%%%%%%%%%%%%
%%%%%%%%%%%%%%%%%%%%%%%%%%%%%%%%%%%%%%%%%%%%%%%%%%%%%%%%%%%%%%%%%%%%%%%%%%%%%%%%
\appendix

\settowidth\MacroIndent{\rmfamily\scriptsize 000\ }

 \DocInput{childdoc.dtx}

\end{document}
%</driver>
% \fi
%
% %%%%%%%%%%%%%%%%%%%%%%%%%%%%%%%%%%%%%%%%%%%%%%%%%%%%%%%%%%%%%%%%%%%%%%%%%%%%%%
% %%%%%%%%%%%%%%%%%%%%%%%%%%%%%%%%%%%%%%%%%%%%%%%%%%%%%%%%%%%%%%%%%%%%%%%%%%%%%%
% \section{Sample}
%\iffalse
%<*samplemain>
%\fi
%
% The following presents a sample document
% with two chapters, two parts, a title page,
% a compile flag as well as three forwarding files to set the flag.
% It consists of eight |.tex| files:
% \begin{center}
% \begin{tabular}{ll}
% |cdocsamp.tex|&main file\\
% |cdocsch1.tex|&include file for chapter 1\\
% |cdocsch2.tex|&include file for chapter 2\\
% |cdocspt3.tex|&include file for part 3\\
% |cdocspt4.tex|&include file for part 4\\
% |cdocsdrf.tex|&forwarding file for main file in draft mode\\
% |cdocsfi1.tex|&forwarding file for final version of chapter 1\\
% |cdocsfi2.tex|&forwarding file for final version of chapter 2\\
% \end{tabular}
% \end{center}
% Each of the eight files can be compiled directly by the \LaTeX{} compiler.
%
% %%%%%%%%%%%%%%%%%%%%%%%%%%%%%%%%%%%%%%
% \paragraph{Main File.}
%
% The main file is called |cdocsamp.tex|.
%
% Load the \textsf{childdoc} definitions and
% declare the filename for the main document:
%    \begin{macrocode}
% \iffalse
%
% childdoc.dtx Copyright (C) 2017-2018 Niklas Beisert
%
% This work may be distributed and/or modified under the
% conditions of the LaTeX Project Public License, either version 1.3
% of this license or (at your option) any later version.
% The latest version of this license is in
%   http://www.latex-project.org/lppl.txt
% and version 1.3 or later is part of all distributions of LaTeX
% version 2005/12/01 or later.
%
% This work has the LPPL maintenance status `maintained'.
%
% The Current Maintainer of this work is Niklas Beisert.
%
% This work consists of the files childdoc.dtx and childdoc.ins
% and the derived files childdoc.def and cdocsamp.tex with
% cdocsch1.tex, cdocsch2.tex, cdocsdrf.tex, cdocsfn1.tex, cdocsfn2.tex.
%
%<package>\ifdefined\childdocmain\endinput\fi
%<package>\ProvidesFile{childdoc.def}[2018/12/30 v2.0 child document driver]
%<samplemain>\ProvidesFile{cdocsamp.tex}[2018/12/30 v2.0 sample for childdoc]
%<*driver>
%\ProvidesFile{childdoc.drv}[2018/12/30 v2.0 childdoc reference manual file]
\PassOptionsToClass{10pt,a4paper}{article}
\documentclass{ltxdoc}

\usepackage[margin=35mm]{geometry}
\usepackage{hyperref}
\usepackage{hyperxmp}
\usepackage[usenames]{color}

\hypersetup{colorlinks=true}
\hypersetup{pdfstartview=FitH}
\hypersetup{pdfpagemode=UseNone}
\hypersetup{pdfsource={}}
\hypersetup{pdflang={en-UK}}
\hypersetup{pdfcopyright={Copyright 2017-2018 Niklas Beisert.
  This work may be distributed and/or modified under the
  conditions of the LaTeX Project Public License, either version 1.3
  of this license or (at your option) any later version.}}
\hypersetup{pdflicenseurl={http://www.latex-project.org/lppl.txt}}
\hypersetup{pdfcontactaddress={ETH Zurich, ITP, HIT K,
  Wolfgang-Pauli-Strasse 27}}
\hypersetup{pdfcontactpostcode={8093}}
\hypersetup{pdfcontactcity={Zurich}}
\hypersetup{pdfcontactcountry={Switzerland}}
\hypersetup{pdfcontactemail={nbeisert@itp.phys.ethz.ch}}
\hypersetup{pdfcontacturl={http://people.phys.ethz.ch/\xmptilde nbeisert/}}

\newcommand{\secref}[1]{\hyperref[#1]{section \ref*{#1}}}

\parskip1ex
\parindent0pt
\let\olditemize\itemize
\def\itemize{\olditemize\parskip0pt}

\begin{document}

\title{The \textsf{childdoc} Package}
\hypersetup{pdftitle={The childdoc Package}}
\author{Niklas Beisert\\[2ex]
  Institut f\"ur Theoretische Physik\\
  Eidgen\"ossische Technische Hochschule Z\"urich\\
  Wolfgang-Pauli-Strasse 27, 8093 Z\"urich, Switzerland\\[1ex]
  \href{mailto:nbeisert@itp.phys.ethz.ch}
  {\texttt{nbeisert@itp.phys.ethz.ch}}}
\hypersetup{pdfauthor={Niklas Beisert}}
\hypersetup{pdfsubject={Manual for the LaTeX2e Package childdoc}}
\date{30 December 2018, \textsf{v2.0}}
\maketitle

\begin{abstract}\noindent
\textsf{childdoc} is a \LaTeXe{} package
that enables the direct compilation
of document sections included by |\include|
to individual files.
\end{abstract}

\begingroup
\parskip0ex
\tableofcontents
\endgroup

%%%%%%%%%%%%%%%%%%%%%%%%%%%%%%%%%%%%%%%%%%%%%%%%%%%%%%%%%%%%%%%%%%%%%%%%%%%%%%%%
%%%%%%%%%%%%%%%%%%%%%%%%%%%%%%%%%%%%%%%%%%%%%%%%%%%%%%%%%%%%%%%%%%%%%%%%%%%%%%%%
\section{Introduction}

\LaTeX{} provides a mechanism to structure a large document (such as a book)
into a main file and several child files (containing the chapters)
using the |\include| command.
This mechanism is beneficial for documents
which span hundreds of pages in order to
make the source file(s) more manageable.
Moreover, compilation can be restricted to
selected child files by means of the |\includeonly| command.
The latter feature can be used to reduce the compilation time while editing
(this was significantly more useful in the earlier days of \LaTeX{})
or to generate a smaller document which is easier to navigate.
Another application of |\includeonly| is to generate
documents consisting of selected parts of the complete document.

However, there are a few drawbacks of the plain |\include| mechanism:
\begin{itemize}
\item
The child files cannot be compiled on their own,
they can only be compiled via the main file.
A naive editing environment
(such as a text editor with an option
to have the current file processed by \LaTeX)
may require one to switch to the main file before compiling;
attempting to compile the child file produces errors.
\item
The main file must be modified (each time)
to adjust the |\includeonly| command
to the present needs. This easily leaves the main file in a messy state.
\item
The generated document will always carry the filename
of the main document. This is inconvenient if
several child files are to be compiled and
to be kept for distribution.
\end{itemize}

The present package provides a simple interface
to make child files individually compilable by \LaTeX{}.
Compiling a child file then has the same effect as compiling
the main file with an |\includeonly| command
to select the appropriate child.
Moreover the generated document will carry the name of the child
rather than the main file.
This resolves all three above issues.

This feature is meant to make the editing of books,
thesis documents and lecture notes somewhat more convenient.
However, the package can also be used efficiently for
composing a series of documents (such as exercise sheets)
which are typically distributed individually.
It then assists the author in generating the individual documents
(potentially in different versions)
as well as a document containing the collected series.
Another application is in developing style files
or other kinds of included material
where compilation of the style file could redirect
to a sample or test file.

%%%%%%%%%%%%%%%%%%%%%%%%%%%%%%%%%%%%%%%%%%%%%%%%%%%%%%%%%%%%%%%%%%%%%%%%%%%%%%%%
%%%%%%%%%%%%%%%%%%%%%%%%%%%%%%%%%%%%%%%%%%%%%%%%%%%%%%%%%%%%%%%%%%%%%%%%%%%%%%%%
\section{Usage}

First of all, the package \textsf{childdoc} is \emph{not} a standard
\LaTeXe{} |.sty| style file! Therefore it needs to be invoked in
a non-standard way.

%%%%%%%%%%%%%%%%%%%%%%%%%%%%%%%%%%%%%%%%%%%%%%%%%%%%%%%%%%%%%%%%%%%%%%%%%%%%%%%%
\subsection{Included Files}
\label{sec:include}

%%%%%%%%%%%%%%%%%%%%%%%%%%%%%%%%%%%%%%%%
\DescribeMacro{\childdocmain}
To use the package, add the commands
\begin{center}
\begin{tabular}{l}
|\input{childdoc.def}|\\
|\childdocmain{}|\\
\end{tabular}
\end{center}
at the very top of the main \LaTeX{} file,
in particular \emph{before} the |\documentclass| statement!
The argument of |\childdocmain| should be left empty
(but it must be present).

%%%%%%%%%%%%%%%%%%%%%%%%%%%%%%%%%%%%%%%%
\DescribeMacro{\childdocof}
Furthermore, add the commands
\begin{center}
\begin{tabular}{l}
|\input{childdoc.def}|\\
|\childdocof{|\textit{main}|}|\\
\end{tabular}
\end{center}
at the top of every child file \textit{child}
which is included by |\include{|\textit{child}|}|
from within the main file
(or at least for those files to be compiled individually).
The argument \textit{main} must be the filename of the main file.

There are a couple of
considerations in setting up the main and child documents:

%%%%%%%%%%%%%%%%%%%%%%%%%%%%%%%%%%%%%%%%
\paragraph{Restrictions.}

Please note the following restrictions:
\begin{itemize}
\item
|\childdocmain| must be called with one argument \textit{main}
to ensure compatibility with earlier version of the package.
It must either be empty (|\childdocmain{}|)
or precisely match the filename of the main file in which it is specified.
See \secref{sec:detection} for further information.
\item
The filename \textit{main} must be specified without the |.tex| extension.
\item
The filename \textit{main} is case sensitive
(even in case-insensitive file systems)
due to internal string comparison.
\item
The argument \textit{main} should be fully expanded, it cannot be a macro.
\item
Subdirectories and special characters should be avoided in filenames.
\item
The command |\childdocmain{|\textit{main}|}| must be followed by a whitespace.
It should not be followed immediately by another command
or by a comment mark `|%|'.
This is because the \TeX{} parser reads the token immediately following
the argument of |\childdocmain| and puts it
at the beginning of every child section;
however, a white\-space is ignored.
\end{itemize}

%%%%%%%%%%%%%%%%%%%%%%%%%%%%%%%%%%%%%%%%
\paragraph{Content of Main File.}

It is advisable to place all content in the child files included by |\include|.
Any output contained in the main file will appear in all child documents
unless suppressed manually;
it cannot be suppressed automatically by the |\includeonly| directive
and thus should normally be avoided.
A method to include some content in the main file
by means of conditional processing is described in \secref{sec:conditional}.

%%%%%%%%%%%%%%%%%%%%%%%%%%%%%%%%%%%%%%%%
\paragraph{Page Numbering.}

When only a part of the document is compiled,
the appropriate numbering of pages
(as well as other status parameters)
is determined from the |.aux| files.
The latter contain information from previous passes.
However this information needs to propagate through
all intermediate child documents.
Therefore the page numbering in child documents may well
be inconsistent until the complete document is compiled at least once.

A useful (if unconventional) way to always ensure a consistent
page numbering is to restart the numbering in each child document
and denote the pages by `\textit{child}|.|\textit{page}'
where \textit{child} represents the chapter/section number of the child file.
This can be achieved by the command
|\numberwithin{page}{|\textit{child}|}|
of the \textsf{amsmath} package
where \textit{child} can be |chapter| or |section|
depending on the chosen structuring.
Alternatively, one can modify the macro |\thepage| appropriately
and reset the counter |page| at the start of each child file.

%%%%%%%%%%%%%%%%%%%%%%%%%%%%%%%%%%%%%%%%%%%%%%%%%%%%%%%%%%%%%%%%%%%%%%%%%%%%%%%%
\subsection{Conditional Processing}
\label{sec:conditional}

The package provides a mechanism to compile different versions
of a document. To customise the versions further some conditional processing
can come in handy to distinguish which version is being compiled.
The package provides two macros to describe the compilation context:

%%%%%%%%%%%%%%%%%%%%%%%%%%%%%%%%%%%%%%%%
\DescribeMacro{\ifchilddoc}
The conditional |\ifchilddoc| distinguishes between the compilation of
child documents and the main document:
%
\begin{center}
|\ifchilddoc |\textit{child-code}| |[|\||else |\textit{main-code}]| \||fi|
\end{center}

%%%%%%%%%%%%%%%%%%%%%%%%%%%%%%%%%%%%%%%%
\DescribeMacro{\childdocname}
\DescribeMacro{\childdocjob}
The macro |\childdocname| contains the filename (without extension)
of the main or child file being processed.
Note that |\childdocjob| will always contain the name of the main file.

%%%%%%%%%%%%%%%%%%%%%%%%%%%%%%%%%%%%%%%%
\paragraph{Title Page.}

Conditional processing can be used to include a title or banner page
in the main document when proper precautions are taken.
Importantly, the code in the main file should ensure that the page counter
(as well as other status parameters which are stored in the |.aux| files)
takes the same value after the conditional processing.
Otherwise the page numbers may take divergent values
depending on which part is compiled.

For example, a title page could be declared by:
%
\begin{center}
\begin{tabular}{l}
|\ifchilddoc\||else|\\
|\addtocounter{page}{-1}|\\
\textit{code for title page}\\
|\newpage|\\
|\||fi|
\end{tabular}
\end{center}
%
A banner page for the child documents can be generated by:
%
\begin{center}
\begin{tabular}{l}
|\ifchilddoc|\\
|\addtocounter{page}{-1}|\\
\textit{code for banner page}\\
|\newpage|\\
|\||fi|
\end{tabular}
\end{center}
%
Here one could write a message such as:
\begin{center}
|This is the part \childdocname{} of \childdocjob{}.|
\end{center}

%%%%%%%%%%%%%%%%%%%%%%%%%%%%%%%%%%%%%%%%%%%%%%%%%%%%%%%%%%%%%%%%%%%%%%%%%%%%%%%%
\subsection{Flags}
\label{sec:flags}

The package makes it easy to generate different versions
of the main or child documents.
To this end compilation flags can be defined
and assigned different default values.
They will be particularly useful in conjunction
with the forwarding mechanism described in \secref{sec:forward}.

For example, it may be useful to have a flag |\version|
which can be set to |draft| or |final|.
The document source will contain some conditional code
depending on the value of |\version|.
Suppose further, the flag should default to |final| for the main file
and to |draft| for child files
which is a natural assignment for editing the document.
This is achieved by placing the following code
in the preamble of the main document
(below the |\childdocmain| directive):
%
\begin{center}
\begin{tabular}{l}
|\ifchilddoc|\\
|\providecommand{\version}{draft}|\\
|\||else|\\
|\providecommand{\version}{final}|\\
|\||fi|
\end{tabular}
\end{center}
%
The definition by |\providecommand| makes sure
that previous definitions are not overwritten.
Further statements |\providecommand{\version}{...}|
can thus be added before the above code to override it.

For the main file, one might add a line
(between |\childdocmain| and the above block)
%
\begin{center}
|%\ifchilddoc\||else\providecommand{\version}{draft}\||fi|
\end{center}
%
which can be uncommented to produce a draft version.
Likewise one can add a line to the very top of a child file
(above the |\childdocof{|\textit{main}|}| directive)
%
\begin{center}
|%\providecommand{\version}{final}|
\end{center}
%
which can be uncommented to produce the final version of this child document.

%%%%%%%%%%%%%%%%%%%%%%%%%%%%%%%%%%%%%%%%%%%%%%%%%%%%%%%%%%%%%%%%%%%%%%%%%%%%%%%%
\subsection{Forwarding}
\label{sec:forward}

Different versions of the main or child documents
using compilation flags as described in \secref{sec:flags}
can be (permanently) stored in different files
for convenient compilation, viewing and distribution.
To this end, the package defines a command
to pass on compilation to a different file:

%%%%%%%%%%%%%%%%%%%%%%%%%%%%%%%%%%%%%%%%
\DescribeMacro{\childdocforward}
The command |\childdocforward| redirects processing to
another source file:
%
\begin{center}
\begin{tabular}{l}
|\input{childdoc.def}|\\
|\childdocforward[|\textit{main}|]{|\textit{dest}|}|\\
\end{tabular}
\end{center}
%
The argument \textit{dest} is the destination file
(without extension).
It should be the main file or one of the child files.
Note that further \textsf{childdoc} directives
such as |\childdocof| and |\childdocforward|
in the indicated file will be processed in this form.
The optional argument \textit{main}
passes on directly to the main file \textit{main}
while pretending to compile the child \textit{dest}.
This form behaves as if \textit{dest}
issues |\childdocof{|\textit{main}|}| right away,
and no further \textsf{childdoc} directives will be processed.

%%%%%%%%%%%%%%%%%%%%%%%%%%%%%%%%%%%%%%%%
\DescribeMacro{\...prefix}
In the alternative form |\childdocforwardprefix|,
%
\begin{center}
\begin{tabular}{l}
|\input{childdoc.def}|\\
|\childdocforwardprefix[|\textit{main}|]{|\textit{prefix}|}{|\textit{dest}|}|
\end{tabular}
\end{center}
%
the destination file is determined by a pattern
depending on the current file:
To make this work, the current file must be called
`{\textit{prefix}\hspace{0.2em}\textit{suffix}}'
with \textit{prefix} matching precisely the argument.
Processing is then passed on to the file
`{\textit{dest}\hspace{0.2em}\textit{suffix}}'.
Surely, the same effect is achieved by
directly specifying the
argument `{\textit{dest}\hspace{0.2em}\textit{suffix}}'
in the first form.
However, that requires to set up a different file
for each child. With the alternative form of the command
all these files can have exactly the same content
which simplifies setting them up and maintaining them.

For example, the following file |draft.tex|
with a compilation flag |\version| as described in \secref{sec:flags}
compiles the main document as a draft:
%
\begin{center}
\begin{tabular}{l}
|\def\version{draft}|\\
|\input{childdoc.def}|\\
|\childdocforward{|\textit{main}|}|
\end{tabular}
\end{center}
%
Likewise, the following files |final|\textit{nn}|.tex|
compile the final version of the child document
|child|\textit{nn}|.tex|:
%
\begin{center}
\begin{tabular}{l}
|\def\version{final}|\\
|\input{childdoc.def}|\\
|\childdocforwardprefix{final}{child}|
\end{tabular}
\end{center}
%

Note that when several versions of a main file and/or of each child file
are to be generated, it may be convenient to set up a |Makefile| or
shell script to automatise the process.

%%%%%%%%%%%%%%%%%%%%%%%%%%%%%%%%%%%%%%%%%%%%%%%%%%%%%%%%%%%%%%%%%%%%%%%%%%%%%%%%
\subsection{Command Line Processing}
\label{sec:commandline}

The effect of redirection files can also be achieved by invoking
the \LaTeX{} compiler with a more elaborate command line.
Most conveniently this should be done as part
of a shell script or a |Makefile|.

When using \textsf{childdoc} in the main file, the following
command lines effectively perform a redirection
(note that depending on the shell being used,
backslashes may have to be doubled: `|\|' $\to$ `|\\|'):
%
\begin{center}
|... -jobname "|\textit{target}|" |\\|"|[\textit{flags}]%
|\input{childdoc.def}\childdocforward[|\textit{main}|]{|\textit{dest}|}"|
\end{center}
%
Here \textit{target} is the name of the output file,
\textit{main} is the name of the main file
and \textit{dest} is the name of the main or child file to be processed
(all filenames without extensions).
The optional argument \textit{main} can be omitted
if \textit{main} matches \textit{dest}.
Optionally, compilation \textit{flags} can be defined via |\def| commands.
This command line makes the \TeX{} engine believe
it is compiling the file \textit{target}
whose content is specified as the latter parameter.
The provided code then forwards the processing to
\textit{main} or \textit{dest} as described in \secref{sec:forward}.

%%%%%%%%%%%%%%%%%%%%%%%%%%%%%%%%%%%%%%%%%%%%%%%%%%%%%%%%%%%%%%%%%%%%%%%%%%%%%%%%
\subsection{Include by Input}
\label{sec:input}

Including child documents by |\include| has some restrictions by design.
Most notably, the content of a child document always occupies
its own set of pages; pages cannot be shared between child documents.
Usually, this behaviour makes perfect sense
because each child document contain an essential part of the document.
However, in some situations it may be desirable to compose
a document from a collection of parts
without having mandatory page breaks between then.
For this case, the package
provides a mechanism to include parts
by |\input| which can also be processed individually.
However, by construction this mechanism
requires manual handling of the content to be output.

%%%%%%%%%%%%%%%%%%%%%%%%%%%%%%%%%%%%%%%%
\DescribeMacro{\ifchilddocmanual}
The main file should be prepared as usual, see \secref{sec:include}.
However, the document body must make a distinction
between processing of an individual part and of the main document, e.g.:
%
\begin{center}
\begin{tabular}{l}
|\ifchilddocmanual|\\
|\input{\childdocname}|\\
|\||else|\\
\textit{document body with }|\input{|\textit{part}|}|\\
|\||fi|
\end{tabular}
\end{center}
%
The conditional |\ifchilddocmanual| is true whenever
a part to be included by |\input| is being compiled,
and the name of the part is stored in |\childdocname|.

%%%%%%%%%%%%%%%%%%%%%%%%%%%%%%%%%%%%%%%%
\DescribeMacro{\childdocby}
Each part to be included by |\input| should start with:
%
\begin{center}
\begin{tabular}{l}
|\input{childdoc.def}|\\
|\childdocby{|\textit{main}|}|\\
\end{tabular}
\end{center}
%
The directive |\childdocby| is similar to |\childdocof|
described in \secref{sec:include},
but the subsequent selection of content must be done manually.
To that end, both |\ifchilddoc| and |\ifchilddocmanual|
will be true upon processing of a part,
and the name of the part is stored in |\childdocname|.
Note that |\jobname| will be set to the filename of the current part
so that each part receives an individual |.aux| file
that does not interfere with the |.aux| file(s) of the main document.
This behaviour can be altered by the alternative form
|\childdocby[*]{|\textit{main}|}| (with a non-empty optional argument)
which uses the |.aux| file of the main document
by setting |\jobname| to \textit{main}.

%%%%%%%%%%%%%%%%%%%%%%%%%%%%%%%%%%%%%%%%%%%%%%%%%%%%%%%%%%%%%%%%%%%%%%%%%%%%%%%%
\subsection{Driver Development}
\label{sec:driver}

The \textsf{childdoc} mechanism can also be use for the development
of definition files such as \LaTeX{} styles or classes.
This case differs from the above setup with multiple parts
included by |\include| in that no |\includeonly| should be invoked.
This can be achieved by starting the include file
(before |\ProvidesPackage|) with:
%
\begin{center}
\begin{tabular}{l}
|\input{childdoc.def}|\\
|\childdocforward{|\textit{main}|}|\\
\end{tabular}
\end{center}
%
or alternatively with:
%
\begin{center}
\begin{tabular}{l}
|\input{childdoc.def}|\\
|\childdocby{|\textit{main}|}|\\
\end{tabular}
\end{center}
%
Both forms have slightly different effects as described above.
The main file is prepared as usual, see \secref{sec:include}.

%%%%%%%%%%%%%%%%%%%%%%%%%%%%%%%%%%%%%%%%%%%%%%%%%%%%%%%%%%%%%%%%%%%%%%%%%%%%%%%%
\subsection{Legacy Detection}
\label{sec:detection}

The directive |\childdocmain| in the main file can detect
whether the complete document or merely a child is to be compiled
even without using the directive |\childdocof|.
This method is deprecated because it is less robust
and there is no compelling reason to use it;
it is merely provided for backward compatibility
and it may be removed in future versions.

If the detection mechanism is to be used,
it is mandatory to correctly specify
the filename of the main file as the argument of |\childdocmain|:
%
\begin{center}
\begin{tabular}{l}
|\input{childdoc.def}|\\
|\childdocmain{|\textit{main}|}|\\
\end{tabular}
\end{center}
%
If |\jobname| does not match the argument \textit{main} of |\childdocmain|,
it is assumed that |\jobname| points to the child file to be compiled.
When using |\childdocmain| with the main file specified as argument,
it suffices to start a child file
with just |\input{|\textit{main}|}|
without loading of the package and using |\childdocof|.
If instead all processing is done
with the appropriate \textsf{childdoc} directives,
the argument of \textit{main} of |\childdocmain| can be empty.

An alternative version of the command line processing described
in \secref{sec:commandline} using the detection mechanism reads:
%
\begin{center}
|... -jobname "|\textit{target}|" "|[\textit{flags}]%
[|\def\jobname{|\textit{dest}|}|]|\input{|\textit{main}|}"|
\end{center}

%%%%%%%%%%%%%%%%%%%%%%%%%%%%%%%%%%%%%%%%%%%%%%%%%%%%%%%%%%%%%%%%%%%%%%%%%%%%%%%%
\subsection{Manual Code}
\label{sec:manual}

In case one cannot be certain whether the definitions file |childdoc.def|
is installed on the target \TeX{} distribution
and one prefers not to ship it,
it is conceivable to paste a few relevant commands into the sources.

To that end, drop all statements |\input{childdoc.def}|
and perform the replacements as outlined below.
Instead of |\childdocmain{|\textit{main}|}| add the following code
to the top of the main file:
%
\begin{center}
\begin{tabular}{l}
|\||ifdefined\childdocname\endinput\||fi\newif\ifchilddoc|\\
|\edef\childdocname{\scantokens\expandafter{\jobname\noexpand}}|\\
|\def\childdocmain{|\textit{main}|}\||ifx\childdocmain\childdocname\||else|\\
|\childdoctrue\includeonly{\childdocname}\let\jobname\childdocmain\||fi|\\
\end{tabular}
\end{center}
%
Instead of |\childdocof{|\textit{main}|}| just include the main file
at the top of each child file:
%
\begin{center}
|\input{|\textit{main}|}|
\end{center}
%
A simple redirection |\childdocforward{|\textit{dest}|}| is achieved by:
%
\begin{center}
|\def\jobname{|\textit{dest}|}\input{\jobname}|
\end{center}
%
The redirection with prefix
|\childdocforwardprefix[|\textit{prefix}|]{|\textit{dest}|}|
is accomplished by:
%
\begin{center}
\begin{tabular}{l}
|{\edef\jobname{\scantokens\expandafter{\jobname\noexpand}}|\\
|\def\redirectjob |\textit{prefix}|#1~~~{\gdef\jobname{|\textit{dest}|#1}}|\\
|\expandafter\redirectjob\jobname~~~}\input{\jobname}|
\end{tabular}
\end{center}

In an alternative approach,
child documents can be compiled by a specific command line
without additional code or specific definitions:
%
\begin{center}
|... -jobname "|\textit{target}|" "|[\textit{flags}]%
|\includeonly{|\textit{dest}|}\input{|\textit{main}|}"|
\end{center}
%

%%%%%%%%%%%%%%%%%%%%%%%%%%%%%%%%%%%%%%%%%%%%%%%%%%%%%%%%%%%%%%%%%%%%%%%%%%%%%%%%
%%%%%%%%%%%%%%%%%%%%%%%%%%%%%%%%%%%%%%%%%%%%%%%%%%%%%%%%%%%%%%%%%%%%%%%%%%%%%%%%
\section{Information}

%%%%%%%%%%%%%%%%%%%%%%%%%%%%%%%%%%%%%%%%%%%%%%%%%%%%%%%%%%%%%%%%%%%%%%%%%%%%%%%%
\subsection{Copyright}

Copyright \copyright{} 2017--2018 Niklas Beisert

This work may be distributed and/or modified under the
conditions of the \LaTeX{} Project Public License, either version 1.3
of this license or (at your option) any later version.
The latest version of this license is in
  \url{http://www.latex-project.org/lppl.txt}
and version 1.3 or later is part of all distributions of \LaTeX{}
version 2005/12/01 or later.

This work has the LPPL maintenance status `maintained'.

The Current Maintainer of this work is Niklas Beisert.

This work consists of the files |README.txt|, |childdoc.ins| and |childdoc.dtx|
as well as the derived files |childdoc.def|, |cdocsamp.tex|
with |cdocsch1.tex|, |cdocsch2.tex|, |cdocspt3.tex|, |cdocspt4.tex|,
|cdocsdrf.tex|, |cdocsfn1.tex|, |cdocsfn2.tex|
as well as |childdoc.pdf|.

%%%%%%%%%%%%%%%%%%%%%%%%%%%%%%%%%%%%%%%%%%%%%%%%%%%%%%%%%%%%%%%%%%%%%%%%%%%%%%%%
\subsection{Files and Installation}

The package consists of the files:
%
\begin{center}
\begin{tabular}{ll}
    |README.txt|   & readme file \\
    |childdoc.ins| & installation file \\
    |childdoc.dtx| & source file \\
    |childdoc.def| & definition file \\
    |cdocsamp.tex| & sample main file \\
    |cdocsch1.tex| & sample include file \\
    |cdocsch2.tex| & sample include file \\
    |cdocspt3.tex| & sample part file \\
    |cdocspt4.tex| & sample part file \\
    |cdocsdrf.tex| & sample redirection file \\
    |cdocsfn1.tex| & sample redirection file \\
    |cdocsfn2.tex| & sample redirection file \\
    |childdoc.pdf| & manual
\end{tabular}
\end{center}
%
The distribution consists of the files
|README.txt|, |childdoc.ins| and |childdoc.dtx|.
%
\begin{itemize}
\item
Run (pdf)\LaTeX{} on |childdoc.dtx|
to compile the manual |childdoc.pdf| (this file).
\item
Run \LaTeX{} on |childdoc.ins| to create the definitions file |childdoc.def|
and the sample |cdocsamp.tex| with include files
|cdocsch1.tex|, |cdocsch2.tex|, |cdocspt3.tex|, |cdocspt4.tex|,
|cdocsdrf.tex|, |cdocsfn1.tex|, |cdocsfn2.tex|.
Then copy the file |childdoc.def| to an appropriate directory of your \LaTeX{}
distribution, e.g.\ \textit{texmf-root}|/tex/latex/childdoc|.
\end{itemize}

%%%%%%%%%%%%%%%%%%%%%%%%%%%%%%%%%%%%%%%%%%%%%%%%%%%%%%%%%%%%%%%%%%%%%%%%%%%%%%%%
\subsection{Related CTAN Packages}

There are several other packages which offer a similar functionality:
%
\begin{itemize}
\item
The packages
\href{http://ctan.org/pkg/docmute}{\textsf{docmute}},
\href{http://ctan.org/pkg/includex}{\textsf{includex}} and
\href{http://ctan.org/pkg/standalone}{\textsf{standalone}}
provide commands to include only the document body of
a child file thus allowing both files to be compiled individually.
\item
The packages \href{http://ctan.org/pkg/subdocs}{\textsf{subdocs}}
and \href{http://ctan.org/pkg/subfiles}{\textsf{subfiles}}
provide structures in which the main and child documents can be
encapsulated and allowing them to be compiled individually.
The inclusion mechanism is different from the conventional |\include|.
\item
The package \href{http://ctan.org/pkg/combine}{\textsf{combine}}
is an elaborate solution to combine several documents into one.
\end{itemize}
%
See also the CTAN topic \href{http://ctan.org/topic/subdocs}{\textsf{subdocs}}
for further related packages.
The present package differs from the above solutions in that
a document structure constructed with the conventional |\include| mechanism
just needs two extra commands at the top of every file
such that all constituent files can be compiled individually.

%%%%%%%%%%%%%%%%%%%%%%%%%%%%%%%%%%%%%%%%%%%%%%%%%%%%%%%%%%%%%%%%%%%%%%%%%%%%%%%%
%\subsection{Feature Suggestions}
%
%The following is a list of features which may be useful for future
%versions of this package:
%%
%\begin{itemize}
%\item
%\ldots
%\end{itemize}

%%%%%%%%%%%%%%%%%%%%%%%%%%%%%%%%%%%%%%%%%%%%%%%%%%%%%%%%%%%%%%%%%%%%%%%%%%%%%%%%
\subsection{Revision History}

%%%%%%%%%%%%%%%%%%%%%%%%%%%%%%%%%%%%%%%%
\paragraph{v2.0:} 2018/12/30

\begin{itemize}
\item
immediate forward processing
\item
added |\childdocby| mechanism
\item
manual restructured
\end{itemize}

%%%%%%%%%%%%%%%%%%%%%%%%%%%%%%%%%%%%%%%%
\paragraph{v1.6:} 2018/01/17

\begin{itemize}
\item
application for development of include files
\item
corrections to manual
\end{itemize}

%%%%%%%%%%%%%%%%%%%%%%%%%%%%%%%%%%%%%%%%
\paragraph{v1.5:} 2017/05/21

\begin{itemize}
\item
more complete structuring introduced
\item
|\childdocof| introduced
\item
|\childdoc| renamed to |\childdocmain|
\item
|\childredirect| renamed to |\childdocforward| and |\childdocforwardprefix|
and functionality expanded
\end{itemize}

%%%%%%%%%%%%%%%%%%%%%%%%%%%%%%%%%%%%%%%%
\paragraph{v1.0:} 2017/04/27

\begin{itemize}
\item
manual and install package
\item
first version published on CTAN
\end{itemize}

%%%%%%%%%%%%%%%%%%%%%%%%%%%%%%%%%%%%%%%%
\paragraph{v0.6:} 2017/04/26

\begin{itemize}
\item
redirection mechanism added
\end{itemize}

%%%%%%%%%%%%%%%%%%%%%%%%%%%%%%%%%%%%%%%%
\paragraph{v0.5:} 2017/04/26

\begin{itemize}
\item
functionality in definition file
\end{itemize}


%%%%%%%%%%%%%%%%%%%%%%%%%%%%%%%%%%%%%%%%%%%%%%%%%%%%%%%%%%%%%%%%%%%%%%%%%%%%%%%%
%%%%%%%%%%%%%%%%%%%%%%%%%%%%%%%%%%%%%%%%%%%%%%%%%%%%%%%%%%%%%%%%%%%%%%%%%%%%%%%%
%%%%%%%%%%%%%%%%%%%%%%%%%%%%%%%%%%%%%%%%%%%%%%%%%%%%%%%%%%%%%%%%%%%%%%%%%%%%%%%%
\appendix

\settowidth\MacroIndent{\rmfamily\scriptsize 000\ }

 \DocInput{childdoc.dtx}

\end{document}
%</driver>
% \fi
%
% %%%%%%%%%%%%%%%%%%%%%%%%%%%%%%%%%%%%%%%%%%%%%%%%%%%%%%%%%%%%%%%%%%%%%%%%%%%%%%
% %%%%%%%%%%%%%%%%%%%%%%%%%%%%%%%%%%%%%%%%%%%%%%%%%%%%%%%%%%%%%%%%%%%%%%%%%%%%%%
% \section{Sample}
%\iffalse
%<*samplemain>
%\fi
%
% The following presents a sample document
% with two chapters, two parts, a title page,
% a compile flag as well as three forwarding files to set the flag.
% It consists of eight |.tex| files:
% \begin{center}
% \begin{tabular}{ll}
% |cdocsamp.tex|&main file\\
% |cdocsch1.tex|&include file for chapter 1\\
% |cdocsch2.tex|&include file for chapter 2\\
% |cdocspt3.tex|&include file for part 3\\
% |cdocspt4.tex|&include file for part 4\\
% |cdocsdrf.tex|&forwarding file for main file in draft mode\\
% |cdocsfi1.tex|&forwarding file for final version of chapter 1\\
% |cdocsfi2.tex|&forwarding file for final version of chapter 2\\
% \end{tabular}
% \end{center}
% Each of the eight files can be compiled directly by the \LaTeX{} compiler.
%
% %%%%%%%%%%%%%%%%%%%%%%%%%%%%%%%%%%%%%%
% \paragraph{Main File.}
%
% The main file is called |cdocsamp.tex|.
%
% Load the \textsf{childdoc} definitions and
% declare the filename for the main document:
%    \begin{macrocode}
\input{childdoc.def}
\childdocmain{}
%    \end{macrocode}

% Optional override for |\version| flag:
%    \begin{macrocode}
%%\ifchilddoc\else\providecommand{\version}{draft}\fi
%    \end{macrocode}

% Define the default values for the |\version| flag
% (|final| for the main file and |draft| for childs):
%    \begin{macrocode}
\ifchilddoc
\providecommand{\version}{draft}
\else
\providecommand{\version}{final}
\fi
%    \end{macrocode}

% Load the standard document class:
%    \begin{macrocode}
\documentclass[12pt]{article}
%    \end{macrocode}

% Start the document body:
%    \begin{macrocode}
\begin{document}
%    \end{macrocode}

% Declare a title page.
% Print title, part of document being processed and version flag:
%    \begin{macrocode}
\addtocounter{page}{-1}
\begin{center}
{\LARGE\bfseries{}childdoc example\par}
\vspace{1cm}
\ifchilddoc
\ifchilddocmanual part\else chapter\fi:
`\childdocname' of `\childdocjob'\par
\else
main document: `\childdocjob'\par
\fi
version: \version\par
\end{center}
\newpage
%    \end{macrocode}

% Manually include selected file,
% otherwise process as usual:
%    \begin{macrocode}
\ifchilddocmanual
\section*{part `\childdocname'}
\input{\childdocname}
\else
%    \end{macrocode}

% Include the two chapters:
%    \begin{macrocode}
\include{cdocsch1}
\include{cdocsch2}
%    \end{macrocode}

% Include the two parts unless only chapters should be displayed:
%    \begin{macrocode}
\ifchilddoc\else
\section{part three}
\input{cdocspt3}
\section{part four}
\input{cdocspt4}
\fi
%    \end{macrocode}

% Process as usual until here:
%    \begin{macrocode}
\fi
%    \end{macrocode}

% End of document body:
%    \begin{macrocode}
\end{document}
%    \end{macrocode}
%\iffalse
%</samplemain>
%\fi
%
% %%%%%%%%%%%%%%%%%%%%%%%%%%%%%%%%%%%%%%
% \paragraph{Chapter Include Files.}
%
% The include files are called |cdocsch1.tex| and |cdocsch2.tex|.
%
%\iffalse
%<*samplechap1|samplechap2>
%\fi

% Optional override for |\version| flag:
%    \begin{macrocode}
%%\providecommand{\version}{final}
%    \end{macrocode}

% Include the main document:
%    \begin{macrocode}
\input{childdoc.def}
\childdocof{cdocsamp}
%    \end{macrocode}

%\iffalse
%</samplechap1|samplechap2>
%\fi
%
%\iffalse
%<*samplechap1>
%\fi
% Some text for chapter 1:
%    \begin{macrocode}
\section{one}
some text in chapter one
%    \end{macrocode}

%\iffalse
%</samplechap1>
%\fi
% Some text for chapter 2:
%\iffalse
%<*samplechap2>
%\fi
%    \begin{macrocode}
\section{two}
more text in chapter two
%    \end{macrocode}

%\iffalse
%</samplechap2>
%\fi
%
% %%%%%%%%%%%%%%%%%%%%%%%%%%%%%%%%%%%%%%
% \paragraph{Part Include Files.}
%
% The include files are called |cdocspt3.tex| and |cdocspt4.tex|.
%
%\iffalse
%<*samplepart3|samplepart4>
%\fi

% Optional override for |\version| flag:
%    \begin{macrocode}
%%\providecommand{\version}{final}
%    \end{macrocode}

% Include the main document:
%    \begin{macrocode}
\input{childdoc.def}
\childdocby{cdocsamp}
%    \end{macrocode}

%\iffalse
%</samplepart3|samplepart4>
%\fi
%
%\iffalse
%<*samplepart3>
%\fi
% Some text for part 3:
%    \begin{macrocode}
some text in part three
%    \end{macrocode}

%\iffalse
%</samplepart3>
%\fi
% Some text for part 4:
%\iffalse
%<*samplepart4>
%\fi
%    \begin{macrocode}
more text in part four
%    \end{macrocode}

%\iffalse
%</samplepart4>
%\fi
%
% %%%%%%%%%%%%%%%%%%%%%%%%%%%%%%%%%%%%%%
% \paragraph{Forwarding for a Complete Draft.}
%
% The following forwarding file |cdocsdrf.tex|
% compiles the main document in draft mode:
%\iffalse
%<*sampledraft>
%\fi
%    \begin{macrocode}
\def\version{draft}
\input{childdoc.def}
\childdocforward{cdocsamp}
%    \end{macrocode}

%\iffalse
%</sampledraft>
%\fi
%
% %%%%%%%%%%%%%%%%%%%%%%%%%%%%%%%%%%%%%%
% \paragraph{Forwarding for Final Version of the Chapters.}
%
% The following forwarding files |cdocsfn1.tex| and |cdocsfn2.tex|
% (with identical content)
% compile the final versions of the child documents
% |cdocsch1.tex| and |cdocsch2.tex|, respectively:
%\iffalse
%<*samplefinal>
%\fi
%    \begin{macrocode}
\def\version{final}
\input{childdoc.def}
\childdocforwardprefix[cdocsamp]{cdocsfn}{cdocsch}
%    \end{macrocode}

%\iffalse
%</samplefinal>
%\fi
%
% %%%%%%%%%%%%%%%%%%%%%%%%%%%%%%%%%%%%%%
% \paragraph{Command Line Processing.}
%
% The following three command lines generate the output files
% |cdocscld|, |cdocscl1| and |cdocscl2|
% which should be identical to
% |cdocsdrf|, |cdocsch1| and |cdocsfn2|, respectively:
% \begin{center}
% \begin{tabular}{l}
% |latex -jobname cdocscld \|\\
% |  "\def\version{draft}\input{childdoc.def}\childdocforward{cdocsamp}"|\\
% |latex -jobname cdocscl1 \|\\
% |  "\input{childdoc.def}\childdocforward[cdocsamp]{cdocsch1}"|\\
% |latex -jobname cdocscl2 \|\\
% |  "\def\version{final}\input{childdoc.def}\childdocforward{cdocsch2}"|
% \end{tabular}
% \end{center}
% Note that the trailing backslash on each first line
% merely continues the input to the second line
% (for convenient cut ant paste).
% Furthermore, the command |latex| can be replaced by any
% of its alternative versions such as |pdflatex|.
%
% %%%%%%%%%%%%%%%%%%%%%%%%%%%%%%%%%%%%%%%%%%%%%%%%%%%%%%%%%%%%%%%%%%%%%%%%%%%%%%
% %%%%%%%%%%%%%%%%%%%%%%%%%%%%%%%%%%%%%%%%%%%%%%%%%%%%%%%%%%%%%%%%%%%%%%%%%%%%%%
% \section{Implementation}
%\iffalse
%<*package>
%\fi
%
% This section describes the definitions file |childdoc.def|.

% The definitions cannot be loaded using |\usepackage| or |\RequirePackage|
% which has a mechanism to prevent loading a style file more than once.
% When loading the definitions by means of |\input|
% multiple instances have to be prevented manually:
%\iffalse
%This code needs to be before the `\ProvidesFile' directive
%which is defined at the beginning of this file.
%Therefore it is also placed there and commented out here.
%</package>
%<*discard>
%\fi
%    \begin{macrocode}
\ifdefined\childdocmain\endinput\fi
%    \end{macrocode}
%\iffalse
%</discard>
%<*package>
%\fi
%
% \macro{\ifchilddoc}
% \macro{\ifchilddocmanual}
% The conditional |\ifchilddoc| tells whether a
% child (true) or main (false) document is being compiled.
% The conditional |\ifchilddocmanual| tells whether
% the |\includeonly| mechanism is used (false) or
% the selection of child files must be performed manually (true).
% The definitions initialise to false:
%    \begin{macrocode}
\newif\ifchilddoc
\newif\ifchilddocmanual
%    \end{macrocode}

% \macro{\childdocname}
% \macro{\childdocjob}
% The macro |\childdocname| stores the name of the main document
% to be compiled. The macro |\childdocjob| stores the name of
% the document on which the \LaTeX{} compiler was originally invoked.
% The content of |\jobname| cannot be compared
% to filenames specified in the source due to different catcodes.
% The following code rescans |\jobname|, stores the result
% in |\childdocname| and saves a copy in |\childdocjob|:
%    \begin{macrocode}
\edef\childdocname{\scantokens\expandafter{\jobname\noexpand}}
\let\childdocjob\childdocname
%    \end{macrocode}

% \macro{\childdocdisable}
% The macro |\childdocdisable| prevents the main file
% from being processed more than once.
% At this stage, the main document command |\childdocmain|
% is assumed to be called once again where it should do nothing.
% Any subsequent call to it should prevent
% a secondary processing of the main document
% It overwrites the forwarding commands
% |\childdocof| and |\childdocforward|
% with empty macros to prevent further inclusions of the main document:
%    \begin{macrocode}
\newcommand{\childdocdisable}
{
  \renewcommand{\childdocmain}[1]{\renewcommand{\childdocmain}[1]{\endinput}}
  \renewcommand{\childdocof}[1]{}
  \renewcommand{\childdocby}[2][]{}
  \renewcommand{\childdocforward}[2][]{}
  \renewcommand{\childdocdisable}{}
}
%    \end{macrocode}

% \macro{\childdocmain}
% The macro |\childdocmain| is to be called at the top of the main file
% with nothing or the main filename (without extension) as argument.
% First, it breaks loops.
% If the argument is not empty and does not match |\childdocname|
% (which is set by the first inclusion of |childdoc.def|),
% |\ifchilddoc| is set to true, |\includeonly| is applied to the child file
% and |\jobname| is set to the main file
% (for proper handling of |.aux| files):
%    \begin{macrocode}
\newcommand{\childdocmain}[1]
{
  \childdocdisable\childdocmain{}
  \if?#1?\else
    \begingroup
      \def\childdoctmp{#1}
      \ifx\childdoctmp\childdocname
        \def\childdoctmp{}
      \else
        \def\childdoctmp
        {
          \childdoctrue
          \includeonly{\childdocname}
          \def\childdocjob{#1}
          \def\jobname{#1}
        }
      \fi
      \expandafter
    \endgroup
    \childdoctmp
  \fi
}
%    \end{macrocode}

% \macro{\childdocof}
% The command |\childdocof| redirects
% compilation to the main file |#1|.
%    \begin{macrocode}
\newcommand{\childdocof}[1]
{
  \childdocdisable
  \childdoctrue
  \includeonly{\childdocname}
  \def\jobname{#1}
  \def\childdocjob{#1}
  \input{#1}
}
%    \end{macrocode}

% \macro{\childdocby}
% The command |\childdocby| ....
%    \begin{macrocode}
\newcommand{\childdocby}[2][]
{
  \childdocdisable
  \childdoctrue
  \childdocmanualtrue
  \if?#1?\else
    \def\jobname{#2}
  \fi
  \def\childdocjob{#2}
  \input{#2}
  \endinput
}
%    \end{macrocode}

% \macro{\childdocforward}
% The command |\childdocforward| redirects
% compilation to the main file or
% (if the optional argument is given) a child file.
% Parameters are set as if the main file
% or a child file starting with |\childdocof| was compiled.
% Then compilation is handed over to the main file:
%    \begin{macrocode}
\newcommand{\childdocforward}[2][]
{
  \begingroup
    \if?#1?
      \def\childdoctmp
      {
        \def\childdocname{#2}
        \def\childdocjob{#2}
        \def\jobname{#2}
        \input{#2}
        \endinput
      }
    \else
      \def\childdoctmp
      {
        \childdocdisable
        \def\childdocname{#2}
        \childdoctrue
        \includeonly{#2}
        \def\childdocjob{#1}
        \def\jobname{#1}
        \input{#1}
        \endinput
      }
    \fi
    \expandafter
  \endgroup
  \childdoctmp
}
%    \end{macrocode}

% \macro{\childdocforwardprefix}
% The command |\childdocforwardprefix| redirects
% compilation to the main or a child file by means of a pattern.
% The prefix |#1| in the current filename is replaced by |#2|
% and the suffix of the current filename is kept
% (it is assumed that the filename does not contain the substring `|~~~|'
% which is used as a delimiter).
% Compilation is handed over to the new file by |\childdocforward|:
%    \begin{macrocode}
\newcommand{\childdocforwardprefix}[3][]
{
  \begingroup
    \def\childdocextract #2##1~~~{\def\childdoctmp{\childdocforward[#1]{#3##1}}}
    \expandafter\childdocextract\childdocname~~~
    \expandafter
  \endgroup
  \childdoctmp
}
%    \end{macrocode}

% \macro{\childdoc}
% The deprecated macro |\childdoc| is a legacy version of |\childdocmain|:
%    \begin{macrocode}
\newcommand{\childdoc}{\childdocmain}
%    \end{macrocode}

% \macro{\childdocredirect}
% The deprecated macro |\childdocredirect| is a legacy version
% of |\childdocforward| and |\childdocforwardprefix|:
%    \begin{macrocode}
\newcommand{\childdocredirect}[2][]
{
  \begingroup
    \if?#1?
      \def\childdoctmp{\childdocforward{#2}}
    \else
      \def\childdoctmp{\childdocforwardprefix{#1}{#2}}
    \fi
    \expandafter
  \endgroup
  \childdoctmp
}
%    \end{macrocode}

%\iffalse
%</package>
%\fi
%
\endinput

\childdocmain{}
%    \end{macrocode}

% Optional override for |\version| flag:
%    \begin{macrocode}
%%\ifchilddoc\else\providecommand{\version}{draft}\fi
%    \end{macrocode}

% Define the default values for the |\version| flag
% (|final| for the main file and |draft| for childs):
%    \begin{macrocode}
\ifchilddoc
\providecommand{\version}{draft}
\else
\providecommand{\version}{final}
\fi
%    \end{macrocode}

% Load the standard document class:
%    \begin{macrocode}
\documentclass[12pt]{article}
%    \end{macrocode}

% Start the document body:
%    \begin{macrocode}
\begin{document}
%    \end{macrocode}

% Declare a title page.
% Print title, part of document being processed and version flag:
%    \begin{macrocode}
\addtocounter{page}{-1}
\begin{center}
{\LARGE\bfseries{}childdoc example\par}
\vspace{1cm}
\ifchilddoc
\ifchilddocmanual part\else chapter\fi:
`\childdocname' of `\childdocjob'\par
\else
main document: `\childdocjob'\par
\fi
version: \version\par
\end{center}
\newpage
%    \end{macrocode}

% Manually include selected file,
% otherwise process as usual:
%    \begin{macrocode}
\ifchilddocmanual
\section*{part `\childdocname'}
\input{\childdocname}
\else
%    \end{macrocode}

% Include the two chapters:
%    \begin{macrocode}
\include{cdocsch1}
\include{cdocsch2}
%    \end{macrocode}

% Include the two parts unless only chapters should be displayed:
%    \begin{macrocode}
\ifchilddoc\else
\section{part three}
\input{cdocspt3}
\section{part four}
\input{cdocspt4}
\fi
%    \end{macrocode}

% Process as usual until here:
%    \begin{macrocode}
\fi
%    \end{macrocode}

% End of document body:
%    \begin{macrocode}
\end{document}
%    \end{macrocode}
%\iffalse
%</samplemain>
%\fi
%
% %%%%%%%%%%%%%%%%%%%%%%%%%%%%%%%%%%%%%%
% \paragraph{Chapter Include Files.}
%
% The include files are called |cdocsch1.tex| and |cdocsch2.tex|.
%
%\iffalse
%<*samplechap1|samplechap2>
%\fi

% Optional override for |\version| flag:
%    \begin{macrocode}
%%\providecommand{\version}{final}
%    \end{macrocode}

% Include the main document:
%    \begin{macrocode}
% \iffalse
%
% childdoc.dtx Copyright (C) 2017-2018 Niklas Beisert
%
% This work may be distributed and/or modified under the
% conditions of the LaTeX Project Public License, either version 1.3
% of this license or (at your option) any later version.
% The latest version of this license is in
%   http://www.latex-project.org/lppl.txt
% and version 1.3 or later is part of all distributions of LaTeX
% version 2005/12/01 or later.
%
% This work has the LPPL maintenance status `maintained'.
%
% The Current Maintainer of this work is Niklas Beisert.
%
% This work consists of the files childdoc.dtx and childdoc.ins
% and the derived files childdoc.def and cdocsamp.tex with
% cdocsch1.tex, cdocsch2.tex, cdocsdrf.tex, cdocsfn1.tex, cdocsfn2.tex.
%
%<package>\ifdefined\childdocmain\endinput\fi
%<package>\ProvidesFile{childdoc.def}[2018/12/30 v2.0 child document driver]
%<samplemain>\ProvidesFile{cdocsamp.tex}[2018/12/30 v2.0 sample for childdoc]
%<*driver>
%\ProvidesFile{childdoc.drv}[2018/12/30 v2.0 childdoc reference manual file]
\PassOptionsToClass{10pt,a4paper}{article}
\documentclass{ltxdoc}

\usepackage[margin=35mm]{geometry}
\usepackage{hyperref}
\usepackage{hyperxmp}
\usepackage[usenames]{color}

\hypersetup{colorlinks=true}
\hypersetup{pdfstartview=FitH}
\hypersetup{pdfpagemode=UseNone}
\hypersetup{pdfsource={}}
\hypersetup{pdflang={en-UK}}
\hypersetup{pdfcopyright={Copyright 2017-2018 Niklas Beisert.
  This work may be distributed and/or modified under the
  conditions of the LaTeX Project Public License, either version 1.3
  of this license or (at your option) any later version.}}
\hypersetup{pdflicenseurl={http://www.latex-project.org/lppl.txt}}
\hypersetup{pdfcontactaddress={ETH Zurich, ITP, HIT K,
  Wolfgang-Pauli-Strasse 27}}
\hypersetup{pdfcontactpostcode={8093}}
\hypersetup{pdfcontactcity={Zurich}}
\hypersetup{pdfcontactcountry={Switzerland}}
\hypersetup{pdfcontactemail={nbeisert@itp.phys.ethz.ch}}
\hypersetup{pdfcontacturl={http://people.phys.ethz.ch/\xmptilde nbeisert/}}

\newcommand{\secref}[1]{\hyperref[#1]{section \ref*{#1}}}

\parskip1ex
\parindent0pt
\let\olditemize\itemize
\def\itemize{\olditemize\parskip0pt}

\begin{document}

\title{The \textsf{childdoc} Package}
\hypersetup{pdftitle={The childdoc Package}}
\author{Niklas Beisert\\[2ex]
  Institut f\"ur Theoretische Physik\\
  Eidgen\"ossische Technische Hochschule Z\"urich\\
  Wolfgang-Pauli-Strasse 27, 8093 Z\"urich, Switzerland\\[1ex]
  \href{mailto:nbeisert@itp.phys.ethz.ch}
  {\texttt{nbeisert@itp.phys.ethz.ch}}}
\hypersetup{pdfauthor={Niklas Beisert}}
\hypersetup{pdfsubject={Manual for the LaTeX2e Package childdoc}}
\date{30 December 2018, \textsf{v2.0}}
\maketitle

\begin{abstract}\noindent
\textsf{childdoc} is a \LaTeXe{} package
that enables the direct compilation
of document sections included by |\include|
to individual files.
\end{abstract}

\begingroup
\parskip0ex
\tableofcontents
\endgroup

%%%%%%%%%%%%%%%%%%%%%%%%%%%%%%%%%%%%%%%%%%%%%%%%%%%%%%%%%%%%%%%%%%%%%%%%%%%%%%%%
%%%%%%%%%%%%%%%%%%%%%%%%%%%%%%%%%%%%%%%%%%%%%%%%%%%%%%%%%%%%%%%%%%%%%%%%%%%%%%%%
\section{Introduction}

\LaTeX{} provides a mechanism to structure a large document (such as a book)
into a main file and several child files (containing the chapters)
using the |\include| command.
This mechanism is beneficial for documents
which span hundreds of pages in order to
make the source file(s) more manageable.
Moreover, compilation can be restricted to
selected child files by means of the |\includeonly| command.
The latter feature can be used to reduce the compilation time while editing
(this was significantly more useful in the earlier days of \LaTeX{})
or to generate a smaller document which is easier to navigate.
Another application of |\includeonly| is to generate
documents consisting of selected parts of the complete document.

However, there are a few drawbacks of the plain |\include| mechanism:
\begin{itemize}
\item
The child files cannot be compiled on their own,
they can only be compiled via the main file.
A naive editing environment
(such as a text editor with an option
to have the current file processed by \LaTeX)
may require one to switch to the main file before compiling;
attempting to compile the child file produces errors.
\item
The main file must be modified (each time)
to adjust the |\includeonly| command
to the present needs. This easily leaves the main file in a messy state.
\item
The generated document will always carry the filename
of the main document. This is inconvenient if
several child files are to be compiled and
to be kept for distribution.
\end{itemize}

The present package provides a simple interface
to make child files individually compilable by \LaTeX{}.
Compiling a child file then has the same effect as compiling
the main file with an |\includeonly| command
to select the appropriate child.
Moreover the generated document will carry the name of the child
rather than the main file.
This resolves all three above issues.

This feature is meant to make the editing of books,
thesis documents and lecture notes somewhat more convenient.
However, the package can also be used efficiently for
composing a series of documents (such as exercise sheets)
which are typically distributed individually.
It then assists the author in generating the individual documents
(potentially in different versions)
as well as a document containing the collected series.
Another application is in developing style files
or other kinds of included material
where compilation of the style file could redirect
to a sample or test file.

%%%%%%%%%%%%%%%%%%%%%%%%%%%%%%%%%%%%%%%%%%%%%%%%%%%%%%%%%%%%%%%%%%%%%%%%%%%%%%%%
%%%%%%%%%%%%%%%%%%%%%%%%%%%%%%%%%%%%%%%%%%%%%%%%%%%%%%%%%%%%%%%%%%%%%%%%%%%%%%%%
\section{Usage}

First of all, the package \textsf{childdoc} is \emph{not} a standard
\LaTeXe{} |.sty| style file! Therefore it needs to be invoked in
a non-standard way.

%%%%%%%%%%%%%%%%%%%%%%%%%%%%%%%%%%%%%%%%%%%%%%%%%%%%%%%%%%%%%%%%%%%%%%%%%%%%%%%%
\subsection{Included Files}
\label{sec:include}

%%%%%%%%%%%%%%%%%%%%%%%%%%%%%%%%%%%%%%%%
\DescribeMacro{\childdocmain}
To use the package, add the commands
\begin{center}
\begin{tabular}{l}
|\input{childdoc.def}|\\
|\childdocmain{}|\\
\end{tabular}
\end{center}
at the very top of the main \LaTeX{} file,
in particular \emph{before} the |\documentclass| statement!
The argument of |\childdocmain| should be left empty
(but it must be present).

%%%%%%%%%%%%%%%%%%%%%%%%%%%%%%%%%%%%%%%%
\DescribeMacro{\childdocof}
Furthermore, add the commands
\begin{center}
\begin{tabular}{l}
|\input{childdoc.def}|\\
|\childdocof{|\textit{main}|}|\\
\end{tabular}
\end{center}
at the top of every child file \textit{child}
which is included by |\include{|\textit{child}|}|
from within the main file
(or at least for those files to be compiled individually).
The argument \textit{main} must be the filename of the main file.

There are a couple of
considerations in setting up the main and child documents:

%%%%%%%%%%%%%%%%%%%%%%%%%%%%%%%%%%%%%%%%
\paragraph{Restrictions.}

Please note the following restrictions:
\begin{itemize}
\item
|\childdocmain| must be called with one argument \textit{main}
to ensure compatibility with earlier version of the package.
It must either be empty (|\childdocmain{}|)
or precisely match the filename of the main file in which it is specified.
See \secref{sec:detection} for further information.
\item
The filename \textit{main} must be specified without the |.tex| extension.
\item
The filename \textit{main} is case sensitive
(even in case-insensitive file systems)
due to internal string comparison.
\item
The argument \textit{main} should be fully expanded, it cannot be a macro.
\item
Subdirectories and special characters should be avoided in filenames.
\item
The command |\childdocmain{|\textit{main}|}| must be followed by a whitespace.
It should not be followed immediately by another command
or by a comment mark `|%|'.
This is because the \TeX{} parser reads the token immediately following
the argument of |\childdocmain| and puts it
at the beginning of every child section;
however, a white\-space is ignored.
\end{itemize}

%%%%%%%%%%%%%%%%%%%%%%%%%%%%%%%%%%%%%%%%
\paragraph{Content of Main File.}

It is advisable to place all content in the child files included by |\include|.
Any output contained in the main file will appear in all child documents
unless suppressed manually;
it cannot be suppressed automatically by the |\includeonly| directive
and thus should normally be avoided.
A method to include some content in the main file
by means of conditional processing is described in \secref{sec:conditional}.

%%%%%%%%%%%%%%%%%%%%%%%%%%%%%%%%%%%%%%%%
\paragraph{Page Numbering.}

When only a part of the document is compiled,
the appropriate numbering of pages
(as well as other status parameters)
is determined from the |.aux| files.
The latter contain information from previous passes.
However this information needs to propagate through
all intermediate child documents.
Therefore the page numbering in child documents may well
be inconsistent until the complete document is compiled at least once.

A useful (if unconventional) way to always ensure a consistent
page numbering is to restart the numbering in each child document
and denote the pages by `\textit{child}|.|\textit{page}'
where \textit{child} represents the chapter/section number of the child file.
This can be achieved by the command
|\numberwithin{page}{|\textit{child}|}|
of the \textsf{amsmath} package
where \textit{child} can be |chapter| or |section|
depending on the chosen structuring.
Alternatively, one can modify the macro |\thepage| appropriately
and reset the counter |page| at the start of each child file.

%%%%%%%%%%%%%%%%%%%%%%%%%%%%%%%%%%%%%%%%%%%%%%%%%%%%%%%%%%%%%%%%%%%%%%%%%%%%%%%%
\subsection{Conditional Processing}
\label{sec:conditional}

The package provides a mechanism to compile different versions
of a document. To customise the versions further some conditional processing
can come in handy to distinguish which version is being compiled.
The package provides two macros to describe the compilation context:

%%%%%%%%%%%%%%%%%%%%%%%%%%%%%%%%%%%%%%%%
\DescribeMacro{\ifchilddoc}
The conditional |\ifchilddoc| distinguishes between the compilation of
child documents and the main document:
%
\begin{center}
|\ifchilddoc |\textit{child-code}| |[|\||else |\textit{main-code}]| \||fi|
\end{center}

%%%%%%%%%%%%%%%%%%%%%%%%%%%%%%%%%%%%%%%%
\DescribeMacro{\childdocname}
\DescribeMacro{\childdocjob}
The macro |\childdocname| contains the filename (without extension)
of the main or child file being processed.
Note that |\childdocjob| will always contain the name of the main file.

%%%%%%%%%%%%%%%%%%%%%%%%%%%%%%%%%%%%%%%%
\paragraph{Title Page.}

Conditional processing can be used to include a title or banner page
in the main document when proper precautions are taken.
Importantly, the code in the main file should ensure that the page counter
(as well as other status parameters which are stored in the |.aux| files)
takes the same value after the conditional processing.
Otherwise the page numbers may take divergent values
depending on which part is compiled.

For example, a title page could be declared by:
%
\begin{center}
\begin{tabular}{l}
|\ifchilddoc\||else|\\
|\addtocounter{page}{-1}|\\
\textit{code for title page}\\
|\newpage|\\
|\||fi|
\end{tabular}
\end{center}
%
A banner page for the child documents can be generated by:
%
\begin{center}
\begin{tabular}{l}
|\ifchilddoc|\\
|\addtocounter{page}{-1}|\\
\textit{code for banner page}\\
|\newpage|\\
|\||fi|
\end{tabular}
\end{center}
%
Here one could write a message such as:
\begin{center}
|This is the part \childdocname{} of \childdocjob{}.|
\end{center}

%%%%%%%%%%%%%%%%%%%%%%%%%%%%%%%%%%%%%%%%%%%%%%%%%%%%%%%%%%%%%%%%%%%%%%%%%%%%%%%%
\subsection{Flags}
\label{sec:flags}

The package makes it easy to generate different versions
of the main or child documents.
To this end compilation flags can be defined
and assigned different default values.
They will be particularly useful in conjunction
with the forwarding mechanism described in \secref{sec:forward}.

For example, it may be useful to have a flag |\version|
which can be set to |draft| or |final|.
The document source will contain some conditional code
depending on the value of |\version|.
Suppose further, the flag should default to |final| for the main file
and to |draft| for child files
which is a natural assignment for editing the document.
This is achieved by placing the following code
in the preamble of the main document
(below the |\childdocmain| directive):
%
\begin{center}
\begin{tabular}{l}
|\ifchilddoc|\\
|\providecommand{\version}{draft}|\\
|\||else|\\
|\providecommand{\version}{final}|\\
|\||fi|
\end{tabular}
\end{center}
%
The definition by |\providecommand| makes sure
that previous definitions are not overwritten.
Further statements |\providecommand{\version}{...}|
can thus be added before the above code to override it.

For the main file, one might add a line
(between |\childdocmain| and the above block)
%
\begin{center}
|%\ifchilddoc\||else\providecommand{\version}{draft}\||fi|
\end{center}
%
which can be uncommented to produce a draft version.
Likewise one can add a line to the very top of a child file
(above the |\childdocof{|\textit{main}|}| directive)
%
\begin{center}
|%\providecommand{\version}{final}|
\end{center}
%
which can be uncommented to produce the final version of this child document.

%%%%%%%%%%%%%%%%%%%%%%%%%%%%%%%%%%%%%%%%%%%%%%%%%%%%%%%%%%%%%%%%%%%%%%%%%%%%%%%%
\subsection{Forwarding}
\label{sec:forward}

Different versions of the main or child documents
using compilation flags as described in \secref{sec:flags}
can be (permanently) stored in different files
for convenient compilation, viewing and distribution.
To this end, the package defines a command
to pass on compilation to a different file:

%%%%%%%%%%%%%%%%%%%%%%%%%%%%%%%%%%%%%%%%
\DescribeMacro{\childdocforward}
The command |\childdocforward| redirects processing to
another source file:
%
\begin{center}
\begin{tabular}{l}
|\input{childdoc.def}|\\
|\childdocforward[|\textit{main}|]{|\textit{dest}|}|\\
\end{tabular}
\end{center}
%
The argument \textit{dest} is the destination file
(without extension).
It should be the main file or one of the child files.
Note that further \textsf{childdoc} directives
such as |\childdocof| and |\childdocforward|
in the indicated file will be processed in this form.
The optional argument \textit{main}
passes on directly to the main file \textit{main}
while pretending to compile the child \textit{dest}.
This form behaves as if \textit{dest}
issues |\childdocof{|\textit{main}|}| right away,
and no further \textsf{childdoc} directives will be processed.

%%%%%%%%%%%%%%%%%%%%%%%%%%%%%%%%%%%%%%%%
\DescribeMacro{\...prefix}
In the alternative form |\childdocforwardprefix|,
%
\begin{center}
\begin{tabular}{l}
|\input{childdoc.def}|\\
|\childdocforwardprefix[|\textit{main}|]{|\textit{prefix}|}{|\textit{dest}|}|
\end{tabular}
\end{center}
%
the destination file is determined by a pattern
depending on the current file:
To make this work, the current file must be called
`{\textit{prefix}\hspace{0.2em}\textit{suffix}}'
with \textit{prefix} matching precisely the argument.
Processing is then passed on to the file
`{\textit{dest}\hspace{0.2em}\textit{suffix}}'.
Surely, the same effect is achieved by
directly specifying the
argument `{\textit{dest}\hspace{0.2em}\textit{suffix}}'
in the first form.
However, that requires to set up a different file
for each child. With the alternative form of the command
all these files can have exactly the same content
which simplifies setting them up and maintaining them.

For example, the following file |draft.tex|
with a compilation flag |\version| as described in \secref{sec:flags}
compiles the main document as a draft:
%
\begin{center}
\begin{tabular}{l}
|\def\version{draft}|\\
|\input{childdoc.def}|\\
|\childdocforward{|\textit{main}|}|
\end{tabular}
\end{center}
%
Likewise, the following files |final|\textit{nn}|.tex|
compile the final version of the child document
|child|\textit{nn}|.tex|:
%
\begin{center}
\begin{tabular}{l}
|\def\version{final}|\\
|\input{childdoc.def}|\\
|\childdocforwardprefix{final}{child}|
\end{tabular}
\end{center}
%

Note that when several versions of a main file and/or of each child file
are to be generated, it may be convenient to set up a |Makefile| or
shell script to automatise the process.

%%%%%%%%%%%%%%%%%%%%%%%%%%%%%%%%%%%%%%%%%%%%%%%%%%%%%%%%%%%%%%%%%%%%%%%%%%%%%%%%
\subsection{Command Line Processing}
\label{sec:commandline}

The effect of redirection files can also be achieved by invoking
the \LaTeX{} compiler with a more elaborate command line.
Most conveniently this should be done as part
of a shell script or a |Makefile|.

When using \textsf{childdoc} in the main file, the following
command lines effectively perform a redirection
(note that depending on the shell being used,
backslashes may have to be doubled: `|\|' $\to$ `|\\|'):
%
\begin{center}
|... -jobname "|\textit{target}|" |\\|"|[\textit{flags}]%
|\input{childdoc.def}\childdocforward[|\textit{main}|]{|\textit{dest}|}"|
\end{center}
%
Here \textit{target} is the name of the output file,
\textit{main} is the name of the main file
and \textit{dest} is the name of the main or child file to be processed
(all filenames without extensions).
The optional argument \textit{main} can be omitted
if \textit{main} matches \textit{dest}.
Optionally, compilation \textit{flags} can be defined via |\def| commands.
This command line makes the \TeX{} engine believe
it is compiling the file \textit{target}
whose content is specified as the latter parameter.
The provided code then forwards the processing to
\textit{main} or \textit{dest} as described in \secref{sec:forward}.

%%%%%%%%%%%%%%%%%%%%%%%%%%%%%%%%%%%%%%%%%%%%%%%%%%%%%%%%%%%%%%%%%%%%%%%%%%%%%%%%
\subsection{Include by Input}
\label{sec:input}

Including child documents by |\include| has some restrictions by design.
Most notably, the content of a child document always occupies
its own set of pages; pages cannot be shared between child documents.
Usually, this behaviour makes perfect sense
because each child document contain an essential part of the document.
However, in some situations it may be desirable to compose
a document from a collection of parts
without having mandatory page breaks between then.
For this case, the package
provides a mechanism to include parts
by |\input| which can also be processed individually.
However, by construction this mechanism
requires manual handling of the content to be output.

%%%%%%%%%%%%%%%%%%%%%%%%%%%%%%%%%%%%%%%%
\DescribeMacro{\ifchilddocmanual}
The main file should be prepared as usual, see \secref{sec:include}.
However, the document body must make a distinction
between processing of an individual part and of the main document, e.g.:
%
\begin{center}
\begin{tabular}{l}
|\ifchilddocmanual|\\
|\input{\childdocname}|\\
|\||else|\\
\textit{document body with }|\input{|\textit{part}|}|\\
|\||fi|
\end{tabular}
\end{center}
%
The conditional |\ifchilddocmanual| is true whenever
a part to be included by |\input| is being compiled,
and the name of the part is stored in |\childdocname|.

%%%%%%%%%%%%%%%%%%%%%%%%%%%%%%%%%%%%%%%%
\DescribeMacro{\childdocby}
Each part to be included by |\input| should start with:
%
\begin{center}
\begin{tabular}{l}
|\input{childdoc.def}|\\
|\childdocby{|\textit{main}|}|\\
\end{tabular}
\end{center}
%
The directive |\childdocby| is similar to |\childdocof|
described in \secref{sec:include},
but the subsequent selection of content must be done manually.
To that end, both |\ifchilddoc| and |\ifchilddocmanual|
will be true upon processing of a part,
and the name of the part is stored in |\childdocname|.
Note that |\jobname| will be set to the filename of the current part
so that each part receives an individual |.aux| file
that does not interfere with the |.aux| file(s) of the main document.
This behaviour can be altered by the alternative form
|\childdocby[*]{|\textit{main}|}| (with a non-empty optional argument)
which uses the |.aux| file of the main document
by setting |\jobname| to \textit{main}.

%%%%%%%%%%%%%%%%%%%%%%%%%%%%%%%%%%%%%%%%%%%%%%%%%%%%%%%%%%%%%%%%%%%%%%%%%%%%%%%%
\subsection{Driver Development}
\label{sec:driver}

The \textsf{childdoc} mechanism can also be use for the development
of definition files such as \LaTeX{} styles or classes.
This case differs from the above setup with multiple parts
included by |\include| in that no |\includeonly| should be invoked.
This can be achieved by starting the include file
(before |\ProvidesPackage|) with:
%
\begin{center}
\begin{tabular}{l}
|\input{childdoc.def}|\\
|\childdocforward{|\textit{main}|}|\\
\end{tabular}
\end{center}
%
or alternatively with:
%
\begin{center}
\begin{tabular}{l}
|\input{childdoc.def}|\\
|\childdocby{|\textit{main}|}|\\
\end{tabular}
\end{center}
%
Both forms have slightly different effects as described above.
The main file is prepared as usual, see \secref{sec:include}.

%%%%%%%%%%%%%%%%%%%%%%%%%%%%%%%%%%%%%%%%%%%%%%%%%%%%%%%%%%%%%%%%%%%%%%%%%%%%%%%%
\subsection{Legacy Detection}
\label{sec:detection}

The directive |\childdocmain| in the main file can detect
whether the complete document or merely a child is to be compiled
even without using the directive |\childdocof|.
This method is deprecated because it is less robust
and there is no compelling reason to use it;
it is merely provided for backward compatibility
and it may be removed in future versions.

If the detection mechanism is to be used,
it is mandatory to correctly specify
the filename of the main file as the argument of |\childdocmain|:
%
\begin{center}
\begin{tabular}{l}
|\input{childdoc.def}|\\
|\childdocmain{|\textit{main}|}|\\
\end{tabular}
\end{center}
%
If |\jobname| does not match the argument \textit{main} of |\childdocmain|,
it is assumed that |\jobname| points to the child file to be compiled.
When using |\childdocmain| with the main file specified as argument,
it suffices to start a child file
with just |\input{|\textit{main}|}|
without loading of the package and using |\childdocof|.
If instead all processing is done
with the appropriate \textsf{childdoc} directives,
the argument of \textit{main} of |\childdocmain| can be empty.

An alternative version of the command line processing described
in \secref{sec:commandline} using the detection mechanism reads:
%
\begin{center}
|... -jobname "|\textit{target}|" "|[\textit{flags}]%
[|\def\jobname{|\textit{dest}|}|]|\input{|\textit{main}|}"|
\end{center}

%%%%%%%%%%%%%%%%%%%%%%%%%%%%%%%%%%%%%%%%%%%%%%%%%%%%%%%%%%%%%%%%%%%%%%%%%%%%%%%%
\subsection{Manual Code}
\label{sec:manual}

In case one cannot be certain whether the definitions file |childdoc.def|
is installed on the target \TeX{} distribution
and one prefers not to ship it,
it is conceivable to paste a few relevant commands into the sources.

To that end, drop all statements |\input{childdoc.def}|
and perform the replacements as outlined below.
Instead of |\childdocmain{|\textit{main}|}| add the following code
to the top of the main file:
%
\begin{center}
\begin{tabular}{l}
|\||ifdefined\childdocname\endinput\||fi\newif\ifchilddoc|\\
|\edef\childdocname{\scantokens\expandafter{\jobname\noexpand}}|\\
|\def\childdocmain{|\textit{main}|}\||ifx\childdocmain\childdocname\||else|\\
|\childdoctrue\includeonly{\childdocname}\let\jobname\childdocmain\||fi|\\
\end{tabular}
\end{center}
%
Instead of |\childdocof{|\textit{main}|}| just include the main file
at the top of each child file:
%
\begin{center}
|\input{|\textit{main}|}|
\end{center}
%
A simple redirection |\childdocforward{|\textit{dest}|}| is achieved by:
%
\begin{center}
|\def\jobname{|\textit{dest}|}\input{\jobname}|
\end{center}
%
The redirection with prefix
|\childdocforwardprefix[|\textit{prefix}|]{|\textit{dest}|}|
is accomplished by:
%
\begin{center}
\begin{tabular}{l}
|{\edef\jobname{\scantokens\expandafter{\jobname\noexpand}}|\\
|\def\redirectjob |\textit{prefix}|#1~~~{\gdef\jobname{|\textit{dest}|#1}}|\\
|\expandafter\redirectjob\jobname~~~}\input{\jobname}|
\end{tabular}
\end{center}

In an alternative approach,
child documents can be compiled by a specific command line
without additional code or specific definitions:
%
\begin{center}
|... -jobname "|\textit{target}|" "|[\textit{flags}]%
|\includeonly{|\textit{dest}|}\input{|\textit{main}|}"|
\end{center}
%

%%%%%%%%%%%%%%%%%%%%%%%%%%%%%%%%%%%%%%%%%%%%%%%%%%%%%%%%%%%%%%%%%%%%%%%%%%%%%%%%
%%%%%%%%%%%%%%%%%%%%%%%%%%%%%%%%%%%%%%%%%%%%%%%%%%%%%%%%%%%%%%%%%%%%%%%%%%%%%%%%
\section{Information}

%%%%%%%%%%%%%%%%%%%%%%%%%%%%%%%%%%%%%%%%%%%%%%%%%%%%%%%%%%%%%%%%%%%%%%%%%%%%%%%%
\subsection{Copyright}

Copyright \copyright{} 2017--2018 Niklas Beisert

This work may be distributed and/or modified under the
conditions of the \LaTeX{} Project Public License, either version 1.3
of this license or (at your option) any later version.
The latest version of this license is in
  \url{http://www.latex-project.org/lppl.txt}
and version 1.3 or later is part of all distributions of \LaTeX{}
version 2005/12/01 or later.

This work has the LPPL maintenance status `maintained'.

The Current Maintainer of this work is Niklas Beisert.

This work consists of the files |README.txt|, |childdoc.ins| and |childdoc.dtx|
as well as the derived files |childdoc.def|, |cdocsamp.tex|
with |cdocsch1.tex|, |cdocsch2.tex|, |cdocspt3.tex|, |cdocspt4.tex|,
|cdocsdrf.tex|, |cdocsfn1.tex|, |cdocsfn2.tex|
as well as |childdoc.pdf|.

%%%%%%%%%%%%%%%%%%%%%%%%%%%%%%%%%%%%%%%%%%%%%%%%%%%%%%%%%%%%%%%%%%%%%%%%%%%%%%%%
\subsection{Files and Installation}

The package consists of the files:
%
\begin{center}
\begin{tabular}{ll}
    |README.txt|   & readme file \\
    |childdoc.ins| & installation file \\
    |childdoc.dtx| & source file \\
    |childdoc.def| & definition file \\
    |cdocsamp.tex| & sample main file \\
    |cdocsch1.tex| & sample include file \\
    |cdocsch2.tex| & sample include file \\
    |cdocspt3.tex| & sample part file \\
    |cdocspt4.tex| & sample part file \\
    |cdocsdrf.tex| & sample redirection file \\
    |cdocsfn1.tex| & sample redirection file \\
    |cdocsfn2.tex| & sample redirection file \\
    |childdoc.pdf| & manual
\end{tabular}
\end{center}
%
The distribution consists of the files
|README.txt|, |childdoc.ins| and |childdoc.dtx|.
%
\begin{itemize}
\item
Run (pdf)\LaTeX{} on |childdoc.dtx|
to compile the manual |childdoc.pdf| (this file).
\item
Run \LaTeX{} on |childdoc.ins| to create the definitions file |childdoc.def|
and the sample |cdocsamp.tex| with include files
|cdocsch1.tex|, |cdocsch2.tex|, |cdocspt3.tex|, |cdocspt4.tex|,
|cdocsdrf.tex|, |cdocsfn1.tex|, |cdocsfn2.tex|.
Then copy the file |childdoc.def| to an appropriate directory of your \LaTeX{}
distribution, e.g.\ \textit{texmf-root}|/tex/latex/childdoc|.
\end{itemize}

%%%%%%%%%%%%%%%%%%%%%%%%%%%%%%%%%%%%%%%%%%%%%%%%%%%%%%%%%%%%%%%%%%%%%%%%%%%%%%%%
\subsection{Related CTAN Packages}

There are several other packages which offer a similar functionality:
%
\begin{itemize}
\item
The packages
\href{http://ctan.org/pkg/docmute}{\textsf{docmute}},
\href{http://ctan.org/pkg/includex}{\textsf{includex}} and
\href{http://ctan.org/pkg/standalone}{\textsf{standalone}}
provide commands to include only the document body of
a child file thus allowing both files to be compiled individually.
\item
The packages \href{http://ctan.org/pkg/subdocs}{\textsf{subdocs}}
and \href{http://ctan.org/pkg/subfiles}{\textsf{subfiles}}
provide structures in which the main and child documents can be
encapsulated and allowing them to be compiled individually.
The inclusion mechanism is different from the conventional |\include|.
\item
The package \href{http://ctan.org/pkg/combine}{\textsf{combine}}
is an elaborate solution to combine several documents into one.
\end{itemize}
%
See also the CTAN topic \href{http://ctan.org/topic/subdocs}{\textsf{subdocs}}
for further related packages.
The present package differs from the above solutions in that
a document structure constructed with the conventional |\include| mechanism
just needs two extra commands at the top of every file
such that all constituent files can be compiled individually.

%%%%%%%%%%%%%%%%%%%%%%%%%%%%%%%%%%%%%%%%%%%%%%%%%%%%%%%%%%%%%%%%%%%%%%%%%%%%%%%%
%\subsection{Feature Suggestions}
%
%The following is a list of features which may be useful for future
%versions of this package:
%%
%\begin{itemize}
%\item
%\ldots
%\end{itemize}

%%%%%%%%%%%%%%%%%%%%%%%%%%%%%%%%%%%%%%%%%%%%%%%%%%%%%%%%%%%%%%%%%%%%%%%%%%%%%%%%
\subsection{Revision History}

%%%%%%%%%%%%%%%%%%%%%%%%%%%%%%%%%%%%%%%%
\paragraph{v2.0:} 2018/12/30

\begin{itemize}
\item
immediate forward processing
\item
added |\childdocby| mechanism
\item
manual restructured
\end{itemize}

%%%%%%%%%%%%%%%%%%%%%%%%%%%%%%%%%%%%%%%%
\paragraph{v1.6:} 2018/01/17

\begin{itemize}
\item
application for development of include files
\item
corrections to manual
\end{itemize}

%%%%%%%%%%%%%%%%%%%%%%%%%%%%%%%%%%%%%%%%
\paragraph{v1.5:} 2017/05/21

\begin{itemize}
\item
more complete structuring introduced
\item
|\childdocof| introduced
\item
|\childdoc| renamed to |\childdocmain|
\item
|\childredirect| renamed to |\childdocforward| and |\childdocforwardprefix|
and functionality expanded
\end{itemize}

%%%%%%%%%%%%%%%%%%%%%%%%%%%%%%%%%%%%%%%%
\paragraph{v1.0:} 2017/04/27

\begin{itemize}
\item
manual and install package
\item
first version published on CTAN
\end{itemize}

%%%%%%%%%%%%%%%%%%%%%%%%%%%%%%%%%%%%%%%%
\paragraph{v0.6:} 2017/04/26

\begin{itemize}
\item
redirection mechanism added
\end{itemize}

%%%%%%%%%%%%%%%%%%%%%%%%%%%%%%%%%%%%%%%%
\paragraph{v0.5:} 2017/04/26

\begin{itemize}
\item
functionality in definition file
\end{itemize}


%%%%%%%%%%%%%%%%%%%%%%%%%%%%%%%%%%%%%%%%%%%%%%%%%%%%%%%%%%%%%%%%%%%%%%%%%%%%%%%%
%%%%%%%%%%%%%%%%%%%%%%%%%%%%%%%%%%%%%%%%%%%%%%%%%%%%%%%%%%%%%%%%%%%%%%%%%%%%%%%%
%%%%%%%%%%%%%%%%%%%%%%%%%%%%%%%%%%%%%%%%%%%%%%%%%%%%%%%%%%%%%%%%%%%%%%%%%%%%%%%%
\appendix

\settowidth\MacroIndent{\rmfamily\scriptsize 000\ }

 \DocInput{childdoc.dtx}

\end{document}
%</driver>
% \fi
%
% %%%%%%%%%%%%%%%%%%%%%%%%%%%%%%%%%%%%%%%%%%%%%%%%%%%%%%%%%%%%%%%%%%%%%%%%%%%%%%
% %%%%%%%%%%%%%%%%%%%%%%%%%%%%%%%%%%%%%%%%%%%%%%%%%%%%%%%%%%%%%%%%%%%%%%%%%%%%%%
% \section{Sample}
%\iffalse
%<*samplemain>
%\fi
%
% The following presents a sample document
% with two chapters, two parts, a title page,
% a compile flag as well as three forwarding files to set the flag.
% It consists of eight |.tex| files:
% \begin{center}
% \begin{tabular}{ll}
% |cdocsamp.tex|&main file\\
% |cdocsch1.tex|&include file for chapter 1\\
% |cdocsch2.tex|&include file for chapter 2\\
% |cdocspt3.tex|&include file for part 3\\
% |cdocspt4.tex|&include file for part 4\\
% |cdocsdrf.tex|&forwarding file for main file in draft mode\\
% |cdocsfi1.tex|&forwarding file for final version of chapter 1\\
% |cdocsfi2.tex|&forwarding file for final version of chapter 2\\
% \end{tabular}
% \end{center}
% Each of the eight files can be compiled directly by the \LaTeX{} compiler.
%
% %%%%%%%%%%%%%%%%%%%%%%%%%%%%%%%%%%%%%%
% \paragraph{Main File.}
%
% The main file is called |cdocsamp.tex|.
%
% Load the \textsf{childdoc} definitions and
% declare the filename for the main document:
%    \begin{macrocode}
\input{childdoc.def}
\childdocmain{}
%    \end{macrocode}

% Optional override for |\version| flag:
%    \begin{macrocode}
%%\ifchilddoc\else\providecommand{\version}{draft}\fi
%    \end{macrocode}

% Define the default values for the |\version| flag
% (|final| for the main file and |draft| for childs):
%    \begin{macrocode}
\ifchilddoc
\providecommand{\version}{draft}
\else
\providecommand{\version}{final}
\fi
%    \end{macrocode}

% Load the standard document class:
%    \begin{macrocode}
\documentclass[12pt]{article}
%    \end{macrocode}

% Start the document body:
%    \begin{macrocode}
\begin{document}
%    \end{macrocode}

% Declare a title page.
% Print title, part of document being processed and version flag:
%    \begin{macrocode}
\addtocounter{page}{-1}
\begin{center}
{\LARGE\bfseries{}childdoc example\par}
\vspace{1cm}
\ifchilddoc
\ifchilddocmanual part\else chapter\fi:
`\childdocname' of `\childdocjob'\par
\else
main document: `\childdocjob'\par
\fi
version: \version\par
\end{center}
\newpage
%    \end{macrocode}

% Manually include selected file,
% otherwise process as usual:
%    \begin{macrocode}
\ifchilddocmanual
\section*{part `\childdocname'}
\input{\childdocname}
\else
%    \end{macrocode}

% Include the two chapters:
%    \begin{macrocode}
\include{cdocsch1}
\include{cdocsch2}
%    \end{macrocode}

% Include the two parts unless only chapters should be displayed:
%    \begin{macrocode}
\ifchilddoc\else
\section{part three}
\input{cdocspt3}
\section{part four}
\input{cdocspt4}
\fi
%    \end{macrocode}

% Process as usual until here:
%    \begin{macrocode}
\fi
%    \end{macrocode}

% End of document body:
%    \begin{macrocode}
\end{document}
%    \end{macrocode}
%\iffalse
%</samplemain>
%\fi
%
% %%%%%%%%%%%%%%%%%%%%%%%%%%%%%%%%%%%%%%
% \paragraph{Chapter Include Files.}
%
% The include files are called |cdocsch1.tex| and |cdocsch2.tex|.
%
%\iffalse
%<*samplechap1|samplechap2>
%\fi

% Optional override for |\version| flag:
%    \begin{macrocode}
%%\providecommand{\version}{final}
%    \end{macrocode}

% Include the main document:
%    \begin{macrocode}
\input{childdoc.def}
\childdocof{cdocsamp}
%    \end{macrocode}

%\iffalse
%</samplechap1|samplechap2>
%\fi
%
%\iffalse
%<*samplechap1>
%\fi
% Some text for chapter 1:
%    \begin{macrocode}
\section{one}
some text in chapter one
%    \end{macrocode}

%\iffalse
%</samplechap1>
%\fi
% Some text for chapter 2:
%\iffalse
%<*samplechap2>
%\fi
%    \begin{macrocode}
\section{two}
more text in chapter two
%    \end{macrocode}

%\iffalse
%</samplechap2>
%\fi
%
% %%%%%%%%%%%%%%%%%%%%%%%%%%%%%%%%%%%%%%
% \paragraph{Part Include Files.}
%
% The include files are called |cdocspt3.tex| and |cdocspt4.tex|.
%
%\iffalse
%<*samplepart3|samplepart4>
%\fi

% Optional override for |\version| flag:
%    \begin{macrocode}
%%\providecommand{\version}{final}
%    \end{macrocode}

% Include the main document:
%    \begin{macrocode}
\input{childdoc.def}
\childdocby{cdocsamp}
%    \end{macrocode}

%\iffalse
%</samplepart3|samplepart4>
%\fi
%
%\iffalse
%<*samplepart3>
%\fi
% Some text for part 3:
%    \begin{macrocode}
some text in part three
%    \end{macrocode}

%\iffalse
%</samplepart3>
%\fi
% Some text for part 4:
%\iffalse
%<*samplepart4>
%\fi
%    \begin{macrocode}
more text in part four
%    \end{macrocode}

%\iffalse
%</samplepart4>
%\fi
%
% %%%%%%%%%%%%%%%%%%%%%%%%%%%%%%%%%%%%%%
% \paragraph{Forwarding for a Complete Draft.}
%
% The following forwarding file |cdocsdrf.tex|
% compiles the main document in draft mode:
%\iffalse
%<*sampledraft>
%\fi
%    \begin{macrocode}
\def\version{draft}
\input{childdoc.def}
\childdocforward{cdocsamp}
%    \end{macrocode}

%\iffalse
%</sampledraft>
%\fi
%
% %%%%%%%%%%%%%%%%%%%%%%%%%%%%%%%%%%%%%%
% \paragraph{Forwarding for Final Version of the Chapters.}
%
% The following forwarding files |cdocsfn1.tex| and |cdocsfn2.tex|
% (with identical content)
% compile the final versions of the child documents
% |cdocsch1.tex| and |cdocsch2.tex|, respectively:
%\iffalse
%<*samplefinal>
%\fi
%    \begin{macrocode}
\def\version{final}
\input{childdoc.def}
\childdocforwardprefix[cdocsamp]{cdocsfn}{cdocsch}
%    \end{macrocode}

%\iffalse
%</samplefinal>
%\fi
%
% %%%%%%%%%%%%%%%%%%%%%%%%%%%%%%%%%%%%%%
% \paragraph{Command Line Processing.}
%
% The following three command lines generate the output files
% |cdocscld|, |cdocscl1| and |cdocscl2|
% which should be identical to
% |cdocsdrf|, |cdocsch1| and |cdocsfn2|, respectively:
% \begin{center}
% \begin{tabular}{l}
% |latex -jobname cdocscld \|\\
% |  "\def\version{draft}\input{childdoc.def}\childdocforward{cdocsamp}"|\\
% |latex -jobname cdocscl1 \|\\
% |  "\input{childdoc.def}\childdocforward[cdocsamp]{cdocsch1}"|\\
% |latex -jobname cdocscl2 \|\\
% |  "\def\version{final}\input{childdoc.def}\childdocforward{cdocsch2}"|
% \end{tabular}
% \end{center}
% Note that the trailing backslash on each first line
% merely continues the input to the second line
% (for convenient cut ant paste).
% Furthermore, the command |latex| can be replaced by any
% of its alternative versions such as |pdflatex|.
%
% %%%%%%%%%%%%%%%%%%%%%%%%%%%%%%%%%%%%%%%%%%%%%%%%%%%%%%%%%%%%%%%%%%%%%%%%%%%%%%
% %%%%%%%%%%%%%%%%%%%%%%%%%%%%%%%%%%%%%%%%%%%%%%%%%%%%%%%%%%%%%%%%%%%%%%%%%%%%%%
% \section{Implementation}
%\iffalse
%<*package>
%\fi
%
% This section describes the definitions file |childdoc.def|.

% The definitions cannot be loaded using |\usepackage| or |\RequirePackage|
% which has a mechanism to prevent loading a style file more than once.
% When loading the definitions by means of |\input|
% multiple instances have to be prevented manually:
%\iffalse
%This code needs to be before the `\ProvidesFile' directive
%which is defined at the beginning of this file.
%Therefore it is also placed there and commented out here.
%</package>
%<*discard>
%\fi
%    \begin{macrocode}
\ifdefined\childdocmain\endinput\fi
%    \end{macrocode}
%\iffalse
%</discard>
%<*package>
%\fi
%
% \macro{\ifchilddoc}
% \macro{\ifchilddocmanual}
% The conditional |\ifchilddoc| tells whether a
% child (true) or main (false) document is being compiled.
% The conditional |\ifchilddocmanual| tells whether
% the |\includeonly| mechanism is used (false) or
% the selection of child files must be performed manually (true).
% The definitions initialise to false:
%    \begin{macrocode}
\newif\ifchilddoc
\newif\ifchilddocmanual
%    \end{macrocode}

% \macro{\childdocname}
% \macro{\childdocjob}
% The macro |\childdocname| stores the name of the main document
% to be compiled. The macro |\childdocjob| stores the name of
% the document on which the \LaTeX{} compiler was originally invoked.
% The content of |\jobname| cannot be compared
% to filenames specified in the source due to different catcodes.
% The following code rescans |\jobname|, stores the result
% in |\childdocname| and saves a copy in |\childdocjob|:
%    \begin{macrocode}
\edef\childdocname{\scantokens\expandafter{\jobname\noexpand}}
\let\childdocjob\childdocname
%    \end{macrocode}

% \macro{\childdocdisable}
% The macro |\childdocdisable| prevents the main file
% from being processed more than once.
% At this stage, the main document command |\childdocmain|
% is assumed to be called once again where it should do nothing.
% Any subsequent call to it should prevent
% a secondary processing of the main document
% It overwrites the forwarding commands
% |\childdocof| and |\childdocforward|
% with empty macros to prevent further inclusions of the main document:
%    \begin{macrocode}
\newcommand{\childdocdisable}
{
  \renewcommand{\childdocmain}[1]{\renewcommand{\childdocmain}[1]{\endinput}}
  \renewcommand{\childdocof}[1]{}
  \renewcommand{\childdocby}[2][]{}
  \renewcommand{\childdocforward}[2][]{}
  \renewcommand{\childdocdisable}{}
}
%    \end{macrocode}

% \macro{\childdocmain}
% The macro |\childdocmain| is to be called at the top of the main file
% with nothing or the main filename (without extension) as argument.
% First, it breaks loops.
% If the argument is not empty and does not match |\childdocname|
% (which is set by the first inclusion of |childdoc.def|),
% |\ifchilddoc| is set to true, |\includeonly| is applied to the child file
% and |\jobname| is set to the main file
% (for proper handling of |.aux| files):
%    \begin{macrocode}
\newcommand{\childdocmain}[1]
{
  \childdocdisable\childdocmain{}
  \if?#1?\else
    \begingroup
      \def\childdoctmp{#1}
      \ifx\childdoctmp\childdocname
        \def\childdoctmp{}
      \else
        \def\childdoctmp
        {
          \childdoctrue
          \includeonly{\childdocname}
          \def\childdocjob{#1}
          \def\jobname{#1}
        }
      \fi
      \expandafter
    \endgroup
    \childdoctmp
  \fi
}
%    \end{macrocode}

% \macro{\childdocof}
% The command |\childdocof| redirects
% compilation to the main file |#1|.
%    \begin{macrocode}
\newcommand{\childdocof}[1]
{
  \childdocdisable
  \childdoctrue
  \includeonly{\childdocname}
  \def\jobname{#1}
  \def\childdocjob{#1}
  \input{#1}
}
%    \end{macrocode}

% \macro{\childdocby}
% The command |\childdocby| ....
%    \begin{macrocode}
\newcommand{\childdocby}[2][]
{
  \childdocdisable
  \childdoctrue
  \childdocmanualtrue
  \if?#1?\else
    \def\jobname{#2}
  \fi
  \def\childdocjob{#2}
  \input{#2}
  \endinput
}
%    \end{macrocode}

% \macro{\childdocforward}
% The command |\childdocforward| redirects
% compilation to the main file or
% (if the optional argument is given) a child file.
% Parameters are set as if the main file
% or a child file starting with |\childdocof| was compiled.
% Then compilation is handed over to the main file:
%    \begin{macrocode}
\newcommand{\childdocforward}[2][]
{
  \begingroup
    \if?#1?
      \def\childdoctmp
      {
        \def\childdocname{#2}
        \def\childdocjob{#2}
        \def\jobname{#2}
        \input{#2}
        \endinput
      }
    \else
      \def\childdoctmp
      {
        \childdocdisable
        \def\childdocname{#2}
        \childdoctrue
        \includeonly{#2}
        \def\childdocjob{#1}
        \def\jobname{#1}
        \input{#1}
        \endinput
      }
    \fi
    \expandafter
  \endgroup
  \childdoctmp
}
%    \end{macrocode}

% \macro{\childdocforwardprefix}
% The command |\childdocforwardprefix| redirects
% compilation to the main or a child file by means of a pattern.
% The prefix |#1| in the current filename is replaced by |#2|
% and the suffix of the current filename is kept
% (it is assumed that the filename does not contain the substring `|~~~|'
% which is used as a delimiter).
% Compilation is handed over to the new file by |\childdocforward|:
%    \begin{macrocode}
\newcommand{\childdocforwardprefix}[3][]
{
  \begingroup
    \def\childdocextract #2##1~~~{\def\childdoctmp{\childdocforward[#1]{#3##1}}}
    \expandafter\childdocextract\childdocname~~~
    \expandafter
  \endgroup
  \childdoctmp
}
%    \end{macrocode}

% \macro{\childdoc}
% The deprecated macro |\childdoc| is a legacy version of |\childdocmain|:
%    \begin{macrocode}
\newcommand{\childdoc}{\childdocmain}
%    \end{macrocode}

% \macro{\childdocredirect}
% The deprecated macro |\childdocredirect| is a legacy version
% of |\childdocforward| and |\childdocforwardprefix|:
%    \begin{macrocode}
\newcommand{\childdocredirect}[2][]
{
  \begingroup
    \if?#1?
      \def\childdoctmp{\childdocforward{#2}}
    \else
      \def\childdoctmp{\childdocforwardprefix{#1}{#2}}
    \fi
    \expandafter
  \endgroup
  \childdoctmp
}
%    \end{macrocode}

%\iffalse
%</package>
%\fi
%
\endinput

\childdocof{cdocsamp}
%    \end{macrocode}

%\iffalse
%</samplechap1|samplechap2>
%\fi
%
%\iffalse
%<*samplechap1>
%\fi
% Some text for chapter 1:
%    \begin{macrocode}
\section{one}
some text in chapter one
%    \end{macrocode}

%\iffalse
%</samplechap1>
%\fi
% Some text for chapter 2:
%\iffalse
%<*samplechap2>
%\fi
%    \begin{macrocode}
\section{two}
more text in chapter two
%    \end{macrocode}

%\iffalse
%</samplechap2>
%\fi
%
% %%%%%%%%%%%%%%%%%%%%%%%%%%%%%%%%%%%%%%
% \paragraph{Part Include Files.}
%
% The include files are called |cdocspt3.tex| and |cdocspt4.tex|.
%
%\iffalse
%<*samplepart3|samplepart4>
%\fi

% Optional override for |\version| flag:
%    \begin{macrocode}
%%\providecommand{\version}{final}
%    \end{macrocode}

% Include the main document:
%    \begin{macrocode}
% \iffalse
%
% childdoc.dtx Copyright (C) 2017-2018 Niklas Beisert
%
% This work may be distributed and/or modified under the
% conditions of the LaTeX Project Public License, either version 1.3
% of this license or (at your option) any later version.
% The latest version of this license is in
%   http://www.latex-project.org/lppl.txt
% and version 1.3 or later is part of all distributions of LaTeX
% version 2005/12/01 or later.
%
% This work has the LPPL maintenance status `maintained'.
%
% The Current Maintainer of this work is Niklas Beisert.
%
% This work consists of the files childdoc.dtx and childdoc.ins
% and the derived files childdoc.def and cdocsamp.tex with
% cdocsch1.tex, cdocsch2.tex, cdocsdrf.tex, cdocsfn1.tex, cdocsfn2.tex.
%
%<package>\ifdefined\childdocmain\endinput\fi
%<package>\ProvidesFile{childdoc.def}[2018/12/30 v2.0 child document driver]
%<samplemain>\ProvidesFile{cdocsamp.tex}[2018/12/30 v2.0 sample for childdoc]
%<*driver>
%\ProvidesFile{childdoc.drv}[2018/12/30 v2.0 childdoc reference manual file]
\PassOptionsToClass{10pt,a4paper}{article}
\documentclass{ltxdoc}

\usepackage[margin=35mm]{geometry}
\usepackage{hyperref}
\usepackage{hyperxmp}
\usepackage[usenames]{color}

\hypersetup{colorlinks=true}
\hypersetup{pdfstartview=FitH}
\hypersetup{pdfpagemode=UseNone}
\hypersetup{pdfsource={}}
\hypersetup{pdflang={en-UK}}
\hypersetup{pdfcopyright={Copyright 2017-2018 Niklas Beisert.
  This work may be distributed and/or modified under the
  conditions of the LaTeX Project Public License, either version 1.3
  of this license or (at your option) any later version.}}
\hypersetup{pdflicenseurl={http://www.latex-project.org/lppl.txt}}
\hypersetup{pdfcontactaddress={ETH Zurich, ITP, HIT K,
  Wolfgang-Pauli-Strasse 27}}
\hypersetup{pdfcontactpostcode={8093}}
\hypersetup{pdfcontactcity={Zurich}}
\hypersetup{pdfcontactcountry={Switzerland}}
\hypersetup{pdfcontactemail={nbeisert@itp.phys.ethz.ch}}
\hypersetup{pdfcontacturl={http://people.phys.ethz.ch/\xmptilde nbeisert/}}

\newcommand{\secref}[1]{\hyperref[#1]{section \ref*{#1}}}

\parskip1ex
\parindent0pt
\let\olditemize\itemize
\def\itemize{\olditemize\parskip0pt}

\begin{document}

\title{The \textsf{childdoc} Package}
\hypersetup{pdftitle={The childdoc Package}}
\author{Niklas Beisert\\[2ex]
  Institut f\"ur Theoretische Physik\\
  Eidgen\"ossische Technische Hochschule Z\"urich\\
  Wolfgang-Pauli-Strasse 27, 8093 Z\"urich, Switzerland\\[1ex]
  \href{mailto:nbeisert@itp.phys.ethz.ch}
  {\texttt{nbeisert@itp.phys.ethz.ch}}}
\hypersetup{pdfauthor={Niklas Beisert}}
\hypersetup{pdfsubject={Manual for the LaTeX2e Package childdoc}}
\date{30 December 2018, \textsf{v2.0}}
\maketitle

\begin{abstract}\noindent
\textsf{childdoc} is a \LaTeXe{} package
that enables the direct compilation
of document sections included by |\include|
to individual files.
\end{abstract}

\begingroup
\parskip0ex
\tableofcontents
\endgroup

%%%%%%%%%%%%%%%%%%%%%%%%%%%%%%%%%%%%%%%%%%%%%%%%%%%%%%%%%%%%%%%%%%%%%%%%%%%%%%%%
%%%%%%%%%%%%%%%%%%%%%%%%%%%%%%%%%%%%%%%%%%%%%%%%%%%%%%%%%%%%%%%%%%%%%%%%%%%%%%%%
\section{Introduction}

\LaTeX{} provides a mechanism to structure a large document (such as a book)
into a main file and several child files (containing the chapters)
using the |\include| command.
This mechanism is beneficial for documents
which span hundreds of pages in order to
make the source file(s) more manageable.
Moreover, compilation can be restricted to
selected child files by means of the |\includeonly| command.
The latter feature can be used to reduce the compilation time while editing
(this was significantly more useful in the earlier days of \LaTeX{})
or to generate a smaller document which is easier to navigate.
Another application of |\includeonly| is to generate
documents consisting of selected parts of the complete document.

However, there are a few drawbacks of the plain |\include| mechanism:
\begin{itemize}
\item
The child files cannot be compiled on their own,
they can only be compiled via the main file.
A naive editing environment
(such as a text editor with an option
to have the current file processed by \LaTeX)
may require one to switch to the main file before compiling;
attempting to compile the child file produces errors.
\item
The main file must be modified (each time)
to adjust the |\includeonly| command
to the present needs. This easily leaves the main file in a messy state.
\item
The generated document will always carry the filename
of the main document. This is inconvenient if
several child files are to be compiled and
to be kept for distribution.
\end{itemize}

The present package provides a simple interface
to make child files individually compilable by \LaTeX{}.
Compiling a child file then has the same effect as compiling
the main file with an |\includeonly| command
to select the appropriate child.
Moreover the generated document will carry the name of the child
rather than the main file.
This resolves all three above issues.

This feature is meant to make the editing of books,
thesis documents and lecture notes somewhat more convenient.
However, the package can also be used efficiently for
composing a series of documents (such as exercise sheets)
which are typically distributed individually.
It then assists the author in generating the individual documents
(potentially in different versions)
as well as a document containing the collected series.
Another application is in developing style files
or other kinds of included material
where compilation of the style file could redirect
to a sample or test file.

%%%%%%%%%%%%%%%%%%%%%%%%%%%%%%%%%%%%%%%%%%%%%%%%%%%%%%%%%%%%%%%%%%%%%%%%%%%%%%%%
%%%%%%%%%%%%%%%%%%%%%%%%%%%%%%%%%%%%%%%%%%%%%%%%%%%%%%%%%%%%%%%%%%%%%%%%%%%%%%%%
\section{Usage}

First of all, the package \textsf{childdoc} is \emph{not} a standard
\LaTeXe{} |.sty| style file! Therefore it needs to be invoked in
a non-standard way.

%%%%%%%%%%%%%%%%%%%%%%%%%%%%%%%%%%%%%%%%%%%%%%%%%%%%%%%%%%%%%%%%%%%%%%%%%%%%%%%%
\subsection{Included Files}
\label{sec:include}

%%%%%%%%%%%%%%%%%%%%%%%%%%%%%%%%%%%%%%%%
\DescribeMacro{\childdocmain}
To use the package, add the commands
\begin{center}
\begin{tabular}{l}
|\input{childdoc.def}|\\
|\childdocmain{}|\\
\end{tabular}
\end{center}
at the very top of the main \LaTeX{} file,
in particular \emph{before} the |\documentclass| statement!
The argument of |\childdocmain| should be left empty
(but it must be present).

%%%%%%%%%%%%%%%%%%%%%%%%%%%%%%%%%%%%%%%%
\DescribeMacro{\childdocof}
Furthermore, add the commands
\begin{center}
\begin{tabular}{l}
|\input{childdoc.def}|\\
|\childdocof{|\textit{main}|}|\\
\end{tabular}
\end{center}
at the top of every child file \textit{child}
which is included by |\include{|\textit{child}|}|
from within the main file
(or at least for those files to be compiled individually).
The argument \textit{main} must be the filename of the main file.

There are a couple of
considerations in setting up the main and child documents:

%%%%%%%%%%%%%%%%%%%%%%%%%%%%%%%%%%%%%%%%
\paragraph{Restrictions.}

Please note the following restrictions:
\begin{itemize}
\item
|\childdocmain| must be called with one argument \textit{main}
to ensure compatibility with earlier version of the package.
It must either be empty (|\childdocmain{}|)
or precisely match the filename of the main file in which it is specified.
See \secref{sec:detection} for further information.
\item
The filename \textit{main} must be specified without the |.tex| extension.
\item
The filename \textit{main} is case sensitive
(even in case-insensitive file systems)
due to internal string comparison.
\item
The argument \textit{main} should be fully expanded, it cannot be a macro.
\item
Subdirectories and special characters should be avoided in filenames.
\item
The command |\childdocmain{|\textit{main}|}| must be followed by a whitespace.
It should not be followed immediately by another command
or by a comment mark `|%|'.
This is because the \TeX{} parser reads the token immediately following
the argument of |\childdocmain| and puts it
at the beginning of every child section;
however, a white\-space is ignored.
\end{itemize}

%%%%%%%%%%%%%%%%%%%%%%%%%%%%%%%%%%%%%%%%
\paragraph{Content of Main File.}

It is advisable to place all content in the child files included by |\include|.
Any output contained in the main file will appear in all child documents
unless suppressed manually;
it cannot be suppressed automatically by the |\includeonly| directive
and thus should normally be avoided.
A method to include some content in the main file
by means of conditional processing is described in \secref{sec:conditional}.

%%%%%%%%%%%%%%%%%%%%%%%%%%%%%%%%%%%%%%%%
\paragraph{Page Numbering.}

When only a part of the document is compiled,
the appropriate numbering of pages
(as well as other status parameters)
is determined from the |.aux| files.
The latter contain information from previous passes.
However this information needs to propagate through
all intermediate child documents.
Therefore the page numbering in child documents may well
be inconsistent until the complete document is compiled at least once.

A useful (if unconventional) way to always ensure a consistent
page numbering is to restart the numbering in each child document
and denote the pages by `\textit{child}|.|\textit{page}'
where \textit{child} represents the chapter/section number of the child file.
This can be achieved by the command
|\numberwithin{page}{|\textit{child}|}|
of the \textsf{amsmath} package
where \textit{child} can be |chapter| or |section|
depending on the chosen structuring.
Alternatively, one can modify the macro |\thepage| appropriately
and reset the counter |page| at the start of each child file.

%%%%%%%%%%%%%%%%%%%%%%%%%%%%%%%%%%%%%%%%%%%%%%%%%%%%%%%%%%%%%%%%%%%%%%%%%%%%%%%%
\subsection{Conditional Processing}
\label{sec:conditional}

The package provides a mechanism to compile different versions
of a document. To customise the versions further some conditional processing
can come in handy to distinguish which version is being compiled.
The package provides two macros to describe the compilation context:

%%%%%%%%%%%%%%%%%%%%%%%%%%%%%%%%%%%%%%%%
\DescribeMacro{\ifchilddoc}
The conditional |\ifchilddoc| distinguishes between the compilation of
child documents and the main document:
%
\begin{center}
|\ifchilddoc |\textit{child-code}| |[|\||else |\textit{main-code}]| \||fi|
\end{center}

%%%%%%%%%%%%%%%%%%%%%%%%%%%%%%%%%%%%%%%%
\DescribeMacro{\childdocname}
\DescribeMacro{\childdocjob}
The macro |\childdocname| contains the filename (without extension)
of the main or child file being processed.
Note that |\childdocjob| will always contain the name of the main file.

%%%%%%%%%%%%%%%%%%%%%%%%%%%%%%%%%%%%%%%%
\paragraph{Title Page.}

Conditional processing can be used to include a title or banner page
in the main document when proper precautions are taken.
Importantly, the code in the main file should ensure that the page counter
(as well as other status parameters which are stored in the |.aux| files)
takes the same value after the conditional processing.
Otherwise the page numbers may take divergent values
depending on which part is compiled.

For example, a title page could be declared by:
%
\begin{center}
\begin{tabular}{l}
|\ifchilddoc\||else|\\
|\addtocounter{page}{-1}|\\
\textit{code for title page}\\
|\newpage|\\
|\||fi|
\end{tabular}
\end{center}
%
A banner page for the child documents can be generated by:
%
\begin{center}
\begin{tabular}{l}
|\ifchilddoc|\\
|\addtocounter{page}{-1}|\\
\textit{code for banner page}\\
|\newpage|\\
|\||fi|
\end{tabular}
\end{center}
%
Here one could write a message such as:
\begin{center}
|This is the part \childdocname{} of \childdocjob{}.|
\end{center}

%%%%%%%%%%%%%%%%%%%%%%%%%%%%%%%%%%%%%%%%%%%%%%%%%%%%%%%%%%%%%%%%%%%%%%%%%%%%%%%%
\subsection{Flags}
\label{sec:flags}

The package makes it easy to generate different versions
of the main or child documents.
To this end compilation flags can be defined
and assigned different default values.
They will be particularly useful in conjunction
with the forwarding mechanism described in \secref{sec:forward}.

For example, it may be useful to have a flag |\version|
which can be set to |draft| or |final|.
The document source will contain some conditional code
depending on the value of |\version|.
Suppose further, the flag should default to |final| for the main file
and to |draft| for child files
which is a natural assignment for editing the document.
This is achieved by placing the following code
in the preamble of the main document
(below the |\childdocmain| directive):
%
\begin{center}
\begin{tabular}{l}
|\ifchilddoc|\\
|\providecommand{\version}{draft}|\\
|\||else|\\
|\providecommand{\version}{final}|\\
|\||fi|
\end{tabular}
\end{center}
%
The definition by |\providecommand| makes sure
that previous definitions are not overwritten.
Further statements |\providecommand{\version}{...}|
can thus be added before the above code to override it.

For the main file, one might add a line
(between |\childdocmain| and the above block)
%
\begin{center}
|%\ifchilddoc\||else\providecommand{\version}{draft}\||fi|
\end{center}
%
which can be uncommented to produce a draft version.
Likewise one can add a line to the very top of a child file
(above the |\childdocof{|\textit{main}|}| directive)
%
\begin{center}
|%\providecommand{\version}{final}|
\end{center}
%
which can be uncommented to produce the final version of this child document.

%%%%%%%%%%%%%%%%%%%%%%%%%%%%%%%%%%%%%%%%%%%%%%%%%%%%%%%%%%%%%%%%%%%%%%%%%%%%%%%%
\subsection{Forwarding}
\label{sec:forward}

Different versions of the main or child documents
using compilation flags as described in \secref{sec:flags}
can be (permanently) stored in different files
for convenient compilation, viewing and distribution.
To this end, the package defines a command
to pass on compilation to a different file:

%%%%%%%%%%%%%%%%%%%%%%%%%%%%%%%%%%%%%%%%
\DescribeMacro{\childdocforward}
The command |\childdocforward| redirects processing to
another source file:
%
\begin{center}
\begin{tabular}{l}
|\input{childdoc.def}|\\
|\childdocforward[|\textit{main}|]{|\textit{dest}|}|\\
\end{tabular}
\end{center}
%
The argument \textit{dest} is the destination file
(without extension).
It should be the main file or one of the child files.
Note that further \textsf{childdoc} directives
such as |\childdocof| and |\childdocforward|
in the indicated file will be processed in this form.
The optional argument \textit{main}
passes on directly to the main file \textit{main}
while pretending to compile the child \textit{dest}.
This form behaves as if \textit{dest}
issues |\childdocof{|\textit{main}|}| right away,
and no further \textsf{childdoc} directives will be processed.

%%%%%%%%%%%%%%%%%%%%%%%%%%%%%%%%%%%%%%%%
\DescribeMacro{\...prefix}
In the alternative form |\childdocforwardprefix|,
%
\begin{center}
\begin{tabular}{l}
|\input{childdoc.def}|\\
|\childdocforwardprefix[|\textit{main}|]{|\textit{prefix}|}{|\textit{dest}|}|
\end{tabular}
\end{center}
%
the destination file is determined by a pattern
depending on the current file:
To make this work, the current file must be called
`{\textit{prefix}\hspace{0.2em}\textit{suffix}}'
with \textit{prefix} matching precisely the argument.
Processing is then passed on to the file
`{\textit{dest}\hspace{0.2em}\textit{suffix}}'.
Surely, the same effect is achieved by
directly specifying the
argument `{\textit{dest}\hspace{0.2em}\textit{suffix}}'
in the first form.
However, that requires to set up a different file
for each child. With the alternative form of the command
all these files can have exactly the same content
which simplifies setting them up and maintaining them.

For example, the following file |draft.tex|
with a compilation flag |\version| as described in \secref{sec:flags}
compiles the main document as a draft:
%
\begin{center}
\begin{tabular}{l}
|\def\version{draft}|\\
|\input{childdoc.def}|\\
|\childdocforward{|\textit{main}|}|
\end{tabular}
\end{center}
%
Likewise, the following files |final|\textit{nn}|.tex|
compile the final version of the child document
|child|\textit{nn}|.tex|:
%
\begin{center}
\begin{tabular}{l}
|\def\version{final}|\\
|\input{childdoc.def}|\\
|\childdocforwardprefix{final}{child}|
\end{tabular}
\end{center}
%

Note that when several versions of a main file and/or of each child file
are to be generated, it may be convenient to set up a |Makefile| or
shell script to automatise the process.

%%%%%%%%%%%%%%%%%%%%%%%%%%%%%%%%%%%%%%%%%%%%%%%%%%%%%%%%%%%%%%%%%%%%%%%%%%%%%%%%
\subsection{Command Line Processing}
\label{sec:commandline}

The effect of redirection files can also be achieved by invoking
the \LaTeX{} compiler with a more elaborate command line.
Most conveniently this should be done as part
of a shell script or a |Makefile|.

When using \textsf{childdoc} in the main file, the following
command lines effectively perform a redirection
(note that depending on the shell being used,
backslashes may have to be doubled: `|\|' $\to$ `|\\|'):
%
\begin{center}
|... -jobname "|\textit{target}|" |\\|"|[\textit{flags}]%
|\input{childdoc.def}\childdocforward[|\textit{main}|]{|\textit{dest}|}"|
\end{center}
%
Here \textit{target} is the name of the output file,
\textit{main} is the name of the main file
and \textit{dest} is the name of the main or child file to be processed
(all filenames without extensions).
The optional argument \textit{main} can be omitted
if \textit{main} matches \textit{dest}.
Optionally, compilation \textit{flags} can be defined via |\def| commands.
This command line makes the \TeX{} engine believe
it is compiling the file \textit{target}
whose content is specified as the latter parameter.
The provided code then forwards the processing to
\textit{main} or \textit{dest} as described in \secref{sec:forward}.

%%%%%%%%%%%%%%%%%%%%%%%%%%%%%%%%%%%%%%%%%%%%%%%%%%%%%%%%%%%%%%%%%%%%%%%%%%%%%%%%
\subsection{Include by Input}
\label{sec:input}

Including child documents by |\include| has some restrictions by design.
Most notably, the content of a child document always occupies
its own set of pages; pages cannot be shared between child documents.
Usually, this behaviour makes perfect sense
because each child document contain an essential part of the document.
However, in some situations it may be desirable to compose
a document from a collection of parts
without having mandatory page breaks between then.
For this case, the package
provides a mechanism to include parts
by |\input| which can also be processed individually.
However, by construction this mechanism
requires manual handling of the content to be output.

%%%%%%%%%%%%%%%%%%%%%%%%%%%%%%%%%%%%%%%%
\DescribeMacro{\ifchilddocmanual}
The main file should be prepared as usual, see \secref{sec:include}.
However, the document body must make a distinction
between processing of an individual part and of the main document, e.g.:
%
\begin{center}
\begin{tabular}{l}
|\ifchilddocmanual|\\
|\input{\childdocname}|\\
|\||else|\\
\textit{document body with }|\input{|\textit{part}|}|\\
|\||fi|
\end{tabular}
\end{center}
%
The conditional |\ifchilddocmanual| is true whenever
a part to be included by |\input| is being compiled,
and the name of the part is stored in |\childdocname|.

%%%%%%%%%%%%%%%%%%%%%%%%%%%%%%%%%%%%%%%%
\DescribeMacro{\childdocby}
Each part to be included by |\input| should start with:
%
\begin{center}
\begin{tabular}{l}
|\input{childdoc.def}|\\
|\childdocby{|\textit{main}|}|\\
\end{tabular}
\end{center}
%
The directive |\childdocby| is similar to |\childdocof|
described in \secref{sec:include},
but the subsequent selection of content must be done manually.
To that end, both |\ifchilddoc| and |\ifchilddocmanual|
will be true upon processing of a part,
and the name of the part is stored in |\childdocname|.
Note that |\jobname| will be set to the filename of the current part
so that each part receives an individual |.aux| file
that does not interfere with the |.aux| file(s) of the main document.
This behaviour can be altered by the alternative form
|\childdocby[*]{|\textit{main}|}| (with a non-empty optional argument)
which uses the |.aux| file of the main document
by setting |\jobname| to \textit{main}.

%%%%%%%%%%%%%%%%%%%%%%%%%%%%%%%%%%%%%%%%%%%%%%%%%%%%%%%%%%%%%%%%%%%%%%%%%%%%%%%%
\subsection{Driver Development}
\label{sec:driver}

The \textsf{childdoc} mechanism can also be use for the development
of definition files such as \LaTeX{} styles or classes.
This case differs from the above setup with multiple parts
included by |\include| in that no |\includeonly| should be invoked.
This can be achieved by starting the include file
(before |\ProvidesPackage|) with:
%
\begin{center}
\begin{tabular}{l}
|\input{childdoc.def}|\\
|\childdocforward{|\textit{main}|}|\\
\end{tabular}
\end{center}
%
or alternatively with:
%
\begin{center}
\begin{tabular}{l}
|\input{childdoc.def}|\\
|\childdocby{|\textit{main}|}|\\
\end{tabular}
\end{center}
%
Both forms have slightly different effects as described above.
The main file is prepared as usual, see \secref{sec:include}.

%%%%%%%%%%%%%%%%%%%%%%%%%%%%%%%%%%%%%%%%%%%%%%%%%%%%%%%%%%%%%%%%%%%%%%%%%%%%%%%%
\subsection{Legacy Detection}
\label{sec:detection}

The directive |\childdocmain| in the main file can detect
whether the complete document or merely a child is to be compiled
even without using the directive |\childdocof|.
This method is deprecated because it is less robust
and there is no compelling reason to use it;
it is merely provided for backward compatibility
and it may be removed in future versions.

If the detection mechanism is to be used,
it is mandatory to correctly specify
the filename of the main file as the argument of |\childdocmain|:
%
\begin{center}
\begin{tabular}{l}
|\input{childdoc.def}|\\
|\childdocmain{|\textit{main}|}|\\
\end{tabular}
\end{center}
%
If |\jobname| does not match the argument \textit{main} of |\childdocmain|,
it is assumed that |\jobname| points to the child file to be compiled.
When using |\childdocmain| with the main file specified as argument,
it suffices to start a child file
with just |\input{|\textit{main}|}|
without loading of the package and using |\childdocof|.
If instead all processing is done
with the appropriate \textsf{childdoc} directives,
the argument of \textit{main} of |\childdocmain| can be empty.

An alternative version of the command line processing described
in \secref{sec:commandline} using the detection mechanism reads:
%
\begin{center}
|... -jobname "|\textit{target}|" "|[\textit{flags}]%
[|\def\jobname{|\textit{dest}|}|]|\input{|\textit{main}|}"|
\end{center}

%%%%%%%%%%%%%%%%%%%%%%%%%%%%%%%%%%%%%%%%%%%%%%%%%%%%%%%%%%%%%%%%%%%%%%%%%%%%%%%%
\subsection{Manual Code}
\label{sec:manual}

In case one cannot be certain whether the definitions file |childdoc.def|
is installed on the target \TeX{} distribution
and one prefers not to ship it,
it is conceivable to paste a few relevant commands into the sources.

To that end, drop all statements |\input{childdoc.def}|
and perform the replacements as outlined below.
Instead of |\childdocmain{|\textit{main}|}| add the following code
to the top of the main file:
%
\begin{center}
\begin{tabular}{l}
|\||ifdefined\childdocname\endinput\||fi\newif\ifchilddoc|\\
|\edef\childdocname{\scantokens\expandafter{\jobname\noexpand}}|\\
|\def\childdocmain{|\textit{main}|}\||ifx\childdocmain\childdocname\||else|\\
|\childdoctrue\includeonly{\childdocname}\let\jobname\childdocmain\||fi|\\
\end{tabular}
\end{center}
%
Instead of |\childdocof{|\textit{main}|}| just include the main file
at the top of each child file:
%
\begin{center}
|\input{|\textit{main}|}|
\end{center}
%
A simple redirection |\childdocforward{|\textit{dest}|}| is achieved by:
%
\begin{center}
|\def\jobname{|\textit{dest}|}\input{\jobname}|
\end{center}
%
The redirection with prefix
|\childdocforwardprefix[|\textit{prefix}|]{|\textit{dest}|}|
is accomplished by:
%
\begin{center}
\begin{tabular}{l}
|{\edef\jobname{\scantokens\expandafter{\jobname\noexpand}}|\\
|\def\redirectjob |\textit{prefix}|#1~~~{\gdef\jobname{|\textit{dest}|#1}}|\\
|\expandafter\redirectjob\jobname~~~}\input{\jobname}|
\end{tabular}
\end{center}

In an alternative approach,
child documents can be compiled by a specific command line
without additional code or specific definitions:
%
\begin{center}
|... -jobname "|\textit{target}|" "|[\textit{flags}]%
|\includeonly{|\textit{dest}|}\input{|\textit{main}|}"|
\end{center}
%

%%%%%%%%%%%%%%%%%%%%%%%%%%%%%%%%%%%%%%%%%%%%%%%%%%%%%%%%%%%%%%%%%%%%%%%%%%%%%%%%
%%%%%%%%%%%%%%%%%%%%%%%%%%%%%%%%%%%%%%%%%%%%%%%%%%%%%%%%%%%%%%%%%%%%%%%%%%%%%%%%
\section{Information}

%%%%%%%%%%%%%%%%%%%%%%%%%%%%%%%%%%%%%%%%%%%%%%%%%%%%%%%%%%%%%%%%%%%%%%%%%%%%%%%%
\subsection{Copyright}

Copyright \copyright{} 2017--2018 Niklas Beisert

This work may be distributed and/or modified under the
conditions of the \LaTeX{} Project Public License, either version 1.3
of this license or (at your option) any later version.
The latest version of this license is in
  \url{http://www.latex-project.org/lppl.txt}
and version 1.3 or later is part of all distributions of \LaTeX{}
version 2005/12/01 or later.

This work has the LPPL maintenance status `maintained'.

The Current Maintainer of this work is Niklas Beisert.

This work consists of the files |README.txt|, |childdoc.ins| and |childdoc.dtx|
as well as the derived files |childdoc.def|, |cdocsamp.tex|
with |cdocsch1.tex|, |cdocsch2.tex|, |cdocspt3.tex|, |cdocspt4.tex|,
|cdocsdrf.tex|, |cdocsfn1.tex|, |cdocsfn2.tex|
as well as |childdoc.pdf|.

%%%%%%%%%%%%%%%%%%%%%%%%%%%%%%%%%%%%%%%%%%%%%%%%%%%%%%%%%%%%%%%%%%%%%%%%%%%%%%%%
\subsection{Files and Installation}

The package consists of the files:
%
\begin{center}
\begin{tabular}{ll}
    |README.txt|   & readme file \\
    |childdoc.ins| & installation file \\
    |childdoc.dtx| & source file \\
    |childdoc.def| & definition file \\
    |cdocsamp.tex| & sample main file \\
    |cdocsch1.tex| & sample include file \\
    |cdocsch2.tex| & sample include file \\
    |cdocspt3.tex| & sample part file \\
    |cdocspt4.tex| & sample part file \\
    |cdocsdrf.tex| & sample redirection file \\
    |cdocsfn1.tex| & sample redirection file \\
    |cdocsfn2.tex| & sample redirection file \\
    |childdoc.pdf| & manual
\end{tabular}
\end{center}
%
The distribution consists of the files
|README.txt|, |childdoc.ins| and |childdoc.dtx|.
%
\begin{itemize}
\item
Run (pdf)\LaTeX{} on |childdoc.dtx|
to compile the manual |childdoc.pdf| (this file).
\item
Run \LaTeX{} on |childdoc.ins| to create the definitions file |childdoc.def|
and the sample |cdocsamp.tex| with include files
|cdocsch1.tex|, |cdocsch2.tex|, |cdocspt3.tex|, |cdocspt4.tex|,
|cdocsdrf.tex|, |cdocsfn1.tex|, |cdocsfn2.tex|.
Then copy the file |childdoc.def| to an appropriate directory of your \LaTeX{}
distribution, e.g.\ \textit{texmf-root}|/tex/latex/childdoc|.
\end{itemize}

%%%%%%%%%%%%%%%%%%%%%%%%%%%%%%%%%%%%%%%%%%%%%%%%%%%%%%%%%%%%%%%%%%%%%%%%%%%%%%%%
\subsection{Related CTAN Packages}

There are several other packages which offer a similar functionality:
%
\begin{itemize}
\item
The packages
\href{http://ctan.org/pkg/docmute}{\textsf{docmute}},
\href{http://ctan.org/pkg/includex}{\textsf{includex}} and
\href{http://ctan.org/pkg/standalone}{\textsf{standalone}}
provide commands to include only the document body of
a child file thus allowing both files to be compiled individually.
\item
The packages \href{http://ctan.org/pkg/subdocs}{\textsf{subdocs}}
and \href{http://ctan.org/pkg/subfiles}{\textsf{subfiles}}
provide structures in which the main and child documents can be
encapsulated and allowing them to be compiled individually.
The inclusion mechanism is different from the conventional |\include|.
\item
The package \href{http://ctan.org/pkg/combine}{\textsf{combine}}
is an elaborate solution to combine several documents into one.
\end{itemize}
%
See also the CTAN topic \href{http://ctan.org/topic/subdocs}{\textsf{subdocs}}
for further related packages.
The present package differs from the above solutions in that
a document structure constructed with the conventional |\include| mechanism
just needs two extra commands at the top of every file
such that all constituent files can be compiled individually.

%%%%%%%%%%%%%%%%%%%%%%%%%%%%%%%%%%%%%%%%%%%%%%%%%%%%%%%%%%%%%%%%%%%%%%%%%%%%%%%%
%\subsection{Feature Suggestions}
%
%The following is a list of features which may be useful for future
%versions of this package:
%%
%\begin{itemize}
%\item
%\ldots
%\end{itemize}

%%%%%%%%%%%%%%%%%%%%%%%%%%%%%%%%%%%%%%%%%%%%%%%%%%%%%%%%%%%%%%%%%%%%%%%%%%%%%%%%
\subsection{Revision History}

%%%%%%%%%%%%%%%%%%%%%%%%%%%%%%%%%%%%%%%%
\paragraph{v2.0:} 2018/12/30

\begin{itemize}
\item
immediate forward processing
\item
added |\childdocby| mechanism
\item
manual restructured
\end{itemize}

%%%%%%%%%%%%%%%%%%%%%%%%%%%%%%%%%%%%%%%%
\paragraph{v1.6:} 2018/01/17

\begin{itemize}
\item
application for development of include files
\item
corrections to manual
\end{itemize}

%%%%%%%%%%%%%%%%%%%%%%%%%%%%%%%%%%%%%%%%
\paragraph{v1.5:} 2017/05/21

\begin{itemize}
\item
more complete structuring introduced
\item
|\childdocof| introduced
\item
|\childdoc| renamed to |\childdocmain|
\item
|\childredirect| renamed to |\childdocforward| and |\childdocforwardprefix|
and functionality expanded
\end{itemize}

%%%%%%%%%%%%%%%%%%%%%%%%%%%%%%%%%%%%%%%%
\paragraph{v1.0:} 2017/04/27

\begin{itemize}
\item
manual and install package
\item
first version published on CTAN
\end{itemize}

%%%%%%%%%%%%%%%%%%%%%%%%%%%%%%%%%%%%%%%%
\paragraph{v0.6:} 2017/04/26

\begin{itemize}
\item
redirection mechanism added
\end{itemize}

%%%%%%%%%%%%%%%%%%%%%%%%%%%%%%%%%%%%%%%%
\paragraph{v0.5:} 2017/04/26

\begin{itemize}
\item
functionality in definition file
\end{itemize}


%%%%%%%%%%%%%%%%%%%%%%%%%%%%%%%%%%%%%%%%%%%%%%%%%%%%%%%%%%%%%%%%%%%%%%%%%%%%%%%%
%%%%%%%%%%%%%%%%%%%%%%%%%%%%%%%%%%%%%%%%%%%%%%%%%%%%%%%%%%%%%%%%%%%%%%%%%%%%%%%%
%%%%%%%%%%%%%%%%%%%%%%%%%%%%%%%%%%%%%%%%%%%%%%%%%%%%%%%%%%%%%%%%%%%%%%%%%%%%%%%%
\appendix

\settowidth\MacroIndent{\rmfamily\scriptsize 000\ }

 \DocInput{childdoc.dtx}

\end{document}
%</driver>
% \fi
%
% %%%%%%%%%%%%%%%%%%%%%%%%%%%%%%%%%%%%%%%%%%%%%%%%%%%%%%%%%%%%%%%%%%%%%%%%%%%%%%
% %%%%%%%%%%%%%%%%%%%%%%%%%%%%%%%%%%%%%%%%%%%%%%%%%%%%%%%%%%%%%%%%%%%%%%%%%%%%%%
% \section{Sample}
%\iffalse
%<*samplemain>
%\fi
%
% The following presents a sample document
% with two chapters, two parts, a title page,
% a compile flag as well as three forwarding files to set the flag.
% It consists of eight |.tex| files:
% \begin{center}
% \begin{tabular}{ll}
% |cdocsamp.tex|&main file\\
% |cdocsch1.tex|&include file for chapter 1\\
% |cdocsch2.tex|&include file for chapter 2\\
% |cdocspt3.tex|&include file for part 3\\
% |cdocspt4.tex|&include file for part 4\\
% |cdocsdrf.tex|&forwarding file for main file in draft mode\\
% |cdocsfi1.tex|&forwarding file for final version of chapter 1\\
% |cdocsfi2.tex|&forwarding file for final version of chapter 2\\
% \end{tabular}
% \end{center}
% Each of the eight files can be compiled directly by the \LaTeX{} compiler.
%
% %%%%%%%%%%%%%%%%%%%%%%%%%%%%%%%%%%%%%%
% \paragraph{Main File.}
%
% The main file is called |cdocsamp.tex|.
%
% Load the \textsf{childdoc} definitions and
% declare the filename for the main document:
%    \begin{macrocode}
\input{childdoc.def}
\childdocmain{}
%    \end{macrocode}

% Optional override for |\version| flag:
%    \begin{macrocode}
%%\ifchilddoc\else\providecommand{\version}{draft}\fi
%    \end{macrocode}

% Define the default values for the |\version| flag
% (|final| for the main file and |draft| for childs):
%    \begin{macrocode}
\ifchilddoc
\providecommand{\version}{draft}
\else
\providecommand{\version}{final}
\fi
%    \end{macrocode}

% Load the standard document class:
%    \begin{macrocode}
\documentclass[12pt]{article}
%    \end{macrocode}

% Start the document body:
%    \begin{macrocode}
\begin{document}
%    \end{macrocode}

% Declare a title page.
% Print title, part of document being processed and version flag:
%    \begin{macrocode}
\addtocounter{page}{-1}
\begin{center}
{\LARGE\bfseries{}childdoc example\par}
\vspace{1cm}
\ifchilddoc
\ifchilddocmanual part\else chapter\fi:
`\childdocname' of `\childdocjob'\par
\else
main document: `\childdocjob'\par
\fi
version: \version\par
\end{center}
\newpage
%    \end{macrocode}

% Manually include selected file,
% otherwise process as usual:
%    \begin{macrocode}
\ifchilddocmanual
\section*{part `\childdocname'}
\input{\childdocname}
\else
%    \end{macrocode}

% Include the two chapters:
%    \begin{macrocode}
\include{cdocsch1}
\include{cdocsch2}
%    \end{macrocode}

% Include the two parts unless only chapters should be displayed:
%    \begin{macrocode}
\ifchilddoc\else
\section{part three}
\input{cdocspt3}
\section{part four}
\input{cdocspt4}
\fi
%    \end{macrocode}

% Process as usual until here:
%    \begin{macrocode}
\fi
%    \end{macrocode}

% End of document body:
%    \begin{macrocode}
\end{document}
%    \end{macrocode}
%\iffalse
%</samplemain>
%\fi
%
% %%%%%%%%%%%%%%%%%%%%%%%%%%%%%%%%%%%%%%
% \paragraph{Chapter Include Files.}
%
% The include files are called |cdocsch1.tex| and |cdocsch2.tex|.
%
%\iffalse
%<*samplechap1|samplechap2>
%\fi

% Optional override for |\version| flag:
%    \begin{macrocode}
%%\providecommand{\version}{final}
%    \end{macrocode}

% Include the main document:
%    \begin{macrocode}
\input{childdoc.def}
\childdocof{cdocsamp}
%    \end{macrocode}

%\iffalse
%</samplechap1|samplechap2>
%\fi
%
%\iffalse
%<*samplechap1>
%\fi
% Some text for chapter 1:
%    \begin{macrocode}
\section{one}
some text in chapter one
%    \end{macrocode}

%\iffalse
%</samplechap1>
%\fi
% Some text for chapter 2:
%\iffalse
%<*samplechap2>
%\fi
%    \begin{macrocode}
\section{two}
more text in chapter two
%    \end{macrocode}

%\iffalse
%</samplechap2>
%\fi
%
% %%%%%%%%%%%%%%%%%%%%%%%%%%%%%%%%%%%%%%
% \paragraph{Part Include Files.}
%
% The include files are called |cdocspt3.tex| and |cdocspt4.tex|.
%
%\iffalse
%<*samplepart3|samplepart4>
%\fi

% Optional override for |\version| flag:
%    \begin{macrocode}
%%\providecommand{\version}{final}
%    \end{macrocode}

% Include the main document:
%    \begin{macrocode}
\input{childdoc.def}
\childdocby{cdocsamp}
%    \end{macrocode}

%\iffalse
%</samplepart3|samplepart4>
%\fi
%
%\iffalse
%<*samplepart3>
%\fi
% Some text for part 3:
%    \begin{macrocode}
some text in part three
%    \end{macrocode}

%\iffalse
%</samplepart3>
%\fi
% Some text for part 4:
%\iffalse
%<*samplepart4>
%\fi
%    \begin{macrocode}
more text in part four
%    \end{macrocode}

%\iffalse
%</samplepart4>
%\fi
%
% %%%%%%%%%%%%%%%%%%%%%%%%%%%%%%%%%%%%%%
% \paragraph{Forwarding for a Complete Draft.}
%
% The following forwarding file |cdocsdrf.tex|
% compiles the main document in draft mode:
%\iffalse
%<*sampledraft>
%\fi
%    \begin{macrocode}
\def\version{draft}
\input{childdoc.def}
\childdocforward{cdocsamp}
%    \end{macrocode}

%\iffalse
%</sampledraft>
%\fi
%
% %%%%%%%%%%%%%%%%%%%%%%%%%%%%%%%%%%%%%%
% \paragraph{Forwarding for Final Version of the Chapters.}
%
% The following forwarding files |cdocsfn1.tex| and |cdocsfn2.tex|
% (with identical content)
% compile the final versions of the child documents
% |cdocsch1.tex| and |cdocsch2.tex|, respectively:
%\iffalse
%<*samplefinal>
%\fi
%    \begin{macrocode}
\def\version{final}
\input{childdoc.def}
\childdocforwardprefix[cdocsamp]{cdocsfn}{cdocsch}
%    \end{macrocode}

%\iffalse
%</samplefinal>
%\fi
%
% %%%%%%%%%%%%%%%%%%%%%%%%%%%%%%%%%%%%%%
% \paragraph{Command Line Processing.}
%
% The following three command lines generate the output files
% |cdocscld|, |cdocscl1| and |cdocscl2|
% which should be identical to
% |cdocsdrf|, |cdocsch1| and |cdocsfn2|, respectively:
% \begin{center}
% \begin{tabular}{l}
% |latex -jobname cdocscld \|\\
% |  "\def\version{draft}\input{childdoc.def}\childdocforward{cdocsamp}"|\\
% |latex -jobname cdocscl1 \|\\
% |  "\input{childdoc.def}\childdocforward[cdocsamp]{cdocsch1}"|\\
% |latex -jobname cdocscl2 \|\\
% |  "\def\version{final}\input{childdoc.def}\childdocforward{cdocsch2}"|
% \end{tabular}
% \end{center}
% Note that the trailing backslash on each first line
% merely continues the input to the second line
% (for convenient cut ant paste).
% Furthermore, the command |latex| can be replaced by any
% of its alternative versions such as |pdflatex|.
%
% %%%%%%%%%%%%%%%%%%%%%%%%%%%%%%%%%%%%%%%%%%%%%%%%%%%%%%%%%%%%%%%%%%%%%%%%%%%%%%
% %%%%%%%%%%%%%%%%%%%%%%%%%%%%%%%%%%%%%%%%%%%%%%%%%%%%%%%%%%%%%%%%%%%%%%%%%%%%%%
% \section{Implementation}
%\iffalse
%<*package>
%\fi
%
% This section describes the definitions file |childdoc.def|.

% The definitions cannot be loaded using |\usepackage| or |\RequirePackage|
% which has a mechanism to prevent loading a style file more than once.
% When loading the definitions by means of |\input|
% multiple instances have to be prevented manually:
%\iffalse
%This code needs to be before the `\ProvidesFile' directive
%which is defined at the beginning of this file.
%Therefore it is also placed there and commented out here.
%</package>
%<*discard>
%\fi
%    \begin{macrocode}
\ifdefined\childdocmain\endinput\fi
%    \end{macrocode}
%\iffalse
%</discard>
%<*package>
%\fi
%
% \macro{\ifchilddoc}
% \macro{\ifchilddocmanual}
% The conditional |\ifchilddoc| tells whether a
% child (true) or main (false) document is being compiled.
% The conditional |\ifchilddocmanual| tells whether
% the |\includeonly| mechanism is used (false) or
% the selection of child files must be performed manually (true).
% The definitions initialise to false:
%    \begin{macrocode}
\newif\ifchilddoc
\newif\ifchilddocmanual
%    \end{macrocode}

% \macro{\childdocname}
% \macro{\childdocjob}
% The macro |\childdocname| stores the name of the main document
% to be compiled. The macro |\childdocjob| stores the name of
% the document on which the \LaTeX{} compiler was originally invoked.
% The content of |\jobname| cannot be compared
% to filenames specified in the source due to different catcodes.
% The following code rescans |\jobname|, stores the result
% in |\childdocname| and saves a copy in |\childdocjob|:
%    \begin{macrocode}
\edef\childdocname{\scantokens\expandafter{\jobname\noexpand}}
\let\childdocjob\childdocname
%    \end{macrocode}

% \macro{\childdocdisable}
% The macro |\childdocdisable| prevents the main file
% from being processed more than once.
% At this stage, the main document command |\childdocmain|
% is assumed to be called once again where it should do nothing.
% Any subsequent call to it should prevent
% a secondary processing of the main document
% It overwrites the forwarding commands
% |\childdocof| and |\childdocforward|
% with empty macros to prevent further inclusions of the main document:
%    \begin{macrocode}
\newcommand{\childdocdisable}
{
  \renewcommand{\childdocmain}[1]{\renewcommand{\childdocmain}[1]{\endinput}}
  \renewcommand{\childdocof}[1]{}
  \renewcommand{\childdocby}[2][]{}
  \renewcommand{\childdocforward}[2][]{}
  \renewcommand{\childdocdisable}{}
}
%    \end{macrocode}

% \macro{\childdocmain}
% The macro |\childdocmain| is to be called at the top of the main file
% with nothing or the main filename (without extension) as argument.
% First, it breaks loops.
% If the argument is not empty and does not match |\childdocname|
% (which is set by the first inclusion of |childdoc.def|),
% |\ifchilddoc| is set to true, |\includeonly| is applied to the child file
% and |\jobname| is set to the main file
% (for proper handling of |.aux| files):
%    \begin{macrocode}
\newcommand{\childdocmain}[1]
{
  \childdocdisable\childdocmain{}
  \if?#1?\else
    \begingroup
      \def\childdoctmp{#1}
      \ifx\childdoctmp\childdocname
        \def\childdoctmp{}
      \else
        \def\childdoctmp
        {
          \childdoctrue
          \includeonly{\childdocname}
          \def\childdocjob{#1}
          \def\jobname{#1}
        }
      \fi
      \expandafter
    \endgroup
    \childdoctmp
  \fi
}
%    \end{macrocode}

% \macro{\childdocof}
% The command |\childdocof| redirects
% compilation to the main file |#1|.
%    \begin{macrocode}
\newcommand{\childdocof}[1]
{
  \childdocdisable
  \childdoctrue
  \includeonly{\childdocname}
  \def\jobname{#1}
  \def\childdocjob{#1}
  \input{#1}
}
%    \end{macrocode}

% \macro{\childdocby}
% The command |\childdocby| ....
%    \begin{macrocode}
\newcommand{\childdocby}[2][]
{
  \childdocdisable
  \childdoctrue
  \childdocmanualtrue
  \if?#1?\else
    \def\jobname{#2}
  \fi
  \def\childdocjob{#2}
  \input{#2}
  \endinput
}
%    \end{macrocode}

% \macro{\childdocforward}
% The command |\childdocforward| redirects
% compilation to the main file or
% (if the optional argument is given) a child file.
% Parameters are set as if the main file
% or a child file starting with |\childdocof| was compiled.
% Then compilation is handed over to the main file:
%    \begin{macrocode}
\newcommand{\childdocforward}[2][]
{
  \begingroup
    \if?#1?
      \def\childdoctmp
      {
        \def\childdocname{#2}
        \def\childdocjob{#2}
        \def\jobname{#2}
        \input{#2}
        \endinput
      }
    \else
      \def\childdoctmp
      {
        \childdocdisable
        \def\childdocname{#2}
        \childdoctrue
        \includeonly{#2}
        \def\childdocjob{#1}
        \def\jobname{#1}
        \input{#1}
        \endinput
      }
    \fi
    \expandafter
  \endgroup
  \childdoctmp
}
%    \end{macrocode}

% \macro{\childdocforwardprefix}
% The command |\childdocforwardprefix| redirects
% compilation to the main or a child file by means of a pattern.
% The prefix |#1| in the current filename is replaced by |#2|
% and the suffix of the current filename is kept
% (it is assumed that the filename does not contain the substring `|~~~|'
% which is used as a delimiter).
% Compilation is handed over to the new file by |\childdocforward|:
%    \begin{macrocode}
\newcommand{\childdocforwardprefix}[3][]
{
  \begingroup
    \def\childdocextract #2##1~~~{\def\childdoctmp{\childdocforward[#1]{#3##1}}}
    \expandafter\childdocextract\childdocname~~~
    \expandafter
  \endgroup
  \childdoctmp
}
%    \end{macrocode}

% \macro{\childdoc}
% The deprecated macro |\childdoc| is a legacy version of |\childdocmain|:
%    \begin{macrocode}
\newcommand{\childdoc}{\childdocmain}
%    \end{macrocode}

% \macro{\childdocredirect}
% The deprecated macro |\childdocredirect| is a legacy version
% of |\childdocforward| and |\childdocforwardprefix|:
%    \begin{macrocode}
\newcommand{\childdocredirect}[2][]
{
  \begingroup
    \if?#1?
      \def\childdoctmp{\childdocforward{#2}}
    \else
      \def\childdoctmp{\childdocforwardprefix{#1}{#2}}
    \fi
    \expandafter
  \endgroup
  \childdoctmp
}
%    \end{macrocode}

%\iffalse
%</package>
%\fi
%
\endinput

\childdocby{cdocsamp}
%    \end{macrocode}

%\iffalse
%</samplepart3|samplepart4>
%\fi
%
%\iffalse
%<*samplepart3>
%\fi
% Some text for part 3:
%    \begin{macrocode}
some text in part three
%    \end{macrocode}

%\iffalse
%</samplepart3>
%\fi
% Some text for part 4:
%\iffalse
%<*samplepart4>
%\fi
%    \begin{macrocode}
more text in part four
%    \end{macrocode}

%\iffalse
%</samplepart4>
%\fi
%
% %%%%%%%%%%%%%%%%%%%%%%%%%%%%%%%%%%%%%%
% \paragraph{Forwarding for a Complete Draft.}
%
% The following forwarding file |cdocsdrf.tex|
% compiles the main document in draft mode:
%\iffalse
%<*sampledraft>
%\fi
%    \begin{macrocode}
\def\version{draft}
% \iffalse
%
% childdoc.dtx Copyright (C) 2017-2018 Niklas Beisert
%
% This work may be distributed and/or modified under the
% conditions of the LaTeX Project Public License, either version 1.3
% of this license or (at your option) any later version.
% The latest version of this license is in
%   http://www.latex-project.org/lppl.txt
% and version 1.3 or later is part of all distributions of LaTeX
% version 2005/12/01 or later.
%
% This work has the LPPL maintenance status `maintained'.
%
% The Current Maintainer of this work is Niklas Beisert.
%
% This work consists of the files childdoc.dtx and childdoc.ins
% and the derived files childdoc.def and cdocsamp.tex with
% cdocsch1.tex, cdocsch2.tex, cdocsdrf.tex, cdocsfn1.tex, cdocsfn2.tex.
%
%<package>\ifdefined\childdocmain\endinput\fi
%<package>\ProvidesFile{childdoc.def}[2018/12/30 v2.0 child document driver]
%<samplemain>\ProvidesFile{cdocsamp.tex}[2018/12/30 v2.0 sample for childdoc]
%<*driver>
%\ProvidesFile{childdoc.drv}[2018/12/30 v2.0 childdoc reference manual file]
\PassOptionsToClass{10pt,a4paper}{article}
\documentclass{ltxdoc}

\usepackage[margin=35mm]{geometry}
\usepackage{hyperref}
\usepackage{hyperxmp}
\usepackage[usenames]{color}

\hypersetup{colorlinks=true}
\hypersetup{pdfstartview=FitH}
\hypersetup{pdfpagemode=UseNone}
\hypersetup{pdfsource={}}
\hypersetup{pdflang={en-UK}}
\hypersetup{pdfcopyright={Copyright 2017-2018 Niklas Beisert.
  This work may be distributed and/or modified under the
  conditions of the LaTeX Project Public License, either version 1.3
  of this license or (at your option) any later version.}}
\hypersetup{pdflicenseurl={http://www.latex-project.org/lppl.txt}}
\hypersetup{pdfcontactaddress={ETH Zurich, ITP, HIT K,
  Wolfgang-Pauli-Strasse 27}}
\hypersetup{pdfcontactpostcode={8093}}
\hypersetup{pdfcontactcity={Zurich}}
\hypersetup{pdfcontactcountry={Switzerland}}
\hypersetup{pdfcontactemail={nbeisert@itp.phys.ethz.ch}}
\hypersetup{pdfcontacturl={http://people.phys.ethz.ch/\xmptilde nbeisert/}}

\newcommand{\secref}[1]{\hyperref[#1]{section \ref*{#1}}}

\parskip1ex
\parindent0pt
\let\olditemize\itemize
\def\itemize{\olditemize\parskip0pt}

\begin{document}

\title{The \textsf{childdoc} Package}
\hypersetup{pdftitle={The childdoc Package}}
\author{Niklas Beisert\\[2ex]
  Institut f\"ur Theoretische Physik\\
  Eidgen\"ossische Technische Hochschule Z\"urich\\
  Wolfgang-Pauli-Strasse 27, 8093 Z\"urich, Switzerland\\[1ex]
  \href{mailto:nbeisert@itp.phys.ethz.ch}
  {\texttt{nbeisert@itp.phys.ethz.ch}}}
\hypersetup{pdfauthor={Niklas Beisert}}
\hypersetup{pdfsubject={Manual for the LaTeX2e Package childdoc}}
\date{30 December 2018, \textsf{v2.0}}
\maketitle

\begin{abstract}\noindent
\textsf{childdoc} is a \LaTeXe{} package
that enables the direct compilation
of document sections included by |\include|
to individual files.
\end{abstract}

\begingroup
\parskip0ex
\tableofcontents
\endgroup

%%%%%%%%%%%%%%%%%%%%%%%%%%%%%%%%%%%%%%%%%%%%%%%%%%%%%%%%%%%%%%%%%%%%%%%%%%%%%%%%
%%%%%%%%%%%%%%%%%%%%%%%%%%%%%%%%%%%%%%%%%%%%%%%%%%%%%%%%%%%%%%%%%%%%%%%%%%%%%%%%
\section{Introduction}

\LaTeX{} provides a mechanism to structure a large document (such as a book)
into a main file and several child files (containing the chapters)
using the |\include| command.
This mechanism is beneficial for documents
which span hundreds of pages in order to
make the source file(s) more manageable.
Moreover, compilation can be restricted to
selected child files by means of the |\includeonly| command.
The latter feature can be used to reduce the compilation time while editing
(this was significantly more useful in the earlier days of \LaTeX{})
or to generate a smaller document which is easier to navigate.
Another application of |\includeonly| is to generate
documents consisting of selected parts of the complete document.

However, there are a few drawbacks of the plain |\include| mechanism:
\begin{itemize}
\item
The child files cannot be compiled on their own,
they can only be compiled via the main file.
A naive editing environment
(such as a text editor with an option
to have the current file processed by \LaTeX)
may require one to switch to the main file before compiling;
attempting to compile the child file produces errors.
\item
The main file must be modified (each time)
to adjust the |\includeonly| command
to the present needs. This easily leaves the main file in a messy state.
\item
The generated document will always carry the filename
of the main document. This is inconvenient if
several child files are to be compiled and
to be kept for distribution.
\end{itemize}

The present package provides a simple interface
to make child files individually compilable by \LaTeX{}.
Compiling a child file then has the same effect as compiling
the main file with an |\includeonly| command
to select the appropriate child.
Moreover the generated document will carry the name of the child
rather than the main file.
This resolves all three above issues.

This feature is meant to make the editing of books,
thesis documents and lecture notes somewhat more convenient.
However, the package can also be used efficiently for
composing a series of documents (such as exercise sheets)
which are typically distributed individually.
It then assists the author in generating the individual documents
(potentially in different versions)
as well as a document containing the collected series.
Another application is in developing style files
or other kinds of included material
where compilation of the style file could redirect
to a sample or test file.

%%%%%%%%%%%%%%%%%%%%%%%%%%%%%%%%%%%%%%%%%%%%%%%%%%%%%%%%%%%%%%%%%%%%%%%%%%%%%%%%
%%%%%%%%%%%%%%%%%%%%%%%%%%%%%%%%%%%%%%%%%%%%%%%%%%%%%%%%%%%%%%%%%%%%%%%%%%%%%%%%
\section{Usage}

First of all, the package \textsf{childdoc} is \emph{not} a standard
\LaTeXe{} |.sty| style file! Therefore it needs to be invoked in
a non-standard way.

%%%%%%%%%%%%%%%%%%%%%%%%%%%%%%%%%%%%%%%%%%%%%%%%%%%%%%%%%%%%%%%%%%%%%%%%%%%%%%%%
\subsection{Included Files}
\label{sec:include}

%%%%%%%%%%%%%%%%%%%%%%%%%%%%%%%%%%%%%%%%
\DescribeMacro{\childdocmain}
To use the package, add the commands
\begin{center}
\begin{tabular}{l}
|\input{childdoc.def}|\\
|\childdocmain{}|\\
\end{tabular}
\end{center}
at the very top of the main \LaTeX{} file,
in particular \emph{before} the |\documentclass| statement!
The argument of |\childdocmain| should be left empty
(but it must be present).

%%%%%%%%%%%%%%%%%%%%%%%%%%%%%%%%%%%%%%%%
\DescribeMacro{\childdocof}
Furthermore, add the commands
\begin{center}
\begin{tabular}{l}
|\input{childdoc.def}|\\
|\childdocof{|\textit{main}|}|\\
\end{tabular}
\end{center}
at the top of every child file \textit{child}
which is included by |\include{|\textit{child}|}|
from within the main file
(or at least for those files to be compiled individually).
The argument \textit{main} must be the filename of the main file.

There are a couple of
considerations in setting up the main and child documents:

%%%%%%%%%%%%%%%%%%%%%%%%%%%%%%%%%%%%%%%%
\paragraph{Restrictions.}

Please note the following restrictions:
\begin{itemize}
\item
|\childdocmain| must be called with one argument \textit{main}
to ensure compatibility with earlier version of the package.
It must either be empty (|\childdocmain{}|)
or precisely match the filename of the main file in which it is specified.
See \secref{sec:detection} for further information.
\item
The filename \textit{main} must be specified without the |.tex| extension.
\item
The filename \textit{main} is case sensitive
(even in case-insensitive file systems)
due to internal string comparison.
\item
The argument \textit{main} should be fully expanded, it cannot be a macro.
\item
Subdirectories and special characters should be avoided in filenames.
\item
The command |\childdocmain{|\textit{main}|}| must be followed by a whitespace.
It should not be followed immediately by another command
or by a comment mark `|%|'.
This is because the \TeX{} parser reads the token immediately following
the argument of |\childdocmain| and puts it
at the beginning of every child section;
however, a white\-space is ignored.
\end{itemize}

%%%%%%%%%%%%%%%%%%%%%%%%%%%%%%%%%%%%%%%%
\paragraph{Content of Main File.}

It is advisable to place all content in the child files included by |\include|.
Any output contained in the main file will appear in all child documents
unless suppressed manually;
it cannot be suppressed automatically by the |\includeonly| directive
and thus should normally be avoided.
A method to include some content in the main file
by means of conditional processing is described in \secref{sec:conditional}.

%%%%%%%%%%%%%%%%%%%%%%%%%%%%%%%%%%%%%%%%
\paragraph{Page Numbering.}

When only a part of the document is compiled,
the appropriate numbering of pages
(as well as other status parameters)
is determined from the |.aux| files.
The latter contain information from previous passes.
However this information needs to propagate through
all intermediate child documents.
Therefore the page numbering in child documents may well
be inconsistent until the complete document is compiled at least once.

A useful (if unconventional) way to always ensure a consistent
page numbering is to restart the numbering in each child document
and denote the pages by `\textit{child}|.|\textit{page}'
where \textit{child} represents the chapter/section number of the child file.
This can be achieved by the command
|\numberwithin{page}{|\textit{child}|}|
of the \textsf{amsmath} package
where \textit{child} can be |chapter| or |section|
depending on the chosen structuring.
Alternatively, one can modify the macro |\thepage| appropriately
and reset the counter |page| at the start of each child file.

%%%%%%%%%%%%%%%%%%%%%%%%%%%%%%%%%%%%%%%%%%%%%%%%%%%%%%%%%%%%%%%%%%%%%%%%%%%%%%%%
\subsection{Conditional Processing}
\label{sec:conditional}

The package provides a mechanism to compile different versions
of a document. To customise the versions further some conditional processing
can come in handy to distinguish which version is being compiled.
The package provides two macros to describe the compilation context:

%%%%%%%%%%%%%%%%%%%%%%%%%%%%%%%%%%%%%%%%
\DescribeMacro{\ifchilddoc}
The conditional |\ifchilddoc| distinguishes between the compilation of
child documents and the main document:
%
\begin{center}
|\ifchilddoc |\textit{child-code}| |[|\||else |\textit{main-code}]| \||fi|
\end{center}

%%%%%%%%%%%%%%%%%%%%%%%%%%%%%%%%%%%%%%%%
\DescribeMacro{\childdocname}
\DescribeMacro{\childdocjob}
The macro |\childdocname| contains the filename (without extension)
of the main or child file being processed.
Note that |\childdocjob| will always contain the name of the main file.

%%%%%%%%%%%%%%%%%%%%%%%%%%%%%%%%%%%%%%%%
\paragraph{Title Page.}

Conditional processing can be used to include a title or banner page
in the main document when proper precautions are taken.
Importantly, the code in the main file should ensure that the page counter
(as well as other status parameters which are stored in the |.aux| files)
takes the same value after the conditional processing.
Otherwise the page numbers may take divergent values
depending on which part is compiled.

For example, a title page could be declared by:
%
\begin{center}
\begin{tabular}{l}
|\ifchilddoc\||else|\\
|\addtocounter{page}{-1}|\\
\textit{code for title page}\\
|\newpage|\\
|\||fi|
\end{tabular}
\end{center}
%
A banner page for the child documents can be generated by:
%
\begin{center}
\begin{tabular}{l}
|\ifchilddoc|\\
|\addtocounter{page}{-1}|\\
\textit{code for banner page}\\
|\newpage|\\
|\||fi|
\end{tabular}
\end{center}
%
Here one could write a message such as:
\begin{center}
|This is the part \childdocname{} of \childdocjob{}.|
\end{center}

%%%%%%%%%%%%%%%%%%%%%%%%%%%%%%%%%%%%%%%%%%%%%%%%%%%%%%%%%%%%%%%%%%%%%%%%%%%%%%%%
\subsection{Flags}
\label{sec:flags}

The package makes it easy to generate different versions
of the main or child documents.
To this end compilation flags can be defined
and assigned different default values.
They will be particularly useful in conjunction
with the forwarding mechanism described in \secref{sec:forward}.

For example, it may be useful to have a flag |\version|
which can be set to |draft| or |final|.
The document source will contain some conditional code
depending on the value of |\version|.
Suppose further, the flag should default to |final| for the main file
and to |draft| for child files
which is a natural assignment for editing the document.
This is achieved by placing the following code
in the preamble of the main document
(below the |\childdocmain| directive):
%
\begin{center}
\begin{tabular}{l}
|\ifchilddoc|\\
|\providecommand{\version}{draft}|\\
|\||else|\\
|\providecommand{\version}{final}|\\
|\||fi|
\end{tabular}
\end{center}
%
The definition by |\providecommand| makes sure
that previous definitions are not overwritten.
Further statements |\providecommand{\version}{...}|
can thus be added before the above code to override it.

For the main file, one might add a line
(between |\childdocmain| and the above block)
%
\begin{center}
|%\ifchilddoc\||else\providecommand{\version}{draft}\||fi|
\end{center}
%
which can be uncommented to produce a draft version.
Likewise one can add a line to the very top of a child file
(above the |\childdocof{|\textit{main}|}| directive)
%
\begin{center}
|%\providecommand{\version}{final}|
\end{center}
%
which can be uncommented to produce the final version of this child document.

%%%%%%%%%%%%%%%%%%%%%%%%%%%%%%%%%%%%%%%%%%%%%%%%%%%%%%%%%%%%%%%%%%%%%%%%%%%%%%%%
\subsection{Forwarding}
\label{sec:forward}

Different versions of the main or child documents
using compilation flags as described in \secref{sec:flags}
can be (permanently) stored in different files
for convenient compilation, viewing and distribution.
To this end, the package defines a command
to pass on compilation to a different file:

%%%%%%%%%%%%%%%%%%%%%%%%%%%%%%%%%%%%%%%%
\DescribeMacro{\childdocforward}
The command |\childdocforward| redirects processing to
another source file:
%
\begin{center}
\begin{tabular}{l}
|\input{childdoc.def}|\\
|\childdocforward[|\textit{main}|]{|\textit{dest}|}|\\
\end{tabular}
\end{center}
%
The argument \textit{dest} is the destination file
(without extension).
It should be the main file or one of the child files.
Note that further \textsf{childdoc} directives
such as |\childdocof| and |\childdocforward|
in the indicated file will be processed in this form.
The optional argument \textit{main}
passes on directly to the main file \textit{main}
while pretending to compile the child \textit{dest}.
This form behaves as if \textit{dest}
issues |\childdocof{|\textit{main}|}| right away,
and no further \textsf{childdoc} directives will be processed.

%%%%%%%%%%%%%%%%%%%%%%%%%%%%%%%%%%%%%%%%
\DescribeMacro{\...prefix}
In the alternative form |\childdocforwardprefix|,
%
\begin{center}
\begin{tabular}{l}
|\input{childdoc.def}|\\
|\childdocforwardprefix[|\textit{main}|]{|\textit{prefix}|}{|\textit{dest}|}|
\end{tabular}
\end{center}
%
the destination file is determined by a pattern
depending on the current file:
To make this work, the current file must be called
`{\textit{prefix}\hspace{0.2em}\textit{suffix}}'
with \textit{prefix} matching precisely the argument.
Processing is then passed on to the file
`{\textit{dest}\hspace{0.2em}\textit{suffix}}'.
Surely, the same effect is achieved by
directly specifying the
argument `{\textit{dest}\hspace{0.2em}\textit{suffix}}'
in the first form.
However, that requires to set up a different file
for each child. With the alternative form of the command
all these files can have exactly the same content
which simplifies setting them up and maintaining them.

For example, the following file |draft.tex|
with a compilation flag |\version| as described in \secref{sec:flags}
compiles the main document as a draft:
%
\begin{center}
\begin{tabular}{l}
|\def\version{draft}|\\
|\input{childdoc.def}|\\
|\childdocforward{|\textit{main}|}|
\end{tabular}
\end{center}
%
Likewise, the following files |final|\textit{nn}|.tex|
compile the final version of the child document
|child|\textit{nn}|.tex|:
%
\begin{center}
\begin{tabular}{l}
|\def\version{final}|\\
|\input{childdoc.def}|\\
|\childdocforwardprefix{final}{child}|
\end{tabular}
\end{center}
%

Note that when several versions of a main file and/or of each child file
are to be generated, it may be convenient to set up a |Makefile| or
shell script to automatise the process.

%%%%%%%%%%%%%%%%%%%%%%%%%%%%%%%%%%%%%%%%%%%%%%%%%%%%%%%%%%%%%%%%%%%%%%%%%%%%%%%%
\subsection{Command Line Processing}
\label{sec:commandline}

The effect of redirection files can also be achieved by invoking
the \LaTeX{} compiler with a more elaborate command line.
Most conveniently this should be done as part
of a shell script or a |Makefile|.

When using \textsf{childdoc} in the main file, the following
command lines effectively perform a redirection
(note that depending on the shell being used,
backslashes may have to be doubled: `|\|' $\to$ `|\\|'):
%
\begin{center}
|... -jobname "|\textit{target}|" |\\|"|[\textit{flags}]%
|\input{childdoc.def}\childdocforward[|\textit{main}|]{|\textit{dest}|}"|
\end{center}
%
Here \textit{target} is the name of the output file,
\textit{main} is the name of the main file
and \textit{dest} is the name of the main or child file to be processed
(all filenames without extensions).
The optional argument \textit{main} can be omitted
if \textit{main} matches \textit{dest}.
Optionally, compilation \textit{flags} can be defined via |\def| commands.
This command line makes the \TeX{} engine believe
it is compiling the file \textit{target}
whose content is specified as the latter parameter.
The provided code then forwards the processing to
\textit{main} or \textit{dest} as described in \secref{sec:forward}.

%%%%%%%%%%%%%%%%%%%%%%%%%%%%%%%%%%%%%%%%%%%%%%%%%%%%%%%%%%%%%%%%%%%%%%%%%%%%%%%%
\subsection{Include by Input}
\label{sec:input}

Including child documents by |\include| has some restrictions by design.
Most notably, the content of a child document always occupies
its own set of pages; pages cannot be shared between child documents.
Usually, this behaviour makes perfect sense
because each child document contain an essential part of the document.
However, in some situations it may be desirable to compose
a document from a collection of parts
without having mandatory page breaks between then.
For this case, the package
provides a mechanism to include parts
by |\input| which can also be processed individually.
However, by construction this mechanism
requires manual handling of the content to be output.

%%%%%%%%%%%%%%%%%%%%%%%%%%%%%%%%%%%%%%%%
\DescribeMacro{\ifchilddocmanual}
The main file should be prepared as usual, see \secref{sec:include}.
However, the document body must make a distinction
between processing of an individual part and of the main document, e.g.:
%
\begin{center}
\begin{tabular}{l}
|\ifchilddocmanual|\\
|\input{\childdocname}|\\
|\||else|\\
\textit{document body with }|\input{|\textit{part}|}|\\
|\||fi|
\end{tabular}
\end{center}
%
The conditional |\ifchilddocmanual| is true whenever
a part to be included by |\input| is being compiled,
and the name of the part is stored in |\childdocname|.

%%%%%%%%%%%%%%%%%%%%%%%%%%%%%%%%%%%%%%%%
\DescribeMacro{\childdocby}
Each part to be included by |\input| should start with:
%
\begin{center}
\begin{tabular}{l}
|\input{childdoc.def}|\\
|\childdocby{|\textit{main}|}|\\
\end{tabular}
\end{center}
%
The directive |\childdocby| is similar to |\childdocof|
described in \secref{sec:include},
but the subsequent selection of content must be done manually.
To that end, both |\ifchilddoc| and |\ifchilddocmanual|
will be true upon processing of a part,
and the name of the part is stored in |\childdocname|.
Note that |\jobname| will be set to the filename of the current part
so that each part receives an individual |.aux| file
that does not interfere with the |.aux| file(s) of the main document.
This behaviour can be altered by the alternative form
|\childdocby[*]{|\textit{main}|}| (with a non-empty optional argument)
which uses the |.aux| file of the main document
by setting |\jobname| to \textit{main}.

%%%%%%%%%%%%%%%%%%%%%%%%%%%%%%%%%%%%%%%%%%%%%%%%%%%%%%%%%%%%%%%%%%%%%%%%%%%%%%%%
\subsection{Driver Development}
\label{sec:driver}

The \textsf{childdoc} mechanism can also be use for the development
of definition files such as \LaTeX{} styles or classes.
This case differs from the above setup with multiple parts
included by |\include| in that no |\includeonly| should be invoked.
This can be achieved by starting the include file
(before |\ProvidesPackage|) with:
%
\begin{center}
\begin{tabular}{l}
|\input{childdoc.def}|\\
|\childdocforward{|\textit{main}|}|\\
\end{tabular}
\end{center}
%
or alternatively with:
%
\begin{center}
\begin{tabular}{l}
|\input{childdoc.def}|\\
|\childdocby{|\textit{main}|}|\\
\end{tabular}
\end{center}
%
Both forms have slightly different effects as described above.
The main file is prepared as usual, see \secref{sec:include}.

%%%%%%%%%%%%%%%%%%%%%%%%%%%%%%%%%%%%%%%%%%%%%%%%%%%%%%%%%%%%%%%%%%%%%%%%%%%%%%%%
\subsection{Legacy Detection}
\label{sec:detection}

The directive |\childdocmain| in the main file can detect
whether the complete document or merely a child is to be compiled
even without using the directive |\childdocof|.
This method is deprecated because it is less robust
and there is no compelling reason to use it;
it is merely provided for backward compatibility
and it may be removed in future versions.

If the detection mechanism is to be used,
it is mandatory to correctly specify
the filename of the main file as the argument of |\childdocmain|:
%
\begin{center}
\begin{tabular}{l}
|\input{childdoc.def}|\\
|\childdocmain{|\textit{main}|}|\\
\end{tabular}
\end{center}
%
If |\jobname| does not match the argument \textit{main} of |\childdocmain|,
it is assumed that |\jobname| points to the child file to be compiled.
When using |\childdocmain| with the main file specified as argument,
it suffices to start a child file
with just |\input{|\textit{main}|}|
without loading of the package and using |\childdocof|.
If instead all processing is done
with the appropriate \textsf{childdoc} directives,
the argument of \textit{main} of |\childdocmain| can be empty.

An alternative version of the command line processing described
in \secref{sec:commandline} using the detection mechanism reads:
%
\begin{center}
|... -jobname "|\textit{target}|" "|[\textit{flags}]%
[|\def\jobname{|\textit{dest}|}|]|\input{|\textit{main}|}"|
\end{center}

%%%%%%%%%%%%%%%%%%%%%%%%%%%%%%%%%%%%%%%%%%%%%%%%%%%%%%%%%%%%%%%%%%%%%%%%%%%%%%%%
\subsection{Manual Code}
\label{sec:manual}

In case one cannot be certain whether the definitions file |childdoc.def|
is installed on the target \TeX{} distribution
and one prefers not to ship it,
it is conceivable to paste a few relevant commands into the sources.

To that end, drop all statements |\input{childdoc.def}|
and perform the replacements as outlined below.
Instead of |\childdocmain{|\textit{main}|}| add the following code
to the top of the main file:
%
\begin{center}
\begin{tabular}{l}
|\||ifdefined\childdocname\endinput\||fi\newif\ifchilddoc|\\
|\edef\childdocname{\scantokens\expandafter{\jobname\noexpand}}|\\
|\def\childdocmain{|\textit{main}|}\||ifx\childdocmain\childdocname\||else|\\
|\childdoctrue\includeonly{\childdocname}\let\jobname\childdocmain\||fi|\\
\end{tabular}
\end{center}
%
Instead of |\childdocof{|\textit{main}|}| just include the main file
at the top of each child file:
%
\begin{center}
|\input{|\textit{main}|}|
\end{center}
%
A simple redirection |\childdocforward{|\textit{dest}|}| is achieved by:
%
\begin{center}
|\def\jobname{|\textit{dest}|}\input{\jobname}|
\end{center}
%
The redirection with prefix
|\childdocforwardprefix[|\textit{prefix}|]{|\textit{dest}|}|
is accomplished by:
%
\begin{center}
\begin{tabular}{l}
|{\edef\jobname{\scantokens\expandafter{\jobname\noexpand}}|\\
|\def\redirectjob |\textit{prefix}|#1~~~{\gdef\jobname{|\textit{dest}|#1}}|\\
|\expandafter\redirectjob\jobname~~~}\input{\jobname}|
\end{tabular}
\end{center}

In an alternative approach,
child documents can be compiled by a specific command line
without additional code or specific definitions:
%
\begin{center}
|... -jobname "|\textit{target}|" "|[\textit{flags}]%
|\includeonly{|\textit{dest}|}\input{|\textit{main}|}"|
\end{center}
%

%%%%%%%%%%%%%%%%%%%%%%%%%%%%%%%%%%%%%%%%%%%%%%%%%%%%%%%%%%%%%%%%%%%%%%%%%%%%%%%%
%%%%%%%%%%%%%%%%%%%%%%%%%%%%%%%%%%%%%%%%%%%%%%%%%%%%%%%%%%%%%%%%%%%%%%%%%%%%%%%%
\section{Information}

%%%%%%%%%%%%%%%%%%%%%%%%%%%%%%%%%%%%%%%%%%%%%%%%%%%%%%%%%%%%%%%%%%%%%%%%%%%%%%%%
\subsection{Copyright}

Copyright \copyright{} 2017--2018 Niklas Beisert

This work may be distributed and/or modified under the
conditions of the \LaTeX{} Project Public License, either version 1.3
of this license or (at your option) any later version.
The latest version of this license is in
  \url{http://www.latex-project.org/lppl.txt}
and version 1.3 or later is part of all distributions of \LaTeX{}
version 2005/12/01 or later.

This work has the LPPL maintenance status `maintained'.

The Current Maintainer of this work is Niklas Beisert.

This work consists of the files |README.txt|, |childdoc.ins| and |childdoc.dtx|
as well as the derived files |childdoc.def|, |cdocsamp.tex|
with |cdocsch1.tex|, |cdocsch2.tex|, |cdocspt3.tex|, |cdocspt4.tex|,
|cdocsdrf.tex|, |cdocsfn1.tex|, |cdocsfn2.tex|
as well as |childdoc.pdf|.

%%%%%%%%%%%%%%%%%%%%%%%%%%%%%%%%%%%%%%%%%%%%%%%%%%%%%%%%%%%%%%%%%%%%%%%%%%%%%%%%
\subsection{Files and Installation}

The package consists of the files:
%
\begin{center}
\begin{tabular}{ll}
    |README.txt|   & readme file \\
    |childdoc.ins| & installation file \\
    |childdoc.dtx| & source file \\
    |childdoc.def| & definition file \\
    |cdocsamp.tex| & sample main file \\
    |cdocsch1.tex| & sample include file \\
    |cdocsch2.tex| & sample include file \\
    |cdocspt3.tex| & sample part file \\
    |cdocspt4.tex| & sample part file \\
    |cdocsdrf.tex| & sample redirection file \\
    |cdocsfn1.tex| & sample redirection file \\
    |cdocsfn2.tex| & sample redirection file \\
    |childdoc.pdf| & manual
\end{tabular}
\end{center}
%
The distribution consists of the files
|README.txt|, |childdoc.ins| and |childdoc.dtx|.
%
\begin{itemize}
\item
Run (pdf)\LaTeX{} on |childdoc.dtx|
to compile the manual |childdoc.pdf| (this file).
\item
Run \LaTeX{} on |childdoc.ins| to create the definitions file |childdoc.def|
and the sample |cdocsamp.tex| with include files
|cdocsch1.tex|, |cdocsch2.tex|, |cdocspt3.tex|, |cdocspt4.tex|,
|cdocsdrf.tex|, |cdocsfn1.tex|, |cdocsfn2.tex|.
Then copy the file |childdoc.def| to an appropriate directory of your \LaTeX{}
distribution, e.g.\ \textit{texmf-root}|/tex/latex/childdoc|.
\end{itemize}

%%%%%%%%%%%%%%%%%%%%%%%%%%%%%%%%%%%%%%%%%%%%%%%%%%%%%%%%%%%%%%%%%%%%%%%%%%%%%%%%
\subsection{Related CTAN Packages}

There are several other packages which offer a similar functionality:
%
\begin{itemize}
\item
The packages
\href{http://ctan.org/pkg/docmute}{\textsf{docmute}},
\href{http://ctan.org/pkg/includex}{\textsf{includex}} and
\href{http://ctan.org/pkg/standalone}{\textsf{standalone}}
provide commands to include only the document body of
a child file thus allowing both files to be compiled individually.
\item
The packages \href{http://ctan.org/pkg/subdocs}{\textsf{subdocs}}
and \href{http://ctan.org/pkg/subfiles}{\textsf{subfiles}}
provide structures in which the main and child documents can be
encapsulated and allowing them to be compiled individually.
The inclusion mechanism is different from the conventional |\include|.
\item
The package \href{http://ctan.org/pkg/combine}{\textsf{combine}}
is an elaborate solution to combine several documents into one.
\end{itemize}
%
See also the CTAN topic \href{http://ctan.org/topic/subdocs}{\textsf{subdocs}}
for further related packages.
The present package differs from the above solutions in that
a document structure constructed with the conventional |\include| mechanism
just needs two extra commands at the top of every file
such that all constituent files can be compiled individually.

%%%%%%%%%%%%%%%%%%%%%%%%%%%%%%%%%%%%%%%%%%%%%%%%%%%%%%%%%%%%%%%%%%%%%%%%%%%%%%%%
%\subsection{Feature Suggestions}
%
%The following is a list of features which may be useful for future
%versions of this package:
%%
%\begin{itemize}
%\item
%\ldots
%\end{itemize}

%%%%%%%%%%%%%%%%%%%%%%%%%%%%%%%%%%%%%%%%%%%%%%%%%%%%%%%%%%%%%%%%%%%%%%%%%%%%%%%%
\subsection{Revision History}

%%%%%%%%%%%%%%%%%%%%%%%%%%%%%%%%%%%%%%%%
\paragraph{v2.0:} 2018/12/30

\begin{itemize}
\item
immediate forward processing
\item
added |\childdocby| mechanism
\item
manual restructured
\end{itemize}

%%%%%%%%%%%%%%%%%%%%%%%%%%%%%%%%%%%%%%%%
\paragraph{v1.6:} 2018/01/17

\begin{itemize}
\item
application for development of include files
\item
corrections to manual
\end{itemize}

%%%%%%%%%%%%%%%%%%%%%%%%%%%%%%%%%%%%%%%%
\paragraph{v1.5:} 2017/05/21

\begin{itemize}
\item
more complete structuring introduced
\item
|\childdocof| introduced
\item
|\childdoc| renamed to |\childdocmain|
\item
|\childredirect| renamed to |\childdocforward| and |\childdocforwardprefix|
and functionality expanded
\end{itemize}

%%%%%%%%%%%%%%%%%%%%%%%%%%%%%%%%%%%%%%%%
\paragraph{v1.0:} 2017/04/27

\begin{itemize}
\item
manual and install package
\item
first version published on CTAN
\end{itemize}

%%%%%%%%%%%%%%%%%%%%%%%%%%%%%%%%%%%%%%%%
\paragraph{v0.6:} 2017/04/26

\begin{itemize}
\item
redirection mechanism added
\end{itemize}

%%%%%%%%%%%%%%%%%%%%%%%%%%%%%%%%%%%%%%%%
\paragraph{v0.5:} 2017/04/26

\begin{itemize}
\item
functionality in definition file
\end{itemize}


%%%%%%%%%%%%%%%%%%%%%%%%%%%%%%%%%%%%%%%%%%%%%%%%%%%%%%%%%%%%%%%%%%%%%%%%%%%%%%%%
%%%%%%%%%%%%%%%%%%%%%%%%%%%%%%%%%%%%%%%%%%%%%%%%%%%%%%%%%%%%%%%%%%%%%%%%%%%%%%%%
%%%%%%%%%%%%%%%%%%%%%%%%%%%%%%%%%%%%%%%%%%%%%%%%%%%%%%%%%%%%%%%%%%%%%%%%%%%%%%%%
\appendix

\settowidth\MacroIndent{\rmfamily\scriptsize 000\ }

 \DocInput{childdoc.dtx}

\end{document}
%</driver>
% \fi
%
% %%%%%%%%%%%%%%%%%%%%%%%%%%%%%%%%%%%%%%%%%%%%%%%%%%%%%%%%%%%%%%%%%%%%%%%%%%%%%%
% %%%%%%%%%%%%%%%%%%%%%%%%%%%%%%%%%%%%%%%%%%%%%%%%%%%%%%%%%%%%%%%%%%%%%%%%%%%%%%
% \section{Sample}
%\iffalse
%<*samplemain>
%\fi
%
% The following presents a sample document
% with two chapters, two parts, a title page,
% a compile flag as well as three forwarding files to set the flag.
% It consists of eight |.tex| files:
% \begin{center}
% \begin{tabular}{ll}
% |cdocsamp.tex|&main file\\
% |cdocsch1.tex|&include file for chapter 1\\
% |cdocsch2.tex|&include file for chapter 2\\
% |cdocspt3.tex|&include file for part 3\\
% |cdocspt4.tex|&include file for part 4\\
% |cdocsdrf.tex|&forwarding file for main file in draft mode\\
% |cdocsfi1.tex|&forwarding file for final version of chapter 1\\
% |cdocsfi2.tex|&forwarding file for final version of chapter 2\\
% \end{tabular}
% \end{center}
% Each of the eight files can be compiled directly by the \LaTeX{} compiler.
%
% %%%%%%%%%%%%%%%%%%%%%%%%%%%%%%%%%%%%%%
% \paragraph{Main File.}
%
% The main file is called |cdocsamp.tex|.
%
% Load the \textsf{childdoc} definitions and
% declare the filename for the main document:
%    \begin{macrocode}
\input{childdoc.def}
\childdocmain{}
%    \end{macrocode}

% Optional override for |\version| flag:
%    \begin{macrocode}
%%\ifchilddoc\else\providecommand{\version}{draft}\fi
%    \end{macrocode}

% Define the default values for the |\version| flag
% (|final| for the main file and |draft| for childs):
%    \begin{macrocode}
\ifchilddoc
\providecommand{\version}{draft}
\else
\providecommand{\version}{final}
\fi
%    \end{macrocode}

% Load the standard document class:
%    \begin{macrocode}
\documentclass[12pt]{article}
%    \end{macrocode}

% Start the document body:
%    \begin{macrocode}
\begin{document}
%    \end{macrocode}

% Declare a title page.
% Print title, part of document being processed and version flag:
%    \begin{macrocode}
\addtocounter{page}{-1}
\begin{center}
{\LARGE\bfseries{}childdoc example\par}
\vspace{1cm}
\ifchilddoc
\ifchilddocmanual part\else chapter\fi:
`\childdocname' of `\childdocjob'\par
\else
main document: `\childdocjob'\par
\fi
version: \version\par
\end{center}
\newpage
%    \end{macrocode}

% Manually include selected file,
% otherwise process as usual:
%    \begin{macrocode}
\ifchilddocmanual
\section*{part `\childdocname'}
\input{\childdocname}
\else
%    \end{macrocode}

% Include the two chapters:
%    \begin{macrocode}
\include{cdocsch1}
\include{cdocsch2}
%    \end{macrocode}

% Include the two parts unless only chapters should be displayed:
%    \begin{macrocode}
\ifchilddoc\else
\section{part three}
\input{cdocspt3}
\section{part four}
\input{cdocspt4}
\fi
%    \end{macrocode}

% Process as usual until here:
%    \begin{macrocode}
\fi
%    \end{macrocode}

% End of document body:
%    \begin{macrocode}
\end{document}
%    \end{macrocode}
%\iffalse
%</samplemain>
%\fi
%
% %%%%%%%%%%%%%%%%%%%%%%%%%%%%%%%%%%%%%%
% \paragraph{Chapter Include Files.}
%
% The include files are called |cdocsch1.tex| and |cdocsch2.tex|.
%
%\iffalse
%<*samplechap1|samplechap2>
%\fi

% Optional override for |\version| flag:
%    \begin{macrocode}
%%\providecommand{\version}{final}
%    \end{macrocode}

% Include the main document:
%    \begin{macrocode}
\input{childdoc.def}
\childdocof{cdocsamp}
%    \end{macrocode}

%\iffalse
%</samplechap1|samplechap2>
%\fi
%
%\iffalse
%<*samplechap1>
%\fi
% Some text for chapter 1:
%    \begin{macrocode}
\section{one}
some text in chapter one
%    \end{macrocode}

%\iffalse
%</samplechap1>
%\fi
% Some text for chapter 2:
%\iffalse
%<*samplechap2>
%\fi
%    \begin{macrocode}
\section{two}
more text in chapter two
%    \end{macrocode}

%\iffalse
%</samplechap2>
%\fi
%
% %%%%%%%%%%%%%%%%%%%%%%%%%%%%%%%%%%%%%%
% \paragraph{Part Include Files.}
%
% The include files are called |cdocspt3.tex| and |cdocspt4.tex|.
%
%\iffalse
%<*samplepart3|samplepart4>
%\fi

% Optional override for |\version| flag:
%    \begin{macrocode}
%%\providecommand{\version}{final}
%    \end{macrocode}

% Include the main document:
%    \begin{macrocode}
\input{childdoc.def}
\childdocby{cdocsamp}
%    \end{macrocode}

%\iffalse
%</samplepart3|samplepart4>
%\fi
%
%\iffalse
%<*samplepart3>
%\fi
% Some text for part 3:
%    \begin{macrocode}
some text in part three
%    \end{macrocode}

%\iffalse
%</samplepart3>
%\fi
% Some text for part 4:
%\iffalse
%<*samplepart4>
%\fi
%    \begin{macrocode}
more text in part four
%    \end{macrocode}

%\iffalse
%</samplepart4>
%\fi
%
% %%%%%%%%%%%%%%%%%%%%%%%%%%%%%%%%%%%%%%
% \paragraph{Forwarding for a Complete Draft.}
%
% The following forwarding file |cdocsdrf.tex|
% compiles the main document in draft mode:
%\iffalse
%<*sampledraft>
%\fi
%    \begin{macrocode}
\def\version{draft}
\input{childdoc.def}
\childdocforward{cdocsamp}
%    \end{macrocode}

%\iffalse
%</sampledraft>
%\fi
%
% %%%%%%%%%%%%%%%%%%%%%%%%%%%%%%%%%%%%%%
% \paragraph{Forwarding for Final Version of the Chapters.}
%
% The following forwarding files |cdocsfn1.tex| and |cdocsfn2.tex|
% (with identical content)
% compile the final versions of the child documents
% |cdocsch1.tex| and |cdocsch2.tex|, respectively:
%\iffalse
%<*samplefinal>
%\fi
%    \begin{macrocode}
\def\version{final}
\input{childdoc.def}
\childdocforwardprefix[cdocsamp]{cdocsfn}{cdocsch}
%    \end{macrocode}

%\iffalse
%</samplefinal>
%\fi
%
% %%%%%%%%%%%%%%%%%%%%%%%%%%%%%%%%%%%%%%
% \paragraph{Command Line Processing.}
%
% The following three command lines generate the output files
% |cdocscld|, |cdocscl1| and |cdocscl2|
% which should be identical to
% |cdocsdrf|, |cdocsch1| and |cdocsfn2|, respectively:
% \begin{center}
% \begin{tabular}{l}
% |latex -jobname cdocscld \|\\
% |  "\def\version{draft}\input{childdoc.def}\childdocforward{cdocsamp}"|\\
% |latex -jobname cdocscl1 \|\\
% |  "\input{childdoc.def}\childdocforward[cdocsamp]{cdocsch1}"|\\
% |latex -jobname cdocscl2 \|\\
% |  "\def\version{final}\input{childdoc.def}\childdocforward{cdocsch2}"|
% \end{tabular}
% \end{center}
% Note that the trailing backslash on each first line
% merely continues the input to the second line
% (for convenient cut ant paste).
% Furthermore, the command |latex| can be replaced by any
% of its alternative versions such as |pdflatex|.
%
% %%%%%%%%%%%%%%%%%%%%%%%%%%%%%%%%%%%%%%%%%%%%%%%%%%%%%%%%%%%%%%%%%%%%%%%%%%%%%%
% %%%%%%%%%%%%%%%%%%%%%%%%%%%%%%%%%%%%%%%%%%%%%%%%%%%%%%%%%%%%%%%%%%%%%%%%%%%%%%
% \section{Implementation}
%\iffalse
%<*package>
%\fi
%
% This section describes the definitions file |childdoc.def|.

% The definitions cannot be loaded using |\usepackage| or |\RequirePackage|
% which has a mechanism to prevent loading a style file more than once.
% When loading the definitions by means of |\input|
% multiple instances have to be prevented manually:
%\iffalse
%This code needs to be before the `\ProvidesFile' directive
%which is defined at the beginning of this file.
%Therefore it is also placed there and commented out here.
%</package>
%<*discard>
%\fi
%    \begin{macrocode}
\ifdefined\childdocmain\endinput\fi
%    \end{macrocode}
%\iffalse
%</discard>
%<*package>
%\fi
%
% \macro{\ifchilddoc}
% \macro{\ifchilddocmanual}
% The conditional |\ifchilddoc| tells whether a
% child (true) or main (false) document is being compiled.
% The conditional |\ifchilddocmanual| tells whether
% the |\includeonly| mechanism is used (false) or
% the selection of child files must be performed manually (true).
% The definitions initialise to false:
%    \begin{macrocode}
\newif\ifchilddoc
\newif\ifchilddocmanual
%    \end{macrocode}

% \macro{\childdocname}
% \macro{\childdocjob}
% The macro |\childdocname| stores the name of the main document
% to be compiled. The macro |\childdocjob| stores the name of
% the document on which the \LaTeX{} compiler was originally invoked.
% The content of |\jobname| cannot be compared
% to filenames specified in the source due to different catcodes.
% The following code rescans |\jobname|, stores the result
% in |\childdocname| and saves a copy in |\childdocjob|:
%    \begin{macrocode}
\edef\childdocname{\scantokens\expandafter{\jobname\noexpand}}
\let\childdocjob\childdocname
%    \end{macrocode}

% \macro{\childdocdisable}
% The macro |\childdocdisable| prevents the main file
% from being processed more than once.
% At this stage, the main document command |\childdocmain|
% is assumed to be called once again where it should do nothing.
% Any subsequent call to it should prevent
% a secondary processing of the main document
% It overwrites the forwarding commands
% |\childdocof| and |\childdocforward|
% with empty macros to prevent further inclusions of the main document:
%    \begin{macrocode}
\newcommand{\childdocdisable}
{
  \renewcommand{\childdocmain}[1]{\renewcommand{\childdocmain}[1]{\endinput}}
  \renewcommand{\childdocof}[1]{}
  \renewcommand{\childdocby}[2][]{}
  \renewcommand{\childdocforward}[2][]{}
  \renewcommand{\childdocdisable}{}
}
%    \end{macrocode}

% \macro{\childdocmain}
% The macro |\childdocmain| is to be called at the top of the main file
% with nothing or the main filename (without extension) as argument.
% First, it breaks loops.
% If the argument is not empty and does not match |\childdocname|
% (which is set by the first inclusion of |childdoc.def|),
% |\ifchilddoc| is set to true, |\includeonly| is applied to the child file
% and |\jobname| is set to the main file
% (for proper handling of |.aux| files):
%    \begin{macrocode}
\newcommand{\childdocmain}[1]
{
  \childdocdisable\childdocmain{}
  \if?#1?\else
    \begingroup
      \def\childdoctmp{#1}
      \ifx\childdoctmp\childdocname
        \def\childdoctmp{}
      \else
        \def\childdoctmp
        {
          \childdoctrue
          \includeonly{\childdocname}
          \def\childdocjob{#1}
          \def\jobname{#1}
        }
      \fi
      \expandafter
    \endgroup
    \childdoctmp
  \fi
}
%    \end{macrocode}

% \macro{\childdocof}
% The command |\childdocof| redirects
% compilation to the main file |#1|.
%    \begin{macrocode}
\newcommand{\childdocof}[1]
{
  \childdocdisable
  \childdoctrue
  \includeonly{\childdocname}
  \def\jobname{#1}
  \def\childdocjob{#1}
  \input{#1}
}
%    \end{macrocode}

% \macro{\childdocby}
% The command |\childdocby| ....
%    \begin{macrocode}
\newcommand{\childdocby}[2][]
{
  \childdocdisable
  \childdoctrue
  \childdocmanualtrue
  \if?#1?\else
    \def\jobname{#2}
  \fi
  \def\childdocjob{#2}
  \input{#2}
  \endinput
}
%    \end{macrocode}

% \macro{\childdocforward}
% The command |\childdocforward| redirects
% compilation to the main file or
% (if the optional argument is given) a child file.
% Parameters are set as if the main file
% or a child file starting with |\childdocof| was compiled.
% Then compilation is handed over to the main file:
%    \begin{macrocode}
\newcommand{\childdocforward}[2][]
{
  \begingroup
    \if?#1?
      \def\childdoctmp
      {
        \def\childdocname{#2}
        \def\childdocjob{#2}
        \def\jobname{#2}
        \input{#2}
        \endinput
      }
    \else
      \def\childdoctmp
      {
        \childdocdisable
        \def\childdocname{#2}
        \childdoctrue
        \includeonly{#2}
        \def\childdocjob{#1}
        \def\jobname{#1}
        \input{#1}
        \endinput
      }
    \fi
    \expandafter
  \endgroup
  \childdoctmp
}
%    \end{macrocode}

% \macro{\childdocforwardprefix}
% The command |\childdocforwardprefix| redirects
% compilation to the main or a child file by means of a pattern.
% The prefix |#1| in the current filename is replaced by |#2|
% and the suffix of the current filename is kept
% (it is assumed that the filename does not contain the substring `|~~~|'
% which is used as a delimiter).
% Compilation is handed over to the new file by |\childdocforward|:
%    \begin{macrocode}
\newcommand{\childdocforwardprefix}[3][]
{
  \begingroup
    \def\childdocextract #2##1~~~{\def\childdoctmp{\childdocforward[#1]{#3##1}}}
    \expandafter\childdocextract\childdocname~~~
    \expandafter
  \endgroup
  \childdoctmp
}
%    \end{macrocode}

% \macro{\childdoc}
% The deprecated macro |\childdoc| is a legacy version of |\childdocmain|:
%    \begin{macrocode}
\newcommand{\childdoc}{\childdocmain}
%    \end{macrocode}

% \macro{\childdocredirect}
% The deprecated macro |\childdocredirect| is a legacy version
% of |\childdocforward| and |\childdocforwardprefix|:
%    \begin{macrocode}
\newcommand{\childdocredirect}[2][]
{
  \begingroup
    \if?#1?
      \def\childdoctmp{\childdocforward{#2}}
    \else
      \def\childdoctmp{\childdocforwardprefix{#1}{#2}}
    \fi
    \expandafter
  \endgroup
  \childdoctmp
}
%    \end{macrocode}

%\iffalse
%</package>
%\fi
%
\endinput

\childdocforward{cdocsamp}
%    \end{macrocode}

%\iffalse
%</sampledraft>
%\fi
%
% %%%%%%%%%%%%%%%%%%%%%%%%%%%%%%%%%%%%%%
% \paragraph{Forwarding for Final Version of the Chapters.}
%
% The following forwarding files |cdocsfn1.tex| and |cdocsfn2.tex|
% (with identical content)
% compile the final versions of the child documents
% |cdocsch1.tex| and |cdocsch2.tex|, respectively:
%\iffalse
%<*samplefinal>
%\fi
%    \begin{macrocode}
\def\version{final}
% \iffalse
%
% childdoc.dtx Copyright (C) 2017-2018 Niklas Beisert
%
% This work may be distributed and/or modified under the
% conditions of the LaTeX Project Public License, either version 1.3
% of this license or (at your option) any later version.
% The latest version of this license is in
%   http://www.latex-project.org/lppl.txt
% and version 1.3 or later is part of all distributions of LaTeX
% version 2005/12/01 or later.
%
% This work has the LPPL maintenance status `maintained'.
%
% The Current Maintainer of this work is Niklas Beisert.
%
% This work consists of the files childdoc.dtx and childdoc.ins
% and the derived files childdoc.def and cdocsamp.tex with
% cdocsch1.tex, cdocsch2.tex, cdocsdrf.tex, cdocsfn1.tex, cdocsfn2.tex.
%
%<package>\ifdefined\childdocmain\endinput\fi
%<package>\ProvidesFile{childdoc.def}[2018/12/30 v2.0 child document driver]
%<samplemain>\ProvidesFile{cdocsamp.tex}[2018/12/30 v2.0 sample for childdoc]
%<*driver>
%\ProvidesFile{childdoc.drv}[2018/12/30 v2.0 childdoc reference manual file]
\PassOptionsToClass{10pt,a4paper}{article}
\documentclass{ltxdoc}

\usepackage[margin=35mm]{geometry}
\usepackage{hyperref}
\usepackage{hyperxmp}
\usepackage[usenames]{color}

\hypersetup{colorlinks=true}
\hypersetup{pdfstartview=FitH}
\hypersetup{pdfpagemode=UseNone}
\hypersetup{pdfsource={}}
\hypersetup{pdflang={en-UK}}
\hypersetup{pdfcopyright={Copyright 2017-2018 Niklas Beisert.
  This work may be distributed and/or modified under the
  conditions of the LaTeX Project Public License, either version 1.3
  of this license or (at your option) any later version.}}
\hypersetup{pdflicenseurl={http://www.latex-project.org/lppl.txt}}
\hypersetup{pdfcontactaddress={ETH Zurich, ITP, HIT K,
  Wolfgang-Pauli-Strasse 27}}
\hypersetup{pdfcontactpostcode={8093}}
\hypersetup{pdfcontactcity={Zurich}}
\hypersetup{pdfcontactcountry={Switzerland}}
\hypersetup{pdfcontactemail={nbeisert@itp.phys.ethz.ch}}
\hypersetup{pdfcontacturl={http://people.phys.ethz.ch/\xmptilde nbeisert/}}

\newcommand{\secref}[1]{\hyperref[#1]{section \ref*{#1}}}

\parskip1ex
\parindent0pt
\let\olditemize\itemize
\def\itemize{\olditemize\parskip0pt}

\begin{document}

\title{The \textsf{childdoc} Package}
\hypersetup{pdftitle={The childdoc Package}}
\author{Niklas Beisert\\[2ex]
  Institut f\"ur Theoretische Physik\\
  Eidgen\"ossische Technische Hochschule Z\"urich\\
  Wolfgang-Pauli-Strasse 27, 8093 Z\"urich, Switzerland\\[1ex]
  \href{mailto:nbeisert@itp.phys.ethz.ch}
  {\texttt{nbeisert@itp.phys.ethz.ch}}}
\hypersetup{pdfauthor={Niklas Beisert}}
\hypersetup{pdfsubject={Manual for the LaTeX2e Package childdoc}}
\date{30 December 2018, \textsf{v2.0}}
\maketitle

\begin{abstract}\noindent
\textsf{childdoc} is a \LaTeXe{} package
that enables the direct compilation
of document sections included by |\include|
to individual files.
\end{abstract}

\begingroup
\parskip0ex
\tableofcontents
\endgroup

%%%%%%%%%%%%%%%%%%%%%%%%%%%%%%%%%%%%%%%%%%%%%%%%%%%%%%%%%%%%%%%%%%%%%%%%%%%%%%%%
%%%%%%%%%%%%%%%%%%%%%%%%%%%%%%%%%%%%%%%%%%%%%%%%%%%%%%%%%%%%%%%%%%%%%%%%%%%%%%%%
\section{Introduction}

\LaTeX{} provides a mechanism to structure a large document (such as a book)
into a main file and several child files (containing the chapters)
using the |\include| command.
This mechanism is beneficial for documents
which span hundreds of pages in order to
make the source file(s) more manageable.
Moreover, compilation can be restricted to
selected child files by means of the |\includeonly| command.
The latter feature can be used to reduce the compilation time while editing
(this was significantly more useful in the earlier days of \LaTeX{})
or to generate a smaller document which is easier to navigate.
Another application of |\includeonly| is to generate
documents consisting of selected parts of the complete document.

However, there are a few drawbacks of the plain |\include| mechanism:
\begin{itemize}
\item
The child files cannot be compiled on their own,
they can only be compiled via the main file.
A naive editing environment
(such as a text editor with an option
to have the current file processed by \LaTeX)
may require one to switch to the main file before compiling;
attempting to compile the child file produces errors.
\item
The main file must be modified (each time)
to adjust the |\includeonly| command
to the present needs. This easily leaves the main file in a messy state.
\item
The generated document will always carry the filename
of the main document. This is inconvenient if
several child files are to be compiled and
to be kept for distribution.
\end{itemize}

The present package provides a simple interface
to make child files individually compilable by \LaTeX{}.
Compiling a child file then has the same effect as compiling
the main file with an |\includeonly| command
to select the appropriate child.
Moreover the generated document will carry the name of the child
rather than the main file.
This resolves all three above issues.

This feature is meant to make the editing of books,
thesis documents and lecture notes somewhat more convenient.
However, the package can also be used efficiently for
composing a series of documents (such as exercise sheets)
which are typically distributed individually.
It then assists the author in generating the individual documents
(potentially in different versions)
as well as a document containing the collected series.
Another application is in developing style files
or other kinds of included material
where compilation of the style file could redirect
to a sample or test file.

%%%%%%%%%%%%%%%%%%%%%%%%%%%%%%%%%%%%%%%%%%%%%%%%%%%%%%%%%%%%%%%%%%%%%%%%%%%%%%%%
%%%%%%%%%%%%%%%%%%%%%%%%%%%%%%%%%%%%%%%%%%%%%%%%%%%%%%%%%%%%%%%%%%%%%%%%%%%%%%%%
\section{Usage}

First of all, the package \textsf{childdoc} is \emph{not} a standard
\LaTeXe{} |.sty| style file! Therefore it needs to be invoked in
a non-standard way.

%%%%%%%%%%%%%%%%%%%%%%%%%%%%%%%%%%%%%%%%%%%%%%%%%%%%%%%%%%%%%%%%%%%%%%%%%%%%%%%%
\subsection{Included Files}
\label{sec:include}

%%%%%%%%%%%%%%%%%%%%%%%%%%%%%%%%%%%%%%%%
\DescribeMacro{\childdocmain}
To use the package, add the commands
\begin{center}
\begin{tabular}{l}
|\input{childdoc.def}|\\
|\childdocmain{}|\\
\end{tabular}
\end{center}
at the very top of the main \LaTeX{} file,
in particular \emph{before} the |\documentclass| statement!
The argument of |\childdocmain| should be left empty
(but it must be present).

%%%%%%%%%%%%%%%%%%%%%%%%%%%%%%%%%%%%%%%%
\DescribeMacro{\childdocof}
Furthermore, add the commands
\begin{center}
\begin{tabular}{l}
|\input{childdoc.def}|\\
|\childdocof{|\textit{main}|}|\\
\end{tabular}
\end{center}
at the top of every child file \textit{child}
which is included by |\include{|\textit{child}|}|
from within the main file
(or at least for those files to be compiled individually).
The argument \textit{main} must be the filename of the main file.

There are a couple of
considerations in setting up the main and child documents:

%%%%%%%%%%%%%%%%%%%%%%%%%%%%%%%%%%%%%%%%
\paragraph{Restrictions.}

Please note the following restrictions:
\begin{itemize}
\item
|\childdocmain| must be called with one argument \textit{main}
to ensure compatibility with earlier version of the package.
It must either be empty (|\childdocmain{}|)
or precisely match the filename of the main file in which it is specified.
See \secref{sec:detection} for further information.
\item
The filename \textit{main} must be specified without the |.tex| extension.
\item
The filename \textit{main} is case sensitive
(even in case-insensitive file systems)
due to internal string comparison.
\item
The argument \textit{main} should be fully expanded, it cannot be a macro.
\item
Subdirectories and special characters should be avoided in filenames.
\item
The command |\childdocmain{|\textit{main}|}| must be followed by a whitespace.
It should not be followed immediately by another command
or by a comment mark `|%|'.
This is because the \TeX{} parser reads the token immediately following
the argument of |\childdocmain| and puts it
at the beginning of every child section;
however, a white\-space is ignored.
\end{itemize}

%%%%%%%%%%%%%%%%%%%%%%%%%%%%%%%%%%%%%%%%
\paragraph{Content of Main File.}

It is advisable to place all content in the child files included by |\include|.
Any output contained in the main file will appear in all child documents
unless suppressed manually;
it cannot be suppressed automatically by the |\includeonly| directive
and thus should normally be avoided.
A method to include some content in the main file
by means of conditional processing is described in \secref{sec:conditional}.

%%%%%%%%%%%%%%%%%%%%%%%%%%%%%%%%%%%%%%%%
\paragraph{Page Numbering.}

When only a part of the document is compiled,
the appropriate numbering of pages
(as well as other status parameters)
is determined from the |.aux| files.
The latter contain information from previous passes.
However this information needs to propagate through
all intermediate child documents.
Therefore the page numbering in child documents may well
be inconsistent until the complete document is compiled at least once.

A useful (if unconventional) way to always ensure a consistent
page numbering is to restart the numbering in each child document
and denote the pages by `\textit{child}|.|\textit{page}'
where \textit{child} represents the chapter/section number of the child file.
This can be achieved by the command
|\numberwithin{page}{|\textit{child}|}|
of the \textsf{amsmath} package
where \textit{child} can be |chapter| or |section|
depending on the chosen structuring.
Alternatively, one can modify the macro |\thepage| appropriately
and reset the counter |page| at the start of each child file.

%%%%%%%%%%%%%%%%%%%%%%%%%%%%%%%%%%%%%%%%%%%%%%%%%%%%%%%%%%%%%%%%%%%%%%%%%%%%%%%%
\subsection{Conditional Processing}
\label{sec:conditional}

The package provides a mechanism to compile different versions
of a document. To customise the versions further some conditional processing
can come in handy to distinguish which version is being compiled.
The package provides two macros to describe the compilation context:

%%%%%%%%%%%%%%%%%%%%%%%%%%%%%%%%%%%%%%%%
\DescribeMacro{\ifchilddoc}
The conditional |\ifchilddoc| distinguishes between the compilation of
child documents and the main document:
%
\begin{center}
|\ifchilddoc |\textit{child-code}| |[|\||else |\textit{main-code}]| \||fi|
\end{center}

%%%%%%%%%%%%%%%%%%%%%%%%%%%%%%%%%%%%%%%%
\DescribeMacro{\childdocname}
\DescribeMacro{\childdocjob}
The macro |\childdocname| contains the filename (without extension)
of the main or child file being processed.
Note that |\childdocjob| will always contain the name of the main file.

%%%%%%%%%%%%%%%%%%%%%%%%%%%%%%%%%%%%%%%%
\paragraph{Title Page.}

Conditional processing can be used to include a title or banner page
in the main document when proper precautions are taken.
Importantly, the code in the main file should ensure that the page counter
(as well as other status parameters which are stored in the |.aux| files)
takes the same value after the conditional processing.
Otherwise the page numbers may take divergent values
depending on which part is compiled.

For example, a title page could be declared by:
%
\begin{center}
\begin{tabular}{l}
|\ifchilddoc\||else|\\
|\addtocounter{page}{-1}|\\
\textit{code for title page}\\
|\newpage|\\
|\||fi|
\end{tabular}
\end{center}
%
A banner page for the child documents can be generated by:
%
\begin{center}
\begin{tabular}{l}
|\ifchilddoc|\\
|\addtocounter{page}{-1}|\\
\textit{code for banner page}\\
|\newpage|\\
|\||fi|
\end{tabular}
\end{center}
%
Here one could write a message such as:
\begin{center}
|This is the part \childdocname{} of \childdocjob{}.|
\end{center}

%%%%%%%%%%%%%%%%%%%%%%%%%%%%%%%%%%%%%%%%%%%%%%%%%%%%%%%%%%%%%%%%%%%%%%%%%%%%%%%%
\subsection{Flags}
\label{sec:flags}

The package makes it easy to generate different versions
of the main or child documents.
To this end compilation flags can be defined
and assigned different default values.
They will be particularly useful in conjunction
with the forwarding mechanism described in \secref{sec:forward}.

For example, it may be useful to have a flag |\version|
which can be set to |draft| or |final|.
The document source will contain some conditional code
depending on the value of |\version|.
Suppose further, the flag should default to |final| for the main file
and to |draft| for child files
which is a natural assignment for editing the document.
This is achieved by placing the following code
in the preamble of the main document
(below the |\childdocmain| directive):
%
\begin{center}
\begin{tabular}{l}
|\ifchilddoc|\\
|\providecommand{\version}{draft}|\\
|\||else|\\
|\providecommand{\version}{final}|\\
|\||fi|
\end{tabular}
\end{center}
%
The definition by |\providecommand| makes sure
that previous definitions are not overwritten.
Further statements |\providecommand{\version}{...}|
can thus be added before the above code to override it.

For the main file, one might add a line
(between |\childdocmain| and the above block)
%
\begin{center}
|%\ifchilddoc\||else\providecommand{\version}{draft}\||fi|
\end{center}
%
which can be uncommented to produce a draft version.
Likewise one can add a line to the very top of a child file
(above the |\childdocof{|\textit{main}|}| directive)
%
\begin{center}
|%\providecommand{\version}{final}|
\end{center}
%
which can be uncommented to produce the final version of this child document.

%%%%%%%%%%%%%%%%%%%%%%%%%%%%%%%%%%%%%%%%%%%%%%%%%%%%%%%%%%%%%%%%%%%%%%%%%%%%%%%%
\subsection{Forwarding}
\label{sec:forward}

Different versions of the main or child documents
using compilation flags as described in \secref{sec:flags}
can be (permanently) stored in different files
for convenient compilation, viewing and distribution.
To this end, the package defines a command
to pass on compilation to a different file:

%%%%%%%%%%%%%%%%%%%%%%%%%%%%%%%%%%%%%%%%
\DescribeMacro{\childdocforward}
The command |\childdocforward| redirects processing to
another source file:
%
\begin{center}
\begin{tabular}{l}
|\input{childdoc.def}|\\
|\childdocforward[|\textit{main}|]{|\textit{dest}|}|\\
\end{tabular}
\end{center}
%
The argument \textit{dest} is the destination file
(without extension).
It should be the main file or one of the child files.
Note that further \textsf{childdoc} directives
such as |\childdocof| and |\childdocforward|
in the indicated file will be processed in this form.
The optional argument \textit{main}
passes on directly to the main file \textit{main}
while pretending to compile the child \textit{dest}.
This form behaves as if \textit{dest}
issues |\childdocof{|\textit{main}|}| right away,
and no further \textsf{childdoc} directives will be processed.

%%%%%%%%%%%%%%%%%%%%%%%%%%%%%%%%%%%%%%%%
\DescribeMacro{\...prefix}
In the alternative form |\childdocforwardprefix|,
%
\begin{center}
\begin{tabular}{l}
|\input{childdoc.def}|\\
|\childdocforwardprefix[|\textit{main}|]{|\textit{prefix}|}{|\textit{dest}|}|
\end{tabular}
\end{center}
%
the destination file is determined by a pattern
depending on the current file:
To make this work, the current file must be called
`{\textit{prefix}\hspace{0.2em}\textit{suffix}}'
with \textit{prefix} matching precisely the argument.
Processing is then passed on to the file
`{\textit{dest}\hspace{0.2em}\textit{suffix}}'.
Surely, the same effect is achieved by
directly specifying the
argument `{\textit{dest}\hspace{0.2em}\textit{suffix}}'
in the first form.
However, that requires to set up a different file
for each child. With the alternative form of the command
all these files can have exactly the same content
which simplifies setting them up and maintaining them.

For example, the following file |draft.tex|
with a compilation flag |\version| as described in \secref{sec:flags}
compiles the main document as a draft:
%
\begin{center}
\begin{tabular}{l}
|\def\version{draft}|\\
|\input{childdoc.def}|\\
|\childdocforward{|\textit{main}|}|
\end{tabular}
\end{center}
%
Likewise, the following files |final|\textit{nn}|.tex|
compile the final version of the child document
|child|\textit{nn}|.tex|:
%
\begin{center}
\begin{tabular}{l}
|\def\version{final}|\\
|\input{childdoc.def}|\\
|\childdocforwardprefix{final}{child}|
\end{tabular}
\end{center}
%

Note that when several versions of a main file and/or of each child file
are to be generated, it may be convenient to set up a |Makefile| or
shell script to automatise the process.

%%%%%%%%%%%%%%%%%%%%%%%%%%%%%%%%%%%%%%%%%%%%%%%%%%%%%%%%%%%%%%%%%%%%%%%%%%%%%%%%
\subsection{Command Line Processing}
\label{sec:commandline}

The effect of redirection files can also be achieved by invoking
the \LaTeX{} compiler with a more elaborate command line.
Most conveniently this should be done as part
of a shell script or a |Makefile|.

When using \textsf{childdoc} in the main file, the following
command lines effectively perform a redirection
(note that depending on the shell being used,
backslashes may have to be doubled: `|\|' $\to$ `|\\|'):
%
\begin{center}
|... -jobname "|\textit{target}|" |\\|"|[\textit{flags}]%
|\input{childdoc.def}\childdocforward[|\textit{main}|]{|\textit{dest}|}"|
\end{center}
%
Here \textit{target} is the name of the output file,
\textit{main} is the name of the main file
and \textit{dest} is the name of the main or child file to be processed
(all filenames without extensions).
The optional argument \textit{main} can be omitted
if \textit{main} matches \textit{dest}.
Optionally, compilation \textit{flags} can be defined via |\def| commands.
This command line makes the \TeX{} engine believe
it is compiling the file \textit{target}
whose content is specified as the latter parameter.
The provided code then forwards the processing to
\textit{main} or \textit{dest} as described in \secref{sec:forward}.

%%%%%%%%%%%%%%%%%%%%%%%%%%%%%%%%%%%%%%%%%%%%%%%%%%%%%%%%%%%%%%%%%%%%%%%%%%%%%%%%
\subsection{Include by Input}
\label{sec:input}

Including child documents by |\include| has some restrictions by design.
Most notably, the content of a child document always occupies
its own set of pages; pages cannot be shared between child documents.
Usually, this behaviour makes perfect sense
because each child document contain an essential part of the document.
However, in some situations it may be desirable to compose
a document from a collection of parts
without having mandatory page breaks between then.
For this case, the package
provides a mechanism to include parts
by |\input| which can also be processed individually.
However, by construction this mechanism
requires manual handling of the content to be output.

%%%%%%%%%%%%%%%%%%%%%%%%%%%%%%%%%%%%%%%%
\DescribeMacro{\ifchilddocmanual}
The main file should be prepared as usual, see \secref{sec:include}.
However, the document body must make a distinction
between processing of an individual part and of the main document, e.g.:
%
\begin{center}
\begin{tabular}{l}
|\ifchilddocmanual|\\
|\input{\childdocname}|\\
|\||else|\\
\textit{document body with }|\input{|\textit{part}|}|\\
|\||fi|
\end{tabular}
\end{center}
%
The conditional |\ifchilddocmanual| is true whenever
a part to be included by |\input| is being compiled,
and the name of the part is stored in |\childdocname|.

%%%%%%%%%%%%%%%%%%%%%%%%%%%%%%%%%%%%%%%%
\DescribeMacro{\childdocby}
Each part to be included by |\input| should start with:
%
\begin{center}
\begin{tabular}{l}
|\input{childdoc.def}|\\
|\childdocby{|\textit{main}|}|\\
\end{tabular}
\end{center}
%
The directive |\childdocby| is similar to |\childdocof|
described in \secref{sec:include},
but the subsequent selection of content must be done manually.
To that end, both |\ifchilddoc| and |\ifchilddocmanual|
will be true upon processing of a part,
and the name of the part is stored in |\childdocname|.
Note that |\jobname| will be set to the filename of the current part
so that each part receives an individual |.aux| file
that does not interfere with the |.aux| file(s) of the main document.
This behaviour can be altered by the alternative form
|\childdocby[*]{|\textit{main}|}| (with a non-empty optional argument)
which uses the |.aux| file of the main document
by setting |\jobname| to \textit{main}.

%%%%%%%%%%%%%%%%%%%%%%%%%%%%%%%%%%%%%%%%%%%%%%%%%%%%%%%%%%%%%%%%%%%%%%%%%%%%%%%%
\subsection{Driver Development}
\label{sec:driver}

The \textsf{childdoc} mechanism can also be use for the development
of definition files such as \LaTeX{} styles or classes.
This case differs from the above setup with multiple parts
included by |\include| in that no |\includeonly| should be invoked.
This can be achieved by starting the include file
(before |\ProvidesPackage|) with:
%
\begin{center}
\begin{tabular}{l}
|\input{childdoc.def}|\\
|\childdocforward{|\textit{main}|}|\\
\end{tabular}
\end{center}
%
or alternatively with:
%
\begin{center}
\begin{tabular}{l}
|\input{childdoc.def}|\\
|\childdocby{|\textit{main}|}|\\
\end{tabular}
\end{center}
%
Both forms have slightly different effects as described above.
The main file is prepared as usual, see \secref{sec:include}.

%%%%%%%%%%%%%%%%%%%%%%%%%%%%%%%%%%%%%%%%%%%%%%%%%%%%%%%%%%%%%%%%%%%%%%%%%%%%%%%%
\subsection{Legacy Detection}
\label{sec:detection}

The directive |\childdocmain| in the main file can detect
whether the complete document or merely a child is to be compiled
even without using the directive |\childdocof|.
This method is deprecated because it is less robust
and there is no compelling reason to use it;
it is merely provided for backward compatibility
and it may be removed in future versions.

If the detection mechanism is to be used,
it is mandatory to correctly specify
the filename of the main file as the argument of |\childdocmain|:
%
\begin{center}
\begin{tabular}{l}
|\input{childdoc.def}|\\
|\childdocmain{|\textit{main}|}|\\
\end{tabular}
\end{center}
%
If |\jobname| does not match the argument \textit{main} of |\childdocmain|,
it is assumed that |\jobname| points to the child file to be compiled.
When using |\childdocmain| with the main file specified as argument,
it suffices to start a child file
with just |\input{|\textit{main}|}|
without loading of the package and using |\childdocof|.
If instead all processing is done
with the appropriate \textsf{childdoc} directives,
the argument of \textit{main} of |\childdocmain| can be empty.

An alternative version of the command line processing described
in \secref{sec:commandline} using the detection mechanism reads:
%
\begin{center}
|... -jobname "|\textit{target}|" "|[\textit{flags}]%
[|\def\jobname{|\textit{dest}|}|]|\input{|\textit{main}|}"|
\end{center}

%%%%%%%%%%%%%%%%%%%%%%%%%%%%%%%%%%%%%%%%%%%%%%%%%%%%%%%%%%%%%%%%%%%%%%%%%%%%%%%%
\subsection{Manual Code}
\label{sec:manual}

In case one cannot be certain whether the definitions file |childdoc.def|
is installed on the target \TeX{} distribution
and one prefers not to ship it,
it is conceivable to paste a few relevant commands into the sources.

To that end, drop all statements |\input{childdoc.def}|
and perform the replacements as outlined below.
Instead of |\childdocmain{|\textit{main}|}| add the following code
to the top of the main file:
%
\begin{center}
\begin{tabular}{l}
|\||ifdefined\childdocname\endinput\||fi\newif\ifchilddoc|\\
|\edef\childdocname{\scantokens\expandafter{\jobname\noexpand}}|\\
|\def\childdocmain{|\textit{main}|}\||ifx\childdocmain\childdocname\||else|\\
|\childdoctrue\includeonly{\childdocname}\let\jobname\childdocmain\||fi|\\
\end{tabular}
\end{center}
%
Instead of |\childdocof{|\textit{main}|}| just include the main file
at the top of each child file:
%
\begin{center}
|\input{|\textit{main}|}|
\end{center}
%
A simple redirection |\childdocforward{|\textit{dest}|}| is achieved by:
%
\begin{center}
|\def\jobname{|\textit{dest}|}\input{\jobname}|
\end{center}
%
The redirection with prefix
|\childdocforwardprefix[|\textit{prefix}|]{|\textit{dest}|}|
is accomplished by:
%
\begin{center}
\begin{tabular}{l}
|{\edef\jobname{\scantokens\expandafter{\jobname\noexpand}}|\\
|\def\redirectjob |\textit{prefix}|#1~~~{\gdef\jobname{|\textit{dest}|#1}}|\\
|\expandafter\redirectjob\jobname~~~}\input{\jobname}|
\end{tabular}
\end{center}

In an alternative approach,
child documents can be compiled by a specific command line
without additional code or specific definitions:
%
\begin{center}
|... -jobname "|\textit{target}|" "|[\textit{flags}]%
|\includeonly{|\textit{dest}|}\input{|\textit{main}|}"|
\end{center}
%

%%%%%%%%%%%%%%%%%%%%%%%%%%%%%%%%%%%%%%%%%%%%%%%%%%%%%%%%%%%%%%%%%%%%%%%%%%%%%%%%
%%%%%%%%%%%%%%%%%%%%%%%%%%%%%%%%%%%%%%%%%%%%%%%%%%%%%%%%%%%%%%%%%%%%%%%%%%%%%%%%
\section{Information}

%%%%%%%%%%%%%%%%%%%%%%%%%%%%%%%%%%%%%%%%%%%%%%%%%%%%%%%%%%%%%%%%%%%%%%%%%%%%%%%%
\subsection{Copyright}

Copyright \copyright{} 2017--2018 Niklas Beisert

This work may be distributed and/or modified under the
conditions of the \LaTeX{} Project Public License, either version 1.3
of this license or (at your option) any later version.
The latest version of this license is in
  \url{http://www.latex-project.org/lppl.txt}
and version 1.3 or later is part of all distributions of \LaTeX{}
version 2005/12/01 or later.

This work has the LPPL maintenance status `maintained'.

The Current Maintainer of this work is Niklas Beisert.

This work consists of the files |README.txt|, |childdoc.ins| and |childdoc.dtx|
as well as the derived files |childdoc.def|, |cdocsamp.tex|
with |cdocsch1.tex|, |cdocsch2.tex|, |cdocspt3.tex|, |cdocspt4.tex|,
|cdocsdrf.tex|, |cdocsfn1.tex|, |cdocsfn2.tex|
as well as |childdoc.pdf|.

%%%%%%%%%%%%%%%%%%%%%%%%%%%%%%%%%%%%%%%%%%%%%%%%%%%%%%%%%%%%%%%%%%%%%%%%%%%%%%%%
\subsection{Files and Installation}

The package consists of the files:
%
\begin{center}
\begin{tabular}{ll}
    |README.txt|   & readme file \\
    |childdoc.ins| & installation file \\
    |childdoc.dtx| & source file \\
    |childdoc.def| & definition file \\
    |cdocsamp.tex| & sample main file \\
    |cdocsch1.tex| & sample include file \\
    |cdocsch2.tex| & sample include file \\
    |cdocspt3.tex| & sample part file \\
    |cdocspt4.tex| & sample part file \\
    |cdocsdrf.tex| & sample redirection file \\
    |cdocsfn1.tex| & sample redirection file \\
    |cdocsfn2.tex| & sample redirection file \\
    |childdoc.pdf| & manual
\end{tabular}
\end{center}
%
The distribution consists of the files
|README.txt|, |childdoc.ins| and |childdoc.dtx|.
%
\begin{itemize}
\item
Run (pdf)\LaTeX{} on |childdoc.dtx|
to compile the manual |childdoc.pdf| (this file).
\item
Run \LaTeX{} on |childdoc.ins| to create the definitions file |childdoc.def|
and the sample |cdocsamp.tex| with include files
|cdocsch1.tex|, |cdocsch2.tex|, |cdocspt3.tex|, |cdocspt4.tex|,
|cdocsdrf.tex|, |cdocsfn1.tex|, |cdocsfn2.tex|.
Then copy the file |childdoc.def| to an appropriate directory of your \LaTeX{}
distribution, e.g.\ \textit{texmf-root}|/tex/latex/childdoc|.
\end{itemize}

%%%%%%%%%%%%%%%%%%%%%%%%%%%%%%%%%%%%%%%%%%%%%%%%%%%%%%%%%%%%%%%%%%%%%%%%%%%%%%%%
\subsection{Related CTAN Packages}

There are several other packages which offer a similar functionality:
%
\begin{itemize}
\item
The packages
\href{http://ctan.org/pkg/docmute}{\textsf{docmute}},
\href{http://ctan.org/pkg/includex}{\textsf{includex}} and
\href{http://ctan.org/pkg/standalone}{\textsf{standalone}}
provide commands to include only the document body of
a child file thus allowing both files to be compiled individually.
\item
The packages \href{http://ctan.org/pkg/subdocs}{\textsf{subdocs}}
and \href{http://ctan.org/pkg/subfiles}{\textsf{subfiles}}
provide structures in which the main and child documents can be
encapsulated and allowing them to be compiled individually.
The inclusion mechanism is different from the conventional |\include|.
\item
The package \href{http://ctan.org/pkg/combine}{\textsf{combine}}
is an elaborate solution to combine several documents into one.
\end{itemize}
%
See also the CTAN topic \href{http://ctan.org/topic/subdocs}{\textsf{subdocs}}
for further related packages.
The present package differs from the above solutions in that
a document structure constructed with the conventional |\include| mechanism
just needs two extra commands at the top of every file
such that all constituent files can be compiled individually.

%%%%%%%%%%%%%%%%%%%%%%%%%%%%%%%%%%%%%%%%%%%%%%%%%%%%%%%%%%%%%%%%%%%%%%%%%%%%%%%%
%\subsection{Feature Suggestions}
%
%The following is a list of features which may be useful for future
%versions of this package:
%%
%\begin{itemize}
%\item
%\ldots
%\end{itemize}

%%%%%%%%%%%%%%%%%%%%%%%%%%%%%%%%%%%%%%%%%%%%%%%%%%%%%%%%%%%%%%%%%%%%%%%%%%%%%%%%
\subsection{Revision History}

%%%%%%%%%%%%%%%%%%%%%%%%%%%%%%%%%%%%%%%%
\paragraph{v2.0:} 2018/12/30

\begin{itemize}
\item
immediate forward processing
\item
added |\childdocby| mechanism
\item
manual restructured
\end{itemize}

%%%%%%%%%%%%%%%%%%%%%%%%%%%%%%%%%%%%%%%%
\paragraph{v1.6:} 2018/01/17

\begin{itemize}
\item
application for development of include files
\item
corrections to manual
\end{itemize}

%%%%%%%%%%%%%%%%%%%%%%%%%%%%%%%%%%%%%%%%
\paragraph{v1.5:} 2017/05/21

\begin{itemize}
\item
more complete structuring introduced
\item
|\childdocof| introduced
\item
|\childdoc| renamed to |\childdocmain|
\item
|\childredirect| renamed to |\childdocforward| and |\childdocforwardprefix|
and functionality expanded
\end{itemize}

%%%%%%%%%%%%%%%%%%%%%%%%%%%%%%%%%%%%%%%%
\paragraph{v1.0:} 2017/04/27

\begin{itemize}
\item
manual and install package
\item
first version published on CTAN
\end{itemize}

%%%%%%%%%%%%%%%%%%%%%%%%%%%%%%%%%%%%%%%%
\paragraph{v0.6:} 2017/04/26

\begin{itemize}
\item
redirection mechanism added
\end{itemize}

%%%%%%%%%%%%%%%%%%%%%%%%%%%%%%%%%%%%%%%%
\paragraph{v0.5:} 2017/04/26

\begin{itemize}
\item
functionality in definition file
\end{itemize}


%%%%%%%%%%%%%%%%%%%%%%%%%%%%%%%%%%%%%%%%%%%%%%%%%%%%%%%%%%%%%%%%%%%%%%%%%%%%%%%%
%%%%%%%%%%%%%%%%%%%%%%%%%%%%%%%%%%%%%%%%%%%%%%%%%%%%%%%%%%%%%%%%%%%%%%%%%%%%%%%%
%%%%%%%%%%%%%%%%%%%%%%%%%%%%%%%%%%%%%%%%%%%%%%%%%%%%%%%%%%%%%%%%%%%%%%%%%%%%%%%%
\appendix

\settowidth\MacroIndent{\rmfamily\scriptsize 000\ }

 \DocInput{childdoc.dtx}

\end{document}
%</driver>
% \fi
%
% %%%%%%%%%%%%%%%%%%%%%%%%%%%%%%%%%%%%%%%%%%%%%%%%%%%%%%%%%%%%%%%%%%%%%%%%%%%%%%
% %%%%%%%%%%%%%%%%%%%%%%%%%%%%%%%%%%%%%%%%%%%%%%%%%%%%%%%%%%%%%%%%%%%%%%%%%%%%%%
% \section{Sample}
%\iffalse
%<*samplemain>
%\fi
%
% The following presents a sample document
% with two chapters, two parts, a title page,
% a compile flag as well as three forwarding files to set the flag.
% It consists of eight |.tex| files:
% \begin{center}
% \begin{tabular}{ll}
% |cdocsamp.tex|&main file\\
% |cdocsch1.tex|&include file for chapter 1\\
% |cdocsch2.tex|&include file for chapter 2\\
% |cdocspt3.tex|&include file for part 3\\
% |cdocspt4.tex|&include file for part 4\\
% |cdocsdrf.tex|&forwarding file for main file in draft mode\\
% |cdocsfi1.tex|&forwarding file for final version of chapter 1\\
% |cdocsfi2.tex|&forwarding file for final version of chapter 2\\
% \end{tabular}
% \end{center}
% Each of the eight files can be compiled directly by the \LaTeX{} compiler.
%
% %%%%%%%%%%%%%%%%%%%%%%%%%%%%%%%%%%%%%%
% \paragraph{Main File.}
%
% The main file is called |cdocsamp.tex|.
%
% Load the \textsf{childdoc} definitions and
% declare the filename for the main document:
%    \begin{macrocode}
\input{childdoc.def}
\childdocmain{}
%    \end{macrocode}

% Optional override for |\version| flag:
%    \begin{macrocode}
%%\ifchilddoc\else\providecommand{\version}{draft}\fi
%    \end{macrocode}

% Define the default values for the |\version| flag
% (|final| for the main file and |draft| for childs):
%    \begin{macrocode}
\ifchilddoc
\providecommand{\version}{draft}
\else
\providecommand{\version}{final}
\fi
%    \end{macrocode}

% Load the standard document class:
%    \begin{macrocode}
\documentclass[12pt]{article}
%    \end{macrocode}

% Start the document body:
%    \begin{macrocode}
\begin{document}
%    \end{macrocode}

% Declare a title page.
% Print title, part of document being processed and version flag:
%    \begin{macrocode}
\addtocounter{page}{-1}
\begin{center}
{\LARGE\bfseries{}childdoc example\par}
\vspace{1cm}
\ifchilddoc
\ifchilddocmanual part\else chapter\fi:
`\childdocname' of `\childdocjob'\par
\else
main document: `\childdocjob'\par
\fi
version: \version\par
\end{center}
\newpage
%    \end{macrocode}

% Manually include selected file,
% otherwise process as usual:
%    \begin{macrocode}
\ifchilddocmanual
\section*{part `\childdocname'}
\input{\childdocname}
\else
%    \end{macrocode}

% Include the two chapters:
%    \begin{macrocode}
\include{cdocsch1}
\include{cdocsch2}
%    \end{macrocode}

% Include the two parts unless only chapters should be displayed:
%    \begin{macrocode}
\ifchilddoc\else
\section{part three}
\input{cdocspt3}
\section{part four}
\input{cdocspt4}
\fi
%    \end{macrocode}

% Process as usual until here:
%    \begin{macrocode}
\fi
%    \end{macrocode}

% End of document body:
%    \begin{macrocode}
\end{document}
%    \end{macrocode}
%\iffalse
%</samplemain>
%\fi
%
% %%%%%%%%%%%%%%%%%%%%%%%%%%%%%%%%%%%%%%
% \paragraph{Chapter Include Files.}
%
% The include files are called |cdocsch1.tex| and |cdocsch2.tex|.
%
%\iffalse
%<*samplechap1|samplechap2>
%\fi

% Optional override for |\version| flag:
%    \begin{macrocode}
%%\providecommand{\version}{final}
%    \end{macrocode}

% Include the main document:
%    \begin{macrocode}
\input{childdoc.def}
\childdocof{cdocsamp}
%    \end{macrocode}

%\iffalse
%</samplechap1|samplechap2>
%\fi
%
%\iffalse
%<*samplechap1>
%\fi
% Some text for chapter 1:
%    \begin{macrocode}
\section{one}
some text in chapter one
%    \end{macrocode}

%\iffalse
%</samplechap1>
%\fi
% Some text for chapter 2:
%\iffalse
%<*samplechap2>
%\fi
%    \begin{macrocode}
\section{two}
more text in chapter two
%    \end{macrocode}

%\iffalse
%</samplechap2>
%\fi
%
% %%%%%%%%%%%%%%%%%%%%%%%%%%%%%%%%%%%%%%
% \paragraph{Part Include Files.}
%
% The include files are called |cdocspt3.tex| and |cdocspt4.tex|.
%
%\iffalse
%<*samplepart3|samplepart4>
%\fi

% Optional override for |\version| flag:
%    \begin{macrocode}
%%\providecommand{\version}{final}
%    \end{macrocode}

% Include the main document:
%    \begin{macrocode}
\input{childdoc.def}
\childdocby{cdocsamp}
%    \end{macrocode}

%\iffalse
%</samplepart3|samplepart4>
%\fi
%
%\iffalse
%<*samplepart3>
%\fi
% Some text for part 3:
%    \begin{macrocode}
some text in part three
%    \end{macrocode}

%\iffalse
%</samplepart3>
%\fi
% Some text for part 4:
%\iffalse
%<*samplepart4>
%\fi
%    \begin{macrocode}
more text in part four
%    \end{macrocode}

%\iffalse
%</samplepart4>
%\fi
%
% %%%%%%%%%%%%%%%%%%%%%%%%%%%%%%%%%%%%%%
% \paragraph{Forwarding for a Complete Draft.}
%
% The following forwarding file |cdocsdrf.tex|
% compiles the main document in draft mode:
%\iffalse
%<*sampledraft>
%\fi
%    \begin{macrocode}
\def\version{draft}
\input{childdoc.def}
\childdocforward{cdocsamp}
%    \end{macrocode}

%\iffalse
%</sampledraft>
%\fi
%
% %%%%%%%%%%%%%%%%%%%%%%%%%%%%%%%%%%%%%%
% \paragraph{Forwarding for Final Version of the Chapters.}
%
% The following forwarding files |cdocsfn1.tex| and |cdocsfn2.tex|
% (with identical content)
% compile the final versions of the child documents
% |cdocsch1.tex| and |cdocsch2.tex|, respectively:
%\iffalse
%<*samplefinal>
%\fi
%    \begin{macrocode}
\def\version{final}
\input{childdoc.def}
\childdocforwardprefix[cdocsamp]{cdocsfn}{cdocsch}
%    \end{macrocode}

%\iffalse
%</samplefinal>
%\fi
%
% %%%%%%%%%%%%%%%%%%%%%%%%%%%%%%%%%%%%%%
% \paragraph{Command Line Processing.}
%
% The following three command lines generate the output files
% |cdocscld|, |cdocscl1| and |cdocscl2|
% which should be identical to
% |cdocsdrf|, |cdocsch1| and |cdocsfn2|, respectively:
% \begin{center}
% \begin{tabular}{l}
% |latex -jobname cdocscld \|\\
% |  "\def\version{draft}\input{childdoc.def}\childdocforward{cdocsamp}"|\\
% |latex -jobname cdocscl1 \|\\
% |  "\input{childdoc.def}\childdocforward[cdocsamp]{cdocsch1}"|\\
% |latex -jobname cdocscl2 \|\\
% |  "\def\version{final}\input{childdoc.def}\childdocforward{cdocsch2}"|
% \end{tabular}
% \end{center}
% Note that the trailing backslash on each first line
% merely continues the input to the second line
% (for convenient cut ant paste).
% Furthermore, the command |latex| can be replaced by any
% of its alternative versions such as |pdflatex|.
%
% %%%%%%%%%%%%%%%%%%%%%%%%%%%%%%%%%%%%%%%%%%%%%%%%%%%%%%%%%%%%%%%%%%%%%%%%%%%%%%
% %%%%%%%%%%%%%%%%%%%%%%%%%%%%%%%%%%%%%%%%%%%%%%%%%%%%%%%%%%%%%%%%%%%%%%%%%%%%%%
% \section{Implementation}
%\iffalse
%<*package>
%\fi
%
% This section describes the definitions file |childdoc.def|.

% The definitions cannot be loaded using |\usepackage| or |\RequirePackage|
% which has a mechanism to prevent loading a style file more than once.
% When loading the definitions by means of |\input|
% multiple instances have to be prevented manually:
%\iffalse
%This code needs to be before the `\ProvidesFile' directive
%which is defined at the beginning of this file.
%Therefore it is also placed there and commented out here.
%</package>
%<*discard>
%\fi
%    \begin{macrocode}
\ifdefined\childdocmain\endinput\fi
%    \end{macrocode}
%\iffalse
%</discard>
%<*package>
%\fi
%
% \macro{\ifchilddoc}
% \macro{\ifchilddocmanual}
% The conditional |\ifchilddoc| tells whether a
% child (true) or main (false) document is being compiled.
% The conditional |\ifchilddocmanual| tells whether
% the |\includeonly| mechanism is used (false) or
% the selection of child files must be performed manually (true).
% The definitions initialise to false:
%    \begin{macrocode}
\newif\ifchilddoc
\newif\ifchilddocmanual
%    \end{macrocode}

% \macro{\childdocname}
% \macro{\childdocjob}
% The macro |\childdocname| stores the name of the main document
% to be compiled. The macro |\childdocjob| stores the name of
% the document on which the \LaTeX{} compiler was originally invoked.
% The content of |\jobname| cannot be compared
% to filenames specified in the source due to different catcodes.
% The following code rescans |\jobname|, stores the result
% in |\childdocname| and saves a copy in |\childdocjob|:
%    \begin{macrocode}
\edef\childdocname{\scantokens\expandafter{\jobname\noexpand}}
\let\childdocjob\childdocname
%    \end{macrocode}

% \macro{\childdocdisable}
% The macro |\childdocdisable| prevents the main file
% from being processed more than once.
% At this stage, the main document command |\childdocmain|
% is assumed to be called once again where it should do nothing.
% Any subsequent call to it should prevent
% a secondary processing of the main document
% It overwrites the forwarding commands
% |\childdocof| and |\childdocforward|
% with empty macros to prevent further inclusions of the main document:
%    \begin{macrocode}
\newcommand{\childdocdisable}
{
  \renewcommand{\childdocmain}[1]{\renewcommand{\childdocmain}[1]{\endinput}}
  \renewcommand{\childdocof}[1]{}
  \renewcommand{\childdocby}[2][]{}
  \renewcommand{\childdocforward}[2][]{}
  \renewcommand{\childdocdisable}{}
}
%    \end{macrocode}

% \macro{\childdocmain}
% The macro |\childdocmain| is to be called at the top of the main file
% with nothing or the main filename (without extension) as argument.
% First, it breaks loops.
% If the argument is not empty and does not match |\childdocname|
% (which is set by the first inclusion of |childdoc.def|),
% |\ifchilddoc| is set to true, |\includeonly| is applied to the child file
% and |\jobname| is set to the main file
% (for proper handling of |.aux| files):
%    \begin{macrocode}
\newcommand{\childdocmain}[1]
{
  \childdocdisable\childdocmain{}
  \if?#1?\else
    \begingroup
      \def\childdoctmp{#1}
      \ifx\childdoctmp\childdocname
        \def\childdoctmp{}
      \else
        \def\childdoctmp
        {
          \childdoctrue
          \includeonly{\childdocname}
          \def\childdocjob{#1}
          \def\jobname{#1}
        }
      \fi
      \expandafter
    \endgroup
    \childdoctmp
  \fi
}
%    \end{macrocode}

% \macro{\childdocof}
% The command |\childdocof| redirects
% compilation to the main file |#1|.
%    \begin{macrocode}
\newcommand{\childdocof}[1]
{
  \childdocdisable
  \childdoctrue
  \includeonly{\childdocname}
  \def\jobname{#1}
  \def\childdocjob{#1}
  \input{#1}
}
%    \end{macrocode}

% \macro{\childdocby}
% The command |\childdocby| ....
%    \begin{macrocode}
\newcommand{\childdocby}[2][]
{
  \childdocdisable
  \childdoctrue
  \childdocmanualtrue
  \if?#1?\else
    \def\jobname{#2}
  \fi
  \def\childdocjob{#2}
  \input{#2}
  \endinput
}
%    \end{macrocode}

% \macro{\childdocforward}
% The command |\childdocforward| redirects
% compilation to the main file or
% (if the optional argument is given) a child file.
% Parameters are set as if the main file
% or a child file starting with |\childdocof| was compiled.
% Then compilation is handed over to the main file:
%    \begin{macrocode}
\newcommand{\childdocforward}[2][]
{
  \begingroup
    \if?#1?
      \def\childdoctmp
      {
        \def\childdocname{#2}
        \def\childdocjob{#2}
        \def\jobname{#2}
        \input{#2}
        \endinput
      }
    \else
      \def\childdoctmp
      {
        \childdocdisable
        \def\childdocname{#2}
        \childdoctrue
        \includeonly{#2}
        \def\childdocjob{#1}
        \def\jobname{#1}
        \input{#1}
        \endinput
      }
    \fi
    \expandafter
  \endgroup
  \childdoctmp
}
%    \end{macrocode}

% \macro{\childdocforwardprefix}
% The command |\childdocforwardprefix| redirects
% compilation to the main or a child file by means of a pattern.
% The prefix |#1| in the current filename is replaced by |#2|
% and the suffix of the current filename is kept
% (it is assumed that the filename does not contain the substring `|~~~|'
% which is used as a delimiter).
% Compilation is handed over to the new file by |\childdocforward|:
%    \begin{macrocode}
\newcommand{\childdocforwardprefix}[3][]
{
  \begingroup
    \def\childdocextract #2##1~~~{\def\childdoctmp{\childdocforward[#1]{#3##1}}}
    \expandafter\childdocextract\childdocname~~~
    \expandafter
  \endgroup
  \childdoctmp
}
%    \end{macrocode}

% \macro{\childdoc}
% The deprecated macro |\childdoc| is a legacy version of |\childdocmain|:
%    \begin{macrocode}
\newcommand{\childdoc}{\childdocmain}
%    \end{macrocode}

% \macro{\childdocredirect}
% The deprecated macro |\childdocredirect| is a legacy version
% of |\childdocforward| and |\childdocforwardprefix|:
%    \begin{macrocode}
\newcommand{\childdocredirect}[2][]
{
  \begingroup
    \if?#1?
      \def\childdoctmp{\childdocforward{#2}}
    \else
      \def\childdoctmp{\childdocforwardprefix{#1}{#2}}
    \fi
    \expandafter
  \endgroup
  \childdoctmp
}
%    \end{macrocode}

%\iffalse
%</package>
%\fi
%
\endinput

\childdocforwardprefix[cdocsamp]{cdocsfn}{cdocsch}
%    \end{macrocode}

%\iffalse
%</samplefinal>
%\fi
%
% %%%%%%%%%%%%%%%%%%%%%%%%%%%%%%%%%%%%%%
% \paragraph{Command Line Processing.}
%
% The following three command lines generate the output files
% |cdocscld|, |cdocscl1| and |cdocscl2|
% which should be identical to
% |cdocsdrf|, |cdocsch1| and |cdocsfn2|, respectively:
% \begin{center}
% \begin{tabular}{l}
% |latex -jobname cdocscld \|\\
% |  "\def\version{draft}% \iffalse
%
% childdoc.dtx Copyright (C) 2017-2018 Niklas Beisert
%
% This work may be distributed and/or modified under the
% conditions of the LaTeX Project Public License, either version 1.3
% of this license or (at your option) any later version.
% The latest version of this license is in
%   http://www.latex-project.org/lppl.txt
% and version 1.3 or later is part of all distributions of LaTeX
% version 2005/12/01 or later.
%
% This work has the LPPL maintenance status `maintained'.
%
% The Current Maintainer of this work is Niklas Beisert.
%
% This work consists of the files childdoc.dtx and childdoc.ins
% and the derived files childdoc.def and cdocsamp.tex with
% cdocsch1.tex, cdocsch2.tex, cdocsdrf.tex, cdocsfn1.tex, cdocsfn2.tex.
%
%<package>\ifdefined\childdocmain\endinput\fi
%<package>\ProvidesFile{childdoc.def}[2018/12/30 v2.0 child document driver]
%<samplemain>\ProvidesFile{cdocsamp.tex}[2018/12/30 v2.0 sample for childdoc]
%<*driver>
%\ProvidesFile{childdoc.drv}[2018/12/30 v2.0 childdoc reference manual file]
\PassOptionsToClass{10pt,a4paper}{article}
\documentclass{ltxdoc}

\usepackage[margin=35mm]{geometry}
\usepackage{hyperref}
\usepackage{hyperxmp}
\usepackage[usenames]{color}

\hypersetup{colorlinks=true}
\hypersetup{pdfstartview=FitH}
\hypersetup{pdfpagemode=UseNone}
\hypersetup{pdfsource={}}
\hypersetup{pdflang={en-UK}}
\hypersetup{pdfcopyright={Copyright 2017-2018 Niklas Beisert.
  This work may be distributed and/or modified under the
  conditions of the LaTeX Project Public License, either version 1.3
  of this license or (at your option) any later version.}}
\hypersetup{pdflicenseurl={http://www.latex-project.org/lppl.txt}}
\hypersetup{pdfcontactaddress={ETH Zurich, ITP, HIT K,
  Wolfgang-Pauli-Strasse 27}}
\hypersetup{pdfcontactpostcode={8093}}
\hypersetup{pdfcontactcity={Zurich}}
\hypersetup{pdfcontactcountry={Switzerland}}
\hypersetup{pdfcontactemail={nbeisert@itp.phys.ethz.ch}}
\hypersetup{pdfcontacturl={http://people.phys.ethz.ch/\xmptilde nbeisert/}}

\newcommand{\secref}[1]{\hyperref[#1]{section \ref*{#1}}}

\parskip1ex
\parindent0pt
\let\olditemize\itemize
\def\itemize{\olditemize\parskip0pt}

\begin{document}

\title{The \textsf{childdoc} Package}
\hypersetup{pdftitle={The childdoc Package}}
\author{Niklas Beisert\\[2ex]
  Institut f\"ur Theoretische Physik\\
  Eidgen\"ossische Technische Hochschule Z\"urich\\
  Wolfgang-Pauli-Strasse 27, 8093 Z\"urich, Switzerland\\[1ex]
  \href{mailto:nbeisert@itp.phys.ethz.ch}
  {\texttt{nbeisert@itp.phys.ethz.ch}}}
\hypersetup{pdfauthor={Niklas Beisert}}
\hypersetup{pdfsubject={Manual for the LaTeX2e Package childdoc}}
\date{30 December 2018, \textsf{v2.0}}
\maketitle

\begin{abstract}\noindent
\textsf{childdoc} is a \LaTeXe{} package
that enables the direct compilation
of document sections included by |\include|
to individual files.
\end{abstract}

\begingroup
\parskip0ex
\tableofcontents
\endgroup

%%%%%%%%%%%%%%%%%%%%%%%%%%%%%%%%%%%%%%%%%%%%%%%%%%%%%%%%%%%%%%%%%%%%%%%%%%%%%%%%
%%%%%%%%%%%%%%%%%%%%%%%%%%%%%%%%%%%%%%%%%%%%%%%%%%%%%%%%%%%%%%%%%%%%%%%%%%%%%%%%
\section{Introduction}

\LaTeX{} provides a mechanism to structure a large document (such as a book)
into a main file and several child files (containing the chapters)
using the |\include| command.
This mechanism is beneficial for documents
which span hundreds of pages in order to
make the source file(s) more manageable.
Moreover, compilation can be restricted to
selected child files by means of the |\includeonly| command.
The latter feature can be used to reduce the compilation time while editing
(this was significantly more useful in the earlier days of \LaTeX{})
or to generate a smaller document which is easier to navigate.
Another application of |\includeonly| is to generate
documents consisting of selected parts of the complete document.

However, there are a few drawbacks of the plain |\include| mechanism:
\begin{itemize}
\item
The child files cannot be compiled on their own,
they can only be compiled via the main file.
A naive editing environment
(such as a text editor with an option
to have the current file processed by \LaTeX)
may require one to switch to the main file before compiling;
attempting to compile the child file produces errors.
\item
The main file must be modified (each time)
to adjust the |\includeonly| command
to the present needs. This easily leaves the main file in a messy state.
\item
The generated document will always carry the filename
of the main document. This is inconvenient if
several child files are to be compiled and
to be kept for distribution.
\end{itemize}

The present package provides a simple interface
to make child files individually compilable by \LaTeX{}.
Compiling a child file then has the same effect as compiling
the main file with an |\includeonly| command
to select the appropriate child.
Moreover the generated document will carry the name of the child
rather than the main file.
This resolves all three above issues.

This feature is meant to make the editing of books,
thesis documents and lecture notes somewhat more convenient.
However, the package can also be used efficiently for
composing a series of documents (such as exercise sheets)
which are typically distributed individually.
It then assists the author in generating the individual documents
(potentially in different versions)
as well as a document containing the collected series.
Another application is in developing style files
or other kinds of included material
where compilation of the style file could redirect
to a sample or test file.

%%%%%%%%%%%%%%%%%%%%%%%%%%%%%%%%%%%%%%%%%%%%%%%%%%%%%%%%%%%%%%%%%%%%%%%%%%%%%%%%
%%%%%%%%%%%%%%%%%%%%%%%%%%%%%%%%%%%%%%%%%%%%%%%%%%%%%%%%%%%%%%%%%%%%%%%%%%%%%%%%
\section{Usage}

First of all, the package \textsf{childdoc} is \emph{not} a standard
\LaTeXe{} |.sty| style file! Therefore it needs to be invoked in
a non-standard way.

%%%%%%%%%%%%%%%%%%%%%%%%%%%%%%%%%%%%%%%%%%%%%%%%%%%%%%%%%%%%%%%%%%%%%%%%%%%%%%%%
\subsection{Included Files}
\label{sec:include}

%%%%%%%%%%%%%%%%%%%%%%%%%%%%%%%%%%%%%%%%
\DescribeMacro{\childdocmain}
To use the package, add the commands
\begin{center}
\begin{tabular}{l}
|\input{childdoc.def}|\\
|\childdocmain{}|\\
\end{tabular}
\end{center}
at the very top of the main \LaTeX{} file,
in particular \emph{before} the |\documentclass| statement!
The argument of |\childdocmain| should be left empty
(but it must be present).

%%%%%%%%%%%%%%%%%%%%%%%%%%%%%%%%%%%%%%%%
\DescribeMacro{\childdocof}
Furthermore, add the commands
\begin{center}
\begin{tabular}{l}
|\input{childdoc.def}|\\
|\childdocof{|\textit{main}|}|\\
\end{tabular}
\end{center}
at the top of every child file \textit{child}
which is included by |\include{|\textit{child}|}|
from within the main file
(or at least for those files to be compiled individually).
The argument \textit{main} must be the filename of the main file.

There are a couple of
considerations in setting up the main and child documents:

%%%%%%%%%%%%%%%%%%%%%%%%%%%%%%%%%%%%%%%%
\paragraph{Restrictions.}

Please note the following restrictions:
\begin{itemize}
\item
|\childdocmain| must be called with one argument \textit{main}
to ensure compatibility with earlier version of the package.
It must either be empty (|\childdocmain{}|)
or precisely match the filename of the main file in which it is specified.
See \secref{sec:detection} for further information.
\item
The filename \textit{main} must be specified without the |.tex| extension.
\item
The filename \textit{main} is case sensitive
(even in case-insensitive file systems)
due to internal string comparison.
\item
The argument \textit{main} should be fully expanded, it cannot be a macro.
\item
Subdirectories and special characters should be avoided in filenames.
\item
The command |\childdocmain{|\textit{main}|}| must be followed by a whitespace.
It should not be followed immediately by another command
or by a comment mark `|%|'.
This is because the \TeX{} parser reads the token immediately following
the argument of |\childdocmain| and puts it
at the beginning of every child section;
however, a white\-space is ignored.
\end{itemize}

%%%%%%%%%%%%%%%%%%%%%%%%%%%%%%%%%%%%%%%%
\paragraph{Content of Main File.}

It is advisable to place all content in the child files included by |\include|.
Any output contained in the main file will appear in all child documents
unless suppressed manually;
it cannot be suppressed automatically by the |\includeonly| directive
and thus should normally be avoided.
A method to include some content in the main file
by means of conditional processing is described in \secref{sec:conditional}.

%%%%%%%%%%%%%%%%%%%%%%%%%%%%%%%%%%%%%%%%
\paragraph{Page Numbering.}

When only a part of the document is compiled,
the appropriate numbering of pages
(as well as other status parameters)
is determined from the |.aux| files.
The latter contain information from previous passes.
However this information needs to propagate through
all intermediate child documents.
Therefore the page numbering in child documents may well
be inconsistent until the complete document is compiled at least once.

A useful (if unconventional) way to always ensure a consistent
page numbering is to restart the numbering in each child document
and denote the pages by `\textit{child}|.|\textit{page}'
where \textit{child} represents the chapter/section number of the child file.
This can be achieved by the command
|\numberwithin{page}{|\textit{child}|}|
of the \textsf{amsmath} package
where \textit{child} can be |chapter| or |section|
depending on the chosen structuring.
Alternatively, one can modify the macro |\thepage| appropriately
and reset the counter |page| at the start of each child file.

%%%%%%%%%%%%%%%%%%%%%%%%%%%%%%%%%%%%%%%%%%%%%%%%%%%%%%%%%%%%%%%%%%%%%%%%%%%%%%%%
\subsection{Conditional Processing}
\label{sec:conditional}

The package provides a mechanism to compile different versions
of a document. To customise the versions further some conditional processing
can come in handy to distinguish which version is being compiled.
The package provides two macros to describe the compilation context:

%%%%%%%%%%%%%%%%%%%%%%%%%%%%%%%%%%%%%%%%
\DescribeMacro{\ifchilddoc}
The conditional |\ifchilddoc| distinguishes between the compilation of
child documents and the main document:
%
\begin{center}
|\ifchilddoc |\textit{child-code}| |[|\||else |\textit{main-code}]| \||fi|
\end{center}

%%%%%%%%%%%%%%%%%%%%%%%%%%%%%%%%%%%%%%%%
\DescribeMacro{\childdocname}
\DescribeMacro{\childdocjob}
The macro |\childdocname| contains the filename (without extension)
of the main or child file being processed.
Note that |\childdocjob| will always contain the name of the main file.

%%%%%%%%%%%%%%%%%%%%%%%%%%%%%%%%%%%%%%%%
\paragraph{Title Page.}

Conditional processing can be used to include a title or banner page
in the main document when proper precautions are taken.
Importantly, the code in the main file should ensure that the page counter
(as well as other status parameters which are stored in the |.aux| files)
takes the same value after the conditional processing.
Otherwise the page numbers may take divergent values
depending on which part is compiled.

For example, a title page could be declared by:
%
\begin{center}
\begin{tabular}{l}
|\ifchilddoc\||else|\\
|\addtocounter{page}{-1}|\\
\textit{code for title page}\\
|\newpage|\\
|\||fi|
\end{tabular}
\end{center}
%
A banner page for the child documents can be generated by:
%
\begin{center}
\begin{tabular}{l}
|\ifchilddoc|\\
|\addtocounter{page}{-1}|\\
\textit{code for banner page}\\
|\newpage|\\
|\||fi|
\end{tabular}
\end{center}
%
Here one could write a message such as:
\begin{center}
|This is the part \childdocname{} of \childdocjob{}.|
\end{center}

%%%%%%%%%%%%%%%%%%%%%%%%%%%%%%%%%%%%%%%%%%%%%%%%%%%%%%%%%%%%%%%%%%%%%%%%%%%%%%%%
\subsection{Flags}
\label{sec:flags}

The package makes it easy to generate different versions
of the main or child documents.
To this end compilation flags can be defined
and assigned different default values.
They will be particularly useful in conjunction
with the forwarding mechanism described in \secref{sec:forward}.

For example, it may be useful to have a flag |\version|
which can be set to |draft| or |final|.
The document source will contain some conditional code
depending on the value of |\version|.
Suppose further, the flag should default to |final| for the main file
and to |draft| for child files
which is a natural assignment for editing the document.
This is achieved by placing the following code
in the preamble of the main document
(below the |\childdocmain| directive):
%
\begin{center}
\begin{tabular}{l}
|\ifchilddoc|\\
|\providecommand{\version}{draft}|\\
|\||else|\\
|\providecommand{\version}{final}|\\
|\||fi|
\end{tabular}
\end{center}
%
The definition by |\providecommand| makes sure
that previous definitions are not overwritten.
Further statements |\providecommand{\version}{...}|
can thus be added before the above code to override it.

For the main file, one might add a line
(between |\childdocmain| and the above block)
%
\begin{center}
|%\ifchilddoc\||else\providecommand{\version}{draft}\||fi|
\end{center}
%
which can be uncommented to produce a draft version.
Likewise one can add a line to the very top of a child file
(above the |\childdocof{|\textit{main}|}| directive)
%
\begin{center}
|%\providecommand{\version}{final}|
\end{center}
%
which can be uncommented to produce the final version of this child document.

%%%%%%%%%%%%%%%%%%%%%%%%%%%%%%%%%%%%%%%%%%%%%%%%%%%%%%%%%%%%%%%%%%%%%%%%%%%%%%%%
\subsection{Forwarding}
\label{sec:forward}

Different versions of the main or child documents
using compilation flags as described in \secref{sec:flags}
can be (permanently) stored in different files
for convenient compilation, viewing and distribution.
To this end, the package defines a command
to pass on compilation to a different file:

%%%%%%%%%%%%%%%%%%%%%%%%%%%%%%%%%%%%%%%%
\DescribeMacro{\childdocforward}
The command |\childdocforward| redirects processing to
another source file:
%
\begin{center}
\begin{tabular}{l}
|\input{childdoc.def}|\\
|\childdocforward[|\textit{main}|]{|\textit{dest}|}|\\
\end{tabular}
\end{center}
%
The argument \textit{dest} is the destination file
(without extension).
It should be the main file or one of the child files.
Note that further \textsf{childdoc} directives
such as |\childdocof| and |\childdocforward|
in the indicated file will be processed in this form.
The optional argument \textit{main}
passes on directly to the main file \textit{main}
while pretending to compile the child \textit{dest}.
This form behaves as if \textit{dest}
issues |\childdocof{|\textit{main}|}| right away,
and no further \textsf{childdoc} directives will be processed.

%%%%%%%%%%%%%%%%%%%%%%%%%%%%%%%%%%%%%%%%
\DescribeMacro{\...prefix}
In the alternative form |\childdocforwardprefix|,
%
\begin{center}
\begin{tabular}{l}
|\input{childdoc.def}|\\
|\childdocforwardprefix[|\textit{main}|]{|\textit{prefix}|}{|\textit{dest}|}|
\end{tabular}
\end{center}
%
the destination file is determined by a pattern
depending on the current file:
To make this work, the current file must be called
`{\textit{prefix}\hspace{0.2em}\textit{suffix}}'
with \textit{prefix} matching precisely the argument.
Processing is then passed on to the file
`{\textit{dest}\hspace{0.2em}\textit{suffix}}'.
Surely, the same effect is achieved by
directly specifying the
argument `{\textit{dest}\hspace{0.2em}\textit{suffix}}'
in the first form.
However, that requires to set up a different file
for each child. With the alternative form of the command
all these files can have exactly the same content
which simplifies setting them up and maintaining them.

For example, the following file |draft.tex|
with a compilation flag |\version| as described in \secref{sec:flags}
compiles the main document as a draft:
%
\begin{center}
\begin{tabular}{l}
|\def\version{draft}|\\
|\input{childdoc.def}|\\
|\childdocforward{|\textit{main}|}|
\end{tabular}
\end{center}
%
Likewise, the following files |final|\textit{nn}|.tex|
compile the final version of the child document
|child|\textit{nn}|.tex|:
%
\begin{center}
\begin{tabular}{l}
|\def\version{final}|\\
|\input{childdoc.def}|\\
|\childdocforwardprefix{final}{child}|
\end{tabular}
\end{center}
%

Note that when several versions of a main file and/or of each child file
are to be generated, it may be convenient to set up a |Makefile| or
shell script to automatise the process.

%%%%%%%%%%%%%%%%%%%%%%%%%%%%%%%%%%%%%%%%%%%%%%%%%%%%%%%%%%%%%%%%%%%%%%%%%%%%%%%%
\subsection{Command Line Processing}
\label{sec:commandline}

The effect of redirection files can also be achieved by invoking
the \LaTeX{} compiler with a more elaborate command line.
Most conveniently this should be done as part
of a shell script or a |Makefile|.

When using \textsf{childdoc} in the main file, the following
command lines effectively perform a redirection
(note that depending on the shell being used,
backslashes may have to be doubled: `|\|' $\to$ `|\\|'):
%
\begin{center}
|... -jobname "|\textit{target}|" |\\|"|[\textit{flags}]%
|\input{childdoc.def}\childdocforward[|\textit{main}|]{|\textit{dest}|}"|
\end{center}
%
Here \textit{target} is the name of the output file,
\textit{main} is the name of the main file
and \textit{dest} is the name of the main or child file to be processed
(all filenames without extensions).
The optional argument \textit{main} can be omitted
if \textit{main} matches \textit{dest}.
Optionally, compilation \textit{flags} can be defined via |\def| commands.
This command line makes the \TeX{} engine believe
it is compiling the file \textit{target}
whose content is specified as the latter parameter.
The provided code then forwards the processing to
\textit{main} or \textit{dest} as described in \secref{sec:forward}.

%%%%%%%%%%%%%%%%%%%%%%%%%%%%%%%%%%%%%%%%%%%%%%%%%%%%%%%%%%%%%%%%%%%%%%%%%%%%%%%%
\subsection{Include by Input}
\label{sec:input}

Including child documents by |\include| has some restrictions by design.
Most notably, the content of a child document always occupies
its own set of pages; pages cannot be shared between child documents.
Usually, this behaviour makes perfect sense
because each child document contain an essential part of the document.
However, in some situations it may be desirable to compose
a document from a collection of parts
without having mandatory page breaks between then.
For this case, the package
provides a mechanism to include parts
by |\input| which can also be processed individually.
However, by construction this mechanism
requires manual handling of the content to be output.

%%%%%%%%%%%%%%%%%%%%%%%%%%%%%%%%%%%%%%%%
\DescribeMacro{\ifchilddocmanual}
The main file should be prepared as usual, see \secref{sec:include}.
However, the document body must make a distinction
between processing of an individual part and of the main document, e.g.:
%
\begin{center}
\begin{tabular}{l}
|\ifchilddocmanual|\\
|\input{\childdocname}|\\
|\||else|\\
\textit{document body with }|\input{|\textit{part}|}|\\
|\||fi|
\end{tabular}
\end{center}
%
The conditional |\ifchilddocmanual| is true whenever
a part to be included by |\input| is being compiled,
and the name of the part is stored in |\childdocname|.

%%%%%%%%%%%%%%%%%%%%%%%%%%%%%%%%%%%%%%%%
\DescribeMacro{\childdocby}
Each part to be included by |\input| should start with:
%
\begin{center}
\begin{tabular}{l}
|\input{childdoc.def}|\\
|\childdocby{|\textit{main}|}|\\
\end{tabular}
\end{center}
%
The directive |\childdocby| is similar to |\childdocof|
described in \secref{sec:include},
but the subsequent selection of content must be done manually.
To that end, both |\ifchilddoc| and |\ifchilddocmanual|
will be true upon processing of a part,
and the name of the part is stored in |\childdocname|.
Note that |\jobname| will be set to the filename of the current part
so that each part receives an individual |.aux| file
that does not interfere with the |.aux| file(s) of the main document.
This behaviour can be altered by the alternative form
|\childdocby[*]{|\textit{main}|}| (with a non-empty optional argument)
which uses the |.aux| file of the main document
by setting |\jobname| to \textit{main}.

%%%%%%%%%%%%%%%%%%%%%%%%%%%%%%%%%%%%%%%%%%%%%%%%%%%%%%%%%%%%%%%%%%%%%%%%%%%%%%%%
\subsection{Driver Development}
\label{sec:driver}

The \textsf{childdoc} mechanism can also be use for the development
of definition files such as \LaTeX{} styles or classes.
This case differs from the above setup with multiple parts
included by |\include| in that no |\includeonly| should be invoked.
This can be achieved by starting the include file
(before |\ProvidesPackage|) with:
%
\begin{center}
\begin{tabular}{l}
|\input{childdoc.def}|\\
|\childdocforward{|\textit{main}|}|\\
\end{tabular}
\end{center}
%
or alternatively with:
%
\begin{center}
\begin{tabular}{l}
|\input{childdoc.def}|\\
|\childdocby{|\textit{main}|}|\\
\end{tabular}
\end{center}
%
Both forms have slightly different effects as described above.
The main file is prepared as usual, see \secref{sec:include}.

%%%%%%%%%%%%%%%%%%%%%%%%%%%%%%%%%%%%%%%%%%%%%%%%%%%%%%%%%%%%%%%%%%%%%%%%%%%%%%%%
\subsection{Legacy Detection}
\label{sec:detection}

The directive |\childdocmain| in the main file can detect
whether the complete document or merely a child is to be compiled
even without using the directive |\childdocof|.
This method is deprecated because it is less robust
and there is no compelling reason to use it;
it is merely provided for backward compatibility
and it may be removed in future versions.

If the detection mechanism is to be used,
it is mandatory to correctly specify
the filename of the main file as the argument of |\childdocmain|:
%
\begin{center}
\begin{tabular}{l}
|\input{childdoc.def}|\\
|\childdocmain{|\textit{main}|}|\\
\end{tabular}
\end{center}
%
If |\jobname| does not match the argument \textit{main} of |\childdocmain|,
it is assumed that |\jobname| points to the child file to be compiled.
When using |\childdocmain| with the main file specified as argument,
it suffices to start a child file
with just |\input{|\textit{main}|}|
without loading of the package and using |\childdocof|.
If instead all processing is done
with the appropriate \textsf{childdoc} directives,
the argument of \textit{main} of |\childdocmain| can be empty.

An alternative version of the command line processing described
in \secref{sec:commandline} using the detection mechanism reads:
%
\begin{center}
|... -jobname "|\textit{target}|" "|[\textit{flags}]%
[|\def\jobname{|\textit{dest}|}|]|\input{|\textit{main}|}"|
\end{center}

%%%%%%%%%%%%%%%%%%%%%%%%%%%%%%%%%%%%%%%%%%%%%%%%%%%%%%%%%%%%%%%%%%%%%%%%%%%%%%%%
\subsection{Manual Code}
\label{sec:manual}

In case one cannot be certain whether the definitions file |childdoc.def|
is installed on the target \TeX{} distribution
and one prefers not to ship it,
it is conceivable to paste a few relevant commands into the sources.

To that end, drop all statements |\input{childdoc.def}|
and perform the replacements as outlined below.
Instead of |\childdocmain{|\textit{main}|}| add the following code
to the top of the main file:
%
\begin{center}
\begin{tabular}{l}
|\||ifdefined\childdocname\endinput\||fi\newif\ifchilddoc|\\
|\edef\childdocname{\scantokens\expandafter{\jobname\noexpand}}|\\
|\def\childdocmain{|\textit{main}|}\||ifx\childdocmain\childdocname\||else|\\
|\childdoctrue\includeonly{\childdocname}\let\jobname\childdocmain\||fi|\\
\end{tabular}
\end{center}
%
Instead of |\childdocof{|\textit{main}|}| just include the main file
at the top of each child file:
%
\begin{center}
|\input{|\textit{main}|}|
\end{center}
%
A simple redirection |\childdocforward{|\textit{dest}|}| is achieved by:
%
\begin{center}
|\def\jobname{|\textit{dest}|}\input{\jobname}|
\end{center}
%
The redirection with prefix
|\childdocforwardprefix[|\textit{prefix}|]{|\textit{dest}|}|
is accomplished by:
%
\begin{center}
\begin{tabular}{l}
|{\edef\jobname{\scantokens\expandafter{\jobname\noexpand}}|\\
|\def\redirectjob |\textit{prefix}|#1~~~{\gdef\jobname{|\textit{dest}|#1}}|\\
|\expandafter\redirectjob\jobname~~~}\input{\jobname}|
\end{tabular}
\end{center}

In an alternative approach,
child documents can be compiled by a specific command line
without additional code or specific definitions:
%
\begin{center}
|... -jobname "|\textit{target}|" "|[\textit{flags}]%
|\includeonly{|\textit{dest}|}\input{|\textit{main}|}"|
\end{center}
%

%%%%%%%%%%%%%%%%%%%%%%%%%%%%%%%%%%%%%%%%%%%%%%%%%%%%%%%%%%%%%%%%%%%%%%%%%%%%%%%%
%%%%%%%%%%%%%%%%%%%%%%%%%%%%%%%%%%%%%%%%%%%%%%%%%%%%%%%%%%%%%%%%%%%%%%%%%%%%%%%%
\section{Information}

%%%%%%%%%%%%%%%%%%%%%%%%%%%%%%%%%%%%%%%%%%%%%%%%%%%%%%%%%%%%%%%%%%%%%%%%%%%%%%%%
\subsection{Copyright}

Copyright \copyright{} 2017--2018 Niklas Beisert

This work may be distributed and/or modified under the
conditions of the \LaTeX{} Project Public License, either version 1.3
of this license or (at your option) any later version.
The latest version of this license is in
  \url{http://www.latex-project.org/lppl.txt}
and version 1.3 or later is part of all distributions of \LaTeX{}
version 2005/12/01 or later.

This work has the LPPL maintenance status `maintained'.

The Current Maintainer of this work is Niklas Beisert.

This work consists of the files |README.txt|, |childdoc.ins| and |childdoc.dtx|
as well as the derived files |childdoc.def|, |cdocsamp.tex|
with |cdocsch1.tex|, |cdocsch2.tex|, |cdocspt3.tex|, |cdocspt4.tex|,
|cdocsdrf.tex|, |cdocsfn1.tex|, |cdocsfn2.tex|
as well as |childdoc.pdf|.

%%%%%%%%%%%%%%%%%%%%%%%%%%%%%%%%%%%%%%%%%%%%%%%%%%%%%%%%%%%%%%%%%%%%%%%%%%%%%%%%
\subsection{Files and Installation}

The package consists of the files:
%
\begin{center}
\begin{tabular}{ll}
    |README.txt|   & readme file \\
    |childdoc.ins| & installation file \\
    |childdoc.dtx| & source file \\
    |childdoc.def| & definition file \\
    |cdocsamp.tex| & sample main file \\
    |cdocsch1.tex| & sample include file \\
    |cdocsch2.tex| & sample include file \\
    |cdocspt3.tex| & sample part file \\
    |cdocspt4.tex| & sample part file \\
    |cdocsdrf.tex| & sample redirection file \\
    |cdocsfn1.tex| & sample redirection file \\
    |cdocsfn2.tex| & sample redirection file \\
    |childdoc.pdf| & manual
\end{tabular}
\end{center}
%
The distribution consists of the files
|README.txt|, |childdoc.ins| and |childdoc.dtx|.
%
\begin{itemize}
\item
Run (pdf)\LaTeX{} on |childdoc.dtx|
to compile the manual |childdoc.pdf| (this file).
\item
Run \LaTeX{} on |childdoc.ins| to create the definitions file |childdoc.def|
and the sample |cdocsamp.tex| with include files
|cdocsch1.tex|, |cdocsch2.tex|, |cdocspt3.tex|, |cdocspt4.tex|,
|cdocsdrf.tex|, |cdocsfn1.tex|, |cdocsfn2.tex|.
Then copy the file |childdoc.def| to an appropriate directory of your \LaTeX{}
distribution, e.g.\ \textit{texmf-root}|/tex/latex/childdoc|.
\end{itemize}

%%%%%%%%%%%%%%%%%%%%%%%%%%%%%%%%%%%%%%%%%%%%%%%%%%%%%%%%%%%%%%%%%%%%%%%%%%%%%%%%
\subsection{Related CTAN Packages}

There are several other packages which offer a similar functionality:
%
\begin{itemize}
\item
The packages
\href{http://ctan.org/pkg/docmute}{\textsf{docmute}},
\href{http://ctan.org/pkg/includex}{\textsf{includex}} and
\href{http://ctan.org/pkg/standalone}{\textsf{standalone}}
provide commands to include only the document body of
a child file thus allowing both files to be compiled individually.
\item
The packages \href{http://ctan.org/pkg/subdocs}{\textsf{subdocs}}
and \href{http://ctan.org/pkg/subfiles}{\textsf{subfiles}}
provide structures in which the main and child documents can be
encapsulated and allowing them to be compiled individually.
The inclusion mechanism is different from the conventional |\include|.
\item
The package \href{http://ctan.org/pkg/combine}{\textsf{combine}}
is an elaborate solution to combine several documents into one.
\end{itemize}
%
See also the CTAN topic \href{http://ctan.org/topic/subdocs}{\textsf{subdocs}}
for further related packages.
The present package differs from the above solutions in that
a document structure constructed with the conventional |\include| mechanism
just needs two extra commands at the top of every file
such that all constituent files can be compiled individually.

%%%%%%%%%%%%%%%%%%%%%%%%%%%%%%%%%%%%%%%%%%%%%%%%%%%%%%%%%%%%%%%%%%%%%%%%%%%%%%%%
%\subsection{Feature Suggestions}
%
%The following is a list of features which may be useful for future
%versions of this package:
%%
%\begin{itemize}
%\item
%\ldots
%\end{itemize}

%%%%%%%%%%%%%%%%%%%%%%%%%%%%%%%%%%%%%%%%%%%%%%%%%%%%%%%%%%%%%%%%%%%%%%%%%%%%%%%%
\subsection{Revision History}

%%%%%%%%%%%%%%%%%%%%%%%%%%%%%%%%%%%%%%%%
\paragraph{v2.0:} 2018/12/30

\begin{itemize}
\item
immediate forward processing
\item
added |\childdocby| mechanism
\item
manual restructured
\end{itemize}

%%%%%%%%%%%%%%%%%%%%%%%%%%%%%%%%%%%%%%%%
\paragraph{v1.6:} 2018/01/17

\begin{itemize}
\item
application for development of include files
\item
corrections to manual
\end{itemize}

%%%%%%%%%%%%%%%%%%%%%%%%%%%%%%%%%%%%%%%%
\paragraph{v1.5:} 2017/05/21

\begin{itemize}
\item
more complete structuring introduced
\item
|\childdocof| introduced
\item
|\childdoc| renamed to |\childdocmain|
\item
|\childredirect| renamed to |\childdocforward| and |\childdocforwardprefix|
and functionality expanded
\end{itemize}

%%%%%%%%%%%%%%%%%%%%%%%%%%%%%%%%%%%%%%%%
\paragraph{v1.0:} 2017/04/27

\begin{itemize}
\item
manual and install package
\item
first version published on CTAN
\end{itemize}

%%%%%%%%%%%%%%%%%%%%%%%%%%%%%%%%%%%%%%%%
\paragraph{v0.6:} 2017/04/26

\begin{itemize}
\item
redirection mechanism added
\end{itemize}

%%%%%%%%%%%%%%%%%%%%%%%%%%%%%%%%%%%%%%%%
\paragraph{v0.5:} 2017/04/26

\begin{itemize}
\item
functionality in definition file
\end{itemize}


%%%%%%%%%%%%%%%%%%%%%%%%%%%%%%%%%%%%%%%%%%%%%%%%%%%%%%%%%%%%%%%%%%%%%%%%%%%%%%%%
%%%%%%%%%%%%%%%%%%%%%%%%%%%%%%%%%%%%%%%%%%%%%%%%%%%%%%%%%%%%%%%%%%%%%%%%%%%%%%%%
%%%%%%%%%%%%%%%%%%%%%%%%%%%%%%%%%%%%%%%%%%%%%%%%%%%%%%%%%%%%%%%%%%%%%%%%%%%%%%%%
\appendix

\settowidth\MacroIndent{\rmfamily\scriptsize 000\ }

 \DocInput{childdoc.dtx}

\end{document}
%</driver>
% \fi
%
% %%%%%%%%%%%%%%%%%%%%%%%%%%%%%%%%%%%%%%%%%%%%%%%%%%%%%%%%%%%%%%%%%%%%%%%%%%%%%%
% %%%%%%%%%%%%%%%%%%%%%%%%%%%%%%%%%%%%%%%%%%%%%%%%%%%%%%%%%%%%%%%%%%%%%%%%%%%%%%
% \section{Sample}
%\iffalse
%<*samplemain>
%\fi
%
% The following presents a sample document
% with two chapters, two parts, a title page,
% a compile flag as well as three forwarding files to set the flag.
% It consists of eight |.tex| files:
% \begin{center}
% \begin{tabular}{ll}
% |cdocsamp.tex|&main file\\
% |cdocsch1.tex|&include file for chapter 1\\
% |cdocsch2.tex|&include file for chapter 2\\
% |cdocspt3.tex|&include file for part 3\\
% |cdocspt4.tex|&include file for part 4\\
% |cdocsdrf.tex|&forwarding file for main file in draft mode\\
% |cdocsfi1.tex|&forwarding file for final version of chapter 1\\
% |cdocsfi2.tex|&forwarding file for final version of chapter 2\\
% \end{tabular}
% \end{center}
% Each of the eight files can be compiled directly by the \LaTeX{} compiler.
%
% %%%%%%%%%%%%%%%%%%%%%%%%%%%%%%%%%%%%%%
% \paragraph{Main File.}
%
% The main file is called |cdocsamp.tex|.
%
% Load the \textsf{childdoc} definitions and
% declare the filename for the main document:
%    \begin{macrocode}
\input{childdoc.def}
\childdocmain{}
%    \end{macrocode}

% Optional override for |\version| flag:
%    \begin{macrocode}
%%\ifchilddoc\else\providecommand{\version}{draft}\fi
%    \end{macrocode}

% Define the default values for the |\version| flag
% (|final| for the main file and |draft| for childs):
%    \begin{macrocode}
\ifchilddoc
\providecommand{\version}{draft}
\else
\providecommand{\version}{final}
\fi
%    \end{macrocode}

% Load the standard document class:
%    \begin{macrocode}
\documentclass[12pt]{article}
%    \end{macrocode}

% Start the document body:
%    \begin{macrocode}
\begin{document}
%    \end{macrocode}

% Declare a title page.
% Print title, part of document being processed and version flag:
%    \begin{macrocode}
\addtocounter{page}{-1}
\begin{center}
{\LARGE\bfseries{}childdoc example\par}
\vspace{1cm}
\ifchilddoc
\ifchilddocmanual part\else chapter\fi:
`\childdocname' of `\childdocjob'\par
\else
main document: `\childdocjob'\par
\fi
version: \version\par
\end{center}
\newpage
%    \end{macrocode}

% Manually include selected file,
% otherwise process as usual:
%    \begin{macrocode}
\ifchilddocmanual
\section*{part `\childdocname'}
\input{\childdocname}
\else
%    \end{macrocode}

% Include the two chapters:
%    \begin{macrocode}
\include{cdocsch1}
\include{cdocsch2}
%    \end{macrocode}

% Include the two parts unless only chapters should be displayed:
%    \begin{macrocode}
\ifchilddoc\else
\section{part three}
\input{cdocspt3}
\section{part four}
\input{cdocspt4}
\fi
%    \end{macrocode}

% Process as usual until here:
%    \begin{macrocode}
\fi
%    \end{macrocode}

% End of document body:
%    \begin{macrocode}
\end{document}
%    \end{macrocode}
%\iffalse
%</samplemain>
%\fi
%
% %%%%%%%%%%%%%%%%%%%%%%%%%%%%%%%%%%%%%%
% \paragraph{Chapter Include Files.}
%
% The include files are called |cdocsch1.tex| and |cdocsch2.tex|.
%
%\iffalse
%<*samplechap1|samplechap2>
%\fi

% Optional override for |\version| flag:
%    \begin{macrocode}
%%\providecommand{\version}{final}
%    \end{macrocode}

% Include the main document:
%    \begin{macrocode}
\input{childdoc.def}
\childdocof{cdocsamp}
%    \end{macrocode}

%\iffalse
%</samplechap1|samplechap2>
%\fi
%
%\iffalse
%<*samplechap1>
%\fi
% Some text for chapter 1:
%    \begin{macrocode}
\section{one}
some text in chapter one
%    \end{macrocode}

%\iffalse
%</samplechap1>
%\fi
% Some text for chapter 2:
%\iffalse
%<*samplechap2>
%\fi
%    \begin{macrocode}
\section{two}
more text in chapter two
%    \end{macrocode}

%\iffalse
%</samplechap2>
%\fi
%
% %%%%%%%%%%%%%%%%%%%%%%%%%%%%%%%%%%%%%%
% \paragraph{Part Include Files.}
%
% The include files are called |cdocspt3.tex| and |cdocspt4.tex|.
%
%\iffalse
%<*samplepart3|samplepart4>
%\fi

% Optional override for |\version| flag:
%    \begin{macrocode}
%%\providecommand{\version}{final}
%    \end{macrocode}

% Include the main document:
%    \begin{macrocode}
\input{childdoc.def}
\childdocby{cdocsamp}
%    \end{macrocode}

%\iffalse
%</samplepart3|samplepart4>
%\fi
%
%\iffalse
%<*samplepart3>
%\fi
% Some text for part 3:
%    \begin{macrocode}
some text in part three
%    \end{macrocode}

%\iffalse
%</samplepart3>
%\fi
% Some text for part 4:
%\iffalse
%<*samplepart4>
%\fi
%    \begin{macrocode}
more text in part four
%    \end{macrocode}

%\iffalse
%</samplepart4>
%\fi
%
% %%%%%%%%%%%%%%%%%%%%%%%%%%%%%%%%%%%%%%
% \paragraph{Forwarding for a Complete Draft.}
%
% The following forwarding file |cdocsdrf.tex|
% compiles the main document in draft mode:
%\iffalse
%<*sampledraft>
%\fi
%    \begin{macrocode}
\def\version{draft}
\input{childdoc.def}
\childdocforward{cdocsamp}
%    \end{macrocode}

%\iffalse
%</sampledraft>
%\fi
%
% %%%%%%%%%%%%%%%%%%%%%%%%%%%%%%%%%%%%%%
% \paragraph{Forwarding for Final Version of the Chapters.}
%
% The following forwarding files |cdocsfn1.tex| and |cdocsfn2.tex|
% (with identical content)
% compile the final versions of the child documents
% |cdocsch1.tex| and |cdocsch2.tex|, respectively:
%\iffalse
%<*samplefinal>
%\fi
%    \begin{macrocode}
\def\version{final}
\input{childdoc.def}
\childdocforwardprefix[cdocsamp]{cdocsfn}{cdocsch}
%    \end{macrocode}

%\iffalse
%</samplefinal>
%\fi
%
% %%%%%%%%%%%%%%%%%%%%%%%%%%%%%%%%%%%%%%
% \paragraph{Command Line Processing.}
%
% The following three command lines generate the output files
% |cdocscld|, |cdocscl1| and |cdocscl2|
% which should be identical to
% |cdocsdrf|, |cdocsch1| and |cdocsfn2|, respectively:
% \begin{center}
% \begin{tabular}{l}
% |latex -jobname cdocscld \|\\
% |  "\def\version{draft}\input{childdoc.def}\childdocforward{cdocsamp}"|\\
% |latex -jobname cdocscl1 \|\\
% |  "\input{childdoc.def}\childdocforward[cdocsamp]{cdocsch1}"|\\
% |latex -jobname cdocscl2 \|\\
% |  "\def\version{final}\input{childdoc.def}\childdocforward{cdocsch2}"|
% \end{tabular}
% \end{center}
% Note that the trailing backslash on each first line
% merely continues the input to the second line
% (for convenient cut ant paste).
% Furthermore, the command |latex| can be replaced by any
% of its alternative versions such as |pdflatex|.
%
% %%%%%%%%%%%%%%%%%%%%%%%%%%%%%%%%%%%%%%%%%%%%%%%%%%%%%%%%%%%%%%%%%%%%%%%%%%%%%%
% %%%%%%%%%%%%%%%%%%%%%%%%%%%%%%%%%%%%%%%%%%%%%%%%%%%%%%%%%%%%%%%%%%%%%%%%%%%%%%
% \section{Implementation}
%\iffalse
%<*package>
%\fi
%
% This section describes the definitions file |childdoc.def|.

% The definitions cannot be loaded using |\usepackage| or |\RequirePackage|
% which has a mechanism to prevent loading a style file more than once.
% When loading the definitions by means of |\input|
% multiple instances have to be prevented manually:
%\iffalse
%This code needs to be before the `\ProvidesFile' directive
%which is defined at the beginning of this file.
%Therefore it is also placed there and commented out here.
%</package>
%<*discard>
%\fi
%    \begin{macrocode}
\ifdefined\childdocmain\endinput\fi
%    \end{macrocode}
%\iffalse
%</discard>
%<*package>
%\fi
%
% \macro{\ifchilddoc}
% \macro{\ifchilddocmanual}
% The conditional |\ifchilddoc| tells whether a
% child (true) or main (false) document is being compiled.
% The conditional |\ifchilddocmanual| tells whether
% the |\includeonly| mechanism is used (false) or
% the selection of child files must be performed manually (true).
% The definitions initialise to false:
%    \begin{macrocode}
\newif\ifchilddoc
\newif\ifchilddocmanual
%    \end{macrocode}

% \macro{\childdocname}
% \macro{\childdocjob}
% The macro |\childdocname| stores the name of the main document
% to be compiled. The macro |\childdocjob| stores the name of
% the document on which the \LaTeX{} compiler was originally invoked.
% The content of |\jobname| cannot be compared
% to filenames specified in the source due to different catcodes.
% The following code rescans |\jobname|, stores the result
% in |\childdocname| and saves a copy in |\childdocjob|:
%    \begin{macrocode}
\edef\childdocname{\scantokens\expandafter{\jobname\noexpand}}
\let\childdocjob\childdocname
%    \end{macrocode}

% \macro{\childdocdisable}
% The macro |\childdocdisable| prevents the main file
% from being processed more than once.
% At this stage, the main document command |\childdocmain|
% is assumed to be called once again where it should do nothing.
% Any subsequent call to it should prevent
% a secondary processing of the main document
% It overwrites the forwarding commands
% |\childdocof| and |\childdocforward|
% with empty macros to prevent further inclusions of the main document:
%    \begin{macrocode}
\newcommand{\childdocdisable}
{
  \renewcommand{\childdocmain}[1]{\renewcommand{\childdocmain}[1]{\endinput}}
  \renewcommand{\childdocof}[1]{}
  \renewcommand{\childdocby}[2][]{}
  \renewcommand{\childdocforward}[2][]{}
  \renewcommand{\childdocdisable}{}
}
%    \end{macrocode}

% \macro{\childdocmain}
% The macro |\childdocmain| is to be called at the top of the main file
% with nothing or the main filename (without extension) as argument.
% First, it breaks loops.
% If the argument is not empty and does not match |\childdocname|
% (which is set by the first inclusion of |childdoc.def|),
% |\ifchilddoc| is set to true, |\includeonly| is applied to the child file
% and |\jobname| is set to the main file
% (for proper handling of |.aux| files):
%    \begin{macrocode}
\newcommand{\childdocmain}[1]
{
  \childdocdisable\childdocmain{}
  \if?#1?\else
    \begingroup
      \def\childdoctmp{#1}
      \ifx\childdoctmp\childdocname
        \def\childdoctmp{}
      \else
        \def\childdoctmp
        {
          \childdoctrue
          \includeonly{\childdocname}
          \def\childdocjob{#1}
          \def\jobname{#1}
        }
      \fi
      \expandafter
    \endgroup
    \childdoctmp
  \fi
}
%    \end{macrocode}

% \macro{\childdocof}
% The command |\childdocof| redirects
% compilation to the main file |#1|.
%    \begin{macrocode}
\newcommand{\childdocof}[1]
{
  \childdocdisable
  \childdoctrue
  \includeonly{\childdocname}
  \def\jobname{#1}
  \def\childdocjob{#1}
  \input{#1}
}
%    \end{macrocode}

% \macro{\childdocby}
% The command |\childdocby| ....
%    \begin{macrocode}
\newcommand{\childdocby}[2][]
{
  \childdocdisable
  \childdoctrue
  \childdocmanualtrue
  \if?#1?\else
    \def\jobname{#2}
  \fi
  \def\childdocjob{#2}
  \input{#2}
  \endinput
}
%    \end{macrocode}

% \macro{\childdocforward}
% The command |\childdocforward| redirects
% compilation to the main file or
% (if the optional argument is given) a child file.
% Parameters are set as if the main file
% or a child file starting with |\childdocof| was compiled.
% Then compilation is handed over to the main file:
%    \begin{macrocode}
\newcommand{\childdocforward}[2][]
{
  \begingroup
    \if?#1?
      \def\childdoctmp
      {
        \def\childdocname{#2}
        \def\childdocjob{#2}
        \def\jobname{#2}
        \input{#2}
        \endinput
      }
    \else
      \def\childdoctmp
      {
        \childdocdisable
        \def\childdocname{#2}
        \childdoctrue
        \includeonly{#2}
        \def\childdocjob{#1}
        \def\jobname{#1}
        \input{#1}
        \endinput
      }
    \fi
    \expandafter
  \endgroup
  \childdoctmp
}
%    \end{macrocode}

% \macro{\childdocforwardprefix}
% The command |\childdocforwardprefix| redirects
% compilation to the main or a child file by means of a pattern.
% The prefix |#1| in the current filename is replaced by |#2|
% and the suffix of the current filename is kept
% (it is assumed that the filename does not contain the substring `|~~~|'
% which is used as a delimiter).
% Compilation is handed over to the new file by |\childdocforward|:
%    \begin{macrocode}
\newcommand{\childdocforwardprefix}[3][]
{
  \begingroup
    \def\childdocextract #2##1~~~{\def\childdoctmp{\childdocforward[#1]{#3##1}}}
    \expandafter\childdocextract\childdocname~~~
    \expandafter
  \endgroup
  \childdoctmp
}
%    \end{macrocode}

% \macro{\childdoc}
% The deprecated macro |\childdoc| is a legacy version of |\childdocmain|:
%    \begin{macrocode}
\newcommand{\childdoc}{\childdocmain}
%    \end{macrocode}

% \macro{\childdocredirect}
% The deprecated macro |\childdocredirect| is a legacy version
% of |\childdocforward| and |\childdocforwardprefix|:
%    \begin{macrocode}
\newcommand{\childdocredirect}[2][]
{
  \begingroup
    \if?#1?
      \def\childdoctmp{\childdocforward{#2}}
    \else
      \def\childdoctmp{\childdocforwardprefix{#1}{#2}}
    \fi
    \expandafter
  \endgroup
  \childdoctmp
}
%    \end{macrocode}

%\iffalse
%</package>
%\fi
%
\endinput
\childdocforward{cdocsamp}"|\\
% |latex -jobname cdocscl1 \|\\
% |  "% \iffalse
%
% childdoc.dtx Copyright (C) 2017-2018 Niklas Beisert
%
% This work may be distributed and/or modified under the
% conditions of the LaTeX Project Public License, either version 1.3
% of this license or (at your option) any later version.
% The latest version of this license is in
%   http://www.latex-project.org/lppl.txt
% and version 1.3 or later is part of all distributions of LaTeX
% version 2005/12/01 or later.
%
% This work has the LPPL maintenance status `maintained'.
%
% The Current Maintainer of this work is Niklas Beisert.
%
% This work consists of the files childdoc.dtx and childdoc.ins
% and the derived files childdoc.def and cdocsamp.tex with
% cdocsch1.tex, cdocsch2.tex, cdocsdrf.tex, cdocsfn1.tex, cdocsfn2.tex.
%
%<package>\ifdefined\childdocmain\endinput\fi
%<package>\ProvidesFile{childdoc.def}[2018/12/30 v2.0 child document driver]
%<samplemain>\ProvidesFile{cdocsamp.tex}[2018/12/30 v2.0 sample for childdoc]
%<*driver>
%\ProvidesFile{childdoc.drv}[2018/12/30 v2.0 childdoc reference manual file]
\PassOptionsToClass{10pt,a4paper}{article}
\documentclass{ltxdoc}

\usepackage[margin=35mm]{geometry}
\usepackage{hyperref}
\usepackage{hyperxmp}
\usepackage[usenames]{color}

\hypersetup{colorlinks=true}
\hypersetup{pdfstartview=FitH}
\hypersetup{pdfpagemode=UseNone}
\hypersetup{pdfsource={}}
\hypersetup{pdflang={en-UK}}
\hypersetup{pdfcopyright={Copyright 2017-2018 Niklas Beisert.
  This work may be distributed and/or modified under the
  conditions of the LaTeX Project Public License, either version 1.3
  of this license or (at your option) any later version.}}
\hypersetup{pdflicenseurl={http://www.latex-project.org/lppl.txt}}
\hypersetup{pdfcontactaddress={ETH Zurich, ITP, HIT K,
  Wolfgang-Pauli-Strasse 27}}
\hypersetup{pdfcontactpostcode={8093}}
\hypersetup{pdfcontactcity={Zurich}}
\hypersetup{pdfcontactcountry={Switzerland}}
\hypersetup{pdfcontactemail={nbeisert@itp.phys.ethz.ch}}
\hypersetup{pdfcontacturl={http://people.phys.ethz.ch/\xmptilde nbeisert/}}

\newcommand{\secref}[1]{\hyperref[#1]{section \ref*{#1}}}

\parskip1ex
\parindent0pt
\let\olditemize\itemize
\def\itemize{\olditemize\parskip0pt}

\begin{document}

\title{The \textsf{childdoc} Package}
\hypersetup{pdftitle={The childdoc Package}}
\author{Niklas Beisert\\[2ex]
  Institut f\"ur Theoretische Physik\\
  Eidgen\"ossische Technische Hochschule Z\"urich\\
  Wolfgang-Pauli-Strasse 27, 8093 Z\"urich, Switzerland\\[1ex]
  \href{mailto:nbeisert@itp.phys.ethz.ch}
  {\texttt{nbeisert@itp.phys.ethz.ch}}}
\hypersetup{pdfauthor={Niklas Beisert}}
\hypersetup{pdfsubject={Manual for the LaTeX2e Package childdoc}}
\date{30 December 2018, \textsf{v2.0}}
\maketitle

\begin{abstract}\noindent
\textsf{childdoc} is a \LaTeXe{} package
that enables the direct compilation
of document sections included by |\include|
to individual files.
\end{abstract}

\begingroup
\parskip0ex
\tableofcontents
\endgroup

%%%%%%%%%%%%%%%%%%%%%%%%%%%%%%%%%%%%%%%%%%%%%%%%%%%%%%%%%%%%%%%%%%%%%%%%%%%%%%%%
%%%%%%%%%%%%%%%%%%%%%%%%%%%%%%%%%%%%%%%%%%%%%%%%%%%%%%%%%%%%%%%%%%%%%%%%%%%%%%%%
\section{Introduction}

\LaTeX{} provides a mechanism to structure a large document (such as a book)
into a main file and several child files (containing the chapters)
using the |\include| command.
This mechanism is beneficial for documents
which span hundreds of pages in order to
make the source file(s) more manageable.
Moreover, compilation can be restricted to
selected child files by means of the |\includeonly| command.
The latter feature can be used to reduce the compilation time while editing
(this was significantly more useful in the earlier days of \LaTeX{})
or to generate a smaller document which is easier to navigate.
Another application of |\includeonly| is to generate
documents consisting of selected parts of the complete document.

However, there are a few drawbacks of the plain |\include| mechanism:
\begin{itemize}
\item
The child files cannot be compiled on their own,
they can only be compiled via the main file.
A naive editing environment
(such as a text editor with an option
to have the current file processed by \LaTeX)
may require one to switch to the main file before compiling;
attempting to compile the child file produces errors.
\item
The main file must be modified (each time)
to adjust the |\includeonly| command
to the present needs. This easily leaves the main file in a messy state.
\item
The generated document will always carry the filename
of the main document. This is inconvenient if
several child files are to be compiled and
to be kept for distribution.
\end{itemize}

The present package provides a simple interface
to make child files individually compilable by \LaTeX{}.
Compiling a child file then has the same effect as compiling
the main file with an |\includeonly| command
to select the appropriate child.
Moreover the generated document will carry the name of the child
rather than the main file.
This resolves all three above issues.

This feature is meant to make the editing of books,
thesis documents and lecture notes somewhat more convenient.
However, the package can also be used efficiently for
composing a series of documents (such as exercise sheets)
which are typically distributed individually.
It then assists the author in generating the individual documents
(potentially in different versions)
as well as a document containing the collected series.
Another application is in developing style files
or other kinds of included material
where compilation of the style file could redirect
to a sample or test file.

%%%%%%%%%%%%%%%%%%%%%%%%%%%%%%%%%%%%%%%%%%%%%%%%%%%%%%%%%%%%%%%%%%%%%%%%%%%%%%%%
%%%%%%%%%%%%%%%%%%%%%%%%%%%%%%%%%%%%%%%%%%%%%%%%%%%%%%%%%%%%%%%%%%%%%%%%%%%%%%%%
\section{Usage}

First of all, the package \textsf{childdoc} is \emph{not} a standard
\LaTeXe{} |.sty| style file! Therefore it needs to be invoked in
a non-standard way.

%%%%%%%%%%%%%%%%%%%%%%%%%%%%%%%%%%%%%%%%%%%%%%%%%%%%%%%%%%%%%%%%%%%%%%%%%%%%%%%%
\subsection{Included Files}
\label{sec:include}

%%%%%%%%%%%%%%%%%%%%%%%%%%%%%%%%%%%%%%%%
\DescribeMacro{\childdocmain}
To use the package, add the commands
\begin{center}
\begin{tabular}{l}
|\input{childdoc.def}|\\
|\childdocmain{}|\\
\end{tabular}
\end{center}
at the very top of the main \LaTeX{} file,
in particular \emph{before} the |\documentclass| statement!
The argument of |\childdocmain| should be left empty
(but it must be present).

%%%%%%%%%%%%%%%%%%%%%%%%%%%%%%%%%%%%%%%%
\DescribeMacro{\childdocof}
Furthermore, add the commands
\begin{center}
\begin{tabular}{l}
|\input{childdoc.def}|\\
|\childdocof{|\textit{main}|}|\\
\end{tabular}
\end{center}
at the top of every child file \textit{child}
which is included by |\include{|\textit{child}|}|
from within the main file
(or at least for those files to be compiled individually).
The argument \textit{main} must be the filename of the main file.

There are a couple of
considerations in setting up the main and child documents:

%%%%%%%%%%%%%%%%%%%%%%%%%%%%%%%%%%%%%%%%
\paragraph{Restrictions.}

Please note the following restrictions:
\begin{itemize}
\item
|\childdocmain| must be called with one argument \textit{main}
to ensure compatibility with earlier version of the package.
It must either be empty (|\childdocmain{}|)
or precisely match the filename of the main file in which it is specified.
See \secref{sec:detection} for further information.
\item
The filename \textit{main} must be specified without the |.tex| extension.
\item
The filename \textit{main} is case sensitive
(even in case-insensitive file systems)
due to internal string comparison.
\item
The argument \textit{main} should be fully expanded, it cannot be a macro.
\item
Subdirectories and special characters should be avoided in filenames.
\item
The command |\childdocmain{|\textit{main}|}| must be followed by a whitespace.
It should not be followed immediately by another command
or by a comment mark `|%|'.
This is because the \TeX{} parser reads the token immediately following
the argument of |\childdocmain| and puts it
at the beginning of every child section;
however, a white\-space is ignored.
\end{itemize}

%%%%%%%%%%%%%%%%%%%%%%%%%%%%%%%%%%%%%%%%
\paragraph{Content of Main File.}

It is advisable to place all content in the child files included by |\include|.
Any output contained in the main file will appear in all child documents
unless suppressed manually;
it cannot be suppressed automatically by the |\includeonly| directive
and thus should normally be avoided.
A method to include some content in the main file
by means of conditional processing is described in \secref{sec:conditional}.

%%%%%%%%%%%%%%%%%%%%%%%%%%%%%%%%%%%%%%%%
\paragraph{Page Numbering.}

When only a part of the document is compiled,
the appropriate numbering of pages
(as well as other status parameters)
is determined from the |.aux| files.
The latter contain information from previous passes.
However this information needs to propagate through
all intermediate child documents.
Therefore the page numbering in child documents may well
be inconsistent until the complete document is compiled at least once.

A useful (if unconventional) way to always ensure a consistent
page numbering is to restart the numbering in each child document
and denote the pages by `\textit{child}|.|\textit{page}'
where \textit{child} represents the chapter/section number of the child file.
This can be achieved by the command
|\numberwithin{page}{|\textit{child}|}|
of the \textsf{amsmath} package
where \textit{child} can be |chapter| or |section|
depending on the chosen structuring.
Alternatively, one can modify the macro |\thepage| appropriately
and reset the counter |page| at the start of each child file.

%%%%%%%%%%%%%%%%%%%%%%%%%%%%%%%%%%%%%%%%%%%%%%%%%%%%%%%%%%%%%%%%%%%%%%%%%%%%%%%%
\subsection{Conditional Processing}
\label{sec:conditional}

The package provides a mechanism to compile different versions
of a document. To customise the versions further some conditional processing
can come in handy to distinguish which version is being compiled.
The package provides two macros to describe the compilation context:

%%%%%%%%%%%%%%%%%%%%%%%%%%%%%%%%%%%%%%%%
\DescribeMacro{\ifchilddoc}
The conditional |\ifchilddoc| distinguishes between the compilation of
child documents and the main document:
%
\begin{center}
|\ifchilddoc |\textit{child-code}| |[|\||else |\textit{main-code}]| \||fi|
\end{center}

%%%%%%%%%%%%%%%%%%%%%%%%%%%%%%%%%%%%%%%%
\DescribeMacro{\childdocname}
\DescribeMacro{\childdocjob}
The macro |\childdocname| contains the filename (without extension)
of the main or child file being processed.
Note that |\childdocjob| will always contain the name of the main file.

%%%%%%%%%%%%%%%%%%%%%%%%%%%%%%%%%%%%%%%%
\paragraph{Title Page.}

Conditional processing can be used to include a title or banner page
in the main document when proper precautions are taken.
Importantly, the code in the main file should ensure that the page counter
(as well as other status parameters which are stored in the |.aux| files)
takes the same value after the conditional processing.
Otherwise the page numbers may take divergent values
depending on which part is compiled.

For example, a title page could be declared by:
%
\begin{center}
\begin{tabular}{l}
|\ifchilddoc\||else|\\
|\addtocounter{page}{-1}|\\
\textit{code for title page}\\
|\newpage|\\
|\||fi|
\end{tabular}
\end{center}
%
A banner page for the child documents can be generated by:
%
\begin{center}
\begin{tabular}{l}
|\ifchilddoc|\\
|\addtocounter{page}{-1}|\\
\textit{code for banner page}\\
|\newpage|\\
|\||fi|
\end{tabular}
\end{center}
%
Here one could write a message such as:
\begin{center}
|This is the part \childdocname{} of \childdocjob{}.|
\end{center}

%%%%%%%%%%%%%%%%%%%%%%%%%%%%%%%%%%%%%%%%%%%%%%%%%%%%%%%%%%%%%%%%%%%%%%%%%%%%%%%%
\subsection{Flags}
\label{sec:flags}

The package makes it easy to generate different versions
of the main or child documents.
To this end compilation flags can be defined
and assigned different default values.
They will be particularly useful in conjunction
with the forwarding mechanism described in \secref{sec:forward}.

For example, it may be useful to have a flag |\version|
which can be set to |draft| or |final|.
The document source will contain some conditional code
depending on the value of |\version|.
Suppose further, the flag should default to |final| for the main file
and to |draft| for child files
which is a natural assignment for editing the document.
This is achieved by placing the following code
in the preamble of the main document
(below the |\childdocmain| directive):
%
\begin{center}
\begin{tabular}{l}
|\ifchilddoc|\\
|\providecommand{\version}{draft}|\\
|\||else|\\
|\providecommand{\version}{final}|\\
|\||fi|
\end{tabular}
\end{center}
%
The definition by |\providecommand| makes sure
that previous definitions are not overwritten.
Further statements |\providecommand{\version}{...}|
can thus be added before the above code to override it.

For the main file, one might add a line
(between |\childdocmain| and the above block)
%
\begin{center}
|%\ifchilddoc\||else\providecommand{\version}{draft}\||fi|
\end{center}
%
which can be uncommented to produce a draft version.
Likewise one can add a line to the very top of a child file
(above the |\childdocof{|\textit{main}|}| directive)
%
\begin{center}
|%\providecommand{\version}{final}|
\end{center}
%
which can be uncommented to produce the final version of this child document.

%%%%%%%%%%%%%%%%%%%%%%%%%%%%%%%%%%%%%%%%%%%%%%%%%%%%%%%%%%%%%%%%%%%%%%%%%%%%%%%%
\subsection{Forwarding}
\label{sec:forward}

Different versions of the main or child documents
using compilation flags as described in \secref{sec:flags}
can be (permanently) stored in different files
for convenient compilation, viewing and distribution.
To this end, the package defines a command
to pass on compilation to a different file:

%%%%%%%%%%%%%%%%%%%%%%%%%%%%%%%%%%%%%%%%
\DescribeMacro{\childdocforward}
The command |\childdocforward| redirects processing to
another source file:
%
\begin{center}
\begin{tabular}{l}
|\input{childdoc.def}|\\
|\childdocforward[|\textit{main}|]{|\textit{dest}|}|\\
\end{tabular}
\end{center}
%
The argument \textit{dest} is the destination file
(without extension).
It should be the main file or one of the child files.
Note that further \textsf{childdoc} directives
such as |\childdocof| and |\childdocforward|
in the indicated file will be processed in this form.
The optional argument \textit{main}
passes on directly to the main file \textit{main}
while pretending to compile the child \textit{dest}.
This form behaves as if \textit{dest}
issues |\childdocof{|\textit{main}|}| right away,
and no further \textsf{childdoc} directives will be processed.

%%%%%%%%%%%%%%%%%%%%%%%%%%%%%%%%%%%%%%%%
\DescribeMacro{\...prefix}
In the alternative form |\childdocforwardprefix|,
%
\begin{center}
\begin{tabular}{l}
|\input{childdoc.def}|\\
|\childdocforwardprefix[|\textit{main}|]{|\textit{prefix}|}{|\textit{dest}|}|
\end{tabular}
\end{center}
%
the destination file is determined by a pattern
depending on the current file:
To make this work, the current file must be called
`{\textit{prefix}\hspace{0.2em}\textit{suffix}}'
with \textit{prefix} matching precisely the argument.
Processing is then passed on to the file
`{\textit{dest}\hspace{0.2em}\textit{suffix}}'.
Surely, the same effect is achieved by
directly specifying the
argument `{\textit{dest}\hspace{0.2em}\textit{suffix}}'
in the first form.
However, that requires to set up a different file
for each child. With the alternative form of the command
all these files can have exactly the same content
which simplifies setting them up and maintaining them.

For example, the following file |draft.tex|
with a compilation flag |\version| as described in \secref{sec:flags}
compiles the main document as a draft:
%
\begin{center}
\begin{tabular}{l}
|\def\version{draft}|\\
|\input{childdoc.def}|\\
|\childdocforward{|\textit{main}|}|
\end{tabular}
\end{center}
%
Likewise, the following files |final|\textit{nn}|.tex|
compile the final version of the child document
|child|\textit{nn}|.tex|:
%
\begin{center}
\begin{tabular}{l}
|\def\version{final}|\\
|\input{childdoc.def}|\\
|\childdocforwardprefix{final}{child}|
\end{tabular}
\end{center}
%

Note that when several versions of a main file and/or of each child file
are to be generated, it may be convenient to set up a |Makefile| or
shell script to automatise the process.

%%%%%%%%%%%%%%%%%%%%%%%%%%%%%%%%%%%%%%%%%%%%%%%%%%%%%%%%%%%%%%%%%%%%%%%%%%%%%%%%
\subsection{Command Line Processing}
\label{sec:commandline}

The effect of redirection files can also be achieved by invoking
the \LaTeX{} compiler with a more elaborate command line.
Most conveniently this should be done as part
of a shell script or a |Makefile|.

When using \textsf{childdoc} in the main file, the following
command lines effectively perform a redirection
(note that depending on the shell being used,
backslashes may have to be doubled: `|\|' $\to$ `|\\|'):
%
\begin{center}
|... -jobname "|\textit{target}|" |\\|"|[\textit{flags}]%
|\input{childdoc.def}\childdocforward[|\textit{main}|]{|\textit{dest}|}"|
\end{center}
%
Here \textit{target} is the name of the output file,
\textit{main} is the name of the main file
and \textit{dest} is the name of the main or child file to be processed
(all filenames without extensions).
The optional argument \textit{main} can be omitted
if \textit{main} matches \textit{dest}.
Optionally, compilation \textit{flags} can be defined via |\def| commands.
This command line makes the \TeX{} engine believe
it is compiling the file \textit{target}
whose content is specified as the latter parameter.
The provided code then forwards the processing to
\textit{main} or \textit{dest} as described in \secref{sec:forward}.

%%%%%%%%%%%%%%%%%%%%%%%%%%%%%%%%%%%%%%%%%%%%%%%%%%%%%%%%%%%%%%%%%%%%%%%%%%%%%%%%
\subsection{Include by Input}
\label{sec:input}

Including child documents by |\include| has some restrictions by design.
Most notably, the content of a child document always occupies
its own set of pages; pages cannot be shared between child documents.
Usually, this behaviour makes perfect sense
because each child document contain an essential part of the document.
However, in some situations it may be desirable to compose
a document from a collection of parts
without having mandatory page breaks between then.
For this case, the package
provides a mechanism to include parts
by |\input| which can also be processed individually.
However, by construction this mechanism
requires manual handling of the content to be output.

%%%%%%%%%%%%%%%%%%%%%%%%%%%%%%%%%%%%%%%%
\DescribeMacro{\ifchilddocmanual}
The main file should be prepared as usual, see \secref{sec:include}.
However, the document body must make a distinction
between processing of an individual part and of the main document, e.g.:
%
\begin{center}
\begin{tabular}{l}
|\ifchilddocmanual|\\
|\input{\childdocname}|\\
|\||else|\\
\textit{document body with }|\input{|\textit{part}|}|\\
|\||fi|
\end{tabular}
\end{center}
%
The conditional |\ifchilddocmanual| is true whenever
a part to be included by |\input| is being compiled,
and the name of the part is stored in |\childdocname|.

%%%%%%%%%%%%%%%%%%%%%%%%%%%%%%%%%%%%%%%%
\DescribeMacro{\childdocby}
Each part to be included by |\input| should start with:
%
\begin{center}
\begin{tabular}{l}
|\input{childdoc.def}|\\
|\childdocby{|\textit{main}|}|\\
\end{tabular}
\end{center}
%
The directive |\childdocby| is similar to |\childdocof|
described in \secref{sec:include},
but the subsequent selection of content must be done manually.
To that end, both |\ifchilddoc| and |\ifchilddocmanual|
will be true upon processing of a part,
and the name of the part is stored in |\childdocname|.
Note that |\jobname| will be set to the filename of the current part
so that each part receives an individual |.aux| file
that does not interfere with the |.aux| file(s) of the main document.
This behaviour can be altered by the alternative form
|\childdocby[*]{|\textit{main}|}| (with a non-empty optional argument)
which uses the |.aux| file of the main document
by setting |\jobname| to \textit{main}.

%%%%%%%%%%%%%%%%%%%%%%%%%%%%%%%%%%%%%%%%%%%%%%%%%%%%%%%%%%%%%%%%%%%%%%%%%%%%%%%%
\subsection{Driver Development}
\label{sec:driver}

The \textsf{childdoc} mechanism can also be use for the development
of definition files such as \LaTeX{} styles or classes.
This case differs from the above setup with multiple parts
included by |\include| in that no |\includeonly| should be invoked.
This can be achieved by starting the include file
(before |\ProvidesPackage|) with:
%
\begin{center}
\begin{tabular}{l}
|\input{childdoc.def}|\\
|\childdocforward{|\textit{main}|}|\\
\end{tabular}
\end{center}
%
or alternatively with:
%
\begin{center}
\begin{tabular}{l}
|\input{childdoc.def}|\\
|\childdocby{|\textit{main}|}|\\
\end{tabular}
\end{center}
%
Both forms have slightly different effects as described above.
The main file is prepared as usual, see \secref{sec:include}.

%%%%%%%%%%%%%%%%%%%%%%%%%%%%%%%%%%%%%%%%%%%%%%%%%%%%%%%%%%%%%%%%%%%%%%%%%%%%%%%%
\subsection{Legacy Detection}
\label{sec:detection}

The directive |\childdocmain| in the main file can detect
whether the complete document or merely a child is to be compiled
even without using the directive |\childdocof|.
This method is deprecated because it is less robust
and there is no compelling reason to use it;
it is merely provided for backward compatibility
and it may be removed in future versions.

If the detection mechanism is to be used,
it is mandatory to correctly specify
the filename of the main file as the argument of |\childdocmain|:
%
\begin{center}
\begin{tabular}{l}
|\input{childdoc.def}|\\
|\childdocmain{|\textit{main}|}|\\
\end{tabular}
\end{center}
%
If |\jobname| does not match the argument \textit{main} of |\childdocmain|,
it is assumed that |\jobname| points to the child file to be compiled.
When using |\childdocmain| with the main file specified as argument,
it suffices to start a child file
with just |\input{|\textit{main}|}|
without loading of the package and using |\childdocof|.
If instead all processing is done
with the appropriate \textsf{childdoc} directives,
the argument of \textit{main} of |\childdocmain| can be empty.

An alternative version of the command line processing described
in \secref{sec:commandline} using the detection mechanism reads:
%
\begin{center}
|... -jobname "|\textit{target}|" "|[\textit{flags}]%
[|\def\jobname{|\textit{dest}|}|]|\input{|\textit{main}|}"|
\end{center}

%%%%%%%%%%%%%%%%%%%%%%%%%%%%%%%%%%%%%%%%%%%%%%%%%%%%%%%%%%%%%%%%%%%%%%%%%%%%%%%%
\subsection{Manual Code}
\label{sec:manual}

In case one cannot be certain whether the definitions file |childdoc.def|
is installed on the target \TeX{} distribution
and one prefers not to ship it,
it is conceivable to paste a few relevant commands into the sources.

To that end, drop all statements |\input{childdoc.def}|
and perform the replacements as outlined below.
Instead of |\childdocmain{|\textit{main}|}| add the following code
to the top of the main file:
%
\begin{center}
\begin{tabular}{l}
|\||ifdefined\childdocname\endinput\||fi\newif\ifchilddoc|\\
|\edef\childdocname{\scantokens\expandafter{\jobname\noexpand}}|\\
|\def\childdocmain{|\textit{main}|}\||ifx\childdocmain\childdocname\||else|\\
|\childdoctrue\includeonly{\childdocname}\let\jobname\childdocmain\||fi|\\
\end{tabular}
\end{center}
%
Instead of |\childdocof{|\textit{main}|}| just include the main file
at the top of each child file:
%
\begin{center}
|\input{|\textit{main}|}|
\end{center}
%
A simple redirection |\childdocforward{|\textit{dest}|}| is achieved by:
%
\begin{center}
|\def\jobname{|\textit{dest}|}\input{\jobname}|
\end{center}
%
The redirection with prefix
|\childdocforwardprefix[|\textit{prefix}|]{|\textit{dest}|}|
is accomplished by:
%
\begin{center}
\begin{tabular}{l}
|{\edef\jobname{\scantokens\expandafter{\jobname\noexpand}}|\\
|\def\redirectjob |\textit{prefix}|#1~~~{\gdef\jobname{|\textit{dest}|#1}}|\\
|\expandafter\redirectjob\jobname~~~}\input{\jobname}|
\end{tabular}
\end{center}

In an alternative approach,
child documents can be compiled by a specific command line
without additional code or specific definitions:
%
\begin{center}
|... -jobname "|\textit{target}|" "|[\textit{flags}]%
|\includeonly{|\textit{dest}|}\input{|\textit{main}|}"|
\end{center}
%

%%%%%%%%%%%%%%%%%%%%%%%%%%%%%%%%%%%%%%%%%%%%%%%%%%%%%%%%%%%%%%%%%%%%%%%%%%%%%%%%
%%%%%%%%%%%%%%%%%%%%%%%%%%%%%%%%%%%%%%%%%%%%%%%%%%%%%%%%%%%%%%%%%%%%%%%%%%%%%%%%
\section{Information}

%%%%%%%%%%%%%%%%%%%%%%%%%%%%%%%%%%%%%%%%%%%%%%%%%%%%%%%%%%%%%%%%%%%%%%%%%%%%%%%%
\subsection{Copyright}

Copyright \copyright{} 2017--2018 Niklas Beisert

This work may be distributed and/or modified under the
conditions of the \LaTeX{} Project Public License, either version 1.3
of this license or (at your option) any later version.
The latest version of this license is in
  \url{http://www.latex-project.org/lppl.txt}
and version 1.3 or later is part of all distributions of \LaTeX{}
version 2005/12/01 or later.

This work has the LPPL maintenance status `maintained'.

The Current Maintainer of this work is Niklas Beisert.

This work consists of the files |README.txt|, |childdoc.ins| and |childdoc.dtx|
as well as the derived files |childdoc.def|, |cdocsamp.tex|
with |cdocsch1.tex|, |cdocsch2.tex|, |cdocspt3.tex|, |cdocspt4.tex|,
|cdocsdrf.tex|, |cdocsfn1.tex|, |cdocsfn2.tex|
as well as |childdoc.pdf|.

%%%%%%%%%%%%%%%%%%%%%%%%%%%%%%%%%%%%%%%%%%%%%%%%%%%%%%%%%%%%%%%%%%%%%%%%%%%%%%%%
\subsection{Files and Installation}

The package consists of the files:
%
\begin{center}
\begin{tabular}{ll}
    |README.txt|   & readme file \\
    |childdoc.ins| & installation file \\
    |childdoc.dtx| & source file \\
    |childdoc.def| & definition file \\
    |cdocsamp.tex| & sample main file \\
    |cdocsch1.tex| & sample include file \\
    |cdocsch2.tex| & sample include file \\
    |cdocspt3.tex| & sample part file \\
    |cdocspt4.tex| & sample part file \\
    |cdocsdrf.tex| & sample redirection file \\
    |cdocsfn1.tex| & sample redirection file \\
    |cdocsfn2.tex| & sample redirection file \\
    |childdoc.pdf| & manual
\end{tabular}
\end{center}
%
The distribution consists of the files
|README.txt|, |childdoc.ins| and |childdoc.dtx|.
%
\begin{itemize}
\item
Run (pdf)\LaTeX{} on |childdoc.dtx|
to compile the manual |childdoc.pdf| (this file).
\item
Run \LaTeX{} on |childdoc.ins| to create the definitions file |childdoc.def|
and the sample |cdocsamp.tex| with include files
|cdocsch1.tex|, |cdocsch2.tex|, |cdocspt3.tex|, |cdocspt4.tex|,
|cdocsdrf.tex|, |cdocsfn1.tex|, |cdocsfn2.tex|.
Then copy the file |childdoc.def| to an appropriate directory of your \LaTeX{}
distribution, e.g.\ \textit{texmf-root}|/tex/latex/childdoc|.
\end{itemize}

%%%%%%%%%%%%%%%%%%%%%%%%%%%%%%%%%%%%%%%%%%%%%%%%%%%%%%%%%%%%%%%%%%%%%%%%%%%%%%%%
\subsection{Related CTAN Packages}

There are several other packages which offer a similar functionality:
%
\begin{itemize}
\item
The packages
\href{http://ctan.org/pkg/docmute}{\textsf{docmute}},
\href{http://ctan.org/pkg/includex}{\textsf{includex}} and
\href{http://ctan.org/pkg/standalone}{\textsf{standalone}}
provide commands to include only the document body of
a child file thus allowing both files to be compiled individually.
\item
The packages \href{http://ctan.org/pkg/subdocs}{\textsf{subdocs}}
and \href{http://ctan.org/pkg/subfiles}{\textsf{subfiles}}
provide structures in which the main and child documents can be
encapsulated and allowing them to be compiled individually.
The inclusion mechanism is different from the conventional |\include|.
\item
The package \href{http://ctan.org/pkg/combine}{\textsf{combine}}
is an elaborate solution to combine several documents into one.
\end{itemize}
%
See also the CTAN topic \href{http://ctan.org/topic/subdocs}{\textsf{subdocs}}
for further related packages.
The present package differs from the above solutions in that
a document structure constructed with the conventional |\include| mechanism
just needs two extra commands at the top of every file
such that all constituent files can be compiled individually.

%%%%%%%%%%%%%%%%%%%%%%%%%%%%%%%%%%%%%%%%%%%%%%%%%%%%%%%%%%%%%%%%%%%%%%%%%%%%%%%%
%\subsection{Feature Suggestions}
%
%The following is a list of features which may be useful for future
%versions of this package:
%%
%\begin{itemize}
%\item
%\ldots
%\end{itemize}

%%%%%%%%%%%%%%%%%%%%%%%%%%%%%%%%%%%%%%%%%%%%%%%%%%%%%%%%%%%%%%%%%%%%%%%%%%%%%%%%
\subsection{Revision History}

%%%%%%%%%%%%%%%%%%%%%%%%%%%%%%%%%%%%%%%%
\paragraph{v2.0:} 2018/12/30

\begin{itemize}
\item
immediate forward processing
\item
added |\childdocby| mechanism
\item
manual restructured
\end{itemize}

%%%%%%%%%%%%%%%%%%%%%%%%%%%%%%%%%%%%%%%%
\paragraph{v1.6:} 2018/01/17

\begin{itemize}
\item
application for development of include files
\item
corrections to manual
\end{itemize}

%%%%%%%%%%%%%%%%%%%%%%%%%%%%%%%%%%%%%%%%
\paragraph{v1.5:} 2017/05/21

\begin{itemize}
\item
more complete structuring introduced
\item
|\childdocof| introduced
\item
|\childdoc| renamed to |\childdocmain|
\item
|\childredirect| renamed to |\childdocforward| and |\childdocforwardprefix|
and functionality expanded
\end{itemize}

%%%%%%%%%%%%%%%%%%%%%%%%%%%%%%%%%%%%%%%%
\paragraph{v1.0:} 2017/04/27

\begin{itemize}
\item
manual and install package
\item
first version published on CTAN
\end{itemize}

%%%%%%%%%%%%%%%%%%%%%%%%%%%%%%%%%%%%%%%%
\paragraph{v0.6:} 2017/04/26

\begin{itemize}
\item
redirection mechanism added
\end{itemize}

%%%%%%%%%%%%%%%%%%%%%%%%%%%%%%%%%%%%%%%%
\paragraph{v0.5:} 2017/04/26

\begin{itemize}
\item
functionality in definition file
\end{itemize}


%%%%%%%%%%%%%%%%%%%%%%%%%%%%%%%%%%%%%%%%%%%%%%%%%%%%%%%%%%%%%%%%%%%%%%%%%%%%%%%%
%%%%%%%%%%%%%%%%%%%%%%%%%%%%%%%%%%%%%%%%%%%%%%%%%%%%%%%%%%%%%%%%%%%%%%%%%%%%%%%%
%%%%%%%%%%%%%%%%%%%%%%%%%%%%%%%%%%%%%%%%%%%%%%%%%%%%%%%%%%%%%%%%%%%%%%%%%%%%%%%%
\appendix

\settowidth\MacroIndent{\rmfamily\scriptsize 000\ }

 \DocInput{childdoc.dtx}

\end{document}
%</driver>
% \fi
%
% %%%%%%%%%%%%%%%%%%%%%%%%%%%%%%%%%%%%%%%%%%%%%%%%%%%%%%%%%%%%%%%%%%%%%%%%%%%%%%
% %%%%%%%%%%%%%%%%%%%%%%%%%%%%%%%%%%%%%%%%%%%%%%%%%%%%%%%%%%%%%%%%%%%%%%%%%%%%%%
% \section{Sample}
%\iffalse
%<*samplemain>
%\fi
%
% The following presents a sample document
% with two chapters, two parts, a title page,
% a compile flag as well as three forwarding files to set the flag.
% It consists of eight |.tex| files:
% \begin{center}
% \begin{tabular}{ll}
% |cdocsamp.tex|&main file\\
% |cdocsch1.tex|&include file for chapter 1\\
% |cdocsch2.tex|&include file for chapter 2\\
% |cdocspt3.tex|&include file for part 3\\
% |cdocspt4.tex|&include file for part 4\\
% |cdocsdrf.tex|&forwarding file for main file in draft mode\\
% |cdocsfi1.tex|&forwarding file for final version of chapter 1\\
% |cdocsfi2.tex|&forwarding file for final version of chapter 2\\
% \end{tabular}
% \end{center}
% Each of the eight files can be compiled directly by the \LaTeX{} compiler.
%
% %%%%%%%%%%%%%%%%%%%%%%%%%%%%%%%%%%%%%%
% \paragraph{Main File.}
%
% The main file is called |cdocsamp.tex|.
%
% Load the \textsf{childdoc} definitions and
% declare the filename for the main document:
%    \begin{macrocode}
\input{childdoc.def}
\childdocmain{}
%    \end{macrocode}

% Optional override for |\version| flag:
%    \begin{macrocode}
%%\ifchilddoc\else\providecommand{\version}{draft}\fi
%    \end{macrocode}

% Define the default values for the |\version| flag
% (|final| for the main file and |draft| for childs):
%    \begin{macrocode}
\ifchilddoc
\providecommand{\version}{draft}
\else
\providecommand{\version}{final}
\fi
%    \end{macrocode}

% Load the standard document class:
%    \begin{macrocode}
\documentclass[12pt]{article}
%    \end{macrocode}

% Start the document body:
%    \begin{macrocode}
\begin{document}
%    \end{macrocode}

% Declare a title page.
% Print title, part of document being processed and version flag:
%    \begin{macrocode}
\addtocounter{page}{-1}
\begin{center}
{\LARGE\bfseries{}childdoc example\par}
\vspace{1cm}
\ifchilddoc
\ifchilddocmanual part\else chapter\fi:
`\childdocname' of `\childdocjob'\par
\else
main document: `\childdocjob'\par
\fi
version: \version\par
\end{center}
\newpage
%    \end{macrocode}

% Manually include selected file,
% otherwise process as usual:
%    \begin{macrocode}
\ifchilddocmanual
\section*{part `\childdocname'}
\input{\childdocname}
\else
%    \end{macrocode}

% Include the two chapters:
%    \begin{macrocode}
\include{cdocsch1}
\include{cdocsch2}
%    \end{macrocode}

% Include the two parts unless only chapters should be displayed:
%    \begin{macrocode}
\ifchilddoc\else
\section{part three}
\input{cdocspt3}
\section{part four}
\input{cdocspt4}
\fi
%    \end{macrocode}

% Process as usual until here:
%    \begin{macrocode}
\fi
%    \end{macrocode}

% End of document body:
%    \begin{macrocode}
\end{document}
%    \end{macrocode}
%\iffalse
%</samplemain>
%\fi
%
% %%%%%%%%%%%%%%%%%%%%%%%%%%%%%%%%%%%%%%
% \paragraph{Chapter Include Files.}
%
% The include files are called |cdocsch1.tex| and |cdocsch2.tex|.
%
%\iffalse
%<*samplechap1|samplechap2>
%\fi

% Optional override for |\version| flag:
%    \begin{macrocode}
%%\providecommand{\version}{final}
%    \end{macrocode}

% Include the main document:
%    \begin{macrocode}
\input{childdoc.def}
\childdocof{cdocsamp}
%    \end{macrocode}

%\iffalse
%</samplechap1|samplechap2>
%\fi
%
%\iffalse
%<*samplechap1>
%\fi
% Some text for chapter 1:
%    \begin{macrocode}
\section{one}
some text in chapter one
%    \end{macrocode}

%\iffalse
%</samplechap1>
%\fi
% Some text for chapter 2:
%\iffalse
%<*samplechap2>
%\fi
%    \begin{macrocode}
\section{two}
more text in chapter two
%    \end{macrocode}

%\iffalse
%</samplechap2>
%\fi
%
% %%%%%%%%%%%%%%%%%%%%%%%%%%%%%%%%%%%%%%
% \paragraph{Part Include Files.}
%
% The include files are called |cdocspt3.tex| and |cdocspt4.tex|.
%
%\iffalse
%<*samplepart3|samplepart4>
%\fi

% Optional override for |\version| flag:
%    \begin{macrocode}
%%\providecommand{\version}{final}
%    \end{macrocode}

% Include the main document:
%    \begin{macrocode}
\input{childdoc.def}
\childdocby{cdocsamp}
%    \end{macrocode}

%\iffalse
%</samplepart3|samplepart4>
%\fi
%
%\iffalse
%<*samplepart3>
%\fi
% Some text for part 3:
%    \begin{macrocode}
some text in part three
%    \end{macrocode}

%\iffalse
%</samplepart3>
%\fi
% Some text for part 4:
%\iffalse
%<*samplepart4>
%\fi
%    \begin{macrocode}
more text in part four
%    \end{macrocode}

%\iffalse
%</samplepart4>
%\fi
%
% %%%%%%%%%%%%%%%%%%%%%%%%%%%%%%%%%%%%%%
% \paragraph{Forwarding for a Complete Draft.}
%
% The following forwarding file |cdocsdrf.tex|
% compiles the main document in draft mode:
%\iffalse
%<*sampledraft>
%\fi
%    \begin{macrocode}
\def\version{draft}
\input{childdoc.def}
\childdocforward{cdocsamp}
%    \end{macrocode}

%\iffalse
%</sampledraft>
%\fi
%
% %%%%%%%%%%%%%%%%%%%%%%%%%%%%%%%%%%%%%%
% \paragraph{Forwarding for Final Version of the Chapters.}
%
% The following forwarding files |cdocsfn1.tex| and |cdocsfn2.tex|
% (with identical content)
% compile the final versions of the child documents
% |cdocsch1.tex| and |cdocsch2.tex|, respectively:
%\iffalse
%<*samplefinal>
%\fi
%    \begin{macrocode}
\def\version{final}
\input{childdoc.def}
\childdocforwardprefix[cdocsamp]{cdocsfn}{cdocsch}
%    \end{macrocode}

%\iffalse
%</samplefinal>
%\fi
%
% %%%%%%%%%%%%%%%%%%%%%%%%%%%%%%%%%%%%%%
% \paragraph{Command Line Processing.}
%
% The following three command lines generate the output files
% |cdocscld|, |cdocscl1| and |cdocscl2|
% which should be identical to
% |cdocsdrf|, |cdocsch1| and |cdocsfn2|, respectively:
% \begin{center}
% \begin{tabular}{l}
% |latex -jobname cdocscld \|\\
% |  "\def\version{draft}\input{childdoc.def}\childdocforward{cdocsamp}"|\\
% |latex -jobname cdocscl1 \|\\
% |  "\input{childdoc.def}\childdocforward[cdocsamp]{cdocsch1}"|\\
% |latex -jobname cdocscl2 \|\\
% |  "\def\version{final}\input{childdoc.def}\childdocforward{cdocsch2}"|
% \end{tabular}
% \end{center}
% Note that the trailing backslash on each first line
% merely continues the input to the second line
% (for convenient cut ant paste).
% Furthermore, the command |latex| can be replaced by any
% of its alternative versions such as |pdflatex|.
%
% %%%%%%%%%%%%%%%%%%%%%%%%%%%%%%%%%%%%%%%%%%%%%%%%%%%%%%%%%%%%%%%%%%%%%%%%%%%%%%
% %%%%%%%%%%%%%%%%%%%%%%%%%%%%%%%%%%%%%%%%%%%%%%%%%%%%%%%%%%%%%%%%%%%%%%%%%%%%%%
% \section{Implementation}
%\iffalse
%<*package>
%\fi
%
% This section describes the definitions file |childdoc.def|.

% The definitions cannot be loaded using |\usepackage| or |\RequirePackage|
% which has a mechanism to prevent loading a style file more than once.
% When loading the definitions by means of |\input|
% multiple instances have to be prevented manually:
%\iffalse
%This code needs to be before the `\ProvidesFile' directive
%which is defined at the beginning of this file.
%Therefore it is also placed there and commented out here.
%</package>
%<*discard>
%\fi
%    \begin{macrocode}
\ifdefined\childdocmain\endinput\fi
%    \end{macrocode}
%\iffalse
%</discard>
%<*package>
%\fi
%
% \macro{\ifchilddoc}
% \macro{\ifchilddocmanual}
% The conditional |\ifchilddoc| tells whether a
% child (true) or main (false) document is being compiled.
% The conditional |\ifchilddocmanual| tells whether
% the |\includeonly| mechanism is used (false) or
% the selection of child files must be performed manually (true).
% The definitions initialise to false:
%    \begin{macrocode}
\newif\ifchilddoc
\newif\ifchilddocmanual
%    \end{macrocode}

% \macro{\childdocname}
% \macro{\childdocjob}
% The macro |\childdocname| stores the name of the main document
% to be compiled. The macro |\childdocjob| stores the name of
% the document on which the \LaTeX{} compiler was originally invoked.
% The content of |\jobname| cannot be compared
% to filenames specified in the source due to different catcodes.
% The following code rescans |\jobname|, stores the result
% in |\childdocname| and saves a copy in |\childdocjob|:
%    \begin{macrocode}
\edef\childdocname{\scantokens\expandafter{\jobname\noexpand}}
\let\childdocjob\childdocname
%    \end{macrocode}

% \macro{\childdocdisable}
% The macro |\childdocdisable| prevents the main file
% from being processed more than once.
% At this stage, the main document command |\childdocmain|
% is assumed to be called once again where it should do nothing.
% Any subsequent call to it should prevent
% a secondary processing of the main document
% It overwrites the forwarding commands
% |\childdocof| and |\childdocforward|
% with empty macros to prevent further inclusions of the main document:
%    \begin{macrocode}
\newcommand{\childdocdisable}
{
  \renewcommand{\childdocmain}[1]{\renewcommand{\childdocmain}[1]{\endinput}}
  \renewcommand{\childdocof}[1]{}
  \renewcommand{\childdocby}[2][]{}
  \renewcommand{\childdocforward}[2][]{}
  \renewcommand{\childdocdisable}{}
}
%    \end{macrocode}

% \macro{\childdocmain}
% The macro |\childdocmain| is to be called at the top of the main file
% with nothing or the main filename (without extension) as argument.
% First, it breaks loops.
% If the argument is not empty and does not match |\childdocname|
% (which is set by the first inclusion of |childdoc.def|),
% |\ifchilddoc| is set to true, |\includeonly| is applied to the child file
% and |\jobname| is set to the main file
% (for proper handling of |.aux| files):
%    \begin{macrocode}
\newcommand{\childdocmain}[1]
{
  \childdocdisable\childdocmain{}
  \if?#1?\else
    \begingroup
      \def\childdoctmp{#1}
      \ifx\childdoctmp\childdocname
        \def\childdoctmp{}
      \else
        \def\childdoctmp
        {
          \childdoctrue
          \includeonly{\childdocname}
          \def\childdocjob{#1}
          \def\jobname{#1}
        }
      \fi
      \expandafter
    \endgroup
    \childdoctmp
  \fi
}
%    \end{macrocode}

% \macro{\childdocof}
% The command |\childdocof| redirects
% compilation to the main file |#1|.
%    \begin{macrocode}
\newcommand{\childdocof}[1]
{
  \childdocdisable
  \childdoctrue
  \includeonly{\childdocname}
  \def\jobname{#1}
  \def\childdocjob{#1}
  \input{#1}
}
%    \end{macrocode}

% \macro{\childdocby}
% The command |\childdocby| ....
%    \begin{macrocode}
\newcommand{\childdocby}[2][]
{
  \childdocdisable
  \childdoctrue
  \childdocmanualtrue
  \if?#1?\else
    \def\jobname{#2}
  \fi
  \def\childdocjob{#2}
  \input{#2}
  \endinput
}
%    \end{macrocode}

% \macro{\childdocforward}
% The command |\childdocforward| redirects
% compilation to the main file or
% (if the optional argument is given) a child file.
% Parameters are set as if the main file
% or a child file starting with |\childdocof| was compiled.
% Then compilation is handed over to the main file:
%    \begin{macrocode}
\newcommand{\childdocforward}[2][]
{
  \begingroup
    \if?#1?
      \def\childdoctmp
      {
        \def\childdocname{#2}
        \def\childdocjob{#2}
        \def\jobname{#2}
        \input{#2}
        \endinput
      }
    \else
      \def\childdoctmp
      {
        \childdocdisable
        \def\childdocname{#2}
        \childdoctrue
        \includeonly{#2}
        \def\childdocjob{#1}
        \def\jobname{#1}
        \input{#1}
        \endinput
      }
    \fi
    \expandafter
  \endgroup
  \childdoctmp
}
%    \end{macrocode}

% \macro{\childdocforwardprefix}
% The command |\childdocforwardprefix| redirects
% compilation to the main or a child file by means of a pattern.
% The prefix |#1| in the current filename is replaced by |#2|
% and the suffix of the current filename is kept
% (it is assumed that the filename does not contain the substring `|~~~|'
% which is used as a delimiter).
% Compilation is handed over to the new file by |\childdocforward|:
%    \begin{macrocode}
\newcommand{\childdocforwardprefix}[3][]
{
  \begingroup
    \def\childdocextract #2##1~~~{\def\childdoctmp{\childdocforward[#1]{#3##1}}}
    \expandafter\childdocextract\childdocname~~~
    \expandafter
  \endgroup
  \childdoctmp
}
%    \end{macrocode}

% \macro{\childdoc}
% The deprecated macro |\childdoc| is a legacy version of |\childdocmain|:
%    \begin{macrocode}
\newcommand{\childdoc}{\childdocmain}
%    \end{macrocode}

% \macro{\childdocredirect}
% The deprecated macro |\childdocredirect| is a legacy version
% of |\childdocforward| and |\childdocforwardprefix|:
%    \begin{macrocode}
\newcommand{\childdocredirect}[2][]
{
  \begingroup
    \if?#1?
      \def\childdoctmp{\childdocforward{#2}}
    \else
      \def\childdoctmp{\childdocforwardprefix{#1}{#2}}
    \fi
    \expandafter
  \endgroup
  \childdoctmp
}
%    \end{macrocode}

%\iffalse
%</package>
%\fi
%
\endinput
\childdocforward[cdocsamp]{cdocsch1}"|\\
% |latex -jobname cdocscl2 \|\\
% |  "\def\version{final}% \iffalse
%
% childdoc.dtx Copyright (C) 2017-2018 Niklas Beisert
%
% This work may be distributed and/or modified under the
% conditions of the LaTeX Project Public License, either version 1.3
% of this license or (at your option) any later version.
% The latest version of this license is in
%   http://www.latex-project.org/lppl.txt
% and version 1.3 or later is part of all distributions of LaTeX
% version 2005/12/01 or later.
%
% This work has the LPPL maintenance status `maintained'.
%
% The Current Maintainer of this work is Niklas Beisert.
%
% This work consists of the files childdoc.dtx and childdoc.ins
% and the derived files childdoc.def and cdocsamp.tex with
% cdocsch1.tex, cdocsch2.tex, cdocsdrf.tex, cdocsfn1.tex, cdocsfn2.tex.
%
%<package>\ifdefined\childdocmain\endinput\fi
%<package>\ProvidesFile{childdoc.def}[2018/12/30 v2.0 child document driver]
%<samplemain>\ProvidesFile{cdocsamp.tex}[2018/12/30 v2.0 sample for childdoc]
%<*driver>
%\ProvidesFile{childdoc.drv}[2018/12/30 v2.0 childdoc reference manual file]
\PassOptionsToClass{10pt,a4paper}{article}
\documentclass{ltxdoc}

\usepackage[margin=35mm]{geometry}
\usepackage{hyperref}
\usepackage{hyperxmp}
\usepackage[usenames]{color}

\hypersetup{colorlinks=true}
\hypersetup{pdfstartview=FitH}
\hypersetup{pdfpagemode=UseNone}
\hypersetup{pdfsource={}}
\hypersetup{pdflang={en-UK}}
\hypersetup{pdfcopyright={Copyright 2017-2018 Niklas Beisert.
  This work may be distributed and/or modified under the
  conditions of the LaTeX Project Public License, either version 1.3
  of this license or (at your option) any later version.}}
\hypersetup{pdflicenseurl={http://www.latex-project.org/lppl.txt}}
\hypersetup{pdfcontactaddress={ETH Zurich, ITP, HIT K,
  Wolfgang-Pauli-Strasse 27}}
\hypersetup{pdfcontactpostcode={8093}}
\hypersetup{pdfcontactcity={Zurich}}
\hypersetup{pdfcontactcountry={Switzerland}}
\hypersetup{pdfcontactemail={nbeisert@itp.phys.ethz.ch}}
\hypersetup{pdfcontacturl={http://people.phys.ethz.ch/\xmptilde nbeisert/}}

\newcommand{\secref}[1]{\hyperref[#1]{section \ref*{#1}}}

\parskip1ex
\parindent0pt
\let\olditemize\itemize
\def\itemize{\olditemize\parskip0pt}

\begin{document}

\title{The \textsf{childdoc} Package}
\hypersetup{pdftitle={The childdoc Package}}
\author{Niklas Beisert\\[2ex]
  Institut f\"ur Theoretische Physik\\
  Eidgen\"ossische Technische Hochschule Z\"urich\\
  Wolfgang-Pauli-Strasse 27, 8093 Z\"urich, Switzerland\\[1ex]
  \href{mailto:nbeisert@itp.phys.ethz.ch}
  {\texttt{nbeisert@itp.phys.ethz.ch}}}
\hypersetup{pdfauthor={Niklas Beisert}}
\hypersetup{pdfsubject={Manual for the LaTeX2e Package childdoc}}
\date{30 December 2018, \textsf{v2.0}}
\maketitle

\begin{abstract}\noindent
\textsf{childdoc} is a \LaTeXe{} package
that enables the direct compilation
of document sections included by |\include|
to individual files.
\end{abstract}

\begingroup
\parskip0ex
\tableofcontents
\endgroup

%%%%%%%%%%%%%%%%%%%%%%%%%%%%%%%%%%%%%%%%%%%%%%%%%%%%%%%%%%%%%%%%%%%%%%%%%%%%%%%%
%%%%%%%%%%%%%%%%%%%%%%%%%%%%%%%%%%%%%%%%%%%%%%%%%%%%%%%%%%%%%%%%%%%%%%%%%%%%%%%%
\section{Introduction}

\LaTeX{} provides a mechanism to structure a large document (such as a book)
into a main file and several child files (containing the chapters)
using the |\include| command.
This mechanism is beneficial for documents
which span hundreds of pages in order to
make the source file(s) more manageable.
Moreover, compilation can be restricted to
selected child files by means of the |\includeonly| command.
The latter feature can be used to reduce the compilation time while editing
(this was significantly more useful in the earlier days of \LaTeX{})
or to generate a smaller document which is easier to navigate.
Another application of |\includeonly| is to generate
documents consisting of selected parts of the complete document.

However, there are a few drawbacks of the plain |\include| mechanism:
\begin{itemize}
\item
The child files cannot be compiled on their own,
they can only be compiled via the main file.
A naive editing environment
(such as a text editor with an option
to have the current file processed by \LaTeX)
may require one to switch to the main file before compiling;
attempting to compile the child file produces errors.
\item
The main file must be modified (each time)
to adjust the |\includeonly| command
to the present needs. This easily leaves the main file in a messy state.
\item
The generated document will always carry the filename
of the main document. This is inconvenient if
several child files are to be compiled and
to be kept for distribution.
\end{itemize}

The present package provides a simple interface
to make child files individually compilable by \LaTeX{}.
Compiling a child file then has the same effect as compiling
the main file with an |\includeonly| command
to select the appropriate child.
Moreover the generated document will carry the name of the child
rather than the main file.
This resolves all three above issues.

This feature is meant to make the editing of books,
thesis documents and lecture notes somewhat more convenient.
However, the package can also be used efficiently for
composing a series of documents (such as exercise sheets)
which are typically distributed individually.
It then assists the author in generating the individual documents
(potentially in different versions)
as well as a document containing the collected series.
Another application is in developing style files
or other kinds of included material
where compilation of the style file could redirect
to a sample or test file.

%%%%%%%%%%%%%%%%%%%%%%%%%%%%%%%%%%%%%%%%%%%%%%%%%%%%%%%%%%%%%%%%%%%%%%%%%%%%%%%%
%%%%%%%%%%%%%%%%%%%%%%%%%%%%%%%%%%%%%%%%%%%%%%%%%%%%%%%%%%%%%%%%%%%%%%%%%%%%%%%%
\section{Usage}

First of all, the package \textsf{childdoc} is \emph{not} a standard
\LaTeXe{} |.sty| style file! Therefore it needs to be invoked in
a non-standard way.

%%%%%%%%%%%%%%%%%%%%%%%%%%%%%%%%%%%%%%%%%%%%%%%%%%%%%%%%%%%%%%%%%%%%%%%%%%%%%%%%
\subsection{Included Files}
\label{sec:include}

%%%%%%%%%%%%%%%%%%%%%%%%%%%%%%%%%%%%%%%%
\DescribeMacro{\childdocmain}
To use the package, add the commands
\begin{center}
\begin{tabular}{l}
|\input{childdoc.def}|\\
|\childdocmain{}|\\
\end{tabular}
\end{center}
at the very top of the main \LaTeX{} file,
in particular \emph{before} the |\documentclass| statement!
The argument of |\childdocmain| should be left empty
(but it must be present).

%%%%%%%%%%%%%%%%%%%%%%%%%%%%%%%%%%%%%%%%
\DescribeMacro{\childdocof}
Furthermore, add the commands
\begin{center}
\begin{tabular}{l}
|\input{childdoc.def}|\\
|\childdocof{|\textit{main}|}|\\
\end{tabular}
\end{center}
at the top of every child file \textit{child}
which is included by |\include{|\textit{child}|}|
from within the main file
(or at least for those files to be compiled individually).
The argument \textit{main} must be the filename of the main file.

There are a couple of
considerations in setting up the main and child documents:

%%%%%%%%%%%%%%%%%%%%%%%%%%%%%%%%%%%%%%%%
\paragraph{Restrictions.}

Please note the following restrictions:
\begin{itemize}
\item
|\childdocmain| must be called with one argument \textit{main}
to ensure compatibility with earlier version of the package.
It must either be empty (|\childdocmain{}|)
or precisely match the filename of the main file in which it is specified.
See \secref{sec:detection} for further information.
\item
The filename \textit{main} must be specified without the |.tex| extension.
\item
The filename \textit{main} is case sensitive
(even in case-insensitive file systems)
due to internal string comparison.
\item
The argument \textit{main} should be fully expanded, it cannot be a macro.
\item
Subdirectories and special characters should be avoided in filenames.
\item
The command |\childdocmain{|\textit{main}|}| must be followed by a whitespace.
It should not be followed immediately by another command
or by a comment mark `|%|'.
This is because the \TeX{} parser reads the token immediately following
the argument of |\childdocmain| and puts it
at the beginning of every child section;
however, a white\-space is ignored.
\end{itemize}

%%%%%%%%%%%%%%%%%%%%%%%%%%%%%%%%%%%%%%%%
\paragraph{Content of Main File.}

It is advisable to place all content in the child files included by |\include|.
Any output contained in the main file will appear in all child documents
unless suppressed manually;
it cannot be suppressed automatically by the |\includeonly| directive
and thus should normally be avoided.
A method to include some content in the main file
by means of conditional processing is described in \secref{sec:conditional}.

%%%%%%%%%%%%%%%%%%%%%%%%%%%%%%%%%%%%%%%%
\paragraph{Page Numbering.}

When only a part of the document is compiled,
the appropriate numbering of pages
(as well as other status parameters)
is determined from the |.aux| files.
The latter contain information from previous passes.
However this information needs to propagate through
all intermediate child documents.
Therefore the page numbering in child documents may well
be inconsistent until the complete document is compiled at least once.

A useful (if unconventional) way to always ensure a consistent
page numbering is to restart the numbering in each child document
and denote the pages by `\textit{child}|.|\textit{page}'
where \textit{child} represents the chapter/section number of the child file.
This can be achieved by the command
|\numberwithin{page}{|\textit{child}|}|
of the \textsf{amsmath} package
where \textit{child} can be |chapter| or |section|
depending on the chosen structuring.
Alternatively, one can modify the macro |\thepage| appropriately
and reset the counter |page| at the start of each child file.

%%%%%%%%%%%%%%%%%%%%%%%%%%%%%%%%%%%%%%%%%%%%%%%%%%%%%%%%%%%%%%%%%%%%%%%%%%%%%%%%
\subsection{Conditional Processing}
\label{sec:conditional}

The package provides a mechanism to compile different versions
of a document. To customise the versions further some conditional processing
can come in handy to distinguish which version is being compiled.
The package provides two macros to describe the compilation context:

%%%%%%%%%%%%%%%%%%%%%%%%%%%%%%%%%%%%%%%%
\DescribeMacro{\ifchilddoc}
The conditional |\ifchilddoc| distinguishes between the compilation of
child documents and the main document:
%
\begin{center}
|\ifchilddoc |\textit{child-code}| |[|\||else |\textit{main-code}]| \||fi|
\end{center}

%%%%%%%%%%%%%%%%%%%%%%%%%%%%%%%%%%%%%%%%
\DescribeMacro{\childdocname}
\DescribeMacro{\childdocjob}
The macro |\childdocname| contains the filename (without extension)
of the main or child file being processed.
Note that |\childdocjob| will always contain the name of the main file.

%%%%%%%%%%%%%%%%%%%%%%%%%%%%%%%%%%%%%%%%
\paragraph{Title Page.}

Conditional processing can be used to include a title or banner page
in the main document when proper precautions are taken.
Importantly, the code in the main file should ensure that the page counter
(as well as other status parameters which are stored in the |.aux| files)
takes the same value after the conditional processing.
Otherwise the page numbers may take divergent values
depending on which part is compiled.

For example, a title page could be declared by:
%
\begin{center}
\begin{tabular}{l}
|\ifchilddoc\||else|\\
|\addtocounter{page}{-1}|\\
\textit{code for title page}\\
|\newpage|\\
|\||fi|
\end{tabular}
\end{center}
%
A banner page for the child documents can be generated by:
%
\begin{center}
\begin{tabular}{l}
|\ifchilddoc|\\
|\addtocounter{page}{-1}|\\
\textit{code for banner page}\\
|\newpage|\\
|\||fi|
\end{tabular}
\end{center}
%
Here one could write a message such as:
\begin{center}
|This is the part \childdocname{} of \childdocjob{}.|
\end{center}

%%%%%%%%%%%%%%%%%%%%%%%%%%%%%%%%%%%%%%%%%%%%%%%%%%%%%%%%%%%%%%%%%%%%%%%%%%%%%%%%
\subsection{Flags}
\label{sec:flags}

The package makes it easy to generate different versions
of the main or child documents.
To this end compilation flags can be defined
and assigned different default values.
They will be particularly useful in conjunction
with the forwarding mechanism described in \secref{sec:forward}.

For example, it may be useful to have a flag |\version|
which can be set to |draft| or |final|.
The document source will contain some conditional code
depending on the value of |\version|.
Suppose further, the flag should default to |final| for the main file
and to |draft| for child files
which is a natural assignment for editing the document.
This is achieved by placing the following code
in the preamble of the main document
(below the |\childdocmain| directive):
%
\begin{center}
\begin{tabular}{l}
|\ifchilddoc|\\
|\providecommand{\version}{draft}|\\
|\||else|\\
|\providecommand{\version}{final}|\\
|\||fi|
\end{tabular}
\end{center}
%
The definition by |\providecommand| makes sure
that previous definitions are not overwritten.
Further statements |\providecommand{\version}{...}|
can thus be added before the above code to override it.

For the main file, one might add a line
(between |\childdocmain| and the above block)
%
\begin{center}
|%\ifchilddoc\||else\providecommand{\version}{draft}\||fi|
\end{center}
%
which can be uncommented to produce a draft version.
Likewise one can add a line to the very top of a child file
(above the |\childdocof{|\textit{main}|}| directive)
%
\begin{center}
|%\providecommand{\version}{final}|
\end{center}
%
which can be uncommented to produce the final version of this child document.

%%%%%%%%%%%%%%%%%%%%%%%%%%%%%%%%%%%%%%%%%%%%%%%%%%%%%%%%%%%%%%%%%%%%%%%%%%%%%%%%
\subsection{Forwarding}
\label{sec:forward}

Different versions of the main or child documents
using compilation flags as described in \secref{sec:flags}
can be (permanently) stored in different files
for convenient compilation, viewing and distribution.
To this end, the package defines a command
to pass on compilation to a different file:

%%%%%%%%%%%%%%%%%%%%%%%%%%%%%%%%%%%%%%%%
\DescribeMacro{\childdocforward}
The command |\childdocforward| redirects processing to
another source file:
%
\begin{center}
\begin{tabular}{l}
|\input{childdoc.def}|\\
|\childdocforward[|\textit{main}|]{|\textit{dest}|}|\\
\end{tabular}
\end{center}
%
The argument \textit{dest} is the destination file
(without extension).
It should be the main file or one of the child files.
Note that further \textsf{childdoc} directives
such as |\childdocof| and |\childdocforward|
in the indicated file will be processed in this form.
The optional argument \textit{main}
passes on directly to the main file \textit{main}
while pretending to compile the child \textit{dest}.
This form behaves as if \textit{dest}
issues |\childdocof{|\textit{main}|}| right away,
and no further \textsf{childdoc} directives will be processed.

%%%%%%%%%%%%%%%%%%%%%%%%%%%%%%%%%%%%%%%%
\DescribeMacro{\...prefix}
In the alternative form |\childdocforwardprefix|,
%
\begin{center}
\begin{tabular}{l}
|\input{childdoc.def}|\\
|\childdocforwardprefix[|\textit{main}|]{|\textit{prefix}|}{|\textit{dest}|}|
\end{tabular}
\end{center}
%
the destination file is determined by a pattern
depending on the current file:
To make this work, the current file must be called
`{\textit{prefix}\hspace{0.2em}\textit{suffix}}'
with \textit{prefix} matching precisely the argument.
Processing is then passed on to the file
`{\textit{dest}\hspace{0.2em}\textit{suffix}}'.
Surely, the same effect is achieved by
directly specifying the
argument `{\textit{dest}\hspace{0.2em}\textit{suffix}}'
in the first form.
However, that requires to set up a different file
for each child. With the alternative form of the command
all these files can have exactly the same content
which simplifies setting them up and maintaining them.

For example, the following file |draft.tex|
with a compilation flag |\version| as described in \secref{sec:flags}
compiles the main document as a draft:
%
\begin{center}
\begin{tabular}{l}
|\def\version{draft}|\\
|\input{childdoc.def}|\\
|\childdocforward{|\textit{main}|}|
\end{tabular}
\end{center}
%
Likewise, the following files |final|\textit{nn}|.tex|
compile the final version of the child document
|child|\textit{nn}|.tex|:
%
\begin{center}
\begin{tabular}{l}
|\def\version{final}|\\
|\input{childdoc.def}|\\
|\childdocforwardprefix{final}{child}|
\end{tabular}
\end{center}
%

Note that when several versions of a main file and/or of each child file
are to be generated, it may be convenient to set up a |Makefile| or
shell script to automatise the process.

%%%%%%%%%%%%%%%%%%%%%%%%%%%%%%%%%%%%%%%%%%%%%%%%%%%%%%%%%%%%%%%%%%%%%%%%%%%%%%%%
\subsection{Command Line Processing}
\label{sec:commandline}

The effect of redirection files can also be achieved by invoking
the \LaTeX{} compiler with a more elaborate command line.
Most conveniently this should be done as part
of a shell script or a |Makefile|.

When using \textsf{childdoc} in the main file, the following
command lines effectively perform a redirection
(note that depending on the shell being used,
backslashes may have to be doubled: `|\|' $\to$ `|\\|'):
%
\begin{center}
|... -jobname "|\textit{target}|" |\\|"|[\textit{flags}]%
|\input{childdoc.def}\childdocforward[|\textit{main}|]{|\textit{dest}|}"|
\end{center}
%
Here \textit{target} is the name of the output file,
\textit{main} is the name of the main file
and \textit{dest} is the name of the main or child file to be processed
(all filenames without extensions).
The optional argument \textit{main} can be omitted
if \textit{main} matches \textit{dest}.
Optionally, compilation \textit{flags} can be defined via |\def| commands.
This command line makes the \TeX{} engine believe
it is compiling the file \textit{target}
whose content is specified as the latter parameter.
The provided code then forwards the processing to
\textit{main} or \textit{dest} as described in \secref{sec:forward}.

%%%%%%%%%%%%%%%%%%%%%%%%%%%%%%%%%%%%%%%%%%%%%%%%%%%%%%%%%%%%%%%%%%%%%%%%%%%%%%%%
\subsection{Include by Input}
\label{sec:input}

Including child documents by |\include| has some restrictions by design.
Most notably, the content of a child document always occupies
its own set of pages; pages cannot be shared between child documents.
Usually, this behaviour makes perfect sense
because each child document contain an essential part of the document.
However, in some situations it may be desirable to compose
a document from a collection of parts
without having mandatory page breaks between then.
For this case, the package
provides a mechanism to include parts
by |\input| which can also be processed individually.
However, by construction this mechanism
requires manual handling of the content to be output.

%%%%%%%%%%%%%%%%%%%%%%%%%%%%%%%%%%%%%%%%
\DescribeMacro{\ifchilddocmanual}
The main file should be prepared as usual, see \secref{sec:include}.
However, the document body must make a distinction
between processing of an individual part and of the main document, e.g.:
%
\begin{center}
\begin{tabular}{l}
|\ifchilddocmanual|\\
|\input{\childdocname}|\\
|\||else|\\
\textit{document body with }|\input{|\textit{part}|}|\\
|\||fi|
\end{tabular}
\end{center}
%
The conditional |\ifchilddocmanual| is true whenever
a part to be included by |\input| is being compiled,
and the name of the part is stored in |\childdocname|.

%%%%%%%%%%%%%%%%%%%%%%%%%%%%%%%%%%%%%%%%
\DescribeMacro{\childdocby}
Each part to be included by |\input| should start with:
%
\begin{center}
\begin{tabular}{l}
|\input{childdoc.def}|\\
|\childdocby{|\textit{main}|}|\\
\end{tabular}
\end{center}
%
The directive |\childdocby| is similar to |\childdocof|
described in \secref{sec:include},
but the subsequent selection of content must be done manually.
To that end, both |\ifchilddoc| and |\ifchilddocmanual|
will be true upon processing of a part,
and the name of the part is stored in |\childdocname|.
Note that |\jobname| will be set to the filename of the current part
so that each part receives an individual |.aux| file
that does not interfere with the |.aux| file(s) of the main document.
This behaviour can be altered by the alternative form
|\childdocby[*]{|\textit{main}|}| (with a non-empty optional argument)
which uses the |.aux| file of the main document
by setting |\jobname| to \textit{main}.

%%%%%%%%%%%%%%%%%%%%%%%%%%%%%%%%%%%%%%%%%%%%%%%%%%%%%%%%%%%%%%%%%%%%%%%%%%%%%%%%
\subsection{Driver Development}
\label{sec:driver}

The \textsf{childdoc} mechanism can also be use for the development
of definition files such as \LaTeX{} styles or classes.
This case differs from the above setup with multiple parts
included by |\include| in that no |\includeonly| should be invoked.
This can be achieved by starting the include file
(before |\ProvidesPackage|) with:
%
\begin{center}
\begin{tabular}{l}
|\input{childdoc.def}|\\
|\childdocforward{|\textit{main}|}|\\
\end{tabular}
\end{center}
%
or alternatively with:
%
\begin{center}
\begin{tabular}{l}
|\input{childdoc.def}|\\
|\childdocby{|\textit{main}|}|\\
\end{tabular}
\end{center}
%
Both forms have slightly different effects as described above.
The main file is prepared as usual, see \secref{sec:include}.

%%%%%%%%%%%%%%%%%%%%%%%%%%%%%%%%%%%%%%%%%%%%%%%%%%%%%%%%%%%%%%%%%%%%%%%%%%%%%%%%
\subsection{Legacy Detection}
\label{sec:detection}

The directive |\childdocmain| in the main file can detect
whether the complete document or merely a child is to be compiled
even without using the directive |\childdocof|.
This method is deprecated because it is less robust
and there is no compelling reason to use it;
it is merely provided for backward compatibility
and it may be removed in future versions.

If the detection mechanism is to be used,
it is mandatory to correctly specify
the filename of the main file as the argument of |\childdocmain|:
%
\begin{center}
\begin{tabular}{l}
|\input{childdoc.def}|\\
|\childdocmain{|\textit{main}|}|\\
\end{tabular}
\end{center}
%
If |\jobname| does not match the argument \textit{main} of |\childdocmain|,
it is assumed that |\jobname| points to the child file to be compiled.
When using |\childdocmain| with the main file specified as argument,
it suffices to start a child file
with just |\input{|\textit{main}|}|
without loading of the package and using |\childdocof|.
If instead all processing is done
with the appropriate \textsf{childdoc} directives,
the argument of \textit{main} of |\childdocmain| can be empty.

An alternative version of the command line processing described
in \secref{sec:commandline} using the detection mechanism reads:
%
\begin{center}
|... -jobname "|\textit{target}|" "|[\textit{flags}]%
[|\def\jobname{|\textit{dest}|}|]|\input{|\textit{main}|}"|
\end{center}

%%%%%%%%%%%%%%%%%%%%%%%%%%%%%%%%%%%%%%%%%%%%%%%%%%%%%%%%%%%%%%%%%%%%%%%%%%%%%%%%
\subsection{Manual Code}
\label{sec:manual}

In case one cannot be certain whether the definitions file |childdoc.def|
is installed on the target \TeX{} distribution
and one prefers not to ship it,
it is conceivable to paste a few relevant commands into the sources.

To that end, drop all statements |\input{childdoc.def}|
and perform the replacements as outlined below.
Instead of |\childdocmain{|\textit{main}|}| add the following code
to the top of the main file:
%
\begin{center}
\begin{tabular}{l}
|\||ifdefined\childdocname\endinput\||fi\newif\ifchilddoc|\\
|\edef\childdocname{\scantokens\expandafter{\jobname\noexpand}}|\\
|\def\childdocmain{|\textit{main}|}\||ifx\childdocmain\childdocname\||else|\\
|\childdoctrue\includeonly{\childdocname}\let\jobname\childdocmain\||fi|\\
\end{tabular}
\end{center}
%
Instead of |\childdocof{|\textit{main}|}| just include the main file
at the top of each child file:
%
\begin{center}
|\input{|\textit{main}|}|
\end{center}
%
A simple redirection |\childdocforward{|\textit{dest}|}| is achieved by:
%
\begin{center}
|\def\jobname{|\textit{dest}|}\input{\jobname}|
\end{center}
%
The redirection with prefix
|\childdocforwardprefix[|\textit{prefix}|]{|\textit{dest}|}|
is accomplished by:
%
\begin{center}
\begin{tabular}{l}
|{\edef\jobname{\scantokens\expandafter{\jobname\noexpand}}|\\
|\def\redirectjob |\textit{prefix}|#1~~~{\gdef\jobname{|\textit{dest}|#1}}|\\
|\expandafter\redirectjob\jobname~~~}\input{\jobname}|
\end{tabular}
\end{center}

In an alternative approach,
child documents can be compiled by a specific command line
without additional code or specific definitions:
%
\begin{center}
|... -jobname "|\textit{target}|" "|[\textit{flags}]%
|\includeonly{|\textit{dest}|}\input{|\textit{main}|}"|
\end{center}
%

%%%%%%%%%%%%%%%%%%%%%%%%%%%%%%%%%%%%%%%%%%%%%%%%%%%%%%%%%%%%%%%%%%%%%%%%%%%%%%%%
%%%%%%%%%%%%%%%%%%%%%%%%%%%%%%%%%%%%%%%%%%%%%%%%%%%%%%%%%%%%%%%%%%%%%%%%%%%%%%%%
\section{Information}

%%%%%%%%%%%%%%%%%%%%%%%%%%%%%%%%%%%%%%%%%%%%%%%%%%%%%%%%%%%%%%%%%%%%%%%%%%%%%%%%
\subsection{Copyright}

Copyright \copyright{} 2017--2018 Niklas Beisert

This work may be distributed and/or modified under the
conditions of the \LaTeX{} Project Public License, either version 1.3
of this license or (at your option) any later version.
The latest version of this license is in
  \url{http://www.latex-project.org/lppl.txt}
and version 1.3 or later is part of all distributions of \LaTeX{}
version 2005/12/01 or later.

This work has the LPPL maintenance status `maintained'.

The Current Maintainer of this work is Niklas Beisert.

This work consists of the files |README.txt|, |childdoc.ins| and |childdoc.dtx|
as well as the derived files |childdoc.def|, |cdocsamp.tex|
with |cdocsch1.tex|, |cdocsch2.tex|, |cdocspt3.tex|, |cdocspt4.tex|,
|cdocsdrf.tex|, |cdocsfn1.tex|, |cdocsfn2.tex|
as well as |childdoc.pdf|.

%%%%%%%%%%%%%%%%%%%%%%%%%%%%%%%%%%%%%%%%%%%%%%%%%%%%%%%%%%%%%%%%%%%%%%%%%%%%%%%%
\subsection{Files and Installation}

The package consists of the files:
%
\begin{center}
\begin{tabular}{ll}
    |README.txt|   & readme file \\
    |childdoc.ins| & installation file \\
    |childdoc.dtx| & source file \\
    |childdoc.def| & definition file \\
    |cdocsamp.tex| & sample main file \\
    |cdocsch1.tex| & sample include file \\
    |cdocsch2.tex| & sample include file \\
    |cdocspt3.tex| & sample part file \\
    |cdocspt4.tex| & sample part file \\
    |cdocsdrf.tex| & sample redirection file \\
    |cdocsfn1.tex| & sample redirection file \\
    |cdocsfn2.tex| & sample redirection file \\
    |childdoc.pdf| & manual
\end{tabular}
\end{center}
%
The distribution consists of the files
|README.txt|, |childdoc.ins| and |childdoc.dtx|.
%
\begin{itemize}
\item
Run (pdf)\LaTeX{} on |childdoc.dtx|
to compile the manual |childdoc.pdf| (this file).
\item
Run \LaTeX{} on |childdoc.ins| to create the definitions file |childdoc.def|
and the sample |cdocsamp.tex| with include files
|cdocsch1.tex|, |cdocsch2.tex|, |cdocspt3.tex|, |cdocspt4.tex|,
|cdocsdrf.tex|, |cdocsfn1.tex|, |cdocsfn2.tex|.
Then copy the file |childdoc.def| to an appropriate directory of your \LaTeX{}
distribution, e.g.\ \textit{texmf-root}|/tex/latex/childdoc|.
\end{itemize}

%%%%%%%%%%%%%%%%%%%%%%%%%%%%%%%%%%%%%%%%%%%%%%%%%%%%%%%%%%%%%%%%%%%%%%%%%%%%%%%%
\subsection{Related CTAN Packages}

There are several other packages which offer a similar functionality:
%
\begin{itemize}
\item
The packages
\href{http://ctan.org/pkg/docmute}{\textsf{docmute}},
\href{http://ctan.org/pkg/includex}{\textsf{includex}} and
\href{http://ctan.org/pkg/standalone}{\textsf{standalone}}
provide commands to include only the document body of
a child file thus allowing both files to be compiled individually.
\item
The packages \href{http://ctan.org/pkg/subdocs}{\textsf{subdocs}}
and \href{http://ctan.org/pkg/subfiles}{\textsf{subfiles}}
provide structures in which the main and child documents can be
encapsulated and allowing them to be compiled individually.
The inclusion mechanism is different from the conventional |\include|.
\item
The package \href{http://ctan.org/pkg/combine}{\textsf{combine}}
is an elaborate solution to combine several documents into one.
\end{itemize}
%
See also the CTAN topic \href{http://ctan.org/topic/subdocs}{\textsf{subdocs}}
for further related packages.
The present package differs from the above solutions in that
a document structure constructed with the conventional |\include| mechanism
just needs two extra commands at the top of every file
such that all constituent files can be compiled individually.

%%%%%%%%%%%%%%%%%%%%%%%%%%%%%%%%%%%%%%%%%%%%%%%%%%%%%%%%%%%%%%%%%%%%%%%%%%%%%%%%
%\subsection{Feature Suggestions}
%
%The following is a list of features which may be useful for future
%versions of this package:
%%
%\begin{itemize}
%\item
%\ldots
%\end{itemize}

%%%%%%%%%%%%%%%%%%%%%%%%%%%%%%%%%%%%%%%%%%%%%%%%%%%%%%%%%%%%%%%%%%%%%%%%%%%%%%%%
\subsection{Revision History}

%%%%%%%%%%%%%%%%%%%%%%%%%%%%%%%%%%%%%%%%
\paragraph{v2.0:} 2018/12/30

\begin{itemize}
\item
immediate forward processing
\item
added |\childdocby| mechanism
\item
manual restructured
\end{itemize}

%%%%%%%%%%%%%%%%%%%%%%%%%%%%%%%%%%%%%%%%
\paragraph{v1.6:} 2018/01/17

\begin{itemize}
\item
application for development of include files
\item
corrections to manual
\end{itemize}

%%%%%%%%%%%%%%%%%%%%%%%%%%%%%%%%%%%%%%%%
\paragraph{v1.5:} 2017/05/21

\begin{itemize}
\item
more complete structuring introduced
\item
|\childdocof| introduced
\item
|\childdoc| renamed to |\childdocmain|
\item
|\childredirect| renamed to |\childdocforward| and |\childdocforwardprefix|
and functionality expanded
\end{itemize}

%%%%%%%%%%%%%%%%%%%%%%%%%%%%%%%%%%%%%%%%
\paragraph{v1.0:} 2017/04/27

\begin{itemize}
\item
manual and install package
\item
first version published on CTAN
\end{itemize}

%%%%%%%%%%%%%%%%%%%%%%%%%%%%%%%%%%%%%%%%
\paragraph{v0.6:} 2017/04/26

\begin{itemize}
\item
redirection mechanism added
\end{itemize}

%%%%%%%%%%%%%%%%%%%%%%%%%%%%%%%%%%%%%%%%
\paragraph{v0.5:} 2017/04/26

\begin{itemize}
\item
functionality in definition file
\end{itemize}


%%%%%%%%%%%%%%%%%%%%%%%%%%%%%%%%%%%%%%%%%%%%%%%%%%%%%%%%%%%%%%%%%%%%%%%%%%%%%%%%
%%%%%%%%%%%%%%%%%%%%%%%%%%%%%%%%%%%%%%%%%%%%%%%%%%%%%%%%%%%%%%%%%%%%%%%%%%%%%%%%
%%%%%%%%%%%%%%%%%%%%%%%%%%%%%%%%%%%%%%%%%%%%%%%%%%%%%%%%%%%%%%%%%%%%%%%%%%%%%%%%
\appendix

\settowidth\MacroIndent{\rmfamily\scriptsize 000\ }

 \DocInput{childdoc.dtx}

\end{document}
%</driver>
% \fi
%
% %%%%%%%%%%%%%%%%%%%%%%%%%%%%%%%%%%%%%%%%%%%%%%%%%%%%%%%%%%%%%%%%%%%%%%%%%%%%%%
% %%%%%%%%%%%%%%%%%%%%%%%%%%%%%%%%%%%%%%%%%%%%%%%%%%%%%%%%%%%%%%%%%%%%%%%%%%%%%%
% \section{Sample}
%\iffalse
%<*samplemain>
%\fi
%
% The following presents a sample document
% with two chapters, two parts, a title page,
% a compile flag as well as three forwarding files to set the flag.
% It consists of eight |.tex| files:
% \begin{center}
% \begin{tabular}{ll}
% |cdocsamp.tex|&main file\\
% |cdocsch1.tex|&include file for chapter 1\\
% |cdocsch2.tex|&include file for chapter 2\\
% |cdocspt3.tex|&include file for part 3\\
% |cdocspt4.tex|&include file for part 4\\
% |cdocsdrf.tex|&forwarding file for main file in draft mode\\
% |cdocsfi1.tex|&forwarding file for final version of chapter 1\\
% |cdocsfi2.tex|&forwarding file for final version of chapter 2\\
% \end{tabular}
% \end{center}
% Each of the eight files can be compiled directly by the \LaTeX{} compiler.
%
% %%%%%%%%%%%%%%%%%%%%%%%%%%%%%%%%%%%%%%
% \paragraph{Main File.}
%
% The main file is called |cdocsamp.tex|.
%
% Load the \textsf{childdoc} definitions and
% declare the filename for the main document:
%    \begin{macrocode}
\input{childdoc.def}
\childdocmain{}
%    \end{macrocode}

% Optional override for |\version| flag:
%    \begin{macrocode}
%%\ifchilddoc\else\providecommand{\version}{draft}\fi
%    \end{macrocode}

% Define the default values for the |\version| flag
% (|final| for the main file and |draft| for childs):
%    \begin{macrocode}
\ifchilddoc
\providecommand{\version}{draft}
\else
\providecommand{\version}{final}
\fi
%    \end{macrocode}

% Load the standard document class:
%    \begin{macrocode}
\documentclass[12pt]{article}
%    \end{macrocode}

% Start the document body:
%    \begin{macrocode}
\begin{document}
%    \end{macrocode}

% Declare a title page.
% Print title, part of document being processed and version flag:
%    \begin{macrocode}
\addtocounter{page}{-1}
\begin{center}
{\LARGE\bfseries{}childdoc example\par}
\vspace{1cm}
\ifchilddoc
\ifchilddocmanual part\else chapter\fi:
`\childdocname' of `\childdocjob'\par
\else
main document: `\childdocjob'\par
\fi
version: \version\par
\end{center}
\newpage
%    \end{macrocode}

% Manually include selected file,
% otherwise process as usual:
%    \begin{macrocode}
\ifchilddocmanual
\section*{part `\childdocname'}
\input{\childdocname}
\else
%    \end{macrocode}

% Include the two chapters:
%    \begin{macrocode}
\include{cdocsch1}
\include{cdocsch2}
%    \end{macrocode}

% Include the two parts unless only chapters should be displayed:
%    \begin{macrocode}
\ifchilddoc\else
\section{part three}
\input{cdocspt3}
\section{part four}
\input{cdocspt4}
\fi
%    \end{macrocode}

% Process as usual until here:
%    \begin{macrocode}
\fi
%    \end{macrocode}

% End of document body:
%    \begin{macrocode}
\end{document}
%    \end{macrocode}
%\iffalse
%</samplemain>
%\fi
%
% %%%%%%%%%%%%%%%%%%%%%%%%%%%%%%%%%%%%%%
% \paragraph{Chapter Include Files.}
%
% The include files are called |cdocsch1.tex| and |cdocsch2.tex|.
%
%\iffalse
%<*samplechap1|samplechap2>
%\fi

% Optional override for |\version| flag:
%    \begin{macrocode}
%%\providecommand{\version}{final}
%    \end{macrocode}

% Include the main document:
%    \begin{macrocode}
\input{childdoc.def}
\childdocof{cdocsamp}
%    \end{macrocode}

%\iffalse
%</samplechap1|samplechap2>
%\fi
%
%\iffalse
%<*samplechap1>
%\fi
% Some text for chapter 1:
%    \begin{macrocode}
\section{one}
some text in chapter one
%    \end{macrocode}

%\iffalse
%</samplechap1>
%\fi
% Some text for chapter 2:
%\iffalse
%<*samplechap2>
%\fi
%    \begin{macrocode}
\section{two}
more text in chapter two
%    \end{macrocode}

%\iffalse
%</samplechap2>
%\fi
%
% %%%%%%%%%%%%%%%%%%%%%%%%%%%%%%%%%%%%%%
% \paragraph{Part Include Files.}
%
% The include files are called |cdocspt3.tex| and |cdocspt4.tex|.
%
%\iffalse
%<*samplepart3|samplepart4>
%\fi

% Optional override for |\version| flag:
%    \begin{macrocode}
%%\providecommand{\version}{final}
%    \end{macrocode}

% Include the main document:
%    \begin{macrocode}
\input{childdoc.def}
\childdocby{cdocsamp}
%    \end{macrocode}

%\iffalse
%</samplepart3|samplepart4>
%\fi
%
%\iffalse
%<*samplepart3>
%\fi
% Some text for part 3:
%    \begin{macrocode}
some text in part three
%    \end{macrocode}

%\iffalse
%</samplepart3>
%\fi
% Some text for part 4:
%\iffalse
%<*samplepart4>
%\fi
%    \begin{macrocode}
more text in part four
%    \end{macrocode}

%\iffalse
%</samplepart4>
%\fi
%
% %%%%%%%%%%%%%%%%%%%%%%%%%%%%%%%%%%%%%%
% \paragraph{Forwarding for a Complete Draft.}
%
% The following forwarding file |cdocsdrf.tex|
% compiles the main document in draft mode:
%\iffalse
%<*sampledraft>
%\fi
%    \begin{macrocode}
\def\version{draft}
\input{childdoc.def}
\childdocforward{cdocsamp}
%    \end{macrocode}

%\iffalse
%</sampledraft>
%\fi
%
% %%%%%%%%%%%%%%%%%%%%%%%%%%%%%%%%%%%%%%
% \paragraph{Forwarding for Final Version of the Chapters.}
%
% The following forwarding files |cdocsfn1.tex| and |cdocsfn2.tex|
% (with identical content)
% compile the final versions of the child documents
% |cdocsch1.tex| and |cdocsch2.tex|, respectively:
%\iffalse
%<*samplefinal>
%\fi
%    \begin{macrocode}
\def\version{final}
\input{childdoc.def}
\childdocforwardprefix[cdocsamp]{cdocsfn}{cdocsch}
%    \end{macrocode}

%\iffalse
%</samplefinal>
%\fi
%
% %%%%%%%%%%%%%%%%%%%%%%%%%%%%%%%%%%%%%%
% \paragraph{Command Line Processing.}
%
% The following three command lines generate the output files
% |cdocscld|, |cdocscl1| and |cdocscl2|
% which should be identical to
% |cdocsdrf|, |cdocsch1| and |cdocsfn2|, respectively:
% \begin{center}
% \begin{tabular}{l}
% |latex -jobname cdocscld \|\\
% |  "\def\version{draft}\input{childdoc.def}\childdocforward{cdocsamp}"|\\
% |latex -jobname cdocscl1 \|\\
% |  "\input{childdoc.def}\childdocforward[cdocsamp]{cdocsch1}"|\\
% |latex -jobname cdocscl2 \|\\
% |  "\def\version{final}\input{childdoc.def}\childdocforward{cdocsch2}"|
% \end{tabular}
% \end{center}
% Note that the trailing backslash on each first line
% merely continues the input to the second line
% (for convenient cut ant paste).
% Furthermore, the command |latex| can be replaced by any
% of its alternative versions such as |pdflatex|.
%
% %%%%%%%%%%%%%%%%%%%%%%%%%%%%%%%%%%%%%%%%%%%%%%%%%%%%%%%%%%%%%%%%%%%%%%%%%%%%%%
% %%%%%%%%%%%%%%%%%%%%%%%%%%%%%%%%%%%%%%%%%%%%%%%%%%%%%%%%%%%%%%%%%%%%%%%%%%%%%%
% \section{Implementation}
%\iffalse
%<*package>
%\fi
%
% This section describes the definitions file |childdoc.def|.

% The definitions cannot be loaded using |\usepackage| or |\RequirePackage|
% which has a mechanism to prevent loading a style file more than once.
% When loading the definitions by means of |\input|
% multiple instances have to be prevented manually:
%\iffalse
%This code needs to be before the `\ProvidesFile' directive
%which is defined at the beginning of this file.
%Therefore it is also placed there and commented out here.
%</package>
%<*discard>
%\fi
%    \begin{macrocode}
\ifdefined\childdocmain\endinput\fi
%    \end{macrocode}
%\iffalse
%</discard>
%<*package>
%\fi
%
% \macro{\ifchilddoc}
% \macro{\ifchilddocmanual}
% The conditional |\ifchilddoc| tells whether a
% child (true) or main (false) document is being compiled.
% The conditional |\ifchilddocmanual| tells whether
% the |\includeonly| mechanism is used (false) or
% the selection of child files must be performed manually (true).
% The definitions initialise to false:
%    \begin{macrocode}
\newif\ifchilddoc
\newif\ifchilddocmanual
%    \end{macrocode}

% \macro{\childdocname}
% \macro{\childdocjob}
% The macro |\childdocname| stores the name of the main document
% to be compiled. The macro |\childdocjob| stores the name of
% the document on which the \LaTeX{} compiler was originally invoked.
% The content of |\jobname| cannot be compared
% to filenames specified in the source due to different catcodes.
% The following code rescans |\jobname|, stores the result
% in |\childdocname| and saves a copy in |\childdocjob|:
%    \begin{macrocode}
\edef\childdocname{\scantokens\expandafter{\jobname\noexpand}}
\let\childdocjob\childdocname
%    \end{macrocode}

% \macro{\childdocdisable}
% The macro |\childdocdisable| prevents the main file
% from being processed more than once.
% At this stage, the main document command |\childdocmain|
% is assumed to be called once again where it should do nothing.
% Any subsequent call to it should prevent
% a secondary processing of the main document
% It overwrites the forwarding commands
% |\childdocof| and |\childdocforward|
% with empty macros to prevent further inclusions of the main document:
%    \begin{macrocode}
\newcommand{\childdocdisable}
{
  \renewcommand{\childdocmain}[1]{\renewcommand{\childdocmain}[1]{\endinput}}
  \renewcommand{\childdocof}[1]{}
  \renewcommand{\childdocby}[2][]{}
  \renewcommand{\childdocforward}[2][]{}
  \renewcommand{\childdocdisable}{}
}
%    \end{macrocode}

% \macro{\childdocmain}
% The macro |\childdocmain| is to be called at the top of the main file
% with nothing or the main filename (without extension) as argument.
% First, it breaks loops.
% If the argument is not empty and does not match |\childdocname|
% (which is set by the first inclusion of |childdoc.def|),
% |\ifchilddoc| is set to true, |\includeonly| is applied to the child file
% and |\jobname| is set to the main file
% (for proper handling of |.aux| files):
%    \begin{macrocode}
\newcommand{\childdocmain}[1]
{
  \childdocdisable\childdocmain{}
  \if?#1?\else
    \begingroup
      \def\childdoctmp{#1}
      \ifx\childdoctmp\childdocname
        \def\childdoctmp{}
      \else
        \def\childdoctmp
        {
          \childdoctrue
          \includeonly{\childdocname}
          \def\childdocjob{#1}
          \def\jobname{#1}
        }
      \fi
      \expandafter
    \endgroup
    \childdoctmp
  \fi
}
%    \end{macrocode}

% \macro{\childdocof}
% The command |\childdocof| redirects
% compilation to the main file |#1|.
%    \begin{macrocode}
\newcommand{\childdocof}[1]
{
  \childdocdisable
  \childdoctrue
  \includeonly{\childdocname}
  \def\jobname{#1}
  \def\childdocjob{#1}
  \input{#1}
}
%    \end{macrocode}

% \macro{\childdocby}
% The command |\childdocby| ....
%    \begin{macrocode}
\newcommand{\childdocby}[2][]
{
  \childdocdisable
  \childdoctrue
  \childdocmanualtrue
  \if?#1?\else
    \def\jobname{#2}
  \fi
  \def\childdocjob{#2}
  \input{#2}
  \endinput
}
%    \end{macrocode}

% \macro{\childdocforward}
% The command |\childdocforward| redirects
% compilation to the main file or
% (if the optional argument is given) a child file.
% Parameters are set as if the main file
% or a child file starting with |\childdocof| was compiled.
% Then compilation is handed over to the main file:
%    \begin{macrocode}
\newcommand{\childdocforward}[2][]
{
  \begingroup
    \if?#1?
      \def\childdoctmp
      {
        \def\childdocname{#2}
        \def\childdocjob{#2}
        \def\jobname{#2}
        \input{#2}
        \endinput
      }
    \else
      \def\childdoctmp
      {
        \childdocdisable
        \def\childdocname{#2}
        \childdoctrue
        \includeonly{#2}
        \def\childdocjob{#1}
        \def\jobname{#1}
        \input{#1}
        \endinput
      }
    \fi
    \expandafter
  \endgroup
  \childdoctmp
}
%    \end{macrocode}

% \macro{\childdocforwardprefix}
% The command |\childdocforwardprefix| redirects
% compilation to the main or a child file by means of a pattern.
% The prefix |#1| in the current filename is replaced by |#2|
% and the suffix of the current filename is kept
% (it is assumed that the filename does not contain the substring `|~~~|'
% which is used as a delimiter).
% Compilation is handed over to the new file by |\childdocforward|:
%    \begin{macrocode}
\newcommand{\childdocforwardprefix}[3][]
{
  \begingroup
    \def\childdocextract #2##1~~~{\def\childdoctmp{\childdocforward[#1]{#3##1}}}
    \expandafter\childdocextract\childdocname~~~
    \expandafter
  \endgroup
  \childdoctmp
}
%    \end{macrocode}

% \macro{\childdoc}
% The deprecated macro |\childdoc| is a legacy version of |\childdocmain|:
%    \begin{macrocode}
\newcommand{\childdoc}{\childdocmain}
%    \end{macrocode}

% \macro{\childdocredirect}
% The deprecated macro |\childdocredirect| is a legacy version
% of |\childdocforward| and |\childdocforwardprefix|:
%    \begin{macrocode}
\newcommand{\childdocredirect}[2][]
{
  \begingroup
    \if?#1?
      \def\childdoctmp{\childdocforward{#2}}
    \else
      \def\childdoctmp{\childdocforwardprefix{#1}{#2}}
    \fi
    \expandafter
  \endgroup
  \childdoctmp
}
%    \end{macrocode}

%\iffalse
%</package>
%\fi
%
\endinput
\childdocforward{cdocsch2}"|
% \end{tabular}
% \end{center}
% Note that the trailing backslash on each first line
% merely continues the input to the second line
% (for convenient cut ant paste).
% Furthermore, the command |latex| can be replaced by any
% of its alternative versions such as |pdflatex|.
%
% %%%%%%%%%%%%%%%%%%%%%%%%%%%%%%%%%%%%%%%%%%%%%%%%%%%%%%%%%%%%%%%%%%%%%%%%%%%%%%
% %%%%%%%%%%%%%%%%%%%%%%%%%%%%%%%%%%%%%%%%%%%%%%%%%%%%%%%%%%%%%%%%%%%%%%%%%%%%%%
% \section{Implementation}
%\iffalse
%<*package>
%\fi
%
% This section describes the definitions file |childdoc.def|.

% The definitions cannot be loaded using |\usepackage| or |\RequirePackage|
% which has a mechanism to prevent loading a style file more than once.
% When loading the definitions by means of |\input|
% multiple instances have to be prevented manually:
%\iffalse
%This code needs to be before the `\ProvidesFile' directive
%which is defined at the beginning of this file.
%Therefore it is also placed there and commented out here.
%</package>
%<*discard>
%\fi
%    \begin{macrocode}
\ifdefined\childdocmain\endinput\fi
%    \end{macrocode}
%\iffalse
%</discard>
%<*package>
%\fi
%
% \macro{\ifchilddoc}
% \macro{\ifchilddocmanual}
% The conditional |\ifchilddoc| tells whether a
% child (true) or main (false) document is being compiled.
% The conditional |\ifchilddocmanual| tells whether
% the |\includeonly| mechanism is used (false) or
% the selection of child files must be performed manually (true).
% The definitions initialise to false:
%    \begin{macrocode}
\newif\ifchilddoc
\newif\ifchilddocmanual
%    \end{macrocode}

% \macro{\childdocname}
% \macro{\childdocjob}
% The macro |\childdocname| stores the name of the main document
% to be compiled. The macro |\childdocjob| stores the name of
% the document on which the \LaTeX{} compiler was originally invoked.
% The content of |\jobname| cannot be compared
% to filenames specified in the source due to different catcodes.
% The following code rescans |\jobname|, stores the result
% in |\childdocname| and saves a copy in |\childdocjob|:
%    \begin{macrocode}
\edef\childdocname{\scantokens\expandafter{\jobname\noexpand}}
\let\childdocjob\childdocname
%    \end{macrocode}

% \macro{\childdocdisable}
% The macro |\childdocdisable| prevents the main file
% from being processed more than once.
% At this stage, the main document command |\childdocmain|
% is assumed to be called once again where it should do nothing.
% Any subsequent call to it should prevent
% a secondary processing of the main document
% It overwrites the forwarding commands
% |\childdocof| and |\childdocforward|
% with empty macros to prevent further inclusions of the main document:
%    \begin{macrocode}
\newcommand{\childdocdisable}
{
  \renewcommand{\childdocmain}[1]{\renewcommand{\childdocmain}[1]{\endinput}}
  \renewcommand{\childdocof}[1]{}
  \renewcommand{\childdocby}[2][]{}
  \renewcommand{\childdocforward}[2][]{}
  \renewcommand{\childdocdisable}{}
}
%    \end{macrocode}

% \macro{\childdocmain}
% The macro |\childdocmain| is to be called at the top of the main file
% with nothing or the main filename (without extension) as argument.
% First, it breaks loops.
% If the argument is not empty and does not match |\childdocname|
% (which is set by the first inclusion of |childdoc.def|),
% |\ifchilddoc| is set to true, |\includeonly| is applied to the child file
% and |\jobname| is set to the main file
% (for proper handling of |.aux| files):
%    \begin{macrocode}
\newcommand{\childdocmain}[1]
{
  \childdocdisable\childdocmain{}
  \if?#1?\else
    \begingroup
      \def\childdoctmp{#1}
      \ifx\childdoctmp\childdocname
        \def\childdoctmp{}
      \else
        \def\childdoctmp
        {
          \childdoctrue
          \includeonly{\childdocname}
          \def\childdocjob{#1}
          \def\jobname{#1}
        }
      \fi
      \expandafter
    \endgroup
    \childdoctmp
  \fi
}
%    \end{macrocode}

% \macro{\childdocof}
% The command |\childdocof| redirects
% compilation to the main file |#1|.
%    \begin{macrocode}
\newcommand{\childdocof}[1]
{
  \childdocdisable
  \childdoctrue
  \includeonly{\childdocname}
  \def\jobname{#1}
  \def\childdocjob{#1}
  \input{#1}
}
%    \end{macrocode}

% \macro{\childdocby}
% The command |\childdocby| ....
%    \begin{macrocode}
\newcommand{\childdocby}[2][]
{
  \childdocdisable
  \childdoctrue
  \childdocmanualtrue
  \if?#1?\else
    \def\jobname{#2}
  \fi
  \def\childdocjob{#2}
  \input{#2}
  \endinput
}
%    \end{macrocode}

% \macro{\childdocforward}
% The command |\childdocforward| redirects
% compilation to the main file or
% (if the optional argument is given) a child file.
% Parameters are set as if the main file
% or a child file starting with |\childdocof| was compiled.
% Then compilation is handed over to the main file:
%    \begin{macrocode}
\newcommand{\childdocforward}[2][]
{
  \begingroup
    \if?#1?
      \def\childdoctmp
      {
        \def\childdocname{#2}
        \def\childdocjob{#2}
        \def\jobname{#2}
        \input{#2}
        \endinput
      }
    \else
      \def\childdoctmp
      {
        \childdocdisable
        \def\childdocname{#2}
        \childdoctrue
        \includeonly{#2}
        \def\childdocjob{#1}
        \def\jobname{#1}
        \input{#1}
        \endinput
      }
    \fi
    \expandafter
  \endgroup
  \childdoctmp
}
%    \end{macrocode}

% \macro{\childdocforwardprefix}
% The command |\childdocforwardprefix| redirects
% compilation to the main or a child file by means of a pattern.
% The prefix |#1| in the current filename is replaced by |#2|
% and the suffix of the current filename is kept
% (it is assumed that the filename does not contain the substring `|~~~|'
% which is used as a delimiter).
% Compilation is handed over to the new file by |\childdocforward|:
%    \begin{macrocode}
\newcommand{\childdocforwardprefix}[3][]
{
  \begingroup
    \def\childdocextract #2##1~~~{\def\childdoctmp{\childdocforward[#1]{#3##1}}}
    \expandafter\childdocextract\childdocname~~~
    \expandafter
  \endgroup
  \childdoctmp
}
%    \end{macrocode}

% \macro{\childdoc}
% The deprecated macro |\childdoc| is a legacy version of |\childdocmain|:
%    \begin{macrocode}
\newcommand{\childdoc}{\childdocmain}
%    \end{macrocode}

% \macro{\childdocredirect}
% The deprecated macro |\childdocredirect| is a legacy version
% of |\childdocforward| and |\childdocforwardprefix|:
%    \begin{macrocode}
\newcommand{\childdocredirect}[2][]
{
  \begingroup
    \if?#1?
      \def\childdoctmp{\childdocforward{#2}}
    \else
      \def\childdoctmp{\childdocforwardprefix{#1}{#2}}
    \fi
    \expandafter
  \endgroup
  \childdoctmp
}
%    \end{macrocode}

%\iffalse
%</package>
%\fi
%
\endinput

\childdocof{cdocsamp}
%    \end{macrocode}

%\iffalse
%</samplechap1|samplechap2>
%\fi
%
%\iffalse
%<*samplechap1>
%\fi
% Some text for chapter 1:
%    \begin{macrocode}
\section{one}
some text in chapter one
%    \end{macrocode}

%\iffalse
%</samplechap1>
%\fi
% Some text for chapter 2:
%\iffalse
%<*samplechap2>
%\fi
%    \begin{macrocode}
\section{two}
more text in chapter two
%    \end{macrocode}

%\iffalse
%</samplechap2>
%\fi
%
% %%%%%%%%%%%%%%%%%%%%%%%%%%%%%%%%%%%%%%
% \paragraph{Part Include Files.}
%
% The include files are called |cdocspt3.tex| and |cdocspt4.tex|.
%
%\iffalse
%<*samplepart3|samplepart4>
%\fi

% Optional override for |\version| flag:
%    \begin{macrocode}
%%\providecommand{\version}{final}
%    \end{macrocode}

% Include the main document:
%    \begin{macrocode}
% \iffalse
%
% childdoc.dtx Copyright (C) 2017-2018 Niklas Beisert
%
% This work may be distributed and/or modified under the
% conditions of the LaTeX Project Public License, either version 1.3
% of this license or (at your option) any later version.
% The latest version of this license is in
%   http://www.latex-project.org/lppl.txt
% and version 1.3 or later is part of all distributions of LaTeX
% version 2005/12/01 or later.
%
% This work has the LPPL maintenance status `maintained'.
%
% The Current Maintainer of this work is Niklas Beisert.
%
% This work consists of the files childdoc.dtx and childdoc.ins
% and the derived files childdoc.def and cdocsamp.tex with
% cdocsch1.tex, cdocsch2.tex, cdocsdrf.tex, cdocsfn1.tex, cdocsfn2.tex.
%
%<package>\ifdefined\childdocmain\endinput\fi
%<package>\ProvidesFile{childdoc.def}[2018/12/30 v2.0 child document driver]
%<samplemain>\ProvidesFile{cdocsamp.tex}[2018/12/30 v2.0 sample for childdoc]
%<*driver>
%\ProvidesFile{childdoc.drv}[2018/12/30 v2.0 childdoc reference manual file]
\PassOptionsToClass{10pt,a4paper}{article}
\documentclass{ltxdoc}

\usepackage[margin=35mm]{geometry}
\usepackage{hyperref}
\usepackage{hyperxmp}
\usepackage[usenames]{color}

\hypersetup{colorlinks=true}
\hypersetup{pdfstartview=FitH}
\hypersetup{pdfpagemode=UseNone}
\hypersetup{pdfsource={}}
\hypersetup{pdflang={en-UK}}
\hypersetup{pdfcopyright={Copyright 2017-2018 Niklas Beisert.
  This work may be distributed and/or modified under the
  conditions of the LaTeX Project Public License, either version 1.3
  of this license or (at your option) any later version.}}
\hypersetup{pdflicenseurl={http://www.latex-project.org/lppl.txt}}
\hypersetup{pdfcontactaddress={ETH Zurich, ITP, HIT K,
  Wolfgang-Pauli-Strasse 27}}
\hypersetup{pdfcontactpostcode={8093}}
\hypersetup{pdfcontactcity={Zurich}}
\hypersetup{pdfcontactcountry={Switzerland}}
\hypersetup{pdfcontactemail={nbeisert@itp.phys.ethz.ch}}
\hypersetup{pdfcontacturl={http://people.phys.ethz.ch/\xmptilde nbeisert/}}

\newcommand{\secref}[1]{\hyperref[#1]{section \ref*{#1}}}

\parskip1ex
\parindent0pt
\let\olditemize\itemize
\def\itemize{\olditemize\parskip0pt}

\begin{document}

\title{The \textsf{childdoc} Package}
\hypersetup{pdftitle={The childdoc Package}}
\author{Niklas Beisert\\[2ex]
  Institut f\"ur Theoretische Physik\\
  Eidgen\"ossische Technische Hochschule Z\"urich\\
  Wolfgang-Pauli-Strasse 27, 8093 Z\"urich, Switzerland\\[1ex]
  \href{mailto:nbeisert@itp.phys.ethz.ch}
  {\texttt{nbeisert@itp.phys.ethz.ch}}}
\hypersetup{pdfauthor={Niklas Beisert}}
\hypersetup{pdfsubject={Manual for the LaTeX2e Package childdoc}}
\date{30 December 2018, \textsf{v2.0}}
\maketitle

\begin{abstract}\noindent
\textsf{childdoc} is a \LaTeXe{} package
that enables the direct compilation
of document sections included by |\include|
to individual files.
\end{abstract}

\begingroup
\parskip0ex
\tableofcontents
\endgroup

%%%%%%%%%%%%%%%%%%%%%%%%%%%%%%%%%%%%%%%%%%%%%%%%%%%%%%%%%%%%%%%%%%%%%%%%%%%%%%%%
%%%%%%%%%%%%%%%%%%%%%%%%%%%%%%%%%%%%%%%%%%%%%%%%%%%%%%%%%%%%%%%%%%%%%%%%%%%%%%%%
\section{Introduction}

\LaTeX{} provides a mechanism to structure a large document (such as a book)
into a main file and several child files (containing the chapters)
using the |\include| command.
This mechanism is beneficial for documents
which span hundreds of pages in order to
make the source file(s) more manageable.
Moreover, compilation can be restricted to
selected child files by means of the |\includeonly| command.
The latter feature can be used to reduce the compilation time while editing
(this was significantly more useful in the earlier days of \LaTeX{})
or to generate a smaller document which is easier to navigate.
Another application of |\includeonly| is to generate
documents consisting of selected parts of the complete document.

However, there are a few drawbacks of the plain |\include| mechanism:
\begin{itemize}
\item
The child files cannot be compiled on their own,
they can only be compiled via the main file.
A naive editing environment
(such as a text editor with an option
to have the current file processed by \LaTeX)
may require one to switch to the main file before compiling;
attempting to compile the child file produces errors.
\item
The main file must be modified (each time)
to adjust the |\includeonly| command
to the present needs. This easily leaves the main file in a messy state.
\item
The generated document will always carry the filename
of the main document. This is inconvenient if
several child files are to be compiled and
to be kept for distribution.
\end{itemize}

The present package provides a simple interface
to make child files individually compilable by \LaTeX{}.
Compiling a child file then has the same effect as compiling
the main file with an |\includeonly| command
to select the appropriate child.
Moreover the generated document will carry the name of the child
rather than the main file.
This resolves all three above issues.

This feature is meant to make the editing of books,
thesis documents and lecture notes somewhat more convenient.
However, the package can also be used efficiently for
composing a series of documents (such as exercise sheets)
which are typically distributed individually.
It then assists the author in generating the individual documents
(potentially in different versions)
as well as a document containing the collected series.
Another application is in developing style files
or other kinds of included material
where compilation of the style file could redirect
to a sample or test file.

%%%%%%%%%%%%%%%%%%%%%%%%%%%%%%%%%%%%%%%%%%%%%%%%%%%%%%%%%%%%%%%%%%%%%%%%%%%%%%%%
%%%%%%%%%%%%%%%%%%%%%%%%%%%%%%%%%%%%%%%%%%%%%%%%%%%%%%%%%%%%%%%%%%%%%%%%%%%%%%%%
\section{Usage}

First of all, the package \textsf{childdoc} is \emph{not} a standard
\LaTeXe{} |.sty| style file! Therefore it needs to be invoked in
a non-standard way.

%%%%%%%%%%%%%%%%%%%%%%%%%%%%%%%%%%%%%%%%%%%%%%%%%%%%%%%%%%%%%%%%%%%%%%%%%%%%%%%%
\subsection{Included Files}
\label{sec:include}

%%%%%%%%%%%%%%%%%%%%%%%%%%%%%%%%%%%%%%%%
\DescribeMacro{\childdocmain}
To use the package, add the commands
\begin{center}
\begin{tabular}{l}
|% \iffalse
%
% childdoc.dtx Copyright (C) 2017-2018 Niklas Beisert
%
% This work may be distributed and/or modified under the
% conditions of the LaTeX Project Public License, either version 1.3
% of this license or (at your option) any later version.
% The latest version of this license is in
%   http://www.latex-project.org/lppl.txt
% and version 1.3 or later is part of all distributions of LaTeX
% version 2005/12/01 or later.
%
% This work has the LPPL maintenance status `maintained'.
%
% The Current Maintainer of this work is Niklas Beisert.
%
% This work consists of the files childdoc.dtx and childdoc.ins
% and the derived files childdoc.def and cdocsamp.tex with
% cdocsch1.tex, cdocsch2.tex, cdocsdrf.tex, cdocsfn1.tex, cdocsfn2.tex.
%
%<package>\ifdefined\childdocmain\endinput\fi
%<package>\ProvidesFile{childdoc.def}[2018/12/30 v2.0 child document driver]
%<samplemain>\ProvidesFile{cdocsamp.tex}[2018/12/30 v2.0 sample for childdoc]
%<*driver>
%\ProvidesFile{childdoc.drv}[2018/12/30 v2.0 childdoc reference manual file]
\PassOptionsToClass{10pt,a4paper}{article}
\documentclass{ltxdoc}

\usepackage[margin=35mm]{geometry}
\usepackage{hyperref}
\usepackage{hyperxmp}
\usepackage[usenames]{color}

\hypersetup{colorlinks=true}
\hypersetup{pdfstartview=FitH}
\hypersetup{pdfpagemode=UseNone}
\hypersetup{pdfsource={}}
\hypersetup{pdflang={en-UK}}
\hypersetup{pdfcopyright={Copyright 2017-2018 Niklas Beisert.
  This work may be distributed and/or modified under the
  conditions of the LaTeX Project Public License, either version 1.3
  of this license or (at your option) any later version.}}
\hypersetup{pdflicenseurl={http://www.latex-project.org/lppl.txt}}
\hypersetup{pdfcontactaddress={ETH Zurich, ITP, HIT K,
  Wolfgang-Pauli-Strasse 27}}
\hypersetup{pdfcontactpostcode={8093}}
\hypersetup{pdfcontactcity={Zurich}}
\hypersetup{pdfcontactcountry={Switzerland}}
\hypersetup{pdfcontactemail={nbeisert@itp.phys.ethz.ch}}
\hypersetup{pdfcontacturl={http://people.phys.ethz.ch/\xmptilde nbeisert/}}

\newcommand{\secref}[1]{\hyperref[#1]{section \ref*{#1}}}

\parskip1ex
\parindent0pt
\let\olditemize\itemize
\def\itemize{\olditemize\parskip0pt}

\begin{document}

\title{The \textsf{childdoc} Package}
\hypersetup{pdftitle={The childdoc Package}}
\author{Niklas Beisert\\[2ex]
  Institut f\"ur Theoretische Physik\\
  Eidgen\"ossische Technische Hochschule Z\"urich\\
  Wolfgang-Pauli-Strasse 27, 8093 Z\"urich, Switzerland\\[1ex]
  \href{mailto:nbeisert@itp.phys.ethz.ch}
  {\texttt{nbeisert@itp.phys.ethz.ch}}}
\hypersetup{pdfauthor={Niklas Beisert}}
\hypersetup{pdfsubject={Manual for the LaTeX2e Package childdoc}}
\date{30 December 2018, \textsf{v2.0}}
\maketitle

\begin{abstract}\noindent
\textsf{childdoc} is a \LaTeXe{} package
that enables the direct compilation
of document sections included by |\include|
to individual files.
\end{abstract}

\begingroup
\parskip0ex
\tableofcontents
\endgroup

%%%%%%%%%%%%%%%%%%%%%%%%%%%%%%%%%%%%%%%%%%%%%%%%%%%%%%%%%%%%%%%%%%%%%%%%%%%%%%%%
%%%%%%%%%%%%%%%%%%%%%%%%%%%%%%%%%%%%%%%%%%%%%%%%%%%%%%%%%%%%%%%%%%%%%%%%%%%%%%%%
\section{Introduction}

\LaTeX{} provides a mechanism to structure a large document (such as a book)
into a main file and several child files (containing the chapters)
using the |\include| command.
This mechanism is beneficial for documents
which span hundreds of pages in order to
make the source file(s) more manageable.
Moreover, compilation can be restricted to
selected child files by means of the |\includeonly| command.
The latter feature can be used to reduce the compilation time while editing
(this was significantly more useful in the earlier days of \LaTeX{})
or to generate a smaller document which is easier to navigate.
Another application of |\includeonly| is to generate
documents consisting of selected parts of the complete document.

However, there are a few drawbacks of the plain |\include| mechanism:
\begin{itemize}
\item
The child files cannot be compiled on their own,
they can only be compiled via the main file.
A naive editing environment
(such as a text editor with an option
to have the current file processed by \LaTeX)
may require one to switch to the main file before compiling;
attempting to compile the child file produces errors.
\item
The main file must be modified (each time)
to adjust the |\includeonly| command
to the present needs. This easily leaves the main file in a messy state.
\item
The generated document will always carry the filename
of the main document. This is inconvenient if
several child files are to be compiled and
to be kept for distribution.
\end{itemize}

The present package provides a simple interface
to make child files individually compilable by \LaTeX{}.
Compiling a child file then has the same effect as compiling
the main file with an |\includeonly| command
to select the appropriate child.
Moreover the generated document will carry the name of the child
rather than the main file.
This resolves all three above issues.

This feature is meant to make the editing of books,
thesis documents and lecture notes somewhat more convenient.
However, the package can also be used efficiently for
composing a series of documents (such as exercise sheets)
which are typically distributed individually.
It then assists the author in generating the individual documents
(potentially in different versions)
as well as a document containing the collected series.
Another application is in developing style files
or other kinds of included material
where compilation of the style file could redirect
to a sample or test file.

%%%%%%%%%%%%%%%%%%%%%%%%%%%%%%%%%%%%%%%%%%%%%%%%%%%%%%%%%%%%%%%%%%%%%%%%%%%%%%%%
%%%%%%%%%%%%%%%%%%%%%%%%%%%%%%%%%%%%%%%%%%%%%%%%%%%%%%%%%%%%%%%%%%%%%%%%%%%%%%%%
\section{Usage}

First of all, the package \textsf{childdoc} is \emph{not} a standard
\LaTeXe{} |.sty| style file! Therefore it needs to be invoked in
a non-standard way.

%%%%%%%%%%%%%%%%%%%%%%%%%%%%%%%%%%%%%%%%%%%%%%%%%%%%%%%%%%%%%%%%%%%%%%%%%%%%%%%%
\subsection{Included Files}
\label{sec:include}

%%%%%%%%%%%%%%%%%%%%%%%%%%%%%%%%%%%%%%%%
\DescribeMacro{\childdocmain}
To use the package, add the commands
\begin{center}
\begin{tabular}{l}
|\input{childdoc.def}|\\
|\childdocmain{}|\\
\end{tabular}
\end{center}
at the very top of the main \LaTeX{} file,
in particular \emph{before} the |\documentclass| statement!
The argument of |\childdocmain| should be left empty
(but it must be present).

%%%%%%%%%%%%%%%%%%%%%%%%%%%%%%%%%%%%%%%%
\DescribeMacro{\childdocof}
Furthermore, add the commands
\begin{center}
\begin{tabular}{l}
|\input{childdoc.def}|\\
|\childdocof{|\textit{main}|}|\\
\end{tabular}
\end{center}
at the top of every child file \textit{child}
which is included by |\include{|\textit{child}|}|
from within the main file
(or at least for those files to be compiled individually).
The argument \textit{main} must be the filename of the main file.

There are a couple of
considerations in setting up the main and child documents:

%%%%%%%%%%%%%%%%%%%%%%%%%%%%%%%%%%%%%%%%
\paragraph{Restrictions.}

Please note the following restrictions:
\begin{itemize}
\item
|\childdocmain| must be called with one argument \textit{main}
to ensure compatibility with earlier version of the package.
It must either be empty (|\childdocmain{}|)
or precisely match the filename of the main file in which it is specified.
See \secref{sec:detection} for further information.
\item
The filename \textit{main} must be specified without the |.tex| extension.
\item
The filename \textit{main} is case sensitive
(even in case-insensitive file systems)
due to internal string comparison.
\item
The argument \textit{main} should be fully expanded, it cannot be a macro.
\item
Subdirectories and special characters should be avoided in filenames.
\item
The command |\childdocmain{|\textit{main}|}| must be followed by a whitespace.
It should not be followed immediately by another command
or by a comment mark `|%|'.
This is because the \TeX{} parser reads the token immediately following
the argument of |\childdocmain| and puts it
at the beginning of every child section;
however, a white\-space is ignored.
\end{itemize}

%%%%%%%%%%%%%%%%%%%%%%%%%%%%%%%%%%%%%%%%
\paragraph{Content of Main File.}

It is advisable to place all content in the child files included by |\include|.
Any output contained in the main file will appear in all child documents
unless suppressed manually;
it cannot be suppressed automatically by the |\includeonly| directive
and thus should normally be avoided.
A method to include some content in the main file
by means of conditional processing is described in \secref{sec:conditional}.

%%%%%%%%%%%%%%%%%%%%%%%%%%%%%%%%%%%%%%%%
\paragraph{Page Numbering.}

When only a part of the document is compiled,
the appropriate numbering of pages
(as well as other status parameters)
is determined from the |.aux| files.
The latter contain information from previous passes.
However this information needs to propagate through
all intermediate child documents.
Therefore the page numbering in child documents may well
be inconsistent until the complete document is compiled at least once.

A useful (if unconventional) way to always ensure a consistent
page numbering is to restart the numbering in each child document
and denote the pages by `\textit{child}|.|\textit{page}'
where \textit{child} represents the chapter/section number of the child file.
This can be achieved by the command
|\numberwithin{page}{|\textit{child}|}|
of the \textsf{amsmath} package
where \textit{child} can be |chapter| or |section|
depending on the chosen structuring.
Alternatively, one can modify the macro |\thepage| appropriately
and reset the counter |page| at the start of each child file.

%%%%%%%%%%%%%%%%%%%%%%%%%%%%%%%%%%%%%%%%%%%%%%%%%%%%%%%%%%%%%%%%%%%%%%%%%%%%%%%%
\subsection{Conditional Processing}
\label{sec:conditional}

The package provides a mechanism to compile different versions
of a document. To customise the versions further some conditional processing
can come in handy to distinguish which version is being compiled.
The package provides two macros to describe the compilation context:

%%%%%%%%%%%%%%%%%%%%%%%%%%%%%%%%%%%%%%%%
\DescribeMacro{\ifchilddoc}
The conditional |\ifchilddoc| distinguishes between the compilation of
child documents and the main document:
%
\begin{center}
|\ifchilddoc |\textit{child-code}| |[|\||else |\textit{main-code}]| \||fi|
\end{center}

%%%%%%%%%%%%%%%%%%%%%%%%%%%%%%%%%%%%%%%%
\DescribeMacro{\childdocname}
\DescribeMacro{\childdocjob}
The macro |\childdocname| contains the filename (without extension)
of the main or child file being processed.
Note that |\childdocjob| will always contain the name of the main file.

%%%%%%%%%%%%%%%%%%%%%%%%%%%%%%%%%%%%%%%%
\paragraph{Title Page.}

Conditional processing can be used to include a title or banner page
in the main document when proper precautions are taken.
Importantly, the code in the main file should ensure that the page counter
(as well as other status parameters which are stored in the |.aux| files)
takes the same value after the conditional processing.
Otherwise the page numbers may take divergent values
depending on which part is compiled.

For example, a title page could be declared by:
%
\begin{center}
\begin{tabular}{l}
|\ifchilddoc\||else|\\
|\addtocounter{page}{-1}|\\
\textit{code for title page}\\
|\newpage|\\
|\||fi|
\end{tabular}
\end{center}
%
A banner page for the child documents can be generated by:
%
\begin{center}
\begin{tabular}{l}
|\ifchilddoc|\\
|\addtocounter{page}{-1}|\\
\textit{code for banner page}\\
|\newpage|\\
|\||fi|
\end{tabular}
\end{center}
%
Here one could write a message such as:
\begin{center}
|This is the part \childdocname{} of \childdocjob{}.|
\end{center}

%%%%%%%%%%%%%%%%%%%%%%%%%%%%%%%%%%%%%%%%%%%%%%%%%%%%%%%%%%%%%%%%%%%%%%%%%%%%%%%%
\subsection{Flags}
\label{sec:flags}

The package makes it easy to generate different versions
of the main or child documents.
To this end compilation flags can be defined
and assigned different default values.
They will be particularly useful in conjunction
with the forwarding mechanism described in \secref{sec:forward}.

For example, it may be useful to have a flag |\version|
which can be set to |draft| or |final|.
The document source will contain some conditional code
depending on the value of |\version|.
Suppose further, the flag should default to |final| for the main file
and to |draft| for child files
which is a natural assignment for editing the document.
This is achieved by placing the following code
in the preamble of the main document
(below the |\childdocmain| directive):
%
\begin{center}
\begin{tabular}{l}
|\ifchilddoc|\\
|\providecommand{\version}{draft}|\\
|\||else|\\
|\providecommand{\version}{final}|\\
|\||fi|
\end{tabular}
\end{center}
%
The definition by |\providecommand| makes sure
that previous definitions are not overwritten.
Further statements |\providecommand{\version}{...}|
can thus be added before the above code to override it.

For the main file, one might add a line
(between |\childdocmain| and the above block)
%
\begin{center}
|%\ifchilddoc\||else\providecommand{\version}{draft}\||fi|
\end{center}
%
which can be uncommented to produce a draft version.
Likewise one can add a line to the very top of a child file
(above the |\childdocof{|\textit{main}|}| directive)
%
\begin{center}
|%\providecommand{\version}{final}|
\end{center}
%
which can be uncommented to produce the final version of this child document.

%%%%%%%%%%%%%%%%%%%%%%%%%%%%%%%%%%%%%%%%%%%%%%%%%%%%%%%%%%%%%%%%%%%%%%%%%%%%%%%%
\subsection{Forwarding}
\label{sec:forward}

Different versions of the main or child documents
using compilation flags as described in \secref{sec:flags}
can be (permanently) stored in different files
for convenient compilation, viewing and distribution.
To this end, the package defines a command
to pass on compilation to a different file:

%%%%%%%%%%%%%%%%%%%%%%%%%%%%%%%%%%%%%%%%
\DescribeMacro{\childdocforward}
The command |\childdocforward| redirects processing to
another source file:
%
\begin{center}
\begin{tabular}{l}
|\input{childdoc.def}|\\
|\childdocforward[|\textit{main}|]{|\textit{dest}|}|\\
\end{tabular}
\end{center}
%
The argument \textit{dest} is the destination file
(without extension).
It should be the main file or one of the child files.
Note that further \textsf{childdoc} directives
such as |\childdocof| and |\childdocforward|
in the indicated file will be processed in this form.
The optional argument \textit{main}
passes on directly to the main file \textit{main}
while pretending to compile the child \textit{dest}.
This form behaves as if \textit{dest}
issues |\childdocof{|\textit{main}|}| right away,
and no further \textsf{childdoc} directives will be processed.

%%%%%%%%%%%%%%%%%%%%%%%%%%%%%%%%%%%%%%%%
\DescribeMacro{\...prefix}
In the alternative form |\childdocforwardprefix|,
%
\begin{center}
\begin{tabular}{l}
|\input{childdoc.def}|\\
|\childdocforwardprefix[|\textit{main}|]{|\textit{prefix}|}{|\textit{dest}|}|
\end{tabular}
\end{center}
%
the destination file is determined by a pattern
depending on the current file:
To make this work, the current file must be called
`{\textit{prefix}\hspace{0.2em}\textit{suffix}}'
with \textit{prefix} matching precisely the argument.
Processing is then passed on to the file
`{\textit{dest}\hspace{0.2em}\textit{suffix}}'.
Surely, the same effect is achieved by
directly specifying the
argument `{\textit{dest}\hspace{0.2em}\textit{suffix}}'
in the first form.
However, that requires to set up a different file
for each child. With the alternative form of the command
all these files can have exactly the same content
which simplifies setting them up and maintaining them.

For example, the following file |draft.tex|
with a compilation flag |\version| as described in \secref{sec:flags}
compiles the main document as a draft:
%
\begin{center}
\begin{tabular}{l}
|\def\version{draft}|\\
|\input{childdoc.def}|\\
|\childdocforward{|\textit{main}|}|
\end{tabular}
\end{center}
%
Likewise, the following files |final|\textit{nn}|.tex|
compile the final version of the child document
|child|\textit{nn}|.tex|:
%
\begin{center}
\begin{tabular}{l}
|\def\version{final}|\\
|\input{childdoc.def}|\\
|\childdocforwardprefix{final}{child}|
\end{tabular}
\end{center}
%

Note that when several versions of a main file and/or of each child file
are to be generated, it may be convenient to set up a |Makefile| or
shell script to automatise the process.

%%%%%%%%%%%%%%%%%%%%%%%%%%%%%%%%%%%%%%%%%%%%%%%%%%%%%%%%%%%%%%%%%%%%%%%%%%%%%%%%
\subsection{Command Line Processing}
\label{sec:commandline}

The effect of redirection files can also be achieved by invoking
the \LaTeX{} compiler with a more elaborate command line.
Most conveniently this should be done as part
of a shell script or a |Makefile|.

When using \textsf{childdoc} in the main file, the following
command lines effectively perform a redirection
(note that depending on the shell being used,
backslashes may have to be doubled: `|\|' $\to$ `|\\|'):
%
\begin{center}
|... -jobname "|\textit{target}|" |\\|"|[\textit{flags}]%
|\input{childdoc.def}\childdocforward[|\textit{main}|]{|\textit{dest}|}"|
\end{center}
%
Here \textit{target} is the name of the output file,
\textit{main} is the name of the main file
and \textit{dest} is the name of the main or child file to be processed
(all filenames without extensions).
The optional argument \textit{main} can be omitted
if \textit{main} matches \textit{dest}.
Optionally, compilation \textit{flags} can be defined via |\def| commands.
This command line makes the \TeX{} engine believe
it is compiling the file \textit{target}
whose content is specified as the latter parameter.
The provided code then forwards the processing to
\textit{main} or \textit{dest} as described in \secref{sec:forward}.

%%%%%%%%%%%%%%%%%%%%%%%%%%%%%%%%%%%%%%%%%%%%%%%%%%%%%%%%%%%%%%%%%%%%%%%%%%%%%%%%
\subsection{Include by Input}
\label{sec:input}

Including child documents by |\include| has some restrictions by design.
Most notably, the content of a child document always occupies
its own set of pages; pages cannot be shared between child documents.
Usually, this behaviour makes perfect sense
because each child document contain an essential part of the document.
However, in some situations it may be desirable to compose
a document from a collection of parts
without having mandatory page breaks between then.
For this case, the package
provides a mechanism to include parts
by |\input| which can also be processed individually.
However, by construction this mechanism
requires manual handling of the content to be output.

%%%%%%%%%%%%%%%%%%%%%%%%%%%%%%%%%%%%%%%%
\DescribeMacro{\ifchilddocmanual}
The main file should be prepared as usual, see \secref{sec:include}.
However, the document body must make a distinction
between processing of an individual part and of the main document, e.g.:
%
\begin{center}
\begin{tabular}{l}
|\ifchilddocmanual|\\
|\input{\childdocname}|\\
|\||else|\\
\textit{document body with }|\input{|\textit{part}|}|\\
|\||fi|
\end{tabular}
\end{center}
%
The conditional |\ifchilddocmanual| is true whenever
a part to be included by |\input| is being compiled,
and the name of the part is stored in |\childdocname|.

%%%%%%%%%%%%%%%%%%%%%%%%%%%%%%%%%%%%%%%%
\DescribeMacro{\childdocby}
Each part to be included by |\input| should start with:
%
\begin{center}
\begin{tabular}{l}
|\input{childdoc.def}|\\
|\childdocby{|\textit{main}|}|\\
\end{tabular}
\end{center}
%
The directive |\childdocby| is similar to |\childdocof|
described in \secref{sec:include},
but the subsequent selection of content must be done manually.
To that end, both |\ifchilddoc| and |\ifchilddocmanual|
will be true upon processing of a part,
and the name of the part is stored in |\childdocname|.
Note that |\jobname| will be set to the filename of the current part
so that each part receives an individual |.aux| file
that does not interfere with the |.aux| file(s) of the main document.
This behaviour can be altered by the alternative form
|\childdocby[*]{|\textit{main}|}| (with a non-empty optional argument)
which uses the |.aux| file of the main document
by setting |\jobname| to \textit{main}.

%%%%%%%%%%%%%%%%%%%%%%%%%%%%%%%%%%%%%%%%%%%%%%%%%%%%%%%%%%%%%%%%%%%%%%%%%%%%%%%%
\subsection{Driver Development}
\label{sec:driver}

The \textsf{childdoc} mechanism can also be use for the development
of definition files such as \LaTeX{} styles or classes.
This case differs from the above setup with multiple parts
included by |\include| in that no |\includeonly| should be invoked.
This can be achieved by starting the include file
(before |\ProvidesPackage|) with:
%
\begin{center}
\begin{tabular}{l}
|\input{childdoc.def}|\\
|\childdocforward{|\textit{main}|}|\\
\end{tabular}
\end{center}
%
or alternatively with:
%
\begin{center}
\begin{tabular}{l}
|\input{childdoc.def}|\\
|\childdocby{|\textit{main}|}|\\
\end{tabular}
\end{center}
%
Both forms have slightly different effects as described above.
The main file is prepared as usual, see \secref{sec:include}.

%%%%%%%%%%%%%%%%%%%%%%%%%%%%%%%%%%%%%%%%%%%%%%%%%%%%%%%%%%%%%%%%%%%%%%%%%%%%%%%%
\subsection{Legacy Detection}
\label{sec:detection}

The directive |\childdocmain| in the main file can detect
whether the complete document or merely a child is to be compiled
even without using the directive |\childdocof|.
This method is deprecated because it is less robust
and there is no compelling reason to use it;
it is merely provided for backward compatibility
and it may be removed in future versions.

If the detection mechanism is to be used,
it is mandatory to correctly specify
the filename of the main file as the argument of |\childdocmain|:
%
\begin{center}
\begin{tabular}{l}
|\input{childdoc.def}|\\
|\childdocmain{|\textit{main}|}|\\
\end{tabular}
\end{center}
%
If |\jobname| does not match the argument \textit{main} of |\childdocmain|,
it is assumed that |\jobname| points to the child file to be compiled.
When using |\childdocmain| with the main file specified as argument,
it suffices to start a child file
with just |\input{|\textit{main}|}|
without loading of the package and using |\childdocof|.
If instead all processing is done
with the appropriate \textsf{childdoc} directives,
the argument of \textit{main} of |\childdocmain| can be empty.

An alternative version of the command line processing described
in \secref{sec:commandline} using the detection mechanism reads:
%
\begin{center}
|... -jobname "|\textit{target}|" "|[\textit{flags}]%
[|\def\jobname{|\textit{dest}|}|]|\input{|\textit{main}|}"|
\end{center}

%%%%%%%%%%%%%%%%%%%%%%%%%%%%%%%%%%%%%%%%%%%%%%%%%%%%%%%%%%%%%%%%%%%%%%%%%%%%%%%%
\subsection{Manual Code}
\label{sec:manual}

In case one cannot be certain whether the definitions file |childdoc.def|
is installed on the target \TeX{} distribution
and one prefers not to ship it,
it is conceivable to paste a few relevant commands into the sources.

To that end, drop all statements |\input{childdoc.def}|
and perform the replacements as outlined below.
Instead of |\childdocmain{|\textit{main}|}| add the following code
to the top of the main file:
%
\begin{center}
\begin{tabular}{l}
|\||ifdefined\childdocname\endinput\||fi\newif\ifchilddoc|\\
|\edef\childdocname{\scantokens\expandafter{\jobname\noexpand}}|\\
|\def\childdocmain{|\textit{main}|}\||ifx\childdocmain\childdocname\||else|\\
|\childdoctrue\includeonly{\childdocname}\let\jobname\childdocmain\||fi|\\
\end{tabular}
\end{center}
%
Instead of |\childdocof{|\textit{main}|}| just include the main file
at the top of each child file:
%
\begin{center}
|\input{|\textit{main}|}|
\end{center}
%
A simple redirection |\childdocforward{|\textit{dest}|}| is achieved by:
%
\begin{center}
|\def\jobname{|\textit{dest}|}\input{\jobname}|
\end{center}
%
The redirection with prefix
|\childdocforwardprefix[|\textit{prefix}|]{|\textit{dest}|}|
is accomplished by:
%
\begin{center}
\begin{tabular}{l}
|{\edef\jobname{\scantokens\expandafter{\jobname\noexpand}}|\\
|\def\redirectjob |\textit{prefix}|#1~~~{\gdef\jobname{|\textit{dest}|#1}}|\\
|\expandafter\redirectjob\jobname~~~}\input{\jobname}|
\end{tabular}
\end{center}

In an alternative approach,
child documents can be compiled by a specific command line
without additional code or specific definitions:
%
\begin{center}
|... -jobname "|\textit{target}|" "|[\textit{flags}]%
|\includeonly{|\textit{dest}|}\input{|\textit{main}|}"|
\end{center}
%

%%%%%%%%%%%%%%%%%%%%%%%%%%%%%%%%%%%%%%%%%%%%%%%%%%%%%%%%%%%%%%%%%%%%%%%%%%%%%%%%
%%%%%%%%%%%%%%%%%%%%%%%%%%%%%%%%%%%%%%%%%%%%%%%%%%%%%%%%%%%%%%%%%%%%%%%%%%%%%%%%
\section{Information}

%%%%%%%%%%%%%%%%%%%%%%%%%%%%%%%%%%%%%%%%%%%%%%%%%%%%%%%%%%%%%%%%%%%%%%%%%%%%%%%%
\subsection{Copyright}

Copyright \copyright{} 2017--2018 Niklas Beisert

This work may be distributed and/or modified under the
conditions of the \LaTeX{} Project Public License, either version 1.3
of this license or (at your option) any later version.
The latest version of this license is in
  \url{http://www.latex-project.org/lppl.txt}
and version 1.3 or later is part of all distributions of \LaTeX{}
version 2005/12/01 or later.

This work has the LPPL maintenance status `maintained'.

The Current Maintainer of this work is Niklas Beisert.

This work consists of the files |README.txt|, |childdoc.ins| and |childdoc.dtx|
as well as the derived files |childdoc.def|, |cdocsamp.tex|
with |cdocsch1.tex|, |cdocsch2.tex|, |cdocspt3.tex|, |cdocspt4.tex|,
|cdocsdrf.tex|, |cdocsfn1.tex|, |cdocsfn2.tex|
as well as |childdoc.pdf|.

%%%%%%%%%%%%%%%%%%%%%%%%%%%%%%%%%%%%%%%%%%%%%%%%%%%%%%%%%%%%%%%%%%%%%%%%%%%%%%%%
\subsection{Files and Installation}

The package consists of the files:
%
\begin{center}
\begin{tabular}{ll}
    |README.txt|   & readme file \\
    |childdoc.ins| & installation file \\
    |childdoc.dtx| & source file \\
    |childdoc.def| & definition file \\
    |cdocsamp.tex| & sample main file \\
    |cdocsch1.tex| & sample include file \\
    |cdocsch2.tex| & sample include file \\
    |cdocspt3.tex| & sample part file \\
    |cdocspt4.tex| & sample part file \\
    |cdocsdrf.tex| & sample redirection file \\
    |cdocsfn1.tex| & sample redirection file \\
    |cdocsfn2.tex| & sample redirection file \\
    |childdoc.pdf| & manual
\end{tabular}
\end{center}
%
The distribution consists of the files
|README.txt|, |childdoc.ins| and |childdoc.dtx|.
%
\begin{itemize}
\item
Run (pdf)\LaTeX{} on |childdoc.dtx|
to compile the manual |childdoc.pdf| (this file).
\item
Run \LaTeX{} on |childdoc.ins| to create the definitions file |childdoc.def|
and the sample |cdocsamp.tex| with include files
|cdocsch1.tex|, |cdocsch2.tex|, |cdocspt3.tex|, |cdocspt4.tex|,
|cdocsdrf.tex|, |cdocsfn1.tex|, |cdocsfn2.tex|.
Then copy the file |childdoc.def| to an appropriate directory of your \LaTeX{}
distribution, e.g.\ \textit{texmf-root}|/tex/latex/childdoc|.
\end{itemize}

%%%%%%%%%%%%%%%%%%%%%%%%%%%%%%%%%%%%%%%%%%%%%%%%%%%%%%%%%%%%%%%%%%%%%%%%%%%%%%%%
\subsection{Related CTAN Packages}

There are several other packages which offer a similar functionality:
%
\begin{itemize}
\item
The packages
\href{http://ctan.org/pkg/docmute}{\textsf{docmute}},
\href{http://ctan.org/pkg/includex}{\textsf{includex}} and
\href{http://ctan.org/pkg/standalone}{\textsf{standalone}}
provide commands to include only the document body of
a child file thus allowing both files to be compiled individually.
\item
The packages \href{http://ctan.org/pkg/subdocs}{\textsf{subdocs}}
and \href{http://ctan.org/pkg/subfiles}{\textsf{subfiles}}
provide structures in which the main and child documents can be
encapsulated and allowing them to be compiled individually.
The inclusion mechanism is different from the conventional |\include|.
\item
The package \href{http://ctan.org/pkg/combine}{\textsf{combine}}
is an elaborate solution to combine several documents into one.
\end{itemize}
%
See also the CTAN topic \href{http://ctan.org/topic/subdocs}{\textsf{subdocs}}
for further related packages.
The present package differs from the above solutions in that
a document structure constructed with the conventional |\include| mechanism
just needs two extra commands at the top of every file
such that all constituent files can be compiled individually.

%%%%%%%%%%%%%%%%%%%%%%%%%%%%%%%%%%%%%%%%%%%%%%%%%%%%%%%%%%%%%%%%%%%%%%%%%%%%%%%%
%\subsection{Feature Suggestions}
%
%The following is a list of features which may be useful for future
%versions of this package:
%%
%\begin{itemize}
%\item
%\ldots
%\end{itemize}

%%%%%%%%%%%%%%%%%%%%%%%%%%%%%%%%%%%%%%%%%%%%%%%%%%%%%%%%%%%%%%%%%%%%%%%%%%%%%%%%
\subsection{Revision History}

%%%%%%%%%%%%%%%%%%%%%%%%%%%%%%%%%%%%%%%%
\paragraph{v2.0:} 2018/12/30

\begin{itemize}
\item
immediate forward processing
\item
added |\childdocby| mechanism
\item
manual restructured
\end{itemize}

%%%%%%%%%%%%%%%%%%%%%%%%%%%%%%%%%%%%%%%%
\paragraph{v1.6:} 2018/01/17

\begin{itemize}
\item
application for development of include files
\item
corrections to manual
\end{itemize}

%%%%%%%%%%%%%%%%%%%%%%%%%%%%%%%%%%%%%%%%
\paragraph{v1.5:} 2017/05/21

\begin{itemize}
\item
more complete structuring introduced
\item
|\childdocof| introduced
\item
|\childdoc| renamed to |\childdocmain|
\item
|\childredirect| renamed to |\childdocforward| and |\childdocforwardprefix|
and functionality expanded
\end{itemize}

%%%%%%%%%%%%%%%%%%%%%%%%%%%%%%%%%%%%%%%%
\paragraph{v1.0:} 2017/04/27

\begin{itemize}
\item
manual and install package
\item
first version published on CTAN
\end{itemize}

%%%%%%%%%%%%%%%%%%%%%%%%%%%%%%%%%%%%%%%%
\paragraph{v0.6:} 2017/04/26

\begin{itemize}
\item
redirection mechanism added
\end{itemize}

%%%%%%%%%%%%%%%%%%%%%%%%%%%%%%%%%%%%%%%%
\paragraph{v0.5:} 2017/04/26

\begin{itemize}
\item
functionality in definition file
\end{itemize}


%%%%%%%%%%%%%%%%%%%%%%%%%%%%%%%%%%%%%%%%%%%%%%%%%%%%%%%%%%%%%%%%%%%%%%%%%%%%%%%%
%%%%%%%%%%%%%%%%%%%%%%%%%%%%%%%%%%%%%%%%%%%%%%%%%%%%%%%%%%%%%%%%%%%%%%%%%%%%%%%%
%%%%%%%%%%%%%%%%%%%%%%%%%%%%%%%%%%%%%%%%%%%%%%%%%%%%%%%%%%%%%%%%%%%%%%%%%%%%%%%%
\appendix

\settowidth\MacroIndent{\rmfamily\scriptsize 000\ }

 \DocInput{childdoc.dtx}

\end{document}
%</driver>
% \fi
%
% %%%%%%%%%%%%%%%%%%%%%%%%%%%%%%%%%%%%%%%%%%%%%%%%%%%%%%%%%%%%%%%%%%%%%%%%%%%%%%
% %%%%%%%%%%%%%%%%%%%%%%%%%%%%%%%%%%%%%%%%%%%%%%%%%%%%%%%%%%%%%%%%%%%%%%%%%%%%%%
% \section{Sample}
%\iffalse
%<*samplemain>
%\fi
%
% The following presents a sample document
% with two chapters, two parts, a title page,
% a compile flag as well as three forwarding files to set the flag.
% It consists of eight |.tex| files:
% \begin{center}
% \begin{tabular}{ll}
% |cdocsamp.tex|&main file\\
% |cdocsch1.tex|&include file for chapter 1\\
% |cdocsch2.tex|&include file for chapter 2\\
% |cdocspt3.tex|&include file for part 3\\
% |cdocspt4.tex|&include file for part 4\\
% |cdocsdrf.tex|&forwarding file for main file in draft mode\\
% |cdocsfi1.tex|&forwarding file for final version of chapter 1\\
% |cdocsfi2.tex|&forwarding file for final version of chapter 2\\
% \end{tabular}
% \end{center}
% Each of the eight files can be compiled directly by the \LaTeX{} compiler.
%
% %%%%%%%%%%%%%%%%%%%%%%%%%%%%%%%%%%%%%%
% \paragraph{Main File.}
%
% The main file is called |cdocsamp.tex|.
%
% Load the \textsf{childdoc} definitions and
% declare the filename for the main document:
%    \begin{macrocode}
\input{childdoc.def}
\childdocmain{}
%    \end{macrocode}

% Optional override for |\version| flag:
%    \begin{macrocode}
%%\ifchilddoc\else\providecommand{\version}{draft}\fi
%    \end{macrocode}

% Define the default values for the |\version| flag
% (|final| for the main file and |draft| for childs):
%    \begin{macrocode}
\ifchilddoc
\providecommand{\version}{draft}
\else
\providecommand{\version}{final}
\fi
%    \end{macrocode}

% Load the standard document class:
%    \begin{macrocode}
\documentclass[12pt]{article}
%    \end{macrocode}

% Start the document body:
%    \begin{macrocode}
\begin{document}
%    \end{macrocode}

% Declare a title page.
% Print title, part of document being processed and version flag:
%    \begin{macrocode}
\addtocounter{page}{-1}
\begin{center}
{\LARGE\bfseries{}childdoc example\par}
\vspace{1cm}
\ifchilddoc
\ifchilddocmanual part\else chapter\fi:
`\childdocname' of `\childdocjob'\par
\else
main document: `\childdocjob'\par
\fi
version: \version\par
\end{center}
\newpage
%    \end{macrocode}

% Manually include selected file,
% otherwise process as usual:
%    \begin{macrocode}
\ifchilddocmanual
\section*{part `\childdocname'}
\input{\childdocname}
\else
%    \end{macrocode}

% Include the two chapters:
%    \begin{macrocode}
\include{cdocsch1}
\include{cdocsch2}
%    \end{macrocode}

% Include the two parts unless only chapters should be displayed:
%    \begin{macrocode}
\ifchilddoc\else
\section{part three}
\input{cdocspt3}
\section{part four}
\input{cdocspt4}
\fi
%    \end{macrocode}

% Process as usual until here:
%    \begin{macrocode}
\fi
%    \end{macrocode}

% End of document body:
%    \begin{macrocode}
\end{document}
%    \end{macrocode}
%\iffalse
%</samplemain>
%\fi
%
% %%%%%%%%%%%%%%%%%%%%%%%%%%%%%%%%%%%%%%
% \paragraph{Chapter Include Files.}
%
% The include files are called |cdocsch1.tex| and |cdocsch2.tex|.
%
%\iffalse
%<*samplechap1|samplechap2>
%\fi

% Optional override for |\version| flag:
%    \begin{macrocode}
%%\providecommand{\version}{final}
%    \end{macrocode}

% Include the main document:
%    \begin{macrocode}
\input{childdoc.def}
\childdocof{cdocsamp}
%    \end{macrocode}

%\iffalse
%</samplechap1|samplechap2>
%\fi
%
%\iffalse
%<*samplechap1>
%\fi
% Some text for chapter 1:
%    \begin{macrocode}
\section{one}
some text in chapter one
%    \end{macrocode}

%\iffalse
%</samplechap1>
%\fi
% Some text for chapter 2:
%\iffalse
%<*samplechap2>
%\fi
%    \begin{macrocode}
\section{two}
more text in chapter two
%    \end{macrocode}

%\iffalse
%</samplechap2>
%\fi
%
% %%%%%%%%%%%%%%%%%%%%%%%%%%%%%%%%%%%%%%
% \paragraph{Part Include Files.}
%
% The include files are called |cdocspt3.tex| and |cdocspt4.tex|.
%
%\iffalse
%<*samplepart3|samplepart4>
%\fi

% Optional override for |\version| flag:
%    \begin{macrocode}
%%\providecommand{\version}{final}
%    \end{macrocode}

% Include the main document:
%    \begin{macrocode}
\input{childdoc.def}
\childdocby{cdocsamp}
%    \end{macrocode}

%\iffalse
%</samplepart3|samplepart4>
%\fi
%
%\iffalse
%<*samplepart3>
%\fi
% Some text for part 3:
%    \begin{macrocode}
some text in part three
%    \end{macrocode}

%\iffalse
%</samplepart3>
%\fi
% Some text for part 4:
%\iffalse
%<*samplepart4>
%\fi
%    \begin{macrocode}
more text in part four
%    \end{macrocode}

%\iffalse
%</samplepart4>
%\fi
%
% %%%%%%%%%%%%%%%%%%%%%%%%%%%%%%%%%%%%%%
% \paragraph{Forwarding for a Complete Draft.}
%
% The following forwarding file |cdocsdrf.tex|
% compiles the main document in draft mode:
%\iffalse
%<*sampledraft>
%\fi
%    \begin{macrocode}
\def\version{draft}
\input{childdoc.def}
\childdocforward{cdocsamp}
%    \end{macrocode}

%\iffalse
%</sampledraft>
%\fi
%
% %%%%%%%%%%%%%%%%%%%%%%%%%%%%%%%%%%%%%%
% \paragraph{Forwarding for Final Version of the Chapters.}
%
% The following forwarding files |cdocsfn1.tex| and |cdocsfn2.tex|
% (with identical content)
% compile the final versions of the child documents
% |cdocsch1.tex| and |cdocsch2.tex|, respectively:
%\iffalse
%<*samplefinal>
%\fi
%    \begin{macrocode}
\def\version{final}
\input{childdoc.def}
\childdocforwardprefix[cdocsamp]{cdocsfn}{cdocsch}
%    \end{macrocode}

%\iffalse
%</samplefinal>
%\fi
%
% %%%%%%%%%%%%%%%%%%%%%%%%%%%%%%%%%%%%%%
% \paragraph{Command Line Processing.}
%
% The following three command lines generate the output files
% |cdocscld|, |cdocscl1| and |cdocscl2|
% which should be identical to
% |cdocsdrf|, |cdocsch1| and |cdocsfn2|, respectively:
% \begin{center}
% \begin{tabular}{l}
% |latex -jobname cdocscld \|\\
% |  "\def\version{draft}\input{childdoc.def}\childdocforward{cdocsamp}"|\\
% |latex -jobname cdocscl1 \|\\
% |  "\input{childdoc.def}\childdocforward[cdocsamp]{cdocsch1}"|\\
% |latex -jobname cdocscl2 \|\\
% |  "\def\version{final}\input{childdoc.def}\childdocforward{cdocsch2}"|
% \end{tabular}
% \end{center}
% Note that the trailing backslash on each first line
% merely continues the input to the second line
% (for convenient cut ant paste).
% Furthermore, the command |latex| can be replaced by any
% of its alternative versions such as |pdflatex|.
%
% %%%%%%%%%%%%%%%%%%%%%%%%%%%%%%%%%%%%%%%%%%%%%%%%%%%%%%%%%%%%%%%%%%%%%%%%%%%%%%
% %%%%%%%%%%%%%%%%%%%%%%%%%%%%%%%%%%%%%%%%%%%%%%%%%%%%%%%%%%%%%%%%%%%%%%%%%%%%%%
% \section{Implementation}
%\iffalse
%<*package>
%\fi
%
% This section describes the definitions file |childdoc.def|.

% The definitions cannot be loaded using |\usepackage| or |\RequirePackage|
% which has a mechanism to prevent loading a style file more than once.
% When loading the definitions by means of |\input|
% multiple instances have to be prevented manually:
%\iffalse
%This code needs to be before the `\ProvidesFile' directive
%which is defined at the beginning of this file.
%Therefore it is also placed there and commented out here.
%</package>
%<*discard>
%\fi
%    \begin{macrocode}
\ifdefined\childdocmain\endinput\fi
%    \end{macrocode}
%\iffalse
%</discard>
%<*package>
%\fi
%
% \macro{\ifchilddoc}
% \macro{\ifchilddocmanual}
% The conditional |\ifchilddoc| tells whether a
% child (true) or main (false) document is being compiled.
% The conditional |\ifchilddocmanual| tells whether
% the |\includeonly| mechanism is used (false) or
% the selection of child files must be performed manually (true).
% The definitions initialise to false:
%    \begin{macrocode}
\newif\ifchilddoc
\newif\ifchilddocmanual
%    \end{macrocode}

% \macro{\childdocname}
% \macro{\childdocjob}
% The macro |\childdocname| stores the name of the main document
% to be compiled. The macro |\childdocjob| stores the name of
% the document on which the \LaTeX{} compiler was originally invoked.
% The content of |\jobname| cannot be compared
% to filenames specified in the source due to different catcodes.
% The following code rescans |\jobname|, stores the result
% in |\childdocname| and saves a copy in |\childdocjob|:
%    \begin{macrocode}
\edef\childdocname{\scantokens\expandafter{\jobname\noexpand}}
\let\childdocjob\childdocname
%    \end{macrocode}

% \macro{\childdocdisable}
% The macro |\childdocdisable| prevents the main file
% from being processed more than once.
% At this stage, the main document command |\childdocmain|
% is assumed to be called once again where it should do nothing.
% Any subsequent call to it should prevent
% a secondary processing of the main document
% It overwrites the forwarding commands
% |\childdocof| and |\childdocforward|
% with empty macros to prevent further inclusions of the main document:
%    \begin{macrocode}
\newcommand{\childdocdisable}
{
  \renewcommand{\childdocmain}[1]{\renewcommand{\childdocmain}[1]{\endinput}}
  \renewcommand{\childdocof}[1]{}
  \renewcommand{\childdocby}[2][]{}
  \renewcommand{\childdocforward}[2][]{}
  \renewcommand{\childdocdisable}{}
}
%    \end{macrocode}

% \macro{\childdocmain}
% The macro |\childdocmain| is to be called at the top of the main file
% with nothing or the main filename (without extension) as argument.
% First, it breaks loops.
% If the argument is not empty and does not match |\childdocname|
% (which is set by the first inclusion of |childdoc.def|),
% |\ifchilddoc| is set to true, |\includeonly| is applied to the child file
% and |\jobname| is set to the main file
% (for proper handling of |.aux| files):
%    \begin{macrocode}
\newcommand{\childdocmain}[1]
{
  \childdocdisable\childdocmain{}
  \if?#1?\else
    \begingroup
      \def\childdoctmp{#1}
      \ifx\childdoctmp\childdocname
        \def\childdoctmp{}
      \else
        \def\childdoctmp
        {
          \childdoctrue
          \includeonly{\childdocname}
          \def\childdocjob{#1}
          \def\jobname{#1}
        }
      \fi
      \expandafter
    \endgroup
    \childdoctmp
  \fi
}
%    \end{macrocode}

% \macro{\childdocof}
% The command |\childdocof| redirects
% compilation to the main file |#1|.
%    \begin{macrocode}
\newcommand{\childdocof}[1]
{
  \childdocdisable
  \childdoctrue
  \includeonly{\childdocname}
  \def\jobname{#1}
  \def\childdocjob{#1}
  \input{#1}
}
%    \end{macrocode}

% \macro{\childdocby}
% The command |\childdocby| ....
%    \begin{macrocode}
\newcommand{\childdocby}[2][]
{
  \childdocdisable
  \childdoctrue
  \childdocmanualtrue
  \if?#1?\else
    \def\jobname{#2}
  \fi
  \def\childdocjob{#2}
  \input{#2}
  \endinput
}
%    \end{macrocode}

% \macro{\childdocforward}
% The command |\childdocforward| redirects
% compilation to the main file or
% (if the optional argument is given) a child file.
% Parameters are set as if the main file
% or a child file starting with |\childdocof| was compiled.
% Then compilation is handed over to the main file:
%    \begin{macrocode}
\newcommand{\childdocforward}[2][]
{
  \begingroup
    \if?#1?
      \def\childdoctmp
      {
        \def\childdocname{#2}
        \def\childdocjob{#2}
        \def\jobname{#2}
        \input{#2}
        \endinput
      }
    \else
      \def\childdoctmp
      {
        \childdocdisable
        \def\childdocname{#2}
        \childdoctrue
        \includeonly{#2}
        \def\childdocjob{#1}
        \def\jobname{#1}
        \input{#1}
        \endinput
      }
    \fi
    \expandafter
  \endgroup
  \childdoctmp
}
%    \end{macrocode}

% \macro{\childdocforwardprefix}
% The command |\childdocforwardprefix| redirects
% compilation to the main or a child file by means of a pattern.
% The prefix |#1| in the current filename is replaced by |#2|
% and the suffix of the current filename is kept
% (it is assumed that the filename does not contain the substring `|~~~|'
% which is used as a delimiter).
% Compilation is handed over to the new file by |\childdocforward|:
%    \begin{macrocode}
\newcommand{\childdocforwardprefix}[3][]
{
  \begingroup
    \def\childdocextract #2##1~~~{\def\childdoctmp{\childdocforward[#1]{#3##1}}}
    \expandafter\childdocextract\childdocname~~~
    \expandafter
  \endgroup
  \childdoctmp
}
%    \end{macrocode}

% \macro{\childdoc}
% The deprecated macro |\childdoc| is a legacy version of |\childdocmain|:
%    \begin{macrocode}
\newcommand{\childdoc}{\childdocmain}
%    \end{macrocode}

% \macro{\childdocredirect}
% The deprecated macro |\childdocredirect| is a legacy version
% of |\childdocforward| and |\childdocforwardprefix|:
%    \begin{macrocode}
\newcommand{\childdocredirect}[2][]
{
  \begingroup
    \if?#1?
      \def\childdoctmp{\childdocforward{#2}}
    \else
      \def\childdoctmp{\childdocforwardprefix{#1}{#2}}
    \fi
    \expandafter
  \endgroup
  \childdoctmp
}
%    \end{macrocode}

%\iffalse
%</package>
%\fi
%
\endinput
|\\
|\childdocmain{}|\\
\end{tabular}
\end{center}
at the very top of the main \LaTeX{} file,
in particular \emph{before} the |\documentclass| statement!
The argument of |\childdocmain| should be left empty
(but it must be present).

%%%%%%%%%%%%%%%%%%%%%%%%%%%%%%%%%%%%%%%%
\DescribeMacro{\childdocof}
Furthermore, add the commands
\begin{center}
\begin{tabular}{l}
|% \iffalse
%
% childdoc.dtx Copyright (C) 2017-2018 Niklas Beisert
%
% This work may be distributed and/or modified under the
% conditions of the LaTeX Project Public License, either version 1.3
% of this license or (at your option) any later version.
% The latest version of this license is in
%   http://www.latex-project.org/lppl.txt
% and version 1.3 or later is part of all distributions of LaTeX
% version 2005/12/01 or later.
%
% This work has the LPPL maintenance status `maintained'.
%
% The Current Maintainer of this work is Niklas Beisert.
%
% This work consists of the files childdoc.dtx and childdoc.ins
% and the derived files childdoc.def and cdocsamp.tex with
% cdocsch1.tex, cdocsch2.tex, cdocsdrf.tex, cdocsfn1.tex, cdocsfn2.tex.
%
%<package>\ifdefined\childdocmain\endinput\fi
%<package>\ProvidesFile{childdoc.def}[2018/12/30 v2.0 child document driver]
%<samplemain>\ProvidesFile{cdocsamp.tex}[2018/12/30 v2.0 sample for childdoc]
%<*driver>
%\ProvidesFile{childdoc.drv}[2018/12/30 v2.0 childdoc reference manual file]
\PassOptionsToClass{10pt,a4paper}{article}
\documentclass{ltxdoc}

\usepackage[margin=35mm]{geometry}
\usepackage{hyperref}
\usepackage{hyperxmp}
\usepackage[usenames]{color}

\hypersetup{colorlinks=true}
\hypersetup{pdfstartview=FitH}
\hypersetup{pdfpagemode=UseNone}
\hypersetup{pdfsource={}}
\hypersetup{pdflang={en-UK}}
\hypersetup{pdfcopyright={Copyright 2017-2018 Niklas Beisert.
  This work may be distributed and/or modified under the
  conditions of the LaTeX Project Public License, either version 1.3
  of this license or (at your option) any later version.}}
\hypersetup{pdflicenseurl={http://www.latex-project.org/lppl.txt}}
\hypersetup{pdfcontactaddress={ETH Zurich, ITP, HIT K,
  Wolfgang-Pauli-Strasse 27}}
\hypersetup{pdfcontactpostcode={8093}}
\hypersetup{pdfcontactcity={Zurich}}
\hypersetup{pdfcontactcountry={Switzerland}}
\hypersetup{pdfcontactemail={nbeisert@itp.phys.ethz.ch}}
\hypersetup{pdfcontacturl={http://people.phys.ethz.ch/\xmptilde nbeisert/}}

\newcommand{\secref}[1]{\hyperref[#1]{section \ref*{#1}}}

\parskip1ex
\parindent0pt
\let\olditemize\itemize
\def\itemize{\olditemize\parskip0pt}

\begin{document}

\title{The \textsf{childdoc} Package}
\hypersetup{pdftitle={The childdoc Package}}
\author{Niklas Beisert\\[2ex]
  Institut f\"ur Theoretische Physik\\
  Eidgen\"ossische Technische Hochschule Z\"urich\\
  Wolfgang-Pauli-Strasse 27, 8093 Z\"urich, Switzerland\\[1ex]
  \href{mailto:nbeisert@itp.phys.ethz.ch}
  {\texttt{nbeisert@itp.phys.ethz.ch}}}
\hypersetup{pdfauthor={Niklas Beisert}}
\hypersetup{pdfsubject={Manual for the LaTeX2e Package childdoc}}
\date{30 December 2018, \textsf{v2.0}}
\maketitle

\begin{abstract}\noindent
\textsf{childdoc} is a \LaTeXe{} package
that enables the direct compilation
of document sections included by |\include|
to individual files.
\end{abstract}

\begingroup
\parskip0ex
\tableofcontents
\endgroup

%%%%%%%%%%%%%%%%%%%%%%%%%%%%%%%%%%%%%%%%%%%%%%%%%%%%%%%%%%%%%%%%%%%%%%%%%%%%%%%%
%%%%%%%%%%%%%%%%%%%%%%%%%%%%%%%%%%%%%%%%%%%%%%%%%%%%%%%%%%%%%%%%%%%%%%%%%%%%%%%%
\section{Introduction}

\LaTeX{} provides a mechanism to structure a large document (such as a book)
into a main file and several child files (containing the chapters)
using the |\include| command.
This mechanism is beneficial for documents
which span hundreds of pages in order to
make the source file(s) more manageable.
Moreover, compilation can be restricted to
selected child files by means of the |\includeonly| command.
The latter feature can be used to reduce the compilation time while editing
(this was significantly more useful in the earlier days of \LaTeX{})
or to generate a smaller document which is easier to navigate.
Another application of |\includeonly| is to generate
documents consisting of selected parts of the complete document.

However, there are a few drawbacks of the plain |\include| mechanism:
\begin{itemize}
\item
The child files cannot be compiled on their own,
they can only be compiled via the main file.
A naive editing environment
(such as a text editor with an option
to have the current file processed by \LaTeX)
may require one to switch to the main file before compiling;
attempting to compile the child file produces errors.
\item
The main file must be modified (each time)
to adjust the |\includeonly| command
to the present needs. This easily leaves the main file in a messy state.
\item
The generated document will always carry the filename
of the main document. This is inconvenient if
several child files are to be compiled and
to be kept for distribution.
\end{itemize}

The present package provides a simple interface
to make child files individually compilable by \LaTeX{}.
Compiling a child file then has the same effect as compiling
the main file with an |\includeonly| command
to select the appropriate child.
Moreover the generated document will carry the name of the child
rather than the main file.
This resolves all three above issues.

This feature is meant to make the editing of books,
thesis documents and lecture notes somewhat more convenient.
However, the package can also be used efficiently for
composing a series of documents (such as exercise sheets)
which are typically distributed individually.
It then assists the author in generating the individual documents
(potentially in different versions)
as well as a document containing the collected series.
Another application is in developing style files
or other kinds of included material
where compilation of the style file could redirect
to a sample or test file.

%%%%%%%%%%%%%%%%%%%%%%%%%%%%%%%%%%%%%%%%%%%%%%%%%%%%%%%%%%%%%%%%%%%%%%%%%%%%%%%%
%%%%%%%%%%%%%%%%%%%%%%%%%%%%%%%%%%%%%%%%%%%%%%%%%%%%%%%%%%%%%%%%%%%%%%%%%%%%%%%%
\section{Usage}

First of all, the package \textsf{childdoc} is \emph{not} a standard
\LaTeXe{} |.sty| style file! Therefore it needs to be invoked in
a non-standard way.

%%%%%%%%%%%%%%%%%%%%%%%%%%%%%%%%%%%%%%%%%%%%%%%%%%%%%%%%%%%%%%%%%%%%%%%%%%%%%%%%
\subsection{Included Files}
\label{sec:include}

%%%%%%%%%%%%%%%%%%%%%%%%%%%%%%%%%%%%%%%%
\DescribeMacro{\childdocmain}
To use the package, add the commands
\begin{center}
\begin{tabular}{l}
|\input{childdoc.def}|\\
|\childdocmain{}|\\
\end{tabular}
\end{center}
at the very top of the main \LaTeX{} file,
in particular \emph{before} the |\documentclass| statement!
The argument of |\childdocmain| should be left empty
(but it must be present).

%%%%%%%%%%%%%%%%%%%%%%%%%%%%%%%%%%%%%%%%
\DescribeMacro{\childdocof}
Furthermore, add the commands
\begin{center}
\begin{tabular}{l}
|\input{childdoc.def}|\\
|\childdocof{|\textit{main}|}|\\
\end{tabular}
\end{center}
at the top of every child file \textit{child}
which is included by |\include{|\textit{child}|}|
from within the main file
(or at least for those files to be compiled individually).
The argument \textit{main} must be the filename of the main file.

There are a couple of
considerations in setting up the main and child documents:

%%%%%%%%%%%%%%%%%%%%%%%%%%%%%%%%%%%%%%%%
\paragraph{Restrictions.}

Please note the following restrictions:
\begin{itemize}
\item
|\childdocmain| must be called with one argument \textit{main}
to ensure compatibility with earlier version of the package.
It must either be empty (|\childdocmain{}|)
or precisely match the filename of the main file in which it is specified.
See \secref{sec:detection} for further information.
\item
The filename \textit{main} must be specified without the |.tex| extension.
\item
The filename \textit{main} is case sensitive
(even in case-insensitive file systems)
due to internal string comparison.
\item
The argument \textit{main} should be fully expanded, it cannot be a macro.
\item
Subdirectories and special characters should be avoided in filenames.
\item
The command |\childdocmain{|\textit{main}|}| must be followed by a whitespace.
It should not be followed immediately by another command
or by a comment mark `|%|'.
This is because the \TeX{} parser reads the token immediately following
the argument of |\childdocmain| and puts it
at the beginning of every child section;
however, a white\-space is ignored.
\end{itemize}

%%%%%%%%%%%%%%%%%%%%%%%%%%%%%%%%%%%%%%%%
\paragraph{Content of Main File.}

It is advisable to place all content in the child files included by |\include|.
Any output contained in the main file will appear in all child documents
unless suppressed manually;
it cannot be suppressed automatically by the |\includeonly| directive
and thus should normally be avoided.
A method to include some content in the main file
by means of conditional processing is described in \secref{sec:conditional}.

%%%%%%%%%%%%%%%%%%%%%%%%%%%%%%%%%%%%%%%%
\paragraph{Page Numbering.}

When only a part of the document is compiled,
the appropriate numbering of pages
(as well as other status parameters)
is determined from the |.aux| files.
The latter contain information from previous passes.
However this information needs to propagate through
all intermediate child documents.
Therefore the page numbering in child documents may well
be inconsistent until the complete document is compiled at least once.

A useful (if unconventional) way to always ensure a consistent
page numbering is to restart the numbering in each child document
and denote the pages by `\textit{child}|.|\textit{page}'
where \textit{child} represents the chapter/section number of the child file.
This can be achieved by the command
|\numberwithin{page}{|\textit{child}|}|
of the \textsf{amsmath} package
where \textit{child} can be |chapter| or |section|
depending on the chosen structuring.
Alternatively, one can modify the macro |\thepage| appropriately
and reset the counter |page| at the start of each child file.

%%%%%%%%%%%%%%%%%%%%%%%%%%%%%%%%%%%%%%%%%%%%%%%%%%%%%%%%%%%%%%%%%%%%%%%%%%%%%%%%
\subsection{Conditional Processing}
\label{sec:conditional}

The package provides a mechanism to compile different versions
of a document. To customise the versions further some conditional processing
can come in handy to distinguish which version is being compiled.
The package provides two macros to describe the compilation context:

%%%%%%%%%%%%%%%%%%%%%%%%%%%%%%%%%%%%%%%%
\DescribeMacro{\ifchilddoc}
The conditional |\ifchilddoc| distinguishes between the compilation of
child documents and the main document:
%
\begin{center}
|\ifchilddoc |\textit{child-code}| |[|\||else |\textit{main-code}]| \||fi|
\end{center}

%%%%%%%%%%%%%%%%%%%%%%%%%%%%%%%%%%%%%%%%
\DescribeMacro{\childdocname}
\DescribeMacro{\childdocjob}
The macro |\childdocname| contains the filename (without extension)
of the main or child file being processed.
Note that |\childdocjob| will always contain the name of the main file.

%%%%%%%%%%%%%%%%%%%%%%%%%%%%%%%%%%%%%%%%
\paragraph{Title Page.}

Conditional processing can be used to include a title or banner page
in the main document when proper precautions are taken.
Importantly, the code in the main file should ensure that the page counter
(as well as other status parameters which are stored in the |.aux| files)
takes the same value after the conditional processing.
Otherwise the page numbers may take divergent values
depending on which part is compiled.

For example, a title page could be declared by:
%
\begin{center}
\begin{tabular}{l}
|\ifchilddoc\||else|\\
|\addtocounter{page}{-1}|\\
\textit{code for title page}\\
|\newpage|\\
|\||fi|
\end{tabular}
\end{center}
%
A banner page for the child documents can be generated by:
%
\begin{center}
\begin{tabular}{l}
|\ifchilddoc|\\
|\addtocounter{page}{-1}|\\
\textit{code for banner page}\\
|\newpage|\\
|\||fi|
\end{tabular}
\end{center}
%
Here one could write a message such as:
\begin{center}
|This is the part \childdocname{} of \childdocjob{}.|
\end{center}

%%%%%%%%%%%%%%%%%%%%%%%%%%%%%%%%%%%%%%%%%%%%%%%%%%%%%%%%%%%%%%%%%%%%%%%%%%%%%%%%
\subsection{Flags}
\label{sec:flags}

The package makes it easy to generate different versions
of the main or child documents.
To this end compilation flags can be defined
and assigned different default values.
They will be particularly useful in conjunction
with the forwarding mechanism described in \secref{sec:forward}.

For example, it may be useful to have a flag |\version|
which can be set to |draft| or |final|.
The document source will contain some conditional code
depending on the value of |\version|.
Suppose further, the flag should default to |final| for the main file
and to |draft| for child files
which is a natural assignment for editing the document.
This is achieved by placing the following code
in the preamble of the main document
(below the |\childdocmain| directive):
%
\begin{center}
\begin{tabular}{l}
|\ifchilddoc|\\
|\providecommand{\version}{draft}|\\
|\||else|\\
|\providecommand{\version}{final}|\\
|\||fi|
\end{tabular}
\end{center}
%
The definition by |\providecommand| makes sure
that previous definitions are not overwritten.
Further statements |\providecommand{\version}{...}|
can thus be added before the above code to override it.

For the main file, one might add a line
(between |\childdocmain| and the above block)
%
\begin{center}
|%\ifchilddoc\||else\providecommand{\version}{draft}\||fi|
\end{center}
%
which can be uncommented to produce a draft version.
Likewise one can add a line to the very top of a child file
(above the |\childdocof{|\textit{main}|}| directive)
%
\begin{center}
|%\providecommand{\version}{final}|
\end{center}
%
which can be uncommented to produce the final version of this child document.

%%%%%%%%%%%%%%%%%%%%%%%%%%%%%%%%%%%%%%%%%%%%%%%%%%%%%%%%%%%%%%%%%%%%%%%%%%%%%%%%
\subsection{Forwarding}
\label{sec:forward}

Different versions of the main or child documents
using compilation flags as described in \secref{sec:flags}
can be (permanently) stored in different files
for convenient compilation, viewing and distribution.
To this end, the package defines a command
to pass on compilation to a different file:

%%%%%%%%%%%%%%%%%%%%%%%%%%%%%%%%%%%%%%%%
\DescribeMacro{\childdocforward}
The command |\childdocforward| redirects processing to
another source file:
%
\begin{center}
\begin{tabular}{l}
|\input{childdoc.def}|\\
|\childdocforward[|\textit{main}|]{|\textit{dest}|}|\\
\end{tabular}
\end{center}
%
The argument \textit{dest} is the destination file
(without extension).
It should be the main file or one of the child files.
Note that further \textsf{childdoc} directives
such as |\childdocof| and |\childdocforward|
in the indicated file will be processed in this form.
The optional argument \textit{main}
passes on directly to the main file \textit{main}
while pretending to compile the child \textit{dest}.
This form behaves as if \textit{dest}
issues |\childdocof{|\textit{main}|}| right away,
and no further \textsf{childdoc} directives will be processed.

%%%%%%%%%%%%%%%%%%%%%%%%%%%%%%%%%%%%%%%%
\DescribeMacro{\...prefix}
In the alternative form |\childdocforwardprefix|,
%
\begin{center}
\begin{tabular}{l}
|\input{childdoc.def}|\\
|\childdocforwardprefix[|\textit{main}|]{|\textit{prefix}|}{|\textit{dest}|}|
\end{tabular}
\end{center}
%
the destination file is determined by a pattern
depending on the current file:
To make this work, the current file must be called
`{\textit{prefix}\hspace{0.2em}\textit{suffix}}'
with \textit{prefix} matching precisely the argument.
Processing is then passed on to the file
`{\textit{dest}\hspace{0.2em}\textit{suffix}}'.
Surely, the same effect is achieved by
directly specifying the
argument `{\textit{dest}\hspace{0.2em}\textit{suffix}}'
in the first form.
However, that requires to set up a different file
for each child. With the alternative form of the command
all these files can have exactly the same content
which simplifies setting them up and maintaining them.

For example, the following file |draft.tex|
with a compilation flag |\version| as described in \secref{sec:flags}
compiles the main document as a draft:
%
\begin{center}
\begin{tabular}{l}
|\def\version{draft}|\\
|\input{childdoc.def}|\\
|\childdocforward{|\textit{main}|}|
\end{tabular}
\end{center}
%
Likewise, the following files |final|\textit{nn}|.tex|
compile the final version of the child document
|child|\textit{nn}|.tex|:
%
\begin{center}
\begin{tabular}{l}
|\def\version{final}|\\
|\input{childdoc.def}|\\
|\childdocforwardprefix{final}{child}|
\end{tabular}
\end{center}
%

Note that when several versions of a main file and/or of each child file
are to be generated, it may be convenient to set up a |Makefile| or
shell script to automatise the process.

%%%%%%%%%%%%%%%%%%%%%%%%%%%%%%%%%%%%%%%%%%%%%%%%%%%%%%%%%%%%%%%%%%%%%%%%%%%%%%%%
\subsection{Command Line Processing}
\label{sec:commandline}

The effect of redirection files can also be achieved by invoking
the \LaTeX{} compiler with a more elaborate command line.
Most conveniently this should be done as part
of a shell script or a |Makefile|.

When using \textsf{childdoc} in the main file, the following
command lines effectively perform a redirection
(note that depending on the shell being used,
backslashes may have to be doubled: `|\|' $\to$ `|\\|'):
%
\begin{center}
|... -jobname "|\textit{target}|" |\\|"|[\textit{flags}]%
|\input{childdoc.def}\childdocforward[|\textit{main}|]{|\textit{dest}|}"|
\end{center}
%
Here \textit{target} is the name of the output file,
\textit{main} is the name of the main file
and \textit{dest} is the name of the main or child file to be processed
(all filenames without extensions).
The optional argument \textit{main} can be omitted
if \textit{main} matches \textit{dest}.
Optionally, compilation \textit{flags} can be defined via |\def| commands.
This command line makes the \TeX{} engine believe
it is compiling the file \textit{target}
whose content is specified as the latter parameter.
The provided code then forwards the processing to
\textit{main} or \textit{dest} as described in \secref{sec:forward}.

%%%%%%%%%%%%%%%%%%%%%%%%%%%%%%%%%%%%%%%%%%%%%%%%%%%%%%%%%%%%%%%%%%%%%%%%%%%%%%%%
\subsection{Include by Input}
\label{sec:input}

Including child documents by |\include| has some restrictions by design.
Most notably, the content of a child document always occupies
its own set of pages; pages cannot be shared between child documents.
Usually, this behaviour makes perfect sense
because each child document contain an essential part of the document.
However, in some situations it may be desirable to compose
a document from a collection of parts
without having mandatory page breaks between then.
For this case, the package
provides a mechanism to include parts
by |\input| which can also be processed individually.
However, by construction this mechanism
requires manual handling of the content to be output.

%%%%%%%%%%%%%%%%%%%%%%%%%%%%%%%%%%%%%%%%
\DescribeMacro{\ifchilddocmanual}
The main file should be prepared as usual, see \secref{sec:include}.
However, the document body must make a distinction
between processing of an individual part and of the main document, e.g.:
%
\begin{center}
\begin{tabular}{l}
|\ifchilddocmanual|\\
|\input{\childdocname}|\\
|\||else|\\
\textit{document body with }|\input{|\textit{part}|}|\\
|\||fi|
\end{tabular}
\end{center}
%
The conditional |\ifchilddocmanual| is true whenever
a part to be included by |\input| is being compiled,
and the name of the part is stored in |\childdocname|.

%%%%%%%%%%%%%%%%%%%%%%%%%%%%%%%%%%%%%%%%
\DescribeMacro{\childdocby}
Each part to be included by |\input| should start with:
%
\begin{center}
\begin{tabular}{l}
|\input{childdoc.def}|\\
|\childdocby{|\textit{main}|}|\\
\end{tabular}
\end{center}
%
The directive |\childdocby| is similar to |\childdocof|
described in \secref{sec:include},
but the subsequent selection of content must be done manually.
To that end, both |\ifchilddoc| and |\ifchilddocmanual|
will be true upon processing of a part,
and the name of the part is stored in |\childdocname|.
Note that |\jobname| will be set to the filename of the current part
so that each part receives an individual |.aux| file
that does not interfere with the |.aux| file(s) of the main document.
This behaviour can be altered by the alternative form
|\childdocby[*]{|\textit{main}|}| (with a non-empty optional argument)
which uses the |.aux| file of the main document
by setting |\jobname| to \textit{main}.

%%%%%%%%%%%%%%%%%%%%%%%%%%%%%%%%%%%%%%%%%%%%%%%%%%%%%%%%%%%%%%%%%%%%%%%%%%%%%%%%
\subsection{Driver Development}
\label{sec:driver}

The \textsf{childdoc} mechanism can also be use for the development
of definition files such as \LaTeX{} styles or classes.
This case differs from the above setup with multiple parts
included by |\include| in that no |\includeonly| should be invoked.
This can be achieved by starting the include file
(before |\ProvidesPackage|) with:
%
\begin{center}
\begin{tabular}{l}
|\input{childdoc.def}|\\
|\childdocforward{|\textit{main}|}|\\
\end{tabular}
\end{center}
%
or alternatively with:
%
\begin{center}
\begin{tabular}{l}
|\input{childdoc.def}|\\
|\childdocby{|\textit{main}|}|\\
\end{tabular}
\end{center}
%
Both forms have slightly different effects as described above.
The main file is prepared as usual, see \secref{sec:include}.

%%%%%%%%%%%%%%%%%%%%%%%%%%%%%%%%%%%%%%%%%%%%%%%%%%%%%%%%%%%%%%%%%%%%%%%%%%%%%%%%
\subsection{Legacy Detection}
\label{sec:detection}

The directive |\childdocmain| in the main file can detect
whether the complete document or merely a child is to be compiled
even without using the directive |\childdocof|.
This method is deprecated because it is less robust
and there is no compelling reason to use it;
it is merely provided for backward compatibility
and it may be removed in future versions.

If the detection mechanism is to be used,
it is mandatory to correctly specify
the filename of the main file as the argument of |\childdocmain|:
%
\begin{center}
\begin{tabular}{l}
|\input{childdoc.def}|\\
|\childdocmain{|\textit{main}|}|\\
\end{tabular}
\end{center}
%
If |\jobname| does not match the argument \textit{main} of |\childdocmain|,
it is assumed that |\jobname| points to the child file to be compiled.
When using |\childdocmain| with the main file specified as argument,
it suffices to start a child file
with just |\input{|\textit{main}|}|
without loading of the package and using |\childdocof|.
If instead all processing is done
with the appropriate \textsf{childdoc} directives,
the argument of \textit{main} of |\childdocmain| can be empty.

An alternative version of the command line processing described
in \secref{sec:commandline} using the detection mechanism reads:
%
\begin{center}
|... -jobname "|\textit{target}|" "|[\textit{flags}]%
[|\def\jobname{|\textit{dest}|}|]|\input{|\textit{main}|}"|
\end{center}

%%%%%%%%%%%%%%%%%%%%%%%%%%%%%%%%%%%%%%%%%%%%%%%%%%%%%%%%%%%%%%%%%%%%%%%%%%%%%%%%
\subsection{Manual Code}
\label{sec:manual}

In case one cannot be certain whether the definitions file |childdoc.def|
is installed on the target \TeX{} distribution
and one prefers not to ship it,
it is conceivable to paste a few relevant commands into the sources.

To that end, drop all statements |\input{childdoc.def}|
and perform the replacements as outlined below.
Instead of |\childdocmain{|\textit{main}|}| add the following code
to the top of the main file:
%
\begin{center}
\begin{tabular}{l}
|\||ifdefined\childdocname\endinput\||fi\newif\ifchilddoc|\\
|\edef\childdocname{\scantokens\expandafter{\jobname\noexpand}}|\\
|\def\childdocmain{|\textit{main}|}\||ifx\childdocmain\childdocname\||else|\\
|\childdoctrue\includeonly{\childdocname}\let\jobname\childdocmain\||fi|\\
\end{tabular}
\end{center}
%
Instead of |\childdocof{|\textit{main}|}| just include the main file
at the top of each child file:
%
\begin{center}
|\input{|\textit{main}|}|
\end{center}
%
A simple redirection |\childdocforward{|\textit{dest}|}| is achieved by:
%
\begin{center}
|\def\jobname{|\textit{dest}|}\input{\jobname}|
\end{center}
%
The redirection with prefix
|\childdocforwardprefix[|\textit{prefix}|]{|\textit{dest}|}|
is accomplished by:
%
\begin{center}
\begin{tabular}{l}
|{\edef\jobname{\scantokens\expandafter{\jobname\noexpand}}|\\
|\def\redirectjob |\textit{prefix}|#1~~~{\gdef\jobname{|\textit{dest}|#1}}|\\
|\expandafter\redirectjob\jobname~~~}\input{\jobname}|
\end{tabular}
\end{center}

In an alternative approach,
child documents can be compiled by a specific command line
without additional code or specific definitions:
%
\begin{center}
|... -jobname "|\textit{target}|" "|[\textit{flags}]%
|\includeonly{|\textit{dest}|}\input{|\textit{main}|}"|
\end{center}
%

%%%%%%%%%%%%%%%%%%%%%%%%%%%%%%%%%%%%%%%%%%%%%%%%%%%%%%%%%%%%%%%%%%%%%%%%%%%%%%%%
%%%%%%%%%%%%%%%%%%%%%%%%%%%%%%%%%%%%%%%%%%%%%%%%%%%%%%%%%%%%%%%%%%%%%%%%%%%%%%%%
\section{Information}

%%%%%%%%%%%%%%%%%%%%%%%%%%%%%%%%%%%%%%%%%%%%%%%%%%%%%%%%%%%%%%%%%%%%%%%%%%%%%%%%
\subsection{Copyright}

Copyright \copyright{} 2017--2018 Niklas Beisert

This work may be distributed and/or modified under the
conditions of the \LaTeX{} Project Public License, either version 1.3
of this license or (at your option) any later version.
The latest version of this license is in
  \url{http://www.latex-project.org/lppl.txt}
and version 1.3 or later is part of all distributions of \LaTeX{}
version 2005/12/01 or later.

This work has the LPPL maintenance status `maintained'.

The Current Maintainer of this work is Niklas Beisert.

This work consists of the files |README.txt|, |childdoc.ins| and |childdoc.dtx|
as well as the derived files |childdoc.def|, |cdocsamp.tex|
with |cdocsch1.tex|, |cdocsch2.tex|, |cdocspt3.tex|, |cdocspt4.tex|,
|cdocsdrf.tex|, |cdocsfn1.tex|, |cdocsfn2.tex|
as well as |childdoc.pdf|.

%%%%%%%%%%%%%%%%%%%%%%%%%%%%%%%%%%%%%%%%%%%%%%%%%%%%%%%%%%%%%%%%%%%%%%%%%%%%%%%%
\subsection{Files and Installation}

The package consists of the files:
%
\begin{center}
\begin{tabular}{ll}
    |README.txt|   & readme file \\
    |childdoc.ins| & installation file \\
    |childdoc.dtx| & source file \\
    |childdoc.def| & definition file \\
    |cdocsamp.tex| & sample main file \\
    |cdocsch1.tex| & sample include file \\
    |cdocsch2.tex| & sample include file \\
    |cdocspt3.tex| & sample part file \\
    |cdocspt4.tex| & sample part file \\
    |cdocsdrf.tex| & sample redirection file \\
    |cdocsfn1.tex| & sample redirection file \\
    |cdocsfn2.tex| & sample redirection file \\
    |childdoc.pdf| & manual
\end{tabular}
\end{center}
%
The distribution consists of the files
|README.txt|, |childdoc.ins| and |childdoc.dtx|.
%
\begin{itemize}
\item
Run (pdf)\LaTeX{} on |childdoc.dtx|
to compile the manual |childdoc.pdf| (this file).
\item
Run \LaTeX{} on |childdoc.ins| to create the definitions file |childdoc.def|
and the sample |cdocsamp.tex| with include files
|cdocsch1.tex|, |cdocsch2.tex|, |cdocspt3.tex|, |cdocspt4.tex|,
|cdocsdrf.tex|, |cdocsfn1.tex|, |cdocsfn2.tex|.
Then copy the file |childdoc.def| to an appropriate directory of your \LaTeX{}
distribution, e.g.\ \textit{texmf-root}|/tex/latex/childdoc|.
\end{itemize}

%%%%%%%%%%%%%%%%%%%%%%%%%%%%%%%%%%%%%%%%%%%%%%%%%%%%%%%%%%%%%%%%%%%%%%%%%%%%%%%%
\subsection{Related CTAN Packages}

There are several other packages which offer a similar functionality:
%
\begin{itemize}
\item
The packages
\href{http://ctan.org/pkg/docmute}{\textsf{docmute}},
\href{http://ctan.org/pkg/includex}{\textsf{includex}} and
\href{http://ctan.org/pkg/standalone}{\textsf{standalone}}
provide commands to include only the document body of
a child file thus allowing both files to be compiled individually.
\item
The packages \href{http://ctan.org/pkg/subdocs}{\textsf{subdocs}}
and \href{http://ctan.org/pkg/subfiles}{\textsf{subfiles}}
provide structures in which the main and child documents can be
encapsulated and allowing them to be compiled individually.
The inclusion mechanism is different from the conventional |\include|.
\item
The package \href{http://ctan.org/pkg/combine}{\textsf{combine}}
is an elaborate solution to combine several documents into one.
\end{itemize}
%
See also the CTAN topic \href{http://ctan.org/topic/subdocs}{\textsf{subdocs}}
for further related packages.
The present package differs from the above solutions in that
a document structure constructed with the conventional |\include| mechanism
just needs two extra commands at the top of every file
such that all constituent files can be compiled individually.

%%%%%%%%%%%%%%%%%%%%%%%%%%%%%%%%%%%%%%%%%%%%%%%%%%%%%%%%%%%%%%%%%%%%%%%%%%%%%%%%
%\subsection{Feature Suggestions}
%
%The following is a list of features which may be useful for future
%versions of this package:
%%
%\begin{itemize}
%\item
%\ldots
%\end{itemize}

%%%%%%%%%%%%%%%%%%%%%%%%%%%%%%%%%%%%%%%%%%%%%%%%%%%%%%%%%%%%%%%%%%%%%%%%%%%%%%%%
\subsection{Revision History}

%%%%%%%%%%%%%%%%%%%%%%%%%%%%%%%%%%%%%%%%
\paragraph{v2.0:} 2018/12/30

\begin{itemize}
\item
immediate forward processing
\item
added |\childdocby| mechanism
\item
manual restructured
\end{itemize}

%%%%%%%%%%%%%%%%%%%%%%%%%%%%%%%%%%%%%%%%
\paragraph{v1.6:} 2018/01/17

\begin{itemize}
\item
application for development of include files
\item
corrections to manual
\end{itemize}

%%%%%%%%%%%%%%%%%%%%%%%%%%%%%%%%%%%%%%%%
\paragraph{v1.5:} 2017/05/21

\begin{itemize}
\item
more complete structuring introduced
\item
|\childdocof| introduced
\item
|\childdoc| renamed to |\childdocmain|
\item
|\childredirect| renamed to |\childdocforward| and |\childdocforwardprefix|
and functionality expanded
\end{itemize}

%%%%%%%%%%%%%%%%%%%%%%%%%%%%%%%%%%%%%%%%
\paragraph{v1.0:} 2017/04/27

\begin{itemize}
\item
manual and install package
\item
first version published on CTAN
\end{itemize}

%%%%%%%%%%%%%%%%%%%%%%%%%%%%%%%%%%%%%%%%
\paragraph{v0.6:} 2017/04/26

\begin{itemize}
\item
redirection mechanism added
\end{itemize}

%%%%%%%%%%%%%%%%%%%%%%%%%%%%%%%%%%%%%%%%
\paragraph{v0.5:} 2017/04/26

\begin{itemize}
\item
functionality in definition file
\end{itemize}


%%%%%%%%%%%%%%%%%%%%%%%%%%%%%%%%%%%%%%%%%%%%%%%%%%%%%%%%%%%%%%%%%%%%%%%%%%%%%%%%
%%%%%%%%%%%%%%%%%%%%%%%%%%%%%%%%%%%%%%%%%%%%%%%%%%%%%%%%%%%%%%%%%%%%%%%%%%%%%%%%
%%%%%%%%%%%%%%%%%%%%%%%%%%%%%%%%%%%%%%%%%%%%%%%%%%%%%%%%%%%%%%%%%%%%%%%%%%%%%%%%
\appendix

\settowidth\MacroIndent{\rmfamily\scriptsize 000\ }

 \DocInput{childdoc.dtx}

\end{document}
%</driver>
% \fi
%
% %%%%%%%%%%%%%%%%%%%%%%%%%%%%%%%%%%%%%%%%%%%%%%%%%%%%%%%%%%%%%%%%%%%%%%%%%%%%%%
% %%%%%%%%%%%%%%%%%%%%%%%%%%%%%%%%%%%%%%%%%%%%%%%%%%%%%%%%%%%%%%%%%%%%%%%%%%%%%%
% \section{Sample}
%\iffalse
%<*samplemain>
%\fi
%
% The following presents a sample document
% with two chapters, two parts, a title page,
% a compile flag as well as three forwarding files to set the flag.
% It consists of eight |.tex| files:
% \begin{center}
% \begin{tabular}{ll}
% |cdocsamp.tex|&main file\\
% |cdocsch1.tex|&include file for chapter 1\\
% |cdocsch2.tex|&include file for chapter 2\\
% |cdocspt3.tex|&include file for part 3\\
% |cdocspt4.tex|&include file for part 4\\
% |cdocsdrf.tex|&forwarding file for main file in draft mode\\
% |cdocsfi1.tex|&forwarding file for final version of chapter 1\\
% |cdocsfi2.tex|&forwarding file for final version of chapter 2\\
% \end{tabular}
% \end{center}
% Each of the eight files can be compiled directly by the \LaTeX{} compiler.
%
% %%%%%%%%%%%%%%%%%%%%%%%%%%%%%%%%%%%%%%
% \paragraph{Main File.}
%
% The main file is called |cdocsamp.tex|.
%
% Load the \textsf{childdoc} definitions and
% declare the filename for the main document:
%    \begin{macrocode}
\input{childdoc.def}
\childdocmain{}
%    \end{macrocode}

% Optional override for |\version| flag:
%    \begin{macrocode}
%%\ifchilddoc\else\providecommand{\version}{draft}\fi
%    \end{macrocode}

% Define the default values for the |\version| flag
% (|final| for the main file and |draft| for childs):
%    \begin{macrocode}
\ifchilddoc
\providecommand{\version}{draft}
\else
\providecommand{\version}{final}
\fi
%    \end{macrocode}

% Load the standard document class:
%    \begin{macrocode}
\documentclass[12pt]{article}
%    \end{macrocode}

% Start the document body:
%    \begin{macrocode}
\begin{document}
%    \end{macrocode}

% Declare a title page.
% Print title, part of document being processed and version flag:
%    \begin{macrocode}
\addtocounter{page}{-1}
\begin{center}
{\LARGE\bfseries{}childdoc example\par}
\vspace{1cm}
\ifchilddoc
\ifchilddocmanual part\else chapter\fi:
`\childdocname' of `\childdocjob'\par
\else
main document: `\childdocjob'\par
\fi
version: \version\par
\end{center}
\newpage
%    \end{macrocode}

% Manually include selected file,
% otherwise process as usual:
%    \begin{macrocode}
\ifchilddocmanual
\section*{part `\childdocname'}
\input{\childdocname}
\else
%    \end{macrocode}

% Include the two chapters:
%    \begin{macrocode}
\include{cdocsch1}
\include{cdocsch2}
%    \end{macrocode}

% Include the two parts unless only chapters should be displayed:
%    \begin{macrocode}
\ifchilddoc\else
\section{part three}
\input{cdocspt3}
\section{part four}
\input{cdocspt4}
\fi
%    \end{macrocode}

% Process as usual until here:
%    \begin{macrocode}
\fi
%    \end{macrocode}

% End of document body:
%    \begin{macrocode}
\end{document}
%    \end{macrocode}
%\iffalse
%</samplemain>
%\fi
%
% %%%%%%%%%%%%%%%%%%%%%%%%%%%%%%%%%%%%%%
% \paragraph{Chapter Include Files.}
%
% The include files are called |cdocsch1.tex| and |cdocsch2.tex|.
%
%\iffalse
%<*samplechap1|samplechap2>
%\fi

% Optional override for |\version| flag:
%    \begin{macrocode}
%%\providecommand{\version}{final}
%    \end{macrocode}

% Include the main document:
%    \begin{macrocode}
\input{childdoc.def}
\childdocof{cdocsamp}
%    \end{macrocode}

%\iffalse
%</samplechap1|samplechap2>
%\fi
%
%\iffalse
%<*samplechap1>
%\fi
% Some text for chapter 1:
%    \begin{macrocode}
\section{one}
some text in chapter one
%    \end{macrocode}

%\iffalse
%</samplechap1>
%\fi
% Some text for chapter 2:
%\iffalse
%<*samplechap2>
%\fi
%    \begin{macrocode}
\section{two}
more text in chapter two
%    \end{macrocode}

%\iffalse
%</samplechap2>
%\fi
%
% %%%%%%%%%%%%%%%%%%%%%%%%%%%%%%%%%%%%%%
% \paragraph{Part Include Files.}
%
% The include files are called |cdocspt3.tex| and |cdocspt4.tex|.
%
%\iffalse
%<*samplepart3|samplepart4>
%\fi

% Optional override for |\version| flag:
%    \begin{macrocode}
%%\providecommand{\version}{final}
%    \end{macrocode}

% Include the main document:
%    \begin{macrocode}
\input{childdoc.def}
\childdocby{cdocsamp}
%    \end{macrocode}

%\iffalse
%</samplepart3|samplepart4>
%\fi
%
%\iffalse
%<*samplepart3>
%\fi
% Some text for part 3:
%    \begin{macrocode}
some text in part three
%    \end{macrocode}

%\iffalse
%</samplepart3>
%\fi
% Some text for part 4:
%\iffalse
%<*samplepart4>
%\fi
%    \begin{macrocode}
more text in part four
%    \end{macrocode}

%\iffalse
%</samplepart4>
%\fi
%
% %%%%%%%%%%%%%%%%%%%%%%%%%%%%%%%%%%%%%%
% \paragraph{Forwarding for a Complete Draft.}
%
% The following forwarding file |cdocsdrf.tex|
% compiles the main document in draft mode:
%\iffalse
%<*sampledraft>
%\fi
%    \begin{macrocode}
\def\version{draft}
\input{childdoc.def}
\childdocforward{cdocsamp}
%    \end{macrocode}

%\iffalse
%</sampledraft>
%\fi
%
% %%%%%%%%%%%%%%%%%%%%%%%%%%%%%%%%%%%%%%
% \paragraph{Forwarding for Final Version of the Chapters.}
%
% The following forwarding files |cdocsfn1.tex| and |cdocsfn2.tex|
% (with identical content)
% compile the final versions of the child documents
% |cdocsch1.tex| and |cdocsch2.tex|, respectively:
%\iffalse
%<*samplefinal>
%\fi
%    \begin{macrocode}
\def\version{final}
\input{childdoc.def}
\childdocforwardprefix[cdocsamp]{cdocsfn}{cdocsch}
%    \end{macrocode}

%\iffalse
%</samplefinal>
%\fi
%
% %%%%%%%%%%%%%%%%%%%%%%%%%%%%%%%%%%%%%%
% \paragraph{Command Line Processing.}
%
% The following three command lines generate the output files
% |cdocscld|, |cdocscl1| and |cdocscl2|
% which should be identical to
% |cdocsdrf|, |cdocsch1| and |cdocsfn2|, respectively:
% \begin{center}
% \begin{tabular}{l}
% |latex -jobname cdocscld \|\\
% |  "\def\version{draft}\input{childdoc.def}\childdocforward{cdocsamp}"|\\
% |latex -jobname cdocscl1 \|\\
% |  "\input{childdoc.def}\childdocforward[cdocsamp]{cdocsch1}"|\\
% |latex -jobname cdocscl2 \|\\
% |  "\def\version{final}\input{childdoc.def}\childdocforward{cdocsch2}"|
% \end{tabular}
% \end{center}
% Note that the trailing backslash on each first line
% merely continues the input to the second line
% (for convenient cut ant paste).
% Furthermore, the command |latex| can be replaced by any
% of its alternative versions such as |pdflatex|.
%
% %%%%%%%%%%%%%%%%%%%%%%%%%%%%%%%%%%%%%%%%%%%%%%%%%%%%%%%%%%%%%%%%%%%%%%%%%%%%%%
% %%%%%%%%%%%%%%%%%%%%%%%%%%%%%%%%%%%%%%%%%%%%%%%%%%%%%%%%%%%%%%%%%%%%%%%%%%%%%%
% \section{Implementation}
%\iffalse
%<*package>
%\fi
%
% This section describes the definitions file |childdoc.def|.

% The definitions cannot be loaded using |\usepackage| or |\RequirePackage|
% which has a mechanism to prevent loading a style file more than once.
% When loading the definitions by means of |\input|
% multiple instances have to be prevented manually:
%\iffalse
%This code needs to be before the `\ProvidesFile' directive
%which is defined at the beginning of this file.
%Therefore it is also placed there and commented out here.
%</package>
%<*discard>
%\fi
%    \begin{macrocode}
\ifdefined\childdocmain\endinput\fi
%    \end{macrocode}
%\iffalse
%</discard>
%<*package>
%\fi
%
% \macro{\ifchilddoc}
% \macro{\ifchilddocmanual}
% The conditional |\ifchilddoc| tells whether a
% child (true) or main (false) document is being compiled.
% The conditional |\ifchilddocmanual| tells whether
% the |\includeonly| mechanism is used (false) or
% the selection of child files must be performed manually (true).
% The definitions initialise to false:
%    \begin{macrocode}
\newif\ifchilddoc
\newif\ifchilddocmanual
%    \end{macrocode}

% \macro{\childdocname}
% \macro{\childdocjob}
% The macro |\childdocname| stores the name of the main document
% to be compiled. The macro |\childdocjob| stores the name of
% the document on which the \LaTeX{} compiler was originally invoked.
% The content of |\jobname| cannot be compared
% to filenames specified in the source due to different catcodes.
% The following code rescans |\jobname|, stores the result
% in |\childdocname| and saves a copy in |\childdocjob|:
%    \begin{macrocode}
\edef\childdocname{\scantokens\expandafter{\jobname\noexpand}}
\let\childdocjob\childdocname
%    \end{macrocode}

% \macro{\childdocdisable}
% The macro |\childdocdisable| prevents the main file
% from being processed more than once.
% At this stage, the main document command |\childdocmain|
% is assumed to be called once again where it should do nothing.
% Any subsequent call to it should prevent
% a secondary processing of the main document
% It overwrites the forwarding commands
% |\childdocof| and |\childdocforward|
% with empty macros to prevent further inclusions of the main document:
%    \begin{macrocode}
\newcommand{\childdocdisable}
{
  \renewcommand{\childdocmain}[1]{\renewcommand{\childdocmain}[1]{\endinput}}
  \renewcommand{\childdocof}[1]{}
  \renewcommand{\childdocby}[2][]{}
  \renewcommand{\childdocforward}[2][]{}
  \renewcommand{\childdocdisable}{}
}
%    \end{macrocode}

% \macro{\childdocmain}
% The macro |\childdocmain| is to be called at the top of the main file
% with nothing or the main filename (without extension) as argument.
% First, it breaks loops.
% If the argument is not empty and does not match |\childdocname|
% (which is set by the first inclusion of |childdoc.def|),
% |\ifchilddoc| is set to true, |\includeonly| is applied to the child file
% and |\jobname| is set to the main file
% (for proper handling of |.aux| files):
%    \begin{macrocode}
\newcommand{\childdocmain}[1]
{
  \childdocdisable\childdocmain{}
  \if?#1?\else
    \begingroup
      \def\childdoctmp{#1}
      \ifx\childdoctmp\childdocname
        \def\childdoctmp{}
      \else
        \def\childdoctmp
        {
          \childdoctrue
          \includeonly{\childdocname}
          \def\childdocjob{#1}
          \def\jobname{#1}
        }
      \fi
      \expandafter
    \endgroup
    \childdoctmp
  \fi
}
%    \end{macrocode}

% \macro{\childdocof}
% The command |\childdocof| redirects
% compilation to the main file |#1|.
%    \begin{macrocode}
\newcommand{\childdocof}[1]
{
  \childdocdisable
  \childdoctrue
  \includeonly{\childdocname}
  \def\jobname{#1}
  \def\childdocjob{#1}
  \input{#1}
}
%    \end{macrocode}

% \macro{\childdocby}
% The command |\childdocby| ....
%    \begin{macrocode}
\newcommand{\childdocby}[2][]
{
  \childdocdisable
  \childdoctrue
  \childdocmanualtrue
  \if?#1?\else
    \def\jobname{#2}
  \fi
  \def\childdocjob{#2}
  \input{#2}
  \endinput
}
%    \end{macrocode}

% \macro{\childdocforward}
% The command |\childdocforward| redirects
% compilation to the main file or
% (if the optional argument is given) a child file.
% Parameters are set as if the main file
% or a child file starting with |\childdocof| was compiled.
% Then compilation is handed over to the main file:
%    \begin{macrocode}
\newcommand{\childdocforward}[2][]
{
  \begingroup
    \if?#1?
      \def\childdoctmp
      {
        \def\childdocname{#2}
        \def\childdocjob{#2}
        \def\jobname{#2}
        \input{#2}
        \endinput
      }
    \else
      \def\childdoctmp
      {
        \childdocdisable
        \def\childdocname{#2}
        \childdoctrue
        \includeonly{#2}
        \def\childdocjob{#1}
        \def\jobname{#1}
        \input{#1}
        \endinput
      }
    \fi
    \expandafter
  \endgroup
  \childdoctmp
}
%    \end{macrocode}

% \macro{\childdocforwardprefix}
% The command |\childdocforwardprefix| redirects
% compilation to the main or a child file by means of a pattern.
% The prefix |#1| in the current filename is replaced by |#2|
% and the suffix of the current filename is kept
% (it is assumed that the filename does not contain the substring `|~~~|'
% which is used as a delimiter).
% Compilation is handed over to the new file by |\childdocforward|:
%    \begin{macrocode}
\newcommand{\childdocforwardprefix}[3][]
{
  \begingroup
    \def\childdocextract #2##1~~~{\def\childdoctmp{\childdocforward[#1]{#3##1}}}
    \expandafter\childdocextract\childdocname~~~
    \expandafter
  \endgroup
  \childdoctmp
}
%    \end{macrocode}

% \macro{\childdoc}
% The deprecated macro |\childdoc| is a legacy version of |\childdocmain|:
%    \begin{macrocode}
\newcommand{\childdoc}{\childdocmain}
%    \end{macrocode}

% \macro{\childdocredirect}
% The deprecated macro |\childdocredirect| is a legacy version
% of |\childdocforward| and |\childdocforwardprefix|:
%    \begin{macrocode}
\newcommand{\childdocredirect}[2][]
{
  \begingroup
    \if?#1?
      \def\childdoctmp{\childdocforward{#2}}
    \else
      \def\childdoctmp{\childdocforwardprefix{#1}{#2}}
    \fi
    \expandafter
  \endgroup
  \childdoctmp
}
%    \end{macrocode}

%\iffalse
%</package>
%\fi
%
\endinput
|\\
|\childdocof{|\textit{main}|}|\\
\end{tabular}
\end{center}
at the top of every child file \textit{child}
which is included by |\include{|\textit{child}|}|
from within the main file
(or at least for those files to be compiled individually).
The argument \textit{main} must be the filename of the main file.

There are a couple of
considerations in setting up the main and child documents:

%%%%%%%%%%%%%%%%%%%%%%%%%%%%%%%%%%%%%%%%
\paragraph{Restrictions.}

Please note the following restrictions:
\begin{itemize}
\item
|\childdocmain| must be called with one argument \textit{main}
to ensure compatibility with earlier version of the package.
It must either be empty (|\childdocmain{}|)
or precisely match the filename of the main file in which it is specified.
See \secref{sec:detection} for further information.
\item
The filename \textit{main} must be specified without the |.tex| extension.
\item
The filename \textit{main} is case sensitive
(even in case-insensitive file systems)
due to internal string comparison.
\item
The argument \textit{main} should be fully expanded, it cannot be a macro.
\item
Subdirectories and special characters should be avoided in filenames.
\item
The command |\childdocmain{|\textit{main}|}| must be followed by a whitespace.
It should not be followed immediately by another command
or by a comment mark `|%|'.
This is because the \TeX{} parser reads the token immediately following
the argument of |\childdocmain| and puts it
at the beginning of every child section;
however, a white\-space is ignored.
\end{itemize}

%%%%%%%%%%%%%%%%%%%%%%%%%%%%%%%%%%%%%%%%
\paragraph{Content of Main File.}

It is advisable to place all content in the child files included by |\include|.
Any output contained in the main file will appear in all child documents
unless suppressed manually;
it cannot be suppressed automatically by the |\includeonly| directive
and thus should normally be avoided.
A method to include some content in the main file
by means of conditional processing is described in \secref{sec:conditional}.

%%%%%%%%%%%%%%%%%%%%%%%%%%%%%%%%%%%%%%%%
\paragraph{Page Numbering.}

When only a part of the document is compiled,
the appropriate numbering of pages
(as well as other status parameters)
is determined from the |.aux| files.
The latter contain information from previous passes.
However this information needs to propagate through
all intermediate child documents.
Therefore the page numbering in child documents may well
be inconsistent until the complete document is compiled at least once.

A useful (if unconventional) way to always ensure a consistent
page numbering is to restart the numbering in each child document
and denote the pages by `\textit{child}|.|\textit{page}'
where \textit{child} represents the chapter/section number of the child file.
This can be achieved by the command
|\numberwithin{page}{|\textit{child}|}|
of the \textsf{amsmath} package
where \textit{child} can be |chapter| or |section|
depending on the chosen structuring.
Alternatively, one can modify the macro |\thepage| appropriately
and reset the counter |page| at the start of each child file.

%%%%%%%%%%%%%%%%%%%%%%%%%%%%%%%%%%%%%%%%%%%%%%%%%%%%%%%%%%%%%%%%%%%%%%%%%%%%%%%%
\subsection{Conditional Processing}
\label{sec:conditional}

The package provides a mechanism to compile different versions
of a document. To customise the versions further some conditional processing
can come in handy to distinguish which version is being compiled.
The package provides two macros to describe the compilation context:

%%%%%%%%%%%%%%%%%%%%%%%%%%%%%%%%%%%%%%%%
\DescribeMacro{\ifchilddoc}
The conditional |\ifchilddoc| distinguishes between the compilation of
child documents and the main document:
%
\begin{center}
|\ifchilddoc |\textit{child-code}| |[|\||else |\textit{main-code}]| \||fi|
\end{center}

%%%%%%%%%%%%%%%%%%%%%%%%%%%%%%%%%%%%%%%%
\DescribeMacro{\childdocname}
\DescribeMacro{\childdocjob}
The macro |\childdocname| contains the filename (without extension)
of the main or child file being processed.
Note that |\childdocjob| will always contain the name of the main file.

%%%%%%%%%%%%%%%%%%%%%%%%%%%%%%%%%%%%%%%%
\paragraph{Title Page.}

Conditional processing can be used to include a title or banner page
in the main document when proper precautions are taken.
Importantly, the code in the main file should ensure that the page counter
(as well as other status parameters which are stored in the |.aux| files)
takes the same value after the conditional processing.
Otherwise the page numbers may take divergent values
depending on which part is compiled.

For example, a title page could be declared by:
%
\begin{center}
\begin{tabular}{l}
|\ifchilddoc\||else|\\
|\addtocounter{page}{-1}|\\
\textit{code for title page}\\
|\newpage|\\
|\||fi|
\end{tabular}
\end{center}
%
A banner page for the child documents can be generated by:
%
\begin{center}
\begin{tabular}{l}
|\ifchilddoc|\\
|\addtocounter{page}{-1}|\\
\textit{code for banner page}\\
|\newpage|\\
|\||fi|
\end{tabular}
\end{center}
%
Here one could write a message such as:
\begin{center}
|This is the part \childdocname{} of \childdocjob{}.|
\end{center}

%%%%%%%%%%%%%%%%%%%%%%%%%%%%%%%%%%%%%%%%%%%%%%%%%%%%%%%%%%%%%%%%%%%%%%%%%%%%%%%%
\subsection{Flags}
\label{sec:flags}

The package makes it easy to generate different versions
of the main or child documents.
To this end compilation flags can be defined
and assigned different default values.
They will be particularly useful in conjunction
with the forwarding mechanism described in \secref{sec:forward}.

For example, it may be useful to have a flag |\version|
which can be set to |draft| or |final|.
The document source will contain some conditional code
depending on the value of |\version|.
Suppose further, the flag should default to |final| for the main file
and to |draft| for child files
which is a natural assignment for editing the document.
This is achieved by placing the following code
in the preamble of the main document
(below the |\childdocmain| directive):
%
\begin{center}
\begin{tabular}{l}
|\ifchilddoc|\\
|\providecommand{\version}{draft}|\\
|\||else|\\
|\providecommand{\version}{final}|\\
|\||fi|
\end{tabular}
\end{center}
%
The definition by |\providecommand| makes sure
that previous definitions are not overwritten.
Further statements |\providecommand{\version}{...}|
can thus be added before the above code to override it.

For the main file, one might add a line
(between |\childdocmain| and the above block)
%
\begin{center}
|%\ifchilddoc\||else\providecommand{\version}{draft}\||fi|
\end{center}
%
which can be uncommented to produce a draft version.
Likewise one can add a line to the very top of a child file
(above the |\childdocof{|\textit{main}|}| directive)
%
\begin{center}
|%\providecommand{\version}{final}|
\end{center}
%
which can be uncommented to produce the final version of this child document.

%%%%%%%%%%%%%%%%%%%%%%%%%%%%%%%%%%%%%%%%%%%%%%%%%%%%%%%%%%%%%%%%%%%%%%%%%%%%%%%%
\subsection{Forwarding}
\label{sec:forward}

Different versions of the main or child documents
using compilation flags as described in \secref{sec:flags}
can be (permanently) stored in different files
for convenient compilation, viewing and distribution.
To this end, the package defines a command
to pass on compilation to a different file:

%%%%%%%%%%%%%%%%%%%%%%%%%%%%%%%%%%%%%%%%
\DescribeMacro{\childdocforward}
The command |\childdocforward| redirects processing to
another source file:
%
\begin{center}
\begin{tabular}{l}
|% \iffalse
%
% childdoc.dtx Copyright (C) 2017-2018 Niklas Beisert
%
% This work may be distributed and/or modified under the
% conditions of the LaTeX Project Public License, either version 1.3
% of this license or (at your option) any later version.
% The latest version of this license is in
%   http://www.latex-project.org/lppl.txt
% and version 1.3 or later is part of all distributions of LaTeX
% version 2005/12/01 or later.
%
% This work has the LPPL maintenance status `maintained'.
%
% The Current Maintainer of this work is Niklas Beisert.
%
% This work consists of the files childdoc.dtx and childdoc.ins
% and the derived files childdoc.def and cdocsamp.tex with
% cdocsch1.tex, cdocsch2.tex, cdocsdrf.tex, cdocsfn1.tex, cdocsfn2.tex.
%
%<package>\ifdefined\childdocmain\endinput\fi
%<package>\ProvidesFile{childdoc.def}[2018/12/30 v2.0 child document driver]
%<samplemain>\ProvidesFile{cdocsamp.tex}[2018/12/30 v2.0 sample for childdoc]
%<*driver>
%\ProvidesFile{childdoc.drv}[2018/12/30 v2.0 childdoc reference manual file]
\PassOptionsToClass{10pt,a4paper}{article}
\documentclass{ltxdoc}

\usepackage[margin=35mm]{geometry}
\usepackage{hyperref}
\usepackage{hyperxmp}
\usepackage[usenames]{color}

\hypersetup{colorlinks=true}
\hypersetup{pdfstartview=FitH}
\hypersetup{pdfpagemode=UseNone}
\hypersetup{pdfsource={}}
\hypersetup{pdflang={en-UK}}
\hypersetup{pdfcopyright={Copyright 2017-2018 Niklas Beisert.
  This work may be distributed and/or modified under the
  conditions of the LaTeX Project Public License, either version 1.3
  of this license or (at your option) any later version.}}
\hypersetup{pdflicenseurl={http://www.latex-project.org/lppl.txt}}
\hypersetup{pdfcontactaddress={ETH Zurich, ITP, HIT K,
  Wolfgang-Pauli-Strasse 27}}
\hypersetup{pdfcontactpostcode={8093}}
\hypersetup{pdfcontactcity={Zurich}}
\hypersetup{pdfcontactcountry={Switzerland}}
\hypersetup{pdfcontactemail={nbeisert@itp.phys.ethz.ch}}
\hypersetup{pdfcontacturl={http://people.phys.ethz.ch/\xmptilde nbeisert/}}

\newcommand{\secref}[1]{\hyperref[#1]{section \ref*{#1}}}

\parskip1ex
\parindent0pt
\let\olditemize\itemize
\def\itemize{\olditemize\parskip0pt}

\begin{document}

\title{The \textsf{childdoc} Package}
\hypersetup{pdftitle={The childdoc Package}}
\author{Niklas Beisert\\[2ex]
  Institut f\"ur Theoretische Physik\\
  Eidgen\"ossische Technische Hochschule Z\"urich\\
  Wolfgang-Pauli-Strasse 27, 8093 Z\"urich, Switzerland\\[1ex]
  \href{mailto:nbeisert@itp.phys.ethz.ch}
  {\texttt{nbeisert@itp.phys.ethz.ch}}}
\hypersetup{pdfauthor={Niklas Beisert}}
\hypersetup{pdfsubject={Manual for the LaTeX2e Package childdoc}}
\date{30 December 2018, \textsf{v2.0}}
\maketitle

\begin{abstract}\noindent
\textsf{childdoc} is a \LaTeXe{} package
that enables the direct compilation
of document sections included by |\include|
to individual files.
\end{abstract}

\begingroup
\parskip0ex
\tableofcontents
\endgroup

%%%%%%%%%%%%%%%%%%%%%%%%%%%%%%%%%%%%%%%%%%%%%%%%%%%%%%%%%%%%%%%%%%%%%%%%%%%%%%%%
%%%%%%%%%%%%%%%%%%%%%%%%%%%%%%%%%%%%%%%%%%%%%%%%%%%%%%%%%%%%%%%%%%%%%%%%%%%%%%%%
\section{Introduction}

\LaTeX{} provides a mechanism to structure a large document (such as a book)
into a main file and several child files (containing the chapters)
using the |\include| command.
This mechanism is beneficial for documents
which span hundreds of pages in order to
make the source file(s) more manageable.
Moreover, compilation can be restricted to
selected child files by means of the |\includeonly| command.
The latter feature can be used to reduce the compilation time while editing
(this was significantly more useful in the earlier days of \LaTeX{})
or to generate a smaller document which is easier to navigate.
Another application of |\includeonly| is to generate
documents consisting of selected parts of the complete document.

However, there are a few drawbacks of the plain |\include| mechanism:
\begin{itemize}
\item
The child files cannot be compiled on their own,
they can only be compiled via the main file.
A naive editing environment
(such as a text editor with an option
to have the current file processed by \LaTeX)
may require one to switch to the main file before compiling;
attempting to compile the child file produces errors.
\item
The main file must be modified (each time)
to adjust the |\includeonly| command
to the present needs. This easily leaves the main file in a messy state.
\item
The generated document will always carry the filename
of the main document. This is inconvenient if
several child files are to be compiled and
to be kept for distribution.
\end{itemize}

The present package provides a simple interface
to make child files individually compilable by \LaTeX{}.
Compiling a child file then has the same effect as compiling
the main file with an |\includeonly| command
to select the appropriate child.
Moreover the generated document will carry the name of the child
rather than the main file.
This resolves all three above issues.

This feature is meant to make the editing of books,
thesis documents and lecture notes somewhat more convenient.
However, the package can also be used efficiently for
composing a series of documents (such as exercise sheets)
which are typically distributed individually.
It then assists the author in generating the individual documents
(potentially in different versions)
as well as a document containing the collected series.
Another application is in developing style files
or other kinds of included material
where compilation of the style file could redirect
to a sample or test file.

%%%%%%%%%%%%%%%%%%%%%%%%%%%%%%%%%%%%%%%%%%%%%%%%%%%%%%%%%%%%%%%%%%%%%%%%%%%%%%%%
%%%%%%%%%%%%%%%%%%%%%%%%%%%%%%%%%%%%%%%%%%%%%%%%%%%%%%%%%%%%%%%%%%%%%%%%%%%%%%%%
\section{Usage}

First of all, the package \textsf{childdoc} is \emph{not} a standard
\LaTeXe{} |.sty| style file! Therefore it needs to be invoked in
a non-standard way.

%%%%%%%%%%%%%%%%%%%%%%%%%%%%%%%%%%%%%%%%%%%%%%%%%%%%%%%%%%%%%%%%%%%%%%%%%%%%%%%%
\subsection{Included Files}
\label{sec:include}

%%%%%%%%%%%%%%%%%%%%%%%%%%%%%%%%%%%%%%%%
\DescribeMacro{\childdocmain}
To use the package, add the commands
\begin{center}
\begin{tabular}{l}
|\input{childdoc.def}|\\
|\childdocmain{}|\\
\end{tabular}
\end{center}
at the very top of the main \LaTeX{} file,
in particular \emph{before} the |\documentclass| statement!
The argument of |\childdocmain| should be left empty
(but it must be present).

%%%%%%%%%%%%%%%%%%%%%%%%%%%%%%%%%%%%%%%%
\DescribeMacro{\childdocof}
Furthermore, add the commands
\begin{center}
\begin{tabular}{l}
|\input{childdoc.def}|\\
|\childdocof{|\textit{main}|}|\\
\end{tabular}
\end{center}
at the top of every child file \textit{child}
which is included by |\include{|\textit{child}|}|
from within the main file
(or at least for those files to be compiled individually).
The argument \textit{main} must be the filename of the main file.

There are a couple of
considerations in setting up the main and child documents:

%%%%%%%%%%%%%%%%%%%%%%%%%%%%%%%%%%%%%%%%
\paragraph{Restrictions.}

Please note the following restrictions:
\begin{itemize}
\item
|\childdocmain| must be called with one argument \textit{main}
to ensure compatibility with earlier version of the package.
It must either be empty (|\childdocmain{}|)
or precisely match the filename of the main file in which it is specified.
See \secref{sec:detection} for further information.
\item
The filename \textit{main} must be specified without the |.tex| extension.
\item
The filename \textit{main} is case sensitive
(even in case-insensitive file systems)
due to internal string comparison.
\item
The argument \textit{main} should be fully expanded, it cannot be a macro.
\item
Subdirectories and special characters should be avoided in filenames.
\item
The command |\childdocmain{|\textit{main}|}| must be followed by a whitespace.
It should not be followed immediately by another command
or by a comment mark `|%|'.
This is because the \TeX{} parser reads the token immediately following
the argument of |\childdocmain| and puts it
at the beginning of every child section;
however, a white\-space is ignored.
\end{itemize}

%%%%%%%%%%%%%%%%%%%%%%%%%%%%%%%%%%%%%%%%
\paragraph{Content of Main File.}

It is advisable to place all content in the child files included by |\include|.
Any output contained in the main file will appear in all child documents
unless suppressed manually;
it cannot be suppressed automatically by the |\includeonly| directive
and thus should normally be avoided.
A method to include some content in the main file
by means of conditional processing is described in \secref{sec:conditional}.

%%%%%%%%%%%%%%%%%%%%%%%%%%%%%%%%%%%%%%%%
\paragraph{Page Numbering.}

When only a part of the document is compiled,
the appropriate numbering of pages
(as well as other status parameters)
is determined from the |.aux| files.
The latter contain information from previous passes.
However this information needs to propagate through
all intermediate child documents.
Therefore the page numbering in child documents may well
be inconsistent until the complete document is compiled at least once.

A useful (if unconventional) way to always ensure a consistent
page numbering is to restart the numbering in each child document
and denote the pages by `\textit{child}|.|\textit{page}'
where \textit{child} represents the chapter/section number of the child file.
This can be achieved by the command
|\numberwithin{page}{|\textit{child}|}|
of the \textsf{amsmath} package
where \textit{child} can be |chapter| or |section|
depending on the chosen structuring.
Alternatively, one can modify the macro |\thepage| appropriately
and reset the counter |page| at the start of each child file.

%%%%%%%%%%%%%%%%%%%%%%%%%%%%%%%%%%%%%%%%%%%%%%%%%%%%%%%%%%%%%%%%%%%%%%%%%%%%%%%%
\subsection{Conditional Processing}
\label{sec:conditional}

The package provides a mechanism to compile different versions
of a document. To customise the versions further some conditional processing
can come in handy to distinguish which version is being compiled.
The package provides two macros to describe the compilation context:

%%%%%%%%%%%%%%%%%%%%%%%%%%%%%%%%%%%%%%%%
\DescribeMacro{\ifchilddoc}
The conditional |\ifchilddoc| distinguishes between the compilation of
child documents and the main document:
%
\begin{center}
|\ifchilddoc |\textit{child-code}| |[|\||else |\textit{main-code}]| \||fi|
\end{center}

%%%%%%%%%%%%%%%%%%%%%%%%%%%%%%%%%%%%%%%%
\DescribeMacro{\childdocname}
\DescribeMacro{\childdocjob}
The macro |\childdocname| contains the filename (without extension)
of the main or child file being processed.
Note that |\childdocjob| will always contain the name of the main file.

%%%%%%%%%%%%%%%%%%%%%%%%%%%%%%%%%%%%%%%%
\paragraph{Title Page.}

Conditional processing can be used to include a title or banner page
in the main document when proper precautions are taken.
Importantly, the code in the main file should ensure that the page counter
(as well as other status parameters which are stored in the |.aux| files)
takes the same value after the conditional processing.
Otherwise the page numbers may take divergent values
depending on which part is compiled.

For example, a title page could be declared by:
%
\begin{center}
\begin{tabular}{l}
|\ifchilddoc\||else|\\
|\addtocounter{page}{-1}|\\
\textit{code for title page}\\
|\newpage|\\
|\||fi|
\end{tabular}
\end{center}
%
A banner page for the child documents can be generated by:
%
\begin{center}
\begin{tabular}{l}
|\ifchilddoc|\\
|\addtocounter{page}{-1}|\\
\textit{code for banner page}\\
|\newpage|\\
|\||fi|
\end{tabular}
\end{center}
%
Here one could write a message such as:
\begin{center}
|This is the part \childdocname{} of \childdocjob{}.|
\end{center}

%%%%%%%%%%%%%%%%%%%%%%%%%%%%%%%%%%%%%%%%%%%%%%%%%%%%%%%%%%%%%%%%%%%%%%%%%%%%%%%%
\subsection{Flags}
\label{sec:flags}

The package makes it easy to generate different versions
of the main or child documents.
To this end compilation flags can be defined
and assigned different default values.
They will be particularly useful in conjunction
with the forwarding mechanism described in \secref{sec:forward}.

For example, it may be useful to have a flag |\version|
which can be set to |draft| or |final|.
The document source will contain some conditional code
depending on the value of |\version|.
Suppose further, the flag should default to |final| for the main file
and to |draft| for child files
which is a natural assignment for editing the document.
This is achieved by placing the following code
in the preamble of the main document
(below the |\childdocmain| directive):
%
\begin{center}
\begin{tabular}{l}
|\ifchilddoc|\\
|\providecommand{\version}{draft}|\\
|\||else|\\
|\providecommand{\version}{final}|\\
|\||fi|
\end{tabular}
\end{center}
%
The definition by |\providecommand| makes sure
that previous definitions are not overwritten.
Further statements |\providecommand{\version}{...}|
can thus be added before the above code to override it.

For the main file, one might add a line
(between |\childdocmain| and the above block)
%
\begin{center}
|%\ifchilddoc\||else\providecommand{\version}{draft}\||fi|
\end{center}
%
which can be uncommented to produce a draft version.
Likewise one can add a line to the very top of a child file
(above the |\childdocof{|\textit{main}|}| directive)
%
\begin{center}
|%\providecommand{\version}{final}|
\end{center}
%
which can be uncommented to produce the final version of this child document.

%%%%%%%%%%%%%%%%%%%%%%%%%%%%%%%%%%%%%%%%%%%%%%%%%%%%%%%%%%%%%%%%%%%%%%%%%%%%%%%%
\subsection{Forwarding}
\label{sec:forward}

Different versions of the main or child documents
using compilation flags as described in \secref{sec:flags}
can be (permanently) stored in different files
for convenient compilation, viewing and distribution.
To this end, the package defines a command
to pass on compilation to a different file:

%%%%%%%%%%%%%%%%%%%%%%%%%%%%%%%%%%%%%%%%
\DescribeMacro{\childdocforward}
The command |\childdocforward| redirects processing to
another source file:
%
\begin{center}
\begin{tabular}{l}
|\input{childdoc.def}|\\
|\childdocforward[|\textit{main}|]{|\textit{dest}|}|\\
\end{tabular}
\end{center}
%
The argument \textit{dest} is the destination file
(without extension).
It should be the main file or one of the child files.
Note that further \textsf{childdoc} directives
such as |\childdocof| and |\childdocforward|
in the indicated file will be processed in this form.
The optional argument \textit{main}
passes on directly to the main file \textit{main}
while pretending to compile the child \textit{dest}.
This form behaves as if \textit{dest}
issues |\childdocof{|\textit{main}|}| right away,
and no further \textsf{childdoc} directives will be processed.

%%%%%%%%%%%%%%%%%%%%%%%%%%%%%%%%%%%%%%%%
\DescribeMacro{\...prefix}
In the alternative form |\childdocforwardprefix|,
%
\begin{center}
\begin{tabular}{l}
|\input{childdoc.def}|\\
|\childdocforwardprefix[|\textit{main}|]{|\textit{prefix}|}{|\textit{dest}|}|
\end{tabular}
\end{center}
%
the destination file is determined by a pattern
depending on the current file:
To make this work, the current file must be called
`{\textit{prefix}\hspace{0.2em}\textit{suffix}}'
with \textit{prefix} matching precisely the argument.
Processing is then passed on to the file
`{\textit{dest}\hspace{0.2em}\textit{suffix}}'.
Surely, the same effect is achieved by
directly specifying the
argument `{\textit{dest}\hspace{0.2em}\textit{suffix}}'
in the first form.
However, that requires to set up a different file
for each child. With the alternative form of the command
all these files can have exactly the same content
which simplifies setting them up and maintaining them.

For example, the following file |draft.tex|
with a compilation flag |\version| as described in \secref{sec:flags}
compiles the main document as a draft:
%
\begin{center}
\begin{tabular}{l}
|\def\version{draft}|\\
|\input{childdoc.def}|\\
|\childdocforward{|\textit{main}|}|
\end{tabular}
\end{center}
%
Likewise, the following files |final|\textit{nn}|.tex|
compile the final version of the child document
|child|\textit{nn}|.tex|:
%
\begin{center}
\begin{tabular}{l}
|\def\version{final}|\\
|\input{childdoc.def}|\\
|\childdocforwardprefix{final}{child}|
\end{tabular}
\end{center}
%

Note that when several versions of a main file and/or of each child file
are to be generated, it may be convenient to set up a |Makefile| or
shell script to automatise the process.

%%%%%%%%%%%%%%%%%%%%%%%%%%%%%%%%%%%%%%%%%%%%%%%%%%%%%%%%%%%%%%%%%%%%%%%%%%%%%%%%
\subsection{Command Line Processing}
\label{sec:commandline}

The effect of redirection files can also be achieved by invoking
the \LaTeX{} compiler with a more elaborate command line.
Most conveniently this should be done as part
of a shell script or a |Makefile|.

When using \textsf{childdoc} in the main file, the following
command lines effectively perform a redirection
(note that depending on the shell being used,
backslashes may have to be doubled: `|\|' $\to$ `|\\|'):
%
\begin{center}
|... -jobname "|\textit{target}|" |\\|"|[\textit{flags}]%
|\input{childdoc.def}\childdocforward[|\textit{main}|]{|\textit{dest}|}"|
\end{center}
%
Here \textit{target} is the name of the output file,
\textit{main} is the name of the main file
and \textit{dest} is the name of the main or child file to be processed
(all filenames without extensions).
The optional argument \textit{main} can be omitted
if \textit{main} matches \textit{dest}.
Optionally, compilation \textit{flags} can be defined via |\def| commands.
This command line makes the \TeX{} engine believe
it is compiling the file \textit{target}
whose content is specified as the latter parameter.
The provided code then forwards the processing to
\textit{main} or \textit{dest} as described in \secref{sec:forward}.

%%%%%%%%%%%%%%%%%%%%%%%%%%%%%%%%%%%%%%%%%%%%%%%%%%%%%%%%%%%%%%%%%%%%%%%%%%%%%%%%
\subsection{Include by Input}
\label{sec:input}

Including child documents by |\include| has some restrictions by design.
Most notably, the content of a child document always occupies
its own set of pages; pages cannot be shared between child documents.
Usually, this behaviour makes perfect sense
because each child document contain an essential part of the document.
However, in some situations it may be desirable to compose
a document from a collection of parts
without having mandatory page breaks between then.
For this case, the package
provides a mechanism to include parts
by |\input| which can also be processed individually.
However, by construction this mechanism
requires manual handling of the content to be output.

%%%%%%%%%%%%%%%%%%%%%%%%%%%%%%%%%%%%%%%%
\DescribeMacro{\ifchilddocmanual}
The main file should be prepared as usual, see \secref{sec:include}.
However, the document body must make a distinction
between processing of an individual part and of the main document, e.g.:
%
\begin{center}
\begin{tabular}{l}
|\ifchilddocmanual|\\
|\input{\childdocname}|\\
|\||else|\\
\textit{document body with }|\input{|\textit{part}|}|\\
|\||fi|
\end{tabular}
\end{center}
%
The conditional |\ifchilddocmanual| is true whenever
a part to be included by |\input| is being compiled,
and the name of the part is stored in |\childdocname|.

%%%%%%%%%%%%%%%%%%%%%%%%%%%%%%%%%%%%%%%%
\DescribeMacro{\childdocby}
Each part to be included by |\input| should start with:
%
\begin{center}
\begin{tabular}{l}
|\input{childdoc.def}|\\
|\childdocby{|\textit{main}|}|\\
\end{tabular}
\end{center}
%
The directive |\childdocby| is similar to |\childdocof|
described in \secref{sec:include},
but the subsequent selection of content must be done manually.
To that end, both |\ifchilddoc| and |\ifchilddocmanual|
will be true upon processing of a part,
and the name of the part is stored in |\childdocname|.
Note that |\jobname| will be set to the filename of the current part
so that each part receives an individual |.aux| file
that does not interfere with the |.aux| file(s) of the main document.
This behaviour can be altered by the alternative form
|\childdocby[*]{|\textit{main}|}| (with a non-empty optional argument)
which uses the |.aux| file of the main document
by setting |\jobname| to \textit{main}.

%%%%%%%%%%%%%%%%%%%%%%%%%%%%%%%%%%%%%%%%%%%%%%%%%%%%%%%%%%%%%%%%%%%%%%%%%%%%%%%%
\subsection{Driver Development}
\label{sec:driver}

The \textsf{childdoc} mechanism can also be use for the development
of definition files such as \LaTeX{} styles or classes.
This case differs from the above setup with multiple parts
included by |\include| in that no |\includeonly| should be invoked.
This can be achieved by starting the include file
(before |\ProvidesPackage|) with:
%
\begin{center}
\begin{tabular}{l}
|\input{childdoc.def}|\\
|\childdocforward{|\textit{main}|}|\\
\end{tabular}
\end{center}
%
or alternatively with:
%
\begin{center}
\begin{tabular}{l}
|\input{childdoc.def}|\\
|\childdocby{|\textit{main}|}|\\
\end{tabular}
\end{center}
%
Both forms have slightly different effects as described above.
The main file is prepared as usual, see \secref{sec:include}.

%%%%%%%%%%%%%%%%%%%%%%%%%%%%%%%%%%%%%%%%%%%%%%%%%%%%%%%%%%%%%%%%%%%%%%%%%%%%%%%%
\subsection{Legacy Detection}
\label{sec:detection}

The directive |\childdocmain| in the main file can detect
whether the complete document or merely a child is to be compiled
even without using the directive |\childdocof|.
This method is deprecated because it is less robust
and there is no compelling reason to use it;
it is merely provided for backward compatibility
and it may be removed in future versions.

If the detection mechanism is to be used,
it is mandatory to correctly specify
the filename of the main file as the argument of |\childdocmain|:
%
\begin{center}
\begin{tabular}{l}
|\input{childdoc.def}|\\
|\childdocmain{|\textit{main}|}|\\
\end{tabular}
\end{center}
%
If |\jobname| does not match the argument \textit{main} of |\childdocmain|,
it is assumed that |\jobname| points to the child file to be compiled.
When using |\childdocmain| with the main file specified as argument,
it suffices to start a child file
with just |\input{|\textit{main}|}|
without loading of the package and using |\childdocof|.
If instead all processing is done
with the appropriate \textsf{childdoc} directives,
the argument of \textit{main} of |\childdocmain| can be empty.

An alternative version of the command line processing described
in \secref{sec:commandline} using the detection mechanism reads:
%
\begin{center}
|... -jobname "|\textit{target}|" "|[\textit{flags}]%
[|\def\jobname{|\textit{dest}|}|]|\input{|\textit{main}|}"|
\end{center}

%%%%%%%%%%%%%%%%%%%%%%%%%%%%%%%%%%%%%%%%%%%%%%%%%%%%%%%%%%%%%%%%%%%%%%%%%%%%%%%%
\subsection{Manual Code}
\label{sec:manual}

In case one cannot be certain whether the definitions file |childdoc.def|
is installed on the target \TeX{} distribution
and one prefers not to ship it,
it is conceivable to paste a few relevant commands into the sources.

To that end, drop all statements |\input{childdoc.def}|
and perform the replacements as outlined below.
Instead of |\childdocmain{|\textit{main}|}| add the following code
to the top of the main file:
%
\begin{center}
\begin{tabular}{l}
|\||ifdefined\childdocname\endinput\||fi\newif\ifchilddoc|\\
|\edef\childdocname{\scantokens\expandafter{\jobname\noexpand}}|\\
|\def\childdocmain{|\textit{main}|}\||ifx\childdocmain\childdocname\||else|\\
|\childdoctrue\includeonly{\childdocname}\let\jobname\childdocmain\||fi|\\
\end{tabular}
\end{center}
%
Instead of |\childdocof{|\textit{main}|}| just include the main file
at the top of each child file:
%
\begin{center}
|\input{|\textit{main}|}|
\end{center}
%
A simple redirection |\childdocforward{|\textit{dest}|}| is achieved by:
%
\begin{center}
|\def\jobname{|\textit{dest}|}\input{\jobname}|
\end{center}
%
The redirection with prefix
|\childdocforwardprefix[|\textit{prefix}|]{|\textit{dest}|}|
is accomplished by:
%
\begin{center}
\begin{tabular}{l}
|{\edef\jobname{\scantokens\expandafter{\jobname\noexpand}}|\\
|\def\redirectjob |\textit{prefix}|#1~~~{\gdef\jobname{|\textit{dest}|#1}}|\\
|\expandafter\redirectjob\jobname~~~}\input{\jobname}|
\end{tabular}
\end{center}

In an alternative approach,
child documents can be compiled by a specific command line
without additional code or specific definitions:
%
\begin{center}
|... -jobname "|\textit{target}|" "|[\textit{flags}]%
|\includeonly{|\textit{dest}|}\input{|\textit{main}|}"|
\end{center}
%

%%%%%%%%%%%%%%%%%%%%%%%%%%%%%%%%%%%%%%%%%%%%%%%%%%%%%%%%%%%%%%%%%%%%%%%%%%%%%%%%
%%%%%%%%%%%%%%%%%%%%%%%%%%%%%%%%%%%%%%%%%%%%%%%%%%%%%%%%%%%%%%%%%%%%%%%%%%%%%%%%
\section{Information}

%%%%%%%%%%%%%%%%%%%%%%%%%%%%%%%%%%%%%%%%%%%%%%%%%%%%%%%%%%%%%%%%%%%%%%%%%%%%%%%%
\subsection{Copyright}

Copyright \copyright{} 2017--2018 Niklas Beisert

This work may be distributed and/or modified under the
conditions of the \LaTeX{} Project Public License, either version 1.3
of this license or (at your option) any later version.
The latest version of this license is in
  \url{http://www.latex-project.org/lppl.txt}
and version 1.3 or later is part of all distributions of \LaTeX{}
version 2005/12/01 or later.

This work has the LPPL maintenance status `maintained'.

The Current Maintainer of this work is Niklas Beisert.

This work consists of the files |README.txt|, |childdoc.ins| and |childdoc.dtx|
as well as the derived files |childdoc.def|, |cdocsamp.tex|
with |cdocsch1.tex|, |cdocsch2.tex|, |cdocspt3.tex|, |cdocspt4.tex|,
|cdocsdrf.tex|, |cdocsfn1.tex|, |cdocsfn2.tex|
as well as |childdoc.pdf|.

%%%%%%%%%%%%%%%%%%%%%%%%%%%%%%%%%%%%%%%%%%%%%%%%%%%%%%%%%%%%%%%%%%%%%%%%%%%%%%%%
\subsection{Files and Installation}

The package consists of the files:
%
\begin{center}
\begin{tabular}{ll}
    |README.txt|   & readme file \\
    |childdoc.ins| & installation file \\
    |childdoc.dtx| & source file \\
    |childdoc.def| & definition file \\
    |cdocsamp.tex| & sample main file \\
    |cdocsch1.tex| & sample include file \\
    |cdocsch2.tex| & sample include file \\
    |cdocspt3.tex| & sample part file \\
    |cdocspt4.tex| & sample part file \\
    |cdocsdrf.tex| & sample redirection file \\
    |cdocsfn1.tex| & sample redirection file \\
    |cdocsfn2.tex| & sample redirection file \\
    |childdoc.pdf| & manual
\end{tabular}
\end{center}
%
The distribution consists of the files
|README.txt|, |childdoc.ins| and |childdoc.dtx|.
%
\begin{itemize}
\item
Run (pdf)\LaTeX{} on |childdoc.dtx|
to compile the manual |childdoc.pdf| (this file).
\item
Run \LaTeX{} on |childdoc.ins| to create the definitions file |childdoc.def|
and the sample |cdocsamp.tex| with include files
|cdocsch1.tex|, |cdocsch2.tex|, |cdocspt3.tex|, |cdocspt4.tex|,
|cdocsdrf.tex|, |cdocsfn1.tex|, |cdocsfn2.tex|.
Then copy the file |childdoc.def| to an appropriate directory of your \LaTeX{}
distribution, e.g.\ \textit{texmf-root}|/tex/latex/childdoc|.
\end{itemize}

%%%%%%%%%%%%%%%%%%%%%%%%%%%%%%%%%%%%%%%%%%%%%%%%%%%%%%%%%%%%%%%%%%%%%%%%%%%%%%%%
\subsection{Related CTAN Packages}

There are several other packages which offer a similar functionality:
%
\begin{itemize}
\item
The packages
\href{http://ctan.org/pkg/docmute}{\textsf{docmute}},
\href{http://ctan.org/pkg/includex}{\textsf{includex}} and
\href{http://ctan.org/pkg/standalone}{\textsf{standalone}}
provide commands to include only the document body of
a child file thus allowing both files to be compiled individually.
\item
The packages \href{http://ctan.org/pkg/subdocs}{\textsf{subdocs}}
and \href{http://ctan.org/pkg/subfiles}{\textsf{subfiles}}
provide structures in which the main and child documents can be
encapsulated and allowing them to be compiled individually.
The inclusion mechanism is different from the conventional |\include|.
\item
The package \href{http://ctan.org/pkg/combine}{\textsf{combine}}
is an elaborate solution to combine several documents into one.
\end{itemize}
%
See also the CTAN topic \href{http://ctan.org/topic/subdocs}{\textsf{subdocs}}
for further related packages.
The present package differs from the above solutions in that
a document structure constructed with the conventional |\include| mechanism
just needs two extra commands at the top of every file
such that all constituent files can be compiled individually.

%%%%%%%%%%%%%%%%%%%%%%%%%%%%%%%%%%%%%%%%%%%%%%%%%%%%%%%%%%%%%%%%%%%%%%%%%%%%%%%%
%\subsection{Feature Suggestions}
%
%The following is a list of features which may be useful for future
%versions of this package:
%%
%\begin{itemize}
%\item
%\ldots
%\end{itemize}

%%%%%%%%%%%%%%%%%%%%%%%%%%%%%%%%%%%%%%%%%%%%%%%%%%%%%%%%%%%%%%%%%%%%%%%%%%%%%%%%
\subsection{Revision History}

%%%%%%%%%%%%%%%%%%%%%%%%%%%%%%%%%%%%%%%%
\paragraph{v2.0:} 2018/12/30

\begin{itemize}
\item
immediate forward processing
\item
added |\childdocby| mechanism
\item
manual restructured
\end{itemize}

%%%%%%%%%%%%%%%%%%%%%%%%%%%%%%%%%%%%%%%%
\paragraph{v1.6:} 2018/01/17

\begin{itemize}
\item
application for development of include files
\item
corrections to manual
\end{itemize}

%%%%%%%%%%%%%%%%%%%%%%%%%%%%%%%%%%%%%%%%
\paragraph{v1.5:} 2017/05/21

\begin{itemize}
\item
more complete structuring introduced
\item
|\childdocof| introduced
\item
|\childdoc| renamed to |\childdocmain|
\item
|\childredirect| renamed to |\childdocforward| and |\childdocforwardprefix|
and functionality expanded
\end{itemize}

%%%%%%%%%%%%%%%%%%%%%%%%%%%%%%%%%%%%%%%%
\paragraph{v1.0:} 2017/04/27

\begin{itemize}
\item
manual and install package
\item
first version published on CTAN
\end{itemize}

%%%%%%%%%%%%%%%%%%%%%%%%%%%%%%%%%%%%%%%%
\paragraph{v0.6:} 2017/04/26

\begin{itemize}
\item
redirection mechanism added
\end{itemize}

%%%%%%%%%%%%%%%%%%%%%%%%%%%%%%%%%%%%%%%%
\paragraph{v0.5:} 2017/04/26

\begin{itemize}
\item
functionality in definition file
\end{itemize}


%%%%%%%%%%%%%%%%%%%%%%%%%%%%%%%%%%%%%%%%%%%%%%%%%%%%%%%%%%%%%%%%%%%%%%%%%%%%%%%%
%%%%%%%%%%%%%%%%%%%%%%%%%%%%%%%%%%%%%%%%%%%%%%%%%%%%%%%%%%%%%%%%%%%%%%%%%%%%%%%%
%%%%%%%%%%%%%%%%%%%%%%%%%%%%%%%%%%%%%%%%%%%%%%%%%%%%%%%%%%%%%%%%%%%%%%%%%%%%%%%%
\appendix

\settowidth\MacroIndent{\rmfamily\scriptsize 000\ }

 \DocInput{childdoc.dtx}

\end{document}
%</driver>
% \fi
%
% %%%%%%%%%%%%%%%%%%%%%%%%%%%%%%%%%%%%%%%%%%%%%%%%%%%%%%%%%%%%%%%%%%%%%%%%%%%%%%
% %%%%%%%%%%%%%%%%%%%%%%%%%%%%%%%%%%%%%%%%%%%%%%%%%%%%%%%%%%%%%%%%%%%%%%%%%%%%%%
% \section{Sample}
%\iffalse
%<*samplemain>
%\fi
%
% The following presents a sample document
% with two chapters, two parts, a title page,
% a compile flag as well as three forwarding files to set the flag.
% It consists of eight |.tex| files:
% \begin{center}
% \begin{tabular}{ll}
% |cdocsamp.tex|&main file\\
% |cdocsch1.tex|&include file for chapter 1\\
% |cdocsch2.tex|&include file for chapter 2\\
% |cdocspt3.tex|&include file for part 3\\
% |cdocspt4.tex|&include file for part 4\\
% |cdocsdrf.tex|&forwarding file for main file in draft mode\\
% |cdocsfi1.tex|&forwarding file for final version of chapter 1\\
% |cdocsfi2.tex|&forwarding file for final version of chapter 2\\
% \end{tabular}
% \end{center}
% Each of the eight files can be compiled directly by the \LaTeX{} compiler.
%
% %%%%%%%%%%%%%%%%%%%%%%%%%%%%%%%%%%%%%%
% \paragraph{Main File.}
%
% The main file is called |cdocsamp.tex|.
%
% Load the \textsf{childdoc} definitions and
% declare the filename for the main document:
%    \begin{macrocode}
\input{childdoc.def}
\childdocmain{}
%    \end{macrocode}

% Optional override for |\version| flag:
%    \begin{macrocode}
%%\ifchilddoc\else\providecommand{\version}{draft}\fi
%    \end{macrocode}

% Define the default values for the |\version| flag
% (|final| for the main file and |draft| for childs):
%    \begin{macrocode}
\ifchilddoc
\providecommand{\version}{draft}
\else
\providecommand{\version}{final}
\fi
%    \end{macrocode}

% Load the standard document class:
%    \begin{macrocode}
\documentclass[12pt]{article}
%    \end{macrocode}

% Start the document body:
%    \begin{macrocode}
\begin{document}
%    \end{macrocode}

% Declare a title page.
% Print title, part of document being processed and version flag:
%    \begin{macrocode}
\addtocounter{page}{-1}
\begin{center}
{\LARGE\bfseries{}childdoc example\par}
\vspace{1cm}
\ifchilddoc
\ifchilddocmanual part\else chapter\fi:
`\childdocname' of `\childdocjob'\par
\else
main document: `\childdocjob'\par
\fi
version: \version\par
\end{center}
\newpage
%    \end{macrocode}

% Manually include selected file,
% otherwise process as usual:
%    \begin{macrocode}
\ifchilddocmanual
\section*{part `\childdocname'}
\input{\childdocname}
\else
%    \end{macrocode}

% Include the two chapters:
%    \begin{macrocode}
\include{cdocsch1}
\include{cdocsch2}
%    \end{macrocode}

% Include the two parts unless only chapters should be displayed:
%    \begin{macrocode}
\ifchilddoc\else
\section{part three}
\input{cdocspt3}
\section{part four}
\input{cdocspt4}
\fi
%    \end{macrocode}

% Process as usual until here:
%    \begin{macrocode}
\fi
%    \end{macrocode}

% End of document body:
%    \begin{macrocode}
\end{document}
%    \end{macrocode}
%\iffalse
%</samplemain>
%\fi
%
% %%%%%%%%%%%%%%%%%%%%%%%%%%%%%%%%%%%%%%
% \paragraph{Chapter Include Files.}
%
% The include files are called |cdocsch1.tex| and |cdocsch2.tex|.
%
%\iffalse
%<*samplechap1|samplechap2>
%\fi

% Optional override for |\version| flag:
%    \begin{macrocode}
%%\providecommand{\version}{final}
%    \end{macrocode}

% Include the main document:
%    \begin{macrocode}
\input{childdoc.def}
\childdocof{cdocsamp}
%    \end{macrocode}

%\iffalse
%</samplechap1|samplechap2>
%\fi
%
%\iffalse
%<*samplechap1>
%\fi
% Some text for chapter 1:
%    \begin{macrocode}
\section{one}
some text in chapter one
%    \end{macrocode}

%\iffalse
%</samplechap1>
%\fi
% Some text for chapter 2:
%\iffalse
%<*samplechap2>
%\fi
%    \begin{macrocode}
\section{two}
more text in chapter two
%    \end{macrocode}

%\iffalse
%</samplechap2>
%\fi
%
% %%%%%%%%%%%%%%%%%%%%%%%%%%%%%%%%%%%%%%
% \paragraph{Part Include Files.}
%
% The include files are called |cdocspt3.tex| and |cdocspt4.tex|.
%
%\iffalse
%<*samplepart3|samplepart4>
%\fi

% Optional override for |\version| flag:
%    \begin{macrocode}
%%\providecommand{\version}{final}
%    \end{macrocode}

% Include the main document:
%    \begin{macrocode}
\input{childdoc.def}
\childdocby{cdocsamp}
%    \end{macrocode}

%\iffalse
%</samplepart3|samplepart4>
%\fi
%
%\iffalse
%<*samplepart3>
%\fi
% Some text for part 3:
%    \begin{macrocode}
some text in part three
%    \end{macrocode}

%\iffalse
%</samplepart3>
%\fi
% Some text for part 4:
%\iffalse
%<*samplepart4>
%\fi
%    \begin{macrocode}
more text in part four
%    \end{macrocode}

%\iffalse
%</samplepart4>
%\fi
%
% %%%%%%%%%%%%%%%%%%%%%%%%%%%%%%%%%%%%%%
% \paragraph{Forwarding for a Complete Draft.}
%
% The following forwarding file |cdocsdrf.tex|
% compiles the main document in draft mode:
%\iffalse
%<*sampledraft>
%\fi
%    \begin{macrocode}
\def\version{draft}
\input{childdoc.def}
\childdocforward{cdocsamp}
%    \end{macrocode}

%\iffalse
%</sampledraft>
%\fi
%
% %%%%%%%%%%%%%%%%%%%%%%%%%%%%%%%%%%%%%%
% \paragraph{Forwarding for Final Version of the Chapters.}
%
% The following forwarding files |cdocsfn1.tex| and |cdocsfn2.tex|
% (with identical content)
% compile the final versions of the child documents
% |cdocsch1.tex| and |cdocsch2.tex|, respectively:
%\iffalse
%<*samplefinal>
%\fi
%    \begin{macrocode}
\def\version{final}
\input{childdoc.def}
\childdocforwardprefix[cdocsamp]{cdocsfn}{cdocsch}
%    \end{macrocode}

%\iffalse
%</samplefinal>
%\fi
%
% %%%%%%%%%%%%%%%%%%%%%%%%%%%%%%%%%%%%%%
% \paragraph{Command Line Processing.}
%
% The following three command lines generate the output files
% |cdocscld|, |cdocscl1| and |cdocscl2|
% which should be identical to
% |cdocsdrf|, |cdocsch1| and |cdocsfn2|, respectively:
% \begin{center}
% \begin{tabular}{l}
% |latex -jobname cdocscld \|\\
% |  "\def\version{draft}\input{childdoc.def}\childdocforward{cdocsamp}"|\\
% |latex -jobname cdocscl1 \|\\
% |  "\input{childdoc.def}\childdocforward[cdocsamp]{cdocsch1}"|\\
% |latex -jobname cdocscl2 \|\\
% |  "\def\version{final}\input{childdoc.def}\childdocforward{cdocsch2}"|
% \end{tabular}
% \end{center}
% Note that the trailing backslash on each first line
% merely continues the input to the second line
% (for convenient cut ant paste).
% Furthermore, the command |latex| can be replaced by any
% of its alternative versions such as |pdflatex|.
%
% %%%%%%%%%%%%%%%%%%%%%%%%%%%%%%%%%%%%%%%%%%%%%%%%%%%%%%%%%%%%%%%%%%%%%%%%%%%%%%
% %%%%%%%%%%%%%%%%%%%%%%%%%%%%%%%%%%%%%%%%%%%%%%%%%%%%%%%%%%%%%%%%%%%%%%%%%%%%%%
% \section{Implementation}
%\iffalse
%<*package>
%\fi
%
% This section describes the definitions file |childdoc.def|.

% The definitions cannot be loaded using |\usepackage| or |\RequirePackage|
% which has a mechanism to prevent loading a style file more than once.
% When loading the definitions by means of |\input|
% multiple instances have to be prevented manually:
%\iffalse
%This code needs to be before the `\ProvidesFile' directive
%which is defined at the beginning of this file.
%Therefore it is also placed there and commented out here.
%</package>
%<*discard>
%\fi
%    \begin{macrocode}
\ifdefined\childdocmain\endinput\fi
%    \end{macrocode}
%\iffalse
%</discard>
%<*package>
%\fi
%
% \macro{\ifchilddoc}
% \macro{\ifchilddocmanual}
% The conditional |\ifchilddoc| tells whether a
% child (true) or main (false) document is being compiled.
% The conditional |\ifchilddocmanual| tells whether
% the |\includeonly| mechanism is used (false) or
% the selection of child files must be performed manually (true).
% The definitions initialise to false:
%    \begin{macrocode}
\newif\ifchilddoc
\newif\ifchilddocmanual
%    \end{macrocode}

% \macro{\childdocname}
% \macro{\childdocjob}
% The macro |\childdocname| stores the name of the main document
% to be compiled. The macro |\childdocjob| stores the name of
% the document on which the \LaTeX{} compiler was originally invoked.
% The content of |\jobname| cannot be compared
% to filenames specified in the source due to different catcodes.
% The following code rescans |\jobname|, stores the result
% in |\childdocname| and saves a copy in |\childdocjob|:
%    \begin{macrocode}
\edef\childdocname{\scantokens\expandafter{\jobname\noexpand}}
\let\childdocjob\childdocname
%    \end{macrocode}

% \macro{\childdocdisable}
% The macro |\childdocdisable| prevents the main file
% from being processed more than once.
% At this stage, the main document command |\childdocmain|
% is assumed to be called once again where it should do nothing.
% Any subsequent call to it should prevent
% a secondary processing of the main document
% It overwrites the forwarding commands
% |\childdocof| and |\childdocforward|
% with empty macros to prevent further inclusions of the main document:
%    \begin{macrocode}
\newcommand{\childdocdisable}
{
  \renewcommand{\childdocmain}[1]{\renewcommand{\childdocmain}[1]{\endinput}}
  \renewcommand{\childdocof}[1]{}
  \renewcommand{\childdocby}[2][]{}
  \renewcommand{\childdocforward}[2][]{}
  \renewcommand{\childdocdisable}{}
}
%    \end{macrocode}

% \macro{\childdocmain}
% The macro |\childdocmain| is to be called at the top of the main file
% with nothing or the main filename (without extension) as argument.
% First, it breaks loops.
% If the argument is not empty and does not match |\childdocname|
% (which is set by the first inclusion of |childdoc.def|),
% |\ifchilddoc| is set to true, |\includeonly| is applied to the child file
% and |\jobname| is set to the main file
% (for proper handling of |.aux| files):
%    \begin{macrocode}
\newcommand{\childdocmain}[1]
{
  \childdocdisable\childdocmain{}
  \if?#1?\else
    \begingroup
      \def\childdoctmp{#1}
      \ifx\childdoctmp\childdocname
        \def\childdoctmp{}
      \else
        \def\childdoctmp
        {
          \childdoctrue
          \includeonly{\childdocname}
          \def\childdocjob{#1}
          \def\jobname{#1}
        }
      \fi
      \expandafter
    \endgroup
    \childdoctmp
  \fi
}
%    \end{macrocode}

% \macro{\childdocof}
% The command |\childdocof| redirects
% compilation to the main file |#1|.
%    \begin{macrocode}
\newcommand{\childdocof}[1]
{
  \childdocdisable
  \childdoctrue
  \includeonly{\childdocname}
  \def\jobname{#1}
  \def\childdocjob{#1}
  \input{#1}
}
%    \end{macrocode}

% \macro{\childdocby}
% The command |\childdocby| ....
%    \begin{macrocode}
\newcommand{\childdocby}[2][]
{
  \childdocdisable
  \childdoctrue
  \childdocmanualtrue
  \if?#1?\else
    \def\jobname{#2}
  \fi
  \def\childdocjob{#2}
  \input{#2}
  \endinput
}
%    \end{macrocode}

% \macro{\childdocforward}
% The command |\childdocforward| redirects
% compilation to the main file or
% (if the optional argument is given) a child file.
% Parameters are set as if the main file
% or a child file starting with |\childdocof| was compiled.
% Then compilation is handed over to the main file:
%    \begin{macrocode}
\newcommand{\childdocforward}[2][]
{
  \begingroup
    \if?#1?
      \def\childdoctmp
      {
        \def\childdocname{#2}
        \def\childdocjob{#2}
        \def\jobname{#2}
        \input{#2}
        \endinput
      }
    \else
      \def\childdoctmp
      {
        \childdocdisable
        \def\childdocname{#2}
        \childdoctrue
        \includeonly{#2}
        \def\childdocjob{#1}
        \def\jobname{#1}
        \input{#1}
        \endinput
      }
    \fi
    \expandafter
  \endgroup
  \childdoctmp
}
%    \end{macrocode}

% \macro{\childdocforwardprefix}
% The command |\childdocforwardprefix| redirects
% compilation to the main or a child file by means of a pattern.
% The prefix |#1| in the current filename is replaced by |#2|
% and the suffix of the current filename is kept
% (it is assumed that the filename does not contain the substring `|~~~|'
% which is used as a delimiter).
% Compilation is handed over to the new file by |\childdocforward|:
%    \begin{macrocode}
\newcommand{\childdocforwardprefix}[3][]
{
  \begingroup
    \def\childdocextract #2##1~~~{\def\childdoctmp{\childdocforward[#1]{#3##1}}}
    \expandafter\childdocextract\childdocname~~~
    \expandafter
  \endgroup
  \childdoctmp
}
%    \end{macrocode}

% \macro{\childdoc}
% The deprecated macro |\childdoc| is a legacy version of |\childdocmain|:
%    \begin{macrocode}
\newcommand{\childdoc}{\childdocmain}
%    \end{macrocode}

% \macro{\childdocredirect}
% The deprecated macro |\childdocredirect| is a legacy version
% of |\childdocforward| and |\childdocforwardprefix|:
%    \begin{macrocode}
\newcommand{\childdocredirect}[2][]
{
  \begingroup
    \if?#1?
      \def\childdoctmp{\childdocforward{#2}}
    \else
      \def\childdoctmp{\childdocforwardprefix{#1}{#2}}
    \fi
    \expandafter
  \endgroup
  \childdoctmp
}
%    \end{macrocode}

%\iffalse
%</package>
%\fi
%
\endinput
|\\
|\childdocforward[|\textit{main}|]{|\textit{dest}|}|\\
\end{tabular}
\end{center}
%
The argument \textit{dest} is the destination file
(without extension).
It should be the main file or one of the child files.
Note that further \textsf{childdoc} directives
such as |\childdocof| and |\childdocforward|
in the indicated file will be processed in this form.
The optional argument \textit{main}
passes on directly to the main file \textit{main}
while pretending to compile the child \textit{dest}.
This form behaves as if \textit{dest}
issues |\childdocof{|\textit{main}|}| right away,
and no further \textsf{childdoc} directives will be processed.

%%%%%%%%%%%%%%%%%%%%%%%%%%%%%%%%%%%%%%%%
\DescribeMacro{\...prefix}
In the alternative form |\childdocforwardprefix|,
%
\begin{center}
\begin{tabular}{l}
|% \iffalse
%
% childdoc.dtx Copyright (C) 2017-2018 Niklas Beisert
%
% This work may be distributed and/or modified under the
% conditions of the LaTeX Project Public License, either version 1.3
% of this license or (at your option) any later version.
% The latest version of this license is in
%   http://www.latex-project.org/lppl.txt
% and version 1.3 or later is part of all distributions of LaTeX
% version 2005/12/01 or later.
%
% This work has the LPPL maintenance status `maintained'.
%
% The Current Maintainer of this work is Niklas Beisert.
%
% This work consists of the files childdoc.dtx and childdoc.ins
% and the derived files childdoc.def and cdocsamp.tex with
% cdocsch1.tex, cdocsch2.tex, cdocsdrf.tex, cdocsfn1.tex, cdocsfn2.tex.
%
%<package>\ifdefined\childdocmain\endinput\fi
%<package>\ProvidesFile{childdoc.def}[2018/12/30 v2.0 child document driver]
%<samplemain>\ProvidesFile{cdocsamp.tex}[2018/12/30 v2.0 sample for childdoc]
%<*driver>
%\ProvidesFile{childdoc.drv}[2018/12/30 v2.0 childdoc reference manual file]
\PassOptionsToClass{10pt,a4paper}{article}
\documentclass{ltxdoc}

\usepackage[margin=35mm]{geometry}
\usepackage{hyperref}
\usepackage{hyperxmp}
\usepackage[usenames]{color}

\hypersetup{colorlinks=true}
\hypersetup{pdfstartview=FitH}
\hypersetup{pdfpagemode=UseNone}
\hypersetup{pdfsource={}}
\hypersetup{pdflang={en-UK}}
\hypersetup{pdfcopyright={Copyright 2017-2018 Niklas Beisert.
  This work may be distributed and/or modified under the
  conditions of the LaTeX Project Public License, either version 1.3
  of this license or (at your option) any later version.}}
\hypersetup{pdflicenseurl={http://www.latex-project.org/lppl.txt}}
\hypersetup{pdfcontactaddress={ETH Zurich, ITP, HIT K,
  Wolfgang-Pauli-Strasse 27}}
\hypersetup{pdfcontactpostcode={8093}}
\hypersetup{pdfcontactcity={Zurich}}
\hypersetup{pdfcontactcountry={Switzerland}}
\hypersetup{pdfcontactemail={nbeisert@itp.phys.ethz.ch}}
\hypersetup{pdfcontacturl={http://people.phys.ethz.ch/\xmptilde nbeisert/}}

\newcommand{\secref}[1]{\hyperref[#1]{section \ref*{#1}}}

\parskip1ex
\parindent0pt
\let\olditemize\itemize
\def\itemize{\olditemize\parskip0pt}

\begin{document}

\title{The \textsf{childdoc} Package}
\hypersetup{pdftitle={The childdoc Package}}
\author{Niklas Beisert\\[2ex]
  Institut f\"ur Theoretische Physik\\
  Eidgen\"ossische Technische Hochschule Z\"urich\\
  Wolfgang-Pauli-Strasse 27, 8093 Z\"urich, Switzerland\\[1ex]
  \href{mailto:nbeisert@itp.phys.ethz.ch}
  {\texttt{nbeisert@itp.phys.ethz.ch}}}
\hypersetup{pdfauthor={Niklas Beisert}}
\hypersetup{pdfsubject={Manual for the LaTeX2e Package childdoc}}
\date{30 December 2018, \textsf{v2.0}}
\maketitle

\begin{abstract}\noindent
\textsf{childdoc} is a \LaTeXe{} package
that enables the direct compilation
of document sections included by |\include|
to individual files.
\end{abstract}

\begingroup
\parskip0ex
\tableofcontents
\endgroup

%%%%%%%%%%%%%%%%%%%%%%%%%%%%%%%%%%%%%%%%%%%%%%%%%%%%%%%%%%%%%%%%%%%%%%%%%%%%%%%%
%%%%%%%%%%%%%%%%%%%%%%%%%%%%%%%%%%%%%%%%%%%%%%%%%%%%%%%%%%%%%%%%%%%%%%%%%%%%%%%%
\section{Introduction}

\LaTeX{} provides a mechanism to structure a large document (such as a book)
into a main file and several child files (containing the chapters)
using the |\include| command.
This mechanism is beneficial for documents
which span hundreds of pages in order to
make the source file(s) more manageable.
Moreover, compilation can be restricted to
selected child files by means of the |\includeonly| command.
The latter feature can be used to reduce the compilation time while editing
(this was significantly more useful in the earlier days of \LaTeX{})
or to generate a smaller document which is easier to navigate.
Another application of |\includeonly| is to generate
documents consisting of selected parts of the complete document.

However, there are a few drawbacks of the plain |\include| mechanism:
\begin{itemize}
\item
The child files cannot be compiled on their own,
they can only be compiled via the main file.
A naive editing environment
(such as a text editor with an option
to have the current file processed by \LaTeX)
may require one to switch to the main file before compiling;
attempting to compile the child file produces errors.
\item
The main file must be modified (each time)
to adjust the |\includeonly| command
to the present needs. This easily leaves the main file in a messy state.
\item
The generated document will always carry the filename
of the main document. This is inconvenient if
several child files are to be compiled and
to be kept for distribution.
\end{itemize}

The present package provides a simple interface
to make child files individually compilable by \LaTeX{}.
Compiling a child file then has the same effect as compiling
the main file with an |\includeonly| command
to select the appropriate child.
Moreover the generated document will carry the name of the child
rather than the main file.
This resolves all three above issues.

This feature is meant to make the editing of books,
thesis documents and lecture notes somewhat more convenient.
However, the package can also be used efficiently for
composing a series of documents (such as exercise sheets)
which are typically distributed individually.
It then assists the author in generating the individual documents
(potentially in different versions)
as well as a document containing the collected series.
Another application is in developing style files
or other kinds of included material
where compilation of the style file could redirect
to a sample or test file.

%%%%%%%%%%%%%%%%%%%%%%%%%%%%%%%%%%%%%%%%%%%%%%%%%%%%%%%%%%%%%%%%%%%%%%%%%%%%%%%%
%%%%%%%%%%%%%%%%%%%%%%%%%%%%%%%%%%%%%%%%%%%%%%%%%%%%%%%%%%%%%%%%%%%%%%%%%%%%%%%%
\section{Usage}

First of all, the package \textsf{childdoc} is \emph{not} a standard
\LaTeXe{} |.sty| style file! Therefore it needs to be invoked in
a non-standard way.

%%%%%%%%%%%%%%%%%%%%%%%%%%%%%%%%%%%%%%%%%%%%%%%%%%%%%%%%%%%%%%%%%%%%%%%%%%%%%%%%
\subsection{Included Files}
\label{sec:include}

%%%%%%%%%%%%%%%%%%%%%%%%%%%%%%%%%%%%%%%%
\DescribeMacro{\childdocmain}
To use the package, add the commands
\begin{center}
\begin{tabular}{l}
|\input{childdoc.def}|\\
|\childdocmain{}|\\
\end{tabular}
\end{center}
at the very top of the main \LaTeX{} file,
in particular \emph{before} the |\documentclass| statement!
The argument of |\childdocmain| should be left empty
(but it must be present).

%%%%%%%%%%%%%%%%%%%%%%%%%%%%%%%%%%%%%%%%
\DescribeMacro{\childdocof}
Furthermore, add the commands
\begin{center}
\begin{tabular}{l}
|\input{childdoc.def}|\\
|\childdocof{|\textit{main}|}|\\
\end{tabular}
\end{center}
at the top of every child file \textit{child}
which is included by |\include{|\textit{child}|}|
from within the main file
(or at least for those files to be compiled individually).
The argument \textit{main} must be the filename of the main file.

There are a couple of
considerations in setting up the main and child documents:

%%%%%%%%%%%%%%%%%%%%%%%%%%%%%%%%%%%%%%%%
\paragraph{Restrictions.}

Please note the following restrictions:
\begin{itemize}
\item
|\childdocmain| must be called with one argument \textit{main}
to ensure compatibility with earlier version of the package.
It must either be empty (|\childdocmain{}|)
or precisely match the filename of the main file in which it is specified.
See \secref{sec:detection} for further information.
\item
The filename \textit{main} must be specified without the |.tex| extension.
\item
The filename \textit{main} is case sensitive
(even in case-insensitive file systems)
due to internal string comparison.
\item
The argument \textit{main} should be fully expanded, it cannot be a macro.
\item
Subdirectories and special characters should be avoided in filenames.
\item
The command |\childdocmain{|\textit{main}|}| must be followed by a whitespace.
It should not be followed immediately by another command
or by a comment mark `|%|'.
This is because the \TeX{} parser reads the token immediately following
the argument of |\childdocmain| and puts it
at the beginning of every child section;
however, a white\-space is ignored.
\end{itemize}

%%%%%%%%%%%%%%%%%%%%%%%%%%%%%%%%%%%%%%%%
\paragraph{Content of Main File.}

It is advisable to place all content in the child files included by |\include|.
Any output contained in the main file will appear in all child documents
unless suppressed manually;
it cannot be suppressed automatically by the |\includeonly| directive
and thus should normally be avoided.
A method to include some content in the main file
by means of conditional processing is described in \secref{sec:conditional}.

%%%%%%%%%%%%%%%%%%%%%%%%%%%%%%%%%%%%%%%%
\paragraph{Page Numbering.}

When only a part of the document is compiled,
the appropriate numbering of pages
(as well as other status parameters)
is determined from the |.aux| files.
The latter contain information from previous passes.
However this information needs to propagate through
all intermediate child documents.
Therefore the page numbering in child documents may well
be inconsistent until the complete document is compiled at least once.

A useful (if unconventional) way to always ensure a consistent
page numbering is to restart the numbering in each child document
and denote the pages by `\textit{child}|.|\textit{page}'
where \textit{child} represents the chapter/section number of the child file.
This can be achieved by the command
|\numberwithin{page}{|\textit{child}|}|
of the \textsf{amsmath} package
where \textit{child} can be |chapter| or |section|
depending on the chosen structuring.
Alternatively, one can modify the macro |\thepage| appropriately
and reset the counter |page| at the start of each child file.

%%%%%%%%%%%%%%%%%%%%%%%%%%%%%%%%%%%%%%%%%%%%%%%%%%%%%%%%%%%%%%%%%%%%%%%%%%%%%%%%
\subsection{Conditional Processing}
\label{sec:conditional}

The package provides a mechanism to compile different versions
of a document. To customise the versions further some conditional processing
can come in handy to distinguish which version is being compiled.
The package provides two macros to describe the compilation context:

%%%%%%%%%%%%%%%%%%%%%%%%%%%%%%%%%%%%%%%%
\DescribeMacro{\ifchilddoc}
The conditional |\ifchilddoc| distinguishes between the compilation of
child documents and the main document:
%
\begin{center}
|\ifchilddoc |\textit{child-code}| |[|\||else |\textit{main-code}]| \||fi|
\end{center}

%%%%%%%%%%%%%%%%%%%%%%%%%%%%%%%%%%%%%%%%
\DescribeMacro{\childdocname}
\DescribeMacro{\childdocjob}
The macro |\childdocname| contains the filename (without extension)
of the main or child file being processed.
Note that |\childdocjob| will always contain the name of the main file.

%%%%%%%%%%%%%%%%%%%%%%%%%%%%%%%%%%%%%%%%
\paragraph{Title Page.}

Conditional processing can be used to include a title or banner page
in the main document when proper precautions are taken.
Importantly, the code in the main file should ensure that the page counter
(as well as other status parameters which are stored in the |.aux| files)
takes the same value after the conditional processing.
Otherwise the page numbers may take divergent values
depending on which part is compiled.

For example, a title page could be declared by:
%
\begin{center}
\begin{tabular}{l}
|\ifchilddoc\||else|\\
|\addtocounter{page}{-1}|\\
\textit{code for title page}\\
|\newpage|\\
|\||fi|
\end{tabular}
\end{center}
%
A banner page for the child documents can be generated by:
%
\begin{center}
\begin{tabular}{l}
|\ifchilddoc|\\
|\addtocounter{page}{-1}|\\
\textit{code for banner page}\\
|\newpage|\\
|\||fi|
\end{tabular}
\end{center}
%
Here one could write a message such as:
\begin{center}
|This is the part \childdocname{} of \childdocjob{}.|
\end{center}

%%%%%%%%%%%%%%%%%%%%%%%%%%%%%%%%%%%%%%%%%%%%%%%%%%%%%%%%%%%%%%%%%%%%%%%%%%%%%%%%
\subsection{Flags}
\label{sec:flags}

The package makes it easy to generate different versions
of the main or child documents.
To this end compilation flags can be defined
and assigned different default values.
They will be particularly useful in conjunction
with the forwarding mechanism described in \secref{sec:forward}.

For example, it may be useful to have a flag |\version|
which can be set to |draft| or |final|.
The document source will contain some conditional code
depending on the value of |\version|.
Suppose further, the flag should default to |final| for the main file
and to |draft| for child files
which is a natural assignment for editing the document.
This is achieved by placing the following code
in the preamble of the main document
(below the |\childdocmain| directive):
%
\begin{center}
\begin{tabular}{l}
|\ifchilddoc|\\
|\providecommand{\version}{draft}|\\
|\||else|\\
|\providecommand{\version}{final}|\\
|\||fi|
\end{tabular}
\end{center}
%
The definition by |\providecommand| makes sure
that previous definitions are not overwritten.
Further statements |\providecommand{\version}{...}|
can thus be added before the above code to override it.

For the main file, one might add a line
(between |\childdocmain| and the above block)
%
\begin{center}
|%\ifchilddoc\||else\providecommand{\version}{draft}\||fi|
\end{center}
%
which can be uncommented to produce a draft version.
Likewise one can add a line to the very top of a child file
(above the |\childdocof{|\textit{main}|}| directive)
%
\begin{center}
|%\providecommand{\version}{final}|
\end{center}
%
which can be uncommented to produce the final version of this child document.

%%%%%%%%%%%%%%%%%%%%%%%%%%%%%%%%%%%%%%%%%%%%%%%%%%%%%%%%%%%%%%%%%%%%%%%%%%%%%%%%
\subsection{Forwarding}
\label{sec:forward}

Different versions of the main or child documents
using compilation flags as described in \secref{sec:flags}
can be (permanently) stored in different files
for convenient compilation, viewing and distribution.
To this end, the package defines a command
to pass on compilation to a different file:

%%%%%%%%%%%%%%%%%%%%%%%%%%%%%%%%%%%%%%%%
\DescribeMacro{\childdocforward}
The command |\childdocforward| redirects processing to
another source file:
%
\begin{center}
\begin{tabular}{l}
|\input{childdoc.def}|\\
|\childdocforward[|\textit{main}|]{|\textit{dest}|}|\\
\end{tabular}
\end{center}
%
The argument \textit{dest} is the destination file
(without extension).
It should be the main file or one of the child files.
Note that further \textsf{childdoc} directives
such as |\childdocof| and |\childdocforward|
in the indicated file will be processed in this form.
The optional argument \textit{main}
passes on directly to the main file \textit{main}
while pretending to compile the child \textit{dest}.
This form behaves as if \textit{dest}
issues |\childdocof{|\textit{main}|}| right away,
and no further \textsf{childdoc} directives will be processed.

%%%%%%%%%%%%%%%%%%%%%%%%%%%%%%%%%%%%%%%%
\DescribeMacro{\...prefix}
In the alternative form |\childdocforwardprefix|,
%
\begin{center}
\begin{tabular}{l}
|\input{childdoc.def}|\\
|\childdocforwardprefix[|\textit{main}|]{|\textit{prefix}|}{|\textit{dest}|}|
\end{tabular}
\end{center}
%
the destination file is determined by a pattern
depending on the current file:
To make this work, the current file must be called
`{\textit{prefix}\hspace{0.2em}\textit{suffix}}'
with \textit{prefix} matching precisely the argument.
Processing is then passed on to the file
`{\textit{dest}\hspace{0.2em}\textit{suffix}}'.
Surely, the same effect is achieved by
directly specifying the
argument `{\textit{dest}\hspace{0.2em}\textit{suffix}}'
in the first form.
However, that requires to set up a different file
for each child. With the alternative form of the command
all these files can have exactly the same content
which simplifies setting them up and maintaining them.

For example, the following file |draft.tex|
with a compilation flag |\version| as described in \secref{sec:flags}
compiles the main document as a draft:
%
\begin{center}
\begin{tabular}{l}
|\def\version{draft}|\\
|\input{childdoc.def}|\\
|\childdocforward{|\textit{main}|}|
\end{tabular}
\end{center}
%
Likewise, the following files |final|\textit{nn}|.tex|
compile the final version of the child document
|child|\textit{nn}|.tex|:
%
\begin{center}
\begin{tabular}{l}
|\def\version{final}|\\
|\input{childdoc.def}|\\
|\childdocforwardprefix{final}{child}|
\end{tabular}
\end{center}
%

Note that when several versions of a main file and/or of each child file
are to be generated, it may be convenient to set up a |Makefile| or
shell script to automatise the process.

%%%%%%%%%%%%%%%%%%%%%%%%%%%%%%%%%%%%%%%%%%%%%%%%%%%%%%%%%%%%%%%%%%%%%%%%%%%%%%%%
\subsection{Command Line Processing}
\label{sec:commandline}

The effect of redirection files can also be achieved by invoking
the \LaTeX{} compiler with a more elaborate command line.
Most conveniently this should be done as part
of a shell script or a |Makefile|.

When using \textsf{childdoc} in the main file, the following
command lines effectively perform a redirection
(note that depending on the shell being used,
backslashes may have to be doubled: `|\|' $\to$ `|\\|'):
%
\begin{center}
|... -jobname "|\textit{target}|" |\\|"|[\textit{flags}]%
|\input{childdoc.def}\childdocforward[|\textit{main}|]{|\textit{dest}|}"|
\end{center}
%
Here \textit{target} is the name of the output file,
\textit{main} is the name of the main file
and \textit{dest} is the name of the main or child file to be processed
(all filenames without extensions).
The optional argument \textit{main} can be omitted
if \textit{main} matches \textit{dest}.
Optionally, compilation \textit{flags} can be defined via |\def| commands.
This command line makes the \TeX{} engine believe
it is compiling the file \textit{target}
whose content is specified as the latter parameter.
The provided code then forwards the processing to
\textit{main} or \textit{dest} as described in \secref{sec:forward}.

%%%%%%%%%%%%%%%%%%%%%%%%%%%%%%%%%%%%%%%%%%%%%%%%%%%%%%%%%%%%%%%%%%%%%%%%%%%%%%%%
\subsection{Include by Input}
\label{sec:input}

Including child documents by |\include| has some restrictions by design.
Most notably, the content of a child document always occupies
its own set of pages; pages cannot be shared between child documents.
Usually, this behaviour makes perfect sense
because each child document contain an essential part of the document.
However, in some situations it may be desirable to compose
a document from a collection of parts
without having mandatory page breaks between then.
For this case, the package
provides a mechanism to include parts
by |\input| which can also be processed individually.
However, by construction this mechanism
requires manual handling of the content to be output.

%%%%%%%%%%%%%%%%%%%%%%%%%%%%%%%%%%%%%%%%
\DescribeMacro{\ifchilddocmanual}
The main file should be prepared as usual, see \secref{sec:include}.
However, the document body must make a distinction
between processing of an individual part and of the main document, e.g.:
%
\begin{center}
\begin{tabular}{l}
|\ifchilddocmanual|\\
|\input{\childdocname}|\\
|\||else|\\
\textit{document body with }|\input{|\textit{part}|}|\\
|\||fi|
\end{tabular}
\end{center}
%
The conditional |\ifchilddocmanual| is true whenever
a part to be included by |\input| is being compiled,
and the name of the part is stored in |\childdocname|.

%%%%%%%%%%%%%%%%%%%%%%%%%%%%%%%%%%%%%%%%
\DescribeMacro{\childdocby}
Each part to be included by |\input| should start with:
%
\begin{center}
\begin{tabular}{l}
|\input{childdoc.def}|\\
|\childdocby{|\textit{main}|}|\\
\end{tabular}
\end{center}
%
The directive |\childdocby| is similar to |\childdocof|
described in \secref{sec:include},
but the subsequent selection of content must be done manually.
To that end, both |\ifchilddoc| and |\ifchilddocmanual|
will be true upon processing of a part,
and the name of the part is stored in |\childdocname|.
Note that |\jobname| will be set to the filename of the current part
so that each part receives an individual |.aux| file
that does not interfere with the |.aux| file(s) of the main document.
This behaviour can be altered by the alternative form
|\childdocby[*]{|\textit{main}|}| (with a non-empty optional argument)
which uses the |.aux| file of the main document
by setting |\jobname| to \textit{main}.

%%%%%%%%%%%%%%%%%%%%%%%%%%%%%%%%%%%%%%%%%%%%%%%%%%%%%%%%%%%%%%%%%%%%%%%%%%%%%%%%
\subsection{Driver Development}
\label{sec:driver}

The \textsf{childdoc} mechanism can also be use for the development
of definition files such as \LaTeX{} styles or classes.
This case differs from the above setup with multiple parts
included by |\include| in that no |\includeonly| should be invoked.
This can be achieved by starting the include file
(before |\ProvidesPackage|) with:
%
\begin{center}
\begin{tabular}{l}
|\input{childdoc.def}|\\
|\childdocforward{|\textit{main}|}|\\
\end{tabular}
\end{center}
%
or alternatively with:
%
\begin{center}
\begin{tabular}{l}
|\input{childdoc.def}|\\
|\childdocby{|\textit{main}|}|\\
\end{tabular}
\end{center}
%
Both forms have slightly different effects as described above.
The main file is prepared as usual, see \secref{sec:include}.

%%%%%%%%%%%%%%%%%%%%%%%%%%%%%%%%%%%%%%%%%%%%%%%%%%%%%%%%%%%%%%%%%%%%%%%%%%%%%%%%
\subsection{Legacy Detection}
\label{sec:detection}

The directive |\childdocmain| in the main file can detect
whether the complete document or merely a child is to be compiled
even without using the directive |\childdocof|.
This method is deprecated because it is less robust
and there is no compelling reason to use it;
it is merely provided for backward compatibility
and it may be removed in future versions.

If the detection mechanism is to be used,
it is mandatory to correctly specify
the filename of the main file as the argument of |\childdocmain|:
%
\begin{center}
\begin{tabular}{l}
|\input{childdoc.def}|\\
|\childdocmain{|\textit{main}|}|\\
\end{tabular}
\end{center}
%
If |\jobname| does not match the argument \textit{main} of |\childdocmain|,
it is assumed that |\jobname| points to the child file to be compiled.
When using |\childdocmain| with the main file specified as argument,
it suffices to start a child file
with just |\input{|\textit{main}|}|
without loading of the package and using |\childdocof|.
If instead all processing is done
with the appropriate \textsf{childdoc} directives,
the argument of \textit{main} of |\childdocmain| can be empty.

An alternative version of the command line processing described
in \secref{sec:commandline} using the detection mechanism reads:
%
\begin{center}
|... -jobname "|\textit{target}|" "|[\textit{flags}]%
[|\def\jobname{|\textit{dest}|}|]|\input{|\textit{main}|}"|
\end{center}

%%%%%%%%%%%%%%%%%%%%%%%%%%%%%%%%%%%%%%%%%%%%%%%%%%%%%%%%%%%%%%%%%%%%%%%%%%%%%%%%
\subsection{Manual Code}
\label{sec:manual}

In case one cannot be certain whether the definitions file |childdoc.def|
is installed on the target \TeX{} distribution
and one prefers not to ship it,
it is conceivable to paste a few relevant commands into the sources.

To that end, drop all statements |\input{childdoc.def}|
and perform the replacements as outlined below.
Instead of |\childdocmain{|\textit{main}|}| add the following code
to the top of the main file:
%
\begin{center}
\begin{tabular}{l}
|\||ifdefined\childdocname\endinput\||fi\newif\ifchilddoc|\\
|\edef\childdocname{\scantokens\expandafter{\jobname\noexpand}}|\\
|\def\childdocmain{|\textit{main}|}\||ifx\childdocmain\childdocname\||else|\\
|\childdoctrue\includeonly{\childdocname}\let\jobname\childdocmain\||fi|\\
\end{tabular}
\end{center}
%
Instead of |\childdocof{|\textit{main}|}| just include the main file
at the top of each child file:
%
\begin{center}
|\input{|\textit{main}|}|
\end{center}
%
A simple redirection |\childdocforward{|\textit{dest}|}| is achieved by:
%
\begin{center}
|\def\jobname{|\textit{dest}|}\input{\jobname}|
\end{center}
%
The redirection with prefix
|\childdocforwardprefix[|\textit{prefix}|]{|\textit{dest}|}|
is accomplished by:
%
\begin{center}
\begin{tabular}{l}
|{\edef\jobname{\scantokens\expandafter{\jobname\noexpand}}|\\
|\def\redirectjob |\textit{prefix}|#1~~~{\gdef\jobname{|\textit{dest}|#1}}|\\
|\expandafter\redirectjob\jobname~~~}\input{\jobname}|
\end{tabular}
\end{center}

In an alternative approach,
child documents can be compiled by a specific command line
without additional code or specific definitions:
%
\begin{center}
|... -jobname "|\textit{target}|" "|[\textit{flags}]%
|\includeonly{|\textit{dest}|}\input{|\textit{main}|}"|
\end{center}
%

%%%%%%%%%%%%%%%%%%%%%%%%%%%%%%%%%%%%%%%%%%%%%%%%%%%%%%%%%%%%%%%%%%%%%%%%%%%%%%%%
%%%%%%%%%%%%%%%%%%%%%%%%%%%%%%%%%%%%%%%%%%%%%%%%%%%%%%%%%%%%%%%%%%%%%%%%%%%%%%%%
\section{Information}

%%%%%%%%%%%%%%%%%%%%%%%%%%%%%%%%%%%%%%%%%%%%%%%%%%%%%%%%%%%%%%%%%%%%%%%%%%%%%%%%
\subsection{Copyright}

Copyright \copyright{} 2017--2018 Niklas Beisert

This work may be distributed and/or modified under the
conditions of the \LaTeX{} Project Public License, either version 1.3
of this license or (at your option) any later version.
The latest version of this license is in
  \url{http://www.latex-project.org/lppl.txt}
and version 1.3 or later is part of all distributions of \LaTeX{}
version 2005/12/01 or later.

This work has the LPPL maintenance status `maintained'.

The Current Maintainer of this work is Niklas Beisert.

This work consists of the files |README.txt|, |childdoc.ins| and |childdoc.dtx|
as well as the derived files |childdoc.def|, |cdocsamp.tex|
with |cdocsch1.tex|, |cdocsch2.tex|, |cdocspt3.tex|, |cdocspt4.tex|,
|cdocsdrf.tex|, |cdocsfn1.tex|, |cdocsfn2.tex|
as well as |childdoc.pdf|.

%%%%%%%%%%%%%%%%%%%%%%%%%%%%%%%%%%%%%%%%%%%%%%%%%%%%%%%%%%%%%%%%%%%%%%%%%%%%%%%%
\subsection{Files and Installation}

The package consists of the files:
%
\begin{center}
\begin{tabular}{ll}
    |README.txt|   & readme file \\
    |childdoc.ins| & installation file \\
    |childdoc.dtx| & source file \\
    |childdoc.def| & definition file \\
    |cdocsamp.tex| & sample main file \\
    |cdocsch1.tex| & sample include file \\
    |cdocsch2.tex| & sample include file \\
    |cdocspt3.tex| & sample part file \\
    |cdocspt4.tex| & sample part file \\
    |cdocsdrf.tex| & sample redirection file \\
    |cdocsfn1.tex| & sample redirection file \\
    |cdocsfn2.tex| & sample redirection file \\
    |childdoc.pdf| & manual
\end{tabular}
\end{center}
%
The distribution consists of the files
|README.txt|, |childdoc.ins| and |childdoc.dtx|.
%
\begin{itemize}
\item
Run (pdf)\LaTeX{} on |childdoc.dtx|
to compile the manual |childdoc.pdf| (this file).
\item
Run \LaTeX{} on |childdoc.ins| to create the definitions file |childdoc.def|
and the sample |cdocsamp.tex| with include files
|cdocsch1.tex|, |cdocsch2.tex|, |cdocspt3.tex|, |cdocspt4.tex|,
|cdocsdrf.tex|, |cdocsfn1.tex|, |cdocsfn2.tex|.
Then copy the file |childdoc.def| to an appropriate directory of your \LaTeX{}
distribution, e.g.\ \textit{texmf-root}|/tex/latex/childdoc|.
\end{itemize}

%%%%%%%%%%%%%%%%%%%%%%%%%%%%%%%%%%%%%%%%%%%%%%%%%%%%%%%%%%%%%%%%%%%%%%%%%%%%%%%%
\subsection{Related CTAN Packages}

There are several other packages which offer a similar functionality:
%
\begin{itemize}
\item
The packages
\href{http://ctan.org/pkg/docmute}{\textsf{docmute}},
\href{http://ctan.org/pkg/includex}{\textsf{includex}} and
\href{http://ctan.org/pkg/standalone}{\textsf{standalone}}
provide commands to include only the document body of
a child file thus allowing both files to be compiled individually.
\item
The packages \href{http://ctan.org/pkg/subdocs}{\textsf{subdocs}}
and \href{http://ctan.org/pkg/subfiles}{\textsf{subfiles}}
provide structures in which the main and child documents can be
encapsulated and allowing them to be compiled individually.
The inclusion mechanism is different from the conventional |\include|.
\item
The package \href{http://ctan.org/pkg/combine}{\textsf{combine}}
is an elaborate solution to combine several documents into one.
\end{itemize}
%
See also the CTAN topic \href{http://ctan.org/topic/subdocs}{\textsf{subdocs}}
for further related packages.
The present package differs from the above solutions in that
a document structure constructed with the conventional |\include| mechanism
just needs two extra commands at the top of every file
such that all constituent files can be compiled individually.

%%%%%%%%%%%%%%%%%%%%%%%%%%%%%%%%%%%%%%%%%%%%%%%%%%%%%%%%%%%%%%%%%%%%%%%%%%%%%%%%
%\subsection{Feature Suggestions}
%
%The following is a list of features which may be useful for future
%versions of this package:
%%
%\begin{itemize}
%\item
%\ldots
%\end{itemize}

%%%%%%%%%%%%%%%%%%%%%%%%%%%%%%%%%%%%%%%%%%%%%%%%%%%%%%%%%%%%%%%%%%%%%%%%%%%%%%%%
\subsection{Revision History}

%%%%%%%%%%%%%%%%%%%%%%%%%%%%%%%%%%%%%%%%
\paragraph{v2.0:} 2018/12/30

\begin{itemize}
\item
immediate forward processing
\item
added |\childdocby| mechanism
\item
manual restructured
\end{itemize}

%%%%%%%%%%%%%%%%%%%%%%%%%%%%%%%%%%%%%%%%
\paragraph{v1.6:} 2018/01/17

\begin{itemize}
\item
application for development of include files
\item
corrections to manual
\end{itemize}

%%%%%%%%%%%%%%%%%%%%%%%%%%%%%%%%%%%%%%%%
\paragraph{v1.5:} 2017/05/21

\begin{itemize}
\item
more complete structuring introduced
\item
|\childdocof| introduced
\item
|\childdoc| renamed to |\childdocmain|
\item
|\childredirect| renamed to |\childdocforward| and |\childdocforwardprefix|
and functionality expanded
\end{itemize}

%%%%%%%%%%%%%%%%%%%%%%%%%%%%%%%%%%%%%%%%
\paragraph{v1.0:} 2017/04/27

\begin{itemize}
\item
manual and install package
\item
first version published on CTAN
\end{itemize}

%%%%%%%%%%%%%%%%%%%%%%%%%%%%%%%%%%%%%%%%
\paragraph{v0.6:} 2017/04/26

\begin{itemize}
\item
redirection mechanism added
\end{itemize}

%%%%%%%%%%%%%%%%%%%%%%%%%%%%%%%%%%%%%%%%
\paragraph{v0.5:} 2017/04/26

\begin{itemize}
\item
functionality in definition file
\end{itemize}


%%%%%%%%%%%%%%%%%%%%%%%%%%%%%%%%%%%%%%%%%%%%%%%%%%%%%%%%%%%%%%%%%%%%%%%%%%%%%%%%
%%%%%%%%%%%%%%%%%%%%%%%%%%%%%%%%%%%%%%%%%%%%%%%%%%%%%%%%%%%%%%%%%%%%%%%%%%%%%%%%
%%%%%%%%%%%%%%%%%%%%%%%%%%%%%%%%%%%%%%%%%%%%%%%%%%%%%%%%%%%%%%%%%%%%%%%%%%%%%%%%
\appendix

\settowidth\MacroIndent{\rmfamily\scriptsize 000\ }

 \DocInput{childdoc.dtx}

\end{document}
%</driver>
% \fi
%
% %%%%%%%%%%%%%%%%%%%%%%%%%%%%%%%%%%%%%%%%%%%%%%%%%%%%%%%%%%%%%%%%%%%%%%%%%%%%%%
% %%%%%%%%%%%%%%%%%%%%%%%%%%%%%%%%%%%%%%%%%%%%%%%%%%%%%%%%%%%%%%%%%%%%%%%%%%%%%%
% \section{Sample}
%\iffalse
%<*samplemain>
%\fi
%
% The following presents a sample document
% with two chapters, two parts, a title page,
% a compile flag as well as three forwarding files to set the flag.
% It consists of eight |.tex| files:
% \begin{center}
% \begin{tabular}{ll}
% |cdocsamp.tex|&main file\\
% |cdocsch1.tex|&include file for chapter 1\\
% |cdocsch2.tex|&include file for chapter 2\\
% |cdocspt3.tex|&include file for part 3\\
% |cdocspt4.tex|&include file for part 4\\
% |cdocsdrf.tex|&forwarding file for main file in draft mode\\
% |cdocsfi1.tex|&forwarding file for final version of chapter 1\\
% |cdocsfi2.tex|&forwarding file for final version of chapter 2\\
% \end{tabular}
% \end{center}
% Each of the eight files can be compiled directly by the \LaTeX{} compiler.
%
% %%%%%%%%%%%%%%%%%%%%%%%%%%%%%%%%%%%%%%
% \paragraph{Main File.}
%
% The main file is called |cdocsamp.tex|.
%
% Load the \textsf{childdoc} definitions and
% declare the filename for the main document:
%    \begin{macrocode}
\input{childdoc.def}
\childdocmain{}
%    \end{macrocode}

% Optional override for |\version| flag:
%    \begin{macrocode}
%%\ifchilddoc\else\providecommand{\version}{draft}\fi
%    \end{macrocode}

% Define the default values for the |\version| flag
% (|final| for the main file and |draft| for childs):
%    \begin{macrocode}
\ifchilddoc
\providecommand{\version}{draft}
\else
\providecommand{\version}{final}
\fi
%    \end{macrocode}

% Load the standard document class:
%    \begin{macrocode}
\documentclass[12pt]{article}
%    \end{macrocode}

% Start the document body:
%    \begin{macrocode}
\begin{document}
%    \end{macrocode}

% Declare a title page.
% Print title, part of document being processed and version flag:
%    \begin{macrocode}
\addtocounter{page}{-1}
\begin{center}
{\LARGE\bfseries{}childdoc example\par}
\vspace{1cm}
\ifchilddoc
\ifchilddocmanual part\else chapter\fi:
`\childdocname' of `\childdocjob'\par
\else
main document: `\childdocjob'\par
\fi
version: \version\par
\end{center}
\newpage
%    \end{macrocode}

% Manually include selected file,
% otherwise process as usual:
%    \begin{macrocode}
\ifchilddocmanual
\section*{part `\childdocname'}
\input{\childdocname}
\else
%    \end{macrocode}

% Include the two chapters:
%    \begin{macrocode}
\include{cdocsch1}
\include{cdocsch2}
%    \end{macrocode}

% Include the two parts unless only chapters should be displayed:
%    \begin{macrocode}
\ifchilddoc\else
\section{part three}
\input{cdocspt3}
\section{part four}
\input{cdocspt4}
\fi
%    \end{macrocode}

% Process as usual until here:
%    \begin{macrocode}
\fi
%    \end{macrocode}

% End of document body:
%    \begin{macrocode}
\end{document}
%    \end{macrocode}
%\iffalse
%</samplemain>
%\fi
%
% %%%%%%%%%%%%%%%%%%%%%%%%%%%%%%%%%%%%%%
% \paragraph{Chapter Include Files.}
%
% The include files are called |cdocsch1.tex| and |cdocsch2.tex|.
%
%\iffalse
%<*samplechap1|samplechap2>
%\fi

% Optional override for |\version| flag:
%    \begin{macrocode}
%%\providecommand{\version}{final}
%    \end{macrocode}

% Include the main document:
%    \begin{macrocode}
\input{childdoc.def}
\childdocof{cdocsamp}
%    \end{macrocode}

%\iffalse
%</samplechap1|samplechap2>
%\fi
%
%\iffalse
%<*samplechap1>
%\fi
% Some text for chapter 1:
%    \begin{macrocode}
\section{one}
some text in chapter one
%    \end{macrocode}

%\iffalse
%</samplechap1>
%\fi
% Some text for chapter 2:
%\iffalse
%<*samplechap2>
%\fi
%    \begin{macrocode}
\section{two}
more text in chapter two
%    \end{macrocode}

%\iffalse
%</samplechap2>
%\fi
%
% %%%%%%%%%%%%%%%%%%%%%%%%%%%%%%%%%%%%%%
% \paragraph{Part Include Files.}
%
% The include files are called |cdocspt3.tex| and |cdocspt4.tex|.
%
%\iffalse
%<*samplepart3|samplepart4>
%\fi

% Optional override for |\version| flag:
%    \begin{macrocode}
%%\providecommand{\version}{final}
%    \end{macrocode}

% Include the main document:
%    \begin{macrocode}
\input{childdoc.def}
\childdocby{cdocsamp}
%    \end{macrocode}

%\iffalse
%</samplepart3|samplepart4>
%\fi
%
%\iffalse
%<*samplepart3>
%\fi
% Some text for part 3:
%    \begin{macrocode}
some text in part three
%    \end{macrocode}

%\iffalse
%</samplepart3>
%\fi
% Some text for part 4:
%\iffalse
%<*samplepart4>
%\fi
%    \begin{macrocode}
more text in part four
%    \end{macrocode}

%\iffalse
%</samplepart4>
%\fi
%
% %%%%%%%%%%%%%%%%%%%%%%%%%%%%%%%%%%%%%%
% \paragraph{Forwarding for a Complete Draft.}
%
% The following forwarding file |cdocsdrf.tex|
% compiles the main document in draft mode:
%\iffalse
%<*sampledraft>
%\fi
%    \begin{macrocode}
\def\version{draft}
\input{childdoc.def}
\childdocforward{cdocsamp}
%    \end{macrocode}

%\iffalse
%</sampledraft>
%\fi
%
% %%%%%%%%%%%%%%%%%%%%%%%%%%%%%%%%%%%%%%
% \paragraph{Forwarding for Final Version of the Chapters.}
%
% The following forwarding files |cdocsfn1.tex| and |cdocsfn2.tex|
% (with identical content)
% compile the final versions of the child documents
% |cdocsch1.tex| and |cdocsch2.tex|, respectively:
%\iffalse
%<*samplefinal>
%\fi
%    \begin{macrocode}
\def\version{final}
\input{childdoc.def}
\childdocforwardprefix[cdocsamp]{cdocsfn}{cdocsch}
%    \end{macrocode}

%\iffalse
%</samplefinal>
%\fi
%
% %%%%%%%%%%%%%%%%%%%%%%%%%%%%%%%%%%%%%%
% \paragraph{Command Line Processing.}
%
% The following three command lines generate the output files
% |cdocscld|, |cdocscl1| and |cdocscl2|
% which should be identical to
% |cdocsdrf|, |cdocsch1| and |cdocsfn2|, respectively:
% \begin{center}
% \begin{tabular}{l}
% |latex -jobname cdocscld \|\\
% |  "\def\version{draft}\input{childdoc.def}\childdocforward{cdocsamp}"|\\
% |latex -jobname cdocscl1 \|\\
% |  "\input{childdoc.def}\childdocforward[cdocsamp]{cdocsch1}"|\\
% |latex -jobname cdocscl2 \|\\
% |  "\def\version{final}\input{childdoc.def}\childdocforward{cdocsch2}"|
% \end{tabular}
% \end{center}
% Note that the trailing backslash on each first line
% merely continues the input to the second line
% (for convenient cut ant paste).
% Furthermore, the command |latex| can be replaced by any
% of its alternative versions such as |pdflatex|.
%
% %%%%%%%%%%%%%%%%%%%%%%%%%%%%%%%%%%%%%%%%%%%%%%%%%%%%%%%%%%%%%%%%%%%%%%%%%%%%%%
% %%%%%%%%%%%%%%%%%%%%%%%%%%%%%%%%%%%%%%%%%%%%%%%%%%%%%%%%%%%%%%%%%%%%%%%%%%%%%%
% \section{Implementation}
%\iffalse
%<*package>
%\fi
%
% This section describes the definitions file |childdoc.def|.

% The definitions cannot be loaded using |\usepackage| or |\RequirePackage|
% which has a mechanism to prevent loading a style file more than once.
% When loading the definitions by means of |\input|
% multiple instances have to be prevented manually:
%\iffalse
%This code needs to be before the `\ProvidesFile' directive
%which is defined at the beginning of this file.
%Therefore it is also placed there and commented out here.
%</package>
%<*discard>
%\fi
%    \begin{macrocode}
\ifdefined\childdocmain\endinput\fi
%    \end{macrocode}
%\iffalse
%</discard>
%<*package>
%\fi
%
% \macro{\ifchilddoc}
% \macro{\ifchilddocmanual}
% The conditional |\ifchilddoc| tells whether a
% child (true) or main (false) document is being compiled.
% The conditional |\ifchilddocmanual| tells whether
% the |\includeonly| mechanism is used (false) or
% the selection of child files must be performed manually (true).
% The definitions initialise to false:
%    \begin{macrocode}
\newif\ifchilddoc
\newif\ifchilddocmanual
%    \end{macrocode}

% \macro{\childdocname}
% \macro{\childdocjob}
% The macro |\childdocname| stores the name of the main document
% to be compiled. The macro |\childdocjob| stores the name of
% the document on which the \LaTeX{} compiler was originally invoked.
% The content of |\jobname| cannot be compared
% to filenames specified in the source due to different catcodes.
% The following code rescans |\jobname|, stores the result
% in |\childdocname| and saves a copy in |\childdocjob|:
%    \begin{macrocode}
\edef\childdocname{\scantokens\expandafter{\jobname\noexpand}}
\let\childdocjob\childdocname
%    \end{macrocode}

% \macro{\childdocdisable}
% The macro |\childdocdisable| prevents the main file
% from being processed more than once.
% At this stage, the main document command |\childdocmain|
% is assumed to be called once again where it should do nothing.
% Any subsequent call to it should prevent
% a secondary processing of the main document
% It overwrites the forwarding commands
% |\childdocof| and |\childdocforward|
% with empty macros to prevent further inclusions of the main document:
%    \begin{macrocode}
\newcommand{\childdocdisable}
{
  \renewcommand{\childdocmain}[1]{\renewcommand{\childdocmain}[1]{\endinput}}
  \renewcommand{\childdocof}[1]{}
  \renewcommand{\childdocby}[2][]{}
  \renewcommand{\childdocforward}[2][]{}
  \renewcommand{\childdocdisable}{}
}
%    \end{macrocode}

% \macro{\childdocmain}
% The macro |\childdocmain| is to be called at the top of the main file
% with nothing or the main filename (without extension) as argument.
% First, it breaks loops.
% If the argument is not empty and does not match |\childdocname|
% (which is set by the first inclusion of |childdoc.def|),
% |\ifchilddoc| is set to true, |\includeonly| is applied to the child file
% and |\jobname| is set to the main file
% (for proper handling of |.aux| files):
%    \begin{macrocode}
\newcommand{\childdocmain}[1]
{
  \childdocdisable\childdocmain{}
  \if?#1?\else
    \begingroup
      \def\childdoctmp{#1}
      \ifx\childdoctmp\childdocname
        \def\childdoctmp{}
      \else
        \def\childdoctmp
        {
          \childdoctrue
          \includeonly{\childdocname}
          \def\childdocjob{#1}
          \def\jobname{#1}
        }
      \fi
      \expandafter
    \endgroup
    \childdoctmp
  \fi
}
%    \end{macrocode}

% \macro{\childdocof}
% The command |\childdocof| redirects
% compilation to the main file |#1|.
%    \begin{macrocode}
\newcommand{\childdocof}[1]
{
  \childdocdisable
  \childdoctrue
  \includeonly{\childdocname}
  \def\jobname{#1}
  \def\childdocjob{#1}
  \input{#1}
}
%    \end{macrocode}

% \macro{\childdocby}
% The command |\childdocby| ....
%    \begin{macrocode}
\newcommand{\childdocby}[2][]
{
  \childdocdisable
  \childdoctrue
  \childdocmanualtrue
  \if?#1?\else
    \def\jobname{#2}
  \fi
  \def\childdocjob{#2}
  \input{#2}
  \endinput
}
%    \end{macrocode}

% \macro{\childdocforward}
% The command |\childdocforward| redirects
% compilation to the main file or
% (if the optional argument is given) a child file.
% Parameters are set as if the main file
% or a child file starting with |\childdocof| was compiled.
% Then compilation is handed over to the main file:
%    \begin{macrocode}
\newcommand{\childdocforward}[2][]
{
  \begingroup
    \if?#1?
      \def\childdoctmp
      {
        \def\childdocname{#2}
        \def\childdocjob{#2}
        \def\jobname{#2}
        \input{#2}
        \endinput
      }
    \else
      \def\childdoctmp
      {
        \childdocdisable
        \def\childdocname{#2}
        \childdoctrue
        \includeonly{#2}
        \def\childdocjob{#1}
        \def\jobname{#1}
        \input{#1}
        \endinput
      }
    \fi
    \expandafter
  \endgroup
  \childdoctmp
}
%    \end{macrocode}

% \macro{\childdocforwardprefix}
% The command |\childdocforwardprefix| redirects
% compilation to the main or a child file by means of a pattern.
% The prefix |#1| in the current filename is replaced by |#2|
% and the suffix of the current filename is kept
% (it is assumed that the filename does not contain the substring `|~~~|'
% which is used as a delimiter).
% Compilation is handed over to the new file by |\childdocforward|:
%    \begin{macrocode}
\newcommand{\childdocforwardprefix}[3][]
{
  \begingroup
    \def\childdocextract #2##1~~~{\def\childdoctmp{\childdocforward[#1]{#3##1}}}
    \expandafter\childdocextract\childdocname~~~
    \expandafter
  \endgroup
  \childdoctmp
}
%    \end{macrocode}

% \macro{\childdoc}
% The deprecated macro |\childdoc| is a legacy version of |\childdocmain|:
%    \begin{macrocode}
\newcommand{\childdoc}{\childdocmain}
%    \end{macrocode}

% \macro{\childdocredirect}
% The deprecated macro |\childdocredirect| is a legacy version
% of |\childdocforward| and |\childdocforwardprefix|:
%    \begin{macrocode}
\newcommand{\childdocredirect}[2][]
{
  \begingroup
    \if?#1?
      \def\childdoctmp{\childdocforward{#2}}
    \else
      \def\childdoctmp{\childdocforwardprefix{#1}{#2}}
    \fi
    \expandafter
  \endgroup
  \childdoctmp
}
%    \end{macrocode}

%\iffalse
%</package>
%\fi
%
\endinput
|\\
|\childdocforwardprefix[|\textit{main}|]{|\textit{prefix}|}{|\textit{dest}|}|
\end{tabular}
\end{center}
%
the destination file is determined by a pattern
depending on the current file:
To make this work, the current file must be called
`{\textit{prefix}\hspace{0.2em}\textit{suffix}}'
with \textit{prefix} matching precisely the argument.
Processing is then passed on to the file
`{\textit{dest}\hspace{0.2em}\textit{suffix}}'.
Surely, the same effect is achieved by
directly specifying the
argument `{\textit{dest}\hspace{0.2em}\textit{suffix}}'
in the first form.
However, that requires to set up a different file
for each child. With the alternative form of the command
all these files can have exactly the same content
which simplifies setting them up and maintaining them.

For example, the following file |draft.tex|
with a compilation flag |\version| as described in \secref{sec:flags}
compiles the main document as a draft:
%
\begin{center}
\begin{tabular}{l}
|\def\version{draft}|\\
|% \iffalse
%
% childdoc.dtx Copyright (C) 2017-2018 Niklas Beisert
%
% This work may be distributed and/or modified under the
% conditions of the LaTeX Project Public License, either version 1.3
% of this license or (at your option) any later version.
% The latest version of this license is in
%   http://www.latex-project.org/lppl.txt
% and version 1.3 or later is part of all distributions of LaTeX
% version 2005/12/01 or later.
%
% This work has the LPPL maintenance status `maintained'.
%
% The Current Maintainer of this work is Niklas Beisert.
%
% This work consists of the files childdoc.dtx and childdoc.ins
% and the derived files childdoc.def and cdocsamp.tex with
% cdocsch1.tex, cdocsch2.tex, cdocsdrf.tex, cdocsfn1.tex, cdocsfn2.tex.
%
%<package>\ifdefined\childdocmain\endinput\fi
%<package>\ProvidesFile{childdoc.def}[2018/12/30 v2.0 child document driver]
%<samplemain>\ProvidesFile{cdocsamp.tex}[2018/12/30 v2.0 sample for childdoc]
%<*driver>
%\ProvidesFile{childdoc.drv}[2018/12/30 v2.0 childdoc reference manual file]
\PassOptionsToClass{10pt,a4paper}{article}
\documentclass{ltxdoc}

\usepackage[margin=35mm]{geometry}
\usepackage{hyperref}
\usepackage{hyperxmp}
\usepackage[usenames]{color}

\hypersetup{colorlinks=true}
\hypersetup{pdfstartview=FitH}
\hypersetup{pdfpagemode=UseNone}
\hypersetup{pdfsource={}}
\hypersetup{pdflang={en-UK}}
\hypersetup{pdfcopyright={Copyright 2017-2018 Niklas Beisert.
  This work may be distributed and/or modified under the
  conditions of the LaTeX Project Public License, either version 1.3
  of this license or (at your option) any later version.}}
\hypersetup{pdflicenseurl={http://www.latex-project.org/lppl.txt}}
\hypersetup{pdfcontactaddress={ETH Zurich, ITP, HIT K,
  Wolfgang-Pauli-Strasse 27}}
\hypersetup{pdfcontactpostcode={8093}}
\hypersetup{pdfcontactcity={Zurich}}
\hypersetup{pdfcontactcountry={Switzerland}}
\hypersetup{pdfcontactemail={nbeisert@itp.phys.ethz.ch}}
\hypersetup{pdfcontacturl={http://people.phys.ethz.ch/\xmptilde nbeisert/}}

\newcommand{\secref}[1]{\hyperref[#1]{section \ref*{#1}}}

\parskip1ex
\parindent0pt
\let\olditemize\itemize
\def\itemize{\olditemize\parskip0pt}

\begin{document}

\title{The \textsf{childdoc} Package}
\hypersetup{pdftitle={The childdoc Package}}
\author{Niklas Beisert\\[2ex]
  Institut f\"ur Theoretische Physik\\
  Eidgen\"ossische Technische Hochschule Z\"urich\\
  Wolfgang-Pauli-Strasse 27, 8093 Z\"urich, Switzerland\\[1ex]
  \href{mailto:nbeisert@itp.phys.ethz.ch}
  {\texttt{nbeisert@itp.phys.ethz.ch}}}
\hypersetup{pdfauthor={Niklas Beisert}}
\hypersetup{pdfsubject={Manual for the LaTeX2e Package childdoc}}
\date{30 December 2018, \textsf{v2.0}}
\maketitle

\begin{abstract}\noindent
\textsf{childdoc} is a \LaTeXe{} package
that enables the direct compilation
of document sections included by |\include|
to individual files.
\end{abstract}

\begingroup
\parskip0ex
\tableofcontents
\endgroup

%%%%%%%%%%%%%%%%%%%%%%%%%%%%%%%%%%%%%%%%%%%%%%%%%%%%%%%%%%%%%%%%%%%%%%%%%%%%%%%%
%%%%%%%%%%%%%%%%%%%%%%%%%%%%%%%%%%%%%%%%%%%%%%%%%%%%%%%%%%%%%%%%%%%%%%%%%%%%%%%%
\section{Introduction}

\LaTeX{} provides a mechanism to structure a large document (such as a book)
into a main file and several child files (containing the chapters)
using the |\include| command.
This mechanism is beneficial for documents
which span hundreds of pages in order to
make the source file(s) more manageable.
Moreover, compilation can be restricted to
selected child files by means of the |\includeonly| command.
The latter feature can be used to reduce the compilation time while editing
(this was significantly more useful in the earlier days of \LaTeX{})
or to generate a smaller document which is easier to navigate.
Another application of |\includeonly| is to generate
documents consisting of selected parts of the complete document.

However, there are a few drawbacks of the plain |\include| mechanism:
\begin{itemize}
\item
The child files cannot be compiled on their own,
they can only be compiled via the main file.
A naive editing environment
(such as a text editor with an option
to have the current file processed by \LaTeX)
may require one to switch to the main file before compiling;
attempting to compile the child file produces errors.
\item
The main file must be modified (each time)
to adjust the |\includeonly| command
to the present needs. This easily leaves the main file in a messy state.
\item
The generated document will always carry the filename
of the main document. This is inconvenient if
several child files are to be compiled and
to be kept for distribution.
\end{itemize}

The present package provides a simple interface
to make child files individually compilable by \LaTeX{}.
Compiling a child file then has the same effect as compiling
the main file with an |\includeonly| command
to select the appropriate child.
Moreover the generated document will carry the name of the child
rather than the main file.
This resolves all three above issues.

This feature is meant to make the editing of books,
thesis documents and lecture notes somewhat more convenient.
However, the package can also be used efficiently for
composing a series of documents (such as exercise sheets)
which are typically distributed individually.
It then assists the author in generating the individual documents
(potentially in different versions)
as well as a document containing the collected series.
Another application is in developing style files
or other kinds of included material
where compilation of the style file could redirect
to a sample or test file.

%%%%%%%%%%%%%%%%%%%%%%%%%%%%%%%%%%%%%%%%%%%%%%%%%%%%%%%%%%%%%%%%%%%%%%%%%%%%%%%%
%%%%%%%%%%%%%%%%%%%%%%%%%%%%%%%%%%%%%%%%%%%%%%%%%%%%%%%%%%%%%%%%%%%%%%%%%%%%%%%%
\section{Usage}

First of all, the package \textsf{childdoc} is \emph{not} a standard
\LaTeXe{} |.sty| style file! Therefore it needs to be invoked in
a non-standard way.

%%%%%%%%%%%%%%%%%%%%%%%%%%%%%%%%%%%%%%%%%%%%%%%%%%%%%%%%%%%%%%%%%%%%%%%%%%%%%%%%
\subsection{Included Files}
\label{sec:include}

%%%%%%%%%%%%%%%%%%%%%%%%%%%%%%%%%%%%%%%%
\DescribeMacro{\childdocmain}
To use the package, add the commands
\begin{center}
\begin{tabular}{l}
|\input{childdoc.def}|\\
|\childdocmain{}|\\
\end{tabular}
\end{center}
at the very top of the main \LaTeX{} file,
in particular \emph{before} the |\documentclass| statement!
The argument of |\childdocmain| should be left empty
(but it must be present).

%%%%%%%%%%%%%%%%%%%%%%%%%%%%%%%%%%%%%%%%
\DescribeMacro{\childdocof}
Furthermore, add the commands
\begin{center}
\begin{tabular}{l}
|\input{childdoc.def}|\\
|\childdocof{|\textit{main}|}|\\
\end{tabular}
\end{center}
at the top of every child file \textit{child}
which is included by |\include{|\textit{child}|}|
from within the main file
(or at least for those files to be compiled individually).
The argument \textit{main} must be the filename of the main file.

There are a couple of
considerations in setting up the main and child documents:

%%%%%%%%%%%%%%%%%%%%%%%%%%%%%%%%%%%%%%%%
\paragraph{Restrictions.}

Please note the following restrictions:
\begin{itemize}
\item
|\childdocmain| must be called with one argument \textit{main}
to ensure compatibility with earlier version of the package.
It must either be empty (|\childdocmain{}|)
or precisely match the filename of the main file in which it is specified.
See \secref{sec:detection} for further information.
\item
The filename \textit{main} must be specified without the |.tex| extension.
\item
The filename \textit{main} is case sensitive
(even in case-insensitive file systems)
due to internal string comparison.
\item
The argument \textit{main} should be fully expanded, it cannot be a macro.
\item
Subdirectories and special characters should be avoided in filenames.
\item
The command |\childdocmain{|\textit{main}|}| must be followed by a whitespace.
It should not be followed immediately by another command
or by a comment mark `|%|'.
This is because the \TeX{} parser reads the token immediately following
the argument of |\childdocmain| and puts it
at the beginning of every child section;
however, a white\-space is ignored.
\end{itemize}

%%%%%%%%%%%%%%%%%%%%%%%%%%%%%%%%%%%%%%%%
\paragraph{Content of Main File.}

It is advisable to place all content in the child files included by |\include|.
Any output contained in the main file will appear in all child documents
unless suppressed manually;
it cannot be suppressed automatically by the |\includeonly| directive
and thus should normally be avoided.
A method to include some content in the main file
by means of conditional processing is described in \secref{sec:conditional}.

%%%%%%%%%%%%%%%%%%%%%%%%%%%%%%%%%%%%%%%%
\paragraph{Page Numbering.}

When only a part of the document is compiled,
the appropriate numbering of pages
(as well as other status parameters)
is determined from the |.aux| files.
The latter contain information from previous passes.
However this information needs to propagate through
all intermediate child documents.
Therefore the page numbering in child documents may well
be inconsistent until the complete document is compiled at least once.

A useful (if unconventional) way to always ensure a consistent
page numbering is to restart the numbering in each child document
and denote the pages by `\textit{child}|.|\textit{page}'
where \textit{child} represents the chapter/section number of the child file.
This can be achieved by the command
|\numberwithin{page}{|\textit{child}|}|
of the \textsf{amsmath} package
where \textit{child} can be |chapter| or |section|
depending on the chosen structuring.
Alternatively, one can modify the macro |\thepage| appropriately
and reset the counter |page| at the start of each child file.

%%%%%%%%%%%%%%%%%%%%%%%%%%%%%%%%%%%%%%%%%%%%%%%%%%%%%%%%%%%%%%%%%%%%%%%%%%%%%%%%
\subsection{Conditional Processing}
\label{sec:conditional}

The package provides a mechanism to compile different versions
of a document. To customise the versions further some conditional processing
can come in handy to distinguish which version is being compiled.
The package provides two macros to describe the compilation context:

%%%%%%%%%%%%%%%%%%%%%%%%%%%%%%%%%%%%%%%%
\DescribeMacro{\ifchilddoc}
The conditional |\ifchilddoc| distinguishes between the compilation of
child documents and the main document:
%
\begin{center}
|\ifchilddoc |\textit{child-code}| |[|\||else |\textit{main-code}]| \||fi|
\end{center}

%%%%%%%%%%%%%%%%%%%%%%%%%%%%%%%%%%%%%%%%
\DescribeMacro{\childdocname}
\DescribeMacro{\childdocjob}
The macro |\childdocname| contains the filename (without extension)
of the main or child file being processed.
Note that |\childdocjob| will always contain the name of the main file.

%%%%%%%%%%%%%%%%%%%%%%%%%%%%%%%%%%%%%%%%
\paragraph{Title Page.}

Conditional processing can be used to include a title or banner page
in the main document when proper precautions are taken.
Importantly, the code in the main file should ensure that the page counter
(as well as other status parameters which are stored in the |.aux| files)
takes the same value after the conditional processing.
Otherwise the page numbers may take divergent values
depending on which part is compiled.

For example, a title page could be declared by:
%
\begin{center}
\begin{tabular}{l}
|\ifchilddoc\||else|\\
|\addtocounter{page}{-1}|\\
\textit{code for title page}\\
|\newpage|\\
|\||fi|
\end{tabular}
\end{center}
%
A banner page for the child documents can be generated by:
%
\begin{center}
\begin{tabular}{l}
|\ifchilddoc|\\
|\addtocounter{page}{-1}|\\
\textit{code for banner page}\\
|\newpage|\\
|\||fi|
\end{tabular}
\end{center}
%
Here one could write a message such as:
\begin{center}
|This is the part \childdocname{} of \childdocjob{}.|
\end{center}

%%%%%%%%%%%%%%%%%%%%%%%%%%%%%%%%%%%%%%%%%%%%%%%%%%%%%%%%%%%%%%%%%%%%%%%%%%%%%%%%
\subsection{Flags}
\label{sec:flags}

The package makes it easy to generate different versions
of the main or child documents.
To this end compilation flags can be defined
and assigned different default values.
They will be particularly useful in conjunction
with the forwarding mechanism described in \secref{sec:forward}.

For example, it may be useful to have a flag |\version|
which can be set to |draft| or |final|.
The document source will contain some conditional code
depending on the value of |\version|.
Suppose further, the flag should default to |final| for the main file
and to |draft| for child files
which is a natural assignment for editing the document.
This is achieved by placing the following code
in the preamble of the main document
(below the |\childdocmain| directive):
%
\begin{center}
\begin{tabular}{l}
|\ifchilddoc|\\
|\providecommand{\version}{draft}|\\
|\||else|\\
|\providecommand{\version}{final}|\\
|\||fi|
\end{tabular}
\end{center}
%
The definition by |\providecommand| makes sure
that previous definitions are not overwritten.
Further statements |\providecommand{\version}{...}|
can thus be added before the above code to override it.

For the main file, one might add a line
(between |\childdocmain| and the above block)
%
\begin{center}
|%\ifchilddoc\||else\providecommand{\version}{draft}\||fi|
\end{center}
%
which can be uncommented to produce a draft version.
Likewise one can add a line to the very top of a child file
(above the |\childdocof{|\textit{main}|}| directive)
%
\begin{center}
|%\providecommand{\version}{final}|
\end{center}
%
which can be uncommented to produce the final version of this child document.

%%%%%%%%%%%%%%%%%%%%%%%%%%%%%%%%%%%%%%%%%%%%%%%%%%%%%%%%%%%%%%%%%%%%%%%%%%%%%%%%
\subsection{Forwarding}
\label{sec:forward}

Different versions of the main or child documents
using compilation flags as described in \secref{sec:flags}
can be (permanently) stored in different files
for convenient compilation, viewing and distribution.
To this end, the package defines a command
to pass on compilation to a different file:

%%%%%%%%%%%%%%%%%%%%%%%%%%%%%%%%%%%%%%%%
\DescribeMacro{\childdocforward}
The command |\childdocforward| redirects processing to
another source file:
%
\begin{center}
\begin{tabular}{l}
|\input{childdoc.def}|\\
|\childdocforward[|\textit{main}|]{|\textit{dest}|}|\\
\end{tabular}
\end{center}
%
The argument \textit{dest} is the destination file
(without extension).
It should be the main file or one of the child files.
Note that further \textsf{childdoc} directives
such as |\childdocof| and |\childdocforward|
in the indicated file will be processed in this form.
The optional argument \textit{main}
passes on directly to the main file \textit{main}
while pretending to compile the child \textit{dest}.
This form behaves as if \textit{dest}
issues |\childdocof{|\textit{main}|}| right away,
and no further \textsf{childdoc} directives will be processed.

%%%%%%%%%%%%%%%%%%%%%%%%%%%%%%%%%%%%%%%%
\DescribeMacro{\...prefix}
In the alternative form |\childdocforwardprefix|,
%
\begin{center}
\begin{tabular}{l}
|\input{childdoc.def}|\\
|\childdocforwardprefix[|\textit{main}|]{|\textit{prefix}|}{|\textit{dest}|}|
\end{tabular}
\end{center}
%
the destination file is determined by a pattern
depending on the current file:
To make this work, the current file must be called
`{\textit{prefix}\hspace{0.2em}\textit{suffix}}'
with \textit{prefix} matching precisely the argument.
Processing is then passed on to the file
`{\textit{dest}\hspace{0.2em}\textit{suffix}}'.
Surely, the same effect is achieved by
directly specifying the
argument `{\textit{dest}\hspace{0.2em}\textit{suffix}}'
in the first form.
However, that requires to set up a different file
for each child. With the alternative form of the command
all these files can have exactly the same content
which simplifies setting them up and maintaining them.

For example, the following file |draft.tex|
with a compilation flag |\version| as described in \secref{sec:flags}
compiles the main document as a draft:
%
\begin{center}
\begin{tabular}{l}
|\def\version{draft}|\\
|\input{childdoc.def}|\\
|\childdocforward{|\textit{main}|}|
\end{tabular}
\end{center}
%
Likewise, the following files |final|\textit{nn}|.tex|
compile the final version of the child document
|child|\textit{nn}|.tex|:
%
\begin{center}
\begin{tabular}{l}
|\def\version{final}|\\
|\input{childdoc.def}|\\
|\childdocforwardprefix{final}{child}|
\end{tabular}
\end{center}
%

Note that when several versions of a main file and/or of each child file
are to be generated, it may be convenient to set up a |Makefile| or
shell script to automatise the process.

%%%%%%%%%%%%%%%%%%%%%%%%%%%%%%%%%%%%%%%%%%%%%%%%%%%%%%%%%%%%%%%%%%%%%%%%%%%%%%%%
\subsection{Command Line Processing}
\label{sec:commandline}

The effect of redirection files can also be achieved by invoking
the \LaTeX{} compiler with a more elaborate command line.
Most conveniently this should be done as part
of a shell script or a |Makefile|.

When using \textsf{childdoc} in the main file, the following
command lines effectively perform a redirection
(note that depending on the shell being used,
backslashes may have to be doubled: `|\|' $\to$ `|\\|'):
%
\begin{center}
|... -jobname "|\textit{target}|" |\\|"|[\textit{flags}]%
|\input{childdoc.def}\childdocforward[|\textit{main}|]{|\textit{dest}|}"|
\end{center}
%
Here \textit{target} is the name of the output file,
\textit{main} is the name of the main file
and \textit{dest} is the name of the main or child file to be processed
(all filenames without extensions).
The optional argument \textit{main} can be omitted
if \textit{main} matches \textit{dest}.
Optionally, compilation \textit{flags} can be defined via |\def| commands.
This command line makes the \TeX{} engine believe
it is compiling the file \textit{target}
whose content is specified as the latter parameter.
The provided code then forwards the processing to
\textit{main} or \textit{dest} as described in \secref{sec:forward}.

%%%%%%%%%%%%%%%%%%%%%%%%%%%%%%%%%%%%%%%%%%%%%%%%%%%%%%%%%%%%%%%%%%%%%%%%%%%%%%%%
\subsection{Include by Input}
\label{sec:input}

Including child documents by |\include| has some restrictions by design.
Most notably, the content of a child document always occupies
its own set of pages; pages cannot be shared between child documents.
Usually, this behaviour makes perfect sense
because each child document contain an essential part of the document.
However, in some situations it may be desirable to compose
a document from a collection of parts
without having mandatory page breaks between then.
For this case, the package
provides a mechanism to include parts
by |\input| which can also be processed individually.
However, by construction this mechanism
requires manual handling of the content to be output.

%%%%%%%%%%%%%%%%%%%%%%%%%%%%%%%%%%%%%%%%
\DescribeMacro{\ifchilddocmanual}
The main file should be prepared as usual, see \secref{sec:include}.
However, the document body must make a distinction
between processing of an individual part and of the main document, e.g.:
%
\begin{center}
\begin{tabular}{l}
|\ifchilddocmanual|\\
|\input{\childdocname}|\\
|\||else|\\
\textit{document body with }|\input{|\textit{part}|}|\\
|\||fi|
\end{tabular}
\end{center}
%
The conditional |\ifchilddocmanual| is true whenever
a part to be included by |\input| is being compiled,
and the name of the part is stored in |\childdocname|.

%%%%%%%%%%%%%%%%%%%%%%%%%%%%%%%%%%%%%%%%
\DescribeMacro{\childdocby}
Each part to be included by |\input| should start with:
%
\begin{center}
\begin{tabular}{l}
|\input{childdoc.def}|\\
|\childdocby{|\textit{main}|}|\\
\end{tabular}
\end{center}
%
The directive |\childdocby| is similar to |\childdocof|
described in \secref{sec:include},
but the subsequent selection of content must be done manually.
To that end, both |\ifchilddoc| and |\ifchilddocmanual|
will be true upon processing of a part,
and the name of the part is stored in |\childdocname|.
Note that |\jobname| will be set to the filename of the current part
so that each part receives an individual |.aux| file
that does not interfere with the |.aux| file(s) of the main document.
This behaviour can be altered by the alternative form
|\childdocby[*]{|\textit{main}|}| (with a non-empty optional argument)
which uses the |.aux| file of the main document
by setting |\jobname| to \textit{main}.

%%%%%%%%%%%%%%%%%%%%%%%%%%%%%%%%%%%%%%%%%%%%%%%%%%%%%%%%%%%%%%%%%%%%%%%%%%%%%%%%
\subsection{Driver Development}
\label{sec:driver}

The \textsf{childdoc} mechanism can also be use for the development
of definition files such as \LaTeX{} styles or classes.
This case differs from the above setup with multiple parts
included by |\include| in that no |\includeonly| should be invoked.
This can be achieved by starting the include file
(before |\ProvidesPackage|) with:
%
\begin{center}
\begin{tabular}{l}
|\input{childdoc.def}|\\
|\childdocforward{|\textit{main}|}|\\
\end{tabular}
\end{center}
%
or alternatively with:
%
\begin{center}
\begin{tabular}{l}
|\input{childdoc.def}|\\
|\childdocby{|\textit{main}|}|\\
\end{tabular}
\end{center}
%
Both forms have slightly different effects as described above.
The main file is prepared as usual, see \secref{sec:include}.

%%%%%%%%%%%%%%%%%%%%%%%%%%%%%%%%%%%%%%%%%%%%%%%%%%%%%%%%%%%%%%%%%%%%%%%%%%%%%%%%
\subsection{Legacy Detection}
\label{sec:detection}

The directive |\childdocmain| in the main file can detect
whether the complete document or merely a child is to be compiled
even without using the directive |\childdocof|.
This method is deprecated because it is less robust
and there is no compelling reason to use it;
it is merely provided for backward compatibility
and it may be removed in future versions.

If the detection mechanism is to be used,
it is mandatory to correctly specify
the filename of the main file as the argument of |\childdocmain|:
%
\begin{center}
\begin{tabular}{l}
|\input{childdoc.def}|\\
|\childdocmain{|\textit{main}|}|\\
\end{tabular}
\end{center}
%
If |\jobname| does not match the argument \textit{main} of |\childdocmain|,
it is assumed that |\jobname| points to the child file to be compiled.
When using |\childdocmain| with the main file specified as argument,
it suffices to start a child file
with just |\input{|\textit{main}|}|
without loading of the package and using |\childdocof|.
If instead all processing is done
with the appropriate \textsf{childdoc} directives,
the argument of \textit{main} of |\childdocmain| can be empty.

An alternative version of the command line processing described
in \secref{sec:commandline} using the detection mechanism reads:
%
\begin{center}
|... -jobname "|\textit{target}|" "|[\textit{flags}]%
[|\def\jobname{|\textit{dest}|}|]|\input{|\textit{main}|}"|
\end{center}

%%%%%%%%%%%%%%%%%%%%%%%%%%%%%%%%%%%%%%%%%%%%%%%%%%%%%%%%%%%%%%%%%%%%%%%%%%%%%%%%
\subsection{Manual Code}
\label{sec:manual}

In case one cannot be certain whether the definitions file |childdoc.def|
is installed on the target \TeX{} distribution
and one prefers not to ship it,
it is conceivable to paste a few relevant commands into the sources.

To that end, drop all statements |\input{childdoc.def}|
and perform the replacements as outlined below.
Instead of |\childdocmain{|\textit{main}|}| add the following code
to the top of the main file:
%
\begin{center}
\begin{tabular}{l}
|\||ifdefined\childdocname\endinput\||fi\newif\ifchilddoc|\\
|\edef\childdocname{\scantokens\expandafter{\jobname\noexpand}}|\\
|\def\childdocmain{|\textit{main}|}\||ifx\childdocmain\childdocname\||else|\\
|\childdoctrue\includeonly{\childdocname}\let\jobname\childdocmain\||fi|\\
\end{tabular}
\end{center}
%
Instead of |\childdocof{|\textit{main}|}| just include the main file
at the top of each child file:
%
\begin{center}
|\input{|\textit{main}|}|
\end{center}
%
A simple redirection |\childdocforward{|\textit{dest}|}| is achieved by:
%
\begin{center}
|\def\jobname{|\textit{dest}|}\input{\jobname}|
\end{center}
%
The redirection with prefix
|\childdocforwardprefix[|\textit{prefix}|]{|\textit{dest}|}|
is accomplished by:
%
\begin{center}
\begin{tabular}{l}
|{\edef\jobname{\scantokens\expandafter{\jobname\noexpand}}|\\
|\def\redirectjob |\textit{prefix}|#1~~~{\gdef\jobname{|\textit{dest}|#1}}|\\
|\expandafter\redirectjob\jobname~~~}\input{\jobname}|
\end{tabular}
\end{center}

In an alternative approach,
child documents can be compiled by a specific command line
without additional code or specific definitions:
%
\begin{center}
|... -jobname "|\textit{target}|" "|[\textit{flags}]%
|\includeonly{|\textit{dest}|}\input{|\textit{main}|}"|
\end{center}
%

%%%%%%%%%%%%%%%%%%%%%%%%%%%%%%%%%%%%%%%%%%%%%%%%%%%%%%%%%%%%%%%%%%%%%%%%%%%%%%%%
%%%%%%%%%%%%%%%%%%%%%%%%%%%%%%%%%%%%%%%%%%%%%%%%%%%%%%%%%%%%%%%%%%%%%%%%%%%%%%%%
\section{Information}

%%%%%%%%%%%%%%%%%%%%%%%%%%%%%%%%%%%%%%%%%%%%%%%%%%%%%%%%%%%%%%%%%%%%%%%%%%%%%%%%
\subsection{Copyright}

Copyright \copyright{} 2017--2018 Niklas Beisert

This work may be distributed and/or modified under the
conditions of the \LaTeX{} Project Public License, either version 1.3
of this license or (at your option) any later version.
The latest version of this license is in
  \url{http://www.latex-project.org/lppl.txt}
and version 1.3 or later is part of all distributions of \LaTeX{}
version 2005/12/01 or later.

This work has the LPPL maintenance status `maintained'.

The Current Maintainer of this work is Niklas Beisert.

This work consists of the files |README.txt|, |childdoc.ins| and |childdoc.dtx|
as well as the derived files |childdoc.def|, |cdocsamp.tex|
with |cdocsch1.tex|, |cdocsch2.tex|, |cdocspt3.tex|, |cdocspt4.tex|,
|cdocsdrf.tex|, |cdocsfn1.tex|, |cdocsfn2.tex|
as well as |childdoc.pdf|.

%%%%%%%%%%%%%%%%%%%%%%%%%%%%%%%%%%%%%%%%%%%%%%%%%%%%%%%%%%%%%%%%%%%%%%%%%%%%%%%%
\subsection{Files and Installation}

The package consists of the files:
%
\begin{center}
\begin{tabular}{ll}
    |README.txt|   & readme file \\
    |childdoc.ins| & installation file \\
    |childdoc.dtx| & source file \\
    |childdoc.def| & definition file \\
    |cdocsamp.tex| & sample main file \\
    |cdocsch1.tex| & sample include file \\
    |cdocsch2.tex| & sample include file \\
    |cdocspt3.tex| & sample part file \\
    |cdocspt4.tex| & sample part file \\
    |cdocsdrf.tex| & sample redirection file \\
    |cdocsfn1.tex| & sample redirection file \\
    |cdocsfn2.tex| & sample redirection file \\
    |childdoc.pdf| & manual
\end{tabular}
\end{center}
%
The distribution consists of the files
|README.txt|, |childdoc.ins| and |childdoc.dtx|.
%
\begin{itemize}
\item
Run (pdf)\LaTeX{} on |childdoc.dtx|
to compile the manual |childdoc.pdf| (this file).
\item
Run \LaTeX{} on |childdoc.ins| to create the definitions file |childdoc.def|
and the sample |cdocsamp.tex| with include files
|cdocsch1.tex|, |cdocsch2.tex|, |cdocspt3.tex|, |cdocspt4.tex|,
|cdocsdrf.tex|, |cdocsfn1.tex|, |cdocsfn2.tex|.
Then copy the file |childdoc.def| to an appropriate directory of your \LaTeX{}
distribution, e.g.\ \textit{texmf-root}|/tex/latex/childdoc|.
\end{itemize}

%%%%%%%%%%%%%%%%%%%%%%%%%%%%%%%%%%%%%%%%%%%%%%%%%%%%%%%%%%%%%%%%%%%%%%%%%%%%%%%%
\subsection{Related CTAN Packages}

There are several other packages which offer a similar functionality:
%
\begin{itemize}
\item
The packages
\href{http://ctan.org/pkg/docmute}{\textsf{docmute}},
\href{http://ctan.org/pkg/includex}{\textsf{includex}} and
\href{http://ctan.org/pkg/standalone}{\textsf{standalone}}
provide commands to include only the document body of
a child file thus allowing both files to be compiled individually.
\item
The packages \href{http://ctan.org/pkg/subdocs}{\textsf{subdocs}}
and \href{http://ctan.org/pkg/subfiles}{\textsf{subfiles}}
provide structures in which the main and child documents can be
encapsulated and allowing them to be compiled individually.
The inclusion mechanism is different from the conventional |\include|.
\item
The package \href{http://ctan.org/pkg/combine}{\textsf{combine}}
is an elaborate solution to combine several documents into one.
\end{itemize}
%
See also the CTAN topic \href{http://ctan.org/topic/subdocs}{\textsf{subdocs}}
for further related packages.
The present package differs from the above solutions in that
a document structure constructed with the conventional |\include| mechanism
just needs two extra commands at the top of every file
such that all constituent files can be compiled individually.

%%%%%%%%%%%%%%%%%%%%%%%%%%%%%%%%%%%%%%%%%%%%%%%%%%%%%%%%%%%%%%%%%%%%%%%%%%%%%%%%
%\subsection{Feature Suggestions}
%
%The following is a list of features which may be useful for future
%versions of this package:
%%
%\begin{itemize}
%\item
%\ldots
%\end{itemize}

%%%%%%%%%%%%%%%%%%%%%%%%%%%%%%%%%%%%%%%%%%%%%%%%%%%%%%%%%%%%%%%%%%%%%%%%%%%%%%%%
\subsection{Revision History}

%%%%%%%%%%%%%%%%%%%%%%%%%%%%%%%%%%%%%%%%
\paragraph{v2.0:} 2018/12/30

\begin{itemize}
\item
immediate forward processing
\item
added |\childdocby| mechanism
\item
manual restructured
\end{itemize}

%%%%%%%%%%%%%%%%%%%%%%%%%%%%%%%%%%%%%%%%
\paragraph{v1.6:} 2018/01/17

\begin{itemize}
\item
application for development of include files
\item
corrections to manual
\end{itemize}

%%%%%%%%%%%%%%%%%%%%%%%%%%%%%%%%%%%%%%%%
\paragraph{v1.5:} 2017/05/21

\begin{itemize}
\item
more complete structuring introduced
\item
|\childdocof| introduced
\item
|\childdoc| renamed to |\childdocmain|
\item
|\childredirect| renamed to |\childdocforward| and |\childdocforwardprefix|
and functionality expanded
\end{itemize}

%%%%%%%%%%%%%%%%%%%%%%%%%%%%%%%%%%%%%%%%
\paragraph{v1.0:} 2017/04/27

\begin{itemize}
\item
manual and install package
\item
first version published on CTAN
\end{itemize}

%%%%%%%%%%%%%%%%%%%%%%%%%%%%%%%%%%%%%%%%
\paragraph{v0.6:} 2017/04/26

\begin{itemize}
\item
redirection mechanism added
\end{itemize}

%%%%%%%%%%%%%%%%%%%%%%%%%%%%%%%%%%%%%%%%
\paragraph{v0.5:} 2017/04/26

\begin{itemize}
\item
functionality in definition file
\end{itemize}


%%%%%%%%%%%%%%%%%%%%%%%%%%%%%%%%%%%%%%%%%%%%%%%%%%%%%%%%%%%%%%%%%%%%%%%%%%%%%%%%
%%%%%%%%%%%%%%%%%%%%%%%%%%%%%%%%%%%%%%%%%%%%%%%%%%%%%%%%%%%%%%%%%%%%%%%%%%%%%%%%
%%%%%%%%%%%%%%%%%%%%%%%%%%%%%%%%%%%%%%%%%%%%%%%%%%%%%%%%%%%%%%%%%%%%%%%%%%%%%%%%
\appendix

\settowidth\MacroIndent{\rmfamily\scriptsize 000\ }

 \DocInput{childdoc.dtx}

\end{document}
%</driver>
% \fi
%
% %%%%%%%%%%%%%%%%%%%%%%%%%%%%%%%%%%%%%%%%%%%%%%%%%%%%%%%%%%%%%%%%%%%%%%%%%%%%%%
% %%%%%%%%%%%%%%%%%%%%%%%%%%%%%%%%%%%%%%%%%%%%%%%%%%%%%%%%%%%%%%%%%%%%%%%%%%%%%%
% \section{Sample}
%\iffalse
%<*samplemain>
%\fi
%
% The following presents a sample document
% with two chapters, two parts, a title page,
% a compile flag as well as three forwarding files to set the flag.
% It consists of eight |.tex| files:
% \begin{center}
% \begin{tabular}{ll}
% |cdocsamp.tex|&main file\\
% |cdocsch1.tex|&include file for chapter 1\\
% |cdocsch2.tex|&include file for chapter 2\\
% |cdocspt3.tex|&include file for part 3\\
% |cdocspt4.tex|&include file for part 4\\
% |cdocsdrf.tex|&forwarding file for main file in draft mode\\
% |cdocsfi1.tex|&forwarding file for final version of chapter 1\\
% |cdocsfi2.tex|&forwarding file for final version of chapter 2\\
% \end{tabular}
% \end{center}
% Each of the eight files can be compiled directly by the \LaTeX{} compiler.
%
% %%%%%%%%%%%%%%%%%%%%%%%%%%%%%%%%%%%%%%
% \paragraph{Main File.}
%
% The main file is called |cdocsamp.tex|.
%
% Load the \textsf{childdoc} definitions and
% declare the filename for the main document:
%    \begin{macrocode}
\input{childdoc.def}
\childdocmain{}
%    \end{macrocode}

% Optional override for |\version| flag:
%    \begin{macrocode}
%%\ifchilddoc\else\providecommand{\version}{draft}\fi
%    \end{macrocode}

% Define the default values for the |\version| flag
% (|final| for the main file and |draft| for childs):
%    \begin{macrocode}
\ifchilddoc
\providecommand{\version}{draft}
\else
\providecommand{\version}{final}
\fi
%    \end{macrocode}

% Load the standard document class:
%    \begin{macrocode}
\documentclass[12pt]{article}
%    \end{macrocode}

% Start the document body:
%    \begin{macrocode}
\begin{document}
%    \end{macrocode}

% Declare a title page.
% Print title, part of document being processed and version flag:
%    \begin{macrocode}
\addtocounter{page}{-1}
\begin{center}
{\LARGE\bfseries{}childdoc example\par}
\vspace{1cm}
\ifchilddoc
\ifchilddocmanual part\else chapter\fi:
`\childdocname' of `\childdocjob'\par
\else
main document: `\childdocjob'\par
\fi
version: \version\par
\end{center}
\newpage
%    \end{macrocode}

% Manually include selected file,
% otherwise process as usual:
%    \begin{macrocode}
\ifchilddocmanual
\section*{part `\childdocname'}
\input{\childdocname}
\else
%    \end{macrocode}

% Include the two chapters:
%    \begin{macrocode}
\include{cdocsch1}
\include{cdocsch2}
%    \end{macrocode}

% Include the two parts unless only chapters should be displayed:
%    \begin{macrocode}
\ifchilddoc\else
\section{part three}
\input{cdocspt3}
\section{part four}
\input{cdocspt4}
\fi
%    \end{macrocode}

% Process as usual until here:
%    \begin{macrocode}
\fi
%    \end{macrocode}

% End of document body:
%    \begin{macrocode}
\end{document}
%    \end{macrocode}
%\iffalse
%</samplemain>
%\fi
%
% %%%%%%%%%%%%%%%%%%%%%%%%%%%%%%%%%%%%%%
% \paragraph{Chapter Include Files.}
%
% The include files are called |cdocsch1.tex| and |cdocsch2.tex|.
%
%\iffalse
%<*samplechap1|samplechap2>
%\fi

% Optional override for |\version| flag:
%    \begin{macrocode}
%%\providecommand{\version}{final}
%    \end{macrocode}

% Include the main document:
%    \begin{macrocode}
\input{childdoc.def}
\childdocof{cdocsamp}
%    \end{macrocode}

%\iffalse
%</samplechap1|samplechap2>
%\fi
%
%\iffalse
%<*samplechap1>
%\fi
% Some text for chapter 1:
%    \begin{macrocode}
\section{one}
some text in chapter one
%    \end{macrocode}

%\iffalse
%</samplechap1>
%\fi
% Some text for chapter 2:
%\iffalse
%<*samplechap2>
%\fi
%    \begin{macrocode}
\section{two}
more text in chapter two
%    \end{macrocode}

%\iffalse
%</samplechap2>
%\fi
%
% %%%%%%%%%%%%%%%%%%%%%%%%%%%%%%%%%%%%%%
% \paragraph{Part Include Files.}
%
% The include files are called |cdocspt3.tex| and |cdocspt4.tex|.
%
%\iffalse
%<*samplepart3|samplepart4>
%\fi

% Optional override for |\version| flag:
%    \begin{macrocode}
%%\providecommand{\version}{final}
%    \end{macrocode}

% Include the main document:
%    \begin{macrocode}
\input{childdoc.def}
\childdocby{cdocsamp}
%    \end{macrocode}

%\iffalse
%</samplepart3|samplepart4>
%\fi
%
%\iffalse
%<*samplepart3>
%\fi
% Some text for part 3:
%    \begin{macrocode}
some text in part three
%    \end{macrocode}

%\iffalse
%</samplepart3>
%\fi
% Some text for part 4:
%\iffalse
%<*samplepart4>
%\fi
%    \begin{macrocode}
more text in part four
%    \end{macrocode}

%\iffalse
%</samplepart4>
%\fi
%
% %%%%%%%%%%%%%%%%%%%%%%%%%%%%%%%%%%%%%%
% \paragraph{Forwarding for a Complete Draft.}
%
% The following forwarding file |cdocsdrf.tex|
% compiles the main document in draft mode:
%\iffalse
%<*sampledraft>
%\fi
%    \begin{macrocode}
\def\version{draft}
\input{childdoc.def}
\childdocforward{cdocsamp}
%    \end{macrocode}

%\iffalse
%</sampledraft>
%\fi
%
% %%%%%%%%%%%%%%%%%%%%%%%%%%%%%%%%%%%%%%
% \paragraph{Forwarding for Final Version of the Chapters.}
%
% The following forwarding files |cdocsfn1.tex| and |cdocsfn2.tex|
% (with identical content)
% compile the final versions of the child documents
% |cdocsch1.tex| and |cdocsch2.tex|, respectively:
%\iffalse
%<*samplefinal>
%\fi
%    \begin{macrocode}
\def\version{final}
\input{childdoc.def}
\childdocforwardprefix[cdocsamp]{cdocsfn}{cdocsch}
%    \end{macrocode}

%\iffalse
%</samplefinal>
%\fi
%
% %%%%%%%%%%%%%%%%%%%%%%%%%%%%%%%%%%%%%%
% \paragraph{Command Line Processing.}
%
% The following three command lines generate the output files
% |cdocscld|, |cdocscl1| and |cdocscl2|
% which should be identical to
% |cdocsdrf|, |cdocsch1| and |cdocsfn2|, respectively:
% \begin{center}
% \begin{tabular}{l}
% |latex -jobname cdocscld \|\\
% |  "\def\version{draft}\input{childdoc.def}\childdocforward{cdocsamp}"|\\
% |latex -jobname cdocscl1 \|\\
% |  "\input{childdoc.def}\childdocforward[cdocsamp]{cdocsch1}"|\\
% |latex -jobname cdocscl2 \|\\
% |  "\def\version{final}\input{childdoc.def}\childdocforward{cdocsch2}"|
% \end{tabular}
% \end{center}
% Note that the trailing backslash on each first line
% merely continues the input to the second line
% (for convenient cut ant paste).
% Furthermore, the command |latex| can be replaced by any
% of its alternative versions such as |pdflatex|.
%
% %%%%%%%%%%%%%%%%%%%%%%%%%%%%%%%%%%%%%%%%%%%%%%%%%%%%%%%%%%%%%%%%%%%%%%%%%%%%%%
% %%%%%%%%%%%%%%%%%%%%%%%%%%%%%%%%%%%%%%%%%%%%%%%%%%%%%%%%%%%%%%%%%%%%%%%%%%%%%%
% \section{Implementation}
%\iffalse
%<*package>
%\fi
%
% This section describes the definitions file |childdoc.def|.

% The definitions cannot be loaded using |\usepackage| or |\RequirePackage|
% which has a mechanism to prevent loading a style file more than once.
% When loading the definitions by means of |\input|
% multiple instances have to be prevented manually:
%\iffalse
%This code needs to be before the `\ProvidesFile' directive
%which is defined at the beginning of this file.
%Therefore it is also placed there and commented out here.
%</package>
%<*discard>
%\fi
%    \begin{macrocode}
\ifdefined\childdocmain\endinput\fi
%    \end{macrocode}
%\iffalse
%</discard>
%<*package>
%\fi
%
% \macro{\ifchilddoc}
% \macro{\ifchilddocmanual}
% The conditional |\ifchilddoc| tells whether a
% child (true) or main (false) document is being compiled.
% The conditional |\ifchilddocmanual| tells whether
% the |\includeonly| mechanism is used (false) or
% the selection of child files must be performed manually (true).
% The definitions initialise to false:
%    \begin{macrocode}
\newif\ifchilddoc
\newif\ifchilddocmanual
%    \end{macrocode}

% \macro{\childdocname}
% \macro{\childdocjob}
% The macro |\childdocname| stores the name of the main document
% to be compiled. The macro |\childdocjob| stores the name of
% the document on which the \LaTeX{} compiler was originally invoked.
% The content of |\jobname| cannot be compared
% to filenames specified in the source due to different catcodes.
% The following code rescans |\jobname|, stores the result
% in |\childdocname| and saves a copy in |\childdocjob|:
%    \begin{macrocode}
\edef\childdocname{\scantokens\expandafter{\jobname\noexpand}}
\let\childdocjob\childdocname
%    \end{macrocode}

% \macro{\childdocdisable}
% The macro |\childdocdisable| prevents the main file
% from being processed more than once.
% At this stage, the main document command |\childdocmain|
% is assumed to be called once again where it should do nothing.
% Any subsequent call to it should prevent
% a secondary processing of the main document
% It overwrites the forwarding commands
% |\childdocof| and |\childdocforward|
% with empty macros to prevent further inclusions of the main document:
%    \begin{macrocode}
\newcommand{\childdocdisable}
{
  \renewcommand{\childdocmain}[1]{\renewcommand{\childdocmain}[1]{\endinput}}
  \renewcommand{\childdocof}[1]{}
  \renewcommand{\childdocby}[2][]{}
  \renewcommand{\childdocforward}[2][]{}
  \renewcommand{\childdocdisable}{}
}
%    \end{macrocode}

% \macro{\childdocmain}
% The macro |\childdocmain| is to be called at the top of the main file
% with nothing or the main filename (without extension) as argument.
% First, it breaks loops.
% If the argument is not empty and does not match |\childdocname|
% (which is set by the first inclusion of |childdoc.def|),
% |\ifchilddoc| is set to true, |\includeonly| is applied to the child file
% and |\jobname| is set to the main file
% (for proper handling of |.aux| files):
%    \begin{macrocode}
\newcommand{\childdocmain}[1]
{
  \childdocdisable\childdocmain{}
  \if?#1?\else
    \begingroup
      \def\childdoctmp{#1}
      \ifx\childdoctmp\childdocname
        \def\childdoctmp{}
      \else
        \def\childdoctmp
        {
          \childdoctrue
          \includeonly{\childdocname}
          \def\childdocjob{#1}
          \def\jobname{#1}
        }
      \fi
      \expandafter
    \endgroup
    \childdoctmp
  \fi
}
%    \end{macrocode}

% \macro{\childdocof}
% The command |\childdocof| redirects
% compilation to the main file |#1|.
%    \begin{macrocode}
\newcommand{\childdocof}[1]
{
  \childdocdisable
  \childdoctrue
  \includeonly{\childdocname}
  \def\jobname{#1}
  \def\childdocjob{#1}
  \input{#1}
}
%    \end{macrocode}

% \macro{\childdocby}
% The command |\childdocby| ....
%    \begin{macrocode}
\newcommand{\childdocby}[2][]
{
  \childdocdisable
  \childdoctrue
  \childdocmanualtrue
  \if?#1?\else
    \def\jobname{#2}
  \fi
  \def\childdocjob{#2}
  \input{#2}
  \endinput
}
%    \end{macrocode}

% \macro{\childdocforward}
% The command |\childdocforward| redirects
% compilation to the main file or
% (if the optional argument is given) a child file.
% Parameters are set as if the main file
% or a child file starting with |\childdocof| was compiled.
% Then compilation is handed over to the main file:
%    \begin{macrocode}
\newcommand{\childdocforward}[2][]
{
  \begingroup
    \if?#1?
      \def\childdoctmp
      {
        \def\childdocname{#2}
        \def\childdocjob{#2}
        \def\jobname{#2}
        \input{#2}
        \endinput
      }
    \else
      \def\childdoctmp
      {
        \childdocdisable
        \def\childdocname{#2}
        \childdoctrue
        \includeonly{#2}
        \def\childdocjob{#1}
        \def\jobname{#1}
        \input{#1}
        \endinput
      }
    \fi
    \expandafter
  \endgroup
  \childdoctmp
}
%    \end{macrocode}

% \macro{\childdocforwardprefix}
% The command |\childdocforwardprefix| redirects
% compilation to the main or a child file by means of a pattern.
% The prefix |#1| in the current filename is replaced by |#2|
% and the suffix of the current filename is kept
% (it is assumed that the filename does not contain the substring `|~~~|'
% which is used as a delimiter).
% Compilation is handed over to the new file by |\childdocforward|:
%    \begin{macrocode}
\newcommand{\childdocforwardprefix}[3][]
{
  \begingroup
    \def\childdocextract #2##1~~~{\def\childdoctmp{\childdocforward[#1]{#3##1}}}
    \expandafter\childdocextract\childdocname~~~
    \expandafter
  \endgroup
  \childdoctmp
}
%    \end{macrocode}

% \macro{\childdoc}
% The deprecated macro |\childdoc| is a legacy version of |\childdocmain|:
%    \begin{macrocode}
\newcommand{\childdoc}{\childdocmain}
%    \end{macrocode}

% \macro{\childdocredirect}
% The deprecated macro |\childdocredirect| is a legacy version
% of |\childdocforward| and |\childdocforwardprefix|:
%    \begin{macrocode}
\newcommand{\childdocredirect}[2][]
{
  \begingroup
    \if?#1?
      \def\childdoctmp{\childdocforward{#2}}
    \else
      \def\childdoctmp{\childdocforwardprefix{#1}{#2}}
    \fi
    \expandafter
  \endgroup
  \childdoctmp
}
%    \end{macrocode}

%\iffalse
%</package>
%\fi
%
\endinput
|\\
|\childdocforward{|\textit{main}|}|
\end{tabular}
\end{center}
%
Likewise, the following files |final|\textit{nn}|.tex|
compile the final version of the child document
|child|\textit{nn}|.tex|:
%
\begin{center}
\begin{tabular}{l}
|\def\version{final}|\\
|% \iffalse
%
% childdoc.dtx Copyright (C) 2017-2018 Niklas Beisert
%
% This work may be distributed and/or modified under the
% conditions of the LaTeX Project Public License, either version 1.3
% of this license or (at your option) any later version.
% The latest version of this license is in
%   http://www.latex-project.org/lppl.txt
% and version 1.3 or later is part of all distributions of LaTeX
% version 2005/12/01 or later.
%
% This work has the LPPL maintenance status `maintained'.
%
% The Current Maintainer of this work is Niklas Beisert.
%
% This work consists of the files childdoc.dtx and childdoc.ins
% and the derived files childdoc.def and cdocsamp.tex with
% cdocsch1.tex, cdocsch2.tex, cdocsdrf.tex, cdocsfn1.tex, cdocsfn2.tex.
%
%<package>\ifdefined\childdocmain\endinput\fi
%<package>\ProvidesFile{childdoc.def}[2018/12/30 v2.0 child document driver]
%<samplemain>\ProvidesFile{cdocsamp.tex}[2018/12/30 v2.0 sample for childdoc]
%<*driver>
%\ProvidesFile{childdoc.drv}[2018/12/30 v2.0 childdoc reference manual file]
\PassOptionsToClass{10pt,a4paper}{article}
\documentclass{ltxdoc}

\usepackage[margin=35mm]{geometry}
\usepackage{hyperref}
\usepackage{hyperxmp}
\usepackage[usenames]{color}

\hypersetup{colorlinks=true}
\hypersetup{pdfstartview=FitH}
\hypersetup{pdfpagemode=UseNone}
\hypersetup{pdfsource={}}
\hypersetup{pdflang={en-UK}}
\hypersetup{pdfcopyright={Copyright 2017-2018 Niklas Beisert.
  This work may be distributed and/or modified under the
  conditions of the LaTeX Project Public License, either version 1.3
  of this license or (at your option) any later version.}}
\hypersetup{pdflicenseurl={http://www.latex-project.org/lppl.txt}}
\hypersetup{pdfcontactaddress={ETH Zurich, ITP, HIT K,
  Wolfgang-Pauli-Strasse 27}}
\hypersetup{pdfcontactpostcode={8093}}
\hypersetup{pdfcontactcity={Zurich}}
\hypersetup{pdfcontactcountry={Switzerland}}
\hypersetup{pdfcontactemail={nbeisert@itp.phys.ethz.ch}}
\hypersetup{pdfcontacturl={http://people.phys.ethz.ch/\xmptilde nbeisert/}}

\newcommand{\secref}[1]{\hyperref[#1]{section \ref*{#1}}}

\parskip1ex
\parindent0pt
\let\olditemize\itemize
\def\itemize{\olditemize\parskip0pt}

\begin{document}

\title{The \textsf{childdoc} Package}
\hypersetup{pdftitle={The childdoc Package}}
\author{Niklas Beisert\\[2ex]
  Institut f\"ur Theoretische Physik\\
  Eidgen\"ossische Technische Hochschule Z\"urich\\
  Wolfgang-Pauli-Strasse 27, 8093 Z\"urich, Switzerland\\[1ex]
  \href{mailto:nbeisert@itp.phys.ethz.ch}
  {\texttt{nbeisert@itp.phys.ethz.ch}}}
\hypersetup{pdfauthor={Niklas Beisert}}
\hypersetup{pdfsubject={Manual for the LaTeX2e Package childdoc}}
\date{30 December 2018, \textsf{v2.0}}
\maketitle

\begin{abstract}\noindent
\textsf{childdoc} is a \LaTeXe{} package
that enables the direct compilation
of document sections included by |\include|
to individual files.
\end{abstract}

\begingroup
\parskip0ex
\tableofcontents
\endgroup

%%%%%%%%%%%%%%%%%%%%%%%%%%%%%%%%%%%%%%%%%%%%%%%%%%%%%%%%%%%%%%%%%%%%%%%%%%%%%%%%
%%%%%%%%%%%%%%%%%%%%%%%%%%%%%%%%%%%%%%%%%%%%%%%%%%%%%%%%%%%%%%%%%%%%%%%%%%%%%%%%
\section{Introduction}

\LaTeX{} provides a mechanism to structure a large document (such as a book)
into a main file and several child files (containing the chapters)
using the |\include| command.
This mechanism is beneficial for documents
which span hundreds of pages in order to
make the source file(s) more manageable.
Moreover, compilation can be restricted to
selected child files by means of the |\includeonly| command.
The latter feature can be used to reduce the compilation time while editing
(this was significantly more useful in the earlier days of \LaTeX{})
or to generate a smaller document which is easier to navigate.
Another application of |\includeonly| is to generate
documents consisting of selected parts of the complete document.

However, there are a few drawbacks of the plain |\include| mechanism:
\begin{itemize}
\item
The child files cannot be compiled on their own,
they can only be compiled via the main file.
A naive editing environment
(such as a text editor with an option
to have the current file processed by \LaTeX)
may require one to switch to the main file before compiling;
attempting to compile the child file produces errors.
\item
The main file must be modified (each time)
to adjust the |\includeonly| command
to the present needs. This easily leaves the main file in a messy state.
\item
The generated document will always carry the filename
of the main document. This is inconvenient if
several child files are to be compiled and
to be kept for distribution.
\end{itemize}

The present package provides a simple interface
to make child files individually compilable by \LaTeX{}.
Compiling a child file then has the same effect as compiling
the main file with an |\includeonly| command
to select the appropriate child.
Moreover the generated document will carry the name of the child
rather than the main file.
This resolves all three above issues.

This feature is meant to make the editing of books,
thesis documents and lecture notes somewhat more convenient.
However, the package can also be used efficiently for
composing a series of documents (such as exercise sheets)
which are typically distributed individually.
It then assists the author in generating the individual documents
(potentially in different versions)
as well as a document containing the collected series.
Another application is in developing style files
or other kinds of included material
where compilation of the style file could redirect
to a sample or test file.

%%%%%%%%%%%%%%%%%%%%%%%%%%%%%%%%%%%%%%%%%%%%%%%%%%%%%%%%%%%%%%%%%%%%%%%%%%%%%%%%
%%%%%%%%%%%%%%%%%%%%%%%%%%%%%%%%%%%%%%%%%%%%%%%%%%%%%%%%%%%%%%%%%%%%%%%%%%%%%%%%
\section{Usage}

First of all, the package \textsf{childdoc} is \emph{not} a standard
\LaTeXe{} |.sty| style file! Therefore it needs to be invoked in
a non-standard way.

%%%%%%%%%%%%%%%%%%%%%%%%%%%%%%%%%%%%%%%%%%%%%%%%%%%%%%%%%%%%%%%%%%%%%%%%%%%%%%%%
\subsection{Included Files}
\label{sec:include}

%%%%%%%%%%%%%%%%%%%%%%%%%%%%%%%%%%%%%%%%
\DescribeMacro{\childdocmain}
To use the package, add the commands
\begin{center}
\begin{tabular}{l}
|\input{childdoc.def}|\\
|\childdocmain{}|\\
\end{tabular}
\end{center}
at the very top of the main \LaTeX{} file,
in particular \emph{before} the |\documentclass| statement!
The argument of |\childdocmain| should be left empty
(but it must be present).

%%%%%%%%%%%%%%%%%%%%%%%%%%%%%%%%%%%%%%%%
\DescribeMacro{\childdocof}
Furthermore, add the commands
\begin{center}
\begin{tabular}{l}
|\input{childdoc.def}|\\
|\childdocof{|\textit{main}|}|\\
\end{tabular}
\end{center}
at the top of every child file \textit{child}
which is included by |\include{|\textit{child}|}|
from within the main file
(or at least for those files to be compiled individually).
The argument \textit{main} must be the filename of the main file.

There are a couple of
considerations in setting up the main and child documents:

%%%%%%%%%%%%%%%%%%%%%%%%%%%%%%%%%%%%%%%%
\paragraph{Restrictions.}

Please note the following restrictions:
\begin{itemize}
\item
|\childdocmain| must be called with one argument \textit{main}
to ensure compatibility with earlier version of the package.
It must either be empty (|\childdocmain{}|)
or precisely match the filename of the main file in which it is specified.
See \secref{sec:detection} for further information.
\item
The filename \textit{main} must be specified without the |.tex| extension.
\item
The filename \textit{main} is case sensitive
(even in case-insensitive file systems)
due to internal string comparison.
\item
The argument \textit{main} should be fully expanded, it cannot be a macro.
\item
Subdirectories and special characters should be avoided in filenames.
\item
The command |\childdocmain{|\textit{main}|}| must be followed by a whitespace.
It should not be followed immediately by another command
or by a comment mark `|%|'.
This is because the \TeX{} parser reads the token immediately following
the argument of |\childdocmain| and puts it
at the beginning of every child section;
however, a white\-space is ignored.
\end{itemize}

%%%%%%%%%%%%%%%%%%%%%%%%%%%%%%%%%%%%%%%%
\paragraph{Content of Main File.}

It is advisable to place all content in the child files included by |\include|.
Any output contained in the main file will appear in all child documents
unless suppressed manually;
it cannot be suppressed automatically by the |\includeonly| directive
and thus should normally be avoided.
A method to include some content in the main file
by means of conditional processing is described in \secref{sec:conditional}.

%%%%%%%%%%%%%%%%%%%%%%%%%%%%%%%%%%%%%%%%
\paragraph{Page Numbering.}

When only a part of the document is compiled,
the appropriate numbering of pages
(as well as other status parameters)
is determined from the |.aux| files.
The latter contain information from previous passes.
However this information needs to propagate through
all intermediate child documents.
Therefore the page numbering in child documents may well
be inconsistent until the complete document is compiled at least once.

A useful (if unconventional) way to always ensure a consistent
page numbering is to restart the numbering in each child document
and denote the pages by `\textit{child}|.|\textit{page}'
where \textit{child} represents the chapter/section number of the child file.
This can be achieved by the command
|\numberwithin{page}{|\textit{child}|}|
of the \textsf{amsmath} package
where \textit{child} can be |chapter| or |section|
depending on the chosen structuring.
Alternatively, one can modify the macro |\thepage| appropriately
and reset the counter |page| at the start of each child file.

%%%%%%%%%%%%%%%%%%%%%%%%%%%%%%%%%%%%%%%%%%%%%%%%%%%%%%%%%%%%%%%%%%%%%%%%%%%%%%%%
\subsection{Conditional Processing}
\label{sec:conditional}

The package provides a mechanism to compile different versions
of a document. To customise the versions further some conditional processing
can come in handy to distinguish which version is being compiled.
The package provides two macros to describe the compilation context:

%%%%%%%%%%%%%%%%%%%%%%%%%%%%%%%%%%%%%%%%
\DescribeMacro{\ifchilddoc}
The conditional |\ifchilddoc| distinguishes between the compilation of
child documents and the main document:
%
\begin{center}
|\ifchilddoc |\textit{child-code}| |[|\||else |\textit{main-code}]| \||fi|
\end{center}

%%%%%%%%%%%%%%%%%%%%%%%%%%%%%%%%%%%%%%%%
\DescribeMacro{\childdocname}
\DescribeMacro{\childdocjob}
The macro |\childdocname| contains the filename (without extension)
of the main or child file being processed.
Note that |\childdocjob| will always contain the name of the main file.

%%%%%%%%%%%%%%%%%%%%%%%%%%%%%%%%%%%%%%%%
\paragraph{Title Page.}

Conditional processing can be used to include a title or banner page
in the main document when proper precautions are taken.
Importantly, the code in the main file should ensure that the page counter
(as well as other status parameters which are stored in the |.aux| files)
takes the same value after the conditional processing.
Otherwise the page numbers may take divergent values
depending on which part is compiled.

For example, a title page could be declared by:
%
\begin{center}
\begin{tabular}{l}
|\ifchilddoc\||else|\\
|\addtocounter{page}{-1}|\\
\textit{code for title page}\\
|\newpage|\\
|\||fi|
\end{tabular}
\end{center}
%
A banner page for the child documents can be generated by:
%
\begin{center}
\begin{tabular}{l}
|\ifchilddoc|\\
|\addtocounter{page}{-1}|\\
\textit{code for banner page}\\
|\newpage|\\
|\||fi|
\end{tabular}
\end{center}
%
Here one could write a message such as:
\begin{center}
|This is the part \childdocname{} of \childdocjob{}.|
\end{center}

%%%%%%%%%%%%%%%%%%%%%%%%%%%%%%%%%%%%%%%%%%%%%%%%%%%%%%%%%%%%%%%%%%%%%%%%%%%%%%%%
\subsection{Flags}
\label{sec:flags}

The package makes it easy to generate different versions
of the main or child documents.
To this end compilation flags can be defined
and assigned different default values.
They will be particularly useful in conjunction
with the forwarding mechanism described in \secref{sec:forward}.

For example, it may be useful to have a flag |\version|
which can be set to |draft| or |final|.
The document source will contain some conditional code
depending on the value of |\version|.
Suppose further, the flag should default to |final| for the main file
and to |draft| for child files
which is a natural assignment for editing the document.
This is achieved by placing the following code
in the preamble of the main document
(below the |\childdocmain| directive):
%
\begin{center}
\begin{tabular}{l}
|\ifchilddoc|\\
|\providecommand{\version}{draft}|\\
|\||else|\\
|\providecommand{\version}{final}|\\
|\||fi|
\end{tabular}
\end{center}
%
The definition by |\providecommand| makes sure
that previous definitions are not overwritten.
Further statements |\providecommand{\version}{...}|
can thus be added before the above code to override it.

For the main file, one might add a line
(between |\childdocmain| and the above block)
%
\begin{center}
|%\ifchilddoc\||else\providecommand{\version}{draft}\||fi|
\end{center}
%
which can be uncommented to produce a draft version.
Likewise one can add a line to the very top of a child file
(above the |\childdocof{|\textit{main}|}| directive)
%
\begin{center}
|%\providecommand{\version}{final}|
\end{center}
%
which can be uncommented to produce the final version of this child document.

%%%%%%%%%%%%%%%%%%%%%%%%%%%%%%%%%%%%%%%%%%%%%%%%%%%%%%%%%%%%%%%%%%%%%%%%%%%%%%%%
\subsection{Forwarding}
\label{sec:forward}

Different versions of the main or child documents
using compilation flags as described in \secref{sec:flags}
can be (permanently) stored in different files
for convenient compilation, viewing and distribution.
To this end, the package defines a command
to pass on compilation to a different file:

%%%%%%%%%%%%%%%%%%%%%%%%%%%%%%%%%%%%%%%%
\DescribeMacro{\childdocforward}
The command |\childdocforward| redirects processing to
another source file:
%
\begin{center}
\begin{tabular}{l}
|\input{childdoc.def}|\\
|\childdocforward[|\textit{main}|]{|\textit{dest}|}|\\
\end{tabular}
\end{center}
%
The argument \textit{dest} is the destination file
(without extension).
It should be the main file or one of the child files.
Note that further \textsf{childdoc} directives
such as |\childdocof| and |\childdocforward|
in the indicated file will be processed in this form.
The optional argument \textit{main}
passes on directly to the main file \textit{main}
while pretending to compile the child \textit{dest}.
This form behaves as if \textit{dest}
issues |\childdocof{|\textit{main}|}| right away,
and no further \textsf{childdoc} directives will be processed.

%%%%%%%%%%%%%%%%%%%%%%%%%%%%%%%%%%%%%%%%
\DescribeMacro{\...prefix}
In the alternative form |\childdocforwardprefix|,
%
\begin{center}
\begin{tabular}{l}
|\input{childdoc.def}|\\
|\childdocforwardprefix[|\textit{main}|]{|\textit{prefix}|}{|\textit{dest}|}|
\end{tabular}
\end{center}
%
the destination file is determined by a pattern
depending on the current file:
To make this work, the current file must be called
`{\textit{prefix}\hspace{0.2em}\textit{suffix}}'
with \textit{prefix} matching precisely the argument.
Processing is then passed on to the file
`{\textit{dest}\hspace{0.2em}\textit{suffix}}'.
Surely, the same effect is achieved by
directly specifying the
argument `{\textit{dest}\hspace{0.2em}\textit{suffix}}'
in the first form.
However, that requires to set up a different file
for each child. With the alternative form of the command
all these files can have exactly the same content
which simplifies setting them up and maintaining them.

For example, the following file |draft.tex|
with a compilation flag |\version| as described in \secref{sec:flags}
compiles the main document as a draft:
%
\begin{center}
\begin{tabular}{l}
|\def\version{draft}|\\
|\input{childdoc.def}|\\
|\childdocforward{|\textit{main}|}|
\end{tabular}
\end{center}
%
Likewise, the following files |final|\textit{nn}|.tex|
compile the final version of the child document
|child|\textit{nn}|.tex|:
%
\begin{center}
\begin{tabular}{l}
|\def\version{final}|\\
|\input{childdoc.def}|\\
|\childdocforwardprefix{final}{child}|
\end{tabular}
\end{center}
%

Note that when several versions of a main file and/or of each child file
are to be generated, it may be convenient to set up a |Makefile| or
shell script to automatise the process.

%%%%%%%%%%%%%%%%%%%%%%%%%%%%%%%%%%%%%%%%%%%%%%%%%%%%%%%%%%%%%%%%%%%%%%%%%%%%%%%%
\subsection{Command Line Processing}
\label{sec:commandline}

The effect of redirection files can also be achieved by invoking
the \LaTeX{} compiler with a more elaborate command line.
Most conveniently this should be done as part
of a shell script or a |Makefile|.

When using \textsf{childdoc} in the main file, the following
command lines effectively perform a redirection
(note that depending on the shell being used,
backslashes may have to be doubled: `|\|' $\to$ `|\\|'):
%
\begin{center}
|... -jobname "|\textit{target}|" |\\|"|[\textit{flags}]%
|\input{childdoc.def}\childdocforward[|\textit{main}|]{|\textit{dest}|}"|
\end{center}
%
Here \textit{target} is the name of the output file,
\textit{main} is the name of the main file
and \textit{dest} is the name of the main or child file to be processed
(all filenames without extensions).
The optional argument \textit{main} can be omitted
if \textit{main} matches \textit{dest}.
Optionally, compilation \textit{flags} can be defined via |\def| commands.
This command line makes the \TeX{} engine believe
it is compiling the file \textit{target}
whose content is specified as the latter parameter.
The provided code then forwards the processing to
\textit{main} or \textit{dest} as described in \secref{sec:forward}.

%%%%%%%%%%%%%%%%%%%%%%%%%%%%%%%%%%%%%%%%%%%%%%%%%%%%%%%%%%%%%%%%%%%%%%%%%%%%%%%%
\subsection{Include by Input}
\label{sec:input}

Including child documents by |\include| has some restrictions by design.
Most notably, the content of a child document always occupies
its own set of pages; pages cannot be shared between child documents.
Usually, this behaviour makes perfect sense
because each child document contain an essential part of the document.
However, in some situations it may be desirable to compose
a document from a collection of parts
without having mandatory page breaks between then.
For this case, the package
provides a mechanism to include parts
by |\input| which can also be processed individually.
However, by construction this mechanism
requires manual handling of the content to be output.

%%%%%%%%%%%%%%%%%%%%%%%%%%%%%%%%%%%%%%%%
\DescribeMacro{\ifchilddocmanual}
The main file should be prepared as usual, see \secref{sec:include}.
However, the document body must make a distinction
between processing of an individual part and of the main document, e.g.:
%
\begin{center}
\begin{tabular}{l}
|\ifchilddocmanual|\\
|\input{\childdocname}|\\
|\||else|\\
\textit{document body with }|\input{|\textit{part}|}|\\
|\||fi|
\end{tabular}
\end{center}
%
The conditional |\ifchilddocmanual| is true whenever
a part to be included by |\input| is being compiled,
and the name of the part is stored in |\childdocname|.

%%%%%%%%%%%%%%%%%%%%%%%%%%%%%%%%%%%%%%%%
\DescribeMacro{\childdocby}
Each part to be included by |\input| should start with:
%
\begin{center}
\begin{tabular}{l}
|\input{childdoc.def}|\\
|\childdocby{|\textit{main}|}|\\
\end{tabular}
\end{center}
%
The directive |\childdocby| is similar to |\childdocof|
described in \secref{sec:include},
but the subsequent selection of content must be done manually.
To that end, both |\ifchilddoc| and |\ifchilddocmanual|
will be true upon processing of a part,
and the name of the part is stored in |\childdocname|.
Note that |\jobname| will be set to the filename of the current part
so that each part receives an individual |.aux| file
that does not interfere with the |.aux| file(s) of the main document.
This behaviour can be altered by the alternative form
|\childdocby[*]{|\textit{main}|}| (with a non-empty optional argument)
which uses the |.aux| file of the main document
by setting |\jobname| to \textit{main}.

%%%%%%%%%%%%%%%%%%%%%%%%%%%%%%%%%%%%%%%%%%%%%%%%%%%%%%%%%%%%%%%%%%%%%%%%%%%%%%%%
\subsection{Driver Development}
\label{sec:driver}

The \textsf{childdoc} mechanism can also be use for the development
of definition files such as \LaTeX{} styles or classes.
This case differs from the above setup with multiple parts
included by |\include| in that no |\includeonly| should be invoked.
This can be achieved by starting the include file
(before |\ProvidesPackage|) with:
%
\begin{center}
\begin{tabular}{l}
|\input{childdoc.def}|\\
|\childdocforward{|\textit{main}|}|\\
\end{tabular}
\end{center}
%
or alternatively with:
%
\begin{center}
\begin{tabular}{l}
|\input{childdoc.def}|\\
|\childdocby{|\textit{main}|}|\\
\end{tabular}
\end{center}
%
Both forms have slightly different effects as described above.
The main file is prepared as usual, see \secref{sec:include}.

%%%%%%%%%%%%%%%%%%%%%%%%%%%%%%%%%%%%%%%%%%%%%%%%%%%%%%%%%%%%%%%%%%%%%%%%%%%%%%%%
\subsection{Legacy Detection}
\label{sec:detection}

The directive |\childdocmain| in the main file can detect
whether the complete document or merely a child is to be compiled
even without using the directive |\childdocof|.
This method is deprecated because it is less robust
and there is no compelling reason to use it;
it is merely provided for backward compatibility
and it may be removed in future versions.

If the detection mechanism is to be used,
it is mandatory to correctly specify
the filename of the main file as the argument of |\childdocmain|:
%
\begin{center}
\begin{tabular}{l}
|\input{childdoc.def}|\\
|\childdocmain{|\textit{main}|}|\\
\end{tabular}
\end{center}
%
If |\jobname| does not match the argument \textit{main} of |\childdocmain|,
it is assumed that |\jobname| points to the child file to be compiled.
When using |\childdocmain| with the main file specified as argument,
it suffices to start a child file
with just |\input{|\textit{main}|}|
without loading of the package and using |\childdocof|.
If instead all processing is done
with the appropriate \textsf{childdoc} directives,
the argument of \textit{main} of |\childdocmain| can be empty.

An alternative version of the command line processing described
in \secref{sec:commandline} using the detection mechanism reads:
%
\begin{center}
|... -jobname "|\textit{target}|" "|[\textit{flags}]%
[|\def\jobname{|\textit{dest}|}|]|\input{|\textit{main}|}"|
\end{center}

%%%%%%%%%%%%%%%%%%%%%%%%%%%%%%%%%%%%%%%%%%%%%%%%%%%%%%%%%%%%%%%%%%%%%%%%%%%%%%%%
\subsection{Manual Code}
\label{sec:manual}

In case one cannot be certain whether the definitions file |childdoc.def|
is installed on the target \TeX{} distribution
and one prefers not to ship it,
it is conceivable to paste a few relevant commands into the sources.

To that end, drop all statements |\input{childdoc.def}|
and perform the replacements as outlined below.
Instead of |\childdocmain{|\textit{main}|}| add the following code
to the top of the main file:
%
\begin{center}
\begin{tabular}{l}
|\||ifdefined\childdocname\endinput\||fi\newif\ifchilddoc|\\
|\edef\childdocname{\scantokens\expandafter{\jobname\noexpand}}|\\
|\def\childdocmain{|\textit{main}|}\||ifx\childdocmain\childdocname\||else|\\
|\childdoctrue\includeonly{\childdocname}\let\jobname\childdocmain\||fi|\\
\end{tabular}
\end{center}
%
Instead of |\childdocof{|\textit{main}|}| just include the main file
at the top of each child file:
%
\begin{center}
|\input{|\textit{main}|}|
\end{center}
%
A simple redirection |\childdocforward{|\textit{dest}|}| is achieved by:
%
\begin{center}
|\def\jobname{|\textit{dest}|}\input{\jobname}|
\end{center}
%
The redirection with prefix
|\childdocforwardprefix[|\textit{prefix}|]{|\textit{dest}|}|
is accomplished by:
%
\begin{center}
\begin{tabular}{l}
|{\edef\jobname{\scantokens\expandafter{\jobname\noexpand}}|\\
|\def\redirectjob |\textit{prefix}|#1~~~{\gdef\jobname{|\textit{dest}|#1}}|\\
|\expandafter\redirectjob\jobname~~~}\input{\jobname}|
\end{tabular}
\end{center}

In an alternative approach,
child documents can be compiled by a specific command line
without additional code or specific definitions:
%
\begin{center}
|... -jobname "|\textit{target}|" "|[\textit{flags}]%
|\includeonly{|\textit{dest}|}\input{|\textit{main}|}"|
\end{center}
%

%%%%%%%%%%%%%%%%%%%%%%%%%%%%%%%%%%%%%%%%%%%%%%%%%%%%%%%%%%%%%%%%%%%%%%%%%%%%%%%%
%%%%%%%%%%%%%%%%%%%%%%%%%%%%%%%%%%%%%%%%%%%%%%%%%%%%%%%%%%%%%%%%%%%%%%%%%%%%%%%%
\section{Information}

%%%%%%%%%%%%%%%%%%%%%%%%%%%%%%%%%%%%%%%%%%%%%%%%%%%%%%%%%%%%%%%%%%%%%%%%%%%%%%%%
\subsection{Copyright}

Copyright \copyright{} 2017--2018 Niklas Beisert

This work may be distributed and/or modified under the
conditions of the \LaTeX{} Project Public License, either version 1.3
of this license or (at your option) any later version.
The latest version of this license is in
  \url{http://www.latex-project.org/lppl.txt}
and version 1.3 or later is part of all distributions of \LaTeX{}
version 2005/12/01 or later.

This work has the LPPL maintenance status `maintained'.

The Current Maintainer of this work is Niklas Beisert.

This work consists of the files |README.txt|, |childdoc.ins| and |childdoc.dtx|
as well as the derived files |childdoc.def|, |cdocsamp.tex|
with |cdocsch1.tex|, |cdocsch2.tex|, |cdocspt3.tex|, |cdocspt4.tex|,
|cdocsdrf.tex|, |cdocsfn1.tex|, |cdocsfn2.tex|
as well as |childdoc.pdf|.

%%%%%%%%%%%%%%%%%%%%%%%%%%%%%%%%%%%%%%%%%%%%%%%%%%%%%%%%%%%%%%%%%%%%%%%%%%%%%%%%
\subsection{Files and Installation}

The package consists of the files:
%
\begin{center}
\begin{tabular}{ll}
    |README.txt|   & readme file \\
    |childdoc.ins| & installation file \\
    |childdoc.dtx| & source file \\
    |childdoc.def| & definition file \\
    |cdocsamp.tex| & sample main file \\
    |cdocsch1.tex| & sample include file \\
    |cdocsch2.tex| & sample include file \\
    |cdocspt3.tex| & sample part file \\
    |cdocspt4.tex| & sample part file \\
    |cdocsdrf.tex| & sample redirection file \\
    |cdocsfn1.tex| & sample redirection file \\
    |cdocsfn2.tex| & sample redirection file \\
    |childdoc.pdf| & manual
\end{tabular}
\end{center}
%
The distribution consists of the files
|README.txt|, |childdoc.ins| and |childdoc.dtx|.
%
\begin{itemize}
\item
Run (pdf)\LaTeX{} on |childdoc.dtx|
to compile the manual |childdoc.pdf| (this file).
\item
Run \LaTeX{} on |childdoc.ins| to create the definitions file |childdoc.def|
and the sample |cdocsamp.tex| with include files
|cdocsch1.tex|, |cdocsch2.tex|, |cdocspt3.tex|, |cdocspt4.tex|,
|cdocsdrf.tex|, |cdocsfn1.tex|, |cdocsfn2.tex|.
Then copy the file |childdoc.def| to an appropriate directory of your \LaTeX{}
distribution, e.g.\ \textit{texmf-root}|/tex/latex/childdoc|.
\end{itemize}

%%%%%%%%%%%%%%%%%%%%%%%%%%%%%%%%%%%%%%%%%%%%%%%%%%%%%%%%%%%%%%%%%%%%%%%%%%%%%%%%
\subsection{Related CTAN Packages}

There are several other packages which offer a similar functionality:
%
\begin{itemize}
\item
The packages
\href{http://ctan.org/pkg/docmute}{\textsf{docmute}},
\href{http://ctan.org/pkg/includex}{\textsf{includex}} and
\href{http://ctan.org/pkg/standalone}{\textsf{standalone}}
provide commands to include only the document body of
a child file thus allowing both files to be compiled individually.
\item
The packages \href{http://ctan.org/pkg/subdocs}{\textsf{subdocs}}
and \href{http://ctan.org/pkg/subfiles}{\textsf{subfiles}}
provide structures in which the main and child documents can be
encapsulated and allowing them to be compiled individually.
The inclusion mechanism is different from the conventional |\include|.
\item
The package \href{http://ctan.org/pkg/combine}{\textsf{combine}}
is an elaborate solution to combine several documents into one.
\end{itemize}
%
See also the CTAN topic \href{http://ctan.org/topic/subdocs}{\textsf{subdocs}}
for further related packages.
The present package differs from the above solutions in that
a document structure constructed with the conventional |\include| mechanism
just needs two extra commands at the top of every file
such that all constituent files can be compiled individually.

%%%%%%%%%%%%%%%%%%%%%%%%%%%%%%%%%%%%%%%%%%%%%%%%%%%%%%%%%%%%%%%%%%%%%%%%%%%%%%%%
%\subsection{Feature Suggestions}
%
%The following is a list of features which may be useful for future
%versions of this package:
%%
%\begin{itemize}
%\item
%\ldots
%\end{itemize}

%%%%%%%%%%%%%%%%%%%%%%%%%%%%%%%%%%%%%%%%%%%%%%%%%%%%%%%%%%%%%%%%%%%%%%%%%%%%%%%%
\subsection{Revision History}

%%%%%%%%%%%%%%%%%%%%%%%%%%%%%%%%%%%%%%%%
\paragraph{v2.0:} 2018/12/30

\begin{itemize}
\item
immediate forward processing
\item
added |\childdocby| mechanism
\item
manual restructured
\end{itemize}

%%%%%%%%%%%%%%%%%%%%%%%%%%%%%%%%%%%%%%%%
\paragraph{v1.6:} 2018/01/17

\begin{itemize}
\item
application for development of include files
\item
corrections to manual
\end{itemize}

%%%%%%%%%%%%%%%%%%%%%%%%%%%%%%%%%%%%%%%%
\paragraph{v1.5:} 2017/05/21

\begin{itemize}
\item
more complete structuring introduced
\item
|\childdocof| introduced
\item
|\childdoc| renamed to |\childdocmain|
\item
|\childredirect| renamed to |\childdocforward| and |\childdocforwardprefix|
and functionality expanded
\end{itemize}

%%%%%%%%%%%%%%%%%%%%%%%%%%%%%%%%%%%%%%%%
\paragraph{v1.0:} 2017/04/27

\begin{itemize}
\item
manual and install package
\item
first version published on CTAN
\end{itemize}

%%%%%%%%%%%%%%%%%%%%%%%%%%%%%%%%%%%%%%%%
\paragraph{v0.6:} 2017/04/26

\begin{itemize}
\item
redirection mechanism added
\end{itemize}

%%%%%%%%%%%%%%%%%%%%%%%%%%%%%%%%%%%%%%%%
\paragraph{v0.5:} 2017/04/26

\begin{itemize}
\item
functionality in definition file
\end{itemize}


%%%%%%%%%%%%%%%%%%%%%%%%%%%%%%%%%%%%%%%%%%%%%%%%%%%%%%%%%%%%%%%%%%%%%%%%%%%%%%%%
%%%%%%%%%%%%%%%%%%%%%%%%%%%%%%%%%%%%%%%%%%%%%%%%%%%%%%%%%%%%%%%%%%%%%%%%%%%%%%%%
%%%%%%%%%%%%%%%%%%%%%%%%%%%%%%%%%%%%%%%%%%%%%%%%%%%%%%%%%%%%%%%%%%%%%%%%%%%%%%%%
\appendix

\settowidth\MacroIndent{\rmfamily\scriptsize 000\ }

 \DocInput{childdoc.dtx}

\end{document}
%</driver>
% \fi
%
% %%%%%%%%%%%%%%%%%%%%%%%%%%%%%%%%%%%%%%%%%%%%%%%%%%%%%%%%%%%%%%%%%%%%%%%%%%%%%%
% %%%%%%%%%%%%%%%%%%%%%%%%%%%%%%%%%%%%%%%%%%%%%%%%%%%%%%%%%%%%%%%%%%%%%%%%%%%%%%
% \section{Sample}
%\iffalse
%<*samplemain>
%\fi
%
% The following presents a sample document
% with two chapters, two parts, a title page,
% a compile flag as well as three forwarding files to set the flag.
% It consists of eight |.tex| files:
% \begin{center}
% \begin{tabular}{ll}
% |cdocsamp.tex|&main file\\
% |cdocsch1.tex|&include file for chapter 1\\
% |cdocsch2.tex|&include file for chapter 2\\
% |cdocspt3.tex|&include file for part 3\\
% |cdocspt4.tex|&include file for part 4\\
% |cdocsdrf.tex|&forwarding file for main file in draft mode\\
% |cdocsfi1.tex|&forwarding file for final version of chapter 1\\
% |cdocsfi2.tex|&forwarding file for final version of chapter 2\\
% \end{tabular}
% \end{center}
% Each of the eight files can be compiled directly by the \LaTeX{} compiler.
%
% %%%%%%%%%%%%%%%%%%%%%%%%%%%%%%%%%%%%%%
% \paragraph{Main File.}
%
% The main file is called |cdocsamp.tex|.
%
% Load the \textsf{childdoc} definitions and
% declare the filename for the main document:
%    \begin{macrocode}
\input{childdoc.def}
\childdocmain{}
%    \end{macrocode}

% Optional override for |\version| flag:
%    \begin{macrocode}
%%\ifchilddoc\else\providecommand{\version}{draft}\fi
%    \end{macrocode}

% Define the default values for the |\version| flag
% (|final| for the main file and |draft| for childs):
%    \begin{macrocode}
\ifchilddoc
\providecommand{\version}{draft}
\else
\providecommand{\version}{final}
\fi
%    \end{macrocode}

% Load the standard document class:
%    \begin{macrocode}
\documentclass[12pt]{article}
%    \end{macrocode}

% Start the document body:
%    \begin{macrocode}
\begin{document}
%    \end{macrocode}

% Declare a title page.
% Print title, part of document being processed and version flag:
%    \begin{macrocode}
\addtocounter{page}{-1}
\begin{center}
{\LARGE\bfseries{}childdoc example\par}
\vspace{1cm}
\ifchilddoc
\ifchilddocmanual part\else chapter\fi:
`\childdocname' of `\childdocjob'\par
\else
main document: `\childdocjob'\par
\fi
version: \version\par
\end{center}
\newpage
%    \end{macrocode}

% Manually include selected file,
% otherwise process as usual:
%    \begin{macrocode}
\ifchilddocmanual
\section*{part `\childdocname'}
\input{\childdocname}
\else
%    \end{macrocode}

% Include the two chapters:
%    \begin{macrocode}
\include{cdocsch1}
\include{cdocsch2}
%    \end{macrocode}

% Include the two parts unless only chapters should be displayed:
%    \begin{macrocode}
\ifchilddoc\else
\section{part three}
\input{cdocspt3}
\section{part four}
\input{cdocspt4}
\fi
%    \end{macrocode}

% Process as usual until here:
%    \begin{macrocode}
\fi
%    \end{macrocode}

% End of document body:
%    \begin{macrocode}
\end{document}
%    \end{macrocode}
%\iffalse
%</samplemain>
%\fi
%
% %%%%%%%%%%%%%%%%%%%%%%%%%%%%%%%%%%%%%%
% \paragraph{Chapter Include Files.}
%
% The include files are called |cdocsch1.tex| and |cdocsch2.tex|.
%
%\iffalse
%<*samplechap1|samplechap2>
%\fi

% Optional override for |\version| flag:
%    \begin{macrocode}
%%\providecommand{\version}{final}
%    \end{macrocode}

% Include the main document:
%    \begin{macrocode}
\input{childdoc.def}
\childdocof{cdocsamp}
%    \end{macrocode}

%\iffalse
%</samplechap1|samplechap2>
%\fi
%
%\iffalse
%<*samplechap1>
%\fi
% Some text for chapter 1:
%    \begin{macrocode}
\section{one}
some text in chapter one
%    \end{macrocode}

%\iffalse
%</samplechap1>
%\fi
% Some text for chapter 2:
%\iffalse
%<*samplechap2>
%\fi
%    \begin{macrocode}
\section{two}
more text in chapter two
%    \end{macrocode}

%\iffalse
%</samplechap2>
%\fi
%
% %%%%%%%%%%%%%%%%%%%%%%%%%%%%%%%%%%%%%%
% \paragraph{Part Include Files.}
%
% The include files are called |cdocspt3.tex| and |cdocspt4.tex|.
%
%\iffalse
%<*samplepart3|samplepart4>
%\fi

% Optional override for |\version| flag:
%    \begin{macrocode}
%%\providecommand{\version}{final}
%    \end{macrocode}

% Include the main document:
%    \begin{macrocode}
\input{childdoc.def}
\childdocby{cdocsamp}
%    \end{macrocode}

%\iffalse
%</samplepart3|samplepart4>
%\fi
%
%\iffalse
%<*samplepart3>
%\fi
% Some text for part 3:
%    \begin{macrocode}
some text in part three
%    \end{macrocode}

%\iffalse
%</samplepart3>
%\fi
% Some text for part 4:
%\iffalse
%<*samplepart4>
%\fi
%    \begin{macrocode}
more text in part four
%    \end{macrocode}

%\iffalse
%</samplepart4>
%\fi
%
% %%%%%%%%%%%%%%%%%%%%%%%%%%%%%%%%%%%%%%
% \paragraph{Forwarding for a Complete Draft.}
%
% The following forwarding file |cdocsdrf.tex|
% compiles the main document in draft mode:
%\iffalse
%<*sampledraft>
%\fi
%    \begin{macrocode}
\def\version{draft}
\input{childdoc.def}
\childdocforward{cdocsamp}
%    \end{macrocode}

%\iffalse
%</sampledraft>
%\fi
%
% %%%%%%%%%%%%%%%%%%%%%%%%%%%%%%%%%%%%%%
% \paragraph{Forwarding for Final Version of the Chapters.}
%
% The following forwarding files |cdocsfn1.tex| and |cdocsfn2.tex|
% (with identical content)
% compile the final versions of the child documents
% |cdocsch1.tex| and |cdocsch2.tex|, respectively:
%\iffalse
%<*samplefinal>
%\fi
%    \begin{macrocode}
\def\version{final}
\input{childdoc.def}
\childdocforwardprefix[cdocsamp]{cdocsfn}{cdocsch}
%    \end{macrocode}

%\iffalse
%</samplefinal>
%\fi
%
% %%%%%%%%%%%%%%%%%%%%%%%%%%%%%%%%%%%%%%
% \paragraph{Command Line Processing.}
%
% The following three command lines generate the output files
% |cdocscld|, |cdocscl1| and |cdocscl2|
% which should be identical to
% |cdocsdrf|, |cdocsch1| and |cdocsfn2|, respectively:
% \begin{center}
% \begin{tabular}{l}
% |latex -jobname cdocscld \|\\
% |  "\def\version{draft}\input{childdoc.def}\childdocforward{cdocsamp}"|\\
% |latex -jobname cdocscl1 \|\\
% |  "\input{childdoc.def}\childdocforward[cdocsamp]{cdocsch1}"|\\
% |latex -jobname cdocscl2 \|\\
% |  "\def\version{final}\input{childdoc.def}\childdocforward{cdocsch2}"|
% \end{tabular}
% \end{center}
% Note that the trailing backslash on each first line
% merely continues the input to the second line
% (for convenient cut ant paste).
% Furthermore, the command |latex| can be replaced by any
% of its alternative versions such as |pdflatex|.
%
% %%%%%%%%%%%%%%%%%%%%%%%%%%%%%%%%%%%%%%%%%%%%%%%%%%%%%%%%%%%%%%%%%%%%%%%%%%%%%%
% %%%%%%%%%%%%%%%%%%%%%%%%%%%%%%%%%%%%%%%%%%%%%%%%%%%%%%%%%%%%%%%%%%%%%%%%%%%%%%
% \section{Implementation}
%\iffalse
%<*package>
%\fi
%
% This section describes the definitions file |childdoc.def|.

% The definitions cannot be loaded using |\usepackage| or |\RequirePackage|
% which has a mechanism to prevent loading a style file more than once.
% When loading the definitions by means of |\input|
% multiple instances have to be prevented manually:
%\iffalse
%This code needs to be before the `\ProvidesFile' directive
%which is defined at the beginning of this file.
%Therefore it is also placed there and commented out here.
%</package>
%<*discard>
%\fi
%    \begin{macrocode}
\ifdefined\childdocmain\endinput\fi
%    \end{macrocode}
%\iffalse
%</discard>
%<*package>
%\fi
%
% \macro{\ifchilddoc}
% \macro{\ifchilddocmanual}
% The conditional |\ifchilddoc| tells whether a
% child (true) or main (false) document is being compiled.
% The conditional |\ifchilddocmanual| tells whether
% the |\includeonly| mechanism is used (false) or
% the selection of child files must be performed manually (true).
% The definitions initialise to false:
%    \begin{macrocode}
\newif\ifchilddoc
\newif\ifchilddocmanual
%    \end{macrocode}

% \macro{\childdocname}
% \macro{\childdocjob}
% The macro |\childdocname| stores the name of the main document
% to be compiled. The macro |\childdocjob| stores the name of
% the document on which the \LaTeX{} compiler was originally invoked.
% The content of |\jobname| cannot be compared
% to filenames specified in the source due to different catcodes.
% The following code rescans |\jobname|, stores the result
% in |\childdocname| and saves a copy in |\childdocjob|:
%    \begin{macrocode}
\edef\childdocname{\scantokens\expandafter{\jobname\noexpand}}
\let\childdocjob\childdocname
%    \end{macrocode}

% \macro{\childdocdisable}
% The macro |\childdocdisable| prevents the main file
% from being processed more than once.
% At this stage, the main document command |\childdocmain|
% is assumed to be called once again where it should do nothing.
% Any subsequent call to it should prevent
% a secondary processing of the main document
% It overwrites the forwarding commands
% |\childdocof| and |\childdocforward|
% with empty macros to prevent further inclusions of the main document:
%    \begin{macrocode}
\newcommand{\childdocdisable}
{
  \renewcommand{\childdocmain}[1]{\renewcommand{\childdocmain}[1]{\endinput}}
  \renewcommand{\childdocof}[1]{}
  \renewcommand{\childdocby}[2][]{}
  \renewcommand{\childdocforward}[2][]{}
  \renewcommand{\childdocdisable}{}
}
%    \end{macrocode}

% \macro{\childdocmain}
% The macro |\childdocmain| is to be called at the top of the main file
% with nothing or the main filename (without extension) as argument.
% First, it breaks loops.
% If the argument is not empty and does not match |\childdocname|
% (which is set by the first inclusion of |childdoc.def|),
% |\ifchilddoc| is set to true, |\includeonly| is applied to the child file
% and |\jobname| is set to the main file
% (for proper handling of |.aux| files):
%    \begin{macrocode}
\newcommand{\childdocmain}[1]
{
  \childdocdisable\childdocmain{}
  \if?#1?\else
    \begingroup
      \def\childdoctmp{#1}
      \ifx\childdoctmp\childdocname
        \def\childdoctmp{}
      \else
        \def\childdoctmp
        {
          \childdoctrue
          \includeonly{\childdocname}
          \def\childdocjob{#1}
          \def\jobname{#1}
        }
      \fi
      \expandafter
    \endgroup
    \childdoctmp
  \fi
}
%    \end{macrocode}

% \macro{\childdocof}
% The command |\childdocof| redirects
% compilation to the main file |#1|.
%    \begin{macrocode}
\newcommand{\childdocof}[1]
{
  \childdocdisable
  \childdoctrue
  \includeonly{\childdocname}
  \def\jobname{#1}
  \def\childdocjob{#1}
  \input{#1}
}
%    \end{macrocode}

% \macro{\childdocby}
% The command |\childdocby| ....
%    \begin{macrocode}
\newcommand{\childdocby}[2][]
{
  \childdocdisable
  \childdoctrue
  \childdocmanualtrue
  \if?#1?\else
    \def\jobname{#2}
  \fi
  \def\childdocjob{#2}
  \input{#2}
  \endinput
}
%    \end{macrocode}

% \macro{\childdocforward}
% The command |\childdocforward| redirects
% compilation to the main file or
% (if the optional argument is given) a child file.
% Parameters are set as if the main file
% or a child file starting with |\childdocof| was compiled.
% Then compilation is handed over to the main file:
%    \begin{macrocode}
\newcommand{\childdocforward}[2][]
{
  \begingroup
    \if?#1?
      \def\childdoctmp
      {
        \def\childdocname{#2}
        \def\childdocjob{#2}
        \def\jobname{#2}
        \input{#2}
        \endinput
      }
    \else
      \def\childdoctmp
      {
        \childdocdisable
        \def\childdocname{#2}
        \childdoctrue
        \includeonly{#2}
        \def\childdocjob{#1}
        \def\jobname{#1}
        \input{#1}
        \endinput
      }
    \fi
    \expandafter
  \endgroup
  \childdoctmp
}
%    \end{macrocode}

% \macro{\childdocforwardprefix}
% The command |\childdocforwardprefix| redirects
% compilation to the main or a child file by means of a pattern.
% The prefix |#1| in the current filename is replaced by |#2|
% and the suffix of the current filename is kept
% (it is assumed that the filename does not contain the substring `|~~~|'
% which is used as a delimiter).
% Compilation is handed over to the new file by |\childdocforward|:
%    \begin{macrocode}
\newcommand{\childdocforwardprefix}[3][]
{
  \begingroup
    \def\childdocextract #2##1~~~{\def\childdoctmp{\childdocforward[#1]{#3##1}}}
    \expandafter\childdocextract\childdocname~~~
    \expandafter
  \endgroup
  \childdoctmp
}
%    \end{macrocode}

% \macro{\childdoc}
% The deprecated macro |\childdoc| is a legacy version of |\childdocmain|:
%    \begin{macrocode}
\newcommand{\childdoc}{\childdocmain}
%    \end{macrocode}

% \macro{\childdocredirect}
% The deprecated macro |\childdocredirect| is a legacy version
% of |\childdocforward| and |\childdocforwardprefix|:
%    \begin{macrocode}
\newcommand{\childdocredirect}[2][]
{
  \begingroup
    \if?#1?
      \def\childdoctmp{\childdocforward{#2}}
    \else
      \def\childdoctmp{\childdocforwardprefix{#1}{#2}}
    \fi
    \expandafter
  \endgroup
  \childdoctmp
}
%    \end{macrocode}

%\iffalse
%</package>
%\fi
%
\endinput
|\\
|\childdocforwardprefix{final}{child}|
\end{tabular}
\end{center}
%

Note that when several versions of a main file and/or of each child file
are to be generated, it may be convenient to set up a |Makefile| or
shell script to automatise the process.

%%%%%%%%%%%%%%%%%%%%%%%%%%%%%%%%%%%%%%%%%%%%%%%%%%%%%%%%%%%%%%%%%%%%%%%%%%%%%%%%
\subsection{Command Line Processing}
\label{sec:commandline}

The effect of redirection files can also be achieved by invoking
the \LaTeX{} compiler with a more elaborate command line.
Most conveniently this should be done as part
of a shell script or a |Makefile|.

When using \textsf{childdoc} in the main file, the following
command lines effectively perform a redirection
(note that depending on the shell being used,
backslashes may have to be doubled: `|\|' $\to$ `|\\|'):
%
\begin{center}
|... -jobname "|\textit{target}|" |\\|"|[\textit{flags}]%
|% \iffalse
%
% childdoc.dtx Copyright (C) 2017-2018 Niklas Beisert
%
% This work may be distributed and/or modified under the
% conditions of the LaTeX Project Public License, either version 1.3
% of this license or (at your option) any later version.
% The latest version of this license is in
%   http://www.latex-project.org/lppl.txt
% and version 1.3 or later is part of all distributions of LaTeX
% version 2005/12/01 or later.
%
% This work has the LPPL maintenance status `maintained'.
%
% The Current Maintainer of this work is Niklas Beisert.
%
% This work consists of the files childdoc.dtx and childdoc.ins
% and the derived files childdoc.def and cdocsamp.tex with
% cdocsch1.tex, cdocsch2.tex, cdocsdrf.tex, cdocsfn1.tex, cdocsfn2.tex.
%
%<package>\ifdefined\childdocmain\endinput\fi
%<package>\ProvidesFile{childdoc.def}[2018/12/30 v2.0 child document driver]
%<samplemain>\ProvidesFile{cdocsamp.tex}[2018/12/30 v2.0 sample for childdoc]
%<*driver>
%\ProvidesFile{childdoc.drv}[2018/12/30 v2.0 childdoc reference manual file]
\PassOptionsToClass{10pt,a4paper}{article}
\documentclass{ltxdoc}

\usepackage[margin=35mm]{geometry}
\usepackage{hyperref}
\usepackage{hyperxmp}
\usepackage[usenames]{color}

\hypersetup{colorlinks=true}
\hypersetup{pdfstartview=FitH}
\hypersetup{pdfpagemode=UseNone}
\hypersetup{pdfsource={}}
\hypersetup{pdflang={en-UK}}
\hypersetup{pdfcopyright={Copyright 2017-2018 Niklas Beisert.
  This work may be distributed and/or modified under the
  conditions of the LaTeX Project Public License, either version 1.3
  of this license or (at your option) any later version.}}
\hypersetup{pdflicenseurl={http://www.latex-project.org/lppl.txt}}
\hypersetup{pdfcontactaddress={ETH Zurich, ITP, HIT K,
  Wolfgang-Pauli-Strasse 27}}
\hypersetup{pdfcontactpostcode={8093}}
\hypersetup{pdfcontactcity={Zurich}}
\hypersetup{pdfcontactcountry={Switzerland}}
\hypersetup{pdfcontactemail={nbeisert@itp.phys.ethz.ch}}
\hypersetup{pdfcontacturl={http://people.phys.ethz.ch/\xmptilde nbeisert/}}

\newcommand{\secref}[1]{\hyperref[#1]{section \ref*{#1}}}

\parskip1ex
\parindent0pt
\let\olditemize\itemize
\def\itemize{\olditemize\parskip0pt}

\begin{document}

\title{The \textsf{childdoc} Package}
\hypersetup{pdftitle={The childdoc Package}}
\author{Niklas Beisert\\[2ex]
  Institut f\"ur Theoretische Physik\\
  Eidgen\"ossische Technische Hochschule Z\"urich\\
  Wolfgang-Pauli-Strasse 27, 8093 Z\"urich, Switzerland\\[1ex]
  \href{mailto:nbeisert@itp.phys.ethz.ch}
  {\texttt{nbeisert@itp.phys.ethz.ch}}}
\hypersetup{pdfauthor={Niklas Beisert}}
\hypersetup{pdfsubject={Manual for the LaTeX2e Package childdoc}}
\date{30 December 2018, \textsf{v2.0}}
\maketitle

\begin{abstract}\noindent
\textsf{childdoc} is a \LaTeXe{} package
that enables the direct compilation
of document sections included by |\include|
to individual files.
\end{abstract}

\begingroup
\parskip0ex
\tableofcontents
\endgroup

%%%%%%%%%%%%%%%%%%%%%%%%%%%%%%%%%%%%%%%%%%%%%%%%%%%%%%%%%%%%%%%%%%%%%%%%%%%%%%%%
%%%%%%%%%%%%%%%%%%%%%%%%%%%%%%%%%%%%%%%%%%%%%%%%%%%%%%%%%%%%%%%%%%%%%%%%%%%%%%%%
\section{Introduction}

\LaTeX{} provides a mechanism to structure a large document (such as a book)
into a main file and several child files (containing the chapters)
using the |\include| command.
This mechanism is beneficial for documents
which span hundreds of pages in order to
make the source file(s) more manageable.
Moreover, compilation can be restricted to
selected child files by means of the |\includeonly| command.
The latter feature can be used to reduce the compilation time while editing
(this was significantly more useful in the earlier days of \LaTeX{})
or to generate a smaller document which is easier to navigate.
Another application of |\includeonly| is to generate
documents consisting of selected parts of the complete document.

However, there are a few drawbacks of the plain |\include| mechanism:
\begin{itemize}
\item
The child files cannot be compiled on their own,
they can only be compiled via the main file.
A naive editing environment
(such as a text editor with an option
to have the current file processed by \LaTeX)
may require one to switch to the main file before compiling;
attempting to compile the child file produces errors.
\item
The main file must be modified (each time)
to adjust the |\includeonly| command
to the present needs. This easily leaves the main file in a messy state.
\item
The generated document will always carry the filename
of the main document. This is inconvenient if
several child files are to be compiled and
to be kept for distribution.
\end{itemize}

The present package provides a simple interface
to make child files individually compilable by \LaTeX{}.
Compiling a child file then has the same effect as compiling
the main file with an |\includeonly| command
to select the appropriate child.
Moreover the generated document will carry the name of the child
rather than the main file.
This resolves all three above issues.

This feature is meant to make the editing of books,
thesis documents and lecture notes somewhat more convenient.
However, the package can also be used efficiently for
composing a series of documents (such as exercise sheets)
which are typically distributed individually.
It then assists the author in generating the individual documents
(potentially in different versions)
as well as a document containing the collected series.
Another application is in developing style files
or other kinds of included material
where compilation of the style file could redirect
to a sample or test file.

%%%%%%%%%%%%%%%%%%%%%%%%%%%%%%%%%%%%%%%%%%%%%%%%%%%%%%%%%%%%%%%%%%%%%%%%%%%%%%%%
%%%%%%%%%%%%%%%%%%%%%%%%%%%%%%%%%%%%%%%%%%%%%%%%%%%%%%%%%%%%%%%%%%%%%%%%%%%%%%%%
\section{Usage}

First of all, the package \textsf{childdoc} is \emph{not} a standard
\LaTeXe{} |.sty| style file! Therefore it needs to be invoked in
a non-standard way.

%%%%%%%%%%%%%%%%%%%%%%%%%%%%%%%%%%%%%%%%%%%%%%%%%%%%%%%%%%%%%%%%%%%%%%%%%%%%%%%%
\subsection{Included Files}
\label{sec:include}

%%%%%%%%%%%%%%%%%%%%%%%%%%%%%%%%%%%%%%%%
\DescribeMacro{\childdocmain}
To use the package, add the commands
\begin{center}
\begin{tabular}{l}
|\input{childdoc.def}|\\
|\childdocmain{}|\\
\end{tabular}
\end{center}
at the very top of the main \LaTeX{} file,
in particular \emph{before} the |\documentclass| statement!
The argument of |\childdocmain| should be left empty
(but it must be present).

%%%%%%%%%%%%%%%%%%%%%%%%%%%%%%%%%%%%%%%%
\DescribeMacro{\childdocof}
Furthermore, add the commands
\begin{center}
\begin{tabular}{l}
|\input{childdoc.def}|\\
|\childdocof{|\textit{main}|}|\\
\end{tabular}
\end{center}
at the top of every child file \textit{child}
which is included by |\include{|\textit{child}|}|
from within the main file
(or at least for those files to be compiled individually).
The argument \textit{main} must be the filename of the main file.

There are a couple of
considerations in setting up the main and child documents:

%%%%%%%%%%%%%%%%%%%%%%%%%%%%%%%%%%%%%%%%
\paragraph{Restrictions.}

Please note the following restrictions:
\begin{itemize}
\item
|\childdocmain| must be called with one argument \textit{main}
to ensure compatibility with earlier version of the package.
It must either be empty (|\childdocmain{}|)
or precisely match the filename of the main file in which it is specified.
See \secref{sec:detection} for further information.
\item
The filename \textit{main} must be specified without the |.tex| extension.
\item
The filename \textit{main} is case sensitive
(even in case-insensitive file systems)
due to internal string comparison.
\item
The argument \textit{main} should be fully expanded, it cannot be a macro.
\item
Subdirectories and special characters should be avoided in filenames.
\item
The command |\childdocmain{|\textit{main}|}| must be followed by a whitespace.
It should not be followed immediately by another command
or by a comment mark `|%|'.
This is because the \TeX{} parser reads the token immediately following
the argument of |\childdocmain| and puts it
at the beginning of every child section;
however, a white\-space is ignored.
\end{itemize}

%%%%%%%%%%%%%%%%%%%%%%%%%%%%%%%%%%%%%%%%
\paragraph{Content of Main File.}

It is advisable to place all content in the child files included by |\include|.
Any output contained in the main file will appear in all child documents
unless suppressed manually;
it cannot be suppressed automatically by the |\includeonly| directive
and thus should normally be avoided.
A method to include some content in the main file
by means of conditional processing is described in \secref{sec:conditional}.

%%%%%%%%%%%%%%%%%%%%%%%%%%%%%%%%%%%%%%%%
\paragraph{Page Numbering.}

When only a part of the document is compiled,
the appropriate numbering of pages
(as well as other status parameters)
is determined from the |.aux| files.
The latter contain information from previous passes.
However this information needs to propagate through
all intermediate child documents.
Therefore the page numbering in child documents may well
be inconsistent until the complete document is compiled at least once.

A useful (if unconventional) way to always ensure a consistent
page numbering is to restart the numbering in each child document
and denote the pages by `\textit{child}|.|\textit{page}'
where \textit{child} represents the chapter/section number of the child file.
This can be achieved by the command
|\numberwithin{page}{|\textit{child}|}|
of the \textsf{amsmath} package
where \textit{child} can be |chapter| or |section|
depending on the chosen structuring.
Alternatively, one can modify the macro |\thepage| appropriately
and reset the counter |page| at the start of each child file.

%%%%%%%%%%%%%%%%%%%%%%%%%%%%%%%%%%%%%%%%%%%%%%%%%%%%%%%%%%%%%%%%%%%%%%%%%%%%%%%%
\subsection{Conditional Processing}
\label{sec:conditional}

The package provides a mechanism to compile different versions
of a document. To customise the versions further some conditional processing
can come in handy to distinguish which version is being compiled.
The package provides two macros to describe the compilation context:

%%%%%%%%%%%%%%%%%%%%%%%%%%%%%%%%%%%%%%%%
\DescribeMacro{\ifchilddoc}
The conditional |\ifchilddoc| distinguishes between the compilation of
child documents and the main document:
%
\begin{center}
|\ifchilddoc |\textit{child-code}| |[|\||else |\textit{main-code}]| \||fi|
\end{center}

%%%%%%%%%%%%%%%%%%%%%%%%%%%%%%%%%%%%%%%%
\DescribeMacro{\childdocname}
\DescribeMacro{\childdocjob}
The macro |\childdocname| contains the filename (without extension)
of the main or child file being processed.
Note that |\childdocjob| will always contain the name of the main file.

%%%%%%%%%%%%%%%%%%%%%%%%%%%%%%%%%%%%%%%%
\paragraph{Title Page.}

Conditional processing can be used to include a title or banner page
in the main document when proper precautions are taken.
Importantly, the code in the main file should ensure that the page counter
(as well as other status parameters which are stored in the |.aux| files)
takes the same value after the conditional processing.
Otherwise the page numbers may take divergent values
depending on which part is compiled.

For example, a title page could be declared by:
%
\begin{center}
\begin{tabular}{l}
|\ifchilddoc\||else|\\
|\addtocounter{page}{-1}|\\
\textit{code for title page}\\
|\newpage|\\
|\||fi|
\end{tabular}
\end{center}
%
A banner page for the child documents can be generated by:
%
\begin{center}
\begin{tabular}{l}
|\ifchilddoc|\\
|\addtocounter{page}{-1}|\\
\textit{code for banner page}\\
|\newpage|\\
|\||fi|
\end{tabular}
\end{center}
%
Here one could write a message such as:
\begin{center}
|This is the part \childdocname{} of \childdocjob{}.|
\end{center}

%%%%%%%%%%%%%%%%%%%%%%%%%%%%%%%%%%%%%%%%%%%%%%%%%%%%%%%%%%%%%%%%%%%%%%%%%%%%%%%%
\subsection{Flags}
\label{sec:flags}

The package makes it easy to generate different versions
of the main or child documents.
To this end compilation flags can be defined
and assigned different default values.
They will be particularly useful in conjunction
with the forwarding mechanism described in \secref{sec:forward}.

For example, it may be useful to have a flag |\version|
which can be set to |draft| or |final|.
The document source will contain some conditional code
depending on the value of |\version|.
Suppose further, the flag should default to |final| for the main file
and to |draft| for child files
which is a natural assignment for editing the document.
This is achieved by placing the following code
in the preamble of the main document
(below the |\childdocmain| directive):
%
\begin{center}
\begin{tabular}{l}
|\ifchilddoc|\\
|\providecommand{\version}{draft}|\\
|\||else|\\
|\providecommand{\version}{final}|\\
|\||fi|
\end{tabular}
\end{center}
%
The definition by |\providecommand| makes sure
that previous definitions are not overwritten.
Further statements |\providecommand{\version}{...}|
can thus be added before the above code to override it.

For the main file, one might add a line
(between |\childdocmain| and the above block)
%
\begin{center}
|%\ifchilddoc\||else\providecommand{\version}{draft}\||fi|
\end{center}
%
which can be uncommented to produce a draft version.
Likewise one can add a line to the very top of a child file
(above the |\childdocof{|\textit{main}|}| directive)
%
\begin{center}
|%\providecommand{\version}{final}|
\end{center}
%
which can be uncommented to produce the final version of this child document.

%%%%%%%%%%%%%%%%%%%%%%%%%%%%%%%%%%%%%%%%%%%%%%%%%%%%%%%%%%%%%%%%%%%%%%%%%%%%%%%%
\subsection{Forwarding}
\label{sec:forward}

Different versions of the main or child documents
using compilation flags as described in \secref{sec:flags}
can be (permanently) stored in different files
for convenient compilation, viewing and distribution.
To this end, the package defines a command
to pass on compilation to a different file:

%%%%%%%%%%%%%%%%%%%%%%%%%%%%%%%%%%%%%%%%
\DescribeMacro{\childdocforward}
The command |\childdocforward| redirects processing to
another source file:
%
\begin{center}
\begin{tabular}{l}
|\input{childdoc.def}|\\
|\childdocforward[|\textit{main}|]{|\textit{dest}|}|\\
\end{tabular}
\end{center}
%
The argument \textit{dest} is the destination file
(without extension).
It should be the main file or one of the child files.
Note that further \textsf{childdoc} directives
such as |\childdocof| and |\childdocforward|
in the indicated file will be processed in this form.
The optional argument \textit{main}
passes on directly to the main file \textit{main}
while pretending to compile the child \textit{dest}.
This form behaves as if \textit{dest}
issues |\childdocof{|\textit{main}|}| right away,
and no further \textsf{childdoc} directives will be processed.

%%%%%%%%%%%%%%%%%%%%%%%%%%%%%%%%%%%%%%%%
\DescribeMacro{\...prefix}
In the alternative form |\childdocforwardprefix|,
%
\begin{center}
\begin{tabular}{l}
|\input{childdoc.def}|\\
|\childdocforwardprefix[|\textit{main}|]{|\textit{prefix}|}{|\textit{dest}|}|
\end{tabular}
\end{center}
%
the destination file is determined by a pattern
depending on the current file:
To make this work, the current file must be called
`{\textit{prefix}\hspace{0.2em}\textit{suffix}}'
with \textit{prefix} matching precisely the argument.
Processing is then passed on to the file
`{\textit{dest}\hspace{0.2em}\textit{suffix}}'.
Surely, the same effect is achieved by
directly specifying the
argument `{\textit{dest}\hspace{0.2em}\textit{suffix}}'
in the first form.
However, that requires to set up a different file
for each child. With the alternative form of the command
all these files can have exactly the same content
which simplifies setting them up and maintaining them.

For example, the following file |draft.tex|
with a compilation flag |\version| as described in \secref{sec:flags}
compiles the main document as a draft:
%
\begin{center}
\begin{tabular}{l}
|\def\version{draft}|\\
|\input{childdoc.def}|\\
|\childdocforward{|\textit{main}|}|
\end{tabular}
\end{center}
%
Likewise, the following files |final|\textit{nn}|.tex|
compile the final version of the child document
|child|\textit{nn}|.tex|:
%
\begin{center}
\begin{tabular}{l}
|\def\version{final}|\\
|\input{childdoc.def}|\\
|\childdocforwardprefix{final}{child}|
\end{tabular}
\end{center}
%

Note that when several versions of a main file and/or of each child file
are to be generated, it may be convenient to set up a |Makefile| or
shell script to automatise the process.

%%%%%%%%%%%%%%%%%%%%%%%%%%%%%%%%%%%%%%%%%%%%%%%%%%%%%%%%%%%%%%%%%%%%%%%%%%%%%%%%
\subsection{Command Line Processing}
\label{sec:commandline}

The effect of redirection files can also be achieved by invoking
the \LaTeX{} compiler with a more elaborate command line.
Most conveniently this should be done as part
of a shell script or a |Makefile|.

When using \textsf{childdoc} in the main file, the following
command lines effectively perform a redirection
(note that depending on the shell being used,
backslashes may have to be doubled: `|\|' $\to$ `|\\|'):
%
\begin{center}
|... -jobname "|\textit{target}|" |\\|"|[\textit{flags}]%
|\input{childdoc.def}\childdocforward[|\textit{main}|]{|\textit{dest}|}"|
\end{center}
%
Here \textit{target} is the name of the output file,
\textit{main} is the name of the main file
and \textit{dest} is the name of the main or child file to be processed
(all filenames without extensions).
The optional argument \textit{main} can be omitted
if \textit{main} matches \textit{dest}.
Optionally, compilation \textit{flags} can be defined via |\def| commands.
This command line makes the \TeX{} engine believe
it is compiling the file \textit{target}
whose content is specified as the latter parameter.
The provided code then forwards the processing to
\textit{main} or \textit{dest} as described in \secref{sec:forward}.

%%%%%%%%%%%%%%%%%%%%%%%%%%%%%%%%%%%%%%%%%%%%%%%%%%%%%%%%%%%%%%%%%%%%%%%%%%%%%%%%
\subsection{Include by Input}
\label{sec:input}

Including child documents by |\include| has some restrictions by design.
Most notably, the content of a child document always occupies
its own set of pages; pages cannot be shared between child documents.
Usually, this behaviour makes perfect sense
because each child document contain an essential part of the document.
However, in some situations it may be desirable to compose
a document from a collection of parts
without having mandatory page breaks between then.
For this case, the package
provides a mechanism to include parts
by |\input| which can also be processed individually.
However, by construction this mechanism
requires manual handling of the content to be output.

%%%%%%%%%%%%%%%%%%%%%%%%%%%%%%%%%%%%%%%%
\DescribeMacro{\ifchilddocmanual}
The main file should be prepared as usual, see \secref{sec:include}.
However, the document body must make a distinction
between processing of an individual part and of the main document, e.g.:
%
\begin{center}
\begin{tabular}{l}
|\ifchilddocmanual|\\
|\input{\childdocname}|\\
|\||else|\\
\textit{document body with }|\input{|\textit{part}|}|\\
|\||fi|
\end{tabular}
\end{center}
%
The conditional |\ifchilddocmanual| is true whenever
a part to be included by |\input| is being compiled,
and the name of the part is stored in |\childdocname|.

%%%%%%%%%%%%%%%%%%%%%%%%%%%%%%%%%%%%%%%%
\DescribeMacro{\childdocby}
Each part to be included by |\input| should start with:
%
\begin{center}
\begin{tabular}{l}
|\input{childdoc.def}|\\
|\childdocby{|\textit{main}|}|\\
\end{tabular}
\end{center}
%
The directive |\childdocby| is similar to |\childdocof|
described in \secref{sec:include},
but the subsequent selection of content must be done manually.
To that end, both |\ifchilddoc| and |\ifchilddocmanual|
will be true upon processing of a part,
and the name of the part is stored in |\childdocname|.
Note that |\jobname| will be set to the filename of the current part
so that each part receives an individual |.aux| file
that does not interfere with the |.aux| file(s) of the main document.
This behaviour can be altered by the alternative form
|\childdocby[*]{|\textit{main}|}| (with a non-empty optional argument)
which uses the |.aux| file of the main document
by setting |\jobname| to \textit{main}.

%%%%%%%%%%%%%%%%%%%%%%%%%%%%%%%%%%%%%%%%%%%%%%%%%%%%%%%%%%%%%%%%%%%%%%%%%%%%%%%%
\subsection{Driver Development}
\label{sec:driver}

The \textsf{childdoc} mechanism can also be use for the development
of definition files such as \LaTeX{} styles or classes.
This case differs from the above setup with multiple parts
included by |\include| in that no |\includeonly| should be invoked.
This can be achieved by starting the include file
(before |\ProvidesPackage|) with:
%
\begin{center}
\begin{tabular}{l}
|\input{childdoc.def}|\\
|\childdocforward{|\textit{main}|}|\\
\end{tabular}
\end{center}
%
or alternatively with:
%
\begin{center}
\begin{tabular}{l}
|\input{childdoc.def}|\\
|\childdocby{|\textit{main}|}|\\
\end{tabular}
\end{center}
%
Both forms have slightly different effects as described above.
The main file is prepared as usual, see \secref{sec:include}.

%%%%%%%%%%%%%%%%%%%%%%%%%%%%%%%%%%%%%%%%%%%%%%%%%%%%%%%%%%%%%%%%%%%%%%%%%%%%%%%%
\subsection{Legacy Detection}
\label{sec:detection}

The directive |\childdocmain| in the main file can detect
whether the complete document or merely a child is to be compiled
even without using the directive |\childdocof|.
This method is deprecated because it is less robust
and there is no compelling reason to use it;
it is merely provided for backward compatibility
and it may be removed in future versions.

If the detection mechanism is to be used,
it is mandatory to correctly specify
the filename of the main file as the argument of |\childdocmain|:
%
\begin{center}
\begin{tabular}{l}
|\input{childdoc.def}|\\
|\childdocmain{|\textit{main}|}|\\
\end{tabular}
\end{center}
%
If |\jobname| does not match the argument \textit{main} of |\childdocmain|,
it is assumed that |\jobname| points to the child file to be compiled.
When using |\childdocmain| with the main file specified as argument,
it suffices to start a child file
with just |\input{|\textit{main}|}|
without loading of the package and using |\childdocof|.
If instead all processing is done
with the appropriate \textsf{childdoc} directives,
the argument of \textit{main} of |\childdocmain| can be empty.

An alternative version of the command line processing described
in \secref{sec:commandline} using the detection mechanism reads:
%
\begin{center}
|... -jobname "|\textit{target}|" "|[\textit{flags}]%
[|\def\jobname{|\textit{dest}|}|]|\input{|\textit{main}|}"|
\end{center}

%%%%%%%%%%%%%%%%%%%%%%%%%%%%%%%%%%%%%%%%%%%%%%%%%%%%%%%%%%%%%%%%%%%%%%%%%%%%%%%%
\subsection{Manual Code}
\label{sec:manual}

In case one cannot be certain whether the definitions file |childdoc.def|
is installed on the target \TeX{} distribution
and one prefers not to ship it,
it is conceivable to paste a few relevant commands into the sources.

To that end, drop all statements |\input{childdoc.def}|
and perform the replacements as outlined below.
Instead of |\childdocmain{|\textit{main}|}| add the following code
to the top of the main file:
%
\begin{center}
\begin{tabular}{l}
|\||ifdefined\childdocname\endinput\||fi\newif\ifchilddoc|\\
|\edef\childdocname{\scantokens\expandafter{\jobname\noexpand}}|\\
|\def\childdocmain{|\textit{main}|}\||ifx\childdocmain\childdocname\||else|\\
|\childdoctrue\includeonly{\childdocname}\let\jobname\childdocmain\||fi|\\
\end{tabular}
\end{center}
%
Instead of |\childdocof{|\textit{main}|}| just include the main file
at the top of each child file:
%
\begin{center}
|\input{|\textit{main}|}|
\end{center}
%
A simple redirection |\childdocforward{|\textit{dest}|}| is achieved by:
%
\begin{center}
|\def\jobname{|\textit{dest}|}\input{\jobname}|
\end{center}
%
The redirection with prefix
|\childdocforwardprefix[|\textit{prefix}|]{|\textit{dest}|}|
is accomplished by:
%
\begin{center}
\begin{tabular}{l}
|{\edef\jobname{\scantokens\expandafter{\jobname\noexpand}}|\\
|\def\redirectjob |\textit{prefix}|#1~~~{\gdef\jobname{|\textit{dest}|#1}}|\\
|\expandafter\redirectjob\jobname~~~}\input{\jobname}|
\end{tabular}
\end{center}

In an alternative approach,
child documents can be compiled by a specific command line
without additional code or specific definitions:
%
\begin{center}
|... -jobname "|\textit{target}|" "|[\textit{flags}]%
|\includeonly{|\textit{dest}|}\input{|\textit{main}|}"|
\end{center}
%

%%%%%%%%%%%%%%%%%%%%%%%%%%%%%%%%%%%%%%%%%%%%%%%%%%%%%%%%%%%%%%%%%%%%%%%%%%%%%%%%
%%%%%%%%%%%%%%%%%%%%%%%%%%%%%%%%%%%%%%%%%%%%%%%%%%%%%%%%%%%%%%%%%%%%%%%%%%%%%%%%
\section{Information}

%%%%%%%%%%%%%%%%%%%%%%%%%%%%%%%%%%%%%%%%%%%%%%%%%%%%%%%%%%%%%%%%%%%%%%%%%%%%%%%%
\subsection{Copyright}

Copyright \copyright{} 2017--2018 Niklas Beisert

This work may be distributed and/or modified under the
conditions of the \LaTeX{} Project Public License, either version 1.3
of this license or (at your option) any later version.
The latest version of this license is in
  \url{http://www.latex-project.org/lppl.txt}
and version 1.3 or later is part of all distributions of \LaTeX{}
version 2005/12/01 or later.

This work has the LPPL maintenance status `maintained'.

The Current Maintainer of this work is Niklas Beisert.

This work consists of the files |README.txt|, |childdoc.ins| and |childdoc.dtx|
as well as the derived files |childdoc.def|, |cdocsamp.tex|
with |cdocsch1.tex|, |cdocsch2.tex|, |cdocspt3.tex|, |cdocspt4.tex|,
|cdocsdrf.tex|, |cdocsfn1.tex|, |cdocsfn2.tex|
as well as |childdoc.pdf|.

%%%%%%%%%%%%%%%%%%%%%%%%%%%%%%%%%%%%%%%%%%%%%%%%%%%%%%%%%%%%%%%%%%%%%%%%%%%%%%%%
\subsection{Files and Installation}

The package consists of the files:
%
\begin{center}
\begin{tabular}{ll}
    |README.txt|   & readme file \\
    |childdoc.ins| & installation file \\
    |childdoc.dtx| & source file \\
    |childdoc.def| & definition file \\
    |cdocsamp.tex| & sample main file \\
    |cdocsch1.tex| & sample include file \\
    |cdocsch2.tex| & sample include file \\
    |cdocspt3.tex| & sample part file \\
    |cdocspt4.tex| & sample part file \\
    |cdocsdrf.tex| & sample redirection file \\
    |cdocsfn1.tex| & sample redirection file \\
    |cdocsfn2.tex| & sample redirection file \\
    |childdoc.pdf| & manual
\end{tabular}
\end{center}
%
The distribution consists of the files
|README.txt|, |childdoc.ins| and |childdoc.dtx|.
%
\begin{itemize}
\item
Run (pdf)\LaTeX{} on |childdoc.dtx|
to compile the manual |childdoc.pdf| (this file).
\item
Run \LaTeX{} on |childdoc.ins| to create the definitions file |childdoc.def|
and the sample |cdocsamp.tex| with include files
|cdocsch1.tex|, |cdocsch2.tex|, |cdocspt3.tex|, |cdocspt4.tex|,
|cdocsdrf.tex|, |cdocsfn1.tex|, |cdocsfn2.tex|.
Then copy the file |childdoc.def| to an appropriate directory of your \LaTeX{}
distribution, e.g.\ \textit{texmf-root}|/tex/latex/childdoc|.
\end{itemize}

%%%%%%%%%%%%%%%%%%%%%%%%%%%%%%%%%%%%%%%%%%%%%%%%%%%%%%%%%%%%%%%%%%%%%%%%%%%%%%%%
\subsection{Related CTAN Packages}

There are several other packages which offer a similar functionality:
%
\begin{itemize}
\item
The packages
\href{http://ctan.org/pkg/docmute}{\textsf{docmute}},
\href{http://ctan.org/pkg/includex}{\textsf{includex}} and
\href{http://ctan.org/pkg/standalone}{\textsf{standalone}}
provide commands to include only the document body of
a child file thus allowing both files to be compiled individually.
\item
The packages \href{http://ctan.org/pkg/subdocs}{\textsf{subdocs}}
and \href{http://ctan.org/pkg/subfiles}{\textsf{subfiles}}
provide structures in which the main and child documents can be
encapsulated and allowing them to be compiled individually.
The inclusion mechanism is different from the conventional |\include|.
\item
The package \href{http://ctan.org/pkg/combine}{\textsf{combine}}
is an elaborate solution to combine several documents into one.
\end{itemize}
%
See also the CTAN topic \href{http://ctan.org/topic/subdocs}{\textsf{subdocs}}
for further related packages.
The present package differs from the above solutions in that
a document structure constructed with the conventional |\include| mechanism
just needs two extra commands at the top of every file
such that all constituent files can be compiled individually.

%%%%%%%%%%%%%%%%%%%%%%%%%%%%%%%%%%%%%%%%%%%%%%%%%%%%%%%%%%%%%%%%%%%%%%%%%%%%%%%%
%\subsection{Feature Suggestions}
%
%The following is a list of features which may be useful for future
%versions of this package:
%%
%\begin{itemize}
%\item
%\ldots
%\end{itemize}

%%%%%%%%%%%%%%%%%%%%%%%%%%%%%%%%%%%%%%%%%%%%%%%%%%%%%%%%%%%%%%%%%%%%%%%%%%%%%%%%
\subsection{Revision History}

%%%%%%%%%%%%%%%%%%%%%%%%%%%%%%%%%%%%%%%%
\paragraph{v2.0:} 2018/12/30

\begin{itemize}
\item
immediate forward processing
\item
added |\childdocby| mechanism
\item
manual restructured
\end{itemize}

%%%%%%%%%%%%%%%%%%%%%%%%%%%%%%%%%%%%%%%%
\paragraph{v1.6:} 2018/01/17

\begin{itemize}
\item
application for development of include files
\item
corrections to manual
\end{itemize}

%%%%%%%%%%%%%%%%%%%%%%%%%%%%%%%%%%%%%%%%
\paragraph{v1.5:} 2017/05/21

\begin{itemize}
\item
more complete structuring introduced
\item
|\childdocof| introduced
\item
|\childdoc| renamed to |\childdocmain|
\item
|\childredirect| renamed to |\childdocforward| and |\childdocforwardprefix|
and functionality expanded
\end{itemize}

%%%%%%%%%%%%%%%%%%%%%%%%%%%%%%%%%%%%%%%%
\paragraph{v1.0:} 2017/04/27

\begin{itemize}
\item
manual and install package
\item
first version published on CTAN
\end{itemize}

%%%%%%%%%%%%%%%%%%%%%%%%%%%%%%%%%%%%%%%%
\paragraph{v0.6:} 2017/04/26

\begin{itemize}
\item
redirection mechanism added
\end{itemize}

%%%%%%%%%%%%%%%%%%%%%%%%%%%%%%%%%%%%%%%%
\paragraph{v0.5:} 2017/04/26

\begin{itemize}
\item
functionality in definition file
\end{itemize}


%%%%%%%%%%%%%%%%%%%%%%%%%%%%%%%%%%%%%%%%%%%%%%%%%%%%%%%%%%%%%%%%%%%%%%%%%%%%%%%%
%%%%%%%%%%%%%%%%%%%%%%%%%%%%%%%%%%%%%%%%%%%%%%%%%%%%%%%%%%%%%%%%%%%%%%%%%%%%%%%%
%%%%%%%%%%%%%%%%%%%%%%%%%%%%%%%%%%%%%%%%%%%%%%%%%%%%%%%%%%%%%%%%%%%%%%%%%%%%%%%%
\appendix

\settowidth\MacroIndent{\rmfamily\scriptsize 000\ }

 \DocInput{childdoc.dtx}

\end{document}
%</driver>
% \fi
%
% %%%%%%%%%%%%%%%%%%%%%%%%%%%%%%%%%%%%%%%%%%%%%%%%%%%%%%%%%%%%%%%%%%%%%%%%%%%%%%
% %%%%%%%%%%%%%%%%%%%%%%%%%%%%%%%%%%%%%%%%%%%%%%%%%%%%%%%%%%%%%%%%%%%%%%%%%%%%%%
% \section{Sample}
%\iffalse
%<*samplemain>
%\fi
%
% The following presents a sample document
% with two chapters, two parts, a title page,
% a compile flag as well as three forwarding files to set the flag.
% It consists of eight |.tex| files:
% \begin{center}
% \begin{tabular}{ll}
% |cdocsamp.tex|&main file\\
% |cdocsch1.tex|&include file for chapter 1\\
% |cdocsch2.tex|&include file for chapter 2\\
% |cdocspt3.tex|&include file for part 3\\
% |cdocspt4.tex|&include file for part 4\\
% |cdocsdrf.tex|&forwarding file for main file in draft mode\\
% |cdocsfi1.tex|&forwarding file for final version of chapter 1\\
% |cdocsfi2.tex|&forwarding file for final version of chapter 2\\
% \end{tabular}
% \end{center}
% Each of the eight files can be compiled directly by the \LaTeX{} compiler.
%
% %%%%%%%%%%%%%%%%%%%%%%%%%%%%%%%%%%%%%%
% \paragraph{Main File.}
%
% The main file is called |cdocsamp.tex|.
%
% Load the \textsf{childdoc} definitions and
% declare the filename for the main document:
%    \begin{macrocode}
\input{childdoc.def}
\childdocmain{}
%    \end{macrocode}

% Optional override for |\version| flag:
%    \begin{macrocode}
%%\ifchilddoc\else\providecommand{\version}{draft}\fi
%    \end{macrocode}

% Define the default values for the |\version| flag
% (|final| for the main file and |draft| for childs):
%    \begin{macrocode}
\ifchilddoc
\providecommand{\version}{draft}
\else
\providecommand{\version}{final}
\fi
%    \end{macrocode}

% Load the standard document class:
%    \begin{macrocode}
\documentclass[12pt]{article}
%    \end{macrocode}

% Start the document body:
%    \begin{macrocode}
\begin{document}
%    \end{macrocode}

% Declare a title page.
% Print title, part of document being processed and version flag:
%    \begin{macrocode}
\addtocounter{page}{-1}
\begin{center}
{\LARGE\bfseries{}childdoc example\par}
\vspace{1cm}
\ifchilddoc
\ifchilddocmanual part\else chapter\fi:
`\childdocname' of `\childdocjob'\par
\else
main document: `\childdocjob'\par
\fi
version: \version\par
\end{center}
\newpage
%    \end{macrocode}

% Manually include selected file,
% otherwise process as usual:
%    \begin{macrocode}
\ifchilddocmanual
\section*{part `\childdocname'}
\input{\childdocname}
\else
%    \end{macrocode}

% Include the two chapters:
%    \begin{macrocode}
\include{cdocsch1}
\include{cdocsch2}
%    \end{macrocode}

% Include the two parts unless only chapters should be displayed:
%    \begin{macrocode}
\ifchilddoc\else
\section{part three}
\input{cdocspt3}
\section{part four}
\input{cdocspt4}
\fi
%    \end{macrocode}

% Process as usual until here:
%    \begin{macrocode}
\fi
%    \end{macrocode}

% End of document body:
%    \begin{macrocode}
\end{document}
%    \end{macrocode}
%\iffalse
%</samplemain>
%\fi
%
% %%%%%%%%%%%%%%%%%%%%%%%%%%%%%%%%%%%%%%
% \paragraph{Chapter Include Files.}
%
% The include files are called |cdocsch1.tex| and |cdocsch2.tex|.
%
%\iffalse
%<*samplechap1|samplechap2>
%\fi

% Optional override for |\version| flag:
%    \begin{macrocode}
%%\providecommand{\version}{final}
%    \end{macrocode}

% Include the main document:
%    \begin{macrocode}
\input{childdoc.def}
\childdocof{cdocsamp}
%    \end{macrocode}

%\iffalse
%</samplechap1|samplechap2>
%\fi
%
%\iffalse
%<*samplechap1>
%\fi
% Some text for chapter 1:
%    \begin{macrocode}
\section{one}
some text in chapter one
%    \end{macrocode}

%\iffalse
%</samplechap1>
%\fi
% Some text for chapter 2:
%\iffalse
%<*samplechap2>
%\fi
%    \begin{macrocode}
\section{two}
more text in chapter two
%    \end{macrocode}

%\iffalse
%</samplechap2>
%\fi
%
% %%%%%%%%%%%%%%%%%%%%%%%%%%%%%%%%%%%%%%
% \paragraph{Part Include Files.}
%
% The include files are called |cdocspt3.tex| and |cdocspt4.tex|.
%
%\iffalse
%<*samplepart3|samplepart4>
%\fi

% Optional override for |\version| flag:
%    \begin{macrocode}
%%\providecommand{\version}{final}
%    \end{macrocode}

% Include the main document:
%    \begin{macrocode}
\input{childdoc.def}
\childdocby{cdocsamp}
%    \end{macrocode}

%\iffalse
%</samplepart3|samplepart4>
%\fi
%
%\iffalse
%<*samplepart3>
%\fi
% Some text for part 3:
%    \begin{macrocode}
some text in part three
%    \end{macrocode}

%\iffalse
%</samplepart3>
%\fi
% Some text for part 4:
%\iffalse
%<*samplepart4>
%\fi
%    \begin{macrocode}
more text in part four
%    \end{macrocode}

%\iffalse
%</samplepart4>
%\fi
%
% %%%%%%%%%%%%%%%%%%%%%%%%%%%%%%%%%%%%%%
% \paragraph{Forwarding for a Complete Draft.}
%
% The following forwarding file |cdocsdrf.tex|
% compiles the main document in draft mode:
%\iffalse
%<*sampledraft>
%\fi
%    \begin{macrocode}
\def\version{draft}
\input{childdoc.def}
\childdocforward{cdocsamp}
%    \end{macrocode}

%\iffalse
%</sampledraft>
%\fi
%
% %%%%%%%%%%%%%%%%%%%%%%%%%%%%%%%%%%%%%%
% \paragraph{Forwarding for Final Version of the Chapters.}
%
% The following forwarding files |cdocsfn1.tex| and |cdocsfn2.tex|
% (with identical content)
% compile the final versions of the child documents
% |cdocsch1.tex| and |cdocsch2.tex|, respectively:
%\iffalse
%<*samplefinal>
%\fi
%    \begin{macrocode}
\def\version{final}
\input{childdoc.def}
\childdocforwardprefix[cdocsamp]{cdocsfn}{cdocsch}
%    \end{macrocode}

%\iffalse
%</samplefinal>
%\fi
%
% %%%%%%%%%%%%%%%%%%%%%%%%%%%%%%%%%%%%%%
% \paragraph{Command Line Processing.}
%
% The following three command lines generate the output files
% |cdocscld|, |cdocscl1| and |cdocscl2|
% which should be identical to
% |cdocsdrf|, |cdocsch1| and |cdocsfn2|, respectively:
% \begin{center}
% \begin{tabular}{l}
% |latex -jobname cdocscld \|\\
% |  "\def\version{draft}\input{childdoc.def}\childdocforward{cdocsamp}"|\\
% |latex -jobname cdocscl1 \|\\
% |  "\input{childdoc.def}\childdocforward[cdocsamp]{cdocsch1}"|\\
% |latex -jobname cdocscl2 \|\\
% |  "\def\version{final}\input{childdoc.def}\childdocforward{cdocsch2}"|
% \end{tabular}
% \end{center}
% Note that the trailing backslash on each first line
% merely continues the input to the second line
% (for convenient cut ant paste).
% Furthermore, the command |latex| can be replaced by any
% of its alternative versions such as |pdflatex|.
%
% %%%%%%%%%%%%%%%%%%%%%%%%%%%%%%%%%%%%%%%%%%%%%%%%%%%%%%%%%%%%%%%%%%%%%%%%%%%%%%
% %%%%%%%%%%%%%%%%%%%%%%%%%%%%%%%%%%%%%%%%%%%%%%%%%%%%%%%%%%%%%%%%%%%%%%%%%%%%%%
% \section{Implementation}
%\iffalse
%<*package>
%\fi
%
% This section describes the definitions file |childdoc.def|.

% The definitions cannot be loaded using |\usepackage| or |\RequirePackage|
% which has a mechanism to prevent loading a style file more than once.
% When loading the definitions by means of |\input|
% multiple instances have to be prevented manually:
%\iffalse
%This code needs to be before the `\ProvidesFile' directive
%which is defined at the beginning of this file.
%Therefore it is also placed there and commented out here.
%</package>
%<*discard>
%\fi
%    \begin{macrocode}
\ifdefined\childdocmain\endinput\fi
%    \end{macrocode}
%\iffalse
%</discard>
%<*package>
%\fi
%
% \macro{\ifchilddoc}
% \macro{\ifchilddocmanual}
% The conditional |\ifchilddoc| tells whether a
% child (true) or main (false) document is being compiled.
% The conditional |\ifchilddocmanual| tells whether
% the |\includeonly| mechanism is used (false) or
% the selection of child files must be performed manually (true).
% The definitions initialise to false:
%    \begin{macrocode}
\newif\ifchilddoc
\newif\ifchilddocmanual
%    \end{macrocode}

% \macro{\childdocname}
% \macro{\childdocjob}
% The macro |\childdocname| stores the name of the main document
% to be compiled. The macro |\childdocjob| stores the name of
% the document on which the \LaTeX{} compiler was originally invoked.
% The content of |\jobname| cannot be compared
% to filenames specified in the source due to different catcodes.
% The following code rescans |\jobname|, stores the result
% in |\childdocname| and saves a copy in |\childdocjob|:
%    \begin{macrocode}
\edef\childdocname{\scantokens\expandafter{\jobname\noexpand}}
\let\childdocjob\childdocname
%    \end{macrocode}

% \macro{\childdocdisable}
% The macro |\childdocdisable| prevents the main file
% from being processed more than once.
% At this stage, the main document command |\childdocmain|
% is assumed to be called once again where it should do nothing.
% Any subsequent call to it should prevent
% a secondary processing of the main document
% It overwrites the forwarding commands
% |\childdocof| and |\childdocforward|
% with empty macros to prevent further inclusions of the main document:
%    \begin{macrocode}
\newcommand{\childdocdisable}
{
  \renewcommand{\childdocmain}[1]{\renewcommand{\childdocmain}[1]{\endinput}}
  \renewcommand{\childdocof}[1]{}
  \renewcommand{\childdocby}[2][]{}
  \renewcommand{\childdocforward}[2][]{}
  \renewcommand{\childdocdisable}{}
}
%    \end{macrocode}

% \macro{\childdocmain}
% The macro |\childdocmain| is to be called at the top of the main file
% with nothing or the main filename (without extension) as argument.
% First, it breaks loops.
% If the argument is not empty and does not match |\childdocname|
% (which is set by the first inclusion of |childdoc.def|),
% |\ifchilddoc| is set to true, |\includeonly| is applied to the child file
% and |\jobname| is set to the main file
% (for proper handling of |.aux| files):
%    \begin{macrocode}
\newcommand{\childdocmain}[1]
{
  \childdocdisable\childdocmain{}
  \if?#1?\else
    \begingroup
      \def\childdoctmp{#1}
      \ifx\childdoctmp\childdocname
        \def\childdoctmp{}
      \else
        \def\childdoctmp
        {
          \childdoctrue
          \includeonly{\childdocname}
          \def\childdocjob{#1}
          \def\jobname{#1}
        }
      \fi
      \expandafter
    \endgroup
    \childdoctmp
  \fi
}
%    \end{macrocode}

% \macro{\childdocof}
% The command |\childdocof| redirects
% compilation to the main file |#1|.
%    \begin{macrocode}
\newcommand{\childdocof}[1]
{
  \childdocdisable
  \childdoctrue
  \includeonly{\childdocname}
  \def\jobname{#1}
  \def\childdocjob{#1}
  \input{#1}
}
%    \end{macrocode}

% \macro{\childdocby}
% The command |\childdocby| ....
%    \begin{macrocode}
\newcommand{\childdocby}[2][]
{
  \childdocdisable
  \childdoctrue
  \childdocmanualtrue
  \if?#1?\else
    \def\jobname{#2}
  \fi
  \def\childdocjob{#2}
  \input{#2}
  \endinput
}
%    \end{macrocode}

% \macro{\childdocforward}
% The command |\childdocforward| redirects
% compilation to the main file or
% (if the optional argument is given) a child file.
% Parameters are set as if the main file
% or a child file starting with |\childdocof| was compiled.
% Then compilation is handed over to the main file:
%    \begin{macrocode}
\newcommand{\childdocforward}[2][]
{
  \begingroup
    \if?#1?
      \def\childdoctmp
      {
        \def\childdocname{#2}
        \def\childdocjob{#2}
        \def\jobname{#2}
        \input{#2}
        \endinput
      }
    \else
      \def\childdoctmp
      {
        \childdocdisable
        \def\childdocname{#2}
        \childdoctrue
        \includeonly{#2}
        \def\childdocjob{#1}
        \def\jobname{#1}
        \input{#1}
        \endinput
      }
    \fi
    \expandafter
  \endgroup
  \childdoctmp
}
%    \end{macrocode}

% \macro{\childdocforwardprefix}
% The command |\childdocforwardprefix| redirects
% compilation to the main or a child file by means of a pattern.
% The prefix |#1| in the current filename is replaced by |#2|
% and the suffix of the current filename is kept
% (it is assumed that the filename does not contain the substring `|~~~|'
% which is used as a delimiter).
% Compilation is handed over to the new file by |\childdocforward|:
%    \begin{macrocode}
\newcommand{\childdocforwardprefix}[3][]
{
  \begingroup
    \def\childdocextract #2##1~~~{\def\childdoctmp{\childdocforward[#1]{#3##1}}}
    \expandafter\childdocextract\childdocname~~~
    \expandafter
  \endgroup
  \childdoctmp
}
%    \end{macrocode}

% \macro{\childdoc}
% The deprecated macro |\childdoc| is a legacy version of |\childdocmain|:
%    \begin{macrocode}
\newcommand{\childdoc}{\childdocmain}
%    \end{macrocode}

% \macro{\childdocredirect}
% The deprecated macro |\childdocredirect| is a legacy version
% of |\childdocforward| and |\childdocforwardprefix|:
%    \begin{macrocode}
\newcommand{\childdocredirect}[2][]
{
  \begingroup
    \if?#1?
      \def\childdoctmp{\childdocforward{#2}}
    \else
      \def\childdoctmp{\childdocforwardprefix{#1}{#2}}
    \fi
    \expandafter
  \endgroup
  \childdoctmp
}
%    \end{macrocode}

%\iffalse
%</package>
%\fi
%
\endinput
\childdocforward[|\textit{main}|]{|\textit{dest}|}"|
\end{center}
%
Here \textit{target} is the name of the output file,
\textit{main} is the name of the main file
and \textit{dest} is the name of the main or child file to be processed
(all filenames without extensions).
The optional argument \textit{main} can be omitted
if \textit{main} matches \textit{dest}.
Optionally, compilation \textit{flags} can be defined via |\def| commands.
This command line makes the \TeX{} engine believe
it is compiling the file \textit{target}
whose content is specified as the latter parameter.
The provided code then forwards the processing to
\textit{main} or \textit{dest} as described in \secref{sec:forward}.

%%%%%%%%%%%%%%%%%%%%%%%%%%%%%%%%%%%%%%%%%%%%%%%%%%%%%%%%%%%%%%%%%%%%%%%%%%%%%%%%
\subsection{Include by Input}
\label{sec:input}

Including child documents by |\include| has some restrictions by design.
Most notably, the content of a child document always occupies
its own set of pages; pages cannot be shared between child documents.
Usually, this behaviour makes perfect sense
because each child document contain an essential part of the document.
However, in some situations it may be desirable to compose
a document from a collection of parts
without having mandatory page breaks between then.
For this case, the package
provides a mechanism to include parts
by |\input| which can also be processed individually.
However, by construction this mechanism
requires manual handling of the content to be output.

%%%%%%%%%%%%%%%%%%%%%%%%%%%%%%%%%%%%%%%%
\DescribeMacro{\ifchilddocmanual}
The main file should be prepared as usual, see \secref{sec:include}.
However, the document body must make a distinction
between processing of an individual part and of the main document, e.g.:
%
\begin{center}
\begin{tabular}{l}
|\ifchilddocmanual|\\
|\input{\childdocname}|\\
|\||else|\\
\textit{document body with }|\input{|\textit{part}|}|\\
|\||fi|
\end{tabular}
\end{center}
%
The conditional |\ifchilddocmanual| is true whenever
a part to be included by |\input| is being compiled,
and the name of the part is stored in |\childdocname|.

%%%%%%%%%%%%%%%%%%%%%%%%%%%%%%%%%%%%%%%%
\DescribeMacro{\childdocby}
Each part to be included by |\input| should start with:
%
\begin{center}
\begin{tabular}{l}
|% \iffalse
%
% childdoc.dtx Copyright (C) 2017-2018 Niklas Beisert
%
% This work may be distributed and/or modified under the
% conditions of the LaTeX Project Public License, either version 1.3
% of this license or (at your option) any later version.
% The latest version of this license is in
%   http://www.latex-project.org/lppl.txt
% and version 1.3 or later is part of all distributions of LaTeX
% version 2005/12/01 or later.
%
% This work has the LPPL maintenance status `maintained'.
%
% The Current Maintainer of this work is Niklas Beisert.
%
% This work consists of the files childdoc.dtx and childdoc.ins
% and the derived files childdoc.def and cdocsamp.tex with
% cdocsch1.tex, cdocsch2.tex, cdocsdrf.tex, cdocsfn1.tex, cdocsfn2.tex.
%
%<package>\ifdefined\childdocmain\endinput\fi
%<package>\ProvidesFile{childdoc.def}[2018/12/30 v2.0 child document driver]
%<samplemain>\ProvidesFile{cdocsamp.tex}[2018/12/30 v2.0 sample for childdoc]
%<*driver>
%\ProvidesFile{childdoc.drv}[2018/12/30 v2.0 childdoc reference manual file]
\PassOptionsToClass{10pt,a4paper}{article}
\documentclass{ltxdoc}

\usepackage[margin=35mm]{geometry}
\usepackage{hyperref}
\usepackage{hyperxmp}
\usepackage[usenames]{color}

\hypersetup{colorlinks=true}
\hypersetup{pdfstartview=FitH}
\hypersetup{pdfpagemode=UseNone}
\hypersetup{pdfsource={}}
\hypersetup{pdflang={en-UK}}
\hypersetup{pdfcopyright={Copyright 2017-2018 Niklas Beisert.
  This work may be distributed and/or modified under the
  conditions of the LaTeX Project Public License, either version 1.3
  of this license or (at your option) any later version.}}
\hypersetup{pdflicenseurl={http://www.latex-project.org/lppl.txt}}
\hypersetup{pdfcontactaddress={ETH Zurich, ITP, HIT K,
  Wolfgang-Pauli-Strasse 27}}
\hypersetup{pdfcontactpostcode={8093}}
\hypersetup{pdfcontactcity={Zurich}}
\hypersetup{pdfcontactcountry={Switzerland}}
\hypersetup{pdfcontactemail={nbeisert@itp.phys.ethz.ch}}
\hypersetup{pdfcontacturl={http://people.phys.ethz.ch/\xmptilde nbeisert/}}

\newcommand{\secref}[1]{\hyperref[#1]{section \ref*{#1}}}

\parskip1ex
\parindent0pt
\let\olditemize\itemize
\def\itemize{\olditemize\parskip0pt}

\begin{document}

\title{The \textsf{childdoc} Package}
\hypersetup{pdftitle={The childdoc Package}}
\author{Niklas Beisert\\[2ex]
  Institut f\"ur Theoretische Physik\\
  Eidgen\"ossische Technische Hochschule Z\"urich\\
  Wolfgang-Pauli-Strasse 27, 8093 Z\"urich, Switzerland\\[1ex]
  \href{mailto:nbeisert@itp.phys.ethz.ch}
  {\texttt{nbeisert@itp.phys.ethz.ch}}}
\hypersetup{pdfauthor={Niklas Beisert}}
\hypersetup{pdfsubject={Manual for the LaTeX2e Package childdoc}}
\date{30 December 2018, \textsf{v2.0}}
\maketitle

\begin{abstract}\noindent
\textsf{childdoc} is a \LaTeXe{} package
that enables the direct compilation
of document sections included by |\include|
to individual files.
\end{abstract}

\begingroup
\parskip0ex
\tableofcontents
\endgroup

%%%%%%%%%%%%%%%%%%%%%%%%%%%%%%%%%%%%%%%%%%%%%%%%%%%%%%%%%%%%%%%%%%%%%%%%%%%%%%%%
%%%%%%%%%%%%%%%%%%%%%%%%%%%%%%%%%%%%%%%%%%%%%%%%%%%%%%%%%%%%%%%%%%%%%%%%%%%%%%%%
\section{Introduction}

\LaTeX{} provides a mechanism to structure a large document (such as a book)
into a main file and several child files (containing the chapters)
using the |\include| command.
This mechanism is beneficial for documents
which span hundreds of pages in order to
make the source file(s) more manageable.
Moreover, compilation can be restricted to
selected child files by means of the |\includeonly| command.
The latter feature can be used to reduce the compilation time while editing
(this was significantly more useful in the earlier days of \LaTeX{})
or to generate a smaller document which is easier to navigate.
Another application of |\includeonly| is to generate
documents consisting of selected parts of the complete document.

However, there are a few drawbacks of the plain |\include| mechanism:
\begin{itemize}
\item
The child files cannot be compiled on their own,
they can only be compiled via the main file.
A naive editing environment
(such as a text editor with an option
to have the current file processed by \LaTeX)
may require one to switch to the main file before compiling;
attempting to compile the child file produces errors.
\item
The main file must be modified (each time)
to adjust the |\includeonly| command
to the present needs. This easily leaves the main file in a messy state.
\item
The generated document will always carry the filename
of the main document. This is inconvenient if
several child files are to be compiled and
to be kept for distribution.
\end{itemize}

The present package provides a simple interface
to make child files individually compilable by \LaTeX{}.
Compiling a child file then has the same effect as compiling
the main file with an |\includeonly| command
to select the appropriate child.
Moreover the generated document will carry the name of the child
rather than the main file.
This resolves all three above issues.

This feature is meant to make the editing of books,
thesis documents and lecture notes somewhat more convenient.
However, the package can also be used efficiently for
composing a series of documents (such as exercise sheets)
which are typically distributed individually.
It then assists the author in generating the individual documents
(potentially in different versions)
as well as a document containing the collected series.
Another application is in developing style files
or other kinds of included material
where compilation of the style file could redirect
to a sample or test file.

%%%%%%%%%%%%%%%%%%%%%%%%%%%%%%%%%%%%%%%%%%%%%%%%%%%%%%%%%%%%%%%%%%%%%%%%%%%%%%%%
%%%%%%%%%%%%%%%%%%%%%%%%%%%%%%%%%%%%%%%%%%%%%%%%%%%%%%%%%%%%%%%%%%%%%%%%%%%%%%%%
\section{Usage}

First of all, the package \textsf{childdoc} is \emph{not} a standard
\LaTeXe{} |.sty| style file! Therefore it needs to be invoked in
a non-standard way.

%%%%%%%%%%%%%%%%%%%%%%%%%%%%%%%%%%%%%%%%%%%%%%%%%%%%%%%%%%%%%%%%%%%%%%%%%%%%%%%%
\subsection{Included Files}
\label{sec:include}

%%%%%%%%%%%%%%%%%%%%%%%%%%%%%%%%%%%%%%%%
\DescribeMacro{\childdocmain}
To use the package, add the commands
\begin{center}
\begin{tabular}{l}
|\input{childdoc.def}|\\
|\childdocmain{}|\\
\end{tabular}
\end{center}
at the very top of the main \LaTeX{} file,
in particular \emph{before} the |\documentclass| statement!
The argument of |\childdocmain| should be left empty
(but it must be present).

%%%%%%%%%%%%%%%%%%%%%%%%%%%%%%%%%%%%%%%%
\DescribeMacro{\childdocof}
Furthermore, add the commands
\begin{center}
\begin{tabular}{l}
|\input{childdoc.def}|\\
|\childdocof{|\textit{main}|}|\\
\end{tabular}
\end{center}
at the top of every child file \textit{child}
which is included by |\include{|\textit{child}|}|
from within the main file
(or at least for those files to be compiled individually).
The argument \textit{main} must be the filename of the main file.

There are a couple of
considerations in setting up the main and child documents:

%%%%%%%%%%%%%%%%%%%%%%%%%%%%%%%%%%%%%%%%
\paragraph{Restrictions.}

Please note the following restrictions:
\begin{itemize}
\item
|\childdocmain| must be called with one argument \textit{main}
to ensure compatibility with earlier version of the package.
It must either be empty (|\childdocmain{}|)
or precisely match the filename of the main file in which it is specified.
See \secref{sec:detection} for further information.
\item
The filename \textit{main} must be specified without the |.tex| extension.
\item
The filename \textit{main} is case sensitive
(even in case-insensitive file systems)
due to internal string comparison.
\item
The argument \textit{main} should be fully expanded, it cannot be a macro.
\item
Subdirectories and special characters should be avoided in filenames.
\item
The command |\childdocmain{|\textit{main}|}| must be followed by a whitespace.
It should not be followed immediately by another command
or by a comment mark `|%|'.
This is because the \TeX{} parser reads the token immediately following
the argument of |\childdocmain| and puts it
at the beginning of every child section;
however, a white\-space is ignored.
\end{itemize}

%%%%%%%%%%%%%%%%%%%%%%%%%%%%%%%%%%%%%%%%
\paragraph{Content of Main File.}

It is advisable to place all content in the child files included by |\include|.
Any output contained in the main file will appear in all child documents
unless suppressed manually;
it cannot be suppressed automatically by the |\includeonly| directive
and thus should normally be avoided.
A method to include some content in the main file
by means of conditional processing is described in \secref{sec:conditional}.

%%%%%%%%%%%%%%%%%%%%%%%%%%%%%%%%%%%%%%%%
\paragraph{Page Numbering.}

When only a part of the document is compiled,
the appropriate numbering of pages
(as well as other status parameters)
is determined from the |.aux| files.
The latter contain information from previous passes.
However this information needs to propagate through
all intermediate child documents.
Therefore the page numbering in child documents may well
be inconsistent until the complete document is compiled at least once.

A useful (if unconventional) way to always ensure a consistent
page numbering is to restart the numbering in each child document
and denote the pages by `\textit{child}|.|\textit{page}'
where \textit{child} represents the chapter/section number of the child file.
This can be achieved by the command
|\numberwithin{page}{|\textit{child}|}|
of the \textsf{amsmath} package
where \textit{child} can be |chapter| or |section|
depending on the chosen structuring.
Alternatively, one can modify the macro |\thepage| appropriately
and reset the counter |page| at the start of each child file.

%%%%%%%%%%%%%%%%%%%%%%%%%%%%%%%%%%%%%%%%%%%%%%%%%%%%%%%%%%%%%%%%%%%%%%%%%%%%%%%%
\subsection{Conditional Processing}
\label{sec:conditional}

The package provides a mechanism to compile different versions
of a document. To customise the versions further some conditional processing
can come in handy to distinguish which version is being compiled.
The package provides two macros to describe the compilation context:

%%%%%%%%%%%%%%%%%%%%%%%%%%%%%%%%%%%%%%%%
\DescribeMacro{\ifchilddoc}
The conditional |\ifchilddoc| distinguishes between the compilation of
child documents and the main document:
%
\begin{center}
|\ifchilddoc |\textit{child-code}| |[|\||else |\textit{main-code}]| \||fi|
\end{center}

%%%%%%%%%%%%%%%%%%%%%%%%%%%%%%%%%%%%%%%%
\DescribeMacro{\childdocname}
\DescribeMacro{\childdocjob}
The macro |\childdocname| contains the filename (without extension)
of the main or child file being processed.
Note that |\childdocjob| will always contain the name of the main file.

%%%%%%%%%%%%%%%%%%%%%%%%%%%%%%%%%%%%%%%%
\paragraph{Title Page.}

Conditional processing can be used to include a title or banner page
in the main document when proper precautions are taken.
Importantly, the code in the main file should ensure that the page counter
(as well as other status parameters which are stored in the |.aux| files)
takes the same value after the conditional processing.
Otherwise the page numbers may take divergent values
depending on which part is compiled.

For example, a title page could be declared by:
%
\begin{center}
\begin{tabular}{l}
|\ifchilddoc\||else|\\
|\addtocounter{page}{-1}|\\
\textit{code for title page}\\
|\newpage|\\
|\||fi|
\end{tabular}
\end{center}
%
A banner page for the child documents can be generated by:
%
\begin{center}
\begin{tabular}{l}
|\ifchilddoc|\\
|\addtocounter{page}{-1}|\\
\textit{code for banner page}\\
|\newpage|\\
|\||fi|
\end{tabular}
\end{center}
%
Here one could write a message such as:
\begin{center}
|This is the part \childdocname{} of \childdocjob{}.|
\end{center}

%%%%%%%%%%%%%%%%%%%%%%%%%%%%%%%%%%%%%%%%%%%%%%%%%%%%%%%%%%%%%%%%%%%%%%%%%%%%%%%%
\subsection{Flags}
\label{sec:flags}

The package makes it easy to generate different versions
of the main or child documents.
To this end compilation flags can be defined
and assigned different default values.
They will be particularly useful in conjunction
with the forwarding mechanism described in \secref{sec:forward}.

For example, it may be useful to have a flag |\version|
which can be set to |draft| or |final|.
The document source will contain some conditional code
depending on the value of |\version|.
Suppose further, the flag should default to |final| for the main file
and to |draft| for child files
which is a natural assignment for editing the document.
This is achieved by placing the following code
in the preamble of the main document
(below the |\childdocmain| directive):
%
\begin{center}
\begin{tabular}{l}
|\ifchilddoc|\\
|\providecommand{\version}{draft}|\\
|\||else|\\
|\providecommand{\version}{final}|\\
|\||fi|
\end{tabular}
\end{center}
%
The definition by |\providecommand| makes sure
that previous definitions are not overwritten.
Further statements |\providecommand{\version}{...}|
can thus be added before the above code to override it.

For the main file, one might add a line
(between |\childdocmain| and the above block)
%
\begin{center}
|%\ifchilddoc\||else\providecommand{\version}{draft}\||fi|
\end{center}
%
which can be uncommented to produce a draft version.
Likewise one can add a line to the very top of a child file
(above the |\childdocof{|\textit{main}|}| directive)
%
\begin{center}
|%\providecommand{\version}{final}|
\end{center}
%
which can be uncommented to produce the final version of this child document.

%%%%%%%%%%%%%%%%%%%%%%%%%%%%%%%%%%%%%%%%%%%%%%%%%%%%%%%%%%%%%%%%%%%%%%%%%%%%%%%%
\subsection{Forwarding}
\label{sec:forward}

Different versions of the main or child documents
using compilation flags as described in \secref{sec:flags}
can be (permanently) stored in different files
for convenient compilation, viewing and distribution.
To this end, the package defines a command
to pass on compilation to a different file:

%%%%%%%%%%%%%%%%%%%%%%%%%%%%%%%%%%%%%%%%
\DescribeMacro{\childdocforward}
The command |\childdocforward| redirects processing to
another source file:
%
\begin{center}
\begin{tabular}{l}
|\input{childdoc.def}|\\
|\childdocforward[|\textit{main}|]{|\textit{dest}|}|\\
\end{tabular}
\end{center}
%
The argument \textit{dest} is the destination file
(without extension).
It should be the main file or one of the child files.
Note that further \textsf{childdoc} directives
such as |\childdocof| and |\childdocforward|
in the indicated file will be processed in this form.
The optional argument \textit{main}
passes on directly to the main file \textit{main}
while pretending to compile the child \textit{dest}.
This form behaves as if \textit{dest}
issues |\childdocof{|\textit{main}|}| right away,
and no further \textsf{childdoc} directives will be processed.

%%%%%%%%%%%%%%%%%%%%%%%%%%%%%%%%%%%%%%%%
\DescribeMacro{\...prefix}
In the alternative form |\childdocforwardprefix|,
%
\begin{center}
\begin{tabular}{l}
|\input{childdoc.def}|\\
|\childdocforwardprefix[|\textit{main}|]{|\textit{prefix}|}{|\textit{dest}|}|
\end{tabular}
\end{center}
%
the destination file is determined by a pattern
depending on the current file:
To make this work, the current file must be called
`{\textit{prefix}\hspace{0.2em}\textit{suffix}}'
with \textit{prefix} matching precisely the argument.
Processing is then passed on to the file
`{\textit{dest}\hspace{0.2em}\textit{suffix}}'.
Surely, the same effect is achieved by
directly specifying the
argument `{\textit{dest}\hspace{0.2em}\textit{suffix}}'
in the first form.
However, that requires to set up a different file
for each child. With the alternative form of the command
all these files can have exactly the same content
which simplifies setting them up and maintaining them.

For example, the following file |draft.tex|
with a compilation flag |\version| as described in \secref{sec:flags}
compiles the main document as a draft:
%
\begin{center}
\begin{tabular}{l}
|\def\version{draft}|\\
|\input{childdoc.def}|\\
|\childdocforward{|\textit{main}|}|
\end{tabular}
\end{center}
%
Likewise, the following files |final|\textit{nn}|.tex|
compile the final version of the child document
|child|\textit{nn}|.tex|:
%
\begin{center}
\begin{tabular}{l}
|\def\version{final}|\\
|\input{childdoc.def}|\\
|\childdocforwardprefix{final}{child}|
\end{tabular}
\end{center}
%

Note that when several versions of a main file and/or of each child file
are to be generated, it may be convenient to set up a |Makefile| or
shell script to automatise the process.

%%%%%%%%%%%%%%%%%%%%%%%%%%%%%%%%%%%%%%%%%%%%%%%%%%%%%%%%%%%%%%%%%%%%%%%%%%%%%%%%
\subsection{Command Line Processing}
\label{sec:commandline}

The effect of redirection files can also be achieved by invoking
the \LaTeX{} compiler with a more elaborate command line.
Most conveniently this should be done as part
of a shell script or a |Makefile|.

When using \textsf{childdoc} in the main file, the following
command lines effectively perform a redirection
(note that depending on the shell being used,
backslashes may have to be doubled: `|\|' $\to$ `|\\|'):
%
\begin{center}
|... -jobname "|\textit{target}|" |\\|"|[\textit{flags}]%
|\input{childdoc.def}\childdocforward[|\textit{main}|]{|\textit{dest}|}"|
\end{center}
%
Here \textit{target} is the name of the output file,
\textit{main} is the name of the main file
and \textit{dest} is the name of the main or child file to be processed
(all filenames without extensions).
The optional argument \textit{main} can be omitted
if \textit{main} matches \textit{dest}.
Optionally, compilation \textit{flags} can be defined via |\def| commands.
This command line makes the \TeX{} engine believe
it is compiling the file \textit{target}
whose content is specified as the latter parameter.
The provided code then forwards the processing to
\textit{main} or \textit{dest} as described in \secref{sec:forward}.

%%%%%%%%%%%%%%%%%%%%%%%%%%%%%%%%%%%%%%%%%%%%%%%%%%%%%%%%%%%%%%%%%%%%%%%%%%%%%%%%
\subsection{Include by Input}
\label{sec:input}

Including child documents by |\include| has some restrictions by design.
Most notably, the content of a child document always occupies
its own set of pages; pages cannot be shared between child documents.
Usually, this behaviour makes perfect sense
because each child document contain an essential part of the document.
However, in some situations it may be desirable to compose
a document from a collection of parts
without having mandatory page breaks between then.
For this case, the package
provides a mechanism to include parts
by |\input| which can also be processed individually.
However, by construction this mechanism
requires manual handling of the content to be output.

%%%%%%%%%%%%%%%%%%%%%%%%%%%%%%%%%%%%%%%%
\DescribeMacro{\ifchilddocmanual}
The main file should be prepared as usual, see \secref{sec:include}.
However, the document body must make a distinction
between processing of an individual part and of the main document, e.g.:
%
\begin{center}
\begin{tabular}{l}
|\ifchilddocmanual|\\
|\input{\childdocname}|\\
|\||else|\\
\textit{document body with }|\input{|\textit{part}|}|\\
|\||fi|
\end{tabular}
\end{center}
%
The conditional |\ifchilddocmanual| is true whenever
a part to be included by |\input| is being compiled,
and the name of the part is stored in |\childdocname|.

%%%%%%%%%%%%%%%%%%%%%%%%%%%%%%%%%%%%%%%%
\DescribeMacro{\childdocby}
Each part to be included by |\input| should start with:
%
\begin{center}
\begin{tabular}{l}
|\input{childdoc.def}|\\
|\childdocby{|\textit{main}|}|\\
\end{tabular}
\end{center}
%
The directive |\childdocby| is similar to |\childdocof|
described in \secref{sec:include},
but the subsequent selection of content must be done manually.
To that end, both |\ifchilddoc| and |\ifchilddocmanual|
will be true upon processing of a part,
and the name of the part is stored in |\childdocname|.
Note that |\jobname| will be set to the filename of the current part
so that each part receives an individual |.aux| file
that does not interfere with the |.aux| file(s) of the main document.
This behaviour can be altered by the alternative form
|\childdocby[*]{|\textit{main}|}| (with a non-empty optional argument)
which uses the |.aux| file of the main document
by setting |\jobname| to \textit{main}.

%%%%%%%%%%%%%%%%%%%%%%%%%%%%%%%%%%%%%%%%%%%%%%%%%%%%%%%%%%%%%%%%%%%%%%%%%%%%%%%%
\subsection{Driver Development}
\label{sec:driver}

The \textsf{childdoc} mechanism can also be use for the development
of definition files such as \LaTeX{} styles or classes.
This case differs from the above setup with multiple parts
included by |\include| in that no |\includeonly| should be invoked.
This can be achieved by starting the include file
(before |\ProvidesPackage|) with:
%
\begin{center}
\begin{tabular}{l}
|\input{childdoc.def}|\\
|\childdocforward{|\textit{main}|}|\\
\end{tabular}
\end{center}
%
or alternatively with:
%
\begin{center}
\begin{tabular}{l}
|\input{childdoc.def}|\\
|\childdocby{|\textit{main}|}|\\
\end{tabular}
\end{center}
%
Both forms have slightly different effects as described above.
The main file is prepared as usual, see \secref{sec:include}.

%%%%%%%%%%%%%%%%%%%%%%%%%%%%%%%%%%%%%%%%%%%%%%%%%%%%%%%%%%%%%%%%%%%%%%%%%%%%%%%%
\subsection{Legacy Detection}
\label{sec:detection}

The directive |\childdocmain| in the main file can detect
whether the complete document or merely a child is to be compiled
even without using the directive |\childdocof|.
This method is deprecated because it is less robust
and there is no compelling reason to use it;
it is merely provided for backward compatibility
and it may be removed in future versions.

If the detection mechanism is to be used,
it is mandatory to correctly specify
the filename of the main file as the argument of |\childdocmain|:
%
\begin{center}
\begin{tabular}{l}
|\input{childdoc.def}|\\
|\childdocmain{|\textit{main}|}|\\
\end{tabular}
\end{center}
%
If |\jobname| does not match the argument \textit{main} of |\childdocmain|,
it is assumed that |\jobname| points to the child file to be compiled.
When using |\childdocmain| with the main file specified as argument,
it suffices to start a child file
with just |\input{|\textit{main}|}|
without loading of the package and using |\childdocof|.
If instead all processing is done
with the appropriate \textsf{childdoc} directives,
the argument of \textit{main} of |\childdocmain| can be empty.

An alternative version of the command line processing described
in \secref{sec:commandline} using the detection mechanism reads:
%
\begin{center}
|... -jobname "|\textit{target}|" "|[\textit{flags}]%
[|\def\jobname{|\textit{dest}|}|]|\input{|\textit{main}|}"|
\end{center}

%%%%%%%%%%%%%%%%%%%%%%%%%%%%%%%%%%%%%%%%%%%%%%%%%%%%%%%%%%%%%%%%%%%%%%%%%%%%%%%%
\subsection{Manual Code}
\label{sec:manual}

In case one cannot be certain whether the definitions file |childdoc.def|
is installed on the target \TeX{} distribution
and one prefers not to ship it,
it is conceivable to paste a few relevant commands into the sources.

To that end, drop all statements |\input{childdoc.def}|
and perform the replacements as outlined below.
Instead of |\childdocmain{|\textit{main}|}| add the following code
to the top of the main file:
%
\begin{center}
\begin{tabular}{l}
|\||ifdefined\childdocname\endinput\||fi\newif\ifchilddoc|\\
|\edef\childdocname{\scantokens\expandafter{\jobname\noexpand}}|\\
|\def\childdocmain{|\textit{main}|}\||ifx\childdocmain\childdocname\||else|\\
|\childdoctrue\includeonly{\childdocname}\let\jobname\childdocmain\||fi|\\
\end{tabular}
\end{center}
%
Instead of |\childdocof{|\textit{main}|}| just include the main file
at the top of each child file:
%
\begin{center}
|\input{|\textit{main}|}|
\end{center}
%
A simple redirection |\childdocforward{|\textit{dest}|}| is achieved by:
%
\begin{center}
|\def\jobname{|\textit{dest}|}\input{\jobname}|
\end{center}
%
The redirection with prefix
|\childdocforwardprefix[|\textit{prefix}|]{|\textit{dest}|}|
is accomplished by:
%
\begin{center}
\begin{tabular}{l}
|{\edef\jobname{\scantokens\expandafter{\jobname\noexpand}}|\\
|\def\redirectjob |\textit{prefix}|#1~~~{\gdef\jobname{|\textit{dest}|#1}}|\\
|\expandafter\redirectjob\jobname~~~}\input{\jobname}|
\end{tabular}
\end{center}

In an alternative approach,
child documents can be compiled by a specific command line
without additional code or specific definitions:
%
\begin{center}
|... -jobname "|\textit{target}|" "|[\textit{flags}]%
|\includeonly{|\textit{dest}|}\input{|\textit{main}|}"|
\end{center}
%

%%%%%%%%%%%%%%%%%%%%%%%%%%%%%%%%%%%%%%%%%%%%%%%%%%%%%%%%%%%%%%%%%%%%%%%%%%%%%%%%
%%%%%%%%%%%%%%%%%%%%%%%%%%%%%%%%%%%%%%%%%%%%%%%%%%%%%%%%%%%%%%%%%%%%%%%%%%%%%%%%
\section{Information}

%%%%%%%%%%%%%%%%%%%%%%%%%%%%%%%%%%%%%%%%%%%%%%%%%%%%%%%%%%%%%%%%%%%%%%%%%%%%%%%%
\subsection{Copyright}

Copyright \copyright{} 2017--2018 Niklas Beisert

This work may be distributed and/or modified under the
conditions of the \LaTeX{} Project Public License, either version 1.3
of this license or (at your option) any later version.
The latest version of this license is in
  \url{http://www.latex-project.org/lppl.txt}
and version 1.3 or later is part of all distributions of \LaTeX{}
version 2005/12/01 or later.

This work has the LPPL maintenance status `maintained'.

The Current Maintainer of this work is Niklas Beisert.

This work consists of the files |README.txt|, |childdoc.ins| and |childdoc.dtx|
as well as the derived files |childdoc.def|, |cdocsamp.tex|
with |cdocsch1.tex|, |cdocsch2.tex|, |cdocspt3.tex|, |cdocspt4.tex|,
|cdocsdrf.tex|, |cdocsfn1.tex|, |cdocsfn2.tex|
as well as |childdoc.pdf|.

%%%%%%%%%%%%%%%%%%%%%%%%%%%%%%%%%%%%%%%%%%%%%%%%%%%%%%%%%%%%%%%%%%%%%%%%%%%%%%%%
\subsection{Files and Installation}

The package consists of the files:
%
\begin{center}
\begin{tabular}{ll}
    |README.txt|   & readme file \\
    |childdoc.ins| & installation file \\
    |childdoc.dtx| & source file \\
    |childdoc.def| & definition file \\
    |cdocsamp.tex| & sample main file \\
    |cdocsch1.tex| & sample include file \\
    |cdocsch2.tex| & sample include file \\
    |cdocspt3.tex| & sample part file \\
    |cdocspt4.tex| & sample part file \\
    |cdocsdrf.tex| & sample redirection file \\
    |cdocsfn1.tex| & sample redirection file \\
    |cdocsfn2.tex| & sample redirection file \\
    |childdoc.pdf| & manual
\end{tabular}
\end{center}
%
The distribution consists of the files
|README.txt|, |childdoc.ins| and |childdoc.dtx|.
%
\begin{itemize}
\item
Run (pdf)\LaTeX{} on |childdoc.dtx|
to compile the manual |childdoc.pdf| (this file).
\item
Run \LaTeX{} on |childdoc.ins| to create the definitions file |childdoc.def|
and the sample |cdocsamp.tex| with include files
|cdocsch1.tex|, |cdocsch2.tex|, |cdocspt3.tex|, |cdocspt4.tex|,
|cdocsdrf.tex|, |cdocsfn1.tex|, |cdocsfn2.tex|.
Then copy the file |childdoc.def| to an appropriate directory of your \LaTeX{}
distribution, e.g.\ \textit{texmf-root}|/tex/latex/childdoc|.
\end{itemize}

%%%%%%%%%%%%%%%%%%%%%%%%%%%%%%%%%%%%%%%%%%%%%%%%%%%%%%%%%%%%%%%%%%%%%%%%%%%%%%%%
\subsection{Related CTAN Packages}

There are several other packages which offer a similar functionality:
%
\begin{itemize}
\item
The packages
\href{http://ctan.org/pkg/docmute}{\textsf{docmute}},
\href{http://ctan.org/pkg/includex}{\textsf{includex}} and
\href{http://ctan.org/pkg/standalone}{\textsf{standalone}}
provide commands to include only the document body of
a child file thus allowing both files to be compiled individually.
\item
The packages \href{http://ctan.org/pkg/subdocs}{\textsf{subdocs}}
and \href{http://ctan.org/pkg/subfiles}{\textsf{subfiles}}
provide structures in which the main and child documents can be
encapsulated and allowing them to be compiled individually.
The inclusion mechanism is different from the conventional |\include|.
\item
The package \href{http://ctan.org/pkg/combine}{\textsf{combine}}
is an elaborate solution to combine several documents into one.
\end{itemize}
%
See also the CTAN topic \href{http://ctan.org/topic/subdocs}{\textsf{subdocs}}
for further related packages.
The present package differs from the above solutions in that
a document structure constructed with the conventional |\include| mechanism
just needs two extra commands at the top of every file
such that all constituent files can be compiled individually.

%%%%%%%%%%%%%%%%%%%%%%%%%%%%%%%%%%%%%%%%%%%%%%%%%%%%%%%%%%%%%%%%%%%%%%%%%%%%%%%%
%\subsection{Feature Suggestions}
%
%The following is a list of features which may be useful for future
%versions of this package:
%%
%\begin{itemize}
%\item
%\ldots
%\end{itemize}

%%%%%%%%%%%%%%%%%%%%%%%%%%%%%%%%%%%%%%%%%%%%%%%%%%%%%%%%%%%%%%%%%%%%%%%%%%%%%%%%
\subsection{Revision History}

%%%%%%%%%%%%%%%%%%%%%%%%%%%%%%%%%%%%%%%%
\paragraph{v2.0:} 2018/12/30

\begin{itemize}
\item
immediate forward processing
\item
added |\childdocby| mechanism
\item
manual restructured
\end{itemize}

%%%%%%%%%%%%%%%%%%%%%%%%%%%%%%%%%%%%%%%%
\paragraph{v1.6:} 2018/01/17

\begin{itemize}
\item
application for development of include files
\item
corrections to manual
\end{itemize}

%%%%%%%%%%%%%%%%%%%%%%%%%%%%%%%%%%%%%%%%
\paragraph{v1.5:} 2017/05/21

\begin{itemize}
\item
more complete structuring introduced
\item
|\childdocof| introduced
\item
|\childdoc| renamed to |\childdocmain|
\item
|\childredirect| renamed to |\childdocforward| and |\childdocforwardprefix|
and functionality expanded
\end{itemize}

%%%%%%%%%%%%%%%%%%%%%%%%%%%%%%%%%%%%%%%%
\paragraph{v1.0:} 2017/04/27

\begin{itemize}
\item
manual and install package
\item
first version published on CTAN
\end{itemize}

%%%%%%%%%%%%%%%%%%%%%%%%%%%%%%%%%%%%%%%%
\paragraph{v0.6:} 2017/04/26

\begin{itemize}
\item
redirection mechanism added
\end{itemize}

%%%%%%%%%%%%%%%%%%%%%%%%%%%%%%%%%%%%%%%%
\paragraph{v0.5:} 2017/04/26

\begin{itemize}
\item
functionality in definition file
\end{itemize}


%%%%%%%%%%%%%%%%%%%%%%%%%%%%%%%%%%%%%%%%%%%%%%%%%%%%%%%%%%%%%%%%%%%%%%%%%%%%%%%%
%%%%%%%%%%%%%%%%%%%%%%%%%%%%%%%%%%%%%%%%%%%%%%%%%%%%%%%%%%%%%%%%%%%%%%%%%%%%%%%%
%%%%%%%%%%%%%%%%%%%%%%%%%%%%%%%%%%%%%%%%%%%%%%%%%%%%%%%%%%%%%%%%%%%%%%%%%%%%%%%%
\appendix

\settowidth\MacroIndent{\rmfamily\scriptsize 000\ }

 \DocInput{childdoc.dtx}

\end{document}
%</driver>
% \fi
%
% %%%%%%%%%%%%%%%%%%%%%%%%%%%%%%%%%%%%%%%%%%%%%%%%%%%%%%%%%%%%%%%%%%%%%%%%%%%%%%
% %%%%%%%%%%%%%%%%%%%%%%%%%%%%%%%%%%%%%%%%%%%%%%%%%%%%%%%%%%%%%%%%%%%%%%%%%%%%%%
% \section{Sample}
%\iffalse
%<*samplemain>
%\fi
%
% The following presents a sample document
% with two chapters, two parts, a title page,
% a compile flag as well as three forwarding files to set the flag.
% It consists of eight |.tex| files:
% \begin{center}
% \begin{tabular}{ll}
% |cdocsamp.tex|&main file\\
% |cdocsch1.tex|&include file for chapter 1\\
% |cdocsch2.tex|&include file for chapter 2\\
% |cdocspt3.tex|&include file for part 3\\
% |cdocspt4.tex|&include file for part 4\\
% |cdocsdrf.tex|&forwarding file for main file in draft mode\\
% |cdocsfi1.tex|&forwarding file for final version of chapter 1\\
% |cdocsfi2.tex|&forwarding file for final version of chapter 2\\
% \end{tabular}
% \end{center}
% Each of the eight files can be compiled directly by the \LaTeX{} compiler.
%
% %%%%%%%%%%%%%%%%%%%%%%%%%%%%%%%%%%%%%%
% \paragraph{Main File.}
%
% The main file is called |cdocsamp.tex|.
%
% Load the \textsf{childdoc} definitions and
% declare the filename for the main document:
%    \begin{macrocode}
\input{childdoc.def}
\childdocmain{}
%    \end{macrocode}

% Optional override for |\version| flag:
%    \begin{macrocode}
%%\ifchilddoc\else\providecommand{\version}{draft}\fi
%    \end{macrocode}

% Define the default values for the |\version| flag
% (|final| for the main file and |draft| for childs):
%    \begin{macrocode}
\ifchilddoc
\providecommand{\version}{draft}
\else
\providecommand{\version}{final}
\fi
%    \end{macrocode}

% Load the standard document class:
%    \begin{macrocode}
\documentclass[12pt]{article}
%    \end{macrocode}

% Start the document body:
%    \begin{macrocode}
\begin{document}
%    \end{macrocode}

% Declare a title page.
% Print title, part of document being processed and version flag:
%    \begin{macrocode}
\addtocounter{page}{-1}
\begin{center}
{\LARGE\bfseries{}childdoc example\par}
\vspace{1cm}
\ifchilddoc
\ifchilddocmanual part\else chapter\fi:
`\childdocname' of `\childdocjob'\par
\else
main document: `\childdocjob'\par
\fi
version: \version\par
\end{center}
\newpage
%    \end{macrocode}

% Manually include selected file,
% otherwise process as usual:
%    \begin{macrocode}
\ifchilddocmanual
\section*{part `\childdocname'}
\input{\childdocname}
\else
%    \end{macrocode}

% Include the two chapters:
%    \begin{macrocode}
\include{cdocsch1}
\include{cdocsch2}
%    \end{macrocode}

% Include the two parts unless only chapters should be displayed:
%    \begin{macrocode}
\ifchilddoc\else
\section{part three}
\input{cdocspt3}
\section{part four}
\input{cdocspt4}
\fi
%    \end{macrocode}

% Process as usual until here:
%    \begin{macrocode}
\fi
%    \end{macrocode}

% End of document body:
%    \begin{macrocode}
\end{document}
%    \end{macrocode}
%\iffalse
%</samplemain>
%\fi
%
% %%%%%%%%%%%%%%%%%%%%%%%%%%%%%%%%%%%%%%
% \paragraph{Chapter Include Files.}
%
% The include files are called |cdocsch1.tex| and |cdocsch2.tex|.
%
%\iffalse
%<*samplechap1|samplechap2>
%\fi

% Optional override for |\version| flag:
%    \begin{macrocode}
%%\providecommand{\version}{final}
%    \end{macrocode}

% Include the main document:
%    \begin{macrocode}
\input{childdoc.def}
\childdocof{cdocsamp}
%    \end{macrocode}

%\iffalse
%</samplechap1|samplechap2>
%\fi
%
%\iffalse
%<*samplechap1>
%\fi
% Some text for chapter 1:
%    \begin{macrocode}
\section{one}
some text in chapter one
%    \end{macrocode}

%\iffalse
%</samplechap1>
%\fi
% Some text for chapter 2:
%\iffalse
%<*samplechap2>
%\fi
%    \begin{macrocode}
\section{two}
more text in chapter two
%    \end{macrocode}

%\iffalse
%</samplechap2>
%\fi
%
% %%%%%%%%%%%%%%%%%%%%%%%%%%%%%%%%%%%%%%
% \paragraph{Part Include Files.}
%
% The include files are called |cdocspt3.tex| and |cdocspt4.tex|.
%
%\iffalse
%<*samplepart3|samplepart4>
%\fi

% Optional override for |\version| flag:
%    \begin{macrocode}
%%\providecommand{\version}{final}
%    \end{macrocode}

% Include the main document:
%    \begin{macrocode}
\input{childdoc.def}
\childdocby{cdocsamp}
%    \end{macrocode}

%\iffalse
%</samplepart3|samplepart4>
%\fi
%
%\iffalse
%<*samplepart3>
%\fi
% Some text for part 3:
%    \begin{macrocode}
some text in part three
%    \end{macrocode}

%\iffalse
%</samplepart3>
%\fi
% Some text for part 4:
%\iffalse
%<*samplepart4>
%\fi
%    \begin{macrocode}
more text in part four
%    \end{macrocode}

%\iffalse
%</samplepart4>
%\fi
%
% %%%%%%%%%%%%%%%%%%%%%%%%%%%%%%%%%%%%%%
% \paragraph{Forwarding for a Complete Draft.}
%
% The following forwarding file |cdocsdrf.tex|
% compiles the main document in draft mode:
%\iffalse
%<*sampledraft>
%\fi
%    \begin{macrocode}
\def\version{draft}
\input{childdoc.def}
\childdocforward{cdocsamp}
%    \end{macrocode}

%\iffalse
%</sampledraft>
%\fi
%
% %%%%%%%%%%%%%%%%%%%%%%%%%%%%%%%%%%%%%%
% \paragraph{Forwarding for Final Version of the Chapters.}
%
% The following forwarding files |cdocsfn1.tex| and |cdocsfn2.tex|
% (with identical content)
% compile the final versions of the child documents
% |cdocsch1.tex| and |cdocsch2.tex|, respectively:
%\iffalse
%<*samplefinal>
%\fi
%    \begin{macrocode}
\def\version{final}
\input{childdoc.def}
\childdocforwardprefix[cdocsamp]{cdocsfn}{cdocsch}
%    \end{macrocode}

%\iffalse
%</samplefinal>
%\fi
%
% %%%%%%%%%%%%%%%%%%%%%%%%%%%%%%%%%%%%%%
% \paragraph{Command Line Processing.}
%
% The following three command lines generate the output files
% |cdocscld|, |cdocscl1| and |cdocscl2|
% which should be identical to
% |cdocsdrf|, |cdocsch1| and |cdocsfn2|, respectively:
% \begin{center}
% \begin{tabular}{l}
% |latex -jobname cdocscld \|\\
% |  "\def\version{draft}\input{childdoc.def}\childdocforward{cdocsamp}"|\\
% |latex -jobname cdocscl1 \|\\
% |  "\input{childdoc.def}\childdocforward[cdocsamp]{cdocsch1}"|\\
% |latex -jobname cdocscl2 \|\\
% |  "\def\version{final}\input{childdoc.def}\childdocforward{cdocsch2}"|
% \end{tabular}
% \end{center}
% Note that the trailing backslash on each first line
% merely continues the input to the second line
% (for convenient cut ant paste).
% Furthermore, the command |latex| can be replaced by any
% of its alternative versions such as |pdflatex|.
%
% %%%%%%%%%%%%%%%%%%%%%%%%%%%%%%%%%%%%%%%%%%%%%%%%%%%%%%%%%%%%%%%%%%%%%%%%%%%%%%
% %%%%%%%%%%%%%%%%%%%%%%%%%%%%%%%%%%%%%%%%%%%%%%%%%%%%%%%%%%%%%%%%%%%%%%%%%%%%%%
% \section{Implementation}
%\iffalse
%<*package>
%\fi
%
% This section describes the definitions file |childdoc.def|.

% The definitions cannot be loaded using |\usepackage| or |\RequirePackage|
% which has a mechanism to prevent loading a style file more than once.
% When loading the definitions by means of |\input|
% multiple instances have to be prevented manually:
%\iffalse
%This code needs to be before the `\ProvidesFile' directive
%which is defined at the beginning of this file.
%Therefore it is also placed there and commented out here.
%</package>
%<*discard>
%\fi
%    \begin{macrocode}
\ifdefined\childdocmain\endinput\fi
%    \end{macrocode}
%\iffalse
%</discard>
%<*package>
%\fi
%
% \macro{\ifchilddoc}
% \macro{\ifchilddocmanual}
% The conditional |\ifchilddoc| tells whether a
% child (true) or main (false) document is being compiled.
% The conditional |\ifchilddocmanual| tells whether
% the |\includeonly| mechanism is used (false) or
% the selection of child files must be performed manually (true).
% The definitions initialise to false:
%    \begin{macrocode}
\newif\ifchilddoc
\newif\ifchilddocmanual
%    \end{macrocode}

% \macro{\childdocname}
% \macro{\childdocjob}
% The macro |\childdocname| stores the name of the main document
% to be compiled. The macro |\childdocjob| stores the name of
% the document on which the \LaTeX{} compiler was originally invoked.
% The content of |\jobname| cannot be compared
% to filenames specified in the source due to different catcodes.
% The following code rescans |\jobname|, stores the result
% in |\childdocname| and saves a copy in |\childdocjob|:
%    \begin{macrocode}
\edef\childdocname{\scantokens\expandafter{\jobname\noexpand}}
\let\childdocjob\childdocname
%    \end{macrocode}

% \macro{\childdocdisable}
% The macro |\childdocdisable| prevents the main file
% from being processed more than once.
% At this stage, the main document command |\childdocmain|
% is assumed to be called once again where it should do nothing.
% Any subsequent call to it should prevent
% a secondary processing of the main document
% It overwrites the forwarding commands
% |\childdocof| and |\childdocforward|
% with empty macros to prevent further inclusions of the main document:
%    \begin{macrocode}
\newcommand{\childdocdisable}
{
  \renewcommand{\childdocmain}[1]{\renewcommand{\childdocmain}[1]{\endinput}}
  \renewcommand{\childdocof}[1]{}
  \renewcommand{\childdocby}[2][]{}
  \renewcommand{\childdocforward}[2][]{}
  \renewcommand{\childdocdisable}{}
}
%    \end{macrocode}

% \macro{\childdocmain}
% The macro |\childdocmain| is to be called at the top of the main file
% with nothing or the main filename (without extension) as argument.
% First, it breaks loops.
% If the argument is not empty and does not match |\childdocname|
% (which is set by the first inclusion of |childdoc.def|),
% |\ifchilddoc| is set to true, |\includeonly| is applied to the child file
% and |\jobname| is set to the main file
% (for proper handling of |.aux| files):
%    \begin{macrocode}
\newcommand{\childdocmain}[1]
{
  \childdocdisable\childdocmain{}
  \if?#1?\else
    \begingroup
      \def\childdoctmp{#1}
      \ifx\childdoctmp\childdocname
        \def\childdoctmp{}
      \else
        \def\childdoctmp
        {
          \childdoctrue
          \includeonly{\childdocname}
          \def\childdocjob{#1}
          \def\jobname{#1}
        }
      \fi
      \expandafter
    \endgroup
    \childdoctmp
  \fi
}
%    \end{macrocode}

% \macro{\childdocof}
% The command |\childdocof| redirects
% compilation to the main file |#1|.
%    \begin{macrocode}
\newcommand{\childdocof}[1]
{
  \childdocdisable
  \childdoctrue
  \includeonly{\childdocname}
  \def\jobname{#1}
  \def\childdocjob{#1}
  \input{#1}
}
%    \end{macrocode}

% \macro{\childdocby}
% The command |\childdocby| ....
%    \begin{macrocode}
\newcommand{\childdocby}[2][]
{
  \childdocdisable
  \childdoctrue
  \childdocmanualtrue
  \if?#1?\else
    \def\jobname{#2}
  \fi
  \def\childdocjob{#2}
  \input{#2}
  \endinput
}
%    \end{macrocode}

% \macro{\childdocforward}
% The command |\childdocforward| redirects
% compilation to the main file or
% (if the optional argument is given) a child file.
% Parameters are set as if the main file
% or a child file starting with |\childdocof| was compiled.
% Then compilation is handed over to the main file:
%    \begin{macrocode}
\newcommand{\childdocforward}[2][]
{
  \begingroup
    \if?#1?
      \def\childdoctmp
      {
        \def\childdocname{#2}
        \def\childdocjob{#2}
        \def\jobname{#2}
        \input{#2}
        \endinput
      }
    \else
      \def\childdoctmp
      {
        \childdocdisable
        \def\childdocname{#2}
        \childdoctrue
        \includeonly{#2}
        \def\childdocjob{#1}
        \def\jobname{#1}
        \input{#1}
        \endinput
      }
    \fi
    \expandafter
  \endgroup
  \childdoctmp
}
%    \end{macrocode}

% \macro{\childdocforwardprefix}
% The command |\childdocforwardprefix| redirects
% compilation to the main or a child file by means of a pattern.
% The prefix |#1| in the current filename is replaced by |#2|
% and the suffix of the current filename is kept
% (it is assumed that the filename does not contain the substring `|~~~|'
% which is used as a delimiter).
% Compilation is handed over to the new file by |\childdocforward|:
%    \begin{macrocode}
\newcommand{\childdocforwardprefix}[3][]
{
  \begingroup
    \def\childdocextract #2##1~~~{\def\childdoctmp{\childdocforward[#1]{#3##1}}}
    \expandafter\childdocextract\childdocname~~~
    \expandafter
  \endgroup
  \childdoctmp
}
%    \end{macrocode}

% \macro{\childdoc}
% The deprecated macro |\childdoc| is a legacy version of |\childdocmain|:
%    \begin{macrocode}
\newcommand{\childdoc}{\childdocmain}
%    \end{macrocode}

% \macro{\childdocredirect}
% The deprecated macro |\childdocredirect| is a legacy version
% of |\childdocforward| and |\childdocforwardprefix|:
%    \begin{macrocode}
\newcommand{\childdocredirect}[2][]
{
  \begingroup
    \if?#1?
      \def\childdoctmp{\childdocforward{#2}}
    \else
      \def\childdoctmp{\childdocforwardprefix{#1}{#2}}
    \fi
    \expandafter
  \endgroup
  \childdoctmp
}
%    \end{macrocode}

%\iffalse
%</package>
%\fi
%
\endinput
|\\
|\childdocby{|\textit{main}|}|\\
\end{tabular}
\end{center}
%
The directive |\childdocby| is similar to |\childdocof|
described in \secref{sec:include},
but the subsequent selection of content must be done manually.
To that end, both |\ifchilddoc| and |\ifchilddocmanual|
will be true upon processing of a part,
and the name of the part is stored in |\childdocname|.
Note that |\jobname| will be set to the filename of the current part
so that each part receives an individual |.aux| file
that does not interfere with the |.aux| file(s) of the main document.
This behaviour can be altered by the alternative form
|\childdocby[*]{|\textit{main}|}| (with a non-empty optional argument)
which uses the |.aux| file of the main document
by setting |\jobname| to \textit{main}.

%%%%%%%%%%%%%%%%%%%%%%%%%%%%%%%%%%%%%%%%%%%%%%%%%%%%%%%%%%%%%%%%%%%%%%%%%%%%%%%%
\subsection{Driver Development}
\label{sec:driver}

The \textsf{childdoc} mechanism can also be use for the development
of definition files such as \LaTeX{} styles or classes.
This case differs from the above setup with multiple parts
included by |\include| in that no |\includeonly| should be invoked.
This can be achieved by starting the include file
(before |\ProvidesPackage|) with:
%
\begin{center}
\begin{tabular}{l}
|% \iffalse
%
% childdoc.dtx Copyright (C) 2017-2018 Niklas Beisert
%
% This work may be distributed and/or modified under the
% conditions of the LaTeX Project Public License, either version 1.3
% of this license or (at your option) any later version.
% The latest version of this license is in
%   http://www.latex-project.org/lppl.txt
% and version 1.3 or later is part of all distributions of LaTeX
% version 2005/12/01 or later.
%
% This work has the LPPL maintenance status `maintained'.
%
% The Current Maintainer of this work is Niklas Beisert.
%
% This work consists of the files childdoc.dtx and childdoc.ins
% and the derived files childdoc.def and cdocsamp.tex with
% cdocsch1.tex, cdocsch2.tex, cdocsdrf.tex, cdocsfn1.tex, cdocsfn2.tex.
%
%<package>\ifdefined\childdocmain\endinput\fi
%<package>\ProvidesFile{childdoc.def}[2018/12/30 v2.0 child document driver]
%<samplemain>\ProvidesFile{cdocsamp.tex}[2018/12/30 v2.0 sample for childdoc]
%<*driver>
%\ProvidesFile{childdoc.drv}[2018/12/30 v2.0 childdoc reference manual file]
\PassOptionsToClass{10pt,a4paper}{article}
\documentclass{ltxdoc}

\usepackage[margin=35mm]{geometry}
\usepackage{hyperref}
\usepackage{hyperxmp}
\usepackage[usenames]{color}

\hypersetup{colorlinks=true}
\hypersetup{pdfstartview=FitH}
\hypersetup{pdfpagemode=UseNone}
\hypersetup{pdfsource={}}
\hypersetup{pdflang={en-UK}}
\hypersetup{pdfcopyright={Copyright 2017-2018 Niklas Beisert.
  This work may be distributed and/or modified under the
  conditions of the LaTeX Project Public License, either version 1.3
  of this license or (at your option) any later version.}}
\hypersetup{pdflicenseurl={http://www.latex-project.org/lppl.txt}}
\hypersetup{pdfcontactaddress={ETH Zurich, ITP, HIT K,
  Wolfgang-Pauli-Strasse 27}}
\hypersetup{pdfcontactpostcode={8093}}
\hypersetup{pdfcontactcity={Zurich}}
\hypersetup{pdfcontactcountry={Switzerland}}
\hypersetup{pdfcontactemail={nbeisert@itp.phys.ethz.ch}}
\hypersetup{pdfcontacturl={http://people.phys.ethz.ch/\xmptilde nbeisert/}}

\newcommand{\secref}[1]{\hyperref[#1]{section \ref*{#1}}}

\parskip1ex
\parindent0pt
\let\olditemize\itemize
\def\itemize{\olditemize\parskip0pt}

\begin{document}

\title{The \textsf{childdoc} Package}
\hypersetup{pdftitle={The childdoc Package}}
\author{Niklas Beisert\\[2ex]
  Institut f\"ur Theoretische Physik\\
  Eidgen\"ossische Technische Hochschule Z\"urich\\
  Wolfgang-Pauli-Strasse 27, 8093 Z\"urich, Switzerland\\[1ex]
  \href{mailto:nbeisert@itp.phys.ethz.ch}
  {\texttt{nbeisert@itp.phys.ethz.ch}}}
\hypersetup{pdfauthor={Niklas Beisert}}
\hypersetup{pdfsubject={Manual for the LaTeX2e Package childdoc}}
\date{30 December 2018, \textsf{v2.0}}
\maketitle

\begin{abstract}\noindent
\textsf{childdoc} is a \LaTeXe{} package
that enables the direct compilation
of document sections included by |\include|
to individual files.
\end{abstract}

\begingroup
\parskip0ex
\tableofcontents
\endgroup

%%%%%%%%%%%%%%%%%%%%%%%%%%%%%%%%%%%%%%%%%%%%%%%%%%%%%%%%%%%%%%%%%%%%%%%%%%%%%%%%
%%%%%%%%%%%%%%%%%%%%%%%%%%%%%%%%%%%%%%%%%%%%%%%%%%%%%%%%%%%%%%%%%%%%%%%%%%%%%%%%
\section{Introduction}

\LaTeX{} provides a mechanism to structure a large document (such as a book)
into a main file and several child files (containing the chapters)
using the |\include| command.
This mechanism is beneficial for documents
which span hundreds of pages in order to
make the source file(s) more manageable.
Moreover, compilation can be restricted to
selected child files by means of the |\includeonly| command.
The latter feature can be used to reduce the compilation time while editing
(this was significantly more useful in the earlier days of \LaTeX{})
or to generate a smaller document which is easier to navigate.
Another application of |\includeonly| is to generate
documents consisting of selected parts of the complete document.

However, there are a few drawbacks of the plain |\include| mechanism:
\begin{itemize}
\item
The child files cannot be compiled on their own,
they can only be compiled via the main file.
A naive editing environment
(such as a text editor with an option
to have the current file processed by \LaTeX)
may require one to switch to the main file before compiling;
attempting to compile the child file produces errors.
\item
The main file must be modified (each time)
to adjust the |\includeonly| command
to the present needs. This easily leaves the main file in a messy state.
\item
The generated document will always carry the filename
of the main document. This is inconvenient if
several child files are to be compiled and
to be kept for distribution.
\end{itemize}

The present package provides a simple interface
to make child files individually compilable by \LaTeX{}.
Compiling a child file then has the same effect as compiling
the main file with an |\includeonly| command
to select the appropriate child.
Moreover the generated document will carry the name of the child
rather than the main file.
This resolves all three above issues.

This feature is meant to make the editing of books,
thesis documents and lecture notes somewhat more convenient.
However, the package can also be used efficiently for
composing a series of documents (such as exercise sheets)
which are typically distributed individually.
It then assists the author in generating the individual documents
(potentially in different versions)
as well as a document containing the collected series.
Another application is in developing style files
or other kinds of included material
where compilation of the style file could redirect
to a sample or test file.

%%%%%%%%%%%%%%%%%%%%%%%%%%%%%%%%%%%%%%%%%%%%%%%%%%%%%%%%%%%%%%%%%%%%%%%%%%%%%%%%
%%%%%%%%%%%%%%%%%%%%%%%%%%%%%%%%%%%%%%%%%%%%%%%%%%%%%%%%%%%%%%%%%%%%%%%%%%%%%%%%
\section{Usage}

First of all, the package \textsf{childdoc} is \emph{not} a standard
\LaTeXe{} |.sty| style file! Therefore it needs to be invoked in
a non-standard way.

%%%%%%%%%%%%%%%%%%%%%%%%%%%%%%%%%%%%%%%%%%%%%%%%%%%%%%%%%%%%%%%%%%%%%%%%%%%%%%%%
\subsection{Included Files}
\label{sec:include}

%%%%%%%%%%%%%%%%%%%%%%%%%%%%%%%%%%%%%%%%
\DescribeMacro{\childdocmain}
To use the package, add the commands
\begin{center}
\begin{tabular}{l}
|\input{childdoc.def}|\\
|\childdocmain{}|\\
\end{tabular}
\end{center}
at the very top of the main \LaTeX{} file,
in particular \emph{before} the |\documentclass| statement!
The argument of |\childdocmain| should be left empty
(but it must be present).

%%%%%%%%%%%%%%%%%%%%%%%%%%%%%%%%%%%%%%%%
\DescribeMacro{\childdocof}
Furthermore, add the commands
\begin{center}
\begin{tabular}{l}
|\input{childdoc.def}|\\
|\childdocof{|\textit{main}|}|\\
\end{tabular}
\end{center}
at the top of every child file \textit{child}
which is included by |\include{|\textit{child}|}|
from within the main file
(or at least for those files to be compiled individually).
The argument \textit{main} must be the filename of the main file.

There are a couple of
considerations in setting up the main and child documents:

%%%%%%%%%%%%%%%%%%%%%%%%%%%%%%%%%%%%%%%%
\paragraph{Restrictions.}

Please note the following restrictions:
\begin{itemize}
\item
|\childdocmain| must be called with one argument \textit{main}
to ensure compatibility with earlier version of the package.
It must either be empty (|\childdocmain{}|)
or precisely match the filename of the main file in which it is specified.
See \secref{sec:detection} for further information.
\item
The filename \textit{main} must be specified without the |.tex| extension.
\item
The filename \textit{main} is case sensitive
(even in case-insensitive file systems)
due to internal string comparison.
\item
The argument \textit{main} should be fully expanded, it cannot be a macro.
\item
Subdirectories and special characters should be avoided in filenames.
\item
The command |\childdocmain{|\textit{main}|}| must be followed by a whitespace.
It should not be followed immediately by another command
or by a comment mark `|%|'.
This is because the \TeX{} parser reads the token immediately following
the argument of |\childdocmain| and puts it
at the beginning of every child section;
however, a white\-space is ignored.
\end{itemize}

%%%%%%%%%%%%%%%%%%%%%%%%%%%%%%%%%%%%%%%%
\paragraph{Content of Main File.}

It is advisable to place all content in the child files included by |\include|.
Any output contained in the main file will appear in all child documents
unless suppressed manually;
it cannot be suppressed automatically by the |\includeonly| directive
and thus should normally be avoided.
A method to include some content in the main file
by means of conditional processing is described in \secref{sec:conditional}.

%%%%%%%%%%%%%%%%%%%%%%%%%%%%%%%%%%%%%%%%
\paragraph{Page Numbering.}

When only a part of the document is compiled,
the appropriate numbering of pages
(as well as other status parameters)
is determined from the |.aux| files.
The latter contain information from previous passes.
However this information needs to propagate through
all intermediate child documents.
Therefore the page numbering in child documents may well
be inconsistent until the complete document is compiled at least once.

A useful (if unconventional) way to always ensure a consistent
page numbering is to restart the numbering in each child document
and denote the pages by `\textit{child}|.|\textit{page}'
where \textit{child} represents the chapter/section number of the child file.
This can be achieved by the command
|\numberwithin{page}{|\textit{child}|}|
of the \textsf{amsmath} package
where \textit{child} can be |chapter| or |section|
depending on the chosen structuring.
Alternatively, one can modify the macro |\thepage| appropriately
and reset the counter |page| at the start of each child file.

%%%%%%%%%%%%%%%%%%%%%%%%%%%%%%%%%%%%%%%%%%%%%%%%%%%%%%%%%%%%%%%%%%%%%%%%%%%%%%%%
\subsection{Conditional Processing}
\label{sec:conditional}

The package provides a mechanism to compile different versions
of a document. To customise the versions further some conditional processing
can come in handy to distinguish which version is being compiled.
The package provides two macros to describe the compilation context:

%%%%%%%%%%%%%%%%%%%%%%%%%%%%%%%%%%%%%%%%
\DescribeMacro{\ifchilddoc}
The conditional |\ifchilddoc| distinguishes between the compilation of
child documents and the main document:
%
\begin{center}
|\ifchilddoc |\textit{child-code}| |[|\||else |\textit{main-code}]| \||fi|
\end{center}

%%%%%%%%%%%%%%%%%%%%%%%%%%%%%%%%%%%%%%%%
\DescribeMacro{\childdocname}
\DescribeMacro{\childdocjob}
The macro |\childdocname| contains the filename (without extension)
of the main or child file being processed.
Note that |\childdocjob| will always contain the name of the main file.

%%%%%%%%%%%%%%%%%%%%%%%%%%%%%%%%%%%%%%%%
\paragraph{Title Page.}

Conditional processing can be used to include a title or banner page
in the main document when proper precautions are taken.
Importantly, the code in the main file should ensure that the page counter
(as well as other status parameters which are stored in the |.aux| files)
takes the same value after the conditional processing.
Otherwise the page numbers may take divergent values
depending on which part is compiled.

For example, a title page could be declared by:
%
\begin{center}
\begin{tabular}{l}
|\ifchilddoc\||else|\\
|\addtocounter{page}{-1}|\\
\textit{code for title page}\\
|\newpage|\\
|\||fi|
\end{tabular}
\end{center}
%
A banner page for the child documents can be generated by:
%
\begin{center}
\begin{tabular}{l}
|\ifchilddoc|\\
|\addtocounter{page}{-1}|\\
\textit{code for banner page}\\
|\newpage|\\
|\||fi|
\end{tabular}
\end{center}
%
Here one could write a message such as:
\begin{center}
|This is the part \childdocname{} of \childdocjob{}.|
\end{center}

%%%%%%%%%%%%%%%%%%%%%%%%%%%%%%%%%%%%%%%%%%%%%%%%%%%%%%%%%%%%%%%%%%%%%%%%%%%%%%%%
\subsection{Flags}
\label{sec:flags}

The package makes it easy to generate different versions
of the main or child documents.
To this end compilation flags can be defined
and assigned different default values.
They will be particularly useful in conjunction
with the forwarding mechanism described in \secref{sec:forward}.

For example, it may be useful to have a flag |\version|
which can be set to |draft| or |final|.
The document source will contain some conditional code
depending on the value of |\version|.
Suppose further, the flag should default to |final| for the main file
and to |draft| for child files
which is a natural assignment for editing the document.
This is achieved by placing the following code
in the preamble of the main document
(below the |\childdocmain| directive):
%
\begin{center}
\begin{tabular}{l}
|\ifchilddoc|\\
|\providecommand{\version}{draft}|\\
|\||else|\\
|\providecommand{\version}{final}|\\
|\||fi|
\end{tabular}
\end{center}
%
The definition by |\providecommand| makes sure
that previous definitions are not overwritten.
Further statements |\providecommand{\version}{...}|
can thus be added before the above code to override it.

For the main file, one might add a line
(between |\childdocmain| and the above block)
%
\begin{center}
|%\ifchilddoc\||else\providecommand{\version}{draft}\||fi|
\end{center}
%
which can be uncommented to produce a draft version.
Likewise one can add a line to the very top of a child file
(above the |\childdocof{|\textit{main}|}| directive)
%
\begin{center}
|%\providecommand{\version}{final}|
\end{center}
%
which can be uncommented to produce the final version of this child document.

%%%%%%%%%%%%%%%%%%%%%%%%%%%%%%%%%%%%%%%%%%%%%%%%%%%%%%%%%%%%%%%%%%%%%%%%%%%%%%%%
\subsection{Forwarding}
\label{sec:forward}

Different versions of the main or child documents
using compilation flags as described in \secref{sec:flags}
can be (permanently) stored in different files
for convenient compilation, viewing and distribution.
To this end, the package defines a command
to pass on compilation to a different file:

%%%%%%%%%%%%%%%%%%%%%%%%%%%%%%%%%%%%%%%%
\DescribeMacro{\childdocforward}
The command |\childdocforward| redirects processing to
another source file:
%
\begin{center}
\begin{tabular}{l}
|\input{childdoc.def}|\\
|\childdocforward[|\textit{main}|]{|\textit{dest}|}|\\
\end{tabular}
\end{center}
%
The argument \textit{dest} is the destination file
(without extension).
It should be the main file or one of the child files.
Note that further \textsf{childdoc} directives
such as |\childdocof| and |\childdocforward|
in the indicated file will be processed in this form.
The optional argument \textit{main}
passes on directly to the main file \textit{main}
while pretending to compile the child \textit{dest}.
This form behaves as if \textit{dest}
issues |\childdocof{|\textit{main}|}| right away,
and no further \textsf{childdoc} directives will be processed.

%%%%%%%%%%%%%%%%%%%%%%%%%%%%%%%%%%%%%%%%
\DescribeMacro{\...prefix}
In the alternative form |\childdocforwardprefix|,
%
\begin{center}
\begin{tabular}{l}
|\input{childdoc.def}|\\
|\childdocforwardprefix[|\textit{main}|]{|\textit{prefix}|}{|\textit{dest}|}|
\end{tabular}
\end{center}
%
the destination file is determined by a pattern
depending on the current file:
To make this work, the current file must be called
`{\textit{prefix}\hspace{0.2em}\textit{suffix}}'
with \textit{prefix} matching precisely the argument.
Processing is then passed on to the file
`{\textit{dest}\hspace{0.2em}\textit{suffix}}'.
Surely, the same effect is achieved by
directly specifying the
argument `{\textit{dest}\hspace{0.2em}\textit{suffix}}'
in the first form.
However, that requires to set up a different file
for each child. With the alternative form of the command
all these files can have exactly the same content
which simplifies setting them up and maintaining them.

For example, the following file |draft.tex|
with a compilation flag |\version| as described in \secref{sec:flags}
compiles the main document as a draft:
%
\begin{center}
\begin{tabular}{l}
|\def\version{draft}|\\
|\input{childdoc.def}|\\
|\childdocforward{|\textit{main}|}|
\end{tabular}
\end{center}
%
Likewise, the following files |final|\textit{nn}|.tex|
compile the final version of the child document
|child|\textit{nn}|.tex|:
%
\begin{center}
\begin{tabular}{l}
|\def\version{final}|\\
|\input{childdoc.def}|\\
|\childdocforwardprefix{final}{child}|
\end{tabular}
\end{center}
%

Note that when several versions of a main file and/or of each child file
are to be generated, it may be convenient to set up a |Makefile| or
shell script to automatise the process.

%%%%%%%%%%%%%%%%%%%%%%%%%%%%%%%%%%%%%%%%%%%%%%%%%%%%%%%%%%%%%%%%%%%%%%%%%%%%%%%%
\subsection{Command Line Processing}
\label{sec:commandline}

The effect of redirection files can also be achieved by invoking
the \LaTeX{} compiler with a more elaborate command line.
Most conveniently this should be done as part
of a shell script or a |Makefile|.

When using \textsf{childdoc} in the main file, the following
command lines effectively perform a redirection
(note that depending on the shell being used,
backslashes may have to be doubled: `|\|' $\to$ `|\\|'):
%
\begin{center}
|... -jobname "|\textit{target}|" |\\|"|[\textit{flags}]%
|\input{childdoc.def}\childdocforward[|\textit{main}|]{|\textit{dest}|}"|
\end{center}
%
Here \textit{target} is the name of the output file,
\textit{main} is the name of the main file
and \textit{dest} is the name of the main or child file to be processed
(all filenames without extensions).
The optional argument \textit{main} can be omitted
if \textit{main} matches \textit{dest}.
Optionally, compilation \textit{flags} can be defined via |\def| commands.
This command line makes the \TeX{} engine believe
it is compiling the file \textit{target}
whose content is specified as the latter parameter.
The provided code then forwards the processing to
\textit{main} or \textit{dest} as described in \secref{sec:forward}.

%%%%%%%%%%%%%%%%%%%%%%%%%%%%%%%%%%%%%%%%%%%%%%%%%%%%%%%%%%%%%%%%%%%%%%%%%%%%%%%%
\subsection{Include by Input}
\label{sec:input}

Including child documents by |\include| has some restrictions by design.
Most notably, the content of a child document always occupies
its own set of pages; pages cannot be shared between child documents.
Usually, this behaviour makes perfect sense
because each child document contain an essential part of the document.
However, in some situations it may be desirable to compose
a document from a collection of parts
without having mandatory page breaks between then.
For this case, the package
provides a mechanism to include parts
by |\input| which can also be processed individually.
However, by construction this mechanism
requires manual handling of the content to be output.

%%%%%%%%%%%%%%%%%%%%%%%%%%%%%%%%%%%%%%%%
\DescribeMacro{\ifchilddocmanual}
The main file should be prepared as usual, see \secref{sec:include}.
However, the document body must make a distinction
between processing of an individual part and of the main document, e.g.:
%
\begin{center}
\begin{tabular}{l}
|\ifchilddocmanual|\\
|\input{\childdocname}|\\
|\||else|\\
\textit{document body with }|\input{|\textit{part}|}|\\
|\||fi|
\end{tabular}
\end{center}
%
The conditional |\ifchilddocmanual| is true whenever
a part to be included by |\input| is being compiled,
and the name of the part is stored in |\childdocname|.

%%%%%%%%%%%%%%%%%%%%%%%%%%%%%%%%%%%%%%%%
\DescribeMacro{\childdocby}
Each part to be included by |\input| should start with:
%
\begin{center}
\begin{tabular}{l}
|\input{childdoc.def}|\\
|\childdocby{|\textit{main}|}|\\
\end{tabular}
\end{center}
%
The directive |\childdocby| is similar to |\childdocof|
described in \secref{sec:include},
but the subsequent selection of content must be done manually.
To that end, both |\ifchilddoc| and |\ifchilddocmanual|
will be true upon processing of a part,
and the name of the part is stored in |\childdocname|.
Note that |\jobname| will be set to the filename of the current part
so that each part receives an individual |.aux| file
that does not interfere with the |.aux| file(s) of the main document.
This behaviour can be altered by the alternative form
|\childdocby[*]{|\textit{main}|}| (with a non-empty optional argument)
which uses the |.aux| file of the main document
by setting |\jobname| to \textit{main}.

%%%%%%%%%%%%%%%%%%%%%%%%%%%%%%%%%%%%%%%%%%%%%%%%%%%%%%%%%%%%%%%%%%%%%%%%%%%%%%%%
\subsection{Driver Development}
\label{sec:driver}

The \textsf{childdoc} mechanism can also be use for the development
of definition files such as \LaTeX{} styles or classes.
This case differs from the above setup with multiple parts
included by |\include| in that no |\includeonly| should be invoked.
This can be achieved by starting the include file
(before |\ProvidesPackage|) with:
%
\begin{center}
\begin{tabular}{l}
|\input{childdoc.def}|\\
|\childdocforward{|\textit{main}|}|\\
\end{tabular}
\end{center}
%
or alternatively with:
%
\begin{center}
\begin{tabular}{l}
|\input{childdoc.def}|\\
|\childdocby{|\textit{main}|}|\\
\end{tabular}
\end{center}
%
Both forms have slightly different effects as described above.
The main file is prepared as usual, see \secref{sec:include}.

%%%%%%%%%%%%%%%%%%%%%%%%%%%%%%%%%%%%%%%%%%%%%%%%%%%%%%%%%%%%%%%%%%%%%%%%%%%%%%%%
\subsection{Legacy Detection}
\label{sec:detection}

The directive |\childdocmain| in the main file can detect
whether the complete document or merely a child is to be compiled
even without using the directive |\childdocof|.
This method is deprecated because it is less robust
and there is no compelling reason to use it;
it is merely provided for backward compatibility
and it may be removed in future versions.

If the detection mechanism is to be used,
it is mandatory to correctly specify
the filename of the main file as the argument of |\childdocmain|:
%
\begin{center}
\begin{tabular}{l}
|\input{childdoc.def}|\\
|\childdocmain{|\textit{main}|}|\\
\end{tabular}
\end{center}
%
If |\jobname| does not match the argument \textit{main} of |\childdocmain|,
it is assumed that |\jobname| points to the child file to be compiled.
When using |\childdocmain| with the main file specified as argument,
it suffices to start a child file
with just |\input{|\textit{main}|}|
without loading of the package and using |\childdocof|.
If instead all processing is done
with the appropriate \textsf{childdoc} directives,
the argument of \textit{main} of |\childdocmain| can be empty.

An alternative version of the command line processing described
in \secref{sec:commandline} using the detection mechanism reads:
%
\begin{center}
|... -jobname "|\textit{target}|" "|[\textit{flags}]%
[|\def\jobname{|\textit{dest}|}|]|\input{|\textit{main}|}"|
\end{center}

%%%%%%%%%%%%%%%%%%%%%%%%%%%%%%%%%%%%%%%%%%%%%%%%%%%%%%%%%%%%%%%%%%%%%%%%%%%%%%%%
\subsection{Manual Code}
\label{sec:manual}

In case one cannot be certain whether the definitions file |childdoc.def|
is installed on the target \TeX{} distribution
and one prefers not to ship it,
it is conceivable to paste a few relevant commands into the sources.

To that end, drop all statements |\input{childdoc.def}|
and perform the replacements as outlined below.
Instead of |\childdocmain{|\textit{main}|}| add the following code
to the top of the main file:
%
\begin{center}
\begin{tabular}{l}
|\||ifdefined\childdocname\endinput\||fi\newif\ifchilddoc|\\
|\edef\childdocname{\scantokens\expandafter{\jobname\noexpand}}|\\
|\def\childdocmain{|\textit{main}|}\||ifx\childdocmain\childdocname\||else|\\
|\childdoctrue\includeonly{\childdocname}\let\jobname\childdocmain\||fi|\\
\end{tabular}
\end{center}
%
Instead of |\childdocof{|\textit{main}|}| just include the main file
at the top of each child file:
%
\begin{center}
|\input{|\textit{main}|}|
\end{center}
%
A simple redirection |\childdocforward{|\textit{dest}|}| is achieved by:
%
\begin{center}
|\def\jobname{|\textit{dest}|}\input{\jobname}|
\end{center}
%
The redirection with prefix
|\childdocforwardprefix[|\textit{prefix}|]{|\textit{dest}|}|
is accomplished by:
%
\begin{center}
\begin{tabular}{l}
|{\edef\jobname{\scantokens\expandafter{\jobname\noexpand}}|\\
|\def\redirectjob |\textit{prefix}|#1~~~{\gdef\jobname{|\textit{dest}|#1}}|\\
|\expandafter\redirectjob\jobname~~~}\input{\jobname}|
\end{tabular}
\end{center}

In an alternative approach,
child documents can be compiled by a specific command line
without additional code or specific definitions:
%
\begin{center}
|... -jobname "|\textit{target}|" "|[\textit{flags}]%
|\includeonly{|\textit{dest}|}\input{|\textit{main}|}"|
\end{center}
%

%%%%%%%%%%%%%%%%%%%%%%%%%%%%%%%%%%%%%%%%%%%%%%%%%%%%%%%%%%%%%%%%%%%%%%%%%%%%%%%%
%%%%%%%%%%%%%%%%%%%%%%%%%%%%%%%%%%%%%%%%%%%%%%%%%%%%%%%%%%%%%%%%%%%%%%%%%%%%%%%%
\section{Information}

%%%%%%%%%%%%%%%%%%%%%%%%%%%%%%%%%%%%%%%%%%%%%%%%%%%%%%%%%%%%%%%%%%%%%%%%%%%%%%%%
\subsection{Copyright}

Copyright \copyright{} 2017--2018 Niklas Beisert

This work may be distributed and/or modified under the
conditions of the \LaTeX{} Project Public License, either version 1.3
of this license or (at your option) any later version.
The latest version of this license is in
  \url{http://www.latex-project.org/lppl.txt}
and version 1.3 or later is part of all distributions of \LaTeX{}
version 2005/12/01 or later.

This work has the LPPL maintenance status `maintained'.

The Current Maintainer of this work is Niklas Beisert.

This work consists of the files |README.txt|, |childdoc.ins| and |childdoc.dtx|
as well as the derived files |childdoc.def|, |cdocsamp.tex|
with |cdocsch1.tex|, |cdocsch2.tex|, |cdocspt3.tex|, |cdocspt4.tex|,
|cdocsdrf.tex|, |cdocsfn1.tex|, |cdocsfn2.tex|
as well as |childdoc.pdf|.

%%%%%%%%%%%%%%%%%%%%%%%%%%%%%%%%%%%%%%%%%%%%%%%%%%%%%%%%%%%%%%%%%%%%%%%%%%%%%%%%
\subsection{Files and Installation}

The package consists of the files:
%
\begin{center}
\begin{tabular}{ll}
    |README.txt|   & readme file \\
    |childdoc.ins| & installation file \\
    |childdoc.dtx| & source file \\
    |childdoc.def| & definition file \\
    |cdocsamp.tex| & sample main file \\
    |cdocsch1.tex| & sample include file \\
    |cdocsch2.tex| & sample include file \\
    |cdocspt3.tex| & sample part file \\
    |cdocspt4.tex| & sample part file \\
    |cdocsdrf.tex| & sample redirection file \\
    |cdocsfn1.tex| & sample redirection file \\
    |cdocsfn2.tex| & sample redirection file \\
    |childdoc.pdf| & manual
\end{tabular}
\end{center}
%
The distribution consists of the files
|README.txt|, |childdoc.ins| and |childdoc.dtx|.
%
\begin{itemize}
\item
Run (pdf)\LaTeX{} on |childdoc.dtx|
to compile the manual |childdoc.pdf| (this file).
\item
Run \LaTeX{} on |childdoc.ins| to create the definitions file |childdoc.def|
and the sample |cdocsamp.tex| with include files
|cdocsch1.tex|, |cdocsch2.tex|, |cdocspt3.tex|, |cdocspt4.tex|,
|cdocsdrf.tex|, |cdocsfn1.tex|, |cdocsfn2.tex|.
Then copy the file |childdoc.def| to an appropriate directory of your \LaTeX{}
distribution, e.g.\ \textit{texmf-root}|/tex/latex/childdoc|.
\end{itemize}

%%%%%%%%%%%%%%%%%%%%%%%%%%%%%%%%%%%%%%%%%%%%%%%%%%%%%%%%%%%%%%%%%%%%%%%%%%%%%%%%
\subsection{Related CTAN Packages}

There are several other packages which offer a similar functionality:
%
\begin{itemize}
\item
The packages
\href{http://ctan.org/pkg/docmute}{\textsf{docmute}},
\href{http://ctan.org/pkg/includex}{\textsf{includex}} and
\href{http://ctan.org/pkg/standalone}{\textsf{standalone}}
provide commands to include only the document body of
a child file thus allowing both files to be compiled individually.
\item
The packages \href{http://ctan.org/pkg/subdocs}{\textsf{subdocs}}
and \href{http://ctan.org/pkg/subfiles}{\textsf{subfiles}}
provide structures in which the main and child documents can be
encapsulated and allowing them to be compiled individually.
The inclusion mechanism is different from the conventional |\include|.
\item
The package \href{http://ctan.org/pkg/combine}{\textsf{combine}}
is an elaborate solution to combine several documents into one.
\end{itemize}
%
See also the CTAN topic \href{http://ctan.org/topic/subdocs}{\textsf{subdocs}}
for further related packages.
The present package differs from the above solutions in that
a document structure constructed with the conventional |\include| mechanism
just needs two extra commands at the top of every file
such that all constituent files can be compiled individually.

%%%%%%%%%%%%%%%%%%%%%%%%%%%%%%%%%%%%%%%%%%%%%%%%%%%%%%%%%%%%%%%%%%%%%%%%%%%%%%%%
%\subsection{Feature Suggestions}
%
%The following is a list of features which may be useful for future
%versions of this package:
%%
%\begin{itemize}
%\item
%\ldots
%\end{itemize}

%%%%%%%%%%%%%%%%%%%%%%%%%%%%%%%%%%%%%%%%%%%%%%%%%%%%%%%%%%%%%%%%%%%%%%%%%%%%%%%%
\subsection{Revision History}

%%%%%%%%%%%%%%%%%%%%%%%%%%%%%%%%%%%%%%%%
\paragraph{v2.0:} 2018/12/30

\begin{itemize}
\item
immediate forward processing
\item
added |\childdocby| mechanism
\item
manual restructured
\end{itemize}

%%%%%%%%%%%%%%%%%%%%%%%%%%%%%%%%%%%%%%%%
\paragraph{v1.6:} 2018/01/17

\begin{itemize}
\item
application for development of include files
\item
corrections to manual
\end{itemize}

%%%%%%%%%%%%%%%%%%%%%%%%%%%%%%%%%%%%%%%%
\paragraph{v1.5:} 2017/05/21

\begin{itemize}
\item
more complete structuring introduced
\item
|\childdocof| introduced
\item
|\childdoc| renamed to |\childdocmain|
\item
|\childredirect| renamed to |\childdocforward| and |\childdocforwardprefix|
and functionality expanded
\end{itemize}

%%%%%%%%%%%%%%%%%%%%%%%%%%%%%%%%%%%%%%%%
\paragraph{v1.0:} 2017/04/27

\begin{itemize}
\item
manual and install package
\item
first version published on CTAN
\end{itemize}

%%%%%%%%%%%%%%%%%%%%%%%%%%%%%%%%%%%%%%%%
\paragraph{v0.6:} 2017/04/26

\begin{itemize}
\item
redirection mechanism added
\end{itemize}

%%%%%%%%%%%%%%%%%%%%%%%%%%%%%%%%%%%%%%%%
\paragraph{v0.5:} 2017/04/26

\begin{itemize}
\item
functionality in definition file
\end{itemize}


%%%%%%%%%%%%%%%%%%%%%%%%%%%%%%%%%%%%%%%%%%%%%%%%%%%%%%%%%%%%%%%%%%%%%%%%%%%%%%%%
%%%%%%%%%%%%%%%%%%%%%%%%%%%%%%%%%%%%%%%%%%%%%%%%%%%%%%%%%%%%%%%%%%%%%%%%%%%%%%%%
%%%%%%%%%%%%%%%%%%%%%%%%%%%%%%%%%%%%%%%%%%%%%%%%%%%%%%%%%%%%%%%%%%%%%%%%%%%%%%%%
\appendix

\settowidth\MacroIndent{\rmfamily\scriptsize 000\ }

 \DocInput{childdoc.dtx}

\end{document}
%</driver>
% \fi
%
% %%%%%%%%%%%%%%%%%%%%%%%%%%%%%%%%%%%%%%%%%%%%%%%%%%%%%%%%%%%%%%%%%%%%%%%%%%%%%%
% %%%%%%%%%%%%%%%%%%%%%%%%%%%%%%%%%%%%%%%%%%%%%%%%%%%%%%%%%%%%%%%%%%%%%%%%%%%%%%
% \section{Sample}
%\iffalse
%<*samplemain>
%\fi
%
% The following presents a sample document
% with two chapters, two parts, a title page,
% a compile flag as well as three forwarding files to set the flag.
% It consists of eight |.tex| files:
% \begin{center}
% \begin{tabular}{ll}
% |cdocsamp.tex|&main file\\
% |cdocsch1.tex|&include file for chapter 1\\
% |cdocsch2.tex|&include file for chapter 2\\
% |cdocspt3.tex|&include file for part 3\\
% |cdocspt4.tex|&include file for part 4\\
% |cdocsdrf.tex|&forwarding file for main file in draft mode\\
% |cdocsfi1.tex|&forwarding file for final version of chapter 1\\
% |cdocsfi2.tex|&forwarding file for final version of chapter 2\\
% \end{tabular}
% \end{center}
% Each of the eight files can be compiled directly by the \LaTeX{} compiler.
%
% %%%%%%%%%%%%%%%%%%%%%%%%%%%%%%%%%%%%%%
% \paragraph{Main File.}
%
% The main file is called |cdocsamp.tex|.
%
% Load the \textsf{childdoc} definitions and
% declare the filename for the main document:
%    \begin{macrocode}
\input{childdoc.def}
\childdocmain{}
%    \end{macrocode}

% Optional override for |\version| flag:
%    \begin{macrocode}
%%\ifchilddoc\else\providecommand{\version}{draft}\fi
%    \end{macrocode}

% Define the default values for the |\version| flag
% (|final| for the main file and |draft| for childs):
%    \begin{macrocode}
\ifchilddoc
\providecommand{\version}{draft}
\else
\providecommand{\version}{final}
\fi
%    \end{macrocode}

% Load the standard document class:
%    \begin{macrocode}
\documentclass[12pt]{article}
%    \end{macrocode}

% Start the document body:
%    \begin{macrocode}
\begin{document}
%    \end{macrocode}

% Declare a title page.
% Print title, part of document being processed and version flag:
%    \begin{macrocode}
\addtocounter{page}{-1}
\begin{center}
{\LARGE\bfseries{}childdoc example\par}
\vspace{1cm}
\ifchilddoc
\ifchilddocmanual part\else chapter\fi:
`\childdocname' of `\childdocjob'\par
\else
main document: `\childdocjob'\par
\fi
version: \version\par
\end{center}
\newpage
%    \end{macrocode}

% Manually include selected file,
% otherwise process as usual:
%    \begin{macrocode}
\ifchilddocmanual
\section*{part `\childdocname'}
\input{\childdocname}
\else
%    \end{macrocode}

% Include the two chapters:
%    \begin{macrocode}
\include{cdocsch1}
\include{cdocsch2}
%    \end{macrocode}

% Include the two parts unless only chapters should be displayed:
%    \begin{macrocode}
\ifchilddoc\else
\section{part three}
\input{cdocspt3}
\section{part four}
\input{cdocspt4}
\fi
%    \end{macrocode}

% Process as usual until here:
%    \begin{macrocode}
\fi
%    \end{macrocode}

% End of document body:
%    \begin{macrocode}
\end{document}
%    \end{macrocode}
%\iffalse
%</samplemain>
%\fi
%
% %%%%%%%%%%%%%%%%%%%%%%%%%%%%%%%%%%%%%%
% \paragraph{Chapter Include Files.}
%
% The include files are called |cdocsch1.tex| and |cdocsch2.tex|.
%
%\iffalse
%<*samplechap1|samplechap2>
%\fi

% Optional override for |\version| flag:
%    \begin{macrocode}
%%\providecommand{\version}{final}
%    \end{macrocode}

% Include the main document:
%    \begin{macrocode}
\input{childdoc.def}
\childdocof{cdocsamp}
%    \end{macrocode}

%\iffalse
%</samplechap1|samplechap2>
%\fi
%
%\iffalse
%<*samplechap1>
%\fi
% Some text for chapter 1:
%    \begin{macrocode}
\section{one}
some text in chapter one
%    \end{macrocode}

%\iffalse
%</samplechap1>
%\fi
% Some text for chapter 2:
%\iffalse
%<*samplechap2>
%\fi
%    \begin{macrocode}
\section{two}
more text in chapter two
%    \end{macrocode}

%\iffalse
%</samplechap2>
%\fi
%
% %%%%%%%%%%%%%%%%%%%%%%%%%%%%%%%%%%%%%%
% \paragraph{Part Include Files.}
%
% The include files are called |cdocspt3.tex| and |cdocspt4.tex|.
%
%\iffalse
%<*samplepart3|samplepart4>
%\fi

% Optional override for |\version| flag:
%    \begin{macrocode}
%%\providecommand{\version}{final}
%    \end{macrocode}

% Include the main document:
%    \begin{macrocode}
\input{childdoc.def}
\childdocby{cdocsamp}
%    \end{macrocode}

%\iffalse
%</samplepart3|samplepart4>
%\fi
%
%\iffalse
%<*samplepart3>
%\fi
% Some text for part 3:
%    \begin{macrocode}
some text in part three
%    \end{macrocode}

%\iffalse
%</samplepart3>
%\fi
% Some text for part 4:
%\iffalse
%<*samplepart4>
%\fi
%    \begin{macrocode}
more text in part four
%    \end{macrocode}

%\iffalse
%</samplepart4>
%\fi
%
% %%%%%%%%%%%%%%%%%%%%%%%%%%%%%%%%%%%%%%
% \paragraph{Forwarding for a Complete Draft.}
%
% The following forwarding file |cdocsdrf.tex|
% compiles the main document in draft mode:
%\iffalse
%<*sampledraft>
%\fi
%    \begin{macrocode}
\def\version{draft}
\input{childdoc.def}
\childdocforward{cdocsamp}
%    \end{macrocode}

%\iffalse
%</sampledraft>
%\fi
%
% %%%%%%%%%%%%%%%%%%%%%%%%%%%%%%%%%%%%%%
% \paragraph{Forwarding for Final Version of the Chapters.}
%
% The following forwarding files |cdocsfn1.tex| and |cdocsfn2.tex|
% (with identical content)
% compile the final versions of the child documents
% |cdocsch1.tex| and |cdocsch2.tex|, respectively:
%\iffalse
%<*samplefinal>
%\fi
%    \begin{macrocode}
\def\version{final}
\input{childdoc.def}
\childdocforwardprefix[cdocsamp]{cdocsfn}{cdocsch}
%    \end{macrocode}

%\iffalse
%</samplefinal>
%\fi
%
% %%%%%%%%%%%%%%%%%%%%%%%%%%%%%%%%%%%%%%
% \paragraph{Command Line Processing.}
%
% The following three command lines generate the output files
% |cdocscld|, |cdocscl1| and |cdocscl2|
% which should be identical to
% |cdocsdrf|, |cdocsch1| and |cdocsfn2|, respectively:
% \begin{center}
% \begin{tabular}{l}
% |latex -jobname cdocscld \|\\
% |  "\def\version{draft}\input{childdoc.def}\childdocforward{cdocsamp}"|\\
% |latex -jobname cdocscl1 \|\\
% |  "\input{childdoc.def}\childdocforward[cdocsamp]{cdocsch1}"|\\
% |latex -jobname cdocscl2 \|\\
% |  "\def\version{final}\input{childdoc.def}\childdocforward{cdocsch2}"|
% \end{tabular}
% \end{center}
% Note that the trailing backslash on each first line
% merely continues the input to the second line
% (for convenient cut ant paste).
% Furthermore, the command |latex| can be replaced by any
% of its alternative versions such as |pdflatex|.
%
% %%%%%%%%%%%%%%%%%%%%%%%%%%%%%%%%%%%%%%%%%%%%%%%%%%%%%%%%%%%%%%%%%%%%%%%%%%%%%%
% %%%%%%%%%%%%%%%%%%%%%%%%%%%%%%%%%%%%%%%%%%%%%%%%%%%%%%%%%%%%%%%%%%%%%%%%%%%%%%
% \section{Implementation}
%\iffalse
%<*package>
%\fi
%
% This section describes the definitions file |childdoc.def|.

% The definitions cannot be loaded using |\usepackage| or |\RequirePackage|
% which has a mechanism to prevent loading a style file more than once.
% When loading the definitions by means of |\input|
% multiple instances have to be prevented manually:
%\iffalse
%This code needs to be before the `\ProvidesFile' directive
%which is defined at the beginning of this file.
%Therefore it is also placed there and commented out here.
%</package>
%<*discard>
%\fi
%    \begin{macrocode}
\ifdefined\childdocmain\endinput\fi
%    \end{macrocode}
%\iffalse
%</discard>
%<*package>
%\fi
%
% \macro{\ifchilddoc}
% \macro{\ifchilddocmanual}
% The conditional |\ifchilddoc| tells whether a
% child (true) or main (false) document is being compiled.
% The conditional |\ifchilddocmanual| tells whether
% the |\includeonly| mechanism is used (false) or
% the selection of child files must be performed manually (true).
% The definitions initialise to false:
%    \begin{macrocode}
\newif\ifchilddoc
\newif\ifchilddocmanual
%    \end{macrocode}

% \macro{\childdocname}
% \macro{\childdocjob}
% The macro |\childdocname| stores the name of the main document
% to be compiled. The macro |\childdocjob| stores the name of
% the document on which the \LaTeX{} compiler was originally invoked.
% The content of |\jobname| cannot be compared
% to filenames specified in the source due to different catcodes.
% The following code rescans |\jobname|, stores the result
% in |\childdocname| and saves a copy in |\childdocjob|:
%    \begin{macrocode}
\edef\childdocname{\scantokens\expandafter{\jobname\noexpand}}
\let\childdocjob\childdocname
%    \end{macrocode}

% \macro{\childdocdisable}
% The macro |\childdocdisable| prevents the main file
% from being processed more than once.
% At this stage, the main document command |\childdocmain|
% is assumed to be called once again where it should do nothing.
% Any subsequent call to it should prevent
% a secondary processing of the main document
% It overwrites the forwarding commands
% |\childdocof| and |\childdocforward|
% with empty macros to prevent further inclusions of the main document:
%    \begin{macrocode}
\newcommand{\childdocdisable}
{
  \renewcommand{\childdocmain}[1]{\renewcommand{\childdocmain}[1]{\endinput}}
  \renewcommand{\childdocof}[1]{}
  \renewcommand{\childdocby}[2][]{}
  \renewcommand{\childdocforward}[2][]{}
  \renewcommand{\childdocdisable}{}
}
%    \end{macrocode}

% \macro{\childdocmain}
% The macro |\childdocmain| is to be called at the top of the main file
% with nothing or the main filename (without extension) as argument.
% First, it breaks loops.
% If the argument is not empty and does not match |\childdocname|
% (which is set by the first inclusion of |childdoc.def|),
% |\ifchilddoc| is set to true, |\includeonly| is applied to the child file
% and |\jobname| is set to the main file
% (for proper handling of |.aux| files):
%    \begin{macrocode}
\newcommand{\childdocmain}[1]
{
  \childdocdisable\childdocmain{}
  \if?#1?\else
    \begingroup
      \def\childdoctmp{#1}
      \ifx\childdoctmp\childdocname
        \def\childdoctmp{}
      \else
        \def\childdoctmp
        {
          \childdoctrue
          \includeonly{\childdocname}
          \def\childdocjob{#1}
          \def\jobname{#1}
        }
      \fi
      \expandafter
    \endgroup
    \childdoctmp
  \fi
}
%    \end{macrocode}

% \macro{\childdocof}
% The command |\childdocof| redirects
% compilation to the main file |#1|.
%    \begin{macrocode}
\newcommand{\childdocof}[1]
{
  \childdocdisable
  \childdoctrue
  \includeonly{\childdocname}
  \def\jobname{#1}
  \def\childdocjob{#1}
  \input{#1}
}
%    \end{macrocode}

% \macro{\childdocby}
% The command |\childdocby| ....
%    \begin{macrocode}
\newcommand{\childdocby}[2][]
{
  \childdocdisable
  \childdoctrue
  \childdocmanualtrue
  \if?#1?\else
    \def\jobname{#2}
  \fi
  \def\childdocjob{#2}
  \input{#2}
  \endinput
}
%    \end{macrocode}

% \macro{\childdocforward}
% The command |\childdocforward| redirects
% compilation to the main file or
% (if the optional argument is given) a child file.
% Parameters are set as if the main file
% or a child file starting with |\childdocof| was compiled.
% Then compilation is handed over to the main file:
%    \begin{macrocode}
\newcommand{\childdocforward}[2][]
{
  \begingroup
    \if?#1?
      \def\childdoctmp
      {
        \def\childdocname{#2}
        \def\childdocjob{#2}
        \def\jobname{#2}
        \input{#2}
        \endinput
      }
    \else
      \def\childdoctmp
      {
        \childdocdisable
        \def\childdocname{#2}
        \childdoctrue
        \includeonly{#2}
        \def\childdocjob{#1}
        \def\jobname{#1}
        \input{#1}
        \endinput
      }
    \fi
    \expandafter
  \endgroup
  \childdoctmp
}
%    \end{macrocode}

% \macro{\childdocforwardprefix}
% The command |\childdocforwardprefix| redirects
% compilation to the main or a child file by means of a pattern.
% The prefix |#1| in the current filename is replaced by |#2|
% and the suffix of the current filename is kept
% (it is assumed that the filename does not contain the substring `|~~~|'
% which is used as a delimiter).
% Compilation is handed over to the new file by |\childdocforward|:
%    \begin{macrocode}
\newcommand{\childdocforwardprefix}[3][]
{
  \begingroup
    \def\childdocextract #2##1~~~{\def\childdoctmp{\childdocforward[#1]{#3##1}}}
    \expandafter\childdocextract\childdocname~~~
    \expandafter
  \endgroup
  \childdoctmp
}
%    \end{macrocode}

% \macro{\childdoc}
% The deprecated macro |\childdoc| is a legacy version of |\childdocmain|:
%    \begin{macrocode}
\newcommand{\childdoc}{\childdocmain}
%    \end{macrocode}

% \macro{\childdocredirect}
% The deprecated macro |\childdocredirect| is a legacy version
% of |\childdocforward| and |\childdocforwardprefix|:
%    \begin{macrocode}
\newcommand{\childdocredirect}[2][]
{
  \begingroup
    \if?#1?
      \def\childdoctmp{\childdocforward{#2}}
    \else
      \def\childdoctmp{\childdocforwardprefix{#1}{#2}}
    \fi
    \expandafter
  \endgroup
  \childdoctmp
}
%    \end{macrocode}

%\iffalse
%</package>
%\fi
%
\endinput
|\\
|\childdocforward{|\textit{main}|}|\\
\end{tabular}
\end{center}
%
or alternatively with:
%
\begin{center}
\begin{tabular}{l}
|% \iffalse
%
% childdoc.dtx Copyright (C) 2017-2018 Niklas Beisert
%
% This work may be distributed and/or modified under the
% conditions of the LaTeX Project Public License, either version 1.3
% of this license or (at your option) any later version.
% The latest version of this license is in
%   http://www.latex-project.org/lppl.txt
% and version 1.3 or later is part of all distributions of LaTeX
% version 2005/12/01 or later.
%
% This work has the LPPL maintenance status `maintained'.
%
% The Current Maintainer of this work is Niklas Beisert.
%
% This work consists of the files childdoc.dtx and childdoc.ins
% and the derived files childdoc.def and cdocsamp.tex with
% cdocsch1.tex, cdocsch2.tex, cdocsdrf.tex, cdocsfn1.tex, cdocsfn2.tex.
%
%<package>\ifdefined\childdocmain\endinput\fi
%<package>\ProvidesFile{childdoc.def}[2018/12/30 v2.0 child document driver]
%<samplemain>\ProvidesFile{cdocsamp.tex}[2018/12/30 v2.0 sample for childdoc]
%<*driver>
%\ProvidesFile{childdoc.drv}[2018/12/30 v2.0 childdoc reference manual file]
\PassOptionsToClass{10pt,a4paper}{article}
\documentclass{ltxdoc}

\usepackage[margin=35mm]{geometry}
\usepackage{hyperref}
\usepackage{hyperxmp}
\usepackage[usenames]{color}

\hypersetup{colorlinks=true}
\hypersetup{pdfstartview=FitH}
\hypersetup{pdfpagemode=UseNone}
\hypersetup{pdfsource={}}
\hypersetup{pdflang={en-UK}}
\hypersetup{pdfcopyright={Copyright 2017-2018 Niklas Beisert.
  This work may be distributed and/or modified under the
  conditions of the LaTeX Project Public License, either version 1.3
  of this license or (at your option) any later version.}}
\hypersetup{pdflicenseurl={http://www.latex-project.org/lppl.txt}}
\hypersetup{pdfcontactaddress={ETH Zurich, ITP, HIT K,
  Wolfgang-Pauli-Strasse 27}}
\hypersetup{pdfcontactpostcode={8093}}
\hypersetup{pdfcontactcity={Zurich}}
\hypersetup{pdfcontactcountry={Switzerland}}
\hypersetup{pdfcontactemail={nbeisert@itp.phys.ethz.ch}}
\hypersetup{pdfcontacturl={http://people.phys.ethz.ch/\xmptilde nbeisert/}}

\newcommand{\secref}[1]{\hyperref[#1]{section \ref*{#1}}}

\parskip1ex
\parindent0pt
\let\olditemize\itemize
\def\itemize{\olditemize\parskip0pt}

\begin{document}

\title{The \textsf{childdoc} Package}
\hypersetup{pdftitle={The childdoc Package}}
\author{Niklas Beisert\\[2ex]
  Institut f\"ur Theoretische Physik\\
  Eidgen\"ossische Technische Hochschule Z\"urich\\
  Wolfgang-Pauli-Strasse 27, 8093 Z\"urich, Switzerland\\[1ex]
  \href{mailto:nbeisert@itp.phys.ethz.ch}
  {\texttt{nbeisert@itp.phys.ethz.ch}}}
\hypersetup{pdfauthor={Niklas Beisert}}
\hypersetup{pdfsubject={Manual for the LaTeX2e Package childdoc}}
\date{30 December 2018, \textsf{v2.0}}
\maketitle

\begin{abstract}\noindent
\textsf{childdoc} is a \LaTeXe{} package
that enables the direct compilation
of document sections included by |\include|
to individual files.
\end{abstract}

\begingroup
\parskip0ex
\tableofcontents
\endgroup

%%%%%%%%%%%%%%%%%%%%%%%%%%%%%%%%%%%%%%%%%%%%%%%%%%%%%%%%%%%%%%%%%%%%%%%%%%%%%%%%
%%%%%%%%%%%%%%%%%%%%%%%%%%%%%%%%%%%%%%%%%%%%%%%%%%%%%%%%%%%%%%%%%%%%%%%%%%%%%%%%
\section{Introduction}

\LaTeX{} provides a mechanism to structure a large document (such as a book)
into a main file and several child files (containing the chapters)
using the |\include| command.
This mechanism is beneficial for documents
which span hundreds of pages in order to
make the source file(s) more manageable.
Moreover, compilation can be restricted to
selected child files by means of the |\includeonly| command.
The latter feature can be used to reduce the compilation time while editing
(this was significantly more useful in the earlier days of \LaTeX{})
or to generate a smaller document which is easier to navigate.
Another application of |\includeonly| is to generate
documents consisting of selected parts of the complete document.

However, there are a few drawbacks of the plain |\include| mechanism:
\begin{itemize}
\item
The child files cannot be compiled on their own,
they can only be compiled via the main file.
A naive editing environment
(such as a text editor with an option
to have the current file processed by \LaTeX)
may require one to switch to the main file before compiling;
attempting to compile the child file produces errors.
\item
The main file must be modified (each time)
to adjust the |\includeonly| command
to the present needs. This easily leaves the main file in a messy state.
\item
The generated document will always carry the filename
of the main document. This is inconvenient if
several child files are to be compiled and
to be kept for distribution.
\end{itemize}

The present package provides a simple interface
to make child files individually compilable by \LaTeX{}.
Compiling a child file then has the same effect as compiling
the main file with an |\includeonly| command
to select the appropriate child.
Moreover the generated document will carry the name of the child
rather than the main file.
This resolves all three above issues.

This feature is meant to make the editing of books,
thesis documents and lecture notes somewhat more convenient.
However, the package can also be used efficiently for
composing a series of documents (such as exercise sheets)
which are typically distributed individually.
It then assists the author in generating the individual documents
(potentially in different versions)
as well as a document containing the collected series.
Another application is in developing style files
or other kinds of included material
where compilation of the style file could redirect
to a sample or test file.

%%%%%%%%%%%%%%%%%%%%%%%%%%%%%%%%%%%%%%%%%%%%%%%%%%%%%%%%%%%%%%%%%%%%%%%%%%%%%%%%
%%%%%%%%%%%%%%%%%%%%%%%%%%%%%%%%%%%%%%%%%%%%%%%%%%%%%%%%%%%%%%%%%%%%%%%%%%%%%%%%
\section{Usage}

First of all, the package \textsf{childdoc} is \emph{not} a standard
\LaTeXe{} |.sty| style file! Therefore it needs to be invoked in
a non-standard way.

%%%%%%%%%%%%%%%%%%%%%%%%%%%%%%%%%%%%%%%%%%%%%%%%%%%%%%%%%%%%%%%%%%%%%%%%%%%%%%%%
\subsection{Included Files}
\label{sec:include}

%%%%%%%%%%%%%%%%%%%%%%%%%%%%%%%%%%%%%%%%
\DescribeMacro{\childdocmain}
To use the package, add the commands
\begin{center}
\begin{tabular}{l}
|\input{childdoc.def}|\\
|\childdocmain{}|\\
\end{tabular}
\end{center}
at the very top of the main \LaTeX{} file,
in particular \emph{before} the |\documentclass| statement!
The argument of |\childdocmain| should be left empty
(but it must be present).

%%%%%%%%%%%%%%%%%%%%%%%%%%%%%%%%%%%%%%%%
\DescribeMacro{\childdocof}
Furthermore, add the commands
\begin{center}
\begin{tabular}{l}
|\input{childdoc.def}|\\
|\childdocof{|\textit{main}|}|\\
\end{tabular}
\end{center}
at the top of every child file \textit{child}
which is included by |\include{|\textit{child}|}|
from within the main file
(or at least for those files to be compiled individually).
The argument \textit{main} must be the filename of the main file.

There are a couple of
considerations in setting up the main and child documents:

%%%%%%%%%%%%%%%%%%%%%%%%%%%%%%%%%%%%%%%%
\paragraph{Restrictions.}

Please note the following restrictions:
\begin{itemize}
\item
|\childdocmain| must be called with one argument \textit{main}
to ensure compatibility with earlier version of the package.
It must either be empty (|\childdocmain{}|)
or precisely match the filename of the main file in which it is specified.
See \secref{sec:detection} for further information.
\item
The filename \textit{main} must be specified without the |.tex| extension.
\item
The filename \textit{main} is case sensitive
(even in case-insensitive file systems)
due to internal string comparison.
\item
The argument \textit{main} should be fully expanded, it cannot be a macro.
\item
Subdirectories and special characters should be avoided in filenames.
\item
The command |\childdocmain{|\textit{main}|}| must be followed by a whitespace.
It should not be followed immediately by another command
or by a comment mark `|%|'.
This is because the \TeX{} parser reads the token immediately following
the argument of |\childdocmain| and puts it
at the beginning of every child section;
however, a white\-space is ignored.
\end{itemize}

%%%%%%%%%%%%%%%%%%%%%%%%%%%%%%%%%%%%%%%%
\paragraph{Content of Main File.}

It is advisable to place all content in the child files included by |\include|.
Any output contained in the main file will appear in all child documents
unless suppressed manually;
it cannot be suppressed automatically by the |\includeonly| directive
and thus should normally be avoided.
A method to include some content in the main file
by means of conditional processing is described in \secref{sec:conditional}.

%%%%%%%%%%%%%%%%%%%%%%%%%%%%%%%%%%%%%%%%
\paragraph{Page Numbering.}

When only a part of the document is compiled,
the appropriate numbering of pages
(as well as other status parameters)
is determined from the |.aux| files.
The latter contain information from previous passes.
However this information needs to propagate through
all intermediate child documents.
Therefore the page numbering in child documents may well
be inconsistent until the complete document is compiled at least once.

A useful (if unconventional) way to always ensure a consistent
page numbering is to restart the numbering in each child document
and denote the pages by `\textit{child}|.|\textit{page}'
where \textit{child} represents the chapter/section number of the child file.
This can be achieved by the command
|\numberwithin{page}{|\textit{child}|}|
of the \textsf{amsmath} package
where \textit{child} can be |chapter| or |section|
depending on the chosen structuring.
Alternatively, one can modify the macro |\thepage| appropriately
and reset the counter |page| at the start of each child file.

%%%%%%%%%%%%%%%%%%%%%%%%%%%%%%%%%%%%%%%%%%%%%%%%%%%%%%%%%%%%%%%%%%%%%%%%%%%%%%%%
\subsection{Conditional Processing}
\label{sec:conditional}

The package provides a mechanism to compile different versions
of a document. To customise the versions further some conditional processing
can come in handy to distinguish which version is being compiled.
The package provides two macros to describe the compilation context:

%%%%%%%%%%%%%%%%%%%%%%%%%%%%%%%%%%%%%%%%
\DescribeMacro{\ifchilddoc}
The conditional |\ifchilddoc| distinguishes between the compilation of
child documents and the main document:
%
\begin{center}
|\ifchilddoc |\textit{child-code}| |[|\||else |\textit{main-code}]| \||fi|
\end{center}

%%%%%%%%%%%%%%%%%%%%%%%%%%%%%%%%%%%%%%%%
\DescribeMacro{\childdocname}
\DescribeMacro{\childdocjob}
The macro |\childdocname| contains the filename (without extension)
of the main or child file being processed.
Note that |\childdocjob| will always contain the name of the main file.

%%%%%%%%%%%%%%%%%%%%%%%%%%%%%%%%%%%%%%%%
\paragraph{Title Page.}

Conditional processing can be used to include a title or banner page
in the main document when proper precautions are taken.
Importantly, the code in the main file should ensure that the page counter
(as well as other status parameters which are stored in the |.aux| files)
takes the same value after the conditional processing.
Otherwise the page numbers may take divergent values
depending on which part is compiled.

For example, a title page could be declared by:
%
\begin{center}
\begin{tabular}{l}
|\ifchilddoc\||else|\\
|\addtocounter{page}{-1}|\\
\textit{code for title page}\\
|\newpage|\\
|\||fi|
\end{tabular}
\end{center}
%
A banner page for the child documents can be generated by:
%
\begin{center}
\begin{tabular}{l}
|\ifchilddoc|\\
|\addtocounter{page}{-1}|\\
\textit{code for banner page}\\
|\newpage|\\
|\||fi|
\end{tabular}
\end{center}
%
Here one could write a message such as:
\begin{center}
|This is the part \childdocname{} of \childdocjob{}.|
\end{center}

%%%%%%%%%%%%%%%%%%%%%%%%%%%%%%%%%%%%%%%%%%%%%%%%%%%%%%%%%%%%%%%%%%%%%%%%%%%%%%%%
\subsection{Flags}
\label{sec:flags}

The package makes it easy to generate different versions
of the main or child documents.
To this end compilation flags can be defined
and assigned different default values.
They will be particularly useful in conjunction
with the forwarding mechanism described in \secref{sec:forward}.

For example, it may be useful to have a flag |\version|
which can be set to |draft| or |final|.
The document source will contain some conditional code
depending on the value of |\version|.
Suppose further, the flag should default to |final| for the main file
and to |draft| for child files
which is a natural assignment for editing the document.
This is achieved by placing the following code
in the preamble of the main document
(below the |\childdocmain| directive):
%
\begin{center}
\begin{tabular}{l}
|\ifchilddoc|\\
|\providecommand{\version}{draft}|\\
|\||else|\\
|\providecommand{\version}{final}|\\
|\||fi|
\end{tabular}
\end{center}
%
The definition by |\providecommand| makes sure
that previous definitions are not overwritten.
Further statements |\providecommand{\version}{...}|
can thus be added before the above code to override it.

For the main file, one might add a line
(between |\childdocmain| and the above block)
%
\begin{center}
|%\ifchilddoc\||else\providecommand{\version}{draft}\||fi|
\end{center}
%
which can be uncommented to produce a draft version.
Likewise one can add a line to the very top of a child file
(above the |\childdocof{|\textit{main}|}| directive)
%
\begin{center}
|%\providecommand{\version}{final}|
\end{center}
%
which can be uncommented to produce the final version of this child document.

%%%%%%%%%%%%%%%%%%%%%%%%%%%%%%%%%%%%%%%%%%%%%%%%%%%%%%%%%%%%%%%%%%%%%%%%%%%%%%%%
\subsection{Forwarding}
\label{sec:forward}

Different versions of the main or child documents
using compilation flags as described in \secref{sec:flags}
can be (permanently) stored in different files
for convenient compilation, viewing and distribution.
To this end, the package defines a command
to pass on compilation to a different file:

%%%%%%%%%%%%%%%%%%%%%%%%%%%%%%%%%%%%%%%%
\DescribeMacro{\childdocforward}
The command |\childdocforward| redirects processing to
another source file:
%
\begin{center}
\begin{tabular}{l}
|\input{childdoc.def}|\\
|\childdocforward[|\textit{main}|]{|\textit{dest}|}|\\
\end{tabular}
\end{center}
%
The argument \textit{dest} is the destination file
(without extension).
It should be the main file or one of the child files.
Note that further \textsf{childdoc} directives
such as |\childdocof| and |\childdocforward|
in the indicated file will be processed in this form.
The optional argument \textit{main}
passes on directly to the main file \textit{main}
while pretending to compile the child \textit{dest}.
This form behaves as if \textit{dest}
issues |\childdocof{|\textit{main}|}| right away,
and no further \textsf{childdoc} directives will be processed.

%%%%%%%%%%%%%%%%%%%%%%%%%%%%%%%%%%%%%%%%
\DescribeMacro{\...prefix}
In the alternative form |\childdocforwardprefix|,
%
\begin{center}
\begin{tabular}{l}
|\input{childdoc.def}|\\
|\childdocforwardprefix[|\textit{main}|]{|\textit{prefix}|}{|\textit{dest}|}|
\end{tabular}
\end{center}
%
the destination file is determined by a pattern
depending on the current file:
To make this work, the current file must be called
`{\textit{prefix}\hspace{0.2em}\textit{suffix}}'
with \textit{prefix} matching precisely the argument.
Processing is then passed on to the file
`{\textit{dest}\hspace{0.2em}\textit{suffix}}'.
Surely, the same effect is achieved by
directly specifying the
argument `{\textit{dest}\hspace{0.2em}\textit{suffix}}'
in the first form.
However, that requires to set up a different file
for each child. With the alternative form of the command
all these files can have exactly the same content
which simplifies setting them up and maintaining them.

For example, the following file |draft.tex|
with a compilation flag |\version| as described in \secref{sec:flags}
compiles the main document as a draft:
%
\begin{center}
\begin{tabular}{l}
|\def\version{draft}|\\
|\input{childdoc.def}|\\
|\childdocforward{|\textit{main}|}|
\end{tabular}
\end{center}
%
Likewise, the following files |final|\textit{nn}|.tex|
compile the final version of the child document
|child|\textit{nn}|.tex|:
%
\begin{center}
\begin{tabular}{l}
|\def\version{final}|\\
|\input{childdoc.def}|\\
|\childdocforwardprefix{final}{child}|
\end{tabular}
\end{center}
%

Note that when several versions of a main file and/or of each child file
are to be generated, it may be convenient to set up a |Makefile| or
shell script to automatise the process.

%%%%%%%%%%%%%%%%%%%%%%%%%%%%%%%%%%%%%%%%%%%%%%%%%%%%%%%%%%%%%%%%%%%%%%%%%%%%%%%%
\subsection{Command Line Processing}
\label{sec:commandline}

The effect of redirection files can also be achieved by invoking
the \LaTeX{} compiler with a more elaborate command line.
Most conveniently this should be done as part
of a shell script or a |Makefile|.

When using \textsf{childdoc} in the main file, the following
command lines effectively perform a redirection
(note that depending on the shell being used,
backslashes may have to be doubled: `|\|' $\to$ `|\\|'):
%
\begin{center}
|... -jobname "|\textit{target}|" |\\|"|[\textit{flags}]%
|\input{childdoc.def}\childdocforward[|\textit{main}|]{|\textit{dest}|}"|
\end{center}
%
Here \textit{target} is the name of the output file,
\textit{main} is the name of the main file
and \textit{dest} is the name of the main or child file to be processed
(all filenames without extensions).
The optional argument \textit{main} can be omitted
if \textit{main} matches \textit{dest}.
Optionally, compilation \textit{flags} can be defined via |\def| commands.
This command line makes the \TeX{} engine believe
it is compiling the file \textit{target}
whose content is specified as the latter parameter.
The provided code then forwards the processing to
\textit{main} or \textit{dest} as described in \secref{sec:forward}.

%%%%%%%%%%%%%%%%%%%%%%%%%%%%%%%%%%%%%%%%%%%%%%%%%%%%%%%%%%%%%%%%%%%%%%%%%%%%%%%%
\subsection{Include by Input}
\label{sec:input}

Including child documents by |\include| has some restrictions by design.
Most notably, the content of a child document always occupies
its own set of pages; pages cannot be shared between child documents.
Usually, this behaviour makes perfect sense
because each child document contain an essential part of the document.
However, in some situations it may be desirable to compose
a document from a collection of parts
without having mandatory page breaks between then.
For this case, the package
provides a mechanism to include parts
by |\input| which can also be processed individually.
However, by construction this mechanism
requires manual handling of the content to be output.

%%%%%%%%%%%%%%%%%%%%%%%%%%%%%%%%%%%%%%%%
\DescribeMacro{\ifchilddocmanual}
The main file should be prepared as usual, see \secref{sec:include}.
However, the document body must make a distinction
between processing of an individual part and of the main document, e.g.:
%
\begin{center}
\begin{tabular}{l}
|\ifchilddocmanual|\\
|\input{\childdocname}|\\
|\||else|\\
\textit{document body with }|\input{|\textit{part}|}|\\
|\||fi|
\end{tabular}
\end{center}
%
The conditional |\ifchilddocmanual| is true whenever
a part to be included by |\input| is being compiled,
and the name of the part is stored in |\childdocname|.

%%%%%%%%%%%%%%%%%%%%%%%%%%%%%%%%%%%%%%%%
\DescribeMacro{\childdocby}
Each part to be included by |\input| should start with:
%
\begin{center}
\begin{tabular}{l}
|\input{childdoc.def}|\\
|\childdocby{|\textit{main}|}|\\
\end{tabular}
\end{center}
%
The directive |\childdocby| is similar to |\childdocof|
described in \secref{sec:include},
but the subsequent selection of content must be done manually.
To that end, both |\ifchilddoc| and |\ifchilddocmanual|
will be true upon processing of a part,
and the name of the part is stored in |\childdocname|.
Note that |\jobname| will be set to the filename of the current part
so that each part receives an individual |.aux| file
that does not interfere with the |.aux| file(s) of the main document.
This behaviour can be altered by the alternative form
|\childdocby[*]{|\textit{main}|}| (with a non-empty optional argument)
which uses the |.aux| file of the main document
by setting |\jobname| to \textit{main}.

%%%%%%%%%%%%%%%%%%%%%%%%%%%%%%%%%%%%%%%%%%%%%%%%%%%%%%%%%%%%%%%%%%%%%%%%%%%%%%%%
\subsection{Driver Development}
\label{sec:driver}

The \textsf{childdoc} mechanism can also be use for the development
of definition files such as \LaTeX{} styles or classes.
This case differs from the above setup with multiple parts
included by |\include| in that no |\includeonly| should be invoked.
This can be achieved by starting the include file
(before |\ProvidesPackage|) with:
%
\begin{center}
\begin{tabular}{l}
|\input{childdoc.def}|\\
|\childdocforward{|\textit{main}|}|\\
\end{tabular}
\end{center}
%
or alternatively with:
%
\begin{center}
\begin{tabular}{l}
|\input{childdoc.def}|\\
|\childdocby{|\textit{main}|}|\\
\end{tabular}
\end{center}
%
Both forms have slightly different effects as described above.
The main file is prepared as usual, see \secref{sec:include}.

%%%%%%%%%%%%%%%%%%%%%%%%%%%%%%%%%%%%%%%%%%%%%%%%%%%%%%%%%%%%%%%%%%%%%%%%%%%%%%%%
\subsection{Legacy Detection}
\label{sec:detection}

The directive |\childdocmain| in the main file can detect
whether the complete document or merely a child is to be compiled
even without using the directive |\childdocof|.
This method is deprecated because it is less robust
and there is no compelling reason to use it;
it is merely provided for backward compatibility
and it may be removed in future versions.

If the detection mechanism is to be used,
it is mandatory to correctly specify
the filename of the main file as the argument of |\childdocmain|:
%
\begin{center}
\begin{tabular}{l}
|\input{childdoc.def}|\\
|\childdocmain{|\textit{main}|}|\\
\end{tabular}
\end{center}
%
If |\jobname| does not match the argument \textit{main} of |\childdocmain|,
it is assumed that |\jobname| points to the child file to be compiled.
When using |\childdocmain| with the main file specified as argument,
it suffices to start a child file
with just |\input{|\textit{main}|}|
without loading of the package and using |\childdocof|.
If instead all processing is done
with the appropriate \textsf{childdoc} directives,
the argument of \textit{main} of |\childdocmain| can be empty.

An alternative version of the command line processing described
in \secref{sec:commandline} using the detection mechanism reads:
%
\begin{center}
|... -jobname "|\textit{target}|" "|[\textit{flags}]%
[|\def\jobname{|\textit{dest}|}|]|\input{|\textit{main}|}"|
\end{center}

%%%%%%%%%%%%%%%%%%%%%%%%%%%%%%%%%%%%%%%%%%%%%%%%%%%%%%%%%%%%%%%%%%%%%%%%%%%%%%%%
\subsection{Manual Code}
\label{sec:manual}

In case one cannot be certain whether the definitions file |childdoc.def|
is installed on the target \TeX{} distribution
and one prefers not to ship it,
it is conceivable to paste a few relevant commands into the sources.

To that end, drop all statements |\input{childdoc.def}|
and perform the replacements as outlined below.
Instead of |\childdocmain{|\textit{main}|}| add the following code
to the top of the main file:
%
\begin{center}
\begin{tabular}{l}
|\||ifdefined\childdocname\endinput\||fi\newif\ifchilddoc|\\
|\edef\childdocname{\scantokens\expandafter{\jobname\noexpand}}|\\
|\def\childdocmain{|\textit{main}|}\||ifx\childdocmain\childdocname\||else|\\
|\childdoctrue\includeonly{\childdocname}\let\jobname\childdocmain\||fi|\\
\end{tabular}
\end{center}
%
Instead of |\childdocof{|\textit{main}|}| just include the main file
at the top of each child file:
%
\begin{center}
|\input{|\textit{main}|}|
\end{center}
%
A simple redirection |\childdocforward{|\textit{dest}|}| is achieved by:
%
\begin{center}
|\def\jobname{|\textit{dest}|}\input{\jobname}|
\end{center}
%
The redirection with prefix
|\childdocforwardprefix[|\textit{prefix}|]{|\textit{dest}|}|
is accomplished by:
%
\begin{center}
\begin{tabular}{l}
|{\edef\jobname{\scantokens\expandafter{\jobname\noexpand}}|\\
|\def\redirectjob |\textit{prefix}|#1~~~{\gdef\jobname{|\textit{dest}|#1}}|\\
|\expandafter\redirectjob\jobname~~~}\input{\jobname}|
\end{tabular}
\end{center}

In an alternative approach,
child documents can be compiled by a specific command line
without additional code or specific definitions:
%
\begin{center}
|... -jobname "|\textit{target}|" "|[\textit{flags}]%
|\includeonly{|\textit{dest}|}\input{|\textit{main}|}"|
\end{center}
%

%%%%%%%%%%%%%%%%%%%%%%%%%%%%%%%%%%%%%%%%%%%%%%%%%%%%%%%%%%%%%%%%%%%%%%%%%%%%%%%%
%%%%%%%%%%%%%%%%%%%%%%%%%%%%%%%%%%%%%%%%%%%%%%%%%%%%%%%%%%%%%%%%%%%%%%%%%%%%%%%%
\section{Information}

%%%%%%%%%%%%%%%%%%%%%%%%%%%%%%%%%%%%%%%%%%%%%%%%%%%%%%%%%%%%%%%%%%%%%%%%%%%%%%%%
\subsection{Copyright}

Copyright \copyright{} 2017--2018 Niklas Beisert

This work may be distributed and/or modified under the
conditions of the \LaTeX{} Project Public License, either version 1.3
of this license or (at your option) any later version.
The latest version of this license is in
  \url{http://www.latex-project.org/lppl.txt}
and version 1.3 or later is part of all distributions of \LaTeX{}
version 2005/12/01 or later.

This work has the LPPL maintenance status `maintained'.

The Current Maintainer of this work is Niklas Beisert.

This work consists of the files |README.txt|, |childdoc.ins| and |childdoc.dtx|
as well as the derived files |childdoc.def|, |cdocsamp.tex|
with |cdocsch1.tex|, |cdocsch2.tex|, |cdocspt3.tex|, |cdocspt4.tex|,
|cdocsdrf.tex|, |cdocsfn1.tex|, |cdocsfn2.tex|
as well as |childdoc.pdf|.

%%%%%%%%%%%%%%%%%%%%%%%%%%%%%%%%%%%%%%%%%%%%%%%%%%%%%%%%%%%%%%%%%%%%%%%%%%%%%%%%
\subsection{Files and Installation}

The package consists of the files:
%
\begin{center}
\begin{tabular}{ll}
    |README.txt|   & readme file \\
    |childdoc.ins| & installation file \\
    |childdoc.dtx| & source file \\
    |childdoc.def| & definition file \\
    |cdocsamp.tex| & sample main file \\
    |cdocsch1.tex| & sample include file \\
    |cdocsch2.tex| & sample include file \\
    |cdocspt3.tex| & sample part file \\
    |cdocspt4.tex| & sample part file \\
    |cdocsdrf.tex| & sample redirection file \\
    |cdocsfn1.tex| & sample redirection file \\
    |cdocsfn2.tex| & sample redirection file \\
    |childdoc.pdf| & manual
\end{tabular}
\end{center}
%
The distribution consists of the files
|README.txt|, |childdoc.ins| and |childdoc.dtx|.
%
\begin{itemize}
\item
Run (pdf)\LaTeX{} on |childdoc.dtx|
to compile the manual |childdoc.pdf| (this file).
\item
Run \LaTeX{} on |childdoc.ins| to create the definitions file |childdoc.def|
and the sample |cdocsamp.tex| with include files
|cdocsch1.tex|, |cdocsch2.tex|, |cdocspt3.tex|, |cdocspt4.tex|,
|cdocsdrf.tex|, |cdocsfn1.tex|, |cdocsfn2.tex|.
Then copy the file |childdoc.def| to an appropriate directory of your \LaTeX{}
distribution, e.g.\ \textit{texmf-root}|/tex/latex/childdoc|.
\end{itemize}

%%%%%%%%%%%%%%%%%%%%%%%%%%%%%%%%%%%%%%%%%%%%%%%%%%%%%%%%%%%%%%%%%%%%%%%%%%%%%%%%
\subsection{Related CTAN Packages}

There are several other packages which offer a similar functionality:
%
\begin{itemize}
\item
The packages
\href{http://ctan.org/pkg/docmute}{\textsf{docmute}},
\href{http://ctan.org/pkg/includex}{\textsf{includex}} and
\href{http://ctan.org/pkg/standalone}{\textsf{standalone}}
provide commands to include only the document body of
a child file thus allowing both files to be compiled individually.
\item
The packages \href{http://ctan.org/pkg/subdocs}{\textsf{subdocs}}
and \href{http://ctan.org/pkg/subfiles}{\textsf{subfiles}}
provide structures in which the main and child documents can be
encapsulated and allowing them to be compiled individually.
The inclusion mechanism is different from the conventional |\include|.
\item
The package \href{http://ctan.org/pkg/combine}{\textsf{combine}}
is an elaborate solution to combine several documents into one.
\end{itemize}
%
See also the CTAN topic \href{http://ctan.org/topic/subdocs}{\textsf{subdocs}}
for further related packages.
The present package differs from the above solutions in that
a document structure constructed with the conventional |\include| mechanism
just needs two extra commands at the top of every file
such that all constituent files can be compiled individually.

%%%%%%%%%%%%%%%%%%%%%%%%%%%%%%%%%%%%%%%%%%%%%%%%%%%%%%%%%%%%%%%%%%%%%%%%%%%%%%%%
%\subsection{Feature Suggestions}
%
%The following is a list of features which may be useful for future
%versions of this package:
%%
%\begin{itemize}
%\item
%\ldots
%\end{itemize}

%%%%%%%%%%%%%%%%%%%%%%%%%%%%%%%%%%%%%%%%%%%%%%%%%%%%%%%%%%%%%%%%%%%%%%%%%%%%%%%%
\subsection{Revision History}

%%%%%%%%%%%%%%%%%%%%%%%%%%%%%%%%%%%%%%%%
\paragraph{v2.0:} 2018/12/30

\begin{itemize}
\item
immediate forward processing
\item
added |\childdocby| mechanism
\item
manual restructured
\end{itemize}

%%%%%%%%%%%%%%%%%%%%%%%%%%%%%%%%%%%%%%%%
\paragraph{v1.6:} 2018/01/17

\begin{itemize}
\item
application for development of include files
\item
corrections to manual
\end{itemize}

%%%%%%%%%%%%%%%%%%%%%%%%%%%%%%%%%%%%%%%%
\paragraph{v1.5:} 2017/05/21

\begin{itemize}
\item
more complete structuring introduced
\item
|\childdocof| introduced
\item
|\childdoc| renamed to |\childdocmain|
\item
|\childredirect| renamed to |\childdocforward| and |\childdocforwardprefix|
and functionality expanded
\end{itemize}

%%%%%%%%%%%%%%%%%%%%%%%%%%%%%%%%%%%%%%%%
\paragraph{v1.0:} 2017/04/27

\begin{itemize}
\item
manual and install package
\item
first version published on CTAN
\end{itemize}

%%%%%%%%%%%%%%%%%%%%%%%%%%%%%%%%%%%%%%%%
\paragraph{v0.6:} 2017/04/26

\begin{itemize}
\item
redirection mechanism added
\end{itemize}

%%%%%%%%%%%%%%%%%%%%%%%%%%%%%%%%%%%%%%%%
\paragraph{v0.5:} 2017/04/26

\begin{itemize}
\item
functionality in definition file
\end{itemize}


%%%%%%%%%%%%%%%%%%%%%%%%%%%%%%%%%%%%%%%%%%%%%%%%%%%%%%%%%%%%%%%%%%%%%%%%%%%%%%%%
%%%%%%%%%%%%%%%%%%%%%%%%%%%%%%%%%%%%%%%%%%%%%%%%%%%%%%%%%%%%%%%%%%%%%%%%%%%%%%%%
%%%%%%%%%%%%%%%%%%%%%%%%%%%%%%%%%%%%%%%%%%%%%%%%%%%%%%%%%%%%%%%%%%%%%%%%%%%%%%%%
\appendix

\settowidth\MacroIndent{\rmfamily\scriptsize 000\ }

 \DocInput{childdoc.dtx}

\end{document}
%</driver>
% \fi
%
% %%%%%%%%%%%%%%%%%%%%%%%%%%%%%%%%%%%%%%%%%%%%%%%%%%%%%%%%%%%%%%%%%%%%%%%%%%%%%%
% %%%%%%%%%%%%%%%%%%%%%%%%%%%%%%%%%%%%%%%%%%%%%%%%%%%%%%%%%%%%%%%%%%%%%%%%%%%%%%
% \section{Sample}
%\iffalse
%<*samplemain>
%\fi
%
% The following presents a sample document
% with two chapters, two parts, a title page,
% a compile flag as well as three forwarding files to set the flag.
% It consists of eight |.tex| files:
% \begin{center}
% \begin{tabular}{ll}
% |cdocsamp.tex|&main file\\
% |cdocsch1.tex|&include file for chapter 1\\
% |cdocsch2.tex|&include file for chapter 2\\
% |cdocspt3.tex|&include file for part 3\\
% |cdocspt4.tex|&include file for part 4\\
% |cdocsdrf.tex|&forwarding file for main file in draft mode\\
% |cdocsfi1.tex|&forwarding file for final version of chapter 1\\
% |cdocsfi2.tex|&forwarding file for final version of chapter 2\\
% \end{tabular}
% \end{center}
% Each of the eight files can be compiled directly by the \LaTeX{} compiler.
%
% %%%%%%%%%%%%%%%%%%%%%%%%%%%%%%%%%%%%%%
% \paragraph{Main File.}
%
% The main file is called |cdocsamp.tex|.
%
% Load the \textsf{childdoc} definitions and
% declare the filename for the main document:
%    \begin{macrocode}
\input{childdoc.def}
\childdocmain{}
%    \end{macrocode}

% Optional override for |\version| flag:
%    \begin{macrocode}
%%\ifchilddoc\else\providecommand{\version}{draft}\fi
%    \end{macrocode}

% Define the default values for the |\version| flag
% (|final| for the main file and |draft| for childs):
%    \begin{macrocode}
\ifchilddoc
\providecommand{\version}{draft}
\else
\providecommand{\version}{final}
\fi
%    \end{macrocode}

% Load the standard document class:
%    \begin{macrocode}
\documentclass[12pt]{article}
%    \end{macrocode}

% Start the document body:
%    \begin{macrocode}
\begin{document}
%    \end{macrocode}

% Declare a title page.
% Print title, part of document being processed and version flag:
%    \begin{macrocode}
\addtocounter{page}{-1}
\begin{center}
{\LARGE\bfseries{}childdoc example\par}
\vspace{1cm}
\ifchilddoc
\ifchilddocmanual part\else chapter\fi:
`\childdocname' of `\childdocjob'\par
\else
main document: `\childdocjob'\par
\fi
version: \version\par
\end{center}
\newpage
%    \end{macrocode}

% Manually include selected file,
% otherwise process as usual:
%    \begin{macrocode}
\ifchilddocmanual
\section*{part `\childdocname'}
\input{\childdocname}
\else
%    \end{macrocode}

% Include the two chapters:
%    \begin{macrocode}
\include{cdocsch1}
\include{cdocsch2}
%    \end{macrocode}

% Include the two parts unless only chapters should be displayed:
%    \begin{macrocode}
\ifchilddoc\else
\section{part three}
\input{cdocspt3}
\section{part four}
\input{cdocspt4}
\fi
%    \end{macrocode}

% Process as usual until here:
%    \begin{macrocode}
\fi
%    \end{macrocode}

% End of document body:
%    \begin{macrocode}
\end{document}
%    \end{macrocode}
%\iffalse
%</samplemain>
%\fi
%
% %%%%%%%%%%%%%%%%%%%%%%%%%%%%%%%%%%%%%%
% \paragraph{Chapter Include Files.}
%
% The include files are called |cdocsch1.tex| and |cdocsch2.tex|.
%
%\iffalse
%<*samplechap1|samplechap2>
%\fi

% Optional override for |\version| flag:
%    \begin{macrocode}
%%\providecommand{\version}{final}
%    \end{macrocode}

% Include the main document:
%    \begin{macrocode}
\input{childdoc.def}
\childdocof{cdocsamp}
%    \end{macrocode}

%\iffalse
%</samplechap1|samplechap2>
%\fi
%
%\iffalse
%<*samplechap1>
%\fi
% Some text for chapter 1:
%    \begin{macrocode}
\section{one}
some text in chapter one
%    \end{macrocode}

%\iffalse
%</samplechap1>
%\fi
% Some text for chapter 2:
%\iffalse
%<*samplechap2>
%\fi
%    \begin{macrocode}
\section{two}
more text in chapter two
%    \end{macrocode}

%\iffalse
%</samplechap2>
%\fi
%
% %%%%%%%%%%%%%%%%%%%%%%%%%%%%%%%%%%%%%%
% \paragraph{Part Include Files.}
%
% The include files are called |cdocspt3.tex| and |cdocspt4.tex|.
%
%\iffalse
%<*samplepart3|samplepart4>
%\fi

% Optional override for |\version| flag:
%    \begin{macrocode}
%%\providecommand{\version}{final}
%    \end{macrocode}

% Include the main document:
%    \begin{macrocode}
\input{childdoc.def}
\childdocby{cdocsamp}
%    \end{macrocode}

%\iffalse
%</samplepart3|samplepart4>
%\fi
%
%\iffalse
%<*samplepart3>
%\fi
% Some text for part 3:
%    \begin{macrocode}
some text in part three
%    \end{macrocode}

%\iffalse
%</samplepart3>
%\fi
% Some text for part 4:
%\iffalse
%<*samplepart4>
%\fi
%    \begin{macrocode}
more text in part four
%    \end{macrocode}

%\iffalse
%</samplepart4>
%\fi
%
% %%%%%%%%%%%%%%%%%%%%%%%%%%%%%%%%%%%%%%
% \paragraph{Forwarding for a Complete Draft.}
%
% The following forwarding file |cdocsdrf.tex|
% compiles the main document in draft mode:
%\iffalse
%<*sampledraft>
%\fi
%    \begin{macrocode}
\def\version{draft}
\input{childdoc.def}
\childdocforward{cdocsamp}
%    \end{macrocode}

%\iffalse
%</sampledraft>
%\fi
%
% %%%%%%%%%%%%%%%%%%%%%%%%%%%%%%%%%%%%%%
% \paragraph{Forwarding for Final Version of the Chapters.}
%
% The following forwarding files |cdocsfn1.tex| and |cdocsfn2.tex|
% (with identical content)
% compile the final versions of the child documents
% |cdocsch1.tex| and |cdocsch2.tex|, respectively:
%\iffalse
%<*samplefinal>
%\fi
%    \begin{macrocode}
\def\version{final}
\input{childdoc.def}
\childdocforwardprefix[cdocsamp]{cdocsfn}{cdocsch}
%    \end{macrocode}

%\iffalse
%</samplefinal>
%\fi
%
% %%%%%%%%%%%%%%%%%%%%%%%%%%%%%%%%%%%%%%
% \paragraph{Command Line Processing.}
%
% The following three command lines generate the output files
% |cdocscld|, |cdocscl1| and |cdocscl2|
% which should be identical to
% |cdocsdrf|, |cdocsch1| and |cdocsfn2|, respectively:
% \begin{center}
% \begin{tabular}{l}
% |latex -jobname cdocscld \|\\
% |  "\def\version{draft}\input{childdoc.def}\childdocforward{cdocsamp}"|\\
% |latex -jobname cdocscl1 \|\\
% |  "\input{childdoc.def}\childdocforward[cdocsamp]{cdocsch1}"|\\
% |latex -jobname cdocscl2 \|\\
% |  "\def\version{final}\input{childdoc.def}\childdocforward{cdocsch2}"|
% \end{tabular}
% \end{center}
% Note that the trailing backslash on each first line
% merely continues the input to the second line
% (for convenient cut ant paste).
% Furthermore, the command |latex| can be replaced by any
% of its alternative versions such as |pdflatex|.
%
% %%%%%%%%%%%%%%%%%%%%%%%%%%%%%%%%%%%%%%%%%%%%%%%%%%%%%%%%%%%%%%%%%%%%%%%%%%%%%%
% %%%%%%%%%%%%%%%%%%%%%%%%%%%%%%%%%%%%%%%%%%%%%%%%%%%%%%%%%%%%%%%%%%%%%%%%%%%%%%
% \section{Implementation}
%\iffalse
%<*package>
%\fi
%
% This section describes the definitions file |childdoc.def|.

% The definitions cannot be loaded using |\usepackage| or |\RequirePackage|
% which has a mechanism to prevent loading a style file more than once.
% When loading the definitions by means of |\input|
% multiple instances have to be prevented manually:
%\iffalse
%This code needs to be before the `\ProvidesFile' directive
%which is defined at the beginning of this file.
%Therefore it is also placed there and commented out here.
%</package>
%<*discard>
%\fi
%    \begin{macrocode}
\ifdefined\childdocmain\endinput\fi
%    \end{macrocode}
%\iffalse
%</discard>
%<*package>
%\fi
%
% \macro{\ifchilddoc}
% \macro{\ifchilddocmanual}
% The conditional |\ifchilddoc| tells whether a
% child (true) or main (false) document is being compiled.
% The conditional |\ifchilddocmanual| tells whether
% the |\includeonly| mechanism is used (false) or
% the selection of child files must be performed manually (true).
% The definitions initialise to false:
%    \begin{macrocode}
\newif\ifchilddoc
\newif\ifchilddocmanual
%    \end{macrocode}

% \macro{\childdocname}
% \macro{\childdocjob}
% The macro |\childdocname| stores the name of the main document
% to be compiled. The macro |\childdocjob| stores the name of
% the document on which the \LaTeX{} compiler was originally invoked.
% The content of |\jobname| cannot be compared
% to filenames specified in the source due to different catcodes.
% The following code rescans |\jobname|, stores the result
% in |\childdocname| and saves a copy in |\childdocjob|:
%    \begin{macrocode}
\edef\childdocname{\scantokens\expandafter{\jobname\noexpand}}
\let\childdocjob\childdocname
%    \end{macrocode}

% \macro{\childdocdisable}
% The macro |\childdocdisable| prevents the main file
% from being processed more than once.
% At this stage, the main document command |\childdocmain|
% is assumed to be called once again where it should do nothing.
% Any subsequent call to it should prevent
% a secondary processing of the main document
% It overwrites the forwarding commands
% |\childdocof| and |\childdocforward|
% with empty macros to prevent further inclusions of the main document:
%    \begin{macrocode}
\newcommand{\childdocdisable}
{
  \renewcommand{\childdocmain}[1]{\renewcommand{\childdocmain}[1]{\endinput}}
  \renewcommand{\childdocof}[1]{}
  \renewcommand{\childdocby}[2][]{}
  \renewcommand{\childdocforward}[2][]{}
  \renewcommand{\childdocdisable}{}
}
%    \end{macrocode}

% \macro{\childdocmain}
% The macro |\childdocmain| is to be called at the top of the main file
% with nothing or the main filename (without extension) as argument.
% First, it breaks loops.
% If the argument is not empty and does not match |\childdocname|
% (which is set by the first inclusion of |childdoc.def|),
% |\ifchilddoc| is set to true, |\includeonly| is applied to the child file
% and |\jobname| is set to the main file
% (for proper handling of |.aux| files):
%    \begin{macrocode}
\newcommand{\childdocmain}[1]
{
  \childdocdisable\childdocmain{}
  \if?#1?\else
    \begingroup
      \def\childdoctmp{#1}
      \ifx\childdoctmp\childdocname
        \def\childdoctmp{}
      \else
        \def\childdoctmp
        {
          \childdoctrue
          \includeonly{\childdocname}
          \def\childdocjob{#1}
          \def\jobname{#1}
        }
      \fi
      \expandafter
    \endgroup
    \childdoctmp
  \fi
}
%    \end{macrocode}

% \macro{\childdocof}
% The command |\childdocof| redirects
% compilation to the main file |#1|.
%    \begin{macrocode}
\newcommand{\childdocof}[1]
{
  \childdocdisable
  \childdoctrue
  \includeonly{\childdocname}
  \def\jobname{#1}
  \def\childdocjob{#1}
  \input{#1}
}
%    \end{macrocode}

% \macro{\childdocby}
% The command |\childdocby| ....
%    \begin{macrocode}
\newcommand{\childdocby}[2][]
{
  \childdocdisable
  \childdoctrue
  \childdocmanualtrue
  \if?#1?\else
    \def\jobname{#2}
  \fi
  \def\childdocjob{#2}
  \input{#2}
  \endinput
}
%    \end{macrocode}

% \macro{\childdocforward}
% The command |\childdocforward| redirects
% compilation to the main file or
% (if the optional argument is given) a child file.
% Parameters are set as if the main file
% or a child file starting with |\childdocof| was compiled.
% Then compilation is handed over to the main file:
%    \begin{macrocode}
\newcommand{\childdocforward}[2][]
{
  \begingroup
    \if?#1?
      \def\childdoctmp
      {
        \def\childdocname{#2}
        \def\childdocjob{#2}
        \def\jobname{#2}
        \input{#2}
        \endinput
      }
    \else
      \def\childdoctmp
      {
        \childdocdisable
        \def\childdocname{#2}
        \childdoctrue
        \includeonly{#2}
        \def\childdocjob{#1}
        \def\jobname{#1}
        \input{#1}
        \endinput
      }
    \fi
    \expandafter
  \endgroup
  \childdoctmp
}
%    \end{macrocode}

% \macro{\childdocforwardprefix}
% The command |\childdocforwardprefix| redirects
% compilation to the main or a child file by means of a pattern.
% The prefix |#1| in the current filename is replaced by |#2|
% and the suffix of the current filename is kept
% (it is assumed that the filename does not contain the substring `|~~~|'
% which is used as a delimiter).
% Compilation is handed over to the new file by |\childdocforward|:
%    \begin{macrocode}
\newcommand{\childdocforwardprefix}[3][]
{
  \begingroup
    \def\childdocextract #2##1~~~{\def\childdoctmp{\childdocforward[#1]{#3##1}}}
    \expandafter\childdocextract\childdocname~~~
    \expandafter
  \endgroup
  \childdoctmp
}
%    \end{macrocode}

% \macro{\childdoc}
% The deprecated macro |\childdoc| is a legacy version of |\childdocmain|:
%    \begin{macrocode}
\newcommand{\childdoc}{\childdocmain}
%    \end{macrocode}

% \macro{\childdocredirect}
% The deprecated macro |\childdocredirect| is a legacy version
% of |\childdocforward| and |\childdocforwardprefix|:
%    \begin{macrocode}
\newcommand{\childdocredirect}[2][]
{
  \begingroup
    \if?#1?
      \def\childdoctmp{\childdocforward{#2}}
    \else
      \def\childdoctmp{\childdocforwardprefix{#1}{#2}}
    \fi
    \expandafter
  \endgroup
  \childdoctmp
}
%    \end{macrocode}

%\iffalse
%</package>
%\fi
%
\endinput
|\\
|\childdocby{|\textit{main}|}|\\
\end{tabular}
\end{center}
%
Both forms have slightly different effects as described above.
The main file is prepared as usual, see \secref{sec:include}.

%%%%%%%%%%%%%%%%%%%%%%%%%%%%%%%%%%%%%%%%%%%%%%%%%%%%%%%%%%%%%%%%%%%%%%%%%%%%%%%%
\subsection{Legacy Detection}
\label{sec:detection}

The directive |\childdocmain| in the main file can detect
whether the complete document or merely a child is to be compiled
even without using the directive |\childdocof|.
This method is deprecated because it is less robust
and there is no compelling reason to use it;
it is merely provided for backward compatibility
and it may be removed in future versions.

If the detection mechanism is to be used,
it is mandatory to correctly specify
the filename of the main file as the argument of |\childdocmain|:
%
\begin{center}
\begin{tabular}{l}
|% \iffalse
%
% childdoc.dtx Copyright (C) 2017-2018 Niklas Beisert
%
% This work may be distributed and/or modified under the
% conditions of the LaTeX Project Public License, either version 1.3
% of this license or (at your option) any later version.
% The latest version of this license is in
%   http://www.latex-project.org/lppl.txt
% and version 1.3 or later is part of all distributions of LaTeX
% version 2005/12/01 or later.
%
% This work has the LPPL maintenance status `maintained'.
%
% The Current Maintainer of this work is Niklas Beisert.
%
% This work consists of the files childdoc.dtx and childdoc.ins
% and the derived files childdoc.def and cdocsamp.tex with
% cdocsch1.tex, cdocsch2.tex, cdocsdrf.tex, cdocsfn1.tex, cdocsfn2.tex.
%
%<package>\ifdefined\childdocmain\endinput\fi
%<package>\ProvidesFile{childdoc.def}[2018/12/30 v2.0 child document driver]
%<samplemain>\ProvidesFile{cdocsamp.tex}[2018/12/30 v2.0 sample for childdoc]
%<*driver>
%\ProvidesFile{childdoc.drv}[2018/12/30 v2.0 childdoc reference manual file]
\PassOptionsToClass{10pt,a4paper}{article}
\documentclass{ltxdoc}

\usepackage[margin=35mm]{geometry}
\usepackage{hyperref}
\usepackage{hyperxmp}
\usepackage[usenames]{color}

\hypersetup{colorlinks=true}
\hypersetup{pdfstartview=FitH}
\hypersetup{pdfpagemode=UseNone}
\hypersetup{pdfsource={}}
\hypersetup{pdflang={en-UK}}
\hypersetup{pdfcopyright={Copyright 2017-2018 Niklas Beisert.
  This work may be distributed and/or modified under the
  conditions of the LaTeX Project Public License, either version 1.3
  of this license or (at your option) any later version.}}
\hypersetup{pdflicenseurl={http://www.latex-project.org/lppl.txt}}
\hypersetup{pdfcontactaddress={ETH Zurich, ITP, HIT K,
  Wolfgang-Pauli-Strasse 27}}
\hypersetup{pdfcontactpostcode={8093}}
\hypersetup{pdfcontactcity={Zurich}}
\hypersetup{pdfcontactcountry={Switzerland}}
\hypersetup{pdfcontactemail={nbeisert@itp.phys.ethz.ch}}
\hypersetup{pdfcontacturl={http://people.phys.ethz.ch/\xmptilde nbeisert/}}

\newcommand{\secref}[1]{\hyperref[#1]{section \ref*{#1}}}

\parskip1ex
\parindent0pt
\let\olditemize\itemize
\def\itemize{\olditemize\parskip0pt}

\begin{document}

\title{The \textsf{childdoc} Package}
\hypersetup{pdftitle={The childdoc Package}}
\author{Niklas Beisert\\[2ex]
  Institut f\"ur Theoretische Physik\\
  Eidgen\"ossische Technische Hochschule Z\"urich\\
  Wolfgang-Pauli-Strasse 27, 8093 Z\"urich, Switzerland\\[1ex]
  \href{mailto:nbeisert@itp.phys.ethz.ch}
  {\texttt{nbeisert@itp.phys.ethz.ch}}}
\hypersetup{pdfauthor={Niklas Beisert}}
\hypersetup{pdfsubject={Manual for the LaTeX2e Package childdoc}}
\date{30 December 2018, \textsf{v2.0}}
\maketitle

\begin{abstract}\noindent
\textsf{childdoc} is a \LaTeXe{} package
that enables the direct compilation
of document sections included by |\include|
to individual files.
\end{abstract}

\begingroup
\parskip0ex
\tableofcontents
\endgroup

%%%%%%%%%%%%%%%%%%%%%%%%%%%%%%%%%%%%%%%%%%%%%%%%%%%%%%%%%%%%%%%%%%%%%%%%%%%%%%%%
%%%%%%%%%%%%%%%%%%%%%%%%%%%%%%%%%%%%%%%%%%%%%%%%%%%%%%%%%%%%%%%%%%%%%%%%%%%%%%%%
\section{Introduction}

\LaTeX{} provides a mechanism to structure a large document (such as a book)
into a main file and several child files (containing the chapters)
using the |\include| command.
This mechanism is beneficial for documents
which span hundreds of pages in order to
make the source file(s) more manageable.
Moreover, compilation can be restricted to
selected child files by means of the |\includeonly| command.
The latter feature can be used to reduce the compilation time while editing
(this was significantly more useful in the earlier days of \LaTeX{})
or to generate a smaller document which is easier to navigate.
Another application of |\includeonly| is to generate
documents consisting of selected parts of the complete document.

However, there are a few drawbacks of the plain |\include| mechanism:
\begin{itemize}
\item
The child files cannot be compiled on their own,
they can only be compiled via the main file.
A naive editing environment
(such as a text editor with an option
to have the current file processed by \LaTeX)
may require one to switch to the main file before compiling;
attempting to compile the child file produces errors.
\item
The main file must be modified (each time)
to adjust the |\includeonly| command
to the present needs. This easily leaves the main file in a messy state.
\item
The generated document will always carry the filename
of the main document. This is inconvenient if
several child files are to be compiled and
to be kept for distribution.
\end{itemize}

The present package provides a simple interface
to make child files individually compilable by \LaTeX{}.
Compiling a child file then has the same effect as compiling
the main file with an |\includeonly| command
to select the appropriate child.
Moreover the generated document will carry the name of the child
rather than the main file.
This resolves all three above issues.

This feature is meant to make the editing of books,
thesis documents and lecture notes somewhat more convenient.
However, the package can also be used efficiently for
composing a series of documents (such as exercise sheets)
which are typically distributed individually.
It then assists the author in generating the individual documents
(potentially in different versions)
as well as a document containing the collected series.
Another application is in developing style files
or other kinds of included material
where compilation of the style file could redirect
to a sample or test file.

%%%%%%%%%%%%%%%%%%%%%%%%%%%%%%%%%%%%%%%%%%%%%%%%%%%%%%%%%%%%%%%%%%%%%%%%%%%%%%%%
%%%%%%%%%%%%%%%%%%%%%%%%%%%%%%%%%%%%%%%%%%%%%%%%%%%%%%%%%%%%%%%%%%%%%%%%%%%%%%%%
\section{Usage}

First of all, the package \textsf{childdoc} is \emph{not} a standard
\LaTeXe{} |.sty| style file! Therefore it needs to be invoked in
a non-standard way.

%%%%%%%%%%%%%%%%%%%%%%%%%%%%%%%%%%%%%%%%%%%%%%%%%%%%%%%%%%%%%%%%%%%%%%%%%%%%%%%%
\subsection{Included Files}
\label{sec:include}

%%%%%%%%%%%%%%%%%%%%%%%%%%%%%%%%%%%%%%%%
\DescribeMacro{\childdocmain}
To use the package, add the commands
\begin{center}
\begin{tabular}{l}
|\input{childdoc.def}|\\
|\childdocmain{}|\\
\end{tabular}
\end{center}
at the very top of the main \LaTeX{} file,
in particular \emph{before} the |\documentclass| statement!
The argument of |\childdocmain| should be left empty
(but it must be present).

%%%%%%%%%%%%%%%%%%%%%%%%%%%%%%%%%%%%%%%%
\DescribeMacro{\childdocof}
Furthermore, add the commands
\begin{center}
\begin{tabular}{l}
|\input{childdoc.def}|\\
|\childdocof{|\textit{main}|}|\\
\end{tabular}
\end{center}
at the top of every child file \textit{child}
which is included by |\include{|\textit{child}|}|
from within the main file
(or at least for those files to be compiled individually).
The argument \textit{main} must be the filename of the main file.

There are a couple of
considerations in setting up the main and child documents:

%%%%%%%%%%%%%%%%%%%%%%%%%%%%%%%%%%%%%%%%
\paragraph{Restrictions.}

Please note the following restrictions:
\begin{itemize}
\item
|\childdocmain| must be called with one argument \textit{main}
to ensure compatibility with earlier version of the package.
It must either be empty (|\childdocmain{}|)
or precisely match the filename of the main file in which it is specified.
See \secref{sec:detection} for further information.
\item
The filename \textit{main} must be specified without the |.tex| extension.
\item
The filename \textit{main} is case sensitive
(even in case-insensitive file systems)
due to internal string comparison.
\item
The argument \textit{main} should be fully expanded, it cannot be a macro.
\item
Subdirectories and special characters should be avoided in filenames.
\item
The command |\childdocmain{|\textit{main}|}| must be followed by a whitespace.
It should not be followed immediately by another command
or by a comment mark `|%|'.
This is because the \TeX{} parser reads the token immediately following
the argument of |\childdocmain| and puts it
at the beginning of every child section;
however, a white\-space is ignored.
\end{itemize}

%%%%%%%%%%%%%%%%%%%%%%%%%%%%%%%%%%%%%%%%
\paragraph{Content of Main File.}

It is advisable to place all content in the child files included by |\include|.
Any output contained in the main file will appear in all child documents
unless suppressed manually;
it cannot be suppressed automatically by the |\includeonly| directive
and thus should normally be avoided.
A method to include some content in the main file
by means of conditional processing is described in \secref{sec:conditional}.

%%%%%%%%%%%%%%%%%%%%%%%%%%%%%%%%%%%%%%%%
\paragraph{Page Numbering.}

When only a part of the document is compiled,
the appropriate numbering of pages
(as well as other status parameters)
is determined from the |.aux| files.
The latter contain information from previous passes.
However this information needs to propagate through
all intermediate child documents.
Therefore the page numbering in child documents may well
be inconsistent until the complete document is compiled at least once.

A useful (if unconventional) way to always ensure a consistent
page numbering is to restart the numbering in each child document
and denote the pages by `\textit{child}|.|\textit{page}'
where \textit{child} represents the chapter/section number of the child file.
This can be achieved by the command
|\numberwithin{page}{|\textit{child}|}|
of the \textsf{amsmath} package
where \textit{child} can be |chapter| or |section|
depending on the chosen structuring.
Alternatively, one can modify the macro |\thepage| appropriately
and reset the counter |page| at the start of each child file.

%%%%%%%%%%%%%%%%%%%%%%%%%%%%%%%%%%%%%%%%%%%%%%%%%%%%%%%%%%%%%%%%%%%%%%%%%%%%%%%%
\subsection{Conditional Processing}
\label{sec:conditional}

The package provides a mechanism to compile different versions
of a document. To customise the versions further some conditional processing
can come in handy to distinguish which version is being compiled.
The package provides two macros to describe the compilation context:

%%%%%%%%%%%%%%%%%%%%%%%%%%%%%%%%%%%%%%%%
\DescribeMacro{\ifchilddoc}
The conditional |\ifchilddoc| distinguishes between the compilation of
child documents and the main document:
%
\begin{center}
|\ifchilddoc |\textit{child-code}| |[|\||else |\textit{main-code}]| \||fi|
\end{center}

%%%%%%%%%%%%%%%%%%%%%%%%%%%%%%%%%%%%%%%%
\DescribeMacro{\childdocname}
\DescribeMacro{\childdocjob}
The macro |\childdocname| contains the filename (without extension)
of the main or child file being processed.
Note that |\childdocjob| will always contain the name of the main file.

%%%%%%%%%%%%%%%%%%%%%%%%%%%%%%%%%%%%%%%%
\paragraph{Title Page.}

Conditional processing can be used to include a title or banner page
in the main document when proper precautions are taken.
Importantly, the code in the main file should ensure that the page counter
(as well as other status parameters which are stored in the |.aux| files)
takes the same value after the conditional processing.
Otherwise the page numbers may take divergent values
depending on which part is compiled.

For example, a title page could be declared by:
%
\begin{center}
\begin{tabular}{l}
|\ifchilddoc\||else|\\
|\addtocounter{page}{-1}|\\
\textit{code for title page}\\
|\newpage|\\
|\||fi|
\end{tabular}
\end{center}
%
A banner page for the child documents can be generated by:
%
\begin{center}
\begin{tabular}{l}
|\ifchilddoc|\\
|\addtocounter{page}{-1}|\\
\textit{code for banner page}\\
|\newpage|\\
|\||fi|
\end{tabular}
\end{center}
%
Here one could write a message such as:
\begin{center}
|This is the part \childdocname{} of \childdocjob{}.|
\end{center}

%%%%%%%%%%%%%%%%%%%%%%%%%%%%%%%%%%%%%%%%%%%%%%%%%%%%%%%%%%%%%%%%%%%%%%%%%%%%%%%%
\subsection{Flags}
\label{sec:flags}

The package makes it easy to generate different versions
of the main or child documents.
To this end compilation flags can be defined
and assigned different default values.
They will be particularly useful in conjunction
with the forwarding mechanism described in \secref{sec:forward}.

For example, it may be useful to have a flag |\version|
which can be set to |draft| or |final|.
The document source will contain some conditional code
depending on the value of |\version|.
Suppose further, the flag should default to |final| for the main file
and to |draft| for child files
which is a natural assignment for editing the document.
This is achieved by placing the following code
in the preamble of the main document
(below the |\childdocmain| directive):
%
\begin{center}
\begin{tabular}{l}
|\ifchilddoc|\\
|\providecommand{\version}{draft}|\\
|\||else|\\
|\providecommand{\version}{final}|\\
|\||fi|
\end{tabular}
\end{center}
%
The definition by |\providecommand| makes sure
that previous definitions are not overwritten.
Further statements |\providecommand{\version}{...}|
can thus be added before the above code to override it.

For the main file, one might add a line
(between |\childdocmain| and the above block)
%
\begin{center}
|%\ifchilddoc\||else\providecommand{\version}{draft}\||fi|
\end{center}
%
which can be uncommented to produce a draft version.
Likewise one can add a line to the very top of a child file
(above the |\childdocof{|\textit{main}|}| directive)
%
\begin{center}
|%\providecommand{\version}{final}|
\end{center}
%
which can be uncommented to produce the final version of this child document.

%%%%%%%%%%%%%%%%%%%%%%%%%%%%%%%%%%%%%%%%%%%%%%%%%%%%%%%%%%%%%%%%%%%%%%%%%%%%%%%%
\subsection{Forwarding}
\label{sec:forward}

Different versions of the main or child documents
using compilation flags as described in \secref{sec:flags}
can be (permanently) stored in different files
for convenient compilation, viewing and distribution.
To this end, the package defines a command
to pass on compilation to a different file:

%%%%%%%%%%%%%%%%%%%%%%%%%%%%%%%%%%%%%%%%
\DescribeMacro{\childdocforward}
The command |\childdocforward| redirects processing to
another source file:
%
\begin{center}
\begin{tabular}{l}
|\input{childdoc.def}|\\
|\childdocforward[|\textit{main}|]{|\textit{dest}|}|\\
\end{tabular}
\end{center}
%
The argument \textit{dest} is the destination file
(without extension).
It should be the main file or one of the child files.
Note that further \textsf{childdoc} directives
such as |\childdocof| and |\childdocforward|
in the indicated file will be processed in this form.
The optional argument \textit{main}
passes on directly to the main file \textit{main}
while pretending to compile the child \textit{dest}.
This form behaves as if \textit{dest}
issues |\childdocof{|\textit{main}|}| right away,
and no further \textsf{childdoc} directives will be processed.

%%%%%%%%%%%%%%%%%%%%%%%%%%%%%%%%%%%%%%%%
\DescribeMacro{\...prefix}
In the alternative form |\childdocforwardprefix|,
%
\begin{center}
\begin{tabular}{l}
|\input{childdoc.def}|\\
|\childdocforwardprefix[|\textit{main}|]{|\textit{prefix}|}{|\textit{dest}|}|
\end{tabular}
\end{center}
%
the destination file is determined by a pattern
depending on the current file:
To make this work, the current file must be called
`{\textit{prefix}\hspace{0.2em}\textit{suffix}}'
with \textit{prefix} matching precisely the argument.
Processing is then passed on to the file
`{\textit{dest}\hspace{0.2em}\textit{suffix}}'.
Surely, the same effect is achieved by
directly specifying the
argument `{\textit{dest}\hspace{0.2em}\textit{suffix}}'
in the first form.
However, that requires to set up a different file
for each child. With the alternative form of the command
all these files can have exactly the same content
which simplifies setting them up and maintaining them.

For example, the following file |draft.tex|
with a compilation flag |\version| as described in \secref{sec:flags}
compiles the main document as a draft:
%
\begin{center}
\begin{tabular}{l}
|\def\version{draft}|\\
|\input{childdoc.def}|\\
|\childdocforward{|\textit{main}|}|
\end{tabular}
\end{center}
%
Likewise, the following files |final|\textit{nn}|.tex|
compile the final version of the child document
|child|\textit{nn}|.tex|:
%
\begin{center}
\begin{tabular}{l}
|\def\version{final}|\\
|\input{childdoc.def}|\\
|\childdocforwardprefix{final}{child}|
\end{tabular}
\end{center}
%

Note that when several versions of a main file and/or of each child file
are to be generated, it may be convenient to set up a |Makefile| or
shell script to automatise the process.

%%%%%%%%%%%%%%%%%%%%%%%%%%%%%%%%%%%%%%%%%%%%%%%%%%%%%%%%%%%%%%%%%%%%%%%%%%%%%%%%
\subsection{Command Line Processing}
\label{sec:commandline}

The effect of redirection files can also be achieved by invoking
the \LaTeX{} compiler with a more elaborate command line.
Most conveniently this should be done as part
of a shell script or a |Makefile|.

When using \textsf{childdoc} in the main file, the following
command lines effectively perform a redirection
(note that depending on the shell being used,
backslashes may have to be doubled: `|\|' $\to$ `|\\|'):
%
\begin{center}
|... -jobname "|\textit{target}|" |\\|"|[\textit{flags}]%
|\input{childdoc.def}\childdocforward[|\textit{main}|]{|\textit{dest}|}"|
\end{center}
%
Here \textit{target} is the name of the output file,
\textit{main} is the name of the main file
and \textit{dest} is the name of the main or child file to be processed
(all filenames without extensions).
The optional argument \textit{main} can be omitted
if \textit{main} matches \textit{dest}.
Optionally, compilation \textit{flags} can be defined via |\def| commands.
This command line makes the \TeX{} engine believe
it is compiling the file \textit{target}
whose content is specified as the latter parameter.
The provided code then forwards the processing to
\textit{main} or \textit{dest} as described in \secref{sec:forward}.

%%%%%%%%%%%%%%%%%%%%%%%%%%%%%%%%%%%%%%%%%%%%%%%%%%%%%%%%%%%%%%%%%%%%%%%%%%%%%%%%
\subsection{Include by Input}
\label{sec:input}

Including child documents by |\include| has some restrictions by design.
Most notably, the content of a child document always occupies
its own set of pages; pages cannot be shared between child documents.
Usually, this behaviour makes perfect sense
because each child document contain an essential part of the document.
However, in some situations it may be desirable to compose
a document from a collection of parts
without having mandatory page breaks between then.
For this case, the package
provides a mechanism to include parts
by |\input| which can also be processed individually.
However, by construction this mechanism
requires manual handling of the content to be output.

%%%%%%%%%%%%%%%%%%%%%%%%%%%%%%%%%%%%%%%%
\DescribeMacro{\ifchilddocmanual}
The main file should be prepared as usual, see \secref{sec:include}.
However, the document body must make a distinction
between processing of an individual part and of the main document, e.g.:
%
\begin{center}
\begin{tabular}{l}
|\ifchilddocmanual|\\
|\input{\childdocname}|\\
|\||else|\\
\textit{document body with }|\input{|\textit{part}|}|\\
|\||fi|
\end{tabular}
\end{center}
%
The conditional |\ifchilddocmanual| is true whenever
a part to be included by |\input| is being compiled,
and the name of the part is stored in |\childdocname|.

%%%%%%%%%%%%%%%%%%%%%%%%%%%%%%%%%%%%%%%%
\DescribeMacro{\childdocby}
Each part to be included by |\input| should start with:
%
\begin{center}
\begin{tabular}{l}
|\input{childdoc.def}|\\
|\childdocby{|\textit{main}|}|\\
\end{tabular}
\end{center}
%
The directive |\childdocby| is similar to |\childdocof|
described in \secref{sec:include},
but the subsequent selection of content must be done manually.
To that end, both |\ifchilddoc| and |\ifchilddocmanual|
will be true upon processing of a part,
and the name of the part is stored in |\childdocname|.
Note that |\jobname| will be set to the filename of the current part
so that each part receives an individual |.aux| file
that does not interfere with the |.aux| file(s) of the main document.
This behaviour can be altered by the alternative form
|\childdocby[*]{|\textit{main}|}| (with a non-empty optional argument)
which uses the |.aux| file of the main document
by setting |\jobname| to \textit{main}.

%%%%%%%%%%%%%%%%%%%%%%%%%%%%%%%%%%%%%%%%%%%%%%%%%%%%%%%%%%%%%%%%%%%%%%%%%%%%%%%%
\subsection{Driver Development}
\label{sec:driver}

The \textsf{childdoc} mechanism can also be use for the development
of definition files such as \LaTeX{} styles or classes.
This case differs from the above setup with multiple parts
included by |\include| in that no |\includeonly| should be invoked.
This can be achieved by starting the include file
(before |\ProvidesPackage|) with:
%
\begin{center}
\begin{tabular}{l}
|\input{childdoc.def}|\\
|\childdocforward{|\textit{main}|}|\\
\end{tabular}
\end{center}
%
or alternatively with:
%
\begin{center}
\begin{tabular}{l}
|\input{childdoc.def}|\\
|\childdocby{|\textit{main}|}|\\
\end{tabular}
\end{center}
%
Both forms have slightly different effects as described above.
The main file is prepared as usual, see \secref{sec:include}.

%%%%%%%%%%%%%%%%%%%%%%%%%%%%%%%%%%%%%%%%%%%%%%%%%%%%%%%%%%%%%%%%%%%%%%%%%%%%%%%%
\subsection{Legacy Detection}
\label{sec:detection}

The directive |\childdocmain| in the main file can detect
whether the complete document or merely a child is to be compiled
even without using the directive |\childdocof|.
This method is deprecated because it is less robust
and there is no compelling reason to use it;
it is merely provided for backward compatibility
and it may be removed in future versions.

If the detection mechanism is to be used,
it is mandatory to correctly specify
the filename of the main file as the argument of |\childdocmain|:
%
\begin{center}
\begin{tabular}{l}
|\input{childdoc.def}|\\
|\childdocmain{|\textit{main}|}|\\
\end{tabular}
\end{center}
%
If |\jobname| does not match the argument \textit{main} of |\childdocmain|,
it is assumed that |\jobname| points to the child file to be compiled.
When using |\childdocmain| with the main file specified as argument,
it suffices to start a child file
with just |\input{|\textit{main}|}|
without loading of the package and using |\childdocof|.
If instead all processing is done
with the appropriate \textsf{childdoc} directives,
the argument of \textit{main} of |\childdocmain| can be empty.

An alternative version of the command line processing described
in \secref{sec:commandline} using the detection mechanism reads:
%
\begin{center}
|... -jobname "|\textit{target}|" "|[\textit{flags}]%
[|\def\jobname{|\textit{dest}|}|]|\input{|\textit{main}|}"|
\end{center}

%%%%%%%%%%%%%%%%%%%%%%%%%%%%%%%%%%%%%%%%%%%%%%%%%%%%%%%%%%%%%%%%%%%%%%%%%%%%%%%%
\subsection{Manual Code}
\label{sec:manual}

In case one cannot be certain whether the definitions file |childdoc.def|
is installed on the target \TeX{} distribution
and one prefers not to ship it,
it is conceivable to paste a few relevant commands into the sources.

To that end, drop all statements |\input{childdoc.def}|
and perform the replacements as outlined below.
Instead of |\childdocmain{|\textit{main}|}| add the following code
to the top of the main file:
%
\begin{center}
\begin{tabular}{l}
|\||ifdefined\childdocname\endinput\||fi\newif\ifchilddoc|\\
|\edef\childdocname{\scantokens\expandafter{\jobname\noexpand}}|\\
|\def\childdocmain{|\textit{main}|}\||ifx\childdocmain\childdocname\||else|\\
|\childdoctrue\includeonly{\childdocname}\let\jobname\childdocmain\||fi|\\
\end{tabular}
\end{center}
%
Instead of |\childdocof{|\textit{main}|}| just include the main file
at the top of each child file:
%
\begin{center}
|\input{|\textit{main}|}|
\end{center}
%
A simple redirection |\childdocforward{|\textit{dest}|}| is achieved by:
%
\begin{center}
|\def\jobname{|\textit{dest}|}\input{\jobname}|
\end{center}
%
The redirection with prefix
|\childdocforwardprefix[|\textit{prefix}|]{|\textit{dest}|}|
is accomplished by:
%
\begin{center}
\begin{tabular}{l}
|{\edef\jobname{\scantokens\expandafter{\jobname\noexpand}}|\\
|\def\redirectjob |\textit{prefix}|#1~~~{\gdef\jobname{|\textit{dest}|#1}}|\\
|\expandafter\redirectjob\jobname~~~}\input{\jobname}|
\end{tabular}
\end{center}

In an alternative approach,
child documents can be compiled by a specific command line
without additional code or specific definitions:
%
\begin{center}
|... -jobname "|\textit{target}|" "|[\textit{flags}]%
|\includeonly{|\textit{dest}|}\input{|\textit{main}|}"|
\end{center}
%

%%%%%%%%%%%%%%%%%%%%%%%%%%%%%%%%%%%%%%%%%%%%%%%%%%%%%%%%%%%%%%%%%%%%%%%%%%%%%%%%
%%%%%%%%%%%%%%%%%%%%%%%%%%%%%%%%%%%%%%%%%%%%%%%%%%%%%%%%%%%%%%%%%%%%%%%%%%%%%%%%
\section{Information}

%%%%%%%%%%%%%%%%%%%%%%%%%%%%%%%%%%%%%%%%%%%%%%%%%%%%%%%%%%%%%%%%%%%%%%%%%%%%%%%%
\subsection{Copyright}

Copyright \copyright{} 2017--2018 Niklas Beisert

This work may be distributed and/or modified under the
conditions of the \LaTeX{} Project Public License, either version 1.3
of this license or (at your option) any later version.
The latest version of this license is in
  \url{http://www.latex-project.org/lppl.txt}
and version 1.3 or later is part of all distributions of \LaTeX{}
version 2005/12/01 or later.

This work has the LPPL maintenance status `maintained'.

The Current Maintainer of this work is Niklas Beisert.

This work consists of the files |README.txt|, |childdoc.ins| and |childdoc.dtx|
as well as the derived files |childdoc.def|, |cdocsamp.tex|
with |cdocsch1.tex|, |cdocsch2.tex|, |cdocspt3.tex|, |cdocspt4.tex|,
|cdocsdrf.tex|, |cdocsfn1.tex|, |cdocsfn2.tex|
as well as |childdoc.pdf|.

%%%%%%%%%%%%%%%%%%%%%%%%%%%%%%%%%%%%%%%%%%%%%%%%%%%%%%%%%%%%%%%%%%%%%%%%%%%%%%%%
\subsection{Files and Installation}

The package consists of the files:
%
\begin{center}
\begin{tabular}{ll}
    |README.txt|   & readme file \\
    |childdoc.ins| & installation file \\
    |childdoc.dtx| & source file \\
    |childdoc.def| & definition file \\
    |cdocsamp.tex| & sample main file \\
    |cdocsch1.tex| & sample include file \\
    |cdocsch2.tex| & sample include file \\
    |cdocspt3.tex| & sample part file \\
    |cdocspt4.tex| & sample part file \\
    |cdocsdrf.tex| & sample redirection file \\
    |cdocsfn1.tex| & sample redirection file \\
    |cdocsfn2.tex| & sample redirection file \\
    |childdoc.pdf| & manual
\end{tabular}
\end{center}
%
The distribution consists of the files
|README.txt|, |childdoc.ins| and |childdoc.dtx|.
%
\begin{itemize}
\item
Run (pdf)\LaTeX{} on |childdoc.dtx|
to compile the manual |childdoc.pdf| (this file).
\item
Run \LaTeX{} on |childdoc.ins| to create the definitions file |childdoc.def|
and the sample |cdocsamp.tex| with include files
|cdocsch1.tex|, |cdocsch2.tex|, |cdocspt3.tex|, |cdocspt4.tex|,
|cdocsdrf.tex|, |cdocsfn1.tex|, |cdocsfn2.tex|.
Then copy the file |childdoc.def| to an appropriate directory of your \LaTeX{}
distribution, e.g.\ \textit{texmf-root}|/tex/latex/childdoc|.
\end{itemize}

%%%%%%%%%%%%%%%%%%%%%%%%%%%%%%%%%%%%%%%%%%%%%%%%%%%%%%%%%%%%%%%%%%%%%%%%%%%%%%%%
\subsection{Related CTAN Packages}

There are several other packages which offer a similar functionality:
%
\begin{itemize}
\item
The packages
\href{http://ctan.org/pkg/docmute}{\textsf{docmute}},
\href{http://ctan.org/pkg/includex}{\textsf{includex}} and
\href{http://ctan.org/pkg/standalone}{\textsf{standalone}}
provide commands to include only the document body of
a child file thus allowing both files to be compiled individually.
\item
The packages \href{http://ctan.org/pkg/subdocs}{\textsf{subdocs}}
and \href{http://ctan.org/pkg/subfiles}{\textsf{subfiles}}
provide structures in which the main and child documents can be
encapsulated and allowing them to be compiled individually.
The inclusion mechanism is different from the conventional |\include|.
\item
The package \href{http://ctan.org/pkg/combine}{\textsf{combine}}
is an elaborate solution to combine several documents into one.
\end{itemize}
%
See also the CTAN topic \href{http://ctan.org/topic/subdocs}{\textsf{subdocs}}
for further related packages.
The present package differs from the above solutions in that
a document structure constructed with the conventional |\include| mechanism
just needs two extra commands at the top of every file
such that all constituent files can be compiled individually.

%%%%%%%%%%%%%%%%%%%%%%%%%%%%%%%%%%%%%%%%%%%%%%%%%%%%%%%%%%%%%%%%%%%%%%%%%%%%%%%%
%\subsection{Feature Suggestions}
%
%The following is a list of features which may be useful for future
%versions of this package:
%%
%\begin{itemize}
%\item
%\ldots
%\end{itemize}

%%%%%%%%%%%%%%%%%%%%%%%%%%%%%%%%%%%%%%%%%%%%%%%%%%%%%%%%%%%%%%%%%%%%%%%%%%%%%%%%
\subsection{Revision History}

%%%%%%%%%%%%%%%%%%%%%%%%%%%%%%%%%%%%%%%%
\paragraph{v2.0:} 2018/12/30

\begin{itemize}
\item
immediate forward processing
\item
added |\childdocby| mechanism
\item
manual restructured
\end{itemize}

%%%%%%%%%%%%%%%%%%%%%%%%%%%%%%%%%%%%%%%%
\paragraph{v1.6:} 2018/01/17

\begin{itemize}
\item
application for development of include files
\item
corrections to manual
\end{itemize}

%%%%%%%%%%%%%%%%%%%%%%%%%%%%%%%%%%%%%%%%
\paragraph{v1.5:} 2017/05/21

\begin{itemize}
\item
more complete structuring introduced
\item
|\childdocof| introduced
\item
|\childdoc| renamed to |\childdocmain|
\item
|\childredirect| renamed to |\childdocforward| and |\childdocforwardprefix|
and functionality expanded
\end{itemize}

%%%%%%%%%%%%%%%%%%%%%%%%%%%%%%%%%%%%%%%%
\paragraph{v1.0:} 2017/04/27

\begin{itemize}
\item
manual and install package
\item
first version published on CTAN
\end{itemize}

%%%%%%%%%%%%%%%%%%%%%%%%%%%%%%%%%%%%%%%%
\paragraph{v0.6:} 2017/04/26

\begin{itemize}
\item
redirection mechanism added
\end{itemize}

%%%%%%%%%%%%%%%%%%%%%%%%%%%%%%%%%%%%%%%%
\paragraph{v0.5:} 2017/04/26

\begin{itemize}
\item
functionality in definition file
\end{itemize}


%%%%%%%%%%%%%%%%%%%%%%%%%%%%%%%%%%%%%%%%%%%%%%%%%%%%%%%%%%%%%%%%%%%%%%%%%%%%%%%%
%%%%%%%%%%%%%%%%%%%%%%%%%%%%%%%%%%%%%%%%%%%%%%%%%%%%%%%%%%%%%%%%%%%%%%%%%%%%%%%%
%%%%%%%%%%%%%%%%%%%%%%%%%%%%%%%%%%%%%%%%%%%%%%%%%%%%%%%%%%%%%%%%%%%%%%%%%%%%%%%%
\appendix

\settowidth\MacroIndent{\rmfamily\scriptsize 000\ }

 \DocInput{childdoc.dtx}

\end{document}
%</driver>
% \fi
%
% %%%%%%%%%%%%%%%%%%%%%%%%%%%%%%%%%%%%%%%%%%%%%%%%%%%%%%%%%%%%%%%%%%%%%%%%%%%%%%
% %%%%%%%%%%%%%%%%%%%%%%%%%%%%%%%%%%%%%%%%%%%%%%%%%%%%%%%%%%%%%%%%%%%%%%%%%%%%%%
% \section{Sample}
%\iffalse
%<*samplemain>
%\fi
%
% The following presents a sample document
% with two chapters, two parts, a title page,
% a compile flag as well as three forwarding files to set the flag.
% It consists of eight |.tex| files:
% \begin{center}
% \begin{tabular}{ll}
% |cdocsamp.tex|&main file\\
% |cdocsch1.tex|&include file for chapter 1\\
% |cdocsch2.tex|&include file for chapter 2\\
% |cdocspt3.tex|&include file for part 3\\
% |cdocspt4.tex|&include file for part 4\\
% |cdocsdrf.tex|&forwarding file for main file in draft mode\\
% |cdocsfi1.tex|&forwarding file for final version of chapter 1\\
% |cdocsfi2.tex|&forwarding file for final version of chapter 2\\
% \end{tabular}
% \end{center}
% Each of the eight files can be compiled directly by the \LaTeX{} compiler.
%
% %%%%%%%%%%%%%%%%%%%%%%%%%%%%%%%%%%%%%%
% \paragraph{Main File.}
%
% The main file is called |cdocsamp.tex|.
%
% Load the \textsf{childdoc} definitions and
% declare the filename for the main document:
%    \begin{macrocode}
\input{childdoc.def}
\childdocmain{}
%    \end{macrocode}

% Optional override for |\version| flag:
%    \begin{macrocode}
%%\ifchilddoc\else\providecommand{\version}{draft}\fi
%    \end{macrocode}

% Define the default values for the |\version| flag
% (|final| for the main file and |draft| for childs):
%    \begin{macrocode}
\ifchilddoc
\providecommand{\version}{draft}
\else
\providecommand{\version}{final}
\fi
%    \end{macrocode}

% Load the standard document class:
%    \begin{macrocode}
\documentclass[12pt]{article}
%    \end{macrocode}

% Start the document body:
%    \begin{macrocode}
\begin{document}
%    \end{macrocode}

% Declare a title page.
% Print title, part of document being processed and version flag:
%    \begin{macrocode}
\addtocounter{page}{-1}
\begin{center}
{\LARGE\bfseries{}childdoc example\par}
\vspace{1cm}
\ifchilddoc
\ifchilddocmanual part\else chapter\fi:
`\childdocname' of `\childdocjob'\par
\else
main document: `\childdocjob'\par
\fi
version: \version\par
\end{center}
\newpage
%    \end{macrocode}

% Manually include selected file,
% otherwise process as usual:
%    \begin{macrocode}
\ifchilddocmanual
\section*{part `\childdocname'}
\input{\childdocname}
\else
%    \end{macrocode}

% Include the two chapters:
%    \begin{macrocode}
\include{cdocsch1}
\include{cdocsch2}
%    \end{macrocode}

% Include the two parts unless only chapters should be displayed:
%    \begin{macrocode}
\ifchilddoc\else
\section{part three}
\input{cdocspt3}
\section{part four}
\input{cdocspt4}
\fi
%    \end{macrocode}

% Process as usual until here:
%    \begin{macrocode}
\fi
%    \end{macrocode}

% End of document body:
%    \begin{macrocode}
\end{document}
%    \end{macrocode}
%\iffalse
%</samplemain>
%\fi
%
% %%%%%%%%%%%%%%%%%%%%%%%%%%%%%%%%%%%%%%
% \paragraph{Chapter Include Files.}
%
% The include files are called |cdocsch1.tex| and |cdocsch2.tex|.
%
%\iffalse
%<*samplechap1|samplechap2>
%\fi

% Optional override for |\version| flag:
%    \begin{macrocode}
%%\providecommand{\version}{final}
%    \end{macrocode}

% Include the main document:
%    \begin{macrocode}
\input{childdoc.def}
\childdocof{cdocsamp}
%    \end{macrocode}

%\iffalse
%</samplechap1|samplechap2>
%\fi
%
%\iffalse
%<*samplechap1>
%\fi
% Some text for chapter 1:
%    \begin{macrocode}
\section{one}
some text in chapter one
%    \end{macrocode}

%\iffalse
%</samplechap1>
%\fi
% Some text for chapter 2:
%\iffalse
%<*samplechap2>
%\fi
%    \begin{macrocode}
\section{two}
more text in chapter two
%    \end{macrocode}

%\iffalse
%</samplechap2>
%\fi
%
% %%%%%%%%%%%%%%%%%%%%%%%%%%%%%%%%%%%%%%
% \paragraph{Part Include Files.}
%
% The include files are called |cdocspt3.tex| and |cdocspt4.tex|.
%
%\iffalse
%<*samplepart3|samplepart4>
%\fi

% Optional override for |\version| flag:
%    \begin{macrocode}
%%\providecommand{\version}{final}
%    \end{macrocode}

% Include the main document:
%    \begin{macrocode}
\input{childdoc.def}
\childdocby{cdocsamp}
%    \end{macrocode}

%\iffalse
%</samplepart3|samplepart4>
%\fi
%
%\iffalse
%<*samplepart3>
%\fi
% Some text for part 3:
%    \begin{macrocode}
some text in part three
%    \end{macrocode}

%\iffalse
%</samplepart3>
%\fi
% Some text for part 4:
%\iffalse
%<*samplepart4>
%\fi
%    \begin{macrocode}
more text in part four
%    \end{macrocode}

%\iffalse
%</samplepart4>
%\fi
%
% %%%%%%%%%%%%%%%%%%%%%%%%%%%%%%%%%%%%%%
% \paragraph{Forwarding for a Complete Draft.}
%
% The following forwarding file |cdocsdrf.tex|
% compiles the main document in draft mode:
%\iffalse
%<*sampledraft>
%\fi
%    \begin{macrocode}
\def\version{draft}
\input{childdoc.def}
\childdocforward{cdocsamp}
%    \end{macrocode}

%\iffalse
%</sampledraft>
%\fi
%
% %%%%%%%%%%%%%%%%%%%%%%%%%%%%%%%%%%%%%%
% \paragraph{Forwarding for Final Version of the Chapters.}
%
% The following forwarding files |cdocsfn1.tex| and |cdocsfn2.tex|
% (with identical content)
% compile the final versions of the child documents
% |cdocsch1.tex| and |cdocsch2.tex|, respectively:
%\iffalse
%<*samplefinal>
%\fi
%    \begin{macrocode}
\def\version{final}
\input{childdoc.def}
\childdocforwardprefix[cdocsamp]{cdocsfn}{cdocsch}
%    \end{macrocode}

%\iffalse
%</samplefinal>
%\fi
%
% %%%%%%%%%%%%%%%%%%%%%%%%%%%%%%%%%%%%%%
% \paragraph{Command Line Processing.}
%
% The following three command lines generate the output files
% |cdocscld|, |cdocscl1| and |cdocscl2|
% which should be identical to
% |cdocsdrf|, |cdocsch1| and |cdocsfn2|, respectively:
% \begin{center}
% \begin{tabular}{l}
% |latex -jobname cdocscld \|\\
% |  "\def\version{draft}\input{childdoc.def}\childdocforward{cdocsamp}"|\\
% |latex -jobname cdocscl1 \|\\
% |  "\input{childdoc.def}\childdocforward[cdocsamp]{cdocsch1}"|\\
% |latex -jobname cdocscl2 \|\\
% |  "\def\version{final}\input{childdoc.def}\childdocforward{cdocsch2}"|
% \end{tabular}
% \end{center}
% Note that the trailing backslash on each first line
% merely continues the input to the second line
% (for convenient cut ant paste).
% Furthermore, the command |latex| can be replaced by any
% of its alternative versions such as |pdflatex|.
%
% %%%%%%%%%%%%%%%%%%%%%%%%%%%%%%%%%%%%%%%%%%%%%%%%%%%%%%%%%%%%%%%%%%%%%%%%%%%%%%
% %%%%%%%%%%%%%%%%%%%%%%%%%%%%%%%%%%%%%%%%%%%%%%%%%%%%%%%%%%%%%%%%%%%%%%%%%%%%%%
% \section{Implementation}
%\iffalse
%<*package>
%\fi
%
% This section describes the definitions file |childdoc.def|.

% The definitions cannot be loaded using |\usepackage| or |\RequirePackage|
% which has a mechanism to prevent loading a style file more than once.
% When loading the definitions by means of |\input|
% multiple instances have to be prevented manually:
%\iffalse
%This code needs to be before the `\ProvidesFile' directive
%which is defined at the beginning of this file.
%Therefore it is also placed there and commented out here.
%</package>
%<*discard>
%\fi
%    \begin{macrocode}
\ifdefined\childdocmain\endinput\fi
%    \end{macrocode}
%\iffalse
%</discard>
%<*package>
%\fi
%
% \macro{\ifchilddoc}
% \macro{\ifchilddocmanual}
% The conditional |\ifchilddoc| tells whether a
% child (true) or main (false) document is being compiled.
% The conditional |\ifchilddocmanual| tells whether
% the |\includeonly| mechanism is used (false) or
% the selection of child files must be performed manually (true).
% The definitions initialise to false:
%    \begin{macrocode}
\newif\ifchilddoc
\newif\ifchilddocmanual
%    \end{macrocode}

% \macro{\childdocname}
% \macro{\childdocjob}
% The macro |\childdocname| stores the name of the main document
% to be compiled. The macro |\childdocjob| stores the name of
% the document on which the \LaTeX{} compiler was originally invoked.
% The content of |\jobname| cannot be compared
% to filenames specified in the source due to different catcodes.
% The following code rescans |\jobname|, stores the result
% in |\childdocname| and saves a copy in |\childdocjob|:
%    \begin{macrocode}
\edef\childdocname{\scantokens\expandafter{\jobname\noexpand}}
\let\childdocjob\childdocname
%    \end{macrocode}

% \macro{\childdocdisable}
% The macro |\childdocdisable| prevents the main file
% from being processed more than once.
% At this stage, the main document command |\childdocmain|
% is assumed to be called once again where it should do nothing.
% Any subsequent call to it should prevent
% a secondary processing of the main document
% It overwrites the forwarding commands
% |\childdocof| and |\childdocforward|
% with empty macros to prevent further inclusions of the main document:
%    \begin{macrocode}
\newcommand{\childdocdisable}
{
  \renewcommand{\childdocmain}[1]{\renewcommand{\childdocmain}[1]{\endinput}}
  \renewcommand{\childdocof}[1]{}
  \renewcommand{\childdocby}[2][]{}
  \renewcommand{\childdocforward}[2][]{}
  \renewcommand{\childdocdisable}{}
}
%    \end{macrocode}

% \macro{\childdocmain}
% The macro |\childdocmain| is to be called at the top of the main file
% with nothing or the main filename (without extension) as argument.
% First, it breaks loops.
% If the argument is not empty and does not match |\childdocname|
% (which is set by the first inclusion of |childdoc.def|),
% |\ifchilddoc| is set to true, |\includeonly| is applied to the child file
% and |\jobname| is set to the main file
% (for proper handling of |.aux| files):
%    \begin{macrocode}
\newcommand{\childdocmain}[1]
{
  \childdocdisable\childdocmain{}
  \if?#1?\else
    \begingroup
      \def\childdoctmp{#1}
      \ifx\childdoctmp\childdocname
        \def\childdoctmp{}
      \else
        \def\childdoctmp
        {
          \childdoctrue
          \includeonly{\childdocname}
          \def\childdocjob{#1}
          \def\jobname{#1}
        }
      \fi
      \expandafter
    \endgroup
    \childdoctmp
  \fi
}
%    \end{macrocode}

% \macro{\childdocof}
% The command |\childdocof| redirects
% compilation to the main file |#1|.
%    \begin{macrocode}
\newcommand{\childdocof}[1]
{
  \childdocdisable
  \childdoctrue
  \includeonly{\childdocname}
  \def\jobname{#1}
  \def\childdocjob{#1}
  \input{#1}
}
%    \end{macrocode}

% \macro{\childdocby}
% The command |\childdocby| ....
%    \begin{macrocode}
\newcommand{\childdocby}[2][]
{
  \childdocdisable
  \childdoctrue
  \childdocmanualtrue
  \if?#1?\else
    \def\jobname{#2}
  \fi
  \def\childdocjob{#2}
  \input{#2}
  \endinput
}
%    \end{macrocode}

% \macro{\childdocforward}
% The command |\childdocforward| redirects
% compilation to the main file or
% (if the optional argument is given) a child file.
% Parameters are set as if the main file
% or a child file starting with |\childdocof| was compiled.
% Then compilation is handed over to the main file:
%    \begin{macrocode}
\newcommand{\childdocforward}[2][]
{
  \begingroup
    \if?#1?
      \def\childdoctmp
      {
        \def\childdocname{#2}
        \def\childdocjob{#2}
        \def\jobname{#2}
        \input{#2}
        \endinput
      }
    \else
      \def\childdoctmp
      {
        \childdocdisable
        \def\childdocname{#2}
        \childdoctrue
        \includeonly{#2}
        \def\childdocjob{#1}
        \def\jobname{#1}
        \input{#1}
        \endinput
      }
    \fi
    \expandafter
  \endgroup
  \childdoctmp
}
%    \end{macrocode}

% \macro{\childdocforwardprefix}
% The command |\childdocforwardprefix| redirects
% compilation to the main or a child file by means of a pattern.
% The prefix |#1| in the current filename is replaced by |#2|
% and the suffix of the current filename is kept
% (it is assumed that the filename does not contain the substring `|~~~|'
% which is used as a delimiter).
% Compilation is handed over to the new file by |\childdocforward|:
%    \begin{macrocode}
\newcommand{\childdocforwardprefix}[3][]
{
  \begingroup
    \def\childdocextract #2##1~~~{\def\childdoctmp{\childdocforward[#1]{#3##1}}}
    \expandafter\childdocextract\childdocname~~~
    \expandafter
  \endgroup
  \childdoctmp
}
%    \end{macrocode}

% \macro{\childdoc}
% The deprecated macro |\childdoc| is a legacy version of |\childdocmain|:
%    \begin{macrocode}
\newcommand{\childdoc}{\childdocmain}
%    \end{macrocode}

% \macro{\childdocredirect}
% The deprecated macro |\childdocredirect| is a legacy version
% of |\childdocforward| and |\childdocforwardprefix|:
%    \begin{macrocode}
\newcommand{\childdocredirect}[2][]
{
  \begingroup
    \if?#1?
      \def\childdoctmp{\childdocforward{#2}}
    \else
      \def\childdoctmp{\childdocforwardprefix{#1}{#2}}
    \fi
    \expandafter
  \endgroup
  \childdoctmp
}
%    \end{macrocode}

%\iffalse
%</package>
%\fi
%
\endinput
|\\
|\childdocmain{|\textit{main}|}|\\
\end{tabular}
\end{center}
%
If |\jobname| does not match the argument \textit{main} of |\childdocmain|,
it is assumed that |\jobname| points to the child file to be compiled.
When using |\childdocmain| with the main file specified as argument,
it suffices to start a child file
with just |\input{|\textit{main}|}|
without loading of the package and using |\childdocof|.
If instead all processing is done
with the appropriate \textsf{childdoc} directives,
the argument of \textit{main} of |\childdocmain| can be empty.

An alternative version of the command line processing described
in \secref{sec:commandline} using the detection mechanism reads:
%
\begin{center}
|... -jobname "|\textit{target}|" "|[\textit{flags}]%
[|\def\jobname{|\textit{dest}|}|]|\input{|\textit{main}|}"|
\end{center}

%%%%%%%%%%%%%%%%%%%%%%%%%%%%%%%%%%%%%%%%%%%%%%%%%%%%%%%%%%%%%%%%%%%%%%%%%%%%%%%%
\subsection{Manual Code}
\label{sec:manual}

In case one cannot be certain whether the definitions file |childdoc.def|
is installed on the target \TeX{} distribution
and one prefers not to ship it,
it is conceivable to paste a few relevant commands into the sources.

To that end, drop all statements |% \iffalse
%
% childdoc.dtx Copyright (C) 2017-2018 Niklas Beisert
%
% This work may be distributed and/or modified under the
% conditions of the LaTeX Project Public License, either version 1.3
% of this license or (at your option) any later version.
% The latest version of this license is in
%   http://www.latex-project.org/lppl.txt
% and version 1.3 or later is part of all distributions of LaTeX
% version 2005/12/01 or later.
%
% This work has the LPPL maintenance status `maintained'.
%
% The Current Maintainer of this work is Niklas Beisert.
%
% This work consists of the files childdoc.dtx and childdoc.ins
% and the derived files childdoc.def and cdocsamp.tex with
% cdocsch1.tex, cdocsch2.tex, cdocsdrf.tex, cdocsfn1.tex, cdocsfn2.tex.
%
%<package>\ifdefined\childdocmain\endinput\fi
%<package>\ProvidesFile{childdoc.def}[2018/12/30 v2.0 child document driver]
%<samplemain>\ProvidesFile{cdocsamp.tex}[2018/12/30 v2.0 sample for childdoc]
%<*driver>
%\ProvidesFile{childdoc.drv}[2018/12/30 v2.0 childdoc reference manual file]
\PassOptionsToClass{10pt,a4paper}{article}
\documentclass{ltxdoc}

\usepackage[margin=35mm]{geometry}
\usepackage{hyperref}
\usepackage{hyperxmp}
\usepackage[usenames]{color}

\hypersetup{colorlinks=true}
\hypersetup{pdfstartview=FitH}
\hypersetup{pdfpagemode=UseNone}
\hypersetup{pdfsource={}}
\hypersetup{pdflang={en-UK}}
\hypersetup{pdfcopyright={Copyright 2017-2018 Niklas Beisert.
  This work may be distributed and/or modified under the
  conditions of the LaTeX Project Public License, either version 1.3
  of this license or (at your option) any later version.}}
\hypersetup{pdflicenseurl={http://www.latex-project.org/lppl.txt}}
\hypersetup{pdfcontactaddress={ETH Zurich, ITP, HIT K,
  Wolfgang-Pauli-Strasse 27}}
\hypersetup{pdfcontactpostcode={8093}}
\hypersetup{pdfcontactcity={Zurich}}
\hypersetup{pdfcontactcountry={Switzerland}}
\hypersetup{pdfcontactemail={nbeisert@itp.phys.ethz.ch}}
\hypersetup{pdfcontacturl={http://people.phys.ethz.ch/\xmptilde nbeisert/}}

\newcommand{\secref}[1]{\hyperref[#1]{section \ref*{#1}}}

\parskip1ex
\parindent0pt
\let\olditemize\itemize
\def\itemize{\olditemize\parskip0pt}

\begin{document}

\title{The \textsf{childdoc} Package}
\hypersetup{pdftitle={The childdoc Package}}
\author{Niklas Beisert\\[2ex]
  Institut f\"ur Theoretische Physik\\
  Eidgen\"ossische Technische Hochschule Z\"urich\\
  Wolfgang-Pauli-Strasse 27, 8093 Z\"urich, Switzerland\\[1ex]
  \href{mailto:nbeisert@itp.phys.ethz.ch}
  {\texttt{nbeisert@itp.phys.ethz.ch}}}
\hypersetup{pdfauthor={Niklas Beisert}}
\hypersetup{pdfsubject={Manual for the LaTeX2e Package childdoc}}
\date{30 December 2018, \textsf{v2.0}}
\maketitle

\begin{abstract}\noindent
\textsf{childdoc} is a \LaTeXe{} package
that enables the direct compilation
of document sections included by |\include|
to individual files.
\end{abstract}

\begingroup
\parskip0ex
\tableofcontents
\endgroup

%%%%%%%%%%%%%%%%%%%%%%%%%%%%%%%%%%%%%%%%%%%%%%%%%%%%%%%%%%%%%%%%%%%%%%%%%%%%%%%%
%%%%%%%%%%%%%%%%%%%%%%%%%%%%%%%%%%%%%%%%%%%%%%%%%%%%%%%%%%%%%%%%%%%%%%%%%%%%%%%%
\section{Introduction}

\LaTeX{} provides a mechanism to structure a large document (such as a book)
into a main file and several child files (containing the chapters)
using the |\include| command.
This mechanism is beneficial for documents
which span hundreds of pages in order to
make the source file(s) more manageable.
Moreover, compilation can be restricted to
selected child files by means of the |\includeonly| command.
The latter feature can be used to reduce the compilation time while editing
(this was significantly more useful in the earlier days of \LaTeX{})
or to generate a smaller document which is easier to navigate.
Another application of |\includeonly| is to generate
documents consisting of selected parts of the complete document.

However, there are a few drawbacks of the plain |\include| mechanism:
\begin{itemize}
\item
The child files cannot be compiled on their own,
they can only be compiled via the main file.
A naive editing environment
(such as a text editor with an option
to have the current file processed by \LaTeX)
may require one to switch to the main file before compiling;
attempting to compile the child file produces errors.
\item
The main file must be modified (each time)
to adjust the |\includeonly| command
to the present needs. This easily leaves the main file in a messy state.
\item
The generated document will always carry the filename
of the main document. This is inconvenient if
several child files are to be compiled and
to be kept for distribution.
\end{itemize}

The present package provides a simple interface
to make child files individually compilable by \LaTeX{}.
Compiling a child file then has the same effect as compiling
the main file with an |\includeonly| command
to select the appropriate child.
Moreover the generated document will carry the name of the child
rather than the main file.
This resolves all three above issues.

This feature is meant to make the editing of books,
thesis documents and lecture notes somewhat more convenient.
However, the package can also be used efficiently for
composing a series of documents (such as exercise sheets)
which are typically distributed individually.
It then assists the author in generating the individual documents
(potentially in different versions)
as well as a document containing the collected series.
Another application is in developing style files
or other kinds of included material
where compilation of the style file could redirect
to a sample or test file.

%%%%%%%%%%%%%%%%%%%%%%%%%%%%%%%%%%%%%%%%%%%%%%%%%%%%%%%%%%%%%%%%%%%%%%%%%%%%%%%%
%%%%%%%%%%%%%%%%%%%%%%%%%%%%%%%%%%%%%%%%%%%%%%%%%%%%%%%%%%%%%%%%%%%%%%%%%%%%%%%%
\section{Usage}

First of all, the package \textsf{childdoc} is \emph{not} a standard
\LaTeXe{} |.sty| style file! Therefore it needs to be invoked in
a non-standard way.

%%%%%%%%%%%%%%%%%%%%%%%%%%%%%%%%%%%%%%%%%%%%%%%%%%%%%%%%%%%%%%%%%%%%%%%%%%%%%%%%
\subsection{Included Files}
\label{sec:include}

%%%%%%%%%%%%%%%%%%%%%%%%%%%%%%%%%%%%%%%%
\DescribeMacro{\childdocmain}
To use the package, add the commands
\begin{center}
\begin{tabular}{l}
|\input{childdoc.def}|\\
|\childdocmain{}|\\
\end{tabular}
\end{center}
at the very top of the main \LaTeX{} file,
in particular \emph{before} the |\documentclass| statement!
The argument of |\childdocmain| should be left empty
(but it must be present).

%%%%%%%%%%%%%%%%%%%%%%%%%%%%%%%%%%%%%%%%
\DescribeMacro{\childdocof}
Furthermore, add the commands
\begin{center}
\begin{tabular}{l}
|\input{childdoc.def}|\\
|\childdocof{|\textit{main}|}|\\
\end{tabular}
\end{center}
at the top of every child file \textit{child}
which is included by |\include{|\textit{child}|}|
from within the main file
(or at least for those files to be compiled individually).
The argument \textit{main} must be the filename of the main file.

There are a couple of
considerations in setting up the main and child documents:

%%%%%%%%%%%%%%%%%%%%%%%%%%%%%%%%%%%%%%%%
\paragraph{Restrictions.}

Please note the following restrictions:
\begin{itemize}
\item
|\childdocmain| must be called with one argument \textit{main}
to ensure compatibility with earlier version of the package.
It must either be empty (|\childdocmain{}|)
or precisely match the filename of the main file in which it is specified.
See \secref{sec:detection} for further information.
\item
The filename \textit{main} must be specified without the |.tex| extension.
\item
The filename \textit{main} is case sensitive
(even in case-insensitive file systems)
due to internal string comparison.
\item
The argument \textit{main} should be fully expanded, it cannot be a macro.
\item
Subdirectories and special characters should be avoided in filenames.
\item
The command |\childdocmain{|\textit{main}|}| must be followed by a whitespace.
It should not be followed immediately by another command
or by a comment mark `|%|'.
This is because the \TeX{} parser reads the token immediately following
the argument of |\childdocmain| and puts it
at the beginning of every child section;
however, a white\-space is ignored.
\end{itemize}

%%%%%%%%%%%%%%%%%%%%%%%%%%%%%%%%%%%%%%%%
\paragraph{Content of Main File.}

It is advisable to place all content in the child files included by |\include|.
Any output contained in the main file will appear in all child documents
unless suppressed manually;
it cannot be suppressed automatically by the |\includeonly| directive
and thus should normally be avoided.
A method to include some content in the main file
by means of conditional processing is described in \secref{sec:conditional}.

%%%%%%%%%%%%%%%%%%%%%%%%%%%%%%%%%%%%%%%%
\paragraph{Page Numbering.}

When only a part of the document is compiled,
the appropriate numbering of pages
(as well as other status parameters)
is determined from the |.aux| files.
The latter contain information from previous passes.
However this information needs to propagate through
all intermediate child documents.
Therefore the page numbering in child documents may well
be inconsistent until the complete document is compiled at least once.

A useful (if unconventional) way to always ensure a consistent
page numbering is to restart the numbering in each child document
and denote the pages by `\textit{child}|.|\textit{page}'
where \textit{child} represents the chapter/section number of the child file.
This can be achieved by the command
|\numberwithin{page}{|\textit{child}|}|
of the \textsf{amsmath} package
where \textit{child} can be |chapter| or |section|
depending on the chosen structuring.
Alternatively, one can modify the macro |\thepage| appropriately
and reset the counter |page| at the start of each child file.

%%%%%%%%%%%%%%%%%%%%%%%%%%%%%%%%%%%%%%%%%%%%%%%%%%%%%%%%%%%%%%%%%%%%%%%%%%%%%%%%
\subsection{Conditional Processing}
\label{sec:conditional}

The package provides a mechanism to compile different versions
of a document. To customise the versions further some conditional processing
can come in handy to distinguish which version is being compiled.
The package provides two macros to describe the compilation context:

%%%%%%%%%%%%%%%%%%%%%%%%%%%%%%%%%%%%%%%%
\DescribeMacro{\ifchilddoc}
The conditional |\ifchilddoc| distinguishes between the compilation of
child documents and the main document:
%
\begin{center}
|\ifchilddoc |\textit{child-code}| |[|\||else |\textit{main-code}]| \||fi|
\end{center}

%%%%%%%%%%%%%%%%%%%%%%%%%%%%%%%%%%%%%%%%
\DescribeMacro{\childdocname}
\DescribeMacro{\childdocjob}
The macro |\childdocname| contains the filename (without extension)
of the main or child file being processed.
Note that |\childdocjob| will always contain the name of the main file.

%%%%%%%%%%%%%%%%%%%%%%%%%%%%%%%%%%%%%%%%
\paragraph{Title Page.}

Conditional processing can be used to include a title or banner page
in the main document when proper precautions are taken.
Importantly, the code in the main file should ensure that the page counter
(as well as other status parameters which are stored in the |.aux| files)
takes the same value after the conditional processing.
Otherwise the page numbers may take divergent values
depending on which part is compiled.

For example, a title page could be declared by:
%
\begin{center}
\begin{tabular}{l}
|\ifchilddoc\||else|\\
|\addtocounter{page}{-1}|\\
\textit{code for title page}\\
|\newpage|\\
|\||fi|
\end{tabular}
\end{center}
%
A banner page for the child documents can be generated by:
%
\begin{center}
\begin{tabular}{l}
|\ifchilddoc|\\
|\addtocounter{page}{-1}|\\
\textit{code for banner page}\\
|\newpage|\\
|\||fi|
\end{tabular}
\end{center}
%
Here one could write a message such as:
\begin{center}
|This is the part \childdocname{} of \childdocjob{}.|
\end{center}

%%%%%%%%%%%%%%%%%%%%%%%%%%%%%%%%%%%%%%%%%%%%%%%%%%%%%%%%%%%%%%%%%%%%%%%%%%%%%%%%
\subsection{Flags}
\label{sec:flags}

The package makes it easy to generate different versions
of the main or child documents.
To this end compilation flags can be defined
and assigned different default values.
They will be particularly useful in conjunction
with the forwarding mechanism described in \secref{sec:forward}.

For example, it may be useful to have a flag |\version|
which can be set to |draft| or |final|.
The document source will contain some conditional code
depending on the value of |\version|.
Suppose further, the flag should default to |final| for the main file
and to |draft| for child files
which is a natural assignment for editing the document.
This is achieved by placing the following code
in the preamble of the main document
(below the |\childdocmain| directive):
%
\begin{center}
\begin{tabular}{l}
|\ifchilddoc|\\
|\providecommand{\version}{draft}|\\
|\||else|\\
|\providecommand{\version}{final}|\\
|\||fi|
\end{tabular}
\end{center}
%
The definition by |\providecommand| makes sure
that previous definitions are not overwritten.
Further statements |\providecommand{\version}{...}|
can thus be added before the above code to override it.

For the main file, one might add a line
(between |\childdocmain| and the above block)
%
\begin{center}
|%\ifchilddoc\||else\providecommand{\version}{draft}\||fi|
\end{center}
%
which can be uncommented to produce a draft version.
Likewise one can add a line to the very top of a child file
(above the |\childdocof{|\textit{main}|}| directive)
%
\begin{center}
|%\providecommand{\version}{final}|
\end{center}
%
which can be uncommented to produce the final version of this child document.

%%%%%%%%%%%%%%%%%%%%%%%%%%%%%%%%%%%%%%%%%%%%%%%%%%%%%%%%%%%%%%%%%%%%%%%%%%%%%%%%
\subsection{Forwarding}
\label{sec:forward}

Different versions of the main or child documents
using compilation flags as described in \secref{sec:flags}
can be (permanently) stored in different files
for convenient compilation, viewing and distribution.
To this end, the package defines a command
to pass on compilation to a different file:

%%%%%%%%%%%%%%%%%%%%%%%%%%%%%%%%%%%%%%%%
\DescribeMacro{\childdocforward}
The command |\childdocforward| redirects processing to
another source file:
%
\begin{center}
\begin{tabular}{l}
|\input{childdoc.def}|\\
|\childdocforward[|\textit{main}|]{|\textit{dest}|}|\\
\end{tabular}
\end{center}
%
The argument \textit{dest} is the destination file
(without extension).
It should be the main file or one of the child files.
Note that further \textsf{childdoc} directives
such as |\childdocof| and |\childdocforward|
in the indicated file will be processed in this form.
The optional argument \textit{main}
passes on directly to the main file \textit{main}
while pretending to compile the child \textit{dest}.
This form behaves as if \textit{dest}
issues |\childdocof{|\textit{main}|}| right away,
and no further \textsf{childdoc} directives will be processed.

%%%%%%%%%%%%%%%%%%%%%%%%%%%%%%%%%%%%%%%%
\DescribeMacro{\...prefix}
In the alternative form |\childdocforwardprefix|,
%
\begin{center}
\begin{tabular}{l}
|\input{childdoc.def}|\\
|\childdocforwardprefix[|\textit{main}|]{|\textit{prefix}|}{|\textit{dest}|}|
\end{tabular}
\end{center}
%
the destination file is determined by a pattern
depending on the current file:
To make this work, the current file must be called
`{\textit{prefix}\hspace{0.2em}\textit{suffix}}'
with \textit{prefix} matching precisely the argument.
Processing is then passed on to the file
`{\textit{dest}\hspace{0.2em}\textit{suffix}}'.
Surely, the same effect is achieved by
directly specifying the
argument `{\textit{dest}\hspace{0.2em}\textit{suffix}}'
in the first form.
However, that requires to set up a different file
for each child. With the alternative form of the command
all these files can have exactly the same content
which simplifies setting them up and maintaining them.

For example, the following file |draft.tex|
with a compilation flag |\version| as described in \secref{sec:flags}
compiles the main document as a draft:
%
\begin{center}
\begin{tabular}{l}
|\def\version{draft}|\\
|\input{childdoc.def}|\\
|\childdocforward{|\textit{main}|}|
\end{tabular}
\end{center}
%
Likewise, the following files |final|\textit{nn}|.tex|
compile the final version of the child document
|child|\textit{nn}|.tex|:
%
\begin{center}
\begin{tabular}{l}
|\def\version{final}|\\
|\input{childdoc.def}|\\
|\childdocforwardprefix{final}{child}|
\end{tabular}
\end{center}
%

Note that when several versions of a main file and/or of each child file
are to be generated, it may be convenient to set up a |Makefile| or
shell script to automatise the process.

%%%%%%%%%%%%%%%%%%%%%%%%%%%%%%%%%%%%%%%%%%%%%%%%%%%%%%%%%%%%%%%%%%%%%%%%%%%%%%%%
\subsection{Command Line Processing}
\label{sec:commandline}

The effect of redirection files can also be achieved by invoking
the \LaTeX{} compiler with a more elaborate command line.
Most conveniently this should be done as part
of a shell script or a |Makefile|.

When using \textsf{childdoc} in the main file, the following
command lines effectively perform a redirection
(note that depending on the shell being used,
backslashes may have to be doubled: `|\|' $\to$ `|\\|'):
%
\begin{center}
|... -jobname "|\textit{target}|" |\\|"|[\textit{flags}]%
|\input{childdoc.def}\childdocforward[|\textit{main}|]{|\textit{dest}|}"|
\end{center}
%
Here \textit{target} is the name of the output file,
\textit{main} is the name of the main file
and \textit{dest} is the name of the main or child file to be processed
(all filenames without extensions).
The optional argument \textit{main} can be omitted
if \textit{main} matches \textit{dest}.
Optionally, compilation \textit{flags} can be defined via |\def| commands.
This command line makes the \TeX{} engine believe
it is compiling the file \textit{target}
whose content is specified as the latter parameter.
The provided code then forwards the processing to
\textit{main} or \textit{dest} as described in \secref{sec:forward}.

%%%%%%%%%%%%%%%%%%%%%%%%%%%%%%%%%%%%%%%%%%%%%%%%%%%%%%%%%%%%%%%%%%%%%%%%%%%%%%%%
\subsection{Include by Input}
\label{sec:input}

Including child documents by |\include| has some restrictions by design.
Most notably, the content of a child document always occupies
its own set of pages; pages cannot be shared between child documents.
Usually, this behaviour makes perfect sense
because each child document contain an essential part of the document.
However, in some situations it may be desirable to compose
a document from a collection of parts
without having mandatory page breaks between then.
For this case, the package
provides a mechanism to include parts
by |\input| which can also be processed individually.
However, by construction this mechanism
requires manual handling of the content to be output.

%%%%%%%%%%%%%%%%%%%%%%%%%%%%%%%%%%%%%%%%
\DescribeMacro{\ifchilddocmanual}
The main file should be prepared as usual, see \secref{sec:include}.
However, the document body must make a distinction
between processing of an individual part and of the main document, e.g.:
%
\begin{center}
\begin{tabular}{l}
|\ifchilddocmanual|\\
|\input{\childdocname}|\\
|\||else|\\
\textit{document body with }|\input{|\textit{part}|}|\\
|\||fi|
\end{tabular}
\end{center}
%
The conditional |\ifchilddocmanual| is true whenever
a part to be included by |\input| is being compiled,
and the name of the part is stored in |\childdocname|.

%%%%%%%%%%%%%%%%%%%%%%%%%%%%%%%%%%%%%%%%
\DescribeMacro{\childdocby}
Each part to be included by |\input| should start with:
%
\begin{center}
\begin{tabular}{l}
|\input{childdoc.def}|\\
|\childdocby{|\textit{main}|}|\\
\end{tabular}
\end{center}
%
The directive |\childdocby| is similar to |\childdocof|
described in \secref{sec:include},
but the subsequent selection of content must be done manually.
To that end, both |\ifchilddoc| and |\ifchilddocmanual|
will be true upon processing of a part,
and the name of the part is stored in |\childdocname|.
Note that |\jobname| will be set to the filename of the current part
so that each part receives an individual |.aux| file
that does not interfere with the |.aux| file(s) of the main document.
This behaviour can be altered by the alternative form
|\childdocby[*]{|\textit{main}|}| (with a non-empty optional argument)
which uses the |.aux| file of the main document
by setting |\jobname| to \textit{main}.

%%%%%%%%%%%%%%%%%%%%%%%%%%%%%%%%%%%%%%%%%%%%%%%%%%%%%%%%%%%%%%%%%%%%%%%%%%%%%%%%
\subsection{Driver Development}
\label{sec:driver}

The \textsf{childdoc} mechanism can also be use for the development
of definition files such as \LaTeX{} styles or classes.
This case differs from the above setup with multiple parts
included by |\include| in that no |\includeonly| should be invoked.
This can be achieved by starting the include file
(before |\ProvidesPackage|) with:
%
\begin{center}
\begin{tabular}{l}
|\input{childdoc.def}|\\
|\childdocforward{|\textit{main}|}|\\
\end{tabular}
\end{center}
%
or alternatively with:
%
\begin{center}
\begin{tabular}{l}
|\input{childdoc.def}|\\
|\childdocby{|\textit{main}|}|\\
\end{tabular}
\end{center}
%
Both forms have slightly different effects as described above.
The main file is prepared as usual, see \secref{sec:include}.

%%%%%%%%%%%%%%%%%%%%%%%%%%%%%%%%%%%%%%%%%%%%%%%%%%%%%%%%%%%%%%%%%%%%%%%%%%%%%%%%
\subsection{Legacy Detection}
\label{sec:detection}

The directive |\childdocmain| in the main file can detect
whether the complete document or merely a child is to be compiled
even without using the directive |\childdocof|.
This method is deprecated because it is less robust
and there is no compelling reason to use it;
it is merely provided for backward compatibility
and it may be removed in future versions.

If the detection mechanism is to be used,
it is mandatory to correctly specify
the filename of the main file as the argument of |\childdocmain|:
%
\begin{center}
\begin{tabular}{l}
|\input{childdoc.def}|\\
|\childdocmain{|\textit{main}|}|\\
\end{tabular}
\end{center}
%
If |\jobname| does not match the argument \textit{main} of |\childdocmain|,
it is assumed that |\jobname| points to the child file to be compiled.
When using |\childdocmain| with the main file specified as argument,
it suffices to start a child file
with just |\input{|\textit{main}|}|
without loading of the package and using |\childdocof|.
If instead all processing is done
with the appropriate \textsf{childdoc} directives,
the argument of \textit{main} of |\childdocmain| can be empty.

An alternative version of the command line processing described
in \secref{sec:commandline} using the detection mechanism reads:
%
\begin{center}
|... -jobname "|\textit{target}|" "|[\textit{flags}]%
[|\def\jobname{|\textit{dest}|}|]|\input{|\textit{main}|}"|
\end{center}

%%%%%%%%%%%%%%%%%%%%%%%%%%%%%%%%%%%%%%%%%%%%%%%%%%%%%%%%%%%%%%%%%%%%%%%%%%%%%%%%
\subsection{Manual Code}
\label{sec:manual}

In case one cannot be certain whether the definitions file |childdoc.def|
is installed on the target \TeX{} distribution
and one prefers not to ship it,
it is conceivable to paste a few relevant commands into the sources.

To that end, drop all statements |\input{childdoc.def}|
and perform the replacements as outlined below.
Instead of |\childdocmain{|\textit{main}|}| add the following code
to the top of the main file:
%
\begin{center}
\begin{tabular}{l}
|\||ifdefined\childdocname\endinput\||fi\newif\ifchilddoc|\\
|\edef\childdocname{\scantokens\expandafter{\jobname\noexpand}}|\\
|\def\childdocmain{|\textit{main}|}\||ifx\childdocmain\childdocname\||else|\\
|\childdoctrue\includeonly{\childdocname}\let\jobname\childdocmain\||fi|\\
\end{tabular}
\end{center}
%
Instead of |\childdocof{|\textit{main}|}| just include the main file
at the top of each child file:
%
\begin{center}
|\input{|\textit{main}|}|
\end{center}
%
A simple redirection |\childdocforward{|\textit{dest}|}| is achieved by:
%
\begin{center}
|\def\jobname{|\textit{dest}|}\input{\jobname}|
\end{center}
%
The redirection with prefix
|\childdocforwardprefix[|\textit{prefix}|]{|\textit{dest}|}|
is accomplished by:
%
\begin{center}
\begin{tabular}{l}
|{\edef\jobname{\scantokens\expandafter{\jobname\noexpand}}|\\
|\def\redirectjob |\textit{prefix}|#1~~~{\gdef\jobname{|\textit{dest}|#1}}|\\
|\expandafter\redirectjob\jobname~~~}\input{\jobname}|
\end{tabular}
\end{center}

In an alternative approach,
child documents can be compiled by a specific command line
without additional code or specific definitions:
%
\begin{center}
|... -jobname "|\textit{target}|" "|[\textit{flags}]%
|\includeonly{|\textit{dest}|}\input{|\textit{main}|}"|
\end{center}
%

%%%%%%%%%%%%%%%%%%%%%%%%%%%%%%%%%%%%%%%%%%%%%%%%%%%%%%%%%%%%%%%%%%%%%%%%%%%%%%%%
%%%%%%%%%%%%%%%%%%%%%%%%%%%%%%%%%%%%%%%%%%%%%%%%%%%%%%%%%%%%%%%%%%%%%%%%%%%%%%%%
\section{Information}

%%%%%%%%%%%%%%%%%%%%%%%%%%%%%%%%%%%%%%%%%%%%%%%%%%%%%%%%%%%%%%%%%%%%%%%%%%%%%%%%
\subsection{Copyright}

Copyright \copyright{} 2017--2018 Niklas Beisert

This work may be distributed and/or modified under the
conditions of the \LaTeX{} Project Public License, either version 1.3
of this license or (at your option) any later version.
The latest version of this license is in
  \url{http://www.latex-project.org/lppl.txt}
and version 1.3 or later is part of all distributions of \LaTeX{}
version 2005/12/01 or later.

This work has the LPPL maintenance status `maintained'.

The Current Maintainer of this work is Niklas Beisert.

This work consists of the files |README.txt|, |childdoc.ins| and |childdoc.dtx|
as well as the derived files |childdoc.def|, |cdocsamp.tex|
with |cdocsch1.tex|, |cdocsch2.tex|, |cdocspt3.tex|, |cdocspt4.tex|,
|cdocsdrf.tex|, |cdocsfn1.tex|, |cdocsfn2.tex|
as well as |childdoc.pdf|.

%%%%%%%%%%%%%%%%%%%%%%%%%%%%%%%%%%%%%%%%%%%%%%%%%%%%%%%%%%%%%%%%%%%%%%%%%%%%%%%%
\subsection{Files and Installation}

The package consists of the files:
%
\begin{center}
\begin{tabular}{ll}
    |README.txt|   & readme file \\
    |childdoc.ins| & installation file \\
    |childdoc.dtx| & source file \\
    |childdoc.def| & definition file \\
    |cdocsamp.tex| & sample main file \\
    |cdocsch1.tex| & sample include file \\
    |cdocsch2.tex| & sample include file \\
    |cdocspt3.tex| & sample part file \\
    |cdocspt4.tex| & sample part file \\
    |cdocsdrf.tex| & sample redirection file \\
    |cdocsfn1.tex| & sample redirection file \\
    |cdocsfn2.tex| & sample redirection file \\
    |childdoc.pdf| & manual
\end{tabular}
\end{center}
%
The distribution consists of the files
|README.txt|, |childdoc.ins| and |childdoc.dtx|.
%
\begin{itemize}
\item
Run (pdf)\LaTeX{} on |childdoc.dtx|
to compile the manual |childdoc.pdf| (this file).
\item
Run \LaTeX{} on |childdoc.ins| to create the definitions file |childdoc.def|
and the sample |cdocsamp.tex| with include files
|cdocsch1.tex|, |cdocsch2.tex|, |cdocspt3.tex|, |cdocspt4.tex|,
|cdocsdrf.tex|, |cdocsfn1.tex|, |cdocsfn2.tex|.
Then copy the file |childdoc.def| to an appropriate directory of your \LaTeX{}
distribution, e.g.\ \textit{texmf-root}|/tex/latex/childdoc|.
\end{itemize}

%%%%%%%%%%%%%%%%%%%%%%%%%%%%%%%%%%%%%%%%%%%%%%%%%%%%%%%%%%%%%%%%%%%%%%%%%%%%%%%%
\subsection{Related CTAN Packages}

There are several other packages which offer a similar functionality:
%
\begin{itemize}
\item
The packages
\href{http://ctan.org/pkg/docmute}{\textsf{docmute}},
\href{http://ctan.org/pkg/includex}{\textsf{includex}} and
\href{http://ctan.org/pkg/standalone}{\textsf{standalone}}
provide commands to include only the document body of
a child file thus allowing both files to be compiled individually.
\item
The packages \href{http://ctan.org/pkg/subdocs}{\textsf{subdocs}}
and \href{http://ctan.org/pkg/subfiles}{\textsf{subfiles}}
provide structures in which the main and child documents can be
encapsulated and allowing them to be compiled individually.
The inclusion mechanism is different from the conventional |\include|.
\item
The package \href{http://ctan.org/pkg/combine}{\textsf{combine}}
is an elaborate solution to combine several documents into one.
\end{itemize}
%
See also the CTAN topic \href{http://ctan.org/topic/subdocs}{\textsf{subdocs}}
for further related packages.
The present package differs from the above solutions in that
a document structure constructed with the conventional |\include| mechanism
just needs two extra commands at the top of every file
such that all constituent files can be compiled individually.

%%%%%%%%%%%%%%%%%%%%%%%%%%%%%%%%%%%%%%%%%%%%%%%%%%%%%%%%%%%%%%%%%%%%%%%%%%%%%%%%
%\subsection{Feature Suggestions}
%
%The following is a list of features which may be useful for future
%versions of this package:
%%
%\begin{itemize}
%\item
%\ldots
%\end{itemize}

%%%%%%%%%%%%%%%%%%%%%%%%%%%%%%%%%%%%%%%%%%%%%%%%%%%%%%%%%%%%%%%%%%%%%%%%%%%%%%%%
\subsection{Revision History}

%%%%%%%%%%%%%%%%%%%%%%%%%%%%%%%%%%%%%%%%
\paragraph{v2.0:} 2018/12/30

\begin{itemize}
\item
immediate forward processing
\item
added |\childdocby| mechanism
\item
manual restructured
\end{itemize}

%%%%%%%%%%%%%%%%%%%%%%%%%%%%%%%%%%%%%%%%
\paragraph{v1.6:} 2018/01/17

\begin{itemize}
\item
application for development of include files
\item
corrections to manual
\end{itemize}

%%%%%%%%%%%%%%%%%%%%%%%%%%%%%%%%%%%%%%%%
\paragraph{v1.5:} 2017/05/21

\begin{itemize}
\item
more complete structuring introduced
\item
|\childdocof| introduced
\item
|\childdoc| renamed to |\childdocmain|
\item
|\childredirect| renamed to |\childdocforward| and |\childdocforwardprefix|
and functionality expanded
\end{itemize}

%%%%%%%%%%%%%%%%%%%%%%%%%%%%%%%%%%%%%%%%
\paragraph{v1.0:} 2017/04/27

\begin{itemize}
\item
manual and install package
\item
first version published on CTAN
\end{itemize}

%%%%%%%%%%%%%%%%%%%%%%%%%%%%%%%%%%%%%%%%
\paragraph{v0.6:} 2017/04/26

\begin{itemize}
\item
redirection mechanism added
\end{itemize}

%%%%%%%%%%%%%%%%%%%%%%%%%%%%%%%%%%%%%%%%
\paragraph{v0.5:} 2017/04/26

\begin{itemize}
\item
functionality in definition file
\end{itemize}


%%%%%%%%%%%%%%%%%%%%%%%%%%%%%%%%%%%%%%%%%%%%%%%%%%%%%%%%%%%%%%%%%%%%%%%%%%%%%%%%
%%%%%%%%%%%%%%%%%%%%%%%%%%%%%%%%%%%%%%%%%%%%%%%%%%%%%%%%%%%%%%%%%%%%%%%%%%%%%%%%
%%%%%%%%%%%%%%%%%%%%%%%%%%%%%%%%%%%%%%%%%%%%%%%%%%%%%%%%%%%%%%%%%%%%%%%%%%%%%%%%
\appendix

\settowidth\MacroIndent{\rmfamily\scriptsize 000\ }

 \DocInput{childdoc.dtx}

\end{document}
%</driver>
% \fi
%
% %%%%%%%%%%%%%%%%%%%%%%%%%%%%%%%%%%%%%%%%%%%%%%%%%%%%%%%%%%%%%%%%%%%%%%%%%%%%%%
% %%%%%%%%%%%%%%%%%%%%%%%%%%%%%%%%%%%%%%%%%%%%%%%%%%%%%%%%%%%%%%%%%%%%%%%%%%%%%%
% \section{Sample}
%\iffalse
%<*samplemain>
%\fi
%
% The following presents a sample document
% with two chapters, two parts, a title page,
% a compile flag as well as three forwarding files to set the flag.
% It consists of eight |.tex| files:
% \begin{center}
% \begin{tabular}{ll}
% |cdocsamp.tex|&main file\\
% |cdocsch1.tex|&include file for chapter 1\\
% |cdocsch2.tex|&include file for chapter 2\\
% |cdocspt3.tex|&include file for part 3\\
% |cdocspt4.tex|&include file for part 4\\
% |cdocsdrf.tex|&forwarding file for main file in draft mode\\
% |cdocsfi1.tex|&forwarding file for final version of chapter 1\\
% |cdocsfi2.tex|&forwarding file for final version of chapter 2\\
% \end{tabular}
% \end{center}
% Each of the eight files can be compiled directly by the \LaTeX{} compiler.
%
% %%%%%%%%%%%%%%%%%%%%%%%%%%%%%%%%%%%%%%
% \paragraph{Main File.}
%
% The main file is called |cdocsamp.tex|.
%
% Load the \textsf{childdoc} definitions and
% declare the filename for the main document:
%    \begin{macrocode}
\input{childdoc.def}
\childdocmain{}
%    \end{macrocode}

% Optional override for |\version| flag:
%    \begin{macrocode}
%%\ifchilddoc\else\providecommand{\version}{draft}\fi
%    \end{macrocode}

% Define the default values for the |\version| flag
% (|final| for the main file and |draft| for childs):
%    \begin{macrocode}
\ifchilddoc
\providecommand{\version}{draft}
\else
\providecommand{\version}{final}
\fi
%    \end{macrocode}

% Load the standard document class:
%    \begin{macrocode}
\documentclass[12pt]{article}
%    \end{macrocode}

% Start the document body:
%    \begin{macrocode}
\begin{document}
%    \end{macrocode}

% Declare a title page.
% Print title, part of document being processed and version flag:
%    \begin{macrocode}
\addtocounter{page}{-1}
\begin{center}
{\LARGE\bfseries{}childdoc example\par}
\vspace{1cm}
\ifchilddoc
\ifchilddocmanual part\else chapter\fi:
`\childdocname' of `\childdocjob'\par
\else
main document: `\childdocjob'\par
\fi
version: \version\par
\end{center}
\newpage
%    \end{macrocode}

% Manually include selected file,
% otherwise process as usual:
%    \begin{macrocode}
\ifchilddocmanual
\section*{part `\childdocname'}
\input{\childdocname}
\else
%    \end{macrocode}

% Include the two chapters:
%    \begin{macrocode}
\include{cdocsch1}
\include{cdocsch2}
%    \end{macrocode}

% Include the two parts unless only chapters should be displayed:
%    \begin{macrocode}
\ifchilddoc\else
\section{part three}
\input{cdocspt3}
\section{part four}
\input{cdocspt4}
\fi
%    \end{macrocode}

% Process as usual until here:
%    \begin{macrocode}
\fi
%    \end{macrocode}

% End of document body:
%    \begin{macrocode}
\end{document}
%    \end{macrocode}
%\iffalse
%</samplemain>
%\fi
%
% %%%%%%%%%%%%%%%%%%%%%%%%%%%%%%%%%%%%%%
% \paragraph{Chapter Include Files.}
%
% The include files are called |cdocsch1.tex| and |cdocsch2.tex|.
%
%\iffalse
%<*samplechap1|samplechap2>
%\fi

% Optional override for |\version| flag:
%    \begin{macrocode}
%%\providecommand{\version}{final}
%    \end{macrocode}

% Include the main document:
%    \begin{macrocode}
\input{childdoc.def}
\childdocof{cdocsamp}
%    \end{macrocode}

%\iffalse
%</samplechap1|samplechap2>
%\fi
%
%\iffalse
%<*samplechap1>
%\fi
% Some text for chapter 1:
%    \begin{macrocode}
\section{one}
some text in chapter one
%    \end{macrocode}

%\iffalse
%</samplechap1>
%\fi
% Some text for chapter 2:
%\iffalse
%<*samplechap2>
%\fi
%    \begin{macrocode}
\section{two}
more text in chapter two
%    \end{macrocode}

%\iffalse
%</samplechap2>
%\fi
%
% %%%%%%%%%%%%%%%%%%%%%%%%%%%%%%%%%%%%%%
% \paragraph{Part Include Files.}
%
% The include files are called |cdocspt3.tex| and |cdocspt4.tex|.
%
%\iffalse
%<*samplepart3|samplepart4>
%\fi

% Optional override for |\version| flag:
%    \begin{macrocode}
%%\providecommand{\version}{final}
%    \end{macrocode}

% Include the main document:
%    \begin{macrocode}
\input{childdoc.def}
\childdocby{cdocsamp}
%    \end{macrocode}

%\iffalse
%</samplepart3|samplepart4>
%\fi
%
%\iffalse
%<*samplepart3>
%\fi
% Some text for part 3:
%    \begin{macrocode}
some text in part three
%    \end{macrocode}

%\iffalse
%</samplepart3>
%\fi
% Some text for part 4:
%\iffalse
%<*samplepart4>
%\fi
%    \begin{macrocode}
more text in part four
%    \end{macrocode}

%\iffalse
%</samplepart4>
%\fi
%
% %%%%%%%%%%%%%%%%%%%%%%%%%%%%%%%%%%%%%%
% \paragraph{Forwarding for a Complete Draft.}
%
% The following forwarding file |cdocsdrf.tex|
% compiles the main document in draft mode:
%\iffalse
%<*sampledraft>
%\fi
%    \begin{macrocode}
\def\version{draft}
\input{childdoc.def}
\childdocforward{cdocsamp}
%    \end{macrocode}

%\iffalse
%</sampledraft>
%\fi
%
% %%%%%%%%%%%%%%%%%%%%%%%%%%%%%%%%%%%%%%
% \paragraph{Forwarding for Final Version of the Chapters.}
%
% The following forwarding files |cdocsfn1.tex| and |cdocsfn2.tex|
% (with identical content)
% compile the final versions of the child documents
% |cdocsch1.tex| and |cdocsch2.tex|, respectively:
%\iffalse
%<*samplefinal>
%\fi
%    \begin{macrocode}
\def\version{final}
\input{childdoc.def}
\childdocforwardprefix[cdocsamp]{cdocsfn}{cdocsch}
%    \end{macrocode}

%\iffalse
%</samplefinal>
%\fi
%
% %%%%%%%%%%%%%%%%%%%%%%%%%%%%%%%%%%%%%%
% \paragraph{Command Line Processing.}
%
% The following three command lines generate the output files
% |cdocscld|, |cdocscl1| and |cdocscl2|
% which should be identical to
% |cdocsdrf|, |cdocsch1| and |cdocsfn2|, respectively:
% \begin{center}
% \begin{tabular}{l}
% |latex -jobname cdocscld \|\\
% |  "\def\version{draft}\input{childdoc.def}\childdocforward{cdocsamp}"|\\
% |latex -jobname cdocscl1 \|\\
% |  "\input{childdoc.def}\childdocforward[cdocsamp]{cdocsch1}"|\\
% |latex -jobname cdocscl2 \|\\
% |  "\def\version{final}\input{childdoc.def}\childdocforward{cdocsch2}"|
% \end{tabular}
% \end{center}
% Note that the trailing backslash on each first line
% merely continues the input to the second line
% (for convenient cut ant paste).
% Furthermore, the command |latex| can be replaced by any
% of its alternative versions such as |pdflatex|.
%
% %%%%%%%%%%%%%%%%%%%%%%%%%%%%%%%%%%%%%%%%%%%%%%%%%%%%%%%%%%%%%%%%%%%%%%%%%%%%%%
% %%%%%%%%%%%%%%%%%%%%%%%%%%%%%%%%%%%%%%%%%%%%%%%%%%%%%%%%%%%%%%%%%%%%%%%%%%%%%%
% \section{Implementation}
%\iffalse
%<*package>
%\fi
%
% This section describes the definitions file |childdoc.def|.

% The definitions cannot be loaded using |\usepackage| or |\RequirePackage|
% which has a mechanism to prevent loading a style file more than once.
% When loading the definitions by means of |\input|
% multiple instances have to be prevented manually:
%\iffalse
%This code needs to be before the `\ProvidesFile' directive
%which is defined at the beginning of this file.
%Therefore it is also placed there and commented out here.
%</package>
%<*discard>
%\fi
%    \begin{macrocode}
\ifdefined\childdocmain\endinput\fi
%    \end{macrocode}
%\iffalse
%</discard>
%<*package>
%\fi
%
% \macro{\ifchilddoc}
% \macro{\ifchilddocmanual}
% The conditional |\ifchilddoc| tells whether a
% child (true) or main (false) document is being compiled.
% The conditional |\ifchilddocmanual| tells whether
% the |\includeonly| mechanism is used (false) or
% the selection of child files must be performed manually (true).
% The definitions initialise to false:
%    \begin{macrocode}
\newif\ifchilddoc
\newif\ifchilddocmanual
%    \end{macrocode}

% \macro{\childdocname}
% \macro{\childdocjob}
% The macro |\childdocname| stores the name of the main document
% to be compiled. The macro |\childdocjob| stores the name of
% the document on which the \LaTeX{} compiler was originally invoked.
% The content of |\jobname| cannot be compared
% to filenames specified in the source due to different catcodes.
% The following code rescans |\jobname|, stores the result
% in |\childdocname| and saves a copy in |\childdocjob|:
%    \begin{macrocode}
\edef\childdocname{\scantokens\expandafter{\jobname\noexpand}}
\let\childdocjob\childdocname
%    \end{macrocode}

% \macro{\childdocdisable}
% The macro |\childdocdisable| prevents the main file
% from being processed more than once.
% At this stage, the main document command |\childdocmain|
% is assumed to be called once again where it should do nothing.
% Any subsequent call to it should prevent
% a secondary processing of the main document
% It overwrites the forwarding commands
% |\childdocof| and |\childdocforward|
% with empty macros to prevent further inclusions of the main document:
%    \begin{macrocode}
\newcommand{\childdocdisable}
{
  \renewcommand{\childdocmain}[1]{\renewcommand{\childdocmain}[1]{\endinput}}
  \renewcommand{\childdocof}[1]{}
  \renewcommand{\childdocby}[2][]{}
  \renewcommand{\childdocforward}[2][]{}
  \renewcommand{\childdocdisable}{}
}
%    \end{macrocode}

% \macro{\childdocmain}
% The macro |\childdocmain| is to be called at the top of the main file
% with nothing or the main filename (without extension) as argument.
% First, it breaks loops.
% If the argument is not empty and does not match |\childdocname|
% (which is set by the first inclusion of |childdoc.def|),
% |\ifchilddoc| is set to true, |\includeonly| is applied to the child file
% and |\jobname| is set to the main file
% (for proper handling of |.aux| files):
%    \begin{macrocode}
\newcommand{\childdocmain}[1]
{
  \childdocdisable\childdocmain{}
  \if?#1?\else
    \begingroup
      \def\childdoctmp{#1}
      \ifx\childdoctmp\childdocname
        \def\childdoctmp{}
      \else
        \def\childdoctmp
        {
          \childdoctrue
          \includeonly{\childdocname}
          \def\childdocjob{#1}
          \def\jobname{#1}
        }
      \fi
      \expandafter
    \endgroup
    \childdoctmp
  \fi
}
%    \end{macrocode}

% \macro{\childdocof}
% The command |\childdocof| redirects
% compilation to the main file |#1|.
%    \begin{macrocode}
\newcommand{\childdocof}[1]
{
  \childdocdisable
  \childdoctrue
  \includeonly{\childdocname}
  \def\jobname{#1}
  \def\childdocjob{#1}
  \input{#1}
}
%    \end{macrocode}

% \macro{\childdocby}
% The command |\childdocby| ....
%    \begin{macrocode}
\newcommand{\childdocby}[2][]
{
  \childdocdisable
  \childdoctrue
  \childdocmanualtrue
  \if?#1?\else
    \def\jobname{#2}
  \fi
  \def\childdocjob{#2}
  \input{#2}
  \endinput
}
%    \end{macrocode}

% \macro{\childdocforward}
% The command |\childdocforward| redirects
% compilation to the main file or
% (if the optional argument is given) a child file.
% Parameters are set as if the main file
% or a child file starting with |\childdocof| was compiled.
% Then compilation is handed over to the main file:
%    \begin{macrocode}
\newcommand{\childdocforward}[2][]
{
  \begingroup
    \if?#1?
      \def\childdoctmp
      {
        \def\childdocname{#2}
        \def\childdocjob{#2}
        \def\jobname{#2}
        \input{#2}
        \endinput
      }
    \else
      \def\childdoctmp
      {
        \childdocdisable
        \def\childdocname{#2}
        \childdoctrue
        \includeonly{#2}
        \def\childdocjob{#1}
        \def\jobname{#1}
        \input{#1}
        \endinput
      }
    \fi
    \expandafter
  \endgroup
  \childdoctmp
}
%    \end{macrocode}

% \macro{\childdocforwardprefix}
% The command |\childdocforwardprefix| redirects
% compilation to the main or a child file by means of a pattern.
% The prefix |#1| in the current filename is replaced by |#2|
% and the suffix of the current filename is kept
% (it is assumed that the filename does not contain the substring `|~~~|'
% which is used as a delimiter).
% Compilation is handed over to the new file by |\childdocforward|:
%    \begin{macrocode}
\newcommand{\childdocforwardprefix}[3][]
{
  \begingroup
    \def\childdocextract #2##1~~~{\def\childdoctmp{\childdocforward[#1]{#3##1}}}
    \expandafter\childdocextract\childdocname~~~
    \expandafter
  \endgroup
  \childdoctmp
}
%    \end{macrocode}

% \macro{\childdoc}
% The deprecated macro |\childdoc| is a legacy version of |\childdocmain|:
%    \begin{macrocode}
\newcommand{\childdoc}{\childdocmain}
%    \end{macrocode}

% \macro{\childdocredirect}
% The deprecated macro |\childdocredirect| is a legacy version
% of |\childdocforward| and |\childdocforwardprefix|:
%    \begin{macrocode}
\newcommand{\childdocredirect}[2][]
{
  \begingroup
    \if?#1?
      \def\childdoctmp{\childdocforward{#2}}
    \else
      \def\childdoctmp{\childdocforwardprefix{#1}{#2}}
    \fi
    \expandafter
  \endgroup
  \childdoctmp
}
%    \end{macrocode}

%\iffalse
%</package>
%\fi
%
\endinput
|
and perform the replacements as outlined below.
Instead of |\childdocmain{|\textit{main}|}| add the following code
to the top of the main file:
%
\begin{center}
\begin{tabular}{l}
|\||ifdefined\childdocname\endinput\||fi\newif\ifchilddoc|\\
|\edef\childdocname{\scantokens\expandafter{\jobname\noexpand}}|\\
|\def\childdocmain{|\textit{main}|}\||ifx\childdocmain\childdocname\||else|\\
|\childdoctrue\includeonly{\childdocname}\let\jobname\childdocmain\||fi|\\
\end{tabular}
\end{center}
%
Instead of |\childdocof{|\textit{main}|}| just include the main file
at the top of each child file:
%
\begin{center}
|\input{|\textit{main}|}|
\end{center}
%
A simple redirection |\childdocforward{|\textit{dest}|}| is achieved by:
%
\begin{center}
|\def\jobname{|\textit{dest}|}\input{\jobname}|
\end{center}
%
The redirection with prefix
|\childdocforwardprefix[|\textit{prefix}|]{|\textit{dest}|}|
is accomplished by:
%
\begin{center}
\begin{tabular}{l}
|{\edef\jobname{\scantokens\expandafter{\jobname\noexpand}}|\\
|\def\redirectjob |\textit{prefix}|#1~~~{\gdef\jobname{|\textit{dest}|#1}}|\\
|\expandafter\redirectjob\jobname~~~}\input{\jobname}|
\end{tabular}
\end{center}

In an alternative approach,
child documents can be compiled by a specific command line
without additional code or specific definitions:
%
\begin{center}
|... -jobname "|\textit{target}|" "|[\textit{flags}]%
|\includeonly{|\textit{dest}|}\input{|\textit{main}|}"|
\end{center}
%

%%%%%%%%%%%%%%%%%%%%%%%%%%%%%%%%%%%%%%%%%%%%%%%%%%%%%%%%%%%%%%%%%%%%%%%%%%%%%%%%
%%%%%%%%%%%%%%%%%%%%%%%%%%%%%%%%%%%%%%%%%%%%%%%%%%%%%%%%%%%%%%%%%%%%%%%%%%%%%%%%
\section{Information}

%%%%%%%%%%%%%%%%%%%%%%%%%%%%%%%%%%%%%%%%%%%%%%%%%%%%%%%%%%%%%%%%%%%%%%%%%%%%%%%%
\subsection{Copyright}

Copyright \copyright{} 2017--2018 Niklas Beisert

This work may be distributed and/or modified under the
conditions of the \LaTeX{} Project Public License, either version 1.3
of this license or (at your option) any later version.
The latest version of this license is in
  \url{http://www.latex-project.org/lppl.txt}
and version 1.3 or later is part of all distributions of \LaTeX{}
version 2005/12/01 or later.

This work has the LPPL maintenance status `maintained'.

The Current Maintainer of this work is Niklas Beisert.

This work consists of the files |README.txt|, |childdoc.ins| and |childdoc.dtx|
as well as the derived files |childdoc.def|, |cdocsamp.tex|
with |cdocsch1.tex|, |cdocsch2.tex|, |cdocspt3.tex|, |cdocspt4.tex|,
|cdocsdrf.tex|, |cdocsfn1.tex|, |cdocsfn2.tex|
as well as |childdoc.pdf|.

%%%%%%%%%%%%%%%%%%%%%%%%%%%%%%%%%%%%%%%%%%%%%%%%%%%%%%%%%%%%%%%%%%%%%%%%%%%%%%%%
\subsection{Files and Installation}

The package consists of the files:
%
\begin{center}
\begin{tabular}{ll}
    |README.txt|   & readme file \\
    |childdoc.ins| & installation file \\
    |childdoc.dtx| & source file \\
    |childdoc.def| & definition file \\
    |cdocsamp.tex| & sample main file \\
    |cdocsch1.tex| & sample include file \\
    |cdocsch2.tex| & sample include file \\
    |cdocspt3.tex| & sample part file \\
    |cdocspt4.tex| & sample part file \\
    |cdocsdrf.tex| & sample redirection file \\
    |cdocsfn1.tex| & sample redirection file \\
    |cdocsfn2.tex| & sample redirection file \\
    |childdoc.pdf| & manual
\end{tabular}
\end{center}
%
The distribution consists of the files
|README.txt|, |childdoc.ins| and |childdoc.dtx|.
%
\begin{itemize}
\item
Run (pdf)\LaTeX{} on |childdoc.dtx|
to compile the manual |childdoc.pdf| (this file).
\item
Run \LaTeX{} on |childdoc.ins| to create the definitions file |childdoc.def|
and the sample |cdocsamp.tex| with include files
|cdocsch1.tex|, |cdocsch2.tex|, |cdocspt3.tex|, |cdocspt4.tex|,
|cdocsdrf.tex|, |cdocsfn1.tex|, |cdocsfn2.tex|.
Then copy the file |childdoc.def| to an appropriate directory of your \LaTeX{}
distribution, e.g.\ \textit{texmf-root}|/tex/latex/childdoc|.
\end{itemize}

%%%%%%%%%%%%%%%%%%%%%%%%%%%%%%%%%%%%%%%%%%%%%%%%%%%%%%%%%%%%%%%%%%%%%%%%%%%%%%%%
\subsection{Related CTAN Packages}

There are several other packages which offer a similar functionality:
%
\begin{itemize}
\item
The packages
\href{http://ctan.org/pkg/docmute}{\textsf{docmute}},
\href{http://ctan.org/pkg/includex}{\textsf{includex}} and
\href{http://ctan.org/pkg/standalone}{\textsf{standalone}}
provide commands to include only the document body of
a child file thus allowing both files to be compiled individually.
\item
The packages \href{http://ctan.org/pkg/subdocs}{\textsf{subdocs}}
and \href{http://ctan.org/pkg/subfiles}{\textsf{subfiles}}
provide structures in which the main and child documents can be
encapsulated and allowing them to be compiled individually.
The inclusion mechanism is different from the conventional |\include|.
\item
The package \href{http://ctan.org/pkg/combine}{\textsf{combine}}
is an elaborate solution to combine several documents into one.
\end{itemize}
%
See also the CTAN topic \href{http://ctan.org/topic/subdocs}{\textsf{subdocs}}
for further related packages.
The present package differs from the above solutions in that
a document structure constructed with the conventional |\include| mechanism
just needs two extra commands at the top of every file
such that all constituent files can be compiled individually.

%%%%%%%%%%%%%%%%%%%%%%%%%%%%%%%%%%%%%%%%%%%%%%%%%%%%%%%%%%%%%%%%%%%%%%%%%%%%%%%%
%\subsection{Feature Suggestions}
%
%The following is a list of features which may be useful for future
%versions of this package:
%%
%\begin{itemize}
%\item
%\ldots
%\end{itemize}

%%%%%%%%%%%%%%%%%%%%%%%%%%%%%%%%%%%%%%%%%%%%%%%%%%%%%%%%%%%%%%%%%%%%%%%%%%%%%%%%
\subsection{Revision History}

%%%%%%%%%%%%%%%%%%%%%%%%%%%%%%%%%%%%%%%%
\paragraph{v2.0:} 2018/12/30

\begin{itemize}
\item
immediate forward processing
\item
added |\childdocby| mechanism
\item
manual restructured
\end{itemize}

%%%%%%%%%%%%%%%%%%%%%%%%%%%%%%%%%%%%%%%%
\paragraph{v1.6:} 2018/01/17

\begin{itemize}
\item
application for development of include files
\item
corrections to manual
\end{itemize}

%%%%%%%%%%%%%%%%%%%%%%%%%%%%%%%%%%%%%%%%
\paragraph{v1.5:} 2017/05/21

\begin{itemize}
\item
more complete structuring introduced
\item
|\childdocof| introduced
\item
|\childdoc| renamed to |\childdocmain|
\item
|\childredirect| renamed to |\childdocforward| and |\childdocforwardprefix|
and functionality expanded
\end{itemize}

%%%%%%%%%%%%%%%%%%%%%%%%%%%%%%%%%%%%%%%%
\paragraph{v1.0:} 2017/04/27

\begin{itemize}
\item
manual and install package
\item
first version published on CTAN
\end{itemize}

%%%%%%%%%%%%%%%%%%%%%%%%%%%%%%%%%%%%%%%%
\paragraph{v0.6:} 2017/04/26

\begin{itemize}
\item
redirection mechanism added
\end{itemize}

%%%%%%%%%%%%%%%%%%%%%%%%%%%%%%%%%%%%%%%%
\paragraph{v0.5:} 2017/04/26

\begin{itemize}
\item
functionality in definition file
\end{itemize}


%%%%%%%%%%%%%%%%%%%%%%%%%%%%%%%%%%%%%%%%%%%%%%%%%%%%%%%%%%%%%%%%%%%%%%%%%%%%%%%%
%%%%%%%%%%%%%%%%%%%%%%%%%%%%%%%%%%%%%%%%%%%%%%%%%%%%%%%%%%%%%%%%%%%%%%%%%%%%%%%%
%%%%%%%%%%%%%%%%%%%%%%%%%%%%%%%%%%%%%%%%%%%%%%%%%%%%%%%%%%%%%%%%%%%%%%%%%%%%%%%%
\appendix

\settowidth\MacroIndent{\rmfamily\scriptsize 000\ }

 \DocInput{childdoc.dtx}

\end{document}
%</driver>
% \fi
%
% %%%%%%%%%%%%%%%%%%%%%%%%%%%%%%%%%%%%%%%%%%%%%%%%%%%%%%%%%%%%%%%%%%%%%%%%%%%%%%
% %%%%%%%%%%%%%%%%%%%%%%%%%%%%%%%%%%%%%%%%%%%%%%%%%%%%%%%%%%%%%%%%%%%%%%%%%%%%%%
% \section{Sample}
%\iffalse
%<*samplemain>
%\fi
%
% The following presents a sample document
% with two chapters, two parts, a title page,
% a compile flag as well as three forwarding files to set the flag.
% It consists of eight |.tex| files:
% \begin{center}
% \begin{tabular}{ll}
% |cdocsamp.tex|&main file\\
% |cdocsch1.tex|&include file for chapter 1\\
% |cdocsch2.tex|&include file for chapter 2\\
% |cdocspt3.tex|&include file for part 3\\
% |cdocspt4.tex|&include file for part 4\\
% |cdocsdrf.tex|&forwarding file for main file in draft mode\\
% |cdocsfi1.tex|&forwarding file for final version of chapter 1\\
% |cdocsfi2.tex|&forwarding file for final version of chapter 2\\
% \end{tabular}
% \end{center}
% Each of the eight files can be compiled directly by the \LaTeX{} compiler.
%
% %%%%%%%%%%%%%%%%%%%%%%%%%%%%%%%%%%%%%%
% \paragraph{Main File.}
%
% The main file is called |cdocsamp.tex|.
%
% Load the \textsf{childdoc} definitions and
% declare the filename for the main document:
%    \begin{macrocode}
% \iffalse
%
% childdoc.dtx Copyright (C) 2017-2018 Niklas Beisert
%
% This work may be distributed and/or modified under the
% conditions of the LaTeX Project Public License, either version 1.3
% of this license or (at your option) any later version.
% The latest version of this license is in
%   http://www.latex-project.org/lppl.txt
% and version 1.3 or later is part of all distributions of LaTeX
% version 2005/12/01 or later.
%
% This work has the LPPL maintenance status `maintained'.
%
% The Current Maintainer of this work is Niklas Beisert.
%
% This work consists of the files childdoc.dtx and childdoc.ins
% and the derived files childdoc.def and cdocsamp.tex with
% cdocsch1.tex, cdocsch2.tex, cdocsdrf.tex, cdocsfn1.tex, cdocsfn2.tex.
%
%<package>\ifdefined\childdocmain\endinput\fi
%<package>\ProvidesFile{childdoc.def}[2018/12/30 v2.0 child document driver]
%<samplemain>\ProvidesFile{cdocsamp.tex}[2018/12/30 v2.0 sample for childdoc]
%<*driver>
%\ProvidesFile{childdoc.drv}[2018/12/30 v2.0 childdoc reference manual file]
\PassOptionsToClass{10pt,a4paper}{article}
\documentclass{ltxdoc}

\usepackage[margin=35mm]{geometry}
\usepackage{hyperref}
\usepackage{hyperxmp}
\usepackage[usenames]{color}

\hypersetup{colorlinks=true}
\hypersetup{pdfstartview=FitH}
\hypersetup{pdfpagemode=UseNone}
\hypersetup{pdfsource={}}
\hypersetup{pdflang={en-UK}}
\hypersetup{pdfcopyright={Copyright 2017-2018 Niklas Beisert.
  This work may be distributed and/or modified under the
  conditions of the LaTeX Project Public License, either version 1.3
  of this license or (at your option) any later version.}}
\hypersetup{pdflicenseurl={http://www.latex-project.org/lppl.txt}}
\hypersetup{pdfcontactaddress={ETH Zurich, ITP, HIT K,
  Wolfgang-Pauli-Strasse 27}}
\hypersetup{pdfcontactpostcode={8093}}
\hypersetup{pdfcontactcity={Zurich}}
\hypersetup{pdfcontactcountry={Switzerland}}
\hypersetup{pdfcontactemail={nbeisert@itp.phys.ethz.ch}}
\hypersetup{pdfcontacturl={http://people.phys.ethz.ch/\xmptilde nbeisert/}}

\newcommand{\secref}[1]{\hyperref[#1]{section \ref*{#1}}}

\parskip1ex
\parindent0pt
\let\olditemize\itemize
\def\itemize{\olditemize\parskip0pt}

\begin{document}

\title{The \textsf{childdoc} Package}
\hypersetup{pdftitle={The childdoc Package}}
\author{Niklas Beisert\\[2ex]
  Institut f\"ur Theoretische Physik\\
  Eidgen\"ossische Technische Hochschule Z\"urich\\
  Wolfgang-Pauli-Strasse 27, 8093 Z\"urich, Switzerland\\[1ex]
  \href{mailto:nbeisert@itp.phys.ethz.ch}
  {\texttt{nbeisert@itp.phys.ethz.ch}}}
\hypersetup{pdfauthor={Niklas Beisert}}
\hypersetup{pdfsubject={Manual for the LaTeX2e Package childdoc}}
\date{30 December 2018, \textsf{v2.0}}
\maketitle

\begin{abstract}\noindent
\textsf{childdoc} is a \LaTeXe{} package
that enables the direct compilation
of document sections included by |\include|
to individual files.
\end{abstract}

\begingroup
\parskip0ex
\tableofcontents
\endgroup

%%%%%%%%%%%%%%%%%%%%%%%%%%%%%%%%%%%%%%%%%%%%%%%%%%%%%%%%%%%%%%%%%%%%%%%%%%%%%%%%
%%%%%%%%%%%%%%%%%%%%%%%%%%%%%%%%%%%%%%%%%%%%%%%%%%%%%%%%%%%%%%%%%%%%%%%%%%%%%%%%
\section{Introduction}

\LaTeX{} provides a mechanism to structure a large document (such as a book)
into a main file and several child files (containing the chapters)
using the |\include| command.
This mechanism is beneficial for documents
which span hundreds of pages in order to
make the source file(s) more manageable.
Moreover, compilation can be restricted to
selected child files by means of the |\includeonly| command.
The latter feature can be used to reduce the compilation time while editing
(this was significantly more useful in the earlier days of \LaTeX{})
or to generate a smaller document which is easier to navigate.
Another application of |\includeonly| is to generate
documents consisting of selected parts of the complete document.

However, there are a few drawbacks of the plain |\include| mechanism:
\begin{itemize}
\item
The child files cannot be compiled on their own,
they can only be compiled via the main file.
A naive editing environment
(such as a text editor with an option
to have the current file processed by \LaTeX)
may require one to switch to the main file before compiling;
attempting to compile the child file produces errors.
\item
The main file must be modified (each time)
to adjust the |\includeonly| command
to the present needs. This easily leaves the main file in a messy state.
\item
The generated document will always carry the filename
of the main document. This is inconvenient if
several child files are to be compiled and
to be kept for distribution.
\end{itemize}

The present package provides a simple interface
to make child files individually compilable by \LaTeX{}.
Compiling a child file then has the same effect as compiling
the main file with an |\includeonly| command
to select the appropriate child.
Moreover the generated document will carry the name of the child
rather than the main file.
This resolves all three above issues.

This feature is meant to make the editing of books,
thesis documents and lecture notes somewhat more convenient.
However, the package can also be used efficiently for
composing a series of documents (such as exercise sheets)
which are typically distributed individually.
It then assists the author in generating the individual documents
(potentially in different versions)
as well as a document containing the collected series.
Another application is in developing style files
or other kinds of included material
where compilation of the style file could redirect
to a sample or test file.

%%%%%%%%%%%%%%%%%%%%%%%%%%%%%%%%%%%%%%%%%%%%%%%%%%%%%%%%%%%%%%%%%%%%%%%%%%%%%%%%
%%%%%%%%%%%%%%%%%%%%%%%%%%%%%%%%%%%%%%%%%%%%%%%%%%%%%%%%%%%%%%%%%%%%%%%%%%%%%%%%
\section{Usage}

First of all, the package \textsf{childdoc} is \emph{not} a standard
\LaTeXe{} |.sty| style file! Therefore it needs to be invoked in
a non-standard way.

%%%%%%%%%%%%%%%%%%%%%%%%%%%%%%%%%%%%%%%%%%%%%%%%%%%%%%%%%%%%%%%%%%%%%%%%%%%%%%%%
\subsection{Included Files}
\label{sec:include}

%%%%%%%%%%%%%%%%%%%%%%%%%%%%%%%%%%%%%%%%
\DescribeMacro{\childdocmain}
To use the package, add the commands
\begin{center}
\begin{tabular}{l}
|\input{childdoc.def}|\\
|\childdocmain{}|\\
\end{tabular}
\end{center}
at the very top of the main \LaTeX{} file,
in particular \emph{before} the |\documentclass| statement!
The argument of |\childdocmain| should be left empty
(but it must be present).

%%%%%%%%%%%%%%%%%%%%%%%%%%%%%%%%%%%%%%%%
\DescribeMacro{\childdocof}
Furthermore, add the commands
\begin{center}
\begin{tabular}{l}
|\input{childdoc.def}|\\
|\childdocof{|\textit{main}|}|\\
\end{tabular}
\end{center}
at the top of every child file \textit{child}
which is included by |\include{|\textit{child}|}|
from within the main file
(or at least for those files to be compiled individually).
The argument \textit{main} must be the filename of the main file.

There are a couple of
considerations in setting up the main and child documents:

%%%%%%%%%%%%%%%%%%%%%%%%%%%%%%%%%%%%%%%%
\paragraph{Restrictions.}

Please note the following restrictions:
\begin{itemize}
\item
|\childdocmain| must be called with one argument \textit{main}
to ensure compatibility with earlier version of the package.
It must either be empty (|\childdocmain{}|)
or precisely match the filename of the main file in which it is specified.
See \secref{sec:detection} for further information.
\item
The filename \textit{main} must be specified without the |.tex| extension.
\item
The filename \textit{main} is case sensitive
(even in case-insensitive file systems)
due to internal string comparison.
\item
The argument \textit{main} should be fully expanded, it cannot be a macro.
\item
Subdirectories and special characters should be avoided in filenames.
\item
The command |\childdocmain{|\textit{main}|}| must be followed by a whitespace.
It should not be followed immediately by another command
or by a comment mark `|%|'.
This is because the \TeX{} parser reads the token immediately following
the argument of |\childdocmain| and puts it
at the beginning of every child section;
however, a white\-space is ignored.
\end{itemize}

%%%%%%%%%%%%%%%%%%%%%%%%%%%%%%%%%%%%%%%%
\paragraph{Content of Main File.}

It is advisable to place all content in the child files included by |\include|.
Any output contained in the main file will appear in all child documents
unless suppressed manually;
it cannot be suppressed automatically by the |\includeonly| directive
and thus should normally be avoided.
A method to include some content in the main file
by means of conditional processing is described in \secref{sec:conditional}.

%%%%%%%%%%%%%%%%%%%%%%%%%%%%%%%%%%%%%%%%
\paragraph{Page Numbering.}

When only a part of the document is compiled,
the appropriate numbering of pages
(as well as other status parameters)
is determined from the |.aux| files.
The latter contain information from previous passes.
However this information needs to propagate through
all intermediate child documents.
Therefore the page numbering in child documents may well
be inconsistent until the complete document is compiled at least once.

A useful (if unconventional) way to always ensure a consistent
page numbering is to restart the numbering in each child document
and denote the pages by `\textit{child}|.|\textit{page}'
where \textit{child} represents the chapter/section number of the child file.
This can be achieved by the command
|\numberwithin{page}{|\textit{child}|}|
of the \textsf{amsmath} package
where \textit{child} can be |chapter| or |section|
depending on the chosen structuring.
Alternatively, one can modify the macro |\thepage| appropriately
and reset the counter |page| at the start of each child file.

%%%%%%%%%%%%%%%%%%%%%%%%%%%%%%%%%%%%%%%%%%%%%%%%%%%%%%%%%%%%%%%%%%%%%%%%%%%%%%%%
\subsection{Conditional Processing}
\label{sec:conditional}

The package provides a mechanism to compile different versions
of a document. To customise the versions further some conditional processing
can come in handy to distinguish which version is being compiled.
The package provides two macros to describe the compilation context:

%%%%%%%%%%%%%%%%%%%%%%%%%%%%%%%%%%%%%%%%
\DescribeMacro{\ifchilddoc}
The conditional |\ifchilddoc| distinguishes between the compilation of
child documents and the main document:
%
\begin{center}
|\ifchilddoc |\textit{child-code}| |[|\||else |\textit{main-code}]| \||fi|
\end{center}

%%%%%%%%%%%%%%%%%%%%%%%%%%%%%%%%%%%%%%%%
\DescribeMacro{\childdocname}
\DescribeMacro{\childdocjob}
The macro |\childdocname| contains the filename (without extension)
of the main or child file being processed.
Note that |\childdocjob| will always contain the name of the main file.

%%%%%%%%%%%%%%%%%%%%%%%%%%%%%%%%%%%%%%%%
\paragraph{Title Page.}

Conditional processing can be used to include a title or banner page
in the main document when proper precautions are taken.
Importantly, the code in the main file should ensure that the page counter
(as well as other status parameters which are stored in the |.aux| files)
takes the same value after the conditional processing.
Otherwise the page numbers may take divergent values
depending on which part is compiled.

For example, a title page could be declared by:
%
\begin{center}
\begin{tabular}{l}
|\ifchilddoc\||else|\\
|\addtocounter{page}{-1}|\\
\textit{code for title page}\\
|\newpage|\\
|\||fi|
\end{tabular}
\end{center}
%
A banner page for the child documents can be generated by:
%
\begin{center}
\begin{tabular}{l}
|\ifchilddoc|\\
|\addtocounter{page}{-1}|\\
\textit{code for banner page}\\
|\newpage|\\
|\||fi|
\end{tabular}
\end{center}
%
Here one could write a message such as:
\begin{center}
|This is the part \childdocname{} of \childdocjob{}.|
\end{center}

%%%%%%%%%%%%%%%%%%%%%%%%%%%%%%%%%%%%%%%%%%%%%%%%%%%%%%%%%%%%%%%%%%%%%%%%%%%%%%%%
\subsection{Flags}
\label{sec:flags}

The package makes it easy to generate different versions
of the main or child documents.
To this end compilation flags can be defined
and assigned different default values.
They will be particularly useful in conjunction
with the forwarding mechanism described in \secref{sec:forward}.

For example, it may be useful to have a flag |\version|
which can be set to |draft| or |final|.
The document source will contain some conditional code
depending on the value of |\version|.
Suppose further, the flag should default to |final| for the main file
and to |draft| for child files
which is a natural assignment for editing the document.
This is achieved by placing the following code
in the preamble of the main document
(below the |\childdocmain| directive):
%
\begin{center}
\begin{tabular}{l}
|\ifchilddoc|\\
|\providecommand{\version}{draft}|\\
|\||else|\\
|\providecommand{\version}{final}|\\
|\||fi|
\end{tabular}
\end{center}
%
The definition by |\providecommand| makes sure
that previous definitions are not overwritten.
Further statements |\providecommand{\version}{...}|
can thus be added before the above code to override it.

For the main file, one might add a line
(between |\childdocmain| and the above block)
%
\begin{center}
|%\ifchilddoc\||else\providecommand{\version}{draft}\||fi|
\end{center}
%
which can be uncommented to produce a draft version.
Likewise one can add a line to the very top of a child file
(above the |\childdocof{|\textit{main}|}| directive)
%
\begin{center}
|%\providecommand{\version}{final}|
\end{center}
%
which can be uncommented to produce the final version of this child document.

%%%%%%%%%%%%%%%%%%%%%%%%%%%%%%%%%%%%%%%%%%%%%%%%%%%%%%%%%%%%%%%%%%%%%%%%%%%%%%%%
\subsection{Forwarding}
\label{sec:forward}

Different versions of the main or child documents
using compilation flags as described in \secref{sec:flags}
can be (permanently) stored in different files
for convenient compilation, viewing and distribution.
To this end, the package defines a command
to pass on compilation to a different file:

%%%%%%%%%%%%%%%%%%%%%%%%%%%%%%%%%%%%%%%%
\DescribeMacro{\childdocforward}
The command |\childdocforward| redirects processing to
another source file:
%
\begin{center}
\begin{tabular}{l}
|\input{childdoc.def}|\\
|\childdocforward[|\textit{main}|]{|\textit{dest}|}|\\
\end{tabular}
\end{center}
%
The argument \textit{dest} is the destination file
(without extension).
It should be the main file or one of the child files.
Note that further \textsf{childdoc} directives
such as |\childdocof| and |\childdocforward|
in the indicated file will be processed in this form.
The optional argument \textit{main}
passes on directly to the main file \textit{main}
while pretending to compile the child \textit{dest}.
This form behaves as if \textit{dest}
issues |\childdocof{|\textit{main}|}| right away,
and no further \textsf{childdoc} directives will be processed.

%%%%%%%%%%%%%%%%%%%%%%%%%%%%%%%%%%%%%%%%
\DescribeMacro{\...prefix}
In the alternative form |\childdocforwardprefix|,
%
\begin{center}
\begin{tabular}{l}
|\input{childdoc.def}|\\
|\childdocforwardprefix[|\textit{main}|]{|\textit{prefix}|}{|\textit{dest}|}|
\end{tabular}
\end{center}
%
the destination file is determined by a pattern
depending on the current file:
To make this work, the current file must be called
`{\textit{prefix}\hspace{0.2em}\textit{suffix}}'
with \textit{prefix} matching precisely the argument.
Processing is then passed on to the file
`{\textit{dest}\hspace{0.2em}\textit{suffix}}'.
Surely, the same effect is achieved by
directly specifying the
argument `{\textit{dest}\hspace{0.2em}\textit{suffix}}'
in the first form.
However, that requires to set up a different file
for each child. With the alternative form of the command
all these files can have exactly the same content
which simplifies setting them up and maintaining them.

For example, the following file |draft.tex|
with a compilation flag |\version| as described in \secref{sec:flags}
compiles the main document as a draft:
%
\begin{center}
\begin{tabular}{l}
|\def\version{draft}|\\
|\input{childdoc.def}|\\
|\childdocforward{|\textit{main}|}|
\end{tabular}
\end{center}
%
Likewise, the following files |final|\textit{nn}|.tex|
compile the final version of the child document
|child|\textit{nn}|.tex|:
%
\begin{center}
\begin{tabular}{l}
|\def\version{final}|\\
|\input{childdoc.def}|\\
|\childdocforwardprefix{final}{child}|
\end{tabular}
\end{center}
%

Note that when several versions of a main file and/or of each child file
are to be generated, it may be convenient to set up a |Makefile| or
shell script to automatise the process.

%%%%%%%%%%%%%%%%%%%%%%%%%%%%%%%%%%%%%%%%%%%%%%%%%%%%%%%%%%%%%%%%%%%%%%%%%%%%%%%%
\subsection{Command Line Processing}
\label{sec:commandline}

The effect of redirection files can also be achieved by invoking
the \LaTeX{} compiler with a more elaborate command line.
Most conveniently this should be done as part
of a shell script or a |Makefile|.

When using \textsf{childdoc} in the main file, the following
command lines effectively perform a redirection
(note that depending on the shell being used,
backslashes may have to be doubled: `|\|' $\to$ `|\\|'):
%
\begin{center}
|... -jobname "|\textit{target}|" |\\|"|[\textit{flags}]%
|\input{childdoc.def}\childdocforward[|\textit{main}|]{|\textit{dest}|}"|
\end{center}
%
Here \textit{target} is the name of the output file,
\textit{main} is the name of the main file
and \textit{dest} is the name of the main or child file to be processed
(all filenames without extensions).
The optional argument \textit{main} can be omitted
if \textit{main} matches \textit{dest}.
Optionally, compilation \textit{flags} can be defined via |\def| commands.
This command line makes the \TeX{} engine believe
it is compiling the file \textit{target}
whose content is specified as the latter parameter.
The provided code then forwards the processing to
\textit{main} or \textit{dest} as described in \secref{sec:forward}.

%%%%%%%%%%%%%%%%%%%%%%%%%%%%%%%%%%%%%%%%%%%%%%%%%%%%%%%%%%%%%%%%%%%%%%%%%%%%%%%%
\subsection{Include by Input}
\label{sec:input}

Including child documents by |\include| has some restrictions by design.
Most notably, the content of a child document always occupies
its own set of pages; pages cannot be shared between child documents.
Usually, this behaviour makes perfect sense
because each child document contain an essential part of the document.
However, in some situations it may be desirable to compose
a document from a collection of parts
without having mandatory page breaks between then.
For this case, the package
provides a mechanism to include parts
by |\input| which can also be processed individually.
However, by construction this mechanism
requires manual handling of the content to be output.

%%%%%%%%%%%%%%%%%%%%%%%%%%%%%%%%%%%%%%%%
\DescribeMacro{\ifchilddocmanual}
The main file should be prepared as usual, see \secref{sec:include}.
However, the document body must make a distinction
between processing of an individual part and of the main document, e.g.:
%
\begin{center}
\begin{tabular}{l}
|\ifchilddocmanual|\\
|\input{\childdocname}|\\
|\||else|\\
\textit{document body with }|\input{|\textit{part}|}|\\
|\||fi|
\end{tabular}
\end{center}
%
The conditional |\ifchilddocmanual| is true whenever
a part to be included by |\input| is being compiled,
and the name of the part is stored in |\childdocname|.

%%%%%%%%%%%%%%%%%%%%%%%%%%%%%%%%%%%%%%%%
\DescribeMacro{\childdocby}
Each part to be included by |\input| should start with:
%
\begin{center}
\begin{tabular}{l}
|\input{childdoc.def}|\\
|\childdocby{|\textit{main}|}|\\
\end{tabular}
\end{center}
%
The directive |\childdocby| is similar to |\childdocof|
described in \secref{sec:include},
but the subsequent selection of content must be done manually.
To that end, both |\ifchilddoc| and |\ifchilddocmanual|
will be true upon processing of a part,
and the name of the part is stored in |\childdocname|.
Note that |\jobname| will be set to the filename of the current part
so that each part receives an individual |.aux| file
that does not interfere with the |.aux| file(s) of the main document.
This behaviour can be altered by the alternative form
|\childdocby[*]{|\textit{main}|}| (with a non-empty optional argument)
which uses the |.aux| file of the main document
by setting |\jobname| to \textit{main}.

%%%%%%%%%%%%%%%%%%%%%%%%%%%%%%%%%%%%%%%%%%%%%%%%%%%%%%%%%%%%%%%%%%%%%%%%%%%%%%%%
\subsection{Driver Development}
\label{sec:driver}

The \textsf{childdoc} mechanism can also be use for the development
of definition files such as \LaTeX{} styles or classes.
This case differs from the above setup with multiple parts
included by |\include| in that no |\includeonly| should be invoked.
This can be achieved by starting the include file
(before |\ProvidesPackage|) with:
%
\begin{center}
\begin{tabular}{l}
|\input{childdoc.def}|\\
|\childdocforward{|\textit{main}|}|\\
\end{tabular}
\end{center}
%
or alternatively with:
%
\begin{center}
\begin{tabular}{l}
|\input{childdoc.def}|\\
|\childdocby{|\textit{main}|}|\\
\end{tabular}
\end{center}
%
Both forms have slightly different effects as described above.
The main file is prepared as usual, see \secref{sec:include}.

%%%%%%%%%%%%%%%%%%%%%%%%%%%%%%%%%%%%%%%%%%%%%%%%%%%%%%%%%%%%%%%%%%%%%%%%%%%%%%%%
\subsection{Legacy Detection}
\label{sec:detection}

The directive |\childdocmain| in the main file can detect
whether the complete document or merely a child is to be compiled
even without using the directive |\childdocof|.
This method is deprecated because it is less robust
and there is no compelling reason to use it;
it is merely provided for backward compatibility
and it may be removed in future versions.

If the detection mechanism is to be used,
it is mandatory to correctly specify
the filename of the main file as the argument of |\childdocmain|:
%
\begin{center}
\begin{tabular}{l}
|\input{childdoc.def}|\\
|\childdocmain{|\textit{main}|}|\\
\end{tabular}
\end{center}
%
If |\jobname| does not match the argument \textit{main} of |\childdocmain|,
it is assumed that |\jobname| points to the child file to be compiled.
When using |\childdocmain| with the main file specified as argument,
it suffices to start a child file
with just |\input{|\textit{main}|}|
without loading of the package and using |\childdocof|.
If instead all processing is done
with the appropriate \textsf{childdoc} directives,
the argument of \textit{main} of |\childdocmain| can be empty.

An alternative version of the command line processing described
in \secref{sec:commandline} using the detection mechanism reads:
%
\begin{center}
|... -jobname "|\textit{target}|" "|[\textit{flags}]%
[|\def\jobname{|\textit{dest}|}|]|\input{|\textit{main}|}"|
\end{center}

%%%%%%%%%%%%%%%%%%%%%%%%%%%%%%%%%%%%%%%%%%%%%%%%%%%%%%%%%%%%%%%%%%%%%%%%%%%%%%%%
\subsection{Manual Code}
\label{sec:manual}

In case one cannot be certain whether the definitions file |childdoc.def|
is installed on the target \TeX{} distribution
and one prefers not to ship it,
it is conceivable to paste a few relevant commands into the sources.

To that end, drop all statements |\input{childdoc.def}|
and perform the replacements as outlined below.
Instead of |\childdocmain{|\textit{main}|}| add the following code
to the top of the main file:
%
\begin{center}
\begin{tabular}{l}
|\||ifdefined\childdocname\endinput\||fi\newif\ifchilddoc|\\
|\edef\childdocname{\scantokens\expandafter{\jobname\noexpand}}|\\
|\def\childdocmain{|\textit{main}|}\||ifx\childdocmain\childdocname\||else|\\
|\childdoctrue\includeonly{\childdocname}\let\jobname\childdocmain\||fi|\\
\end{tabular}
\end{center}
%
Instead of |\childdocof{|\textit{main}|}| just include the main file
at the top of each child file:
%
\begin{center}
|\input{|\textit{main}|}|
\end{center}
%
A simple redirection |\childdocforward{|\textit{dest}|}| is achieved by:
%
\begin{center}
|\def\jobname{|\textit{dest}|}\input{\jobname}|
\end{center}
%
The redirection with prefix
|\childdocforwardprefix[|\textit{prefix}|]{|\textit{dest}|}|
is accomplished by:
%
\begin{center}
\begin{tabular}{l}
|{\edef\jobname{\scantokens\expandafter{\jobname\noexpand}}|\\
|\def\redirectjob |\textit{prefix}|#1~~~{\gdef\jobname{|\textit{dest}|#1}}|\\
|\expandafter\redirectjob\jobname~~~}\input{\jobname}|
\end{tabular}
\end{center}

In an alternative approach,
child documents can be compiled by a specific command line
without additional code or specific definitions:
%
\begin{center}
|... -jobname "|\textit{target}|" "|[\textit{flags}]%
|\includeonly{|\textit{dest}|}\input{|\textit{main}|}"|
\end{center}
%

%%%%%%%%%%%%%%%%%%%%%%%%%%%%%%%%%%%%%%%%%%%%%%%%%%%%%%%%%%%%%%%%%%%%%%%%%%%%%%%%
%%%%%%%%%%%%%%%%%%%%%%%%%%%%%%%%%%%%%%%%%%%%%%%%%%%%%%%%%%%%%%%%%%%%%%%%%%%%%%%%
\section{Information}

%%%%%%%%%%%%%%%%%%%%%%%%%%%%%%%%%%%%%%%%%%%%%%%%%%%%%%%%%%%%%%%%%%%%%%%%%%%%%%%%
\subsection{Copyright}

Copyright \copyright{} 2017--2018 Niklas Beisert

This work may be distributed and/or modified under the
conditions of the \LaTeX{} Project Public License, either version 1.3
of this license or (at your option) any later version.
The latest version of this license is in
  \url{http://www.latex-project.org/lppl.txt}
and version 1.3 or later is part of all distributions of \LaTeX{}
version 2005/12/01 or later.

This work has the LPPL maintenance status `maintained'.

The Current Maintainer of this work is Niklas Beisert.

This work consists of the files |README.txt|, |childdoc.ins| and |childdoc.dtx|
as well as the derived files |childdoc.def|, |cdocsamp.tex|
with |cdocsch1.tex|, |cdocsch2.tex|, |cdocspt3.tex|, |cdocspt4.tex|,
|cdocsdrf.tex|, |cdocsfn1.tex|, |cdocsfn2.tex|
as well as |childdoc.pdf|.

%%%%%%%%%%%%%%%%%%%%%%%%%%%%%%%%%%%%%%%%%%%%%%%%%%%%%%%%%%%%%%%%%%%%%%%%%%%%%%%%
\subsection{Files and Installation}

The package consists of the files:
%
\begin{center}
\begin{tabular}{ll}
    |README.txt|   & readme file \\
    |childdoc.ins| & installation file \\
    |childdoc.dtx| & source file \\
    |childdoc.def| & definition file \\
    |cdocsamp.tex| & sample main file \\
    |cdocsch1.tex| & sample include file \\
    |cdocsch2.tex| & sample include file \\
    |cdocspt3.tex| & sample part file \\
    |cdocspt4.tex| & sample part file \\
    |cdocsdrf.tex| & sample redirection file \\
    |cdocsfn1.tex| & sample redirection file \\
    |cdocsfn2.tex| & sample redirection file \\
    |childdoc.pdf| & manual
\end{tabular}
\end{center}
%
The distribution consists of the files
|README.txt|, |childdoc.ins| and |childdoc.dtx|.
%
\begin{itemize}
\item
Run (pdf)\LaTeX{} on |childdoc.dtx|
to compile the manual |childdoc.pdf| (this file).
\item
Run \LaTeX{} on |childdoc.ins| to create the definitions file |childdoc.def|
and the sample |cdocsamp.tex| with include files
|cdocsch1.tex|, |cdocsch2.tex|, |cdocspt3.tex|, |cdocspt4.tex|,
|cdocsdrf.tex|, |cdocsfn1.tex|, |cdocsfn2.tex|.
Then copy the file |childdoc.def| to an appropriate directory of your \LaTeX{}
distribution, e.g.\ \textit{texmf-root}|/tex/latex/childdoc|.
\end{itemize}

%%%%%%%%%%%%%%%%%%%%%%%%%%%%%%%%%%%%%%%%%%%%%%%%%%%%%%%%%%%%%%%%%%%%%%%%%%%%%%%%
\subsection{Related CTAN Packages}

There are several other packages which offer a similar functionality:
%
\begin{itemize}
\item
The packages
\href{http://ctan.org/pkg/docmute}{\textsf{docmute}},
\href{http://ctan.org/pkg/includex}{\textsf{includex}} and
\href{http://ctan.org/pkg/standalone}{\textsf{standalone}}
provide commands to include only the document body of
a child file thus allowing both files to be compiled individually.
\item
The packages \href{http://ctan.org/pkg/subdocs}{\textsf{subdocs}}
and \href{http://ctan.org/pkg/subfiles}{\textsf{subfiles}}
provide structures in which the main and child documents can be
encapsulated and allowing them to be compiled individually.
The inclusion mechanism is different from the conventional |\include|.
\item
The package \href{http://ctan.org/pkg/combine}{\textsf{combine}}
is an elaborate solution to combine several documents into one.
\end{itemize}
%
See also the CTAN topic \href{http://ctan.org/topic/subdocs}{\textsf{subdocs}}
for further related packages.
The present package differs from the above solutions in that
a document structure constructed with the conventional |\include| mechanism
just needs two extra commands at the top of every file
such that all constituent files can be compiled individually.

%%%%%%%%%%%%%%%%%%%%%%%%%%%%%%%%%%%%%%%%%%%%%%%%%%%%%%%%%%%%%%%%%%%%%%%%%%%%%%%%
%\subsection{Feature Suggestions}
%
%The following is a list of features which may be useful for future
%versions of this package:
%%
%\begin{itemize}
%\item
%\ldots
%\end{itemize}

%%%%%%%%%%%%%%%%%%%%%%%%%%%%%%%%%%%%%%%%%%%%%%%%%%%%%%%%%%%%%%%%%%%%%%%%%%%%%%%%
\subsection{Revision History}

%%%%%%%%%%%%%%%%%%%%%%%%%%%%%%%%%%%%%%%%
\paragraph{v2.0:} 2018/12/30

\begin{itemize}
\item
immediate forward processing
\item
added |\childdocby| mechanism
\item
manual restructured
\end{itemize}

%%%%%%%%%%%%%%%%%%%%%%%%%%%%%%%%%%%%%%%%
\paragraph{v1.6:} 2018/01/17

\begin{itemize}
\item
application for development of include files
\item
corrections to manual
\end{itemize}

%%%%%%%%%%%%%%%%%%%%%%%%%%%%%%%%%%%%%%%%
\paragraph{v1.5:} 2017/05/21

\begin{itemize}
\item
more complete structuring introduced
\item
|\childdocof| introduced
\item
|\childdoc| renamed to |\childdocmain|
\item
|\childredirect| renamed to |\childdocforward| and |\childdocforwardprefix|
and functionality expanded
\end{itemize}

%%%%%%%%%%%%%%%%%%%%%%%%%%%%%%%%%%%%%%%%
\paragraph{v1.0:} 2017/04/27

\begin{itemize}
\item
manual and install package
\item
first version published on CTAN
\end{itemize}

%%%%%%%%%%%%%%%%%%%%%%%%%%%%%%%%%%%%%%%%
\paragraph{v0.6:} 2017/04/26

\begin{itemize}
\item
redirection mechanism added
\end{itemize}

%%%%%%%%%%%%%%%%%%%%%%%%%%%%%%%%%%%%%%%%
\paragraph{v0.5:} 2017/04/26

\begin{itemize}
\item
functionality in definition file
\end{itemize}


%%%%%%%%%%%%%%%%%%%%%%%%%%%%%%%%%%%%%%%%%%%%%%%%%%%%%%%%%%%%%%%%%%%%%%%%%%%%%%%%
%%%%%%%%%%%%%%%%%%%%%%%%%%%%%%%%%%%%%%%%%%%%%%%%%%%%%%%%%%%%%%%%%%%%%%%%%%%%%%%%
%%%%%%%%%%%%%%%%%%%%%%%%%%%%%%%%%%%%%%%%%%%%%%%%%%%%%%%%%%%%%%%%%%%%%%%%%%%%%%%%
\appendix

\settowidth\MacroIndent{\rmfamily\scriptsize 000\ }

 \DocInput{childdoc.dtx}

\end{document}
%</driver>
% \fi
%
% %%%%%%%%%%%%%%%%%%%%%%%%%%%%%%%%%%%%%%%%%%%%%%%%%%%%%%%%%%%%%%%%%%%%%%%%%%%%%%
% %%%%%%%%%%%%%%%%%%%%%%%%%%%%%%%%%%%%%%%%%%%%%%%%%%%%%%%%%%%%%%%%%%%%%%%%%%%%%%
% \section{Sample}
%\iffalse
%<*samplemain>
%\fi
%
% The following presents a sample document
% with two chapters, two parts, a title page,
% a compile flag as well as three forwarding files to set the flag.
% It consists of eight |.tex| files:
% \begin{center}
% \begin{tabular}{ll}
% |cdocsamp.tex|&main file\\
% |cdocsch1.tex|&include file for chapter 1\\
% |cdocsch2.tex|&include file for chapter 2\\
% |cdocspt3.tex|&include file for part 3\\
% |cdocspt4.tex|&include file for part 4\\
% |cdocsdrf.tex|&forwarding file for main file in draft mode\\
% |cdocsfi1.tex|&forwarding file for final version of chapter 1\\
% |cdocsfi2.tex|&forwarding file for final version of chapter 2\\
% \end{tabular}
% \end{center}
% Each of the eight files can be compiled directly by the \LaTeX{} compiler.
%
% %%%%%%%%%%%%%%%%%%%%%%%%%%%%%%%%%%%%%%
% \paragraph{Main File.}
%
% The main file is called |cdocsamp.tex|.
%
% Load the \textsf{childdoc} definitions and
% declare the filename for the main document:
%    \begin{macrocode}
\input{childdoc.def}
\childdocmain{}
%    \end{macrocode}

% Optional override for |\version| flag:
%    \begin{macrocode}
%%\ifchilddoc\else\providecommand{\version}{draft}\fi
%    \end{macrocode}

% Define the default values for the |\version| flag
% (|final| for the main file and |draft| for childs):
%    \begin{macrocode}
\ifchilddoc
\providecommand{\version}{draft}
\else
\providecommand{\version}{final}
\fi
%    \end{macrocode}

% Load the standard document class:
%    \begin{macrocode}
\documentclass[12pt]{article}
%    \end{macrocode}

% Start the document body:
%    \begin{macrocode}
\begin{document}
%    \end{macrocode}

% Declare a title page.
% Print title, part of document being processed and version flag:
%    \begin{macrocode}
\addtocounter{page}{-1}
\begin{center}
{\LARGE\bfseries{}childdoc example\par}
\vspace{1cm}
\ifchilddoc
\ifchilddocmanual part\else chapter\fi:
`\childdocname' of `\childdocjob'\par
\else
main document: `\childdocjob'\par
\fi
version: \version\par
\end{center}
\newpage
%    \end{macrocode}

% Manually include selected file,
% otherwise process as usual:
%    \begin{macrocode}
\ifchilddocmanual
\section*{part `\childdocname'}
\input{\childdocname}
\else
%    \end{macrocode}

% Include the two chapters:
%    \begin{macrocode}
\include{cdocsch1}
\include{cdocsch2}
%    \end{macrocode}

% Include the two parts unless only chapters should be displayed:
%    \begin{macrocode}
\ifchilddoc\else
\section{part three}
\input{cdocspt3}
\section{part four}
\input{cdocspt4}
\fi
%    \end{macrocode}

% Process as usual until here:
%    \begin{macrocode}
\fi
%    \end{macrocode}

% End of document body:
%    \begin{macrocode}
\end{document}
%    \end{macrocode}
%\iffalse
%</samplemain>
%\fi
%
% %%%%%%%%%%%%%%%%%%%%%%%%%%%%%%%%%%%%%%
% \paragraph{Chapter Include Files.}
%
% The include files are called |cdocsch1.tex| and |cdocsch2.tex|.
%
%\iffalse
%<*samplechap1|samplechap2>
%\fi

% Optional override for |\version| flag:
%    \begin{macrocode}
%%\providecommand{\version}{final}
%    \end{macrocode}

% Include the main document:
%    \begin{macrocode}
\input{childdoc.def}
\childdocof{cdocsamp}
%    \end{macrocode}

%\iffalse
%</samplechap1|samplechap2>
%\fi
%
%\iffalse
%<*samplechap1>
%\fi
% Some text for chapter 1:
%    \begin{macrocode}
\section{one}
some text in chapter one
%    \end{macrocode}

%\iffalse
%</samplechap1>
%\fi
% Some text for chapter 2:
%\iffalse
%<*samplechap2>
%\fi
%    \begin{macrocode}
\section{two}
more text in chapter two
%    \end{macrocode}

%\iffalse
%</samplechap2>
%\fi
%
% %%%%%%%%%%%%%%%%%%%%%%%%%%%%%%%%%%%%%%
% \paragraph{Part Include Files.}
%
% The include files are called |cdocspt3.tex| and |cdocspt4.tex|.
%
%\iffalse
%<*samplepart3|samplepart4>
%\fi

% Optional override for |\version| flag:
%    \begin{macrocode}
%%\providecommand{\version}{final}
%    \end{macrocode}

% Include the main document:
%    \begin{macrocode}
\input{childdoc.def}
\childdocby{cdocsamp}
%    \end{macrocode}

%\iffalse
%</samplepart3|samplepart4>
%\fi
%
%\iffalse
%<*samplepart3>
%\fi
% Some text for part 3:
%    \begin{macrocode}
some text in part three
%    \end{macrocode}

%\iffalse
%</samplepart3>
%\fi
% Some text for part 4:
%\iffalse
%<*samplepart4>
%\fi
%    \begin{macrocode}
more text in part four
%    \end{macrocode}

%\iffalse
%</samplepart4>
%\fi
%
% %%%%%%%%%%%%%%%%%%%%%%%%%%%%%%%%%%%%%%
% \paragraph{Forwarding for a Complete Draft.}
%
% The following forwarding file |cdocsdrf.tex|
% compiles the main document in draft mode:
%\iffalse
%<*sampledraft>
%\fi
%    \begin{macrocode}
\def\version{draft}
\input{childdoc.def}
\childdocforward{cdocsamp}
%    \end{macrocode}

%\iffalse
%</sampledraft>
%\fi
%
% %%%%%%%%%%%%%%%%%%%%%%%%%%%%%%%%%%%%%%
% \paragraph{Forwarding for Final Version of the Chapters.}
%
% The following forwarding files |cdocsfn1.tex| and |cdocsfn2.tex|
% (with identical content)
% compile the final versions of the child documents
% |cdocsch1.tex| and |cdocsch2.tex|, respectively:
%\iffalse
%<*samplefinal>
%\fi
%    \begin{macrocode}
\def\version{final}
\input{childdoc.def}
\childdocforwardprefix[cdocsamp]{cdocsfn}{cdocsch}
%    \end{macrocode}

%\iffalse
%</samplefinal>
%\fi
%
% %%%%%%%%%%%%%%%%%%%%%%%%%%%%%%%%%%%%%%
% \paragraph{Command Line Processing.}
%
% The following three command lines generate the output files
% |cdocscld|, |cdocscl1| and |cdocscl2|
% which should be identical to
% |cdocsdrf|, |cdocsch1| and |cdocsfn2|, respectively:
% \begin{center}
% \begin{tabular}{l}
% |latex -jobname cdocscld \|\\
% |  "\def\version{draft}\input{childdoc.def}\childdocforward{cdocsamp}"|\\
% |latex -jobname cdocscl1 \|\\
% |  "\input{childdoc.def}\childdocforward[cdocsamp]{cdocsch1}"|\\
% |latex -jobname cdocscl2 \|\\
% |  "\def\version{final}\input{childdoc.def}\childdocforward{cdocsch2}"|
% \end{tabular}
% \end{center}
% Note that the trailing backslash on each first line
% merely continues the input to the second line
% (for convenient cut ant paste).
% Furthermore, the command |latex| can be replaced by any
% of its alternative versions such as |pdflatex|.
%
% %%%%%%%%%%%%%%%%%%%%%%%%%%%%%%%%%%%%%%%%%%%%%%%%%%%%%%%%%%%%%%%%%%%%%%%%%%%%%%
% %%%%%%%%%%%%%%%%%%%%%%%%%%%%%%%%%%%%%%%%%%%%%%%%%%%%%%%%%%%%%%%%%%%%%%%%%%%%%%
% \section{Implementation}
%\iffalse
%<*package>
%\fi
%
% This section describes the definitions file |childdoc.def|.

% The definitions cannot be loaded using |\usepackage| or |\RequirePackage|
% which has a mechanism to prevent loading a style file more than once.
% When loading the definitions by means of |\input|
% multiple instances have to be prevented manually:
%\iffalse
%This code needs to be before the `\ProvidesFile' directive
%which is defined at the beginning of this file.
%Therefore it is also placed there and commented out here.
%</package>
%<*discard>
%\fi
%    \begin{macrocode}
\ifdefined\childdocmain\endinput\fi
%    \end{macrocode}
%\iffalse
%</discard>
%<*package>
%\fi
%
% \macro{\ifchilddoc}
% \macro{\ifchilddocmanual}
% The conditional |\ifchilddoc| tells whether a
% child (true) or main (false) document is being compiled.
% The conditional |\ifchilddocmanual| tells whether
% the |\includeonly| mechanism is used (false) or
% the selection of child files must be performed manually (true).
% The definitions initialise to false:
%    \begin{macrocode}
\newif\ifchilddoc
\newif\ifchilddocmanual
%    \end{macrocode}

% \macro{\childdocname}
% \macro{\childdocjob}
% The macro |\childdocname| stores the name of the main document
% to be compiled. The macro |\childdocjob| stores the name of
% the document on which the \LaTeX{} compiler was originally invoked.
% The content of |\jobname| cannot be compared
% to filenames specified in the source due to different catcodes.
% The following code rescans |\jobname|, stores the result
% in |\childdocname| and saves a copy in |\childdocjob|:
%    \begin{macrocode}
\edef\childdocname{\scantokens\expandafter{\jobname\noexpand}}
\let\childdocjob\childdocname
%    \end{macrocode}

% \macro{\childdocdisable}
% The macro |\childdocdisable| prevents the main file
% from being processed more than once.
% At this stage, the main document command |\childdocmain|
% is assumed to be called once again where it should do nothing.
% Any subsequent call to it should prevent
% a secondary processing of the main document
% It overwrites the forwarding commands
% |\childdocof| and |\childdocforward|
% with empty macros to prevent further inclusions of the main document:
%    \begin{macrocode}
\newcommand{\childdocdisable}
{
  \renewcommand{\childdocmain}[1]{\renewcommand{\childdocmain}[1]{\endinput}}
  \renewcommand{\childdocof}[1]{}
  \renewcommand{\childdocby}[2][]{}
  \renewcommand{\childdocforward}[2][]{}
  \renewcommand{\childdocdisable}{}
}
%    \end{macrocode}

% \macro{\childdocmain}
% The macro |\childdocmain| is to be called at the top of the main file
% with nothing or the main filename (without extension) as argument.
% First, it breaks loops.
% If the argument is not empty and does not match |\childdocname|
% (which is set by the first inclusion of |childdoc.def|),
% |\ifchilddoc| is set to true, |\includeonly| is applied to the child file
% and |\jobname| is set to the main file
% (for proper handling of |.aux| files):
%    \begin{macrocode}
\newcommand{\childdocmain}[1]
{
  \childdocdisable\childdocmain{}
  \if?#1?\else
    \begingroup
      \def\childdoctmp{#1}
      \ifx\childdoctmp\childdocname
        \def\childdoctmp{}
      \else
        \def\childdoctmp
        {
          \childdoctrue
          \includeonly{\childdocname}
          \def\childdocjob{#1}
          \def\jobname{#1}
        }
      \fi
      \expandafter
    \endgroup
    \childdoctmp
  \fi
}
%    \end{macrocode}

% \macro{\childdocof}
% The command |\childdocof| redirects
% compilation to the main file |#1|.
%    \begin{macrocode}
\newcommand{\childdocof}[1]
{
  \childdocdisable
  \childdoctrue
  \includeonly{\childdocname}
  \def\jobname{#1}
  \def\childdocjob{#1}
  \input{#1}
}
%    \end{macrocode}

% \macro{\childdocby}
% The command |\childdocby| ....
%    \begin{macrocode}
\newcommand{\childdocby}[2][]
{
  \childdocdisable
  \childdoctrue
  \childdocmanualtrue
  \if?#1?\else
    \def\jobname{#2}
  \fi
  \def\childdocjob{#2}
  \input{#2}
  \endinput
}
%    \end{macrocode}

% \macro{\childdocforward}
% The command |\childdocforward| redirects
% compilation to the main file or
% (if the optional argument is given) a child file.
% Parameters are set as if the main file
% or a child file starting with |\childdocof| was compiled.
% Then compilation is handed over to the main file:
%    \begin{macrocode}
\newcommand{\childdocforward}[2][]
{
  \begingroup
    \if?#1?
      \def\childdoctmp
      {
        \def\childdocname{#2}
        \def\childdocjob{#2}
        \def\jobname{#2}
        \input{#2}
        \endinput
      }
    \else
      \def\childdoctmp
      {
        \childdocdisable
        \def\childdocname{#2}
        \childdoctrue
        \includeonly{#2}
        \def\childdocjob{#1}
        \def\jobname{#1}
        \input{#1}
        \endinput
      }
    \fi
    \expandafter
  \endgroup
  \childdoctmp
}
%    \end{macrocode}

% \macro{\childdocforwardprefix}
% The command |\childdocforwardprefix| redirects
% compilation to the main or a child file by means of a pattern.
% The prefix |#1| in the current filename is replaced by |#2|
% and the suffix of the current filename is kept
% (it is assumed that the filename does not contain the substring `|~~~|'
% which is used as a delimiter).
% Compilation is handed over to the new file by |\childdocforward|:
%    \begin{macrocode}
\newcommand{\childdocforwardprefix}[3][]
{
  \begingroup
    \def\childdocextract #2##1~~~{\def\childdoctmp{\childdocforward[#1]{#3##1}}}
    \expandafter\childdocextract\childdocname~~~
    \expandafter
  \endgroup
  \childdoctmp
}
%    \end{macrocode}

% \macro{\childdoc}
% The deprecated macro |\childdoc| is a legacy version of |\childdocmain|:
%    \begin{macrocode}
\newcommand{\childdoc}{\childdocmain}
%    \end{macrocode}

% \macro{\childdocredirect}
% The deprecated macro |\childdocredirect| is a legacy version
% of |\childdocforward| and |\childdocforwardprefix|:
%    \begin{macrocode}
\newcommand{\childdocredirect}[2][]
{
  \begingroup
    \if?#1?
      \def\childdoctmp{\childdocforward{#2}}
    \else
      \def\childdoctmp{\childdocforwardprefix{#1}{#2}}
    \fi
    \expandafter
  \endgroup
  \childdoctmp
}
%    \end{macrocode}

%\iffalse
%</package>
%\fi
%
\endinput

\childdocmain{}
%    \end{macrocode}

% Optional override for |\version| flag:
%    \begin{macrocode}
%%\ifchilddoc\else\providecommand{\version}{draft}\fi
%    \end{macrocode}

% Define the default values for the |\version| flag
% (|final| for the main file and |draft| for childs):
%    \begin{macrocode}
\ifchilddoc
\providecommand{\version}{draft}
\else
\providecommand{\version}{final}
\fi
%    \end{macrocode}

% Load the standard document class:
%    \begin{macrocode}
\documentclass[12pt]{article}
%    \end{macrocode}

% Start the document body:
%    \begin{macrocode}
\begin{document}
%    \end{macrocode}

% Declare a title page.
% Print title, part of document being processed and version flag:
%    \begin{macrocode}
\addtocounter{page}{-1}
\begin{center}
{\LARGE\bfseries{}childdoc example\par}
\vspace{1cm}
\ifchilddoc
\ifchilddocmanual part\else chapter\fi:
`\childdocname' of `\childdocjob'\par
\else
main document: `\childdocjob'\par
\fi
version: \version\par
\end{center}
\newpage
%    \end{macrocode}

% Manually include selected file,
% otherwise process as usual:
%    \begin{macrocode}
\ifchilddocmanual
\section*{part `\childdocname'}
\input{\childdocname}
\else
%    \end{macrocode}

% Include the two chapters:
%    \begin{macrocode}
\include{cdocsch1}
\include{cdocsch2}
%    \end{macrocode}

% Include the two parts unless only chapters should be displayed:
%    \begin{macrocode}
\ifchilddoc\else
\section{part three}
\input{cdocspt3}
\section{part four}
\input{cdocspt4}
\fi
%    \end{macrocode}

% Process as usual until here:
%    \begin{macrocode}
\fi
%    \end{macrocode}

% End of document body:
%    \begin{macrocode}
\end{document}
%    \end{macrocode}
%\iffalse
%</samplemain>
%\fi
%
% %%%%%%%%%%%%%%%%%%%%%%%%%%%%%%%%%%%%%%
% \paragraph{Chapter Include Files.}
%
% The include files are called |cdocsch1.tex| and |cdocsch2.tex|.
%
%\iffalse
%<*samplechap1|samplechap2>
%\fi

% Optional override for |\version| flag:
%    \begin{macrocode}
%%\providecommand{\version}{final}
%    \end{macrocode}

% Include the main document:
%    \begin{macrocode}
% \iffalse
%
% childdoc.dtx Copyright (C) 2017-2018 Niklas Beisert
%
% This work may be distributed and/or modified under the
% conditions of the LaTeX Project Public License, either version 1.3
% of this license or (at your option) any later version.
% The latest version of this license is in
%   http://www.latex-project.org/lppl.txt
% and version 1.3 or later is part of all distributions of LaTeX
% version 2005/12/01 or later.
%
% This work has the LPPL maintenance status `maintained'.
%
% The Current Maintainer of this work is Niklas Beisert.
%
% This work consists of the files childdoc.dtx and childdoc.ins
% and the derived files childdoc.def and cdocsamp.tex with
% cdocsch1.tex, cdocsch2.tex, cdocsdrf.tex, cdocsfn1.tex, cdocsfn2.tex.
%
%<package>\ifdefined\childdocmain\endinput\fi
%<package>\ProvidesFile{childdoc.def}[2018/12/30 v2.0 child document driver]
%<samplemain>\ProvidesFile{cdocsamp.tex}[2018/12/30 v2.0 sample for childdoc]
%<*driver>
%\ProvidesFile{childdoc.drv}[2018/12/30 v2.0 childdoc reference manual file]
\PassOptionsToClass{10pt,a4paper}{article}
\documentclass{ltxdoc}

\usepackage[margin=35mm]{geometry}
\usepackage{hyperref}
\usepackage{hyperxmp}
\usepackage[usenames]{color}

\hypersetup{colorlinks=true}
\hypersetup{pdfstartview=FitH}
\hypersetup{pdfpagemode=UseNone}
\hypersetup{pdfsource={}}
\hypersetup{pdflang={en-UK}}
\hypersetup{pdfcopyright={Copyright 2017-2018 Niklas Beisert.
  This work may be distributed and/or modified under the
  conditions of the LaTeX Project Public License, either version 1.3
  of this license or (at your option) any later version.}}
\hypersetup{pdflicenseurl={http://www.latex-project.org/lppl.txt}}
\hypersetup{pdfcontactaddress={ETH Zurich, ITP, HIT K,
  Wolfgang-Pauli-Strasse 27}}
\hypersetup{pdfcontactpostcode={8093}}
\hypersetup{pdfcontactcity={Zurich}}
\hypersetup{pdfcontactcountry={Switzerland}}
\hypersetup{pdfcontactemail={nbeisert@itp.phys.ethz.ch}}
\hypersetup{pdfcontacturl={http://people.phys.ethz.ch/\xmptilde nbeisert/}}

\newcommand{\secref}[1]{\hyperref[#1]{section \ref*{#1}}}

\parskip1ex
\parindent0pt
\let\olditemize\itemize
\def\itemize{\olditemize\parskip0pt}

\begin{document}

\title{The \textsf{childdoc} Package}
\hypersetup{pdftitle={The childdoc Package}}
\author{Niklas Beisert\\[2ex]
  Institut f\"ur Theoretische Physik\\
  Eidgen\"ossische Technische Hochschule Z\"urich\\
  Wolfgang-Pauli-Strasse 27, 8093 Z\"urich, Switzerland\\[1ex]
  \href{mailto:nbeisert@itp.phys.ethz.ch}
  {\texttt{nbeisert@itp.phys.ethz.ch}}}
\hypersetup{pdfauthor={Niklas Beisert}}
\hypersetup{pdfsubject={Manual for the LaTeX2e Package childdoc}}
\date{30 December 2018, \textsf{v2.0}}
\maketitle

\begin{abstract}\noindent
\textsf{childdoc} is a \LaTeXe{} package
that enables the direct compilation
of document sections included by |\include|
to individual files.
\end{abstract}

\begingroup
\parskip0ex
\tableofcontents
\endgroup

%%%%%%%%%%%%%%%%%%%%%%%%%%%%%%%%%%%%%%%%%%%%%%%%%%%%%%%%%%%%%%%%%%%%%%%%%%%%%%%%
%%%%%%%%%%%%%%%%%%%%%%%%%%%%%%%%%%%%%%%%%%%%%%%%%%%%%%%%%%%%%%%%%%%%%%%%%%%%%%%%
\section{Introduction}

\LaTeX{} provides a mechanism to structure a large document (such as a book)
into a main file and several child files (containing the chapters)
using the |\include| command.
This mechanism is beneficial for documents
which span hundreds of pages in order to
make the source file(s) more manageable.
Moreover, compilation can be restricted to
selected child files by means of the |\includeonly| command.
The latter feature can be used to reduce the compilation time while editing
(this was significantly more useful in the earlier days of \LaTeX{})
or to generate a smaller document which is easier to navigate.
Another application of |\includeonly| is to generate
documents consisting of selected parts of the complete document.

However, there are a few drawbacks of the plain |\include| mechanism:
\begin{itemize}
\item
The child files cannot be compiled on their own,
they can only be compiled via the main file.
A naive editing environment
(such as a text editor with an option
to have the current file processed by \LaTeX)
may require one to switch to the main file before compiling;
attempting to compile the child file produces errors.
\item
The main file must be modified (each time)
to adjust the |\includeonly| command
to the present needs. This easily leaves the main file in a messy state.
\item
The generated document will always carry the filename
of the main document. This is inconvenient if
several child files are to be compiled and
to be kept for distribution.
\end{itemize}

The present package provides a simple interface
to make child files individually compilable by \LaTeX{}.
Compiling a child file then has the same effect as compiling
the main file with an |\includeonly| command
to select the appropriate child.
Moreover the generated document will carry the name of the child
rather than the main file.
This resolves all three above issues.

This feature is meant to make the editing of books,
thesis documents and lecture notes somewhat more convenient.
However, the package can also be used efficiently for
composing a series of documents (such as exercise sheets)
which are typically distributed individually.
It then assists the author in generating the individual documents
(potentially in different versions)
as well as a document containing the collected series.
Another application is in developing style files
or other kinds of included material
where compilation of the style file could redirect
to a sample or test file.

%%%%%%%%%%%%%%%%%%%%%%%%%%%%%%%%%%%%%%%%%%%%%%%%%%%%%%%%%%%%%%%%%%%%%%%%%%%%%%%%
%%%%%%%%%%%%%%%%%%%%%%%%%%%%%%%%%%%%%%%%%%%%%%%%%%%%%%%%%%%%%%%%%%%%%%%%%%%%%%%%
\section{Usage}

First of all, the package \textsf{childdoc} is \emph{not} a standard
\LaTeXe{} |.sty| style file! Therefore it needs to be invoked in
a non-standard way.

%%%%%%%%%%%%%%%%%%%%%%%%%%%%%%%%%%%%%%%%%%%%%%%%%%%%%%%%%%%%%%%%%%%%%%%%%%%%%%%%
\subsection{Included Files}
\label{sec:include}

%%%%%%%%%%%%%%%%%%%%%%%%%%%%%%%%%%%%%%%%
\DescribeMacro{\childdocmain}
To use the package, add the commands
\begin{center}
\begin{tabular}{l}
|\input{childdoc.def}|\\
|\childdocmain{}|\\
\end{tabular}
\end{center}
at the very top of the main \LaTeX{} file,
in particular \emph{before} the |\documentclass| statement!
The argument of |\childdocmain| should be left empty
(but it must be present).

%%%%%%%%%%%%%%%%%%%%%%%%%%%%%%%%%%%%%%%%
\DescribeMacro{\childdocof}
Furthermore, add the commands
\begin{center}
\begin{tabular}{l}
|\input{childdoc.def}|\\
|\childdocof{|\textit{main}|}|\\
\end{tabular}
\end{center}
at the top of every child file \textit{child}
which is included by |\include{|\textit{child}|}|
from within the main file
(or at least for those files to be compiled individually).
The argument \textit{main} must be the filename of the main file.

There are a couple of
considerations in setting up the main and child documents:

%%%%%%%%%%%%%%%%%%%%%%%%%%%%%%%%%%%%%%%%
\paragraph{Restrictions.}

Please note the following restrictions:
\begin{itemize}
\item
|\childdocmain| must be called with one argument \textit{main}
to ensure compatibility with earlier version of the package.
It must either be empty (|\childdocmain{}|)
or precisely match the filename of the main file in which it is specified.
See \secref{sec:detection} for further information.
\item
The filename \textit{main} must be specified without the |.tex| extension.
\item
The filename \textit{main} is case sensitive
(even in case-insensitive file systems)
due to internal string comparison.
\item
The argument \textit{main} should be fully expanded, it cannot be a macro.
\item
Subdirectories and special characters should be avoided in filenames.
\item
The command |\childdocmain{|\textit{main}|}| must be followed by a whitespace.
It should not be followed immediately by another command
or by a comment mark `|%|'.
This is because the \TeX{} parser reads the token immediately following
the argument of |\childdocmain| and puts it
at the beginning of every child section;
however, a white\-space is ignored.
\end{itemize}

%%%%%%%%%%%%%%%%%%%%%%%%%%%%%%%%%%%%%%%%
\paragraph{Content of Main File.}

It is advisable to place all content in the child files included by |\include|.
Any output contained in the main file will appear in all child documents
unless suppressed manually;
it cannot be suppressed automatically by the |\includeonly| directive
and thus should normally be avoided.
A method to include some content in the main file
by means of conditional processing is described in \secref{sec:conditional}.

%%%%%%%%%%%%%%%%%%%%%%%%%%%%%%%%%%%%%%%%
\paragraph{Page Numbering.}

When only a part of the document is compiled,
the appropriate numbering of pages
(as well as other status parameters)
is determined from the |.aux| files.
The latter contain information from previous passes.
However this information needs to propagate through
all intermediate child documents.
Therefore the page numbering in child documents may well
be inconsistent until the complete document is compiled at least once.

A useful (if unconventional) way to always ensure a consistent
page numbering is to restart the numbering in each child document
and denote the pages by `\textit{child}|.|\textit{page}'
where \textit{child} represents the chapter/section number of the child file.
This can be achieved by the command
|\numberwithin{page}{|\textit{child}|}|
of the \textsf{amsmath} package
where \textit{child} can be |chapter| or |section|
depending on the chosen structuring.
Alternatively, one can modify the macro |\thepage| appropriately
and reset the counter |page| at the start of each child file.

%%%%%%%%%%%%%%%%%%%%%%%%%%%%%%%%%%%%%%%%%%%%%%%%%%%%%%%%%%%%%%%%%%%%%%%%%%%%%%%%
\subsection{Conditional Processing}
\label{sec:conditional}

The package provides a mechanism to compile different versions
of a document. To customise the versions further some conditional processing
can come in handy to distinguish which version is being compiled.
The package provides two macros to describe the compilation context:

%%%%%%%%%%%%%%%%%%%%%%%%%%%%%%%%%%%%%%%%
\DescribeMacro{\ifchilddoc}
The conditional |\ifchilddoc| distinguishes between the compilation of
child documents and the main document:
%
\begin{center}
|\ifchilddoc |\textit{child-code}| |[|\||else |\textit{main-code}]| \||fi|
\end{center}

%%%%%%%%%%%%%%%%%%%%%%%%%%%%%%%%%%%%%%%%
\DescribeMacro{\childdocname}
\DescribeMacro{\childdocjob}
The macro |\childdocname| contains the filename (without extension)
of the main or child file being processed.
Note that |\childdocjob| will always contain the name of the main file.

%%%%%%%%%%%%%%%%%%%%%%%%%%%%%%%%%%%%%%%%
\paragraph{Title Page.}

Conditional processing can be used to include a title or banner page
in the main document when proper precautions are taken.
Importantly, the code in the main file should ensure that the page counter
(as well as other status parameters which are stored in the |.aux| files)
takes the same value after the conditional processing.
Otherwise the page numbers may take divergent values
depending on which part is compiled.

For example, a title page could be declared by:
%
\begin{center}
\begin{tabular}{l}
|\ifchilddoc\||else|\\
|\addtocounter{page}{-1}|\\
\textit{code for title page}\\
|\newpage|\\
|\||fi|
\end{tabular}
\end{center}
%
A banner page for the child documents can be generated by:
%
\begin{center}
\begin{tabular}{l}
|\ifchilddoc|\\
|\addtocounter{page}{-1}|\\
\textit{code for banner page}\\
|\newpage|\\
|\||fi|
\end{tabular}
\end{center}
%
Here one could write a message such as:
\begin{center}
|This is the part \childdocname{} of \childdocjob{}.|
\end{center}

%%%%%%%%%%%%%%%%%%%%%%%%%%%%%%%%%%%%%%%%%%%%%%%%%%%%%%%%%%%%%%%%%%%%%%%%%%%%%%%%
\subsection{Flags}
\label{sec:flags}

The package makes it easy to generate different versions
of the main or child documents.
To this end compilation flags can be defined
and assigned different default values.
They will be particularly useful in conjunction
with the forwarding mechanism described in \secref{sec:forward}.

For example, it may be useful to have a flag |\version|
which can be set to |draft| or |final|.
The document source will contain some conditional code
depending on the value of |\version|.
Suppose further, the flag should default to |final| for the main file
and to |draft| for child files
which is a natural assignment for editing the document.
This is achieved by placing the following code
in the preamble of the main document
(below the |\childdocmain| directive):
%
\begin{center}
\begin{tabular}{l}
|\ifchilddoc|\\
|\providecommand{\version}{draft}|\\
|\||else|\\
|\providecommand{\version}{final}|\\
|\||fi|
\end{tabular}
\end{center}
%
The definition by |\providecommand| makes sure
that previous definitions are not overwritten.
Further statements |\providecommand{\version}{...}|
can thus be added before the above code to override it.

For the main file, one might add a line
(between |\childdocmain| and the above block)
%
\begin{center}
|%\ifchilddoc\||else\providecommand{\version}{draft}\||fi|
\end{center}
%
which can be uncommented to produce a draft version.
Likewise one can add a line to the very top of a child file
(above the |\childdocof{|\textit{main}|}| directive)
%
\begin{center}
|%\providecommand{\version}{final}|
\end{center}
%
which can be uncommented to produce the final version of this child document.

%%%%%%%%%%%%%%%%%%%%%%%%%%%%%%%%%%%%%%%%%%%%%%%%%%%%%%%%%%%%%%%%%%%%%%%%%%%%%%%%
\subsection{Forwarding}
\label{sec:forward}

Different versions of the main or child documents
using compilation flags as described in \secref{sec:flags}
can be (permanently) stored in different files
for convenient compilation, viewing and distribution.
To this end, the package defines a command
to pass on compilation to a different file:

%%%%%%%%%%%%%%%%%%%%%%%%%%%%%%%%%%%%%%%%
\DescribeMacro{\childdocforward}
The command |\childdocforward| redirects processing to
another source file:
%
\begin{center}
\begin{tabular}{l}
|\input{childdoc.def}|\\
|\childdocforward[|\textit{main}|]{|\textit{dest}|}|\\
\end{tabular}
\end{center}
%
The argument \textit{dest} is the destination file
(without extension).
It should be the main file or one of the child files.
Note that further \textsf{childdoc} directives
such as |\childdocof| and |\childdocforward|
in the indicated file will be processed in this form.
The optional argument \textit{main}
passes on directly to the main file \textit{main}
while pretending to compile the child \textit{dest}.
This form behaves as if \textit{dest}
issues |\childdocof{|\textit{main}|}| right away,
and no further \textsf{childdoc} directives will be processed.

%%%%%%%%%%%%%%%%%%%%%%%%%%%%%%%%%%%%%%%%
\DescribeMacro{\...prefix}
In the alternative form |\childdocforwardprefix|,
%
\begin{center}
\begin{tabular}{l}
|\input{childdoc.def}|\\
|\childdocforwardprefix[|\textit{main}|]{|\textit{prefix}|}{|\textit{dest}|}|
\end{tabular}
\end{center}
%
the destination file is determined by a pattern
depending on the current file:
To make this work, the current file must be called
`{\textit{prefix}\hspace{0.2em}\textit{suffix}}'
with \textit{prefix} matching precisely the argument.
Processing is then passed on to the file
`{\textit{dest}\hspace{0.2em}\textit{suffix}}'.
Surely, the same effect is achieved by
directly specifying the
argument `{\textit{dest}\hspace{0.2em}\textit{suffix}}'
in the first form.
However, that requires to set up a different file
for each child. With the alternative form of the command
all these files can have exactly the same content
which simplifies setting them up and maintaining them.

For example, the following file |draft.tex|
with a compilation flag |\version| as described in \secref{sec:flags}
compiles the main document as a draft:
%
\begin{center}
\begin{tabular}{l}
|\def\version{draft}|\\
|\input{childdoc.def}|\\
|\childdocforward{|\textit{main}|}|
\end{tabular}
\end{center}
%
Likewise, the following files |final|\textit{nn}|.tex|
compile the final version of the child document
|child|\textit{nn}|.tex|:
%
\begin{center}
\begin{tabular}{l}
|\def\version{final}|\\
|\input{childdoc.def}|\\
|\childdocforwardprefix{final}{child}|
\end{tabular}
\end{center}
%

Note that when several versions of a main file and/or of each child file
are to be generated, it may be convenient to set up a |Makefile| or
shell script to automatise the process.

%%%%%%%%%%%%%%%%%%%%%%%%%%%%%%%%%%%%%%%%%%%%%%%%%%%%%%%%%%%%%%%%%%%%%%%%%%%%%%%%
\subsection{Command Line Processing}
\label{sec:commandline}

The effect of redirection files can also be achieved by invoking
the \LaTeX{} compiler with a more elaborate command line.
Most conveniently this should be done as part
of a shell script or a |Makefile|.

When using \textsf{childdoc} in the main file, the following
command lines effectively perform a redirection
(note that depending on the shell being used,
backslashes may have to be doubled: `|\|' $\to$ `|\\|'):
%
\begin{center}
|... -jobname "|\textit{target}|" |\\|"|[\textit{flags}]%
|\input{childdoc.def}\childdocforward[|\textit{main}|]{|\textit{dest}|}"|
\end{center}
%
Here \textit{target} is the name of the output file,
\textit{main} is the name of the main file
and \textit{dest} is the name of the main or child file to be processed
(all filenames without extensions).
The optional argument \textit{main} can be omitted
if \textit{main} matches \textit{dest}.
Optionally, compilation \textit{flags} can be defined via |\def| commands.
This command line makes the \TeX{} engine believe
it is compiling the file \textit{target}
whose content is specified as the latter parameter.
The provided code then forwards the processing to
\textit{main} or \textit{dest} as described in \secref{sec:forward}.

%%%%%%%%%%%%%%%%%%%%%%%%%%%%%%%%%%%%%%%%%%%%%%%%%%%%%%%%%%%%%%%%%%%%%%%%%%%%%%%%
\subsection{Include by Input}
\label{sec:input}

Including child documents by |\include| has some restrictions by design.
Most notably, the content of a child document always occupies
its own set of pages; pages cannot be shared between child documents.
Usually, this behaviour makes perfect sense
because each child document contain an essential part of the document.
However, in some situations it may be desirable to compose
a document from a collection of parts
without having mandatory page breaks between then.
For this case, the package
provides a mechanism to include parts
by |\input| which can also be processed individually.
However, by construction this mechanism
requires manual handling of the content to be output.

%%%%%%%%%%%%%%%%%%%%%%%%%%%%%%%%%%%%%%%%
\DescribeMacro{\ifchilddocmanual}
The main file should be prepared as usual, see \secref{sec:include}.
However, the document body must make a distinction
between processing of an individual part and of the main document, e.g.:
%
\begin{center}
\begin{tabular}{l}
|\ifchilddocmanual|\\
|\input{\childdocname}|\\
|\||else|\\
\textit{document body with }|\input{|\textit{part}|}|\\
|\||fi|
\end{tabular}
\end{center}
%
The conditional |\ifchilddocmanual| is true whenever
a part to be included by |\input| is being compiled,
and the name of the part is stored in |\childdocname|.

%%%%%%%%%%%%%%%%%%%%%%%%%%%%%%%%%%%%%%%%
\DescribeMacro{\childdocby}
Each part to be included by |\input| should start with:
%
\begin{center}
\begin{tabular}{l}
|\input{childdoc.def}|\\
|\childdocby{|\textit{main}|}|\\
\end{tabular}
\end{center}
%
The directive |\childdocby| is similar to |\childdocof|
described in \secref{sec:include},
but the subsequent selection of content must be done manually.
To that end, both |\ifchilddoc| and |\ifchilddocmanual|
will be true upon processing of a part,
and the name of the part is stored in |\childdocname|.
Note that |\jobname| will be set to the filename of the current part
so that each part receives an individual |.aux| file
that does not interfere with the |.aux| file(s) of the main document.
This behaviour can be altered by the alternative form
|\childdocby[*]{|\textit{main}|}| (with a non-empty optional argument)
which uses the |.aux| file of the main document
by setting |\jobname| to \textit{main}.

%%%%%%%%%%%%%%%%%%%%%%%%%%%%%%%%%%%%%%%%%%%%%%%%%%%%%%%%%%%%%%%%%%%%%%%%%%%%%%%%
\subsection{Driver Development}
\label{sec:driver}

The \textsf{childdoc} mechanism can also be use for the development
of definition files such as \LaTeX{} styles or classes.
This case differs from the above setup with multiple parts
included by |\include| in that no |\includeonly| should be invoked.
This can be achieved by starting the include file
(before |\ProvidesPackage|) with:
%
\begin{center}
\begin{tabular}{l}
|\input{childdoc.def}|\\
|\childdocforward{|\textit{main}|}|\\
\end{tabular}
\end{center}
%
or alternatively with:
%
\begin{center}
\begin{tabular}{l}
|\input{childdoc.def}|\\
|\childdocby{|\textit{main}|}|\\
\end{tabular}
\end{center}
%
Both forms have slightly different effects as described above.
The main file is prepared as usual, see \secref{sec:include}.

%%%%%%%%%%%%%%%%%%%%%%%%%%%%%%%%%%%%%%%%%%%%%%%%%%%%%%%%%%%%%%%%%%%%%%%%%%%%%%%%
\subsection{Legacy Detection}
\label{sec:detection}

The directive |\childdocmain| in the main file can detect
whether the complete document or merely a child is to be compiled
even without using the directive |\childdocof|.
This method is deprecated because it is less robust
and there is no compelling reason to use it;
it is merely provided for backward compatibility
and it may be removed in future versions.

If the detection mechanism is to be used,
it is mandatory to correctly specify
the filename of the main file as the argument of |\childdocmain|:
%
\begin{center}
\begin{tabular}{l}
|\input{childdoc.def}|\\
|\childdocmain{|\textit{main}|}|\\
\end{tabular}
\end{center}
%
If |\jobname| does not match the argument \textit{main} of |\childdocmain|,
it is assumed that |\jobname| points to the child file to be compiled.
When using |\childdocmain| with the main file specified as argument,
it suffices to start a child file
with just |\input{|\textit{main}|}|
without loading of the package and using |\childdocof|.
If instead all processing is done
with the appropriate \textsf{childdoc} directives,
the argument of \textit{main} of |\childdocmain| can be empty.

An alternative version of the command line processing described
in \secref{sec:commandline} using the detection mechanism reads:
%
\begin{center}
|... -jobname "|\textit{target}|" "|[\textit{flags}]%
[|\def\jobname{|\textit{dest}|}|]|\input{|\textit{main}|}"|
\end{center}

%%%%%%%%%%%%%%%%%%%%%%%%%%%%%%%%%%%%%%%%%%%%%%%%%%%%%%%%%%%%%%%%%%%%%%%%%%%%%%%%
\subsection{Manual Code}
\label{sec:manual}

In case one cannot be certain whether the definitions file |childdoc.def|
is installed on the target \TeX{} distribution
and one prefers not to ship it,
it is conceivable to paste a few relevant commands into the sources.

To that end, drop all statements |\input{childdoc.def}|
and perform the replacements as outlined below.
Instead of |\childdocmain{|\textit{main}|}| add the following code
to the top of the main file:
%
\begin{center}
\begin{tabular}{l}
|\||ifdefined\childdocname\endinput\||fi\newif\ifchilddoc|\\
|\edef\childdocname{\scantokens\expandafter{\jobname\noexpand}}|\\
|\def\childdocmain{|\textit{main}|}\||ifx\childdocmain\childdocname\||else|\\
|\childdoctrue\includeonly{\childdocname}\let\jobname\childdocmain\||fi|\\
\end{tabular}
\end{center}
%
Instead of |\childdocof{|\textit{main}|}| just include the main file
at the top of each child file:
%
\begin{center}
|\input{|\textit{main}|}|
\end{center}
%
A simple redirection |\childdocforward{|\textit{dest}|}| is achieved by:
%
\begin{center}
|\def\jobname{|\textit{dest}|}\input{\jobname}|
\end{center}
%
The redirection with prefix
|\childdocforwardprefix[|\textit{prefix}|]{|\textit{dest}|}|
is accomplished by:
%
\begin{center}
\begin{tabular}{l}
|{\edef\jobname{\scantokens\expandafter{\jobname\noexpand}}|\\
|\def\redirectjob |\textit{prefix}|#1~~~{\gdef\jobname{|\textit{dest}|#1}}|\\
|\expandafter\redirectjob\jobname~~~}\input{\jobname}|
\end{tabular}
\end{center}

In an alternative approach,
child documents can be compiled by a specific command line
without additional code or specific definitions:
%
\begin{center}
|... -jobname "|\textit{target}|" "|[\textit{flags}]%
|\includeonly{|\textit{dest}|}\input{|\textit{main}|}"|
\end{center}
%

%%%%%%%%%%%%%%%%%%%%%%%%%%%%%%%%%%%%%%%%%%%%%%%%%%%%%%%%%%%%%%%%%%%%%%%%%%%%%%%%
%%%%%%%%%%%%%%%%%%%%%%%%%%%%%%%%%%%%%%%%%%%%%%%%%%%%%%%%%%%%%%%%%%%%%%%%%%%%%%%%
\section{Information}

%%%%%%%%%%%%%%%%%%%%%%%%%%%%%%%%%%%%%%%%%%%%%%%%%%%%%%%%%%%%%%%%%%%%%%%%%%%%%%%%
\subsection{Copyright}

Copyright \copyright{} 2017--2018 Niklas Beisert

This work may be distributed and/or modified under the
conditions of the \LaTeX{} Project Public License, either version 1.3
of this license or (at your option) any later version.
The latest version of this license is in
  \url{http://www.latex-project.org/lppl.txt}
and version 1.3 or later is part of all distributions of \LaTeX{}
version 2005/12/01 or later.

This work has the LPPL maintenance status `maintained'.

The Current Maintainer of this work is Niklas Beisert.

This work consists of the files |README.txt|, |childdoc.ins| and |childdoc.dtx|
as well as the derived files |childdoc.def|, |cdocsamp.tex|
with |cdocsch1.tex|, |cdocsch2.tex|, |cdocspt3.tex|, |cdocspt4.tex|,
|cdocsdrf.tex|, |cdocsfn1.tex|, |cdocsfn2.tex|
as well as |childdoc.pdf|.

%%%%%%%%%%%%%%%%%%%%%%%%%%%%%%%%%%%%%%%%%%%%%%%%%%%%%%%%%%%%%%%%%%%%%%%%%%%%%%%%
\subsection{Files and Installation}

The package consists of the files:
%
\begin{center}
\begin{tabular}{ll}
    |README.txt|   & readme file \\
    |childdoc.ins| & installation file \\
    |childdoc.dtx| & source file \\
    |childdoc.def| & definition file \\
    |cdocsamp.tex| & sample main file \\
    |cdocsch1.tex| & sample include file \\
    |cdocsch2.tex| & sample include file \\
    |cdocspt3.tex| & sample part file \\
    |cdocspt4.tex| & sample part file \\
    |cdocsdrf.tex| & sample redirection file \\
    |cdocsfn1.tex| & sample redirection file \\
    |cdocsfn2.tex| & sample redirection file \\
    |childdoc.pdf| & manual
\end{tabular}
\end{center}
%
The distribution consists of the files
|README.txt|, |childdoc.ins| and |childdoc.dtx|.
%
\begin{itemize}
\item
Run (pdf)\LaTeX{} on |childdoc.dtx|
to compile the manual |childdoc.pdf| (this file).
\item
Run \LaTeX{} on |childdoc.ins| to create the definitions file |childdoc.def|
and the sample |cdocsamp.tex| with include files
|cdocsch1.tex|, |cdocsch2.tex|, |cdocspt3.tex|, |cdocspt4.tex|,
|cdocsdrf.tex|, |cdocsfn1.tex|, |cdocsfn2.tex|.
Then copy the file |childdoc.def| to an appropriate directory of your \LaTeX{}
distribution, e.g.\ \textit{texmf-root}|/tex/latex/childdoc|.
\end{itemize}

%%%%%%%%%%%%%%%%%%%%%%%%%%%%%%%%%%%%%%%%%%%%%%%%%%%%%%%%%%%%%%%%%%%%%%%%%%%%%%%%
\subsection{Related CTAN Packages}

There are several other packages which offer a similar functionality:
%
\begin{itemize}
\item
The packages
\href{http://ctan.org/pkg/docmute}{\textsf{docmute}},
\href{http://ctan.org/pkg/includex}{\textsf{includex}} and
\href{http://ctan.org/pkg/standalone}{\textsf{standalone}}
provide commands to include only the document body of
a child file thus allowing both files to be compiled individually.
\item
The packages \href{http://ctan.org/pkg/subdocs}{\textsf{subdocs}}
and \href{http://ctan.org/pkg/subfiles}{\textsf{subfiles}}
provide structures in which the main and child documents can be
encapsulated and allowing them to be compiled individually.
The inclusion mechanism is different from the conventional |\include|.
\item
The package \href{http://ctan.org/pkg/combine}{\textsf{combine}}
is an elaborate solution to combine several documents into one.
\end{itemize}
%
See also the CTAN topic \href{http://ctan.org/topic/subdocs}{\textsf{subdocs}}
for further related packages.
The present package differs from the above solutions in that
a document structure constructed with the conventional |\include| mechanism
just needs two extra commands at the top of every file
such that all constituent files can be compiled individually.

%%%%%%%%%%%%%%%%%%%%%%%%%%%%%%%%%%%%%%%%%%%%%%%%%%%%%%%%%%%%%%%%%%%%%%%%%%%%%%%%
%\subsection{Feature Suggestions}
%
%The following is a list of features which may be useful for future
%versions of this package:
%%
%\begin{itemize}
%\item
%\ldots
%\end{itemize}

%%%%%%%%%%%%%%%%%%%%%%%%%%%%%%%%%%%%%%%%%%%%%%%%%%%%%%%%%%%%%%%%%%%%%%%%%%%%%%%%
\subsection{Revision History}

%%%%%%%%%%%%%%%%%%%%%%%%%%%%%%%%%%%%%%%%
\paragraph{v2.0:} 2018/12/30

\begin{itemize}
\item
immediate forward processing
\item
added |\childdocby| mechanism
\item
manual restructured
\end{itemize}

%%%%%%%%%%%%%%%%%%%%%%%%%%%%%%%%%%%%%%%%
\paragraph{v1.6:} 2018/01/17

\begin{itemize}
\item
application for development of include files
\item
corrections to manual
\end{itemize}

%%%%%%%%%%%%%%%%%%%%%%%%%%%%%%%%%%%%%%%%
\paragraph{v1.5:} 2017/05/21

\begin{itemize}
\item
more complete structuring introduced
\item
|\childdocof| introduced
\item
|\childdoc| renamed to |\childdocmain|
\item
|\childredirect| renamed to |\childdocforward| and |\childdocforwardprefix|
and functionality expanded
\end{itemize}

%%%%%%%%%%%%%%%%%%%%%%%%%%%%%%%%%%%%%%%%
\paragraph{v1.0:} 2017/04/27

\begin{itemize}
\item
manual and install package
\item
first version published on CTAN
\end{itemize}

%%%%%%%%%%%%%%%%%%%%%%%%%%%%%%%%%%%%%%%%
\paragraph{v0.6:} 2017/04/26

\begin{itemize}
\item
redirection mechanism added
\end{itemize}

%%%%%%%%%%%%%%%%%%%%%%%%%%%%%%%%%%%%%%%%
\paragraph{v0.5:} 2017/04/26

\begin{itemize}
\item
functionality in definition file
\end{itemize}


%%%%%%%%%%%%%%%%%%%%%%%%%%%%%%%%%%%%%%%%%%%%%%%%%%%%%%%%%%%%%%%%%%%%%%%%%%%%%%%%
%%%%%%%%%%%%%%%%%%%%%%%%%%%%%%%%%%%%%%%%%%%%%%%%%%%%%%%%%%%%%%%%%%%%%%%%%%%%%%%%
%%%%%%%%%%%%%%%%%%%%%%%%%%%%%%%%%%%%%%%%%%%%%%%%%%%%%%%%%%%%%%%%%%%%%%%%%%%%%%%%
\appendix

\settowidth\MacroIndent{\rmfamily\scriptsize 000\ }

 \DocInput{childdoc.dtx}

\end{document}
%</driver>
% \fi
%
% %%%%%%%%%%%%%%%%%%%%%%%%%%%%%%%%%%%%%%%%%%%%%%%%%%%%%%%%%%%%%%%%%%%%%%%%%%%%%%
% %%%%%%%%%%%%%%%%%%%%%%%%%%%%%%%%%%%%%%%%%%%%%%%%%%%%%%%%%%%%%%%%%%%%%%%%%%%%%%
% \section{Sample}
%\iffalse
%<*samplemain>
%\fi
%
% The following presents a sample document
% with two chapters, two parts, a title page,
% a compile flag as well as three forwarding files to set the flag.
% It consists of eight |.tex| files:
% \begin{center}
% \begin{tabular}{ll}
% |cdocsamp.tex|&main file\\
% |cdocsch1.tex|&include file for chapter 1\\
% |cdocsch2.tex|&include file for chapter 2\\
% |cdocspt3.tex|&include file for part 3\\
% |cdocspt4.tex|&include file for part 4\\
% |cdocsdrf.tex|&forwarding file for main file in draft mode\\
% |cdocsfi1.tex|&forwarding file for final version of chapter 1\\
% |cdocsfi2.tex|&forwarding file for final version of chapter 2\\
% \end{tabular}
% \end{center}
% Each of the eight files can be compiled directly by the \LaTeX{} compiler.
%
% %%%%%%%%%%%%%%%%%%%%%%%%%%%%%%%%%%%%%%
% \paragraph{Main File.}
%
% The main file is called |cdocsamp.tex|.
%
% Load the \textsf{childdoc} definitions and
% declare the filename for the main document:
%    \begin{macrocode}
\input{childdoc.def}
\childdocmain{}
%    \end{macrocode}

% Optional override for |\version| flag:
%    \begin{macrocode}
%%\ifchilddoc\else\providecommand{\version}{draft}\fi
%    \end{macrocode}

% Define the default values for the |\version| flag
% (|final| for the main file and |draft| for childs):
%    \begin{macrocode}
\ifchilddoc
\providecommand{\version}{draft}
\else
\providecommand{\version}{final}
\fi
%    \end{macrocode}

% Load the standard document class:
%    \begin{macrocode}
\documentclass[12pt]{article}
%    \end{macrocode}

% Start the document body:
%    \begin{macrocode}
\begin{document}
%    \end{macrocode}

% Declare a title page.
% Print title, part of document being processed and version flag:
%    \begin{macrocode}
\addtocounter{page}{-1}
\begin{center}
{\LARGE\bfseries{}childdoc example\par}
\vspace{1cm}
\ifchilddoc
\ifchilddocmanual part\else chapter\fi:
`\childdocname' of `\childdocjob'\par
\else
main document: `\childdocjob'\par
\fi
version: \version\par
\end{center}
\newpage
%    \end{macrocode}

% Manually include selected file,
% otherwise process as usual:
%    \begin{macrocode}
\ifchilddocmanual
\section*{part `\childdocname'}
\input{\childdocname}
\else
%    \end{macrocode}

% Include the two chapters:
%    \begin{macrocode}
\include{cdocsch1}
\include{cdocsch2}
%    \end{macrocode}

% Include the two parts unless only chapters should be displayed:
%    \begin{macrocode}
\ifchilddoc\else
\section{part three}
\input{cdocspt3}
\section{part four}
\input{cdocspt4}
\fi
%    \end{macrocode}

% Process as usual until here:
%    \begin{macrocode}
\fi
%    \end{macrocode}

% End of document body:
%    \begin{macrocode}
\end{document}
%    \end{macrocode}
%\iffalse
%</samplemain>
%\fi
%
% %%%%%%%%%%%%%%%%%%%%%%%%%%%%%%%%%%%%%%
% \paragraph{Chapter Include Files.}
%
% The include files are called |cdocsch1.tex| and |cdocsch2.tex|.
%
%\iffalse
%<*samplechap1|samplechap2>
%\fi

% Optional override for |\version| flag:
%    \begin{macrocode}
%%\providecommand{\version}{final}
%    \end{macrocode}

% Include the main document:
%    \begin{macrocode}
\input{childdoc.def}
\childdocof{cdocsamp}
%    \end{macrocode}

%\iffalse
%</samplechap1|samplechap2>
%\fi
%
%\iffalse
%<*samplechap1>
%\fi
% Some text for chapter 1:
%    \begin{macrocode}
\section{one}
some text in chapter one
%    \end{macrocode}

%\iffalse
%</samplechap1>
%\fi
% Some text for chapter 2:
%\iffalse
%<*samplechap2>
%\fi
%    \begin{macrocode}
\section{two}
more text in chapter two
%    \end{macrocode}

%\iffalse
%</samplechap2>
%\fi
%
% %%%%%%%%%%%%%%%%%%%%%%%%%%%%%%%%%%%%%%
% \paragraph{Part Include Files.}
%
% The include files are called |cdocspt3.tex| and |cdocspt4.tex|.
%
%\iffalse
%<*samplepart3|samplepart4>
%\fi

% Optional override for |\version| flag:
%    \begin{macrocode}
%%\providecommand{\version}{final}
%    \end{macrocode}

% Include the main document:
%    \begin{macrocode}
\input{childdoc.def}
\childdocby{cdocsamp}
%    \end{macrocode}

%\iffalse
%</samplepart3|samplepart4>
%\fi
%
%\iffalse
%<*samplepart3>
%\fi
% Some text for part 3:
%    \begin{macrocode}
some text in part three
%    \end{macrocode}

%\iffalse
%</samplepart3>
%\fi
% Some text for part 4:
%\iffalse
%<*samplepart4>
%\fi
%    \begin{macrocode}
more text in part four
%    \end{macrocode}

%\iffalse
%</samplepart4>
%\fi
%
% %%%%%%%%%%%%%%%%%%%%%%%%%%%%%%%%%%%%%%
% \paragraph{Forwarding for a Complete Draft.}
%
% The following forwarding file |cdocsdrf.tex|
% compiles the main document in draft mode:
%\iffalse
%<*sampledraft>
%\fi
%    \begin{macrocode}
\def\version{draft}
\input{childdoc.def}
\childdocforward{cdocsamp}
%    \end{macrocode}

%\iffalse
%</sampledraft>
%\fi
%
% %%%%%%%%%%%%%%%%%%%%%%%%%%%%%%%%%%%%%%
% \paragraph{Forwarding for Final Version of the Chapters.}
%
% The following forwarding files |cdocsfn1.tex| and |cdocsfn2.tex|
% (with identical content)
% compile the final versions of the child documents
% |cdocsch1.tex| and |cdocsch2.tex|, respectively:
%\iffalse
%<*samplefinal>
%\fi
%    \begin{macrocode}
\def\version{final}
\input{childdoc.def}
\childdocforwardprefix[cdocsamp]{cdocsfn}{cdocsch}
%    \end{macrocode}

%\iffalse
%</samplefinal>
%\fi
%
% %%%%%%%%%%%%%%%%%%%%%%%%%%%%%%%%%%%%%%
% \paragraph{Command Line Processing.}
%
% The following three command lines generate the output files
% |cdocscld|, |cdocscl1| and |cdocscl2|
% which should be identical to
% |cdocsdrf|, |cdocsch1| and |cdocsfn2|, respectively:
% \begin{center}
% \begin{tabular}{l}
% |latex -jobname cdocscld \|\\
% |  "\def\version{draft}\input{childdoc.def}\childdocforward{cdocsamp}"|\\
% |latex -jobname cdocscl1 \|\\
% |  "\input{childdoc.def}\childdocforward[cdocsamp]{cdocsch1}"|\\
% |latex -jobname cdocscl2 \|\\
% |  "\def\version{final}\input{childdoc.def}\childdocforward{cdocsch2}"|
% \end{tabular}
% \end{center}
% Note that the trailing backslash on each first line
% merely continues the input to the second line
% (for convenient cut ant paste).
% Furthermore, the command |latex| can be replaced by any
% of its alternative versions such as |pdflatex|.
%
% %%%%%%%%%%%%%%%%%%%%%%%%%%%%%%%%%%%%%%%%%%%%%%%%%%%%%%%%%%%%%%%%%%%%%%%%%%%%%%
% %%%%%%%%%%%%%%%%%%%%%%%%%%%%%%%%%%%%%%%%%%%%%%%%%%%%%%%%%%%%%%%%%%%%%%%%%%%%%%
% \section{Implementation}
%\iffalse
%<*package>
%\fi
%
% This section describes the definitions file |childdoc.def|.

% The definitions cannot be loaded using |\usepackage| or |\RequirePackage|
% which has a mechanism to prevent loading a style file more than once.
% When loading the definitions by means of |\input|
% multiple instances have to be prevented manually:
%\iffalse
%This code needs to be before the `\ProvidesFile' directive
%which is defined at the beginning of this file.
%Therefore it is also placed there and commented out here.
%</package>
%<*discard>
%\fi
%    \begin{macrocode}
\ifdefined\childdocmain\endinput\fi
%    \end{macrocode}
%\iffalse
%</discard>
%<*package>
%\fi
%
% \macro{\ifchilddoc}
% \macro{\ifchilddocmanual}
% The conditional |\ifchilddoc| tells whether a
% child (true) or main (false) document is being compiled.
% The conditional |\ifchilddocmanual| tells whether
% the |\includeonly| mechanism is used (false) or
% the selection of child files must be performed manually (true).
% The definitions initialise to false:
%    \begin{macrocode}
\newif\ifchilddoc
\newif\ifchilddocmanual
%    \end{macrocode}

% \macro{\childdocname}
% \macro{\childdocjob}
% The macro |\childdocname| stores the name of the main document
% to be compiled. The macro |\childdocjob| stores the name of
% the document on which the \LaTeX{} compiler was originally invoked.
% The content of |\jobname| cannot be compared
% to filenames specified in the source due to different catcodes.
% The following code rescans |\jobname|, stores the result
% in |\childdocname| and saves a copy in |\childdocjob|:
%    \begin{macrocode}
\edef\childdocname{\scantokens\expandafter{\jobname\noexpand}}
\let\childdocjob\childdocname
%    \end{macrocode}

% \macro{\childdocdisable}
% The macro |\childdocdisable| prevents the main file
% from being processed more than once.
% At this stage, the main document command |\childdocmain|
% is assumed to be called once again where it should do nothing.
% Any subsequent call to it should prevent
% a secondary processing of the main document
% It overwrites the forwarding commands
% |\childdocof| and |\childdocforward|
% with empty macros to prevent further inclusions of the main document:
%    \begin{macrocode}
\newcommand{\childdocdisable}
{
  \renewcommand{\childdocmain}[1]{\renewcommand{\childdocmain}[1]{\endinput}}
  \renewcommand{\childdocof}[1]{}
  \renewcommand{\childdocby}[2][]{}
  \renewcommand{\childdocforward}[2][]{}
  \renewcommand{\childdocdisable}{}
}
%    \end{macrocode}

% \macro{\childdocmain}
% The macro |\childdocmain| is to be called at the top of the main file
% with nothing or the main filename (without extension) as argument.
% First, it breaks loops.
% If the argument is not empty and does not match |\childdocname|
% (which is set by the first inclusion of |childdoc.def|),
% |\ifchilddoc| is set to true, |\includeonly| is applied to the child file
% and |\jobname| is set to the main file
% (for proper handling of |.aux| files):
%    \begin{macrocode}
\newcommand{\childdocmain}[1]
{
  \childdocdisable\childdocmain{}
  \if?#1?\else
    \begingroup
      \def\childdoctmp{#1}
      \ifx\childdoctmp\childdocname
        \def\childdoctmp{}
      \else
        \def\childdoctmp
        {
          \childdoctrue
          \includeonly{\childdocname}
          \def\childdocjob{#1}
          \def\jobname{#1}
        }
      \fi
      \expandafter
    \endgroup
    \childdoctmp
  \fi
}
%    \end{macrocode}

% \macro{\childdocof}
% The command |\childdocof| redirects
% compilation to the main file |#1|.
%    \begin{macrocode}
\newcommand{\childdocof}[1]
{
  \childdocdisable
  \childdoctrue
  \includeonly{\childdocname}
  \def\jobname{#1}
  \def\childdocjob{#1}
  \input{#1}
}
%    \end{macrocode}

% \macro{\childdocby}
% The command |\childdocby| ....
%    \begin{macrocode}
\newcommand{\childdocby}[2][]
{
  \childdocdisable
  \childdoctrue
  \childdocmanualtrue
  \if?#1?\else
    \def\jobname{#2}
  \fi
  \def\childdocjob{#2}
  \input{#2}
  \endinput
}
%    \end{macrocode}

% \macro{\childdocforward}
% The command |\childdocforward| redirects
% compilation to the main file or
% (if the optional argument is given) a child file.
% Parameters are set as if the main file
% or a child file starting with |\childdocof| was compiled.
% Then compilation is handed over to the main file:
%    \begin{macrocode}
\newcommand{\childdocforward}[2][]
{
  \begingroup
    \if?#1?
      \def\childdoctmp
      {
        \def\childdocname{#2}
        \def\childdocjob{#2}
        \def\jobname{#2}
        \input{#2}
        \endinput
      }
    \else
      \def\childdoctmp
      {
        \childdocdisable
        \def\childdocname{#2}
        \childdoctrue
        \includeonly{#2}
        \def\childdocjob{#1}
        \def\jobname{#1}
        \input{#1}
        \endinput
      }
    \fi
    \expandafter
  \endgroup
  \childdoctmp
}
%    \end{macrocode}

% \macro{\childdocforwardprefix}
% The command |\childdocforwardprefix| redirects
% compilation to the main or a child file by means of a pattern.
% The prefix |#1| in the current filename is replaced by |#2|
% and the suffix of the current filename is kept
% (it is assumed that the filename does not contain the substring `|~~~|'
% which is used as a delimiter).
% Compilation is handed over to the new file by |\childdocforward|:
%    \begin{macrocode}
\newcommand{\childdocforwardprefix}[3][]
{
  \begingroup
    \def\childdocextract #2##1~~~{\def\childdoctmp{\childdocforward[#1]{#3##1}}}
    \expandafter\childdocextract\childdocname~~~
    \expandafter
  \endgroup
  \childdoctmp
}
%    \end{macrocode}

% \macro{\childdoc}
% The deprecated macro |\childdoc| is a legacy version of |\childdocmain|:
%    \begin{macrocode}
\newcommand{\childdoc}{\childdocmain}
%    \end{macrocode}

% \macro{\childdocredirect}
% The deprecated macro |\childdocredirect| is a legacy version
% of |\childdocforward| and |\childdocforwardprefix|:
%    \begin{macrocode}
\newcommand{\childdocredirect}[2][]
{
  \begingroup
    \if?#1?
      \def\childdoctmp{\childdocforward{#2}}
    \else
      \def\childdoctmp{\childdocforwardprefix{#1}{#2}}
    \fi
    \expandafter
  \endgroup
  \childdoctmp
}
%    \end{macrocode}

%\iffalse
%</package>
%\fi
%
\endinput

\childdocof{cdocsamp}
%    \end{macrocode}

%\iffalse
%</samplechap1|samplechap2>
%\fi
%
%\iffalse
%<*samplechap1>
%\fi
% Some text for chapter 1:
%    \begin{macrocode}
\section{one}
some text in chapter one
%    \end{macrocode}

%\iffalse
%</samplechap1>
%\fi
% Some text for chapter 2:
%\iffalse
%<*samplechap2>
%\fi
%    \begin{macrocode}
\section{two}
more text in chapter two
%    \end{macrocode}

%\iffalse
%</samplechap2>
%\fi
%
% %%%%%%%%%%%%%%%%%%%%%%%%%%%%%%%%%%%%%%
% \paragraph{Part Include Files.}
%
% The include files are called |cdocspt3.tex| and |cdocspt4.tex|.
%
%\iffalse
%<*samplepart3|samplepart4>
%\fi

% Optional override for |\version| flag:
%    \begin{macrocode}
%%\providecommand{\version}{final}
%    \end{macrocode}

% Include the main document:
%    \begin{macrocode}
% \iffalse
%
% childdoc.dtx Copyright (C) 2017-2018 Niklas Beisert
%
% This work may be distributed and/or modified under the
% conditions of the LaTeX Project Public License, either version 1.3
% of this license or (at your option) any later version.
% The latest version of this license is in
%   http://www.latex-project.org/lppl.txt
% and version 1.3 or later is part of all distributions of LaTeX
% version 2005/12/01 or later.
%
% This work has the LPPL maintenance status `maintained'.
%
% The Current Maintainer of this work is Niklas Beisert.
%
% This work consists of the files childdoc.dtx and childdoc.ins
% and the derived files childdoc.def and cdocsamp.tex with
% cdocsch1.tex, cdocsch2.tex, cdocsdrf.tex, cdocsfn1.tex, cdocsfn2.tex.
%
%<package>\ifdefined\childdocmain\endinput\fi
%<package>\ProvidesFile{childdoc.def}[2018/12/30 v2.0 child document driver]
%<samplemain>\ProvidesFile{cdocsamp.tex}[2018/12/30 v2.0 sample for childdoc]
%<*driver>
%\ProvidesFile{childdoc.drv}[2018/12/30 v2.0 childdoc reference manual file]
\PassOptionsToClass{10pt,a4paper}{article}
\documentclass{ltxdoc}

\usepackage[margin=35mm]{geometry}
\usepackage{hyperref}
\usepackage{hyperxmp}
\usepackage[usenames]{color}

\hypersetup{colorlinks=true}
\hypersetup{pdfstartview=FitH}
\hypersetup{pdfpagemode=UseNone}
\hypersetup{pdfsource={}}
\hypersetup{pdflang={en-UK}}
\hypersetup{pdfcopyright={Copyright 2017-2018 Niklas Beisert.
  This work may be distributed and/or modified under the
  conditions of the LaTeX Project Public License, either version 1.3
  of this license or (at your option) any later version.}}
\hypersetup{pdflicenseurl={http://www.latex-project.org/lppl.txt}}
\hypersetup{pdfcontactaddress={ETH Zurich, ITP, HIT K,
  Wolfgang-Pauli-Strasse 27}}
\hypersetup{pdfcontactpostcode={8093}}
\hypersetup{pdfcontactcity={Zurich}}
\hypersetup{pdfcontactcountry={Switzerland}}
\hypersetup{pdfcontactemail={nbeisert@itp.phys.ethz.ch}}
\hypersetup{pdfcontacturl={http://people.phys.ethz.ch/\xmptilde nbeisert/}}

\newcommand{\secref}[1]{\hyperref[#1]{section \ref*{#1}}}

\parskip1ex
\parindent0pt
\let\olditemize\itemize
\def\itemize{\olditemize\parskip0pt}

\begin{document}

\title{The \textsf{childdoc} Package}
\hypersetup{pdftitle={The childdoc Package}}
\author{Niklas Beisert\\[2ex]
  Institut f\"ur Theoretische Physik\\
  Eidgen\"ossische Technische Hochschule Z\"urich\\
  Wolfgang-Pauli-Strasse 27, 8093 Z\"urich, Switzerland\\[1ex]
  \href{mailto:nbeisert@itp.phys.ethz.ch}
  {\texttt{nbeisert@itp.phys.ethz.ch}}}
\hypersetup{pdfauthor={Niklas Beisert}}
\hypersetup{pdfsubject={Manual for the LaTeX2e Package childdoc}}
\date{30 December 2018, \textsf{v2.0}}
\maketitle

\begin{abstract}\noindent
\textsf{childdoc} is a \LaTeXe{} package
that enables the direct compilation
of document sections included by |\include|
to individual files.
\end{abstract}

\begingroup
\parskip0ex
\tableofcontents
\endgroup

%%%%%%%%%%%%%%%%%%%%%%%%%%%%%%%%%%%%%%%%%%%%%%%%%%%%%%%%%%%%%%%%%%%%%%%%%%%%%%%%
%%%%%%%%%%%%%%%%%%%%%%%%%%%%%%%%%%%%%%%%%%%%%%%%%%%%%%%%%%%%%%%%%%%%%%%%%%%%%%%%
\section{Introduction}

\LaTeX{} provides a mechanism to structure a large document (such as a book)
into a main file and several child files (containing the chapters)
using the |\include| command.
This mechanism is beneficial for documents
which span hundreds of pages in order to
make the source file(s) more manageable.
Moreover, compilation can be restricted to
selected child files by means of the |\includeonly| command.
The latter feature can be used to reduce the compilation time while editing
(this was significantly more useful in the earlier days of \LaTeX{})
or to generate a smaller document which is easier to navigate.
Another application of |\includeonly| is to generate
documents consisting of selected parts of the complete document.

However, there are a few drawbacks of the plain |\include| mechanism:
\begin{itemize}
\item
The child files cannot be compiled on their own,
they can only be compiled via the main file.
A naive editing environment
(such as a text editor with an option
to have the current file processed by \LaTeX)
may require one to switch to the main file before compiling;
attempting to compile the child file produces errors.
\item
The main file must be modified (each time)
to adjust the |\includeonly| command
to the present needs. This easily leaves the main file in a messy state.
\item
The generated document will always carry the filename
of the main document. This is inconvenient if
several child files are to be compiled and
to be kept for distribution.
\end{itemize}

The present package provides a simple interface
to make child files individually compilable by \LaTeX{}.
Compiling a child file then has the same effect as compiling
the main file with an |\includeonly| command
to select the appropriate child.
Moreover the generated document will carry the name of the child
rather than the main file.
This resolves all three above issues.

This feature is meant to make the editing of books,
thesis documents and lecture notes somewhat more convenient.
However, the package can also be used efficiently for
composing a series of documents (such as exercise sheets)
which are typically distributed individually.
It then assists the author in generating the individual documents
(potentially in different versions)
as well as a document containing the collected series.
Another application is in developing style files
or other kinds of included material
where compilation of the style file could redirect
to a sample or test file.

%%%%%%%%%%%%%%%%%%%%%%%%%%%%%%%%%%%%%%%%%%%%%%%%%%%%%%%%%%%%%%%%%%%%%%%%%%%%%%%%
%%%%%%%%%%%%%%%%%%%%%%%%%%%%%%%%%%%%%%%%%%%%%%%%%%%%%%%%%%%%%%%%%%%%%%%%%%%%%%%%
\section{Usage}

First of all, the package \textsf{childdoc} is \emph{not} a standard
\LaTeXe{} |.sty| style file! Therefore it needs to be invoked in
a non-standard way.

%%%%%%%%%%%%%%%%%%%%%%%%%%%%%%%%%%%%%%%%%%%%%%%%%%%%%%%%%%%%%%%%%%%%%%%%%%%%%%%%
\subsection{Included Files}
\label{sec:include}

%%%%%%%%%%%%%%%%%%%%%%%%%%%%%%%%%%%%%%%%
\DescribeMacro{\childdocmain}
To use the package, add the commands
\begin{center}
\begin{tabular}{l}
|\input{childdoc.def}|\\
|\childdocmain{}|\\
\end{tabular}
\end{center}
at the very top of the main \LaTeX{} file,
in particular \emph{before} the |\documentclass| statement!
The argument of |\childdocmain| should be left empty
(but it must be present).

%%%%%%%%%%%%%%%%%%%%%%%%%%%%%%%%%%%%%%%%
\DescribeMacro{\childdocof}
Furthermore, add the commands
\begin{center}
\begin{tabular}{l}
|\input{childdoc.def}|\\
|\childdocof{|\textit{main}|}|\\
\end{tabular}
\end{center}
at the top of every child file \textit{child}
which is included by |\include{|\textit{child}|}|
from within the main file
(or at least for those files to be compiled individually).
The argument \textit{main} must be the filename of the main file.

There are a couple of
considerations in setting up the main and child documents:

%%%%%%%%%%%%%%%%%%%%%%%%%%%%%%%%%%%%%%%%
\paragraph{Restrictions.}

Please note the following restrictions:
\begin{itemize}
\item
|\childdocmain| must be called with one argument \textit{main}
to ensure compatibility with earlier version of the package.
It must either be empty (|\childdocmain{}|)
or precisely match the filename of the main file in which it is specified.
See \secref{sec:detection} for further information.
\item
The filename \textit{main} must be specified without the |.tex| extension.
\item
The filename \textit{main} is case sensitive
(even in case-insensitive file systems)
due to internal string comparison.
\item
The argument \textit{main} should be fully expanded, it cannot be a macro.
\item
Subdirectories and special characters should be avoided in filenames.
\item
The command |\childdocmain{|\textit{main}|}| must be followed by a whitespace.
It should not be followed immediately by another command
or by a comment mark `|%|'.
This is because the \TeX{} parser reads the token immediately following
the argument of |\childdocmain| and puts it
at the beginning of every child section;
however, a white\-space is ignored.
\end{itemize}

%%%%%%%%%%%%%%%%%%%%%%%%%%%%%%%%%%%%%%%%
\paragraph{Content of Main File.}

It is advisable to place all content in the child files included by |\include|.
Any output contained in the main file will appear in all child documents
unless suppressed manually;
it cannot be suppressed automatically by the |\includeonly| directive
and thus should normally be avoided.
A method to include some content in the main file
by means of conditional processing is described in \secref{sec:conditional}.

%%%%%%%%%%%%%%%%%%%%%%%%%%%%%%%%%%%%%%%%
\paragraph{Page Numbering.}

When only a part of the document is compiled,
the appropriate numbering of pages
(as well as other status parameters)
is determined from the |.aux| files.
The latter contain information from previous passes.
However this information needs to propagate through
all intermediate child documents.
Therefore the page numbering in child documents may well
be inconsistent until the complete document is compiled at least once.

A useful (if unconventional) way to always ensure a consistent
page numbering is to restart the numbering in each child document
and denote the pages by `\textit{child}|.|\textit{page}'
where \textit{child} represents the chapter/section number of the child file.
This can be achieved by the command
|\numberwithin{page}{|\textit{child}|}|
of the \textsf{amsmath} package
where \textit{child} can be |chapter| or |section|
depending on the chosen structuring.
Alternatively, one can modify the macro |\thepage| appropriately
and reset the counter |page| at the start of each child file.

%%%%%%%%%%%%%%%%%%%%%%%%%%%%%%%%%%%%%%%%%%%%%%%%%%%%%%%%%%%%%%%%%%%%%%%%%%%%%%%%
\subsection{Conditional Processing}
\label{sec:conditional}

The package provides a mechanism to compile different versions
of a document. To customise the versions further some conditional processing
can come in handy to distinguish which version is being compiled.
The package provides two macros to describe the compilation context:

%%%%%%%%%%%%%%%%%%%%%%%%%%%%%%%%%%%%%%%%
\DescribeMacro{\ifchilddoc}
The conditional |\ifchilddoc| distinguishes between the compilation of
child documents and the main document:
%
\begin{center}
|\ifchilddoc |\textit{child-code}| |[|\||else |\textit{main-code}]| \||fi|
\end{center}

%%%%%%%%%%%%%%%%%%%%%%%%%%%%%%%%%%%%%%%%
\DescribeMacro{\childdocname}
\DescribeMacro{\childdocjob}
The macro |\childdocname| contains the filename (without extension)
of the main or child file being processed.
Note that |\childdocjob| will always contain the name of the main file.

%%%%%%%%%%%%%%%%%%%%%%%%%%%%%%%%%%%%%%%%
\paragraph{Title Page.}

Conditional processing can be used to include a title or banner page
in the main document when proper precautions are taken.
Importantly, the code in the main file should ensure that the page counter
(as well as other status parameters which are stored in the |.aux| files)
takes the same value after the conditional processing.
Otherwise the page numbers may take divergent values
depending on which part is compiled.

For example, a title page could be declared by:
%
\begin{center}
\begin{tabular}{l}
|\ifchilddoc\||else|\\
|\addtocounter{page}{-1}|\\
\textit{code for title page}\\
|\newpage|\\
|\||fi|
\end{tabular}
\end{center}
%
A banner page for the child documents can be generated by:
%
\begin{center}
\begin{tabular}{l}
|\ifchilddoc|\\
|\addtocounter{page}{-1}|\\
\textit{code for banner page}\\
|\newpage|\\
|\||fi|
\end{tabular}
\end{center}
%
Here one could write a message such as:
\begin{center}
|This is the part \childdocname{} of \childdocjob{}.|
\end{center}

%%%%%%%%%%%%%%%%%%%%%%%%%%%%%%%%%%%%%%%%%%%%%%%%%%%%%%%%%%%%%%%%%%%%%%%%%%%%%%%%
\subsection{Flags}
\label{sec:flags}

The package makes it easy to generate different versions
of the main or child documents.
To this end compilation flags can be defined
and assigned different default values.
They will be particularly useful in conjunction
with the forwarding mechanism described in \secref{sec:forward}.

For example, it may be useful to have a flag |\version|
which can be set to |draft| or |final|.
The document source will contain some conditional code
depending on the value of |\version|.
Suppose further, the flag should default to |final| for the main file
and to |draft| for child files
which is a natural assignment for editing the document.
This is achieved by placing the following code
in the preamble of the main document
(below the |\childdocmain| directive):
%
\begin{center}
\begin{tabular}{l}
|\ifchilddoc|\\
|\providecommand{\version}{draft}|\\
|\||else|\\
|\providecommand{\version}{final}|\\
|\||fi|
\end{tabular}
\end{center}
%
The definition by |\providecommand| makes sure
that previous definitions are not overwritten.
Further statements |\providecommand{\version}{...}|
can thus be added before the above code to override it.

For the main file, one might add a line
(between |\childdocmain| and the above block)
%
\begin{center}
|%\ifchilddoc\||else\providecommand{\version}{draft}\||fi|
\end{center}
%
which can be uncommented to produce a draft version.
Likewise one can add a line to the very top of a child file
(above the |\childdocof{|\textit{main}|}| directive)
%
\begin{center}
|%\providecommand{\version}{final}|
\end{center}
%
which can be uncommented to produce the final version of this child document.

%%%%%%%%%%%%%%%%%%%%%%%%%%%%%%%%%%%%%%%%%%%%%%%%%%%%%%%%%%%%%%%%%%%%%%%%%%%%%%%%
\subsection{Forwarding}
\label{sec:forward}

Different versions of the main or child documents
using compilation flags as described in \secref{sec:flags}
can be (permanently) stored in different files
for convenient compilation, viewing and distribution.
To this end, the package defines a command
to pass on compilation to a different file:

%%%%%%%%%%%%%%%%%%%%%%%%%%%%%%%%%%%%%%%%
\DescribeMacro{\childdocforward}
The command |\childdocforward| redirects processing to
another source file:
%
\begin{center}
\begin{tabular}{l}
|\input{childdoc.def}|\\
|\childdocforward[|\textit{main}|]{|\textit{dest}|}|\\
\end{tabular}
\end{center}
%
The argument \textit{dest} is the destination file
(without extension).
It should be the main file or one of the child files.
Note that further \textsf{childdoc} directives
such as |\childdocof| and |\childdocforward|
in the indicated file will be processed in this form.
The optional argument \textit{main}
passes on directly to the main file \textit{main}
while pretending to compile the child \textit{dest}.
This form behaves as if \textit{dest}
issues |\childdocof{|\textit{main}|}| right away,
and no further \textsf{childdoc} directives will be processed.

%%%%%%%%%%%%%%%%%%%%%%%%%%%%%%%%%%%%%%%%
\DescribeMacro{\...prefix}
In the alternative form |\childdocforwardprefix|,
%
\begin{center}
\begin{tabular}{l}
|\input{childdoc.def}|\\
|\childdocforwardprefix[|\textit{main}|]{|\textit{prefix}|}{|\textit{dest}|}|
\end{tabular}
\end{center}
%
the destination file is determined by a pattern
depending on the current file:
To make this work, the current file must be called
`{\textit{prefix}\hspace{0.2em}\textit{suffix}}'
with \textit{prefix} matching precisely the argument.
Processing is then passed on to the file
`{\textit{dest}\hspace{0.2em}\textit{suffix}}'.
Surely, the same effect is achieved by
directly specifying the
argument `{\textit{dest}\hspace{0.2em}\textit{suffix}}'
in the first form.
However, that requires to set up a different file
for each child. With the alternative form of the command
all these files can have exactly the same content
which simplifies setting them up and maintaining them.

For example, the following file |draft.tex|
with a compilation flag |\version| as described in \secref{sec:flags}
compiles the main document as a draft:
%
\begin{center}
\begin{tabular}{l}
|\def\version{draft}|\\
|\input{childdoc.def}|\\
|\childdocforward{|\textit{main}|}|
\end{tabular}
\end{center}
%
Likewise, the following files |final|\textit{nn}|.tex|
compile the final version of the child document
|child|\textit{nn}|.tex|:
%
\begin{center}
\begin{tabular}{l}
|\def\version{final}|\\
|\input{childdoc.def}|\\
|\childdocforwardprefix{final}{child}|
\end{tabular}
\end{center}
%

Note that when several versions of a main file and/or of each child file
are to be generated, it may be convenient to set up a |Makefile| or
shell script to automatise the process.

%%%%%%%%%%%%%%%%%%%%%%%%%%%%%%%%%%%%%%%%%%%%%%%%%%%%%%%%%%%%%%%%%%%%%%%%%%%%%%%%
\subsection{Command Line Processing}
\label{sec:commandline}

The effect of redirection files can also be achieved by invoking
the \LaTeX{} compiler with a more elaborate command line.
Most conveniently this should be done as part
of a shell script or a |Makefile|.

When using \textsf{childdoc} in the main file, the following
command lines effectively perform a redirection
(note that depending on the shell being used,
backslashes may have to be doubled: `|\|' $\to$ `|\\|'):
%
\begin{center}
|... -jobname "|\textit{target}|" |\\|"|[\textit{flags}]%
|\input{childdoc.def}\childdocforward[|\textit{main}|]{|\textit{dest}|}"|
\end{center}
%
Here \textit{target} is the name of the output file,
\textit{main} is the name of the main file
and \textit{dest} is the name of the main or child file to be processed
(all filenames without extensions).
The optional argument \textit{main} can be omitted
if \textit{main} matches \textit{dest}.
Optionally, compilation \textit{flags} can be defined via |\def| commands.
This command line makes the \TeX{} engine believe
it is compiling the file \textit{target}
whose content is specified as the latter parameter.
The provided code then forwards the processing to
\textit{main} or \textit{dest} as described in \secref{sec:forward}.

%%%%%%%%%%%%%%%%%%%%%%%%%%%%%%%%%%%%%%%%%%%%%%%%%%%%%%%%%%%%%%%%%%%%%%%%%%%%%%%%
\subsection{Include by Input}
\label{sec:input}

Including child documents by |\include| has some restrictions by design.
Most notably, the content of a child document always occupies
its own set of pages; pages cannot be shared between child documents.
Usually, this behaviour makes perfect sense
because each child document contain an essential part of the document.
However, in some situations it may be desirable to compose
a document from a collection of parts
without having mandatory page breaks between then.
For this case, the package
provides a mechanism to include parts
by |\input| which can also be processed individually.
However, by construction this mechanism
requires manual handling of the content to be output.

%%%%%%%%%%%%%%%%%%%%%%%%%%%%%%%%%%%%%%%%
\DescribeMacro{\ifchilddocmanual}
The main file should be prepared as usual, see \secref{sec:include}.
However, the document body must make a distinction
between processing of an individual part and of the main document, e.g.:
%
\begin{center}
\begin{tabular}{l}
|\ifchilddocmanual|\\
|\input{\childdocname}|\\
|\||else|\\
\textit{document body with }|\input{|\textit{part}|}|\\
|\||fi|
\end{tabular}
\end{center}
%
The conditional |\ifchilddocmanual| is true whenever
a part to be included by |\input| is being compiled,
and the name of the part is stored in |\childdocname|.

%%%%%%%%%%%%%%%%%%%%%%%%%%%%%%%%%%%%%%%%
\DescribeMacro{\childdocby}
Each part to be included by |\input| should start with:
%
\begin{center}
\begin{tabular}{l}
|\input{childdoc.def}|\\
|\childdocby{|\textit{main}|}|\\
\end{tabular}
\end{center}
%
The directive |\childdocby| is similar to |\childdocof|
described in \secref{sec:include},
but the subsequent selection of content must be done manually.
To that end, both |\ifchilddoc| and |\ifchilddocmanual|
will be true upon processing of a part,
and the name of the part is stored in |\childdocname|.
Note that |\jobname| will be set to the filename of the current part
so that each part receives an individual |.aux| file
that does not interfere with the |.aux| file(s) of the main document.
This behaviour can be altered by the alternative form
|\childdocby[*]{|\textit{main}|}| (with a non-empty optional argument)
which uses the |.aux| file of the main document
by setting |\jobname| to \textit{main}.

%%%%%%%%%%%%%%%%%%%%%%%%%%%%%%%%%%%%%%%%%%%%%%%%%%%%%%%%%%%%%%%%%%%%%%%%%%%%%%%%
\subsection{Driver Development}
\label{sec:driver}

The \textsf{childdoc} mechanism can also be use for the development
of definition files such as \LaTeX{} styles or classes.
This case differs from the above setup with multiple parts
included by |\include| in that no |\includeonly| should be invoked.
This can be achieved by starting the include file
(before |\ProvidesPackage|) with:
%
\begin{center}
\begin{tabular}{l}
|\input{childdoc.def}|\\
|\childdocforward{|\textit{main}|}|\\
\end{tabular}
\end{center}
%
or alternatively with:
%
\begin{center}
\begin{tabular}{l}
|\input{childdoc.def}|\\
|\childdocby{|\textit{main}|}|\\
\end{tabular}
\end{center}
%
Both forms have slightly different effects as described above.
The main file is prepared as usual, see \secref{sec:include}.

%%%%%%%%%%%%%%%%%%%%%%%%%%%%%%%%%%%%%%%%%%%%%%%%%%%%%%%%%%%%%%%%%%%%%%%%%%%%%%%%
\subsection{Legacy Detection}
\label{sec:detection}

The directive |\childdocmain| in the main file can detect
whether the complete document or merely a child is to be compiled
even without using the directive |\childdocof|.
This method is deprecated because it is less robust
and there is no compelling reason to use it;
it is merely provided for backward compatibility
and it may be removed in future versions.

If the detection mechanism is to be used,
it is mandatory to correctly specify
the filename of the main file as the argument of |\childdocmain|:
%
\begin{center}
\begin{tabular}{l}
|\input{childdoc.def}|\\
|\childdocmain{|\textit{main}|}|\\
\end{tabular}
\end{center}
%
If |\jobname| does not match the argument \textit{main} of |\childdocmain|,
it is assumed that |\jobname| points to the child file to be compiled.
When using |\childdocmain| with the main file specified as argument,
it suffices to start a child file
with just |\input{|\textit{main}|}|
without loading of the package and using |\childdocof|.
If instead all processing is done
with the appropriate \textsf{childdoc} directives,
the argument of \textit{main} of |\childdocmain| can be empty.

An alternative version of the command line processing described
in \secref{sec:commandline} using the detection mechanism reads:
%
\begin{center}
|... -jobname "|\textit{target}|" "|[\textit{flags}]%
[|\def\jobname{|\textit{dest}|}|]|\input{|\textit{main}|}"|
\end{center}

%%%%%%%%%%%%%%%%%%%%%%%%%%%%%%%%%%%%%%%%%%%%%%%%%%%%%%%%%%%%%%%%%%%%%%%%%%%%%%%%
\subsection{Manual Code}
\label{sec:manual}

In case one cannot be certain whether the definitions file |childdoc.def|
is installed on the target \TeX{} distribution
and one prefers not to ship it,
it is conceivable to paste a few relevant commands into the sources.

To that end, drop all statements |\input{childdoc.def}|
and perform the replacements as outlined below.
Instead of |\childdocmain{|\textit{main}|}| add the following code
to the top of the main file:
%
\begin{center}
\begin{tabular}{l}
|\||ifdefined\childdocname\endinput\||fi\newif\ifchilddoc|\\
|\edef\childdocname{\scantokens\expandafter{\jobname\noexpand}}|\\
|\def\childdocmain{|\textit{main}|}\||ifx\childdocmain\childdocname\||else|\\
|\childdoctrue\includeonly{\childdocname}\let\jobname\childdocmain\||fi|\\
\end{tabular}
\end{center}
%
Instead of |\childdocof{|\textit{main}|}| just include the main file
at the top of each child file:
%
\begin{center}
|\input{|\textit{main}|}|
\end{center}
%
A simple redirection |\childdocforward{|\textit{dest}|}| is achieved by:
%
\begin{center}
|\def\jobname{|\textit{dest}|}\input{\jobname}|
\end{center}
%
The redirection with prefix
|\childdocforwardprefix[|\textit{prefix}|]{|\textit{dest}|}|
is accomplished by:
%
\begin{center}
\begin{tabular}{l}
|{\edef\jobname{\scantokens\expandafter{\jobname\noexpand}}|\\
|\def\redirectjob |\textit{prefix}|#1~~~{\gdef\jobname{|\textit{dest}|#1}}|\\
|\expandafter\redirectjob\jobname~~~}\input{\jobname}|
\end{tabular}
\end{center}

In an alternative approach,
child documents can be compiled by a specific command line
without additional code or specific definitions:
%
\begin{center}
|... -jobname "|\textit{target}|" "|[\textit{flags}]%
|\includeonly{|\textit{dest}|}\input{|\textit{main}|}"|
\end{center}
%

%%%%%%%%%%%%%%%%%%%%%%%%%%%%%%%%%%%%%%%%%%%%%%%%%%%%%%%%%%%%%%%%%%%%%%%%%%%%%%%%
%%%%%%%%%%%%%%%%%%%%%%%%%%%%%%%%%%%%%%%%%%%%%%%%%%%%%%%%%%%%%%%%%%%%%%%%%%%%%%%%
\section{Information}

%%%%%%%%%%%%%%%%%%%%%%%%%%%%%%%%%%%%%%%%%%%%%%%%%%%%%%%%%%%%%%%%%%%%%%%%%%%%%%%%
\subsection{Copyright}

Copyright \copyright{} 2017--2018 Niklas Beisert

This work may be distributed and/or modified under the
conditions of the \LaTeX{} Project Public License, either version 1.3
of this license or (at your option) any later version.
The latest version of this license is in
  \url{http://www.latex-project.org/lppl.txt}
and version 1.3 or later is part of all distributions of \LaTeX{}
version 2005/12/01 or later.

This work has the LPPL maintenance status `maintained'.

The Current Maintainer of this work is Niklas Beisert.

This work consists of the files |README.txt|, |childdoc.ins| and |childdoc.dtx|
as well as the derived files |childdoc.def|, |cdocsamp.tex|
with |cdocsch1.tex|, |cdocsch2.tex|, |cdocspt3.tex|, |cdocspt4.tex|,
|cdocsdrf.tex|, |cdocsfn1.tex|, |cdocsfn2.tex|
as well as |childdoc.pdf|.

%%%%%%%%%%%%%%%%%%%%%%%%%%%%%%%%%%%%%%%%%%%%%%%%%%%%%%%%%%%%%%%%%%%%%%%%%%%%%%%%
\subsection{Files and Installation}

The package consists of the files:
%
\begin{center}
\begin{tabular}{ll}
    |README.txt|   & readme file \\
    |childdoc.ins| & installation file \\
    |childdoc.dtx| & source file \\
    |childdoc.def| & definition file \\
    |cdocsamp.tex| & sample main file \\
    |cdocsch1.tex| & sample include file \\
    |cdocsch2.tex| & sample include file \\
    |cdocspt3.tex| & sample part file \\
    |cdocspt4.tex| & sample part file \\
    |cdocsdrf.tex| & sample redirection file \\
    |cdocsfn1.tex| & sample redirection file \\
    |cdocsfn2.tex| & sample redirection file \\
    |childdoc.pdf| & manual
\end{tabular}
\end{center}
%
The distribution consists of the files
|README.txt|, |childdoc.ins| and |childdoc.dtx|.
%
\begin{itemize}
\item
Run (pdf)\LaTeX{} on |childdoc.dtx|
to compile the manual |childdoc.pdf| (this file).
\item
Run \LaTeX{} on |childdoc.ins| to create the definitions file |childdoc.def|
and the sample |cdocsamp.tex| with include files
|cdocsch1.tex|, |cdocsch2.tex|, |cdocspt3.tex|, |cdocspt4.tex|,
|cdocsdrf.tex|, |cdocsfn1.tex|, |cdocsfn2.tex|.
Then copy the file |childdoc.def| to an appropriate directory of your \LaTeX{}
distribution, e.g.\ \textit{texmf-root}|/tex/latex/childdoc|.
\end{itemize}

%%%%%%%%%%%%%%%%%%%%%%%%%%%%%%%%%%%%%%%%%%%%%%%%%%%%%%%%%%%%%%%%%%%%%%%%%%%%%%%%
\subsection{Related CTAN Packages}

There are several other packages which offer a similar functionality:
%
\begin{itemize}
\item
The packages
\href{http://ctan.org/pkg/docmute}{\textsf{docmute}},
\href{http://ctan.org/pkg/includex}{\textsf{includex}} and
\href{http://ctan.org/pkg/standalone}{\textsf{standalone}}
provide commands to include only the document body of
a child file thus allowing both files to be compiled individually.
\item
The packages \href{http://ctan.org/pkg/subdocs}{\textsf{subdocs}}
and \href{http://ctan.org/pkg/subfiles}{\textsf{subfiles}}
provide structures in which the main and child documents can be
encapsulated and allowing them to be compiled individually.
The inclusion mechanism is different from the conventional |\include|.
\item
The package \href{http://ctan.org/pkg/combine}{\textsf{combine}}
is an elaborate solution to combine several documents into one.
\end{itemize}
%
See also the CTAN topic \href{http://ctan.org/topic/subdocs}{\textsf{subdocs}}
for further related packages.
The present package differs from the above solutions in that
a document structure constructed with the conventional |\include| mechanism
just needs two extra commands at the top of every file
such that all constituent files can be compiled individually.

%%%%%%%%%%%%%%%%%%%%%%%%%%%%%%%%%%%%%%%%%%%%%%%%%%%%%%%%%%%%%%%%%%%%%%%%%%%%%%%%
%\subsection{Feature Suggestions}
%
%The following is a list of features which may be useful for future
%versions of this package:
%%
%\begin{itemize}
%\item
%\ldots
%\end{itemize}

%%%%%%%%%%%%%%%%%%%%%%%%%%%%%%%%%%%%%%%%%%%%%%%%%%%%%%%%%%%%%%%%%%%%%%%%%%%%%%%%
\subsection{Revision History}

%%%%%%%%%%%%%%%%%%%%%%%%%%%%%%%%%%%%%%%%
\paragraph{v2.0:} 2018/12/30

\begin{itemize}
\item
immediate forward processing
\item
added |\childdocby| mechanism
\item
manual restructured
\end{itemize}

%%%%%%%%%%%%%%%%%%%%%%%%%%%%%%%%%%%%%%%%
\paragraph{v1.6:} 2018/01/17

\begin{itemize}
\item
application for development of include files
\item
corrections to manual
\end{itemize}

%%%%%%%%%%%%%%%%%%%%%%%%%%%%%%%%%%%%%%%%
\paragraph{v1.5:} 2017/05/21

\begin{itemize}
\item
more complete structuring introduced
\item
|\childdocof| introduced
\item
|\childdoc| renamed to |\childdocmain|
\item
|\childredirect| renamed to |\childdocforward| and |\childdocforwardprefix|
and functionality expanded
\end{itemize}

%%%%%%%%%%%%%%%%%%%%%%%%%%%%%%%%%%%%%%%%
\paragraph{v1.0:} 2017/04/27

\begin{itemize}
\item
manual and install package
\item
first version published on CTAN
\end{itemize}

%%%%%%%%%%%%%%%%%%%%%%%%%%%%%%%%%%%%%%%%
\paragraph{v0.6:} 2017/04/26

\begin{itemize}
\item
redirection mechanism added
\end{itemize}

%%%%%%%%%%%%%%%%%%%%%%%%%%%%%%%%%%%%%%%%
\paragraph{v0.5:} 2017/04/26

\begin{itemize}
\item
functionality in definition file
\end{itemize}


%%%%%%%%%%%%%%%%%%%%%%%%%%%%%%%%%%%%%%%%%%%%%%%%%%%%%%%%%%%%%%%%%%%%%%%%%%%%%%%%
%%%%%%%%%%%%%%%%%%%%%%%%%%%%%%%%%%%%%%%%%%%%%%%%%%%%%%%%%%%%%%%%%%%%%%%%%%%%%%%%
%%%%%%%%%%%%%%%%%%%%%%%%%%%%%%%%%%%%%%%%%%%%%%%%%%%%%%%%%%%%%%%%%%%%%%%%%%%%%%%%
\appendix

\settowidth\MacroIndent{\rmfamily\scriptsize 000\ }

 \DocInput{childdoc.dtx}

\end{document}
%</driver>
% \fi
%
% %%%%%%%%%%%%%%%%%%%%%%%%%%%%%%%%%%%%%%%%%%%%%%%%%%%%%%%%%%%%%%%%%%%%%%%%%%%%%%
% %%%%%%%%%%%%%%%%%%%%%%%%%%%%%%%%%%%%%%%%%%%%%%%%%%%%%%%%%%%%%%%%%%%%%%%%%%%%%%
% \section{Sample}
%\iffalse
%<*samplemain>
%\fi
%
% The following presents a sample document
% with two chapters, two parts, a title page,
% a compile flag as well as three forwarding files to set the flag.
% It consists of eight |.tex| files:
% \begin{center}
% \begin{tabular}{ll}
% |cdocsamp.tex|&main file\\
% |cdocsch1.tex|&include file for chapter 1\\
% |cdocsch2.tex|&include file for chapter 2\\
% |cdocspt3.tex|&include file for part 3\\
% |cdocspt4.tex|&include file for part 4\\
% |cdocsdrf.tex|&forwarding file for main file in draft mode\\
% |cdocsfi1.tex|&forwarding file for final version of chapter 1\\
% |cdocsfi2.tex|&forwarding file for final version of chapter 2\\
% \end{tabular}
% \end{center}
% Each of the eight files can be compiled directly by the \LaTeX{} compiler.
%
% %%%%%%%%%%%%%%%%%%%%%%%%%%%%%%%%%%%%%%
% \paragraph{Main File.}
%
% The main file is called |cdocsamp.tex|.
%
% Load the \textsf{childdoc} definitions and
% declare the filename for the main document:
%    \begin{macrocode}
\input{childdoc.def}
\childdocmain{}
%    \end{macrocode}

% Optional override for |\version| flag:
%    \begin{macrocode}
%%\ifchilddoc\else\providecommand{\version}{draft}\fi
%    \end{macrocode}

% Define the default values for the |\version| flag
% (|final| for the main file and |draft| for childs):
%    \begin{macrocode}
\ifchilddoc
\providecommand{\version}{draft}
\else
\providecommand{\version}{final}
\fi
%    \end{macrocode}

% Load the standard document class:
%    \begin{macrocode}
\documentclass[12pt]{article}
%    \end{macrocode}

% Start the document body:
%    \begin{macrocode}
\begin{document}
%    \end{macrocode}

% Declare a title page.
% Print title, part of document being processed and version flag:
%    \begin{macrocode}
\addtocounter{page}{-1}
\begin{center}
{\LARGE\bfseries{}childdoc example\par}
\vspace{1cm}
\ifchilddoc
\ifchilddocmanual part\else chapter\fi:
`\childdocname' of `\childdocjob'\par
\else
main document: `\childdocjob'\par
\fi
version: \version\par
\end{center}
\newpage
%    \end{macrocode}

% Manually include selected file,
% otherwise process as usual:
%    \begin{macrocode}
\ifchilddocmanual
\section*{part `\childdocname'}
\input{\childdocname}
\else
%    \end{macrocode}

% Include the two chapters:
%    \begin{macrocode}
\include{cdocsch1}
\include{cdocsch2}
%    \end{macrocode}

% Include the two parts unless only chapters should be displayed:
%    \begin{macrocode}
\ifchilddoc\else
\section{part three}
\input{cdocspt3}
\section{part four}
\input{cdocspt4}
\fi
%    \end{macrocode}

% Process as usual until here:
%    \begin{macrocode}
\fi
%    \end{macrocode}

% End of document body:
%    \begin{macrocode}
\end{document}
%    \end{macrocode}
%\iffalse
%</samplemain>
%\fi
%
% %%%%%%%%%%%%%%%%%%%%%%%%%%%%%%%%%%%%%%
% \paragraph{Chapter Include Files.}
%
% The include files are called |cdocsch1.tex| and |cdocsch2.tex|.
%
%\iffalse
%<*samplechap1|samplechap2>
%\fi

% Optional override for |\version| flag:
%    \begin{macrocode}
%%\providecommand{\version}{final}
%    \end{macrocode}

% Include the main document:
%    \begin{macrocode}
\input{childdoc.def}
\childdocof{cdocsamp}
%    \end{macrocode}

%\iffalse
%</samplechap1|samplechap2>
%\fi
%
%\iffalse
%<*samplechap1>
%\fi
% Some text for chapter 1:
%    \begin{macrocode}
\section{one}
some text in chapter one
%    \end{macrocode}

%\iffalse
%</samplechap1>
%\fi
% Some text for chapter 2:
%\iffalse
%<*samplechap2>
%\fi
%    \begin{macrocode}
\section{two}
more text in chapter two
%    \end{macrocode}

%\iffalse
%</samplechap2>
%\fi
%
% %%%%%%%%%%%%%%%%%%%%%%%%%%%%%%%%%%%%%%
% \paragraph{Part Include Files.}
%
% The include files are called |cdocspt3.tex| and |cdocspt4.tex|.
%
%\iffalse
%<*samplepart3|samplepart4>
%\fi

% Optional override for |\version| flag:
%    \begin{macrocode}
%%\providecommand{\version}{final}
%    \end{macrocode}

% Include the main document:
%    \begin{macrocode}
\input{childdoc.def}
\childdocby{cdocsamp}
%    \end{macrocode}

%\iffalse
%</samplepart3|samplepart4>
%\fi
%
%\iffalse
%<*samplepart3>
%\fi
% Some text for part 3:
%    \begin{macrocode}
some text in part three
%    \end{macrocode}

%\iffalse
%</samplepart3>
%\fi
% Some text for part 4:
%\iffalse
%<*samplepart4>
%\fi
%    \begin{macrocode}
more text in part four
%    \end{macrocode}

%\iffalse
%</samplepart4>
%\fi
%
% %%%%%%%%%%%%%%%%%%%%%%%%%%%%%%%%%%%%%%
% \paragraph{Forwarding for a Complete Draft.}
%
% The following forwarding file |cdocsdrf.tex|
% compiles the main document in draft mode:
%\iffalse
%<*sampledraft>
%\fi
%    \begin{macrocode}
\def\version{draft}
\input{childdoc.def}
\childdocforward{cdocsamp}
%    \end{macrocode}

%\iffalse
%</sampledraft>
%\fi
%
% %%%%%%%%%%%%%%%%%%%%%%%%%%%%%%%%%%%%%%
% \paragraph{Forwarding for Final Version of the Chapters.}
%
% The following forwarding files |cdocsfn1.tex| and |cdocsfn2.tex|
% (with identical content)
% compile the final versions of the child documents
% |cdocsch1.tex| and |cdocsch2.tex|, respectively:
%\iffalse
%<*samplefinal>
%\fi
%    \begin{macrocode}
\def\version{final}
\input{childdoc.def}
\childdocforwardprefix[cdocsamp]{cdocsfn}{cdocsch}
%    \end{macrocode}

%\iffalse
%</samplefinal>
%\fi
%
% %%%%%%%%%%%%%%%%%%%%%%%%%%%%%%%%%%%%%%
% \paragraph{Command Line Processing.}
%
% The following three command lines generate the output files
% |cdocscld|, |cdocscl1| and |cdocscl2|
% which should be identical to
% |cdocsdrf|, |cdocsch1| and |cdocsfn2|, respectively:
% \begin{center}
% \begin{tabular}{l}
% |latex -jobname cdocscld \|\\
% |  "\def\version{draft}\input{childdoc.def}\childdocforward{cdocsamp}"|\\
% |latex -jobname cdocscl1 \|\\
% |  "\input{childdoc.def}\childdocforward[cdocsamp]{cdocsch1}"|\\
% |latex -jobname cdocscl2 \|\\
% |  "\def\version{final}\input{childdoc.def}\childdocforward{cdocsch2}"|
% \end{tabular}
% \end{center}
% Note that the trailing backslash on each first line
% merely continues the input to the second line
% (for convenient cut ant paste).
% Furthermore, the command |latex| can be replaced by any
% of its alternative versions such as |pdflatex|.
%
% %%%%%%%%%%%%%%%%%%%%%%%%%%%%%%%%%%%%%%%%%%%%%%%%%%%%%%%%%%%%%%%%%%%%%%%%%%%%%%
% %%%%%%%%%%%%%%%%%%%%%%%%%%%%%%%%%%%%%%%%%%%%%%%%%%%%%%%%%%%%%%%%%%%%%%%%%%%%%%
% \section{Implementation}
%\iffalse
%<*package>
%\fi
%
% This section describes the definitions file |childdoc.def|.

% The definitions cannot be loaded using |\usepackage| or |\RequirePackage|
% which has a mechanism to prevent loading a style file more than once.
% When loading the definitions by means of |\input|
% multiple instances have to be prevented manually:
%\iffalse
%This code needs to be before the `\ProvidesFile' directive
%which is defined at the beginning of this file.
%Therefore it is also placed there and commented out here.
%</package>
%<*discard>
%\fi
%    \begin{macrocode}
\ifdefined\childdocmain\endinput\fi
%    \end{macrocode}
%\iffalse
%</discard>
%<*package>
%\fi
%
% \macro{\ifchilddoc}
% \macro{\ifchilddocmanual}
% The conditional |\ifchilddoc| tells whether a
% child (true) or main (false) document is being compiled.
% The conditional |\ifchilddocmanual| tells whether
% the |\includeonly| mechanism is used (false) or
% the selection of child files must be performed manually (true).
% The definitions initialise to false:
%    \begin{macrocode}
\newif\ifchilddoc
\newif\ifchilddocmanual
%    \end{macrocode}

% \macro{\childdocname}
% \macro{\childdocjob}
% The macro |\childdocname| stores the name of the main document
% to be compiled. The macro |\childdocjob| stores the name of
% the document on which the \LaTeX{} compiler was originally invoked.
% The content of |\jobname| cannot be compared
% to filenames specified in the source due to different catcodes.
% The following code rescans |\jobname|, stores the result
% in |\childdocname| and saves a copy in |\childdocjob|:
%    \begin{macrocode}
\edef\childdocname{\scantokens\expandafter{\jobname\noexpand}}
\let\childdocjob\childdocname
%    \end{macrocode}

% \macro{\childdocdisable}
% The macro |\childdocdisable| prevents the main file
% from being processed more than once.
% At this stage, the main document command |\childdocmain|
% is assumed to be called once again where it should do nothing.
% Any subsequent call to it should prevent
% a secondary processing of the main document
% It overwrites the forwarding commands
% |\childdocof| and |\childdocforward|
% with empty macros to prevent further inclusions of the main document:
%    \begin{macrocode}
\newcommand{\childdocdisable}
{
  \renewcommand{\childdocmain}[1]{\renewcommand{\childdocmain}[1]{\endinput}}
  \renewcommand{\childdocof}[1]{}
  \renewcommand{\childdocby}[2][]{}
  \renewcommand{\childdocforward}[2][]{}
  \renewcommand{\childdocdisable}{}
}
%    \end{macrocode}

% \macro{\childdocmain}
% The macro |\childdocmain| is to be called at the top of the main file
% with nothing or the main filename (without extension) as argument.
% First, it breaks loops.
% If the argument is not empty and does not match |\childdocname|
% (which is set by the first inclusion of |childdoc.def|),
% |\ifchilddoc| is set to true, |\includeonly| is applied to the child file
% and |\jobname| is set to the main file
% (for proper handling of |.aux| files):
%    \begin{macrocode}
\newcommand{\childdocmain}[1]
{
  \childdocdisable\childdocmain{}
  \if?#1?\else
    \begingroup
      \def\childdoctmp{#1}
      \ifx\childdoctmp\childdocname
        \def\childdoctmp{}
      \else
        \def\childdoctmp
        {
          \childdoctrue
          \includeonly{\childdocname}
          \def\childdocjob{#1}
          \def\jobname{#1}
        }
      \fi
      \expandafter
    \endgroup
    \childdoctmp
  \fi
}
%    \end{macrocode}

% \macro{\childdocof}
% The command |\childdocof| redirects
% compilation to the main file |#1|.
%    \begin{macrocode}
\newcommand{\childdocof}[1]
{
  \childdocdisable
  \childdoctrue
  \includeonly{\childdocname}
  \def\jobname{#1}
  \def\childdocjob{#1}
  \input{#1}
}
%    \end{macrocode}

% \macro{\childdocby}
% The command |\childdocby| ....
%    \begin{macrocode}
\newcommand{\childdocby}[2][]
{
  \childdocdisable
  \childdoctrue
  \childdocmanualtrue
  \if?#1?\else
    \def\jobname{#2}
  \fi
  \def\childdocjob{#2}
  \input{#2}
  \endinput
}
%    \end{macrocode}

% \macro{\childdocforward}
% The command |\childdocforward| redirects
% compilation to the main file or
% (if the optional argument is given) a child file.
% Parameters are set as if the main file
% or a child file starting with |\childdocof| was compiled.
% Then compilation is handed over to the main file:
%    \begin{macrocode}
\newcommand{\childdocforward}[2][]
{
  \begingroup
    \if?#1?
      \def\childdoctmp
      {
        \def\childdocname{#2}
        \def\childdocjob{#2}
        \def\jobname{#2}
        \input{#2}
        \endinput
      }
    \else
      \def\childdoctmp
      {
        \childdocdisable
        \def\childdocname{#2}
        \childdoctrue
        \includeonly{#2}
        \def\childdocjob{#1}
        \def\jobname{#1}
        \input{#1}
        \endinput
      }
    \fi
    \expandafter
  \endgroup
  \childdoctmp
}
%    \end{macrocode}

% \macro{\childdocforwardprefix}
% The command |\childdocforwardprefix| redirects
% compilation to the main or a child file by means of a pattern.
% The prefix |#1| in the current filename is replaced by |#2|
% and the suffix of the current filename is kept
% (it is assumed that the filename does not contain the substring `|~~~|'
% which is used as a delimiter).
% Compilation is handed over to the new file by |\childdocforward|:
%    \begin{macrocode}
\newcommand{\childdocforwardprefix}[3][]
{
  \begingroup
    \def\childdocextract #2##1~~~{\def\childdoctmp{\childdocforward[#1]{#3##1}}}
    \expandafter\childdocextract\childdocname~~~
    \expandafter
  \endgroup
  \childdoctmp
}
%    \end{macrocode}

% \macro{\childdoc}
% The deprecated macro |\childdoc| is a legacy version of |\childdocmain|:
%    \begin{macrocode}
\newcommand{\childdoc}{\childdocmain}
%    \end{macrocode}

% \macro{\childdocredirect}
% The deprecated macro |\childdocredirect| is a legacy version
% of |\childdocforward| and |\childdocforwardprefix|:
%    \begin{macrocode}
\newcommand{\childdocredirect}[2][]
{
  \begingroup
    \if?#1?
      \def\childdoctmp{\childdocforward{#2}}
    \else
      \def\childdoctmp{\childdocforwardprefix{#1}{#2}}
    \fi
    \expandafter
  \endgroup
  \childdoctmp
}
%    \end{macrocode}

%\iffalse
%</package>
%\fi
%
\endinput

\childdocby{cdocsamp}
%    \end{macrocode}

%\iffalse
%</samplepart3|samplepart4>
%\fi
%
%\iffalse
%<*samplepart3>
%\fi
% Some text for part 3:
%    \begin{macrocode}
some text in part three
%    \end{macrocode}

%\iffalse
%</samplepart3>
%\fi
% Some text for part 4:
%\iffalse
%<*samplepart4>
%\fi
%    \begin{macrocode}
more text in part four
%    \end{macrocode}

%\iffalse
%</samplepart4>
%\fi
%
% %%%%%%%%%%%%%%%%%%%%%%%%%%%%%%%%%%%%%%
% \paragraph{Forwarding for a Complete Draft.}
%
% The following forwarding file |cdocsdrf.tex|
% compiles the main document in draft mode:
%\iffalse
%<*sampledraft>
%\fi
%    \begin{macrocode}
\def\version{draft}
% \iffalse
%
% childdoc.dtx Copyright (C) 2017-2018 Niklas Beisert
%
% This work may be distributed and/or modified under the
% conditions of the LaTeX Project Public License, either version 1.3
% of this license or (at your option) any later version.
% The latest version of this license is in
%   http://www.latex-project.org/lppl.txt
% and version 1.3 or later is part of all distributions of LaTeX
% version 2005/12/01 or later.
%
% This work has the LPPL maintenance status `maintained'.
%
% The Current Maintainer of this work is Niklas Beisert.
%
% This work consists of the files childdoc.dtx and childdoc.ins
% and the derived files childdoc.def and cdocsamp.tex with
% cdocsch1.tex, cdocsch2.tex, cdocsdrf.tex, cdocsfn1.tex, cdocsfn2.tex.
%
%<package>\ifdefined\childdocmain\endinput\fi
%<package>\ProvidesFile{childdoc.def}[2018/12/30 v2.0 child document driver]
%<samplemain>\ProvidesFile{cdocsamp.tex}[2018/12/30 v2.0 sample for childdoc]
%<*driver>
%\ProvidesFile{childdoc.drv}[2018/12/30 v2.0 childdoc reference manual file]
\PassOptionsToClass{10pt,a4paper}{article}
\documentclass{ltxdoc}

\usepackage[margin=35mm]{geometry}
\usepackage{hyperref}
\usepackage{hyperxmp}
\usepackage[usenames]{color}

\hypersetup{colorlinks=true}
\hypersetup{pdfstartview=FitH}
\hypersetup{pdfpagemode=UseNone}
\hypersetup{pdfsource={}}
\hypersetup{pdflang={en-UK}}
\hypersetup{pdfcopyright={Copyright 2017-2018 Niklas Beisert.
  This work may be distributed and/or modified under the
  conditions of the LaTeX Project Public License, either version 1.3
  of this license or (at your option) any later version.}}
\hypersetup{pdflicenseurl={http://www.latex-project.org/lppl.txt}}
\hypersetup{pdfcontactaddress={ETH Zurich, ITP, HIT K,
  Wolfgang-Pauli-Strasse 27}}
\hypersetup{pdfcontactpostcode={8093}}
\hypersetup{pdfcontactcity={Zurich}}
\hypersetup{pdfcontactcountry={Switzerland}}
\hypersetup{pdfcontactemail={nbeisert@itp.phys.ethz.ch}}
\hypersetup{pdfcontacturl={http://people.phys.ethz.ch/\xmptilde nbeisert/}}

\newcommand{\secref}[1]{\hyperref[#1]{section \ref*{#1}}}

\parskip1ex
\parindent0pt
\let\olditemize\itemize
\def\itemize{\olditemize\parskip0pt}

\begin{document}

\title{The \textsf{childdoc} Package}
\hypersetup{pdftitle={The childdoc Package}}
\author{Niklas Beisert\\[2ex]
  Institut f\"ur Theoretische Physik\\
  Eidgen\"ossische Technische Hochschule Z\"urich\\
  Wolfgang-Pauli-Strasse 27, 8093 Z\"urich, Switzerland\\[1ex]
  \href{mailto:nbeisert@itp.phys.ethz.ch}
  {\texttt{nbeisert@itp.phys.ethz.ch}}}
\hypersetup{pdfauthor={Niklas Beisert}}
\hypersetup{pdfsubject={Manual for the LaTeX2e Package childdoc}}
\date{30 December 2018, \textsf{v2.0}}
\maketitle

\begin{abstract}\noindent
\textsf{childdoc} is a \LaTeXe{} package
that enables the direct compilation
of document sections included by |\include|
to individual files.
\end{abstract}

\begingroup
\parskip0ex
\tableofcontents
\endgroup

%%%%%%%%%%%%%%%%%%%%%%%%%%%%%%%%%%%%%%%%%%%%%%%%%%%%%%%%%%%%%%%%%%%%%%%%%%%%%%%%
%%%%%%%%%%%%%%%%%%%%%%%%%%%%%%%%%%%%%%%%%%%%%%%%%%%%%%%%%%%%%%%%%%%%%%%%%%%%%%%%
\section{Introduction}

\LaTeX{} provides a mechanism to structure a large document (such as a book)
into a main file and several child files (containing the chapters)
using the |\include| command.
This mechanism is beneficial for documents
which span hundreds of pages in order to
make the source file(s) more manageable.
Moreover, compilation can be restricted to
selected child files by means of the |\includeonly| command.
The latter feature can be used to reduce the compilation time while editing
(this was significantly more useful in the earlier days of \LaTeX{})
or to generate a smaller document which is easier to navigate.
Another application of |\includeonly| is to generate
documents consisting of selected parts of the complete document.

However, there are a few drawbacks of the plain |\include| mechanism:
\begin{itemize}
\item
The child files cannot be compiled on their own,
they can only be compiled via the main file.
A naive editing environment
(such as a text editor with an option
to have the current file processed by \LaTeX)
may require one to switch to the main file before compiling;
attempting to compile the child file produces errors.
\item
The main file must be modified (each time)
to adjust the |\includeonly| command
to the present needs. This easily leaves the main file in a messy state.
\item
The generated document will always carry the filename
of the main document. This is inconvenient if
several child files are to be compiled and
to be kept for distribution.
\end{itemize}

The present package provides a simple interface
to make child files individually compilable by \LaTeX{}.
Compiling a child file then has the same effect as compiling
the main file with an |\includeonly| command
to select the appropriate child.
Moreover the generated document will carry the name of the child
rather than the main file.
This resolves all three above issues.

This feature is meant to make the editing of books,
thesis documents and lecture notes somewhat more convenient.
However, the package can also be used efficiently for
composing a series of documents (such as exercise sheets)
which are typically distributed individually.
It then assists the author in generating the individual documents
(potentially in different versions)
as well as a document containing the collected series.
Another application is in developing style files
or other kinds of included material
where compilation of the style file could redirect
to a sample or test file.

%%%%%%%%%%%%%%%%%%%%%%%%%%%%%%%%%%%%%%%%%%%%%%%%%%%%%%%%%%%%%%%%%%%%%%%%%%%%%%%%
%%%%%%%%%%%%%%%%%%%%%%%%%%%%%%%%%%%%%%%%%%%%%%%%%%%%%%%%%%%%%%%%%%%%%%%%%%%%%%%%
\section{Usage}

First of all, the package \textsf{childdoc} is \emph{not} a standard
\LaTeXe{} |.sty| style file! Therefore it needs to be invoked in
a non-standard way.

%%%%%%%%%%%%%%%%%%%%%%%%%%%%%%%%%%%%%%%%%%%%%%%%%%%%%%%%%%%%%%%%%%%%%%%%%%%%%%%%
\subsection{Included Files}
\label{sec:include}

%%%%%%%%%%%%%%%%%%%%%%%%%%%%%%%%%%%%%%%%
\DescribeMacro{\childdocmain}
To use the package, add the commands
\begin{center}
\begin{tabular}{l}
|\input{childdoc.def}|\\
|\childdocmain{}|\\
\end{tabular}
\end{center}
at the very top of the main \LaTeX{} file,
in particular \emph{before} the |\documentclass| statement!
The argument of |\childdocmain| should be left empty
(but it must be present).

%%%%%%%%%%%%%%%%%%%%%%%%%%%%%%%%%%%%%%%%
\DescribeMacro{\childdocof}
Furthermore, add the commands
\begin{center}
\begin{tabular}{l}
|\input{childdoc.def}|\\
|\childdocof{|\textit{main}|}|\\
\end{tabular}
\end{center}
at the top of every child file \textit{child}
which is included by |\include{|\textit{child}|}|
from within the main file
(or at least for those files to be compiled individually).
The argument \textit{main} must be the filename of the main file.

There are a couple of
considerations in setting up the main and child documents:

%%%%%%%%%%%%%%%%%%%%%%%%%%%%%%%%%%%%%%%%
\paragraph{Restrictions.}

Please note the following restrictions:
\begin{itemize}
\item
|\childdocmain| must be called with one argument \textit{main}
to ensure compatibility with earlier version of the package.
It must either be empty (|\childdocmain{}|)
or precisely match the filename of the main file in which it is specified.
See \secref{sec:detection} for further information.
\item
The filename \textit{main} must be specified without the |.tex| extension.
\item
The filename \textit{main} is case sensitive
(even in case-insensitive file systems)
due to internal string comparison.
\item
The argument \textit{main} should be fully expanded, it cannot be a macro.
\item
Subdirectories and special characters should be avoided in filenames.
\item
The command |\childdocmain{|\textit{main}|}| must be followed by a whitespace.
It should not be followed immediately by another command
or by a comment mark `|%|'.
This is because the \TeX{} parser reads the token immediately following
the argument of |\childdocmain| and puts it
at the beginning of every child section;
however, a white\-space is ignored.
\end{itemize}

%%%%%%%%%%%%%%%%%%%%%%%%%%%%%%%%%%%%%%%%
\paragraph{Content of Main File.}

It is advisable to place all content in the child files included by |\include|.
Any output contained in the main file will appear in all child documents
unless suppressed manually;
it cannot be suppressed automatically by the |\includeonly| directive
and thus should normally be avoided.
A method to include some content in the main file
by means of conditional processing is described in \secref{sec:conditional}.

%%%%%%%%%%%%%%%%%%%%%%%%%%%%%%%%%%%%%%%%
\paragraph{Page Numbering.}

When only a part of the document is compiled,
the appropriate numbering of pages
(as well as other status parameters)
is determined from the |.aux| files.
The latter contain information from previous passes.
However this information needs to propagate through
all intermediate child documents.
Therefore the page numbering in child documents may well
be inconsistent until the complete document is compiled at least once.

A useful (if unconventional) way to always ensure a consistent
page numbering is to restart the numbering in each child document
and denote the pages by `\textit{child}|.|\textit{page}'
where \textit{child} represents the chapter/section number of the child file.
This can be achieved by the command
|\numberwithin{page}{|\textit{child}|}|
of the \textsf{amsmath} package
where \textit{child} can be |chapter| or |section|
depending on the chosen structuring.
Alternatively, one can modify the macro |\thepage| appropriately
and reset the counter |page| at the start of each child file.

%%%%%%%%%%%%%%%%%%%%%%%%%%%%%%%%%%%%%%%%%%%%%%%%%%%%%%%%%%%%%%%%%%%%%%%%%%%%%%%%
\subsection{Conditional Processing}
\label{sec:conditional}

The package provides a mechanism to compile different versions
of a document. To customise the versions further some conditional processing
can come in handy to distinguish which version is being compiled.
The package provides two macros to describe the compilation context:

%%%%%%%%%%%%%%%%%%%%%%%%%%%%%%%%%%%%%%%%
\DescribeMacro{\ifchilddoc}
The conditional |\ifchilddoc| distinguishes between the compilation of
child documents and the main document:
%
\begin{center}
|\ifchilddoc |\textit{child-code}| |[|\||else |\textit{main-code}]| \||fi|
\end{center}

%%%%%%%%%%%%%%%%%%%%%%%%%%%%%%%%%%%%%%%%
\DescribeMacro{\childdocname}
\DescribeMacro{\childdocjob}
The macro |\childdocname| contains the filename (without extension)
of the main or child file being processed.
Note that |\childdocjob| will always contain the name of the main file.

%%%%%%%%%%%%%%%%%%%%%%%%%%%%%%%%%%%%%%%%
\paragraph{Title Page.}

Conditional processing can be used to include a title or banner page
in the main document when proper precautions are taken.
Importantly, the code in the main file should ensure that the page counter
(as well as other status parameters which are stored in the |.aux| files)
takes the same value after the conditional processing.
Otherwise the page numbers may take divergent values
depending on which part is compiled.

For example, a title page could be declared by:
%
\begin{center}
\begin{tabular}{l}
|\ifchilddoc\||else|\\
|\addtocounter{page}{-1}|\\
\textit{code for title page}\\
|\newpage|\\
|\||fi|
\end{tabular}
\end{center}
%
A banner page for the child documents can be generated by:
%
\begin{center}
\begin{tabular}{l}
|\ifchilddoc|\\
|\addtocounter{page}{-1}|\\
\textit{code for banner page}\\
|\newpage|\\
|\||fi|
\end{tabular}
\end{center}
%
Here one could write a message such as:
\begin{center}
|This is the part \childdocname{} of \childdocjob{}.|
\end{center}

%%%%%%%%%%%%%%%%%%%%%%%%%%%%%%%%%%%%%%%%%%%%%%%%%%%%%%%%%%%%%%%%%%%%%%%%%%%%%%%%
\subsection{Flags}
\label{sec:flags}

The package makes it easy to generate different versions
of the main or child documents.
To this end compilation flags can be defined
and assigned different default values.
They will be particularly useful in conjunction
with the forwarding mechanism described in \secref{sec:forward}.

For example, it may be useful to have a flag |\version|
which can be set to |draft| or |final|.
The document source will contain some conditional code
depending on the value of |\version|.
Suppose further, the flag should default to |final| for the main file
and to |draft| for child files
which is a natural assignment for editing the document.
This is achieved by placing the following code
in the preamble of the main document
(below the |\childdocmain| directive):
%
\begin{center}
\begin{tabular}{l}
|\ifchilddoc|\\
|\providecommand{\version}{draft}|\\
|\||else|\\
|\providecommand{\version}{final}|\\
|\||fi|
\end{tabular}
\end{center}
%
The definition by |\providecommand| makes sure
that previous definitions are not overwritten.
Further statements |\providecommand{\version}{...}|
can thus be added before the above code to override it.

For the main file, one might add a line
(between |\childdocmain| and the above block)
%
\begin{center}
|%\ifchilddoc\||else\providecommand{\version}{draft}\||fi|
\end{center}
%
which can be uncommented to produce a draft version.
Likewise one can add a line to the very top of a child file
(above the |\childdocof{|\textit{main}|}| directive)
%
\begin{center}
|%\providecommand{\version}{final}|
\end{center}
%
which can be uncommented to produce the final version of this child document.

%%%%%%%%%%%%%%%%%%%%%%%%%%%%%%%%%%%%%%%%%%%%%%%%%%%%%%%%%%%%%%%%%%%%%%%%%%%%%%%%
\subsection{Forwarding}
\label{sec:forward}

Different versions of the main or child documents
using compilation flags as described in \secref{sec:flags}
can be (permanently) stored in different files
for convenient compilation, viewing and distribution.
To this end, the package defines a command
to pass on compilation to a different file:

%%%%%%%%%%%%%%%%%%%%%%%%%%%%%%%%%%%%%%%%
\DescribeMacro{\childdocforward}
The command |\childdocforward| redirects processing to
another source file:
%
\begin{center}
\begin{tabular}{l}
|\input{childdoc.def}|\\
|\childdocforward[|\textit{main}|]{|\textit{dest}|}|\\
\end{tabular}
\end{center}
%
The argument \textit{dest} is the destination file
(without extension).
It should be the main file or one of the child files.
Note that further \textsf{childdoc} directives
such as |\childdocof| and |\childdocforward|
in the indicated file will be processed in this form.
The optional argument \textit{main}
passes on directly to the main file \textit{main}
while pretending to compile the child \textit{dest}.
This form behaves as if \textit{dest}
issues |\childdocof{|\textit{main}|}| right away,
and no further \textsf{childdoc} directives will be processed.

%%%%%%%%%%%%%%%%%%%%%%%%%%%%%%%%%%%%%%%%
\DescribeMacro{\...prefix}
In the alternative form |\childdocforwardprefix|,
%
\begin{center}
\begin{tabular}{l}
|\input{childdoc.def}|\\
|\childdocforwardprefix[|\textit{main}|]{|\textit{prefix}|}{|\textit{dest}|}|
\end{tabular}
\end{center}
%
the destination file is determined by a pattern
depending on the current file:
To make this work, the current file must be called
`{\textit{prefix}\hspace{0.2em}\textit{suffix}}'
with \textit{prefix} matching precisely the argument.
Processing is then passed on to the file
`{\textit{dest}\hspace{0.2em}\textit{suffix}}'.
Surely, the same effect is achieved by
directly specifying the
argument `{\textit{dest}\hspace{0.2em}\textit{suffix}}'
in the first form.
However, that requires to set up a different file
for each child. With the alternative form of the command
all these files can have exactly the same content
which simplifies setting them up and maintaining them.

For example, the following file |draft.tex|
with a compilation flag |\version| as described in \secref{sec:flags}
compiles the main document as a draft:
%
\begin{center}
\begin{tabular}{l}
|\def\version{draft}|\\
|\input{childdoc.def}|\\
|\childdocforward{|\textit{main}|}|
\end{tabular}
\end{center}
%
Likewise, the following files |final|\textit{nn}|.tex|
compile the final version of the child document
|child|\textit{nn}|.tex|:
%
\begin{center}
\begin{tabular}{l}
|\def\version{final}|\\
|\input{childdoc.def}|\\
|\childdocforwardprefix{final}{child}|
\end{tabular}
\end{center}
%

Note that when several versions of a main file and/or of each child file
are to be generated, it may be convenient to set up a |Makefile| or
shell script to automatise the process.

%%%%%%%%%%%%%%%%%%%%%%%%%%%%%%%%%%%%%%%%%%%%%%%%%%%%%%%%%%%%%%%%%%%%%%%%%%%%%%%%
\subsection{Command Line Processing}
\label{sec:commandline}

The effect of redirection files can also be achieved by invoking
the \LaTeX{} compiler with a more elaborate command line.
Most conveniently this should be done as part
of a shell script or a |Makefile|.

When using \textsf{childdoc} in the main file, the following
command lines effectively perform a redirection
(note that depending on the shell being used,
backslashes may have to be doubled: `|\|' $\to$ `|\\|'):
%
\begin{center}
|... -jobname "|\textit{target}|" |\\|"|[\textit{flags}]%
|\input{childdoc.def}\childdocforward[|\textit{main}|]{|\textit{dest}|}"|
\end{center}
%
Here \textit{target} is the name of the output file,
\textit{main} is the name of the main file
and \textit{dest} is the name of the main or child file to be processed
(all filenames without extensions).
The optional argument \textit{main} can be omitted
if \textit{main} matches \textit{dest}.
Optionally, compilation \textit{flags} can be defined via |\def| commands.
This command line makes the \TeX{} engine believe
it is compiling the file \textit{target}
whose content is specified as the latter parameter.
The provided code then forwards the processing to
\textit{main} or \textit{dest} as described in \secref{sec:forward}.

%%%%%%%%%%%%%%%%%%%%%%%%%%%%%%%%%%%%%%%%%%%%%%%%%%%%%%%%%%%%%%%%%%%%%%%%%%%%%%%%
\subsection{Include by Input}
\label{sec:input}

Including child documents by |\include| has some restrictions by design.
Most notably, the content of a child document always occupies
its own set of pages; pages cannot be shared between child documents.
Usually, this behaviour makes perfect sense
because each child document contain an essential part of the document.
However, in some situations it may be desirable to compose
a document from a collection of parts
without having mandatory page breaks between then.
For this case, the package
provides a mechanism to include parts
by |\input| which can also be processed individually.
However, by construction this mechanism
requires manual handling of the content to be output.

%%%%%%%%%%%%%%%%%%%%%%%%%%%%%%%%%%%%%%%%
\DescribeMacro{\ifchilddocmanual}
The main file should be prepared as usual, see \secref{sec:include}.
However, the document body must make a distinction
between processing of an individual part and of the main document, e.g.:
%
\begin{center}
\begin{tabular}{l}
|\ifchilddocmanual|\\
|\input{\childdocname}|\\
|\||else|\\
\textit{document body with }|\input{|\textit{part}|}|\\
|\||fi|
\end{tabular}
\end{center}
%
The conditional |\ifchilddocmanual| is true whenever
a part to be included by |\input| is being compiled,
and the name of the part is stored in |\childdocname|.

%%%%%%%%%%%%%%%%%%%%%%%%%%%%%%%%%%%%%%%%
\DescribeMacro{\childdocby}
Each part to be included by |\input| should start with:
%
\begin{center}
\begin{tabular}{l}
|\input{childdoc.def}|\\
|\childdocby{|\textit{main}|}|\\
\end{tabular}
\end{center}
%
The directive |\childdocby| is similar to |\childdocof|
described in \secref{sec:include},
but the subsequent selection of content must be done manually.
To that end, both |\ifchilddoc| and |\ifchilddocmanual|
will be true upon processing of a part,
and the name of the part is stored in |\childdocname|.
Note that |\jobname| will be set to the filename of the current part
so that each part receives an individual |.aux| file
that does not interfere with the |.aux| file(s) of the main document.
This behaviour can be altered by the alternative form
|\childdocby[*]{|\textit{main}|}| (with a non-empty optional argument)
which uses the |.aux| file of the main document
by setting |\jobname| to \textit{main}.

%%%%%%%%%%%%%%%%%%%%%%%%%%%%%%%%%%%%%%%%%%%%%%%%%%%%%%%%%%%%%%%%%%%%%%%%%%%%%%%%
\subsection{Driver Development}
\label{sec:driver}

The \textsf{childdoc} mechanism can also be use for the development
of definition files such as \LaTeX{} styles or classes.
This case differs from the above setup with multiple parts
included by |\include| in that no |\includeonly| should be invoked.
This can be achieved by starting the include file
(before |\ProvidesPackage|) with:
%
\begin{center}
\begin{tabular}{l}
|\input{childdoc.def}|\\
|\childdocforward{|\textit{main}|}|\\
\end{tabular}
\end{center}
%
or alternatively with:
%
\begin{center}
\begin{tabular}{l}
|\input{childdoc.def}|\\
|\childdocby{|\textit{main}|}|\\
\end{tabular}
\end{center}
%
Both forms have slightly different effects as described above.
The main file is prepared as usual, see \secref{sec:include}.

%%%%%%%%%%%%%%%%%%%%%%%%%%%%%%%%%%%%%%%%%%%%%%%%%%%%%%%%%%%%%%%%%%%%%%%%%%%%%%%%
\subsection{Legacy Detection}
\label{sec:detection}

The directive |\childdocmain| in the main file can detect
whether the complete document or merely a child is to be compiled
even without using the directive |\childdocof|.
This method is deprecated because it is less robust
and there is no compelling reason to use it;
it is merely provided for backward compatibility
and it may be removed in future versions.

If the detection mechanism is to be used,
it is mandatory to correctly specify
the filename of the main file as the argument of |\childdocmain|:
%
\begin{center}
\begin{tabular}{l}
|\input{childdoc.def}|\\
|\childdocmain{|\textit{main}|}|\\
\end{tabular}
\end{center}
%
If |\jobname| does not match the argument \textit{main} of |\childdocmain|,
it is assumed that |\jobname| points to the child file to be compiled.
When using |\childdocmain| with the main file specified as argument,
it suffices to start a child file
with just |\input{|\textit{main}|}|
without loading of the package and using |\childdocof|.
If instead all processing is done
with the appropriate \textsf{childdoc} directives,
the argument of \textit{main} of |\childdocmain| can be empty.

An alternative version of the command line processing described
in \secref{sec:commandline} using the detection mechanism reads:
%
\begin{center}
|... -jobname "|\textit{target}|" "|[\textit{flags}]%
[|\def\jobname{|\textit{dest}|}|]|\input{|\textit{main}|}"|
\end{center}

%%%%%%%%%%%%%%%%%%%%%%%%%%%%%%%%%%%%%%%%%%%%%%%%%%%%%%%%%%%%%%%%%%%%%%%%%%%%%%%%
\subsection{Manual Code}
\label{sec:manual}

In case one cannot be certain whether the definitions file |childdoc.def|
is installed on the target \TeX{} distribution
and one prefers not to ship it,
it is conceivable to paste a few relevant commands into the sources.

To that end, drop all statements |\input{childdoc.def}|
and perform the replacements as outlined below.
Instead of |\childdocmain{|\textit{main}|}| add the following code
to the top of the main file:
%
\begin{center}
\begin{tabular}{l}
|\||ifdefined\childdocname\endinput\||fi\newif\ifchilddoc|\\
|\edef\childdocname{\scantokens\expandafter{\jobname\noexpand}}|\\
|\def\childdocmain{|\textit{main}|}\||ifx\childdocmain\childdocname\||else|\\
|\childdoctrue\includeonly{\childdocname}\let\jobname\childdocmain\||fi|\\
\end{tabular}
\end{center}
%
Instead of |\childdocof{|\textit{main}|}| just include the main file
at the top of each child file:
%
\begin{center}
|\input{|\textit{main}|}|
\end{center}
%
A simple redirection |\childdocforward{|\textit{dest}|}| is achieved by:
%
\begin{center}
|\def\jobname{|\textit{dest}|}\input{\jobname}|
\end{center}
%
The redirection with prefix
|\childdocforwardprefix[|\textit{prefix}|]{|\textit{dest}|}|
is accomplished by:
%
\begin{center}
\begin{tabular}{l}
|{\edef\jobname{\scantokens\expandafter{\jobname\noexpand}}|\\
|\def\redirectjob |\textit{prefix}|#1~~~{\gdef\jobname{|\textit{dest}|#1}}|\\
|\expandafter\redirectjob\jobname~~~}\input{\jobname}|
\end{tabular}
\end{center}

In an alternative approach,
child documents can be compiled by a specific command line
without additional code or specific definitions:
%
\begin{center}
|... -jobname "|\textit{target}|" "|[\textit{flags}]%
|\includeonly{|\textit{dest}|}\input{|\textit{main}|}"|
\end{center}
%

%%%%%%%%%%%%%%%%%%%%%%%%%%%%%%%%%%%%%%%%%%%%%%%%%%%%%%%%%%%%%%%%%%%%%%%%%%%%%%%%
%%%%%%%%%%%%%%%%%%%%%%%%%%%%%%%%%%%%%%%%%%%%%%%%%%%%%%%%%%%%%%%%%%%%%%%%%%%%%%%%
\section{Information}

%%%%%%%%%%%%%%%%%%%%%%%%%%%%%%%%%%%%%%%%%%%%%%%%%%%%%%%%%%%%%%%%%%%%%%%%%%%%%%%%
\subsection{Copyright}

Copyright \copyright{} 2017--2018 Niklas Beisert

This work may be distributed and/or modified under the
conditions of the \LaTeX{} Project Public License, either version 1.3
of this license or (at your option) any later version.
The latest version of this license is in
  \url{http://www.latex-project.org/lppl.txt}
and version 1.3 or later is part of all distributions of \LaTeX{}
version 2005/12/01 or later.

This work has the LPPL maintenance status `maintained'.

The Current Maintainer of this work is Niklas Beisert.

This work consists of the files |README.txt|, |childdoc.ins| and |childdoc.dtx|
as well as the derived files |childdoc.def|, |cdocsamp.tex|
with |cdocsch1.tex|, |cdocsch2.tex|, |cdocspt3.tex|, |cdocspt4.tex|,
|cdocsdrf.tex|, |cdocsfn1.tex|, |cdocsfn2.tex|
as well as |childdoc.pdf|.

%%%%%%%%%%%%%%%%%%%%%%%%%%%%%%%%%%%%%%%%%%%%%%%%%%%%%%%%%%%%%%%%%%%%%%%%%%%%%%%%
\subsection{Files and Installation}

The package consists of the files:
%
\begin{center}
\begin{tabular}{ll}
    |README.txt|   & readme file \\
    |childdoc.ins| & installation file \\
    |childdoc.dtx| & source file \\
    |childdoc.def| & definition file \\
    |cdocsamp.tex| & sample main file \\
    |cdocsch1.tex| & sample include file \\
    |cdocsch2.tex| & sample include file \\
    |cdocspt3.tex| & sample part file \\
    |cdocspt4.tex| & sample part file \\
    |cdocsdrf.tex| & sample redirection file \\
    |cdocsfn1.tex| & sample redirection file \\
    |cdocsfn2.tex| & sample redirection file \\
    |childdoc.pdf| & manual
\end{tabular}
\end{center}
%
The distribution consists of the files
|README.txt|, |childdoc.ins| and |childdoc.dtx|.
%
\begin{itemize}
\item
Run (pdf)\LaTeX{} on |childdoc.dtx|
to compile the manual |childdoc.pdf| (this file).
\item
Run \LaTeX{} on |childdoc.ins| to create the definitions file |childdoc.def|
and the sample |cdocsamp.tex| with include files
|cdocsch1.tex|, |cdocsch2.tex|, |cdocspt3.tex|, |cdocspt4.tex|,
|cdocsdrf.tex|, |cdocsfn1.tex|, |cdocsfn2.tex|.
Then copy the file |childdoc.def| to an appropriate directory of your \LaTeX{}
distribution, e.g.\ \textit{texmf-root}|/tex/latex/childdoc|.
\end{itemize}

%%%%%%%%%%%%%%%%%%%%%%%%%%%%%%%%%%%%%%%%%%%%%%%%%%%%%%%%%%%%%%%%%%%%%%%%%%%%%%%%
\subsection{Related CTAN Packages}

There are several other packages which offer a similar functionality:
%
\begin{itemize}
\item
The packages
\href{http://ctan.org/pkg/docmute}{\textsf{docmute}},
\href{http://ctan.org/pkg/includex}{\textsf{includex}} and
\href{http://ctan.org/pkg/standalone}{\textsf{standalone}}
provide commands to include only the document body of
a child file thus allowing both files to be compiled individually.
\item
The packages \href{http://ctan.org/pkg/subdocs}{\textsf{subdocs}}
and \href{http://ctan.org/pkg/subfiles}{\textsf{subfiles}}
provide structures in which the main and child documents can be
encapsulated and allowing them to be compiled individually.
The inclusion mechanism is different from the conventional |\include|.
\item
The package \href{http://ctan.org/pkg/combine}{\textsf{combine}}
is an elaborate solution to combine several documents into one.
\end{itemize}
%
See also the CTAN topic \href{http://ctan.org/topic/subdocs}{\textsf{subdocs}}
for further related packages.
The present package differs from the above solutions in that
a document structure constructed with the conventional |\include| mechanism
just needs two extra commands at the top of every file
such that all constituent files can be compiled individually.

%%%%%%%%%%%%%%%%%%%%%%%%%%%%%%%%%%%%%%%%%%%%%%%%%%%%%%%%%%%%%%%%%%%%%%%%%%%%%%%%
%\subsection{Feature Suggestions}
%
%The following is a list of features which may be useful for future
%versions of this package:
%%
%\begin{itemize}
%\item
%\ldots
%\end{itemize}

%%%%%%%%%%%%%%%%%%%%%%%%%%%%%%%%%%%%%%%%%%%%%%%%%%%%%%%%%%%%%%%%%%%%%%%%%%%%%%%%
\subsection{Revision History}

%%%%%%%%%%%%%%%%%%%%%%%%%%%%%%%%%%%%%%%%
\paragraph{v2.0:} 2018/12/30

\begin{itemize}
\item
immediate forward processing
\item
added |\childdocby| mechanism
\item
manual restructured
\end{itemize}

%%%%%%%%%%%%%%%%%%%%%%%%%%%%%%%%%%%%%%%%
\paragraph{v1.6:} 2018/01/17

\begin{itemize}
\item
application for development of include files
\item
corrections to manual
\end{itemize}

%%%%%%%%%%%%%%%%%%%%%%%%%%%%%%%%%%%%%%%%
\paragraph{v1.5:} 2017/05/21

\begin{itemize}
\item
more complete structuring introduced
\item
|\childdocof| introduced
\item
|\childdoc| renamed to |\childdocmain|
\item
|\childredirect| renamed to |\childdocforward| and |\childdocforwardprefix|
and functionality expanded
\end{itemize}

%%%%%%%%%%%%%%%%%%%%%%%%%%%%%%%%%%%%%%%%
\paragraph{v1.0:} 2017/04/27

\begin{itemize}
\item
manual and install package
\item
first version published on CTAN
\end{itemize}

%%%%%%%%%%%%%%%%%%%%%%%%%%%%%%%%%%%%%%%%
\paragraph{v0.6:} 2017/04/26

\begin{itemize}
\item
redirection mechanism added
\end{itemize}

%%%%%%%%%%%%%%%%%%%%%%%%%%%%%%%%%%%%%%%%
\paragraph{v0.5:} 2017/04/26

\begin{itemize}
\item
functionality in definition file
\end{itemize}


%%%%%%%%%%%%%%%%%%%%%%%%%%%%%%%%%%%%%%%%%%%%%%%%%%%%%%%%%%%%%%%%%%%%%%%%%%%%%%%%
%%%%%%%%%%%%%%%%%%%%%%%%%%%%%%%%%%%%%%%%%%%%%%%%%%%%%%%%%%%%%%%%%%%%%%%%%%%%%%%%
%%%%%%%%%%%%%%%%%%%%%%%%%%%%%%%%%%%%%%%%%%%%%%%%%%%%%%%%%%%%%%%%%%%%%%%%%%%%%%%%
\appendix

\settowidth\MacroIndent{\rmfamily\scriptsize 000\ }

 \DocInput{childdoc.dtx}

\end{document}
%</driver>
% \fi
%
% %%%%%%%%%%%%%%%%%%%%%%%%%%%%%%%%%%%%%%%%%%%%%%%%%%%%%%%%%%%%%%%%%%%%%%%%%%%%%%
% %%%%%%%%%%%%%%%%%%%%%%%%%%%%%%%%%%%%%%%%%%%%%%%%%%%%%%%%%%%%%%%%%%%%%%%%%%%%%%
% \section{Sample}
%\iffalse
%<*samplemain>
%\fi
%
% The following presents a sample document
% with two chapters, two parts, a title page,
% a compile flag as well as three forwarding files to set the flag.
% It consists of eight |.tex| files:
% \begin{center}
% \begin{tabular}{ll}
% |cdocsamp.tex|&main file\\
% |cdocsch1.tex|&include file for chapter 1\\
% |cdocsch2.tex|&include file for chapter 2\\
% |cdocspt3.tex|&include file for part 3\\
% |cdocspt4.tex|&include file for part 4\\
% |cdocsdrf.tex|&forwarding file for main file in draft mode\\
% |cdocsfi1.tex|&forwarding file for final version of chapter 1\\
% |cdocsfi2.tex|&forwarding file for final version of chapter 2\\
% \end{tabular}
% \end{center}
% Each of the eight files can be compiled directly by the \LaTeX{} compiler.
%
% %%%%%%%%%%%%%%%%%%%%%%%%%%%%%%%%%%%%%%
% \paragraph{Main File.}
%
% The main file is called |cdocsamp.tex|.
%
% Load the \textsf{childdoc} definitions and
% declare the filename for the main document:
%    \begin{macrocode}
\input{childdoc.def}
\childdocmain{}
%    \end{macrocode}

% Optional override for |\version| flag:
%    \begin{macrocode}
%%\ifchilddoc\else\providecommand{\version}{draft}\fi
%    \end{macrocode}

% Define the default values for the |\version| flag
% (|final| for the main file and |draft| for childs):
%    \begin{macrocode}
\ifchilddoc
\providecommand{\version}{draft}
\else
\providecommand{\version}{final}
\fi
%    \end{macrocode}

% Load the standard document class:
%    \begin{macrocode}
\documentclass[12pt]{article}
%    \end{macrocode}

% Start the document body:
%    \begin{macrocode}
\begin{document}
%    \end{macrocode}

% Declare a title page.
% Print title, part of document being processed and version flag:
%    \begin{macrocode}
\addtocounter{page}{-1}
\begin{center}
{\LARGE\bfseries{}childdoc example\par}
\vspace{1cm}
\ifchilddoc
\ifchilddocmanual part\else chapter\fi:
`\childdocname' of `\childdocjob'\par
\else
main document: `\childdocjob'\par
\fi
version: \version\par
\end{center}
\newpage
%    \end{macrocode}

% Manually include selected file,
% otherwise process as usual:
%    \begin{macrocode}
\ifchilddocmanual
\section*{part `\childdocname'}
\input{\childdocname}
\else
%    \end{macrocode}

% Include the two chapters:
%    \begin{macrocode}
\include{cdocsch1}
\include{cdocsch2}
%    \end{macrocode}

% Include the two parts unless only chapters should be displayed:
%    \begin{macrocode}
\ifchilddoc\else
\section{part three}
\input{cdocspt3}
\section{part four}
\input{cdocspt4}
\fi
%    \end{macrocode}

% Process as usual until here:
%    \begin{macrocode}
\fi
%    \end{macrocode}

% End of document body:
%    \begin{macrocode}
\end{document}
%    \end{macrocode}
%\iffalse
%</samplemain>
%\fi
%
% %%%%%%%%%%%%%%%%%%%%%%%%%%%%%%%%%%%%%%
% \paragraph{Chapter Include Files.}
%
% The include files are called |cdocsch1.tex| and |cdocsch2.tex|.
%
%\iffalse
%<*samplechap1|samplechap2>
%\fi

% Optional override for |\version| flag:
%    \begin{macrocode}
%%\providecommand{\version}{final}
%    \end{macrocode}

% Include the main document:
%    \begin{macrocode}
\input{childdoc.def}
\childdocof{cdocsamp}
%    \end{macrocode}

%\iffalse
%</samplechap1|samplechap2>
%\fi
%
%\iffalse
%<*samplechap1>
%\fi
% Some text for chapter 1:
%    \begin{macrocode}
\section{one}
some text in chapter one
%    \end{macrocode}

%\iffalse
%</samplechap1>
%\fi
% Some text for chapter 2:
%\iffalse
%<*samplechap2>
%\fi
%    \begin{macrocode}
\section{two}
more text in chapter two
%    \end{macrocode}

%\iffalse
%</samplechap2>
%\fi
%
% %%%%%%%%%%%%%%%%%%%%%%%%%%%%%%%%%%%%%%
% \paragraph{Part Include Files.}
%
% The include files are called |cdocspt3.tex| and |cdocspt4.tex|.
%
%\iffalse
%<*samplepart3|samplepart4>
%\fi

% Optional override for |\version| flag:
%    \begin{macrocode}
%%\providecommand{\version}{final}
%    \end{macrocode}

% Include the main document:
%    \begin{macrocode}
\input{childdoc.def}
\childdocby{cdocsamp}
%    \end{macrocode}

%\iffalse
%</samplepart3|samplepart4>
%\fi
%
%\iffalse
%<*samplepart3>
%\fi
% Some text for part 3:
%    \begin{macrocode}
some text in part three
%    \end{macrocode}

%\iffalse
%</samplepart3>
%\fi
% Some text for part 4:
%\iffalse
%<*samplepart4>
%\fi
%    \begin{macrocode}
more text in part four
%    \end{macrocode}

%\iffalse
%</samplepart4>
%\fi
%
% %%%%%%%%%%%%%%%%%%%%%%%%%%%%%%%%%%%%%%
% \paragraph{Forwarding for a Complete Draft.}
%
% The following forwarding file |cdocsdrf.tex|
% compiles the main document in draft mode:
%\iffalse
%<*sampledraft>
%\fi
%    \begin{macrocode}
\def\version{draft}
\input{childdoc.def}
\childdocforward{cdocsamp}
%    \end{macrocode}

%\iffalse
%</sampledraft>
%\fi
%
% %%%%%%%%%%%%%%%%%%%%%%%%%%%%%%%%%%%%%%
% \paragraph{Forwarding for Final Version of the Chapters.}
%
% The following forwarding files |cdocsfn1.tex| and |cdocsfn2.tex|
% (with identical content)
% compile the final versions of the child documents
% |cdocsch1.tex| and |cdocsch2.tex|, respectively:
%\iffalse
%<*samplefinal>
%\fi
%    \begin{macrocode}
\def\version{final}
\input{childdoc.def}
\childdocforwardprefix[cdocsamp]{cdocsfn}{cdocsch}
%    \end{macrocode}

%\iffalse
%</samplefinal>
%\fi
%
% %%%%%%%%%%%%%%%%%%%%%%%%%%%%%%%%%%%%%%
% \paragraph{Command Line Processing.}
%
% The following three command lines generate the output files
% |cdocscld|, |cdocscl1| and |cdocscl2|
% which should be identical to
% |cdocsdrf|, |cdocsch1| and |cdocsfn2|, respectively:
% \begin{center}
% \begin{tabular}{l}
% |latex -jobname cdocscld \|\\
% |  "\def\version{draft}\input{childdoc.def}\childdocforward{cdocsamp}"|\\
% |latex -jobname cdocscl1 \|\\
% |  "\input{childdoc.def}\childdocforward[cdocsamp]{cdocsch1}"|\\
% |latex -jobname cdocscl2 \|\\
% |  "\def\version{final}\input{childdoc.def}\childdocforward{cdocsch2}"|
% \end{tabular}
% \end{center}
% Note that the trailing backslash on each first line
% merely continues the input to the second line
% (for convenient cut ant paste).
% Furthermore, the command |latex| can be replaced by any
% of its alternative versions such as |pdflatex|.
%
% %%%%%%%%%%%%%%%%%%%%%%%%%%%%%%%%%%%%%%%%%%%%%%%%%%%%%%%%%%%%%%%%%%%%%%%%%%%%%%
% %%%%%%%%%%%%%%%%%%%%%%%%%%%%%%%%%%%%%%%%%%%%%%%%%%%%%%%%%%%%%%%%%%%%%%%%%%%%%%
% \section{Implementation}
%\iffalse
%<*package>
%\fi
%
% This section describes the definitions file |childdoc.def|.

% The definitions cannot be loaded using |\usepackage| or |\RequirePackage|
% which has a mechanism to prevent loading a style file more than once.
% When loading the definitions by means of |\input|
% multiple instances have to be prevented manually:
%\iffalse
%This code needs to be before the `\ProvidesFile' directive
%which is defined at the beginning of this file.
%Therefore it is also placed there and commented out here.
%</package>
%<*discard>
%\fi
%    \begin{macrocode}
\ifdefined\childdocmain\endinput\fi
%    \end{macrocode}
%\iffalse
%</discard>
%<*package>
%\fi
%
% \macro{\ifchilddoc}
% \macro{\ifchilddocmanual}
% The conditional |\ifchilddoc| tells whether a
% child (true) or main (false) document is being compiled.
% The conditional |\ifchilddocmanual| tells whether
% the |\includeonly| mechanism is used (false) or
% the selection of child files must be performed manually (true).
% The definitions initialise to false:
%    \begin{macrocode}
\newif\ifchilddoc
\newif\ifchilddocmanual
%    \end{macrocode}

% \macro{\childdocname}
% \macro{\childdocjob}
% The macro |\childdocname| stores the name of the main document
% to be compiled. The macro |\childdocjob| stores the name of
% the document on which the \LaTeX{} compiler was originally invoked.
% The content of |\jobname| cannot be compared
% to filenames specified in the source due to different catcodes.
% The following code rescans |\jobname|, stores the result
% in |\childdocname| and saves a copy in |\childdocjob|:
%    \begin{macrocode}
\edef\childdocname{\scantokens\expandafter{\jobname\noexpand}}
\let\childdocjob\childdocname
%    \end{macrocode}

% \macro{\childdocdisable}
% The macro |\childdocdisable| prevents the main file
% from being processed more than once.
% At this stage, the main document command |\childdocmain|
% is assumed to be called once again where it should do nothing.
% Any subsequent call to it should prevent
% a secondary processing of the main document
% It overwrites the forwarding commands
% |\childdocof| and |\childdocforward|
% with empty macros to prevent further inclusions of the main document:
%    \begin{macrocode}
\newcommand{\childdocdisable}
{
  \renewcommand{\childdocmain}[1]{\renewcommand{\childdocmain}[1]{\endinput}}
  \renewcommand{\childdocof}[1]{}
  \renewcommand{\childdocby}[2][]{}
  \renewcommand{\childdocforward}[2][]{}
  \renewcommand{\childdocdisable}{}
}
%    \end{macrocode}

% \macro{\childdocmain}
% The macro |\childdocmain| is to be called at the top of the main file
% with nothing or the main filename (without extension) as argument.
% First, it breaks loops.
% If the argument is not empty and does not match |\childdocname|
% (which is set by the first inclusion of |childdoc.def|),
% |\ifchilddoc| is set to true, |\includeonly| is applied to the child file
% and |\jobname| is set to the main file
% (for proper handling of |.aux| files):
%    \begin{macrocode}
\newcommand{\childdocmain}[1]
{
  \childdocdisable\childdocmain{}
  \if?#1?\else
    \begingroup
      \def\childdoctmp{#1}
      \ifx\childdoctmp\childdocname
        \def\childdoctmp{}
      \else
        \def\childdoctmp
        {
          \childdoctrue
          \includeonly{\childdocname}
          \def\childdocjob{#1}
          \def\jobname{#1}
        }
      \fi
      \expandafter
    \endgroup
    \childdoctmp
  \fi
}
%    \end{macrocode}

% \macro{\childdocof}
% The command |\childdocof| redirects
% compilation to the main file |#1|.
%    \begin{macrocode}
\newcommand{\childdocof}[1]
{
  \childdocdisable
  \childdoctrue
  \includeonly{\childdocname}
  \def\jobname{#1}
  \def\childdocjob{#1}
  \input{#1}
}
%    \end{macrocode}

% \macro{\childdocby}
% The command |\childdocby| ....
%    \begin{macrocode}
\newcommand{\childdocby}[2][]
{
  \childdocdisable
  \childdoctrue
  \childdocmanualtrue
  \if?#1?\else
    \def\jobname{#2}
  \fi
  \def\childdocjob{#2}
  \input{#2}
  \endinput
}
%    \end{macrocode}

% \macro{\childdocforward}
% The command |\childdocforward| redirects
% compilation to the main file or
% (if the optional argument is given) a child file.
% Parameters are set as if the main file
% or a child file starting with |\childdocof| was compiled.
% Then compilation is handed over to the main file:
%    \begin{macrocode}
\newcommand{\childdocforward}[2][]
{
  \begingroup
    \if?#1?
      \def\childdoctmp
      {
        \def\childdocname{#2}
        \def\childdocjob{#2}
        \def\jobname{#2}
        \input{#2}
        \endinput
      }
    \else
      \def\childdoctmp
      {
        \childdocdisable
        \def\childdocname{#2}
        \childdoctrue
        \includeonly{#2}
        \def\childdocjob{#1}
        \def\jobname{#1}
        \input{#1}
        \endinput
      }
    \fi
    \expandafter
  \endgroup
  \childdoctmp
}
%    \end{macrocode}

% \macro{\childdocforwardprefix}
% The command |\childdocforwardprefix| redirects
% compilation to the main or a child file by means of a pattern.
% The prefix |#1| in the current filename is replaced by |#2|
% and the suffix of the current filename is kept
% (it is assumed that the filename does not contain the substring `|~~~|'
% which is used as a delimiter).
% Compilation is handed over to the new file by |\childdocforward|:
%    \begin{macrocode}
\newcommand{\childdocforwardprefix}[3][]
{
  \begingroup
    \def\childdocextract #2##1~~~{\def\childdoctmp{\childdocforward[#1]{#3##1}}}
    \expandafter\childdocextract\childdocname~~~
    \expandafter
  \endgroup
  \childdoctmp
}
%    \end{macrocode}

% \macro{\childdoc}
% The deprecated macro |\childdoc| is a legacy version of |\childdocmain|:
%    \begin{macrocode}
\newcommand{\childdoc}{\childdocmain}
%    \end{macrocode}

% \macro{\childdocredirect}
% The deprecated macro |\childdocredirect| is a legacy version
% of |\childdocforward| and |\childdocforwardprefix|:
%    \begin{macrocode}
\newcommand{\childdocredirect}[2][]
{
  \begingroup
    \if?#1?
      \def\childdoctmp{\childdocforward{#2}}
    \else
      \def\childdoctmp{\childdocforwardprefix{#1}{#2}}
    \fi
    \expandafter
  \endgroup
  \childdoctmp
}
%    \end{macrocode}

%\iffalse
%</package>
%\fi
%
\endinput

\childdocforward{cdocsamp}
%    \end{macrocode}

%\iffalse
%</sampledraft>
%\fi
%
% %%%%%%%%%%%%%%%%%%%%%%%%%%%%%%%%%%%%%%
% \paragraph{Forwarding for Final Version of the Chapters.}
%
% The following forwarding files |cdocsfn1.tex| and |cdocsfn2.tex|
% (with identical content)
% compile the final versions of the child documents
% |cdocsch1.tex| and |cdocsch2.tex|, respectively:
%\iffalse
%<*samplefinal>
%\fi
%    \begin{macrocode}
\def\version{final}
% \iffalse
%
% childdoc.dtx Copyright (C) 2017-2018 Niklas Beisert
%
% This work may be distributed and/or modified under the
% conditions of the LaTeX Project Public License, either version 1.3
% of this license or (at your option) any later version.
% The latest version of this license is in
%   http://www.latex-project.org/lppl.txt
% and version 1.3 or later is part of all distributions of LaTeX
% version 2005/12/01 or later.
%
% This work has the LPPL maintenance status `maintained'.
%
% The Current Maintainer of this work is Niklas Beisert.
%
% This work consists of the files childdoc.dtx and childdoc.ins
% and the derived files childdoc.def and cdocsamp.tex with
% cdocsch1.tex, cdocsch2.tex, cdocsdrf.tex, cdocsfn1.tex, cdocsfn2.tex.
%
%<package>\ifdefined\childdocmain\endinput\fi
%<package>\ProvidesFile{childdoc.def}[2018/12/30 v2.0 child document driver]
%<samplemain>\ProvidesFile{cdocsamp.tex}[2018/12/30 v2.0 sample for childdoc]
%<*driver>
%\ProvidesFile{childdoc.drv}[2018/12/30 v2.0 childdoc reference manual file]
\PassOptionsToClass{10pt,a4paper}{article}
\documentclass{ltxdoc}

\usepackage[margin=35mm]{geometry}
\usepackage{hyperref}
\usepackage{hyperxmp}
\usepackage[usenames]{color}

\hypersetup{colorlinks=true}
\hypersetup{pdfstartview=FitH}
\hypersetup{pdfpagemode=UseNone}
\hypersetup{pdfsource={}}
\hypersetup{pdflang={en-UK}}
\hypersetup{pdfcopyright={Copyright 2017-2018 Niklas Beisert.
  This work may be distributed and/or modified under the
  conditions of the LaTeX Project Public License, either version 1.3
  of this license or (at your option) any later version.}}
\hypersetup{pdflicenseurl={http://www.latex-project.org/lppl.txt}}
\hypersetup{pdfcontactaddress={ETH Zurich, ITP, HIT K,
  Wolfgang-Pauli-Strasse 27}}
\hypersetup{pdfcontactpostcode={8093}}
\hypersetup{pdfcontactcity={Zurich}}
\hypersetup{pdfcontactcountry={Switzerland}}
\hypersetup{pdfcontactemail={nbeisert@itp.phys.ethz.ch}}
\hypersetup{pdfcontacturl={http://people.phys.ethz.ch/\xmptilde nbeisert/}}

\newcommand{\secref}[1]{\hyperref[#1]{section \ref*{#1}}}

\parskip1ex
\parindent0pt
\let\olditemize\itemize
\def\itemize{\olditemize\parskip0pt}

\begin{document}

\title{The \textsf{childdoc} Package}
\hypersetup{pdftitle={The childdoc Package}}
\author{Niklas Beisert\\[2ex]
  Institut f\"ur Theoretische Physik\\
  Eidgen\"ossische Technische Hochschule Z\"urich\\
  Wolfgang-Pauli-Strasse 27, 8093 Z\"urich, Switzerland\\[1ex]
  \href{mailto:nbeisert@itp.phys.ethz.ch}
  {\texttt{nbeisert@itp.phys.ethz.ch}}}
\hypersetup{pdfauthor={Niklas Beisert}}
\hypersetup{pdfsubject={Manual for the LaTeX2e Package childdoc}}
\date{30 December 2018, \textsf{v2.0}}
\maketitle

\begin{abstract}\noindent
\textsf{childdoc} is a \LaTeXe{} package
that enables the direct compilation
of document sections included by |\include|
to individual files.
\end{abstract}

\begingroup
\parskip0ex
\tableofcontents
\endgroup

%%%%%%%%%%%%%%%%%%%%%%%%%%%%%%%%%%%%%%%%%%%%%%%%%%%%%%%%%%%%%%%%%%%%%%%%%%%%%%%%
%%%%%%%%%%%%%%%%%%%%%%%%%%%%%%%%%%%%%%%%%%%%%%%%%%%%%%%%%%%%%%%%%%%%%%%%%%%%%%%%
\section{Introduction}

\LaTeX{} provides a mechanism to structure a large document (such as a book)
into a main file and several child files (containing the chapters)
using the |\include| command.
This mechanism is beneficial for documents
which span hundreds of pages in order to
make the source file(s) more manageable.
Moreover, compilation can be restricted to
selected child files by means of the |\includeonly| command.
The latter feature can be used to reduce the compilation time while editing
(this was significantly more useful in the earlier days of \LaTeX{})
or to generate a smaller document which is easier to navigate.
Another application of |\includeonly| is to generate
documents consisting of selected parts of the complete document.

However, there are a few drawbacks of the plain |\include| mechanism:
\begin{itemize}
\item
The child files cannot be compiled on their own,
they can only be compiled via the main file.
A naive editing environment
(such as a text editor with an option
to have the current file processed by \LaTeX)
may require one to switch to the main file before compiling;
attempting to compile the child file produces errors.
\item
The main file must be modified (each time)
to adjust the |\includeonly| command
to the present needs. This easily leaves the main file in a messy state.
\item
The generated document will always carry the filename
of the main document. This is inconvenient if
several child files are to be compiled and
to be kept for distribution.
\end{itemize}

The present package provides a simple interface
to make child files individually compilable by \LaTeX{}.
Compiling a child file then has the same effect as compiling
the main file with an |\includeonly| command
to select the appropriate child.
Moreover the generated document will carry the name of the child
rather than the main file.
This resolves all three above issues.

This feature is meant to make the editing of books,
thesis documents and lecture notes somewhat more convenient.
However, the package can also be used efficiently for
composing a series of documents (such as exercise sheets)
which are typically distributed individually.
It then assists the author in generating the individual documents
(potentially in different versions)
as well as a document containing the collected series.
Another application is in developing style files
or other kinds of included material
where compilation of the style file could redirect
to a sample or test file.

%%%%%%%%%%%%%%%%%%%%%%%%%%%%%%%%%%%%%%%%%%%%%%%%%%%%%%%%%%%%%%%%%%%%%%%%%%%%%%%%
%%%%%%%%%%%%%%%%%%%%%%%%%%%%%%%%%%%%%%%%%%%%%%%%%%%%%%%%%%%%%%%%%%%%%%%%%%%%%%%%
\section{Usage}

First of all, the package \textsf{childdoc} is \emph{not} a standard
\LaTeXe{} |.sty| style file! Therefore it needs to be invoked in
a non-standard way.

%%%%%%%%%%%%%%%%%%%%%%%%%%%%%%%%%%%%%%%%%%%%%%%%%%%%%%%%%%%%%%%%%%%%%%%%%%%%%%%%
\subsection{Included Files}
\label{sec:include}

%%%%%%%%%%%%%%%%%%%%%%%%%%%%%%%%%%%%%%%%
\DescribeMacro{\childdocmain}
To use the package, add the commands
\begin{center}
\begin{tabular}{l}
|\input{childdoc.def}|\\
|\childdocmain{}|\\
\end{tabular}
\end{center}
at the very top of the main \LaTeX{} file,
in particular \emph{before} the |\documentclass| statement!
The argument of |\childdocmain| should be left empty
(but it must be present).

%%%%%%%%%%%%%%%%%%%%%%%%%%%%%%%%%%%%%%%%
\DescribeMacro{\childdocof}
Furthermore, add the commands
\begin{center}
\begin{tabular}{l}
|\input{childdoc.def}|\\
|\childdocof{|\textit{main}|}|\\
\end{tabular}
\end{center}
at the top of every child file \textit{child}
which is included by |\include{|\textit{child}|}|
from within the main file
(or at least for those files to be compiled individually).
The argument \textit{main} must be the filename of the main file.

There are a couple of
considerations in setting up the main and child documents:

%%%%%%%%%%%%%%%%%%%%%%%%%%%%%%%%%%%%%%%%
\paragraph{Restrictions.}

Please note the following restrictions:
\begin{itemize}
\item
|\childdocmain| must be called with one argument \textit{main}
to ensure compatibility with earlier version of the package.
It must either be empty (|\childdocmain{}|)
or precisely match the filename of the main file in which it is specified.
See \secref{sec:detection} for further information.
\item
The filename \textit{main} must be specified without the |.tex| extension.
\item
The filename \textit{main} is case sensitive
(even in case-insensitive file systems)
due to internal string comparison.
\item
The argument \textit{main} should be fully expanded, it cannot be a macro.
\item
Subdirectories and special characters should be avoided in filenames.
\item
The command |\childdocmain{|\textit{main}|}| must be followed by a whitespace.
It should not be followed immediately by another command
or by a comment mark `|%|'.
This is because the \TeX{} parser reads the token immediately following
the argument of |\childdocmain| and puts it
at the beginning of every child section;
however, a white\-space is ignored.
\end{itemize}

%%%%%%%%%%%%%%%%%%%%%%%%%%%%%%%%%%%%%%%%
\paragraph{Content of Main File.}

It is advisable to place all content in the child files included by |\include|.
Any output contained in the main file will appear in all child documents
unless suppressed manually;
it cannot be suppressed automatically by the |\includeonly| directive
and thus should normally be avoided.
A method to include some content in the main file
by means of conditional processing is described in \secref{sec:conditional}.

%%%%%%%%%%%%%%%%%%%%%%%%%%%%%%%%%%%%%%%%
\paragraph{Page Numbering.}

When only a part of the document is compiled,
the appropriate numbering of pages
(as well as other status parameters)
is determined from the |.aux| files.
The latter contain information from previous passes.
However this information needs to propagate through
all intermediate child documents.
Therefore the page numbering in child documents may well
be inconsistent until the complete document is compiled at least once.

A useful (if unconventional) way to always ensure a consistent
page numbering is to restart the numbering in each child document
and denote the pages by `\textit{child}|.|\textit{page}'
where \textit{child} represents the chapter/section number of the child file.
This can be achieved by the command
|\numberwithin{page}{|\textit{child}|}|
of the \textsf{amsmath} package
where \textit{child} can be |chapter| or |section|
depending on the chosen structuring.
Alternatively, one can modify the macro |\thepage| appropriately
and reset the counter |page| at the start of each child file.

%%%%%%%%%%%%%%%%%%%%%%%%%%%%%%%%%%%%%%%%%%%%%%%%%%%%%%%%%%%%%%%%%%%%%%%%%%%%%%%%
\subsection{Conditional Processing}
\label{sec:conditional}

The package provides a mechanism to compile different versions
of a document. To customise the versions further some conditional processing
can come in handy to distinguish which version is being compiled.
The package provides two macros to describe the compilation context:

%%%%%%%%%%%%%%%%%%%%%%%%%%%%%%%%%%%%%%%%
\DescribeMacro{\ifchilddoc}
The conditional |\ifchilddoc| distinguishes between the compilation of
child documents and the main document:
%
\begin{center}
|\ifchilddoc |\textit{child-code}| |[|\||else |\textit{main-code}]| \||fi|
\end{center}

%%%%%%%%%%%%%%%%%%%%%%%%%%%%%%%%%%%%%%%%
\DescribeMacro{\childdocname}
\DescribeMacro{\childdocjob}
The macro |\childdocname| contains the filename (without extension)
of the main or child file being processed.
Note that |\childdocjob| will always contain the name of the main file.

%%%%%%%%%%%%%%%%%%%%%%%%%%%%%%%%%%%%%%%%
\paragraph{Title Page.}

Conditional processing can be used to include a title or banner page
in the main document when proper precautions are taken.
Importantly, the code in the main file should ensure that the page counter
(as well as other status parameters which are stored in the |.aux| files)
takes the same value after the conditional processing.
Otherwise the page numbers may take divergent values
depending on which part is compiled.

For example, a title page could be declared by:
%
\begin{center}
\begin{tabular}{l}
|\ifchilddoc\||else|\\
|\addtocounter{page}{-1}|\\
\textit{code for title page}\\
|\newpage|\\
|\||fi|
\end{tabular}
\end{center}
%
A banner page for the child documents can be generated by:
%
\begin{center}
\begin{tabular}{l}
|\ifchilddoc|\\
|\addtocounter{page}{-1}|\\
\textit{code for banner page}\\
|\newpage|\\
|\||fi|
\end{tabular}
\end{center}
%
Here one could write a message such as:
\begin{center}
|This is the part \childdocname{} of \childdocjob{}.|
\end{center}

%%%%%%%%%%%%%%%%%%%%%%%%%%%%%%%%%%%%%%%%%%%%%%%%%%%%%%%%%%%%%%%%%%%%%%%%%%%%%%%%
\subsection{Flags}
\label{sec:flags}

The package makes it easy to generate different versions
of the main or child documents.
To this end compilation flags can be defined
and assigned different default values.
They will be particularly useful in conjunction
with the forwarding mechanism described in \secref{sec:forward}.

For example, it may be useful to have a flag |\version|
which can be set to |draft| or |final|.
The document source will contain some conditional code
depending on the value of |\version|.
Suppose further, the flag should default to |final| for the main file
and to |draft| for child files
which is a natural assignment for editing the document.
This is achieved by placing the following code
in the preamble of the main document
(below the |\childdocmain| directive):
%
\begin{center}
\begin{tabular}{l}
|\ifchilddoc|\\
|\providecommand{\version}{draft}|\\
|\||else|\\
|\providecommand{\version}{final}|\\
|\||fi|
\end{tabular}
\end{center}
%
The definition by |\providecommand| makes sure
that previous definitions are not overwritten.
Further statements |\providecommand{\version}{...}|
can thus be added before the above code to override it.

For the main file, one might add a line
(between |\childdocmain| and the above block)
%
\begin{center}
|%\ifchilddoc\||else\providecommand{\version}{draft}\||fi|
\end{center}
%
which can be uncommented to produce a draft version.
Likewise one can add a line to the very top of a child file
(above the |\childdocof{|\textit{main}|}| directive)
%
\begin{center}
|%\providecommand{\version}{final}|
\end{center}
%
which can be uncommented to produce the final version of this child document.

%%%%%%%%%%%%%%%%%%%%%%%%%%%%%%%%%%%%%%%%%%%%%%%%%%%%%%%%%%%%%%%%%%%%%%%%%%%%%%%%
\subsection{Forwarding}
\label{sec:forward}

Different versions of the main or child documents
using compilation flags as described in \secref{sec:flags}
can be (permanently) stored in different files
for convenient compilation, viewing and distribution.
To this end, the package defines a command
to pass on compilation to a different file:

%%%%%%%%%%%%%%%%%%%%%%%%%%%%%%%%%%%%%%%%
\DescribeMacro{\childdocforward}
The command |\childdocforward| redirects processing to
another source file:
%
\begin{center}
\begin{tabular}{l}
|\input{childdoc.def}|\\
|\childdocforward[|\textit{main}|]{|\textit{dest}|}|\\
\end{tabular}
\end{center}
%
The argument \textit{dest} is the destination file
(without extension).
It should be the main file or one of the child files.
Note that further \textsf{childdoc} directives
such as |\childdocof| and |\childdocforward|
in the indicated file will be processed in this form.
The optional argument \textit{main}
passes on directly to the main file \textit{main}
while pretending to compile the child \textit{dest}.
This form behaves as if \textit{dest}
issues |\childdocof{|\textit{main}|}| right away,
and no further \textsf{childdoc} directives will be processed.

%%%%%%%%%%%%%%%%%%%%%%%%%%%%%%%%%%%%%%%%
\DescribeMacro{\...prefix}
In the alternative form |\childdocforwardprefix|,
%
\begin{center}
\begin{tabular}{l}
|\input{childdoc.def}|\\
|\childdocforwardprefix[|\textit{main}|]{|\textit{prefix}|}{|\textit{dest}|}|
\end{tabular}
\end{center}
%
the destination file is determined by a pattern
depending on the current file:
To make this work, the current file must be called
`{\textit{prefix}\hspace{0.2em}\textit{suffix}}'
with \textit{prefix} matching precisely the argument.
Processing is then passed on to the file
`{\textit{dest}\hspace{0.2em}\textit{suffix}}'.
Surely, the same effect is achieved by
directly specifying the
argument `{\textit{dest}\hspace{0.2em}\textit{suffix}}'
in the first form.
However, that requires to set up a different file
for each child. With the alternative form of the command
all these files can have exactly the same content
which simplifies setting them up and maintaining them.

For example, the following file |draft.tex|
with a compilation flag |\version| as described in \secref{sec:flags}
compiles the main document as a draft:
%
\begin{center}
\begin{tabular}{l}
|\def\version{draft}|\\
|\input{childdoc.def}|\\
|\childdocforward{|\textit{main}|}|
\end{tabular}
\end{center}
%
Likewise, the following files |final|\textit{nn}|.tex|
compile the final version of the child document
|child|\textit{nn}|.tex|:
%
\begin{center}
\begin{tabular}{l}
|\def\version{final}|\\
|\input{childdoc.def}|\\
|\childdocforwardprefix{final}{child}|
\end{tabular}
\end{center}
%

Note that when several versions of a main file and/or of each child file
are to be generated, it may be convenient to set up a |Makefile| or
shell script to automatise the process.

%%%%%%%%%%%%%%%%%%%%%%%%%%%%%%%%%%%%%%%%%%%%%%%%%%%%%%%%%%%%%%%%%%%%%%%%%%%%%%%%
\subsection{Command Line Processing}
\label{sec:commandline}

The effect of redirection files can also be achieved by invoking
the \LaTeX{} compiler with a more elaborate command line.
Most conveniently this should be done as part
of a shell script or a |Makefile|.

When using \textsf{childdoc} in the main file, the following
command lines effectively perform a redirection
(note that depending on the shell being used,
backslashes may have to be doubled: `|\|' $\to$ `|\\|'):
%
\begin{center}
|... -jobname "|\textit{target}|" |\\|"|[\textit{flags}]%
|\input{childdoc.def}\childdocforward[|\textit{main}|]{|\textit{dest}|}"|
\end{center}
%
Here \textit{target} is the name of the output file,
\textit{main} is the name of the main file
and \textit{dest} is the name of the main or child file to be processed
(all filenames without extensions).
The optional argument \textit{main} can be omitted
if \textit{main} matches \textit{dest}.
Optionally, compilation \textit{flags} can be defined via |\def| commands.
This command line makes the \TeX{} engine believe
it is compiling the file \textit{target}
whose content is specified as the latter parameter.
The provided code then forwards the processing to
\textit{main} or \textit{dest} as described in \secref{sec:forward}.

%%%%%%%%%%%%%%%%%%%%%%%%%%%%%%%%%%%%%%%%%%%%%%%%%%%%%%%%%%%%%%%%%%%%%%%%%%%%%%%%
\subsection{Include by Input}
\label{sec:input}

Including child documents by |\include| has some restrictions by design.
Most notably, the content of a child document always occupies
its own set of pages; pages cannot be shared between child documents.
Usually, this behaviour makes perfect sense
because each child document contain an essential part of the document.
However, in some situations it may be desirable to compose
a document from a collection of parts
without having mandatory page breaks between then.
For this case, the package
provides a mechanism to include parts
by |\input| which can also be processed individually.
However, by construction this mechanism
requires manual handling of the content to be output.

%%%%%%%%%%%%%%%%%%%%%%%%%%%%%%%%%%%%%%%%
\DescribeMacro{\ifchilddocmanual}
The main file should be prepared as usual, see \secref{sec:include}.
However, the document body must make a distinction
between processing of an individual part and of the main document, e.g.:
%
\begin{center}
\begin{tabular}{l}
|\ifchilddocmanual|\\
|\input{\childdocname}|\\
|\||else|\\
\textit{document body with }|\input{|\textit{part}|}|\\
|\||fi|
\end{tabular}
\end{center}
%
The conditional |\ifchilddocmanual| is true whenever
a part to be included by |\input| is being compiled,
and the name of the part is stored in |\childdocname|.

%%%%%%%%%%%%%%%%%%%%%%%%%%%%%%%%%%%%%%%%
\DescribeMacro{\childdocby}
Each part to be included by |\input| should start with:
%
\begin{center}
\begin{tabular}{l}
|\input{childdoc.def}|\\
|\childdocby{|\textit{main}|}|\\
\end{tabular}
\end{center}
%
The directive |\childdocby| is similar to |\childdocof|
described in \secref{sec:include},
but the subsequent selection of content must be done manually.
To that end, both |\ifchilddoc| and |\ifchilddocmanual|
will be true upon processing of a part,
and the name of the part is stored in |\childdocname|.
Note that |\jobname| will be set to the filename of the current part
so that each part receives an individual |.aux| file
that does not interfere with the |.aux| file(s) of the main document.
This behaviour can be altered by the alternative form
|\childdocby[*]{|\textit{main}|}| (with a non-empty optional argument)
which uses the |.aux| file of the main document
by setting |\jobname| to \textit{main}.

%%%%%%%%%%%%%%%%%%%%%%%%%%%%%%%%%%%%%%%%%%%%%%%%%%%%%%%%%%%%%%%%%%%%%%%%%%%%%%%%
\subsection{Driver Development}
\label{sec:driver}

The \textsf{childdoc} mechanism can also be use for the development
of definition files such as \LaTeX{} styles or classes.
This case differs from the above setup with multiple parts
included by |\include| in that no |\includeonly| should be invoked.
This can be achieved by starting the include file
(before |\ProvidesPackage|) with:
%
\begin{center}
\begin{tabular}{l}
|\input{childdoc.def}|\\
|\childdocforward{|\textit{main}|}|\\
\end{tabular}
\end{center}
%
or alternatively with:
%
\begin{center}
\begin{tabular}{l}
|\input{childdoc.def}|\\
|\childdocby{|\textit{main}|}|\\
\end{tabular}
\end{center}
%
Both forms have slightly different effects as described above.
The main file is prepared as usual, see \secref{sec:include}.

%%%%%%%%%%%%%%%%%%%%%%%%%%%%%%%%%%%%%%%%%%%%%%%%%%%%%%%%%%%%%%%%%%%%%%%%%%%%%%%%
\subsection{Legacy Detection}
\label{sec:detection}

The directive |\childdocmain| in the main file can detect
whether the complete document or merely a child is to be compiled
even without using the directive |\childdocof|.
This method is deprecated because it is less robust
and there is no compelling reason to use it;
it is merely provided for backward compatibility
and it may be removed in future versions.

If the detection mechanism is to be used,
it is mandatory to correctly specify
the filename of the main file as the argument of |\childdocmain|:
%
\begin{center}
\begin{tabular}{l}
|\input{childdoc.def}|\\
|\childdocmain{|\textit{main}|}|\\
\end{tabular}
\end{center}
%
If |\jobname| does not match the argument \textit{main} of |\childdocmain|,
it is assumed that |\jobname| points to the child file to be compiled.
When using |\childdocmain| with the main file specified as argument,
it suffices to start a child file
with just |\input{|\textit{main}|}|
without loading of the package and using |\childdocof|.
If instead all processing is done
with the appropriate \textsf{childdoc} directives,
the argument of \textit{main} of |\childdocmain| can be empty.

An alternative version of the command line processing described
in \secref{sec:commandline} using the detection mechanism reads:
%
\begin{center}
|... -jobname "|\textit{target}|" "|[\textit{flags}]%
[|\def\jobname{|\textit{dest}|}|]|\input{|\textit{main}|}"|
\end{center}

%%%%%%%%%%%%%%%%%%%%%%%%%%%%%%%%%%%%%%%%%%%%%%%%%%%%%%%%%%%%%%%%%%%%%%%%%%%%%%%%
\subsection{Manual Code}
\label{sec:manual}

In case one cannot be certain whether the definitions file |childdoc.def|
is installed on the target \TeX{} distribution
and one prefers not to ship it,
it is conceivable to paste a few relevant commands into the sources.

To that end, drop all statements |\input{childdoc.def}|
and perform the replacements as outlined below.
Instead of |\childdocmain{|\textit{main}|}| add the following code
to the top of the main file:
%
\begin{center}
\begin{tabular}{l}
|\||ifdefined\childdocname\endinput\||fi\newif\ifchilddoc|\\
|\edef\childdocname{\scantokens\expandafter{\jobname\noexpand}}|\\
|\def\childdocmain{|\textit{main}|}\||ifx\childdocmain\childdocname\||else|\\
|\childdoctrue\includeonly{\childdocname}\let\jobname\childdocmain\||fi|\\
\end{tabular}
\end{center}
%
Instead of |\childdocof{|\textit{main}|}| just include the main file
at the top of each child file:
%
\begin{center}
|\input{|\textit{main}|}|
\end{center}
%
A simple redirection |\childdocforward{|\textit{dest}|}| is achieved by:
%
\begin{center}
|\def\jobname{|\textit{dest}|}\input{\jobname}|
\end{center}
%
The redirection with prefix
|\childdocforwardprefix[|\textit{prefix}|]{|\textit{dest}|}|
is accomplished by:
%
\begin{center}
\begin{tabular}{l}
|{\edef\jobname{\scantokens\expandafter{\jobname\noexpand}}|\\
|\def\redirectjob |\textit{prefix}|#1~~~{\gdef\jobname{|\textit{dest}|#1}}|\\
|\expandafter\redirectjob\jobname~~~}\input{\jobname}|
\end{tabular}
\end{center}

In an alternative approach,
child documents can be compiled by a specific command line
without additional code or specific definitions:
%
\begin{center}
|... -jobname "|\textit{target}|" "|[\textit{flags}]%
|\includeonly{|\textit{dest}|}\input{|\textit{main}|}"|
\end{center}
%

%%%%%%%%%%%%%%%%%%%%%%%%%%%%%%%%%%%%%%%%%%%%%%%%%%%%%%%%%%%%%%%%%%%%%%%%%%%%%%%%
%%%%%%%%%%%%%%%%%%%%%%%%%%%%%%%%%%%%%%%%%%%%%%%%%%%%%%%%%%%%%%%%%%%%%%%%%%%%%%%%
\section{Information}

%%%%%%%%%%%%%%%%%%%%%%%%%%%%%%%%%%%%%%%%%%%%%%%%%%%%%%%%%%%%%%%%%%%%%%%%%%%%%%%%
\subsection{Copyright}

Copyright \copyright{} 2017--2018 Niklas Beisert

This work may be distributed and/or modified under the
conditions of the \LaTeX{} Project Public License, either version 1.3
of this license or (at your option) any later version.
The latest version of this license is in
  \url{http://www.latex-project.org/lppl.txt}
and version 1.3 or later is part of all distributions of \LaTeX{}
version 2005/12/01 or later.

This work has the LPPL maintenance status `maintained'.

The Current Maintainer of this work is Niklas Beisert.

This work consists of the files |README.txt|, |childdoc.ins| and |childdoc.dtx|
as well as the derived files |childdoc.def|, |cdocsamp.tex|
with |cdocsch1.tex|, |cdocsch2.tex|, |cdocspt3.tex|, |cdocspt4.tex|,
|cdocsdrf.tex|, |cdocsfn1.tex|, |cdocsfn2.tex|
as well as |childdoc.pdf|.

%%%%%%%%%%%%%%%%%%%%%%%%%%%%%%%%%%%%%%%%%%%%%%%%%%%%%%%%%%%%%%%%%%%%%%%%%%%%%%%%
\subsection{Files and Installation}

The package consists of the files:
%
\begin{center}
\begin{tabular}{ll}
    |README.txt|   & readme file \\
    |childdoc.ins| & installation file \\
    |childdoc.dtx| & source file \\
    |childdoc.def| & definition file \\
    |cdocsamp.tex| & sample main file \\
    |cdocsch1.tex| & sample include file \\
    |cdocsch2.tex| & sample include file \\
    |cdocspt3.tex| & sample part file \\
    |cdocspt4.tex| & sample part file \\
    |cdocsdrf.tex| & sample redirection file \\
    |cdocsfn1.tex| & sample redirection file \\
    |cdocsfn2.tex| & sample redirection file \\
    |childdoc.pdf| & manual
\end{tabular}
\end{center}
%
The distribution consists of the files
|README.txt|, |childdoc.ins| and |childdoc.dtx|.
%
\begin{itemize}
\item
Run (pdf)\LaTeX{} on |childdoc.dtx|
to compile the manual |childdoc.pdf| (this file).
\item
Run \LaTeX{} on |childdoc.ins| to create the definitions file |childdoc.def|
and the sample |cdocsamp.tex| with include files
|cdocsch1.tex|, |cdocsch2.tex|, |cdocspt3.tex|, |cdocspt4.tex|,
|cdocsdrf.tex|, |cdocsfn1.tex|, |cdocsfn2.tex|.
Then copy the file |childdoc.def| to an appropriate directory of your \LaTeX{}
distribution, e.g.\ \textit{texmf-root}|/tex/latex/childdoc|.
\end{itemize}

%%%%%%%%%%%%%%%%%%%%%%%%%%%%%%%%%%%%%%%%%%%%%%%%%%%%%%%%%%%%%%%%%%%%%%%%%%%%%%%%
\subsection{Related CTAN Packages}

There are several other packages which offer a similar functionality:
%
\begin{itemize}
\item
The packages
\href{http://ctan.org/pkg/docmute}{\textsf{docmute}},
\href{http://ctan.org/pkg/includex}{\textsf{includex}} and
\href{http://ctan.org/pkg/standalone}{\textsf{standalone}}
provide commands to include only the document body of
a child file thus allowing both files to be compiled individually.
\item
The packages \href{http://ctan.org/pkg/subdocs}{\textsf{subdocs}}
and \href{http://ctan.org/pkg/subfiles}{\textsf{subfiles}}
provide structures in which the main and child documents can be
encapsulated and allowing them to be compiled individually.
The inclusion mechanism is different from the conventional |\include|.
\item
The package \href{http://ctan.org/pkg/combine}{\textsf{combine}}
is an elaborate solution to combine several documents into one.
\end{itemize}
%
See also the CTAN topic \href{http://ctan.org/topic/subdocs}{\textsf{subdocs}}
for further related packages.
The present package differs from the above solutions in that
a document structure constructed with the conventional |\include| mechanism
just needs two extra commands at the top of every file
such that all constituent files can be compiled individually.

%%%%%%%%%%%%%%%%%%%%%%%%%%%%%%%%%%%%%%%%%%%%%%%%%%%%%%%%%%%%%%%%%%%%%%%%%%%%%%%%
%\subsection{Feature Suggestions}
%
%The following is a list of features which may be useful for future
%versions of this package:
%%
%\begin{itemize}
%\item
%\ldots
%\end{itemize}

%%%%%%%%%%%%%%%%%%%%%%%%%%%%%%%%%%%%%%%%%%%%%%%%%%%%%%%%%%%%%%%%%%%%%%%%%%%%%%%%
\subsection{Revision History}

%%%%%%%%%%%%%%%%%%%%%%%%%%%%%%%%%%%%%%%%
\paragraph{v2.0:} 2018/12/30

\begin{itemize}
\item
immediate forward processing
\item
added |\childdocby| mechanism
\item
manual restructured
\end{itemize}

%%%%%%%%%%%%%%%%%%%%%%%%%%%%%%%%%%%%%%%%
\paragraph{v1.6:} 2018/01/17

\begin{itemize}
\item
application for development of include files
\item
corrections to manual
\end{itemize}

%%%%%%%%%%%%%%%%%%%%%%%%%%%%%%%%%%%%%%%%
\paragraph{v1.5:} 2017/05/21

\begin{itemize}
\item
more complete structuring introduced
\item
|\childdocof| introduced
\item
|\childdoc| renamed to |\childdocmain|
\item
|\childredirect| renamed to |\childdocforward| and |\childdocforwardprefix|
and functionality expanded
\end{itemize}

%%%%%%%%%%%%%%%%%%%%%%%%%%%%%%%%%%%%%%%%
\paragraph{v1.0:} 2017/04/27

\begin{itemize}
\item
manual and install package
\item
first version published on CTAN
\end{itemize}

%%%%%%%%%%%%%%%%%%%%%%%%%%%%%%%%%%%%%%%%
\paragraph{v0.6:} 2017/04/26

\begin{itemize}
\item
redirection mechanism added
\end{itemize}

%%%%%%%%%%%%%%%%%%%%%%%%%%%%%%%%%%%%%%%%
\paragraph{v0.5:} 2017/04/26

\begin{itemize}
\item
functionality in definition file
\end{itemize}


%%%%%%%%%%%%%%%%%%%%%%%%%%%%%%%%%%%%%%%%%%%%%%%%%%%%%%%%%%%%%%%%%%%%%%%%%%%%%%%%
%%%%%%%%%%%%%%%%%%%%%%%%%%%%%%%%%%%%%%%%%%%%%%%%%%%%%%%%%%%%%%%%%%%%%%%%%%%%%%%%
%%%%%%%%%%%%%%%%%%%%%%%%%%%%%%%%%%%%%%%%%%%%%%%%%%%%%%%%%%%%%%%%%%%%%%%%%%%%%%%%
\appendix

\settowidth\MacroIndent{\rmfamily\scriptsize 000\ }

 \DocInput{childdoc.dtx}

\end{document}
%</driver>
% \fi
%
% %%%%%%%%%%%%%%%%%%%%%%%%%%%%%%%%%%%%%%%%%%%%%%%%%%%%%%%%%%%%%%%%%%%%%%%%%%%%%%
% %%%%%%%%%%%%%%%%%%%%%%%%%%%%%%%%%%%%%%%%%%%%%%%%%%%%%%%%%%%%%%%%%%%%%%%%%%%%%%
% \section{Sample}
%\iffalse
%<*samplemain>
%\fi
%
% The following presents a sample document
% with two chapters, two parts, a title page,
% a compile flag as well as three forwarding files to set the flag.
% It consists of eight |.tex| files:
% \begin{center}
% \begin{tabular}{ll}
% |cdocsamp.tex|&main file\\
% |cdocsch1.tex|&include file for chapter 1\\
% |cdocsch2.tex|&include file for chapter 2\\
% |cdocspt3.tex|&include file for part 3\\
% |cdocspt4.tex|&include file for part 4\\
% |cdocsdrf.tex|&forwarding file for main file in draft mode\\
% |cdocsfi1.tex|&forwarding file for final version of chapter 1\\
% |cdocsfi2.tex|&forwarding file for final version of chapter 2\\
% \end{tabular}
% \end{center}
% Each of the eight files can be compiled directly by the \LaTeX{} compiler.
%
% %%%%%%%%%%%%%%%%%%%%%%%%%%%%%%%%%%%%%%
% \paragraph{Main File.}
%
% The main file is called |cdocsamp.tex|.
%
% Load the \textsf{childdoc} definitions and
% declare the filename for the main document:
%    \begin{macrocode}
\input{childdoc.def}
\childdocmain{}
%    \end{macrocode}

% Optional override for |\version| flag:
%    \begin{macrocode}
%%\ifchilddoc\else\providecommand{\version}{draft}\fi
%    \end{macrocode}

% Define the default values for the |\version| flag
% (|final| for the main file and |draft| for childs):
%    \begin{macrocode}
\ifchilddoc
\providecommand{\version}{draft}
\else
\providecommand{\version}{final}
\fi
%    \end{macrocode}

% Load the standard document class:
%    \begin{macrocode}
\documentclass[12pt]{article}
%    \end{macrocode}

% Start the document body:
%    \begin{macrocode}
\begin{document}
%    \end{macrocode}

% Declare a title page.
% Print title, part of document being processed and version flag:
%    \begin{macrocode}
\addtocounter{page}{-1}
\begin{center}
{\LARGE\bfseries{}childdoc example\par}
\vspace{1cm}
\ifchilddoc
\ifchilddocmanual part\else chapter\fi:
`\childdocname' of `\childdocjob'\par
\else
main document: `\childdocjob'\par
\fi
version: \version\par
\end{center}
\newpage
%    \end{macrocode}

% Manually include selected file,
% otherwise process as usual:
%    \begin{macrocode}
\ifchilddocmanual
\section*{part `\childdocname'}
\input{\childdocname}
\else
%    \end{macrocode}

% Include the two chapters:
%    \begin{macrocode}
\include{cdocsch1}
\include{cdocsch2}
%    \end{macrocode}

% Include the two parts unless only chapters should be displayed:
%    \begin{macrocode}
\ifchilddoc\else
\section{part three}
\input{cdocspt3}
\section{part four}
\input{cdocspt4}
\fi
%    \end{macrocode}

% Process as usual until here:
%    \begin{macrocode}
\fi
%    \end{macrocode}

% End of document body:
%    \begin{macrocode}
\end{document}
%    \end{macrocode}
%\iffalse
%</samplemain>
%\fi
%
% %%%%%%%%%%%%%%%%%%%%%%%%%%%%%%%%%%%%%%
% \paragraph{Chapter Include Files.}
%
% The include files are called |cdocsch1.tex| and |cdocsch2.tex|.
%
%\iffalse
%<*samplechap1|samplechap2>
%\fi

% Optional override for |\version| flag:
%    \begin{macrocode}
%%\providecommand{\version}{final}
%    \end{macrocode}

% Include the main document:
%    \begin{macrocode}
\input{childdoc.def}
\childdocof{cdocsamp}
%    \end{macrocode}

%\iffalse
%</samplechap1|samplechap2>
%\fi
%
%\iffalse
%<*samplechap1>
%\fi
% Some text for chapter 1:
%    \begin{macrocode}
\section{one}
some text in chapter one
%    \end{macrocode}

%\iffalse
%</samplechap1>
%\fi
% Some text for chapter 2:
%\iffalse
%<*samplechap2>
%\fi
%    \begin{macrocode}
\section{two}
more text in chapter two
%    \end{macrocode}

%\iffalse
%</samplechap2>
%\fi
%
% %%%%%%%%%%%%%%%%%%%%%%%%%%%%%%%%%%%%%%
% \paragraph{Part Include Files.}
%
% The include files are called |cdocspt3.tex| and |cdocspt4.tex|.
%
%\iffalse
%<*samplepart3|samplepart4>
%\fi

% Optional override for |\version| flag:
%    \begin{macrocode}
%%\providecommand{\version}{final}
%    \end{macrocode}

% Include the main document:
%    \begin{macrocode}
\input{childdoc.def}
\childdocby{cdocsamp}
%    \end{macrocode}

%\iffalse
%</samplepart3|samplepart4>
%\fi
%
%\iffalse
%<*samplepart3>
%\fi
% Some text for part 3:
%    \begin{macrocode}
some text in part three
%    \end{macrocode}

%\iffalse
%</samplepart3>
%\fi
% Some text for part 4:
%\iffalse
%<*samplepart4>
%\fi
%    \begin{macrocode}
more text in part four
%    \end{macrocode}

%\iffalse
%</samplepart4>
%\fi
%
% %%%%%%%%%%%%%%%%%%%%%%%%%%%%%%%%%%%%%%
% \paragraph{Forwarding for a Complete Draft.}
%
% The following forwarding file |cdocsdrf.tex|
% compiles the main document in draft mode:
%\iffalse
%<*sampledraft>
%\fi
%    \begin{macrocode}
\def\version{draft}
\input{childdoc.def}
\childdocforward{cdocsamp}
%    \end{macrocode}

%\iffalse
%</sampledraft>
%\fi
%
% %%%%%%%%%%%%%%%%%%%%%%%%%%%%%%%%%%%%%%
% \paragraph{Forwarding for Final Version of the Chapters.}
%
% The following forwarding files |cdocsfn1.tex| and |cdocsfn2.tex|
% (with identical content)
% compile the final versions of the child documents
% |cdocsch1.tex| and |cdocsch2.tex|, respectively:
%\iffalse
%<*samplefinal>
%\fi
%    \begin{macrocode}
\def\version{final}
\input{childdoc.def}
\childdocforwardprefix[cdocsamp]{cdocsfn}{cdocsch}
%    \end{macrocode}

%\iffalse
%</samplefinal>
%\fi
%
% %%%%%%%%%%%%%%%%%%%%%%%%%%%%%%%%%%%%%%
% \paragraph{Command Line Processing.}
%
% The following three command lines generate the output files
% |cdocscld|, |cdocscl1| and |cdocscl2|
% which should be identical to
% |cdocsdrf|, |cdocsch1| and |cdocsfn2|, respectively:
% \begin{center}
% \begin{tabular}{l}
% |latex -jobname cdocscld \|\\
% |  "\def\version{draft}\input{childdoc.def}\childdocforward{cdocsamp}"|\\
% |latex -jobname cdocscl1 \|\\
% |  "\input{childdoc.def}\childdocforward[cdocsamp]{cdocsch1}"|\\
% |latex -jobname cdocscl2 \|\\
% |  "\def\version{final}\input{childdoc.def}\childdocforward{cdocsch2}"|
% \end{tabular}
% \end{center}
% Note that the trailing backslash on each first line
% merely continues the input to the second line
% (for convenient cut ant paste).
% Furthermore, the command |latex| can be replaced by any
% of its alternative versions such as |pdflatex|.
%
% %%%%%%%%%%%%%%%%%%%%%%%%%%%%%%%%%%%%%%%%%%%%%%%%%%%%%%%%%%%%%%%%%%%%%%%%%%%%%%
% %%%%%%%%%%%%%%%%%%%%%%%%%%%%%%%%%%%%%%%%%%%%%%%%%%%%%%%%%%%%%%%%%%%%%%%%%%%%%%
% \section{Implementation}
%\iffalse
%<*package>
%\fi
%
% This section describes the definitions file |childdoc.def|.

% The definitions cannot be loaded using |\usepackage| or |\RequirePackage|
% which has a mechanism to prevent loading a style file more than once.
% When loading the definitions by means of |\input|
% multiple instances have to be prevented manually:
%\iffalse
%This code needs to be before the `\ProvidesFile' directive
%which is defined at the beginning of this file.
%Therefore it is also placed there and commented out here.
%</package>
%<*discard>
%\fi
%    \begin{macrocode}
\ifdefined\childdocmain\endinput\fi
%    \end{macrocode}
%\iffalse
%</discard>
%<*package>
%\fi
%
% \macro{\ifchilddoc}
% \macro{\ifchilddocmanual}
% The conditional |\ifchilddoc| tells whether a
% child (true) or main (false) document is being compiled.
% The conditional |\ifchilddocmanual| tells whether
% the |\includeonly| mechanism is used (false) or
% the selection of child files must be performed manually (true).
% The definitions initialise to false:
%    \begin{macrocode}
\newif\ifchilddoc
\newif\ifchilddocmanual
%    \end{macrocode}

% \macro{\childdocname}
% \macro{\childdocjob}
% The macro |\childdocname| stores the name of the main document
% to be compiled. The macro |\childdocjob| stores the name of
% the document on which the \LaTeX{} compiler was originally invoked.
% The content of |\jobname| cannot be compared
% to filenames specified in the source due to different catcodes.
% The following code rescans |\jobname|, stores the result
% in |\childdocname| and saves a copy in |\childdocjob|:
%    \begin{macrocode}
\edef\childdocname{\scantokens\expandafter{\jobname\noexpand}}
\let\childdocjob\childdocname
%    \end{macrocode}

% \macro{\childdocdisable}
% The macro |\childdocdisable| prevents the main file
% from being processed more than once.
% At this stage, the main document command |\childdocmain|
% is assumed to be called once again where it should do nothing.
% Any subsequent call to it should prevent
% a secondary processing of the main document
% It overwrites the forwarding commands
% |\childdocof| and |\childdocforward|
% with empty macros to prevent further inclusions of the main document:
%    \begin{macrocode}
\newcommand{\childdocdisable}
{
  \renewcommand{\childdocmain}[1]{\renewcommand{\childdocmain}[1]{\endinput}}
  \renewcommand{\childdocof}[1]{}
  \renewcommand{\childdocby}[2][]{}
  \renewcommand{\childdocforward}[2][]{}
  \renewcommand{\childdocdisable}{}
}
%    \end{macrocode}

% \macro{\childdocmain}
% The macro |\childdocmain| is to be called at the top of the main file
% with nothing or the main filename (without extension) as argument.
% First, it breaks loops.
% If the argument is not empty and does not match |\childdocname|
% (which is set by the first inclusion of |childdoc.def|),
% |\ifchilddoc| is set to true, |\includeonly| is applied to the child file
% and |\jobname| is set to the main file
% (for proper handling of |.aux| files):
%    \begin{macrocode}
\newcommand{\childdocmain}[1]
{
  \childdocdisable\childdocmain{}
  \if?#1?\else
    \begingroup
      \def\childdoctmp{#1}
      \ifx\childdoctmp\childdocname
        \def\childdoctmp{}
      \else
        \def\childdoctmp
        {
          \childdoctrue
          \includeonly{\childdocname}
          \def\childdocjob{#1}
          \def\jobname{#1}
        }
      \fi
      \expandafter
    \endgroup
    \childdoctmp
  \fi
}
%    \end{macrocode}

% \macro{\childdocof}
% The command |\childdocof| redirects
% compilation to the main file |#1|.
%    \begin{macrocode}
\newcommand{\childdocof}[1]
{
  \childdocdisable
  \childdoctrue
  \includeonly{\childdocname}
  \def\jobname{#1}
  \def\childdocjob{#1}
  \input{#1}
}
%    \end{macrocode}

% \macro{\childdocby}
% The command |\childdocby| ....
%    \begin{macrocode}
\newcommand{\childdocby}[2][]
{
  \childdocdisable
  \childdoctrue
  \childdocmanualtrue
  \if?#1?\else
    \def\jobname{#2}
  \fi
  \def\childdocjob{#2}
  \input{#2}
  \endinput
}
%    \end{macrocode}

% \macro{\childdocforward}
% The command |\childdocforward| redirects
% compilation to the main file or
% (if the optional argument is given) a child file.
% Parameters are set as if the main file
% or a child file starting with |\childdocof| was compiled.
% Then compilation is handed over to the main file:
%    \begin{macrocode}
\newcommand{\childdocforward}[2][]
{
  \begingroup
    \if?#1?
      \def\childdoctmp
      {
        \def\childdocname{#2}
        \def\childdocjob{#2}
        \def\jobname{#2}
        \input{#2}
        \endinput
      }
    \else
      \def\childdoctmp
      {
        \childdocdisable
        \def\childdocname{#2}
        \childdoctrue
        \includeonly{#2}
        \def\childdocjob{#1}
        \def\jobname{#1}
        \input{#1}
        \endinput
      }
    \fi
    \expandafter
  \endgroup
  \childdoctmp
}
%    \end{macrocode}

% \macro{\childdocforwardprefix}
% The command |\childdocforwardprefix| redirects
% compilation to the main or a child file by means of a pattern.
% The prefix |#1| in the current filename is replaced by |#2|
% and the suffix of the current filename is kept
% (it is assumed that the filename does not contain the substring `|~~~|'
% which is used as a delimiter).
% Compilation is handed over to the new file by |\childdocforward|:
%    \begin{macrocode}
\newcommand{\childdocforwardprefix}[3][]
{
  \begingroup
    \def\childdocextract #2##1~~~{\def\childdoctmp{\childdocforward[#1]{#3##1}}}
    \expandafter\childdocextract\childdocname~~~
    \expandafter
  \endgroup
  \childdoctmp
}
%    \end{macrocode}

% \macro{\childdoc}
% The deprecated macro |\childdoc| is a legacy version of |\childdocmain|:
%    \begin{macrocode}
\newcommand{\childdoc}{\childdocmain}
%    \end{macrocode}

% \macro{\childdocredirect}
% The deprecated macro |\childdocredirect| is a legacy version
% of |\childdocforward| and |\childdocforwardprefix|:
%    \begin{macrocode}
\newcommand{\childdocredirect}[2][]
{
  \begingroup
    \if?#1?
      \def\childdoctmp{\childdocforward{#2}}
    \else
      \def\childdoctmp{\childdocforwardprefix{#1}{#2}}
    \fi
    \expandafter
  \endgroup
  \childdoctmp
}
%    \end{macrocode}

%\iffalse
%</package>
%\fi
%
\endinput

\childdocforwardprefix[cdocsamp]{cdocsfn}{cdocsch}
%    \end{macrocode}

%\iffalse
%</samplefinal>
%\fi
%
% %%%%%%%%%%%%%%%%%%%%%%%%%%%%%%%%%%%%%%
% \paragraph{Command Line Processing.}
%
% The following three command lines generate the output files
% |cdocscld|, |cdocscl1| and |cdocscl2|
% which should be identical to
% |cdocsdrf|, |cdocsch1| and |cdocsfn2|, respectively:
% \begin{center}
% \begin{tabular}{l}
% |latex -jobname cdocscld \|\\
% |  "\def\version{draft}% \iffalse
%
% childdoc.dtx Copyright (C) 2017-2018 Niklas Beisert
%
% This work may be distributed and/or modified under the
% conditions of the LaTeX Project Public License, either version 1.3
% of this license or (at your option) any later version.
% The latest version of this license is in
%   http://www.latex-project.org/lppl.txt
% and version 1.3 or later is part of all distributions of LaTeX
% version 2005/12/01 or later.
%
% This work has the LPPL maintenance status `maintained'.
%
% The Current Maintainer of this work is Niklas Beisert.
%
% This work consists of the files childdoc.dtx and childdoc.ins
% and the derived files childdoc.def and cdocsamp.tex with
% cdocsch1.tex, cdocsch2.tex, cdocsdrf.tex, cdocsfn1.tex, cdocsfn2.tex.
%
%<package>\ifdefined\childdocmain\endinput\fi
%<package>\ProvidesFile{childdoc.def}[2018/12/30 v2.0 child document driver]
%<samplemain>\ProvidesFile{cdocsamp.tex}[2018/12/30 v2.0 sample for childdoc]
%<*driver>
%\ProvidesFile{childdoc.drv}[2018/12/30 v2.0 childdoc reference manual file]
\PassOptionsToClass{10pt,a4paper}{article}
\documentclass{ltxdoc}

\usepackage[margin=35mm]{geometry}
\usepackage{hyperref}
\usepackage{hyperxmp}
\usepackage[usenames]{color}

\hypersetup{colorlinks=true}
\hypersetup{pdfstartview=FitH}
\hypersetup{pdfpagemode=UseNone}
\hypersetup{pdfsource={}}
\hypersetup{pdflang={en-UK}}
\hypersetup{pdfcopyright={Copyright 2017-2018 Niklas Beisert.
  This work may be distributed and/or modified under the
  conditions of the LaTeX Project Public License, either version 1.3
  of this license or (at your option) any later version.}}
\hypersetup{pdflicenseurl={http://www.latex-project.org/lppl.txt}}
\hypersetup{pdfcontactaddress={ETH Zurich, ITP, HIT K,
  Wolfgang-Pauli-Strasse 27}}
\hypersetup{pdfcontactpostcode={8093}}
\hypersetup{pdfcontactcity={Zurich}}
\hypersetup{pdfcontactcountry={Switzerland}}
\hypersetup{pdfcontactemail={nbeisert@itp.phys.ethz.ch}}
\hypersetup{pdfcontacturl={http://people.phys.ethz.ch/\xmptilde nbeisert/}}

\newcommand{\secref}[1]{\hyperref[#1]{section \ref*{#1}}}

\parskip1ex
\parindent0pt
\let\olditemize\itemize
\def\itemize{\olditemize\parskip0pt}

\begin{document}

\title{The \textsf{childdoc} Package}
\hypersetup{pdftitle={The childdoc Package}}
\author{Niklas Beisert\\[2ex]
  Institut f\"ur Theoretische Physik\\
  Eidgen\"ossische Technische Hochschule Z\"urich\\
  Wolfgang-Pauli-Strasse 27, 8093 Z\"urich, Switzerland\\[1ex]
  \href{mailto:nbeisert@itp.phys.ethz.ch}
  {\texttt{nbeisert@itp.phys.ethz.ch}}}
\hypersetup{pdfauthor={Niklas Beisert}}
\hypersetup{pdfsubject={Manual for the LaTeX2e Package childdoc}}
\date{30 December 2018, \textsf{v2.0}}
\maketitle

\begin{abstract}\noindent
\textsf{childdoc} is a \LaTeXe{} package
that enables the direct compilation
of document sections included by |\include|
to individual files.
\end{abstract}

\begingroup
\parskip0ex
\tableofcontents
\endgroup

%%%%%%%%%%%%%%%%%%%%%%%%%%%%%%%%%%%%%%%%%%%%%%%%%%%%%%%%%%%%%%%%%%%%%%%%%%%%%%%%
%%%%%%%%%%%%%%%%%%%%%%%%%%%%%%%%%%%%%%%%%%%%%%%%%%%%%%%%%%%%%%%%%%%%%%%%%%%%%%%%
\section{Introduction}

\LaTeX{} provides a mechanism to structure a large document (such as a book)
into a main file and several child files (containing the chapters)
using the |\include| command.
This mechanism is beneficial for documents
which span hundreds of pages in order to
make the source file(s) more manageable.
Moreover, compilation can be restricted to
selected child files by means of the |\includeonly| command.
The latter feature can be used to reduce the compilation time while editing
(this was significantly more useful in the earlier days of \LaTeX{})
or to generate a smaller document which is easier to navigate.
Another application of |\includeonly| is to generate
documents consisting of selected parts of the complete document.

However, there are a few drawbacks of the plain |\include| mechanism:
\begin{itemize}
\item
The child files cannot be compiled on their own,
they can only be compiled via the main file.
A naive editing environment
(such as a text editor with an option
to have the current file processed by \LaTeX)
may require one to switch to the main file before compiling;
attempting to compile the child file produces errors.
\item
The main file must be modified (each time)
to adjust the |\includeonly| command
to the present needs. This easily leaves the main file in a messy state.
\item
The generated document will always carry the filename
of the main document. This is inconvenient if
several child files are to be compiled and
to be kept for distribution.
\end{itemize}

The present package provides a simple interface
to make child files individually compilable by \LaTeX{}.
Compiling a child file then has the same effect as compiling
the main file with an |\includeonly| command
to select the appropriate child.
Moreover the generated document will carry the name of the child
rather than the main file.
This resolves all three above issues.

This feature is meant to make the editing of books,
thesis documents and lecture notes somewhat more convenient.
However, the package can also be used efficiently for
composing a series of documents (such as exercise sheets)
which are typically distributed individually.
It then assists the author in generating the individual documents
(potentially in different versions)
as well as a document containing the collected series.
Another application is in developing style files
or other kinds of included material
where compilation of the style file could redirect
to a sample or test file.

%%%%%%%%%%%%%%%%%%%%%%%%%%%%%%%%%%%%%%%%%%%%%%%%%%%%%%%%%%%%%%%%%%%%%%%%%%%%%%%%
%%%%%%%%%%%%%%%%%%%%%%%%%%%%%%%%%%%%%%%%%%%%%%%%%%%%%%%%%%%%%%%%%%%%%%%%%%%%%%%%
\section{Usage}

First of all, the package \textsf{childdoc} is \emph{not} a standard
\LaTeXe{} |.sty| style file! Therefore it needs to be invoked in
a non-standard way.

%%%%%%%%%%%%%%%%%%%%%%%%%%%%%%%%%%%%%%%%%%%%%%%%%%%%%%%%%%%%%%%%%%%%%%%%%%%%%%%%
\subsection{Included Files}
\label{sec:include}

%%%%%%%%%%%%%%%%%%%%%%%%%%%%%%%%%%%%%%%%
\DescribeMacro{\childdocmain}
To use the package, add the commands
\begin{center}
\begin{tabular}{l}
|\input{childdoc.def}|\\
|\childdocmain{}|\\
\end{tabular}
\end{center}
at the very top of the main \LaTeX{} file,
in particular \emph{before} the |\documentclass| statement!
The argument of |\childdocmain| should be left empty
(but it must be present).

%%%%%%%%%%%%%%%%%%%%%%%%%%%%%%%%%%%%%%%%
\DescribeMacro{\childdocof}
Furthermore, add the commands
\begin{center}
\begin{tabular}{l}
|\input{childdoc.def}|\\
|\childdocof{|\textit{main}|}|\\
\end{tabular}
\end{center}
at the top of every child file \textit{child}
which is included by |\include{|\textit{child}|}|
from within the main file
(or at least for those files to be compiled individually).
The argument \textit{main} must be the filename of the main file.

There are a couple of
considerations in setting up the main and child documents:

%%%%%%%%%%%%%%%%%%%%%%%%%%%%%%%%%%%%%%%%
\paragraph{Restrictions.}

Please note the following restrictions:
\begin{itemize}
\item
|\childdocmain| must be called with one argument \textit{main}
to ensure compatibility with earlier version of the package.
It must either be empty (|\childdocmain{}|)
or precisely match the filename of the main file in which it is specified.
See \secref{sec:detection} for further information.
\item
The filename \textit{main} must be specified without the |.tex| extension.
\item
The filename \textit{main} is case sensitive
(even in case-insensitive file systems)
due to internal string comparison.
\item
The argument \textit{main} should be fully expanded, it cannot be a macro.
\item
Subdirectories and special characters should be avoided in filenames.
\item
The command |\childdocmain{|\textit{main}|}| must be followed by a whitespace.
It should not be followed immediately by another command
or by a comment mark `|%|'.
This is because the \TeX{} parser reads the token immediately following
the argument of |\childdocmain| and puts it
at the beginning of every child section;
however, a white\-space is ignored.
\end{itemize}

%%%%%%%%%%%%%%%%%%%%%%%%%%%%%%%%%%%%%%%%
\paragraph{Content of Main File.}

It is advisable to place all content in the child files included by |\include|.
Any output contained in the main file will appear in all child documents
unless suppressed manually;
it cannot be suppressed automatically by the |\includeonly| directive
and thus should normally be avoided.
A method to include some content in the main file
by means of conditional processing is described in \secref{sec:conditional}.

%%%%%%%%%%%%%%%%%%%%%%%%%%%%%%%%%%%%%%%%
\paragraph{Page Numbering.}

When only a part of the document is compiled,
the appropriate numbering of pages
(as well as other status parameters)
is determined from the |.aux| files.
The latter contain information from previous passes.
However this information needs to propagate through
all intermediate child documents.
Therefore the page numbering in child documents may well
be inconsistent until the complete document is compiled at least once.

A useful (if unconventional) way to always ensure a consistent
page numbering is to restart the numbering in each child document
and denote the pages by `\textit{child}|.|\textit{page}'
where \textit{child} represents the chapter/section number of the child file.
This can be achieved by the command
|\numberwithin{page}{|\textit{child}|}|
of the \textsf{amsmath} package
where \textit{child} can be |chapter| or |section|
depending on the chosen structuring.
Alternatively, one can modify the macro |\thepage| appropriately
and reset the counter |page| at the start of each child file.

%%%%%%%%%%%%%%%%%%%%%%%%%%%%%%%%%%%%%%%%%%%%%%%%%%%%%%%%%%%%%%%%%%%%%%%%%%%%%%%%
\subsection{Conditional Processing}
\label{sec:conditional}

The package provides a mechanism to compile different versions
of a document. To customise the versions further some conditional processing
can come in handy to distinguish which version is being compiled.
The package provides two macros to describe the compilation context:

%%%%%%%%%%%%%%%%%%%%%%%%%%%%%%%%%%%%%%%%
\DescribeMacro{\ifchilddoc}
The conditional |\ifchilddoc| distinguishes between the compilation of
child documents and the main document:
%
\begin{center}
|\ifchilddoc |\textit{child-code}| |[|\||else |\textit{main-code}]| \||fi|
\end{center}

%%%%%%%%%%%%%%%%%%%%%%%%%%%%%%%%%%%%%%%%
\DescribeMacro{\childdocname}
\DescribeMacro{\childdocjob}
The macro |\childdocname| contains the filename (without extension)
of the main or child file being processed.
Note that |\childdocjob| will always contain the name of the main file.

%%%%%%%%%%%%%%%%%%%%%%%%%%%%%%%%%%%%%%%%
\paragraph{Title Page.}

Conditional processing can be used to include a title or banner page
in the main document when proper precautions are taken.
Importantly, the code in the main file should ensure that the page counter
(as well as other status parameters which are stored in the |.aux| files)
takes the same value after the conditional processing.
Otherwise the page numbers may take divergent values
depending on which part is compiled.

For example, a title page could be declared by:
%
\begin{center}
\begin{tabular}{l}
|\ifchilddoc\||else|\\
|\addtocounter{page}{-1}|\\
\textit{code for title page}\\
|\newpage|\\
|\||fi|
\end{tabular}
\end{center}
%
A banner page for the child documents can be generated by:
%
\begin{center}
\begin{tabular}{l}
|\ifchilddoc|\\
|\addtocounter{page}{-1}|\\
\textit{code for banner page}\\
|\newpage|\\
|\||fi|
\end{tabular}
\end{center}
%
Here one could write a message such as:
\begin{center}
|This is the part \childdocname{} of \childdocjob{}.|
\end{center}

%%%%%%%%%%%%%%%%%%%%%%%%%%%%%%%%%%%%%%%%%%%%%%%%%%%%%%%%%%%%%%%%%%%%%%%%%%%%%%%%
\subsection{Flags}
\label{sec:flags}

The package makes it easy to generate different versions
of the main or child documents.
To this end compilation flags can be defined
and assigned different default values.
They will be particularly useful in conjunction
with the forwarding mechanism described in \secref{sec:forward}.

For example, it may be useful to have a flag |\version|
which can be set to |draft| or |final|.
The document source will contain some conditional code
depending on the value of |\version|.
Suppose further, the flag should default to |final| for the main file
and to |draft| for child files
which is a natural assignment for editing the document.
This is achieved by placing the following code
in the preamble of the main document
(below the |\childdocmain| directive):
%
\begin{center}
\begin{tabular}{l}
|\ifchilddoc|\\
|\providecommand{\version}{draft}|\\
|\||else|\\
|\providecommand{\version}{final}|\\
|\||fi|
\end{tabular}
\end{center}
%
The definition by |\providecommand| makes sure
that previous definitions are not overwritten.
Further statements |\providecommand{\version}{...}|
can thus be added before the above code to override it.

For the main file, one might add a line
(between |\childdocmain| and the above block)
%
\begin{center}
|%\ifchilddoc\||else\providecommand{\version}{draft}\||fi|
\end{center}
%
which can be uncommented to produce a draft version.
Likewise one can add a line to the very top of a child file
(above the |\childdocof{|\textit{main}|}| directive)
%
\begin{center}
|%\providecommand{\version}{final}|
\end{center}
%
which can be uncommented to produce the final version of this child document.

%%%%%%%%%%%%%%%%%%%%%%%%%%%%%%%%%%%%%%%%%%%%%%%%%%%%%%%%%%%%%%%%%%%%%%%%%%%%%%%%
\subsection{Forwarding}
\label{sec:forward}

Different versions of the main or child documents
using compilation flags as described in \secref{sec:flags}
can be (permanently) stored in different files
for convenient compilation, viewing and distribution.
To this end, the package defines a command
to pass on compilation to a different file:

%%%%%%%%%%%%%%%%%%%%%%%%%%%%%%%%%%%%%%%%
\DescribeMacro{\childdocforward}
The command |\childdocforward| redirects processing to
another source file:
%
\begin{center}
\begin{tabular}{l}
|\input{childdoc.def}|\\
|\childdocforward[|\textit{main}|]{|\textit{dest}|}|\\
\end{tabular}
\end{center}
%
The argument \textit{dest} is the destination file
(without extension).
It should be the main file or one of the child files.
Note that further \textsf{childdoc} directives
such as |\childdocof| and |\childdocforward|
in the indicated file will be processed in this form.
The optional argument \textit{main}
passes on directly to the main file \textit{main}
while pretending to compile the child \textit{dest}.
This form behaves as if \textit{dest}
issues |\childdocof{|\textit{main}|}| right away,
and no further \textsf{childdoc} directives will be processed.

%%%%%%%%%%%%%%%%%%%%%%%%%%%%%%%%%%%%%%%%
\DescribeMacro{\...prefix}
In the alternative form |\childdocforwardprefix|,
%
\begin{center}
\begin{tabular}{l}
|\input{childdoc.def}|\\
|\childdocforwardprefix[|\textit{main}|]{|\textit{prefix}|}{|\textit{dest}|}|
\end{tabular}
\end{center}
%
the destination file is determined by a pattern
depending on the current file:
To make this work, the current file must be called
`{\textit{prefix}\hspace{0.2em}\textit{suffix}}'
with \textit{prefix} matching precisely the argument.
Processing is then passed on to the file
`{\textit{dest}\hspace{0.2em}\textit{suffix}}'.
Surely, the same effect is achieved by
directly specifying the
argument `{\textit{dest}\hspace{0.2em}\textit{suffix}}'
in the first form.
However, that requires to set up a different file
for each child. With the alternative form of the command
all these files can have exactly the same content
which simplifies setting them up and maintaining them.

For example, the following file |draft.tex|
with a compilation flag |\version| as described in \secref{sec:flags}
compiles the main document as a draft:
%
\begin{center}
\begin{tabular}{l}
|\def\version{draft}|\\
|\input{childdoc.def}|\\
|\childdocforward{|\textit{main}|}|
\end{tabular}
\end{center}
%
Likewise, the following files |final|\textit{nn}|.tex|
compile the final version of the child document
|child|\textit{nn}|.tex|:
%
\begin{center}
\begin{tabular}{l}
|\def\version{final}|\\
|\input{childdoc.def}|\\
|\childdocforwardprefix{final}{child}|
\end{tabular}
\end{center}
%

Note that when several versions of a main file and/or of each child file
are to be generated, it may be convenient to set up a |Makefile| or
shell script to automatise the process.

%%%%%%%%%%%%%%%%%%%%%%%%%%%%%%%%%%%%%%%%%%%%%%%%%%%%%%%%%%%%%%%%%%%%%%%%%%%%%%%%
\subsection{Command Line Processing}
\label{sec:commandline}

The effect of redirection files can also be achieved by invoking
the \LaTeX{} compiler with a more elaborate command line.
Most conveniently this should be done as part
of a shell script or a |Makefile|.

When using \textsf{childdoc} in the main file, the following
command lines effectively perform a redirection
(note that depending on the shell being used,
backslashes may have to be doubled: `|\|' $\to$ `|\\|'):
%
\begin{center}
|... -jobname "|\textit{target}|" |\\|"|[\textit{flags}]%
|\input{childdoc.def}\childdocforward[|\textit{main}|]{|\textit{dest}|}"|
\end{center}
%
Here \textit{target} is the name of the output file,
\textit{main} is the name of the main file
and \textit{dest} is the name of the main or child file to be processed
(all filenames without extensions).
The optional argument \textit{main} can be omitted
if \textit{main} matches \textit{dest}.
Optionally, compilation \textit{flags} can be defined via |\def| commands.
This command line makes the \TeX{} engine believe
it is compiling the file \textit{target}
whose content is specified as the latter parameter.
The provided code then forwards the processing to
\textit{main} or \textit{dest} as described in \secref{sec:forward}.

%%%%%%%%%%%%%%%%%%%%%%%%%%%%%%%%%%%%%%%%%%%%%%%%%%%%%%%%%%%%%%%%%%%%%%%%%%%%%%%%
\subsection{Include by Input}
\label{sec:input}

Including child documents by |\include| has some restrictions by design.
Most notably, the content of a child document always occupies
its own set of pages; pages cannot be shared between child documents.
Usually, this behaviour makes perfect sense
because each child document contain an essential part of the document.
However, in some situations it may be desirable to compose
a document from a collection of parts
without having mandatory page breaks between then.
For this case, the package
provides a mechanism to include parts
by |\input| which can also be processed individually.
However, by construction this mechanism
requires manual handling of the content to be output.

%%%%%%%%%%%%%%%%%%%%%%%%%%%%%%%%%%%%%%%%
\DescribeMacro{\ifchilddocmanual}
The main file should be prepared as usual, see \secref{sec:include}.
However, the document body must make a distinction
between processing of an individual part and of the main document, e.g.:
%
\begin{center}
\begin{tabular}{l}
|\ifchilddocmanual|\\
|\input{\childdocname}|\\
|\||else|\\
\textit{document body with }|\input{|\textit{part}|}|\\
|\||fi|
\end{tabular}
\end{center}
%
The conditional |\ifchilddocmanual| is true whenever
a part to be included by |\input| is being compiled,
and the name of the part is stored in |\childdocname|.

%%%%%%%%%%%%%%%%%%%%%%%%%%%%%%%%%%%%%%%%
\DescribeMacro{\childdocby}
Each part to be included by |\input| should start with:
%
\begin{center}
\begin{tabular}{l}
|\input{childdoc.def}|\\
|\childdocby{|\textit{main}|}|\\
\end{tabular}
\end{center}
%
The directive |\childdocby| is similar to |\childdocof|
described in \secref{sec:include},
but the subsequent selection of content must be done manually.
To that end, both |\ifchilddoc| and |\ifchilddocmanual|
will be true upon processing of a part,
and the name of the part is stored in |\childdocname|.
Note that |\jobname| will be set to the filename of the current part
so that each part receives an individual |.aux| file
that does not interfere with the |.aux| file(s) of the main document.
This behaviour can be altered by the alternative form
|\childdocby[*]{|\textit{main}|}| (with a non-empty optional argument)
which uses the |.aux| file of the main document
by setting |\jobname| to \textit{main}.

%%%%%%%%%%%%%%%%%%%%%%%%%%%%%%%%%%%%%%%%%%%%%%%%%%%%%%%%%%%%%%%%%%%%%%%%%%%%%%%%
\subsection{Driver Development}
\label{sec:driver}

The \textsf{childdoc} mechanism can also be use for the development
of definition files such as \LaTeX{} styles or classes.
This case differs from the above setup with multiple parts
included by |\include| in that no |\includeonly| should be invoked.
This can be achieved by starting the include file
(before |\ProvidesPackage|) with:
%
\begin{center}
\begin{tabular}{l}
|\input{childdoc.def}|\\
|\childdocforward{|\textit{main}|}|\\
\end{tabular}
\end{center}
%
or alternatively with:
%
\begin{center}
\begin{tabular}{l}
|\input{childdoc.def}|\\
|\childdocby{|\textit{main}|}|\\
\end{tabular}
\end{center}
%
Both forms have slightly different effects as described above.
The main file is prepared as usual, see \secref{sec:include}.

%%%%%%%%%%%%%%%%%%%%%%%%%%%%%%%%%%%%%%%%%%%%%%%%%%%%%%%%%%%%%%%%%%%%%%%%%%%%%%%%
\subsection{Legacy Detection}
\label{sec:detection}

The directive |\childdocmain| in the main file can detect
whether the complete document or merely a child is to be compiled
even without using the directive |\childdocof|.
This method is deprecated because it is less robust
and there is no compelling reason to use it;
it is merely provided for backward compatibility
and it may be removed in future versions.

If the detection mechanism is to be used,
it is mandatory to correctly specify
the filename of the main file as the argument of |\childdocmain|:
%
\begin{center}
\begin{tabular}{l}
|\input{childdoc.def}|\\
|\childdocmain{|\textit{main}|}|\\
\end{tabular}
\end{center}
%
If |\jobname| does not match the argument \textit{main} of |\childdocmain|,
it is assumed that |\jobname| points to the child file to be compiled.
When using |\childdocmain| with the main file specified as argument,
it suffices to start a child file
with just |\input{|\textit{main}|}|
without loading of the package and using |\childdocof|.
If instead all processing is done
with the appropriate \textsf{childdoc} directives,
the argument of \textit{main} of |\childdocmain| can be empty.

An alternative version of the command line processing described
in \secref{sec:commandline} using the detection mechanism reads:
%
\begin{center}
|... -jobname "|\textit{target}|" "|[\textit{flags}]%
[|\def\jobname{|\textit{dest}|}|]|\input{|\textit{main}|}"|
\end{center}

%%%%%%%%%%%%%%%%%%%%%%%%%%%%%%%%%%%%%%%%%%%%%%%%%%%%%%%%%%%%%%%%%%%%%%%%%%%%%%%%
\subsection{Manual Code}
\label{sec:manual}

In case one cannot be certain whether the definitions file |childdoc.def|
is installed on the target \TeX{} distribution
and one prefers not to ship it,
it is conceivable to paste a few relevant commands into the sources.

To that end, drop all statements |\input{childdoc.def}|
and perform the replacements as outlined below.
Instead of |\childdocmain{|\textit{main}|}| add the following code
to the top of the main file:
%
\begin{center}
\begin{tabular}{l}
|\||ifdefined\childdocname\endinput\||fi\newif\ifchilddoc|\\
|\edef\childdocname{\scantokens\expandafter{\jobname\noexpand}}|\\
|\def\childdocmain{|\textit{main}|}\||ifx\childdocmain\childdocname\||else|\\
|\childdoctrue\includeonly{\childdocname}\let\jobname\childdocmain\||fi|\\
\end{tabular}
\end{center}
%
Instead of |\childdocof{|\textit{main}|}| just include the main file
at the top of each child file:
%
\begin{center}
|\input{|\textit{main}|}|
\end{center}
%
A simple redirection |\childdocforward{|\textit{dest}|}| is achieved by:
%
\begin{center}
|\def\jobname{|\textit{dest}|}\input{\jobname}|
\end{center}
%
The redirection with prefix
|\childdocforwardprefix[|\textit{prefix}|]{|\textit{dest}|}|
is accomplished by:
%
\begin{center}
\begin{tabular}{l}
|{\edef\jobname{\scantokens\expandafter{\jobname\noexpand}}|\\
|\def\redirectjob |\textit{prefix}|#1~~~{\gdef\jobname{|\textit{dest}|#1}}|\\
|\expandafter\redirectjob\jobname~~~}\input{\jobname}|
\end{tabular}
\end{center}

In an alternative approach,
child documents can be compiled by a specific command line
without additional code or specific definitions:
%
\begin{center}
|... -jobname "|\textit{target}|" "|[\textit{flags}]%
|\includeonly{|\textit{dest}|}\input{|\textit{main}|}"|
\end{center}
%

%%%%%%%%%%%%%%%%%%%%%%%%%%%%%%%%%%%%%%%%%%%%%%%%%%%%%%%%%%%%%%%%%%%%%%%%%%%%%%%%
%%%%%%%%%%%%%%%%%%%%%%%%%%%%%%%%%%%%%%%%%%%%%%%%%%%%%%%%%%%%%%%%%%%%%%%%%%%%%%%%
\section{Information}

%%%%%%%%%%%%%%%%%%%%%%%%%%%%%%%%%%%%%%%%%%%%%%%%%%%%%%%%%%%%%%%%%%%%%%%%%%%%%%%%
\subsection{Copyright}

Copyright \copyright{} 2017--2018 Niklas Beisert

This work may be distributed and/or modified under the
conditions of the \LaTeX{} Project Public License, either version 1.3
of this license or (at your option) any later version.
The latest version of this license is in
  \url{http://www.latex-project.org/lppl.txt}
and version 1.3 or later is part of all distributions of \LaTeX{}
version 2005/12/01 or later.

This work has the LPPL maintenance status `maintained'.

The Current Maintainer of this work is Niklas Beisert.

This work consists of the files |README.txt|, |childdoc.ins| and |childdoc.dtx|
as well as the derived files |childdoc.def|, |cdocsamp.tex|
with |cdocsch1.tex|, |cdocsch2.tex|, |cdocspt3.tex|, |cdocspt4.tex|,
|cdocsdrf.tex|, |cdocsfn1.tex|, |cdocsfn2.tex|
as well as |childdoc.pdf|.

%%%%%%%%%%%%%%%%%%%%%%%%%%%%%%%%%%%%%%%%%%%%%%%%%%%%%%%%%%%%%%%%%%%%%%%%%%%%%%%%
\subsection{Files and Installation}

The package consists of the files:
%
\begin{center}
\begin{tabular}{ll}
    |README.txt|   & readme file \\
    |childdoc.ins| & installation file \\
    |childdoc.dtx| & source file \\
    |childdoc.def| & definition file \\
    |cdocsamp.tex| & sample main file \\
    |cdocsch1.tex| & sample include file \\
    |cdocsch2.tex| & sample include file \\
    |cdocspt3.tex| & sample part file \\
    |cdocspt4.tex| & sample part file \\
    |cdocsdrf.tex| & sample redirection file \\
    |cdocsfn1.tex| & sample redirection file \\
    |cdocsfn2.tex| & sample redirection file \\
    |childdoc.pdf| & manual
\end{tabular}
\end{center}
%
The distribution consists of the files
|README.txt|, |childdoc.ins| and |childdoc.dtx|.
%
\begin{itemize}
\item
Run (pdf)\LaTeX{} on |childdoc.dtx|
to compile the manual |childdoc.pdf| (this file).
\item
Run \LaTeX{} on |childdoc.ins| to create the definitions file |childdoc.def|
and the sample |cdocsamp.tex| with include files
|cdocsch1.tex|, |cdocsch2.tex|, |cdocspt3.tex|, |cdocspt4.tex|,
|cdocsdrf.tex|, |cdocsfn1.tex|, |cdocsfn2.tex|.
Then copy the file |childdoc.def| to an appropriate directory of your \LaTeX{}
distribution, e.g.\ \textit{texmf-root}|/tex/latex/childdoc|.
\end{itemize}

%%%%%%%%%%%%%%%%%%%%%%%%%%%%%%%%%%%%%%%%%%%%%%%%%%%%%%%%%%%%%%%%%%%%%%%%%%%%%%%%
\subsection{Related CTAN Packages}

There are several other packages which offer a similar functionality:
%
\begin{itemize}
\item
The packages
\href{http://ctan.org/pkg/docmute}{\textsf{docmute}},
\href{http://ctan.org/pkg/includex}{\textsf{includex}} and
\href{http://ctan.org/pkg/standalone}{\textsf{standalone}}
provide commands to include only the document body of
a child file thus allowing both files to be compiled individually.
\item
The packages \href{http://ctan.org/pkg/subdocs}{\textsf{subdocs}}
and \href{http://ctan.org/pkg/subfiles}{\textsf{subfiles}}
provide structures in which the main and child documents can be
encapsulated and allowing them to be compiled individually.
The inclusion mechanism is different from the conventional |\include|.
\item
The package \href{http://ctan.org/pkg/combine}{\textsf{combine}}
is an elaborate solution to combine several documents into one.
\end{itemize}
%
See also the CTAN topic \href{http://ctan.org/topic/subdocs}{\textsf{subdocs}}
for further related packages.
The present package differs from the above solutions in that
a document structure constructed with the conventional |\include| mechanism
just needs two extra commands at the top of every file
such that all constituent files can be compiled individually.

%%%%%%%%%%%%%%%%%%%%%%%%%%%%%%%%%%%%%%%%%%%%%%%%%%%%%%%%%%%%%%%%%%%%%%%%%%%%%%%%
%\subsection{Feature Suggestions}
%
%The following is a list of features which may be useful for future
%versions of this package:
%%
%\begin{itemize}
%\item
%\ldots
%\end{itemize}

%%%%%%%%%%%%%%%%%%%%%%%%%%%%%%%%%%%%%%%%%%%%%%%%%%%%%%%%%%%%%%%%%%%%%%%%%%%%%%%%
\subsection{Revision History}

%%%%%%%%%%%%%%%%%%%%%%%%%%%%%%%%%%%%%%%%
\paragraph{v2.0:} 2018/12/30

\begin{itemize}
\item
immediate forward processing
\item
added |\childdocby| mechanism
\item
manual restructured
\end{itemize}

%%%%%%%%%%%%%%%%%%%%%%%%%%%%%%%%%%%%%%%%
\paragraph{v1.6:} 2018/01/17

\begin{itemize}
\item
application for development of include files
\item
corrections to manual
\end{itemize}

%%%%%%%%%%%%%%%%%%%%%%%%%%%%%%%%%%%%%%%%
\paragraph{v1.5:} 2017/05/21

\begin{itemize}
\item
more complete structuring introduced
\item
|\childdocof| introduced
\item
|\childdoc| renamed to |\childdocmain|
\item
|\childredirect| renamed to |\childdocforward| and |\childdocforwardprefix|
and functionality expanded
\end{itemize}

%%%%%%%%%%%%%%%%%%%%%%%%%%%%%%%%%%%%%%%%
\paragraph{v1.0:} 2017/04/27

\begin{itemize}
\item
manual and install package
\item
first version published on CTAN
\end{itemize}

%%%%%%%%%%%%%%%%%%%%%%%%%%%%%%%%%%%%%%%%
\paragraph{v0.6:} 2017/04/26

\begin{itemize}
\item
redirection mechanism added
\end{itemize}

%%%%%%%%%%%%%%%%%%%%%%%%%%%%%%%%%%%%%%%%
\paragraph{v0.5:} 2017/04/26

\begin{itemize}
\item
functionality in definition file
\end{itemize}


%%%%%%%%%%%%%%%%%%%%%%%%%%%%%%%%%%%%%%%%%%%%%%%%%%%%%%%%%%%%%%%%%%%%%%%%%%%%%%%%
%%%%%%%%%%%%%%%%%%%%%%%%%%%%%%%%%%%%%%%%%%%%%%%%%%%%%%%%%%%%%%%%%%%%%%%%%%%%%%%%
%%%%%%%%%%%%%%%%%%%%%%%%%%%%%%%%%%%%%%%%%%%%%%%%%%%%%%%%%%%%%%%%%%%%%%%%%%%%%%%%
\appendix

\settowidth\MacroIndent{\rmfamily\scriptsize 000\ }

 \DocInput{childdoc.dtx}

\end{document}
%</driver>
% \fi
%
% %%%%%%%%%%%%%%%%%%%%%%%%%%%%%%%%%%%%%%%%%%%%%%%%%%%%%%%%%%%%%%%%%%%%%%%%%%%%%%
% %%%%%%%%%%%%%%%%%%%%%%%%%%%%%%%%%%%%%%%%%%%%%%%%%%%%%%%%%%%%%%%%%%%%%%%%%%%%%%
% \section{Sample}
%\iffalse
%<*samplemain>
%\fi
%
% The following presents a sample document
% with two chapters, two parts, a title page,
% a compile flag as well as three forwarding files to set the flag.
% It consists of eight |.tex| files:
% \begin{center}
% \begin{tabular}{ll}
% |cdocsamp.tex|&main file\\
% |cdocsch1.tex|&include file for chapter 1\\
% |cdocsch2.tex|&include file for chapter 2\\
% |cdocspt3.tex|&include file for part 3\\
% |cdocspt4.tex|&include file for part 4\\
% |cdocsdrf.tex|&forwarding file for main file in draft mode\\
% |cdocsfi1.tex|&forwarding file for final version of chapter 1\\
% |cdocsfi2.tex|&forwarding file for final version of chapter 2\\
% \end{tabular}
% \end{center}
% Each of the eight files can be compiled directly by the \LaTeX{} compiler.
%
% %%%%%%%%%%%%%%%%%%%%%%%%%%%%%%%%%%%%%%
% \paragraph{Main File.}
%
% The main file is called |cdocsamp.tex|.
%
% Load the \textsf{childdoc} definitions and
% declare the filename for the main document:
%    \begin{macrocode}
\input{childdoc.def}
\childdocmain{}
%    \end{macrocode}

% Optional override for |\version| flag:
%    \begin{macrocode}
%%\ifchilddoc\else\providecommand{\version}{draft}\fi
%    \end{macrocode}

% Define the default values for the |\version| flag
% (|final| for the main file and |draft| for childs):
%    \begin{macrocode}
\ifchilddoc
\providecommand{\version}{draft}
\else
\providecommand{\version}{final}
\fi
%    \end{macrocode}

% Load the standard document class:
%    \begin{macrocode}
\documentclass[12pt]{article}
%    \end{macrocode}

% Start the document body:
%    \begin{macrocode}
\begin{document}
%    \end{macrocode}

% Declare a title page.
% Print title, part of document being processed and version flag:
%    \begin{macrocode}
\addtocounter{page}{-1}
\begin{center}
{\LARGE\bfseries{}childdoc example\par}
\vspace{1cm}
\ifchilddoc
\ifchilddocmanual part\else chapter\fi:
`\childdocname' of `\childdocjob'\par
\else
main document: `\childdocjob'\par
\fi
version: \version\par
\end{center}
\newpage
%    \end{macrocode}

% Manually include selected file,
% otherwise process as usual:
%    \begin{macrocode}
\ifchilddocmanual
\section*{part `\childdocname'}
\input{\childdocname}
\else
%    \end{macrocode}

% Include the two chapters:
%    \begin{macrocode}
\include{cdocsch1}
\include{cdocsch2}
%    \end{macrocode}

% Include the two parts unless only chapters should be displayed:
%    \begin{macrocode}
\ifchilddoc\else
\section{part three}
\input{cdocspt3}
\section{part four}
\input{cdocspt4}
\fi
%    \end{macrocode}

% Process as usual until here:
%    \begin{macrocode}
\fi
%    \end{macrocode}

% End of document body:
%    \begin{macrocode}
\end{document}
%    \end{macrocode}
%\iffalse
%</samplemain>
%\fi
%
% %%%%%%%%%%%%%%%%%%%%%%%%%%%%%%%%%%%%%%
% \paragraph{Chapter Include Files.}
%
% The include files are called |cdocsch1.tex| and |cdocsch2.tex|.
%
%\iffalse
%<*samplechap1|samplechap2>
%\fi

% Optional override for |\version| flag:
%    \begin{macrocode}
%%\providecommand{\version}{final}
%    \end{macrocode}

% Include the main document:
%    \begin{macrocode}
\input{childdoc.def}
\childdocof{cdocsamp}
%    \end{macrocode}

%\iffalse
%</samplechap1|samplechap2>
%\fi
%
%\iffalse
%<*samplechap1>
%\fi
% Some text for chapter 1:
%    \begin{macrocode}
\section{one}
some text in chapter one
%    \end{macrocode}

%\iffalse
%</samplechap1>
%\fi
% Some text for chapter 2:
%\iffalse
%<*samplechap2>
%\fi
%    \begin{macrocode}
\section{two}
more text in chapter two
%    \end{macrocode}

%\iffalse
%</samplechap2>
%\fi
%
% %%%%%%%%%%%%%%%%%%%%%%%%%%%%%%%%%%%%%%
% \paragraph{Part Include Files.}
%
% The include files are called |cdocspt3.tex| and |cdocspt4.tex|.
%
%\iffalse
%<*samplepart3|samplepart4>
%\fi

% Optional override for |\version| flag:
%    \begin{macrocode}
%%\providecommand{\version}{final}
%    \end{macrocode}

% Include the main document:
%    \begin{macrocode}
\input{childdoc.def}
\childdocby{cdocsamp}
%    \end{macrocode}

%\iffalse
%</samplepart3|samplepart4>
%\fi
%
%\iffalse
%<*samplepart3>
%\fi
% Some text for part 3:
%    \begin{macrocode}
some text in part three
%    \end{macrocode}

%\iffalse
%</samplepart3>
%\fi
% Some text for part 4:
%\iffalse
%<*samplepart4>
%\fi
%    \begin{macrocode}
more text in part four
%    \end{macrocode}

%\iffalse
%</samplepart4>
%\fi
%
% %%%%%%%%%%%%%%%%%%%%%%%%%%%%%%%%%%%%%%
% \paragraph{Forwarding for a Complete Draft.}
%
% The following forwarding file |cdocsdrf.tex|
% compiles the main document in draft mode:
%\iffalse
%<*sampledraft>
%\fi
%    \begin{macrocode}
\def\version{draft}
\input{childdoc.def}
\childdocforward{cdocsamp}
%    \end{macrocode}

%\iffalse
%</sampledraft>
%\fi
%
% %%%%%%%%%%%%%%%%%%%%%%%%%%%%%%%%%%%%%%
% \paragraph{Forwarding for Final Version of the Chapters.}
%
% The following forwarding files |cdocsfn1.tex| and |cdocsfn2.tex|
% (with identical content)
% compile the final versions of the child documents
% |cdocsch1.tex| and |cdocsch2.tex|, respectively:
%\iffalse
%<*samplefinal>
%\fi
%    \begin{macrocode}
\def\version{final}
\input{childdoc.def}
\childdocforwardprefix[cdocsamp]{cdocsfn}{cdocsch}
%    \end{macrocode}

%\iffalse
%</samplefinal>
%\fi
%
% %%%%%%%%%%%%%%%%%%%%%%%%%%%%%%%%%%%%%%
% \paragraph{Command Line Processing.}
%
% The following three command lines generate the output files
% |cdocscld|, |cdocscl1| and |cdocscl2|
% which should be identical to
% |cdocsdrf|, |cdocsch1| and |cdocsfn2|, respectively:
% \begin{center}
% \begin{tabular}{l}
% |latex -jobname cdocscld \|\\
% |  "\def\version{draft}\input{childdoc.def}\childdocforward{cdocsamp}"|\\
% |latex -jobname cdocscl1 \|\\
% |  "\input{childdoc.def}\childdocforward[cdocsamp]{cdocsch1}"|\\
% |latex -jobname cdocscl2 \|\\
% |  "\def\version{final}\input{childdoc.def}\childdocforward{cdocsch2}"|
% \end{tabular}
% \end{center}
% Note that the trailing backslash on each first line
% merely continues the input to the second line
% (for convenient cut ant paste).
% Furthermore, the command |latex| can be replaced by any
% of its alternative versions such as |pdflatex|.
%
% %%%%%%%%%%%%%%%%%%%%%%%%%%%%%%%%%%%%%%%%%%%%%%%%%%%%%%%%%%%%%%%%%%%%%%%%%%%%%%
% %%%%%%%%%%%%%%%%%%%%%%%%%%%%%%%%%%%%%%%%%%%%%%%%%%%%%%%%%%%%%%%%%%%%%%%%%%%%%%
% \section{Implementation}
%\iffalse
%<*package>
%\fi
%
% This section describes the definitions file |childdoc.def|.

% The definitions cannot be loaded using |\usepackage| or |\RequirePackage|
% which has a mechanism to prevent loading a style file more than once.
% When loading the definitions by means of |\input|
% multiple instances have to be prevented manually:
%\iffalse
%This code needs to be before the `\ProvidesFile' directive
%which is defined at the beginning of this file.
%Therefore it is also placed there and commented out here.
%</package>
%<*discard>
%\fi
%    \begin{macrocode}
\ifdefined\childdocmain\endinput\fi
%    \end{macrocode}
%\iffalse
%</discard>
%<*package>
%\fi
%
% \macro{\ifchilddoc}
% \macro{\ifchilddocmanual}
% The conditional |\ifchilddoc| tells whether a
% child (true) or main (false) document is being compiled.
% The conditional |\ifchilddocmanual| tells whether
% the |\includeonly| mechanism is used (false) or
% the selection of child files must be performed manually (true).
% The definitions initialise to false:
%    \begin{macrocode}
\newif\ifchilddoc
\newif\ifchilddocmanual
%    \end{macrocode}

% \macro{\childdocname}
% \macro{\childdocjob}
% The macro |\childdocname| stores the name of the main document
% to be compiled. The macro |\childdocjob| stores the name of
% the document on which the \LaTeX{} compiler was originally invoked.
% The content of |\jobname| cannot be compared
% to filenames specified in the source due to different catcodes.
% The following code rescans |\jobname|, stores the result
% in |\childdocname| and saves a copy in |\childdocjob|:
%    \begin{macrocode}
\edef\childdocname{\scantokens\expandafter{\jobname\noexpand}}
\let\childdocjob\childdocname
%    \end{macrocode}

% \macro{\childdocdisable}
% The macro |\childdocdisable| prevents the main file
% from being processed more than once.
% At this stage, the main document command |\childdocmain|
% is assumed to be called once again where it should do nothing.
% Any subsequent call to it should prevent
% a secondary processing of the main document
% It overwrites the forwarding commands
% |\childdocof| and |\childdocforward|
% with empty macros to prevent further inclusions of the main document:
%    \begin{macrocode}
\newcommand{\childdocdisable}
{
  \renewcommand{\childdocmain}[1]{\renewcommand{\childdocmain}[1]{\endinput}}
  \renewcommand{\childdocof}[1]{}
  \renewcommand{\childdocby}[2][]{}
  \renewcommand{\childdocforward}[2][]{}
  \renewcommand{\childdocdisable}{}
}
%    \end{macrocode}

% \macro{\childdocmain}
% The macro |\childdocmain| is to be called at the top of the main file
% with nothing or the main filename (without extension) as argument.
% First, it breaks loops.
% If the argument is not empty and does not match |\childdocname|
% (which is set by the first inclusion of |childdoc.def|),
% |\ifchilddoc| is set to true, |\includeonly| is applied to the child file
% and |\jobname| is set to the main file
% (for proper handling of |.aux| files):
%    \begin{macrocode}
\newcommand{\childdocmain}[1]
{
  \childdocdisable\childdocmain{}
  \if?#1?\else
    \begingroup
      \def\childdoctmp{#1}
      \ifx\childdoctmp\childdocname
        \def\childdoctmp{}
      \else
        \def\childdoctmp
        {
          \childdoctrue
          \includeonly{\childdocname}
          \def\childdocjob{#1}
          \def\jobname{#1}
        }
      \fi
      \expandafter
    \endgroup
    \childdoctmp
  \fi
}
%    \end{macrocode}

% \macro{\childdocof}
% The command |\childdocof| redirects
% compilation to the main file |#1|.
%    \begin{macrocode}
\newcommand{\childdocof}[1]
{
  \childdocdisable
  \childdoctrue
  \includeonly{\childdocname}
  \def\jobname{#1}
  \def\childdocjob{#1}
  \input{#1}
}
%    \end{macrocode}

% \macro{\childdocby}
% The command |\childdocby| ....
%    \begin{macrocode}
\newcommand{\childdocby}[2][]
{
  \childdocdisable
  \childdoctrue
  \childdocmanualtrue
  \if?#1?\else
    \def\jobname{#2}
  \fi
  \def\childdocjob{#2}
  \input{#2}
  \endinput
}
%    \end{macrocode}

% \macro{\childdocforward}
% The command |\childdocforward| redirects
% compilation to the main file or
% (if the optional argument is given) a child file.
% Parameters are set as if the main file
% or a child file starting with |\childdocof| was compiled.
% Then compilation is handed over to the main file:
%    \begin{macrocode}
\newcommand{\childdocforward}[2][]
{
  \begingroup
    \if?#1?
      \def\childdoctmp
      {
        \def\childdocname{#2}
        \def\childdocjob{#2}
        \def\jobname{#2}
        \input{#2}
        \endinput
      }
    \else
      \def\childdoctmp
      {
        \childdocdisable
        \def\childdocname{#2}
        \childdoctrue
        \includeonly{#2}
        \def\childdocjob{#1}
        \def\jobname{#1}
        \input{#1}
        \endinput
      }
    \fi
    \expandafter
  \endgroup
  \childdoctmp
}
%    \end{macrocode}

% \macro{\childdocforwardprefix}
% The command |\childdocforwardprefix| redirects
% compilation to the main or a child file by means of a pattern.
% The prefix |#1| in the current filename is replaced by |#2|
% and the suffix of the current filename is kept
% (it is assumed that the filename does not contain the substring `|~~~|'
% which is used as a delimiter).
% Compilation is handed over to the new file by |\childdocforward|:
%    \begin{macrocode}
\newcommand{\childdocforwardprefix}[3][]
{
  \begingroup
    \def\childdocextract #2##1~~~{\def\childdoctmp{\childdocforward[#1]{#3##1}}}
    \expandafter\childdocextract\childdocname~~~
    \expandafter
  \endgroup
  \childdoctmp
}
%    \end{macrocode}

% \macro{\childdoc}
% The deprecated macro |\childdoc| is a legacy version of |\childdocmain|:
%    \begin{macrocode}
\newcommand{\childdoc}{\childdocmain}
%    \end{macrocode}

% \macro{\childdocredirect}
% The deprecated macro |\childdocredirect| is a legacy version
% of |\childdocforward| and |\childdocforwardprefix|:
%    \begin{macrocode}
\newcommand{\childdocredirect}[2][]
{
  \begingroup
    \if?#1?
      \def\childdoctmp{\childdocforward{#2}}
    \else
      \def\childdoctmp{\childdocforwardprefix{#1}{#2}}
    \fi
    \expandafter
  \endgroup
  \childdoctmp
}
%    \end{macrocode}

%\iffalse
%</package>
%\fi
%
\endinput
\childdocforward{cdocsamp}"|\\
% |latex -jobname cdocscl1 \|\\
% |  "% \iffalse
%
% childdoc.dtx Copyright (C) 2017-2018 Niklas Beisert
%
% This work may be distributed and/or modified under the
% conditions of the LaTeX Project Public License, either version 1.3
% of this license or (at your option) any later version.
% The latest version of this license is in
%   http://www.latex-project.org/lppl.txt
% and version 1.3 or later is part of all distributions of LaTeX
% version 2005/12/01 or later.
%
% This work has the LPPL maintenance status `maintained'.
%
% The Current Maintainer of this work is Niklas Beisert.
%
% This work consists of the files childdoc.dtx and childdoc.ins
% and the derived files childdoc.def and cdocsamp.tex with
% cdocsch1.tex, cdocsch2.tex, cdocsdrf.tex, cdocsfn1.tex, cdocsfn2.tex.
%
%<package>\ifdefined\childdocmain\endinput\fi
%<package>\ProvidesFile{childdoc.def}[2018/12/30 v2.0 child document driver]
%<samplemain>\ProvidesFile{cdocsamp.tex}[2018/12/30 v2.0 sample for childdoc]
%<*driver>
%\ProvidesFile{childdoc.drv}[2018/12/30 v2.0 childdoc reference manual file]
\PassOptionsToClass{10pt,a4paper}{article}
\documentclass{ltxdoc}

\usepackage[margin=35mm]{geometry}
\usepackage{hyperref}
\usepackage{hyperxmp}
\usepackage[usenames]{color}

\hypersetup{colorlinks=true}
\hypersetup{pdfstartview=FitH}
\hypersetup{pdfpagemode=UseNone}
\hypersetup{pdfsource={}}
\hypersetup{pdflang={en-UK}}
\hypersetup{pdfcopyright={Copyright 2017-2018 Niklas Beisert.
  This work may be distributed and/or modified under the
  conditions of the LaTeX Project Public License, either version 1.3
  of this license or (at your option) any later version.}}
\hypersetup{pdflicenseurl={http://www.latex-project.org/lppl.txt}}
\hypersetup{pdfcontactaddress={ETH Zurich, ITP, HIT K,
  Wolfgang-Pauli-Strasse 27}}
\hypersetup{pdfcontactpostcode={8093}}
\hypersetup{pdfcontactcity={Zurich}}
\hypersetup{pdfcontactcountry={Switzerland}}
\hypersetup{pdfcontactemail={nbeisert@itp.phys.ethz.ch}}
\hypersetup{pdfcontacturl={http://people.phys.ethz.ch/\xmptilde nbeisert/}}

\newcommand{\secref}[1]{\hyperref[#1]{section \ref*{#1}}}

\parskip1ex
\parindent0pt
\let\olditemize\itemize
\def\itemize{\olditemize\parskip0pt}

\begin{document}

\title{The \textsf{childdoc} Package}
\hypersetup{pdftitle={The childdoc Package}}
\author{Niklas Beisert\\[2ex]
  Institut f\"ur Theoretische Physik\\
  Eidgen\"ossische Technische Hochschule Z\"urich\\
  Wolfgang-Pauli-Strasse 27, 8093 Z\"urich, Switzerland\\[1ex]
  \href{mailto:nbeisert@itp.phys.ethz.ch}
  {\texttt{nbeisert@itp.phys.ethz.ch}}}
\hypersetup{pdfauthor={Niklas Beisert}}
\hypersetup{pdfsubject={Manual for the LaTeX2e Package childdoc}}
\date{30 December 2018, \textsf{v2.0}}
\maketitle

\begin{abstract}\noindent
\textsf{childdoc} is a \LaTeXe{} package
that enables the direct compilation
of document sections included by |\include|
to individual files.
\end{abstract}

\begingroup
\parskip0ex
\tableofcontents
\endgroup

%%%%%%%%%%%%%%%%%%%%%%%%%%%%%%%%%%%%%%%%%%%%%%%%%%%%%%%%%%%%%%%%%%%%%%%%%%%%%%%%
%%%%%%%%%%%%%%%%%%%%%%%%%%%%%%%%%%%%%%%%%%%%%%%%%%%%%%%%%%%%%%%%%%%%%%%%%%%%%%%%
\section{Introduction}

\LaTeX{} provides a mechanism to structure a large document (such as a book)
into a main file and several child files (containing the chapters)
using the |\include| command.
This mechanism is beneficial for documents
which span hundreds of pages in order to
make the source file(s) more manageable.
Moreover, compilation can be restricted to
selected child files by means of the |\includeonly| command.
The latter feature can be used to reduce the compilation time while editing
(this was significantly more useful in the earlier days of \LaTeX{})
or to generate a smaller document which is easier to navigate.
Another application of |\includeonly| is to generate
documents consisting of selected parts of the complete document.

However, there are a few drawbacks of the plain |\include| mechanism:
\begin{itemize}
\item
The child files cannot be compiled on their own,
they can only be compiled via the main file.
A naive editing environment
(such as a text editor with an option
to have the current file processed by \LaTeX)
may require one to switch to the main file before compiling;
attempting to compile the child file produces errors.
\item
The main file must be modified (each time)
to adjust the |\includeonly| command
to the present needs. This easily leaves the main file in a messy state.
\item
The generated document will always carry the filename
of the main document. This is inconvenient if
several child files are to be compiled and
to be kept for distribution.
\end{itemize}

The present package provides a simple interface
to make child files individually compilable by \LaTeX{}.
Compiling a child file then has the same effect as compiling
the main file with an |\includeonly| command
to select the appropriate child.
Moreover the generated document will carry the name of the child
rather than the main file.
This resolves all three above issues.

This feature is meant to make the editing of books,
thesis documents and lecture notes somewhat more convenient.
However, the package can also be used efficiently for
composing a series of documents (such as exercise sheets)
which are typically distributed individually.
It then assists the author in generating the individual documents
(potentially in different versions)
as well as a document containing the collected series.
Another application is in developing style files
or other kinds of included material
where compilation of the style file could redirect
to a sample or test file.

%%%%%%%%%%%%%%%%%%%%%%%%%%%%%%%%%%%%%%%%%%%%%%%%%%%%%%%%%%%%%%%%%%%%%%%%%%%%%%%%
%%%%%%%%%%%%%%%%%%%%%%%%%%%%%%%%%%%%%%%%%%%%%%%%%%%%%%%%%%%%%%%%%%%%%%%%%%%%%%%%
\section{Usage}

First of all, the package \textsf{childdoc} is \emph{not} a standard
\LaTeXe{} |.sty| style file! Therefore it needs to be invoked in
a non-standard way.

%%%%%%%%%%%%%%%%%%%%%%%%%%%%%%%%%%%%%%%%%%%%%%%%%%%%%%%%%%%%%%%%%%%%%%%%%%%%%%%%
\subsection{Included Files}
\label{sec:include}

%%%%%%%%%%%%%%%%%%%%%%%%%%%%%%%%%%%%%%%%
\DescribeMacro{\childdocmain}
To use the package, add the commands
\begin{center}
\begin{tabular}{l}
|\input{childdoc.def}|\\
|\childdocmain{}|\\
\end{tabular}
\end{center}
at the very top of the main \LaTeX{} file,
in particular \emph{before} the |\documentclass| statement!
The argument of |\childdocmain| should be left empty
(but it must be present).

%%%%%%%%%%%%%%%%%%%%%%%%%%%%%%%%%%%%%%%%
\DescribeMacro{\childdocof}
Furthermore, add the commands
\begin{center}
\begin{tabular}{l}
|\input{childdoc.def}|\\
|\childdocof{|\textit{main}|}|\\
\end{tabular}
\end{center}
at the top of every child file \textit{child}
which is included by |\include{|\textit{child}|}|
from within the main file
(or at least for those files to be compiled individually).
The argument \textit{main} must be the filename of the main file.

There are a couple of
considerations in setting up the main and child documents:

%%%%%%%%%%%%%%%%%%%%%%%%%%%%%%%%%%%%%%%%
\paragraph{Restrictions.}

Please note the following restrictions:
\begin{itemize}
\item
|\childdocmain| must be called with one argument \textit{main}
to ensure compatibility with earlier version of the package.
It must either be empty (|\childdocmain{}|)
or precisely match the filename of the main file in which it is specified.
See \secref{sec:detection} for further information.
\item
The filename \textit{main} must be specified without the |.tex| extension.
\item
The filename \textit{main} is case sensitive
(even in case-insensitive file systems)
due to internal string comparison.
\item
The argument \textit{main} should be fully expanded, it cannot be a macro.
\item
Subdirectories and special characters should be avoided in filenames.
\item
The command |\childdocmain{|\textit{main}|}| must be followed by a whitespace.
It should not be followed immediately by another command
or by a comment mark `|%|'.
This is because the \TeX{} parser reads the token immediately following
the argument of |\childdocmain| and puts it
at the beginning of every child section;
however, a white\-space is ignored.
\end{itemize}

%%%%%%%%%%%%%%%%%%%%%%%%%%%%%%%%%%%%%%%%
\paragraph{Content of Main File.}

It is advisable to place all content in the child files included by |\include|.
Any output contained in the main file will appear in all child documents
unless suppressed manually;
it cannot be suppressed automatically by the |\includeonly| directive
and thus should normally be avoided.
A method to include some content in the main file
by means of conditional processing is described in \secref{sec:conditional}.

%%%%%%%%%%%%%%%%%%%%%%%%%%%%%%%%%%%%%%%%
\paragraph{Page Numbering.}

When only a part of the document is compiled,
the appropriate numbering of pages
(as well as other status parameters)
is determined from the |.aux| files.
The latter contain information from previous passes.
However this information needs to propagate through
all intermediate child documents.
Therefore the page numbering in child documents may well
be inconsistent until the complete document is compiled at least once.

A useful (if unconventional) way to always ensure a consistent
page numbering is to restart the numbering in each child document
and denote the pages by `\textit{child}|.|\textit{page}'
where \textit{child} represents the chapter/section number of the child file.
This can be achieved by the command
|\numberwithin{page}{|\textit{child}|}|
of the \textsf{amsmath} package
where \textit{child} can be |chapter| or |section|
depending on the chosen structuring.
Alternatively, one can modify the macro |\thepage| appropriately
and reset the counter |page| at the start of each child file.

%%%%%%%%%%%%%%%%%%%%%%%%%%%%%%%%%%%%%%%%%%%%%%%%%%%%%%%%%%%%%%%%%%%%%%%%%%%%%%%%
\subsection{Conditional Processing}
\label{sec:conditional}

The package provides a mechanism to compile different versions
of a document. To customise the versions further some conditional processing
can come in handy to distinguish which version is being compiled.
The package provides two macros to describe the compilation context:

%%%%%%%%%%%%%%%%%%%%%%%%%%%%%%%%%%%%%%%%
\DescribeMacro{\ifchilddoc}
The conditional |\ifchilddoc| distinguishes between the compilation of
child documents and the main document:
%
\begin{center}
|\ifchilddoc |\textit{child-code}| |[|\||else |\textit{main-code}]| \||fi|
\end{center}

%%%%%%%%%%%%%%%%%%%%%%%%%%%%%%%%%%%%%%%%
\DescribeMacro{\childdocname}
\DescribeMacro{\childdocjob}
The macro |\childdocname| contains the filename (without extension)
of the main or child file being processed.
Note that |\childdocjob| will always contain the name of the main file.

%%%%%%%%%%%%%%%%%%%%%%%%%%%%%%%%%%%%%%%%
\paragraph{Title Page.}

Conditional processing can be used to include a title or banner page
in the main document when proper precautions are taken.
Importantly, the code in the main file should ensure that the page counter
(as well as other status parameters which are stored in the |.aux| files)
takes the same value after the conditional processing.
Otherwise the page numbers may take divergent values
depending on which part is compiled.

For example, a title page could be declared by:
%
\begin{center}
\begin{tabular}{l}
|\ifchilddoc\||else|\\
|\addtocounter{page}{-1}|\\
\textit{code for title page}\\
|\newpage|\\
|\||fi|
\end{tabular}
\end{center}
%
A banner page for the child documents can be generated by:
%
\begin{center}
\begin{tabular}{l}
|\ifchilddoc|\\
|\addtocounter{page}{-1}|\\
\textit{code for banner page}\\
|\newpage|\\
|\||fi|
\end{tabular}
\end{center}
%
Here one could write a message such as:
\begin{center}
|This is the part \childdocname{} of \childdocjob{}.|
\end{center}

%%%%%%%%%%%%%%%%%%%%%%%%%%%%%%%%%%%%%%%%%%%%%%%%%%%%%%%%%%%%%%%%%%%%%%%%%%%%%%%%
\subsection{Flags}
\label{sec:flags}

The package makes it easy to generate different versions
of the main or child documents.
To this end compilation flags can be defined
and assigned different default values.
They will be particularly useful in conjunction
with the forwarding mechanism described in \secref{sec:forward}.

For example, it may be useful to have a flag |\version|
which can be set to |draft| or |final|.
The document source will contain some conditional code
depending on the value of |\version|.
Suppose further, the flag should default to |final| for the main file
and to |draft| for child files
which is a natural assignment for editing the document.
This is achieved by placing the following code
in the preamble of the main document
(below the |\childdocmain| directive):
%
\begin{center}
\begin{tabular}{l}
|\ifchilddoc|\\
|\providecommand{\version}{draft}|\\
|\||else|\\
|\providecommand{\version}{final}|\\
|\||fi|
\end{tabular}
\end{center}
%
The definition by |\providecommand| makes sure
that previous definitions are not overwritten.
Further statements |\providecommand{\version}{...}|
can thus be added before the above code to override it.

For the main file, one might add a line
(between |\childdocmain| and the above block)
%
\begin{center}
|%\ifchilddoc\||else\providecommand{\version}{draft}\||fi|
\end{center}
%
which can be uncommented to produce a draft version.
Likewise one can add a line to the very top of a child file
(above the |\childdocof{|\textit{main}|}| directive)
%
\begin{center}
|%\providecommand{\version}{final}|
\end{center}
%
which can be uncommented to produce the final version of this child document.

%%%%%%%%%%%%%%%%%%%%%%%%%%%%%%%%%%%%%%%%%%%%%%%%%%%%%%%%%%%%%%%%%%%%%%%%%%%%%%%%
\subsection{Forwarding}
\label{sec:forward}

Different versions of the main or child documents
using compilation flags as described in \secref{sec:flags}
can be (permanently) stored in different files
for convenient compilation, viewing and distribution.
To this end, the package defines a command
to pass on compilation to a different file:

%%%%%%%%%%%%%%%%%%%%%%%%%%%%%%%%%%%%%%%%
\DescribeMacro{\childdocforward}
The command |\childdocforward| redirects processing to
another source file:
%
\begin{center}
\begin{tabular}{l}
|\input{childdoc.def}|\\
|\childdocforward[|\textit{main}|]{|\textit{dest}|}|\\
\end{tabular}
\end{center}
%
The argument \textit{dest} is the destination file
(without extension).
It should be the main file or one of the child files.
Note that further \textsf{childdoc} directives
such as |\childdocof| and |\childdocforward|
in the indicated file will be processed in this form.
The optional argument \textit{main}
passes on directly to the main file \textit{main}
while pretending to compile the child \textit{dest}.
This form behaves as if \textit{dest}
issues |\childdocof{|\textit{main}|}| right away,
and no further \textsf{childdoc} directives will be processed.

%%%%%%%%%%%%%%%%%%%%%%%%%%%%%%%%%%%%%%%%
\DescribeMacro{\...prefix}
In the alternative form |\childdocforwardprefix|,
%
\begin{center}
\begin{tabular}{l}
|\input{childdoc.def}|\\
|\childdocforwardprefix[|\textit{main}|]{|\textit{prefix}|}{|\textit{dest}|}|
\end{tabular}
\end{center}
%
the destination file is determined by a pattern
depending on the current file:
To make this work, the current file must be called
`{\textit{prefix}\hspace{0.2em}\textit{suffix}}'
with \textit{prefix} matching precisely the argument.
Processing is then passed on to the file
`{\textit{dest}\hspace{0.2em}\textit{suffix}}'.
Surely, the same effect is achieved by
directly specifying the
argument `{\textit{dest}\hspace{0.2em}\textit{suffix}}'
in the first form.
However, that requires to set up a different file
for each child. With the alternative form of the command
all these files can have exactly the same content
which simplifies setting them up and maintaining them.

For example, the following file |draft.tex|
with a compilation flag |\version| as described in \secref{sec:flags}
compiles the main document as a draft:
%
\begin{center}
\begin{tabular}{l}
|\def\version{draft}|\\
|\input{childdoc.def}|\\
|\childdocforward{|\textit{main}|}|
\end{tabular}
\end{center}
%
Likewise, the following files |final|\textit{nn}|.tex|
compile the final version of the child document
|child|\textit{nn}|.tex|:
%
\begin{center}
\begin{tabular}{l}
|\def\version{final}|\\
|\input{childdoc.def}|\\
|\childdocforwardprefix{final}{child}|
\end{tabular}
\end{center}
%

Note that when several versions of a main file and/or of each child file
are to be generated, it may be convenient to set up a |Makefile| or
shell script to automatise the process.

%%%%%%%%%%%%%%%%%%%%%%%%%%%%%%%%%%%%%%%%%%%%%%%%%%%%%%%%%%%%%%%%%%%%%%%%%%%%%%%%
\subsection{Command Line Processing}
\label{sec:commandline}

The effect of redirection files can also be achieved by invoking
the \LaTeX{} compiler with a more elaborate command line.
Most conveniently this should be done as part
of a shell script or a |Makefile|.

When using \textsf{childdoc} in the main file, the following
command lines effectively perform a redirection
(note that depending on the shell being used,
backslashes may have to be doubled: `|\|' $\to$ `|\\|'):
%
\begin{center}
|... -jobname "|\textit{target}|" |\\|"|[\textit{flags}]%
|\input{childdoc.def}\childdocforward[|\textit{main}|]{|\textit{dest}|}"|
\end{center}
%
Here \textit{target} is the name of the output file,
\textit{main} is the name of the main file
and \textit{dest} is the name of the main or child file to be processed
(all filenames without extensions).
The optional argument \textit{main} can be omitted
if \textit{main} matches \textit{dest}.
Optionally, compilation \textit{flags} can be defined via |\def| commands.
This command line makes the \TeX{} engine believe
it is compiling the file \textit{target}
whose content is specified as the latter parameter.
The provided code then forwards the processing to
\textit{main} or \textit{dest} as described in \secref{sec:forward}.

%%%%%%%%%%%%%%%%%%%%%%%%%%%%%%%%%%%%%%%%%%%%%%%%%%%%%%%%%%%%%%%%%%%%%%%%%%%%%%%%
\subsection{Include by Input}
\label{sec:input}

Including child documents by |\include| has some restrictions by design.
Most notably, the content of a child document always occupies
its own set of pages; pages cannot be shared between child documents.
Usually, this behaviour makes perfect sense
because each child document contain an essential part of the document.
However, in some situations it may be desirable to compose
a document from a collection of parts
without having mandatory page breaks between then.
For this case, the package
provides a mechanism to include parts
by |\input| which can also be processed individually.
However, by construction this mechanism
requires manual handling of the content to be output.

%%%%%%%%%%%%%%%%%%%%%%%%%%%%%%%%%%%%%%%%
\DescribeMacro{\ifchilddocmanual}
The main file should be prepared as usual, see \secref{sec:include}.
However, the document body must make a distinction
between processing of an individual part and of the main document, e.g.:
%
\begin{center}
\begin{tabular}{l}
|\ifchilddocmanual|\\
|\input{\childdocname}|\\
|\||else|\\
\textit{document body with }|\input{|\textit{part}|}|\\
|\||fi|
\end{tabular}
\end{center}
%
The conditional |\ifchilddocmanual| is true whenever
a part to be included by |\input| is being compiled,
and the name of the part is stored in |\childdocname|.

%%%%%%%%%%%%%%%%%%%%%%%%%%%%%%%%%%%%%%%%
\DescribeMacro{\childdocby}
Each part to be included by |\input| should start with:
%
\begin{center}
\begin{tabular}{l}
|\input{childdoc.def}|\\
|\childdocby{|\textit{main}|}|\\
\end{tabular}
\end{center}
%
The directive |\childdocby| is similar to |\childdocof|
described in \secref{sec:include},
but the subsequent selection of content must be done manually.
To that end, both |\ifchilddoc| and |\ifchilddocmanual|
will be true upon processing of a part,
and the name of the part is stored in |\childdocname|.
Note that |\jobname| will be set to the filename of the current part
so that each part receives an individual |.aux| file
that does not interfere with the |.aux| file(s) of the main document.
This behaviour can be altered by the alternative form
|\childdocby[*]{|\textit{main}|}| (with a non-empty optional argument)
which uses the |.aux| file of the main document
by setting |\jobname| to \textit{main}.

%%%%%%%%%%%%%%%%%%%%%%%%%%%%%%%%%%%%%%%%%%%%%%%%%%%%%%%%%%%%%%%%%%%%%%%%%%%%%%%%
\subsection{Driver Development}
\label{sec:driver}

The \textsf{childdoc} mechanism can also be use for the development
of definition files such as \LaTeX{} styles or classes.
This case differs from the above setup with multiple parts
included by |\include| in that no |\includeonly| should be invoked.
This can be achieved by starting the include file
(before |\ProvidesPackage|) with:
%
\begin{center}
\begin{tabular}{l}
|\input{childdoc.def}|\\
|\childdocforward{|\textit{main}|}|\\
\end{tabular}
\end{center}
%
or alternatively with:
%
\begin{center}
\begin{tabular}{l}
|\input{childdoc.def}|\\
|\childdocby{|\textit{main}|}|\\
\end{tabular}
\end{center}
%
Both forms have slightly different effects as described above.
The main file is prepared as usual, see \secref{sec:include}.

%%%%%%%%%%%%%%%%%%%%%%%%%%%%%%%%%%%%%%%%%%%%%%%%%%%%%%%%%%%%%%%%%%%%%%%%%%%%%%%%
\subsection{Legacy Detection}
\label{sec:detection}

The directive |\childdocmain| in the main file can detect
whether the complete document or merely a child is to be compiled
even without using the directive |\childdocof|.
This method is deprecated because it is less robust
and there is no compelling reason to use it;
it is merely provided for backward compatibility
and it may be removed in future versions.

If the detection mechanism is to be used,
it is mandatory to correctly specify
the filename of the main file as the argument of |\childdocmain|:
%
\begin{center}
\begin{tabular}{l}
|\input{childdoc.def}|\\
|\childdocmain{|\textit{main}|}|\\
\end{tabular}
\end{center}
%
If |\jobname| does not match the argument \textit{main} of |\childdocmain|,
it is assumed that |\jobname| points to the child file to be compiled.
When using |\childdocmain| with the main file specified as argument,
it suffices to start a child file
with just |\input{|\textit{main}|}|
without loading of the package and using |\childdocof|.
If instead all processing is done
with the appropriate \textsf{childdoc} directives,
the argument of \textit{main} of |\childdocmain| can be empty.

An alternative version of the command line processing described
in \secref{sec:commandline} using the detection mechanism reads:
%
\begin{center}
|... -jobname "|\textit{target}|" "|[\textit{flags}]%
[|\def\jobname{|\textit{dest}|}|]|\input{|\textit{main}|}"|
\end{center}

%%%%%%%%%%%%%%%%%%%%%%%%%%%%%%%%%%%%%%%%%%%%%%%%%%%%%%%%%%%%%%%%%%%%%%%%%%%%%%%%
\subsection{Manual Code}
\label{sec:manual}

In case one cannot be certain whether the definitions file |childdoc.def|
is installed on the target \TeX{} distribution
and one prefers not to ship it,
it is conceivable to paste a few relevant commands into the sources.

To that end, drop all statements |\input{childdoc.def}|
and perform the replacements as outlined below.
Instead of |\childdocmain{|\textit{main}|}| add the following code
to the top of the main file:
%
\begin{center}
\begin{tabular}{l}
|\||ifdefined\childdocname\endinput\||fi\newif\ifchilddoc|\\
|\edef\childdocname{\scantokens\expandafter{\jobname\noexpand}}|\\
|\def\childdocmain{|\textit{main}|}\||ifx\childdocmain\childdocname\||else|\\
|\childdoctrue\includeonly{\childdocname}\let\jobname\childdocmain\||fi|\\
\end{tabular}
\end{center}
%
Instead of |\childdocof{|\textit{main}|}| just include the main file
at the top of each child file:
%
\begin{center}
|\input{|\textit{main}|}|
\end{center}
%
A simple redirection |\childdocforward{|\textit{dest}|}| is achieved by:
%
\begin{center}
|\def\jobname{|\textit{dest}|}\input{\jobname}|
\end{center}
%
The redirection with prefix
|\childdocforwardprefix[|\textit{prefix}|]{|\textit{dest}|}|
is accomplished by:
%
\begin{center}
\begin{tabular}{l}
|{\edef\jobname{\scantokens\expandafter{\jobname\noexpand}}|\\
|\def\redirectjob |\textit{prefix}|#1~~~{\gdef\jobname{|\textit{dest}|#1}}|\\
|\expandafter\redirectjob\jobname~~~}\input{\jobname}|
\end{tabular}
\end{center}

In an alternative approach,
child documents can be compiled by a specific command line
without additional code or specific definitions:
%
\begin{center}
|... -jobname "|\textit{target}|" "|[\textit{flags}]%
|\includeonly{|\textit{dest}|}\input{|\textit{main}|}"|
\end{center}
%

%%%%%%%%%%%%%%%%%%%%%%%%%%%%%%%%%%%%%%%%%%%%%%%%%%%%%%%%%%%%%%%%%%%%%%%%%%%%%%%%
%%%%%%%%%%%%%%%%%%%%%%%%%%%%%%%%%%%%%%%%%%%%%%%%%%%%%%%%%%%%%%%%%%%%%%%%%%%%%%%%
\section{Information}

%%%%%%%%%%%%%%%%%%%%%%%%%%%%%%%%%%%%%%%%%%%%%%%%%%%%%%%%%%%%%%%%%%%%%%%%%%%%%%%%
\subsection{Copyright}

Copyright \copyright{} 2017--2018 Niklas Beisert

This work may be distributed and/or modified under the
conditions of the \LaTeX{} Project Public License, either version 1.3
of this license or (at your option) any later version.
The latest version of this license is in
  \url{http://www.latex-project.org/lppl.txt}
and version 1.3 or later is part of all distributions of \LaTeX{}
version 2005/12/01 or later.

This work has the LPPL maintenance status `maintained'.

The Current Maintainer of this work is Niklas Beisert.

This work consists of the files |README.txt|, |childdoc.ins| and |childdoc.dtx|
as well as the derived files |childdoc.def|, |cdocsamp.tex|
with |cdocsch1.tex|, |cdocsch2.tex|, |cdocspt3.tex|, |cdocspt4.tex|,
|cdocsdrf.tex|, |cdocsfn1.tex|, |cdocsfn2.tex|
as well as |childdoc.pdf|.

%%%%%%%%%%%%%%%%%%%%%%%%%%%%%%%%%%%%%%%%%%%%%%%%%%%%%%%%%%%%%%%%%%%%%%%%%%%%%%%%
\subsection{Files and Installation}

The package consists of the files:
%
\begin{center}
\begin{tabular}{ll}
    |README.txt|   & readme file \\
    |childdoc.ins| & installation file \\
    |childdoc.dtx| & source file \\
    |childdoc.def| & definition file \\
    |cdocsamp.tex| & sample main file \\
    |cdocsch1.tex| & sample include file \\
    |cdocsch2.tex| & sample include file \\
    |cdocspt3.tex| & sample part file \\
    |cdocspt4.tex| & sample part file \\
    |cdocsdrf.tex| & sample redirection file \\
    |cdocsfn1.tex| & sample redirection file \\
    |cdocsfn2.tex| & sample redirection file \\
    |childdoc.pdf| & manual
\end{tabular}
\end{center}
%
The distribution consists of the files
|README.txt|, |childdoc.ins| and |childdoc.dtx|.
%
\begin{itemize}
\item
Run (pdf)\LaTeX{} on |childdoc.dtx|
to compile the manual |childdoc.pdf| (this file).
\item
Run \LaTeX{} on |childdoc.ins| to create the definitions file |childdoc.def|
and the sample |cdocsamp.tex| with include files
|cdocsch1.tex|, |cdocsch2.tex|, |cdocspt3.tex|, |cdocspt4.tex|,
|cdocsdrf.tex|, |cdocsfn1.tex|, |cdocsfn2.tex|.
Then copy the file |childdoc.def| to an appropriate directory of your \LaTeX{}
distribution, e.g.\ \textit{texmf-root}|/tex/latex/childdoc|.
\end{itemize}

%%%%%%%%%%%%%%%%%%%%%%%%%%%%%%%%%%%%%%%%%%%%%%%%%%%%%%%%%%%%%%%%%%%%%%%%%%%%%%%%
\subsection{Related CTAN Packages}

There are several other packages which offer a similar functionality:
%
\begin{itemize}
\item
The packages
\href{http://ctan.org/pkg/docmute}{\textsf{docmute}},
\href{http://ctan.org/pkg/includex}{\textsf{includex}} and
\href{http://ctan.org/pkg/standalone}{\textsf{standalone}}
provide commands to include only the document body of
a child file thus allowing both files to be compiled individually.
\item
The packages \href{http://ctan.org/pkg/subdocs}{\textsf{subdocs}}
and \href{http://ctan.org/pkg/subfiles}{\textsf{subfiles}}
provide structures in which the main and child documents can be
encapsulated and allowing them to be compiled individually.
The inclusion mechanism is different from the conventional |\include|.
\item
The package \href{http://ctan.org/pkg/combine}{\textsf{combine}}
is an elaborate solution to combine several documents into one.
\end{itemize}
%
See also the CTAN topic \href{http://ctan.org/topic/subdocs}{\textsf{subdocs}}
for further related packages.
The present package differs from the above solutions in that
a document structure constructed with the conventional |\include| mechanism
just needs two extra commands at the top of every file
such that all constituent files can be compiled individually.

%%%%%%%%%%%%%%%%%%%%%%%%%%%%%%%%%%%%%%%%%%%%%%%%%%%%%%%%%%%%%%%%%%%%%%%%%%%%%%%%
%\subsection{Feature Suggestions}
%
%The following is a list of features which may be useful for future
%versions of this package:
%%
%\begin{itemize}
%\item
%\ldots
%\end{itemize}

%%%%%%%%%%%%%%%%%%%%%%%%%%%%%%%%%%%%%%%%%%%%%%%%%%%%%%%%%%%%%%%%%%%%%%%%%%%%%%%%
\subsection{Revision History}

%%%%%%%%%%%%%%%%%%%%%%%%%%%%%%%%%%%%%%%%
\paragraph{v2.0:} 2018/12/30

\begin{itemize}
\item
immediate forward processing
\item
added |\childdocby| mechanism
\item
manual restructured
\end{itemize}

%%%%%%%%%%%%%%%%%%%%%%%%%%%%%%%%%%%%%%%%
\paragraph{v1.6:} 2018/01/17

\begin{itemize}
\item
application for development of include files
\item
corrections to manual
\end{itemize}

%%%%%%%%%%%%%%%%%%%%%%%%%%%%%%%%%%%%%%%%
\paragraph{v1.5:} 2017/05/21

\begin{itemize}
\item
more complete structuring introduced
\item
|\childdocof| introduced
\item
|\childdoc| renamed to |\childdocmain|
\item
|\childredirect| renamed to |\childdocforward| and |\childdocforwardprefix|
and functionality expanded
\end{itemize}

%%%%%%%%%%%%%%%%%%%%%%%%%%%%%%%%%%%%%%%%
\paragraph{v1.0:} 2017/04/27

\begin{itemize}
\item
manual and install package
\item
first version published on CTAN
\end{itemize}

%%%%%%%%%%%%%%%%%%%%%%%%%%%%%%%%%%%%%%%%
\paragraph{v0.6:} 2017/04/26

\begin{itemize}
\item
redirection mechanism added
\end{itemize}

%%%%%%%%%%%%%%%%%%%%%%%%%%%%%%%%%%%%%%%%
\paragraph{v0.5:} 2017/04/26

\begin{itemize}
\item
functionality in definition file
\end{itemize}


%%%%%%%%%%%%%%%%%%%%%%%%%%%%%%%%%%%%%%%%%%%%%%%%%%%%%%%%%%%%%%%%%%%%%%%%%%%%%%%%
%%%%%%%%%%%%%%%%%%%%%%%%%%%%%%%%%%%%%%%%%%%%%%%%%%%%%%%%%%%%%%%%%%%%%%%%%%%%%%%%
%%%%%%%%%%%%%%%%%%%%%%%%%%%%%%%%%%%%%%%%%%%%%%%%%%%%%%%%%%%%%%%%%%%%%%%%%%%%%%%%
\appendix

\settowidth\MacroIndent{\rmfamily\scriptsize 000\ }

 \DocInput{childdoc.dtx}

\end{document}
%</driver>
% \fi
%
% %%%%%%%%%%%%%%%%%%%%%%%%%%%%%%%%%%%%%%%%%%%%%%%%%%%%%%%%%%%%%%%%%%%%%%%%%%%%%%
% %%%%%%%%%%%%%%%%%%%%%%%%%%%%%%%%%%%%%%%%%%%%%%%%%%%%%%%%%%%%%%%%%%%%%%%%%%%%%%
% \section{Sample}
%\iffalse
%<*samplemain>
%\fi
%
% The following presents a sample document
% with two chapters, two parts, a title page,
% a compile flag as well as three forwarding files to set the flag.
% It consists of eight |.tex| files:
% \begin{center}
% \begin{tabular}{ll}
% |cdocsamp.tex|&main file\\
% |cdocsch1.tex|&include file for chapter 1\\
% |cdocsch2.tex|&include file for chapter 2\\
% |cdocspt3.tex|&include file for part 3\\
% |cdocspt4.tex|&include file for part 4\\
% |cdocsdrf.tex|&forwarding file for main file in draft mode\\
% |cdocsfi1.tex|&forwarding file for final version of chapter 1\\
% |cdocsfi2.tex|&forwarding file for final version of chapter 2\\
% \end{tabular}
% \end{center}
% Each of the eight files can be compiled directly by the \LaTeX{} compiler.
%
% %%%%%%%%%%%%%%%%%%%%%%%%%%%%%%%%%%%%%%
% \paragraph{Main File.}
%
% The main file is called |cdocsamp.tex|.
%
% Load the \textsf{childdoc} definitions and
% declare the filename for the main document:
%    \begin{macrocode}
\input{childdoc.def}
\childdocmain{}
%    \end{macrocode}

% Optional override for |\version| flag:
%    \begin{macrocode}
%%\ifchilddoc\else\providecommand{\version}{draft}\fi
%    \end{macrocode}

% Define the default values for the |\version| flag
% (|final| for the main file and |draft| for childs):
%    \begin{macrocode}
\ifchilddoc
\providecommand{\version}{draft}
\else
\providecommand{\version}{final}
\fi
%    \end{macrocode}

% Load the standard document class:
%    \begin{macrocode}
\documentclass[12pt]{article}
%    \end{macrocode}

% Start the document body:
%    \begin{macrocode}
\begin{document}
%    \end{macrocode}

% Declare a title page.
% Print title, part of document being processed and version flag:
%    \begin{macrocode}
\addtocounter{page}{-1}
\begin{center}
{\LARGE\bfseries{}childdoc example\par}
\vspace{1cm}
\ifchilddoc
\ifchilddocmanual part\else chapter\fi:
`\childdocname' of `\childdocjob'\par
\else
main document: `\childdocjob'\par
\fi
version: \version\par
\end{center}
\newpage
%    \end{macrocode}

% Manually include selected file,
% otherwise process as usual:
%    \begin{macrocode}
\ifchilddocmanual
\section*{part `\childdocname'}
\input{\childdocname}
\else
%    \end{macrocode}

% Include the two chapters:
%    \begin{macrocode}
\include{cdocsch1}
\include{cdocsch2}
%    \end{macrocode}

% Include the two parts unless only chapters should be displayed:
%    \begin{macrocode}
\ifchilddoc\else
\section{part three}
\input{cdocspt3}
\section{part four}
\input{cdocspt4}
\fi
%    \end{macrocode}

% Process as usual until here:
%    \begin{macrocode}
\fi
%    \end{macrocode}

% End of document body:
%    \begin{macrocode}
\end{document}
%    \end{macrocode}
%\iffalse
%</samplemain>
%\fi
%
% %%%%%%%%%%%%%%%%%%%%%%%%%%%%%%%%%%%%%%
% \paragraph{Chapter Include Files.}
%
% The include files are called |cdocsch1.tex| and |cdocsch2.tex|.
%
%\iffalse
%<*samplechap1|samplechap2>
%\fi

% Optional override for |\version| flag:
%    \begin{macrocode}
%%\providecommand{\version}{final}
%    \end{macrocode}

% Include the main document:
%    \begin{macrocode}
\input{childdoc.def}
\childdocof{cdocsamp}
%    \end{macrocode}

%\iffalse
%</samplechap1|samplechap2>
%\fi
%
%\iffalse
%<*samplechap1>
%\fi
% Some text for chapter 1:
%    \begin{macrocode}
\section{one}
some text in chapter one
%    \end{macrocode}

%\iffalse
%</samplechap1>
%\fi
% Some text for chapter 2:
%\iffalse
%<*samplechap2>
%\fi
%    \begin{macrocode}
\section{two}
more text in chapter two
%    \end{macrocode}

%\iffalse
%</samplechap2>
%\fi
%
% %%%%%%%%%%%%%%%%%%%%%%%%%%%%%%%%%%%%%%
% \paragraph{Part Include Files.}
%
% The include files are called |cdocspt3.tex| and |cdocspt4.tex|.
%
%\iffalse
%<*samplepart3|samplepart4>
%\fi

% Optional override for |\version| flag:
%    \begin{macrocode}
%%\providecommand{\version}{final}
%    \end{macrocode}

% Include the main document:
%    \begin{macrocode}
\input{childdoc.def}
\childdocby{cdocsamp}
%    \end{macrocode}

%\iffalse
%</samplepart3|samplepart4>
%\fi
%
%\iffalse
%<*samplepart3>
%\fi
% Some text for part 3:
%    \begin{macrocode}
some text in part three
%    \end{macrocode}

%\iffalse
%</samplepart3>
%\fi
% Some text for part 4:
%\iffalse
%<*samplepart4>
%\fi
%    \begin{macrocode}
more text in part four
%    \end{macrocode}

%\iffalse
%</samplepart4>
%\fi
%
% %%%%%%%%%%%%%%%%%%%%%%%%%%%%%%%%%%%%%%
% \paragraph{Forwarding for a Complete Draft.}
%
% The following forwarding file |cdocsdrf.tex|
% compiles the main document in draft mode:
%\iffalse
%<*sampledraft>
%\fi
%    \begin{macrocode}
\def\version{draft}
\input{childdoc.def}
\childdocforward{cdocsamp}
%    \end{macrocode}

%\iffalse
%</sampledraft>
%\fi
%
% %%%%%%%%%%%%%%%%%%%%%%%%%%%%%%%%%%%%%%
% \paragraph{Forwarding for Final Version of the Chapters.}
%
% The following forwarding files |cdocsfn1.tex| and |cdocsfn2.tex|
% (with identical content)
% compile the final versions of the child documents
% |cdocsch1.tex| and |cdocsch2.tex|, respectively:
%\iffalse
%<*samplefinal>
%\fi
%    \begin{macrocode}
\def\version{final}
\input{childdoc.def}
\childdocforwardprefix[cdocsamp]{cdocsfn}{cdocsch}
%    \end{macrocode}

%\iffalse
%</samplefinal>
%\fi
%
% %%%%%%%%%%%%%%%%%%%%%%%%%%%%%%%%%%%%%%
% \paragraph{Command Line Processing.}
%
% The following three command lines generate the output files
% |cdocscld|, |cdocscl1| and |cdocscl2|
% which should be identical to
% |cdocsdrf|, |cdocsch1| and |cdocsfn2|, respectively:
% \begin{center}
% \begin{tabular}{l}
% |latex -jobname cdocscld \|\\
% |  "\def\version{draft}\input{childdoc.def}\childdocforward{cdocsamp}"|\\
% |latex -jobname cdocscl1 \|\\
% |  "\input{childdoc.def}\childdocforward[cdocsamp]{cdocsch1}"|\\
% |latex -jobname cdocscl2 \|\\
% |  "\def\version{final}\input{childdoc.def}\childdocforward{cdocsch2}"|
% \end{tabular}
% \end{center}
% Note that the trailing backslash on each first line
% merely continues the input to the second line
% (for convenient cut ant paste).
% Furthermore, the command |latex| can be replaced by any
% of its alternative versions such as |pdflatex|.
%
% %%%%%%%%%%%%%%%%%%%%%%%%%%%%%%%%%%%%%%%%%%%%%%%%%%%%%%%%%%%%%%%%%%%%%%%%%%%%%%
% %%%%%%%%%%%%%%%%%%%%%%%%%%%%%%%%%%%%%%%%%%%%%%%%%%%%%%%%%%%%%%%%%%%%%%%%%%%%%%
% \section{Implementation}
%\iffalse
%<*package>
%\fi
%
% This section describes the definitions file |childdoc.def|.

% The definitions cannot be loaded using |\usepackage| or |\RequirePackage|
% which has a mechanism to prevent loading a style file more than once.
% When loading the definitions by means of |\input|
% multiple instances have to be prevented manually:
%\iffalse
%This code needs to be before the `\ProvidesFile' directive
%which is defined at the beginning of this file.
%Therefore it is also placed there and commented out here.
%</package>
%<*discard>
%\fi
%    \begin{macrocode}
\ifdefined\childdocmain\endinput\fi
%    \end{macrocode}
%\iffalse
%</discard>
%<*package>
%\fi
%
% \macro{\ifchilddoc}
% \macro{\ifchilddocmanual}
% The conditional |\ifchilddoc| tells whether a
% child (true) or main (false) document is being compiled.
% The conditional |\ifchilddocmanual| tells whether
% the |\includeonly| mechanism is used (false) or
% the selection of child files must be performed manually (true).
% The definitions initialise to false:
%    \begin{macrocode}
\newif\ifchilddoc
\newif\ifchilddocmanual
%    \end{macrocode}

% \macro{\childdocname}
% \macro{\childdocjob}
% The macro |\childdocname| stores the name of the main document
% to be compiled. The macro |\childdocjob| stores the name of
% the document on which the \LaTeX{} compiler was originally invoked.
% The content of |\jobname| cannot be compared
% to filenames specified in the source due to different catcodes.
% The following code rescans |\jobname|, stores the result
% in |\childdocname| and saves a copy in |\childdocjob|:
%    \begin{macrocode}
\edef\childdocname{\scantokens\expandafter{\jobname\noexpand}}
\let\childdocjob\childdocname
%    \end{macrocode}

% \macro{\childdocdisable}
% The macro |\childdocdisable| prevents the main file
% from being processed more than once.
% At this stage, the main document command |\childdocmain|
% is assumed to be called once again where it should do nothing.
% Any subsequent call to it should prevent
% a secondary processing of the main document
% It overwrites the forwarding commands
% |\childdocof| and |\childdocforward|
% with empty macros to prevent further inclusions of the main document:
%    \begin{macrocode}
\newcommand{\childdocdisable}
{
  \renewcommand{\childdocmain}[1]{\renewcommand{\childdocmain}[1]{\endinput}}
  \renewcommand{\childdocof}[1]{}
  \renewcommand{\childdocby}[2][]{}
  \renewcommand{\childdocforward}[2][]{}
  \renewcommand{\childdocdisable}{}
}
%    \end{macrocode}

% \macro{\childdocmain}
% The macro |\childdocmain| is to be called at the top of the main file
% with nothing or the main filename (without extension) as argument.
% First, it breaks loops.
% If the argument is not empty and does not match |\childdocname|
% (which is set by the first inclusion of |childdoc.def|),
% |\ifchilddoc| is set to true, |\includeonly| is applied to the child file
% and |\jobname| is set to the main file
% (for proper handling of |.aux| files):
%    \begin{macrocode}
\newcommand{\childdocmain}[1]
{
  \childdocdisable\childdocmain{}
  \if?#1?\else
    \begingroup
      \def\childdoctmp{#1}
      \ifx\childdoctmp\childdocname
        \def\childdoctmp{}
      \else
        \def\childdoctmp
        {
          \childdoctrue
          \includeonly{\childdocname}
          \def\childdocjob{#1}
          \def\jobname{#1}
        }
      \fi
      \expandafter
    \endgroup
    \childdoctmp
  \fi
}
%    \end{macrocode}

% \macro{\childdocof}
% The command |\childdocof| redirects
% compilation to the main file |#1|.
%    \begin{macrocode}
\newcommand{\childdocof}[1]
{
  \childdocdisable
  \childdoctrue
  \includeonly{\childdocname}
  \def\jobname{#1}
  \def\childdocjob{#1}
  \input{#1}
}
%    \end{macrocode}

% \macro{\childdocby}
% The command |\childdocby| ....
%    \begin{macrocode}
\newcommand{\childdocby}[2][]
{
  \childdocdisable
  \childdoctrue
  \childdocmanualtrue
  \if?#1?\else
    \def\jobname{#2}
  \fi
  \def\childdocjob{#2}
  \input{#2}
  \endinput
}
%    \end{macrocode}

% \macro{\childdocforward}
% The command |\childdocforward| redirects
% compilation to the main file or
% (if the optional argument is given) a child file.
% Parameters are set as if the main file
% or a child file starting with |\childdocof| was compiled.
% Then compilation is handed over to the main file:
%    \begin{macrocode}
\newcommand{\childdocforward}[2][]
{
  \begingroup
    \if?#1?
      \def\childdoctmp
      {
        \def\childdocname{#2}
        \def\childdocjob{#2}
        \def\jobname{#2}
        \input{#2}
        \endinput
      }
    \else
      \def\childdoctmp
      {
        \childdocdisable
        \def\childdocname{#2}
        \childdoctrue
        \includeonly{#2}
        \def\childdocjob{#1}
        \def\jobname{#1}
        \input{#1}
        \endinput
      }
    \fi
    \expandafter
  \endgroup
  \childdoctmp
}
%    \end{macrocode}

% \macro{\childdocforwardprefix}
% The command |\childdocforwardprefix| redirects
% compilation to the main or a child file by means of a pattern.
% The prefix |#1| in the current filename is replaced by |#2|
% and the suffix of the current filename is kept
% (it is assumed that the filename does not contain the substring `|~~~|'
% which is used as a delimiter).
% Compilation is handed over to the new file by |\childdocforward|:
%    \begin{macrocode}
\newcommand{\childdocforwardprefix}[3][]
{
  \begingroup
    \def\childdocextract #2##1~~~{\def\childdoctmp{\childdocforward[#1]{#3##1}}}
    \expandafter\childdocextract\childdocname~~~
    \expandafter
  \endgroup
  \childdoctmp
}
%    \end{macrocode}

% \macro{\childdoc}
% The deprecated macro |\childdoc| is a legacy version of |\childdocmain|:
%    \begin{macrocode}
\newcommand{\childdoc}{\childdocmain}
%    \end{macrocode}

% \macro{\childdocredirect}
% The deprecated macro |\childdocredirect| is a legacy version
% of |\childdocforward| and |\childdocforwardprefix|:
%    \begin{macrocode}
\newcommand{\childdocredirect}[2][]
{
  \begingroup
    \if?#1?
      \def\childdoctmp{\childdocforward{#2}}
    \else
      \def\childdoctmp{\childdocforwardprefix{#1}{#2}}
    \fi
    \expandafter
  \endgroup
  \childdoctmp
}
%    \end{macrocode}

%\iffalse
%</package>
%\fi
%
\endinput
\childdocforward[cdocsamp]{cdocsch1}"|\\
% |latex -jobname cdocscl2 \|\\
% |  "\def\version{final}% \iffalse
%
% childdoc.dtx Copyright (C) 2017-2018 Niklas Beisert
%
% This work may be distributed and/or modified under the
% conditions of the LaTeX Project Public License, either version 1.3
% of this license or (at your option) any later version.
% The latest version of this license is in
%   http://www.latex-project.org/lppl.txt
% and version 1.3 or later is part of all distributions of LaTeX
% version 2005/12/01 or later.
%
% This work has the LPPL maintenance status `maintained'.
%
% The Current Maintainer of this work is Niklas Beisert.
%
% This work consists of the files childdoc.dtx and childdoc.ins
% and the derived files childdoc.def and cdocsamp.tex with
% cdocsch1.tex, cdocsch2.tex, cdocsdrf.tex, cdocsfn1.tex, cdocsfn2.tex.
%
%<package>\ifdefined\childdocmain\endinput\fi
%<package>\ProvidesFile{childdoc.def}[2018/12/30 v2.0 child document driver]
%<samplemain>\ProvidesFile{cdocsamp.tex}[2018/12/30 v2.0 sample for childdoc]
%<*driver>
%\ProvidesFile{childdoc.drv}[2018/12/30 v2.0 childdoc reference manual file]
\PassOptionsToClass{10pt,a4paper}{article}
\documentclass{ltxdoc}

\usepackage[margin=35mm]{geometry}
\usepackage{hyperref}
\usepackage{hyperxmp}
\usepackage[usenames]{color}

\hypersetup{colorlinks=true}
\hypersetup{pdfstartview=FitH}
\hypersetup{pdfpagemode=UseNone}
\hypersetup{pdfsource={}}
\hypersetup{pdflang={en-UK}}
\hypersetup{pdfcopyright={Copyright 2017-2018 Niklas Beisert.
  This work may be distributed and/or modified under the
  conditions of the LaTeX Project Public License, either version 1.3
  of this license or (at your option) any later version.}}
\hypersetup{pdflicenseurl={http://www.latex-project.org/lppl.txt}}
\hypersetup{pdfcontactaddress={ETH Zurich, ITP, HIT K,
  Wolfgang-Pauli-Strasse 27}}
\hypersetup{pdfcontactpostcode={8093}}
\hypersetup{pdfcontactcity={Zurich}}
\hypersetup{pdfcontactcountry={Switzerland}}
\hypersetup{pdfcontactemail={nbeisert@itp.phys.ethz.ch}}
\hypersetup{pdfcontacturl={http://people.phys.ethz.ch/\xmptilde nbeisert/}}

\newcommand{\secref}[1]{\hyperref[#1]{section \ref*{#1}}}

\parskip1ex
\parindent0pt
\let\olditemize\itemize
\def\itemize{\olditemize\parskip0pt}

\begin{document}

\title{The \textsf{childdoc} Package}
\hypersetup{pdftitle={The childdoc Package}}
\author{Niklas Beisert\\[2ex]
  Institut f\"ur Theoretische Physik\\
  Eidgen\"ossische Technische Hochschule Z\"urich\\
  Wolfgang-Pauli-Strasse 27, 8093 Z\"urich, Switzerland\\[1ex]
  \href{mailto:nbeisert@itp.phys.ethz.ch}
  {\texttt{nbeisert@itp.phys.ethz.ch}}}
\hypersetup{pdfauthor={Niklas Beisert}}
\hypersetup{pdfsubject={Manual for the LaTeX2e Package childdoc}}
\date{30 December 2018, \textsf{v2.0}}
\maketitle

\begin{abstract}\noindent
\textsf{childdoc} is a \LaTeXe{} package
that enables the direct compilation
of document sections included by |\include|
to individual files.
\end{abstract}

\begingroup
\parskip0ex
\tableofcontents
\endgroup

%%%%%%%%%%%%%%%%%%%%%%%%%%%%%%%%%%%%%%%%%%%%%%%%%%%%%%%%%%%%%%%%%%%%%%%%%%%%%%%%
%%%%%%%%%%%%%%%%%%%%%%%%%%%%%%%%%%%%%%%%%%%%%%%%%%%%%%%%%%%%%%%%%%%%%%%%%%%%%%%%
\section{Introduction}

\LaTeX{} provides a mechanism to structure a large document (such as a book)
into a main file and several child files (containing the chapters)
using the |\include| command.
This mechanism is beneficial for documents
which span hundreds of pages in order to
make the source file(s) more manageable.
Moreover, compilation can be restricted to
selected child files by means of the |\includeonly| command.
The latter feature can be used to reduce the compilation time while editing
(this was significantly more useful in the earlier days of \LaTeX{})
or to generate a smaller document which is easier to navigate.
Another application of |\includeonly| is to generate
documents consisting of selected parts of the complete document.

However, there are a few drawbacks of the plain |\include| mechanism:
\begin{itemize}
\item
The child files cannot be compiled on their own,
they can only be compiled via the main file.
A naive editing environment
(such as a text editor with an option
to have the current file processed by \LaTeX)
may require one to switch to the main file before compiling;
attempting to compile the child file produces errors.
\item
The main file must be modified (each time)
to adjust the |\includeonly| command
to the present needs. This easily leaves the main file in a messy state.
\item
The generated document will always carry the filename
of the main document. This is inconvenient if
several child files are to be compiled and
to be kept for distribution.
\end{itemize}

The present package provides a simple interface
to make child files individually compilable by \LaTeX{}.
Compiling a child file then has the same effect as compiling
the main file with an |\includeonly| command
to select the appropriate child.
Moreover the generated document will carry the name of the child
rather than the main file.
This resolves all three above issues.

This feature is meant to make the editing of books,
thesis documents and lecture notes somewhat more convenient.
However, the package can also be used efficiently for
composing a series of documents (such as exercise sheets)
which are typically distributed individually.
It then assists the author in generating the individual documents
(potentially in different versions)
as well as a document containing the collected series.
Another application is in developing style files
or other kinds of included material
where compilation of the style file could redirect
to a sample or test file.

%%%%%%%%%%%%%%%%%%%%%%%%%%%%%%%%%%%%%%%%%%%%%%%%%%%%%%%%%%%%%%%%%%%%%%%%%%%%%%%%
%%%%%%%%%%%%%%%%%%%%%%%%%%%%%%%%%%%%%%%%%%%%%%%%%%%%%%%%%%%%%%%%%%%%%%%%%%%%%%%%
\section{Usage}

First of all, the package \textsf{childdoc} is \emph{not} a standard
\LaTeXe{} |.sty| style file! Therefore it needs to be invoked in
a non-standard way.

%%%%%%%%%%%%%%%%%%%%%%%%%%%%%%%%%%%%%%%%%%%%%%%%%%%%%%%%%%%%%%%%%%%%%%%%%%%%%%%%
\subsection{Included Files}
\label{sec:include}

%%%%%%%%%%%%%%%%%%%%%%%%%%%%%%%%%%%%%%%%
\DescribeMacro{\childdocmain}
To use the package, add the commands
\begin{center}
\begin{tabular}{l}
|\input{childdoc.def}|\\
|\childdocmain{}|\\
\end{tabular}
\end{center}
at the very top of the main \LaTeX{} file,
in particular \emph{before} the |\documentclass| statement!
The argument of |\childdocmain| should be left empty
(but it must be present).

%%%%%%%%%%%%%%%%%%%%%%%%%%%%%%%%%%%%%%%%
\DescribeMacro{\childdocof}
Furthermore, add the commands
\begin{center}
\begin{tabular}{l}
|\input{childdoc.def}|\\
|\childdocof{|\textit{main}|}|\\
\end{tabular}
\end{center}
at the top of every child file \textit{child}
which is included by |\include{|\textit{child}|}|
from within the main file
(or at least for those files to be compiled individually).
The argument \textit{main} must be the filename of the main file.

There are a couple of
considerations in setting up the main and child documents:

%%%%%%%%%%%%%%%%%%%%%%%%%%%%%%%%%%%%%%%%
\paragraph{Restrictions.}

Please note the following restrictions:
\begin{itemize}
\item
|\childdocmain| must be called with one argument \textit{main}
to ensure compatibility with earlier version of the package.
It must either be empty (|\childdocmain{}|)
or precisely match the filename of the main file in which it is specified.
See \secref{sec:detection} for further information.
\item
The filename \textit{main} must be specified without the |.tex| extension.
\item
The filename \textit{main} is case sensitive
(even in case-insensitive file systems)
due to internal string comparison.
\item
The argument \textit{main} should be fully expanded, it cannot be a macro.
\item
Subdirectories and special characters should be avoided in filenames.
\item
The command |\childdocmain{|\textit{main}|}| must be followed by a whitespace.
It should not be followed immediately by another command
or by a comment mark `|%|'.
This is because the \TeX{} parser reads the token immediately following
the argument of |\childdocmain| and puts it
at the beginning of every child section;
however, a white\-space is ignored.
\end{itemize}

%%%%%%%%%%%%%%%%%%%%%%%%%%%%%%%%%%%%%%%%
\paragraph{Content of Main File.}

It is advisable to place all content in the child files included by |\include|.
Any output contained in the main file will appear in all child documents
unless suppressed manually;
it cannot be suppressed automatically by the |\includeonly| directive
and thus should normally be avoided.
A method to include some content in the main file
by means of conditional processing is described in \secref{sec:conditional}.

%%%%%%%%%%%%%%%%%%%%%%%%%%%%%%%%%%%%%%%%
\paragraph{Page Numbering.}

When only a part of the document is compiled,
the appropriate numbering of pages
(as well as other status parameters)
is determined from the |.aux| files.
The latter contain information from previous passes.
However this information needs to propagate through
all intermediate child documents.
Therefore the page numbering in child documents may well
be inconsistent until the complete document is compiled at least once.

A useful (if unconventional) way to always ensure a consistent
page numbering is to restart the numbering in each child document
and denote the pages by `\textit{child}|.|\textit{page}'
where \textit{child} represents the chapter/section number of the child file.
This can be achieved by the command
|\numberwithin{page}{|\textit{child}|}|
of the \textsf{amsmath} package
where \textit{child} can be |chapter| or |section|
depending on the chosen structuring.
Alternatively, one can modify the macro |\thepage| appropriately
and reset the counter |page| at the start of each child file.

%%%%%%%%%%%%%%%%%%%%%%%%%%%%%%%%%%%%%%%%%%%%%%%%%%%%%%%%%%%%%%%%%%%%%%%%%%%%%%%%
\subsection{Conditional Processing}
\label{sec:conditional}

The package provides a mechanism to compile different versions
of a document. To customise the versions further some conditional processing
can come in handy to distinguish which version is being compiled.
The package provides two macros to describe the compilation context:

%%%%%%%%%%%%%%%%%%%%%%%%%%%%%%%%%%%%%%%%
\DescribeMacro{\ifchilddoc}
The conditional |\ifchilddoc| distinguishes between the compilation of
child documents and the main document:
%
\begin{center}
|\ifchilddoc |\textit{child-code}| |[|\||else |\textit{main-code}]| \||fi|
\end{center}

%%%%%%%%%%%%%%%%%%%%%%%%%%%%%%%%%%%%%%%%
\DescribeMacro{\childdocname}
\DescribeMacro{\childdocjob}
The macro |\childdocname| contains the filename (without extension)
of the main or child file being processed.
Note that |\childdocjob| will always contain the name of the main file.

%%%%%%%%%%%%%%%%%%%%%%%%%%%%%%%%%%%%%%%%
\paragraph{Title Page.}

Conditional processing can be used to include a title or banner page
in the main document when proper precautions are taken.
Importantly, the code in the main file should ensure that the page counter
(as well as other status parameters which are stored in the |.aux| files)
takes the same value after the conditional processing.
Otherwise the page numbers may take divergent values
depending on which part is compiled.

For example, a title page could be declared by:
%
\begin{center}
\begin{tabular}{l}
|\ifchilddoc\||else|\\
|\addtocounter{page}{-1}|\\
\textit{code for title page}\\
|\newpage|\\
|\||fi|
\end{tabular}
\end{center}
%
A banner page for the child documents can be generated by:
%
\begin{center}
\begin{tabular}{l}
|\ifchilddoc|\\
|\addtocounter{page}{-1}|\\
\textit{code for banner page}\\
|\newpage|\\
|\||fi|
\end{tabular}
\end{center}
%
Here one could write a message such as:
\begin{center}
|This is the part \childdocname{} of \childdocjob{}.|
\end{center}

%%%%%%%%%%%%%%%%%%%%%%%%%%%%%%%%%%%%%%%%%%%%%%%%%%%%%%%%%%%%%%%%%%%%%%%%%%%%%%%%
\subsection{Flags}
\label{sec:flags}

The package makes it easy to generate different versions
of the main or child documents.
To this end compilation flags can be defined
and assigned different default values.
They will be particularly useful in conjunction
with the forwarding mechanism described in \secref{sec:forward}.

For example, it may be useful to have a flag |\version|
which can be set to |draft| or |final|.
The document source will contain some conditional code
depending on the value of |\version|.
Suppose further, the flag should default to |final| for the main file
and to |draft| for child files
which is a natural assignment for editing the document.
This is achieved by placing the following code
in the preamble of the main document
(below the |\childdocmain| directive):
%
\begin{center}
\begin{tabular}{l}
|\ifchilddoc|\\
|\providecommand{\version}{draft}|\\
|\||else|\\
|\providecommand{\version}{final}|\\
|\||fi|
\end{tabular}
\end{center}
%
The definition by |\providecommand| makes sure
that previous definitions are not overwritten.
Further statements |\providecommand{\version}{...}|
can thus be added before the above code to override it.

For the main file, one might add a line
(between |\childdocmain| and the above block)
%
\begin{center}
|%\ifchilddoc\||else\providecommand{\version}{draft}\||fi|
\end{center}
%
which can be uncommented to produce a draft version.
Likewise one can add a line to the very top of a child file
(above the |\childdocof{|\textit{main}|}| directive)
%
\begin{center}
|%\providecommand{\version}{final}|
\end{center}
%
which can be uncommented to produce the final version of this child document.

%%%%%%%%%%%%%%%%%%%%%%%%%%%%%%%%%%%%%%%%%%%%%%%%%%%%%%%%%%%%%%%%%%%%%%%%%%%%%%%%
\subsection{Forwarding}
\label{sec:forward}

Different versions of the main or child documents
using compilation flags as described in \secref{sec:flags}
can be (permanently) stored in different files
for convenient compilation, viewing and distribution.
To this end, the package defines a command
to pass on compilation to a different file:

%%%%%%%%%%%%%%%%%%%%%%%%%%%%%%%%%%%%%%%%
\DescribeMacro{\childdocforward}
The command |\childdocforward| redirects processing to
another source file:
%
\begin{center}
\begin{tabular}{l}
|\input{childdoc.def}|\\
|\childdocforward[|\textit{main}|]{|\textit{dest}|}|\\
\end{tabular}
\end{center}
%
The argument \textit{dest} is the destination file
(without extension).
It should be the main file or one of the child files.
Note that further \textsf{childdoc} directives
such as |\childdocof| and |\childdocforward|
in the indicated file will be processed in this form.
The optional argument \textit{main}
passes on directly to the main file \textit{main}
while pretending to compile the child \textit{dest}.
This form behaves as if \textit{dest}
issues |\childdocof{|\textit{main}|}| right away,
and no further \textsf{childdoc} directives will be processed.

%%%%%%%%%%%%%%%%%%%%%%%%%%%%%%%%%%%%%%%%
\DescribeMacro{\...prefix}
In the alternative form |\childdocforwardprefix|,
%
\begin{center}
\begin{tabular}{l}
|\input{childdoc.def}|\\
|\childdocforwardprefix[|\textit{main}|]{|\textit{prefix}|}{|\textit{dest}|}|
\end{tabular}
\end{center}
%
the destination file is determined by a pattern
depending on the current file:
To make this work, the current file must be called
`{\textit{prefix}\hspace{0.2em}\textit{suffix}}'
with \textit{prefix} matching precisely the argument.
Processing is then passed on to the file
`{\textit{dest}\hspace{0.2em}\textit{suffix}}'.
Surely, the same effect is achieved by
directly specifying the
argument `{\textit{dest}\hspace{0.2em}\textit{suffix}}'
in the first form.
However, that requires to set up a different file
for each child. With the alternative form of the command
all these files can have exactly the same content
which simplifies setting them up and maintaining them.

For example, the following file |draft.tex|
with a compilation flag |\version| as described in \secref{sec:flags}
compiles the main document as a draft:
%
\begin{center}
\begin{tabular}{l}
|\def\version{draft}|\\
|\input{childdoc.def}|\\
|\childdocforward{|\textit{main}|}|
\end{tabular}
\end{center}
%
Likewise, the following files |final|\textit{nn}|.tex|
compile the final version of the child document
|child|\textit{nn}|.tex|:
%
\begin{center}
\begin{tabular}{l}
|\def\version{final}|\\
|\input{childdoc.def}|\\
|\childdocforwardprefix{final}{child}|
\end{tabular}
\end{center}
%

Note that when several versions of a main file and/or of each child file
are to be generated, it may be convenient to set up a |Makefile| or
shell script to automatise the process.

%%%%%%%%%%%%%%%%%%%%%%%%%%%%%%%%%%%%%%%%%%%%%%%%%%%%%%%%%%%%%%%%%%%%%%%%%%%%%%%%
\subsection{Command Line Processing}
\label{sec:commandline}

The effect of redirection files can also be achieved by invoking
the \LaTeX{} compiler with a more elaborate command line.
Most conveniently this should be done as part
of a shell script or a |Makefile|.

When using \textsf{childdoc} in the main file, the following
command lines effectively perform a redirection
(note that depending on the shell being used,
backslashes may have to be doubled: `|\|' $\to$ `|\\|'):
%
\begin{center}
|... -jobname "|\textit{target}|" |\\|"|[\textit{flags}]%
|\input{childdoc.def}\childdocforward[|\textit{main}|]{|\textit{dest}|}"|
\end{center}
%
Here \textit{target} is the name of the output file,
\textit{main} is the name of the main file
and \textit{dest} is the name of the main or child file to be processed
(all filenames without extensions).
The optional argument \textit{main} can be omitted
if \textit{main} matches \textit{dest}.
Optionally, compilation \textit{flags} can be defined via |\def| commands.
This command line makes the \TeX{} engine believe
it is compiling the file \textit{target}
whose content is specified as the latter parameter.
The provided code then forwards the processing to
\textit{main} or \textit{dest} as described in \secref{sec:forward}.

%%%%%%%%%%%%%%%%%%%%%%%%%%%%%%%%%%%%%%%%%%%%%%%%%%%%%%%%%%%%%%%%%%%%%%%%%%%%%%%%
\subsection{Include by Input}
\label{sec:input}

Including child documents by |\include| has some restrictions by design.
Most notably, the content of a child document always occupies
its own set of pages; pages cannot be shared between child documents.
Usually, this behaviour makes perfect sense
because each child document contain an essential part of the document.
However, in some situations it may be desirable to compose
a document from a collection of parts
without having mandatory page breaks between then.
For this case, the package
provides a mechanism to include parts
by |\input| which can also be processed individually.
However, by construction this mechanism
requires manual handling of the content to be output.

%%%%%%%%%%%%%%%%%%%%%%%%%%%%%%%%%%%%%%%%
\DescribeMacro{\ifchilddocmanual}
The main file should be prepared as usual, see \secref{sec:include}.
However, the document body must make a distinction
between processing of an individual part and of the main document, e.g.:
%
\begin{center}
\begin{tabular}{l}
|\ifchilddocmanual|\\
|\input{\childdocname}|\\
|\||else|\\
\textit{document body with }|\input{|\textit{part}|}|\\
|\||fi|
\end{tabular}
\end{center}
%
The conditional |\ifchilddocmanual| is true whenever
a part to be included by |\input| is being compiled,
and the name of the part is stored in |\childdocname|.

%%%%%%%%%%%%%%%%%%%%%%%%%%%%%%%%%%%%%%%%
\DescribeMacro{\childdocby}
Each part to be included by |\input| should start with:
%
\begin{center}
\begin{tabular}{l}
|\input{childdoc.def}|\\
|\childdocby{|\textit{main}|}|\\
\end{tabular}
\end{center}
%
The directive |\childdocby| is similar to |\childdocof|
described in \secref{sec:include},
but the subsequent selection of content must be done manually.
To that end, both |\ifchilddoc| and |\ifchilddocmanual|
will be true upon processing of a part,
and the name of the part is stored in |\childdocname|.
Note that |\jobname| will be set to the filename of the current part
so that each part receives an individual |.aux| file
that does not interfere with the |.aux| file(s) of the main document.
This behaviour can be altered by the alternative form
|\childdocby[*]{|\textit{main}|}| (with a non-empty optional argument)
which uses the |.aux| file of the main document
by setting |\jobname| to \textit{main}.

%%%%%%%%%%%%%%%%%%%%%%%%%%%%%%%%%%%%%%%%%%%%%%%%%%%%%%%%%%%%%%%%%%%%%%%%%%%%%%%%
\subsection{Driver Development}
\label{sec:driver}

The \textsf{childdoc} mechanism can also be use for the development
of definition files such as \LaTeX{} styles or classes.
This case differs from the above setup with multiple parts
included by |\include| in that no |\includeonly| should be invoked.
This can be achieved by starting the include file
(before |\ProvidesPackage|) with:
%
\begin{center}
\begin{tabular}{l}
|\input{childdoc.def}|\\
|\childdocforward{|\textit{main}|}|\\
\end{tabular}
\end{center}
%
or alternatively with:
%
\begin{center}
\begin{tabular}{l}
|\input{childdoc.def}|\\
|\childdocby{|\textit{main}|}|\\
\end{tabular}
\end{center}
%
Both forms have slightly different effects as described above.
The main file is prepared as usual, see \secref{sec:include}.

%%%%%%%%%%%%%%%%%%%%%%%%%%%%%%%%%%%%%%%%%%%%%%%%%%%%%%%%%%%%%%%%%%%%%%%%%%%%%%%%
\subsection{Legacy Detection}
\label{sec:detection}

The directive |\childdocmain| in the main file can detect
whether the complete document or merely a child is to be compiled
even without using the directive |\childdocof|.
This method is deprecated because it is less robust
and there is no compelling reason to use it;
it is merely provided for backward compatibility
and it may be removed in future versions.

If the detection mechanism is to be used,
it is mandatory to correctly specify
the filename of the main file as the argument of |\childdocmain|:
%
\begin{center}
\begin{tabular}{l}
|\input{childdoc.def}|\\
|\childdocmain{|\textit{main}|}|\\
\end{tabular}
\end{center}
%
If |\jobname| does not match the argument \textit{main} of |\childdocmain|,
it is assumed that |\jobname| points to the child file to be compiled.
When using |\childdocmain| with the main file specified as argument,
it suffices to start a child file
with just |\input{|\textit{main}|}|
without loading of the package and using |\childdocof|.
If instead all processing is done
with the appropriate \textsf{childdoc} directives,
the argument of \textit{main} of |\childdocmain| can be empty.

An alternative version of the command line processing described
in \secref{sec:commandline} using the detection mechanism reads:
%
\begin{center}
|... -jobname "|\textit{target}|" "|[\textit{flags}]%
[|\def\jobname{|\textit{dest}|}|]|\input{|\textit{main}|}"|
\end{center}

%%%%%%%%%%%%%%%%%%%%%%%%%%%%%%%%%%%%%%%%%%%%%%%%%%%%%%%%%%%%%%%%%%%%%%%%%%%%%%%%
\subsection{Manual Code}
\label{sec:manual}

In case one cannot be certain whether the definitions file |childdoc.def|
is installed on the target \TeX{} distribution
and one prefers not to ship it,
it is conceivable to paste a few relevant commands into the sources.

To that end, drop all statements |\input{childdoc.def}|
and perform the replacements as outlined below.
Instead of |\childdocmain{|\textit{main}|}| add the following code
to the top of the main file:
%
\begin{center}
\begin{tabular}{l}
|\||ifdefined\childdocname\endinput\||fi\newif\ifchilddoc|\\
|\edef\childdocname{\scantokens\expandafter{\jobname\noexpand}}|\\
|\def\childdocmain{|\textit{main}|}\||ifx\childdocmain\childdocname\||else|\\
|\childdoctrue\includeonly{\childdocname}\let\jobname\childdocmain\||fi|\\
\end{tabular}
\end{center}
%
Instead of |\childdocof{|\textit{main}|}| just include the main file
at the top of each child file:
%
\begin{center}
|\input{|\textit{main}|}|
\end{center}
%
A simple redirection |\childdocforward{|\textit{dest}|}| is achieved by:
%
\begin{center}
|\def\jobname{|\textit{dest}|}\input{\jobname}|
\end{center}
%
The redirection with prefix
|\childdocforwardprefix[|\textit{prefix}|]{|\textit{dest}|}|
is accomplished by:
%
\begin{center}
\begin{tabular}{l}
|{\edef\jobname{\scantokens\expandafter{\jobname\noexpand}}|\\
|\def\redirectjob |\textit{prefix}|#1~~~{\gdef\jobname{|\textit{dest}|#1}}|\\
|\expandafter\redirectjob\jobname~~~}\input{\jobname}|
\end{tabular}
\end{center}

In an alternative approach,
child documents can be compiled by a specific command line
without additional code or specific definitions:
%
\begin{center}
|... -jobname "|\textit{target}|" "|[\textit{flags}]%
|\includeonly{|\textit{dest}|}\input{|\textit{main}|}"|
\end{center}
%

%%%%%%%%%%%%%%%%%%%%%%%%%%%%%%%%%%%%%%%%%%%%%%%%%%%%%%%%%%%%%%%%%%%%%%%%%%%%%%%%
%%%%%%%%%%%%%%%%%%%%%%%%%%%%%%%%%%%%%%%%%%%%%%%%%%%%%%%%%%%%%%%%%%%%%%%%%%%%%%%%
\section{Information}

%%%%%%%%%%%%%%%%%%%%%%%%%%%%%%%%%%%%%%%%%%%%%%%%%%%%%%%%%%%%%%%%%%%%%%%%%%%%%%%%
\subsection{Copyright}

Copyright \copyright{} 2017--2018 Niklas Beisert

This work may be distributed and/or modified under the
conditions of the \LaTeX{} Project Public License, either version 1.3
of this license or (at your option) any later version.
The latest version of this license is in
  \url{http://www.latex-project.org/lppl.txt}
and version 1.3 or later is part of all distributions of \LaTeX{}
version 2005/12/01 or later.

This work has the LPPL maintenance status `maintained'.

The Current Maintainer of this work is Niklas Beisert.

This work consists of the files |README.txt|, |childdoc.ins| and |childdoc.dtx|
as well as the derived files |childdoc.def|, |cdocsamp.tex|
with |cdocsch1.tex|, |cdocsch2.tex|, |cdocspt3.tex|, |cdocspt4.tex|,
|cdocsdrf.tex|, |cdocsfn1.tex|, |cdocsfn2.tex|
as well as |childdoc.pdf|.

%%%%%%%%%%%%%%%%%%%%%%%%%%%%%%%%%%%%%%%%%%%%%%%%%%%%%%%%%%%%%%%%%%%%%%%%%%%%%%%%
\subsection{Files and Installation}

The package consists of the files:
%
\begin{center}
\begin{tabular}{ll}
    |README.txt|   & readme file \\
    |childdoc.ins| & installation file \\
    |childdoc.dtx| & source file \\
    |childdoc.def| & definition file \\
    |cdocsamp.tex| & sample main file \\
    |cdocsch1.tex| & sample include file \\
    |cdocsch2.tex| & sample include file \\
    |cdocspt3.tex| & sample part file \\
    |cdocspt4.tex| & sample part file \\
    |cdocsdrf.tex| & sample redirection file \\
    |cdocsfn1.tex| & sample redirection file \\
    |cdocsfn2.tex| & sample redirection file \\
    |childdoc.pdf| & manual
\end{tabular}
\end{center}
%
The distribution consists of the files
|README.txt|, |childdoc.ins| and |childdoc.dtx|.
%
\begin{itemize}
\item
Run (pdf)\LaTeX{} on |childdoc.dtx|
to compile the manual |childdoc.pdf| (this file).
\item
Run \LaTeX{} on |childdoc.ins| to create the definitions file |childdoc.def|
and the sample |cdocsamp.tex| with include files
|cdocsch1.tex|, |cdocsch2.tex|, |cdocspt3.tex|, |cdocspt4.tex|,
|cdocsdrf.tex|, |cdocsfn1.tex|, |cdocsfn2.tex|.
Then copy the file |childdoc.def| to an appropriate directory of your \LaTeX{}
distribution, e.g.\ \textit{texmf-root}|/tex/latex/childdoc|.
\end{itemize}

%%%%%%%%%%%%%%%%%%%%%%%%%%%%%%%%%%%%%%%%%%%%%%%%%%%%%%%%%%%%%%%%%%%%%%%%%%%%%%%%
\subsection{Related CTAN Packages}

There are several other packages which offer a similar functionality:
%
\begin{itemize}
\item
The packages
\href{http://ctan.org/pkg/docmute}{\textsf{docmute}},
\href{http://ctan.org/pkg/includex}{\textsf{includex}} and
\href{http://ctan.org/pkg/standalone}{\textsf{standalone}}
provide commands to include only the document body of
a child file thus allowing both files to be compiled individually.
\item
The packages \href{http://ctan.org/pkg/subdocs}{\textsf{subdocs}}
and \href{http://ctan.org/pkg/subfiles}{\textsf{subfiles}}
provide structures in which the main and child documents can be
encapsulated and allowing them to be compiled individually.
The inclusion mechanism is different from the conventional |\include|.
\item
The package \href{http://ctan.org/pkg/combine}{\textsf{combine}}
is an elaborate solution to combine several documents into one.
\end{itemize}
%
See also the CTAN topic \href{http://ctan.org/topic/subdocs}{\textsf{subdocs}}
for further related packages.
The present package differs from the above solutions in that
a document structure constructed with the conventional |\include| mechanism
just needs two extra commands at the top of every file
such that all constituent files can be compiled individually.

%%%%%%%%%%%%%%%%%%%%%%%%%%%%%%%%%%%%%%%%%%%%%%%%%%%%%%%%%%%%%%%%%%%%%%%%%%%%%%%%
%\subsection{Feature Suggestions}
%
%The following is a list of features which may be useful for future
%versions of this package:
%%
%\begin{itemize}
%\item
%\ldots
%\end{itemize}

%%%%%%%%%%%%%%%%%%%%%%%%%%%%%%%%%%%%%%%%%%%%%%%%%%%%%%%%%%%%%%%%%%%%%%%%%%%%%%%%
\subsection{Revision History}

%%%%%%%%%%%%%%%%%%%%%%%%%%%%%%%%%%%%%%%%
\paragraph{v2.0:} 2018/12/30

\begin{itemize}
\item
immediate forward processing
\item
added |\childdocby| mechanism
\item
manual restructured
\end{itemize}

%%%%%%%%%%%%%%%%%%%%%%%%%%%%%%%%%%%%%%%%
\paragraph{v1.6:} 2018/01/17

\begin{itemize}
\item
application for development of include files
\item
corrections to manual
\end{itemize}

%%%%%%%%%%%%%%%%%%%%%%%%%%%%%%%%%%%%%%%%
\paragraph{v1.5:} 2017/05/21

\begin{itemize}
\item
more complete structuring introduced
\item
|\childdocof| introduced
\item
|\childdoc| renamed to |\childdocmain|
\item
|\childredirect| renamed to |\childdocforward| and |\childdocforwardprefix|
and functionality expanded
\end{itemize}

%%%%%%%%%%%%%%%%%%%%%%%%%%%%%%%%%%%%%%%%
\paragraph{v1.0:} 2017/04/27

\begin{itemize}
\item
manual and install package
\item
first version published on CTAN
\end{itemize}

%%%%%%%%%%%%%%%%%%%%%%%%%%%%%%%%%%%%%%%%
\paragraph{v0.6:} 2017/04/26

\begin{itemize}
\item
redirection mechanism added
\end{itemize}

%%%%%%%%%%%%%%%%%%%%%%%%%%%%%%%%%%%%%%%%
\paragraph{v0.5:} 2017/04/26

\begin{itemize}
\item
functionality in definition file
\end{itemize}


%%%%%%%%%%%%%%%%%%%%%%%%%%%%%%%%%%%%%%%%%%%%%%%%%%%%%%%%%%%%%%%%%%%%%%%%%%%%%%%%
%%%%%%%%%%%%%%%%%%%%%%%%%%%%%%%%%%%%%%%%%%%%%%%%%%%%%%%%%%%%%%%%%%%%%%%%%%%%%%%%
%%%%%%%%%%%%%%%%%%%%%%%%%%%%%%%%%%%%%%%%%%%%%%%%%%%%%%%%%%%%%%%%%%%%%%%%%%%%%%%%
\appendix

\settowidth\MacroIndent{\rmfamily\scriptsize 000\ }

 \DocInput{childdoc.dtx}

\end{document}
%</driver>
% \fi
%
% %%%%%%%%%%%%%%%%%%%%%%%%%%%%%%%%%%%%%%%%%%%%%%%%%%%%%%%%%%%%%%%%%%%%%%%%%%%%%%
% %%%%%%%%%%%%%%%%%%%%%%%%%%%%%%%%%%%%%%%%%%%%%%%%%%%%%%%%%%%%%%%%%%%%%%%%%%%%%%
% \section{Sample}
%\iffalse
%<*samplemain>
%\fi
%
% The following presents a sample document
% with two chapters, two parts, a title page,
% a compile flag as well as three forwarding files to set the flag.
% It consists of eight |.tex| files:
% \begin{center}
% \begin{tabular}{ll}
% |cdocsamp.tex|&main file\\
% |cdocsch1.tex|&include file for chapter 1\\
% |cdocsch2.tex|&include file for chapter 2\\
% |cdocspt3.tex|&include file for part 3\\
% |cdocspt4.tex|&include file for part 4\\
% |cdocsdrf.tex|&forwarding file for main file in draft mode\\
% |cdocsfi1.tex|&forwarding file for final version of chapter 1\\
% |cdocsfi2.tex|&forwarding file for final version of chapter 2\\
% \end{tabular}
% \end{center}
% Each of the eight files can be compiled directly by the \LaTeX{} compiler.
%
% %%%%%%%%%%%%%%%%%%%%%%%%%%%%%%%%%%%%%%
% \paragraph{Main File.}
%
% The main file is called |cdocsamp.tex|.
%
% Load the \textsf{childdoc} definitions and
% declare the filename for the main document:
%    \begin{macrocode}
\input{childdoc.def}
\childdocmain{}
%    \end{macrocode}

% Optional override for |\version| flag:
%    \begin{macrocode}
%%\ifchilddoc\else\providecommand{\version}{draft}\fi
%    \end{macrocode}

% Define the default values for the |\version| flag
% (|final| for the main file and |draft| for childs):
%    \begin{macrocode}
\ifchilddoc
\providecommand{\version}{draft}
\else
\providecommand{\version}{final}
\fi
%    \end{macrocode}

% Load the standard document class:
%    \begin{macrocode}
\documentclass[12pt]{article}
%    \end{macrocode}

% Start the document body:
%    \begin{macrocode}
\begin{document}
%    \end{macrocode}

% Declare a title page.
% Print title, part of document being processed and version flag:
%    \begin{macrocode}
\addtocounter{page}{-1}
\begin{center}
{\LARGE\bfseries{}childdoc example\par}
\vspace{1cm}
\ifchilddoc
\ifchilddocmanual part\else chapter\fi:
`\childdocname' of `\childdocjob'\par
\else
main document: `\childdocjob'\par
\fi
version: \version\par
\end{center}
\newpage
%    \end{macrocode}

% Manually include selected file,
% otherwise process as usual:
%    \begin{macrocode}
\ifchilddocmanual
\section*{part `\childdocname'}
\input{\childdocname}
\else
%    \end{macrocode}

% Include the two chapters:
%    \begin{macrocode}
\include{cdocsch1}
\include{cdocsch2}
%    \end{macrocode}

% Include the two parts unless only chapters should be displayed:
%    \begin{macrocode}
\ifchilddoc\else
\section{part three}
\input{cdocspt3}
\section{part four}
\input{cdocspt4}
\fi
%    \end{macrocode}

% Process as usual until here:
%    \begin{macrocode}
\fi
%    \end{macrocode}

% End of document body:
%    \begin{macrocode}
\end{document}
%    \end{macrocode}
%\iffalse
%</samplemain>
%\fi
%
% %%%%%%%%%%%%%%%%%%%%%%%%%%%%%%%%%%%%%%
% \paragraph{Chapter Include Files.}
%
% The include files are called |cdocsch1.tex| and |cdocsch2.tex|.
%
%\iffalse
%<*samplechap1|samplechap2>
%\fi

% Optional override for |\version| flag:
%    \begin{macrocode}
%%\providecommand{\version}{final}
%    \end{macrocode}

% Include the main document:
%    \begin{macrocode}
\input{childdoc.def}
\childdocof{cdocsamp}
%    \end{macrocode}

%\iffalse
%</samplechap1|samplechap2>
%\fi
%
%\iffalse
%<*samplechap1>
%\fi
% Some text for chapter 1:
%    \begin{macrocode}
\section{one}
some text in chapter one
%    \end{macrocode}

%\iffalse
%</samplechap1>
%\fi
% Some text for chapter 2:
%\iffalse
%<*samplechap2>
%\fi
%    \begin{macrocode}
\section{two}
more text in chapter two
%    \end{macrocode}

%\iffalse
%</samplechap2>
%\fi
%
% %%%%%%%%%%%%%%%%%%%%%%%%%%%%%%%%%%%%%%
% \paragraph{Part Include Files.}
%
% The include files are called |cdocspt3.tex| and |cdocspt4.tex|.
%
%\iffalse
%<*samplepart3|samplepart4>
%\fi

% Optional override for |\version| flag:
%    \begin{macrocode}
%%\providecommand{\version}{final}
%    \end{macrocode}

% Include the main document:
%    \begin{macrocode}
\input{childdoc.def}
\childdocby{cdocsamp}
%    \end{macrocode}

%\iffalse
%</samplepart3|samplepart4>
%\fi
%
%\iffalse
%<*samplepart3>
%\fi
% Some text for part 3:
%    \begin{macrocode}
some text in part three
%    \end{macrocode}

%\iffalse
%</samplepart3>
%\fi
% Some text for part 4:
%\iffalse
%<*samplepart4>
%\fi
%    \begin{macrocode}
more text in part four
%    \end{macrocode}

%\iffalse
%</samplepart4>
%\fi
%
% %%%%%%%%%%%%%%%%%%%%%%%%%%%%%%%%%%%%%%
% \paragraph{Forwarding for a Complete Draft.}
%
% The following forwarding file |cdocsdrf.tex|
% compiles the main document in draft mode:
%\iffalse
%<*sampledraft>
%\fi
%    \begin{macrocode}
\def\version{draft}
\input{childdoc.def}
\childdocforward{cdocsamp}
%    \end{macrocode}

%\iffalse
%</sampledraft>
%\fi
%
% %%%%%%%%%%%%%%%%%%%%%%%%%%%%%%%%%%%%%%
% \paragraph{Forwarding for Final Version of the Chapters.}
%
% The following forwarding files |cdocsfn1.tex| and |cdocsfn2.tex|
% (with identical content)
% compile the final versions of the child documents
% |cdocsch1.tex| and |cdocsch2.tex|, respectively:
%\iffalse
%<*samplefinal>
%\fi
%    \begin{macrocode}
\def\version{final}
\input{childdoc.def}
\childdocforwardprefix[cdocsamp]{cdocsfn}{cdocsch}
%    \end{macrocode}

%\iffalse
%</samplefinal>
%\fi
%
% %%%%%%%%%%%%%%%%%%%%%%%%%%%%%%%%%%%%%%
% \paragraph{Command Line Processing.}
%
% The following three command lines generate the output files
% |cdocscld|, |cdocscl1| and |cdocscl2|
% which should be identical to
% |cdocsdrf|, |cdocsch1| and |cdocsfn2|, respectively:
% \begin{center}
% \begin{tabular}{l}
% |latex -jobname cdocscld \|\\
% |  "\def\version{draft}\input{childdoc.def}\childdocforward{cdocsamp}"|\\
% |latex -jobname cdocscl1 \|\\
% |  "\input{childdoc.def}\childdocforward[cdocsamp]{cdocsch1}"|\\
% |latex -jobname cdocscl2 \|\\
% |  "\def\version{final}\input{childdoc.def}\childdocforward{cdocsch2}"|
% \end{tabular}
% \end{center}
% Note that the trailing backslash on each first line
% merely continues the input to the second line
% (for convenient cut ant paste).
% Furthermore, the command |latex| can be replaced by any
% of its alternative versions such as |pdflatex|.
%
% %%%%%%%%%%%%%%%%%%%%%%%%%%%%%%%%%%%%%%%%%%%%%%%%%%%%%%%%%%%%%%%%%%%%%%%%%%%%%%
% %%%%%%%%%%%%%%%%%%%%%%%%%%%%%%%%%%%%%%%%%%%%%%%%%%%%%%%%%%%%%%%%%%%%%%%%%%%%%%
% \section{Implementation}
%\iffalse
%<*package>
%\fi
%
% This section describes the definitions file |childdoc.def|.

% The definitions cannot be loaded using |\usepackage| or |\RequirePackage|
% which has a mechanism to prevent loading a style file more than once.
% When loading the definitions by means of |\input|
% multiple instances have to be prevented manually:
%\iffalse
%This code needs to be before the `\ProvidesFile' directive
%which is defined at the beginning of this file.
%Therefore it is also placed there and commented out here.
%</package>
%<*discard>
%\fi
%    \begin{macrocode}
\ifdefined\childdocmain\endinput\fi
%    \end{macrocode}
%\iffalse
%</discard>
%<*package>
%\fi
%
% \macro{\ifchilddoc}
% \macro{\ifchilddocmanual}
% The conditional |\ifchilddoc| tells whether a
% child (true) or main (false) document is being compiled.
% The conditional |\ifchilddocmanual| tells whether
% the |\includeonly| mechanism is used (false) or
% the selection of child files must be performed manually (true).
% The definitions initialise to false:
%    \begin{macrocode}
\newif\ifchilddoc
\newif\ifchilddocmanual
%    \end{macrocode}

% \macro{\childdocname}
% \macro{\childdocjob}
% The macro |\childdocname| stores the name of the main document
% to be compiled. The macro |\childdocjob| stores the name of
% the document on which the \LaTeX{} compiler was originally invoked.
% The content of |\jobname| cannot be compared
% to filenames specified in the source due to different catcodes.
% The following code rescans |\jobname|, stores the result
% in |\childdocname| and saves a copy in |\childdocjob|:
%    \begin{macrocode}
\edef\childdocname{\scantokens\expandafter{\jobname\noexpand}}
\let\childdocjob\childdocname
%    \end{macrocode}

% \macro{\childdocdisable}
% The macro |\childdocdisable| prevents the main file
% from being processed more than once.
% At this stage, the main document command |\childdocmain|
% is assumed to be called once again where it should do nothing.
% Any subsequent call to it should prevent
% a secondary processing of the main document
% It overwrites the forwarding commands
% |\childdocof| and |\childdocforward|
% with empty macros to prevent further inclusions of the main document:
%    \begin{macrocode}
\newcommand{\childdocdisable}
{
  \renewcommand{\childdocmain}[1]{\renewcommand{\childdocmain}[1]{\endinput}}
  \renewcommand{\childdocof}[1]{}
  \renewcommand{\childdocby}[2][]{}
  \renewcommand{\childdocforward}[2][]{}
  \renewcommand{\childdocdisable}{}
}
%    \end{macrocode}

% \macro{\childdocmain}
% The macro |\childdocmain| is to be called at the top of the main file
% with nothing or the main filename (without extension) as argument.
% First, it breaks loops.
% If the argument is not empty and does not match |\childdocname|
% (which is set by the first inclusion of |childdoc.def|),
% |\ifchilddoc| is set to true, |\includeonly| is applied to the child file
% and |\jobname| is set to the main file
% (for proper handling of |.aux| files):
%    \begin{macrocode}
\newcommand{\childdocmain}[1]
{
  \childdocdisable\childdocmain{}
  \if?#1?\else
    \begingroup
      \def\childdoctmp{#1}
      \ifx\childdoctmp\childdocname
        \def\childdoctmp{}
      \else
        \def\childdoctmp
        {
          \childdoctrue
          \includeonly{\childdocname}
          \def\childdocjob{#1}
          \def\jobname{#1}
        }
      \fi
      \expandafter
    \endgroup
    \childdoctmp
  \fi
}
%    \end{macrocode}

% \macro{\childdocof}
% The command |\childdocof| redirects
% compilation to the main file |#1|.
%    \begin{macrocode}
\newcommand{\childdocof}[1]
{
  \childdocdisable
  \childdoctrue
  \includeonly{\childdocname}
  \def\jobname{#1}
  \def\childdocjob{#1}
  \input{#1}
}
%    \end{macrocode}

% \macro{\childdocby}
% The command |\childdocby| ....
%    \begin{macrocode}
\newcommand{\childdocby}[2][]
{
  \childdocdisable
  \childdoctrue
  \childdocmanualtrue
  \if?#1?\else
    \def\jobname{#2}
  \fi
  \def\childdocjob{#2}
  \input{#2}
  \endinput
}
%    \end{macrocode}

% \macro{\childdocforward}
% The command |\childdocforward| redirects
% compilation to the main file or
% (if the optional argument is given) a child file.
% Parameters are set as if the main file
% or a child file starting with |\childdocof| was compiled.
% Then compilation is handed over to the main file:
%    \begin{macrocode}
\newcommand{\childdocforward}[2][]
{
  \begingroup
    \if?#1?
      \def\childdoctmp
      {
        \def\childdocname{#2}
        \def\childdocjob{#2}
        \def\jobname{#2}
        \input{#2}
        \endinput
      }
    \else
      \def\childdoctmp
      {
        \childdocdisable
        \def\childdocname{#2}
        \childdoctrue
        \includeonly{#2}
        \def\childdocjob{#1}
        \def\jobname{#1}
        \input{#1}
        \endinput
      }
    \fi
    \expandafter
  \endgroup
  \childdoctmp
}
%    \end{macrocode}

% \macro{\childdocforwardprefix}
% The command |\childdocforwardprefix| redirects
% compilation to the main or a child file by means of a pattern.
% The prefix |#1| in the current filename is replaced by |#2|
% and the suffix of the current filename is kept
% (it is assumed that the filename does not contain the substring `|~~~|'
% which is used as a delimiter).
% Compilation is handed over to the new file by |\childdocforward|:
%    \begin{macrocode}
\newcommand{\childdocforwardprefix}[3][]
{
  \begingroup
    \def\childdocextract #2##1~~~{\def\childdoctmp{\childdocforward[#1]{#3##1}}}
    \expandafter\childdocextract\childdocname~~~
    \expandafter
  \endgroup
  \childdoctmp
}
%    \end{macrocode}

% \macro{\childdoc}
% The deprecated macro |\childdoc| is a legacy version of |\childdocmain|:
%    \begin{macrocode}
\newcommand{\childdoc}{\childdocmain}
%    \end{macrocode}

% \macro{\childdocredirect}
% The deprecated macro |\childdocredirect| is a legacy version
% of |\childdocforward| and |\childdocforwardprefix|:
%    \begin{macrocode}
\newcommand{\childdocredirect}[2][]
{
  \begingroup
    \if?#1?
      \def\childdoctmp{\childdocforward{#2}}
    \else
      \def\childdoctmp{\childdocforwardprefix{#1}{#2}}
    \fi
    \expandafter
  \endgroup
  \childdoctmp
}
%    \end{macrocode}

%\iffalse
%</package>
%\fi
%
\endinput
\childdocforward{cdocsch2}"|
% \end{tabular}
% \end{center}
% Note that the trailing backslash on each first line
% merely continues the input to the second line
% (for convenient cut ant paste).
% Furthermore, the command |latex| can be replaced by any
% of its alternative versions such as |pdflatex|.
%
% %%%%%%%%%%%%%%%%%%%%%%%%%%%%%%%%%%%%%%%%%%%%%%%%%%%%%%%%%%%%%%%%%%%%%%%%%%%%%%
% %%%%%%%%%%%%%%%%%%%%%%%%%%%%%%%%%%%%%%%%%%%%%%%%%%%%%%%%%%%%%%%%%%%%%%%%%%%%%%
% \section{Implementation}
%\iffalse
%<*package>
%\fi
%
% This section describes the definitions file |childdoc.def|.

% The definitions cannot be loaded using |\usepackage| or |\RequirePackage|
% which has a mechanism to prevent loading a style file more than once.
% When loading the definitions by means of |\input|
% multiple instances have to be prevented manually:
%\iffalse
%This code needs to be before the `\ProvidesFile' directive
%which is defined at the beginning of this file.
%Therefore it is also placed there and commented out here.
%</package>
%<*discard>
%\fi
%    \begin{macrocode}
\ifdefined\childdocmain\endinput\fi
%    \end{macrocode}
%\iffalse
%</discard>
%<*package>
%\fi
%
% \macro{\ifchilddoc}
% \macro{\ifchilddocmanual}
% The conditional |\ifchilddoc| tells whether a
% child (true) or main (false) document is being compiled.
% The conditional |\ifchilddocmanual| tells whether
% the |\includeonly| mechanism is used (false) or
% the selection of child files must be performed manually (true).
% The definitions initialise to false:
%    \begin{macrocode}
\newif\ifchilddoc
\newif\ifchilddocmanual
%    \end{macrocode}

% \macro{\childdocname}
% \macro{\childdocjob}
% The macro |\childdocname| stores the name of the main document
% to be compiled. The macro |\childdocjob| stores the name of
% the document on which the \LaTeX{} compiler was originally invoked.
% The content of |\jobname| cannot be compared
% to filenames specified in the source due to different catcodes.
% The following code rescans |\jobname|, stores the result
% in |\childdocname| and saves a copy in |\childdocjob|:
%    \begin{macrocode}
\edef\childdocname{\scantokens\expandafter{\jobname\noexpand}}
\let\childdocjob\childdocname
%    \end{macrocode}

% \macro{\childdocdisable}
% The macro |\childdocdisable| prevents the main file
% from being processed more than once.
% At this stage, the main document command |\childdocmain|
% is assumed to be called once again where it should do nothing.
% Any subsequent call to it should prevent
% a secondary processing of the main document
% It overwrites the forwarding commands
% |\childdocof| and |\childdocforward|
% with empty macros to prevent further inclusions of the main document:
%    \begin{macrocode}
\newcommand{\childdocdisable}
{
  \renewcommand{\childdocmain}[1]{\renewcommand{\childdocmain}[1]{\endinput}}
  \renewcommand{\childdocof}[1]{}
  \renewcommand{\childdocby}[2][]{}
  \renewcommand{\childdocforward}[2][]{}
  \renewcommand{\childdocdisable}{}
}
%    \end{macrocode}

% \macro{\childdocmain}
% The macro |\childdocmain| is to be called at the top of the main file
% with nothing or the main filename (without extension) as argument.
% First, it breaks loops.
% If the argument is not empty and does not match |\childdocname|
% (which is set by the first inclusion of |childdoc.def|),
% |\ifchilddoc| is set to true, |\includeonly| is applied to the child file
% and |\jobname| is set to the main file
% (for proper handling of |.aux| files):
%    \begin{macrocode}
\newcommand{\childdocmain}[1]
{
  \childdocdisable\childdocmain{}
  \if?#1?\else
    \begingroup
      \def\childdoctmp{#1}
      \ifx\childdoctmp\childdocname
        \def\childdoctmp{}
      \else
        \def\childdoctmp
        {
          \childdoctrue
          \includeonly{\childdocname}
          \def\childdocjob{#1}
          \def\jobname{#1}
        }
      \fi
      \expandafter
    \endgroup
    \childdoctmp
  \fi
}
%    \end{macrocode}

% \macro{\childdocof}
% The command |\childdocof| redirects
% compilation to the main file |#1|.
%    \begin{macrocode}
\newcommand{\childdocof}[1]
{
  \childdocdisable
  \childdoctrue
  \includeonly{\childdocname}
  \def\jobname{#1}
  \def\childdocjob{#1}
  \input{#1}
}
%    \end{macrocode}

% \macro{\childdocby}
% The command |\childdocby| ....
%    \begin{macrocode}
\newcommand{\childdocby}[2][]
{
  \childdocdisable
  \childdoctrue
  \childdocmanualtrue
  \if?#1?\else
    \def\jobname{#2}
  \fi
  \def\childdocjob{#2}
  \input{#2}
  \endinput
}
%    \end{macrocode}

% \macro{\childdocforward}
% The command |\childdocforward| redirects
% compilation to the main file or
% (if the optional argument is given) a child file.
% Parameters are set as if the main file
% or a child file starting with |\childdocof| was compiled.
% Then compilation is handed over to the main file:
%    \begin{macrocode}
\newcommand{\childdocforward}[2][]
{
  \begingroup
    \if?#1?
      \def\childdoctmp
      {
        \def\childdocname{#2}
        \def\childdocjob{#2}
        \def\jobname{#2}
        \input{#2}
        \endinput
      }
    \else
      \def\childdoctmp
      {
        \childdocdisable
        \def\childdocname{#2}
        \childdoctrue
        \includeonly{#2}
        \def\childdocjob{#1}
        \def\jobname{#1}
        \input{#1}
        \endinput
      }
    \fi
    \expandafter
  \endgroup
  \childdoctmp
}
%    \end{macrocode}

% \macro{\childdocforwardprefix}
% The command |\childdocforwardprefix| redirects
% compilation to the main or a child file by means of a pattern.
% The prefix |#1| in the current filename is replaced by |#2|
% and the suffix of the current filename is kept
% (it is assumed that the filename does not contain the substring `|~~~|'
% which is used as a delimiter).
% Compilation is handed over to the new file by |\childdocforward|:
%    \begin{macrocode}
\newcommand{\childdocforwardprefix}[3][]
{
  \begingroup
    \def\childdocextract #2##1~~~{\def\childdoctmp{\childdocforward[#1]{#3##1}}}
    \expandafter\childdocextract\childdocname~~~
    \expandafter
  \endgroup
  \childdoctmp
}
%    \end{macrocode}

% \macro{\childdoc}
% The deprecated macro |\childdoc| is a legacy version of |\childdocmain|:
%    \begin{macrocode}
\newcommand{\childdoc}{\childdocmain}
%    \end{macrocode}

% \macro{\childdocredirect}
% The deprecated macro |\childdocredirect| is a legacy version
% of |\childdocforward| and |\childdocforwardprefix|:
%    \begin{macrocode}
\newcommand{\childdocredirect}[2][]
{
  \begingroup
    \if?#1?
      \def\childdoctmp{\childdocforward{#2}}
    \else
      \def\childdoctmp{\childdocforwardprefix{#1}{#2}}
    \fi
    \expandafter
  \endgroup
  \childdoctmp
}
%    \end{macrocode}

%\iffalse
%</package>
%\fi
%
\endinput

\childdocby{cdocsamp}
%    \end{macrocode}

%\iffalse
%</samplepart3|samplepart4>
%\fi
%
%\iffalse
%<*samplepart3>
%\fi
% Some text for part 3:
%    \begin{macrocode}
some text in part three
%    \end{macrocode}

%\iffalse
%</samplepart3>
%\fi
% Some text for part 4:
%\iffalse
%<*samplepart4>
%\fi
%    \begin{macrocode}
more text in part four
%    \end{macrocode}

%\iffalse
%</samplepart4>
%\fi
%
% %%%%%%%%%%%%%%%%%%%%%%%%%%%%%%%%%%%%%%
% \paragraph{Forwarding for a Complete Draft.}
%
% The following forwarding file |cdocsdrf.tex|
% compiles the main document in draft mode:
%\iffalse
%<*sampledraft>
%\fi
%    \begin{macrocode}
\def\version{draft}
% \iffalse
%
% childdoc.dtx Copyright (C) 2017-2018 Niklas Beisert
%
% This work may be distributed and/or modified under the
% conditions of the LaTeX Project Public License, either version 1.3
% of this license or (at your option) any later version.
% The latest version of this license is in
%   http://www.latex-project.org/lppl.txt
% and version 1.3 or later is part of all distributions of LaTeX
% version 2005/12/01 or later.
%
% This work has the LPPL maintenance status `maintained'.
%
% The Current Maintainer of this work is Niklas Beisert.
%
% This work consists of the files childdoc.dtx and childdoc.ins
% and the derived files childdoc.def and cdocsamp.tex with
% cdocsch1.tex, cdocsch2.tex, cdocsdrf.tex, cdocsfn1.tex, cdocsfn2.tex.
%
%<package>\ifdefined\childdocmain\endinput\fi
%<package>\ProvidesFile{childdoc.def}[2018/12/30 v2.0 child document driver]
%<samplemain>\ProvidesFile{cdocsamp.tex}[2018/12/30 v2.0 sample for childdoc]
%<*driver>
%\ProvidesFile{childdoc.drv}[2018/12/30 v2.0 childdoc reference manual file]
\PassOptionsToClass{10pt,a4paper}{article}
\documentclass{ltxdoc}

\usepackage[margin=35mm]{geometry}
\usepackage{hyperref}
\usepackage{hyperxmp}
\usepackage[usenames]{color}

\hypersetup{colorlinks=true}
\hypersetup{pdfstartview=FitH}
\hypersetup{pdfpagemode=UseNone}
\hypersetup{pdfsource={}}
\hypersetup{pdflang={en-UK}}
\hypersetup{pdfcopyright={Copyright 2017-2018 Niklas Beisert.
  This work may be distributed and/or modified under the
  conditions of the LaTeX Project Public License, either version 1.3
  of this license or (at your option) any later version.}}
\hypersetup{pdflicenseurl={http://www.latex-project.org/lppl.txt}}
\hypersetup{pdfcontactaddress={ETH Zurich, ITP, HIT K,
  Wolfgang-Pauli-Strasse 27}}
\hypersetup{pdfcontactpostcode={8093}}
\hypersetup{pdfcontactcity={Zurich}}
\hypersetup{pdfcontactcountry={Switzerland}}
\hypersetup{pdfcontactemail={nbeisert@itp.phys.ethz.ch}}
\hypersetup{pdfcontacturl={http://people.phys.ethz.ch/\xmptilde nbeisert/}}

\newcommand{\secref}[1]{\hyperref[#1]{section \ref*{#1}}}

\parskip1ex
\parindent0pt
\let\olditemize\itemize
\def\itemize{\olditemize\parskip0pt}

\begin{document}

\title{The \textsf{childdoc} Package}
\hypersetup{pdftitle={The childdoc Package}}
\author{Niklas Beisert\\[2ex]
  Institut f\"ur Theoretische Physik\\
  Eidgen\"ossische Technische Hochschule Z\"urich\\
  Wolfgang-Pauli-Strasse 27, 8093 Z\"urich, Switzerland\\[1ex]
  \href{mailto:nbeisert@itp.phys.ethz.ch}
  {\texttt{nbeisert@itp.phys.ethz.ch}}}
\hypersetup{pdfauthor={Niklas Beisert}}
\hypersetup{pdfsubject={Manual for the LaTeX2e Package childdoc}}
\date{30 December 2018, \textsf{v2.0}}
\maketitle

\begin{abstract}\noindent
\textsf{childdoc} is a \LaTeXe{} package
that enables the direct compilation
of document sections included by |\include|
to individual files.
\end{abstract}

\begingroup
\parskip0ex
\tableofcontents
\endgroup

%%%%%%%%%%%%%%%%%%%%%%%%%%%%%%%%%%%%%%%%%%%%%%%%%%%%%%%%%%%%%%%%%%%%%%%%%%%%%%%%
%%%%%%%%%%%%%%%%%%%%%%%%%%%%%%%%%%%%%%%%%%%%%%%%%%%%%%%%%%%%%%%%%%%%%%%%%%%%%%%%
\section{Introduction}

\LaTeX{} provides a mechanism to structure a large document (such as a book)
into a main file and several child files (containing the chapters)
using the |\include| command.
This mechanism is beneficial for documents
which span hundreds of pages in order to
make the source file(s) more manageable.
Moreover, compilation can be restricted to
selected child files by means of the |\includeonly| command.
The latter feature can be used to reduce the compilation time while editing
(this was significantly more useful in the earlier days of \LaTeX{})
or to generate a smaller document which is easier to navigate.
Another application of |\includeonly| is to generate
documents consisting of selected parts of the complete document.

However, there are a few drawbacks of the plain |\include| mechanism:
\begin{itemize}
\item
The child files cannot be compiled on their own,
they can only be compiled via the main file.
A naive editing environment
(such as a text editor with an option
to have the current file processed by \LaTeX)
may require one to switch to the main file before compiling;
attempting to compile the child file produces errors.
\item
The main file must be modified (each time)
to adjust the |\includeonly| command
to the present needs. This easily leaves the main file in a messy state.
\item
The generated document will always carry the filename
of the main document. This is inconvenient if
several child files are to be compiled and
to be kept for distribution.
\end{itemize}

The present package provides a simple interface
to make child files individually compilable by \LaTeX{}.
Compiling a child file then has the same effect as compiling
the main file with an |\includeonly| command
to select the appropriate child.
Moreover the generated document will carry the name of the child
rather than the main file.
This resolves all three above issues.

This feature is meant to make the editing of books,
thesis documents and lecture notes somewhat more convenient.
However, the package can also be used efficiently for
composing a series of documents (such as exercise sheets)
which are typically distributed individually.
It then assists the author in generating the individual documents
(potentially in different versions)
as well as a document containing the collected series.
Another application is in developing style files
or other kinds of included material
where compilation of the style file could redirect
to a sample or test file.

%%%%%%%%%%%%%%%%%%%%%%%%%%%%%%%%%%%%%%%%%%%%%%%%%%%%%%%%%%%%%%%%%%%%%%%%%%%%%%%%
%%%%%%%%%%%%%%%%%%%%%%%%%%%%%%%%%%%%%%%%%%%%%%%%%%%%%%%%%%%%%%%%%%%%%%%%%%%%%%%%
\section{Usage}

First of all, the package \textsf{childdoc} is \emph{not} a standard
\LaTeXe{} |.sty| style file! Therefore it needs to be invoked in
a non-standard way.

%%%%%%%%%%%%%%%%%%%%%%%%%%%%%%%%%%%%%%%%%%%%%%%%%%%%%%%%%%%%%%%%%%%%%%%%%%%%%%%%
\subsection{Included Files}
\label{sec:include}

%%%%%%%%%%%%%%%%%%%%%%%%%%%%%%%%%%%%%%%%
\DescribeMacro{\childdocmain}
To use the package, add the commands
\begin{center}
\begin{tabular}{l}
|% \iffalse
%
% childdoc.dtx Copyright (C) 2017-2018 Niklas Beisert
%
% This work may be distributed and/or modified under the
% conditions of the LaTeX Project Public License, either version 1.3
% of this license or (at your option) any later version.
% The latest version of this license is in
%   http://www.latex-project.org/lppl.txt
% and version 1.3 or later is part of all distributions of LaTeX
% version 2005/12/01 or later.
%
% This work has the LPPL maintenance status `maintained'.
%
% The Current Maintainer of this work is Niklas Beisert.
%
% This work consists of the files childdoc.dtx and childdoc.ins
% and the derived files childdoc.def and cdocsamp.tex with
% cdocsch1.tex, cdocsch2.tex, cdocsdrf.tex, cdocsfn1.tex, cdocsfn2.tex.
%
%<package>\ifdefined\childdocmain\endinput\fi
%<package>\ProvidesFile{childdoc.def}[2018/12/30 v2.0 child document driver]
%<samplemain>\ProvidesFile{cdocsamp.tex}[2018/12/30 v2.0 sample for childdoc]
%<*driver>
%\ProvidesFile{childdoc.drv}[2018/12/30 v2.0 childdoc reference manual file]
\PassOptionsToClass{10pt,a4paper}{article}
\documentclass{ltxdoc}

\usepackage[margin=35mm]{geometry}
\usepackage{hyperref}
\usepackage{hyperxmp}
\usepackage[usenames]{color}

\hypersetup{colorlinks=true}
\hypersetup{pdfstartview=FitH}
\hypersetup{pdfpagemode=UseNone}
\hypersetup{pdfsource={}}
\hypersetup{pdflang={en-UK}}
\hypersetup{pdfcopyright={Copyright 2017-2018 Niklas Beisert.
  This work may be distributed and/or modified under the
  conditions of the LaTeX Project Public License, either version 1.3
  of this license or (at your option) any later version.}}
\hypersetup{pdflicenseurl={http://www.latex-project.org/lppl.txt}}
\hypersetup{pdfcontactaddress={ETH Zurich, ITP, HIT K,
  Wolfgang-Pauli-Strasse 27}}
\hypersetup{pdfcontactpostcode={8093}}
\hypersetup{pdfcontactcity={Zurich}}
\hypersetup{pdfcontactcountry={Switzerland}}
\hypersetup{pdfcontactemail={nbeisert@itp.phys.ethz.ch}}
\hypersetup{pdfcontacturl={http://people.phys.ethz.ch/\xmptilde nbeisert/}}

\newcommand{\secref}[1]{\hyperref[#1]{section \ref*{#1}}}

\parskip1ex
\parindent0pt
\let\olditemize\itemize
\def\itemize{\olditemize\parskip0pt}

\begin{document}

\title{The \textsf{childdoc} Package}
\hypersetup{pdftitle={The childdoc Package}}
\author{Niklas Beisert\\[2ex]
  Institut f\"ur Theoretische Physik\\
  Eidgen\"ossische Technische Hochschule Z\"urich\\
  Wolfgang-Pauli-Strasse 27, 8093 Z\"urich, Switzerland\\[1ex]
  \href{mailto:nbeisert@itp.phys.ethz.ch}
  {\texttt{nbeisert@itp.phys.ethz.ch}}}
\hypersetup{pdfauthor={Niklas Beisert}}
\hypersetup{pdfsubject={Manual for the LaTeX2e Package childdoc}}
\date{30 December 2018, \textsf{v2.0}}
\maketitle

\begin{abstract}\noindent
\textsf{childdoc} is a \LaTeXe{} package
that enables the direct compilation
of document sections included by |\include|
to individual files.
\end{abstract}

\begingroup
\parskip0ex
\tableofcontents
\endgroup

%%%%%%%%%%%%%%%%%%%%%%%%%%%%%%%%%%%%%%%%%%%%%%%%%%%%%%%%%%%%%%%%%%%%%%%%%%%%%%%%
%%%%%%%%%%%%%%%%%%%%%%%%%%%%%%%%%%%%%%%%%%%%%%%%%%%%%%%%%%%%%%%%%%%%%%%%%%%%%%%%
\section{Introduction}

\LaTeX{} provides a mechanism to structure a large document (such as a book)
into a main file and several child files (containing the chapters)
using the |\include| command.
This mechanism is beneficial for documents
which span hundreds of pages in order to
make the source file(s) more manageable.
Moreover, compilation can be restricted to
selected child files by means of the |\includeonly| command.
The latter feature can be used to reduce the compilation time while editing
(this was significantly more useful in the earlier days of \LaTeX{})
or to generate a smaller document which is easier to navigate.
Another application of |\includeonly| is to generate
documents consisting of selected parts of the complete document.

However, there are a few drawbacks of the plain |\include| mechanism:
\begin{itemize}
\item
The child files cannot be compiled on their own,
they can only be compiled via the main file.
A naive editing environment
(such as a text editor with an option
to have the current file processed by \LaTeX)
may require one to switch to the main file before compiling;
attempting to compile the child file produces errors.
\item
The main file must be modified (each time)
to adjust the |\includeonly| command
to the present needs. This easily leaves the main file in a messy state.
\item
The generated document will always carry the filename
of the main document. This is inconvenient if
several child files are to be compiled and
to be kept for distribution.
\end{itemize}

The present package provides a simple interface
to make child files individually compilable by \LaTeX{}.
Compiling a child file then has the same effect as compiling
the main file with an |\includeonly| command
to select the appropriate child.
Moreover the generated document will carry the name of the child
rather than the main file.
This resolves all three above issues.

This feature is meant to make the editing of books,
thesis documents and lecture notes somewhat more convenient.
However, the package can also be used efficiently for
composing a series of documents (such as exercise sheets)
which are typically distributed individually.
It then assists the author in generating the individual documents
(potentially in different versions)
as well as a document containing the collected series.
Another application is in developing style files
or other kinds of included material
where compilation of the style file could redirect
to a sample or test file.

%%%%%%%%%%%%%%%%%%%%%%%%%%%%%%%%%%%%%%%%%%%%%%%%%%%%%%%%%%%%%%%%%%%%%%%%%%%%%%%%
%%%%%%%%%%%%%%%%%%%%%%%%%%%%%%%%%%%%%%%%%%%%%%%%%%%%%%%%%%%%%%%%%%%%%%%%%%%%%%%%
\section{Usage}

First of all, the package \textsf{childdoc} is \emph{not} a standard
\LaTeXe{} |.sty| style file! Therefore it needs to be invoked in
a non-standard way.

%%%%%%%%%%%%%%%%%%%%%%%%%%%%%%%%%%%%%%%%%%%%%%%%%%%%%%%%%%%%%%%%%%%%%%%%%%%%%%%%
\subsection{Included Files}
\label{sec:include}

%%%%%%%%%%%%%%%%%%%%%%%%%%%%%%%%%%%%%%%%
\DescribeMacro{\childdocmain}
To use the package, add the commands
\begin{center}
\begin{tabular}{l}
|\input{childdoc.def}|\\
|\childdocmain{}|\\
\end{tabular}
\end{center}
at the very top of the main \LaTeX{} file,
in particular \emph{before} the |\documentclass| statement!
The argument of |\childdocmain| should be left empty
(but it must be present).

%%%%%%%%%%%%%%%%%%%%%%%%%%%%%%%%%%%%%%%%
\DescribeMacro{\childdocof}
Furthermore, add the commands
\begin{center}
\begin{tabular}{l}
|\input{childdoc.def}|\\
|\childdocof{|\textit{main}|}|\\
\end{tabular}
\end{center}
at the top of every child file \textit{child}
which is included by |\include{|\textit{child}|}|
from within the main file
(or at least for those files to be compiled individually).
The argument \textit{main} must be the filename of the main file.

There are a couple of
considerations in setting up the main and child documents:

%%%%%%%%%%%%%%%%%%%%%%%%%%%%%%%%%%%%%%%%
\paragraph{Restrictions.}

Please note the following restrictions:
\begin{itemize}
\item
|\childdocmain| must be called with one argument \textit{main}
to ensure compatibility with earlier version of the package.
It must either be empty (|\childdocmain{}|)
or precisely match the filename of the main file in which it is specified.
See \secref{sec:detection} for further information.
\item
The filename \textit{main} must be specified without the |.tex| extension.
\item
The filename \textit{main} is case sensitive
(even in case-insensitive file systems)
due to internal string comparison.
\item
The argument \textit{main} should be fully expanded, it cannot be a macro.
\item
Subdirectories and special characters should be avoided in filenames.
\item
The command |\childdocmain{|\textit{main}|}| must be followed by a whitespace.
It should not be followed immediately by another command
or by a comment mark `|%|'.
This is because the \TeX{} parser reads the token immediately following
the argument of |\childdocmain| and puts it
at the beginning of every child section;
however, a white\-space is ignored.
\end{itemize}

%%%%%%%%%%%%%%%%%%%%%%%%%%%%%%%%%%%%%%%%
\paragraph{Content of Main File.}

It is advisable to place all content in the child files included by |\include|.
Any output contained in the main file will appear in all child documents
unless suppressed manually;
it cannot be suppressed automatically by the |\includeonly| directive
and thus should normally be avoided.
A method to include some content in the main file
by means of conditional processing is described in \secref{sec:conditional}.

%%%%%%%%%%%%%%%%%%%%%%%%%%%%%%%%%%%%%%%%
\paragraph{Page Numbering.}

When only a part of the document is compiled,
the appropriate numbering of pages
(as well as other status parameters)
is determined from the |.aux| files.
The latter contain information from previous passes.
However this information needs to propagate through
all intermediate child documents.
Therefore the page numbering in child documents may well
be inconsistent until the complete document is compiled at least once.

A useful (if unconventional) way to always ensure a consistent
page numbering is to restart the numbering in each child document
and denote the pages by `\textit{child}|.|\textit{page}'
where \textit{child} represents the chapter/section number of the child file.
This can be achieved by the command
|\numberwithin{page}{|\textit{child}|}|
of the \textsf{amsmath} package
where \textit{child} can be |chapter| or |section|
depending on the chosen structuring.
Alternatively, one can modify the macro |\thepage| appropriately
and reset the counter |page| at the start of each child file.

%%%%%%%%%%%%%%%%%%%%%%%%%%%%%%%%%%%%%%%%%%%%%%%%%%%%%%%%%%%%%%%%%%%%%%%%%%%%%%%%
\subsection{Conditional Processing}
\label{sec:conditional}

The package provides a mechanism to compile different versions
of a document. To customise the versions further some conditional processing
can come in handy to distinguish which version is being compiled.
The package provides two macros to describe the compilation context:

%%%%%%%%%%%%%%%%%%%%%%%%%%%%%%%%%%%%%%%%
\DescribeMacro{\ifchilddoc}
The conditional |\ifchilddoc| distinguishes between the compilation of
child documents and the main document:
%
\begin{center}
|\ifchilddoc |\textit{child-code}| |[|\||else |\textit{main-code}]| \||fi|
\end{center}

%%%%%%%%%%%%%%%%%%%%%%%%%%%%%%%%%%%%%%%%
\DescribeMacro{\childdocname}
\DescribeMacro{\childdocjob}
The macro |\childdocname| contains the filename (without extension)
of the main or child file being processed.
Note that |\childdocjob| will always contain the name of the main file.

%%%%%%%%%%%%%%%%%%%%%%%%%%%%%%%%%%%%%%%%
\paragraph{Title Page.}

Conditional processing can be used to include a title or banner page
in the main document when proper precautions are taken.
Importantly, the code in the main file should ensure that the page counter
(as well as other status parameters which are stored in the |.aux| files)
takes the same value after the conditional processing.
Otherwise the page numbers may take divergent values
depending on which part is compiled.

For example, a title page could be declared by:
%
\begin{center}
\begin{tabular}{l}
|\ifchilddoc\||else|\\
|\addtocounter{page}{-1}|\\
\textit{code for title page}\\
|\newpage|\\
|\||fi|
\end{tabular}
\end{center}
%
A banner page for the child documents can be generated by:
%
\begin{center}
\begin{tabular}{l}
|\ifchilddoc|\\
|\addtocounter{page}{-1}|\\
\textit{code for banner page}\\
|\newpage|\\
|\||fi|
\end{tabular}
\end{center}
%
Here one could write a message such as:
\begin{center}
|This is the part \childdocname{} of \childdocjob{}.|
\end{center}

%%%%%%%%%%%%%%%%%%%%%%%%%%%%%%%%%%%%%%%%%%%%%%%%%%%%%%%%%%%%%%%%%%%%%%%%%%%%%%%%
\subsection{Flags}
\label{sec:flags}

The package makes it easy to generate different versions
of the main or child documents.
To this end compilation flags can be defined
and assigned different default values.
They will be particularly useful in conjunction
with the forwarding mechanism described in \secref{sec:forward}.

For example, it may be useful to have a flag |\version|
which can be set to |draft| or |final|.
The document source will contain some conditional code
depending on the value of |\version|.
Suppose further, the flag should default to |final| for the main file
and to |draft| for child files
which is a natural assignment for editing the document.
This is achieved by placing the following code
in the preamble of the main document
(below the |\childdocmain| directive):
%
\begin{center}
\begin{tabular}{l}
|\ifchilddoc|\\
|\providecommand{\version}{draft}|\\
|\||else|\\
|\providecommand{\version}{final}|\\
|\||fi|
\end{tabular}
\end{center}
%
The definition by |\providecommand| makes sure
that previous definitions are not overwritten.
Further statements |\providecommand{\version}{...}|
can thus be added before the above code to override it.

For the main file, one might add a line
(between |\childdocmain| and the above block)
%
\begin{center}
|%\ifchilddoc\||else\providecommand{\version}{draft}\||fi|
\end{center}
%
which can be uncommented to produce a draft version.
Likewise one can add a line to the very top of a child file
(above the |\childdocof{|\textit{main}|}| directive)
%
\begin{center}
|%\providecommand{\version}{final}|
\end{center}
%
which can be uncommented to produce the final version of this child document.

%%%%%%%%%%%%%%%%%%%%%%%%%%%%%%%%%%%%%%%%%%%%%%%%%%%%%%%%%%%%%%%%%%%%%%%%%%%%%%%%
\subsection{Forwarding}
\label{sec:forward}

Different versions of the main or child documents
using compilation flags as described in \secref{sec:flags}
can be (permanently) stored in different files
for convenient compilation, viewing and distribution.
To this end, the package defines a command
to pass on compilation to a different file:

%%%%%%%%%%%%%%%%%%%%%%%%%%%%%%%%%%%%%%%%
\DescribeMacro{\childdocforward}
The command |\childdocforward| redirects processing to
another source file:
%
\begin{center}
\begin{tabular}{l}
|\input{childdoc.def}|\\
|\childdocforward[|\textit{main}|]{|\textit{dest}|}|\\
\end{tabular}
\end{center}
%
The argument \textit{dest} is the destination file
(without extension).
It should be the main file or one of the child files.
Note that further \textsf{childdoc} directives
such as |\childdocof| and |\childdocforward|
in the indicated file will be processed in this form.
The optional argument \textit{main}
passes on directly to the main file \textit{main}
while pretending to compile the child \textit{dest}.
This form behaves as if \textit{dest}
issues |\childdocof{|\textit{main}|}| right away,
and no further \textsf{childdoc} directives will be processed.

%%%%%%%%%%%%%%%%%%%%%%%%%%%%%%%%%%%%%%%%
\DescribeMacro{\...prefix}
In the alternative form |\childdocforwardprefix|,
%
\begin{center}
\begin{tabular}{l}
|\input{childdoc.def}|\\
|\childdocforwardprefix[|\textit{main}|]{|\textit{prefix}|}{|\textit{dest}|}|
\end{tabular}
\end{center}
%
the destination file is determined by a pattern
depending on the current file:
To make this work, the current file must be called
`{\textit{prefix}\hspace{0.2em}\textit{suffix}}'
with \textit{prefix} matching precisely the argument.
Processing is then passed on to the file
`{\textit{dest}\hspace{0.2em}\textit{suffix}}'.
Surely, the same effect is achieved by
directly specifying the
argument `{\textit{dest}\hspace{0.2em}\textit{suffix}}'
in the first form.
However, that requires to set up a different file
for each child. With the alternative form of the command
all these files can have exactly the same content
which simplifies setting them up and maintaining them.

For example, the following file |draft.tex|
with a compilation flag |\version| as described in \secref{sec:flags}
compiles the main document as a draft:
%
\begin{center}
\begin{tabular}{l}
|\def\version{draft}|\\
|\input{childdoc.def}|\\
|\childdocforward{|\textit{main}|}|
\end{tabular}
\end{center}
%
Likewise, the following files |final|\textit{nn}|.tex|
compile the final version of the child document
|child|\textit{nn}|.tex|:
%
\begin{center}
\begin{tabular}{l}
|\def\version{final}|\\
|\input{childdoc.def}|\\
|\childdocforwardprefix{final}{child}|
\end{tabular}
\end{center}
%

Note that when several versions of a main file and/or of each child file
are to be generated, it may be convenient to set up a |Makefile| or
shell script to automatise the process.

%%%%%%%%%%%%%%%%%%%%%%%%%%%%%%%%%%%%%%%%%%%%%%%%%%%%%%%%%%%%%%%%%%%%%%%%%%%%%%%%
\subsection{Command Line Processing}
\label{sec:commandline}

The effect of redirection files can also be achieved by invoking
the \LaTeX{} compiler with a more elaborate command line.
Most conveniently this should be done as part
of a shell script or a |Makefile|.

When using \textsf{childdoc} in the main file, the following
command lines effectively perform a redirection
(note that depending on the shell being used,
backslashes may have to be doubled: `|\|' $\to$ `|\\|'):
%
\begin{center}
|... -jobname "|\textit{target}|" |\\|"|[\textit{flags}]%
|\input{childdoc.def}\childdocforward[|\textit{main}|]{|\textit{dest}|}"|
\end{center}
%
Here \textit{target} is the name of the output file,
\textit{main} is the name of the main file
and \textit{dest} is the name of the main or child file to be processed
(all filenames without extensions).
The optional argument \textit{main} can be omitted
if \textit{main} matches \textit{dest}.
Optionally, compilation \textit{flags} can be defined via |\def| commands.
This command line makes the \TeX{} engine believe
it is compiling the file \textit{target}
whose content is specified as the latter parameter.
The provided code then forwards the processing to
\textit{main} or \textit{dest} as described in \secref{sec:forward}.

%%%%%%%%%%%%%%%%%%%%%%%%%%%%%%%%%%%%%%%%%%%%%%%%%%%%%%%%%%%%%%%%%%%%%%%%%%%%%%%%
\subsection{Include by Input}
\label{sec:input}

Including child documents by |\include| has some restrictions by design.
Most notably, the content of a child document always occupies
its own set of pages; pages cannot be shared between child documents.
Usually, this behaviour makes perfect sense
because each child document contain an essential part of the document.
However, in some situations it may be desirable to compose
a document from a collection of parts
without having mandatory page breaks between then.
For this case, the package
provides a mechanism to include parts
by |\input| which can also be processed individually.
However, by construction this mechanism
requires manual handling of the content to be output.

%%%%%%%%%%%%%%%%%%%%%%%%%%%%%%%%%%%%%%%%
\DescribeMacro{\ifchilddocmanual}
The main file should be prepared as usual, see \secref{sec:include}.
However, the document body must make a distinction
between processing of an individual part and of the main document, e.g.:
%
\begin{center}
\begin{tabular}{l}
|\ifchilddocmanual|\\
|\input{\childdocname}|\\
|\||else|\\
\textit{document body with }|\input{|\textit{part}|}|\\
|\||fi|
\end{tabular}
\end{center}
%
The conditional |\ifchilddocmanual| is true whenever
a part to be included by |\input| is being compiled,
and the name of the part is stored in |\childdocname|.

%%%%%%%%%%%%%%%%%%%%%%%%%%%%%%%%%%%%%%%%
\DescribeMacro{\childdocby}
Each part to be included by |\input| should start with:
%
\begin{center}
\begin{tabular}{l}
|\input{childdoc.def}|\\
|\childdocby{|\textit{main}|}|\\
\end{tabular}
\end{center}
%
The directive |\childdocby| is similar to |\childdocof|
described in \secref{sec:include},
but the subsequent selection of content must be done manually.
To that end, both |\ifchilddoc| and |\ifchilddocmanual|
will be true upon processing of a part,
and the name of the part is stored in |\childdocname|.
Note that |\jobname| will be set to the filename of the current part
so that each part receives an individual |.aux| file
that does not interfere with the |.aux| file(s) of the main document.
This behaviour can be altered by the alternative form
|\childdocby[*]{|\textit{main}|}| (with a non-empty optional argument)
which uses the |.aux| file of the main document
by setting |\jobname| to \textit{main}.

%%%%%%%%%%%%%%%%%%%%%%%%%%%%%%%%%%%%%%%%%%%%%%%%%%%%%%%%%%%%%%%%%%%%%%%%%%%%%%%%
\subsection{Driver Development}
\label{sec:driver}

The \textsf{childdoc} mechanism can also be use for the development
of definition files such as \LaTeX{} styles or classes.
This case differs from the above setup with multiple parts
included by |\include| in that no |\includeonly| should be invoked.
This can be achieved by starting the include file
(before |\ProvidesPackage|) with:
%
\begin{center}
\begin{tabular}{l}
|\input{childdoc.def}|\\
|\childdocforward{|\textit{main}|}|\\
\end{tabular}
\end{center}
%
or alternatively with:
%
\begin{center}
\begin{tabular}{l}
|\input{childdoc.def}|\\
|\childdocby{|\textit{main}|}|\\
\end{tabular}
\end{center}
%
Both forms have slightly different effects as described above.
The main file is prepared as usual, see \secref{sec:include}.

%%%%%%%%%%%%%%%%%%%%%%%%%%%%%%%%%%%%%%%%%%%%%%%%%%%%%%%%%%%%%%%%%%%%%%%%%%%%%%%%
\subsection{Legacy Detection}
\label{sec:detection}

The directive |\childdocmain| in the main file can detect
whether the complete document or merely a child is to be compiled
even without using the directive |\childdocof|.
This method is deprecated because it is less robust
and there is no compelling reason to use it;
it is merely provided for backward compatibility
and it may be removed in future versions.

If the detection mechanism is to be used,
it is mandatory to correctly specify
the filename of the main file as the argument of |\childdocmain|:
%
\begin{center}
\begin{tabular}{l}
|\input{childdoc.def}|\\
|\childdocmain{|\textit{main}|}|\\
\end{tabular}
\end{center}
%
If |\jobname| does not match the argument \textit{main} of |\childdocmain|,
it is assumed that |\jobname| points to the child file to be compiled.
When using |\childdocmain| with the main file specified as argument,
it suffices to start a child file
with just |\input{|\textit{main}|}|
without loading of the package and using |\childdocof|.
If instead all processing is done
with the appropriate \textsf{childdoc} directives,
the argument of \textit{main} of |\childdocmain| can be empty.

An alternative version of the command line processing described
in \secref{sec:commandline} using the detection mechanism reads:
%
\begin{center}
|... -jobname "|\textit{target}|" "|[\textit{flags}]%
[|\def\jobname{|\textit{dest}|}|]|\input{|\textit{main}|}"|
\end{center}

%%%%%%%%%%%%%%%%%%%%%%%%%%%%%%%%%%%%%%%%%%%%%%%%%%%%%%%%%%%%%%%%%%%%%%%%%%%%%%%%
\subsection{Manual Code}
\label{sec:manual}

In case one cannot be certain whether the definitions file |childdoc.def|
is installed on the target \TeX{} distribution
and one prefers not to ship it,
it is conceivable to paste a few relevant commands into the sources.

To that end, drop all statements |\input{childdoc.def}|
and perform the replacements as outlined below.
Instead of |\childdocmain{|\textit{main}|}| add the following code
to the top of the main file:
%
\begin{center}
\begin{tabular}{l}
|\||ifdefined\childdocname\endinput\||fi\newif\ifchilddoc|\\
|\edef\childdocname{\scantokens\expandafter{\jobname\noexpand}}|\\
|\def\childdocmain{|\textit{main}|}\||ifx\childdocmain\childdocname\||else|\\
|\childdoctrue\includeonly{\childdocname}\let\jobname\childdocmain\||fi|\\
\end{tabular}
\end{center}
%
Instead of |\childdocof{|\textit{main}|}| just include the main file
at the top of each child file:
%
\begin{center}
|\input{|\textit{main}|}|
\end{center}
%
A simple redirection |\childdocforward{|\textit{dest}|}| is achieved by:
%
\begin{center}
|\def\jobname{|\textit{dest}|}\input{\jobname}|
\end{center}
%
The redirection with prefix
|\childdocforwardprefix[|\textit{prefix}|]{|\textit{dest}|}|
is accomplished by:
%
\begin{center}
\begin{tabular}{l}
|{\edef\jobname{\scantokens\expandafter{\jobname\noexpand}}|\\
|\def\redirectjob |\textit{prefix}|#1~~~{\gdef\jobname{|\textit{dest}|#1}}|\\
|\expandafter\redirectjob\jobname~~~}\input{\jobname}|
\end{tabular}
\end{center}

In an alternative approach,
child documents can be compiled by a specific command line
without additional code or specific definitions:
%
\begin{center}
|... -jobname "|\textit{target}|" "|[\textit{flags}]%
|\includeonly{|\textit{dest}|}\input{|\textit{main}|}"|
\end{center}
%

%%%%%%%%%%%%%%%%%%%%%%%%%%%%%%%%%%%%%%%%%%%%%%%%%%%%%%%%%%%%%%%%%%%%%%%%%%%%%%%%
%%%%%%%%%%%%%%%%%%%%%%%%%%%%%%%%%%%%%%%%%%%%%%%%%%%%%%%%%%%%%%%%%%%%%%%%%%%%%%%%
\section{Information}

%%%%%%%%%%%%%%%%%%%%%%%%%%%%%%%%%%%%%%%%%%%%%%%%%%%%%%%%%%%%%%%%%%%%%%%%%%%%%%%%
\subsection{Copyright}

Copyright \copyright{} 2017--2018 Niklas Beisert

This work may be distributed and/or modified under the
conditions of the \LaTeX{} Project Public License, either version 1.3
of this license or (at your option) any later version.
The latest version of this license is in
  \url{http://www.latex-project.org/lppl.txt}
and version 1.3 or later is part of all distributions of \LaTeX{}
version 2005/12/01 or later.

This work has the LPPL maintenance status `maintained'.

The Current Maintainer of this work is Niklas Beisert.

This work consists of the files |README.txt|, |childdoc.ins| and |childdoc.dtx|
as well as the derived files |childdoc.def|, |cdocsamp.tex|
with |cdocsch1.tex|, |cdocsch2.tex|, |cdocspt3.tex|, |cdocspt4.tex|,
|cdocsdrf.tex|, |cdocsfn1.tex|, |cdocsfn2.tex|
as well as |childdoc.pdf|.

%%%%%%%%%%%%%%%%%%%%%%%%%%%%%%%%%%%%%%%%%%%%%%%%%%%%%%%%%%%%%%%%%%%%%%%%%%%%%%%%
\subsection{Files and Installation}

The package consists of the files:
%
\begin{center}
\begin{tabular}{ll}
    |README.txt|   & readme file \\
    |childdoc.ins| & installation file \\
    |childdoc.dtx| & source file \\
    |childdoc.def| & definition file \\
    |cdocsamp.tex| & sample main file \\
    |cdocsch1.tex| & sample include file \\
    |cdocsch2.tex| & sample include file \\
    |cdocspt3.tex| & sample part file \\
    |cdocspt4.tex| & sample part file \\
    |cdocsdrf.tex| & sample redirection file \\
    |cdocsfn1.tex| & sample redirection file \\
    |cdocsfn2.tex| & sample redirection file \\
    |childdoc.pdf| & manual
\end{tabular}
\end{center}
%
The distribution consists of the files
|README.txt|, |childdoc.ins| and |childdoc.dtx|.
%
\begin{itemize}
\item
Run (pdf)\LaTeX{} on |childdoc.dtx|
to compile the manual |childdoc.pdf| (this file).
\item
Run \LaTeX{} on |childdoc.ins| to create the definitions file |childdoc.def|
and the sample |cdocsamp.tex| with include files
|cdocsch1.tex|, |cdocsch2.tex|, |cdocspt3.tex|, |cdocspt4.tex|,
|cdocsdrf.tex|, |cdocsfn1.tex|, |cdocsfn2.tex|.
Then copy the file |childdoc.def| to an appropriate directory of your \LaTeX{}
distribution, e.g.\ \textit{texmf-root}|/tex/latex/childdoc|.
\end{itemize}

%%%%%%%%%%%%%%%%%%%%%%%%%%%%%%%%%%%%%%%%%%%%%%%%%%%%%%%%%%%%%%%%%%%%%%%%%%%%%%%%
\subsection{Related CTAN Packages}

There are several other packages which offer a similar functionality:
%
\begin{itemize}
\item
The packages
\href{http://ctan.org/pkg/docmute}{\textsf{docmute}},
\href{http://ctan.org/pkg/includex}{\textsf{includex}} and
\href{http://ctan.org/pkg/standalone}{\textsf{standalone}}
provide commands to include only the document body of
a child file thus allowing both files to be compiled individually.
\item
The packages \href{http://ctan.org/pkg/subdocs}{\textsf{subdocs}}
and \href{http://ctan.org/pkg/subfiles}{\textsf{subfiles}}
provide structures in which the main and child documents can be
encapsulated and allowing them to be compiled individually.
The inclusion mechanism is different from the conventional |\include|.
\item
The package \href{http://ctan.org/pkg/combine}{\textsf{combine}}
is an elaborate solution to combine several documents into one.
\end{itemize}
%
See also the CTAN topic \href{http://ctan.org/topic/subdocs}{\textsf{subdocs}}
for further related packages.
The present package differs from the above solutions in that
a document structure constructed with the conventional |\include| mechanism
just needs two extra commands at the top of every file
such that all constituent files can be compiled individually.

%%%%%%%%%%%%%%%%%%%%%%%%%%%%%%%%%%%%%%%%%%%%%%%%%%%%%%%%%%%%%%%%%%%%%%%%%%%%%%%%
%\subsection{Feature Suggestions}
%
%The following is a list of features which may be useful for future
%versions of this package:
%%
%\begin{itemize}
%\item
%\ldots
%\end{itemize}

%%%%%%%%%%%%%%%%%%%%%%%%%%%%%%%%%%%%%%%%%%%%%%%%%%%%%%%%%%%%%%%%%%%%%%%%%%%%%%%%
\subsection{Revision History}

%%%%%%%%%%%%%%%%%%%%%%%%%%%%%%%%%%%%%%%%
\paragraph{v2.0:} 2018/12/30

\begin{itemize}
\item
immediate forward processing
\item
added |\childdocby| mechanism
\item
manual restructured
\end{itemize}

%%%%%%%%%%%%%%%%%%%%%%%%%%%%%%%%%%%%%%%%
\paragraph{v1.6:} 2018/01/17

\begin{itemize}
\item
application for development of include files
\item
corrections to manual
\end{itemize}

%%%%%%%%%%%%%%%%%%%%%%%%%%%%%%%%%%%%%%%%
\paragraph{v1.5:} 2017/05/21

\begin{itemize}
\item
more complete structuring introduced
\item
|\childdocof| introduced
\item
|\childdoc| renamed to |\childdocmain|
\item
|\childredirect| renamed to |\childdocforward| and |\childdocforwardprefix|
and functionality expanded
\end{itemize}

%%%%%%%%%%%%%%%%%%%%%%%%%%%%%%%%%%%%%%%%
\paragraph{v1.0:} 2017/04/27

\begin{itemize}
\item
manual and install package
\item
first version published on CTAN
\end{itemize}

%%%%%%%%%%%%%%%%%%%%%%%%%%%%%%%%%%%%%%%%
\paragraph{v0.6:} 2017/04/26

\begin{itemize}
\item
redirection mechanism added
\end{itemize}

%%%%%%%%%%%%%%%%%%%%%%%%%%%%%%%%%%%%%%%%
\paragraph{v0.5:} 2017/04/26

\begin{itemize}
\item
functionality in definition file
\end{itemize}


%%%%%%%%%%%%%%%%%%%%%%%%%%%%%%%%%%%%%%%%%%%%%%%%%%%%%%%%%%%%%%%%%%%%%%%%%%%%%%%%
%%%%%%%%%%%%%%%%%%%%%%%%%%%%%%%%%%%%%%%%%%%%%%%%%%%%%%%%%%%%%%%%%%%%%%%%%%%%%%%%
%%%%%%%%%%%%%%%%%%%%%%%%%%%%%%%%%%%%%%%%%%%%%%%%%%%%%%%%%%%%%%%%%%%%%%%%%%%%%%%%
\appendix

\settowidth\MacroIndent{\rmfamily\scriptsize 000\ }

 \DocInput{childdoc.dtx}

\end{document}
%</driver>
% \fi
%
% %%%%%%%%%%%%%%%%%%%%%%%%%%%%%%%%%%%%%%%%%%%%%%%%%%%%%%%%%%%%%%%%%%%%%%%%%%%%%%
% %%%%%%%%%%%%%%%%%%%%%%%%%%%%%%%%%%%%%%%%%%%%%%%%%%%%%%%%%%%%%%%%%%%%%%%%%%%%%%
% \section{Sample}
%\iffalse
%<*samplemain>
%\fi
%
% The following presents a sample document
% with two chapters, two parts, a title page,
% a compile flag as well as three forwarding files to set the flag.
% It consists of eight |.tex| files:
% \begin{center}
% \begin{tabular}{ll}
% |cdocsamp.tex|&main file\\
% |cdocsch1.tex|&include file for chapter 1\\
% |cdocsch2.tex|&include file for chapter 2\\
% |cdocspt3.tex|&include file for part 3\\
% |cdocspt4.tex|&include file for part 4\\
% |cdocsdrf.tex|&forwarding file for main file in draft mode\\
% |cdocsfi1.tex|&forwarding file for final version of chapter 1\\
% |cdocsfi2.tex|&forwarding file for final version of chapter 2\\
% \end{tabular}
% \end{center}
% Each of the eight files can be compiled directly by the \LaTeX{} compiler.
%
% %%%%%%%%%%%%%%%%%%%%%%%%%%%%%%%%%%%%%%
% \paragraph{Main File.}
%
% The main file is called |cdocsamp.tex|.
%
% Load the \textsf{childdoc} definitions and
% declare the filename for the main document:
%    \begin{macrocode}
\input{childdoc.def}
\childdocmain{}
%    \end{macrocode}

% Optional override for |\version| flag:
%    \begin{macrocode}
%%\ifchilddoc\else\providecommand{\version}{draft}\fi
%    \end{macrocode}

% Define the default values for the |\version| flag
% (|final| for the main file and |draft| for childs):
%    \begin{macrocode}
\ifchilddoc
\providecommand{\version}{draft}
\else
\providecommand{\version}{final}
\fi
%    \end{macrocode}

% Load the standard document class:
%    \begin{macrocode}
\documentclass[12pt]{article}
%    \end{macrocode}

% Start the document body:
%    \begin{macrocode}
\begin{document}
%    \end{macrocode}

% Declare a title page.
% Print title, part of document being processed and version flag:
%    \begin{macrocode}
\addtocounter{page}{-1}
\begin{center}
{\LARGE\bfseries{}childdoc example\par}
\vspace{1cm}
\ifchilddoc
\ifchilddocmanual part\else chapter\fi:
`\childdocname' of `\childdocjob'\par
\else
main document: `\childdocjob'\par
\fi
version: \version\par
\end{center}
\newpage
%    \end{macrocode}

% Manually include selected file,
% otherwise process as usual:
%    \begin{macrocode}
\ifchilddocmanual
\section*{part `\childdocname'}
\input{\childdocname}
\else
%    \end{macrocode}

% Include the two chapters:
%    \begin{macrocode}
\include{cdocsch1}
\include{cdocsch2}
%    \end{macrocode}

% Include the two parts unless only chapters should be displayed:
%    \begin{macrocode}
\ifchilddoc\else
\section{part three}
\input{cdocspt3}
\section{part four}
\input{cdocspt4}
\fi
%    \end{macrocode}

% Process as usual until here:
%    \begin{macrocode}
\fi
%    \end{macrocode}

% End of document body:
%    \begin{macrocode}
\end{document}
%    \end{macrocode}
%\iffalse
%</samplemain>
%\fi
%
% %%%%%%%%%%%%%%%%%%%%%%%%%%%%%%%%%%%%%%
% \paragraph{Chapter Include Files.}
%
% The include files are called |cdocsch1.tex| and |cdocsch2.tex|.
%
%\iffalse
%<*samplechap1|samplechap2>
%\fi

% Optional override for |\version| flag:
%    \begin{macrocode}
%%\providecommand{\version}{final}
%    \end{macrocode}

% Include the main document:
%    \begin{macrocode}
\input{childdoc.def}
\childdocof{cdocsamp}
%    \end{macrocode}

%\iffalse
%</samplechap1|samplechap2>
%\fi
%
%\iffalse
%<*samplechap1>
%\fi
% Some text for chapter 1:
%    \begin{macrocode}
\section{one}
some text in chapter one
%    \end{macrocode}

%\iffalse
%</samplechap1>
%\fi
% Some text for chapter 2:
%\iffalse
%<*samplechap2>
%\fi
%    \begin{macrocode}
\section{two}
more text in chapter two
%    \end{macrocode}

%\iffalse
%</samplechap2>
%\fi
%
% %%%%%%%%%%%%%%%%%%%%%%%%%%%%%%%%%%%%%%
% \paragraph{Part Include Files.}
%
% The include files are called |cdocspt3.tex| and |cdocspt4.tex|.
%
%\iffalse
%<*samplepart3|samplepart4>
%\fi

% Optional override for |\version| flag:
%    \begin{macrocode}
%%\providecommand{\version}{final}
%    \end{macrocode}

% Include the main document:
%    \begin{macrocode}
\input{childdoc.def}
\childdocby{cdocsamp}
%    \end{macrocode}

%\iffalse
%</samplepart3|samplepart4>
%\fi
%
%\iffalse
%<*samplepart3>
%\fi
% Some text for part 3:
%    \begin{macrocode}
some text in part three
%    \end{macrocode}

%\iffalse
%</samplepart3>
%\fi
% Some text for part 4:
%\iffalse
%<*samplepart4>
%\fi
%    \begin{macrocode}
more text in part four
%    \end{macrocode}

%\iffalse
%</samplepart4>
%\fi
%
% %%%%%%%%%%%%%%%%%%%%%%%%%%%%%%%%%%%%%%
% \paragraph{Forwarding for a Complete Draft.}
%
% The following forwarding file |cdocsdrf.tex|
% compiles the main document in draft mode:
%\iffalse
%<*sampledraft>
%\fi
%    \begin{macrocode}
\def\version{draft}
\input{childdoc.def}
\childdocforward{cdocsamp}
%    \end{macrocode}

%\iffalse
%</sampledraft>
%\fi
%
% %%%%%%%%%%%%%%%%%%%%%%%%%%%%%%%%%%%%%%
% \paragraph{Forwarding for Final Version of the Chapters.}
%
% The following forwarding files |cdocsfn1.tex| and |cdocsfn2.tex|
% (with identical content)
% compile the final versions of the child documents
% |cdocsch1.tex| and |cdocsch2.tex|, respectively:
%\iffalse
%<*samplefinal>
%\fi
%    \begin{macrocode}
\def\version{final}
\input{childdoc.def}
\childdocforwardprefix[cdocsamp]{cdocsfn}{cdocsch}
%    \end{macrocode}

%\iffalse
%</samplefinal>
%\fi
%
% %%%%%%%%%%%%%%%%%%%%%%%%%%%%%%%%%%%%%%
% \paragraph{Command Line Processing.}
%
% The following three command lines generate the output files
% |cdocscld|, |cdocscl1| and |cdocscl2|
% which should be identical to
% |cdocsdrf|, |cdocsch1| and |cdocsfn2|, respectively:
% \begin{center}
% \begin{tabular}{l}
% |latex -jobname cdocscld \|\\
% |  "\def\version{draft}\input{childdoc.def}\childdocforward{cdocsamp}"|\\
% |latex -jobname cdocscl1 \|\\
% |  "\input{childdoc.def}\childdocforward[cdocsamp]{cdocsch1}"|\\
% |latex -jobname cdocscl2 \|\\
% |  "\def\version{final}\input{childdoc.def}\childdocforward{cdocsch2}"|
% \end{tabular}
% \end{center}
% Note that the trailing backslash on each first line
% merely continues the input to the second line
% (for convenient cut ant paste).
% Furthermore, the command |latex| can be replaced by any
% of its alternative versions such as |pdflatex|.
%
% %%%%%%%%%%%%%%%%%%%%%%%%%%%%%%%%%%%%%%%%%%%%%%%%%%%%%%%%%%%%%%%%%%%%%%%%%%%%%%
% %%%%%%%%%%%%%%%%%%%%%%%%%%%%%%%%%%%%%%%%%%%%%%%%%%%%%%%%%%%%%%%%%%%%%%%%%%%%%%
% \section{Implementation}
%\iffalse
%<*package>
%\fi
%
% This section describes the definitions file |childdoc.def|.

% The definitions cannot be loaded using |\usepackage| or |\RequirePackage|
% which has a mechanism to prevent loading a style file more than once.
% When loading the definitions by means of |\input|
% multiple instances have to be prevented manually:
%\iffalse
%This code needs to be before the `\ProvidesFile' directive
%which is defined at the beginning of this file.
%Therefore it is also placed there and commented out here.
%</package>
%<*discard>
%\fi
%    \begin{macrocode}
\ifdefined\childdocmain\endinput\fi
%    \end{macrocode}
%\iffalse
%</discard>
%<*package>
%\fi
%
% \macro{\ifchilddoc}
% \macro{\ifchilddocmanual}
% The conditional |\ifchilddoc| tells whether a
% child (true) or main (false) document is being compiled.
% The conditional |\ifchilddocmanual| tells whether
% the |\includeonly| mechanism is used (false) or
% the selection of child files must be performed manually (true).
% The definitions initialise to false:
%    \begin{macrocode}
\newif\ifchilddoc
\newif\ifchilddocmanual
%    \end{macrocode}

% \macro{\childdocname}
% \macro{\childdocjob}
% The macro |\childdocname| stores the name of the main document
% to be compiled. The macro |\childdocjob| stores the name of
% the document on which the \LaTeX{} compiler was originally invoked.
% The content of |\jobname| cannot be compared
% to filenames specified in the source due to different catcodes.
% The following code rescans |\jobname|, stores the result
% in |\childdocname| and saves a copy in |\childdocjob|:
%    \begin{macrocode}
\edef\childdocname{\scantokens\expandafter{\jobname\noexpand}}
\let\childdocjob\childdocname
%    \end{macrocode}

% \macro{\childdocdisable}
% The macro |\childdocdisable| prevents the main file
% from being processed more than once.
% At this stage, the main document command |\childdocmain|
% is assumed to be called once again where it should do nothing.
% Any subsequent call to it should prevent
% a secondary processing of the main document
% It overwrites the forwarding commands
% |\childdocof| and |\childdocforward|
% with empty macros to prevent further inclusions of the main document:
%    \begin{macrocode}
\newcommand{\childdocdisable}
{
  \renewcommand{\childdocmain}[1]{\renewcommand{\childdocmain}[1]{\endinput}}
  \renewcommand{\childdocof}[1]{}
  \renewcommand{\childdocby}[2][]{}
  \renewcommand{\childdocforward}[2][]{}
  \renewcommand{\childdocdisable}{}
}
%    \end{macrocode}

% \macro{\childdocmain}
% The macro |\childdocmain| is to be called at the top of the main file
% with nothing or the main filename (without extension) as argument.
% First, it breaks loops.
% If the argument is not empty and does not match |\childdocname|
% (which is set by the first inclusion of |childdoc.def|),
% |\ifchilddoc| is set to true, |\includeonly| is applied to the child file
% and |\jobname| is set to the main file
% (for proper handling of |.aux| files):
%    \begin{macrocode}
\newcommand{\childdocmain}[1]
{
  \childdocdisable\childdocmain{}
  \if?#1?\else
    \begingroup
      \def\childdoctmp{#1}
      \ifx\childdoctmp\childdocname
        \def\childdoctmp{}
      \else
        \def\childdoctmp
        {
          \childdoctrue
          \includeonly{\childdocname}
          \def\childdocjob{#1}
          \def\jobname{#1}
        }
      \fi
      \expandafter
    \endgroup
    \childdoctmp
  \fi
}
%    \end{macrocode}

% \macro{\childdocof}
% The command |\childdocof| redirects
% compilation to the main file |#1|.
%    \begin{macrocode}
\newcommand{\childdocof}[1]
{
  \childdocdisable
  \childdoctrue
  \includeonly{\childdocname}
  \def\jobname{#1}
  \def\childdocjob{#1}
  \input{#1}
}
%    \end{macrocode}

% \macro{\childdocby}
% The command |\childdocby| ....
%    \begin{macrocode}
\newcommand{\childdocby}[2][]
{
  \childdocdisable
  \childdoctrue
  \childdocmanualtrue
  \if?#1?\else
    \def\jobname{#2}
  \fi
  \def\childdocjob{#2}
  \input{#2}
  \endinput
}
%    \end{macrocode}

% \macro{\childdocforward}
% The command |\childdocforward| redirects
% compilation to the main file or
% (if the optional argument is given) a child file.
% Parameters are set as if the main file
% or a child file starting with |\childdocof| was compiled.
% Then compilation is handed over to the main file:
%    \begin{macrocode}
\newcommand{\childdocforward}[2][]
{
  \begingroup
    \if?#1?
      \def\childdoctmp
      {
        \def\childdocname{#2}
        \def\childdocjob{#2}
        \def\jobname{#2}
        \input{#2}
        \endinput
      }
    \else
      \def\childdoctmp
      {
        \childdocdisable
        \def\childdocname{#2}
        \childdoctrue
        \includeonly{#2}
        \def\childdocjob{#1}
        \def\jobname{#1}
        \input{#1}
        \endinput
      }
    \fi
    \expandafter
  \endgroup
  \childdoctmp
}
%    \end{macrocode}

% \macro{\childdocforwardprefix}
% The command |\childdocforwardprefix| redirects
% compilation to the main or a child file by means of a pattern.
% The prefix |#1| in the current filename is replaced by |#2|
% and the suffix of the current filename is kept
% (it is assumed that the filename does not contain the substring `|~~~|'
% which is used as a delimiter).
% Compilation is handed over to the new file by |\childdocforward|:
%    \begin{macrocode}
\newcommand{\childdocforwardprefix}[3][]
{
  \begingroup
    \def\childdocextract #2##1~~~{\def\childdoctmp{\childdocforward[#1]{#3##1}}}
    \expandafter\childdocextract\childdocname~~~
    \expandafter
  \endgroup
  \childdoctmp
}
%    \end{macrocode}

% \macro{\childdoc}
% The deprecated macro |\childdoc| is a legacy version of |\childdocmain|:
%    \begin{macrocode}
\newcommand{\childdoc}{\childdocmain}
%    \end{macrocode}

% \macro{\childdocredirect}
% The deprecated macro |\childdocredirect| is a legacy version
% of |\childdocforward| and |\childdocforwardprefix|:
%    \begin{macrocode}
\newcommand{\childdocredirect}[2][]
{
  \begingroup
    \if?#1?
      \def\childdoctmp{\childdocforward{#2}}
    \else
      \def\childdoctmp{\childdocforwardprefix{#1}{#2}}
    \fi
    \expandafter
  \endgroup
  \childdoctmp
}
%    \end{macrocode}

%\iffalse
%</package>
%\fi
%
\endinput
|\\
|\childdocmain{}|\\
\end{tabular}
\end{center}
at the very top of the main \LaTeX{} file,
in particular \emph{before} the |\documentclass| statement!
The argument of |\childdocmain| should be left empty
(but it must be present).

%%%%%%%%%%%%%%%%%%%%%%%%%%%%%%%%%%%%%%%%
\DescribeMacro{\childdocof}
Furthermore, add the commands
\begin{center}
\begin{tabular}{l}
|% \iffalse
%
% childdoc.dtx Copyright (C) 2017-2018 Niklas Beisert
%
% This work may be distributed and/or modified under the
% conditions of the LaTeX Project Public License, either version 1.3
% of this license or (at your option) any later version.
% The latest version of this license is in
%   http://www.latex-project.org/lppl.txt
% and version 1.3 or later is part of all distributions of LaTeX
% version 2005/12/01 or later.
%
% This work has the LPPL maintenance status `maintained'.
%
% The Current Maintainer of this work is Niklas Beisert.
%
% This work consists of the files childdoc.dtx and childdoc.ins
% and the derived files childdoc.def and cdocsamp.tex with
% cdocsch1.tex, cdocsch2.tex, cdocsdrf.tex, cdocsfn1.tex, cdocsfn2.tex.
%
%<package>\ifdefined\childdocmain\endinput\fi
%<package>\ProvidesFile{childdoc.def}[2018/12/30 v2.0 child document driver]
%<samplemain>\ProvidesFile{cdocsamp.tex}[2018/12/30 v2.0 sample for childdoc]
%<*driver>
%\ProvidesFile{childdoc.drv}[2018/12/30 v2.0 childdoc reference manual file]
\PassOptionsToClass{10pt,a4paper}{article}
\documentclass{ltxdoc}

\usepackage[margin=35mm]{geometry}
\usepackage{hyperref}
\usepackage{hyperxmp}
\usepackage[usenames]{color}

\hypersetup{colorlinks=true}
\hypersetup{pdfstartview=FitH}
\hypersetup{pdfpagemode=UseNone}
\hypersetup{pdfsource={}}
\hypersetup{pdflang={en-UK}}
\hypersetup{pdfcopyright={Copyright 2017-2018 Niklas Beisert.
  This work may be distributed and/or modified under the
  conditions of the LaTeX Project Public License, either version 1.3
  of this license or (at your option) any later version.}}
\hypersetup{pdflicenseurl={http://www.latex-project.org/lppl.txt}}
\hypersetup{pdfcontactaddress={ETH Zurich, ITP, HIT K,
  Wolfgang-Pauli-Strasse 27}}
\hypersetup{pdfcontactpostcode={8093}}
\hypersetup{pdfcontactcity={Zurich}}
\hypersetup{pdfcontactcountry={Switzerland}}
\hypersetup{pdfcontactemail={nbeisert@itp.phys.ethz.ch}}
\hypersetup{pdfcontacturl={http://people.phys.ethz.ch/\xmptilde nbeisert/}}

\newcommand{\secref}[1]{\hyperref[#1]{section \ref*{#1}}}

\parskip1ex
\parindent0pt
\let\olditemize\itemize
\def\itemize{\olditemize\parskip0pt}

\begin{document}

\title{The \textsf{childdoc} Package}
\hypersetup{pdftitle={The childdoc Package}}
\author{Niklas Beisert\\[2ex]
  Institut f\"ur Theoretische Physik\\
  Eidgen\"ossische Technische Hochschule Z\"urich\\
  Wolfgang-Pauli-Strasse 27, 8093 Z\"urich, Switzerland\\[1ex]
  \href{mailto:nbeisert@itp.phys.ethz.ch}
  {\texttt{nbeisert@itp.phys.ethz.ch}}}
\hypersetup{pdfauthor={Niklas Beisert}}
\hypersetup{pdfsubject={Manual for the LaTeX2e Package childdoc}}
\date{30 December 2018, \textsf{v2.0}}
\maketitle

\begin{abstract}\noindent
\textsf{childdoc} is a \LaTeXe{} package
that enables the direct compilation
of document sections included by |\include|
to individual files.
\end{abstract}

\begingroup
\parskip0ex
\tableofcontents
\endgroup

%%%%%%%%%%%%%%%%%%%%%%%%%%%%%%%%%%%%%%%%%%%%%%%%%%%%%%%%%%%%%%%%%%%%%%%%%%%%%%%%
%%%%%%%%%%%%%%%%%%%%%%%%%%%%%%%%%%%%%%%%%%%%%%%%%%%%%%%%%%%%%%%%%%%%%%%%%%%%%%%%
\section{Introduction}

\LaTeX{} provides a mechanism to structure a large document (such as a book)
into a main file and several child files (containing the chapters)
using the |\include| command.
This mechanism is beneficial for documents
which span hundreds of pages in order to
make the source file(s) more manageable.
Moreover, compilation can be restricted to
selected child files by means of the |\includeonly| command.
The latter feature can be used to reduce the compilation time while editing
(this was significantly more useful in the earlier days of \LaTeX{})
or to generate a smaller document which is easier to navigate.
Another application of |\includeonly| is to generate
documents consisting of selected parts of the complete document.

However, there are a few drawbacks of the plain |\include| mechanism:
\begin{itemize}
\item
The child files cannot be compiled on their own,
they can only be compiled via the main file.
A naive editing environment
(such as a text editor with an option
to have the current file processed by \LaTeX)
may require one to switch to the main file before compiling;
attempting to compile the child file produces errors.
\item
The main file must be modified (each time)
to adjust the |\includeonly| command
to the present needs. This easily leaves the main file in a messy state.
\item
The generated document will always carry the filename
of the main document. This is inconvenient if
several child files are to be compiled and
to be kept for distribution.
\end{itemize}

The present package provides a simple interface
to make child files individually compilable by \LaTeX{}.
Compiling a child file then has the same effect as compiling
the main file with an |\includeonly| command
to select the appropriate child.
Moreover the generated document will carry the name of the child
rather than the main file.
This resolves all three above issues.

This feature is meant to make the editing of books,
thesis documents and lecture notes somewhat more convenient.
However, the package can also be used efficiently for
composing a series of documents (such as exercise sheets)
which are typically distributed individually.
It then assists the author in generating the individual documents
(potentially in different versions)
as well as a document containing the collected series.
Another application is in developing style files
or other kinds of included material
where compilation of the style file could redirect
to a sample or test file.

%%%%%%%%%%%%%%%%%%%%%%%%%%%%%%%%%%%%%%%%%%%%%%%%%%%%%%%%%%%%%%%%%%%%%%%%%%%%%%%%
%%%%%%%%%%%%%%%%%%%%%%%%%%%%%%%%%%%%%%%%%%%%%%%%%%%%%%%%%%%%%%%%%%%%%%%%%%%%%%%%
\section{Usage}

First of all, the package \textsf{childdoc} is \emph{not} a standard
\LaTeXe{} |.sty| style file! Therefore it needs to be invoked in
a non-standard way.

%%%%%%%%%%%%%%%%%%%%%%%%%%%%%%%%%%%%%%%%%%%%%%%%%%%%%%%%%%%%%%%%%%%%%%%%%%%%%%%%
\subsection{Included Files}
\label{sec:include}

%%%%%%%%%%%%%%%%%%%%%%%%%%%%%%%%%%%%%%%%
\DescribeMacro{\childdocmain}
To use the package, add the commands
\begin{center}
\begin{tabular}{l}
|\input{childdoc.def}|\\
|\childdocmain{}|\\
\end{tabular}
\end{center}
at the very top of the main \LaTeX{} file,
in particular \emph{before} the |\documentclass| statement!
The argument of |\childdocmain| should be left empty
(but it must be present).

%%%%%%%%%%%%%%%%%%%%%%%%%%%%%%%%%%%%%%%%
\DescribeMacro{\childdocof}
Furthermore, add the commands
\begin{center}
\begin{tabular}{l}
|\input{childdoc.def}|\\
|\childdocof{|\textit{main}|}|\\
\end{tabular}
\end{center}
at the top of every child file \textit{child}
which is included by |\include{|\textit{child}|}|
from within the main file
(or at least for those files to be compiled individually).
The argument \textit{main} must be the filename of the main file.

There are a couple of
considerations in setting up the main and child documents:

%%%%%%%%%%%%%%%%%%%%%%%%%%%%%%%%%%%%%%%%
\paragraph{Restrictions.}

Please note the following restrictions:
\begin{itemize}
\item
|\childdocmain| must be called with one argument \textit{main}
to ensure compatibility with earlier version of the package.
It must either be empty (|\childdocmain{}|)
or precisely match the filename of the main file in which it is specified.
See \secref{sec:detection} for further information.
\item
The filename \textit{main} must be specified without the |.tex| extension.
\item
The filename \textit{main} is case sensitive
(even in case-insensitive file systems)
due to internal string comparison.
\item
The argument \textit{main} should be fully expanded, it cannot be a macro.
\item
Subdirectories and special characters should be avoided in filenames.
\item
The command |\childdocmain{|\textit{main}|}| must be followed by a whitespace.
It should not be followed immediately by another command
or by a comment mark `|%|'.
This is because the \TeX{} parser reads the token immediately following
the argument of |\childdocmain| and puts it
at the beginning of every child section;
however, a white\-space is ignored.
\end{itemize}

%%%%%%%%%%%%%%%%%%%%%%%%%%%%%%%%%%%%%%%%
\paragraph{Content of Main File.}

It is advisable to place all content in the child files included by |\include|.
Any output contained in the main file will appear in all child documents
unless suppressed manually;
it cannot be suppressed automatically by the |\includeonly| directive
and thus should normally be avoided.
A method to include some content in the main file
by means of conditional processing is described in \secref{sec:conditional}.

%%%%%%%%%%%%%%%%%%%%%%%%%%%%%%%%%%%%%%%%
\paragraph{Page Numbering.}

When only a part of the document is compiled,
the appropriate numbering of pages
(as well as other status parameters)
is determined from the |.aux| files.
The latter contain information from previous passes.
However this information needs to propagate through
all intermediate child documents.
Therefore the page numbering in child documents may well
be inconsistent until the complete document is compiled at least once.

A useful (if unconventional) way to always ensure a consistent
page numbering is to restart the numbering in each child document
and denote the pages by `\textit{child}|.|\textit{page}'
where \textit{child} represents the chapter/section number of the child file.
This can be achieved by the command
|\numberwithin{page}{|\textit{child}|}|
of the \textsf{amsmath} package
where \textit{child} can be |chapter| or |section|
depending on the chosen structuring.
Alternatively, one can modify the macro |\thepage| appropriately
and reset the counter |page| at the start of each child file.

%%%%%%%%%%%%%%%%%%%%%%%%%%%%%%%%%%%%%%%%%%%%%%%%%%%%%%%%%%%%%%%%%%%%%%%%%%%%%%%%
\subsection{Conditional Processing}
\label{sec:conditional}

The package provides a mechanism to compile different versions
of a document. To customise the versions further some conditional processing
can come in handy to distinguish which version is being compiled.
The package provides two macros to describe the compilation context:

%%%%%%%%%%%%%%%%%%%%%%%%%%%%%%%%%%%%%%%%
\DescribeMacro{\ifchilddoc}
The conditional |\ifchilddoc| distinguishes between the compilation of
child documents and the main document:
%
\begin{center}
|\ifchilddoc |\textit{child-code}| |[|\||else |\textit{main-code}]| \||fi|
\end{center}

%%%%%%%%%%%%%%%%%%%%%%%%%%%%%%%%%%%%%%%%
\DescribeMacro{\childdocname}
\DescribeMacro{\childdocjob}
The macro |\childdocname| contains the filename (without extension)
of the main or child file being processed.
Note that |\childdocjob| will always contain the name of the main file.

%%%%%%%%%%%%%%%%%%%%%%%%%%%%%%%%%%%%%%%%
\paragraph{Title Page.}

Conditional processing can be used to include a title or banner page
in the main document when proper precautions are taken.
Importantly, the code in the main file should ensure that the page counter
(as well as other status parameters which are stored in the |.aux| files)
takes the same value after the conditional processing.
Otherwise the page numbers may take divergent values
depending on which part is compiled.

For example, a title page could be declared by:
%
\begin{center}
\begin{tabular}{l}
|\ifchilddoc\||else|\\
|\addtocounter{page}{-1}|\\
\textit{code for title page}\\
|\newpage|\\
|\||fi|
\end{tabular}
\end{center}
%
A banner page for the child documents can be generated by:
%
\begin{center}
\begin{tabular}{l}
|\ifchilddoc|\\
|\addtocounter{page}{-1}|\\
\textit{code for banner page}\\
|\newpage|\\
|\||fi|
\end{tabular}
\end{center}
%
Here one could write a message such as:
\begin{center}
|This is the part \childdocname{} of \childdocjob{}.|
\end{center}

%%%%%%%%%%%%%%%%%%%%%%%%%%%%%%%%%%%%%%%%%%%%%%%%%%%%%%%%%%%%%%%%%%%%%%%%%%%%%%%%
\subsection{Flags}
\label{sec:flags}

The package makes it easy to generate different versions
of the main or child documents.
To this end compilation flags can be defined
and assigned different default values.
They will be particularly useful in conjunction
with the forwarding mechanism described in \secref{sec:forward}.

For example, it may be useful to have a flag |\version|
which can be set to |draft| or |final|.
The document source will contain some conditional code
depending on the value of |\version|.
Suppose further, the flag should default to |final| for the main file
and to |draft| for child files
which is a natural assignment for editing the document.
This is achieved by placing the following code
in the preamble of the main document
(below the |\childdocmain| directive):
%
\begin{center}
\begin{tabular}{l}
|\ifchilddoc|\\
|\providecommand{\version}{draft}|\\
|\||else|\\
|\providecommand{\version}{final}|\\
|\||fi|
\end{tabular}
\end{center}
%
The definition by |\providecommand| makes sure
that previous definitions are not overwritten.
Further statements |\providecommand{\version}{...}|
can thus be added before the above code to override it.

For the main file, one might add a line
(between |\childdocmain| and the above block)
%
\begin{center}
|%\ifchilddoc\||else\providecommand{\version}{draft}\||fi|
\end{center}
%
which can be uncommented to produce a draft version.
Likewise one can add a line to the very top of a child file
(above the |\childdocof{|\textit{main}|}| directive)
%
\begin{center}
|%\providecommand{\version}{final}|
\end{center}
%
which can be uncommented to produce the final version of this child document.

%%%%%%%%%%%%%%%%%%%%%%%%%%%%%%%%%%%%%%%%%%%%%%%%%%%%%%%%%%%%%%%%%%%%%%%%%%%%%%%%
\subsection{Forwarding}
\label{sec:forward}

Different versions of the main or child documents
using compilation flags as described in \secref{sec:flags}
can be (permanently) stored in different files
for convenient compilation, viewing and distribution.
To this end, the package defines a command
to pass on compilation to a different file:

%%%%%%%%%%%%%%%%%%%%%%%%%%%%%%%%%%%%%%%%
\DescribeMacro{\childdocforward}
The command |\childdocforward| redirects processing to
another source file:
%
\begin{center}
\begin{tabular}{l}
|\input{childdoc.def}|\\
|\childdocforward[|\textit{main}|]{|\textit{dest}|}|\\
\end{tabular}
\end{center}
%
The argument \textit{dest} is the destination file
(without extension).
It should be the main file or one of the child files.
Note that further \textsf{childdoc} directives
such as |\childdocof| and |\childdocforward|
in the indicated file will be processed in this form.
The optional argument \textit{main}
passes on directly to the main file \textit{main}
while pretending to compile the child \textit{dest}.
This form behaves as if \textit{dest}
issues |\childdocof{|\textit{main}|}| right away,
and no further \textsf{childdoc} directives will be processed.

%%%%%%%%%%%%%%%%%%%%%%%%%%%%%%%%%%%%%%%%
\DescribeMacro{\...prefix}
In the alternative form |\childdocforwardprefix|,
%
\begin{center}
\begin{tabular}{l}
|\input{childdoc.def}|\\
|\childdocforwardprefix[|\textit{main}|]{|\textit{prefix}|}{|\textit{dest}|}|
\end{tabular}
\end{center}
%
the destination file is determined by a pattern
depending on the current file:
To make this work, the current file must be called
`{\textit{prefix}\hspace{0.2em}\textit{suffix}}'
with \textit{prefix} matching precisely the argument.
Processing is then passed on to the file
`{\textit{dest}\hspace{0.2em}\textit{suffix}}'.
Surely, the same effect is achieved by
directly specifying the
argument `{\textit{dest}\hspace{0.2em}\textit{suffix}}'
in the first form.
However, that requires to set up a different file
for each child. With the alternative form of the command
all these files can have exactly the same content
which simplifies setting them up and maintaining them.

For example, the following file |draft.tex|
with a compilation flag |\version| as described in \secref{sec:flags}
compiles the main document as a draft:
%
\begin{center}
\begin{tabular}{l}
|\def\version{draft}|\\
|\input{childdoc.def}|\\
|\childdocforward{|\textit{main}|}|
\end{tabular}
\end{center}
%
Likewise, the following files |final|\textit{nn}|.tex|
compile the final version of the child document
|child|\textit{nn}|.tex|:
%
\begin{center}
\begin{tabular}{l}
|\def\version{final}|\\
|\input{childdoc.def}|\\
|\childdocforwardprefix{final}{child}|
\end{tabular}
\end{center}
%

Note that when several versions of a main file and/or of each child file
are to be generated, it may be convenient to set up a |Makefile| or
shell script to automatise the process.

%%%%%%%%%%%%%%%%%%%%%%%%%%%%%%%%%%%%%%%%%%%%%%%%%%%%%%%%%%%%%%%%%%%%%%%%%%%%%%%%
\subsection{Command Line Processing}
\label{sec:commandline}

The effect of redirection files can also be achieved by invoking
the \LaTeX{} compiler with a more elaborate command line.
Most conveniently this should be done as part
of a shell script or a |Makefile|.

When using \textsf{childdoc} in the main file, the following
command lines effectively perform a redirection
(note that depending on the shell being used,
backslashes may have to be doubled: `|\|' $\to$ `|\\|'):
%
\begin{center}
|... -jobname "|\textit{target}|" |\\|"|[\textit{flags}]%
|\input{childdoc.def}\childdocforward[|\textit{main}|]{|\textit{dest}|}"|
\end{center}
%
Here \textit{target} is the name of the output file,
\textit{main} is the name of the main file
and \textit{dest} is the name of the main or child file to be processed
(all filenames without extensions).
The optional argument \textit{main} can be omitted
if \textit{main} matches \textit{dest}.
Optionally, compilation \textit{flags} can be defined via |\def| commands.
This command line makes the \TeX{} engine believe
it is compiling the file \textit{target}
whose content is specified as the latter parameter.
The provided code then forwards the processing to
\textit{main} or \textit{dest} as described in \secref{sec:forward}.

%%%%%%%%%%%%%%%%%%%%%%%%%%%%%%%%%%%%%%%%%%%%%%%%%%%%%%%%%%%%%%%%%%%%%%%%%%%%%%%%
\subsection{Include by Input}
\label{sec:input}

Including child documents by |\include| has some restrictions by design.
Most notably, the content of a child document always occupies
its own set of pages; pages cannot be shared between child documents.
Usually, this behaviour makes perfect sense
because each child document contain an essential part of the document.
However, in some situations it may be desirable to compose
a document from a collection of parts
without having mandatory page breaks between then.
For this case, the package
provides a mechanism to include parts
by |\input| which can also be processed individually.
However, by construction this mechanism
requires manual handling of the content to be output.

%%%%%%%%%%%%%%%%%%%%%%%%%%%%%%%%%%%%%%%%
\DescribeMacro{\ifchilddocmanual}
The main file should be prepared as usual, see \secref{sec:include}.
However, the document body must make a distinction
between processing of an individual part and of the main document, e.g.:
%
\begin{center}
\begin{tabular}{l}
|\ifchilddocmanual|\\
|\input{\childdocname}|\\
|\||else|\\
\textit{document body with }|\input{|\textit{part}|}|\\
|\||fi|
\end{tabular}
\end{center}
%
The conditional |\ifchilddocmanual| is true whenever
a part to be included by |\input| is being compiled,
and the name of the part is stored in |\childdocname|.

%%%%%%%%%%%%%%%%%%%%%%%%%%%%%%%%%%%%%%%%
\DescribeMacro{\childdocby}
Each part to be included by |\input| should start with:
%
\begin{center}
\begin{tabular}{l}
|\input{childdoc.def}|\\
|\childdocby{|\textit{main}|}|\\
\end{tabular}
\end{center}
%
The directive |\childdocby| is similar to |\childdocof|
described in \secref{sec:include},
but the subsequent selection of content must be done manually.
To that end, both |\ifchilddoc| and |\ifchilddocmanual|
will be true upon processing of a part,
and the name of the part is stored in |\childdocname|.
Note that |\jobname| will be set to the filename of the current part
so that each part receives an individual |.aux| file
that does not interfere with the |.aux| file(s) of the main document.
This behaviour can be altered by the alternative form
|\childdocby[*]{|\textit{main}|}| (with a non-empty optional argument)
which uses the |.aux| file of the main document
by setting |\jobname| to \textit{main}.

%%%%%%%%%%%%%%%%%%%%%%%%%%%%%%%%%%%%%%%%%%%%%%%%%%%%%%%%%%%%%%%%%%%%%%%%%%%%%%%%
\subsection{Driver Development}
\label{sec:driver}

The \textsf{childdoc} mechanism can also be use for the development
of definition files such as \LaTeX{} styles or classes.
This case differs from the above setup with multiple parts
included by |\include| in that no |\includeonly| should be invoked.
This can be achieved by starting the include file
(before |\ProvidesPackage|) with:
%
\begin{center}
\begin{tabular}{l}
|\input{childdoc.def}|\\
|\childdocforward{|\textit{main}|}|\\
\end{tabular}
\end{center}
%
or alternatively with:
%
\begin{center}
\begin{tabular}{l}
|\input{childdoc.def}|\\
|\childdocby{|\textit{main}|}|\\
\end{tabular}
\end{center}
%
Both forms have slightly different effects as described above.
The main file is prepared as usual, see \secref{sec:include}.

%%%%%%%%%%%%%%%%%%%%%%%%%%%%%%%%%%%%%%%%%%%%%%%%%%%%%%%%%%%%%%%%%%%%%%%%%%%%%%%%
\subsection{Legacy Detection}
\label{sec:detection}

The directive |\childdocmain| in the main file can detect
whether the complete document or merely a child is to be compiled
even without using the directive |\childdocof|.
This method is deprecated because it is less robust
and there is no compelling reason to use it;
it is merely provided for backward compatibility
and it may be removed in future versions.

If the detection mechanism is to be used,
it is mandatory to correctly specify
the filename of the main file as the argument of |\childdocmain|:
%
\begin{center}
\begin{tabular}{l}
|\input{childdoc.def}|\\
|\childdocmain{|\textit{main}|}|\\
\end{tabular}
\end{center}
%
If |\jobname| does not match the argument \textit{main} of |\childdocmain|,
it is assumed that |\jobname| points to the child file to be compiled.
When using |\childdocmain| with the main file specified as argument,
it suffices to start a child file
with just |\input{|\textit{main}|}|
without loading of the package and using |\childdocof|.
If instead all processing is done
with the appropriate \textsf{childdoc} directives,
the argument of \textit{main} of |\childdocmain| can be empty.

An alternative version of the command line processing described
in \secref{sec:commandline} using the detection mechanism reads:
%
\begin{center}
|... -jobname "|\textit{target}|" "|[\textit{flags}]%
[|\def\jobname{|\textit{dest}|}|]|\input{|\textit{main}|}"|
\end{center}

%%%%%%%%%%%%%%%%%%%%%%%%%%%%%%%%%%%%%%%%%%%%%%%%%%%%%%%%%%%%%%%%%%%%%%%%%%%%%%%%
\subsection{Manual Code}
\label{sec:manual}

In case one cannot be certain whether the definitions file |childdoc.def|
is installed on the target \TeX{} distribution
and one prefers not to ship it,
it is conceivable to paste a few relevant commands into the sources.

To that end, drop all statements |\input{childdoc.def}|
and perform the replacements as outlined below.
Instead of |\childdocmain{|\textit{main}|}| add the following code
to the top of the main file:
%
\begin{center}
\begin{tabular}{l}
|\||ifdefined\childdocname\endinput\||fi\newif\ifchilddoc|\\
|\edef\childdocname{\scantokens\expandafter{\jobname\noexpand}}|\\
|\def\childdocmain{|\textit{main}|}\||ifx\childdocmain\childdocname\||else|\\
|\childdoctrue\includeonly{\childdocname}\let\jobname\childdocmain\||fi|\\
\end{tabular}
\end{center}
%
Instead of |\childdocof{|\textit{main}|}| just include the main file
at the top of each child file:
%
\begin{center}
|\input{|\textit{main}|}|
\end{center}
%
A simple redirection |\childdocforward{|\textit{dest}|}| is achieved by:
%
\begin{center}
|\def\jobname{|\textit{dest}|}\input{\jobname}|
\end{center}
%
The redirection with prefix
|\childdocforwardprefix[|\textit{prefix}|]{|\textit{dest}|}|
is accomplished by:
%
\begin{center}
\begin{tabular}{l}
|{\edef\jobname{\scantokens\expandafter{\jobname\noexpand}}|\\
|\def\redirectjob |\textit{prefix}|#1~~~{\gdef\jobname{|\textit{dest}|#1}}|\\
|\expandafter\redirectjob\jobname~~~}\input{\jobname}|
\end{tabular}
\end{center}

In an alternative approach,
child documents can be compiled by a specific command line
without additional code or specific definitions:
%
\begin{center}
|... -jobname "|\textit{target}|" "|[\textit{flags}]%
|\includeonly{|\textit{dest}|}\input{|\textit{main}|}"|
\end{center}
%

%%%%%%%%%%%%%%%%%%%%%%%%%%%%%%%%%%%%%%%%%%%%%%%%%%%%%%%%%%%%%%%%%%%%%%%%%%%%%%%%
%%%%%%%%%%%%%%%%%%%%%%%%%%%%%%%%%%%%%%%%%%%%%%%%%%%%%%%%%%%%%%%%%%%%%%%%%%%%%%%%
\section{Information}

%%%%%%%%%%%%%%%%%%%%%%%%%%%%%%%%%%%%%%%%%%%%%%%%%%%%%%%%%%%%%%%%%%%%%%%%%%%%%%%%
\subsection{Copyright}

Copyright \copyright{} 2017--2018 Niklas Beisert

This work may be distributed and/or modified under the
conditions of the \LaTeX{} Project Public License, either version 1.3
of this license or (at your option) any later version.
The latest version of this license is in
  \url{http://www.latex-project.org/lppl.txt}
and version 1.3 or later is part of all distributions of \LaTeX{}
version 2005/12/01 or later.

This work has the LPPL maintenance status `maintained'.

The Current Maintainer of this work is Niklas Beisert.

This work consists of the files |README.txt|, |childdoc.ins| and |childdoc.dtx|
as well as the derived files |childdoc.def|, |cdocsamp.tex|
with |cdocsch1.tex|, |cdocsch2.tex|, |cdocspt3.tex|, |cdocspt4.tex|,
|cdocsdrf.tex|, |cdocsfn1.tex|, |cdocsfn2.tex|
as well as |childdoc.pdf|.

%%%%%%%%%%%%%%%%%%%%%%%%%%%%%%%%%%%%%%%%%%%%%%%%%%%%%%%%%%%%%%%%%%%%%%%%%%%%%%%%
\subsection{Files and Installation}

The package consists of the files:
%
\begin{center}
\begin{tabular}{ll}
    |README.txt|   & readme file \\
    |childdoc.ins| & installation file \\
    |childdoc.dtx| & source file \\
    |childdoc.def| & definition file \\
    |cdocsamp.tex| & sample main file \\
    |cdocsch1.tex| & sample include file \\
    |cdocsch2.tex| & sample include file \\
    |cdocspt3.tex| & sample part file \\
    |cdocspt4.tex| & sample part file \\
    |cdocsdrf.tex| & sample redirection file \\
    |cdocsfn1.tex| & sample redirection file \\
    |cdocsfn2.tex| & sample redirection file \\
    |childdoc.pdf| & manual
\end{tabular}
\end{center}
%
The distribution consists of the files
|README.txt|, |childdoc.ins| and |childdoc.dtx|.
%
\begin{itemize}
\item
Run (pdf)\LaTeX{} on |childdoc.dtx|
to compile the manual |childdoc.pdf| (this file).
\item
Run \LaTeX{} on |childdoc.ins| to create the definitions file |childdoc.def|
and the sample |cdocsamp.tex| with include files
|cdocsch1.tex|, |cdocsch2.tex|, |cdocspt3.tex|, |cdocspt4.tex|,
|cdocsdrf.tex|, |cdocsfn1.tex|, |cdocsfn2.tex|.
Then copy the file |childdoc.def| to an appropriate directory of your \LaTeX{}
distribution, e.g.\ \textit{texmf-root}|/tex/latex/childdoc|.
\end{itemize}

%%%%%%%%%%%%%%%%%%%%%%%%%%%%%%%%%%%%%%%%%%%%%%%%%%%%%%%%%%%%%%%%%%%%%%%%%%%%%%%%
\subsection{Related CTAN Packages}

There are several other packages which offer a similar functionality:
%
\begin{itemize}
\item
The packages
\href{http://ctan.org/pkg/docmute}{\textsf{docmute}},
\href{http://ctan.org/pkg/includex}{\textsf{includex}} and
\href{http://ctan.org/pkg/standalone}{\textsf{standalone}}
provide commands to include only the document body of
a child file thus allowing both files to be compiled individually.
\item
The packages \href{http://ctan.org/pkg/subdocs}{\textsf{subdocs}}
and \href{http://ctan.org/pkg/subfiles}{\textsf{subfiles}}
provide structures in which the main and child documents can be
encapsulated and allowing them to be compiled individually.
The inclusion mechanism is different from the conventional |\include|.
\item
The package \href{http://ctan.org/pkg/combine}{\textsf{combine}}
is an elaborate solution to combine several documents into one.
\end{itemize}
%
See also the CTAN topic \href{http://ctan.org/topic/subdocs}{\textsf{subdocs}}
for further related packages.
The present package differs from the above solutions in that
a document structure constructed with the conventional |\include| mechanism
just needs two extra commands at the top of every file
such that all constituent files can be compiled individually.

%%%%%%%%%%%%%%%%%%%%%%%%%%%%%%%%%%%%%%%%%%%%%%%%%%%%%%%%%%%%%%%%%%%%%%%%%%%%%%%%
%\subsection{Feature Suggestions}
%
%The following is a list of features which may be useful for future
%versions of this package:
%%
%\begin{itemize}
%\item
%\ldots
%\end{itemize}

%%%%%%%%%%%%%%%%%%%%%%%%%%%%%%%%%%%%%%%%%%%%%%%%%%%%%%%%%%%%%%%%%%%%%%%%%%%%%%%%
\subsection{Revision History}

%%%%%%%%%%%%%%%%%%%%%%%%%%%%%%%%%%%%%%%%
\paragraph{v2.0:} 2018/12/30

\begin{itemize}
\item
immediate forward processing
\item
added |\childdocby| mechanism
\item
manual restructured
\end{itemize}

%%%%%%%%%%%%%%%%%%%%%%%%%%%%%%%%%%%%%%%%
\paragraph{v1.6:} 2018/01/17

\begin{itemize}
\item
application for development of include files
\item
corrections to manual
\end{itemize}

%%%%%%%%%%%%%%%%%%%%%%%%%%%%%%%%%%%%%%%%
\paragraph{v1.5:} 2017/05/21

\begin{itemize}
\item
more complete structuring introduced
\item
|\childdocof| introduced
\item
|\childdoc| renamed to |\childdocmain|
\item
|\childredirect| renamed to |\childdocforward| and |\childdocforwardprefix|
and functionality expanded
\end{itemize}

%%%%%%%%%%%%%%%%%%%%%%%%%%%%%%%%%%%%%%%%
\paragraph{v1.0:} 2017/04/27

\begin{itemize}
\item
manual and install package
\item
first version published on CTAN
\end{itemize}

%%%%%%%%%%%%%%%%%%%%%%%%%%%%%%%%%%%%%%%%
\paragraph{v0.6:} 2017/04/26

\begin{itemize}
\item
redirection mechanism added
\end{itemize}

%%%%%%%%%%%%%%%%%%%%%%%%%%%%%%%%%%%%%%%%
\paragraph{v0.5:} 2017/04/26

\begin{itemize}
\item
functionality in definition file
\end{itemize}


%%%%%%%%%%%%%%%%%%%%%%%%%%%%%%%%%%%%%%%%%%%%%%%%%%%%%%%%%%%%%%%%%%%%%%%%%%%%%%%%
%%%%%%%%%%%%%%%%%%%%%%%%%%%%%%%%%%%%%%%%%%%%%%%%%%%%%%%%%%%%%%%%%%%%%%%%%%%%%%%%
%%%%%%%%%%%%%%%%%%%%%%%%%%%%%%%%%%%%%%%%%%%%%%%%%%%%%%%%%%%%%%%%%%%%%%%%%%%%%%%%
\appendix

\settowidth\MacroIndent{\rmfamily\scriptsize 000\ }

 \DocInput{childdoc.dtx}

\end{document}
%</driver>
% \fi
%
% %%%%%%%%%%%%%%%%%%%%%%%%%%%%%%%%%%%%%%%%%%%%%%%%%%%%%%%%%%%%%%%%%%%%%%%%%%%%%%
% %%%%%%%%%%%%%%%%%%%%%%%%%%%%%%%%%%%%%%%%%%%%%%%%%%%%%%%%%%%%%%%%%%%%%%%%%%%%%%
% \section{Sample}
%\iffalse
%<*samplemain>
%\fi
%
% The following presents a sample document
% with two chapters, two parts, a title page,
% a compile flag as well as three forwarding files to set the flag.
% It consists of eight |.tex| files:
% \begin{center}
% \begin{tabular}{ll}
% |cdocsamp.tex|&main file\\
% |cdocsch1.tex|&include file for chapter 1\\
% |cdocsch2.tex|&include file for chapter 2\\
% |cdocspt3.tex|&include file for part 3\\
% |cdocspt4.tex|&include file for part 4\\
% |cdocsdrf.tex|&forwarding file for main file in draft mode\\
% |cdocsfi1.tex|&forwarding file for final version of chapter 1\\
% |cdocsfi2.tex|&forwarding file for final version of chapter 2\\
% \end{tabular}
% \end{center}
% Each of the eight files can be compiled directly by the \LaTeX{} compiler.
%
% %%%%%%%%%%%%%%%%%%%%%%%%%%%%%%%%%%%%%%
% \paragraph{Main File.}
%
% The main file is called |cdocsamp.tex|.
%
% Load the \textsf{childdoc} definitions and
% declare the filename for the main document:
%    \begin{macrocode}
\input{childdoc.def}
\childdocmain{}
%    \end{macrocode}

% Optional override for |\version| flag:
%    \begin{macrocode}
%%\ifchilddoc\else\providecommand{\version}{draft}\fi
%    \end{macrocode}

% Define the default values for the |\version| flag
% (|final| for the main file and |draft| for childs):
%    \begin{macrocode}
\ifchilddoc
\providecommand{\version}{draft}
\else
\providecommand{\version}{final}
\fi
%    \end{macrocode}

% Load the standard document class:
%    \begin{macrocode}
\documentclass[12pt]{article}
%    \end{macrocode}

% Start the document body:
%    \begin{macrocode}
\begin{document}
%    \end{macrocode}

% Declare a title page.
% Print title, part of document being processed and version flag:
%    \begin{macrocode}
\addtocounter{page}{-1}
\begin{center}
{\LARGE\bfseries{}childdoc example\par}
\vspace{1cm}
\ifchilddoc
\ifchilddocmanual part\else chapter\fi:
`\childdocname' of `\childdocjob'\par
\else
main document: `\childdocjob'\par
\fi
version: \version\par
\end{center}
\newpage
%    \end{macrocode}

% Manually include selected file,
% otherwise process as usual:
%    \begin{macrocode}
\ifchilddocmanual
\section*{part `\childdocname'}
\input{\childdocname}
\else
%    \end{macrocode}

% Include the two chapters:
%    \begin{macrocode}
\include{cdocsch1}
\include{cdocsch2}
%    \end{macrocode}

% Include the two parts unless only chapters should be displayed:
%    \begin{macrocode}
\ifchilddoc\else
\section{part three}
\input{cdocspt3}
\section{part four}
\input{cdocspt4}
\fi
%    \end{macrocode}

% Process as usual until here:
%    \begin{macrocode}
\fi
%    \end{macrocode}

% End of document body:
%    \begin{macrocode}
\end{document}
%    \end{macrocode}
%\iffalse
%</samplemain>
%\fi
%
% %%%%%%%%%%%%%%%%%%%%%%%%%%%%%%%%%%%%%%
% \paragraph{Chapter Include Files.}
%
% The include files are called |cdocsch1.tex| and |cdocsch2.tex|.
%
%\iffalse
%<*samplechap1|samplechap2>
%\fi

% Optional override for |\version| flag:
%    \begin{macrocode}
%%\providecommand{\version}{final}
%    \end{macrocode}

% Include the main document:
%    \begin{macrocode}
\input{childdoc.def}
\childdocof{cdocsamp}
%    \end{macrocode}

%\iffalse
%</samplechap1|samplechap2>
%\fi
%
%\iffalse
%<*samplechap1>
%\fi
% Some text for chapter 1:
%    \begin{macrocode}
\section{one}
some text in chapter one
%    \end{macrocode}

%\iffalse
%</samplechap1>
%\fi
% Some text for chapter 2:
%\iffalse
%<*samplechap2>
%\fi
%    \begin{macrocode}
\section{two}
more text in chapter two
%    \end{macrocode}

%\iffalse
%</samplechap2>
%\fi
%
% %%%%%%%%%%%%%%%%%%%%%%%%%%%%%%%%%%%%%%
% \paragraph{Part Include Files.}
%
% The include files are called |cdocspt3.tex| and |cdocspt4.tex|.
%
%\iffalse
%<*samplepart3|samplepart4>
%\fi

% Optional override for |\version| flag:
%    \begin{macrocode}
%%\providecommand{\version}{final}
%    \end{macrocode}

% Include the main document:
%    \begin{macrocode}
\input{childdoc.def}
\childdocby{cdocsamp}
%    \end{macrocode}

%\iffalse
%</samplepart3|samplepart4>
%\fi
%
%\iffalse
%<*samplepart3>
%\fi
% Some text for part 3:
%    \begin{macrocode}
some text in part three
%    \end{macrocode}

%\iffalse
%</samplepart3>
%\fi
% Some text for part 4:
%\iffalse
%<*samplepart4>
%\fi
%    \begin{macrocode}
more text in part four
%    \end{macrocode}

%\iffalse
%</samplepart4>
%\fi
%
% %%%%%%%%%%%%%%%%%%%%%%%%%%%%%%%%%%%%%%
% \paragraph{Forwarding for a Complete Draft.}
%
% The following forwarding file |cdocsdrf.tex|
% compiles the main document in draft mode:
%\iffalse
%<*sampledraft>
%\fi
%    \begin{macrocode}
\def\version{draft}
\input{childdoc.def}
\childdocforward{cdocsamp}
%    \end{macrocode}

%\iffalse
%</sampledraft>
%\fi
%
% %%%%%%%%%%%%%%%%%%%%%%%%%%%%%%%%%%%%%%
% \paragraph{Forwarding for Final Version of the Chapters.}
%
% The following forwarding files |cdocsfn1.tex| and |cdocsfn2.tex|
% (with identical content)
% compile the final versions of the child documents
% |cdocsch1.tex| and |cdocsch2.tex|, respectively:
%\iffalse
%<*samplefinal>
%\fi
%    \begin{macrocode}
\def\version{final}
\input{childdoc.def}
\childdocforwardprefix[cdocsamp]{cdocsfn}{cdocsch}
%    \end{macrocode}

%\iffalse
%</samplefinal>
%\fi
%
% %%%%%%%%%%%%%%%%%%%%%%%%%%%%%%%%%%%%%%
% \paragraph{Command Line Processing.}
%
% The following three command lines generate the output files
% |cdocscld|, |cdocscl1| and |cdocscl2|
% which should be identical to
% |cdocsdrf|, |cdocsch1| and |cdocsfn2|, respectively:
% \begin{center}
% \begin{tabular}{l}
% |latex -jobname cdocscld \|\\
% |  "\def\version{draft}\input{childdoc.def}\childdocforward{cdocsamp}"|\\
% |latex -jobname cdocscl1 \|\\
% |  "\input{childdoc.def}\childdocforward[cdocsamp]{cdocsch1}"|\\
% |latex -jobname cdocscl2 \|\\
% |  "\def\version{final}\input{childdoc.def}\childdocforward{cdocsch2}"|
% \end{tabular}
% \end{center}
% Note that the trailing backslash on each first line
% merely continues the input to the second line
% (for convenient cut ant paste).
% Furthermore, the command |latex| can be replaced by any
% of its alternative versions such as |pdflatex|.
%
% %%%%%%%%%%%%%%%%%%%%%%%%%%%%%%%%%%%%%%%%%%%%%%%%%%%%%%%%%%%%%%%%%%%%%%%%%%%%%%
% %%%%%%%%%%%%%%%%%%%%%%%%%%%%%%%%%%%%%%%%%%%%%%%%%%%%%%%%%%%%%%%%%%%%%%%%%%%%%%
% \section{Implementation}
%\iffalse
%<*package>
%\fi
%
% This section describes the definitions file |childdoc.def|.

% The definitions cannot be loaded using |\usepackage| or |\RequirePackage|
% which has a mechanism to prevent loading a style file more than once.
% When loading the definitions by means of |\input|
% multiple instances have to be prevented manually:
%\iffalse
%This code needs to be before the `\ProvidesFile' directive
%which is defined at the beginning of this file.
%Therefore it is also placed there and commented out here.
%</package>
%<*discard>
%\fi
%    \begin{macrocode}
\ifdefined\childdocmain\endinput\fi
%    \end{macrocode}
%\iffalse
%</discard>
%<*package>
%\fi
%
% \macro{\ifchilddoc}
% \macro{\ifchilddocmanual}
% The conditional |\ifchilddoc| tells whether a
% child (true) or main (false) document is being compiled.
% The conditional |\ifchilddocmanual| tells whether
% the |\includeonly| mechanism is used (false) or
% the selection of child files must be performed manually (true).
% The definitions initialise to false:
%    \begin{macrocode}
\newif\ifchilddoc
\newif\ifchilddocmanual
%    \end{macrocode}

% \macro{\childdocname}
% \macro{\childdocjob}
% The macro |\childdocname| stores the name of the main document
% to be compiled. The macro |\childdocjob| stores the name of
% the document on which the \LaTeX{} compiler was originally invoked.
% The content of |\jobname| cannot be compared
% to filenames specified in the source due to different catcodes.
% The following code rescans |\jobname|, stores the result
% in |\childdocname| and saves a copy in |\childdocjob|:
%    \begin{macrocode}
\edef\childdocname{\scantokens\expandafter{\jobname\noexpand}}
\let\childdocjob\childdocname
%    \end{macrocode}

% \macro{\childdocdisable}
% The macro |\childdocdisable| prevents the main file
% from being processed more than once.
% At this stage, the main document command |\childdocmain|
% is assumed to be called once again where it should do nothing.
% Any subsequent call to it should prevent
% a secondary processing of the main document
% It overwrites the forwarding commands
% |\childdocof| and |\childdocforward|
% with empty macros to prevent further inclusions of the main document:
%    \begin{macrocode}
\newcommand{\childdocdisable}
{
  \renewcommand{\childdocmain}[1]{\renewcommand{\childdocmain}[1]{\endinput}}
  \renewcommand{\childdocof}[1]{}
  \renewcommand{\childdocby}[2][]{}
  \renewcommand{\childdocforward}[2][]{}
  \renewcommand{\childdocdisable}{}
}
%    \end{macrocode}

% \macro{\childdocmain}
% The macro |\childdocmain| is to be called at the top of the main file
% with nothing or the main filename (without extension) as argument.
% First, it breaks loops.
% If the argument is not empty and does not match |\childdocname|
% (which is set by the first inclusion of |childdoc.def|),
% |\ifchilddoc| is set to true, |\includeonly| is applied to the child file
% and |\jobname| is set to the main file
% (for proper handling of |.aux| files):
%    \begin{macrocode}
\newcommand{\childdocmain}[1]
{
  \childdocdisable\childdocmain{}
  \if?#1?\else
    \begingroup
      \def\childdoctmp{#1}
      \ifx\childdoctmp\childdocname
        \def\childdoctmp{}
      \else
        \def\childdoctmp
        {
          \childdoctrue
          \includeonly{\childdocname}
          \def\childdocjob{#1}
          \def\jobname{#1}
        }
      \fi
      \expandafter
    \endgroup
    \childdoctmp
  \fi
}
%    \end{macrocode}

% \macro{\childdocof}
% The command |\childdocof| redirects
% compilation to the main file |#1|.
%    \begin{macrocode}
\newcommand{\childdocof}[1]
{
  \childdocdisable
  \childdoctrue
  \includeonly{\childdocname}
  \def\jobname{#1}
  \def\childdocjob{#1}
  \input{#1}
}
%    \end{macrocode}

% \macro{\childdocby}
% The command |\childdocby| ....
%    \begin{macrocode}
\newcommand{\childdocby}[2][]
{
  \childdocdisable
  \childdoctrue
  \childdocmanualtrue
  \if?#1?\else
    \def\jobname{#2}
  \fi
  \def\childdocjob{#2}
  \input{#2}
  \endinput
}
%    \end{macrocode}

% \macro{\childdocforward}
% The command |\childdocforward| redirects
% compilation to the main file or
% (if the optional argument is given) a child file.
% Parameters are set as if the main file
% or a child file starting with |\childdocof| was compiled.
% Then compilation is handed over to the main file:
%    \begin{macrocode}
\newcommand{\childdocforward}[2][]
{
  \begingroup
    \if?#1?
      \def\childdoctmp
      {
        \def\childdocname{#2}
        \def\childdocjob{#2}
        \def\jobname{#2}
        \input{#2}
        \endinput
      }
    \else
      \def\childdoctmp
      {
        \childdocdisable
        \def\childdocname{#2}
        \childdoctrue
        \includeonly{#2}
        \def\childdocjob{#1}
        \def\jobname{#1}
        \input{#1}
        \endinput
      }
    \fi
    \expandafter
  \endgroup
  \childdoctmp
}
%    \end{macrocode}

% \macro{\childdocforwardprefix}
% The command |\childdocforwardprefix| redirects
% compilation to the main or a child file by means of a pattern.
% The prefix |#1| in the current filename is replaced by |#2|
% and the suffix of the current filename is kept
% (it is assumed that the filename does not contain the substring `|~~~|'
% which is used as a delimiter).
% Compilation is handed over to the new file by |\childdocforward|:
%    \begin{macrocode}
\newcommand{\childdocforwardprefix}[3][]
{
  \begingroup
    \def\childdocextract #2##1~~~{\def\childdoctmp{\childdocforward[#1]{#3##1}}}
    \expandafter\childdocextract\childdocname~~~
    \expandafter
  \endgroup
  \childdoctmp
}
%    \end{macrocode}

% \macro{\childdoc}
% The deprecated macro |\childdoc| is a legacy version of |\childdocmain|:
%    \begin{macrocode}
\newcommand{\childdoc}{\childdocmain}
%    \end{macrocode}

% \macro{\childdocredirect}
% The deprecated macro |\childdocredirect| is a legacy version
% of |\childdocforward| and |\childdocforwardprefix|:
%    \begin{macrocode}
\newcommand{\childdocredirect}[2][]
{
  \begingroup
    \if?#1?
      \def\childdoctmp{\childdocforward{#2}}
    \else
      \def\childdoctmp{\childdocforwardprefix{#1}{#2}}
    \fi
    \expandafter
  \endgroup
  \childdoctmp
}
%    \end{macrocode}

%\iffalse
%</package>
%\fi
%
\endinput
|\\
|\childdocof{|\textit{main}|}|\\
\end{tabular}
\end{center}
at the top of every child file \textit{child}
which is included by |\include{|\textit{child}|}|
from within the main file
(or at least for those files to be compiled individually).
The argument \textit{main} must be the filename of the main file.

There are a couple of
considerations in setting up the main and child documents:

%%%%%%%%%%%%%%%%%%%%%%%%%%%%%%%%%%%%%%%%
\paragraph{Restrictions.}

Please note the following restrictions:
\begin{itemize}
\item
|\childdocmain| must be called with one argument \textit{main}
to ensure compatibility with earlier version of the package.
It must either be empty (|\childdocmain{}|)
or precisely match the filename of the main file in which it is specified.
See \secref{sec:detection} for further information.
\item
The filename \textit{main} must be specified without the |.tex| extension.
\item
The filename \textit{main} is case sensitive
(even in case-insensitive file systems)
due to internal string comparison.
\item
The argument \textit{main} should be fully expanded, it cannot be a macro.
\item
Subdirectories and special characters should be avoided in filenames.
\item
The command |\childdocmain{|\textit{main}|}| must be followed by a whitespace.
It should not be followed immediately by another command
or by a comment mark `|%|'.
This is because the \TeX{} parser reads the token immediately following
the argument of |\childdocmain| and puts it
at the beginning of every child section;
however, a white\-space is ignored.
\end{itemize}

%%%%%%%%%%%%%%%%%%%%%%%%%%%%%%%%%%%%%%%%
\paragraph{Content of Main File.}

It is advisable to place all content in the child files included by |\include|.
Any output contained in the main file will appear in all child documents
unless suppressed manually;
it cannot be suppressed automatically by the |\includeonly| directive
and thus should normally be avoided.
A method to include some content in the main file
by means of conditional processing is described in \secref{sec:conditional}.

%%%%%%%%%%%%%%%%%%%%%%%%%%%%%%%%%%%%%%%%
\paragraph{Page Numbering.}

When only a part of the document is compiled,
the appropriate numbering of pages
(as well as other status parameters)
is determined from the |.aux| files.
The latter contain information from previous passes.
However this information needs to propagate through
all intermediate child documents.
Therefore the page numbering in child documents may well
be inconsistent until the complete document is compiled at least once.

A useful (if unconventional) way to always ensure a consistent
page numbering is to restart the numbering in each child document
and denote the pages by `\textit{child}|.|\textit{page}'
where \textit{child} represents the chapter/section number of the child file.
This can be achieved by the command
|\numberwithin{page}{|\textit{child}|}|
of the \textsf{amsmath} package
where \textit{child} can be |chapter| or |section|
depending on the chosen structuring.
Alternatively, one can modify the macro |\thepage| appropriately
and reset the counter |page| at the start of each child file.

%%%%%%%%%%%%%%%%%%%%%%%%%%%%%%%%%%%%%%%%%%%%%%%%%%%%%%%%%%%%%%%%%%%%%%%%%%%%%%%%
\subsection{Conditional Processing}
\label{sec:conditional}

The package provides a mechanism to compile different versions
of a document. To customise the versions further some conditional processing
can come in handy to distinguish which version is being compiled.
The package provides two macros to describe the compilation context:

%%%%%%%%%%%%%%%%%%%%%%%%%%%%%%%%%%%%%%%%
\DescribeMacro{\ifchilddoc}
The conditional |\ifchilddoc| distinguishes between the compilation of
child documents and the main document:
%
\begin{center}
|\ifchilddoc |\textit{child-code}| |[|\||else |\textit{main-code}]| \||fi|
\end{center}

%%%%%%%%%%%%%%%%%%%%%%%%%%%%%%%%%%%%%%%%
\DescribeMacro{\childdocname}
\DescribeMacro{\childdocjob}
The macro |\childdocname| contains the filename (without extension)
of the main or child file being processed.
Note that |\childdocjob| will always contain the name of the main file.

%%%%%%%%%%%%%%%%%%%%%%%%%%%%%%%%%%%%%%%%
\paragraph{Title Page.}

Conditional processing can be used to include a title or banner page
in the main document when proper precautions are taken.
Importantly, the code in the main file should ensure that the page counter
(as well as other status parameters which are stored in the |.aux| files)
takes the same value after the conditional processing.
Otherwise the page numbers may take divergent values
depending on which part is compiled.

For example, a title page could be declared by:
%
\begin{center}
\begin{tabular}{l}
|\ifchilddoc\||else|\\
|\addtocounter{page}{-1}|\\
\textit{code for title page}\\
|\newpage|\\
|\||fi|
\end{tabular}
\end{center}
%
A banner page for the child documents can be generated by:
%
\begin{center}
\begin{tabular}{l}
|\ifchilddoc|\\
|\addtocounter{page}{-1}|\\
\textit{code for banner page}\\
|\newpage|\\
|\||fi|
\end{tabular}
\end{center}
%
Here one could write a message such as:
\begin{center}
|This is the part \childdocname{} of \childdocjob{}.|
\end{center}

%%%%%%%%%%%%%%%%%%%%%%%%%%%%%%%%%%%%%%%%%%%%%%%%%%%%%%%%%%%%%%%%%%%%%%%%%%%%%%%%
\subsection{Flags}
\label{sec:flags}

The package makes it easy to generate different versions
of the main or child documents.
To this end compilation flags can be defined
and assigned different default values.
They will be particularly useful in conjunction
with the forwarding mechanism described in \secref{sec:forward}.

For example, it may be useful to have a flag |\version|
which can be set to |draft| or |final|.
The document source will contain some conditional code
depending on the value of |\version|.
Suppose further, the flag should default to |final| for the main file
and to |draft| for child files
which is a natural assignment for editing the document.
This is achieved by placing the following code
in the preamble of the main document
(below the |\childdocmain| directive):
%
\begin{center}
\begin{tabular}{l}
|\ifchilddoc|\\
|\providecommand{\version}{draft}|\\
|\||else|\\
|\providecommand{\version}{final}|\\
|\||fi|
\end{tabular}
\end{center}
%
The definition by |\providecommand| makes sure
that previous definitions are not overwritten.
Further statements |\providecommand{\version}{...}|
can thus be added before the above code to override it.

For the main file, one might add a line
(between |\childdocmain| and the above block)
%
\begin{center}
|%\ifchilddoc\||else\providecommand{\version}{draft}\||fi|
\end{center}
%
which can be uncommented to produce a draft version.
Likewise one can add a line to the very top of a child file
(above the |\childdocof{|\textit{main}|}| directive)
%
\begin{center}
|%\providecommand{\version}{final}|
\end{center}
%
which can be uncommented to produce the final version of this child document.

%%%%%%%%%%%%%%%%%%%%%%%%%%%%%%%%%%%%%%%%%%%%%%%%%%%%%%%%%%%%%%%%%%%%%%%%%%%%%%%%
\subsection{Forwarding}
\label{sec:forward}

Different versions of the main or child documents
using compilation flags as described in \secref{sec:flags}
can be (permanently) stored in different files
for convenient compilation, viewing and distribution.
To this end, the package defines a command
to pass on compilation to a different file:

%%%%%%%%%%%%%%%%%%%%%%%%%%%%%%%%%%%%%%%%
\DescribeMacro{\childdocforward}
The command |\childdocforward| redirects processing to
another source file:
%
\begin{center}
\begin{tabular}{l}
|% \iffalse
%
% childdoc.dtx Copyright (C) 2017-2018 Niklas Beisert
%
% This work may be distributed and/or modified under the
% conditions of the LaTeX Project Public License, either version 1.3
% of this license or (at your option) any later version.
% The latest version of this license is in
%   http://www.latex-project.org/lppl.txt
% and version 1.3 or later is part of all distributions of LaTeX
% version 2005/12/01 or later.
%
% This work has the LPPL maintenance status `maintained'.
%
% The Current Maintainer of this work is Niklas Beisert.
%
% This work consists of the files childdoc.dtx and childdoc.ins
% and the derived files childdoc.def and cdocsamp.tex with
% cdocsch1.tex, cdocsch2.tex, cdocsdrf.tex, cdocsfn1.tex, cdocsfn2.tex.
%
%<package>\ifdefined\childdocmain\endinput\fi
%<package>\ProvidesFile{childdoc.def}[2018/12/30 v2.0 child document driver]
%<samplemain>\ProvidesFile{cdocsamp.tex}[2018/12/30 v2.0 sample for childdoc]
%<*driver>
%\ProvidesFile{childdoc.drv}[2018/12/30 v2.0 childdoc reference manual file]
\PassOptionsToClass{10pt,a4paper}{article}
\documentclass{ltxdoc}

\usepackage[margin=35mm]{geometry}
\usepackage{hyperref}
\usepackage{hyperxmp}
\usepackage[usenames]{color}

\hypersetup{colorlinks=true}
\hypersetup{pdfstartview=FitH}
\hypersetup{pdfpagemode=UseNone}
\hypersetup{pdfsource={}}
\hypersetup{pdflang={en-UK}}
\hypersetup{pdfcopyright={Copyright 2017-2018 Niklas Beisert.
  This work may be distributed and/or modified under the
  conditions of the LaTeX Project Public License, either version 1.3
  of this license or (at your option) any later version.}}
\hypersetup{pdflicenseurl={http://www.latex-project.org/lppl.txt}}
\hypersetup{pdfcontactaddress={ETH Zurich, ITP, HIT K,
  Wolfgang-Pauli-Strasse 27}}
\hypersetup{pdfcontactpostcode={8093}}
\hypersetup{pdfcontactcity={Zurich}}
\hypersetup{pdfcontactcountry={Switzerland}}
\hypersetup{pdfcontactemail={nbeisert@itp.phys.ethz.ch}}
\hypersetup{pdfcontacturl={http://people.phys.ethz.ch/\xmptilde nbeisert/}}

\newcommand{\secref}[1]{\hyperref[#1]{section \ref*{#1}}}

\parskip1ex
\parindent0pt
\let\olditemize\itemize
\def\itemize{\olditemize\parskip0pt}

\begin{document}

\title{The \textsf{childdoc} Package}
\hypersetup{pdftitle={The childdoc Package}}
\author{Niklas Beisert\\[2ex]
  Institut f\"ur Theoretische Physik\\
  Eidgen\"ossische Technische Hochschule Z\"urich\\
  Wolfgang-Pauli-Strasse 27, 8093 Z\"urich, Switzerland\\[1ex]
  \href{mailto:nbeisert@itp.phys.ethz.ch}
  {\texttt{nbeisert@itp.phys.ethz.ch}}}
\hypersetup{pdfauthor={Niklas Beisert}}
\hypersetup{pdfsubject={Manual for the LaTeX2e Package childdoc}}
\date{30 December 2018, \textsf{v2.0}}
\maketitle

\begin{abstract}\noindent
\textsf{childdoc} is a \LaTeXe{} package
that enables the direct compilation
of document sections included by |\include|
to individual files.
\end{abstract}

\begingroup
\parskip0ex
\tableofcontents
\endgroup

%%%%%%%%%%%%%%%%%%%%%%%%%%%%%%%%%%%%%%%%%%%%%%%%%%%%%%%%%%%%%%%%%%%%%%%%%%%%%%%%
%%%%%%%%%%%%%%%%%%%%%%%%%%%%%%%%%%%%%%%%%%%%%%%%%%%%%%%%%%%%%%%%%%%%%%%%%%%%%%%%
\section{Introduction}

\LaTeX{} provides a mechanism to structure a large document (such as a book)
into a main file and several child files (containing the chapters)
using the |\include| command.
This mechanism is beneficial for documents
which span hundreds of pages in order to
make the source file(s) more manageable.
Moreover, compilation can be restricted to
selected child files by means of the |\includeonly| command.
The latter feature can be used to reduce the compilation time while editing
(this was significantly more useful in the earlier days of \LaTeX{})
or to generate a smaller document which is easier to navigate.
Another application of |\includeonly| is to generate
documents consisting of selected parts of the complete document.

However, there are a few drawbacks of the plain |\include| mechanism:
\begin{itemize}
\item
The child files cannot be compiled on their own,
they can only be compiled via the main file.
A naive editing environment
(such as a text editor with an option
to have the current file processed by \LaTeX)
may require one to switch to the main file before compiling;
attempting to compile the child file produces errors.
\item
The main file must be modified (each time)
to adjust the |\includeonly| command
to the present needs. This easily leaves the main file in a messy state.
\item
The generated document will always carry the filename
of the main document. This is inconvenient if
several child files are to be compiled and
to be kept for distribution.
\end{itemize}

The present package provides a simple interface
to make child files individually compilable by \LaTeX{}.
Compiling a child file then has the same effect as compiling
the main file with an |\includeonly| command
to select the appropriate child.
Moreover the generated document will carry the name of the child
rather than the main file.
This resolves all three above issues.

This feature is meant to make the editing of books,
thesis documents and lecture notes somewhat more convenient.
However, the package can also be used efficiently for
composing a series of documents (such as exercise sheets)
which are typically distributed individually.
It then assists the author in generating the individual documents
(potentially in different versions)
as well as a document containing the collected series.
Another application is in developing style files
or other kinds of included material
where compilation of the style file could redirect
to a sample or test file.

%%%%%%%%%%%%%%%%%%%%%%%%%%%%%%%%%%%%%%%%%%%%%%%%%%%%%%%%%%%%%%%%%%%%%%%%%%%%%%%%
%%%%%%%%%%%%%%%%%%%%%%%%%%%%%%%%%%%%%%%%%%%%%%%%%%%%%%%%%%%%%%%%%%%%%%%%%%%%%%%%
\section{Usage}

First of all, the package \textsf{childdoc} is \emph{not} a standard
\LaTeXe{} |.sty| style file! Therefore it needs to be invoked in
a non-standard way.

%%%%%%%%%%%%%%%%%%%%%%%%%%%%%%%%%%%%%%%%%%%%%%%%%%%%%%%%%%%%%%%%%%%%%%%%%%%%%%%%
\subsection{Included Files}
\label{sec:include}

%%%%%%%%%%%%%%%%%%%%%%%%%%%%%%%%%%%%%%%%
\DescribeMacro{\childdocmain}
To use the package, add the commands
\begin{center}
\begin{tabular}{l}
|\input{childdoc.def}|\\
|\childdocmain{}|\\
\end{tabular}
\end{center}
at the very top of the main \LaTeX{} file,
in particular \emph{before} the |\documentclass| statement!
The argument of |\childdocmain| should be left empty
(but it must be present).

%%%%%%%%%%%%%%%%%%%%%%%%%%%%%%%%%%%%%%%%
\DescribeMacro{\childdocof}
Furthermore, add the commands
\begin{center}
\begin{tabular}{l}
|\input{childdoc.def}|\\
|\childdocof{|\textit{main}|}|\\
\end{tabular}
\end{center}
at the top of every child file \textit{child}
which is included by |\include{|\textit{child}|}|
from within the main file
(or at least for those files to be compiled individually).
The argument \textit{main} must be the filename of the main file.

There are a couple of
considerations in setting up the main and child documents:

%%%%%%%%%%%%%%%%%%%%%%%%%%%%%%%%%%%%%%%%
\paragraph{Restrictions.}

Please note the following restrictions:
\begin{itemize}
\item
|\childdocmain| must be called with one argument \textit{main}
to ensure compatibility with earlier version of the package.
It must either be empty (|\childdocmain{}|)
or precisely match the filename of the main file in which it is specified.
See \secref{sec:detection} for further information.
\item
The filename \textit{main} must be specified without the |.tex| extension.
\item
The filename \textit{main} is case sensitive
(even in case-insensitive file systems)
due to internal string comparison.
\item
The argument \textit{main} should be fully expanded, it cannot be a macro.
\item
Subdirectories and special characters should be avoided in filenames.
\item
The command |\childdocmain{|\textit{main}|}| must be followed by a whitespace.
It should not be followed immediately by another command
or by a comment mark `|%|'.
This is because the \TeX{} parser reads the token immediately following
the argument of |\childdocmain| and puts it
at the beginning of every child section;
however, a white\-space is ignored.
\end{itemize}

%%%%%%%%%%%%%%%%%%%%%%%%%%%%%%%%%%%%%%%%
\paragraph{Content of Main File.}

It is advisable to place all content in the child files included by |\include|.
Any output contained in the main file will appear in all child documents
unless suppressed manually;
it cannot be suppressed automatically by the |\includeonly| directive
and thus should normally be avoided.
A method to include some content in the main file
by means of conditional processing is described in \secref{sec:conditional}.

%%%%%%%%%%%%%%%%%%%%%%%%%%%%%%%%%%%%%%%%
\paragraph{Page Numbering.}

When only a part of the document is compiled,
the appropriate numbering of pages
(as well as other status parameters)
is determined from the |.aux| files.
The latter contain information from previous passes.
However this information needs to propagate through
all intermediate child documents.
Therefore the page numbering in child documents may well
be inconsistent until the complete document is compiled at least once.

A useful (if unconventional) way to always ensure a consistent
page numbering is to restart the numbering in each child document
and denote the pages by `\textit{child}|.|\textit{page}'
where \textit{child} represents the chapter/section number of the child file.
This can be achieved by the command
|\numberwithin{page}{|\textit{child}|}|
of the \textsf{amsmath} package
where \textit{child} can be |chapter| or |section|
depending on the chosen structuring.
Alternatively, one can modify the macro |\thepage| appropriately
and reset the counter |page| at the start of each child file.

%%%%%%%%%%%%%%%%%%%%%%%%%%%%%%%%%%%%%%%%%%%%%%%%%%%%%%%%%%%%%%%%%%%%%%%%%%%%%%%%
\subsection{Conditional Processing}
\label{sec:conditional}

The package provides a mechanism to compile different versions
of a document. To customise the versions further some conditional processing
can come in handy to distinguish which version is being compiled.
The package provides two macros to describe the compilation context:

%%%%%%%%%%%%%%%%%%%%%%%%%%%%%%%%%%%%%%%%
\DescribeMacro{\ifchilddoc}
The conditional |\ifchilddoc| distinguishes between the compilation of
child documents and the main document:
%
\begin{center}
|\ifchilddoc |\textit{child-code}| |[|\||else |\textit{main-code}]| \||fi|
\end{center}

%%%%%%%%%%%%%%%%%%%%%%%%%%%%%%%%%%%%%%%%
\DescribeMacro{\childdocname}
\DescribeMacro{\childdocjob}
The macro |\childdocname| contains the filename (without extension)
of the main or child file being processed.
Note that |\childdocjob| will always contain the name of the main file.

%%%%%%%%%%%%%%%%%%%%%%%%%%%%%%%%%%%%%%%%
\paragraph{Title Page.}

Conditional processing can be used to include a title or banner page
in the main document when proper precautions are taken.
Importantly, the code in the main file should ensure that the page counter
(as well as other status parameters which are stored in the |.aux| files)
takes the same value after the conditional processing.
Otherwise the page numbers may take divergent values
depending on which part is compiled.

For example, a title page could be declared by:
%
\begin{center}
\begin{tabular}{l}
|\ifchilddoc\||else|\\
|\addtocounter{page}{-1}|\\
\textit{code for title page}\\
|\newpage|\\
|\||fi|
\end{tabular}
\end{center}
%
A banner page for the child documents can be generated by:
%
\begin{center}
\begin{tabular}{l}
|\ifchilddoc|\\
|\addtocounter{page}{-1}|\\
\textit{code for banner page}\\
|\newpage|\\
|\||fi|
\end{tabular}
\end{center}
%
Here one could write a message such as:
\begin{center}
|This is the part \childdocname{} of \childdocjob{}.|
\end{center}

%%%%%%%%%%%%%%%%%%%%%%%%%%%%%%%%%%%%%%%%%%%%%%%%%%%%%%%%%%%%%%%%%%%%%%%%%%%%%%%%
\subsection{Flags}
\label{sec:flags}

The package makes it easy to generate different versions
of the main or child documents.
To this end compilation flags can be defined
and assigned different default values.
They will be particularly useful in conjunction
with the forwarding mechanism described in \secref{sec:forward}.

For example, it may be useful to have a flag |\version|
which can be set to |draft| or |final|.
The document source will contain some conditional code
depending on the value of |\version|.
Suppose further, the flag should default to |final| for the main file
and to |draft| for child files
which is a natural assignment for editing the document.
This is achieved by placing the following code
in the preamble of the main document
(below the |\childdocmain| directive):
%
\begin{center}
\begin{tabular}{l}
|\ifchilddoc|\\
|\providecommand{\version}{draft}|\\
|\||else|\\
|\providecommand{\version}{final}|\\
|\||fi|
\end{tabular}
\end{center}
%
The definition by |\providecommand| makes sure
that previous definitions are not overwritten.
Further statements |\providecommand{\version}{...}|
can thus be added before the above code to override it.

For the main file, one might add a line
(between |\childdocmain| and the above block)
%
\begin{center}
|%\ifchilddoc\||else\providecommand{\version}{draft}\||fi|
\end{center}
%
which can be uncommented to produce a draft version.
Likewise one can add a line to the very top of a child file
(above the |\childdocof{|\textit{main}|}| directive)
%
\begin{center}
|%\providecommand{\version}{final}|
\end{center}
%
which can be uncommented to produce the final version of this child document.

%%%%%%%%%%%%%%%%%%%%%%%%%%%%%%%%%%%%%%%%%%%%%%%%%%%%%%%%%%%%%%%%%%%%%%%%%%%%%%%%
\subsection{Forwarding}
\label{sec:forward}

Different versions of the main or child documents
using compilation flags as described in \secref{sec:flags}
can be (permanently) stored in different files
for convenient compilation, viewing and distribution.
To this end, the package defines a command
to pass on compilation to a different file:

%%%%%%%%%%%%%%%%%%%%%%%%%%%%%%%%%%%%%%%%
\DescribeMacro{\childdocforward}
The command |\childdocforward| redirects processing to
another source file:
%
\begin{center}
\begin{tabular}{l}
|\input{childdoc.def}|\\
|\childdocforward[|\textit{main}|]{|\textit{dest}|}|\\
\end{tabular}
\end{center}
%
The argument \textit{dest} is the destination file
(without extension).
It should be the main file or one of the child files.
Note that further \textsf{childdoc} directives
such as |\childdocof| and |\childdocforward|
in the indicated file will be processed in this form.
The optional argument \textit{main}
passes on directly to the main file \textit{main}
while pretending to compile the child \textit{dest}.
This form behaves as if \textit{dest}
issues |\childdocof{|\textit{main}|}| right away,
and no further \textsf{childdoc} directives will be processed.

%%%%%%%%%%%%%%%%%%%%%%%%%%%%%%%%%%%%%%%%
\DescribeMacro{\...prefix}
In the alternative form |\childdocforwardprefix|,
%
\begin{center}
\begin{tabular}{l}
|\input{childdoc.def}|\\
|\childdocforwardprefix[|\textit{main}|]{|\textit{prefix}|}{|\textit{dest}|}|
\end{tabular}
\end{center}
%
the destination file is determined by a pattern
depending on the current file:
To make this work, the current file must be called
`{\textit{prefix}\hspace{0.2em}\textit{suffix}}'
with \textit{prefix} matching precisely the argument.
Processing is then passed on to the file
`{\textit{dest}\hspace{0.2em}\textit{suffix}}'.
Surely, the same effect is achieved by
directly specifying the
argument `{\textit{dest}\hspace{0.2em}\textit{suffix}}'
in the first form.
However, that requires to set up a different file
for each child. With the alternative form of the command
all these files can have exactly the same content
which simplifies setting them up and maintaining them.

For example, the following file |draft.tex|
with a compilation flag |\version| as described in \secref{sec:flags}
compiles the main document as a draft:
%
\begin{center}
\begin{tabular}{l}
|\def\version{draft}|\\
|\input{childdoc.def}|\\
|\childdocforward{|\textit{main}|}|
\end{tabular}
\end{center}
%
Likewise, the following files |final|\textit{nn}|.tex|
compile the final version of the child document
|child|\textit{nn}|.tex|:
%
\begin{center}
\begin{tabular}{l}
|\def\version{final}|\\
|\input{childdoc.def}|\\
|\childdocforwardprefix{final}{child}|
\end{tabular}
\end{center}
%

Note that when several versions of a main file and/or of each child file
are to be generated, it may be convenient to set up a |Makefile| or
shell script to automatise the process.

%%%%%%%%%%%%%%%%%%%%%%%%%%%%%%%%%%%%%%%%%%%%%%%%%%%%%%%%%%%%%%%%%%%%%%%%%%%%%%%%
\subsection{Command Line Processing}
\label{sec:commandline}

The effect of redirection files can also be achieved by invoking
the \LaTeX{} compiler with a more elaborate command line.
Most conveniently this should be done as part
of a shell script or a |Makefile|.

When using \textsf{childdoc} in the main file, the following
command lines effectively perform a redirection
(note that depending on the shell being used,
backslashes may have to be doubled: `|\|' $\to$ `|\\|'):
%
\begin{center}
|... -jobname "|\textit{target}|" |\\|"|[\textit{flags}]%
|\input{childdoc.def}\childdocforward[|\textit{main}|]{|\textit{dest}|}"|
\end{center}
%
Here \textit{target} is the name of the output file,
\textit{main} is the name of the main file
and \textit{dest} is the name of the main or child file to be processed
(all filenames without extensions).
The optional argument \textit{main} can be omitted
if \textit{main} matches \textit{dest}.
Optionally, compilation \textit{flags} can be defined via |\def| commands.
This command line makes the \TeX{} engine believe
it is compiling the file \textit{target}
whose content is specified as the latter parameter.
The provided code then forwards the processing to
\textit{main} or \textit{dest} as described in \secref{sec:forward}.

%%%%%%%%%%%%%%%%%%%%%%%%%%%%%%%%%%%%%%%%%%%%%%%%%%%%%%%%%%%%%%%%%%%%%%%%%%%%%%%%
\subsection{Include by Input}
\label{sec:input}

Including child documents by |\include| has some restrictions by design.
Most notably, the content of a child document always occupies
its own set of pages; pages cannot be shared between child documents.
Usually, this behaviour makes perfect sense
because each child document contain an essential part of the document.
However, in some situations it may be desirable to compose
a document from a collection of parts
without having mandatory page breaks between then.
For this case, the package
provides a mechanism to include parts
by |\input| which can also be processed individually.
However, by construction this mechanism
requires manual handling of the content to be output.

%%%%%%%%%%%%%%%%%%%%%%%%%%%%%%%%%%%%%%%%
\DescribeMacro{\ifchilddocmanual}
The main file should be prepared as usual, see \secref{sec:include}.
However, the document body must make a distinction
between processing of an individual part and of the main document, e.g.:
%
\begin{center}
\begin{tabular}{l}
|\ifchilddocmanual|\\
|\input{\childdocname}|\\
|\||else|\\
\textit{document body with }|\input{|\textit{part}|}|\\
|\||fi|
\end{tabular}
\end{center}
%
The conditional |\ifchilddocmanual| is true whenever
a part to be included by |\input| is being compiled,
and the name of the part is stored in |\childdocname|.

%%%%%%%%%%%%%%%%%%%%%%%%%%%%%%%%%%%%%%%%
\DescribeMacro{\childdocby}
Each part to be included by |\input| should start with:
%
\begin{center}
\begin{tabular}{l}
|\input{childdoc.def}|\\
|\childdocby{|\textit{main}|}|\\
\end{tabular}
\end{center}
%
The directive |\childdocby| is similar to |\childdocof|
described in \secref{sec:include},
but the subsequent selection of content must be done manually.
To that end, both |\ifchilddoc| and |\ifchilddocmanual|
will be true upon processing of a part,
and the name of the part is stored in |\childdocname|.
Note that |\jobname| will be set to the filename of the current part
so that each part receives an individual |.aux| file
that does not interfere with the |.aux| file(s) of the main document.
This behaviour can be altered by the alternative form
|\childdocby[*]{|\textit{main}|}| (with a non-empty optional argument)
which uses the |.aux| file of the main document
by setting |\jobname| to \textit{main}.

%%%%%%%%%%%%%%%%%%%%%%%%%%%%%%%%%%%%%%%%%%%%%%%%%%%%%%%%%%%%%%%%%%%%%%%%%%%%%%%%
\subsection{Driver Development}
\label{sec:driver}

The \textsf{childdoc} mechanism can also be use for the development
of definition files such as \LaTeX{} styles or classes.
This case differs from the above setup with multiple parts
included by |\include| in that no |\includeonly| should be invoked.
This can be achieved by starting the include file
(before |\ProvidesPackage|) with:
%
\begin{center}
\begin{tabular}{l}
|\input{childdoc.def}|\\
|\childdocforward{|\textit{main}|}|\\
\end{tabular}
\end{center}
%
or alternatively with:
%
\begin{center}
\begin{tabular}{l}
|\input{childdoc.def}|\\
|\childdocby{|\textit{main}|}|\\
\end{tabular}
\end{center}
%
Both forms have slightly different effects as described above.
The main file is prepared as usual, see \secref{sec:include}.

%%%%%%%%%%%%%%%%%%%%%%%%%%%%%%%%%%%%%%%%%%%%%%%%%%%%%%%%%%%%%%%%%%%%%%%%%%%%%%%%
\subsection{Legacy Detection}
\label{sec:detection}

The directive |\childdocmain| in the main file can detect
whether the complete document or merely a child is to be compiled
even without using the directive |\childdocof|.
This method is deprecated because it is less robust
and there is no compelling reason to use it;
it is merely provided for backward compatibility
and it may be removed in future versions.

If the detection mechanism is to be used,
it is mandatory to correctly specify
the filename of the main file as the argument of |\childdocmain|:
%
\begin{center}
\begin{tabular}{l}
|\input{childdoc.def}|\\
|\childdocmain{|\textit{main}|}|\\
\end{tabular}
\end{center}
%
If |\jobname| does not match the argument \textit{main} of |\childdocmain|,
it is assumed that |\jobname| points to the child file to be compiled.
When using |\childdocmain| with the main file specified as argument,
it suffices to start a child file
with just |\input{|\textit{main}|}|
without loading of the package and using |\childdocof|.
If instead all processing is done
with the appropriate \textsf{childdoc} directives,
the argument of \textit{main} of |\childdocmain| can be empty.

An alternative version of the command line processing described
in \secref{sec:commandline} using the detection mechanism reads:
%
\begin{center}
|... -jobname "|\textit{target}|" "|[\textit{flags}]%
[|\def\jobname{|\textit{dest}|}|]|\input{|\textit{main}|}"|
\end{center}

%%%%%%%%%%%%%%%%%%%%%%%%%%%%%%%%%%%%%%%%%%%%%%%%%%%%%%%%%%%%%%%%%%%%%%%%%%%%%%%%
\subsection{Manual Code}
\label{sec:manual}

In case one cannot be certain whether the definitions file |childdoc.def|
is installed on the target \TeX{} distribution
and one prefers not to ship it,
it is conceivable to paste a few relevant commands into the sources.

To that end, drop all statements |\input{childdoc.def}|
and perform the replacements as outlined below.
Instead of |\childdocmain{|\textit{main}|}| add the following code
to the top of the main file:
%
\begin{center}
\begin{tabular}{l}
|\||ifdefined\childdocname\endinput\||fi\newif\ifchilddoc|\\
|\edef\childdocname{\scantokens\expandafter{\jobname\noexpand}}|\\
|\def\childdocmain{|\textit{main}|}\||ifx\childdocmain\childdocname\||else|\\
|\childdoctrue\includeonly{\childdocname}\let\jobname\childdocmain\||fi|\\
\end{tabular}
\end{center}
%
Instead of |\childdocof{|\textit{main}|}| just include the main file
at the top of each child file:
%
\begin{center}
|\input{|\textit{main}|}|
\end{center}
%
A simple redirection |\childdocforward{|\textit{dest}|}| is achieved by:
%
\begin{center}
|\def\jobname{|\textit{dest}|}\input{\jobname}|
\end{center}
%
The redirection with prefix
|\childdocforwardprefix[|\textit{prefix}|]{|\textit{dest}|}|
is accomplished by:
%
\begin{center}
\begin{tabular}{l}
|{\edef\jobname{\scantokens\expandafter{\jobname\noexpand}}|\\
|\def\redirectjob |\textit{prefix}|#1~~~{\gdef\jobname{|\textit{dest}|#1}}|\\
|\expandafter\redirectjob\jobname~~~}\input{\jobname}|
\end{tabular}
\end{center}

In an alternative approach,
child documents can be compiled by a specific command line
without additional code or specific definitions:
%
\begin{center}
|... -jobname "|\textit{target}|" "|[\textit{flags}]%
|\includeonly{|\textit{dest}|}\input{|\textit{main}|}"|
\end{center}
%

%%%%%%%%%%%%%%%%%%%%%%%%%%%%%%%%%%%%%%%%%%%%%%%%%%%%%%%%%%%%%%%%%%%%%%%%%%%%%%%%
%%%%%%%%%%%%%%%%%%%%%%%%%%%%%%%%%%%%%%%%%%%%%%%%%%%%%%%%%%%%%%%%%%%%%%%%%%%%%%%%
\section{Information}

%%%%%%%%%%%%%%%%%%%%%%%%%%%%%%%%%%%%%%%%%%%%%%%%%%%%%%%%%%%%%%%%%%%%%%%%%%%%%%%%
\subsection{Copyright}

Copyright \copyright{} 2017--2018 Niklas Beisert

This work may be distributed and/or modified under the
conditions of the \LaTeX{} Project Public License, either version 1.3
of this license or (at your option) any later version.
The latest version of this license is in
  \url{http://www.latex-project.org/lppl.txt}
and version 1.3 or later is part of all distributions of \LaTeX{}
version 2005/12/01 or later.

This work has the LPPL maintenance status `maintained'.

The Current Maintainer of this work is Niklas Beisert.

This work consists of the files |README.txt|, |childdoc.ins| and |childdoc.dtx|
as well as the derived files |childdoc.def|, |cdocsamp.tex|
with |cdocsch1.tex|, |cdocsch2.tex|, |cdocspt3.tex|, |cdocspt4.tex|,
|cdocsdrf.tex|, |cdocsfn1.tex|, |cdocsfn2.tex|
as well as |childdoc.pdf|.

%%%%%%%%%%%%%%%%%%%%%%%%%%%%%%%%%%%%%%%%%%%%%%%%%%%%%%%%%%%%%%%%%%%%%%%%%%%%%%%%
\subsection{Files and Installation}

The package consists of the files:
%
\begin{center}
\begin{tabular}{ll}
    |README.txt|   & readme file \\
    |childdoc.ins| & installation file \\
    |childdoc.dtx| & source file \\
    |childdoc.def| & definition file \\
    |cdocsamp.tex| & sample main file \\
    |cdocsch1.tex| & sample include file \\
    |cdocsch2.tex| & sample include file \\
    |cdocspt3.tex| & sample part file \\
    |cdocspt4.tex| & sample part file \\
    |cdocsdrf.tex| & sample redirection file \\
    |cdocsfn1.tex| & sample redirection file \\
    |cdocsfn2.tex| & sample redirection file \\
    |childdoc.pdf| & manual
\end{tabular}
\end{center}
%
The distribution consists of the files
|README.txt|, |childdoc.ins| and |childdoc.dtx|.
%
\begin{itemize}
\item
Run (pdf)\LaTeX{} on |childdoc.dtx|
to compile the manual |childdoc.pdf| (this file).
\item
Run \LaTeX{} on |childdoc.ins| to create the definitions file |childdoc.def|
and the sample |cdocsamp.tex| with include files
|cdocsch1.tex|, |cdocsch2.tex|, |cdocspt3.tex|, |cdocspt4.tex|,
|cdocsdrf.tex|, |cdocsfn1.tex|, |cdocsfn2.tex|.
Then copy the file |childdoc.def| to an appropriate directory of your \LaTeX{}
distribution, e.g.\ \textit{texmf-root}|/tex/latex/childdoc|.
\end{itemize}

%%%%%%%%%%%%%%%%%%%%%%%%%%%%%%%%%%%%%%%%%%%%%%%%%%%%%%%%%%%%%%%%%%%%%%%%%%%%%%%%
\subsection{Related CTAN Packages}

There are several other packages which offer a similar functionality:
%
\begin{itemize}
\item
The packages
\href{http://ctan.org/pkg/docmute}{\textsf{docmute}},
\href{http://ctan.org/pkg/includex}{\textsf{includex}} and
\href{http://ctan.org/pkg/standalone}{\textsf{standalone}}
provide commands to include only the document body of
a child file thus allowing both files to be compiled individually.
\item
The packages \href{http://ctan.org/pkg/subdocs}{\textsf{subdocs}}
and \href{http://ctan.org/pkg/subfiles}{\textsf{subfiles}}
provide structures in which the main and child documents can be
encapsulated and allowing them to be compiled individually.
The inclusion mechanism is different from the conventional |\include|.
\item
The package \href{http://ctan.org/pkg/combine}{\textsf{combine}}
is an elaborate solution to combine several documents into one.
\end{itemize}
%
See also the CTAN topic \href{http://ctan.org/topic/subdocs}{\textsf{subdocs}}
for further related packages.
The present package differs from the above solutions in that
a document structure constructed with the conventional |\include| mechanism
just needs two extra commands at the top of every file
such that all constituent files can be compiled individually.

%%%%%%%%%%%%%%%%%%%%%%%%%%%%%%%%%%%%%%%%%%%%%%%%%%%%%%%%%%%%%%%%%%%%%%%%%%%%%%%%
%\subsection{Feature Suggestions}
%
%The following is a list of features which may be useful for future
%versions of this package:
%%
%\begin{itemize}
%\item
%\ldots
%\end{itemize}

%%%%%%%%%%%%%%%%%%%%%%%%%%%%%%%%%%%%%%%%%%%%%%%%%%%%%%%%%%%%%%%%%%%%%%%%%%%%%%%%
\subsection{Revision History}

%%%%%%%%%%%%%%%%%%%%%%%%%%%%%%%%%%%%%%%%
\paragraph{v2.0:} 2018/12/30

\begin{itemize}
\item
immediate forward processing
\item
added |\childdocby| mechanism
\item
manual restructured
\end{itemize}

%%%%%%%%%%%%%%%%%%%%%%%%%%%%%%%%%%%%%%%%
\paragraph{v1.6:} 2018/01/17

\begin{itemize}
\item
application for development of include files
\item
corrections to manual
\end{itemize}

%%%%%%%%%%%%%%%%%%%%%%%%%%%%%%%%%%%%%%%%
\paragraph{v1.5:} 2017/05/21

\begin{itemize}
\item
more complete structuring introduced
\item
|\childdocof| introduced
\item
|\childdoc| renamed to |\childdocmain|
\item
|\childredirect| renamed to |\childdocforward| and |\childdocforwardprefix|
and functionality expanded
\end{itemize}

%%%%%%%%%%%%%%%%%%%%%%%%%%%%%%%%%%%%%%%%
\paragraph{v1.0:} 2017/04/27

\begin{itemize}
\item
manual and install package
\item
first version published on CTAN
\end{itemize}

%%%%%%%%%%%%%%%%%%%%%%%%%%%%%%%%%%%%%%%%
\paragraph{v0.6:} 2017/04/26

\begin{itemize}
\item
redirection mechanism added
\end{itemize}

%%%%%%%%%%%%%%%%%%%%%%%%%%%%%%%%%%%%%%%%
\paragraph{v0.5:} 2017/04/26

\begin{itemize}
\item
functionality in definition file
\end{itemize}


%%%%%%%%%%%%%%%%%%%%%%%%%%%%%%%%%%%%%%%%%%%%%%%%%%%%%%%%%%%%%%%%%%%%%%%%%%%%%%%%
%%%%%%%%%%%%%%%%%%%%%%%%%%%%%%%%%%%%%%%%%%%%%%%%%%%%%%%%%%%%%%%%%%%%%%%%%%%%%%%%
%%%%%%%%%%%%%%%%%%%%%%%%%%%%%%%%%%%%%%%%%%%%%%%%%%%%%%%%%%%%%%%%%%%%%%%%%%%%%%%%
\appendix

\settowidth\MacroIndent{\rmfamily\scriptsize 000\ }

 \DocInput{childdoc.dtx}

\end{document}
%</driver>
% \fi
%
% %%%%%%%%%%%%%%%%%%%%%%%%%%%%%%%%%%%%%%%%%%%%%%%%%%%%%%%%%%%%%%%%%%%%%%%%%%%%%%
% %%%%%%%%%%%%%%%%%%%%%%%%%%%%%%%%%%%%%%%%%%%%%%%%%%%%%%%%%%%%%%%%%%%%%%%%%%%%%%
% \section{Sample}
%\iffalse
%<*samplemain>
%\fi
%
% The following presents a sample document
% with two chapters, two parts, a title page,
% a compile flag as well as three forwarding files to set the flag.
% It consists of eight |.tex| files:
% \begin{center}
% \begin{tabular}{ll}
% |cdocsamp.tex|&main file\\
% |cdocsch1.tex|&include file for chapter 1\\
% |cdocsch2.tex|&include file for chapter 2\\
% |cdocspt3.tex|&include file for part 3\\
% |cdocspt4.tex|&include file for part 4\\
% |cdocsdrf.tex|&forwarding file for main file in draft mode\\
% |cdocsfi1.tex|&forwarding file for final version of chapter 1\\
% |cdocsfi2.tex|&forwarding file for final version of chapter 2\\
% \end{tabular}
% \end{center}
% Each of the eight files can be compiled directly by the \LaTeX{} compiler.
%
% %%%%%%%%%%%%%%%%%%%%%%%%%%%%%%%%%%%%%%
% \paragraph{Main File.}
%
% The main file is called |cdocsamp.tex|.
%
% Load the \textsf{childdoc} definitions and
% declare the filename for the main document:
%    \begin{macrocode}
\input{childdoc.def}
\childdocmain{}
%    \end{macrocode}

% Optional override for |\version| flag:
%    \begin{macrocode}
%%\ifchilddoc\else\providecommand{\version}{draft}\fi
%    \end{macrocode}

% Define the default values for the |\version| flag
% (|final| for the main file and |draft| for childs):
%    \begin{macrocode}
\ifchilddoc
\providecommand{\version}{draft}
\else
\providecommand{\version}{final}
\fi
%    \end{macrocode}

% Load the standard document class:
%    \begin{macrocode}
\documentclass[12pt]{article}
%    \end{macrocode}

% Start the document body:
%    \begin{macrocode}
\begin{document}
%    \end{macrocode}

% Declare a title page.
% Print title, part of document being processed and version flag:
%    \begin{macrocode}
\addtocounter{page}{-1}
\begin{center}
{\LARGE\bfseries{}childdoc example\par}
\vspace{1cm}
\ifchilddoc
\ifchilddocmanual part\else chapter\fi:
`\childdocname' of `\childdocjob'\par
\else
main document: `\childdocjob'\par
\fi
version: \version\par
\end{center}
\newpage
%    \end{macrocode}

% Manually include selected file,
% otherwise process as usual:
%    \begin{macrocode}
\ifchilddocmanual
\section*{part `\childdocname'}
\input{\childdocname}
\else
%    \end{macrocode}

% Include the two chapters:
%    \begin{macrocode}
\include{cdocsch1}
\include{cdocsch2}
%    \end{macrocode}

% Include the two parts unless only chapters should be displayed:
%    \begin{macrocode}
\ifchilddoc\else
\section{part three}
\input{cdocspt3}
\section{part four}
\input{cdocspt4}
\fi
%    \end{macrocode}

% Process as usual until here:
%    \begin{macrocode}
\fi
%    \end{macrocode}

% End of document body:
%    \begin{macrocode}
\end{document}
%    \end{macrocode}
%\iffalse
%</samplemain>
%\fi
%
% %%%%%%%%%%%%%%%%%%%%%%%%%%%%%%%%%%%%%%
% \paragraph{Chapter Include Files.}
%
% The include files are called |cdocsch1.tex| and |cdocsch2.tex|.
%
%\iffalse
%<*samplechap1|samplechap2>
%\fi

% Optional override for |\version| flag:
%    \begin{macrocode}
%%\providecommand{\version}{final}
%    \end{macrocode}

% Include the main document:
%    \begin{macrocode}
\input{childdoc.def}
\childdocof{cdocsamp}
%    \end{macrocode}

%\iffalse
%</samplechap1|samplechap2>
%\fi
%
%\iffalse
%<*samplechap1>
%\fi
% Some text for chapter 1:
%    \begin{macrocode}
\section{one}
some text in chapter one
%    \end{macrocode}

%\iffalse
%</samplechap1>
%\fi
% Some text for chapter 2:
%\iffalse
%<*samplechap2>
%\fi
%    \begin{macrocode}
\section{two}
more text in chapter two
%    \end{macrocode}

%\iffalse
%</samplechap2>
%\fi
%
% %%%%%%%%%%%%%%%%%%%%%%%%%%%%%%%%%%%%%%
% \paragraph{Part Include Files.}
%
% The include files are called |cdocspt3.tex| and |cdocspt4.tex|.
%
%\iffalse
%<*samplepart3|samplepart4>
%\fi

% Optional override for |\version| flag:
%    \begin{macrocode}
%%\providecommand{\version}{final}
%    \end{macrocode}

% Include the main document:
%    \begin{macrocode}
\input{childdoc.def}
\childdocby{cdocsamp}
%    \end{macrocode}

%\iffalse
%</samplepart3|samplepart4>
%\fi
%
%\iffalse
%<*samplepart3>
%\fi
% Some text for part 3:
%    \begin{macrocode}
some text in part three
%    \end{macrocode}

%\iffalse
%</samplepart3>
%\fi
% Some text for part 4:
%\iffalse
%<*samplepart4>
%\fi
%    \begin{macrocode}
more text in part four
%    \end{macrocode}

%\iffalse
%</samplepart4>
%\fi
%
% %%%%%%%%%%%%%%%%%%%%%%%%%%%%%%%%%%%%%%
% \paragraph{Forwarding for a Complete Draft.}
%
% The following forwarding file |cdocsdrf.tex|
% compiles the main document in draft mode:
%\iffalse
%<*sampledraft>
%\fi
%    \begin{macrocode}
\def\version{draft}
\input{childdoc.def}
\childdocforward{cdocsamp}
%    \end{macrocode}

%\iffalse
%</sampledraft>
%\fi
%
% %%%%%%%%%%%%%%%%%%%%%%%%%%%%%%%%%%%%%%
% \paragraph{Forwarding for Final Version of the Chapters.}
%
% The following forwarding files |cdocsfn1.tex| and |cdocsfn2.tex|
% (with identical content)
% compile the final versions of the child documents
% |cdocsch1.tex| and |cdocsch2.tex|, respectively:
%\iffalse
%<*samplefinal>
%\fi
%    \begin{macrocode}
\def\version{final}
\input{childdoc.def}
\childdocforwardprefix[cdocsamp]{cdocsfn}{cdocsch}
%    \end{macrocode}

%\iffalse
%</samplefinal>
%\fi
%
% %%%%%%%%%%%%%%%%%%%%%%%%%%%%%%%%%%%%%%
% \paragraph{Command Line Processing.}
%
% The following three command lines generate the output files
% |cdocscld|, |cdocscl1| and |cdocscl2|
% which should be identical to
% |cdocsdrf|, |cdocsch1| and |cdocsfn2|, respectively:
% \begin{center}
% \begin{tabular}{l}
% |latex -jobname cdocscld \|\\
% |  "\def\version{draft}\input{childdoc.def}\childdocforward{cdocsamp}"|\\
% |latex -jobname cdocscl1 \|\\
% |  "\input{childdoc.def}\childdocforward[cdocsamp]{cdocsch1}"|\\
% |latex -jobname cdocscl2 \|\\
% |  "\def\version{final}\input{childdoc.def}\childdocforward{cdocsch2}"|
% \end{tabular}
% \end{center}
% Note that the trailing backslash on each first line
% merely continues the input to the second line
% (for convenient cut ant paste).
% Furthermore, the command |latex| can be replaced by any
% of its alternative versions such as |pdflatex|.
%
% %%%%%%%%%%%%%%%%%%%%%%%%%%%%%%%%%%%%%%%%%%%%%%%%%%%%%%%%%%%%%%%%%%%%%%%%%%%%%%
% %%%%%%%%%%%%%%%%%%%%%%%%%%%%%%%%%%%%%%%%%%%%%%%%%%%%%%%%%%%%%%%%%%%%%%%%%%%%%%
% \section{Implementation}
%\iffalse
%<*package>
%\fi
%
% This section describes the definitions file |childdoc.def|.

% The definitions cannot be loaded using |\usepackage| or |\RequirePackage|
% which has a mechanism to prevent loading a style file more than once.
% When loading the definitions by means of |\input|
% multiple instances have to be prevented manually:
%\iffalse
%This code needs to be before the `\ProvidesFile' directive
%which is defined at the beginning of this file.
%Therefore it is also placed there and commented out here.
%</package>
%<*discard>
%\fi
%    \begin{macrocode}
\ifdefined\childdocmain\endinput\fi
%    \end{macrocode}
%\iffalse
%</discard>
%<*package>
%\fi
%
% \macro{\ifchilddoc}
% \macro{\ifchilddocmanual}
% The conditional |\ifchilddoc| tells whether a
% child (true) or main (false) document is being compiled.
% The conditional |\ifchilddocmanual| tells whether
% the |\includeonly| mechanism is used (false) or
% the selection of child files must be performed manually (true).
% The definitions initialise to false:
%    \begin{macrocode}
\newif\ifchilddoc
\newif\ifchilddocmanual
%    \end{macrocode}

% \macro{\childdocname}
% \macro{\childdocjob}
% The macro |\childdocname| stores the name of the main document
% to be compiled. The macro |\childdocjob| stores the name of
% the document on which the \LaTeX{} compiler was originally invoked.
% The content of |\jobname| cannot be compared
% to filenames specified in the source due to different catcodes.
% The following code rescans |\jobname|, stores the result
% in |\childdocname| and saves a copy in |\childdocjob|:
%    \begin{macrocode}
\edef\childdocname{\scantokens\expandafter{\jobname\noexpand}}
\let\childdocjob\childdocname
%    \end{macrocode}

% \macro{\childdocdisable}
% The macro |\childdocdisable| prevents the main file
% from being processed more than once.
% At this stage, the main document command |\childdocmain|
% is assumed to be called once again where it should do nothing.
% Any subsequent call to it should prevent
% a secondary processing of the main document
% It overwrites the forwarding commands
% |\childdocof| and |\childdocforward|
% with empty macros to prevent further inclusions of the main document:
%    \begin{macrocode}
\newcommand{\childdocdisable}
{
  \renewcommand{\childdocmain}[1]{\renewcommand{\childdocmain}[1]{\endinput}}
  \renewcommand{\childdocof}[1]{}
  \renewcommand{\childdocby}[2][]{}
  \renewcommand{\childdocforward}[2][]{}
  \renewcommand{\childdocdisable}{}
}
%    \end{macrocode}

% \macro{\childdocmain}
% The macro |\childdocmain| is to be called at the top of the main file
% with nothing or the main filename (without extension) as argument.
% First, it breaks loops.
% If the argument is not empty and does not match |\childdocname|
% (which is set by the first inclusion of |childdoc.def|),
% |\ifchilddoc| is set to true, |\includeonly| is applied to the child file
% and |\jobname| is set to the main file
% (for proper handling of |.aux| files):
%    \begin{macrocode}
\newcommand{\childdocmain}[1]
{
  \childdocdisable\childdocmain{}
  \if?#1?\else
    \begingroup
      \def\childdoctmp{#1}
      \ifx\childdoctmp\childdocname
        \def\childdoctmp{}
      \else
        \def\childdoctmp
        {
          \childdoctrue
          \includeonly{\childdocname}
          \def\childdocjob{#1}
          \def\jobname{#1}
        }
      \fi
      \expandafter
    \endgroup
    \childdoctmp
  \fi
}
%    \end{macrocode}

% \macro{\childdocof}
% The command |\childdocof| redirects
% compilation to the main file |#1|.
%    \begin{macrocode}
\newcommand{\childdocof}[1]
{
  \childdocdisable
  \childdoctrue
  \includeonly{\childdocname}
  \def\jobname{#1}
  \def\childdocjob{#1}
  \input{#1}
}
%    \end{macrocode}

% \macro{\childdocby}
% The command |\childdocby| ....
%    \begin{macrocode}
\newcommand{\childdocby}[2][]
{
  \childdocdisable
  \childdoctrue
  \childdocmanualtrue
  \if?#1?\else
    \def\jobname{#2}
  \fi
  \def\childdocjob{#2}
  \input{#2}
  \endinput
}
%    \end{macrocode}

% \macro{\childdocforward}
% The command |\childdocforward| redirects
% compilation to the main file or
% (if the optional argument is given) a child file.
% Parameters are set as if the main file
% or a child file starting with |\childdocof| was compiled.
% Then compilation is handed over to the main file:
%    \begin{macrocode}
\newcommand{\childdocforward}[2][]
{
  \begingroup
    \if?#1?
      \def\childdoctmp
      {
        \def\childdocname{#2}
        \def\childdocjob{#2}
        \def\jobname{#2}
        \input{#2}
        \endinput
      }
    \else
      \def\childdoctmp
      {
        \childdocdisable
        \def\childdocname{#2}
        \childdoctrue
        \includeonly{#2}
        \def\childdocjob{#1}
        \def\jobname{#1}
        \input{#1}
        \endinput
      }
    \fi
    \expandafter
  \endgroup
  \childdoctmp
}
%    \end{macrocode}

% \macro{\childdocforwardprefix}
% The command |\childdocforwardprefix| redirects
% compilation to the main or a child file by means of a pattern.
% The prefix |#1| in the current filename is replaced by |#2|
% and the suffix of the current filename is kept
% (it is assumed that the filename does not contain the substring `|~~~|'
% which is used as a delimiter).
% Compilation is handed over to the new file by |\childdocforward|:
%    \begin{macrocode}
\newcommand{\childdocforwardprefix}[3][]
{
  \begingroup
    \def\childdocextract #2##1~~~{\def\childdoctmp{\childdocforward[#1]{#3##1}}}
    \expandafter\childdocextract\childdocname~~~
    \expandafter
  \endgroup
  \childdoctmp
}
%    \end{macrocode}

% \macro{\childdoc}
% The deprecated macro |\childdoc| is a legacy version of |\childdocmain|:
%    \begin{macrocode}
\newcommand{\childdoc}{\childdocmain}
%    \end{macrocode}

% \macro{\childdocredirect}
% The deprecated macro |\childdocredirect| is a legacy version
% of |\childdocforward| and |\childdocforwardprefix|:
%    \begin{macrocode}
\newcommand{\childdocredirect}[2][]
{
  \begingroup
    \if?#1?
      \def\childdoctmp{\childdocforward{#2}}
    \else
      \def\childdoctmp{\childdocforwardprefix{#1}{#2}}
    \fi
    \expandafter
  \endgroup
  \childdoctmp
}
%    \end{macrocode}

%\iffalse
%</package>
%\fi
%
\endinput
|\\
|\childdocforward[|\textit{main}|]{|\textit{dest}|}|\\
\end{tabular}
\end{center}
%
The argument \textit{dest} is the destination file
(without extension).
It should be the main file or one of the child files.
Note that further \textsf{childdoc} directives
such as |\childdocof| and |\childdocforward|
in the indicated file will be processed in this form.
The optional argument \textit{main}
passes on directly to the main file \textit{main}
while pretending to compile the child \textit{dest}.
This form behaves as if \textit{dest}
issues |\childdocof{|\textit{main}|}| right away,
and no further \textsf{childdoc} directives will be processed.

%%%%%%%%%%%%%%%%%%%%%%%%%%%%%%%%%%%%%%%%
\DescribeMacro{\...prefix}
In the alternative form |\childdocforwardprefix|,
%
\begin{center}
\begin{tabular}{l}
|% \iffalse
%
% childdoc.dtx Copyright (C) 2017-2018 Niklas Beisert
%
% This work may be distributed and/or modified under the
% conditions of the LaTeX Project Public License, either version 1.3
% of this license or (at your option) any later version.
% The latest version of this license is in
%   http://www.latex-project.org/lppl.txt
% and version 1.3 or later is part of all distributions of LaTeX
% version 2005/12/01 or later.
%
% This work has the LPPL maintenance status `maintained'.
%
% The Current Maintainer of this work is Niklas Beisert.
%
% This work consists of the files childdoc.dtx and childdoc.ins
% and the derived files childdoc.def and cdocsamp.tex with
% cdocsch1.tex, cdocsch2.tex, cdocsdrf.tex, cdocsfn1.tex, cdocsfn2.tex.
%
%<package>\ifdefined\childdocmain\endinput\fi
%<package>\ProvidesFile{childdoc.def}[2018/12/30 v2.0 child document driver]
%<samplemain>\ProvidesFile{cdocsamp.tex}[2018/12/30 v2.0 sample for childdoc]
%<*driver>
%\ProvidesFile{childdoc.drv}[2018/12/30 v2.0 childdoc reference manual file]
\PassOptionsToClass{10pt,a4paper}{article}
\documentclass{ltxdoc}

\usepackage[margin=35mm]{geometry}
\usepackage{hyperref}
\usepackage{hyperxmp}
\usepackage[usenames]{color}

\hypersetup{colorlinks=true}
\hypersetup{pdfstartview=FitH}
\hypersetup{pdfpagemode=UseNone}
\hypersetup{pdfsource={}}
\hypersetup{pdflang={en-UK}}
\hypersetup{pdfcopyright={Copyright 2017-2018 Niklas Beisert.
  This work may be distributed and/or modified under the
  conditions of the LaTeX Project Public License, either version 1.3
  of this license or (at your option) any later version.}}
\hypersetup{pdflicenseurl={http://www.latex-project.org/lppl.txt}}
\hypersetup{pdfcontactaddress={ETH Zurich, ITP, HIT K,
  Wolfgang-Pauli-Strasse 27}}
\hypersetup{pdfcontactpostcode={8093}}
\hypersetup{pdfcontactcity={Zurich}}
\hypersetup{pdfcontactcountry={Switzerland}}
\hypersetup{pdfcontactemail={nbeisert@itp.phys.ethz.ch}}
\hypersetup{pdfcontacturl={http://people.phys.ethz.ch/\xmptilde nbeisert/}}

\newcommand{\secref}[1]{\hyperref[#1]{section \ref*{#1}}}

\parskip1ex
\parindent0pt
\let\olditemize\itemize
\def\itemize{\olditemize\parskip0pt}

\begin{document}

\title{The \textsf{childdoc} Package}
\hypersetup{pdftitle={The childdoc Package}}
\author{Niklas Beisert\\[2ex]
  Institut f\"ur Theoretische Physik\\
  Eidgen\"ossische Technische Hochschule Z\"urich\\
  Wolfgang-Pauli-Strasse 27, 8093 Z\"urich, Switzerland\\[1ex]
  \href{mailto:nbeisert@itp.phys.ethz.ch}
  {\texttt{nbeisert@itp.phys.ethz.ch}}}
\hypersetup{pdfauthor={Niklas Beisert}}
\hypersetup{pdfsubject={Manual for the LaTeX2e Package childdoc}}
\date{30 December 2018, \textsf{v2.0}}
\maketitle

\begin{abstract}\noindent
\textsf{childdoc} is a \LaTeXe{} package
that enables the direct compilation
of document sections included by |\include|
to individual files.
\end{abstract}

\begingroup
\parskip0ex
\tableofcontents
\endgroup

%%%%%%%%%%%%%%%%%%%%%%%%%%%%%%%%%%%%%%%%%%%%%%%%%%%%%%%%%%%%%%%%%%%%%%%%%%%%%%%%
%%%%%%%%%%%%%%%%%%%%%%%%%%%%%%%%%%%%%%%%%%%%%%%%%%%%%%%%%%%%%%%%%%%%%%%%%%%%%%%%
\section{Introduction}

\LaTeX{} provides a mechanism to structure a large document (such as a book)
into a main file and several child files (containing the chapters)
using the |\include| command.
This mechanism is beneficial for documents
which span hundreds of pages in order to
make the source file(s) more manageable.
Moreover, compilation can be restricted to
selected child files by means of the |\includeonly| command.
The latter feature can be used to reduce the compilation time while editing
(this was significantly more useful in the earlier days of \LaTeX{})
or to generate a smaller document which is easier to navigate.
Another application of |\includeonly| is to generate
documents consisting of selected parts of the complete document.

However, there are a few drawbacks of the plain |\include| mechanism:
\begin{itemize}
\item
The child files cannot be compiled on their own,
they can only be compiled via the main file.
A naive editing environment
(such as a text editor with an option
to have the current file processed by \LaTeX)
may require one to switch to the main file before compiling;
attempting to compile the child file produces errors.
\item
The main file must be modified (each time)
to adjust the |\includeonly| command
to the present needs. This easily leaves the main file in a messy state.
\item
The generated document will always carry the filename
of the main document. This is inconvenient if
several child files are to be compiled and
to be kept for distribution.
\end{itemize}

The present package provides a simple interface
to make child files individually compilable by \LaTeX{}.
Compiling a child file then has the same effect as compiling
the main file with an |\includeonly| command
to select the appropriate child.
Moreover the generated document will carry the name of the child
rather than the main file.
This resolves all three above issues.

This feature is meant to make the editing of books,
thesis documents and lecture notes somewhat more convenient.
However, the package can also be used efficiently for
composing a series of documents (such as exercise sheets)
which are typically distributed individually.
It then assists the author in generating the individual documents
(potentially in different versions)
as well as a document containing the collected series.
Another application is in developing style files
or other kinds of included material
where compilation of the style file could redirect
to a sample or test file.

%%%%%%%%%%%%%%%%%%%%%%%%%%%%%%%%%%%%%%%%%%%%%%%%%%%%%%%%%%%%%%%%%%%%%%%%%%%%%%%%
%%%%%%%%%%%%%%%%%%%%%%%%%%%%%%%%%%%%%%%%%%%%%%%%%%%%%%%%%%%%%%%%%%%%%%%%%%%%%%%%
\section{Usage}

First of all, the package \textsf{childdoc} is \emph{not} a standard
\LaTeXe{} |.sty| style file! Therefore it needs to be invoked in
a non-standard way.

%%%%%%%%%%%%%%%%%%%%%%%%%%%%%%%%%%%%%%%%%%%%%%%%%%%%%%%%%%%%%%%%%%%%%%%%%%%%%%%%
\subsection{Included Files}
\label{sec:include}

%%%%%%%%%%%%%%%%%%%%%%%%%%%%%%%%%%%%%%%%
\DescribeMacro{\childdocmain}
To use the package, add the commands
\begin{center}
\begin{tabular}{l}
|\input{childdoc.def}|\\
|\childdocmain{}|\\
\end{tabular}
\end{center}
at the very top of the main \LaTeX{} file,
in particular \emph{before} the |\documentclass| statement!
The argument of |\childdocmain| should be left empty
(but it must be present).

%%%%%%%%%%%%%%%%%%%%%%%%%%%%%%%%%%%%%%%%
\DescribeMacro{\childdocof}
Furthermore, add the commands
\begin{center}
\begin{tabular}{l}
|\input{childdoc.def}|\\
|\childdocof{|\textit{main}|}|\\
\end{tabular}
\end{center}
at the top of every child file \textit{child}
which is included by |\include{|\textit{child}|}|
from within the main file
(or at least for those files to be compiled individually).
The argument \textit{main} must be the filename of the main file.

There are a couple of
considerations in setting up the main and child documents:

%%%%%%%%%%%%%%%%%%%%%%%%%%%%%%%%%%%%%%%%
\paragraph{Restrictions.}

Please note the following restrictions:
\begin{itemize}
\item
|\childdocmain| must be called with one argument \textit{main}
to ensure compatibility with earlier version of the package.
It must either be empty (|\childdocmain{}|)
or precisely match the filename of the main file in which it is specified.
See \secref{sec:detection} for further information.
\item
The filename \textit{main} must be specified without the |.tex| extension.
\item
The filename \textit{main} is case sensitive
(even in case-insensitive file systems)
due to internal string comparison.
\item
The argument \textit{main} should be fully expanded, it cannot be a macro.
\item
Subdirectories and special characters should be avoided in filenames.
\item
The command |\childdocmain{|\textit{main}|}| must be followed by a whitespace.
It should not be followed immediately by another command
or by a comment mark `|%|'.
This is because the \TeX{} parser reads the token immediately following
the argument of |\childdocmain| and puts it
at the beginning of every child section;
however, a white\-space is ignored.
\end{itemize}

%%%%%%%%%%%%%%%%%%%%%%%%%%%%%%%%%%%%%%%%
\paragraph{Content of Main File.}

It is advisable to place all content in the child files included by |\include|.
Any output contained in the main file will appear in all child documents
unless suppressed manually;
it cannot be suppressed automatically by the |\includeonly| directive
and thus should normally be avoided.
A method to include some content in the main file
by means of conditional processing is described in \secref{sec:conditional}.

%%%%%%%%%%%%%%%%%%%%%%%%%%%%%%%%%%%%%%%%
\paragraph{Page Numbering.}

When only a part of the document is compiled,
the appropriate numbering of pages
(as well as other status parameters)
is determined from the |.aux| files.
The latter contain information from previous passes.
However this information needs to propagate through
all intermediate child documents.
Therefore the page numbering in child documents may well
be inconsistent until the complete document is compiled at least once.

A useful (if unconventional) way to always ensure a consistent
page numbering is to restart the numbering in each child document
and denote the pages by `\textit{child}|.|\textit{page}'
where \textit{child} represents the chapter/section number of the child file.
This can be achieved by the command
|\numberwithin{page}{|\textit{child}|}|
of the \textsf{amsmath} package
where \textit{child} can be |chapter| or |section|
depending on the chosen structuring.
Alternatively, one can modify the macro |\thepage| appropriately
and reset the counter |page| at the start of each child file.

%%%%%%%%%%%%%%%%%%%%%%%%%%%%%%%%%%%%%%%%%%%%%%%%%%%%%%%%%%%%%%%%%%%%%%%%%%%%%%%%
\subsection{Conditional Processing}
\label{sec:conditional}

The package provides a mechanism to compile different versions
of a document. To customise the versions further some conditional processing
can come in handy to distinguish which version is being compiled.
The package provides two macros to describe the compilation context:

%%%%%%%%%%%%%%%%%%%%%%%%%%%%%%%%%%%%%%%%
\DescribeMacro{\ifchilddoc}
The conditional |\ifchilddoc| distinguishes between the compilation of
child documents and the main document:
%
\begin{center}
|\ifchilddoc |\textit{child-code}| |[|\||else |\textit{main-code}]| \||fi|
\end{center}

%%%%%%%%%%%%%%%%%%%%%%%%%%%%%%%%%%%%%%%%
\DescribeMacro{\childdocname}
\DescribeMacro{\childdocjob}
The macro |\childdocname| contains the filename (without extension)
of the main or child file being processed.
Note that |\childdocjob| will always contain the name of the main file.

%%%%%%%%%%%%%%%%%%%%%%%%%%%%%%%%%%%%%%%%
\paragraph{Title Page.}

Conditional processing can be used to include a title or banner page
in the main document when proper precautions are taken.
Importantly, the code in the main file should ensure that the page counter
(as well as other status parameters which are stored in the |.aux| files)
takes the same value after the conditional processing.
Otherwise the page numbers may take divergent values
depending on which part is compiled.

For example, a title page could be declared by:
%
\begin{center}
\begin{tabular}{l}
|\ifchilddoc\||else|\\
|\addtocounter{page}{-1}|\\
\textit{code for title page}\\
|\newpage|\\
|\||fi|
\end{tabular}
\end{center}
%
A banner page for the child documents can be generated by:
%
\begin{center}
\begin{tabular}{l}
|\ifchilddoc|\\
|\addtocounter{page}{-1}|\\
\textit{code for banner page}\\
|\newpage|\\
|\||fi|
\end{tabular}
\end{center}
%
Here one could write a message such as:
\begin{center}
|This is the part \childdocname{} of \childdocjob{}.|
\end{center}

%%%%%%%%%%%%%%%%%%%%%%%%%%%%%%%%%%%%%%%%%%%%%%%%%%%%%%%%%%%%%%%%%%%%%%%%%%%%%%%%
\subsection{Flags}
\label{sec:flags}

The package makes it easy to generate different versions
of the main or child documents.
To this end compilation flags can be defined
and assigned different default values.
They will be particularly useful in conjunction
with the forwarding mechanism described in \secref{sec:forward}.

For example, it may be useful to have a flag |\version|
which can be set to |draft| or |final|.
The document source will contain some conditional code
depending on the value of |\version|.
Suppose further, the flag should default to |final| for the main file
and to |draft| for child files
which is a natural assignment for editing the document.
This is achieved by placing the following code
in the preamble of the main document
(below the |\childdocmain| directive):
%
\begin{center}
\begin{tabular}{l}
|\ifchilddoc|\\
|\providecommand{\version}{draft}|\\
|\||else|\\
|\providecommand{\version}{final}|\\
|\||fi|
\end{tabular}
\end{center}
%
The definition by |\providecommand| makes sure
that previous definitions are not overwritten.
Further statements |\providecommand{\version}{...}|
can thus be added before the above code to override it.

For the main file, one might add a line
(between |\childdocmain| and the above block)
%
\begin{center}
|%\ifchilddoc\||else\providecommand{\version}{draft}\||fi|
\end{center}
%
which can be uncommented to produce a draft version.
Likewise one can add a line to the very top of a child file
(above the |\childdocof{|\textit{main}|}| directive)
%
\begin{center}
|%\providecommand{\version}{final}|
\end{center}
%
which can be uncommented to produce the final version of this child document.

%%%%%%%%%%%%%%%%%%%%%%%%%%%%%%%%%%%%%%%%%%%%%%%%%%%%%%%%%%%%%%%%%%%%%%%%%%%%%%%%
\subsection{Forwarding}
\label{sec:forward}

Different versions of the main or child documents
using compilation flags as described in \secref{sec:flags}
can be (permanently) stored in different files
for convenient compilation, viewing and distribution.
To this end, the package defines a command
to pass on compilation to a different file:

%%%%%%%%%%%%%%%%%%%%%%%%%%%%%%%%%%%%%%%%
\DescribeMacro{\childdocforward}
The command |\childdocforward| redirects processing to
another source file:
%
\begin{center}
\begin{tabular}{l}
|\input{childdoc.def}|\\
|\childdocforward[|\textit{main}|]{|\textit{dest}|}|\\
\end{tabular}
\end{center}
%
The argument \textit{dest} is the destination file
(without extension).
It should be the main file or one of the child files.
Note that further \textsf{childdoc} directives
such as |\childdocof| and |\childdocforward|
in the indicated file will be processed in this form.
The optional argument \textit{main}
passes on directly to the main file \textit{main}
while pretending to compile the child \textit{dest}.
This form behaves as if \textit{dest}
issues |\childdocof{|\textit{main}|}| right away,
and no further \textsf{childdoc} directives will be processed.

%%%%%%%%%%%%%%%%%%%%%%%%%%%%%%%%%%%%%%%%
\DescribeMacro{\...prefix}
In the alternative form |\childdocforwardprefix|,
%
\begin{center}
\begin{tabular}{l}
|\input{childdoc.def}|\\
|\childdocforwardprefix[|\textit{main}|]{|\textit{prefix}|}{|\textit{dest}|}|
\end{tabular}
\end{center}
%
the destination file is determined by a pattern
depending on the current file:
To make this work, the current file must be called
`{\textit{prefix}\hspace{0.2em}\textit{suffix}}'
with \textit{prefix} matching precisely the argument.
Processing is then passed on to the file
`{\textit{dest}\hspace{0.2em}\textit{suffix}}'.
Surely, the same effect is achieved by
directly specifying the
argument `{\textit{dest}\hspace{0.2em}\textit{suffix}}'
in the first form.
However, that requires to set up a different file
for each child. With the alternative form of the command
all these files can have exactly the same content
which simplifies setting them up and maintaining them.

For example, the following file |draft.tex|
with a compilation flag |\version| as described in \secref{sec:flags}
compiles the main document as a draft:
%
\begin{center}
\begin{tabular}{l}
|\def\version{draft}|\\
|\input{childdoc.def}|\\
|\childdocforward{|\textit{main}|}|
\end{tabular}
\end{center}
%
Likewise, the following files |final|\textit{nn}|.tex|
compile the final version of the child document
|child|\textit{nn}|.tex|:
%
\begin{center}
\begin{tabular}{l}
|\def\version{final}|\\
|\input{childdoc.def}|\\
|\childdocforwardprefix{final}{child}|
\end{tabular}
\end{center}
%

Note that when several versions of a main file and/or of each child file
are to be generated, it may be convenient to set up a |Makefile| or
shell script to automatise the process.

%%%%%%%%%%%%%%%%%%%%%%%%%%%%%%%%%%%%%%%%%%%%%%%%%%%%%%%%%%%%%%%%%%%%%%%%%%%%%%%%
\subsection{Command Line Processing}
\label{sec:commandline}

The effect of redirection files can also be achieved by invoking
the \LaTeX{} compiler with a more elaborate command line.
Most conveniently this should be done as part
of a shell script or a |Makefile|.

When using \textsf{childdoc} in the main file, the following
command lines effectively perform a redirection
(note that depending on the shell being used,
backslashes may have to be doubled: `|\|' $\to$ `|\\|'):
%
\begin{center}
|... -jobname "|\textit{target}|" |\\|"|[\textit{flags}]%
|\input{childdoc.def}\childdocforward[|\textit{main}|]{|\textit{dest}|}"|
\end{center}
%
Here \textit{target} is the name of the output file,
\textit{main} is the name of the main file
and \textit{dest} is the name of the main or child file to be processed
(all filenames without extensions).
The optional argument \textit{main} can be omitted
if \textit{main} matches \textit{dest}.
Optionally, compilation \textit{flags} can be defined via |\def| commands.
This command line makes the \TeX{} engine believe
it is compiling the file \textit{target}
whose content is specified as the latter parameter.
The provided code then forwards the processing to
\textit{main} or \textit{dest} as described in \secref{sec:forward}.

%%%%%%%%%%%%%%%%%%%%%%%%%%%%%%%%%%%%%%%%%%%%%%%%%%%%%%%%%%%%%%%%%%%%%%%%%%%%%%%%
\subsection{Include by Input}
\label{sec:input}

Including child documents by |\include| has some restrictions by design.
Most notably, the content of a child document always occupies
its own set of pages; pages cannot be shared between child documents.
Usually, this behaviour makes perfect sense
because each child document contain an essential part of the document.
However, in some situations it may be desirable to compose
a document from a collection of parts
without having mandatory page breaks between then.
For this case, the package
provides a mechanism to include parts
by |\input| which can also be processed individually.
However, by construction this mechanism
requires manual handling of the content to be output.

%%%%%%%%%%%%%%%%%%%%%%%%%%%%%%%%%%%%%%%%
\DescribeMacro{\ifchilddocmanual}
The main file should be prepared as usual, see \secref{sec:include}.
However, the document body must make a distinction
between processing of an individual part and of the main document, e.g.:
%
\begin{center}
\begin{tabular}{l}
|\ifchilddocmanual|\\
|\input{\childdocname}|\\
|\||else|\\
\textit{document body with }|\input{|\textit{part}|}|\\
|\||fi|
\end{tabular}
\end{center}
%
The conditional |\ifchilddocmanual| is true whenever
a part to be included by |\input| is being compiled,
and the name of the part is stored in |\childdocname|.

%%%%%%%%%%%%%%%%%%%%%%%%%%%%%%%%%%%%%%%%
\DescribeMacro{\childdocby}
Each part to be included by |\input| should start with:
%
\begin{center}
\begin{tabular}{l}
|\input{childdoc.def}|\\
|\childdocby{|\textit{main}|}|\\
\end{tabular}
\end{center}
%
The directive |\childdocby| is similar to |\childdocof|
described in \secref{sec:include},
but the subsequent selection of content must be done manually.
To that end, both |\ifchilddoc| and |\ifchilddocmanual|
will be true upon processing of a part,
and the name of the part is stored in |\childdocname|.
Note that |\jobname| will be set to the filename of the current part
so that each part receives an individual |.aux| file
that does not interfere with the |.aux| file(s) of the main document.
This behaviour can be altered by the alternative form
|\childdocby[*]{|\textit{main}|}| (with a non-empty optional argument)
which uses the |.aux| file of the main document
by setting |\jobname| to \textit{main}.

%%%%%%%%%%%%%%%%%%%%%%%%%%%%%%%%%%%%%%%%%%%%%%%%%%%%%%%%%%%%%%%%%%%%%%%%%%%%%%%%
\subsection{Driver Development}
\label{sec:driver}

The \textsf{childdoc} mechanism can also be use for the development
of definition files such as \LaTeX{} styles or classes.
This case differs from the above setup with multiple parts
included by |\include| in that no |\includeonly| should be invoked.
This can be achieved by starting the include file
(before |\ProvidesPackage|) with:
%
\begin{center}
\begin{tabular}{l}
|\input{childdoc.def}|\\
|\childdocforward{|\textit{main}|}|\\
\end{tabular}
\end{center}
%
or alternatively with:
%
\begin{center}
\begin{tabular}{l}
|\input{childdoc.def}|\\
|\childdocby{|\textit{main}|}|\\
\end{tabular}
\end{center}
%
Both forms have slightly different effects as described above.
The main file is prepared as usual, see \secref{sec:include}.

%%%%%%%%%%%%%%%%%%%%%%%%%%%%%%%%%%%%%%%%%%%%%%%%%%%%%%%%%%%%%%%%%%%%%%%%%%%%%%%%
\subsection{Legacy Detection}
\label{sec:detection}

The directive |\childdocmain| in the main file can detect
whether the complete document or merely a child is to be compiled
even without using the directive |\childdocof|.
This method is deprecated because it is less robust
and there is no compelling reason to use it;
it is merely provided for backward compatibility
and it may be removed in future versions.

If the detection mechanism is to be used,
it is mandatory to correctly specify
the filename of the main file as the argument of |\childdocmain|:
%
\begin{center}
\begin{tabular}{l}
|\input{childdoc.def}|\\
|\childdocmain{|\textit{main}|}|\\
\end{tabular}
\end{center}
%
If |\jobname| does not match the argument \textit{main} of |\childdocmain|,
it is assumed that |\jobname| points to the child file to be compiled.
When using |\childdocmain| with the main file specified as argument,
it suffices to start a child file
with just |\input{|\textit{main}|}|
without loading of the package and using |\childdocof|.
If instead all processing is done
with the appropriate \textsf{childdoc} directives,
the argument of \textit{main} of |\childdocmain| can be empty.

An alternative version of the command line processing described
in \secref{sec:commandline} using the detection mechanism reads:
%
\begin{center}
|... -jobname "|\textit{target}|" "|[\textit{flags}]%
[|\def\jobname{|\textit{dest}|}|]|\input{|\textit{main}|}"|
\end{center}

%%%%%%%%%%%%%%%%%%%%%%%%%%%%%%%%%%%%%%%%%%%%%%%%%%%%%%%%%%%%%%%%%%%%%%%%%%%%%%%%
\subsection{Manual Code}
\label{sec:manual}

In case one cannot be certain whether the definitions file |childdoc.def|
is installed on the target \TeX{} distribution
and one prefers not to ship it,
it is conceivable to paste a few relevant commands into the sources.

To that end, drop all statements |\input{childdoc.def}|
and perform the replacements as outlined below.
Instead of |\childdocmain{|\textit{main}|}| add the following code
to the top of the main file:
%
\begin{center}
\begin{tabular}{l}
|\||ifdefined\childdocname\endinput\||fi\newif\ifchilddoc|\\
|\edef\childdocname{\scantokens\expandafter{\jobname\noexpand}}|\\
|\def\childdocmain{|\textit{main}|}\||ifx\childdocmain\childdocname\||else|\\
|\childdoctrue\includeonly{\childdocname}\let\jobname\childdocmain\||fi|\\
\end{tabular}
\end{center}
%
Instead of |\childdocof{|\textit{main}|}| just include the main file
at the top of each child file:
%
\begin{center}
|\input{|\textit{main}|}|
\end{center}
%
A simple redirection |\childdocforward{|\textit{dest}|}| is achieved by:
%
\begin{center}
|\def\jobname{|\textit{dest}|}\input{\jobname}|
\end{center}
%
The redirection with prefix
|\childdocforwardprefix[|\textit{prefix}|]{|\textit{dest}|}|
is accomplished by:
%
\begin{center}
\begin{tabular}{l}
|{\edef\jobname{\scantokens\expandafter{\jobname\noexpand}}|\\
|\def\redirectjob |\textit{prefix}|#1~~~{\gdef\jobname{|\textit{dest}|#1}}|\\
|\expandafter\redirectjob\jobname~~~}\input{\jobname}|
\end{tabular}
\end{center}

In an alternative approach,
child documents can be compiled by a specific command line
without additional code or specific definitions:
%
\begin{center}
|... -jobname "|\textit{target}|" "|[\textit{flags}]%
|\includeonly{|\textit{dest}|}\input{|\textit{main}|}"|
\end{center}
%

%%%%%%%%%%%%%%%%%%%%%%%%%%%%%%%%%%%%%%%%%%%%%%%%%%%%%%%%%%%%%%%%%%%%%%%%%%%%%%%%
%%%%%%%%%%%%%%%%%%%%%%%%%%%%%%%%%%%%%%%%%%%%%%%%%%%%%%%%%%%%%%%%%%%%%%%%%%%%%%%%
\section{Information}

%%%%%%%%%%%%%%%%%%%%%%%%%%%%%%%%%%%%%%%%%%%%%%%%%%%%%%%%%%%%%%%%%%%%%%%%%%%%%%%%
\subsection{Copyright}

Copyright \copyright{} 2017--2018 Niklas Beisert

This work may be distributed and/or modified under the
conditions of the \LaTeX{} Project Public License, either version 1.3
of this license or (at your option) any later version.
The latest version of this license is in
  \url{http://www.latex-project.org/lppl.txt}
and version 1.3 or later is part of all distributions of \LaTeX{}
version 2005/12/01 or later.

This work has the LPPL maintenance status `maintained'.

The Current Maintainer of this work is Niklas Beisert.

This work consists of the files |README.txt|, |childdoc.ins| and |childdoc.dtx|
as well as the derived files |childdoc.def|, |cdocsamp.tex|
with |cdocsch1.tex|, |cdocsch2.tex|, |cdocspt3.tex|, |cdocspt4.tex|,
|cdocsdrf.tex|, |cdocsfn1.tex|, |cdocsfn2.tex|
as well as |childdoc.pdf|.

%%%%%%%%%%%%%%%%%%%%%%%%%%%%%%%%%%%%%%%%%%%%%%%%%%%%%%%%%%%%%%%%%%%%%%%%%%%%%%%%
\subsection{Files and Installation}

The package consists of the files:
%
\begin{center}
\begin{tabular}{ll}
    |README.txt|   & readme file \\
    |childdoc.ins| & installation file \\
    |childdoc.dtx| & source file \\
    |childdoc.def| & definition file \\
    |cdocsamp.tex| & sample main file \\
    |cdocsch1.tex| & sample include file \\
    |cdocsch2.tex| & sample include file \\
    |cdocspt3.tex| & sample part file \\
    |cdocspt4.tex| & sample part file \\
    |cdocsdrf.tex| & sample redirection file \\
    |cdocsfn1.tex| & sample redirection file \\
    |cdocsfn2.tex| & sample redirection file \\
    |childdoc.pdf| & manual
\end{tabular}
\end{center}
%
The distribution consists of the files
|README.txt|, |childdoc.ins| and |childdoc.dtx|.
%
\begin{itemize}
\item
Run (pdf)\LaTeX{} on |childdoc.dtx|
to compile the manual |childdoc.pdf| (this file).
\item
Run \LaTeX{} on |childdoc.ins| to create the definitions file |childdoc.def|
and the sample |cdocsamp.tex| with include files
|cdocsch1.tex|, |cdocsch2.tex|, |cdocspt3.tex|, |cdocspt4.tex|,
|cdocsdrf.tex|, |cdocsfn1.tex|, |cdocsfn2.tex|.
Then copy the file |childdoc.def| to an appropriate directory of your \LaTeX{}
distribution, e.g.\ \textit{texmf-root}|/tex/latex/childdoc|.
\end{itemize}

%%%%%%%%%%%%%%%%%%%%%%%%%%%%%%%%%%%%%%%%%%%%%%%%%%%%%%%%%%%%%%%%%%%%%%%%%%%%%%%%
\subsection{Related CTAN Packages}

There are several other packages which offer a similar functionality:
%
\begin{itemize}
\item
The packages
\href{http://ctan.org/pkg/docmute}{\textsf{docmute}},
\href{http://ctan.org/pkg/includex}{\textsf{includex}} and
\href{http://ctan.org/pkg/standalone}{\textsf{standalone}}
provide commands to include only the document body of
a child file thus allowing both files to be compiled individually.
\item
The packages \href{http://ctan.org/pkg/subdocs}{\textsf{subdocs}}
and \href{http://ctan.org/pkg/subfiles}{\textsf{subfiles}}
provide structures in which the main and child documents can be
encapsulated and allowing them to be compiled individually.
The inclusion mechanism is different from the conventional |\include|.
\item
The package \href{http://ctan.org/pkg/combine}{\textsf{combine}}
is an elaborate solution to combine several documents into one.
\end{itemize}
%
See also the CTAN topic \href{http://ctan.org/topic/subdocs}{\textsf{subdocs}}
for further related packages.
The present package differs from the above solutions in that
a document structure constructed with the conventional |\include| mechanism
just needs two extra commands at the top of every file
such that all constituent files can be compiled individually.

%%%%%%%%%%%%%%%%%%%%%%%%%%%%%%%%%%%%%%%%%%%%%%%%%%%%%%%%%%%%%%%%%%%%%%%%%%%%%%%%
%\subsection{Feature Suggestions}
%
%The following is a list of features which may be useful for future
%versions of this package:
%%
%\begin{itemize}
%\item
%\ldots
%\end{itemize}

%%%%%%%%%%%%%%%%%%%%%%%%%%%%%%%%%%%%%%%%%%%%%%%%%%%%%%%%%%%%%%%%%%%%%%%%%%%%%%%%
\subsection{Revision History}

%%%%%%%%%%%%%%%%%%%%%%%%%%%%%%%%%%%%%%%%
\paragraph{v2.0:} 2018/12/30

\begin{itemize}
\item
immediate forward processing
\item
added |\childdocby| mechanism
\item
manual restructured
\end{itemize}

%%%%%%%%%%%%%%%%%%%%%%%%%%%%%%%%%%%%%%%%
\paragraph{v1.6:} 2018/01/17

\begin{itemize}
\item
application for development of include files
\item
corrections to manual
\end{itemize}

%%%%%%%%%%%%%%%%%%%%%%%%%%%%%%%%%%%%%%%%
\paragraph{v1.5:} 2017/05/21

\begin{itemize}
\item
more complete structuring introduced
\item
|\childdocof| introduced
\item
|\childdoc| renamed to |\childdocmain|
\item
|\childredirect| renamed to |\childdocforward| and |\childdocforwardprefix|
and functionality expanded
\end{itemize}

%%%%%%%%%%%%%%%%%%%%%%%%%%%%%%%%%%%%%%%%
\paragraph{v1.0:} 2017/04/27

\begin{itemize}
\item
manual and install package
\item
first version published on CTAN
\end{itemize}

%%%%%%%%%%%%%%%%%%%%%%%%%%%%%%%%%%%%%%%%
\paragraph{v0.6:} 2017/04/26

\begin{itemize}
\item
redirection mechanism added
\end{itemize}

%%%%%%%%%%%%%%%%%%%%%%%%%%%%%%%%%%%%%%%%
\paragraph{v0.5:} 2017/04/26

\begin{itemize}
\item
functionality in definition file
\end{itemize}


%%%%%%%%%%%%%%%%%%%%%%%%%%%%%%%%%%%%%%%%%%%%%%%%%%%%%%%%%%%%%%%%%%%%%%%%%%%%%%%%
%%%%%%%%%%%%%%%%%%%%%%%%%%%%%%%%%%%%%%%%%%%%%%%%%%%%%%%%%%%%%%%%%%%%%%%%%%%%%%%%
%%%%%%%%%%%%%%%%%%%%%%%%%%%%%%%%%%%%%%%%%%%%%%%%%%%%%%%%%%%%%%%%%%%%%%%%%%%%%%%%
\appendix

\settowidth\MacroIndent{\rmfamily\scriptsize 000\ }

 \DocInput{childdoc.dtx}

\end{document}
%</driver>
% \fi
%
% %%%%%%%%%%%%%%%%%%%%%%%%%%%%%%%%%%%%%%%%%%%%%%%%%%%%%%%%%%%%%%%%%%%%%%%%%%%%%%
% %%%%%%%%%%%%%%%%%%%%%%%%%%%%%%%%%%%%%%%%%%%%%%%%%%%%%%%%%%%%%%%%%%%%%%%%%%%%%%
% \section{Sample}
%\iffalse
%<*samplemain>
%\fi
%
% The following presents a sample document
% with two chapters, two parts, a title page,
% a compile flag as well as three forwarding files to set the flag.
% It consists of eight |.tex| files:
% \begin{center}
% \begin{tabular}{ll}
% |cdocsamp.tex|&main file\\
% |cdocsch1.tex|&include file for chapter 1\\
% |cdocsch2.tex|&include file for chapter 2\\
% |cdocspt3.tex|&include file for part 3\\
% |cdocspt4.tex|&include file for part 4\\
% |cdocsdrf.tex|&forwarding file for main file in draft mode\\
% |cdocsfi1.tex|&forwarding file for final version of chapter 1\\
% |cdocsfi2.tex|&forwarding file for final version of chapter 2\\
% \end{tabular}
% \end{center}
% Each of the eight files can be compiled directly by the \LaTeX{} compiler.
%
% %%%%%%%%%%%%%%%%%%%%%%%%%%%%%%%%%%%%%%
% \paragraph{Main File.}
%
% The main file is called |cdocsamp.tex|.
%
% Load the \textsf{childdoc} definitions and
% declare the filename for the main document:
%    \begin{macrocode}
\input{childdoc.def}
\childdocmain{}
%    \end{macrocode}

% Optional override for |\version| flag:
%    \begin{macrocode}
%%\ifchilddoc\else\providecommand{\version}{draft}\fi
%    \end{macrocode}

% Define the default values for the |\version| flag
% (|final| for the main file and |draft| for childs):
%    \begin{macrocode}
\ifchilddoc
\providecommand{\version}{draft}
\else
\providecommand{\version}{final}
\fi
%    \end{macrocode}

% Load the standard document class:
%    \begin{macrocode}
\documentclass[12pt]{article}
%    \end{macrocode}

% Start the document body:
%    \begin{macrocode}
\begin{document}
%    \end{macrocode}

% Declare a title page.
% Print title, part of document being processed and version flag:
%    \begin{macrocode}
\addtocounter{page}{-1}
\begin{center}
{\LARGE\bfseries{}childdoc example\par}
\vspace{1cm}
\ifchilddoc
\ifchilddocmanual part\else chapter\fi:
`\childdocname' of `\childdocjob'\par
\else
main document: `\childdocjob'\par
\fi
version: \version\par
\end{center}
\newpage
%    \end{macrocode}

% Manually include selected file,
% otherwise process as usual:
%    \begin{macrocode}
\ifchilddocmanual
\section*{part `\childdocname'}
\input{\childdocname}
\else
%    \end{macrocode}

% Include the two chapters:
%    \begin{macrocode}
\include{cdocsch1}
\include{cdocsch2}
%    \end{macrocode}

% Include the two parts unless only chapters should be displayed:
%    \begin{macrocode}
\ifchilddoc\else
\section{part three}
\input{cdocspt3}
\section{part four}
\input{cdocspt4}
\fi
%    \end{macrocode}

% Process as usual until here:
%    \begin{macrocode}
\fi
%    \end{macrocode}

% End of document body:
%    \begin{macrocode}
\end{document}
%    \end{macrocode}
%\iffalse
%</samplemain>
%\fi
%
% %%%%%%%%%%%%%%%%%%%%%%%%%%%%%%%%%%%%%%
% \paragraph{Chapter Include Files.}
%
% The include files are called |cdocsch1.tex| and |cdocsch2.tex|.
%
%\iffalse
%<*samplechap1|samplechap2>
%\fi

% Optional override for |\version| flag:
%    \begin{macrocode}
%%\providecommand{\version}{final}
%    \end{macrocode}

% Include the main document:
%    \begin{macrocode}
\input{childdoc.def}
\childdocof{cdocsamp}
%    \end{macrocode}

%\iffalse
%</samplechap1|samplechap2>
%\fi
%
%\iffalse
%<*samplechap1>
%\fi
% Some text for chapter 1:
%    \begin{macrocode}
\section{one}
some text in chapter one
%    \end{macrocode}

%\iffalse
%</samplechap1>
%\fi
% Some text for chapter 2:
%\iffalse
%<*samplechap2>
%\fi
%    \begin{macrocode}
\section{two}
more text in chapter two
%    \end{macrocode}

%\iffalse
%</samplechap2>
%\fi
%
% %%%%%%%%%%%%%%%%%%%%%%%%%%%%%%%%%%%%%%
% \paragraph{Part Include Files.}
%
% The include files are called |cdocspt3.tex| and |cdocspt4.tex|.
%
%\iffalse
%<*samplepart3|samplepart4>
%\fi

% Optional override for |\version| flag:
%    \begin{macrocode}
%%\providecommand{\version}{final}
%    \end{macrocode}

% Include the main document:
%    \begin{macrocode}
\input{childdoc.def}
\childdocby{cdocsamp}
%    \end{macrocode}

%\iffalse
%</samplepart3|samplepart4>
%\fi
%
%\iffalse
%<*samplepart3>
%\fi
% Some text for part 3:
%    \begin{macrocode}
some text in part three
%    \end{macrocode}

%\iffalse
%</samplepart3>
%\fi
% Some text for part 4:
%\iffalse
%<*samplepart4>
%\fi
%    \begin{macrocode}
more text in part four
%    \end{macrocode}

%\iffalse
%</samplepart4>
%\fi
%
% %%%%%%%%%%%%%%%%%%%%%%%%%%%%%%%%%%%%%%
% \paragraph{Forwarding for a Complete Draft.}
%
% The following forwarding file |cdocsdrf.tex|
% compiles the main document in draft mode:
%\iffalse
%<*sampledraft>
%\fi
%    \begin{macrocode}
\def\version{draft}
\input{childdoc.def}
\childdocforward{cdocsamp}
%    \end{macrocode}

%\iffalse
%</sampledraft>
%\fi
%
% %%%%%%%%%%%%%%%%%%%%%%%%%%%%%%%%%%%%%%
% \paragraph{Forwarding for Final Version of the Chapters.}
%
% The following forwarding files |cdocsfn1.tex| and |cdocsfn2.tex|
% (with identical content)
% compile the final versions of the child documents
% |cdocsch1.tex| and |cdocsch2.tex|, respectively:
%\iffalse
%<*samplefinal>
%\fi
%    \begin{macrocode}
\def\version{final}
\input{childdoc.def}
\childdocforwardprefix[cdocsamp]{cdocsfn}{cdocsch}
%    \end{macrocode}

%\iffalse
%</samplefinal>
%\fi
%
% %%%%%%%%%%%%%%%%%%%%%%%%%%%%%%%%%%%%%%
% \paragraph{Command Line Processing.}
%
% The following three command lines generate the output files
% |cdocscld|, |cdocscl1| and |cdocscl2|
% which should be identical to
% |cdocsdrf|, |cdocsch1| and |cdocsfn2|, respectively:
% \begin{center}
% \begin{tabular}{l}
% |latex -jobname cdocscld \|\\
% |  "\def\version{draft}\input{childdoc.def}\childdocforward{cdocsamp}"|\\
% |latex -jobname cdocscl1 \|\\
% |  "\input{childdoc.def}\childdocforward[cdocsamp]{cdocsch1}"|\\
% |latex -jobname cdocscl2 \|\\
% |  "\def\version{final}\input{childdoc.def}\childdocforward{cdocsch2}"|
% \end{tabular}
% \end{center}
% Note that the trailing backslash on each first line
% merely continues the input to the second line
% (for convenient cut ant paste).
% Furthermore, the command |latex| can be replaced by any
% of its alternative versions such as |pdflatex|.
%
% %%%%%%%%%%%%%%%%%%%%%%%%%%%%%%%%%%%%%%%%%%%%%%%%%%%%%%%%%%%%%%%%%%%%%%%%%%%%%%
% %%%%%%%%%%%%%%%%%%%%%%%%%%%%%%%%%%%%%%%%%%%%%%%%%%%%%%%%%%%%%%%%%%%%%%%%%%%%%%
% \section{Implementation}
%\iffalse
%<*package>
%\fi
%
% This section describes the definitions file |childdoc.def|.

% The definitions cannot be loaded using |\usepackage| or |\RequirePackage|
% which has a mechanism to prevent loading a style file more than once.
% When loading the definitions by means of |\input|
% multiple instances have to be prevented manually:
%\iffalse
%This code needs to be before the `\ProvidesFile' directive
%which is defined at the beginning of this file.
%Therefore it is also placed there and commented out here.
%</package>
%<*discard>
%\fi
%    \begin{macrocode}
\ifdefined\childdocmain\endinput\fi
%    \end{macrocode}
%\iffalse
%</discard>
%<*package>
%\fi
%
% \macro{\ifchilddoc}
% \macro{\ifchilddocmanual}
% The conditional |\ifchilddoc| tells whether a
% child (true) or main (false) document is being compiled.
% The conditional |\ifchilddocmanual| tells whether
% the |\includeonly| mechanism is used (false) or
% the selection of child files must be performed manually (true).
% The definitions initialise to false:
%    \begin{macrocode}
\newif\ifchilddoc
\newif\ifchilddocmanual
%    \end{macrocode}

% \macro{\childdocname}
% \macro{\childdocjob}
% The macro |\childdocname| stores the name of the main document
% to be compiled. The macro |\childdocjob| stores the name of
% the document on which the \LaTeX{} compiler was originally invoked.
% The content of |\jobname| cannot be compared
% to filenames specified in the source due to different catcodes.
% The following code rescans |\jobname|, stores the result
% in |\childdocname| and saves a copy in |\childdocjob|:
%    \begin{macrocode}
\edef\childdocname{\scantokens\expandafter{\jobname\noexpand}}
\let\childdocjob\childdocname
%    \end{macrocode}

% \macro{\childdocdisable}
% The macro |\childdocdisable| prevents the main file
% from being processed more than once.
% At this stage, the main document command |\childdocmain|
% is assumed to be called once again where it should do nothing.
% Any subsequent call to it should prevent
% a secondary processing of the main document
% It overwrites the forwarding commands
% |\childdocof| and |\childdocforward|
% with empty macros to prevent further inclusions of the main document:
%    \begin{macrocode}
\newcommand{\childdocdisable}
{
  \renewcommand{\childdocmain}[1]{\renewcommand{\childdocmain}[1]{\endinput}}
  \renewcommand{\childdocof}[1]{}
  \renewcommand{\childdocby}[2][]{}
  \renewcommand{\childdocforward}[2][]{}
  \renewcommand{\childdocdisable}{}
}
%    \end{macrocode}

% \macro{\childdocmain}
% The macro |\childdocmain| is to be called at the top of the main file
% with nothing or the main filename (without extension) as argument.
% First, it breaks loops.
% If the argument is not empty and does not match |\childdocname|
% (which is set by the first inclusion of |childdoc.def|),
% |\ifchilddoc| is set to true, |\includeonly| is applied to the child file
% and |\jobname| is set to the main file
% (for proper handling of |.aux| files):
%    \begin{macrocode}
\newcommand{\childdocmain}[1]
{
  \childdocdisable\childdocmain{}
  \if?#1?\else
    \begingroup
      \def\childdoctmp{#1}
      \ifx\childdoctmp\childdocname
        \def\childdoctmp{}
      \else
        \def\childdoctmp
        {
          \childdoctrue
          \includeonly{\childdocname}
          \def\childdocjob{#1}
          \def\jobname{#1}
        }
      \fi
      \expandafter
    \endgroup
    \childdoctmp
  \fi
}
%    \end{macrocode}

% \macro{\childdocof}
% The command |\childdocof| redirects
% compilation to the main file |#1|.
%    \begin{macrocode}
\newcommand{\childdocof}[1]
{
  \childdocdisable
  \childdoctrue
  \includeonly{\childdocname}
  \def\jobname{#1}
  \def\childdocjob{#1}
  \input{#1}
}
%    \end{macrocode}

% \macro{\childdocby}
% The command |\childdocby| ....
%    \begin{macrocode}
\newcommand{\childdocby}[2][]
{
  \childdocdisable
  \childdoctrue
  \childdocmanualtrue
  \if?#1?\else
    \def\jobname{#2}
  \fi
  \def\childdocjob{#2}
  \input{#2}
  \endinput
}
%    \end{macrocode}

% \macro{\childdocforward}
% The command |\childdocforward| redirects
% compilation to the main file or
% (if the optional argument is given) a child file.
% Parameters are set as if the main file
% or a child file starting with |\childdocof| was compiled.
% Then compilation is handed over to the main file:
%    \begin{macrocode}
\newcommand{\childdocforward}[2][]
{
  \begingroup
    \if?#1?
      \def\childdoctmp
      {
        \def\childdocname{#2}
        \def\childdocjob{#2}
        \def\jobname{#2}
        \input{#2}
        \endinput
      }
    \else
      \def\childdoctmp
      {
        \childdocdisable
        \def\childdocname{#2}
        \childdoctrue
        \includeonly{#2}
        \def\childdocjob{#1}
        \def\jobname{#1}
        \input{#1}
        \endinput
      }
    \fi
    \expandafter
  \endgroup
  \childdoctmp
}
%    \end{macrocode}

% \macro{\childdocforwardprefix}
% The command |\childdocforwardprefix| redirects
% compilation to the main or a child file by means of a pattern.
% The prefix |#1| in the current filename is replaced by |#2|
% and the suffix of the current filename is kept
% (it is assumed that the filename does not contain the substring `|~~~|'
% which is used as a delimiter).
% Compilation is handed over to the new file by |\childdocforward|:
%    \begin{macrocode}
\newcommand{\childdocforwardprefix}[3][]
{
  \begingroup
    \def\childdocextract #2##1~~~{\def\childdoctmp{\childdocforward[#1]{#3##1}}}
    \expandafter\childdocextract\childdocname~~~
    \expandafter
  \endgroup
  \childdoctmp
}
%    \end{macrocode}

% \macro{\childdoc}
% The deprecated macro |\childdoc| is a legacy version of |\childdocmain|:
%    \begin{macrocode}
\newcommand{\childdoc}{\childdocmain}
%    \end{macrocode}

% \macro{\childdocredirect}
% The deprecated macro |\childdocredirect| is a legacy version
% of |\childdocforward| and |\childdocforwardprefix|:
%    \begin{macrocode}
\newcommand{\childdocredirect}[2][]
{
  \begingroup
    \if?#1?
      \def\childdoctmp{\childdocforward{#2}}
    \else
      \def\childdoctmp{\childdocforwardprefix{#1}{#2}}
    \fi
    \expandafter
  \endgroup
  \childdoctmp
}
%    \end{macrocode}

%\iffalse
%</package>
%\fi
%
\endinput
|\\
|\childdocforwardprefix[|\textit{main}|]{|\textit{prefix}|}{|\textit{dest}|}|
\end{tabular}
\end{center}
%
the destination file is determined by a pattern
depending on the current file:
To make this work, the current file must be called
`{\textit{prefix}\hspace{0.2em}\textit{suffix}}'
with \textit{prefix} matching precisely the argument.
Processing is then passed on to the file
`{\textit{dest}\hspace{0.2em}\textit{suffix}}'.
Surely, the same effect is achieved by
directly specifying the
argument `{\textit{dest}\hspace{0.2em}\textit{suffix}}'
in the first form.
However, that requires to set up a different file
for each child. With the alternative form of the command
all these files can have exactly the same content
which simplifies setting them up and maintaining them.

For example, the following file |draft.tex|
with a compilation flag |\version| as described in \secref{sec:flags}
compiles the main document as a draft:
%
\begin{center}
\begin{tabular}{l}
|\def\version{draft}|\\
|% \iffalse
%
% childdoc.dtx Copyright (C) 2017-2018 Niklas Beisert
%
% This work may be distributed and/or modified under the
% conditions of the LaTeX Project Public License, either version 1.3
% of this license or (at your option) any later version.
% The latest version of this license is in
%   http://www.latex-project.org/lppl.txt
% and version 1.3 or later is part of all distributions of LaTeX
% version 2005/12/01 or later.
%
% This work has the LPPL maintenance status `maintained'.
%
% The Current Maintainer of this work is Niklas Beisert.
%
% This work consists of the files childdoc.dtx and childdoc.ins
% and the derived files childdoc.def and cdocsamp.tex with
% cdocsch1.tex, cdocsch2.tex, cdocsdrf.tex, cdocsfn1.tex, cdocsfn2.tex.
%
%<package>\ifdefined\childdocmain\endinput\fi
%<package>\ProvidesFile{childdoc.def}[2018/12/30 v2.0 child document driver]
%<samplemain>\ProvidesFile{cdocsamp.tex}[2018/12/30 v2.0 sample for childdoc]
%<*driver>
%\ProvidesFile{childdoc.drv}[2018/12/30 v2.0 childdoc reference manual file]
\PassOptionsToClass{10pt,a4paper}{article}
\documentclass{ltxdoc}

\usepackage[margin=35mm]{geometry}
\usepackage{hyperref}
\usepackage{hyperxmp}
\usepackage[usenames]{color}

\hypersetup{colorlinks=true}
\hypersetup{pdfstartview=FitH}
\hypersetup{pdfpagemode=UseNone}
\hypersetup{pdfsource={}}
\hypersetup{pdflang={en-UK}}
\hypersetup{pdfcopyright={Copyright 2017-2018 Niklas Beisert.
  This work may be distributed and/or modified under the
  conditions of the LaTeX Project Public License, either version 1.3
  of this license or (at your option) any later version.}}
\hypersetup{pdflicenseurl={http://www.latex-project.org/lppl.txt}}
\hypersetup{pdfcontactaddress={ETH Zurich, ITP, HIT K,
  Wolfgang-Pauli-Strasse 27}}
\hypersetup{pdfcontactpostcode={8093}}
\hypersetup{pdfcontactcity={Zurich}}
\hypersetup{pdfcontactcountry={Switzerland}}
\hypersetup{pdfcontactemail={nbeisert@itp.phys.ethz.ch}}
\hypersetup{pdfcontacturl={http://people.phys.ethz.ch/\xmptilde nbeisert/}}

\newcommand{\secref}[1]{\hyperref[#1]{section \ref*{#1}}}

\parskip1ex
\parindent0pt
\let\olditemize\itemize
\def\itemize{\olditemize\parskip0pt}

\begin{document}

\title{The \textsf{childdoc} Package}
\hypersetup{pdftitle={The childdoc Package}}
\author{Niklas Beisert\\[2ex]
  Institut f\"ur Theoretische Physik\\
  Eidgen\"ossische Technische Hochschule Z\"urich\\
  Wolfgang-Pauli-Strasse 27, 8093 Z\"urich, Switzerland\\[1ex]
  \href{mailto:nbeisert@itp.phys.ethz.ch}
  {\texttt{nbeisert@itp.phys.ethz.ch}}}
\hypersetup{pdfauthor={Niklas Beisert}}
\hypersetup{pdfsubject={Manual for the LaTeX2e Package childdoc}}
\date{30 December 2018, \textsf{v2.0}}
\maketitle

\begin{abstract}\noindent
\textsf{childdoc} is a \LaTeXe{} package
that enables the direct compilation
of document sections included by |\include|
to individual files.
\end{abstract}

\begingroup
\parskip0ex
\tableofcontents
\endgroup

%%%%%%%%%%%%%%%%%%%%%%%%%%%%%%%%%%%%%%%%%%%%%%%%%%%%%%%%%%%%%%%%%%%%%%%%%%%%%%%%
%%%%%%%%%%%%%%%%%%%%%%%%%%%%%%%%%%%%%%%%%%%%%%%%%%%%%%%%%%%%%%%%%%%%%%%%%%%%%%%%
\section{Introduction}

\LaTeX{} provides a mechanism to structure a large document (such as a book)
into a main file and several child files (containing the chapters)
using the |\include| command.
This mechanism is beneficial for documents
which span hundreds of pages in order to
make the source file(s) more manageable.
Moreover, compilation can be restricted to
selected child files by means of the |\includeonly| command.
The latter feature can be used to reduce the compilation time while editing
(this was significantly more useful in the earlier days of \LaTeX{})
or to generate a smaller document which is easier to navigate.
Another application of |\includeonly| is to generate
documents consisting of selected parts of the complete document.

However, there are a few drawbacks of the plain |\include| mechanism:
\begin{itemize}
\item
The child files cannot be compiled on their own,
they can only be compiled via the main file.
A naive editing environment
(such as a text editor with an option
to have the current file processed by \LaTeX)
may require one to switch to the main file before compiling;
attempting to compile the child file produces errors.
\item
The main file must be modified (each time)
to adjust the |\includeonly| command
to the present needs. This easily leaves the main file in a messy state.
\item
The generated document will always carry the filename
of the main document. This is inconvenient if
several child files are to be compiled and
to be kept for distribution.
\end{itemize}

The present package provides a simple interface
to make child files individually compilable by \LaTeX{}.
Compiling a child file then has the same effect as compiling
the main file with an |\includeonly| command
to select the appropriate child.
Moreover the generated document will carry the name of the child
rather than the main file.
This resolves all three above issues.

This feature is meant to make the editing of books,
thesis documents and lecture notes somewhat more convenient.
However, the package can also be used efficiently for
composing a series of documents (such as exercise sheets)
which are typically distributed individually.
It then assists the author in generating the individual documents
(potentially in different versions)
as well as a document containing the collected series.
Another application is in developing style files
or other kinds of included material
where compilation of the style file could redirect
to a sample or test file.

%%%%%%%%%%%%%%%%%%%%%%%%%%%%%%%%%%%%%%%%%%%%%%%%%%%%%%%%%%%%%%%%%%%%%%%%%%%%%%%%
%%%%%%%%%%%%%%%%%%%%%%%%%%%%%%%%%%%%%%%%%%%%%%%%%%%%%%%%%%%%%%%%%%%%%%%%%%%%%%%%
\section{Usage}

First of all, the package \textsf{childdoc} is \emph{not} a standard
\LaTeXe{} |.sty| style file! Therefore it needs to be invoked in
a non-standard way.

%%%%%%%%%%%%%%%%%%%%%%%%%%%%%%%%%%%%%%%%%%%%%%%%%%%%%%%%%%%%%%%%%%%%%%%%%%%%%%%%
\subsection{Included Files}
\label{sec:include}

%%%%%%%%%%%%%%%%%%%%%%%%%%%%%%%%%%%%%%%%
\DescribeMacro{\childdocmain}
To use the package, add the commands
\begin{center}
\begin{tabular}{l}
|\input{childdoc.def}|\\
|\childdocmain{}|\\
\end{tabular}
\end{center}
at the very top of the main \LaTeX{} file,
in particular \emph{before} the |\documentclass| statement!
The argument of |\childdocmain| should be left empty
(but it must be present).

%%%%%%%%%%%%%%%%%%%%%%%%%%%%%%%%%%%%%%%%
\DescribeMacro{\childdocof}
Furthermore, add the commands
\begin{center}
\begin{tabular}{l}
|\input{childdoc.def}|\\
|\childdocof{|\textit{main}|}|\\
\end{tabular}
\end{center}
at the top of every child file \textit{child}
which is included by |\include{|\textit{child}|}|
from within the main file
(or at least for those files to be compiled individually).
The argument \textit{main} must be the filename of the main file.

There are a couple of
considerations in setting up the main and child documents:

%%%%%%%%%%%%%%%%%%%%%%%%%%%%%%%%%%%%%%%%
\paragraph{Restrictions.}

Please note the following restrictions:
\begin{itemize}
\item
|\childdocmain| must be called with one argument \textit{main}
to ensure compatibility with earlier version of the package.
It must either be empty (|\childdocmain{}|)
or precisely match the filename of the main file in which it is specified.
See \secref{sec:detection} for further information.
\item
The filename \textit{main} must be specified without the |.tex| extension.
\item
The filename \textit{main} is case sensitive
(even in case-insensitive file systems)
due to internal string comparison.
\item
The argument \textit{main} should be fully expanded, it cannot be a macro.
\item
Subdirectories and special characters should be avoided in filenames.
\item
The command |\childdocmain{|\textit{main}|}| must be followed by a whitespace.
It should not be followed immediately by another command
or by a comment mark `|%|'.
This is because the \TeX{} parser reads the token immediately following
the argument of |\childdocmain| and puts it
at the beginning of every child section;
however, a white\-space is ignored.
\end{itemize}

%%%%%%%%%%%%%%%%%%%%%%%%%%%%%%%%%%%%%%%%
\paragraph{Content of Main File.}

It is advisable to place all content in the child files included by |\include|.
Any output contained in the main file will appear in all child documents
unless suppressed manually;
it cannot be suppressed automatically by the |\includeonly| directive
and thus should normally be avoided.
A method to include some content in the main file
by means of conditional processing is described in \secref{sec:conditional}.

%%%%%%%%%%%%%%%%%%%%%%%%%%%%%%%%%%%%%%%%
\paragraph{Page Numbering.}

When only a part of the document is compiled,
the appropriate numbering of pages
(as well as other status parameters)
is determined from the |.aux| files.
The latter contain information from previous passes.
However this information needs to propagate through
all intermediate child documents.
Therefore the page numbering in child documents may well
be inconsistent until the complete document is compiled at least once.

A useful (if unconventional) way to always ensure a consistent
page numbering is to restart the numbering in each child document
and denote the pages by `\textit{child}|.|\textit{page}'
where \textit{child} represents the chapter/section number of the child file.
This can be achieved by the command
|\numberwithin{page}{|\textit{child}|}|
of the \textsf{amsmath} package
where \textit{child} can be |chapter| or |section|
depending on the chosen structuring.
Alternatively, one can modify the macro |\thepage| appropriately
and reset the counter |page| at the start of each child file.

%%%%%%%%%%%%%%%%%%%%%%%%%%%%%%%%%%%%%%%%%%%%%%%%%%%%%%%%%%%%%%%%%%%%%%%%%%%%%%%%
\subsection{Conditional Processing}
\label{sec:conditional}

The package provides a mechanism to compile different versions
of a document. To customise the versions further some conditional processing
can come in handy to distinguish which version is being compiled.
The package provides two macros to describe the compilation context:

%%%%%%%%%%%%%%%%%%%%%%%%%%%%%%%%%%%%%%%%
\DescribeMacro{\ifchilddoc}
The conditional |\ifchilddoc| distinguishes between the compilation of
child documents and the main document:
%
\begin{center}
|\ifchilddoc |\textit{child-code}| |[|\||else |\textit{main-code}]| \||fi|
\end{center}

%%%%%%%%%%%%%%%%%%%%%%%%%%%%%%%%%%%%%%%%
\DescribeMacro{\childdocname}
\DescribeMacro{\childdocjob}
The macro |\childdocname| contains the filename (without extension)
of the main or child file being processed.
Note that |\childdocjob| will always contain the name of the main file.

%%%%%%%%%%%%%%%%%%%%%%%%%%%%%%%%%%%%%%%%
\paragraph{Title Page.}

Conditional processing can be used to include a title or banner page
in the main document when proper precautions are taken.
Importantly, the code in the main file should ensure that the page counter
(as well as other status parameters which are stored in the |.aux| files)
takes the same value after the conditional processing.
Otherwise the page numbers may take divergent values
depending on which part is compiled.

For example, a title page could be declared by:
%
\begin{center}
\begin{tabular}{l}
|\ifchilddoc\||else|\\
|\addtocounter{page}{-1}|\\
\textit{code for title page}\\
|\newpage|\\
|\||fi|
\end{tabular}
\end{center}
%
A banner page for the child documents can be generated by:
%
\begin{center}
\begin{tabular}{l}
|\ifchilddoc|\\
|\addtocounter{page}{-1}|\\
\textit{code for banner page}\\
|\newpage|\\
|\||fi|
\end{tabular}
\end{center}
%
Here one could write a message such as:
\begin{center}
|This is the part \childdocname{} of \childdocjob{}.|
\end{center}

%%%%%%%%%%%%%%%%%%%%%%%%%%%%%%%%%%%%%%%%%%%%%%%%%%%%%%%%%%%%%%%%%%%%%%%%%%%%%%%%
\subsection{Flags}
\label{sec:flags}

The package makes it easy to generate different versions
of the main or child documents.
To this end compilation flags can be defined
and assigned different default values.
They will be particularly useful in conjunction
with the forwarding mechanism described in \secref{sec:forward}.

For example, it may be useful to have a flag |\version|
which can be set to |draft| or |final|.
The document source will contain some conditional code
depending on the value of |\version|.
Suppose further, the flag should default to |final| for the main file
and to |draft| for child files
which is a natural assignment for editing the document.
This is achieved by placing the following code
in the preamble of the main document
(below the |\childdocmain| directive):
%
\begin{center}
\begin{tabular}{l}
|\ifchilddoc|\\
|\providecommand{\version}{draft}|\\
|\||else|\\
|\providecommand{\version}{final}|\\
|\||fi|
\end{tabular}
\end{center}
%
The definition by |\providecommand| makes sure
that previous definitions are not overwritten.
Further statements |\providecommand{\version}{...}|
can thus be added before the above code to override it.

For the main file, one might add a line
(between |\childdocmain| and the above block)
%
\begin{center}
|%\ifchilddoc\||else\providecommand{\version}{draft}\||fi|
\end{center}
%
which can be uncommented to produce a draft version.
Likewise one can add a line to the very top of a child file
(above the |\childdocof{|\textit{main}|}| directive)
%
\begin{center}
|%\providecommand{\version}{final}|
\end{center}
%
which can be uncommented to produce the final version of this child document.

%%%%%%%%%%%%%%%%%%%%%%%%%%%%%%%%%%%%%%%%%%%%%%%%%%%%%%%%%%%%%%%%%%%%%%%%%%%%%%%%
\subsection{Forwarding}
\label{sec:forward}

Different versions of the main or child documents
using compilation flags as described in \secref{sec:flags}
can be (permanently) stored in different files
for convenient compilation, viewing and distribution.
To this end, the package defines a command
to pass on compilation to a different file:

%%%%%%%%%%%%%%%%%%%%%%%%%%%%%%%%%%%%%%%%
\DescribeMacro{\childdocforward}
The command |\childdocforward| redirects processing to
another source file:
%
\begin{center}
\begin{tabular}{l}
|\input{childdoc.def}|\\
|\childdocforward[|\textit{main}|]{|\textit{dest}|}|\\
\end{tabular}
\end{center}
%
The argument \textit{dest} is the destination file
(without extension).
It should be the main file or one of the child files.
Note that further \textsf{childdoc} directives
such as |\childdocof| and |\childdocforward|
in the indicated file will be processed in this form.
The optional argument \textit{main}
passes on directly to the main file \textit{main}
while pretending to compile the child \textit{dest}.
This form behaves as if \textit{dest}
issues |\childdocof{|\textit{main}|}| right away,
and no further \textsf{childdoc} directives will be processed.

%%%%%%%%%%%%%%%%%%%%%%%%%%%%%%%%%%%%%%%%
\DescribeMacro{\...prefix}
In the alternative form |\childdocforwardprefix|,
%
\begin{center}
\begin{tabular}{l}
|\input{childdoc.def}|\\
|\childdocforwardprefix[|\textit{main}|]{|\textit{prefix}|}{|\textit{dest}|}|
\end{tabular}
\end{center}
%
the destination file is determined by a pattern
depending on the current file:
To make this work, the current file must be called
`{\textit{prefix}\hspace{0.2em}\textit{suffix}}'
with \textit{prefix} matching precisely the argument.
Processing is then passed on to the file
`{\textit{dest}\hspace{0.2em}\textit{suffix}}'.
Surely, the same effect is achieved by
directly specifying the
argument `{\textit{dest}\hspace{0.2em}\textit{suffix}}'
in the first form.
However, that requires to set up a different file
for each child. With the alternative form of the command
all these files can have exactly the same content
which simplifies setting them up and maintaining them.

For example, the following file |draft.tex|
with a compilation flag |\version| as described in \secref{sec:flags}
compiles the main document as a draft:
%
\begin{center}
\begin{tabular}{l}
|\def\version{draft}|\\
|\input{childdoc.def}|\\
|\childdocforward{|\textit{main}|}|
\end{tabular}
\end{center}
%
Likewise, the following files |final|\textit{nn}|.tex|
compile the final version of the child document
|child|\textit{nn}|.tex|:
%
\begin{center}
\begin{tabular}{l}
|\def\version{final}|\\
|\input{childdoc.def}|\\
|\childdocforwardprefix{final}{child}|
\end{tabular}
\end{center}
%

Note that when several versions of a main file and/or of each child file
are to be generated, it may be convenient to set up a |Makefile| or
shell script to automatise the process.

%%%%%%%%%%%%%%%%%%%%%%%%%%%%%%%%%%%%%%%%%%%%%%%%%%%%%%%%%%%%%%%%%%%%%%%%%%%%%%%%
\subsection{Command Line Processing}
\label{sec:commandline}

The effect of redirection files can also be achieved by invoking
the \LaTeX{} compiler with a more elaborate command line.
Most conveniently this should be done as part
of a shell script or a |Makefile|.

When using \textsf{childdoc} in the main file, the following
command lines effectively perform a redirection
(note that depending on the shell being used,
backslashes may have to be doubled: `|\|' $\to$ `|\\|'):
%
\begin{center}
|... -jobname "|\textit{target}|" |\\|"|[\textit{flags}]%
|\input{childdoc.def}\childdocforward[|\textit{main}|]{|\textit{dest}|}"|
\end{center}
%
Here \textit{target} is the name of the output file,
\textit{main} is the name of the main file
and \textit{dest} is the name of the main or child file to be processed
(all filenames without extensions).
The optional argument \textit{main} can be omitted
if \textit{main} matches \textit{dest}.
Optionally, compilation \textit{flags} can be defined via |\def| commands.
This command line makes the \TeX{} engine believe
it is compiling the file \textit{target}
whose content is specified as the latter parameter.
The provided code then forwards the processing to
\textit{main} or \textit{dest} as described in \secref{sec:forward}.

%%%%%%%%%%%%%%%%%%%%%%%%%%%%%%%%%%%%%%%%%%%%%%%%%%%%%%%%%%%%%%%%%%%%%%%%%%%%%%%%
\subsection{Include by Input}
\label{sec:input}

Including child documents by |\include| has some restrictions by design.
Most notably, the content of a child document always occupies
its own set of pages; pages cannot be shared between child documents.
Usually, this behaviour makes perfect sense
because each child document contain an essential part of the document.
However, in some situations it may be desirable to compose
a document from a collection of parts
without having mandatory page breaks between then.
For this case, the package
provides a mechanism to include parts
by |\input| which can also be processed individually.
However, by construction this mechanism
requires manual handling of the content to be output.

%%%%%%%%%%%%%%%%%%%%%%%%%%%%%%%%%%%%%%%%
\DescribeMacro{\ifchilddocmanual}
The main file should be prepared as usual, see \secref{sec:include}.
However, the document body must make a distinction
between processing of an individual part and of the main document, e.g.:
%
\begin{center}
\begin{tabular}{l}
|\ifchilddocmanual|\\
|\input{\childdocname}|\\
|\||else|\\
\textit{document body with }|\input{|\textit{part}|}|\\
|\||fi|
\end{tabular}
\end{center}
%
The conditional |\ifchilddocmanual| is true whenever
a part to be included by |\input| is being compiled,
and the name of the part is stored in |\childdocname|.

%%%%%%%%%%%%%%%%%%%%%%%%%%%%%%%%%%%%%%%%
\DescribeMacro{\childdocby}
Each part to be included by |\input| should start with:
%
\begin{center}
\begin{tabular}{l}
|\input{childdoc.def}|\\
|\childdocby{|\textit{main}|}|\\
\end{tabular}
\end{center}
%
The directive |\childdocby| is similar to |\childdocof|
described in \secref{sec:include},
but the subsequent selection of content must be done manually.
To that end, both |\ifchilddoc| and |\ifchilddocmanual|
will be true upon processing of a part,
and the name of the part is stored in |\childdocname|.
Note that |\jobname| will be set to the filename of the current part
so that each part receives an individual |.aux| file
that does not interfere with the |.aux| file(s) of the main document.
This behaviour can be altered by the alternative form
|\childdocby[*]{|\textit{main}|}| (with a non-empty optional argument)
which uses the |.aux| file of the main document
by setting |\jobname| to \textit{main}.

%%%%%%%%%%%%%%%%%%%%%%%%%%%%%%%%%%%%%%%%%%%%%%%%%%%%%%%%%%%%%%%%%%%%%%%%%%%%%%%%
\subsection{Driver Development}
\label{sec:driver}

The \textsf{childdoc} mechanism can also be use for the development
of definition files such as \LaTeX{} styles or classes.
This case differs from the above setup with multiple parts
included by |\include| in that no |\includeonly| should be invoked.
This can be achieved by starting the include file
(before |\ProvidesPackage|) with:
%
\begin{center}
\begin{tabular}{l}
|\input{childdoc.def}|\\
|\childdocforward{|\textit{main}|}|\\
\end{tabular}
\end{center}
%
or alternatively with:
%
\begin{center}
\begin{tabular}{l}
|\input{childdoc.def}|\\
|\childdocby{|\textit{main}|}|\\
\end{tabular}
\end{center}
%
Both forms have slightly different effects as described above.
The main file is prepared as usual, see \secref{sec:include}.

%%%%%%%%%%%%%%%%%%%%%%%%%%%%%%%%%%%%%%%%%%%%%%%%%%%%%%%%%%%%%%%%%%%%%%%%%%%%%%%%
\subsection{Legacy Detection}
\label{sec:detection}

The directive |\childdocmain| in the main file can detect
whether the complete document or merely a child is to be compiled
even without using the directive |\childdocof|.
This method is deprecated because it is less robust
and there is no compelling reason to use it;
it is merely provided for backward compatibility
and it may be removed in future versions.

If the detection mechanism is to be used,
it is mandatory to correctly specify
the filename of the main file as the argument of |\childdocmain|:
%
\begin{center}
\begin{tabular}{l}
|\input{childdoc.def}|\\
|\childdocmain{|\textit{main}|}|\\
\end{tabular}
\end{center}
%
If |\jobname| does not match the argument \textit{main} of |\childdocmain|,
it is assumed that |\jobname| points to the child file to be compiled.
When using |\childdocmain| with the main file specified as argument,
it suffices to start a child file
with just |\input{|\textit{main}|}|
without loading of the package and using |\childdocof|.
If instead all processing is done
with the appropriate \textsf{childdoc} directives,
the argument of \textit{main} of |\childdocmain| can be empty.

An alternative version of the command line processing described
in \secref{sec:commandline} using the detection mechanism reads:
%
\begin{center}
|... -jobname "|\textit{target}|" "|[\textit{flags}]%
[|\def\jobname{|\textit{dest}|}|]|\input{|\textit{main}|}"|
\end{center}

%%%%%%%%%%%%%%%%%%%%%%%%%%%%%%%%%%%%%%%%%%%%%%%%%%%%%%%%%%%%%%%%%%%%%%%%%%%%%%%%
\subsection{Manual Code}
\label{sec:manual}

In case one cannot be certain whether the definitions file |childdoc.def|
is installed on the target \TeX{} distribution
and one prefers not to ship it,
it is conceivable to paste a few relevant commands into the sources.

To that end, drop all statements |\input{childdoc.def}|
and perform the replacements as outlined below.
Instead of |\childdocmain{|\textit{main}|}| add the following code
to the top of the main file:
%
\begin{center}
\begin{tabular}{l}
|\||ifdefined\childdocname\endinput\||fi\newif\ifchilddoc|\\
|\edef\childdocname{\scantokens\expandafter{\jobname\noexpand}}|\\
|\def\childdocmain{|\textit{main}|}\||ifx\childdocmain\childdocname\||else|\\
|\childdoctrue\includeonly{\childdocname}\let\jobname\childdocmain\||fi|\\
\end{tabular}
\end{center}
%
Instead of |\childdocof{|\textit{main}|}| just include the main file
at the top of each child file:
%
\begin{center}
|\input{|\textit{main}|}|
\end{center}
%
A simple redirection |\childdocforward{|\textit{dest}|}| is achieved by:
%
\begin{center}
|\def\jobname{|\textit{dest}|}\input{\jobname}|
\end{center}
%
The redirection with prefix
|\childdocforwardprefix[|\textit{prefix}|]{|\textit{dest}|}|
is accomplished by:
%
\begin{center}
\begin{tabular}{l}
|{\edef\jobname{\scantokens\expandafter{\jobname\noexpand}}|\\
|\def\redirectjob |\textit{prefix}|#1~~~{\gdef\jobname{|\textit{dest}|#1}}|\\
|\expandafter\redirectjob\jobname~~~}\input{\jobname}|
\end{tabular}
\end{center}

In an alternative approach,
child documents can be compiled by a specific command line
without additional code or specific definitions:
%
\begin{center}
|... -jobname "|\textit{target}|" "|[\textit{flags}]%
|\includeonly{|\textit{dest}|}\input{|\textit{main}|}"|
\end{center}
%

%%%%%%%%%%%%%%%%%%%%%%%%%%%%%%%%%%%%%%%%%%%%%%%%%%%%%%%%%%%%%%%%%%%%%%%%%%%%%%%%
%%%%%%%%%%%%%%%%%%%%%%%%%%%%%%%%%%%%%%%%%%%%%%%%%%%%%%%%%%%%%%%%%%%%%%%%%%%%%%%%
\section{Information}

%%%%%%%%%%%%%%%%%%%%%%%%%%%%%%%%%%%%%%%%%%%%%%%%%%%%%%%%%%%%%%%%%%%%%%%%%%%%%%%%
\subsection{Copyright}

Copyright \copyright{} 2017--2018 Niklas Beisert

This work may be distributed and/or modified under the
conditions of the \LaTeX{} Project Public License, either version 1.3
of this license or (at your option) any later version.
The latest version of this license is in
  \url{http://www.latex-project.org/lppl.txt}
and version 1.3 or later is part of all distributions of \LaTeX{}
version 2005/12/01 or later.

This work has the LPPL maintenance status `maintained'.

The Current Maintainer of this work is Niklas Beisert.

This work consists of the files |README.txt|, |childdoc.ins| and |childdoc.dtx|
as well as the derived files |childdoc.def|, |cdocsamp.tex|
with |cdocsch1.tex|, |cdocsch2.tex|, |cdocspt3.tex|, |cdocspt4.tex|,
|cdocsdrf.tex|, |cdocsfn1.tex|, |cdocsfn2.tex|
as well as |childdoc.pdf|.

%%%%%%%%%%%%%%%%%%%%%%%%%%%%%%%%%%%%%%%%%%%%%%%%%%%%%%%%%%%%%%%%%%%%%%%%%%%%%%%%
\subsection{Files and Installation}

The package consists of the files:
%
\begin{center}
\begin{tabular}{ll}
    |README.txt|   & readme file \\
    |childdoc.ins| & installation file \\
    |childdoc.dtx| & source file \\
    |childdoc.def| & definition file \\
    |cdocsamp.tex| & sample main file \\
    |cdocsch1.tex| & sample include file \\
    |cdocsch2.tex| & sample include file \\
    |cdocspt3.tex| & sample part file \\
    |cdocspt4.tex| & sample part file \\
    |cdocsdrf.tex| & sample redirection file \\
    |cdocsfn1.tex| & sample redirection file \\
    |cdocsfn2.tex| & sample redirection file \\
    |childdoc.pdf| & manual
\end{tabular}
\end{center}
%
The distribution consists of the files
|README.txt|, |childdoc.ins| and |childdoc.dtx|.
%
\begin{itemize}
\item
Run (pdf)\LaTeX{} on |childdoc.dtx|
to compile the manual |childdoc.pdf| (this file).
\item
Run \LaTeX{} on |childdoc.ins| to create the definitions file |childdoc.def|
and the sample |cdocsamp.tex| with include files
|cdocsch1.tex|, |cdocsch2.tex|, |cdocspt3.tex|, |cdocspt4.tex|,
|cdocsdrf.tex|, |cdocsfn1.tex|, |cdocsfn2.tex|.
Then copy the file |childdoc.def| to an appropriate directory of your \LaTeX{}
distribution, e.g.\ \textit{texmf-root}|/tex/latex/childdoc|.
\end{itemize}

%%%%%%%%%%%%%%%%%%%%%%%%%%%%%%%%%%%%%%%%%%%%%%%%%%%%%%%%%%%%%%%%%%%%%%%%%%%%%%%%
\subsection{Related CTAN Packages}

There are several other packages which offer a similar functionality:
%
\begin{itemize}
\item
The packages
\href{http://ctan.org/pkg/docmute}{\textsf{docmute}},
\href{http://ctan.org/pkg/includex}{\textsf{includex}} and
\href{http://ctan.org/pkg/standalone}{\textsf{standalone}}
provide commands to include only the document body of
a child file thus allowing both files to be compiled individually.
\item
The packages \href{http://ctan.org/pkg/subdocs}{\textsf{subdocs}}
and \href{http://ctan.org/pkg/subfiles}{\textsf{subfiles}}
provide structures in which the main and child documents can be
encapsulated and allowing them to be compiled individually.
The inclusion mechanism is different from the conventional |\include|.
\item
The package \href{http://ctan.org/pkg/combine}{\textsf{combine}}
is an elaborate solution to combine several documents into one.
\end{itemize}
%
See also the CTAN topic \href{http://ctan.org/topic/subdocs}{\textsf{subdocs}}
for further related packages.
The present package differs from the above solutions in that
a document structure constructed with the conventional |\include| mechanism
just needs two extra commands at the top of every file
such that all constituent files can be compiled individually.

%%%%%%%%%%%%%%%%%%%%%%%%%%%%%%%%%%%%%%%%%%%%%%%%%%%%%%%%%%%%%%%%%%%%%%%%%%%%%%%%
%\subsection{Feature Suggestions}
%
%The following is a list of features which may be useful for future
%versions of this package:
%%
%\begin{itemize}
%\item
%\ldots
%\end{itemize}

%%%%%%%%%%%%%%%%%%%%%%%%%%%%%%%%%%%%%%%%%%%%%%%%%%%%%%%%%%%%%%%%%%%%%%%%%%%%%%%%
\subsection{Revision History}

%%%%%%%%%%%%%%%%%%%%%%%%%%%%%%%%%%%%%%%%
\paragraph{v2.0:} 2018/12/30

\begin{itemize}
\item
immediate forward processing
\item
added |\childdocby| mechanism
\item
manual restructured
\end{itemize}

%%%%%%%%%%%%%%%%%%%%%%%%%%%%%%%%%%%%%%%%
\paragraph{v1.6:} 2018/01/17

\begin{itemize}
\item
application for development of include files
\item
corrections to manual
\end{itemize}

%%%%%%%%%%%%%%%%%%%%%%%%%%%%%%%%%%%%%%%%
\paragraph{v1.5:} 2017/05/21

\begin{itemize}
\item
more complete structuring introduced
\item
|\childdocof| introduced
\item
|\childdoc| renamed to |\childdocmain|
\item
|\childredirect| renamed to |\childdocforward| and |\childdocforwardprefix|
and functionality expanded
\end{itemize}

%%%%%%%%%%%%%%%%%%%%%%%%%%%%%%%%%%%%%%%%
\paragraph{v1.0:} 2017/04/27

\begin{itemize}
\item
manual and install package
\item
first version published on CTAN
\end{itemize}

%%%%%%%%%%%%%%%%%%%%%%%%%%%%%%%%%%%%%%%%
\paragraph{v0.6:} 2017/04/26

\begin{itemize}
\item
redirection mechanism added
\end{itemize}

%%%%%%%%%%%%%%%%%%%%%%%%%%%%%%%%%%%%%%%%
\paragraph{v0.5:} 2017/04/26

\begin{itemize}
\item
functionality in definition file
\end{itemize}


%%%%%%%%%%%%%%%%%%%%%%%%%%%%%%%%%%%%%%%%%%%%%%%%%%%%%%%%%%%%%%%%%%%%%%%%%%%%%%%%
%%%%%%%%%%%%%%%%%%%%%%%%%%%%%%%%%%%%%%%%%%%%%%%%%%%%%%%%%%%%%%%%%%%%%%%%%%%%%%%%
%%%%%%%%%%%%%%%%%%%%%%%%%%%%%%%%%%%%%%%%%%%%%%%%%%%%%%%%%%%%%%%%%%%%%%%%%%%%%%%%
\appendix

\settowidth\MacroIndent{\rmfamily\scriptsize 000\ }

 \DocInput{childdoc.dtx}

\end{document}
%</driver>
% \fi
%
% %%%%%%%%%%%%%%%%%%%%%%%%%%%%%%%%%%%%%%%%%%%%%%%%%%%%%%%%%%%%%%%%%%%%%%%%%%%%%%
% %%%%%%%%%%%%%%%%%%%%%%%%%%%%%%%%%%%%%%%%%%%%%%%%%%%%%%%%%%%%%%%%%%%%%%%%%%%%%%
% \section{Sample}
%\iffalse
%<*samplemain>
%\fi
%
% The following presents a sample document
% with two chapters, two parts, a title page,
% a compile flag as well as three forwarding files to set the flag.
% It consists of eight |.tex| files:
% \begin{center}
% \begin{tabular}{ll}
% |cdocsamp.tex|&main file\\
% |cdocsch1.tex|&include file for chapter 1\\
% |cdocsch2.tex|&include file for chapter 2\\
% |cdocspt3.tex|&include file for part 3\\
% |cdocspt4.tex|&include file for part 4\\
% |cdocsdrf.tex|&forwarding file for main file in draft mode\\
% |cdocsfi1.tex|&forwarding file for final version of chapter 1\\
% |cdocsfi2.tex|&forwarding file for final version of chapter 2\\
% \end{tabular}
% \end{center}
% Each of the eight files can be compiled directly by the \LaTeX{} compiler.
%
% %%%%%%%%%%%%%%%%%%%%%%%%%%%%%%%%%%%%%%
% \paragraph{Main File.}
%
% The main file is called |cdocsamp.tex|.
%
% Load the \textsf{childdoc} definitions and
% declare the filename for the main document:
%    \begin{macrocode}
\input{childdoc.def}
\childdocmain{}
%    \end{macrocode}

% Optional override for |\version| flag:
%    \begin{macrocode}
%%\ifchilddoc\else\providecommand{\version}{draft}\fi
%    \end{macrocode}

% Define the default values for the |\version| flag
% (|final| for the main file and |draft| for childs):
%    \begin{macrocode}
\ifchilddoc
\providecommand{\version}{draft}
\else
\providecommand{\version}{final}
\fi
%    \end{macrocode}

% Load the standard document class:
%    \begin{macrocode}
\documentclass[12pt]{article}
%    \end{macrocode}

% Start the document body:
%    \begin{macrocode}
\begin{document}
%    \end{macrocode}

% Declare a title page.
% Print title, part of document being processed and version flag:
%    \begin{macrocode}
\addtocounter{page}{-1}
\begin{center}
{\LARGE\bfseries{}childdoc example\par}
\vspace{1cm}
\ifchilddoc
\ifchilddocmanual part\else chapter\fi:
`\childdocname' of `\childdocjob'\par
\else
main document: `\childdocjob'\par
\fi
version: \version\par
\end{center}
\newpage
%    \end{macrocode}

% Manually include selected file,
% otherwise process as usual:
%    \begin{macrocode}
\ifchilddocmanual
\section*{part `\childdocname'}
\input{\childdocname}
\else
%    \end{macrocode}

% Include the two chapters:
%    \begin{macrocode}
\include{cdocsch1}
\include{cdocsch2}
%    \end{macrocode}

% Include the two parts unless only chapters should be displayed:
%    \begin{macrocode}
\ifchilddoc\else
\section{part three}
\input{cdocspt3}
\section{part four}
\input{cdocspt4}
\fi
%    \end{macrocode}

% Process as usual until here:
%    \begin{macrocode}
\fi
%    \end{macrocode}

% End of document body:
%    \begin{macrocode}
\end{document}
%    \end{macrocode}
%\iffalse
%</samplemain>
%\fi
%
% %%%%%%%%%%%%%%%%%%%%%%%%%%%%%%%%%%%%%%
% \paragraph{Chapter Include Files.}
%
% The include files are called |cdocsch1.tex| and |cdocsch2.tex|.
%
%\iffalse
%<*samplechap1|samplechap2>
%\fi

% Optional override for |\version| flag:
%    \begin{macrocode}
%%\providecommand{\version}{final}
%    \end{macrocode}

% Include the main document:
%    \begin{macrocode}
\input{childdoc.def}
\childdocof{cdocsamp}
%    \end{macrocode}

%\iffalse
%</samplechap1|samplechap2>
%\fi
%
%\iffalse
%<*samplechap1>
%\fi
% Some text for chapter 1:
%    \begin{macrocode}
\section{one}
some text in chapter one
%    \end{macrocode}

%\iffalse
%</samplechap1>
%\fi
% Some text for chapter 2:
%\iffalse
%<*samplechap2>
%\fi
%    \begin{macrocode}
\section{two}
more text in chapter two
%    \end{macrocode}

%\iffalse
%</samplechap2>
%\fi
%
% %%%%%%%%%%%%%%%%%%%%%%%%%%%%%%%%%%%%%%
% \paragraph{Part Include Files.}
%
% The include files are called |cdocspt3.tex| and |cdocspt4.tex|.
%
%\iffalse
%<*samplepart3|samplepart4>
%\fi

% Optional override for |\version| flag:
%    \begin{macrocode}
%%\providecommand{\version}{final}
%    \end{macrocode}

% Include the main document:
%    \begin{macrocode}
\input{childdoc.def}
\childdocby{cdocsamp}
%    \end{macrocode}

%\iffalse
%</samplepart3|samplepart4>
%\fi
%
%\iffalse
%<*samplepart3>
%\fi
% Some text for part 3:
%    \begin{macrocode}
some text in part three
%    \end{macrocode}

%\iffalse
%</samplepart3>
%\fi
% Some text for part 4:
%\iffalse
%<*samplepart4>
%\fi
%    \begin{macrocode}
more text in part four
%    \end{macrocode}

%\iffalse
%</samplepart4>
%\fi
%
% %%%%%%%%%%%%%%%%%%%%%%%%%%%%%%%%%%%%%%
% \paragraph{Forwarding for a Complete Draft.}
%
% The following forwarding file |cdocsdrf.tex|
% compiles the main document in draft mode:
%\iffalse
%<*sampledraft>
%\fi
%    \begin{macrocode}
\def\version{draft}
\input{childdoc.def}
\childdocforward{cdocsamp}
%    \end{macrocode}

%\iffalse
%</sampledraft>
%\fi
%
% %%%%%%%%%%%%%%%%%%%%%%%%%%%%%%%%%%%%%%
% \paragraph{Forwarding for Final Version of the Chapters.}
%
% The following forwarding files |cdocsfn1.tex| and |cdocsfn2.tex|
% (with identical content)
% compile the final versions of the child documents
% |cdocsch1.tex| and |cdocsch2.tex|, respectively:
%\iffalse
%<*samplefinal>
%\fi
%    \begin{macrocode}
\def\version{final}
\input{childdoc.def}
\childdocforwardprefix[cdocsamp]{cdocsfn}{cdocsch}
%    \end{macrocode}

%\iffalse
%</samplefinal>
%\fi
%
% %%%%%%%%%%%%%%%%%%%%%%%%%%%%%%%%%%%%%%
% \paragraph{Command Line Processing.}
%
% The following three command lines generate the output files
% |cdocscld|, |cdocscl1| and |cdocscl2|
% which should be identical to
% |cdocsdrf|, |cdocsch1| and |cdocsfn2|, respectively:
% \begin{center}
% \begin{tabular}{l}
% |latex -jobname cdocscld \|\\
% |  "\def\version{draft}\input{childdoc.def}\childdocforward{cdocsamp}"|\\
% |latex -jobname cdocscl1 \|\\
% |  "\input{childdoc.def}\childdocforward[cdocsamp]{cdocsch1}"|\\
% |latex -jobname cdocscl2 \|\\
% |  "\def\version{final}\input{childdoc.def}\childdocforward{cdocsch2}"|
% \end{tabular}
% \end{center}
% Note that the trailing backslash on each first line
% merely continues the input to the second line
% (for convenient cut ant paste).
% Furthermore, the command |latex| can be replaced by any
% of its alternative versions such as |pdflatex|.
%
% %%%%%%%%%%%%%%%%%%%%%%%%%%%%%%%%%%%%%%%%%%%%%%%%%%%%%%%%%%%%%%%%%%%%%%%%%%%%%%
% %%%%%%%%%%%%%%%%%%%%%%%%%%%%%%%%%%%%%%%%%%%%%%%%%%%%%%%%%%%%%%%%%%%%%%%%%%%%%%
% \section{Implementation}
%\iffalse
%<*package>
%\fi
%
% This section describes the definitions file |childdoc.def|.

% The definitions cannot be loaded using |\usepackage| or |\RequirePackage|
% which has a mechanism to prevent loading a style file more than once.
% When loading the definitions by means of |\input|
% multiple instances have to be prevented manually:
%\iffalse
%This code needs to be before the `\ProvidesFile' directive
%which is defined at the beginning of this file.
%Therefore it is also placed there and commented out here.
%</package>
%<*discard>
%\fi
%    \begin{macrocode}
\ifdefined\childdocmain\endinput\fi
%    \end{macrocode}
%\iffalse
%</discard>
%<*package>
%\fi
%
% \macro{\ifchilddoc}
% \macro{\ifchilddocmanual}
% The conditional |\ifchilddoc| tells whether a
% child (true) or main (false) document is being compiled.
% The conditional |\ifchilddocmanual| tells whether
% the |\includeonly| mechanism is used (false) or
% the selection of child files must be performed manually (true).
% The definitions initialise to false:
%    \begin{macrocode}
\newif\ifchilddoc
\newif\ifchilddocmanual
%    \end{macrocode}

% \macro{\childdocname}
% \macro{\childdocjob}
% The macro |\childdocname| stores the name of the main document
% to be compiled. The macro |\childdocjob| stores the name of
% the document on which the \LaTeX{} compiler was originally invoked.
% The content of |\jobname| cannot be compared
% to filenames specified in the source due to different catcodes.
% The following code rescans |\jobname|, stores the result
% in |\childdocname| and saves a copy in |\childdocjob|:
%    \begin{macrocode}
\edef\childdocname{\scantokens\expandafter{\jobname\noexpand}}
\let\childdocjob\childdocname
%    \end{macrocode}

% \macro{\childdocdisable}
% The macro |\childdocdisable| prevents the main file
% from being processed more than once.
% At this stage, the main document command |\childdocmain|
% is assumed to be called once again where it should do nothing.
% Any subsequent call to it should prevent
% a secondary processing of the main document
% It overwrites the forwarding commands
% |\childdocof| and |\childdocforward|
% with empty macros to prevent further inclusions of the main document:
%    \begin{macrocode}
\newcommand{\childdocdisable}
{
  \renewcommand{\childdocmain}[1]{\renewcommand{\childdocmain}[1]{\endinput}}
  \renewcommand{\childdocof}[1]{}
  \renewcommand{\childdocby}[2][]{}
  \renewcommand{\childdocforward}[2][]{}
  \renewcommand{\childdocdisable}{}
}
%    \end{macrocode}

% \macro{\childdocmain}
% The macro |\childdocmain| is to be called at the top of the main file
% with nothing or the main filename (without extension) as argument.
% First, it breaks loops.
% If the argument is not empty and does not match |\childdocname|
% (which is set by the first inclusion of |childdoc.def|),
% |\ifchilddoc| is set to true, |\includeonly| is applied to the child file
% and |\jobname| is set to the main file
% (for proper handling of |.aux| files):
%    \begin{macrocode}
\newcommand{\childdocmain}[1]
{
  \childdocdisable\childdocmain{}
  \if?#1?\else
    \begingroup
      \def\childdoctmp{#1}
      \ifx\childdoctmp\childdocname
        \def\childdoctmp{}
      \else
        \def\childdoctmp
        {
          \childdoctrue
          \includeonly{\childdocname}
          \def\childdocjob{#1}
          \def\jobname{#1}
        }
      \fi
      \expandafter
    \endgroup
    \childdoctmp
  \fi
}
%    \end{macrocode}

% \macro{\childdocof}
% The command |\childdocof| redirects
% compilation to the main file |#1|.
%    \begin{macrocode}
\newcommand{\childdocof}[1]
{
  \childdocdisable
  \childdoctrue
  \includeonly{\childdocname}
  \def\jobname{#1}
  \def\childdocjob{#1}
  \input{#1}
}
%    \end{macrocode}

% \macro{\childdocby}
% The command |\childdocby| ....
%    \begin{macrocode}
\newcommand{\childdocby}[2][]
{
  \childdocdisable
  \childdoctrue
  \childdocmanualtrue
  \if?#1?\else
    \def\jobname{#2}
  \fi
  \def\childdocjob{#2}
  \input{#2}
  \endinput
}
%    \end{macrocode}

% \macro{\childdocforward}
% The command |\childdocforward| redirects
% compilation to the main file or
% (if the optional argument is given) a child file.
% Parameters are set as if the main file
% or a child file starting with |\childdocof| was compiled.
% Then compilation is handed over to the main file:
%    \begin{macrocode}
\newcommand{\childdocforward}[2][]
{
  \begingroup
    \if?#1?
      \def\childdoctmp
      {
        \def\childdocname{#2}
        \def\childdocjob{#2}
        \def\jobname{#2}
        \input{#2}
        \endinput
      }
    \else
      \def\childdoctmp
      {
        \childdocdisable
        \def\childdocname{#2}
        \childdoctrue
        \includeonly{#2}
        \def\childdocjob{#1}
        \def\jobname{#1}
        \input{#1}
        \endinput
      }
    \fi
    \expandafter
  \endgroup
  \childdoctmp
}
%    \end{macrocode}

% \macro{\childdocforwardprefix}
% The command |\childdocforwardprefix| redirects
% compilation to the main or a child file by means of a pattern.
% The prefix |#1| in the current filename is replaced by |#2|
% and the suffix of the current filename is kept
% (it is assumed that the filename does not contain the substring `|~~~|'
% which is used as a delimiter).
% Compilation is handed over to the new file by |\childdocforward|:
%    \begin{macrocode}
\newcommand{\childdocforwardprefix}[3][]
{
  \begingroup
    \def\childdocextract #2##1~~~{\def\childdoctmp{\childdocforward[#1]{#3##1}}}
    \expandafter\childdocextract\childdocname~~~
    \expandafter
  \endgroup
  \childdoctmp
}
%    \end{macrocode}

% \macro{\childdoc}
% The deprecated macro |\childdoc| is a legacy version of |\childdocmain|:
%    \begin{macrocode}
\newcommand{\childdoc}{\childdocmain}
%    \end{macrocode}

% \macro{\childdocredirect}
% The deprecated macro |\childdocredirect| is a legacy version
% of |\childdocforward| and |\childdocforwardprefix|:
%    \begin{macrocode}
\newcommand{\childdocredirect}[2][]
{
  \begingroup
    \if?#1?
      \def\childdoctmp{\childdocforward{#2}}
    \else
      \def\childdoctmp{\childdocforwardprefix{#1}{#2}}
    \fi
    \expandafter
  \endgroup
  \childdoctmp
}
%    \end{macrocode}

%\iffalse
%</package>
%\fi
%
\endinput
|\\
|\childdocforward{|\textit{main}|}|
\end{tabular}
\end{center}
%
Likewise, the following files |final|\textit{nn}|.tex|
compile the final version of the child document
|child|\textit{nn}|.tex|:
%
\begin{center}
\begin{tabular}{l}
|\def\version{final}|\\
|% \iffalse
%
% childdoc.dtx Copyright (C) 2017-2018 Niklas Beisert
%
% This work may be distributed and/or modified under the
% conditions of the LaTeX Project Public License, either version 1.3
% of this license or (at your option) any later version.
% The latest version of this license is in
%   http://www.latex-project.org/lppl.txt
% and version 1.3 or later is part of all distributions of LaTeX
% version 2005/12/01 or later.
%
% This work has the LPPL maintenance status `maintained'.
%
% The Current Maintainer of this work is Niklas Beisert.
%
% This work consists of the files childdoc.dtx and childdoc.ins
% and the derived files childdoc.def and cdocsamp.tex with
% cdocsch1.tex, cdocsch2.tex, cdocsdrf.tex, cdocsfn1.tex, cdocsfn2.tex.
%
%<package>\ifdefined\childdocmain\endinput\fi
%<package>\ProvidesFile{childdoc.def}[2018/12/30 v2.0 child document driver]
%<samplemain>\ProvidesFile{cdocsamp.tex}[2018/12/30 v2.0 sample for childdoc]
%<*driver>
%\ProvidesFile{childdoc.drv}[2018/12/30 v2.0 childdoc reference manual file]
\PassOptionsToClass{10pt,a4paper}{article}
\documentclass{ltxdoc}

\usepackage[margin=35mm]{geometry}
\usepackage{hyperref}
\usepackage{hyperxmp}
\usepackage[usenames]{color}

\hypersetup{colorlinks=true}
\hypersetup{pdfstartview=FitH}
\hypersetup{pdfpagemode=UseNone}
\hypersetup{pdfsource={}}
\hypersetup{pdflang={en-UK}}
\hypersetup{pdfcopyright={Copyright 2017-2018 Niklas Beisert.
  This work may be distributed and/or modified under the
  conditions of the LaTeX Project Public License, either version 1.3
  of this license or (at your option) any later version.}}
\hypersetup{pdflicenseurl={http://www.latex-project.org/lppl.txt}}
\hypersetup{pdfcontactaddress={ETH Zurich, ITP, HIT K,
  Wolfgang-Pauli-Strasse 27}}
\hypersetup{pdfcontactpostcode={8093}}
\hypersetup{pdfcontactcity={Zurich}}
\hypersetup{pdfcontactcountry={Switzerland}}
\hypersetup{pdfcontactemail={nbeisert@itp.phys.ethz.ch}}
\hypersetup{pdfcontacturl={http://people.phys.ethz.ch/\xmptilde nbeisert/}}

\newcommand{\secref}[1]{\hyperref[#1]{section \ref*{#1}}}

\parskip1ex
\parindent0pt
\let\olditemize\itemize
\def\itemize{\olditemize\parskip0pt}

\begin{document}

\title{The \textsf{childdoc} Package}
\hypersetup{pdftitle={The childdoc Package}}
\author{Niklas Beisert\\[2ex]
  Institut f\"ur Theoretische Physik\\
  Eidgen\"ossische Technische Hochschule Z\"urich\\
  Wolfgang-Pauli-Strasse 27, 8093 Z\"urich, Switzerland\\[1ex]
  \href{mailto:nbeisert@itp.phys.ethz.ch}
  {\texttt{nbeisert@itp.phys.ethz.ch}}}
\hypersetup{pdfauthor={Niklas Beisert}}
\hypersetup{pdfsubject={Manual for the LaTeX2e Package childdoc}}
\date{30 December 2018, \textsf{v2.0}}
\maketitle

\begin{abstract}\noindent
\textsf{childdoc} is a \LaTeXe{} package
that enables the direct compilation
of document sections included by |\include|
to individual files.
\end{abstract}

\begingroup
\parskip0ex
\tableofcontents
\endgroup

%%%%%%%%%%%%%%%%%%%%%%%%%%%%%%%%%%%%%%%%%%%%%%%%%%%%%%%%%%%%%%%%%%%%%%%%%%%%%%%%
%%%%%%%%%%%%%%%%%%%%%%%%%%%%%%%%%%%%%%%%%%%%%%%%%%%%%%%%%%%%%%%%%%%%%%%%%%%%%%%%
\section{Introduction}

\LaTeX{} provides a mechanism to structure a large document (such as a book)
into a main file and several child files (containing the chapters)
using the |\include| command.
This mechanism is beneficial for documents
which span hundreds of pages in order to
make the source file(s) more manageable.
Moreover, compilation can be restricted to
selected child files by means of the |\includeonly| command.
The latter feature can be used to reduce the compilation time while editing
(this was significantly more useful in the earlier days of \LaTeX{})
or to generate a smaller document which is easier to navigate.
Another application of |\includeonly| is to generate
documents consisting of selected parts of the complete document.

However, there are a few drawbacks of the plain |\include| mechanism:
\begin{itemize}
\item
The child files cannot be compiled on their own,
they can only be compiled via the main file.
A naive editing environment
(such as a text editor with an option
to have the current file processed by \LaTeX)
may require one to switch to the main file before compiling;
attempting to compile the child file produces errors.
\item
The main file must be modified (each time)
to adjust the |\includeonly| command
to the present needs. This easily leaves the main file in a messy state.
\item
The generated document will always carry the filename
of the main document. This is inconvenient if
several child files are to be compiled and
to be kept for distribution.
\end{itemize}

The present package provides a simple interface
to make child files individually compilable by \LaTeX{}.
Compiling a child file then has the same effect as compiling
the main file with an |\includeonly| command
to select the appropriate child.
Moreover the generated document will carry the name of the child
rather than the main file.
This resolves all three above issues.

This feature is meant to make the editing of books,
thesis documents and lecture notes somewhat more convenient.
However, the package can also be used efficiently for
composing a series of documents (such as exercise sheets)
which are typically distributed individually.
It then assists the author in generating the individual documents
(potentially in different versions)
as well as a document containing the collected series.
Another application is in developing style files
or other kinds of included material
where compilation of the style file could redirect
to a sample or test file.

%%%%%%%%%%%%%%%%%%%%%%%%%%%%%%%%%%%%%%%%%%%%%%%%%%%%%%%%%%%%%%%%%%%%%%%%%%%%%%%%
%%%%%%%%%%%%%%%%%%%%%%%%%%%%%%%%%%%%%%%%%%%%%%%%%%%%%%%%%%%%%%%%%%%%%%%%%%%%%%%%
\section{Usage}

First of all, the package \textsf{childdoc} is \emph{not} a standard
\LaTeXe{} |.sty| style file! Therefore it needs to be invoked in
a non-standard way.

%%%%%%%%%%%%%%%%%%%%%%%%%%%%%%%%%%%%%%%%%%%%%%%%%%%%%%%%%%%%%%%%%%%%%%%%%%%%%%%%
\subsection{Included Files}
\label{sec:include}

%%%%%%%%%%%%%%%%%%%%%%%%%%%%%%%%%%%%%%%%
\DescribeMacro{\childdocmain}
To use the package, add the commands
\begin{center}
\begin{tabular}{l}
|\input{childdoc.def}|\\
|\childdocmain{}|\\
\end{tabular}
\end{center}
at the very top of the main \LaTeX{} file,
in particular \emph{before} the |\documentclass| statement!
The argument of |\childdocmain| should be left empty
(but it must be present).

%%%%%%%%%%%%%%%%%%%%%%%%%%%%%%%%%%%%%%%%
\DescribeMacro{\childdocof}
Furthermore, add the commands
\begin{center}
\begin{tabular}{l}
|\input{childdoc.def}|\\
|\childdocof{|\textit{main}|}|\\
\end{tabular}
\end{center}
at the top of every child file \textit{child}
which is included by |\include{|\textit{child}|}|
from within the main file
(or at least for those files to be compiled individually).
The argument \textit{main} must be the filename of the main file.

There are a couple of
considerations in setting up the main and child documents:

%%%%%%%%%%%%%%%%%%%%%%%%%%%%%%%%%%%%%%%%
\paragraph{Restrictions.}

Please note the following restrictions:
\begin{itemize}
\item
|\childdocmain| must be called with one argument \textit{main}
to ensure compatibility with earlier version of the package.
It must either be empty (|\childdocmain{}|)
or precisely match the filename of the main file in which it is specified.
See \secref{sec:detection} for further information.
\item
The filename \textit{main} must be specified without the |.tex| extension.
\item
The filename \textit{main} is case sensitive
(even in case-insensitive file systems)
due to internal string comparison.
\item
The argument \textit{main} should be fully expanded, it cannot be a macro.
\item
Subdirectories and special characters should be avoided in filenames.
\item
The command |\childdocmain{|\textit{main}|}| must be followed by a whitespace.
It should not be followed immediately by another command
or by a comment mark `|%|'.
This is because the \TeX{} parser reads the token immediately following
the argument of |\childdocmain| and puts it
at the beginning of every child section;
however, a white\-space is ignored.
\end{itemize}

%%%%%%%%%%%%%%%%%%%%%%%%%%%%%%%%%%%%%%%%
\paragraph{Content of Main File.}

It is advisable to place all content in the child files included by |\include|.
Any output contained in the main file will appear in all child documents
unless suppressed manually;
it cannot be suppressed automatically by the |\includeonly| directive
and thus should normally be avoided.
A method to include some content in the main file
by means of conditional processing is described in \secref{sec:conditional}.

%%%%%%%%%%%%%%%%%%%%%%%%%%%%%%%%%%%%%%%%
\paragraph{Page Numbering.}

When only a part of the document is compiled,
the appropriate numbering of pages
(as well as other status parameters)
is determined from the |.aux| files.
The latter contain information from previous passes.
However this information needs to propagate through
all intermediate child documents.
Therefore the page numbering in child documents may well
be inconsistent until the complete document is compiled at least once.

A useful (if unconventional) way to always ensure a consistent
page numbering is to restart the numbering in each child document
and denote the pages by `\textit{child}|.|\textit{page}'
where \textit{child} represents the chapter/section number of the child file.
This can be achieved by the command
|\numberwithin{page}{|\textit{child}|}|
of the \textsf{amsmath} package
where \textit{child} can be |chapter| or |section|
depending on the chosen structuring.
Alternatively, one can modify the macro |\thepage| appropriately
and reset the counter |page| at the start of each child file.

%%%%%%%%%%%%%%%%%%%%%%%%%%%%%%%%%%%%%%%%%%%%%%%%%%%%%%%%%%%%%%%%%%%%%%%%%%%%%%%%
\subsection{Conditional Processing}
\label{sec:conditional}

The package provides a mechanism to compile different versions
of a document. To customise the versions further some conditional processing
can come in handy to distinguish which version is being compiled.
The package provides two macros to describe the compilation context:

%%%%%%%%%%%%%%%%%%%%%%%%%%%%%%%%%%%%%%%%
\DescribeMacro{\ifchilddoc}
The conditional |\ifchilddoc| distinguishes between the compilation of
child documents and the main document:
%
\begin{center}
|\ifchilddoc |\textit{child-code}| |[|\||else |\textit{main-code}]| \||fi|
\end{center}

%%%%%%%%%%%%%%%%%%%%%%%%%%%%%%%%%%%%%%%%
\DescribeMacro{\childdocname}
\DescribeMacro{\childdocjob}
The macro |\childdocname| contains the filename (without extension)
of the main or child file being processed.
Note that |\childdocjob| will always contain the name of the main file.

%%%%%%%%%%%%%%%%%%%%%%%%%%%%%%%%%%%%%%%%
\paragraph{Title Page.}

Conditional processing can be used to include a title or banner page
in the main document when proper precautions are taken.
Importantly, the code in the main file should ensure that the page counter
(as well as other status parameters which are stored in the |.aux| files)
takes the same value after the conditional processing.
Otherwise the page numbers may take divergent values
depending on which part is compiled.

For example, a title page could be declared by:
%
\begin{center}
\begin{tabular}{l}
|\ifchilddoc\||else|\\
|\addtocounter{page}{-1}|\\
\textit{code for title page}\\
|\newpage|\\
|\||fi|
\end{tabular}
\end{center}
%
A banner page for the child documents can be generated by:
%
\begin{center}
\begin{tabular}{l}
|\ifchilddoc|\\
|\addtocounter{page}{-1}|\\
\textit{code for banner page}\\
|\newpage|\\
|\||fi|
\end{tabular}
\end{center}
%
Here one could write a message such as:
\begin{center}
|This is the part \childdocname{} of \childdocjob{}.|
\end{center}

%%%%%%%%%%%%%%%%%%%%%%%%%%%%%%%%%%%%%%%%%%%%%%%%%%%%%%%%%%%%%%%%%%%%%%%%%%%%%%%%
\subsection{Flags}
\label{sec:flags}

The package makes it easy to generate different versions
of the main or child documents.
To this end compilation flags can be defined
and assigned different default values.
They will be particularly useful in conjunction
with the forwarding mechanism described in \secref{sec:forward}.

For example, it may be useful to have a flag |\version|
which can be set to |draft| or |final|.
The document source will contain some conditional code
depending on the value of |\version|.
Suppose further, the flag should default to |final| for the main file
and to |draft| for child files
which is a natural assignment for editing the document.
This is achieved by placing the following code
in the preamble of the main document
(below the |\childdocmain| directive):
%
\begin{center}
\begin{tabular}{l}
|\ifchilddoc|\\
|\providecommand{\version}{draft}|\\
|\||else|\\
|\providecommand{\version}{final}|\\
|\||fi|
\end{tabular}
\end{center}
%
The definition by |\providecommand| makes sure
that previous definitions are not overwritten.
Further statements |\providecommand{\version}{...}|
can thus be added before the above code to override it.

For the main file, one might add a line
(between |\childdocmain| and the above block)
%
\begin{center}
|%\ifchilddoc\||else\providecommand{\version}{draft}\||fi|
\end{center}
%
which can be uncommented to produce a draft version.
Likewise one can add a line to the very top of a child file
(above the |\childdocof{|\textit{main}|}| directive)
%
\begin{center}
|%\providecommand{\version}{final}|
\end{center}
%
which can be uncommented to produce the final version of this child document.

%%%%%%%%%%%%%%%%%%%%%%%%%%%%%%%%%%%%%%%%%%%%%%%%%%%%%%%%%%%%%%%%%%%%%%%%%%%%%%%%
\subsection{Forwarding}
\label{sec:forward}

Different versions of the main or child documents
using compilation flags as described in \secref{sec:flags}
can be (permanently) stored in different files
for convenient compilation, viewing and distribution.
To this end, the package defines a command
to pass on compilation to a different file:

%%%%%%%%%%%%%%%%%%%%%%%%%%%%%%%%%%%%%%%%
\DescribeMacro{\childdocforward}
The command |\childdocforward| redirects processing to
another source file:
%
\begin{center}
\begin{tabular}{l}
|\input{childdoc.def}|\\
|\childdocforward[|\textit{main}|]{|\textit{dest}|}|\\
\end{tabular}
\end{center}
%
The argument \textit{dest} is the destination file
(without extension).
It should be the main file or one of the child files.
Note that further \textsf{childdoc} directives
such as |\childdocof| and |\childdocforward|
in the indicated file will be processed in this form.
The optional argument \textit{main}
passes on directly to the main file \textit{main}
while pretending to compile the child \textit{dest}.
This form behaves as if \textit{dest}
issues |\childdocof{|\textit{main}|}| right away,
and no further \textsf{childdoc} directives will be processed.

%%%%%%%%%%%%%%%%%%%%%%%%%%%%%%%%%%%%%%%%
\DescribeMacro{\...prefix}
In the alternative form |\childdocforwardprefix|,
%
\begin{center}
\begin{tabular}{l}
|\input{childdoc.def}|\\
|\childdocforwardprefix[|\textit{main}|]{|\textit{prefix}|}{|\textit{dest}|}|
\end{tabular}
\end{center}
%
the destination file is determined by a pattern
depending on the current file:
To make this work, the current file must be called
`{\textit{prefix}\hspace{0.2em}\textit{suffix}}'
with \textit{prefix} matching precisely the argument.
Processing is then passed on to the file
`{\textit{dest}\hspace{0.2em}\textit{suffix}}'.
Surely, the same effect is achieved by
directly specifying the
argument `{\textit{dest}\hspace{0.2em}\textit{suffix}}'
in the first form.
However, that requires to set up a different file
for each child. With the alternative form of the command
all these files can have exactly the same content
which simplifies setting them up and maintaining them.

For example, the following file |draft.tex|
with a compilation flag |\version| as described in \secref{sec:flags}
compiles the main document as a draft:
%
\begin{center}
\begin{tabular}{l}
|\def\version{draft}|\\
|\input{childdoc.def}|\\
|\childdocforward{|\textit{main}|}|
\end{tabular}
\end{center}
%
Likewise, the following files |final|\textit{nn}|.tex|
compile the final version of the child document
|child|\textit{nn}|.tex|:
%
\begin{center}
\begin{tabular}{l}
|\def\version{final}|\\
|\input{childdoc.def}|\\
|\childdocforwardprefix{final}{child}|
\end{tabular}
\end{center}
%

Note that when several versions of a main file and/or of each child file
are to be generated, it may be convenient to set up a |Makefile| or
shell script to automatise the process.

%%%%%%%%%%%%%%%%%%%%%%%%%%%%%%%%%%%%%%%%%%%%%%%%%%%%%%%%%%%%%%%%%%%%%%%%%%%%%%%%
\subsection{Command Line Processing}
\label{sec:commandline}

The effect of redirection files can also be achieved by invoking
the \LaTeX{} compiler with a more elaborate command line.
Most conveniently this should be done as part
of a shell script or a |Makefile|.

When using \textsf{childdoc} in the main file, the following
command lines effectively perform a redirection
(note that depending on the shell being used,
backslashes may have to be doubled: `|\|' $\to$ `|\\|'):
%
\begin{center}
|... -jobname "|\textit{target}|" |\\|"|[\textit{flags}]%
|\input{childdoc.def}\childdocforward[|\textit{main}|]{|\textit{dest}|}"|
\end{center}
%
Here \textit{target} is the name of the output file,
\textit{main} is the name of the main file
and \textit{dest} is the name of the main or child file to be processed
(all filenames without extensions).
The optional argument \textit{main} can be omitted
if \textit{main} matches \textit{dest}.
Optionally, compilation \textit{flags} can be defined via |\def| commands.
This command line makes the \TeX{} engine believe
it is compiling the file \textit{target}
whose content is specified as the latter parameter.
The provided code then forwards the processing to
\textit{main} or \textit{dest} as described in \secref{sec:forward}.

%%%%%%%%%%%%%%%%%%%%%%%%%%%%%%%%%%%%%%%%%%%%%%%%%%%%%%%%%%%%%%%%%%%%%%%%%%%%%%%%
\subsection{Include by Input}
\label{sec:input}

Including child documents by |\include| has some restrictions by design.
Most notably, the content of a child document always occupies
its own set of pages; pages cannot be shared between child documents.
Usually, this behaviour makes perfect sense
because each child document contain an essential part of the document.
However, in some situations it may be desirable to compose
a document from a collection of parts
without having mandatory page breaks between then.
For this case, the package
provides a mechanism to include parts
by |\input| which can also be processed individually.
However, by construction this mechanism
requires manual handling of the content to be output.

%%%%%%%%%%%%%%%%%%%%%%%%%%%%%%%%%%%%%%%%
\DescribeMacro{\ifchilddocmanual}
The main file should be prepared as usual, see \secref{sec:include}.
However, the document body must make a distinction
between processing of an individual part and of the main document, e.g.:
%
\begin{center}
\begin{tabular}{l}
|\ifchilddocmanual|\\
|\input{\childdocname}|\\
|\||else|\\
\textit{document body with }|\input{|\textit{part}|}|\\
|\||fi|
\end{tabular}
\end{center}
%
The conditional |\ifchilddocmanual| is true whenever
a part to be included by |\input| is being compiled,
and the name of the part is stored in |\childdocname|.

%%%%%%%%%%%%%%%%%%%%%%%%%%%%%%%%%%%%%%%%
\DescribeMacro{\childdocby}
Each part to be included by |\input| should start with:
%
\begin{center}
\begin{tabular}{l}
|\input{childdoc.def}|\\
|\childdocby{|\textit{main}|}|\\
\end{tabular}
\end{center}
%
The directive |\childdocby| is similar to |\childdocof|
described in \secref{sec:include},
but the subsequent selection of content must be done manually.
To that end, both |\ifchilddoc| and |\ifchilddocmanual|
will be true upon processing of a part,
and the name of the part is stored in |\childdocname|.
Note that |\jobname| will be set to the filename of the current part
so that each part receives an individual |.aux| file
that does not interfere with the |.aux| file(s) of the main document.
This behaviour can be altered by the alternative form
|\childdocby[*]{|\textit{main}|}| (with a non-empty optional argument)
which uses the |.aux| file of the main document
by setting |\jobname| to \textit{main}.

%%%%%%%%%%%%%%%%%%%%%%%%%%%%%%%%%%%%%%%%%%%%%%%%%%%%%%%%%%%%%%%%%%%%%%%%%%%%%%%%
\subsection{Driver Development}
\label{sec:driver}

The \textsf{childdoc} mechanism can also be use for the development
of definition files such as \LaTeX{} styles or classes.
This case differs from the above setup with multiple parts
included by |\include| in that no |\includeonly| should be invoked.
This can be achieved by starting the include file
(before |\ProvidesPackage|) with:
%
\begin{center}
\begin{tabular}{l}
|\input{childdoc.def}|\\
|\childdocforward{|\textit{main}|}|\\
\end{tabular}
\end{center}
%
or alternatively with:
%
\begin{center}
\begin{tabular}{l}
|\input{childdoc.def}|\\
|\childdocby{|\textit{main}|}|\\
\end{tabular}
\end{center}
%
Both forms have slightly different effects as described above.
The main file is prepared as usual, see \secref{sec:include}.

%%%%%%%%%%%%%%%%%%%%%%%%%%%%%%%%%%%%%%%%%%%%%%%%%%%%%%%%%%%%%%%%%%%%%%%%%%%%%%%%
\subsection{Legacy Detection}
\label{sec:detection}

The directive |\childdocmain| in the main file can detect
whether the complete document or merely a child is to be compiled
even without using the directive |\childdocof|.
This method is deprecated because it is less robust
and there is no compelling reason to use it;
it is merely provided for backward compatibility
and it may be removed in future versions.

If the detection mechanism is to be used,
it is mandatory to correctly specify
the filename of the main file as the argument of |\childdocmain|:
%
\begin{center}
\begin{tabular}{l}
|\input{childdoc.def}|\\
|\childdocmain{|\textit{main}|}|\\
\end{tabular}
\end{center}
%
If |\jobname| does not match the argument \textit{main} of |\childdocmain|,
it is assumed that |\jobname| points to the child file to be compiled.
When using |\childdocmain| with the main file specified as argument,
it suffices to start a child file
with just |\input{|\textit{main}|}|
without loading of the package and using |\childdocof|.
If instead all processing is done
with the appropriate \textsf{childdoc} directives,
the argument of \textit{main} of |\childdocmain| can be empty.

An alternative version of the command line processing described
in \secref{sec:commandline} using the detection mechanism reads:
%
\begin{center}
|... -jobname "|\textit{target}|" "|[\textit{flags}]%
[|\def\jobname{|\textit{dest}|}|]|\input{|\textit{main}|}"|
\end{center}

%%%%%%%%%%%%%%%%%%%%%%%%%%%%%%%%%%%%%%%%%%%%%%%%%%%%%%%%%%%%%%%%%%%%%%%%%%%%%%%%
\subsection{Manual Code}
\label{sec:manual}

In case one cannot be certain whether the definitions file |childdoc.def|
is installed on the target \TeX{} distribution
and one prefers not to ship it,
it is conceivable to paste a few relevant commands into the sources.

To that end, drop all statements |\input{childdoc.def}|
and perform the replacements as outlined below.
Instead of |\childdocmain{|\textit{main}|}| add the following code
to the top of the main file:
%
\begin{center}
\begin{tabular}{l}
|\||ifdefined\childdocname\endinput\||fi\newif\ifchilddoc|\\
|\edef\childdocname{\scantokens\expandafter{\jobname\noexpand}}|\\
|\def\childdocmain{|\textit{main}|}\||ifx\childdocmain\childdocname\||else|\\
|\childdoctrue\includeonly{\childdocname}\let\jobname\childdocmain\||fi|\\
\end{tabular}
\end{center}
%
Instead of |\childdocof{|\textit{main}|}| just include the main file
at the top of each child file:
%
\begin{center}
|\input{|\textit{main}|}|
\end{center}
%
A simple redirection |\childdocforward{|\textit{dest}|}| is achieved by:
%
\begin{center}
|\def\jobname{|\textit{dest}|}\input{\jobname}|
\end{center}
%
The redirection with prefix
|\childdocforwardprefix[|\textit{prefix}|]{|\textit{dest}|}|
is accomplished by:
%
\begin{center}
\begin{tabular}{l}
|{\edef\jobname{\scantokens\expandafter{\jobname\noexpand}}|\\
|\def\redirectjob |\textit{prefix}|#1~~~{\gdef\jobname{|\textit{dest}|#1}}|\\
|\expandafter\redirectjob\jobname~~~}\input{\jobname}|
\end{tabular}
\end{center}

In an alternative approach,
child documents can be compiled by a specific command line
without additional code or specific definitions:
%
\begin{center}
|... -jobname "|\textit{target}|" "|[\textit{flags}]%
|\includeonly{|\textit{dest}|}\input{|\textit{main}|}"|
\end{center}
%

%%%%%%%%%%%%%%%%%%%%%%%%%%%%%%%%%%%%%%%%%%%%%%%%%%%%%%%%%%%%%%%%%%%%%%%%%%%%%%%%
%%%%%%%%%%%%%%%%%%%%%%%%%%%%%%%%%%%%%%%%%%%%%%%%%%%%%%%%%%%%%%%%%%%%%%%%%%%%%%%%
\section{Information}

%%%%%%%%%%%%%%%%%%%%%%%%%%%%%%%%%%%%%%%%%%%%%%%%%%%%%%%%%%%%%%%%%%%%%%%%%%%%%%%%
\subsection{Copyright}

Copyright \copyright{} 2017--2018 Niklas Beisert

This work may be distributed and/or modified under the
conditions of the \LaTeX{} Project Public License, either version 1.3
of this license or (at your option) any later version.
The latest version of this license is in
  \url{http://www.latex-project.org/lppl.txt}
and version 1.3 or later is part of all distributions of \LaTeX{}
version 2005/12/01 or later.

This work has the LPPL maintenance status `maintained'.

The Current Maintainer of this work is Niklas Beisert.

This work consists of the files |README.txt|, |childdoc.ins| and |childdoc.dtx|
as well as the derived files |childdoc.def|, |cdocsamp.tex|
with |cdocsch1.tex|, |cdocsch2.tex|, |cdocspt3.tex|, |cdocspt4.tex|,
|cdocsdrf.tex|, |cdocsfn1.tex|, |cdocsfn2.tex|
as well as |childdoc.pdf|.

%%%%%%%%%%%%%%%%%%%%%%%%%%%%%%%%%%%%%%%%%%%%%%%%%%%%%%%%%%%%%%%%%%%%%%%%%%%%%%%%
\subsection{Files and Installation}

The package consists of the files:
%
\begin{center}
\begin{tabular}{ll}
    |README.txt|   & readme file \\
    |childdoc.ins| & installation file \\
    |childdoc.dtx| & source file \\
    |childdoc.def| & definition file \\
    |cdocsamp.tex| & sample main file \\
    |cdocsch1.tex| & sample include file \\
    |cdocsch2.tex| & sample include file \\
    |cdocspt3.tex| & sample part file \\
    |cdocspt4.tex| & sample part file \\
    |cdocsdrf.tex| & sample redirection file \\
    |cdocsfn1.tex| & sample redirection file \\
    |cdocsfn2.tex| & sample redirection file \\
    |childdoc.pdf| & manual
\end{tabular}
\end{center}
%
The distribution consists of the files
|README.txt|, |childdoc.ins| and |childdoc.dtx|.
%
\begin{itemize}
\item
Run (pdf)\LaTeX{} on |childdoc.dtx|
to compile the manual |childdoc.pdf| (this file).
\item
Run \LaTeX{} on |childdoc.ins| to create the definitions file |childdoc.def|
and the sample |cdocsamp.tex| with include files
|cdocsch1.tex|, |cdocsch2.tex|, |cdocspt3.tex|, |cdocspt4.tex|,
|cdocsdrf.tex|, |cdocsfn1.tex|, |cdocsfn2.tex|.
Then copy the file |childdoc.def| to an appropriate directory of your \LaTeX{}
distribution, e.g.\ \textit{texmf-root}|/tex/latex/childdoc|.
\end{itemize}

%%%%%%%%%%%%%%%%%%%%%%%%%%%%%%%%%%%%%%%%%%%%%%%%%%%%%%%%%%%%%%%%%%%%%%%%%%%%%%%%
\subsection{Related CTAN Packages}

There are several other packages which offer a similar functionality:
%
\begin{itemize}
\item
The packages
\href{http://ctan.org/pkg/docmute}{\textsf{docmute}},
\href{http://ctan.org/pkg/includex}{\textsf{includex}} and
\href{http://ctan.org/pkg/standalone}{\textsf{standalone}}
provide commands to include only the document body of
a child file thus allowing both files to be compiled individually.
\item
The packages \href{http://ctan.org/pkg/subdocs}{\textsf{subdocs}}
and \href{http://ctan.org/pkg/subfiles}{\textsf{subfiles}}
provide structures in which the main and child documents can be
encapsulated and allowing them to be compiled individually.
The inclusion mechanism is different from the conventional |\include|.
\item
The package \href{http://ctan.org/pkg/combine}{\textsf{combine}}
is an elaborate solution to combine several documents into one.
\end{itemize}
%
See also the CTAN topic \href{http://ctan.org/topic/subdocs}{\textsf{subdocs}}
for further related packages.
The present package differs from the above solutions in that
a document structure constructed with the conventional |\include| mechanism
just needs two extra commands at the top of every file
such that all constituent files can be compiled individually.

%%%%%%%%%%%%%%%%%%%%%%%%%%%%%%%%%%%%%%%%%%%%%%%%%%%%%%%%%%%%%%%%%%%%%%%%%%%%%%%%
%\subsection{Feature Suggestions}
%
%The following is a list of features which may be useful for future
%versions of this package:
%%
%\begin{itemize}
%\item
%\ldots
%\end{itemize}

%%%%%%%%%%%%%%%%%%%%%%%%%%%%%%%%%%%%%%%%%%%%%%%%%%%%%%%%%%%%%%%%%%%%%%%%%%%%%%%%
\subsection{Revision History}

%%%%%%%%%%%%%%%%%%%%%%%%%%%%%%%%%%%%%%%%
\paragraph{v2.0:} 2018/12/30

\begin{itemize}
\item
immediate forward processing
\item
added |\childdocby| mechanism
\item
manual restructured
\end{itemize}

%%%%%%%%%%%%%%%%%%%%%%%%%%%%%%%%%%%%%%%%
\paragraph{v1.6:} 2018/01/17

\begin{itemize}
\item
application for development of include files
\item
corrections to manual
\end{itemize}

%%%%%%%%%%%%%%%%%%%%%%%%%%%%%%%%%%%%%%%%
\paragraph{v1.5:} 2017/05/21

\begin{itemize}
\item
more complete structuring introduced
\item
|\childdocof| introduced
\item
|\childdoc| renamed to |\childdocmain|
\item
|\childredirect| renamed to |\childdocforward| and |\childdocforwardprefix|
and functionality expanded
\end{itemize}

%%%%%%%%%%%%%%%%%%%%%%%%%%%%%%%%%%%%%%%%
\paragraph{v1.0:} 2017/04/27

\begin{itemize}
\item
manual and install package
\item
first version published on CTAN
\end{itemize}

%%%%%%%%%%%%%%%%%%%%%%%%%%%%%%%%%%%%%%%%
\paragraph{v0.6:} 2017/04/26

\begin{itemize}
\item
redirection mechanism added
\end{itemize}

%%%%%%%%%%%%%%%%%%%%%%%%%%%%%%%%%%%%%%%%
\paragraph{v0.5:} 2017/04/26

\begin{itemize}
\item
functionality in definition file
\end{itemize}


%%%%%%%%%%%%%%%%%%%%%%%%%%%%%%%%%%%%%%%%%%%%%%%%%%%%%%%%%%%%%%%%%%%%%%%%%%%%%%%%
%%%%%%%%%%%%%%%%%%%%%%%%%%%%%%%%%%%%%%%%%%%%%%%%%%%%%%%%%%%%%%%%%%%%%%%%%%%%%%%%
%%%%%%%%%%%%%%%%%%%%%%%%%%%%%%%%%%%%%%%%%%%%%%%%%%%%%%%%%%%%%%%%%%%%%%%%%%%%%%%%
\appendix

\settowidth\MacroIndent{\rmfamily\scriptsize 000\ }

 \DocInput{childdoc.dtx}

\end{document}
%</driver>
% \fi
%
% %%%%%%%%%%%%%%%%%%%%%%%%%%%%%%%%%%%%%%%%%%%%%%%%%%%%%%%%%%%%%%%%%%%%%%%%%%%%%%
% %%%%%%%%%%%%%%%%%%%%%%%%%%%%%%%%%%%%%%%%%%%%%%%%%%%%%%%%%%%%%%%%%%%%%%%%%%%%%%
% \section{Sample}
%\iffalse
%<*samplemain>
%\fi
%
% The following presents a sample document
% with two chapters, two parts, a title page,
% a compile flag as well as three forwarding files to set the flag.
% It consists of eight |.tex| files:
% \begin{center}
% \begin{tabular}{ll}
% |cdocsamp.tex|&main file\\
% |cdocsch1.tex|&include file for chapter 1\\
% |cdocsch2.tex|&include file for chapter 2\\
% |cdocspt3.tex|&include file for part 3\\
% |cdocspt4.tex|&include file for part 4\\
% |cdocsdrf.tex|&forwarding file for main file in draft mode\\
% |cdocsfi1.tex|&forwarding file for final version of chapter 1\\
% |cdocsfi2.tex|&forwarding file for final version of chapter 2\\
% \end{tabular}
% \end{center}
% Each of the eight files can be compiled directly by the \LaTeX{} compiler.
%
% %%%%%%%%%%%%%%%%%%%%%%%%%%%%%%%%%%%%%%
% \paragraph{Main File.}
%
% The main file is called |cdocsamp.tex|.
%
% Load the \textsf{childdoc} definitions and
% declare the filename for the main document:
%    \begin{macrocode}
\input{childdoc.def}
\childdocmain{}
%    \end{macrocode}

% Optional override for |\version| flag:
%    \begin{macrocode}
%%\ifchilddoc\else\providecommand{\version}{draft}\fi
%    \end{macrocode}

% Define the default values for the |\version| flag
% (|final| for the main file and |draft| for childs):
%    \begin{macrocode}
\ifchilddoc
\providecommand{\version}{draft}
\else
\providecommand{\version}{final}
\fi
%    \end{macrocode}

% Load the standard document class:
%    \begin{macrocode}
\documentclass[12pt]{article}
%    \end{macrocode}

% Start the document body:
%    \begin{macrocode}
\begin{document}
%    \end{macrocode}

% Declare a title page.
% Print title, part of document being processed and version flag:
%    \begin{macrocode}
\addtocounter{page}{-1}
\begin{center}
{\LARGE\bfseries{}childdoc example\par}
\vspace{1cm}
\ifchilddoc
\ifchilddocmanual part\else chapter\fi:
`\childdocname' of `\childdocjob'\par
\else
main document: `\childdocjob'\par
\fi
version: \version\par
\end{center}
\newpage
%    \end{macrocode}

% Manually include selected file,
% otherwise process as usual:
%    \begin{macrocode}
\ifchilddocmanual
\section*{part `\childdocname'}
\input{\childdocname}
\else
%    \end{macrocode}

% Include the two chapters:
%    \begin{macrocode}
\include{cdocsch1}
\include{cdocsch2}
%    \end{macrocode}

% Include the two parts unless only chapters should be displayed:
%    \begin{macrocode}
\ifchilddoc\else
\section{part three}
\input{cdocspt3}
\section{part four}
\input{cdocspt4}
\fi
%    \end{macrocode}

% Process as usual until here:
%    \begin{macrocode}
\fi
%    \end{macrocode}

% End of document body:
%    \begin{macrocode}
\end{document}
%    \end{macrocode}
%\iffalse
%</samplemain>
%\fi
%
% %%%%%%%%%%%%%%%%%%%%%%%%%%%%%%%%%%%%%%
% \paragraph{Chapter Include Files.}
%
% The include files are called |cdocsch1.tex| and |cdocsch2.tex|.
%
%\iffalse
%<*samplechap1|samplechap2>
%\fi

% Optional override for |\version| flag:
%    \begin{macrocode}
%%\providecommand{\version}{final}
%    \end{macrocode}

% Include the main document:
%    \begin{macrocode}
\input{childdoc.def}
\childdocof{cdocsamp}
%    \end{macrocode}

%\iffalse
%</samplechap1|samplechap2>
%\fi
%
%\iffalse
%<*samplechap1>
%\fi
% Some text for chapter 1:
%    \begin{macrocode}
\section{one}
some text in chapter one
%    \end{macrocode}

%\iffalse
%</samplechap1>
%\fi
% Some text for chapter 2:
%\iffalse
%<*samplechap2>
%\fi
%    \begin{macrocode}
\section{two}
more text in chapter two
%    \end{macrocode}

%\iffalse
%</samplechap2>
%\fi
%
% %%%%%%%%%%%%%%%%%%%%%%%%%%%%%%%%%%%%%%
% \paragraph{Part Include Files.}
%
% The include files are called |cdocspt3.tex| and |cdocspt4.tex|.
%
%\iffalse
%<*samplepart3|samplepart4>
%\fi

% Optional override for |\version| flag:
%    \begin{macrocode}
%%\providecommand{\version}{final}
%    \end{macrocode}

% Include the main document:
%    \begin{macrocode}
\input{childdoc.def}
\childdocby{cdocsamp}
%    \end{macrocode}

%\iffalse
%</samplepart3|samplepart4>
%\fi
%
%\iffalse
%<*samplepart3>
%\fi
% Some text for part 3:
%    \begin{macrocode}
some text in part three
%    \end{macrocode}

%\iffalse
%</samplepart3>
%\fi
% Some text for part 4:
%\iffalse
%<*samplepart4>
%\fi
%    \begin{macrocode}
more text in part four
%    \end{macrocode}

%\iffalse
%</samplepart4>
%\fi
%
% %%%%%%%%%%%%%%%%%%%%%%%%%%%%%%%%%%%%%%
% \paragraph{Forwarding for a Complete Draft.}
%
% The following forwarding file |cdocsdrf.tex|
% compiles the main document in draft mode:
%\iffalse
%<*sampledraft>
%\fi
%    \begin{macrocode}
\def\version{draft}
\input{childdoc.def}
\childdocforward{cdocsamp}
%    \end{macrocode}

%\iffalse
%</sampledraft>
%\fi
%
% %%%%%%%%%%%%%%%%%%%%%%%%%%%%%%%%%%%%%%
% \paragraph{Forwarding for Final Version of the Chapters.}
%
% The following forwarding files |cdocsfn1.tex| and |cdocsfn2.tex|
% (with identical content)
% compile the final versions of the child documents
% |cdocsch1.tex| and |cdocsch2.tex|, respectively:
%\iffalse
%<*samplefinal>
%\fi
%    \begin{macrocode}
\def\version{final}
\input{childdoc.def}
\childdocforwardprefix[cdocsamp]{cdocsfn}{cdocsch}
%    \end{macrocode}

%\iffalse
%</samplefinal>
%\fi
%
% %%%%%%%%%%%%%%%%%%%%%%%%%%%%%%%%%%%%%%
% \paragraph{Command Line Processing.}
%
% The following three command lines generate the output files
% |cdocscld|, |cdocscl1| and |cdocscl2|
% which should be identical to
% |cdocsdrf|, |cdocsch1| and |cdocsfn2|, respectively:
% \begin{center}
% \begin{tabular}{l}
% |latex -jobname cdocscld \|\\
% |  "\def\version{draft}\input{childdoc.def}\childdocforward{cdocsamp}"|\\
% |latex -jobname cdocscl1 \|\\
% |  "\input{childdoc.def}\childdocforward[cdocsamp]{cdocsch1}"|\\
% |latex -jobname cdocscl2 \|\\
% |  "\def\version{final}\input{childdoc.def}\childdocforward{cdocsch2}"|
% \end{tabular}
% \end{center}
% Note that the trailing backslash on each first line
% merely continues the input to the second line
% (for convenient cut ant paste).
% Furthermore, the command |latex| can be replaced by any
% of its alternative versions such as |pdflatex|.
%
% %%%%%%%%%%%%%%%%%%%%%%%%%%%%%%%%%%%%%%%%%%%%%%%%%%%%%%%%%%%%%%%%%%%%%%%%%%%%%%
% %%%%%%%%%%%%%%%%%%%%%%%%%%%%%%%%%%%%%%%%%%%%%%%%%%%%%%%%%%%%%%%%%%%%%%%%%%%%%%
% \section{Implementation}
%\iffalse
%<*package>
%\fi
%
% This section describes the definitions file |childdoc.def|.

% The definitions cannot be loaded using |\usepackage| or |\RequirePackage|
% which has a mechanism to prevent loading a style file more than once.
% When loading the definitions by means of |\input|
% multiple instances have to be prevented manually:
%\iffalse
%This code needs to be before the `\ProvidesFile' directive
%which is defined at the beginning of this file.
%Therefore it is also placed there and commented out here.
%</package>
%<*discard>
%\fi
%    \begin{macrocode}
\ifdefined\childdocmain\endinput\fi
%    \end{macrocode}
%\iffalse
%</discard>
%<*package>
%\fi
%
% \macro{\ifchilddoc}
% \macro{\ifchilddocmanual}
% The conditional |\ifchilddoc| tells whether a
% child (true) or main (false) document is being compiled.
% The conditional |\ifchilddocmanual| tells whether
% the |\includeonly| mechanism is used (false) or
% the selection of child files must be performed manually (true).
% The definitions initialise to false:
%    \begin{macrocode}
\newif\ifchilddoc
\newif\ifchilddocmanual
%    \end{macrocode}

% \macro{\childdocname}
% \macro{\childdocjob}
% The macro |\childdocname| stores the name of the main document
% to be compiled. The macro |\childdocjob| stores the name of
% the document on which the \LaTeX{} compiler was originally invoked.
% The content of |\jobname| cannot be compared
% to filenames specified in the source due to different catcodes.
% The following code rescans |\jobname|, stores the result
% in |\childdocname| and saves a copy in |\childdocjob|:
%    \begin{macrocode}
\edef\childdocname{\scantokens\expandafter{\jobname\noexpand}}
\let\childdocjob\childdocname
%    \end{macrocode}

% \macro{\childdocdisable}
% The macro |\childdocdisable| prevents the main file
% from being processed more than once.
% At this stage, the main document command |\childdocmain|
% is assumed to be called once again where it should do nothing.
% Any subsequent call to it should prevent
% a secondary processing of the main document
% It overwrites the forwarding commands
% |\childdocof| and |\childdocforward|
% with empty macros to prevent further inclusions of the main document:
%    \begin{macrocode}
\newcommand{\childdocdisable}
{
  \renewcommand{\childdocmain}[1]{\renewcommand{\childdocmain}[1]{\endinput}}
  \renewcommand{\childdocof}[1]{}
  \renewcommand{\childdocby}[2][]{}
  \renewcommand{\childdocforward}[2][]{}
  \renewcommand{\childdocdisable}{}
}
%    \end{macrocode}

% \macro{\childdocmain}
% The macro |\childdocmain| is to be called at the top of the main file
% with nothing or the main filename (without extension) as argument.
% First, it breaks loops.
% If the argument is not empty and does not match |\childdocname|
% (which is set by the first inclusion of |childdoc.def|),
% |\ifchilddoc| is set to true, |\includeonly| is applied to the child file
% and |\jobname| is set to the main file
% (for proper handling of |.aux| files):
%    \begin{macrocode}
\newcommand{\childdocmain}[1]
{
  \childdocdisable\childdocmain{}
  \if?#1?\else
    \begingroup
      \def\childdoctmp{#1}
      \ifx\childdoctmp\childdocname
        \def\childdoctmp{}
      \else
        \def\childdoctmp
        {
          \childdoctrue
          \includeonly{\childdocname}
          \def\childdocjob{#1}
          \def\jobname{#1}
        }
      \fi
      \expandafter
    \endgroup
    \childdoctmp
  \fi
}
%    \end{macrocode}

% \macro{\childdocof}
% The command |\childdocof| redirects
% compilation to the main file |#1|.
%    \begin{macrocode}
\newcommand{\childdocof}[1]
{
  \childdocdisable
  \childdoctrue
  \includeonly{\childdocname}
  \def\jobname{#1}
  \def\childdocjob{#1}
  \input{#1}
}
%    \end{macrocode}

% \macro{\childdocby}
% The command |\childdocby| ....
%    \begin{macrocode}
\newcommand{\childdocby}[2][]
{
  \childdocdisable
  \childdoctrue
  \childdocmanualtrue
  \if?#1?\else
    \def\jobname{#2}
  \fi
  \def\childdocjob{#2}
  \input{#2}
  \endinput
}
%    \end{macrocode}

% \macro{\childdocforward}
% The command |\childdocforward| redirects
% compilation to the main file or
% (if the optional argument is given) a child file.
% Parameters are set as if the main file
% or a child file starting with |\childdocof| was compiled.
% Then compilation is handed over to the main file:
%    \begin{macrocode}
\newcommand{\childdocforward}[2][]
{
  \begingroup
    \if?#1?
      \def\childdoctmp
      {
        \def\childdocname{#2}
        \def\childdocjob{#2}
        \def\jobname{#2}
        \input{#2}
        \endinput
      }
    \else
      \def\childdoctmp
      {
        \childdocdisable
        \def\childdocname{#2}
        \childdoctrue
        \includeonly{#2}
        \def\childdocjob{#1}
        \def\jobname{#1}
        \input{#1}
        \endinput
      }
    \fi
    \expandafter
  \endgroup
  \childdoctmp
}
%    \end{macrocode}

% \macro{\childdocforwardprefix}
% The command |\childdocforwardprefix| redirects
% compilation to the main or a child file by means of a pattern.
% The prefix |#1| in the current filename is replaced by |#2|
% and the suffix of the current filename is kept
% (it is assumed that the filename does not contain the substring `|~~~|'
% which is used as a delimiter).
% Compilation is handed over to the new file by |\childdocforward|:
%    \begin{macrocode}
\newcommand{\childdocforwardprefix}[3][]
{
  \begingroup
    \def\childdocextract #2##1~~~{\def\childdoctmp{\childdocforward[#1]{#3##1}}}
    \expandafter\childdocextract\childdocname~~~
    \expandafter
  \endgroup
  \childdoctmp
}
%    \end{macrocode}

% \macro{\childdoc}
% The deprecated macro |\childdoc| is a legacy version of |\childdocmain|:
%    \begin{macrocode}
\newcommand{\childdoc}{\childdocmain}
%    \end{macrocode}

% \macro{\childdocredirect}
% The deprecated macro |\childdocredirect| is a legacy version
% of |\childdocforward| and |\childdocforwardprefix|:
%    \begin{macrocode}
\newcommand{\childdocredirect}[2][]
{
  \begingroup
    \if?#1?
      \def\childdoctmp{\childdocforward{#2}}
    \else
      \def\childdoctmp{\childdocforwardprefix{#1}{#2}}
    \fi
    \expandafter
  \endgroup
  \childdoctmp
}
%    \end{macrocode}

%\iffalse
%</package>
%\fi
%
\endinput
|\\
|\childdocforwardprefix{final}{child}|
\end{tabular}
\end{center}
%

Note that when several versions of a main file and/or of each child file
are to be generated, it may be convenient to set up a |Makefile| or
shell script to automatise the process.

%%%%%%%%%%%%%%%%%%%%%%%%%%%%%%%%%%%%%%%%%%%%%%%%%%%%%%%%%%%%%%%%%%%%%%%%%%%%%%%%
\subsection{Command Line Processing}
\label{sec:commandline}

The effect of redirection files can also be achieved by invoking
the \LaTeX{} compiler with a more elaborate command line.
Most conveniently this should be done as part
of a shell script or a |Makefile|.

When using \textsf{childdoc} in the main file, the following
command lines effectively perform a redirection
(note that depending on the shell being used,
backslashes may have to be doubled: `|\|' $\to$ `|\\|'):
%
\begin{center}
|... -jobname "|\textit{target}|" |\\|"|[\textit{flags}]%
|% \iffalse
%
% childdoc.dtx Copyright (C) 2017-2018 Niklas Beisert
%
% This work may be distributed and/or modified under the
% conditions of the LaTeX Project Public License, either version 1.3
% of this license or (at your option) any later version.
% The latest version of this license is in
%   http://www.latex-project.org/lppl.txt
% and version 1.3 or later is part of all distributions of LaTeX
% version 2005/12/01 or later.
%
% This work has the LPPL maintenance status `maintained'.
%
% The Current Maintainer of this work is Niklas Beisert.
%
% This work consists of the files childdoc.dtx and childdoc.ins
% and the derived files childdoc.def and cdocsamp.tex with
% cdocsch1.tex, cdocsch2.tex, cdocsdrf.tex, cdocsfn1.tex, cdocsfn2.tex.
%
%<package>\ifdefined\childdocmain\endinput\fi
%<package>\ProvidesFile{childdoc.def}[2018/12/30 v2.0 child document driver]
%<samplemain>\ProvidesFile{cdocsamp.tex}[2018/12/30 v2.0 sample for childdoc]
%<*driver>
%\ProvidesFile{childdoc.drv}[2018/12/30 v2.0 childdoc reference manual file]
\PassOptionsToClass{10pt,a4paper}{article}
\documentclass{ltxdoc}

\usepackage[margin=35mm]{geometry}
\usepackage{hyperref}
\usepackage{hyperxmp}
\usepackage[usenames]{color}

\hypersetup{colorlinks=true}
\hypersetup{pdfstartview=FitH}
\hypersetup{pdfpagemode=UseNone}
\hypersetup{pdfsource={}}
\hypersetup{pdflang={en-UK}}
\hypersetup{pdfcopyright={Copyright 2017-2018 Niklas Beisert.
  This work may be distributed and/or modified under the
  conditions of the LaTeX Project Public License, either version 1.3
  of this license or (at your option) any later version.}}
\hypersetup{pdflicenseurl={http://www.latex-project.org/lppl.txt}}
\hypersetup{pdfcontactaddress={ETH Zurich, ITP, HIT K,
  Wolfgang-Pauli-Strasse 27}}
\hypersetup{pdfcontactpostcode={8093}}
\hypersetup{pdfcontactcity={Zurich}}
\hypersetup{pdfcontactcountry={Switzerland}}
\hypersetup{pdfcontactemail={nbeisert@itp.phys.ethz.ch}}
\hypersetup{pdfcontacturl={http://people.phys.ethz.ch/\xmptilde nbeisert/}}

\newcommand{\secref}[1]{\hyperref[#1]{section \ref*{#1}}}

\parskip1ex
\parindent0pt
\let\olditemize\itemize
\def\itemize{\olditemize\parskip0pt}

\begin{document}

\title{The \textsf{childdoc} Package}
\hypersetup{pdftitle={The childdoc Package}}
\author{Niklas Beisert\\[2ex]
  Institut f\"ur Theoretische Physik\\
  Eidgen\"ossische Technische Hochschule Z\"urich\\
  Wolfgang-Pauli-Strasse 27, 8093 Z\"urich, Switzerland\\[1ex]
  \href{mailto:nbeisert@itp.phys.ethz.ch}
  {\texttt{nbeisert@itp.phys.ethz.ch}}}
\hypersetup{pdfauthor={Niklas Beisert}}
\hypersetup{pdfsubject={Manual for the LaTeX2e Package childdoc}}
\date{30 December 2018, \textsf{v2.0}}
\maketitle

\begin{abstract}\noindent
\textsf{childdoc} is a \LaTeXe{} package
that enables the direct compilation
of document sections included by |\include|
to individual files.
\end{abstract}

\begingroup
\parskip0ex
\tableofcontents
\endgroup

%%%%%%%%%%%%%%%%%%%%%%%%%%%%%%%%%%%%%%%%%%%%%%%%%%%%%%%%%%%%%%%%%%%%%%%%%%%%%%%%
%%%%%%%%%%%%%%%%%%%%%%%%%%%%%%%%%%%%%%%%%%%%%%%%%%%%%%%%%%%%%%%%%%%%%%%%%%%%%%%%
\section{Introduction}

\LaTeX{} provides a mechanism to structure a large document (such as a book)
into a main file and several child files (containing the chapters)
using the |\include| command.
This mechanism is beneficial for documents
which span hundreds of pages in order to
make the source file(s) more manageable.
Moreover, compilation can be restricted to
selected child files by means of the |\includeonly| command.
The latter feature can be used to reduce the compilation time while editing
(this was significantly more useful in the earlier days of \LaTeX{})
or to generate a smaller document which is easier to navigate.
Another application of |\includeonly| is to generate
documents consisting of selected parts of the complete document.

However, there are a few drawbacks of the plain |\include| mechanism:
\begin{itemize}
\item
The child files cannot be compiled on their own,
they can only be compiled via the main file.
A naive editing environment
(such as a text editor with an option
to have the current file processed by \LaTeX)
may require one to switch to the main file before compiling;
attempting to compile the child file produces errors.
\item
The main file must be modified (each time)
to adjust the |\includeonly| command
to the present needs. This easily leaves the main file in a messy state.
\item
The generated document will always carry the filename
of the main document. This is inconvenient if
several child files are to be compiled and
to be kept for distribution.
\end{itemize}

The present package provides a simple interface
to make child files individually compilable by \LaTeX{}.
Compiling a child file then has the same effect as compiling
the main file with an |\includeonly| command
to select the appropriate child.
Moreover the generated document will carry the name of the child
rather than the main file.
This resolves all three above issues.

This feature is meant to make the editing of books,
thesis documents and lecture notes somewhat more convenient.
However, the package can also be used efficiently for
composing a series of documents (such as exercise sheets)
which are typically distributed individually.
It then assists the author in generating the individual documents
(potentially in different versions)
as well as a document containing the collected series.
Another application is in developing style files
or other kinds of included material
where compilation of the style file could redirect
to a sample or test file.

%%%%%%%%%%%%%%%%%%%%%%%%%%%%%%%%%%%%%%%%%%%%%%%%%%%%%%%%%%%%%%%%%%%%%%%%%%%%%%%%
%%%%%%%%%%%%%%%%%%%%%%%%%%%%%%%%%%%%%%%%%%%%%%%%%%%%%%%%%%%%%%%%%%%%%%%%%%%%%%%%
\section{Usage}

First of all, the package \textsf{childdoc} is \emph{not} a standard
\LaTeXe{} |.sty| style file! Therefore it needs to be invoked in
a non-standard way.

%%%%%%%%%%%%%%%%%%%%%%%%%%%%%%%%%%%%%%%%%%%%%%%%%%%%%%%%%%%%%%%%%%%%%%%%%%%%%%%%
\subsection{Included Files}
\label{sec:include}

%%%%%%%%%%%%%%%%%%%%%%%%%%%%%%%%%%%%%%%%
\DescribeMacro{\childdocmain}
To use the package, add the commands
\begin{center}
\begin{tabular}{l}
|\input{childdoc.def}|\\
|\childdocmain{}|\\
\end{tabular}
\end{center}
at the very top of the main \LaTeX{} file,
in particular \emph{before} the |\documentclass| statement!
The argument of |\childdocmain| should be left empty
(but it must be present).

%%%%%%%%%%%%%%%%%%%%%%%%%%%%%%%%%%%%%%%%
\DescribeMacro{\childdocof}
Furthermore, add the commands
\begin{center}
\begin{tabular}{l}
|\input{childdoc.def}|\\
|\childdocof{|\textit{main}|}|\\
\end{tabular}
\end{center}
at the top of every child file \textit{child}
which is included by |\include{|\textit{child}|}|
from within the main file
(or at least for those files to be compiled individually).
The argument \textit{main} must be the filename of the main file.

There are a couple of
considerations in setting up the main and child documents:

%%%%%%%%%%%%%%%%%%%%%%%%%%%%%%%%%%%%%%%%
\paragraph{Restrictions.}

Please note the following restrictions:
\begin{itemize}
\item
|\childdocmain| must be called with one argument \textit{main}
to ensure compatibility with earlier version of the package.
It must either be empty (|\childdocmain{}|)
or precisely match the filename of the main file in which it is specified.
See \secref{sec:detection} for further information.
\item
The filename \textit{main} must be specified without the |.tex| extension.
\item
The filename \textit{main} is case sensitive
(even in case-insensitive file systems)
due to internal string comparison.
\item
The argument \textit{main} should be fully expanded, it cannot be a macro.
\item
Subdirectories and special characters should be avoided in filenames.
\item
The command |\childdocmain{|\textit{main}|}| must be followed by a whitespace.
It should not be followed immediately by another command
or by a comment mark `|%|'.
This is because the \TeX{} parser reads the token immediately following
the argument of |\childdocmain| and puts it
at the beginning of every child section;
however, a white\-space is ignored.
\end{itemize}

%%%%%%%%%%%%%%%%%%%%%%%%%%%%%%%%%%%%%%%%
\paragraph{Content of Main File.}

It is advisable to place all content in the child files included by |\include|.
Any output contained in the main file will appear in all child documents
unless suppressed manually;
it cannot be suppressed automatically by the |\includeonly| directive
and thus should normally be avoided.
A method to include some content in the main file
by means of conditional processing is described in \secref{sec:conditional}.

%%%%%%%%%%%%%%%%%%%%%%%%%%%%%%%%%%%%%%%%
\paragraph{Page Numbering.}

When only a part of the document is compiled,
the appropriate numbering of pages
(as well as other status parameters)
is determined from the |.aux| files.
The latter contain information from previous passes.
However this information needs to propagate through
all intermediate child documents.
Therefore the page numbering in child documents may well
be inconsistent until the complete document is compiled at least once.

A useful (if unconventional) way to always ensure a consistent
page numbering is to restart the numbering in each child document
and denote the pages by `\textit{child}|.|\textit{page}'
where \textit{child} represents the chapter/section number of the child file.
This can be achieved by the command
|\numberwithin{page}{|\textit{child}|}|
of the \textsf{amsmath} package
where \textit{child} can be |chapter| or |section|
depending on the chosen structuring.
Alternatively, one can modify the macro |\thepage| appropriately
and reset the counter |page| at the start of each child file.

%%%%%%%%%%%%%%%%%%%%%%%%%%%%%%%%%%%%%%%%%%%%%%%%%%%%%%%%%%%%%%%%%%%%%%%%%%%%%%%%
\subsection{Conditional Processing}
\label{sec:conditional}

The package provides a mechanism to compile different versions
of a document. To customise the versions further some conditional processing
can come in handy to distinguish which version is being compiled.
The package provides two macros to describe the compilation context:

%%%%%%%%%%%%%%%%%%%%%%%%%%%%%%%%%%%%%%%%
\DescribeMacro{\ifchilddoc}
The conditional |\ifchilddoc| distinguishes between the compilation of
child documents and the main document:
%
\begin{center}
|\ifchilddoc |\textit{child-code}| |[|\||else |\textit{main-code}]| \||fi|
\end{center}

%%%%%%%%%%%%%%%%%%%%%%%%%%%%%%%%%%%%%%%%
\DescribeMacro{\childdocname}
\DescribeMacro{\childdocjob}
The macro |\childdocname| contains the filename (without extension)
of the main or child file being processed.
Note that |\childdocjob| will always contain the name of the main file.

%%%%%%%%%%%%%%%%%%%%%%%%%%%%%%%%%%%%%%%%
\paragraph{Title Page.}

Conditional processing can be used to include a title or banner page
in the main document when proper precautions are taken.
Importantly, the code in the main file should ensure that the page counter
(as well as other status parameters which are stored in the |.aux| files)
takes the same value after the conditional processing.
Otherwise the page numbers may take divergent values
depending on which part is compiled.

For example, a title page could be declared by:
%
\begin{center}
\begin{tabular}{l}
|\ifchilddoc\||else|\\
|\addtocounter{page}{-1}|\\
\textit{code for title page}\\
|\newpage|\\
|\||fi|
\end{tabular}
\end{center}
%
A banner page for the child documents can be generated by:
%
\begin{center}
\begin{tabular}{l}
|\ifchilddoc|\\
|\addtocounter{page}{-1}|\\
\textit{code for banner page}\\
|\newpage|\\
|\||fi|
\end{tabular}
\end{center}
%
Here one could write a message such as:
\begin{center}
|This is the part \childdocname{} of \childdocjob{}.|
\end{center}

%%%%%%%%%%%%%%%%%%%%%%%%%%%%%%%%%%%%%%%%%%%%%%%%%%%%%%%%%%%%%%%%%%%%%%%%%%%%%%%%
\subsection{Flags}
\label{sec:flags}

The package makes it easy to generate different versions
of the main or child documents.
To this end compilation flags can be defined
and assigned different default values.
They will be particularly useful in conjunction
with the forwarding mechanism described in \secref{sec:forward}.

For example, it may be useful to have a flag |\version|
which can be set to |draft| or |final|.
The document source will contain some conditional code
depending on the value of |\version|.
Suppose further, the flag should default to |final| for the main file
and to |draft| for child files
which is a natural assignment for editing the document.
This is achieved by placing the following code
in the preamble of the main document
(below the |\childdocmain| directive):
%
\begin{center}
\begin{tabular}{l}
|\ifchilddoc|\\
|\providecommand{\version}{draft}|\\
|\||else|\\
|\providecommand{\version}{final}|\\
|\||fi|
\end{tabular}
\end{center}
%
The definition by |\providecommand| makes sure
that previous definitions are not overwritten.
Further statements |\providecommand{\version}{...}|
can thus be added before the above code to override it.

For the main file, one might add a line
(between |\childdocmain| and the above block)
%
\begin{center}
|%\ifchilddoc\||else\providecommand{\version}{draft}\||fi|
\end{center}
%
which can be uncommented to produce a draft version.
Likewise one can add a line to the very top of a child file
(above the |\childdocof{|\textit{main}|}| directive)
%
\begin{center}
|%\providecommand{\version}{final}|
\end{center}
%
which can be uncommented to produce the final version of this child document.

%%%%%%%%%%%%%%%%%%%%%%%%%%%%%%%%%%%%%%%%%%%%%%%%%%%%%%%%%%%%%%%%%%%%%%%%%%%%%%%%
\subsection{Forwarding}
\label{sec:forward}

Different versions of the main or child documents
using compilation flags as described in \secref{sec:flags}
can be (permanently) stored in different files
for convenient compilation, viewing and distribution.
To this end, the package defines a command
to pass on compilation to a different file:

%%%%%%%%%%%%%%%%%%%%%%%%%%%%%%%%%%%%%%%%
\DescribeMacro{\childdocforward}
The command |\childdocforward| redirects processing to
another source file:
%
\begin{center}
\begin{tabular}{l}
|\input{childdoc.def}|\\
|\childdocforward[|\textit{main}|]{|\textit{dest}|}|\\
\end{tabular}
\end{center}
%
The argument \textit{dest} is the destination file
(without extension).
It should be the main file or one of the child files.
Note that further \textsf{childdoc} directives
such as |\childdocof| and |\childdocforward|
in the indicated file will be processed in this form.
The optional argument \textit{main}
passes on directly to the main file \textit{main}
while pretending to compile the child \textit{dest}.
This form behaves as if \textit{dest}
issues |\childdocof{|\textit{main}|}| right away,
and no further \textsf{childdoc} directives will be processed.

%%%%%%%%%%%%%%%%%%%%%%%%%%%%%%%%%%%%%%%%
\DescribeMacro{\...prefix}
In the alternative form |\childdocforwardprefix|,
%
\begin{center}
\begin{tabular}{l}
|\input{childdoc.def}|\\
|\childdocforwardprefix[|\textit{main}|]{|\textit{prefix}|}{|\textit{dest}|}|
\end{tabular}
\end{center}
%
the destination file is determined by a pattern
depending on the current file:
To make this work, the current file must be called
`{\textit{prefix}\hspace{0.2em}\textit{suffix}}'
with \textit{prefix} matching precisely the argument.
Processing is then passed on to the file
`{\textit{dest}\hspace{0.2em}\textit{suffix}}'.
Surely, the same effect is achieved by
directly specifying the
argument `{\textit{dest}\hspace{0.2em}\textit{suffix}}'
in the first form.
However, that requires to set up a different file
for each child. With the alternative form of the command
all these files can have exactly the same content
which simplifies setting them up and maintaining them.

For example, the following file |draft.tex|
with a compilation flag |\version| as described in \secref{sec:flags}
compiles the main document as a draft:
%
\begin{center}
\begin{tabular}{l}
|\def\version{draft}|\\
|\input{childdoc.def}|\\
|\childdocforward{|\textit{main}|}|
\end{tabular}
\end{center}
%
Likewise, the following files |final|\textit{nn}|.tex|
compile the final version of the child document
|child|\textit{nn}|.tex|:
%
\begin{center}
\begin{tabular}{l}
|\def\version{final}|\\
|\input{childdoc.def}|\\
|\childdocforwardprefix{final}{child}|
\end{tabular}
\end{center}
%

Note that when several versions of a main file and/or of each child file
are to be generated, it may be convenient to set up a |Makefile| or
shell script to automatise the process.

%%%%%%%%%%%%%%%%%%%%%%%%%%%%%%%%%%%%%%%%%%%%%%%%%%%%%%%%%%%%%%%%%%%%%%%%%%%%%%%%
\subsection{Command Line Processing}
\label{sec:commandline}

The effect of redirection files can also be achieved by invoking
the \LaTeX{} compiler with a more elaborate command line.
Most conveniently this should be done as part
of a shell script or a |Makefile|.

When using \textsf{childdoc} in the main file, the following
command lines effectively perform a redirection
(note that depending on the shell being used,
backslashes may have to be doubled: `|\|' $\to$ `|\\|'):
%
\begin{center}
|... -jobname "|\textit{target}|" |\\|"|[\textit{flags}]%
|\input{childdoc.def}\childdocforward[|\textit{main}|]{|\textit{dest}|}"|
\end{center}
%
Here \textit{target} is the name of the output file,
\textit{main} is the name of the main file
and \textit{dest} is the name of the main or child file to be processed
(all filenames without extensions).
The optional argument \textit{main} can be omitted
if \textit{main} matches \textit{dest}.
Optionally, compilation \textit{flags} can be defined via |\def| commands.
This command line makes the \TeX{} engine believe
it is compiling the file \textit{target}
whose content is specified as the latter parameter.
The provided code then forwards the processing to
\textit{main} or \textit{dest} as described in \secref{sec:forward}.

%%%%%%%%%%%%%%%%%%%%%%%%%%%%%%%%%%%%%%%%%%%%%%%%%%%%%%%%%%%%%%%%%%%%%%%%%%%%%%%%
\subsection{Include by Input}
\label{sec:input}

Including child documents by |\include| has some restrictions by design.
Most notably, the content of a child document always occupies
its own set of pages; pages cannot be shared between child documents.
Usually, this behaviour makes perfect sense
because each child document contain an essential part of the document.
However, in some situations it may be desirable to compose
a document from a collection of parts
without having mandatory page breaks between then.
For this case, the package
provides a mechanism to include parts
by |\input| which can also be processed individually.
However, by construction this mechanism
requires manual handling of the content to be output.

%%%%%%%%%%%%%%%%%%%%%%%%%%%%%%%%%%%%%%%%
\DescribeMacro{\ifchilddocmanual}
The main file should be prepared as usual, see \secref{sec:include}.
However, the document body must make a distinction
between processing of an individual part and of the main document, e.g.:
%
\begin{center}
\begin{tabular}{l}
|\ifchilddocmanual|\\
|\input{\childdocname}|\\
|\||else|\\
\textit{document body with }|\input{|\textit{part}|}|\\
|\||fi|
\end{tabular}
\end{center}
%
The conditional |\ifchilddocmanual| is true whenever
a part to be included by |\input| is being compiled,
and the name of the part is stored in |\childdocname|.

%%%%%%%%%%%%%%%%%%%%%%%%%%%%%%%%%%%%%%%%
\DescribeMacro{\childdocby}
Each part to be included by |\input| should start with:
%
\begin{center}
\begin{tabular}{l}
|\input{childdoc.def}|\\
|\childdocby{|\textit{main}|}|\\
\end{tabular}
\end{center}
%
The directive |\childdocby| is similar to |\childdocof|
described in \secref{sec:include},
but the subsequent selection of content must be done manually.
To that end, both |\ifchilddoc| and |\ifchilddocmanual|
will be true upon processing of a part,
and the name of the part is stored in |\childdocname|.
Note that |\jobname| will be set to the filename of the current part
so that each part receives an individual |.aux| file
that does not interfere with the |.aux| file(s) of the main document.
This behaviour can be altered by the alternative form
|\childdocby[*]{|\textit{main}|}| (with a non-empty optional argument)
which uses the |.aux| file of the main document
by setting |\jobname| to \textit{main}.

%%%%%%%%%%%%%%%%%%%%%%%%%%%%%%%%%%%%%%%%%%%%%%%%%%%%%%%%%%%%%%%%%%%%%%%%%%%%%%%%
\subsection{Driver Development}
\label{sec:driver}

The \textsf{childdoc} mechanism can also be use for the development
of definition files such as \LaTeX{} styles or classes.
This case differs from the above setup with multiple parts
included by |\include| in that no |\includeonly| should be invoked.
This can be achieved by starting the include file
(before |\ProvidesPackage|) with:
%
\begin{center}
\begin{tabular}{l}
|\input{childdoc.def}|\\
|\childdocforward{|\textit{main}|}|\\
\end{tabular}
\end{center}
%
or alternatively with:
%
\begin{center}
\begin{tabular}{l}
|\input{childdoc.def}|\\
|\childdocby{|\textit{main}|}|\\
\end{tabular}
\end{center}
%
Both forms have slightly different effects as described above.
The main file is prepared as usual, see \secref{sec:include}.

%%%%%%%%%%%%%%%%%%%%%%%%%%%%%%%%%%%%%%%%%%%%%%%%%%%%%%%%%%%%%%%%%%%%%%%%%%%%%%%%
\subsection{Legacy Detection}
\label{sec:detection}

The directive |\childdocmain| in the main file can detect
whether the complete document or merely a child is to be compiled
even without using the directive |\childdocof|.
This method is deprecated because it is less robust
and there is no compelling reason to use it;
it is merely provided for backward compatibility
and it may be removed in future versions.

If the detection mechanism is to be used,
it is mandatory to correctly specify
the filename of the main file as the argument of |\childdocmain|:
%
\begin{center}
\begin{tabular}{l}
|\input{childdoc.def}|\\
|\childdocmain{|\textit{main}|}|\\
\end{tabular}
\end{center}
%
If |\jobname| does not match the argument \textit{main} of |\childdocmain|,
it is assumed that |\jobname| points to the child file to be compiled.
When using |\childdocmain| with the main file specified as argument,
it suffices to start a child file
with just |\input{|\textit{main}|}|
without loading of the package and using |\childdocof|.
If instead all processing is done
with the appropriate \textsf{childdoc} directives,
the argument of \textit{main} of |\childdocmain| can be empty.

An alternative version of the command line processing described
in \secref{sec:commandline} using the detection mechanism reads:
%
\begin{center}
|... -jobname "|\textit{target}|" "|[\textit{flags}]%
[|\def\jobname{|\textit{dest}|}|]|\input{|\textit{main}|}"|
\end{center}

%%%%%%%%%%%%%%%%%%%%%%%%%%%%%%%%%%%%%%%%%%%%%%%%%%%%%%%%%%%%%%%%%%%%%%%%%%%%%%%%
\subsection{Manual Code}
\label{sec:manual}

In case one cannot be certain whether the definitions file |childdoc.def|
is installed on the target \TeX{} distribution
and one prefers not to ship it,
it is conceivable to paste a few relevant commands into the sources.

To that end, drop all statements |\input{childdoc.def}|
and perform the replacements as outlined below.
Instead of |\childdocmain{|\textit{main}|}| add the following code
to the top of the main file:
%
\begin{center}
\begin{tabular}{l}
|\||ifdefined\childdocname\endinput\||fi\newif\ifchilddoc|\\
|\edef\childdocname{\scantokens\expandafter{\jobname\noexpand}}|\\
|\def\childdocmain{|\textit{main}|}\||ifx\childdocmain\childdocname\||else|\\
|\childdoctrue\includeonly{\childdocname}\let\jobname\childdocmain\||fi|\\
\end{tabular}
\end{center}
%
Instead of |\childdocof{|\textit{main}|}| just include the main file
at the top of each child file:
%
\begin{center}
|\input{|\textit{main}|}|
\end{center}
%
A simple redirection |\childdocforward{|\textit{dest}|}| is achieved by:
%
\begin{center}
|\def\jobname{|\textit{dest}|}\input{\jobname}|
\end{center}
%
The redirection with prefix
|\childdocforwardprefix[|\textit{prefix}|]{|\textit{dest}|}|
is accomplished by:
%
\begin{center}
\begin{tabular}{l}
|{\edef\jobname{\scantokens\expandafter{\jobname\noexpand}}|\\
|\def\redirectjob |\textit{prefix}|#1~~~{\gdef\jobname{|\textit{dest}|#1}}|\\
|\expandafter\redirectjob\jobname~~~}\input{\jobname}|
\end{tabular}
\end{center}

In an alternative approach,
child documents can be compiled by a specific command line
without additional code or specific definitions:
%
\begin{center}
|... -jobname "|\textit{target}|" "|[\textit{flags}]%
|\includeonly{|\textit{dest}|}\input{|\textit{main}|}"|
\end{center}
%

%%%%%%%%%%%%%%%%%%%%%%%%%%%%%%%%%%%%%%%%%%%%%%%%%%%%%%%%%%%%%%%%%%%%%%%%%%%%%%%%
%%%%%%%%%%%%%%%%%%%%%%%%%%%%%%%%%%%%%%%%%%%%%%%%%%%%%%%%%%%%%%%%%%%%%%%%%%%%%%%%
\section{Information}

%%%%%%%%%%%%%%%%%%%%%%%%%%%%%%%%%%%%%%%%%%%%%%%%%%%%%%%%%%%%%%%%%%%%%%%%%%%%%%%%
\subsection{Copyright}

Copyright \copyright{} 2017--2018 Niklas Beisert

This work may be distributed and/or modified under the
conditions of the \LaTeX{} Project Public License, either version 1.3
of this license or (at your option) any later version.
The latest version of this license is in
  \url{http://www.latex-project.org/lppl.txt}
and version 1.3 or later is part of all distributions of \LaTeX{}
version 2005/12/01 or later.

This work has the LPPL maintenance status `maintained'.

The Current Maintainer of this work is Niklas Beisert.

This work consists of the files |README.txt|, |childdoc.ins| and |childdoc.dtx|
as well as the derived files |childdoc.def|, |cdocsamp.tex|
with |cdocsch1.tex|, |cdocsch2.tex|, |cdocspt3.tex|, |cdocspt4.tex|,
|cdocsdrf.tex|, |cdocsfn1.tex|, |cdocsfn2.tex|
as well as |childdoc.pdf|.

%%%%%%%%%%%%%%%%%%%%%%%%%%%%%%%%%%%%%%%%%%%%%%%%%%%%%%%%%%%%%%%%%%%%%%%%%%%%%%%%
\subsection{Files and Installation}

The package consists of the files:
%
\begin{center}
\begin{tabular}{ll}
    |README.txt|   & readme file \\
    |childdoc.ins| & installation file \\
    |childdoc.dtx| & source file \\
    |childdoc.def| & definition file \\
    |cdocsamp.tex| & sample main file \\
    |cdocsch1.tex| & sample include file \\
    |cdocsch2.tex| & sample include file \\
    |cdocspt3.tex| & sample part file \\
    |cdocspt4.tex| & sample part file \\
    |cdocsdrf.tex| & sample redirection file \\
    |cdocsfn1.tex| & sample redirection file \\
    |cdocsfn2.tex| & sample redirection file \\
    |childdoc.pdf| & manual
\end{tabular}
\end{center}
%
The distribution consists of the files
|README.txt|, |childdoc.ins| and |childdoc.dtx|.
%
\begin{itemize}
\item
Run (pdf)\LaTeX{} on |childdoc.dtx|
to compile the manual |childdoc.pdf| (this file).
\item
Run \LaTeX{} on |childdoc.ins| to create the definitions file |childdoc.def|
and the sample |cdocsamp.tex| with include files
|cdocsch1.tex|, |cdocsch2.tex|, |cdocspt3.tex|, |cdocspt4.tex|,
|cdocsdrf.tex|, |cdocsfn1.tex|, |cdocsfn2.tex|.
Then copy the file |childdoc.def| to an appropriate directory of your \LaTeX{}
distribution, e.g.\ \textit{texmf-root}|/tex/latex/childdoc|.
\end{itemize}

%%%%%%%%%%%%%%%%%%%%%%%%%%%%%%%%%%%%%%%%%%%%%%%%%%%%%%%%%%%%%%%%%%%%%%%%%%%%%%%%
\subsection{Related CTAN Packages}

There are several other packages which offer a similar functionality:
%
\begin{itemize}
\item
The packages
\href{http://ctan.org/pkg/docmute}{\textsf{docmute}},
\href{http://ctan.org/pkg/includex}{\textsf{includex}} and
\href{http://ctan.org/pkg/standalone}{\textsf{standalone}}
provide commands to include only the document body of
a child file thus allowing both files to be compiled individually.
\item
The packages \href{http://ctan.org/pkg/subdocs}{\textsf{subdocs}}
and \href{http://ctan.org/pkg/subfiles}{\textsf{subfiles}}
provide structures in which the main and child documents can be
encapsulated and allowing them to be compiled individually.
The inclusion mechanism is different from the conventional |\include|.
\item
The package \href{http://ctan.org/pkg/combine}{\textsf{combine}}
is an elaborate solution to combine several documents into one.
\end{itemize}
%
See also the CTAN topic \href{http://ctan.org/topic/subdocs}{\textsf{subdocs}}
for further related packages.
The present package differs from the above solutions in that
a document structure constructed with the conventional |\include| mechanism
just needs two extra commands at the top of every file
such that all constituent files can be compiled individually.

%%%%%%%%%%%%%%%%%%%%%%%%%%%%%%%%%%%%%%%%%%%%%%%%%%%%%%%%%%%%%%%%%%%%%%%%%%%%%%%%
%\subsection{Feature Suggestions}
%
%The following is a list of features which may be useful for future
%versions of this package:
%%
%\begin{itemize}
%\item
%\ldots
%\end{itemize}

%%%%%%%%%%%%%%%%%%%%%%%%%%%%%%%%%%%%%%%%%%%%%%%%%%%%%%%%%%%%%%%%%%%%%%%%%%%%%%%%
\subsection{Revision History}

%%%%%%%%%%%%%%%%%%%%%%%%%%%%%%%%%%%%%%%%
\paragraph{v2.0:} 2018/12/30

\begin{itemize}
\item
immediate forward processing
\item
added |\childdocby| mechanism
\item
manual restructured
\end{itemize}

%%%%%%%%%%%%%%%%%%%%%%%%%%%%%%%%%%%%%%%%
\paragraph{v1.6:} 2018/01/17

\begin{itemize}
\item
application for development of include files
\item
corrections to manual
\end{itemize}

%%%%%%%%%%%%%%%%%%%%%%%%%%%%%%%%%%%%%%%%
\paragraph{v1.5:} 2017/05/21

\begin{itemize}
\item
more complete structuring introduced
\item
|\childdocof| introduced
\item
|\childdoc| renamed to |\childdocmain|
\item
|\childredirect| renamed to |\childdocforward| and |\childdocforwardprefix|
and functionality expanded
\end{itemize}

%%%%%%%%%%%%%%%%%%%%%%%%%%%%%%%%%%%%%%%%
\paragraph{v1.0:} 2017/04/27

\begin{itemize}
\item
manual and install package
\item
first version published on CTAN
\end{itemize}

%%%%%%%%%%%%%%%%%%%%%%%%%%%%%%%%%%%%%%%%
\paragraph{v0.6:} 2017/04/26

\begin{itemize}
\item
redirection mechanism added
\end{itemize}

%%%%%%%%%%%%%%%%%%%%%%%%%%%%%%%%%%%%%%%%
\paragraph{v0.5:} 2017/04/26

\begin{itemize}
\item
functionality in definition file
\end{itemize}


%%%%%%%%%%%%%%%%%%%%%%%%%%%%%%%%%%%%%%%%%%%%%%%%%%%%%%%%%%%%%%%%%%%%%%%%%%%%%%%%
%%%%%%%%%%%%%%%%%%%%%%%%%%%%%%%%%%%%%%%%%%%%%%%%%%%%%%%%%%%%%%%%%%%%%%%%%%%%%%%%
%%%%%%%%%%%%%%%%%%%%%%%%%%%%%%%%%%%%%%%%%%%%%%%%%%%%%%%%%%%%%%%%%%%%%%%%%%%%%%%%
\appendix

\settowidth\MacroIndent{\rmfamily\scriptsize 000\ }

 \DocInput{childdoc.dtx}

\end{document}
%</driver>
% \fi
%
% %%%%%%%%%%%%%%%%%%%%%%%%%%%%%%%%%%%%%%%%%%%%%%%%%%%%%%%%%%%%%%%%%%%%%%%%%%%%%%
% %%%%%%%%%%%%%%%%%%%%%%%%%%%%%%%%%%%%%%%%%%%%%%%%%%%%%%%%%%%%%%%%%%%%%%%%%%%%%%
% \section{Sample}
%\iffalse
%<*samplemain>
%\fi
%
% The following presents a sample document
% with two chapters, two parts, a title page,
% a compile flag as well as three forwarding files to set the flag.
% It consists of eight |.tex| files:
% \begin{center}
% \begin{tabular}{ll}
% |cdocsamp.tex|&main file\\
% |cdocsch1.tex|&include file for chapter 1\\
% |cdocsch2.tex|&include file for chapter 2\\
% |cdocspt3.tex|&include file for part 3\\
% |cdocspt4.tex|&include file for part 4\\
% |cdocsdrf.tex|&forwarding file for main file in draft mode\\
% |cdocsfi1.tex|&forwarding file for final version of chapter 1\\
% |cdocsfi2.tex|&forwarding file for final version of chapter 2\\
% \end{tabular}
% \end{center}
% Each of the eight files can be compiled directly by the \LaTeX{} compiler.
%
% %%%%%%%%%%%%%%%%%%%%%%%%%%%%%%%%%%%%%%
% \paragraph{Main File.}
%
% The main file is called |cdocsamp.tex|.
%
% Load the \textsf{childdoc} definitions and
% declare the filename for the main document:
%    \begin{macrocode}
\input{childdoc.def}
\childdocmain{}
%    \end{macrocode}

% Optional override for |\version| flag:
%    \begin{macrocode}
%%\ifchilddoc\else\providecommand{\version}{draft}\fi
%    \end{macrocode}

% Define the default values for the |\version| flag
% (|final| for the main file and |draft| for childs):
%    \begin{macrocode}
\ifchilddoc
\providecommand{\version}{draft}
\else
\providecommand{\version}{final}
\fi
%    \end{macrocode}

% Load the standard document class:
%    \begin{macrocode}
\documentclass[12pt]{article}
%    \end{macrocode}

% Start the document body:
%    \begin{macrocode}
\begin{document}
%    \end{macrocode}

% Declare a title page.
% Print title, part of document being processed and version flag:
%    \begin{macrocode}
\addtocounter{page}{-1}
\begin{center}
{\LARGE\bfseries{}childdoc example\par}
\vspace{1cm}
\ifchilddoc
\ifchilddocmanual part\else chapter\fi:
`\childdocname' of `\childdocjob'\par
\else
main document: `\childdocjob'\par
\fi
version: \version\par
\end{center}
\newpage
%    \end{macrocode}

% Manually include selected file,
% otherwise process as usual:
%    \begin{macrocode}
\ifchilddocmanual
\section*{part `\childdocname'}
\input{\childdocname}
\else
%    \end{macrocode}

% Include the two chapters:
%    \begin{macrocode}
\include{cdocsch1}
\include{cdocsch2}
%    \end{macrocode}

% Include the two parts unless only chapters should be displayed:
%    \begin{macrocode}
\ifchilddoc\else
\section{part three}
\input{cdocspt3}
\section{part four}
\input{cdocspt4}
\fi
%    \end{macrocode}

% Process as usual until here:
%    \begin{macrocode}
\fi
%    \end{macrocode}

% End of document body:
%    \begin{macrocode}
\end{document}
%    \end{macrocode}
%\iffalse
%</samplemain>
%\fi
%
% %%%%%%%%%%%%%%%%%%%%%%%%%%%%%%%%%%%%%%
% \paragraph{Chapter Include Files.}
%
% The include files are called |cdocsch1.tex| and |cdocsch2.tex|.
%
%\iffalse
%<*samplechap1|samplechap2>
%\fi

% Optional override for |\version| flag:
%    \begin{macrocode}
%%\providecommand{\version}{final}
%    \end{macrocode}

% Include the main document:
%    \begin{macrocode}
\input{childdoc.def}
\childdocof{cdocsamp}
%    \end{macrocode}

%\iffalse
%</samplechap1|samplechap2>
%\fi
%
%\iffalse
%<*samplechap1>
%\fi
% Some text for chapter 1:
%    \begin{macrocode}
\section{one}
some text in chapter one
%    \end{macrocode}

%\iffalse
%</samplechap1>
%\fi
% Some text for chapter 2:
%\iffalse
%<*samplechap2>
%\fi
%    \begin{macrocode}
\section{two}
more text in chapter two
%    \end{macrocode}

%\iffalse
%</samplechap2>
%\fi
%
% %%%%%%%%%%%%%%%%%%%%%%%%%%%%%%%%%%%%%%
% \paragraph{Part Include Files.}
%
% The include files are called |cdocspt3.tex| and |cdocspt4.tex|.
%
%\iffalse
%<*samplepart3|samplepart4>
%\fi

% Optional override for |\version| flag:
%    \begin{macrocode}
%%\providecommand{\version}{final}
%    \end{macrocode}

% Include the main document:
%    \begin{macrocode}
\input{childdoc.def}
\childdocby{cdocsamp}
%    \end{macrocode}

%\iffalse
%</samplepart3|samplepart4>
%\fi
%
%\iffalse
%<*samplepart3>
%\fi
% Some text for part 3:
%    \begin{macrocode}
some text in part three
%    \end{macrocode}

%\iffalse
%</samplepart3>
%\fi
% Some text for part 4:
%\iffalse
%<*samplepart4>
%\fi
%    \begin{macrocode}
more text in part four
%    \end{macrocode}

%\iffalse
%</samplepart4>
%\fi
%
% %%%%%%%%%%%%%%%%%%%%%%%%%%%%%%%%%%%%%%
% \paragraph{Forwarding for a Complete Draft.}
%
% The following forwarding file |cdocsdrf.tex|
% compiles the main document in draft mode:
%\iffalse
%<*sampledraft>
%\fi
%    \begin{macrocode}
\def\version{draft}
\input{childdoc.def}
\childdocforward{cdocsamp}
%    \end{macrocode}

%\iffalse
%</sampledraft>
%\fi
%
% %%%%%%%%%%%%%%%%%%%%%%%%%%%%%%%%%%%%%%
% \paragraph{Forwarding for Final Version of the Chapters.}
%
% The following forwarding files |cdocsfn1.tex| and |cdocsfn2.tex|
% (with identical content)
% compile the final versions of the child documents
% |cdocsch1.tex| and |cdocsch2.tex|, respectively:
%\iffalse
%<*samplefinal>
%\fi
%    \begin{macrocode}
\def\version{final}
\input{childdoc.def}
\childdocforwardprefix[cdocsamp]{cdocsfn}{cdocsch}
%    \end{macrocode}

%\iffalse
%</samplefinal>
%\fi
%
% %%%%%%%%%%%%%%%%%%%%%%%%%%%%%%%%%%%%%%
% \paragraph{Command Line Processing.}
%
% The following three command lines generate the output files
% |cdocscld|, |cdocscl1| and |cdocscl2|
% which should be identical to
% |cdocsdrf|, |cdocsch1| and |cdocsfn2|, respectively:
% \begin{center}
% \begin{tabular}{l}
% |latex -jobname cdocscld \|\\
% |  "\def\version{draft}\input{childdoc.def}\childdocforward{cdocsamp}"|\\
% |latex -jobname cdocscl1 \|\\
% |  "\input{childdoc.def}\childdocforward[cdocsamp]{cdocsch1}"|\\
% |latex -jobname cdocscl2 \|\\
% |  "\def\version{final}\input{childdoc.def}\childdocforward{cdocsch2}"|
% \end{tabular}
% \end{center}
% Note that the trailing backslash on each first line
% merely continues the input to the second line
% (for convenient cut ant paste).
% Furthermore, the command |latex| can be replaced by any
% of its alternative versions such as |pdflatex|.
%
% %%%%%%%%%%%%%%%%%%%%%%%%%%%%%%%%%%%%%%%%%%%%%%%%%%%%%%%%%%%%%%%%%%%%%%%%%%%%%%
% %%%%%%%%%%%%%%%%%%%%%%%%%%%%%%%%%%%%%%%%%%%%%%%%%%%%%%%%%%%%%%%%%%%%%%%%%%%%%%
% \section{Implementation}
%\iffalse
%<*package>
%\fi
%
% This section describes the definitions file |childdoc.def|.

% The definitions cannot be loaded using |\usepackage| or |\RequirePackage|
% which has a mechanism to prevent loading a style file more than once.
% When loading the definitions by means of |\input|
% multiple instances have to be prevented manually:
%\iffalse
%This code needs to be before the `\ProvidesFile' directive
%which is defined at the beginning of this file.
%Therefore it is also placed there and commented out here.
%</package>
%<*discard>
%\fi
%    \begin{macrocode}
\ifdefined\childdocmain\endinput\fi
%    \end{macrocode}
%\iffalse
%</discard>
%<*package>
%\fi
%
% \macro{\ifchilddoc}
% \macro{\ifchilddocmanual}
% The conditional |\ifchilddoc| tells whether a
% child (true) or main (false) document is being compiled.
% The conditional |\ifchilddocmanual| tells whether
% the |\includeonly| mechanism is used (false) or
% the selection of child files must be performed manually (true).
% The definitions initialise to false:
%    \begin{macrocode}
\newif\ifchilddoc
\newif\ifchilddocmanual
%    \end{macrocode}

% \macro{\childdocname}
% \macro{\childdocjob}
% The macro |\childdocname| stores the name of the main document
% to be compiled. The macro |\childdocjob| stores the name of
% the document on which the \LaTeX{} compiler was originally invoked.
% The content of |\jobname| cannot be compared
% to filenames specified in the source due to different catcodes.
% The following code rescans |\jobname|, stores the result
% in |\childdocname| and saves a copy in |\childdocjob|:
%    \begin{macrocode}
\edef\childdocname{\scantokens\expandafter{\jobname\noexpand}}
\let\childdocjob\childdocname
%    \end{macrocode}

% \macro{\childdocdisable}
% The macro |\childdocdisable| prevents the main file
% from being processed more than once.
% At this stage, the main document command |\childdocmain|
% is assumed to be called once again where it should do nothing.
% Any subsequent call to it should prevent
% a secondary processing of the main document
% It overwrites the forwarding commands
% |\childdocof| and |\childdocforward|
% with empty macros to prevent further inclusions of the main document:
%    \begin{macrocode}
\newcommand{\childdocdisable}
{
  \renewcommand{\childdocmain}[1]{\renewcommand{\childdocmain}[1]{\endinput}}
  \renewcommand{\childdocof}[1]{}
  \renewcommand{\childdocby}[2][]{}
  \renewcommand{\childdocforward}[2][]{}
  \renewcommand{\childdocdisable}{}
}
%    \end{macrocode}

% \macro{\childdocmain}
% The macro |\childdocmain| is to be called at the top of the main file
% with nothing or the main filename (without extension) as argument.
% First, it breaks loops.
% If the argument is not empty and does not match |\childdocname|
% (which is set by the first inclusion of |childdoc.def|),
% |\ifchilddoc| is set to true, |\includeonly| is applied to the child file
% and |\jobname| is set to the main file
% (for proper handling of |.aux| files):
%    \begin{macrocode}
\newcommand{\childdocmain}[1]
{
  \childdocdisable\childdocmain{}
  \if?#1?\else
    \begingroup
      \def\childdoctmp{#1}
      \ifx\childdoctmp\childdocname
        \def\childdoctmp{}
      \else
        \def\childdoctmp
        {
          \childdoctrue
          \includeonly{\childdocname}
          \def\childdocjob{#1}
          \def\jobname{#1}
        }
      \fi
      \expandafter
    \endgroup
    \childdoctmp
  \fi
}
%    \end{macrocode}

% \macro{\childdocof}
% The command |\childdocof| redirects
% compilation to the main file |#1|.
%    \begin{macrocode}
\newcommand{\childdocof}[1]
{
  \childdocdisable
  \childdoctrue
  \includeonly{\childdocname}
  \def\jobname{#1}
  \def\childdocjob{#1}
  \input{#1}
}
%    \end{macrocode}

% \macro{\childdocby}
% The command |\childdocby| ....
%    \begin{macrocode}
\newcommand{\childdocby}[2][]
{
  \childdocdisable
  \childdoctrue
  \childdocmanualtrue
  \if?#1?\else
    \def\jobname{#2}
  \fi
  \def\childdocjob{#2}
  \input{#2}
  \endinput
}
%    \end{macrocode}

% \macro{\childdocforward}
% The command |\childdocforward| redirects
% compilation to the main file or
% (if the optional argument is given) a child file.
% Parameters are set as if the main file
% or a child file starting with |\childdocof| was compiled.
% Then compilation is handed over to the main file:
%    \begin{macrocode}
\newcommand{\childdocforward}[2][]
{
  \begingroup
    \if?#1?
      \def\childdoctmp
      {
        \def\childdocname{#2}
        \def\childdocjob{#2}
        \def\jobname{#2}
        \input{#2}
        \endinput
      }
    \else
      \def\childdoctmp
      {
        \childdocdisable
        \def\childdocname{#2}
        \childdoctrue
        \includeonly{#2}
        \def\childdocjob{#1}
        \def\jobname{#1}
        \input{#1}
        \endinput
      }
    \fi
    \expandafter
  \endgroup
  \childdoctmp
}
%    \end{macrocode}

% \macro{\childdocforwardprefix}
% The command |\childdocforwardprefix| redirects
% compilation to the main or a child file by means of a pattern.
% The prefix |#1| in the current filename is replaced by |#2|
% and the suffix of the current filename is kept
% (it is assumed that the filename does not contain the substring `|~~~|'
% which is used as a delimiter).
% Compilation is handed over to the new file by |\childdocforward|:
%    \begin{macrocode}
\newcommand{\childdocforwardprefix}[3][]
{
  \begingroup
    \def\childdocextract #2##1~~~{\def\childdoctmp{\childdocforward[#1]{#3##1}}}
    \expandafter\childdocextract\childdocname~~~
    \expandafter
  \endgroup
  \childdoctmp
}
%    \end{macrocode}

% \macro{\childdoc}
% The deprecated macro |\childdoc| is a legacy version of |\childdocmain|:
%    \begin{macrocode}
\newcommand{\childdoc}{\childdocmain}
%    \end{macrocode}

% \macro{\childdocredirect}
% The deprecated macro |\childdocredirect| is a legacy version
% of |\childdocforward| and |\childdocforwardprefix|:
%    \begin{macrocode}
\newcommand{\childdocredirect}[2][]
{
  \begingroup
    \if?#1?
      \def\childdoctmp{\childdocforward{#2}}
    \else
      \def\childdoctmp{\childdocforwardprefix{#1}{#2}}
    \fi
    \expandafter
  \endgroup
  \childdoctmp
}
%    \end{macrocode}

%\iffalse
%</package>
%\fi
%
\endinput
\childdocforward[|\textit{main}|]{|\textit{dest}|}"|
\end{center}
%
Here \textit{target} is the name of the output file,
\textit{main} is the name of the main file
and \textit{dest} is the name of the main or child file to be processed
(all filenames without extensions).
The optional argument \textit{main} can be omitted
if \textit{main} matches \textit{dest}.
Optionally, compilation \textit{flags} can be defined via |\def| commands.
This command line makes the \TeX{} engine believe
it is compiling the file \textit{target}
whose content is specified as the latter parameter.
The provided code then forwards the processing to
\textit{main} or \textit{dest} as described in \secref{sec:forward}.

%%%%%%%%%%%%%%%%%%%%%%%%%%%%%%%%%%%%%%%%%%%%%%%%%%%%%%%%%%%%%%%%%%%%%%%%%%%%%%%%
\subsection{Include by Input}
\label{sec:input}

Including child documents by |\include| has some restrictions by design.
Most notably, the content of a child document always occupies
its own set of pages; pages cannot be shared between child documents.
Usually, this behaviour makes perfect sense
because each child document contain an essential part of the document.
However, in some situations it may be desirable to compose
a document from a collection of parts
without having mandatory page breaks between then.
For this case, the package
provides a mechanism to include parts
by |\input| which can also be processed individually.
However, by construction this mechanism
requires manual handling of the content to be output.

%%%%%%%%%%%%%%%%%%%%%%%%%%%%%%%%%%%%%%%%
\DescribeMacro{\ifchilddocmanual}
The main file should be prepared as usual, see \secref{sec:include}.
However, the document body must make a distinction
between processing of an individual part and of the main document, e.g.:
%
\begin{center}
\begin{tabular}{l}
|\ifchilddocmanual|\\
|\input{\childdocname}|\\
|\||else|\\
\textit{document body with }|\input{|\textit{part}|}|\\
|\||fi|
\end{tabular}
\end{center}
%
The conditional |\ifchilddocmanual| is true whenever
a part to be included by |\input| is being compiled,
and the name of the part is stored in |\childdocname|.

%%%%%%%%%%%%%%%%%%%%%%%%%%%%%%%%%%%%%%%%
\DescribeMacro{\childdocby}
Each part to be included by |\input| should start with:
%
\begin{center}
\begin{tabular}{l}
|% \iffalse
%
% childdoc.dtx Copyright (C) 2017-2018 Niklas Beisert
%
% This work may be distributed and/or modified under the
% conditions of the LaTeX Project Public License, either version 1.3
% of this license or (at your option) any later version.
% The latest version of this license is in
%   http://www.latex-project.org/lppl.txt
% and version 1.3 or later is part of all distributions of LaTeX
% version 2005/12/01 or later.
%
% This work has the LPPL maintenance status `maintained'.
%
% The Current Maintainer of this work is Niklas Beisert.
%
% This work consists of the files childdoc.dtx and childdoc.ins
% and the derived files childdoc.def and cdocsamp.tex with
% cdocsch1.tex, cdocsch2.tex, cdocsdrf.tex, cdocsfn1.tex, cdocsfn2.tex.
%
%<package>\ifdefined\childdocmain\endinput\fi
%<package>\ProvidesFile{childdoc.def}[2018/12/30 v2.0 child document driver]
%<samplemain>\ProvidesFile{cdocsamp.tex}[2018/12/30 v2.0 sample for childdoc]
%<*driver>
%\ProvidesFile{childdoc.drv}[2018/12/30 v2.0 childdoc reference manual file]
\PassOptionsToClass{10pt,a4paper}{article}
\documentclass{ltxdoc}

\usepackage[margin=35mm]{geometry}
\usepackage{hyperref}
\usepackage{hyperxmp}
\usepackage[usenames]{color}

\hypersetup{colorlinks=true}
\hypersetup{pdfstartview=FitH}
\hypersetup{pdfpagemode=UseNone}
\hypersetup{pdfsource={}}
\hypersetup{pdflang={en-UK}}
\hypersetup{pdfcopyright={Copyright 2017-2018 Niklas Beisert.
  This work may be distributed and/or modified under the
  conditions of the LaTeX Project Public License, either version 1.3
  of this license or (at your option) any later version.}}
\hypersetup{pdflicenseurl={http://www.latex-project.org/lppl.txt}}
\hypersetup{pdfcontactaddress={ETH Zurich, ITP, HIT K,
  Wolfgang-Pauli-Strasse 27}}
\hypersetup{pdfcontactpostcode={8093}}
\hypersetup{pdfcontactcity={Zurich}}
\hypersetup{pdfcontactcountry={Switzerland}}
\hypersetup{pdfcontactemail={nbeisert@itp.phys.ethz.ch}}
\hypersetup{pdfcontacturl={http://people.phys.ethz.ch/\xmptilde nbeisert/}}

\newcommand{\secref}[1]{\hyperref[#1]{section \ref*{#1}}}

\parskip1ex
\parindent0pt
\let\olditemize\itemize
\def\itemize{\olditemize\parskip0pt}

\begin{document}

\title{The \textsf{childdoc} Package}
\hypersetup{pdftitle={The childdoc Package}}
\author{Niklas Beisert\\[2ex]
  Institut f\"ur Theoretische Physik\\
  Eidgen\"ossische Technische Hochschule Z\"urich\\
  Wolfgang-Pauli-Strasse 27, 8093 Z\"urich, Switzerland\\[1ex]
  \href{mailto:nbeisert@itp.phys.ethz.ch}
  {\texttt{nbeisert@itp.phys.ethz.ch}}}
\hypersetup{pdfauthor={Niklas Beisert}}
\hypersetup{pdfsubject={Manual for the LaTeX2e Package childdoc}}
\date{30 December 2018, \textsf{v2.0}}
\maketitle

\begin{abstract}\noindent
\textsf{childdoc} is a \LaTeXe{} package
that enables the direct compilation
of document sections included by |\include|
to individual files.
\end{abstract}

\begingroup
\parskip0ex
\tableofcontents
\endgroup

%%%%%%%%%%%%%%%%%%%%%%%%%%%%%%%%%%%%%%%%%%%%%%%%%%%%%%%%%%%%%%%%%%%%%%%%%%%%%%%%
%%%%%%%%%%%%%%%%%%%%%%%%%%%%%%%%%%%%%%%%%%%%%%%%%%%%%%%%%%%%%%%%%%%%%%%%%%%%%%%%
\section{Introduction}

\LaTeX{} provides a mechanism to structure a large document (such as a book)
into a main file and several child files (containing the chapters)
using the |\include| command.
This mechanism is beneficial for documents
which span hundreds of pages in order to
make the source file(s) more manageable.
Moreover, compilation can be restricted to
selected child files by means of the |\includeonly| command.
The latter feature can be used to reduce the compilation time while editing
(this was significantly more useful in the earlier days of \LaTeX{})
or to generate a smaller document which is easier to navigate.
Another application of |\includeonly| is to generate
documents consisting of selected parts of the complete document.

However, there are a few drawbacks of the plain |\include| mechanism:
\begin{itemize}
\item
The child files cannot be compiled on their own,
they can only be compiled via the main file.
A naive editing environment
(such as a text editor with an option
to have the current file processed by \LaTeX)
may require one to switch to the main file before compiling;
attempting to compile the child file produces errors.
\item
The main file must be modified (each time)
to adjust the |\includeonly| command
to the present needs. This easily leaves the main file in a messy state.
\item
The generated document will always carry the filename
of the main document. This is inconvenient if
several child files are to be compiled and
to be kept for distribution.
\end{itemize}

The present package provides a simple interface
to make child files individually compilable by \LaTeX{}.
Compiling a child file then has the same effect as compiling
the main file with an |\includeonly| command
to select the appropriate child.
Moreover the generated document will carry the name of the child
rather than the main file.
This resolves all three above issues.

This feature is meant to make the editing of books,
thesis documents and lecture notes somewhat more convenient.
However, the package can also be used efficiently for
composing a series of documents (such as exercise sheets)
which are typically distributed individually.
It then assists the author in generating the individual documents
(potentially in different versions)
as well as a document containing the collected series.
Another application is in developing style files
or other kinds of included material
where compilation of the style file could redirect
to a sample or test file.

%%%%%%%%%%%%%%%%%%%%%%%%%%%%%%%%%%%%%%%%%%%%%%%%%%%%%%%%%%%%%%%%%%%%%%%%%%%%%%%%
%%%%%%%%%%%%%%%%%%%%%%%%%%%%%%%%%%%%%%%%%%%%%%%%%%%%%%%%%%%%%%%%%%%%%%%%%%%%%%%%
\section{Usage}

First of all, the package \textsf{childdoc} is \emph{not} a standard
\LaTeXe{} |.sty| style file! Therefore it needs to be invoked in
a non-standard way.

%%%%%%%%%%%%%%%%%%%%%%%%%%%%%%%%%%%%%%%%%%%%%%%%%%%%%%%%%%%%%%%%%%%%%%%%%%%%%%%%
\subsection{Included Files}
\label{sec:include}

%%%%%%%%%%%%%%%%%%%%%%%%%%%%%%%%%%%%%%%%
\DescribeMacro{\childdocmain}
To use the package, add the commands
\begin{center}
\begin{tabular}{l}
|\input{childdoc.def}|\\
|\childdocmain{}|\\
\end{tabular}
\end{center}
at the very top of the main \LaTeX{} file,
in particular \emph{before} the |\documentclass| statement!
The argument of |\childdocmain| should be left empty
(but it must be present).

%%%%%%%%%%%%%%%%%%%%%%%%%%%%%%%%%%%%%%%%
\DescribeMacro{\childdocof}
Furthermore, add the commands
\begin{center}
\begin{tabular}{l}
|\input{childdoc.def}|\\
|\childdocof{|\textit{main}|}|\\
\end{tabular}
\end{center}
at the top of every child file \textit{child}
which is included by |\include{|\textit{child}|}|
from within the main file
(or at least for those files to be compiled individually).
The argument \textit{main} must be the filename of the main file.

There are a couple of
considerations in setting up the main and child documents:

%%%%%%%%%%%%%%%%%%%%%%%%%%%%%%%%%%%%%%%%
\paragraph{Restrictions.}

Please note the following restrictions:
\begin{itemize}
\item
|\childdocmain| must be called with one argument \textit{main}
to ensure compatibility with earlier version of the package.
It must either be empty (|\childdocmain{}|)
or precisely match the filename of the main file in which it is specified.
See \secref{sec:detection} for further information.
\item
The filename \textit{main} must be specified without the |.tex| extension.
\item
The filename \textit{main} is case sensitive
(even in case-insensitive file systems)
due to internal string comparison.
\item
The argument \textit{main} should be fully expanded, it cannot be a macro.
\item
Subdirectories and special characters should be avoided in filenames.
\item
The command |\childdocmain{|\textit{main}|}| must be followed by a whitespace.
It should not be followed immediately by another command
or by a comment mark `|%|'.
This is because the \TeX{} parser reads the token immediately following
the argument of |\childdocmain| and puts it
at the beginning of every child section;
however, a white\-space is ignored.
\end{itemize}

%%%%%%%%%%%%%%%%%%%%%%%%%%%%%%%%%%%%%%%%
\paragraph{Content of Main File.}

It is advisable to place all content in the child files included by |\include|.
Any output contained in the main file will appear in all child documents
unless suppressed manually;
it cannot be suppressed automatically by the |\includeonly| directive
and thus should normally be avoided.
A method to include some content in the main file
by means of conditional processing is described in \secref{sec:conditional}.

%%%%%%%%%%%%%%%%%%%%%%%%%%%%%%%%%%%%%%%%
\paragraph{Page Numbering.}

When only a part of the document is compiled,
the appropriate numbering of pages
(as well as other status parameters)
is determined from the |.aux| files.
The latter contain information from previous passes.
However this information needs to propagate through
all intermediate child documents.
Therefore the page numbering in child documents may well
be inconsistent until the complete document is compiled at least once.

A useful (if unconventional) way to always ensure a consistent
page numbering is to restart the numbering in each child document
and denote the pages by `\textit{child}|.|\textit{page}'
where \textit{child} represents the chapter/section number of the child file.
This can be achieved by the command
|\numberwithin{page}{|\textit{child}|}|
of the \textsf{amsmath} package
where \textit{child} can be |chapter| or |section|
depending on the chosen structuring.
Alternatively, one can modify the macro |\thepage| appropriately
and reset the counter |page| at the start of each child file.

%%%%%%%%%%%%%%%%%%%%%%%%%%%%%%%%%%%%%%%%%%%%%%%%%%%%%%%%%%%%%%%%%%%%%%%%%%%%%%%%
\subsection{Conditional Processing}
\label{sec:conditional}

The package provides a mechanism to compile different versions
of a document. To customise the versions further some conditional processing
can come in handy to distinguish which version is being compiled.
The package provides two macros to describe the compilation context:

%%%%%%%%%%%%%%%%%%%%%%%%%%%%%%%%%%%%%%%%
\DescribeMacro{\ifchilddoc}
The conditional |\ifchilddoc| distinguishes between the compilation of
child documents and the main document:
%
\begin{center}
|\ifchilddoc |\textit{child-code}| |[|\||else |\textit{main-code}]| \||fi|
\end{center}

%%%%%%%%%%%%%%%%%%%%%%%%%%%%%%%%%%%%%%%%
\DescribeMacro{\childdocname}
\DescribeMacro{\childdocjob}
The macro |\childdocname| contains the filename (without extension)
of the main or child file being processed.
Note that |\childdocjob| will always contain the name of the main file.

%%%%%%%%%%%%%%%%%%%%%%%%%%%%%%%%%%%%%%%%
\paragraph{Title Page.}

Conditional processing can be used to include a title or banner page
in the main document when proper precautions are taken.
Importantly, the code in the main file should ensure that the page counter
(as well as other status parameters which are stored in the |.aux| files)
takes the same value after the conditional processing.
Otherwise the page numbers may take divergent values
depending on which part is compiled.

For example, a title page could be declared by:
%
\begin{center}
\begin{tabular}{l}
|\ifchilddoc\||else|\\
|\addtocounter{page}{-1}|\\
\textit{code for title page}\\
|\newpage|\\
|\||fi|
\end{tabular}
\end{center}
%
A banner page for the child documents can be generated by:
%
\begin{center}
\begin{tabular}{l}
|\ifchilddoc|\\
|\addtocounter{page}{-1}|\\
\textit{code for banner page}\\
|\newpage|\\
|\||fi|
\end{tabular}
\end{center}
%
Here one could write a message such as:
\begin{center}
|This is the part \childdocname{} of \childdocjob{}.|
\end{center}

%%%%%%%%%%%%%%%%%%%%%%%%%%%%%%%%%%%%%%%%%%%%%%%%%%%%%%%%%%%%%%%%%%%%%%%%%%%%%%%%
\subsection{Flags}
\label{sec:flags}

The package makes it easy to generate different versions
of the main or child documents.
To this end compilation flags can be defined
and assigned different default values.
They will be particularly useful in conjunction
with the forwarding mechanism described in \secref{sec:forward}.

For example, it may be useful to have a flag |\version|
which can be set to |draft| or |final|.
The document source will contain some conditional code
depending on the value of |\version|.
Suppose further, the flag should default to |final| for the main file
and to |draft| for child files
which is a natural assignment for editing the document.
This is achieved by placing the following code
in the preamble of the main document
(below the |\childdocmain| directive):
%
\begin{center}
\begin{tabular}{l}
|\ifchilddoc|\\
|\providecommand{\version}{draft}|\\
|\||else|\\
|\providecommand{\version}{final}|\\
|\||fi|
\end{tabular}
\end{center}
%
The definition by |\providecommand| makes sure
that previous definitions are not overwritten.
Further statements |\providecommand{\version}{...}|
can thus be added before the above code to override it.

For the main file, one might add a line
(between |\childdocmain| and the above block)
%
\begin{center}
|%\ifchilddoc\||else\providecommand{\version}{draft}\||fi|
\end{center}
%
which can be uncommented to produce a draft version.
Likewise one can add a line to the very top of a child file
(above the |\childdocof{|\textit{main}|}| directive)
%
\begin{center}
|%\providecommand{\version}{final}|
\end{center}
%
which can be uncommented to produce the final version of this child document.

%%%%%%%%%%%%%%%%%%%%%%%%%%%%%%%%%%%%%%%%%%%%%%%%%%%%%%%%%%%%%%%%%%%%%%%%%%%%%%%%
\subsection{Forwarding}
\label{sec:forward}

Different versions of the main or child documents
using compilation flags as described in \secref{sec:flags}
can be (permanently) stored in different files
for convenient compilation, viewing and distribution.
To this end, the package defines a command
to pass on compilation to a different file:

%%%%%%%%%%%%%%%%%%%%%%%%%%%%%%%%%%%%%%%%
\DescribeMacro{\childdocforward}
The command |\childdocforward| redirects processing to
another source file:
%
\begin{center}
\begin{tabular}{l}
|\input{childdoc.def}|\\
|\childdocforward[|\textit{main}|]{|\textit{dest}|}|\\
\end{tabular}
\end{center}
%
The argument \textit{dest} is the destination file
(without extension).
It should be the main file or one of the child files.
Note that further \textsf{childdoc} directives
such as |\childdocof| and |\childdocforward|
in the indicated file will be processed in this form.
The optional argument \textit{main}
passes on directly to the main file \textit{main}
while pretending to compile the child \textit{dest}.
This form behaves as if \textit{dest}
issues |\childdocof{|\textit{main}|}| right away,
and no further \textsf{childdoc} directives will be processed.

%%%%%%%%%%%%%%%%%%%%%%%%%%%%%%%%%%%%%%%%
\DescribeMacro{\...prefix}
In the alternative form |\childdocforwardprefix|,
%
\begin{center}
\begin{tabular}{l}
|\input{childdoc.def}|\\
|\childdocforwardprefix[|\textit{main}|]{|\textit{prefix}|}{|\textit{dest}|}|
\end{tabular}
\end{center}
%
the destination file is determined by a pattern
depending on the current file:
To make this work, the current file must be called
`{\textit{prefix}\hspace{0.2em}\textit{suffix}}'
with \textit{prefix} matching precisely the argument.
Processing is then passed on to the file
`{\textit{dest}\hspace{0.2em}\textit{suffix}}'.
Surely, the same effect is achieved by
directly specifying the
argument `{\textit{dest}\hspace{0.2em}\textit{suffix}}'
in the first form.
However, that requires to set up a different file
for each child. With the alternative form of the command
all these files can have exactly the same content
which simplifies setting them up and maintaining them.

For example, the following file |draft.tex|
with a compilation flag |\version| as described in \secref{sec:flags}
compiles the main document as a draft:
%
\begin{center}
\begin{tabular}{l}
|\def\version{draft}|\\
|\input{childdoc.def}|\\
|\childdocforward{|\textit{main}|}|
\end{tabular}
\end{center}
%
Likewise, the following files |final|\textit{nn}|.tex|
compile the final version of the child document
|child|\textit{nn}|.tex|:
%
\begin{center}
\begin{tabular}{l}
|\def\version{final}|\\
|\input{childdoc.def}|\\
|\childdocforwardprefix{final}{child}|
\end{tabular}
\end{center}
%

Note that when several versions of a main file and/or of each child file
are to be generated, it may be convenient to set up a |Makefile| or
shell script to automatise the process.

%%%%%%%%%%%%%%%%%%%%%%%%%%%%%%%%%%%%%%%%%%%%%%%%%%%%%%%%%%%%%%%%%%%%%%%%%%%%%%%%
\subsection{Command Line Processing}
\label{sec:commandline}

The effect of redirection files can also be achieved by invoking
the \LaTeX{} compiler with a more elaborate command line.
Most conveniently this should be done as part
of a shell script or a |Makefile|.

When using \textsf{childdoc} in the main file, the following
command lines effectively perform a redirection
(note that depending on the shell being used,
backslashes may have to be doubled: `|\|' $\to$ `|\\|'):
%
\begin{center}
|... -jobname "|\textit{target}|" |\\|"|[\textit{flags}]%
|\input{childdoc.def}\childdocforward[|\textit{main}|]{|\textit{dest}|}"|
\end{center}
%
Here \textit{target} is the name of the output file,
\textit{main} is the name of the main file
and \textit{dest} is the name of the main or child file to be processed
(all filenames without extensions).
The optional argument \textit{main} can be omitted
if \textit{main} matches \textit{dest}.
Optionally, compilation \textit{flags} can be defined via |\def| commands.
This command line makes the \TeX{} engine believe
it is compiling the file \textit{target}
whose content is specified as the latter parameter.
The provided code then forwards the processing to
\textit{main} or \textit{dest} as described in \secref{sec:forward}.

%%%%%%%%%%%%%%%%%%%%%%%%%%%%%%%%%%%%%%%%%%%%%%%%%%%%%%%%%%%%%%%%%%%%%%%%%%%%%%%%
\subsection{Include by Input}
\label{sec:input}

Including child documents by |\include| has some restrictions by design.
Most notably, the content of a child document always occupies
its own set of pages; pages cannot be shared between child documents.
Usually, this behaviour makes perfect sense
because each child document contain an essential part of the document.
However, in some situations it may be desirable to compose
a document from a collection of parts
without having mandatory page breaks between then.
For this case, the package
provides a mechanism to include parts
by |\input| which can also be processed individually.
However, by construction this mechanism
requires manual handling of the content to be output.

%%%%%%%%%%%%%%%%%%%%%%%%%%%%%%%%%%%%%%%%
\DescribeMacro{\ifchilddocmanual}
The main file should be prepared as usual, see \secref{sec:include}.
However, the document body must make a distinction
between processing of an individual part and of the main document, e.g.:
%
\begin{center}
\begin{tabular}{l}
|\ifchilddocmanual|\\
|\input{\childdocname}|\\
|\||else|\\
\textit{document body with }|\input{|\textit{part}|}|\\
|\||fi|
\end{tabular}
\end{center}
%
The conditional |\ifchilddocmanual| is true whenever
a part to be included by |\input| is being compiled,
and the name of the part is stored in |\childdocname|.

%%%%%%%%%%%%%%%%%%%%%%%%%%%%%%%%%%%%%%%%
\DescribeMacro{\childdocby}
Each part to be included by |\input| should start with:
%
\begin{center}
\begin{tabular}{l}
|\input{childdoc.def}|\\
|\childdocby{|\textit{main}|}|\\
\end{tabular}
\end{center}
%
The directive |\childdocby| is similar to |\childdocof|
described in \secref{sec:include},
but the subsequent selection of content must be done manually.
To that end, both |\ifchilddoc| and |\ifchilddocmanual|
will be true upon processing of a part,
and the name of the part is stored in |\childdocname|.
Note that |\jobname| will be set to the filename of the current part
so that each part receives an individual |.aux| file
that does not interfere with the |.aux| file(s) of the main document.
This behaviour can be altered by the alternative form
|\childdocby[*]{|\textit{main}|}| (with a non-empty optional argument)
which uses the |.aux| file of the main document
by setting |\jobname| to \textit{main}.

%%%%%%%%%%%%%%%%%%%%%%%%%%%%%%%%%%%%%%%%%%%%%%%%%%%%%%%%%%%%%%%%%%%%%%%%%%%%%%%%
\subsection{Driver Development}
\label{sec:driver}

The \textsf{childdoc} mechanism can also be use for the development
of definition files such as \LaTeX{} styles or classes.
This case differs from the above setup with multiple parts
included by |\include| in that no |\includeonly| should be invoked.
This can be achieved by starting the include file
(before |\ProvidesPackage|) with:
%
\begin{center}
\begin{tabular}{l}
|\input{childdoc.def}|\\
|\childdocforward{|\textit{main}|}|\\
\end{tabular}
\end{center}
%
or alternatively with:
%
\begin{center}
\begin{tabular}{l}
|\input{childdoc.def}|\\
|\childdocby{|\textit{main}|}|\\
\end{tabular}
\end{center}
%
Both forms have slightly different effects as described above.
The main file is prepared as usual, see \secref{sec:include}.

%%%%%%%%%%%%%%%%%%%%%%%%%%%%%%%%%%%%%%%%%%%%%%%%%%%%%%%%%%%%%%%%%%%%%%%%%%%%%%%%
\subsection{Legacy Detection}
\label{sec:detection}

The directive |\childdocmain| in the main file can detect
whether the complete document or merely a child is to be compiled
even without using the directive |\childdocof|.
This method is deprecated because it is less robust
and there is no compelling reason to use it;
it is merely provided for backward compatibility
and it may be removed in future versions.

If the detection mechanism is to be used,
it is mandatory to correctly specify
the filename of the main file as the argument of |\childdocmain|:
%
\begin{center}
\begin{tabular}{l}
|\input{childdoc.def}|\\
|\childdocmain{|\textit{main}|}|\\
\end{tabular}
\end{center}
%
If |\jobname| does not match the argument \textit{main} of |\childdocmain|,
it is assumed that |\jobname| points to the child file to be compiled.
When using |\childdocmain| with the main file specified as argument,
it suffices to start a child file
with just |\input{|\textit{main}|}|
without loading of the package and using |\childdocof|.
If instead all processing is done
with the appropriate \textsf{childdoc} directives,
the argument of \textit{main} of |\childdocmain| can be empty.

An alternative version of the command line processing described
in \secref{sec:commandline} using the detection mechanism reads:
%
\begin{center}
|... -jobname "|\textit{target}|" "|[\textit{flags}]%
[|\def\jobname{|\textit{dest}|}|]|\input{|\textit{main}|}"|
\end{center}

%%%%%%%%%%%%%%%%%%%%%%%%%%%%%%%%%%%%%%%%%%%%%%%%%%%%%%%%%%%%%%%%%%%%%%%%%%%%%%%%
\subsection{Manual Code}
\label{sec:manual}

In case one cannot be certain whether the definitions file |childdoc.def|
is installed on the target \TeX{} distribution
and one prefers not to ship it,
it is conceivable to paste a few relevant commands into the sources.

To that end, drop all statements |\input{childdoc.def}|
and perform the replacements as outlined below.
Instead of |\childdocmain{|\textit{main}|}| add the following code
to the top of the main file:
%
\begin{center}
\begin{tabular}{l}
|\||ifdefined\childdocname\endinput\||fi\newif\ifchilddoc|\\
|\edef\childdocname{\scantokens\expandafter{\jobname\noexpand}}|\\
|\def\childdocmain{|\textit{main}|}\||ifx\childdocmain\childdocname\||else|\\
|\childdoctrue\includeonly{\childdocname}\let\jobname\childdocmain\||fi|\\
\end{tabular}
\end{center}
%
Instead of |\childdocof{|\textit{main}|}| just include the main file
at the top of each child file:
%
\begin{center}
|\input{|\textit{main}|}|
\end{center}
%
A simple redirection |\childdocforward{|\textit{dest}|}| is achieved by:
%
\begin{center}
|\def\jobname{|\textit{dest}|}\input{\jobname}|
\end{center}
%
The redirection with prefix
|\childdocforwardprefix[|\textit{prefix}|]{|\textit{dest}|}|
is accomplished by:
%
\begin{center}
\begin{tabular}{l}
|{\edef\jobname{\scantokens\expandafter{\jobname\noexpand}}|\\
|\def\redirectjob |\textit{prefix}|#1~~~{\gdef\jobname{|\textit{dest}|#1}}|\\
|\expandafter\redirectjob\jobname~~~}\input{\jobname}|
\end{tabular}
\end{center}

In an alternative approach,
child documents can be compiled by a specific command line
without additional code or specific definitions:
%
\begin{center}
|... -jobname "|\textit{target}|" "|[\textit{flags}]%
|\includeonly{|\textit{dest}|}\input{|\textit{main}|}"|
\end{center}
%

%%%%%%%%%%%%%%%%%%%%%%%%%%%%%%%%%%%%%%%%%%%%%%%%%%%%%%%%%%%%%%%%%%%%%%%%%%%%%%%%
%%%%%%%%%%%%%%%%%%%%%%%%%%%%%%%%%%%%%%%%%%%%%%%%%%%%%%%%%%%%%%%%%%%%%%%%%%%%%%%%
\section{Information}

%%%%%%%%%%%%%%%%%%%%%%%%%%%%%%%%%%%%%%%%%%%%%%%%%%%%%%%%%%%%%%%%%%%%%%%%%%%%%%%%
\subsection{Copyright}

Copyright \copyright{} 2017--2018 Niklas Beisert

This work may be distributed and/or modified under the
conditions of the \LaTeX{} Project Public License, either version 1.3
of this license or (at your option) any later version.
The latest version of this license is in
  \url{http://www.latex-project.org/lppl.txt}
and version 1.3 or later is part of all distributions of \LaTeX{}
version 2005/12/01 or later.

This work has the LPPL maintenance status `maintained'.

The Current Maintainer of this work is Niklas Beisert.

This work consists of the files |README.txt|, |childdoc.ins| and |childdoc.dtx|
as well as the derived files |childdoc.def|, |cdocsamp.tex|
with |cdocsch1.tex|, |cdocsch2.tex|, |cdocspt3.tex|, |cdocspt4.tex|,
|cdocsdrf.tex|, |cdocsfn1.tex|, |cdocsfn2.tex|
as well as |childdoc.pdf|.

%%%%%%%%%%%%%%%%%%%%%%%%%%%%%%%%%%%%%%%%%%%%%%%%%%%%%%%%%%%%%%%%%%%%%%%%%%%%%%%%
\subsection{Files and Installation}

The package consists of the files:
%
\begin{center}
\begin{tabular}{ll}
    |README.txt|   & readme file \\
    |childdoc.ins| & installation file \\
    |childdoc.dtx| & source file \\
    |childdoc.def| & definition file \\
    |cdocsamp.tex| & sample main file \\
    |cdocsch1.tex| & sample include file \\
    |cdocsch2.tex| & sample include file \\
    |cdocspt3.tex| & sample part file \\
    |cdocspt4.tex| & sample part file \\
    |cdocsdrf.tex| & sample redirection file \\
    |cdocsfn1.tex| & sample redirection file \\
    |cdocsfn2.tex| & sample redirection file \\
    |childdoc.pdf| & manual
\end{tabular}
\end{center}
%
The distribution consists of the files
|README.txt|, |childdoc.ins| and |childdoc.dtx|.
%
\begin{itemize}
\item
Run (pdf)\LaTeX{} on |childdoc.dtx|
to compile the manual |childdoc.pdf| (this file).
\item
Run \LaTeX{} on |childdoc.ins| to create the definitions file |childdoc.def|
and the sample |cdocsamp.tex| with include files
|cdocsch1.tex|, |cdocsch2.tex|, |cdocspt3.tex|, |cdocspt4.tex|,
|cdocsdrf.tex|, |cdocsfn1.tex|, |cdocsfn2.tex|.
Then copy the file |childdoc.def| to an appropriate directory of your \LaTeX{}
distribution, e.g.\ \textit{texmf-root}|/tex/latex/childdoc|.
\end{itemize}

%%%%%%%%%%%%%%%%%%%%%%%%%%%%%%%%%%%%%%%%%%%%%%%%%%%%%%%%%%%%%%%%%%%%%%%%%%%%%%%%
\subsection{Related CTAN Packages}

There are several other packages which offer a similar functionality:
%
\begin{itemize}
\item
The packages
\href{http://ctan.org/pkg/docmute}{\textsf{docmute}},
\href{http://ctan.org/pkg/includex}{\textsf{includex}} and
\href{http://ctan.org/pkg/standalone}{\textsf{standalone}}
provide commands to include only the document body of
a child file thus allowing both files to be compiled individually.
\item
The packages \href{http://ctan.org/pkg/subdocs}{\textsf{subdocs}}
and \href{http://ctan.org/pkg/subfiles}{\textsf{subfiles}}
provide structures in which the main and child documents can be
encapsulated and allowing them to be compiled individually.
The inclusion mechanism is different from the conventional |\include|.
\item
The package \href{http://ctan.org/pkg/combine}{\textsf{combine}}
is an elaborate solution to combine several documents into one.
\end{itemize}
%
See also the CTAN topic \href{http://ctan.org/topic/subdocs}{\textsf{subdocs}}
for further related packages.
The present package differs from the above solutions in that
a document structure constructed with the conventional |\include| mechanism
just needs two extra commands at the top of every file
such that all constituent files can be compiled individually.

%%%%%%%%%%%%%%%%%%%%%%%%%%%%%%%%%%%%%%%%%%%%%%%%%%%%%%%%%%%%%%%%%%%%%%%%%%%%%%%%
%\subsection{Feature Suggestions}
%
%The following is a list of features which may be useful for future
%versions of this package:
%%
%\begin{itemize}
%\item
%\ldots
%\end{itemize}

%%%%%%%%%%%%%%%%%%%%%%%%%%%%%%%%%%%%%%%%%%%%%%%%%%%%%%%%%%%%%%%%%%%%%%%%%%%%%%%%
\subsection{Revision History}

%%%%%%%%%%%%%%%%%%%%%%%%%%%%%%%%%%%%%%%%
\paragraph{v2.0:} 2018/12/30

\begin{itemize}
\item
immediate forward processing
\item
added |\childdocby| mechanism
\item
manual restructured
\end{itemize}

%%%%%%%%%%%%%%%%%%%%%%%%%%%%%%%%%%%%%%%%
\paragraph{v1.6:} 2018/01/17

\begin{itemize}
\item
application for development of include files
\item
corrections to manual
\end{itemize}

%%%%%%%%%%%%%%%%%%%%%%%%%%%%%%%%%%%%%%%%
\paragraph{v1.5:} 2017/05/21

\begin{itemize}
\item
more complete structuring introduced
\item
|\childdocof| introduced
\item
|\childdoc| renamed to |\childdocmain|
\item
|\childredirect| renamed to |\childdocforward| and |\childdocforwardprefix|
and functionality expanded
\end{itemize}

%%%%%%%%%%%%%%%%%%%%%%%%%%%%%%%%%%%%%%%%
\paragraph{v1.0:} 2017/04/27

\begin{itemize}
\item
manual and install package
\item
first version published on CTAN
\end{itemize}

%%%%%%%%%%%%%%%%%%%%%%%%%%%%%%%%%%%%%%%%
\paragraph{v0.6:} 2017/04/26

\begin{itemize}
\item
redirection mechanism added
\end{itemize}

%%%%%%%%%%%%%%%%%%%%%%%%%%%%%%%%%%%%%%%%
\paragraph{v0.5:} 2017/04/26

\begin{itemize}
\item
functionality in definition file
\end{itemize}


%%%%%%%%%%%%%%%%%%%%%%%%%%%%%%%%%%%%%%%%%%%%%%%%%%%%%%%%%%%%%%%%%%%%%%%%%%%%%%%%
%%%%%%%%%%%%%%%%%%%%%%%%%%%%%%%%%%%%%%%%%%%%%%%%%%%%%%%%%%%%%%%%%%%%%%%%%%%%%%%%
%%%%%%%%%%%%%%%%%%%%%%%%%%%%%%%%%%%%%%%%%%%%%%%%%%%%%%%%%%%%%%%%%%%%%%%%%%%%%%%%
\appendix

\settowidth\MacroIndent{\rmfamily\scriptsize 000\ }

 \DocInput{childdoc.dtx}

\end{document}
%</driver>
% \fi
%
% %%%%%%%%%%%%%%%%%%%%%%%%%%%%%%%%%%%%%%%%%%%%%%%%%%%%%%%%%%%%%%%%%%%%%%%%%%%%%%
% %%%%%%%%%%%%%%%%%%%%%%%%%%%%%%%%%%%%%%%%%%%%%%%%%%%%%%%%%%%%%%%%%%%%%%%%%%%%%%
% \section{Sample}
%\iffalse
%<*samplemain>
%\fi
%
% The following presents a sample document
% with two chapters, two parts, a title page,
% a compile flag as well as three forwarding files to set the flag.
% It consists of eight |.tex| files:
% \begin{center}
% \begin{tabular}{ll}
% |cdocsamp.tex|&main file\\
% |cdocsch1.tex|&include file for chapter 1\\
% |cdocsch2.tex|&include file for chapter 2\\
% |cdocspt3.tex|&include file for part 3\\
% |cdocspt4.tex|&include file for part 4\\
% |cdocsdrf.tex|&forwarding file for main file in draft mode\\
% |cdocsfi1.tex|&forwarding file for final version of chapter 1\\
% |cdocsfi2.tex|&forwarding file for final version of chapter 2\\
% \end{tabular}
% \end{center}
% Each of the eight files can be compiled directly by the \LaTeX{} compiler.
%
% %%%%%%%%%%%%%%%%%%%%%%%%%%%%%%%%%%%%%%
% \paragraph{Main File.}
%
% The main file is called |cdocsamp.tex|.
%
% Load the \textsf{childdoc} definitions and
% declare the filename for the main document:
%    \begin{macrocode}
\input{childdoc.def}
\childdocmain{}
%    \end{macrocode}

% Optional override for |\version| flag:
%    \begin{macrocode}
%%\ifchilddoc\else\providecommand{\version}{draft}\fi
%    \end{macrocode}

% Define the default values for the |\version| flag
% (|final| for the main file and |draft| for childs):
%    \begin{macrocode}
\ifchilddoc
\providecommand{\version}{draft}
\else
\providecommand{\version}{final}
\fi
%    \end{macrocode}

% Load the standard document class:
%    \begin{macrocode}
\documentclass[12pt]{article}
%    \end{macrocode}

% Start the document body:
%    \begin{macrocode}
\begin{document}
%    \end{macrocode}

% Declare a title page.
% Print title, part of document being processed and version flag:
%    \begin{macrocode}
\addtocounter{page}{-1}
\begin{center}
{\LARGE\bfseries{}childdoc example\par}
\vspace{1cm}
\ifchilddoc
\ifchilddocmanual part\else chapter\fi:
`\childdocname' of `\childdocjob'\par
\else
main document: `\childdocjob'\par
\fi
version: \version\par
\end{center}
\newpage
%    \end{macrocode}

% Manually include selected file,
% otherwise process as usual:
%    \begin{macrocode}
\ifchilddocmanual
\section*{part `\childdocname'}
\input{\childdocname}
\else
%    \end{macrocode}

% Include the two chapters:
%    \begin{macrocode}
\include{cdocsch1}
\include{cdocsch2}
%    \end{macrocode}

% Include the two parts unless only chapters should be displayed:
%    \begin{macrocode}
\ifchilddoc\else
\section{part three}
\input{cdocspt3}
\section{part four}
\input{cdocspt4}
\fi
%    \end{macrocode}

% Process as usual until here:
%    \begin{macrocode}
\fi
%    \end{macrocode}

% End of document body:
%    \begin{macrocode}
\end{document}
%    \end{macrocode}
%\iffalse
%</samplemain>
%\fi
%
% %%%%%%%%%%%%%%%%%%%%%%%%%%%%%%%%%%%%%%
% \paragraph{Chapter Include Files.}
%
% The include files are called |cdocsch1.tex| and |cdocsch2.tex|.
%
%\iffalse
%<*samplechap1|samplechap2>
%\fi

% Optional override for |\version| flag:
%    \begin{macrocode}
%%\providecommand{\version}{final}
%    \end{macrocode}

% Include the main document:
%    \begin{macrocode}
\input{childdoc.def}
\childdocof{cdocsamp}
%    \end{macrocode}

%\iffalse
%</samplechap1|samplechap2>
%\fi
%
%\iffalse
%<*samplechap1>
%\fi
% Some text for chapter 1:
%    \begin{macrocode}
\section{one}
some text in chapter one
%    \end{macrocode}

%\iffalse
%</samplechap1>
%\fi
% Some text for chapter 2:
%\iffalse
%<*samplechap2>
%\fi
%    \begin{macrocode}
\section{two}
more text in chapter two
%    \end{macrocode}

%\iffalse
%</samplechap2>
%\fi
%
% %%%%%%%%%%%%%%%%%%%%%%%%%%%%%%%%%%%%%%
% \paragraph{Part Include Files.}
%
% The include files are called |cdocspt3.tex| and |cdocspt4.tex|.
%
%\iffalse
%<*samplepart3|samplepart4>
%\fi

% Optional override for |\version| flag:
%    \begin{macrocode}
%%\providecommand{\version}{final}
%    \end{macrocode}

% Include the main document:
%    \begin{macrocode}
\input{childdoc.def}
\childdocby{cdocsamp}
%    \end{macrocode}

%\iffalse
%</samplepart3|samplepart4>
%\fi
%
%\iffalse
%<*samplepart3>
%\fi
% Some text for part 3:
%    \begin{macrocode}
some text in part three
%    \end{macrocode}

%\iffalse
%</samplepart3>
%\fi
% Some text for part 4:
%\iffalse
%<*samplepart4>
%\fi
%    \begin{macrocode}
more text in part four
%    \end{macrocode}

%\iffalse
%</samplepart4>
%\fi
%
% %%%%%%%%%%%%%%%%%%%%%%%%%%%%%%%%%%%%%%
% \paragraph{Forwarding for a Complete Draft.}
%
% The following forwarding file |cdocsdrf.tex|
% compiles the main document in draft mode:
%\iffalse
%<*sampledraft>
%\fi
%    \begin{macrocode}
\def\version{draft}
\input{childdoc.def}
\childdocforward{cdocsamp}
%    \end{macrocode}

%\iffalse
%</sampledraft>
%\fi
%
% %%%%%%%%%%%%%%%%%%%%%%%%%%%%%%%%%%%%%%
% \paragraph{Forwarding for Final Version of the Chapters.}
%
% The following forwarding files |cdocsfn1.tex| and |cdocsfn2.tex|
% (with identical content)
% compile the final versions of the child documents
% |cdocsch1.tex| and |cdocsch2.tex|, respectively:
%\iffalse
%<*samplefinal>
%\fi
%    \begin{macrocode}
\def\version{final}
\input{childdoc.def}
\childdocforwardprefix[cdocsamp]{cdocsfn}{cdocsch}
%    \end{macrocode}

%\iffalse
%</samplefinal>
%\fi
%
% %%%%%%%%%%%%%%%%%%%%%%%%%%%%%%%%%%%%%%
% \paragraph{Command Line Processing.}
%
% The following three command lines generate the output files
% |cdocscld|, |cdocscl1| and |cdocscl2|
% which should be identical to
% |cdocsdrf|, |cdocsch1| and |cdocsfn2|, respectively:
% \begin{center}
% \begin{tabular}{l}
% |latex -jobname cdocscld \|\\
% |  "\def\version{draft}\input{childdoc.def}\childdocforward{cdocsamp}"|\\
% |latex -jobname cdocscl1 \|\\
% |  "\input{childdoc.def}\childdocforward[cdocsamp]{cdocsch1}"|\\
% |latex -jobname cdocscl2 \|\\
% |  "\def\version{final}\input{childdoc.def}\childdocforward{cdocsch2}"|
% \end{tabular}
% \end{center}
% Note that the trailing backslash on each first line
% merely continues the input to the second line
% (for convenient cut ant paste).
% Furthermore, the command |latex| can be replaced by any
% of its alternative versions such as |pdflatex|.
%
% %%%%%%%%%%%%%%%%%%%%%%%%%%%%%%%%%%%%%%%%%%%%%%%%%%%%%%%%%%%%%%%%%%%%%%%%%%%%%%
% %%%%%%%%%%%%%%%%%%%%%%%%%%%%%%%%%%%%%%%%%%%%%%%%%%%%%%%%%%%%%%%%%%%%%%%%%%%%%%
% \section{Implementation}
%\iffalse
%<*package>
%\fi
%
% This section describes the definitions file |childdoc.def|.

% The definitions cannot be loaded using |\usepackage| or |\RequirePackage|
% which has a mechanism to prevent loading a style file more than once.
% When loading the definitions by means of |\input|
% multiple instances have to be prevented manually:
%\iffalse
%This code needs to be before the `\ProvidesFile' directive
%which is defined at the beginning of this file.
%Therefore it is also placed there and commented out here.
%</package>
%<*discard>
%\fi
%    \begin{macrocode}
\ifdefined\childdocmain\endinput\fi
%    \end{macrocode}
%\iffalse
%</discard>
%<*package>
%\fi
%
% \macro{\ifchilddoc}
% \macro{\ifchilddocmanual}
% The conditional |\ifchilddoc| tells whether a
% child (true) or main (false) document is being compiled.
% The conditional |\ifchilddocmanual| tells whether
% the |\includeonly| mechanism is used (false) or
% the selection of child files must be performed manually (true).
% The definitions initialise to false:
%    \begin{macrocode}
\newif\ifchilddoc
\newif\ifchilddocmanual
%    \end{macrocode}

% \macro{\childdocname}
% \macro{\childdocjob}
% The macro |\childdocname| stores the name of the main document
% to be compiled. The macro |\childdocjob| stores the name of
% the document on which the \LaTeX{} compiler was originally invoked.
% The content of |\jobname| cannot be compared
% to filenames specified in the source due to different catcodes.
% The following code rescans |\jobname|, stores the result
% in |\childdocname| and saves a copy in |\childdocjob|:
%    \begin{macrocode}
\edef\childdocname{\scantokens\expandafter{\jobname\noexpand}}
\let\childdocjob\childdocname
%    \end{macrocode}

% \macro{\childdocdisable}
% The macro |\childdocdisable| prevents the main file
% from being processed more than once.
% At this stage, the main document command |\childdocmain|
% is assumed to be called once again where it should do nothing.
% Any subsequent call to it should prevent
% a secondary processing of the main document
% It overwrites the forwarding commands
% |\childdocof| and |\childdocforward|
% with empty macros to prevent further inclusions of the main document:
%    \begin{macrocode}
\newcommand{\childdocdisable}
{
  \renewcommand{\childdocmain}[1]{\renewcommand{\childdocmain}[1]{\endinput}}
  \renewcommand{\childdocof}[1]{}
  \renewcommand{\childdocby}[2][]{}
  \renewcommand{\childdocforward}[2][]{}
  \renewcommand{\childdocdisable}{}
}
%    \end{macrocode}

% \macro{\childdocmain}
% The macro |\childdocmain| is to be called at the top of the main file
% with nothing or the main filename (without extension) as argument.
% First, it breaks loops.
% If the argument is not empty and does not match |\childdocname|
% (which is set by the first inclusion of |childdoc.def|),
% |\ifchilddoc| is set to true, |\includeonly| is applied to the child file
% and |\jobname| is set to the main file
% (for proper handling of |.aux| files):
%    \begin{macrocode}
\newcommand{\childdocmain}[1]
{
  \childdocdisable\childdocmain{}
  \if?#1?\else
    \begingroup
      \def\childdoctmp{#1}
      \ifx\childdoctmp\childdocname
        \def\childdoctmp{}
      \else
        \def\childdoctmp
        {
          \childdoctrue
          \includeonly{\childdocname}
          \def\childdocjob{#1}
          \def\jobname{#1}
        }
      \fi
      \expandafter
    \endgroup
    \childdoctmp
  \fi
}
%    \end{macrocode}

% \macro{\childdocof}
% The command |\childdocof| redirects
% compilation to the main file |#1|.
%    \begin{macrocode}
\newcommand{\childdocof}[1]
{
  \childdocdisable
  \childdoctrue
  \includeonly{\childdocname}
  \def\jobname{#1}
  \def\childdocjob{#1}
  \input{#1}
}
%    \end{macrocode}

% \macro{\childdocby}
% The command |\childdocby| ....
%    \begin{macrocode}
\newcommand{\childdocby}[2][]
{
  \childdocdisable
  \childdoctrue
  \childdocmanualtrue
  \if?#1?\else
    \def\jobname{#2}
  \fi
  \def\childdocjob{#2}
  \input{#2}
  \endinput
}
%    \end{macrocode}

% \macro{\childdocforward}
% The command |\childdocforward| redirects
% compilation to the main file or
% (if the optional argument is given) a child file.
% Parameters are set as if the main file
% or a child file starting with |\childdocof| was compiled.
% Then compilation is handed over to the main file:
%    \begin{macrocode}
\newcommand{\childdocforward}[2][]
{
  \begingroup
    \if?#1?
      \def\childdoctmp
      {
        \def\childdocname{#2}
        \def\childdocjob{#2}
        \def\jobname{#2}
        \input{#2}
        \endinput
      }
    \else
      \def\childdoctmp
      {
        \childdocdisable
        \def\childdocname{#2}
        \childdoctrue
        \includeonly{#2}
        \def\childdocjob{#1}
        \def\jobname{#1}
        \input{#1}
        \endinput
      }
    \fi
    \expandafter
  \endgroup
  \childdoctmp
}
%    \end{macrocode}

% \macro{\childdocforwardprefix}
% The command |\childdocforwardprefix| redirects
% compilation to the main or a child file by means of a pattern.
% The prefix |#1| in the current filename is replaced by |#2|
% and the suffix of the current filename is kept
% (it is assumed that the filename does not contain the substring `|~~~|'
% which is used as a delimiter).
% Compilation is handed over to the new file by |\childdocforward|:
%    \begin{macrocode}
\newcommand{\childdocforwardprefix}[3][]
{
  \begingroup
    \def\childdocextract #2##1~~~{\def\childdoctmp{\childdocforward[#1]{#3##1}}}
    \expandafter\childdocextract\childdocname~~~
    \expandafter
  \endgroup
  \childdoctmp
}
%    \end{macrocode}

% \macro{\childdoc}
% The deprecated macro |\childdoc| is a legacy version of |\childdocmain|:
%    \begin{macrocode}
\newcommand{\childdoc}{\childdocmain}
%    \end{macrocode}

% \macro{\childdocredirect}
% The deprecated macro |\childdocredirect| is a legacy version
% of |\childdocforward| and |\childdocforwardprefix|:
%    \begin{macrocode}
\newcommand{\childdocredirect}[2][]
{
  \begingroup
    \if?#1?
      \def\childdoctmp{\childdocforward{#2}}
    \else
      \def\childdoctmp{\childdocforwardprefix{#1}{#2}}
    \fi
    \expandafter
  \endgroup
  \childdoctmp
}
%    \end{macrocode}

%\iffalse
%</package>
%\fi
%
\endinput
|\\
|\childdocby{|\textit{main}|}|\\
\end{tabular}
\end{center}
%
The directive |\childdocby| is similar to |\childdocof|
described in \secref{sec:include},
but the subsequent selection of content must be done manually.
To that end, both |\ifchilddoc| and |\ifchilddocmanual|
will be true upon processing of a part,
and the name of the part is stored in |\childdocname|.
Note that |\jobname| will be set to the filename of the current part
so that each part receives an individual |.aux| file
that does not interfere with the |.aux| file(s) of the main document.
This behaviour can be altered by the alternative form
|\childdocby[*]{|\textit{main}|}| (with a non-empty optional argument)
which uses the |.aux| file of the main document
by setting |\jobname| to \textit{main}.

%%%%%%%%%%%%%%%%%%%%%%%%%%%%%%%%%%%%%%%%%%%%%%%%%%%%%%%%%%%%%%%%%%%%%%%%%%%%%%%%
\subsection{Driver Development}
\label{sec:driver}

The \textsf{childdoc} mechanism can also be use for the development
of definition files such as \LaTeX{} styles or classes.
This case differs from the above setup with multiple parts
included by |\include| in that no |\includeonly| should be invoked.
This can be achieved by starting the include file
(before |\ProvidesPackage|) with:
%
\begin{center}
\begin{tabular}{l}
|% \iffalse
%
% childdoc.dtx Copyright (C) 2017-2018 Niklas Beisert
%
% This work may be distributed and/or modified under the
% conditions of the LaTeX Project Public License, either version 1.3
% of this license or (at your option) any later version.
% The latest version of this license is in
%   http://www.latex-project.org/lppl.txt
% and version 1.3 or later is part of all distributions of LaTeX
% version 2005/12/01 or later.
%
% This work has the LPPL maintenance status `maintained'.
%
% The Current Maintainer of this work is Niklas Beisert.
%
% This work consists of the files childdoc.dtx and childdoc.ins
% and the derived files childdoc.def and cdocsamp.tex with
% cdocsch1.tex, cdocsch2.tex, cdocsdrf.tex, cdocsfn1.tex, cdocsfn2.tex.
%
%<package>\ifdefined\childdocmain\endinput\fi
%<package>\ProvidesFile{childdoc.def}[2018/12/30 v2.0 child document driver]
%<samplemain>\ProvidesFile{cdocsamp.tex}[2018/12/30 v2.0 sample for childdoc]
%<*driver>
%\ProvidesFile{childdoc.drv}[2018/12/30 v2.0 childdoc reference manual file]
\PassOptionsToClass{10pt,a4paper}{article}
\documentclass{ltxdoc}

\usepackage[margin=35mm]{geometry}
\usepackage{hyperref}
\usepackage{hyperxmp}
\usepackage[usenames]{color}

\hypersetup{colorlinks=true}
\hypersetup{pdfstartview=FitH}
\hypersetup{pdfpagemode=UseNone}
\hypersetup{pdfsource={}}
\hypersetup{pdflang={en-UK}}
\hypersetup{pdfcopyright={Copyright 2017-2018 Niklas Beisert.
  This work may be distributed and/or modified under the
  conditions of the LaTeX Project Public License, either version 1.3
  of this license or (at your option) any later version.}}
\hypersetup{pdflicenseurl={http://www.latex-project.org/lppl.txt}}
\hypersetup{pdfcontactaddress={ETH Zurich, ITP, HIT K,
  Wolfgang-Pauli-Strasse 27}}
\hypersetup{pdfcontactpostcode={8093}}
\hypersetup{pdfcontactcity={Zurich}}
\hypersetup{pdfcontactcountry={Switzerland}}
\hypersetup{pdfcontactemail={nbeisert@itp.phys.ethz.ch}}
\hypersetup{pdfcontacturl={http://people.phys.ethz.ch/\xmptilde nbeisert/}}

\newcommand{\secref}[1]{\hyperref[#1]{section \ref*{#1}}}

\parskip1ex
\parindent0pt
\let\olditemize\itemize
\def\itemize{\olditemize\parskip0pt}

\begin{document}

\title{The \textsf{childdoc} Package}
\hypersetup{pdftitle={The childdoc Package}}
\author{Niklas Beisert\\[2ex]
  Institut f\"ur Theoretische Physik\\
  Eidgen\"ossische Technische Hochschule Z\"urich\\
  Wolfgang-Pauli-Strasse 27, 8093 Z\"urich, Switzerland\\[1ex]
  \href{mailto:nbeisert@itp.phys.ethz.ch}
  {\texttt{nbeisert@itp.phys.ethz.ch}}}
\hypersetup{pdfauthor={Niklas Beisert}}
\hypersetup{pdfsubject={Manual for the LaTeX2e Package childdoc}}
\date{30 December 2018, \textsf{v2.0}}
\maketitle

\begin{abstract}\noindent
\textsf{childdoc} is a \LaTeXe{} package
that enables the direct compilation
of document sections included by |\include|
to individual files.
\end{abstract}

\begingroup
\parskip0ex
\tableofcontents
\endgroup

%%%%%%%%%%%%%%%%%%%%%%%%%%%%%%%%%%%%%%%%%%%%%%%%%%%%%%%%%%%%%%%%%%%%%%%%%%%%%%%%
%%%%%%%%%%%%%%%%%%%%%%%%%%%%%%%%%%%%%%%%%%%%%%%%%%%%%%%%%%%%%%%%%%%%%%%%%%%%%%%%
\section{Introduction}

\LaTeX{} provides a mechanism to structure a large document (such as a book)
into a main file and several child files (containing the chapters)
using the |\include| command.
This mechanism is beneficial for documents
which span hundreds of pages in order to
make the source file(s) more manageable.
Moreover, compilation can be restricted to
selected child files by means of the |\includeonly| command.
The latter feature can be used to reduce the compilation time while editing
(this was significantly more useful in the earlier days of \LaTeX{})
or to generate a smaller document which is easier to navigate.
Another application of |\includeonly| is to generate
documents consisting of selected parts of the complete document.

However, there are a few drawbacks of the plain |\include| mechanism:
\begin{itemize}
\item
The child files cannot be compiled on their own,
they can only be compiled via the main file.
A naive editing environment
(such as a text editor with an option
to have the current file processed by \LaTeX)
may require one to switch to the main file before compiling;
attempting to compile the child file produces errors.
\item
The main file must be modified (each time)
to adjust the |\includeonly| command
to the present needs. This easily leaves the main file in a messy state.
\item
The generated document will always carry the filename
of the main document. This is inconvenient if
several child files are to be compiled and
to be kept for distribution.
\end{itemize}

The present package provides a simple interface
to make child files individually compilable by \LaTeX{}.
Compiling a child file then has the same effect as compiling
the main file with an |\includeonly| command
to select the appropriate child.
Moreover the generated document will carry the name of the child
rather than the main file.
This resolves all three above issues.

This feature is meant to make the editing of books,
thesis documents and lecture notes somewhat more convenient.
However, the package can also be used efficiently for
composing a series of documents (such as exercise sheets)
which are typically distributed individually.
It then assists the author in generating the individual documents
(potentially in different versions)
as well as a document containing the collected series.
Another application is in developing style files
or other kinds of included material
where compilation of the style file could redirect
to a sample or test file.

%%%%%%%%%%%%%%%%%%%%%%%%%%%%%%%%%%%%%%%%%%%%%%%%%%%%%%%%%%%%%%%%%%%%%%%%%%%%%%%%
%%%%%%%%%%%%%%%%%%%%%%%%%%%%%%%%%%%%%%%%%%%%%%%%%%%%%%%%%%%%%%%%%%%%%%%%%%%%%%%%
\section{Usage}

First of all, the package \textsf{childdoc} is \emph{not} a standard
\LaTeXe{} |.sty| style file! Therefore it needs to be invoked in
a non-standard way.

%%%%%%%%%%%%%%%%%%%%%%%%%%%%%%%%%%%%%%%%%%%%%%%%%%%%%%%%%%%%%%%%%%%%%%%%%%%%%%%%
\subsection{Included Files}
\label{sec:include}

%%%%%%%%%%%%%%%%%%%%%%%%%%%%%%%%%%%%%%%%
\DescribeMacro{\childdocmain}
To use the package, add the commands
\begin{center}
\begin{tabular}{l}
|\input{childdoc.def}|\\
|\childdocmain{}|\\
\end{tabular}
\end{center}
at the very top of the main \LaTeX{} file,
in particular \emph{before} the |\documentclass| statement!
The argument of |\childdocmain| should be left empty
(but it must be present).

%%%%%%%%%%%%%%%%%%%%%%%%%%%%%%%%%%%%%%%%
\DescribeMacro{\childdocof}
Furthermore, add the commands
\begin{center}
\begin{tabular}{l}
|\input{childdoc.def}|\\
|\childdocof{|\textit{main}|}|\\
\end{tabular}
\end{center}
at the top of every child file \textit{child}
which is included by |\include{|\textit{child}|}|
from within the main file
(or at least for those files to be compiled individually).
The argument \textit{main} must be the filename of the main file.

There are a couple of
considerations in setting up the main and child documents:

%%%%%%%%%%%%%%%%%%%%%%%%%%%%%%%%%%%%%%%%
\paragraph{Restrictions.}

Please note the following restrictions:
\begin{itemize}
\item
|\childdocmain| must be called with one argument \textit{main}
to ensure compatibility with earlier version of the package.
It must either be empty (|\childdocmain{}|)
or precisely match the filename of the main file in which it is specified.
See \secref{sec:detection} for further information.
\item
The filename \textit{main} must be specified without the |.tex| extension.
\item
The filename \textit{main} is case sensitive
(even in case-insensitive file systems)
due to internal string comparison.
\item
The argument \textit{main} should be fully expanded, it cannot be a macro.
\item
Subdirectories and special characters should be avoided in filenames.
\item
The command |\childdocmain{|\textit{main}|}| must be followed by a whitespace.
It should not be followed immediately by another command
or by a comment mark `|%|'.
This is because the \TeX{} parser reads the token immediately following
the argument of |\childdocmain| and puts it
at the beginning of every child section;
however, a white\-space is ignored.
\end{itemize}

%%%%%%%%%%%%%%%%%%%%%%%%%%%%%%%%%%%%%%%%
\paragraph{Content of Main File.}

It is advisable to place all content in the child files included by |\include|.
Any output contained in the main file will appear in all child documents
unless suppressed manually;
it cannot be suppressed automatically by the |\includeonly| directive
and thus should normally be avoided.
A method to include some content in the main file
by means of conditional processing is described in \secref{sec:conditional}.

%%%%%%%%%%%%%%%%%%%%%%%%%%%%%%%%%%%%%%%%
\paragraph{Page Numbering.}

When only a part of the document is compiled,
the appropriate numbering of pages
(as well as other status parameters)
is determined from the |.aux| files.
The latter contain information from previous passes.
However this information needs to propagate through
all intermediate child documents.
Therefore the page numbering in child documents may well
be inconsistent until the complete document is compiled at least once.

A useful (if unconventional) way to always ensure a consistent
page numbering is to restart the numbering in each child document
and denote the pages by `\textit{child}|.|\textit{page}'
where \textit{child} represents the chapter/section number of the child file.
This can be achieved by the command
|\numberwithin{page}{|\textit{child}|}|
of the \textsf{amsmath} package
where \textit{child} can be |chapter| or |section|
depending on the chosen structuring.
Alternatively, one can modify the macro |\thepage| appropriately
and reset the counter |page| at the start of each child file.

%%%%%%%%%%%%%%%%%%%%%%%%%%%%%%%%%%%%%%%%%%%%%%%%%%%%%%%%%%%%%%%%%%%%%%%%%%%%%%%%
\subsection{Conditional Processing}
\label{sec:conditional}

The package provides a mechanism to compile different versions
of a document. To customise the versions further some conditional processing
can come in handy to distinguish which version is being compiled.
The package provides two macros to describe the compilation context:

%%%%%%%%%%%%%%%%%%%%%%%%%%%%%%%%%%%%%%%%
\DescribeMacro{\ifchilddoc}
The conditional |\ifchilddoc| distinguishes between the compilation of
child documents and the main document:
%
\begin{center}
|\ifchilddoc |\textit{child-code}| |[|\||else |\textit{main-code}]| \||fi|
\end{center}

%%%%%%%%%%%%%%%%%%%%%%%%%%%%%%%%%%%%%%%%
\DescribeMacro{\childdocname}
\DescribeMacro{\childdocjob}
The macro |\childdocname| contains the filename (without extension)
of the main or child file being processed.
Note that |\childdocjob| will always contain the name of the main file.

%%%%%%%%%%%%%%%%%%%%%%%%%%%%%%%%%%%%%%%%
\paragraph{Title Page.}

Conditional processing can be used to include a title or banner page
in the main document when proper precautions are taken.
Importantly, the code in the main file should ensure that the page counter
(as well as other status parameters which are stored in the |.aux| files)
takes the same value after the conditional processing.
Otherwise the page numbers may take divergent values
depending on which part is compiled.

For example, a title page could be declared by:
%
\begin{center}
\begin{tabular}{l}
|\ifchilddoc\||else|\\
|\addtocounter{page}{-1}|\\
\textit{code for title page}\\
|\newpage|\\
|\||fi|
\end{tabular}
\end{center}
%
A banner page for the child documents can be generated by:
%
\begin{center}
\begin{tabular}{l}
|\ifchilddoc|\\
|\addtocounter{page}{-1}|\\
\textit{code for banner page}\\
|\newpage|\\
|\||fi|
\end{tabular}
\end{center}
%
Here one could write a message such as:
\begin{center}
|This is the part \childdocname{} of \childdocjob{}.|
\end{center}

%%%%%%%%%%%%%%%%%%%%%%%%%%%%%%%%%%%%%%%%%%%%%%%%%%%%%%%%%%%%%%%%%%%%%%%%%%%%%%%%
\subsection{Flags}
\label{sec:flags}

The package makes it easy to generate different versions
of the main or child documents.
To this end compilation flags can be defined
and assigned different default values.
They will be particularly useful in conjunction
with the forwarding mechanism described in \secref{sec:forward}.

For example, it may be useful to have a flag |\version|
which can be set to |draft| or |final|.
The document source will contain some conditional code
depending on the value of |\version|.
Suppose further, the flag should default to |final| for the main file
and to |draft| for child files
which is a natural assignment for editing the document.
This is achieved by placing the following code
in the preamble of the main document
(below the |\childdocmain| directive):
%
\begin{center}
\begin{tabular}{l}
|\ifchilddoc|\\
|\providecommand{\version}{draft}|\\
|\||else|\\
|\providecommand{\version}{final}|\\
|\||fi|
\end{tabular}
\end{center}
%
The definition by |\providecommand| makes sure
that previous definitions are not overwritten.
Further statements |\providecommand{\version}{...}|
can thus be added before the above code to override it.

For the main file, one might add a line
(between |\childdocmain| and the above block)
%
\begin{center}
|%\ifchilddoc\||else\providecommand{\version}{draft}\||fi|
\end{center}
%
which can be uncommented to produce a draft version.
Likewise one can add a line to the very top of a child file
(above the |\childdocof{|\textit{main}|}| directive)
%
\begin{center}
|%\providecommand{\version}{final}|
\end{center}
%
which can be uncommented to produce the final version of this child document.

%%%%%%%%%%%%%%%%%%%%%%%%%%%%%%%%%%%%%%%%%%%%%%%%%%%%%%%%%%%%%%%%%%%%%%%%%%%%%%%%
\subsection{Forwarding}
\label{sec:forward}

Different versions of the main or child documents
using compilation flags as described in \secref{sec:flags}
can be (permanently) stored in different files
for convenient compilation, viewing and distribution.
To this end, the package defines a command
to pass on compilation to a different file:

%%%%%%%%%%%%%%%%%%%%%%%%%%%%%%%%%%%%%%%%
\DescribeMacro{\childdocforward}
The command |\childdocforward| redirects processing to
another source file:
%
\begin{center}
\begin{tabular}{l}
|\input{childdoc.def}|\\
|\childdocforward[|\textit{main}|]{|\textit{dest}|}|\\
\end{tabular}
\end{center}
%
The argument \textit{dest} is the destination file
(without extension).
It should be the main file or one of the child files.
Note that further \textsf{childdoc} directives
such as |\childdocof| and |\childdocforward|
in the indicated file will be processed in this form.
The optional argument \textit{main}
passes on directly to the main file \textit{main}
while pretending to compile the child \textit{dest}.
This form behaves as if \textit{dest}
issues |\childdocof{|\textit{main}|}| right away,
and no further \textsf{childdoc} directives will be processed.

%%%%%%%%%%%%%%%%%%%%%%%%%%%%%%%%%%%%%%%%
\DescribeMacro{\...prefix}
In the alternative form |\childdocforwardprefix|,
%
\begin{center}
\begin{tabular}{l}
|\input{childdoc.def}|\\
|\childdocforwardprefix[|\textit{main}|]{|\textit{prefix}|}{|\textit{dest}|}|
\end{tabular}
\end{center}
%
the destination file is determined by a pattern
depending on the current file:
To make this work, the current file must be called
`{\textit{prefix}\hspace{0.2em}\textit{suffix}}'
with \textit{prefix} matching precisely the argument.
Processing is then passed on to the file
`{\textit{dest}\hspace{0.2em}\textit{suffix}}'.
Surely, the same effect is achieved by
directly specifying the
argument `{\textit{dest}\hspace{0.2em}\textit{suffix}}'
in the first form.
However, that requires to set up a different file
for each child. With the alternative form of the command
all these files can have exactly the same content
which simplifies setting them up and maintaining them.

For example, the following file |draft.tex|
with a compilation flag |\version| as described in \secref{sec:flags}
compiles the main document as a draft:
%
\begin{center}
\begin{tabular}{l}
|\def\version{draft}|\\
|\input{childdoc.def}|\\
|\childdocforward{|\textit{main}|}|
\end{tabular}
\end{center}
%
Likewise, the following files |final|\textit{nn}|.tex|
compile the final version of the child document
|child|\textit{nn}|.tex|:
%
\begin{center}
\begin{tabular}{l}
|\def\version{final}|\\
|\input{childdoc.def}|\\
|\childdocforwardprefix{final}{child}|
\end{tabular}
\end{center}
%

Note that when several versions of a main file and/or of each child file
are to be generated, it may be convenient to set up a |Makefile| or
shell script to automatise the process.

%%%%%%%%%%%%%%%%%%%%%%%%%%%%%%%%%%%%%%%%%%%%%%%%%%%%%%%%%%%%%%%%%%%%%%%%%%%%%%%%
\subsection{Command Line Processing}
\label{sec:commandline}

The effect of redirection files can also be achieved by invoking
the \LaTeX{} compiler with a more elaborate command line.
Most conveniently this should be done as part
of a shell script or a |Makefile|.

When using \textsf{childdoc} in the main file, the following
command lines effectively perform a redirection
(note that depending on the shell being used,
backslashes may have to be doubled: `|\|' $\to$ `|\\|'):
%
\begin{center}
|... -jobname "|\textit{target}|" |\\|"|[\textit{flags}]%
|\input{childdoc.def}\childdocforward[|\textit{main}|]{|\textit{dest}|}"|
\end{center}
%
Here \textit{target} is the name of the output file,
\textit{main} is the name of the main file
and \textit{dest} is the name of the main or child file to be processed
(all filenames without extensions).
The optional argument \textit{main} can be omitted
if \textit{main} matches \textit{dest}.
Optionally, compilation \textit{flags} can be defined via |\def| commands.
This command line makes the \TeX{} engine believe
it is compiling the file \textit{target}
whose content is specified as the latter parameter.
The provided code then forwards the processing to
\textit{main} or \textit{dest} as described in \secref{sec:forward}.

%%%%%%%%%%%%%%%%%%%%%%%%%%%%%%%%%%%%%%%%%%%%%%%%%%%%%%%%%%%%%%%%%%%%%%%%%%%%%%%%
\subsection{Include by Input}
\label{sec:input}

Including child documents by |\include| has some restrictions by design.
Most notably, the content of a child document always occupies
its own set of pages; pages cannot be shared between child documents.
Usually, this behaviour makes perfect sense
because each child document contain an essential part of the document.
However, in some situations it may be desirable to compose
a document from a collection of parts
without having mandatory page breaks between then.
For this case, the package
provides a mechanism to include parts
by |\input| which can also be processed individually.
However, by construction this mechanism
requires manual handling of the content to be output.

%%%%%%%%%%%%%%%%%%%%%%%%%%%%%%%%%%%%%%%%
\DescribeMacro{\ifchilddocmanual}
The main file should be prepared as usual, see \secref{sec:include}.
However, the document body must make a distinction
between processing of an individual part and of the main document, e.g.:
%
\begin{center}
\begin{tabular}{l}
|\ifchilddocmanual|\\
|\input{\childdocname}|\\
|\||else|\\
\textit{document body with }|\input{|\textit{part}|}|\\
|\||fi|
\end{tabular}
\end{center}
%
The conditional |\ifchilddocmanual| is true whenever
a part to be included by |\input| is being compiled,
and the name of the part is stored in |\childdocname|.

%%%%%%%%%%%%%%%%%%%%%%%%%%%%%%%%%%%%%%%%
\DescribeMacro{\childdocby}
Each part to be included by |\input| should start with:
%
\begin{center}
\begin{tabular}{l}
|\input{childdoc.def}|\\
|\childdocby{|\textit{main}|}|\\
\end{tabular}
\end{center}
%
The directive |\childdocby| is similar to |\childdocof|
described in \secref{sec:include},
but the subsequent selection of content must be done manually.
To that end, both |\ifchilddoc| and |\ifchilddocmanual|
will be true upon processing of a part,
and the name of the part is stored in |\childdocname|.
Note that |\jobname| will be set to the filename of the current part
so that each part receives an individual |.aux| file
that does not interfere with the |.aux| file(s) of the main document.
This behaviour can be altered by the alternative form
|\childdocby[*]{|\textit{main}|}| (with a non-empty optional argument)
which uses the |.aux| file of the main document
by setting |\jobname| to \textit{main}.

%%%%%%%%%%%%%%%%%%%%%%%%%%%%%%%%%%%%%%%%%%%%%%%%%%%%%%%%%%%%%%%%%%%%%%%%%%%%%%%%
\subsection{Driver Development}
\label{sec:driver}

The \textsf{childdoc} mechanism can also be use for the development
of definition files such as \LaTeX{} styles or classes.
This case differs from the above setup with multiple parts
included by |\include| in that no |\includeonly| should be invoked.
This can be achieved by starting the include file
(before |\ProvidesPackage|) with:
%
\begin{center}
\begin{tabular}{l}
|\input{childdoc.def}|\\
|\childdocforward{|\textit{main}|}|\\
\end{tabular}
\end{center}
%
or alternatively with:
%
\begin{center}
\begin{tabular}{l}
|\input{childdoc.def}|\\
|\childdocby{|\textit{main}|}|\\
\end{tabular}
\end{center}
%
Both forms have slightly different effects as described above.
The main file is prepared as usual, see \secref{sec:include}.

%%%%%%%%%%%%%%%%%%%%%%%%%%%%%%%%%%%%%%%%%%%%%%%%%%%%%%%%%%%%%%%%%%%%%%%%%%%%%%%%
\subsection{Legacy Detection}
\label{sec:detection}

The directive |\childdocmain| in the main file can detect
whether the complete document or merely a child is to be compiled
even without using the directive |\childdocof|.
This method is deprecated because it is less robust
and there is no compelling reason to use it;
it is merely provided for backward compatibility
and it may be removed in future versions.

If the detection mechanism is to be used,
it is mandatory to correctly specify
the filename of the main file as the argument of |\childdocmain|:
%
\begin{center}
\begin{tabular}{l}
|\input{childdoc.def}|\\
|\childdocmain{|\textit{main}|}|\\
\end{tabular}
\end{center}
%
If |\jobname| does not match the argument \textit{main} of |\childdocmain|,
it is assumed that |\jobname| points to the child file to be compiled.
When using |\childdocmain| with the main file specified as argument,
it suffices to start a child file
with just |\input{|\textit{main}|}|
without loading of the package and using |\childdocof|.
If instead all processing is done
with the appropriate \textsf{childdoc} directives,
the argument of \textit{main} of |\childdocmain| can be empty.

An alternative version of the command line processing described
in \secref{sec:commandline} using the detection mechanism reads:
%
\begin{center}
|... -jobname "|\textit{target}|" "|[\textit{flags}]%
[|\def\jobname{|\textit{dest}|}|]|\input{|\textit{main}|}"|
\end{center}

%%%%%%%%%%%%%%%%%%%%%%%%%%%%%%%%%%%%%%%%%%%%%%%%%%%%%%%%%%%%%%%%%%%%%%%%%%%%%%%%
\subsection{Manual Code}
\label{sec:manual}

In case one cannot be certain whether the definitions file |childdoc.def|
is installed on the target \TeX{} distribution
and one prefers not to ship it,
it is conceivable to paste a few relevant commands into the sources.

To that end, drop all statements |\input{childdoc.def}|
and perform the replacements as outlined below.
Instead of |\childdocmain{|\textit{main}|}| add the following code
to the top of the main file:
%
\begin{center}
\begin{tabular}{l}
|\||ifdefined\childdocname\endinput\||fi\newif\ifchilddoc|\\
|\edef\childdocname{\scantokens\expandafter{\jobname\noexpand}}|\\
|\def\childdocmain{|\textit{main}|}\||ifx\childdocmain\childdocname\||else|\\
|\childdoctrue\includeonly{\childdocname}\let\jobname\childdocmain\||fi|\\
\end{tabular}
\end{center}
%
Instead of |\childdocof{|\textit{main}|}| just include the main file
at the top of each child file:
%
\begin{center}
|\input{|\textit{main}|}|
\end{center}
%
A simple redirection |\childdocforward{|\textit{dest}|}| is achieved by:
%
\begin{center}
|\def\jobname{|\textit{dest}|}\input{\jobname}|
\end{center}
%
The redirection with prefix
|\childdocforwardprefix[|\textit{prefix}|]{|\textit{dest}|}|
is accomplished by:
%
\begin{center}
\begin{tabular}{l}
|{\edef\jobname{\scantokens\expandafter{\jobname\noexpand}}|\\
|\def\redirectjob |\textit{prefix}|#1~~~{\gdef\jobname{|\textit{dest}|#1}}|\\
|\expandafter\redirectjob\jobname~~~}\input{\jobname}|
\end{tabular}
\end{center}

In an alternative approach,
child documents can be compiled by a specific command line
without additional code or specific definitions:
%
\begin{center}
|... -jobname "|\textit{target}|" "|[\textit{flags}]%
|\includeonly{|\textit{dest}|}\input{|\textit{main}|}"|
\end{center}
%

%%%%%%%%%%%%%%%%%%%%%%%%%%%%%%%%%%%%%%%%%%%%%%%%%%%%%%%%%%%%%%%%%%%%%%%%%%%%%%%%
%%%%%%%%%%%%%%%%%%%%%%%%%%%%%%%%%%%%%%%%%%%%%%%%%%%%%%%%%%%%%%%%%%%%%%%%%%%%%%%%
\section{Information}

%%%%%%%%%%%%%%%%%%%%%%%%%%%%%%%%%%%%%%%%%%%%%%%%%%%%%%%%%%%%%%%%%%%%%%%%%%%%%%%%
\subsection{Copyright}

Copyright \copyright{} 2017--2018 Niklas Beisert

This work may be distributed and/or modified under the
conditions of the \LaTeX{} Project Public License, either version 1.3
of this license or (at your option) any later version.
The latest version of this license is in
  \url{http://www.latex-project.org/lppl.txt}
and version 1.3 or later is part of all distributions of \LaTeX{}
version 2005/12/01 or later.

This work has the LPPL maintenance status `maintained'.

The Current Maintainer of this work is Niklas Beisert.

This work consists of the files |README.txt|, |childdoc.ins| and |childdoc.dtx|
as well as the derived files |childdoc.def|, |cdocsamp.tex|
with |cdocsch1.tex|, |cdocsch2.tex|, |cdocspt3.tex|, |cdocspt4.tex|,
|cdocsdrf.tex|, |cdocsfn1.tex|, |cdocsfn2.tex|
as well as |childdoc.pdf|.

%%%%%%%%%%%%%%%%%%%%%%%%%%%%%%%%%%%%%%%%%%%%%%%%%%%%%%%%%%%%%%%%%%%%%%%%%%%%%%%%
\subsection{Files and Installation}

The package consists of the files:
%
\begin{center}
\begin{tabular}{ll}
    |README.txt|   & readme file \\
    |childdoc.ins| & installation file \\
    |childdoc.dtx| & source file \\
    |childdoc.def| & definition file \\
    |cdocsamp.tex| & sample main file \\
    |cdocsch1.tex| & sample include file \\
    |cdocsch2.tex| & sample include file \\
    |cdocspt3.tex| & sample part file \\
    |cdocspt4.tex| & sample part file \\
    |cdocsdrf.tex| & sample redirection file \\
    |cdocsfn1.tex| & sample redirection file \\
    |cdocsfn2.tex| & sample redirection file \\
    |childdoc.pdf| & manual
\end{tabular}
\end{center}
%
The distribution consists of the files
|README.txt|, |childdoc.ins| and |childdoc.dtx|.
%
\begin{itemize}
\item
Run (pdf)\LaTeX{} on |childdoc.dtx|
to compile the manual |childdoc.pdf| (this file).
\item
Run \LaTeX{} on |childdoc.ins| to create the definitions file |childdoc.def|
and the sample |cdocsamp.tex| with include files
|cdocsch1.tex|, |cdocsch2.tex|, |cdocspt3.tex|, |cdocspt4.tex|,
|cdocsdrf.tex|, |cdocsfn1.tex|, |cdocsfn2.tex|.
Then copy the file |childdoc.def| to an appropriate directory of your \LaTeX{}
distribution, e.g.\ \textit{texmf-root}|/tex/latex/childdoc|.
\end{itemize}

%%%%%%%%%%%%%%%%%%%%%%%%%%%%%%%%%%%%%%%%%%%%%%%%%%%%%%%%%%%%%%%%%%%%%%%%%%%%%%%%
\subsection{Related CTAN Packages}

There are several other packages which offer a similar functionality:
%
\begin{itemize}
\item
The packages
\href{http://ctan.org/pkg/docmute}{\textsf{docmute}},
\href{http://ctan.org/pkg/includex}{\textsf{includex}} and
\href{http://ctan.org/pkg/standalone}{\textsf{standalone}}
provide commands to include only the document body of
a child file thus allowing both files to be compiled individually.
\item
The packages \href{http://ctan.org/pkg/subdocs}{\textsf{subdocs}}
and \href{http://ctan.org/pkg/subfiles}{\textsf{subfiles}}
provide structures in which the main and child documents can be
encapsulated and allowing them to be compiled individually.
The inclusion mechanism is different from the conventional |\include|.
\item
The package \href{http://ctan.org/pkg/combine}{\textsf{combine}}
is an elaborate solution to combine several documents into one.
\end{itemize}
%
See also the CTAN topic \href{http://ctan.org/topic/subdocs}{\textsf{subdocs}}
for further related packages.
The present package differs from the above solutions in that
a document structure constructed with the conventional |\include| mechanism
just needs two extra commands at the top of every file
such that all constituent files can be compiled individually.

%%%%%%%%%%%%%%%%%%%%%%%%%%%%%%%%%%%%%%%%%%%%%%%%%%%%%%%%%%%%%%%%%%%%%%%%%%%%%%%%
%\subsection{Feature Suggestions}
%
%The following is a list of features which may be useful for future
%versions of this package:
%%
%\begin{itemize}
%\item
%\ldots
%\end{itemize}

%%%%%%%%%%%%%%%%%%%%%%%%%%%%%%%%%%%%%%%%%%%%%%%%%%%%%%%%%%%%%%%%%%%%%%%%%%%%%%%%
\subsection{Revision History}

%%%%%%%%%%%%%%%%%%%%%%%%%%%%%%%%%%%%%%%%
\paragraph{v2.0:} 2018/12/30

\begin{itemize}
\item
immediate forward processing
\item
added |\childdocby| mechanism
\item
manual restructured
\end{itemize}

%%%%%%%%%%%%%%%%%%%%%%%%%%%%%%%%%%%%%%%%
\paragraph{v1.6:} 2018/01/17

\begin{itemize}
\item
application for development of include files
\item
corrections to manual
\end{itemize}

%%%%%%%%%%%%%%%%%%%%%%%%%%%%%%%%%%%%%%%%
\paragraph{v1.5:} 2017/05/21

\begin{itemize}
\item
more complete structuring introduced
\item
|\childdocof| introduced
\item
|\childdoc| renamed to |\childdocmain|
\item
|\childredirect| renamed to |\childdocforward| and |\childdocforwardprefix|
and functionality expanded
\end{itemize}

%%%%%%%%%%%%%%%%%%%%%%%%%%%%%%%%%%%%%%%%
\paragraph{v1.0:} 2017/04/27

\begin{itemize}
\item
manual and install package
\item
first version published on CTAN
\end{itemize}

%%%%%%%%%%%%%%%%%%%%%%%%%%%%%%%%%%%%%%%%
\paragraph{v0.6:} 2017/04/26

\begin{itemize}
\item
redirection mechanism added
\end{itemize}

%%%%%%%%%%%%%%%%%%%%%%%%%%%%%%%%%%%%%%%%
\paragraph{v0.5:} 2017/04/26

\begin{itemize}
\item
functionality in definition file
\end{itemize}


%%%%%%%%%%%%%%%%%%%%%%%%%%%%%%%%%%%%%%%%%%%%%%%%%%%%%%%%%%%%%%%%%%%%%%%%%%%%%%%%
%%%%%%%%%%%%%%%%%%%%%%%%%%%%%%%%%%%%%%%%%%%%%%%%%%%%%%%%%%%%%%%%%%%%%%%%%%%%%%%%
%%%%%%%%%%%%%%%%%%%%%%%%%%%%%%%%%%%%%%%%%%%%%%%%%%%%%%%%%%%%%%%%%%%%%%%%%%%%%%%%
\appendix

\settowidth\MacroIndent{\rmfamily\scriptsize 000\ }

 \DocInput{childdoc.dtx}

\end{document}
%</driver>
% \fi
%
% %%%%%%%%%%%%%%%%%%%%%%%%%%%%%%%%%%%%%%%%%%%%%%%%%%%%%%%%%%%%%%%%%%%%%%%%%%%%%%
% %%%%%%%%%%%%%%%%%%%%%%%%%%%%%%%%%%%%%%%%%%%%%%%%%%%%%%%%%%%%%%%%%%%%%%%%%%%%%%
% \section{Sample}
%\iffalse
%<*samplemain>
%\fi
%
% The following presents a sample document
% with two chapters, two parts, a title page,
% a compile flag as well as three forwarding files to set the flag.
% It consists of eight |.tex| files:
% \begin{center}
% \begin{tabular}{ll}
% |cdocsamp.tex|&main file\\
% |cdocsch1.tex|&include file for chapter 1\\
% |cdocsch2.tex|&include file for chapter 2\\
% |cdocspt3.tex|&include file for part 3\\
% |cdocspt4.tex|&include file for part 4\\
% |cdocsdrf.tex|&forwarding file for main file in draft mode\\
% |cdocsfi1.tex|&forwarding file for final version of chapter 1\\
% |cdocsfi2.tex|&forwarding file for final version of chapter 2\\
% \end{tabular}
% \end{center}
% Each of the eight files can be compiled directly by the \LaTeX{} compiler.
%
% %%%%%%%%%%%%%%%%%%%%%%%%%%%%%%%%%%%%%%
% \paragraph{Main File.}
%
% The main file is called |cdocsamp.tex|.
%
% Load the \textsf{childdoc} definitions and
% declare the filename for the main document:
%    \begin{macrocode}
\input{childdoc.def}
\childdocmain{}
%    \end{macrocode}

% Optional override for |\version| flag:
%    \begin{macrocode}
%%\ifchilddoc\else\providecommand{\version}{draft}\fi
%    \end{macrocode}

% Define the default values for the |\version| flag
% (|final| for the main file and |draft| for childs):
%    \begin{macrocode}
\ifchilddoc
\providecommand{\version}{draft}
\else
\providecommand{\version}{final}
\fi
%    \end{macrocode}

% Load the standard document class:
%    \begin{macrocode}
\documentclass[12pt]{article}
%    \end{macrocode}

% Start the document body:
%    \begin{macrocode}
\begin{document}
%    \end{macrocode}

% Declare a title page.
% Print title, part of document being processed and version flag:
%    \begin{macrocode}
\addtocounter{page}{-1}
\begin{center}
{\LARGE\bfseries{}childdoc example\par}
\vspace{1cm}
\ifchilddoc
\ifchilddocmanual part\else chapter\fi:
`\childdocname' of `\childdocjob'\par
\else
main document: `\childdocjob'\par
\fi
version: \version\par
\end{center}
\newpage
%    \end{macrocode}

% Manually include selected file,
% otherwise process as usual:
%    \begin{macrocode}
\ifchilddocmanual
\section*{part `\childdocname'}
\input{\childdocname}
\else
%    \end{macrocode}

% Include the two chapters:
%    \begin{macrocode}
\include{cdocsch1}
\include{cdocsch2}
%    \end{macrocode}

% Include the two parts unless only chapters should be displayed:
%    \begin{macrocode}
\ifchilddoc\else
\section{part three}
\input{cdocspt3}
\section{part four}
\input{cdocspt4}
\fi
%    \end{macrocode}

% Process as usual until here:
%    \begin{macrocode}
\fi
%    \end{macrocode}

% End of document body:
%    \begin{macrocode}
\end{document}
%    \end{macrocode}
%\iffalse
%</samplemain>
%\fi
%
% %%%%%%%%%%%%%%%%%%%%%%%%%%%%%%%%%%%%%%
% \paragraph{Chapter Include Files.}
%
% The include files are called |cdocsch1.tex| and |cdocsch2.tex|.
%
%\iffalse
%<*samplechap1|samplechap2>
%\fi

% Optional override for |\version| flag:
%    \begin{macrocode}
%%\providecommand{\version}{final}
%    \end{macrocode}

% Include the main document:
%    \begin{macrocode}
\input{childdoc.def}
\childdocof{cdocsamp}
%    \end{macrocode}

%\iffalse
%</samplechap1|samplechap2>
%\fi
%
%\iffalse
%<*samplechap1>
%\fi
% Some text for chapter 1:
%    \begin{macrocode}
\section{one}
some text in chapter one
%    \end{macrocode}

%\iffalse
%</samplechap1>
%\fi
% Some text for chapter 2:
%\iffalse
%<*samplechap2>
%\fi
%    \begin{macrocode}
\section{two}
more text in chapter two
%    \end{macrocode}

%\iffalse
%</samplechap2>
%\fi
%
% %%%%%%%%%%%%%%%%%%%%%%%%%%%%%%%%%%%%%%
% \paragraph{Part Include Files.}
%
% The include files are called |cdocspt3.tex| and |cdocspt4.tex|.
%
%\iffalse
%<*samplepart3|samplepart4>
%\fi

% Optional override for |\version| flag:
%    \begin{macrocode}
%%\providecommand{\version}{final}
%    \end{macrocode}

% Include the main document:
%    \begin{macrocode}
\input{childdoc.def}
\childdocby{cdocsamp}
%    \end{macrocode}

%\iffalse
%</samplepart3|samplepart4>
%\fi
%
%\iffalse
%<*samplepart3>
%\fi
% Some text for part 3:
%    \begin{macrocode}
some text in part three
%    \end{macrocode}

%\iffalse
%</samplepart3>
%\fi
% Some text for part 4:
%\iffalse
%<*samplepart4>
%\fi
%    \begin{macrocode}
more text in part four
%    \end{macrocode}

%\iffalse
%</samplepart4>
%\fi
%
% %%%%%%%%%%%%%%%%%%%%%%%%%%%%%%%%%%%%%%
% \paragraph{Forwarding for a Complete Draft.}
%
% The following forwarding file |cdocsdrf.tex|
% compiles the main document in draft mode:
%\iffalse
%<*sampledraft>
%\fi
%    \begin{macrocode}
\def\version{draft}
\input{childdoc.def}
\childdocforward{cdocsamp}
%    \end{macrocode}

%\iffalse
%</sampledraft>
%\fi
%
% %%%%%%%%%%%%%%%%%%%%%%%%%%%%%%%%%%%%%%
% \paragraph{Forwarding for Final Version of the Chapters.}
%
% The following forwarding files |cdocsfn1.tex| and |cdocsfn2.tex|
% (with identical content)
% compile the final versions of the child documents
% |cdocsch1.tex| and |cdocsch2.tex|, respectively:
%\iffalse
%<*samplefinal>
%\fi
%    \begin{macrocode}
\def\version{final}
\input{childdoc.def}
\childdocforwardprefix[cdocsamp]{cdocsfn}{cdocsch}
%    \end{macrocode}

%\iffalse
%</samplefinal>
%\fi
%
% %%%%%%%%%%%%%%%%%%%%%%%%%%%%%%%%%%%%%%
% \paragraph{Command Line Processing.}
%
% The following three command lines generate the output files
% |cdocscld|, |cdocscl1| and |cdocscl2|
% which should be identical to
% |cdocsdrf|, |cdocsch1| and |cdocsfn2|, respectively:
% \begin{center}
% \begin{tabular}{l}
% |latex -jobname cdocscld \|\\
% |  "\def\version{draft}\input{childdoc.def}\childdocforward{cdocsamp}"|\\
% |latex -jobname cdocscl1 \|\\
% |  "\input{childdoc.def}\childdocforward[cdocsamp]{cdocsch1}"|\\
% |latex -jobname cdocscl2 \|\\
% |  "\def\version{final}\input{childdoc.def}\childdocforward{cdocsch2}"|
% \end{tabular}
% \end{center}
% Note that the trailing backslash on each first line
% merely continues the input to the second line
% (for convenient cut ant paste).
% Furthermore, the command |latex| can be replaced by any
% of its alternative versions such as |pdflatex|.
%
% %%%%%%%%%%%%%%%%%%%%%%%%%%%%%%%%%%%%%%%%%%%%%%%%%%%%%%%%%%%%%%%%%%%%%%%%%%%%%%
% %%%%%%%%%%%%%%%%%%%%%%%%%%%%%%%%%%%%%%%%%%%%%%%%%%%%%%%%%%%%%%%%%%%%%%%%%%%%%%
% \section{Implementation}
%\iffalse
%<*package>
%\fi
%
% This section describes the definitions file |childdoc.def|.

% The definitions cannot be loaded using |\usepackage| or |\RequirePackage|
% which has a mechanism to prevent loading a style file more than once.
% When loading the definitions by means of |\input|
% multiple instances have to be prevented manually:
%\iffalse
%This code needs to be before the `\ProvidesFile' directive
%which is defined at the beginning of this file.
%Therefore it is also placed there and commented out here.
%</package>
%<*discard>
%\fi
%    \begin{macrocode}
\ifdefined\childdocmain\endinput\fi
%    \end{macrocode}
%\iffalse
%</discard>
%<*package>
%\fi
%
% \macro{\ifchilddoc}
% \macro{\ifchilddocmanual}
% The conditional |\ifchilddoc| tells whether a
% child (true) or main (false) document is being compiled.
% The conditional |\ifchilddocmanual| tells whether
% the |\includeonly| mechanism is used (false) or
% the selection of child files must be performed manually (true).
% The definitions initialise to false:
%    \begin{macrocode}
\newif\ifchilddoc
\newif\ifchilddocmanual
%    \end{macrocode}

% \macro{\childdocname}
% \macro{\childdocjob}
% The macro |\childdocname| stores the name of the main document
% to be compiled. The macro |\childdocjob| stores the name of
% the document on which the \LaTeX{} compiler was originally invoked.
% The content of |\jobname| cannot be compared
% to filenames specified in the source due to different catcodes.
% The following code rescans |\jobname|, stores the result
% in |\childdocname| and saves a copy in |\childdocjob|:
%    \begin{macrocode}
\edef\childdocname{\scantokens\expandafter{\jobname\noexpand}}
\let\childdocjob\childdocname
%    \end{macrocode}

% \macro{\childdocdisable}
% The macro |\childdocdisable| prevents the main file
% from being processed more than once.
% At this stage, the main document command |\childdocmain|
% is assumed to be called once again where it should do nothing.
% Any subsequent call to it should prevent
% a secondary processing of the main document
% It overwrites the forwarding commands
% |\childdocof| and |\childdocforward|
% with empty macros to prevent further inclusions of the main document:
%    \begin{macrocode}
\newcommand{\childdocdisable}
{
  \renewcommand{\childdocmain}[1]{\renewcommand{\childdocmain}[1]{\endinput}}
  \renewcommand{\childdocof}[1]{}
  \renewcommand{\childdocby}[2][]{}
  \renewcommand{\childdocforward}[2][]{}
  \renewcommand{\childdocdisable}{}
}
%    \end{macrocode}

% \macro{\childdocmain}
% The macro |\childdocmain| is to be called at the top of the main file
% with nothing or the main filename (without extension) as argument.
% First, it breaks loops.
% If the argument is not empty and does not match |\childdocname|
% (which is set by the first inclusion of |childdoc.def|),
% |\ifchilddoc| is set to true, |\includeonly| is applied to the child file
% and |\jobname| is set to the main file
% (for proper handling of |.aux| files):
%    \begin{macrocode}
\newcommand{\childdocmain}[1]
{
  \childdocdisable\childdocmain{}
  \if?#1?\else
    \begingroup
      \def\childdoctmp{#1}
      \ifx\childdoctmp\childdocname
        \def\childdoctmp{}
      \else
        \def\childdoctmp
        {
          \childdoctrue
          \includeonly{\childdocname}
          \def\childdocjob{#1}
          \def\jobname{#1}
        }
      \fi
      \expandafter
    \endgroup
    \childdoctmp
  \fi
}
%    \end{macrocode}

% \macro{\childdocof}
% The command |\childdocof| redirects
% compilation to the main file |#1|.
%    \begin{macrocode}
\newcommand{\childdocof}[1]
{
  \childdocdisable
  \childdoctrue
  \includeonly{\childdocname}
  \def\jobname{#1}
  \def\childdocjob{#1}
  \input{#1}
}
%    \end{macrocode}

% \macro{\childdocby}
% The command |\childdocby| ....
%    \begin{macrocode}
\newcommand{\childdocby}[2][]
{
  \childdocdisable
  \childdoctrue
  \childdocmanualtrue
  \if?#1?\else
    \def\jobname{#2}
  \fi
  \def\childdocjob{#2}
  \input{#2}
  \endinput
}
%    \end{macrocode}

% \macro{\childdocforward}
% The command |\childdocforward| redirects
% compilation to the main file or
% (if the optional argument is given) a child file.
% Parameters are set as if the main file
% or a child file starting with |\childdocof| was compiled.
% Then compilation is handed over to the main file:
%    \begin{macrocode}
\newcommand{\childdocforward}[2][]
{
  \begingroup
    \if?#1?
      \def\childdoctmp
      {
        \def\childdocname{#2}
        \def\childdocjob{#2}
        \def\jobname{#2}
        \input{#2}
        \endinput
      }
    \else
      \def\childdoctmp
      {
        \childdocdisable
        \def\childdocname{#2}
        \childdoctrue
        \includeonly{#2}
        \def\childdocjob{#1}
        \def\jobname{#1}
        \input{#1}
        \endinput
      }
    \fi
    \expandafter
  \endgroup
  \childdoctmp
}
%    \end{macrocode}

% \macro{\childdocforwardprefix}
% The command |\childdocforwardprefix| redirects
% compilation to the main or a child file by means of a pattern.
% The prefix |#1| in the current filename is replaced by |#2|
% and the suffix of the current filename is kept
% (it is assumed that the filename does not contain the substring `|~~~|'
% which is used as a delimiter).
% Compilation is handed over to the new file by |\childdocforward|:
%    \begin{macrocode}
\newcommand{\childdocforwardprefix}[3][]
{
  \begingroup
    \def\childdocextract #2##1~~~{\def\childdoctmp{\childdocforward[#1]{#3##1}}}
    \expandafter\childdocextract\childdocname~~~
    \expandafter
  \endgroup
  \childdoctmp
}
%    \end{macrocode}

% \macro{\childdoc}
% The deprecated macro |\childdoc| is a legacy version of |\childdocmain|:
%    \begin{macrocode}
\newcommand{\childdoc}{\childdocmain}
%    \end{macrocode}

% \macro{\childdocredirect}
% The deprecated macro |\childdocredirect| is a legacy version
% of |\childdocforward| and |\childdocforwardprefix|:
%    \begin{macrocode}
\newcommand{\childdocredirect}[2][]
{
  \begingroup
    \if?#1?
      \def\childdoctmp{\childdocforward{#2}}
    \else
      \def\childdoctmp{\childdocforwardprefix{#1}{#2}}
    \fi
    \expandafter
  \endgroup
  \childdoctmp
}
%    \end{macrocode}

%\iffalse
%</package>
%\fi
%
\endinput
|\\
|\childdocforward{|\textit{main}|}|\\
\end{tabular}
\end{center}
%
or alternatively with:
%
\begin{center}
\begin{tabular}{l}
|% \iffalse
%
% childdoc.dtx Copyright (C) 2017-2018 Niklas Beisert
%
% This work may be distributed and/or modified under the
% conditions of the LaTeX Project Public License, either version 1.3
% of this license or (at your option) any later version.
% The latest version of this license is in
%   http://www.latex-project.org/lppl.txt
% and version 1.3 or later is part of all distributions of LaTeX
% version 2005/12/01 or later.
%
% This work has the LPPL maintenance status `maintained'.
%
% The Current Maintainer of this work is Niklas Beisert.
%
% This work consists of the files childdoc.dtx and childdoc.ins
% and the derived files childdoc.def and cdocsamp.tex with
% cdocsch1.tex, cdocsch2.tex, cdocsdrf.tex, cdocsfn1.tex, cdocsfn2.tex.
%
%<package>\ifdefined\childdocmain\endinput\fi
%<package>\ProvidesFile{childdoc.def}[2018/12/30 v2.0 child document driver]
%<samplemain>\ProvidesFile{cdocsamp.tex}[2018/12/30 v2.0 sample for childdoc]
%<*driver>
%\ProvidesFile{childdoc.drv}[2018/12/30 v2.0 childdoc reference manual file]
\PassOptionsToClass{10pt,a4paper}{article}
\documentclass{ltxdoc}

\usepackage[margin=35mm]{geometry}
\usepackage{hyperref}
\usepackage{hyperxmp}
\usepackage[usenames]{color}

\hypersetup{colorlinks=true}
\hypersetup{pdfstartview=FitH}
\hypersetup{pdfpagemode=UseNone}
\hypersetup{pdfsource={}}
\hypersetup{pdflang={en-UK}}
\hypersetup{pdfcopyright={Copyright 2017-2018 Niklas Beisert.
  This work may be distributed and/or modified under the
  conditions of the LaTeX Project Public License, either version 1.3
  of this license or (at your option) any later version.}}
\hypersetup{pdflicenseurl={http://www.latex-project.org/lppl.txt}}
\hypersetup{pdfcontactaddress={ETH Zurich, ITP, HIT K,
  Wolfgang-Pauli-Strasse 27}}
\hypersetup{pdfcontactpostcode={8093}}
\hypersetup{pdfcontactcity={Zurich}}
\hypersetup{pdfcontactcountry={Switzerland}}
\hypersetup{pdfcontactemail={nbeisert@itp.phys.ethz.ch}}
\hypersetup{pdfcontacturl={http://people.phys.ethz.ch/\xmptilde nbeisert/}}

\newcommand{\secref}[1]{\hyperref[#1]{section \ref*{#1}}}

\parskip1ex
\parindent0pt
\let\olditemize\itemize
\def\itemize{\olditemize\parskip0pt}

\begin{document}

\title{The \textsf{childdoc} Package}
\hypersetup{pdftitle={The childdoc Package}}
\author{Niklas Beisert\\[2ex]
  Institut f\"ur Theoretische Physik\\
  Eidgen\"ossische Technische Hochschule Z\"urich\\
  Wolfgang-Pauli-Strasse 27, 8093 Z\"urich, Switzerland\\[1ex]
  \href{mailto:nbeisert@itp.phys.ethz.ch}
  {\texttt{nbeisert@itp.phys.ethz.ch}}}
\hypersetup{pdfauthor={Niklas Beisert}}
\hypersetup{pdfsubject={Manual for the LaTeX2e Package childdoc}}
\date{30 December 2018, \textsf{v2.0}}
\maketitle

\begin{abstract}\noindent
\textsf{childdoc} is a \LaTeXe{} package
that enables the direct compilation
of document sections included by |\include|
to individual files.
\end{abstract}

\begingroup
\parskip0ex
\tableofcontents
\endgroup

%%%%%%%%%%%%%%%%%%%%%%%%%%%%%%%%%%%%%%%%%%%%%%%%%%%%%%%%%%%%%%%%%%%%%%%%%%%%%%%%
%%%%%%%%%%%%%%%%%%%%%%%%%%%%%%%%%%%%%%%%%%%%%%%%%%%%%%%%%%%%%%%%%%%%%%%%%%%%%%%%
\section{Introduction}

\LaTeX{} provides a mechanism to structure a large document (such as a book)
into a main file and several child files (containing the chapters)
using the |\include| command.
This mechanism is beneficial for documents
which span hundreds of pages in order to
make the source file(s) more manageable.
Moreover, compilation can be restricted to
selected child files by means of the |\includeonly| command.
The latter feature can be used to reduce the compilation time while editing
(this was significantly more useful in the earlier days of \LaTeX{})
or to generate a smaller document which is easier to navigate.
Another application of |\includeonly| is to generate
documents consisting of selected parts of the complete document.

However, there are a few drawbacks of the plain |\include| mechanism:
\begin{itemize}
\item
The child files cannot be compiled on their own,
they can only be compiled via the main file.
A naive editing environment
(such as a text editor with an option
to have the current file processed by \LaTeX)
may require one to switch to the main file before compiling;
attempting to compile the child file produces errors.
\item
The main file must be modified (each time)
to adjust the |\includeonly| command
to the present needs. This easily leaves the main file in a messy state.
\item
The generated document will always carry the filename
of the main document. This is inconvenient if
several child files are to be compiled and
to be kept for distribution.
\end{itemize}

The present package provides a simple interface
to make child files individually compilable by \LaTeX{}.
Compiling a child file then has the same effect as compiling
the main file with an |\includeonly| command
to select the appropriate child.
Moreover the generated document will carry the name of the child
rather than the main file.
This resolves all three above issues.

This feature is meant to make the editing of books,
thesis documents and lecture notes somewhat more convenient.
However, the package can also be used efficiently for
composing a series of documents (such as exercise sheets)
which are typically distributed individually.
It then assists the author in generating the individual documents
(potentially in different versions)
as well as a document containing the collected series.
Another application is in developing style files
or other kinds of included material
where compilation of the style file could redirect
to a sample or test file.

%%%%%%%%%%%%%%%%%%%%%%%%%%%%%%%%%%%%%%%%%%%%%%%%%%%%%%%%%%%%%%%%%%%%%%%%%%%%%%%%
%%%%%%%%%%%%%%%%%%%%%%%%%%%%%%%%%%%%%%%%%%%%%%%%%%%%%%%%%%%%%%%%%%%%%%%%%%%%%%%%
\section{Usage}

First of all, the package \textsf{childdoc} is \emph{not} a standard
\LaTeXe{} |.sty| style file! Therefore it needs to be invoked in
a non-standard way.

%%%%%%%%%%%%%%%%%%%%%%%%%%%%%%%%%%%%%%%%%%%%%%%%%%%%%%%%%%%%%%%%%%%%%%%%%%%%%%%%
\subsection{Included Files}
\label{sec:include}

%%%%%%%%%%%%%%%%%%%%%%%%%%%%%%%%%%%%%%%%
\DescribeMacro{\childdocmain}
To use the package, add the commands
\begin{center}
\begin{tabular}{l}
|\input{childdoc.def}|\\
|\childdocmain{}|\\
\end{tabular}
\end{center}
at the very top of the main \LaTeX{} file,
in particular \emph{before} the |\documentclass| statement!
The argument of |\childdocmain| should be left empty
(but it must be present).

%%%%%%%%%%%%%%%%%%%%%%%%%%%%%%%%%%%%%%%%
\DescribeMacro{\childdocof}
Furthermore, add the commands
\begin{center}
\begin{tabular}{l}
|\input{childdoc.def}|\\
|\childdocof{|\textit{main}|}|\\
\end{tabular}
\end{center}
at the top of every child file \textit{child}
which is included by |\include{|\textit{child}|}|
from within the main file
(or at least for those files to be compiled individually).
The argument \textit{main} must be the filename of the main file.

There are a couple of
considerations in setting up the main and child documents:

%%%%%%%%%%%%%%%%%%%%%%%%%%%%%%%%%%%%%%%%
\paragraph{Restrictions.}

Please note the following restrictions:
\begin{itemize}
\item
|\childdocmain| must be called with one argument \textit{main}
to ensure compatibility with earlier version of the package.
It must either be empty (|\childdocmain{}|)
or precisely match the filename of the main file in which it is specified.
See \secref{sec:detection} for further information.
\item
The filename \textit{main} must be specified without the |.tex| extension.
\item
The filename \textit{main} is case sensitive
(even in case-insensitive file systems)
due to internal string comparison.
\item
The argument \textit{main} should be fully expanded, it cannot be a macro.
\item
Subdirectories and special characters should be avoided in filenames.
\item
The command |\childdocmain{|\textit{main}|}| must be followed by a whitespace.
It should not be followed immediately by another command
or by a comment mark `|%|'.
This is because the \TeX{} parser reads the token immediately following
the argument of |\childdocmain| and puts it
at the beginning of every child section;
however, a white\-space is ignored.
\end{itemize}

%%%%%%%%%%%%%%%%%%%%%%%%%%%%%%%%%%%%%%%%
\paragraph{Content of Main File.}

It is advisable to place all content in the child files included by |\include|.
Any output contained in the main file will appear in all child documents
unless suppressed manually;
it cannot be suppressed automatically by the |\includeonly| directive
and thus should normally be avoided.
A method to include some content in the main file
by means of conditional processing is described in \secref{sec:conditional}.

%%%%%%%%%%%%%%%%%%%%%%%%%%%%%%%%%%%%%%%%
\paragraph{Page Numbering.}

When only a part of the document is compiled,
the appropriate numbering of pages
(as well as other status parameters)
is determined from the |.aux| files.
The latter contain information from previous passes.
However this information needs to propagate through
all intermediate child documents.
Therefore the page numbering in child documents may well
be inconsistent until the complete document is compiled at least once.

A useful (if unconventional) way to always ensure a consistent
page numbering is to restart the numbering in each child document
and denote the pages by `\textit{child}|.|\textit{page}'
where \textit{child} represents the chapter/section number of the child file.
This can be achieved by the command
|\numberwithin{page}{|\textit{child}|}|
of the \textsf{amsmath} package
where \textit{child} can be |chapter| or |section|
depending on the chosen structuring.
Alternatively, one can modify the macro |\thepage| appropriately
and reset the counter |page| at the start of each child file.

%%%%%%%%%%%%%%%%%%%%%%%%%%%%%%%%%%%%%%%%%%%%%%%%%%%%%%%%%%%%%%%%%%%%%%%%%%%%%%%%
\subsection{Conditional Processing}
\label{sec:conditional}

The package provides a mechanism to compile different versions
of a document. To customise the versions further some conditional processing
can come in handy to distinguish which version is being compiled.
The package provides two macros to describe the compilation context:

%%%%%%%%%%%%%%%%%%%%%%%%%%%%%%%%%%%%%%%%
\DescribeMacro{\ifchilddoc}
The conditional |\ifchilddoc| distinguishes between the compilation of
child documents and the main document:
%
\begin{center}
|\ifchilddoc |\textit{child-code}| |[|\||else |\textit{main-code}]| \||fi|
\end{center}

%%%%%%%%%%%%%%%%%%%%%%%%%%%%%%%%%%%%%%%%
\DescribeMacro{\childdocname}
\DescribeMacro{\childdocjob}
The macro |\childdocname| contains the filename (without extension)
of the main or child file being processed.
Note that |\childdocjob| will always contain the name of the main file.

%%%%%%%%%%%%%%%%%%%%%%%%%%%%%%%%%%%%%%%%
\paragraph{Title Page.}

Conditional processing can be used to include a title or banner page
in the main document when proper precautions are taken.
Importantly, the code in the main file should ensure that the page counter
(as well as other status parameters which are stored in the |.aux| files)
takes the same value after the conditional processing.
Otherwise the page numbers may take divergent values
depending on which part is compiled.

For example, a title page could be declared by:
%
\begin{center}
\begin{tabular}{l}
|\ifchilddoc\||else|\\
|\addtocounter{page}{-1}|\\
\textit{code for title page}\\
|\newpage|\\
|\||fi|
\end{tabular}
\end{center}
%
A banner page for the child documents can be generated by:
%
\begin{center}
\begin{tabular}{l}
|\ifchilddoc|\\
|\addtocounter{page}{-1}|\\
\textit{code for banner page}\\
|\newpage|\\
|\||fi|
\end{tabular}
\end{center}
%
Here one could write a message such as:
\begin{center}
|This is the part \childdocname{} of \childdocjob{}.|
\end{center}

%%%%%%%%%%%%%%%%%%%%%%%%%%%%%%%%%%%%%%%%%%%%%%%%%%%%%%%%%%%%%%%%%%%%%%%%%%%%%%%%
\subsection{Flags}
\label{sec:flags}

The package makes it easy to generate different versions
of the main or child documents.
To this end compilation flags can be defined
and assigned different default values.
They will be particularly useful in conjunction
with the forwarding mechanism described in \secref{sec:forward}.

For example, it may be useful to have a flag |\version|
which can be set to |draft| or |final|.
The document source will contain some conditional code
depending on the value of |\version|.
Suppose further, the flag should default to |final| for the main file
and to |draft| for child files
which is a natural assignment for editing the document.
This is achieved by placing the following code
in the preamble of the main document
(below the |\childdocmain| directive):
%
\begin{center}
\begin{tabular}{l}
|\ifchilddoc|\\
|\providecommand{\version}{draft}|\\
|\||else|\\
|\providecommand{\version}{final}|\\
|\||fi|
\end{tabular}
\end{center}
%
The definition by |\providecommand| makes sure
that previous definitions are not overwritten.
Further statements |\providecommand{\version}{...}|
can thus be added before the above code to override it.

For the main file, one might add a line
(between |\childdocmain| and the above block)
%
\begin{center}
|%\ifchilddoc\||else\providecommand{\version}{draft}\||fi|
\end{center}
%
which can be uncommented to produce a draft version.
Likewise one can add a line to the very top of a child file
(above the |\childdocof{|\textit{main}|}| directive)
%
\begin{center}
|%\providecommand{\version}{final}|
\end{center}
%
which can be uncommented to produce the final version of this child document.

%%%%%%%%%%%%%%%%%%%%%%%%%%%%%%%%%%%%%%%%%%%%%%%%%%%%%%%%%%%%%%%%%%%%%%%%%%%%%%%%
\subsection{Forwarding}
\label{sec:forward}

Different versions of the main or child documents
using compilation flags as described in \secref{sec:flags}
can be (permanently) stored in different files
for convenient compilation, viewing and distribution.
To this end, the package defines a command
to pass on compilation to a different file:

%%%%%%%%%%%%%%%%%%%%%%%%%%%%%%%%%%%%%%%%
\DescribeMacro{\childdocforward}
The command |\childdocforward| redirects processing to
another source file:
%
\begin{center}
\begin{tabular}{l}
|\input{childdoc.def}|\\
|\childdocforward[|\textit{main}|]{|\textit{dest}|}|\\
\end{tabular}
\end{center}
%
The argument \textit{dest} is the destination file
(without extension).
It should be the main file or one of the child files.
Note that further \textsf{childdoc} directives
such as |\childdocof| and |\childdocforward|
in the indicated file will be processed in this form.
The optional argument \textit{main}
passes on directly to the main file \textit{main}
while pretending to compile the child \textit{dest}.
This form behaves as if \textit{dest}
issues |\childdocof{|\textit{main}|}| right away,
and no further \textsf{childdoc} directives will be processed.

%%%%%%%%%%%%%%%%%%%%%%%%%%%%%%%%%%%%%%%%
\DescribeMacro{\...prefix}
In the alternative form |\childdocforwardprefix|,
%
\begin{center}
\begin{tabular}{l}
|\input{childdoc.def}|\\
|\childdocforwardprefix[|\textit{main}|]{|\textit{prefix}|}{|\textit{dest}|}|
\end{tabular}
\end{center}
%
the destination file is determined by a pattern
depending on the current file:
To make this work, the current file must be called
`{\textit{prefix}\hspace{0.2em}\textit{suffix}}'
with \textit{prefix} matching precisely the argument.
Processing is then passed on to the file
`{\textit{dest}\hspace{0.2em}\textit{suffix}}'.
Surely, the same effect is achieved by
directly specifying the
argument `{\textit{dest}\hspace{0.2em}\textit{suffix}}'
in the first form.
However, that requires to set up a different file
for each child. With the alternative form of the command
all these files can have exactly the same content
which simplifies setting them up and maintaining them.

For example, the following file |draft.tex|
with a compilation flag |\version| as described in \secref{sec:flags}
compiles the main document as a draft:
%
\begin{center}
\begin{tabular}{l}
|\def\version{draft}|\\
|\input{childdoc.def}|\\
|\childdocforward{|\textit{main}|}|
\end{tabular}
\end{center}
%
Likewise, the following files |final|\textit{nn}|.tex|
compile the final version of the child document
|child|\textit{nn}|.tex|:
%
\begin{center}
\begin{tabular}{l}
|\def\version{final}|\\
|\input{childdoc.def}|\\
|\childdocforwardprefix{final}{child}|
\end{tabular}
\end{center}
%

Note that when several versions of a main file and/or of each child file
are to be generated, it may be convenient to set up a |Makefile| or
shell script to automatise the process.

%%%%%%%%%%%%%%%%%%%%%%%%%%%%%%%%%%%%%%%%%%%%%%%%%%%%%%%%%%%%%%%%%%%%%%%%%%%%%%%%
\subsection{Command Line Processing}
\label{sec:commandline}

The effect of redirection files can also be achieved by invoking
the \LaTeX{} compiler with a more elaborate command line.
Most conveniently this should be done as part
of a shell script or a |Makefile|.

When using \textsf{childdoc} in the main file, the following
command lines effectively perform a redirection
(note that depending on the shell being used,
backslashes may have to be doubled: `|\|' $\to$ `|\\|'):
%
\begin{center}
|... -jobname "|\textit{target}|" |\\|"|[\textit{flags}]%
|\input{childdoc.def}\childdocforward[|\textit{main}|]{|\textit{dest}|}"|
\end{center}
%
Here \textit{target} is the name of the output file,
\textit{main} is the name of the main file
and \textit{dest} is the name of the main or child file to be processed
(all filenames without extensions).
The optional argument \textit{main} can be omitted
if \textit{main} matches \textit{dest}.
Optionally, compilation \textit{flags} can be defined via |\def| commands.
This command line makes the \TeX{} engine believe
it is compiling the file \textit{target}
whose content is specified as the latter parameter.
The provided code then forwards the processing to
\textit{main} or \textit{dest} as described in \secref{sec:forward}.

%%%%%%%%%%%%%%%%%%%%%%%%%%%%%%%%%%%%%%%%%%%%%%%%%%%%%%%%%%%%%%%%%%%%%%%%%%%%%%%%
\subsection{Include by Input}
\label{sec:input}

Including child documents by |\include| has some restrictions by design.
Most notably, the content of a child document always occupies
its own set of pages; pages cannot be shared between child documents.
Usually, this behaviour makes perfect sense
because each child document contain an essential part of the document.
However, in some situations it may be desirable to compose
a document from a collection of parts
without having mandatory page breaks between then.
For this case, the package
provides a mechanism to include parts
by |\input| which can also be processed individually.
However, by construction this mechanism
requires manual handling of the content to be output.

%%%%%%%%%%%%%%%%%%%%%%%%%%%%%%%%%%%%%%%%
\DescribeMacro{\ifchilddocmanual}
The main file should be prepared as usual, see \secref{sec:include}.
However, the document body must make a distinction
between processing of an individual part and of the main document, e.g.:
%
\begin{center}
\begin{tabular}{l}
|\ifchilddocmanual|\\
|\input{\childdocname}|\\
|\||else|\\
\textit{document body with }|\input{|\textit{part}|}|\\
|\||fi|
\end{tabular}
\end{center}
%
The conditional |\ifchilddocmanual| is true whenever
a part to be included by |\input| is being compiled,
and the name of the part is stored in |\childdocname|.

%%%%%%%%%%%%%%%%%%%%%%%%%%%%%%%%%%%%%%%%
\DescribeMacro{\childdocby}
Each part to be included by |\input| should start with:
%
\begin{center}
\begin{tabular}{l}
|\input{childdoc.def}|\\
|\childdocby{|\textit{main}|}|\\
\end{tabular}
\end{center}
%
The directive |\childdocby| is similar to |\childdocof|
described in \secref{sec:include},
but the subsequent selection of content must be done manually.
To that end, both |\ifchilddoc| and |\ifchilddocmanual|
will be true upon processing of a part,
and the name of the part is stored in |\childdocname|.
Note that |\jobname| will be set to the filename of the current part
so that each part receives an individual |.aux| file
that does not interfere with the |.aux| file(s) of the main document.
This behaviour can be altered by the alternative form
|\childdocby[*]{|\textit{main}|}| (with a non-empty optional argument)
which uses the |.aux| file of the main document
by setting |\jobname| to \textit{main}.

%%%%%%%%%%%%%%%%%%%%%%%%%%%%%%%%%%%%%%%%%%%%%%%%%%%%%%%%%%%%%%%%%%%%%%%%%%%%%%%%
\subsection{Driver Development}
\label{sec:driver}

The \textsf{childdoc} mechanism can also be use for the development
of definition files such as \LaTeX{} styles or classes.
This case differs from the above setup with multiple parts
included by |\include| in that no |\includeonly| should be invoked.
This can be achieved by starting the include file
(before |\ProvidesPackage|) with:
%
\begin{center}
\begin{tabular}{l}
|\input{childdoc.def}|\\
|\childdocforward{|\textit{main}|}|\\
\end{tabular}
\end{center}
%
or alternatively with:
%
\begin{center}
\begin{tabular}{l}
|\input{childdoc.def}|\\
|\childdocby{|\textit{main}|}|\\
\end{tabular}
\end{center}
%
Both forms have slightly different effects as described above.
The main file is prepared as usual, see \secref{sec:include}.

%%%%%%%%%%%%%%%%%%%%%%%%%%%%%%%%%%%%%%%%%%%%%%%%%%%%%%%%%%%%%%%%%%%%%%%%%%%%%%%%
\subsection{Legacy Detection}
\label{sec:detection}

The directive |\childdocmain| in the main file can detect
whether the complete document or merely a child is to be compiled
even without using the directive |\childdocof|.
This method is deprecated because it is less robust
and there is no compelling reason to use it;
it is merely provided for backward compatibility
and it may be removed in future versions.

If the detection mechanism is to be used,
it is mandatory to correctly specify
the filename of the main file as the argument of |\childdocmain|:
%
\begin{center}
\begin{tabular}{l}
|\input{childdoc.def}|\\
|\childdocmain{|\textit{main}|}|\\
\end{tabular}
\end{center}
%
If |\jobname| does not match the argument \textit{main} of |\childdocmain|,
it is assumed that |\jobname| points to the child file to be compiled.
When using |\childdocmain| with the main file specified as argument,
it suffices to start a child file
with just |\input{|\textit{main}|}|
without loading of the package and using |\childdocof|.
If instead all processing is done
with the appropriate \textsf{childdoc} directives,
the argument of \textit{main} of |\childdocmain| can be empty.

An alternative version of the command line processing described
in \secref{sec:commandline} using the detection mechanism reads:
%
\begin{center}
|... -jobname "|\textit{target}|" "|[\textit{flags}]%
[|\def\jobname{|\textit{dest}|}|]|\input{|\textit{main}|}"|
\end{center}

%%%%%%%%%%%%%%%%%%%%%%%%%%%%%%%%%%%%%%%%%%%%%%%%%%%%%%%%%%%%%%%%%%%%%%%%%%%%%%%%
\subsection{Manual Code}
\label{sec:manual}

In case one cannot be certain whether the definitions file |childdoc.def|
is installed on the target \TeX{} distribution
and one prefers not to ship it,
it is conceivable to paste a few relevant commands into the sources.

To that end, drop all statements |\input{childdoc.def}|
and perform the replacements as outlined below.
Instead of |\childdocmain{|\textit{main}|}| add the following code
to the top of the main file:
%
\begin{center}
\begin{tabular}{l}
|\||ifdefined\childdocname\endinput\||fi\newif\ifchilddoc|\\
|\edef\childdocname{\scantokens\expandafter{\jobname\noexpand}}|\\
|\def\childdocmain{|\textit{main}|}\||ifx\childdocmain\childdocname\||else|\\
|\childdoctrue\includeonly{\childdocname}\let\jobname\childdocmain\||fi|\\
\end{tabular}
\end{center}
%
Instead of |\childdocof{|\textit{main}|}| just include the main file
at the top of each child file:
%
\begin{center}
|\input{|\textit{main}|}|
\end{center}
%
A simple redirection |\childdocforward{|\textit{dest}|}| is achieved by:
%
\begin{center}
|\def\jobname{|\textit{dest}|}\input{\jobname}|
\end{center}
%
The redirection with prefix
|\childdocforwardprefix[|\textit{prefix}|]{|\textit{dest}|}|
is accomplished by:
%
\begin{center}
\begin{tabular}{l}
|{\edef\jobname{\scantokens\expandafter{\jobname\noexpand}}|\\
|\def\redirectjob |\textit{prefix}|#1~~~{\gdef\jobname{|\textit{dest}|#1}}|\\
|\expandafter\redirectjob\jobname~~~}\input{\jobname}|
\end{tabular}
\end{center}

In an alternative approach,
child documents can be compiled by a specific command line
without additional code or specific definitions:
%
\begin{center}
|... -jobname "|\textit{target}|" "|[\textit{flags}]%
|\includeonly{|\textit{dest}|}\input{|\textit{main}|}"|
\end{center}
%

%%%%%%%%%%%%%%%%%%%%%%%%%%%%%%%%%%%%%%%%%%%%%%%%%%%%%%%%%%%%%%%%%%%%%%%%%%%%%%%%
%%%%%%%%%%%%%%%%%%%%%%%%%%%%%%%%%%%%%%%%%%%%%%%%%%%%%%%%%%%%%%%%%%%%%%%%%%%%%%%%
\section{Information}

%%%%%%%%%%%%%%%%%%%%%%%%%%%%%%%%%%%%%%%%%%%%%%%%%%%%%%%%%%%%%%%%%%%%%%%%%%%%%%%%
\subsection{Copyright}

Copyright \copyright{} 2017--2018 Niklas Beisert

This work may be distributed and/or modified under the
conditions of the \LaTeX{} Project Public License, either version 1.3
of this license or (at your option) any later version.
The latest version of this license is in
  \url{http://www.latex-project.org/lppl.txt}
and version 1.3 or later is part of all distributions of \LaTeX{}
version 2005/12/01 or later.

This work has the LPPL maintenance status `maintained'.

The Current Maintainer of this work is Niklas Beisert.

This work consists of the files |README.txt|, |childdoc.ins| and |childdoc.dtx|
as well as the derived files |childdoc.def|, |cdocsamp.tex|
with |cdocsch1.tex|, |cdocsch2.tex|, |cdocspt3.tex|, |cdocspt4.tex|,
|cdocsdrf.tex|, |cdocsfn1.tex|, |cdocsfn2.tex|
as well as |childdoc.pdf|.

%%%%%%%%%%%%%%%%%%%%%%%%%%%%%%%%%%%%%%%%%%%%%%%%%%%%%%%%%%%%%%%%%%%%%%%%%%%%%%%%
\subsection{Files and Installation}

The package consists of the files:
%
\begin{center}
\begin{tabular}{ll}
    |README.txt|   & readme file \\
    |childdoc.ins| & installation file \\
    |childdoc.dtx| & source file \\
    |childdoc.def| & definition file \\
    |cdocsamp.tex| & sample main file \\
    |cdocsch1.tex| & sample include file \\
    |cdocsch2.tex| & sample include file \\
    |cdocspt3.tex| & sample part file \\
    |cdocspt4.tex| & sample part file \\
    |cdocsdrf.tex| & sample redirection file \\
    |cdocsfn1.tex| & sample redirection file \\
    |cdocsfn2.tex| & sample redirection file \\
    |childdoc.pdf| & manual
\end{tabular}
\end{center}
%
The distribution consists of the files
|README.txt|, |childdoc.ins| and |childdoc.dtx|.
%
\begin{itemize}
\item
Run (pdf)\LaTeX{} on |childdoc.dtx|
to compile the manual |childdoc.pdf| (this file).
\item
Run \LaTeX{} on |childdoc.ins| to create the definitions file |childdoc.def|
and the sample |cdocsamp.tex| with include files
|cdocsch1.tex|, |cdocsch2.tex|, |cdocspt3.tex|, |cdocspt4.tex|,
|cdocsdrf.tex|, |cdocsfn1.tex|, |cdocsfn2.tex|.
Then copy the file |childdoc.def| to an appropriate directory of your \LaTeX{}
distribution, e.g.\ \textit{texmf-root}|/tex/latex/childdoc|.
\end{itemize}

%%%%%%%%%%%%%%%%%%%%%%%%%%%%%%%%%%%%%%%%%%%%%%%%%%%%%%%%%%%%%%%%%%%%%%%%%%%%%%%%
\subsection{Related CTAN Packages}

There are several other packages which offer a similar functionality:
%
\begin{itemize}
\item
The packages
\href{http://ctan.org/pkg/docmute}{\textsf{docmute}},
\href{http://ctan.org/pkg/includex}{\textsf{includex}} and
\href{http://ctan.org/pkg/standalone}{\textsf{standalone}}
provide commands to include only the document body of
a child file thus allowing both files to be compiled individually.
\item
The packages \href{http://ctan.org/pkg/subdocs}{\textsf{subdocs}}
and \href{http://ctan.org/pkg/subfiles}{\textsf{subfiles}}
provide structures in which the main and child documents can be
encapsulated and allowing them to be compiled individually.
The inclusion mechanism is different from the conventional |\include|.
\item
The package \href{http://ctan.org/pkg/combine}{\textsf{combine}}
is an elaborate solution to combine several documents into one.
\end{itemize}
%
See also the CTAN topic \href{http://ctan.org/topic/subdocs}{\textsf{subdocs}}
for further related packages.
The present package differs from the above solutions in that
a document structure constructed with the conventional |\include| mechanism
just needs two extra commands at the top of every file
such that all constituent files can be compiled individually.

%%%%%%%%%%%%%%%%%%%%%%%%%%%%%%%%%%%%%%%%%%%%%%%%%%%%%%%%%%%%%%%%%%%%%%%%%%%%%%%%
%\subsection{Feature Suggestions}
%
%The following is a list of features which may be useful for future
%versions of this package:
%%
%\begin{itemize}
%\item
%\ldots
%\end{itemize}

%%%%%%%%%%%%%%%%%%%%%%%%%%%%%%%%%%%%%%%%%%%%%%%%%%%%%%%%%%%%%%%%%%%%%%%%%%%%%%%%
\subsection{Revision History}

%%%%%%%%%%%%%%%%%%%%%%%%%%%%%%%%%%%%%%%%
\paragraph{v2.0:} 2018/12/30

\begin{itemize}
\item
immediate forward processing
\item
added |\childdocby| mechanism
\item
manual restructured
\end{itemize}

%%%%%%%%%%%%%%%%%%%%%%%%%%%%%%%%%%%%%%%%
\paragraph{v1.6:} 2018/01/17

\begin{itemize}
\item
application for development of include files
\item
corrections to manual
\end{itemize}

%%%%%%%%%%%%%%%%%%%%%%%%%%%%%%%%%%%%%%%%
\paragraph{v1.5:} 2017/05/21

\begin{itemize}
\item
more complete structuring introduced
\item
|\childdocof| introduced
\item
|\childdoc| renamed to |\childdocmain|
\item
|\childredirect| renamed to |\childdocforward| and |\childdocforwardprefix|
and functionality expanded
\end{itemize}

%%%%%%%%%%%%%%%%%%%%%%%%%%%%%%%%%%%%%%%%
\paragraph{v1.0:} 2017/04/27

\begin{itemize}
\item
manual and install package
\item
first version published on CTAN
\end{itemize}

%%%%%%%%%%%%%%%%%%%%%%%%%%%%%%%%%%%%%%%%
\paragraph{v0.6:} 2017/04/26

\begin{itemize}
\item
redirection mechanism added
\end{itemize}

%%%%%%%%%%%%%%%%%%%%%%%%%%%%%%%%%%%%%%%%
\paragraph{v0.5:} 2017/04/26

\begin{itemize}
\item
functionality in definition file
\end{itemize}


%%%%%%%%%%%%%%%%%%%%%%%%%%%%%%%%%%%%%%%%%%%%%%%%%%%%%%%%%%%%%%%%%%%%%%%%%%%%%%%%
%%%%%%%%%%%%%%%%%%%%%%%%%%%%%%%%%%%%%%%%%%%%%%%%%%%%%%%%%%%%%%%%%%%%%%%%%%%%%%%%
%%%%%%%%%%%%%%%%%%%%%%%%%%%%%%%%%%%%%%%%%%%%%%%%%%%%%%%%%%%%%%%%%%%%%%%%%%%%%%%%
\appendix

\settowidth\MacroIndent{\rmfamily\scriptsize 000\ }

 \DocInput{childdoc.dtx}

\end{document}
%</driver>
% \fi
%
% %%%%%%%%%%%%%%%%%%%%%%%%%%%%%%%%%%%%%%%%%%%%%%%%%%%%%%%%%%%%%%%%%%%%%%%%%%%%%%
% %%%%%%%%%%%%%%%%%%%%%%%%%%%%%%%%%%%%%%%%%%%%%%%%%%%%%%%%%%%%%%%%%%%%%%%%%%%%%%
% \section{Sample}
%\iffalse
%<*samplemain>
%\fi
%
% The following presents a sample document
% with two chapters, two parts, a title page,
% a compile flag as well as three forwarding files to set the flag.
% It consists of eight |.tex| files:
% \begin{center}
% \begin{tabular}{ll}
% |cdocsamp.tex|&main file\\
% |cdocsch1.tex|&include file for chapter 1\\
% |cdocsch2.tex|&include file for chapter 2\\
% |cdocspt3.tex|&include file for part 3\\
% |cdocspt4.tex|&include file for part 4\\
% |cdocsdrf.tex|&forwarding file for main file in draft mode\\
% |cdocsfi1.tex|&forwarding file for final version of chapter 1\\
% |cdocsfi2.tex|&forwarding file for final version of chapter 2\\
% \end{tabular}
% \end{center}
% Each of the eight files can be compiled directly by the \LaTeX{} compiler.
%
% %%%%%%%%%%%%%%%%%%%%%%%%%%%%%%%%%%%%%%
% \paragraph{Main File.}
%
% The main file is called |cdocsamp.tex|.
%
% Load the \textsf{childdoc} definitions and
% declare the filename for the main document:
%    \begin{macrocode}
\input{childdoc.def}
\childdocmain{}
%    \end{macrocode}

% Optional override for |\version| flag:
%    \begin{macrocode}
%%\ifchilddoc\else\providecommand{\version}{draft}\fi
%    \end{macrocode}

% Define the default values for the |\version| flag
% (|final| for the main file and |draft| for childs):
%    \begin{macrocode}
\ifchilddoc
\providecommand{\version}{draft}
\else
\providecommand{\version}{final}
\fi
%    \end{macrocode}

% Load the standard document class:
%    \begin{macrocode}
\documentclass[12pt]{article}
%    \end{macrocode}

% Start the document body:
%    \begin{macrocode}
\begin{document}
%    \end{macrocode}

% Declare a title page.
% Print title, part of document being processed and version flag:
%    \begin{macrocode}
\addtocounter{page}{-1}
\begin{center}
{\LARGE\bfseries{}childdoc example\par}
\vspace{1cm}
\ifchilddoc
\ifchilddocmanual part\else chapter\fi:
`\childdocname' of `\childdocjob'\par
\else
main document: `\childdocjob'\par
\fi
version: \version\par
\end{center}
\newpage
%    \end{macrocode}

% Manually include selected file,
% otherwise process as usual:
%    \begin{macrocode}
\ifchilddocmanual
\section*{part `\childdocname'}
\input{\childdocname}
\else
%    \end{macrocode}

% Include the two chapters:
%    \begin{macrocode}
\include{cdocsch1}
\include{cdocsch2}
%    \end{macrocode}

% Include the two parts unless only chapters should be displayed:
%    \begin{macrocode}
\ifchilddoc\else
\section{part three}
\input{cdocspt3}
\section{part four}
\input{cdocspt4}
\fi
%    \end{macrocode}

% Process as usual until here:
%    \begin{macrocode}
\fi
%    \end{macrocode}

% End of document body:
%    \begin{macrocode}
\end{document}
%    \end{macrocode}
%\iffalse
%</samplemain>
%\fi
%
% %%%%%%%%%%%%%%%%%%%%%%%%%%%%%%%%%%%%%%
% \paragraph{Chapter Include Files.}
%
% The include files are called |cdocsch1.tex| and |cdocsch2.tex|.
%
%\iffalse
%<*samplechap1|samplechap2>
%\fi

% Optional override for |\version| flag:
%    \begin{macrocode}
%%\providecommand{\version}{final}
%    \end{macrocode}

% Include the main document:
%    \begin{macrocode}
\input{childdoc.def}
\childdocof{cdocsamp}
%    \end{macrocode}

%\iffalse
%</samplechap1|samplechap2>
%\fi
%
%\iffalse
%<*samplechap1>
%\fi
% Some text for chapter 1:
%    \begin{macrocode}
\section{one}
some text in chapter one
%    \end{macrocode}

%\iffalse
%</samplechap1>
%\fi
% Some text for chapter 2:
%\iffalse
%<*samplechap2>
%\fi
%    \begin{macrocode}
\section{two}
more text in chapter two
%    \end{macrocode}

%\iffalse
%</samplechap2>
%\fi
%
% %%%%%%%%%%%%%%%%%%%%%%%%%%%%%%%%%%%%%%
% \paragraph{Part Include Files.}
%
% The include files are called |cdocspt3.tex| and |cdocspt4.tex|.
%
%\iffalse
%<*samplepart3|samplepart4>
%\fi

% Optional override for |\version| flag:
%    \begin{macrocode}
%%\providecommand{\version}{final}
%    \end{macrocode}

% Include the main document:
%    \begin{macrocode}
\input{childdoc.def}
\childdocby{cdocsamp}
%    \end{macrocode}

%\iffalse
%</samplepart3|samplepart4>
%\fi
%
%\iffalse
%<*samplepart3>
%\fi
% Some text for part 3:
%    \begin{macrocode}
some text in part three
%    \end{macrocode}

%\iffalse
%</samplepart3>
%\fi
% Some text for part 4:
%\iffalse
%<*samplepart4>
%\fi
%    \begin{macrocode}
more text in part four
%    \end{macrocode}

%\iffalse
%</samplepart4>
%\fi
%
% %%%%%%%%%%%%%%%%%%%%%%%%%%%%%%%%%%%%%%
% \paragraph{Forwarding for a Complete Draft.}
%
% The following forwarding file |cdocsdrf.tex|
% compiles the main document in draft mode:
%\iffalse
%<*sampledraft>
%\fi
%    \begin{macrocode}
\def\version{draft}
\input{childdoc.def}
\childdocforward{cdocsamp}
%    \end{macrocode}

%\iffalse
%</sampledraft>
%\fi
%
% %%%%%%%%%%%%%%%%%%%%%%%%%%%%%%%%%%%%%%
% \paragraph{Forwarding for Final Version of the Chapters.}
%
% The following forwarding files |cdocsfn1.tex| and |cdocsfn2.tex|
% (with identical content)
% compile the final versions of the child documents
% |cdocsch1.tex| and |cdocsch2.tex|, respectively:
%\iffalse
%<*samplefinal>
%\fi
%    \begin{macrocode}
\def\version{final}
\input{childdoc.def}
\childdocforwardprefix[cdocsamp]{cdocsfn}{cdocsch}
%    \end{macrocode}

%\iffalse
%</samplefinal>
%\fi
%
% %%%%%%%%%%%%%%%%%%%%%%%%%%%%%%%%%%%%%%
% \paragraph{Command Line Processing.}
%
% The following three command lines generate the output files
% |cdocscld|, |cdocscl1| and |cdocscl2|
% which should be identical to
% |cdocsdrf|, |cdocsch1| and |cdocsfn2|, respectively:
% \begin{center}
% \begin{tabular}{l}
% |latex -jobname cdocscld \|\\
% |  "\def\version{draft}\input{childdoc.def}\childdocforward{cdocsamp}"|\\
% |latex -jobname cdocscl1 \|\\
% |  "\input{childdoc.def}\childdocforward[cdocsamp]{cdocsch1}"|\\
% |latex -jobname cdocscl2 \|\\
% |  "\def\version{final}\input{childdoc.def}\childdocforward{cdocsch2}"|
% \end{tabular}
% \end{center}
% Note that the trailing backslash on each first line
% merely continues the input to the second line
% (for convenient cut ant paste).
% Furthermore, the command |latex| can be replaced by any
% of its alternative versions such as |pdflatex|.
%
% %%%%%%%%%%%%%%%%%%%%%%%%%%%%%%%%%%%%%%%%%%%%%%%%%%%%%%%%%%%%%%%%%%%%%%%%%%%%%%
% %%%%%%%%%%%%%%%%%%%%%%%%%%%%%%%%%%%%%%%%%%%%%%%%%%%%%%%%%%%%%%%%%%%%%%%%%%%%%%
% \section{Implementation}
%\iffalse
%<*package>
%\fi
%
% This section describes the definitions file |childdoc.def|.

% The definitions cannot be loaded using |\usepackage| or |\RequirePackage|
% which has a mechanism to prevent loading a style file more than once.
% When loading the definitions by means of |\input|
% multiple instances have to be prevented manually:
%\iffalse
%This code needs to be before the `\ProvidesFile' directive
%which is defined at the beginning of this file.
%Therefore it is also placed there and commented out here.
%</package>
%<*discard>
%\fi
%    \begin{macrocode}
\ifdefined\childdocmain\endinput\fi
%    \end{macrocode}
%\iffalse
%</discard>
%<*package>
%\fi
%
% \macro{\ifchilddoc}
% \macro{\ifchilddocmanual}
% The conditional |\ifchilddoc| tells whether a
% child (true) or main (false) document is being compiled.
% The conditional |\ifchilddocmanual| tells whether
% the |\includeonly| mechanism is used (false) or
% the selection of child files must be performed manually (true).
% The definitions initialise to false:
%    \begin{macrocode}
\newif\ifchilddoc
\newif\ifchilddocmanual
%    \end{macrocode}

% \macro{\childdocname}
% \macro{\childdocjob}
% The macro |\childdocname| stores the name of the main document
% to be compiled. The macro |\childdocjob| stores the name of
% the document on which the \LaTeX{} compiler was originally invoked.
% The content of |\jobname| cannot be compared
% to filenames specified in the source due to different catcodes.
% The following code rescans |\jobname|, stores the result
% in |\childdocname| and saves a copy in |\childdocjob|:
%    \begin{macrocode}
\edef\childdocname{\scantokens\expandafter{\jobname\noexpand}}
\let\childdocjob\childdocname
%    \end{macrocode}

% \macro{\childdocdisable}
% The macro |\childdocdisable| prevents the main file
% from being processed more than once.
% At this stage, the main document command |\childdocmain|
% is assumed to be called once again where it should do nothing.
% Any subsequent call to it should prevent
% a secondary processing of the main document
% It overwrites the forwarding commands
% |\childdocof| and |\childdocforward|
% with empty macros to prevent further inclusions of the main document:
%    \begin{macrocode}
\newcommand{\childdocdisable}
{
  \renewcommand{\childdocmain}[1]{\renewcommand{\childdocmain}[1]{\endinput}}
  \renewcommand{\childdocof}[1]{}
  \renewcommand{\childdocby}[2][]{}
  \renewcommand{\childdocforward}[2][]{}
  \renewcommand{\childdocdisable}{}
}
%    \end{macrocode}

% \macro{\childdocmain}
% The macro |\childdocmain| is to be called at the top of the main file
% with nothing or the main filename (without extension) as argument.
% First, it breaks loops.
% If the argument is not empty and does not match |\childdocname|
% (which is set by the first inclusion of |childdoc.def|),
% |\ifchilddoc| is set to true, |\includeonly| is applied to the child file
% and |\jobname| is set to the main file
% (for proper handling of |.aux| files):
%    \begin{macrocode}
\newcommand{\childdocmain}[1]
{
  \childdocdisable\childdocmain{}
  \if?#1?\else
    \begingroup
      \def\childdoctmp{#1}
      \ifx\childdoctmp\childdocname
        \def\childdoctmp{}
      \else
        \def\childdoctmp
        {
          \childdoctrue
          \includeonly{\childdocname}
          \def\childdocjob{#1}
          \def\jobname{#1}
        }
      \fi
      \expandafter
    \endgroup
    \childdoctmp
  \fi
}
%    \end{macrocode}

% \macro{\childdocof}
% The command |\childdocof| redirects
% compilation to the main file |#1|.
%    \begin{macrocode}
\newcommand{\childdocof}[1]
{
  \childdocdisable
  \childdoctrue
  \includeonly{\childdocname}
  \def\jobname{#1}
  \def\childdocjob{#1}
  \input{#1}
}
%    \end{macrocode}

% \macro{\childdocby}
% The command |\childdocby| ....
%    \begin{macrocode}
\newcommand{\childdocby}[2][]
{
  \childdocdisable
  \childdoctrue
  \childdocmanualtrue
  \if?#1?\else
    \def\jobname{#2}
  \fi
  \def\childdocjob{#2}
  \input{#2}
  \endinput
}
%    \end{macrocode}

% \macro{\childdocforward}
% The command |\childdocforward| redirects
% compilation to the main file or
% (if the optional argument is given) a child file.
% Parameters are set as if the main file
% or a child file starting with |\childdocof| was compiled.
% Then compilation is handed over to the main file:
%    \begin{macrocode}
\newcommand{\childdocforward}[2][]
{
  \begingroup
    \if?#1?
      \def\childdoctmp
      {
        \def\childdocname{#2}
        \def\childdocjob{#2}
        \def\jobname{#2}
        \input{#2}
        \endinput
      }
    \else
      \def\childdoctmp
      {
        \childdocdisable
        \def\childdocname{#2}
        \childdoctrue
        \includeonly{#2}
        \def\childdocjob{#1}
        \def\jobname{#1}
        \input{#1}
        \endinput
      }
    \fi
    \expandafter
  \endgroup
  \childdoctmp
}
%    \end{macrocode}

% \macro{\childdocforwardprefix}
% The command |\childdocforwardprefix| redirects
% compilation to the main or a child file by means of a pattern.
% The prefix |#1| in the current filename is replaced by |#2|
% and the suffix of the current filename is kept
% (it is assumed that the filename does not contain the substring `|~~~|'
% which is used as a delimiter).
% Compilation is handed over to the new file by |\childdocforward|:
%    \begin{macrocode}
\newcommand{\childdocforwardprefix}[3][]
{
  \begingroup
    \def\childdocextract #2##1~~~{\def\childdoctmp{\childdocforward[#1]{#3##1}}}
    \expandafter\childdocextract\childdocname~~~
    \expandafter
  \endgroup
  \childdoctmp
}
%    \end{macrocode}

% \macro{\childdoc}
% The deprecated macro |\childdoc| is a legacy version of |\childdocmain|:
%    \begin{macrocode}
\newcommand{\childdoc}{\childdocmain}
%    \end{macrocode}

% \macro{\childdocredirect}
% The deprecated macro |\childdocredirect| is a legacy version
% of |\childdocforward| and |\childdocforwardprefix|:
%    \begin{macrocode}
\newcommand{\childdocredirect}[2][]
{
  \begingroup
    \if?#1?
      \def\childdoctmp{\childdocforward{#2}}
    \else
      \def\childdoctmp{\childdocforwardprefix{#1}{#2}}
    \fi
    \expandafter
  \endgroup
  \childdoctmp
}
%    \end{macrocode}

%\iffalse
%</package>
%\fi
%
\endinput
|\\
|\childdocby{|\textit{main}|}|\\
\end{tabular}
\end{center}
%
Both forms have slightly different effects as described above.
The main file is prepared as usual, see \secref{sec:include}.

%%%%%%%%%%%%%%%%%%%%%%%%%%%%%%%%%%%%%%%%%%%%%%%%%%%%%%%%%%%%%%%%%%%%%%%%%%%%%%%%
\subsection{Legacy Detection}
\label{sec:detection}

The directive |\childdocmain| in the main file can detect
whether the complete document or merely a child is to be compiled
even without using the directive |\childdocof|.
This method is deprecated because it is less robust
and there is no compelling reason to use it;
it is merely provided for backward compatibility
and it may be removed in future versions.

If the detection mechanism is to be used,
it is mandatory to correctly specify
the filename of the main file as the argument of |\childdocmain|:
%
\begin{center}
\begin{tabular}{l}
|% \iffalse
%
% childdoc.dtx Copyright (C) 2017-2018 Niklas Beisert
%
% This work may be distributed and/or modified under the
% conditions of the LaTeX Project Public License, either version 1.3
% of this license or (at your option) any later version.
% The latest version of this license is in
%   http://www.latex-project.org/lppl.txt
% and version 1.3 or later is part of all distributions of LaTeX
% version 2005/12/01 or later.
%
% This work has the LPPL maintenance status `maintained'.
%
% The Current Maintainer of this work is Niklas Beisert.
%
% This work consists of the files childdoc.dtx and childdoc.ins
% and the derived files childdoc.def and cdocsamp.tex with
% cdocsch1.tex, cdocsch2.tex, cdocsdrf.tex, cdocsfn1.tex, cdocsfn2.tex.
%
%<package>\ifdefined\childdocmain\endinput\fi
%<package>\ProvidesFile{childdoc.def}[2018/12/30 v2.0 child document driver]
%<samplemain>\ProvidesFile{cdocsamp.tex}[2018/12/30 v2.0 sample for childdoc]
%<*driver>
%\ProvidesFile{childdoc.drv}[2018/12/30 v2.0 childdoc reference manual file]
\PassOptionsToClass{10pt,a4paper}{article}
\documentclass{ltxdoc}

\usepackage[margin=35mm]{geometry}
\usepackage{hyperref}
\usepackage{hyperxmp}
\usepackage[usenames]{color}

\hypersetup{colorlinks=true}
\hypersetup{pdfstartview=FitH}
\hypersetup{pdfpagemode=UseNone}
\hypersetup{pdfsource={}}
\hypersetup{pdflang={en-UK}}
\hypersetup{pdfcopyright={Copyright 2017-2018 Niklas Beisert.
  This work may be distributed and/or modified under the
  conditions of the LaTeX Project Public License, either version 1.3
  of this license or (at your option) any later version.}}
\hypersetup{pdflicenseurl={http://www.latex-project.org/lppl.txt}}
\hypersetup{pdfcontactaddress={ETH Zurich, ITP, HIT K,
  Wolfgang-Pauli-Strasse 27}}
\hypersetup{pdfcontactpostcode={8093}}
\hypersetup{pdfcontactcity={Zurich}}
\hypersetup{pdfcontactcountry={Switzerland}}
\hypersetup{pdfcontactemail={nbeisert@itp.phys.ethz.ch}}
\hypersetup{pdfcontacturl={http://people.phys.ethz.ch/\xmptilde nbeisert/}}

\newcommand{\secref}[1]{\hyperref[#1]{section \ref*{#1}}}

\parskip1ex
\parindent0pt
\let\olditemize\itemize
\def\itemize{\olditemize\parskip0pt}

\begin{document}

\title{The \textsf{childdoc} Package}
\hypersetup{pdftitle={The childdoc Package}}
\author{Niklas Beisert\\[2ex]
  Institut f\"ur Theoretische Physik\\
  Eidgen\"ossische Technische Hochschule Z\"urich\\
  Wolfgang-Pauli-Strasse 27, 8093 Z\"urich, Switzerland\\[1ex]
  \href{mailto:nbeisert@itp.phys.ethz.ch}
  {\texttt{nbeisert@itp.phys.ethz.ch}}}
\hypersetup{pdfauthor={Niklas Beisert}}
\hypersetup{pdfsubject={Manual for the LaTeX2e Package childdoc}}
\date{30 December 2018, \textsf{v2.0}}
\maketitle

\begin{abstract}\noindent
\textsf{childdoc} is a \LaTeXe{} package
that enables the direct compilation
of document sections included by |\include|
to individual files.
\end{abstract}

\begingroup
\parskip0ex
\tableofcontents
\endgroup

%%%%%%%%%%%%%%%%%%%%%%%%%%%%%%%%%%%%%%%%%%%%%%%%%%%%%%%%%%%%%%%%%%%%%%%%%%%%%%%%
%%%%%%%%%%%%%%%%%%%%%%%%%%%%%%%%%%%%%%%%%%%%%%%%%%%%%%%%%%%%%%%%%%%%%%%%%%%%%%%%
\section{Introduction}

\LaTeX{} provides a mechanism to structure a large document (such as a book)
into a main file and several child files (containing the chapters)
using the |\include| command.
This mechanism is beneficial for documents
which span hundreds of pages in order to
make the source file(s) more manageable.
Moreover, compilation can be restricted to
selected child files by means of the |\includeonly| command.
The latter feature can be used to reduce the compilation time while editing
(this was significantly more useful in the earlier days of \LaTeX{})
or to generate a smaller document which is easier to navigate.
Another application of |\includeonly| is to generate
documents consisting of selected parts of the complete document.

However, there are a few drawbacks of the plain |\include| mechanism:
\begin{itemize}
\item
The child files cannot be compiled on their own,
they can only be compiled via the main file.
A naive editing environment
(such as a text editor with an option
to have the current file processed by \LaTeX)
may require one to switch to the main file before compiling;
attempting to compile the child file produces errors.
\item
The main file must be modified (each time)
to adjust the |\includeonly| command
to the present needs. This easily leaves the main file in a messy state.
\item
The generated document will always carry the filename
of the main document. This is inconvenient if
several child files are to be compiled and
to be kept for distribution.
\end{itemize}

The present package provides a simple interface
to make child files individually compilable by \LaTeX{}.
Compiling a child file then has the same effect as compiling
the main file with an |\includeonly| command
to select the appropriate child.
Moreover the generated document will carry the name of the child
rather than the main file.
This resolves all three above issues.

This feature is meant to make the editing of books,
thesis documents and lecture notes somewhat more convenient.
However, the package can also be used efficiently for
composing a series of documents (such as exercise sheets)
which are typically distributed individually.
It then assists the author in generating the individual documents
(potentially in different versions)
as well as a document containing the collected series.
Another application is in developing style files
or other kinds of included material
where compilation of the style file could redirect
to a sample or test file.

%%%%%%%%%%%%%%%%%%%%%%%%%%%%%%%%%%%%%%%%%%%%%%%%%%%%%%%%%%%%%%%%%%%%%%%%%%%%%%%%
%%%%%%%%%%%%%%%%%%%%%%%%%%%%%%%%%%%%%%%%%%%%%%%%%%%%%%%%%%%%%%%%%%%%%%%%%%%%%%%%
\section{Usage}

First of all, the package \textsf{childdoc} is \emph{not} a standard
\LaTeXe{} |.sty| style file! Therefore it needs to be invoked in
a non-standard way.

%%%%%%%%%%%%%%%%%%%%%%%%%%%%%%%%%%%%%%%%%%%%%%%%%%%%%%%%%%%%%%%%%%%%%%%%%%%%%%%%
\subsection{Included Files}
\label{sec:include}

%%%%%%%%%%%%%%%%%%%%%%%%%%%%%%%%%%%%%%%%
\DescribeMacro{\childdocmain}
To use the package, add the commands
\begin{center}
\begin{tabular}{l}
|\input{childdoc.def}|\\
|\childdocmain{}|\\
\end{tabular}
\end{center}
at the very top of the main \LaTeX{} file,
in particular \emph{before} the |\documentclass| statement!
The argument of |\childdocmain| should be left empty
(but it must be present).

%%%%%%%%%%%%%%%%%%%%%%%%%%%%%%%%%%%%%%%%
\DescribeMacro{\childdocof}
Furthermore, add the commands
\begin{center}
\begin{tabular}{l}
|\input{childdoc.def}|\\
|\childdocof{|\textit{main}|}|\\
\end{tabular}
\end{center}
at the top of every child file \textit{child}
which is included by |\include{|\textit{child}|}|
from within the main file
(or at least for those files to be compiled individually).
The argument \textit{main} must be the filename of the main file.

There are a couple of
considerations in setting up the main and child documents:

%%%%%%%%%%%%%%%%%%%%%%%%%%%%%%%%%%%%%%%%
\paragraph{Restrictions.}

Please note the following restrictions:
\begin{itemize}
\item
|\childdocmain| must be called with one argument \textit{main}
to ensure compatibility with earlier version of the package.
It must either be empty (|\childdocmain{}|)
or precisely match the filename of the main file in which it is specified.
See \secref{sec:detection} for further information.
\item
The filename \textit{main} must be specified without the |.tex| extension.
\item
The filename \textit{main} is case sensitive
(even in case-insensitive file systems)
due to internal string comparison.
\item
The argument \textit{main} should be fully expanded, it cannot be a macro.
\item
Subdirectories and special characters should be avoided in filenames.
\item
The command |\childdocmain{|\textit{main}|}| must be followed by a whitespace.
It should not be followed immediately by another command
or by a comment mark `|%|'.
This is because the \TeX{} parser reads the token immediately following
the argument of |\childdocmain| and puts it
at the beginning of every child section;
however, a white\-space is ignored.
\end{itemize}

%%%%%%%%%%%%%%%%%%%%%%%%%%%%%%%%%%%%%%%%
\paragraph{Content of Main File.}

It is advisable to place all content in the child files included by |\include|.
Any output contained in the main file will appear in all child documents
unless suppressed manually;
it cannot be suppressed automatically by the |\includeonly| directive
and thus should normally be avoided.
A method to include some content in the main file
by means of conditional processing is described in \secref{sec:conditional}.

%%%%%%%%%%%%%%%%%%%%%%%%%%%%%%%%%%%%%%%%
\paragraph{Page Numbering.}

When only a part of the document is compiled,
the appropriate numbering of pages
(as well as other status parameters)
is determined from the |.aux| files.
The latter contain information from previous passes.
However this information needs to propagate through
all intermediate child documents.
Therefore the page numbering in child documents may well
be inconsistent until the complete document is compiled at least once.

A useful (if unconventional) way to always ensure a consistent
page numbering is to restart the numbering in each child document
and denote the pages by `\textit{child}|.|\textit{page}'
where \textit{child} represents the chapter/section number of the child file.
This can be achieved by the command
|\numberwithin{page}{|\textit{child}|}|
of the \textsf{amsmath} package
where \textit{child} can be |chapter| or |section|
depending on the chosen structuring.
Alternatively, one can modify the macro |\thepage| appropriately
and reset the counter |page| at the start of each child file.

%%%%%%%%%%%%%%%%%%%%%%%%%%%%%%%%%%%%%%%%%%%%%%%%%%%%%%%%%%%%%%%%%%%%%%%%%%%%%%%%
\subsection{Conditional Processing}
\label{sec:conditional}

The package provides a mechanism to compile different versions
of a document. To customise the versions further some conditional processing
can come in handy to distinguish which version is being compiled.
The package provides two macros to describe the compilation context:

%%%%%%%%%%%%%%%%%%%%%%%%%%%%%%%%%%%%%%%%
\DescribeMacro{\ifchilddoc}
The conditional |\ifchilddoc| distinguishes between the compilation of
child documents and the main document:
%
\begin{center}
|\ifchilddoc |\textit{child-code}| |[|\||else |\textit{main-code}]| \||fi|
\end{center}

%%%%%%%%%%%%%%%%%%%%%%%%%%%%%%%%%%%%%%%%
\DescribeMacro{\childdocname}
\DescribeMacro{\childdocjob}
The macro |\childdocname| contains the filename (without extension)
of the main or child file being processed.
Note that |\childdocjob| will always contain the name of the main file.

%%%%%%%%%%%%%%%%%%%%%%%%%%%%%%%%%%%%%%%%
\paragraph{Title Page.}

Conditional processing can be used to include a title or banner page
in the main document when proper precautions are taken.
Importantly, the code in the main file should ensure that the page counter
(as well as other status parameters which are stored in the |.aux| files)
takes the same value after the conditional processing.
Otherwise the page numbers may take divergent values
depending on which part is compiled.

For example, a title page could be declared by:
%
\begin{center}
\begin{tabular}{l}
|\ifchilddoc\||else|\\
|\addtocounter{page}{-1}|\\
\textit{code for title page}\\
|\newpage|\\
|\||fi|
\end{tabular}
\end{center}
%
A banner page for the child documents can be generated by:
%
\begin{center}
\begin{tabular}{l}
|\ifchilddoc|\\
|\addtocounter{page}{-1}|\\
\textit{code for banner page}\\
|\newpage|\\
|\||fi|
\end{tabular}
\end{center}
%
Here one could write a message such as:
\begin{center}
|This is the part \childdocname{} of \childdocjob{}.|
\end{center}

%%%%%%%%%%%%%%%%%%%%%%%%%%%%%%%%%%%%%%%%%%%%%%%%%%%%%%%%%%%%%%%%%%%%%%%%%%%%%%%%
\subsection{Flags}
\label{sec:flags}

The package makes it easy to generate different versions
of the main or child documents.
To this end compilation flags can be defined
and assigned different default values.
They will be particularly useful in conjunction
with the forwarding mechanism described in \secref{sec:forward}.

For example, it may be useful to have a flag |\version|
which can be set to |draft| or |final|.
The document source will contain some conditional code
depending on the value of |\version|.
Suppose further, the flag should default to |final| for the main file
and to |draft| for child files
which is a natural assignment for editing the document.
This is achieved by placing the following code
in the preamble of the main document
(below the |\childdocmain| directive):
%
\begin{center}
\begin{tabular}{l}
|\ifchilddoc|\\
|\providecommand{\version}{draft}|\\
|\||else|\\
|\providecommand{\version}{final}|\\
|\||fi|
\end{tabular}
\end{center}
%
The definition by |\providecommand| makes sure
that previous definitions are not overwritten.
Further statements |\providecommand{\version}{...}|
can thus be added before the above code to override it.

For the main file, one might add a line
(between |\childdocmain| and the above block)
%
\begin{center}
|%\ifchilddoc\||else\providecommand{\version}{draft}\||fi|
\end{center}
%
which can be uncommented to produce a draft version.
Likewise one can add a line to the very top of a child file
(above the |\childdocof{|\textit{main}|}| directive)
%
\begin{center}
|%\providecommand{\version}{final}|
\end{center}
%
which can be uncommented to produce the final version of this child document.

%%%%%%%%%%%%%%%%%%%%%%%%%%%%%%%%%%%%%%%%%%%%%%%%%%%%%%%%%%%%%%%%%%%%%%%%%%%%%%%%
\subsection{Forwarding}
\label{sec:forward}

Different versions of the main or child documents
using compilation flags as described in \secref{sec:flags}
can be (permanently) stored in different files
for convenient compilation, viewing and distribution.
To this end, the package defines a command
to pass on compilation to a different file:

%%%%%%%%%%%%%%%%%%%%%%%%%%%%%%%%%%%%%%%%
\DescribeMacro{\childdocforward}
The command |\childdocforward| redirects processing to
another source file:
%
\begin{center}
\begin{tabular}{l}
|\input{childdoc.def}|\\
|\childdocforward[|\textit{main}|]{|\textit{dest}|}|\\
\end{tabular}
\end{center}
%
The argument \textit{dest} is the destination file
(without extension).
It should be the main file or one of the child files.
Note that further \textsf{childdoc} directives
such as |\childdocof| and |\childdocforward|
in the indicated file will be processed in this form.
The optional argument \textit{main}
passes on directly to the main file \textit{main}
while pretending to compile the child \textit{dest}.
This form behaves as if \textit{dest}
issues |\childdocof{|\textit{main}|}| right away,
and no further \textsf{childdoc} directives will be processed.

%%%%%%%%%%%%%%%%%%%%%%%%%%%%%%%%%%%%%%%%
\DescribeMacro{\...prefix}
In the alternative form |\childdocforwardprefix|,
%
\begin{center}
\begin{tabular}{l}
|\input{childdoc.def}|\\
|\childdocforwardprefix[|\textit{main}|]{|\textit{prefix}|}{|\textit{dest}|}|
\end{tabular}
\end{center}
%
the destination file is determined by a pattern
depending on the current file:
To make this work, the current file must be called
`{\textit{prefix}\hspace{0.2em}\textit{suffix}}'
with \textit{prefix} matching precisely the argument.
Processing is then passed on to the file
`{\textit{dest}\hspace{0.2em}\textit{suffix}}'.
Surely, the same effect is achieved by
directly specifying the
argument `{\textit{dest}\hspace{0.2em}\textit{suffix}}'
in the first form.
However, that requires to set up a different file
for each child. With the alternative form of the command
all these files can have exactly the same content
which simplifies setting them up and maintaining them.

For example, the following file |draft.tex|
with a compilation flag |\version| as described in \secref{sec:flags}
compiles the main document as a draft:
%
\begin{center}
\begin{tabular}{l}
|\def\version{draft}|\\
|\input{childdoc.def}|\\
|\childdocforward{|\textit{main}|}|
\end{tabular}
\end{center}
%
Likewise, the following files |final|\textit{nn}|.tex|
compile the final version of the child document
|child|\textit{nn}|.tex|:
%
\begin{center}
\begin{tabular}{l}
|\def\version{final}|\\
|\input{childdoc.def}|\\
|\childdocforwardprefix{final}{child}|
\end{tabular}
\end{center}
%

Note that when several versions of a main file and/or of each child file
are to be generated, it may be convenient to set up a |Makefile| or
shell script to automatise the process.

%%%%%%%%%%%%%%%%%%%%%%%%%%%%%%%%%%%%%%%%%%%%%%%%%%%%%%%%%%%%%%%%%%%%%%%%%%%%%%%%
\subsection{Command Line Processing}
\label{sec:commandline}

The effect of redirection files can also be achieved by invoking
the \LaTeX{} compiler with a more elaborate command line.
Most conveniently this should be done as part
of a shell script or a |Makefile|.

When using \textsf{childdoc} in the main file, the following
command lines effectively perform a redirection
(note that depending on the shell being used,
backslashes may have to be doubled: `|\|' $\to$ `|\\|'):
%
\begin{center}
|... -jobname "|\textit{target}|" |\\|"|[\textit{flags}]%
|\input{childdoc.def}\childdocforward[|\textit{main}|]{|\textit{dest}|}"|
\end{center}
%
Here \textit{target} is the name of the output file,
\textit{main} is the name of the main file
and \textit{dest} is the name of the main or child file to be processed
(all filenames without extensions).
The optional argument \textit{main} can be omitted
if \textit{main} matches \textit{dest}.
Optionally, compilation \textit{flags} can be defined via |\def| commands.
This command line makes the \TeX{} engine believe
it is compiling the file \textit{target}
whose content is specified as the latter parameter.
The provided code then forwards the processing to
\textit{main} or \textit{dest} as described in \secref{sec:forward}.

%%%%%%%%%%%%%%%%%%%%%%%%%%%%%%%%%%%%%%%%%%%%%%%%%%%%%%%%%%%%%%%%%%%%%%%%%%%%%%%%
\subsection{Include by Input}
\label{sec:input}

Including child documents by |\include| has some restrictions by design.
Most notably, the content of a child document always occupies
its own set of pages; pages cannot be shared between child documents.
Usually, this behaviour makes perfect sense
because each child document contain an essential part of the document.
However, in some situations it may be desirable to compose
a document from a collection of parts
without having mandatory page breaks between then.
For this case, the package
provides a mechanism to include parts
by |\input| which can also be processed individually.
However, by construction this mechanism
requires manual handling of the content to be output.

%%%%%%%%%%%%%%%%%%%%%%%%%%%%%%%%%%%%%%%%
\DescribeMacro{\ifchilddocmanual}
The main file should be prepared as usual, see \secref{sec:include}.
However, the document body must make a distinction
between processing of an individual part and of the main document, e.g.:
%
\begin{center}
\begin{tabular}{l}
|\ifchilddocmanual|\\
|\input{\childdocname}|\\
|\||else|\\
\textit{document body with }|\input{|\textit{part}|}|\\
|\||fi|
\end{tabular}
\end{center}
%
The conditional |\ifchilddocmanual| is true whenever
a part to be included by |\input| is being compiled,
and the name of the part is stored in |\childdocname|.

%%%%%%%%%%%%%%%%%%%%%%%%%%%%%%%%%%%%%%%%
\DescribeMacro{\childdocby}
Each part to be included by |\input| should start with:
%
\begin{center}
\begin{tabular}{l}
|\input{childdoc.def}|\\
|\childdocby{|\textit{main}|}|\\
\end{tabular}
\end{center}
%
The directive |\childdocby| is similar to |\childdocof|
described in \secref{sec:include},
but the subsequent selection of content must be done manually.
To that end, both |\ifchilddoc| and |\ifchilddocmanual|
will be true upon processing of a part,
and the name of the part is stored in |\childdocname|.
Note that |\jobname| will be set to the filename of the current part
so that each part receives an individual |.aux| file
that does not interfere with the |.aux| file(s) of the main document.
This behaviour can be altered by the alternative form
|\childdocby[*]{|\textit{main}|}| (with a non-empty optional argument)
which uses the |.aux| file of the main document
by setting |\jobname| to \textit{main}.

%%%%%%%%%%%%%%%%%%%%%%%%%%%%%%%%%%%%%%%%%%%%%%%%%%%%%%%%%%%%%%%%%%%%%%%%%%%%%%%%
\subsection{Driver Development}
\label{sec:driver}

The \textsf{childdoc} mechanism can also be use for the development
of definition files such as \LaTeX{} styles or classes.
This case differs from the above setup with multiple parts
included by |\include| in that no |\includeonly| should be invoked.
This can be achieved by starting the include file
(before |\ProvidesPackage|) with:
%
\begin{center}
\begin{tabular}{l}
|\input{childdoc.def}|\\
|\childdocforward{|\textit{main}|}|\\
\end{tabular}
\end{center}
%
or alternatively with:
%
\begin{center}
\begin{tabular}{l}
|\input{childdoc.def}|\\
|\childdocby{|\textit{main}|}|\\
\end{tabular}
\end{center}
%
Both forms have slightly different effects as described above.
The main file is prepared as usual, see \secref{sec:include}.

%%%%%%%%%%%%%%%%%%%%%%%%%%%%%%%%%%%%%%%%%%%%%%%%%%%%%%%%%%%%%%%%%%%%%%%%%%%%%%%%
\subsection{Legacy Detection}
\label{sec:detection}

The directive |\childdocmain| in the main file can detect
whether the complete document or merely a child is to be compiled
even without using the directive |\childdocof|.
This method is deprecated because it is less robust
and there is no compelling reason to use it;
it is merely provided for backward compatibility
and it may be removed in future versions.

If the detection mechanism is to be used,
it is mandatory to correctly specify
the filename of the main file as the argument of |\childdocmain|:
%
\begin{center}
\begin{tabular}{l}
|\input{childdoc.def}|\\
|\childdocmain{|\textit{main}|}|\\
\end{tabular}
\end{center}
%
If |\jobname| does not match the argument \textit{main} of |\childdocmain|,
it is assumed that |\jobname| points to the child file to be compiled.
When using |\childdocmain| with the main file specified as argument,
it suffices to start a child file
with just |\input{|\textit{main}|}|
without loading of the package and using |\childdocof|.
If instead all processing is done
with the appropriate \textsf{childdoc} directives,
the argument of \textit{main} of |\childdocmain| can be empty.

An alternative version of the command line processing described
in \secref{sec:commandline} using the detection mechanism reads:
%
\begin{center}
|... -jobname "|\textit{target}|" "|[\textit{flags}]%
[|\def\jobname{|\textit{dest}|}|]|\input{|\textit{main}|}"|
\end{center}

%%%%%%%%%%%%%%%%%%%%%%%%%%%%%%%%%%%%%%%%%%%%%%%%%%%%%%%%%%%%%%%%%%%%%%%%%%%%%%%%
\subsection{Manual Code}
\label{sec:manual}

In case one cannot be certain whether the definitions file |childdoc.def|
is installed on the target \TeX{} distribution
and one prefers not to ship it,
it is conceivable to paste a few relevant commands into the sources.

To that end, drop all statements |\input{childdoc.def}|
and perform the replacements as outlined below.
Instead of |\childdocmain{|\textit{main}|}| add the following code
to the top of the main file:
%
\begin{center}
\begin{tabular}{l}
|\||ifdefined\childdocname\endinput\||fi\newif\ifchilddoc|\\
|\edef\childdocname{\scantokens\expandafter{\jobname\noexpand}}|\\
|\def\childdocmain{|\textit{main}|}\||ifx\childdocmain\childdocname\||else|\\
|\childdoctrue\includeonly{\childdocname}\let\jobname\childdocmain\||fi|\\
\end{tabular}
\end{center}
%
Instead of |\childdocof{|\textit{main}|}| just include the main file
at the top of each child file:
%
\begin{center}
|\input{|\textit{main}|}|
\end{center}
%
A simple redirection |\childdocforward{|\textit{dest}|}| is achieved by:
%
\begin{center}
|\def\jobname{|\textit{dest}|}\input{\jobname}|
\end{center}
%
The redirection with prefix
|\childdocforwardprefix[|\textit{prefix}|]{|\textit{dest}|}|
is accomplished by:
%
\begin{center}
\begin{tabular}{l}
|{\edef\jobname{\scantokens\expandafter{\jobname\noexpand}}|\\
|\def\redirectjob |\textit{prefix}|#1~~~{\gdef\jobname{|\textit{dest}|#1}}|\\
|\expandafter\redirectjob\jobname~~~}\input{\jobname}|
\end{tabular}
\end{center}

In an alternative approach,
child documents can be compiled by a specific command line
without additional code or specific definitions:
%
\begin{center}
|... -jobname "|\textit{target}|" "|[\textit{flags}]%
|\includeonly{|\textit{dest}|}\input{|\textit{main}|}"|
\end{center}
%

%%%%%%%%%%%%%%%%%%%%%%%%%%%%%%%%%%%%%%%%%%%%%%%%%%%%%%%%%%%%%%%%%%%%%%%%%%%%%%%%
%%%%%%%%%%%%%%%%%%%%%%%%%%%%%%%%%%%%%%%%%%%%%%%%%%%%%%%%%%%%%%%%%%%%%%%%%%%%%%%%
\section{Information}

%%%%%%%%%%%%%%%%%%%%%%%%%%%%%%%%%%%%%%%%%%%%%%%%%%%%%%%%%%%%%%%%%%%%%%%%%%%%%%%%
\subsection{Copyright}

Copyright \copyright{} 2017--2018 Niklas Beisert

This work may be distributed and/or modified under the
conditions of the \LaTeX{} Project Public License, either version 1.3
of this license or (at your option) any later version.
The latest version of this license is in
  \url{http://www.latex-project.org/lppl.txt}
and version 1.3 or later is part of all distributions of \LaTeX{}
version 2005/12/01 or later.

This work has the LPPL maintenance status `maintained'.

The Current Maintainer of this work is Niklas Beisert.

This work consists of the files |README.txt|, |childdoc.ins| and |childdoc.dtx|
as well as the derived files |childdoc.def|, |cdocsamp.tex|
with |cdocsch1.tex|, |cdocsch2.tex|, |cdocspt3.tex|, |cdocspt4.tex|,
|cdocsdrf.tex|, |cdocsfn1.tex|, |cdocsfn2.tex|
as well as |childdoc.pdf|.

%%%%%%%%%%%%%%%%%%%%%%%%%%%%%%%%%%%%%%%%%%%%%%%%%%%%%%%%%%%%%%%%%%%%%%%%%%%%%%%%
\subsection{Files and Installation}

The package consists of the files:
%
\begin{center}
\begin{tabular}{ll}
    |README.txt|   & readme file \\
    |childdoc.ins| & installation file \\
    |childdoc.dtx| & source file \\
    |childdoc.def| & definition file \\
    |cdocsamp.tex| & sample main file \\
    |cdocsch1.tex| & sample include file \\
    |cdocsch2.tex| & sample include file \\
    |cdocspt3.tex| & sample part file \\
    |cdocspt4.tex| & sample part file \\
    |cdocsdrf.tex| & sample redirection file \\
    |cdocsfn1.tex| & sample redirection file \\
    |cdocsfn2.tex| & sample redirection file \\
    |childdoc.pdf| & manual
\end{tabular}
\end{center}
%
The distribution consists of the files
|README.txt|, |childdoc.ins| and |childdoc.dtx|.
%
\begin{itemize}
\item
Run (pdf)\LaTeX{} on |childdoc.dtx|
to compile the manual |childdoc.pdf| (this file).
\item
Run \LaTeX{} on |childdoc.ins| to create the definitions file |childdoc.def|
and the sample |cdocsamp.tex| with include files
|cdocsch1.tex|, |cdocsch2.tex|, |cdocspt3.tex|, |cdocspt4.tex|,
|cdocsdrf.tex|, |cdocsfn1.tex|, |cdocsfn2.tex|.
Then copy the file |childdoc.def| to an appropriate directory of your \LaTeX{}
distribution, e.g.\ \textit{texmf-root}|/tex/latex/childdoc|.
\end{itemize}

%%%%%%%%%%%%%%%%%%%%%%%%%%%%%%%%%%%%%%%%%%%%%%%%%%%%%%%%%%%%%%%%%%%%%%%%%%%%%%%%
\subsection{Related CTAN Packages}

There are several other packages which offer a similar functionality:
%
\begin{itemize}
\item
The packages
\href{http://ctan.org/pkg/docmute}{\textsf{docmute}},
\href{http://ctan.org/pkg/includex}{\textsf{includex}} and
\href{http://ctan.org/pkg/standalone}{\textsf{standalone}}
provide commands to include only the document body of
a child file thus allowing both files to be compiled individually.
\item
The packages \href{http://ctan.org/pkg/subdocs}{\textsf{subdocs}}
and \href{http://ctan.org/pkg/subfiles}{\textsf{subfiles}}
provide structures in which the main and child documents can be
encapsulated and allowing them to be compiled individually.
The inclusion mechanism is different from the conventional |\include|.
\item
The package \href{http://ctan.org/pkg/combine}{\textsf{combine}}
is an elaborate solution to combine several documents into one.
\end{itemize}
%
See also the CTAN topic \href{http://ctan.org/topic/subdocs}{\textsf{subdocs}}
for further related packages.
The present package differs from the above solutions in that
a document structure constructed with the conventional |\include| mechanism
just needs two extra commands at the top of every file
such that all constituent files can be compiled individually.

%%%%%%%%%%%%%%%%%%%%%%%%%%%%%%%%%%%%%%%%%%%%%%%%%%%%%%%%%%%%%%%%%%%%%%%%%%%%%%%%
%\subsection{Feature Suggestions}
%
%The following is a list of features which may be useful for future
%versions of this package:
%%
%\begin{itemize}
%\item
%\ldots
%\end{itemize}

%%%%%%%%%%%%%%%%%%%%%%%%%%%%%%%%%%%%%%%%%%%%%%%%%%%%%%%%%%%%%%%%%%%%%%%%%%%%%%%%
\subsection{Revision History}

%%%%%%%%%%%%%%%%%%%%%%%%%%%%%%%%%%%%%%%%
\paragraph{v2.0:} 2018/12/30

\begin{itemize}
\item
immediate forward processing
\item
added |\childdocby| mechanism
\item
manual restructured
\end{itemize}

%%%%%%%%%%%%%%%%%%%%%%%%%%%%%%%%%%%%%%%%
\paragraph{v1.6:} 2018/01/17

\begin{itemize}
\item
application for development of include files
\item
corrections to manual
\end{itemize}

%%%%%%%%%%%%%%%%%%%%%%%%%%%%%%%%%%%%%%%%
\paragraph{v1.5:} 2017/05/21

\begin{itemize}
\item
more complete structuring introduced
\item
|\childdocof| introduced
\item
|\childdoc| renamed to |\childdocmain|
\item
|\childredirect| renamed to |\childdocforward| and |\childdocforwardprefix|
and functionality expanded
\end{itemize}

%%%%%%%%%%%%%%%%%%%%%%%%%%%%%%%%%%%%%%%%
\paragraph{v1.0:} 2017/04/27

\begin{itemize}
\item
manual and install package
\item
first version published on CTAN
\end{itemize}

%%%%%%%%%%%%%%%%%%%%%%%%%%%%%%%%%%%%%%%%
\paragraph{v0.6:} 2017/04/26

\begin{itemize}
\item
redirection mechanism added
\end{itemize}

%%%%%%%%%%%%%%%%%%%%%%%%%%%%%%%%%%%%%%%%
\paragraph{v0.5:} 2017/04/26

\begin{itemize}
\item
functionality in definition file
\end{itemize}


%%%%%%%%%%%%%%%%%%%%%%%%%%%%%%%%%%%%%%%%%%%%%%%%%%%%%%%%%%%%%%%%%%%%%%%%%%%%%%%%
%%%%%%%%%%%%%%%%%%%%%%%%%%%%%%%%%%%%%%%%%%%%%%%%%%%%%%%%%%%%%%%%%%%%%%%%%%%%%%%%
%%%%%%%%%%%%%%%%%%%%%%%%%%%%%%%%%%%%%%%%%%%%%%%%%%%%%%%%%%%%%%%%%%%%%%%%%%%%%%%%
\appendix

\settowidth\MacroIndent{\rmfamily\scriptsize 000\ }

 \DocInput{childdoc.dtx}

\end{document}
%</driver>
% \fi
%
% %%%%%%%%%%%%%%%%%%%%%%%%%%%%%%%%%%%%%%%%%%%%%%%%%%%%%%%%%%%%%%%%%%%%%%%%%%%%%%
% %%%%%%%%%%%%%%%%%%%%%%%%%%%%%%%%%%%%%%%%%%%%%%%%%%%%%%%%%%%%%%%%%%%%%%%%%%%%%%
% \section{Sample}
%\iffalse
%<*samplemain>
%\fi
%
% The following presents a sample document
% with two chapters, two parts, a title page,
% a compile flag as well as three forwarding files to set the flag.
% It consists of eight |.tex| files:
% \begin{center}
% \begin{tabular}{ll}
% |cdocsamp.tex|&main file\\
% |cdocsch1.tex|&include file for chapter 1\\
% |cdocsch2.tex|&include file for chapter 2\\
% |cdocspt3.tex|&include file for part 3\\
% |cdocspt4.tex|&include file for part 4\\
% |cdocsdrf.tex|&forwarding file for main file in draft mode\\
% |cdocsfi1.tex|&forwarding file for final version of chapter 1\\
% |cdocsfi2.tex|&forwarding file for final version of chapter 2\\
% \end{tabular}
% \end{center}
% Each of the eight files can be compiled directly by the \LaTeX{} compiler.
%
% %%%%%%%%%%%%%%%%%%%%%%%%%%%%%%%%%%%%%%
% \paragraph{Main File.}
%
% The main file is called |cdocsamp.tex|.
%
% Load the \textsf{childdoc} definitions and
% declare the filename for the main document:
%    \begin{macrocode}
\input{childdoc.def}
\childdocmain{}
%    \end{macrocode}

% Optional override for |\version| flag:
%    \begin{macrocode}
%%\ifchilddoc\else\providecommand{\version}{draft}\fi
%    \end{macrocode}

% Define the default values for the |\version| flag
% (|final| for the main file and |draft| for childs):
%    \begin{macrocode}
\ifchilddoc
\providecommand{\version}{draft}
\else
\providecommand{\version}{final}
\fi
%    \end{macrocode}

% Load the standard document class:
%    \begin{macrocode}
\documentclass[12pt]{article}
%    \end{macrocode}

% Start the document body:
%    \begin{macrocode}
\begin{document}
%    \end{macrocode}

% Declare a title page.
% Print title, part of document being processed and version flag:
%    \begin{macrocode}
\addtocounter{page}{-1}
\begin{center}
{\LARGE\bfseries{}childdoc example\par}
\vspace{1cm}
\ifchilddoc
\ifchilddocmanual part\else chapter\fi:
`\childdocname' of `\childdocjob'\par
\else
main document: `\childdocjob'\par
\fi
version: \version\par
\end{center}
\newpage
%    \end{macrocode}

% Manually include selected file,
% otherwise process as usual:
%    \begin{macrocode}
\ifchilddocmanual
\section*{part `\childdocname'}
\input{\childdocname}
\else
%    \end{macrocode}

% Include the two chapters:
%    \begin{macrocode}
\include{cdocsch1}
\include{cdocsch2}
%    \end{macrocode}

% Include the two parts unless only chapters should be displayed:
%    \begin{macrocode}
\ifchilddoc\else
\section{part three}
\input{cdocspt3}
\section{part four}
\input{cdocspt4}
\fi
%    \end{macrocode}

% Process as usual until here:
%    \begin{macrocode}
\fi
%    \end{macrocode}

% End of document body:
%    \begin{macrocode}
\end{document}
%    \end{macrocode}
%\iffalse
%</samplemain>
%\fi
%
% %%%%%%%%%%%%%%%%%%%%%%%%%%%%%%%%%%%%%%
% \paragraph{Chapter Include Files.}
%
% The include files are called |cdocsch1.tex| and |cdocsch2.tex|.
%
%\iffalse
%<*samplechap1|samplechap2>
%\fi

% Optional override for |\version| flag:
%    \begin{macrocode}
%%\providecommand{\version}{final}
%    \end{macrocode}

% Include the main document:
%    \begin{macrocode}
\input{childdoc.def}
\childdocof{cdocsamp}
%    \end{macrocode}

%\iffalse
%</samplechap1|samplechap2>
%\fi
%
%\iffalse
%<*samplechap1>
%\fi
% Some text for chapter 1:
%    \begin{macrocode}
\section{one}
some text in chapter one
%    \end{macrocode}

%\iffalse
%</samplechap1>
%\fi
% Some text for chapter 2:
%\iffalse
%<*samplechap2>
%\fi
%    \begin{macrocode}
\section{two}
more text in chapter two
%    \end{macrocode}

%\iffalse
%</samplechap2>
%\fi
%
% %%%%%%%%%%%%%%%%%%%%%%%%%%%%%%%%%%%%%%
% \paragraph{Part Include Files.}
%
% The include files are called |cdocspt3.tex| and |cdocspt4.tex|.
%
%\iffalse
%<*samplepart3|samplepart4>
%\fi

% Optional override for |\version| flag:
%    \begin{macrocode}
%%\providecommand{\version}{final}
%    \end{macrocode}

% Include the main document:
%    \begin{macrocode}
\input{childdoc.def}
\childdocby{cdocsamp}
%    \end{macrocode}

%\iffalse
%</samplepart3|samplepart4>
%\fi
%
%\iffalse
%<*samplepart3>
%\fi
% Some text for part 3:
%    \begin{macrocode}
some text in part three
%    \end{macrocode}

%\iffalse
%</samplepart3>
%\fi
% Some text for part 4:
%\iffalse
%<*samplepart4>
%\fi
%    \begin{macrocode}
more text in part four
%    \end{macrocode}

%\iffalse
%</samplepart4>
%\fi
%
% %%%%%%%%%%%%%%%%%%%%%%%%%%%%%%%%%%%%%%
% \paragraph{Forwarding for a Complete Draft.}
%
% The following forwarding file |cdocsdrf.tex|
% compiles the main document in draft mode:
%\iffalse
%<*sampledraft>
%\fi
%    \begin{macrocode}
\def\version{draft}
\input{childdoc.def}
\childdocforward{cdocsamp}
%    \end{macrocode}

%\iffalse
%</sampledraft>
%\fi
%
% %%%%%%%%%%%%%%%%%%%%%%%%%%%%%%%%%%%%%%
% \paragraph{Forwarding for Final Version of the Chapters.}
%
% The following forwarding files |cdocsfn1.tex| and |cdocsfn2.tex|
% (with identical content)
% compile the final versions of the child documents
% |cdocsch1.tex| and |cdocsch2.tex|, respectively:
%\iffalse
%<*samplefinal>
%\fi
%    \begin{macrocode}
\def\version{final}
\input{childdoc.def}
\childdocforwardprefix[cdocsamp]{cdocsfn}{cdocsch}
%    \end{macrocode}

%\iffalse
%</samplefinal>
%\fi
%
% %%%%%%%%%%%%%%%%%%%%%%%%%%%%%%%%%%%%%%
% \paragraph{Command Line Processing.}
%
% The following three command lines generate the output files
% |cdocscld|, |cdocscl1| and |cdocscl2|
% which should be identical to
% |cdocsdrf|, |cdocsch1| and |cdocsfn2|, respectively:
% \begin{center}
% \begin{tabular}{l}
% |latex -jobname cdocscld \|\\
% |  "\def\version{draft}\input{childdoc.def}\childdocforward{cdocsamp}"|\\
% |latex -jobname cdocscl1 \|\\
% |  "\input{childdoc.def}\childdocforward[cdocsamp]{cdocsch1}"|\\
% |latex -jobname cdocscl2 \|\\
% |  "\def\version{final}\input{childdoc.def}\childdocforward{cdocsch2}"|
% \end{tabular}
% \end{center}
% Note that the trailing backslash on each first line
% merely continues the input to the second line
% (for convenient cut ant paste).
% Furthermore, the command |latex| can be replaced by any
% of its alternative versions such as |pdflatex|.
%
% %%%%%%%%%%%%%%%%%%%%%%%%%%%%%%%%%%%%%%%%%%%%%%%%%%%%%%%%%%%%%%%%%%%%%%%%%%%%%%
% %%%%%%%%%%%%%%%%%%%%%%%%%%%%%%%%%%%%%%%%%%%%%%%%%%%%%%%%%%%%%%%%%%%%%%%%%%%%%%
% \section{Implementation}
%\iffalse
%<*package>
%\fi
%
% This section describes the definitions file |childdoc.def|.

% The definitions cannot be loaded using |\usepackage| or |\RequirePackage|
% which has a mechanism to prevent loading a style file more than once.
% When loading the definitions by means of |\input|
% multiple instances have to be prevented manually:
%\iffalse
%This code needs to be before the `\ProvidesFile' directive
%which is defined at the beginning of this file.
%Therefore it is also placed there and commented out here.
%</package>
%<*discard>
%\fi
%    \begin{macrocode}
\ifdefined\childdocmain\endinput\fi
%    \end{macrocode}
%\iffalse
%</discard>
%<*package>
%\fi
%
% \macro{\ifchilddoc}
% \macro{\ifchilddocmanual}
% The conditional |\ifchilddoc| tells whether a
% child (true) or main (false) document is being compiled.
% The conditional |\ifchilddocmanual| tells whether
% the |\includeonly| mechanism is used (false) or
% the selection of child files must be performed manually (true).
% The definitions initialise to false:
%    \begin{macrocode}
\newif\ifchilddoc
\newif\ifchilddocmanual
%    \end{macrocode}

% \macro{\childdocname}
% \macro{\childdocjob}
% The macro |\childdocname| stores the name of the main document
% to be compiled. The macro |\childdocjob| stores the name of
% the document on which the \LaTeX{} compiler was originally invoked.
% The content of |\jobname| cannot be compared
% to filenames specified in the source due to different catcodes.
% The following code rescans |\jobname|, stores the result
% in |\childdocname| and saves a copy in |\childdocjob|:
%    \begin{macrocode}
\edef\childdocname{\scantokens\expandafter{\jobname\noexpand}}
\let\childdocjob\childdocname
%    \end{macrocode}

% \macro{\childdocdisable}
% The macro |\childdocdisable| prevents the main file
% from being processed more than once.
% At this stage, the main document command |\childdocmain|
% is assumed to be called once again where it should do nothing.
% Any subsequent call to it should prevent
% a secondary processing of the main document
% It overwrites the forwarding commands
% |\childdocof| and |\childdocforward|
% with empty macros to prevent further inclusions of the main document:
%    \begin{macrocode}
\newcommand{\childdocdisable}
{
  \renewcommand{\childdocmain}[1]{\renewcommand{\childdocmain}[1]{\endinput}}
  \renewcommand{\childdocof}[1]{}
  \renewcommand{\childdocby}[2][]{}
  \renewcommand{\childdocforward}[2][]{}
  \renewcommand{\childdocdisable}{}
}
%    \end{macrocode}

% \macro{\childdocmain}
% The macro |\childdocmain| is to be called at the top of the main file
% with nothing or the main filename (without extension) as argument.
% First, it breaks loops.
% If the argument is not empty and does not match |\childdocname|
% (which is set by the first inclusion of |childdoc.def|),
% |\ifchilddoc| is set to true, |\includeonly| is applied to the child file
% and |\jobname| is set to the main file
% (for proper handling of |.aux| files):
%    \begin{macrocode}
\newcommand{\childdocmain}[1]
{
  \childdocdisable\childdocmain{}
  \if?#1?\else
    \begingroup
      \def\childdoctmp{#1}
      \ifx\childdoctmp\childdocname
        \def\childdoctmp{}
      \else
        \def\childdoctmp
        {
          \childdoctrue
          \includeonly{\childdocname}
          \def\childdocjob{#1}
          \def\jobname{#1}
        }
      \fi
      \expandafter
    \endgroup
    \childdoctmp
  \fi
}
%    \end{macrocode}

% \macro{\childdocof}
% The command |\childdocof| redirects
% compilation to the main file |#1|.
%    \begin{macrocode}
\newcommand{\childdocof}[1]
{
  \childdocdisable
  \childdoctrue
  \includeonly{\childdocname}
  \def\jobname{#1}
  \def\childdocjob{#1}
  \input{#1}
}
%    \end{macrocode}

% \macro{\childdocby}
% The command |\childdocby| ....
%    \begin{macrocode}
\newcommand{\childdocby}[2][]
{
  \childdocdisable
  \childdoctrue
  \childdocmanualtrue
  \if?#1?\else
    \def\jobname{#2}
  \fi
  \def\childdocjob{#2}
  \input{#2}
  \endinput
}
%    \end{macrocode}

% \macro{\childdocforward}
% The command |\childdocforward| redirects
% compilation to the main file or
% (if the optional argument is given) a child file.
% Parameters are set as if the main file
% or a child file starting with |\childdocof| was compiled.
% Then compilation is handed over to the main file:
%    \begin{macrocode}
\newcommand{\childdocforward}[2][]
{
  \begingroup
    \if?#1?
      \def\childdoctmp
      {
        \def\childdocname{#2}
        \def\childdocjob{#2}
        \def\jobname{#2}
        \input{#2}
        \endinput
      }
    \else
      \def\childdoctmp
      {
        \childdocdisable
        \def\childdocname{#2}
        \childdoctrue
        \includeonly{#2}
        \def\childdocjob{#1}
        \def\jobname{#1}
        \input{#1}
        \endinput
      }
    \fi
    \expandafter
  \endgroup
  \childdoctmp
}
%    \end{macrocode}

% \macro{\childdocforwardprefix}
% The command |\childdocforwardprefix| redirects
% compilation to the main or a child file by means of a pattern.
% The prefix |#1| in the current filename is replaced by |#2|
% and the suffix of the current filename is kept
% (it is assumed that the filename does not contain the substring `|~~~|'
% which is used as a delimiter).
% Compilation is handed over to the new file by |\childdocforward|:
%    \begin{macrocode}
\newcommand{\childdocforwardprefix}[3][]
{
  \begingroup
    \def\childdocextract #2##1~~~{\def\childdoctmp{\childdocforward[#1]{#3##1}}}
    \expandafter\childdocextract\childdocname~~~
    \expandafter
  \endgroup
  \childdoctmp
}
%    \end{macrocode}

% \macro{\childdoc}
% The deprecated macro |\childdoc| is a legacy version of |\childdocmain|:
%    \begin{macrocode}
\newcommand{\childdoc}{\childdocmain}
%    \end{macrocode}

% \macro{\childdocredirect}
% The deprecated macro |\childdocredirect| is a legacy version
% of |\childdocforward| and |\childdocforwardprefix|:
%    \begin{macrocode}
\newcommand{\childdocredirect}[2][]
{
  \begingroup
    \if?#1?
      \def\childdoctmp{\childdocforward{#2}}
    \else
      \def\childdoctmp{\childdocforwardprefix{#1}{#2}}
    \fi
    \expandafter
  \endgroup
  \childdoctmp
}
%    \end{macrocode}

%\iffalse
%</package>
%\fi
%
\endinput
|\\
|\childdocmain{|\textit{main}|}|\\
\end{tabular}
\end{center}
%
If |\jobname| does not match the argument \textit{main} of |\childdocmain|,
it is assumed that |\jobname| points to the child file to be compiled.
When using |\childdocmain| with the main file specified as argument,
it suffices to start a child file
with just |\input{|\textit{main}|}|
without loading of the package and using |\childdocof|.
If instead all processing is done
with the appropriate \textsf{childdoc} directives,
the argument of \textit{main} of |\childdocmain| can be empty.

An alternative version of the command line processing described
in \secref{sec:commandline} using the detection mechanism reads:
%
\begin{center}
|... -jobname "|\textit{target}|" "|[\textit{flags}]%
[|\def\jobname{|\textit{dest}|}|]|\input{|\textit{main}|}"|
\end{center}

%%%%%%%%%%%%%%%%%%%%%%%%%%%%%%%%%%%%%%%%%%%%%%%%%%%%%%%%%%%%%%%%%%%%%%%%%%%%%%%%
\subsection{Manual Code}
\label{sec:manual}

In case one cannot be certain whether the definitions file |childdoc.def|
is installed on the target \TeX{} distribution
and one prefers not to ship it,
it is conceivable to paste a few relevant commands into the sources.

To that end, drop all statements |% \iffalse
%
% childdoc.dtx Copyright (C) 2017-2018 Niklas Beisert
%
% This work may be distributed and/or modified under the
% conditions of the LaTeX Project Public License, either version 1.3
% of this license or (at your option) any later version.
% The latest version of this license is in
%   http://www.latex-project.org/lppl.txt
% and version 1.3 or later is part of all distributions of LaTeX
% version 2005/12/01 or later.
%
% This work has the LPPL maintenance status `maintained'.
%
% The Current Maintainer of this work is Niklas Beisert.
%
% This work consists of the files childdoc.dtx and childdoc.ins
% and the derived files childdoc.def and cdocsamp.tex with
% cdocsch1.tex, cdocsch2.tex, cdocsdrf.tex, cdocsfn1.tex, cdocsfn2.tex.
%
%<package>\ifdefined\childdocmain\endinput\fi
%<package>\ProvidesFile{childdoc.def}[2018/12/30 v2.0 child document driver]
%<samplemain>\ProvidesFile{cdocsamp.tex}[2018/12/30 v2.0 sample for childdoc]
%<*driver>
%\ProvidesFile{childdoc.drv}[2018/12/30 v2.0 childdoc reference manual file]
\PassOptionsToClass{10pt,a4paper}{article}
\documentclass{ltxdoc}

\usepackage[margin=35mm]{geometry}
\usepackage{hyperref}
\usepackage{hyperxmp}
\usepackage[usenames]{color}

\hypersetup{colorlinks=true}
\hypersetup{pdfstartview=FitH}
\hypersetup{pdfpagemode=UseNone}
\hypersetup{pdfsource={}}
\hypersetup{pdflang={en-UK}}
\hypersetup{pdfcopyright={Copyright 2017-2018 Niklas Beisert.
  This work may be distributed and/or modified under the
  conditions of the LaTeX Project Public License, either version 1.3
  of this license or (at your option) any later version.}}
\hypersetup{pdflicenseurl={http://www.latex-project.org/lppl.txt}}
\hypersetup{pdfcontactaddress={ETH Zurich, ITP, HIT K,
  Wolfgang-Pauli-Strasse 27}}
\hypersetup{pdfcontactpostcode={8093}}
\hypersetup{pdfcontactcity={Zurich}}
\hypersetup{pdfcontactcountry={Switzerland}}
\hypersetup{pdfcontactemail={nbeisert@itp.phys.ethz.ch}}
\hypersetup{pdfcontacturl={http://people.phys.ethz.ch/\xmptilde nbeisert/}}

\newcommand{\secref}[1]{\hyperref[#1]{section \ref*{#1}}}

\parskip1ex
\parindent0pt
\let\olditemize\itemize
\def\itemize{\olditemize\parskip0pt}

\begin{document}

\title{The \textsf{childdoc} Package}
\hypersetup{pdftitle={The childdoc Package}}
\author{Niklas Beisert\\[2ex]
  Institut f\"ur Theoretische Physik\\
  Eidgen\"ossische Technische Hochschule Z\"urich\\
  Wolfgang-Pauli-Strasse 27, 8093 Z\"urich, Switzerland\\[1ex]
  \href{mailto:nbeisert@itp.phys.ethz.ch}
  {\texttt{nbeisert@itp.phys.ethz.ch}}}
\hypersetup{pdfauthor={Niklas Beisert}}
\hypersetup{pdfsubject={Manual for the LaTeX2e Package childdoc}}
\date{30 December 2018, \textsf{v2.0}}
\maketitle

\begin{abstract}\noindent
\textsf{childdoc} is a \LaTeXe{} package
that enables the direct compilation
of document sections included by |\include|
to individual files.
\end{abstract}

\begingroup
\parskip0ex
\tableofcontents
\endgroup

%%%%%%%%%%%%%%%%%%%%%%%%%%%%%%%%%%%%%%%%%%%%%%%%%%%%%%%%%%%%%%%%%%%%%%%%%%%%%%%%
%%%%%%%%%%%%%%%%%%%%%%%%%%%%%%%%%%%%%%%%%%%%%%%%%%%%%%%%%%%%%%%%%%%%%%%%%%%%%%%%
\section{Introduction}

\LaTeX{} provides a mechanism to structure a large document (such as a book)
into a main file and several child files (containing the chapters)
using the |\include| command.
This mechanism is beneficial for documents
which span hundreds of pages in order to
make the source file(s) more manageable.
Moreover, compilation can be restricted to
selected child files by means of the |\includeonly| command.
The latter feature can be used to reduce the compilation time while editing
(this was significantly more useful in the earlier days of \LaTeX{})
or to generate a smaller document which is easier to navigate.
Another application of |\includeonly| is to generate
documents consisting of selected parts of the complete document.

However, there are a few drawbacks of the plain |\include| mechanism:
\begin{itemize}
\item
The child files cannot be compiled on their own,
they can only be compiled via the main file.
A naive editing environment
(such as a text editor with an option
to have the current file processed by \LaTeX)
may require one to switch to the main file before compiling;
attempting to compile the child file produces errors.
\item
The main file must be modified (each time)
to adjust the |\includeonly| command
to the present needs. This easily leaves the main file in a messy state.
\item
The generated document will always carry the filename
of the main document. This is inconvenient if
several child files are to be compiled and
to be kept for distribution.
\end{itemize}

The present package provides a simple interface
to make child files individually compilable by \LaTeX{}.
Compiling a child file then has the same effect as compiling
the main file with an |\includeonly| command
to select the appropriate child.
Moreover the generated document will carry the name of the child
rather than the main file.
This resolves all three above issues.

This feature is meant to make the editing of books,
thesis documents and lecture notes somewhat more convenient.
However, the package can also be used efficiently for
composing a series of documents (such as exercise sheets)
which are typically distributed individually.
It then assists the author in generating the individual documents
(potentially in different versions)
as well as a document containing the collected series.
Another application is in developing style files
or other kinds of included material
where compilation of the style file could redirect
to a sample or test file.

%%%%%%%%%%%%%%%%%%%%%%%%%%%%%%%%%%%%%%%%%%%%%%%%%%%%%%%%%%%%%%%%%%%%%%%%%%%%%%%%
%%%%%%%%%%%%%%%%%%%%%%%%%%%%%%%%%%%%%%%%%%%%%%%%%%%%%%%%%%%%%%%%%%%%%%%%%%%%%%%%
\section{Usage}

First of all, the package \textsf{childdoc} is \emph{not} a standard
\LaTeXe{} |.sty| style file! Therefore it needs to be invoked in
a non-standard way.

%%%%%%%%%%%%%%%%%%%%%%%%%%%%%%%%%%%%%%%%%%%%%%%%%%%%%%%%%%%%%%%%%%%%%%%%%%%%%%%%
\subsection{Included Files}
\label{sec:include}

%%%%%%%%%%%%%%%%%%%%%%%%%%%%%%%%%%%%%%%%
\DescribeMacro{\childdocmain}
To use the package, add the commands
\begin{center}
\begin{tabular}{l}
|\input{childdoc.def}|\\
|\childdocmain{}|\\
\end{tabular}
\end{center}
at the very top of the main \LaTeX{} file,
in particular \emph{before} the |\documentclass| statement!
The argument of |\childdocmain| should be left empty
(but it must be present).

%%%%%%%%%%%%%%%%%%%%%%%%%%%%%%%%%%%%%%%%
\DescribeMacro{\childdocof}
Furthermore, add the commands
\begin{center}
\begin{tabular}{l}
|\input{childdoc.def}|\\
|\childdocof{|\textit{main}|}|\\
\end{tabular}
\end{center}
at the top of every child file \textit{child}
which is included by |\include{|\textit{child}|}|
from within the main file
(or at least for those files to be compiled individually).
The argument \textit{main} must be the filename of the main file.

There are a couple of
considerations in setting up the main and child documents:

%%%%%%%%%%%%%%%%%%%%%%%%%%%%%%%%%%%%%%%%
\paragraph{Restrictions.}

Please note the following restrictions:
\begin{itemize}
\item
|\childdocmain| must be called with one argument \textit{main}
to ensure compatibility with earlier version of the package.
It must either be empty (|\childdocmain{}|)
or precisely match the filename of the main file in which it is specified.
See \secref{sec:detection} for further information.
\item
The filename \textit{main} must be specified without the |.tex| extension.
\item
The filename \textit{main} is case sensitive
(even in case-insensitive file systems)
due to internal string comparison.
\item
The argument \textit{main} should be fully expanded, it cannot be a macro.
\item
Subdirectories and special characters should be avoided in filenames.
\item
The command |\childdocmain{|\textit{main}|}| must be followed by a whitespace.
It should not be followed immediately by another command
or by a comment mark `|%|'.
This is because the \TeX{} parser reads the token immediately following
the argument of |\childdocmain| and puts it
at the beginning of every child section;
however, a white\-space is ignored.
\end{itemize}

%%%%%%%%%%%%%%%%%%%%%%%%%%%%%%%%%%%%%%%%
\paragraph{Content of Main File.}

It is advisable to place all content in the child files included by |\include|.
Any output contained in the main file will appear in all child documents
unless suppressed manually;
it cannot be suppressed automatically by the |\includeonly| directive
and thus should normally be avoided.
A method to include some content in the main file
by means of conditional processing is described in \secref{sec:conditional}.

%%%%%%%%%%%%%%%%%%%%%%%%%%%%%%%%%%%%%%%%
\paragraph{Page Numbering.}

When only a part of the document is compiled,
the appropriate numbering of pages
(as well as other status parameters)
is determined from the |.aux| files.
The latter contain information from previous passes.
However this information needs to propagate through
all intermediate child documents.
Therefore the page numbering in child documents may well
be inconsistent until the complete document is compiled at least once.

A useful (if unconventional) way to always ensure a consistent
page numbering is to restart the numbering in each child document
and denote the pages by `\textit{child}|.|\textit{page}'
where \textit{child} represents the chapter/section number of the child file.
This can be achieved by the command
|\numberwithin{page}{|\textit{child}|}|
of the \textsf{amsmath} package
where \textit{child} can be |chapter| or |section|
depending on the chosen structuring.
Alternatively, one can modify the macro |\thepage| appropriately
and reset the counter |page| at the start of each child file.

%%%%%%%%%%%%%%%%%%%%%%%%%%%%%%%%%%%%%%%%%%%%%%%%%%%%%%%%%%%%%%%%%%%%%%%%%%%%%%%%
\subsection{Conditional Processing}
\label{sec:conditional}

The package provides a mechanism to compile different versions
of a document. To customise the versions further some conditional processing
can come in handy to distinguish which version is being compiled.
The package provides two macros to describe the compilation context:

%%%%%%%%%%%%%%%%%%%%%%%%%%%%%%%%%%%%%%%%
\DescribeMacro{\ifchilddoc}
The conditional |\ifchilddoc| distinguishes between the compilation of
child documents and the main document:
%
\begin{center}
|\ifchilddoc |\textit{child-code}| |[|\||else |\textit{main-code}]| \||fi|
\end{center}

%%%%%%%%%%%%%%%%%%%%%%%%%%%%%%%%%%%%%%%%
\DescribeMacro{\childdocname}
\DescribeMacro{\childdocjob}
The macro |\childdocname| contains the filename (without extension)
of the main or child file being processed.
Note that |\childdocjob| will always contain the name of the main file.

%%%%%%%%%%%%%%%%%%%%%%%%%%%%%%%%%%%%%%%%
\paragraph{Title Page.}

Conditional processing can be used to include a title or banner page
in the main document when proper precautions are taken.
Importantly, the code in the main file should ensure that the page counter
(as well as other status parameters which are stored in the |.aux| files)
takes the same value after the conditional processing.
Otherwise the page numbers may take divergent values
depending on which part is compiled.

For example, a title page could be declared by:
%
\begin{center}
\begin{tabular}{l}
|\ifchilddoc\||else|\\
|\addtocounter{page}{-1}|\\
\textit{code for title page}\\
|\newpage|\\
|\||fi|
\end{tabular}
\end{center}
%
A banner page for the child documents can be generated by:
%
\begin{center}
\begin{tabular}{l}
|\ifchilddoc|\\
|\addtocounter{page}{-1}|\\
\textit{code for banner page}\\
|\newpage|\\
|\||fi|
\end{tabular}
\end{center}
%
Here one could write a message such as:
\begin{center}
|This is the part \childdocname{} of \childdocjob{}.|
\end{center}

%%%%%%%%%%%%%%%%%%%%%%%%%%%%%%%%%%%%%%%%%%%%%%%%%%%%%%%%%%%%%%%%%%%%%%%%%%%%%%%%
\subsection{Flags}
\label{sec:flags}

The package makes it easy to generate different versions
of the main or child documents.
To this end compilation flags can be defined
and assigned different default values.
They will be particularly useful in conjunction
with the forwarding mechanism described in \secref{sec:forward}.

For example, it may be useful to have a flag |\version|
which can be set to |draft| or |final|.
The document source will contain some conditional code
depending on the value of |\version|.
Suppose further, the flag should default to |final| for the main file
and to |draft| for child files
which is a natural assignment for editing the document.
This is achieved by placing the following code
in the preamble of the main document
(below the |\childdocmain| directive):
%
\begin{center}
\begin{tabular}{l}
|\ifchilddoc|\\
|\providecommand{\version}{draft}|\\
|\||else|\\
|\providecommand{\version}{final}|\\
|\||fi|
\end{tabular}
\end{center}
%
The definition by |\providecommand| makes sure
that previous definitions are not overwritten.
Further statements |\providecommand{\version}{...}|
can thus be added before the above code to override it.

For the main file, one might add a line
(between |\childdocmain| and the above block)
%
\begin{center}
|%\ifchilddoc\||else\providecommand{\version}{draft}\||fi|
\end{center}
%
which can be uncommented to produce a draft version.
Likewise one can add a line to the very top of a child file
(above the |\childdocof{|\textit{main}|}| directive)
%
\begin{center}
|%\providecommand{\version}{final}|
\end{center}
%
which can be uncommented to produce the final version of this child document.

%%%%%%%%%%%%%%%%%%%%%%%%%%%%%%%%%%%%%%%%%%%%%%%%%%%%%%%%%%%%%%%%%%%%%%%%%%%%%%%%
\subsection{Forwarding}
\label{sec:forward}

Different versions of the main or child documents
using compilation flags as described in \secref{sec:flags}
can be (permanently) stored in different files
for convenient compilation, viewing and distribution.
To this end, the package defines a command
to pass on compilation to a different file:

%%%%%%%%%%%%%%%%%%%%%%%%%%%%%%%%%%%%%%%%
\DescribeMacro{\childdocforward}
The command |\childdocforward| redirects processing to
another source file:
%
\begin{center}
\begin{tabular}{l}
|\input{childdoc.def}|\\
|\childdocforward[|\textit{main}|]{|\textit{dest}|}|\\
\end{tabular}
\end{center}
%
The argument \textit{dest} is the destination file
(without extension).
It should be the main file or one of the child files.
Note that further \textsf{childdoc} directives
such as |\childdocof| and |\childdocforward|
in the indicated file will be processed in this form.
The optional argument \textit{main}
passes on directly to the main file \textit{main}
while pretending to compile the child \textit{dest}.
This form behaves as if \textit{dest}
issues |\childdocof{|\textit{main}|}| right away,
and no further \textsf{childdoc} directives will be processed.

%%%%%%%%%%%%%%%%%%%%%%%%%%%%%%%%%%%%%%%%
\DescribeMacro{\...prefix}
In the alternative form |\childdocforwardprefix|,
%
\begin{center}
\begin{tabular}{l}
|\input{childdoc.def}|\\
|\childdocforwardprefix[|\textit{main}|]{|\textit{prefix}|}{|\textit{dest}|}|
\end{tabular}
\end{center}
%
the destination file is determined by a pattern
depending on the current file:
To make this work, the current file must be called
`{\textit{prefix}\hspace{0.2em}\textit{suffix}}'
with \textit{prefix} matching precisely the argument.
Processing is then passed on to the file
`{\textit{dest}\hspace{0.2em}\textit{suffix}}'.
Surely, the same effect is achieved by
directly specifying the
argument `{\textit{dest}\hspace{0.2em}\textit{suffix}}'
in the first form.
However, that requires to set up a different file
for each child. With the alternative form of the command
all these files can have exactly the same content
which simplifies setting them up and maintaining them.

For example, the following file |draft.tex|
with a compilation flag |\version| as described in \secref{sec:flags}
compiles the main document as a draft:
%
\begin{center}
\begin{tabular}{l}
|\def\version{draft}|\\
|\input{childdoc.def}|\\
|\childdocforward{|\textit{main}|}|
\end{tabular}
\end{center}
%
Likewise, the following files |final|\textit{nn}|.tex|
compile the final version of the child document
|child|\textit{nn}|.tex|:
%
\begin{center}
\begin{tabular}{l}
|\def\version{final}|\\
|\input{childdoc.def}|\\
|\childdocforwardprefix{final}{child}|
\end{tabular}
\end{center}
%

Note that when several versions of a main file and/or of each child file
are to be generated, it may be convenient to set up a |Makefile| or
shell script to automatise the process.

%%%%%%%%%%%%%%%%%%%%%%%%%%%%%%%%%%%%%%%%%%%%%%%%%%%%%%%%%%%%%%%%%%%%%%%%%%%%%%%%
\subsection{Command Line Processing}
\label{sec:commandline}

The effect of redirection files can also be achieved by invoking
the \LaTeX{} compiler with a more elaborate command line.
Most conveniently this should be done as part
of a shell script or a |Makefile|.

When using \textsf{childdoc} in the main file, the following
command lines effectively perform a redirection
(note that depending on the shell being used,
backslashes may have to be doubled: `|\|' $\to$ `|\\|'):
%
\begin{center}
|... -jobname "|\textit{target}|" |\\|"|[\textit{flags}]%
|\input{childdoc.def}\childdocforward[|\textit{main}|]{|\textit{dest}|}"|
\end{center}
%
Here \textit{target} is the name of the output file,
\textit{main} is the name of the main file
and \textit{dest} is the name of the main or child file to be processed
(all filenames without extensions).
The optional argument \textit{main} can be omitted
if \textit{main} matches \textit{dest}.
Optionally, compilation \textit{flags} can be defined via |\def| commands.
This command line makes the \TeX{} engine believe
it is compiling the file \textit{target}
whose content is specified as the latter parameter.
The provided code then forwards the processing to
\textit{main} or \textit{dest} as described in \secref{sec:forward}.

%%%%%%%%%%%%%%%%%%%%%%%%%%%%%%%%%%%%%%%%%%%%%%%%%%%%%%%%%%%%%%%%%%%%%%%%%%%%%%%%
\subsection{Include by Input}
\label{sec:input}

Including child documents by |\include| has some restrictions by design.
Most notably, the content of a child document always occupies
its own set of pages; pages cannot be shared between child documents.
Usually, this behaviour makes perfect sense
because each child document contain an essential part of the document.
However, in some situations it may be desirable to compose
a document from a collection of parts
without having mandatory page breaks between then.
For this case, the package
provides a mechanism to include parts
by |\input| which can also be processed individually.
However, by construction this mechanism
requires manual handling of the content to be output.

%%%%%%%%%%%%%%%%%%%%%%%%%%%%%%%%%%%%%%%%
\DescribeMacro{\ifchilddocmanual}
The main file should be prepared as usual, see \secref{sec:include}.
However, the document body must make a distinction
between processing of an individual part and of the main document, e.g.:
%
\begin{center}
\begin{tabular}{l}
|\ifchilddocmanual|\\
|\input{\childdocname}|\\
|\||else|\\
\textit{document body with }|\input{|\textit{part}|}|\\
|\||fi|
\end{tabular}
\end{center}
%
The conditional |\ifchilddocmanual| is true whenever
a part to be included by |\input| is being compiled,
and the name of the part is stored in |\childdocname|.

%%%%%%%%%%%%%%%%%%%%%%%%%%%%%%%%%%%%%%%%
\DescribeMacro{\childdocby}
Each part to be included by |\input| should start with:
%
\begin{center}
\begin{tabular}{l}
|\input{childdoc.def}|\\
|\childdocby{|\textit{main}|}|\\
\end{tabular}
\end{center}
%
The directive |\childdocby| is similar to |\childdocof|
described in \secref{sec:include},
but the subsequent selection of content must be done manually.
To that end, both |\ifchilddoc| and |\ifchilddocmanual|
will be true upon processing of a part,
and the name of the part is stored in |\childdocname|.
Note that |\jobname| will be set to the filename of the current part
so that each part receives an individual |.aux| file
that does not interfere with the |.aux| file(s) of the main document.
This behaviour can be altered by the alternative form
|\childdocby[*]{|\textit{main}|}| (with a non-empty optional argument)
which uses the |.aux| file of the main document
by setting |\jobname| to \textit{main}.

%%%%%%%%%%%%%%%%%%%%%%%%%%%%%%%%%%%%%%%%%%%%%%%%%%%%%%%%%%%%%%%%%%%%%%%%%%%%%%%%
\subsection{Driver Development}
\label{sec:driver}

The \textsf{childdoc} mechanism can also be use for the development
of definition files such as \LaTeX{} styles or classes.
This case differs from the above setup with multiple parts
included by |\include| in that no |\includeonly| should be invoked.
This can be achieved by starting the include file
(before |\ProvidesPackage|) with:
%
\begin{center}
\begin{tabular}{l}
|\input{childdoc.def}|\\
|\childdocforward{|\textit{main}|}|\\
\end{tabular}
\end{center}
%
or alternatively with:
%
\begin{center}
\begin{tabular}{l}
|\input{childdoc.def}|\\
|\childdocby{|\textit{main}|}|\\
\end{tabular}
\end{center}
%
Both forms have slightly different effects as described above.
The main file is prepared as usual, see \secref{sec:include}.

%%%%%%%%%%%%%%%%%%%%%%%%%%%%%%%%%%%%%%%%%%%%%%%%%%%%%%%%%%%%%%%%%%%%%%%%%%%%%%%%
\subsection{Legacy Detection}
\label{sec:detection}

The directive |\childdocmain| in the main file can detect
whether the complete document or merely a child is to be compiled
even without using the directive |\childdocof|.
This method is deprecated because it is less robust
and there is no compelling reason to use it;
it is merely provided for backward compatibility
and it may be removed in future versions.

If the detection mechanism is to be used,
it is mandatory to correctly specify
the filename of the main file as the argument of |\childdocmain|:
%
\begin{center}
\begin{tabular}{l}
|\input{childdoc.def}|\\
|\childdocmain{|\textit{main}|}|\\
\end{tabular}
\end{center}
%
If |\jobname| does not match the argument \textit{main} of |\childdocmain|,
it is assumed that |\jobname| points to the child file to be compiled.
When using |\childdocmain| with the main file specified as argument,
it suffices to start a child file
with just |\input{|\textit{main}|}|
without loading of the package and using |\childdocof|.
If instead all processing is done
with the appropriate \textsf{childdoc} directives,
the argument of \textit{main} of |\childdocmain| can be empty.

An alternative version of the command line processing described
in \secref{sec:commandline} using the detection mechanism reads:
%
\begin{center}
|... -jobname "|\textit{target}|" "|[\textit{flags}]%
[|\def\jobname{|\textit{dest}|}|]|\input{|\textit{main}|}"|
\end{center}

%%%%%%%%%%%%%%%%%%%%%%%%%%%%%%%%%%%%%%%%%%%%%%%%%%%%%%%%%%%%%%%%%%%%%%%%%%%%%%%%
\subsection{Manual Code}
\label{sec:manual}

In case one cannot be certain whether the definitions file |childdoc.def|
is installed on the target \TeX{} distribution
and one prefers not to ship it,
it is conceivable to paste a few relevant commands into the sources.

To that end, drop all statements |\input{childdoc.def}|
and perform the replacements as outlined below.
Instead of |\childdocmain{|\textit{main}|}| add the following code
to the top of the main file:
%
\begin{center}
\begin{tabular}{l}
|\||ifdefined\childdocname\endinput\||fi\newif\ifchilddoc|\\
|\edef\childdocname{\scantokens\expandafter{\jobname\noexpand}}|\\
|\def\childdocmain{|\textit{main}|}\||ifx\childdocmain\childdocname\||else|\\
|\childdoctrue\includeonly{\childdocname}\let\jobname\childdocmain\||fi|\\
\end{tabular}
\end{center}
%
Instead of |\childdocof{|\textit{main}|}| just include the main file
at the top of each child file:
%
\begin{center}
|\input{|\textit{main}|}|
\end{center}
%
A simple redirection |\childdocforward{|\textit{dest}|}| is achieved by:
%
\begin{center}
|\def\jobname{|\textit{dest}|}\input{\jobname}|
\end{center}
%
The redirection with prefix
|\childdocforwardprefix[|\textit{prefix}|]{|\textit{dest}|}|
is accomplished by:
%
\begin{center}
\begin{tabular}{l}
|{\edef\jobname{\scantokens\expandafter{\jobname\noexpand}}|\\
|\def\redirectjob |\textit{prefix}|#1~~~{\gdef\jobname{|\textit{dest}|#1}}|\\
|\expandafter\redirectjob\jobname~~~}\input{\jobname}|
\end{tabular}
\end{center}

In an alternative approach,
child documents can be compiled by a specific command line
without additional code or specific definitions:
%
\begin{center}
|... -jobname "|\textit{target}|" "|[\textit{flags}]%
|\includeonly{|\textit{dest}|}\input{|\textit{main}|}"|
\end{center}
%

%%%%%%%%%%%%%%%%%%%%%%%%%%%%%%%%%%%%%%%%%%%%%%%%%%%%%%%%%%%%%%%%%%%%%%%%%%%%%%%%
%%%%%%%%%%%%%%%%%%%%%%%%%%%%%%%%%%%%%%%%%%%%%%%%%%%%%%%%%%%%%%%%%%%%%%%%%%%%%%%%
\section{Information}

%%%%%%%%%%%%%%%%%%%%%%%%%%%%%%%%%%%%%%%%%%%%%%%%%%%%%%%%%%%%%%%%%%%%%%%%%%%%%%%%
\subsection{Copyright}

Copyright \copyright{} 2017--2018 Niklas Beisert

This work may be distributed and/or modified under the
conditions of the \LaTeX{} Project Public License, either version 1.3
of this license or (at your option) any later version.
The latest version of this license is in
  \url{http://www.latex-project.org/lppl.txt}
and version 1.3 or later is part of all distributions of \LaTeX{}
version 2005/12/01 or later.

This work has the LPPL maintenance status `maintained'.

The Current Maintainer of this work is Niklas Beisert.

This work consists of the files |README.txt|, |childdoc.ins| and |childdoc.dtx|
as well as the derived files |childdoc.def|, |cdocsamp.tex|
with |cdocsch1.tex|, |cdocsch2.tex|, |cdocspt3.tex|, |cdocspt4.tex|,
|cdocsdrf.tex|, |cdocsfn1.tex|, |cdocsfn2.tex|
as well as |childdoc.pdf|.

%%%%%%%%%%%%%%%%%%%%%%%%%%%%%%%%%%%%%%%%%%%%%%%%%%%%%%%%%%%%%%%%%%%%%%%%%%%%%%%%
\subsection{Files and Installation}

The package consists of the files:
%
\begin{center}
\begin{tabular}{ll}
    |README.txt|   & readme file \\
    |childdoc.ins| & installation file \\
    |childdoc.dtx| & source file \\
    |childdoc.def| & definition file \\
    |cdocsamp.tex| & sample main file \\
    |cdocsch1.tex| & sample include file \\
    |cdocsch2.tex| & sample include file \\
    |cdocspt3.tex| & sample part file \\
    |cdocspt4.tex| & sample part file \\
    |cdocsdrf.tex| & sample redirection file \\
    |cdocsfn1.tex| & sample redirection file \\
    |cdocsfn2.tex| & sample redirection file \\
    |childdoc.pdf| & manual
\end{tabular}
\end{center}
%
The distribution consists of the files
|README.txt|, |childdoc.ins| and |childdoc.dtx|.
%
\begin{itemize}
\item
Run (pdf)\LaTeX{} on |childdoc.dtx|
to compile the manual |childdoc.pdf| (this file).
\item
Run \LaTeX{} on |childdoc.ins| to create the definitions file |childdoc.def|
and the sample |cdocsamp.tex| with include files
|cdocsch1.tex|, |cdocsch2.tex|, |cdocspt3.tex|, |cdocspt4.tex|,
|cdocsdrf.tex|, |cdocsfn1.tex|, |cdocsfn2.tex|.
Then copy the file |childdoc.def| to an appropriate directory of your \LaTeX{}
distribution, e.g.\ \textit{texmf-root}|/tex/latex/childdoc|.
\end{itemize}

%%%%%%%%%%%%%%%%%%%%%%%%%%%%%%%%%%%%%%%%%%%%%%%%%%%%%%%%%%%%%%%%%%%%%%%%%%%%%%%%
\subsection{Related CTAN Packages}

There are several other packages which offer a similar functionality:
%
\begin{itemize}
\item
The packages
\href{http://ctan.org/pkg/docmute}{\textsf{docmute}},
\href{http://ctan.org/pkg/includex}{\textsf{includex}} and
\href{http://ctan.org/pkg/standalone}{\textsf{standalone}}
provide commands to include only the document body of
a child file thus allowing both files to be compiled individually.
\item
The packages \href{http://ctan.org/pkg/subdocs}{\textsf{subdocs}}
and \href{http://ctan.org/pkg/subfiles}{\textsf{subfiles}}
provide structures in which the main and child documents can be
encapsulated and allowing them to be compiled individually.
The inclusion mechanism is different from the conventional |\include|.
\item
The package \href{http://ctan.org/pkg/combine}{\textsf{combine}}
is an elaborate solution to combine several documents into one.
\end{itemize}
%
See also the CTAN topic \href{http://ctan.org/topic/subdocs}{\textsf{subdocs}}
for further related packages.
The present package differs from the above solutions in that
a document structure constructed with the conventional |\include| mechanism
just needs two extra commands at the top of every file
such that all constituent files can be compiled individually.

%%%%%%%%%%%%%%%%%%%%%%%%%%%%%%%%%%%%%%%%%%%%%%%%%%%%%%%%%%%%%%%%%%%%%%%%%%%%%%%%
%\subsection{Feature Suggestions}
%
%The following is a list of features which may be useful for future
%versions of this package:
%%
%\begin{itemize}
%\item
%\ldots
%\end{itemize}

%%%%%%%%%%%%%%%%%%%%%%%%%%%%%%%%%%%%%%%%%%%%%%%%%%%%%%%%%%%%%%%%%%%%%%%%%%%%%%%%
\subsection{Revision History}

%%%%%%%%%%%%%%%%%%%%%%%%%%%%%%%%%%%%%%%%
\paragraph{v2.0:} 2018/12/30

\begin{itemize}
\item
immediate forward processing
\item
added |\childdocby| mechanism
\item
manual restructured
\end{itemize}

%%%%%%%%%%%%%%%%%%%%%%%%%%%%%%%%%%%%%%%%
\paragraph{v1.6:} 2018/01/17

\begin{itemize}
\item
application for development of include files
\item
corrections to manual
\end{itemize}

%%%%%%%%%%%%%%%%%%%%%%%%%%%%%%%%%%%%%%%%
\paragraph{v1.5:} 2017/05/21

\begin{itemize}
\item
more complete structuring introduced
\item
|\childdocof| introduced
\item
|\childdoc| renamed to |\childdocmain|
\item
|\childredirect| renamed to |\childdocforward| and |\childdocforwardprefix|
and functionality expanded
\end{itemize}

%%%%%%%%%%%%%%%%%%%%%%%%%%%%%%%%%%%%%%%%
\paragraph{v1.0:} 2017/04/27

\begin{itemize}
\item
manual and install package
\item
first version published on CTAN
\end{itemize}

%%%%%%%%%%%%%%%%%%%%%%%%%%%%%%%%%%%%%%%%
\paragraph{v0.6:} 2017/04/26

\begin{itemize}
\item
redirection mechanism added
\end{itemize}

%%%%%%%%%%%%%%%%%%%%%%%%%%%%%%%%%%%%%%%%
\paragraph{v0.5:} 2017/04/26

\begin{itemize}
\item
functionality in definition file
\end{itemize}


%%%%%%%%%%%%%%%%%%%%%%%%%%%%%%%%%%%%%%%%%%%%%%%%%%%%%%%%%%%%%%%%%%%%%%%%%%%%%%%%
%%%%%%%%%%%%%%%%%%%%%%%%%%%%%%%%%%%%%%%%%%%%%%%%%%%%%%%%%%%%%%%%%%%%%%%%%%%%%%%%
%%%%%%%%%%%%%%%%%%%%%%%%%%%%%%%%%%%%%%%%%%%%%%%%%%%%%%%%%%%%%%%%%%%%%%%%%%%%%%%%
\appendix

\settowidth\MacroIndent{\rmfamily\scriptsize 000\ }

 \DocInput{childdoc.dtx}

\end{document}
%</driver>
% \fi
%
% %%%%%%%%%%%%%%%%%%%%%%%%%%%%%%%%%%%%%%%%%%%%%%%%%%%%%%%%%%%%%%%%%%%%%%%%%%%%%%
% %%%%%%%%%%%%%%%%%%%%%%%%%%%%%%%%%%%%%%%%%%%%%%%%%%%%%%%%%%%%%%%%%%%%%%%%%%%%%%
% \section{Sample}
%\iffalse
%<*samplemain>
%\fi
%
% The following presents a sample document
% with two chapters, two parts, a title page,
% a compile flag as well as three forwarding files to set the flag.
% It consists of eight |.tex| files:
% \begin{center}
% \begin{tabular}{ll}
% |cdocsamp.tex|&main file\\
% |cdocsch1.tex|&include file for chapter 1\\
% |cdocsch2.tex|&include file for chapter 2\\
% |cdocspt3.tex|&include file for part 3\\
% |cdocspt4.tex|&include file for part 4\\
% |cdocsdrf.tex|&forwarding file for main file in draft mode\\
% |cdocsfi1.tex|&forwarding file for final version of chapter 1\\
% |cdocsfi2.tex|&forwarding file for final version of chapter 2\\
% \end{tabular}
% \end{center}
% Each of the eight files can be compiled directly by the \LaTeX{} compiler.
%
% %%%%%%%%%%%%%%%%%%%%%%%%%%%%%%%%%%%%%%
% \paragraph{Main File.}
%
% The main file is called |cdocsamp.tex|.
%
% Load the \textsf{childdoc} definitions and
% declare the filename for the main document:
%    \begin{macrocode}
\input{childdoc.def}
\childdocmain{}
%    \end{macrocode}

% Optional override for |\version| flag:
%    \begin{macrocode}
%%\ifchilddoc\else\providecommand{\version}{draft}\fi
%    \end{macrocode}

% Define the default values for the |\version| flag
% (|final| for the main file and |draft| for childs):
%    \begin{macrocode}
\ifchilddoc
\providecommand{\version}{draft}
\else
\providecommand{\version}{final}
\fi
%    \end{macrocode}

% Load the standard document class:
%    \begin{macrocode}
\documentclass[12pt]{article}
%    \end{macrocode}

% Start the document body:
%    \begin{macrocode}
\begin{document}
%    \end{macrocode}

% Declare a title page.
% Print title, part of document being processed and version flag:
%    \begin{macrocode}
\addtocounter{page}{-1}
\begin{center}
{\LARGE\bfseries{}childdoc example\par}
\vspace{1cm}
\ifchilddoc
\ifchilddocmanual part\else chapter\fi:
`\childdocname' of `\childdocjob'\par
\else
main document: `\childdocjob'\par
\fi
version: \version\par
\end{center}
\newpage
%    \end{macrocode}

% Manually include selected file,
% otherwise process as usual:
%    \begin{macrocode}
\ifchilddocmanual
\section*{part `\childdocname'}
\input{\childdocname}
\else
%    \end{macrocode}

% Include the two chapters:
%    \begin{macrocode}
\include{cdocsch1}
\include{cdocsch2}
%    \end{macrocode}

% Include the two parts unless only chapters should be displayed:
%    \begin{macrocode}
\ifchilddoc\else
\section{part three}
\input{cdocspt3}
\section{part four}
\input{cdocspt4}
\fi
%    \end{macrocode}

% Process as usual until here:
%    \begin{macrocode}
\fi
%    \end{macrocode}

% End of document body:
%    \begin{macrocode}
\end{document}
%    \end{macrocode}
%\iffalse
%</samplemain>
%\fi
%
% %%%%%%%%%%%%%%%%%%%%%%%%%%%%%%%%%%%%%%
% \paragraph{Chapter Include Files.}
%
% The include files are called |cdocsch1.tex| and |cdocsch2.tex|.
%
%\iffalse
%<*samplechap1|samplechap2>
%\fi

% Optional override for |\version| flag:
%    \begin{macrocode}
%%\providecommand{\version}{final}
%    \end{macrocode}

% Include the main document:
%    \begin{macrocode}
\input{childdoc.def}
\childdocof{cdocsamp}
%    \end{macrocode}

%\iffalse
%</samplechap1|samplechap2>
%\fi
%
%\iffalse
%<*samplechap1>
%\fi
% Some text for chapter 1:
%    \begin{macrocode}
\section{one}
some text in chapter one
%    \end{macrocode}

%\iffalse
%</samplechap1>
%\fi
% Some text for chapter 2:
%\iffalse
%<*samplechap2>
%\fi
%    \begin{macrocode}
\section{two}
more text in chapter two
%    \end{macrocode}

%\iffalse
%</samplechap2>
%\fi
%
% %%%%%%%%%%%%%%%%%%%%%%%%%%%%%%%%%%%%%%
% \paragraph{Part Include Files.}
%
% The include files are called |cdocspt3.tex| and |cdocspt4.tex|.
%
%\iffalse
%<*samplepart3|samplepart4>
%\fi

% Optional override for |\version| flag:
%    \begin{macrocode}
%%\providecommand{\version}{final}
%    \end{macrocode}

% Include the main document:
%    \begin{macrocode}
\input{childdoc.def}
\childdocby{cdocsamp}
%    \end{macrocode}

%\iffalse
%</samplepart3|samplepart4>
%\fi
%
%\iffalse
%<*samplepart3>
%\fi
% Some text for part 3:
%    \begin{macrocode}
some text in part three
%    \end{macrocode}

%\iffalse
%</samplepart3>
%\fi
% Some text for part 4:
%\iffalse
%<*samplepart4>
%\fi
%    \begin{macrocode}
more text in part four
%    \end{macrocode}

%\iffalse
%</samplepart4>
%\fi
%
% %%%%%%%%%%%%%%%%%%%%%%%%%%%%%%%%%%%%%%
% \paragraph{Forwarding for a Complete Draft.}
%
% The following forwarding file |cdocsdrf.tex|
% compiles the main document in draft mode:
%\iffalse
%<*sampledraft>
%\fi
%    \begin{macrocode}
\def\version{draft}
\input{childdoc.def}
\childdocforward{cdocsamp}
%    \end{macrocode}

%\iffalse
%</sampledraft>
%\fi
%
% %%%%%%%%%%%%%%%%%%%%%%%%%%%%%%%%%%%%%%
% \paragraph{Forwarding for Final Version of the Chapters.}
%
% The following forwarding files |cdocsfn1.tex| and |cdocsfn2.tex|
% (with identical content)
% compile the final versions of the child documents
% |cdocsch1.tex| and |cdocsch2.tex|, respectively:
%\iffalse
%<*samplefinal>
%\fi
%    \begin{macrocode}
\def\version{final}
\input{childdoc.def}
\childdocforwardprefix[cdocsamp]{cdocsfn}{cdocsch}
%    \end{macrocode}

%\iffalse
%</samplefinal>
%\fi
%
% %%%%%%%%%%%%%%%%%%%%%%%%%%%%%%%%%%%%%%
% \paragraph{Command Line Processing.}
%
% The following three command lines generate the output files
% |cdocscld|, |cdocscl1| and |cdocscl2|
% which should be identical to
% |cdocsdrf|, |cdocsch1| and |cdocsfn2|, respectively:
% \begin{center}
% \begin{tabular}{l}
% |latex -jobname cdocscld \|\\
% |  "\def\version{draft}\input{childdoc.def}\childdocforward{cdocsamp}"|\\
% |latex -jobname cdocscl1 \|\\
% |  "\input{childdoc.def}\childdocforward[cdocsamp]{cdocsch1}"|\\
% |latex -jobname cdocscl2 \|\\
% |  "\def\version{final}\input{childdoc.def}\childdocforward{cdocsch2}"|
% \end{tabular}
% \end{center}
% Note that the trailing backslash on each first line
% merely continues the input to the second line
% (for convenient cut ant paste).
% Furthermore, the command |latex| can be replaced by any
% of its alternative versions such as |pdflatex|.
%
% %%%%%%%%%%%%%%%%%%%%%%%%%%%%%%%%%%%%%%%%%%%%%%%%%%%%%%%%%%%%%%%%%%%%%%%%%%%%%%
% %%%%%%%%%%%%%%%%%%%%%%%%%%%%%%%%%%%%%%%%%%%%%%%%%%%%%%%%%%%%%%%%%%%%%%%%%%%%%%
% \section{Implementation}
%\iffalse
%<*package>
%\fi
%
% This section describes the definitions file |childdoc.def|.

% The definitions cannot be loaded using |\usepackage| or |\RequirePackage|
% which has a mechanism to prevent loading a style file more than once.
% When loading the definitions by means of |\input|
% multiple instances have to be prevented manually:
%\iffalse
%This code needs to be before the `\ProvidesFile' directive
%which is defined at the beginning of this file.
%Therefore it is also placed there and commented out here.
%</package>
%<*discard>
%\fi
%    \begin{macrocode}
\ifdefined\childdocmain\endinput\fi
%    \end{macrocode}
%\iffalse
%</discard>
%<*package>
%\fi
%
% \macro{\ifchilddoc}
% \macro{\ifchilddocmanual}
% The conditional |\ifchilddoc| tells whether a
% child (true) or main (false) document is being compiled.
% The conditional |\ifchilddocmanual| tells whether
% the |\includeonly| mechanism is used (false) or
% the selection of child files must be performed manually (true).
% The definitions initialise to false:
%    \begin{macrocode}
\newif\ifchilddoc
\newif\ifchilddocmanual
%    \end{macrocode}

% \macro{\childdocname}
% \macro{\childdocjob}
% The macro |\childdocname| stores the name of the main document
% to be compiled. The macro |\childdocjob| stores the name of
% the document on which the \LaTeX{} compiler was originally invoked.
% The content of |\jobname| cannot be compared
% to filenames specified in the source due to different catcodes.
% The following code rescans |\jobname|, stores the result
% in |\childdocname| and saves a copy in |\childdocjob|:
%    \begin{macrocode}
\edef\childdocname{\scantokens\expandafter{\jobname\noexpand}}
\let\childdocjob\childdocname
%    \end{macrocode}

% \macro{\childdocdisable}
% The macro |\childdocdisable| prevents the main file
% from being processed more than once.
% At this stage, the main document command |\childdocmain|
% is assumed to be called once again where it should do nothing.
% Any subsequent call to it should prevent
% a secondary processing of the main document
% It overwrites the forwarding commands
% |\childdocof| and |\childdocforward|
% with empty macros to prevent further inclusions of the main document:
%    \begin{macrocode}
\newcommand{\childdocdisable}
{
  \renewcommand{\childdocmain}[1]{\renewcommand{\childdocmain}[1]{\endinput}}
  \renewcommand{\childdocof}[1]{}
  \renewcommand{\childdocby}[2][]{}
  \renewcommand{\childdocforward}[2][]{}
  \renewcommand{\childdocdisable}{}
}
%    \end{macrocode}

% \macro{\childdocmain}
% The macro |\childdocmain| is to be called at the top of the main file
% with nothing or the main filename (without extension) as argument.
% First, it breaks loops.
% If the argument is not empty and does not match |\childdocname|
% (which is set by the first inclusion of |childdoc.def|),
% |\ifchilddoc| is set to true, |\includeonly| is applied to the child file
% and |\jobname| is set to the main file
% (for proper handling of |.aux| files):
%    \begin{macrocode}
\newcommand{\childdocmain}[1]
{
  \childdocdisable\childdocmain{}
  \if?#1?\else
    \begingroup
      \def\childdoctmp{#1}
      \ifx\childdoctmp\childdocname
        \def\childdoctmp{}
      \else
        \def\childdoctmp
        {
          \childdoctrue
          \includeonly{\childdocname}
          \def\childdocjob{#1}
          \def\jobname{#1}
        }
      \fi
      \expandafter
    \endgroup
    \childdoctmp
  \fi
}
%    \end{macrocode}

% \macro{\childdocof}
% The command |\childdocof| redirects
% compilation to the main file |#1|.
%    \begin{macrocode}
\newcommand{\childdocof}[1]
{
  \childdocdisable
  \childdoctrue
  \includeonly{\childdocname}
  \def\jobname{#1}
  \def\childdocjob{#1}
  \input{#1}
}
%    \end{macrocode}

% \macro{\childdocby}
% The command |\childdocby| ....
%    \begin{macrocode}
\newcommand{\childdocby}[2][]
{
  \childdocdisable
  \childdoctrue
  \childdocmanualtrue
  \if?#1?\else
    \def\jobname{#2}
  \fi
  \def\childdocjob{#2}
  \input{#2}
  \endinput
}
%    \end{macrocode}

% \macro{\childdocforward}
% The command |\childdocforward| redirects
% compilation to the main file or
% (if the optional argument is given) a child file.
% Parameters are set as if the main file
% or a child file starting with |\childdocof| was compiled.
% Then compilation is handed over to the main file:
%    \begin{macrocode}
\newcommand{\childdocforward}[2][]
{
  \begingroup
    \if?#1?
      \def\childdoctmp
      {
        \def\childdocname{#2}
        \def\childdocjob{#2}
        \def\jobname{#2}
        \input{#2}
        \endinput
      }
    \else
      \def\childdoctmp
      {
        \childdocdisable
        \def\childdocname{#2}
        \childdoctrue
        \includeonly{#2}
        \def\childdocjob{#1}
        \def\jobname{#1}
        \input{#1}
        \endinput
      }
    \fi
    \expandafter
  \endgroup
  \childdoctmp
}
%    \end{macrocode}

% \macro{\childdocforwardprefix}
% The command |\childdocforwardprefix| redirects
% compilation to the main or a child file by means of a pattern.
% The prefix |#1| in the current filename is replaced by |#2|
% and the suffix of the current filename is kept
% (it is assumed that the filename does not contain the substring `|~~~|'
% which is used as a delimiter).
% Compilation is handed over to the new file by |\childdocforward|:
%    \begin{macrocode}
\newcommand{\childdocforwardprefix}[3][]
{
  \begingroup
    \def\childdocextract #2##1~~~{\def\childdoctmp{\childdocforward[#1]{#3##1}}}
    \expandafter\childdocextract\childdocname~~~
    \expandafter
  \endgroup
  \childdoctmp
}
%    \end{macrocode}

% \macro{\childdoc}
% The deprecated macro |\childdoc| is a legacy version of |\childdocmain|:
%    \begin{macrocode}
\newcommand{\childdoc}{\childdocmain}
%    \end{macrocode}

% \macro{\childdocredirect}
% The deprecated macro |\childdocredirect| is a legacy version
% of |\childdocforward| and |\childdocforwardprefix|:
%    \begin{macrocode}
\newcommand{\childdocredirect}[2][]
{
  \begingroup
    \if?#1?
      \def\childdoctmp{\childdocforward{#2}}
    \else
      \def\childdoctmp{\childdocforwardprefix{#1}{#2}}
    \fi
    \expandafter
  \endgroup
  \childdoctmp
}
%    \end{macrocode}

%\iffalse
%</package>
%\fi
%
\endinput
|
and perform the replacements as outlined below.
Instead of |\childdocmain{|\textit{main}|}| add the following code
to the top of the main file:
%
\begin{center}
\begin{tabular}{l}
|\||ifdefined\childdocname\endinput\||fi\newif\ifchilddoc|\\
|\edef\childdocname{\scantokens\expandafter{\jobname\noexpand}}|\\
|\def\childdocmain{|\textit{main}|}\||ifx\childdocmain\childdocname\||else|\\
|\childdoctrue\includeonly{\childdocname}\let\jobname\childdocmain\||fi|\\
\end{tabular}
\end{center}
%
Instead of |\childdocof{|\textit{main}|}| just include the main file
at the top of each child file:
%
\begin{center}
|\input{|\textit{main}|}|
\end{center}
%
A simple redirection |\childdocforward{|\textit{dest}|}| is achieved by:
%
\begin{center}
|\def\jobname{|\textit{dest}|}\input{\jobname}|
\end{center}
%
The redirection with prefix
|\childdocforwardprefix[|\textit{prefix}|]{|\textit{dest}|}|
is accomplished by:
%
\begin{center}
\begin{tabular}{l}
|{\edef\jobname{\scantokens\expandafter{\jobname\noexpand}}|\\
|\def\redirectjob |\textit{prefix}|#1~~~{\gdef\jobname{|\textit{dest}|#1}}|\\
|\expandafter\redirectjob\jobname~~~}\input{\jobname}|
\end{tabular}
\end{center}

In an alternative approach,
child documents can be compiled by a specific command line
without additional code or specific definitions:
%
\begin{center}
|... -jobname "|\textit{target}|" "|[\textit{flags}]%
|\includeonly{|\textit{dest}|}\input{|\textit{main}|}"|
\end{center}
%

%%%%%%%%%%%%%%%%%%%%%%%%%%%%%%%%%%%%%%%%%%%%%%%%%%%%%%%%%%%%%%%%%%%%%%%%%%%%%%%%
%%%%%%%%%%%%%%%%%%%%%%%%%%%%%%%%%%%%%%%%%%%%%%%%%%%%%%%%%%%%%%%%%%%%%%%%%%%%%%%%
\section{Information}

%%%%%%%%%%%%%%%%%%%%%%%%%%%%%%%%%%%%%%%%%%%%%%%%%%%%%%%%%%%%%%%%%%%%%%%%%%%%%%%%
\subsection{Copyright}

Copyright \copyright{} 2017--2018 Niklas Beisert

This work may be distributed and/or modified under the
conditions of the \LaTeX{} Project Public License, either version 1.3
of this license or (at your option) any later version.
The latest version of this license is in
  \url{http://www.latex-project.org/lppl.txt}
and version 1.3 or later is part of all distributions of \LaTeX{}
version 2005/12/01 or later.

This work has the LPPL maintenance status `maintained'.

The Current Maintainer of this work is Niklas Beisert.

This work consists of the files |README.txt|, |childdoc.ins| and |childdoc.dtx|
as well as the derived files |childdoc.def|, |cdocsamp.tex|
with |cdocsch1.tex|, |cdocsch2.tex|, |cdocspt3.tex|, |cdocspt4.tex|,
|cdocsdrf.tex|, |cdocsfn1.tex|, |cdocsfn2.tex|
as well as |childdoc.pdf|.

%%%%%%%%%%%%%%%%%%%%%%%%%%%%%%%%%%%%%%%%%%%%%%%%%%%%%%%%%%%%%%%%%%%%%%%%%%%%%%%%
\subsection{Files and Installation}

The package consists of the files:
%
\begin{center}
\begin{tabular}{ll}
    |README.txt|   & readme file \\
    |childdoc.ins| & installation file \\
    |childdoc.dtx| & source file \\
    |childdoc.def| & definition file \\
    |cdocsamp.tex| & sample main file \\
    |cdocsch1.tex| & sample include file \\
    |cdocsch2.tex| & sample include file \\
    |cdocspt3.tex| & sample part file \\
    |cdocspt4.tex| & sample part file \\
    |cdocsdrf.tex| & sample redirection file \\
    |cdocsfn1.tex| & sample redirection file \\
    |cdocsfn2.tex| & sample redirection file \\
    |childdoc.pdf| & manual
\end{tabular}
\end{center}
%
The distribution consists of the files
|README.txt|, |childdoc.ins| and |childdoc.dtx|.
%
\begin{itemize}
\item
Run (pdf)\LaTeX{} on |childdoc.dtx|
to compile the manual |childdoc.pdf| (this file).
\item
Run \LaTeX{} on |childdoc.ins| to create the definitions file |childdoc.def|
and the sample |cdocsamp.tex| with include files
|cdocsch1.tex|, |cdocsch2.tex|, |cdocspt3.tex|, |cdocspt4.tex|,
|cdocsdrf.tex|, |cdocsfn1.tex|, |cdocsfn2.tex|.
Then copy the file |childdoc.def| to an appropriate directory of your \LaTeX{}
distribution, e.g.\ \textit{texmf-root}|/tex/latex/childdoc|.
\end{itemize}

%%%%%%%%%%%%%%%%%%%%%%%%%%%%%%%%%%%%%%%%%%%%%%%%%%%%%%%%%%%%%%%%%%%%%%%%%%%%%%%%
\subsection{Related CTAN Packages}

There are several other packages which offer a similar functionality:
%
\begin{itemize}
\item
The packages
\href{http://ctan.org/pkg/docmute}{\textsf{docmute}},
\href{http://ctan.org/pkg/includex}{\textsf{includex}} and
\href{http://ctan.org/pkg/standalone}{\textsf{standalone}}
provide commands to include only the document body of
a child file thus allowing both files to be compiled individually.
\item
The packages \href{http://ctan.org/pkg/subdocs}{\textsf{subdocs}}
and \href{http://ctan.org/pkg/subfiles}{\textsf{subfiles}}
provide structures in which the main and child documents can be
encapsulated and allowing them to be compiled individually.
The inclusion mechanism is different from the conventional |\include|.
\item
The package \href{http://ctan.org/pkg/combine}{\textsf{combine}}
is an elaborate solution to combine several documents into one.
\end{itemize}
%
See also the CTAN topic \href{http://ctan.org/topic/subdocs}{\textsf{subdocs}}
for further related packages.
The present package differs from the above solutions in that
a document structure constructed with the conventional |\include| mechanism
just needs two extra commands at the top of every file
such that all constituent files can be compiled individually.

%%%%%%%%%%%%%%%%%%%%%%%%%%%%%%%%%%%%%%%%%%%%%%%%%%%%%%%%%%%%%%%%%%%%%%%%%%%%%%%%
%\subsection{Feature Suggestions}
%
%The following is a list of features which may be useful for future
%versions of this package:
%%
%\begin{itemize}
%\item
%\ldots
%\end{itemize}

%%%%%%%%%%%%%%%%%%%%%%%%%%%%%%%%%%%%%%%%%%%%%%%%%%%%%%%%%%%%%%%%%%%%%%%%%%%%%%%%
\subsection{Revision History}

%%%%%%%%%%%%%%%%%%%%%%%%%%%%%%%%%%%%%%%%
\paragraph{v2.0:} 2018/12/30

\begin{itemize}
\item
immediate forward processing
\item
added |\childdocby| mechanism
\item
manual restructured
\end{itemize}

%%%%%%%%%%%%%%%%%%%%%%%%%%%%%%%%%%%%%%%%
\paragraph{v1.6:} 2018/01/17

\begin{itemize}
\item
application for development of include files
\item
corrections to manual
\end{itemize}

%%%%%%%%%%%%%%%%%%%%%%%%%%%%%%%%%%%%%%%%
\paragraph{v1.5:} 2017/05/21

\begin{itemize}
\item
more complete structuring introduced
\item
|\childdocof| introduced
\item
|\childdoc| renamed to |\childdocmain|
\item
|\childredirect| renamed to |\childdocforward| and |\childdocforwardprefix|
and functionality expanded
\end{itemize}

%%%%%%%%%%%%%%%%%%%%%%%%%%%%%%%%%%%%%%%%
\paragraph{v1.0:} 2017/04/27

\begin{itemize}
\item
manual and install package
\item
first version published on CTAN
\end{itemize}

%%%%%%%%%%%%%%%%%%%%%%%%%%%%%%%%%%%%%%%%
\paragraph{v0.6:} 2017/04/26

\begin{itemize}
\item
redirection mechanism added
\end{itemize}

%%%%%%%%%%%%%%%%%%%%%%%%%%%%%%%%%%%%%%%%
\paragraph{v0.5:} 2017/04/26

\begin{itemize}
\item
functionality in definition file
\end{itemize}


%%%%%%%%%%%%%%%%%%%%%%%%%%%%%%%%%%%%%%%%%%%%%%%%%%%%%%%%%%%%%%%%%%%%%%%%%%%%%%%%
%%%%%%%%%%%%%%%%%%%%%%%%%%%%%%%%%%%%%%%%%%%%%%%%%%%%%%%%%%%%%%%%%%%%%%%%%%%%%%%%
%%%%%%%%%%%%%%%%%%%%%%%%%%%%%%%%%%%%%%%%%%%%%%%%%%%%%%%%%%%%%%%%%%%%%%%%%%%%%%%%
\appendix

\settowidth\MacroIndent{\rmfamily\scriptsize 000\ }

 \DocInput{childdoc.dtx}

\end{document}
%</driver>
% \fi
%
% %%%%%%%%%%%%%%%%%%%%%%%%%%%%%%%%%%%%%%%%%%%%%%%%%%%%%%%%%%%%%%%%%%%%%%%%%%%%%%
% %%%%%%%%%%%%%%%%%%%%%%%%%%%%%%%%%%%%%%%%%%%%%%%%%%%%%%%%%%%%%%%%%%%%%%%%%%%%%%
% \section{Sample}
%\iffalse
%<*samplemain>
%\fi
%
% The following presents a sample document
% with two chapters, two parts, a title page,
% a compile flag as well as three forwarding files to set the flag.
% It consists of eight |.tex| files:
% \begin{center}
% \begin{tabular}{ll}
% |cdocsamp.tex|&main file\\
% |cdocsch1.tex|&include file for chapter 1\\
% |cdocsch2.tex|&include file for chapter 2\\
% |cdocspt3.tex|&include file for part 3\\
% |cdocspt4.tex|&include file for part 4\\
% |cdocsdrf.tex|&forwarding file for main file in draft mode\\
% |cdocsfi1.tex|&forwarding file for final version of chapter 1\\
% |cdocsfi2.tex|&forwarding file for final version of chapter 2\\
% \end{tabular}
% \end{center}
% Each of the eight files can be compiled directly by the \LaTeX{} compiler.
%
% %%%%%%%%%%%%%%%%%%%%%%%%%%%%%%%%%%%%%%
% \paragraph{Main File.}
%
% The main file is called |cdocsamp.tex|.
%
% Load the \textsf{childdoc} definitions and
% declare the filename for the main document:
%    \begin{macrocode}
% \iffalse
%
% childdoc.dtx Copyright (C) 2017-2018 Niklas Beisert
%
% This work may be distributed and/or modified under the
% conditions of the LaTeX Project Public License, either version 1.3
% of this license or (at your option) any later version.
% The latest version of this license is in
%   http://www.latex-project.org/lppl.txt
% and version 1.3 or later is part of all distributions of LaTeX
% version 2005/12/01 or later.
%
% This work has the LPPL maintenance status `maintained'.
%
% The Current Maintainer of this work is Niklas Beisert.
%
% This work consists of the files childdoc.dtx and childdoc.ins
% and the derived files childdoc.def and cdocsamp.tex with
% cdocsch1.tex, cdocsch2.tex, cdocsdrf.tex, cdocsfn1.tex, cdocsfn2.tex.
%
%<package>\ifdefined\childdocmain\endinput\fi
%<package>\ProvidesFile{childdoc.def}[2018/12/30 v2.0 child document driver]
%<samplemain>\ProvidesFile{cdocsamp.tex}[2018/12/30 v2.0 sample for childdoc]
%<*driver>
%\ProvidesFile{childdoc.drv}[2018/12/30 v2.0 childdoc reference manual file]
\PassOptionsToClass{10pt,a4paper}{article}
\documentclass{ltxdoc}

\usepackage[margin=35mm]{geometry}
\usepackage{hyperref}
\usepackage{hyperxmp}
\usepackage[usenames]{color}

\hypersetup{colorlinks=true}
\hypersetup{pdfstartview=FitH}
\hypersetup{pdfpagemode=UseNone}
\hypersetup{pdfsource={}}
\hypersetup{pdflang={en-UK}}
\hypersetup{pdfcopyright={Copyright 2017-2018 Niklas Beisert.
  This work may be distributed and/or modified under the
  conditions of the LaTeX Project Public License, either version 1.3
  of this license or (at your option) any later version.}}
\hypersetup{pdflicenseurl={http://www.latex-project.org/lppl.txt}}
\hypersetup{pdfcontactaddress={ETH Zurich, ITP, HIT K,
  Wolfgang-Pauli-Strasse 27}}
\hypersetup{pdfcontactpostcode={8093}}
\hypersetup{pdfcontactcity={Zurich}}
\hypersetup{pdfcontactcountry={Switzerland}}
\hypersetup{pdfcontactemail={nbeisert@itp.phys.ethz.ch}}
\hypersetup{pdfcontacturl={http://people.phys.ethz.ch/\xmptilde nbeisert/}}

\newcommand{\secref}[1]{\hyperref[#1]{section \ref*{#1}}}

\parskip1ex
\parindent0pt
\let\olditemize\itemize
\def\itemize{\olditemize\parskip0pt}

\begin{document}

\title{The \textsf{childdoc} Package}
\hypersetup{pdftitle={The childdoc Package}}
\author{Niklas Beisert\\[2ex]
  Institut f\"ur Theoretische Physik\\
  Eidgen\"ossische Technische Hochschule Z\"urich\\
  Wolfgang-Pauli-Strasse 27, 8093 Z\"urich, Switzerland\\[1ex]
  \href{mailto:nbeisert@itp.phys.ethz.ch}
  {\texttt{nbeisert@itp.phys.ethz.ch}}}
\hypersetup{pdfauthor={Niklas Beisert}}
\hypersetup{pdfsubject={Manual for the LaTeX2e Package childdoc}}
\date{30 December 2018, \textsf{v2.0}}
\maketitle

\begin{abstract}\noindent
\textsf{childdoc} is a \LaTeXe{} package
that enables the direct compilation
of document sections included by |\include|
to individual files.
\end{abstract}

\begingroup
\parskip0ex
\tableofcontents
\endgroup

%%%%%%%%%%%%%%%%%%%%%%%%%%%%%%%%%%%%%%%%%%%%%%%%%%%%%%%%%%%%%%%%%%%%%%%%%%%%%%%%
%%%%%%%%%%%%%%%%%%%%%%%%%%%%%%%%%%%%%%%%%%%%%%%%%%%%%%%%%%%%%%%%%%%%%%%%%%%%%%%%
\section{Introduction}

\LaTeX{} provides a mechanism to structure a large document (such as a book)
into a main file and several child files (containing the chapters)
using the |\include| command.
This mechanism is beneficial for documents
which span hundreds of pages in order to
make the source file(s) more manageable.
Moreover, compilation can be restricted to
selected child files by means of the |\includeonly| command.
The latter feature can be used to reduce the compilation time while editing
(this was significantly more useful in the earlier days of \LaTeX{})
or to generate a smaller document which is easier to navigate.
Another application of |\includeonly| is to generate
documents consisting of selected parts of the complete document.

However, there are a few drawbacks of the plain |\include| mechanism:
\begin{itemize}
\item
The child files cannot be compiled on their own,
they can only be compiled via the main file.
A naive editing environment
(such as a text editor with an option
to have the current file processed by \LaTeX)
may require one to switch to the main file before compiling;
attempting to compile the child file produces errors.
\item
The main file must be modified (each time)
to adjust the |\includeonly| command
to the present needs. This easily leaves the main file in a messy state.
\item
The generated document will always carry the filename
of the main document. This is inconvenient if
several child files are to be compiled and
to be kept for distribution.
\end{itemize}

The present package provides a simple interface
to make child files individually compilable by \LaTeX{}.
Compiling a child file then has the same effect as compiling
the main file with an |\includeonly| command
to select the appropriate child.
Moreover the generated document will carry the name of the child
rather than the main file.
This resolves all three above issues.

This feature is meant to make the editing of books,
thesis documents and lecture notes somewhat more convenient.
However, the package can also be used efficiently for
composing a series of documents (such as exercise sheets)
which are typically distributed individually.
It then assists the author in generating the individual documents
(potentially in different versions)
as well as a document containing the collected series.
Another application is in developing style files
or other kinds of included material
where compilation of the style file could redirect
to a sample or test file.

%%%%%%%%%%%%%%%%%%%%%%%%%%%%%%%%%%%%%%%%%%%%%%%%%%%%%%%%%%%%%%%%%%%%%%%%%%%%%%%%
%%%%%%%%%%%%%%%%%%%%%%%%%%%%%%%%%%%%%%%%%%%%%%%%%%%%%%%%%%%%%%%%%%%%%%%%%%%%%%%%
\section{Usage}

First of all, the package \textsf{childdoc} is \emph{not} a standard
\LaTeXe{} |.sty| style file! Therefore it needs to be invoked in
a non-standard way.

%%%%%%%%%%%%%%%%%%%%%%%%%%%%%%%%%%%%%%%%%%%%%%%%%%%%%%%%%%%%%%%%%%%%%%%%%%%%%%%%
\subsection{Included Files}
\label{sec:include}

%%%%%%%%%%%%%%%%%%%%%%%%%%%%%%%%%%%%%%%%
\DescribeMacro{\childdocmain}
To use the package, add the commands
\begin{center}
\begin{tabular}{l}
|\input{childdoc.def}|\\
|\childdocmain{}|\\
\end{tabular}
\end{center}
at the very top of the main \LaTeX{} file,
in particular \emph{before} the |\documentclass| statement!
The argument of |\childdocmain| should be left empty
(but it must be present).

%%%%%%%%%%%%%%%%%%%%%%%%%%%%%%%%%%%%%%%%
\DescribeMacro{\childdocof}
Furthermore, add the commands
\begin{center}
\begin{tabular}{l}
|\input{childdoc.def}|\\
|\childdocof{|\textit{main}|}|\\
\end{tabular}
\end{center}
at the top of every child file \textit{child}
which is included by |\include{|\textit{child}|}|
from within the main file
(or at least for those files to be compiled individually).
The argument \textit{main} must be the filename of the main file.

There are a couple of
considerations in setting up the main and child documents:

%%%%%%%%%%%%%%%%%%%%%%%%%%%%%%%%%%%%%%%%
\paragraph{Restrictions.}

Please note the following restrictions:
\begin{itemize}
\item
|\childdocmain| must be called with one argument \textit{main}
to ensure compatibility with earlier version of the package.
It must either be empty (|\childdocmain{}|)
or precisely match the filename of the main file in which it is specified.
See \secref{sec:detection} for further information.
\item
The filename \textit{main} must be specified without the |.tex| extension.
\item
The filename \textit{main} is case sensitive
(even in case-insensitive file systems)
due to internal string comparison.
\item
The argument \textit{main} should be fully expanded, it cannot be a macro.
\item
Subdirectories and special characters should be avoided in filenames.
\item
The command |\childdocmain{|\textit{main}|}| must be followed by a whitespace.
It should not be followed immediately by another command
or by a comment mark `|%|'.
This is because the \TeX{} parser reads the token immediately following
the argument of |\childdocmain| and puts it
at the beginning of every child section;
however, a white\-space is ignored.
\end{itemize}

%%%%%%%%%%%%%%%%%%%%%%%%%%%%%%%%%%%%%%%%
\paragraph{Content of Main File.}

It is advisable to place all content in the child files included by |\include|.
Any output contained in the main file will appear in all child documents
unless suppressed manually;
it cannot be suppressed automatically by the |\includeonly| directive
and thus should normally be avoided.
A method to include some content in the main file
by means of conditional processing is described in \secref{sec:conditional}.

%%%%%%%%%%%%%%%%%%%%%%%%%%%%%%%%%%%%%%%%
\paragraph{Page Numbering.}

When only a part of the document is compiled,
the appropriate numbering of pages
(as well as other status parameters)
is determined from the |.aux| files.
The latter contain information from previous passes.
However this information needs to propagate through
all intermediate child documents.
Therefore the page numbering in child documents may well
be inconsistent until the complete document is compiled at least once.

A useful (if unconventional) way to always ensure a consistent
page numbering is to restart the numbering in each child document
and denote the pages by `\textit{child}|.|\textit{page}'
where \textit{child} represents the chapter/section number of the child file.
This can be achieved by the command
|\numberwithin{page}{|\textit{child}|}|
of the \textsf{amsmath} package
where \textit{child} can be |chapter| or |section|
depending on the chosen structuring.
Alternatively, one can modify the macro |\thepage| appropriately
and reset the counter |page| at the start of each child file.

%%%%%%%%%%%%%%%%%%%%%%%%%%%%%%%%%%%%%%%%%%%%%%%%%%%%%%%%%%%%%%%%%%%%%%%%%%%%%%%%
\subsection{Conditional Processing}
\label{sec:conditional}

The package provides a mechanism to compile different versions
of a document. To customise the versions further some conditional processing
can come in handy to distinguish which version is being compiled.
The package provides two macros to describe the compilation context:

%%%%%%%%%%%%%%%%%%%%%%%%%%%%%%%%%%%%%%%%
\DescribeMacro{\ifchilddoc}
The conditional |\ifchilddoc| distinguishes between the compilation of
child documents and the main document:
%
\begin{center}
|\ifchilddoc |\textit{child-code}| |[|\||else |\textit{main-code}]| \||fi|
\end{center}

%%%%%%%%%%%%%%%%%%%%%%%%%%%%%%%%%%%%%%%%
\DescribeMacro{\childdocname}
\DescribeMacro{\childdocjob}
The macro |\childdocname| contains the filename (without extension)
of the main or child file being processed.
Note that |\childdocjob| will always contain the name of the main file.

%%%%%%%%%%%%%%%%%%%%%%%%%%%%%%%%%%%%%%%%
\paragraph{Title Page.}

Conditional processing can be used to include a title or banner page
in the main document when proper precautions are taken.
Importantly, the code in the main file should ensure that the page counter
(as well as other status parameters which are stored in the |.aux| files)
takes the same value after the conditional processing.
Otherwise the page numbers may take divergent values
depending on which part is compiled.

For example, a title page could be declared by:
%
\begin{center}
\begin{tabular}{l}
|\ifchilddoc\||else|\\
|\addtocounter{page}{-1}|\\
\textit{code for title page}\\
|\newpage|\\
|\||fi|
\end{tabular}
\end{center}
%
A banner page for the child documents can be generated by:
%
\begin{center}
\begin{tabular}{l}
|\ifchilddoc|\\
|\addtocounter{page}{-1}|\\
\textit{code for banner page}\\
|\newpage|\\
|\||fi|
\end{tabular}
\end{center}
%
Here one could write a message such as:
\begin{center}
|This is the part \childdocname{} of \childdocjob{}.|
\end{center}

%%%%%%%%%%%%%%%%%%%%%%%%%%%%%%%%%%%%%%%%%%%%%%%%%%%%%%%%%%%%%%%%%%%%%%%%%%%%%%%%
\subsection{Flags}
\label{sec:flags}

The package makes it easy to generate different versions
of the main or child documents.
To this end compilation flags can be defined
and assigned different default values.
They will be particularly useful in conjunction
with the forwarding mechanism described in \secref{sec:forward}.

For example, it may be useful to have a flag |\version|
which can be set to |draft| or |final|.
The document source will contain some conditional code
depending on the value of |\version|.
Suppose further, the flag should default to |final| for the main file
and to |draft| for child files
which is a natural assignment for editing the document.
This is achieved by placing the following code
in the preamble of the main document
(below the |\childdocmain| directive):
%
\begin{center}
\begin{tabular}{l}
|\ifchilddoc|\\
|\providecommand{\version}{draft}|\\
|\||else|\\
|\providecommand{\version}{final}|\\
|\||fi|
\end{tabular}
\end{center}
%
The definition by |\providecommand| makes sure
that previous definitions are not overwritten.
Further statements |\providecommand{\version}{...}|
can thus be added before the above code to override it.

For the main file, one might add a line
(between |\childdocmain| and the above block)
%
\begin{center}
|%\ifchilddoc\||else\providecommand{\version}{draft}\||fi|
\end{center}
%
which can be uncommented to produce a draft version.
Likewise one can add a line to the very top of a child file
(above the |\childdocof{|\textit{main}|}| directive)
%
\begin{center}
|%\providecommand{\version}{final}|
\end{center}
%
which can be uncommented to produce the final version of this child document.

%%%%%%%%%%%%%%%%%%%%%%%%%%%%%%%%%%%%%%%%%%%%%%%%%%%%%%%%%%%%%%%%%%%%%%%%%%%%%%%%
\subsection{Forwarding}
\label{sec:forward}

Different versions of the main or child documents
using compilation flags as described in \secref{sec:flags}
can be (permanently) stored in different files
for convenient compilation, viewing and distribution.
To this end, the package defines a command
to pass on compilation to a different file:

%%%%%%%%%%%%%%%%%%%%%%%%%%%%%%%%%%%%%%%%
\DescribeMacro{\childdocforward}
The command |\childdocforward| redirects processing to
another source file:
%
\begin{center}
\begin{tabular}{l}
|\input{childdoc.def}|\\
|\childdocforward[|\textit{main}|]{|\textit{dest}|}|\\
\end{tabular}
\end{center}
%
The argument \textit{dest} is the destination file
(without extension).
It should be the main file or one of the child files.
Note that further \textsf{childdoc} directives
such as |\childdocof| and |\childdocforward|
in the indicated file will be processed in this form.
The optional argument \textit{main}
passes on directly to the main file \textit{main}
while pretending to compile the child \textit{dest}.
This form behaves as if \textit{dest}
issues |\childdocof{|\textit{main}|}| right away,
and no further \textsf{childdoc} directives will be processed.

%%%%%%%%%%%%%%%%%%%%%%%%%%%%%%%%%%%%%%%%
\DescribeMacro{\...prefix}
In the alternative form |\childdocforwardprefix|,
%
\begin{center}
\begin{tabular}{l}
|\input{childdoc.def}|\\
|\childdocforwardprefix[|\textit{main}|]{|\textit{prefix}|}{|\textit{dest}|}|
\end{tabular}
\end{center}
%
the destination file is determined by a pattern
depending on the current file:
To make this work, the current file must be called
`{\textit{prefix}\hspace{0.2em}\textit{suffix}}'
with \textit{prefix} matching precisely the argument.
Processing is then passed on to the file
`{\textit{dest}\hspace{0.2em}\textit{suffix}}'.
Surely, the same effect is achieved by
directly specifying the
argument `{\textit{dest}\hspace{0.2em}\textit{suffix}}'
in the first form.
However, that requires to set up a different file
for each child. With the alternative form of the command
all these files can have exactly the same content
which simplifies setting them up and maintaining them.

For example, the following file |draft.tex|
with a compilation flag |\version| as described in \secref{sec:flags}
compiles the main document as a draft:
%
\begin{center}
\begin{tabular}{l}
|\def\version{draft}|\\
|\input{childdoc.def}|\\
|\childdocforward{|\textit{main}|}|
\end{tabular}
\end{center}
%
Likewise, the following files |final|\textit{nn}|.tex|
compile the final version of the child document
|child|\textit{nn}|.tex|:
%
\begin{center}
\begin{tabular}{l}
|\def\version{final}|\\
|\input{childdoc.def}|\\
|\childdocforwardprefix{final}{child}|
\end{tabular}
\end{center}
%

Note that when several versions of a main file and/or of each child file
are to be generated, it may be convenient to set up a |Makefile| or
shell script to automatise the process.

%%%%%%%%%%%%%%%%%%%%%%%%%%%%%%%%%%%%%%%%%%%%%%%%%%%%%%%%%%%%%%%%%%%%%%%%%%%%%%%%
\subsection{Command Line Processing}
\label{sec:commandline}

The effect of redirection files can also be achieved by invoking
the \LaTeX{} compiler with a more elaborate command line.
Most conveniently this should be done as part
of a shell script or a |Makefile|.

When using \textsf{childdoc} in the main file, the following
command lines effectively perform a redirection
(note that depending on the shell being used,
backslashes may have to be doubled: `|\|' $\to$ `|\\|'):
%
\begin{center}
|... -jobname "|\textit{target}|" |\\|"|[\textit{flags}]%
|\input{childdoc.def}\childdocforward[|\textit{main}|]{|\textit{dest}|}"|
\end{center}
%
Here \textit{target} is the name of the output file,
\textit{main} is the name of the main file
and \textit{dest} is the name of the main or child file to be processed
(all filenames without extensions).
The optional argument \textit{main} can be omitted
if \textit{main} matches \textit{dest}.
Optionally, compilation \textit{flags} can be defined via |\def| commands.
This command line makes the \TeX{} engine believe
it is compiling the file \textit{target}
whose content is specified as the latter parameter.
The provided code then forwards the processing to
\textit{main} or \textit{dest} as described in \secref{sec:forward}.

%%%%%%%%%%%%%%%%%%%%%%%%%%%%%%%%%%%%%%%%%%%%%%%%%%%%%%%%%%%%%%%%%%%%%%%%%%%%%%%%
\subsection{Include by Input}
\label{sec:input}

Including child documents by |\include| has some restrictions by design.
Most notably, the content of a child document always occupies
its own set of pages; pages cannot be shared between child documents.
Usually, this behaviour makes perfect sense
because each child document contain an essential part of the document.
However, in some situations it may be desirable to compose
a document from a collection of parts
without having mandatory page breaks between then.
For this case, the package
provides a mechanism to include parts
by |\input| which can also be processed individually.
However, by construction this mechanism
requires manual handling of the content to be output.

%%%%%%%%%%%%%%%%%%%%%%%%%%%%%%%%%%%%%%%%
\DescribeMacro{\ifchilddocmanual}
The main file should be prepared as usual, see \secref{sec:include}.
However, the document body must make a distinction
between processing of an individual part and of the main document, e.g.:
%
\begin{center}
\begin{tabular}{l}
|\ifchilddocmanual|\\
|\input{\childdocname}|\\
|\||else|\\
\textit{document body with }|\input{|\textit{part}|}|\\
|\||fi|
\end{tabular}
\end{center}
%
The conditional |\ifchilddocmanual| is true whenever
a part to be included by |\input| is being compiled,
and the name of the part is stored in |\childdocname|.

%%%%%%%%%%%%%%%%%%%%%%%%%%%%%%%%%%%%%%%%
\DescribeMacro{\childdocby}
Each part to be included by |\input| should start with:
%
\begin{center}
\begin{tabular}{l}
|\input{childdoc.def}|\\
|\childdocby{|\textit{main}|}|\\
\end{tabular}
\end{center}
%
The directive |\childdocby| is similar to |\childdocof|
described in \secref{sec:include},
but the subsequent selection of content must be done manually.
To that end, both |\ifchilddoc| and |\ifchilddocmanual|
will be true upon processing of a part,
and the name of the part is stored in |\childdocname|.
Note that |\jobname| will be set to the filename of the current part
so that each part receives an individual |.aux| file
that does not interfere with the |.aux| file(s) of the main document.
This behaviour can be altered by the alternative form
|\childdocby[*]{|\textit{main}|}| (with a non-empty optional argument)
which uses the |.aux| file of the main document
by setting |\jobname| to \textit{main}.

%%%%%%%%%%%%%%%%%%%%%%%%%%%%%%%%%%%%%%%%%%%%%%%%%%%%%%%%%%%%%%%%%%%%%%%%%%%%%%%%
\subsection{Driver Development}
\label{sec:driver}

The \textsf{childdoc} mechanism can also be use for the development
of definition files such as \LaTeX{} styles or classes.
This case differs from the above setup with multiple parts
included by |\include| in that no |\includeonly| should be invoked.
This can be achieved by starting the include file
(before |\ProvidesPackage|) with:
%
\begin{center}
\begin{tabular}{l}
|\input{childdoc.def}|\\
|\childdocforward{|\textit{main}|}|\\
\end{tabular}
\end{center}
%
or alternatively with:
%
\begin{center}
\begin{tabular}{l}
|\input{childdoc.def}|\\
|\childdocby{|\textit{main}|}|\\
\end{tabular}
\end{center}
%
Both forms have slightly different effects as described above.
The main file is prepared as usual, see \secref{sec:include}.

%%%%%%%%%%%%%%%%%%%%%%%%%%%%%%%%%%%%%%%%%%%%%%%%%%%%%%%%%%%%%%%%%%%%%%%%%%%%%%%%
\subsection{Legacy Detection}
\label{sec:detection}

The directive |\childdocmain| in the main file can detect
whether the complete document or merely a child is to be compiled
even without using the directive |\childdocof|.
This method is deprecated because it is less robust
and there is no compelling reason to use it;
it is merely provided for backward compatibility
and it may be removed in future versions.

If the detection mechanism is to be used,
it is mandatory to correctly specify
the filename of the main file as the argument of |\childdocmain|:
%
\begin{center}
\begin{tabular}{l}
|\input{childdoc.def}|\\
|\childdocmain{|\textit{main}|}|\\
\end{tabular}
\end{center}
%
If |\jobname| does not match the argument \textit{main} of |\childdocmain|,
it is assumed that |\jobname| points to the child file to be compiled.
When using |\childdocmain| with the main file specified as argument,
it suffices to start a child file
with just |\input{|\textit{main}|}|
without loading of the package and using |\childdocof|.
If instead all processing is done
with the appropriate \textsf{childdoc} directives,
the argument of \textit{main} of |\childdocmain| can be empty.

An alternative version of the command line processing described
in \secref{sec:commandline} using the detection mechanism reads:
%
\begin{center}
|... -jobname "|\textit{target}|" "|[\textit{flags}]%
[|\def\jobname{|\textit{dest}|}|]|\input{|\textit{main}|}"|
\end{center}

%%%%%%%%%%%%%%%%%%%%%%%%%%%%%%%%%%%%%%%%%%%%%%%%%%%%%%%%%%%%%%%%%%%%%%%%%%%%%%%%
\subsection{Manual Code}
\label{sec:manual}

In case one cannot be certain whether the definitions file |childdoc.def|
is installed on the target \TeX{} distribution
and one prefers not to ship it,
it is conceivable to paste a few relevant commands into the sources.

To that end, drop all statements |\input{childdoc.def}|
and perform the replacements as outlined below.
Instead of |\childdocmain{|\textit{main}|}| add the following code
to the top of the main file:
%
\begin{center}
\begin{tabular}{l}
|\||ifdefined\childdocname\endinput\||fi\newif\ifchilddoc|\\
|\edef\childdocname{\scantokens\expandafter{\jobname\noexpand}}|\\
|\def\childdocmain{|\textit{main}|}\||ifx\childdocmain\childdocname\||else|\\
|\childdoctrue\includeonly{\childdocname}\let\jobname\childdocmain\||fi|\\
\end{tabular}
\end{center}
%
Instead of |\childdocof{|\textit{main}|}| just include the main file
at the top of each child file:
%
\begin{center}
|\input{|\textit{main}|}|
\end{center}
%
A simple redirection |\childdocforward{|\textit{dest}|}| is achieved by:
%
\begin{center}
|\def\jobname{|\textit{dest}|}\input{\jobname}|
\end{center}
%
The redirection with prefix
|\childdocforwardprefix[|\textit{prefix}|]{|\textit{dest}|}|
is accomplished by:
%
\begin{center}
\begin{tabular}{l}
|{\edef\jobname{\scantokens\expandafter{\jobname\noexpand}}|\\
|\def\redirectjob |\textit{prefix}|#1~~~{\gdef\jobname{|\textit{dest}|#1}}|\\
|\expandafter\redirectjob\jobname~~~}\input{\jobname}|
\end{tabular}
\end{center}

In an alternative approach,
child documents can be compiled by a specific command line
without additional code or specific definitions:
%
\begin{center}
|... -jobname "|\textit{target}|" "|[\textit{flags}]%
|\includeonly{|\textit{dest}|}\input{|\textit{main}|}"|
\end{center}
%

%%%%%%%%%%%%%%%%%%%%%%%%%%%%%%%%%%%%%%%%%%%%%%%%%%%%%%%%%%%%%%%%%%%%%%%%%%%%%%%%
%%%%%%%%%%%%%%%%%%%%%%%%%%%%%%%%%%%%%%%%%%%%%%%%%%%%%%%%%%%%%%%%%%%%%%%%%%%%%%%%
\section{Information}

%%%%%%%%%%%%%%%%%%%%%%%%%%%%%%%%%%%%%%%%%%%%%%%%%%%%%%%%%%%%%%%%%%%%%%%%%%%%%%%%
\subsection{Copyright}

Copyright \copyright{} 2017--2018 Niklas Beisert

This work may be distributed and/or modified under the
conditions of the \LaTeX{} Project Public License, either version 1.3
of this license or (at your option) any later version.
The latest version of this license is in
  \url{http://www.latex-project.org/lppl.txt}
and version 1.3 or later is part of all distributions of \LaTeX{}
version 2005/12/01 or later.

This work has the LPPL maintenance status `maintained'.

The Current Maintainer of this work is Niklas Beisert.

This work consists of the files |README.txt|, |childdoc.ins| and |childdoc.dtx|
as well as the derived files |childdoc.def|, |cdocsamp.tex|
with |cdocsch1.tex|, |cdocsch2.tex|, |cdocspt3.tex|, |cdocspt4.tex|,
|cdocsdrf.tex|, |cdocsfn1.tex|, |cdocsfn2.tex|
as well as |childdoc.pdf|.

%%%%%%%%%%%%%%%%%%%%%%%%%%%%%%%%%%%%%%%%%%%%%%%%%%%%%%%%%%%%%%%%%%%%%%%%%%%%%%%%
\subsection{Files and Installation}

The package consists of the files:
%
\begin{center}
\begin{tabular}{ll}
    |README.txt|   & readme file \\
    |childdoc.ins| & installation file \\
    |childdoc.dtx| & source file \\
    |childdoc.def| & definition file \\
    |cdocsamp.tex| & sample main file \\
    |cdocsch1.tex| & sample include file \\
    |cdocsch2.tex| & sample include file \\
    |cdocspt3.tex| & sample part file \\
    |cdocspt4.tex| & sample part file \\
    |cdocsdrf.tex| & sample redirection file \\
    |cdocsfn1.tex| & sample redirection file \\
    |cdocsfn2.tex| & sample redirection file \\
    |childdoc.pdf| & manual
\end{tabular}
\end{center}
%
The distribution consists of the files
|README.txt|, |childdoc.ins| and |childdoc.dtx|.
%
\begin{itemize}
\item
Run (pdf)\LaTeX{} on |childdoc.dtx|
to compile the manual |childdoc.pdf| (this file).
\item
Run \LaTeX{} on |childdoc.ins| to create the definitions file |childdoc.def|
and the sample |cdocsamp.tex| with include files
|cdocsch1.tex|, |cdocsch2.tex|, |cdocspt3.tex|, |cdocspt4.tex|,
|cdocsdrf.tex|, |cdocsfn1.tex|, |cdocsfn2.tex|.
Then copy the file |childdoc.def| to an appropriate directory of your \LaTeX{}
distribution, e.g.\ \textit{texmf-root}|/tex/latex/childdoc|.
\end{itemize}

%%%%%%%%%%%%%%%%%%%%%%%%%%%%%%%%%%%%%%%%%%%%%%%%%%%%%%%%%%%%%%%%%%%%%%%%%%%%%%%%
\subsection{Related CTAN Packages}

There are several other packages which offer a similar functionality:
%
\begin{itemize}
\item
The packages
\href{http://ctan.org/pkg/docmute}{\textsf{docmute}},
\href{http://ctan.org/pkg/includex}{\textsf{includex}} and
\href{http://ctan.org/pkg/standalone}{\textsf{standalone}}
provide commands to include only the document body of
a child file thus allowing both files to be compiled individually.
\item
The packages \href{http://ctan.org/pkg/subdocs}{\textsf{subdocs}}
and \href{http://ctan.org/pkg/subfiles}{\textsf{subfiles}}
provide structures in which the main and child documents can be
encapsulated and allowing them to be compiled individually.
The inclusion mechanism is different from the conventional |\include|.
\item
The package \href{http://ctan.org/pkg/combine}{\textsf{combine}}
is an elaborate solution to combine several documents into one.
\end{itemize}
%
See also the CTAN topic \href{http://ctan.org/topic/subdocs}{\textsf{subdocs}}
for further related packages.
The present package differs from the above solutions in that
a document structure constructed with the conventional |\include| mechanism
just needs two extra commands at the top of every file
such that all constituent files can be compiled individually.

%%%%%%%%%%%%%%%%%%%%%%%%%%%%%%%%%%%%%%%%%%%%%%%%%%%%%%%%%%%%%%%%%%%%%%%%%%%%%%%%
%\subsection{Feature Suggestions}
%
%The following is a list of features which may be useful for future
%versions of this package:
%%
%\begin{itemize}
%\item
%\ldots
%\end{itemize}

%%%%%%%%%%%%%%%%%%%%%%%%%%%%%%%%%%%%%%%%%%%%%%%%%%%%%%%%%%%%%%%%%%%%%%%%%%%%%%%%
\subsection{Revision History}

%%%%%%%%%%%%%%%%%%%%%%%%%%%%%%%%%%%%%%%%
\paragraph{v2.0:} 2018/12/30

\begin{itemize}
\item
immediate forward processing
\item
added |\childdocby| mechanism
\item
manual restructured
\end{itemize}

%%%%%%%%%%%%%%%%%%%%%%%%%%%%%%%%%%%%%%%%
\paragraph{v1.6:} 2018/01/17

\begin{itemize}
\item
application for development of include files
\item
corrections to manual
\end{itemize}

%%%%%%%%%%%%%%%%%%%%%%%%%%%%%%%%%%%%%%%%
\paragraph{v1.5:} 2017/05/21

\begin{itemize}
\item
more complete structuring introduced
\item
|\childdocof| introduced
\item
|\childdoc| renamed to |\childdocmain|
\item
|\childredirect| renamed to |\childdocforward| and |\childdocforwardprefix|
and functionality expanded
\end{itemize}

%%%%%%%%%%%%%%%%%%%%%%%%%%%%%%%%%%%%%%%%
\paragraph{v1.0:} 2017/04/27

\begin{itemize}
\item
manual and install package
\item
first version published on CTAN
\end{itemize}

%%%%%%%%%%%%%%%%%%%%%%%%%%%%%%%%%%%%%%%%
\paragraph{v0.6:} 2017/04/26

\begin{itemize}
\item
redirection mechanism added
\end{itemize}

%%%%%%%%%%%%%%%%%%%%%%%%%%%%%%%%%%%%%%%%
\paragraph{v0.5:} 2017/04/26

\begin{itemize}
\item
functionality in definition file
\end{itemize}


%%%%%%%%%%%%%%%%%%%%%%%%%%%%%%%%%%%%%%%%%%%%%%%%%%%%%%%%%%%%%%%%%%%%%%%%%%%%%%%%
%%%%%%%%%%%%%%%%%%%%%%%%%%%%%%%%%%%%%%%%%%%%%%%%%%%%%%%%%%%%%%%%%%%%%%%%%%%%%%%%
%%%%%%%%%%%%%%%%%%%%%%%%%%%%%%%%%%%%%%%%%%%%%%%%%%%%%%%%%%%%%%%%%%%%%%%%%%%%%%%%
\appendix

\settowidth\MacroIndent{\rmfamily\scriptsize 000\ }

 \DocInput{childdoc.dtx}

\end{document}
%</driver>
% \fi
%
% %%%%%%%%%%%%%%%%%%%%%%%%%%%%%%%%%%%%%%%%%%%%%%%%%%%%%%%%%%%%%%%%%%%%%%%%%%%%%%
% %%%%%%%%%%%%%%%%%%%%%%%%%%%%%%%%%%%%%%%%%%%%%%%%%%%%%%%%%%%%%%%%%%%%%%%%%%%%%%
% \section{Sample}
%\iffalse
%<*samplemain>
%\fi
%
% The following presents a sample document
% with two chapters, two parts, a title page,
% a compile flag as well as three forwarding files to set the flag.
% It consists of eight |.tex| files:
% \begin{center}
% \begin{tabular}{ll}
% |cdocsamp.tex|&main file\\
% |cdocsch1.tex|&include file for chapter 1\\
% |cdocsch2.tex|&include file for chapter 2\\
% |cdocspt3.tex|&include file for part 3\\
% |cdocspt4.tex|&include file for part 4\\
% |cdocsdrf.tex|&forwarding file for main file in draft mode\\
% |cdocsfi1.tex|&forwarding file for final version of chapter 1\\
% |cdocsfi2.tex|&forwarding file for final version of chapter 2\\
% \end{tabular}
% \end{center}
% Each of the eight files can be compiled directly by the \LaTeX{} compiler.
%
% %%%%%%%%%%%%%%%%%%%%%%%%%%%%%%%%%%%%%%
% \paragraph{Main File.}
%
% The main file is called |cdocsamp.tex|.
%
% Load the \textsf{childdoc} definitions and
% declare the filename for the main document:
%    \begin{macrocode}
\input{childdoc.def}
\childdocmain{}
%    \end{macrocode}

% Optional override for |\version| flag:
%    \begin{macrocode}
%%\ifchilddoc\else\providecommand{\version}{draft}\fi
%    \end{macrocode}

% Define the default values for the |\version| flag
% (|final| for the main file and |draft| for childs):
%    \begin{macrocode}
\ifchilddoc
\providecommand{\version}{draft}
\else
\providecommand{\version}{final}
\fi
%    \end{macrocode}

% Load the standard document class:
%    \begin{macrocode}
\documentclass[12pt]{article}
%    \end{macrocode}

% Start the document body:
%    \begin{macrocode}
\begin{document}
%    \end{macrocode}

% Declare a title page.
% Print title, part of document being processed and version flag:
%    \begin{macrocode}
\addtocounter{page}{-1}
\begin{center}
{\LARGE\bfseries{}childdoc example\par}
\vspace{1cm}
\ifchilddoc
\ifchilddocmanual part\else chapter\fi:
`\childdocname' of `\childdocjob'\par
\else
main document: `\childdocjob'\par
\fi
version: \version\par
\end{center}
\newpage
%    \end{macrocode}

% Manually include selected file,
% otherwise process as usual:
%    \begin{macrocode}
\ifchilddocmanual
\section*{part `\childdocname'}
\input{\childdocname}
\else
%    \end{macrocode}

% Include the two chapters:
%    \begin{macrocode}
\include{cdocsch1}
\include{cdocsch2}
%    \end{macrocode}

% Include the two parts unless only chapters should be displayed:
%    \begin{macrocode}
\ifchilddoc\else
\section{part three}
\input{cdocspt3}
\section{part four}
\input{cdocspt4}
\fi
%    \end{macrocode}

% Process as usual until here:
%    \begin{macrocode}
\fi
%    \end{macrocode}

% End of document body:
%    \begin{macrocode}
\end{document}
%    \end{macrocode}
%\iffalse
%</samplemain>
%\fi
%
% %%%%%%%%%%%%%%%%%%%%%%%%%%%%%%%%%%%%%%
% \paragraph{Chapter Include Files.}
%
% The include files are called |cdocsch1.tex| and |cdocsch2.tex|.
%
%\iffalse
%<*samplechap1|samplechap2>
%\fi

% Optional override for |\version| flag:
%    \begin{macrocode}
%%\providecommand{\version}{final}
%    \end{macrocode}

% Include the main document:
%    \begin{macrocode}
\input{childdoc.def}
\childdocof{cdocsamp}
%    \end{macrocode}

%\iffalse
%</samplechap1|samplechap2>
%\fi
%
%\iffalse
%<*samplechap1>
%\fi
% Some text for chapter 1:
%    \begin{macrocode}
\section{one}
some text in chapter one
%    \end{macrocode}

%\iffalse
%</samplechap1>
%\fi
% Some text for chapter 2:
%\iffalse
%<*samplechap2>
%\fi
%    \begin{macrocode}
\section{two}
more text in chapter two
%    \end{macrocode}

%\iffalse
%</samplechap2>
%\fi
%
% %%%%%%%%%%%%%%%%%%%%%%%%%%%%%%%%%%%%%%
% \paragraph{Part Include Files.}
%
% The include files are called |cdocspt3.tex| and |cdocspt4.tex|.
%
%\iffalse
%<*samplepart3|samplepart4>
%\fi

% Optional override for |\version| flag:
%    \begin{macrocode}
%%\providecommand{\version}{final}
%    \end{macrocode}

% Include the main document:
%    \begin{macrocode}
\input{childdoc.def}
\childdocby{cdocsamp}
%    \end{macrocode}

%\iffalse
%</samplepart3|samplepart4>
%\fi
%
%\iffalse
%<*samplepart3>
%\fi
% Some text for part 3:
%    \begin{macrocode}
some text in part three
%    \end{macrocode}

%\iffalse
%</samplepart3>
%\fi
% Some text for part 4:
%\iffalse
%<*samplepart4>
%\fi
%    \begin{macrocode}
more text in part four
%    \end{macrocode}

%\iffalse
%</samplepart4>
%\fi
%
% %%%%%%%%%%%%%%%%%%%%%%%%%%%%%%%%%%%%%%
% \paragraph{Forwarding for a Complete Draft.}
%
% The following forwarding file |cdocsdrf.tex|
% compiles the main document in draft mode:
%\iffalse
%<*sampledraft>
%\fi
%    \begin{macrocode}
\def\version{draft}
\input{childdoc.def}
\childdocforward{cdocsamp}
%    \end{macrocode}

%\iffalse
%</sampledraft>
%\fi
%
% %%%%%%%%%%%%%%%%%%%%%%%%%%%%%%%%%%%%%%
% \paragraph{Forwarding for Final Version of the Chapters.}
%
% The following forwarding files |cdocsfn1.tex| and |cdocsfn2.tex|
% (with identical content)
% compile the final versions of the child documents
% |cdocsch1.tex| and |cdocsch2.tex|, respectively:
%\iffalse
%<*samplefinal>
%\fi
%    \begin{macrocode}
\def\version{final}
\input{childdoc.def}
\childdocforwardprefix[cdocsamp]{cdocsfn}{cdocsch}
%    \end{macrocode}

%\iffalse
%</samplefinal>
%\fi
%
% %%%%%%%%%%%%%%%%%%%%%%%%%%%%%%%%%%%%%%
% \paragraph{Command Line Processing.}
%
% The following three command lines generate the output files
% |cdocscld|, |cdocscl1| and |cdocscl2|
% which should be identical to
% |cdocsdrf|, |cdocsch1| and |cdocsfn2|, respectively:
% \begin{center}
% \begin{tabular}{l}
% |latex -jobname cdocscld \|\\
% |  "\def\version{draft}\input{childdoc.def}\childdocforward{cdocsamp}"|\\
% |latex -jobname cdocscl1 \|\\
% |  "\input{childdoc.def}\childdocforward[cdocsamp]{cdocsch1}"|\\
% |latex -jobname cdocscl2 \|\\
% |  "\def\version{final}\input{childdoc.def}\childdocforward{cdocsch2}"|
% \end{tabular}
% \end{center}
% Note that the trailing backslash on each first line
% merely continues the input to the second line
% (for convenient cut ant paste).
% Furthermore, the command |latex| can be replaced by any
% of its alternative versions such as |pdflatex|.
%
% %%%%%%%%%%%%%%%%%%%%%%%%%%%%%%%%%%%%%%%%%%%%%%%%%%%%%%%%%%%%%%%%%%%%%%%%%%%%%%
% %%%%%%%%%%%%%%%%%%%%%%%%%%%%%%%%%%%%%%%%%%%%%%%%%%%%%%%%%%%%%%%%%%%%%%%%%%%%%%
% \section{Implementation}
%\iffalse
%<*package>
%\fi
%
% This section describes the definitions file |childdoc.def|.

% The definitions cannot be loaded using |\usepackage| or |\RequirePackage|
% which has a mechanism to prevent loading a style file more than once.
% When loading the definitions by means of |\input|
% multiple instances have to be prevented manually:
%\iffalse
%This code needs to be before the `\ProvidesFile' directive
%which is defined at the beginning of this file.
%Therefore it is also placed there and commented out here.
%</package>
%<*discard>
%\fi
%    \begin{macrocode}
\ifdefined\childdocmain\endinput\fi
%    \end{macrocode}
%\iffalse
%</discard>
%<*package>
%\fi
%
% \macro{\ifchilddoc}
% \macro{\ifchilddocmanual}
% The conditional |\ifchilddoc| tells whether a
% child (true) or main (false) document is being compiled.
% The conditional |\ifchilddocmanual| tells whether
% the |\includeonly| mechanism is used (false) or
% the selection of child files must be performed manually (true).
% The definitions initialise to false:
%    \begin{macrocode}
\newif\ifchilddoc
\newif\ifchilddocmanual
%    \end{macrocode}

% \macro{\childdocname}
% \macro{\childdocjob}
% The macro |\childdocname| stores the name of the main document
% to be compiled. The macro |\childdocjob| stores the name of
% the document on which the \LaTeX{} compiler was originally invoked.
% The content of |\jobname| cannot be compared
% to filenames specified in the source due to different catcodes.
% The following code rescans |\jobname|, stores the result
% in |\childdocname| and saves a copy in |\childdocjob|:
%    \begin{macrocode}
\edef\childdocname{\scantokens\expandafter{\jobname\noexpand}}
\let\childdocjob\childdocname
%    \end{macrocode}

% \macro{\childdocdisable}
% The macro |\childdocdisable| prevents the main file
% from being processed more than once.
% At this stage, the main document command |\childdocmain|
% is assumed to be called once again where it should do nothing.
% Any subsequent call to it should prevent
% a secondary processing of the main document
% It overwrites the forwarding commands
% |\childdocof| and |\childdocforward|
% with empty macros to prevent further inclusions of the main document:
%    \begin{macrocode}
\newcommand{\childdocdisable}
{
  \renewcommand{\childdocmain}[1]{\renewcommand{\childdocmain}[1]{\endinput}}
  \renewcommand{\childdocof}[1]{}
  \renewcommand{\childdocby}[2][]{}
  \renewcommand{\childdocforward}[2][]{}
  \renewcommand{\childdocdisable}{}
}
%    \end{macrocode}

% \macro{\childdocmain}
% The macro |\childdocmain| is to be called at the top of the main file
% with nothing or the main filename (without extension) as argument.
% First, it breaks loops.
% If the argument is not empty and does not match |\childdocname|
% (which is set by the first inclusion of |childdoc.def|),
% |\ifchilddoc| is set to true, |\includeonly| is applied to the child file
% and |\jobname| is set to the main file
% (for proper handling of |.aux| files):
%    \begin{macrocode}
\newcommand{\childdocmain}[1]
{
  \childdocdisable\childdocmain{}
  \if?#1?\else
    \begingroup
      \def\childdoctmp{#1}
      \ifx\childdoctmp\childdocname
        \def\childdoctmp{}
      \else
        \def\childdoctmp
        {
          \childdoctrue
          \includeonly{\childdocname}
          \def\childdocjob{#1}
          \def\jobname{#1}
        }
      \fi
      \expandafter
    \endgroup
    \childdoctmp
  \fi
}
%    \end{macrocode}

% \macro{\childdocof}
% The command |\childdocof| redirects
% compilation to the main file |#1|.
%    \begin{macrocode}
\newcommand{\childdocof}[1]
{
  \childdocdisable
  \childdoctrue
  \includeonly{\childdocname}
  \def\jobname{#1}
  \def\childdocjob{#1}
  \input{#1}
}
%    \end{macrocode}

% \macro{\childdocby}
% The command |\childdocby| ....
%    \begin{macrocode}
\newcommand{\childdocby}[2][]
{
  \childdocdisable
  \childdoctrue
  \childdocmanualtrue
  \if?#1?\else
    \def\jobname{#2}
  \fi
  \def\childdocjob{#2}
  \input{#2}
  \endinput
}
%    \end{macrocode}

% \macro{\childdocforward}
% The command |\childdocforward| redirects
% compilation to the main file or
% (if the optional argument is given) a child file.
% Parameters are set as if the main file
% or a child file starting with |\childdocof| was compiled.
% Then compilation is handed over to the main file:
%    \begin{macrocode}
\newcommand{\childdocforward}[2][]
{
  \begingroup
    \if?#1?
      \def\childdoctmp
      {
        \def\childdocname{#2}
        \def\childdocjob{#2}
        \def\jobname{#2}
        \input{#2}
        \endinput
      }
    \else
      \def\childdoctmp
      {
        \childdocdisable
        \def\childdocname{#2}
        \childdoctrue
        \includeonly{#2}
        \def\childdocjob{#1}
        \def\jobname{#1}
        \input{#1}
        \endinput
      }
    \fi
    \expandafter
  \endgroup
  \childdoctmp
}
%    \end{macrocode}

% \macro{\childdocforwardprefix}
% The command |\childdocforwardprefix| redirects
% compilation to the main or a child file by means of a pattern.
% The prefix |#1| in the current filename is replaced by |#2|
% and the suffix of the current filename is kept
% (it is assumed that the filename does not contain the substring `|~~~|'
% which is used as a delimiter).
% Compilation is handed over to the new file by |\childdocforward|:
%    \begin{macrocode}
\newcommand{\childdocforwardprefix}[3][]
{
  \begingroup
    \def\childdocextract #2##1~~~{\def\childdoctmp{\childdocforward[#1]{#3##1}}}
    \expandafter\childdocextract\childdocname~~~
    \expandafter
  \endgroup
  \childdoctmp
}
%    \end{macrocode}

% \macro{\childdoc}
% The deprecated macro |\childdoc| is a legacy version of |\childdocmain|:
%    \begin{macrocode}
\newcommand{\childdoc}{\childdocmain}
%    \end{macrocode}

% \macro{\childdocredirect}
% The deprecated macro |\childdocredirect| is a legacy version
% of |\childdocforward| and |\childdocforwardprefix|:
%    \begin{macrocode}
\newcommand{\childdocredirect}[2][]
{
  \begingroup
    \if?#1?
      \def\childdoctmp{\childdocforward{#2}}
    \else
      \def\childdoctmp{\childdocforwardprefix{#1}{#2}}
    \fi
    \expandafter
  \endgroup
  \childdoctmp
}
%    \end{macrocode}

%\iffalse
%</package>
%\fi
%
\endinput

\childdocmain{}
%    \end{macrocode}

% Optional override for |\version| flag:
%    \begin{macrocode}
%%\ifchilddoc\else\providecommand{\version}{draft}\fi
%    \end{macrocode}

% Define the default values for the |\version| flag
% (|final| for the main file and |draft| for childs):
%    \begin{macrocode}
\ifchilddoc
\providecommand{\version}{draft}
\else
\providecommand{\version}{final}
\fi
%    \end{macrocode}

% Load the standard document class:
%    \begin{macrocode}
\documentclass[12pt]{article}
%    \end{macrocode}

% Start the document body:
%    \begin{macrocode}
\begin{document}
%    \end{macrocode}

% Declare a title page.
% Print title, part of document being processed and version flag:
%    \begin{macrocode}
\addtocounter{page}{-1}
\begin{center}
{\LARGE\bfseries{}childdoc example\par}
\vspace{1cm}
\ifchilddoc
\ifchilddocmanual part\else chapter\fi:
`\childdocname' of `\childdocjob'\par
\else
main document: `\childdocjob'\par
\fi
version: \version\par
\end{center}
\newpage
%    \end{macrocode}

% Manually include selected file,
% otherwise process as usual:
%    \begin{macrocode}
\ifchilddocmanual
\section*{part `\childdocname'}
\input{\childdocname}
\else
%    \end{macrocode}

% Include the two chapters:
%    \begin{macrocode}
\include{cdocsch1}
\include{cdocsch2}
%    \end{macrocode}

% Include the two parts unless only chapters should be displayed:
%    \begin{macrocode}
\ifchilddoc\else
\section{part three}
\input{cdocspt3}
\section{part four}
\input{cdocspt4}
\fi
%    \end{macrocode}

% Process as usual until here:
%    \begin{macrocode}
\fi
%    \end{macrocode}

% End of document body:
%    \begin{macrocode}
\end{document}
%    \end{macrocode}
%\iffalse
%</samplemain>
%\fi
%
% %%%%%%%%%%%%%%%%%%%%%%%%%%%%%%%%%%%%%%
% \paragraph{Chapter Include Files.}
%
% The include files are called |cdocsch1.tex| and |cdocsch2.tex|.
%
%\iffalse
%<*samplechap1|samplechap2>
%\fi

% Optional override for |\version| flag:
%    \begin{macrocode}
%%\providecommand{\version}{final}
%    \end{macrocode}

% Include the main document:
%    \begin{macrocode}
% \iffalse
%
% childdoc.dtx Copyright (C) 2017-2018 Niklas Beisert
%
% This work may be distributed and/or modified under the
% conditions of the LaTeX Project Public License, either version 1.3
% of this license or (at your option) any later version.
% The latest version of this license is in
%   http://www.latex-project.org/lppl.txt
% and version 1.3 or later is part of all distributions of LaTeX
% version 2005/12/01 or later.
%
% This work has the LPPL maintenance status `maintained'.
%
% The Current Maintainer of this work is Niklas Beisert.
%
% This work consists of the files childdoc.dtx and childdoc.ins
% and the derived files childdoc.def and cdocsamp.tex with
% cdocsch1.tex, cdocsch2.tex, cdocsdrf.tex, cdocsfn1.tex, cdocsfn2.tex.
%
%<package>\ifdefined\childdocmain\endinput\fi
%<package>\ProvidesFile{childdoc.def}[2018/12/30 v2.0 child document driver]
%<samplemain>\ProvidesFile{cdocsamp.tex}[2018/12/30 v2.0 sample for childdoc]
%<*driver>
%\ProvidesFile{childdoc.drv}[2018/12/30 v2.0 childdoc reference manual file]
\PassOptionsToClass{10pt,a4paper}{article}
\documentclass{ltxdoc}

\usepackage[margin=35mm]{geometry}
\usepackage{hyperref}
\usepackage{hyperxmp}
\usepackage[usenames]{color}

\hypersetup{colorlinks=true}
\hypersetup{pdfstartview=FitH}
\hypersetup{pdfpagemode=UseNone}
\hypersetup{pdfsource={}}
\hypersetup{pdflang={en-UK}}
\hypersetup{pdfcopyright={Copyright 2017-2018 Niklas Beisert.
  This work may be distributed and/or modified under the
  conditions of the LaTeX Project Public License, either version 1.3
  of this license or (at your option) any later version.}}
\hypersetup{pdflicenseurl={http://www.latex-project.org/lppl.txt}}
\hypersetup{pdfcontactaddress={ETH Zurich, ITP, HIT K,
  Wolfgang-Pauli-Strasse 27}}
\hypersetup{pdfcontactpostcode={8093}}
\hypersetup{pdfcontactcity={Zurich}}
\hypersetup{pdfcontactcountry={Switzerland}}
\hypersetup{pdfcontactemail={nbeisert@itp.phys.ethz.ch}}
\hypersetup{pdfcontacturl={http://people.phys.ethz.ch/\xmptilde nbeisert/}}

\newcommand{\secref}[1]{\hyperref[#1]{section \ref*{#1}}}

\parskip1ex
\parindent0pt
\let\olditemize\itemize
\def\itemize{\olditemize\parskip0pt}

\begin{document}

\title{The \textsf{childdoc} Package}
\hypersetup{pdftitle={The childdoc Package}}
\author{Niklas Beisert\\[2ex]
  Institut f\"ur Theoretische Physik\\
  Eidgen\"ossische Technische Hochschule Z\"urich\\
  Wolfgang-Pauli-Strasse 27, 8093 Z\"urich, Switzerland\\[1ex]
  \href{mailto:nbeisert@itp.phys.ethz.ch}
  {\texttt{nbeisert@itp.phys.ethz.ch}}}
\hypersetup{pdfauthor={Niklas Beisert}}
\hypersetup{pdfsubject={Manual for the LaTeX2e Package childdoc}}
\date{30 December 2018, \textsf{v2.0}}
\maketitle

\begin{abstract}\noindent
\textsf{childdoc} is a \LaTeXe{} package
that enables the direct compilation
of document sections included by |\include|
to individual files.
\end{abstract}

\begingroup
\parskip0ex
\tableofcontents
\endgroup

%%%%%%%%%%%%%%%%%%%%%%%%%%%%%%%%%%%%%%%%%%%%%%%%%%%%%%%%%%%%%%%%%%%%%%%%%%%%%%%%
%%%%%%%%%%%%%%%%%%%%%%%%%%%%%%%%%%%%%%%%%%%%%%%%%%%%%%%%%%%%%%%%%%%%%%%%%%%%%%%%
\section{Introduction}

\LaTeX{} provides a mechanism to structure a large document (such as a book)
into a main file and several child files (containing the chapters)
using the |\include| command.
This mechanism is beneficial for documents
which span hundreds of pages in order to
make the source file(s) more manageable.
Moreover, compilation can be restricted to
selected child files by means of the |\includeonly| command.
The latter feature can be used to reduce the compilation time while editing
(this was significantly more useful in the earlier days of \LaTeX{})
or to generate a smaller document which is easier to navigate.
Another application of |\includeonly| is to generate
documents consisting of selected parts of the complete document.

However, there are a few drawbacks of the plain |\include| mechanism:
\begin{itemize}
\item
The child files cannot be compiled on their own,
they can only be compiled via the main file.
A naive editing environment
(such as a text editor with an option
to have the current file processed by \LaTeX)
may require one to switch to the main file before compiling;
attempting to compile the child file produces errors.
\item
The main file must be modified (each time)
to adjust the |\includeonly| command
to the present needs. This easily leaves the main file in a messy state.
\item
The generated document will always carry the filename
of the main document. This is inconvenient if
several child files are to be compiled and
to be kept for distribution.
\end{itemize}

The present package provides a simple interface
to make child files individually compilable by \LaTeX{}.
Compiling a child file then has the same effect as compiling
the main file with an |\includeonly| command
to select the appropriate child.
Moreover the generated document will carry the name of the child
rather than the main file.
This resolves all three above issues.

This feature is meant to make the editing of books,
thesis documents and lecture notes somewhat more convenient.
However, the package can also be used efficiently for
composing a series of documents (such as exercise sheets)
which are typically distributed individually.
It then assists the author in generating the individual documents
(potentially in different versions)
as well as a document containing the collected series.
Another application is in developing style files
or other kinds of included material
where compilation of the style file could redirect
to a sample or test file.

%%%%%%%%%%%%%%%%%%%%%%%%%%%%%%%%%%%%%%%%%%%%%%%%%%%%%%%%%%%%%%%%%%%%%%%%%%%%%%%%
%%%%%%%%%%%%%%%%%%%%%%%%%%%%%%%%%%%%%%%%%%%%%%%%%%%%%%%%%%%%%%%%%%%%%%%%%%%%%%%%
\section{Usage}

First of all, the package \textsf{childdoc} is \emph{not} a standard
\LaTeXe{} |.sty| style file! Therefore it needs to be invoked in
a non-standard way.

%%%%%%%%%%%%%%%%%%%%%%%%%%%%%%%%%%%%%%%%%%%%%%%%%%%%%%%%%%%%%%%%%%%%%%%%%%%%%%%%
\subsection{Included Files}
\label{sec:include}

%%%%%%%%%%%%%%%%%%%%%%%%%%%%%%%%%%%%%%%%
\DescribeMacro{\childdocmain}
To use the package, add the commands
\begin{center}
\begin{tabular}{l}
|\input{childdoc.def}|\\
|\childdocmain{}|\\
\end{tabular}
\end{center}
at the very top of the main \LaTeX{} file,
in particular \emph{before} the |\documentclass| statement!
The argument of |\childdocmain| should be left empty
(but it must be present).

%%%%%%%%%%%%%%%%%%%%%%%%%%%%%%%%%%%%%%%%
\DescribeMacro{\childdocof}
Furthermore, add the commands
\begin{center}
\begin{tabular}{l}
|\input{childdoc.def}|\\
|\childdocof{|\textit{main}|}|\\
\end{tabular}
\end{center}
at the top of every child file \textit{child}
which is included by |\include{|\textit{child}|}|
from within the main file
(or at least for those files to be compiled individually).
The argument \textit{main} must be the filename of the main file.

There are a couple of
considerations in setting up the main and child documents:

%%%%%%%%%%%%%%%%%%%%%%%%%%%%%%%%%%%%%%%%
\paragraph{Restrictions.}

Please note the following restrictions:
\begin{itemize}
\item
|\childdocmain| must be called with one argument \textit{main}
to ensure compatibility with earlier version of the package.
It must either be empty (|\childdocmain{}|)
or precisely match the filename of the main file in which it is specified.
See \secref{sec:detection} for further information.
\item
The filename \textit{main} must be specified without the |.tex| extension.
\item
The filename \textit{main} is case sensitive
(even in case-insensitive file systems)
due to internal string comparison.
\item
The argument \textit{main} should be fully expanded, it cannot be a macro.
\item
Subdirectories and special characters should be avoided in filenames.
\item
The command |\childdocmain{|\textit{main}|}| must be followed by a whitespace.
It should not be followed immediately by another command
or by a comment mark `|%|'.
This is because the \TeX{} parser reads the token immediately following
the argument of |\childdocmain| and puts it
at the beginning of every child section;
however, a white\-space is ignored.
\end{itemize}

%%%%%%%%%%%%%%%%%%%%%%%%%%%%%%%%%%%%%%%%
\paragraph{Content of Main File.}

It is advisable to place all content in the child files included by |\include|.
Any output contained in the main file will appear in all child documents
unless suppressed manually;
it cannot be suppressed automatically by the |\includeonly| directive
and thus should normally be avoided.
A method to include some content in the main file
by means of conditional processing is described in \secref{sec:conditional}.

%%%%%%%%%%%%%%%%%%%%%%%%%%%%%%%%%%%%%%%%
\paragraph{Page Numbering.}

When only a part of the document is compiled,
the appropriate numbering of pages
(as well as other status parameters)
is determined from the |.aux| files.
The latter contain information from previous passes.
However this information needs to propagate through
all intermediate child documents.
Therefore the page numbering in child documents may well
be inconsistent until the complete document is compiled at least once.

A useful (if unconventional) way to always ensure a consistent
page numbering is to restart the numbering in each child document
and denote the pages by `\textit{child}|.|\textit{page}'
where \textit{child} represents the chapter/section number of the child file.
This can be achieved by the command
|\numberwithin{page}{|\textit{child}|}|
of the \textsf{amsmath} package
where \textit{child} can be |chapter| or |section|
depending on the chosen structuring.
Alternatively, one can modify the macro |\thepage| appropriately
and reset the counter |page| at the start of each child file.

%%%%%%%%%%%%%%%%%%%%%%%%%%%%%%%%%%%%%%%%%%%%%%%%%%%%%%%%%%%%%%%%%%%%%%%%%%%%%%%%
\subsection{Conditional Processing}
\label{sec:conditional}

The package provides a mechanism to compile different versions
of a document. To customise the versions further some conditional processing
can come in handy to distinguish which version is being compiled.
The package provides two macros to describe the compilation context:

%%%%%%%%%%%%%%%%%%%%%%%%%%%%%%%%%%%%%%%%
\DescribeMacro{\ifchilddoc}
The conditional |\ifchilddoc| distinguishes between the compilation of
child documents and the main document:
%
\begin{center}
|\ifchilddoc |\textit{child-code}| |[|\||else |\textit{main-code}]| \||fi|
\end{center}

%%%%%%%%%%%%%%%%%%%%%%%%%%%%%%%%%%%%%%%%
\DescribeMacro{\childdocname}
\DescribeMacro{\childdocjob}
The macro |\childdocname| contains the filename (without extension)
of the main or child file being processed.
Note that |\childdocjob| will always contain the name of the main file.

%%%%%%%%%%%%%%%%%%%%%%%%%%%%%%%%%%%%%%%%
\paragraph{Title Page.}

Conditional processing can be used to include a title or banner page
in the main document when proper precautions are taken.
Importantly, the code in the main file should ensure that the page counter
(as well as other status parameters which are stored in the |.aux| files)
takes the same value after the conditional processing.
Otherwise the page numbers may take divergent values
depending on which part is compiled.

For example, a title page could be declared by:
%
\begin{center}
\begin{tabular}{l}
|\ifchilddoc\||else|\\
|\addtocounter{page}{-1}|\\
\textit{code for title page}\\
|\newpage|\\
|\||fi|
\end{tabular}
\end{center}
%
A banner page for the child documents can be generated by:
%
\begin{center}
\begin{tabular}{l}
|\ifchilddoc|\\
|\addtocounter{page}{-1}|\\
\textit{code for banner page}\\
|\newpage|\\
|\||fi|
\end{tabular}
\end{center}
%
Here one could write a message such as:
\begin{center}
|This is the part \childdocname{} of \childdocjob{}.|
\end{center}

%%%%%%%%%%%%%%%%%%%%%%%%%%%%%%%%%%%%%%%%%%%%%%%%%%%%%%%%%%%%%%%%%%%%%%%%%%%%%%%%
\subsection{Flags}
\label{sec:flags}

The package makes it easy to generate different versions
of the main or child documents.
To this end compilation flags can be defined
and assigned different default values.
They will be particularly useful in conjunction
with the forwarding mechanism described in \secref{sec:forward}.

For example, it may be useful to have a flag |\version|
which can be set to |draft| or |final|.
The document source will contain some conditional code
depending on the value of |\version|.
Suppose further, the flag should default to |final| for the main file
and to |draft| for child files
which is a natural assignment for editing the document.
This is achieved by placing the following code
in the preamble of the main document
(below the |\childdocmain| directive):
%
\begin{center}
\begin{tabular}{l}
|\ifchilddoc|\\
|\providecommand{\version}{draft}|\\
|\||else|\\
|\providecommand{\version}{final}|\\
|\||fi|
\end{tabular}
\end{center}
%
The definition by |\providecommand| makes sure
that previous definitions are not overwritten.
Further statements |\providecommand{\version}{...}|
can thus be added before the above code to override it.

For the main file, one might add a line
(between |\childdocmain| and the above block)
%
\begin{center}
|%\ifchilddoc\||else\providecommand{\version}{draft}\||fi|
\end{center}
%
which can be uncommented to produce a draft version.
Likewise one can add a line to the very top of a child file
(above the |\childdocof{|\textit{main}|}| directive)
%
\begin{center}
|%\providecommand{\version}{final}|
\end{center}
%
which can be uncommented to produce the final version of this child document.

%%%%%%%%%%%%%%%%%%%%%%%%%%%%%%%%%%%%%%%%%%%%%%%%%%%%%%%%%%%%%%%%%%%%%%%%%%%%%%%%
\subsection{Forwarding}
\label{sec:forward}

Different versions of the main or child documents
using compilation flags as described in \secref{sec:flags}
can be (permanently) stored in different files
for convenient compilation, viewing and distribution.
To this end, the package defines a command
to pass on compilation to a different file:

%%%%%%%%%%%%%%%%%%%%%%%%%%%%%%%%%%%%%%%%
\DescribeMacro{\childdocforward}
The command |\childdocforward| redirects processing to
another source file:
%
\begin{center}
\begin{tabular}{l}
|\input{childdoc.def}|\\
|\childdocforward[|\textit{main}|]{|\textit{dest}|}|\\
\end{tabular}
\end{center}
%
The argument \textit{dest} is the destination file
(without extension).
It should be the main file or one of the child files.
Note that further \textsf{childdoc} directives
such as |\childdocof| and |\childdocforward|
in the indicated file will be processed in this form.
The optional argument \textit{main}
passes on directly to the main file \textit{main}
while pretending to compile the child \textit{dest}.
This form behaves as if \textit{dest}
issues |\childdocof{|\textit{main}|}| right away,
and no further \textsf{childdoc} directives will be processed.

%%%%%%%%%%%%%%%%%%%%%%%%%%%%%%%%%%%%%%%%
\DescribeMacro{\...prefix}
In the alternative form |\childdocforwardprefix|,
%
\begin{center}
\begin{tabular}{l}
|\input{childdoc.def}|\\
|\childdocforwardprefix[|\textit{main}|]{|\textit{prefix}|}{|\textit{dest}|}|
\end{tabular}
\end{center}
%
the destination file is determined by a pattern
depending on the current file:
To make this work, the current file must be called
`{\textit{prefix}\hspace{0.2em}\textit{suffix}}'
with \textit{prefix} matching precisely the argument.
Processing is then passed on to the file
`{\textit{dest}\hspace{0.2em}\textit{suffix}}'.
Surely, the same effect is achieved by
directly specifying the
argument `{\textit{dest}\hspace{0.2em}\textit{suffix}}'
in the first form.
However, that requires to set up a different file
for each child. With the alternative form of the command
all these files can have exactly the same content
which simplifies setting them up and maintaining them.

For example, the following file |draft.tex|
with a compilation flag |\version| as described in \secref{sec:flags}
compiles the main document as a draft:
%
\begin{center}
\begin{tabular}{l}
|\def\version{draft}|\\
|\input{childdoc.def}|\\
|\childdocforward{|\textit{main}|}|
\end{tabular}
\end{center}
%
Likewise, the following files |final|\textit{nn}|.tex|
compile the final version of the child document
|child|\textit{nn}|.tex|:
%
\begin{center}
\begin{tabular}{l}
|\def\version{final}|\\
|\input{childdoc.def}|\\
|\childdocforwardprefix{final}{child}|
\end{tabular}
\end{center}
%

Note that when several versions of a main file and/or of each child file
are to be generated, it may be convenient to set up a |Makefile| or
shell script to automatise the process.

%%%%%%%%%%%%%%%%%%%%%%%%%%%%%%%%%%%%%%%%%%%%%%%%%%%%%%%%%%%%%%%%%%%%%%%%%%%%%%%%
\subsection{Command Line Processing}
\label{sec:commandline}

The effect of redirection files can also be achieved by invoking
the \LaTeX{} compiler with a more elaborate command line.
Most conveniently this should be done as part
of a shell script or a |Makefile|.

When using \textsf{childdoc} in the main file, the following
command lines effectively perform a redirection
(note that depending on the shell being used,
backslashes may have to be doubled: `|\|' $\to$ `|\\|'):
%
\begin{center}
|... -jobname "|\textit{target}|" |\\|"|[\textit{flags}]%
|\input{childdoc.def}\childdocforward[|\textit{main}|]{|\textit{dest}|}"|
\end{center}
%
Here \textit{target} is the name of the output file,
\textit{main} is the name of the main file
and \textit{dest} is the name of the main or child file to be processed
(all filenames without extensions).
The optional argument \textit{main} can be omitted
if \textit{main} matches \textit{dest}.
Optionally, compilation \textit{flags} can be defined via |\def| commands.
This command line makes the \TeX{} engine believe
it is compiling the file \textit{target}
whose content is specified as the latter parameter.
The provided code then forwards the processing to
\textit{main} or \textit{dest} as described in \secref{sec:forward}.

%%%%%%%%%%%%%%%%%%%%%%%%%%%%%%%%%%%%%%%%%%%%%%%%%%%%%%%%%%%%%%%%%%%%%%%%%%%%%%%%
\subsection{Include by Input}
\label{sec:input}

Including child documents by |\include| has some restrictions by design.
Most notably, the content of a child document always occupies
its own set of pages; pages cannot be shared between child documents.
Usually, this behaviour makes perfect sense
because each child document contain an essential part of the document.
However, in some situations it may be desirable to compose
a document from a collection of parts
without having mandatory page breaks between then.
For this case, the package
provides a mechanism to include parts
by |\input| which can also be processed individually.
However, by construction this mechanism
requires manual handling of the content to be output.

%%%%%%%%%%%%%%%%%%%%%%%%%%%%%%%%%%%%%%%%
\DescribeMacro{\ifchilddocmanual}
The main file should be prepared as usual, see \secref{sec:include}.
However, the document body must make a distinction
between processing of an individual part and of the main document, e.g.:
%
\begin{center}
\begin{tabular}{l}
|\ifchilddocmanual|\\
|\input{\childdocname}|\\
|\||else|\\
\textit{document body with }|\input{|\textit{part}|}|\\
|\||fi|
\end{tabular}
\end{center}
%
The conditional |\ifchilddocmanual| is true whenever
a part to be included by |\input| is being compiled,
and the name of the part is stored in |\childdocname|.

%%%%%%%%%%%%%%%%%%%%%%%%%%%%%%%%%%%%%%%%
\DescribeMacro{\childdocby}
Each part to be included by |\input| should start with:
%
\begin{center}
\begin{tabular}{l}
|\input{childdoc.def}|\\
|\childdocby{|\textit{main}|}|\\
\end{tabular}
\end{center}
%
The directive |\childdocby| is similar to |\childdocof|
described in \secref{sec:include},
but the subsequent selection of content must be done manually.
To that end, both |\ifchilddoc| and |\ifchilddocmanual|
will be true upon processing of a part,
and the name of the part is stored in |\childdocname|.
Note that |\jobname| will be set to the filename of the current part
so that each part receives an individual |.aux| file
that does not interfere with the |.aux| file(s) of the main document.
This behaviour can be altered by the alternative form
|\childdocby[*]{|\textit{main}|}| (with a non-empty optional argument)
which uses the |.aux| file of the main document
by setting |\jobname| to \textit{main}.

%%%%%%%%%%%%%%%%%%%%%%%%%%%%%%%%%%%%%%%%%%%%%%%%%%%%%%%%%%%%%%%%%%%%%%%%%%%%%%%%
\subsection{Driver Development}
\label{sec:driver}

The \textsf{childdoc} mechanism can also be use for the development
of definition files such as \LaTeX{} styles or classes.
This case differs from the above setup with multiple parts
included by |\include| in that no |\includeonly| should be invoked.
This can be achieved by starting the include file
(before |\ProvidesPackage|) with:
%
\begin{center}
\begin{tabular}{l}
|\input{childdoc.def}|\\
|\childdocforward{|\textit{main}|}|\\
\end{tabular}
\end{center}
%
or alternatively with:
%
\begin{center}
\begin{tabular}{l}
|\input{childdoc.def}|\\
|\childdocby{|\textit{main}|}|\\
\end{tabular}
\end{center}
%
Both forms have slightly different effects as described above.
The main file is prepared as usual, see \secref{sec:include}.

%%%%%%%%%%%%%%%%%%%%%%%%%%%%%%%%%%%%%%%%%%%%%%%%%%%%%%%%%%%%%%%%%%%%%%%%%%%%%%%%
\subsection{Legacy Detection}
\label{sec:detection}

The directive |\childdocmain| in the main file can detect
whether the complete document or merely a child is to be compiled
even without using the directive |\childdocof|.
This method is deprecated because it is less robust
and there is no compelling reason to use it;
it is merely provided for backward compatibility
and it may be removed in future versions.

If the detection mechanism is to be used,
it is mandatory to correctly specify
the filename of the main file as the argument of |\childdocmain|:
%
\begin{center}
\begin{tabular}{l}
|\input{childdoc.def}|\\
|\childdocmain{|\textit{main}|}|\\
\end{tabular}
\end{center}
%
If |\jobname| does not match the argument \textit{main} of |\childdocmain|,
it is assumed that |\jobname| points to the child file to be compiled.
When using |\childdocmain| with the main file specified as argument,
it suffices to start a child file
with just |\input{|\textit{main}|}|
without loading of the package and using |\childdocof|.
If instead all processing is done
with the appropriate \textsf{childdoc} directives,
the argument of \textit{main} of |\childdocmain| can be empty.

An alternative version of the command line processing described
in \secref{sec:commandline} using the detection mechanism reads:
%
\begin{center}
|... -jobname "|\textit{target}|" "|[\textit{flags}]%
[|\def\jobname{|\textit{dest}|}|]|\input{|\textit{main}|}"|
\end{center}

%%%%%%%%%%%%%%%%%%%%%%%%%%%%%%%%%%%%%%%%%%%%%%%%%%%%%%%%%%%%%%%%%%%%%%%%%%%%%%%%
\subsection{Manual Code}
\label{sec:manual}

In case one cannot be certain whether the definitions file |childdoc.def|
is installed on the target \TeX{} distribution
and one prefers not to ship it,
it is conceivable to paste a few relevant commands into the sources.

To that end, drop all statements |\input{childdoc.def}|
and perform the replacements as outlined below.
Instead of |\childdocmain{|\textit{main}|}| add the following code
to the top of the main file:
%
\begin{center}
\begin{tabular}{l}
|\||ifdefined\childdocname\endinput\||fi\newif\ifchilddoc|\\
|\edef\childdocname{\scantokens\expandafter{\jobname\noexpand}}|\\
|\def\childdocmain{|\textit{main}|}\||ifx\childdocmain\childdocname\||else|\\
|\childdoctrue\includeonly{\childdocname}\let\jobname\childdocmain\||fi|\\
\end{tabular}
\end{center}
%
Instead of |\childdocof{|\textit{main}|}| just include the main file
at the top of each child file:
%
\begin{center}
|\input{|\textit{main}|}|
\end{center}
%
A simple redirection |\childdocforward{|\textit{dest}|}| is achieved by:
%
\begin{center}
|\def\jobname{|\textit{dest}|}\input{\jobname}|
\end{center}
%
The redirection with prefix
|\childdocforwardprefix[|\textit{prefix}|]{|\textit{dest}|}|
is accomplished by:
%
\begin{center}
\begin{tabular}{l}
|{\edef\jobname{\scantokens\expandafter{\jobname\noexpand}}|\\
|\def\redirectjob |\textit{prefix}|#1~~~{\gdef\jobname{|\textit{dest}|#1}}|\\
|\expandafter\redirectjob\jobname~~~}\input{\jobname}|
\end{tabular}
\end{center}

In an alternative approach,
child documents can be compiled by a specific command line
without additional code or specific definitions:
%
\begin{center}
|... -jobname "|\textit{target}|" "|[\textit{flags}]%
|\includeonly{|\textit{dest}|}\input{|\textit{main}|}"|
\end{center}
%

%%%%%%%%%%%%%%%%%%%%%%%%%%%%%%%%%%%%%%%%%%%%%%%%%%%%%%%%%%%%%%%%%%%%%%%%%%%%%%%%
%%%%%%%%%%%%%%%%%%%%%%%%%%%%%%%%%%%%%%%%%%%%%%%%%%%%%%%%%%%%%%%%%%%%%%%%%%%%%%%%
\section{Information}

%%%%%%%%%%%%%%%%%%%%%%%%%%%%%%%%%%%%%%%%%%%%%%%%%%%%%%%%%%%%%%%%%%%%%%%%%%%%%%%%
\subsection{Copyright}

Copyright \copyright{} 2017--2018 Niklas Beisert

This work may be distributed and/or modified under the
conditions of the \LaTeX{} Project Public License, either version 1.3
of this license or (at your option) any later version.
The latest version of this license is in
  \url{http://www.latex-project.org/lppl.txt}
and version 1.3 or later is part of all distributions of \LaTeX{}
version 2005/12/01 or later.

This work has the LPPL maintenance status `maintained'.

The Current Maintainer of this work is Niklas Beisert.

This work consists of the files |README.txt|, |childdoc.ins| and |childdoc.dtx|
as well as the derived files |childdoc.def|, |cdocsamp.tex|
with |cdocsch1.tex|, |cdocsch2.tex|, |cdocspt3.tex|, |cdocspt4.tex|,
|cdocsdrf.tex|, |cdocsfn1.tex|, |cdocsfn2.tex|
as well as |childdoc.pdf|.

%%%%%%%%%%%%%%%%%%%%%%%%%%%%%%%%%%%%%%%%%%%%%%%%%%%%%%%%%%%%%%%%%%%%%%%%%%%%%%%%
\subsection{Files and Installation}

The package consists of the files:
%
\begin{center}
\begin{tabular}{ll}
    |README.txt|   & readme file \\
    |childdoc.ins| & installation file \\
    |childdoc.dtx| & source file \\
    |childdoc.def| & definition file \\
    |cdocsamp.tex| & sample main file \\
    |cdocsch1.tex| & sample include file \\
    |cdocsch2.tex| & sample include file \\
    |cdocspt3.tex| & sample part file \\
    |cdocspt4.tex| & sample part file \\
    |cdocsdrf.tex| & sample redirection file \\
    |cdocsfn1.tex| & sample redirection file \\
    |cdocsfn2.tex| & sample redirection file \\
    |childdoc.pdf| & manual
\end{tabular}
\end{center}
%
The distribution consists of the files
|README.txt|, |childdoc.ins| and |childdoc.dtx|.
%
\begin{itemize}
\item
Run (pdf)\LaTeX{} on |childdoc.dtx|
to compile the manual |childdoc.pdf| (this file).
\item
Run \LaTeX{} on |childdoc.ins| to create the definitions file |childdoc.def|
and the sample |cdocsamp.tex| with include files
|cdocsch1.tex|, |cdocsch2.tex|, |cdocspt3.tex|, |cdocspt4.tex|,
|cdocsdrf.tex|, |cdocsfn1.tex|, |cdocsfn2.tex|.
Then copy the file |childdoc.def| to an appropriate directory of your \LaTeX{}
distribution, e.g.\ \textit{texmf-root}|/tex/latex/childdoc|.
\end{itemize}

%%%%%%%%%%%%%%%%%%%%%%%%%%%%%%%%%%%%%%%%%%%%%%%%%%%%%%%%%%%%%%%%%%%%%%%%%%%%%%%%
\subsection{Related CTAN Packages}

There are several other packages which offer a similar functionality:
%
\begin{itemize}
\item
The packages
\href{http://ctan.org/pkg/docmute}{\textsf{docmute}},
\href{http://ctan.org/pkg/includex}{\textsf{includex}} and
\href{http://ctan.org/pkg/standalone}{\textsf{standalone}}
provide commands to include only the document body of
a child file thus allowing both files to be compiled individually.
\item
The packages \href{http://ctan.org/pkg/subdocs}{\textsf{subdocs}}
and \href{http://ctan.org/pkg/subfiles}{\textsf{subfiles}}
provide structures in which the main and child documents can be
encapsulated and allowing them to be compiled individually.
The inclusion mechanism is different from the conventional |\include|.
\item
The package \href{http://ctan.org/pkg/combine}{\textsf{combine}}
is an elaborate solution to combine several documents into one.
\end{itemize}
%
See also the CTAN topic \href{http://ctan.org/topic/subdocs}{\textsf{subdocs}}
for further related packages.
The present package differs from the above solutions in that
a document structure constructed with the conventional |\include| mechanism
just needs two extra commands at the top of every file
such that all constituent files can be compiled individually.

%%%%%%%%%%%%%%%%%%%%%%%%%%%%%%%%%%%%%%%%%%%%%%%%%%%%%%%%%%%%%%%%%%%%%%%%%%%%%%%%
%\subsection{Feature Suggestions}
%
%The following is a list of features which may be useful for future
%versions of this package:
%%
%\begin{itemize}
%\item
%\ldots
%\end{itemize}

%%%%%%%%%%%%%%%%%%%%%%%%%%%%%%%%%%%%%%%%%%%%%%%%%%%%%%%%%%%%%%%%%%%%%%%%%%%%%%%%
\subsection{Revision History}

%%%%%%%%%%%%%%%%%%%%%%%%%%%%%%%%%%%%%%%%
\paragraph{v2.0:} 2018/12/30

\begin{itemize}
\item
immediate forward processing
\item
added |\childdocby| mechanism
\item
manual restructured
\end{itemize}

%%%%%%%%%%%%%%%%%%%%%%%%%%%%%%%%%%%%%%%%
\paragraph{v1.6:} 2018/01/17

\begin{itemize}
\item
application for development of include files
\item
corrections to manual
\end{itemize}

%%%%%%%%%%%%%%%%%%%%%%%%%%%%%%%%%%%%%%%%
\paragraph{v1.5:} 2017/05/21

\begin{itemize}
\item
more complete structuring introduced
\item
|\childdocof| introduced
\item
|\childdoc| renamed to |\childdocmain|
\item
|\childredirect| renamed to |\childdocforward| and |\childdocforwardprefix|
and functionality expanded
\end{itemize}

%%%%%%%%%%%%%%%%%%%%%%%%%%%%%%%%%%%%%%%%
\paragraph{v1.0:} 2017/04/27

\begin{itemize}
\item
manual and install package
\item
first version published on CTAN
\end{itemize}

%%%%%%%%%%%%%%%%%%%%%%%%%%%%%%%%%%%%%%%%
\paragraph{v0.6:} 2017/04/26

\begin{itemize}
\item
redirection mechanism added
\end{itemize}

%%%%%%%%%%%%%%%%%%%%%%%%%%%%%%%%%%%%%%%%
\paragraph{v0.5:} 2017/04/26

\begin{itemize}
\item
functionality in definition file
\end{itemize}


%%%%%%%%%%%%%%%%%%%%%%%%%%%%%%%%%%%%%%%%%%%%%%%%%%%%%%%%%%%%%%%%%%%%%%%%%%%%%%%%
%%%%%%%%%%%%%%%%%%%%%%%%%%%%%%%%%%%%%%%%%%%%%%%%%%%%%%%%%%%%%%%%%%%%%%%%%%%%%%%%
%%%%%%%%%%%%%%%%%%%%%%%%%%%%%%%%%%%%%%%%%%%%%%%%%%%%%%%%%%%%%%%%%%%%%%%%%%%%%%%%
\appendix

\settowidth\MacroIndent{\rmfamily\scriptsize 000\ }

 \DocInput{childdoc.dtx}

\end{document}
%</driver>
% \fi
%
% %%%%%%%%%%%%%%%%%%%%%%%%%%%%%%%%%%%%%%%%%%%%%%%%%%%%%%%%%%%%%%%%%%%%%%%%%%%%%%
% %%%%%%%%%%%%%%%%%%%%%%%%%%%%%%%%%%%%%%%%%%%%%%%%%%%%%%%%%%%%%%%%%%%%%%%%%%%%%%
% \section{Sample}
%\iffalse
%<*samplemain>
%\fi
%
% The following presents a sample document
% with two chapters, two parts, a title page,
% a compile flag as well as three forwarding files to set the flag.
% It consists of eight |.tex| files:
% \begin{center}
% \begin{tabular}{ll}
% |cdocsamp.tex|&main file\\
% |cdocsch1.tex|&include file for chapter 1\\
% |cdocsch2.tex|&include file for chapter 2\\
% |cdocspt3.tex|&include file for part 3\\
% |cdocspt4.tex|&include file for part 4\\
% |cdocsdrf.tex|&forwarding file for main file in draft mode\\
% |cdocsfi1.tex|&forwarding file for final version of chapter 1\\
% |cdocsfi2.tex|&forwarding file for final version of chapter 2\\
% \end{tabular}
% \end{center}
% Each of the eight files can be compiled directly by the \LaTeX{} compiler.
%
% %%%%%%%%%%%%%%%%%%%%%%%%%%%%%%%%%%%%%%
% \paragraph{Main File.}
%
% The main file is called |cdocsamp.tex|.
%
% Load the \textsf{childdoc} definitions and
% declare the filename for the main document:
%    \begin{macrocode}
\input{childdoc.def}
\childdocmain{}
%    \end{macrocode}

% Optional override for |\version| flag:
%    \begin{macrocode}
%%\ifchilddoc\else\providecommand{\version}{draft}\fi
%    \end{macrocode}

% Define the default values for the |\version| flag
% (|final| for the main file and |draft| for childs):
%    \begin{macrocode}
\ifchilddoc
\providecommand{\version}{draft}
\else
\providecommand{\version}{final}
\fi
%    \end{macrocode}

% Load the standard document class:
%    \begin{macrocode}
\documentclass[12pt]{article}
%    \end{macrocode}

% Start the document body:
%    \begin{macrocode}
\begin{document}
%    \end{macrocode}

% Declare a title page.
% Print title, part of document being processed and version flag:
%    \begin{macrocode}
\addtocounter{page}{-1}
\begin{center}
{\LARGE\bfseries{}childdoc example\par}
\vspace{1cm}
\ifchilddoc
\ifchilddocmanual part\else chapter\fi:
`\childdocname' of `\childdocjob'\par
\else
main document: `\childdocjob'\par
\fi
version: \version\par
\end{center}
\newpage
%    \end{macrocode}

% Manually include selected file,
% otherwise process as usual:
%    \begin{macrocode}
\ifchilddocmanual
\section*{part `\childdocname'}
\input{\childdocname}
\else
%    \end{macrocode}

% Include the two chapters:
%    \begin{macrocode}
\include{cdocsch1}
\include{cdocsch2}
%    \end{macrocode}

% Include the two parts unless only chapters should be displayed:
%    \begin{macrocode}
\ifchilddoc\else
\section{part three}
\input{cdocspt3}
\section{part four}
\input{cdocspt4}
\fi
%    \end{macrocode}

% Process as usual until here:
%    \begin{macrocode}
\fi
%    \end{macrocode}

% End of document body:
%    \begin{macrocode}
\end{document}
%    \end{macrocode}
%\iffalse
%</samplemain>
%\fi
%
% %%%%%%%%%%%%%%%%%%%%%%%%%%%%%%%%%%%%%%
% \paragraph{Chapter Include Files.}
%
% The include files are called |cdocsch1.tex| and |cdocsch2.tex|.
%
%\iffalse
%<*samplechap1|samplechap2>
%\fi

% Optional override for |\version| flag:
%    \begin{macrocode}
%%\providecommand{\version}{final}
%    \end{macrocode}

% Include the main document:
%    \begin{macrocode}
\input{childdoc.def}
\childdocof{cdocsamp}
%    \end{macrocode}

%\iffalse
%</samplechap1|samplechap2>
%\fi
%
%\iffalse
%<*samplechap1>
%\fi
% Some text for chapter 1:
%    \begin{macrocode}
\section{one}
some text in chapter one
%    \end{macrocode}

%\iffalse
%</samplechap1>
%\fi
% Some text for chapter 2:
%\iffalse
%<*samplechap2>
%\fi
%    \begin{macrocode}
\section{two}
more text in chapter two
%    \end{macrocode}

%\iffalse
%</samplechap2>
%\fi
%
% %%%%%%%%%%%%%%%%%%%%%%%%%%%%%%%%%%%%%%
% \paragraph{Part Include Files.}
%
% The include files are called |cdocspt3.tex| and |cdocspt4.tex|.
%
%\iffalse
%<*samplepart3|samplepart4>
%\fi

% Optional override for |\version| flag:
%    \begin{macrocode}
%%\providecommand{\version}{final}
%    \end{macrocode}

% Include the main document:
%    \begin{macrocode}
\input{childdoc.def}
\childdocby{cdocsamp}
%    \end{macrocode}

%\iffalse
%</samplepart3|samplepart4>
%\fi
%
%\iffalse
%<*samplepart3>
%\fi
% Some text for part 3:
%    \begin{macrocode}
some text in part three
%    \end{macrocode}

%\iffalse
%</samplepart3>
%\fi
% Some text for part 4:
%\iffalse
%<*samplepart4>
%\fi
%    \begin{macrocode}
more text in part four
%    \end{macrocode}

%\iffalse
%</samplepart4>
%\fi
%
% %%%%%%%%%%%%%%%%%%%%%%%%%%%%%%%%%%%%%%
% \paragraph{Forwarding for a Complete Draft.}
%
% The following forwarding file |cdocsdrf.tex|
% compiles the main document in draft mode:
%\iffalse
%<*sampledraft>
%\fi
%    \begin{macrocode}
\def\version{draft}
\input{childdoc.def}
\childdocforward{cdocsamp}
%    \end{macrocode}

%\iffalse
%</sampledraft>
%\fi
%
% %%%%%%%%%%%%%%%%%%%%%%%%%%%%%%%%%%%%%%
% \paragraph{Forwarding for Final Version of the Chapters.}
%
% The following forwarding files |cdocsfn1.tex| and |cdocsfn2.tex|
% (with identical content)
% compile the final versions of the child documents
% |cdocsch1.tex| and |cdocsch2.tex|, respectively:
%\iffalse
%<*samplefinal>
%\fi
%    \begin{macrocode}
\def\version{final}
\input{childdoc.def}
\childdocforwardprefix[cdocsamp]{cdocsfn}{cdocsch}
%    \end{macrocode}

%\iffalse
%</samplefinal>
%\fi
%
% %%%%%%%%%%%%%%%%%%%%%%%%%%%%%%%%%%%%%%
% \paragraph{Command Line Processing.}
%
% The following three command lines generate the output files
% |cdocscld|, |cdocscl1| and |cdocscl2|
% which should be identical to
% |cdocsdrf|, |cdocsch1| and |cdocsfn2|, respectively:
% \begin{center}
% \begin{tabular}{l}
% |latex -jobname cdocscld \|\\
% |  "\def\version{draft}\input{childdoc.def}\childdocforward{cdocsamp}"|\\
% |latex -jobname cdocscl1 \|\\
% |  "\input{childdoc.def}\childdocforward[cdocsamp]{cdocsch1}"|\\
% |latex -jobname cdocscl2 \|\\
% |  "\def\version{final}\input{childdoc.def}\childdocforward{cdocsch2}"|
% \end{tabular}
% \end{center}
% Note that the trailing backslash on each first line
% merely continues the input to the second line
% (for convenient cut ant paste).
% Furthermore, the command |latex| can be replaced by any
% of its alternative versions such as |pdflatex|.
%
% %%%%%%%%%%%%%%%%%%%%%%%%%%%%%%%%%%%%%%%%%%%%%%%%%%%%%%%%%%%%%%%%%%%%%%%%%%%%%%
% %%%%%%%%%%%%%%%%%%%%%%%%%%%%%%%%%%%%%%%%%%%%%%%%%%%%%%%%%%%%%%%%%%%%%%%%%%%%%%
% \section{Implementation}
%\iffalse
%<*package>
%\fi
%
% This section describes the definitions file |childdoc.def|.

% The definitions cannot be loaded using |\usepackage| or |\RequirePackage|
% which has a mechanism to prevent loading a style file more than once.
% When loading the definitions by means of |\input|
% multiple instances have to be prevented manually:
%\iffalse
%This code needs to be before the `\ProvidesFile' directive
%which is defined at the beginning of this file.
%Therefore it is also placed there and commented out here.
%</package>
%<*discard>
%\fi
%    \begin{macrocode}
\ifdefined\childdocmain\endinput\fi
%    \end{macrocode}
%\iffalse
%</discard>
%<*package>
%\fi
%
% \macro{\ifchilddoc}
% \macro{\ifchilddocmanual}
% The conditional |\ifchilddoc| tells whether a
% child (true) or main (false) document is being compiled.
% The conditional |\ifchilddocmanual| tells whether
% the |\includeonly| mechanism is used (false) or
% the selection of child files must be performed manually (true).
% The definitions initialise to false:
%    \begin{macrocode}
\newif\ifchilddoc
\newif\ifchilddocmanual
%    \end{macrocode}

% \macro{\childdocname}
% \macro{\childdocjob}
% The macro |\childdocname| stores the name of the main document
% to be compiled. The macro |\childdocjob| stores the name of
% the document on which the \LaTeX{} compiler was originally invoked.
% The content of |\jobname| cannot be compared
% to filenames specified in the source due to different catcodes.
% The following code rescans |\jobname|, stores the result
% in |\childdocname| and saves a copy in |\childdocjob|:
%    \begin{macrocode}
\edef\childdocname{\scantokens\expandafter{\jobname\noexpand}}
\let\childdocjob\childdocname
%    \end{macrocode}

% \macro{\childdocdisable}
% The macro |\childdocdisable| prevents the main file
% from being processed more than once.
% At this stage, the main document command |\childdocmain|
% is assumed to be called once again where it should do nothing.
% Any subsequent call to it should prevent
% a secondary processing of the main document
% It overwrites the forwarding commands
% |\childdocof| and |\childdocforward|
% with empty macros to prevent further inclusions of the main document:
%    \begin{macrocode}
\newcommand{\childdocdisable}
{
  \renewcommand{\childdocmain}[1]{\renewcommand{\childdocmain}[1]{\endinput}}
  \renewcommand{\childdocof}[1]{}
  \renewcommand{\childdocby}[2][]{}
  \renewcommand{\childdocforward}[2][]{}
  \renewcommand{\childdocdisable}{}
}
%    \end{macrocode}

% \macro{\childdocmain}
% The macro |\childdocmain| is to be called at the top of the main file
% with nothing or the main filename (without extension) as argument.
% First, it breaks loops.
% If the argument is not empty and does not match |\childdocname|
% (which is set by the first inclusion of |childdoc.def|),
% |\ifchilddoc| is set to true, |\includeonly| is applied to the child file
% and |\jobname| is set to the main file
% (for proper handling of |.aux| files):
%    \begin{macrocode}
\newcommand{\childdocmain}[1]
{
  \childdocdisable\childdocmain{}
  \if?#1?\else
    \begingroup
      \def\childdoctmp{#1}
      \ifx\childdoctmp\childdocname
        \def\childdoctmp{}
      \else
        \def\childdoctmp
        {
          \childdoctrue
          \includeonly{\childdocname}
          \def\childdocjob{#1}
          \def\jobname{#1}
        }
      \fi
      \expandafter
    \endgroup
    \childdoctmp
  \fi
}
%    \end{macrocode}

% \macro{\childdocof}
% The command |\childdocof| redirects
% compilation to the main file |#1|.
%    \begin{macrocode}
\newcommand{\childdocof}[1]
{
  \childdocdisable
  \childdoctrue
  \includeonly{\childdocname}
  \def\jobname{#1}
  \def\childdocjob{#1}
  \input{#1}
}
%    \end{macrocode}

% \macro{\childdocby}
% The command |\childdocby| ....
%    \begin{macrocode}
\newcommand{\childdocby}[2][]
{
  \childdocdisable
  \childdoctrue
  \childdocmanualtrue
  \if?#1?\else
    \def\jobname{#2}
  \fi
  \def\childdocjob{#2}
  \input{#2}
  \endinput
}
%    \end{macrocode}

% \macro{\childdocforward}
% The command |\childdocforward| redirects
% compilation to the main file or
% (if the optional argument is given) a child file.
% Parameters are set as if the main file
% or a child file starting with |\childdocof| was compiled.
% Then compilation is handed over to the main file:
%    \begin{macrocode}
\newcommand{\childdocforward}[2][]
{
  \begingroup
    \if?#1?
      \def\childdoctmp
      {
        \def\childdocname{#2}
        \def\childdocjob{#2}
        \def\jobname{#2}
        \input{#2}
        \endinput
      }
    \else
      \def\childdoctmp
      {
        \childdocdisable
        \def\childdocname{#2}
        \childdoctrue
        \includeonly{#2}
        \def\childdocjob{#1}
        \def\jobname{#1}
        \input{#1}
        \endinput
      }
    \fi
    \expandafter
  \endgroup
  \childdoctmp
}
%    \end{macrocode}

% \macro{\childdocforwardprefix}
% The command |\childdocforwardprefix| redirects
% compilation to the main or a child file by means of a pattern.
% The prefix |#1| in the current filename is replaced by |#2|
% and the suffix of the current filename is kept
% (it is assumed that the filename does not contain the substring `|~~~|'
% which is used as a delimiter).
% Compilation is handed over to the new file by |\childdocforward|:
%    \begin{macrocode}
\newcommand{\childdocforwardprefix}[3][]
{
  \begingroup
    \def\childdocextract #2##1~~~{\def\childdoctmp{\childdocforward[#1]{#3##1}}}
    \expandafter\childdocextract\childdocname~~~
    \expandafter
  \endgroup
  \childdoctmp
}
%    \end{macrocode}

% \macro{\childdoc}
% The deprecated macro |\childdoc| is a legacy version of |\childdocmain|:
%    \begin{macrocode}
\newcommand{\childdoc}{\childdocmain}
%    \end{macrocode}

% \macro{\childdocredirect}
% The deprecated macro |\childdocredirect| is a legacy version
% of |\childdocforward| and |\childdocforwardprefix|:
%    \begin{macrocode}
\newcommand{\childdocredirect}[2][]
{
  \begingroup
    \if?#1?
      \def\childdoctmp{\childdocforward{#2}}
    \else
      \def\childdoctmp{\childdocforwardprefix{#1}{#2}}
    \fi
    \expandafter
  \endgroup
  \childdoctmp
}
%    \end{macrocode}

%\iffalse
%</package>
%\fi
%
\endinput

\childdocof{cdocsamp}
%    \end{macrocode}

%\iffalse
%</samplechap1|samplechap2>
%\fi
%
%\iffalse
%<*samplechap1>
%\fi
% Some text for chapter 1:
%    \begin{macrocode}
\section{one}
some text in chapter one
%    \end{macrocode}

%\iffalse
%</samplechap1>
%\fi
% Some text for chapter 2:
%\iffalse
%<*samplechap2>
%\fi
%    \begin{macrocode}
\section{two}
more text in chapter two
%    \end{macrocode}

%\iffalse
%</samplechap2>
%\fi
%
% %%%%%%%%%%%%%%%%%%%%%%%%%%%%%%%%%%%%%%
% \paragraph{Part Include Files.}
%
% The include files are called |cdocspt3.tex| and |cdocspt4.tex|.
%
%\iffalse
%<*samplepart3|samplepart4>
%\fi

% Optional override for |\version| flag:
%    \begin{macrocode}
%%\providecommand{\version}{final}
%    \end{macrocode}

% Include the main document:
%    \begin{macrocode}
% \iffalse
%
% childdoc.dtx Copyright (C) 2017-2018 Niklas Beisert
%
% This work may be distributed and/or modified under the
% conditions of the LaTeX Project Public License, either version 1.3
% of this license or (at your option) any later version.
% The latest version of this license is in
%   http://www.latex-project.org/lppl.txt
% and version 1.3 or later is part of all distributions of LaTeX
% version 2005/12/01 or later.
%
% This work has the LPPL maintenance status `maintained'.
%
% The Current Maintainer of this work is Niklas Beisert.
%
% This work consists of the files childdoc.dtx and childdoc.ins
% and the derived files childdoc.def and cdocsamp.tex with
% cdocsch1.tex, cdocsch2.tex, cdocsdrf.tex, cdocsfn1.tex, cdocsfn2.tex.
%
%<package>\ifdefined\childdocmain\endinput\fi
%<package>\ProvidesFile{childdoc.def}[2018/12/30 v2.0 child document driver]
%<samplemain>\ProvidesFile{cdocsamp.tex}[2018/12/30 v2.0 sample for childdoc]
%<*driver>
%\ProvidesFile{childdoc.drv}[2018/12/30 v2.0 childdoc reference manual file]
\PassOptionsToClass{10pt,a4paper}{article}
\documentclass{ltxdoc}

\usepackage[margin=35mm]{geometry}
\usepackage{hyperref}
\usepackage{hyperxmp}
\usepackage[usenames]{color}

\hypersetup{colorlinks=true}
\hypersetup{pdfstartview=FitH}
\hypersetup{pdfpagemode=UseNone}
\hypersetup{pdfsource={}}
\hypersetup{pdflang={en-UK}}
\hypersetup{pdfcopyright={Copyright 2017-2018 Niklas Beisert.
  This work may be distributed and/or modified under the
  conditions of the LaTeX Project Public License, either version 1.3
  of this license or (at your option) any later version.}}
\hypersetup{pdflicenseurl={http://www.latex-project.org/lppl.txt}}
\hypersetup{pdfcontactaddress={ETH Zurich, ITP, HIT K,
  Wolfgang-Pauli-Strasse 27}}
\hypersetup{pdfcontactpostcode={8093}}
\hypersetup{pdfcontactcity={Zurich}}
\hypersetup{pdfcontactcountry={Switzerland}}
\hypersetup{pdfcontactemail={nbeisert@itp.phys.ethz.ch}}
\hypersetup{pdfcontacturl={http://people.phys.ethz.ch/\xmptilde nbeisert/}}

\newcommand{\secref}[1]{\hyperref[#1]{section \ref*{#1}}}

\parskip1ex
\parindent0pt
\let\olditemize\itemize
\def\itemize{\olditemize\parskip0pt}

\begin{document}

\title{The \textsf{childdoc} Package}
\hypersetup{pdftitle={The childdoc Package}}
\author{Niklas Beisert\\[2ex]
  Institut f\"ur Theoretische Physik\\
  Eidgen\"ossische Technische Hochschule Z\"urich\\
  Wolfgang-Pauli-Strasse 27, 8093 Z\"urich, Switzerland\\[1ex]
  \href{mailto:nbeisert@itp.phys.ethz.ch}
  {\texttt{nbeisert@itp.phys.ethz.ch}}}
\hypersetup{pdfauthor={Niklas Beisert}}
\hypersetup{pdfsubject={Manual for the LaTeX2e Package childdoc}}
\date{30 December 2018, \textsf{v2.0}}
\maketitle

\begin{abstract}\noindent
\textsf{childdoc} is a \LaTeXe{} package
that enables the direct compilation
of document sections included by |\include|
to individual files.
\end{abstract}

\begingroup
\parskip0ex
\tableofcontents
\endgroup

%%%%%%%%%%%%%%%%%%%%%%%%%%%%%%%%%%%%%%%%%%%%%%%%%%%%%%%%%%%%%%%%%%%%%%%%%%%%%%%%
%%%%%%%%%%%%%%%%%%%%%%%%%%%%%%%%%%%%%%%%%%%%%%%%%%%%%%%%%%%%%%%%%%%%%%%%%%%%%%%%
\section{Introduction}

\LaTeX{} provides a mechanism to structure a large document (such as a book)
into a main file and several child files (containing the chapters)
using the |\include| command.
This mechanism is beneficial for documents
which span hundreds of pages in order to
make the source file(s) more manageable.
Moreover, compilation can be restricted to
selected child files by means of the |\includeonly| command.
The latter feature can be used to reduce the compilation time while editing
(this was significantly more useful in the earlier days of \LaTeX{})
or to generate a smaller document which is easier to navigate.
Another application of |\includeonly| is to generate
documents consisting of selected parts of the complete document.

However, there are a few drawbacks of the plain |\include| mechanism:
\begin{itemize}
\item
The child files cannot be compiled on their own,
they can only be compiled via the main file.
A naive editing environment
(such as a text editor with an option
to have the current file processed by \LaTeX)
may require one to switch to the main file before compiling;
attempting to compile the child file produces errors.
\item
The main file must be modified (each time)
to adjust the |\includeonly| command
to the present needs. This easily leaves the main file in a messy state.
\item
The generated document will always carry the filename
of the main document. This is inconvenient if
several child files are to be compiled and
to be kept for distribution.
\end{itemize}

The present package provides a simple interface
to make child files individually compilable by \LaTeX{}.
Compiling a child file then has the same effect as compiling
the main file with an |\includeonly| command
to select the appropriate child.
Moreover the generated document will carry the name of the child
rather than the main file.
This resolves all three above issues.

This feature is meant to make the editing of books,
thesis documents and lecture notes somewhat more convenient.
However, the package can also be used efficiently for
composing a series of documents (such as exercise sheets)
which are typically distributed individually.
It then assists the author in generating the individual documents
(potentially in different versions)
as well as a document containing the collected series.
Another application is in developing style files
or other kinds of included material
where compilation of the style file could redirect
to a sample or test file.

%%%%%%%%%%%%%%%%%%%%%%%%%%%%%%%%%%%%%%%%%%%%%%%%%%%%%%%%%%%%%%%%%%%%%%%%%%%%%%%%
%%%%%%%%%%%%%%%%%%%%%%%%%%%%%%%%%%%%%%%%%%%%%%%%%%%%%%%%%%%%%%%%%%%%%%%%%%%%%%%%
\section{Usage}

First of all, the package \textsf{childdoc} is \emph{not} a standard
\LaTeXe{} |.sty| style file! Therefore it needs to be invoked in
a non-standard way.

%%%%%%%%%%%%%%%%%%%%%%%%%%%%%%%%%%%%%%%%%%%%%%%%%%%%%%%%%%%%%%%%%%%%%%%%%%%%%%%%
\subsection{Included Files}
\label{sec:include}

%%%%%%%%%%%%%%%%%%%%%%%%%%%%%%%%%%%%%%%%
\DescribeMacro{\childdocmain}
To use the package, add the commands
\begin{center}
\begin{tabular}{l}
|\input{childdoc.def}|\\
|\childdocmain{}|\\
\end{tabular}
\end{center}
at the very top of the main \LaTeX{} file,
in particular \emph{before} the |\documentclass| statement!
The argument of |\childdocmain| should be left empty
(but it must be present).

%%%%%%%%%%%%%%%%%%%%%%%%%%%%%%%%%%%%%%%%
\DescribeMacro{\childdocof}
Furthermore, add the commands
\begin{center}
\begin{tabular}{l}
|\input{childdoc.def}|\\
|\childdocof{|\textit{main}|}|\\
\end{tabular}
\end{center}
at the top of every child file \textit{child}
which is included by |\include{|\textit{child}|}|
from within the main file
(or at least for those files to be compiled individually).
The argument \textit{main} must be the filename of the main file.

There are a couple of
considerations in setting up the main and child documents:

%%%%%%%%%%%%%%%%%%%%%%%%%%%%%%%%%%%%%%%%
\paragraph{Restrictions.}

Please note the following restrictions:
\begin{itemize}
\item
|\childdocmain| must be called with one argument \textit{main}
to ensure compatibility with earlier version of the package.
It must either be empty (|\childdocmain{}|)
or precisely match the filename of the main file in which it is specified.
See \secref{sec:detection} for further information.
\item
The filename \textit{main} must be specified without the |.tex| extension.
\item
The filename \textit{main} is case sensitive
(even in case-insensitive file systems)
due to internal string comparison.
\item
The argument \textit{main} should be fully expanded, it cannot be a macro.
\item
Subdirectories and special characters should be avoided in filenames.
\item
The command |\childdocmain{|\textit{main}|}| must be followed by a whitespace.
It should not be followed immediately by another command
or by a comment mark `|%|'.
This is because the \TeX{} parser reads the token immediately following
the argument of |\childdocmain| and puts it
at the beginning of every child section;
however, a white\-space is ignored.
\end{itemize}

%%%%%%%%%%%%%%%%%%%%%%%%%%%%%%%%%%%%%%%%
\paragraph{Content of Main File.}

It is advisable to place all content in the child files included by |\include|.
Any output contained in the main file will appear in all child documents
unless suppressed manually;
it cannot be suppressed automatically by the |\includeonly| directive
and thus should normally be avoided.
A method to include some content in the main file
by means of conditional processing is described in \secref{sec:conditional}.

%%%%%%%%%%%%%%%%%%%%%%%%%%%%%%%%%%%%%%%%
\paragraph{Page Numbering.}

When only a part of the document is compiled,
the appropriate numbering of pages
(as well as other status parameters)
is determined from the |.aux| files.
The latter contain information from previous passes.
However this information needs to propagate through
all intermediate child documents.
Therefore the page numbering in child documents may well
be inconsistent until the complete document is compiled at least once.

A useful (if unconventional) way to always ensure a consistent
page numbering is to restart the numbering in each child document
and denote the pages by `\textit{child}|.|\textit{page}'
where \textit{child} represents the chapter/section number of the child file.
This can be achieved by the command
|\numberwithin{page}{|\textit{child}|}|
of the \textsf{amsmath} package
where \textit{child} can be |chapter| or |section|
depending on the chosen structuring.
Alternatively, one can modify the macro |\thepage| appropriately
and reset the counter |page| at the start of each child file.

%%%%%%%%%%%%%%%%%%%%%%%%%%%%%%%%%%%%%%%%%%%%%%%%%%%%%%%%%%%%%%%%%%%%%%%%%%%%%%%%
\subsection{Conditional Processing}
\label{sec:conditional}

The package provides a mechanism to compile different versions
of a document. To customise the versions further some conditional processing
can come in handy to distinguish which version is being compiled.
The package provides two macros to describe the compilation context:

%%%%%%%%%%%%%%%%%%%%%%%%%%%%%%%%%%%%%%%%
\DescribeMacro{\ifchilddoc}
The conditional |\ifchilddoc| distinguishes between the compilation of
child documents and the main document:
%
\begin{center}
|\ifchilddoc |\textit{child-code}| |[|\||else |\textit{main-code}]| \||fi|
\end{center}

%%%%%%%%%%%%%%%%%%%%%%%%%%%%%%%%%%%%%%%%
\DescribeMacro{\childdocname}
\DescribeMacro{\childdocjob}
The macro |\childdocname| contains the filename (without extension)
of the main or child file being processed.
Note that |\childdocjob| will always contain the name of the main file.

%%%%%%%%%%%%%%%%%%%%%%%%%%%%%%%%%%%%%%%%
\paragraph{Title Page.}

Conditional processing can be used to include a title or banner page
in the main document when proper precautions are taken.
Importantly, the code in the main file should ensure that the page counter
(as well as other status parameters which are stored in the |.aux| files)
takes the same value after the conditional processing.
Otherwise the page numbers may take divergent values
depending on which part is compiled.

For example, a title page could be declared by:
%
\begin{center}
\begin{tabular}{l}
|\ifchilddoc\||else|\\
|\addtocounter{page}{-1}|\\
\textit{code for title page}\\
|\newpage|\\
|\||fi|
\end{tabular}
\end{center}
%
A banner page for the child documents can be generated by:
%
\begin{center}
\begin{tabular}{l}
|\ifchilddoc|\\
|\addtocounter{page}{-1}|\\
\textit{code for banner page}\\
|\newpage|\\
|\||fi|
\end{tabular}
\end{center}
%
Here one could write a message such as:
\begin{center}
|This is the part \childdocname{} of \childdocjob{}.|
\end{center}

%%%%%%%%%%%%%%%%%%%%%%%%%%%%%%%%%%%%%%%%%%%%%%%%%%%%%%%%%%%%%%%%%%%%%%%%%%%%%%%%
\subsection{Flags}
\label{sec:flags}

The package makes it easy to generate different versions
of the main or child documents.
To this end compilation flags can be defined
and assigned different default values.
They will be particularly useful in conjunction
with the forwarding mechanism described in \secref{sec:forward}.

For example, it may be useful to have a flag |\version|
which can be set to |draft| or |final|.
The document source will contain some conditional code
depending on the value of |\version|.
Suppose further, the flag should default to |final| for the main file
and to |draft| for child files
which is a natural assignment for editing the document.
This is achieved by placing the following code
in the preamble of the main document
(below the |\childdocmain| directive):
%
\begin{center}
\begin{tabular}{l}
|\ifchilddoc|\\
|\providecommand{\version}{draft}|\\
|\||else|\\
|\providecommand{\version}{final}|\\
|\||fi|
\end{tabular}
\end{center}
%
The definition by |\providecommand| makes sure
that previous definitions are not overwritten.
Further statements |\providecommand{\version}{...}|
can thus be added before the above code to override it.

For the main file, one might add a line
(between |\childdocmain| and the above block)
%
\begin{center}
|%\ifchilddoc\||else\providecommand{\version}{draft}\||fi|
\end{center}
%
which can be uncommented to produce a draft version.
Likewise one can add a line to the very top of a child file
(above the |\childdocof{|\textit{main}|}| directive)
%
\begin{center}
|%\providecommand{\version}{final}|
\end{center}
%
which can be uncommented to produce the final version of this child document.

%%%%%%%%%%%%%%%%%%%%%%%%%%%%%%%%%%%%%%%%%%%%%%%%%%%%%%%%%%%%%%%%%%%%%%%%%%%%%%%%
\subsection{Forwarding}
\label{sec:forward}

Different versions of the main or child documents
using compilation flags as described in \secref{sec:flags}
can be (permanently) stored in different files
for convenient compilation, viewing and distribution.
To this end, the package defines a command
to pass on compilation to a different file:

%%%%%%%%%%%%%%%%%%%%%%%%%%%%%%%%%%%%%%%%
\DescribeMacro{\childdocforward}
The command |\childdocforward| redirects processing to
another source file:
%
\begin{center}
\begin{tabular}{l}
|\input{childdoc.def}|\\
|\childdocforward[|\textit{main}|]{|\textit{dest}|}|\\
\end{tabular}
\end{center}
%
The argument \textit{dest} is the destination file
(without extension).
It should be the main file or one of the child files.
Note that further \textsf{childdoc} directives
such as |\childdocof| and |\childdocforward|
in the indicated file will be processed in this form.
The optional argument \textit{main}
passes on directly to the main file \textit{main}
while pretending to compile the child \textit{dest}.
This form behaves as if \textit{dest}
issues |\childdocof{|\textit{main}|}| right away,
and no further \textsf{childdoc} directives will be processed.

%%%%%%%%%%%%%%%%%%%%%%%%%%%%%%%%%%%%%%%%
\DescribeMacro{\...prefix}
In the alternative form |\childdocforwardprefix|,
%
\begin{center}
\begin{tabular}{l}
|\input{childdoc.def}|\\
|\childdocforwardprefix[|\textit{main}|]{|\textit{prefix}|}{|\textit{dest}|}|
\end{tabular}
\end{center}
%
the destination file is determined by a pattern
depending on the current file:
To make this work, the current file must be called
`{\textit{prefix}\hspace{0.2em}\textit{suffix}}'
with \textit{prefix} matching precisely the argument.
Processing is then passed on to the file
`{\textit{dest}\hspace{0.2em}\textit{suffix}}'.
Surely, the same effect is achieved by
directly specifying the
argument `{\textit{dest}\hspace{0.2em}\textit{suffix}}'
in the first form.
However, that requires to set up a different file
for each child. With the alternative form of the command
all these files can have exactly the same content
which simplifies setting them up and maintaining them.

For example, the following file |draft.tex|
with a compilation flag |\version| as described in \secref{sec:flags}
compiles the main document as a draft:
%
\begin{center}
\begin{tabular}{l}
|\def\version{draft}|\\
|\input{childdoc.def}|\\
|\childdocforward{|\textit{main}|}|
\end{tabular}
\end{center}
%
Likewise, the following files |final|\textit{nn}|.tex|
compile the final version of the child document
|child|\textit{nn}|.tex|:
%
\begin{center}
\begin{tabular}{l}
|\def\version{final}|\\
|\input{childdoc.def}|\\
|\childdocforwardprefix{final}{child}|
\end{tabular}
\end{center}
%

Note that when several versions of a main file and/or of each child file
are to be generated, it may be convenient to set up a |Makefile| or
shell script to automatise the process.

%%%%%%%%%%%%%%%%%%%%%%%%%%%%%%%%%%%%%%%%%%%%%%%%%%%%%%%%%%%%%%%%%%%%%%%%%%%%%%%%
\subsection{Command Line Processing}
\label{sec:commandline}

The effect of redirection files can also be achieved by invoking
the \LaTeX{} compiler with a more elaborate command line.
Most conveniently this should be done as part
of a shell script or a |Makefile|.

When using \textsf{childdoc} in the main file, the following
command lines effectively perform a redirection
(note that depending on the shell being used,
backslashes may have to be doubled: `|\|' $\to$ `|\\|'):
%
\begin{center}
|... -jobname "|\textit{target}|" |\\|"|[\textit{flags}]%
|\input{childdoc.def}\childdocforward[|\textit{main}|]{|\textit{dest}|}"|
\end{center}
%
Here \textit{target} is the name of the output file,
\textit{main} is the name of the main file
and \textit{dest} is the name of the main or child file to be processed
(all filenames without extensions).
The optional argument \textit{main} can be omitted
if \textit{main} matches \textit{dest}.
Optionally, compilation \textit{flags} can be defined via |\def| commands.
This command line makes the \TeX{} engine believe
it is compiling the file \textit{target}
whose content is specified as the latter parameter.
The provided code then forwards the processing to
\textit{main} or \textit{dest} as described in \secref{sec:forward}.

%%%%%%%%%%%%%%%%%%%%%%%%%%%%%%%%%%%%%%%%%%%%%%%%%%%%%%%%%%%%%%%%%%%%%%%%%%%%%%%%
\subsection{Include by Input}
\label{sec:input}

Including child documents by |\include| has some restrictions by design.
Most notably, the content of a child document always occupies
its own set of pages; pages cannot be shared between child documents.
Usually, this behaviour makes perfect sense
because each child document contain an essential part of the document.
However, in some situations it may be desirable to compose
a document from a collection of parts
without having mandatory page breaks between then.
For this case, the package
provides a mechanism to include parts
by |\input| which can also be processed individually.
However, by construction this mechanism
requires manual handling of the content to be output.

%%%%%%%%%%%%%%%%%%%%%%%%%%%%%%%%%%%%%%%%
\DescribeMacro{\ifchilddocmanual}
The main file should be prepared as usual, see \secref{sec:include}.
However, the document body must make a distinction
between processing of an individual part and of the main document, e.g.:
%
\begin{center}
\begin{tabular}{l}
|\ifchilddocmanual|\\
|\input{\childdocname}|\\
|\||else|\\
\textit{document body with }|\input{|\textit{part}|}|\\
|\||fi|
\end{tabular}
\end{center}
%
The conditional |\ifchilddocmanual| is true whenever
a part to be included by |\input| is being compiled,
and the name of the part is stored in |\childdocname|.

%%%%%%%%%%%%%%%%%%%%%%%%%%%%%%%%%%%%%%%%
\DescribeMacro{\childdocby}
Each part to be included by |\input| should start with:
%
\begin{center}
\begin{tabular}{l}
|\input{childdoc.def}|\\
|\childdocby{|\textit{main}|}|\\
\end{tabular}
\end{center}
%
The directive |\childdocby| is similar to |\childdocof|
described in \secref{sec:include},
but the subsequent selection of content must be done manually.
To that end, both |\ifchilddoc| and |\ifchilddocmanual|
will be true upon processing of a part,
and the name of the part is stored in |\childdocname|.
Note that |\jobname| will be set to the filename of the current part
so that each part receives an individual |.aux| file
that does not interfere with the |.aux| file(s) of the main document.
This behaviour can be altered by the alternative form
|\childdocby[*]{|\textit{main}|}| (with a non-empty optional argument)
which uses the |.aux| file of the main document
by setting |\jobname| to \textit{main}.

%%%%%%%%%%%%%%%%%%%%%%%%%%%%%%%%%%%%%%%%%%%%%%%%%%%%%%%%%%%%%%%%%%%%%%%%%%%%%%%%
\subsection{Driver Development}
\label{sec:driver}

The \textsf{childdoc} mechanism can also be use for the development
of definition files such as \LaTeX{} styles or classes.
This case differs from the above setup with multiple parts
included by |\include| in that no |\includeonly| should be invoked.
This can be achieved by starting the include file
(before |\ProvidesPackage|) with:
%
\begin{center}
\begin{tabular}{l}
|\input{childdoc.def}|\\
|\childdocforward{|\textit{main}|}|\\
\end{tabular}
\end{center}
%
or alternatively with:
%
\begin{center}
\begin{tabular}{l}
|\input{childdoc.def}|\\
|\childdocby{|\textit{main}|}|\\
\end{tabular}
\end{center}
%
Both forms have slightly different effects as described above.
The main file is prepared as usual, see \secref{sec:include}.

%%%%%%%%%%%%%%%%%%%%%%%%%%%%%%%%%%%%%%%%%%%%%%%%%%%%%%%%%%%%%%%%%%%%%%%%%%%%%%%%
\subsection{Legacy Detection}
\label{sec:detection}

The directive |\childdocmain| in the main file can detect
whether the complete document or merely a child is to be compiled
even without using the directive |\childdocof|.
This method is deprecated because it is less robust
and there is no compelling reason to use it;
it is merely provided for backward compatibility
and it may be removed in future versions.

If the detection mechanism is to be used,
it is mandatory to correctly specify
the filename of the main file as the argument of |\childdocmain|:
%
\begin{center}
\begin{tabular}{l}
|\input{childdoc.def}|\\
|\childdocmain{|\textit{main}|}|\\
\end{tabular}
\end{center}
%
If |\jobname| does not match the argument \textit{main} of |\childdocmain|,
it is assumed that |\jobname| points to the child file to be compiled.
When using |\childdocmain| with the main file specified as argument,
it suffices to start a child file
with just |\input{|\textit{main}|}|
without loading of the package and using |\childdocof|.
If instead all processing is done
with the appropriate \textsf{childdoc} directives,
the argument of \textit{main} of |\childdocmain| can be empty.

An alternative version of the command line processing described
in \secref{sec:commandline} using the detection mechanism reads:
%
\begin{center}
|... -jobname "|\textit{target}|" "|[\textit{flags}]%
[|\def\jobname{|\textit{dest}|}|]|\input{|\textit{main}|}"|
\end{center}

%%%%%%%%%%%%%%%%%%%%%%%%%%%%%%%%%%%%%%%%%%%%%%%%%%%%%%%%%%%%%%%%%%%%%%%%%%%%%%%%
\subsection{Manual Code}
\label{sec:manual}

In case one cannot be certain whether the definitions file |childdoc.def|
is installed on the target \TeX{} distribution
and one prefers not to ship it,
it is conceivable to paste a few relevant commands into the sources.

To that end, drop all statements |\input{childdoc.def}|
and perform the replacements as outlined below.
Instead of |\childdocmain{|\textit{main}|}| add the following code
to the top of the main file:
%
\begin{center}
\begin{tabular}{l}
|\||ifdefined\childdocname\endinput\||fi\newif\ifchilddoc|\\
|\edef\childdocname{\scantokens\expandafter{\jobname\noexpand}}|\\
|\def\childdocmain{|\textit{main}|}\||ifx\childdocmain\childdocname\||else|\\
|\childdoctrue\includeonly{\childdocname}\let\jobname\childdocmain\||fi|\\
\end{tabular}
\end{center}
%
Instead of |\childdocof{|\textit{main}|}| just include the main file
at the top of each child file:
%
\begin{center}
|\input{|\textit{main}|}|
\end{center}
%
A simple redirection |\childdocforward{|\textit{dest}|}| is achieved by:
%
\begin{center}
|\def\jobname{|\textit{dest}|}\input{\jobname}|
\end{center}
%
The redirection with prefix
|\childdocforwardprefix[|\textit{prefix}|]{|\textit{dest}|}|
is accomplished by:
%
\begin{center}
\begin{tabular}{l}
|{\edef\jobname{\scantokens\expandafter{\jobname\noexpand}}|\\
|\def\redirectjob |\textit{prefix}|#1~~~{\gdef\jobname{|\textit{dest}|#1}}|\\
|\expandafter\redirectjob\jobname~~~}\input{\jobname}|
\end{tabular}
\end{center}

In an alternative approach,
child documents can be compiled by a specific command line
without additional code or specific definitions:
%
\begin{center}
|... -jobname "|\textit{target}|" "|[\textit{flags}]%
|\includeonly{|\textit{dest}|}\input{|\textit{main}|}"|
\end{center}
%

%%%%%%%%%%%%%%%%%%%%%%%%%%%%%%%%%%%%%%%%%%%%%%%%%%%%%%%%%%%%%%%%%%%%%%%%%%%%%%%%
%%%%%%%%%%%%%%%%%%%%%%%%%%%%%%%%%%%%%%%%%%%%%%%%%%%%%%%%%%%%%%%%%%%%%%%%%%%%%%%%
\section{Information}

%%%%%%%%%%%%%%%%%%%%%%%%%%%%%%%%%%%%%%%%%%%%%%%%%%%%%%%%%%%%%%%%%%%%%%%%%%%%%%%%
\subsection{Copyright}

Copyright \copyright{} 2017--2018 Niklas Beisert

This work may be distributed and/or modified under the
conditions of the \LaTeX{} Project Public License, either version 1.3
of this license or (at your option) any later version.
The latest version of this license is in
  \url{http://www.latex-project.org/lppl.txt}
and version 1.3 or later is part of all distributions of \LaTeX{}
version 2005/12/01 or later.

This work has the LPPL maintenance status `maintained'.

The Current Maintainer of this work is Niklas Beisert.

This work consists of the files |README.txt|, |childdoc.ins| and |childdoc.dtx|
as well as the derived files |childdoc.def|, |cdocsamp.tex|
with |cdocsch1.tex|, |cdocsch2.tex|, |cdocspt3.tex|, |cdocspt4.tex|,
|cdocsdrf.tex|, |cdocsfn1.tex|, |cdocsfn2.tex|
as well as |childdoc.pdf|.

%%%%%%%%%%%%%%%%%%%%%%%%%%%%%%%%%%%%%%%%%%%%%%%%%%%%%%%%%%%%%%%%%%%%%%%%%%%%%%%%
\subsection{Files and Installation}

The package consists of the files:
%
\begin{center}
\begin{tabular}{ll}
    |README.txt|   & readme file \\
    |childdoc.ins| & installation file \\
    |childdoc.dtx| & source file \\
    |childdoc.def| & definition file \\
    |cdocsamp.tex| & sample main file \\
    |cdocsch1.tex| & sample include file \\
    |cdocsch2.tex| & sample include file \\
    |cdocspt3.tex| & sample part file \\
    |cdocspt4.tex| & sample part file \\
    |cdocsdrf.tex| & sample redirection file \\
    |cdocsfn1.tex| & sample redirection file \\
    |cdocsfn2.tex| & sample redirection file \\
    |childdoc.pdf| & manual
\end{tabular}
\end{center}
%
The distribution consists of the files
|README.txt|, |childdoc.ins| and |childdoc.dtx|.
%
\begin{itemize}
\item
Run (pdf)\LaTeX{} on |childdoc.dtx|
to compile the manual |childdoc.pdf| (this file).
\item
Run \LaTeX{} on |childdoc.ins| to create the definitions file |childdoc.def|
and the sample |cdocsamp.tex| with include files
|cdocsch1.tex|, |cdocsch2.tex|, |cdocspt3.tex|, |cdocspt4.tex|,
|cdocsdrf.tex|, |cdocsfn1.tex|, |cdocsfn2.tex|.
Then copy the file |childdoc.def| to an appropriate directory of your \LaTeX{}
distribution, e.g.\ \textit{texmf-root}|/tex/latex/childdoc|.
\end{itemize}

%%%%%%%%%%%%%%%%%%%%%%%%%%%%%%%%%%%%%%%%%%%%%%%%%%%%%%%%%%%%%%%%%%%%%%%%%%%%%%%%
\subsection{Related CTAN Packages}

There are several other packages which offer a similar functionality:
%
\begin{itemize}
\item
The packages
\href{http://ctan.org/pkg/docmute}{\textsf{docmute}},
\href{http://ctan.org/pkg/includex}{\textsf{includex}} and
\href{http://ctan.org/pkg/standalone}{\textsf{standalone}}
provide commands to include only the document body of
a child file thus allowing both files to be compiled individually.
\item
The packages \href{http://ctan.org/pkg/subdocs}{\textsf{subdocs}}
and \href{http://ctan.org/pkg/subfiles}{\textsf{subfiles}}
provide structures in which the main and child documents can be
encapsulated and allowing them to be compiled individually.
The inclusion mechanism is different from the conventional |\include|.
\item
The package \href{http://ctan.org/pkg/combine}{\textsf{combine}}
is an elaborate solution to combine several documents into one.
\end{itemize}
%
See also the CTAN topic \href{http://ctan.org/topic/subdocs}{\textsf{subdocs}}
for further related packages.
The present package differs from the above solutions in that
a document structure constructed with the conventional |\include| mechanism
just needs two extra commands at the top of every file
such that all constituent files can be compiled individually.

%%%%%%%%%%%%%%%%%%%%%%%%%%%%%%%%%%%%%%%%%%%%%%%%%%%%%%%%%%%%%%%%%%%%%%%%%%%%%%%%
%\subsection{Feature Suggestions}
%
%The following is a list of features which may be useful for future
%versions of this package:
%%
%\begin{itemize}
%\item
%\ldots
%\end{itemize}

%%%%%%%%%%%%%%%%%%%%%%%%%%%%%%%%%%%%%%%%%%%%%%%%%%%%%%%%%%%%%%%%%%%%%%%%%%%%%%%%
\subsection{Revision History}

%%%%%%%%%%%%%%%%%%%%%%%%%%%%%%%%%%%%%%%%
\paragraph{v2.0:} 2018/12/30

\begin{itemize}
\item
immediate forward processing
\item
added |\childdocby| mechanism
\item
manual restructured
\end{itemize}

%%%%%%%%%%%%%%%%%%%%%%%%%%%%%%%%%%%%%%%%
\paragraph{v1.6:} 2018/01/17

\begin{itemize}
\item
application for development of include files
\item
corrections to manual
\end{itemize}

%%%%%%%%%%%%%%%%%%%%%%%%%%%%%%%%%%%%%%%%
\paragraph{v1.5:} 2017/05/21

\begin{itemize}
\item
more complete structuring introduced
\item
|\childdocof| introduced
\item
|\childdoc| renamed to |\childdocmain|
\item
|\childredirect| renamed to |\childdocforward| and |\childdocforwardprefix|
and functionality expanded
\end{itemize}

%%%%%%%%%%%%%%%%%%%%%%%%%%%%%%%%%%%%%%%%
\paragraph{v1.0:} 2017/04/27

\begin{itemize}
\item
manual and install package
\item
first version published on CTAN
\end{itemize}

%%%%%%%%%%%%%%%%%%%%%%%%%%%%%%%%%%%%%%%%
\paragraph{v0.6:} 2017/04/26

\begin{itemize}
\item
redirection mechanism added
\end{itemize}

%%%%%%%%%%%%%%%%%%%%%%%%%%%%%%%%%%%%%%%%
\paragraph{v0.5:} 2017/04/26

\begin{itemize}
\item
functionality in definition file
\end{itemize}


%%%%%%%%%%%%%%%%%%%%%%%%%%%%%%%%%%%%%%%%%%%%%%%%%%%%%%%%%%%%%%%%%%%%%%%%%%%%%%%%
%%%%%%%%%%%%%%%%%%%%%%%%%%%%%%%%%%%%%%%%%%%%%%%%%%%%%%%%%%%%%%%%%%%%%%%%%%%%%%%%
%%%%%%%%%%%%%%%%%%%%%%%%%%%%%%%%%%%%%%%%%%%%%%%%%%%%%%%%%%%%%%%%%%%%%%%%%%%%%%%%
\appendix

\settowidth\MacroIndent{\rmfamily\scriptsize 000\ }

 \DocInput{childdoc.dtx}

\end{document}
%</driver>
% \fi
%
% %%%%%%%%%%%%%%%%%%%%%%%%%%%%%%%%%%%%%%%%%%%%%%%%%%%%%%%%%%%%%%%%%%%%%%%%%%%%%%
% %%%%%%%%%%%%%%%%%%%%%%%%%%%%%%%%%%%%%%%%%%%%%%%%%%%%%%%%%%%%%%%%%%%%%%%%%%%%%%
% \section{Sample}
%\iffalse
%<*samplemain>
%\fi
%
% The following presents a sample document
% with two chapters, two parts, a title page,
% a compile flag as well as three forwarding files to set the flag.
% It consists of eight |.tex| files:
% \begin{center}
% \begin{tabular}{ll}
% |cdocsamp.tex|&main file\\
% |cdocsch1.tex|&include file for chapter 1\\
% |cdocsch2.tex|&include file for chapter 2\\
% |cdocspt3.tex|&include file for part 3\\
% |cdocspt4.tex|&include file for part 4\\
% |cdocsdrf.tex|&forwarding file for main file in draft mode\\
% |cdocsfi1.tex|&forwarding file for final version of chapter 1\\
% |cdocsfi2.tex|&forwarding file for final version of chapter 2\\
% \end{tabular}
% \end{center}
% Each of the eight files can be compiled directly by the \LaTeX{} compiler.
%
% %%%%%%%%%%%%%%%%%%%%%%%%%%%%%%%%%%%%%%
% \paragraph{Main File.}
%
% The main file is called |cdocsamp.tex|.
%
% Load the \textsf{childdoc} definitions and
% declare the filename for the main document:
%    \begin{macrocode}
\input{childdoc.def}
\childdocmain{}
%    \end{macrocode}

% Optional override for |\version| flag:
%    \begin{macrocode}
%%\ifchilddoc\else\providecommand{\version}{draft}\fi
%    \end{macrocode}

% Define the default values for the |\version| flag
% (|final| for the main file and |draft| for childs):
%    \begin{macrocode}
\ifchilddoc
\providecommand{\version}{draft}
\else
\providecommand{\version}{final}
\fi
%    \end{macrocode}

% Load the standard document class:
%    \begin{macrocode}
\documentclass[12pt]{article}
%    \end{macrocode}

% Start the document body:
%    \begin{macrocode}
\begin{document}
%    \end{macrocode}

% Declare a title page.
% Print title, part of document being processed and version flag:
%    \begin{macrocode}
\addtocounter{page}{-1}
\begin{center}
{\LARGE\bfseries{}childdoc example\par}
\vspace{1cm}
\ifchilddoc
\ifchilddocmanual part\else chapter\fi:
`\childdocname' of `\childdocjob'\par
\else
main document: `\childdocjob'\par
\fi
version: \version\par
\end{center}
\newpage
%    \end{macrocode}

% Manually include selected file,
% otherwise process as usual:
%    \begin{macrocode}
\ifchilddocmanual
\section*{part `\childdocname'}
\input{\childdocname}
\else
%    \end{macrocode}

% Include the two chapters:
%    \begin{macrocode}
\include{cdocsch1}
\include{cdocsch2}
%    \end{macrocode}

% Include the two parts unless only chapters should be displayed:
%    \begin{macrocode}
\ifchilddoc\else
\section{part three}
\input{cdocspt3}
\section{part four}
\input{cdocspt4}
\fi
%    \end{macrocode}

% Process as usual until here:
%    \begin{macrocode}
\fi
%    \end{macrocode}

% End of document body:
%    \begin{macrocode}
\end{document}
%    \end{macrocode}
%\iffalse
%</samplemain>
%\fi
%
% %%%%%%%%%%%%%%%%%%%%%%%%%%%%%%%%%%%%%%
% \paragraph{Chapter Include Files.}
%
% The include files are called |cdocsch1.tex| and |cdocsch2.tex|.
%
%\iffalse
%<*samplechap1|samplechap2>
%\fi

% Optional override for |\version| flag:
%    \begin{macrocode}
%%\providecommand{\version}{final}
%    \end{macrocode}

% Include the main document:
%    \begin{macrocode}
\input{childdoc.def}
\childdocof{cdocsamp}
%    \end{macrocode}

%\iffalse
%</samplechap1|samplechap2>
%\fi
%
%\iffalse
%<*samplechap1>
%\fi
% Some text for chapter 1:
%    \begin{macrocode}
\section{one}
some text in chapter one
%    \end{macrocode}

%\iffalse
%</samplechap1>
%\fi
% Some text for chapter 2:
%\iffalse
%<*samplechap2>
%\fi
%    \begin{macrocode}
\section{two}
more text in chapter two
%    \end{macrocode}

%\iffalse
%</samplechap2>
%\fi
%
% %%%%%%%%%%%%%%%%%%%%%%%%%%%%%%%%%%%%%%
% \paragraph{Part Include Files.}
%
% The include files are called |cdocspt3.tex| and |cdocspt4.tex|.
%
%\iffalse
%<*samplepart3|samplepart4>
%\fi

% Optional override for |\version| flag:
%    \begin{macrocode}
%%\providecommand{\version}{final}
%    \end{macrocode}

% Include the main document:
%    \begin{macrocode}
\input{childdoc.def}
\childdocby{cdocsamp}
%    \end{macrocode}

%\iffalse
%</samplepart3|samplepart4>
%\fi
%
%\iffalse
%<*samplepart3>
%\fi
% Some text for part 3:
%    \begin{macrocode}
some text in part three
%    \end{macrocode}

%\iffalse
%</samplepart3>
%\fi
% Some text for part 4:
%\iffalse
%<*samplepart4>
%\fi
%    \begin{macrocode}
more text in part four
%    \end{macrocode}

%\iffalse
%</samplepart4>
%\fi
%
% %%%%%%%%%%%%%%%%%%%%%%%%%%%%%%%%%%%%%%
% \paragraph{Forwarding for a Complete Draft.}
%
% The following forwarding file |cdocsdrf.tex|
% compiles the main document in draft mode:
%\iffalse
%<*sampledraft>
%\fi
%    \begin{macrocode}
\def\version{draft}
\input{childdoc.def}
\childdocforward{cdocsamp}
%    \end{macrocode}

%\iffalse
%</sampledraft>
%\fi
%
% %%%%%%%%%%%%%%%%%%%%%%%%%%%%%%%%%%%%%%
% \paragraph{Forwarding for Final Version of the Chapters.}
%
% The following forwarding files |cdocsfn1.tex| and |cdocsfn2.tex|
% (with identical content)
% compile the final versions of the child documents
% |cdocsch1.tex| and |cdocsch2.tex|, respectively:
%\iffalse
%<*samplefinal>
%\fi
%    \begin{macrocode}
\def\version{final}
\input{childdoc.def}
\childdocforwardprefix[cdocsamp]{cdocsfn}{cdocsch}
%    \end{macrocode}

%\iffalse
%</samplefinal>
%\fi
%
% %%%%%%%%%%%%%%%%%%%%%%%%%%%%%%%%%%%%%%
% \paragraph{Command Line Processing.}
%
% The following three command lines generate the output files
% |cdocscld|, |cdocscl1| and |cdocscl2|
% which should be identical to
% |cdocsdrf|, |cdocsch1| and |cdocsfn2|, respectively:
% \begin{center}
% \begin{tabular}{l}
% |latex -jobname cdocscld \|\\
% |  "\def\version{draft}\input{childdoc.def}\childdocforward{cdocsamp}"|\\
% |latex -jobname cdocscl1 \|\\
% |  "\input{childdoc.def}\childdocforward[cdocsamp]{cdocsch1}"|\\
% |latex -jobname cdocscl2 \|\\
% |  "\def\version{final}\input{childdoc.def}\childdocforward{cdocsch2}"|
% \end{tabular}
% \end{center}
% Note that the trailing backslash on each first line
% merely continues the input to the second line
% (for convenient cut ant paste).
% Furthermore, the command |latex| can be replaced by any
% of its alternative versions such as |pdflatex|.
%
% %%%%%%%%%%%%%%%%%%%%%%%%%%%%%%%%%%%%%%%%%%%%%%%%%%%%%%%%%%%%%%%%%%%%%%%%%%%%%%
% %%%%%%%%%%%%%%%%%%%%%%%%%%%%%%%%%%%%%%%%%%%%%%%%%%%%%%%%%%%%%%%%%%%%%%%%%%%%%%
% \section{Implementation}
%\iffalse
%<*package>
%\fi
%
% This section describes the definitions file |childdoc.def|.

% The definitions cannot be loaded using |\usepackage| or |\RequirePackage|
% which has a mechanism to prevent loading a style file more than once.
% When loading the definitions by means of |\input|
% multiple instances have to be prevented manually:
%\iffalse
%This code needs to be before the `\ProvidesFile' directive
%which is defined at the beginning of this file.
%Therefore it is also placed there and commented out here.
%</package>
%<*discard>
%\fi
%    \begin{macrocode}
\ifdefined\childdocmain\endinput\fi
%    \end{macrocode}
%\iffalse
%</discard>
%<*package>
%\fi
%
% \macro{\ifchilddoc}
% \macro{\ifchilddocmanual}
% The conditional |\ifchilddoc| tells whether a
% child (true) or main (false) document is being compiled.
% The conditional |\ifchilddocmanual| tells whether
% the |\includeonly| mechanism is used (false) or
% the selection of child files must be performed manually (true).
% The definitions initialise to false:
%    \begin{macrocode}
\newif\ifchilddoc
\newif\ifchilddocmanual
%    \end{macrocode}

% \macro{\childdocname}
% \macro{\childdocjob}
% The macro |\childdocname| stores the name of the main document
% to be compiled. The macro |\childdocjob| stores the name of
% the document on which the \LaTeX{} compiler was originally invoked.
% The content of |\jobname| cannot be compared
% to filenames specified in the source due to different catcodes.
% The following code rescans |\jobname|, stores the result
% in |\childdocname| and saves a copy in |\childdocjob|:
%    \begin{macrocode}
\edef\childdocname{\scantokens\expandafter{\jobname\noexpand}}
\let\childdocjob\childdocname
%    \end{macrocode}

% \macro{\childdocdisable}
% The macro |\childdocdisable| prevents the main file
% from being processed more than once.
% At this stage, the main document command |\childdocmain|
% is assumed to be called once again where it should do nothing.
% Any subsequent call to it should prevent
% a secondary processing of the main document
% It overwrites the forwarding commands
% |\childdocof| and |\childdocforward|
% with empty macros to prevent further inclusions of the main document:
%    \begin{macrocode}
\newcommand{\childdocdisable}
{
  \renewcommand{\childdocmain}[1]{\renewcommand{\childdocmain}[1]{\endinput}}
  \renewcommand{\childdocof}[1]{}
  \renewcommand{\childdocby}[2][]{}
  \renewcommand{\childdocforward}[2][]{}
  \renewcommand{\childdocdisable}{}
}
%    \end{macrocode}

% \macro{\childdocmain}
% The macro |\childdocmain| is to be called at the top of the main file
% with nothing or the main filename (without extension) as argument.
% First, it breaks loops.
% If the argument is not empty and does not match |\childdocname|
% (which is set by the first inclusion of |childdoc.def|),
% |\ifchilddoc| is set to true, |\includeonly| is applied to the child file
% and |\jobname| is set to the main file
% (for proper handling of |.aux| files):
%    \begin{macrocode}
\newcommand{\childdocmain}[1]
{
  \childdocdisable\childdocmain{}
  \if?#1?\else
    \begingroup
      \def\childdoctmp{#1}
      \ifx\childdoctmp\childdocname
        \def\childdoctmp{}
      \else
        \def\childdoctmp
        {
          \childdoctrue
          \includeonly{\childdocname}
          \def\childdocjob{#1}
          \def\jobname{#1}
        }
      \fi
      \expandafter
    \endgroup
    \childdoctmp
  \fi
}
%    \end{macrocode}

% \macro{\childdocof}
% The command |\childdocof| redirects
% compilation to the main file |#1|.
%    \begin{macrocode}
\newcommand{\childdocof}[1]
{
  \childdocdisable
  \childdoctrue
  \includeonly{\childdocname}
  \def\jobname{#1}
  \def\childdocjob{#1}
  \input{#1}
}
%    \end{macrocode}

% \macro{\childdocby}
% The command |\childdocby| ....
%    \begin{macrocode}
\newcommand{\childdocby}[2][]
{
  \childdocdisable
  \childdoctrue
  \childdocmanualtrue
  \if?#1?\else
    \def\jobname{#2}
  \fi
  \def\childdocjob{#2}
  \input{#2}
  \endinput
}
%    \end{macrocode}

% \macro{\childdocforward}
% The command |\childdocforward| redirects
% compilation to the main file or
% (if the optional argument is given) a child file.
% Parameters are set as if the main file
% or a child file starting with |\childdocof| was compiled.
% Then compilation is handed over to the main file:
%    \begin{macrocode}
\newcommand{\childdocforward}[2][]
{
  \begingroup
    \if?#1?
      \def\childdoctmp
      {
        \def\childdocname{#2}
        \def\childdocjob{#2}
        \def\jobname{#2}
        \input{#2}
        \endinput
      }
    \else
      \def\childdoctmp
      {
        \childdocdisable
        \def\childdocname{#2}
        \childdoctrue
        \includeonly{#2}
        \def\childdocjob{#1}
        \def\jobname{#1}
        \input{#1}
        \endinput
      }
    \fi
    \expandafter
  \endgroup
  \childdoctmp
}
%    \end{macrocode}

% \macro{\childdocforwardprefix}
% The command |\childdocforwardprefix| redirects
% compilation to the main or a child file by means of a pattern.
% The prefix |#1| in the current filename is replaced by |#2|
% and the suffix of the current filename is kept
% (it is assumed that the filename does not contain the substring `|~~~|'
% which is used as a delimiter).
% Compilation is handed over to the new file by |\childdocforward|:
%    \begin{macrocode}
\newcommand{\childdocforwardprefix}[3][]
{
  \begingroup
    \def\childdocextract #2##1~~~{\def\childdoctmp{\childdocforward[#1]{#3##1}}}
    \expandafter\childdocextract\childdocname~~~
    \expandafter
  \endgroup
  \childdoctmp
}
%    \end{macrocode}

% \macro{\childdoc}
% The deprecated macro |\childdoc| is a legacy version of |\childdocmain|:
%    \begin{macrocode}
\newcommand{\childdoc}{\childdocmain}
%    \end{macrocode}

% \macro{\childdocredirect}
% The deprecated macro |\childdocredirect| is a legacy version
% of |\childdocforward| and |\childdocforwardprefix|:
%    \begin{macrocode}
\newcommand{\childdocredirect}[2][]
{
  \begingroup
    \if?#1?
      \def\childdoctmp{\childdocforward{#2}}
    \else
      \def\childdoctmp{\childdocforwardprefix{#1}{#2}}
    \fi
    \expandafter
  \endgroup
  \childdoctmp
}
%    \end{macrocode}

%\iffalse
%</package>
%\fi
%
\endinput

\childdocby{cdocsamp}
%    \end{macrocode}

%\iffalse
%</samplepart3|samplepart4>
%\fi
%
%\iffalse
%<*samplepart3>
%\fi
% Some text for part 3:
%    \begin{macrocode}
some text in part three
%    \end{macrocode}

%\iffalse
%</samplepart3>
%\fi
% Some text for part 4:
%\iffalse
%<*samplepart4>
%\fi
%    \begin{macrocode}
more text in part four
%    \end{macrocode}

%\iffalse
%</samplepart4>
%\fi
%
% %%%%%%%%%%%%%%%%%%%%%%%%%%%%%%%%%%%%%%
% \paragraph{Forwarding for a Complete Draft.}
%
% The following forwarding file |cdocsdrf.tex|
% compiles the main document in draft mode:
%\iffalse
%<*sampledraft>
%\fi
%    \begin{macrocode}
\def\version{draft}
% \iffalse
%
% childdoc.dtx Copyright (C) 2017-2018 Niklas Beisert
%
% This work may be distributed and/or modified under the
% conditions of the LaTeX Project Public License, either version 1.3
% of this license or (at your option) any later version.
% The latest version of this license is in
%   http://www.latex-project.org/lppl.txt
% and version 1.3 or later is part of all distributions of LaTeX
% version 2005/12/01 or later.
%
% This work has the LPPL maintenance status `maintained'.
%
% The Current Maintainer of this work is Niklas Beisert.
%
% This work consists of the files childdoc.dtx and childdoc.ins
% and the derived files childdoc.def and cdocsamp.tex with
% cdocsch1.tex, cdocsch2.tex, cdocsdrf.tex, cdocsfn1.tex, cdocsfn2.tex.
%
%<package>\ifdefined\childdocmain\endinput\fi
%<package>\ProvidesFile{childdoc.def}[2018/12/30 v2.0 child document driver]
%<samplemain>\ProvidesFile{cdocsamp.tex}[2018/12/30 v2.0 sample for childdoc]
%<*driver>
%\ProvidesFile{childdoc.drv}[2018/12/30 v2.0 childdoc reference manual file]
\PassOptionsToClass{10pt,a4paper}{article}
\documentclass{ltxdoc}

\usepackage[margin=35mm]{geometry}
\usepackage{hyperref}
\usepackage{hyperxmp}
\usepackage[usenames]{color}

\hypersetup{colorlinks=true}
\hypersetup{pdfstartview=FitH}
\hypersetup{pdfpagemode=UseNone}
\hypersetup{pdfsource={}}
\hypersetup{pdflang={en-UK}}
\hypersetup{pdfcopyright={Copyright 2017-2018 Niklas Beisert.
  This work may be distributed and/or modified under the
  conditions of the LaTeX Project Public License, either version 1.3
  of this license or (at your option) any later version.}}
\hypersetup{pdflicenseurl={http://www.latex-project.org/lppl.txt}}
\hypersetup{pdfcontactaddress={ETH Zurich, ITP, HIT K,
  Wolfgang-Pauli-Strasse 27}}
\hypersetup{pdfcontactpostcode={8093}}
\hypersetup{pdfcontactcity={Zurich}}
\hypersetup{pdfcontactcountry={Switzerland}}
\hypersetup{pdfcontactemail={nbeisert@itp.phys.ethz.ch}}
\hypersetup{pdfcontacturl={http://people.phys.ethz.ch/\xmptilde nbeisert/}}

\newcommand{\secref}[1]{\hyperref[#1]{section \ref*{#1}}}

\parskip1ex
\parindent0pt
\let\olditemize\itemize
\def\itemize{\olditemize\parskip0pt}

\begin{document}

\title{The \textsf{childdoc} Package}
\hypersetup{pdftitle={The childdoc Package}}
\author{Niklas Beisert\\[2ex]
  Institut f\"ur Theoretische Physik\\
  Eidgen\"ossische Technische Hochschule Z\"urich\\
  Wolfgang-Pauli-Strasse 27, 8093 Z\"urich, Switzerland\\[1ex]
  \href{mailto:nbeisert@itp.phys.ethz.ch}
  {\texttt{nbeisert@itp.phys.ethz.ch}}}
\hypersetup{pdfauthor={Niklas Beisert}}
\hypersetup{pdfsubject={Manual for the LaTeX2e Package childdoc}}
\date{30 December 2018, \textsf{v2.0}}
\maketitle

\begin{abstract}\noindent
\textsf{childdoc} is a \LaTeXe{} package
that enables the direct compilation
of document sections included by |\include|
to individual files.
\end{abstract}

\begingroup
\parskip0ex
\tableofcontents
\endgroup

%%%%%%%%%%%%%%%%%%%%%%%%%%%%%%%%%%%%%%%%%%%%%%%%%%%%%%%%%%%%%%%%%%%%%%%%%%%%%%%%
%%%%%%%%%%%%%%%%%%%%%%%%%%%%%%%%%%%%%%%%%%%%%%%%%%%%%%%%%%%%%%%%%%%%%%%%%%%%%%%%
\section{Introduction}

\LaTeX{} provides a mechanism to structure a large document (such as a book)
into a main file and several child files (containing the chapters)
using the |\include| command.
This mechanism is beneficial for documents
which span hundreds of pages in order to
make the source file(s) more manageable.
Moreover, compilation can be restricted to
selected child files by means of the |\includeonly| command.
The latter feature can be used to reduce the compilation time while editing
(this was significantly more useful in the earlier days of \LaTeX{})
or to generate a smaller document which is easier to navigate.
Another application of |\includeonly| is to generate
documents consisting of selected parts of the complete document.

However, there are a few drawbacks of the plain |\include| mechanism:
\begin{itemize}
\item
The child files cannot be compiled on their own,
they can only be compiled via the main file.
A naive editing environment
(such as a text editor with an option
to have the current file processed by \LaTeX)
may require one to switch to the main file before compiling;
attempting to compile the child file produces errors.
\item
The main file must be modified (each time)
to adjust the |\includeonly| command
to the present needs. This easily leaves the main file in a messy state.
\item
The generated document will always carry the filename
of the main document. This is inconvenient if
several child files are to be compiled and
to be kept for distribution.
\end{itemize}

The present package provides a simple interface
to make child files individually compilable by \LaTeX{}.
Compiling a child file then has the same effect as compiling
the main file with an |\includeonly| command
to select the appropriate child.
Moreover the generated document will carry the name of the child
rather than the main file.
This resolves all three above issues.

This feature is meant to make the editing of books,
thesis documents and lecture notes somewhat more convenient.
However, the package can also be used efficiently for
composing a series of documents (such as exercise sheets)
which are typically distributed individually.
It then assists the author in generating the individual documents
(potentially in different versions)
as well as a document containing the collected series.
Another application is in developing style files
or other kinds of included material
where compilation of the style file could redirect
to a sample or test file.

%%%%%%%%%%%%%%%%%%%%%%%%%%%%%%%%%%%%%%%%%%%%%%%%%%%%%%%%%%%%%%%%%%%%%%%%%%%%%%%%
%%%%%%%%%%%%%%%%%%%%%%%%%%%%%%%%%%%%%%%%%%%%%%%%%%%%%%%%%%%%%%%%%%%%%%%%%%%%%%%%
\section{Usage}

First of all, the package \textsf{childdoc} is \emph{not} a standard
\LaTeXe{} |.sty| style file! Therefore it needs to be invoked in
a non-standard way.

%%%%%%%%%%%%%%%%%%%%%%%%%%%%%%%%%%%%%%%%%%%%%%%%%%%%%%%%%%%%%%%%%%%%%%%%%%%%%%%%
\subsection{Included Files}
\label{sec:include}

%%%%%%%%%%%%%%%%%%%%%%%%%%%%%%%%%%%%%%%%
\DescribeMacro{\childdocmain}
To use the package, add the commands
\begin{center}
\begin{tabular}{l}
|\input{childdoc.def}|\\
|\childdocmain{}|\\
\end{tabular}
\end{center}
at the very top of the main \LaTeX{} file,
in particular \emph{before} the |\documentclass| statement!
The argument of |\childdocmain| should be left empty
(but it must be present).

%%%%%%%%%%%%%%%%%%%%%%%%%%%%%%%%%%%%%%%%
\DescribeMacro{\childdocof}
Furthermore, add the commands
\begin{center}
\begin{tabular}{l}
|\input{childdoc.def}|\\
|\childdocof{|\textit{main}|}|\\
\end{tabular}
\end{center}
at the top of every child file \textit{child}
which is included by |\include{|\textit{child}|}|
from within the main file
(or at least for those files to be compiled individually).
The argument \textit{main} must be the filename of the main file.

There are a couple of
considerations in setting up the main and child documents:

%%%%%%%%%%%%%%%%%%%%%%%%%%%%%%%%%%%%%%%%
\paragraph{Restrictions.}

Please note the following restrictions:
\begin{itemize}
\item
|\childdocmain| must be called with one argument \textit{main}
to ensure compatibility with earlier version of the package.
It must either be empty (|\childdocmain{}|)
or precisely match the filename of the main file in which it is specified.
See \secref{sec:detection} for further information.
\item
The filename \textit{main} must be specified without the |.tex| extension.
\item
The filename \textit{main} is case sensitive
(even in case-insensitive file systems)
due to internal string comparison.
\item
The argument \textit{main} should be fully expanded, it cannot be a macro.
\item
Subdirectories and special characters should be avoided in filenames.
\item
The command |\childdocmain{|\textit{main}|}| must be followed by a whitespace.
It should not be followed immediately by another command
or by a comment mark `|%|'.
This is because the \TeX{} parser reads the token immediately following
the argument of |\childdocmain| and puts it
at the beginning of every child section;
however, a white\-space is ignored.
\end{itemize}

%%%%%%%%%%%%%%%%%%%%%%%%%%%%%%%%%%%%%%%%
\paragraph{Content of Main File.}

It is advisable to place all content in the child files included by |\include|.
Any output contained in the main file will appear in all child documents
unless suppressed manually;
it cannot be suppressed automatically by the |\includeonly| directive
and thus should normally be avoided.
A method to include some content in the main file
by means of conditional processing is described in \secref{sec:conditional}.

%%%%%%%%%%%%%%%%%%%%%%%%%%%%%%%%%%%%%%%%
\paragraph{Page Numbering.}

When only a part of the document is compiled,
the appropriate numbering of pages
(as well as other status parameters)
is determined from the |.aux| files.
The latter contain information from previous passes.
However this information needs to propagate through
all intermediate child documents.
Therefore the page numbering in child documents may well
be inconsistent until the complete document is compiled at least once.

A useful (if unconventional) way to always ensure a consistent
page numbering is to restart the numbering in each child document
and denote the pages by `\textit{child}|.|\textit{page}'
where \textit{child} represents the chapter/section number of the child file.
This can be achieved by the command
|\numberwithin{page}{|\textit{child}|}|
of the \textsf{amsmath} package
where \textit{child} can be |chapter| or |section|
depending on the chosen structuring.
Alternatively, one can modify the macro |\thepage| appropriately
and reset the counter |page| at the start of each child file.

%%%%%%%%%%%%%%%%%%%%%%%%%%%%%%%%%%%%%%%%%%%%%%%%%%%%%%%%%%%%%%%%%%%%%%%%%%%%%%%%
\subsection{Conditional Processing}
\label{sec:conditional}

The package provides a mechanism to compile different versions
of a document. To customise the versions further some conditional processing
can come in handy to distinguish which version is being compiled.
The package provides two macros to describe the compilation context:

%%%%%%%%%%%%%%%%%%%%%%%%%%%%%%%%%%%%%%%%
\DescribeMacro{\ifchilddoc}
The conditional |\ifchilddoc| distinguishes between the compilation of
child documents and the main document:
%
\begin{center}
|\ifchilddoc |\textit{child-code}| |[|\||else |\textit{main-code}]| \||fi|
\end{center}

%%%%%%%%%%%%%%%%%%%%%%%%%%%%%%%%%%%%%%%%
\DescribeMacro{\childdocname}
\DescribeMacro{\childdocjob}
The macro |\childdocname| contains the filename (without extension)
of the main or child file being processed.
Note that |\childdocjob| will always contain the name of the main file.

%%%%%%%%%%%%%%%%%%%%%%%%%%%%%%%%%%%%%%%%
\paragraph{Title Page.}

Conditional processing can be used to include a title or banner page
in the main document when proper precautions are taken.
Importantly, the code in the main file should ensure that the page counter
(as well as other status parameters which are stored in the |.aux| files)
takes the same value after the conditional processing.
Otherwise the page numbers may take divergent values
depending on which part is compiled.

For example, a title page could be declared by:
%
\begin{center}
\begin{tabular}{l}
|\ifchilddoc\||else|\\
|\addtocounter{page}{-1}|\\
\textit{code for title page}\\
|\newpage|\\
|\||fi|
\end{tabular}
\end{center}
%
A banner page for the child documents can be generated by:
%
\begin{center}
\begin{tabular}{l}
|\ifchilddoc|\\
|\addtocounter{page}{-1}|\\
\textit{code for banner page}\\
|\newpage|\\
|\||fi|
\end{tabular}
\end{center}
%
Here one could write a message such as:
\begin{center}
|This is the part \childdocname{} of \childdocjob{}.|
\end{center}

%%%%%%%%%%%%%%%%%%%%%%%%%%%%%%%%%%%%%%%%%%%%%%%%%%%%%%%%%%%%%%%%%%%%%%%%%%%%%%%%
\subsection{Flags}
\label{sec:flags}

The package makes it easy to generate different versions
of the main or child documents.
To this end compilation flags can be defined
and assigned different default values.
They will be particularly useful in conjunction
with the forwarding mechanism described in \secref{sec:forward}.

For example, it may be useful to have a flag |\version|
which can be set to |draft| or |final|.
The document source will contain some conditional code
depending on the value of |\version|.
Suppose further, the flag should default to |final| for the main file
and to |draft| for child files
which is a natural assignment for editing the document.
This is achieved by placing the following code
in the preamble of the main document
(below the |\childdocmain| directive):
%
\begin{center}
\begin{tabular}{l}
|\ifchilddoc|\\
|\providecommand{\version}{draft}|\\
|\||else|\\
|\providecommand{\version}{final}|\\
|\||fi|
\end{tabular}
\end{center}
%
The definition by |\providecommand| makes sure
that previous definitions are not overwritten.
Further statements |\providecommand{\version}{...}|
can thus be added before the above code to override it.

For the main file, one might add a line
(between |\childdocmain| and the above block)
%
\begin{center}
|%\ifchilddoc\||else\providecommand{\version}{draft}\||fi|
\end{center}
%
which can be uncommented to produce a draft version.
Likewise one can add a line to the very top of a child file
(above the |\childdocof{|\textit{main}|}| directive)
%
\begin{center}
|%\providecommand{\version}{final}|
\end{center}
%
which can be uncommented to produce the final version of this child document.

%%%%%%%%%%%%%%%%%%%%%%%%%%%%%%%%%%%%%%%%%%%%%%%%%%%%%%%%%%%%%%%%%%%%%%%%%%%%%%%%
\subsection{Forwarding}
\label{sec:forward}

Different versions of the main or child documents
using compilation flags as described in \secref{sec:flags}
can be (permanently) stored in different files
for convenient compilation, viewing and distribution.
To this end, the package defines a command
to pass on compilation to a different file:

%%%%%%%%%%%%%%%%%%%%%%%%%%%%%%%%%%%%%%%%
\DescribeMacro{\childdocforward}
The command |\childdocforward| redirects processing to
another source file:
%
\begin{center}
\begin{tabular}{l}
|\input{childdoc.def}|\\
|\childdocforward[|\textit{main}|]{|\textit{dest}|}|\\
\end{tabular}
\end{center}
%
The argument \textit{dest} is the destination file
(without extension).
It should be the main file or one of the child files.
Note that further \textsf{childdoc} directives
such as |\childdocof| and |\childdocforward|
in the indicated file will be processed in this form.
The optional argument \textit{main}
passes on directly to the main file \textit{main}
while pretending to compile the child \textit{dest}.
This form behaves as if \textit{dest}
issues |\childdocof{|\textit{main}|}| right away,
and no further \textsf{childdoc} directives will be processed.

%%%%%%%%%%%%%%%%%%%%%%%%%%%%%%%%%%%%%%%%
\DescribeMacro{\...prefix}
In the alternative form |\childdocforwardprefix|,
%
\begin{center}
\begin{tabular}{l}
|\input{childdoc.def}|\\
|\childdocforwardprefix[|\textit{main}|]{|\textit{prefix}|}{|\textit{dest}|}|
\end{tabular}
\end{center}
%
the destination file is determined by a pattern
depending on the current file:
To make this work, the current file must be called
`{\textit{prefix}\hspace{0.2em}\textit{suffix}}'
with \textit{prefix} matching precisely the argument.
Processing is then passed on to the file
`{\textit{dest}\hspace{0.2em}\textit{suffix}}'.
Surely, the same effect is achieved by
directly specifying the
argument `{\textit{dest}\hspace{0.2em}\textit{suffix}}'
in the first form.
However, that requires to set up a different file
for each child. With the alternative form of the command
all these files can have exactly the same content
which simplifies setting them up and maintaining them.

For example, the following file |draft.tex|
with a compilation flag |\version| as described in \secref{sec:flags}
compiles the main document as a draft:
%
\begin{center}
\begin{tabular}{l}
|\def\version{draft}|\\
|\input{childdoc.def}|\\
|\childdocforward{|\textit{main}|}|
\end{tabular}
\end{center}
%
Likewise, the following files |final|\textit{nn}|.tex|
compile the final version of the child document
|child|\textit{nn}|.tex|:
%
\begin{center}
\begin{tabular}{l}
|\def\version{final}|\\
|\input{childdoc.def}|\\
|\childdocforwardprefix{final}{child}|
\end{tabular}
\end{center}
%

Note that when several versions of a main file and/or of each child file
are to be generated, it may be convenient to set up a |Makefile| or
shell script to automatise the process.

%%%%%%%%%%%%%%%%%%%%%%%%%%%%%%%%%%%%%%%%%%%%%%%%%%%%%%%%%%%%%%%%%%%%%%%%%%%%%%%%
\subsection{Command Line Processing}
\label{sec:commandline}

The effect of redirection files can also be achieved by invoking
the \LaTeX{} compiler with a more elaborate command line.
Most conveniently this should be done as part
of a shell script or a |Makefile|.

When using \textsf{childdoc} in the main file, the following
command lines effectively perform a redirection
(note that depending on the shell being used,
backslashes may have to be doubled: `|\|' $\to$ `|\\|'):
%
\begin{center}
|... -jobname "|\textit{target}|" |\\|"|[\textit{flags}]%
|\input{childdoc.def}\childdocforward[|\textit{main}|]{|\textit{dest}|}"|
\end{center}
%
Here \textit{target} is the name of the output file,
\textit{main} is the name of the main file
and \textit{dest} is the name of the main or child file to be processed
(all filenames without extensions).
The optional argument \textit{main} can be omitted
if \textit{main} matches \textit{dest}.
Optionally, compilation \textit{flags} can be defined via |\def| commands.
This command line makes the \TeX{} engine believe
it is compiling the file \textit{target}
whose content is specified as the latter parameter.
The provided code then forwards the processing to
\textit{main} or \textit{dest} as described in \secref{sec:forward}.

%%%%%%%%%%%%%%%%%%%%%%%%%%%%%%%%%%%%%%%%%%%%%%%%%%%%%%%%%%%%%%%%%%%%%%%%%%%%%%%%
\subsection{Include by Input}
\label{sec:input}

Including child documents by |\include| has some restrictions by design.
Most notably, the content of a child document always occupies
its own set of pages; pages cannot be shared between child documents.
Usually, this behaviour makes perfect sense
because each child document contain an essential part of the document.
However, in some situations it may be desirable to compose
a document from a collection of parts
without having mandatory page breaks between then.
For this case, the package
provides a mechanism to include parts
by |\input| which can also be processed individually.
However, by construction this mechanism
requires manual handling of the content to be output.

%%%%%%%%%%%%%%%%%%%%%%%%%%%%%%%%%%%%%%%%
\DescribeMacro{\ifchilddocmanual}
The main file should be prepared as usual, see \secref{sec:include}.
However, the document body must make a distinction
between processing of an individual part and of the main document, e.g.:
%
\begin{center}
\begin{tabular}{l}
|\ifchilddocmanual|\\
|\input{\childdocname}|\\
|\||else|\\
\textit{document body with }|\input{|\textit{part}|}|\\
|\||fi|
\end{tabular}
\end{center}
%
The conditional |\ifchilddocmanual| is true whenever
a part to be included by |\input| is being compiled,
and the name of the part is stored in |\childdocname|.

%%%%%%%%%%%%%%%%%%%%%%%%%%%%%%%%%%%%%%%%
\DescribeMacro{\childdocby}
Each part to be included by |\input| should start with:
%
\begin{center}
\begin{tabular}{l}
|\input{childdoc.def}|\\
|\childdocby{|\textit{main}|}|\\
\end{tabular}
\end{center}
%
The directive |\childdocby| is similar to |\childdocof|
described in \secref{sec:include},
but the subsequent selection of content must be done manually.
To that end, both |\ifchilddoc| and |\ifchilddocmanual|
will be true upon processing of a part,
and the name of the part is stored in |\childdocname|.
Note that |\jobname| will be set to the filename of the current part
so that each part receives an individual |.aux| file
that does not interfere with the |.aux| file(s) of the main document.
This behaviour can be altered by the alternative form
|\childdocby[*]{|\textit{main}|}| (with a non-empty optional argument)
which uses the |.aux| file of the main document
by setting |\jobname| to \textit{main}.

%%%%%%%%%%%%%%%%%%%%%%%%%%%%%%%%%%%%%%%%%%%%%%%%%%%%%%%%%%%%%%%%%%%%%%%%%%%%%%%%
\subsection{Driver Development}
\label{sec:driver}

The \textsf{childdoc} mechanism can also be use for the development
of definition files such as \LaTeX{} styles or classes.
This case differs from the above setup with multiple parts
included by |\include| in that no |\includeonly| should be invoked.
This can be achieved by starting the include file
(before |\ProvidesPackage|) with:
%
\begin{center}
\begin{tabular}{l}
|\input{childdoc.def}|\\
|\childdocforward{|\textit{main}|}|\\
\end{tabular}
\end{center}
%
or alternatively with:
%
\begin{center}
\begin{tabular}{l}
|\input{childdoc.def}|\\
|\childdocby{|\textit{main}|}|\\
\end{tabular}
\end{center}
%
Both forms have slightly different effects as described above.
The main file is prepared as usual, see \secref{sec:include}.

%%%%%%%%%%%%%%%%%%%%%%%%%%%%%%%%%%%%%%%%%%%%%%%%%%%%%%%%%%%%%%%%%%%%%%%%%%%%%%%%
\subsection{Legacy Detection}
\label{sec:detection}

The directive |\childdocmain| in the main file can detect
whether the complete document or merely a child is to be compiled
even without using the directive |\childdocof|.
This method is deprecated because it is less robust
and there is no compelling reason to use it;
it is merely provided for backward compatibility
and it may be removed in future versions.

If the detection mechanism is to be used,
it is mandatory to correctly specify
the filename of the main file as the argument of |\childdocmain|:
%
\begin{center}
\begin{tabular}{l}
|\input{childdoc.def}|\\
|\childdocmain{|\textit{main}|}|\\
\end{tabular}
\end{center}
%
If |\jobname| does not match the argument \textit{main} of |\childdocmain|,
it is assumed that |\jobname| points to the child file to be compiled.
When using |\childdocmain| with the main file specified as argument,
it suffices to start a child file
with just |\input{|\textit{main}|}|
without loading of the package and using |\childdocof|.
If instead all processing is done
with the appropriate \textsf{childdoc} directives,
the argument of \textit{main} of |\childdocmain| can be empty.

An alternative version of the command line processing described
in \secref{sec:commandline} using the detection mechanism reads:
%
\begin{center}
|... -jobname "|\textit{target}|" "|[\textit{flags}]%
[|\def\jobname{|\textit{dest}|}|]|\input{|\textit{main}|}"|
\end{center}

%%%%%%%%%%%%%%%%%%%%%%%%%%%%%%%%%%%%%%%%%%%%%%%%%%%%%%%%%%%%%%%%%%%%%%%%%%%%%%%%
\subsection{Manual Code}
\label{sec:manual}

In case one cannot be certain whether the definitions file |childdoc.def|
is installed on the target \TeX{} distribution
and one prefers not to ship it,
it is conceivable to paste a few relevant commands into the sources.

To that end, drop all statements |\input{childdoc.def}|
and perform the replacements as outlined below.
Instead of |\childdocmain{|\textit{main}|}| add the following code
to the top of the main file:
%
\begin{center}
\begin{tabular}{l}
|\||ifdefined\childdocname\endinput\||fi\newif\ifchilddoc|\\
|\edef\childdocname{\scantokens\expandafter{\jobname\noexpand}}|\\
|\def\childdocmain{|\textit{main}|}\||ifx\childdocmain\childdocname\||else|\\
|\childdoctrue\includeonly{\childdocname}\let\jobname\childdocmain\||fi|\\
\end{tabular}
\end{center}
%
Instead of |\childdocof{|\textit{main}|}| just include the main file
at the top of each child file:
%
\begin{center}
|\input{|\textit{main}|}|
\end{center}
%
A simple redirection |\childdocforward{|\textit{dest}|}| is achieved by:
%
\begin{center}
|\def\jobname{|\textit{dest}|}\input{\jobname}|
\end{center}
%
The redirection with prefix
|\childdocforwardprefix[|\textit{prefix}|]{|\textit{dest}|}|
is accomplished by:
%
\begin{center}
\begin{tabular}{l}
|{\edef\jobname{\scantokens\expandafter{\jobname\noexpand}}|\\
|\def\redirectjob |\textit{prefix}|#1~~~{\gdef\jobname{|\textit{dest}|#1}}|\\
|\expandafter\redirectjob\jobname~~~}\input{\jobname}|
\end{tabular}
\end{center}

In an alternative approach,
child documents can be compiled by a specific command line
without additional code or specific definitions:
%
\begin{center}
|... -jobname "|\textit{target}|" "|[\textit{flags}]%
|\includeonly{|\textit{dest}|}\input{|\textit{main}|}"|
\end{center}
%

%%%%%%%%%%%%%%%%%%%%%%%%%%%%%%%%%%%%%%%%%%%%%%%%%%%%%%%%%%%%%%%%%%%%%%%%%%%%%%%%
%%%%%%%%%%%%%%%%%%%%%%%%%%%%%%%%%%%%%%%%%%%%%%%%%%%%%%%%%%%%%%%%%%%%%%%%%%%%%%%%
\section{Information}

%%%%%%%%%%%%%%%%%%%%%%%%%%%%%%%%%%%%%%%%%%%%%%%%%%%%%%%%%%%%%%%%%%%%%%%%%%%%%%%%
\subsection{Copyright}

Copyright \copyright{} 2017--2018 Niklas Beisert

This work may be distributed and/or modified under the
conditions of the \LaTeX{} Project Public License, either version 1.3
of this license or (at your option) any later version.
The latest version of this license is in
  \url{http://www.latex-project.org/lppl.txt}
and version 1.3 or later is part of all distributions of \LaTeX{}
version 2005/12/01 or later.

This work has the LPPL maintenance status `maintained'.

The Current Maintainer of this work is Niklas Beisert.

This work consists of the files |README.txt|, |childdoc.ins| and |childdoc.dtx|
as well as the derived files |childdoc.def|, |cdocsamp.tex|
with |cdocsch1.tex|, |cdocsch2.tex|, |cdocspt3.tex|, |cdocspt4.tex|,
|cdocsdrf.tex|, |cdocsfn1.tex|, |cdocsfn2.tex|
as well as |childdoc.pdf|.

%%%%%%%%%%%%%%%%%%%%%%%%%%%%%%%%%%%%%%%%%%%%%%%%%%%%%%%%%%%%%%%%%%%%%%%%%%%%%%%%
\subsection{Files and Installation}

The package consists of the files:
%
\begin{center}
\begin{tabular}{ll}
    |README.txt|   & readme file \\
    |childdoc.ins| & installation file \\
    |childdoc.dtx| & source file \\
    |childdoc.def| & definition file \\
    |cdocsamp.tex| & sample main file \\
    |cdocsch1.tex| & sample include file \\
    |cdocsch2.tex| & sample include file \\
    |cdocspt3.tex| & sample part file \\
    |cdocspt4.tex| & sample part file \\
    |cdocsdrf.tex| & sample redirection file \\
    |cdocsfn1.tex| & sample redirection file \\
    |cdocsfn2.tex| & sample redirection file \\
    |childdoc.pdf| & manual
\end{tabular}
\end{center}
%
The distribution consists of the files
|README.txt|, |childdoc.ins| and |childdoc.dtx|.
%
\begin{itemize}
\item
Run (pdf)\LaTeX{} on |childdoc.dtx|
to compile the manual |childdoc.pdf| (this file).
\item
Run \LaTeX{} on |childdoc.ins| to create the definitions file |childdoc.def|
and the sample |cdocsamp.tex| with include files
|cdocsch1.tex|, |cdocsch2.tex|, |cdocspt3.tex|, |cdocspt4.tex|,
|cdocsdrf.tex|, |cdocsfn1.tex|, |cdocsfn2.tex|.
Then copy the file |childdoc.def| to an appropriate directory of your \LaTeX{}
distribution, e.g.\ \textit{texmf-root}|/tex/latex/childdoc|.
\end{itemize}

%%%%%%%%%%%%%%%%%%%%%%%%%%%%%%%%%%%%%%%%%%%%%%%%%%%%%%%%%%%%%%%%%%%%%%%%%%%%%%%%
\subsection{Related CTAN Packages}

There are several other packages which offer a similar functionality:
%
\begin{itemize}
\item
The packages
\href{http://ctan.org/pkg/docmute}{\textsf{docmute}},
\href{http://ctan.org/pkg/includex}{\textsf{includex}} and
\href{http://ctan.org/pkg/standalone}{\textsf{standalone}}
provide commands to include only the document body of
a child file thus allowing both files to be compiled individually.
\item
The packages \href{http://ctan.org/pkg/subdocs}{\textsf{subdocs}}
and \href{http://ctan.org/pkg/subfiles}{\textsf{subfiles}}
provide structures in which the main and child documents can be
encapsulated and allowing them to be compiled individually.
The inclusion mechanism is different from the conventional |\include|.
\item
The package \href{http://ctan.org/pkg/combine}{\textsf{combine}}
is an elaborate solution to combine several documents into one.
\end{itemize}
%
See also the CTAN topic \href{http://ctan.org/topic/subdocs}{\textsf{subdocs}}
for further related packages.
The present package differs from the above solutions in that
a document structure constructed with the conventional |\include| mechanism
just needs two extra commands at the top of every file
such that all constituent files can be compiled individually.

%%%%%%%%%%%%%%%%%%%%%%%%%%%%%%%%%%%%%%%%%%%%%%%%%%%%%%%%%%%%%%%%%%%%%%%%%%%%%%%%
%\subsection{Feature Suggestions}
%
%The following is a list of features which may be useful for future
%versions of this package:
%%
%\begin{itemize}
%\item
%\ldots
%\end{itemize}

%%%%%%%%%%%%%%%%%%%%%%%%%%%%%%%%%%%%%%%%%%%%%%%%%%%%%%%%%%%%%%%%%%%%%%%%%%%%%%%%
\subsection{Revision History}

%%%%%%%%%%%%%%%%%%%%%%%%%%%%%%%%%%%%%%%%
\paragraph{v2.0:} 2018/12/30

\begin{itemize}
\item
immediate forward processing
\item
added |\childdocby| mechanism
\item
manual restructured
\end{itemize}

%%%%%%%%%%%%%%%%%%%%%%%%%%%%%%%%%%%%%%%%
\paragraph{v1.6:} 2018/01/17

\begin{itemize}
\item
application for development of include files
\item
corrections to manual
\end{itemize}

%%%%%%%%%%%%%%%%%%%%%%%%%%%%%%%%%%%%%%%%
\paragraph{v1.5:} 2017/05/21

\begin{itemize}
\item
more complete structuring introduced
\item
|\childdocof| introduced
\item
|\childdoc| renamed to |\childdocmain|
\item
|\childredirect| renamed to |\childdocforward| and |\childdocforwardprefix|
and functionality expanded
\end{itemize}

%%%%%%%%%%%%%%%%%%%%%%%%%%%%%%%%%%%%%%%%
\paragraph{v1.0:} 2017/04/27

\begin{itemize}
\item
manual and install package
\item
first version published on CTAN
\end{itemize}

%%%%%%%%%%%%%%%%%%%%%%%%%%%%%%%%%%%%%%%%
\paragraph{v0.6:} 2017/04/26

\begin{itemize}
\item
redirection mechanism added
\end{itemize}

%%%%%%%%%%%%%%%%%%%%%%%%%%%%%%%%%%%%%%%%
\paragraph{v0.5:} 2017/04/26

\begin{itemize}
\item
functionality in definition file
\end{itemize}


%%%%%%%%%%%%%%%%%%%%%%%%%%%%%%%%%%%%%%%%%%%%%%%%%%%%%%%%%%%%%%%%%%%%%%%%%%%%%%%%
%%%%%%%%%%%%%%%%%%%%%%%%%%%%%%%%%%%%%%%%%%%%%%%%%%%%%%%%%%%%%%%%%%%%%%%%%%%%%%%%
%%%%%%%%%%%%%%%%%%%%%%%%%%%%%%%%%%%%%%%%%%%%%%%%%%%%%%%%%%%%%%%%%%%%%%%%%%%%%%%%
\appendix

\settowidth\MacroIndent{\rmfamily\scriptsize 000\ }

 \DocInput{childdoc.dtx}

\end{document}
%</driver>
% \fi
%
% %%%%%%%%%%%%%%%%%%%%%%%%%%%%%%%%%%%%%%%%%%%%%%%%%%%%%%%%%%%%%%%%%%%%%%%%%%%%%%
% %%%%%%%%%%%%%%%%%%%%%%%%%%%%%%%%%%%%%%%%%%%%%%%%%%%%%%%%%%%%%%%%%%%%%%%%%%%%%%
% \section{Sample}
%\iffalse
%<*samplemain>
%\fi
%
% The following presents a sample document
% with two chapters, two parts, a title page,
% a compile flag as well as three forwarding files to set the flag.
% It consists of eight |.tex| files:
% \begin{center}
% \begin{tabular}{ll}
% |cdocsamp.tex|&main file\\
% |cdocsch1.tex|&include file for chapter 1\\
% |cdocsch2.tex|&include file for chapter 2\\
% |cdocspt3.tex|&include file for part 3\\
% |cdocspt4.tex|&include file for part 4\\
% |cdocsdrf.tex|&forwarding file for main file in draft mode\\
% |cdocsfi1.tex|&forwarding file for final version of chapter 1\\
% |cdocsfi2.tex|&forwarding file for final version of chapter 2\\
% \end{tabular}
% \end{center}
% Each of the eight files can be compiled directly by the \LaTeX{} compiler.
%
% %%%%%%%%%%%%%%%%%%%%%%%%%%%%%%%%%%%%%%
% \paragraph{Main File.}
%
% The main file is called |cdocsamp.tex|.
%
% Load the \textsf{childdoc} definitions and
% declare the filename for the main document:
%    \begin{macrocode}
\input{childdoc.def}
\childdocmain{}
%    \end{macrocode}

% Optional override for |\version| flag:
%    \begin{macrocode}
%%\ifchilddoc\else\providecommand{\version}{draft}\fi
%    \end{macrocode}

% Define the default values for the |\version| flag
% (|final| for the main file and |draft| for childs):
%    \begin{macrocode}
\ifchilddoc
\providecommand{\version}{draft}
\else
\providecommand{\version}{final}
\fi
%    \end{macrocode}

% Load the standard document class:
%    \begin{macrocode}
\documentclass[12pt]{article}
%    \end{macrocode}

% Start the document body:
%    \begin{macrocode}
\begin{document}
%    \end{macrocode}

% Declare a title page.
% Print title, part of document being processed and version flag:
%    \begin{macrocode}
\addtocounter{page}{-1}
\begin{center}
{\LARGE\bfseries{}childdoc example\par}
\vspace{1cm}
\ifchilddoc
\ifchilddocmanual part\else chapter\fi:
`\childdocname' of `\childdocjob'\par
\else
main document: `\childdocjob'\par
\fi
version: \version\par
\end{center}
\newpage
%    \end{macrocode}

% Manually include selected file,
% otherwise process as usual:
%    \begin{macrocode}
\ifchilddocmanual
\section*{part `\childdocname'}
\input{\childdocname}
\else
%    \end{macrocode}

% Include the two chapters:
%    \begin{macrocode}
\include{cdocsch1}
\include{cdocsch2}
%    \end{macrocode}

% Include the two parts unless only chapters should be displayed:
%    \begin{macrocode}
\ifchilddoc\else
\section{part three}
\input{cdocspt3}
\section{part four}
\input{cdocspt4}
\fi
%    \end{macrocode}

% Process as usual until here:
%    \begin{macrocode}
\fi
%    \end{macrocode}

% End of document body:
%    \begin{macrocode}
\end{document}
%    \end{macrocode}
%\iffalse
%</samplemain>
%\fi
%
% %%%%%%%%%%%%%%%%%%%%%%%%%%%%%%%%%%%%%%
% \paragraph{Chapter Include Files.}
%
% The include files are called |cdocsch1.tex| and |cdocsch2.tex|.
%
%\iffalse
%<*samplechap1|samplechap2>
%\fi

% Optional override for |\version| flag:
%    \begin{macrocode}
%%\providecommand{\version}{final}
%    \end{macrocode}

% Include the main document:
%    \begin{macrocode}
\input{childdoc.def}
\childdocof{cdocsamp}
%    \end{macrocode}

%\iffalse
%</samplechap1|samplechap2>
%\fi
%
%\iffalse
%<*samplechap1>
%\fi
% Some text for chapter 1:
%    \begin{macrocode}
\section{one}
some text in chapter one
%    \end{macrocode}

%\iffalse
%</samplechap1>
%\fi
% Some text for chapter 2:
%\iffalse
%<*samplechap2>
%\fi
%    \begin{macrocode}
\section{two}
more text in chapter two
%    \end{macrocode}

%\iffalse
%</samplechap2>
%\fi
%
% %%%%%%%%%%%%%%%%%%%%%%%%%%%%%%%%%%%%%%
% \paragraph{Part Include Files.}
%
% The include files are called |cdocspt3.tex| and |cdocspt4.tex|.
%
%\iffalse
%<*samplepart3|samplepart4>
%\fi

% Optional override for |\version| flag:
%    \begin{macrocode}
%%\providecommand{\version}{final}
%    \end{macrocode}

% Include the main document:
%    \begin{macrocode}
\input{childdoc.def}
\childdocby{cdocsamp}
%    \end{macrocode}

%\iffalse
%</samplepart3|samplepart4>
%\fi
%
%\iffalse
%<*samplepart3>
%\fi
% Some text for part 3:
%    \begin{macrocode}
some text in part three
%    \end{macrocode}

%\iffalse
%</samplepart3>
%\fi
% Some text for part 4:
%\iffalse
%<*samplepart4>
%\fi
%    \begin{macrocode}
more text in part four
%    \end{macrocode}

%\iffalse
%</samplepart4>
%\fi
%
% %%%%%%%%%%%%%%%%%%%%%%%%%%%%%%%%%%%%%%
% \paragraph{Forwarding for a Complete Draft.}
%
% The following forwarding file |cdocsdrf.tex|
% compiles the main document in draft mode:
%\iffalse
%<*sampledraft>
%\fi
%    \begin{macrocode}
\def\version{draft}
\input{childdoc.def}
\childdocforward{cdocsamp}
%    \end{macrocode}

%\iffalse
%</sampledraft>
%\fi
%
% %%%%%%%%%%%%%%%%%%%%%%%%%%%%%%%%%%%%%%
% \paragraph{Forwarding for Final Version of the Chapters.}
%
% The following forwarding files |cdocsfn1.tex| and |cdocsfn2.tex|
% (with identical content)
% compile the final versions of the child documents
% |cdocsch1.tex| and |cdocsch2.tex|, respectively:
%\iffalse
%<*samplefinal>
%\fi
%    \begin{macrocode}
\def\version{final}
\input{childdoc.def}
\childdocforwardprefix[cdocsamp]{cdocsfn}{cdocsch}
%    \end{macrocode}

%\iffalse
%</samplefinal>
%\fi
%
% %%%%%%%%%%%%%%%%%%%%%%%%%%%%%%%%%%%%%%
% \paragraph{Command Line Processing.}
%
% The following three command lines generate the output files
% |cdocscld|, |cdocscl1| and |cdocscl2|
% which should be identical to
% |cdocsdrf|, |cdocsch1| and |cdocsfn2|, respectively:
% \begin{center}
% \begin{tabular}{l}
% |latex -jobname cdocscld \|\\
% |  "\def\version{draft}\input{childdoc.def}\childdocforward{cdocsamp}"|\\
% |latex -jobname cdocscl1 \|\\
% |  "\input{childdoc.def}\childdocforward[cdocsamp]{cdocsch1}"|\\
% |latex -jobname cdocscl2 \|\\
% |  "\def\version{final}\input{childdoc.def}\childdocforward{cdocsch2}"|
% \end{tabular}
% \end{center}
% Note that the trailing backslash on each first line
% merely continues the input to the second line
% (for convenient cut ant paste).
% Furthermore, the command |latex| can be replaced by any
% of its alternative versions such as |pdflatex|.
%
% %%%%%%%%%%%%%%%%%%%%%%%%%%%%%%%%%%%%%%%%%%%%%%%%%%%%%%%%%%%%%%%%%%%%%%%%%%%%%%
% %%%%%%%%%%%%%%%%%%%%%%%%%%%%%%%%%%%%%%%%%%%%%%%%%%%%%%%%%%%%%%%%%%%%%%%%%%%%%%
% \section{Implementation}
%\iffalse
%<*package>
%\fi
%
% This section describes the definitions file |childdoc.def|.

% The definitions cannot be loaded using |\usepackage| or |\RequirePackage|
% which has a mechanism to prevent loading a style file more than once.
% When loading the definitions by means of |\input|
% multiple instances have to be prevented manually:
%\iffalse
%This code needs to be before the `\ProvidesFile' directive
%which is defined at the beginning of this file.
%Therefore it is also placed there and commented out here.
%</package>
%<*discard>
%\fi
%    \begin{macrocode}
\ifdefined\childdocmain\endinput\fi
%    \end{macrocode}
%\iffalse
%</discard>
%<*package>
%\fi
%
% \macro{\ifchilddoc}
% \macro{\ifchilddocmanual}
% The conditional |\ifchilddoc| tells whether a
% child (true) or main (false) document is being compiled.
% The conditional |\ifchilddocmanual| tells whether
% the |\includeonly| mechanism is used (false) or
% the selection of child files must be performed manually (true).
% The definitions initialise to false:
%    \begin{macrocode}
\newif\ifchilddoc
\newif\ifchilddocmanual
%    \end{macrocode}

% \macro{\childdocname}
% \macro{\childdocjob}
% The macro |\childdocname| stores the name of the main document
% to be compiled. The macro |\childdocjob| stores the name of
% the document on which the \LaTeX{} compiler was originally invoked.
% The content of |\jobname| cannot be compared
% to filenames specified in the source due to different catcodes.
% The following code rescans |\jobname|, stores the result
% in |\childdocname| and saves a copy in |\childdocjob|:
%    \begin{macrocode}
\edef\childdocname{\scantokens\expandafter{\jobname\noexpand}}
\let\childdocjob\childdocname
%    \end{macrocode}

% \macro{\childdocdisable}
% The macro |\childdocdisable| prevents the main file
% from being processed more than once.
% At this stage, the main document command |\childdocmain|
% is assumed to be called once again where it should do nothing.
% Any subsequent call to it should prevent
% a secondary processing of the main document
% It overwrites the forwarding commands
% |\childdocof| and |\childdocforward|
% with empty macros to prevent further inclusions of the main document:
%    \begin{macrocode}
\newcommand{\childdocdisable}
{
  \renewcommand{\childdocmain}[1]{\renewcommand{\childdocmain}[1]{\endinput}}
  \renewcommand{\childdocof}[1]{}
  \renewcommand{\childdocby}[2][]{}
  \renewcommand{\childdocforward}[2][]{}
  \renewcommand{\childdocdisable}{}
}
%    \end{macrocode}

% \macro{\childdocmain}
% The macro |\childdocmain| is to be called at the top of the main file
% with nothing or the main filename (without extension) as argument.
% First, it breaks loops.
% If the argument is not empty and does not match |\childdocname|
% (which is set by the first inclusion of |childdoc.def|),
% |\ifchilddoc| is set to true, |\includeonly| is applied to the child file
% and |\jobname| is set to the main file
% (for proper handling of |.aux| files):
%    \begin{macrocode}
\newcommand{\childdocmain}[1]
{
  \childdocdisable\childdocmain{}
  \if?#1?\else
    \begingroup
      \def\childdoctmp{#1}
      \ifx\childdoctmp\childdocname
        \def\childdoctmp{}
      \else
        \def\childdoctmp
        {
          \childdoctrue
          \includeonly{\childdocname}
          \def\childdocjob{#1}
          \def\jobname{#1}
        }
      \fi
      \expandafter
    \endgroup
    \childdoctmp
  \fi
}
%    \end{macrocode}

% \macro{\childdocof}
% The command |\childdocof| redirects
% compilation to the main file |#1|.
%    \begin{macrocode}
\newcommand{\childdocof}[1]
{
  \childdocdisable
  \childdoctrue
  \includeonly{\childdocname}
  \def\jobname{#1}
  \def\childdocjob{#1}
  \input{#1}
}
%    \end{macrocode}

% \macro{\childdocby}
% The command |\childdocby| ....
%    \begin{macrocode}
\newcommand{\childdocby}[2][]
{
  \childdocdisable
  \childdoctrue
  \childdocmanualtrue
  \if?#1?\else
    \def\jobname{#2}
  \fi
  \def\childdocjob{#2}
  \input{#2}
  \endinput
}
%    \end{macrocode}

% \macro{\childdocforward}
% The command |\childdocforward| redirects
% compilation to the main file or
% (if the optional argument is given) a child file.
% Parameters are set as if the main file
% or a child file starting with |\childdocof| was compiled.
% Then compilation is handed over to the main file:
%    \begin{macrocode}
\newcommand{\childdocforward}[2][]
{
  \begingroup
    \if?#1?
      \def\childdoctmp
      {
        \def\childdocname{#2}
        \def\childdocjob{#2}
        \def\jobname{#2}
        \input{#2}
        \endinput
      }
    \else
      \def\childdoctmp
      {
        \childdocdisable
        \def\childdocname{#2}
        \childdoctrue
        \includeonly{#2}
        \def\childdocjob{#1}
        \def\jobname{#1}
        \input{#1}
        \endinput
      }
    \fi
    \expandafter
  \endgroup
  \childdoctmp
}
%    \end{macrocode}

% \macro{\childdocforwardprefix}
% The command |\childdocforwardprefix| redirects
% compilation to the main or a child file by means of a pattern.
% The prefix |#1| in the current filename is replaced by |#2|
% and the suffix of the current filename is kept
% (it is assumed that the filename does not contain the substring `|~~~|'
% which is used as a delimiter).
% Compilation is handed over to the new file by |\childdocforward|:
%    \begin{macrocode}
\newcommand{\childdocforwardprefix}[3][]
{
  \begingroup
    \def\childdocextract #2##1~~~{\def\childdoctmp{\childdocforward[#1]{#3##1}}}
    \expandafter\childdocextract\childdocname~~~
    \expandafter
  \endgroup
  \childdoctmp
}
%    \end{macrocode}

% \macro{\childdoc}
% The deprecated macro |\childdoc| is a legacy version of |\childdocmain|:
%    \begin{macrocode}
\newcommand{\childdoc}{\childdocmain}
%    \end{macrocode}

% \macro{\childdocredirect}
% The deprecated macro |\childdocredirect| is a legacy version
% of |\childdocforward| and |\childdocforwardprefix|:
%    \begin{macrocode}
\newcommand{\childdocredirect}[2][]
{
  \begingroup
    \if?#1?
      \def\childdoctmp{\childdocforward{#2}}
    \else
      \def\childdoctmp{\childdocforwardprefix{#1}{#2}}
    \fi
    \expandafter
  \endgroup
  \childdoctmp
}
%    \end{macrocode}

%\iffalse
%</package>
%\fi
%
\endinput

\childdocforward{cdocsamp}
%    \end{macrocode}

%\iffalse
%</sampledraft>
%\fi
%
% %%%%%%%%%%%%%%%%%%%%%%%%%%%%%%%%%%%%%%
% \paragraph{Forwarding for Final Version of the Chapters.}
%
% The following forwarding files |cdocsfn1.tex| and |cdocsfn2.tex|
% (with identical content)
% compile the final versions of the child documents
% |cdocsch1.tex| and |cdocsch2.tex|, respectively:
%\iffalse
%<*samplefinal>
%\fi
%    \begin{macrocode}
\def\version{final}
% \iffalse
%
% childdoc.dtx Copyright (C) 2017-2018 Niklas Beisert
%
% This work may be distributed and/or modified under the
% conditions of the LaTeX Project Public License, either version 1.3
% of this license or (at your option) any later version.
% The latest version of this license is in
%   http://www.latex-project.org/lppl.txt
% and version 1.3 or later is part of all distributions of LaTeX
% version 2005/12/01 or later.
%
% This work has the LPPL maintenance status `maintained'.
%
% The Current Maintainer of this work is Niklas Beisert.
%
% This work consists of the files childdoc.dtx and childdoc.ins
% and the derived files childdoc.def and cdocsamp.tex with
% cdocsch1.tex, cdocsch2.tex, cdocsdrf.tex, cdocsfn1.tex, cdocsfn2.tex.
%
%<package>\ifdefined\childdocmain\endinput\fi
%<package>\ProvidesFile{childdoc.def}[2018/12/30 v2.0 child document driver]
%<samplemain>\ProvidesFile{cdocsamp.tex}[2018/12/30 v2.0 sample for childdoc]
%<*driver>
%\ProvidesFile{childdoc.drv}[2018/12/30 v2.0 childdoc reference manual file]
\PassOptionsToClass{10pt,a4paper}{article}
\documentclass{ltxdoc}

\usepackage[margin=35mm]{geometry}
\usepackage{hyperref}
\usepackage{hyperxmp}
\usepackage[usenames]{color}

\hypersetup{colorlinks=true}
\hypersetup{pdfstartview=FitH}
\hypersetup{pdfpagemode=UseNone}
\hypersetup{pdfsource={}}
\hypersetup{pdflang={en-UK}}
\hypersetup{pdfcopyright={Copyright 2017-2018 Niklas Beisert.
  This work may be distributed and/or modified under the
  conditions of the LaTeX Project Public License, either version 1.3
  of this license or (at your option) any later version.}}
\hypersetup{pdflicenseurl={http://www.latex-project.org/lppl.txt}}
\hypersetup{pdfcontactaddress={ETH Zurich, ITP, HIT K,
  Wolfgang-Pauli-Strasse 27}}
\hypersetup{pdfcontactpostcode={8093}}
\hypersetup{pdfcontactcity={Zurich}}
\hypersetup{pdfcontactcountry={Switzerland}}
\hypersetup{pdfcontactemail={nbeisert@itp.phys.ethz.ch}}
\hypersetup{pdfcontacturl={http://people.phys.ethz.ch/\xmptilde nbeisert/}}

\newcommand{\secref}[1]{\hyperref[#1]{section \ref*{#1}}}

\parskip1ex
\parindent0pt
\let\olditemize\itemize
\def\itemize{\olditemize\parskip0pt}

\begin{document}

\title{The \textsf{childdoc} Package}
\hypersetup{pdftitle={The childdoc Package}}
\author{Niklas Beisert\\[2ex]
  Institut f\"ur Theoretische Physik\\
  Eidgen\"ossische Technische Hochschule Z\"urich\\
  Wolfgang-Pauli-Strasse 27, 8093 Z\"urich, Switzerland\\[1ex]
  \href{mailto:nbeisert@itp.phys.ethz.ch}
  {\texttt{nbeisert@itp.phys.ethz.ch}}}
\hypersetup{pdfauthor={Niklas Beisert}}
\hypersetup{pdfsubject={Manual for the LaTeX2e Package childdoc}}
\date{30 December 2018, \textsf{v2.0}}
\maketitle

\begin{abstract}\noindent
\textsf{childdoc} is a \LaTeXe{} package
that enables the direct compilation
of document sections included by |\include|
to individual files.
\end{abstract}

\begingroup
\parskip0ex
\tableofcontents
\endgroup

%%%%%%%%%%%%%%%%%%%%%%%%%%%%%%%%%%%%%%%%%%%%%%%%%%%%%%%%%%%%%%%%%%%%%%%%%%%%%%%%
%%%%%%%%%%%%%%%%%%%%%%%%%%%%%%%%%%%%%%%%%%%%%%%%%%%%%%%%%%%%%%%%%%%%%%%%%%%%%%%%
\section{Introduction}

\LaTeX{} provides a mechanism to structure a large document (such as a book)
into a main file and several child files (containing the chapters)
using the |\include| command.
This mechanism is beneficial for documents
which span hundreds of pages in order to
make the source file(s) more manageable.
Moreover, compilation can be restricted to
selected child files by means of the |\includeonly| command.
The latter feature can be used to reduce the compilation time while editing
(this was significantly more useful in the earlier days of \LaTeX{})
or to generate a smaller document which is easier to navigate.
Another application of |\includeonly| is to generate
documents consisting of selected parts of the complete document.

However, there are a few drawbacks of the plain |\include| mechanism:
\begin{itemize}
\item
The child files cannot be compiled on their own,
they can only be compiled via the main file.
A naive editing environment
(such as a text editor with an option
to have the current file processed by \LaTeX)
may require one to switch to the main file before compiling;
attempting to compile the child file produces errors.
\item
The main file must be modified (each time)
to adjust the |\includeonly| command
to the present needs. This easily leaves the main file in a messy state.
\item
The generated document will always carry the filename
of the main document. This is inconvenient if
several child files are to be compiled and
to be kept for distribution.
\end{itemize}

The present package provides a simple interface
to make child files individually compilable by \LaTeX{}.
Compiling a child file then has the same effect as compiling
the main file with an |\includeonly| command
to select the appropriate child.
Moreover the generated document will carry the name of the child
rather than the main file.
This resolves all three above issues.

This feature is meant to make the editing of books,
thesis documents and lecture notes somewhat more convenient.
However, the package can also be used efficiently for
composing a series of documents (such as exercise sheets)
which are typically distributed individually.
It then assists the author in generating the individual documents
(potentially in different versions)
as well as a document containing the collected series.
Another application is in developing style files
or other kinds of included material
where compilation of the style file could redirect
to a sample or test file.

%%%%%%%%%%%%%%%%%%%%%%%%%%%%%%%%%%%%%%%%%%%%%%%%%%%%%%%%%%%%%%%%%%%%%%%%%%%%%%%%
%%%%%%%%%%%%%%%%%%%%%%%%%%%%%%%%%%%%%%%%%%%%%%%%%%%%%%%%%%%%%%%%%%%%%%%%%%%%%%%%
\section{Usage}

First of all, the package \textsf{childdoc} is \emph{not} a standard
\LaTeXe{} |.sty| style file! Therefore it needs to be invoked in
a non-standard way.

%%%%%%%%%%%%%%%%%%%%%%%%%%%%%%%%%%%%%%%%%%%%%%%%%%%%%%%%%%%%%%%%%%%%%%%%%%%%%%%%
\subsection{Included Files}
\label{sec:include}

%%%%%%%%%%%%%%%%%%%%%%%%%%%%%%%%%%%%%%%%
\DescribeMacro{\childdocmain}
To use the package, add the commands
\begin{center}
\begin{tabular}{l}
|\input{childdoc.def}|\\
|\childdocmain{}|\\
\end{tabular}
\end{center}
at the very top of the main \LaTeX{} file,
in particular \emph{before} the |\documentclass| statement!
The argument of |\childdocmain| should be left empty
(but it must be present).

%%%%%%%%%%%%%%%%%%%%%%%%%%%%%%%%%%%%%%%%
\DescribeMacro{\childdocof}
Furthermore, add the commands
\begin{center}
\begin{tabular}{l}
|\input{childdoc.def}|\\
|\childdocof{|\textit{main}|}|\\
\end{tabular}
\end{center}
at the top of every child file \textit{child}
which is included by |\include{|\textit{child}|}|
from within the main file
(or at least for those files to be compiled individually).
The argument \textit{main} must be the filename of the main file.

There are a couple of
considerations in setting up the main and child documents:

%%%%%%%%%%%%%%%%%%%%%%%%%%%%%%%%%%%%%%%%
\paragraph{Restrictions.}

Please note the following restrictions:
\begin{itemize}
\item
|\childdocmain| must be called with one argument \textit{main}
to ensure compatibility with earlier version of the package.
It must either be empty (|\childdocmain{}|)
or precisely match the filename of the main file in which it is specified.
See \secref{sec:detection} for further information.
\item
The filename \textit{main} must be specified without the |.tex| extension.
\item
The filename \textit{main} is case sensitive
(even in case-insensitive file systems)
due to internal string comparison.
\item
The argument \textit{main} should be fully expanded, it cannot be a macro.
\item
Subdirectories and special characters should be avoided in filenames.
\item
The command |\childdocmain{|\textit{main}|}| must be followed by a whitespace.
It should not be followed immediately by another command
or by a comment mark `|%|'.
This is because the \TeX{} parser reads the token immediately following
the argument of |\childdocmain| and puts it
at the beginning of every child section;
however, a white\-space is ignored.
\end{itemize}

%%%%%%%%%%%%%%%%%%%%%%%%%%%%%%%%%%%%%%%%
\paragraph{Content of Main File.}

It is advisable to place all content in the child files included by |\include|.
Any output contained in the main file will appear in all child documents
unless suppressed manually;
it cannot be suppressed automatically by the |\includeonly| directive
and thus should normally be avoided.
A method to include some content in the main file
by means of conditional processing is described in \secref{sec:conditional}.

%%%%%%%%%%%%%%%%%%%%%%%%%%%%%%%%%%%%%%%%
\paragraph{Page Numbering.}

When only a part of the document is compiled,
the appropriate numbering of pages
(as well as other status parameters)
is determined from the |.aux| files.
The latter contain information from previous passes.
However this information needs to propagate through
all intermediate child documents.
Therefore the page numbering in child documents may well
be inconsistent until the complete document is compiled at least once.

A useful (if unconventional) way to always ensure a consistent
page numbering is to restart the numbering in each child document
and denote the pages by `\textit{child}|.|\textit{page}'
where \textit{child} represents the chapter/section number of the child file.
This can be achieved by the command
|\numberwithin{page}{|\textit{child}|}|
of the \textsf{amsmath} package
where \textit{child} can be |chapter| or |section|
depending on the chosen structuring.
Alternatively, one can modify the macro |\thepage| appropriately
and reset the counter |page| at the start of each child file.

%%%%%%%%%%%%%%%%%%%%%%%%%%%%%%%%%%%%%%%%%%%%%%%%%%%%%%%%%%%%%%%%%%%%%%%%%%%%%%%%
\subsection{Conditional Processing}
\label{sec:conditional}

The package provides a mechanism to compile different versions
of a document. To customise the versions further some conditional processing
can come in handy to distinguish which version is being compiled.
The package provides two macros to describe the compilation context:

%%%%%%%%%%%%%%%%%%%%%%%%%%%%%%%%%%%%%%%%
\DescribeMacro{\ifchilddoc}
The conditional |\ifchilddoc| distinguishes between the compilation of
child documents and the main document:
%
\begin{center}
|\ifchilddoc |\textit{child-code}| |[|\||else |\textit{main-code}]| \||fi|
\end{center}

%%%%%%%%%%%%%%%%%%%%%%%%%%%%%%%%%%%%%%%%
\DescribeMacro{\childdocname}
\DescribeMacro{\childdocjob}
The macro |\childdocname| contains the filename (without extension)
of the main or child file being processed.
Note that |\childdocjob| will always contain the name of the main file.

%%%%%%%%%%%%%%%%%%%%%%%%%%%%%%%%%%%%%%%%
\paragraph{Title Page.}

Conditional processing can be used to include a title or banner page
in the main document when proper precautions are taken.
Importantly, the code in the main file should ensure that the page counter
(as well as other status parameters which are stored in the |.aux| files)
takes the same value after the conditional processing.
Otherwise the page numbers may take divergent values
depending on which part is compiled.

For example, a title page could be declared by:
%
\begin{center}
\begin{tabular}{l}
|\ifchilddoc\||else|\\
|\addtocounter{page}{-1}|\\
\textit{code for title page}\\
|\newpage|\\
|\||fi|
\end{tabular}
\end{center}
%
A banner page for the child documents can be generated by:
%
\begin{center}
\begin{tabular}{l}
|\ifchilddoc|\\
|\addtocounter{page}{-1}|\\
\textit{code for banner page}\\
|\newpage|\\
|\||fi|
\end{tabular}
\end{center}
%
Here one could write a message such as:
\begin{center}
|This is the part \childdocname{} of \childdocjob{}.|
\end{center}

%%%%%%%%%%%%%%%%%%%%%%%%%%%%%%%%%%%%%%%%%%%%%%%%%%%%%%%%%%%%%%%%%%%%%%%%%%%%%%%%
\subsection{Flags}
\label{sec:flags}

The package makes it easy to generate different versions
of the main or child documents.
To this end compilation flags can be defined
and assigned different default values.
They will be particularly useful in conjunction
with the forwarding mechanism described in \secref{sec:forward}.

For example, it may be useful to have a flag |\version|
which can be set to |draft| or |final|.
The document source will contain some conditional code
depending on the value of |\version|.
Suppose further, the flag should default to |final| for the main file
and to |draft| for child files
which is a natural assignment for editing the document.
This is achieved by placing the following code
in the preamble of the main document
(below the |\childdocmain| directive):
%
\begin{center}
\begin{tabular}{l}
|\ifchilddoc|\\
|\providecommand{\version}{draft}|\\
|\||else|\\
|\providecommand{\version}{final}|\\
|\||fi|
\end{tabular}
\end{center}
%
The definition by |\providecommand| makes sure
that previous definitions are not overwritten.
Further statements |\providecommand{\version}{...}|
can thus be added before the above code to override it.

For the main file, one might add a line
(between |\childdocmain| and the above block)
%
\begin{center}
|%\ifchilddoc\||else\providecommand{\version}{draft}\||fi|
\end{center}
%
which can be uncommented to produce a draft version.
Likewise one can add a line to the very top of a child file
(above the |\childdocof{|\textit{main}|}| directive)
%
\begin{center}
|%\providecommand{\version}{final}|
\end{center}
%
which can be uncommented to produce the final version of this child document.

%%%%%%%%%%%%%%%%%%%%%%%%%%%%%%%%%%%%%%%%%%%%%%%%%%%%%%%%%%%%%%%%%%%%%%%%%%%%%%%%
\subsection{Forwarding}
\label{sec:forward}

Different versions of the main or child documents
using compilation flags as described in \secref{sec:flags}
can be (permanently) stored in different files
for convenient compilation, viewing and distribution.
To this end, the package defines a command
to pass on compilation to a different file:

%%%%%%%%%%%%%%%%%%%%%%%%%%%%%%%%%%%%%%%%
\DescribeMacro{\childdocforward}
The command |\childdocforward| redirects processing to
another source file:
%
\begin{center}
\begin{tabular}{l}
|\input{childdoc.def}|\\
|\childdocforward[|\textit{main}|]{|\textit{dest}|}|\\
\end{tabular}
\end{center}
%
The argument \textit{dest} is the destination file
(without extension).
It should be the main file or one of the child files.
Note that further \textsf{childdoc} directives
such as |\childdocof| and |\childdocforward|
in the indicated file will be processed in this form.
The optional argument \textit{main}
passes on directly to the main file \textit{main}
while pretending to compile the child \textit{dest}.
This form behaves as if \textit{dest}
issues |\childdocof{|\textit{main}|}| right away,
and no further \textsf{childdoc} directives will be processed.

%%%%%%%%%%%%%%%%%%%%%%%%%%%%%%%%%%%%%%%%
\DescribeMacro{\...prefix}
In the alternative form |\childdocforwardprefix|,
%
\begin{center}
\begin{tabular}{l}
|\input{childdoc.def}|\\
|\childdocforwardprefix[|\textit{main}|]{|\textit{prefix}|}{|\textit{dest}|}|
\end{tabular}
\end{center}
%
the destination file is determined by a pattern
depending on the current file:
To make this work, the current file must be called
`{\textit{prefix}\hspace{0.2em}\textit{suffix}}'
with \textit{prefix} matching precisely the argument.
Processing is then passed on to the file
`{\textit{dest}\hspace{0.2em}\textit{suffix}}'.
Surely, the same effect is achieved by
directly specifying the
argument `{\textit{dest}\hspace{0.2em}\textit{suffix}}'
in the first form.
However, that requires to set up a different file
for each child. With the alternative form of the command
all these files can have exactly the same content
which simplifies setting them up and maintaining them.

For example, the following file |draft.tex|
with a compilation flag |\version| as described in \secref{sec:flags}
compiles the main document as a draft:
%
\begin{center}
\begin{tabular}{l}
|\def\version{draft}|\\
|\input{childdoc.def}|\\
|\childdocforward{|\textit{main}|}|
\end{tabular}
\end{center}
%
Likewise, the following files |final|\textit{nn}|.tex|
compile the final version of the child document
|child|\textit{nn}|.tex|:
%
\begin{center}
\begin{tabular}{l}
|\def\version{final}|\\
|\input{childdoc.def}|\\
|\childdocforwardprefix{final}{child}|
\end{tabular}
\end{center}
%

Note that when several versions of a main file and/or of each child file
are to be generated, it may be convenient to set up a |Makefile| or
shell script to automatise the process.

%%%%%%%%%%%%%%%%%%%%%%%%%%%%%%%%%%%%%%%%%%%%%%%%%%%%%%%%%%%%%%%%%%%%%%%%%%%%%%%%
\subsection{Command Line Processing}
\label{sec:commandline}

The effect of redirection files can also be achieved by invoking
the \LaTeX{} compiler with a more elaborate command line.
Most conveniently this should be done as part
of a shell script or a |Makefile|.

When using \textsf{childdoc} in the main file, the following
command lines effectively perform a redirection
(note that depending on the shell being used,
backslashes may have to be doubled: `|\|' $\to$ `|\\|'):
%
\begin{center}
|... -jobname "|\textit{target}|" |\\|"|[\textit{flags}]%
|\input{childdoc.def}\childdocforward[|\textit{main}|]{|\textit{dest}|}"|
\end{center}
%
Here \textit{target} is the name of the output file,
\textit{main} is the name of the main file
and \textit{dest} is the name of the main or child file to be processed
(all filenames without extensions).
The optional argument \textit{main} can be omitted
if \textit{main} matches \textit{dest}.
Optionally, compilation \textit{flags} can be defined via |\def| commands.
This command line makes the \TeX{} engine believe
it is compiling the file \textit{target}
whose content is specified as the latter parameter.
The provided code then forwards the processing to
\textit{main} or \textit{dest} as described in \secref{sec:forward}.

%%%%%%%%%%%%%%%%%%%%%%%%%%%%%%%%%%%%%%%%%%%%%%%%%%%%%%%%%%%%%%%%%%%%%%%%%%%%%%%%
\subsection{Include by Input}
\label{sec:input}

Including child documents by |\include| has some restrictions by design.
Most notably, the content of a child document always occupies
its own set of pages; pages cannot be shared between child documents.
Usually, this behaviour makes perfect sense
because each child document contain an essential part of the document.
However, in some situations it may be desirable to compose
a document from a collection of parts
without having mandatory page breaks between then.
For this case, the package
provides a mechanism to include parts
by |\input| which can also be processed individually.
However, by construction this mechanism
requires manual handling of the content to be output.

%%%%%%%%%%%%%%%%%%%%%%%%%%%%%%%%%%%%%%%%
\DescribeMacro{\ifchilddocmanual}
The main file should be prepared as usual, see \secref{sec:include}.
However, the document body must make a distinction
between processing of an individual part and of the main document, e.g.:
%
\begin{center}
\begin{tabular}{l}
|\ifchilddocmanual|\\
|\input{\childdocname}|\\
|\||else|\\
\textit{document body with }|\input{|\textit{part}|}|\\
|\||fi|
\end{tabular}
\end{center}
%
The conditional |\ifchilddocmanual| is true whenever
a part to be included by |\input| is being compiled,
and the name of the part is stored in |\childdocname|.

%%%%%%%%%%%%%%%%%%%%%%%%%%%%%%%%%%%%%%%%
\DescribeMacro{\childdocby}
Each part to be included by |\input| should start with:
%
\begin{center}
\begin{tabular}{l}
|\input{childdoc.def}|\\
|\childdocby{|\textit{main}|}|\\
\end{tabular}
\end{center}
%
The directive |\childdocby| is similar to |\childdocof|
described in \secref{sec:include},
but the subsequent selection of content must be done manually.
To that end, both |\ifchilddoc| and |\ifchilddocmanual|
will be true upon processing of a part,
and the name of the part is stored in |\childdocname|.
Note that |\jobname| will be set to the filename of the current part
so that each part receives an individual |.aux| file
that does not interfere with the |.aux| file(s) of the main document.
This behaviour can be altered by the alternative form
|\childdocby[*]{|\textit{main}|}| (with a non-empty optional argument)
which uses the |.aux| file of the main document
by setting |\jobname| to \textit{main}.

%%%%%%%%%%%%%%%%%%%%%%%%%%%%%%%%%%%%%%%%%%%%%%%%%%%%%%%%%%%%%%%%%%%%%%%%%%%%%%%%
\subsection{Driver Development}
\label{sec:driver}

The \textsf{childdoc} mechanism can also be use for the development
of definition files such as \LaTeX{} styles or classes.
This case differs from the above setup with multiple parts
included by |\include| in that no |\includeonly| should be invoked.
This can be achieved by starting the include file
(before |\ProvidesPackage|) with:
%
\begin{center}
\begin{tabular}{l}
|\input{childdoc.def}|\\
|\childdocforward{|\textit{main}|}|\\
\end{tabular}
\end{center}
%
or alternatively with:
%
\begin{center}
\begin{tabular}{l}
|\input{childdoc.def}|\\
|\childdocby{|\textit{main}|}|\\
\end{tabular}
\end{center}
%
Both forms have slightly different effects as described above.
The main file is prepared as usual, see \secref{sec:include}.

%%%%%%%%%%%%%%%%%%%%%%%%%%%%%%%%%%%%%%%%%%%%%%%%%%%%%%%%%%%%%%%%%%%%%%%%%%%%%%%%
\subsection{Legacy Detection}
\label{sec:detection}

The directive |\childdocmain| in the main file can detect
whether the complete document or merely a child is to be compiled
even without using the directive |\childdocof|.
This method is deprecated because it is less robust
and there is no compelling reason to use it;
it is merely provided for backward compatibility
and it may be removed in future versions.

If the detection mechanism is to be used,
it is mandatory to correctly specify
the filename of the main file as the argument of |\childdocmain|:
%
\begin{center}
\begin{tabular}{l}
|\input{childdoc.def}|\\
|\childdocmain{|\textit{main}|}|\\
\end{tabular}
\end{center}
%
If |\jobname| does not match the argument \textit{main} of |\childdocmain|,
it is assumed that |\jobname| points to the child file to be compiled.
When using |\childdocmain| with the main file specified as argument,
it suffices to start a child file
with just |\input{|\textit{main}|}|
without loading of the package and using |\childdocof|.
If instead all processing is done
with the appropriate \textsf{childdoc} directives,
the argument of \textit{main} of |\childdocmain| can be empty.

An alternative version of the command line processing described
in \secref{sec:commandline} using the detection mechanism reads:
%
\begin{center}
|... -jobname "|\textit{target}|" "|[\textit{flags}]%
[|\def\jobname{|\textit{dest}|}|]|\input{|\textit{main}|}"|
\end{center}

%%%%%%%%%%%%%%%%%%%%%%%%%%%%%%%%%%%%%%%%%%%%%%%%%%%%%%%%%%%%%%%%%%%%%%%%%%%%%%%%
\subsection{Manual Code}
\label{sec:manual}

In case one cannot be certain whether the definitions file |childdoc.def|
is installed on the target \TeX{} distribution
and one prefers not to ship it,
it is conceivable to paste a few relevant commands into the sources.

To that end, drop all statements |\input{childdoc.def}|
and perform the replacements as outlined below.
Instead of |\childdocmain{|\textit{main}|}| add the following code
to the top of the main file:
%
\begin{center}
\begin{tabular}{l}
|\||ifdefined\childdocname\endinput\||fi\newif\ifchilddoc|\\
|\edef\childdocname{\scantokens\expandafter{\jobname\noexpand}}|\\
|\def\childdocmain{|\textit{main}|}\||ifx\childdocmain\childdocname\||else|\\
|\childdoctrue\includeonly{\childdocname}\let\jobname\childdocmain\||fi|\\
\end{tabular}
\end{center}
%
Instead of |\childdocof{|\textit{main}|}| just include the main file
at the top of each child file:
%
\begin{center}
|\input{|\textit{main}|}|
\end{center}
%
A simple redirection |\childdocforward{|\textit{dest}|}| is achieved by:
%
\begin{center}
|\def\jobname{|\textit{dest}|}\input{\jobname}|
\end{center}
%
The redirection with prefix
|\childdocforwardprefix[|\textit{prefix}|]{|\textit{dest}|}|
is accomplished by:
%
\begin{center}
\begin{tabular}{l}
|{\edef\jobname{\scantokens\expandafter{\jobname\noexpand}}|\\
|\def\redirectjob |\textit{prefix}|#1~~~{\gdef\jobname{|\textit{dest}|#1}}|\\
|\expandafter\redirectjob\jobname~~~}\input{\jobname}|
\end{tabular}
\end{center}

In an alternative approach,
child documents can be compiled by a specific command line
without additional code or specific definitions:
%
\begin{center}
|... -jobname "|\textit{target}|" "|[\textit{flags}]%
|\includeonly{|\textit{dest}|}\input{|\textit{main}|}"|
\end{center}
%

%%%%%%%%%%%%%%%%%%%%%%%%%%%%%%%%%%%%%%%%%%%%%%%%%%%%%%%%%%%%%%%%%%%%%%%%%%%%%%%%
%%%%%%%%%%%%%%%%%%%%%%%%%%%%%%%%%%%%%%%%%%%%%%%%%%%%%%%%%%%%%%%%%%%%%%%%%%%%%%%%
\section{Information}

%%%%%%%%%%%%%%%%%%%%%%%%%%%%%%%%%%%%%%%%%%%%%%%%%%%%%%%%%%%%%%%%%%%%%%%%%%%%%%%%
\subsection{Copyright}

Copyright \copyright{} 2017--2018 Niklas Beisert

This work may be distributed and/or modified under the
conditions of the \LaTeX{} Project Public License, either version 1.3
of this license or (at your option) any later version.
The latest version of this license is in
  \url{http://www.latex-project.org/lppl.txt}
and version 1.3 or later is part of all distributions of \LaTeX{}
version 2005/12/01 or later.

This work has the LPPL maintenance status `maintained'.

The Current Maintainer of this work is Niklas Beisert.

This work consists of the files |README.txt|, |childdoc.ins| and |childdoc.dtx|
as well as the derived files |childdoc.def|, |cdocsamp.tex|
with |cdocsch1.tex|, |cdocsch2.tex|, |cdocspt3.tex|, |cdocspt4.tex|,
|cdocsdrf.tex|, |cdocsfn1.tex|, |cdocsfn2.tex|
as well as |childdoc.pdf|.

%%%%%%%%%%%%%%%%%%%%%%%%%%%%%%%%%%%%%%%%%%%%%%%%%%%%%%%%%%%%%%%%%%%%%%%%%%%%%%%%
\subsection{Files and Installation}

The package consists of the files:
%
\begin{center}
\begin{tabular}{ll}
    |README.txt|   & readme file \\
    |childdoc.ins| & installation file \\
    |childdoc.dtx| & source file \\
    |childdoc.def| & definition file \\
    |cdocsamp.tex| & sample main file \\
    |cdocsch1.tex| & sample include file \\
    |cdocsch2.tex| & sample include file \\
    |cdocspt3.tex| & sample part file \\
    |cdocspt4.tex| & sample part file \\
    |cdocsdrf.tex| & sample redirection file \\
    |cdocsfn1.tex| & sample redirection file \\
    |cdocsfn2.tex| & sample redirection file \\
    |childdoc.pdf| & manual
\end{tabular}
\end{center}
%
The distribution consists of the files
|README.txt|, |childdoc.ins| and |childdoc.dtx|.
%
\begin{itemize}
\item
Run (pdf)\LaTeX{} on |childdoc.dtx|
to compile the manual |childdoc.pdf| (this file).
\item
Run \LaTeX{} on |childdoc.ins| to create the definitions file |childdoc.def|
and the sample |cdocsamp.tex| with include files
|cdocsch1.tex|, |cdocsch2.tex|, |cdocspt3.tex|, |cdocspt4.tex|,
|cdocsdrf.tex|, |cdocsfn1.tex|, |cdocsfn2.tex|.
Then copy the file |childdoc.def| to an appropriate directory of your \LaTeX{}
distribution, e.g.\ \textit{texmf-root}|/tex/latex/childdoc|.
\end{itemize}

%%%%%%%%%%%%%%%%%%%%%%%%%%%%%%%%%%%%%%%%%%%%%%%%%%%%%%%%%%%%%%%%%%%%%%%%%%%%%%%%
\subsection{Related CTAN Packages}

There are several other packages which offer a similar functionality:
%
\begin{itemize}
\item
The packages
\href{http://ctan.org/pkg/docmute}{\textsf{docmute}},
\href{http://ctan.org/pkg/includex}{\textsf{includex}} and
\href{http://ctan.org/pkg/standalone}{\textsf{standalone}}
provide commands to include only the document body of
a child file thus allowing both files to be compiled individually.
\item
The packages \href{http://ctan.org/pkg/subdocs}{\textsf{subdocs}}
and \href{http://ctan.org/pkg/subfiles}{\textsf{subfiles}}
provide structures in which the main and child documents can be
encapsulated and allowing them to be compiled individually.
The inclusion mechanism is different from the conventional |\include|.
\item
The package \href{http://ctan.org/pkg/combine}{\textsf{combine}}
is an elaborate solution to combine several documents into one.
\end{itemize}
%
See also the CTAN topic \href{http://ctan.org/topic/subdocs}{\textsf{subdocs}}
for further related packages.
The present package differs from the above solutions in that
a document structure constructed with the conventional |\include| mechanism
just needs two extra commands at the top of every file
such that all constituent files can be compiled individually.

%%%%%%%%%%%%%%%%%%%%%%%%%%%%%%%%%%%%%%%%%%%%%%%%%%%%%%%%%%%%%%%%%%%%%%%%%%%%%%%%
%\subsection{Feature Suggestions}
%
%The following is a list of features which may be useful for future
%versions of this package:
%%
%\begin{itemize}
%\item
%\ldots
%\end{itemize}

%%%%%%%%%%%%%%%%%%%%%%%%%%%%%%%%%%%%%%%%%%%%%%%%%%%%%%%%%%%%%%%%%%%%%%%%%%%%%%%%
\subsection{Revision History}

%%%%%%%%%%%%%%%%%%%%%%%%%%%%%%%%%%%%%%%%
\paragraph{v2.0:} 2018/12/30

\begin{itemize}
\item
immediate forward processing
\item
added |\childdocby| mechanism
\item
manual restructured
\end{itemize}

%%%%%%%%%%%%%%%%%%%%%%%%%%%%%%%%%%%%%%%%
\paragraph{v1.6:} 2018/01/17

\begin{itemize}
\item
application for development of include files
\item
corrections to manual
\end{itemize}

%%%%%%%%%%%%%%%%%%%%%%%%%%%%%%%%%%%%%%%%
\paragraph{v1.5:} 2017/05/21

\begin{itemize}
\item
more complete structuring introduced
\item
|\childdocof| introduced
\item
|\childdoc| renamed to |\childdocmain|
\item
|\childredirect| renamed to |\childdocforward| and |\childdocforwardprefix|
and functionality expanded
\end{itemize}

%%%%%%%%%%%%%%%%%%%%%%%%%%%%%%%%%%%%%%%%
\paragraph{v1.0:} 2017/04/27

\begin{itemize}
\item
manual and install package
\item
first version published on CTAN
\end{itemize}

%%%%%%%%%%%%%%%%%%%%%%%%%%%%%%%%%%%%%%%%
\paragraph{v0.6:} 2017/04/26

\begin{itemize}
\item
redirection mechanism added
\end{itemize}

%%%%%%%%%%%%%%%%%%%%%%%%%%%%%%%%%%%%%%%%
\paragraph{v0.5:} 2017/04/26

\begin{itemize}
\item
functionality in definition file
\end{itemize}


%%%%%%%%%%%%%%%%%%%%%%%%%%%%%%%%%%%%%%%%%%%%%%%%%%%%%%%%%%%%%%%%%%%%%%%%%%%%%%%%
%%%%%%%%%%%%%%%%%%%%%%%%%%%%%%%%%%%%%%%%%%%%%%%%%%%%%%%%%%%%%%%%%%%%%%%%%%%%%%%%
%%%%%%%%%%%%%%%%%%%%%%%%%%%%%%%%%%%%%%%%%%%%%%%%%%%%%%%%%%%%%%%%%%%%%%%%%%%%%%%%
\appendix

\settowidth\MacroIndent{\rmfamily\scriptsize 000\ }

 \DocInput{childdoc.dtx}

\end{document}
%</driver>
% \fi
%
% %%%%%%%%%%%%%%%%%%%%%%%%%%%%%%%%%%%%%%%%%%%%%%%%%%%%%%%%%%%%%%%%%%%%%%%%%%%%%%
% %%%%%%%%%%%%%%%%%%%%%%%%%%%%%%%%%%%%%%%%%%%%%%%%%%%%%%%%%%%%%%%%%%%%%%%%%%%%%%
% \section{Sample}
%\iffalse
%<*samplemain>
%\fi
%
% The following presents a sample document
% with two chapters, two parts, a title page,
% a compile flag as well as three forwarding files to set the flag.
% It consists of eight |.tex| files:
% \begin{center}
% \begin{tabular}{ll}
% |cdocsamp.tex|&main file\\
% |cdocsch1.tex|&include file for chapter 1\\
% |cdocsch2.tex|&include file for chapter 2\\
% |cdocspt3.tex|&include file for part 3\\
% |cdocspt4.tex|&include file for part 4\\
% |cdocsdrf.tex|&forwarding file for main file in draft mode\\
% |cdocsfi1.tex|&forwarding file for final version of chapter 1\\
% |cdocsfi2.tex|&forwarding file for final version of chapter 2\\
% \end{tabular}
% \end{center}
% Each of the eight files can be compiled directly by the \LaTeX{} compiler.
%
% %%%%%%%%%%%%%%%%%%%%%%%%%%%%%%%%%%%%%%
% \paragraph{Main File.}
%
% The main file is called |cdocsamp.tex|.
%
% Load the \textsf{childdoc} definitions and
% declare the filename for the main document:
%    \begin{macrocode}
\input{childdoc.def}
\childdocmain{}
%    \end{macrocode}

% Optional override for |\version| flag:
%    \begin{macrocode}
%%\ifchilddoc\else\providecommand{\version}{draft}\fi
%    \end{macrocode}

% Define the default values for the |\version| flag
% (|final| for the main file and |draft| for childs):
%    \begin{macrocode}
\ifchilddoc
\providecommand{\version}{draft}
\else
\providecommand{\version}{final}
\fi
%    \end{macrocode}

% Load the standard document class:
%    \begin{macrocode}
\documentclass[12pt]{article}
%    \end{macrocode}

% Start the document body:
%    \begin{macrocode}
\begin{document}
%    \end{macrocode}

% Declare a title page.
% Print title, part of document being processed and version flag:
%    \begin{macrocode}
\addtocounter{page}{-1}
\begin{center}
{\LARGE\bfseries{}childdoc example\par}
\vspace{1cm}
\ifchilddoc
\ifchilddocmanual part\else chapter\fi:
`\childdocname' of `\childdocjob'\par
\else
main document: `\childdocjob'\par
\fi
version: \version\par
\end{center}
\newpage
%    \end{macrocode}

% Manually include selected file,
% otherwise process as usual:
%    \begin{macrocode}
\ifchilddocmanual
\section*{part `\childdocname'}
\input{\childdocname}
\else
%    \end{macrocode}

% Include the two chapters:
%    \begin{macrocode}
\include{cdocsch1}
\include{cdocsch2}
%    \end{macrocode}

% Include the two parts unless only chapters should be displayed:
%    \begin{macrocode}
\ifchilddoc\else
\section{part three}
\input{cdocspt3}
\section{part four}
\input{cdocspt4}
\fi
%    \end{macrocode}

% Process as usual until here:
%    \begin{macrocode}
\fi
%    \end{macrocode}

% End of document body:
%    \begin{macrocode}
\end{document}
%    \end{macrocode}
%\iffalse
%</samplemain>
%\fi
%
% %%%%%%%%%%%%%%%%%%%%%%%%%%%%%%%%%%%%%%
% \paragraph{Chapter Include Files.}
%
% The include files are called |cdocsch1.tex| and |cdocsch2.tex|.
%
%\iffalse
%<*samplechap1|samplechap2>
%\fi

% Optional override for |\version| flag:
%    \begin{macrocode}
%%\providecommand{\version}{final}
%    \end{macrocode}

% Include the main document:
%    \begin{macrocode}
\input{childdoc.def}
\childdocof{cdocsamp}
%    \end{macrocode}

%\iffalse
%</samplechap1|samplechap2>
%\fi
%
%\iffalse
%<*samplechap1>
%\fi
% Some text for chapter 1:
%    \begin{macrocode}
\section{one}
some text in chapter one
%    \end{macrocode}

%\iffalse
%</samplechap1>
%\fi
% Some text for chapter 2:
%\iffalse
%<*samplechap2>
%\fi
%    \begin{macrocode}
\section{two}
more text in chapter two
%    \end{macrocode}

%\iffalse
%</samplechap2>
%\fi
%
% %%%%%%%%%%%%%%%%%%%%%%%%%%%%%%%%%%%%%%
% \paragraph{Part Include Files.}
%
% The include files are called |cdocspt3.tex| and |cdocspt4.tex|.
%
%\iffalse
%<*samplepart3|samplepart4>
%\fi

% Optional override for |\version| flag:
%    \begin{macrocode}
%%\providecommand{\version}{final}
%    \end{macrocode}

% Include the main document:
%    \begin{macrocode}
\input{childdoc.def}
\childdocby{cdocsamp}
%    \end{macrocode}

%\iffalse
%</samplepart3|samplepart4>
%\fi
%
%\iffalse
%<*samplepart3>
%\fi
% Some text for part 3:
%    \begin{macrocode}
some text in part three
%    \end{macrocode}

%\iffalse
%</samplepart3>
%\fi
% Some text for part 4:
%\iffalse
%<*samplepart4>
%\fi
%    \begin{macrocode}
more text in part four
%    \end{macrocode}

%\iffalse
%</samplepart4>
%\fi
%
% %%%%%%%%%%%%%%%%%%%%%%%%%%%%%%%%%%%%%%
% \paragraph{Forwarding for a Complete Draft.}
%
% The following forwarding file |cdocsdrf.tex|
% compiles the main document in draft mode:
%\iffalse
%<*sampledraft>
%\fi
%    \begin{macrocode}
\def\version{draft}
\input{childdoc.def}
\childdocforward{cdocsamp}
%    \end{macrocode}

%\iffalse
%</sampledraft>
%\fi
%
% %%%%%%%%%%%%%%%%%%%%%%%%%%%%%%%%%%%%%%
% \paragraph{Forwarding for Final Version of the Chapters.}
%
% The following forwarding files |cdocsfn1.tex| and |cdocsfn2.tex|
% (with identical content)
% compile the final versions of the child documents
% |cdocsch1.tex| and |cdocsch2.tex|, respectively:
%\iffalse
%<*samplefinal>
%\fi
%    \begin{macrocode}
\def\version{final}
\input{childdoc.def}
\childdocforwardprefix[cdocsamp]{cdocsfn}{cdocsch}
%    \end{macrocode}

%\iffalse
%</samplefinal>
%\fi
%
% %%%%%%%%%%%%%%%%%%%%%%%%%%%%%%%%%%%%%%
% \paragraph{Command Line Processing.}
%
% The following three command lines generate the output files
% |cdocscld|, |cdocscl1| and |cdocscl2|
% which should be identical to
% |cdocsdrf|, |cdocsch1| and |cdocsfn2|, respectively:
% \begin{center}
% \begin{tabular}{l}
% |latex -jobname cdocscld \|\\
% |  "\def\version{draft}\input{childdoc.def}\childdocforward{cdocsamp}"|\\
% |latex -jobname cdocscl1 \|\\
% |  "\input{childdoc.def}\childdocforward[cdocsamp]{cdocsch1}"|\\
% |latex -jobname cdocscl2 \|\\
% |  "\def\version{final}\input{childdoc.def}\childdocforward{cdocsch2}"|
% \end{tabular}
% \end{center}
% Note that the trailing backslash on each first line
% merely continues the input to the second line
% (for convenient cut ant paste).
% Furthermore, the command |latex| can be replaced by any
% of its alternative versions such as |pdflatex|.
%
% %%%%%%%%%%%%%%%%%%%%%%%%%%%%%%%%%%%%%%%%%%%%%%%%%%%%%%%%%%%%%%%%%%%%%%%%%%%%%%
% %%%%%%%%%%%%%%%%%%%%%%%%%%%%%%%%%%%%%%%%%%%%%%%%%%%%%%%%%%%%%%%%%%%%%%%%%%%%%%
% \section{Implementation}
%\iffalse
%<*package>
%\fi
%
% This section describes the definitions file |childdoc.def|.

% The definitions cannot be loaded using |\usepackage| or |\RequirePackage|
% which has a mechanism to prevent loading a style file more than once.
% When loading the definitions by means of |\input|
% multiple instances have to be prevented manually:
%\iffalse
%This code needs to be before the `\ProvidesFile' directive
%which is defined at the beginning of this file.
%Therefore it is also placed there and commented out here.
%</package>
%<*discard>
%\fi
%    \begin{macrocode}
\ifdefined\childdocmain\endinput\fi
%    \end{macrocode}
%\iffalse
%</discard>
%<*package>
%\fi
%
% \macro{\ifchilddoc}
% \macro{\ifchilddocmanual}
% The conditional |\ifchilddoc| tells whether a
% child (true) or main (false) document is being compiled.
% The conditional |\ifchilddocmanual| tells whether
% the |\includeonly| mechanism is used (false) or
% the selection of child files must be performed manually (true).
% The definitions initialise to false:
%    \begin{macrocode}
\newif\ifchilddoc
\newif\ifchilddocmanual
%    \end{macrocode}

% \macro{\childdocname}
% \macro{\childdocjob}
% The macro |\childdocname| stores the name of the main document
% to be compiled. The macro |\childdocjob| stores the name of
% the document on which the \LaTeX{} compiler was originally invoked.
% The content of |\jobname| cannot be compared
% to filenames specified in the source due to different catcodes.
% The following code rescans |\jobname|, stores the result
% in |\childdocname| and saves a copy in |\childdocjob|:
%    \begin{macrocode}
\edef\childdocname{\scantokens\expandafter{\jobname\noexpand}}
\let\childdocjob\childdocname
%    \end{macrocode}

% \macro{\childdocdisable}
% The macro |\childdocdisable| prevents the main file
% from being processed more than once.
% At this stage, the main document command |\childdocmain|
% is assumed to be called once again where it should do nothing.
% Any subsequent call to it should prevent
% a secondary processing of the main document
% It overwrites the forwarding commands
% |\childdocof| and |\childdocforward|
% with empty macros to prevent further inclusions of the main document:
%    \begin{macrocode}
\newcommand{\childdocdisable}
{
  \renewcommand{\childdocmain}[1]{\renewcommand{\childdocmain}[1]{\endinput}}
  \renewcommand{\childdocof}[1]{}
  \renewcommand{\childdocby}[2][]{}
  \renewcommand{\childdocforward}[2][]{}
  \renewcommand{\childdocdisable}{}
}
%    \end{macrocode}

% \macro{\childdocmain}
% The macro |\childdocmain| is to be called at the top of the main file
% with nothing or the main filename (without extension) as argument.
% First, it breaks loops.
% If the argument is not empty and does not match |\childdocname|
% (which is set by the first inclusion of |childdoc.def|),
% |\ifchilddoc| is set to true, |\includeonly| is applied to the child file
% and |\jobname| is set to the main file
% (for proper handling of |.aux| files):
%    \begin{macrocode}
\newcommand{\childdocmain}[1]
{
  \childdocdisable\childdocmain{}
  \if?#1?\else
    \begingroup
      \def\childdoctmp{#1}
      \ifx\childdoctmp\childdocname
        \def\childdoctmp{}
      \else
        \def\childdoctmp
        {
          \childdoctrue
          \includeonly{\childdocname}
          \def\childdocjob{#1}
          \def\jobname{#1}
        }
      \fi
      \expandafter
    \endgroup
    \childdoctmp
  \fi
}
%    \end{macrocode}

% \macro{\childdocof}
% The command |\childdocof| redirects
% compilation to the main file |#1|.
%    \begin{macrocode}
\newcommand{\childdocof}[1]
{
  \childdocdisable
  \childdoctrue
  \includeonly{\childdocname}
  \def\jobname{#1}
  \def\childdocjob{#1}
  \input{#1}
}
%    \end{macrocode}

% \macro{\childdocby}
% The command |\childdocby| ....
%    \begin{macrocode}
\newcommand{\childdocby}[2][]
{
  \childdocdisable
  \childdoctrue
  \childdocmanualtrue
  \if?#1?\else
    \def\jobname{#2}
  \fi
  \def\childdocjob{#2}
  \input{#2}
  \endinput
}
%    \end{macrocode}

% \macro{\childdocforward}
% The command |\childdocforward| redirects
% compilation to the main file or
% (if the optional argument is given) a child file.
% Parameters are set as if the main file
% or a child file starting with |\childdocof| was compiled.
% Then compilation is handed over to the main file:
%    \begin{macrocode}
\newcommand{\childdocforward}[2][]
{
  \begingroup
    \if?#1?
      \def\childdoctmp
      {
        \def\childdocname{#2}
        \def\childdocjob{#2}
        \def\jobname{#2}
        \input{#2}
        \endinput
      }
    \else
      \def\childdoctmp
      {
        \childdocdisable
        \def\childdocname{#2}
        \childdoctrue
        \includeonly{#2}
        \def\childdocjob{#1}
        \def\jobname{#1}
        \input{#1}
        \endinput
      }
    \fi
    \expandafter
  \endgroup
  \childdoctmp
}
%    \end{macrocode}

% \macro{\childdocforwardprefix}
% The command |\childdocforwardprefix| redirects
% compilation to the main or a child file by means of a pattern.
% The prefix |#1| in the current filename is replaced by |#2|
% and the suffix of the current filename is kept
% (it is assumed that the filename does not contain the substring `|~~~|'
% which is used as a delimiter).
% Compilation is handed over to the new file by |\childdocforward|:
%    \begin{macrocode}
\newcommand{\childdocforwardprefix}[3][]
{
  \begingroup
    \def\childdocextract #2##1~~~{\def\childdoctmp{\childdocforward[#1]{#3##1}}}
    \expandafter\childdocextract\childdocname~~~
    \expandafter
  \endgroup
  \childdoctmp
}
%    \end{macrocode}

% \macro{\childdoc}
% The deprecated macro |\childdoc| is a legacy version of |\childdocmain|:
%    \begin{macrocode}
\newcommand{\childdoc}{\childdocmain}
%    \end{macrocode}

% \macro{\childdocredirect}
% The deprecated macro |\childdocredirect| is a legacy version
% of |\childdocforward| and |\childdocforwardprefix|:
%    \begin{macrocode}
\newcommand{\childdocredirect}[2][]
{
  \begingroup
    \if?#1?
      \def\childdoctmp{\childdocforward{#2}}
    \else
      \def\childdoctmp{\childdocforwardprefix{#1}{#2}}
    \fi
    \expandafter
  \endgroup
  \childdoctmp
}
%    \end{macrocode}

%\iffalse
%</package>
%\fi
%
\endinput

\childdocforwardprefix[cdocsamp]{cdocsfn}{cdocsch}
%    \end{macrocode}

%\iffalse
%</samplefinal>
%\fi
%
% %%%%%%%%%%%%%%%%%%%%%%%%%%%%%%%%%%%%%%
% \paragraph{Command Line Processing.}
%
% The following three command lines generate the output files
% |cdocscld|, |cdocscl1| and |cdocscl2|
% which should be identical to
% |cdocsdrf|, |cdocsch1| and |cdocsfn2|, respectively:
% \begin{center}
% \begin{tabular}{l}
% |latex -jobname cdocscld \|\\
% |  "\def\version{draft}% \iffalse
%
% childdoc.dtx Copyright (C) 2017-2018 Niklas Beisert
%
% This work may be distributed and/or modified under the
% conditions of the LaTeX Project Public License, either version 1.3
% of this license or (at your option) any later version.
% The latest version of this license is in
%   http://www.latex-project.org/lppl.txt
% and version 1.3 or later is part of all distributions of LaTeX
% version 2005/12/01 or later.
%
% This work has the LPPL maintenance status `maintained'.
%
% The Current Maintainer of this work is Niklas Beisert.
%
% This work consists of the files childdoc.dtx and childdoc.ins
% and the derived files childdoc.def and cdocsamp.tex with
% cdocsch1.tex, cdocsch2.tex, cdocsdrf.tex, cdocsfn1.tex, cdocsfn2.tex.
%
%<package>\ifdefined\childdocmain\endinput\fi
%<package>\ProvidesFile{childdoc.def}[2018/12/30 v2.0 child document driver]
%<samplemain>\ProvidesFile{cdocsamp.tex}[2018/12/30 v2.0 sample for childdoc]
%<*driver>
%\ProvidesFile{childdoc.drv}[2018/12/30 v2.0 childdoc reference manual file]
\PassOptionsToClass{10pt,a4paper}{article}
\documentclass{ltxdoc}

\usepackage[margin=35mm]{geometry}
\usepackage{hyperref}
\usepackage{hyperxmp}
\usepackage[usenames]{color}

\hypersetup{colorlinks=true}
\hypersetup{pdfstartview=FitH}
\hypersetup{pdfpagemode=UseNone}
\hypersetup{pdfsource={}}
\hypersetup{pdflang={en-UK}}
\hypersetup{pdfcopyright={Copyright 2017-2018 Niklas Beisert.
  This work may be distributed and/or modified under the
  conditions of the LaTeX Project Public License, either version 1.3
  of this license or (at your option) any later version.}}
\hypersetup{pdflicenseurl={http://www.latex-project.org/lppl.txt}}
\hypersetup{pdfcontactaddress={ETH Zurich, ITP, HIT K,
  Wolfgang-Pauli-Strasse 27}}
\hypersetup{pdfcontactpostcode={8093}}
\hypersetup{pdfcontactcity={Zurich}}
\hypersetup{pdfcontactcountry={Switzerland}}
\hypersetup{pdfcontactemail={nbeisert@itp.phys.ethz.ch}}
\hypersetup{pdfcontacturl={http://people.phys.ethz.ch/\xmptilde nbeisert/}}

\newcommand{\secref}[1]{\hyperref[#1]{section \ref*{#1}}}

\parskip1ex
\parindent0pt
\let\olditemize\itemize
\def\itemize{\olditemize\parskip0pt}

\begin{document}

\title{The \textsf{childdoc} Package}
\hypersetup{pdftitle={The childdoc Package}}
\author{Niklas Beisert\\[2ex]
  Institut f\"ur Theoretische Physik\\
  Eidgen\"ossische Technische Hochschule Z\"urich\\
  Wolfgang-Pauli-Strasse 27, 8093 Z\"urich, Switzerland\\[1ex]
  \href{mailto:nbeisert@itp.phys.ethz.ch}
  {\texttt{nbeisert@itp.phys.ethz.ch}}}
\hypersetup{pdfauthor={Niklas Beisert}}
\hypersetup{pdfsubject={Manual for the LaTeX2e Package childdoc}}
\date{30 December 2018, \textsf{v2.0}}
\maketitle

\begin{abstract}\noindent
\textsf{childdoc} is a \LaTeXe{} package
that enables the direct compilation
of document sections included by |\include|
to individual files.
\end{abstract}

\begingroup
\parskip0ex
\tableofcontents
\endgroup

%%%%%%%%%%%%%%%%%%%%%%%%%%%%%%%%%%%%%%%%%%%%%%%%%%%%%%%%%%%%%%%%%%%%%%%%%%%%%%%%
%%%%%%%%%%%%%%%%%%%%%%%%%%%%%%%%%%%%%%%%%%%%%%%%%%%%%%%%%%%%%%%%%%%%%%%%%%%%%%%%
\section{Introduction}

\LaTeX{} provides a mechanism to structure a large document (such as a book)
into a main file and several child files (containing the chapters)
using the |\include| command.
This mechanism is beneficial for documents
which span hundreds of pages in order to
make the source file(s) more manageable.
Moreover, compilation can be restricted to
selected child files by means of the |\includeonly| command.
The latter feature can be used to reduce the compilation time while editing
(this was significantly more useful in the earlier days of \LaTeX{})
or to generate a smaller document which is easier to navigate.
Another application of |\includeonly| is to generate
documents consisting of selected parts of the complete document.

However, there are a few drawbacks of the plain |\include| mechanism:
\begin{itemize}
\item
The child files cannot be compiled on their own,
they can only be compiled via the main file.
A naive editing environment
(such as a text editor with an option
to have the current file processed by \LaTeX)
may require one to switch to the main file before compiling;
attempting to compile the child file produces errors.
\item
The main file must be modified (each time)
to adjust the |\includeonly| command
to the present needs. This easily leaves the main file in a messy state.
\item
The generated document will always carry the filename
of the main document. This is inconvenient if
several child files are to be compiled and
to be kept for distribution.
\end{itemize}

The present package provides a simple interface
to make child files individually compilable by \LaTeX{}.
Compiling a child file then has the same effect as compiling
the main file with an |\includeonly| command
to select the appropriate child.
Moreover the generated document will carry the name of the child
rather than the main file.
This resolves all three above issues.

This feature is meant to make the editing of books,
thesis documents and lecture notes somewhat more convenient.
However, the package can also be used efficiently for
composing a series of documents (such as exercise sheets)
which are typically distributed individually.
It then assists the author in generating the individual documents
(potentially in different versions)
as well as a document containing the collected series.
Another application is in developing style files
or other kinds of included material
where compilation of the style file could redirect
to a sample or test file.

%%%%%%%%%%%%%%%%%%%%%%%%%%%%%%%%%%%%%%%%%%%%%%%%%%%%%%%%%%%%%%%%%%%%%%%%%%%%%%%%
%%%%%%%%%%%%%%%%%%%%%%%%%%%%%%%%%%%%%%%%%%%%%%%%%%%%%%%%%%%%%%%%%%%%%%%%%%%%%%%%
\section{Usage}

First of all, the package \textsf{childdoc} is \emph{not} a standard
\LaTeXe{} |.sty| style file! Therefore it needs to be invoked in
a non-standard way.

%%%%%%%%%%%%%%%%%%%%%%%%%%%%%%%%%%%%%%%%%%%%%%%%%%%%%%%%%%%%%%%%%%%%%%%%%%%%%%%%
\subsection{Included Files}
\label{sec:include}

%%%%%%%%%%%%%%%%%%%%%%%%%%%%%%%%%%%%%%%%
\DescribeMacro{\childdocmain}
To use the package, add the commands
\begin{center}
\begin{tabular}{l}
|\input{childdoc.def}|\\
|\childdocmain{}|\\
\end{tabular}
\end{center}
at the very top of the main \LaTeX{} file,
in particular \emph{before} the |\documentclass| statement!
The argument of |\childdocmain| should be left empty
(but it must be present).

%%%%%%%%%%%%%%%%%%%%%%%%%%%%%%%%%%%%%%%%
\DescribeMacro{\childdocof}
Furthermore, add the commands
\begin{center}
\begin{tabular}{l}
|\input{childdoc.def}|\\
|\childdocof{|\textit{main}|}|\\
\end{tabular}
\end{center}
at the top of every child file \textit{child}
which is included by |\include{|\textit{child}|}|
from within the main file
(or at least for those files to be compiled individually).
The argument \textit{main} must be the filename of the main file.

There are a couple of
considerations in setting up the main and child documents:

%%%%%%%%%%%%%%%%%%%%%%%%%%%%%%%%%%%%%%%%
\paragraph{Restrictions.}

Please note the following restrictions:
\begin{itemize}
\item
|\childdocmain| must be called with one argument \textit{main}
to ensure compatibility with earlier version of the package.
It must either be empty (|\childdocmain{}|)
or precisely match the filename of the main file in which it is specified.
See \secref{sec:detection} for further information.
\item
The filename \textit{main} must be specified without the |.tex| extension.
\item
The filename \textit{main} is case sensitive
(even in case-insensitive file systems)
due to internal string comparison.
\item
The argument \textit{main} should be fully expanded, it cannot be a macro.
\item
Subdirectories and special characters should be avoided in filenames.
\item
The command |\childdocmain{|\textit{main}|}| must be followed by a whitespace.
It should not be followed immediately by another command
or by a comment mark `|%|'.
This is because the \TeX{} parser reads the token immediately following
the argument of |\childdocmain| and puts it
at the beginning of every child section;
however, a white\-space is ignored.
\end{itemize}

%%%%%%%%%%%%%%%%%%%%%%%%%%%%%%%%%%%%%%%%
\paragraph{Content of Main File.}

It is advisable to place all content in the child files included by |\include|.
Any output contained in the main file will appear in all child documents
unless suppressed manually;
it cannot be suppressed automatically by the |\includeonly| directive
and thus should normally be avoided.
A method to include some content in the main file
by means of conditional processing is described in \secref{sec:conditional}.

%%%%%%%%%%%%%%%%%%%%%%%%%%%%%%%%%%%%%%%%
\paragraph{Page Numbering.}

When only a part of the document is compiled,
the appropriate numbering of pages
(as well as other status parameters)
is determined from the |.aux| files.
The latter contain information from previous passes.
However this information needs to propagate through
all intermediate child documents.
Therefore the page numbering in child documents may well
be inconsistent until the complete document is compiled at least once.

A useful (if unconventional) way to always ensure a consistent
page numbering is to restart the numbering in each child document
and denote the pages by `\textit{child}|.|\textit{page}'
where \textit{child} represents the chapter/section number of the child file.
This can be achieved by the command
|\numberwithin{page}{|\textit{child}|}|
of the \textsf{amsmath} package
where \textit{child} can be |chapter| or |section|
depending on the chosen structuring.
Alternatively, one can modify the macro |\thepage| appropriately
and reset the counter |page| at the start of each child file.

%%%%%%%%%%%%%%%%%%%%%%%%%%%%%%%%%%%%%%%%%%%%%%%%%%%%%%%%%%%%%%%%%%%%%%%%%%%%%%%%
\subsection{Conditional Processing}
\label{sec:conditional}

The package provides a mechanism to compile different versions
of a document. To customise the versions further some conditional processing
can come in handy to distinguish which version is being compiled.
The package provides two macros to describe the compilation context:

%%%%%%%%%%%%%%%%%%%%%%%%%%%%%%%%%%%%%%%%
\DescribeMacro{\ifchilddoc}
The conditional |\ifchilddoc| distinguishes between the compilation of
child documents and the main document:
%
\begin{center}
|\ifchilddoc |\textit{child-code}| |[|\||else |\textit{main-code}]| \||fi|
\end{center}

%%%%%%%%%%%%%%%%%%%%%%%%%%%%%%%%%%%%%%%%
\DescribeMacro{\childdocname}
\DescribeMacro{\childdocjob}
The macro |\childdocname| contains the filename (without extension)
of the main or child file being processed.
Note that |\childdocjob| will always contain the name of the main file.

%%%%%%%%%%%%%%%%%%%%%%%%%%%%%%%%%%%%%%%%
\paragraph{Title Page.}

Conditional processing can be used to include a title or banner page
in the main document when proper precautions are taken.
Importantly, the code in the main file should ensure that the page counter
(as well as other status parameters which are stored in the |.aux| files)
takes the same value after the conditional processing.
Otherwise the page numbers may take divergent values
depending on which part is compiled.

For example, a title page could be declared by:
%
\begin{center}
\begin{tabular}{l}
|\ifchilddoc\||else|\\
|\addtocounter{page}{-1}|\\
\textit{code for title page}\\
|\newpage|\\
|\||fi|
\end{tabular}
\end{center}
%
A banner page for the child documents can be generated by:
%
\begin{center}
\begin{tabular}{l}
|\ifchilddoc|\\
|\addtocounter{page}{-1}|\\
\textit{code for banner page}\\
|\newpage|\\
|\||fi|
\end{tabular}
\end{center}
%
Here one could write a message such as:
\begin{center}
|This is the part \childdocname{} of \childdocjob{}.|
\end{center}

%%%%%%%%%%%%%%%%%%%%%%%%%%%%%%%%%%%%%%%%%%%%%%%%%%%%%%%%%%%%%%%%%%%%%%%%%%%%%%%%
\subsection{Flags}
\label{sec:flags}

The package makes it easy to generate different versions
of the main or child documents.
To this end compilation flags can be defined
and assigned different default values.
They will be particularly useful in conjunction
with the forwarding mechanism described in \secref{sec:forward}.

For example, it may be useful to have a flag |\version|
which can be set to |draft| or |final|.
The document source will contain some conditional code
depending on the value of |\version|.
Suppose further, the flag should default to |final| for the main file
and to |draft| for child files
which is a natural assignment for editing the document.
This is achieved by placing the following code
in the preamble of the main document
(below the |\childdocmain| directive):
%
\begin{center}
\begin{tabular}{l}
|\ifchilddoc|\\
|\providecommand{\version}{draft}|\\
|\||else|\\
|\providecommand{\version}{final}|\\
|\||fi|
\end{tabular}
\end{center}
%
The definition by |\providecommand| makes sure
that previous definitions are not overwritten.
Further statements |\providecommand{\version}{...}|
can thus be added before the above code to override it.

For the main file, one might add a line
(between |\childdocmain| and the above block)
%
\begin{center}
|%\ifchilddoc\||else\providecommand{\version}{draft}\||fi|
\end{center}
%
which can be uncommented to produce a draft version.
Likewise one can add a line to the very top of a child file
(above the |\childdocof{|\textit{main}|}| directive)
%
\begin{center}
|%\providecommand{\version}{final}|
\end{center}
%
which can be uncommented to produce the final version of this child document.

%%%%%%%%%%%%%%%%%%%%%%%%%%%%%%%%%%%%%%%%%%%%%%%%%%%%%%%%%%%%%%%%%%%%%%%%%%%%%%%%
\subsection{Forwarding}
\label{sec:forward}

Different versions of the main or child documents
using compilation flags as described in \secref{sec:flags}
can be (permanently) stored in different files
for convenient compilation, viewing and distribution.
To this end, the package defines a command
to pass on compilation to a different file:

%%%%%%%%%%%%%%%%%%%%%%%%%%%%%%%%%%%%%%%%
\DescribeMacro{\childdocforward}
The command |\childdocforward| redirects processing to
another source file:
%
\begin{center}
\begin{tabular}{l}
|\input{childdoc.def}|\\
|\childdocforward[|\textit{main}|]{|\textit{dest}|}|\\
\end{tabular}
\end{center}
%
The argument \textit{dest} is the destination file
(without extension).
It should be the main file or one of the child files.
Note that further \textsf{childdoc} directives
such as |\childdocof| and |\childdocforward|
in the indicated file will be processed in this form.
The optional argument \textit{main}
passes on directly to the main file \textit{main}
while pretending to compile the child \textit{dest}.
This form behaves as if \textit{dest}
issues |\childdocof{|\textit{main}|}| right away,
and no further \textsf{childdoc} directives will be processed.

%%%%%%%%%%%%%%%%%%%%%%%%%%%%%%%%%%%%%%%%
\DescribeMacro{\...prefix}
In the alternative form |\childdocforwardprefix|,
%
\begin{center}
\begin{tabular}{l}
|\input{childdoc.def}|\\
|\childdocforwardprefix[|\textit{main}|]{|\textit{prefix}|}{|\textit{dest}|}|
\end{tabular}
\end{center}
%
the destination file is determined by a pattern
depending on the current file:
To make this work, the current file must be called
`{\textit{prefix}\hspace{0.2em}\textit{suffix}}'
with \textit{prefix} matching precisely the argument.
Processing is then passed on to the file
`{\textit{dest}\hspace{0.2em}\textit{suffix}}'.
Surely, the same effect is achieved by
directly specifying the
argument `{\textit{dest}\hspace{0.2em}\textit{suffix}}'
in the first form.
However, that requires to set up a different file
for each child. With the alternative form of the command
all these files can have exactly the same content
which simplifies setting them up and maintaining them.

For example, the following file |draft.tex|
with a compilation flag |\version| as described in \secref{sec:flags}
compiles the main document as a draft:
%
\begin{center}
\begin{tabular}{l}
|\def\version{draft}|\\
|\input{childdoc.def}|\\
|\childdocforward{|\textit{main}|}|
\end{tabular}
\end{center}
%
Likewise, the following files |final|\textit{nn}|.tex|
compile the final version of the child document
|child|\textit{nn}|.tex|:
%
\begin{center}
\begin{tabular}{l}
|\def\version{final}|\\
|\input{childdoc.def}|\\
|\childdocforwardprefix{final}{child}|
\end{tabular}
\end{center}
%

Note that when several versions of a main file and/or of each child file
are to be generated, it may be convenient to set up a |Makefile| or
shell script to automatise the process.

%%%%%%%%%%%%%%%%%%%%%%%%%%%%%%%%%%%%%%%%%%%%%%%%%%%%%%%%%%%%%%%%%%%%%%%%%%%%%%%%
\subsection{Command Line Processing}
\label{sec:commandline}

The effect of redirection files can also be achieved by invoking
the \LaTeX{} compiler with a more elaborate command line.
Most conveniently this should be done as part
of a shell script or a |Makefile|.

When using \textsf{childdoc} in the main file, the following
command lines effectively perform a redirection
(note that depending on the shell being used,
backslashes may have to be doubled: `|\|' $\to$ `|\\|'):
%
\begin{center}
|... -jobname "|\textit{target}|" |\\|"|[\textit{flags}]%
|\input{childdoc.def}\childdocforward[|\textit{main}|]{|\textit{dest}|}"|
\end{center}
%
Here \textit{target} is the name of the output file,
\textit{main} is the name of the main file
and \textit{dest} is the name of the main or child file to be processed
(all filenames without extensions).
The optional argument \textit{main} can be omitted
if \textit{main} matches \textit{dest}.
Optionally, compilation \textit{flags} can be defined via |\def| commands.
This command line makes the \TeX{} engine believe
it is compiling the file \textit{target}
whose content is specified as the latter parameter.
The provided code then forwards the processing to
\textit{main} or \textit{dest} as described in \secref{sec:forward}.

%%%%%%%%%%%%%%%%%%%%%%%%%%%%%%%%%%%%%%%%%%%%%%%%%%%%%%%%%%%%%%%%%%%%%%%%%%%%%%%%
\subsection{Include by Input}
\label{sec:input}

Including child documents by |\include| has some restrictions by design.
Most notably, the content of a child document always occupies
its own set of pages; pages cannot be shared between child documents.
Usually, this behaviour makes perfect sense
because each child document contain an essential part of the document.
However, in some situations it may be desirable to compose
a document from a collection of parts
without having mandatory page breaks between then.
For this case, the package
provides a mechanism to include parts
by |\input| which can also be processed individually.
However, by construction this mechanism
requires manual handling of the content to be output.

%%%%%%%%%%%%%%%%%%%%%%%%%%%%%%%%%%%%%%%%
\DescribeMacro{\ifchilddocmanual}
The main file should be prepared as usual, see \secref{sec:include}.
However, the document body must make a distinction
between processing of an individual part and of the main document, e.g.:
%
\begin{center}
\begin{tabular}{l}
|\ifchilddocmanual|\\
|\input{\childdocname}|\\
|\||else|\\
\textit{document body with }|\input{|\textit{part}|}|\\
|\||fi|
\end{tabular}
\end{center}
%
The conditional |\ifchilddocmanual| is true whenever
a part to be included by |\input| is being compiled,
and the name of the part is stored in |\childdocname|.

%%%%%%%%%%%%%%%%%%%%%%%%%%%%%%%%%%%%%%%%
\DescribeMacro{\childdocby}
Each part to be included by |\input| should start with:
%
\begin{center}
\begin{tabular}{l}
|\input{childdoc.def}|\\
|\childdocby{|\textit{main}|}|\\
\end{tabular}
\end{center}
%
The directive |\childdocby| is similar to |\childdocof|
described in \secref{sec:include},
but the subsequent selection of content must be done manually.
To that end, both |\ifchilddoc| and |\ifchilddocmanual|
will be true upon processing of a part,
and the name of the part is stored in |\childdocname|.
Note that |\jobname| will be set to the filename of the current part
so that each part receives an individual |.aux| file
that does not interfere with the |.aux| file(s) of the main document.
This behaviour can be altered by the alternative form
|\childdocby[*]{|\textit{main}|}| (with a non-empty optional argument)
which uses the |.aux| file of the main document
by setting |\jobname| to \textit{main}.

%%%%%%%%%%%%%%%%%%%%%%%%%%%%%%%%%%%%%%%%%%%%%%%%%%%%%%%%%%%%%%%%%%%%%%%%%%%%%%%%
\subsection{Driver Development}
\label{sec:driver}

The \textsf{childdoc} mechanism can also be use for the development
of definition files such as \LaTeX{} styles or classes.
This case differs from the above setup with multiple parts
included by |\include| in that no |\includeonly| should be invoked.
This can be achieved by starting the include file
(before |\ProvidesPackage|) with:
%
\begin{center}
\begin{tabular}{l}
|\input{childdoc.def}|\\
|\childdocforward{|\textit{main}|}|\\
\end{tabular}
\end{center}
%
or alternatively with:
%
\begin{center}
\begin{tabular}{l}
|\input{childdoc.def}|\\
|\childdocby{|\textit{main}|}|\\
\end{tabular}
\end{center}
%
Both forms have slightly different effects as described above.
The main file is prepared as usual, see \secref{sec:include}.

%%%%%%%%%%%%%%%%%%%%%%%%%%%%%%%%%%%%%%%%%%%%%%%%%%%%%%%%%%%%%%%%%%%%%%%%%%%%%%%%
\subsection{Legacy Detection}
\label{sec:detection}

The directive |\childdocmain| in the main file can detect
whether the complete document or merely a child is to be compiled
even without using the directive |\childdocof|.
This method is deprecated because it is less robust
and there is no compelling reason to use it;
it is merely provided for backward compatibility
and it may be removed in future versions.

If the detection mechanism is to be used,
it is mandatory to correctly specify
the filename of the main file as the argument of |\childdocmain|:
%
\begin{center}
\begin{tabular}{l}
|\input{childdoc.def}|\\
|\childdocmain{|\textit{main}|}|\\
\end{tabular}
\end{center}
%
If |\jobname| does not match the argument \textit{main} of |\childdocmain|,
it is assumed that |\jobname| points to the child file to be compiled.
When using |\childdocmain| with the main file specified as argument,
it suffices to start a child file
with just |\input{|\textit{main}|}|
without loading of the package and using |\childdocof|.
If instead all processing is done
with the appropriate \textsf{childdoc} directives,
the argument of \textit{main} of |\childdocmain| can be empty.

An alternative version of the command line processing described
in \secref{sec:commandline} using the detection mechanism reads:
%
\begin{center}
|... -jobname "|\textit{target}|" "|[\textit{flags}]%
[|\def\jobname{|\textit{dest}|}|]|\input{|\textit{main}|}"|
\end{center}

%%%%%%%%%%%%%%%%%%%%%%%%%%%%%%%%%%%%%%%%%%%%%%%%%%%%%%%%%%%%%%%%%%%%%%%%%%%%%%%%
\subsection{Manual Code}
\label{sec:manual}

In case one cannot be certain whether the definitions file |childdoc.def|
is installed on the target \TeX{} distribution
and one prefers not to ship it,
it is conceivable to paste a few relevant commands into the sources.

To that end, drop all statements |\input{childdoc.def}|
and perform the replacements as outlined below.
Instead of |\childdocmain{|\textit{main}|}| add the following code
to the top of the main file:
%
\begin{center}
\begin{tabular}{l}
|\||ifdefined\childdocname\endinput\||fi\newif\ifchilddoc|\\
|\edef\childdocname{\scantokens\expandafter{\jobname\noexpand}}|\\
|\def\childdocmain{|\textit{main}|}\||ifx\childdocmain\childdocname\||else|\\
|\childdoctrue\includeonly{\childdocname}\let\jobname\childdocmain\||fi|\\
\end{tabular}
\end{center}
%
Instead of |\childdocof{|\textit{main}|}| just include the main file
at the top of each child file:
%
\begin{center}
|\input{|\textit{main}|}|
\end{center}
%
A simple redirection |\childdocforward{|\textit{dest}|}| is achieved by:
%
\begin{center}
|\def\jobname{|\textit{dest}|}\input{\jobname}|
\end{center}
%
The redirection with prefix
|\childdocforwardprefix[|\textit{prefix}|]{|\textit{dest}|}|
is accomplished by:
%
\begin{center}
\begin{tabular}{l}
|{\edef\jobname{\scantokens\expandafter{\jobname\noexpand}}|\\
|\def\redirectjob |\textit{prefix}|#1~~~{\gdef\jobname{|\textit{dest}|#1}}|\\
|\expandafter\redirectjob\jobname~~~}\input{\jobname}|
\end{tabular}
\end{center}

In an alternative approach,
child documents can be compiled by a specific command line
without additional code or specific definitions:
%
\begin{center}
|... -jobname "|\textit{target}|" "|[\textit{flags}]%
|\includeonly{|\textit{dest}|}\input{|\textit{main}|}"|
\end{center}
%

%%%%%%%%%%%%%%%%%%%%%%%%%%%%%%%%%%%%%%%%%%%%%%%%%%%%%%%%%%%%%%%%%%%%%%%%%%%%%%%%
%%%%%%%%%%%%%%%%%%%%%%%%%%%%%%%%%%%%%%%%%%%%%%%%%%%%%%%%%%%%%%%%%%%%%%%%%%%%%%%%
\section{Information}

%%%%%%%%%%%%%%%%%%%%%%%%%%%%%%%%%%%%%%%%%%%%%%%%%%%%%%%%%%%%%%%%%%%%%%%%%%%%%%%%
\subsection{Copyright}

Copyright \copyright{} 2017--2018 Niklas Beisert

This work may be distributed and/or modified under the
conditions of the \LaTeX{} Project Public License, either version 1.3
of this license or (at your option) any later version.
The latest version of this license is in
  \url{http://www.latex-project.org/lppl.txt}
and version 1.3 or later is part of all distributions of \LaTeX{}
version 2005/12/01 or later.

This work has the LPPL maintenance status `maintained'.

The Current Maintainer of this work is Niklas Beisert.

This work consists of the files |README.txt|, |childdoc.ins| and |childdoc.dtx|
as well as the derived files |childdoc.def|, |cdocsamp.tex|
with |cdocsch1.tex|, |cdocsch2.tex|, |cdocspt3.tex|, |cdocspt4.tex|,
|cdocsdrf.tex|, |cdocsfn1.tex|, |cdocsfn2.tex|
as well as |childdoc.pdf|.

%%%%%%%%%%%%%%%%%%%%%%%%%%%%%%%%%%%%%%%%%%%%%%%%%%%%%%%%%%%%%%%%%%%%%%%%%%%%%%%%
\subsection{Files and Installation}

The package consists of the files:
%
\begin{center}
\begin{tabular}{ll}
    |README.txt|   & readme file \\
    |childdoc.ins| & installation file \\
    |childdoc.dtx| & source file \\
    |childdoc.def| & definition file \\
    |cdocsamp.tex| & sample main file \\
    |cdocsch1.tex| & sample include file \\
    |cdocsch2.tex| & sample include file \\
    |cdocspt3.tex| & sample part file \\
    |cdocspt4.tex| & sample part file \\
    |cdocsdrf.tex| & sample redirection file \\
    |cdocsfn1.tex| & sample redirection file \\
    |cdocsfn2.tex| & sample redirection file \\
    |childdoc.pdf| & manual
\end{tabular}
\end{center}
%
The distribution consists of the files
|README.txt|, |childdoc.ins| and |childdoc.dtx|.
%
\begin{itemize}
\item
Run (pdf)\LaTeX{} on |childdoc.dtx|
to compile the manual |childdoc.pdf| (this file).
\item
Run \LaTeX{} on |childdoc.ins| to create the definitions file |childdoc.def|
and the sample |cdocsamp.tex| with include files
|cdocsch1.tex|, |cdocsch2.tex|, |cdocspt3.tex|, |cdocspt4.tex|,
|cdocsdrf.tex|, |cdocsfn1.tex|, |cdocsfn2.tex|.
Then copy the file |childdoc.def| to an appropriate directory of your \LaTeX{}
distribution, e.g.\ \textit{texmf-root}|/tex/latex/childdoc|.
\end{itemize}

%%%%%%%%%%%%%%%%%%%%%%%%%%%%%%%%%%%%%%%%%%%%%%%%%%%%%%%%%%%%%%%%%%%%%%%%%%%%%%%%
\subsection{Related CTAN Packages}

There are several other packages which offer a similar functionality:
%
\begin{itemize}
\item
The packages
\href{http://ctan.org/pkg/docmute}{\textsf{docmute}},
\href{http://ctan.org/pkg/includex}{\textsf{includex}} and
\href{http://ctan.org/pkg/standalone}{\textsf{standalone}}
provide commands to include only the document body of
a child file thus allowing both files to be compiled individually.
\item
The packages \href{http://ctan.org/pkg/subdocs}{\textsf{subdocs}}
and \href{http://ctan.org/pkg/subfiles}{\textsf{subfiles}}
provide structures in which the main and child documents can be
encapsulated and allowing them to be compiled individually.
The inclusion mechanism is different from the conventional |\include|.
\item
The package \href{http://ctan.org/pkg/combine}{\textsf{combine}}
is an elaborate solution to combine several documents into one.
\end{itemize}
%
See also the CTAN topic \href{http://ctan.org/topic/subdocs}{\textsf{subdocs}}
for further related packages.
The present package differs from the above solutions in that
a document structure constructed with the conventional |\include| mechanism
just needs two extra commands at the top of every file
such that all constituent files can be compiled individually.

%%%%%%%%%%%%%%%%%%%%%%%%%%%%%%%%%%%%%%%%%%%%%%%%%%%%%%%%%%%%%%%%%%%%%%%%%%%%%%%%
%\subsection{Feature Suggestions}
%
%The following is a list of features which may be useful for future
%versions of this package:
%%
%\begin{itemize}
%\item
%\ldots
%\end{itemize}

%%%%%%%%%%%%%%%%%%%%%%%%%%%%%%%%%%%%%%%%%%%%%%%%%%%%%%%%%%%%%%%%%%%%%%%%%%%%%%%%
\subsection{Revision History}

%%%%%%%%%%%%%%%%%%%%%%%%%%%%%%%%%%%%%%%%
\paragraph{v2.0:} 2018/12/30

\begin{itemize}
\item
immediate forward processing
\item
added |\childdocby| mechanism
\item
manual restructured
\end{itemize}

%%%%%%%%%%%%%%%%%%%%%%%%%%%%%%%%%%%%%%%%
\paragraph{v1.6:} 2018/01/17

\begin{itemize}
\item
application for development of include files
\item
corrections to manual
\end{itemize}

%%%%%%%%%%%%%%%%%%%%%%%%%%%%%%%%%%%%%%%%
\paragraph{v1.5:} 2017/05/21

\begin{itemize}
\item
more complete structuring introduced
\item
|\childdocof| introduced
\item
|\childdoc| renamed to |\childdocmain|
\item
|\childredirect| renamed to |\childdocforward| and |\childdocforwardprefix|
and functionality expanded
\end{itemize}

%%%%%%%%%%%%%%%%%%%%%%%%%%%%%%%%%%%%%%%%
\paragraph{v1.0:} 2017/04/27

\begin{itemize}
\item
manual and install package
\item
first version published on CTAN
\end{itemize}

%%%%%%%%%%%%%%%%%%%%%%%%%%%%%%%%%%%%%%%%
\paragraph{v0.6:} 2017/04/26

\begin{itemize}
\item
redirection mechanism added
\end{itemize}

%%%%%%%%%%%%%%%%%%%%%%%%%%%%%%%%%%%%%%%%
\paragraph{v0.5:} 2017/04/26

\begin{itemize}
\item
functionality in definition file
\end{itemize}


%%%%%%%%%%%%%%%%%%%%%%%%%%%%%%%%%%%%%%%%%%%%%%%%%%%%%%%%%%%%%%%%%%%%%%%%%%%%%%%%
%%%%%%%%%%%%%%%%%%%%%%%%%%%%%%%%%%%%%%%%%%%%%%%%%%%%%%%%%%%%%%%%%%%%%%%%%%%%%%%%
%%%%%%%%%%%%%%%%%%%%%%%%%%%%%%%%%%%%%%%%%%%%%%%%%%%%%%%%%%%%%%%%%%%%%%%%%%%%%%%%
\appendix

\settowidth\MacroIndent{\rmfamily\scriptsize 000\ }

 \DocInput{childdoc.dtx}

\end{document}
%</driver>
% \fi
%
% %%%%%%%%%%%%%%%%%%%%%%%%%%%%%%%%%%%%%%%%%%%%%%%%%%%%%%%%%%%%%%%%%%%%%%%%%%%%%%
% %%%%%%%%%%%%%%%%%%%%%%%%%%%%%%%%%%%%%%%%%%%%%%%%%%%%%%%%%%%%%%%%%%%%%%%%%%%%%%
% \section{Sample}
%\iffalse
%<*samplemain>
%\fi
%
% The following presents a sample document
% with two chapters, two parts, a title page,
% a compile flag as well as three forwarding files to set the flag.
% It consists of eight |.tex| files:
% \begin{center}
% \begin{tabular}{ll}
% |cdocsamp.tex|&main file\\
% |cdocsch1.tex|&include file for chapter 1\\
% |cdocsch2.tex|&include file for chapter 2\\
% |cdocspt3.tex|&include file for part 3\\
% |cdocspt4.tex|&include file for part 4\\
% |cdocsdrf.tex|&forwarding file for main file in draft mode\\
% |cdocsfi1.tex|&forwarding file for final version of chapter 1\\
% |cdocsfi2.tex|&forwarding file for final version of chapter 2\\
% \end{tabular}
% \end{center}
% Each of the eight files can be compiled directly by the \LaTeX{} compiler.
%
% %%%%%%%%%%%%%%%%%%%%%%%%%%%%%%%%%%%%%%
% \paragraph{Main File.}
%
% The main file is called |cdocsamp.tex|.
%
% Load the \textsf{childdoc} definitions and
% declare the filename for the main document:
%    \begin{macrocode}
\input{childdoc.def}
\childdocmain{}
%    \end{macrocode}

% Optional override for |\version| flag:
%    \begin{macrocode}
%%\ifchilddoc\else\providecommand{\version}{draft}\fi
%    \end{macrocode}

% Define the default values for the |\version| flag
% (|final| for the main file and |draft| for childs):
%    \begin{macrocode}
\ifchilddoc
\providecommand{\version}{draft}
\else
\providecommand{\version}{final}
\fi
%    \end{macrocode}

% Load the standard document class:
%    \begin{macrocode}
\documentclass[12pt]{article}
%    \end{macrocode}

% Start the document body:
%    \begin{macrocode}
\begin{document}
%    \end{macrocode}

% Declare a title page.
% Print title, part of document being processed and version flag:
%    \begin{macrocode}
\addtocounter{page}{-1}
\begin{center}
{\LARGE\bfseries{}childdoc example\par}
\vspace{1cm}
\ifchilddoc
\ifchilddocmanual part\else chapter\fi:
`\childdocname' of `\childdocjob'\par
\else
main document: `\childdocjob'\par
\fi
version: \version\par
\end{center}
\newpage
%    \end{macrocode}

% Manually include selected file,
% otherwise process as usual:
%    \begin{macrocode}
\ifchilddocmanual
\section*{part `\childdocname'}
\input{\childdocname}
\else
%    \end{macrocode}

% Include the two chapters:
%    \begin{macrocode}
\include{cdocsch1}
\include{cdocsch2}
%    \end{macrocode}

% Include the two parts unless only chapters should be displayed:
%    \begin{macrocode}
\ifchilddoc\else
\section{part three}
\input{cdocspt3}
\section{part four}
\input{cdocspt4}
\fi
%    \end{macrocode}

% Process as usual until here:
%    \begin{macrocode}
\fi
%    \end{macrocode}

% End of document body:
%    \begin{macrocode}
\end{document}
%    \end{macrocode}
%\iffalse
%</samplemain>
%\fi
%
% %%%%%%%%%%%%%%%%%%%%%%%%%%%%%%%%%%%%%%
% \paragraph{Chapter Include Files.}
%
% The include files are called |cdocsch1.tex| and |cdocsch2.tex|.
%
%\iffalse
%<*samplechap1|samplechap2>
%\fi

% Optional override for |\version| flag:
%    \begin{macrocode}
%%\providecommand{\version}{final}
%    \end{macrocode}

% Include the main document:
%    \begin{macrocode}
\input{childdoc.def}
\childdocof{cdocsamp}
%    \end{macrocode}

%\iffalse
%</samplechap1|samplechap2>
%\fi
%
%\iffalse
%<*samplechap1>
%\fi
% Some text for chapter 1:
%    \begin{macrocode}
\section{one}
some text in chapter one
%    \end{macrocode}

%\iffalse
%</samplechap1>
%\fi
% Some text for chapter 2:
%\iffalse
%<*samplechap2>
%\fi
%    \begin{macrocode}
\section{two}
more text in chapter two
%    \end{macrocode}

%\iffalse
%</samplechap2>
%\fi
%
% %%%%%%%%%%%%%%%%%%%%%%%%%%%%%%%%%%%%%%
% \paragraph{Part Include Files.}
%
% The include files are called |cdocspt3.tex| and |cdocspt4.tex|.
%
%\iffalse
%<*samplepart3|samplepart4>
%\fi

% Optional override for |\version| flag:
%    \begin{macrocode}
%%\providecommand{\version}{final}
%    \end{macrocode}

% Include the main document:
%    \begin{macrocode}
\input{childdoc.def}
\childdocby{cdocsamp}
%    \end{macrocode}

%\iffalse
%</samplepart3|samplepart4>
%\fi
%
%\iffalse
%<*samplepart3>
%\fi
% Some text for part 3:
%    \begin{macrocode}
some text in part three
%    \end{macrocode}

%\iffalse
%</samplepart3>
%\fi
% Some text for part 4:
%\iffalse
%<*samplepart4>
%\fi
%    \begin{macrocode}
more text in part four
%    \end{macrocode}

%\iffalse
%</samplepart4>
%\fi
%
% %%%%%%%%%%%%%%%%%%%%%%%%%%%%%%%%%%%%%%
% \paragraph{Forwarding for a Complete Draft.}
%
% The following forwarding file |cdocsdrf.tex|
% compiles the main document in draft mode:
%\iffalse
%<*sampledraft>
%\fi
%    \begin{macrocode}
\def\version{draft}
\input{childdoc.def}
\childdocforward{cdocsamp}
%    \end{macrocode}

%\iffalse
%</sampledraft>
%\fi
%
% %%%%%%%%%%%%%%%%%%%%%%%%%%%%%%%%%%%%%%
% \paragraph{Forwarding for Final Version of the Chapters.}
%
% The following forwarding files |cdocsfn1.tex| and |cdocsfn2.tex|
% (with identical content)
% compile the final versions of the child documents
% |cdocsch1.tex| and |cdocsch2.tex|, respectively:
%\iffalse
%<*samplefinal>
%\fi
%    \begin{macrocode}
\def\version{final}
\input{childdoc.def}
\childdocforwardprefix[cdocsamp]{cdocsfn}{cdocsch}
%    \end{macrocode}

%\iffalse
%</samplefinal>
%\fi
%
% %%%%%%%%%%%%%%%%%%%%%%%%%%%%%%%%%%%%%%
% \paragraph{Command Line Processing.}
%
% The following three command lines generate the output files
% |cdocscld|, |cdocscl1| and |cdocscl2|
% which should be identical to
% |cdocsdrf|, |cdocsch1| and |cdocsfn2|, respectively:
% \begin{center}
% \begin{tabular}{l}
% |latex -jobname cdocscld \|\\
% |  "\def\version{draft}\input{childdoc.def}\childdocforward{cdocsamp}"|\\
% |latex -jobname cdocscl1 \|\\
% |  "\input{childdoc.def}\childdocforward[cdocsamp]{cdocsch1}"|\\
% |latex -jobname cdocscl2 \|\\
% |  "\def\version{final}\input{childdoc.def}\childdocforward{cdocsch2}"|
% \end{tabular}
% \end{center}
% Note that the trailing backslash on each first line
% merely continues the input to the second line
% (for convenient cut ant paste).
% Furthermore, the command |latex| can be replaced by any
% of its alternative versions such as |pdflatex|.
%
% %%%%%%%%%%%%%%%%%%%%%%%%%%%%%%%%%%%%%%%%%%%%%%%%%%%%%%%%%%%%%%%%%%%%%%%%%%%%%%
% %%%%%%%%%%%%%%%%%%%%%%%%%%%%%%%%%%%%%%%%%%%%%%%%%%%%%%%%%%%%%%%%%%%%%%%%%%%%%%
% \section{Implementation}
%\iffalse
%<*package>
%\fi
%
% This section describes the definitions file |childdoc.def|.

% The definitions cannot be loaded using |\usepackage| or |\RequirePackage|
% which has a mechanism to prevent loading a style file more than once.
% When loading the definitions by means of |\input|
% multiple instances have to be prevented manually:
%\iffalse
%This code needs to be before the `\ProvidesFile' directive
%which is defined at the beginning of this file.
%Therefore it is also placed there and commented out here.
%</package>
%<*discard>
%\fi
%    \begin{macrocode}
\ifdefined\childdocmain\endinput\fi
%    \end{macrocode}
%\iffalse
%</discard>
%<*package>
%\fi
%
% \macro{\ifchilddoc}
% \macro{\ifchilddocmanual}
% The conditional |\ifchilddoc| tells whether a
% child (true) or main (false) document is being compiled.
% The conditional |\ifchilddocmanual| tells whether
% the |\includeonly| mechanism is used (false) or
% the selection of child files must be performed manually (true).
% The definitions initialise to false:
%    \begin{macrocode}
\newif\ifchilddoc
\newif\ifchilddocmanual
%    \end{macrocode}

% \macro{\childdocname}
% \macro{\childdocjob}
% The macro |\childdocname| stores the name of the main document
% to be compiled. The macro |\childdocjob| stores the name of
% the document on which the \LaTeX{} compiler was originally invoked.
% The content of |\jobname| cannot be compared
% to filenames specified in the source due to different catcodes.
% The following code rescans |\jobname|, stores the result
% in |\childdocname| and saves a copy in |\childdocjob|:
%    \begin{macrocode}
\edef\childdocname{\scantokens\expandafter{\jobname\noexpand}}
\let\childdocjob\childdocname
%    \end{macrocode}

% \macro{\childdocdisable}
% The macro |\childdocdisable| prevents the main file
% from being processed more than once.
% At this stage, the main document command |\childdocmain|
% is assumed to be called once again where it should do nothing.
% Any subsequent call to it should prevent
% a secondary processing of the main document
% It overwrites the forwarding commands
% |\childdocof| and |\childdocforward|
% with empty macros to prevent further inclusions of the main document:
%    \begin{macrocode}
\newcommand{\childdocdisable}
{
  \renewcommand{\childdocmain}[1]{\renewcommand{\childdocmain}[1]{\endinput}}
  \renewcommand{\childdocof}[1]{}
  \renewcommand{\childdocby}[2][]{}
  \renewcommand{\childdocforward}[2][]{}
  \renewcommand{\childdocdisable}{}
}
%    \end{macrocode}

% \macro{\childdocmain}
% The macro |\childdocmain| is to be called at the top of the main file
% with nothing or the main filename (without extension) as argument.
% First, it breaks loops.
% If the argument is not empty and does not match |\childdocname|
% (which is set by the first inclusion of |childdoc.def|),
% |\ifchilddoc| is set to true, |\includeonly| is applied to the child file
% and |\jobname| is set to the main file
% (for proper handling of |.aux| files):
%    \begin{macrocode}
\newcommand{\childdocmain}[1]
{
  \childdocdisable\childdocmain{}
  \if?#1?\else
    \begingroup
      \def\childdoctmp{#1}
      \ifx\childdoctmp\childdocname
        \def\childdoctmp{}
      \else
        \def\childdoctmp
        {
          \childdoctrue
          \includeonly{\childdocname}
          \def\childdocjob{#1}
          \def\jobname{#1}
        }
      \fi
      \expandafter
    \endgroup
    \childdoctmp
  \fi
}
%    \end{macrocode}

% \macro{\childdocof}
% The command |\childdocof| redirects
% compilation to the main file |#1|.
%    \begin{macrocode}
\newcommand{\childdocof}[1]
{
  \childdocdisable
  \childdoctrue
  \includeonly{\childdocname}
  \def\jobname{#1}
  \def\childdocjob{#1}
  \input{#1}
}
%    \end{macrocode}

% \macro{\childdocby}
% The command |\childdocby| ....
%    \begin{macrocode}
\newcommand{\childdocby}[2][]
{
  \childdocdisable
  \childdoctrue
  \childdocmanualtrue
  \if?#1?\else
    \def\jobname{#2}
  \fi
  \def\childdocjob{#2}
  \input{#2}
  \endinput
}
%    \end{macrocode}

% \macro{\childdocforward}
% The command |\childdocforward| redirects
% compilation to the main file or
% (if the optional argument is given) a child file.
% Parameters are set as if the main file
% or a child file starting with |\childdocof| was compiled.
% Then compilation is handed over to the main file:
%    \begin{macrocode}
\newcommand{\childdocforward}[2][]
{
  \begingroup
    \if?#1?
      \def\childdoctmp
      {
        \def\childdocname{#2}
        \def\childdocjob{#2}
        \def\jobname{#2}
        \input{#2}
        \endinput
      }
    \else
      \def\childdoctmp
      {
        \childdocdisable
        \def\childdocname{#2}
        \childdoctrue
        \includeonly{#2}
        \def\childdocjob{#1}
        \def\jobname{#1}
        \input{#1}
        \endinput
      }
    \fi
    \expandafter
  \endgroup
  \childdoctmp
}
%    \end{macrocode}

% \macro{\childdocforwardprefix}
% The command |\childdocforwardprefix| redirects
% compilation to the main or a child file by means of a pattern.
% The prefix |#1| in the current filename is replaced by |#2|
% and the suffix of the current filename is kept
% (it is assumed that the filename does not contain the substring `|~~~|'
% which is used as a delimiter).
% Compilation is handed over to the new file by |\childdocforward|:
%    \begin{macrocode}
\newcommand{\childdocforwardprefix}[3][]
{
  \begingroup
    \def\childdocextract #2##1~~~{\def\childdoctmp{\childdocforward[#1]{#3##1}}}
    \expandafter\childdocextract\childdocname~~~
    \expandafter
  \endgroup
  \childdoctmp
}
%    \end{macrocode}

% \macro{\childdoc}
% The deprecated macro |\childdoc| is a legacy version of |\childdocmain|:
%    \begin{macrocode}
\newcommand{\childdoc}{\childdocmain}
%    \end{macrocode}

% \macro{\childdocredirect}
% The deprecated macro |\childdocredirect| is a legacy version
% of |\childdocforward| and |\childdocforwardprefix|:
%    \begin{macrocode}
\newcommand{\childdocredirect}[2][]
{
  \begingroup
    \if?#1?
      \def\childdoctmp{\childdocforward{#2}}
    \else
      \def\childdoctmp{\childdocforwardprefix{#1}{#2}}
    \fi
    \expandafter
  \endgroup
  \childdoctmp
}
%    \end{macrocode}

%\iffalse
%</package>
%\fi
%
\endinput
\childdocforward{cdocsamp}"|\\
% |latex -jobname cdocscl1 \|\\
% |  "% \iffalse
%
% childdoc.dtx Copyright (C) 2017-2018 Niklas Beisert
%
% This work may be distributed and/or modified under the
% conditions of the LaTeX Project Public License, either version 1.3
% of this license or (at your option) any later version.
% The latest version of this license is in
%   http://www.latex-project.org/lppl.txt
% and version 1.3 or later is part of all distributions of LaTeX
% version 2005/12/01 or later.
%
% This work has the LPPL maintenance status `maintained'.
%
% The Current Maintainer of this work is Niklas Beisert.
%
% This work consists of the files childdoc.dtx and childdoc.ins
% and the derived files childdoc.def and cdocsamp.tex with
% cdocsch1.tex, cdocsch2.tex, cdocsdrf.tex, cdocsfn1.tex, cdocsfn2.tex.
%
%<package>\ifdefined\childdocmain\endinput\fi
%<package>\ProvidesFile{childdoc.def}[2018/12/30 v2.0 child document driver]
%<samplemain>\ProvidesFile{cdocsamp.tex}[2018/12/30 v2.0 sample for childdoc]
%<*driver>
%\ProvidesFile{childdoc.drv}[2018/12/30 v2.0 childdoc reference manual file]
\PassOptionsToClass{10pt,a4paper}{article}
\documentclass{ltxdoc}

\usepackage[margin=35mm]{geometry}
\usepackage{hyperref}
\usepackage{hyperxmp}
\usepackage[usenames]{color}

\hypersetup{colorlinks=true}
\hypersetup{pdfstartview=FitH}
\hypersetup{pdfpagemode=UseNone}
\hypersetup{pdfsource={}}
\hypersetup{pdflang={en-UK}}
\hypersetup{pdfcopyright={Copyright 2017-2018 Niklas Beisert.
  This work may be distributed and/or modified under the
  conditions of the LaTeX Project Public License, either version 1.3
  of this license or (at your option) any later version.}}
\hypersetup{pdflicenseurl={http://www.latex-project.org/lppl.txt}}
\hypersetup{pdfcontactaddress={ETH Zurich, ITP, HIT K,
  Wolfgang-Pauli-Strasse 27}}
\hypersetup{pdfcontactpostcode={8093}}
\hypersetup{pdfcontactcity={Zurich}}
\hypersetup{pdfcontactcountry={Switzerland}}
\hypersetup{pdfcontactemail={nbeisert@itp.phys.ethz.ch}}
\hypersetup{pdfcontacturl={http://people.phys.ethz.ch/\xmptilde nbeisert/}}

\newcommand{\secref}[1]{\hyperref[#1]{section \ref*{#1}}}

\parskip1ex
\parindent0pt
\let\olditemize\itemize
\def\itemize{\olditemize\parskip0pt}

\begin{document}

\title{The \textsf{childdoc} Package}
\hypersetup{pdftitle={The childdoc Package}}
\author{Niklas Beisert\\[2ex]
  Institut f\"ur Theoretische Physik\\
  Eidgen\"ossische Technische Hochschule Z\"urich\\
  Wolfgang-Pauli-Strasse 27, 8093 Z\"urich, Switzerland\\[1ex]
  \href{mailto:nbeisert@itp.phys.ethz.ch}
  {\texttt{nbeisert@itp.phys.ethz.ch}}}
\hypersetup{pdfauthor={Niklas Beisert}}
\hypersetup{pdfsubject={Manual for the LaTeX2e Package childdoc}}
\date{30 December 2018, \textsf{v2.0}}
\maketitle

\begin{abstract}\noindent
\textsf{childdoc} is a \LaTeXe{} package
that enables the direct compilation
of document sections included by |\include|
to individual files.
\end{abstract}

\begingroup
\parskip0ex
\tableofcontents
\endgroup

%%%%%%%%%%%%%%%%%%%%%%%%%%%%%%%%%%%%%%%%%%%%%%%%%%%%%%%%%%%%%%%%%%%%%%%%%%%%%%%%
%%%%%%%%%%%%%%%%%%%%%%%%%%%%%%%%%%%%%%%%%%%%%%%%%%%%%%%%%%%%%%%%%%%%%%%%%%%%%%%%
\section{Introduction}

\LaTeX{} provides a mechanism to structure a large document (such as a book)
into a main file and several child files (containing the chapters)
using the |\include| command.
This mechanism is beneficial for documents
which span hundreds of pages in order to
make the source file(s) more manageable.
Moreover, compilation can be restricted to
selected child files by means of the |\includeonly| command.
The latter feature can be used to reduce the compilation time while editing
(this was significantly more useful in the earlier days of \LaTeX{})
or to generate a smaller document which is easier to navigate.
Another application of |\includeonly| is to generate
documents consisting of selected parts of the complete document.

However, there are a few drawbacks of the plain |\include| mechanism:
\begin{itemize}
\item
The child files cannot be compiled on their own,
they can only be compiled via the main file.
A naive editing environment
(such as a text editor with an option
to have the current file processed by \LaTeX)
may require one to switch to the main file before compiling;
attempting to compile the child file produces errors.
\item
The main file must be modified (each time)
to adjust the |\includeonly| command
to the present needs. This easily leaves the main file in a messy state.
\item
The generated document will always carry the filename
of the main document. This is inconvenient if
several child files are to be compiled and
to be kept for distribution.
\end{itemize}

The present package provides a simple interface
to make child files individually compilable by \LaTeX{}.
Compiling a child file then has the same effect as compiling
the main file with an |\includeonly| command
to select the appropriate child.
Moreover the generated document will carry the name of the child
rather than the main file.
This resolves all three above issues.

This feature is meant to make the editing of books,
thesis documents and lecture notes somewhat more convenient.
However, the package can also be used efficiently for
composing a series of documents (such as exercise sheets)
which are typically distributed individually.
It then assists the author in generating the individual documents
(potentially in different versions)
as well as a document containing the collected series.
Another application is in developing style files
or other kinds of included material
where compilation of the style file could redirect
to a sample or test file.

%%%%%%%%%%%%%%%%%%%%%%%%%%%%%%%%%%%%%%%%%%%%%%%%%%%%%%%%%%%%%%%%%%%%%%%%%%%%%%%%
%%%%%%%%%%%%%%%%%%%%%%%%%%%%%%%%%%%%%%%%%%%%%%%%%%%%%%%%%%%%%%%%%%%%%%%%%%%%%%%%
\section{Usage}

First of all, the package \textsf{childdoc} is \emph{not} a standard
\LaTeXe{} |.sty| style file! Therefore it needs to be invoked in
a non-standard way.

%%%%%%%%%%%%%%%%%%%%%%%%%%%%%%%%%%%%%%%%%%%%%%%%%%%%%%%%%%%%%%%%%%%%%%%%%%%%%%%%
\subsection{Included Files}
\label{sec:include}

%%%%%%%%%%%%%%%%%%%%%%%%%%%%%%%%%%%%%%%%
\DescribeMacro{\childdocmain}
To use the package, add the commands
\begin{center}
\begin{tabular}{l}
|\input{childdoc.def}|\\
|\childdocmain{}|\\
\end{tabular}
\end{center}
at the very top of the main \LaTeX{} file,
in particular \emph{before} the |\documentclass| statement!
The argument of |\childdocmain| should be left empty
(but it must be present).

%%%%%%%%%%%%%%%%%%%%%%%%%%%%%%%%%%%%%%%%
\DescribeMacro{\childdocof}
Furthermore, add the commands
\begin{center}
\begin{tabular}{l}
|\input{childdoc.def}|\\
|\childdocof{|\textit{main}|}|\\
\end{tabular}
\end{center}
at the top of every child file \textit{child}
which is included by |\include{|\textit{child}|}|
from within the main file
(or at least for those files to be compiled individually).
The argument \textit{main} must be the filename of the main file.

There are a couple of
considerations in setting up the main and child documents:

%%%%%%%%%%%%%%%%%%%%%%%%%%%%%%%%%%%%%%%%
\paragraph{Restrictions.}

Please note the following restrictions:
\begin{itemize}
\item
|\childdocmain| must be called with one argument \textit{main}
to ensure compatibility with earlier version of the package.
It must either be empty (|\childdocmain{}|)
or precisely match the filename of the main file in which it is specified.
See \secref{sec:detection} for further information.
\item
The filename \textit{main} must be specified without the |.tex| extension.
\item
The filename \textit{main} is case sensitive
(even in case-insensitive file systems)
due to internal string comparison.
\item
The argument \textit{main} should be fully expanded, it cannot be a macro.
\item
Subdirectories and special characters should be avoided in filenames.
\item
The command |\childdocmain{|\textit{main}|}| must be followed by a whitespace.
It should not be followed immediately by another command
or by a comment mark `|%|'.
This is because the \TeX{} parser reads the token immediately following
the argument of |\childdocmain| and puts it
at the beginning of every child section;
however, a white\-space is ignored.
\end{itemize}

%%%%%%%%%%%%%%%%%%%%%%%%%%%%%%%%%%%%%%%%
\paragraph{Content of Main File.}

It is advisable to place all content in the child files included by |\include|.
Any output contained in the main file will appear in all child documents
unless suppressed manually;
it cannot be suppressed automatically by the |\includeonly| directive
and thus should normally be avoided.
A method to include some content in the main file
by means of conditional processing is described in \secref{sec:conditional}.

%%%%%%%%%%%%%%%%%%%%%%%%%%%%%%%%%%%%%%%%
\paragraph{Page Numbering.}

When only a part of the document is compiled,
the appropriate numbering of pages
(as well as other status parameters)
is determined from the |.aux| files.
The latter contain information from previous passes.
However this information needs to propagate through
all intermediate child documents.
Therefore the page numbering in child documents may well
be inconsistent until the complete document is compiled at least once.

A useful (if unconventional) way to always ensure a consistent
page numbering is to restart the numbering in each child document
and denote the pages by `\textit{child}|.|\textit{page}'
where \textit{child} represents the chapter/section number of the child file.
This can be achieved by the command
|\numberwithin{page}{|\textit{child}|}|
of the \textsf{amsmath} package
where \textit{child} can be |chapter| or |section|
depending on the chosen structuring.
Alternatively, one can modify the macro |\thepage| appropriately
and reset the counter |page| at the start of each child file.

%%%%%%%%%%%%%%%%%%%%%%%%%%%%%%%%%%%%%%%%%%%%%%%%%%%%%%%%%%%%%%%%%%%%%%%%%%%%%%%%
\subsection{Conditional Processing}
\label{sec:conditional}

The package provides a mechanism to compile different versions
of a document. To customise the versions further some conditional processing
can come in handy to distinguish which version is being compiled.
The package provides two macros to describe the compilation context:

%%%%%%%%%%%%%%%%%%%%%%%%%%%%%%%%%%%%%%%%
\DescribeMacro{\ifchilddoc}
The conditional |\ifchilddoc| distinguishes between the compilation of
child documents and the main document:
%
\begin{center}
|\ifchilddoc |\textit{child-code}| |[|\||else |\textit{main-code}]| \||fi|
\end{center}

%%%%%%%%%%%%%%%%%%%%%%%%%%%%%%%%%%%%%%%%
\DescribeMacro{\childdocname}
\DescribeMacro{\childdocjob}
The macro |\childdocname| contains the filename (without extension)
of the main or child file being processed.
Note that |\childdocjob| will always contain the name of the main file.

%%%%%%%%%%%%%%%%%%%%%%%%%%%%%%%%%%%%%%%%
\paragraph{Title Page.}

Conditional processing can be used to include a title or banner page
in the main document when proper precautions are taken.
Importantly, the code in the main file should ensure that the page counter
(as well as other status parameters which are stored in the |.aux| files)
takes the same value after the conditional processing.
Otherwise the page numbers may take divergent values
depending on which part is compiled.

For example, a title page could be declared by:
%
\begin{center}
\begin{tabular}{l}
|\ifchilddoc\||else|\\
|\addtocounter{page}{-1}|\\
\textit{code for title page}\\
|\newpage|\\
|\||fi|
\end{tabular}
\end{center}
%
A banner page for the child documents can be generated by:
%
\begin{center}
\begin{tabular}{l}
|\ifchilddoc|\\
|\addtocounter{page}{-1}|\\
\textit{code for banner page}\\
|\newpage|\\
|\||fi|
\end{tabular}
\end{center}
%
Here one could write a message such as:
\begin{center}
|This is the part \childdocname{} of \childdocjob{}.|
\end{center}

%%%%%%%%%%%%%%%%%%%%%%%%%%%%%%%%%%%%%%%%%%%%%%%%%%%%%%%%%%%%%%%%%%%%%%%%%%%%%%%%
\subsection{Flags}
\label{sec:flags}

The package makes it easy to generate different versions
of the main or child documents.
To this end compilation flags can be defined
and assigned different default values.
They will be particularly useful in conjunction
with the forwarding mechanism described in \secref{sec:forward}.

For example, it may be useful to have a flag |\version|
which can be set to |draft| or |final|.
The document source will contain some conditional code
depending on the value of |\version|.
Suppose further, the flag should default to |final| for the main file
and to |draft| for child files
which is a natural assignment for editing the document.
This is achieved by placing the following code
in the preamble of the main document
(below the |\childdocmain| directive):
%
\begin{center}
\begin{tabular}{l}
|\ifchilddoc|\\
|\providecommand{\version}{draft}|\\
|\||else|\\
|\providecommand{\version}{final}|\\
|\||fi|
\end{tabular}
\end{center}
%
The definition by |\providecommand| makes sure
that previous definitions are not overwritten.
Further statements |\providecommand{\version}{...}|
can thus be added before the above code to override it.

For the main file, one might add a line
(between |\childdocmain| and the above block)
%
\begin{center}
|%\ifchilddoc\||else\providecommand{\version}{draft}\||fi|
\end{center}
%
which can be uncommented to produce a draft version.
Likewise one can add a line to the very top of a child file
(above the |\childdocof{|\textit{main}|}| directive)
%
\begin{center}
|%\providecommand{\version}{final}|
\end{center}
%
which can be uncommented to produce the final version of this child document.

%%%%%%%%%%%%%%%%%%%%%%%%%%%%%%%%%%%%%%%%%%%%%%%%%%%%%%%%%%%%%%%%%%%%%%%%%%%%%%%%
\subsection{Forwarding}
\label{sec:forward}

Different versions of the main or child documents
using compilation flags as described in \secref{sec:flags}
can be (permanently) stored in different files
for convenient compilation, viewing and distribution.
To this end, the package defines a command
to pass on compilation to a different file:

%%%%%%%%%%%%%%%%%%%%%%%%%%%%%%%%%%%%%%%%
\DescribeMacro{\childdocforward}
The command |\childdocforward| redirects processing to
another source file:
%
\begin{center}
\begin{tabular}{l}
|\input{childdoc.def}|\\
|\childdocforward[|\textit{main}|]{|\textit{dest}|}|\\
\end{tabular}
\end{center}
%
The argument \textit{dest} is the destination file
(without extension).
It should be the main file or one of the child files.
Note that further \textsf{childdoc} directives
such as |\childdocof| and |\childdocforward|
in the indicated file will be processed in this form.
The optional argument \textit{main}
passes on directly to the main file \textit{main}
while pretending to compile the child \textit{dest}.
This form behaves as if \textit{dest}
issues |\childdocof{|\textit{main}|}| right away,
and no further \textsf{childdoc} directives will be processed.

%%%%%%%%%%%%%%%%%%%%%%%%%%%%%%%%%%%%%%%%
\DescribeMacro{\...prefix}
In the alternative form |\childdocforwardprefix|,
%
\begin{center}
\begin{tabular}{l}
|\input{childdoc.def}|\\
|\childdocforwardprefix[|\textit{main}|]{|\textit{prefix}|}{|\textit{dest}|}|
\end{tabular}
\end{center}
%
the destination file is determined by a pattern
depending on the current file:
To make this work, the current file must be called
`{\textit{prefix}\hspace{0.2em}\textit{suffix}}'
with \textit{prefix} matching precisely the argument.
Processing is then passed on to the file
`{\textit{dest}\hspace{0.2em}\textit{suffix}}'.
Surely, the same effect is achieved by
directly specifying the
argument `{\textit{dest}\hspace{0.2em}\textit{suffix}}'
in the first form.
However, that requires to set up a different file
for each child. With the alternative form of the command
all these files can have exactly the same content
which simplifies setting them up and maintaining them.

For example, the following file |draft.tex|
with a compilation flag |\version| as described in \secref{sec:flags}
compiles the main document as a draft:
%
\begin{center}
\begin{tabular}{l}
|\def\version{draft}|\\
|\input{childdoc.def}|\\
|\childdocforward{|\textit{main}|}|
\end{tabular}
\end{center}
%
Likewise, the following files |final|\textit{nn}|.tex|
compile the final version of the child document
|child|\textit{nn}|.tex|:
%
\begin{center}
\begin{tabular}{l}
|\def\version{final}|\\
|\input{childdoc.def}|\\
|\childdocforwardprefix{final}{child}|
\end{tabular}
\end{center}
%

Note that when several versions of a main file and/or of each child file
are to be generated, it may be convenient to set up a |Makefile| or
shell script to automatise the process.

%%%%%%%%%%%%%%%%%%%%%%%%%%%%%%%%%%%%%%%%%%%%%%%%%%%%%%%%%%%%%%%%%%%%%%%%%%%%%%%%
\subsection{Command Line Processing}
\label{sec:commandline}

The effect of redirection files can also be achieved by invoking
the \LaTeX{} compiler with a more elaborate command line.
Most conveniently this should be done as part
of a shell script or a |Makefile|.

When using \textsf{childdoc} in the main file, the following
command lines effectively perform a redirection
(note that depending on the shell being used,
backslashes may have to be doubled: `|\|' $\to$ `|\\|'):
%
\begin{center}
|... -jobname "|\textit{target}|" |\\|"|[\textit{flags}]%
|\input{childdoc.def}\childdocforward[|\textit{main}|]{|\textit{dest}|}"|
\end{center}
%
Here \textit{target} is the name of the output file,
\textit{main} is the name of the main file
and \textit{dest} is the name of the main or child file to be processed
(all filenames without extensions).
The optional argument \textit{main} can be omitted
if \textit{main} matches \textit{dest}.
Optionally, compilation \textit{flags} can be defined via |\def| commands.
This command line makes the \TeX{} engine believe
it is compiling the file \textit{target}
whose content is specified as the latter parameter.
The provided code then forwards the processing to
\textit{main} or \textit{dest} as described in \secref{sec:forward}.

%%%%%%%%%%%%%%%%%%%%%%%%%%%%%%%%%%%%%%%%%%%%%%%%%%%%%%%%%%%%%%%%%%%%%%%%%%%%%%%%
\subsection{Include by Input}
\label{sec:input}

Including child documents by |\include| has some restrictions by design.
Most notably, the content of a child document always occupies
its own set of pages; pages cannot be shared between child documents.
Usually, this behaviour makes perfect sense
because each child document contain an essential part of the document.
However, in some situations it may be desirable to compose
a document from a collection of parts
without having mandatory page breaks between then.
For this case, the package
provides a mechanism to include parts
by |\input| which can also be processed individually.
However, by construction this mechanism
requires manual handling of the content to be output.

%%%%%%%%%%%%%%%%%%%%%%%%%%%%%%%%%%%%%%%%
\DescribeMacro{\ifchilddocmanual}
The main file should be prepared as usual, see \secref{sec:include}.
However, the document body must make a distinction
between processing of an individual part and of the main document, e.g.:
%
\begin{center}
\begin{tabular}{l}
|\ifchilddocmanual|\\
|\input{\childdocname}|\\
|\||else|\\
\textit{document body with }|\input{|\textit{part}|}|\\
|\||fi|
\end{tabular}
\end{center}
%
The conditional |\ifchilddocmanual| is true whenever
a part to be included by |\input| is being compiled,
and the name of the part is stored in |\childdocname|.

%%%%%%%%%%%%%%%%%%%%%%%%%%%%%%%%%%%%%%%%
\DescribeMacro{\childdocby}
Each part to be included by |\input| should start with:
%
\begin{center}
\begin{tabular}{l}
|\input{childdoc.def}|\\
|\childdocby{|\textit{main}|}|\\
\end{tabular}
\end{center}
%
The directive |\childdocby| is similar to |\childdocof|
described in \secref{sec:include},
but the subsequent selection of content must be done manually.
To that end, both |\ifchilddoc| and |\ifchilddocmanual|
will be true upon processing of a part,
and the name of the part is stored in |\childdocname|.
Note that |\jobname| will be set to the filename of the current part
so that each part receives an individual |.aux| file
that does not interfere with the |.aux| file(s) of the main document.
This behaviour can be altered by the alternative form
|\childdocby[*]{|\textit{main}|}| (with a non-empty optional argument)
which uses the |.aux| file of the main document
by setting |\jobname| to \textit{main}.

%%%%%%%%%%%%%%%%%%%%%%%%%%%%%%%%%%%%%%%%%%%%%%%%%%%%%%%%%%%%%%%%%%%%%%%%%%%%%%%%
\subsection{Driver Development}
\label{sec:driver}

The \textsf{childdoc} mechanism can also be use for the development
of definition files such as \LaTeX{} styles or classes.
This case differs from the above setup with multiple parts
included by |\include| in that no |\includeonly| should be invoked.
This can be achieved by starting the include file
(before |\ProvidesPackage|) with:
%
\begin{center}
\begin{tabular}{l}
|\input{childdoc.def}|\\
|\childdocforward{|\textit{main}|}|\\
\end{tabular}
\end{center}
%
or alternatively with:
%
\begin{center}
\begin{tabular}{l}
|\input{childdoc.def}|\\
|\childdocby{|\textit{main}|}|\\
\end{tabular}
\end{center}
%
Both forms have slightly different effects as described above.
The main file is prepared as usual, see \secref{sec:include}.

%%%%%%%%%%%%%%%%%%%%%%%%%%%%%%%%%%%%%%%%%%%%%%%%%%%%%%%%%%%%%%%%%%%%%%%%%%%%%%%%
\subsection{Legacy Detection}
\label{sec:detection}

The directive |\childdocmain| in the main file can detect
whether the complete document or merely a child is to be compiled
even without using the directive |\childdocof|.
This method is deprecated because it is less robust
and there is no compelling reason to use it;
it is merely provided for backward compatibility
and it may be removed in future versions.

If the detection mechanism is to be used,
it is mandatory to correctly specify
the filename of the main file as the argument of |\childdocmain|:
%
\begin{center}
\begin{tabular}{l}
|\input{childdoc.def}|\\
|\childdocmain{|\textit{main}|}|\\
\end{tabular}
\end{center}
%
If |\jobname| does not match the argument \textit{main} of |\childdocmain|,
it is assumed that |\jobname| points to the child file to be compiled.
When using |\childdocmain| with the main file specified as argument,
it suffices to start a child file
with just |\input{|\textit{main}|}|
without loading of the package and using |\childdocof|.
If instead all processing is done
with the appropriate \textsf{childdoc} directives,
the argument of \textit{main} of |\childdocmain| can be empty.

An alternative version of the command line processing described
in \secref{sec:commandline} using the detection mechanism reads:
%
\begin{center}
|... -jobname "|\textit{target}|" "|[\textit{flags}]%
[|\def\jobname{|\textit{dest}|}|]|\input{|\textit{main}|}"|
\end{center}

%%%%%%%%%%%%%%%%%%%%%%%%%%%%%%%%%%%%%%%%%%%%%%%%%%%%%%%%%%%%%%%%%%%%%%%%%%%%%%%%
\subsection{Manual Code}
\label{sec:manual}

In case one cannot be certain whether the definitions file |childdoc.def|
is installed on the target \TeX{} distribution
and one prefers not to ship it,
it is conceivable to paste a few relevant commands into the sources.

To that end, drop all statements |\input{childdoc.def}|
and perform the replacements as outlined below.
Instead of |\childdocmain{|\textit{main}|}| add the following code
to the top of the main file:
%
\begin{center}
\begin{tabular}{l}
|\||ifdefined\childdocname\endinput\||fi\newif\ifchilddoc|\\
|\edef\childdocname{\scantokens\expandafter{\jobname\noexpand}}|\\
|\def\childdocmain{|\textit{main}|}\||ifx\childdocmain\childdocname\||else|\\
|\childdoctrue\includeonly{\childdocname}\let\jobname\childdocmain\||fi|\\
\end{tabular}
\end{center}
%
Instead of |\childdocof{|\textit{main}|}| just include the main file
at the top of each child file:
%
\begin{center}
|\input{|\textit{main}|}|
\end{center}
%
A simple redirection |\childdocforward{|\textit{dest}|}| is achieved by:
%
\begin{center}
|\def\jobname{|\textit{dest}|}\input{\jobname}|
\end{center}
%
The redirection with prefix
|\childdocforwardprefix[|\textit{prefix}|]{|\textit{dest}|}|
is accomplished by:
%
\begin{center}
\begin{tabular}{l}
|{\edef\jobname{\scantokens\expandafter{\jobname\noexpand}}|\\
|\def\redirectjob |\textit{prefix}|#1~~~{\gdef\jobname{|\textit{dest}|#1}}|\\
|\expandafter\redirectjob\jobname~~~}\input{\jobname}|
\end{tabular}
\end{center}

In an alternative approach,
child documents can be compiled by a specific command line
without additional code or specific definitions:
%
\begin{center}
|... -jobname "|\textit{target}|" "|[\textit{flags}]%
|\includeonly{|\textit{dest}|}\input{|\textit{main}|}"|
\end{center}
%

%%%%%%%%%%%%%%%%%%%%%%%%%%%%%%%%%%%%%%%%%%%%%%%%%%%%%%%%%%%%%%%%%%%%%%%%%%%%%%%%
%%%%%%%%%%%%%%%%%%%%%%%%%%%%%%%%%%%%%%%%%%%%%%%%%%%%%%%%%%%%%%%%%%%%%%%%%%%%%%%%
\section{Information}

%%%%%%%%%%%%%%%%%%%%%%%%%%%%%%%%%%%%%%%%%%%%%%%%%%%%%%%%%%%%%%%%%%%%%%%%%%%%%%%%
\subsection{Copyright}

Copyright \copyright{} 2017--2018 Niklas Beisert

This work may be distributed and/or modified under the
conditions of the \LaTeX{} Project Public License, either version 1.3
of this license or (at your option) any later version.
The latest version of this license is in
  \url{http://www.latex-project.org/lppl.txt}
and version 1.3 or later is part of all distributions of \LaTeX{}
version 2005/12/01 or later.

This work has the LPPL maintenance status `maintained'.

The Current Maintainer of this work is Niklas Beisert.

This work consists of the files |README.txt|, |childdoc.ins| and |childdoc.dtx|
as well as the derived files |childdoc.def|, |cdocsamp.tex|
with |cdocsch1.tex|, |cdocsch2.tex|, |cdocspt3.tex|, |cdocspt4.tex|,
|cdocsdrf.tex|, |cdocsfn1.tex|, |cdocsfn2.tex|
as well as |childdoc.pdf|.

%%%%%%%%%%%%%%%%%%%%%%%%%%%%%%%%%%%%%%%%%%%%%%%%%%%%%%%%%%%%%%%%%%%%%%%%%%%%%%%%
\subsection{Files and Installation}

The package consists of the files:
%
\begin{center}
\begin{tabular}{ll}
    |README.txt|   & readme file \\
    |childdoc.ins| & installation file \\
    |childdoc.dtx| & source file \\
    |childdoc.def| & definition file \\
    |cdocsamp.tex| & sample main file \\
    |cdocsch1.tex| & sample include file \\
    |cdocsch2.tex| & sample include file \\
    |cdocspt3.tex| & sample part file \\
    |cdocspt4.tex| & sample part file \\
    |cdocsdrf.tex| & sample redirection file \\
    |cdocsfn1.tex| & sample redirection file \\
    |cdocsfn2.tex| & sample redirection file \\
    |childdoc.pdf| & manual
\end{tabular}
\end{center}
%
The distribution consists of the files
|README.txt|, |childdoc.ins| and |childdoc.dtx|.
%
\begin{itemize}
\item
Run (pdf)\LaTeX{} on |childdoc.dtx|
to compile the manual |childdoc.pdf| (this file).
\item
Run \LaTeX{} on |childdoc.ins| to create the definitions file |childdoc.def|
and the sample |cdocsamp.tex| with include files
|cdocsch1.tex|, |cdocsch2.tex|, |cdocspt3.tex|, |cdocspt4.tex|,
|cdocsdrf.tex|, |cdocsfn1.tex|, |cdocsfn2.tex|.
Then copy the file |childdoc.def| to an appropriate directory of your \LaTeX{}
distribution, e.g.\ \textit{texmf-root}|/tex/latex/childdoc|.
\end{itemize}

%%%%%%%%%%%%%%%%%%%%%%%%%%%%%%%%%%%%%%%%%%%%%%%%%%%%%%%%%%%%%%%%%%%%%%%%%%%%%%%%
\subsection{Related CTAN Packages}

There are several other packages which offer a similar functionality:
%
\begin{itemize}
\item
The packages
\href{http://ctan.org/pkg/docmute}{\textsf{docmute}},
\href{http://ctan.org/pkg/includex}{\textsf{includex}} and
\href{http://ctan.org/pkg/standalone}{\textsf{standalone}}
provide commands to include only the document body of
a child file thus allowing both files to be compiled individually.
\item
The packages \href{http://ctan.org/pkg/subdocs}{\textsf{subdocs}}
and \href{http://ctan.org/pkg/subfiles}{\textsf{subfiles}}
provide structures in which the main and child documents can be
encapsulated and allowing them to be compiled individually.
The inclusion mechanism is different from the conventional |\include|.
\item
The package \href{http://ctan.org/pkg/combine}{\textsf{combine}}
is an elaborate solution to combine several documents into one.
\end{itemize}
%
See also the CTAN topic \href{http://ctan.org/topic/subdocs}{\textsf{subdocs}}
for further related packages.
The present package differs from the above solutions in that
a document structure constructed with the conventional |\include| mechanism
just needs two extra commands at the top of every file
such that all constituent files can be compiled individually.

%%%%%%%%%%%%%%%%%%%%%%%%%%%%%%%%%%%%%%%%%%%%%%%%%%%%%%%%%%%%%%%%%%%%%%%%%%%%%%%%
%\subsection{Feature Suggestions}
%
%The following is a list of features which may be useful for future
%versions of this package:
%%
%\begin{itemize}
%\item
%\ldots
%\end{itemize}

%%%%%%%%%%%%%%%%%%%%%%%%%%%%%%%%%%%%%%%%%%%%%%%%%%%%%%%%%%%%%%%%%%%%%%%%%%%%%%%%
\subsection{Revision History}

%%%%%%%%%%%%%%%%%%%%%%%%%%%%%%%%%%%%%%%%
\paragraph{v2.0:} 2018/12/30

\begin{itemize}
\item
immediate forward processing
\item
added |\childdocby| mechanism
\item
manual restructured
\end{itemize}

%%%%%%%%%%%%%%%%%%%%%%%%%%%%%%%%%%%%%%%%
\paragraph{v1.6:} 2018/01/17

\begin{itemize}
\item
application for development of include files
\item
corrections to manual
\end{itemize}

%%%%%%%%%%%%%%%%%%%%%%%%%%%%%%%%%%%%%%%%
\paragraph{v1.5:} 2017/05/21

\begin{itemize}
\item
more complete structuring introduced
\item
|\childdocof| introduced
\item
|\childdoc| renamed to |\childdocmain|
\item
|\childredirect| renamed to |\childdocforward| and |\childdocforwardprefix|
and functionality expanded
\end{itemize}

%%%%%%%%%%%%%%%%%%%%%%%%%%%%%%%%%%%%%%%%
\paragraph{v1.0:} 2017/04/27

\begin{itemize}
\item
manual and install package
\item
first version published on CTAN
\end{itemize}

%%%%%%%%%%%%%%%%%%%%%%%%%%%%%%%%%%%%%%%%
\paragraph{v0.6:} 2017/04/26

\begin{itemize}
\item
redirection mechanism added
\end{itemize}

%%%%%%%%%%%%%%%%%%%%%%%%%%%%%%%%%%%%%%%%
\paragraph{v0.5:} 2017/04/26

\begin{itemize}
\item
functionality in definition file
\end{itemize}


%%%%%%%%%%%%%%%%%%%%%%%%%%%%%%%%%%%%%%%%%%%%%%%%%%%%%%%%%%%%%%%%%%%%%%%%%%%%%%%%
%%%%%%%%%%%%%%%%%%%%%%%%%%%%%%%%%%%%%%%%%%%%%%%%%%%%%%%%%%%%%%%%%%%%%%%%%%%%%%%%
%%%%%%%%%%%%%%%%%%%%%%%%%%%%%%%%%%%%%%%%%%%%%%%%%%%%%%%%%%%%%%%%%%%%%%%%%%%%%%%%
\appendix

\settowidth\MacroIndent{\rmfamily\scriptsize 000\ }

 \DocInput{childdoc.dtx}

\end{document}
%</driver>
% \fi
%
% %%%%%%%%%%%%%%%%%%%%%%%%%%%%%%%%%%%%%%%%%%%%%%%%%%%%%%%%%%%%%%%%%%%%%%%%%%%%%%
% %%%%%%%%%%%%%%%%%%%%%%%%%%%%%%%%%%%%%%%%%%%%%%%%%%%%%%%%%%%%%%%%%%%%%%%%%%%%%%
% \section{Sample}
%\iffalse
%<*samplemain>
%\fi
%
% The following presents a sample document
% with two chapters, two parts, a title page,
% a compile flag as well as three forwarding files to set the flag.
% It consists of eight |.tex| files:
% \begin{center}
% \begin{tabular}{ll}
% |cdocsamp.tex|&main file\\
% |cdocsch1.tex|&include file for chapter 1\\
% |cdocsch2.tex|&include file for chapter 2\\
% |cdocspt3.tex|&include file for part 3\\
% |cdocspt4.tex|&include file for part 4\\
% |cdocsdrf.tex|&forwarding file for main file in draft mode\\
% |cdocsfi1.tex|&forwarding file for final version of chapter 1\\
% |cdocsfi2.tex|&forwarding file for final version of chapter 2\\
% \end{tabular}
% \end{center}
% Each of the eight files can be compiled directly by the \LaTeX{} compiler.
%
% %%%%%%%%%%%%%%%%%%%%%%%%%%%%%%%%%%%%%%
% \paragraph{Main File.}
%
% The main file is called |cdocsamp.tex|.
%
% Load the \textsf{childdoc} definitions and
% declare the filename for the main document:
%    \begin{macrocode}
\input{childdoc.def}
\childdocmain{}
%    \end{macrocode}

% Optional override for |\version| flag:
%    \begin{macrocode}
%%\ifchilddoc\else\providecommand{\version}{draft}\fi
%    \end{macrocode}

% Define the default values for the |\version| flag
% (|final| for the main file and |draft| for childs):
%    \begin{macrocode}
\ifchilddoc
\providecommand{\version}{draft}
\else
\providecommand{\version}{final}
\fi
%    \end{macrocode}

% Load the standard document class:
%    \begin{macrocode}
\documentclass[12pt]{article}
%    \end{macrocode}

% Start the document body:
%    \begin{macrocode}
\begin{document}
%    \end{macrocode}

% Declare a title page.
% Print title, part of document being processed and version flag:
%    \begin{macrocode}
\addtocounter{page}{-1}
\begin{center}
{\LARGE\bfseries{}childdoc example\par}
\vspace{1cm}
\ifchilddoc
\ifchilddocmanual part\else chapter\fi:
`\childdocname' of `\childdocjob'\par
\else
main document: `\childdocjob'\par
\fi
version: \version\par
\end{center}
\newpage
%    \end{macrocode}

% Manually include selected file,
% otherwise process as usual:
%    \begin{macrocode}
\ifchilddocmanual
\section*{part `\childdocname'}
\input{\childdocname}
\else
%    \end{macrocode}

% Include the two chapters:
%    \begin{macrocode}
\include{cdocsch1}
\include{cdocsch2}
%    \end{macrocode}

% Include the two parts unless only chapters should be displayed:
%    \begin{macrocode}
\ifchilddoc\else
\section{part three}
\input{cdocspt3}
\section{part four}
\input{cdocspt4}
\fi
%    \end{macrocode}

% Process as usual until here:
%    \begin{macrocode}
\fi
%    \end{macrocode}

% End of document body:
%    \begin{macrocode}
\end{document}
%    \end{macrocode}
%\iffalse
%</samplemain>
%\fi
%
% %%%%%%%%%%%%%%%%%%%%%%%%%%%%%%%%%%%%%%
% \paragraph{Chapter Include Files.}
%
% The include files are called |cdocsch1.tex| and |cdocsch2.tex|.
%
%\iffalse
%<*samplechap1|samplechap2>
%\fi

% Optional override for |\version| flag:
%    \begin{macrocode}
%%\providecommand{\version}{final}
%    \end{macrocode}

% Include the main document:
%    \begin{macrocode}
\input{childdoc.def}
\childdocof{cdocsamp}
%    \end{macrocode}

%\iffalse
%</samplechap1|samplechap2>
%\fi
%
%\iffalse
%<*samplechap1>
%\fi
% Some text for chapter 1:
%    \begin{macrocode}
\section{one}
some text in chapter one
%    \end{macrocode}

%\iffalse
%</samplechap1>
%\fi
% Some text for chapter 2:
%\iffalse
%<*samplechap2>
%\fi
%    \begin{macrocode}
\section{two}
more text in chapter two
%    \end{macrocode}

%\iffalse
%</samplechap2>
%\fi
%
% %%%%%%%%%%%%%%%%%%%%%%%%%%%%%%%%%%%%%%
% \paragraph{Part Include Files.}
%
% The include files are called |cdocspt3.tex| and |cdocspt4.tex|.
%
%\iffalse
%<*samplepart3|samplepart4>
%\fi

% Optional override for |\version| flag:
%    \begin{macrocode}
%%\providecommand{\version}{final}
%    \end{macrocode}

% Include the main document:
%    \begin{macrocode}
\input{childdoc.def}
\childdocby{cdocsamp}
%    \end{macrocode}

%\iffalse
%</samplepart3|samplepart4>
%\fi
%
%\iffalse
%<*samplepart3>
%\fi
% Some text for part 3:
%    \begin{macrocode}
some text in part three
%    \end{macrocode}

%\iffalse
%</samplepart3>
%\fi
% Some text for part 4:
%\iffalse
%<*samplepart4>
%\fi
%    \begin{macrocode}
more text in part four
%    \end{macrocode}

%\iffalse
%</samplepart4>
%\fi
%
% %%%%%%%%%%%%%%%%%%%%%%%%%%%%%%%%%%%%%%
% \paragraph{Forwarding for a Complete Draft.}
%
% The following forwarding file |cdocsdrf.tex|
% compiles the main document in draft mode:
%\iffalse
%<*sampledraft>
%\fi
%    \begin{macrocode}
\def\version{draft}
\input{childdoc.def}
\childdocforward{cdocsamp}
%    \end{macrocode}

%\iffalse
%</sampledraft>
%\fi
%
% %%%%%%%%%%%%%%%%%%%%%%%%%%%%%%%%%%%%%%
% \paragraph{Forwarding for Final Version of the Chapters.}
%
% The following forwarding files |cdocsfn1.tex| and |cdocsfn2.tex|
% (with identical content)
% compile the final versions of the child documents
% |cdocsch1.tex| and |cdocsch2.tex|, respectively:
%\iffalse
%<*samplefinal>
%\fi
%    \begin{macrocode}
\def\version{final}
\input{childdoc.def}
\childdocforwardprefix[cdocsamp]{cdocsfn}{cdocsch}
%    \end{macrocode}

%\iffalse
%</samplefinal>
%\fi
%
% %%%%%%%%%%%%%%%%%%%%%%%%%%%%%%%%%%%%%%
% \paragraph{Command Line Processing.}
%
% The following three command lines generate the output files
% |cdocscld|, |cdocscl1| and |cdocscl2|
% which should be identical to
% |cdocsdrf|, |cdocsch1| and |cdocsfn2|, respectively:
% \begin{center}
% \begin{tabular}{l}
% |latex -jobname cdocscld \|\\
% |  "\def\version{draft}\input{childdoc.def}\childdocforward{cdocsamp}"|\\
% |latex -jobname cdocscl1 \|\\
% |  "\input{childdoc.def}\childdocforward[cdocsamp]{cdocsch1}"|\\
% |latex -jobname cdocscl2 \|\\
% |  "\def\version{final}\input{childdoc.def}\childdocforward{cdocsch2}"|
% \end{tabular}
% \end{center}
% Note that the trailing backslash on each first line
% merely continues the input to the second line
% (for convenient cut ant paste).
% Furthermore, the command |latex| can be replaced by any
% of its alternative versions such as |pdflatex|.
%
% %%%%%%%%%%%%%%%%%%%%%%%%%%%%%%%%%%%%%%%%%%%%%%%%%%%%%%%%%%%%%%%%%%%%%%%%%%%%%%
% %%%%%%%%%%%%%%%%%%%%%%%%%%%%%%%%%%%%%%%%%%%%%%%%%%%%%%%%%%%%%%%%%%%%%%%%%%%%%%
% \section{Implementation}
%\iffalse
%<*package>
%\fi
%
% This section describes the definitions file |childdoc.def|.

% The definitions cannot be loaded using |\usepackage| or |\RequirePackage|
% which has a mechanism to prevent loading a style file more than once.
% When loading the definitions by means of |\input|
% multiple instances have to be prevented manually:
%\iffalse
%This code needs to be before the `\ProvidesFile' directive
%which is defined at the beginning of this file.
%Therefore it is also placed there and commented out here.
%</package>
%<*discard>
%\fi
%    \begin{macrocode}
\ifdefined\childdocmain\endinput\fi
%    \end{macrocode}
%\iffalse
%</discard>
%<*package>
%\fi
%
% \macro{\ifchilddoc}
% \macro{\ifchilddocmanual}
% The conditional |\ifchilddoc| tells whether a
% child (true) or main (false) document is being compiled.
% The conditional |\ifchilddocmanual| tells whether
% the |\includeonly| mechanism is used (false) or
% the selection of child files must be performed manually (true).
% The definitions initialise to false:
%    \begin{macrocode}
\newif\ifchilddoc
\newif\ifchilddocmanual
%    \end{macrocode}

% \macro{\childdocname}
% \macro{\childdocjob}
% The macro |\childdocname| stores the name of the main document
% to be compiled. The macro |\childdocjob| stores the name of
% the document on which the \LaTeX{} compiler was originally invoked.
% The content of |\jobname| cannot be compared
% to filenames specified in the source due to different catcodes.
% The following code rescans |\jobname|, stores the result
% in |\childdocname| and saves a copy in |\childdocjob|:
%    \begin{macrocode}
\edef\childdocname{\scantokens\expandafter{\jobname\noexpand}}
\let\childdocjob\childdocname
%    \end{macrocode}

% \macro{\childdocdisable}
% The macro |\childdocdisable| prevents the main file
% from being processed more than once.
% At this stage, the main document command |\childdocmain|
% is assumed to be called once again where it should do nothing.
% Any subsequent call to it should prevent
% a secondary processing of the main document
% It overwrites the forwarding commands
% |\childdocof| and |\childdocforward|
% with empty macros to prevent further inclusions of the main document:
%    \begin{macrocode}
\newcommand{\childdocdisable}
{
  \renewcommand{\childdocmain}[1]{\renewcommand{\childdocmain}[1]{\endinput}}
  \renewcommand{\childdocof}[1]{}
  \renewcommand{\childdocby}[2][]{}
  \renewcommand{\childdocforward}[2][]{}
  \renewcommand{\childdocdisable}{}
}
%    \end{macrocode}

% \macro{\childdocmain}
% The macro |\childdocmain| is to be called at the top of the main file
% with nothing or the main filename (without extension) as argument.
% First, it breaks loops.
% If the argument is not empty and does not match |\childdocname|
% (which is set by the first inclusion of |childdoc.def|),
% |\ifchilddoc| is set to true, |\includeonly| is applied to the child file
% and |\jobname| is set to the main file
% (for proper handling of |.aux| files):
%    \begin{macrocode}
\newcommand{\childdocmain}[1]
{
  \childdocdisable\childdocmain{}
  \if?#1?\else
    \begingroup
      \def\childdoctmp{#1}
      \ifx\childdoctmp\childdocname
        \def\childdoctmp{}
      \else
        \def\childdoctmp
        {
          \childdoctrue
          \includeonly{\childdocname}
          \def\childdocjob{#1}
          \def\jobname{#1}
        }
      \fi
      \expandafter
    \endgroup
    \childdoctmp
  \fi
}
%    \end{macrocode}

% \macro{\childdocof}
% The command |\childdocof| redirects
% compilation to the main file |#1|.
%    \begin{macrocode}
\newcommand{\childdocof}[1]
{
  \childdocdisable
  \childdoctrue
  \includeonly{\childdocname}
  \def\jobname{#1}
  \def\childdocjob{#1}
  \input{#1}
}
%    \end{macrocode}

% \macro{\childdocby}
% The command |\childdocby| ....
%    \begin{macrocode}
\newcommand{\childdocby}[2][]
{
  \childdocdisable
  \childdoctrue
  \childdocmanualtrue
  \if?#1?\else
    \def\jobname{#2}
  \fi
  \def\childdocjob{#2}
  \input{#2}
  \endinput
}
%    \end{macrocode}

% \macro{\childdocforward}
% The command |\childdocforward| redirects
% compilation to the main file or
% (if the optional argument is given) a child file.
% Parameters are set as if the main file
% or a child file starting with |\childdocof| was compiled.
% Then compilation is handed over to the main file:
%    \begin{macrocode}
\newcommand{\childdocforward}[2][]
{
  \begingroup
    \if?#1?
      \def\childdoctmp
      {
        \def\childdocname{#2}
        \def\childdocjob{#2}
        \def\jobname{#2}
        \input{#2}
        \endinput
      }
    \else
      \def\childdoctmp
      {
        \childdocdisable
        \def\childdocname{#2}
        \childdoctrue
        \includeonly{#2}
        \def\childdocjob{#1}
        \def\jobname{#1}
        \input{#1}
        \endinput
      }
    \fi
    \expandafter
  \endgroup
  \childdoctmp
}
%    \end{macrocode}

% \macro{\childdocforwardprefix}
% The command |\childdocforwardprefix| redirects
% compilation to the main or a child file by means of a pattern.
% The prefix |#1| in the current filename is replaced by |#2|
% and the suffix of the current filename is kept
% (it is assumed that the filename does not contain the substring `|~~~|'
% which is used as a delimiter).
% Compilation is handed over to the new file by |\childdocforward|:
%    \begin{macrocode}
\newcommand{\childdocforwardprefix}[3][]
{
  \begingroup
    \def\childdocextract #2##1~~~{\def\childdoctmp{\childdocforward[#1]{#3##1}}}
    \expandafter\childdocextract\childdocname~~~
    \expandafter
  \endgroup
  \childdoctmp
}
%    \end{macrocode}

% \macro{\childdoc}
% The deprecated macro |\childdoc| is a legacy version of |\childdocmain|:
%    \begin{macrocode}
\newcommand{\childdoc}{\childdocmain}
%    \end{macrocode}

% \macro{\childdocredirect}
% The deprecated macro |\childdocredirect| is a legacy version
% of |\childdocforward| and |\childdocforwardprefix|:
%    \begin{macrocode}
\newcommand{\childdocredirect}[2][]
{
  \begingroup
    \if?#1?
      \def\childdoctmp{\childdocforward{#2}}
    \else
      \def\childdoctmp{\childdocforwardprefix{#1}{#2}}
    \fi
    \expandafter
  \endgroup
  \childdoctmp
}
%    \end{macrocode}

%\iffalse
%</package>
%\fi
%
\endinput
\childdocforward[cdocsamp]{cdocsch1}"|\\
% |latex -jobname cdocscl2 \|\\
% |  "\def\version{final}% \iffalse
%
% childdoc.dtx Copyright (C) 2017-2018 Niklas Beisert
%
% This work may be distributed and/or modified under the
% conditions of the LaTeX Project Public License, either version 1.3
% of this license or (at your option) any later version.
% The latest version of this license is in
%   http://www.latex-project.org/lppl.txt
% and version 1.3 or later is part of all distributions of LaTeX
% version 2005/12/01 or later.
%
% This work has the LPPL maintenance status `maintained'.
%
% The Current Maintainer of this work is Niklas Beisert.
%
% This work consists of the files childdoc.dtx and childdoc.ins
% and the derived files childdoc.def and cdocsamp.tex with
% cdocsch1.tex, cdocsch2.tex, cdocsdrf.tex, cdocsfn1.tex, cdocsfn2.tex.
%
%<package>\ifdefined\childdocmain\endinput\fi
%<package>\ProvidesFile{childdoc.def}[2018/12/30 v2.0 child document driver]
%<samplemain>\ProvidesFile{cdocsamp.tex}[2018/12/30 v2.0 sample for childdoc]
%<*driver>
%\ProvidesFile{childdoc.drv}[2018/12/30 v2.0 childdoc reference manual file]
\PassOptionsToClass{10pt,a4paper}{article}
\documentclass{ltxdoc}

\usepackage[margin=35mm]{geometry}
\usepackage{hyperref}
\usepackage{hyperxmp}
\usepackage[usenames]{color}

\hypersetup{colorlinks=true}
\hypersetup{pdfstartview=FitH}
\hypersetup{pdfpagemode=UseNone}
\hypersetup{pdfsource={}}
\hypersetup{pdflang={en-UK}}
\hypersetup{pdfcopyright={Copyright 2017-2018 Niklas Beisert.
  This work may be distributed and/or modified under the
  conditions of the LaTeX Project Public License, either version 1.3
  of this license or (at your option) any later version.}}
\hypersetup{pdflicenseurl={http://www.latex-project.org/lppl.txt}}
\hypersetup{pdfcontactaddress={ETH Zurich, ITP, HIT K,
  Wolfgang-Pauli-Strasse 27}}
\hypersetup{pdfcontactpostcode={8093}}
\hypersetup{pdfcontactcity={Zurich}}
\hypersetup{pdfcontactcountry={Switzerland}}
\hypersetup{pdfcontactemail={nbeisert@itp.phys.ethz.ch}}
\hypersetup{pdfcontacturl={http://people.phys.ethz.ch/\xmptilde nbeisert/}}

\newcommand{\secref}[1]{\hyperref[#1]{section \ref*{#1}}}

\parskip1ex
\parindent0pt
\let\olditemize\itemize
\def\itemize{\olditemize\parskip0pt}

\begin{document}

\title{The \textsf{childdoc} Package}
\hypersetup{pdftitle={The childdoc Package}}
\author{Niklas Beisert\\[2ex]
  Institut f\"ur Theoretische Physik\\
  Eidgen\"ossische Technische Hochschule Z\"urich\\
  Wolfgang-Pauli-Strasse 27, 8093 Z\"urich, Switzerland\\[1ex]
  \href{mailto:nbeisert@itp.phys.ethz.ch}
  {\texttt{nbeisert@itp.phys.ethz.ch}}}
\hypersetup{pdfauthor={Niklas Beisert}}
\hypersetup{pdfsubject={Manual for the LaTeX2e Package childdoc}}
\date{30 December 2018, \textsf{v2.0}}
\maketitle

\begin{abstract}\noindent
\textsf{childdoc} is a \LaTeXe{} package
that enables the direct compilation
of document sections included by |\include|
to individual files.
\end{abstract}

\begingroup
\parskip0ex
\tableofcontents
\endgroup

%%%%%%%%%%%%%%%%%%%%%%%%%%%%%%%%%%%%%%%%%%%%%%%%%%%%%%%%%%%%%%%%%%%%%%%%%%%%%%%%
%%%%%%%%%%%%%%%%%%%%%%%%%%%%%%%%%%%%%%%%%%%%%%%%%%%%%%%%%%%%%%%%%%%%%%%%%%%%%%%%
\section{Introduction}

\LaTeX{} provides a mechanism to structure a large document (such as a book)
into a main file and several child files (containing the chapters)
using the |\include| command.
This mechanism is beneficial for documents
which span hundreds of pages in order to
make the source file(s) more manageable.
Moreover, compilation can be restricted to
selected child files by means of the |\includeonly| command.
The latter feature can be used to reduce the compilation time while editing
(this was significantly more useful in the earlier days of \LaTeX{})
or to generate a smaller document which is easier to navigate.
Another application of |\includeonly| is to generate
documents consisting of selected parts of the complete document.

However, there are a few drawbacks of the plain |\include| mechanism:
\begin{itemize}
\item
The child files cannot be compiled on their own,
they can only be compiled via the main file.
A naive editing environment
(such as a text editor with an option
to have the current file processed by \LaTeX)
may require one to switch to the main file before compiling;
attempting to compile the child file produces errors.
\item
The main file must be modified (each time)
to adjust the |\includeonly| command
to the present needs. This easily leaves the main file in a messy state.
\item
The generated document will always carry the filename
of the main document. This is inconvenient if
several child files are to be compiled and
to be kept for distribution.
\end{itemize}

The present package provides a simple interface
to make child files individually compilable by \LaTeX{}.
Compiling a child file then has the same effect as compiling
the main file with an |\includeonly| command
to select the appropriate child.
Moreover the generated document will carry the name of the child
rather than the main file.
This resolves all three above issues.

This feature is meant to make the editing of books,
thesis documents and lecture notes somewhat more convenient.
However, the package can also be used efficiently for
composing a series of documents (such as exercise sheets)
which are typically distributed individually.
It then assists the author in generating the individual documents
(potentially in different versions)
as well as a document containing the collected series.
Another application is in developing style files
or other kinds of included material
where compilation of the style file could redirect
to a sample or test file.

%%%%%%%%%%%%%%%%%%%%%%%%%%%%%%%%%%%%%%%%%%%%%%%%%%%%%%%%%%%%%%%%%%%%%%%%%%%%%%%%
%%%%%%%%%%%%%%%%%%%%%%%%%%%%%%%%%%%%%%%%%%%%%%%%%%%%%%%%%%%%%%%%%%%%%%%%%%%%%%%%
\section{Usage}

First of all, the package \textsf{childdoc} is \emph{not} a standard
\LaTeXe{} |.sty| style file! Therefore it needs to be invoked in
a non-standard way.

%%%%%%%%%%%%%%%%%%%%%%%%%%%%%%%%%%%%%%%%%%%%%%%%%%%%%%%%%%%%%%%%%%%%%%%%%%%%%%%%
\subsection{Included Files}
\label{sec:include}

%%%%%%%%%%%%%%%%%%%%%%%%%%%%%%%%%%%%%%%%
\DescribeMacro{\childdocmain}
To use the package, add the commands
\begin{center}
\begin{tabular}{l}
|\input{childdoc.def}|\\
|\childdocmain{}|\\
\end{tabular}
\end{center}
at the very top of the main \LaTeX{} file,
in particular \emph{before} the |\documentclass| statement!
The argument of |\childdocmain| should be left empty
(but it must be present).

%%%%%%%%%%%%%%%%%%%%%%%%%%%%%%%%%%%%%%%%
\DescribeMacro{\childdocof}
Furthermore, add the commands
\begin{center}
\begin{tabular}{l}
|\input{childdoc.def}|\\
|\childdocof{|\textit{main}|}|\\
\end{tabular}
\end{center}
at the top of every child file \textit{child}
which is included by |\include{|\textit{child}|}|
from within the main file
(or at least for those files to be compiled individually).
The argument \textit{main} must be the filename of the main file.

There are a couple of
considerations in setting up the main and child documents:

%%%%%%%%%%%%%%%%%%%%%%%%%%%%%%%%%%%%%%%%
\paragraph{Restrictions.}

Please note the following restrictions:
\begin{itemize}
\item
|\childdocmain| must be called with one argument \textit{main}
to ensure compatibility with earlier version of the package.
It must either be empty (|\childdocmain{}|)
or precisely match the filename of the main file in which it is specified.
See \secref{sec:detection} for further information.
\item
The filename \textit{main} must be specified without the |.tex| extension.
\item
The filename \textit{main} is case sensitive
(even in case-insensitive file systems)
due to internal string comparison.
\item
The argument \textit{main} should be fully expanded, it cannot be a macro.
\item
Subdirectories and special characters should be avoided in filenames.
\item
The command |\childdocmain{|\textit{main}|}| must be followed by a whitespace.
It should not be followed immediately by another command
or by a comment mark `|%|'.
This is because the \TeX{} parser reads the token immediately following
the argument of |\childdocmain| and puts it
at the beginning of every child section;
however, a white\-space is ignored.
\end{itemize}

%%%%%%%%%%%%%%%%%%%%%%%%%%%%%%%%%%%%%%%%
\paragraph{Content of Main File.}

It is advisable to place all content in the child files included by |\include|.
Any output contained in the main file will appear in all child documents
unless suppressed manually;
it cannot be suppressed automatically by the |\includeonly| directive
and thus should normally be avoided.
A method to include some content in the main file
by means of conditional processing is described in \secref{sec:conditional}.

%%%%%%%%%%%%%%%%%%%%%%%%%%%%%%%%%%%%%%%%
\paragraph{Page Numbering.}

When only a part of the document is compiled,
the appropriate numbering of pages
(as well as other status parameters)
is determined from the |.aux| files.
The latter contain information from previous passes.
However this information needs to propagate through
all intermediate child documents.
Therefore the page numbering in child documents may well
be inconsistent until the complete document is compiled at least once.

A useful (if unconventional) way to always ensure a consistent
page numbering is to restart the numbering in each child document
and denote the pages by `\textit{child}|.|\textit{page}'
where \textit{child} represents the chapter/section number of the child file.
This can be achieved by the command
|\numberwithin{page}{|\textit{child}|}|
of the \textsf{amsmath} package
where \textit{child} can be |chapter| or |section|
depending on the chosen structuring.
Alternatively, one can modify the macro |\thepage| appropriately
and reset the counter |page| at the start of each child file.

%%%%%%%%%%%%%%%%%%%%%%%%%%%%%%%%%%%%%%%%%%%%%%%%%%%%%%%%%%%%%%%%%%%%%%%%%%%%%%%%
\subsection{Conditional Processing}
\label{sec:conditional}

The package provides a mechanism to compile different versions
of a document. To customise the versions further some conditional processing
can come in handy to distinguish which version is being compiled.
The package provides two macros to describe the compilation context:

%%%%%%%%%%%%%%%%%%%%%%%%%%%%%%%%%%%%%%%%
\DescribeMacro{\ifchilddoc}
The conditional |\ifchilddoc| distinguishes between the compilation of
child documents and the main document:
%
\begin{center}
|\ifchilddoc |\textit{child-code}| |[|\||else |\textit{main-code}]| \||fi|
\end{center}

%%%%%%%%%%%%%%%%%%%%%%%%%%%%%%%%%%%%%%%%
\DescribeMacro{\childdocname}
\DescribeMacro{\childdocjob}
The macro |\childdocname| contains the filename (without extension)
of the main or child file being processed.
Note that |\childdocjob| will always contain the name of the main file.

%%%%%%%%%%%%%%%%%%%%%%%%%%%%%%%%%%%%%%%%
\paragraph{Title Page.}

Conditional processing can be used to include a title or banner page
in the main document when proper precautions are taken.
Importantly, the code in the main file should ensure that the page counter
(as well as other status parameters which are stored in the |.aux| files)
takes the same value after the conditional processing.
Otherwise the page numbers may take divergent values
depending on which part is compiled.

For example, a title page could be declared by:
%
\begin{center}
\begin{tabular}{l}
|\ifchilddoc\||else|\\
|\addtocounter{page}{-1}|\\
\textit{code for title page}\\
|\newpage|\\
|\||fi|
\end{tabular}
\end{center}
%
A banner page for the child documents can be generated by:
%
\begin{center}
\begin{tabular}{l}
|\ifchilddoc|\\
|\addtocounter{page}{-1}|\\
\textit{code for banner page}\\
|\newpage|\\
|\||fi|
\end{tabular}
\end{center}
%
Here one could write a message such as:
\begin{center}
|This is the part \childdocname{} of \childdocjob{}.|
\end{center}

%%%%%%%%%%%%%%%%%%%%%%%%%%%%%%%%%%%%%%%%%%%%%%%%%%%%%%%%%%%%%%%%%%%%%%%%%%%%%%%%
\subsection{Flags}
\label{sec:flags}

The package makes it easy to generate different versions
of the main or child documents.
To this end compilation flags can be defined
and assigned different default values.
They will be particularly useful in conjunction
with the forwarding mechanism described in \secref{sec:forward}.

For example, it may be useful to have a flag |\version|
which can be set to |draft| or |final|.
The document source will contain some conditional code
depending on the value of |\version|.
Suppose further, the flag should default to |final| for the main file
and to |draft| for child files
which is a natural assignment for editing the document.
This is achieved by placing the following code
in the preamble of the main document
(below the |\childdocmain| directive):
%
\begin{center}
\begin{tabular}{l}
|\ifchilddoc|\\
|\providecommand{\version}{draft}|\\
|\||else|\\
|\providecommand{\version}{final}|\\
|\||fi|
\end{tabular}
\end{center}
%
The definition by |\providecommand| makes sure
that previous definitions are not overwritten.
Further statements |\providecommand{\version}{...}|
can thus be added before the above code to override it.

For the main file, one might add a line
(between |\childdocmain| and the above block)
%
\begin{center}
|%\ifchilddoc\||else\providecommand{\version}{draft}\||fi|
\end{center}
%
which can be uncommented to produce a draft version.
Likewise one can add a line to the very top of a child file
(above the |\childdocof{|\textit{main}|}| directive)
%
\begin{center}
|%\providecommand{\version}{final}|
\end{center}
%
which can be uncommented to produce the final version of this child document.

%%%%%%%%%%%%%%%%%%%%%%%%%%%%%%%%%%%%%%%%%%%%%%%%%%%%%%%%%%%%%%%%%%%%%%%%%%%%%%%%
\subsection{Forwarding}
\label{sec:forward}

Different versions of the main or child documents
using compilation flags as described in \secref{sec:flags}
can be (permanently) stored in different files
for convenient compilation, viewing and distribution.
To this end, the package defines a command
to pass on compilation to a different file:

%%%%%%%%%%%%%%%%%%%%%%%%%%%%%%%%%%%%%%%%
\DescribeMacro{\childdocforward}
The command |\childdocforward| redirects processing to
another source file:
%
\begin{center}
\begin{tabular}{l}
|\input{childdoc.def}|\\
|\childdocforward[|\textit{main}|]{|\textit{dest}|}|\\
\end{tabular}
\end{center}
%
The argument \textit{dest} is the destination file
(without extension).
It should be the main file or one of the child files.
Note that further \textsf{childdoc} directives
such as |\childdocof| and |\childdocforward|
in the indicated file will be processed in this form.
The optional argument \textit{main}
passes on directly to the main file \textit{main}
while pretending to compile the child \textit{dest}.
This form behaves as if \textit{dest}
issues |\childdocof{|\textit{main}|}| right away,
and no further \textsf{childdoc} directives will be processed.

%%%%%%%%%%%%%%%%%%%%%%%%%%%%%%%%%%%%%%%%
\DescribeMacro{\...prefix}
In the alternative form |\childdocforwardprefix|,
%
\begin{center}
\begin{tabular}{l}
|\input{childdoc.def}|\\
|\childdocforwardprefix[|\textit{main}|]{|\textit{prefix}|}{|\textit{dest}|}|
\end{tabular}
\end{center}
%
the destination file is determined by a pattern
depending on the current file:
To make this work, the current file must be called
`{\textit{prefix}\hspace{0.2em}\textit{suffix}}'
with \textit{prefix} matching precisely the argument.
Processing is then passed on to the file
`{\textit{dest}\hspace{0.2em}\textit{suffix}}'.
Surely, the same effect is achieved by
directly specifying the
argument `{\textit{dest}\hspace{0.2em}\textit{suffix}}'
in the first form.
However, that requires to set up a different file
for each child. With the alternative form of the command
all these files can have exactly the same content
which simplifies setting them up and maintaining them.

For example, the following file |draft.tex|
with a compilation flag |\version| as described in \secref{sec:flags}
compiles the main document as a draft:
%
\begin{center}
\begin{tabular}{l}
|\def\version{draft}|\\
|\input{childdoc.def}|\\
|\childdocforward{|\textit{main}|}|
\end{tabular}
\end{center}
%
Likewise, the following files |final|\textit{nn}|.tex|
compile the final version of the child document
|child|\textit{nn}|.tex|:
%
\begin{center}
\begin{tabular}{l}
|\def\version{final}|\\
|\input{childdoc.def}|\\
|\childdocforwardprefix{final}{child}|
\end{tabular}
\end{center}
%

Note that when several versions of a main file and/or of each child file
are to be generated, it may be convenient to set up a |Makefile| or
shell script to automatise the process.

%%%%%%%%%%%%%%%%%%%%%%%%%%%%%%%%%%%%%%%%%%%%%%%%%%%%%%%%%%%%%%%%%%%%%%%%%%%%%%%%
\subsection{Command Line Processing}
\label{sec:commandline}

The effect of redirection files can also be achieved by invoking
the \LaTeX{} compiler with a more elaborate command line.
Most conveniently this should be done as part
of a shell script or a |Makefile|.

When using \textsf{childdoc} in the main file, the following
command lines effectively perform a redirection
(note that depending on the shell being used,
backslashes may have to be doubled: `|\|' $\to$ `|\\|'):
%
\begin{center}
|... -jobname "|\textit{target}|" |\\|"|[\textit{flags}]%
|\input{childdoc.def}\childdocforward[|\textit{main}|]{|\textit{dest}|}"|
\end{center}
%
Here \textit{target} is the name of the output file,
\textit{main} is the name of the main file
and \textit{dest} is the name of the main or child file to be processed
(all filenames without extensions).
The optional argument \textit{main} can be omitted
if \textit{main} matches \textit{dest}.
Optionally, compilation \textit{flags} can be defined via |\def| commands.
This command line makes the \TeX{} engine believe
it is compiling the file \textit{target}
whose content is specified as the latter parameter.
The provided code then forwards the processing to
\textit{main} or \textit{dest} as described in \secref{sec:forward}.

%%%%%%%%%%%%%%%%%%%%%%%%%%%%%%%%%%%%%%%%%%%%%%%%%%%%%%%%%%%%%%%%%%%%%%%%%%%%%%%%
\subsection{Include by Input}
\label{sec:input}

Including child documents by |\include| has some restrictions by design.
Most notably, the content of a child document always occupies
its own set of pages; pages cannot be shared between child documents.
Usually, this behaviour makes perfect sense
because each child document contain an essential part of the document.
However, in some situations it may be desirable to compose
a document from a collection of parts
without having mandatory page breaks between then.
For this case, the package
provides a mechanism to include parts
by |\input| which can also be processed individually.
However, by construction this mechanism
requires manual handling of the content to be output.

%%%%%%%%%%%%%%%%%%%%%%%%%%%%%%%%%%%%%%%%
\DescribeMacro{\ifchilddocmanual}
The main file should be prepared as usual, see \secref{sec:include}.
However, the document body must make a distinction
between processing of an individual part and of the main document, e.g.:
%
\begin{center}
\begin{tabular}{l}
|\ifchilddocmanual|\\
|\input{\childdocname}|\\
|\||else|\\
\textit{document body with }|\input{|\textit{part}|}|\\
|\||fi|
\end{tabular}
\end{center}
%
The conditional |\ifchilddocmanual| is true whenever
a part to be included by |\input| is being compiled,
and the name of the part is stored in |\childdocname|.

%%%%%%%%%%%%%%%%%%%%%%%%%%%%%%%%%%%%%%%%
\DescribeMacro{\childdocby}
Each part to be included by |\input| should start with:
%
\begin{center}
\begin{tabular}{l}
|\input{childdoc.def}|\\
|\childdocby{|\textit{main}|}|\\
\end{tabular}
\end{center}
%
The directive |\childdocby| is similar to |\childdocof|
described in \secref{sec:include},
but the subsequent selection of content must be done manually.
To that end, both |\ifchilddoc| and |\ifchilddocmanual|
will be true upon processing of a part,
and the name of the part is stored in |\childdocname|.
Note that |\jobname| will be set to the filename of the current part
so that each part receives an individual |.aux| file
that does not interfere with the |.aux| file(s) of the main document.
This behaviour can be altered by the alternative form
|\childdocby[*]{|\textit{main}|}| (with a non-empty optional argument)
which uses the |.aux| file of the main document
by setting |\jobname| to \textit{main}.

%%%%%%%%%%%%%%%%%%%%%%%%%%%%%%%%%%%%%%%%%%%%%%%%%%%%%%%%%%%%%%%%%%%%%%%%%%%%%%%%
\subsection{Driver Development}
\label{sec:driver}

The \textsf{childdoc} mechanism can also be use for the development
of definition files such as \LaTeX{} styles or classes.
This case differs from the above setup with multiple parts
included by |\include| in that no |\includeonly| should be invoked.
This can be achieved by starting the include file
(before |\ProvidesPackage|) with:
%
\begin{center}
\begin{tabular}{l}
|\input{childdoc.def}|\\
|\childdocforward{|\textit{main}|}|\\
\end{tabular}
\end{center}
%
or alternatively with:
%
\begin{center}
\begin{tabular}{l}
|\input{childdoc.def}|\\
|\childdocby{|\textit{main}|}|\\
\end{tabular}
\end{center}
%
Both forms have slightly different effects as described above.
The main file is prepared as usual, see \secref{sec:include}.

%%%%%%%%%%%%%%%%%%%%%%%%%%%%%%%%%%%%%%%%%%%%%%%%%%%%%%%%%%%%%%%%%%%%%%%%%%%%%%%%
\subsection{Legacy Detection}
\label{sec:detection}

The directive |\childdocmain| in the main file can detect
whether the complete document or merely a child is to be compiled
even without using the directive |\childdocof|.
This method is deprecated because it is less robust
and there is no compelling reason to use it;
it is merely provided for backward compatibility
and it may be removed in future versions.

If the detection mechanism is to be used,
it is mandatory to correctly specify
the filename of the main file as the argument of |\childdocmain|:
%
\begin{center}
\begin{tabular}{l}
|\input{childdoc.def}|\\
|\childdocmain{|\textit{main}|}|\\
\end{tabular}
\end{center}
%
If |\jobname| does not match the argument \textit{main} of |\childdocmain|,
it is assumed that |\jobname| points to the child file to be compiled.
When using |\childdocmain| with the main file specified as argument,
it suffices to start a child file
with just |\input{|\textit{main}|}|
without loading of the package and using |\childdocof|.
If instead all processing is done
with the appropriate \textsf{childdoc} directives,
the argument of \textit{main} of |\childdocmain| can be empty.

An alternative version of the command line processing described
in \secref{sec:commandline} using the detection mechanism reads:
%
\begin{center}
|... -jobname "|\textit{target}|" "|[\textit{flags}]%
[|\def\jobname{|\textit{dest}|}|]|\input{|\textit{main}|}"|
\end{center}

%%%%%%%%%%%%%%%%%%%%%%%%%%%%%%%%%%%%%%%%%%%%%%%%%%%%%%%%%%%%%%%%%%%%%%%%%%%%%%%%
\subsection{Manual Code}
\label{sec:manual}

In case one cannot be certain whether the definitions file |childdoc.def|
is installed on the target \TeX{} distribution
and one prefers not to ship it,
it is conceivable to paste a few relevant commands into the sources.

To that end, drop all statements |\input{childdoc.def}|
and perform the replacements as outlined below.
Instead of |\childdocmain{|\textit{main}|}| add the following code
to the top of the main file:
%
\begin{center}
\begin{tabular}{l}
|\||ifdefined\childdocname\endinput\||fi\newif\ifchilddoc|\\
|\edef\childdocname{\scantokens\expandafter{\jobname\noexpand}}|\\
|\def\childdocmain{|\textit{main}|}\||ifx\childdocmain\childdocname\||else|\\
|\childdoctrue\includeonly{\childdocname}\let\jobname\childdocmain\||fi|\\
\end{tabular}
\end{center}
%
Instead of |\childdocof{|\textit{main}|}| just include the main file
at the top of each child file:
%
\begin{center}
|\input{|\textit{main}|}|
\end{center}
%
A simple redirection |\childdocforward{|\textit{dest}|}| is achieved by:
%
\begin{center}
|\def\jobname{|\textit{dest}|}\input{\jobname}|
\end{center}
%
The redirection with prefix
|\childdocforwardprefix[|\textit{prefix}|]{|\textit{dest}|}|
is accomplished by:
%
\begin{center}
\begin{tabular}{l}
|{\edef\jobname{\scantokens\expandafter{\jobname\noexpand}}|\\
|\def\redirectjob |\textit{prefix}|#1~~~{\gdef\jobname{|\textit{dest}|#1}}|\\
|\expandafter\redirectjob\jobname~~~}\input{\jobname}|
\end{tabular}
\end{center}

In an alternative approach,
child documents can be compiled by a specific command line
without additional code or specific definitions:
%
\begin{center}
|... -jobname "|\textit{target}|" "|[\textit{flags}]%
|\includeonly{|\textit{dest}|}\input{|\textit{main}|}"|
\end{center}
%

%%%%%%%%%%%%%%%%%%%%%%%%%%%%%%%%%%%%%%%%%%%%%%%%%%%%%%%%%%%%%%%%%%%%%%%%%%%%%%%%
%%%%%%%%%%%%%%%%%%%%%%%%%%%%%%%%%%%%%%%%%%%%%%%%%%%%%%%%%%%%%%%%%%%%%%%%%%%%%%%%
\section{Information}

%%%%%%%%%%%%%%%%%%%%%%%%%%%%%%%%%%%%%%%%%%%%%%%%%%%%%%%%%%%%%%%%%%%%%%%%%%%%%%%%
\subsection{Copyright}

Copyright \copyright{} 2017--2018 Niklas Beisert

This work may be distributed and/or modified under the
conditions of the \LaTeX{} Project Public License, either version 1.3
of this license or (at your option) any later version.
The latest version of this license is in
  \url{http://www.latex-project.org/lppl.txt}
and version 1.3 or later is part of all distributions of \LaTeX{}
version 2005/12/01 or later.

This work has the LPPL maintenance status `maintained'.

The Current Maintainer of this work is Niklas Beisert.

This work consists of the files |README.txt|, |childdoc.ins| and |childdoc.dtx|
as well as the derived files |childdoc.def|, |cdocsamp.tex|
with |cdocsch1.tex|, |cdocsch2.tex|, |cdocspt3.tex|, |cdocspt4.tex|,
|cdocsdrf.tex|, |cdocsfn1.tex|, |cdocsfn2.tex|
as well as |childdoc.pdf|.

%%%%%%%%%%%%%%%%%%%%%%%%%%%%%%%%%%%%%%%%%%%%%%%%%%%%%%%%%%%%%%%%%%%%%%%%%%%%%%%%
\subsection{Files and Installation}

The package consists of the files:
%
\begin{center}
\begin{tabular}{ll}
    |README.txt|   & readme file \\
    |childdoc.ins| & installation file \\
    |childdoc.dtx| & source file \\
    |childdoc.def| & definition file \\
    |cdocsamp.tex| & sample main file \\
    |cdocsch1.tex| & sample include file \\
    |cdocsch2.tex| & sample include file \\
    |cdocspt3.tex| & sample part file \\
    |cdocspt4.tex| & sample part file \\
    |cdocsdrf.tex| & sample redirection file \\
    |cdocsfn1.tex| & sample redirection file \\
    |cdocsfn2.tex| & sample redirection file \\
    |childdoc.pdf| & manual
\end{tabular}
\end{center}
%
The distribution consists of the files
|README.txt|, |childdoc.ins| and |childdoc.dtx|.
%
\begin{itemize}
\item
Run (pdf)\LaTeX{} on |childdoc.dtx|
to compile the manual |childdoc.pdf| (this file).
\item
Run \LaTeX{} on |childdoc.ins| to create the definitions file |childdoc.def|
and the sample |cdocsamp.tex| with include files
|cdocsch1.tex|, |cdocsch2.tex|, |cdocspt3.tex|, |cdocspt4.tex|,
|cdocsdrf.tex|, |cdocsfn1.tex|, |cdocsfn2.tex|.
Then copy the file |childdoc.def| to an appropriate directory of your \LaTeX{}
distribution, e.g.\ \textit{texmf-root}|/tex/latex/childdoc|.
\end{itemize}

%%%%%%%%%%%%%%%%%%%%%%%%%%%%%%%%%%%%%%%%%%%%%%%%%%%%%%%%%%%%%%%%%%%%%%%%%%%%%%%%
\subsection{Related CTAN Packages}

There are several other packages which offer a similar functionality:
%
\begin{itemize}
\item
The packages
\href{http://ctan.org/pkg/docmute}{\textsf{docmute}},
\href{http://ctan.org/pkg/includex}{\textsf{includex}} and
\href{http://ctan.org/pkg/standalone}{\textsf{standalone}}
provide commands to include only the document body of
a child file thus allowing both files to be compiled individually.
\item
The packages \href{http://ctan.org/pkg/subdocs}{\textsf{subdocs}}
and \href{http://ctan.org/pkg/subfiles}{\textsf{subfiles}}
provide structures in which the main and child documents can be
encapsulated and allowing them to be compiled individually.
The inclusion mechanism is different from the conventional |\include|.
\item
The package \href{http://ctan.org/pkg/combine}{\textsf{combine}}
is an elaborate solution to combine several documents into one.
\end{itemize}
%
See also the CTAN topic \href{http://ctan.org/topic/subdocs}{\textsf{subdocs}}
for further related packages.
The present package differs from the above solutions in that
a document structure constructed with the conventional |\include| mechanism
just needs two extra commands at the top of every file
such that all constituent files can be compiled individually.

%%%%%%%%%%%%%%%%%%%%%%%%%%%%%%%%%%%%%%%%%%%%%%%%%%%%%%%%%%%%%%%%%%%%%%%%%%%%%%%%
%\subsection{Feature Suggestions}
%
%The following is a list of features which may be useful for future
%versions of this package:
%%
%\begin{itemize}
%\item
%\ldots
%\end{itemize}

%%%%%%%%%%%%%%%%%%%%%%%%%%%%%%%%%%%%%%%%%%%%%%%%%%%%%%%%%%%%%%%%%%%%%%%%%%%%%%%%
\subsection{Revision History}

%%%%%%%%%%%%%%%%%%%%%%%%%%%%%%%%%%%%%%%%
\paragraph{v2.0:} 2018/12/30

\begin{itemize}
\item
immediate forward processing
\item
added |\childdocby| mechanism
\item
manual restructured
\end{itemize}

%%%%%%%%%%%%%%%%%%%%%%%%%%%%%%%%%%%%%%%%
\paragraph{v1.6:} 2018/01/17

\begin{itemize}
\item
application for development of include files
\item
corrections to manual
\end{itemize}

%%%%%%%%%%%%%%%%%%%%%%%%%%%%%%%%%%%%%%%%
\paragraph{v1.5:} 2017/05/21

\begin{itemize}
\item
more complete structuring introduced
\item
|\childdocof| introduced
\item
|\childdoc| renamed to |\childdocmain|
\item
|\childredirect| renamed to |\childdocforward| and |\childdocforwardprefix|
and functionality expanded
\end{itemize}

%%%%%%%%%%%%%%%%%%%%%%%%%%%%%%%%%%%%%%%%
\paragraph{v1.0:} 2017/04/27

\begin{itemize}
\item
manual and install package
\item
first version published on CTAN
\end{itemize}

%%%%%%%%%%%%%%%%%%%%%%%%%%%%%%%%%%%%%%%%
\paragraph{v0.6:} 2017/04/26

\begin{itemize}
\item
redirection mechanism added
\end{itemize}

%%%%%%%%%%%%%%%%%%%%%%%%%%%%%%%%%%%%%%%%
\paragraph{v0.5:} 2017/04/26

\begin{itemize}
\item
functionality in definition file
\end{itemize}


%%%%%%%%%%%%%%%%%%%%%%%%%%%%%%%%%%%%%%%%%%%%%%%%%%%%%%%%%%%%%%%%%%%%%%%%%%%%%%%%
%%%%%%%%%%%%%%%%%%%%%%%%%%%%%%%%%%%%%%%%%%%%%%%%%%%%%%%%%%%%%%%%%%%%%%%%%%%%%%%%
%%%%%%%%%%%%%%%%%%%%%%%%%%%%%%%%%%%%%%%%%%%%%%%%%%%%%%%%%%%%%%%%%%%%%%%%%%%%%%%%
\appendix

\settowidth\MacroIndent{\rmfamily\scriptsize 000\ }

 \DocInput{childdoc.dtx}

\end{document}
%</driver>
% \fi
%
% %%%%%%%%%%%%%%%%%%%%%%%%%%%%%%%%%%%%%%%%%%%%%%%%%%%%%%%%%%%%%%%%%%%%%%%%%%%%%%
% %%%%%%%%%%%%%%%%%%%%%%%%%%%%%%%%%%%%%%%%%%%%%%%%%%%%%%%%%%%%%%%%%%%%%%%%%%%%%%
% \section{Sample}
%\iffalse
%<*samplemain>
%\fi
%
% The following presents a sample document
% with two chapters, two parts, a title page,
% a compile flag as well as three forwarding files to set the flag.
% It consists of eight |.tex| files:
% \begin{center}
% \begin{tabular}{ll}
% |cdocsamp.tex|&main file\\
% |cdocsch1.tex|&include file for chapter 1\\
% |cdocsch2.tex|&include file for chapter 2\\
% |cdocspt3.tex|&include file for part 3\\
% |cdocspt4.tex|&include file for part 4\\
% |cdocsdrf.tex|&forwarding file for main file in draft mode\\
% |cdocsfi1.tex|&forwarding file for final version of chapter 1\\
% |cdocsfi2.tex|&forwarding file for final version of chapter 2\\
% \end{tabular}
% \end{center}
% Each of the eight files can be compiled directly by the \LaTeX{} compiler.
%
% %%%%%%%%%%%%%%%%%%%%%%%%%%%%%%%%%%%%%%
% \paragraph{Main File.}
%
% The main file is called |cdocsamp.tex|.
%
% Load the \textsf{childdoc} definitions and
% declare the filename for the main document:
%    \begin{macrocode}
\input{childdoc.def}
\childdocmain{}
%    \end{macrocode}

% Optional override for |\version| flag:
%    \begin{macrocode}
%%\ifchilddoc\else\providecommand{\version}{draft}\fi
%    \end{macrocode}

% Define the default values for the |\version| flag
% (|final| for the main file and |draft| for childs):
%    \begin{macrocode}
\ifchilddoc
\providecommand{\version}{draft}
\else
\providecommand{\version}{final}
\fi
%    \end{macrocode}

% Load the standard document class:
%    \begin{macrocode}
\documentclass[12pt]{article}
%    \end{macrocode}

% Start the document body:
%    \begin{macrocode}
\begin{document}
%    \end{macrocode}

% Declare a title page.
% Print title, part of document being processed and version flag:
%    \begin{macrocode}
\addtocounter{page}{-1}
\begin{center}
{\LARGE\bfseries{}childdoc example\par}
\vspace{1cm}
\ifchilddoc
\ifchilddocmanual part\else chapter\fi:
`\childdocname' of `\childdocjob'\par
\else
main document: `\childdocjob'\par
\fi
version: \version\par
\end{center}
\newpage
%    \end{macrocode}

% Manually include selected file,
% otherwise process as usual:
%    \begin{macrocode}
\ifchilddocmanual
\section*{part `\childdocname'}
\input{\childdocname}
\else
%    \end{macrocode}

% Include the two chapters:
%    \begin{macrocode}
\include{cdocsch1}
\include{cdocsch2}
%    \end{macrocode}

% Include the two parts unless only chapters should be displayed:
%    \begin{macrocode}
\ifchilddoc\else
\section{part three}
\input{cdocspt3}
\section{part four}
\input{cdocspt4}
\fi
%    \end{macrocode}

% Process as usual until here:
%    \begin{macrocode}
\fi
%    \end{macrocode}

% End of document body:
%    \begin{macrocode}
\end{document}
%    \end{macrocode}
%\iffalse
%</samplemain>
%\fi
%
% %%%%%%%%%%%%%%%%%%%%%%%%%%%%%%%%%%%%%%
% \paragraph{Chapter Include Files.}
%
% The include files are called |cdocsch1.tex| and |cdocsch2.tex|.
%
%\iffalse
%<*samplechap1|samplechap2>
%\fi

% Optional override for |\version| flag:
%    \begin{macrocode}
%%\providecommand{\version}{final}
%    \end{macrocode}

% Include the main document:
%    \begin{macrocode}
\input{childdoc.def}
\childdocof{cdocsamp}
%    \end{macrocode}

%\iffalse
%</samplechap1|samplechap2>
%\fi
%
%\iffalse
%<*samplechap1>
%\fi
% Some text for chapter 1:
%    \begin{macrocode}
\section{one}
some text in chapter one
%    \end{macrocode}

%\iffalse
%</samplechap1>
%\fi
% Some text for chapter 2:
%\iffalse
%<*samplechap2>
%\fi
%    \begin{macrocode}
\section{two}
more text in chapter two
%    \end{macrocode}

%\iffalse
%</samplechap2>
%\fi
%
% %%%%%%%%%%%%%%%%%%%%%%%%%%%%%%%%%%%%%%
% \paragraph{Part Include Files.}
%
% The include files are called |cdocspt3.tex| and |cdocspt4.tex|.
%
%\iffalse
%<*samplepart3|samplepart4>
%\fi

% Optional override for |\version| flag:
%    \begin{macrocode}
%%\providecommand{\version}{final}
%    \end{macrocode}

% Include the main document:
%    \begin{macrocode}
\input{childdoc.def}
\childdocby{cdocsamp}
%    \end{macrocode}

%\iffalse
%</samplepart3|samplepart4>
%\fi
%
%\iffalse
%<*samplepart3>
%\fi
% Some text for part 3:
%    \begin{macrocode}
some text in part three
%    \end{macrocode}

%\iffalse
%</samplepart3>
%\fi
% Some text for part 4:
%\iffalse
%<*samplepart4>
%\fi
%    \begin{macrocode}
more text in part four
%    \end{macrocode}

%\iffalse
%</samplepart4>
%\fi
%
% %%%%%%%%%%%%%%%%%%%%%%%%%%%%%%%%%%%%%%
% \paragraph{Forwarding for a Complete Draft.}
%
% The following forwarding file |cdocsdrf.tex|
% compiles the main document in draft mode:
%\iffalse
%<*sampledraft>
%\fi
%    \begin{macrocode}
\def\version{draft}
\input{childdoc.def}
\childdocforward{cdocsamp}
%    \end{macrocode}

%\iffalse
%</sampledraft>
%\fi
%
% %%%%%%%%%%%%%%%%%%%%%%%%%%%%%%%%%%%%%%
% \paragraph{Forwarding for Final Version of the Chapters.}
%
% The following forwarding files |cdocsfn1.tex| and |cdocsfn2.tex|
% (with identical content)
% compile the final versions of the child documents
% |cdocsch1.tex| and |cdocsch2.tex|, respectively:
%\iffalse
%<*samplefinal>
%\fi
%    \begin{macrocode}
\def\version{final}
\input{childdoc.def}
\childdocforwardprefix[cdocsamp]{cdocsfn}{cdocsch}
%    \end{macrocode}

%\iffalse
%</samplefinal>
%\fi
%
% %%%%%%%%%%%%%%%%%%%%%%%%%%%%%%%%%%%%%%
% \paragraph{Command Line Processing.}
%
% The following three command lines generate the output files
% |cdocscld|, |cdocscl1| and |cdocscl2|
% which should be identical to
% |cdocsdrf|, |cdocsch1| and |cdocsfn2|, respectively:
% \begin{center}
% \begin{tabular}{l}
% |latex -jobname cdocscld \|\\
% |  "\def\version{draft}\input{childdoc.def}\childdocforward{cdocsamp}"|\\
% |latex -jobname cdocscl1 \|\\
% |  "\input{childdoc.def}\childdocforward[cdocsamp]{cdocsch1}"|\\
% |latex -jobname cdocscl2 \|\\
% |  "\def\version{final}\input{childdoc.def}\childdocforward{cdocsch2}"|
% \end{tabular}
% \end{center}
% Note that the trailing backslash on each first line
% merely continues the input to the second line
% (for convenient cut ant paste).
% Furthermore, the command |latex| can be replaced by any
% of its alternative versions such as |pdflatex|.
%
% %%%%%%%%%%%%%%%%%%%%%%%%%%%%%%%%%%%%%%%%%%%%%%%%%%%%%%%%%%%%%%%%%%%%%%%%%%%%%%
% %%%%%%%%%%%%%%%%%%%%%%%%%%%%%%%%%%%%%%%%%%%%%%%%%%%%%%%%%%%%%%%%%%%%%%%%%%%%%%
% \section{Implementation}
%\iffalse
%<*package>
%\fi
%
% This section describes the definitions file |childdoc.def|.

% The definitions cannot be loaded using |\usepackage| or |\RequirePackage|
% which has a mechanism to prevent loading a style file more than once.
% When loading the definitions by means of |\input|
% multiple instances have to be prevented manually:
%\iffalse
%This code needs to be before the `\ProvidesFile' directive
%which is defined at the beginning of this file.
%Therefore it is also placed there and commented out here.
%</package>
%<*discard>
%\fi
%    \begin{macrocode}
\ifdefined\childdocmain\endinput\fi
%    \end{macrocode}
%\iffalse
%</discard>
%<*package>
%\fi
%
% \macro{\ifchilddoc}
% \macro{\ifchilddocmanual}
% The conditional |\ifchilddoc| tells whether a
% child (true) or main (false) document is being compiled.
% The conditional |\ifchilddocmanual| tells whether
% the |\includeonly| mechanism is used (false) or
% the selection of child files must be performed manually (true).
% The definitions initialise to false:
%    \begin{macrocode}
\newif\ifchilddoc
\newif\ifchilddocmanual
%    \end{macrocode}

% \macro{\childdocname}
% \macro{\childdocjob}
% The macro |\childdocname| stores the name of the main document
% to be compiled. The macro |\childdocjob| stores the name of
% the document on which the \LaTeX{} compiler was originally invoked.
% The content of |\jobname| cannot be compared
% to filenames specified in the source due to different catcodes.
% The following code rescans |\jobname|, stores the result
% in |\childdocname| and saves a copy in |\childdocjob|:
%    \begin{macrocode}
\edef\childdocname{\scantokens\expandafter{\jobname\noexpand}}
\let\childdocjob\childdocname
%    \end{macrocode}

% \macro{\childdocdisable}
% The macro |\childdocdisable| prevents the main file
% from being processed more than once.
% At this stage, the main document command |\childdocmain|
% is assumed to be called once again where it should do nothing.
% Any subsequent call to it should prevent
% a secondary processing of the main document
% It overwrites the forwarding commands
% |\childdocof| and |\childdocforward|
% with empty macros to prevent further inclusions of the main document:
%    \begin{macrocode}
\newcommand{\childdocdisable}
{
  \renewcommand{\childdocmain}[1]{\renewcommand{\childdocmain}[1]{\endinput}}
  \renewcommand{\childdocof}[1]{}
  \renewcommand{\childdocby}[2][]{}
  \renewcommand{\childdocforward}[2][]{}
  \renewcommand{\childdocdisable}{}
}
%    \end{macrocode}

% \macro{\childdocmain}
% The macro |\childdocmain| is to be called at the top of the main file
% with nothing or the main filename (without extension) as argument.
% First, it breaks loops.
% If the argument is not empty and does not match |\childdocname|
% (which is set by the first inclusion of |childdoc.def|),
% |\ifchilddoc| is set to true, |\includeonly| is applied to the child file
% and |\jobname| is set to the main file
% (for proper handling of |.aux| files):
%    \begin{macrocode}
\newcommand{\childdocmain}[1]
{
  \childdocdisable\childdocmain{}
  \if?#1?\else
    \begingroup
      \def\childdoctmp{#1}
      \ifx\childdoctmp\childdocname
        \def\childdoctmp{}
      \else
        \def\childdoctmp
        {
          \childdoctrue
          \includeonly{\childdocname}
          \def\childdocjob{#1}
          \def\jobname{#1}
        }
      \fi
      \expandafter
    \endgroup
    \childdoctmp
  \fi
}
%    \end{macrocode}

% \macro{\childdocof}
% The command |\childdocof| redirects
% compilation to the main file |#1|.
%    \begin{macrocode}
\newcommand{\childdocof}[1]
{
  \childdocdisable
  \childdoctrue
  \includeonly{\childdocname}
  \def\jobname{#1}
  \def\childdocjob{#1}
  \input{#1}
}
%    \end{macrocode}

% \macro{\childdocby}
% The command |\childdocby| ....
%    \begin{macrocode}
\newcommand{\childdocby}[2][]
{
  \childdocdisable
  \childdoctrue
  \childdocmanualtrue
  \if?#1?\else
    \def\jobname{#2}
  \fi
  \def\childdocjob{#2}
  \input{#2}
  \endinput
}
%    \end{macrocode}

% \macro{\childdocforward}
% The command |\childdocforward| redirects
% compilation to the main file or
% (if the optional argument is given) a child file.
% Parameters are set as if the main file
% or a child file starting with |\childdocof| was compiled.
% Then compilation is handed over to the main file:
%    \begin{macrocode}
\newcommand{\childdocforward}[2][]
{
  \begingroup
    \if?#1?
      \def\childdoctmp
      {
        \def\childdocname{#2}
        \def\childdocjob{#2}
        \def\jobname{#2}
        \input{#2}
        \endinput
      }
    \else
      \def\childdoctmp
      {
        \childdocdisable
        \def\childdocname{#2}
        \childdoctrue
        \includeonly{#2}
        \def\childdocjob{#1}
        \def\jobname{#1}
        \input{#1}
        \endinput
      }
    \fi
    \expandafter
  \endgroup
  \childdoctmp
}
%    \end{macrocode}

% \macro{\childdocforwardprefix}
% The command |\childdocforwardprefix| redirects
% compilation to the main or a child file by means of a pattern.
% The prefix |#1| in the current filename is replaced by |#2|
% and the suffix of the current filename is kept
% (it is assumed that the filename does not contain the substring `|~~~|'
% which is used as a delimiter).
% Compilation is handed over to the new file by |\childdocforward|:
%    \begin{macrocode}
\newcommand{\childdocforwardprefix}[3][]
{
  \begingroup
    \def\childdocextract #2##1~~~{\def\childdoctmp{\childdocforward[#1]{#3##1}}}
    \expandafter\childdocextract\childdocname~~~
    \expandafter
  \endgroup
  \childdoctmp
}
%    \end{macrocode}

% \macro{\childdoc}
% The deprecated macro |\childdoc| is a legacy version of |\childdocmain|:
%    \begin{macrocode}
\newcommand{\childdoc}{\childdocmain}
%    \end{macrocode}

% \macro{\childdocredirect}
% The deprecated macro |\childdocredirect| is a legacy version
% of |\childdocforward| and |\childdocforwardprefix|:
%    \begin{macrocode}
\newcommand{\childdocredirect}[2][]
{
  \begingroup
    \if?#1?
      \def\childdoctmp{\childdocforward{#2}}
    \else
      \def\childdoctmp{\childdocforwardprefix{#1}{#2}}
    \fi
    \expandafter
  \endgroup
  \childdoctmp
}
%    \end{macrocode}

%\iffalse
%</package>
%\fi
%
\endinput
\childdocforward{cdocsch2}"|
% \end{tabular}
% \end{center}
% Note that the trailing backslash on each first line
% merely continues the input to the second line
% (for convenient cut ant paste).
% Furthermore, the command |latex| can be replaced by any
% of its alternative versions such as |pdflatex|.
%
% %%%%%%%%%%%%%%%%%%%%%%%%%%%%%%%%%%%%%%%%%%%%%%%%%%%%%%%%%%%%%%%%%%%%%%%%%%%%%%
% %%%%%%%%%%%%%%%%%%%%%%%%%%%%%%%%%%%%%%%%%%%%%%%%%%%%%%%%%%%%%%%%%%%%%%%%%%%%%%
% \section{Implementation}
%\iffalse
%<*package>
%\fi
%
% This section describes the definitions file |childdoc.def|.

% The definitions cannot be loaded using |\usepackage| or |\RequirePackage|
% which has a mechanism to prevent loading a style file more than once.
% When loading the definitions by means of |\input|
% multiple instances have to be prevented manually:
%\iffalse
%This code needs to be before the `\ProvidesFile' directive
%which is defined at the beginning of this file.
%Therefore it is also placed there and commented out here.
%</package>
%<*discard>
%\fi
%    \begin{macrocode}
\ifdefined\childdocmain\endinput\fi
%    \end{macrocode}
%\iffalse
%</discard>
%<*package>
%\fi
%
% \macro{\ifchilddoc}
% \macro{\ifchilddocmanual}
% The conditional |\ifchilddoc| tells whether a
% child (true) or main (false) document is being compiled.
% The conditional |\ifchilddocmanual| tells whether
% the |\includeonly| mechanism is used (false) or
% the selection of child files must be performed manually (true).
% The definitions initialise to false:
%    \begin{macrocode}
\newif\ifchilddoc
\newif\ifchilddocmanual
%    \end{macrocode}

% \macro{\childdocname}
% \macro{\childdocjob}
% The macro |\childdocname| stores the name of the main document
% to be compiled. The macro |\childdocjob| stores the name of
% the document on which the \LaTeX{} compiler was originally invoked.
% The content of |\jobname| cannot be compared
% to filenames specified in the source due to different catcodes.
% The following code rescans |\jobname|, stores the result
% in |\childdocname| and saves a copy in |\childdocjob|:
%    \begin{macrocode}
\edef\childdocname{\scantokens\expandafter{\jobname\noexpand}}
\let\childdocjob\childdocname
%    \end{macrocode}

% \macro{\childdocdisable}
% The macro |\childdocdisable| prevents the main file
% from being processed more than once.
% At this stage, the main document command |\childdocmain|
% is assumed to be called once again where it should do nothing.
% Any subsequent call to it should prevent
% a secondary processing of the main document
% It overwrites the forwarding commands
% |\childdocof| and |\childdocforward|
% with empty macros to prevent further inclusions of the main document:
%    \begin{macrocode}
\newcommand{\childdocdisable}
{
  \renewcommand{\childdocmain}[1]{\renewcommand{\childdocmain}[1]{\endinput}}
  \renewcommand{\childdocof}[1]{}
  \renewcommand{\childdocby}[2][]{}
  \renewcommand{\childdocforward}[2][]{}
  \renewcommand{\childdocdisable}{}
}
%    \end{macrocode}

% \macro{\childdocmain}
% The macro |\childdocmain| is to be called at the top of the main file
% with nothing or the main filename (without extension) as argument.
% First, it breaks loops.
% If the argument is not empty and does not match |\childdocname|
% (which is set by the first inclusion of |childdoc.def|),
% |\ifchilddoc| is set to true, |\includeonly| is applied to the child file
% and |\jobname| is set to the main file
% (for proper handling of |.aux| files):
%    \begin{macrocode}
\newcommand{\childdocmain}[1]
{
  \childdocdisable\childdocmain{}
  \if?#1?\else
    \begingroup
      \def\childdoctmp{#1}
      \ifx\childdoctmp\childdocname
        \def\childdoctmp{}
      \else
        \def\childdoctmp
        {
          \childdoctrue
          \includeonly{\childdocname}
          \def\childdocjob{#1}
          \def\jobname{#1}
        }
      \fi
      \expandafter
    \endgroup
    \childdoctmp
  \fi
}
%    \end{macrocode}

% \macro{\childdocof}
% The command |\childdocof| redirects
% compilation to the main file |#1|.
%    \begin{macrocode}
\newcommand{\childdocof}[1]
{
  \childdocdisable
  \childdoctrue
  \includeonly{\childdocname}
  \def\jobname{#1}
  \def\childdocjob{#1}
  \input{#1}
}
%    \end{macrocode}

% \macro{\childdocby}
% The command |\childdocby| ....
%    \begin{macrocode}
\newcommand{\childdocby}[2][]
{
  \childdocdisable
  \childdoctrue
  \childdocmanualtrue
  \if?#1?\else
    \def\jobname{#2}
  \fi
  \def\childdocjob{#2}
  \input{#2}
  \endinput
}
%    \end{macrocode}

% \macro{\childdocforward}
% The command |\childdocforward| redirects
% compilation to the main file or
% (if the optional argument is given) a child file.
% Parameters are set as if the main file
% or a child file starting with |\childdocof| was compiled.
% Then compilation is handed over to the main file:
%    \begin{macrocode}
\newcommand{\childdocforward}[2][]
{
  \begingroup
    \if?#1?
      \def\childdoctmp
      {
        \def\childdocname{#2}
        \def\childdocjob{#2}
        \def\jobname{#2}
        \input{#2}
        \endinput
      }
    \else
      \def\childdoctmp
      {
        \childdocdisable
        \def\childdocname{#2}
        \childdoctrue
        \includeonly{#2}
        \def\childdocjob{#1}
        \def\jobname{#1}
        \input{#1}
        \endinput
      }
    \fi
    \expandafter
  \endgroup
  \childdoctmp
}
%    \end{macrocode}

% \macro{\childdocforwardprefix}
% The command |\childdocforwardprefix| redirects
% compilation to the main or a child file by means of a pattern.
% The prefix |#1| in the current filename is replaced by |#2|
% and the suffix of the current filename is kept
% (it is assumed that the filename does not contain the substring `|~~~|'
% which is used as a delimiter).
% Compilation is handed over to the new file by |\childdocforward|:
%    \begin{macrocode}
\newcommand{\childdocforwardprefix}[3][]
{
  \begingroup
    \def\childdocextract #2##1~~~{\def\childdoctmp{\childdocforward[#1]{#3##1}}}
    \expandafter\childdocextract\childdocname~~~
    \expandafter
  \endgroup
  \childdoctmp
}
%    \end{macrocode}

% \macro{\childdoc}
% The deprecated macro |\childdoc| is a legacy version of |\childdocmain|:
%    \begin{macrocode}
\newcommand{\childdoc}{\childdocmain}
%    \end{macrocode}

% \macro{\childdocredirect}
% The deprecated macro |\childdocredirect| is a legacy version
% of |\childdocforward| and |\childdocforwardprefix|:
%    \begin{macrocode}
\newcommand{\childdocredirect}[2][]
{
  \begingroup
    \if?#1?
      \def\childdoctmp{\childdocforward{#2}}
    \else
      \def\childdoctmp{\childdocforwardprefix{#1}{#2}}
    \fi
    \expandafter
  \endgroup
  \childdoctmp
}
%    \end{macrocode}

%\iffalse
%</package>
%\fi
%
\endinput

\childdocforward{cdocsamp}
%    \end{macrocode}

%\iffalse
%</sampledraft>
%\fi
%
% %%%%%%%%%%%%%%%%%%%%%%%%%%%%%%%%%%%%%%
% \paragraph{Forwarding for Final Version of the Chapters.}
%
% The following forwarding files |cdocsfn1.tex| and |cdocsfn2.tex|
% (with identical content)
% compile the final versions of the child documents
% |cdocsch1.tex| and |cdocsch2.tex|, respectively:
%\iffalse
%<*samplefinal>
%\fi
%    \begin{macrocode}
\def\version{final}
% \iffalse
%
% childdoc.dtx Copyright (C) 2017-2018 Niklas Beisert
%
% This work may be distributed and/or modified under the
% conditions of the LaTeX Project Public License, either version 1.3
% of this license or (at your option) any later version.
% The latest version of this license is in
%   http://www.latex-project.org/lppl.txt
% and version 1.3 or later is part of all distributions of LaTeX
% version 2005/12/01 or later.
%
% This work has the LPPL maintenance status `maintained'.
%
% The Current Maintainer of this work is Niklas Beisert.
%
% This work consists of the files childdoc.dtx and childdoc.ins
% and the derived files childdoc.def and cdocsamp.tex with
% cdocsch1.tex, cdocsch2.tex, cdocsdrf.tex, cdocsfn1.tex, cdocsfn2.tex.
%
%<package>\ifdefined\childdocmain\endinput\fi
%<package>\ProvidesFile{childdoc.def}[2018/12/30 v2.0 child document driver]
%<samplemain>\ProvidesFile{cdocsamp.tex}[2018/12/30 v2.0 sample for childdoc]
%<*driver>
%\ProvidesFile{childdoc.drv}[2018/12/30 v2.0 childdoc reference manual file]
\PassOptionsToClass{10pt,a4paper}{article}
\documentclass{ltxdoc}

\usepackage[margin=35mm]{geometry}
\usepackage{hyperref}
\usepackage{hyperxmp}
\usepackage[usenames]{color}

\hypersetup{colorlinks=true}
\hypersetup{pdfstartview=FitH}
\hypersetup{pdfpagemode=UseNone}
\hypersetup{pdfsource={}}
\hypersetup{pdflang={en-UK}}
\hypersetup{pdfcopyright={Copyright 2017-2018 Niklas Beisert.
  This work may be distributed and/or modified under the
  conditions of the LaTeX Project Public License, either version 1.3
  of this license or (at your option) any later version.}}
\hypersetup{pdflicenseurl={http://www.latex-project.org/lppl.txt}}
\hypersetup{pdfcontactaddress={ETH Zurich, ITP, HIT K,
  Wolfgang-Pauli-Strasse 27}}
\hypersetup{pdfcontactpostcode={8093}}
\hypersetup{pdfcontactcity={Zurich}}
\hypersetup{pdfcontactcountry={Switzerland}}
\hypersetup{pdfcontactemail={nbeisert@itp.phys.ethz.ch}}
\hypersetup{pdfcontacturl={http://people.phys.ethz.ch/\xmptilde nbeisert/}}

\newcommand{\secref}[1]{\hyperref[#1]{section \ref*{#1}}}

\parskip1ex
\parindent0pt
\let\olditemize\itemize
\def\itemize{\olditemize\parskip0pt}

\begin{document}

\title{The \textsf{childdoc} Package}
\hypersetup{pdftitle={The childdoc Package}}
\author{Niklas Beisert\\[2ex]
  Institut f\"ur Theoretische Physik\\
  Eidgen\"ossische Technische Hochschule Z\"urich\\
  Wolfgang-Pauli-Strasse 27, 8093 Z\"urich, Switzerland\\[1ex]
  \href{mailto:nbeisert@itp.phys.ethz.ch}
  {\texttt{nbeisert@itp.phys.ethz.ch}}}
\hypersetup{pdfauthor={Niklas Beisert}}
\hypersetup{pdfsubject={Manual for the LaTeX2e Package childdoc}}
\date{30 December 2018, \textsf{v2.0}}
\maketitle

\begin{abstract}\noindent
\textsf{childdoc} is a \LaTeXe{} package
that enables the direct compilation
of document sections included by |\include|
to individual files.
\end{abstract}

\begingroup
\parskip0ex
\tableofcontents
\endgroup

%%%%%%%%%%%%%%%%%%%%%%%%%%%%%%%%%%%%%%%%%%%%%%%%%%%%%%%%%%%%%%%%%%%%%%%%%%%%%%%%
%%%%%%%%%%%%%%%%%%%%%%%%%%%%%%%%%%%%%%%%%%%%%%%%%%%%%%%%%%%%%%%%%%%%%%%%%%%%%%%%
\section{Introduction}

\LaTeX{} provides a mechanism to structure a large document (such as a book)
into a main file and several child files (containing the chapters)
using the |\include| command.
This mechanism is beneficial for documents
which span hundreds of pages in order to
make the source file(s) more manageable.
Moreover, compilation can be restricted to
selected child files by means of the |\includeonly| command.
The latter feature can be used to reduce the compilation time while editing
(this was significantly more useful in the earlier days of \LaTeX{})
or to generate a smaller document which is easier to navigate.
Another application of |\includeonly| is to generate
documents consisting of selected parts of the complete document.

However, there are a few drawbacks of the plain |\include| mechanism:
\begin{itemize}
\item
The child files cannot be compiled on their own,
they can only be compiled via the main file.
A naive editing environment
(such as a text editor with an option
to have the current file processed by \LaTeX)
may require one to switch to the main file before compiling;
attempting to compile the child file produces errors.
\item
The main file must be modified (each time)
to adjust the |\includeonly| command
to the present needs. This easily leaves the main file in a messy state.
\item
The generated document will always carry the filename
of the main document. This is inconvenient if
several child files are to be compiled and
to be kept for distribution.
\end{itemize}

The present package provides a simple interface
to make child files individually compilable by \LaTeX{}.
Compiling a child file then has the same effect as compiling
the main file with an |\includeonly| command
to select the appropriate child.
Moreover the generated document will carry the name of the child
rather than the main file.
This resolves all three above issues.

This feature is meant to make the editing of books,
thesis documents and lecture notes somewhat more convenient.
However, the package can also be used efficiently for
composing a series of documents (such as exercise sheets)
which are typically distributed individually.
It then assists the author in generating the individual documents
(potentially in different versions)
as well as a document containing the collected series.
Another application is in developing style files
or other kinds of included material
where compilation of the style file could redirect
to a sample or test file.

%%%%%%%%%%%%%%%%%%%%%%%%%%%%%%%%%%%%%%%%%%%%%%%%%%%%%%%%%%%%%%%%%%%%%%%%%%%%%%%%
%%%%%%%%%%%%%%%%%%%%%%%%%%%%%%%%%%%%%%%%%%%%%%%%%%%%%%%%%%%%%%%%%%%%%%%%%%%%%%%%
\section{Usage}

First of all, the package \textsf{childdoc} is \emph{not} a standard
\LaTeXe{} |.sty| style file! Therefore it needs to be invoked in
a non-standard way.

%%%%%%%%%%%%%%%%%%%%%%%%%%%%%%%%%%%%%%%%%%%%%%%%%%%%%%%%%%%%%%%%%%%%%%%%%%%%%%%%
\subsection{Included Files}
\label{sec:include}

%%%%%%%%%%%%%%%%%%%%%%%%%%%%%%%%%%%%%%%%
\DescribeMacro{\childdocmain}
To use the package, add the commands
\begin{center}
\begin{tabular}{l}
|% \iffalse
%
% childdoc.dtx Copyright (C) 2017-2018 Niklas Beisert
%
% This work may be distributed and/or modified under the
% conditions of the LaTeX Project Public License, either version 1.3
% of this license or (at your option) any later version.
% The latest version of this license is in
%   http://www.latex-project.org/lppl.txt
% and version 1.3 or later is part of all distributions of LaTeX
% version 2005/12/01 or later.
%
% This work has the LPPL maintenance status `maintained'.
%
% The Current Maintainer of this work is Niklas Beisert.
%
% This work consists of the files childdoc.dtx and childdoc.ins
% and the derived files childdoc.def and cdocsamp.tex with
% cdocsch1.tex, cdocsch2.tex, cdocsdrf.tex, cdocsfn1.tex, cdocsfn2.tex.
%
%<package>\ifdefined\childdocmain\endinput\fi
%<package>\ProvidesFile{childdoc.def}[2018/12/30 v2.0 child document driver]
%<samplemain>\ProvidesFile{cdocsamp.tex}[2018/12/30 v2.0 sample for childdoc]
%<*driver>
%\ProvidesFile{childdoc.drv}[2018/12/30 v2.0 childdoc reference manual file]
\PassOptionsToClass{10pt,a4paper}{article}
\documentclass{ltxdoc}

\usepackage[margin=35mm]{geometry}
\usepackage{hyperref}
\usepackage{hyperxmp}
\usepackage[usenames]{color}

\hypersetup{colorlinks=true}
\hypersetup{pdfstartview=FitH}
\hypersetup{pdfpagemode=UseNone}
\hypersetup{pdfsource={}}
\hypersetup{pdflang={en-UK}}
\hypersetup{pdfcopyright={Copyright 2017-2018 Niklas Beisert.
  This work may be distributed and/or modified under the
  conditions of the LaTeX Project Public License, either version 1.3
  of this license or (at your option) any later version.}}
\hypersetup{pdflicenseurl={http://www.latex-project.org/lppl.txt}}
\hypersetup{pdfcontactaddress={ETH Zurich, ITP, HIT K,
  Wolfgang-Pauli-Strasse 27}}
\hypersetup{pdfcontactpostcode={8093}}
\hypersetup{pdfcontactcity={Zurich}}
\hypersetup{pdfcontactcountry={Switzerland}}
\hypersetup{pdfcontactemail={nbeisert@itp.phys.ethz.ch}}
\hypersetup{pdfcontacturl={http://people.phys.ethz.ch/\xmptilde nbeisert/}}

\newcommand{\secref}[1]{\hyperref[#1]{section \ref*{#1}}}

\parskip1ex
\parindent0pt
\let\olditemize\itemize
\def\itemize{\olditemize\parskip0pt}

\begin{document}

\title{The \textsf{childdoc} Package}
\hypersetup{pdftitle={The childdoc Package}}
\author{Niklas Beisert\\[2ex]
  Institut f\"ur Theoretische Physik\\
  Eidgen\"ossische Technische Hochschule Z\"urich\\
  Wolfgang-Pauli-Strasse 27, 8093 Z\"urich, Switzerland\\[1ex]
  \href{mailto:nbeisert@itp.phys.ethz.ch}
  {\texttt{nbeisert@itp.phys.ethz.ch}}}
\hypersetup{pdfauthor={Niklas Beisert}}
\hypersetup{pdfsubject={Manual for the LaTeX2e Package childdoc}}
\date{30 December 2018, \textsf{v2.0}}
\maketitle

\begin{abstract}\noindent
\textsf{childdoc} is a \LaTeXe{} package
that enables the direct compilation
of document sections included by |\include|
to individual files.
\end{abstract}

\begingroup
\parskip0ex
\tableofcontents
\endgroup

%%%%%%%%%%%%%%%%%%%%%%%%%%%%%%%%%%%%%%%%%%%%%%%%%%%%%%%%%%%%%%%%%%%%%%%%%%%%%%%%
%%%%%%%%%%%%%%%%%%%%%%%%%%%%%%%%%%%%%%%%%%%%%%%%%%%%%%%%%%%%%%%%%%%%%%%%%%%%%%%%
\section{Introduction}

\LaTeX{} provides a mechanism to structure a large document (such as a book)
into a main file and several child files (containing the chapters)
using the |\include| command.
This mechanism is beneficial for documents
which span hundreds of pages in order to
make the source file(s) more manageable.
Moreover, compilation can be restricted to
selected child files by means of the |\includeonly| command.
The latter feature can be used to reduce the compilation time while editing
(this was significantly more useful in the earlier days of \LaTeX{})
or to generate a smaller document which is easier to navigate.
Another application of |\includeonly| is to generate
documents consisting of selected parts of the complete document.

However, there are a few drawbacks of the plain |\include| mechanism:
\begin{itemize}
\item
The child files cannot be compiled on their own,
they can only be compiled via the main file.
A naive editing environment
(such as a text editor with an option
to have the current file processed by \LaTeX)
may require one to switch to the main file before compiling;
attempting to compile the child file produces errors.
\item
The main file must be modified (each time)
to adjust the |\includeonly| command
to the present needs. This easily leaves the main file in a messy state.
\item
The generated document will always carry the filename
of the main document. This is inconvenient if
several child files are to be compiled and
to be kept for distribution.
\end{itemize}

The present package provides a simple interface
to make child files individually compilable by \LaTeX{}.
Compiling a child file then has the same effect as compiling
the main file with an |\includeonly| command
to select the appropriate child.
Moreover the generated document will carry the name of the child
rather than the main file.
This resolves all three above issues.

This feature is meant to make the editing of books,
thesis documents and lecture notes somewhat more convenient.
However, the package can also be used efficiently for
composing a series of documents (such as exercise sheets)
which are typically distributed individually.
It then assists the author in generating the individual documents
(potentially in different versions)
as well as a document containing the collected series.
Another application is in developing style files
or other kinds of included material
where compilation of the style file could redirect
to a sample or test file.

%%%%%%%%%%%%%%%%%%%%%%%%%%%%%%%%%%%%%%%%%%%%%%%%%%%%%%%%%%%%%%%%%%%%%%%%%%%%%%%%
%%%%%%%%%%%%%%%%%%%%%%%%%%%%%%%%%%%%%%%%%%%%%%%%%%%%%%%%%%%%%%%%%%%%%%%%%%%%%%%%
\section{Usage}

First of all, the package \textsf{childdoc} is \emph{not} a standard
\LaTeXe{} |.sty| style file! Therefore it needs to be invoked in
a non-standard way.

%%%%%%%%%%%%%%%%%%%%%%%%%%%%%%%%%%%%%%%%%%%%%%%%%%%%%%%%%%%%%%%%%%%%%%%%%%%%%%%%
\subsection{Included Files}
\label{sec:include}

%%%%%%%%%%%%%%%%%%%%%%%%%%%%%%%%%%%%%%%%
\DescribeMacro{\childdocmain}
To use the package, add the commands
\begin{center}
\begin{tabular}{l}
|\input{childdoc.def}|\\
|\childdocmain{}|\\
\end{tabular}
\end{center}
at the very top of the main \LaTeX{} file,
in particular \emph{before} the |\documentclass| statement!
The argument of |\childdocmain| should be left empty
(but it must be present).

%%%%%%%%%%%%%%%%%%%%%%%%%%%%%%%%%%%%%%%%
\DescribeMacro{\childdocof}
Furthermore, add the commands
\begin{center}
\begin{tabular}{l}
|\input{childdoc.def}|\\
|\childdocof{|\textit{main}|}|\\
\end{tabular}
\end{center}
at the top of every child file \textit{child}
which is included by |\include{|\textit{child}|}|
from within the main file
(or at least for those files to be compiled individually).
The argument \textit{main} must be the filename of the main file.

There are a couple of
considerations in setting up the main and child documents:

%%%%%%%%%%%%%%%%%%%%%%%%%%%%%%%%%%%%%%%%
\paragraph{Restrictions.}

Please note the following restrictions:
\begin{itemize}
\item
|\childdocmain| must be called with one argument \textit{main}
to ensure compatibility with earlier version of the package.
It must either be empty (|\childdocmain{}|)
or precisely match the filename of the main file in which it is specified.
See \secref{sec:detection} for further information.
\item
The filename \textit{main} must be specified without the |.tex| extension.
\item
The filename \textit{main} is case sensitive
(even in case-insensitive file systems)
due to internal string comparison.
\item
The argument \textit{main} should be fully expanded, it cannot be a macro.
\item
Subdirectories and special characters should be avoided in filenames.
\item
The command |\childdocmain{|\textit{main}|}| must be followed by a whitespace.
It should not be followed immediately by another command
or by a comment mark `|%|'.
This is because the \TeX{} parser reads the token immediately following
the argument of |\childdocmain| and puts it
at the beginning of every child section;
however, a white\-space is ignored.
\end{itemize}

%%%%%%%%%%%%%%%%%%%%%%%%%%%%%%%%%%%%%%%%
\paragraph{Content of Main File.}

It is advisable to place all content in the child files included by |\include|.
Any output contained in the main file will appear in all child documents
unless suppressed manually;
it cannot be suppressed automatically by the |\includeonly| directive
and thus should normally be avoided.
A method to include some content in the main file
by means of conditional processing is described in \secref{sec:conditional}.

%%%%%%%%%%%%%%%%%%%%%%%%%%%%%%%%%%%%%%%%
\paragraph{Page Numbering.}

When only a part of the document is compiled,
the appropriate numbering of pages
(as well as other status parameters)
is determined from the |.aux| files.
The latter contain information from previous passes.
However this information needs to propagate through
all intermediate child documents.
Therefore the page numbering in child documents may well
be inconsistent until the complete document is compiled at least once.

A useful (if unconventional) way to always ensure a consistent
page numbering is to restart the numbering in each child document
and denote the pages by `\textit{child}|.|\textit{page}'
where \textit{child} represents the chapter/section number of the child file.
This can be achieved by the command
|\numberwithin{page}{|\textit{child}|}|
of the \textsf{amsmath} package
where \textit{child} can be |chapter| or |section|
depending on the chosen structuring.
Alternatively, one can modify the macro |\thepage| appropriately
and reset the counter |page| at the start of each child file.

%%%%%%%%%%%%%%%%%%%%%%%%%%%%%%%%%%%%%%%%%%%%%%%%%%%%%%%%%%%%%%%%%%%%%%%%%%%%%%%%
\subsection{Conditional Processing}
\label{sec:conditional}

The package provides a mechanism to compile different versions
of a document. To customise the versions further some conditional processing
can come in handy to distinguish which version is being compiled.
The package provides two macros to describe the compilation context:

%%%%%%%%%%%%%%%%%%%%%%%%%%%%%%%%%%%%%%%%
\DescribeMacro{\ifchilddoc}
The conditional |\ifchilddoc| distinguishes between the compilation of
child documents and the main document:
%
\begin{center}
|\ifchilddoc |\textit{child-code}| |[|\||else |\textit{main-code}]| \||fi|
\end{center}

%%%%%%%%%%%%%%%%%%%%%%%%%%%%%%%%%%%%%%%%
\DescribeMacro{\childdocname}
\DescribeMacro{\childdocjob}
The macro |\childdocname| contains the filename (without extension)
of the main or child file being processed.
Note that |\childdocjob| will always contain the name of the main file.

%%%%%%%%%%%%%%%%%%%%%%%%%%%%%%%%%%%%%%%%
\paragraph{Title Page.}

Conditional processing can be used to include a title or banner page
in the main document when proper precautions are taken.
Importantly, the code in the main file should ensure that the page counter
(as well as other status parameters which are stored in the |.aux| files)
takes the same value after the conditional processing.
Otherwise the page numbers may take divergent values
depending on which part is compiled.

For example, a title page could be declared by:
%
\begin{center}
\begin{tabular}{l}
|\ifchilddoc\||else|\\
|\addtocounter{page}{-1}|\\
\textit{code for title page}\\
|\newpage|\\
|\||fi|
\end{tabular}
\end{center}
%
A banner page for the child documents can be generated by:
%
\begin{center}
\begin{tabular}{l}
|\ifchilddoc|\\
|\addtocounter{page}{-1}|\\
\textit{code for banner page}\\
|\newpage|\\
|\||fi|
\end{tabular}
\end{center}
%
Here one could write a message such as:
\begin{center}
|This is the part \childdocname{} of \childdocjob{}.|
\end{center}

%%%%%%%%%%%%%%%%%%%%%%%%%%%%%%%%%%%%%%%%%%%%%%%%%%%%%%%%%%%%%%%%%%%%%%%%%%%%%%%%
\subsection{Flags}
\label{sec:flags}

The package makes it easy to generate different versions
of the main or child documents.
To this end compilation flags can be defined
and assigned different default values.
They will be particularly useful in conjunction
with the forwarding mechanism described in \secref{sec:forward}.

For example, it may be useful to have a flag |\version|
which can be set to |draft| or |final|.
The document source will contain some conditional code
depending on the value of |\version|.
Suppose further, the flag should default to |final| for the main file
and to |draft| for child files
which is a natural assignment for editing the document.
This is achieved by placing the following code
in the preamble of the main document
(below the |\childdocmain| directive):
%
\begin{center}
\begin{tabular}{l}
|\ifchilddoc|\\
|\providecommand{\version}{draft}|\\
|\||else|\\
|\providecommand{\version}{final}|\\
|\||fi|
\end{tabular}
\end{center}
%
The definition by |\providecommand| makes sure
that previous definitions are not overwritten.
Further statements |\providecommand{\version}{...}|
can thus be added before the above code to override it.

For the main file, one might add a line
(between |\childdocmain| and the above block)
%
\begin{center}
|%\ifchilddoc\||else\providecommand{\version}{draft}\||fi|
\end{center}
%
which can be uncommented to produce a draft version.
Likewise one can add a line to the very top of a child file
(above the |\childdocof{|\textit{main}|}| directive)
%
\begin{center}
|%\providecommand{\version}{final}|
\end{center}
%
which can be uncommented to produce the final version of this child document.

%%%%%%%%%%%%%%%%%%%%%%%%%%%%%%%%%%%%%%%%%%%%%%%%%%%%%%%%%%%%%%%%%%%%%%%%%%%%%%%%
\subsection{Forwarding}
\label{sec:forward}

Different versions of the main or child documents
using compilation flags as described in \secref{sec:flags}
can be (permanently) stored in different files
for convenient compilation, viewing and distribution.
To this end, the package defines a command
to pass on compilation to a different file:

%%%%%%%%%%%%%%%%%%%%%%%%%%%%%%%%%%%%%%%%
\DescribeMacro{\childdocforward}
The command |\childdocforward| redirects processing to
another source file:
%
\begin{center}
\begin{tabular}{l}
|\input{childdoc.def}|\\
|\childdocforward[|\textit{main}|]{|\textit{dest}|}|\\
\end{tabular}
\end{center}
%
The argument \textit{dest} is the destination file
(without extension).
It should be the main file or one of the child files.
Note that further \textsf{childdoc} directives
such as |\childdocof| and |\childdocforward|
in the indicated file will be processed in this form.
The optional argument \textit{main}
passes on directly to the main file \textit{main}
while pretending to compile the child \textit{dest}.
This form behaves as if \textit{dest}
issues |\childdocof{|\textit{main}|}| right away,
and no further \textsf{childdoc} directives will be processed.

%%%%%%%%%%%%%%%%%%%%%%%%%%%%%%%%%%%%%%%%
\DescribeMacro{\...prefix}
In the alternative form |\childdocforwardprefix|,
%
\begin{center}
\begin{tabular}{l}
|\input{childdoc.def}|\\
|\childdocforwardprefix[|\textit{main}|]{|\textit{prefix}|}{|\textit{dest}|}|
\end{tabular}
\end{center}
%
the destination file is determined by a pattern
depending on the current file:
To make this work, the current file must be called
`{\textit{prefix}\hspace{0.2em}\textit{suffix}}'
with \textit{prefix} matching precisely the argument.
Processing is then passed on to the file
`{\textit{dest}\hspace{0.2em}\textit{suffix}}'.
Surely, the same effect is achieved by
directly specifying the
argument `{\textit{dest}\hspace{0.2em}\textit{suffix}}'
in the first form.
However, that requires to set up a different file
for each child. With the alternative form of the command
all these files can have exactly the same content
which simplifies setting them up and maintaining them.

For example, the following file |draft.tex|
with a compilation flag |\version| as described in \secref{sec:flags}
compiles the main document as a draft:
%
\begin{center}
\begin{tabular}{l}
|\def\version{draft}|\\
|\input{childdoc.def}|\\
|\childdocforward{|\textit{main}|}|
\end{tabular}
\end{center}
%
Likewise, the following files |final|\textit{nn}|.tex|
compile the final version of the child document
|child|\textit{nn}|.tex|:
%
\begin{center}
\begin{tabular}{l}
|\def\version{final}|\\
|\input{childdoc.def}|\\
|\childdocforwardprefix{final}{child}|
\end{tabular}
\end{center}
%

Note that when several versions of a main file and/or of each child file
are to be generated, it may be convenient to set up a |Makefile| or
shell script to automatise the process.

%%%%%%%%%%%%%%%%%%%%%%%%%%%%%%%%%%%%%%%%%%%%%%%%%%%%%%%%%%%%%%%%%%%%%%%%%%%%%%%%
\subsection{Command Line Processing}
\label{sec:commandline}

The effect of redirection files can also be achieved by invoking
the \LaTeX{} compiler with a more elaborate command line.
Most conveniently this should be done as part
of a shell script or a |Makefile|.

When using \textsf{childdoc} in the main file, the following
command lines effectively perform a redirection
(note that depending on the shell being used,
backslashes may have to be doubled: `|\|' $\to$ `|\\|'):
%
\begin{center}
|... -jobname "|\textit{target}|" |\\|"|[\textit{flags}]%
|\input{childdoc.def}\childdocforward[|\textit{main}|]{|\textit{dest}|}"|
\end{center}
%
Here \textit{target} is the name of the output file,
\textit{main} is the name of the main file
and \textit{dest} is the name of the main or child file to be processed
(all filenames without extensions).
The optional argument \textit{main} can be omitted
if \textit{main} matches \textit{dest}.
Optionally, compilation \textit{flags} can be defined via |\def| commands.
This command line makes the \TeX{} engine believe
it is compiling the file \textit{target}
whose content is specified as the latter parameter.
The provided code then forwards the processing to
\textit{main} or \textit{dest} as described in \secref{sec:forward}.

%%%%%%%%%%%%%%%%%%%%%%%%%%%%%%%%%%%%%%%%%%%%%%%%%%%%%%%%%%%%%%%%%%%%%%%%%%%%%%%%
\subsection{Include by Input}
\label{sec:input}

Including child documents by |\include| has some restrictions by design.
Most notably, the content of a child document always occupies
its own set of pages; pages cannot be shared between child documents.
Usually, this behaviour makes perfect sense
because each child document contain an essential part of the document.
However, in some situations it may be desirable to compose
a document from a collection of parts
without having mandatory page breaks between then.
For this case, the package
provides a mechanism to include parts
by |\input| which can also be processed individually.
However, by construction this mechanism
requires manual handling of the content to be output.

%%%%%%%%%%%%%%%%%%%%%%%%%%%%%%%%%%%%%%%%
\DescribeMacro{\ifchilddocmanual}
The main file should be prepared as usual, see \secref{sec:include}.
However, the document body must make a distinction
between processing of an individual part and of the main document, e.g.:
%
\begin{center}
\begin{tabular}{l}
|\ifchilddocmanual|\\
|\input{\childdocname}|\\
|\||else|\\
\textit{document body with }|\input{|\textit{part}|}|\\
|\||fi|
\end{tabular}
\end{center}
%
The conditional |\ifchilddocmanual| is true whenever
a part to be included by |\input| is being compiled,
and the name of the part is stored in |\childdocname|.

%%%%%%%%%%%%%%%%%%%%%%%%%%%%%%%%%%%%%%%%
\DescribeMacro{\childdocby}
Each part to be included by |\input| should start with:
%
\begin{center}
\begin{tabular}{l}
|\input{childdoc.def}|\\
|\childdocby{|\textit{main}|}|\\
\end{tabular}
\end{center}
%
The directive |\childdocby| is similar to |\childdocof|
described in \secref{sec:include},
but the subsequent selection of content must be done manually.
To that end, both |\ifchilddoc| and |\ifchilddocmanual|
will be true upon processing of a part,
and the name of the part is stored in |\childdocname|.
Note that |\jobname| will be set to the filename of the current part
so that each part receives an individual |.aux| file
that does not interfere with the |.aux| file(s) of the main document.
This behaviour can be altered by the alternative form
|\childdocby[*]{|\textit{main}|}| (with a non-empty optional argument)
which uses the |.aux| file of the main document
by setting |\jobname| to \textit{main}.

%%%%%%%%%%%%%%%%%%%%%%%%%%%%%%%%%%%%%%%%%%%%%%%%%%%%%%%%%%%%%%%%%%%%%%%%%%%%%%%%
\subsection{Driver Development}
\label{sec:driver}

The \textsf{childdoc} mechanism can also be use for the development
of definition files such as \LaTeX{} styles or classes.
This case differs from the above setup with multiple parts
included by |\include| in that no |\includeonly| should be invoked.
This can be achieved by starting the include file
(before |\ProvidesPackage|) with:
%
\begin{center}
\begin{tabular}{l}
|\input{childdoc.def}|\\
|\childdocforward{|\textit{main}|}|\\
\end{tabular}
\end{center}
%
or alternatively with:
%
\begin{center}
\begin{tabular}{l}
|\input{childdoc.def}|\\
|\childdocby{|\textit{main}|}|\\
\end{tabular}
\end{center}
%
Both forms have slightly different effects as described above.
The main file is prepared as usual, see \secref{sec:include}.

%%%%%%%%%%%%%%%%%%%%%%%%%%%%%%%%%%%%%%%%%%%%%%%%%%%%%%%%%%%%%%%%%%%%%%%%%%%%%%%%
\subsection{Legacy Detection}
\label{sec:detection}

The directive |\childdocmain| in the main file can detect
whether the complete document or merely a child is to be compiled
even without using the directive |\childdocof|.
This method is deprecated because it is less robust
and there is no compelling reason to use it;
it is merely provided for backward compatibility
and it may be removed in future versions.

If the detection mechanism is to be used,
it is mandatory to correctly specify
the filename of the main file as the argument of |\childdocmain|:
%
\begin{center}
\begin{tabular}{l}
|\input{childdoc.def}|\\
|\childdocmain{|\textit{main}|}|\\
\end{tabular}
\end{center}
%
If |\jobname| does not match the argument \textit{main} of |\childdocmain|,
it is assumed that |\jobname| points to the child file to be compiled.
When using |\childdocmain| with the main file specified as argument,
it suffices to start a child file
with just |\input{|\textit{main}|}|
without loading of the package and using |\childdocof|.
If instead all processing is done
with the appropriate \textsf{childdoc} directives,
the argument of \textit{main} of |\childdocmain| can be empty.

An alternative version of the command line processing described
in \secref{sec:commandline} using the detection mechanism reads:
%
\begin{center}
|... -jobname "|\textit{target}|" "|[\textit{flags}]%
[|\def\jobname{|\textit{dest}|}|]|\input{|\textit{main}|}"|
\end{center}

%%%%%%%%%%%%%%%%%%%%%%%%%%%%%%%%%%%%%%%%%%%%%%%%%%%%%%%%%%%%%%%%%%%%%%%%%%%%%%%%
\subsection{Manual Code}
\label{sec:manual}

In case one cannot be certain whether the definitions file |childdoc.def|
is installed on the target \TeX{} distribution
and one prefers not to ship it,
it is conceivable to paste a few relevant commands into the sources.

To that end, drop all statements |\input{childdoc.def}|
and perform the replacements as outlined below.
Instead of |\childdocmain{|\textit{main}|}| add the following code
to the top of the main file:
%
\begin{center}
\begin{tabular}{l}
|\||ifdefined\childdocname\endinput\||fi\newif\ifchilddoc|\\
|\edef\childdocname{\scantokens\expandafter{\jobname\noexpand}}|\\
|\def\childdocmain{|\textit{main}|}\||ifx\childdocmain\childdocname\||else|\\
|\childdoctrue\includeonly{\childdocname}\let\jobname\childdocmain\||fi|\\
\end{tabular}
\end{center}
%
Instead of |\childdocof{|\textit{main}|}| just include the main file
at the top of each child file:
%
\begin{center}
|\input{|\textit{main}|}|
\end{center}
%
A simple redirection |\childdocforward{|\textit{dest}|}| is achieved by:
%
\begin{center}
|\def\jobname{|\textit{dest}|}\input{\jobname}|
\end{center}
%
The redirection with prefix
|\childdocforwardprefix[|\textit{prefix}|]{|\textit{dest}|}|
is accomplished by:
%
\begin{center}
\begin{tabular}{l}
|{\edef\jobname{\scantokens\expandafter{\jobname\noexpand}}|\\
|\def\redirectjob |\textit{prefix}|#1~~~{\gdef\jobname{|\textit{dest}|#1}}|\\
|\expandafter\redirectjob\jobname~~~}\input{\jobname}|
\end{tabular}
\end{center}

In an alternative approach,
child documents can be compiled by a specific command line
without additional code or specific definitions:
%
\begin{center}
|... -jobname "|\textit{target}|" "|[\textit{flags}]%
|\includeonly{|\textit{dest}|}\input{|\textit{main}|}"|
\end{center}
%

%%%%%%%%%%%%%%%%%%%%%%%%%%%%%%%%%%%%%%%%%%%%%%%%%%%%%%%%%%%%%%%%%%%%%%%%%%%%%%%%
%%%%%%%%%%%%%%%%%%%%%%%%%%%%%%%%%%%%%%%%%%%%%%%%%%%%%%%%%%%%%%%%%%%%%%%%%%%%%%%%
\section{Information}

%%%%%%%%%%%%%%%%%%%%%%%%%%%%%%%%%%%%%%%%%%%%%%%%%%%%%%%%%%%%%%%%%%%%%%%%%%%%%%%%
\subsection{Copyright}

Copyright \copyright{} 2017--2018 Niklas Beisert

This work may be distributed and/or modified under the
conditions of the \LaTeX{} Project Public License, either version 1.3
of this license or (at your option) any later version.
The latest version of this license is in
  \url{http://www.latex-project.org/lppl.txt}
and version 1.3 or later is part of all distributions of \LaTeX{}
version 2005/12/01 or later.

This work has the LPPL maintenance status `maintained'.

The Current Maintainer of this work is Niklas Beisert.

This work consists of the files |README.txt|, |childdoc.ins| and |childdoc.dtx|
as well as the derived files |childdoc.def|, |cdocsamp.tex|
with |cdocsch1.tex|, |cdocsch2.tex|, |cdocspt3.tex|, |cdocspt4.tex|,
|cdocsdrf.tex|, |cdocsfn1.tex|, |cdocsfn2.tex|
as well as |childdoc.pdf|.

%%%%%%%%%%%%%%%%%%%%%%%%%%%%%%%%%%%%%%%%%%%%%%%%%%%%%%%%%%%%%%%%%%%%%%%%%%%%%%%%
\subsection{Files and Installation}

The package consists of the files:
%
\begin{center}
\begin{tabular}{ll}
    |README.txt|   & readme file \\
    |childdoc.ins| & installation file \\
    |childdoc.dtx| & source file \\
    |childdoc.def| & definition file \\
    |cdocsamp.tex| & sample main file \\
    |cdocsch1.tex| & sample include file \\
    |cdocsch2.tex| & sample include file \\
    |cdocspt3.tex| & sample part file \\
    |cdocspt4.tex| & sample part file \\
    |cdocsdrf.tex| & sample redirection file \\
    |cdocsfn1.tex| & sample redirection file \\
    |cdocsfn2.tex| & sample redirection file \\
    |childdoc.pdf| & manual
\end{tabular}
\end{center}
%
The distribution consists of the files
|README.txt|, |childdoc.ins| and |childdoc.dtx|.
%
\begin{itemize}
\item
Run (pdf)\LaTeX{} on |childdoc.dtx|
to compile the manual |childdoc.pdf| (this file).
\item
Run \LaTeX{} on |childdoc.ins| to create the definitions file |childdoc.def|
and the sample |cdocsamp.tex| with include files
|cdocsch1.tex|, |cdocsch2.tex|, |cdocspt3.tex|, |cdocspt4.tex|,
|cdocsdrf.tex|, |cdocsfn1.tex|, |cdocsfn2.tex|.
Then copy the file |childdoc.def| to an appropriate directory of your \LaTeX{}
distribution, e.g.\ \textit{texmf-root}|/tex/latex/childdoc|.
\end{itemize}

%%%%%%%%%%%%%%%%%%%%%%%%%%%%%%%%%%%%%%%%%%%%%%%%%%%%%%%%%%%%%%%%%%%%%%%%%%%%%%%%
\subsection{Related CTAN Packages}

There are several other packages which offer a similar functionality:
%
\begin{itemize}
\item
The packages
\href{http://ctan.org/pkg/docmute}{\textsf{docmute}},
\href{http://ctan.org/pkg/includex}{\textsf{includex}} and
\href{http://ctan.org/pkg/standalone}{\textsf{standalone}}
provide commands to include only the document body of
a child file thus allowing both files to be compiled individually.
\item
The packages \href{http://ctan.org/pkg/subdocs}{\textsf{subdocs}}
and \href{http://ctan.org/pkg/subfiles}{\textsf{subfiles}}
provide structures in which the main and child documents can be
encapsulated and allowing them to be compiled individually.
The inclusion mechanism is different from the conventional |\include|.
\item
The package \href{http://ctan.org/pkg/combine}{\textsf{combine}}
is an elaborate solution to combine several documents into one.
\end{itemize}
%
See also the CTAN topic \href{http://ctan.org/topic/subdocs}{\textsf{subdocs}}
for further related packages.
The present package differs from the above solutions in that
a document structure constructed with the conventional |\include| mechanism
just needs two extra commands at the top of every file
such that all constituent files can be compiled individually.

%%%%%%%%%%%%%%%%%%%%%%%%%%%%%%%%%%%%%%%%%%%%%%%%%%%%%%%%%%%%%%%%%%%%%%%%%%%%%%%%
%\subsection{Feature Suggestions}
%
%The following is a list of features which may be useful for future
%versions of this package:
%%
%\begin{itemize}
%\item
%\ldots
%\end{itemize}

%%%%%%%%%%%%%%%%%%%%%%%%%%%%%%%%%%%%%%%%%%%%%%%%%%%%%%%%%%%%%%%%%%%%%%%%%%%%%%%%
\subsection{Revision History}

%%%%%%%%%%%%%%%%%%%%%%%%%%%%%%%%%%%%%%%%
\paragraph{v2.0:} 2018/12/30

\begin{itemize}
\item
immediate forward processing
\item
added |\childdocby| mechanism
\item
manual restructured
\end{itemize}

%%%%%%%%%%%%%%%%%%%%%%%%%%%%%%%%%%%%%%%%
\paragraph{v1.6:} 2018/01/17

\begin{itemize}
\item
application for development of include files
\item
corrections to manual
\end{itemize}

%%%%%%%%%%%%%%%%%%%%%%%%%%%%%%%%%%%%%%%%
\paragraph{v1.5:} 2017/05/21

\begin{itemize}
\item
more complete structuring introduced
\item
|\childdocof| introduced
\item
|\childdoc| renamed to |\childdocmain|
\item
|\childredirect| renamed to |\childdocforward| and |\childdocforwardprefix|
and functionality expanded
\end{itemize}

%%%%%%%%%%%%%%%%%%%%%%%%%%%%%%%%%%%%%%%%
\paragraph{v1.0:} 2017/04/27

\begin{itemize}
\item
manual and install package
\item
first version published on CTAN
\end{itemize}

%%%%%%%%%%%%%%%%%%%%%%%%%%%%%%%%%%%%%%%%
\paragraph{v0.6:} 2017/04/26

\begin{itemize}
\item
redirection mechanism added
\end{itemize}

%%%%%%%%%%%%%%%%%%%%%%%%%%%%%%%%%%%%%%%%
\paragraph{v0.5:} 2017/04/26

\begin{itemize}
\item
functionality in definition file
\end{itemize}


%%%%%%%%%%%%%%%%%%%%%%%%%%%%%%%%%%%%%%%%%%%%%%%%%%%%%%%%%%%%%%%%%%%%%%%%%%%%%%%%
%%%%%%%%%%%%%%%%%%%%%%%%%%%%%%%%%%%%%%%%%%%%%%%%%%%%%%%%%%%%%%%%%%%%%%%%%%%%%%%%
%%%%%%%%%%%%%%%%%%%%%%%%%%%%%%%%%%%%%%%%%%%%%%%%%%%%%%%%%%%%%%%%%%%%%%%%%%%%%%%%
\appendix

\settowidth\MacroIndent{\rmfamily\scriptsize 000\ }

 \DocInput{childdoc.dtx}

\end{document}
%</driver>
% \fi
%
% %%%%%%%%%%%%%%%%%%%%%%%%%%%%%%%%%%%%%%%%%%%%%%%%%%%%%%%%%%%%%%%%%%%%%%%%%%%%%%
% %%%%%%%%%%%%%%%%%%%%%%%%%%%%%%%%%%%%%%%%%%%%%%%%%%%%%%%%%%%%%%%%%%%%%%%%%%%%%%
% \section{Sample}
%\iffalse
%<*samplemain>
%\fi
%
% The following presents a sample document
% with two chapters, two parts, a title page,
% a compile flag as well as three forwarding files to set the flag.
% It consists of eight |.tex| files:
% \begin{center}
% \begin{tabular}{ll}
% |cdocsamp.tex|&main file\\
% |cdocsch1.tex|&include file for chapter 1\\
% |cdocsch2.tex|&include file for chapter 2\\
% |cdocspt3.tex|&include file for part 3\\
% |cdocspt4.tex|&include file for part 4\\
% |cdocsdrf.tex|&forwarding file for main file in draft mode\\
% |cdocsfi1.tex|&forwarding file for final version of chapter 1\\
% |cdocsfi2.tex|&forwarding file for final version of chapter 2\\
% \end{tabular}
% \end{center}
% Each of the eight files can be compiled directly by the \LaTeX{} compiler.
%
% %%%%%%%%%%%%%%%%%%%%%%%%%%%%%%%%%%%%%%
% \paragraph{Main File.}
%
% The main file is called |cdocsamp.tex|.
%
% Load the \textsf{childdoc} definitions and
% declare the filename for the main document:
%    \begin{macrocode}
\input{childdoc.def}
\childdocmain{}
%    \end{macrocode}

% Optional override for |\version| flag:
%    \begin{macrocode}
%%\ifchilddoc\else\providecommand{\version}{draft}\fi
%    \end{macrocode}

% Define the default values for the |\version| flag
% (|final| for the main file and |draft| for childs):
%    \begin{macrocode}
\ifchilddoc
\providecommand{\version}{draft}
\else
\providecommand{\version}{final}
\fi
%    \end{macrocode}

% Load the standard document class:
%    \begin{macrocode}
\documentclass[12pt]{article}
%    \end{macrocode}

% Start the document body:
%    \begin{macrocode}
\begin{document}
%    \end{macrocode}

% Declare a title page.
% Print title, part of document being processed and version flag:
%    \begin{macrocode}
\addtocounter{page}{-1}
\begin{center}
{\LARGE\bfseries{}childdoc example\par}
\vspace{1cm}
\ifchilddoc
\ifchilddocmanual part\else chapter\fi:
`\childdocname' of `\childdocjob'\par
\else
main document: `\childdocjob'\par
\fi
version: \version\par
\end{center}
\newpage
%    \end{macrocode}

% Manually include selected file,
% otherwise process as usual:
%    \begin{macrocode}
\ifchilddocmanual
\section*{part `\childdocname'}
\input{\childdocname}
\else
%    \end{macrocode}

% Include the two chapters:
%    \begin{macrocode}
\include{cdocsch1}
\include{cdocsch2}
%    \end{macrocode}

% Include the two parts unless only chapters should be displayed:
%    \begin{macrocode}
\ifchilddoc\else
\section{part three}
\input{cdocspt3}
\section{part four}
\input{cdocspt4}
\fi
%    \end{macrocode}

% Process as usual until here:
%    \begin{macrocode}
\fi
%    \end{macrocode}

% End of document body:
%    \begin{macrocode}
\end{document}
%    \end{macrocode}
%\iffalse
%</samplemain>
%\fi
%
% %%%%%%%%%%%%%%%%%%%%%%%%%%%%%%%%%%%%%%
% \paragraph{Chapter Include Files.}
%
% The include files are called |cdocsch1.tex| and |cdocsch2.tex|.
%
%\iffalse
%<*samplechap1|samplechap2>
%\fi

% Optional override for |\version| flag:
%    \begin{macrocode}
%%\providecommand{\version}{final}
%    \end{macrocode}

% Include the main document:
%    \begin{macrocode}
\input{childdoc.def}
\childdocof{cdocsamp}
%    \end{macrocode}

%\iffalse
%</samplechap1|samplechap2>
%\fi
%
%\iffalse
%<*samplechap1>
%\fi
% Some text for chapter 1:
%    \begin{macrocode}
\section{one}
some text in chapter one
%    \end{macrocode}

%\iffalse
%</samplechap1>
%\fi
% Some text for chapter 2:
%\iffalse
%<*samplechap2>
%\fi
%    \begin{macrocode}
\section{two}
more text in chapter two
%    \end{macrocode}

%\iffalse
%</samplechap2>
%\fi
%
% %%%%%%%%%%%%%%%%%%%%%%%%%%%%%%%%%%%%%%
% \paragraph{Part Include Files.}
%
% The include files are called |cdocspt3.tex| and |cdocspt4.tex|.
%
%\iffalse
%<*samplepart3|samplepart4>
%\fi

% Optional override for |\version| flag:
%    \begin{macrocode}
%%\providecommand{\version}{final}
%    \end{macrocode}

% Include the main document:
%    \begin{macrocode}
\input{childdoc.def}
\childdocby{cdocsamp}
%    \end{macrocode}

%\iffalse
%</samplepart3|samplepart4>
%\fi
%
%\iffalse
%<*samplepart3>
%\fi
% Some text for part 3:
%    \begin{macrocode}
some text in part three
%    \end{macrocode}

%\iffalse
%</samplepart3>
%\fi
% Some text for part 4:
%\iffalse
%<*samplepart4>
%\fi
%    \begin{macrocode}
more text in part four
%    \end{macrocode}

%\iffalse
%</samplepart4>
%\fi
%
% %%%%%%%%%%%%%%%%%%%%%%%%%%%%%%%%%%%%%%
% \paragraph{Forwarding for a Complete Draft.}
%
% The following forwarding file |cdocsdrf.tex|
% compiles the main document in draft mode:
%\iffalse
%<*sampledraft>
%\fi
%    \begin{macrocode}
\def\version{draft}
\input{childdoc.def}
\childdocforward{cdocsamp}
%    \end{macrocode}

%\iffalse
%</sampledraft>
%\fi
%
% %%%%%%%%%%%%%%%%%%%%%%%%%%%%%%%%%%%%%%
% \paragraph{Forwarding for Final Version of the Chapters.}
%
% The following forwarding files |cdocsfn1.tex| and |cdocsfn2.tex|
% (with identical content)
% compile the final versions of the child documents
% |cdocsch1.tex| and |cdocsch2.tex|, respectively:
%\iffalse
%<*samplefinal>
%\fi
%    \begin{macrocode}
\def\version{final}
\input{childdoc.def}
\childdocforwardprefix[cdocsamp]{cdocsfn}{cdocsch}
%    \end{macrocode}

%\iffalse
%</samplefinal>
%\fi
%
% %%%%%%%%%%%%%%%%%%%%%%%%%%%%%%%%%%%%%%
% \paragraph{Command Line Processing.}
%
% The following three command lines generate the output files
% |cdocscld|, |cdocscl1| and |cdocscl2|
% which should be identical to
% |cdocsdrf|, |cdocsch1| and |cdocsfn2|, respectively:
% \begin{center}
% \begin{tabular}{l}
% |latex -jobname cdocscld \|\\
% |  "\def\version{draft}\input{childdoc.def}\childdocforward{cdocsamp}"|\\
% |latex -jobname cdocscl1 \|\\
% |  "\input{childdoc.def}\childdocforward[cdocsamp]{cdocsch1}"|\\
% |latex -jobname cdocscl2 \|\\
% |  "\def\version{final}\input{childdoc.def}\childdocforward{cdocsch2}"|
% \end{tabular}
% \end{center}
% Note that the trailing backslash on each first line
% merely continues the input to the second line
% (for convenient cut ant paste).
% Furthermore, the command |latex| can be replaced by any
% of its alternative versions such as |pdflatex|.
%
% %%%%%%%%%%%%%%%%%%%%%%%%%%%%%%%%%%%%%%%%%%%%%%%%%%%%%%%%%%%%%%%%%%%%%%%%%%%%%%
% %%%%%%%%%%%%%%%%%%%%%%%%%%%%%%%%%%%%%%%%%%%%%%%%%%%%%%%%%%%%%%%%%%%%%%%%%%%%%%
% \section{Implementation}
%\iffalse
%<*package>
%\fi
%
% This section describes the definitions file |childdoc.def|.

% The definitions cannot be loaded using |\usepackage| or |\RequirePackage|
% which has a mechanism to prevent loading a style file more than once.
% When loading the definitions by means of |\input|
% multiple instances have to be prevented manually:
%\iffalse
%This code needs to be before the `\ProvidesFile' directive
%which is defined at the beginning of this file.
%Therefore it is also placed there and commented out here.
%</package>
%<*discard>
%\fi
%    \begin{macrocode}
\ifdefined\childdocmain\endinput\fi
%    \end{macrocode}
%\iffalse
%</discard>
%<*package>
%\fi
%
% \macro{\ifchilddoc}
% \macro{\ifchilddocmanual}
% The conditional |\ifchilddoc| tells whether a
% child (true) or main (false) document is being compiled.
% The conditional |\ifchilddocmanual| tells whether
% the |\includeonly| mechanism is used (false) or
% the selection of child files must be performed manually (true).
% The definitions initialise to false:
%    \begin{macrocode}
\newif\ifchilddoc
\newif\ifchilddocmanual
%    \end{macrocode}

% \macro{\childdocname}
% \macro{\childdocjob}
% The macro |\childdocname| stores the name of the main document
% to be compiled. The macro |\childdocjob| stores the name of
% the document on which the \LaTeX{} compiler was originally invoked.
% The content of |\jobname| cannot be compared
% to filenames specified in the source due to different catcodes.
% The following code rescans |\jobname|, stores the result
% in |\childdocname| and saves a copy in |\childdocjob|:
%    \begin{macrocode}
\edef\childdocname{\scantokens\expandafter{\jobname\noexpand}}
\let\childdocjob\childdocname
%    \end{macrocode}

% \macro{\childdocdisable}
% The macro |\childdocdisable| prevents the main file
% from being processed more than once.
% At this stage, the main document command |\childdocmain|
% is assumed to be called once again where it should do nothing.
% Any subsequent call to it should prevent
% a secondary processing of the main document
% It overwrites the forwarding commands
% |\childdocof| and |\childdocforward|
% with empty macros to prevent further inclusions of the main document:
%    \begin{macrocode}
\newcommand{\childdocdisable}
{
  \renewcommand{\childdocmain}[1]{\renewcommand{\childdocmain}[1]{\endinput}}
  \renewcommand{\childdocof}[1]{}
  \renewcommand{\childdocby}[2][]{}
  \renewcommand{\childdocforward}[2][]{}
  \renewcommand{\childdocdisable}{}
}
%    \end{macrocode}

% \macro{\childdocmain}
% The macro |\childdocmain| is to be called at the top of the main file
% with nothing or the main filename (without extension) as argument.
% First, it breaks loops.
% If the argument is not empty and does not match |\childdocname|
% (which is set by the first inclusion of |childdoc.def|),
% |\ifchilddoc| is set to true, |\includeonly| is applied to the child file
% and |\jobname| is set to the main file
% (for proper handling of |.aux| files):
%    \begin{macrocode}
\newcommand{\childdocmain}[1]
{
  \childdocdisable\childdocmain{}
  \if?#1?\else
    \begingroup
      \def\childdoctmp{#1}
      \ifx\childdoctmp\childdocname
        \def\childdoctmp{}
      \else
        \def\childdoctmp
        {
          \childdoctrue
          \includeonly{\childdocname}
          \def\childdocjob{#1}
          \def\jobname{#1}
        }
      \fi
      \expandafter
    \endgroup
    \childdoctmp
  \fi
}
%    \end{macrocode}

% \macro{\childdocof}
% The command |\childdocof| redirects
% compilation to the main file |#1|.
%    \begin{macrocode}
\newcommand{\childdocof}[1]
{
  \childdocdisable
  \childdoctrue
  \includeonly{\childdocname}
  \def\jobname{#1}
  \def\childdocjob{#1}
  \input{#1}
}
%    \end{macrocode}

% \macro{\childdocby}
% The command |\childdocby| ....
%    \begin{macrocode}
\newcommand{\childdocby}[2][]
{
  \childdocdisable
  \childdoctrue
  \childdocmanualtrue
  \if?#1?\else
    \def\jobname{#2}
  \fi
  \def\childdocjob{#2}
  \input{#2}
  \endinput
}
%    \end{macrocode}

% \macro{\childdocforward}
% The command |\childdocforward| redirects
% compilation to the main file or
% (if the optional argument is given) a child file.
% Parameters are set as if the main file
% or a child file starting with |\childdocof| was compiled.
% Then compilation is handed over to the main file:
%    \begin{macrocode}
\newcommand{\childdocforward}[2][]
{
  \begingroup
    \if?#1?
      \def\childdoctmp
      {
        \def\childdocname{#2}
        \def\childdocjob{#2}
        \def\jobname{#2}
        \input{#2}
        \endinput
      }
    \else
      \def\childdoctmp
      {
        \childdocdisable
        \def\childdocname{#2}
        \childdoctrue
        \includeonly{#2}
        \def\childdocjob{#1}
        \def\jobname{#1}
        \input{#1}
        \endinput
      }
    \fi
    \expandafter
  \endgroup
  \childdoctmp
}
%    \end{macrocode}

% \macro{\childdocforwardprefix}
% The command |\childdocforwardprefix| redirects
% compilation to the main or a child file by means of a pattern.
% The prefix |#1| in the current filename is replaced by |#2|
% and the suffix of the current filename is kept
% (it is assumed that the filename does not contain the substring `|~~~|'
% which is used as a delimiter).
% Compilation is handed over to the new file by |\childdocforward|:
%    \begin{macrocode}
\newcommand{\childdocforwardprefix}[3][]
{
  \begingroup
    \def\childdocextract #2##1~~~{\def\childdoctmp{\childdocforward[#1]{#3##1}}}
    \expandafter\childdocextract\childdocname~~~
    \expandafter
  \endgroup
  \childdoctmp
}
%    \end{macrocode}

% \macro{\childdoc}
% The deprecated macro |\childdoc| is a legacy version of |\childdocmain|:
%    \begin{macrocode}
\newcommand{\childdoc}{\childdocmain}
%    \end{macrocode}

% \macro{\childdocredirect}
% The deprecated macro |\childdocredirect| is a legacy version
% of |\childdocforward| and |\childdocforwardprefix|:
%    \begin{macrocode}
\newcommand{\childdocredirect}[2][]
{
  \begingroup
    \if?#1?
      \def\childdoctmp{\childdocforward{#2}}
    \else
      \def\childdoctmp{\childdocforwardprefix{#1}{#2}}
    \fi
    \expandafter
  \endgroup
  \childdoctmp
}
%    \end{macrocode}

%\iffalse
%</package>
%\fi
%
\endinput
|\\
|\childdocmain{}|\\
\end{tabular}
\end{center}
at the very top of the main \LaTeX{} file,
in particular \emph{before} the |\documentclass| statement!
The argument of |\childdocmain| should be left empty
(but it must be present).

%%%%%%%%%%%%%%%%%%%%%%%%%%%%%%%%%%%%%%%%
\DescribeMacro{\childdocof}
Furthermore, add the commands
\begin{center}
\begin{tabular}{l}
|% \iffalse
%
% childdoc.dtx Copyright (C) 2017-2018 Niklas Beisert
%
% This work may be distributed and/or modified under the
% conditions of the LaTeX Project Public License, either version 1.3
% of this license or (at your option) any later version.
% The latest version of this license is in
%   http://www.latex-project.org/lppl.txt
% and version 1.3 or later is part of all distributions of LaTeX
% version 2005/12/01 or later.
%
% This work has the LPPL maintenance status `maintained'.
%
% The Current Maintainer of this work is Niklas Beisert.
%
% This work consists of the files childdoc.dtx and childdoc.ins
% and the derived files childdoc.def and cdocsamp.tex with
% cdocsch1.tex, cdocsch2.tex, cdocsdrf.tex, cdocsfn1.tex, cdocsfn2.tex.
%
%<package>\ifdefined\childdocmain\endinput\fi
%<package>\ProvidesFile{childdoc.def}[2018/12/30 v2.0 child document driver]
%<samplemain>\ProvidesFile{cdocsamp.tex}[2018/12/30 v2.0 sample for childdoc]
%<*driver>
%\ProvidesFile{childdoc.drv}[2018/12/30 v2.0 childdoc reference manual file]
\PassOptionsToClass{10pt,a4paper}{article}
\documentclass{ltxdoc}

\usepackage[margin=35mm]{geometry}
\usepackage{hyperref}
\usepackage{hyperxmp}
\usepackage[usenames]{color}

\hypersetup{colorlinks=true}
\hypersetup{pdfstartview=FitH}
\hypersetup{pdfpagemode=UseNone}
\hypersetup{pdfsource={}}
\hypersetup{pdflang={en-UK}}
\hypersetup{pdfcopyright={Copyright 2017-2018 Niklas Beisert.
  This work may be distributed and/or modified under the
  conditions of the LaTeX Project Public License, either version 1.3
  of this license or (at your option) any later version.}}
\hypersetup{pdflicenseurl={http://www.latex-project.org/lppl.txt}}
\hypersetup{pdfcontactaddress={ETH Zurich, ITP, HIT K,
  Wolfgang-Pauli-Strasse 27}}
\hypersetup{pdfcontactpostcode={8093}}
\hypersetup{pdfcontactcity={Zurich}}
\hypersetup{pdfcontactcountry={Switzerland}}
\hypersetup{pdfcontactemail={nbeisert@itp.phys.ethz.ch}}
\hypersetup{pdfcontacturl={http://people.phys.ethz.ch/\xmptilde nbeisert/}}

\newcommand{\secref}[1]{\hyperref[#1]{section \ref*{#1}}}

\parskip1ex
\parindent0pt
\let\olditemize\itemize
\def\itemize{\olditemize\parskip0pt}

\begin{document}

\title{The \textsf{childdoc} Package}
\hypersetup{pdftitle={The childdoc Package}}
\author{Niklas Beisert\\[2ex]
  Institut f\"ur Theoretische Physik\\
  Eidgen\"ossische Technische Hochschule Z\"urich\\
  Wolfgang-Pauli-Strasse 27, 8093 Z\"urich, Switzerland\\[1ex]
  \href{mailto:nbeisert@itp.phys.ethz.ch}
  {\texttt{nbeisert@itp.phys.ethz.ch}}}
\hypersetup{pdfauthor={Niklas Beisert}}
\hypersetup{pdfsubject={Manual for the LaTeX2e Package childdoc}}
\date{30 December 2018, \textsf{v2.0}}
\maketitle

\begin{abstract}\noindent
\textsf{childdoc} is a \LaTeXe{} package
that enables the direct compilation
of document sections included by |\include|
to individual files.
\end{abstract}

\begingroup
\parskip0ex
\tableofcontents
\endgroup

%%%%%%%%%%%%%%%%%%%%%%%%%%%%%%%%%%%%%%%%%%%%%%%%%%%%%%%%%%%%%%%%%%%%%%%%%%%%%%%%
%%%%%%%%%%%%%%%%%%%%%%%%%%%%%%%%%%%%%%%%%%%%%%%%%%%%%%%%%%%%%%%%%%%%%%%%%%%%%%%%
\section{Introduction}

\LaTeX{} provides a mechanism to structure a large document (such as a book)
into a main file and several child files (containing the chapters)
using the |\include| command.
This mechanism is beneficial for documents
which span hundreds of pages in order to
make the source file(s) more manageable.
Moreover, compilation can be restricted to
selected child files by means of the |\includeonly| command.
The latter feature can be used to reduce the compilation time while editing
(this was significantly more useful in the earlier days of \LaTeX{})
or to generate a smaller document which is easier to navigate.
Another application of |\includeonly| is to generate
documents consisting of selected parts of the complete document.

However, there are a few drawbacks of the plain |\include| mechanism:
\begin{itemize}
\item
The child files cannot be compiled on their own,
they can only be compiled via the main file.
A naive editing environment
(such as a text editor with an option
to have the current file processed by \LaTeX)
may require one to switch to the main file before compiling;
attempting to compile the child file produces errors.
\item
The main file must be modified (each time)
to adjust the |\includeonly| command
to the present needs. This easily leaves the main file in a messy state.
\item
The generated document will always carry the filename
of the main document. This is inconvenient if
several child files are to be compiled and
to be kept for distribution.
\end{itemize}

The present package provides a simple interface
to make child files individually compilable by \LaTeX{}.
Compiling a child file then has the same effect as compiling
the main file with an |\includeonly| command
to select the appropriate child.
Moreover the generated document will carry the name of the child
rather than the main file.
This resolves all three above issues.

This feature is meant to make the editing of books,
thesis documents and lecture notes somewhat more convenient.
However, the package can also be used efficiently for
composing a series of documents (such as exercise sheets)
which are typically distributed individually.
It then assists the author in generating the individual documents
(potentially in different versions)
as well as a document containing the collected series.
Another application is in developing style files
or other kinds of included material
where compilation of the style file could redirect
to a sample or test file.

%%%%%%%%%%%%%%%%%%%%%%%%%%%%%%%%%%%%%%%%%%%%%%%%%%%%%%%%%%%%%%%%%%%%%%%%%%%%%%%%
%%%%%%%%%%%%%%%%%%%%%%%%%%%%%%%%%%%%%%%%%%%%%%%%%%%%%%%%%%%%%%%%%%%%%%%%%%%%%%%%
\section{Usage}

First of all, the package \textsf{childdoc} is \emph{not} a standard
\LaTeXe{} |.sty| style file! Therefore it needs to be invoked in
a non-standard way.

%%%%%%%%%%%%%%%%%%%%%%%%%%%%%%%%%%%%%%%%%%%%%%%%%%%%%%%%%%%%%%%%%%%%%%%%%%%%%%%%
\subsection{Included Files}
\label{sec:include}

%%%%%%%%%%%%%%%%%%%%%%%%%%%%%%%%%%%%%%%%
\DescribeMacro{\childdocmain}
To use the package, add the commands
\begin{center}
\begin{tabular}{l}
|\input{childdoc.def}|\\
|\childdocmain{}|\\
\end{tabular}
\end{center}
at the very top of the main \LaTeX{} file,
in particular \emph{before} the |\documentclass| statement!
The argument of |\childdocmain| should be left empty
(but it must be present).

%%%%%%%%%%%%%%%%%%%%%%%%%%%%%%%%%%%%%%%%
\DescribeMacro{\childdocof}
Furthermore, add the commands
\begin{center}
\begin{tabular}{l}
|\input{childdoc.def}|\\
|\childdocof{|\textit{main}|}|\\
\end{tabular}
\end{center}
at the top of every child file \textit{child}
which is included by |\include{|\textit{child}|}|
from within the main file
(or at least for those files to be compiled individually).
The argument \textit{main} must be the filename of the main file.

There are a couple of
considerations in setting up the main and child documents:

%%%%%%%%%%%%%%%%%%%%%%%%%%%%%%%%%%%%%%%%
\paragraph{Restrictions.}

Please note the following restrictions:
\begin{itemize}
\item
|\childdocmain| must be called with one argument \textit{main}
to ensure compatibility with earlier version of the package.
It must either be empty (|\childdocmain{}|)
or precisely match the filename of the main file in which it is specified.
See \secref{sec:detection} for further information.
\item
The filename \textit{main} must be specified without the |.tex| extension.
\item
The filename \textit{main} is case sensitive
(even in case-insensitive file systems)
due to internal string comparison.
\item
The argument \textit{main} should be fully expanded, it cannot be a macro.
\item
Subdirectories and special characters should be avoided in filenames.
\item
The command |\childdocmain{|\textit{main}|}| must be followed by a whitespace.
It should not be followed immediately by another command
or by a comment mark `|%|'.
This is because the \TeX{} parser reads the token immediately following
the argument of |\childdocmain| and puts it
at the beginning of every child section;
however, a white\-space is ignored.
\end{itemize}

%%%%%%%%%%%%%%%%%%%%%%%%%%%%%%%%%%%%%%%%
\paragraph{Content of Main File.}

It is advisable to place all content in the child files included by |\include|.
Any output contained in the main file will appear in all child documents
unless suppressed manually;
it cannot be suppressed automatically by the |\includeonly| directive
and thus should normally be avoided.
A method to include some content in the main file
by means of conditional processing is described in \secref{sec:conditional}.

%%%%%%%%%%%%%%%%%%%%%%%%%%%%%%%%%%%%%%%%
\paragraph{Page Numbering.}

When only a part of the document is compiled,
the appropriate numbering of pages
(as well as other status parameters)
is determined from the |.aux| files.
The latter contain information from previous passes.
However this information needs to propagate through
all intermediate child documents.
Therefore the page numbering in child documents may well
be inconsistent until the complete document is compiled at least once.

A useful (if unconventional) way to always ensure a consistent
page numbering is to restart the numbering in each child document
and denote the pages by `\textit{child}|.|\textit{page}'
where \textit{child} represents the chapter/section number of the child file.
This can be achieved by the command
|\numberwithin{page}{|\textit{child}|}|
of the \textsf{amsmath} package
where \textit{child} can be |chapter| or |section|
depending on the chosen structuring.
Alternatively, one can modify the macro |\thepage| appropriately
and reset the counter |page| at the start of each child file.

%%%%%%%%%%%%%%%%%%%%%%%%%%%%%%%%%%%%%%%%%%%%%%%%%%%%%%%%%%%%%%%%%%%%%%%%%%%%%%%%
\subsection{Conditional Processing}
\label{sec:conditional}

The package provides a mechanism to compile different versions
of a document. To customise the versions further some conditional processing
can come in handy to distinguish which version is being compiled.
The package provides two macros to describe the compilation context:

%%%%%%%%%%%%%%%%%%%%%%%%%%%%%%%%%%%%%%%%
\DescribeMacro{\ifchilddoc}
The conditional |\ifchilddoc| distinguishes between the compilation of
child documents and the main document:
%
\begin{center}
|\ifchilddoc |\textit{child-code}| |[|\||else |\textit{main-code}]| \||fi|
\end{center}

%%%%%%%%%%%%%%%%%%%%%%%%%%%%%%%%%%%%%%%%
\DescribeMacro{\childdocname}
\DescribeMacro{\childdocjob}
The macro |\childdocname| contains the filename (without extension)
of the main or child file being processed.
Note that |\childdocjob| will always contain the name of the main file.

%%%%%%%%%%%%%%%%%%%%%%%%%%%%%%%%%%%%%%%%
\paragraph{Title Page.}

Conditional processing can be used to include a title or banner page
in the main document when proper precautions are taken.
Importantly, the code in the main file should ensure that the page counter
(as well as other status parameters which are stored in the |.aux| files)
takes the same value after the conditional processing.
Otherwise the page numbers may take divergent values
depending on which part is compiled.

For example, a title page could be declared by:
%
\begin{center}
\begin{tabular}{l}
|\ifchilddoc\||else|\\
|\addtocounter{page}{-1}|\\
\textit{code for title page}\\
|\newpage|\\
|\||fi|
\end{tabular}
\end{center}
%
A banner page for the child documents can be generated by:
%
\begin{center}
\begin{tabular}{l}
|\ifchilddoc|\\
|\addtocounter{page}{-1}|\\
\textit{code for banner page}\\
|\newpage|\\
|\||fi|
\end{tabular}
\end{center}
%
Here one could write a message such as:
\begin{center}
|This is the part \childdocname{} of \childdocjob{}.|
\end{center}

%%%%%%%%%%%%%%%%%%%%%%%%%%%%%%%%%%%%%%%%%%%%%%%%%%%%%%%%%%%%%%%%%%%%%%%%%%%%%%%%
\subsection{Flags}
\label{sec:flags}

The package makes it easy to generate different versions
of the main or child documents.
To this end compilation flags can be defined
and assigned different default values.
They will be particularly useful in conjunction
with the forwarding mechanism described in \secref{sec:forward}.

For example, it may be useful to have a flag |\version|
which can be set to |draft| or |final|.
The document source will contain some conditional code
depending on the value of |\version|.
Suppose further, the flag should default to |final| for the main file
and to |draft| for child files
which is a natural assignment for editing the document.
This is achieved by placing the following code
in the preamble of the main document
(below the |\childdocmain| directive):
%
\begin{center}
\begin{tabular}{l}
|\ifchilddoc|\\
|\providecommand{\version}{draft}|\\
|\||else|\\
|\providecommand{\version}{final}|\\
|\||fi|
\end{tabular}
\end{center}
%
The definition by |\providecommand| makes sure
that previous definitions are not overwritten.
Further statements |\providecommand{\version}{...}|
can thus be added before the above code to override it.

For the main file, one might add a line
(between |\childdocmain| and the above block)
%
\begin{center}
|%\ifchilddoc\||else\providecommand{\version}{draft}\||fi|
\end{center}
%
which can be uncommented to produce a draft version.
Likewise one can add a line to the very top of a child file
(above the |\childdocof{|\textit{main}|}| directive)
%
\begin{center}
|%\providecommand{\version}{final}|
\end{center}
%
which can be uncommented to produce the final version of this child document.

%%%%%%%%%%%%%%%%%%%%%%%%%%%%%%%%%%%%%%%%%%%%%%%%%%%%%%%%%%%%%%%%%%%%%%%%%%%%%%%%
\subsection{Forwarding}
\label{sec:forward}

Different versions of the main or child documents
using compilation flags as described in \secref{sec:flags}
can be (permanently) stored in different files
for convenient compilation, viewing and distribution.
To this end, the package defines a command
to pass on compilation to a different file:

%%%%%%%%%%%%%%%%%%%%%%%%%%%%%%%%%%%%%%%%
\DescribeMacro{\childdocforward}
The command |\childdocforward| redirects processing to
another source file:
%
\begin{center}
\begin{tabular}{l}
|\input{childdoc.def}|\\
|\childdocforward[|\textit{main}|]{|\textit{dest}|}|\\
\end{tabular}
\end{center}
%
The argument \textit{dest} is the destination file
(without extension).
It should be the main file or one of the child files.
Note that further \textsf{childdoc} directives
such as |\childdocof| and |\childdocforward|
in the indicated file will be processed in this form.
The optional argument \textit{main}
passes on directly to the main file \textit{main}
while pretending to compile the child \textit{dest}.
This form behaves as if \textit{dest}
issues |\childdocof{|\textit{main}|}| right away,
and no further \textsf{childdoc} directives will be processed.

%%%%%%%%%%%%%%%%%%%%%%%%%%%%%%%%%%%%%%%%
\DescribeMacro{\...prefix}
In the alternative form |\childdocforwardprefix|,
%
\begin{center}
\begin{tabular}{l}
|\input{childdoc.def}|\\
|\childdocforwardprefix[|\textit{main}|]{|\textit{prefix}|}{|\textit{dest}|}|
\end{tabular}
\end{center}
%
the destination file is determined by a pattern
depending on the current file:
To make this work, the current file must be called
`{\textit{prefix}\hspace{0.2em}\textit{suffix}}'
with \textit{prefix} matching precisely the argument.
Processing is then passed on to the file
`{\textit{dest}\hspace{0.2em}\textit{suffix}}'.
Surely, the same effect is achieved by
directly specifying the
argument `{\textit{dest}\hspace{0.2em}\textit{suffix}}'
in the first form.
However, that requires to set up a different file
for each child. With the alternative form of the command
all these files can have exactly the same content
which simplifies setting them up and maintaining them.

For example, the following file |draft.tex|
with a compilation flag |\version| as described in \secref{sec:flags}
compiles the main document as a draft:
%
\begin{center}
\begin{tabular}{l}
|\def\version{draft}|\\
|\input{childdoc.def}|\\
|\childdocforward{|\textit{main}|}|
\end{tabular}
\end{center}
%
Likewise, the following files |final|\textit{nn}|.tex|
compile the final version of the child document
|child|\textit{nn}|.tex|:
%
\begin{center}
\begin{tabular}{l}
|\def\version{final}|\\
|\input{childdoc.def}|\\
|\childdocforwardprefix{final}{child}|
\end{tabular}
\end{center}
%

Note that when several versions of a main file and/or of each child file
are to be generated, it may be convenient to set up a |Makefile| or
shell script to automatise the process.

%%%%%%%%%%%%%%%%%%%%%%%%%%%%%%%%%%%%%%%%%%%%%%%%%%%%%%%%%%%%%%%%%%%%%%%%%%%%%%%%
\subsection{Command Line Processing}
\label{sec:commandline}

The effect of redirection files can also be achieved by invoking
the \LaTeX{} compiler with a more elaborate command line.
Most conveniently this should be done as part
of a shell script or a |Makefile|.

When using \textsf{childdoc} in the main file, the following
command lines effectively perform a redirection
(note that depending on the shell being used,
backslashes may have to be doubled: `|\|' $\to$ `|\\|'):
%
\begin{center}
|... -jobname "|\textit{target}|" |\\|"|[\textit{flags}]%
|\input{childdoc.def}\childdocforward[|\textit{main}|]{|\textit{dest}|}"|
\end{center}
%
Here \textit{target} is the name of the output file,
\textit{main} is the name of the main file
and \textit{dest} is the name of the main or child file to be processed
(all filenames without extensions).
The optional argument \textit{main} can be omitted
if \textit{main} matches \textit{dest}.
Optionally, compilation \textit{flags} can be defined via |\def| commands.
This command line makes the \TeX{} engine believe
it is compiling the file \textit{target}
whose content is specified as the latter parameter.
The provided code then forwards the processing to
\textit{main} or \textit{dest} as described in \secref{sec:forward}.

%%%%%%%%%%%%%%%%%%%%%%%%%%%%%%%%%%%%%%%%%%%%%%%%%%%%%%%%%%%%%%%%%%%%%%%%%%%%%%%%
\subsection{Include by Input}
\label{sec:input}

Including child documents by |\include| has some restrictions by design.
Most notably, the content of a child document always occupies
its own set of pages; pages cannot be shared between child documents.
Usually, this behaviour makes perfect sense
because each child document contain an essential part of the document.
However, in some situations it may be desirable to compose
a document from a collection of parts
without having mandatory page breaks between then.
For this case, the package
provides a mechanism to include parts
by |\input| which can also be processed individually.
However, by construction this mechanism
requires manual handling of the content to be output.

%%%%%%%%%%%%%%%%%%%%%%%%%%%%%%%%%%%%%%%%
\DescribeMacro{\ifchilddocmanual}
The main file should be prepared as usual, see \secref{sec:include}.
However, the document body must make a distinction
between processing of an individual part and of the main document, e.g.:
%
\begin{center}
\begin{tabular}{l}
|\ifchilddocmanual|\\
|\input{\childdocname}|\\
|\||else|\\
\textit{document body with }|\input{|\textit{part}|}|\\
|\||fi|
\end{tabular}
\end{center}
%
The conditional |\ifchilddocmanual| is true whenever
a part to be included by |\input| is being compiled,
and the name of the part is stored in |\childdocname|.

%%%%%%%%%%%%%%%%%%%%%%%%%%%%%%%%%%%%%%%%
\DescribeMacro{\childdocby}
Each part to be included by |\input| should start with:
%
\begin{center}
\begin{tabular}{l}
|\input{childdoc.def}|\\
|\childdocby{|\textit{main}|}|\\
\end{tabular}
\end{center}
%
The directive |\childdocby| is similar to |\childdocof|
described in \secref{sec:include},
but the subsequent selection of content must be done manually.
To that end, both |\ifchilddoc| and |\ifchilddocmanual|
will be true upon processing of a part,
and the name of the part is stored in |\childdocname|.
Note that |\jobname| will be set to the filename of the current part
so that each part receives an individual |.aux| file
that does not interfere with the |.aux| file(s) of the main document.
This behaviour can be altered by the alternative form
|\childdocby[*]{|\textit{main}|}| (with a non-empty optional argument)
which uses the |.aux| file of the main document
by setting |\jobname| to \textit{main}.

%%%%%%%%%%%%%%%%%%%%%%%%%%%%%%%%%%%%%%%%%%%%%%%%%%%%%%%%%%%%%%%%%%%%%%%%%%%%%%%%
\subsection{Driver Development}
\label{sec:driver}

The \textsf{childdoc} mechanism can also be use for the development
of definition files such as \LaTeX{} styles or classes.
This case differs from the above setup with multiple parts
included by |\include| in that no |\includeonly| should be invoked.
This can be achieved by starting the include file
(before |\ProvidesPackage|) with:
%
\begin{center}
\begin{tabular}{l}
|\input{childdoc.def}|\\
|\childdocforward{|\textit{main}|}|\\
\end{tabular}
\end{center}
%
or alternatively with:
%
\begin{center}
\begin{tabular}{l}
|\input{childdoc.def}|\\
|\childdocby{|\textit{main}|}|\\
\end{tabular}
\end{center}
%
Both forms have slightly different effects as described above.
The main file is prepared as usual, see \secref{sec:include}.

%%%%%%%%%%%%%%%%%%%%%%%%%%%%%%%%%%%%%%%%%%%%%%%%%%%%%%%%%%%%%%%%%%%%%%%%%%%%%%%%
\subsection{Legacy Detection}
\label{sec:detection}

The directive |\childdocmain| in the main file can detect
whether the complete document or merely a child is to be compiled
even without using the directive |\childdocof|.
This method is deprecated because it is less robust
and there is no compelling reason to use it;
it is merely provided for backward compatibility
and it may be removed in future versions.

If the detection mechanism is to be used,
it is mandatory to correctly specify
the filename of the main file as the argument of |\childdocmain|:
%
\begin{center}
\begin{tabular}{l}
|\input{childdoc.def}|\\
|\childdocmain{|\textit{main}|}|\\
\end{tabular}
\end{center}
%
If |\jobname| does not match the argument \textit{main} of |\childdocmain|,
it is assumed that |\jobname| points to the child file to be compiled.
When using |\childdocmain| with the main file specified as argument,
it suffices to start a child file
with just |\input{|\textit{main}|}|
without loading of the package and using |\childdocof|.
If instead all processing is done
with the appropriate \textsf{childdoc} directives,
the argument of \textit{main} of |\childdocmain| can be empty.

An alternative version of the command line processing described
in \secref{sec:commandline} using the detection mechanism reads:
%
\begin{center}
|... -jobname "|\textit{target}|" "|[\textit{flags}]%
[|\def\jobname{|\textit{dest}|}|]|\input{|\textit{main}|}"|
\end{center}

%%%%%%%%%%%%%%%%%%%%%%%%%%%%%%%%%%%%%%%%%%%%%%%%%%%%%%%%%%%%%%%%%%%%%%%%%%%%%%%%
\subsection{Manual Code}
\label{sec:manual}

In case one cannot be certain whether the definitions file |childdoc.def|
is installed on the target \TeX{} distribution
and one prefers not to ship it,
it is conceivable to paste a few relevant commands into the sources.

To that end, drop all statements |\input{childdoc.def}|
and perform the replacements as outlined below.
Instead of |\childdocmain{|\textit{main}|}| add the following code
to the top of the main file:
%
\begin{center}
\begin{tabular}{l}
|\||ifdefined\childdocname\endinput\||fi\newif\ifchilddoc|\\
|\edef\childdocname{\scantokens\expandafter{\jobname\noexpand}}|\\
|\def\childdocmain{|\textit{main}|}\||ifx\childdocmain\childdocname\||else|\\
|\childdoctrue\includeonly{\childdocname}\let\jobname\childdocmain\||fi|\\
\end{tabular}
\end{center}
%
Instead of |\childdocof{|\textit{main}|}| just include the main file
at the top of each child file:
%
\begin{center}
|\input{|\textit{main}|}|
\end{center}
%
A simple redirection |\childdocforward{|\textit{dest}|}| is achieved by:
%
\begin{center}
|\def\jobname{|\textit{dest}|}\input{\jobname}|
\end{center}
%
The redirection with prefix
|\childdocforwardprefix[|\textit{prefix}|]{|\textit{dest}|}|
is accomplished by:
%
\begin{center}
\begin{tabular}{l}
|{\edef\jobname{\scantokens\expandafter{\jobname\noexpand}}|\\
|\def\redirectjob |\textit{prefix}|#1~~~{\gdef\jobname{|\textit{dest}|#1}}|\\
|\expandafter\redirectjob\jobname~~~}\input{\jobname}|
\end{tabular}
\end{center}

In an alternative approach,
child documents can be compiled by a specific command line
without additional code or specific definitions:
%
\begin{center}
|... -jobname "|\textit{target}|" "|[\textit{flags}]%
|\includeonly{|\textit{dest}|}\input{|\textit{main}|}"|
\end{center}
%

%%%%%%%%%%%%%%%%%%%%%%%%%%%%%%%%%%%%%%%%%%%%%%%%%%%%%%%%%%%%%%%%%%%%%%%%%%%%%%%%
%%%%%%%%%%%%%%%%%%%%%%%%%%%%%%%%%%%%%%%%%%%%%%%%%%%%%%%%%%%%%%%%%%%%%%%%%%%%%%%%
\section{Information}

%%%%%%%%%%%%%%%%%%%%%%%%%%%%%%%%%%%%%%%%%%%%%%%%%%%%%%%%%%%%%%%%%%%%%%%%%%%%%%%%
\subsection{Copyright}

Copyright \copyright{} 2017--2018 Niklas Beisert

This work may be distributed and/or modified under the
conditions of the \LaTeX{} Project Public License, either version 1.3
of this license or (at your option) any later version.
The latest version of this license is in
  \url{http://www.latex-project.org/lppl.txt}
and version 1.3 or later is part of all distributions of \LaTeX{}
version 2005/12/01 or later.

This work has the LPPL maintenance status `maintained'.

The Current Maintainer of this work is Niklas Beisert.

This work consists of the files |README.txt|, |childdoc.ins| and |childdoc.dtx|
as well as the derived files |childdoc.def|, |cdocsamp.tex|
with |cdocsch1.tex|, |cdocsch2.tex|, |cdocspt3.tex|, |cdocspt4.tex|,
|cdocsdrf.tex|, |cdocsfn1.tex|, |cdocsfn2.tex|
as well as |childdoc.pdf|.

%%%%%%%%%%%%%%%%%%%%%%%%%%%%%%%%%%%%%%%%%%%%%%%%%%%%%%%%%%%%%%%%%%%%%%%%%%%%%%%%
\subsection{Files and Installation}

The package consists of the files:
%
\begin{center}
\begin{tabular}{ll}
    |README.txt|   & readme file \\
    |childdoc.ins| & installation file \\
    |childdoc.dtx| & source file \\
    |childdoc.def| & definition file \\
    |cdocsamp.tex| & sample main file \\
    |cdocsch1.tex| & sample include file \\
    |cdocsch2.tex| & sample include file \\
    |cdocspt3.tex| & sample part file \\
    |cdocspt4.tex| & sample part file \\
    |cdocsdrf.tex| & sample redirection file \\
    |cdocsfn1.tex| & sample redirection file \\
    |cdocsfn2.tex| & sample redirection file \\
    |childdoc.pdf| & manual
\end{tabular}
\end{center}
%
The distribution consists of the files
|README.txt|, |childdoc.ins| and |childdoc.dtx|.
%
\begin{itemize}
\item
Run (pdf)\LaTeX{} on |childdoc.dtx|
to compile the manual |childdoc.pdf| (this file).
\item
Run \LaTeX{} on |childdoc.ins| to create the definitions file |childdoc.def|
and the sample |cdocsamp.tex| with include files
|cdocsch1.tex|, |cdocsch2.tex|, |cdocspt3.tex|, |cdocspt4.tex|,
|cdocsdrf.tex|, |cdocsfn1.tex|, |cdocsfn2.tex|.
Then copy the file |childdoc.def| to an appropriate directory of your \LaTeX{}
distribution, e.g.\ \textit{texmf-root}|/tex/latex/childdoc|.
\end{itemize}

%%%%%%%%%%%%%%%%%%%%%%%%%%%%%%%%%%%%%%%%%%%%%%%%%%%%%%%%%%%%%%%%%%%%%%%%%%%%%%%%
\subsection{Related CTAN Packages}

There are several other packages which offer a similar functionality:
%
\begin{itemize}
\item
The packages
\href{http://ctan.org/pkg/docmute}{\textsf{docmute}},
\href{http://ctan.org/pkg/includex}{\textsf{includex}} and
\href{http://ctan.org/pkg/standalone}{\textsf{standalone}}
provide commands to include only the document body of
a child file thus allowing both files to be compiled individually.
\item
The packages \href{http://ctan.org/pkg/subdocs}{\textsf{subdocs}}
and \href{http://ctan.org/pkg/subfiles}{\textsf{subfiles}}
provide structures in which the main and child documents can be
encapsulated and allowing them to be compiled individually.
The inclusion mechanism is different from the conventional |\include|.
\item
The package \href{http://ctan.org/pkg/combine}{\textsf{combine}}
is an elaborate solution to combine several documents into one.
\end{itemize}
%
See also the CTAN topic \href{http://ctan.org/topic/subdocs}{\textsf{subdocs}}
for further related packages.
The present package differs from the above solutions in that
a document structure constructed with the conventional |\include| mechanism
just needs two extra commands at the top of every file
such that all constituent files can be compiled individually.

%%%%%%%%%%%%%%%%%%%%%%%%%%%%%%%%%%%%%%%%%%%%%%%%%%%%%%%%%%%%%%%%%%%%%%%%%%%%%%%%
%\subsection{Feature Suggestions}
%
%The following is a list of features which may be useful for future
%versions of this package:
%%
%\begin{itemize}
%\item
%\ldots
%\end{itemize}

%%%%%%%%%%%%%%%%%%%%%%%%%%%%%%%%%%%%%%%%%%%%%%%%%%%%%%%%%%%%%%%%%%%%%%%%%%%%%%%%
\subsection{Revision History}

%%%%%%%%%%%%%%%%%%%%%%%%%%%%%%%%%%%%%%%%
\paragraph{v2.0:} 2018/12/30

\begin{itemize}
\item
immediate forward processing
\item
added |\childdocby| mechanism
\item
manual restructured
\end{itemize}

%%%%%%%%%%%%%%%%%%%%%%%%%%%%%%%%%%%%%%%%
\paragraph{v1.6:} 2018/01/17

\begin{itemize}
\item
application for development of include files
\item
corrections to manual
\end{itemize}

%%%%%%%%%%%%%%%%%%%%%%%%%%%%%%%%%%%%%%%%
\paragraph{v1.5:} 2017/05/21

\begin{itemize}
\item
more complete structuring introduced
\item
|\childdocof| introduced
\item
|\childdoc| renamed to |\childdocmain|
\item
|\childredirect| renamed to |\childdocforward| and |\childdocforwardprefix|
and functionality expanded
\end{itemize}

%%%%%%%%%%%%%%%%%%%%%%%%%%%%%%%%%%%%%%%%
\paragraph{v1.0:} 2017/04/27

\begin{itemize}
\item
manual and install package
\item
first version published on CTAN
\end{itemize}

%%%%%%%%%%%%%%%%%%%%%%%%%%%%%%%%%%%%%%%%
\paragraph{v0.6:} 2017/04/26

\begin{itemize}
\item
redirection mechanism added
\end{itemize}

%%%%%%%%%%%%%%%%%%%%%%%%%%%%%%%%%%%%%%%%
\paragraph{v0.5:} 2017/04/26

\begin{itemize}
\item
functionality in definition file
\end{itemize}


%%%%%%%%%%%%%%%%%%%%%%%%%%%%%%%%%%%%%%%%%%%%%%%%%%%%%%%%%%%%%%%%%%%%%%%%%%%%%%%%
%%%%%%%%%%%%%%%%%%%%%%%%%%%%%%%%%%%%%%%%%%%%%%%%%%%%%%%%%%%%%%%%%%%%%%%%%%%%%%%%
%%%%%%%%%%%%%%%%%%%%%%%%%%%%%%%%%%%%%%%%%%%%%%%%%%%%%%%%%%%%%%%%%%%%%%%%%%%%%%%%
\appendix

\settowidth\MacroIndent{\rmfamily\scriptsize 000\ }

 \DocInput{childdoc.dtx}

\end{document}
%</driver>
% \fi
%
% %%%%%%%%%%%%%%%%%%%%%%%%%%%%%%%%%%%%%%%%%%%%%%%%%%%%%%%%%%%%%%%%%%%%%%%%%%%%%%
% %%%%%%%%%%%%%%%%%%%%%%%%%%%%%%%%%%%%%%%%%%%%%%%%%%%%%%%%%%%%%%%%%%%%%%%%%%%%%%
% \section{Sample}
%\iffalse
%<*samplemain>
%\fi
%
% The following presents a sample document
% with two chapters, two parts, a title page,
% a compile flag as well as three forwarding files to set the flag.
% It consists of eight |.tex| files:
% \begin{center}
% \begin{tabular}{ll}
% |cdocsamp.tex|&main file\\
% |cdocsch1.tex|&include file for chapter 1\\
% |cdocsch2.tex|&include file for chapter 2\\
% |cdocspt3.tex|&include file for part 3\\
% |cdocspt4.tex|&include file for part 4\\
% |cdocsdrf.tex|&forwarding file for main file in draft mode\\
% |cdocsfi1.tex|&forwarding file for final version of chapter 1\\
% |cdocsfi2.tex|&forwarding file for final version of chapter 2\\
% \end{tabular}
% \end{center}
% Each of the eight files can be compiled directly by the \LaTeX{} compiler.
%
% %%%%%%%%%%%%%%%%%%%%%%%%%%%%%%%%%%%%%%
% \paragraph{Main File.}
%
% The main file is called |cdocsamp.tex|.
%
% Load the \textsf{childdoc} definitions and
% declare the filename for the main document:
%    \begin{macrocode}
\input{childdoc.def}
\childdocmain{}
%    \end{macrocode}

% Optional override for |\version| flag:
%    \begin{macrocode}
%%\ifchilddoc\else\providecommand{\version}{draft}\fi
%    \end{macrocode}

% Define the default values for the |\version| flag
% (|final| for the main file and |draft| for childs):
%    \begin{macrocode}
\ifchilddoc
\providecommand{\version}{draft}
\else
\providecommand{\version}{final}
\fi
%    \end{macrocode}

% Load the standard document class:
%    \begin{macrocode}
\documentclass[12pt]{article}
%    \end{macrocode}

% Start the document body:
%    \begin{macrocode}
\begin{document}
%    \end{macrocode}

% Declare a title page.
% Print title, part of document being processed and version flag:
%    \begin{macrocode}
\addtocounter{page}{-1}
\begin{center}
{\LARGE\bfseries{}childdoc example\par}
\vspace{1cm}
\ifchilddoc
\ifchilddocmanual part\else chapter\fi:
`\childdocname' of `\childdocjob'\par
\else
main document: `\childdocjob'\par
\fi
version: \version\par
\end{center}
\newpage
%    \end{macrocode}

% Manually include selected file,
% otherwise process as usual:
%    \begin{macrocode}
\ifchilddocmanual
\section*{part `\childdocname'}
\input{\childdocname}
\else
%    \end{macrocode}

% Include the two chapters:
%    \begin{macrocode}
\include{cdocsch1}
\include{cdocsch2}
%    \end{macrocode}

% Include the two parts unless only chapters should be displayed:
%    \begin{macrocode}
\ifchilddoc\else
\section{part three}
\input{cdocspt3}
\section{part four}
\input{cdocspt4}
\fi
%    \end{macrocode}

% Process as usual until here:
%    \begin{macrocode}
\fi
%    \end{macrocode}

% End of document body:
%    \begin{macrocode}
\end{document}
%    \end{macrocode}
%\iffalse
%</samplemain>
%\fi
%
% %%%%%%%%%%%%%%%%%%%%%%%%%%%%%%%%%%%%%%
% \paragraph{Chapter Include Files.}
%
% The include files are called |cdocsch1.tex| and |cdocsch2.tex|.
%
%\iffalse
%<*samplechap1|samplechap2>
%\fi

% Optional override for |\version| flag:
%    \begin{macrocode}
%%\providecommand{\version}{final}
%    \end{macrocode}

% Include the main document:
%    \begin{macrocode}
\input{childdoc.def}
\childdocof{cdocsamp}
%    \end{macrocode}

%\iffalse
%</samplechap1|samplechap2>
%\fi
%
%\iffalse
%<*samplechap1>
%\fi
% Some text for chapter 1:
%    \begin{macrocode}
\section{one}
some text in chapter one
%    \end{macrocode}

%\iffalse
%</samplechap1>
%\fi
% Some text for chapter 2:
%\iffalse
%<*samplechap2>
%\fi
%    \begin{macrocode}
\section{two}
more text in chapter two
%    \end{macrocode}

%\iffalse
%</samplechap2>
%\fi
%
% %%%%%%%%%%%%%%%%%%%%%%%%%%%%%%%%%%%%%%
% \paragraph{Part Include Files.}
%
% The include files are called |cdocspt3.tex| and |cdocspt4.tex|.
%
%\iffalse
%<*samplepart3|samplepart4>
%\fi

% Optional override for |\version| flag:
%    \begin{macrocode}
%%\providecommand{\version}{final}
%    \end{macrocode}

% Include the main document:
%    \begin{macrocode}
\input{childdoc.def}
\childdocby{cdocsamp}
%    \end{macrocode}

%\iffalse
%</samplepart3|samplepart4>
%\fi
%
%\iffalse
%<*samplepart3>
%\fi
% Some text for part 3:
%    \begin{macrocode}
some text in part three
%    \end{macrocode}

%\iffalse
%</samplepart3>
%\fi
% Some text for part 4:
%\iffalse
%<*samplepart4>
%\fi
%    \begin{macrocode}
more text in part four
%    \end{macrocode}

%\iffalse
%</samplepart4>
%\fi
%
% %%%%%%%%%%%%%%%%%%%%%%%%%%%%%%%%%%%%%%
% \paragraph{Forwarding for a Complete Draft.}
%
% The following forwarding file |cdocsdrf.tex|
% compiles the main document in draft mode:
%\iffalse
%<*sampledraft>
%\fi
%    \begin{macrocode}
\def\version{draft}
\input{childdoc.def}
\childdocforward{cdocsamp}
%    \end{macrocode}

%\iffalse
%</sampledraft>
%\fi
%
% %%%%%%%%%%%%%%%%%%%%%%%%%%%%%%%%%%%%%%
% \paragraph{Forwarding for Final Version of the Chapters.}
%
% The following forwarding files |cdocsfn1.tex| and |cdocsfn2.tex|
% (with identical content)
% compile the final versions of the child documents
% |cdocsch1.tex| and |cdocsch2.tex|, respectively:
%\iffalse
%<*samplefinal>
%\fi
%    \begin{macrocode}
\def\version{final}
\input{childdoc.def}
\childdocforwardprefix[cdocsamp]{cdocsfn}{cdocsch}
%    \end{macrocode}

%\iffalse
%</samplefinal>
%\fi
%
% %%%%%%%%%%%%%%%%%%%%%%%%%%%%%%%%%%%%%%
% \paragraph{Command Line Processing.}
%
% The following three command lines generate the output files
% |cdocscld|, |cdocscl1| and |cdocscl2|
% which should be identical to
% |cdocsdrf|, |cdocsch1| and |cdocsfn2|, respectively:
% \begin{center}
% \begin{tabular}{l}
% |latex -jobname cdocscld \|\\
% |  "\def\version{draft}\input{childdoc.def}\childdocforward{cdocsamp}"|\\
% |latex -jobname cdocscl1 \|\\
% |  "\input{childdoc.def}\childdocforward[cdocsamp]{cdocsch1}"|\\
% |latex -jobname cdocscl2 \|\\
% |  "\def\version{final}\input{childdoc.def}\childdocforward{cdocsch2}"|
% \end{tabular}
% \end{center}
% Note that the trailing backslash on each first line
% merely continues the input to the second line
% (for convenient cut ant paste).
% Furthermore, the command |latex| can be replaced by any
% of its alternative versions such as |pdflatex|.
%
% %%%%%%%%%%%%%%%%%%%%%%%%%%%%%%%%%%%%%%%%%%%%%%%%%%%%%%%%%%%%%%%%%%%%%%%%%%%%%%
% %%%%%%%%%%%%%%%%%%%%%%%%%%%%%%%%%%%%%%%%%%%%%%%%%%%%%%%%%%%%%%%%%%%%%%%%%%%%%%
% \section{Implementation}
%\iffalse
%<*package>
%\fi
%
% This section describes the definitions file |childdoc.def|.

% The definitions cannot be loaded using |\usepackage| or |\RequirePackage|
% which has a mechanism to prevent loading a style file more than once.
% When loading the definitions by means of |\input|
% multiple instances have to be prevented manually:
%\iffalse
%This code needs to be before the `\ProvidesFile' directive
%which is defined at the beginning of this file.
%Therefore it is also placed there and commented out here.
%</package>
%<*discard>
%\fi
%    \begin{macrocode}
\ifdefined\childdocmain\endinput\fi
%    \end{macrocode}
%\iffalse
%</discard>
%<*package>
%\fi
%
% \macro{\ifchilddoc}
% \macro{\ifchilddocmanual}
% The conditional |\ifchilddoc| tells whether a
% child (true) or main (false) document is being compiled.
% The conditional |\ifchilddocmanual| tells whether
% the |\includeonly| mechanism is used (false) or
% the selection of child files must be performed manually (true).
% The definitions initialise to false:
%    \begin{macrocode}
\newif\ifchilddoc
\newif\ifchilddocmanual
%    \end{macrocode}

% \macro{\childdocname}
% \macro{\childdocjob}
% The macro |\childdocname| stores the name of the main document
% to be compiled. The macro |\childdocjob| stores the name of
% the document on which the \LaTeX{} compiler was originally invoked.
% The content of |\jobname| cannot be compared
% to filenames specified in the source due to different catcodes.
% The following code rescans |\jobname|, stores the result
% in |\childdocname| and saves a copy in |\childdocjob|:
%    \begin{macrocode}
\edef\childdocname{\scantokens\expandafter{\jobname\noexpand}}
\let\childdocjob\childdocname
%    \end{macrocode}

% \macro{\childdocdisable}
% The macro |\childdocdisable| prevents the main file
% from being processed more than once.
% At this stage, the main document command |\childdocmain|
% is assumed to be called once again where it should do nothing.
% Any subsequent call to it should prevent
% a secondary processing of the main document
% It overwrites the forwarding commands
% |\childdocof| and |\childdocforward|
% with empty macros to prevent further inclusions of the main document:
%    \begin{macrocode}
\newcommand{\childdocdisable}
{
  \renewcommand{\childdocmain}[1]{\renewcommand{\childdocmain}[1]{\endinput}}
  \renewcommand{\childdocof}[1]{}
  \renewcommand{\childdocby}[2][]{}
  \renewcommand{\childdocforward}[2][]{}
  \renewcommand{\childdocdisable}{}
}
%    \end{macrocode}

% \macro{\childdocmain}
% The macro |\childdocmain| is to be called at the top of the main file
% with nothing or the main filename (without extension) as argument.
% First, it breaks loops.
% If the argument is not empty and does not match |\childdocname|
% (which is set by the first inclusion of |childdoc.def|),
% |\ifchilddoc| is set to true, |\includeonly| is applied to the child file
% and |\jobname| is set to the main file
% (for proper handling of |.aux| files):
%    \begin{macrocode}
\newcommand{\childdocmain}[1]
{
  \childdocdisable\childdocmain{}
  \if?#1?\else
    \begingroup
      \def\childdoctmp{#1}
      \ifx\childdoctmp\childdocname
        \def\childdoctmp{}
      \else
        \def\childdoctmp
        {
          \childdoctrue
          \includeonly{\childdocname}
          \def\childdocjob{#1}
          \def\jobname{#1}
        }
      \fi
      \expandafter
    \endgroup
    \childdoctmp
  \fi
}
%    \end{macrocode}

% \macro{\childdocof}
% The command |\childdocof| redirects
% compilation to the main file |#1|.
%    \begin{macrocode}
\newcommand{\childdocof}[1]
{
  \childdocdisable
  \childdoctrue
  \includeonly{\childdocname}
  \def\jobname{#1}
  \def\childdocjob{#1}
  \input{#1}
}
%    \end{macrocode}

% \macro{\childdocby}
% The command |\childdocby| ....
%    \begin{macrocode}
\newcommand{\childdocby}[2][]
{
  \childdocdisable
  \childdoctrue
  \childdocmanualtrue
  \if?#1?\else
    \def\jobname{#2}
  \fi
  \def\childdocjob{#2}
  \input{#2}
  \endinput
}
%    \end{macrocode}

% \macro{\childdocforward}
% The command |\childdocforward| redirects
% compilation to the main file or
% (if the optional argument is given) a child file.
% Parameters are set as if the main file
% or a child file starting with |\childdocof| was compiled.
% Then compilation is handed over to the main file:
%    \begin{macrocode}
\newcommand{\childdocforward}[2][]
{
  \begingroup
    \if?#1?
      \def\childdoctmp
      {
        \def\childdocname{#2}
        \def\childdocjob{#2}
        \def\jobname{#2}
        \input{#2}
        \endinput
      }
    \else
      \def\childdoctmp
      {
        \childdocdisable
        \def\childdocname{#2}
        \childdoctrue
        \includeonly{#2}
        \def\childdocjob{#1}
        \def\jobname{#1}
        \input{#1}
        \endinput
      }
    \fi
    \expandafter
  \endgroup
  \childdoctmp
}
%    \end{macrocode}

% \macro{\childdocforwardprefix}
% The command |\childdocforwardprefix| redirects
% compilation to the main or a child file by means of a pattern.
% The prefix |#1| in the current filename is replaced by |#2|
% and the suffix of the current filename is kept
% (it is assumed that the filename does not contain the substring `|~~~|'
% which is used as a delimiter).
% Compilation is handed over to the new file by |\childdocforward|:
%    \begin{macrocode}
\newcommand{\childdocforwardprefix}[3][]
{
  \begingroup
    \def\childdocextract #2##1~~~{\def\childdoctmp{\childdocforward[#1]{#3##1}}}
    \expandafter\childdocextract\childdocname~~~
    \expandafter
  \endgroup
  \childdoctmp
}
%    \end{macrocode}

% \macro{\childdoc}
% The deprecated macro |\childdoc| is a legacy version of |\childdocmain|:
%    \begin{macrocode}
\newcommand{\childdoc}{\childdocmain}
%    \end{macrocode}

% \macro{\childdocredirect}
% The deprecated macro |\childdocredirect| is a legacy version
% of |\childdocforward| and |\childdocforwardprefix|:
%    \begin{macrocode}
\newcommand{\childdocredirect}[2][]
{
  \begingroup
    \if?#1?
      \def\childdoctmp{\childdocforward{#2}}
    \else
      \def\childdoctmp{\childdocforwardprefix{#1}{#2}}
    \fi
    \expandafter
  \endgroup
  \childdoctmp
}
%    \end{macrocode}

%\iffalse
%</package>
%\fi
%
\endinput
|\\
|\childdocof{|\textit{main}|}|\\
\end{tabular}
\end{center}
at the top of every child file \textit{child}
which is included by |\include{|\textit{child}|}|
from within the main file
(or at least for those files to be compiled individually).
The argument \textit{main} must be the filename of the main file.

There are a couple of
considerations in setting up the main and child documents:

%%%%%%%%%%%%%%%%%%%%%%%%%%%%%%%%%%%%%%%%
\paragraph{Restrictions.}

Please note the following restrictions:
\begin{itemize}
\item
|\childdocmain| must be called with one argument \textit{main}
to ensure compatibility with earlier version of the package.
It must either be empty (|\childdocmain{}|)
or precisely match the filename of the main file in which it is specified.
See \secref{sec:detection} for further information.
\item
The filename \textit{main} must be specified without the |.tex| extension.
\item
The filename \textit{main} is case sensitive
(even in case-insensitive file systems)
due to internal string comparison.
\item
The argument \textit{main} should be fully expanded, it cannot be a macro.
\item
Subdirectories and special characters should be avoided in filenames.
\item
The command |\childdocmain{|\textit{main}|}| must be followed by a whitespace.
It should not be followed immediately by another command
or by a comment mark `|%|'.
This is because the \TeX{} parser reads the token immediately following
the argument of |\childdocmain| and puts it
at the beginning of every child section;
however, a white\-space is ignored.
\end{itemize}

%%%%%%%%%%%%%%%%%%%%%%%%%%%%%%%%%%%%%%%%
\paragraph{Content of Main File.}

It is advisable to place all content in the child files included by |\include|.
Any output contained in the main file will appear in all child documents
unless suppressed manually;
it cannot be suppressed automatically by the |\includeonly| directive
and thus should normally be avoided.
A method to include some content in the main file
by means of conditional processing is described in \secref{sec:conditional}.

%%%%%%%%%%%%%%%%%%%%%%%%%%%%%%%%%%%%%%%%
\paragraph{Page Numbering.}

When only a part of the document is compiled,
the appropriate numbering of pages
(as well as other status parameters)
is determined from the |.aux| files.
The latter contain information from previous passes.
However this information needs to propagate through
all intermediate child documents.
Therefore the page numbering in child documents may well
be inconsistent until the complete document is compiled at least once.

A useful (if unconventional) way to always ensure a consistent
page numbering is to restart the numbering in each child document
and denote the pages by `\textit{child}|.|\textit{page}'
where \textit{child} represents the chapter/section number of the child file.
This can be achieved by the command
|\numberwithin{page}{|\textit{child}|}|
of the \textsf{amsmath} package
where \textit{child} can be |chapter| or |section|
depending on the chosen structuring.
Alternatively, one can modify the macro |\thepage| appropriately
and reset the counter |page| at the start of each child file.

%%%%%%%%%%%%%%%%%%%%%%%%%%%%%%%%%%%%%%%%%%%%%%%%%%%%%%%%%%%%%%%%%%%%%%%%%%%%%%%%
\subsection{Conditional Processing}
\label{sec:conditional}

The package provides a mechanism to compile different versions
of a document. To customise the versions further some conditional processing
can come in handy to distinguish which version is being compiled.
The package provides two macros to describe the compilation context:

%%%%%%%%%%%%%%%%%%%%%%%%%%%%%%%%%%%%%%%%
\DescribeMacro{\ifchilddoc}
The conditional |\ifchilddoc| distinguishes between the compilation of
child documents and the main document:
%
\begin{center}
|\ifchilddoc |\textit{child-code}| |[|\||else |\textit{main-code}]| \||fi|
\end{center}

%%%%%%%%%%%%%%%%%%%%%%%%%%%%%%%%%%%%%%%%
\DescribeMacro{\childdocname}
\DescribeMacro{\childdocjob}
The macro |\childdocname| contains the filename (without extension)
of the main or child file being processed.
Note that |\childdocjob| will always contain the name of the main file.

%%%%%%%%%%%%%%%%%%%%%%%%%%%%%%%%%%%%%%%%
\paragraph{Title Page.}

Conditional processing can be used to include a title or banner page
in the main document when proper precautions are taken.
Importantly, the code in the main file should ensure that the page counter
(as well as other status parameters which are stored in the |.aux| files)
takes the same value after the conditional processing.
Otherwise the page numbers may take divergent values
depending on which part is compiled.

For example, a title page could be declared by:
%
\begin{center}
\begin{tabular}{l}
|\ifchilddoc\||else|\\
|\addtocounter{page}{-1}|\\
\textit{code for title page}\\
|\newpage|\\
|\||fi|
\end{tabular}
\end{center}
%
A banner page for the child documents can be generated by:
%
\begin{center}
\begin{tabular}{l}
|\ifchilddoc|\\
|\addtocounter{page}{-1}|\\
\textit{code for banner page}\\
|\newpage|\\
|\||fi|
\end{tabular}
\end{center}
%
Here one could write a message such as:
\begin{center}
|This is the part \childdocname{} of \childdocjob{}.|
\end{center}

%%%%%%%%%%%%%%%%%%%%%%%%%%%%%%%%%%%%%%%%%%%%%%%%%%%%%%%%%%%%%%%%%%%%%%%%%%%%%%%%
\subsection{Flags}
\label{sec:flags}

The package makes it easy to generate different versions
of the main or child documents.
To this end compilation flags can be defined
and assigned different default values.
They will be particularly useful in conjunction
with the forwarding mechanism described in \secref{sec:forward}.

For example, it may be useful to have a flag |\version|
which can be set to |draft| or |final|.
The document source will contain some conditional code
depending on the value of |\version|.
Suppose further, the flag should default to |final| for the main file
and to |draft| for child files
which is a natural assignment for editing the document.
This is achieved by placing the following code
in the preamble of the main document
(below the |\childdocmain| directive):
%
\begin{center}
\begin{tabular}{l}
|\ifchilddoc|\\
|\providecommand{\version}{draft}|\\
|\||else|\\
|\providecommand{\version}{final}|\\
|\||fi|
\end{tabular}
\end{center}
%
The definition by |\providecommand| makes sure
that previous definitions are not overwritten.
Further statements |\providecommand{\version}{...}|
can thus be added before the above code to override it.

For the main file, one might add a line
(between |\childdocmain| and the above block)
%
\begin{center}
|%\ifchilddoc\||else\providecommand{\version}{draft}\||fi|
\end{center}
%
which can be uncommented to produce a draft version.
Likewise one can add a line to the very top of a child file
(above the |\childdocof{|\textit{main}|}| directive)
%
\begin{center}
|%\providecommand{\version}{final}|
\end{center}
%
which can be uncommented to produce the final version of this child document.

%%%%%%%%%%%%%%%%%%%%%%%%%%%%%%%%%%%%%%%%%%%%%%%%%%%%%%%%%%%%%%%%%%%%%%%%%%%%%%%%
\subsection{Forwarding}
\label{sec:forward}

Different versions of the main or child documents
using compilation flags as described in \secref{sec:flags}
can be (permanently) stored in different files
for convenient compilation, viewing and distribution.
To this end, the package defines a command
to pass on compilation to a different file:

%%%%%%%%%%%%%%%%%%%%%%%%%%%%%%%%%%%%%%%%
\DescribeMacro{\childdocforward}
The command |\childdocforward| redirects processing to
another source file:
%
\begin{center}
\begin{tabular}{l}
|% \iffalse
%
% childdoc.dtx Copyright (C) 2017-2018 Niklas Beisert
%
% This work may be distributed and/or modified under the
% conditions of the LaTeX Project Public License, either version 1.3
% of this license or (at your option) any later version.
% The latest version of this license is in
%   http://www.latex-project.org/lppl.txt
% and version 1.3 or later is part of all distributions of LaTeX
% version 2005/12/01 or later.
%
% This work has the LPPL maintenance status `maintained'.
%
% The Current Maintainer of this work is Niklas Beisert.
%
% This work consists of the files childdoc.dtx and childdoc.ins
% and the derived files childdoc.def and cdocsamp.tex with
% cdocsch1.tex, cdocsch2.tex, cdocsdrf.tex, cdocsfn1.tex, cdocsfn2.tex.
%
%<package>\ifdefined\childdocmain\endinput\fi
%<package>\ProvidesFile{childdoc.def}[2018/12/30 v2.0 child document driver]
%<samplemain>\ProvidesFile{cdocsamp.tex}[2018/12/30 v2.0 sample for childdoc]
%<*driver>
%\ProvidesFile{childdoc.drv}[2018/12/30 v2.0 childdoc reference manual file]
\PassOptionsToClass{10pt,a4paper}{article}
\documentclass{ltxdoc}

\usepackage[margin=35mm]{geometry}
\usepackage{hyperref}
\usepackage{hyperxmp}
\usepackage[usenames]{color}

\hypersetup{colorlinks=true}
\hypersetup{pdfstartview=FitH}
\hypersetup{pdfpagemode=UseNone}
\hypersetup{pdfsource={}}
\hypersetup{pdflang={en-UK}}
\hypersetup{pdfcopyright={Copyright 2017-2018 Niklas Beisert.
  This work may be distributed and/or modified under the
  conditions of the LaTeX Project Public License, either version 1.3
  of this license or (at your option) any later version.}}
\hypersetup{pdflicenseurl={http://www.latex-project.org/lppl.txt}}
\hypersetup{pdfcontactaddress={ETH Zurich, ITP, HIT K,
  Wolfgang-Pauli-Strasse 27}}
\hypersetup{pdfcontactpostcode={8093}}
\hypersetup{pdfcontactcity={Zurich}}
\hypersetup{pdfcontactcountry={Switzerland}}
\hypersetup{pdfcontactemail={nbeisert@itp.phys.ethz.ch}}
\hypersetup{pdfcontacturl={http://people.phys.ethz.ch/\xmptilde nbeisert/}}

\newcommand{\secref}[1]{\hyperref[#1]{section \ref*{#1}}}

\parskip1ex
\parindent0pt
\let\olditemize\itemize
\def\itemize{\olditemize\parskip0pt}

\begin{document}

\title{The \textsf{childdoc} Package}
\hypersetup{pdftitle={The childdoc Package}}
\author{Niklas Beisert\\[2ex]
  Institut f\"ur Theoretische Physik\\
  Eidgen\"ossische Technische Hochschule Z\"urich\\
  Wolfgang-Pauli-Strasse 27, 8093 Z\"urich, Switzerland\\[1ex]
  \href{mailto:nbeisert@itp.phys.ethz.ch}
  {\texttt{nbeisert@itp.phys.ethz.ch}}}
\hypersetup{pdfauthor={Niklas Beisert}}
\hypersetup{pdfsubject={Manual for the LaTeX2e Package childdoc}}
\date{30 December 2018, \textsf{v2.0}}
\maketitle

\begin{abstract}\noindent
\textsf{childdoc} is a \LaTeXe{} package
that enables the direct compilation
of document sections included by |\include|
to individual files.
\end{abstract}

\begingroup
\parskip0ex
\tableofcontents
\endgroup

%%%%%%%%%%%%%%%%%%%%%%%%%%%%%%%%%%%%%%%%%%%%%%%%%%%%%%%%%%%%%%%%%%%%%%%%%%%%%%%%
%%%%%%%%%%%%%%%%%%%%%%%%%%%%%%%%%%%%%%%%%%%%%%%%%%%%%%%%%%%%%%%%%%%%%%%%%%%%%%%%
\section{Introduction}

\LaTeX{} provides a mechanism to structure a large document (such as a book)
into a main file and several child files (containing the chapters)
using the |\include| command.
This mechanism is beneficial for documents
which span hundreds of pages in order to
make the source file(s) more manageable.
Moreover, compilation can be restricted to
selected child files by means of the |\includeonly| command.
The latter feature can be used to reduce the compilation time while editing
(this was significantly more useful in the earlier days of \LaTeX{})
or to generate a smaller document which is easier to navigate.
Another application of |\includeonly| is to generate
documents consisting of selected parts of the complete document.

However, there are a few drawbacks of the plain |\include| mechanism:
\begin{itemize}
\item
The child files cannot be compiled on their own,
they can only be compiled via the main file.
A naive editing environment
(such as a text editor with an option
to have the current file processed by \LaTeX)
may require one to switch to the main file before compiling;
attempting to compile the child file produces errors.
\item
The main file must be modified (each time)
to adjust the |\includeonly| command
to the present needs. This easily leaves the main file in a messy state.
\item
The generated document will always carry the filename
of the main document. This is inconvenient if
several child files are to be compiled and
to be kept for distribution.
\end{itemize}

The present package provides a simple interface
to make child files individually compilable by \LaTeX{}.
Compiling a child file then has the same effect as compiling
the main file with an |\includeonly| command
to select the appropriate child.
Moreover the generated document will carry the name of the child
rather than the main file.
This resolves all three above issues.

This feature is meant to make the editing of books,
thesis documents and lecture notes somewhat more convenient.
However, the package can also be used efficiently for
composing a series of documents (such as exercise sheets)
which are typically distributed individually.
It then assists the author in generating the individual documents
(potentially in different versions)
as well as a document containing the collected series.
Another application is in developing style files
or other kinds of included material
where compilation of the style file could redirect
to a sample or test file.

%%%%%%%%%%%%%%%%%%%%%%%%%%%%%%%%%%%%%%%%%%%%%%%%%%%%%%%%%%%%%%%%%%%%%%%%%%%%%%%%
%%%%%%%%%%%%%%%%%%%%%%%%%%%%%%%%%%%%%%%%%%%%%%%%%%%%%%%%%%%%%%%%%%%%%%%%%%%%%%%%
\section{Usage}

First of all, the package \textsf{childdoc} is \emph{not} a standard
\LaTeXe{} |.sty| style file! Therefore it needs to be invoked in
a non-standard way.

%%%%%%%%%%%%%%%%%%%%%%%%%%%%%%%%%%%%%%%%%%%%%%%%%%%%%%%%%%%%%%%%%%%%%%%%%%%%%%%%
\subsection{Included Files}
\label{sec:include}

%%%%%%%%%%%%%%%%%%%%%%%%%%%%%%%%%%%%%%%%
\DescribeMacro{\childdocmain}
To use the package, add the commands
\begin{center}
\begin{tabular}{l}
|\input{childdoc.def}|\\
|\childdocmain{}|\\
\end{tabular}
\end{center}
at the very top of the main \LaTeX{} file,
in particular \emph{before} the |\documentclass| statement!
The argument of |\childdocmain| should be left empty
(but it must be present).

%%%%%%%%%%%%%%%%%%%%%%%%%%%%%%%%%%%%%%%%
\DescribeMacro{\childdocof}
Furthermore, add the commands
\begin{center}
\begin{tabular}{l}
|\input{childdoc.def}|\\
|\childdocof{|\textit{main}|}|\\
\end{tabular}
\end{center}
at the top of every child file \textit{child}
which is included by |\include{|\textit{child}|}|
from within the main file
(or at least for those files to be compiled individually).
The argument \textit{main} must be the filename of the main file.

There are a couple of
considerations in setting up the main and child documents:

%%%%%%%%%%%%%%%%%%%%%%%%%%%%%%%%%%%%%%%%
\paragraph{Restrictions.}

Please note the following restrictions:
\begin{itemize}
\item
|\childdocmain| must be called with one argument \textit{main}
to ensure compatibility with earlier version of the package.
It must either be empty (|\childdocmain{}|)
or precisely match the filename of the main file in which it is specified.
See \secref{sec:detection} for further information.
\item
The filename \textit{main} must be specified without the |.tex| extension.
\item
The filename \textit{main} is case sensitive
(even in case-insensitive file systems)
due to internal string comparison.
\item
The argument \textit{main} should be fully expanded, it cannot be a macro.
\item
Subdirectories and special characters should be avoided in filenames.
\item
The command |\childdocmain{|\textit{main}|}| must be followed by a whitespace.
It should not be followed immediately by another command
or by a comment mark `|%|'.
This is because the \TeX{} parser reads the token immediately following
the argument of |\childdocmain| and puts it
at the beginning of every child section;
however, a white\-space is ignored.
\end{itemize}

%%%%%%%%%%%%%%%%%%%%%%%%%%%%%%%%%%%%%%%%
\paragraph{Content of Main File.}

It is advisable to place all content in the child files included by |\include|.
Any output contained in the main file will appear in all child documents
unless suppressed manually;
it cannot be suppressed automatically by the |\includeonly| directive
and thus should normally be avoided.
A method to include some content in the main file
by means of conditional processing is described in \secref{sec:conditional}.

%%%%%%%%%%%%%%%%%%%%%%%%%%%%%%%%%%%%%%%%
\paragraph{Page Numbering.}

When only a part of the document is compiled,
the appropriate numbering of pages
(as well as other status parameters)
is determined from the |.aux| files.
The latter contain information from previous passes.
However this information needs to propagate through
all intermediate child documents.
Therefore the page numbering in child documents may well
be inconsistent until the complete document is compiled at least once.

A useful (if unconventional) way to always ensure a consistent
page numbering is to restart the numbering in each child document
and denote the pages by `\textit{child}|.|\textit{page}'
where \textit{child} represents the chapter/section number of the child file.
This can be achieved by the command
|\numberwithin{page}{|\textit{child}|}|
of the \textsf{amsmath} package
where \textit{child} can be |chapter| or |section|
depending on the chosen structuring.
Alternatively, one can modify the macro |\thepage| appropriately
and reset the counter |page| at the start of each child file.

%%%%%%%%%%%%%%%%%%%%%%%%%%%%%%%%%%%%%%%%%%%%%%%%%%%%%%%%%%%%%%%%%%%%%%%%%%%%%%%%
\subsection{Conditional Processing}
\label{sec:conditional}

The package provides a mechanism to compile different versions
of a document. To customise the versions further some conditional processing
can come in handy to distinguish which version is being compiled.
The package provides two macros to describe the compilation context:

%%%%%%%%%%%%%%%%%%%%%%%%%%%%%%%%%%%%%%%%
\DescribeMacro{\ifchilddoc}
The conditional |\ifchilddoc| distinguishes between the compilation of
child documents and the main document:
%
\begin{center}
|\ifchilddoc |\textit{child-code}| |[|\||else |\textit{main-code}]| \||fi|
\end{center}

%%%%%%%%%%%%%%%%%%%%%%%%%%%%%%%%%%%%%%%%
\DescribeMacro{\childdocname}
\DescribeMacro{\childdocjob}
The macro |\childdocname| contains the filename (without extension)
of the main or child file being processed.
Note that |\childdocjob| will always contain the name of the main file.

%%%%%%%%%%%%%%%%%%%%%%%%%%%%%%%%%%%%%%%%
\paragraph{Title Page.}

Conditional processing can be used to include a title or banner page
in the main document when proper precautions are taken.
Importantly, the code in the main file should ensure that the page counter
(as well as other status parameters which are stored in the |.aux| files)
takes the same value after the conditional processing.
Otherwise the page numbers may take divergent values
depending on which part is compiled.

For example, a title page could be declared by:
%
\begin{center}
\begin{tabular}{l}
|\ifchilddoc\||else|\\
|\addtocounter{page}{-1}|\\
\textit{code for title page}\\
|\newpage|\\
|\||fi|
\end{tabular}
\end{center}
%
A banner page for the child documents can be generated by:
%
\begin{center}
\begin{tabular}{l}
|\ifchilddoc|\\
|\addtocounter{page}{-1}|\\
\textit{code for banner page}\\
|\newpage|\\
|\||fi|
\end{tabular}
\end{center}
%
Here one could write a message such as:
\begin{center}
|This is the part \childdocname{} of \childdocjob{}.|
\end{center}

%%%%%%%%%%%%%%%%%%%%%%%%%%%%%%%%%%%%%%%%%%%%%%%%%%%%%%%%%%%%%%%%%%%%%%%%%%%%%%%%
\subsection{Flags}
\label{sec:flags}

The package makes it easy to generate different versions
of the main or child documents.
To this end compilation flags can be defined
and assigned different default values.
They will be particularly useful in conjunction
with the forwarding mechanism described in \secref{sec:forward}.

For example, it may be useful to have a flag |\version|
which can be set to |draft| or |final|.
The document source will contain some conditional code
depending on the value of |\version|.
Suppose further, the flag should default to |final| for the main file
and to |draft| for child files
which is a natural assignment for editing the document.
This is achieved by placing the following code
in the preamble of the main document
(below the |\childdocmain| directive):
%
\begin{center}
\begin{tabular}{l}
|\ifchilddoc|\\
|\providecommand{\version}{draft}|\\
|\||else|\\
|\providecommand{\version}{final}|\\
|\||fi|
\end{tabular}
\end{center}
%
The definition by |\providecommand| makes sure
that previous definitions are not overwritten.
Further statements |\providecommand{\version}{...}|
can thus be added before the above code to override it.

For the main file, one might add a line
(between |\childdocmain| and the above block)
%
\begin{center}
|%\ifchilddoc\||else\providecommand{\version}{draft}\||fi|
\end{center}
%
which can be uncommented to produce a draft version.
Likewise one can add a line to the very top of a child file
(above the |\childdocof{|\textit{main}|}| directive)
%
\begin{center}
|%\providecommand{\version}{final}|
\end{center}
%
which can be uncommented to produce the final version of this child document.

%%%%%%%%%%%%%%%%%%%%%%%%%%%%%%%%%%%%%%%%%%%%%%%%%%%%%%%%%%%%%%%%%%%%%%%%%%%%%%%%
\subsection{Forwarding}
\label{sec:forward}

Different versions of the main or child documents
using compilation flags as described in \secref{sec:flags}
can be (permanently) stored in different files
for convenient compilation, viewing and distribution.
To this end, the package defines a command
to pass on compilation to a different file:

%%%%%%%%%%%%%%%%%%%%%%%%%%%%%%%%%%%%%%%%
\DescribeMacro{\childdocforward}
The command |\childdocforward| redirects processing to
another source file:
%
\begin{center}
\begin{tabular}{l}
|\input{childdoc.def}|\\
|\childdocforward[|\textit{main}|]{|\textit{dest}|}|\\
\end{tabular}
\end{center}
%
The argument \textit{dest} is the destination file
(without extension).
It should be the main file or one of the child files.
Note that further \textsf{childdoc} directives
such as |\childdocof| and |\childdocforward|
in the indicated file will be processed in this form.
The optional argument \textit{main}
passes on directly to the main file \textit{main}
while pretending to compile the child \textit{dest}.
This form behaves as if \textit{dest}
issues |\childdocof{|\textit{main}|}| right away,
and no further \textsf{childdoc} directives will be processed.

%%%%%%%%%%%%%%%%%%%%%%%%%%%%%%%%%%%%%%%%
\DescribeMacro{\...prefix}
In the alternative form |\childdocforwardprefix|,
%
\begin{center}
\begin{tabular}{l}
|\input{childdoc.def}|\\
|\childdocforwardprefix[|\textit{main}|]{|\textit{prefix}|}{|\textit{dest}|}|
\end{tabular}
\end{center}
%
the destination file is determined by a pattern
depending on the current file:
To make this work, the current file must be called
`{\textit{prefix}\hspace{0.2em}\textit{suffix}}'
with \textit{prefix} matching precisely the argument.
Processing is then passed on to the file
`{\textit{dest}\hspace{0.2em}\textit{suffix}}'.
Surely, the same effect is achieved by
directly specifying the
argument `{\textit{dest}\hspace{0.2em}\textit{suffix}}'
in the first form.
However, that requires to set up a different file
for each child. With the alternative form of the command
all these files can have exactly the same content
which simplifies setting them up and maintaining them.

For example, the following file |draft.tex|
with a compilation flag |\version| as described in \secref{sec:flags}
compiles the main document as a draft:
%
\begin{center}
\begin{tabular}{l}
|\def\version{draft}|\\
|\input{childdoc.def}|\\
|\childdocforward{|\textit{main}|}|
\end{tabular}
\end{center}
%
Likewise, the following files |final|\textit{nn}|.tex|
compile the final version of the child document
|child|\textit{nn}|.tex|:
%
\begin{center}
\begin{tabular}{l}
|\def\version{final}|\\
|\input{childdoc.def}|\\
|\childdocforwardprefix{final}{child}|
\end{tabular}
\end{center}
%

Note that when several versions of a main file and/or of each child file
are to be generated, it may be convenient to set up a |Makefile| or
shell script to automatise the process.

%%%%%%%%%%%%%%%%%%%%%%%%%%%%%%%%%%%%%%%%%%%%%%%%%%%%%%%%%%%%%%%%%%%%%%%%%%%%%%%%
\subsection{Command Line Processing}
\label{sec:commandline}

The effect of redirection files can also be achieved by invoking
the \LaTeX{} compiler with a more elaborate command line.
Most conveniently this should be done as part
of a shell script or a |Makefile|.

When using \textsf{childdoc} in the main file, the following
command lines effectively perform a redirection
(note that depending on the shell being used,
backslashes may have to be doubled: `|\|' $\to$ `|\\|'):
%
\begin{center}
|... -jobname "|\textit{target}|" |\\|"|[\textit{flags}]%
|\input{childdoc.def}\childdocforward[|\textit{main}|]{|\textit{dest}|}"|
\end{center}
%
Here \textit{target} is the name of the output file,
\textit{main} is the name of the main file
and \textit{dest} is the name of the main or child file to be processed
(all filenames without extensions).
The optional argument \textit{main} can be omitted
if \textit{main} matches \textit{dest}.
Optionally, compilation \textit{flags} can be defined via |\def| commands.
This command line makes the \TeX{} engine believe
it is compiling the file \textit{target}
whose content is specified as the latter parameter.
The provided code then forwards the processing to
\textit{main} or \textit{dest} as described in \secref{sec:forward}.

%%%%%%%%%%%%%%%%%%%%%%%%%%%%%%%%%%%%%%%%%%%%%%%%%%%%%%%%%%%%%%%%%%%%%%%%%%%%%%%%
\subsection{Include by Input}
\label{sec:input}

Including child documents by |\include| has some restrictions by design.
Most notably, the content of a child document always occupies
its own set of pages; pages cannot be shared between child documents.
Usually, this behaviour makes perfect sense
because each child document contain an essential part of the document.
However, in some situations it may be desirable to compose
a document from a collection of parts
without having mandatory page breaks between then.
For this case, the package
provides a mechanism to include parts
by |\input| which can also be processed individually.
However, by construction this mechanism
requires manual handling of the content to be output.

%%%%%%%%%%%%%%%%%%%%%%%%%%%%%%%%%%%%%%%%
\DescribeMacro{\ifchilddocmanual}
The main file should be prepared as usual, see \secref{sec:include}.
However, the document body must make a distinction
between processing of an individual part and of the main document, e.g.:
%
\begin{center}
\begin{tabular}{l}
|\ifchilddocmanual|\\
|\input{\childdocname}|\\
|\||else|\\
\textit{document body with }|\input{|\textit{part}|}|\\
|\||fi|
\end{tabular}
\end{center}
%
The conditional |\ifchilddocmanual| is true whenever
a part to be included by |\input| is being compiled,
and the name of the part is stored in |\childdocname|.

%%%%%%%%%%%%%%%%%%%%%%%%%%%%%%%%%%%%%%%%
\DescribeMacro{\childdocby}
Each part to be included by |\input| should start with:
%
\begin{center}
\begin{tabular}{l}
|\input{childdoc.def}|\\
|\childdocby{|\textit{main}|}|\\
\end{tabular}
\end{center}
%
The directive |\childdocby| is similar to |\childdocof|
described in \secref{sec:include},
but the subsequent selection of content must be done manually.
To that end, both |\ifchilddoc| and |\ifchilddocmanual|
will be true upon processing of a part,
and the name of the part is stored in |\childdocname|.
Note that |\jobname| will be set to the filename of the current part
so that each part receives an individual |.aux| file
that does not interfere with the |.aux| file(s) of the main document.
This behaviour can be altered by the alternative form
|\childdocby[*]{|\textit{main}|}| (with a non-empty optional argument)
which uses the |.aux| file of the main document
by setting |\jobname| to \textit{main}.

%%%%%%%%%%%%%%%%%%%%%%%%%%%%%%%%%%%%%%%%%%%%%%%%%%%%%%%%%%%%%%%%%%%%%%%%%%%%%%%%
\subsection{Driver Development}
\label{sec:driver}

The \textsf{childdoc} mechanism can also be use for the development
of definition files such as \LaTeX{} styles or classes.
This case differs from the above setup with multiple parts
included by |\include| in that no |\includeonly| should be invoked.
This can be achieved by starting the include file
(before |\ProvidesPackage|) with:
%
\begin{center}
\begin{tabular}{l}
|\input{childdoc.def}|\\
|\childdocforward{|\textit{main}|}|\\
\end{tabular}
\end{center}
%
or alternatively with:
%
\begin{center}
\begin{tabular}{l}
|\input{childdoc.def}|\\
|\childdocby{|\textit{main}|}|\\
\end{tabular}
\end{center}
%
Both forms have slightly different effects as described above.
The main file is prepared as usual, see \secref{sec:include}.

%%%%%%%%%%%%%%%%%%%%%%%%%%%%%%%%%%%%%%%%%%%%%%%%%%%%%%%%%%%%%%%%%%%%%%%%%%%%%%%%
\subsection{Legacy Detection}
\label{sec:detection}

The directive |\childdocmain| in the main file can detect
whether the complete document or merely a child is to be compiled
even without using the directive |\childdocof|.
This method is deprecated because it is less robust
and there is no compelling reason to use it;
it is merely provided for backward compatibility
and it may be removed in future versions.

If the detection mechanism is to be used,
it is mandatory to correctly specify
the filename of the main file as the argument of |\childdocmain|:
%
\begin{center}
\begin{tabular}{l}
|\input{childdoc.def}|\\
|\childdocmain{|\textit{main}|}|\\
\end{tabular}
\end{center}
%
If |\jobname| does not match the argument \textit{main} of |\childdocmain|,
it is assumed that |\jobname| points to the child file to be compiled.
When using |\childdocmain| with the main file specified as argument,
it suffices to start a child file
with just |\input{|\textit{main}|}|
without loading of the package and using |\childdocof|.
If instead all processing is done
with the appropriate \textsf{childdoc} directives,
the argument of \textit{main} of |\childdocmain| can be empty.

An alternative version of the command line processing described
in \secref{sec:commandline} using the detection mechanism reads:
%
\begin{center}
|... -jobname "|\textit{target}|" "|[\textit{flags}]%
[|\def\jobname{|\textit{dest}|}|]|\input{|\textit{main}|}"|
\end{center}

%%%%%%%%%%%%%%%%%%%%%%%%%%%%%%%%%%%%%%%%%%%%%%%%%%%%%%%%%%%%%%%%%%%%%%%%%%%%%%%%
\subsection{Manual Code}
\label{sec:manual}

In case one cannot be certain whether the definitions file |childdoc.def|
is installed on the target \TeX{} distribution
and one prefers not to ship it,
it is conceivable to paste a few relevant commands into the sources.

To that end, drop all statements |\input{childdoc.def}|
and perform the replacements as outlined below.
Instead of |\childdocmain{|\textit{main}|}| add the following code
to the top of the main file:
%
\begin{center}
\begin{tabular}{l}
|\||ifdefined\childdocname\endinput\||fi\newif\ifchilddoc|\\
|\edef\childdocname{\scantokens\expandafter{\jobname\noexpand}}|\\
|\def\childdocmain{|\textit{main}|}\||ifx\childdocmain\childdocname\||else|\\
|\childdoctrue\includeonly{\childdocname}\let\jobname\childdocmain\||fi|\\
\end{tabular}
\end{center}
%
Instead of |\childdocof{|\textit{main}|}| just include the main file
at the top of each child file:
%
\begin{center}
|\input{|\textit{main}|}|
\end{center}
%
A simple redirection |\childdocforward{|\textit{dest}|}| is achieved by:
%
\begin{center}
|\def\jobname{|\textit{dest}|}\input{\jobname}|
\end{center}
%
The redirection with prefix
|\childdocforwardprefix[|\textit{prefix}|]{|\textit{dest}|}|
is accomplished by:
%
\begin{center}
\begin{tabular}{l}
|{\edef\jobname{\scantokens\expandafter{\jobname\noexpand}}|\\
|\def\redirectjob |\textit{prefix}|#1~~~{\gdef\jobname{|\textit{dest}|#1}}|\\
|\expandafter\redirectjob\jobname~~~}\input{\jobname}|
\end{tabular}
\end{center}

In an alternative approach,
child documents can be compiled by a specific command line
without additional code or specific definitions:
%
\begin{center}
|... -jobname "|\textit{target}|" "|[\textit{flags}]%
|\includeonly{|\textit{dest}|}\input{|\textit{main}|}"|
\end{center}
%

%%%%%%%%%%%%%%%%%%%%%%%%%%%%%%%%%%%%%%%%%%%%%%%%%%%%%%%%%%%%%%%%%%%%%%%%%%%%%%%%
%%%%%%%%%%%%%%%%%%%%%%%%%%%%%%%%%%%%%%%%%%%%%%%%%%%%%%%%%%%%%%%%%%%%%%%%%%%%%%%%
\section{Information}

%%%%%%%%%%%%%%%%%%%%%%%%%%%%%%%%%%%%%%%%%%%%%%%%%%%%%%%%%%%%%%%%%%%%%%%%%%%%%%%%
\subsection{Copyright}

Copyright \copyright{} 2017--2018 Niklas Beisert

This work may be distributed and/or modified under the
conditions of the \LaTeX{} Project Public License, either version 1.3
of this license or (at your option) any later version.
The latest version of this license is in
  \url{http://www.latex-project.org/lppl.txt}
and version 1.3 or later is part of all distributions of \LaTeX{}
version 2005/12/01 or later.

This work has the LPPL maintenance status `maintained'.

The Current Maintainer of this work is Niklas Beisert.

This work consists of the files |README.txt|, |childdoc.ins| and |childdoc.dtx|
as well as the derived files |childdoc.def|, |cdocsamp.tex|
with |cdocsch1.tex|, |cdocsch2.tex|, |cdocspt3.tex|, |cdocspt4.tex|,
|cdocsdrf.tex|, |cdocsfn1.tex|, |cdocsfn2.tex|
as well as |childdoc.pdf|.

%%%%%%%%%%%%%%%%%%%%%%%%%%%%%%%%%%%%%%%%%%%%%%%%%%%%%%%%%%%%%%%%%%%%%%%%%%%%%%%%
\subsection{Files and Installation}

The package consists of the files:
%
\begin{center}
\begin{tabular}{ll}
    |README.txt|   & readme file \\
    |childdoc.ins| & installation file \\
    |childdoc.dtx| & source file \\
    |childdoc.def| & definition file \\
    |cdocsamp.tex| & sample main file \\
    |cdocsch1.tex| & sample include file \\
    |cdocsch2.tex| & sample include file \\
    |cdocspt3.tex| & sample part file \\
    |cdocspt4.tex| & sample part file \\
    |cdocsdrf.tex| & sample redirection file \\
    |cdocsfn1.tex| & sample redirection file \\
    |cdocsfn2.tex| & sample redirection file \\
    |childdoc.pdf| & manual
\end{tabular}
\end{center}
%
The distribution consists of the files
|README.txt|, |childdoc.ins| and |childdoc.dtx|.
%
\begin{itemize}
\item
Run (pdf)\LaTeX{} on |childdoc.dtx|
to compile the manual |childdoc.pdf| (this file).
\item
Run \LaTeX{} on |childdoc.ins| to create the definitions file |childdoc.def|
and the sample |cdocsamp.tex| with include files
|cdocsch1.tex|, |cdocsch2.tex|, |cdocspt3.tex|, |cdocspt4.tex|,
|cdocsdrf.tex|, |cdocsfn1.tex|, |cdocsfn2.tex|.
Then copy the file |childdoc.def| to an appropriate directory of your \LaTeX{}
distribution, e.g.\ \textit{texmf-root}|/tex/latex/childdoc|.
\end{itemize}

%%%%%%%%%%%%%%%%%%%%%%%%%%%%%%%%%%%%%%%%%%%%%%%%%%%%%%%%%%%%%%%%%%%%%%%%%%%%%%%%
\subsection{Related CTAN Packages}

There are several other packages which offer a similar functionality:
%
\begin{itemize}
\item
The packages
\href{http://ctan.org/pkg/docmute}{\textsf{docmute}},
\href{http://ctan.org/pkg/includex}{\textsf{includex}} and
\href{http://ctan.org/pkg/standalone}{\textsf{standalone}}
provide commands to include only the document body of
a child file thus allowing both files to be compiled individually.
\item
The packages \href{http://ctan.org/pkg/subdocs}{\textsf{subdocs}}
and \href{http://ctan.org/pkg/subfiles}{\textsf{subfiles}}
provide structures in which the main and child documents can be
encapsulated and allowing them to be compiled individually.
The inclusion mechanism is different from the conventional |\include|.
\item
The package \href{http://ctan.org/pkg/combine}{\textsf{combine}}
is an elaborate solution to combine several documents into one.
\end{itemize}
%
See also the CTAN topic \href{http://ctan.org/topic/subdocs}{\textsf{subdocs}}
for further related packages.
The present package differs from the above solutions in that
a document structure constructed with the conventional |\include| mechanism
just needs two extra commands at the top of every file
such that all constituent files can be compiled individually.

%%%%%%%%%%%%%%%%%%%%%%%%%%%%%%%%%%%%%%%%%%%%%%%%%%%%%%%%%%%%%%%%%%%%%%%%%%%%%%%%
%\subsection{Feature Suggestions}
%
%The following is a list of features which may be useful for future
%versions of this package:
%%
%\begin{itemize}
%\item
%\ldots
%\end{itemize}

%%%%%%%%%%%%%%%%%%%%%%%%%%%%%%%%%%%%%%%%%%%%%%%%%%%%%%%%%%%%%%%%%%%%%%%%%%%%%%%%
\subsection{Revision History}

%%%%%%%%%%%%%%%%%%%%%%%%%%%%%%%%%%%%%%%%
\paragraph{v2.0:} 2018/12/30

\begin{itemize}
\item
immediate forward processing
\item
added |\childdocby| mechanism
\item
manual restructured
\end{itemize}

%%%%%%%%%%%%%%%%%%%%%%%%%%%%%%%%%%%%%%%%
\paragraph{v1.6:} 2018/01/17

\begin{itemize}
\item
application for development of include files
\item
corrections to manual
\end{itemize}

%%%%%%%%%%%%%%%%%%%%%%%%%%%%%%%%%%%%%%%%
\paragraph{v1.5:} 2017/05/21

\begin{itemize}
\item
more complete structuring introduced
\item
|\childdocof| introduced
\item
|\childdoc| renamed to |\childdocmain|
\item
|\childredirect| renamed to |\childdocforward| and |\childdocforwardprefix|
and functionality expanded
\end{itemize}

%%%%%%%%%%%%%%%%%%%%%%%%%%%%%%%%%%%%%%%%
\paragraph{v1.0:} 2017/04/27

\begin{itemize}
\item
manual and install package
\item
first version published on CTAN
\end{itemize}

%%%%%%%%%%%%%%%%%%%%%%%%%%%%%%%%%%%%%%%%
\paragraph{v0.6:} 2017/04/26

\begin{itemize}
\item
redirection mechanism added
\end{itemize}

%%%%%%%%%%%%%%%%%%%%%%%%%%%%%%%%%%%%%%%%
\paragraph{v0.5:} 2017/04/26

\begin{itemize}
\item
functionality in definition file
\end{itemize}


%%%%%%%%%%%%%%%%%%%%%%%%%%%%%%%%%%%%%%%%%%%%%%%%%%%%%%%%%%%%%%%%%%%%%%%%%%%%%%%%
%%%%%%%%%%%%%%%%%%%%%%%%%%%%%%%%%%%%%%%%%%%%%%%%%%%%%%%%%%%%%%%%%%%%%%%%%%%%%%%%
%%%%%%%%%%%%%%%%%%%%%%%%%%%%%%%%%%%%%%%%%%%%%%%%%%%%%%%%%%%%%%%%%%%%%%%%%%%%%%%%
\appendix

\settowidth\MacroIndent{\rmfamily\scriptsize 000\ }

 \DocInput{childdoc.dtx}

\end{document}
%</driver>
% \fi
%
% %%%%%%%%%%%%%%%%%%%%%%%%%%%%%%%%%%%%%%%%%%%%%%%%%%%%%%%%%%%%%%%%%%%%%%%%%%%%%%
% %%%%%%%%%%%%%%%%%%%%%%%%%%%%%%%%%%%%%%%%%%%%%%%%%%%%%%%%%%%%%%%%%%%%%%%%%%%%%%
% \section{Sample}
%\iffalse
%<*samplemain>
%\fi
%
% The following presents a sample document
% with two chapters, two parts, a title page,
% a compile flag as well as three forwarding files to set the flag.
% It consists of eight |.tex| files:
% \begin{center}
% \begin{tabular}{ll}
% |cdocsamp.tex|&main file\\
% |cdocsch1.tex|&include file for chapter 1\\
% |cdocsch2.tex|&include file for chapter 2\\
% |cdocspt3.tex|&include file for part 3\\
% |cdocspt4.tex|&include file for part 4\\
% |cdocsdrf.tex|&forwarding file for main file in draft mode\\
% |cdocsfi1.tex|&forwarding file for final version of chapter 1\\
% |cdocsfi2.tex|&forwarding file for final version of chapter 2\\
% \end{tabular}
% \end{center}
% Each of the eight files can be compiled directly by the \LaTeX{} compiler.
%
% %%%%%%%%%%%%%%%%%%%%%%%%%%%%%%%%%%%%%%
% \paragraph{Main File.}
%
% The main file is called |cdocsamp.tex|.
%
% Load the \textsf{childdoc} definitions and
% declare the filename for the main document:
%    \begin{macrocode}
\input{childdoc.def}
\childdocmain{}
%    \end{macrocode}

% Optional override for |\version| flag:
%    \begin{macrocode}
%%\ifchilddoc\else\providecommand{\version}{draft}\fi
%    \end{macrocode}

% Define the default values for the |\version| flag
% (|final| for the main file and |draft| for childs):
%    \begin{macrocode}
\ifchilddoc
\providecommand{\version}{draft}
\else
\providecommand{\version}{final}
\fi
%    \end{macrocode}

% Load the standard document class:
%    \begin{macrocode}
\documentclass[12pt]{article}
%    \end{macrocode}

% Start the document body:
%    \begin{macrocode}
\begin{document}
%    \end{macrocode}

% Declare a title page.
% Print title, part of document being processed and version flag:
%    \begin{macrocode}
\addtocounter{page}{-1}
\begin{center}
{\LARGE\bfseries{}childdoc example\par}
\vspace{1cm}
\ifchilddoc
\ifchilddocmanual part\else chapter\fi:
`\childdocname' of `\childdocjob'\par
\else
main document: `\childdocjob'\par
\fi
version: \version\par
\end{center}
\newpage
%    \end{macrocode}

% Manually include selected file,
% otherwise process as usual:
%    \begin{macrocode}
\ifchilddocmanual
\section*{part `\childdocname'}
\input{\childdocname}
\else
%    \end{macrocode}

% Include the two chapters:
%    \begin{macrocode}
\include{cdocsch1}
\include{cdocsch2}
%    \end{macrocode}

% Include the two parts unless only chapters should be displayed:
%    \begin{macrocode}
\ifchilddoc\else
\section{part three}
\input{cdocspt3}
\section{part four}
\input{cdocspt4}
\fi
%    \end{macrocode}

% Process as usual until here:
%    \begin{macrocode}
\fi
%    \end{macrocode}

% End of document body:
%    \begin{macrocode}
\end{document}
%    \end{macrocode}
%\iffalse
%</samplemain>
%\fi
%
% %%%%%%%%%%%%%%%%%%%%%%%%%%%%%%%%%%%%%%
% \paragraph{Chapter Include Files.}
%
% The include files are called |cdocsch1.tex| and |cdocsch2.tex|.
%
%\iffalse
%<*samplechap1|samplechap2>
%\fi

% Optional override for |\version| flag:
%    \begin{macrocode}
%%\providecommand{\version}{final}
%    \end{macrocode}

% Include the main document:
%    \begin{macrocode}
\input{childdoc.def}
\childdocof{cdocsamp}
%    \end{macrocode}

%\iffalse
%</samplechap1|samplechap2>
%\fi
%
%\iffalse
%<*samplechap1>
%\fi
% Some text for chapter 1:
%    \begin{macrocode}
\section{one}
some text in chapter one
%    \end{macrocode}

%\iffalse
%</samplechap1>
%\fi
% Some text for chapter 2:
%\iffalse
%<*samplechap2>
%\fi
%    \begin{macrocode}
\section{two}
more text in chapter two
%    \end{macrocode}

%\iffalse
%</samplechap2>
%\fi
%
% %%%%%%%%%%%%%%%%%%%%%%%%%%%%%%%%%%%%%%
% \paragraph{Part Include Files.}
%
% The include files are called |cdocspt3.tex| and |cdocspt4.tex|.
%
%\iffalse
%<*samplepart3|samplepart4>
%\fi

% Optional override for |\version| flag:
%    \begin{macrocode}
%%\providecommand{\version}{final}
%    \end{macrocode}

% Include the main document:
%    \begin{macrocode}
\input{childdoc.def}
\childdocby{cdocsamp}
%    \end{macrocode}

%\iffalse
%</samplepart3|samplepart4>
%\fi
%
%\iffalse
%<*samplepart3>
%\fi
% Some text for part 3:
%    \begin{macrocode}
some text in part three
%    \end{macrocode}

%\iffalse
%</samplepart3>
%\fi
% Some text for part 4:
%\iffalse
%<*samplepart4>
%\fi
%    \begin{macrocode}
more text in part four
%    \end{macrocode}

%\iffalse
%</samplepart4>
%\fi
%
% %%%%%%%%%%%%%%%%%%%%%%%%%%%%%%%%%%%%%%
% \paragraph{Forwarding for a Complete Draft.}
%
% The following forwarding file |cdocsdrf.tex|
% compiles the main document in draft mode:
%\iffalse
%<*sampledraft>
%\fi
%    \begin{macrocode}
\def\version{draft}
\input{childdoc.def}
\childdocforward{cdocsamp}
%    \end{macrocode}

%\iffalse
%</sampledraft>
%\fi
%
% %%%%%%%%%%%%%%%%%%%%%%%%%%%%%%%%%%%%%%
% \paragraph{Forwarding for Final Version of the Chapters.}
%
% The following forwarding files |cdocsfn1.tex| and |cdocsfn2.tex|
% (with identical content)
% compile the final versions of the child documents
% |cdocsch1.tex| and |cdocsch2.tex|, respectively:
%\iffalse
%<*samplefinal>
%\fi
%    \begin{macrocode}
\def\version{final}
\input{childdoc.def}
\childdocforwardprefix[cdocsamp]{cdocsfn}{cdocsch}
%    \end{macrocode}

%\iffalse
%</samplefinal>
%\fi
%
% %%%%%%%%%%%%%%%%%%%%%%%%%%%%%%%%%%%%%%
% \paragraph{Command Line Processing.}
%
% The following three command lines generate the output files
% |cdocscld|, |cdocscl1| and |cdocscl2|
% which should be identical to
% |cdocsdrf|, |cdocsch1| and |cdocsfn2|, respectively:
% \begin{center}
% \begin{tabular}{l}
% |latex -jobname cdocscld \|\\
% |  "\def\version{draft}\input{childdoc.def}\childdocforward{cdocsamp}"|\\
% |latex -jobname cdocscl1 \|\\
% |  "\input{childdoc.def}\childdocforward[cdocsamp]{cdocsch1}"|\\
% |latex -jobname cdocscl2 \|\\
% |  "\def\version{final}\input{childdoc.def}\childdocforward{cdocsch2}"|
% \end{tabular}
% \end{center}
% Note that the trailing backslash on each first line
% merely continues the input to the second line
% (for convenient cut ant paste).
% Furthermore, the command |latex| can be replaced by any
% of its alternative versions such as |pdflatex|.
%
% %%%%%%%%%%%%%%%%%%%%%%%%%%%%%%%%%%%%%%%%%%%%%%%%%%%%%%%%%%%%%%%%%%%%%%%%%%%%%%
% %%%%%%%%%%%%%%%%%%%%%%%%%%%%%%%%%%%%%%%%%%%%%%%%%%%%%%%%%%%%%%%%%%%%%%%%%%%%%%
% \section{Implementation}
%\iffalse
%<*package>
%\fi
%
% This section describes the definitions file |childdoc.def|.

% The definitions cannot be loaded using |\usepackage| or |\RequirePackage|
% which has a mechanism to prevent loading a style file more than once.
% When loading the definitions by means of |\input|
% multiple instances have to be prevented manually:
%\iffalse
%This code needs to be before the `\ProvidesFile' directive
%which is defined at the beginning of this file.
%Therefore it is also placed there and commented out here.
%</package>
%<*discard>
%\fi
%    \begin{macrocode}
\ifdefined\childdocmain\endinput\fi
%    \end{macrocode}
%\iffalse
%</discard>
%<*package>
%\fi
%
% \macro{\ifchilddoc}
% \macro{\ifchilddocmanual}
% The conditional |\ifchilddoc| tells whether a
% child (true) or main (false) document is being compiled.
% The conditional |\ifchilddocmanual| tells whether
% the |\includeonly| mechanism is used (false) or
% the selection of child files must be performed manually (true).
% The definitions initialise to false:
%    \begin{macrocode}
\newif\ifchilddoc
\newif\ifchilddocmanual
%    \end{macrocode}

% \macro{\childdocname}
% \macro{\childdocjob}
% The macro |\childdocname| stores the name of the main document
% to be compiled. The macro |\childdocjob| stores the name of
% the document on which the \LaTeX{} compiler was originally invoked.
% The content of |\jobname| cannot be compared
% to filenames specified in the source due to different catcodes.
% The following code rescans |\jobname|, stores the result
% in |\childdocname| and saves a copy in |\childdocjob|:
%    \begin{macrocode}
\edef\childdocname{\scantokens\expandafter{\jobname\noexpand}}
\let\childdocjob\childdocname
%    \end{macrocode}

% \macro{\childdocdisable}
% The macro |\childdocdisable| prevents the main file
% from being processed more than once.
% At this stage, the main document command |\childdocmain|
% is assumed to be called once again where it should do nothing.
% Any subsequent call to it should prevent
% a secondary processing of the main document
% It overwrites the forwarding commands
% |\childdocof| and |\childdocforward|
% with empty macros to prevent further inclusions of the main document:
%    \begin{macrocode}
\newcommand{\childdocdisable}
{
  \renewcommand{\childdocmain}[1]{\renewcommand{\childdocmain}[1]{\endinput}}
  \renewcommand{\childdocof}[1]{}
  \renewcommand{\childdocby}[2][]{}
  \renewcommand{\childdocforward}[2][]{}
  \renewcommand{\childdocdisable}{}
}
%    \end{macrocode}

% \macro{\childdocmain}
% The macro |\childdocmain| is to be called at the top of the main file
% with nothing or the main filename (without extension) as argument.
% First, it breaks loops.
% If the argument is not empty and does not match |\childdocname|
% (which is set by the first inclusion of |childdoc.def|),
% |\ifchilddoc| is set to true, |\includeonly| is applied to the child file
% and |\jobname| is set to the main file
% (for proper handling of |.aux| files):
%    \begin{macrocode}
\newcommand{\childdocmain}[1]
{
  \childdocdisable\childdocmain{}
  \if?#1?\else
    \begingroup
      \def\childdoctmp{#1}
      \ifx\childdoctmp\childdocname
        \def\childdoctmp{}
      \else
        \def\childdoctmp
        {
          \childdoctrue
          \includeonly{\childdocname}
          \def\childdocjob{#1}
          \def\jobname{#1}
        }
      \fi
      \expandafter
    \endgroup
    \childdoctmp
  \fi
}
%    \end{macrocode}

% \macro{\childdocof}
% The command |\childdocof| redirects
% compilation to the main file |#1|.
%    \begin{macrocode}
\newcommand{\childdocof}[1]
{
  \childdocdisable
  \childdoctrue
  \includeonly{\childdocname}
  \def\jobname{#1}
  \def\childdocjob{#1}
  \input{#1}
}
%    \end{macrocode}

% \macro{\childdocby}
% The command |\childdocby| ....
%    \begin{macrocode}
\newcommand{\childdocby}[2][]
{
  \childdocdisable
  \childdoctrue
  \childdocmanualtrue
  \if?#1?\else
    \def\jobname{#2}
  \fi
  \def\childdocjob{#2}
  \input{#2}
  \endinput
}
%    \end{macrocode}

% \macro{\childdocforward}
% The command |\childdocforward| redirects
% compilation to the main file or
% (if the optional argument is given) a child file.
% Parameters are set as if the main file
% or a child file starting with |\childdocof| was compiled.
% Then compilation is handed over to the main file:
%    \begin{macrocode}
\newcommand{\childdocforward}[2][]
{
  \begingroup
    \if?#1?
      \def\childdoctmp
      {
        \def\childdocname{#2}
        \def\childdocjob{#2}
        \def\jobname{#2}
        \input{#2}
        \endinput
      }
    \else
      \def\childdoctmp
      {
        \childdocdisable
        \def\childdocname{#2}
        \childdoctrue
        \includeonly{#2}
        \def\childdocjob{#1}
        \def\jobname{#1}
        \input{#1}
        \endinput
      }
    \fi
    \expandafter
  \endgroup
  \childdoctmp
}
%    \end{macrocode}

% \macro{\childdocforwardprefix}
% The command |\childdocforwardprefix| redirects
% compilation to the main or a child file by means of a pattern.
% The prefix |#1| in the current filename is replaced by |#2|
% and the suffix of the current filename is kept
% (it is assumed that the filename does not contain the substring `|~~~|'
% which is used as a delimiter).
% Compilation is handed over to the new file by |\childdocforward|:
%    \begin{macrocode}
\newcommand{\childdocforwardprefix}[3][]
{
  \begingroup
    \def\childdocextract #2##1~~~{\def\childdoctmp{\childdocforward[#1]{#3##1}}}
    \expandafter\childdocextract\childdocname~~~
    \expandafter
  \endgroup
  \childdoctmp
}
%    \end{macrocode}

% \macro{\childdoc}
% The deprecated macro |\childdoc| is a legacy version of |\childdocmain|:
%    \begin{macrocode}
\newcommand{\childdoc}{\childdocmain}
%    \end{macrocode}

% \macro{\childdocredirect}
% The deprecated macro |\childdocredirect| is a legacy version
% of |\childdocforward| and |\childdocforwardprefix|:
%    \begin{macrocode}
\newcommand{\childdocredirect}[2][]
{
  \begingroup
    \if?#1?
      \def\childdoctmp{\childdocforward{#2}}
    \else
      \def\childdoctmp{\childdocforwardprefix{#1}{#2}}
    \fi
    \expandafter
  \endgroup
  \childdoctmp
}
%    \end{macrocode}

%\iffalse
%</package>
%\fi
%
\endinput
|\\
|\childdocforward[|\textit{main}|]{|\textit{dest}|}|\\
\end{tabular}
\end{center}
%
The argument \textit{dest} is the destination file
(without extension).
It should be the main file or one of the child files.
Note that further \textsf{childdoc} directives
such as |\childdocof| and |\childdocforward|
in the indicated file will be processed in this form.
The optional argument \textit{main}
passes on directly to the main file \textit{main}
while pretending to compile the child \textit{dest}.
This form behaves as if \textit{dest}
issues |\childdocof{|\textit{main}|}| right away,
and no further \textsf{childdoc} directives will be processed.

%%%%%%%%%%%%%%%%%%%%%%%%%%%%%%%%%%%%%%%%
\DescribeMacro{\...prefix}
In the alternative form |\childdocforwardprefix|,
%
\begin{center}
\begin{tabular}{l}
|% \iffalse
%
% childdoc.dtx Copyright (C) 2017-2018 Niklas Beisert
%
% This work may be distributed and/or modified under the
% conditions of the LaTeX Project Public License, either version 1.3
% of this license or (at your option) any later version.
% The latest version of this license is in
%   http://www.latex-project.org/lppl.txt
% and version 1.3 or later is part of all distributions of LaTeX
% version 2005/12/01 or later.
%
% This work has the LPPL maintenance status `maintained'.
%
% The Current Maintainer of this work is Niklas Beisert.
%
% This work consists of the files childdoc.dtx and childdoc.ins
% and the derived files childdoc.def and cdocsamp.tex with
% cdocsch1.tex, cdocsch2.tex, cdocsdrf.tex, cdocsfn1.tex, cdocsfn2.tex.
%
%<package>\ifdefined\childdocmain\endinput\fi
%<package>\ProvidesFile{childdoc.def}[2018/12/30 v2.0 child document driver]
%<samplemain>\ProvidesFile{cdocsamp.tex}[2018/12/30 v2.0 sample for childdoc]
%<*driver>
%\ProvidesFile{childdoc.drv}[2018/12/30 v2.0 childdoc reference manual file]
\PassOptionsToClass{10pt,a4paper}{article}
\documentclass{ltxdoc}

\usepackage[margin=35mm]{geometry}
\usepackage{hyperref}
\usepackage{hyperxmp}
\usepackage[usenames]{color}

\hypersetup{colorlinks=true}
\hypersetup{pdfstartview=FitH}
\hypersetup{pdfpagemode=UseNone}
\hypersetup{pdfsource={}}
\hypersetup{pdflang={en-UK}}
\hypersetup{pdfcopyright={Copyright 2017-2018 Niklas Beisert.
  This work may be distributed and/or modified under the
  conditions of the LaTeX Project Public License, either version 1.3
  of this license or (at your option) any later version.}}
\hypersetup{pdflicenseurl={http://www.latex-project.org/lppl.txt}}
\hypersetup{pdfcontactaddress={ETH Zurich, ITP, HIT K,
  Wolfgang-Pauli-Strasse 27}}
\hypersetup{pdfcontactpostcode={8093}}
\hypersetup{pdfcontactcity={Zurich}}
\hypersetup{pdfcontactcountry={Switzerland}}
\hypersetup{pdfcontactemail={nbeisert@itp.phys.ethz.ch}}
\hypersetup{pdfcontacturl={http://people.phys.ethz.ch/\xmptilde nbeisert/}}

\newcommand{\secref}[1]{\hyperref[#1]{section \ref*{#1}}}

\parskip1ex
\parindent0pt
\let\olditemize\itemize
\def\itemize{\olditemize\parskip0pt}

\begin{document}

\title{The \textsf{childdoc} Package}
\hypersetup{pdftitle={The childdoc Package}}
\author{Niklas Beisert\\[2ex]
  Institut f\"ur Theoretische Physik\\
  Eidgen\"ossische Technische Hochschule Z\"urich\\
  Wolfgang-Pauli-Strasse 27, 8093 Z\"urich, Switzerland\\[1ex]
  \href{mailto:nbeisert@itp.phys.ethz.ch}
  {\texttt{nbeisert@itp.phys.ethz.ch}}}
\hypersetup{pdfauthor={Niklas Beisert}}
\hypersetup{pdfsubject={Manual for the LaTeX2e Package childdoc}}
\date{30 December 2018, \textsf{v2.0}}
\maketitle

\begin{abstract}\noindent
\textsf{childdoc} is a \LaTeXe{} package
that enables the direct compilation
of document sections included by |\include|
to individual files.
\end{abstract}

\begingroup
\parskip0ex
\tableofcontents
\endgroup

%%%%%%%%%%%%%%%%%%%%%%%%%%%%%%%%%%%%%%%%%%%%%%%%%%%%%%%%%%%%%%%%%%%%%%%%%%%%%%%%
%%%%%%%%%%%%%%%%%%%%%%%%%%%%%%%%%%%%%%%%%%%%%%%%%%%%%%%%%%%%%%%%%%%%%%%%%%%%%%%%
\section{Introduction}

\LaTeX{} provides a mechanism to structure a large document (such as a book)
into a main file and several child files (containing the chapters)
using the |\include| command.
This mechanism is beneficial for documents
which span hundreds of pages in order to
make the source file(s) more manageable.
Moreover, compilation can be restricted to
selected child files by means of the |\includeonly| command.
The latter feature can be used to reduce the compilation time while editing
(this was significantly more useful in the earlier days of \LaTeX{})
or to generate a smaller document which is easier to navigate.
Another application of |\includeonly| is to generate
documents consisting of selected parts of the complete document.

However, there are a few drawbacks of the plain |\include| mechanism:
\begin{itemize}
\item
The child files cannot be compiled on their own,
they can only be compiled via the main file.
A naive editing environment
(such as a text editor with an option
to have the current file processed by \LaTeX)
may require one to switch to the main file before compiling;
attempting to compile the child file produces errors.
\item
The main file must be modified (each time)
to adjust the |\includeonly| command
to the present needs. This easily leaves the main file in a messy state.
\item
The generated document will always carry the filename
of the main document. This is inconvenient if
several child files are to be compiled and
to be kept for distribution.
\end{itemize}

The present package provides a simple interface
to make child files individually compilable by \LaTeX{}.
Compiling a child file then has the same effect as compiling
the main file with an |\includeonly| command
to select the appropriate child.
Moreover the generated document will carry the name of the child
rather than the main file.
This resolves all three above issues.

This feature is meant to make the editing of books,
thesis documents and lecture notes somewhat more convenient.
However, the package can also be used efficiently for
composing a series of documents (such as exercise sheets)
which are typically distributed individually.
It then assists the author in generating the individual documents
(potentially in different versions)
as well as a document containing the collected series.
Another application is in developing style files
or other kinds of included material
where compilation of the style file could redirect
to a sample or test file.

%%%%%%%%%%%%%%%%%%%%%%%%%%%%%%%%%%%%%%%%%%%%%%%%%%%%%%%%%%%%%%%%%%%%%%%%%%%%%%%%
%%%%%%%%%%%%%%%%%%%%%%%%%%%%%%%%%%%%%%%%%%%%%%%%%%%%%%%%%%%%%%%%%%%%%%%%%%%%%%%%
\section{Usage}

First of all, the package \textsf{childdoc} is \emph{not} a standard
\LaTeXe{} |.sty| style file! Therefore it needs to be invoked in
a non-standard way.

%%%%%%%%%%%%%%%%%%%%%%%%%%%%%%%%%%%%%%%%%%%%%%%%%%%%%%%%%%%%%%%%%%%%%%%%%%%%%%%%
\subsection{Included Files}
\label{sec:include}

%%%%%%%%%%%%%%%%%%%%%%%%%%%%%%%%%%%%%%%%
\DescribeMacro{\childdocmain}
To use the package, add the commands
\begin{center}
\begin{tabular}{l}
|\input{childdoc.def}|\\
|\childdocmain{}|\\
\end{tabular}
\end{center}
at the very top of the main \LaTeX{} file,
in particular \emph{before} the |\documentclass| statement!
The argument of |\childdocmain| should be left empty
(but it must be present).

%%%%%%%%%%%%%%%%%%%%%%%%%%%%%%%%%%%%%%%%
\DescribeMacro{\childdocof}
Furthermore, add the commands
\begin{center}
\begin{tabular}{l}
|\input{childdoc.def}|\\
|\childdocof{|\textit{main}|}|\\
\end{tabular}
\end{center}
at the top of every child file \textit{child}
which is included by |\include{|\textit{child}|}|
from within the main file
(or at least for those files to be compiled individually).
The argument \textit{main} must be the filename of the main file.

There are a couple of
considerations in setting up the main and child documents:

%%%%%%%%%%%%%%%%%%%%%%%%%%%%%%%%%%%%%%%%
\paragraph{Restrictions.}

Please note the following restrictions:
\begin{itemize}
\item
|\childdocmain| must be called with one argument \textit{main}
to ensure compatibility with earlier version of the package.
It must either be empty (|\childdocmain{}|)
or precisely match the filename of the main file in which it is specified.
See \secref{sec:detection} for further information.
\item
The filename \textit{main} must be specified without the |.tex| extension.
\item
The filename \textit{main} is case sensitive
(even in case-insensitive file systems)
due to internal string comparison.
\item
The argument \textit{main} should be fully expanded, it cannot be a macro.
\item
Subdirectories and special characters should be avoided in filenames.
\item
The command |\childdocmain{|\textit{main}|}| must be followed by a whitespace.
It should not be followed immediately by another command
or by a comment mark `|%|'.
This is because the \TeX{} parser reads the token immediately following
the argument of |\childdocmain| and puts it
at the beginning of every child section;
however, a white\-space is ignored.
\end{itemize}

%%%%%%%%%%%%%%%%%%%%%%%%%%%%%%%%%%%%%%%%
\paragraph{Content of Main File.}

It is advisable to place all content in the child files included by |\include|.
Any output contained in the main file will appear in all child documents
unless suppressed manually;
it cannot be suppressed automatically by the |\includeonly| directive
and thus should normally be avoided.
A method to include some content in the main file
by means of conditional processing is described in \secref{sec:conditional}.

%%%%%%%%%%%%%%%%%%%%%%%%%%%%%%%%%%%%%%%%
\paragraph{Page Numbering.}

When only a part of the document is compiled,
the appropriate numbering of pages
(as well as other status parameters)
is determined from the |.aux| files.
The latter contain information from previous passes.
However this information needs to propagate through
all intermediate child documents.
Therefore the page numbering in child documents may well
be inconsistent until the complete document is compiled at least once.

A useful (if unconventional) way to always ensure a consistent
page numbering is to restart the numbering in each child document
and denote the pages by `\textit{child}|.|\textit{page}'
where \textit{child} represents the chapter/section number of the child file.
This can be achieved by the command
|\numberwithin{page}{|\textit{child}|}|
of the \textsf{amsmath} package
where \textit{child} can be |chapter| or |section|
depending on the chosen structuring.
Alternatively, one can modify the macro |\thepage| appropriately
and reset the counter |page| at the start of each child file.

%%%%%%%%%%%%%%%%%%%%%%%%%%%%%%%%%%%%%%%%%%%%%%%%%%%%%%%%%%%%%%%%%%%%%%%%%%%%%%%%
\subsection{Conditional Processing}
\label{sec:conditional}

The package provides a mechanism to compile different versions
of a document. To customise the versions further some conditional processing
can come in handy to distinguish which version is being compiled.
The package provides two macros to describe the compilation context:

%%%%%%%%%%%%%%%%%%%%%%%%%%%%%%%%%%%%%%%%
\DescribeMacro{\ifchilddoc}
The conditional |\ifchilddoc| distinguishes between the compilation of
child documents and the main document:
%
\begin{center}
|\ifchilddoc |\textit{child-code}| |[|\||else |\textit{main-code}]| \||fi|
\end{center}

%%%%%%%%%%%%%%%%%%%%%%%%%%%%%%%%%%%%%%%%
\DescribeMacro{\childdocname}
\DescribeMacro{\childdocjob}
The macro |\childdocname| contains the filename (without extension)
of the main or child file being processed.
Note that |\childdocjob| will always contain the name of the main file.

%%%%%%%%%%%%%%%%%%%%%%%%%%%%%%%%%%%%%%%%
\paragraph{Title Page.}

Conditional processing can be used to include a title or banner page
in the main document when proper precautions are taken.
Importantly, the code in the main file should ensure that the page counter
(as well as other status parameters which are stored in the |.aux| files)
takes the same value after the conditional processing.
Otherwise the page numbers may take divergent values
depending on which part is compiled.

For example, a title page could be declared by:
%
\begin{center}
\begin{tabular}{l}
|\ifchilddoc\||else|\\
|\addtocounter{page}{-1}|\\
\textit{code for title page}\\
|\newpage|\\
|\||fi|
\end{tabular}
\end{center}
%
A banner page for the child documents can be generated by:
%
\begin{center}
\begin{tabular}{l}
|\ifchilddoc|\\
|\addtocounter{page}{-1}|\\
\textit{code for banner page}\\
|\newpage|\\
|\||fi|
\end{tabular}
\end{center}
%
Here one could write a message such as:
\begin{center}
|This is the part \childdocname{} of \childdocjob{}.|
\end{center}

%%%%%%%%%%%%%%%%%%%%%%%%%%%%%%%%%%%%%%%%%%%%%%%%%%%%%%%%%%%%%%%%%%%%%%%%%%%%%%%%
\subsection{Flags}
\label{sec:flags}

The package makes it easy to generate different versions
of the main or child documents.
To this end compilation flags can be defined
and assigned different default values.
They will be particularly useful in conjunction
with the forwarding mechanism described in \secref{sec:forward}.

For example, it may be useful to have a flag |\version|
which can be set to |draft| or |final|.
The document source will contain some conditional code
depending on the value of |\version|.
Suppose further, the flag should default to |final| for the main file
and to |draft| for child files
which is a natural assignment for editing the document.
This is achieved by placing the following code
in the preamble of the main document
(below the |\childdocmain| directive):
%
\begin{center}
\begin{tabular}{l}
|\ifchilddoc|\\
|\providecommand{\version}{draft}|\\
|\||else|\\
|\providecommand{\version}{final}|\\
|\||fi|
\end{tabular}
\end{center}
%
The definition by |\providecommand| makes sure
that previous definitions are not overwritten.
Further statements |\providecommand{\version}{...}|
can thus be added before the above code to override it.

For the main file, one might add a line
(between |\childdocmain| and the above block)
%
\begin{center}
|%\ifchilddoc\||else\providecommand{\version}{draft}\||fi|
\end{center}
%
which can be uncommented to produce a draft version.
Likewise one can add a line to the very top of a child file
(above the |\childdocof{|\textit{main}|}| directive)
%
\begin{center}
|%\providecommand{\version}{final}|
\end{center}
%
which can be uncommented to produce the final version of this child document.

%%%%%%%%%%%%%%%%%%%%%%%%%%%%%%%%%%%%%%%%%%%%%%%%%%%%%%%%%%%%%%%%%%%%%%%%%%%%%%%%
\subsection{Forwarding}
\label{sec:forward}

Different versions of the main or child documents
using compilation flags as described in \secref{sec:flags}
can be (permanently) stored in different files
for convenient compilation, viewing and distribution.
To this end, the package defines a command
to pass on compilation to a different file:

%%%%%%%%%%%%%%%%%%%%%%%%%%%%%%%%%%%%%%%%
\DescribeMacro{\childdocforward}
The command |\childdocforward| redirects processing to
another source file:
%
\begin{center}
\begin{tabular}{l}
|\input{childdoc.def}|\\
|\childdocforward[|\textit{main}|]{|\textit{dest}|}|\\
\end{tabular}
\end{center}
%
The argument \textit{dest} is the destination file
(without extension).
It should be the main file or one of the child files.
Note that further \textsf{childdoc} directives
such as |\childdocof| and |\childdocforward|
in the indicated file will be processed in this form.
The optional argument \textit{main}
passes on directly to the main file \textit{main}
while pretending to compile the child \textit{dest}.
This form behaves as if \textit{dest}
issues |\childdocof{|\textit{main}|}| right away,
and no further \textsf{childdoc} directives will be processed.

%%%%%%%%%%%%%%%%%%%%%%%%%%%%%%%%%%%%%%%%
\DescribeMacro{\...prefix}
In the alternative form |\childdocforwardprefix|,
%
\begin{center}
\begin{tabular}{l}
|\input{childdoc.def}|\\
|\childdocforwardprefix[|\textit{main}|]{|\textit{prefix}|}{|\textit{dest}|}|
\end{tabular}
\end{center}
%
the destination file is determined by a pattern
depending on the current file:
To make this work, the current file must be called
`{\textit{prefix}\hspace{0.2em}\textit{suffix}}'
with \textit{prefix} matching precisely the argument.
Processing is then passed on to the file
`{\textit{dest}\hspace{0.2em}\textit{suffix}}'.
Surely, the same effect is achieved by
directly specifying the
argument `{\textit{dest}\hspace{0.2em}\textit{suffix}}'
in the first form.
However, that requires to set up a different file
for each child. With the alternative form of the command
all these files can have exactly the same content
which simplifies setting them up and maintaining them.

For example, the following file |draft.tex|
with a compilation flag |\version| as described in \secref{sec:flags}
compiles the main document as a draft:
%
\begin{center}
\begin{tabular}{l}
|\def\version{draft}|\\
|\input{childdoc.def}|\\
|\childdocforward{|\textit{main}|}|
\end{tabular}
\end{center}
%
Likewise, the following files |final|\textit{nn}|.tex|
compile the final version of the child document
|child|\textit{nn}|.tex|:
%
\begin{center}
\begin{tabular}{l}
|\def\version{final}|\\
|\input{childdoc.def}|\\
|\childdocforwardprefix{final}{child}|
\end{tabular}
\end{center}
%

Note that when several versions of a main file and/or of each child file
are to be generated, it may be convenient to set up a |Makefile| or
shell script to automatise the process.

%%%%%%%%%%%%%%%%%%%%%%%%%%%%%%%%%%%%%%%%%%%%%%%%%%%%%%%%%%%%%%%%%%%%%%%%%%%%%%%%
\subsection{Command Line Processing}
\label{sec:commandline}

The effect of redirection files can also be achieved by invoking
the \LaTeX{} compiler with a more elaborate command line.
Most conveniently this should be done as part
of a shell script or a |Makefile|.

When using \textsf{childdoc} in the main file, the following
command lines effectively perform a redirection
(note that depending on the shell being used,
backslashes may have to be doubled: `|\|' $\to$ `|\\|'):
%
\begin{center}
|... -jobname "|\textit{target}|" |\\|"|[\textit{flags}]%
|\input{childdoc.def}\childdocforward[|\textit{main}|]{|\textit{dest}|}"|
\end{center}
%
Here \textit{target} is the name of the output file,
\textit{main} is the name of the main file
and \textit{dest} is the name of the main or child file to be processed
(all filenames without extensions).
The optional argument \textit{main} can be omitted
if \textit{main} matches \textit{dest}.
Optionally, compilation \textit{flags} can be defined via |\def| commands.
This command line makes the \TeX{} engine believe
it is compiling the file \textit{target}
whose content is specified as the latter parameter.
The provided code then forwards the processing to
\textit{main} or \textit{dest} as described in \secref{sec:forward}.

%%%%%%%%%%%%%%%%%%%%%%%%%%%%%%%%%%%%%%%%%%%%%%%%%%%%%%%%%%%%%%%%%%%%%%%%%%%%%%%%
\subsection{Include by Input}
\label{sec:input}

Including child documents by |\include| has some restrictions by design.
Most notably, the content of a child document always occupies
its own set of pages; pages cannot be shared between child documents.
Usually, this behaviour makes perfect sense
because each child document contain an essential part of the document.
However, in some situations it may be desirable to compose
a document from a collection of parts
without having mandatory page breaks between then.
For this case, the package
provides a mechanism to include parts
by |\input| which can also be processed individually.
However, by construction this mechanism
requires manual handling of the content to be output.

%%%%%%%%%%%%%%%%%%%%%%%%%%%%%%%%%%%%%%%%
\DescribeMacro{\ifchilddocmanual}
The main file should be prepared as usual, see \secref{sec:include}.
However, the document body must make a distinction
between processing of an individual part and of the main document, e.g.:
%
\begin{center}
\begin{tabular}{l}
|\ifchilddocmanual|\\
|\input{\childdocname}|\\
|\||else|\\
\textit{document body with }|\input{|\textit{part}|}|\\
|\||fi|
\end{tabular}
\end{center}
%
The conditional |\ifchilddocmanual| is true whenever
a part to be included by |\input| is being compiled,
and the name of the part is stored in |\childdocname|.

%%%%%%%%%%%%%%%%%%%%%%%%%%%%%%%%%%%%%%%%
\DescribeMacro{\childdocby}
Each part to be included by |\input| should start with:
%
\begin{center}
\begin{tabular}{l}
|\input{childdoc.def}|\\
|\childdocby{|\textit{main}|}|\\
\end{tabular}
\end{center}
%
The directive |\childdocby| is similar to |\childdocof|
described in \secref{sec:include},
but the subsequent selection of content must be done manually.
To that end, both |\ifchilddoc| and |\ifchilddocmanual|
will be true upon processing of a part,
and the name of the part is stored in |\childdocname|.
Note that |\jobname| will be set to the filename of the current part
so that each part receives an individual |.aux| file
that does not interfere with the |.aux| file(s) of the main document.
This behaviour can be altered by the alternative form
|\childdocby[*]{|\textit{main}|}| (with a non-empty optional argument)
which uses the |.aux| file of the main document
by setting |\jobname| to \textit{main}.

%%%%%%%%%%%%%%%%%%%%%%%%%%%%%%%%%%%%%%%%%%%%%%%%%%%%%%%%%%%%%%%%%%%%%%%%%%%%%%%%
\subsection{Driver Development}
\label{sec:driver}

The \textsf{childdoc} mechanism can also be use for the development
of definition files such as \LaTeX{} styles or classes.
This case differs from the above setup with multiple parts
included by |\include| in that no |\includeonly| should be invoked.
This can be achieved by starting the include file
(before |\ProvidesPackage|) with:
%
\begin{center}
\begin{tabular}{l}
|\input{childdoc.def}|\\
|\childdocforward{|\textit{main}|}|\\
\end{tabular}
\end{center}
%
or alternatively with:
%
\begin{center}
\begin{tabular}{l}
|\input{childdoc.def}|\\
|\childdocby{|\textit{main}|}|\\
\end{tabular}
\end{center}
%
Both forms have slightly different effects as described above.
The main file is prepared as usual, see \secref{sec:include}.

%%%%%%%%%%%%%%%%%%%%%%%%%%%%%%%%%%%%%%%%%%%%%%%%%%%%%%%%%%%%%%%%%%%%%%%%%%%%%%%%
\subsection{Legacy Detection}
\label{sec:detection}

The directive |\childdocmain| in the main file can detect
whether the complete document or merely a child is to be compiled
even without using the directive |\childdocof|.
This method is deprecated because it is less robust
and there is no compelling reason to use it;
it is merely provided for backward compatibility
and it may be removed in future versions.

If the detection mechanism is to be used,
it is mandatory to correctly specify
the filename of the main file as the argument of |\childdocmain|:
%
\begin{center}
\begin{tabular}{l}
|\input{childdoc.def}|\\
|\childdocmain{|\textit{main}|}|\\
\end{tabular}
\end{center}
%
If |\jobname| does not match the argument \textit{main} of |\childdocmain|,
it is assumed that |\jobname| points to the child file to be compiled.
When using |\childdocmain| with the main file specified as argument,
it suffices to start a child file
with just |\input{|\textit{main}|}|
without loading of the package and using |\childdocof|.
If instead all processing is done
with the appropriate \textsf{childdoc} directives,
the argument of \textit{main} of |\childdocmain| can be empty.

An alternative version of the command line processing described
in \secref{sec:commandline} using the detection mechanism reads:
%
\begin{center}
|... -jobname "|\textit{target}|" "|[\textit{flags}]%
[|\def\jobname{|\textit{dest}|}|]|\input{|\textit{main}|}"|
\end{center}

%%%%%%%%%%%%%%%%%%%%%%%%%%%%%%%%%%%%%%%%%%%%%%%%%%%%%%%%%%%%%%%%%%%%%%%%%%%%%%%%
\subsection{Manual Code}
\label{sec:manual}

In case one cannot be certain whether the definitions file |childdoc.def|
is installed on the target \TeX{} distribution
and one prefers not to ship it,
it is conceivable to paste a few relevant commands into the sources.

To that end, drop all statements |\input{childdoc.def}|
and perform the replacements as outlined below.
Instead of |\childdocmain{|\textit{main}|}| add the following code
to the top of the main file:
%
\begin{center}
\begin{tabular}{l}
|\||ifdefined\childdocname\endinput\||fi\newif\ifchilddoc|\\
|\edef\childdocname{\scantokens\expandafter{\jobname\noexpand}}|\\
|\def\childdocmain{|\textit{main}|}\||ifx\childdocmain\childdocname\||else|\\
|\childdoctrue\includeonly{\childdocname}\let\jobname\childdocmain\||fi|\\
\end{tabular}
\end{center}
%
Instead of |\childdocof{|\textit{main}|}| just include the main file
at the top of each child file:
%
\begin{center}
|\input{|\textit{main}|}|
\end{center}
%
A simple redirection |\childdocforward{|\textit{dest}|}| is achieved by:
%
\begin{center}
|\def\jobname{|\textit{dest}|}\input{\jobname}|
\end{center}
%
The redirection with prefix
|\childdocforwardprefix[|\textit{prefix}|]{|\textit{dest}|}|
is accomplished by:
%
\begin{center}
\begin{tabular}{l}
|{\edef\jobname{\scantokens\expandafter{\jobname\noexpand}}|\\
|\def\redirectjob |\textit{prefix}|#1~~~{\gdef\jobname{|\textit{dest}|#1}}|\\
|\expandafter\redirectjob\jobname~~~}\input{\jobname}|
\end{tabular}
\end{center}

In an alternative approach,
child documents can be compiled by a specific command line
without additional code or specific definitions:
%
\begin{center}
|... -jobname "|\textit{target}|" "|[\textit{flags}]%
|\includeonly{|\textit{dest}|}\input{|\textit{main}|}"|
\end{center}
%

%%%%%%%%%%%%%%%%%%%%%%%%%%%%%%%%%%%%%%%%%%%%%%%%%%%%%%%%%%%%%%%%%%%%%%%%%%%%%%%%
%%%%%%%%%%%%%%%%%%%%%%%%%%%%%%%%%%%%%%%%%%%%%%%%%%%%%%%%%%%%%%%%%%%%%%%%%%%%%%%%
\section{Information}

%%%%%%%%%%%%%%%%%%%%%%%%%%%%%%%%%%%%%%%%%%%%%%%%%%%%%%%%%%%%%%%%%%%%%%%%%%%%%%%%
\subsection{Copyright}

Copyright \copyright{} 2017--2018 Niklas Beisert

This work may be distributed and/or modified under the
conditions of the \LaTeX{} Project Public License, either version 1.3
of this license or (at your option) any later version.
The latest version of this license is in
  \url{http://www.latex-project.org/lppl.txt}
and version 1.3 or later is part of all distributions of \LaTeX{}
version 2005/12/01 or later.

This work has the LPPL maintenance status `maintained'.

The Current Maintainer of this work is Niklas Beisert.

This work consists of the files |README.txt|, |childdoc.ins| and |childdoc.dtx|
as well as the derived files |childdoc.def|, |cdocsamp.tex|
with |cdocsch1.tex|, |cdocsch2.tex|, |cdocspt3.tex|, |cdocspt4.tex|,
|cdocsdrf.tex|, |cdocsfn1.tex|, |cdocsfn2.tex|
as well as |childdoc.pdf|.

%%%%%%%%%%%%%%%%%%%%%%%%%%%%%%%%%%%%%%%%%%%%%%%%%%%%%%%%%%%%%%%%%%%%%%%%%%%%%%%%
\subsection{Files and Installation}

The package consists of the files:
%
\begin{center}
\begin{tabular}{ll}
    |README.txt|   & readme file \\
    |childdoc.ins| & installation file \\
    |childdoc.dtx| & source file \\
    |childdoc.def| & definition file \\
    |cdocsamp.tex| & sample main file \\
    |cdocsch1.tex| & sample include file \\
    |cdocsch2.tex| & sample include file \\
    |cdocspt3.tex| & sample part file \\
    |cdocspt4.tex| & sample part file \\
    |cdocsdrf.tex| & sample redirection file \\
    |cdocsfn1.tex| & sample redirection file \\
    |cdocsfn2.tex| & sample redirection file \\
    |childdoc.pdf| & manual
\end{tabular}
\end{center}
%
The distribution consists of the files
|README.txt|, |childdoc.ins| and |childdoc.dtx|.
%
\begin{itemize}
\item
Run (pdf)\LaTeX{} on |childdoc.dtx|
to compile the manual |childdoc.pdf| (this file).
\item
Run \LaTeX{} on |childdoc.ins| to create the definitions file |childdoc.def|
and the sample |cdocsamp.tex| with include files
|cdocsch1.tex|, |cdocsch2.tex|, |cdocspt3.tex|, |cdocspt4.tex|,
|cdocsdrf.tex|, |cdocsfn1.tex|, |cdocsfn2.tex|.
Then copy the file |childdoc.def| to an appropriate directory of your \LaTeX{}
distribution, e.g.\ \textit{texmf-root}|/tex/latex/childdoc|.
\end{itemize}

%%%%%%%%%%%%%%%%%%%%%%%%%%%%%%%%%%%%%%%%%%%%%%%%%%%%%%%%%%%%%%%%%%%%%%%%%%%%%%%%
\subsection{Related CTAN Packages}

There are several other packages which offer a similar functionality:
%
\begin{itemize}
\item
The packages
\href{http://ctan.org/pkg/docmute}{\textsf{docmute}},
\href{http://ctan.org/pkg/includex}{\textsf{includex}} and
\href{http://ctan.org/pkg/standalone}{\textsf{standalone}}
provide commands to include only the document body of
a child file thus allowing both files to be compiled individually.
\item
The packages \href{http://ctan.org/pkg/subdocs}{\textsf{subdocs}}
and \href{http://ctan.org/pkg/subfiles}{\textsf{subfiles}}
provide structures in which the main and child documents can be
encapsulated and allowing them to be compiled individually.
The inclusion mechanism is different from the conventional |\include|.
\item
The package \href{http://ctan.org/pkg/combine}{\textsf{combine}}
is an elaborate solution to combine several documents into one.
\end{itemize}
%
See also the CTAN topic \href{http://ctan.org/topic/subdocs}{\textsf{subdocs}}
for further related packages.
The present package differs from the above solutions in that
a document structure constructed with the conventional |\include| mechanism
just needs two extra commands at the top of every file
such that all constituent files can be compiled individually.

%%%%%%%%%%%%%%%%%%%%%%%%%%%%%%%%%%%%%%%%%%%%%%%%%%%%%%%%%%%%%%%%%%%%%%%%%%%%%%%%
%\subsection{Feature Suggestions}
%
%The following is a list of features which may be useful for future
%versions of this package:
%%
%\begin{itemize}
%\item
%\ldots
%\end{itemize}

%%%%%%%%%%%%%%%%%%%%%%%%%%%%%%%%%%%%%%%%%%%%%%%%%%%%%%%%%%%%%%%%%%%%%%%%%%%%%%%%
\subsection{Revision History}

%%%%%%%%%%%%%%%%%%%%%%%%%%%%%%%%%%%%%%%%
\paragraph{v2.0:} 2018/12/30

\begin{itemize}
\item
immediate forward processing
\item
added |\childdocby| mechanism
\item
manual restructured
\end{itemize}

%%%%%%%%%%%%%%%%%%%%%%%%%%%%%%%%%%%%%%%%
\paragraph{v1.6:} 2018/01/17

\begin{itemize}
\item
application for development of include files
\item
corrections to manual
\end{itemize}

%%%%%%%%%%%%%%%%%%%%%%%%%%%%%%%%%%%%%%%%
\paragraph{v1.5:} 2017/05/21

\begin{itemize}
\item
more complete structuring introduced
\item
|\childdocof| introduced
\item
|\childdoc| renamed to |\childdocmain|
\item
|\childredirect| renamed to |\childdocforward| and |\childdocforwardprefix|
and functionality expanded
\end{itemize}

%%%%%%%%%%%%%%%%%%%%%%%%%%%%%%%%%%%%%%%%
\paragraph{v1.0:} 2017/04/27

\begin{itemize}
\item
manual and install package
\item
first version published on CTAN
\end{itemize}

%%%%%%%%%%%%%%%%%%%%%%%%%%%%%%%%%%%%%%%%
\paragraph{v0.6:} 2017/04/26

\begin{itemize}
\item
redirection mechanism added
\end{itemize}

%%%%%%%%%%%%%%%%%%%%%%%%%%%%%%%%%%%%%%%%
\paragraph{v0.5:} 2017/04/26

\begin{itemize}
\item
functionality in definition file
\end{itemize}


%%%%%%%%%%%%%%%%%%%%%%%%%%%%%%%%%%%%%%%%%%%%%%%%%%%%%%%%%%%%%%%%%%%%%%%%%%%%%%%%
%%%%%%%%%%%%%%%%%%%%%%%%%%%%%%%%%%%%%%%%%%%%%%%%%%%%%%%%%%%%%%%%%%%%%%%%%%%%%%%%
%%%%%%%%%%%%%%%%%%%%%%%%%%%%%%%%%%%%%%%%%%%%%%%%%%%%%%%%%%%%%%%%%%%%%%%%%%%%%%%%
\appendix

\settowidth\MacroIndent{\rmfamily\scriptsize 000\ }

 \DocInput{childdoc.dtx}

\end{document}
%</driver>
% \fi
%
% %%%%%%%%%%%%%%%%%%%%%%%%%%%%%%%%%%%%%%%%%%%%%%%%%%%%%%%%%%%%%%%%%%%%%%%%%%%%%%
% %%%%%%%%%%%%%%%%%%%%%%%%%%%%%%%%%%%%%%%%%%%%%%%%%%%%%%%%%%%%%%%%%%%%%%%%%%%%%%
% \section{Sample}
%\iffalse
%<*samplemain>
%\fi
%
% The following presents a sample document
% with two chapters, two parts, a title page,
% a compile flag as well as three forwarding files to set the flag.
% It consists of eight |.tex| files:
% \begin{center}
% \begin{tabular}{ll}
% |cdocsamp.tex|&main file\\
% |cdocsch1.tex|&include file for chapter 1\\
% |cdocsch2.tex|&include file for chapter 2\\
% |cdocspt3.tex|&include file for part 3\\
% |cdocspt4.tex|&include file for part 4\\
% |cdocsdrf.tex|&forwarding file for main file in draft mode\\
% |cdocsfi1.tex|&forwarding file for final version of chapter 1\\
% |cdocsfi2.tex|&forwarding file for final version of chapter 2\\
% \end{tabular}
% \end{center}
% Each of the eight files can be compiled directly by the \LaTeX{} compiler.
%
% %%%%%%%%%%%%%%%%%%%%%%%%%%%%%%%%%%%%%%
% \paragraph{Main File.}
%
% The main file is called |cdocsamp.tex|.
%
% Load the \textsf{childdoc} definitions and
% declare the filename for the main document:
%    \begin{macrocode}
\input{childdoc.def}
\childdocmain{}
%    \end{macrocode}

% Optional override for |\version| flag:
%    \begin{macrocode}
%%\ifchilddoc\else\providecommand{\version}{draft}\fi
%    \end{macrocode}

% Define the default values for the |\version| flag
% (|final| for the main file and |draft| for childs):
%    \begin{macrocode}
\ifchilddoc
\providecommand{\version}{draft}
\else
\providecommand{\version}{final}
\fi
%    \end{macrocode}

% Load the standard document class:
%    \begin{macrocode}
\documentclass[12pt]{article}
%    \end{macrocode}

% Start the document body:
%    \begin{macrocode}
\begin{document}
%    \end{macrocode}

% Declare a title page.
% Print title, part of document being processed and version flag:
%    \begin{macrocode}
\addtocounter{page}{-1}
\begin{center}
{\LARGE\bfseries{}childdoc example\par}
\vspace{1cm}
\ifchilddoc
\ifchilddocmanual part\else chapter\fi:
`\childdocname' of `\childdocjob'\par
\else
main document: `\childdocjob'\par
\fi
version: \version\par
\end{center}
\newpage
%    \end{macrocode}

% Manually include selected file,
% otherwise process as usual:
%    \begin{macrocode}
\ifchilddocmanual
\section*{part `\childdocname'}
\input{\childdocname}
\else
%    \end{macrocode}

% Include the two chapters:
%    \begin{macrocode}
\include{cdocsch1}
\include{cdocsch2}
%    \end{macrocode}

% Include the two parts unless only chapters should be displayed:
%    \begin{macrocode}
\ifchilddoc\else
\section{part three}
\input{cdocspt3}
\section{part four}
\input{cdocspt4}
\fi
%    \end{macrocode}

% Process as usual until here:
%    \begin{macrocode}
\fi
%    \end{macrocode}

% End of document body:
%    \begin{macrocode}
\end{document}
%    \end{macrocode}
%\iffalse
%</samplemain>
%\fi
%
% %%%%%%%%%%%%%%%%%%%%%%%%%%%%%%%%%%%%%%
% \paragraph{Chapter Include Files.}
%
% The include files are called |cdocsch1.tex| and |cdocsch2.tex|.
%
%\iffalse
%<*samplechap1|samplechap2>
%\fi

% Optional override for |\version| flag:
%    \begin{macrocode}
%%\providecommand{\version}{final}
%    \end{macrocode}

% Include the main document:
%    \begin{macrocode}
\input{childdoc.def}
\childdocof{cdocsamp}
%    \end{macrocode}

%\iffalse
%</samplechap1|samplechap2>
%\fi
%
%\iffalse
%<*samplechap1>
%\fi
% Some text for chapter 1:
%    \begin{macrocode}
\section{one}
some text in chapter one
%    \end{macrocode}

%\iffalse
%</samplechap1>
%\fi
% Some text for chapter 2:
%\iffalse
%<*samplechap2>
%\fi
%    \begin{macrocode}
\section{two}
more text in chapter two
%    \end{macrocode}

%\iffalse
%</samplechap2>
%\fi
%
% %%%%%%%%%%%%%%%%%%%%%%%%%%%%%%%%%%%%%%
% \paragraph{Part Include Files.}
%
% The include files are called |cdocspt3.tex| and |cdocspt4.tex|.
%
%\iffalse
%<*samplepart3|samplepart4>
%\fi

% Optional override for |\version| flag:
%    \begin{macrocode}
%%\providecommand{\version}{final}
%    \end{macrocode}

% Include the main document:
%    \begin{macrocode}
\input{childdoc.def}
\childdocby{cdocsamp}
%    \end{macrocode}

%\iffalse
%</samplepart3|samplepart4>
%\fi
%
%\iffalse
%<*samplepart3>
%\fi
% Some text for part 3:
%    \begin{macrocode}
some text in part three
%    \end{macrocode}

%\iffalse
%</samplepart3>
%\fi
% Some text for part 4:
%\iffalse
%<*samplepart4>
%\fi
%    \begin{macrocode}
more text in part four
%    \end{macrocode}

%\iffalse
%</samplepart4>
%\fi
%
% %%%%%%%%%%%%%%%%%%%%%%%%%%%%%%%%%%%%%%
% \paragraph{Forwarding for a Complete Draft.}
%
% The following forwarding file |cdocsdrf.tex|
% compiles the main document in draft mode:
%\iffalse
%<*sampledraft>
%\fi
%    \begin{macrocode}
\def\version{draft}
\input{childdoc.def}
\childdocforward{cdocsamp}
%    \end{macrocode}

%\iffalse
%</sampledraft>
%\fi
%
% %%%%%%%%%%%%%%%%%%%%%%%%%%%%%%%%%%%%%%
% \paragraph{Forwarding for Final Version of the Chapters.}
%
% The following forwarding files |cdocsfn1.tex| and |cdocsfn2.tex|
% (with identical content)
% compile the final versions of the child documents
% |cdocsch1.tex| and |cdocsch2.tex|, respectively:
%\iffalse
%<*samplefinal>
%\fi
%    \begin{macrocode}
\def\version{final}
\input{childdoc.def}
\childdocforwardprefix[cdocsamp]{cdocsfn}{cdocsch}
%    \end{macrocode}

%\iffalse
%</samplefinal>
%\fi
%
% %%%%%%%%%%%%%%%%%%%%%%%%%%%%%%%%%%%%%%
% \paragraph{Command Line Processing.}
%
% The following three command lines generate the output files
% |cdocscld|, |cdocscl1| and |cdocscl2|
% which should be identical to
% |cdocsdrf|, |cdocsch1| and |cdocsfn2|, respectively:
% \begin{center}
% \begin{tabular}{l}
% |latex -jobname cdocscld \|\\
% |  "\def\version{draft}\input{childdoc.def}\childdocforward{cdocsamp}"|\\
% |latex -jobname cdocscl1 \|\\
% |  "\input{childdoc.def}\childdocforward[cdocsamp]{cdocsch1}"|\\
% |latex -jobname cdocscl2 \|\\
% |  "\def\version{final}\input{childdoc.def}\childdocforward{cdocsch2}"|
% \end{tabular}
% \end{center}
% Note that the trailing backslash on each first line
% merely continues the input to the second line
% (for convenient cut ant paste).
% Furthermore, the command |latex| can be replaced by any
% of its alternative versions such as |pdflatex|.
%
% %%%%%%%%%%%%%%%%%%%%%%%%%%%%%%%%%%%%%%%%%%%%%%%%%%%%%%%%%%%%%%%%%%%%%%%%%%%%%%
% %%%%%%%%%%%%%%%%%%%%%%%%%%%%%%%%%%%%%%%%%%%%%%%%%%%%%%%%%%%%%%%%%%%%%%%%%%%%%%
% \section{Implementation}
%\iffalse
%<*package>
%\fi
%
% This section describes the definitions file |childdoc.def|.

% The definitions cannot be loaded using |\usepackage| or |\RequirePackage|
% which has a mechanism to prevent loading a style file more than once.
% When loading the definitions by means of |\input|
% multiple instances have to be prevented manually:
%\iffalse
%This code needs to be before the `\ProvidesFile' directive
%which is defined at the beginning of this file.
%Therefore it is also placed there and commented out here.
%</package>
%<*discard>
%\fi
%    \begin{macrocode}
\ifdefined\childdocmain\endinput\fi
%    \end{macrocode}
%\iffalse
%</discard>
%<*package>
%\fi
%
% \macro{\ifchilddoc}
% \macro{\ifchilddocmanual}
% The conditional |\ifchilddoc| tells whether a
% child (true) or main (false) document is being compiled.
% The conditional |\ifchilddocmanual| tells whether
% the |\includeonly| mechanism is used (false) or
% the selection of child files must be performed manually (true).
% The definitions initialise to false:
%    \begin{macrocode}
\newif\ifchilddoc
\newif\ifchilddocmanual
%    \end{macrocode}

% \macro{\childdocname}
% \macro{\childdocjob}
% The macro |\childdocname| stores the name of the main document
% to be compiled. The macro |\childdocjob| stores the name of
% the document on which the \LaTeX{} compiler was originally invoked.
% The content of |\jobname| cannot be compared
% to filenames specified in the source due to different catcodes.
% The following code rescans |\jobname|, stores the result
% in |\childdocname| and saves a copy in |\childdocjob|:
%    \begin{macrocode}
\edef\childdocname{\scantokens\expandafter{\jobname\noexpand}}
\let\childdocjob\childdocname
%    \end{macrocode}

% \macro{\childdocdisable}
% The macro |\childdocdisable| prevents the main file
% from being processed more than once.
% At this stage, the main document command |\childdocmain|
% is assumed to be called once again where it should do nothing.
% Any subsequent call to it should prevent
% a secondary processing of the main document
% It overwrites the forwarding commands
% |\childdocof| and |\childdocforward|
% with empty macros to prevent further inclusions of the main document:
%    \begin{macrocode}
\newcommand{\childdocdisable}
{
  \renewcommand{\childdocmain}[1]{\renewcommand{\childdocmain}[1]{\endinput}}
  \renewcommand{\childdocof}[1]{}
  \renewcommand{\childdocby}[2][]{}
  \renewcommand{\childdocforward}[2][]{}
  \renewcommand{\childdocdisable}{}
}
%    \end{macrocode}

% \macro{\childdocmain}
% The macro |\childdocmain| is to be called at the top of the main file
% with nothing or the main filename (without extension) as argument.
% First, it breaks loops.
% If the argument is not empty and does not match |\childdocname|
% (which is set by the first inclusion of |childdoc.def|),
% |\ifchilddoc| is set to true, |\includeonly| is applied to the child file
% and |\jobname| is set to the main file
% (for proper handling of |.aux| files):
%    \begin{macrocode}
\newcommand{\childdocmain}[1]
{
  \childdocdisable\childdocmain{}
  \if?#1?\else
    \begingroup
      \def\childdoctmp{#1}
      \ifx\childdoctmp\childdocname
        \def\childdoctmp{}
      \else
        \def\childdoctmp
        {
          \childdoctrue
          \includeonly{\childdocname}
          \def\childdocjob{#1}
          \def\jobname{#1}
        }
      \fi
      \expandafter
    \endgroup
    \childdoctmp
  \fi
}
%    \end{macrocode}

% \macro{\childdocof}
% The command |\childdocof| redirects
% compilation to the main file |#1|.
%    \begin{macrocode}
\newcommand{\childdocof}[1]
{
  \childdocdisable
  \childdoctrue
  \includeonly{\childdocname}
  \def\jobname{#1}
  \def\childdocjob{#1}
  \input{#1}
}
%    \end{macrocode}

% \macro{\childdocby}
% The command |\childdocby| ....
%    \begin{macrocode}
\newcommand{\childdocby}[2][]
{
  \childdocdisable
  \childdoctrue
  \childdocmanualtrue
  \if?#1?\else
    \def\jobname{#2}
  \fi
  \def\childdocjob{#2}
  \input{#2}
  \endinput
}
%    \end{macrocode}

% \macro{\childdocforward}
% The command |\childdocforward| redirects
% compilation to the main file or
% (if the optional argument is given) a child file.
% Parameters are set as if the main file
% or a child file starting with |\childdocof| was compiled.
% Then compilation is handed over to the main file:
%    \begin{macrocode}
\newcommand{\childdocforward}[2][]
{
  \begingroup
    \if?#1?
      \def\childdoctmp
      {
        \def\childdocname{#2}
        \def\childdocjob{#2}
        \def\jobname{#2}
        \input{#2}
        \endinput
      }
    \else
      \def\childdoctmp
      {
        \childdocdisable
        \def\childdocname{#2}
        \childdoctrue
        \includeonly{#2}
        \def\childdocjob{#1}
        \def\jobname{#1}
        \input{#1}
        \endinput
      }
    \fi
    \expandafter
  \endgroup
  \childdoctmp
}
%    \end{macrocode}

% \macro{\childdocforwardprefix}
% The command |\childdocforwardprefix| redirects
% compilation to the main or a child file by means of a pattern.
% The prefix |#1| in the current filename is replaced by |#2|
% and the suffix of the current filename is kept
% (it is assumed that the filename does not contain the substring `|~~~|'
% which is used as a delimiter).
% Compilation is handed over to the new file by |\childdocforward|:
%    \begin{macrocode}
\newcommand{\childdocforwardprefix}[3][]
{
  \begingroup
    \def\childdocextract #2##1~~~{\def\childdoctmp{\childdocforward[#1]{#3##1}}}
    \expandafter\childdocextract\childdocname~~~
    \expandafter
  \endgroup
  \childdoctmp
}
%    \end{macrocode}

% \macro{\childdoc}
% The deprecated macro |\childdoc| is a legacy version of |\childdocmain|:
%    \begin{macrocode}
\newcommand{\childdoc}{\childdocmain}
%    \end{macrocode}

% \macro{\childdocredirect}
% The deprecated macro |\childdocredirect| is a legacy version
% of |\childdocforward| and |\childdocforwardprefix|:
%    \begin{macrocode}
\newcommand{\childdocredirect}[2][]
{
  \begingroup
    \if?#1?
      \def\childdoctmp{\childdocforward{#2}}
    \else
      \def\childdoctmp{\childdocforwardprefix{#1}{#2}}
    \fi
    \expandafter
  \endgroup
  \childdoctmp
}
%    \end{macrocode}

%\iffalse
%</package>
%\fi
%
\endinput
|\\
|\childdocforwardprefix[|\textit{main}|]{|\textit{prefix}|}{|\textit{dest}|}|
\end{tabular}
\end{center}
%
the destination file is determined by a pattern
depending on the current file:
To make this work, the current file must be called
`{\textit{prefix}\hspace{0.2em}\textit{suffix}}'
with \textit{prefix} matching precisely the argument.
Processing is then passed on to the file
`{\textit{dest}\hspace{0.2em}\textit{suffix}}'.
Surely, the same effect is achieved by
directly specifying the
argument `{\textit{dest}\hspace{0.2em}\textit{suffix}}'
in the first form.
However, that requires to set up a different file
for each child. With the alternative form of the command
all these files can have exactly the same content
which simplifies setting them up and maintaining them.

For example, the following file |draft.tex|
with a compilation flag |\version| as described in \secref{sec:flags}
compiles the main document as a draft:
%
\begin{center}
\begin{tabular}{l}
|\def\version{draft}|\\
|% \iffalse
%
% childdoc.dtx Copyright (C) 2017-2018 Niklas Beisert
%
% This work may be distributed and/or modified under the
% conditions of the LaTeX Project Public License, either version 1.3
% of this license or (at your option) any later version.
% The latest version of this license is in
%   http://www.latex-project.org/lppl.txt
% and version 1.3 or later is part of all distributions of LaTeX
% version 2005/12/01 or later.
%
% This work has the LPPL maintenance status `maintained'.
%
% The Current Maintainer of this work is Niklas Beisert.
%
% This work consists of the files childdoc.dtx and childdoc.ins
% and the derived files childdoc.def and cdocsamp.tex with
% cdocsch1.tex, cdocsch2.tex, cdocsdrf.tex, cdocsfn1.tex, cdocsfn2.tex.
%
%<package>\ifdefined\childdocmain\endinput\fi
%<package>\ProvidesFile{childdoc.def}[2018/12/30 v2.0 child document driver]
%<samplemain>\ProvidesFile{cdocsamp.tex}[2018/12/30 v2.0 sample for childdoc]
%<*driver>
%\ProvidesFile{childdoc.drv}[2018/12/30 v2.0 childdoc reference manual file]
\PassOptionsToClass{10pt,a4paper}{article}
\documentclass{ltxdoc}

\usepackage[margin=35mm]{geometry}
\usepackage{hyperref}
\usepackage{hyperxmp}
\usepackage[usenames]{color}

\hypersetup{colorlinks=true}
\hypersetup{pdfstartview=FitH}
\hypersetup{pdfpagemode=UseNone}
\hypersetup{pdfsource={}}
\hypersetup{pdflang={en-UK}}
\hypersetup{pdfcopyright={Copyright 2017-2018 Niklas Beisert.
  This work may be distributed and/or modified under the
  conditions of the LaTeX Project Public License, either version 1.3
  of this license or (at your option) any later version.}}
\hypersetup{pdflicenseurl={http://www.latex-project.org/lppl.txt}}
\hypersetup{pdfcontactaddress={ETH Zurich, ITP, HIT K,
  Wolfgang-Pauli-Strasse 27}}
\hypersetup{pdfcontactpostcode={8093}}
\hypersetup{pdfcontactcity={Zurich}}
\hypersetup{pdfcontactcountry={Switzerland}}
\hypersetup{pdfcontactemail={nbeisert@itp.phys.ethz.ch}}
\hypersetup{pdfcontacturl={http://people.phys.ethz.ch/\xmptilde nbeisert/}}

\newcommand{\secref}[1]{\hyperref[#1]{section \ref*{#1}}}

\parskip1ex
\parindent0pt
\let\olditemize\itemize
\def\itemize{\olditemize\parskip0pt}

\begin{document}

\title{The \textsf{childdoc} Package}
\hypersetup{pdftitle={The childdoc Package}}
\author{Niklas Beisert\\[2ex]
  Institut f\"ur Theoretische Physik\\
  Eidgen\"ossische Technische Hochschule Z\"urich\\
  Wolfgang-Pauli-Strasse 27, 8093 Z\"urich, Switzerland\\[1ex]
  \href{mailto:nbeisert@itp.phys.ethz.ch}
  {\texttt{nbeisert@itp.phys.ethz.ch}}}
\hypersetup{pdfauthor={Niklas Beisert}}
\hypersetup{pdfsubject={Manual for the LaTeX2e Package childdoc}}
\date{30 December 2018, \textsf{v2.0}}
\maketitle

\begin{abstract}\noindent
\textsf{childdoc} is a \LaTeXe{} package
that enables the direct compilation
of document sections included by |\include|
to individual files.
\end{abstract}

\begingroup
\parskip0ex
\tableofcontents
\endgroup

%%%%%%%%%%%%%%%%%%%%%%%%%%%%%%%%%%%%%%%%%%%%%%%%%%%%%%%%%%%%%%%%%%%%%%%%%%%%%%%%
%%%%%%%%%%%%%%%%%%%%%%%%%%%%%%%%%%%%%%%%%%%%%%%%%%%%%%%%%%%%%%%%%%%%%%%%%%%%%%%%
\section{Introduction}

\LaTeX{} provides a mechanism to structure a large document (such as a book)
into a main file and several child files (containing the chapters)
using the |\include| command.
This mechanism is beneficial for documents
which span hundreds of pages in order to
make the source file(s) more manageable.
Moreover, compilation can be restricted to
selected child files by means of the |\includeonly| command.
The latter feature can be used to reduce the compilation time while editing
(this was significantly more useful in the earlier days of \LaTeX{})
or to generate a smaller document which is easier to navigate.
Another application of |\includeonly| is to generate
documents consisting of selected parts of the complete document.

However, there are a few drawbacks of the plain |\include| mechanism:
\begin{itemize}
\item
The child files cannot be compiled on their own,
they can only be compiled via the main file.
A naive editing environment
(such as a text editor with an option
to have the current file processed by \LaTeX)
may require one to switch to the main file before compiling;
attempting to compile the child file produces errors.
\item
The main file must be modified (each time)
to adjust the |\includeonly| command
to the present needs. This easily leaves the main file in a messy state.
\item
The generated document will always carry the filename
of the main document. This is inconvenient if
several child files are to be compiled and
to be kept for distribution.
\end{itemize}

The present package provides a simple interface
to make child files individually compilable by \LaTeX{}.
Compiling a child file then has the same effect as compiling
the main file with an |\includeonly| command
to select the appropriate child.
Moreover the generated document will carry the name of the child
rather than the main file.
This resolves all three above issues.

This feature is meant to make the editing of books,
thesis documents and lecture notes somewhat more convenient.
However, the package can also be used efficiently for
composing a series of documents (such as exercise sheets)
which are typically distributed individually.
It then assists the author in generating the individual documents
(potentially in different versions)
as well as a document containing the collected series.
Another application is in developing style files
or other kinds of included material
where compilation of the style file could redirect
to a sample or test file.

%%%%%%%%%%%%%%%%%%%%%%%%%%%%%%%%%%%%%%%%%%%%%%%%%%%%%%%%%%%%%%%%%%%%%%%%%%%%%%%%
%%%%%%%%%%%%%%%%%%%%%%%%%%%%%%%%%%%%%%%%%%%%%%%%%%%%%%%%%%%%%%%%%%%%%%%%%%%%%%%%
\section{Usage}

First of all, the package \textsf{childdoc} is \emph{not} a standard
\LaTeXe{} |.sty| style file! Therefore it needs to be invoked in
a non-standard way.

%%%%%%%%%%%%%%%%%%%%%%%%%%%%%%%%%%%%%%%%%%%%%%%%%%%%%%%%%%%%%%%%%%%%%%%%%%%%%%%%
\subsection{Included Files}
\label{sec:include}

%%%%%%%%%%%%%%%%%%%%%%%%%%%%%%%%%%%%%%%%
\DescribeMacro{\childdocmain}
To use the package, add the commands
\begin{center}
\begin{tabular}{l}
|\input{childdoc.def}|\\
|\childdocmain{}|\\
\end{tabular}
\end{center}
at the very top of the main \LaTeX{} file,
in particular \emph{before} the |\documentclass| statement!
The argument of |\childdocmain| should be left empty
(but it must be present).

%%%%%%%%%%%%%%%%%%%%%%%%%%%%%%%%%%%%%%%%
\DescribeMacro{\childdocof}
Furthermore, add the commands
\begin{center}
\begin{tabular}{l}
|\input{childdoc.def}|\\
|\childdocof{|\textit{main}|}|\\
\end{tabular}
\end{center}
at the top of every child file \textit{child}
which is included by |\include{|\textit{child}|}|
from within the main file
(or at least for those files to be compiled individually).
The argument \textit{main} must be the filename of the main file.

There are a couple of
considerations in setting up the main and child documents:

%%%%%%%%%%%%%%%%%%%%%%%%%%%%%%%%%%%%%%%%
\paragraph{Restrictions.}

Please note the following restrictions:
\begin{itemize}
\item
|\childdocmain| must be called with one argument \textit{main}
to ensure compatibility with earlier version of the package.
It must either be empty (|\childdocmain{}|)
or precisely match the filename of the main file in which it is specified.
See \secref{sec:detection} for further information.
\item
The filename \textit{main} must be specified without the |.tex| extension.
\item
The filename \textit{main} is case sensitive
(even in case-insensitive file systems)
due to internal string comparison.
\item
The argument \textit{main} should be fully expanded, it cannot be a macro.
\item
Subdirectories and special characters should be avoided in filenames.
\item
The command |\childdocmain{|\textit{main}|}| must be followed by a whitespace.
It should not be followed immediately by another command
or by a comment mark `|%|'.
This is because the \TeX{} parser reads the token immediately following
the argument of |\childdocmain| and puts it
at the beginning of every child section;
however, a white\-space is ignored.
\end{itemize}

%%%%%%%%%%%%%%%%%%%%%%%%%%%%%%%%%%%%%%%%
\paragraph{Content of Main File.}

It is advisable to place all content in the child files included by |\include|.
Any output contained in the main file will appear in all child documents
unless suppressed manually;
it cannot be suppressed automatically by the |\includeonly| directive
and thus should normally be avoided.
A method to include some content in the main file
by means of conditional processing is described in \secref{sec:conditional}.

%%%%%%%%%%%%%%%%%%%%%%%%%%%%%%%%%%%%%%%%
\paragraph{Page Numbering.}

When only a part of the document is compiled,
the appropriate numbering of pages
(as well as other status parameters)
is determined from the |.aux| files.
The latter contain information from previous passes.
However this information needs to propagate through
all intermediate child documents.
Therefore the page numbering in child documents may well
be inconsistent until the complete document is compiled at least once.

A useful (if unconventional) way to always ensure a consistent
page numbering is to restart the numbering in each child document
and denote the pages by `\textit{child}|.|\textit{page}'
where \textit{child} represents the chapter/section number of the child file.
This can be achieved by the command
|\numberwithin{page}{|\textit{child}|}|
of the \textsf{amsmath} package
where \textit{child} can be |chapter| or |section|
depending on the chosen structuring.
Alternatively, one can modify the macro |\thepage| appropriately
and reset the counter |page| at the start of each child file.

%%%%%%%%%%%%%%%%%%%%%%%%%%%%%%%%%%%%%%%%%%%%%%%%%%%%%%%%%%%%%%%%%%%%%%%%%%%%%%%%
\subsection{Conditional Processing}
\label{sec:conditional}

The package provides a mechanism to compile different versions
of a document. To customise the versions further some conditional processing
can come in handy to distinguish which version is being compiled.
The package provides two macros to describe the compilation context:

%%%%%%%%%%%%%%%%%%%%%%%%%%%%%%%%%%%%%%%%
\DescribeMacro{\ifchilddoc}
The conditional |\ifchilddoc| distinguishes between the compilation of
child documents and the main document:
%
\begin{center}
|\ifchilddoc |\textit{child-code}| |[|\||else |\textit{main-code}]| \||fi|
\end{center}

%%%%%%%%%%%%%%%%%%%%%%%%%%%%%%%%%%%%%%%%
\DescribeMacro{\childdocname}
\DescribeMacro{\childdocjob}
The macro |\childdocname| contains the filename (without extension)
of the main or child file being processed.
Note that |\childdocjob| will always contain the name of the main file.

%%%%%%%%%%%%%%%%%%%%%%%%%%%%%%%%%%%%%%%%
\paragraph{Title Page.}

Conditional processing can be used to include a title or banner page
in the main document when proper precautions are taken.
Importantly, the code in the main file should ensure that the page counter
(as well as other status parameters which are stored in the |.aux| files)
takes the same value after the conditional processing.
Otherwise the page numbers may take divergent values
depending on which part is compiled.

For example, a title page could be declared by:
%
\begin{center}
\begin{tabular}{l}
|\ifchilddoc\||else|\\
|\addtocounter{page}{-1}|\\
\textit{code for title page}\\
|\newpage|\\
|\||fi|
\end{tabular}
\end{center}
%
A banner page for the child documents can be generated by:
%
\begin{center}
\begin{tabular}{l}
|\ifchilddoc|\\
|\addtocounter{page}{-1}|\\
\textit{code for banner page}\\
|\newpage|\\
|\||fi|
\end{tabular}
\end{center}
%
Here one could write a message such as:
\begin{center}
|This is the part \childdocname{} of \childdocjob{}.|
\end{center}

%%%%%%%%%%%%%%%%%%%%%%%%%%%%%%%%%%%%%%%%%%%%%%%%%%%%%%%%%%%%%%%%%%%%%%%%%%%%%%%%
\subsection{Flags}
\label{sec:flags}

The package makes it easy to generate different versions
of the main or child documents.
To this end compilation flags can be defined
and assigned different default values.
They will be particularly useful in conjunction
with the forwarding mechanism described in \secref{sec:forward}.

For example, it may be useful to have a flag |\version|
which can be set to |draft| or |final|.
The document source will contain some conditional code
depending on the value of |\version|.
Suppose further, the flag should default to |final| for the main file
and to |draft| for child files
which is a natural assignment for editing the document.
This is achieved by placing the following code
in the preamble of the main document
(below the |\childdocmain| directive):
%
\begin{center}
\begin{tabular}{l}
|\ifchilddoc|\\
|\providecommand{\version}{draft}|\\
|\||else|\\
|\providecommand{\version}{final}|\\
|\||fi|
\end{tabular}
\end{center}
%
The definition by |\providecommand| makes sure
that previous definitions are not overwritten.
Further statements |\providecommand{\version}{...}|
can thus be added before the above code to override it.

For the main file, one might add a line
(between |\childdocmain| and the above block)
%
\begin{center}
|%\ifchilddoc\||else\providecommand{\version}{draft}\||fi|
\end{center}
%
which can be uncommented to produce a draft version.
Likewise one can add a line to the very top of a child file
(above the |\childdocof{|\textit{main}|}| directive)
%
\begin{center}
|%\providecommand{\version}{final}|
\end{center}
%
which can be uncommented to produce the final version of this child document.

%%%%%%%%%%%%%%%%%%%%%%%%%%%%%%%%%%%%%%%%%%%%%%%%%%%%%%%%%%%%%%%%%%%%%%%%%%%%%%%%
\subsection{Forwarding}
\label{sec:forward}

Different versions of the main or child documents
using compilation flags as described in \secref{sec:flags}
can be (permanently) stored in different files
for convenient compilation, viewing and distribution.
To this end, the package defines a command
to pass on compilation to a different file:

%%%%%%%%%%%%%%%%%%%%%%%%%%%%%%%%%%%%%%%%
\DescribeMacro{\childdocforward}
The command |\childdocforward| redirects processing to
another source file:
%
\begin{center}
\begin{tabular}{l}
|\input{childdoc.def}|\\
|\childdocforward[|\textit{main}|]{|\textit{dest}|}|\\
\end{tabular}
\end{center}
%
The argument \textit{dest} is the destination file
(without extension).
It should be the main file or one of the child files.
Note that further \textsf{childdoc} directives
such as |\childdocof| and |\childdocforward|
in the indicated file will be processed in this form.
The optional argument \textit{main}
passes on directly to the main file \textit{main}
while pretending to compile the child \textit{dest}.
This form behaves as if \textit{dest}
issues |\childdocof{|\textit{main}|}| right away,
and no further \textsf{childdoc} directives will be processed.

%%%%%%%%%%%%%%%%%%%%%%%%%%%%%%%%%%%%%%%%
\DescribeMacro{\...prefix}
In the alternative form |\childdocforwardprefix|,
%
\begin{center}
\begin{tabular}{l}
|\input{childdoc.def}|\\
|\childdocforwardprefix[|\textit{main}|]{|\textit{prefix}|}{|\textit{dest}|}|
\end{tabular}
\end{center}
%
the destination file is determined by a pattern
depending on the current file:
To make this work, the current file must be called
`{\textit{prefix}\hspace{0.2em}\textit{suffix}}'
with \textit{prefix} matching precisely the argument.
Processing is then passed on to the file
`{\textit{dest}\hspace{0.2em}\textit{suffix}}'.
Surely, the same effect is achieved by
directly specifying the
argument `{\textit{dest}\hspace{0.2em}\textit{suffix}}'
in the first form.
However, that requires to set up a different file
for each child. With the alternative form of the command
all these files can have exactly the same content
which simplifies setting them up and maintaining them.

For example, the following file |draft.tex|
with a compilation flag |\version| as described in \secref{sec:flags}
compiles the main document as a draft:
%
\begin{center}
\begin{tabular}{l}
|\def\version{draft}|\\
|\input{childdoc.def}|\\
|\childdocforward{|\textit{main}|}|
\end{tabular}
\end{center}
%
Likewise, the following files |final|\textit{nn}|.tex|
compile the final version of the child document
|child|\textit{nn}|.tex|:
%
\begin{center}
\begin{tabular}{l}
|\def\version{final}|\\
|\input{childdoc.def}|\\
|\childdocforwardprefix{final}{child}|
\end{tabular}
\end{center}
%

Note that when several versions of a main file and/or of each child file
are to be generated, it may be convenient to set up a |Makefile| or
shell script to automatise the process.

%%%%%%%%%%%%%%%%%%%%%%%%%%%%%%%%%%%%%%%%%%%%%%%%%%%%%%%%%%%%%%%%%%%%%%%%%%%%%%%%
\subsection{Command Line Processing}
\label{sec:commandline}

The effect of redirection files can also be achieved by invoking
the \LaTeX{} compiler with a more elaborate command line.
Most conveniently this should be done as part
of a shell script or a |Makefile|.

When using \textsf{childdoc} in the main file, the following
command lines effectively perform a redirection
(note that depending on the shell being used,
backslashes may have to be doubled: `|\|' $\to$ `|\\|'):
%
\begin{center}
|... -jobname "|\textit{target}|" |\\|"|[\textit{flags}]%
|\input{childdoc.def}\childdocforward[|\textit{main}|]{|\textit{dest}|}"|
\end{center}
%
Here \textit{target} is the name of the output file,
\textit{main} is the name of the main file
and \textit{dest} is the name of the main or child file to be processed
(all filenames without extensions).
The optional argument \textit{main} can be omitted
if \textit{main} matches \textit{dest}.
Optionally, compilation \textit{flags} can be defined via |\def| commands.
This command line makes the \TeX{} engine believe
it is compiling the file \textit{target}
whose content is specified as the latter parameter.
The provided code then forwards the processing to
\textit{main} or \textit{dest} as described in \secref{sec:forward}.

%%%%%%%%%%%%%%%%%%%%%%%%%%%%%%%%%%%%%%%%%%%%%%%%%%%%%%%%%%%%%%%%%%%%%%%%%%%%%%%%
\subsection{Include by Input}
\label{sec:input}

Including child documents by |\include| has some restrictions by design.
Most notably, the content of a child document always occupies
its own set of pages; pages cannot be shared between child documents.
Usually, this behaviour makes perfect sense
because each child document contain an essential part of the document.
However, in some situations it may be desirable to compose
a document from a collection of parts
without having mandatory page breaks between then.
For this case, the package
provides a mechanism to include parts
by |\input| which can also be processed individually.
However, by construction this mechanism
requires manual handling of the content to be output.

%%%%%%%%%%%%%%%%%%%%%%%%%%%%%%%%%%%%%%%%
\DescribeMacro{\ifchilddocmanual}
The main file should be prepared as usual, see \secref{sec:include}.
However, the document body must make a distinction
between processing of an individual part and of the main document, e.g.:
%
\begin{center}
\begin{tabular}{l}
|\ifchilddocmanual|\\
|\input{\childdocname}|\\
|\||else|\\
\textit{document body with }|\input{|\textit{part}|}|\\
|\||fi|
\end{tabular}
\end{center}
%
The conditional |\ifchilddocmanual| is true whenever
a part to be included by |\input| is being compiled,
and the name of the part is stored in |\childdocname|.

%%%%%%%%%%%%%%%%%%%%%%%%%%%%%%%%%%%%%%%%
\DescribeMacro{\childdocby}
Each part to be included by |\input| should start with:
%
\begin{center}
\begin{tabular}{l}
|\input{childdoc.def}|\\
|\childdocby{|\textit{main}|}|\\
\end{tabular}
\end{center}
%
The directive |\childdocby| is similar to |\childdocof|
described in \secref{sec:include},
but the subsequent selection of content must be done manually.
To that end, both |\ifchilddoc| and |\ifchilddocmanual|
will be true upon processing of a part,
and the name of the part is stored in |\childdocname|.
Note that |\jobname| will be set to the filename of the current part
so that each part receives an individual |.aux| file
that does not interfere with the |.aux| file(s) of the main document.
This behaviour can be altered by the alternative form
|\childdocby[*]{|\textit{main}|}| (with a non-empty optional argument)
which uses the |.aux| file of the main document
by setting |\jobname| to \textit{main}.

%%%%%%%%%%%%%%%%%%%%%%%%%%%%%%%%%%%%%%%%%%%%%%%%%%%%%%%%%%%%%%%%%%%%%%%%%%%%%%%%
\subsection{Driver Development}
\label{sec:driver}

The \textsf{childdoc} mechanism can also be use for the development
of definition files such as \LaTeX{} styles or classes.
This case differs from the above setup with multiple parts
included by |\include| in that no |\includeonly| should be invoked.
This can be achieved by starting the include file
(before |\ProvidesPackage|) with:
%
\begin{center}
\begin{tabular}{l}
|\input{childdoc.def}|\\
|\childdocforward{|\textit{main}|}|\\
\end{tabular}
\end{center}
%
or alternatively with:
%
\begin{center}
\begin{tabular}{l}
|\input{childdoc.def}|\\
|\childdocby{|\textit{main}|}|\\
\end{tabular}
\end{center}
%
Both forms have slightly different effects as described above.
The main file is prepared as usual, see \secref{sec:include}.

%%%%%%%%%%%%%%%%%%%%%%%%%%%%%%%%%%%%%%%%%%%%%%%%%%%%%%%%%%%%%%%%%%%%%%%%%%%%%%%%
\subsection{Legacy Detection}
\label{sec:detection}

The directive |\childdocmain| in the main file can detect
whether the complete document or merely a child is to be compiled
even without using the directive |\childdocof|.
This method is deprecated because it is less robust
and there is no compelling reason to use it;
it is merely provided for backward compatibility
and it may be removed in future versions.

If the detection mechanism is to be used,
it is mandatory to correctly specify
the filename of the main file as the argument of |\childdocmain|:
%
\begin{center}
\begin{tabular}{l}
|\input{childdoc.def}|\\
|\childdocmain{|\textit{main}|}|\\
\end{tabular}
\end{center}
%
If |\jobname| does not match the argument \textit{main} of |\childdocmain|,
it is assumed that |\jobname| points to the child file to be compiled.
When using |\childdocmain| with the main file specified as argument,
it suffices to start a child file
with just |\input{|\textit{main}|}|
without loading of the package and using |\childdocof|.
If instead all processing is done
with the appropriate \textsf{childdoc} directives,
the argument of \textit{main} of |\childdocmain| can be empty.

An alternative version of the command line processing described
in \secref{sec:commandline} using the detection mechanism reads:
%
\begin{center}
|... -jobname "|\textit{target}|" "|[\textit{flags}]%
[|\def\jobname{|\textit{dest}|}|]|\input{|\textit{main}|}"|
\end{center}

%%%%%%%%%%%%%%%%%%%%%%%%%%%%%%%%%%%%%%%%%%%%%%%%%%%%%%%%%%%%%%%%%%%%%%%%%%%%%%%%
\subsection{Manual Code}
\label{sec:manual}

In case one cannot be certain whether the definitions file |childdoc.def|
is installed on the target \TeX{} distribution
and one prefers not to ship it,
it is conceivable to paste a few relevant commands into the sources.

To that end, drop all statements |\input{childdoc.def}|
and perform the replacements as outlined below.
Instead of |\childdocmain{|\textit{main}|}| add the following code
to the top of the main file:
%
\begin{center}
\begin{tabular}{l}
|\||ifdefined\childdocname\endinput\||fi\newif\ifchilddoc|\\
|\edef\childdocname{\scantokens\expandafter{\jobname\noexpand}}|\\
|\def\childdocmain{|\textit{main}|}\||ifx\childdocmain\childdocname\||else|\\
|\childdoctrue\includeonly{\childdocname}\let\jobname\childdocmain\||fi|\\
\end{tabular}
\end{center}
%
Instead of |\childdocof{|\textit{main}|}| just include the main file
at the top of each child file:
%
\begin{center}
|\input{|\textit{main}|}|
\end{center}
%
A simple redirection |\childdocforward{|\textit{dest}|}| is achieved by:
%
\begin{center}
|\def\jobname{|\textit{dest}|}\input{\jobname}|
\end{center}
%
The redirection with prefix
|\childdocforwardprefix[|\textit{prefix}|]{|\textit{dest}|}|
is accomplished by:
%
\begin{center}
\begin{tabular}{l}
|{\edef\jobname{\scantokens\expandafter{\jobname\noexpand}}|\\
|\def\redirectjob |\textit{prefix}|#1~~~{\gdef\jobname{|\textit{dest}|#1}}|\\
|\expandafter\redirectjob\jobname~~~}\input{\jobname}|
\end{tabular}
\end{center}

In an alternative approach,
child documents can be compiled by a specific command line
without additional code or specific definitions:
%
\begin{center}
|... -jobname "|\textit{target}|" "|[\textit{flags}]%
|\includeonly{|\textit{dest}|}\input{|\textit{main}|}"|
\end{center}
%

%%%%%%%%%%%%%%%%%%%%%%%%%%%%%%%%%%%%%%%%%%%%%%%%%%%%%%%%%%%%%%%%%%%%%%%%%%%%%%%%
%%%%%%%%%%%%%%%%%%%%%%%%%%%%%%%%%%%%%%%%%%%%%%%%%%%%%%%%%%%%%%%%%%%%%%%%%%%%%%%%
\section{Information}

%%%%%%%%%%%%%%%%%%%%%%%%%%%%%%%%%%%%%%%%%%%%%%%%%%%%%%%%%%%%%%%%%%%%%%%%%%%%%%%%
\subsection{Copyright}

Copyright \copyright{} 2017--2018 Niklas Beisert

This work may be distributed and/or modified under the
conditions of the \LaTeX{} Project Public License, either version 1.3
of this license or (at your option) any later version.
The latest version of this license is in
  \url{http://www.latex-project.org/lppl.txt}
and version 1.3 or later is part of all distributions of \LaTeX{}
version 2005/12/01 or later.

This work has the LPPL maintenance status `maintained'.

The Current Maintainer of this work is Niklas Beisert.

This work consists of the files |README.txt|, |childdoc.ins| and |childdoc.dtx|
as well as the derived files |childdoc.def|, |cdocsamp.tex|
with |cdocsch1.tex|, |cdocsch2.tex|, |cdocspt3.tex|, |cdocspt4.tex|,
|cdocsdrf.tex|, |cdocsfn1.tex|, |cdocsfn2.tex|
as well as |childdoc.pdf|.

%%%%%%%%%%%%%%%%%%%%%%%%%%%%%%%%%%%%%%%%%%%%%%%%%%%%%%%%%%%%%%%%%%%%%%%%%%%%%%%%
\subsection{Files and Installation}

The package consists of the files:
%
\begin{center}
\begin{tabular}{ll}
    |README.txt|   & readme file \\
    |childdoc.ins| & installation file \\
    |childdoc.dtx| & source file \\
    |childdoc.def| & definition file \\
    |cdocsamp.tex| & sample main file \\
    |cdocsch1.tex| & sample include file \\
    |cdocsch2.tex| & sample include file \\
    |cdocspt3.tex| & sample part file \\
    |cdocspt4.tex| & sample part file \\
    |cdocsdrf.tex| & sample redirection file \\
    |cdocsfn1.tex| & sample redirection file \\
    |cdocsfn2.tex| & sample redirection file \\
    |childdoc.pdf| & manual
\end{tabular}
\end{center}
%
The distribution consists of the files
|README.txt|, |childdoc.ins| and |childdoc.dtx|.
%
\begin{itemize}
\item
Run (pdf)\LaTeX{} on |childdoc.dtx|
to compile the manual |childdoc.pdf| (this file).
\item
Run \LaTeX{} on |childdoc.ins| to create the definitions file |childdoc.def|
and the sample |cdocsamp.tex| with include files
|cdocsch1.tex|, |cdocsch2.tex|, |cdocspt3.tex|, |cdocspt4.tex|,
|cdocsdrf.tex|, |cdocsfn1.tex|, |cdocsfn2.tex|.
Then copy the file |childdoc.def| to an appropriate directory of your \LaTeX{}
distribution, e.g.\ \textit{texmf-root}|/tex/latex/childdoc|.
\end{itemize}

%%%%%%%%%%%%%%%%%%%%%%%%%%%%%%%%%%%%%%%%%%%%%%%%%%%%%%%%%%%%%%%%%%%%%%%%%%%%%%%%
\subsection{Related CTAN Packages}

There are several other packages which offer a similar functionality:
%
\begin{itemize}
\item
The packages
\href{http://ctan.org/pkg/docmute}{\textsf{docmute}},
\href{http://ctan.org/pkg/includex}{\textsf{includex}} and
\href{http://ctan.org/pkg/standalone}{\textsf{standalone}}
provide commands to include only the document body of
a child file thus allowing both files to be compiled individually.
\item
The packages \href{http://ctan.org/pkg/subdocs}{\textsf{subdocs}}
and \href{http://ctan.org/pkg/subfiles}{\textsf{subfiles}}
provide structures in which the main and child documents can be
encapsulated and allowing them to be compiled individually.
The inclusion mechanism is different from the conventional |\include|.
\item
The package \href{http://ctan.org/pkg/combine}{\textsf{combine}}
is an elaborate solution to combine several documents into one.
\end{itemize}
%
See also the CTAN topic \href{http://ctan.org/topic/subdocs}{\textsf{subdocs}}
for further related packages.
The present package differs from the above solutions in that
a document structure constructed with the conventional |\include| mechanism
just needs two extra commands at the top of every file
such that all constituent files can be compiled individually.

%%%%%%%%%%%%%%%%%%%%%%%%%%%%%%%%%%%%%%%%%%%%%%%%%%%%%%%%%%%%%%%%%%%%%%%%%%%%%%%%
%\subsection{Feature Suggestions}
%
%The following is a list of features which may be useful for future
%versions of this package:
%%
%\begin{itemize}
%\item
%\ldots
%\end{itemize}

%%%%%%%%%%%%%%%%%%%%%%%%%%%%%%%%%%%%%%%%%%%%%%%%%%%%%%%%%%%%%%%%%%%%%%%%%%%%%%%%
\subsection{Revision History}

%%%%%%%%%%%%%%%%%%%%%%%%%%%%%%%%%%%%%%%%
\paragraph{v2.0:} 2018/12/30

\begin{itemize}
\item
immediate forward processing
\item
added |\childdocby| mechanism
\item
manual restructured
\end{itemize}

%%%%%%%%%%%%%%%%%%%%%%%%%%%%%%%%%%%%%%%%
\paragraph{v1.6:} 2018/01/17

\begin{itemize}
\item
application for development of include files
\item
corrections to manual
\end{itemize}

%%%%%%%%%%%%%%%%%%%%%%%%%%%%%%%%%%%%%%%%
\paragraph{v1.5:} 2017/05/21

\begin{itemize}
\item
more complete structuring introduced
\item
|\childdocof| introduced
\item
|\childdoc| renamed to |\childdocmain|
\item
|\childredirect| renamed to |\childdocforward| and |\childdocforwardprefix|
and functionality expanded
\end{itemize}

%%%%%%%%%%%%%%%%%%%%%%%%%%%%%%%%%%%%%%%%
\paragraph{v1.0:} 2017/04/27

\begin{itemize}
\item
manual and install package
\item
first version published on CTAN
\end{itemize}

%%%%%%%%%%%%%%%%%%%%%%%%%%%%%%%%%%%%%%%%
\paragraph{v0.6:} 2017/04/26

\begin{itemize}
\item
redirection mechanism added
\end{itemize}

%%%%%%%%%%%%%%%%%%%%%%%%%%%%%%%%%%%%%%%%
\paragraph{v0.5:} 2017/04/26

\begin{itemize}
\item
functionality in definition file
\end{itemize}


%%%%%%%%%%%%%%%%%%%%%%%%%%%%%%%%%%%%%%%%%%%%%%%%%%%%%%%%%%%%%%%%%%%%%%%%%%%%%%%%
%%%%%%%%%%%%%%%%%%%%%%%%%%%%%%%%%%%%%%%%%%%%%%%%%%%%%%%%%%%%%%%%%%%%%%%%%%%%%%%%
%%%%%%%%%%%%%%%%%%%%%%%%%%%%%%%%%%%%%%%%%%%%%%%%%%%%%%%%%%%%%%%%%%%%%%%%%%%%%%%%
\appendix

\settowidth\MacroIndent{\rmfamily\scriptsize 000\ }

 \DocInput{childdoc.dtx}

\end{document}
%</driver>
% \fi
%
% %%%%%%%%%%%%%%%%%%%%%%%%%%%%%%%%%%%%%%%%%%%%%%%%%%%%%%%%%%%%%%%%%%%%%%%%%%%%%%
% %%%%%%%%%%%%%%%%%%%%%%%%%%%%%%%%%%%%%%%%%%%%%%%%%%%%%%%%%%%%%%%%%%%%%%%%%%%%%%
% \section{Sample}
%\iffalse
%<*samplemain>
%\fi
%
% The following presents a sample document
% with two chapters, two parts, a title page,
% a compile flag as well as three forwarding files to set the flag.
% It consists of eight |.tex| files:
% \begin{center}
% \begin{tabular}{ll}
% |cdocsamp.tex|&main file\\
% |cdocsch1.tex|&include file for chapter 1\\
% |cdocsch2.tex|&include file for chapter 2\\
% |cdocspt3.tex|&include file for part 3\\
% |cdocspt4.tex|&include file for part 4\\
% |cdocsdrf.tex|&forwarding file for main file in draft mode\\
% |cdocsfi1.tex|&forwarding file for final version of chapter 1\\
% |cdocsfi2.tex|&forwarding file for final version of chapter 2\\
% \end{tabular}
% \end{center}
% Each of the eight files can be compiled directly by the \LaTeX{} compiler.
%
% %%%%%%%%%%%%%%%%%%%%%%%%%%%%%%%%%%%%%%
% \paragraph{Main File.}
%
% The main file is called |cdocsamp.tex|.
%
% Load the \textsf{childdoc} definitions and
% declare the filename for the main document:
%    \begin{macrocode}
\input{childdoc.def}
\childdocmain{}
%    \end{macrocode}

% Optional override for |\version| flag:
%    \begin{macrocode}
%%\ifchilddoc\else\providecommand{\version}{draft}\fi
%    \end{macrocode}

% Define the default values for the |\version| flag
% (|final| for the main file and |draft| for childs):
%    \begin{macrocode}
\ifchilddoc
\providecommand{\version}{draft}
\else
\providecommand{\version}{final}
\fi
%    \end{macrocode}

% Load the standard document class:
%    \begin{macrocode}
\documentclass[12pt]{article}
%    \end{macrocode}

% Start the document body:
%    \begin{macrocode}
\begin{document}
%    \end{macrocode}

% Declare a title page.
% Print title, part of document being processed and version flag:
%    \begin{macrocode}
\addtocounter{page}{-1}
\begin{center}
{\LARGE\bfseries{}childdoc example\par}
\vspace{1cm}
\ifchilddoc
\ifchilddocmanual part\else chapter\fi:
`\childdocname' of `\childdocjob'\par
\else
main document: `\childdocjob'\par
\fi
version: \version\par
\end{center}
\newpage
%    \end{macrocode}

% Manually include selected file,
% otherwise process as usual:
%    \begin{macrocode}
\ifchilddocmanual
\section*{part `\childdocname'}
\input{\childdocname}
\else
%    \end{macrocode}

% Include the two chapters:
%    \begin{macrocode}
\include{cdocsch1}
\include{cdocsch2}
%    \end{macrocode}

% Include the two parts unless only chapters should be displayed:
%    \begin{macrocode}
\ifchilddoc\else
\section{part three}
\input{cdocspt3}
\section{part four}
\input{cdocspt4}
\fi
%    \end{macrocode}

% Process as usual until here:
%    \begin{macrocode}
\fi
%    \end{macrocode}

% End of document body:
%    \begin{macrocode}
\end{document}
%    \end{macrocode}
%\iffalse
%</samplemain>
%\fi
%
% %%%%%%%%%%%%%%%%%%%%%%%%%%%%%%%%%%%%%%
% \paragraph{Chapter Include Files.}
%
% The include files are called |cdocsch1.tex| and |cdocsch2.tex|.
%
%\iffalse
%<*samplechap1|samplechap2>
%\fi

% Optional override for |\version| flag:
%    \begin{macrocode}
%%\providecommand{\version}{final}
%    \end{macrocode}

% Include the main document:
%    \begin{macrocode}
\input{childdoc.def}
\childdocof{cdocsamp}
%    \end{macrocode}

%\iffalse
%</samplechap1|samplechap2>
%\fi
%
%\iffalse
%<*samplechap1>
%\fi
% Some text for chapter 1:
%    \begin{macrocode}
\section{one}
some text in chapter one
%    \end{macrocode}

%\iffalse
%</samplechap1>
%\fi
% Some text for chapter 2:
%\iffalse
%<*samplechap2>
%\fi
%    \begin{macrocode}
\section{two}
more text in chapter two
%    \end{macrocode}

%\iffalse
%</samplechap2>
%\fi
%
% %%%%%%%%%%%%%%%%%%%%%%%%%%%%%%%%%%%%%%
% \paragraph{Part Include Files.}
%
% The include files are called |cdocspt3.tex| and |cdocspt4.tex|.
%
%\iffalse
%<*samplepart3|samplepart4>
%\fi

% Optional override for |\version| flag:
%    \begin{macrocode}
%%\providecommand{\version}{final}
%    \end{macrocode}

% Include the main document:
%    \begin{macrocode}
\input{childdoc.def}
\childdocby{cdocsamp}
%    \end{macrocode}

%\iffalse
%</samplepart3|samplepart4>
%\fi
%
%\iffalse
%<*samplepart3>
%\fi
% Some text for part 3:
%    \begin{macrocode}
some text in part three
%    \end{macrocode}

%\iffalse
%</samplepart3>
%\fi
% Some text for part 4:
%\iffalse
%<*samplepart4>
%\fi
%    \begin{macrocode}
more text in part four
%    \end{macrocode}

%\iffalse
%</samplepart4>
%\fi
%
% %%%%%%%%%%%%%%%%%%%%%%%%%%%%%%%%%%%%%%
% \paragraph{Forwarding for a Complete Draft.}
%
% The following forwarding file |cdocsdrf.tex|
% compiles the main document in draft mode:
%\iffalse
%<*sampledraft>
%\fi
%    \begin{macrocode}
\def\version{draft}
\input{childdoc.def}
\childdocforward{cdocsamp}
%    \end{macrocode}

%\iffalse
%</sampledraft>
%\fi
%
% %%%%%%%%%%%%%%%%%%%%%%%%%%%%%%%%%%%%%%
% \paragraph{Forwarding for Final Version of the Chapters.}
%
% The following forwarding files |cdocsfn1.tex| and |cdocsfn2.tex|
% (with identical content)
% compile the final versions of the child documents
% |cdocsch1.tex| and |cdocsch2.tex|, respectively:
%\iffalse
%<*samplefinal>
%\fi
%    \begin{macrocode}
\def\version{final}
\input{childdoc.def}
\childdocforwardprefix[cdocsamp]{cdocsfn}{cdocsch}
%    \end{macrocode}

%\iffalse
%</samplefinal>
%\fi
%
% %%%%%%%%%%%%%%%%%%%%%%%%%%%%%%%%%%%%%%
% \paragraph{Command Line Processing.}
%
% The following three command lines generate the output files
% |cdocscld|, |cdocscl1| and |cdocscl2|
% which should be identical to
% |cdocsdrf|, |cdocsch1| and |cdocsfn2|, respectively:
% \begin{center}
% \begin{tabular}{l}
% |latex -jobname cdocscld \|\\
% |  "\def\version{draft}\input{childdoc.def}\childdocforward{cdocsamp}"|\\
% |latex -jobname cdocscl1 \|\\
% |  "\input{childdoc.def}\childdocforward[cdocsamp]{cdocsch1}"|\\
% |latex -jobname cdocscl2 \|\\
% |  "\def\version{final}\input{childdoc.def}\childdocforward{cdocsch2}"|
% \end{tabular}
% \end{center}
% Note that the trailing backslash on each first line
% merely continues the input to the second line
% (for convenient cut ant paste).
% Furthermore, the command |latex| can be replaced by any
% of its alternative versions such as |pdflatex|.
%
% %%%%%%%%%%%%%%%%%%%%%%%%%%%%%%%%%%%%%%%%%%%%%%%%%%%%%%%%%%%%%%%%%%%%%%%%%%%%%%
% %%%%%%%%%%%%%%%%%%%%%%%%%%%%%%%%%%%%%%%%%%%%%%%%%%%%%%%%%%%%%%%%%%%%%%%%%%%%%%
% \section{Implementation}
%\iffalse
%<*package>
%\fi
%
% This section describes the definitions file |childdoc.def|.

% The definitions cannot be loaded using |\usepackage| or |\RequirePackage|
% which has a mechanism to prevent loading a style file more than once.
% When loading the definitions by means of |\input|
% multiple instances have to be prevented manually:
%\iffalse
%This code needs to be before the `\ProvidesFile' directive
%which is defined at the beginning of this file.
%Therefore it is also placed there and commented out here.
%</package>
%<*discard>
%\fi
%    \begin{macrocode}
\ifdefined\childdocmain\endinput\fi
%    \end{macrocode}
%\iffalse
%</discard>
%<*package>
%\fi
%
% \macro{\ifchilddoc}
% \macro{\ifchilddocmanual}
% The conditional |\ifchilddoc| tells whether a
% child (true) or main (false) document is being compiled.
% The conditional |\ifchilddocmanual| tells whether
% the |\includeonly| mechanism is used (false) or
% the selection of child files must be performed manually (true).
% The definitions initialise to false:
%    \begin{macrocode}
\newif\ifchilddoc
\newif\ifchilddocmanual
%    \end{macrocode}

% \macro{\childdocname}
% \macro{\childdocjob}
% The macro |\childdocname| stores the name of the main document
% to be compiled. The macro |\childdocjob| stores the name of
% the document on which the \LaTeX{} compiler was originally invoked.
% The content of |\jobname| cannot be compared
% to filenames specified in the source due to different catcodes.
% The following code rescans |\jobname|, stores the result
% in |\childdocname| and saves a copy in |\childdocjob|:
%    \begin{macrocode}
\edef\childdocname{\scantokens\expandafter{\jobname\noexpand}}
\let\childdocjob\childdocname
%    \end{macrocode}

% \macro{\childdocdisable}
% The macro |\childdocdisable| prevents the main file
% from being processed more than once.
% At this stage, the main document command |\childdocmain|
% is assumed to be called once again where it should do nothing.
% Any subsequent call to it should prevent
% a secondary processing of the main document
% It overwrites the forwarding commands
% |\childdocof| and |\childdocforward|
% with empty macros to prevent further inclusions of the main document:
%    \begin{macrocode}
\newcommand{\childdocdisable}
{
  \renewcommand{\childdocmain}[1]{\renewcommand{\childdocmain}[1]{\endinput}}
  \renewcommand{\childdocof}[1]{}
  \renewcommand{\childdocby}[2][]{}
  \renewcommand{\childdocforward}[2][]{}
  \renewcommand{\childdocdisable}{}
}
%    \end{macrocode}

% \macro{\childdocmain}
% The macro |\childdocmain| is to be called at the top of the main file
% with nothing or the main filename (without extension) as argument.
% First, it breaks loops.
% If the argument is not empty and does not match |\childdocname|
% (which is set by the first inclusion of |childdoc.def|),
% |\ifchilddoc| is set to true, |\includeonly| is applied to the child file
% and |\jobname| is set to the main file
% (for proper handling of |.aux| files):
%    \begin{macrocode}
\newcommand{\childdocmain}[1]
{
  \childdocdisable\childdocmain{}
  \if?#1?\else
    \begingroup
      \def\childdoctmp{#1}
      \ifx\childdoctmp\childdocname
        \def\childdoctmp{}
      \else
        \def\childdoctmp
        {
          \childdoctrue
          \includeonly{\childdocname}
          \def\childdocjob{#1}
          \def\jobname{#1}
        }
      \fi
      \expandafter
    \endgroup
    \childdoctmp
  \fi
}
%    \end{macrocode}

% \macro{\childdocof}
% The command |\childdocof| redirects
% compilation to the main file |#1|.
%    \begin{macrocode}
\newcommand{\childdocof}[1]
{
  \childdocdisable
  \childdoctrue
  \includeonly{\childdocname}
  \def\jobname{#1}
  \def\childdocjob{#1}
  \input{#1}
}
%    \end{macrocode}

% \macro{\childdocby}
% The command |\childdocby| ....
%    \begin{macrocode}
\newcommand{\childdocby}[2][]
{
  \childdocdisable
  \childdoctrue
  \childdocmanualtrue
  \if?#1?\else
    \def\jobname{#2}
  \fi
  \def\childdocjob{#2}
  \input{#2}
  \endinput
}
%    \end{macrocode}

% \macro{\childdocforward}
% The command |\childdocforward| redirects
% compilation to the main file or
% (if the optional argument is given) a child file.
% Parameters are set as if the main file
% or a child file starting with |\childdocof| was compiled.
% Then compilation is handed over to the main file:
%    \begin{macrocode}
\newcommand{\childdocforward}[2][]
{
  \begingroup
    \if?#1?
      \def\childdoctmp
      {
        \def\childdocname{#2}
        \def\childdocjob{#2}
        \def\jobname{#2}
        \input{#2}
        \endinput
      }
    \else
      \def\childdoctmp
      {
        \childdocdisable
        \def\childdocname{#2}
        \childdoctrue
        \includeonly{#2}
        \def\childdocjob{#1}
        \def\jobname{#1}
        \input{#1}
        \endinput
      }
    \fi
    \expandafter
  \endgroup
  \childdoctmp
}
%    \end{macrocode}

% \macro{\childdocforwardprefix}
% The command |\childdocforwardprefix| redirects
% compilation to the main or a child file by means of a pattern.
% The prefix |#1| in the current filename is replaced by |#2|
% and the suffix of the current filename is kept
% (it is assumed that the filename does not contain the substring `|~~~|'
% which is used as a delimiter).
% Compilation is handed over to the new file by |\childdocforward|:
%    \begin{macrocode}
\newcommand{\childdocforwardprefix}[3][]
{
  \begingroup
    \def\childdocextract #2##1~~~{\def\childdoctmp{\childdocforward[#1]{#3##1}}}
    \expandafter\childdocextract\childdocname~~~
    \expandafter
  \endgroup
  \childdoctmp
}
%    \end{macrocode}

% \macro{\childdoc}
% The deprecated macro |\childdoc| is a legacy version of |\childdocmain|:
%    \begin{macrocode}
\newcommand{\childdoc}{\childdocmain}
%    \end{macrocode}

% \macro{\childdocredirect}
% The deprecated macro |\childdocredirect| is a legacy version
% of |\childdocforward| and |\childdocforwardprefix|:
%    \begin{macrocode}
\newcommand{\childdocredirect}[2][]
{
  \begingroup
    \if?#1?
      \def\childdoctmp{\childdocforward{#2}}
    \else
      \def\childdoctmp{\childdocforwardprefix{#1}{#2}}
    \fi
    \expandafter
  \endgroup
  \childdoctmp
}
%    \end{macrocode}

%\iffalse
%</package>
%\fi
%
\endinput
|\\
|\childdocforward{|\textit{main}|}|
\end{tabular}
\end{center}
%
Likewise, the following files |final|\textit{nn}|.tex|
compile the final version of the child document
|child|\textit{nn}|.tex|:
%
\begin{center}
\begin{tabular}{l}
|\def\version{final}|\\
|% \iffalse
%
% childdoc.dtx Copyright (C) 2017-2018 Niklas Beisert
%
% This work may be distributed and/or modified under the
% conditions of the LaTeX Project Public License, either version 1.3
% of this license or (at your option) any later version.
% The latest version of this license is in
%   http://www.latex-project.org/lppl.txt
% and version 1.3 or later is part of all distributions of LaTeX
% version 2005/12/01 or later.
%
% This work has the LPPL maintenance status `maintained'.
%
% The Current Maintainer of this work is Niklas Beisert.
%
% This work consists of the files childdoc.dtx and childdoc.ins
% and the derived files childdoc.def and cdocsamp.tex with
% cdocsch1.tex, cdocsch2.tex, cdocsdrf.tex, cdocsfn1.tex, cdocsfn2.tex.
%
%<package>\ifdefined\childdocmain\endinput\fi
%<package>\ProvidesFile{childdoc.def}[2018/12/30 v2.0 child document driver]
%<samplemain>\ProvidesFile{cdocsamp.tex}[2018/12/30 v2.0 sample for childdoc]
%<*driver>
%\ProvidesFile{childdoc.drv}[2018/12/30 v2.0 childdoc reference manual file]
\PassOptionsToClass{10pt,a4paper}{article}
\documentclass{ltxdoc}

\usepackage[margin=35mm]{geometry}
\usepackage{hyperref}
\usepackage{hyperxmp}
\usepackage[usenames]{color}

\hypersetup{colorlinks=true}
\hypersetup{pdfstartview=FitH}
\hypersetup{pdfpagemode=UseNone}
\hypersetup{pdfsource={}}
\hypersetup{pdflang={en-UK}}
\hypersetup{pdfcopyright={Copyright 2017-2018 Niklas Beisert.
  This work may be distributed and/or modified under the
  conditions of the LaTeX Project Public License, either version 1.3
  of this license or (at your option) any later version.}}
\hypersetup{pdflicenseurl={http://www.latex-project.org/lppl.txt}}
\hypersetup{pdfcontactaddress={ETH Zurich, ITP, HIT K,
  Wolfgang-Pauli-Strasse 27}}
\hypersetup{pdfcontactpostcode={8093}}
\hypersetup{pdfcontactcity={Zurich}}
\hypersetup{pdfcontactcountry={Switzerland}}
\hypersetup{pdfcontactemail={nbeisert@itp.phys.ethz.ch}}
\hypersetup{pdfcontacturl={http://people.phys.ethz.ch/\xmptilde nbeisert/}}

\newcommand{\secref}[1]{\hyperref[#1]{section \ref*{#1}}}

\parskip1ex
\parindent0pt
\let\olditemize\itemize
\def\itemize{\olditemize\parskip0pt}

\begin{document}

\title{The \textsf{childdoc} Package}
\hypersetup{pdftitle={The childdoc Package}}
\author{Niklas Beisert\\[2ex]
  Institut f\"ur Theoretische Physik\\
  Eidgen\"ossische Technische Hochschule Z\"urich\\
  Wolfgang-Pauli-Strasse 27, 8093 Z\"urich, Switzerland\\[1ex]
  \href{mailto:nbeisert@itp.phys.ethz.ch}
  {\texttt{nbeisert@itp.phys.ethz.ch}}}
\hypersetup{pdfauthor={Niklas Beisert}}
\hypersetup{pdfsubject={Manual for the LaTeX2e Package childdoc}}
\date{30 December 2018, \textsf{v2.0}}
\maketitle

\begin{abstract}\noindent
\textsf{childdoc} is a \LaTeXe{} package
that enables the direct compilation
of document sections included by |\include|
to individual files.
\end{abstract}

\begingroup
\parskip0ex
\tableofcontents
\endgroup

%%%%%%%%%%%%%%%%%%%%%%%%%%%%%%%%%%%%%%%%%%%%%%%%%%%%%%%%%%%%%%%%%%%%%%%%%%%%%%%%
%%%%%%%%%%%%%%%%%%%%%%%%%%%%%%%%%%%%%%%%%%%%%%%%%%%%%%%%%%%%%%%%%%%%%%%%%%%%%%%%
\section{Introduction}

\LaTeX{} provides a mechanism to structure a large document (such as a book)
into a main file and several child files (containing the chapters)
using the |\include| command.
This mechanism is beneficial for documents
which span hundreds of pages in order to
make the source file(s) more manageable.
Moreover, compilation can be restricted to
selected child files by means of the |\includeonly| command.
The latter feature can be used to reduce the compilation time while editing
(this was significantly more useful in the earlier days of \LaTeX{})
or to generate a smaller document which is easier to navigate.
Another application of |\includeonly| is to generate
documents consisting of selected parts of the complete document.

However, there are a few drawbacks of the plain |\include| mechanism:
\begin{itemize}
\item
The child files cannot be compiled on their own,
they can only be compiled via the main file.
A naive editing environment
(such as a text editor with an option
to have the current file processed by \LaTeX)
may require one to switch to the main file before compiling;
attempting to compile the child file produces errors.
\item
The main file must be modified (each time)
to adjust the |\includeonly| command
to the present needs. This easily leaves the main file in a messy state.
\item
The generated document will always carry the filename
of the main document. This is inconvenient if
several child files are to be compiled and
to be kept for distribution.
\end{itemize}

The present package provides a simple interface
to make child files individually compilable by \LaTeX{}.
Compiling a child file then has the same effect as compiling
the main file with an |\includeonly| command
to select the appropriate child.
Moreover the generated document will carry the name of the child
rather than the main file.
This resolves all three above issues.

This feature is meant to make the editing of books,
thesis documents and lecture notes somewhat more convenient.
However, the package can also be used efficiently for
composing a series of documents (such as exercise sheets)
which are typically distributed individually.
It then assists the author in generating the individual documents
(potentially in different versions)
as well as a document containing the collected series.
Another application is in developing style files
or other kinds of included material
where compilation of the style file could redirect
to a sample or test file.

%%%%%%%%%%%%%%%%%%%%%%%%%%%%%%%%%%%%%%%%%%%%%%%%%%%%%%%%%%%%%%%%%%%%%%%%%%%%%%%%
%%%%%%%%%%%%%%%%%%%%%%%%%%%%%%%%%%%%%%%%%%%%%%%%%%%%%%%%%%%%%%%%%%%%%%%%%%%%%%%%
\section{Usage}

First of all, the package \textsf{childdoc} is \emph{not} a standard
\LaTeXe{} |.sty| style file! Therefore it needs to be invoked in
a non-standard way.

%%%%%%%%%%%%%%%%%%%%%%%%%%%%%%%%%%%%%%%%%%%%%%%%%%%%%%%%%%%%%%%%%%%%%%%%%%%%%%%%
\subsection{Included Files}
\label{sec:include}

%%%%%%%%%%%%%%%%%%%%%%%%%%%%%%%%%%%%%%%%
\DescribeMacro{\childdocmain}
To use the package, add the commands
\begin{center}
\begin{tabular}{l}
|\input{childdoc.def}|\\
|\childdocmain{}|\\
\end{tabular}
\end{center}
at the very top of the main \LaTeX{} file,
in particular \emph{before} the |\documentclass| statement!
The argument of |\childdocmain| should be left empty
(but it must be present).

%%%%%%%%%%%%%%%%%%%%%%%%%%%%%%%%%%%%%%%%
\DescribeMacro{\childdocof}
Furthermore, add the commands
\begin{center}
\begin{tabular}{l}
|\input{childdoc.def}|\\
|\childdocof{|\textit{main}|}|\\
\end{tabular}
\end{center}
at the top of every child file \textit{child}
which is included by |\include{|\textit{child}|}|
from within the main file
(or at least for those files to be compiled individually).
The argument \textit{main} must be the filename of the main file.

There are a couple of
considerations in setting up the main and child documents:

%%%%%%%%%%%%%%%%%%%%%%%%%%%%%%%%%%%%%%%%
\paragraph{Restrictions.}

Please note the following restrictions:
\begin{itemize}
\item
|\childdocmain| must be called with one argument \textit{main}
to ensure compatibility with earlier version of the package.
It must either be empty (|\childdocmain{}|)
or precisely match the filename of the main file in which it is specified.
See \secref{sec:detection} for further information.
\item
The filename \textit{main} must be specified without the |.tex| extension.
\item
The filename \textit{main} is case sensitive
(even in case-insensitive file systems)
due to internal string comparison.
\item
The argument \textit{main} should be fully expanded, it cannot be a macro.
\item
Subdirectories and special characters should be avoided in filenames.
\item
The command |\childdocmain{|\textit{main}|}| must be followed by a whitespace.
It should not be followed immediately by another command
or by a comment mark `|%|'.
This is because the \TeX{} parser reads the token immediately following
the argument of |\childdocmain| and puts it
at the beginning of every child section;
however, a white\-space is ignored.
\end{itemize}

%%%%%%%%%%%%%%%%%%%%%%%%%%%%%%%%%%%%%%%%
\paragraph{Content of Main File.}

It is advisable to place all content in the child files included by |\include|.
Any output contained in the main file will appear in all child documents
unless suppressed manually;
it cannot be suppressed automatically by the |\includeonly| directive
and thus should normally be avoided.
A method to include some content in the main file
by means of conditional processing is described in \secref{sec:conditional}.

%%%%%%%%%%%%%%%%%%%%%%%%%%%%%%%%%%%%%%%%
\paragraph{Page Numbering.}

When only a part of the document is compiled,
the appropriate numbering of pages
(as well as other status parameters)
is determined from the |.aux| files.
The latter contain information from previous passes.
However this information needs to propagate through
all intermediate child documents.
Therefore the page numbering in child documents may well
be inconsistent until the complete document is compiled at least once.

A useful (if unconventional) way to always ensure a consistent
page numbering is to restart the numbering in each child document
and denote the pages by `\textit{child}|.|\textit{page}'
where \textit{child} represents the chapter/section number of the child file.
This can be achieved by the command
|\numberwithin{page}{|\textit{child}|}|
of the \textsf{amsmath} package
where \textit{child} can be |chapter| or |section|
depending on the chosen structuring.
Alternatively, one can modify the macro |\thepage| appropriately
and reset the counter |page| at the start of each child file.

%%%%%%%%%%%%%%%%%%%%%%%%%%%%%%%%%%%%%%%%%%%%%%%%%%%%%%%%%%%%%%%%%%%%%%%%%%%%%%%%
\subsection{Conditional Processing}
\label{sec:conditional}

The package provides a mechanism to compile different versions
of a document. To customise the versions further some conditional processing
can come in handy to distinguish which version is being compiled.
The package provides two macros to describe the compilation context:

%%%%%%%%%%%%%%%%%%%%%%%%%%%%%%%%%%%%%%%%
\DescribeMacro{\ifchilddoc}
The conditional |\ifchilddoc| distinguishes between the compilation of
child documents and the main document:
%
\begin{center}
|\ifchilddoc |\textit{child-code}| |[|\||else |\textit{main-code}]| \||fi|
\end{center}

%%%%%%%%%%%%%%%%%%%%%%%%%%%%%%%%%%%%%%%%
\DescribeMacro{\childdocname}
\DescribeMacro{\childdocjob}
The macro |\childdocname| contains the filename (without extension)
of the main or child file being processed.
Note that |\childdocjob| will always contain the name of the main file.

%%%%%%%%%%%%%%%%%%%%%%%%%%%%%%%%%%%%%%%%
\paragraph{Title Page.}

Conditional processing can be used to include a title or banner page
in the main document when proper precautions are taken.
Importantly, the code in the main file should ensure that the page counter
(as well as other status parameters which are stored in the |.aux| files)
takes the same value after the conditional processing.
Otherwise the page numbers may take divergent values
depending on which part is compiled.

For example, a title page could be declared by:
%
\begin{center}
\begin{tabular}{l}
|\ifchilddoc\||else|\\
|\addtocounter{page}{-1}|\\
\textit{code for title page}\\
|\newpage|\\
|\||fi|
\end{tabular}
\end{center}
%
A banner page for the child documents can be generated by:
%
\begin{center}
\begin{tabular}{l}
|\ifchilddoc|\\
|\addtocounter{page}{-1}|\\
\textit{code for banner page}\\
|\newpage|\\
|\||fi|
\end{tabular}
\end{center}
%
Here one could write a message such as:
\begin{center}
|This is the part \childdocname{} of \childdocjob{}.|
\end{center}

%%%%%%%%%%%%%%%%%%%%%%%%%%%%%%%%%%%%%%%%%%%%%%%%%%%%%%%%%%%%%%%%%%%%%%%%%%%%%%%%
\subsection{Flags}
\label{sec:flags}

The package makes it easy to generate different versions
of the main or child documents.
To this end compilation flags can be defined
and assigned different default values.
They will be particularly useful in conjunction
with the forwarding mechanism described in \secref{sec:forward}.

For example, it may be useful to have a flag |\version|
which can be set to |draft| or |final|.
The document source will contain some conditional code
depending on the value of |\version|.
Suppose further, the flag should default to |final| for the main file
and to |draft| for child files
which is a natural assignment for editing the document.
This is achieved by placing the following code
in the preamble of the main document
(below the |\childdocmain| directive):
%
\begin{center}
\begin{tabular}{l}
|\ifchilddoc|\\
|\providecommand{\version}{draft}|\\
|\||else|\\
|\providecommand{\version}{final}|\\
|\||fi|
\end{tabular}
\end{center}
%
The definition by |\providecommand| makes sure
that previous definitions are not overwritten.
Further statements |\providecommand{\version}{...}|
can thus be added before the above code to override it.

For the main file, one might add a line
(between |\childdocmain| and the above block)
%
\begin{center}
|%\ifchilddoc\||else\providecommand{\version}{draft}\||fi|
\end{center}
%
which can be uncommented to produce a draft version.
Likewise one can add a line to the very top of a child file
(above the |\childdocof{|\textit{main}|}| directive)
%
\begin{center}
|%\providecommand{\version}{final}|
\end{center}
%
which can be uncommented to produce the final version of this child document.

%%%%%%%%%%%%%%%%%%%%%%%%%%%%%%%%%%%%%%%%%%%%%%%%%%%%%%%%%%%%%%%%%%%%%%%%%%%%%%%%
\subsection{Forwarding}
\label{sec:forward}

Different versions of the main or child documents
using compilation flags as described in \secref{sec:flags}
can be (permanently) stored in different files
for convenient compilation, viewing and distribution.
To this end, the package defines a command
to pass on compilation to a different file:

%%%%%%%%%%%%%%%%%%%%%%%%%%%%%%%%%%%%%%%%
\DescribeMacro{\childdocforward}
The command |\childdocforward| redirects processing to
another source file:
%
\begin{center}
\begin{tabular}{l}
|\input{childdoc.def}|\\
|\childdocforward[|\textit{main}|]{|\textit{dest}|}|\\
\end{tabular}
\end{center}
%
The argument \textit{dest} is the destination file
(without extension).
It should be the main file or one of the child files.
Note that further \textsf{childdoc} directives
such as |\childdocof| and |\childdocforward|
in the indicated file will be processed in this form.
The optional argument \textit{main}
passes on directly to the main file \textit{main}
while pretending to compile the child \textit{dest}.
This form behaves as if \textit{dest}
issues |\childdocof{|\textit{main}|}| right away,
and no further \textsf{childdoc} directives will be processed.

%%%%%%%%%%%%%%%%%%%%%%%%%%%%%%%%%%%%%%%%
\DescribeMacro{\...prefix}
In the alternative form |\childdocforwardprefix|,
%
\begin{center}
\begin{tabular}{l}
|\input{childdoc.def}|\\
|\childdocforwardprefix[|\textit{main}|]{|\textit{prefix}|}{|\textit{dest}|}|
\end{tabular}
\end{center}
%
the destination file is determined by a pattern
depending on the current file:
To make this work, the current file must be called
`{\textit{prefix}\hspace{0.2em}\textit{suffix}}'
with \textit{prefix} matching precisely the argument.
Processing is then passed on to the file
`{\textit{dest}\hspace{0.2em}\textit{suffix}}'.
Surely, the same effect is achieved by
directly specifying the
argument `{\textit{dest}\hspace{0.2em}\textit{suffix}}'
in the first form.
However, that requires to set up a different file
for each child. With the alternative form of the command
all these files can have exactly the same content
which simplifies setting them up and maintaining them.

For example, the following file |draft.tex|
with a compilation flag |\version| as described in \secref{sec:flags}
compiles the main document as a draft:
%
\begin{center}
\begin{tabular}{l}
|\def\version{draft}|\\
|\input{childdoc.def}|\\
|\childdocforward{|\textit{main}|}|
\end{tabular}
\end{center}
%
Likewise, the following files |final|\textit{nn}|.tex|
compile the final version of the child document
|child|\textit{nn}|.tex|:
%
\begin{center}
\begin{tabular}{l}
|\def\version{final}|\\
|\input{childdoc.def}|\\
|\childdocforwardprefix{final}{child}|
\end{tabular}
\end{center}
%

Note that when several versions of a main file and/or of each child file
are to be generated, it may be convenient to set up a |Makefile| or
shell script to automatise the process.

%%%%%%%%%%%%%%%%%%%%%%%%%%%%%%%%%%%%%%%%%%%%%%%%%%%%%%%%%%%%%%%%%%%%%%%%%%%%%%%%
\subsection{Command Line Processing}
\label{sec:commandline}

The effect of redirection files can also be achieved by invoking
the \LaTeX{} compiler with a more elaborate command line.
Most conveniently this should be done as part
of a shell script or a |Makefile|.

When using \textsf{childdoc} in the main file, the following
command lines effectively perform a redirection
(note that depending on the shell being used,
backslashes may have to be doubled: `|\|' $\to$ `|\\|'):
%
\begin{center}
|... -jobname "|\textit{target}|" |\\|"|[\textit{flags}]%
|\input{childdoc.def}\childdocforward[|\textit{main}|]{|\textit{dest}|}"|
\end{center}
%
Here \textit{target} is the name of the output file,
\textit{main} is the name of the main file
and \textit{dest} is the name of the main or child file to be processed
(all filenames without extensions).
The optional argument \textit{main} can be omitted
if \textit{main} matches \textit{dest}.
Optionally, compilation \textit{flags} can be defined via |\def| commands.
This command line makes the \TeX{} engine believe
it is compiling the file \textit{target}
whose content is specified as the latter parameter.
The provided code then forwards the processing to
\textit{main} or \textit{dest} as described in \secref{sec:forward}.

%%%%%%%%%%%%%%%%%%%%%%%%%%%%%%%%%%%%%%%%%%%%%%%%%%%%%%%%%%%%%%%%%%%%%%%%%%%%%%%%
\subsection{Include by Input}
\label{sec:input}

Including child documents by |\include| has some restrictions by design.
Most notably, the content of a child document always occupies
its own set of pages; pages cannot be shared between child documents.
Usually, this behaviour makes perfect sense
because each child document contain an essential part of the document.
However, in some situations it may be desirable to compose
a document from a collection of parts
without having mandatory page breaks between then.
For this case, the package
provides a mechanism to include parts
by |\input| which can also be processed individually.
However, by construction this mechanism
requires manual handling of the content to be output.

%%%%%%%%%%%%%%%%%%%%%%%%%%%%%%%%%%%%%%%%
\DescribeMacro{\ifchilddocmanual}
The main file should be prepared as usual, see \secref{sec:include}.
However, the document body must make a distinction
between processing of an individual part and of the main document, e.g.:
%
\begin{center}
\begin{tabular}{l}
|\ifchilddocmanual|\\
|\input{\childdocname}|\\
|\||else|\\
\textit{document body with }|\input{|\textit{part}|}|\\
|\||fi|
\end{tabular}
\end{center}
%
The conditional |\ifchilddocmanual| is true whenever
a part to be included by |\input| is being compiled,
and the name of the part is stored in |\childdocname|.

%%%%%%%%%%%%%%%%%%%%%%%%%%%%%%%%%%%%%%%%
\DescribeMacro{\childdocby}
Each part to be included by |\input| should start with:
%
\begin{center}
\begin{tabular}{l}
|\input{childdoc.def}|\\
|\childdocby{|\textit{main}|}|\\
\end{tabular}
\end{center}
%
The directive |\childdocby| is similar to |\childdocof|
described in \secref{sec:include},
but the subsequent selection of content must be done manually.
To that end, both |\ifchilddoc| and |\ifchilddocmanual|
will be true upon processing of a part,
and the name of the part is stored in |\childdocname|.
Note that |\jobname| will be set to the filename of the current part
so that each part receives an individual |.aux| file
that does not interfere with the |.aux| file(s) of the main document.
This behaviour can be altered by the alternative form
|\childdocby[*]{|\textit{main}|}| (with a non-empty optional argument)
which uses the |.aux| file of the main document
by setting |\jobname| to \textit{main}.

%%%%%%%%%%%%%%%%%%%%%%%%%%%%%%%%%%%%%%%%%%%%%%%%%%%%%%%%%%%%%%%%%%%%%%%%%%%%%%%%
\subsection{Driver Development}
\label{sec:driver}

The \textsf{childdoc} mechanism can also be use for the development
of definition files such as \LaTeX{} styles or classes.
This case differs from the above setup with multiple parts
included by |\include| in that no |\includeonly| should be invoked.
This can be achieved by starting the include file
(before |\ProvidesPackage|) with:
%
\begin{center}
\begin{tabular}{l}
|\input{childdoc.def}|\\
|\childdocforward{|\textit{main}|}|\\
\end{tabular}
\end{center}
%
or alternatively with:
%
\begin{center}
\begin{tabular}{l}
|\input{childdoc.def}|\\
|\childdocby{|\textit{main}|}|\\
\end{tabular}
\end{center}
%
Both forms have slightly different effects as described above.
The main file is prepared as usual, see \secref{sec:include}.

%%%%%%%%%%%%%%%%%%%%%%%%%%%%%%%%%%%%%%%%%%%%%%%%%%%%%%%%%%%%%%%%%%%%%%%%%%%%%%%%
\subsection{Legacy Detection}
\label{sec:detection}

The directive |\childdocmain| in the main file can detect
whether the complete document or merely a child is to be compiled
even without using the directive |\childdocof|.
This method is deprecated because it is less robust
and there is no compelling reason to use it;
it is merely provided for backward compatibility
and it may be removed in future versions.

If the detection mechanism is to be used,
it is mandatory to correctly specify
the filename of the main file as the argument of |\childdocmain|:
%
\begin{center}
\begin{tabular}{l}
|\input{childdoc.def}|\\
|\childdocmain{|\textit{main}|}|\\
\end{tabular}
\end{center}
%
If |\jobname| does not match the argument \textit{main} of |\childdocmain|,
it is assumed that |\jobname| points to the child file to be compiled.
When using |\childdocmain| with the main file specified as argument,
it suffices to start a child file
with just |\input{|\textit{main}|}|
without loading of the package and using |\childdocof|.
If instead all processing is done
with the appropriate \textsf{childdoc} directives,
the argument of \textit{main} of |\childdocmain| can be empty.

An alternative version of the command line processing described
in \secref{sec:commandline} using the detection mechanism reads:
%
\begin{center}
|... -jobname "|\textit{target}|" "|[\textit{flags}]%
[|\def\jobname{|\textit{dest}|}|]|\input{|\textit{main}|}"|
\end{center}

%%%%%%%%%%%%%%%%%%%%%%%%%%%%%%%%%%%%%%%%%%%%%%%%%%%%%%%%%%%%%%%%%%%%%%%%%%%%%%%%
\subsection{Manual Code}
\label{sec:manual}

In case one cannot be certain whether the definitions file |childdoc.def|
is installed on the target \TeX{} distribution
and one prefers not to ship it,
it is conceivable to paste a few relevant commands into the sources.

To that end, drop all statements |\input{childdoc.def}|
and perform the replacements as outlined below.
Instead of |\childdocmain{|\textit{main}|}| add the following code
to the top of the main file:
%
\begin{center}
\begin{tabular}{l}
|\||ifdefined\childdocname\endinput\||fi\newif\ifchilddoc|\\
|\edef\childdocname{\scantokens\expandafter{\jobname\noexpand}}|\\
|\def\childdocmain{|\textit{main}|}\||ifx\childdocmain\childdocname\||else|\\
|\childdoctrue\includeonly{\childdocname}\let\jobname\childdocmain\||fi|\\
\end{tabular}
\end{center}
%
Instead of |\childdocof{|\textit{main}|}| just include the main file
at the top of each child file:
%
\begin{center}
|\input{|\textit{main}|}|
\end{center}
%
A simple redirection |\childdocforward{|\textit{dest}|}| is achieved by:
%
\begin{center}
|\def\jobname{|\textit{dest}|}\input{\jobname}|
\end{center}
%
The redirection with prefix
|\childdocforwardprefix[|\textit{prefix}|]{|\textit{dest}|}|
is accomplished by:
%
\begin{center}
\begin{tabular}{l}
|{\edef\jobname{\scantokens\expandafter{\jobname\noexpand}}|\\
|\def\redirectjob |\textit{prefix}|#1~~~{\gdef\jobname{|\textit{dest}|#1}}|\\
|\expandafter\redirectjob\jobname~~~}\input{\jobname}|
\end{tabular}
\end{center}

In an alternative approach,
child documents can be compiled by a specific command line
without additional code or specific definitions:
%
\begin{center}
|... -jobname "|\textit{target}|" "|[\textit{flags}]%
|\includeonly{|\textit{dest}|}\input{|\textit{main}|}"|
\end{center}
%

%%%%%%%%%%%%%%%%%%%%%%%%%%%%%%%%%%%%%%%%%%%%%%%%%%%%%%%%%%%%%%%%%%%%%%%%%%%%%%%%
%%%%%%%%%%%%%%%%%%%%%%%%%%%%%%%%%%%%%%%%%%%%%%%%%%%%%%%%%%%%%%%%%%%%%%%%%%%%%%%%
\section{Information}

%%%%%%%%%%%%%%%%%%%%%%%%%%%%%%%%%%%%%%%%%%%%%%%%%%%%%%%%%%%%%%%%%%%%%%%%%%%%%%%%
\subsection{Copyright}

Copyright \copyright{} 2017--2018 Niklas Beisert

This work may be distributed and/or modified under the
conditions of the \LaTeX{} Project Public License, either version 1.3
of this license or (at your option) any later version.
The latest version of this license is in
  \url{http://www.latex-project.org/lppl.txt}
and version 1.3 or later is part of all distributions of \LaTeX{}
version 2005/12/01 or later.

This work has the LPPL maintenance status `maintained'.

The Current Maintainer of this work is Niklas Beisert.

This work consists of the files |README.txt|, |childdoc.ins| and |childdoc.dtx|
as well as the derived files |childdoc.def|, |cdocsamp.tex|
with |cdocsch1.tex|, |cdocsch2.tex|, |cdocspt3.tex|, |cdocspt4.tex|,
|cdocsdrf.tex|, |cdocsfn1.tex|, |cdocsfn2.tex|
as well as |childdoc.pdf|.

%%%%%%%%%%%%%%%%%%%%%%%%%%%%%%%%%%%%%%%%%%%%%%%%%%%%%%%%%%%%%%%%%%%%%%%%%%%%%%%%
\subsection{Files and Installation}

The package consists of the files:
%
\begin{center}
\begin{tabular}{ll}
    |README.txt|   & readme file \\
    |childdoc.ins| & installation file \\
    |childdoc.dtx| & source file \\
    |childdoc.def| & definition file \\
    |cdocsamp.tex| & sample main file \\
    |cdocsch1.tex| & sample include file \\
    |cdocsch2.tex| & sample include file \\
    |cdocspt3.tex| & sample part file \\
    |cdocspt4.tex| & sample part file \\
    |cdocsdrf.tex| & sample redirection file \\
    |cdocsfn1.tex| & sample redirection file \\
    |cdocsfn2.tex| & sample redirection file \\
    |childdoc.pdf| & manual
\end{tabular}
\end{center}
%
The distribution consists of the files
|README.txt|, |childdoc.ins| and |childdoc.dtx|.
%
\begin{itemize}
\item
Run (pdf)\LaTeX{} on |childdoc.dtx|
to compile the manual |childdoc.pdf| (this file).
\item
Run \LaTeX{} on |childdoc.ins| to create the definitions file |childdoc.def|
and the sample |cdocsamp.tex| with include files
|cdocsch1.tex|, |cdocsch2.tex|, |cdocspt3.tex|, |cdocspt4.tex|,
|cdocsdrf.tex|, |cdocsfn1.tex|, |cdocsfn2.tex|.
Then copy the file |childdoc.def| to an appropriate directory of your \LaTeX{}
distribution, e.g.\ \textit{texmf-root}|/tex/latex/childdoc|.
\end{itemize}

%%%%%%%%%%%%%%%%%%%%%%%%%%%%%%%%%%%%%%%%%%%%%%%%%%%%%%%%%%%%%%%%%%%%%%%%%%%%%%%%
\subsection{Related CTAN Packages}

There are several other packages which offer a similar functionality:
%
\begin{itemize}
\item
The packages
\href{http://ctan.org/pkg/docmute}{\textsf{docmute}},
\href{http://ctan.org/pkg/includex}{\textsf{includex}} and
\href{http://ctan.org/pkg/standalone}{\textsf{standalone}}
provide commands to include only the document body of
a child file thus allowing both files to be compiled individually.
\item
The packages \href{http://ctan.org/pkg/subdocs}{\textsf{subdocs}}
and \href{http://ctan.org/pkg/subfiles}{\textsf{subfiles}}
provide structures in which the main and child documents can be
encapsulated and allowing them to be compiled individually.
The inclusion mechanism is different from the conventional |\include|.
\item
The package \href{http://ctan.org/pkg/combine}{\textsf{combine}}
is an elaborate solution to combine several documents into one.
\end{itemize}
%
See also the CTAN topic \href{http://ctan.org/topic/subdocs}{\textsf{subdocs}}
for further related packages.
The present package differs from the above solutions in that
a document structure constructed with the conventional |\include| mechanism
just needs two extra commands at the top of every file
such that all constituent files can be compiled individually.

%%%%%%%%%%%%%%%%%%%%%%%%%%%%%%%%%%%%%%%%%%%%%%%%%%%%%%%%%%%%%%%%%%%%%%%%%%%%%%%%
%\subsection{Feature Suggestions}
%
%The following is a list of features which may be useful for future
%versions of this package:
%%
%\begin{itemize}
%\item
%\ldots
%\end{itemize}

%%%%%%%%%%%%%%%%%%%%%%%%%%%%%%%%%%%%%%%%%%%%%%%%%%%%%%%%%%%%%%%%%%%%%%%%%%%%%%%%
\subsection{Revision History}

%%%%%%%%%%%%%%%%%%%%%%%%%%%%%%%%%%%%%%%%
\paragraph{v2.0:} 2018/12/30

\begin{itemize}
\item
immediate forward processing
\item
added |\childdocby| mechanism
\item
manual restructured
\end{itemize}

%%%%%%%%%%%%%%%%%%%%%%%%%%%%%%%%%%%%%%%%
\paragraph{v1.6:} 2018/01/17

\begin{itemize}
\item
application for development of include files
\item
corrections to manual
\end{itemize}

%%%%%%%%%%%%%%%%%%%%%%%%%%%%%%%%%%%%%%%%
\paragraph{v1.5:} 2017/05/21

\begin{itemize}
\item
more complete structuring introduced
\item
|\childdocof| introduced
\item
|\childdoc| renamed to |\childdocmain|
\item
|\childredirect| renamed to |\childdocforward| and |\childdocforwardprefix|
and functionality expanded
\end{itemize}

%%%%%%%%%%%%%%%%%%%%%%%%%%%%%%%%%%%%%%%%
\paragraph{v1.0:} 2017/04/27

\begin{itemize}
\item
manual and install package
\item
first version published on CTAN
\end{itemize}

%%%%%%%%%%%%%%%%%%%%%%%%%%%%%%%%%%%%%%%%
\paragraph{v0.6:} 2017/04/26

\begin{itemize}
\item
redirection mechanism added
\end{itemize}

%%%%%%%%%%%%%%%%%%%%%%%%%%%%%%%%%%%%%%%%
\paragraph{v0.5:} 2017/04/26

\begin{itemize}
\item
functionality in definition file
\end{itemize}


%%%%%%%%%%%%%%%%%%%%%%%%%%%%%%%%%%%%%%%%%%%%%%%%%%%%%%%%%%%%%%%%%%%%%%%%%%%%%%%%
%%%%%%%%%%%%%%%%%%%%%%%%%%%%%%%%%%%%%%%%%%%%%%%%%%%%%%%%%%%%%%%%%%%%%%%%%%%%%%%%
%%%%%%%%%%%%%%%%%%%%%%%%%%%%%%%%%%%%%%%%%%%%%%%%%%%%%%%%%%%%%%%%%%%%%%%%%%%%%%%%
\appendix

\settowidth\MacroIndent{\rmfamily\scriptsize 000\ }

 \DocInput{childdoc.dtx}

\end{document}
%</driver>
% \fi
%
% %%%%%%%%%%%%%%%%%%%%%%%%%%%%%%%%%%%%%%%%%%%%%%%%%%%%%%%%%%%%%%%%%%%%%%%%%%%%%%
% %%%%%%%%%%%%%%%%%%%%%%%%%%%%%%%%%%%%%%%%%%%%%%%%%%%%%%%%%%%%%%%%%%%%%%%%%%%%%%
% \section{Sample}
%\iffalse
%<*samplemain>
%\fi
%
% The following presents a sample document
% with two chapters, two parts, a title page,
% a compile flag as well as three forwarding files to set the flag.
% It consists of eight |.tex| files:
% \begin{center}
% \begin{tabular}{ll}
% |cdocsamp.tex|&main file\\
% |cdocsch1.tex|&include file for chapter 1\\
% |cdocsch2.tex|&include file for chapter 2\\
% |cdocspt3.tex|&include file for part 3\\
% |cdocspt4.tex|&include file for part 4\\
% |cdocsdrf.tex|&forwarding file for main file in draft mode\\
% |cdocsfi1.tex|&forwarding file for final version of chapter 1\\
% |cdocsfi2.tex|&forwarding file for final version of chapter 2\\
% \end{tabular}
% \end{center}
% Each of the eight files can be compiled directly by the \LaTeX{} compiler.
%
% %%%%%%%%%%%%%%%%%%%%%%%%%%%%%%%%%%%%%%
% \paragraph{Main File.}
%
% The main file is called |cdocsamp.tex|.
%
% Load the \textsf{childdoc} definitions and
% declare the filename for the main document:
%    \begin{macrocode}
\input{childdoc.def}
\childdocmain{}
%    \end{macrocode}

% Optional override for |\version| flag:
%    \begin{macrocode}
%%\ifchilddoc\else\providecommand{\version}{draft}\fi
%    \end{macrocode}

% Define the default values for the |\version| flag
% (|final| for the main file and |draft| for childs):
%    \begin{macrocode}
\ifchilddoc
\providecommand{\version}{draft}
\else
\providecommand{\version}{final}
\fi
%    \end{macrocode}

% Load the standard document class:
%    \begin{macrocode}
\documentclass[12pt]{article}
%    \end{macrocode}

% Start the document body:
%    \begin{macrocode}
\begin{document}
%    \end{macrocode}

% Declare a title page.
% Print title, part of document being processed and version flag:
%    \begin{macrocode}
\addtocounter{page}{-1}
\begin{center}
{\LARGE\bfseries{}childdoc example\par}
\vspace{1cm}
\ifchilddoc
\ifchilddocmanual part\else chapter\fi:
`\childdocname' of `\childdocjob'\par
\else
main document: `\childdocjob'\par
\fi
version: \version\par
\end{center}
\newpage
%    \end{macrocode}

% Manually include selected file,
% otherwise process as usual:
%    \begin{macrocode}
\ifchilddocmanual
\section*{part `\childdocname'}
\input{\childdocname}
\else
%    \end{macrocode}

% Include the two chapters:
%    \begin{macrocode}
\include{cdocsch1}
\include{cdocsch2}
%    \end{macrocode}

% Include the two parts unless only chapters should be displayed:
%    \begin{macrocode}
\ifchilddoc\else
\section{part three}
\input{cdocspt3}
\section{part four}
\input{cdocspt4}
\fi
%    \end{macrocode}

% Process as usual until here:
%    \begin{macrocode}
\fi
%    \end{macrocode}

% End of document body:
%    \begin{macrocode}
\end{document}
%    \end{macrocode}
%\iffalse
%</samplemain>
%\fi
%
% %%%%%%%%%%%%%%%%%%%%%%%%%%%%%%%%%%%%%%
% \paragraph{Chapter Include Files.}
%
% The include files are called |cdocsch1.tex| and |cdocsch2.tex|.
%
%\iffalse
%<*samplechap1|samplechap2>
%\fi

% Optional override for |\version| flag:
%    \begin{macrocode}
%%\providecommand{\version}{final}
%    \end{macrocode}

% Include the main document:
%    \begin{macrocode}
\input{childdoc.def}
\childdocof{cdocsamp}
%    \end{macrocode}

%\iffalse
%</samplechap1|samplechap2>
%\fi
%
%\iffalse
%<*samplechap1>
%\fi
% Some text for chapter 1:
%    \begin{macrocode}
\section{one}
some text in chapter one
%    \end{macrocode}

%\iffalse
%</samplechap1>
%\fi
% Some text for chapter 2:
%\iffalse
%<*samplechap2>
%\fi
%    \begin{macrocode}
\section{two}
more text in chapter two
%    \end{macrocode}

%\iffalse
%</samplechap2>
%\fi
%
% %%%%%%%%%%%%%%%%%%%%%%%%%%%%%%%%%%%%%%
% \paragraph{Part Include Files.}
%
% The include files are called |cdocspt3.tex| and |cdocspt4.tex|.
%
%\iffalse
%<*samplepart3|samplepart4>
%\fi

% Optional override for |\version| flag:
%    \begin{macrocode}
%%\providecommand{\version}{final}
%    \end{macrocode}

% Include the main document:
%    \begin{macrocode}
\input{childdoc.def}
\childdocby{cdocsamp}
%    \end{macrocode}

%\iffalse
%</samplepart3|samplepart4>
%\fi
%
%\iffalse
%<*samplepart3>
%\fi
% Some text for part 3:
%    \begin{macrocode}
some text in part three
%    \end{macrocode}

%\iffalse
%</samplepart3>
%\fi
% Some text for part 4:
%\iffalse
%<*samplepart4>
%\fi
%    \begin{macrocode}
more text in part four
%    \end{macrocode}

%\iffalse
%</samplepart4>
%\fi
%
% %%%%%%%%%%%%%%%%%%%%%%%%%%%%%%%%%%%%%%
% \paragraph{Forwarding for a Complete Draft.}
%
% The following forwarding file |cdocsdrf.tex|
% compiles the main document in draft mode:
%\iffalse
%<*sampledraft>
%\fi
%    \begin{macrocode}
\def\version{draft}
\input{childdoc.def}
\childdocforward{cdocsamp}
%    \end{macrocode}

%\iffalse
%</sampledraft>
%\fi
%
% %%%%%%%%%%%%%%%%%%%%%%%%%%%%%%%%%%%%%%
% \paragraph{Forwarding for Final Version of the Chapters.}
%
% The following forwarding files |cdocsfn1.tex| and |cdocsfn2.tex|
% (with identical content)
% compile the final versions of the child documents
% |cdocsch1.tex| and |cdocsch2.tex|, respectively:
%\iffalse
%<*samplefinal>
%\fi
%    \begin{macrocode}
\def\version{final}
\input{childdoc.def}
\childdocforwardprefix[cdocsamp]{cdocsfn}{cdocsch}
%    \end{macrocode}

%\iffalse
%</samplefinal>
%\fi
%
% %%%%%%%%%%%%%%%%%%%%%%%%%%%%%%%%%%%%%%
% \paragraph{Command Line Processing.}
%
% The following three command lines generate the output files
% |cdocscld|, |cdocscl1| and |cdocscl2|
% which should be identical to
% |cdocsdrf|, |cdocsch1| and |cdocsfn2|, respectively:
% \begin{center}
% \begin{tabular}{l}
% |latex -jobname cdocscld \|\\
% |  "\def\version{draft}\input{childdoc.def}\childdocforward{cdocsamp}"|\\
% |latex -jobname cdocscl1 \|\\
% |  "\input{childdoc.def}\childdocforward[cdocsamp]{cdocsch1}"|\\
% |latex -jobname cdocscl2 \|\\
% |  "\def\version{final}\input{childdoc.def}\childdocforward{cdocsch2}"|
% \end{tabular}
% \end{center}
% Note that the trailing backslash on each first line
% merely continues the input to the second line
% (for convenient cut ant paste).
% Furthermore, the command |latex| can be replaced by any
% of its alternative versions such as |pdflatex|.
%
% %%%%%%%%%%%%%%%%%%%%%%%%%%%%%%%%%%%%%%%%%%%%%%%%%%%%%%%%%%%%%%%%%%%%%%%%%%%%%%
% %%%%%%%%%%%%%%%%%%%%%%%%%%%%%%%%%%%%%%%%%%%%%%%%%%%%%%%%%%%%%%%%%%%%%%%%%%%%%%
% \section{Implementation}
%\iffalse
%<*package>
%\fi
%
% This section describes the definitions file |childdoc.def|.

% The definitions cannot be loaded using |\usepackage| or |\RequirePackage|
% which has a mechanism to prevent loading a style file more than once.
% When loading the definitions by means of |\input|
% multiple instances have to be prevented manually:
%\iffalse
%This code needs to be before the `\ProvidesFile' directive
%which is defined at the beginning of this file.
%Therefore it is also placed there and commented out here.
%</package>
%<*discard>
%\fi
%    \begin{macrocode}
\ifdefined\childdocmain\endinput\fi
%    \end{macrocode}
%\iffalse
%</discard>
%<*package>
%\fi
%
% \macro{\ifchilddoc}
% \macro{\ifchilddocmanual}
% The conditional |\ifchilddoc| tells whether a
% child (true) or main (false) document is being compiled.
% The conditional |\ifchilddocmanual| tells whether
% the |\includeonly| mechanism is used (false) or
% the selection of child files must be performed manually (true).
% The definitions initialise to false:
%    \begin{macrocode}
\newif\ifchilddoc
\newif\ifchilddocmanual
%    \end{macrocode}

% \macro{\childdocname}
% \macro{\childdocjob}
% The macro |\childdocname| stores the name of the main document
% to be compiled. The macro |\childdocjob| stores the name of
% the document on which the \LaTeX{} compiler was originally invoked.
% The content of |\jobname| cannot be compared
% to filenames specified in the source due to different catcodes.
% The following code rescans |\jobname|, stores the result
% in |\childdocname| and saves a copy in |\childdocjob|:
%    \begin{macrocode}
\edef\childdocname{\scantokens\expandafter{\jobname\noexpand}}
\let\childdocjob\childdocname
%    \end{macrocode}

% \macro{\childdocdisable}
% The macro |\childdocdisable| prevents the main file
% from being processed more than once.
% At this stage, the main document command |\childdocmain|
% is assumed to be called once again where it should do nothing.
% Any subsequent call to it should prevent
% a secondary processing of the main document
% It overwrites the forwarding commands
% |\childdocof| and |\childdocforward|
% with empty macros to prevent further inclusions of the main document:
%    \begin{macrocode}
\newcommand{\childdocdisable}
{
  \renewcommand{\childdocmain}[1]{\renewcommand{\childdocmain}[1]{\endinput}}
  \renewcommand{\childdocof}[1]{}
  \renewcommand{\childdocby}[2][]{}
  \renewcommand{\childdocforward}[2][]{}
  \renewcommand{\childdocdisable}{}
}
%    \end{macrocode}

% \macro{\childdocmain}
% The macro |\childdocmain| is to be called at the top of the main file
% with nothing or the main filename (without extension) as argument.
% First, it breaks loops.
% If the argument is not empty and does not match |\childdocname|
% (which is set by the first inclusion of |childdoc.def|),
% |\ifchilddoc| is set to true, |\includeonly| is applied to the child file
% and |\jobname| is set to the main file
% (for proper handling of |.aux| files):
%    \begin{macrocode}
\newcommand{\childdocmain}[1]
{
  \childdocdisable\childdocmain{}
  \if?#1?\else
    \begingroup
      \def\childdoctmp{#1}
      \ifx\childdoctmp\childdocname
        \def\childdoctmp{}
      \else
        \def\childdoctmp
        {
          \childdoctrue
          \includeonly{\childdocname}
          \def\childdocjob{#1}
          \def\jobname{#1}
        }
      \fi
      \expandafter
    \endgroup
    \childdoctmp
  \fi
}
%    \end{macrocode}

% \macro{\childdocof}
% The command |\childdocof| redirects
% compilation to the main file |#1|.
%    \begin{macrocode}
\newcommand{\childdocof}[1]
{
  \childdocdisable
  \childdoctrue
  \includeonly{\childdocname}
  \def\jobname{#1}
  \def\childdocjob{#1}
  \input{#1}
}
%    \end{macrocode}

% \macro{\childdocby}
% The command |\childdocby| ....
%    \begin{macrocode}
\newcommand{\childdocby}[2][]
{
  \childdocdisable
  \childdoctrue
  \childdocmanualtrue
  \if?#1?\else
    \def\jobname{#2}
  \fi
  \def\childdocjob{#2}
  \input{#2}
  \endinput
}
%    \end{macrocode}

% \macro{\childdocforward}
% The command |\childdocforward| redirects
% compilation to the main file or
% (if the optional argument is given) a child file.
% Parameters are set as if the main file
% or a child file starting with |\childdocof| was compiled.
% Then compilation is handed over to the main file:
%    \begin{macrocode}
\newcommand{\childdocforward}[2][]
{
  \begingroup
    \if?#1?
      \def\childdoctmp
      {
        \def\childdocname{#2}
        \def\childdocjob{#2}
        \def\jobname{#2}
        \input{#2}
        \endinput
      }
    \else
      \def\childdoctmp
      {
        \childdocdisable
        \def\childdocname{#2}
        \childdoctrue
        \includeonly{#2}
        \def\childdocjob{#1}
        \def\jobname{#1}
        \input{#1}
        \endinput
      }
    \fi
    \expandafter
  \endgroup
  \childdoctmp
}
%    \end{macrocode}

% \macro{\childdocforwardprefix}
% The command |\childdocforwardprefix| redirects
% compilation to the main or a child file by means of a pattern.
% The prefix |#1| in the current filename is replaced by |#2|
% and the suffix of the current filename is kept
% (it is assumed that the filename does not contain the substring `|~~~|'
% which is used as a delimiter).
% Compilation is handed over to the new file by |\childdocforward|:
%    \begin{macrocode}
\newcommand{\childdocforwardprefix}[3][]
{
  \begingroup
    \def\childdocextract #2##1~~~{\def\childdoctmp{\childdocforward[#1]{#3##1}}}
    \expandafter\childdocextract\childdocname~~~
    \expandafter
  \endgroup
  \childdoctmp
}
%    \end{macrocode}

% \macro{\childdoc}
% The deprecated macro |\childdoc| is a legacy version of |\childdocmain|:
%    \begin{macrocode}
\newcommand{\childdoc}{\childdocmain}
%    \end{macrocode}

% \macro{\childdocredirect}
% The deprecated macro |\childdocredirect| is a legacy version
% of |\childdocforward| and |\childdocforwardprefix|:
%    \begin{macrocode}
\newcommand{\childdocredirect}[2][]
{
  \begingroup
    \if?#1?
      \def\childdoctmp{\childdocforward{#2}}
    \else
      \def\childdoctmp{\childdocforwardprefix{#1}{#2}}
    \fi
    \expandafter
  \endgroup
  \childdoctmp
}
%    \end{macrocode}

%\iffalse
%</package>
%\fi
%
\endinput
|\\
|\childdocforwardprefix{final}{child}|
\end{tabular}
\end{center}
%

Note that when several versions of a main file and/or of each child file
are to be generated, it may be convenient to set up a |Makefile| or
shell script to automatise the process.

%%%%%%%%%%%%%%%%%%%%%%%%%%%%%%%%%%%%%%%%%%%%%%%%%%%%%%%%%%%%%%%%%%%%%%%%%%%%%%%%
\subsection{Command Line Processing}
\label{sec:commandline}

The effect of redirection files can also be achieved by invoking
the \LaTeX{} compiler with a more elaborate command line.
Most conveniently this should be done as part
of a shell script or a |Makefile|.

When using \textsf{childdoc} in the main file, the following
command lines effectively perform a redirection
(note that depending on the shell being used,
backslashes may have to be doubled: `|\|' $\to$ `|\\|'):
%
\begin{center}
|... -jobname "|\textit{target}|" |\\|"|[\textit{flags}]%
|% \iffalse
%
% childdoc.dtx Copyright (C) 2017-2018 Niklas Beisert
%
% This work may be distributed and/or modified under the
% conditions of the LaTeX Project Public License, either version 1.3
% of this license or (at your option) any later version.
% The latest version of this license is in
%   http://www.latex-project.org/lppl.txt
% and version 1.3 or later is part of all distributions of LaTeX
% version 2005/12/01 or later.
%
% This work has the LPPL maintenance status `maintained'.
%
% The Current Maintainer of this work is Niklas Beisert.
%
% This work consists of the files childdoc.dtx and childdoc.ins
% and the derived files childdoc.def and cdocsamp.tex with
% cdocsch1.tex, cdocsch2.tex, cdocsdrf.tex, cdocsfn1.tex, cdocsfn2.tex.
%
%<package>\ifdefined\childdocmain\endinput\fi
%<package>\ProvidesFile{childdoc.def}[2018/12/30 v2.0 child document driver]
%<samplemain>\ProvidesFile{cdocsamp.tex}[2018/12/30 v2.0 sample for childdoc]
%<*driver>
%\ProvidesFile{childdoc.drv}[2018/12/30 v2.0 childdoc reference manual file]
\PassOptionsToClass{10pt,a4paper}{article}
\documentclass{ltxdoc}

\usepackage[margin=35mm]{geometry}
\usepackage{hyperref}
\usepackage{hyperxmp}
\usepackage[usenames]{color}

\hypersetup{colorlinks=true}
\hypersetup{pdfstartview=FitH}
\hypersetup{pdfpagemode=UseNone}
\hypersetup{pdfsource={}}
\hypersetup{pdflang={en-UK}}
\hypersetup{pdfcopyright={Copyright 2017-2018 Niklas Beisert.
  This work may be distributed and/or modified under the
  conditions of the LaTeX Project Public License, either version 1.3
  of this license or (at your option) any later version.}}
\hypersetup{pdflicenseurl={http://www.latex-project.org/lppl.txt}}
\hypersetup{pdfcontactaddress={ETH Zurich, ITP, HIT K,
  Wolfgang-Pauli-Strasse 27}}
\hypersetup{pdfcontactpostcode={8093}}
\hypersetup{pdfcontactcity={Zurich}}
\hypersetup{pdfcontactcountry={Switzerland}}
\hypersetup{pdfcontactemail={nbeisert@itp.phys.ethz.ch}}
\hypersetup{pdfcontacturl={http://people.phys.ethz.ch/\xmptilde nbeisert/}}

\newcommand{\secref}[1]{\hyperref[#1]{section \ref*{#1}}}

\parskip1ex
\parindent0pt
\let\olditemize\itemize
\def\itemize{\olditemize\parskip0pt}

\begin{document}

\title{The \textsf{childdoc} Package}
\hypersetup{pdftitle={The childdoc Package}}
\author{Niklas Beisert\\[2ex]
  Institut f\"ur Theoretische Physik\\
  Eidgen\"ossische Technische Hochschule Z\"urich\\
  Wolfgang-Pauli-Strasse 27, 8093 Z\"urich, Switzerland\\[1ex]
  \href{mailto:nbeisert@itp.phys.ethz.ch}
  {\texttt{nbeisert@itp.phys.ethz.ch}}}
\hypersetup{pdfauthor={Niklas Beisert}}
\hypersetup{pdfsubject={Manual for the LaTeX2e Package childdoc}}
\date{30 December 2018, \textsf{v2.0}}
\maketitle

\begin{abstract}\noindent
\textsf{childdoc} is a \LaTeXe{} package
that enables the direct compilation
of document sections included by |\include|
to individual files.
\end{abstract}

\begingroup
\parskip0ex
\tableofcontents
\endgroup

%%%%%%%%%%%%%%%%%%%%%%%%%%%%%%%%%%%%%%%%%%%%%%%%%%%%%%%%%%%%%%%%%%%%%%%%%%%%%%%%
%%%%%%%%%%%%%%%%%%%%%%%%%%%%%%%%%%%%%%%%%%%%%%%%%%%%%%%%%%%%%%%%%%%%%%%%%%%%%%%%
\section{Introduction}

\LaTeX{} provides a mechanism to structure a large document (such as a book)
into a main file and several child files (containing the chapters)
using the |\include| command.
This mechanism is beneficial for documents
which span hundreds of pages in order to
make the source file(s) more manageable.
Moreover, compilation can be restricted to
selected child files by means of the |\includeonly| command.
The latter feature can be used to reduce the compilation time while editing
(this was significantly more useful in the earlier days of \LaTeX{})
or to generate a smaller document which is easier to navigate.
Another application of |\includeonly| is to generate
documents consisting of selected parts of the complete document.

However, there are a few drawbacks of the plain |\include| mechanism:
\begin{itemize}
\item
The child files cannot be compiled on their own,
they can only be compiled via the main file.
A naive editing environment
(such as a text editor with an option
to have the current file processed by \LaTeX)
may require one to switch to the main file before compiling;
attempting to compile the child file produces errors.
\item
The main file must be modified (each time)
to adjust the |\includeonly| command
to the present needs. This easily leaves the main file in a messy state.
\item
The generated document will always carry the filename
of the main document. This is inconvenient if
several child files are to be compiled and
to be kept for distribution.
\end{itemize}

The present package provides a simple interface
to make child files individually compilable by \LaTeX{}.
Compiling a child file then has the same effect as compiling
the main file with an |\includeonly| command
to select the appropriate child.
Moreover the generated document will carry the name of the child
rather than the main file.
This resolves all three above issues.

This feature is meant to make the editing of books,
thesis documents and lecture notes somewhat more convenient.
However, the package can also be used efficiently for
composing a series of documents (such as exercise sheets)
which are typically distributed individually.
It then assists the author in generating the individual documents
(potentially in different versions)
as well as a document containing the collected series.
Another application is in developing style files
or other kinds of included material
where compilation of the style file could redirect
to a sample or test file.

%%%%%%%%%%%%%%%%%%%%%%%%%%%%%%%%%%%%%%%%%%%%%%%%%%%%%%%%%%%%%%%%%%%%%%%%%%%%%%%%
%%%%%%%%%%%%%%%%%%%%%%%%%%%%%%%%%%%%%%%%%%%%%%%%%%%%%%%%%%%%%%%%%%%%%%%%%%%%%%%%
\section{Usage}

First of all, the package \textsf{childdoc} is \emph{not} a standard
\LaTeXe{} |.sty| style file! Therefore it needs to be invoked in
a non-standard way.

%%%%%%%%%%%%%%%%%%%%%%%%%%%%%%%%%%%%%%%%%%%%%%%%%%%%%%%%%%%%%%%%%%%%%%%%%%%%%%%%
\subsection{Included Files}
\label{sec:include}

%%%%%%%%%%%%%%%%%%%%%%%%%%%%%%%%%%%%%%%%
\DescribeMacro{\childdocmain}
To use the package, add the commands
\begin{center}
\begin{tabular}{l}
|\input{childdoc.def}|\\
|\childdocmain{}|\\
\end{tabular}
\end{center}
at the very top of the main \LaTeX{} file,
in particular \emph{before} the |\documentclass| statement!
The argument of |\childdocmain| should be left empty
(but it must be present).

%%%%%%%%%%%%%%%%%%%%%%%%%%%%%%%%%%%%%%%%
\DescribeMacro{\childdocof}
Furthermore, add the commands
\begin{center}
\begin{tabular}{l}
|\input{childdoc.def}|\\
|\childdocof{|\textit{main}|}|\\
\end{tabular}
\end{center}
at the top of every child file \textit{child}
which is included by |\include{|\textit{child}|}|
from within the main file
(or at least for those files to be compiled individually).
The argument \textit{main} must be the filename of the main file.

There are a couple of
considerations in setting up the main and child documents:

%%%%%%%%%%%%%%%%%%%%%%%%%%%%%%%%%%%%%%%%
\paragraph{Restrictions.}

Please note the following restrictions:
\begin{itemize}
\item
|\childdocmain| must be called with one argument \textit{main}
to ensure compatibility with earlier version of the package.
It must either be empty (|\childdocmain{}|)
or precisely match the filename of the main file in which it is specified.
See \secref{sec:detection} for further information.
\item
The filename \textit{main} must be specified without the |.tex| extension.
\item
The filename \textit{main} is case sensitive
(even in case-insensitive file systems)
due to internal string comparison.
\item
The argument \textit{main} should be fully expanded, it cannot be a macro.
\item
Subdirectories and special characters should be avoided in filenames.
\item
The command |\childdocmain{|\textit{main}|}| must be followed by a whitespace.
It should not be followed immediately by another command
or by a comment mark `|%|'.
This is because the \TeX{} parser reads the token immediately following
the argument of |\childdocmain| and puts it
at the beginning of every child section;
however, a white\-space is ignored.
\end{itemize}

%%%%%%%%%%%%%%%%%%%%%%%%%%%%%%%%%%%%%%%%
\paragraph{Content of Main File.}

It is advisable to place all content in the child files included by |\include|.
Any output contained in the main file will appear in all child documents
unless suppressed manually;
it cannot be suppressed automatically by the |\includeonly| directive
and thus should normally be avoided.
A method to include some content in the main file
by means of conditional processing is described in \secref{sec:conditional}.

%%%%%%%%%%%%%%%%%%%%%%%%%%%%%%%%%%%%%%%%
\paragraph{Page Numbering.}

When only a part of the document is compiled,
the appropriate numbering of pages
(as well as other status parameters)
is determined from the |.aux| files.
The latter contain information from previous passes.
However this information needs to propagate through
all intermediate child documents.
Therefore the page numbering in child documents may well
be inconsistent until the complete document is compiled at least once.

A useful (if unconventional) way to always ensure a consistent
page numbering is to restart the numbering in each child document
and denote the pages by `\textit{child}|.|\textit{page}'
where \textit{child} represents the chapter/section number of the child file.
This can be achieved by the command
|\numberwithin{page}{|\textit{child}|}|
of the \textsf{amsmath} package
where \textit{child} can be |chapter| or |section|
depending on the chosen structuring.
Alternatively, one can modify the macro |\thepage| appropriately
and reset the counter |page| at the start of each child file.

%%%%%%%%%%%%%%%%%%%%%%%%%%%%%%%%%%%%%%%%%%%%%%%%%%%%%%%%%%%%%%%%%%%%%%%%%%%%%%%%
\subsection{Conditional Processing}
\label{sec:conditional}

The package provides a mechanism to compile different versions
of a document. To customise the versions further some conditional processing
can come in handy to distinguish which version is being compiled.
The package provides two macros to describe the compilation context:

%%%%%%%%%%%%%%%%%%%%%%%%%%%%%%%%%%%%%%%%
\DescribeMacro{\ifchilddoc}
The conditional |\ifchilddoc| distinguishes between the compilation of
child documents and the main document:
%
\begin{center}
|\ifchilddoc |\textit{child-code}| |[|\||else |\textit{main-code}]| \||fi|
\end{center}

%%%%%%%%%%%%%%%%%%%%%%%%%%%%%%%%%%%%%%%%
\DescribeMacro{\childdocname}
\DescribeMacro{\childdocjob}
The macro |\childdocname| contains the filename (without extension)
of the main or child file being processed.
Note that |\childdocjob| will always contain the name of the main file.

%%%%%%%%%%%%%%%%%%%%%%%%%%%%%%%%%%%%%%%%
\paragraph{Title Page.}

Conditional processing can be used to include a title or banner page
in the main document when proper precautions are taken.
Importantly, the code in the main file should ensure that the page counter
(as well as other status parameters which are stored in the |.aux| files)
takes the same value after the conditional processing.
Otherwise the page numbers may take divergent values
depending on which part is compiled.

For example, a title page could be declared by:
%
\begin{center}
\begin{tabular}{l}
|\ifchilddoc\||else|\\
|\addtocounter{page}{-1}|\\
\textit{code for title page}\\
|\newpage|\\
|\||fi|
\end{tabular}
\end{center}
%
A banner page for the child documents can be generated by:
%
\begin{center}
\begin{tabular}{l}
|\ifchilddoc|\\
|\addtocounter{page}{-1}|\\
\textit{code for banner page}\\
|\newpage|\\
|\||fi|
\end{tabular}
\end{center}
%
Here one could write a message such as:
\begin{center}
|This is the part \childdocname{} of \childdocjob{}.|
\end{center}

%%%%%%%%%%%%%%%%%%%%%%%%%%%%%%%%%%%%%%%%%%%%%%%%%%%%%%%%%%%%%%%%%%%%%%%%%%%%%%%%
\subsection{Flags}
\label{sec:flags}

The package makes it easy to generate different versions
of the main or child documents.
To this end compilation flags can be defined
and assigned different default values.
They will be particularly useful in conjunction
with the forwarding mechanism described in \secref{sec:forward}.

For example, it may be useful to have a flag |\version|
which can be set to |draft| or |final|.
The document source will contain some conditional code
depending on the value of |\version|.
Suppose further, the flag should default to |final| for the main file
and to |draft| for child files
which is a natural assignment for editing the document.
This is achieved by placing the following code
in the preamble of the main document
(below the |\childdocmain| directive):
%
\begin{center}
\begin{tabular}{l}
|\ifchilddoc|\\
|\providecommand{\version}{draft}|\\
|\||else|\\
|\providecommand{\version}{final}|\\
|\||fi|
\end{tabular}
\end{center}
%
The definition by |\providecommand| makes sure
that previous definitions are not overwritten.
Further statements |\providecommand{\version}{...}|
can thus be added before the above code to override it.

For the main file, one might add a line
(between |\childdocmain| and the above block)
%
\begin{center}
|%\ifchilddoc\||else\providecommand{\version}{draft}\||fi|
\end{center}
%
which can be uncommented to produce a draft version.
Likewise one can add a line to the very top of a child file
(above the |\childdocof{|\textit{main}|}| directive)
%
\begin{center}
|%\providecommand{\version}{final}|
\end{center}
%
which can be uncommented to produce the final version of this child document.

%%%%%%%%%%%%%%%%%%%%%%%%%%%%%%%%%%%%%%%%%%%%%%%%%%%%%%%%%%%%%%%%%%%%%%%%%%%%%%%%
\subsection{Forwarding}
\label{sec:forward}

Different versions of the main or child documents
using compilation flags as described in \secref{sec:flags}
can be (permanently) stored in different files
for convenient compilation, viewing and distribution.
To this end, the package defines a command
to pass on compilation to a different file:

%%%%%%%%%%%%%%%%%%%%%%%%%%%%%%%%%%%%%%%%
\DescribeMacro{\childdocforward}
The command |\childdocforward| redirects processing to
another source file:
%
\begin{center}
\begin{tabular}{l}
|\input{childdoc.def}|\\
|\childdocforward[|\textit{main}|]{|\textit{dest}|}|\\
\end{tabular}
\end{center}
%
The argument \textit{dest} is the destination file
(without extension).
It should be the main file or one of the child files.
Note that further \textsf{childdoc} directives
such as |\childdocof| and |\childdocforward|
in the indicated file will be processed in this form.
The optional argument \textit{main}
passes on directly to the main file \textit{main}
while pretending to compile the child \textit{dest}.
This form behaves as if \textit{dest}
issues |\childdocof{|\textit{main}|}| right away,
and no further \textsf{childdoc} directives will be processed.

%%%%%%%%%%%%%%%%%%%%%%%%%%%%%%%%%%%%%%%%
\DescribeMacro{\...prefix}
In the alternative form |\childdocforwardprefix|,
%
\begin{center}
\begin{tabular}{l}
|\input{childdoc.def}|\\
|\childdocforwardprefix[|\textit{main}|]{|\textit{prefix}|}{|\textit{dest}|}|
\end{tabular}
\end{center}
%
the destination file is determined by a pattern
depending on the current file:
To make this work, the current file must be called
`{\textit{prefix}\hspace{0.2em}\textit{suffix}}'
with \textit{prefix} matching precisely the argument.
Processing is then passed on to the file
`{\textit{dest}\hspace{0.2em}\textit{suffix}}'.
Surely, the same effect is achieved by
directly specifying the
argument `{\textit{dest}\hspace{0.2em}\textit{suffix}}'
in the first form.
However, that requires to set up a different file
for each child. With the alternative form of the command
all these files can have exactly the same content
which simplifies setting them up and maintaining them.

For example, the following file |draft.tex|
with a compilation flag |\version| as described in \secref{sec:flags}
compiles the main document as a draft:
%
\begin{center}
\begin{tabular}{l}
|\def\version{draft}|\\
|\input{childdoc.def}|\\
|\childdocforward{|\textit{main}|}|
\end{tabular}
\end{center}
%
Likewise, the following files |final|\textit{nn}|.tex|
compile the final version of the child document
|child|\textit{nn}|.tex|:
%
\begin{center}
\begin{tabular}{l}
|\def\version{final}|\\
|\input{childdoc.def}|\\
|\childdocforwardprefix{final}{child}|
\end{tabular}
\end{center}
%

Note that when several versions of a main file and/or of each child file
are to be generated, it may be convenient to set up a |Makefile| or
shell script to automatise the process.

%%%%%%%%%%%%%%%%%%%%%%%%%%%%%%%%%%%%%%%%%%%%%%%%%%%%%%%%%%%%%%%%%%%%%%%%%%%%%%%%
\subsection{Command Line Processing}
\label{sec:commandline}

The effect of redirection files can also be achieved by invoking
the \LaTeX{} compiler with a more elaborate command line.
Most conveniently this should be done as part
of a shell script or a |Makefile|.

When using \textsf{childdoc} in the main file, the following
command lines effectively perform a redirection
(note that depending on the shell being used,
backslashes may have to be doubled: `|\|' $\to$ `|\\|'):
%
\begin{center}
|... -jobname "|\textit{target}|" |\\|"|[\textit{flags}]%
|\input{childdoc.def}\childdocforward[|\textit{main}|]{|\textit{dest}|}"|
\end{center}
%
Here \textit{target} is the name of the output file,
\textit{main} is the name of the main file
and \textit{dest} is the name of the main or child file to be processed
(all filenames without extensions).
The optional argument \textit{main} can be omitted
if \textit{main} matches \textit{dest}.
Optionally, compilation \textit{flags} can be defined via |\def| commands.
This command line makes the \TeX{} engine believe
it is compiling the file \textit{target}
whose content is specified as the latter parameter.
The provided code then forwards the processing to
\textit{main} or \textit{dest} as described in \secref{sec:forward}.

%%%%%%%%%%%%%%%%%%%%%%%%%%%%%%%%%%%%%%%%%%%%%%%%%%%%%%%%%%%%%%%%%%%%%%%%%%%%%%%%
\subsection{Include by Input}
\label{sec:input}

Including child documents by |\include| has some restrictions by design.
Most notably, the content of a child document always occupies
its own set of pages; pages cannot be shared between child documents.
Usually, this behaviour makes perfect sense
because each child document contain an essential part of the document.
However, in some situations it may be desirable to compose
a document from a collection of parts
without having mandatory page breaks between then.
For this case, the package
provides a mechanism to include parts
by |\input| which can also be processed individually.
However, by construction this mechanism
requires manual handling of the content to be output.

%%%%%%%%%%%%%%%%%%%%%%%%%%%%%%%%%%%%%%%%
\DescribeMacro{\ifchilddocmanual}
The main file should be prepared as usual, see \secref{sec:include}.
However, the document body must make a distinction
between processing of an individual part and of the main document, e.g.:
%
\begin{center}
\begin{tabular}{l}
|\ifchilddocmanual|\\
|\input{\childdocname}|\\
|\||else|\\
\textit{document body with }|\input{|\textit{part}|}|\\
|\||fi|
\end{tabular}
\end{center}
%
The conditional |\ifchilddocmanual| is true whenever
a part to be included by |\input| is being compiled,
and the name of the part is stored in |\childdocname|.

%%%%%%%%%%%%%%%%%%%%%%%%%%%%%%%%%%%%%%%%
\DescribeMacro{\childdocby}
Each part to be included by |\input| should start with:
%
\begin{center}
\begin{tabular}{l}
|\input{childdoc.def}|\\
|\childdocby{|\textit{main}|}|\\
\end{tabular}
\end{center}
%
The directive |\childdocby| is similar to |\childdocof|
described in \secref{sec:include},
but the subsequent selection of content must be done manually.
To that end, both |\ifchilddoc| and |\ifchilddocmanual|
will be true upon processing of a part,
and the name of the part is stored in |\childdocname|.
Note that |\jobname| will be set to the filename of the current part
so that each part receives an individual |.aux| file
that does not interfere with the |.aux| file(s) of the main document.
This behaviour can be altered by the alternative form
|\childdocby[*]{|\textit{main}|}| (with a non-empty optional argument)
which uses the |.aux| file of the main document
by setting |\jobname| to \textit{main}.

%%%%%%%%%%%%%%%%%%%%%%%%%%%%%%%%%%%%%%%%%%%%%%%%%%%%%%%%%%%%%%%%%%%%%%%%%%%%%%%%
\subsection{Driver Development}
\label{sec:driver}

The \textsf{childdoc} mechanism can also be use for the development
of definition files such as \LaTeX{} styles or classes.
This case differs from the above setup with multiple parts
included by |\include| in that no |\includeonly| should be invoked.
This can be achieved by starting the include file
(before |\ProvidesPackage|) with:
%
\begin{center}
\begin{tabular}{l}
|\input{childdoc.def}|\\
|\childdocforward{|\textit{main}|}|\\
\end{tabular}
\end{center}
%
or alternatively with:
%
\begin{center}
\begin{tabular}{l}
|\input{childdoc.def}|\\
|\childdocby{|\textit{main}|}|\\
\end{tabular}
\end{center}
%
Both forms have slightly different effects as described above.
The main file is prepared as usual, see \secref{sec:include}.

%%%%%%%%%%%%%%%%%%%%%%%%%%%%%%%%%%%%%%%%%%%%%%%%%%%%%%%%%%%%%%%%%%%%%%%%%%%%%%%%
\subsection{Legacy Detection}
\label{sec:detection}

The directive |\childdocmain| in the main file can detect
whether the complete document or merely a child is to be compiled
even without using the directive |\childdocof|.
This method is deprecated because it is less robust
and there is no compelling reason to use it;
it is merely provided for backward compatibility
and it may be removed in future versions.

If the detection mechanism is to be used,
it is mandatory to correctly specify
the filename of the main file as the argument of |\childdocmain|:
%
\begin{center}
\begin{tabular}{l}
|\input{childdoc.def}|\\
|\childdocmain{|\textit{main}|}|\\
\end{tabular}
\end{center}
%
If |\jobname| does not match the argument \textit{main} of |\childdocmain|,
it is assumed that |\jobname| points to the child file to be compiled.
When using |\childdocmain| with the main file specified as argument,
it suffices to start a child file
with just |\input{|\textit{main}|}|
without loading of the package and using |\childdocof|.
If instead all processing is done
with the appropriate \textsf{childdoc} directives,
the argument of \textit{main} of |\childdocmain| can be empty.

An alternative version of the command line processing described
in \secref{sec:commandline} using the detection mechanism reads:
%
\begin{center}
|... -jobname "|\textit{target}|" "|[\textit{flags}]%
[|\def\jobname{|\textit{dest}|}|]|\input{|\textit{main}|}"|
\end{center}

%%%%%%%%%%%%%%%%%%%%%%%%%%%%%%%%%%%%%%%%%%%%%%%%%%%%%%%%%%%%%%%%%%%%%%%%%%%%%%%%
\subsection{Manual Code}
\label{sec:manual}

In case one cannot be certain whether the definitions file |childdoc.def|
is installed on the target \TeX{} distribution
and one prefers not to ship it,
it is conceivable to paste a few relevant commands into the sources.

To that end, drop all statements |\input{childdoc.def}|
and perform the replacements as outlined below.
Instead of |\childdocmain{|\textit{main}|}| add the following code
to the top of the main file:
%
\begin{center}
\begin{tabular}{l}
|\||ifdefined\childdocname\endinput\||fi\newif\ifchilddoc|\\
|\edef\childdocname{\scantokens\expandafter{\jobname\noexpand}}|\\
|\def\childdocmain{|\textit{main}|}\||ifx\childdocmain\childdocname\||else|\\
|\childdoctrue\includeonly{\childdocname}\let\jobname\childdocmain\||fi|\\
\end{tabular}
\end{center}
%
Instead of |\childdocof{|\textit{main}|}| just include the main file
at the top of each child file:
%
\begin{center}
|\input{|\textit{main}|}|
\end{center}
%
A simple redirection |\childdocforward{|\textit{dest}|}| is achieved by:
%
\begin{center}
|\def\jobname{|\textit{dest}|}\input{\jobname}|
\end{center}
%
The redirection with prefix
|\childdocforwardprefix[|\textit{prefix}|]{|\textit{dest}|}|
is accomplished by:
%
\begin{center}
\begin{tabular}{l}
|{\edef\jobname{\scantokens\expandafter{\jobname\noexpand}}|\\
|\def\redirectjob |\textit{prefix}|#1~~~{\gdef\jobname{|\textit{dest}|#1}}|\\
|\expandafter\redirectjob\jobname~~~}\input{\jobname}|
\end{tabular}
\end{center}

In an alternative approach,
child documents can be compiled by a specific command line
without additional code or specific definitions:
%
\begin{center}
|... -jobname "|\textit{target}|" "|[\textit{flags}]%
|\includeonly{|\textit{dest}|}\input{|\textit{main}|}"|
\end{center}
%

%%%%%%%%%%%%%%%%%%%%%%%%%%%%%%%%%%%%%%%%%%%%%%%%%%%%%%%%%%%%%%%%%%%%%%%%%%%%%%%%
%%%%%%%%%%%%%%%%%%%%%%%%%%%%%%%%%%%%%%%%%%%%%%%%%%%%%%%%%%%%%%%%%%%%%%%%%%%%%%%%
\section{Information}

%%%%%%%%%%%%%%%%%%%%%%%%%%%%%%%%%%%%%%%%%%%%%%%%%%%%%%%%%%%%%%%%%%%%%%%%%%%%%%%%
\subsection{Copyright}

Copyright \copyright{} 2017--2018 Niklas Beisert

This work may be distributed and/or modified under the
conditions of the \LaTeX{} Project Public License, either version 1.3
of this license or (at your option) any later version.
The latest version of this license is in
  \url{http://www.latex-project.org/lppl.txt}
and version 1.3 or later is part of all distributions of \LaTeX{}
version 2005/12/01 or later.

This work has the LPPL maintenance status `maintained'.

The Current Maintainer of this work is Niklas Beisert.

This work consists of the files |README.txt|, |childdoc.ins| and |childdoc.dtx|
as well as the derived files |childdoc.def|, |cdocsamp.tex|
with |cdocsch1.tex|, |cdocsch2.tex|, |cdocspt3.tex|, |cdocspt4.tex|,
|cdocsdrf.tex|, |cdocsfn1.tex|, |cdocsfn2.tex|
as well as |childdoc.pdf|.

%%%%%%%%%%%%%%%%%%%%%%%%%%%%%%%%%%%%%%%%%%%%%%%%%%%%%%%%%%%%%%%%%%%%%%%%%%%%%%%%
\subsection{Files and Installation}

The package consists of the files:
%
\begin{center}
\begin{tabular}{ll}
    |README.txt|   & readme file \\
    |childdoc.ins| & installation file \\
    |childdoc.dtx| & source file \\
    |childdoc.def| & definition file \\
    |cdocsamp.tex| & sample main file \\
    |cdocsch1.tex| & sample include file \\
    |cdocsch2.tex| & sample include file \\
    |cdocspt3.tex| & sample part file \\
    |cdocspt4.tex| & sample part file \\
    |cdocsdrf.tex| & sample redirection file \\
    |cdocsfn1.tex| & sample redirection file \\
    |cdocsfn2.tex| & sample redirection file \\
    |childdoc.pdf| & manual
\end{tabular}
\end{center}
%
The distribution consists of the files
|README.txt|, |childdoc.ins| and |childdoc.dtx|.
%
\begin{itemize}
\item
Run (pdf)\LaTeX{} on |childdoc.dtx|
to compile the manual |childdoc.pdf| (this file).
\item
Run \LaTeX{} on |childdoc.ins| to create the definitions file |childdoc.def|
and the sample |cdocsamp.tex| with include files
|cdocsch1.tex|, |cdocsch2.tex|, |cdocspt3.tex|, |cdocspt4.tex|,
|cdocsdrf.tex|, |cdocsfn1.tex|, |cdocsfn2.tex|.
Then copy the file |childdoc.def| to an appropriate directory of your \LaTeX{}
distribution, e.g.\ \textit{texmf-root}|/tex/latex/childdoc|.
\end{itemize}

%%%%%%%%%%%%%%%%%%%%%%%%%%%%%%%%%%%%%%%%%%%%%%%%%%%%%%%%%%%%%%%%%%%%%%%%%%%%%%%%
\subsection{Related CTAN Packages}

There are several other packages which offer a similar functionality:
%
\begin{itemize}
\item
The packages
\href{http://ctan.org/pkg/docmute}{\textsf{docmute}},
\href{http://ctan.org/pkg/includex}{\textsf{includex}} and
\href{http://ctan.org/pkg/standalone}{\textsf{standalone}}
provide commands to include only the document body of
a child file thus allowing both files to be compiled individually.
\item
The packages \href{http://ctan.org/pkg/subdocs}{\textsf{subdocs}}
and \href{http://ctan.org/pkg/subfiles}{\textsf{subfiles}}
provide structures in which the main and child documents can be
encapsulated and allowing them to be compiled individually.
The inclusion mechanism is different from the conventional |\include|.
\item
The package \href{http://ctan.org/pkg/combine}{\textsf{combine}}
is an elaborate solution to combine several documents into one.
\end{itemize}
%
See also the CTAN topic \href{http://ctan.org/topic/subdocs}{\textsf{subdocs}}
for further related packages.
The present package differs from the above solutions in that
a document structure constructed with the conventional |\include| mechanism
just needs two extra commands at the top of every file
such that all constituent files can be compiled individually.

%%%%%%%%%%%%%%%%%%%%%%%%%%%%%%%%%%%%%%%%%%%%%%%%%%%%%%%%%%%%%%%%%%%%%%%%%%%%%%%%
%\subsection{Feature Suggestions}
%
%The following is a list of features which may be useful for future
%versions of this package:
%%
%\begin{itemize}
%\item
%\ldots
%\end{itemize}

%%%%%%%%%%%%%%%%%%%%%%%%%%%%%%%%%%%%%%%%%%%%%%%%%%%%%%%%%%%%%%%%%%%%%%%%%%%%%%%%
\subsection{Revision History}

%%%%%%%%%%%%%%%%%%%%%%%%%%%%%%%%%%%%%%%%
\paragraph{v2.0:} 2018/12/30

\begin{itemize}
\item
immediate forward processing
\item
added |\childdocby| mechanism
\item
manual restructured
\end{itemize}

%%%%%%%%%%%%%%%%%%%%%%%%%%%%%%%%%%%%%%%%
\paragraph{v1.6:} 2018/01/17

\begin{itemize}
\item
application for development of include files
\item
corrections to manual
\end{itemize}

%%%%%%%%%%%%%%%%%%%%%%%%%%%%%%%%%%%%%%%%
\paragraph{v1.5:} 2017/05/21

\begin{itemize}
\item
more complete structuring introduced
\item
|\childdocof| introduced
\item
|\childdoc| renamed to |\childdocmain|
\item
|\childredirect| renamed to |\childdocforward| and |\childdocforwardprefix|
and functionality expanded
\end{itemize}

%%%%%%%%%%%%%%%%%%%%%%%%%%%%%%%%%%%%%%%%
\paragraph{v1.0:} 2017/04/27

\begin{itemize}
\item
manual and install package
\item
first version published on CTAN
\end{itemize}

%%%%%%%%%%%%%%%%%%%%%%%%%%%%%%%%%%%%%%%%
\paragraph{v0.6:} 2017/04/26

\begin{itemize}
\item
redirection mechanism added
\end{itemize}

%%%%%%%%%%%%%%%%%%%%%%%%%%%%%%%%%%%%%%%%
\paragraph{v0.5:} 2017/04/26

\begin{itemize}
\item
functionality in definition file
\end{itemize}


%%%%%%%%%%%%%%%%%%%%%%%%%%%%%%%%%%%%%%%%%%%%%%%%%%%%%%%%%%%%%%%%%%%%%%%%%%%%%%%%
%%%%%%%%%%%%%%%%%%%%%%%%%%%%%%%%%%%%%%%%%%%%%%%%%%%%%%%%%%%%%%%%%%%%%%%%%%%%%%%%
%%%%%%%%%%%%%%%%%%%%%%%%%%%%%%%%%%%%%%%%%%%%%%%%%%%%%%%%%%%%%%%%%%%%%%%%%%%%%%%%
\appendix

\settowidth\MacroIndent{\rmfamily\scriptsize 000\ }

 \DocInput{childdoc.dtx}

\end{document}
%</driver>
% \fi
%
% %%%%%%%%%%%%%%%%%%%%%%%%%%%%%%%%%%%%%%%%%%%%%%%%%%%%%%%%%%%%%%%%%%%%%%%%%%%%%%
% %%%%%%%%%%%%%%%%%%%%%%%%%%%%%%%%%%%%%%%%%%%%%%%%%%%%%%%%%%%%%%%%%%%%%%%%%%%%%%
% \section{Sample}
%\iffalse
%<*samplemain>
%\fi
%
% The following presents a sample document
% with two chapters, two parts, a title page,
% a compile flag as well as three forwarding files to set the flag.
% It consists of eight |.tex| files:
% \begin{center}
% \begin{tabular}{ll}
% |cdocsamp.tex|&main file\\
% |cdocsch1.tex|&include file for chapter 1\\
% |cdocsch2.tex|&include file for chapter 2\\
% |cdocspt3.tex|&include file for part 3\\
% |cdocspt4.tex|&include file for part 4\\
% |cdocsdrf.tex|&forwarding file for main file in draft mode\\
% |cdocsfi1.tex|&forwarding file for final version of chapter 1\\
% |cdocsfi2.tex|&forwarding file for final version of chapter 2\\
% \end{tabular}
% \end{center}
% Each of the eight files can be compiled directly by the \LaTeX{} compiler.
%
% %%%%%%%%%%%%%%%%%%%%%%%%%%%%%%%%%%%%%%
% \paragraph{Main File.}
%
% The main file is called |cdocsamp.tex|.
%
% Load the \textsf{childdoc} definitions and
% declare the filename for the main document:
%    \begin{macrocode}
\input{childdoc.def}
\childdocmain{}
%    \end{macrocode}

% Optional override for |\version| flag:
%    \begin{macrocode}
%%\ifchilddoc\else\providecommand{\version}{draft}\fi
%    \end{macrocode}

% Define the default values for the |\version| flag
% (|final| for the main file and |draft| for childs):
%    \begin{macrocode}
\ifchilddoc
\providecommand{\version}{draft}
\else
\providecommand{\version}{final}
\fi
%    \end{macrocode}

% Load the standard document class:
%    \begin{macrocode}
\documentclass[12pt]{article}
%    \end{macrocode}

% Start the document body:
%    \begin{macrocode}
\begin{document}
%    \end{macrocode}

% Declare a title page.
% Print title, part of document being processed and version flag:
%    \begin{macrocode}
\addtocounter{page}{-1}
\begin{center}
{\LARGE\bfseries{}childdoc example\par}
\vspace{1cm}
\ifchilddoc
\ifchilddocmanual part\else chapter\fi:
`\childdocname' of `\childdocjob'\par
\else
main document: `\childdocjob'\par
\fi
version: \version\par
\end{center}
\newpage
%    \end{macrocode}

% Manually include selected file,
% otherwise process as usual:
%    \begin{macrocode}
\ifchilddocmanual
\section*{part `\childdocname'}
\input{\childdocname}
\else
%    \end{macrocode}

% Include the two chapters:
%    \begin{macrocode}
\include{cdocsch1}
\include{cdocsch2}
%    \end{macrocode}

% Include the two parts unless only chapters should be displayed:
%    \begin{macrocode}
\ifchilddoc\else
\section{part three}
\input{cdocspt3}
\section{part four}
\input{cdocspt4}
\fi
%    \end{macrocode}

% Process as usual until here:
%    \begin{macrocode}
\fi
%    \end{macrocode}

% End of document body:
%    \begin{macrocode}
\end{document}
%    \end{macrocode}
%\iffalse
%</samplemain>
%\fi
%
% %%%%%%%%%%%%%%%%%%%%%%%%%%%%%%%%%%%%%%
% \paragraph{Chapter Include Files.}
%
% The include files are called |cdocsch1.tex| and |cdocsch2.tex|.
%
%\iffalse
%<*samplechap1|samplechap2>
%\fi

% Optional override for |\version| flag:
%    \begin{macrocode}
%%\providecommand{\version}{final}
%    \end{macrocode}

% Include the main document:
%    \begin{macrocode}
\input{childdoc.def}
\childdocof{cdocsamp}
%    \end{macrocode}

%\iffalse
%</samplechap1|samplechap2>
%\fi
%
%\iffalse
%<*samplechap1>
%\fi
% Some text for chapter 1:
%    \begin{macrocode}
\section{one}
some text in chapter one
%    \end{macrocode}

%\iffalse
%</samplechap1>
%\fi
% Some text for chapter 2:
%\iffalse
%<*samplechap2>
%\fi
%    \begin{macrocode}
\section{two}
more text in chapter two
%    \end{macrocode}

%\iffalse
%</samplechap2>
%\fi
%
% %%%%%%%%%%%%%%%%%%%%%%%%%%%%%%%%%%%%%%
% \paragraph{Part Include Files.}
%
% The include files are called |cdocspt3.tex| and |cdocspt4.tex|.
%
%\iffalse
%<*samplepart3|samplepart4>
%\fi

% Optional override for |\version| flag:
%    \begin{macrocode}
%%\providecommand{\version}{final}
%    \end{macrocode}

% Include the main document:
%    \begin{macrocode}
\input{childdoc.def}
\childdocby{cdocsamp}
%    \end{macrocode}

%\iffalse
%</samplepart3|samplepart4>
%\fi
%
%\iffalse
%<*samplepart3>
%\fi
% Some text for part 3:
%    \begin{macrocode}
some text in part three
%    \end{macrocode}

%\iffalse
%</samplepart3>
%\fi
% Some text for part 4:
%\iffalse
%<*samplepart4>
%\fi
%    \begin{macrocode}
more text in part four
%    \end{macrocode}

%\iffalse
%</samplepart4>
%\fi
%
% %%%%%%%%%%%%%%%%%%%%%%%%%%%%%%%%%%%%%%
% \paragraph{Forwarding for a Complete Draft.}
%
% The following forwarding file |cdocsdrf.tex|
% compiles the main document in draft mode:
%\iffalse
%<*sampledraft>
%\fi
%    \begin{macrocode}
\def\version{draft}
\input{childdoc.def}
\childdocforward{cdocsamp}
%    \end{macrocode}

%\iffalse
%</sampledraft>
%\fi
%
% %%%%%%%%%%%%%%%%%%%%%%%%%%%%%%%%%%%%%%
% \paragraph{Forwarding for Final Version of the Chapters.}
%
% The following forwarding files |cdocsfn1.tex| and |cdocsfn2.tex|
% (with identical content)
% compile the final versions of the child documents
% |cdocsch1.tex| and |cdocsch2.tex|, respectively:
%\iffalse
%<*samplefinal>
%\fi
%    \begin{macrocode}
\def\version{final}
\input{childdoc.def}
\childdocforwardprefix[cdocsamp]{cdocsfn}{cdocsch}
%    \end{macrocode}

%\iffalse
%</samplefinal>
%\fi
%
% %%%%%%%%%%%%%%%%%%%%%%%%%%%%%%%%%%%%%%
% \paragraph{Command Line Processing.}
%
% The following three command lines generate the output files
% |cdocscld|, |cdocscl1| and |cdocscl2|
% which should be identical to
% |cdocsdrf|, |cdocsch1| and |cdocsfn2|, respectively:
% \begin{center}
% \begin{tabular}{l}
% |latex -jobname cdocscld \|\\
% |  "\def\version{draft}\input{childdoc.def}\childdocforward{cdocsamp}"|\\
% |latex -jobname cdocscl1 \|\\
% |  "\input{childdoc.def}\childdocforward[cdocsamp]{cdocsch1}"|\\
% |latex -jobname cdocscl2 \|\\
% |  "\def\version{final}\input{childdoc.def}\childdocforward{cdocsch2}"|
% \end{tabular}
% \end{center}
% Note that the trailing backslash on each first line
% merely continues the input to the second line
% (for convenient cut ant paste).
% Furthermore, the command |latex| can be replaced by any
% of its alternative versions such as |pdflatex|.
%
% %%%%%%%%%%%%%%%%%%%%%%%%%%%%%%%%%%%%%%%%%%%%%%%%%%%%%%%%%%%%%%%%%%%%%%%%%%%%%%
% %%%%%%%%%%%%%%%%%%%%%%%%%%%%%%%%%%%%%%%%%%%%%%%%%%%%%%%%%%%%%%%%%%%%%%%%%%%%%%
% \section{Implementation}
%\iffalse
%<*package>
%\fi
%
% This section describes the definitions file |childdoc.def|.

% The definitions cannot be loaded using |\usepackage| or |\RequirePackage|
% which has a mechanism to prevent loading a style file more than once.
% When loading the definitions by means of |\input|
% multiple instances have to be prevented manually:
%\iffalse
%This code needs to be before the `\ProvidesFile' directive
%which is defined at the beginning of this file.
%Therefore it is also placed there and commented out here.
%</package>
%<*discard>
%\fi
%    \begin{macrocode}
\ifdefined\childdocmain\endinput\fi
%    \end{macrocode}
%\iffalse
%</discard>
%<*package>
%\fi
%
% \macro{\ifchilddoc}
% \macro{\ifchilddocmanual}
% The conditional |\ifchilddoc| tells whether a
% child (true) or main (false) document is being compiled.
% The conditional |\ifchilddocmanual| tells whether
% the |\includeonly| mechanism is used (false) or
% the selection of child files must be performed manually (true).
% The definitions initialise to false:
%    \begin{macrocode}
\newif\ifchilddoc
\newif\ifchilddocmanual
%    \end{macrocode}

% \macro{\childdocname}
% \macro{\childdocjob}
% The macro |\childdocname| stores the name of the main document
% to be compiled. The macro |\childdocjob| stores the name of
% the document on which the \LaTeX{} compiler was originally invoked.
% The content of |\jobname| cannot be compared
% to filenames specified in the source due to different catcodes.
% The following code rescans |\jobname|, stores the result
% in |\childdocname| and saves a copy in |\childdocjob|:
%    \begin{macrocode}
\edef\childdocname{\scantokens\expandafter{\jobname\noexpand}}
\let\childdocjob\childdocname
%    \end{macrocode}

% \macro{\childdocdisable}
% The macro |\childdocdisable| prevents the main file
% from being processed more than once.
% At this stage, the main document command |\childdocmain|
% is assumed to be called once again where it should do nothing.
% Any subsequent call to it should prevent
% a secondary processing of the main document
% It overwrites the forwarding commands
% |\childdocof| and |\childdocforward|
% with empty macros to prevent further inclusions of the main document:
%    \begin{macrocode}
\newcommand{\childdocdisable}
{
  \renewcommand{\childdocmain}[1]{\renewcommand{\childdocmain}[1]{\endinput}}
  \renewcommand{\childdocof}[1]{}
  \renewcommand{\childdocby}[2][]{}
  \renewcommand{\childdocforward}[2][]{}
  \renewcommand{\childdocdisable}{}
}
%    \end{macrocode}

% \macro{\childdocmain}
% The macro |\childdocmain| is to be called at the top of the main file
% with nothing or the main filename (without extension) as argument.
% First, it breaks loops.
% If the argument is not empty and does not match |\childdocname|
% (which is set by the first inclusion of |childdoc.def|),
% |\ifchilddoc| is set to true, |\includeonly| is applied to the child file
% and |\jobname| is set to the main file
% (for proper handling of |.aux| files):
%    \begin{macrocode}
\newcommand{\childdocmain}[1]
{
  \childdocdisable\childdocmain{}
  \if?#1?\else
    \begingroup
      \def\childdoctmp{#1}
      \ifx\childdoctmp\childdocname
        \def\childdoctmp{}
      \else
        \def\childdoctmp
        {
          \childdoctrue
          \includeonly{\childdocname}
          \def\childdocjob{#1}
          \def\jobname{#1}
        }
      \fi
      \expandafter
    \endgroup
    \childdoctmp
  \fi
}
%    \end{macrocode}

% \macro{\childdocof}
% The command |\childdocof| redirects
% compilation to the main file |#1|.
%    \begin{macrocode}
\newcommand{\childdocof}[1]
{
  \childdocdisable
  \childdoctrue
  \includeonly{\childdocname}
  \def\jobname{#1}
  \def\childdocjob{#1}
  \input{#1}
}
%    \end{macrocode}

% \macro{\childdocby}
% The command |\childdocby| ....
%    \begin{macrocode}
\newcommand{\childdocby}[2][]
{
  \childdocdisable
  \childdoctrue
  \childdocmanualtrue
  \if?#1?\else
    \def\jobname{#2}
  \fi
  \def\childdocjob{#2}
  \input{#2}
  \endinput
}
%    \end{macrocode}

% \macro{\childdocforward}
% The command |\childdocforward| redirects
% compilation to the main file or
% (if the optional argument is given) a child file.
% Parameters are set as if the main file
% or a child file starting with |\childdocof| was compiled.
% Then compilation is handed over to the main file:
%    \begin{macrocode}
\newcommand{\childdocforward}[2][]
{
  \begingroup
    \if?#1?
      \def\childdoctmp
      {
        \def\childdocname{#2}
        \def\childdocjob{#2}
        \def\jobname{#2}
        \input{#2}
        \endinput
      }
    \else
      \def\childdoctmp
      {
        \childdocdisable
        \def\childdocname{#2}
        \childdoctrue
        \includeonly{#2}
        \def\childdocjob{#1}
        \def\jobname{#1}
        \input{#1}
        \endinput
      }
    \fi
    \expandafter
  \endgroup
  \childdoctmp
}
%    \end{macrocode}

% \macro{\childdocforwardprefix}
% The command |\childdocforwardprefix| redirects
% compilation to the main or a child file by means of a pattern.
% The prefix |#1| in the current filename is replaced by |#2|
% and the suffix of the current filename is kept
% (it is assumed that the filename does not contain the substring `|~~~|'
% which is used as a delimiter).
% Compilation is handed over to the new file by |\childdocforward|:
%    \begin{macrocode}
\newcommand{\childdocforwardprefix}[3][]
{
  \begingroup
    \def\childdocextract #2##1~~~{\def\childdoctmp{\childdocforward[#1]{#3##1}}}
    \expandafter\childdocextract\childdocname~~~
    \expandafter
  \endgroup
  \childdoctmp
}
%    \end{macrocode}

% \macro{\childdoc}
% The deprecated macro |\childdoc| is a legacy version of |\childdocmain|:
%    \begin{macrocode}
\newcommand{\childdoc}{\childdocmain}
%    \end{macrocode}

% \macro{\childdocredirect}
% The deprecated macro |\childdocredirect| is a legacy version
% of |\childdocforward| and |\childdocforwardprefix|:
%    \begin{macrocode}
\newcommand{\childdocredirect}[2][]
{
  \begingroup
    \if?#1?
      \def\childdoctmp{\childdocforward{#2}}
    \else
      \def\childdoctmp{\childdocforwardprefix{#1}{#2}}
    \fi
    \expandafter
  \endgroup
  \childdoctmp
}
%    \end{macrocode}

%\iffalse
%</package>
%\fi
%
\endinput
\childdocforward[|\textit{main}|]{|\textit{dest}|}"|
\end{center}
%
Here \textit{target} is the name of the output file,
\textit{main} is the name of the main file
and \textit{dest} is the name of the main or child file to be processed
(all filenames without extensions).
The optional argument \textit{main} can be omitted
if \textit{main} matches \textit{dest}.
Optionally, compilation \textit{flags} can be defined via |\def| commands.
This command line makes the \TeX{} engine believe
it is compiling the file \textit{target}
whose content is specified as the latter parameter.
The provided code then forwards the processing to
\textit{main} or \textit{dest} as described in \secref{sec:forward}.

%%%%%%%%%%%%%%%%%%%%%%%%%%%%%%%%%%%%%%%%%%%%%%%%%%%%%%%%%%%%%%%%%%%%%%%%%%%%%%%%
\subsection{Include by Input}
\label{sec:input}

Including child documents by |\include| has some restrictions by design.
Most notably, the content of a child document always occupies
its own set of pages; pages cannot be shared between child documents.
Usually, this behaviour makes perfect sense
because each child document contain an essential part of the document.
However, in some situations it may be desirable to compose
a document from a collection of parts
without having mandatory page breaks between then.
For this case, the package
provides a mechanism to include parts
by |\input| which can also be processed individually.
However, by construction this mechanism
requires manual handling of the content to be output.

%%%%%%%%%%%%%%%%%%%%%%%%%%%%%%%%%%%%%%%%
\DescribeMacro{\ifchilddocmanual}
The main file should be prepared as usual, see \secref{sec:include}.
However, the document body must make a distinction
between processing of an individual part and of the main document, e.g.:
%
\begin{center}
\begin{tabular}{l}
|\ifchilddocmanual|\\
|\input{\childdocname}|\\
|\||else|\\
\textit{document body with }|\input{|\textit{part}|}|\\
|\||fi|
\end{tabular}
\end{center}
%
The conditional |\ifchilddocmanual| is true whenever
a part to be included by |\input| is being compiled,
and the name of the part is stored in |\childdocname|.

%%%%%%%%%%%%%%%%%%%%%%%%%%%%%%%%%%%%%%%%
\DescribeMacro{\childdocby}
Each part to be included by |\input| should start with:
%
\begin{center}
\begin{tabular}{l}
|% \iffalse
%
% childdoc.dtx Copyright (C) 2017-2018 Niklas Beisert
%
% This work may be distributed and/or modified under the
% conditions of the LaTeX Project Public License, either version 1.3
% of this license or (at your option) any later version.
% The latest version of this license is in
%   http://www.latex-project.org/lppl.txt
% and version 1.3 or later is part of all distributions of LaTeX
% version 2005/12/01 or later.
%
% This work has the LPPL maintenance status `maintained'.
%
% The Current Maintainer of this work is Niklas Beisert.
%
% This work consists of the files childdoc.dtx and childdoc.ins
% and the derived files childdoc.def and cdocsamp.tex with
% cdocsch1.tex, cdocsch2.tex, cdocsdrf.tex, cdocsfn1.tex, cdocsfn2.tex.
%
%<package>\ifdefined\childdocmain\endinput\fi
%<package>\ProvidesFile{childdoc.def}[2018/12/30 v2.0 child document driver]
%<samplemain>\ProvidesFile{cdocsamp.tex}[2018/12/30 v2.0 sample for childdoc]
%<*driver>
%\ProvidesFile{childdoc.drv}[2018/12/30 v2.0 childdoc reference manual file]
\PassOptionsToClass{10pt,a4paper}{article}
\documentclass{ltxdoc}

\usepackage[margin=35mm]{geometry}
\usepackage{hyperref}
\usepackage{hyperxmp}
\usepackage[usenames]{color}

\hypersetup{colorlinks=true}
\hypersetup{pdfstartview=FitH}
\hypersetup{pdfpagemode=UseNone}
\hypersetup{pdfsource={}}
\hypersetup{pdflang={en-UK}}
\hypersetup{pdfcopyright={Copyright 2017-2018 Niklas Beisert.
  This work may be distributed and/or modified under the
  conditions of the LaTeX Project Public License, either version 1.3
  of this license or (at your option) any later version.}}
\hypersetup{pdflicenseurl={http://www.latex-project.org/lppl.txt}}
\hypersetup{pdfcontactaddress={ETH Zurich, ITP, HIT K,
  Wolfgang-Pauli-Strasse 27}}
\hypersetup{pdfcontactpostcode={8093}}
\hypersetup{pdfcontactcity={Zurich}}
\hypersetup{pdfcontactcountry={Switzerland}}
\hypersetup{pdfcontactemail={nbeisert@itp.phys.ethz.ch}}
\hypersetup{pdfcontacturl={http://people.phys.ethz.ch/\xmptilde nbeisert/}}

\newcommand{\secref}[1]{\hyperref[#1]{section \ref*{#1}}}

\parskip1ex
\parindent0pt
\let\olditemize\itemize
\def\itemize{\olditemize\parskip0pt}

\begin{document}

\title{The \textsf{childdoc} Package}
\hypersetup{pdftitle={The childdoc Package}}
\author{Niklas Beisert\\[2ex]
  Institut f\"ur Theoretische Physik\\
  Eidgen\"ossische Technische Hochschule Z\"urich\\
  Wolfgang-Pauli-Strasse 27, 8093 Z\"urich, Switzerland\\[1ex]
  \href{mailto:nbeisert@itp.phys.ethz.ch}
  {\texttt{nbeisert@itp.phys.ethz.ch}}}
\hypersetup{pdfauthor={Niklas Beisert}}
\hypersetup{pdfsubject={Manual for the LaTeX2e Package childdoc}}
\date{30 December 2018, \textsf{v2.0}}
\maketitle

\begin{abstract}\noindent
\textsf{childdoc} is a \LaTeXe{} package
that enables the direct compilation
of document sections included by |\include|
to individual files.
\end{abstract}

\begingroup
\parskip0ex
\tableofcontents
\endgroup

%%%%%%%%%%%%%%%%%%%%%%%%%%%%%%%%%%%%%%%%%%%%%%%%%%%%%%%%%%%%%%%%%%%%%%%%%%%%%%%%
%%%%%%%%%%%%%%%%%%%%%%%%%%%%%%%%%%%%%%%%%%%%%%%%%%%%%%%%%%%%%%%%%%%%%%%%%%%%%%%%
\section{Introduction}

\LaTeX{} provides a mechanism to structure a large document (such as a book)
into a main file and several child files (containing the chapters)
using the |\include| command.
This mechanism is beneficial for documents
which span hundreds of pages in order to
make the source file(s) more manageable.
Moreover, compilation can be restricted to
selected child files by means of the |\includeonly| command.
The latter feature can be used to reduce the compilation time while editing
(this was significantly more useful in the earlier days of \LaTeX{})
or to generate a smaller document which is easier to navigate.
Another application of |\includeonly| is to generate
documents consisting of selected parts of the complete document.

However, there are a few drawbacks of the plain |\include| mechanism:
\begin{itemize}
\item
The child files cannot be compiled on their own,
they can only be compiled via the main file.
A naive editing environment
(such as a text editor with an option
to have the current file processed by \LaTeX)
may require one to switch to the main file before compiling;
attempting to compile the child file produces errors.
\item
The main file must be modified (each time)
to adjust the |\includeonly| command
to the present needs. This easily leaves the main file in a messy state.
\item
The generated document will always carry the filename
of the main document. This is inconvenient if
several child files are to be compiled and
to be kept for distribution.
\end{itemize}

The present package provides a simple interface
to make child files individually compilable by \LaTeX{}.
Compiling a child file then has the same effect as compiling
the main file with an |\includeonly| command
to select the appropriate child.
Moreover the generated document will carry the name of the child
rather than the main file.
This resolves all three above issues.

This feature is meant to make the editing of books,
thesis documents and lecture notes somewhat more convenient.
However, the package can also be used efficiently for
composing a series of documents (such as exercise sheets)
which are typically distributed individually.
It then assists the author in generating the individual documents
(potentially in different versions)
as well as a document containing the collected series.
Another application is in developing style files
or other kinds of included material
where compilation of the style file could redirect
to a sample or test file.

%%%%%%%%%%%%%%%%%%%%%%%%%%%%%%%%%%%%%%%%%%%%%%%%%%%%%%%%%%%%%%%%%%%%%%%%%%%%%%%%
%%%%%%%%%%%%%%%%%%%%%%%%%%%%%%%%%%%%%%%%%%%%%%%%%%%%%%%%%%%%%%%%%%%%%%%%%%%%%%%%
\section{Usage}

First of all, the package \textsf{childdoc} is \emph{not} a standard
\LaTeXe{} |.sty| style file! Therefore it needs to be invoked in
a non-standard way.

%%%%%%%%%%%%%%%%%%%%%%%%%%%%%%%%%%%%%%%%%%%%%%%%%%%%%%%%%%%%%%%%%%%%%%%%%%%%%%%%
\subsection{Included Files}
\label{sec:include}

%%%%%%%%%%%%%%%%%%%%%%%%%%%%%%%%%%%%%%%%
\DescribeMacro{\childdocmain}
To use the package, add the commands
\begin{center}
\begin{tabular}{l}
|\input{childdoc.def}|\\
|\childdocmain{}|\\
\end{tabular}
\end{center}
at the very top of the main \LaTeX{} file,
in particular \emph{before} the |\documentclass| statement!
The argument of |\childdocmain| should be left empty
(but it must be present).

%%%%%%%%%%%%%%%%%%%%%%%%%%%%%%%%%%%%%%%%
\DescribeMacro{\childdocof}
Furthermore, add the commands
\begin{center}
\begin{tabular}{l}
|\input{childdoc.def}|\\
|\childdocof{|\textit{main}|}|\\
\end{tabular}
\end{center}
at the top of every child file \textit{child}
which is included by |\include{|\textit{child}|}|
from within the main file
(or at least for those files to be compiled individually).
The argument \textit{main} must be the filename of the main file.

There are a couple of
considerations in setting up the main and child documents:

%%%%%%%%%%%%%%%%%%%%%%%%%%%%%%%%%%%%%%%%
\paragraph{Restrictions.}

Please note the following restrictions:
\begin{itemize}
\item
|\childdocmain| must be called with one argument \textit{main}
to ensure compatibility with earlier version of the package.
It must either be empty (|\childdocmain{}|)
or precisely match the filename of the main file in which it is specified.
See \secref{sec:detection} for further information.
\item
The filename \textit{main} must be specified without the |.tex| extension.
\item
The filename \textit{main} is case sensitive
(even in case-insensitive file systems)
due to internal string comparison.
\item
The argument \textit{main} should be fully expanded, it cannot be a macro.
\item
Subdirectories and special characters should be avoided in filenames.
\item
The command |\childdocmain{|\textit{main}|}| must be followed by a whitespace.
It should not be followed immediately by another command
or by a comment mark `|%|'.
This is because the \TeX{} parser reads the token immediately following
the argument of |\childdocmain| and puts it
at the beginning of every child section;
however, a white\-space is ignored.
\end{itemize}

%%%%%%%%%%%%%%%%%%%%%%%%%%%%%%%%%%%%%%%%
\paragraph{Content of Main File.}

It is advisable to place all content in the child files included by |\include|.
Any output contained in the main file will appear in all child documents
unless suppressed manually;
it cannot be suppressed automatically by the |\includeonly| directive
and thus should normally be avoided.
A method to include some content in the main file
by means of conditional processing is described in \secref{sec:conditional}.

%%%%%%%%%%%%%%%%%%%%%%%%%%%%%%%%%%%%%%%%
\paragraph{Page Numbering.}

When only a part of the document is compiled,
the appropriate numbering of pages
(as well as other status parameters)
is determined from the |.aux| files.
The latter contain information from previous passes.
However this information needs to propagate through
all intermediate child documents.
Therefore the page numbering in child documents may well
be inconsistent until the complete document is compiled at least once.

A useful (if unconventional) way to always ensure a consistent
page numbering is to restart the numbering in each child document
and denote the pages by `\textit{child}|.|\textit{page}'
where \textit{child} represents the chapter/section number of the child file.
This can be achieved by the command
|\numberwithin{page}{|\textit{child}|}|
of the \textsf{amsmath} package
where \textit{child} can be |chapter| or |section|
depending on the chosen structuring.
Alternatively, one can modify the macro |\thepage| appropriately
and reset the counter |page| at the start of each child file.

%%%%%%%%%%%%%%%%%%%%%%%%%%%%%%%%%%%%%%%%%%%%%%%%%%%%%%%%%%%%%%%%%%%%%%%%%%%%%%%%
\subsection{Conditional Processing}
\label{sec:conditional}

The package provides a mechanism to compile different versions
of a document. To customise the versions further some conditional processing
can come in handy to distinguish which version is being compiled.
The package provides two macros to describe the compilation context:

%%%%%%%%%%%%%%%%%%%%%%%%%%%%%%%%%%%%%%%%
\DescribeMacro{\ifchilddoc}
The conditional |\ifchilddoc| distinguishes between the compilation of
child documents and the main document:
%
\begin{center}
|\ifchilddoc |\textit{child-code}| |[|\||else |\textit{main-code}]| \||fi|
\end{center}

%%%%%%%%%%%%%%%%%%%%%%%%%%%%%%%%%%%%%%%%
\DescribeMacro{\childdocname}
\DescribeMacro{\childdocjob}
The macro |\childdocname| contains the filename (without extension)
of the main or child file being processed.
Note that |\childdocjob| will always contain the name of the main file.

%%%%%%%%%%%%%%%%%%%%%%%%%%%%%%%%%%%%%%%%
\paragraph{Title Page.}

Conditional processing can be used to include a title or banner page
in the main document when proper precautions are taken.
Importantly, the code in the main file should ensure that the page counter
(as well as other status parameters which are stored in the |.aux| files)
takes the same value after the conditional processing.
Otherwise the page numbers may take divergent values
depending on which part is compiled.

For example, a title page could be declared by:
%
\begin{center}
\begin{tabular}{l}
|\ifchilddoc\||else|\\
|\addtocounter{page}{-1}|\\
\textit{code for title page}\\
|\newpage|\\
|\||fi|
\end{tabular}
\end{center}
%
A banner page for the child documents can be generated by:
%
\begin{center}
\begin{tabular}{l}
|\ifchilddoc|\\
|\addtocounter{page}{-1}|\\
\textit{code for banner page}\\
|\newpage|\\
|\||fi|
\end{tabular}
\end{center}
%
Here one could write a message such as:
\begin{center}
|This is the part \childdocname{} of \childdocjob{}.|
\end{center}

%%%%%%%%%%%%%%%%%%%%%%%%%%%%%%%%%%%%%%%%%%%%%%%%%%%%%%%%%%%%%%%%%%%%%%%%%%%%%%%%
\subsection{Flags}
\label{sec:flags}

The package makes it easy to generate different versions
of the main or child documents.
To this end compilation flags can be defined
and assigned different default values.
They will be particularly useful in conjunction
with the forwarding mechanism described in \secref{sec:forward}.

For example, it may be useful to have a flag |\version|
which can be set to |draft| or |final|.
The document source will contain some conditional code
depending on the value of |\version|.
Suppose further, the flag should default to |final| for the main file
and to |draft| for child files
which is a natural assignment for editing the document.
This is achieved by placing the following code
in the preamble of the main document
(below the |\childdocmain| directive):
%
\begin{center}
\begin{tabular}{l}
|\ifchilddoc|\\
|\providecommand{\version}{draft}|\\
|\||else|\\
|\providecommand{\version}{final}|\\
|\||fi|
\end{tabular}
\end{center}
%
The definition by |\providecommand| makes sure
that previous definitions are not overwritten.
Further statements |\providecommand{\version}{...}|
can thus be added before the above code to override it.

For the main file, one might add a line
(between |\childdocmain| and the above block)
%
\begin{center}
|%\ifchilddoc\||else\providecommand{\version}{draft}\||fi|
\end{center}
%
which can be uncommented to produce a draft version.
Likewise one can add a line to the very top of a child file
(above the |\childdocof{|\textit{main}|}| directive)
%
\begin{center}
|%\providecommand{\version}{final}|
\end{center}
%
which can be uncommented to produce the final version of this child document.

%%%%%%%%%%%%%%%%%%%%%%%%%%%%%%%%%%%%%%%%%%%%%%%%%%%%%%%%%%%%%%%%%%%%%%%%%%%%%%%%
\subsection{Forwarding}
\label{sec:forward}

Different versions of the main or child documents
using compilation flags as described in \secref{sec:flags}
can be (permanently) stored in different files
for convenient compilation, viewing and distribution.
To this end, the package defines a command
to pass on compilation to a different file:

%%%%%%%%%%%%%%%%%%%%%%%%%%%%%%%%%%%%%%%%
\DescribeMacro{\childdocforward}
The command |\childdocforward| redirects processing to
another source file:
%
\begin{center}
\begin{tabular}{l}
|\input{childdoc.def}|\\
|\childdocforward[|\textit{main}|]{|\textit{dest}|}|\\
\end{tabular}
\end{center}
%
The argument \textit{dest} is the destination file
(without extension).
It should be the main file or one of the child files.
Note that further \textsf{childdoc} directives
such as |\childdocof| and |\childdocforward|
in the indicated file will be processed in this form.
The optional argument \textit{main}
passes on directly to the main file \textit{main}
while pretending to compile the child \textit{dest}.
This form behaves as if \textit{dest}
issues |\childdocof{|\textit{main}|}| right away,
and no further \textsf{childdoc} directives will be processed.

%%%%%%%%%%%%%%%%%%%%%%%%%%%%%%%%%%%%%%%%
\DescribeMacro{\...prefix}
In the alternative form |\childdocforwardprefix|,
%
\begin{center}
\begin{tabular}{l}
|\input{childdoc.def}|\\
|\childdocforwardprefix[|\textit{main}|]{|\textit{prefix}|}{|\textit{dest}|}|
\end{tabular}
\end{center}
%
the destination file is determined by a pattern
depending on the current file:
To make this work, the current file must be called
`{\textit{prefix}\hspace{0.2em}\textit{suffix}}'
with \textit{prefix} matching precisely the argument.
Processing is then passed on to the file
`{\textit{dest}\hspace{0.2em}\textit{suffix}}'.
Surely, the same effect is achieved by
directly specifying the
argument `{\textit{dest}\hspace{0.2em}\textit{suffix}}'
in the first form.
However, that requires to set up a different file
for each child. With the alternative form of the command
all these files can have exactly the same content
which simplifies setting them up and maintaining them.

For example, the following file |draft.tex|
with a compilation flag |\version| as described in \secref{sec:flags}
compiles the main document as a draft:
%
\begin{center}
\begin{tabular}{l}
|\def\version{draft}|\\
|\input{childdoc.def}|\\
|\childdocforward{|\textit{main}|}|
\end{tabular}
\end{center}
%
Likewise, the following files |final|\textit{nn}|.tex|
compile the final version of the child document
|child|\textit{nn}|.tex|:
%
\begin{center}
\begin{tabular}{l}
|\def\version{final}|\\
|\input{childdoc.def}|\\
|\childdocforwardprefix{final}{child}|
\end{tabular}
\end{center}
%

Note that when several versions of a main file and/or of each child file
are to be generated, it may be convenient to set up a |Makefile| or
shell script to automatise the process.

%%%%%%%%%%%%%%%%%%%%%%%%%%%%%%%%%%%%%%%%%%%%%%%%%%%%%%%%%%%%%%%%%%%%%%%%%%%%%%%%
\subsection{Command Line Processing}
\label{sec:commandline}

The effect of redirection files can also be achieved by invoking
the \LaTeX{} compiler with a more elaborate command line.
Most conveniently this should be done as part
of a shell script or a |Makefile|.

When using \textsf{childdoc} in the main file, the following
command lines effectively perform a redirection
(note that depending on the shell being used,
backslashes may have to be doubled: `|\|' $\to$ `|\\|'):
%
\begin{center}
|... -jobname "|\textit{target}|" |\\|"|[\textit{flags}]%
|\input{childdoc.def}\childdocforward[|\textit{main}|]{|\textit{dest}|}"|
\end{center}
%
Here \textit{target} is the name of the output file,
\textit{main} is the name of the main file
and \textit{dest} is the name of the main or child file to be processed
(all filenames without extensions).
The optional argument \textit{main} can be omitted
if \textit{main} matches \textit{dest}.
Optionally, compilation \textit{flags} can be defined via |\def| commands.
This command line makes the \TeX{} engine believe
it is compiling the file \textit{target}
whose content is specified as the latter parameter.
The provided code then forwards the processing to
\textit{main} or \textit{dest} as described in \secref{sec:forward}.

%%%%%%%%%%%%%%%%%%%%%%%%%%%%%%%%%%%%%%%%%%%%%%%%%%%%%%%%%%%%%%%%%%%%%%%%%%%%%%%%
\subsection{Include by Input}
\label{sec:input}

Including child documents by |\include| has some restrictions by design.
Most notably, the content of a child document always occupies
its own set of pages; pages cannot be shared between child documents.
Usually, this behaviour makes perfect sense
because each child document contain an essential part of the document.
However, in some situations it may be desirable to compose
a document from a collection of parts
without having mandatory page breaks between then.
For this case, the package
provides a mechanism to include parts
by |\input| which can also be processed individually.
However, by construction this mechanism
requires manual handling of the content to be output.

%%%%%%%%%%%%%%%%%%%%%%%%%%%%%%%%%%%%%%%%
\DescribeMacro{\ifchilddocmanual}
The main file should be prepared as usual, see \secref{sec:include}.
However, the document body must make a distinction
between processing of an individual part and of the main document, e.g.:
%
\begin{center}
\begin{tabular}{l}
|\ifchilddocmanual|\\
|\input{\childdocname}|\\
|\||else|\\
\textit{document body with }|\input{|\textit{part}|}|\\
|\||fi|
\end{tabular}
\end{center}
%
The conditional |\ifchilddocmanual| is true whenever
a part to be included by |\input| is being compiled,
and the name of the part is stored in |\childdocname|.

%%%%%%%%%%%%%%%%%%%%%%%%%%%%%%%%%%%%%%%%
\DescribeMacro{\childdocby}
Each part to be included by |\input| should start with:
%
\begin{center}
\begin{tabular}{l}
|\input{childdoc.def}|\\
|\childdocby{|\textit{main}|}|\\
\end{tabular}
\end{center}
%
The directive |\childdocby| is similar to |\childdocof|
described in \secref{sec:include},
but the subsequent selection of content must be done manually.
To that end, both |\ifchilddoc| and |\ifchilddocmanual|
will be true upon processing of a part,
and the name of the part is stored in |\childdocname|.
Note that |\jobname| will be set to the filename of the current part
so that each part receives an individual |.aux| file
that does not interfere with the |.aux| file(s) of the main document.
This behaviour can be altered by the alternative form
|\childdocby[*]{|\textit{main}|}| (with a non-empty optional argument)
which uses the |.aux| file of the main document
by setting |\jobname| to \textit{main}.

%%%%%%%%%%%%%%%%%%%%%%%%%%%%%%%%%%%%%%%%%%%%%%%%%%%%%%%%%%%%%%%%%%%%%%%%%%%%%%%%
\subsection{Driver Development}
\label{sec:driver}

The \textsf{childdoc} mechanism can also be use for the development
of definition files such as \LaTeX{} styles or classes.
This case differs from the above setup with multiple parts
included by |\include| in that no |\includeonly| should be invoked.
This can be achieved by starting the include file
(before |\ProvidesPackage|) with:
%
\begin{center}
\begin{tabular}{l}
|\input{childdoc.def}|\\
|\childdocforward{|\textit{main}|}|\\
\end{tabular}
\end{center}
%
or alternatively with:
%
\begin{center}
\begin{tabular}{l}
|\input{childdoc.def}|\\
|\childdocby{|\textit{main}|}|\\
\end{tabular}
\end{center}
%
Both forms have slightly different effects as described above.
The main file is prepared as usual, see \secref{sec:include}.

%%%%%%%%%%%%%%%%%%%%%%%%%%%%%%%%%%%%%%%%%%%%%%%%%%%%%%%%%%%%%%%%%%%%%%%%%%%%%%%%
\subsection{Legacy Detection}
\label{sec:detection}

The directive |\childdocmain| in the main file can detect
whether the complete document or merely a child is to be compiled
even without using the directive |\childdocof|.
This method is deprecated because it is less robust
and there is no compelling reason to use it;
it is merely provided for backward compatibility
and it may be removed in future versions.

If the detection mechanism is to be used,
it is mandatory to correctly specify
the filename of the main file as the argument of |\childdocmain|:
%
\begin{center}
\begin{tabular}{l}
|\input{childdoc.def}|\\
|\childdocmain{|\textit{main}|}|\\
\end{tabular}
\end{center}
%
If |\jobname| does not match the argument \textit{main} of |\childdocmain|,
it is assumed that |\jobname| points to the child file to be compiled.
When using |\childdocmain| with the main file specified as argument,
it suffices to start a child file
with just |\input{|\textit{main}|}|
without loading of the package and using |\childdocof|.
If instead all processing is done
with the appropriate \textsf{childdoc} directives,
the argument of \textit{main} of |\childdocmain| can be empty.

An alternative version of the command line processing described
in \secref{sec:commandline} using the detection mechanism reads:
%
\begin{center}
|... -jobname "|\textit{target}|" "|[\textit{flags}]%
[|\def\jobname{|\textit{dest}|}|]|\input{|\textit{main}|}"|
\end{center}

%%%%%%%%%%%%%%%%%%%%%%%%%%%%%%%%%%%%%%%%%%%%%%%%%%%%%%%%%%%%%%%%%%%%%%%%%%%%%%%%
\subsection{Manual Code}
\label{sec:manual}

In case one cannot be certain whether the definitions file |childdoc.def|
is installed on the target \TeX{} distribution
and one prefers not to ship it,
it is conceivable to paste a few relevant commands into the sources.

To that end, drop all statements |\input{childdoc.def}|
and perform the replacements as outlined below.
Instead of |\childdocmain{|\textit{main}|}| add the following code
to the top of the main file:
%
\begin{center}
\begin{tabular}{l}
|\||ifdefined\childdocname\endinput\||fi\newif\ifchilddoc|\\
|\edef\childdocname{\scantokens\expandafter{\jobname\noexpand}}|\\
|\def\childdocmain{|\textit{main}|}\||ifx\childdocmain\childdocname\||else|\\
|\childdoctrue\includeonly{\childdocname}\let\jobname\childdocmain\||fi|\\
\end{tabular}
\end{center}
%
Instead of |\childdocof{|\textit{main}|}| just include the main file
at the top of each child file:
%
\begin{center}
|\input{|\textit{main}|}|
\end{center}
%
A simple redirection |\childdocforward{|\textit{dest}|}| is achieved by:
%
\begin{center}
|\def\jobname{|\textit{dest}|}\input{\jobname}|
\end{center}
%
The redirection with prefix
|\childdocforwardprefix[|\textit{prefix}|]{|\textit{dest}|}|
is accomplished by:
%
\begin{center}
\begin{tabular}{l}
|{\edef\jobname{\scantokens\expandafter{\jobname\noexpand}}|\\
|\def\redirectjob |\textit{prefix}|#1~~~{\gdef\jobname{|\textit{dest}|#1}}|\\
|\expandafter\redirectjob\jobname~~~}\input{\jobname}|
\end{tabular}
\end{center}

In an alternative approach,
child documents can be compiled by a specific command line
without additional code or specific definitions:
%
\begin{center}
|... -jobname "|\textit{target}|" "|[\textit{flags}]%
|\includeonly{|\textit{dest}|}\input{|\textit{main}|}"|
\end{center}
%

%%%%%%%%%%%%%%%%%%%%%%%%%%%%%%%%%%%%%%%%%%%%%%%%%%%%%%%%%%%%%%%%%%%%%%%%%%%%%%%%
%%%%%%%%%%%%%%%%%%%%%%%%%%%%%%%%%%%%%%%%%%%%%%%%%%%%%%%%%%%%%%%%%%%%%%%%%%%%%%%%
\section{Information}

%%%%%%%%%%%%%%%%%%%%%%%%%%%%%%%%%%%%%%%%%%%%%%%%%%%%%%%%%%%%%%%%%%%%%%%%%%%%%%%%
\subsection{Copyright}

Copyright \copyright{} 2017--2018 Niklas Beisert

This work may be distributed and/or modified under the
conditions of the \LaTeX{} Project Public License, either version 1.3
of this license or (at your option) any later version.
The latest version of this license is in
  \url{http://www.latex-project.org/lppl.txt}
and version 1.3 or later is part of all distributions of \LaTeX{}
version 2005/12/01 or later.

This work has the LPPL maintenance status `maintained'.

The Current Maintainer of this work is Niklas Beisert.

This work consists of the files |README.txt|, |childdoc.ins| and |childdoc.dtx|
as well as the derived files |childdoc.def|, |cdocsamp.tex|
with |cdocsch1.tex|, |cdocsch2.tex|, |cdocspt3.tex|, |cdocspt4.tex|,
|cdocsdrf.tex|, |cdocsfn1.tex|, |cdocsfn2.tex|
as well as |childdoc.pdf|.

%%%%%%%%%%%%%%%%%%%%%%%%%%%%%%%%%%%%%%%%%%%%%%%%%%%%%%%%%%%%%%%%%%%%%%%%%%%%%%%%
\subsection{Files and Installation}

The package consists of the files:
%
\begin{center}
\begin{tabular}{ll}
    |README.txt|   & readme file \\
    |childdoc.ins| & installation file \\
    |childdoc.dtx| & source file \\
    |childdoc.def| & definition file \\
    |cdocsamp.tex| & sample main file \\
    |cdocsch1.tex| & sample include file \\
    |cdocsch2.tex| & sample include file \\
    |cdocspt3.tex| & sample part file \\
    |cdocspt4.tex| & sample part file \\
    |cdocsdrf.tex| & sample redirection file \\
    |cdocsfn1.tex| & sample redirection file \\
    |cdocsfn2.tex| & sample redirection file \\
    |childdoc.pdf| & manual
\end{tabular}
\end{center}
%
The distribution consists of the files
|README.txt|, |childdoc.ins| and |childdoc.dtx|.
%
\begin{itemize}
\item
Run (pdf)\LaTeX{} on |childdoc.dtx|
to compile the manual |childdoc.pdf| (this file).
\item
Run \LaTeX{} on |childdoc.ins| to create the definitions file |childdoc.def|
and the sample |cdocsamp.tex| with include files
|cdocsch1.tex|, |cdocsch2.tex|, |cdocspt3.tex|, |cdocspt4.tex|,
|cdocsdrf.tex|, |cdocsfn1.tex|, |cdocsfn2.tex|.
Then copy the file |childdoc.def| to an appropriate directory of your \LaTeX{}
distribution, e.g.\ \textit{texmf-root}|/tex/latex/childdoc|.
\end{itemize}

%%%%%%%%%%%%%%%%%%%%%%%%%%%%%%%%%%%%%%%%%%%%%%%%%%%%%%%%%%%%%%%%%%%%%%%%%%%%%%%%
\subsection{Related CTAN Packages}

There are several other packages which offer a similar functionality:
%
\begin{itemize}
\item
The packages
\href{http://ctan.org/pkg/docmute}{\textsf{docmute}},
\href{http://ctan.org/pkg/includex}{\textsf{includex}} and
\href{http://ctan.org/pkg/standalone}{\textsf{standalone}}
provide commands to include only the document body of
a child file thus allowing both files to be compiled individually.
\item
The packages \href{http://ctan.org/pkg/subdocs}{\textsf{subdocs}}
and \href{http://ctan.org/pkg/subfiles}{\textsf{subfiles}}
provide structures in which the main and child documents can be
encapsulated and allowing them to be compiled individually.
The inclusion mechanism is different from the conventional |\include|.
\item
The package \href{http://ctan.org/pkg/combine}{\textsf{combine}}
is an elaborate solution to combine several documents into one.
\end{itemize}
%
See also the CTAN topic \href{http://ctan.org/topic/subdocs}{\textsf{subdocs}}
for further related packages.
The present package differs from the above solutions in that
a document structure constructed with the conventional |\include| mechanism
just needs two extra commands at the top of every file
such that all constituent files can be compiled individually.

%%%%%%%%%%%%%%%%%%%%%%%%%%%%%%%%%%%%%%%%%%%%%%%%%%%%%%%%%%%%%%%%%%%%%%%%%%%%%%%%
%\subsection{Feature Suggestions}
%
%The following is a list of features which may be useful for future
%versions of this package:
%%
%\begin{itemize}
%\item
%\ldots
%\end{itemize}

%%%%%%%%%%%%%%%%%%%%%%%%%%%%%%%%%%%%%%%%%%%%%%%%%%%%%%%%%%%%%%%%%%%%%%%%%%%%%%%%
\subsection{Revision History}

%%%%%%%%%%%%%%%%%%%%%%%%%%%%%%%%%%%%%%%%
\paragraph{v2.0:} 2018/12/30

\begin{itemize}
\item
immediate forward processing
\item
added |\childdocby| mechanism
\item
manual restructured
\end{itemize}

%%%%%%%%%%%%%%%%%%%%%%%%%%%%%%%%%%%%%%%%
\paragraph{v1.6:} 2018/01/17

\begin{itemize}
\item
application for development of include files
\item
corrections to manual
\end{itemize}

%%%%%%%%%%%%%%%%%%%%%%%%%%%%%%%%%%%%%%%%
\paragraph{v1.5:} 2017/05/21

\begin{itemize}
\item
more complete structuring introduced
\item
|\childdocof| introduced
\item
|\childdoc| renamed to |\childdocmain|
\item
|\childredirect| renamed to |\childdocforward| and |\childdocforwardprefix|
and functionality expanded
\end{itemize}

%%%%%%%%%%%%%%%%%%%%%%%%%%%%%%%%%%%%%%%%
\paragraph{v1.0:} 2017/04/27

\begin{itemize}
\item
manual and install package
\item
first version published on CTAN
\end{itemize}

%%%%%%%%%%%%%%%%%%%%%%%%%%%%%%%%%%%%%%%%
\paragraph{v0.6:} 2017/04/26

\begin{itemize}
\item
redirection mechanism added
\end{itemize}

%%%%%%%%%%%%%%%%%%%%%%%%%%%%%%%%%%%%%%%%
\paragraph{v0.5:} 2017/04/26

\begin{itemize}
\item
functionality in definition file
\end{itemize}


%%%%%%%%%%%%%%%%%%%%%%%%%%%%%%%%%%%%%%%%%%%%%%%%%%%%%%%%%%%%%%%%%%%%%%%%%%%%%%%%
%%%%%%%%%%%%%%%%%%%%%%%%%%%%%%%%%%%%%%%%%%%%%%%%%%%%%%%%%%%%%%%%%%%%%%%%%%%%%%%%
%%%%%%%%%%%%%%%%%%%%%%%%%%%%%%%%%%%%%%%%%%%%%%%%%%%%%%%%%%%%%%%%%%%%%%%%%%%%%%%%
\appendix

\settowidth\MacroIndent{\rmfamily\scriptsize 000\ }

 \DocInput{childdoc.dtx}

\end{document}
%</driver>
% \fi
%
% %%%%%%%%%%%%%%%%%%%%%%%%%%%%%%%%%%%%%%%%%%%%%%%%%%%%%%%%%%%%%%%%%%%%%%%%%%%%%%
% %%%%%%%%%%%%%%%%%%%%%%%%%%%%%%%%%%%%%%%%%%%%%%%%%%%%%%%%%%%%%%%%%%%%%%%%%%%%%%
% \section{Sample}
%\iffalse
%<*samplemain>
%\fi
%
% The following presents a sample document
% with two chapters, two parts, a title page,
% a compile flag as well as three forwarding files to set the flag.
% It consists of eight |.tex| files:
% \begin{center}
% \begin{tabular}{ll}
% |cdocsamp.tex|&main file\\
% |cdocsch1.tex|&include file for chapter 1\\
% |cdocsch2.tex|&include file for chapter 2\\
% |cdocspt3.tex|&include file for part 3\\
% |cdocspt4.tex|&include file for part 4\\
% |cdocsdrf.tex|&forwarding file for main file in draft mode\\
% |cdocsfi1.tex|&forwarding file for final version of chapter 1\\
% |cdocsfi2.tex|&forwarding file for final version of chapter 2\\
% \end{tabular}
% \end{center}
% Each of the eight files can be compiled directly by the \LaTeX{} compiler.
%
% %%%%%%%%%%%%%%%%%%%%%%%%%%%%%%%%%%%%%%
% \paragraph{Main File.}
%
% The main file is called |cdocsamp.tex|.
%
% Load the \textsf{childdoc} definitions and
% declare the filename for the main document:
%    \begin{macrocode}
\input{childdoc.def}
\childdocmain{}
%    \end{macrocode}

% Optional override for |\version| flag:
%    \begin{macrocode}
%%\ifchilddoc\else\providecommand{\version}{draft}\fi
%    \end{macrocode}

% Define the default values for the |\version| flag
% (|final| for the main file and |draft| for childs):
%    \begin{macrocode}
\ifchilddoc
\providecommand{\version}{draft}
\else
\providecommand{\version}{final}
\fi
%    \end{macrocode}

% Load the standard document class:
%    \begin{macrocode}
\documentclass[12pt]{article}
%    \end{macrocode}

% Start the document body:
%    \begin{macrocode}
\begin{document}
%    \end{macrocode}

% Declare a title page.
% Print title, part of document being processed and version flag:
%    \begin{macrocode}
\addtocounter{page}{-1}
\begin{center}
{\LARGE\bfseries{}childdoc example\par}
\vspace{1cm}
\ifchilddoc
\ifchilddocmanual part\else chapter\fi:
`\childdocname' of `\childdocjob'\par
\else
main document: `\childdocjob'\par
\fi
version: \version\par
\end{center}
\newpage
%    \end{macrocode}

% Manually include selected file,
% otherwise process as usual:
%    \begin{macrocode}
\ifchilddocmanual
\section*{part `\childdocname'}
\input{\childdocname}
\else
%    \end{macrocode}

% Include the two chapters:
%    \begin{macrocode}
\include{cdocsch1}
\include{cdocsch2}
%    \end{macrocode}

% Include the two parts unless only chapters should be displayed:
%    \begin{macrocode}
\ifchilddoc\else
\section{part three}
\input{cdocspt3}
\section{part four}
\input{cdocspt4}
\fi
%    \end{macrocode}

% Process as usual until here:
%    \begin{macrocode}
\fi
%    \end{macrocode}

% End of document body:
%    \begin{macrocode}
\end{document}
%    \end{macrocode}
%\iffalse
%</samplemain>
%\fi
%
% %%%%%%%%%%%%%%%%%%%%%%%%%%%%%%%%%%%%%%
% \paragraph{Chapter Include Files.}
%
% The include files are called |cdocsch1.tex| and |cdocsch2.tex|.
%
%\iffalse
%<*samplechap1|samplechap2>
%\fi

% Optional override for |\version| flag:
%    \begin{macrocode}
%%\providecommand{\version}{final}
%    \end{macrocode}

% Include the main document:
%    \begin{macrocode}
\input{childdoc.def}
\childdocof{cdocsamp}
%    \end{macrocode}

%\iffalse
%</samplechap1|samplechap2>
%\fi
%
%\iffalse
%<*samplechap1>
%\fi
% Some text for chapter 1:
%    \begin{macrocode}
\section{one}
some text in chapter one
%    \end{macrocode}

%\iffalse
%</samplechap1>
%\fi
% Some text for chapter 2:
%\iffalse
%<*samplechap2>
%\fi
%    \begin{macrocode}
\section{two}
more text in chapter two
%    \end{macrocode}

%\iffalse
%</samplechap2>
%\fi
%
% %%%%%%%%%%%%%%%%%%%%%%%%%%%%%%%%%%%%%%
% \paragraph{Part Include Files.}
%
% The include files are called |cdocspt3.tex| and |cdocspt4.tex|.
%
%\iffalse
%<*samplepart3|samplepart4>
%\fi

% Optional override for |\version| flag:
%    \begin{macrocode}
%%\providecommand{\version}{final}
%    \end{macrocode}

% Include the main document:
%    \begin{macrocode}
\input{childdoc.def}
\childdocby{cdocsamp}
%    \end{macrocode}

%\iffalse
%</samplepart3|samplepart4>
%\fi
%
%\iffalse
%<*samplepart3>
%\fi
% Some text for part 3:
%    \begin{macrocode}
some text in part three
%    \end{macrocode}

%\iffalse
%</samplepart3>
%\fi
% Some text for part 4:
%\iffalse
%<*samplepart4>
%\fi
%    \begin{macrocode}
more text in part four
%    \end{macrocode}

%\iffalse
%</samplepart4>
%\fi
%
% %%%%%%%%%%%%%%%%%%%%%%%%%%%%%%%%%%%%%%
% \paragraph{Forwarding for a Complete Draft.}
%
% The following forwarding file |cdocsdrf.tex|
% compiles the main document in draft mode:
%\iffalse
%<*sampledraft>
%\fi
%    \begin{macrocode}
\def\version{draft}
\input{childdoc.def}
\childdocforward{cdocsamp}
%    \end{macrocode}

%\iffalse
%</sampledraft>
%\fi
%
% %%%%%%%%%%%%%%%%%%%%%%%%%%%%%%%%%%%%%%
% \paragraph{Forwarding for Final Version of the Chapters.}
%
% The following forwarding files |cdocsfn1.tex| and |cdocsfn2.tex|
% (with identical content)
% compile the final versions of the child documents
% |cdocsch1.tex| and |cdocsch2.tex|, respectively:
%\iffalse
%<*samplefinal>
%\fi
%    \begin{macrocode}
\def\version{final}
\input{childdoc.def}
\childdocforwardprefix[cdocsamp]{cdocsfn}{cdocsch}
%    \end{macrocode}

%\iffalse
%</samplefinal>
%\fi
%
% %%%%%%%%%%%%%%%%%%%%%%%%%%%%%%%%%%%%%%
% \paragraph{Command Line Processing.}
%
% The following three command lines generate the output files
% |cdocscld|, |cdocscl1| and |cdocscl2|
% which should be identical to
% |cdocsdrf|, |cdocsch1| and |cdocsfn2|, respectively:
% \begin{center}
% \begin{tabular}{l}
% |latex -jobname cdocscld \|\\
% |  "\def\version{draft}\input{childdoc.def}\childdocforward{cdocsamp}"|\\
% |latex -jobname cdocscl1 \|\\
% |  "\input{childdoc.def}\childdocforward[cdocsamp]{cdocsch1}"|\\
% |latex -jobname cdocscl2 \|\\
% |  "\def\version{final}\input{childdoc.def}\childdocforward{cdocsch2}"|
% \end{tabular}
% \end{center}
% Note that the trailing backslash on each first line
% merely continues the input to the second line
% (for convenient cut ant paste).
% Furthermore, the command |latex| can be replaced by any
% of its alternative versions such as |pdflatex|.
%
% %%%%%%%%%%%%%%%%%%%%%%%%%%%%%%%%%%%%%%%%%%%%%%%%%%%%%%%%%%%%%%%%%%%%%%%%%%%%%%
% %%%%%%%%%%%%%%%%%%%%%%%%%%%%%%%%%%%%%%%%%%%%%%%%%%%%%%%%%%%%%%%%%%%%%%%%%%%%%%
% \section{Implementation}
%\iffalse
%<*package>
%\fi
%
% This section describes the definitions file |childdoc.def|.

% The definitions cannot be loaded using |\usepackage| or |\RequirePackage|
% which has a mechanism to prevent loading a style file more than once.
% When loading the definitions by means of |\input|
% multiple instances have to be prevented manually:
%\iffalse
%This code needs to be before the `\ProvidesFile' directive
%which is defined at the beginning of this file.
%Therefore it is also placed there and commented out here.
%</package>
%<*discard>
%\fi
%    \begin{macrocode}
\ifdefined\childdocmain\endinput\fi
%    \end{macrocode}
%\iffalse
%</discard>
%<*package>
%\fi
%
% \macro{\ifchilddoc}
% \macro{\ifchilddocmanual}
% The conditional |\ifchilddoc| tells whether a
% child (true) or main (false) document is being compiled.
% The conditional |\ifchilddocmanual| tells whether
% the |\includeonly| mechanism is used (false) or
% the selection of child files must be performed manually (true).
% The definitions initialise to false:
%    \begin{macrocode}
\newif\ifchilddoc
\newif\ifchilddocmanual
%    \end{macrocode}

% \macro{\childdocname}
% \macro{\childdocjob}
% The macro |\childdocname| stores the name of the main document
% to be compiled. The macro |\childdocjob| stores the name of
% the document on which the \LaTeX{} compiler was originally invoked.
% The content of |\jobname| cannot be compared
% to filenames specified in the source due to different catcodes.
% The following code rescans |\jobname|, stores the result
% in |\childdocname| and saves a copy in |\childdocjob|:
%    \begin{macrocode}
\edef\childdocname{\scantokens\expandafter{\jobname\noexpand}}
\let\childdocjob\childdocname
%    \end{macrocode}

% \macro{\childdocdisable}
% The macro |\childdocdisable| prevents the main file
% from being processed more than once.
% At this stage, the main document command |\childdocmain|
% is assumed to be called once again where it should do nothing.
% Any subsequent call to it should prevent
% a secondary processing of the main document
% It overwrites the forwarding commands
% |\childdocof| and |\childdocforward|
% with empty macros to prevent further inclusions of the main document:
%    \begin{macrocode}
\newcommand{\childdocdisable}
{
  \renewcommand{\childdocmain}[1]{\renewcommand{\childdocmain}[1]{\endinput}}
  \renewcommand{\childdocof}[1]{}
  \renewcommand{\childdocby}[2][]{}
  \renewcommand{\childdocforward}[2][]{}
  \renewcommand{\childdocdisable}{}
}
%    \end{macrocode}

% \macro{\childdocmain}
% The macro |\childdocmain| is to be called at the top of the main file
% with nothing or the main filename (without extension) as argument.
% First, it breaks loops.
% If the argument is not empty and does not match |\childdocname|
% (which is set by the first inclusion of |childdoc.def|),
% |\ifchilddoc| is set to true, |\includeonly| is applied to the child file
% and |\jobname| is set to the main file
% (for proper handling of |.aux| files):
%    \begin{macrocode}
\newcommand{\childdocmain}[1]
{
  \childdocdisable\childdocmain{}
  \if?#1?\else
    \begingroup
      \def\childdoctmp{#1}
      \ifx\childdoctmp\childdocname
        \def\childdoctmp{}
      \else
        \def\childdoctmp
        {
          \childdoctrue
          \includeonly{\childdocname}
          \def\childdocjob{#1}
          \def\jobname{#1}
        }
      \fi
      \expandafter
    \endgroup
    \childdoctmp
  \fi
}
%    \end{macrocode}

% \macro{\childdocof}
% The command |\childdocof| redirects
% compilation to the main file |#1|.
%    \begin{macrocode}
\newcommand{\childdocof}[1]
{
  \childdocdisable
  \childdoctrue
  \includeonly{\childdocname}
  \def\jobname{#1}
  \def\childdocjob{#1}
  \input{#1}
}
%    \end{macrocode}

% \macro{\childdocby}
% The command |\childdocby| ....
%    \begin{macrocode}
\newcommand{\childdocby}[2][]
{
  \childdocdisable
  \childdoctrue
  \childdocmanualtrue
  \if?#1?\else
    \def\jobname{#2}
  \fi
  \def\childdocjob{#2}
  \input{#2}
  \endinput
}
%    \end{macrocode}

% \macro{\childdocforward}
% The command |\childdocforward| redirects
% compilation to the main file or
% (if the optional argument is given) a child file.
% Parameters are set as if the main file
% or a child file starting with |\childdocof| was compiled.
% Then compilation is handed over to the main file:
%    \begin{macrocode}
\newcommand{\childdocforward}[2][]
{
  \begingroup
    \if?#1?
      \def\childdoctmp
      {
        \def\childdocname{#2}
        \def\childdocjob{#2}
        \def\jobname{#2}
        \input{#2}
        \endinput
      }
    \else
      \def\childdoctmp
      {
        \childdocdisable
        \def\childdocname{#2}
        \childdoctrue
        \includeonly{#2}
        \def\childdocjob{#1}
        \def\jobname{#1}
        \input{#1}
        \endinput
      }
    \fi
    \expandafter
  \endgroup
  \childdoctmp
}
%    \end{macrocode}

% \macro{\childdocforwardprefix}
% The command |\childdocforwardprefix| redirects
% compilation to the main or a child file by means of a pattern.
% The prefix |#1| in the current filename is replaced by |#2|
% and the suffix of the current filename is kept
% (it is assumed that the filename does not contain the substring `|~~~|'
% which is used as a delimiter).
% Compilation is handed over to the new file by |\childdocforward|:
%    \begin{macrocode}
\newcommand{\childdocforwardprefix}[3][]
{
  \begingroup
    \def\childdocextract #2##1~~~{\def\childdoctmp{\childdocforward[#1]{#3##1}}}
    \expandafter\childdocextract\childdocname~~~
    \expandafter
  \endgroup
  \childdoctmp
}
%    \end{macrocode}

% \macro{\childdoc}
% The deprecated macro |\childdoc| is a legacy version of |\childdocmain|:
%    \begin{macrocode}
\newcommand{\childdoc}{\childdocmain}
%    \end{macrocode}

% \macro{\childdocredirect}
% The deprecated macro |\childdocredirect| is a legacy version
% of |\childdocforward| and |\childdocforwardprefix|:
%    \begin{macrocode}
\newcommand{\childdocredirect}[2][]
{
  \begingroup
    \if?#1?
      \def\childdoctmp{\childdocforward{#2}}
    \else
      \def\childdoctmp{\childdocforwardprefix{#1}{#2}}
    \fi
    \expandafter
  \endgroup
  \childdoctmp
}
%    \end{macrocode}

%\iffalse
%</package>
%\fi
%
\endinput
|\\
|\childdocby{|\textit{main}|}|\\
\end{tabular}
\end{center}
%
The directive |\childdocby| is similar to |\childdocof|
described in \secref{sec:include},
but the subsequent selection of content must be done manually.
To that end, both |\ifchilddoc| and |\ifchilddocmanual|
will be true upon processing of a part,
and the name of the part is stored in |\childdocname|.
Note that |\jobname| will be set to the filename of the current part
so that each part receives an individual |.aux| file
that does not interfere with the |.aux| file(s) of the main document.
This behaviour can be altered by the alternative form
|\childdocby[*]{|\textit{main}|}| (with a non-empty optional argument)
which uses the |.aux| file of the main document
by setting |\jobname| to \textit{main}.

%%%%%%%%%%%%%%%%%%%%%%%%%%%%%%%%%%%%%%%%%%%%%%%%%%%%%%%%%%%%%%%%%%%%%%%%%%%%%%%%
\subsection{Driver Development}
\label{sec:driver}

The \textsf{childdoc} mechanism can also be use for the development
of definition files such as \LaTeX{} styles or classes.
This case differs from the above setup with multiple parts
included by |\include| in that no |\includeonly| should be invoked.
This can be achieved by starting the include file
(before |\ProvidesPackage|) with:
%
\begin{center}
\begin{tabular}{l}
|% \iffalse
%
% childdoc.dtx Copyright (C) 2017-2018 Niklas Beisert
%
% This work may be distributed and/or modified under the
% conditions of the LaTeX Project Public License, either version 1.3
% of this license or (at your option) any later version.
% The latest version of this license is in
%   http://www.latex-project.org/lppl.txt
% and version 1.3 or later is part of all distributions of LaTeX
% version 2005/12/01 or later.
%
% This work has the LPPL maintenance status `maintained'.
%
% The Current Maintainer of this work is Niklas Beisert.
%
% This work consists of the files childdoc.dtx and childdoc.ins
% and the derived files childdoc.def and cdocsamp.tex with
% cdocsch1.tex, cdocsch2.tex, cdocsdrf.tex, cdocsfn1.tex, cdocsfn2.tex.
%
%<package>\ifdefined\childdocmain\endinput\fi
%<package>\ProvidesFile{childdoc.def}[2018/12/30 v2.0 child document driver]
%<samplemain>\ProvidesFile{cdocsamp.tex}[2018/12/30 v2.0 sample for childdoc]
%<*driver>
%\ProvidesFile{childdoc.drv}[2018/12/30 v2.0 childdoc reference manual file]
\PassOptionsToClass{10pt,a4paper}{article}
\documentclass{ltxdoc}

\usepackage[margin=35mm]{geometry}
\usepackage{hyperref}
\usepackage{hyperxmp}
\usepackage[usenames]{color}

\hypersetup{colorlinks=true}
\hypersetup{pdfstartview=FitH}
\hypersetup{pdfpagemode=UseNone}
\hypersetup{pdfsource={}}
\hypersetup{pdflang={en-UK}}
\hypersetup{pdfcopyright={Copyright 2017-2018 Niklas Beisert.
  This work may be distributed and/or modified under the
  conditions of the LaTeX Project Public License, either version 1.3
  of this license or (at your option) any later version.}}
\hypersetup{pdflicenseurl={http://www.latex-project.org/lppl.txt}}
\hypersetup{pdfcontactaddress={ETH Zurich, ITP, HIT K,
  Wolfgang-Pauli-Strasse 27}}
\hypersetup{pdfcontactpostcode={8093}}
\hypersetup{pdfcontactcity={Zurich}}
\hypersetup{pdfcontactcountry={Switzerland}}
\hypersetup{pdfcontactemail={nbeisert@itp.phys.ethz.ch}}
\hypersetup{pdfcontacturl={http://people.phys.ethz.ch/\xmptilde nbeisert/}}

\newcommand{\secref}[1]{\hyperref[#1]{section \ref*{#1}}}

\parskip1ex
\parindent0pt
\let\olditemize\itemize
\def\itemize{\olditemize\parskip0pt}

\begin{document}

\title{The \textsf{childdoc} Package}
\hypersetup{pdftitle={The childdoc Package}}
\author{Niklas Beisert\\[2ex]
  Institut f\"ur Theoretische Physik\\
  Eidgen\"ossische Technische Hochschule Z\"urich\\
  Wolfgang-Pauli-Strasse 27, 8093 Z\"urich, Switzerland\\[1ex]
  \href{mailto:nbeisert@itp.phys.ethz.ch}
  {\texttt{nbeisert@itp.phys.ethz.ch}}}
\hypersetup{pdfauthor={Niklas Beisert}}
\hypersetup{pdfsubject={Manual for the LaTeX2e Package childdoc}}
\date{30 December 2018, \textsf{v2.0}}
\maketitle

\begin{abstract}\noindent
\textsf{childdoc} is a \LaTeXe{} package
that enables the direct compilation
of document sections included by |\include|
to individual files.
\end{abstract}

\begingroup
\parskip0ex
\tableofcontents
\endgroup

%%%%%%%%%%%%%%%%%%%%%%%%%%%%%%%%%%%%%%%%%%%%%%%%%%%%%%%%%%%%%%%%%%%%%%%%%%%%%%%%
%%%%%%%%%%%%%%%%%%%%%%%%%%%%%%%%%%%%%%%%%%%%%%%%%%%%%%%%%%%%%%%%%%%%%%%%%%%%%%%%
\section{Introduction}

\LaTeX{} provides a mechanism to structure a large document (such as a book)
into a main file and several child files (containing the chapters)
using the |\include| command.
This mechanism is beneficial for documents
which span hundreds of pages in order to
make the source file(s) more manageable.
Moreover, compilation can be restricted to
selected child files by means of the |\includeonly| command.
The latter feature can be used to reduce the compilation time while editing
(this was significantly more useful in the earlier days of \LaTeX{})
or to generate a smaller document which is easier to navigate.
Another application of |\includeonly| is to generate
documents consisting of selected parts of the complete document.

However, there are a few drawbacks of the plain |\include| mechanism:
\begin{itemize}
\item
The child files cannot be compiled on their own,
they can only be compiled via the main file.
A naive editing environment
(such as a text editor with an option
to have the current file processed by \LaTeX)
may require one to switch to the main file before compiling;
attempting to compile the child file produces errors.
\item
The main file must be modified (each time)
to adjust the |\includeonly| command
to the present needs. This easily leaves the main file in a messy state.
\item
The generated document will always carry the filename
of the main document. This is inconvenient if
several child files are to be compiled and
to be kept for distribution.
\end{itemize}

The present package provides a simple interface
to make child files individually compilable by \LaTeX{}.
Compiling a child file then has the same effect as compiling
the main file with an |\includeonly| command
to select the appropriate child.
Moreover the generated document will carry the name of the child
rather than the main file.
This resolves all three above issues.

This feature is meant to make the editing of books,
thesis documents and lecture notes somewhat more convenient.
However, the package can also be used efficiently for
composing a series of documents (such as exercise sheets)
which are typically distributed individually.
It then assists the author in generating the individual documents
(potentially in different versions)
as well as a document containing the collected series.
Another application is in developing style files
or other kinds of included material
where compilation of the style file could redirect
to a sample or test file.

%%%%%%%%%%%%%%%%%%%%%%%%%%%%%%%%%%%%%%%%%%%%%%%%%%%%%%%%%%%%%%%%%%%%%%%%%%%%%%%%
%%%%%%%%%%%%%%%%%%%%%%%%%%%%%%%%%%%%%%%%%%%%%%%%%%%%%%%%%%%%%%%%%%%%%%%%%%%%%%%%
\section{Usage}

First of all, the package \textsf{childdoc} is \emph{not} a standard
\LaTeXe{} |.sty| style file! Therefore it needs to be invoked in
a non-standard way.

%%%%%%%%%%%%%%%%%%%%%%%%%%%%%%%%%%%%%%%%%%%%%%%%%%%%%%%%%%%%%%%%%%%%%%%%%%%%%%%%
\subsection{Included Files}
\label{sec:include}

%%%%%%%%%%%%%%%%%%%%%%%%%%%%%%%%%%%%%%%%
\DescribeMacro{\childdocmain}
To use the package, add the commands
\begin{center}
\begin{tabular}{l}
|\input{childdoc.def}|\\
|\childdocmain{}|\\
\end{tabular}
\end{center}
at the very top of the main \LaTeX{} file,
in particular \emph{before} the |\documentclass| statement!
The argument of |\childdocmain| should be left empty
(but it must be present).

%%%%%%%%%%%%%%%%%%%%%%%%%%%%%%%%%%%%%%%%
\DescribeMacro{\childdocof}
Furthermore, add the commands
\begin{center}
\begin{tabular}{l}
|\input{childdoc.def}|\\
|\childdocof{|\textit{main}|}|\\
\end{tabular}
\end{center}
at the top of every child file \textit{child}
which is included by |\include{|\textit{child}|}|
from within the main file
(or at least for those files to be compiled individually).
The argument \textit{main} must be the filename of the main file.

There are a couple of
considerations in setting up the main and child documents:

%%%%%%%%%%%%%%%%%%%%%%%%%%%%%%%%%%%%%%%%
\paragraph{Restrictions.}

Please note the following restrictions:
\begin{itemize}
\item
|\childdocmain| must be called with one argument \textit{main}
to ensure compatibility with earlier version of the package.
It must either be empty (|\childdocmain{}|)
or precisely match the filename of the main file in which it is specified.
See \secref{sec:detection} for further information.
\item
The filename \textit{main} must be specified without the |.tex| extension.
\item
The filename \textit{main} is case sensitive
(even in case-insensitive file systems)
due to internal string comparison.
\item
The argument \textit{main} should be fully expanded, it cannot be a macro.
\item
Subdirectories and special characters should be avoided in filenames.
\item
The command |\childdocmain{|\textit{main}|}| must be followed by a whitespace.
It should not be followed immediately by another command
or by a comment mark `|%|'.
This is because the \TeX{} parser reads the token immediately following
the argument of |\childdocmain| and puts it
at the beginning of every child section;
however, a white\-space is ignored.
\end{itemize}

%%%%%%%%%%%%%%%%%%%%%%%%%%%%%%%%%%%%%%%%
\paragraph{Content of Main File.}

It is advisable to place all content in the child files included by |\include|.
Any output contained in the main file will appear in all child documents
unless suppressed manually;
it cannot be suppressed automatically by the |\includeonly| directive
and thus should normally be avoided.
A method to include some content in the main file
by means of conditional processing is described in \secref{sec:conditional}.

%%%%%%%%%%%%%%%%%%%%%%%%%%%%%%%%%%%%%%%%
\paragraph{Page Numbering.}

When only a part of the document is compiled,
the appropriate numbering of pages
(as well as other status parameters)
is determined from the |.aux| files.
The latter contain information from previous passes.
However this information needs to propagate through
all intermediate child documents.
Therefore the page numbering in child documents may well
be inconsistent until the complete document is compiled at least once.

A useful (if unconventional) way to always ensure a consistent
page numbering is to restart the numbering in each child document
and denote the pages by `\textit{child}|.|\textit{page}'
where \textit{child} represents the chapter/section number of the child file.
This can be achieved by the command
|\numberwithin{page}{|\textit{child}|}|
of the \textsf{amsmath} package
where \textit{child} can be |chapter| or |section|
depending on the chosen structuring.
Alternatively, one can modify the macro |\thepage| appropriately
and reset the counter |page| at the start of each child file.

%%%%%%%%%%%%%%%%%%%%%%%%%%%%%%%%%%%%%%%%%%%%%%%%%%%%%%%%%%%%%%%%%%%%%%%%%%%%%%%%
\subsection{Conditional Processing}
\label{sec:conditional}

The package provides a mechanism to compile different versions
of a document. To customise the versions further some conditional processing
can come in handy to distinguish which version is being compiled.
The package provides two macros to describe the compilation context:

%%%%%%%%%%%%%%%%%%%%%%%%%%%%%%%%%%%%%%%%
\DescribeMacro{\ifchilddoc}
The conditional |\ifchilddoc| distinguishes between the compilation of
child documents and the main document:
%
\begin{center}
|\ifchilddoc |\textit{child-code}| |[|\||else |\textit{main-code}]| \||fi|
\end{center}

%%%%%%%%%%%%%%%%%%%%%%%%%%%%%%%%%%%%%%%%
\DescribeMacro{\childdocname}
\DescribeMacro{\childdocjob}
The macro |\childdocname| contains the filename (without extension)
of the main or child file being processed.
Note that |\childdocjob| will always contain the name of the main file.

%%%%%%%%%%%%%%%%%%%%%%%%%%%%%%%%%%%%%%%%
\paragraph{Title Page.}

Conditional processing can be used to include a title or banner page
in the main document when proper precautions are taken.
Importantly, the code in the main file should ensure that the page counter
(as well as other status parameters which are stored in the |.aux| files)
takes the same value after the conditional processing.
Otherwise the page numbers may take divergent values
depending on which part is compiled.

For example, a title page could be declared by:
%
\begin{center}
\begin{tabular}{l}
|\ifchilddoc\||else|\\
|\addtocounter{page}{-1}|\\
\textit{code for title page}\\
|\newpage|\\
|\||fi|
\end{tabular}
\end{center}
%
A banner page for the child documents can be generated by:
%
\begin{center}
\begin{tabular}{l}
|\ifchilddoc|\\
|\addtocounter{page}{-1}|\\
\textit{code for banner page}\\
|\newpage|\\
|\||fi|
\end{tabular}
\end{center}
%
Here one could write a message such as:
\begin{center}
|This is the part \childdocname{} of \childdocjob{}.|
\end{center}

%%%%%%%%%%%%%%%%%%%%%%%%%%%%%%%%%%%%%%%%%%%%%%%%%%%%%%%%%%%%%%%%%%%%%%%%%%%%%%%%
\subsection{Flags}
\label{sec:flags}

The package makes it easy to generate different versions
of the main or child documents.
To this end compilation flags can be defined
and assigned different default values.
They will be particularly useful in conjunction
with the forwarding mechanism described in \secref{sec:forward}.

For example, it may be useful to have a flag |\version|
which can be set to |draft| or |final|.
The document source will contain some conditional code
depending on the value of |\version|.
Suppose further, the flag should default to |final| for the main file
and to |draft| for child files
which is a natural assignment for editing the document.
This is achieved by placing the following code
in the preamble of the main document
(below the |\childdocmain| directive):
%
\begin{center}
\begin{tabular}{l}
|\ifchilddoc|\\
|\providecommand{\version}{draft}|\\
|\||else|\\
|\providecommand{\version}{final}|\\
|\||fi|
\end{tabular}
\end{center}
%
The definition by |\providecommand| makes sure
that previous definitions are not overwritten.
Further statements |\providecommand{\version}{...}|
can thus be added before the above code to override it.

For the main file, one might add a line
(between |\childdocmain| and the above block)
%
\begin{center}
|%\ifchilddoc\||else\providecommand{\version}{draft}\||fi|
\end{center}
%
which can be uncommented to produce a draft version.
Likewise one can add a line to the very top of a child file
(above the |\childdocof{|\textit{main}|}| directive)
%
\begin{center}
|%\providecommand{\version}{final}|
\end{center}
%
which can be uncommented to produce the final version of this child document.

%%%%%%%%%%%%%%%%%%%%%%%%%%%%%%%%%%%%%%%%%%%%%%%%%%%%%%%%%%%%%%%%%%%%%%%%%%%%%%%%
\subsection{Forwarding}
\label{sec:forward}

Different versions of the main or child documents
using compilation flags as described in \secref{sec:flags}
can be (permanently) stored in different files
for convenient compilation, viewing and distribution.
To this end, the package defines a command
to pass on compilation to a different file:

%%%%%%%%%%%%%%%%%%%%%%%%%%%%%%%%%%%%%%%%
\DescribeMacro{\childdocforward}
The command |\childdocforward| redirects processing to
another source file:
%
\begin{center}
\begin{tabular}{l}
|\input{childdoc.def}|\\
|\childdocforward[|\textit{main}|]{|\textit{dest}|}|\\
\end{tabular}
\end{center}
%
The argument \textit{dest} is the destination file
(without extension).
It should be the main file or one of the child files.
Note that further \textsf{childdoc} directives
such as |\childdocof| and |\childdocforward|
in the indicated file will be processed in this form.
The optional argument \textit{main}
passes on directly to the main file \textit{main}
while pretending to compile the child \textit{dest}.
This form behaves as if \textit{dest}
issues |\childdocof{|\textit{main}|}| right away,
and no further \textsf{childdoc} directives will be processed.

%%%%%%%%%%%%%%%%%%%%%%%%%%%%%%%%%%%%%%%%
\DescribeMacro{\...prefix}
In the alternative form |\childdocforwardprefix|,
%
\begin{center}
\begin{tabular}{l}
|\input{childdoc.def}|\\
|\childdocforwardprefix[|\textit{main}|]{|\textit{prefix}|}{|\textit{dest}|}|
\end{tabular}
\end{center}
%
the destination file is determined by a pattern
depending on the current file:
To make this work, the current file must be called
`{\textit{prefix}\hspace{0.2em}\textit{suffix}}'
with \textit{prefix} matching precisely the argument.
Processing is then passed on to the file
`{\textit{dest}\hspace{0.2em}\textit{suffix}}'.
Surely, the same effect is achieved by
directly specifying the
argument `{\textit{dest}\hspace{0.2em}\textit{suffix}}'
in the first form.
However, that requires to set up a different file
for each child. With the alternative form of the command
all these files can have exactly the same content
which simplifies setting them up and maintaining them.

For example, the following file |draft.tex|
with a compilation flag |\version| as described in \secref{sec:flags}
compiles the main document as a draft:
%
\begin{center}
\begin{tabular}{l}
|\def\version{draft}|\\
|\input{childdoc.def}|\\
|\childdocforward{|\textit{main}|}|
\end{tabular}
\end{center}
%
Likewise, the following files |final|\textit{nn}|.tex|
compile the final version of the child document
|child|\textit{nn}|.tex|:
%
\begin{center}
\begin{tabular}{l}
|\def\version{final}|\\
|\input{childdoc.def}|\\
|\childdocforwardprefix{final}{child}|
\end{tabular}
\end{center}
%

Note that when several versions of a main file and/or of each child file
are to be generated, it may be convenient to set up a |Makefile| or
shell script to automatise the process.

%%%%%%%%%%%%%%%%%%%%%%%%%%%%%%%%%%%%%%%%%%%%%%%%%%%%%%%%%%%%%%%%%%%%%%%%%%%%%%%%
\subsection{Command Line Processing}
\label{sec:commandline}

The effect of redirection files can also be achieved by invoking
the \LaTeX{} compiler with a more elaborate command line.
Most conveniently this should be done as part
of a shell script or a |Makefile|.

When using \textsf{childdoc} in the main file, the following
command lines effectively perform a redirection
(note that depending on the shell being used,
backslashes may have to be doubled: `|\|' $\to$ `|\\|'):
%
\begin{center}
|... -jobname "|\textit{target}|" |\\|"|[\textit{flags}]%
|\input{childdoc.def}\childdocforward[|\textit{main}|]{|\textit{dest}|}"|
\end{center}
%
Here \textit{target} is the name of the output file,
\textit{main} is the name of the main file
and \textit{dest} is the name of the main or child file to be processed
(all filenames without extensions).
The optional argument \textit{main} can be omitted
if \textit{main} matches \textit{dest}.
Optionally, compilation \textit{flags} can be defined via |\def| commands.
This command line makes the \TeX{} engine believe
it is compiling the file \textit{target}
whose content is specified as the latter parameter.
The provided code then forwards the processing to
\textit{main} or \textit{dest} as described in \secref{sec:forward}.

%%%%%%%%%%%%%%%%%%%%%%%%%%%%%%%%%%%%%%%%%%%%%%%%%%%%%%%%%%%%%%%%%%%%%%%%%%%%%%%%
\subsection{Include by Input}
\label{sec:input}

Including child documents by |\include| has some restrictions by design.
Most notably, the content of a child document always occupies
its own set of pages; pages cannot be shared between child documents.
Usually, this behaviour makes perfect sense
because each child document contain an essential part of the document.
However, in some situations it may be desirable to compose
a document from a collection of parts
without having mandatory page breaks between then.
For this case, the package
provides a mechanism to include parts
by |\input| which can also be processed individually.
However, by construction this mechanism
requires manual handling of the content to be output.

%%%%%%%%%%%%%%%%%%%%%%%%%%%%%%%%%%%%%%%%
\DescribeMacro{\ifchilddocmanual}
The main file should be prepared as usual, see \secref{sec:include}.
However, the document body must make a distinction
between processing of an individual part and of the main document, e.g.:
%
\begin{center}
\begin{tabular}{l}
|\ifchilddocmanual|\\
|\input{\childdocname}|\\
|\||else|\\
\textit{document body with }|\input{|\textit{part}|}|\\
|\||fi|
\end{tabular}
\end{center}
%
The conditional |\ifchilddocmanual| is true whenever
a part to be included by |\input| is being compiled,
and the name of the part is stored in |\childdocname|.

%%%%%%%%%%%%%%%%%%%%%%%%%%%%%%%%%%%%%%%%
\DescribeMacro{\childdocby}
Each part to be included by |\input| should start with:
%
\begin{center}
\begin{tabular}{l}
|\input{childdoc.def}|\\
|\childdocby{|\textit{main}|}|\\
\end{tabular}
\end{center}
%
The directive |\childdocby| is similar to |\childdocof|
described in \secref{sec:include},
but the subsequent selection of content must be done manually.
To that end, both |\ifchilddoc| and |\ifchilddocmanual|
will be true upon processing of a part,
and the name of the part is stored in |\childdocname|.
Note that |\jobname| will be set to the filename of the current part
so that each part receives an individual |.aux| file
that does not interfere with the |.aux| file(s) of the main document.
This behaviour can be altered by the alternative form
|\childdocby[*]{|\textit{main}|}| (with a non-empty optional argument)
which uses the |.aux| file of the main document
by setting |\jobname| to \textit{main}.

%%%%%%%%%%%%%%%%%%%%%%%%%%%%%%%%%%%%%%%%%%%%%%%%%%%%%%%%%%%%%%%%%%%%%%%%%%%%%%%%
\subsection{Driver Development}
\label{sec:driver}

The \textsf{childdoc} mechanism can also be use for the development
of definition files such as \LaTeX{} styles or classes.
This case differs from the above setup with multiple parts
included by |\include| in that no |\includeonly| should be invoked.
This can be achieved by starting the include file
(before |\ProvidesPackage|) with:
%
\begin{center}
\begin{tabular}{l}
|\input{childdoc.def}|\\
|\childdocforward{|\textit{main}|}|\\
\end{tabular}
\end{center}
%
or alternatively with:
%
\begin{center}
\begin{tabular}{l}
|\input{childdoc.def}|\\
|\childdocby{|\textit{main}|}|\\
\end{tabular}
\end{center}
%
Both forms have slightly different effects as described above.
The main file is prepared as usual, see \secref{sec:include}.

%%%%%%%%%%%%%%%%%%%%%%%%%%%%%%%%%%%%%%%%%%%%%%%%%%%%%%%%%%%%%%%%%%%%%%%%%%%%%%%%
\subsection{Legacy Detection}
\label{sec:detection}

The directive |\childdocmain| in the main file can detect
whether the complete document or merely a child is to be compiled
even without using the directive |\childdocof|.
This method is deprecated because it is less robust
and there is no compelling reason to use it;
it is merely provided for backward compatibility
and it may be removed in future versions.

If the detection mechanism is to be used,
it is mandatory to correctly specify
the filename of the main file as the argument of |\childdocmain|:
%
\begin{center}
\begin{tabular}{l}
|\input{childdoc.def}|\\
|\childdocmain{|\textit{main}|}|\\
\end{tabular}
\end{center}
%
If |\jobname| does not match the argument \textit{main} of |\childdocmain|,
it is assumed that |\jobname| points to the child file to be compiled.
When using |\childdocmain| with the main file specified as argument,
it suffices to start a child file
with just |\input{|\textit{main}|}|
without loading of the package and using |\childdocof|.
If instead all processing is done
with the appropriate \textsf{childdoc} directives,
the argument of \textit{main} of |\childdocmain| can be empty.

An alternative version of the command line processing described
in \secref{sec:commandline} using the detection mechanism reads:
%
\begin{center}
|... -jobname "|\textit{target}|" "|[\textit{flags}]%
[|\def\jobname{|\textit{dest}|}|]|\input{|\textit{main}|}"|
\end{center}

%%%%%%%%%%%%%%%%%%%%%%%%%%%%%%%%%%%%%%%%%%%%%%%%%%%%%%%%%%%%%%%%%%%%%%%%%%%%%%%%
\subsection{Manual Code}
\label{sec:manual}

In case one cannot be certain whether the definitions file |childdoc.def|
is installed on the target \TeX{} distribution
and one prefers not to ship it,
it is conceivable to paste a few relevant commands into the sources.

To that end, drop all statements |\input{childdoc.def}|
and perform the replacements as outlined below.
Instead of |\childdocmain{|\textit{main}|}| add the following code
to the top of the main file:
%
\begin{center}
\begin{tabular}{l}
|\||ifdefined\childdocname\endinput\||fi\newif\ifchilddoc|\\
|\edef\childdocname{\scantokens\expandafter{\jobname\noexpand}}|\\
|\def\childdocmain{|\textit{main}|}\||ifx\childdocmain\childdocname\||else|\\
|\childdoctrue\includeonly{\childdocname}\let\jobname\childdocmain\||fi|\\
\end{tabular}
\end{center}
%
Instead of |\childdocof{|\textit{main}|}| just include the main file
at the top of each child file:
%
\begin{center}
|\input{|\textit{main}|}|
\end{center}
%
A simple redirection |\childdocforward{|\textit{dest}|}| is achieved by:
%
\begin{center}
|\def\jobname{|\textit{dest}|}\input{\jobname}|
\end{center}
%
The redirection with prefix
|\childdocforwardprefix[|\textit{prefix}|]{|\textit{dest}|}|
is accomplished by:
%
\begin{center}
\begin{tabular}{l}
|{\edef\jobname{\scantokens\expandafter{\jobname\noexpand}}|\\
|\def\redirectjob |\textit{prefix}|#1~~~{\gdef\jobname{|\textit{dest}|#1}}|\\
|\expandafter\redirectjob\jobname~~~}\input{\jobname}|
\end{tabular}
\end{center}

In an alternative approach,
child documents can be compiled by a specific command line
without additional code or specific definitions:
%
\begin{center}
|... -jobname "|\textit{target}|" "|[\textit{flags}]%
|\includeonly{|\textit{dest}|}\input{|\textit{main}|}"|
\end{center}
%

%%%%%%%%%%%%%%%%%%%%%%%%%%%%%%%%%%%%%%%%%%%%%%%%%%%%%%%%%%%%%%%%%%%%%%%%%%%%%%%%
%%%%%%%%%%%%%%%%%%%%%%%%%%%%%%%%%%%%%%%%%%%%%%%%%%%%%%%%%%%%%%%%%%%%%%%%%%%%%%%%
\section{Information}

%%%%%%%%%%%%%%%%%%%%%%%%%%%%%%%%%%%%%%%%%%%%%%%%%%%%%%%%%%%%%%%%%%%%%%%%%%%%%%%%
\subsection{Copyright}

Copyright \copyright{} 2017--2018 Niklas Beisert

This work may be distributed and/or modified under the
conditions of the \LaTeX{} Project Public License, either version 1.3
of this license or (at your option) any later version.
The latest version of this license is in
  \url{http://www.latex-project.org/lppl.txt}
and version 1.3 or later is part of all distributions of \LaTeX{}
version 2005/12/01 or later.

This work has the LPPL maintenance status `maintained'.

The Current Maintainer of this work is Niklas Beisert.

This work consists of the files |README.txt|, |childdoc.ins| and |childdoc.dtx|
as well as the derived files |childdoc.def|, |cdocsamp.tex|
with |cdocsch1.tex|, |cdocsch2.tex|, |cdocspt3.tex|, |cdocspt4.tex|,
|cdocsdrf.tex|, |cdocsfn1.tex|, |cdocsfn2.tex|
as well as |childdoc.pdf|.

%%%%%%%%%%%%%%%%%%%%%%%%%%%%%%%%%%%%%%%%%%%%%%%%%%%%%%%%%%%%%%%%%%%%%%%%%%%%%%%%
\subsection{Files and Installation}

The package consists of the files:
%
\begin{center}
\begin{tabular}{ll}
    |README.txt|   & readme file \\
    |childdoc.ins| & installation file \\
    |childdoc.dtx| & source file \\
    |childdoc.def| & definition file \\
    |cdocsamp.tex| & sample main file \\
    |cdocsch1.tex| & sample include file \\
    |cdocsch2.tex| & sample include file \\
    |cdocspt3.tex| & sample part file \\
    |cdocspt4.tex| & sample part file \\
    |cdocsdrf.tex| & sample redirection file \\
    |cdocsfn1.tex| & sample redirection file \\
    |cdocsfn2.tex| & sample redirection file \\
    |childdoc.pdf| & manual
\end{tabular}
\end{center}
%
The distribution consists of the files
|README.txt|, |childdoc.ins| and |childdoc.dtx|.
%
\begin{itemize}
\item
Run (pdf)\LaTeX{} on |childdoc.dtx|
to compile the manual |childdoc.pdf| (this file).
\item
Run \LaTeX{} on |childdoc.ins| to create the definitions file |childdoc.def|
and the sample |cdocsamp.tex| with include files
|cdocsch1.tex|, |cdocsch2.tex|, |cdocspt3.tex|, |cdocspt4.tex|,
|cdocsdrf.tex|, |cdocsfn1.tex|, |cdocsfn2.tex|.
Then copy the file |childdoc.def| to an appropriate directory of your \LaTeX{}
distribution, e.g.\ \textit{texmf-root}|/tex/latex/childdoc|.
\end{itemize}

%%%%%%%%%%%%%%%%%%%%%%%%%%%%%%%%%%%%%%%%%%%%%%%%%%%%%%%%%%%%%%%%%%%%%%%%%%%%%%%%
\subsection{Related CTAN Packages}

There are several other packages which offer a similar functionality:
%
\begin{itemize}
\item
The packages
\href{http://ctan.org/pkg/docmute}{\textsf{docmute}},
\href{http://ctan.org/pkg/includex}{\textsf{includex}} and
\href{http://ctan.org/pkg/standalone}{\textsf{standalone}}
provide commands to include only the document body of
a child file thus allowing both files to be compiled individually.
\item
The packages \href{http://ctan.org/pkg/subdocs}{\textsf{subdocs}}
and \href{http://ctan.org/pkg/subfiles}{\textsf{subfiles}}
provide structures in which the main and child documents can be
encapsulated and allowing them to be compiled individually.
The inclusion mechanism is different from the conventional |\include|.
\item
The package \href{http://ctan.org/pkg/combine}{\textsf{combine}}
is an elaborate solution to combine several documents into one.
\end{itemize}
%
See also the CTAN topic \href{http://ctan.org/topic/subdocs}{\textsf{subdocs}}
for further related packages.
The present package differs from the above solutions in that
a document structure constructed with the conventional |\include| mechanism
just needs two extra commands at the top of every file
such that all constituent files can be compiled individually.

%%%%%%%%%%%%%%%%%%%%%%%%%%%%%%%%%%%%%%%%%%%%%%%%%%%%%%%%%%%%%%%%%%%%%%%%%%%%%%%%
%\subsection{Feature Suggestions}
%
%The following is a list of features which may be useful for future
%versions of this package:
%%
%\begin{itemize}
%\item
%\ldots
%\end{itemize}

%%%%%%%%%%%%%%%%%%%%%%%%%%%%%%%%%%%%%%%%%%%%%%%%%%%%%%%%%%%%%%%%%%%%%%%%%%%%%%%%
\subsection{Revision History}

%%%%%%%%%%%%%%%%%%%%%%%%%%%%%%%%%%%%%%%%
\paragraph{v2.0:} 2018/12/30

\begin{itemize}
\item
immediate forward processing
\item
added |\childdocby| mechanism
\item
manual restructured
\end{itemize}

%%%%%%%%%%%%%%%%%%%%%%%%%%%%%%%%%%%%%%%%
\paragraph{v1.6:} 2018/01/17

\begin{itemize}
\item
application for development of include files
\item
corrections to manual
\end{itemize}

%%%%%%%%%%%%%%%%%%%%%%%%%%%%%%%%%%%%%%%%
\paragraph{v1.5:} 2017/05/21

\begin{itemize}
\item
more complete structuring introduced
\item
|\childdocof| introduced
\item
|\childdoc| renamed to |\childdocmain|
\item
|\childredirect| renamed to |\childdocforward| and |\childdocforwardprefix|
and functionality expanded
\end{itemize}

%%%%%%%%%%%%%%%%%%%%%%%%%%%%%%%%%%%%%%%%
\paragraph{v1.0:} 2017/04/27

\begin{itemize}
\item
manual and install package
\item
first version published on CTAN
\end{itemize}

%%%%%%%%%%%%%%%%%%%%%%%%%%%%%%%%%%%%%%%%
\paragraph{v0.6:} 2017/04/26

\begin{itemize}
\item
redirection mechanism added
\end{itemize}

%%%%%%%%%%%%%%%%%%%%%%%%%%%%%%%%%%%%%%%%
\paragraph{v0.5:} 2017/04/26

\begin{itemize}
\item
functionality in definition file
\end{itemize}


%%%%%%%%%%%%%%%%%%%%%%%%%%%%%%%%%%%%%%%%%%%%%%%%%%%%%%%%%%%%%%%%%%%%%%%%%%%%%%%%
%%%%%%%%%%%%%%%%%%%%%%%%%%%%%%%%%%%%%%%%%%%%%%%%%%%%%%%%%%%%%%%%%%%%%%%%%%%%%%%%
%%%%%%%%%%%%%%%%%%%%%%%%%%%%%%%%%%%%%%%%%%%%%%%%%%%%%%%%%%%%%%%%%%%%%%%%%%%%%%%%
\appendix

\settowidth\MacroIndent{\rmfamily\scriptsize 000\ }

 \DocInput{childdoc.dtx}

\end{document}
%</driver>
% \fi
%
% %%%%%%%%%%%%%%%%%%%%%%%%%%%%%%%%%%%%%%%%%%%%%%%%%%%%%%%%%%%%%%%%%%%%%%%%%%%%%%
% %%%%%%%%%%%%%%%%%%%%%%%%%%%%%%%%%%%%%%%%%%%%%%%%%%%%%%%%%%%%%%%%%%%%%%%%%%%%%%
% \section{Sample}
%\iffalse
%<*samplemain>
%\fi
%
% The following presents a sample document
% with two chapters, two parts, a title page,
% a compile flag as well as three forwarding files to set the flag.
% It consists of eight |.tex| files:
% \begin{center}
% \begin{tabular}{ll}
% |cdocsamp.tex|&main file\\
% |cdocsch1.tex|&include file for chapter 1\\
% |cdocsch2.tex|&include file for chapter 2\\
% |cdocspt3.tex|&include file for part 3\\
% |cdocspt4.tex|&include file for part 4\\
% |cdocsdrf.tex|&forwarding file for main file in draft mode\\
% |cdocsfi1.tex|&forwarding file for final version of chapter 1\\
% |cdocsfi2.tex|&forwarding file for final version of chapter 2\\
% \end{tabular}
% \end{center}
% Each of the eight files can be compiled directly by the \LaTeX{} compiler.
%
% %%%%%%%%%%%%%%%%%%%%%%%%%%%%%%%%%%%%%%
% \paragraph{Main File.}
%
% The main file is called |cdocsamp.tex|.
%
% Load the \textsf{childdoc} definitions and
% declare the filename for the main document:
%    \begin{macrocode}
\input{childdoc.def}
\childdocmain{}
%    \end{macrocode}

% Optional override for |\version| flag:
%    \begin{macrocode}
%%\ifchilddoc\else\providecommand{\version}{draft}\fi
%    \end{macrocode}

% Define the default values for the |\version| flag
% (|final| for the main file and |draft| for childs):
%    \begin{macrocode}
\ifchilddoc
\providecommand{\version}{draft}
\else
\providecommand{\version}{final}
\fi
%    \end{macrocode}

% Load the standard document class:
%    \begin{macrocode}
\documentclass[12pt]{article}
%    \end{macrocode}

% Start the document body:
%    \begin{macrocode}
\begin{document}
%    \end{macrocode}

% Declare a title page.
% Print title, part of document being processed and version flag:
%    \begin{macrocode}
\addtocounter{page}{-1}
\begin{center}
{\LARGE\bfseries{}childdoc example\par}
\vspace{1cm}
\ifchilddoc
\ifchilddocmanual part\else chapter\fi:
`\childdocname' of `\childdocjob'\par
\else
main document: `\childdocjob'\par
\fi
version: \version\par
\end{center}
\newpage
%    \end{macrocode}

% Manually include selected file,
% otherwise process as usual:
%    \begin{macrocode}
\ifchilddocmanual
\section*{part `\childdocname'}
\input{\childdocname}
\else
%    \end{macrocode}

% Include the two chapters:
%    \begin{macrocode}
\include{cdocsch1}
\include{cdocsch2}
%    \end{macrocode}

% Include the two parts unless only chapters should be displayed:
%    \begin{macrocode}
\ifchilddoc\else
\section{part three}
\input{cdocspt3}
\section{part four}
\input{cdocspt4}
\fi
%    \end{macrocode}

% Process as usual until here:
%    \begin{macrocode}
\fi
%    \end{macrocode}

% End of document body:
%    \begin{macrocode}
\end{document}
%    \end{macrocode}
%\iffalse
%</samplemain>
%\fi
%
% %%%%%%%%%%%%%%%%%%%%%%%%%%%%%%%%%%%%%%
% \paragraph{Chapter Include Files.}
%
% The include files are called |cdocsch1.tex| and |cdocsch2.tex|.
%
%\iffalse
%<*samplechap1|samplechap2>
%\fi

% Optional override for |\version| flag:
%    \begin{macrocode}
%%\providecommand{\version}{final}
%    \end{macrocode}

% Include the main document:
%    \begin{macrocode}
\input{childdoc.def}
\childdocof{cdocsamp}
%    \end{macrocode}

%\iffalse
%</samplechap1|samplechap2>
%\fi
%
%\iffalse
%<*samplechap1>
%\fi
% Some text for chapter 1:
%    \begin{macrocode}
\section{one}
some text in chapter one
%    \end{macrocode}

%\iffalse
%</samplechap1>
%\fi
% Some text for chapter 2:
%\iffalse
%<*samplechap2>
%\fi
%    \begin{macrocode}
\section{two}
more text in chapter two
%    \end{macrocode}

%\iffalse
%</samplechap2>
%\fi
%
% %%%%%%%%%%%%%%%%%%%%%%%%%%%%%%%%%%%%%%
% \paragraph{Part Include Files.}
%
% The include files are called |cdocspt3.tex| and |cdocspt4.tex|.
%
%\iffalse
%<*samplepart3|samplepart4>
%\fi

% Optional override for |\version| flag:
%    \begin{macrocode}
%%\providecommand{\version}{final}
%    \end{macrocode}

% Include the main document:
%    \begin{macrocode}
\input{childdoc.def}
\childdocby{cdocsamp}
%    \end{macrocode}

%\iffalse
%</samplepart3|samplepart4>
%\fi
%
%\iffalse
%<*samplepart3>
%\fi
% Some text for part 3:
%    \begin{macrocode}
some text in part three
%    \end{macrocode}

%\iffalse
%</samplepart3>
%\fi
% Some text for part 4:
%\iffalse
%<*samplepart4>
%\fi
%    \begin{macrocode}
more text in part four
%    \end{macrocode}

%\iffalse
%</samplepart4>
%\fi
%
% %%%%%%%%%%%%%%%%%%%%%%%%%%%%%%%%%%%%%%
% \paragraph{Forwarding for a Complete Draft.}
%
% The following forwarding file |cdocsdrf.tex|
% compiles the main document in draft mode:
%\iffalse
%<*sampledraft>
%\fi
%    \begin{macrocode}
\def\version{draft}
\input{childdoc.def}
\childdocforward{cdocsamp}
%    \end{macrocode}

%\iffalse
%</sampledraft>
%\fi
%
% %%%%%%%%%%%%%%%%%%%%%%%%%%%%%%%%%%%%%%
% \paragraph{Forwarding for Final Version of the Chapters.}
%
% The following forwarding files |cdocsfn1.tex| and |cdocsfn2.tex|
% (with identical content)
% compile the final versions of the child documents
% |cdocsch1.tex| and |cdocsch2.tex|, respectively:
%\iffalse
%<*samplefinal>
%\fi
%    \begin{macrocode}
\def\version{final}
\input{childdoc.def}
\childdocforwardprefix[cdocsamp]{cdocsfn}{cdocsch}
%    \end{macrocode}

%\iffalse
%</samplefinal>
%\fi
%
% %%%%%%%%%%%%%%%%%%%%%%%%%%%%%%%%%%%%%%
% \paragraph{Command Line Processing.}
%
% The following three command lines generate the output files
% |cdocscld|, |cdocscl1| and |cdocscl2|
% which should be identical to
% |cdocsdrf|, |cdocsch1| and |cdocsfn2|, respectively:
% \begin{center}
% \begin{tabular}{l}
% |latex -jobname cdocscld \|\\
% |  "\def\version{draft}\input{childdoc.def}\childdocforward{cdocsamp}"|\\
% |latex -jobname cdocscl1 \|\\
% |  "\input{childdoc.def}\childdocforward[cdocsamp]{cdocsch1}"|\\
% |latex -jobname cdocscl2 \|\\
% |  "\def\version{final}\input{childdoc.def}\childdocforward{cdocsch2}"|
% \end{tabular}
% \end{center}
% Note that the trailing backslash on each first line
% merely continues the input to the second line
% (for convenient cut ant paste).
% Furthermore, the command |latex| can be replaced by any
% of its alternative versions such as |pdflatex|.
%
% %%%%%%%%%%%%%%%%%%%%%%%%%%%%%%%%%%%%%%%%%%%%%%%%%%%%%%%%%%%%%%%%%%%%%%%%%%%%%%
% %%%%%%%%%%%%%%%%%%%%%%%%%%%%%%%%%%%%%%%%%%%%%%%%%%%%%%%%%%%%%%%%%%%%%%%%%%%%%%
% \section{Implementation}
%\iffalse
%<*package>
%\fi
%
% This section describes the definitions file |childdoc.def|.

% The definitions cannot be loaded using |\usepackage| or |\RequirePackage|
% which has a mechanism to prevent loading a style file more than once.
% When loading the definitions by means of |\input|
% multiple instances have to be prevented manually:
%\iffalse
%This code needs to be before the `\ProvidesFile' directive
%which is defined at the beginning of this file.
%Therefore it is also placed there and commented out here.
%</package>
%<*discard>
%\fi
%    \begin{macrocode}
\ifdefined\childdocmain\endinput\fi
%    \end{macrocode}
%\iffalse
%</discard>
%<*package>
%\fi
%
% \macro{\ifchilddoc}
% \macro{\ifchilddocmanual}
% The conditional |\ifchilddoc| tells whether a
% child (true) or main (false) document is being compiled.
% The conditional |\ifchilddocmanual| tells whether
% the |\includeonly| mechanism is used (false) or
% the selection of child files must be performed manually (true).
% The definitions initialise to false:
%    \begin{macrocode}
\newif\ifchilddoc
\newif\ifchilddocmanual
%    \end{macrocode}

% \macro{\childdocname}
% \macro{\childdocjob}
% The macro |\childdocname| stores the name of the main document
% to be compiled. The macro |\childdocjob| stores the name of
% the document on which the \LaTeX{} compiler was originally invoked.
% The content of |\jobname| cannot be compared
% to filenames specified in the source due to different catcodes.
% The following code rescans |\jobname|, stores the result
% in |\childdocname| and saves a copy in |\childdocjob|:
%    \begin{macrocode}
\edef\childdocname{\scantokens\expandafter{\jobname\noexpand}}
\let\childdocjob\childdocname
%    \end{macrocode}

% \macro{\childdocdisable}
% The macro |\childdocdisable| prevents the main file
% from being processed more than once.
% At this stage, the main document command |\childdocmain|
% is assumed to be called once again where it should do nothing.
% Any subsequent call to it should prevent
% a secondary processing of the main document
% It overwrites the forwarding commands
% |\childdocof| and |\childdocforward|
% with empty macros to prevent further inclusions of the main document:
%    \begin{macrocode}
\newcommand{\childdocdisable}
{
  \renewcommand{\childdocmain}[1]{\renewcommand{\childdocmain}[1]{\endinput}}
  \renewcommand{\childdocof}[1]{}
  \renewcommand{\childdocby}[2][]{}
  \renewcommand{\childdocforward}[2][]{}
  \renewcommand{\childdocdisable}{}
}
%    \end{macrocode}

% \macro{\childdocmain}
% The macro |\childdocmain| is to be called at the top of the main file
% with nothing or the main filename (without extension) as argument.
% First, it breaks loops.
% If the argument is not empty and does not match |\childdocname|
% (which is set by the first inclusion of |childdoc.def|),
% |\ifchilddoc| is set to true, |\includeonly| is applied to the child file
% and |\jobname| is set to the main file
% (for proper handling of |.aux| files):
%    \begin{macrocode}
\newcommand{\childdocmain}[1]
{
  \childdocdisable\childdocmain{}
  \if?#1?\else
    \begingroup
      \def\childdoctmp{#1}
      \ifx\childdoctmp\childdocname
        \def\childdoctmp{}
      \else
        \def\childdoctmp
        {
          \childdoctrue
          \includeonly{\childdocname}
          \def\childdocjob{#1}
          \def\jobname{#1}
        }
      \fi
      \expandafter
    \endgroup
    \childdoctmp
  \fi
}
%    \end{macrocode}

% \macro{\childdocof}
% The command |\childdocof| redirects
% compilation to the main file |#1|.
%    \begin{macrocode}
\newcommand{\childdocof}[1]
{
  \childdocdisable
  \childdoctrue
  \includeonly{\childdocname}
  \def\jobname{#1}
  \def\childdocjob{#1}
  \input{#1}
}
%    \end{macrocode}

% \macro{\childdocby}
% The command |\childdocby| ....
%    \begin{macrocode}
\newcommand{\childdocby}[2][]
{
  \childdocdisable
  \childdoctrue
  \childdocmanualtrue
  \if?#1?\else
    \def\jobname{#2}
  \fi
  \def\childdocjob{#2}
  \input{#2}
  \endinput
}
%    \end{macrocode}

% \macro{\childdocforward}
% The command |\childdocforward| redirects
% compilation to the main file or
% (if the optional argument is given) a child file.
% Parameters are set as if the main file
% or a child file starting with |\childdocof| was compiled.
% Then compilation is handed over to the main file:
%    \begin{macrocode}
\newcommand{\childdocforward}[2][]
{
  \begingroup
    \if?#1?
      \def\childdoctmp
      {
        \def\childdocname{#2}
        \def\childdocjob{#2}
        \def\jobname{#2}
        \input{#2}
        \endinput
      }
    \else
      \def\childdoctmp
      {
        \childdocdisable
        \def\childdocname{#2}
        \childdoctrue
        \includeonly{#2}
        \def\childdocjob{#1}
        \def\jobname{#1}
        \input{#1}
        \endinput
      }
    \fi
    \expandafter
  \endgroup
  \childdoctmp
}
%    \end{macrocode}

% \macro{\childdocforwardprefix}
% The command |\childdocforwardprefix| redirects
% compilation to the main or a child file by means of a pattern.
% The prefix |#1| in the current filename is replaced by |#2|
% and the suffix of the current filename is kept
% (it is assumed that the filename does not contain the substring `|~~~|'
% which is used as a delimiter).
% Compilation is handed over to the new file by |\childdocforward|:
%    \begin{macrocode}
\newcommand{\childdocforwardprefix}[3][]
{
  \begingroup
    \def\childdocextract #2##1~~~{\def\childdoctmp{\childdocforward[#1]{#3##1}}}
    \expandafter\childdocextract\childdocname~~~
    \expandafter
  \endgroup
  \childdoctmp
}
%    \end{macrocode}

% \macro{\childdoc}
% The deprecated macro |\childdoc| is a legacy version of |\childdocmain|:
%    \begin{macrocode}
\newcommand{\childdoc}{\childdocmain}
%    \end{macrocode}

% \macro{\childdocredirect}
% The deprecated macro |\childdocredirect| is a legacy version
% of |\childdocforward| and |\childdocforwardprefix|:
%    \begin{macrocode}
\newcommand{\childdocredirect}[2][]
{
  \begingroup
    \if?#1?
      \def\childdoctmp{\childdocforward{#2}}
    \else
      \def\childdoctmp{\childdocforwardprefix{#1}{#2}}
    \fi
    \expandafter
  \endgroup
  \childdoctmp
}
%    \end{macrocode}

%\iffalse
%</package>
%\fi
%
\endinput
|\\
|\childdocforward{|\textit{main}|}|\\
\end{tabular}
\end{center}
%
or alternatively with:
%
\begin{center}
\begin{tabular}{l}
|% \iffalse
%
% childdoc.dtx Copyright (C) 2017-2018 Niklas Beisert
%
% This work may be distributed and/or modified under the
% conditions of the LaTeX Project Public License, either version 1.3
% of this license or (at your option) any later version.
% The latest version of this license is in
%   http://www.latex-project.org/lppl.txt
% and version 1.3 or later is part of all distributions of LaTeX
% version 2005/12/01 or later.
%
% This work has the LPPL maintenance status `maintained'.
%
% The Current Maintainer of this work is Niklas Beisert.
%
% This work consists of the files childdoc.dtx and childdoc.ins
% and the derived files childdoc.def and cdocsamp.tex with
% cdocsch1.tex, cdocsch2.tex, cdocsdrf.tex, cdocsfn1.tex, cdocsfn2.tex.
%
%<package>\ifdefined\childdocmain\endinput\fi
%<package>\ProvidesFile{childdoc.def}[2018/12/30 v2.0 child document driver]
%<samplemain>\ProvidesFile{cdocsamp.tex}[2018/12/30 v2.0 sample for childdoc]
%<*driver>
%\ProvidesFile{childdoc.drv}[2018/12/30 v2.0 childdoc reference manual file]
\PassOptionsToClass{10pt,a4paper}{article}
\documentclass{ltxdoc}

\usepackage[margin=35mm]{geometry}
\usepackage{hyperref}
\usepackage{hyperxmp}
\usepackage[usenames]{color}

\hypersetup{colorlinks=true}
\hypersetup{pdfstartview=FitH}
\hypersetup{pdfpagemode=UseNone}
\hypersetup{pdfsource={}}
\hypersetup{pdflang={en-UK}}
\hypersetup{pdfcopyright={Copyright 2017-2018 Niklas Beisert.
  This work may be distributed and/or modified under the
  conditions of the LaTeX Project Public License, either version 1.3
  of this license or (at your option) any later version.}}
\hypersetup{pdflicenseurl={http://www.latex-project.org/lppl.txt}}
\hypersetup{pdfcontactaddress={ETH Zurich, ITP, HIT K,
  Wolfgang-Pauli-Strasse 27}}
\hypersetup{pdfcontactpostcode={8093}}
\hypersetup{pdfcontactcity={Zurich}}
\hypersetup{pdfcontactcountry={Switzerland}}
\hypersetup{pdfcontactemail={nbeisert@itp.phys.ethz.ch}}
\hypersetup{pdfcontacturl={http://people.phys.ethz.ch/\xmptilde nbeisert/}}

\newcommand{\secref}[1]{\hyperref[#1]{section \ref*{#1}}}

\parskip1ex
\parindent0pt
\let\olditemize\itemize
\def\itemize{\olditemize\parskip0pt}

\begin{document}

\title{The \textsf{childdoc} Package}
\hypersetup{pdftitle={The childdoc Package}}
\author{Niklas Beisert\\[2ex]
  Institut f\"ur Theoretische Physik\\
  Eidgen\"ossische Technische Hochschule Z\"urich\\
  Wolfgang-Pauli-Strasse 27, 8093 Z\"urich, Switzerland\\[1ex]
  \href{mailto:nbeisert@itp.phys.ethz.ch}
  {\texttt{nbeisert@itp.phys.ethz.ch}}}
\hypersetup{pdfauthor={Niklas Beisert}}
\hypersetup{pdfsubject={Manual for the LaTeX2e Package childdoc}}
\date{30 December 2018, \textsf{v2.0}}
\maketitle

\begin{abstract}\noindent
\textsf{childdoc} is a \LaTeXe{} package
that enables the direct compilation
of document sections included by |\include|
to individual files.
\end{abstract}

\begingroup
\parskip0ex
\tableofcontents
\endgroup

%%%%%%%%%%%%%%%%%%%%%%%%%%%%%%%%%%%%%%%%%%%%%%%%%%%%%%%%%%%%%%%%%%%%%%%%%%%%%%%%
%%%%%%%%%%%%%%%%%%%%%%%%%%%%%%%%%%%%%%%%%%%%%%%%%%%%%%%%%%%%%%%%%%%%%%%%%%%%%%%%
\section{Introduction}

\LaTeX{} provides a mechanism to structure a large document (such as a book)
into a main file and several child files (containing the chapters)
using the |\include| command.
This mechanism is beneficial for documents
which span hundreds of pages in order to
make the source file(s) more manageable.
Moreover, compilation can be restricted to
selected child files by means of the |\includeonly| command.
The latter feature can be used to reduce the compilation time while editing
(this was significantly more useful in the earlier days of \LaTeX{})
or to generate a smaller document which is easier to navigate.
Another application of |\includeonly| is to generate
documents consisting of selected parts of the complete document.

However, there are a few drawbacks of the plain |\include| mechanism:
\begin{itemize}
\item
The child files cannot be compiled on their own,
they can only be compiled via the main file.
A naive editing environment
(such as a text editor with an option
to have the current file processed by \LaTeX)
may require one to switch to the main file before compiling;
attempting to compile the child file produces errors.
\item
The main file must be modified (each time)
to adjust the |\includeonly| command
to the present needs. This easily leaves the main file in a messy state.
\item
The generated document will always carry the filename
of the main document. This is inconvenient if
several child files are to be compiled and
to be kept for distribution.
\end{itemize}

The present package provides a simple interface
to make child files individually compilable by \LaTeX{}.
Compiling a child file then has the same effect as compiling
the main file with an |\includeonly| command
to select the appropriate child.
Moreover the generated document will carry the name of the child
rather than the main file.
This resolves all three above issues.

This feature is meant to make the editing of books,
thesis documents and lecture notes somewhat more convenient.
However, the package can also be used efficiently for
composing a series of documents (such as exercise sheets)
which are typically distributed individually.
It then assists the author in generating the individual documents
(potentially in different versions)
as well as a document containing the collected series.
Another application is in developing style files
or other kinds of included material
where compilation of the style file could redirect
to a sample or test file.

%%%%%%%%%%%%%%%%%%%%%%%%%%%%%%%%%%%%%%%%%%%%%%%%%%%%%%%%%%%%%%%%%%%%%%%%%%%%%%%%
%%%%%%%%%%%%%%%%%%%%%%%%%%%%%%%%%%%%%%%%%%%%%%%%%%%%%%%%%%%%%%%%%%%%%%%%%%%%%%%%
\section{Usage}

First of all, the package \textsf{childdoc} is \emph{not} a standard
\LaTeXe{} |.sty| style file! Therefore it needs to be invoked in
a non-standard way.

%%%%%%%%%%%%%%%%%%%%%%%%%%%%%%%%%%%%%%%%%%%%%%%%%%%%%%%%%%%%%%%%%%%%%%%%%%%%%%%%
\subsection{Included Files}
\label{sec:include}

%%%%%%%%%%%%%%%%%%%%%%%%%%%%%%%%%%%%%%%%
\DescribeMacro{\childdocmain}
To use the package, add the commands
\begin{center}
\begin{tabular}{l}
|\input{childdoc.def}|\\
|\childdocmain{}|\\
\end{tabular}
\end{center}
at the very top of the main \LaTeX{} file,
in particular \emph{before} the |\documentclass| statement!
The argument of |\childdocmain| should be left empty
(but it must be present).

%%%%%%%%%%%%%%%%%%%%%%%%%%%%%%%%%%%%%%%%
\DescribeMacro{\childdocof}
Furthermore, add the commands
\begin{center}
\begin{tabular}{l}
|\input{childdoc.def}|\\
|\childdocof{|\textit{main}|}|\\
\end{tabular}
\end{center}
at the top of every child file \textit{child}
which is included by |\include{|\textit{child}|}|
from within the main file
(or at least for those files to be compiled individually).
The argument \textit{main} must be the filename of the main file.

There are a couple of
considerations in setting up the main and child documents:

%%%%%%%%%%%%%%%%%%%%%%%%%%%%%%%%%%%%%%%%
\paragraph{Restrictions.}

Please note the following restrictions:
\begin{itemize}
\item
|\childdocmain| must be called with one argument \textit{main}
to ensure compatibility with earlier version of the package.
It must either be empty (|\childdocmain{}|)
or precisely match the filename of the main file in which it is specified.
See \secref{sec:detection} for further information.
\item
The filename \textit{main} must be specified without the |.tex| extension.
\item
The filename \textit{main} is case sensitive
(even in case-insensitive file systems)
due to internal string comparison.
\item
The argument \textit{main} should be fully expanded, it cannot be a macro.
\item
Subdirectories and special characters should be avoided in filenames.
\item
The command |\childdocmain{|\textit{main}|}| must be followed by a whitespace.
It should not be followed immediately by another command
or by a comment mark `|%|'.
This is because the \TeX{} parser reads the token immediately following
the argument of |\childdocmain| and puts it
at the beginning of every child section;
however, a white\-space is ignored.
\end{itemize}

%%%%%%%%%%%%%%%%%%%%%%%%%%%%%%%%%%%%%%%%
\paragraph{Content of Main File.}

It is advisable to place all content in the child files included by |\include|.
Any output contained in the main file will appear in all child documents
unless suppressed manually;
it cannot be suppressed automatically by the |\includeonly| directive
and thus should normally be avoided.
A method to include some content in the main file
by means of conditional processing is described in \secref{sec:conditional}.

%%%%%%%%%%%%%%%%%%%%%%%%%%%%%%%%%%%%%%%%
\paragraph{Page Numbering.}

When only a part of the document is compiled,
the appropriate numbering of pages
(as well as other status parameters)
is determined from the |.aux| files.
The latter contain information from previous passes.
However this information needs to propagate through
all intermediate child documents.
Therefore the page numbering in child documents may well
be inconsistent until the complete document is compiled at least once.

A useful (if unconventional) way to always ensure a consistent
page numbering is to restart the numbering in each child document
and denote the pages by `\textit{child}|.|\textit{page}'
where \textit{child} represents the chapter/section number of the child file.
This can be achieved by the command
|\numberwithin{page}{|\textit{child}|}|
of the \textsf{amsmath} package
where \textit{child} can be |chapter| or |section|
depending on the chosen structuring.
Alternatively, one can modify the macro |\thepage| appropriately
and reset the counter |page| at the start of each child file.

%%%%%%%%%%%%%%%%%%%%%%%%%%%%%%%%%%%%%%%%%%%%%%%%%%%%%%%%%%%%%%%%%%%%%%%%%%%%%%%%
\subsection{Conditional Processing}
\label{sec:conditional}

The package provides a mechanism to compile different versions
of a document. To customise the versions further some conditional processing
can come in handy to distinguish which version is being compiled.
The package provides two macros to describe the compilation context:

%%%%%%%%%%%%%%%%%%%%%%%%%%%%%%%%%%%%%%%%
\DescribeMacro{\ifchilddoc}
The conditional |\ifchilddoc| distinguishes between the compilation of
child documents and the main document:
%
\begin{center}
|\ifchilddoc |\textit{child-code}| |[|\||else |\textit{main-code}]| \||fi|
\end{center}

%%%%%%%%%%%%%%%%%%%%%%%%%%%%%%%%%%%%%%%%
\DescribeMacro{\childdocname}
\DescribeMacro{\childdocjob}
The macro |\childdocname| contains the filename (without extension)
of the main or child file being processed.
Note that |\childdocjob| will always contain the name of the main file.

%%%%%%%%%%%%%%%%%%%%%%%%%%%%%%%%%%%%%%%%
\paragraph{Title Page.}

Conditional processing can be used to include a title or banner page
in the main document when proper precautions are taken.
Importantly, the code in the main file should ensure that the page counter
(as well as other status parameters which are stored in the |.aux| files)
takes the same value after the conditional processing.
Otherwise the page numbers may take divergent values
depending on which part is compiled.

For example, a title page could be declared by:
%
\begin{center}
\begin{tabular}{l}
|\ifchilddoc\||else|\\
|\addtocounter{page}{-1}|\\
\textit{code for title page}\\
|\newpage|\\
|\||fi|
\end{tabular}
\end{center}
%
A banner page for the child documents can be generated by:
%
\begin{center}
\begin{tabular}{l}
|\ifchilddoc|\\
|\addtocounter{page}{-1}|\\
\textit{code for banner page}\\
|\newpage|\\
|\||fi|
\end{tabular}
\end{center}
%
Here one could write a message such as:
\begin{center}
|This is the part \childdocname{} of \childdocjob{}.|
\end{center}

%%%%%%%%%%%%%%%%%%%%%%%%%%%%%%%%%%%%%%%%%%%%%%%%%%%%%%%%%%%%%%%%%%%%%%%%%%%%%%%%
\subsection{Flags}
\label{sec:flags}

The package makes it easy to generate different versions
of the main or child documents.
To this end compilation flags can be defined
and assigned different default values.
They will be particularly useful in conjunction
with the forwarding mechanism described in \secref{sec:forward}.

For example, it may be useful to have a flag |\version|
which can be set to |draft| or |final|.
The document source will contain some conditional code
depending on the value of |\version|.
Suppose further, the flag should default to |final| for the main file
and to |draft| for child files
which is a natural assignment for editing the document.
This is achieved by placing the following code
in the preamble of the main document
(below the |\childdocmain| directive):
%
\begin{center}
\begin{tabular}{l}
|\ifchilddoc|\\
|\providecommand{\version}{draft}|\\
|\||else|\\
|\providecommand{\version}{final}|\\
|\||fi|
\end{tabular}
\end{center}
%
The definition by |\providecommand| makes sure
that previous definitions are not overwritten.
Further statements |\providecommand{\version}{...}|
can thus be added before the above code to override it.

For the main file, one might add a line
(between |\childdocmain| and the above block)
%
\begin{center}
|%\ifchilddoc\||else\providecommand{\version}{draft}\||fi|
\end{center}
%
which can be uncommented to produce a draft version.
Likewise one can add a line to the very top of a child file
(above the |\childdocof{|\textit{main}|}| directive)
%
\begin{center}
|%\providecommand{\version}{final}|
\end{center}
%
which can be uncommented to produce the final version of this child document.

%%%%%%%%%%%%%%%%%%%%%%%%%%%%%%%%%%%%%%%%%%%%%%%%%%%%%%%%%%%%%%%%%%%%%%%%%%%%%%%%
\subsection{Forwarding}
\label{sec:forward}

Different versions of the main or child documents
using compilation flags as described in \secref{sec:flags}
can be (permanently) stored in different files
for convenient compilation, viewing and distribution.
To this end, the package defines a command
to pass on compilation to a different file:

%%%%%%%%%%%%%%%%%%%%%%%%%%%%%%%%%%%%%%%%
\DescribeMacro{\childdocforward}
The command |\childdocforward| redirects processing to
another source file:
%
\begin{center}
\begin{tabular}{l}
|\input{childdoc.def}|\\
|\childdocforward[|\textit{main}|]{|\textit{dest}|}|\\
\end{tabular}
\end{center}
%
The argument \textit{dest} is the destination file
(without extension).
It should be the main file or one of the child files.
Note that further \textsf{childdoc} directives
such as |\childdocof| and |\childdocforward|
in the indicated file will be processed in this form.
The optional argument \textit{main}
passes on directly to the main file \textit{main}
while pretending to compile the child \textit{dest}.
This form behaves as if \textit{dest}
issues |\childdocof{|\textit{main}|}| right away,
and no further \textsf{childdoc} directives will be processed.

%%%%%%%%%%%%%%%%%%%%%%%%%%%%%%%%%%%%%%%%
\DescribeMacro{\...prefix}
In the alternative form |\childdocforwardprefix|,
%
\begin{center}
\begin{tabular}{l}
|\input{childdoc.def}|\\
|\childdocforwardprefix[|\textit{main}|]{|\textit{prefix}|}{|\textit{dest}|}|
\end{tabular}
\end{center}
%
the destination file is determined by a pattern
depending on the current file:
To make this work, the current file must be called
`{\textit{prefix}\hspace{0.2em}\textit{suffix}}'
with \textit{prefix} matching precisely the argument.
Processing is then passed on to the file
`{\textit{dest}\hspace{0.2em}\textit{suffix}}'.
Surely, the same effect is achieved by
directly specifying the
argument `{\textit{dest}\hspace{0.2em}\textit{suffix}}'
in the first form.
However, that requires to set up a different file
for each child. With the alternative form of the command
all these files can have exactly the same content
which simplifies setting them up and maintaining them.

For example, the following file |draft.tex|
with a compilation flag |\version| as described in \secref{sec:flags}
compiles the main document as a draft:
%
\begin{center}
\begin{tabular}{l}
|\def\version{draft}|\\
|\input{childdoc.def}|\\
|\childdocforward{|\textit{main}|}|
\end{tabular}
\end{center}
%
Likewise, the following files |final|\textit{nn}|.tex|
compile the final version of the child document
|child|\textit{nn}|.tex|:
%
\begin{center}
\begin{tabular}{l}
|\def\version{final}|\\
|\input{childdoc.def}|\\
|\childdocforwardprefix{final}{child}|
\end{tabular}
\end{center}
%

Note that when several versions of a main file and/or of each child file
are to be generated, it may be convenient to set up a |Makefile| or
shell script to automatise the process.

%%%%%%%%%%%%%%%%%%%%%%%%%%%%%%%%%%%%%%%%%%%%%%%%%%%%%%%%%%%%%%%%%%%%%%%%%%%%%%%%
\subsection{Command Line Processing}
\label{sec:commandline}

The effect of redirection files can also be achieved by invoking
the \LaTeX{} compiler with a more elaborate command line.
Most conveniently this should be done as part
of a shell script or a |Makefile|.

When using \textsf{childdoc} in the main file, the following
command lines effectively perform a redirection
(note that depending on the shell being used,
backslashes may have to be doubled: `|\|' $\to$ `|\\|'):
%
\begin{center}
|... -jobname "|\textit{target}|" |\\|"|[\textit{flags}]%
|\input{childdoc.def}\childdocforward[|\textit{main}|]{|\textit{dest}|}"|
\end{center}
%
Here \textit{target} is the name of the output file,
\textit{main} is the name of the main file
and \textit{dest} is the name of the main or child file to be processed
(all filenames without extensions).
The optional argument \textit{main} can be omitted
if \textit{main} matches \textit{dest}.
Optionally, compilation \textit{flags} can be defined via |\def| commands.
This command line makes the \TeX{} engine believe
it is compiling the file \textit{target}
whose content is specified as the latter parameter.
The provided code then forwards the processing to
\textit{main} or \textit{dest} as described in \secref{sec:forward}.

%%%%%%%%%%%%%%%%%%%%%%%%%%%%%%%%%%%%%%%%%%%%%%%%%%%%%%%%%%%%%%%%%%%%%%%%%%%%%%%%
\subsection{Include by Input}
\label{sec:input}

Including child documents by |\include| has some restrictions by design.
Most notably, the content of a child document always occupies
its own set of pages; pages cannot be shared between child documents.
Usually, this behaviour makes perfect sense
because each child document contain an essential part of the document.
However, in some situations it may be desirable to compose
a document from a collection of parts
without having mandatory page breaks between then.
For this case, the package
provides a mechanism to include parts
by |\input| which can also be processed individually.
However, by construction this mechanism
requires manual handling of the content to be output.

%%%%%%%%%%%%%%%%%%%%%%%%%%%%%%%%%%%%%%%%
\DescribeMacro{\ifchilddocmanual}
The main file should be prepared as usual, see \secref{sec:include}.
However, the document body must make a distinction
between processing of an individual part and of the main document, e.g.:
%
\begin{center}
\begin{tabular}{l}
|\ifchilddocmanual|\\
|\input{\childdocname}|\\
|\||else|\\
\textit{document body with }|\input{|\textit{part}|}|\\
|\||fi|
\end{tabular}
\end{center}
%
The conditional |\ifchilddocmanual| is true whenever
a part to be included by |\input| is being compiled,
and the name of the part is stored in |\childdocname|.

%%%%%%%%%%%%%%%%%%%%%%%%%%%%%%%%%%%%%%%%
\DescribeMacro{\childdocby}
Each part to be included by |\input| should start with:
%
\begin{center}
\begin{tabular}{l}
|\input{childdoc.def}|\\
|\childdocby{|\textit{main}|}|\\
\end{tabular}
\end{center}
%
The directive |\childdocby| is similar to |\childdocof|
described in \secref{sec:include},
but the subsequent selection of content must be done manually.
To that end, both |\ifchilddoc| and |\ifchilddocmanual|
will be true upon processing of a part,
and the name of the part is stored in |\childdocname|.
Note that |\jobname| will be set to the filename of the current part
so that each part receives an individual |.aux| file
that does not interfere with the |.aux| file(s) of the main document.
This behaviour can be altered by the alternative form
|\childdocby[*]{|\textit{main}|}| (with a non-empty optional argument)
which uses the |.aux| file of the main document
by setting |\jobname| to \textit{main}.

%%%%%%%%%%%%%%%%%%%%%%%%%%%%%%%%%%%%%%%%%%%%%%%%%%%%%%%%%%%%%%%%%%%%%%%%%%%%%%%%
\subsection{Driver Development}
\label{sec:driver}

The \textsf{childdoc} mechanism can also be use for the development
of definition files such as \LaTeX{} styles or classes.
This case differs from the above setup with multiple parts
included by |\include| in that no |\includeonly| should be invoked.
This can be achieved by starting the include file
(before |\ProvidesPackage|) with:
%
\begin{center}
\begin{tabular}{l}
|\input{childdoc.def}|\\
|\childdocforward{|\textit{main}|}|\\
\end{tabular}
\end{center}
%
or alternatively with:
%
\begin{center}
\begin{tabular}{l}
|\input{childdoc.def}|\\
|\childdocby{|\textit{main}|}|\\
\end{tabular}
\end{center}
%
Both forms have slightly different effects as described above.
The main file is prepared as usual, see \secref{sec:include}.

%%%%%%%%%%%%%%%%%%%%%%%%%%%%%%%%%%%%%%%%%%%%%%%%%%%%%%%%%%%%%%%%%%%%%%%%%%%%%%%%
\subsection{Legacy Detection}
\label{sec:detection}

The directive |\childdocmain| in the main file can detect
whether the complete document or merely a child is to be compiled
even without using the directive |\childdocof|.
This method is deprecated because it is less robust
and there is no compelling reason to use it;
it is merely provided for backward compatibility
and it may be removed in future versions.

If the detection mechanism is to be used,
it is mandatory to correctly specify
the filename of the main file as the argument of |\childdocmain|:
%
\begin{center}
\begin{tabular}{l}
|\input{childdoc.def}|\\
|\childdocmain{|\textit{main}|}|\\
\end{tabular}
\end{center}
%
If |\jobname| does not match the argument \textit{main} of |\childdocmain|,
it is assumed that |\jobname| points to the child file to be compiled.
When using |\childdocmain| with the main file specified as argument,
it suffices to start a child file
with just |\input{|\textit{main}|}|
without loading of the package and using |\childdocof|.
If instead all processing is done
with the appropriate \textsf{childdoc} directives,
the argument of \textit{main} of |\childdocmain| can be empty.

An alternative version of the command line processing described
in \secref{sec:commandline} using the detection mechanism reads:
%
\begin{center}
|... -jobname "|\textit{target}|" "|[\textit{flags}]%
[|\def\jobname{|\textit{dest}|}|]|\input{|\textit{main}|}"|
\end{center}

%%%%%%%%%%%%%%%%%%%%%%%%%%%%%%%%%%%%%%%%%%%%%%%%%%%%%%%%%%%%%%%%%%%%%%%%%%%%%%%%
\subsection{Manual Code}
\label{sec:manual}

In case one cannot be certain whether the definitions file |childdoc.def|
is installed on the target \TeX{} distribution
and one prefers not to ship it,
it is conceivable to paste a few relevant commands into the sources.

To that end, drop all statements |\input{childdoc.def}|
and perform the replacements as outlined below.
Instead of |\childdocmain{|\textit{main}|}| add the following code
to the top of the main file:
%
\begin{center}
\begin{tabular}{l}
|\||ifdefined\childdocname\endinput\||fi\newif\ifchilddoc|\\
|\edef\childdocname{\scantokens\expandafter{\jobname\noexpand}}|\\
|\def\childdocmain{|\textit{main}|}\||ifx\childdocmain\childdocname\||else|\\
|\childdoctrue\includeonly{\childdocname}\let\jobname\childdocmain\||fi|\\
\end{tabular}
\end{center}
%
Instead of |\childdocof{|\textit{main}|}| just include the main file
at the top of each child file:
%
\begin{center}
|\input{|\textit{main}|}|
\end{center}
%
A simple redirection |\childdocforward{|\textit{dest}|}| is achieved by:
%
\begin{center}
|\def\jobname{|\textit{dest}|}\input{\jobname}|
\end{center}
%
The redirection with prefix
|\childdocforwardprefix[|\textit{prefix}|]{|\textit{dest}|}|
is accomplished by:
%
\begin{center}
\begin{tabular}{l}
|{\edef\jobname{\scantokens\expandafter{\jobname\noexpand}}|\\
|\def\redirectjob |\textit{prefix}|#1~~~{\gdef\jobname{|\textit{dest}|#1}}|\\
|\expandafter\redirectjob\jobname~~~}\input{\jobname}|
\end{tabular}
\end{center}

In an alternative approach,
child documents can be compiled by a specific command line
without additional code or specific definitions:
%
\begin{center}
|... -jobname "|\textit{target}|" "|[\textit{flags}]%
|\includeonly{|\textit{dest}|}\input{|\textit{main}|}"|
\end{center}
%

%%%%%%%%%%%%%%%%%%%%%%%%%%%%%%%%%%%%%%%%%%%%%%%%%%%%%%%%%%%%%%%%%%%%%%%%%%%%%%%%
%%%%%%%%%%%%%%%%%%%%%%%%%%%%%%%%%%%%%%%%%%%%%%%%%%%%%%%%%%%%%%%%%%%%%%%%%%%%%%%%
\section{Information}

%%%%%%%%%%%%%%%%%%%%%%%%%%%%%%%%%%%%%%%%%%%%%%%%%%%%%%%%%%%%%%%%%%%%%%%%%%%%%%%%
\subsection{Copyright}

Copyright \copyright{} 2017--2018 Niklas Beisert

This work may be distributed and/or modified under the
conditions of the \LaTeX{} Project Public License, either version 1.3
of this license or (at your option) any later version.
The latest version of this license is in
  \url{http://www.latex-project.org/lppl.txt}
and version 1.3 or later is part of all distributions of \LaTeX{}
version 2005/12/01 or later.

This work has the LPPL maintenance status `maintained'.

The Current Maintainer of this work is Niklas Beisert.

This work consists of the files |README.txt|, |childdoc.ins| and |childdoc.dtx|
as well as the derived files |childdoc.def|, |cdocsamp.tex|
with |cdocsch1.tex|, |cdocsch2.tex|, |cdocspt3.tex|, |cdocspt4.tex|,
|cdocsdrf.tex|, |cdocsfn1.tex|, |cdocsfn2.tex|
as well as |childdoc.pdf|.

%%%%%%%%%%%%%%%%%%%%%%%%%%%%%%%%%%%%%%%%%%%%%%%%%%%%%%%%%%%%%%%%%%%%%%%%%%%%%%%%
\subsection{Files and Installation}

The package consists of the files:
%
\begin{center}
\begin{tabular}{ll}
    |README.txt|   & readme file \\
    |childdoc.ins| & installation file \\
    |childdoc.dtx| & source file \\
    |childdoc.def| & definition file \\
    |cdocsamp.tex| & sample main file \\
    |cdocsch1.tex| & sample include file \\
    |cdocsch2.tex| & sample include file \\
    |cdocspt3.tex| & sample part file \\
    |cdocspt4.tex| & sample part file \\
    |cdocsdrf.tex| & sample redirection file \\
    |cdocsfn1.tex| & sample redirection file \\
    |cdocsfn2.tex| & sample redirection file \\
    |childdoc.pdf| & manual
\end{tabular}
\end{center}
%
The distribution consists of the files
|README.txt|, |childdoc.ins| and |childdoc.dtx|.
%
\begin{itemize}
\item
Run (pdf)\LaTeX{} on |childdoc.dtx|
to compile the manual |childdoc.pdf| (this file).
\item
Run \LaTeX{} on |childdoc.ins| to create the definitions file |childdoc.def|
and the sample |cdocsamp.tex| with include files
|cdocsch1.tex|, |cdocsch2.tex|, |cdocspt3.tex|, |cdocspt4.tex|,
|cdocsdrf.tex|, |cdocsfn1.tex|, |cdocsfn2.tex|.
Then copy the file |childdoc.def| to an appropriate directory of your \LaTeX{}
distribution, e.g.\ \textit{texmf-root}|/tex/latex/childdoc|.
\end{itemize}

%%%%%%%%%%%%%%%%%%%%%%%%%%%%%%%%%%%%%%%%%%%%%%%%%%%%%%%%%%%%%%%%%%%%%%%%%%%%%%%%
\subsection{Related CTAN Packages}

There are several other packages which offer a similar functionality:
%
\begin{itemize}
\item
The packages
\href{http://ctan.org/pkg/docmute}{\textsf{docmute}},
\href{http://ctan.org/pkg/includex}{\textsf{includex}} and
\href{http://ctan.org/pkg/standalone}{\textsf{standalone}}
provide commands to include only the document body of
a child file thus allowing both files to be compiled individually.
\item
The packages \href{http://ctan.org/pkg/subdocs}{\textsf{subdocs}}
and \href{http://ctan.org/pkg/subfiles}{\textsf{subfiles}}
provide structures in which the main and child documents can be
encapsulated and allowing them to be compiled individually.
The inclusion mechanism is different from the conventional |\include|.
\item
The package \href{http://ctan.org/pkg/combine}{\textsf{combine}}
is an elaborate solution to combine several documents into one.
\end{itemize}
%
See also the CTAN topic \href{http://ctan.org/topic/subdocs}{\textsf{subdocs}}
for further related packages.
The present package differs from the above solutions in that
a document structure constructed with the conventional |\include| mechanism
just needs two extra commands at the top of every file
such that all constituent files can be compiled individually.

%%%%%%%%%%%%%%%%%%%%%%%%%%%%%%%%%%%%%%%%%%%%%%%%%%%%%%%%%%%%%%%%%%%%%%%%%%%%%%%%
%\subsection{Feature Suggestions}
%
%The following is a list of features which may be useful for future
%versions of this package:
%%
%\begin{itemize}
%\item
%\ldots
%\end{itemize}

%%%%%%%%%%%%%%%%%%%%%%%%%%%%%%%%%%%%%%%%%%%%%%%%%%%%%%%%%%%%%%%%%%%%%%%%%%%%%%%%
\subsection{Revision History}

%%%%%%%%%%%%%%%%%%%%%%%%%%%%%%%%%%%%%%%%
\paragraph{v2.0:} 2018/12/30

\begin{itemize}
\item
immediate forward processing
\item
added |\childdocby| mechanism
\item
manual restructured
\end{itemize}

%%%%%%%%%%%%%%%%%%%%%%%%%%%%%%%%%%%%%%%%
\paragraph{v1.6:} 2018/01/17

\begin{itemize}
\item
application for development of include files
\item
corrections to manual
\end{itemize}

%%%%%%%%%%%%%%%%%%%%%%%%%%%%%%%%%%%%%%%%
\paragraph{v1.5:} 2017/05/21

\begin{itemize}
\item
more complete structuring introduced
\item
|\childdocof| introduced
\item
|\childdoc| renamed to |\childdocmain|
\item
|\childredirect| renamed to |\childdocforward| and |\childdocforwardprefix|
and functionality expanded
\end{itemize}

%%%%%%%%%%%%%%%%%%%%%%%%%%%%%%%%%%%%%%%%
\paragraph{v1.0:} 2017/04/27

\begin{itemize}
\item
manual and install package
\item
first version published on CTAN
\end{itemize}

%%%%%%%%%%%%%%%%%%%%%%%%%%%%%%%%%%%%%%%%
\paragraph{v0.6:} 2017/04/26

\begin{itemize}
\item
redirection mechanism added
\end{itemize}

%%%%%%%%%%%%%%%%%%%%%%%%%%%%%%%%%%%%%%%%
\paragraph{v0.5:} 2017/04/26

\begin{itemize}
\item
functionality in definition file
\end{itemize}


%%%%%%%%%%%%%%%%%%%%%%%%%%%%%%%%%%%%%%%%%%%%%%%%%%%%%%%%%%%%%%%%%%%%%%%%%%%%%%%%
%%%%%%%%%%%%%%%%%%%%%%%%%%%%%%%%%%%%%%%%%%%%%%%%%%%%%%%%%%%%%%%%%%%%%%%%%%%%%%%%
%%%%%%%%%%%%%%%%%%%%%%%%%%%%%%%%%%%%%%%%%%%%%%%%%%%%%%%%%%%%%%%%%%%%%%%%%%%%%%%%
\appendix

\settowidth\MacroIndent{\rmfamily\scriptsize 000\ }

 \DocInput{childdoc.dtx}

\end{document}
%</driver>
% \fi
%
% %%%%%%%%%%%%%%%%%%%%%%%%%%%%%%%%%%%%%%%%%%%%%%%%%%%%%%%%%%%%%%%%%%%%%%%%%%%%%%
% %%%%%%%%%%%%%%%%%%%%%%%%%%%%%%%%%%%%%%%%%%%%%%%%%%%%%%%%%%%%%%%%%%%%%%%%%%%%%%
% \section{Sample}
%\iffalse
%<*samplemain>
%\fi
%
% The following presents a sample document
% with two chapters, two parts, a title page,
% a compile flag as well as three forwarding files to set the flag.
% It consists of eight |.tex| files:
% \begin{center}
% \begin{tabular}{ll}
% |cdocsamp.tex|&main file\\
% |cdocsch1.tex|&include file for chapter 1\\
% |cdocsch2.tex|&include file for chapter 2\\
% |cdocspt3.tex|&include file for part 3\\
% |cdocspt4.tex|&include file for part 4\\
% |cdocsdrf.tex|&forwarding file for main file in draft mode\\
% |cdocsfi1.tex|&forwarding file for final version of chapter 1\\
% |cdocsfi2.tex|&forwarding file for final version of chapter 2\\
% \end{tabular}
% \end{center}
% Each of the eight files can be compiled directly by the \LaTeX{} compiler.
%
% %%%%%%%%%%%%%%%%%%%%%%%%%%%%%%%%%%%%%%
% \paragraph{Main File.}
%
% The main file is called |cdocsamp.tex|.
%
% Load the \textsf{childdoc} definitions and
% declare the filename for the main document:
%    \begin{macrocode}
\input{childdoc.def}
\childdocmain{}
%    \end{macrocode}

% Optional override for |\version| flag:
%    \begin{macrocode}
%%\ifchilddoc\else\providecommand{\version}{draft}\fi
%    \end{macrocode}

% Define the default values for the |\version| flag
% (|final| for the main file and |draft| for childs):
%    \begin{macrocode}
\ifchilddoc
\providecommand{\version}{draft}
\else
\providecommand{\version}{final}
\fi
%    \end{macrocode}

% Load the standard document class:
%    \begin{macrocode}
\documentclass[12pt]{article}
%    \end{macrocode}

% Start the document body:
%    \begin{macrocode}
\begin{document}
%    \end{macrocode}

% Declare a title page.
% Print title, part of document being processed and version flag:
%    \begin{macrocode}
\addtocounter{page}{-1}
\begin{center}
{\LARGE\bfseries{}childdoc example\par}
\vspace{1cm}
\ifchilddoc
\ifchilddocmanual part\else chapter\fi:
`\childdocname' of `\childdocjob'\par
\else
main document: `\childdocjob'\par
\fi
version: \version\par
\end{center}
\newpage
%    \end{macrocode}

% Manually include selected file,
% otherwise process as usual:
%    \begin{macrocode}
\ifchilddocmanual
\section*{part `\childdocname'}
\input{\childdocname}
\else
%    \end{macrocode}

% Include the two chapters:
%    \begin{macrocode}
\include{cdocsch1}
\include{cdocsch2}
%    \end{macrocode}

% Include the two parts unless only chapters should be displayed:
%    \begin{macrocode}
\ifchilddoc\else
\section{part three}
\input{cdocspt3}
\section{part four}
\input{cdocspt4}
\fi
%    \end{macrocode}

% Process as usual until here:
%    \begin{macrocode}
\fi
%    \end{macrocode}

% End of document body:
%    \begin{macrocode}
\end{document}
%    \end{macrocode}
%\iffalse
%</samplemain>
%\fi
%
% %%%%%%%%%%%%%%%%%%%%%%%%%%%%%%%%%%%%%%
% \paragraph{Chapter Include Files.}
%
% The include files are called |cdocsch1.tex| and |cdocsch2.tex|.
%
%\iffalse
%<*samplechap1|samplechap2>
%\fi

% Optional override for |\version| flag:
%    \begin{macrocode}
%%\providecommand{\version}{final}
%    \end{macrocode}

% Include the main document:
%    \begin{macrocode}
\input{childdoc.def}
\childdocof{cdocsamp}
%    \end{macrocode}

%\iffalse
%</samplechap1|samplechap2>
%\fi
%
%\iffalse
%<*samplechap1>
%\fi
% Some text for chapter 1:
%    \begin{macrocode}
\section{one}
some text in chapter one
%    \end{macrocode}

%\iffalse
%</samplechap1>
%\fi
% Some text for chapter 2:
%\iffalse
%<*samplechap2>
%\fi
%    \begin{macrocode}
\section{two}
more text in chapter two
%    \end{macrocode}

%\iffalse
%</samplechap2>
%\fi
%
% %%%%%%%%%%%%%%%%%%%%%%%%%%%%%%%%%%%%%%
% \paragraph{Part Include Files.}
%
% The include files are called |cdocspt3.tex| and |cdocspt4.tex|.
%
%\iffalse
%<*samplepart3|samplepart4>
%\fi

% Optional override for |\version| flag:
%    \begin{macrocode}
%%\providecommand{\version}{final}
%    \end{macrocode}

% Include the main document:
%    \begin{macrocode}
\input{childdoc.def}
\childdocby{cdocsamp}
%    \end{macrocode}

%\iffalse
%</samplepart3|samplepart4>
%\fi
%
%\iffalse
%<*samplepart3>
%\fi
% Some text for part 3:
%    \begin{macrocode}
some text in part three
%    \end{macrocode}

%\iffalse
%</samplepart3>
%\fi
% Some text for part 4:
%\iffalse
%<*samplepart4>
%\fi
%    \begin{macrocode}
more text in part four
%    \end{macrocode}

%\iffalse
%</samplepart4>
%\fi
%
% %%%%%%%%%%%%%%%%%%%%%%%%%%%%%%%%%%%%%%
% \paragraph{Forwarding for a Complete Draft.}
%
% The following forwarding file |cdocsdrf.tex|
% compiles the main document in draft mode:
%\iffalse
%<*sampledraft>
%\fi
%    \begin{macrocode}
\def\version{draft}
\input{childdoc.def}
\childdocforward{cdocsamp}
%    \end{macrocode}

%\iffalse
%</sampledraft>
%\fi
%
% %%%%%%%%%%%%%%%%%%%%%%%%%%%%%%%%%%%%%%
% \paragraph{Forwarding for Final Version of the Chapters.}
%
% The following forwarding files |cdocsfn1.tex| and |cdocsfn2.tex|
% (with identical content)
% compile the final versions of the child documents
% |cdocsch1.tex| and |cdocsch2.tex|, respectively:
%\iffalse
%<*samplefinal>
%\fi
%    \begin{macrocode}
\def\version{final}
\input{childdoc.def}
\childdocforwardprefix[cdocsamp]{cdocsfn}{cdocsch}
%    \end{macrocode}

%\iffalse
%</samplefinal>
%\fi
%
% %%%%%%%%%%%%%%%%%%%%%%%%%%%%%%%%%%%%%%
% \paragraph{Command Line Processing.}
%
% The following three command lines generate the output files
% |cdocscld|, |cdocscl1| and |cdocscl2|
% which should be identical to
% |cdocsdrf|, |cdocsch1| and |cdocsfn2|, respectively:
% \begin{center}
% \begin{tabular}{l}
% |latex -jobname cdocscld \|\\
% |  "\def\version{draft}\input{childdoc.def}\childdocforward{cdocsamp}"|\\
% |latex -jobname cdocscl1 \|\\
% |  "\input{childdoc.def}\childdocforward[cdocsamp]{cdocsch1}"|\\
% |latex -jobname cdocscl2 \|\\
% |  "\def\version{final}\input{childdoc.def}\childdocforward{cdocsch2}"|
% \end{tabular}
% \end{center}
% Note that the trailing backslash on each first line
% merely continues the input to the second line
% (for convenient cut ant paste).
% Furthermore, the command |latex| can be replaced by any
% of its alternative versions such as |pdflatex|.
%
% %%%%%%%%%%%%%%%%%%%%%%%%%%%%%%%%%%%%%%%%%%%%%%%%%%%%%%%%%%%%%%%%%%%%%%%%%%%%%%
% %%%%%%%%%%%%%%%%%%%%%%%%%%%%%%%%%%%%%%%%%%%%%%%%%%%%%%%%%%%%%%%%%%%%%%%%%%%%%%
% \section{Implementation}
%\iffalse
%<*package>
%\fi
%
% This section describes the definitions file |childdoc.def|.

% The definitions cannot be loaded using |\usepackage| or |\RequirePackage|
% which has a mechanism to prevent loading a style file more than once.
% When loading the definitions by means of |\input|
% multiple instances have to be prevented manually:
%\iffalse
%This code needs to be before the `\ProvidesFile' directive
%which is defined at the beginning of this file.
%Therefore it is also placed there and commented out here.
%</package>
%<*discard>
%\fi
%    \begin{macrocode}
\ifdefined\childdocmain\endinput\fi
%    \end{macrocode}
%\iffalse
%</discard>
%<*package>
%\fi
%
% \macro{\ifchilddoc}
% \macro{\ifchilddocmanual}
% The conditional |\ifchilddoc| tells whether a
% child (true) or main (false) document is being compiled.
% The conditional |\ifchilddocmanual| tells whether
% the |\includeonly| mechanism is used (false) or
% the selection of child files must be performed manually (true).
% The definitions initialise to false:
%    \begin{macrocode}
\newif\ifchilddoc
\newif\ifchilddocmanual
%    \end{macrocode}

% \macro{\childdocname}
% \macro{\childdocjob}
% The macro |\childdocname| stores the name of the main document
% to be compiled. The macro |\childdocjob| stores the name of
% the document on which the \LaTeX{} compiler was originally invoked.
% The content of |\jobname| cannot be compared
% to filenames specified in the source due to different catcodes.
% The following code rescans |\jobname|, stores the result
% in |\childdocname| and saves a copy in |\childdocjob|:
%    \begin{macrocode}
\edef\childdocname{\scantokens\expandafter{\jobname\noexpand}}
\let\childdocjob\childdocname
%    \end{macrocode}

% \macro{\childdocdisable}
% The macro |\childdocdisable| prevents the main file
% from being processed more than once.
% At this stage, the main document command |\childdocmain|
% is assumed to be called once again where it should do nothing.
% Any subsequent call to it should prevent
% a secondary processing of the main document
% It overwrites the forwarding commands
% |\childdocof| and |\childdocforward|
% with empty macros to prevent further inclusions of the main document:
%    \begin{macrocode}
\newcommand{\childdocdisable}
{
  \renewcommand{\childdocmain}[1]{\renewcommand{\childdocmain}[1]{\endinput}}
  \renewcommand{\childdocof}[1]{}
  \renewcommand{\childdocby}[2][]{}
  \renewcommand{\childdocforward}[2][]{}
  \renewcommand{\childdocdisable}{}
}
%    \end{macrocode}

% \macro{\childdocmain}
% The macro |\childdocmain| is to be called at the top of the main file
% with nothing or the main filename (without extension) as argument.
% First, it breaks loops.
% If the argument is not empty and does not match |\childdocname|
% (which is set by the first inclusion of |childdoc.def|),
% |\ifchilddoc| is set to true, |\includeonly| is applied to the child file
% and |\jobname| is set to the main file
% (for proper handling of |.aux| files):
%    \begin{macrocode}
\newcommand{\childdocmain}[1]
{
  \childdocdisable\childdocmain{}
  \if?#1?\else
    \begingroup
      \def\childdoctmp{#1}
      \ifx\childdoctmp\childdocname
        \def\childdoctmp{}
      \else
        \def\childdoctmp
        {
          \childdoctrue
          \includeonly{\childdocname}
          \def\childdocjob{#1}
          \def\jobname{#1}
        }
      \fi
      \expandafter
    \endgroup
    \childdoctmp
  \fi
}
%    \end{macrocode}

% \macro{\childdocof}
% The command |\childdocof| redirects
% compilation to the main file |#1|.
%    \begin{macrocode}
\newcommand{\childdocof}[1]
{
  \childdocdisable
  \childdoctrue
  \includeonly{\childdocname}
  \def\jobname{#1}
  \def\childdocjob{#1}
  \input{#1}
}
%    \end{macrocode}

% \macro{\childdocby}
% The command |\childdocby| ....
%    \begin{macrocode}
\newcommand{\childdocby}[2][]
{
  \childdocdisable
  \childdoctrue
  \childdocmanualtrue
  \if?#1?\else
    \def\jobname{#2}
  \fi
  \def\childdocjob{#2}
  \input{#2}
  \endinput
}
%    \end{macrocode}

% \macro{\childdocforward}
% The command |\childdocforward| redirects
% compilation to the main file or
% (if the optional argument is given) a child file.
% Parameters are set as if the main file
% or a child file starting with |\childdocof| was compiled.
% Then compilation is handed over to the main file:
%    \begin{macrocode}
\newcommand{\childdocforward}[2][]
{
  \begingroup
    \if?#1?
      \def\childdoctmp
      {
        \def\childdocname{#2}
        \def\childdocjob{#2}
        \def\jobname{#2}
        \input{#2}
        \endinput
      }
    \else
      \def\childdoctmp
      {
        \childdocdisable
        \def\childdocname{#2}
        \childdoctrue
        \includeonly{#2}
        \def\childdocjob{#1}
        \def\jobname{#1}
        \input{#1}
        \endinput
      }
    \fi
    \expandafter
  \endgroup
  \childdoctmp
}
%    \end{macrocode}

% \macro{\childdocforwardprefix}
% The command |\childdocforwardprefix| redirects
% compilation to the main or a child file by means of a pattern.
% The prefix |#1| in the current filename is replaced by |#2|
% and the suffix of the current filename is kept
% (it is assumed that the filename does not contain the substring `|~~~|'
% which is used as a delimiter).
% Compilation is handed over to the new file by |\childdocforward|:
%    \begin{macrocode}
\newcommand{\childdocforwardprefix}[3][]
{
  \begingroup
    \def\childdocextract #2##1~~~{\def\childdoctmp{\childdocforward[#1]{#3##1}}}
    \expandafter\childdocextract\childdocname~~~
    \expandafter
  \endgroup
  \childdoctmp
}
%    \end{macrocode}

% \macro{\childdoc}
% The deprecated macro |\childdoc| is a legacy version of |\childdocmain|:
%    \begin{macrocode}
\newcommand{\childdoc}{\childdocmain}
%    \end{macrocode}

% \macro{\childdocredirect}
% The deprecated macro |\childdocredirect| is a legacy version
% of |\childdocforward| and |\childdocforwardprefix|:
%    \begin{macrocode}
\newcommand{\childdocredirect}[2][]
{
  \begingroup
    \if?#1?
      \def\childdoctmp{\childdocforward{#2}}
    \else
      \def\childdoctmp{\childdocforwardprefix{#1}{#2}}
    \fi
    \expandafter
  \endgroup
  \childdoctmp
}
%    \end{macrocode}

%\iffalse
%</package>
%\fi
%
\endinput
|\\
|\childdocby{|\textit{main}|}|\\
\end{tabular}
\end{center}
%
Both forms have slightly different effects as described above.
The main file is prepared as usual, see \secref{sec:include}.

%%%%%%%%%%%%%%%%%%%%%%%%%%%%%%%%%%%%%%%%%%%%%%%%%%%%%%%%%%%%%%%%%%%%%%%%%%%%%%%%
\subsection{Legacy Detection}
\label{sec:detection}

The directive |\childdocmain| in the main file can detect
whether the complete document or merely a child is to be compiled
even without using the directive |\childdocof|.
This method is deprecated because it is less robust
and there is no compelling reason to use it;
it is merely provided for backward compatibility
and it may be removed in future versions.

If the detection mechanism is to be used,
it is mandatory to correctly specify
the filename of the main file as the argument of |\childdocmain|:
%
\begin{center}
\begin{tabular}{l}
|% \iffalse
%
% childdoc.dtx Copyright (C) 2017-2018 Niklas Beisert
%
% This work may be distributed and/or modified under the
% conditions of the LaTeX Project Public License, either version 1.3
% of this license or (at your option) any later version.
% The latest version of this license is in
%   http://www.latex-project.org/lppl.txt
% and version 1.3 or later is part of all distributions of LaTeX
% version 2005/12/01 or later.
%
% This work has the LPPL maintenance status `maintained'.
%
% The Current Maintainer of this work is Niklas Beisert.
%
% This work consists of the files childdoc.dtx and childdoc.ins
% and the derived files childdoc.def and cdocsamp.tex with
% cdocsch1.tex, cdocsch2.tex, cdocsdrf.tex, cdocsfn1.tex, cdocsfn2.tex.
%
%<package>\ifdefined\childdocmain\endinput\fi
%<package>\ProvidesFile{childdoc.def}[2018/12/30 v2.0 child document driver]
%<samplemain>\ProvidesFile{cdocsamp.tex}[2018/12/30 v2.0 sample for childdoc]
%<*driver>
%\ProvidesFile{childdoc.drv}[2018/12/30 v2.0 childdoc reference manual file]
\PassOptionsToClass{10pt,a4paper}{article}
\documentclass{ltxdoc}

\usepackage[margin=35mm]{geometry}
\usepackage{hyperref}
\usepackage{hyperxmp}
\usepackage[usenames]{color}

\hypersetup{colorlinks=true}
\hypersetup{pdfstartview=FitH}
\hypersetup{pdfpagemode=UseNone}
\hypersetup{pdfsource={}}
\hypersetup{pdflang={en-UK}}
\hypersetup{pdfcopyright={Copyright 2017-2018 Niklas Beisert.
  This work may be distributed and/or modified under the
  conditions of the LaTeX Project Public License, either version 1.3
  of this license or (at your option) any later version.}}
\hypersetup{pdflicenseurl={http://www.latex-project.org/lppl.txt}}
\hypersetup{pdfcontactaddress={ETH Zurich, ITP, HIT K,
  Wolfgang-Pauli-Strasse 27}}
\hypersetup{pdfcontactpostcode={8093}}
\hypersetup{pdfcontactcity={Zurich}}
\hypersetup{pdfcontactcountry={Switzerland}}
\hypersetup{pdfcontactemail={nbeisert@itp.phys.ethz.ch}}
\hypersetup{pdfcontacturl={http://people.phys.ethz.ch/\xmptilde nbeisert/}}

\newcommand{\secref}[1]{\hyperref[#1]{section \ref*{#1}}}

\parskip1ex
\parindent0pt
\let\olditemize\itemize
\def\itemize{\olditemize\parskip0pt}

\begin{document}

\title{The \textsf{childdoc} Package}
\hypersetup{pdftitle={The childdoc Package}}
\author{Niklas Beisert\\[2ex]
  Institut f\"ur Theoretische Physik\\
  Eidgen\"ossische Technische Hochschule Z\"urich\\
  Wolfgang-Pauli-Strasse 27, 8093 Z\"urich, Switzerland\\[1ex]
  \href{mailto:nbeisert@itp.phys.ethz.ch}
  {\texttt{nbeisert@itp.phys.ethz.ch}}}
\hypersetup{pdfauthor={Niklas Beisert}}
\hypersetup{pdfsubject={Manual for the LaTeX2e Package childdoc}}
\date{30 December 2018, \textsf{v2.0}}
\maketitle

\begin{abstract}\noindent
\textsf{childdoc} is a \LaTeXe{} package
that enables the direct compilation
of document sections included by |\include|
to individual files.
\end{abstract}

\begingroup
\parskip0ex
\tableofcontents
\endgroup

%%%%%%%%%%%%%%%%%%%%%%%%%%%%%%%%%%%%%%%%%%%%%%%%%%%%%%%%%%%%%%%%%%%%%%%%%%%%%%%%
%%%%%%%%%%%%%%%%%%%%%%%%%%%%%%%%%%%%%%%%%%%%%%%%%%%%%%%%%%%%%%%%%%%%%%%%%%%%%%%%
\section{Introduction}

\LaTeX{} provides a mechanism to structure a large document (such as a book)
into a main file and several child files (containing the chapters)
using the |\include| command.
This mechanism is beneficial for documents
which span hundreds of pages in order to
make the source file(s) more manageable.
Moreover, compilation can be restricted to
selected child files by means of the |\includeonly| command.
The latter feature can be used to reduce the compilation time while editing
(this was significantly more useful in the earlier days of \LaTeX{})
or to generate a smaller document which is easier to navigate.
Another application of |\includeonly| is to generate
documents consisting of selected parts of the complete document.

However, there are a few drawbacks of the plain |\include| mechanism:
\begin{itemize}
\item
The child files cannot be compiled on their own,
they can only be compiled via the main file.
A naive editing environment
(such as a text editor with an option
to have the current file processed by \LaTeX)
may require one to switch to the main file before compiling;
attempting to compile the child file produces errors.
\item
The main file must be modified (each time)
to adjust the |\includeonly| command
to the present needs. This easily leaves the main file in a messy state.
\item
The generated document will always carry the filename
of the main document. This is inconvenient if
several child files are to be compiled and
to be kept for distribution.
\end{itemize}

The present package provides a simple interface
to make child files individually compilable by \LaTeX{}.
Compiling a child file then has the same effect as compiling
the main file with an |\includeonly| command
to select the appropriate child.
Moreover the generated document will carry the name of the child
rather than the main file.
This resolves all three above issues.

This feature is meant to make the editing of books,
thesis documents and lecture notes somewhat more convenient.
However, the package can also be used efficiently for
composing a series of documents (such as exercise sheets)
which are typically distributed individually.
It then assists the author in generating the individual documents
(potentially in different versions)
as well as a document containing the collected series.
Another application is in developing style files
or other kinds of included material
where compilation of the style file could redirect
to a sample or test file.

%%%%%%%%%%%%%%%%%%%%%%%%%%%%%%%%%%%%%%%%%%%%%%%%%%%%%%%%%%%%%%%%%%%%%%%%%%%%%%%%
%%%%%%%%%%%%%%%%%%%%%%%%%%%%%%%%%%%%%%%%%%%%%%%%%%%%%%%%%%%%%%%%%%%%%%%%%%%%%%%%
\section{Usage}

First of all, the package \textsf{childdoc} is \emph{not} a standard
\LaTeXe{} |.sty| style file! Therefore it needs to be invoked in
a non-standard way.

%%%%%%%%%%%%%%%%%%%%%%%%%%%%%%%%%%%%%%%%%%%%%%%%%%%%%%%%%%%%%%%%%%%%%%%%%%%%%%%%
\subsection{Included Files}
\label{sec:include}

%%%%%%%%%%%%%%%%%%%%%%%%%%%%%%%%%%%%%%%%
\DescribeMacro{\childdocmain}
To use the package, add the commands
\begin{center}
\begin{tabular}{l}
|\input{childdoc.def}|\\
|\childdocmain{}|\\
\end{tabular}
\end{center}
at the very top of the main \LaTeX{} file,
in particular \emph{before} the |\documentclass| statement!
The argument of |\childdocmain| should be left empty
(but it must be present).

%%%%%%%%%%%%%%%%%%%%%%%%%%%%%%%%%%%%%%%%
\DescribeMacro{\childdocof}
Furthermore, add the commands
\begin{center}
\begin{tabular}{l}
|\input{childdoc.def}|\\
|\childdocof{|\textit{main}|}|\\
\end{tabular}
\end{center}
at the top of every child file \textit{child}
which is included by |\include{|\textit{child}|}|
from within the main file
(or at least for those files to be compiled individually).
The argument \textit{main} must be the filename of the main file.

There are a couple of
considerations in setting up the main and child documents:

%%%%%%%%%%%%%%%%%%%%%%%%%%%%%%%%%%%%%%%%
\paragraph{Restrictions.}

Please note the following restrictions:
\begin{itemize}
\item
|\childdocmain| must be called with one argument \textit{main}
to ensure compatibility with earlier version of the package.
It must either be empty (|\childdocmain{}|)
or precisely match the filename of the main file in which it is specified.
See \secref{sec:detection} for further information.
\item
The filename \textit{main} must be specified without the |.tex| extension.
\item
The filename \textit{main} is case sensitive
(even in case-insensitive file systems)
due to internal string comparison.
\item
The argument \textit{main} should be fully expanded, it cannot be a macro.
\item
Subdirectories and special characters should be avoided in filenames.
\item
The command |\childdocmain{|\textit{main}|}| must be followed by a whitespace.
It should not be followed immediately by another command
or by a comment mark `|%|'.
This is because the \TeX{} parser reads the token immediately following
the argument of |\childdocmain| and puts it
at the beginning of every child section;
however, a white\-space is ignored.
\end{itemize}

%%%%%%%%%%%%%%%%%%%%%%%%%%%%%%%%%%%%%%%%
\paragraph{Content of Main File.}

It is advisable to place all content in the child files included by |\include|.
Any output contained in the main file will appear in all child documents
unless suppressed manually;
it cannot be suppressed automatically by the |\includeonly| directive
and thus should normally be avoided.
A method to include some content in the main file
by means of conditional processing is described in \secref{sec:conditional}.

%%%%%%%%%%%%%%%%%%%%%%%%%%%%%%%%%%%%%%%%
\paragraph{Page Numbering.}

When only a part of the document is compiled,
the appropriate numbering of pages
(as well as other status parameters)
is determined from the |.aux| files.
The latter contain information from previous passes.
However this information needs to propagate through
all intermediate child documents.
Therefore the page numbering in child documents may well
be inconsistent until the complete document is compiled at least once.

A useful (if unconventional) way to always ensure a consistent
page numbering is to restart the numbering in each child document
and denote the pages by `\textit{child}|.|\textit{page}'
where \textit{child} represents the chapter/section number of the child file.
This can be achieved by the command
|\numberwithin{page}{|\textit{child}|}|
of the \textsf{amsmath} package
where \textit{child} can be |chapter| or |section|
depending on the chosen structuring.
Alternatively, one can modify the macro |\thepage| appropriately
and reset the counter |page| at the start of each child file.

%%%%%%%%%%%%%%%%%%%%%%%%%%%%%%%%%%%%%%%%%%%%%%%%%%%%%%%%%%%%%%%%%%%%%%%%%%%%%%%%
\subsection{Conditional Processing}
\label{sec:conditional}

The package provides a mechanism to compile different versions
of a document. To customise the versions further some conditional processing
can come in handy to distinguish which version is being compiled.
The package provides two macros to describe the compilation context:

%%%%%%%%%%%%%%%%%%%%%%%%%%%%%%%%%%%%%%%%
\DescribeMacro{\ifchilddoc}
The conditional |\ifchilddoc| distinguishes between the compilation of
child documents and the main document:
%
\begin{center}
|\ifchilddoc |\textit{child-code}| |[|\||else |\textit{main-code}]| \||fi|
\end{center}

%%%%%%%%%%%%%%%%%%%%%%%%%%%%%%%%%%%%%%%%
\DescribeMacro{\childdocname}
\DescribeMacro{\childdocjob}
The macro |\childdocname| contains the filename (without extension)
of the main or child file being processed.
Note that |\childdocjob| will always contain the name of the main file.

%%%%%%%%%%%%%%%%%%%%%%%%%%%%%%%%%%%%%%%%
\paragraph{Title Page.}

Conditional processing can be used to include a title or banner page
in the main document when proper precautions are taken.
Importantly, the code in the main file should ensure that the page counter
(as well as other status parameters which are stored in the |.aux| files)
takes the same value after the conditional processing.
Otherwise the page numbers may take divergent values
depending on which part is compiled.

For example, a title page could be declared by:
%
\begin{center}
\begin{tabular}{l}
|\ifchilddoc\||else|\\
|\addtocounter{page}{-1}|\\
\textit{code for title page}\\
|\newpage|\\
|\||fi|
\end{tabular}
\end{center}
%
A banner page for the child documents can be generated by:
%
\begin{center}
\begin{tabular}{l}
|\ifchilddoc|\\
|\addtocounter{page}{-1}|\\
\textit{code for banner page}\\
|\newpage|\\
|\||fi|
\end{tabular}
\end{center}
%
Here one could write a message such as:
\begin{center}
|This is the part \childdocname{} of \childdocjob{}.|
\end{center}

%%%%%%%%%%%%%%%%%%%%%%%%%%%%%%%%%%%%%%%%%%%%%%%%%%%%%%%%%%%%%%%%%%%%%%%%%%%%%%%%
\subsection{Flags}
\label{sec:flags}

The package makes it easy to generate different versions
of the main or child documents.
To this end compilation flags can be defined
and assigned different default values.
They will be particularly useful in conjunction
with the forwarding mechanism described in \secref{sec:forward}.

For example, it may be useful to have a flag |\version|
which can be set to |draft| or |final|.
The document source will contain some conditional code
depending on the value of |\version|.
Suppose further, the flag should default to |final| for the main file
and to |draft| for child files
which is a natural assignment for editing the document.
This is achieved by placing the following code
in the preamble of the main document
(below the |\childdocmain| directive):
%
\begin{center}
\begin{tabular}{l}
|\ifchilddoc|\\
|\providecommand{\version}{draft}|\\
|\||else|\\
|\providecommand{\version}{final}|\\
|\||fi|
\end{tabular}
\end{center}
%
The definition by |\providecommand| makes sure
that previous definitions are not overwritten.
Further statements |\providecommand{\version}{...}|
can thus be added before the above code to override it.

For the main file, one might add a line
(between |\childdocmain| and the above block)
%
\begin{center}
|%\ifchilddoc\||else\providecommand{\version}{draft}\||fi|
\end{center}
%
which can be uncommented to produce a draft version.
Likewise one can add a line to the very top of a child file
(above the |\childdocof{|\textit{main}|}| directive)
%
\begin{center}
|%\providecommand{\version}{final}|
\end{center}
%
which can be uncommented to produce the final version of this child document.

%%%%%%%%%%%%%%%%%%%%%%%%%%%%%%%%%%%%%%%%%%%%%%%%%%%%%%%%%%%%%%%%%%%%%%%%%%%%%%%%
\subsection{Forwarding}
\label{sec:forward}

Different versions of the main or child documents
using compilation flags as described in \secref{sec:flags}
can be (permanently) stored in different files
for convenient compilation, viewing and distribution.
To this end, the package defines a command
to pass on compilation to a different file:

%%%%%%%%%%%%%%%%%%%%%%%%%%%%%%%%%%%%%%%%
\DescribeMacro{\childdocforward}
The command |\childdocforward| redirects processing to
another source file:
%
\begin{center}
\begin{tabular}{l}
|\input{childdoc.def}|\\
|\childdocforward[|\textit{main}|]{|\textit{dest}|}|\\
\end{tabular}
\end{center}
%
The argument \textit{dest} is the destination file
(without extension).
It should be the main file or one of the child files.
Note that further \textsf{childdoc} directives
such as |\childdocof| and |\childdocforward|
in the indicated file will be processed in this form.
The optional argument \textit{main}
passes on directly to the main file \textit{main}
while pretending to compile the child \textit{dest}.
This form behaves as if \textit{dest}
issues |\childdocof{|\textit{main}|}| right away,
and no further \textsf{childdoc} directives will be processed.

%%%%%%%%%%%%%%%%%%%%%%%%%%%%%%%%%%%%%%%%
\DescribeMacro{\...prefix}
In the alternative form |\childdocforwardprefix|,
%
\begin{center}
\begin{tabular}{l}
|\input{childdoc.def}|\\
|\childdocforwardprefix[|\textit{main}|]{|\textit{prefix}|}{|\textit{dest}|}|
\end{tabular}
\end{center}
%
the destination file is determined by a pattern
depending on the current file:
To make this work, the current file must be called
`{\textit{prefix}\hspace{0.2em}\textit{suffix}}'
with \textit{prefix} matching precisely the argument.
Processing is then passed on to the file
`{\textit{dest}\hspace{0.2em}\textit{suffix}}'.
Surely, the same effect is achieved by
directly specifying the
argument `{\textit{dest}\hspace{0.2em}\textit{suffix}}'
in the first form.
However, that requires to set up a different file
for each child. With the alternative form of the command
all these files can have exactly the same content
which simplifies setting them up and maintaining them.

For example, the following file |draft.tex|
with a compilation flag |\version| as described in \secref{sec:flags}
compiles the main document as a draft:
%
\begin{center}
\begin{tabular}{l}
|\def\version{draft}|\\
|\input{childdoc.def}|\\
|\childdocforward{|\textit{main}|}|
\end{tabular}
\end{center}
%
Likewise, the following files |final|\textit{nn}|.tex|
compile the final version of the child document
|child|\textit{nn}|.tex|:
%
\begin{center}
\begin{tabular}{l}
|\def\version{final}|\\
|\input{childdoc.def}|\\
|\childdocforwardprefix{final}{child}|
\end{tabular}
\end{center}
%

Note that when several versions of a main file and/or of each child file
are to be generated, it may be convenient to set up a |Makefile| or
shell script to automatise the process.

%%%%%%%%%%%%%%%%%%%%%%%%%%%%%%%%%%%%%%%%%%%%%%%%%%%%%%%%%%%%%%%%%%%%%%%%%%%%%%%%
\subsection{Command Line Processing}
\label{sec:commandline}

The effect of redirection files can also be achieved by invoking
the \LaTeX{} compiler with a more elaborate command line.
Most conveniently this should be done as part
of a shell script or a |Makefile|.

When using \textsf{childdoc} in the main file, the following
command lines effectively perform a redirection
(note that depending on the shell being used,
backslashes may have to be doubled: `|\|' $\to$ `|\\|'):
%
\begin{center}
|... -jobname "|\textit{target}|" |\\|"|[\textit{flags}]%
|\input{childdoc.def}\childdocforward[|\textit{main}|]{|\textit{dest}|}"|
\end{center}
%
Here \textit{target} is the name of the output file,
\textit{main} is the name of the main file
and \textit{dest} is the name of the main or child file to be processed
(all filenames without extensions).
The optional argument \textit{main} can be omitted
if \textit{main} matches \textit{dest}.
Optionally, compilation \textit{flags} can be defined via |\def| commands.
This command line makes the \TeX{} engine believe
it is compiling the file \textit{target}
whose content is specified as the latter parameter.
The provided code then forwards the processing to
\textit{main} or \textit{dest} as described in \secref{sec:forward}.

%%%%%%%%%%%%%%%%%%%%%%%%%%%%%%%%%%%%%%%%%%%%%%%%%%%%%%%%%%%%%%%%%%%%%%%%%%%%%%%%
\subsection{Include by Input}
\label{sec:input}

Including child documents by |\include| has some restrictions by design.
Most notably, the content of a child document always occupies
its own set of pages; pages cannot be shared between child documents.
Usually, this behaviour makes perfect sense
because each child document contain an essential part of the document.
However, in some situations it may be desirable to compose
a document from a collection of parts
without having mandatory page breaks between then.
For this case, the package
provides a mechanism to include parts
by |\input| which can also be processed individually.
However, by construction this mechanism
requires manual handling of the content to be output.

%%%%%%%%%%%%%%%%%%%%%%%%%%%%%%%%%%%%%%%%
\DescribeMacro{\ifchilddocmanual}
The main file should be prepared as usual, see \secref{sec:include}.
However, the document body must make a distinction
between processing of an individual part and of the main document, e.g.:
%
\begin{center}
\begin{tabular}{l}
|\ifchilddocmanual|\\
|\input{\childdocname}|\\
|\||else|\\
\textit{document body with }|\input{|\textit{part}|}|\\
|\||fi|
\end{tabular}
\end{center}
%
The conditional |\ifchilddocmanual| is true whenever
a part to be included by |\input| is being compiled,
and the name of the part is stored in |\childdocname|.

%%%%%%%%%%%%%%%%%%%%%%%%%%%%%%%%%%%%%%%%
\DescribeMacro{\childdocby}
Each part to be included by |\input| should start with:
%
\begin{center}
\begin{tabular}{l}
|\input{childdoc.def}|\\
|\childdocby{|\textit{main}|}|\\
\end{tabular}
\end{center}
%
The directive |\childdocby| is similar to |\childdocof|
described in \secref{sec:include},
but the subsequent selection of content must be done manually.
To that end, both |\ifchilddoc| and |\ifchilddocmanual|
will be true upon processing of a part,
and the name of the part is stored in |\childdocname|.
Note that |\jobname| will be set to the filename of the current part
so that each part receives an individual |.aux| file
that does not interfere with the |.aux| file(s) of the main document.
This behaviour can be altered by the alternative form
|\childdocby[*]{|\textit{main}|}| (with a non-empty optional argument)
which uses the |.aux| file of the main document
by setting |\jobname| to \textit{main}.

%%%%%%%%%%%%%%%%%%%%%%%%%%%%%%%%%%%%%%%%%%%%%%%%%%%%%%%%%%%%%%%%%%%%%%%%%%%%%%%%
\subsection{Driver Development}
\label{sec:driver}

The \textsf{childdoc} mechanism can also be use for the development
of definition files such as \LaTeX{} styles or classes.
This case differs from the above setup with multiple parts
included by |\include| in that no |\includeonly| should be invoked.
This can be achieved by starting the include file
(before |\ProvidesPackage|) with:
%
\begin{center}
\begin{tabular}{l}
|\input{childdoc.def}|\\
|\childdocforward{|\textit{main}|}|\\
\end{tabular}
\end{center}
%
or alternatively with:
%
\begin{center}
\begin{tabular}{l}
|\input{childdoc.def}|\\
|\childdocby{|\textit{main}|}|\\
\end{tabular}
\end{center}
%
Both forms have slightly different effects as described above.
The main file is prepared as usual, see \secref{sec:include}.

%%%%%%%%%%%%%%%%%%%%%%%%%%%%%%%%%%%%%%%%%%%%%%%%%%%%%%%%%%%%%%%%%%%%%%%%%%%%%%%%
\subsection{Legacy Detection}
\label{sec:detection}

The directive |\childdocmain| in the main file can detect
whether the complete document or merely a child is to be compiled
even without using the directive |\childdocof|.
This method is deprecated because it is less robust
and there is no compelling reason to use it;
it is merely provided for backward compatibility
and it may be removed in future versions.

If the detection mechanism is to be used,
it is mandatory to correctly specify
the filename of the main file as the argument of |\childdocmain|:
%
\begin{center}
\begin{tabular}{l}
|\input{childdoc.def}|\\
|\childdocmain{|\textit{main}|}|\\
\end{tabular}
\end{center}
%
If |\jobname| does not match the argument \textit{main} of |\childdocmain|,
it is assumed that |\jobname| points to the child file to be compiled.
When using |\childdocmain| with the main file specified as argument,
it suffices to start a child file
with just |\input{|\textit{main}|}|
without loading of the package and using |\childdocof|.
If instead all processing is done
with the appropriate \textsf{childdoc} directives,
the argument of \textit{main} of |\childdocmain| can be empty.

An alternative version of the command line processing described
in \secref{sec:commandline} using the detection mechanism reads:
%
\begin{center}
|... -jobname "|\textit{target}|" "|[\textit{flags}]%
[|\def\jobname{|\textit{dest}|}|]|\input{|\textit{main}|}"|
\end{center}

%%%%%%%%%%%%%%%%%%%%%%%%%%%%%%%%%%%%%%%%%%%%%%%%%%%%%%%%%%%%%%%%%%%%%%%%%%%%%%%%
\subsection{Manual Code}
\label{sec:manual}

In case one cannot be certain whether the definitions file |childdoc.def|
is installed on the target \TeX{} distribution
and one prefers not to ship it,
it is conceivable to paste a few relevant commands into the sources.

To that end, drop all statements |\input{childdoc.def}|
and perform the replacements as outlined below.
Instead of |\childdocmain{|\textit{main}|}| add the following code
to the top of the main file:
%
\begin{center}
\begin{tabular}{l}
|\||ifdefined\childdocname\endinput\||fi\newif\ifchilddoc|\\
|\edef\childdocname{\scantokens\expandafter{\jobname\noexpand}}|\\
|\def\childdocmain{|\textit{main}|}\||ifx\childdocmain\childdocname\||else|\\
|\childdoctrue\includeonly{\childdocname}\let\jobname\childdocmain\||fi|\\
\end{tabular}
\end{center}
%
Instead of |\childdocof{|\textit{main}|}| just include the main file
at the top of each child file:
%
\begin{center}
|\input{|\textit{main}|}|
\end{center}
%
A simple redirection |\childdocforward{|\textit{dest}|}| is achieved by:
%
\begin{center}
|\def\jobname{|\textit{dest}|}\input{\jobname}|
\end{center}
%
The redirection with prefix
|\childdocforwardprefix[|\textit{prefix}|]{|\textit{dest}|}|
is accomplished by:
%
\begin{center}
\begin{tabular}{l}
|{\edef\jobname{\scantokens\expandafter{\jobname\noexpand}}|\\
|\def\redirectjob |\textit{prefix}|#1~~~{\gdef\jobname{|\textit{dest}|#1}}|\\
|\expandafter\redirectjob\jobname~~~}\input{\jobname}|
\end{tabular}
\end{center}

In an alternative approach,
child documents can be compiled by a specific command line
without additional code or specific definitions:
%
\begin{center}
|... -jobname "|\textit{target}|" "|[\textit{flags}]%
|\includeonly{|\textit{dest}|}\input{|\textit{main}|}"|
\end{center}
%

%%%%%%%%%%%%%%%%%%%%%%%%%%%%%%%%%%%%%%%%%%%%%%%%%%%%%%%%%%%%%%%%%%%%%%%%%%%%%%%%
%%%%%%%%%%%%%%%%%%%%%%%%%%%%%%%%%%%%%%%%%%%%%%%%%%%%%%%%%%%%%%%%%%%%%%%%%%%%%%%%
\section{Information}

%%%%%%%%%%%%%%%%%%%%%%%%%%%%%%%%%%%%%%%%%%%%%%%%%%%%%%%%%%%%%%%%%%%%%%%%%%%%%%%%
\subsection{Copyright}

Copyright \copyright{} 2017--2018 Niklas Beisert

This work may be distributed and/or modified under the
conditions of the \LaTeX{} Project Public License, either version 1.3
of this license or (at your option) any later version.
The latest version of this license is in
  \url{http://www.latex-project.org/lppl.txt}
and version 1.3 or later is part of all distributions of \LaTeX{}
version 2005/12/01 or later.

This work has the LPPL maintenance status `maintained'.

The Current Maintainer of this work is Niklas Beisert.

This work consists of the files |README.txt|, |childdoc.ins| and |childdoc.dtx|
as well as the derived files |childdoc.def|, |cdocsamp.tex|
with |cdocsch1.tex|, |cdocsch2.tex|, |cdocspt3.tex|, |cdocspt4.tex|,
|cdocsdrf.tex|, |cdocsfn1.tex|, |cdocsfn2.tex|
as well as |childdoc.pdf|.

%%%%%%%%%%%%%%%%%%%%%%%%%%%%%%%%%%%%%%%%%%%%%%%%%%%%%%%%%%%%%%%%%%%%%%%%%%%%%%%%
\subsection{Files and Installation}

The package consists of the files:
%
\begin{center}
\begin{tabular}{ll}
    |README.txt|   & readme file \\
    |childdoc.ins| & installation file \\
    |childdoc.dtx| & source file \\
    |childdoc.def| & definition file \\
    |cdocsamp.tex| & sample main file \\
    |cdocsch1.tex| & sample include file \\
    |cdocsch2.tex| & sample include file \\
    |cdocspt3.tex| & sample part file \\
    |cdocspt4.tex| & sample part file \\
    |cdocsdrf.tex| & sample redirection file \\
    |cdocsfn1.tex| & sample redirection file \\
    |cdocsfn2.tex| & sample redirection file \\
    |childdoc.pdf| & manual
\end{tabular}
\end{center}
%
The distribution consists of the files
|README.txt|, |childdoc.ins| and |childdoc.dtx|.
%
\begin{itemize}
\item
Run (pdf)\LaTeX{} on |childdoc.dtx|
to compile the manual |childdoc.pdf| (this file).
\item
Run \LaTeX{} on |childdoc.ins| to create the definitions file |childdoc.def|
and the sample |cdocsamp.tex| with include files
|cdocsch1.tex|, |cdocsch2.tex|, |cdocspt3.tex|, |cdocspt4.tex|,
|cdocsdrf.tex|, |cdocsfn1.tex|, |cdocsfn2.tex|.
Then copy the file |childdoc.def| to an appropriate directory of your \LaTeX{}
distribution, e.g.\ \textit{texmf-root}|/tex/latex/childdoc|.
\end{itemize}

%%%%%%%%%%%%%%%%%%%%%%%%%%%%%%%%%%%%%%%%%%%%%%%%%%%%%%%%%%%%%%%%%%%%%%%%%%%%%%%%
\subsection{Related CTAN Packages}

There are several other packages which offer a similar functionality:
%
\begin{itemize}
\item
The packages
\href{http://ctan.org/pkg/docmute}{\textsf{docmute}},
\href{http://ctan.org/pkg/includex}{\textsf{includex}} and
\href{http://ctan.org/pkg/standalone}{\textsf{standalone}}
provide commands to include only the document body of
a child file thus allowing both files to be compiled individually.
\item
The packages \href{http://ctan.org/pkg/subdocs}{\textsf{subdocs}}
and \href{http://ctan.org/pkg/subfiles}{\textsf{subfiles}}
provide structures in which the main and child documents can be
encapsulated and allowing them to be compiled individually.
The inclusion mechanism is different from the conventional |\include|.
\item
The package \href{http://ctan.org/pkg/combine}{\textsf{combine}}
is an elaborate solution to combine several documents into one.
\end{itemize}
%
See also the CTAN topic \href{http://ctan.org/topic/subdocs}{\textsf{subdocs}}
for further related packages.
The present package differs from the above solutions in that
a document structure constructed with the conventional |\include| mechanism
just needs two extra commands at the top of every file
such that all constituent files can be compiled individually.

%%%%%%%%%%%%%%%%%%%%%%%%%%%%%%%%%%%%%%%%%%%%%%%%%%%%%%%%%%%%%%%%%%%%%%%%%%%%%%%%
%\subsection{Feature Suggestions}
%
%The following is a list of features which may be useful for future
%versions of this package:
%%
%\begin{itemize}
%\item
%\ldots
%\end{itemize}

%%%%%%%%%%%%%%%%%%%%%%%%%%%%%%%%%%%%%%%%%%%%%%%%%%%%%%%%%%%%%%%%%%%%%%%%%%%%%%%%
\subsection{Revision History}

%%%%%%%%%%%%%%%%%%%%%%%%%%%%%%%%%%%%%%%%
\paragraph{v2.0:} 2018/12/30

\begin{itemize}
\item
immediate forward processing
\item
added |\childdocby| mechanism
\item
manual restructured
\end{itemize}

%%%%%%%%%%%%%%%%%%%%%%%%%%%%%%%%%%%%%%%%
\paragraph{v1.6:} 2018/01/17

\begin{itemize}
\item
application for development of include files
\item
corrections to manual
\end{itemize}

%%%%%%%%%%%%%%%%%%%%%%%%%%%%%%%%%%%%%%%%
\paragraph{v1.5:} 2017/05/21

\begin{itemize}
\item
more complete structuring introduced
\item
|\childdocof| introduced
\item
|\childdoc| renamed to |\childdocmain|
\item
|\childredirect| renamed to |\childdocforward| and |\childdocforwardprefix|
and functionality expanded
\end{itemize}

%%%%%%%%%%%%%%%%%%%%%%%%%%%%%%%%%%%%%%%%
\paragraph{v1.0:} 2017/04/27

\begin{itemize}
\item
manual and install package
\item
first version published on CTAN
\end{itemize}

%%%%%%%%%%%%%%%%%%%%%%%%%%%%%%%%%%%%%%%%
\paragraph{v0.6:} 2017/04/26

\begin{itemize}
\item
redirection mechanism added
\end{itemize}

%%%%%%%%%%%%%%%%%%%%%%%%%%%%%%%%%%%%%%%%
\paragraph{v0.5:} 2017/04/26

\begin{itemize}
\item
functionality in definition file
\end{itemize}


%%%%%%%%%%%%%%%%%%%%%%%%%%%%%%%%%%%%%%%%%%%%%%%%%%%%%%%%%%%%%%%%%%%%%%%%%%%%%%%%
%%%%%%%%%%%%%%%%%%%%%%%%%%%%%%%%%%%%%%%%%%%%%%%%%%%%%%%%%%%%%%%%%%%%%%%%%%%%%%%%
%%%%%%%%%%%%%%%%%%%%%%%%%%%%%%%%%%%%%%%%%%%%%%%%%%%%%%%%%%%%%%%%%%%%%%%%%%%%%%%%
\appendix

\settowidth\MacroIndent{\rmfamily\scriptsize 000\ }

 \DocInput{childdoc.dtx}

\end{document}
%</driver>
% \fi
%
% %%%%%%%%%%%%%%%%%%%%%%%%%%%%%%%%%%%%%%%%%%%%%%%%%%%%%%%%%%%%%%%%%%%%%%%%%%%%%%
% %%%%%%%%%%%%%%%%%%%%%%%%%%%%%%%%%%%%%%%%%%%%%%%%%%%%%%%%%%%%%%%%%%%%%%%%%%%%%%
% \section{Sample}
%\iffalse
%<*samplemain>
%\fi
%
% The following presents a sample document
% with two chapters, two parts, a title page,
% a compile flag as well as three forwarding files to set the flag.
% It consists of eight |.tex| files:
% \begin{center}
% \begin{tabular}{ll}
% |cdocsamp.tex|&main file\\
% |cdocsch1.tex|&include file for chapter 1\\
% |cdocsch2.tex|&include file for chapter 2\\
% |cdocspt3.tex|&include file for part 3\\
% |cdocspt4.tex|&include file for part 4\\
% |cdocsdrf.tex|&forwarding file for main file in draft mode\\
% |cdocsfi1.tex|&forwarding file for final version of chapter 1\\
% |cdocsfi2.tex|&forwarding file for final version of chapter 2\\
% \end{tabular}
% \end{center}
% Each of the eight files can be compiled directly by the \LaTeX{} compiler.
%
% %%%%%%%%%%%%%%%%%%%%%%%%%%%%%%%%%%%%%%
% \paragraph{Main File.}
%
% The main file is called |cdocsamp.tex|.
%
% Load the \textsf{childdoc} definitions and
% declare the filename for the main document:
%    \begin{macrocode}
\input{childdoc.def}
\childdocmain{}
%    \end{macrocode}

% Optional override for |\version| flag:
%    \begin{macrocode}
%%\ifchilddoc\else\providecommand{\version}{draft}\fi
%    \end{macrocode}

% Define the default values for the |\version| flag
% (|final| for the main file and |draft| for childs):
%    \begin{macrocode}
\ifchilddoc
\providecommand{\version}{draft}
\else
\providecommand{\version}{final}
\fi
%    \end{macrocode}

% Load the standard document class:
%    \begin{macrocode}
\documentclass[12pt]{article}
%    \end{macrocode}

% Start the document body:
%    \begin{macrocode}
\begin{document}
%    \end{macrocode}

% Declare a title page.
% Print title, part of document being processed and version flag:
%    \begin{macrocode}
\addtocounter{page}{-1}
\begin{center}
{\LARGE\bfseries{}childdoc example\par}
\vspace{1cm}
\ifchilddoc
\ifchilddocmanual part\else chapter\fi:
`\childdocname' of `\childdocjob'\par
\else
main document: `\childdocjob'\par
\fi
version: \version\par
\end{center}
\newpage
%    \end{macrocode}

% Manually include selected file,
% otherwise process as usual:
%    \begin{macrocode}
\ifchilddocmanual
\section*{part `\childdocname'}
\input{\childdocname}
\else
%    \end{macrocode}

% Include the two chapters:
%    \begin{macrocode}
\include{cdocsch1}
\include{cdocsch2}
%    \end{macrocode}

% Include the two parts unless only chapters should be displayed:
%    \begin{macrocode}
\ifchilddoc\else
\section{part three}
\input{cdocspt3}
\section{part four}
\input{cdocspt4}
\fi
%    \end{macrocode}

% Process as usual until here:
%    \begin{macrocode}
\fi
%    \end{macrocode}

% End of document body:
%    \begin{macrocode}
\end{document}
%    \end{macrocode}
%\iffalse
%</samplemain>
%\fi
%
% %%%%%%%%%%%%%%%%%%%%%%%%%%%%%%%%%%%%%%
% \paragraph{Chapter Include Files.}
%
% The include files are called |cdocsch1.tex| and |cdocsch2.tex|.
%
%\iffalse
%<*samplechap1|samplechap2>
%\fi

% Optional override for |\version| flag:
%    \begin{macrocode}
%%\providecommand{\version}{final}
%    \end{macrocode}

% Include the main document:
%    \begin{macrocode}
\input{childdoc.def}
\childdocof{cdocsamp}
%    \end{macrocode}

%\iffalse
%</samplechap1|samplechap2>
%\fi
%
%\iffalse
%<*samplechap1>
%\fi
% Some text for chapter 1:
%    \begin{macrocode}
\section{one}
some text in chapter one
%    \end{macrocode}

%\iffalse
%</samplechap1>
%\fi
% Some text for chapter 2:
%\iffalse
%<*samplechap2>
%\fi
%    \begin{macrocode}
\section{two}
more text in chapter two
%    \end{macrocode}

%\iffalse
%</samplechap2>
%\fi
%
% %%%%%%%%%%%%%%%%%%%%%%%%%%%%%%%%%%%%%%
% \paragraph{Part Include Files.}
%
% The include files are called |cdocspt3.tex| and |cdocspt4.tex|.
%
%\iffalse
%<*samplepart3|samplepart4>
%\fi

% Optional override for |\version| flag:
%    \begin{macrocode}
%%\providecommand{\version}{final}
%    \end{macrocode}

% Include the main document:
%    \begin{macrocode}
\input{childdoc.def}
\childdocby{cdocsamp}
%    \end{macrocode}

%\iffalse
%</samplepart3|samplepart4>
%\fi
%
%\iffalse
%<*samplepart3>
%\fi
% Some text for part 3:
%    \begin{macrocode}
some text in part three
%    \end{macrocode}

%\iffalse
%</samplepart3>
%\fi
% Some text for part 4:
%\iffalse
%<*samplepart4>
%\fi
%    \begin{macrocode}
more text in part four
%    \end{macrocode}

%\iffalse
%</samplepart4>
%\fi
%
% %%%%%%%%%%%%%%%%%%%%%%%%%%%%%%%%%%%%%%
% \paragraph{Forwarding for a Complete Draft.}
%
% The following forwarding file |cdocsdrf.tex|
% compiles the main document in draft mode:
%\iffalse
%<*sampledraft>
%\fi
%    \begin{macrocode}
\def\version{draft}
\input{childdoc.def}
\childdocforward{cdocsamp}
%    \end{macrocode}

%\iffalse
%</sampledraft>
%\fi
%
% %%%%%%%%%%%%%%%%%%%%%%%%%%%%%%%%%%%%%%
% \paragraph{Forwarding for Final Version of the Chapters.}
%
% The following forwarding files |cdocsfn1.tex| and |cdocsfn2.tex|
% (with identical content)
% compile the final versions of the child documents
% |cdocsch1.tex| and |cdocsch2.tex|, respectively:
%\iffalse
%<*samplefinal>
%\fi
%    \begin{macrocode}
\def\version{final}
\input{childdoc.def}
\childdocforwardprefix[cdocsamp]{cdocsfn}{cdocsch}
%    \end{macrocode}

%\iffalse
%</samplefinal>
%\fi
%
% %%%%%%%%%%%%%%%%%%%%%%%%%%%%%%%%%%%%%%
% \paragraph{Command Line Processing.}
%
% The following three command lines generate the output files
% |cdocscld|, |cdocscl1| and |cdocscl2|
% which should be identical to
% |cdocsdrf|, |cdocsch1| and |cdocsfn2|, respectively:
% \begin{center}
% \begin{tabular}{l}
% |latex -jobname cdocscld \|\\
% |  "\def\version{draft}\input{childdoc.def}\childdocforward{cdocsamp}"|\\
% |latex -jobname cdocscl1 \|\\
% |  "\input{childdoc.def}\childdocforward[cdocsamp]{cdocsch1}"|\\
% |latex -jobname cdocscl2 \|\\
% |  "\def\version{final}\input{childdoc.def}\childdocforward{cdocsch2}"|
% \end{tabular}
% \end{center}
% Note that the trailing backslash on each first line
% merely continues the input to the second line
% (for convenient cut ant paste).
% Furthermore, the command |latex| can be replaced by any
% of its alternative versions such as |pdflatex|.
%
% %%%%%%%%%%%%%%%%%%%%%%%%%%%%%%%%%%%%%%%%%%%%%%%%%%%%%%%%%%%%%%%%%%%%%%%%%%%%%%
% %%%%%%%%%%%%%%%%%%%%%%%%%%%%%%%%%%%%%%%%%%%%%%%%%%%%%%%%%%%%%%%%%%%%%%%%%%%%%%
% \section{Implementation}
%\iffalse
%<*package>
%\fi
%
% This section describes the definitions file |childdoc.def|.

% The definitions cannot be loaded using |\usepackage| or |\RequirePackage|
% which has a mechanism to prevent loading a style file more than once.
% When loading the definitions by means of |\input|
% multiple instances have to be prevented manually:
%\iffalse
%This code needs to be before the `\ProvidesFile' directive
%which is defined at the beginning of this file.
%Therefore it is also placed there and commented out here.
%</package>
%<*discard>
%\fi
%    \begin{macrocode}
\ifdefined\childdocmain\endinput\fi
%    \end{macrocode}
%\iffalse
%</discard>
%<*package>
%\fi
%
% \macro{\ifchilddoc}
% \macro{\ifchilddocmanual}
% The conditional |\ifchilddoc| tells whether a
% child (true) or main (false) document is being compiled.
% The conditional |\ifchilddocmanual| tells whether
% the |\includeonly| mechanism is used (false) or
% the selection of child files must be performed manually (true).
% The definitions initialise to false:
%    \begin{macrocode}
\newif\ifchilddoc
\newif\ifchilddocmanual
%    \end{macrocode}

% \macro{\childdocname}
% \macro{\childdocjob}
% The macro |\childdocname| stores the name of the main document
% to be compiled. The macro |\childdocjob| stores the name of
% the document on which the \LaTeX{} compiler was originally invoked.
% The content of |\jobname| cannot be compared
% to filenames specified in the source due to different catcodes.
% The following code rescans |\jobname|, stores the result
% in |\childdocname| and saves a copy in |\childdocjob|:
%    \begin{macrocode}
\edef\childdocname{\scantokens\expandafter{\jobname\noexpand}}
\let\childdocjob\childdocname
%    \end{macrocode}

% \macro{\childdocdisable}
% The macro |\childdocdisable| prevents the main file
% from being processed more than once.
% At this stage, the main document command |\childdocmain|
% is assumed to be called once again where it should do nothing.
% Any subsequent call to it should prevent
% a secondary processing of the main document
% It overwrites the forwarding commands
% |\childdocof| and |\childdocforward|
% with empty macros to prevent further inclusions of the main document:
%    \begin{macrocode}
\newcommand{\childdocdisable}
{
  \renewcommand{\childdocmain}[1]{\renewcommand{\childdocmain}[1]{\endinput}}
  \renewcommand{\childdocof}[1]{}
  \renewcommand{\childdocby}[2][]{}
  \renewcommand{\childdocforward}[2][]{}
  \renewcommand{\childdocdisable}{}
}
%    \end{macrocode}

% \macro{\childdocmain}
% The macro |\childdocmain| is to be called at the top of the main file
% with nothing or the main filename (without extension) as argument.
% First, it breaks loops.
% If the argument is not empty and does not match |\childdocname|
% (which is set by the first inclusion of |childdoc.def|),
% |\ifchilddoc| is set to true, |\includeonly| is applied to the child file
% and |\jobname| is set to the main file
% (for proper handling of |.aux| files):
%    \begin{macrocode}
\newcommand{\childdocmain}[1]
{
  \childdocdisable\childdocmain{}
  \if?#1?\else
    \begingroup
      \def\childdoctmp{#1}
      \ifx\childdoctmp\childdocname
        \def\childdoctmp{}
      \else
        \def\childdoctmp
        {
          \childdoctrue
          \includeonly{\childdocname}
          \def\childdocjob{#1}
          \def\jobname{#1}
        }
      \fi
      \expandafter
    \endgroup
    \childdoctmp
  \fi
}
%    \end{macrocode}

% \macro{\childdocof}
% The command |\childdocof| redirects
% compilation to the main file |#1|.
%    \begin{macrocode}
\newcommand{\childdocof}[1]
{
  \childdocdisable
  \childdoctrue
  \includeonly{\childdocname}
  \def\jobname{#1}
  \def\childdocjob{#1}
  \input{#1}
}
%    \end{macrocode}

% \macro{\childdocby}
% The command |\childdocby| ....
%    \begin{macrocode}
\newcommand{\childdocby}[2][]
{
  \childdocdisable
  \childdoctrue
  \childdocmanualtrue
  \if?#1?\else
    \def\jobname{#2}
  \fi
  \def\childdocjob{#2}
  \input{#2}
  \endinput
}
%    \end{macrocode}

% \macro{\childdocforward}
% The command |\childdocforward| redirects
% compilation to the main file or
% (if the optional argument is given) a child file.
% Parameters are set as if the main file
% or a child file starting with |\childdocof| was compiled.
% Then compilation is handed over to the main file:
%    \begin{macrocode}
\newcommand{\childdocforward}[2][]
{
  \begingroup
    \if?#1?
      \def\childdoctmp
      {
        \def\childdocname{#2}
        \def\childdocjob{#2}
        \def\jobname{#2}
        \input{#2}
        \endinput
      }
    \else
      \def\childdoctmp
      {
        \childdocdisable
        \def\childdocname{#2}
        \childdoctrue
        \includeonly{#2}
        \def\childdocjob{#1}
        \def\jobname{#1}
        \input{#1}
        \endinput
      }
    \fi
    \expandafter
  \endgroup
  \childdoctmp
}
%    \end{macrocode}

% \macro{\childdocforwardprefix}
% The command |\childdocforwardprefix| redirects
% compilation to the main or a child file by means of a pattern.
% The prefix |#1| in the current filename is replaced by |#2|
% and the suffix of the current filename is kept
% (it is assumed that the filename does not contain the substring `|~~~|'
% which is used as a delimiter).
% Compilation is handed over to the new file by |\childdocforward|:
%    \begin{macrocode}
\newcommand{\childdocforwardprefix}[3][]
{
  \begingroup
    \def\childdocextract #2##1~~~{\def\childdoctmp{\childdocforward[#1]{#3##1}}}
    \expandafter\childdocextract\childdocname~~~
    \expandafter
  \endgroup
  \childdoctmp
}
%    \end{macrocode}

% \macro{\childdoc}
% The deprecated macro |\childdoc| is a legacy version of |\childdocmain|:
%    \begin{macrocode}
\newcommand{\childdoc}{\childdocmain}
%    \end{macrocode}

% \macro{\childdocredirect}
% The deprecated macro |\childdocredirect| is a legacy version
% of |\childdocforward| and |\childdocforwardprefix|:
%    \begin{macrocode}
\newcommand{\childdocredirect}[2][]
{
  \begingroup
    \if?#1?
      \def\childdoctmp{\childdocforward{#2}}
    \else
      \def\childdoctmp{\childdocforwardprefix{#1}{#2}}
    \fi
    \expandafter
  \endgroup
  \childdoctmp
}
%    \end{macrocode}

%\iffalse
%</package>
%\fi
%
\endinput
|\\
|\childdocmain{|\textit{main}|}|\\
\end{tabular}
\end{center}
%
If |\jobname| does not match the argument \textit{main} of |\childdocmain|,
it is assumed that |\jobname| points to the child file to be compiled.
When using |\childdocmain| with the main file specified as argument,
it suffices to start a child file
with just |\input{|\textit{main}|}|
without loading of the package and using |\childdocof|.
If instead all processing is done
with the appropriate \textsf{childdoc} directives,
the argument of \textit{main} of |\childdocmain| can be empty.

An alternative version of the command line processing described
in \secref{sec:commandline} using the detection mechanism reads:
%
\begin{center}
|... -jobname "|\textit{target}|" "|[\textit{flags}]%
[|\def\jobname{|\textit{dest}|}|]|\input{|\textit{main}|}"|
\end{center}

%%%%%%%%%%%%%%%%%%%%%%%%%%%%%%%%%%%%%%%%%%%%%%%%%%%%%%%%%%%%%%%%%%%%%%%%%%%%%%%%
\subsection{Manual Code}
\label{sec:manual}

In case one cannot be certain whether the definitions file |childdoc.def|
is installed on the target \TeX{} distribution
and one prefers not to ship it,
it is conceivable to paste a few relevant commands into the sources.

To that end, drop all statements |% \iffalse
%
% childdoc.dtx Copyright (C) 2017-2018 Niklas Beisert
%
% This work may be distributed and/or modified under the
% conditions of the LaTeX Project Public License, either version 1.3
% of this license or (at your option) any later version.
% The latest version of this license is in
%   http://www.latex-project.org/lppl.txt
% and version 1.3 or later is part of all distributions of LaTeX
% version 2005/12/01 or later.
%
% This work has the LPPL maintenance status `maintained'.
%
% The Current Maintainer of this work is Niklas Beisert.
%
% This work consists of the files childdoc.dtx and childdoc.ins
% and the derived files childdoc.def and cdocsamp.tex with
% cdocsch1.tex, cdocsch2.tex, cdocsdrf.tex, cdocsfn1.tex, cdocsfn2.tex.
%
%<package>\ifdefined\childdocmain\endinput\fi
%<package>\ProvidesFile{childdoc.def}[2018/12/30 v2.0 child document driver]
%<samplemain>\ProvidesFile{cdocsamp.tex}[2018/12/30 v2.0 sample for childdoc]
%<*driver>
%\ProvidesFile{childdoc.drv}[2018/12/30 v2.0 childdoc reference manual file]
\PassOptionsToClass{10pt,a4paper}{article}
\documentclass{ltxdoc}

\usepackage[margin=35mm]{geometry}
\usepackage{hyperref}
\usepackage{hyperxmp}
\usepackage[usenames]{color}

\hypersetup{colorlinks=true}
\hypersetup{pdfstartview=FitH}
\hypersetup{pdfpagemode=UseNone}
\hypersetup{pdfsource={}}
\hypersetup{pdflang={en-UK}}
\hypersetup{pdfcopyright={Copyright 2017-2018 Niklas Beisert.
  This work may be distributed and/or modified under the
  conditions of the LaTeX Project Public License, either version 1.3
  of this license or (at your option) any later version.}}
\hypersetup{pdflicenseurl={http://www.latex-project.org/lppl.txt}}
\hypersetup{pdfcontactaddress={ETH Zurich, ITP, HIT K,
  Wolfgang-Pauli-Strasse 27}}
\hypersetup{pdfcontactpostcode={8093}}
\hypersetup{pdfcontactcity={Zurich}}
\hypersetup{pdfcontactcountry={Switzerland}}
\hypersetup{pdfcontactemail={nbeisert@itp.phys.ethz.ch}}
\hypersetup{pdfcontacturl={http://people.phys.ethz.ch/\xmptilde nbeisert/}}

\newcommand{\secref}[1]{\hyperref[#1]{section \ref*{#1}}}

\parskip1ex
\parindent0pt
\let\olditemize\itemize
\def\itemize{\olditemize\parskip0pt}

\begin{document}

\title{The \textsf{childdoc} Package}
\hypersetup{pdftitle={The childdoc Package}}
\author{Niklas Beisert\\[2ex]
  Institut f\"ur Theoretische Physik\\
  Eidgen\"ossische Technische Hochschule Z\"urich\\
  Wolfgang-Pauli-Strasse 27, 8093 Z\"urich, Switzerland\\[1ex]
  \href{mailto:nbeisert@itp.phys.ethz.ch}
  {\texttt{nbeisert@itp.phys.ethz.ch}}}
\hypersetup{pdfauthor={Niklas Beisert}}
\hypersetup{pdfsubject={Manual for the LaTeX2e Package childdoc}}
\date{30 December 2018, \textsf{v2.0}}
\maketitle

\begin{abstract}\noindent
\textsf{childdoc} is a \LaTeXe{} package
that enables the direct compilation
of document sections included by |\include|
to individual files.
\end{abstract}

\begingroup
\parskip0ex
\tableofcontents
\endgroup

%%%%%%%%%%%%%%%%%%%%%%%%%%%%%%%%%%%%%%%%%%%%%%%%%%%%%%%%%%%%%%%%%%%%%%%%%%%%%%%%
%%%%%%%%%%%%%%%%%%%%%%%%%%%%%%%%%%%%%%%%%%%%%%%%%%%%%%%%%%%%%%%%%%%%%%%%%%%%%%%%
\section{Introduction}

\LaTeX{} provides a mechanism to structure a large document (such as a book)
into a main file and several child files (containing the chapters)
using the |\include| command.
This mechanism is beneficial for documents
which span hundreds of pages in order to
make the source file(s) more manageable.
Moreover, compilation can be restricted to
selected child files by means of the |\includeonly| command.
The latter feature can be used to reduce the compilation time while editing
(this was significantly more useful in the earlier days of \LaTeX{})
or to generate a smaller document which is easier to navigate.
Another application of |\includeonly| is to generate
documents consisting of selected parts of the complete document.

However, there are a few drawbacks of the plain |\include| mechanism:
\begin{itemize}
\item
The child files cannot be compiled on their own,
they can only be compiled via the main file.
A naive editing environment
(such as a text editor with an option
to have the current file processed by \LaTeX)
may require one to switch to the main file before compiling;
attempting to compile the child file produces errors.
\item
The main file must be modified (each time)
to adjust the |\includeonly| command
to the present needs. This easily leaves the main file in a messy state.
\item
The generated document will always carry the filename
of the main document. This is inconvenient if
several child files are to be compiled and
to be kept for distribution.
\end{itemize}

The present package provides a simple interface
to make child files individually compilable by \LaTeX{}.
Compiling a child file then has the same effect as compiling
the main file with an |\includeonly| command
to select the appropriate child.
Moreover the generated document will carry the name of the child
rather than the main file.
This resolves all three above issues.

This feature is meant to make the editing of books,
thesis documents and lecture notes somewhat more convenient.
However, the package can also be used efficiently for
composing a series of documents (such as exercise sheets)
which are typically distributed individually.
It then assists the author in generating the individual documents
(potentially in different versions)
as well as a document containing the collected series.
Another application is in developing style files
or other kinds of included material
where compilation of the style file could redirect
to a sample or test file.

%%%%%%%%%%%%%%%%%%%%%%%%%%%%%%%%%%%%%%%%%%%%%%%%%%%%%%%%%%%%%%%%%%%%%%%%%%%%%%%%
%%%%%%%%%%%%%%%%%%%%%%%%%%%%%%%%%%%%%%%%%%%%%%%%%%%%%%%%%%%%%%%%%%%%%%%%%%%%%%%%
\section{Usage}

First of all, the package \textsf{childdoc} is \emph{not} a standard
\LaTeXe{} |.sty| style file! Therefore it needs to be invoked in
a non-standard way.

%%%%%%%%%%%%%%%%%%%%%%%%%%%%%%%%%%%%%%%%%%%%%%%%%%%%%%%%%%%%%%%%%%%%%%%%%%%%%%%%
\subsection{Included Files}
\label{sec:include}

%%%%%%%%%%%%%%%%%%%%%%%%%%%%%%%%%%%%%%%%
\DescribeMacro{\childdocmain}
To use the package, add the commands
\begin{center}
\begin{tabular}{l}
|\input{childdoc.def}|\\
|\childdocmain{}|\\
\end{tabular}
\end{center}
at the very top of the main \LaTeX{} file,
in particular \emph{before} the |\documentclass| statement!
The argument of |\childdocmain| should be left empty
(but it must be present).

%%%%%%%%%%%%%%%%%%%%%%%%%%%%%%%%%%%%%%%%
\DescribeMacro{\childdocof}
Furthermore, add the commands
\begin{center}
\begin{tabular}{l}
|\input{childdoc.def}|\\
|\childdocof{|\textit{main}|}|\\
\end{tabular}
\end{center}
at the top of every child file \textit{child}
which is included by |\include{|\textit{child}|}|
from within the main file
(or at least for those files to be compiled individually).
The argument \textit{main} must be the filename of the main file.

There are a couple of
considerations in setting up the main and child documents:

%%%%%%%%%%%%%%%%%%%%%%%%%%%%%%%%%%%%%%%%
\paragraph{Restrictions.}

Please note the following restrictions:
\begin{itemize}
\item
|\childdocmain| must be called with one argument \textit{main}
to ensure compatibility with earlier version of the package.
It must either be empty (|\childdocmain{}|)
or precisely match the filename of the main file in which it is specified.
See \secref{sec:detection} for further information.
\item
The filename \textit{main} must be specified without the |.tex| extension.
\item
The filename \textit{main} is case sensitive
(even in case-insensitive file systems)
due to internal string comparison.
\item
The argument \textit{main} should be fully expanded, it cannot be a macro.
\item
Subdirectories and special characters should be avoided in filenames.
\item
The command |\childdocmain{|\textit{main}|}| must be followed by a whitespace.
It should not be followed immediately by another command
or by a comment mark `|%|'.
This is because the \TeX{} parser reads the token immediately following
the argument of |\childdocmain| and puts it
at the beginning of every child section;
however, a white\-space is ignored.
\end{itemize}

%%%%%%%%%%%%%%%%%%%%%%%%%%%%%%%%%%%%%%%%
\paragraph{Content of Main File.}

It is advisable to place all content in the child files included by |\include|.
Any output contained in the main file will appear in all child documents
unless suppressed manually;
it cannot be suppressed automatically by the |\includeonly| directive
and thus should normally be avoided.
A method to include some content in the main file
by means of conditional processing is described in \secref{sec:conditional}.

%%%%%%%%%%%%%%%%%%%%%%%%%%%%%%%%%%%%%%%%
\paragraph{Page Numbering.}

When only a part of the document is compiled,
the appropriate numbering of pages
(as well as other status parameters)
is determined from the |.aux| files.
The latter contain information from previous passes.
However this information needs to propagate through
all intermediate child documents.
Therefore the page numbering in child documents may well
be inconsistent until the complete document is compiled at least once.

A useful (if unconventional) way to always ensure a consistent
page numbering is to restart the numbering in each child document
and denote the pages by `\textit{child}|.|\textit{page}'
where \textit{child} represents the chapter/section number of the child file.
This can be achieved by the command
|\numberwithin{page}{|\textit{child}|}|
of the \textsf{amsmath} package
where \textit{child} can be |chapter| or |section|
depending on the chosen structuring.
Alternatively, one can modify the macro |\thepage| appropriately
and reset the counter |page| at the start of each child file.

%%%%%%%%%%%%%%%%%%%%%%%%%%%%%%%%%%%%%%%%%%%%%%%%%%%%%%%%%%%%%%%%%%%%%%%%%%%%%%%%
\subsection{Conditional Processing}
\label{sec:conditional}

The package provides a mechanism to compile different versions
of a document. To customise the versions further some conditional processing
can come in handy to distinguish which version is being compiled.
The package provides two macros to describe the compilation context:

%%%%%%%%%%%%%%%%%%%%%%%%%%%%%%%%%%%%%%%%
\DescribeMacro{\ifchilddoc}
The conditional |\ifchilddoc| distinguishes between the compilation of
child documents and the main document:
%
\begin{center}
|\ifchilddoc |\textit{child-code}| |[|\||else |\textit{main-code}]| \||fi|
\end{center}

%%%%%%%%%%%%%%%%%%%%%%%%%%%%%%%%%%%%%%%%
\DescribeMacro{\childdocname}
\DescribeMacro{\childdocjob}
The macro |\childdocname| contains the filename (without extension)
of the main or child file being processed.
Note that |\childdocjob| will always contain the name of the main file.

%%%%%%%%%%%%%%%%%%%%%%%%%%%%%%%%%%%%%%%%
\paragraph{Title Page.}

Conditional processing can be used to include a title or banner page
in the main document when proper precautions are taken.
Importantly, the code in the main file should ensure that the page counter
(as well as other status parameters which are stored in the |.aux| files)
takes the same value after the conditional processing.
Otherwise the page numbers may take divergent values
depending on which part is compiled.

For example, a title page could be declared by:
%
\begin{center}
\begin{tabular}{l}
|\ifchilddoc\||else|\\
|\addtocounter{page}{-1}|\\
\textit{code for title page}\\
|\newpage|\\
|\||fi|
\end{tabular}
\end{center}
%
A banner page for the child documents can be generated by:
%
\begin{center}
\begin{tabular}{l}
|\ifchilddoc|\\
|\addtocounter{page}{-1}|\\
\textit{code for banner page}\\
|\newpage|\\
|\||fi|
\end{tabular}
\end{center}
%
Here one could write a message such as:
\begin{center}
|This is the part \childdocname{} of \childdocjob{}.|
\end{center}

%%%%%%%%%%%%%%%%%%%%%%%%%%%%%%%%%%%%%%%%%%%%%%%%%%%%%%%%%%%%%%%%%%%%%%%%%%%%%%%%
\subsection{Flags}
\label{sec:flags}

The package makes it easy to generate different versions
of the main or child documents.
To this end compilation flags can be defined
and assigned different default values.
They will be particularly useful in conjunction
with the forwarding mechanism described in \secref{sec:forward}.

For example, it may be useful to have a flag |\version|
which can be set to |draft| or |final|.
The document source will contain some conditional code
depending on the value of |\version|.
Suppose further, the flag should default to |final| for the main file
and to |draft| for child files
which is a natural assignment for editing the document.
This is achieved by placing the following code
in the preamble of the main document
(below the |\childdocmain| directive):
%
\begin{center}
\begin{tabular}{l}
|\ifchilddoc|\\
|\providecommand{\version}{draft}|\\
|\||else|\\
|\providecommand{\version}{final}|\\
|\||fi|
\end{tabular}
\end{center}
%
The definition by |\providecommand| makes sure
that previous definitions are not overwritten.
Further statements |\providecommand{\version}{...}|
can thus be added before the above code to override it.

For the main file, one might add a line
(between |\childdocmain| and the above block)
%
\begin{center}
|%\ifchilddoc\||else\providecommand{\version}{draft}\||fi|
\end{center}
%
which can be uncommented to produce a draft version.
Likewise one can add a line to the very top of a child file
(above the |\childdocof{|\textit{main}|}| directive)
%
\begin{center}
|%\providecommand{\version}{final}|
\end{center}
%
which can be uncommented to produce the final version of this child document.

%%%%%%%%%%%%%%%%%%%%%%%%%%%%%%%%%%%%%%%%%%%%%%%%%%%%%%%%%%%%%%%%%%%%%%%%%%%%%%%%
\subsection{Forwarding}
\label{sec:forward}

Different versions of the main or child documents
using compilation flags as described in \secref{sec:flags}
can be (permanently) stored in different files
for convenient compilation, viewing and distribution.
To this end, the package defines a command
to pass on compilation to a different file:

%%%%%%%%%%%%%%%%%%%%%%%%%%%%%%%%%%%%%%%%
\DescribeMacro{\childdocforward}
The command |\childdocforward| redirects processing to
another source file:
%
\begin{center}
\begin{tabular}{l}
|\input{childdoc.def}|\\
|\childdocforward[|\textit{main}|]{|\textit{dest}|}|\\
\end{tabular}
\end{center}
%
The argument \textit{dest} is the destination file
(without extension).
It should be the main file or one of the child files.
Note that further \textsf{childdoc} directives
such as |\childdocof| and |\childdocforward|
in the indicated file will be processed in this form.
The optional argument \textit{main}
passes on directly to the main file \textit{main}
while pretending to compile the child \textit{dest}.
This form behaves as if \textit{dest}
issues |\childdocof{|\textit{main}|}| right away,
and no further \textsf{childdoc} directives will be processed.

%%%%%%%%%%%%%%%%%%%%%%%%%%%%%%%%%%%%%%%%
\DescribeMacro{\...prefix}
In the alternative form |\childdocforwardprefix|,
%
\begin{center}
\begin{tabular}{l}
|\input{childdoc.def}|\\
|\childdocforwardprefix[|\textit{main}|]{|\textit{prefix}|}{|\textit{dest}|}|
\end{tabular}
\end{center}
%
the destination file is determined by a pattern
depending on the current file:
To make this work, the current file must be called
`{\textit{prefix}\hspace{0.2em}\textit{suffix}}'
with \textit{prefix} matching precisely the argument.
Processing is then passed on to the file
`{\textit{dest}\hspace{0.2em}\textit{suffix}}'.
Surely, the same effect is achieved by
directly specifying the
argument `{\textit{dest}\hspace{0.2em}\textit{suffix}}'
in the first form.
However, that requires to set up a different file
for each child. With the alternative form of the command
all these files can have exactly the same content
which simplifies setting them up and maintaining them.

For example, the following file |draft.tex|
with a compilation flag |\version| as described in \secref{sec:flags}
compiles the main document as a draft:
%
\begin{center}
\begin{tabular}{l}
|\def\version{draft}|\\
|\input{childdoc.def}|\\
|\childdocforward{|\textit{main}|}|
\end{tabular}
\end{center}
%
Likewise, the following files |final|\textit{nn}|.tex|
compile the final version of the child document
|child|\textit{nn}|.tex|:
%
\begin{center}
\begin{tabular}{l}
|\def\version{final}|\\
|\input{childdoc.def}|\\
|\childdocforwardprefix{final}{child}|
\end{tabular}
\end{center}
%

Note that when several versions of a main file and/or of each child file
are to be generated, it may be convenient to set up a |Makefile| or
shell script to automatise the process.

%%%%%%%%%%%%%%%%%%%%%%%%%%%%%%%%%%%%%%%%%%%%%%%%%%%%%%%%%%%%%%%%%%%%%%%%%%%%%%%%
\subsection{Command Line Processing}
\label{sec:commandline}

The effect of redirection files can also be achieved by invoking
the \LaTeX{} compiler with a more elaborate command line.
Most conveniently this should be done as part
of a shell script or a |Makefile|.

When using \textsf{childdoc} in the main file, the following
command lines effectively perform a redirection
(note that depending on the shell being used,
backslashes may have to be doubled: `|\|' $\to$ `|\\|'):
%
\begin{center}
|... -jobname "|\textit{target}|" |\\|"|[\textit{flags}]%
|\input{childdoc.def}\childdocforward[|\textit{main}|]{|\textit{dest}|}"|
\end{center}
%
Here \textit{target} is the name of the output file,
\textit{main} is the name of the main file
and \textit{dest} is the name of the main or child file to be processed
(all filenames without extensions).
The optional argument \textit{main} can be omitted
if \textit{main} matches \textit{dest}.
Optionally, compilation \textit{flags} can be defined via |\def| commands.
This command line makes the \TeX{} engine believe
it is compiling the file \textit{target}
whose content is specified as the latter parameter.
The provided code then forwards the processing to
\textit{main} or \textit{dest} as described in \secref{sec:forward}.

%%%%%%%%%%%%%%%%%%%%%%%%%%%%%%%%%%%%%%%%%%%%%%%%%%%%%%%%%%%%%%%%%%%%%%%%%%%%%%%%
\subsection{Include by Input}
\label{sec:input}

Including child documents by |\include| has some restrictions by design.
Most notably, the content of a child document always occupies
its own set of pages; pages cannot be shared between child documents.
Usually, this behaviour makes perfect sense
because each child document contain an essential part of the document.
However, in some situations it may be desirable to compose
a document from a collection of parts
without having mandatory page breaks between then.
For this case, the package
provides a mechanism to include parts
by |\input| which can also be processed individually.
However, by construction this mechanism
requires manual handling of the content to be output.

%%%%%%%%%%%%%%%%%%%%%%%%%%%%%%%%%%%%%%%%
\DescribeMacro{\ifchilddocmanual}
The main file should be prepared as usual, see \secref{sec:include}.
However, the document body must make a distinction
between processing of an individual part and of the main document, e.g.:
%
\begin{center}
\begin{tabular}{l}
|\ifchilddocmanual|\\
|\input{\childdocname}|\\
|\||else|\\
\textit{document body with }|\input{|\textit{part}|}|\\
|\||fi|
\end{tabular}
\end{center}
%
The conditional |\ifchilddocmanual| is true whenever
a part to be included by |\input| is being compiled,
and the name of the part is stored in |\childdocname|.

%%%%%%%%%%%%%%%%%%%%%%%%%%%%%%%%%%%%%%%%
\DescribeMacro{\childdocby}
Each part to be included by |\input| should start with:
%
\begin{center}
\begin{tabular}{l}
|\input{childdoc.def}|\\
|\childdocby{|\textit{main}|}|\\
\end{tabular}
\end{center}
%
The directive |\childdocby| is similar to |\childdocof|
described in \secref{sec:include},
but the subsequent selection of content must be done manually.
To that end, both |\ifchilddoc| and |\ifchilddocmanual|
will be true upon processing of a part,
and the name of the part is stored in |\childdocname|.
Note that |\jobname| will be set to the filename of the current part
so that each part receives an individual |.aux| file
that does not interfere with the |.aux| file(s) of the main document.
This behaviour can be altered by the alternative form
|\childdocby[*]{|\textit{main}|}| (with a non-empty optional argument)
which uses the |.aux| file of the main document
by setting |\jobname| to \textit{main}.

%%%%%%%%%%%%%%%%%%%%%%%%%%%%%%%%%%%%%%%%%%%%%%%%%%%%%%%%%%%%%%%%%%%%%%%%%%%%%%%%
\subsection{Driver Development}
\label{sec:driver}

The \textsf{childdoc} mechanism can also be use for the development
of definition files such as \LaTeX{} styles or classes.
This case differs from the above setup with multiple parts
included by |\include| in that no |\includeonly| should be invoked.
This can be achieved by starting the include file
(before |\ProvidesPackage|) with:
%
\begin{center}
\begin{tabular}{l}
|\input{childdoc.def}|\\
|\childdocforward{|\textit{main}|}|\\
\end{tabular}
\end{center}
%
or alternatively with:
%
\begin{center}
\begin{tabular}{l}
|\input{childdoc.def}|\\
|\childdocby{|\textit{main}|}|\\
\end{tabular}
\end{center}
%
Both forms have slightly different effects as described above.
The main file is prepared as usual, see \secref{sec:include}.

%%%%%%%%%%%%%%%%%%%%%%%%%%%%%%%%%%%%%%%%%%%%%%%%%%%%%%%%%%%%%%%%%%%%%%%%%%%%%%%%
\subsection{Legacy Detection}
\label{sec:detection}

The directive |\childdocmain| in the main file can detect
whether the complete document or merely a child is to be compiled
even without using the directive |\childdocof|.
This method is deprecated because it is less robust
and there is no compelling reason to use it;
it is merely provided for backward compatibility
and it may be removed in future versions.

If the detection mechanism is to be used,
it is mandatory to correctly specify
the filename of the main file as the argument of |\childdocmain|:
%
\begin{center}
\begin{tabular}{l}
|\input{childdoc.def}|\\
|\childdocmain{|\textit{main}|}|\\
\end{tabular}
\end{center}
%
If |\jobname| does not match the argument \textit{main} of |\childdocmain|,
it is assumed that |\jobname| points to the child file to be compiled.
When using |\childdocmain| with the main file specified as argument,
it suffices to start a child file
with just |\input{|\textit{main}|}|
without loading of the package and using |\childdocof|.
If instead all processing is done
with the appropriate \textsf{childdoc} directives,
the argument of \textit{main} of |\childdocmain| can be empty.

An alternative version of the command line processing described
in \secref{sec:commandline} using the detection mechanism reads:
%
\begin{center}
|... -jobname "|\textit{target}|" "|[\textit{flags}]%
[|\def\jobname{|\textit{dest}|}|]|\input{|\textit{main}|}"|
\end{center}

%%%%%%%%%%%%%%%%%%%%%%%%%%%%%%%%%%%%%%%%%%%%%%%%%%%%%%%%%%%%%%%%%%%%%%%%%%%%%%%%
\subsection{Manual Code}
\label{sec:manual}

In case one cannot be certain whether the definitions file |childdoc.def|
is installed on the target \TeX{} distribution
and one prefers not to ship it,
it is conceivable to paste a few relevant commands into the sources.

To that end, drop all statements |\input{childdoc.def}|
and perform the replacements as outlined below.
Instead of |\childdocmain{|\textit{main}|}| add the following code
to the top of the main file:
%
\begin{center}
\begin{tabular}{l}
|\||ifdefined\childdocname\endinput\||fi\newif\ifchilddoc|\\
|\edef\childdocname{\scantokens\expandafter{\jobname\noexpand}}|\\
|\def\childdocmain{|\textit{main}|}\||ifx\childdocmain\childdocname\||else|\\
|\childdoctrue\includeonly{\childdocname}\let\jobname\childdocmain\||fi|\\
\end{tabular}
\end{center}
%
Instead of |\childdocof{|\textit{main}|}| just include the main file
at the top of each child file:
%
\begin{center}
|\input{|\textit{main}|}|
\end{center}
%
A simple redirection |\childdocforward{|\textit{dest}|}| is achieved by:
%
\begin{center}
|\def\jobname{|\textit{dest}|}\input{\jobname}|
\end{center}
%
The redirection with prefix
|\childdocforwardprefix[|\textit{prefix}|]{|\textit{dest}|}|
is accomplished by:
%
\begin{center}
\begin{tabular}{l}
|{\edef\jobname{\scantokens\expandafter{\jobname\noexpand}}|\\
|\def\redirectjob |\textit{prefix}|#1~~~{\gdef\jobname{|\textit{dest}|#1}}|\\
|\expandafter\redirectjob\jobname~~~}\input{\jobname}|
\end{tabular}
\end{center}

In an alternative approach,
child documents can be compiled by a specific command line
without additional code or specific definitions:
%
\begin{center}
|... -jobname "|\textit{target}|" "|[\textit{flags}]%
|\includeonly{|\textit{dest}|}\input{|\textit{main}|}"|
\end{center}
%

%%%%%%%%%%%%%%%%%%%%%%%%%%%%%%%%%%%%%%%%%%%%%%%%%%%%%%%%%%%%%%%%%%%%%%%%%%%%%%%%
%%%%%%%%%%%%%%%%%%%%%%%%%%%%%%%%%%%%%%%%%%%%%%%%%%%%%%%%%%%%%%%%%%%%%%%%%%%%%%%%
\section{Information}

%%%%%%%%%%%%%%%%%%%%%%%%%%%%%%%%%%%%%%%%%%%%%%%%%%%%%%%%%%%%%%%%%%%%%%%%%%%%%%%%
\subsection{Copyright}

Copyright \copyright{} 2017--2018 Niklas Beisert

This work may be distributed and/or modified under the
conditions of the \LaTeX{} Project Public License, either version 1.3
of this license or (at your option) any later version.
The latest version of this license is in
  \url{http://www.latex-project.org/lppl.txt}
and version 1.3 or later is part of all distributions of \LaTeX{}
version 2005/12/01 or later.

This work has the LPPL maintenance status `maintained'.

The Current Maintainer of this work is Niklas Beisert.

This work consists of the files |README.txt|, |childdoc.ins| and |childdoc.dtx|
as well as the derived files |childdoc.def|, |cdocsamp.tex|
with |cdocsch1.tex|, |cdocsch2.tex|, |cdocspt3.tex|, |cdocspt4.tex|,
|cdocsdrf.tex|, |cdocsfn1.tex|, |cdocsfn2.tex|
as well as |childdoc.pdf|.

%%%%%%%%%%%%%%%%%%%%%%%%%%%%%%%%%%%%%%%%%%%%%%%%%%%%%%%%%%%%%%%%%%%%%%%%%%%%%%%%
\subsection{Files and Installation}

The package consists of the files:
%
\begin{center}
\begin{tabular}{ll}
    |README.txt|   & readme file \\
    |childdoc.ins| & installation file \\
    |childdoc.dtx| & source file \\
    |childdoc.def| & definition file \\
    |cdocsamp.tex| & sample main file \\
    |cdocsch1.tex| & sample include file \\
    |cdocsch2.tex| & sample include file \\
    |cdocspt3.tex| & sample part file \\
    |cdocspt4.tex| & sample part file \\
    |cdocsdrf.tex| & sample redirection file \\
    |cdocsfn1.tex| & sample redirection file \\
    |cdocsfn2.tex| & sample redirection file \\
    |childdoc.pdf| & manual
\end{tabular}
\end{center}
%
The distribution consists of the files
|README.txt|, |childdoc.ins| and |childdoc.dtx|.
%
\begin{itemize}
\item
Run (pdf)\LaTeX{} on |childdoc.dtx|
to compile the manual |childdoc.pdf| (this file).
\item
Run \LaTeX{} on |childdoc.ins| to create the definitions file |childdoc.def|
and the sample |cdocsamp.tex| with include files
|cdocsch1.tex|, |cdocsch2.tex|, |cdocspt3.tex|, |cdocspt4.tex|,
|cdocsdrf.tex|, |cdocsfn1.tex|, |cdocsfn2.tex|.
Then copy the file |childdoc.def| to an appropriate directory of your \LaTeX{}
distribution, e.g.\ \textit{texmf-root}|/tex/latex/childdoc|.
\end{itemize}

%%%%%%%%%%%%%%%%%%%%%%%%%%%%%%%%%%%%%%%%%%%%%%%%%%%%%%%%%%%%%%%%%%%%%%%%%%%%%%%%
\subsection{Related CTAN Packages}

There are several other packages which offer a similar functionality:
%
\begin{itemize}
\item
The packages
\href{http://ctan.org/pkg/docmute}{\textsf{docmute}},
\href{http://ctan.org/pkg/includex}{\textsf{includex}} and
\href{http://ctan.org/pkg/standalone}{\textsf{standalone}}
provide commands to include only the document body of
a child file thus allowing both files to be compiled individually.
\item
The packages \href{http://ctan.org/pkg/subdocs}{\textsf{subdocs}}
and \href{http://ctan.org/pkg/subfiles}{\textsf{subfiles}}
provide structures in which the main and child documents can be
encapsulated and allowing them to be compiled individually.
The inclusion mechanism is different from the conventional |\include|.
\item
The package \href{http://ctan.org/pkg/combine}{\textsf{combine}}
is an elaborate solution to combine several documents into one.
\end{itemize}
%
See also the CTAN topic \href{http://ctan.org/topic/subdocs}{\textsf{subdocs}}
for further related packages.
The present package differs from the above solutions in that
a document structure constructed with the conventional |\include| mechanism
just needs two extra commands at the top of every file
such that all constituent files can be compiled individually.

%%%%%%%%%%%%%%%%%%%%%%%%%%%%%%%%%%%%%%%%%%%%%%%%%%%%%%%%%%%%%%%%%%%%%%%%%%%%%%%%
%\subsection{Feature Suggestions}
%
%The following is a list of features which may be useful for future
%versions of this package:
%%
%\begin{itemize}
%\item
%\ldots
%\end{itemize}

%%%%%%%%%%%%%%%%%%%%%%%%%%%%%%%%%%%%%%%%%%%%%%%%%%%%%%%%%%%%%%%%%%%%%%%%%%%%%%%%
\subsection{Revision History}

%%%%%%%%%%%%%%%%%%%%%%%%%%%%%%%%%%%%%%%%
\paragraph{v2.0:} 2018/12/30

\begin{itemize}
\item
immediate forward processing
\item
added |\childdocby| mechanism
\item
manual restructured
\end{itemize}

%%%%%%%%%%%%%%%%%%%%%%%%%%%%%%%%%%%%%%%%
\paragraph{v1.6:} 2018/01/17

\begin{itemize}
\item
application for development of include files
\item
corrections to manual
\end{itemize}

%%%%%%%%%%%%%%%%%%%%%%%%%%%%%%%%%%%%%%%%
\paragraph{v1.5:} 2017/05/21

\begin{itemize}
\item
more complete structuring introduced
\item
|\childdocof| introduced
\item
|\childdoc| renamed to |\childdocmain|
\item
|\childredirect| renamed to |\childdocforward| and |\childdocforwardprefix|
and functionality expanded
\end{itemize}

%%%%%%%%%%%%%%%%%%%%%%%%%%%%%%%%%%%%%%%%
\paragraph{v1.0:} 2017/04/27

\begin{itemize}
\item
manual and install package
\item
first version published on CTAN
\end{itemize}

%%%%%%%%%%%%%%%%%%%%%%%%%%%%%%%%%%%%%%%%
\paragraph{v0.6:} 2017/04/26

\begin{itemize}
\item
redirection mechanism added
\end{itemize}

%%%%%%%%%%%%%%%%%%%%%%%%%%%%%%%%%%%%%%%%
\paragraph{v0.5:} 2017/04/26

\begin{itemize}
\item
functionality in definition file
\end{itemize}


%%%%%%%%%%%%%%%%%%%%%%%%%%%%%%%%%%%%%%%%%%%%%%%%%%%%%%%%%%%%%%%%%%%%%%%%%%%%%%%%
%%%%%%%%%%%%%%%%%%%%%%%%%%%%%%%%%%%%%%%%%%%%%%%%%%%%%%%%%%%%%%%%%%%%%%%%%%%%%%%%
%%%%%%%%%%%%%%%%%%%%%%%%%%%%%%%%%%%%%%%%%%%%%%%%%%%%%%%%%%%%%%%%%%%%%%%%%%%%%%%%
\appendix

\settowidth\MacroIndent{\rmfamily\scriptsize 000\ }

 \DocInput{childdoc.dtx}

\end{document}
%</driver>
% \fi
%
% %%%%%%%%%%%%%%%%%%%%%%%%%%%%%%%%%%%%%%%%%%%%%%%%%%%%%%%%%%%%%%%%%%%%%%%%%%%%%%
% %%%%%%%%%%%%%%%%%%%%%%%%%%%%%%%%%%%%%%%%%%%%%%%%%%%%%%%%%%%%%%%%%%%%%%%%%%%%%%
% \section{Sample}
%\iffalse
%<*samplemain>
%\fi
%
% The following presents a sample document
% with two chapters, two parts, a title page,
% a compile flag as well as three forwarding files to set the flag.
% It consists of eight |.tex| files:
% \begin{center}
% \begin{tabular}{ll}
% |cdocsamp.tex|&main file\\
% |cdocsch1.tex|&include file for chapter 1\\
% |cdocsch2.tex|&include file for chapter 2\\
% |cdocspt3.tex|&include file for part 3\\
% |cdocspt4.tex|&include file for part 4\\
% |cdocsdrf.tex|&forwarding file for main file in draft mode\\
% |cdocsfi1.tex|&forwarding file for final version of chapter 1\\
% |cdocsfi2.tex|&forwarding file for final version of chapter 2\\
% \end{tabular}
% \end{center}
% Each of the eight files can be compiled directly by the \LaTeX{} compiler.
%
% %%%%%%%%%%%%%%%%%%%%%%%%%%%%%%%%%%%%%%
% \paragraph{Main File.}
%
% The main file is called |cdocsamp.tex|.
%
% Load the \textsf{childdoc} definitions and
% declare the filename for the main document:
%    \begin{macrocode}
\input{childdoc.def}
\childdocmain{}
%    \end{macrocode}

% Optional override for |\version| flag:
%    \begin{macrocode}
%%\ifchilddoc\else\providecommand{\version}{draft}\fi
%    \end{macrocode}

% Define the default values for the |\version| flag
% (|final| for the main file and |draft| for childs):
%    \begin{macrocode}
\ifchilddoc
\providecommand{\version}{draft}
\else
\providecommand{\version}{final}
\fi
%    \end{macrocode}

% Load the standard document class:
%    \begin{macrocode}
\documentclass[12pt]{article}
%    \end{macrocode}

% Start the document body:
%    \begin{macrocode}
\begin{document}
%    \end{macrocode}

% Declare a title page.
% Print title, part of document being processed and version flag:
%    \begin{macrocode}
\addtocounter{page}{-1}
\begin{center}
{\LARGE\bfseries{}childdoc example\par}
\vspace{1cm}
\ifchilddoc
\ifchilddocmanual part\else chapter\fi:
`\childdocname' of `\childdocjob'\par
\else
main document: `\childdocjob'\par
\fi
version: \version\par
\end{center}
\newpage
%    \end{macrocode}

% Manually include selected file,
% otherwise process as usual:
%    \begin{macrocode}
\ifchilddocmanual
\section*{part `\childdocname'}
\input{\childdocname}
\else
%    \end{macrocode}

% Include the two chapters:
%    \begin{macrocode}
\include{cdocsch1}
\include{cdocsch2}
%    \end{macrocode}

% Include the two parts unless only chapters should be displayed:
%    \begin{macrocode}
\ifchilddoc\else
\section{part three}
\input{cdocspt3}
\section{part four}
\input{cdocspt4}
\fi
%    \end{macrocode}

% Process as usual until here:
%    \begin{macrocode}
\fi
%    \end{macrocode}

% End of document body:
%    \begin{macrocode}
\end{document}
%    \end{macrocode}
%\iffalse
%</samplemain>
%\fi
%
% %%%%%%%%%%%%%%%%%%%%%%%%%%%%%%%%%%%%%%
% \paragraph{Chapter Include Files.}
%
% The include files are called |cdocsch1.tex| and |cdocsch2.tex|.
%
%\iffalse
%<*samplechap1|samplechap2>
%\fi

% Optional override for |\version| flag:
%    \begin{macrocode}
%%\providecommand{\version}{final}
%    \end{macrocode}

% Include the main document:
%    \begin{macrocode}
\input{childdoc.def}
\childdocof{cdocsamp}
%    \end{macrocode}

%\iffalse
%</samplechap1|samplechap2>
%\fi
%
%\iffalse
%<*samplechap1>
%\fi
% Some text for chapter 1:
%    \begin{macrocode}
\section{one}
some text in chapter one
%    \end{macrocode}

%\iffalse
%</samplechap1>
%\fi
% Some text for chapter 2:
%\iffalse
%<*samplechap2>
%\fi
%    \begin{macrocode}
\section{two}
more text in chapter two
%    \end{macrocode}

%\iffalse
%</samplechap2>
%\fi
%
% %%%%%%%%%%%%%%%%%%%%%%%%%%%%%%%%%%%%%%
% \paragraph{Part Include Files.}
%
% The include files are called |cdocspt3.tex| and |cdocspt4.tex|.
%
%\iffalse
%<*samplepart3|samplepart4>
%\fi

% Optional override for |\version| flag:
%    \begin{macrocode}
%%\providecommand{\version}{final}
%    \end{macrocode}

% Include the main document:
%    \begin{macrocode}
\input{childdoc.def}
\childdocby{cdocsamp}
%    \end{macrocode}

%\iffalse
%</samplepart3|samplepart4>
%\fi
%
%\iffalse
%<*samplepart3>
%\fi
% Some text for part 3:
%    \begin{macrocode}
some text in part three
%    \end{macrocode}

%\iffalse
%</samplepart3>
%\fi
% Some text for part 4:
%\iffalse
%<*samplepart4>
%\fi
%    \begin{macrocode}
more text in part four
%    \end{macrocode}

%\iffalse
%</samplepart4>
%\fi
%
% %%%%%%%%%%%%%%%%%%%%%%%%%%%%%%%%%%%%%%
% \paragraph{Forwarding for a Complete Draft.}
%
% The following forwarding file |cdocsdrf.tex|
% compiles the main document in draft mode:
%\iffalse
%<*sampledraft>
%\fi
%    \begin{macrocode}
\def\version{draft}
\input{childdoc.def}
\childdocforward{cdocsamp}
%    \end{macrocode}

%\iffalse
%</sampledraft>
%\fi
%
% %%%%%%%%%%%%%%%%%%%%%%%%%%%%%%%%%%%%%%
% \paragraph{Forwarding for Final Version of the Chapters.}
%
% The following forwarding files |cdocsfn1.tex| and |cdocsfn2.tex|
% (with identical content)
% compile the final versions of the child documents
% |cdocsch1.tex| and |cdocsch2.tex|, respectively:
%\iffalse
%<*samplefinal>
%\fi
%    \begin{macrocode}
\def\version{final}
\input{childdoc.def}
\childdocforwardprefix[cdocsamp]{cdocsfn}{cdocsch}
%    \end{macrocode}

%\iffalse
%</samplefinal>
%\fi
%
% %%%%%%%%%%%%%%%%%%%%%%%%%%%%%%%%%%%%%%
% \paragraph{Command Line Processing.}
%
% The following three command lines generate the output files
% |cdocscld|, |cdocscl1| and |cdocscl2|
% which should be identical to
% |cdocsdrf|, |cdocsch1| and |cdocsfn2|, respectively:
% \begin{center}
% \begin{tabular}{l}
% |latex -jobname cdocscld \|\\
% |  "\def\version{draft}\input{childdoc.def}\childdocforward{cdocsamp}"|\\
% |latex -jobname cdocscl1 \|\\
% |  "\input{childdoc.def}\childdocforward[cdocsamp]{cdocsch1}"|\\
% |latex -jobname cdocscl2 \|\\
% |  "\def\version{final}\input{childdoc.def}\childdocforward{cdocsch2}"|
% \end{tabular}
% \end{center}
% Note that the trailing backslash on each first line
% merely continues the input to the second line
% (for convenient cut ant paste).
% Furthermore, the command |latex| can be replaced by any
% of its alternative versions such as |pdflatex|.
%
% %%%%%%%%%%%%%%%%%%%%%%%%%%%%%%%%%%%%%%%%%%%%%%%%%%%%%%%%%%%%%%%%%%%%%%%%%%%%%%
% %%%%%%%%%%%%%%%%%%%%%%%%%%%%%%%%%%%%%%%%%%%%%%%%%%%%%%%%%%%%%%%%%%%%%%%%%%%%%%
% \section{Implementation}
%\iffalse
%<*package>
%\fi
%
% This section describes the definitions file |childdoc.def|.

% The definitions cannot be loaded using |\usepackage| or |\RequirePackage|
% which has a mechanism to prevent loading a style file more than once.
% When loading the definitions by means of |\input|
% multiple instances have to be prevented manually:
%\iffalse
%This code needs to be before the `\ProvidesFile' directive
%which is defined at the beginning of this file.
%Therefore it is also placed there and commented out here.
%</package>
%<*discard>
%\fi
%    \begin{macrocode}
\ifdefined\childdocmain\endinput\fi
%    \end{macrocode}
%\iffalse
%</discard>
%<*package>
%\fi
%
% \macro{\ifchilddoc}
% \macro{\ifchilddocmanual}
% The conditional |\ifchilddoc| tells whether a
% child (true) or main (false) document is being compiled.
% The conditional |\ifchilddocmanual| tells whether
% the |\includeonly| mechanism is used (false) or
% the selection of child files must be performed manually (true).
% The definitions initialise to false:
%    \begin{macrocode}
\newif\ifchilddoc
\newif\ifchilddocmanual
%    \end{macrocode}

% \macro{\childdocname}
% \macro{\childdocjob}
% The macro |\childdocname| stores the name of the main document
% to be compiled. The macro |\childdocjob| stores the name of
% the document on which the \LaTeX{} compiler was originally invoked.
% The content of |\jobname| cannot be compared
% to filenames specified in the source due to different catcodes.
% The following code rescans |\jobname|, stores the result
% in |\childdocname| and saves a copy in |\childdocjob|:
%    \begin{macrocode}
\edef\childdocname{\scantokens\expandafter{\jobname\noexpand}}
\let\childdocjob\childdocname
%    \end{macrocode}

% \macro{\childdocdisable}
% The macro |\childdocdisable| prevents the main file
% from being processed more than once.
% At this stage, the main document command |\childdocmain|
% is assumed to be called once again where it should do nothing.
% Any subsequent call to it should prevent
% a secondary processing of the main document
% It overwrites the forwarding commands
% |\childdocof| and |\childdocforward|
% with empty macros to prevent further inclusions of the main document:
%    \begin{macrocode}
\newcommand{\childdocdisable}
{
  \renewcommand{\childdocmain}[1]{\renewcommand{\childdocmain}[1]{\endinput}}
  \renewcommand{\childdocof}[1]{}
  \renewcommand{\childdocby}[2][]{}
  \renewcommand{\childdocforward}[2][]{}
  \renewcommand{\childdocdisable}{}
}
%    \end{macrocode}

% \macro{\childdocmain}
% The macro |\childdocmain| is to be called at the top of the main file
% with nothing or the main filename (without extension) as argument.
% First, it breaks loops.
% If the argument is not empty and does not match |\childdocname|
% (which is set by the first inclusion of |childdoc.def|),
% |\ifchilddoc| is set to true, |\includeonly| is applied to the child file
% and |\jobname| is set to the main file
% (for proper handling of |.aux| files):
%    \begin{macrocode}
\newcommand{\childdocmain}[1]
{
  \childdocdisable\childdocmain{}
  \if?#1?\else
    \begingroup
      \def\childdoctmp{#1}
      \ifx\childdoctmp\childdocname
        \def\childdoctmp{}
      \else
        \def\childdoctmp
        {
          \childdoctrue
          \includeonly{\childdocname}
          \def\childdocjob{#1}
          \def\jobname{#1}
        }
      \fi
      \expandafter
    \endgroup
    \childdoctmp
  \fi
}
%    \end{macrocode}

% \macro{\childdocof}
% The command |\childdocof| redirects
% compilation to the main file |#1|.
%    \begin{macrocode}
\newcommand{\childdocof}[1]
{
  \childdocdisable
  \childdoctrue
  \includeonly{\childdocname}
  \def\jobname{#1}
  \def\childdocjob{#1}
  \input{#1}
}
%    \end{macrocode}

% \macro{\childdocby}
% The command |\childdocby| ....
%    \begin{macrocode}
\newcommand{\childdocby}[2][]
{
  \childdocdisable
  \childdoctrue
  \childdocmanualtrue
  \if?#1?\else
    \def\jobname{#2}
  \fi
  \def\childdocjob{#2}
  \input{#2}
  \endinput
}
%    \end{macrocode}

% \macro{\childdocforward}
% The command |\childdocforward| redirects
% compilation to the main file or
% (if the optional argument is given) a child file.
% Parameters are set as if the main file
% or a child file starting with |\childdocof| was compiled.
% Then compilation is handed over to the main file:
%    \begin{macrocode}
\newcommand{\childdocforward}[2][]
{
  \begingroup
    \if?#1?
      \def\childdoctmp
      {
        \def\childdocname{#2}
        \def\childdocjob{#2}
        \def\jobname{#2}
        \input{#2}
        \endinput
      }
    \else
      \def\childdoctmp
      {
        \childdocdisable
        \def\childdocname{#2}
        \childdoctrue
        \includeonly{#2}
        \def\childdocjob{#1}
        \def\jobname{#1}
        \input{#1}
        \endinput
      }
    \fi
    \expandafter
  \endgroup
  \childdoctmp
}
%    \end{macrocode}

% \macro{\childdocforwardprefix}
% The command |\childdocforwardprefix| redirects
% compilation to the main or a child file by means of a pattern.
% The prefix |#1| in the current filename is replaced by |#2|
% and the suffix of the current filename is kept
% (it is assumed that the filename does not contain the substring `|~~~|'
% which is used as a delimiter).
% Compilation is handed over to the new file by |\childdocforward|:
%    \begin{macrocode}
\newcommand{\childdocforwardprefix}[3][]
{
  \begingroup
    \def\childdocextract #2##1~~~{\def\childdoctmp{\childdocforward[#1]{#3##1}}}
    \expandafter\childdocextract\childdocname~~~
    \expandafter
  \endgroup
  \childdoctmp
}
%    \end{macrocode}

% \macro{\childdoc}
% The deprecated macro |\childdoc| is a legacy version of |\childdocmain|:
%    \begin{macrocode}
\newcommand{\childdoc}{\childdocmain}
%    \end{macrocode}

% \macro{\childdocredirect}
% The deprecated macro |\childdocredirect| is a legacy version
% of |\childdocforward| and |\childdocforwardprefix|:
%    \begin{macrocode}
\newcommand{\childdocredirect}[2][]
{
  \begingroup
    \if?#1?
      \def\childdoctmp{\childdocforward{#2}}
    \else
      \def\childdoctmp{\childdocforwardprefix{#1}{#2}}
    \fi
    \expandafter
  \endgroup
  \childdoctmp
}
%    \end{macrocode}

%\iffalse
%</package>
%\fi
%
\endinput
|
and perform the replacements as outlined below.
Instead of |\childdocmain{|\textit{main}|}| add the following code
to the top of the main file:
%
\begin{center}
\begin{tabular}{l}
|\||ifdefined\childdocname\endinput\||fi\newif\ifchilddoc|\\
|\edef\childdocname{\scantokens\expandafter{\jobname\noexpand}}|\\
|\def\childdocmain{|\textit{main}|}\||ifx\childdocmain\childdocname\||else|\\
|\childdoctrue\includeonly{\childdocname}\let\jobname\childdocmain\||fi|\\
\end{tabular}
\end{center}
%
Instead of |\childdocof{|\textit{main}|}| just include the main file
at the top of each child file:
%
\begin{center}
|\input{|\textit{main}|}|
\end{center}
%
A simple redirection |\childdocforward{|\textit{dest}|}| is achieved by:
%
\begin{center}
|\def\jobname{|\textit{dest}|}\input{\jobname}|
\end{center}
%
The redirection with prefix
|\childdocforwardprefix[|\textit{prefix}|]{|\textit{dest}|}|
is accomplished by:
%
\begin{center}
\begin{tabular}{l}
|{\edef\jobname{\scantokens\expandafter{\jobname\noexpand}}|\\
|\def\redirectjob |\textit{prefix}|#1~~~{\gdef\jobname{|\textit{dest}|#1}}|\\
|\expandafter\redirectjob\jobname~~~}\input{\jobname}|
\end{tabular}
\end{center}

In an alternative approach,
child documents can be compiled by a specific command line
without additional code or specific definitions:
%
\begin{center}
|... -jobname "|\textit{target}|" "|[\textit{flags}]%
|\includeonly{|\textit{dest}|}\input{|\textit{main}|}"|
\end{center}
%

%%%%%%%%%%%%%%%%%%%%%%%%%%%%%%%%%%%%%%%%%%%%%%%%%%%%%%%%%%%%%%%%%%%%%%%%%%%%%%%%
%%%%%%%%%%%%%%%%%%%%%%%%%%%%%%%%%%%%%%%%%%%%%%%%%%%%%%%%%%%%%%%%%%%%%%%%%%%%%%%%
\section{Information}

%%%%%%%%%%%%%%%%%%%%%%%%%%%%%%%%%%%%%%%%%%%%%%%%%%%%%%%%%%%%%%%%%%%%%%%%%%%%%%%%
\subsection{Copyright}

Copyright \copyright{} 2017--2018 Niklas Beisert

This work may be distributed and/or modified under the
conditions of the \LaTeX{} Project Public License, either version 1.3
of this license or (at your option) any later version.
The latest version of this license is in
  \url{http://www.latex-project.org/lppl.txt}
and version 1.3 or later is part of all distributions of \LaTeX{}
version 2005/12/01 or later.

This work has the LPPL maintenance status `maintained'.

The Current Maintainer of this work is Niklas Beisert.

This work consists of the files |README.txt|, |childdoc.ins| and |childdoc.dtx|
as well as the derived files |childdoc.def|, |cdocsamp.tex|
with |cdocsch1.tex|, |cdocsch2.tex|, |cdocspt3.tex|, |cdocspt4.tex|,
|cdocsdrf.tex|, |cdocsfn1.tex|, |cdocsfn2.tex|
as well as |childdoc.pdf|.

%%%%%%%%%%%%%%%%%%%%%%%%%%%%%%%%%%%%%%%%%%%%%%%%%%%%%%%%%%%%%%%%%%%%%%%%%%%%%%%%
\subsection{Files and Installation}

The package consists of the files:
%
\begin{center}
\begin{tabular}{ll}
    |README.txt|   & readme file \\
    |childdoc.ins| & installation file \\
    |childdoc.dtx| & source file \\
    |childdoc.def| & definition file \\
    |cdocsamp.tex| & sample main file \\
    |cdocsch1.tex| & sample include file \\
    |cdocsch2.tex| & sample include file \\
    |cdocspt3.tex| & sample part file \\
    |cdocspt4.tex| & sample part file \\
    |cdocsdrf.tex| & sample redirection file \\
    |cdocsfn1.tex| & sample redirection file \\
    |cdocsfn2.tex| & sample redirection file \\
    |childdoc.pdf| & manual
\end{tabular}
\end{center}
%
The distribution consists of the files
|README.txt|, |childdoc.ins| and |childdoc.dtx|.
%
\begin{itemize}
\item
Run (pdf)\LaTeX{} on |childdoc.dtx|
to compile the manual |childdoc.pdf| (this file).
\item
Run \LaTeX{} on |childdoc.ins| to create the definitions file |childdoc.def|
and the sample |cdocsamp.tex| with include files
|cdocsch1.tex|, |cdocsch2.tex|, |cdocspt3.tex|, |cdocspt4.tex|,
|cdocsdrf.tex|, |cdocsfn1.tex|, |cdocsfn2.tex|.
Then copy the file |childdoc.def| to an appropriate directory of your \LaTeX{}
distribution, e.g.\ \textit{texmf-root}|/tex/latex/childdoc|.
\end{itemize}

%%%%%%%%%%%%%%%%%%%%%%%%%%%%%%%%%%%%%%%%%%%%%%%%%%%%%%%%%%%%%%%%%%%%%%%%%%%%%%%%
\subsection{Related CTAN Packages}

There are several other packages which offer a similar functionality:
%
\begin{itemize}
\item
The packages
\href{http://ctan.org/pkg/docmute}{\textsf{docmute}},
\href{http://ctan.org/pkg/includex}{\textsf{includex}} and
\href{http://ctan.org/pkg/standalone}{\textsf{standalone}}
provide commands to include only the document body of
a child file thus allowing both files to be compiled individually.
\item
The packages \href{http://ctan.org/pkg/subdocs}{\textsf{subdocs}}
and \href{http://ctan.org/pkg/subfiles}{\textsf{subfiles}}
provide structures in which the main and child documents can be
encapsulated and allowing them to be compiled individually.
The inclusion mechanism is different from the conventional |\include|.
\item
The package \href{http://ctan.org/pkg/combine}{\textsf{combine}}
is an elaborate solution to combine several documents into one.
\end{itemize}
%
See also the CTAN topic \href{http://ctan.org/topic/subdocs}{\textsf{subdocs}}
for further related packages.
The present package differs from the above solutions in that
a document structure constructed with the conventional |\include| mechanism
just needs two extra commands at the top of every file
such that all constituent files can be compiled individually.

%%%%%%%%%%%%%%%%%%%%%%%%%%%%%%%%%%%%%%%%%%%%%%%%%%%%%%%%%%%%%%%%%%%%%%%%%%%%%%%%
%\subsection{Feature Suggestions}
%
%The following is a list of features which may be useful for future
%versions of this package:
%%
%\begin{itemize}
%\item
%\ldots
%\end{itemize}

%%%%%%%%%%%%%%%%%%%%%%%%%%%%%%%%%%%%%%%%%%%%%%%%%%%%%%%%%%%%%%%%%%%%%%%%%%%%%%%%
\subsection{Revision History}

%%%%%%%%%%%%%%%%%%%%%%%%%%%%%%%%%%%%%%%%
\paragraph{v2.0:} 2018/12/30

\begin{itemize}
\item
immediate forward processing
\item
added |\childdocby| mechanism
\item
manual restructured
\end{itemize}

%%%%%%%%%%%%%%%%%%%%%%%%%%%%%%%%%%%%%%%%
\paragraph{v1.6:} 2018/01/17

\begin{itemize}
\item
application for development of include files
\item
corrections to manual
\end{itemize}

%%%%%%%%%%%%%%%%%%%%%%%%%%%%%%%%%%%%%%%%
\paragraph{v1.5:} 2017/05/21

\begin{itemize}
\item
more complete structuring introduced
\item
|\childdocof| introduced
\item
|\childdoc| renamed to |\childdocmain|
\item
|\childredirect| renamed to |\childdocforward| and |\childdocforwardprefix|
and functionality expanded
\end{itemize}

%%%%%%%%%%%%%%%%%%%%%%%%%%%%%%%%%%%%%%%%
\paragraph{v1.0:} 2017/04/27

\begin{itemize}
\item
manual and install package
\item
first version published on CTAN
\end{itemize}

%%%%%%%%%%%%%%%%%%%%%%%%%%%%%%%%%%%%%%%%
\paragraph{v0.6:} 2017/04/26

\begin{itemize}
\item
redirection mechanism added
\end{itemize}

%%%%%%%%%%%%%%%%%%%%%%%%%%%%%%%%%%%%%%%%
\paragraph{v0.5:} 2017/04/26

\begin{itemize}
\item
functionality in definition file
\end{itemize}


%%%%%%%%%%%%%%%%%%%%%%%%%%%%%%%%%%%%%%%%%%%%%%%%%%%%%%%%%%%%%%%%%%%%%%%%%%%%%%%%
%%%%%%%%%%%%%%%%%%%%%%%%%%%%%%%%%%%%%%%%%%%%%%%%%%%%%%%%%%%%%%%%%%%%%%%%%%%%%%%%
%%%%%%%%%%%%%%%%%%%%%%%%%%%%%%%%%%%%%%%%%%%%%%%%%%%%%%%%%%%%%%%%%%%%%%%%%%%%%%%%
\appendix

\settowidth\MacroIndent{\rmfamily\scriptsize 000\ }

 \DocInput{childdoc.dtx}

\end{document}
%</driver>
% \fi
%
% %%%%%%%%%%%%%%%%%%%%%%%%%%%%%%%%%%%%%%%%%%%%%%%%%%%%%%%%%%%%%%%%%%%%%%%%%%%%%%
% %%%%%%%%%%%%%%%%%%%%%%%%%%%%%%%%%%%%%%%%%%%%%%%%%%%%%%%%%%%%%%%%%%%%%%%%%%%%%%
% \section{Sample}
%\iffalse
%<*samplemain>
%\fi
%
% The following presents a sample document
% with two chapters, two parts, a title page,
% a compile flag as well as three forwarding files to set the flag.
% It consists of eight |.tex| files:
% \begin{center}
% \begin{tabular}{ll}
% |cdocsamp.tex|&main file\\
% |cdocsch1.tex|&include file for chapter 1\\
% |cdocsch2.tex|&include file for chapter 2\\
% |cdocspt3.tex|&include file for part 3\\
% |cdocspt4.tex|&include file for part 4\\
% |cdocsdrf.tex|&forwarding file for main file in draft mode\\
% |cdocsfi1.tex|&forwarding file for final version of chapter 1\\
% |cdocsfi2.tex|&forwarding file for final version of chapter 2\\
% \end{tabular}
% \end{center}
% Each of the eight files can be compiled directly by the \LaTeX{} compiler.
%
% %%%%%%%%%%%%%%%%%%%%%%%%%%%%%%%%%%%%%%
% \paragraph{Main File.}
%
% The main file is called |cdocsamp.tex|.
%
% Load the \textsf{childdoc} definitions and
% declare the filename for the main document:
%    \begin{macrocode}
% \iffalse
%
% childdoc.dtx Copyright (C) 2017-2018 Niklas Beisert
%
% This work may be distributed and/or modified under the
% conditions of the LaTeX Project Public License, either version 1.3
% of this license or (at your option) any later version.
% The latest version of this license is in
%   http://www.latex-project.org/lppl.txt
% and version 1.3 or later is part of all distributions of LaTeX
% version 2005/12/01 or later.
%
% This work has the LPPL maintenance status `maintained'.
%
% The Current Maintainer of this work is Niklas Beisert.
%
% This work consists of the files childdoc.dtx and childdoc.ins
% and the derived files childdoc.def and cdocsamp.tex with
% cdocsch1.tex, cdocsch2.tex, cdocsdrf.tex, cdocsfn1.tex, cdocsfn2.tex.
%
%<package>\ifdefined\childdocmain\endinput\fi
%<package>\ProvidesFile{childdoc.def}[2018/12/30 v2.0 child document driver]
%<samplemain>\ProvidesFile{cdocsamp.tex}[2018/12/30 v2.0 sample for childdoc]
%<*driver>
%\ProvidesFile{childdoc.drv}[2018/12/30 v2.0 childdoc reference manual file]
\PassOptionsToClass{10pt,a4paper}{article}
\documentclass{ltxdoc}

\usepackage[margin=35mm]{geometry}
\usepackage{hyperref}
\usepackage{hyperxmp}
\usepackage[usenames]{color}

\hypersetup{colorlinks=true}
\hypersetup{pdfstartview=FitH}
\hypersetup{pdfpagemode=UseNone}
\hypersetup{pdfsource={}}
\hypersetup{pdflang={en-UK}}
\hypersetup{pdfcopyright={Copyright 2017-2018 Niklas Beisert.
  This work may be distributed and/or modified under the
  conditions of the LaTeX Project Public License, either version 1.3
  of this license or (at your option) any later version.}}
\hypersetup{pdflicenseurl={http://www.latex-project.org/lppl.txt}}
\hypersetup{pdfcontactaddress={ETH Zurich, ITP, HIT K,
  Wolfgang-Pauli-Strasse 27}}
\hypersetup{pdfcontactpostcode={8093}}
\hypersetup{pdfcontactcity={Zurich}}
\hypersetup{pdfcontactcountry={Switzerland}}
\hypersetup{pdfcontactemail={nbeisert@itp.phys.ethz.ch}}
\hypersetup{pdfcontacturl={http://people.phys.ethz.ch/\xmptilde nbeisert/}}

\newcommand{\secref}[1]{\hyperref[#1]{section \ref*{#1}}}

\parskip1ex
\parindent0pt
\let\olditemize\itemize
\def\itemize{\olditemize\parskip0pt}

\begin{document}

\title{The \textsf{childdoc} Package}
\hypersetup{pdftitle={The childdoc Package}}
\author{Niklas Beisert\\[2ex]
  Institut f\"ur Theoretische Physik\\
  Eidgen\"ossische Technische Hochschule Z\"urich\\
  Wolfgang-Pauli-Strasse 27, 8093 Z\"urich, Switzerland\\[1ex]
  \href{mailto:nbeisert@itp.phys.ethz.ch}
  {\texttt{nbeisert@itp.phys.ethz.ch}}}
\hypersetup{pdfauthor={Niklas Beisert}}
\hypersetup{pdfsubject={Manual for the LaTeX2e Package childdoc}}
\date{30 December 2018, \textsf{v2.0}}
\maketitle

\begin{abstract}\noindent
\textsf{childdoc} is a \LaTeXe{} package
that enables the direct compilation
of document sections included by |\include|
to individual files.
\end{abstract}

\begingroup
\parskip0ex
\tableofcontents
\endgroup

%%%%%%%%%%%%%%%%%%%%%%%%%%%%%%%%%%%%%%%%%%%%%%%%%%%%%%%%%%%%%%%%%%%%%%%%%%%%%%%%
%%%%%%%%%%%%%%%%%%%%%%%%%%%%%%%%%%%%%%%%%%%%%%%%%%%%%%%%%%%%%%%%%%%%%%%%%%%%%%%%
\section{Introduction}

\LaTeX{} provides a mechanism to structure a large document (such as a book)
into a main file and several child files (containing the chapters)
using the |\include| command.
This mechanism is beneficial for documents
which span hundreds of pages in order to
make the source file(s) more manageable.
Moreover, compilation can be restricted to
selected child files by means of the |\includeonly| command.
The latter feature can be used to reduce the compilation time while editing
(this was significantly more useful in the earlier days of \LaTeX{})
or to generate a smaller document which is easier to navigate.
Another application of |\includeonly| is to generate
documents consisting of selected parts of the complete document.

However, there are a few drawbacks of the plain |\include| mechanism:
\begin{itemize}
\item
The child files cannot be compiled on their own,
they can only be compiled via the main file.
A naive editing environment
(such as a text editor with an option
to have the current file processed by \LaTeX)
may require one to switch to the main file before compiling;
attempting to compile the child file produces errors.
\item
The main file must be modified (each time)
to adjust the |\includeonly| command
to the present needs. This easily leaves the main file in a messy state.
\item
The generated document will always carry the filename
of the main document. This is inconvenient if
several child files are to be compiled and
to be kept for distribution.
\end{itemize}

The present package provides a simple interface
to make child files individually compilable by \LaTeX{}.
Compiling a child file then has the same effect as compiling
the main file with an |\includeonly| command
to select the appropriate child.
Moreover the generated document will carry the name of the child
rather than the main file.
This resolves all three above issues.

This feature is meant to make the editing of books,
thesis documents and lecture notes somewhat more convenient.
However, the package can also be used efficiently for
composing a series of documents (such as exercise sheets)
which are typically distributed individually.
It then assists the author in generating the individual documents
(potentially in different versions)
as well as a document containing the collected series.
Another application is in developing style files
or other kinds of included material
where compilation of the style file could redirect
to a sample or test file.

%%%%%%%%%%%%%%%%%%%%%%%%%%%%%%%%%%%%%%%%%%%%%%%%%%%%%%%%%%%%%%%%%%%%%%%%%%%%%%%%
%%%%%%%%%%%%%%%%%%%%%%%%%%%%%%%%%%%%%%%%%%%%%%%%%%%%%%%%%%%%%%%%%%%%%%%%%%%%%%%%
\section{Usage}

First of all, the package \textsf{childdoc} is \emph{not} a standard
\LaTeXe{} |.sty| style file! Therefore it needs to be invoked in
a non-standard way.

%%%%%%%%%%%%%%%%%%%%%%%%%%%%%%%%%%%%%%%%%%%%%%%%%%%%%%%%%%%%%%%%%%%%%%%%%%%%%%%%
\subsection{Included Files}
\label{sec:include}

%%%%%%%%%%%%%%%%%%%%%%%%%%%%%%%%%%%%%%%%
\DescribeMacro{\childdocmain}
To use the package, add the commands
\begin{center}
\begin{tabular}{l}
|\input{childdoc.def}|\\
|\childdocmain{}|\\
\end{tabular}
\end{center}
at the very top of the main \LaTeX{} file,
in particular \emph{before} the |\documentclass| statement!
The argument of |\childdocmain| should be left empty
(but it must be present).

%%%%%%%%%%%%%%%%%%%%%%%%%%%%%%%%%%%%%%%%
\DescribeMacro{\childdocof}
Furthermore, add the commands
\begin{center}
\begin{tabular}{l}
|\input{childdoc.def}|\\
|\childdocof{|\textit{main}|}|\\
\end{tabular}
\end{center}
at the top of every child file \textit{child}
which is included by |\include{|\textit{child}|}|
from within the main file
(or at least for those files to be compiled individually).
The argument \textit{main} must be the filename of the main file.

There are a couple of
considerations in setting up the main and child documents:

%%%%%%%%%%%%%%%%%%%%%%%%%%%%%%%%%%%%%%%%
\paragraph{Restrictions.}

Please note the following restrictions:
\begin{itemize}
\item
|\childdocmain| must be called with one argument \textit{main}
to ensure compatibility with earlier version of the package.
It must either be empty (|\childdocmain{}|)
or precisely match the filename of the main file in which it is specified.
See \secref{sec:detection} for further information.
\item
The filename \textit{main} must be specified without the |.tex| extension.
\item
The filename \textit{main} is case sensitive
(even in case-insensitive file systems)
due to internal string comparison.
\item
The argument \textit{main} should be fully expanded, it cannot be a macro.
\item
Subdirectories and special characters should be avoided in filenames.
\item
The command |\childdocmain{|\textit{main}|}| must be followed by a whitespace.
It should not be followed immediately by another command
or by a comment mark `|%|'.
This is because the \TeX{} parser reads the token immediately following
the argument of |\childdocmain| and puts it
at the beginning of every child section;
however, a white\-space is ignored.
\end{itemize}

%%%%%%%%%%%%%%%%%%%%%%%%%%%%%%%%%%%%%%%%
\paragraph{Content of Main File.}

It is advisable to place all content in the child files included by |\include|.
Any output contained in the main file will appear in all child documents
unless suppressed manually;
it cannot be suppressed automatically by the |\includeonly| directive
and thus should normally be avoided.
A method to include some content in the main file
by means of conditional processing is described in \secref{sec:conditional}.

%%%%%%%%%%%%%%%%%%%%%%%%%%%%%%%%%%%%%%%%
\paragraph{Page Numbering.}

When only a part of the document is compiled,
the appropriate numbering of pages
(as well as other status parameters)
is determined from the |.aux| files.
The latter contain information from previous passes.
However this information needs to propagate through
all intermediate child documents.
Therefore the page numbering in child documents may well
be inconsistent until the complete document is compiled at least once.

A useful (if unconventional) way to always ensure a consistent
page numbering is to restart the numbering in each child document
and denote the pages by `\textit{child}|.|\textit{page}'
where \textit{child} represents the chapter/section number of the child file.
This can be achieved by the command
|\numberwithin{page}{|\textit{child}|}|
of the \textsf{amsmath} package
where \textit{child} can be |chapter| or |section|
depending on the chosen structuring.
Alternatively, one can modify the macro |\thepage| appropriately
and reset the counter |page| at the start of each child file.

%%%%%%%%%%%%%%%%%%%%%%%%%%%%%%%%%%%%%%%%%%%%%%%%%%%%%%%%%%%%%%%%%%%%%%%%%%%%%%%%
\subsection{Conditional Processing}
\label{sec:conditional}

The package provides a mechanism to compile different versions
of a document. To customise the versions further some conditional processing
can come in handy to distinguish which version is being compiled.
The package provides two macros to describe the compilation context:

%%%%%%%%%%%%%%%%%%%%%%%%%%%%%%%%%%%%%%%%
\DescribeMacro{\ifchilddoc}
The conditional |\ifchilddoc| distinguishes between the compilation of
child documents and the main document:
%
\begin{center}
|\ifchilddoc |\textit{child-code}| |[|\||else |\textit{main-code}]| \||fi|
\end{center}

%%%%%%%%%%%%%%%%%%%%%%%%%%%%%%%%%%%%%%%%
\DescribeMacro{\childdocname}
\DescribeMacro{\childdocjob}
The macro |\childdocname| contains the filename (without extension)
of the main or child file being processed.
Note that |\childdocjob| will always contain the name of the main file.

%%%%%%%%%%%%%%%%%%%%%%%%%%%%%%%%%%%%%%%%
\paragraph{Title Page.}

Conditional processing can be used to include a title or banner page
in the main document when proper precautions are taken.
Importantly, the code in the main file should ensure that the page counter
(as well as other status parameters which are stored in the |.aux| files)
takes the same value after the conditional processing.
Otherwise the page numbers may take divergent values
depending on which part is compiled.

For example, a title page could be declared by:
%
\begin{center}
\begin{tabular}{l}
|\ifchilddoc\||else|\\
|\addtocounter{page}{-1}|\\
\textit{code for title page}\\
|\newpage|\\
|\||fi|
\end{tabular}
\end{center}
%
A banner page for the child documents can be generated by:
%
\begin{center}
\begin{tabular}{l}
|\ifchilddoc|\\
|\addtocounter{page}{-1}|\\
\textit{code for banner page}\\
|\newpage|\\
|\||fi|
\end{tabular}
\end{center}
%
Here one could write a message such as:
\begin{center}
|This is the part \childdocname{} of \childdocjob{}.|
\end{center}

%%%%%%%%%%%%%%%%%%%%%%%%%%%%%%%%%%%%%%%%%%%%%%%%%%%%%%%%%%%%%%%%%%%%%%%%%%%%%%%%
\subsection{Flags}
\label{sec:flags}

The package makes it easy to generate different versions
of the main or child documents.
To this end compilation flags can be defined
and assigned different default values.
They will be particularly useful in conjunction
with the forwarding mechanism described in \secref{sec:forward}.

For example, it may be useful to have a flag |\version|
which can be set to |draft| or |final|.
The document source will contain some conditional code
depending on the value of |\version|.
Suppose further, the flag should default to |final| for the main file
and to |draft| for child files
which is a natural assignment for editing the document.
This is achieved by placing the following code
in the preamble of the main document
(below the |\childdocmain| directive):
%
\begin{center}
\begin{tabular}{l}
|\ifchilddoc|\\
|\providecommand{\version}{draft}|\\
|\||else|\\
|\providecommand{\version}{final}|\\
|\||fi|
\end{tabular}
\end{center}
%
The definition by |\providecommand| makes sure
that previous definitions are not overwritten.
Further statements |\providecommand{\version}{...}|
can thus be added before the above code to override it.

For the main file, one might add a line
(between |\childdocmain| and the above block)
%
\begin{center}
|%\ifchilddoc\||else\providecommand{\version}{draft}\||fi|
\end{center}
%
which can be uncommented to produce a draft version.
Likewise one can add a line to the very top of a child file
(above the |\childdocof{|\textit{main}|}| directive)
%
\begin{center}
|%\providecommand{\version}{final}|
\end{center}
%
which can be uncommented to produce the final version of this child document.

%%%%%%%%%%%%%%%%%%%%%%%%%%%%%%%%%%%%%%%%%%%%%%%%%%%%%%%%%%%%%%%%%%%%%%%%%%%%%%%%
\subsection{Forwarding}
\label{sec:forward}

Different versions of the main or child documents
using compilation flags as described in \secref{sec:flags}
can be (permanently) stored in different files
for convenient compilation, viewing and distribution.
To this end, the package defines a command
to pass on compilation to a different file:

%%%%%%%%%%%%%%%%%%%%%%%%%%%%%%%%%%%%%%%%
\DescribeMacro{\childdocforward}
The command |\childdocforward| redirects processing to
another source file:
%
\begin{center}
\begin{tabular}{l}
|\input{childdoc.def}|\\
|\childdocforward[|\textit{main}|]{|\textit{dest}|}|\\
\end{tabular}
\end{center}
%
The argument \textit{dest} is the destination file
(without extension).
It should be the main file or one of the child files.
Note that further \textsf{childdoc} directives
such as |\childdocof| and |\childdocforward|
in the indicated file will be processed in this form.
The optional argument \textit{main}
passes on directly to the main file \textit{main}
while pretending to compile the child \textit{dest}.
This form behaves as if \textit{dest}
issues |\childdocof{|\textit{main}|}| right away,
and no further \textsf{childdoc} directives will be processed.

%%%%%%%%%%%%%%%%%%%%%%%%%%%%%%%%%%%%%%%%
\DescribeMacro{\...prefix}
In the alternative form |\childdocforwardprefix|,
%
\begin{center}
\begin{tabular}{l}
|\input{childdoc.def}|\\
|\childdocforwardprefix[|\textit{main}|]{|\textit{prefix}|}{|\textit{dest}|}|
\end{tabular}
\end{center}
%
the destination file is determined by a pattern
depending on the current file:
To make this work, the current file must be called
`{\textit{prefix}\hspace{0.2em}\textit{suffix}}'
with \textit{prefix} matching precisely the argument.
Processing is then passed on to the file
`{\textit{dest}\hspace{0.2em}\textit{suffix}}'.
Surely, the same effect is achieved by
directly specifying the
argument `{\textit{dest}\hspace{0.2em}\textit{suffix}}'
in the first form.
However, that requires to set up a different file
for each child. With the alternative form of the command
all these files can have exactly the same content
which simplifies setting them up and maintaining them.

For example, the following file |draft.tex|
with a compilation flag |\version| as described in \secref{sec:flags}
compiles the main document as a draft:
%
\begin{center}
\begin{tabular}{l}
|\def\version{draft}|\\
|\input{childdoc.def}|\\
|\childdocforward{|\textit{main}|}|
\end{tabular}
\end{center}
%
Likewise, the following files |final|\textit{nn}|.tex|
compile the final version of the child document
|child|\textit{nn}|.tex|:
%
\begin{center}
\begin{tabular}{l}
|\def\version{final}|\\
|\input{childdoc.def}|\\
|\childdocforwardprefix{final}{child}|
\end{tabular}
\end{center}
%

Note that when several versions of a main file and/or of each child file
are to be generated, it may be convenient to set up a |Makefile| or
shell script to automatise the process.

%%%%%%%%%%%%%%%%%%%%%%%%%%%%%%%%%%%%%%%%%%%%%%%%%%%%%%%%%%%%%%%%%%%%%%%%%%%%%%%%
\subsection{Command Line Processing}
\label{sec:commandline}

The effect of redirection files can also be achieved by invoking
the \LaTeX{} compiler with a more elaborate command line.
Most conveniently this should be done as part
of a shell script or a |Makefile|.

When using \textsf{childdoc} in the main file, the following
command lines effectively perform a redirection
(note that depending on the shell being used,
backslashes may have to be doubled: `|\|' $\to$ `|\\|'):
%
\begin{center}
|... -jobname "|\textit{target}|" |\\|"|[\textit{flags}]%
|\input{childdoc.def}\childdocforward[|\textit{main}|]{|\textit{dest}|}"|
\end{center}
%
Here \textit{target} is the name of the output file,
\textit{main} is the name of the main file
and \textit{dest} is the name of the main or child file to be processed
(all filenames without extensions).
The optional argument \textit{main} can be omitted
if \textit{main} matches \textit{dest}.
Optionally, compilation \textit{flags} can be defined via |\def| commands.
This command line makes the \TeX{} engine believe
it is compiling the file \textit{target}
whose content is specified as the latter parameter.
The provided code then forwards the processing to
\textit{main} or \textit{dest} as described in \secref{sec:forward}.

%%%%%%%%%%%%%%%%%%%%%%%%%%%%%%%%%%%%%%%%%%%%%%%%%%%%%%%%%%%%%%%%%%%%%%%%%%%%%%%%
\subsection{Include by Input}
\label{sec:input}

Including child documents by |\include| has some restrictions by design.
Most notably, the content of a child document always occupies
its own set of pages; pages cannot be shared between child documents.
Usually, this behaviour makes perfect sense
because each child document contain an essential part of the document.
However, in some situations it may be desirable to compose
a document from a collection of parts
without having mandatory page breaks between then.
For this case, the package
provides a mechanism to include parts
by |\input| which can also be processed individually.
However, by construction this mechanism
requires manual handling of the content to be output.

%%%%%%%%%%%%%%%%%%%%%%%%%%%%%%%%%%%%%%%%
\DescribeMacro{\ifchilddocmanual}
The main file should be prepared as usual, see \secref{sec:include}.
However, the document body must make a distinction
between processing of an individual part and of the main document, e.g.:
%
\begin{center}
\begin{tabular}{l}
|\ifchilddocmanual|\\
|\input{\childdocname}|\\
|\||else|\\
\textit{document body with }|\input{|\textit{part}|}|\\
|\||fi|
\end{tabular}
\end{center}
%
The conditional |\ifchilddocmanual| is true whenever
a part to be included by |\input| is being compiled,
and the name of the part is stored in |\childdocname|.

%%%%%%%%%%%%%%%%%%%%%%%%%%%%%%%%%%%%%%%%
\DescribeMacro{\childdocby}
Each part to be included by |\input| should start with:
%
\begin{center}
\begin{tabular}{l}
|\input{childdoc.def}|\\
|\childdocby{|\textit{main}|}|\\
\end{tabular}
\end{center}
%
The directive |\childdocby| is similar to |\childdocof|
described in \secref{sec:include},
but the subsequent selection of content must be done manually.
To that end, both |\ifchilddoc| and |\ifchilddocmanual|
will be true upon processing of a part,
and the name of the part is stored in |\childdocname|.
Note that |\jobname| will be set to the filename of the current part
so that each part receives an individual |.aux| file
that does not interfere with the |.aux| file(s) of the main document.
This behaviour can be altered by the alternative form
|\childdocby[*]{|\textit{main}|}| (with a non-empty optional argument)
which uses the |.aux| file of the main document
by setting |\jobname| to \textit{main}.

%%%%%%%%%%%%%%%%%%%%%%%%%%%%%%%%%%%%%%%%%%%%%%%%%%%%%%%%%%%%%%%%%%%%%%%%%%%%%%%%
\subsection{Driver Development}
\label{sec:driver}

The \textsf{childdoc} mechanism can also be use for the development
of definition files such as \LaTeX{} styles or classes.
This case differs from the above setup with multiple parts
included by |\include| in that no |\includeonly| should be invoked.
This can be achieved by starting the include file
(before |\ProvidesPackage|) with:
%
\begin{center}
\begin{tabular}{l}
|\input{childdoc.def}|\\
|\childdocforward{|\textit{main}|}|\\
\end{tabular}
\end{center}
%
or alternatively with:
%
\begin{center}
\begin{tabular}{l}
|\input{childdoc.def}|\\
|\childdocby{|\textit{main}|}|\\
\end{tabular}
\end{center}
%
Both forms have slightly different effects as described above.
The main file is prepared as usual, see \secref{sec:include}.

%%%%%%%%%%%%%%%%%%%%%%%%%%%%%%%%%%%%%%%%%%%%%%%%%%%%%%%%%%%%%%%%%%%%%%%%%%%%%%%%
\subsection{Legacy Detection}
\label{sec:detection}

The directive |\childdocmain| in the main file can detect
whether the complete document or merely a child is to be compiled
even without using the directive |\childdocof|.
This method is deprecated because it is less robust
and there is no compelling reason to use it;
it is merely provided for backward compatibility
and it may be removed in future versions.

If the detection mechanism is to be used,
it is mandatory to correctly specify
the filename of the main file as the argument of |\childdocmain|:
%
\begin{center}
\begin{tabular}{l}
|\input{childdoc.def}|\\
|\childdocmain{|\textit{main}|}|\\
\end{tabular}
\end{center}
%
If |\jobname| does not match the argument \textit{main} of |\childdocmain|,
it is assumed that |\jobname| points to the child file to be compiled.
When using |\childdocmain| with the main file specified as argument,
it suffices to start a child file
with just |\input{|\textit{main}|}|
without loading of the package and using |\childdocof|.
If instead all processing is done
with the appropriate \textsf{childdoc} directives,
the argument of \textit{main} of |\childdocmain| can be empty.

An alternative version of the command line processing described
in \secref{sec:commandline} using the detection mechanism reads:
%
\begin{center}
|... -jobname "|\textit{target}|" "|[\textit{flags}]%
[|\def\jobname{|\textit{dest}|}|]|\input{|\textit{main}|}"|
\end{center}

%%%%%%%%%%%%%%%%%%%%%%%%%%%%%%%%%%%%%%%%%%%%%%%%%%%%%%%%%%%%%%%%%%%%%%%%%%%%%%%%
\subsection{Manual Code}
\label{sec:manual}

In case one cannot be certain whether the definitions file |childdoc.def|
is installed on the target \TeX{} distribution
and one prefers not to ship it,
it is conceivable to paste a few relevant commands into the sources.

To that end, drop all statements |\input{childdoc.def}|
and perform the replacements as outlined below.
Instead of |\childdocmain{|\textit{main}|}| add the following code
to the top of the main file:
%
\begin{center}
\begin{tabular}{l}
|\||ifdefined\childdocname\endinput\||fi\newif\ifchilddoc|\\
|\edef\childdocname{\scantokens\expandafter{\jobname\noexpand}}|\\
|\def\childdocmain{|\textit{main}|}\||ifx\childdocmain\childdocname\||else|\\
|\childdoctrue\includeonly{\childdocname}\let\jobname\childdocmain\||fi|\\
\end{tabular}
\end{center}
%
Instead of |\childdocof{|\textit{main}|}| just include the main file
at the top of each child file:
%
\begin{center}
|\input{|\textit{main}|}|
\end{center}
%
A simple redirection |\childdocforward{|\textit{dest}|}| is achieved by:
%
\begin{center}
|\def\jobname{|\textit{dest}|}\input{\jobname}|
\end{center}
%
The redirection with prefix
|\childdocforwardprefix[|\textit{prefix}|]{|\textit{dest}|}|
is accomplished by:
%
\begin{center}
\begin{tabular}{l}
|{\edef\jobname{\scantokens\expandafter{\jobname\noexpand}}|\\
|\def\redirectjob |\textit{prefix}|#1~~~{\gdef\jobname{|\textit{dest}|#1}}|\\
|\expandafter\redirectjob\jobname~~~}\input{\jobname}|
\end{tabular}
\end{center}

In an alternative approach,
child documents can be compiled by a specific command line
without additional code or specific definitions:
%
\begin{center}
|... -jobname "|\textit{target}|" "|[\textit{flags}]%
|\includeonly{|\textit{dest}|}\input{|\textit{main}|}"|
\end{center}
%

%%%%%%%%%%%%%%%%%%%%%%%%%%%%%%%%%%%%%%%%%%%%%%%%%%%%%%%%%%%%%%%%%%%%%%%%%%%%%%%%
%%%%%%%%%%%%%%%%%%%%%%%%%%%%%%%%%%%%%%%%%%%%%%%%%%%%%%%%%%%%%%%%%%%%%%%%%%%%%%%%
\section{Information}

%%%%%%%%%%%%%%%%%%%%%%%%%%%%%%%%%%%%%%%%%%%%%%%%%%%%%%%%%%%%%%%%%%%%%%%%%%%%%%%%
\subsection{Copyright}

Copyright \copyright{} 2017--2018 Niklas Beisert

This work may be distributed and/or modified under the
conditions of the \LaTeX{} Project Public License, either version 1.3
of this license or (at your option) any later version.
The latest version of this license is in
  \url{http://www.latex-project.org/lppl.txt}
and version 1.3 or later is part of all distributions of \LaTeX{}
version 2005/12/01 or later.

This work has the LPPL maintenance status `maintained'.

The Current Maintainer of this work is Niklas Beisert.

This work consists of the files |README.txt|, |childdoc.ins| and |childdoc.dtx|
as well as the derived files |childdoc.def|, |cdocsamp.tex|
with |cdocsch1.tex|, |cdocsch2.tex|, |cdocspt3.tex|, |cdocspt4.tex|,
|cdocsdrf.tex|, |cdocsfn1.tex|, |cdocsfn2.tex|
as well as |childdoc.pdf|.

%%%%%%%%%%%%%%%%%%%%%%%%%%%%%%%%%%%%%%%%%%%%%%%%%%%%%%%%%%%%%%%%%%%%%%%%%%%%%%%%
\subsection{Files and Installation}

The package consists of the files:
%
\begin{center}
\begin{tabular}{ll}
    |README.txt|   & readme file \\
    |childdoc.ins| & installation file \\
    |childdoc.dtx| & source file \\
    |childdoc.def| & definition file \\
    |cdocsamp.tex| & sample main file \\
    |cdocsch1.tex| & sample include file \\
    |cdocsch2.tex| & sample include file \\
    |cdocspt3.tex| & sample part file \\
    |cdocspt4.tex| & sample part file \\
    |cdocsdrf.tex| & sample redirection file \\
    |cdocsfn1.tex| & sample redirection file \\
    |cdocsfn2.tex| & sample redirection file \\
    |childdoc.pdf| & manual
\end{tabular}
\end{center}
%
The distribution consists of the files
|README.txt|, |childdoc.ins| and |childdoc.dtx|.
%
\begin{itemize}
\item
Run (pdf)\LaTeX{} on |childdoc.dtx|
to compile the manual |childdoc.pdf| (this file).
\item
Run \LaTeX{} on |childdoc.ins| to create the definitions file |childdoc.def|
and the sample |cdocsamp.tex| with include files
|cdocsch1.tex|, |cdocsch2.tex|, |cdocspt3.tex|, |cdocspt4.tex|,
|cdocsdrf.tex|, |cdocsfn1.tex|, |cdocsfn2.tex|.
Then copy the file |childdoc.def| to an appropriate directory of your \LaTeX{}
distribution, e.g.\ \textit{texmf-root}|/tex/latex/childdoc|.
\end{itemize}

%%%%%%%%%%%%%%%%%%%%%%%%%%%%%%%%%%%%%%%%%%%%%%%%%%%%%%%%%%%%%%%%%%%%%%%%%%%%%%%%
\subsection{Related CTAN Packages}

There are several other packages which offer a similar functionality:
%
\begin{itemize}
\item
The packages
\href{http://ctan.org/pkg/docmute}{\textsf{docmute}},
\href{http://ctan.org/pkg/includex}{\textsf{includex}} and
\href{http://ctan.org/pkg/standalone}{\textsf{standalone}}
provide commands to include only the document body of
a child file thus allowing both files to be compiled individually.
\item
The packages \href{http://ctan.org/pkg/subdocs}{\textsf{subdocs}}
and \href{http://ctan.org/pkg/subfiles}{\textsf{subfiles}}
provide structures in which the main and child documents can be
encapsulated and allowing them to be compiled individually.
The inclusion mechanism is different from the conventional |\include|.
\item
The package \href{http://ctan.org/pkg/combine}{\textsf{combine}}
is an elaborate solution to combine several documents into one.
\end{itemize}
%
See also the CTAN topic \href{http://ctan.org/topic/subdocs}{\textsf{subdocs}}
for further related packages.
The present package differs from the above solutions in that
a document structure constructed with the conventional |\include| mechanism
just needs two extra commands at the top of every file
such that all constituent files can be compiled individually.

%%%%%%%%%%%%%%%%%%%%%%%%%%%%%%%%%%%%%%%%%%%%%%%%%%%%%%%%%%%%%%%%%%%%%%%%%%%%%%%%
%\subsection{Feature Suggestions}
%
%The following is a list of features which may be useful for future
%versions of this package:
%%
%\begin{itemize}
%\item
%\ldots
%\end{itemize}

%%%%%%%%%%%%%%%%%%%%%%%%%%%%%%%%%%%%%%%%%%%%%%%%%%%%%%%%%%%%%%%%%%%%%%%%%%%%%%%%
\subsection{Revision History}

%%%%%%%%%%%%%%%%%%%%%%%%%%%%%%%%%%%%%%%%
\paragraph{v2.0:} 2018/12/30

\begin{itemize}
\item
immediate forward processing
\item
added |\childdocby| mechanism
\item
manual restructured
\end{itemize}

%%%%%%%%%%%%%%%%%%%%%%%%%%%%%%%%%%%%%%%%
\paragraph{v1.6:} 2018/01/17

\begin{itemize}
\item
application for development of include files
\item
corrections to manual
\end{itemize}

%%%%%%%%%%%%%%%%%%%%%%%%%%%%%%%%%%%%%%%%
\paragraph{v1.5:} 2017/05/21

\begin{itemize}
\item
more complete structuring introduced
\item
|\childdocof| introduced
\item
|\childdoc| renamed to |\childdocmain|
\item
|\childredirect| renamed to |\childdocforward| and |\childdocforwardprefix|
and functionality expanded
\end{itemize}

%%%%%%%%%%%%%%%%%%%%%%%%%%%%%%%%%%%%%%%%
\paragraph{v1.0:} 2017/04/27

\begin{itemize}
\item
manual and install package
\item
first version published on CTAN
\end{itemize}

%%%%%%%%%%%%%%%%%%%%%%%%%%%%%%%%%%%%%%%%
\paragraph{v0.6:} 2017/04/26

\begin{itemize}
\item
redirection mechanism added
\end{itemize}

%%%%%%%%%%%%%%%%%%%%%%%%%%%%%%%%%%%%%%%%
\paragraph{v0.5:} 2017/04/26

\begin{itemize}
\item
functionality in definition file
\end{itemize}


%%%%%%%%%%%%%%%%%%%%%%%%%%%%%%%%%%%%%%%%%%%%%%%%%%%%%%%%%%%%%%%%%%%%%%%%%%%%%%%%
%%%%%%%%%%%%%%%%%%%%%%%%%%%%%%%%%%%%%%%%%%%%%%%%%%%%%%%%%%%%%%%%%%%%%%%%%%%%%%%%
%%%%%%%%%%%%%%%%%%%%%%%%%%%%%%%%%%%%%%%%%%%%%%%%%%%%%%%%%%%%%%%%%%%%%%%%%%%%%%%%
\appendix

\settowidth\MacroIndent{\rmfamily\scriptsize 000\ }

 \DocInput{childdoc.dtx}

\end{document}
%</driver>
% \fi
%
% %%%%%%%%%%%%%%%%%%%%%%%%%%%%%%%%%%%%%%%%%%%%%%%%%%%%%%%%%%%%%%%%%%%%%%%%%%%%%%
% %%%%%%%%%%%%%%%%%%%%%%%%%%%%%%%%%%%%%%%%%%%%%%%%%%%%%%%%%%%%%%%%%%%%%%%%%%%%%%
% \section{Sample}
%\iffalse
%<*samplemain>
%\fi
%
% The following presents a sample document
% with two chapters, two parts, a title page,
% a compile flag as well as three forwarding files to set the flag.
% It consists of eight |.tex| files:
% \begin{center}
% \begin{tabular}{ll}
% |cdocsamp.tex|&main file\\
% |cdocsch1.tex|&include file for chapter 1\\
% |cdocsch2.tex|&include file for chapter 2\\
% |cdocspt3.tex|&include file for part 3\\
% |cdocspt4.tex|&include file for part 4\\
% |cdocsdrf.tex|&forwarding file for main file in draft mode\\
% |cdocsfi1.tex|&forwarding file for final version of chapter 1\\
% |cdocsfi2.tex|&forwarding file for final version of chapter 2\\
% \end{tabular}
% \end{center}
% Each of the eight files can be compiled directly by the \LaTeX{} compiler.
%
% %%%%%%%%%%%%%%%%%%%%%%%%%%%%%%%%%%%%%%
% \paragraph{Main File.}
%
% The main file is called |cdocsamp.tex|.
%
% Load the \textsf{childdoc} definitions and
% declare the filename for the main document:
%    \begin{macrocode}
\input{childdoc.def}
\childdocmain{}
%    \end{macrocode}

% Optional override for |\version| flag:
%    \begin{macrocode}
%%\ifchilddoc\else\providecommand{\version}{draft}\fi
%    \end{macrocode}

% Define the default values for the |\version| flag
% (|final| for the main file and |draft| for childs):
%    \begin{macrocode}
\ifchilddoc
\providecommand{\version}{draft}
\else
\providecommand{\version}{final}
\fi
%    \end{macrocode}

% Load the standard document class:
%    \begin{macrocode}
\documentclass[12pt]{article}
%    \end{macrocode}

% Start the document body:
%    \begin{macrocode}
\begin{document}
%    \end{macrocode}

% Declare a title page.
% Print title, part of document being processed and version flag:
%    \begin{macrocode}
\addtocounter{page}{-1}
\begin{center}
{\LARGE\bfseries{}childdoc example\par}
\vspace{1cm}
\ifchilddoc
\ifchilddocmanual part\else chapter\fi:
`\childdocname' of `\childdocjob'\par
\else
main document: `\childdocjob'\par
\fi
version: \version\par
\end{center}
\newpage
%    \end{macrocode}

% Manually include selected file,
% otherwise process as usual:
%    \begin{macrocode}
\ifchilddocmanual
\section*{part `\childdocname'}
\input{\childdocname}
\else
%    \end{macrocode}

% Include the two chapters:
%    \begin{macrocode}
\include{cdocsch1}
\include{cdocsch2}
%    \end{macrocode}

% Include the two parts unless only chapters should be displayed:
%    \begin{macrocode}
\ifchilddoc\else
\section{part three}
\input{cdocspt3}
\section{part four}
\input{cdocspt4}
\fi
%    \end{macrocode}

% Process as usual until here:
%    \begin{macrocode}
\fi
%    \end{macrocode}

% End of document body:
%    \begin{macrocode}
\end{document}
%    \end{macrocode}
%\iffalse
%</samplemain>
%\fi
%
% %%%%%%%%%%%%%%%%%%%%%%%%%%%%%%%%%%%%%%
% \paragraph{Chapter Include Files.}
%
% The include files are called |cdocsch1.tex| and |cdocsch2.tex|.
%
%\iffalse
%<*samplechap1|samplechap2>
%\fi

% Optional override for |\version| flag:
%    \begin{macrocode}
%%\providecommand{\version}{final}
%    \end{macrocode}

% Include the main document:
%    \begin{macrocode}
\input{childdoc.def}
\childdocof{cdocsamp}
%    \end{macrocode}

%\iffalse
%</samplechap1|samplechap2>
%\fi
%
%\iffalse
%<*samplechap1>
%\fi
% Some text for chapter 1:
%    \begin{macrocode}
\section{one}
some text in chapter one
%    \end{macrocode}

%\iffalse
%</samplechap1>
%\fi
% Some text for chapter 2:
%\iffalse
%<*samplechap2>
%\fi
%    \begin{macrocode}
\section{two}
more text in chapter two
%    \end{macrocode}

%\iffalse
%</samplechap2>
%\fi
%
% %%%%%%%%%%%%%%%%%%%%%%%%%%%%%%%%%%%%%%
% \paragraph{Part Include Files.}
%
% The include files are called |cdocspt3.tex| and |cdocspt4.tex|.
%
%\iffalse
%<*samplepart3|samplepart4>
%\fi

% Optional override for |\version| flag:
%    \begin{macrocode}
%%\providecommand{\version}{final}
%    \end{macrocode}

% Include the main document:
%    \begin{macrocode}
\input{childdoc.def}
\childdocby{cdocsamp}
%    \end{macrocode}

%\iffalse
%</samplepart3|samplepart4>
%\fi
%
%\iffalse
%<*samplepart3>
%\fi
% Some text for part 3:
%    \begin{macrocode}
some text in part three
%    \end{macrocode}

%\iffalse
%</samplepart3>
%\fi
% Some text for part 4:
%\iffalse
%<*samplepart4>
%\fi
%    \begin{macrocode}
more text in part four
%    \end{macrocode}

%\iffalse
%</samplepart4>
%\fi
%
% %%%%%%%%%%%%%%%%%%%%%%%%%%%%%%%%%%%%%%
% \paragraph{Forwarding for a Complete Draft.}
%
% The following forwarding file |cdocsdrf.tex|
% compiles the main document in draft mode:
%\iffalse
%<*sampledraft>
%\fi
%    \begin{macrocode}
\def\version{draft}
\input{childdoc.def}
\childdocforward{cdocsamp}
%    \end{macrocode}

%\iffalse
%</sampledraft>
%\fi
%
% %%%%%%%%%%%%%%%%%%%%%%%%%%%%%%%%%%%%%%
% \paragraph{Forwarding for Final Version of the Chapters.}
%
% The following forwarding files |cdocsfn1.tex| and |cdocsfn2.tex|
% (with identical content)
% compile the final versions of the child documents
% |cdocsch1.tex| and |cdocsch2.tex|, respectively:
%\iffalse
%<*samplefinal>
%\fi
%    \begin{macrocode}
\def\version{final}
\input{childdoc.def}
\childdocforwardprefix[cdocsamp]{cdocsfn}{cdocsch}
%    \end{macrocode}

%\iffalse
%</samplefinal>
%\fi
%
% %%%%%%%%%%%%%%%%%%%%%%%%%%%%%%%%%%%%%%
% \paragraph{Command Line Processing.}
%
% The following three command lines generate the output files
% |cdocscld|, |cdocscl1| and |cdocscl2|
% which should be identical to
% |cdocsdrf|, |cdocsch1| and |cdocsfn2|, respectively:
% \begin{center}
% \begin{tabular}{l}
% |latex -jobname cdocscld \|\\
% |  "\def\version{draft}\input{childdoc.def}\childdocforward{cdocsamp}"|\\
% |latex -jobname cdocscl1 \|\\
% |  "\input{childdoc.def}\childdocforward[cdocsamp]{cdocsch1}"|\\
% |latex -jobname cdocscl2 \|\\
% |  "\def\version{final}\input{childdoc.def}\childdocforward{cdocsch2}"|
% \end{tabular}
% \end{center}
% Note that the trailing backslash on each first line
% merely continues the input to the second line
% (for convenient cut ant paste).
% Furthermore, the command |latex| can be replaced by any
% of its alternative versions such as |pdflatex|.
%
% %%%%%%%%%%%%%%%%%%%%%%%%%%%%%%%%%%%%%%%%%%%%%%%%%%%%%%%%%%%%%%%%%%%%%%%%%%%%%%
% %%%%%%%%%%%%%%%%%%%%%%%%%%%%%%%%%%%%%%%%%%%%%%%%%%%%%%%%%%%%%%%%%%%%%%%%%%%%%%
% \section{Implementation}
%\iffalse
%<*package>
%\fi
%
% This section describes the definitions file |childdoc.def|.

% The definitions cannot be loaded using |\usepackage| or |\RequirePackage|
% which has a mechanism to prevent loading a style file more than once.
% When loading the definitions by means of |\input|
% multiple instances have to be prevented manually:
%\iffalse
%This code needs to be before the `\ProvidesFile' directive
%which is defined at the beginning of this file.
%Therefore it is also placed there and commented out here.
%</package>
%<*discard>
%\fi
%    \begin{macrocode}
\ifdefined\childdocmain\endinput\fi
%    \end{macrocode}
%\iffalse
%</discard>
%<*package>
%\fi
%
% \macro{\ifchilddoc}
% \macro{\ifchilddocmanual}
% The conditional |\ifchilddoc| tells whether a
% child (true) or main (false) document is being compiled.
% The conditional |\ifchilddocmanual| tells whether
% the |\includeonly| mechanism is used (false) or
% the selection of child files must be performed manually (true).
% The definitions initialise to false:
%    \begin{macrocode}
\newif\ifchilddoc
\newif\ifchilddocmanual
%    \end{macrocode}

% \macro{\childdocname}
% \macro{\childdocjob}
% The macro |\childdocname| stores the name of the main document
% to be compiled. The macro |\childdocjob| stores the name of
% the document on which the \LaTeX{} compiler was originally invoked.
% The content of |\jobname| cannot be compared
% to filenames specified in the source due to different catcodes.
% The following code rescans |\jobname|, stores the result
% in |\childdocname| and saves a copy in |\childdocjob|:
%    \begin{macrocode}
\edef\childdocname{\scantokens\expandafter{\jobname\noexpand}}
\let\childdocjob\childdocname
%    \end{macrocode}

% \macro{\childdocdisable}
% The macro |\childdocdisable| prevents the main file
% from being processed more than once.
% At this stage, the main document command |\childdocmain|
% is assumed to be called once again where it should do nothing.
% Any subsequent call to it should prevent
% a secondary processing of the main document
% It overwrites the forwarding commands
% |\childdocof| and |\childdocforward|
% with empty macros to prevent further inclusions of the main document:
%    \begin{macrocode}
\newcommand{\childdocdisable}
{
  \renewcommand{\childdocmain}[1]{\renewcommand{\childdocmain}[1]{\endinput}}
  \renewcommand{\childdocof}[1]{}
  \renewcommand{\childdocby}[2][]{}
  \renewcommand{\childdocforward}[2][]{}
  \renewcommand{\childdocdisable}{}
}
%    \end{macrocode}

% \macro{\childdocmain}
% The macro |\childdocmain| is to be called at the top of the main file
% with nothing or the main filename (without extension) as argument.
% First, it breaks loops.
% If the argument is not empty and does not match |\childdocname|
% (which is set by the first inclusion of |childdoc.def|),
% |\ifchilddoc| is set to true, |\includeonly| is applied to the child file
% and |\jobname| is set to the main file
% (for proper handling of |.aux| files):
%    \begin{macrocode}
\newcommand{\childdocmain}[1]
{
  \childdocdisable\childdocmain{}
  \if?#1?\else
    \begingroup
      \def\childdoctmp{#1}
      \ifx\childdoctmp\childdocname
        \def\childdoctmp{}
      \else
        \def\childdoctmp
        {
          \childdoctrue
          \includeonly{\childdocname}
          \def\childdocjob{#1}
          \def\jobname{#1}
        }
      \fi
      \expandafter
    \endgroup
    \childdoctmp
  \fi
}
%    \end{macrocode}

% \macro{\childdocof}
% The command |\childdocof| redirects
% compilation to the main file |#1|.
%    \begin{macrocode}
\newcommand{\childdocof}[1]
{
  \childdocdisable
  \childdoctrue
  \includeonly{\childdocname}
  \def\jobname{#1}
  \def\childdocjob{#1}
  \input{#1}
}
%    \end{macrocode}

% \macro{\childdocby}
% The command |\childdocby| ....
%    \begin{macrocode}
\newcommand{\childdocby}[2][]
{
  \childdocdisable
  \childdoctrue
  \childdocmanualtrue
  \if?#1?\else
    \def\jobname{#2}
  \fi
  \def\childdocjob{#2}
  \input{#2}
  \endinput
}
%    \end{macrocode}

% \macro{\childdocforward}
% The command |\childdocforward| redirects
% compilation to the main file or
% (if the optional argument is given) a child file.
% Parameters are set as if the main file
% or a child file starting with |\childdocof| was compiled.
% Then compilation is handed over to the main file:
%    \begin{macrocode}
\newcommand{\childdocforward}[2][]
{
  \begingroup
    \if?#1?
      \def\childdoctmp
      {
        \def\childdocname{#2}
        \def\childdocjob{#2}
        \def\jobname{#2}
        \input{#2}
        \endinput
      }
    \else
      \def\childdoctmp
      {
        \childdocdisable
        \def\childdocname{#2}
        \childdoctrue
        \includeonly{#2}
        \def\childdocjob{#1}
        \def\jobname{#1}
        \input{#1}
        \endinput
      }
    \fi
    \expandafter
  \endgroup
  \childdoctmp
}
%    \end{macrocode}

% \macro{\childdocforwardprefix}
% The command |\childdocforwardprefix| redirects
% compilation to the main or a child file by means of a pattern.
% The prefix |#1| in the current filename is replaced by |#2|
% and the suffix of the current filename is kept
% (it is assumed that the filename does not contain the substring `|~~~|'
% which is used as a delimiter).
% Compilation is handed over to the new file by |\childdocforward|:
%    \begin{macrocode}
\newcommand{\childdocforwardprefix}[3][]
{
  \begingroup
    \def\childdocextract #2##1~~~{\def\childdoctmp{\childdocforward[#1]{#3##1}}}
    \expandafter\childdocextract\childdocname~~~
    \expandafter
  \endgroup
  \childdoctmp
}
%    \end{macrocode}

% \macro{\childdoc}
% The deprecated macro |\childdoc| is a legacy version of |\childdocmain|:
%    \begin{macrocode}
\newcommand{\childdoc}{\childdocmain}
%    \end{macrocode}

% \macro{\childdocredirect}
% The deprecated macro |\childdocredirect| is a legacy version
% of |\childdocforward| and |\childdocforwardprefix|:
%    \begin{macrocode}
\newcommand{\childdocredirect}[2][]
{
  \begingroup
    \if?#1?
      \def\childdoctmp{\childdocforward{#2}}
    \else
      \def\childdoctmp{\childdocforwardprefix{#1}{#2}}
    \fi
    \expandafter
  \endgroup
  \childdoctmp
}
%    \end{macrocode}

%\iffalse
%</package>
%\fi
%
\endinput

\childdocmain{}
%    \end{macrocode}

% Optional override for |\version| flag:
%    \begin{macrocode}
%%\ifchilddoc\else\providecommand{\version}{draft}\fi
%    \end{macrocode}

% Define the default values for the |\version| flag
% (|final| for the main file and |draft| for childs):
%    \begin{macrocode}
\ifchilddoc
\providecommand{\version}{draft}
\else
\providecommand{\version}{final}
\fi
%    \end{macrocode}

% Load the standard document class:
%    \begin{macrocode}
\documentclass[12pt]{article}
%    \end{macrocode}

% Start the document body:
%    \begin{macrocode}
\begin{document}
%    \end{macrocode}

% Declare a title page.
% Print title, part of document being processed and version flag:
%    \begin{macrocode}
\addtocounter{page}{-1}
\begin{center}
{\LARGE\bfseries{}childdoc example\par}
\vspace{1cm}
\ifchilddoc
\ifchilddocmanual part\else chapter\fi:
`\childdocname' of `\childdocjob'\par
\else
main document: `\childdocjob'\par
\fi
version: \version\par
\end{center}
\newpage
%    \end{macrocode}

% Manually include selected file,
% otherwise process as usual:
%    \begin{macrocode}
\ifchilddocmanual
\section*{part `\childdocname'}
\input{\childdocname}
\else
%    \end{macrocode}

% Include the two chapters:
%    \begin{macrocode}
\include{cdocsch1}
\include{cdocsch2}
%    \end{macrocode}

% Include the two parts unless only chapters should be displayed:
%    \begin{macrocode}
\ifchilddoc\else
\section{part three}
\input{cdocspt3}
\section{part four}
\input{cdocspt4}
\fi
%    \end{macrocode}

% Process as usual until here:
%    \begin{macrocode}
\fi
%    \end{macrocode}

% End of document body:
%    \begin{macrocode}
\end{document}
%    \end{macrocode}
%\iffalse
%</samplemain>
%\fi
%
% %%%%%%%%%%%%%%%%%%%%%%%%%%%%%%%%%%%%%%
% \paragraph{Chapter Include Files.}
%
% The include files are called |cdocsch1.tex| and |cdocsch2.tex|.
%
%\iffalse
%<*samplechap1|samplechap2>
%\fi

% Optional override for |\version| flag:
%    \begin{macrocode}
%%\providecommand{\version}{final}
%    \end{macrocode}

% Include the main document:
%    \begin{macrocode}
% \iffalse
%
% childdoc.dtx Copyright (C) 2017-2018 Niklas Beisert
%
% This work may be distributed and/or modified under the
% conditions of the LaTeX Project Public License, either version 1.3
% of this license or (at your option) any later version.
% The latest version of this license is in
%   http://www.latex-project.org/lppl.txt
% and version 1.3 or later is part of all distributions of LaTeX
% version 2005/12/01 or later.
%
% This work has the LPPL maintenance status `maintained'.
%
% The Current Maintainer of this work is Niklas Beisert.
%
% This work consists of the files childdoc.dtx and childdoc.ins
% and the derived files childdoc.def and cdocsamp.tex with
% cdocsch1.tex, cdocsch2.tex, cdocsdrf.tex, cdocsfn1.tex, cdocsfn2.tex.
%
%<package>\ifdefined\childdocmain\endinput\fi
%<package>\ProvidesFile{childdoc.def}[2018/12/30 v2.0 child document driver]
%<samplemain>\ProvidesFile{cdocsamp.tex}[2018/12/30 v2.0 sample for childdoc]
%<*driver>
%\ProvidesFile{childdoc.drv}[2018/12/30 v2.0 childdoc reference manual file]
\PassOptionsToClass{10pt,a4paper}{article}
\documentclass{ltxdoc}

\usepackage[margin=35mm]{geometry}
\usepackage{hyperref}
\usepackage{hyperxmp}
\usepackage[usenames]{color}

\hypersetup{colorlinks=true}
\hypersetup{pdfstartview=FitH}
\hypersetup{pdfpagemode=UseNone}
\hypersetup{pdfsource={}}
\hypersetup{pdflang={en-UK}}
\hypersetup{pdfcopyright={Copyright 2017-2018 Niklas Beisert.
  This work may be distributed and/or modified under the
  conditions of the LaTeX Project Public License, either version 1.3
  of this license or (at your option) any later version.}}
\hypersetup{pdflicenseurl={http://www.latex-project.org/lppl.txt}}
\hypersetup{pdfcontactaddress={ETH Zurich, ITP, HIT K,
  Wolfgang-Pauli-Strasse 27}}
\hypersetup{pdfcontactpostcode={8093}}
\hypersetup{pdfcontactcity={Zurich}}
\hypersetup{pdfcontactcountry={Switzerland}}
\hypersetup{pdfcontactemail={nbeisert@itp.phys.ethz.ch}}
\hypersetup{pdfcontacturl={http://people.phys.ethz.ch/\xmptilde nbeisert/}}

\newcommand{\secref}[1]{\hyperref[#1]{section \ref*{#1}}}

\parskip1ex
\parindent0pt
\let\olditemize\itemize
\def\itemize{\olditemize\parskip0pt}

\begin{document}

\title{The \textsf{childdoc} Package}
\hypersetup{pdftitle={The childdoc Package}}
\author{Niklas Beisert\\[2ex]
  Institut f\"ur Theoretische Physik\\
  Eidgen\"ossische Technische Hochschule Z\"urich\\
  Wolfgang-Pauli-Strasse 27, 8093 Z\"urich, Switzerland\\[1ex]
  \href{mailto:nbeisert@itp.phys.ethz.ch}
  {\texttt{nbeisert@itp.phys.ethz.ch}}}
\hypersetup{pdfauthor={Niklas Beisert}}
\hypersetup{pdfsubject={Manual for the LaTeX2e Package childdoc}}
\date{30 December 2018, \textsf{v2.0}}
\maketitle

\begin{abstract}\noindent
\textsf{childdoc} is a \LaTeXe{} package
that enables the direct compilation
of document sections included by |\include|
to individual files.
\end{abstract}

\begingroup
\parskip0ex
\tableofcontents
\endgroup

%%%%%%%%%%%%%%%%%%%%%%%%%%%%%%%%%%%%%%%%%%%%%%%%%%%%%%%%%%%%%%%%%%%%%%%%%%%%%%%%
%%%%%%%%%%%%%%%%%%%%%%%%%%%%%%%%%%%%%%%%%%%%%%%%%%%%%%%%%%%%%%%%%%%%%%%%%%%%%%%%
\section{Introduction}

\LaTeX{} provides a mechanism to structure a large document (such as a book)
into a main file and several child files (containing the chapters)
using the |\include| command.
This mechanism is beneficial for documents
which span hundreds of pages in order to
make the source file(s) more manageable.
Moreover, compilation can be restricted to
selected child files by means of the |\includeonly| command.
The latter feature can be used to reduce the compilation time while editing
(this was significantly more useful in the earlier days of \LaTeX{})
or to generate a smaller document which is easier to navigate.
Another application of |\includeonly| is to generate
documents consisting of selected parts of the complete document.

However, there are a few drawbacks of the plain |\include| mechanism:
\begin{itemize}
\item
The child files cannot be compiled on their own,
they can only be compiled via the main file.
A naive editing environment
(such as a text editor with an option
to have the current file processed by \LaTeX)
may require one to switch to the main file before compiling;
attempting to compile the child file produces errors.
\item
The main file must be modified (each time)
to adjust the |\includeonly| command
to the present needs. This easily leaves the main file in a messy state.
\item
The generated document will always carry the filename
of the main document. This is inconvenient if
several child files are to be compiled and
to be kept for distribution.
\end{itemize}

The present package provides a simple interface
to make child files individually compilable by \LaTeX{}.
Compiling a child file then has the same effect as compiling
the main file with an |\includeonly| command
to select the appropriate child.
Moreover the generated document will carry the name of the child
rather than the main file.
This resolves all three above issues.

This feature is meant to make the editing of books,
thesis documents and lecture notes somewhat more convenient.
However, the package can also be used efficiently for
composing a series of documents (such as exercise sheets)
which are typically distributed individually.
It then assists the author in generating the individual documents
(potentially in different versions)
as well as a document containing the collected series.
Another application is in developing style files
or other kinds of included material
where compilation of the style file could redirect
to a sample or test file.

%%%%%%%%%%%%%%%%%%%%%%%%%%%%%%%%%%%%%%%%%%%%%%%%%%%%%%%%%%%%%%%%%%%%%%%%%%%%%%%%
%%%%%%%%%%%%%%%%%%%%%%%%%%%%%%%%%%%%%%%%%%%%%%%%%%%%%%%%%%%%%%%%%%%%%%%%%%%%%%%%
\section{Usage}

First of all, the package \textsf{childdoc} is \emph{not} a standard
\LaTeXe{} |.sty| style file! Therefore it needs to be invoked in
a non-standard way.

%%%%%%%%%%%%%%%%%%%%%%%%%%%%%%%%%%%%%%%%%%%%%%%%%%%%%%%%%%%%%%%%%%%%%%%%%%%%%%%%
\subsection{Included Files}
\label{sec:include}

%%%%%%%%%%%%%%%%%%%%%%%%%%%%%%%%%%%%%%%%
\DescribeMacro{\childdocmain}
To use the package, add the commands
\begin{center}
\begin{tabular}{l}
|\input{childdoc.def}|\\
|\childdocmain{}|\\
\end{tabular}
\end{center}
at the very top of the main \LaTeX{} file,
in particular \emph{before} the |\documentclass| statement!
The argument of |\childdocmain| should be left empty
(but it must be present).

%%%%%%%%%%%%%%%%%%%%%%%%%%%%%%%%%%%%%%%%
\DescribeMacro{\childdocof}
Furthermore, add the commands
\begin{center}
\begin{tabular}{l}
|\input{childdoc.def}|\\
|\childdocof{|\textit{main}|}|\\
\end{tabular}
\end{center}
at the top of every child file \textit{child}
which is included by |\include{|\textit{child}|}|
from within the main file
(or at least for those files to be compiled individually).
The argument \textit{main} must be the filename of the main file.

There are a couple of
considerations in setting up the main and child documents:

%%%%%%%%%%%%%%%%%%%%%%%%%%%%%%%%%%%%%%%%
\paragraph{Restrictions.}

Please note the following restrictions:
\begin{itemize}
\item
|\childdocmain| must be called with one argument \textit{main}
to ensure compatibility with earlier version of the package.
It must either be empty (|\childdocmain{}|)
or precisely match the filename of the main file in which it is specified.
See \secref{sec:detection} for further information.
\item
The filename \textit{main} must be specified without the |.tex| extension.
\item
The filename \textit{main} is case sensitive
(even in case-insensitive file systems)
due to internal string comparison.
\item
The argument \textit{main} should be fully expanded, it cannot be a macro.
\item
Subdirectories and special characters should be avoided in filenames.
\item
The command |\childdocmain{|\textit{main}|}| must be followed by a whitespace.
It should not be followed immediately by another command
or by a comment mark `|%|'.
This is because the \TeX{} parser reads the token immediately following
the argument of |\childdocmain| and puts it
at the beginning of every child section;
however, a white\-space is ignored.
\end{itemize}

%%%%%%%%%%%%%%%%%%%%%%%%%%%%%%%%%%%%%%%%
\paragraph{Content of Main File.}

It is advisable to place all content in the child files included by |\include|.
Any output contained in the main file will appear in all child documents
unless suppressed manually;
it cannot be suppressed automatically by the |\includeonly| directive
and thus should normally be avoided.
A method to include some content in the main file
by means of conditional processing is described in \secref{sec:conditional}.

%%%%%%%%%%%%%%%%%%%%%%%%%%%%%%%%%%%%%%%%
\paragraph{Page Numbering.}

When only a part of the document is compiled,
the appropriate numbering of pages
(as well as other status parameters)
is determined from the |.aux| files.
The latter contain information from previous passes.
However this information needs to propagate through
all intermediate child documents.
Therefore the page numbering in child documents may well
be inconsistent until the complete document is compiled at least once.

A useful (if unconventional) way to always ensure a consistent
page numbering is to restart the numbering in each child document
and denote the pages by `\textit{child}|.|\textit{page}'
where \textit{child} represents the chapter/section number of the child file.
This can be achieved by the command
|\numberwithin{page}{|\textit{child}|}|
of the \textsf{amsmath} package
where \textit{child} can be |chapter| or |section|
depending on the chosen structuring.
Alternatively, one can modify the macro |\thepage| appropriately
and reset the counter |page| at the start of each child file.

%%%%%%%%%%%%%%%%%%%%%%%%%%%%%%%%%%%%%%%%%%%%%%%%%%%%%%%%%%%%%%%%%%%%%%%%%%%%%%%%
\subsection{Conditional Processing}
\label{sec:conditional}

The package provides a mechanism to compile different versions
of a document. To customise the versions further some conditional processing
can come in handy to distinguish which version is being compiled.
The package provides two macros to describe the compilation context:

%%%%%%%%%%%%%%%%%%%%%%%%%%%%%%%%%%%%%%%%
\DescribeMacro{\ifchilddoc}
The conditional |\ifchilddoc| distinguishes between the compilation of
child documents and the main document:
%
\begin{center}
|\ifchilddoc |\textit{child-code}| |[|\||else |\textit{main-code}]| \||fi|
\end{center}

%%%%%%%%%%%%%%%%%%%%%%%%%%%%%%%%%%%%%%%%
\DescribeMacro{\childdocname}
\DescribeMacro{\childdocjob}
The macro |\childdocname| contains the filename (without extension)
of the main or child file being processed.
Note that |\childdocjob| will always contain the name of the main file.

%%%%%%%%%%%%%%%%%%%%%%%%%%%%%%%%%%%%%%%%
\paragraph{Title Page.}

Conditional processing can be used to include a title or banner page
in the main document when proper precautions are taken.
Importantly, the code in the main file should ensure that the page counter
(as well as other status parameters which are stored in the |.aux| files)
takes the same value after the conditional processing.
Otherwise the page numbers may take divergent values
depending on which part is compiled.

For example, a title page could be declared by:
%
\begin{center}
\begin{tabular}{l}
|\ifchilddoc\||else|\\
|\addtocounter{page}{-1}|\\
\textit{code for title page}\\
|\newpage|\\
|\||fi|
\end{tabular}
\end{center}
%
A banner page for the child documents can be generated by:
%
\begin{center}
\begin{tabular}{l}
|\ifchilddoc|\\
|\addtocounter{page}{-1}|\\
\textit{code for banner page}\\
|\newpage|\\
|\||fi|
\end{tabular}
\end{center}
%
Here one could write a message such as:
\begin{center}
|This is the part \childdocname{} of \childdocjob{}.|
\end{center}

%%%%%%%%%%%%%%%%%%%%%%%%%%%%%%%%%%%%%%%%%%%%%%%%%%%%%%%%%%%%%%%%%%%%%%%%%%%%%%%%
\subsection{Flags}
\label{sec:flags}

The package makes it easy to generate different versions
of the main or child documents.
To this end compilation flags can be defined
and assigned different default values.
They will be particularly useful in conjunction
with the forwarding mechanism described in \secref{sec:forward}.

For example, it may be useful to have a flag |\version|
which can be set to |draft| or |final|.
The document source will contain some conditional code
depending on the value of |\version|.
Suppose further, the flag should default to |final| for the main file
and to |draft| for child files
which is a natural assignment for editing the document.
This is achieved by placing the following code
in the preamble of the main document
(below the |\childdocmain| directive):
%
\begin{center}
\begin{tabular}{l}
|\ifchilddoc|\\
|\providecommand{\version}{draft}|\\
|\||else|\\
|\providecommand{\version}{final}|\\
|\||fi|
\end{tabular}
\end{center}
%
The definition by |\providecommand| makes sure
that previous definitions are not overwritten.
Further statements |\providecommand{\version}{...}|
can thus be added before the above code to override it.

For the main file, one might add a line
(between |\childdocmain| and the above block)
%
\begin{center}
|%\ifchilddoc\||else\providecommand{\version}{draft}\||fi|
\end{center}
%
which can be uncommented to produce a draft version.
Likewise one can add a line to the very top of a child file
(above the |\childdocof{|\textit{main}|}| directive)
%
\begin{center}
|%\providecommand{\version}{final}|
\end{center}
%
which can be uncommented to produce the final version of this child document.

%%%%%%%%%%%%%%%%%%%%%%%%%%%%%%%%%%%%%%%%%%%%%%%%%%%%%%%%%%%%%%%%%%%%%%%%%%%%%%%%
\subsection{Forwarding}
\label{sec:forward}

Different versions of the main or child documents
using compilation flags as described in \secref{sec:flags}
can be (permanently) stored in different files
for convenient compilation, viewing and distribution.
To this end, the package defines a command
to pass on compilation to a different file:

%%%%%%%%%%%%%%%%%%%%%%%%%%%%%%%%%%%%%%%%
\DescribeMacro{\childdocforward}
The command |\childdocforward| redirects processing to
another source file:
%
\begin{center}
\begin{tabular}{l}
|\input{childdoc.def}|\\
|\childdocforward[|\textit{main}|]{|\textit{dest}|}|\\
\end{tabular}
\end{center}
%
The argument \textit{dest} is the destination file
(without extension).
It should be the main file or one of the child files.
Note that further \textsf{childdoc} directives
such as |\childdocof| and |\childdocforward|
in the indicated file will be processed in this form.
The optional argument \textit{main}
passes on directly to the main file \textit{main}
while pretending to compile the child \textit{dest}.
This form behaves as if \textit{dest}
issues |\childdocof{|\textit{main}|}| right away,
and no further \textsf{childdoc} directives will be processed.

%%%%%%%%%%%%%%%%%%%%%%%%%%%%%%%%%%%%%%%%
\DescribeMacro{\...prefix}
In the alternative form |\childdocforwardprefix|,
%
\begin{center}
\begin{tabular}{l}
|\input{childdoc.def}|\\
|\childdocforwardprefix[|\textit{main}|]{|\textit{prefix}|}{|\textit{dest}|}|
\end{tabular}
\end{center}
%
the destination file is determined by a pattern
depending on the current file:
To make this work, the current file must be called
`{\textit{prefix}\hspace{0.2em}\textit{suffix}}'
with \textit{prefix} matching precisely the argument.
Processing is then passed on to the file
`{\textit{dest}\hspace{0.2em}\textit{suffix}}'.
Surely, the same effect is achieved by
directly specifying the
argument `{\textit{dest}\hspace{0.2em}\textit{suffix}}'
in the first form.
However, that requires to set up a different file
for each child. With the alternative form of the command
all these files can have exactly the same content
which simplifies setting them up and maintaining them.

For example, the following file |draft.tex|
with a compilation flag |\version| as described in \secref{sec:flags}
compiles the main document as a draft:
%
\begin{center}
\begin{tabular}{l}
|\def\version{draft}|\\
|\input{childdoc.def}|\\
|\childdocforward{|\textit{main}|}|
\end{tabular}
\end{center}
%
Likewise, the following files |final|\textit{nn}|.tex|
compile the final version of the child document
|child|\textit{nn}|.tex|:
%
\begin{center}
\begin{tabular}{l}
|\def\version{final}|\\
|\input{childdoc.def}|\\
|\childdocforwardprefix{final}{child}|
\end{tabular}
\end{center}
%

Note that when several versions of a main file and/or of each child file
are to be generated, it may be convenient to set up a |Makefile| or
shell script to automatise the process.

%%%%%%%%%%%%%%%%%%%%%%%%%%%%%%%%%%%%%%%%%%%%%%%%%%%%%%%%%%%%%%%%%%%%%%%%%%%%%%%%
\subsection{Command Line Processing}
\label{sec:commandline}

The effect of redirection files can also be achieved by invoking
the \LaTeX{} compiler with a more elaborate command line.
Most conveniently this should be done as part
of a shell script or a |Makefile|.

When using \textsf{childdoc} in the main file, the following
command lines effectively perform a redirection
(note that depending on the shell being used,
backslashes may have to be doubled: `|\|' $\to$ `|\\|'):
%
\begin{center}
|... -jobname "|\textit{target}|" |\\|"|[\textit{flags}]%
|\input{childdoc.def}\childdocforward[|\textit{main}|]{|\textit{dest}|}"|
\end{center}
%
Here \textit{target} is the name of the output file,
\textit{main} is the name of the main file
and \textit{dest} is the name of the main or child file to be processed
(all filenames without extensions).
The optional argument \textit{main} can be omitted
if \textit{main} matches \textit{dest}.
Optionally, compilation \textit{flags} can be defined via |\def| commands.
This command line makes the \TeX{} engine believe
it is compiling the file \textit{target}
whose content is specified as the latter parameter.
The provided code then forwards the processing to
\textit{main} or \textit{dest} as described in \secref{sec:forward}.

%%%%%%%%%%%%%%%%%%%%%%%%%%%%%%%%%%%%%%%%%%%%%%%%%%%%%%%%%%%%%%%%%%%%%%%%%%%%%%%%
\subsection{Include by Input}
\label{sec:input}

Including child documents by |\include| has some restrictions by design.
Most notably, the content of a child document always occupies
its own set of pages; pages cannot be shared between child documents.
Usually, this behaviour makes perfect sense
because each child document contain an essential part of the document.
However, in some situations it may be desirable to compose
a document from a collection of parts
without having mandatory page breaks between then.
For this case, the package
provides a mechanism to include parts
by |\input| which can also be processed individually.
However, by construction this mechanism
requires manual handling of the content to be output.

%%%%%%%%%%%%%%%%%%%%%%%%%%%%%%%%%%%%%%%%
\DescribeMacro{\ifchilddocmanual}
The main file should be prepared as usual, see \secref{sec:include}.
However, the document body must make a distinction
between processing of an individual part and of the main document, e.g.:
%
\begin{center}
\begin{tabular}{l}
|\ifchilddocmanual|\\
|\input{\childdocname}|\\
|\||else|\\
\textit{document body with }|\input{|\textit{part}|}|\\
|\||fi|
\end{tabular}
\end{center}
%
The conditional |\ifchilddocmanual| is true whenever
a part to be included by |\input| is being compiled,
and the name of the part is stored in |\childdocname|.

%%%%%%%%%%%%%%%%%%%%%%%%%%%%%%%%%%%%%%%%
\DescribeMacro{\childdocby}
Each part to be included by |\input| should start with:
%
\begin{center}
\begin{tabular}{l}
|\input{childdoc.def}|\\
|\childdocby{|\textit{main}|}|\\
\end{tabular}
\end{center}
%
The directive |\childdocby| is similar to |\childdocof|
described in \secref{sec:include},
but the subsequent selection of content must be done manually.
To that end, both |\ifchilddoc| and |\ifchilddocmanual|
will be true upon processing of a part,
and the name of the part is stored in |\childdocname|.
Note that |\jobname| will be set to the filename of the current part
so that each part receives an individual |.aux| file
that does not interfere with the |.aux| file(s) of the main document.
This behaviour can be altered by the alternative form
|\childdocby[*]{|\textit{main}|}| (with a non-empty optional argument)
which uses the |.aux| file of the main document
by setting |\jobname| to \textit{main}.

%%%%%%%%%%%%%%%%%%%%%%%%%%%%%%%%%%%%%%%%%%%%%%%%%%%%%%%%%%%%%%%%%%%%%%%%%%%%%%%%
\subsection{Driver Development}
\label{sec:driver}

The \textsf{childdoc} mechanism can also be use for the development
of definition files such as \LaTeX{} styles or classes.
This case differs from the above setup with multiple parts
included by |\include| in that no |\includeonly| should be invoked.
This can be achieved by starting the include file
(before |\ProvidesPackage|) with:
%
\begin{center}
\begin{tabular}{l}
|\input{childdoc.def}|\\
|\childdocforward{|\textit{main}|}|\\
\end{tabular}
\end{center}
%
or alternatively with:
%
\begin{center}
\begin{tabular}{l}
|\input{childdoc.def}|\\
|\childdocby{|\textit{main}|}|\\
\end{tabular}
\end{center}
%
Both forms have slightly different effects as described above.
The main file is prepared as usual, see \secref{sec:include}.

%%%%%%%%%%%%%%%%%%%%%%%%%%%%%%%%%%%%%%%%%%%%%%%%%%%%%%%%%%%%%%%%%%%%%%%%%%%%%%%%
\subsection{Legacy Detection}
\label{sec:detection}

The directive |\childdocmain| in the main file can detect
whether the complete document or merely a child is to be compiled
even without using the directive |\childdocof|.
This method is deprecated because it is less robust
and there is no compelling reason to use it;
it is merely provided for backward compatibility
and it may be removed in future versions.

If the detection mechanism is to be used,
it is mandatory to correctly specify
the filename of the main file as the argument of |\childdocmain|:
%
\begin{center}
\begin{tabular}{l}
|\input{childdoc.def}|\\
|\childdocmain{|\textit{main}|}|\\
\end{tabular}
\end{center}
%
If |\jobname| does not match the argument \textit{main} of |\childdocmain|,
it is assumed that |\jobname| points to the child file to be compiled.
When using |\childdocmain| with the main file specified as argument,
it suffices to start a child file
with just |\input{|\textit{main}|}|
without loading of the package and using |\childdocof|.
If instead all processing is done
with the appropriate \textsf{childdoc} directives,
the argument of \textit{main} of |\childdocmain| can be empty.

An alternative version of the command line processing described
in \secref{sec:commandline} using the detection mechanism reads:
%
\begin{center}
|... -jobname "|\textit{target}|" "|[\textit{flags}]%
[|\def\jobname{|\textit{dest}|}|]|\input{|\textit{main}|}"|
\end{center}

%%%%%%%%%%%%%%%%%%%%%%%%%%%%%%%%%%%%%%%%%%%%%%%%%%%%%%%%%%%%%%%%%%%%%%%%%%%%%%%%
\subsection{Manual Code}
\label{sec:manual}

In case one cannot be certain whether the definitions file |childdoc.def|
is installed on the target \TeX{} distribution
and one prefers not to ship it,
it is conceivable to paste a few relevant commands into the sources.

To that end, drop all statements |\input{childdoc.def}|
and perform the replacements as outlined below.
Instead of |\childdocmain{|\textit{main}|}| add the following code
to the top of the main file:
%
\begin{center}
\begin{tabular}{l}
|\||ifdefined\childdocname\endinput\||fi\newif\ifchilddoc|\\
|\edef\childdocname{\scantokens\expandafter{\jobname\noexpand}}|\\
|\def\childdocmain{|\textit{main}|}\||ifx\childdocmain\childdocname\||else|\\
|\childdoctrue\includeonly{\childdocname}\let\jobname\childdocmain\||fi|\\
\end{tabular}
\end{center}
%
Instead of |\childdocof{|\textit{main}|}| just include the main file
at the top of each child file:
%
\begin{center}
|\input{|\textit{main}|}|
\end{center}
%
A simple redirection |\childdocforward{|\textit{dest}|}| is achieved by:
%
\begin{center}
|\def\jobname{|\textit{dest}|}\input{\jobname}|
\end{center}
%
The redirection with prefix
|\childdocforwardprefix[|\textit{prefix}|]{|\textit{dest}|}|
is accomplished by:
%
\begin{center}
\begin{tabular}{l}
|{\edef\jobname{\scantokens\expandafter{\jobname\noexpand}}|\\
|\def\redirectjob |\textit{prefix}|#1~~~{\gdef\jobname{|\textit{dest}|#1}}|\\
|\expandafter\redirectjob\jobname~~~}\input{\jobname}|
\end{tabular}
\end{center}

In an alternative approach,
child documents can be compiled by a specific command line
without additional code or specific definitions:
%
\begin{center}
|... -jobname "|\textit{target}|" "|[\textit{flags}]%
|\includeonly{|\textit{dest}|}\input{|\textit{main}|}"|
\end{center}
%

%%%%%%%%%%%%%%%%%%%%%%%%%%%%%%%%%%%%%%%%%%%%%%%%%%%%%%%%%%%%%%%%%%%%%%%%%%%%%%%%
%%%%%%%%%%%%%%%%%%%%%%%%%%%%%%%%%%%%%%%%%%%%%%%%%%%%%%%%%%%%%%%%%%%%%%%%%%%%%%%%
\section{Information}

%%%%%%%%%%%%%%%%%%%%%%%%%%%%%%%%%%%%%%%%%%%%%%%%%%%%%%%%%%%%%%%%%%%%%%%%%%%%%%%%
\subsection{Copyright}

Copyright \copyright{} 2017--2018 Niklas Beisert

This work may be distributed and/or modified under the
conditions of the \LaTeX{} Project Public License, either version 1.3
of this license or (at your option) any later version.
The latest version of this license is in
  \url{http://www.latex-project.org/lppl.txt}
and version 1.3 or later is part of all distributions of \LaTeX{}
version 2005/12/01 or later.

This work has the LPPL maintenance status `maintained'.

The Current Maintainer of this work is Niklas Beisert.

This work consists of the files |README.txt|, |childdoc.ins| and |childdoc.dtx|
as well as the derived files |childdoc.def|, |cdocsamp.tex|
with |cdocsch1.tex|, |cdocsch2.tex|, |cdocspt3.tex|, |cdocspt4.tex|,
|cdocsdrf.tex|, |cdocsfn1.tex|, |cdocsfn2.tex|
as well as |childdoc.pdf|.

%%%%%%%%%%%%%%%%%%%%%%%%%%%%%%%%%%%%%%%%%%%%%%%%%%%%%%%%%%%%%%%%%%%%%%%%%%%%%%%%
\subsection{Files and Installation}

The package consists of the files:
%
\begin{center}
\begin{tabular}{ll}
    |README.txt|   & readme file \\
    |childdoc.ins| & installation file \\
    |childdoc.dtx| & source file \\
    |childdoc.def| & definition file \\
    |cdocsamp.tex| & sample main file \\
    |cdocsch1.tex| & sample include file \\
    |cdocsch2.tex| & sample include file \\
    |cdocspt3.tex| & sample part file \\
    |cdocspt4.tex| & sample part file \\
    |cdocsdrf.tex| & sample redirection file \\
    |cdocsfn1.tex| & sample redirection file \\
    |cdocsfn2.tex| & sample redirection file \\
    |childdoc.pdf| & manual
\end{tabular}
\end{center}
%
The distribution consists of the files
|README.txt|, |childdoc.ins| and |childdoc.dtx|.
%
\begin{itemize}
\item
Run (pdf)\LaTeX{} on |childdoc.dtx|
to compile the manual |childdoc.pdf| (this file).
\item
Run \LaTeX{} on |childdoc.ins| to create the definitions file |childdoc.def|
and the sample |cdocsamp.tex| with include files
|cdocsch1.tex|, |cdocsch2.tex|, |cdocspt3.tex|, |cdocspt4.tex|,
|cdocsdrf.tex|, |cdocsfn1.tex|, |cdocsfn2.tex|.
Then copy the file |childdoc.def| to an appropriate directory of your \LaTeX{}
distribution, e.g.\ \textit{texmf-root}|/tex/latex/childdoc|.
\end{itemize}

%%%%%%%%%%%%%%%%%%%%%%%%%%%%%%%%%%%%%%%%%%%%%%%%%%%%%%%%%%%%%%%%%%%%%%%%%%%%%%%%
\subsection{Related CTAN Packages}

There are several other packages which offer a similar functionality:
%
\begin{itemize}
\item
The packages
\href{http://ctan.org/pkg/docmute}{\textsf{docmute}},
\href{http://ctan.org/pkg/includex}{\textsf{includex}} and
\href{http://ctan.org/pkg/standalone}{\textsf{standalone}}
provide commands to include only the document body of
a child file thus allowing both files to be compiled individually.
\item
The packages \href{http://ctan.org/pkg/subdocs}{\textsf{subdocs}}
and \href{http://ctan.org/pkg/subfiles}{\textsf{subfiles}}
provide structures in which the main and child documents can be
encapsulated and allowing them to be compiled individually.
The inclusion mechanism is different from the conventional |\include|.
\item
The package \href{http://ctan.org/pkg/combine}{\textsf{combine}}
is an elaborate solution to combine several documents into one.
\end{itemize}
%
See also the CTAN topic \href{http://ctan.org/topic/subdocs}{\textsf{subdocs}}
for further related packages.
The present package differs from the above solutions in that
a document structure constructed with the conventional |\include| mechanism
just needs two extra commands at the top of every file
such that all constituent files can be compiled individually.

%%%%%%%%%%%%%%%%%%%%%%%%%%%%%%%%%%%%%%%%%%%%%%%%%%%%%%%%%%%%%%%%%%%%%%%%%%%%%%%%
%\subsection{Feature Suggestions}
%
%The following is a list of features which may be useful for future
%versions of this package:
%%
%\begin{itemize}
%\item
%\ldots
%\end{itemize}

%%%%%%%%%%%%%%%%%%%%%%%%%%%%%%%%%%%%%%%%%%%%%%%%%%%%%%%%%%%%%%%%%%%%%%%%%%%%%%%%
\subsection{Revision History}

%%%%%%%%%%%%%%%%%%%%%%%%%%%%%%%%%%%%%%%%
\paragraph{v2.0:} 2018/12/30

\begin{itemize}
\item
immediate forward processing
\item
added |\childdocby| mechanism
\item
manual restructured
\end{itemize}

%%%%%%%%%%%%%%%%%%%%%%%%%%%%%%%%%%%%%%%%
\paragraph{v1.6:} 2018/01/17

\begin{itemize}
\item
application for development of include files
\item
corrections to manual
\end{itemize}

%%%%%%%%%%%%%%%%%%%%%%%%%%%%%%%%%%%%%%%%
\paragraph{v1.5:} 2017/05/21

\begin{itemize}
\item
more complete structuring introduced
\item
|\childdocof| introduced
\item
|\childdoc| renamed to |\childdocmain|
\item
|\childredirect| renamed to |\childdocforward| and |\childdocforwardprefix|
and functionality expanded
\end{itemize}

%%%%%%%%%%%%%%%%%%%%%%%%%%%%%%%%%%%%%%%%
\paragraph{v1.0:} 2017/04/27

\begin{itemize}
\item
manual and install package
\item
first version published on CTAN
\end{itemize}

%%%%%%%%%%%%%%%%%%%%%%%%%%%%%%%%%%%%%%%%
\paragraph{v0.6:} 2017/04/26

\begin{itemize}
\item
redirection mechanism added
\end{itemize}

%%%%%%%%%%%%%%%%%%%%%%%%%%%%%%%%%%%%%%%%
\paragraph{v0.5:} 2017/04/26

\begin{itemize}
\item
functionality in definition file
\end{itemize}


%%%%%%%%%%%%%%%%%%%%%%%%%%%%%%%%%%%%%%%%%%%%%%%%%%%%%%%%%%%%%%%%%%%%%%%%%%%%%%%%
%%%%%%%%%%%%%%%%%%%%%%%%%%%%%%%%%%%%%%%%%%%%%%%%%%%%%%%%%%%%%%%%%%%%%%%%%%%%%%%%
%%%%%%%%%%%%%%%%%%%%%%%%%%%%%%%%%%%%%%%%%%%%%%%%%%%%%%%%%%%%%%%%%%%%%%%%%%%%%%%%
\appendix

\settowidth\MacroIndent{\rmfamily\scriptsize 000\ }

 \DocInput{childdoc.dtx}

\end{document}
%</driver>
% \fi
%
% %%%%%%%%%%%%%%%%%%%%%%%%%%%%%%%%%%%%%%%%%%%%%%%%%%%%%%%%%%%%%%%%%%%%%%%%%%%%%%
% %%%%%%%%%%%%%%%%%%%%%%%%%%%%%%%%%%%%%%%%%%%%%%%%%%%%%%%%%%%%%%%%%%%%%%%%%%%%%%
% \section{Sample}
%\iffalse
%<*samplemain>
%\fi
%
% The following presents a sample document
% with two chapters, two parts, a title page,
% a compile flag as well as three forwarding files to set the flag.
% It consists of eight |.tex| files:
% \begin{center}
% \begin{tabular}{ll}
% |cdocsamp.tex|&main file\\
% |cdocsch1.tex|&include file for chapter 1\\
% |cdocsch2.tex|&include file for chapter 2\\
% |cdocspt3.tex|&include file for part 3\\
% |cdocspt4.tex|&include file for part 4\\
% |cdocsdrf.tex|&forwarding file for main file in draft mode\\
% |cdocsfi1.tex|&forwarding file for final version of chapter 1\\
% |cdocsfi2.tex|&forwarding file for final version of chapter 2\\
% \end{tabular}
% \end{center}
% Each of the eight files can be compiled directly by the \LaTeX{} compiler.
%
% %%%%%%%%%%%%%%%%%%%%%%%%%%%%%%%%%%%%%%
% \paragraph{Main File.}
%
% The main file is called |cdocsamp.tex|.
%
% Load the \textsf{childdoc} definitions and
% declare the filename for the main document:
%    \begin{macrocode}
\input{childdoc.def}
\childdocmain{}
%    \end{macrocode}

% Optional override for |\version| flag:
%    \begin{macrocode}
%%\ifchilddoc\else\providecommand{\version}{draft}\fi
%    \end{macrocode}

% Define the default values for the |\version| flag
% (|final| for the main file and |draft| for childs):
%    \begin{macrocode}
\ifchilddoc
\providecommand{\version}{draft}
\else
\providecommand{\version}{final}
\fi
%    \end{macrocode}

% Load the standard document class:
%    \begin{macrocode}
\documentclass[12pt]{article}
%    \end{macrocode}

% Start the document body:
%    \begin{macrocode}
\begin{document}
%    \end{macrocode}

% Declare a title page.
% Print title, part of document being processed and version flag:
%    \begin{macrocode}
\addtocounter{page}{-1}
\begin{center}
{\LARGE\bfseries{}childdoc example\par}
\vspace{1cm}
\ifchilddoc
\ifchilddocmanual part\else chapter\fi:
`\childdocname' of `\childdocjob'\par
\else
main document: `\childdocjob'\par
\fi
version: \version\par
\end{center}
\newpage
%    \end{macrocode}

% Manually include selected file,
% otherwise process as usual:
%    \begin{macrocode}
\ifchilddocmanual
\section*{part `\childdocname'}
\input{\childdocname}
\else
%    \end{macrocode}

% Include the two chapters:
%    \begin{macrocode}
\include{cdocsch1}
\include{cdocsch2}
%    \end{macrocode}

% Include the two parts unless only chapters should be displayed:
%    \begin{macrocode}
\ifchilddoc\else
\section{part three}
\input{cdocspt3}
\section{part four}
\input{cdocspt4}
\fi
%    \end{macrocode}

% Process as usual until here:
%    \begin{macrocode}
\fi
%    \end{macrocode}

% End of document body:
%    \begin{macrocode}
\end{document}
%    \end{macrocode}
%\iffalse
%</samplemain>
%\fi
%
% %%%%%%%%%%%%%%%%%%%%%%%%%%%%%%%%%%%%%%
% \paragraph{Chapter Include Files.}
%
% The include files are called |cdocsch1.tex| and |cdocsch2.tex|.
%
%\iffalse
%<*samplechap1|samplechap2>
%\fi

% Optional override for |\version| flag:
%    \begin{macrocode}
%%\providecommand{\version}{final}
%    \end{macrocode}

% Include the main document:
%    \begin{macrocode}
\input{childdoc.def}
\childdocof{cdocsamp}
%    \end{macrocode}

%\iffalse
%</samplechap1|samplechap2>
%\fi
%
%\iffalse
%<*samplechap1>
%\fi
% Some text for chapter 1:
%    \begin{macrocode}
\section{one}
some text in chapter one
%    \end{macrocode}

%\iffalse
%</samplechap1>
%\fi
% Some text for chapter 2:
%\iffalse
%<*samplechap2>
%\fi
%    \begin{macrocode}
\section{two}
more text in chapter two
%    \end{macrocode}

%\iffalse
%</samplechap2>
%\fi
%
% %%%%%%%%%%%%%%%%%%%%%%%%%%%%%%%%%%%%%%
% \paragraph{Part Include Files.}
%
% The include files are called |cdocspt3.tex| and |cdocspt4.tex|.
%
%\iffalse
%<*samplepart3|samplepart4>
%\fi

% Optional override for |\version| flag:
%    \begin{macrocode}
%%\providecommand{\version}{final}
%    \end{macrocode}

% Include the main document:
%    \begin{macrocode}
\input{childdoc.def}
\childdocby{cdocsamp}
%    \end{macrocode}

%\iffalse
%</samplepart3|samplepart4>
%\fi
%
%\iffalse
%<*samplepart3>
%\fi
% Some text for part 3:
%    \begin{macrocode}
some text in part three
%    \end{macrocode}

%\iffalse
%</samplepart3>
%\fi
% Some text for part 4:
%\iffalse
%<*samplepart4>
%\fi
%    \begin{macrocode}
more text in part four
%    \end{macrocode}

%\iffalse
%</samplepart4>
%\fi
%
% %%%%%%%%%%%%%%%%%%%%%%%%%%%%%%%%%%%%%%
% \paragraph{Forwarding for a Complete Draft.}
%
% The following forwarding file |cdocsdrf.tex|
% compiles the main document in draft mode:
%\iffalse
%<*sampledraft>
%\fi
%    \begin{macrocode}
\def\version{draft}
\input{childdoc.def}
\childdocforward{cdocsamp}
%    \end{macrocode}

%\iffalse
%</sampledraft>
%\fi
%
% %%%%%%%%%%%%%%%%%%%%%%%%%%%%%%%%%%%%%%
% \paragraph{Forwarding for Final Version of the Chapters.}
%
% The following forwarding files |cdocsfn1.tex| and |cdocsfn2.tex|
% (with identical content)
% compile the final versions of the child documents
% |cdocsch1.tex| and |cdocsch2.tex|, respectively:
%\iffalse
%<*samplefinal>
%\fi
%    \begin{macrocode}
\def\version{final}
\input{childdoc.def}
\childdocforwardprefix[cdocsamp]{cdocsfn}{cdocsch}
%    \end{macrocode}

%\iffalse
%</samplefinal>
%\fi
%
% %%%%%%%%%%%%%%%%%%%%%%%%%%%%%%%%%%%%%%
% \paragraph{Command Line Processing.}
%
% The following three command lines generate the output files
% |cdocscld|, |cdocscl1| and |cdocscl2|
% which should be identical to
% |cdocsdrf|, |cdocsch1| and |cdocsfn2|, respectively:
% \begin{center}
% \begin{tabular}{l}
% |latex -jobname cdocscld \|\\
% |  "\def\version{draft}\input{childdoc.def}\childdocforward{cdocsamp}"|\\
% |latex -jobname cdocscl1 \|\\
% |  "\input{childdoc.def}\childdocforward[cdocsamp]{cdocsch1}"|\\
% |latex -jobname cdocscl2 \|\\
% |  "\def\version{final}\input{childdoc.def}\childdocforward{cdocsch2}"|
% \end{tabular}
% \end{center}
% Note that the trailing backslash on each first line
% merely continues the input to the second line
% (for convenient cut ant paste).
% Furthermore, the command |latex| can be replaced by any
% of its alternative versions such as |pdflatex|.
%
% %%%%%%%%%%%%%%%%%%%%%%%%%%%%%%%%%%%%%%%%%%%%%%%%%%%%%%%%%%%%%%%%%%%%%%%%%%%%%%
% %%%%%%%%%%%%%%%%%%%%%%%%%%%%%%%%%%%%%%%%%%%%%%%%%%%%%%%%%%%%%%%%%%%%%%%%%%%%%%
% \section{Implementation}
%\iffalse
%<*package>
%\fi
%
% This section describes the definitions file |childdoc.def|.

% The definitions cannot be loaded using |\usepackage| or |\RequirePackage|
% which has a mechanism to prevent loading a style file more than once.
% When loading the definitions by means of |\input|
% multiple instances have to be prevented manually:
%\iffalse
%This code needs to be before the `\ProvidesFile' directive
%which is defined at the beginning of this file.
%Therefore it is also placed there and commented out here.
%</package>
%<*discard>
%\fi
%    \begin{macrocode}
\ifdefined\childdocmain\endinput\fi
%    \end{macrocode}
%\iffalse
%</discard>
%<*package>
%\fi
%
% \macro{\ifchilddoc}
% \macro{\ifchilddocmanual}
% The conditional |\ifchilddoc| tells whether a
% child (true) or main (false) document is being compiled.
% The conditional |\ifchilddocmanual| tells whether
% the |\includeonly| mechanism is used (false) or
% the selection of child files must be performed manually (true).
% The definitions initialise to false:
%    \begin{macrocode}
\newif\ifchilddoc
\newif\ifchilddocmanual
%    \end{macrocode}

% \macro{\childdocname}
% \macro{\childdocjob}
% The macro |\childdocname| stores the name of the main document
% to be compiled. The macro |\childdocjob| stores the name of
% the document on which the \LaTeX{} compiler was originally invoked.
% The content of |\jobname| cannot be compared
% to filenames specified in the source due to different catcodes.
% The following code rescans |\jobname|, stores the result
% in |\childdocname| and saves a copy in |\childdocjob|:
%    \begin{macrocode}
\edef\childdocname{\scantokens\expandafter{\jobname\noexpand}}
\let\childdocjob\childdocname
%    \end{macrocode}

% \macro{\childdocdisable}
% The macro |\childdocdisable| prevents the main file
% from being processed more than once.
% At this stage, the main document command |\childdocmain|
% is assumed to be called once again where it should do nothing.
% Any subsequent call to it should prevent
% a secondary processing of the main document
% It overwrites the forwarding commands
% |\childdocof| and |\childdocforward|
% with empty macros to prevent further inclusions of the main document:
%    \begin{macrocode}
\newcommand{\childdocdisable}
{
  \renewcommand{\childdocmain}[1]{\renewcommand{\childdocmain}[1]{\endinput}}
  \renewcommand{\childdocof}[1]{}
  \renewcommand{\childdocby}[2][]{}
  \renewcommand{\childdocforward}[2][]{}
  \renewcommand{\childdocdisable}{}
}
%    \end{macrocode}

% \macro{\childdocmain}
% The macro |\childdocmain| is to be called at the top of the main file
% with nothing or the main filename (without extension) as argument.
% First, it breaks loops.
% If the argument is not empty and does not match |\childdocname|
% (which is set by the first inclusion of |childdoc.def|),
% |\ifchilddoc| is set to true, |\includeonly| is applied to the child file
% and |\jobname| is set to the main file
% (for proper handling of |.aux| files):
%    \begin{macrocode}
\newcommand{\childdocmain}[1]
{
  \childdocdisable\childdocmain{}
  \if?#1?\else
    \begingroup
      \def\childdoctmp{#1}
      \ifx\childdoctmp\childdocname
        \def\childdoctmp{}
      \else
        \def\childdoctmp
        {
          \childdoctrue
          \includeonly{\childdocname}
          \def\childdocjob{#1}
          \def\jobname{#1}
        }
      \fi
      \expandafter
    \endgroup
    \childdoctmp
  \fi
}
%    \end{macrocode}

% \macro{\childdocof}
% The command |\childdocof| redirects
% compilation to the main file |#1|.
%    \begin{macrocode}
\newcommand{\childdocof}[1]
{
  \childdocdisable
  \childdoctrue
  \includeonly{\childdocname}
  \def\jobname{#1}
  \def\childdocjob{#1}
  \input{#1}
}
%    \end{macrocode}

% \macro{\childdocby}
% The command |\childdocby| ....
%    \begin{macrocode}
\newcommand{\childdocby}[2][]
{
  \childdocdisable
  \childdoctrue
  \childdocmanualtrue
  \if?#1?\else
    \def\jobname{#2}
  \fi
  \def\childdocjob{#2}
  \input{#2}
  \endinput
}
%    \end{macrocode}

% \macro{\childdocforward}
% The command |\childdocforward| redirects
% compilation to the main file or
% (if the optional argument is given) a child file.
% Parameters are set as if the main file
% or a child file starting with |\childdocof| was compiled.
% Then compilation is handed over to the main file:
%    \begin{macrocode}
\newcommand{\childdocforward}[2][]
{
  \begingroup
    \if?#1?
      \def\childdoctmp
      {
        \def\childdocname{#2}
        \def\childdocjob{#2}
        \def\jobname{#2}
        \input{#2}
        \endinput
      }
    \else
      \def\childdoctmp
      {
        \childdocdisable
        \def\childdocname{#2}
        \childdoctrue
        \includeonly{#2}
        \def\childdocjob{#1}
        \def\jobname{#1}
        \input{#1}
        \endinput
      }
    \fi
    \expandafter
  \endgroup
  \childdoctmp
}
%    \end{macrocode}

% \macro{\childdocforwardprefix}
% The command |\childdocforwardprefix| redirects
% compilation to the main or a child file by means of a pattern.
% The prefix |#1| in the current filename is replaced by |#2|
% and the suffix of the current filename is kept
% (it is assumed that the filename does not contain the substring `|~~~|'
% which is used as a delimiter).
% Compilation is handed over to the new file by |\childdocforward|:
%    \begin{macrocode}
\newcommand{\childdocforwardprefix}[3][]
{
  \begingroup
    \def\childdocextract #2##1~~~{\def\childdoctmp{\childdocforward[#1]{#3##1}}}
    \expandafter\childdocextract\childdocname~~~
    \expandafter
  \endgroup
  \childdoctmp
}
%    \end{macrocode}

% \macro{\childdoc}
% The deprecated macro |\childdoc| is a legacy version of |\childdocmain|:
%    \begin{macrocode}
\newcommand{\childdoc}{\childdocmain}
%    \end{macrocode}

% \macro{\childdocredirect}
% The deprecated macro |\childdocredirect| is a legacy version
% of |\childdocforward| and |\childdocforwardprefix|:
%    \begin{macrocode}
\newcommand{\childdocredirect}[2][]
{
  \begingroup
    \if?#1?
      \def\childdoctmp{\childdocforward{#2}}
    \else
      \def\childdoctmp{\childdocforwardprefix{#1}{#2}}
    \fi
    \expandafter
  \endgroup
  \childdoctmp
}
%    \end{macrocode}

%\iffalse
%</package>
%\fi
%
\endinput

\childdocof{cdocsamp}
%    \end{macrocode}

%\iffalse
%</samplechap1|samplechap2>
%\fi
%
%\iffalse
%<*samplechap1>
%\fi
% Some text for chapter 1:
%    \begin{macrocode}
\section{one}
some text in chapter one
%    \end{macrocode}

%\iffalse
%</samplechap1>
%\fi
% Some text for chapter 2:
%\iffalse
%<*samplechap2>
%\fi
%    \begin{macrocode}
\section{two}
more text in chapter two
%    \end{macrocode}

%\iffalse
%</samplechap2>
%\fi
%
% %%%%%%%%%%%%%%%%%%%%%%%%%%%%%%%%%%%%%%
% \paragraph{Part Include Files.}
%
% The include files are called |cdocspt3.tex| and |cdocspt4.tex|.
%
%\iffalse
%<*samplepart3|samplepart4>
%\fi

% Optional override for |\version| flag:
%    \begin{macrocode}
%%\providecommand{\version}{final}
%    \end{macrocode}

% Include the main document:
%    \begin{macrocode}
% \iffalse
%
% childdoc.dtx Copyright (C) 2017-2018 Niklas Beisert
%
% This work may be distributed and/or modified under the
% conditions of the LaTeX Project Public License, either version 1.3
% of this license or (at your option) any later version.
% The latest version of this license is in
%   http://www.latex-project.org/lppl.txt
% and version 1.3 or later is part of all distributions of LaTeX
% version 2005/12/01 or later.
%
% This work has the LPPL maintenance status `maintained'.
%
% The Current Maintainer of this work is Niklas Beisert.
%
% This work consists of the files childdoc.dtx and childdoc.ins
% and the derived files childdoc.def and cdocsamp.tex with
% cdocsch1.tex, cdocsch2.tex, cdocsdrf.tex, cdocsfn1.tex, cdocsfn2.tex.
%
%<package>\ifdefined\childdocmain\endinput\fi
%<package>\ProvidesFile{childdoc.def}[2018/12/30 v2.0 child document driver]
%<samplemain>\ProvidesFile{cdocsamp.tex}[2018/12/30 v2.0 sample for childdoc]
%<*driver>
%\ProvidesFile{childdoc.drv}[2018/12/30 v2.0 childdoc reference manual file]
\PassOptionsToClass{10pt,a4paper}{article}
\documentclass{ltxdoc}

\usepackage[margin=35mm]{geometry}
\usepackage{hyperref}
\usepackage{hyperxmp}
\usepackage[usenames]{color}

\hypersetup{colorlinks=true}
\hypersetup{pdfstartview=FitH}
\hypersetup{pdfpagemode=UseNone}
\hypersetup{pdfsource={}}
\hypersetup{pdflang={en-UK}}
\hypersetup{pdfcopyright={Copyright 2017-2018 Niklas Beisert.
  This work may be distributed and/or modified under the
  conditions of the LaTeX Project Public License, either version 1.3
  of this license or (at your option) any later version.}}
\hypersetup{pdflicenseurl={http://www.latex-project.org/lppl.txt}}
\hypersetup{pdfcontactaddress={ETH Zurich, ITP, HIT K,
  Wolfgang-Pauli-Strasse 27}}
\hypersetup{pdfcontactpostcode={8093}}
\hypersetup{pdfcontactcity={Zurich}}
\hypersetup{pdfcontactcountry={Switzerland}}
\hypersetup{pdfcontactemail={nbeisert@itp.phys.ethz.ch}}
\hypersetup{pdfcontacturl={http://people.phys.ethz.ch/\xmptilde nbeisert/}}

\newcommand{\secref}[1]{\hyperref[#1]{section \ref*{#1}}}

\parskip1ex
\parindent0pt
\let\olditemize\itemize
\def\itemize{\olditemize\parskip0pt}

\begin{document}

\title{The \textsf{childdoc} Package}
\hypersetup{pdftitle={The childdoc Package}}
\author{Niklas Beisert\\[2ex]
  Institut f\"ur Theoretische Physik\\
  Eidgen\"ossische Technische Hochschule Z\"urich\\
  Wolfgang-Pauli-Strasse 27, 8093 Z\"urich, Switzerland\\[1ex]
  \href{mailto:nbeisert@itp.phys.ethz.ch}
  {\texttt{nbeisert@itp.phys.ethz.ch}}}
\hypersetup{pdfauthor={Niklas Beisert}}
\hypersetup{pdfsubject={Manual for the LaTeX2e Package childdoc}}
\date{30 December 2018, \textsf{v2.0}}
\maketitle

\begin{abstract}\noindent
\textsf{childdoc} is a \LaTeXe{} package
that enables the direct compilation
of document sections included by |\include|
to individual files.
\end{abstract}

\begingroup
\parskip0ex
\tableofcontents
\endgroup

%%%%%%%%%%%%%%%%%%%%%%%%%%%%%%%%%%%%%%%%%%%%%%%%%%%%%%%%%%%%%%%%%%%%%%%%%%%%%%%%
%%%%%%%%%%%%%%%%%%%%%%%%%%%%%%%%%%%%%%%%%%%%%%%%%%%%%%%%%%%%%%%%%%%%%%%%%%%%%%%%
\section{Introduction}

\LaTeX{} provides a mechanism to structure a large document (such as a book)
into a main file and several child files (containing the chapters)
using the |\include| command.
This mechanism is beneficial for documents
which span hundreds of pages in order to
make the source file(s) more manageable.
Moreover, compilation can be restricted to
selected child files by means of the |\includeonly| command.
The latter feature can be used to reduce the compilation time while editing
(this was significantly more useful in the earlier days of \LaTeX{})
or to generate a smaller document which is easier to navigate.
Another application of |\includeonly| is to generate
documents consisting of selected parts of the complete document.

However, there are a few drawbacks of the plain |\include| mechanism:
\begin{itemize}
\item
The child files cannot be compiled on their own,
they can only be compiled via the main file.
A naive editing environment
(such as a text editor with an option
to have the current file processed by \LaTeX)
may require one to switch to the main file before compiling;
attempting to compile the child file produces errors.
\item
The main file must be modified (each time)
to adjust the |\includeonly| command
to the present needs. This easily leaves the main file in a messy state.
\item
The generated document will always carry the filename
of the main document. This is inconvenient if
several child files are to be compiled and
to be kept for distribution.
\end{itemize}

The present package provides a simple interface
to make child files individually compilable by \LaTeX{}.
Compiling a child file then has the same effect as compiling
the main file with an |\includeonly| command
to select the appropriate child.
Moreover the generated document will carry the name of the child
rather than the main file.
This resolves all three above issues.

This feature is meant to make the editing of books,
thesis documents and lecture notes somewhat more convenient.
However, the package can also be used efficiently for
composing a series of documents (such as exercise sheets)
which are typically distributed individually.
It then assists the author in generating the individual documents
(potentially in different versions)
as well as a document containing the collected series.
Another application is in developing style files
or other kinds of included material
where compilation of the style file could redirect
to a sample or test file.

%%%%%%%%%%%%%%%%%%%%%%%%%%%%%%%%%%%%%%%%%%%%%%%%%%%%%%%%%%%%%%%%%%%%%%%%%%%%%%%%
%%%%%%%%%%%%%%%%%%%%%%%%%%%%%%%%%%%%%%%%%%%%%%%%%%%%%%%%%%%%%%%%%%%%%%%%%%%%%%%%
\section{Usage}

First of all, the package \textsf{childdoc} is \emph{not} a standard
\LaTeXe{} |.sty| style file! Therefore it needs to be invoked in
a non-standard way.

%%%%%%%%%%%%%%%%%%%%%%%%%%%%%%%%%%%%%%%%%%%%%%%%%%%%%%%%%%%%%%%%%%%%%%%%%%%%%%%%
\subsection{Included Files}
\label{sec:include}

%%%%%%%%%%%%%%%%%%%%%%%%%%%%%%%%%%%%%%%%
\DescribeMacro{\childdocmain}
To use the package, add the commands
\begin{center}
\begin{tabular}{l}
|\input{childdoc.def}|\\
|\childdocmain{}|\\
\end{tabular}
\end{center}
at the very top of the main \LaTeX{} file,
in particular \emph{before} the |\documentclass| statement!
The argument of |\childdocmain| should be left empty
(but it must be present).

%%%%%%%%%%%%%%%%%%%%%%%%%%%%%%%%%%%%%%%%
\DescribeMacro{\childdocof}
Furthermore, add the commands
\begin{center}
\begin{tabular}{l}
|\input{childdoc.def}|\\
|\childdocof{|\textit{main}|}|\\
\end{tabular}
\end{center}
at the top of every child file \textit{child}
which is included by |\include{|\textit{child}|}|
from within the main file
(or at least for those files to be compiled individually).
The argument \textit{main} must be the filename of the main file.

There are a couple of
considerations in setting up the main and child documents:

%%%%%%%%%%%%%%%%%%%%%%%%%%%%%%%%%%%%%%%%
\paragraph{Restrictions.}

Please note the following restrictions:
\begin{itemize}
\item
|\childdocmain| must be called with one argument \textit{main}
to ensure compatibility with earlier version of the package.
It must either be empty (|\childdocmain{}|)
or precisely match the filename of the main file in which it is specified.
See \secref{sec:detection} for further information.
\item
The filename \textit{main} must be specified without the |.tex| extension.
\item
The filename \textit{main} is case sensitive
(even in case-insensitive file systems)
due to internal string comparison.
\item
The argument \textit{main} should be fully expanded, it cannot be a macro.
\item
Subdirectories and special characters should be avoided in filenames.
\item
The command |\childdocmain{|\textit{main}|}| must be followed by a whitespace.
It should not be followed immediately by another command
or by a comment mark `|%|'.
This is because the \TeX{} parser reads the token immediately following
the argument of |\childdocmain| and puts it
at the beginning of every child section;
however, a white\-space is ignored.
\end{itemize}

%%%%%%%%%%%%%%%%%%%%%%%%%%%%%%%%%%%%%%%%
\paragraph{Content of Main File.}

It is advisable to place all content in the child files included by |\include|.
Any output contained in the main file will appear in all child documents
unless suppressed manually;
it cannot be suppressed automatically by the |\includeonly| directive
and thus should normally be avoided.
A method to include some content in the main file
by means of conditional processing is described in \secref{sec:conditional}.

%%%%%%%%%%%%%%%%%%%%%%%%%%%%%%%%%%%%%%%%
\paragraph{Page Numbering.}

When only a part of the document is compiled,
the appropriate numbering of pages
(as well as other status parameters)
is determined from the |.aux| files.
The latter contain information from previous passes.
However this information needs to propagate through
all intermediate child documents.
Therefore the page numbering in child documents may well
be inconsistent until the complete document is compiled at least once.

A useful (if unconventional) way to always ensure a consistent
page numbering is to restart the numbering in each child document
and denote the pages by `\textit{child}|.|\textit{page}'
where \textit{child} represents the chapter/section number of the child file.
This can be achieved by the command
|\numberwithin{page}{|\textit{child}|}|
of the \textsf{amsmath} package
where \textit{child} can be |chapter| or |section|
depending on the chosen structuring.
Alternatively, one can modify the macro |\thepage| appropriately
and reset the counter |page| at the start of each child file.

%%%%%%%%%%%%%%%%%%%%%%%%%%%%%%%%%%%%%%%%%%%%%%%%%%%%%%%%%%%%%%%%%%%%%%%%%%%%%%%%
\subsection{Conditional Processing}
\label{sec:conditional}

The package provides a mechanism to compile different versions
of a document. To customise the versions further some conditional processing
can come in handy to distinguish which version is being compiled.
The package provides two macros to describe the compilation context:

%%%%%%%%%%%%%%%%%%%%%%%%%%%%%%%%%%%%%%%%
\DescribeMacro{\ifchilddoc}
The conditional |\ifchilddoc| distinguishes between the compilation of
child documents and the main document:
%
\begin{center}
|\ifchilddoc |\textit{child-code}| |[|\||else |\textit{main-code}]| \||fi|
\end{center}

%%%%%%%%%%%%%%%%%%%%%%%%%%%%%%%%%%%%%%%%
\DescribeMacro{\childdocname}
\DescribeMacro{\childdocjob}
The macro |\childdocname| contains the filename (without extension)
of the main or child file being processed.
Note that |\childdocjob| will always contain the name of the main file.

%%%%%%%%%%%%%%%%%%%%%%%%%%%%%%%%%%%%%%%%
\paragraph{Title Page.}

Conditional processing can be used to include a title or banner page
in the main document when proper precautions are taken.
Importantly, the code in the main file should ensure that the page counter
(as well as other status parameters which are stored in the |.aux| files)
takes the same value after the conditional processing.
Otherwise the page numbers may take divergent values
depending on which part is compiled.

For example, a title page could be declared by:
%
\begin{center}
\begin{tabular}{l}
|\ifchilddoc\||else|\\
|\addtocounter{page}{-1}|\\
\textit{code for title page}\\
|\newpage|\\
|\||fi|
\end{tabular}
\end{center}
%
A banner page for the child documents can be generated by:
%
\begin{center}
\begin{tabular}{l}
|\ifchilddoc|\\
|\addtocounter{page}{-1}|\\
\textit{code for banner page}\\
|\newpage|\\
|\||fi|
\end{tabular}
\end{center}
%
Here one could write a message such as:
\begin{center}
|This is the part \childdocname{} of \childdocjob{}.|
\end{center}

%%%%%%%%%%%%%%%%%%%%%%%%%%%%%%%%%%%%%%%%%%%%%%%%%%%%%%%%%%%%%%%%%%%%%%%%%%%%%%%%
\subsection{Flags}
\label{sec:flags}

The package makes it easy to generate different versions
of the main or child documents.
To this end compilation flags can be defined
and assigned different default values.
They will be particularly useful in conjunction
with the forwarding mechanism described in \secref{sec:forward}.

For example, it may be useful to have a flag |\version|
which can be set to |draft| or |final|.
The document source will contain some conditional code
depending on the value of |\version|.
Suppose further, the flag should default to |final| for the main file
and to |draft| for child files
which is a natural assignment for editing the document.
This is achieved by placing the following code
in the preamble of the main document
(below the |\childdocmain| directive):
%
\begin{center}
\begin{tabular}{l}
|\ifchilddoc|\\
|\providecommand{\version}{draft}|\\
|\||else|\\
|\providecommand{\version}{final}|\\
|\||fi|
\end{tabular}
\end{center}
%
The definition by |\providecommand| makes sure
that previous definitions are not overwritten.
Further statements |\providecommand{\version}{...}|
can thus be added before the above code to override it.

For the main file, one might add a line
(between |\childdocmain| and the above block)
%
\begin{center}
|%\ifchilddoc\||else\providecommand{\version}{draft}\||fi|
\end{center}
%
which can be uncommented to produce a draft version.
Likewise one can add a line to the very top of a child file
(above the |\childdocof{|\textit{main}|}| directive)
%
\begin{center}
|%\providecommand{\version}{final}|
\end{center}
%
which can be uncommented to produce the final version of this child document.

%%%%%%%%%%%%%%%%%%%%%%%%%%%%%%%%%%%%%%%%%%%%%%%%%%%%%%%%%%%%%%%%%%%%%%%%%%%%%%%%
\subsection{Forwarding}
\label{sec:forward}

Different versions of the main or child documents
using compilation flags as described in \secref{sec:flags}
can be (permanently) stored in different files
for convenient compilation, viewing and distribution.
To this end, the package defines a command
to pass on compilation to a different file:

%%%%%%%%%%%%%%%%%%%%%%%%%%%%%%%%%%%%%%%%
\DescribeMacro{\childdocforward}
The command |\childdocforward| redirects processing to
another source file:
%
\begin{center}
\begin{tabular}{l}
|\input{childdoc.def}|\\
|\childdocforward[|\textit{main}|]{|\textit{dest}|}|\\
\end{tabular}
\end{center}
%
The argument \textit{dest} is the destination file
(without extension).
It should be the main file or one of the child files.
Note that further \textsf{childdoc} directives
such as |\childdocof| and |\childdocforward|
in the indicated file will be processed in this form.
The optional argument \textit{main}
passes on directly to the main file \textit{main}
while pretending to compile the child \textit{dest}.
This form behaves as if \textit{dest}
issues |\childdocof{|\textit{main}|}| right away,
and no further \textsf{childdoc} directives will be processed.

%%%%%%%%%%%%%%%%%%%%%%%%%%%%%%%%%%%%%%%%
\DescribeMacro{\...prefix}
In the alternative form |\childdocforwardprefix|,
%
\begin{center}
\begin{tabular}{l}
|\input{childdoc.def}|\\
|\childdocforwardprefix[|\textit{main}|]{|\textit{prefix}|}{|\textit{dest}|}|
\end{tabular}
\end{center}
%
the destination file is determined by a pattern
depending on the current file:
To make this work, the current file must be called
`{\textit{prefix}\hspace{0.2em}\textit{suffix}}'
with \textit{prefix} matching precisely the argument.
Processing is then passed on to the file
`{\textit{dest}\hspace{0.2em}\textit{suffix}}'.
Surely, the same effect is achieved by
directly specifying the
argument `{\textit{dest}\hspace{0.2em}\textit{suffix}}'
in the first form.
However, that requires to set up a different file
for each child. With the alternative form of the command
all these files can have exactly the same content
which simplifies setting them up and maintaining them.

For example, the following file |draft.tex|
with a compilation flag |\version| as described in \secref{sec:flags}
compiles the main document as a draft:
%
\begin{center}
\begin{tabular}{l}
|\def\version{draft}|\\
|\input{childdoc.def}|\\
|\childdocforward{|\textit{main}|}|
\end{tabular}
\end{center}
%
Likewise, the following files |final|\textit{nn}|.tex|
compile the final version of the child document
|child|\textit{nn}|.tex|:
%
\begin{center}
\begin{tabular}{l}
|\def\version{final}|\\
|\input{childdoc.def}|\\
|\childdocforwardprefix{final}{child}|
\end{tabular}
\end{center}
%

Note that when several versions of a main file and/or of each child file
are to be generated, it may be convenient to set up a |Makefile| or
shell script to automatise the process.

%%%%%%%%%%%%%%%%%%%%%%%%%%%%%%%%%%%%%%%%%%%%%%%%%%%%%%%%%%%%%%%%%%%%%%%%%%%%%%%%
\subsection{Command Line Processing}
\label{sec:commandline}

The effect of redirection files can also be achieved by invoking
the \LaTeX{} compiler with a more elaborate command line.
Most conveniently this should be done as part
of a shell script or a |Makefile|.

When using \textsf{childdoc} in the main file, the following
command lines effectively perform a redirection
(note that depending on the shell being used,
backslashes may have to be doubled: `|\|' $\to$ `|\\|'):
%
\begin{center}
|... -jobname "|\textit{target}|" |\\|"|[\textit{flags}]%
|\input{childdoc.def}\childdocforward[|\textit{main}|]{|\textit{dest}|}"|
\end{center}
%
Here \textit{target} is the name of the output file,
\textit{main} is the name of the main file
and \textit{dest} is the name of the main or child file to be processed
(all filenames without extensions).
The optional argument \textit{main} can be omitted
if \textit{main} matches \textit{dest}.
Optionally, compilation \textit{flags} can be defined via |\def| commands.
This command line makes the \TeX{} engine believe
it is compiling the file \textit{target}
whose content is specified as the latter parameter.
The provided code then forwards the processing to
\textit{main} or \textit{dest} as described in \secref{sec:forward}.

%%%%%%%%%%%%%%%%%%%%%%%%%%%%%%%%%%%%%%%%%%%%%%%%%%%%%%%%%%%%%%%%%%%%%%%%%%%%%%%%
\subsection{Include by Input}
\label{sec:input}

Including child documents by |\include| has some restrictions by design.
Most notably, the content of a child document always occupies
its own set of pages; pages cannot be shared between child documents.
Usually, this behaviour makes perfect sense
because each child document contain an essential part of the document.
However, in some situations it may be desirable to compose
a document from a collection of parts
without having mandatory page breaks between then.
For this case, the package
provides a mechanism to include parts
by |\input| which can also be processed individually.
However, by construction this mechanism
requires manual handling of the content to be output.

%%%%%%%%%%%%%%%%%%%%%%%%%%%%%%%%%%%%%%%%
\DescribeMacro{\ifchilddocmanual}
The main file should be prepared as usual, see \secref{sec:include}.
However, the document body must make a distinction
between processing of an individual part and of the main document, e.g.:
%
\begin{center}
\begin{tabular}{l}
|\ifchilddocmanual|\\
|\input{\childdocname}|\\
|\||else|\\
\textit{document body with }|\input{|\textit{part}|}|\\
|\||fi|
\end{tabular}
\end{center}
%
The conditional |\ifchilddocmanual| is true whenever
a part to be included by |\input| is being compiled,
and the name of the part is stored in |\childdocname|.

%%%%%%%%%%%%%%%%%%%%%%%%%%%%%%%%%%%%%%%%
\DescribeMacro{\childdocby}
Each part to be included by |\input| should start with:
%
\begin{center}
\begin{tabular}{l}
|\input{childdoc.def}|\\
|\childdocby{|\textit{main}|}|\\
\end{tabular}
\end{center}
%
The directive |\childdocby| is similar to |\childdocof|
described in \secref{sec:include},
but the subsequent selection of content must be done manually.
To that end, both |\ifchilddoc| and |\ifchilddocmanual|
will be true upon processing of a part,
and the name of the part is stored in |\childdocname|.
Note that |\jobname| will be set to the filename of the current part
so that each part receives an individual |.aux| file
that does not interfere with the |.aux| file(s) of the main document.
This behaviour can be altered by the alternative form
|\childdocby[*]{|\textit{main}|}| (with a non-empty optional argument)
which uses the |.aux| file of the main document
by setting |\jobname| to \textit{main}.

%%%%%%%%%%%%%%%%%%%%%%%%%%%%%%%%%%%%%%%%%%%%%%%%%%%%%%%%%%%%%%%%%%%%%%%%%%%%%%%%
\subsection{Driver Development}
\label{sec:driver}

The \textsf{childdoc} mechanism can also be use for the development
of definition files such as \LaTeX{} styles or classes.
This case differs from the above setup with multiple parts
included by |\include| in that no |\includeonly| should be invoked.
This can be achieved by starting the include file
(before |\ProvidesPackage|) with:
%
\begin{center}
\begin{tabular}{l}
|\input{childdoc.def}|\\
|\childdocforward{|\textit{main}|}|\\
\end{tabular}
\end{center}
%
or alternatively with:
%
\begin{center}
\begin{tabular}{l}
|\input{childdoc.def}|\\
|\childdocby{|\textit{main}|}|\\
\end{tabular}
\end{center}
%
Both forms have slightly different effects as described above.
The main file is prepared as usual, see \secref{sec:include}.

%%%%%%%%%%%%%%%%%%%%%%%%%%%%%%%%%%%%%%%%%%%%%%%%%%%%%%%%%%%%%%%%%%%%%%%%%%%%%%%%
\subsection{Legacy Detection}
\label{sec:detection}

The directive |\childdocmain| in the main file can detect
whether the complete document or merely a child is to be compiled
even without using the directive |\childdocof|.
This method is deprecated because it is less robust
and there is no compelling reason to use it;
it is merely provided for backward compatibility
and it may be removed in future versions.

If the detection mechanism is to be used,
it is mandatory to correctly specify
the filename of the main file as the argument of |\childdocmain|:
%
\begin{center}
\begin{tabular}{l}
|\input{childdoc.def}|\\
|\childdocmain{|\textit{main}|}|\\
\end{tabular}
\end{center}
%
If |\jobname| does not match the argument \textit{main} of |\childdocmain|,
it is assumed that |\jobname| points to the child file to be compiled.
When using |\childdocmain| with the main file specified as argument,
it suffices to start a child file
with just |\input{|\textit{main}|}|
without loading of the package and using |\childdocof|.
If instead all processing is done
with the appropriate \textsf{childdoc} directives,
the argument of \textit{main} of |\childdocmain| can be empty.

An alternative version of the command line processing described
in \secref{sec:commandline} using the detection mechanism reads:
%
\begin{center}
|... -jobname "|\textit{target}|" "|[\textit{flags}]%
[|\def\jobname{|\textit{dest}|}|]|\input{|\textit{main}|}"|
\end{center}

%%%%%%%%%%%%%%%%%%%%%%%%%%%%%%%%%%%%%%%%%%%%%%%%%%%%%%%%%%%%%%%%%%%%%%%%%%%%%%%%
\subsection{Manual Code}
\label{sec:manual}

In case one cannot be certain whether the definitions file |childdoc.def|
is installed on the target \TeX{} distribution
and one prefers not to ship it,
it is conceivable to paste a few relevant commands into the sources.

To that end, drop all statements |\input{childdoc.def}|
and perform the replacements as outlined below.
Instead of |\childdocmain{|\textit{main}|}| add the following code
to the top of the main file:
%
\begin{center}
\begin{tabular}{l}
|\||ifdefined\childdocname\endinput\||fi\newif\ifchilddoc|\\
|\edef\childdocname{\scantokens\expandafter{\jobname\noexpand}}|\\
|\def\childdocmain{|\textit{main}|}\||ifx\childdocmain\childdocname\||else|\\
|\childdoctrue\includeonly{\childdocname}\let\jobname\childdocmain\||fi|\\
\end{tabular}
\end{center}
%
Instead of |\childdocof{|\textit{main}|}| just include the main file
at the top of each child file:
%
\begin{center}
|\input{|\textit{main}|}|
\end{center}
%
A simple redirection |\childdocforward{|\textit{dest}|}| is achieved by:
%
\begin{center}
|\def\jobname{|\textit{dest}|}\input{\jobname}|
\end{center}
%
The redirection with prefix
|\childdocforwardprefix[|\textit{prefix}|]{|\textit{dest}|}|
is accomplished by:
%
\begin{center}
\begin{tabular}{l}
|{\edef\jobname{\scantokens\expandafter{\jobname\noexpand}}|\\
|\def\redirectjob |\textit{prefix}|#1~~~{\gdef\jobname{|\textit{dest}|#1}}|\\
|\expandafter\redirectjob\jobname~~~}\input{\jobname}|
\end{tabular}
\end{center}

In an alternative approach,
child documents can be compiled by a specific command line
without additional code or specific definitions:
%
\begin{center}
|... -jobname "|\textit{target}|" "|[\textit{flags}]%
|\includeonly{|\textit{dest}|}\input{|\textit{main}|}"|
\end{center}
%

%%%%%%%%%%%%%%%%%%%%%%%%%%%%%%%%%%%%%%%%%%%%%%%%%%%%%%%%%%%%%%%%%%%%%%%%%%%%%%%%
%%%%%%%%%%%%%%%%%%%%%%%%%%%%%%%%%%%%%%%%%%%%%%%%%%%%%%%%%%%%%%%%%%%%%%%%%%%%%%%%
\section{Information}

%%%%%%%%%%%%%%%%%%%%%%%%%%%%%%%%%%%%%%%%%%%%%%%%%%%%%%%%%%%%%%%%%%%%%%%%%%%%%%%%
\subsection{Copyright}

Copyright \copyright{} 2017--2018 Niklas Beisert

This work may be distributed and/or modified under the
conditions of the \LaTeX{} Project Public License, either version 1.3
of this license or (at your option) any later version.
The latest version of this license is in
  \url{http://www.latex-project.org/lppl.txt}
and version 1.3 or later is part of all distributions of \LaTeX{}
version 2005/12/01 or later.

This work has the LPPL maintenance status `maintained'.

The Current Maintainer of this work is Niklas Beisert.

This work consists of the files |README.txt|, |childdoc.ins| and |childdoc.dtx|
as well as the derived files |childdoc.def|, |cdocsamp.tex|
with |cdocsch1.tex|, |cdocsch2.tex|, |cdocspt3.tex|, |cdocspt4.tex|,
|cdocsdrf.tex|, |cdocsfn1.tex|, |cdocsfn2.tex|
as well as |childdoc.pdf|.

%%%%%%%%%%%%%%%%%%%%%%%%%%%%%%%%%%%%%%%%%%%%%%%%%%%%%%%%%%%%%%%%%%%%%%%%%%%%%%%%
\subsection{Files and Installation}

The package consists of the files:
%
\begin{center}
\begin{tabular}{ll}
    |README.txt|   & readme file \\
    |childdoc.ins| & installation file \\
    |childdoc.dtx| & source file \\
    |childdoc.def| & definition file \\
    |cdocsamp.tex| & sample main file \\
    |cdocsch1.tex| & sample include file \\
    |cdocsch2.tex| & sample include file \\
    |cdocspt3.tex| & sample part file \\
    |cdocspt4.tex| & sample part file \\
    |cdocsdrf.tex| & sample redirection file \\
    |cdocsfn1.tex| & sample redirection file \\
    |cdocsfn2.tex| & sample redirection file \\
    |childdoc.pdf| & manual
\end{tabular}
\end{center}
%
The distribution consists of the files
|README.txt|, |childdoc.ins| and |childdoc.dtx|.
%
\begin{itemize}
\item
Run (pdf)\LaTeX{} on |childdoc.dtx|
to compile the manual |childdoc.pdf| (this file).
\item
Run \LaTeX{} on |childdoc.ins| to create the definitions file |childdoc.def|
and the sample |cdocsamp.tex| with include files
|cdocsch1.tex|, |cdocsch2.tex|, |cdocspt3.tex|, |cdocspt4.tex|,
|cdocsdrf.tex|, |cdocsfn1.tex|, |cdocsfn2.tex|.
Then copy the file |childdoc.def| to an appropriate directory of your \LaTeX{}
distribution, e.g.\ \textit{texmf-root}|/tex/latex/childdoc|.
\end{itemize}

%%%%%%%%%%%%%%%%%%%%%%%%%%%%%%%%%%%%%%%%%%%%%%%%%%%%%%%%%%%%%%%%%%%%%%%%%%%%%%%%
\subsection{Related CTAN Packages}

There are several other packages which offer a similar functionality:
%
\begin{itemize}
\item
The packages
\href{http://ctan.org/pkg/docmute}{\textsf{docmute}},
\href{http://ctan.org/pkg/includex}{\textsf{includex}} and
\href{http://ctan.org/pkg/standalone}{\textsf{standalone}}
provide commands to include only the document body of
a child file thus allowing both files to be compiled individually.
\item
The packages \href{http://ctan.org/pkg/subdocs}{\textsf{subdocs}}
and \href{http://ctan.org/pkg/subfiles}{\textsf{subfiles}}
provide structures in which the main and child documents can be
encapsulated and allowing them to be compiled individually.
The inclusion mechanism is different from the conventional |\include|.
\item
The package \href{http://ctan.org/pkg/combine}{\textsf{combine}}
is an elaborate solution to combine several documents into one.
\end{itemize}
%
See also the CTAN topic \href{http://ctan.org/topic/subdocs}{\textsf{subdocs}}
for further related packages.
The present package differs from the above solutions in that
a document structure constructed with the conventional |\include| mechanism
just needs two extra commands at the top of every file
such that all constituent files can be compiled individually.

%%%%%%%%%%%%%%%%%%%%%%%%%%%%%%%%%%%%%%%%%%%%%%%%%%%%%%%%%%%%%%%%%%%%%%%%%%%%%%%%
%\subsection{Feature Suggestions}
%
%The following is a list of features which may be useful for future
%versions of this package:
%%
%\begin{itemize}
%\item
%\ldots
%\end{itemize}

%%%%%%%%%%%%%%%%%%%%%%%%%%%%%%%%%%%%%%%%%%%%%%%%%%%%%%%%%%%%%%%%%%%%%%%%%%%%%%%%
\subsection{Revision History}

%%%%%%%%%%%%%%%%%%%%%%%%%%%%%%%%%%%%%%%%
\paragraph{v2.0:} 2018/12/30

\begin{itemize}
\item
immediate forward processing
\item
added |\childdocby| mechanism
\item
manual restructured
\end{itemize}

%%%%%%%%%%%%%%%%%%%%%%%%%%%%%%%%%%%%%%%%
\paragraph{v1.6:} 2018/01/17

\begin{itemize}
\item
application for development of include files
\item
corrections to manual
\end{itemize}

%%%%%%%%%%%%%%%%%%%%%%%%%%%%%%%%%%%%%%%%
\paragraph{v1.5:} 2017/05/21

\begin{itemize}
\item
more complete structuring introduced
\item
|\childdocof| introduced
\item
|\childdoc| renamed to |\childdocmain|
\item
|\childredirect| renamed to |\childdocforward| and |\childdocforwardprefix|
and functionality expanded
\end{itemize}

%%%%%%%%%%%%%%%%%%%%%%%%%%%%%%%%%%%%%%%%
\paragraph{v1.0:} 2017/04/27

\begin{itemize}
\item
manual and install package
\item
first version published on CTAN
\end{itemize}

%%%%%%%%%%%%%%%%%%%%%%%%%%%%%%%%%%%%%%%%
\paragraph{v0.6:} 2017/04/26

\begin{itemize}
\item
redirection mechanism added
\end{itemize}

%%%%%%%%%%%%%%%%%%%%%%%%%%%%%%%%%%%%%%%%
\paragraph{v0.5:} 2017/04/26

\begin{itemize}
\item
functionality in definition file
\end{itemize}


%%%%%%%%%%%%%%%%%%%%%%%%%%%%%%%%%%%%%%%%%%%%%%%%%%%%%%%%%%%%%%%%%%%%%%%%%%%%%%%%
%%%%%%%%%%%%%%%%%%%%%%%%%%%%%%%%%%%%%%%%%%%%%%%%%%%%%%%%%%%%%%%%%%%%%%%%%%%%%%%%
%%%%%%%%%%%%%%%%%%%%%%%%%%%%%%%%%%%%%%%%%%%%%%%%%%%%%%%%%%%%%%%%%%%%%%%%%%%%%%%%
\appendix

\settowidth\MacroIndent{\rmfamily\scriptsize 000\ }

 \DocInput{childdoc.dtx}

\end{document}
%</driver>
% \fi
%
% %%%%%%%%%%%%%%%%%%%%%%%%%%%%%%%%%%%%%%%%%%%%%%%%%%%%%%%%%%%%%%%%%%%%%%%%%%%%%%
% %%%%%%%%%%%%%%%%%%%%%%%%%%%%%%%%%%%%%%%%%%%%%%%%%%%%%%%%%%%%%%%%%%%%%%%%%%%%%%
% \section{Sample}
%\iffalse
%<*samplemain>
%\fi
%
% The following presents a sample document
% with two chapters, two parts, a title page,
% a compile flag as well as three forwarding files to set the flag.
% It consists of eight |.tex| files:
% \begin{center}
% \begin{tabular}{ll}
% |cdocsamp.tex|&main file\\
% |cdocsch1.tex|&include file for chapter 1\\
% |cdocsch2.tex|&include file for chapter 2\\
% |cdocspt3.tex|&include file for part 3\\
% |cdocspt4.tex|&include file for part 4\\
% |cdocsdrf.tex|&forwarding file for main file in draft mode\\
% |cdocsfi1.tex|&forwarding file for final version of chapter 1\\
% |cdocsfi2.tex|&forwarding file for final version of chapter 2\\
% \end{tabular}
% \end{center}
% Each of the eight files can be compiled directly by the \LaTeX{} compiler.
%
% %%%%%%%%%%%%%%%%%%%%%%%%%%%%%%%%%%%%%%
% \paragraph{Main File.}
%
% The main file is called |cdocsamp.tex|.
%
% Load the \textsf{childdoc} definitions and
% declare the filename for the main document:
%    \begin{macrocode}
\input{childdoc.def}
\childdocmain{}
%    \end{macrocode}

% Optional override for |\version| flag:
%    \begin{macrocode}
%%\ifchilddoc\else\providecommand{\version}{draft}\fi
%    \end{macrocode}

% Define the default values for the |\version| flag
% (|final| for the main file and |draft| for childs):
%    \begin{macrocode}
\ifchilddoc
\providecommand{\version}{draft}
\else
\providecommand{\version}{final}
\fi
%    \end{macrocode}

% Load the standard document class:
%    \begin{macrocode}
\documentclass[12pt]{article}
%    \end{macrocode}

% Start the document body:
%    \begin{macrocode}
\begin{document}
%    \end{macrocode}

% Declare a title page.
% Print title, part of document being processed and version flag:
%    \begin{macrocode}
\addtocounter{page}{-1}
\begin{center}
{\LARGE\bfseries{}childdoc example\par}
\vspace{1cm}
\ifchilddoc
\ifchilddocmanual part\else chapter\fi:
`\childdocname' of `\childdocjob'\par
\else
main document: `\childdocjob'\par
\fi
version: \version\par
\end{center}
\newpage
%    \end{macrocode}

% Manually include selected file,
% otherwise process as usual:
%    \begin{macrocode}
\ifchilddocmanual
\section*{part `\childdocname'}
\input{\childdocname}
\else
%    \end{macrocode}

% Include the two chapters:
%    \begin{macrocode}
\include{cdocsch1}
\include{cdocsch2}
%    \end{macrocode}

% Include the two parts unless only chapters should be displayed:
%    \begin{macrocode}
\ifchilddoc\else
\section{part three}
\input{cdocspt3}
\section{part four}
\input{cdocspt4}
\fi
%    \end{macrocode}

% Process as usual until here:
%    \begin{macrocode}
\fi
%    \end{macrocode}

% End of document body:
%    \begin{macrocode}
\end{document}
%    \end{macrocode}
%\iffalse
%</samplemain>
%\fi
%
% %%%%%%%%%%%%%%%%%%%%%%%%%%%%%%%%%%%%%%
% \paragraph{Chapter Include Files.}
%
% The include files are called |cdocsch1.tex| and |cdocsch2.tex|.
%
%\iffalse
%<*samplechap1|samplechap2>
%\fi

% Optional override for |\version| flag:
%    \begin{macrocode}
%%\providecommand{\version}{final}
%    \end{macrocode}

% Include the main document:
%    \begin{macrocode}
\input{childdoc.def}
\childdocof{cdocsamp}
%    \end{macrocode}

%\iffalse
%</samplechap1|samplechap2>
%\fi
%
%\iffalse
%<*samplechap1>
%\fi
% Some text for chapter 1:
%    \begin{macrocode}
\section{one}
some text in chapter one
%    \end{macrocode}

%\iffalse
%</samplechap1>
%\fi
% Some text for chapter 2:
%\iffalse
%<*samplechap2>
%\fi
%    \begin{macrocode}
\section{two}
more text in chapter two
%    \end{macrocode}

%\iffalse
%</samplechap2>
%\fi
%
% %%%%%%%%%%%%%%%%%%%%%%%%%%%%%%%%%%%%%%
% \paragraph{Part Include Files.}
%
% The include files are called |cdocspt3.tex| and |cdocspt4.tex|.
%
%\iffalse
%<*samplepart3|samplepart4>
%\fi

% Optional override for |\version| flag:
%    \begin{macrocode}
%%\providecommand{\version}{final}
%    \end{macrocode}

% Include the main document:
%    \begin{macrocode}
\input{childdoc.def}
\childdocby{cdocsamp}
%    \end{macrocode}

%\iffalse
%</samplepart3|samplepart4>
%\fi
%
%\iffalse
%<*samplepart3>
%\fi
% Some text for part 3:
%    \begin{macrocode}
some text in part three
%    \end{macrocode}

%\iffalse
%</samplepart3>
%\fi
% Some text for part 4:
%\iffalse
%<*samplepart4>
%\fi
%    \begin{macrocode}
more text in part four
%    \end{macrocode}

%\iffalse
%</samplepart4>
%\fi
%
% %%%%%%%%%%%%%%%%%%%%%%%%%%%%%%%%%%%%%%
% \paragraph{Forwarding for a Complete Draft.}
%
% The following forwarding file |cdocsdrf.tex|
% compiles the main document in draft mode:
%\iffalse
%<*sampledraft>
%\fi
%    \begin{macrocode}
\def\version{draft}
\input{childdoc.def}
\childdocforward{cdocsamp}
%    \end{macrocode}

%\iffalse
%</sampledraft>
%\fi
%
% %%%%%%%%%%%%%%%%%%%%%%%%%%%%%%%%%%%%%%
% \paragraph{Forwarding for Final Version of the Chapters.}
%
% The following forwarding files |cdocsfn1.tex| and |cdocsfn2.tex|
% (with identical content)
% compile the final versions of the child documents
% |cdocsch1.tex| and |cdocsch2.tex|, respectively:
%\iffalse
%<*samplefinal>
%\fi
%    \begin{macrocode}
\def\version{final}
\input{childdoc.def}
\childdocforwardprefix[cdocsamp]{cdocsfn}{cdocsch}
%    \end{macrocode}

%\iffalse
%</samplefinal>
%\fi
%
% %%%%%%%%%%%%%%%%%%%%%%%%%%%%%%%%%%%%%%
% \paragraph{Command Line Processing.}
%
% The following three command lines generate the output files
% |cdocscld|, |cdocscl1| and |cdocscl2|
% which should be identical to
% |cdocsdrf|, |cdocsch1| and |cdocsfn2|, respectively:
% \begin{center}
% \begin{tabular}{l}
% |latex -jobname cdocscld \|\\
% |  "\def\version{draft}\input{childdoc.def}\childdocforward{cdocsamp}"|\\
% |latex -jobname cdocscl1 \|\\
% |  "\input{childdoc.def}\childdocforward[cdocsamp]{cdocsch1}"|\\
% |latex -jobname cdocscl2 \|\\
% |  "\def\version{final}\input{childdoc.def}\childdocforward{cdocsch2}"|
% \end{tabular}
% \end{center}
% Note that the trailing backslash on each first line
% merely continues the input to the second line
% (for convenient cut ant paste).
% Furthermore, the command |latex| can be replaced by any
% of its alternative versions such as |pdflatex|.
%
% %%%%%%%%%%%%%%%%%%%%%%%%%%%%%%%%%%%%%%%%%%%%%%%%%%%%%%%%%%%%%%%%%%%%%%%%%%%%%%
% %%%%%%%%%%%%%%%%%%%%%%%%%%%%%%%%%%%%%%%%%%%%%%%%%%%%%%%%%%%%%%%%%%%%%%%%%%%%%%
% \section{Implementation}
%\iffalse
%<*package>
%\fi
%
% This section describes the definitions file |childdoc.def|.

% The definitions cannot be loaded using |\usepackage| or |\RequirePackage|
% which has a mechanism to prevent loading a style file more than once.
% When loading the definitions by means of |\input|
% multiple instances have to be prevented manually:
%\iffalse
%This code needs to be before the `\ProvidesFile' directive
%which is defined at the beginning of this file.
%Therefore it is also placed there and commented out here.
%</package>
%<*discard>
%\fi
%    \begin{macrocode}
\ifdefined\childdocmain\endinput\fi
%    \end{macrocode}
%\iffalse
%</discard>
%<*package>
%\fi
%
% \macro{\ifchilddoc}
% \macro{\ifchilddocmanual}
% The conditional |\ifchilddoc| tells whether a
% child (true) or main (false) document is being compiled.
% The conditional |\ifchilddocmanual| tells whether
% the |\includeonly| mechanism is used (false) or
% the selection of child files must be performed manually (true).
% The definitions initialise to false:
%    \begin{macrocode}
\newif\ifchilddoc
\newif\ifchilddocmanual
%    \end{macrocode}

% \macro{\childdocname}
% \macro{\childdocjob}
% The macro |\childdocname| stores the name of the main document
% to be compiled. The macro |\childdocjob| stores the name of
% the document on which the \LaTeX{} compiler was originally invoked.
% The content of |\jobname| cannot be compared
% to filenames specified in the source due to different catcodes.
% The following code rescans |\jobname|, stores the result
% in |\childdocname| and saves a copy in |\childdocjob|:
%    \begin{macrocode}
\edef\childdocname{\scantokens\expandafter{\jobname\noexpand}}
\let\childdocjob\childdocname
%    \end{macrocode}

% \macro{\childdocdisable}
% The macro |\childdocdisable| prevents the main file
% from being processed more than once.
% At this stage, the main document command |\childdocmain|
% is assumed to be called once again where it should do nothing.
% Any subsequent call to it should prevent
% a secondary processing of the main document
% It overwrites the forwarding commands
% |\childdocof| and |\childdocforward|
% with empty macros to prevent further inclusions of the main document:
%    \begin{macrocode}
\newcommand{\childdocdisable}
{
  \renewcommand{\childdocmain}[1]{\renewcommand{\childdocmain}[1]{\endinput}}
  \renewcommand{\childdocof}[1]{}
  \renewcommand{\childdocby}[2][]{}
  \renewcommand{\childdocforward}[2][]{}
  \renewcommand{\childdocdisable}{}
}
%    \end{macrocode}

% \macro{\childdocmain}
% The macro |\childdocmain| is to be called at the top of the main file
% with nothing or the main filename (without extension) as argument.
% First, it breaks loops.
% If the argument is not empty and does not match |\childdocname|
% (which is set by the first inclusion of |childdoc.def|),
% |\ifchilddoc| is set to true, |\includeonly| is applied to the child file
% and |\jobname| is set to the main file
% (for proper handling of |.aux| files):
%    \begin{macrocode}
\newcommand{\childdocmain}[1]
{
  \childdocdisable\childdocmain{}
  \if?#1?\else
    \begingroup
      \def\childdoctmp{#1}
      \ifx\childdoctmp\childdocname
        \def\childdoctmp{}
      \else
        \def\childdoctmp
        {
          \childdoctrue
          \includeonly{\childdocname}
          \def\childdocjob{#1}
          \def\jobname{#1}
        }
      \fi
      \expandafter
    \endgroup
    \childdoctmp
  \fi
}
%    \end{macrocode}

% \macro{\childdocof}
% The command |\childdocof| redirects
% compilation to the main file |#1|.
%    \begin{macrocode}
\newcommand{\childdocof}[1]
{
  \childdocdisable
  \childdoctrue
  \includeonly{\childdocname}
  \def\jobname{#1}
  \def\childdocjob{#1}
  \input{#1}
}
%    \end{macrocode}

% \macro{\childdocby}
% The command |\childdocby| ....
%    \begin{macrocode}
\newcommand{\childdocby}[2][]
{
  \childdocdisable
  \childdoctrue
  \childdocmanualtrue
  \if?#1?\else
    \def\jobname{#2}
  \fi
  \def\childdocjob{#2}
  \input{#2}
  \endinput
}
%    \end{macrocode}

% \macro{\childdocforward}
% The command |\childdocforward| redirects
% compilation to the main file or
% (if the optional argument is given) a child file.
% Parameters are set as if the main file
% or a child file starting with |\childdocof| was compiled.
% Then compilation is handed over to the main file:
%    \begin{macrocode}
\newcommand{\childdocforward}[2][]
{
  \begingroup
    \if?#1?
      \def\childdoctmp
      {
        \def\childdocname{#2}
        \def\childdocjob{#2}
        \def\jobname{#2}
        \input{#2}
        \endinput
      }
    \else
      \def\childdoctmp
      {
        \childdocdisable
        \def\childdocname{#2}
        \childdoctrue
        \includeonly{#2}
        \def\childdocjob{#1}
        \def\jobname{#1}
        \input{#1}
        \endinput
      }
    \fi
    \expandafter
  \endgroup
  \childdoctmp
}
%    \end{macrocode}

% \macro{\childdocforwardprefix}
% The command |\childdocforwardprefix| redirects
% compilation to the main or a child file by means of a pattern.
% The prefix |#1| in the current filename is replaced by |#2|
% and the suffix of the current filename is kept
% (it is assumed that the filename does not contain the substring `|~~~|'
% which is used as a delimiter).
% Compilation is handed over to the new file by |\childdocforward|:
%    \begin{macrocode}
\newcommand{\childdocforwardprefix}[3][]
{
  \begingroup
    \def\childdocextract #2##1~~~{\def\childdoctmp{\childdocforward[#1]{#3##1}}}
    \expandafter\childdocextract\childdocname~~~
    \expandafter
  \endgroup
  \childdoctmp
}
%    \end{macrocode}

% \macro{\childdoc}
% The deprecated macro |\childdoc| is a legacy version of |\childdocmain|:
%    \begin{macrocode}
\newcommand{\childdoc}{\childdocmain}
%    \end{macrocode}

% \macro{\childdocredirect}
% The deprecated macro |\childdocredirect| is a legacy version
% of |\childdocforward| and |\childdocforwardprefix|:
%    \begin{macrocode}
\newcommand{\childdocredirect}[2][]
{
  \begingroup
    \if?#1?
      \def\childdoctmp{\childdocforward{#2}}
    \else
      \def\childdoctmp{\childdocforwardprefix{#1}{#2}}
    \fi
    \expandafter
  \endgroup
  \childdoctmp
}
%    \end{macrocode}

%\iffalse
%</package>
%\fi
%
\endinput

\childdocby{cdocsamp}
%    \end{macrocode}

%\iffalse
%</samplepart3|samplepart4>
%\fi
%
%\iffalse
%<*samplepart3>
%\fi
% Some text for part 3:
%    \begin{macrocode}
some text in part three
%    \end{macrocode}

%\iffalse
%</samplepart3>
%\fi
% Some text for part 4:
%\iffalse
%<*samplepart4>
%\fi
%    \begin{macrocode}
more text in part four
%    \end{macrocode}

%\iffalse
%</samplepart4>
%\fi
%
% %%%%%%%%%%%%%%%%%%%%%%%%%%%%%%%%%%%%%%
% \paragraph{Forwarding for a Complete Draft.}
%
% The following forwarding file |cdocsdrf.tex|
% compiles the main document in draft mode:
%\iffalse
%<*sampledraft>
%\fi
%    \begin{macrocode}
\def\version{draft}
% \iffalse
%
% childdoc.dtx Copyright (C) 2017-2018 Niklas Beisert
%
% This work may be distributed and/or modified under the
% conditions of the LaTeX Project Public License, either version 1.3
% of this license or (at your option) any later version.
% The latest version of this license is in
%   http://www.latex-project.org/lppl.txt
% and version 1.3 or later is part of all distributions of LaTeX
% version 2005/12/01 or later.
%
% This work has the LPPL maintenance status `maintained'.
%
% The Current Maintainer of this work is Niklas Beisert.
%
% This work consists of the files childdoc.dtx and childdoc.ins
% and the derived files childdoc.def and cdocsamp.tex with
% cdocsch1.tex, cdocsch2.tex, cdocsdrf.tex, cdocsfn1.tex, cdocsfn2.tex.
%
%<package>\ifdefined\childdocmain\endinput\fi
%<package>\ProvidesFile{childdoc.def}[2018/12/30 v2.0 child document driver]
%<samplemain>\ProvidesFile{cdocsamp.tex}[2018/12/30 v2.0 sample for childdoc]
%<*driver>
%\ProvidesFile{childdoc.drv}[2018/12/30 v2.0 childdoc reference manual file]
\PassOptionsToClass{10pt,a4paper}{article}
\documentclass{ltxdoc}

\usepackage[margin=35mm]{geometry}
\usepackage{hyperref}
\usepackage{hyperxmp}
\usepackage[usenames]{color}

\hypersetup{colorlinks=true}
\hypersetup{pdfstartview=FitH}
\hypersetup{pdfpagemode=UseNone}
\hypersetup{pdfsource={}}
\hypersetup{pdflang={en-UK}}
\hypersetup{pdfcopyright={Copyright 2017-2018 Niklas Beisert.
  This work may be distributed and/or modified under the
  conditions of the LaTeX Project Public License, either version 1.3
  of this license or (at your option) any later version.}}
\hypersetup{pdflicenseurl={http://www.latex-project.org/lppl.txt}}
\hypersetup{pdfcontactaddress={ETH Zurich, ITP, HIT K,
  Wolfgang-Pauli-Strasse 27}}
\hypersetup{pdfcontactpostcode={8093}}
\hypersetup{pdfcontactcity={Zurich}}
\hypersetup{pdfcontactcountry={Switzerland}}
\hypersetup{pdfcontactemail={nbeisert@itp.phys.ethz.ch}}
\hypersetup{pdfcontacturl={http://people.phys.ethz.ch/\xmptilde nbeisert/}}

\newcommand{\secref}[1]{\hyperref[#1]{section \ref*{#1}}}

\parskip1ex
\parindent0pt
\let\olditemize\itemize
\def\itemize{\olditemize\parskip0pt}

\begin{document}

\title{The \textsf{childdoc} Package}
\hypersetup{pdftitle={The childdoc Package}}
\author{Niklas Beisert\\[2ex]
  Institut f\"ur Theoretische Physik\\
  Eidgen\"ossische Technische Hochschule Z\"urich\\
  Wolfgang-Pauli-Strasse 27, 8093 Z\"urich, Switzerland\\[1ex]
  \href{mailto:nbeisert@itp.phys.ethz.ch}
  {\texttt{nbeisert@itp.phys.ethz.ch}}}
\hypersetup{pdfauthor={Niklas Beisert}}
\hypersetup{pdfsubject={Manual for the LaTeX2e Package childdoc}}
\date{30 December 2018, \textsf{v2.0}}
\maketitle

\begin{abstract}\noindent
\textsf{childdoc} is a \LaTeXe{} package
that enables the direct compilation
of document sections included by |\include|
to individual files.
\end{abstract}

\begingroup
\parskip0ex
\tableofcontents
\endgroup

%%%%%%%%%%%%%%%%%%%%%%%%%%%%%%%%%%%%%%%%%%%%%%%%%%%%%%%%%%%%%%%%%%%%%%%%%%%%%%%%
%%%%%%%%%%%%%%%%%%%%%%%%%%%%%%%%%%%%%%%%%%%%%%%%%%%%%%%%%%%%%%%%%%%%%%%%%%%%%%%%
\section{Introduction}

\LaTeX{} provides a mechanism to structure a large document (such as a book)
into a main file and several child files (containing the chapters)
using the |\include| command.
This mechanism is beneficial for documents
which span hundreds of pages in order to
make the source file(s) more manageable.
Moreover, compilation can be restricted to
selected child files by means of the |\includeonly| command.
The latter feature can be used to reduce the compilation time while editing
(this was significantly more useful in the earlier days of \LaTeX{})
or to generate a smaller document which is easier to navigate.
Another application of |\includeonly| is to generate
documents consisting of selected parts of the complete document.

However, there are a few drawbacks of the plain |\include| mechanism:
\begin{itemize}
\item
The child files cannot be compiled on their own,
they can only be compiled via the main file.
A naive editing environment
(such as a text editor with an option
to have the current file processed by \LaTeX)
may require one to switch to the main file before compiling;
attempting to compile the child file produces errors.
\item
The main file must be modified (each time)
to adjust the |\includeonly| command
to the present needs. This easily leaves the main file in a messy state.
\item
The generated document will always carry the filename
of the main document. This is inconvenient if
several child files are to be compiled and
to be kept for distribution.
\end{itemize}

The present package provides a simple interface
to make child files individually compilable by \LaTeX{}.
Compiling a child file then has the same effect as compiling
the main file with an |\includeonly| command
to select the appropriate child.
Moreover the generated document will carry the name of the child
rather than the main file.
This resolves all three above issues.

This feature is meant to make the editing of books,
thesis documents and lecture notes somewhat more convenient.
However, the package can also be used efficiently for
composing a series of documents (such as exercise sheets)
which are typically distributed individually.
It then assists the author in generating the individual documents
(potentially in different versions)
as well as a document containing the collected series.
Another application is in developing style files
or other kinds of included material
where compilation of the style file could redirect
to a sample or test file.

%%%%%%%%%%%%%%%%%%%%%%%%%%%%%%%%%%%%%%%%%%%%%%%%%%%%%%%%%%%%%%%%%%%%%%%%%%%%%%%%
%%%%%%%%%%%%%%%%%%%%%%%%%%%%%%%%%%%%%%%%%%%%%%%%%%%%%%%%%%%%%%%%%%%%%%%%%%%%%%%%
\section{Usage}

First of all, the package \textsf{childdoc} is \emph{not} a standard
\LaTeXe{} |.sty| style file! Therefore it needs to be invoked in
a non-standard way.

%%%%%%%%%%%%%%%%%%%%%%%%%%%%%%%%%%%%%%%%%%%%%%%%%%%%%%%%%%%%%%%%%%%%%%%%%%%%%%%%
\subsection{Included Files}
\label{sec:include}

%%%%%%%%%%%%%%%%%%%%%%%%%%%%%%%%%%%%%%%%
\DescribeMacro{\childdocmain}
To use the package, add the commands
\begin{center}
\begin{tabular}{l}
|\input{childdoc.def}|\\
|\childdocmain{}|\\
\end{tabular}
\end{center}
at the very top of the main \LaTeX{} file,
in particular \emph{before} the |\documentclass| statement!
The argument of |\childdocmain| should be left empty
(but it must be present).

%%%%%%%%%%%%%%%%%%%%%%%%%%%%%%%%%%%%%%%%
\DescribeMacro{\childdocof}
Furthermore, add the commands
\begin{center}
\begin{tabular}{l}
|\input{childdoc.def}|\\
|\childdocof{|\textit{main}|}|\\
\end{tabular}
\end{center}
at the top of every child file \textit{child}
which is included by |\include{|\textit{child}|}|
from within the main file
(or at least for those files to be compiled individually).
The argument \textit{main} must be the filename of the main file.

There are a couple of
considerations in setting up the main and child documents:

%%%%%%%%%%%%%%%%%%%%%%%%%%%%%%%%%%%%%%%%
\paragraph{Restrictions.}

Please note the following restrictions:
\begin{itemize}
\item
|\childdocmain| must be called with one argument \textit{main}
to ensure compatibility with earlier version of the package.
It must either be empty (|\childdocmain{}|)
or precisely match the filename of the main file in which it is specified.
See \secref{sec:detection} for further information.
\item
The filename \textit{main} must be specified without the |.tex| extension.
\item
The filename \textit{main} is case sensitive
(even in case-insensitive file systems)
due to internal string comparison.
\item
The argument \textit{main} should be fully expanded, it cannot be a macro.
\item
Subdirectories and special characters should be avoided in filenames.
\item
The command |\childdocmain{|\textit{main}|}| must be followed by a whitespace.
It should not be followed immediately by another command
or by a comment mark `|%|'.
This is because the \TeX{} parser reads the token immediately following
the argument of |\childdocmain| and puts it
at the beginning of every child section;
however, a white\-space is ignored.
\end{itemize}

%%%%%%%%%%%%%%%%%%%%%%%%%%%%%%%%%%%%%%%%
\paragraph{Content of Main File.}

It is advisable to place all content in the child files included by |\include|.
Any output contained in the main file will appear in all child documents
unless suppressed manually;
it cannot be suppressed automatically by the |\includeonly| directive
and thus should normally be avoided.
A method to include some content in the main file
by means of conditional processing is described in \secref{sec:conditional}.

%%%%%%%%%%%%%%%%%%%%%%%%%%%%%%%%%%%%%%%%
\paragraph{Page Numbering.}

When only a part of the document is compiled,
the appropriate numbering of pages
(as well as other status parameters)
is determined from the |.aux| files.
The latter contain information from previous passes.
However this information needs to propagate through
all intermediate child documents.
Therefore the page numbering in child documents may well
be inconsistent until the complete document is compiled at least once.

A useful (if unconventional) way to always ensure a consistent
page numbering is to restart the numbering in each child document
and denote the pages by `\textit{child}|.|\textit{page}'
where \textit{child} represents the chapter/section number of the child file.
This can be achieved by the command
|\numberwithin{page}{|\textit{child}|}|
of the \textsf{amsmath} package
where \textit{child} can be |chapter| or |section|
depending on the chosen structuring.
Alternatively, one can modify the macro |\thepage| appropriately
and reset the counter |page| at the start of each child file.

%%%%%%%%%%%%%%%%%%%%%%%%%%%%%%%%%%%%%%%%%%%%%%%%%%%%%%%%%%%%%%%%%%%%%%%%%%%%%%%%
\subsection{Conditional Processing}
\label{sec:conditional}

The package provides a mechanism to compile different versions
of a document. To customise the versions further some conditional processing
can come in handy to distinguish which version is being compiled.
The package provides two macros to describe the compilation context:

%%%%%%%%%%%%%%%%%%%%%%%%%%%%%%%%%%%%%%%%
\DescribeMacro{\ifchilddoc}
The conditional |\ifchilddoc| distinguishes between the compilation of
child documents and the main document:
%
\begin{center}
|\ifchilddoc |\textit{child-code}| |[|\||else |\textit{main-code}]| \||fi|
\end{center}

%%%%%%%%%%%%%%%%%%%%%%%%%%%%%%%%%%%%%%%%
\DescribeMacro{\childdocname}
\DescribeMacro{\childdocjob}
The macro |\childdocname| contains the filename (without extension)
of the main or child file being processed.
Note that |\childdocjob| will always contain the name of the main file.

%%%%%%%%%%%%%%%%%%%%%%%%%%%%%%%%%%%%%%%%
\paragraph{Title Page.}

Conditional processing can be used to include a title or banner page
in the main document when proper precautions are taken.
Importantly, the code in the main file should ensure that the page counter
(as well as other status parameters which are stored in the |.aux| files)
takes the same value after the conditional processing.
Otherwise the page numbers may take divergent values
depending on which part is compiled.

For example, a title page could be declared by:
%
\begin{center}
\begin{tabular}{l}
|\ifchilddoc\||else|\\
|\addtocounter{page}{-1}|\\
\textit{code for title page}\\
|\newpage|\\
|\||fi|
\end{tabular}
\end{center}
%
A banner page for the child documents can be generated by:
%
\begin{center}
\begin{tabular}{l}
|\ifchilddoc|\\
|\addtocounter{page}{-1}|\\
\textit{code for banner page}\\
|\newpage|\\
|\||fi|
\end{tabular}
\end{center}
%
Here one could write a message such as:
\begin{center}
|This is the part \childdocname{} of \childdocjob{}.|
\end{center}

%%%%%%%%%%%%%%%%%%%%%%%%%%%%%%%%%%%%%%%%%%%%%%%%%%%%%%%%%%%%%%%%%%%%%%%%%%%%%%%%
\subsection{Flags}
\label{sec:flags}

The package makes it easy to generate different versions
of the main or child documents.
To this end compilation flags can be defined
and assigned different default values.
They will be particularly useful in conjunction
with the forwarding mechanism described in \secref{sec:forward}.

For example, it may be useful to have a flag |\version|
which can be set to |draft| or |final|.
The document source will contain some conditional code
depending on the value of |\version|.
Suppose further, the flag should default to |final| for the main file
and to |draft| for child files
which is a natural assignment for editing the document.
This is achieved by placing the following code
in the preamble of the main document
(below the |\childdocmain| directive):
%
\begin{center}
\begin{tabular}{l}
|\ifchilddoc|\\
|\providecommand{\version}{draft}|\\
|\||else|\\
|\providecommand{\version}{final}|\\
|\||fi|
\end{tabular}
\end{center}
%
The definition by |\providecommand| makes sure
that previous definitions are not overwritten.
Further statements |\providecommand{\version}{...}|
can thus be added before the above code to override it.

For the main file, one might add a line
(between |\childdocmain| and the above block)
%
\begin{center}
|%\ifchilddoc\||else\providecommand{\version}{draft}\||fi|
\end{center}
%
which can be uncommented to produce a draft version.
Likewise one can add a line to the very top of a child file
(above the |\childdocof{|\textit{main}|}| directive)
%
\begin{center}
|%\providecommand{\version}{final}|
\end{center}
%
which can be uncommented to produce the final version of this child document.

%%%%%%%%%%%%%%%%%%%%%%%%%%%%%%%%%%%%%%%%%%%%%%%%%%%%%%%%%%%%%%%%%%%%%%%%%%%%%%%%
\subsection{Forwarding}
\label{sec:forward}

Different versions of the main or child documents
using compilation flags as described in \secref{sec:flags}
can be (permanently) stored in different files
for convenient compilation, viewing and distribution.
To this end, the package defines a command
to pass on compilation to a different file:

%%%%%%%%%%%%%%%%%%%%%%%%%%%%%%%%%%%%%%%%
\DescribeMacro{\childdocforward}
The command |\childdocforward| redirects processing to
another source file:
%
\begin{center}
\begin{tabular}{l}
|\input{childdoc.def}|\\
|\childdocforward[|\textit{main}|]{|\textit{dest}|}|\\
\end{tabular}
\end{center}
%
The argument \textit{dest} is the destination file
(without extension).
It should be the main file or one of the child files.
Note that further \textsf{childdoc} directives
such as |\childdocof| and |\childdocforward|
in the indicated file will be processed in this form.
The optional argument \textit{main}
passes on directly to the main file \textit{main}
while pretending to compile the child \textit{dest}.
This form behaves as if \textit{dest}
issues |\childdocof{|\textit{main}|}| right away,
and no further \textsf{childdoc} directives will be processed.

%%%%%%%%%%%%%%%%%%%%%%%%%%%%%%%%%%%%%%%%
\DescribeMacro{\...prefix}
In the alternative form |\childdocforwardprefix|,
%
\begin{center}
\begin{tabular}{l}
|\input{childdoc.def}|\\
|\childdocforwardprefix[|\textit{main}|]{|\textit{prefix}|}{|\textit{dest}|}|
\end{tabular}
\end{center}
%
the destination file is determined by a pattern
depending on the current file:
To make this work, the current file must be called
`{\textit{prefix}\hspace{0.2em}\textit{suffix}}'
with \textit{prefix} matching precisely the argument.
Processing is then passed on to the file
`{\textit{dest}\hspace{0.2em}\textit{suffix}}'.
Surely, the same effect is achieved by
directly specifying the
argument `{\textit{dest}\hspace{0.2em}\textit{suffix}}'
in the first form.
However, that requires to set up a different file
for each child. With the alternative form of the command
all these files can have exactly the same content
which simplifies setting them up and maintaining them.

For example, the following file |draft.tex|
with a compilation flag |\version| as described in \secref{sec:flags}
compiles the main document as a draft:
%
\begin{center}
\begin{tabular}{l}
|\def\version{draft}|\\
|\input{childdoc.def}|\\
|\childdocforward{|\textit{main}|}|
\end{tabular}
\end{center}
%
Likewise, the following files |final|\textit{nn}|.tex|
compile the final version of the child document
|child|\textit{nn}|.tex|:
%
\begin{center}
\begin{tabular}{l}
|\def\version{final}|\\
|\input{childdoc.def}|\\
|\childdocforwardprefix{final}{child}|
\end{tabular}
\end{center}
%

Note that when several versions of a main file and/or of each child file
are to be generated, it may be convenient to set up a |Makefile| or
shell script to automatise the process.

%%%%%%%%%%%%%%%%%%%%%%%%%%%%%%%%%%%%%%%%%%%%%%%%%%%%%%%%%%%%%%%%%%%%%%%%%%%%%%%%
\subsection{Command Line Processing}
\label{sec:commandline}

The effect of redirection files can also be achieved by invoking
the \LaTeX{} compiler with a more elaborate command line.
Most conveniently this should be done as part
of a shell script or a |Makefile|.

When using \textsf{childdoc} in the main file, the following
command lines effectively perform a redirection
(note that depending on the shell being used,
backslashes may have to be doubled: `|\|' $\to$ `|\\|'):
%
\begin{center}
|... -jobname "|\textit{target}|" |\\|"|[\textit{flags}]%
|\input{childdoc.def}\childdocforward[|\textit{main}|]{|\textit{dest}|}"|
\end{center}
%
Here \textit{target} is the name of the output file,
\textit{main} is the name of the main file
and \textit{dest} is the name of the main or child file to be processed
(all filenames without extensions).
The optional argument \textit{main} can be omitted
if \textit{main} matches \textit{dest}.
Optionally, compilation \textit{flags} can be defined via |\def| commands.
This command line makes the \TeX{} engine believe
it is compiling the file \textit{target}
whose content is specified as the latter parameter.
The provided code then forwards the processing to
\textit{main} or \textit{dest} as described in \secref{sec:forward}.

%%%%%%%%%%%%%%%%%%%%%%%%%%%%%%%%%%%%%%%%%%%%%%%%%%%%%%%%%%%%%%%%%%%%%%%%%%%%%%%%
\subsection{Include by Input}
\label{sec:input}

Including child documents by |\include| has some restrictions by design.
Most notably, the content of a child document always occupies
its own set of pages; pages cannot be shared between child documents.
Usually, this behaviour makes perfect sense
because each child document contain an essential part of the document.
However, in some situations it may be desirable to compose
a document from a collection of parts
without having mandatory page breaks between then.
For this case, the package
provides a mechanism to include parts
by |\input| which can also be processed individually.
However, by construction this mechanism
requires manual handling of the content to be output.

%%%%%%%%%%%%%%%%%%%%%%%%%%%%%%%%%%%%%%%%
\DescribeMacro{\ifchilddocmanual}
The main file should be prepared as usual, see \secref{sec:include}.
However, the document body must make a distinction
between processing of an individual part and of the main document, e.g.:
%
\begin{center}
\begin{tabular}{l}
|\ifchilddocmanual|\\
|\input{\childdocname}|\\
|\||else|\\
\textit{document body with }|\input{|\textit{part}|}|\\
|\||fi|
\end{tabular}
\end{center}
%
The conditional |\ifchilddocmanual| is true whenever
a part to be included by |\input| is being compiled,
and the name of the part is stored in |\childdocname|.

%%%%%%%%%%%%%%%%%%%%%%%%%%%%%%%%%%%%%%%%
\DescribeMacro{\childdocby}
Each part to be included by |\input| should start with:
%
\begin{center}
\begin{tabular}{l}
|\input{childdoc.def}|\\
|\childdocby{|\textit{main}|}|\\
\end{tabular}
\end{center}
%
The directive |\childdocby| is similar to |\childdocof|
described in \secref{sec:include},
but the subsequent selection of content must be done manually.
To that end, both |\ifchilddoc| and |\ifchilddocmanual|
will be true upon processing of a part,
and the name of the part is stored in |\childdocname|.
Note that |\jobname| will be set to the filename of the current part
so that each part receives an individual |.aux| file
that does not interfere with the |.aux| file(s) of the main document.
This behaviour can be altered by the alternative form
|\childdocby[*]{|\textit{main}|}| (with a non-empty optional argument)
which uses the |.aux| file of the main document
by setting |\jobname| to \textit{main}.

%%%%%%%%%%%%%%%%%%%%%%%%%%%%%%%%%%%%%%%%%%%%%%%%%%%%%%%%%%%%%%%%%%%%%%%%%%%%%%%%
\subsection{Driver Development}
\label{sec:driver}

The \textsf{childdoc} mechanism can also be use for the development
of definition files such as \LaTeX{} styles or classes.
This case differs from the above setup with multiple parts
included by |\include| in that no |\includeonly| should be invoked.
This can be achieved by starting the include file
(before |\ProvidesPackage|) with:
%
\begin{center}
\begin{tabular}{l}
|\input{childdoc.def}|\\
|\childdocforward{|\textit{main}|}|\\
\end{tabular}
\end{center}
%
or alternatively with:
%
\begin{center}
\begin{tabular}{l}
|\input{childdoc.def}|\\
|\childdocby{|\textit{main}|}|\\
\end{tabular}
\end{center}
%
Both forms have slightly different effects as described above.
The main file is prepared as usual, see \secref{sec:include}.

%%%%%%%%%%%%%%%%%%%%%%%%%%%%%%%%%%%%%%%%%%%%%%%%%%%%%%%%%%%%%%%%%%%%%%%%%%%%%%%%
\subsection{Legacy Detection}
\label{sec:detection}

The directive |\childdocmain| in the main file can detect
whether the complete document or merely a child is to be compiled
even without using the directive |\childdocof|.
This method is deprecated because it is less robust
and there is no compelling reason to use it;
it is merely provided for backward compatibility
and it may be removed in future versions.

If the detection mechanism is to be used,
it is mandatory to correctly specify
the filename of the main file as the argument of |\childdocmain|:
%
\begin{center}
\begin{tabular}{l}
|\input{childdoc.def}|\\
|\childdocmain{|\textit{main}|}|\\
\end{tabular}
\end{center}
%
If |\jobname| does not match the argument \textit{main} of |\childdocmain|,
it is assumed that |\jobname| points to the child file to be compiled.
When using |\childdocmain| with the main file specified as argument,
it suffices to start a child file
with just |\input{|\textit{main}|}|
without loading of the package and using |\childdocof|.
If instead all processing is done
with the appropriate \textsf{childdoc} directives,
the argument of \textit{main} of |\childdocmain| can be empty.

An alternative version of the command line processing described
in \secref{sec:commandline} using the detection mechanism reads:
%
\begin{center}
|... -jobname "|\textit{target}|" "|[\textit{flags}]%
[|\def\jobname{|\textit{dest}|}|]|\input{|\textit{main}|}"|
\end{center}

%%%%%%%%%%%%%%%%%%%%%%%%%%%%%%%%%%%%%%%%%%%%%%%%%%%%%%%%%%%%%%%%%%%%%%%%%%%%%%%%
\subsection{Manual Code}
\label{sec:manual}

In case one cannot be certain whether the definitions file |childdoc.def|
is installed on the target \TeX{} distribution
and one prefers not to ship it,
it is conceivable to paste a few relevant commands into the sources.

To that end, drop all statements |\input{childdoc.def}|
and perform the replacements as outlined below.
Instead of |\childdocmain{|\textit{main}|}| add the following code
to the top of the main file:
%
\begin{center}
\begin{tabular}{l}
|\||ifdefined\childdocname\endinput\||fi\newif\ifchilddoc|\\
|\edef\childdocname{\scantokens\expandafter{\jobname\noexpand}}|\\
|\def\childdocmain{|\textit{main}|}\||ifx\childdocmain\childdocname\||else|\\
|\childdoctrue\includeonly{\childdocname}\let\jobname\childdocmain\||fi|\\
\end{tabular}
\end{center}
%
Instead of |\childdocof{|\textit{main}|}| just include the main file
at the top of each child file:
%
\begin{center}
|\input{|\textit{main}|}|
\end{center}
%
A simple redirection |\childdocforward{|\textit{dest}|}| is achieved by:
%
\begin{center}
|\def\jobname{|\textit{dest}|}\input{\jobname}|
\end{center}
%
The redirection with prefix
|\childdocforwardprefix[|\textit{prefix}|]{|\textit{dest}|}|
is accomplished by:
%
\begin{center}
\begin{tabular}{l}
|{\edef\jobname{\scantokens\expandafter{\jobname\noexpand}}|\\
|\def\redirectjob |\textit{prefix}|#1~~~{\gdef\jobname{|\textit{dest}|#1}}|\\
|\expandafter\redirectjob\jobname~~~}\input{\jobname}|
\end{tabular}
\end{center}

In an alternative approach,
child documents can be compiled by a specific command line
without additional code or specific definitions:
%
\begin{center}
|... -jobname "|\textit{target}|" "|[\textit{flags}]%
|\includeonly{|\textit{dest}|}\input{|\textit{main}|}"|
\end{center}
%

%%%%%%%%%%%%%%%%%%%%%%%%%%%%%%%%%%%%%%%%%%%%%%%%%%%%%%%%%%%%%%%%%%%%%%%%%%%%%%%%
%%%%%%%%%%%%%%%%%%%%%%%%%%%%%%%%%%%%%%%%%%%%%%%%%%%%%%%%%%%%%%%%%%%%%%%%%%%%%%%%
\section{Information}

%%%%%%%%%%%%%%%%%%%%%%%%%%%%%%%%%%%%%%%%%%%%%%%%%%%%%%%%%%%%%%%%%%%%%%%%%%%%%%%%
\subsection{Copyright}

Copyright \copyright{} 2017--2018 Niklas Beisert

This work may be distributed and/or modified under the
conditions of the \LaTeX{} Project Public License, either version 1.3
of this license or (at your option) any later version.
The latest version of this license is in
  \url{http://www.latex-project.org/lppl.txt}
and version 1.3 or later is part of all distributions of \LaTeX{}
version 2005/12/01 or later.

This work has the LPPL maintenance status `maintained'.

The Current Maintainer of this work is Niklas Beisert.

This work consists of the files |README.txt|, |childdoc.ins| and |childdoc.dtx|
as well as the derived files |childdoc.def|, |cdocsamp.tex|
with |cdocsch1.tex|, |cdocsch2.tex|, |cdocspt3.tex|, |cdocspt4.tex|,
|cdocsdrf.tex|, |cdocsfn1.tex|, |cdocsfn2.tex|
as well as |childdoc.pdf|.

%%%%%%%%%%%%%%%%%%%%%%%%%%%%%%%%%%%%%%%%%%%%%%%%%%%%%%%%%%%%%%%%%%%%%%%%%%%%%%%%
\subsection{Files and Installation}

The package consists of the files:
%
\begin{center}
\begin{tabular}{ll}
    |README.txt|   & readme file \\
    |childdoc.ins| & installation file \\
    |childdoc.dtx| & source file \\
    |childdoc.def| & definition file \\
    |cdocsamp.tex| & sample main file \\
    |cdocsch1.tex| & sample include file \\
    |cdocsch2.tex| & sample include file \\
    |cdocspt3.tex| & sample part file \\
    |cdocspt4.tex| & sample part file \\
    |cdocsdrf.tex| & sample redirection file \\
    |cdocsfn1.tex| & sample redirection file \\
    |cdocsfn2.tex| & sample redirection file \\
    |childdoc.pdf| & manual
\end{tabular}
\end{center}
%
The distribution consists of the files
|README.txt|, |childdoc.ins| and |childdoc.dtx|.
%
\begin{itemize}
\item
Run (pdf)\LaTeX{} on |childdoc.dtx|
to compile the manual |childdoc.pdf| (this file).
\item
Run \LaTeX{} on |childdoc.ins| to create the definitions file |childdoc.def|
and the sample |cdocsamp.tex| with include files
|cdocsch1.tex|, |cdocsch2.tex|, |cdocspt3.tex|, |cdocspt4.tex|,
|cdocsdrf.tex|, |cdocsfn1.tex|, |cdocsfn2.tex|.
Then copy the file |childdoc.def| to an appropriate directory of your \LaTeX{}
distribution, e.g.\ \textit{texmf-root}|/tex/latex/childdoc|.
\end{itemize}

%%%%%%%%%%%%%%%%%%%%%%%%%%%%%%%%%%%%%%%%%%%%%%%%%%%%%%%%%%%%%%%%%%%%%%%%%%%%%%%%
\subsection{Related CTAN Packages}

There are several other packages which offer a similar functionality:
%
\begin{itemize}
\item
The packages
\href{http://ctan.org/pkg/docmute}{\textsf{docmute}},
\href{http://ctan.org/pkg/includex}{\textsf{includex}} and
\href{http://ctan.org/pkg/standalone}{\textsf{standalone}}
provide commands to include only the document body of
a child file thus allowing both files to be compiled individually.
\item
The packages \href{http://ctan.org/pkg/subdocs}{\textsf{subdocs}}
and \href{http://ctan.org/pkg/subfiles}{\textsf{subfiles}}
provide structures in which the main and child documents can be
encapsulated and allowing them to be compiled individually.
The inclusion mechanism is different from the conventional |\include|.
\item
The package \href{http://ctan.org/pkg/combine}{\textsf{combine}}
is an elaborate solution to combine several documents into one.
\end{itemize}
%
See also the CTAN topic \href{http://ctan.org/topic/subdocs}{\textsf{subdocs}}
for further related packages.
The present package differs from the above solutions in that
a document structure constructed with the conventional |\include| mechanism
just needs two extra commands at the top of every file
such that all constituent files can be compiled individually.

%%%%%%%%%%%%%%%%%%%%%%%%%%%%%%%%%%%%%%%%%%%%%%%%%%%%%%%%%%%%%%%%%%%%%%%%%%%%%%%%
%\subsection{Feature Suggestions}
%
%The following is a list of features which may be useful for future
%versions of this package:
%%
%\begin{itemize}
%\item
%\ldots
%\end{itemize}

%%%%%%%%%%%%%%%%%%%%%%%%%%%%%%%%%%%%%%%%%%%%%%%%%%%%%%%%%%%%%%%%%%%%%%%%%%%%%%%%
\subsection{Revision History}

%%%%%%%%%%%%%%%%%%%%%%%%%%%%%%%%%%%%%%%%
\paragraph{v2.0:} 2018/12/30

\begin{itemize}
\item
immediate forward processing
\item
added |\childdocby| mechanism
\item
manual restructured
\end{itemize}

%%%%%%%%%%%%%%%%%%%%%%%%%%%%%%%%%%%%%%%%
\paragraph{v1.6:} 2018/01/17

\begin{itemize}
\item
application for development of include files
\item
corrections to manual
\end{itemize}

%%%%%%%%%%%%%%%%%%%%%%%%%%%%%%%%%%%%%%%%
\paragraph{v1.5:} 2017/05/21

\begin{itemize}
\item
more complete structuring introduced
\item
|\childdocof| introduced
\item
|\childdoc| renamed to |\childdocmain|
\item
|\childredirect| renamed to |\childdocforward| and |\childdocforwardprefix|
and functionality expanded
\end{itemize}

%%%%%%%%%%%%%%%%%%%%%%%%%%%%%%%%%%%%%%%%
\paragraph{v1.0:} 2017/04/27

\begin{itemize}
\item
manual and install package
\item
first version published on CTAN
\end{itemize}

%%%%%%%%%%%%%%%%%%%%%%%%%%%%%%%%%%%%%%%%
\paragraph{v0.6:} 2017/04/26

\begin{itemize}
\item
redirection mechanism added
\end{itemize}

%%%%%%%%%%%%%%%%%%%%%%%%%%%%%%%%%%%%%%%%
\paragraph{v0.5:} 2017/04/26

\begin{itemize}
\item
functionality in definition file
\end{itemize}


%%%%%%%%%%%%%%%%%%%%%%%%%%%%%%%%%%%%%%%%%%%%%%%%%%%%%%%%%%%%%%%%%%%%%%%%%%%%%%%%
%%%%%%%%%%%%%%%%%%%%%%%%%%%%%%%%%%%%%%%%%%%%%%%%%%%%%%%%%%%%%%%%%%%%%%%%%%%%%%%%
%%%%%%%%%%%%%%%%%%%%%%%%%%%%%%%%%%%%%%%%%%%%%%%%%%%%%%%%%%%%%%%%%%%%%%%%%%%%%%%%
\appendix

\settowidth\MacroIndent{\rmfamily\scriptsize 000\ }

 \DocInput{childdoc.dtx}

\end{document}
%</driver>
% \fi
%
% %%%%%%%%%%%%%%%%%%%%%%%%%%%%%%%%%%%%%%%%%%%%%%%%%%%%%%%%%%%%%%%%%%%%%%%%%%%%%%
% %%%%%%%%%%%%%%%%%%%%%%%%%%%%%%%%%%%%%%%%%%%%%%%%%%%%%%%%%%%%%%%%%%%%%%%%%%%%%%
% \section{Sample}
%\iffalse
%<*samplemain>
%\fi
%
% The following presents a sample document
% with two chapters, two parts, a title page,
% a compile flag as well as three forwarding files to set the flag.
% It consists of eight |.tex| files:
% \begin{center}
% \begin{tabular}{ll}
% |cdocsamp.tex|&main file\\
% |cdocsch1.tex|&include file for chapter 1\\
% |cdocsch2.tex|&include file for chapter 2\\
% |cdocspt3.tex|&include file for part 3\\
% |cdocspt4.tex|&include file for part 4\\
% |cdocsdrf.tex|&forwarding file for main file in draft mode\\
% |cdocsfi1.tex|&forwarding file for final version of chapter 1\\
% |cdocsfi2.tex|&forwarding file for final version of chapter 2\\
% \end{tabular}
% \end{center}
% Each of the eight files can be compiled directly by the \LaTeX{} compiler.
%
% %%%%%%%%%%%%%%%%%%%%%%%%%%%%%%%%%%%%%%
% \paragraph{Main File.}
%
% The main file is called |cdocsamp.tex|.
%
% Load the \textsf{childdoc} definitions and
% declare the filename for the main document:
%    \begin{macrocode}
\input{childdoc.def}
\childdocmain{}
%    \end{macrocode}

% Optional override for |\version| flag:
%    \begin{macrocode}
%%\ifchilddoc\else\providecommand{\version}{draft}\fi
%    \end{macrocode}

% Define the default values for the |\version| flag
% (|final| for the main file and |draft| for childs):
%    \begin{macrocode}
\ifchilddoc
\providecommand{\version}{draft}
\else
\providecommand{\version}{final}
\fi
%    \end{macrocode}

% Load the standard document class:
%    \begin{macrocode}
\documentclass[12pt]{article}
%    \end{macrocode}

% Start the document body:
%    \begin{macrocode}
\begin{document}
%    \end{macrocode}

% Declare a title page.
% Print title, part of document being processed and version flag:
%    \begin{macrocode}
\addtocounter{page}{-1}
\begin{center}
{\LARGE\bfseries{}childdoc example\par}
\vspace{1cm}
\ifchilddoc
\ifchilddocmanual part\else chapter\fi:
`\childdocname' of `\childdocjob'\par
\else
main document: `\childdocjob'\par
\fi
version: \version\par
\end{center}
\newpage
%    \end{macrocode}

% Manually include selected file,
% otherwise process as usual:
%    \begin{macrocode}
\ifchilddocmanual
\section*{part `\childdocname'}
\input{\childdocname}
\else
%    \end{macrocode}

% Include the two chapters:
%    \begin{macrocode}
\include{cdocsch1}
\include{cdocsch2}
%    \end{macrocode}

% Include the two parts unless only chapters should be displayed:
%    \begin{macrocode}
\ifchilddoc\else
\section{part three}
\input{cdocspt3}
\section{part four}
\input{cdocspt4}
\fi
%    \end{macrocode}

% Process as usual until here:
%    \begin{macrocode}
\fi
%    \end{macrocode}

% End of document body:
%    \begin{macrocode}
\end{document}
%    \end{macrocode}
%\iffalse
%</samplemain>
%\fi
%
% %%%%%%%%%%%%%%%%%%%%%%%%%%%%%%%%%%%%%%
% \paragraph{Chapter Include Files.}
%
% The include files are called |cdocsch1.tex| and |cdocsch2.tex|.
%
%\iffalse
%<*samplechap1|samplechap2>
%\fi

% Optional override for |\version| flag:
%    \begin{macrocode}
%%\providecommand{\version}{final}
%    \end{macrocode}

% Include the main document:
%    \begin{macrocode}
\input{childdoc.def}
\childdocof{cdocsamp}
%    \end{macrocode}

%\iffalse
%</samplechap1|samplechap2>
%\fi
%
%\iffalse
%<*samplechap1>
%\fi
% Some text for chapter 1:
%    \begin{macrocode}
\section{one}
some text in chapter one
%    \end{macrocode}

%\iffalse
%</samplechap1>
%\fi
% Some text for chapter 2:
%\iffalse
%<*samplechap2>
%\fi
%    \begin{macrocode}
\section{two}
more text in chapter two
%    \end{macrocode}

%\iffalse
%</samplechap2>
%\fi
%
% %%%%%%%%%%%%%%%%%%%%%%%%%%%%%%%%%%%%%%
% \paragraph{Part Include Files.}
%
% The include files are called |cdocspt3.tex| and |cdocspt4.tex|.
%
%\iffalse
%<*samplepart3|samplepart4>
%\fi

% Optional override for |\version| flag:
%    \begin{macrocode}
%%\providecommand{\version}{final}
%    \end{macrocode}

% Include the main document:
%    \begin{macrocode}
\input{childdoc.def}
\childdocby{cdocsamp}
%    \end{macrocode}

%\iffalse
%</samplepart3|samplepart4>
%\fi
%
%\iffalse
%<*samplepart3>
%\fi
% Some text for part 3:
%    \begin{macrocode}
some text in part three
%    \end{macrocode}

%\iffalse
%</samplepart3>
%\fi
% Some text for part 4:
%\iffalse
%<*samplepart4>
%\fi
%    \begin{macrocode}
more text in part four
%    \end{macrocode}

%\iffalse
%</samplepart4>
%\fi
%
% %%%%%%%%%%%%%%%%%%%%%%%%%%%%%%%%%%%%%%
% \paragraph{Forwarding for a Complete Draft.}
%
% The following forwarding file |cdocsdrf.tex|
% compiles the main document in draft mode:
%\iffalse
%<*sampledraft>
%\fi
%    \begin{macrocode}
\def\version{draft}
\input{childdoc.def}
\childdocforward{cdocsamp}
%    \end{macrocode}

%\iffalse
%</sampledraft>
%\fi
%
% %%%%%%%%%%%%%%%%%%%%%%%%%%%%%%%%%%%%%%
% \paragraph{Forwarding for Final Version of the Chapters.}
%
% The following forwarding files |cdocsfn1.tex| and |cdocsfn2.tex|
% (with identical content)
% compile the final versions of the child documents
% |cdocsch1.tex| and |cdocsch2.tex|, respectively:
%\iffalse
%<*samplefinal>
%\fi
%    \begin{macrocode}
\def\version{final}
\input{childdoc.def}
\childdocforwardprefix[cdocsamp]{cdocsfn}{cdocsch}
%    \end{macrocode}

%\iffalse
%</samplefinal>
%\fi
%
% %%%%%%%%%%%%%%%%%%%%%%%%%%%%%%%%%%%%%%
% \paragraph{Command Line Processing.}
%
% The following three command lines generate the output files
% |cdocscld|, |cdocscl1| and |cdocscl2|
% which should be identical to
% |cdocsdrf|, |cdocsch1| and |cdocsfn2|, respectively:
% \begin{center}
% \begin{tabular}{l}
% |latex -jobname cdocscld \|\\
% |  "\def\version{draft}\input{childdoc.def}\childdocforward{cdocsamp}"|\\
% |latex -jobname cdocscl1 \|\\
% |  "\input{childdoc.def}\childdocforward[cdocsamp]{cdocsch1}"|\\
% |latex -jobname cdocscl2 \|\\
% |  "\def\version{final}\input{childdoc.def}\childdocforward{cdocsch2}"|
% \end{tabular}
% \end{center}
% Note that the trailing backslash on each first line
% merely continues the input to the second line
% (for convenient cut ant paste).
% Furthermore, the command |latex| can be replaced by any
% of its alternative versions such as |pdflatex|.
%
% %%%%%%%%%%%%%%%%%%%%%%%%%%%%%%%%%%%%%%%%%%%%%%%%%%%%%%%%%%%%%%%%%%%%%%%%%%%%%%
% %%%%%%%%%%%%%%%%%%%%%%%%%%%%%%%%%%%%%%%%%%%%%%%%%%%%%%%%%%%%%%%%%%%%%%%%%%%%%%
% \section{Implementation}
%\iffalse
%<*package>
%\fi
%
% This section describes the definitions file |childdoc.def|.

% The definitions cannot be loaded using |\usepackage| or |\RequirePackage|
% which has a mechanism to prevent loading a style file more than once.
% When loading the definitions by means of |\input|
% multiple instances have to be prevented manually:
%\iffalse
%This code needs to be before the `\ProvidesFile' directive
%which is defined at the beginning of this file.
%Therefore it is also placed there and commented out here.
%</package>
%<*discard>
%\fi
%    \begin{macrocode}
\ifdefined\childdocmain\endinput\fi
%    \end{macrocode}
%\iffalse
%</discard>
%<*package>
%\fi
%
% \macro{\ifchilddoc}
% \macro{\ifchilddocmanual}
% The conditional |\ifchilddoc| tells whether a
% child (true) or main (false) document is being compiled.
% The conditional |\ifchilddocmanual| tells whether
% the |\includeonly| mechanism is used (false) or
% the selection of child files must be performed manually (true).
% The definitions initialise to false:
%    \begin{macrocode}
\newif\ifchilddoc
\newif\ifchilddocmanual
%    \end{macrocode}

% \macro{\childdocname}
% \macro{\childdocjob}
% The macro |\childdocname| stores the name of the main document
% to be compiled. The macro |\childdocjob| stores the name of
% the document on which the \LaTeX{} compiler was originally invoked.
% The content of |\jobname| cannot be compared
% to filenames specified in the source due to different catcodes.
% The following code rescans |\jobname|, stores the result
% in |\childdocname| and saves a copy in |\childdocjob|:
%    \begin{macrocode}
\edef\childdocname{\scantokens\expandafter{\jobname\noexpand}}
\let\childdocjob\childdocname
%    \end{macrocode}

% \macro{\childdocdisable}
% The macro |\childdocdisable| prevents the main file
% from being processed more than once.
% At this stage, the main document command |\childdocmain|
% is assumed to be called once again where it should do nothing.
% Any subsequent call to it should prevent
% a secondary processing of the main document
% It overwrites the forwarding commands
% |\childdocof| and |\childdocforward|
% with empty macros to prevent further inclusions of the main document:
%    \begin{macrocode}
\newcommand{\childdocdisable}
{
  \renewcommand{\childdocmain}[1]{\renewcommand{\childdocmain}[1]{\endinput}}
  \renewcommand{\childdocof}[1]{}
  \renewcommand{\childdocby}[2][]{}
  \renewcommand{\childdocforward}[2][]{}
  \renewcommand{\childdocdisable}{}
}
%    \end{macrocode}

% \macro{\childdocmain}
% The macro |\childdocmain| is to be called at the top of the main file
% with nothing or the main filename (without extension) as argument.
% First, it breaks loops.
% If the argument is not empty and does not match |\childdocname|
% (which is set by the first inclusion of |childdoc.def|),
% |\ifchilddoc| is set to true, |\includeonly| is applied to the child file
% and |\jobname| is set to the main file
% (for proper handling of |.aux| files):
%    \begin{macrocode}
\newcommand{\childdocmain}[1]
{
  \childdocdisable\childdocmain{}
  \if?#1?\else
    \begingroup
      \def\childdoctmp{#1}
      \ifx\childdoctmp\childdocname
        \def\childdoctmp{}
      \else
        \def\childdoctmp
        {
          \childdoctrue
          \includeonly{\childdocname}
          \def\childdocjob{#1}
          \def\jobname{#1}
        }
      \fi
      \expandafter
    \endgroup
    \childdoctmp
  \fi
}
%    \end{macrocode}

% \macro{\childdocof}
% The command |\childdocof| redirects
% compilation to the main file |#1|.
%    \begin{macrocode}
\newcommand{\childdocof}[1]
{
  \childdocdisable
  \childdoctrue
  \includeonly{\childdocname}
  \def\jobname{#1}
  \def\childdocjob{#1}
  \input{#1}
}
%    \end{macrocode}

% \macro{\childdocby}
% The command |\childdocby| ....
%    \begin{macrocode}
\newcommand{\childdocby}[2][]
{
  \childdocdisable
  \childdoctrue
  \childdocmanualtrue
  \if?#1?\else
    \def\jobname{#2}
  \fi
  \def\childdocjob{#2}
  \input{#2}
  \endinput
}
%    \end{macrocode}

% \macro{\childdocforward}
% The command |\childdocforward| redirects
% compilation to the main file or
% (if the optional argument is given) a child file.
% Parameters are set as if the main file
% or a child file starting with |\childdocof| was compiled.
% Then compilation is handed over to the main file:
%    \begin{macrocode}
\newcommand{\childdocforward}[2][]
{
  \begingroup
    \if?#1?
      \def\childdoctmp
      {
        \def\childdocname{#2}
        \def\childdocjob{#2}
        \def\jobname{#2}
        \input{#2}
        \endinput
      }
    \else
      \def\childdoctmp
      {
        \childdocdisable
        \def\childdocname{#2}
        \childdoctrue
        \includeonly{#2}
        \def\childdocjob{#1}
        \def\jobname{#1}
        \input{#1}
        \endinput
      }
    \fi
    \expandafter
  \endgroup
  \childdoctmp
}
%    \end{macrocode}

% \macro{\childdocforwardprefix}
% The command |\childdocforwardprefix| redirects
% compilation to the main or a child file by means of a pattern.
% The prefix |#1| in the current filename is replaced by |#2|
% and the suffix of the current filename is kept
% (it is assumed that the filename does not contain the substring `|~~~|'
% which is used as a delimiter).
% Compilation is handed over to the new file by |\childdocforward|:
%    \begin{macrocode}
\newcommand{\childdocforwardprefix}[3][]
{
  \begingroup
    \def\childdocextract #2##1~~~{\def\childdoctmp{\childdocforward[#1]{#3##1}}}
    \expandafter\childdocextract\childdocname~~~
    \expandafter
  \endgroup
  \childdoctmp
}
%    \end{macrocode}

% \macro{\childdoc}
% The deprecated macro |\childdoc| is a legacy version of |\childdocmain|:
%    \begin{macrocode}
\newcommand{\childdoc}{\childdocmain}
%    \end{macrocode}

% \macro{\childdocredirect}
% The deprecated macro |\childdocredirect| is a legacy version
% of |\childdocforward| and |\childdocforwardprefix|:
%    \begin{macrocode}
\newcommand{\childdocredirect}[2][]
{
  \begingroup
    \if?#1?
      \def\childdoctmp{\childdocforward{#2}}
    \else
      \def\childdoctmp{\childdocforwardprefix{#1}{#2}}
    \fi
    \expandafter
  \endgroup
  \childdoctmp
}
%    \end{macrocode}

%\iffalse
%</package>
%\fi
%
\endinput

\childdocforward{cdocsamp}
%    \end{macrocode}

%\iffalse
%</sampledraft>
%\fi
%
% %%%%%%%%%%%%%%%%%%%%%%%%%%%%%%%%%%%%%%
% \paragraph{Forwarding for Final Version of the Chapters.}
%
% The following forwarding files |cdocsfn1.tex| and |cdocsfn2.tex|
% (with identical content)
% compile the final versions of the child documents
% |cdocsch1.tex| and |cdocsch2.tex|, respectively:
%\iffalse
%<*samplefinal>
%\fi
%    \begin{macrocode}
\def\version{final}
% \iffalse
%
% childdoc.dtx Copyright (C) 2017-2018 Niklas Beisert
%
% This work may be distributed and/or modified under the
% conditions of the LaTeX Project Public License, either version 1.3
% of this license or (at your option) any later version.
% The latest version of this license is in
%   http://www.latex-project.org/lppl.txt
% and version 1.3 or later is part of all distributions of LaTeX
% version 2005/12/01 or later.
%
% This work has the LPPL maintenance status `maintained'.
%
% The Current Maintainer of this work is Niklas Beisert.
%
% This work consists of the files childdoc.dtx and childdoc.ins
% and the derived files childdoc.def and cdocsamp.tex with
% cdocsch1.tex, cdocsch2.tex, cdocsdrf.tex, cdocsfn1.tex, cdocsfn2.tex.
%
%<package>\ifdefined\childdocmain\endinput\fi
%<package>\ProvidesFile{childdoc.def}[2018/12/30 v2.0 child document driver]
%<samplemain>\ProvidesFile{cdocsamp.tex}[2018/12/30 v2.0 sample for childdoc]
%<*driver>
%\ProvidesFile{childdoc.drv}[2018/12/30 v2.0 childdoc reference manual file]
\PassOptionsToClass{10pt,a4paper}{article}
\documentclass{ltxdoc}

\usepackage[margin=35mm]{geometry}
\usepackage{hyperref}
\usepackage{hyperxmp}
\usepackage[usenames]{color}

\hypersetup{colorlinks=true}
\hypersetup{pdfstartview=FitH}
\hypersetup{pdfpagemode=UseNone}
\hypersetup{pdfsource={}}
\hypersetup{pdflang={en-UK}}
\hypersetup{pdfcopyright={Copyright 2017-2018 Niklas Beisert.
  This work may be distributed and/or modified under the
  conditions of the LaTeX Project Public License, either version 1.3
  of this license or (at your option) any later version.}}
\hypersetup{pdflicenseurl={http://www.latex-project.org/lppl.txt}}
\hypersetup{pdfcontactaddress={ETH Zurich, ITP, HIT K,
  Wolfgang-Pauli-Strasse 27}}
\hypersetup{pdfcontactpostcode={8093}}
\hypersetup{pdfcontactcity={Zurich}}
\hypersetup{pdfcontactcountry={Switzerland}}
\hypersetup{pdfcontactemail={nbeisert@itp.phys.ethz.ch}}
\hypersetup{pdfcontacturl={http://people.phys.ethz.ch/\xmptilde nbeisert/}}

\newcommand{\secref}[1]{\hyperref[#1]{section \ref*{#1}}}

\parskip1ex
\parindent0pt
\let\olditemize\itemize
\def\itemize{\olditemize\parskip0pt}

\begin{document}

\title{The \textsf{childdoc} Package}
\hypersetup{pdftitle={The childdoc Package}}
\author{Niklas Beisert\\[2ex]
  Institut f\"ur Theoretische Physik\\
  Eidgen\"ossische Technische Hochschule Z\"urich\\
  Wolfgang-Pauli-Strasse 27, 8093 Z\"urich, Switzerland\\[1ex]
  \href{mailto:nbeisert@itp.phys.ethz.ch}
  {\texttt{nbeisert@itp.phys.ethz.ch}}}
\hypersetup{pdfauthor={Niklas Beisert}}
\hypersetup{pdfsubject={Manual for the LaTeX2e Package childdoc}}
\date{30 December 2018, \textsf{v2.0}}
\maketitle

\begin{abstract}\noindent
\textsf{childdoc} is a \LaTeXe{} package
that enables the direct compilation
of document sections included by |\include|
to individual files.
\end{abstract}

\begingroup
\parskip0ex
\tableofcontents
\endgroup

%%%%%%%%%%%%%%%%%%%%%%%%%%%%%%%%%%%%%%%%%%%%%%%%%%%%%%%%%%%%%%%%%%%%%%%%%%%%%%%%
%%%%%%%%%%%%%%%%%%%%%%%%%%%%%%%%%%%%%%%%%%%%%%%%%%%%%%%%%%%%%%%%%%%%%%%%%%%%%%%%
\section{Introduction}

\LaTeX{} provides a mechanism to structure a large document (such as a book)
into a main file and several child files (containing the chapters)
using the |\include| command.
This mechanism is beneficial for documents
which span hundreds of pages in order to
make the source file(s) more manageable.
Moreover, compilation can be restricted to
selected child files by means of the |\includeonly| command.
The latter feature can be used to reduce the compilation time while editing
(this was significantly more useful in the earlier days of \LaTeX{})
or to generate a smaller document which is easier to navigate.
Another application of |\includeonly| is to generate
documents consisting of selected parts of the complete document.

However, there are a few drawbacks of the plain |\include| mechanism:
\begin{itemize}
\item
The child files cannot be compiled on their own,
they can only be compiled via the main file.
A naive editing environment
(such as a text editor with an option
to have the current file processed by \LaTeX)
may require one to switch to the main file before compiling;
attempting to compile the child file produces errors.
\item
The main file must be modified (each time)
to adjust the |\includeonly| command
to the present needs. This easily leaves the main file in a messy state.
\item
The generated document will always carry the filename
of the main document. This is inconvenient if
several child files are to be compiled and
to be kept for distribution.
\end{itemize}

The present package provides a simple interface
to make child files individually compilable by \LaTeX{}.
Compiling a child file then has the same effect as compiling
the main file with an |\includeonly| command
to select the appropriate child.
Moreover the generated document will carry the name of the child
rather than the main file.
This resolves all three above issues.

This feature is meant to make the editing of books,
thesis documents and lecture notes somewhat more convenient.
However, the package can also be used efficiently for
composing a series of documents (such as exercise sheets)
which are typically distributed individually.
It then assists the author in generating the individual documents
(potentially in different versions)
as well as a document containing the collected series.
Another application is in developing style files
or other kinds of included material
where compilation of the style file could redirect
to a sample or test file.

%%%%%%%%%%%%%%%%%%%%%%%%%%%%%%%%%%%%%%%%%%%%%%%%%%%%%%%%%%%%%%%%%%%%%%%%%%%%%%%%
%%%%%%%%%%%%%%%%%%%%%%%%%%%%%%%%%%%%%%%%%%%%%%%%%%%%%%%%%%%%%%%%%%%%%%%%%%%%%%%%
\section{Usage}

First of all, the package \textsf{childdoc} is \emph{not} a standard
\LaTeXe{} |.sty| style file! Therefore it needs to be invoked in
a non-standard way.

%%%%%%%%%%%%%%%%%%%%%%%%%%%%%%%%%%%%%%%%%%%%%%%%%%%%%%%%%%%%%%%%%%%%%%%%%%%%%%%%
\subsection{Included Files}
\label{sec:include}

%%%%%%%%%%%%%%%%%%%%%%%%%%%%%%%%%%%%%%%%
\DescribeMacro{\childdocmain}
To use the package, add the commands
\begin{center}
\begin{tabular}{l}
|\input{childdoc.def}|\\
|\childdocmain{}|\\
\end{tabular}
\end{center}
at the very top of the main \LaTeX{} file,
in particular \emph{before} the |\documentclass| statement!
The argument of |\childdocmain| should be left empty
(but it must be present).

%%%%%%%%%%%%%%%%%%%%%%%%%%%%%%%%%%%%%%%%
\DescribeMacro{\childdocof}
Furthermore, add the commands
\begin{center}
\begin{tabular}{l}
|\input{childdoc.def}|\\
|\childdocof{|\textit{main}|}|\\
\end{tabular}
\end{center}
at the top of every child file \textit{child}
which is included by |\include{|\textit{child}|}|
from within the main file
(or at least for those files to be compiled individually).
The argument \textit{main} must be the filename of the main file.

There are a couple of
considerations in setting up the main and child documents:

%%%%%%%%%%%%%%%%%%%%%%%%%%%%%%%%%%%%%%%%
\paragraph{Restrictions.}

Please note the following restrictions:
\begin{itemize}
\item
|\childdocmain| must be called with one argument \textit{main}
to ensure compatibility with earlier version of the package.
It must either be empty (|\childdocmain{}|)
or precisely match the filename of the main file in which it is specified.
See \secref{sec:detection} for further information.
\item
The filename \textit{main} must be specified without the |.tex| extension.
\item
The filename \textit{main} is case sensitive
(even in case-insensitive file systems)
due to internal string comparison.
\item
The argument \textit{main} should be fully expanded, it cannot be a macro.
\item
Subdirectories and special characters should be avoided in filenames.
\item
The command |\childdocmain{|\textit{main}|}| must be followed by a whitespace.
It should not be followed immediately by another command
or by a comment mark `|%|'.
This is because the \TeX{} parser reads the token immediately following
the argument of |\childdocmain| and puts it
at the beginning of every child section;
however, a white\-space is ignored.
\end{itemize}

%%%%%%%%%%%%%%%%%%%%%%%%%%%%%%%%%%%%%%%%
\paragraph{Content of Main File.}

It is advisable to place all content in the child files included by |\include|.
Any output contained in the main file will appear in all child documents
unless suppressed manually;
it cannot be suppressed automatically by the |\includeonly| directive
and thus should normally be avoided.
A method to include some content in the main file
by means of conditional processing is described in \secref{sec:conditional}.

%%%%%%%%%%%%%%%%%%%%%%%%%%%%%%%%%%%%%%%%
\paragraph{Page Numbering.}

When only a part of the document is compiled,
the appropriate numbering of pages
(as well as other status parameters)
is determined from the |.aux| files.
The latter contain information from previous passes.
However this information needs to propagate through
all intermediate child documents.
Therefore the page numbering in child documents may well
be inconsistent until the complete document is compiled at least once.

A useful (if unconventional) way to always ensure a consistent
page numbering is to restart the numbering in each child document
and denote the pages by `\textit{child}|.|\textit{page}'
where \textit{child} represents the chapter/section number of the child file.
This can be achieved by the command
|\numberwithin{page}{|\textit{child}|}|
of the \textsf{amsmath} package
where \textit{child} can be |chapter| or |section|
depending on the chosen structuring.
Alternatively, one can modify the macro |\thepage| appropriately
and reset the counter |page| at the start of each child file.

%%%%%%%%%%%%%%%%%%%%%%%%%%%%%%%%%%%%%%%%%%%%%%%%%%%%%%%%%%%%%%%%%%%%%%%%%%%%%%%%
\subsection{Conditional Processing}
\label{sec:conditional}

The package provides a mechanism to compile different versions
of a document. To customise the versions further some conditional processing
can come in handy to distinguish which version is being compiled.
The package provides two macros to describe the compilation context:

%%%%%%%%%%%%%%%%%%%%%%%%%%%%%%%%%%%%%%%%
\DescribeMacro{\ifchilddoc}
The conditional |\ifchilddoc| distinguishes between the compilation of
child documents and the main document:
%
\begin{center}
|\ifchilddoc |\textit{child-code}| |[|\||else |\textit{main-code}]| \||fi|
\end{center}

%%%%%%%%%%%%%%%%%%%%%%%%%%%%%%%%%%%%%%%%
\DescribeMacro{\childdocname}
\DescribeMacro{\childdocjob}
The macro |\childdocname| contains the filename (without extension)
of the main or child file being processed.
Note that |\childdocjob| will always contain the name of the main file.

%%%%%%%%%%%%%%%%%%%%%%%%%%%%%%%%%%%%%%%%
\paragraph{Title Page.}

Conditional processing can be used to include a title or banner page
in the main document when proper precautions are taken.
Importantly, the code in the main file should ensure that the page counter
(as well as other status parameters which are stored in the |.aux| files)
takes the same value after the conditional processing.
Otherwise the page numbers may take divergent values
depending on which part is compiled.

For example, a title page could be declared by:
%
\begin{center}
\begin{tabular}{l}
|\ifchilddoc\||else|\\
|\addtocounter{page}{-1}|\\
\textit{code for title page}\\
|\newpage|\\
|\||fi|
\end{tabular}
\end{center}
%
A banner page for the child documents can be generated by:
%
\begin{center}
\begin{tabular}{l}
|\ifchilddoc|\\
|\addtocounter{page}{-1}|\\
\textit{code for banner page}\\
|\newpage|\\
|\||fi|
\end{tabular}
\end{center}
%
Here one could write a message such as:
\begin{center}
|This is the part \childdocname{} of \childdocjob{}.|
\end{center}

%%%%%%%%%%%%%%%%%%%%%%%%%%%%%%%%%%%%%%%%%%%%%%%%%%%%%%%%%%%%%%%%%%%%%%%%%%%%%%%%
\subsection{Flags}
\label{sec:flags}

The package makes it easy to generate different versions
of the main or child documents.
To this end compilation flags can be defined
and assigned different default values.
They will be particularly useful in conjunction
with the forwarding mechanism described in \secref{sec:forward}.

For example, it may be useful to have a flag |\version|
which can be set to |draft| or |final|.
The document source will contain some conditional code
depending on the value of |\version|.
Suppose further, the flag should default to |final| for the main file
and to |draft| for child files
which is a natural assignment for editing the document.
This is achieved by placing the following code
in the preamble of the main document
(below the |\childdocmain| directive):
%
\begin{center}
\begin{tabular}{l}
|\ifchilddoc|\\
|\providecommand{\version}{draft}|\\
|\||else|\\
|\providecommand{\version}{final}|\\
|\||fi|
\end{tabular}
\end{center}
%
The definition by |\providecommand| makes sure
that previous definitions are not overwritten.
Further statements |\providecommand{\version}{...}|
can thus be added before the above code to override it.

For the main file, one might add a line
(between |\childdocmain| and the above block)
%
\begin{center}
|%\ifchilddoc\||else\providecommand{\version}{draft}\||fi|
\end{center}
%
which can be uncommented to produce a draft version.
Likewise one can add a line to the very top of a child file
(above the |\childdocof{|\textit{main}|}| directive)
%
\begin{center}
|%\providecommand{\version}{final}|
\end{center}
%
which can be uncommented to produce the final version of this child document.

%%%%%%%%%%%%%%%%%%%%%%%%%%%%%%%%%%%%%%%%%%%%%%%%%%%%%%%%%%%%%%%%%%%%%%%%%%%%%%%%
\subsection{Forwarding}
\label{sec:forward}

Different versions of the main or child documents
using compilation flags as described in \secref{sec:flags}
can be (permanently) stored in different files
for convenient compilation, viewing and distribution.
To this end, the package defines a command
to pass on compilation to a different file:

%%%%%%%%%%%%%%%%%%%%%%%%%%%%%%%%%%%%%%%%
\DescribeMacro{\childdocforward}
The command |\childdocforward| redirects processing to
another source file:
%
\begin{center}
\begin{tabular}{l}
|\input{childdoc.def}|\\
|\childdocforward[|\textit{main}|]{|\textit{dest}|}|\\
\end{tabular}
\end{center}
%
The argument \textit{dest} is the destination file
(without extension).
It should be the main file or one of the child files.
Note that further \textsf{childdoc} directives
such as |\childdocof| and |\childdocforward|
in the indicated file will be processed in this form.
The optional argument \textit{main}
passes on directly to the main file \textit{main}
while pretending to compile the child \textit{dest}.
This form behaves as if \textit{dest}
issues |\childdocof{|\textit{main}|}| right away,
and no further \textsf{childdoc} directives will be processed.

%%%%%%%%%%%%%%%%%%%%%%%%%%%%%%%%%%%%%%%%
\DescribeMacro{\...prefix}
In the alternative form |\childdocforwardprefix|,
%
\begin{center}
\begin{tabular}{l}
|\input{childdoc.def}|\\
|\childdocforwardprefix[|\textit{main}|]{|\textit{prefix}|}{|\textit{dest}|}|
\end{tabular}
\end{center}
%
the destination file is determined by a pattern
depending on the current file:
To make this work, the current file must be called
`{\textit{prefix}\hspace{0.2em}\textit{suffix}}'
with \textit{prefix} matching precisely the argument.
Processing is then passed on to the file
`{\textit{dest}\hspace{0.2em}\textit{suffix}}'.
Surely, the same effect is achieved by
directly specifying the
argument `{\textit{dest}\hspace{0.2em}\textit{suffix}}'
in the first form.
However, that requires to set up a different file
for each child. With the alternative form of the command
all these files can have exactly the same content
which simplifies setting them up and maintaining them.

For example, the following file |draft.tex|
with a compilation flag |\version| as described in \secref{sec:flags}
compiles the main document as a draft:
%
\begin{center}
\begin{tabular}{l}
|\def\version{draft}|\\
|\input{childdoc.def}|\\
|\childdocforward{|\textit{main}|}|
\end{tabular}
\end{center}
%
Likewise, the following files |final|\textit{nn}|.tex|
compile the final version of the child document
|child|\textit{nn}|.tex|:
%
\begin{center}
\begin{tabular}{l}
|\def\version{final}|\\
|\input{childdoc.def}|\\
|\childdocforwardprefix{final}{child}|
\end{tabular}
\end{center}
%

Note that when several versions of a main file and/or of each child file
are to be generated, it may be convenient to set up a |Makefile| or
shell script to automatise the process.

%%%%%%%%%%%%%%%%%%%%%%%%%%%%%%%%%%%%%%%%%%%%%%%%%%%%%%%%%%%%%%%%%%%%%%%%%%%%%%%%
\subsection{Command Line Processing}
\label{sec:commandline}

The effect of redirection files can also be achieved by invoking
the \LaTeX{} compiler with a more elaborate command line.
Most conveniently this should be done as part
of a shell script or a |Makefile|.

When using \textsf{childdoc} in the main file, the following
command lines effectively perform a redirection
(note that depending on the shell being used,
backslashes may have to be doubled: `|\|' $\to$ `|\\|'):
%
\begin{center}
|... -jobname "|\textit{target}|" |\\|"|[\textit{flags}]%
|\input{childdoc.def}\childdocforward[|\textit{main}|]{|\textit{dest}|}"|
\end{center}
%
Here \textit{target} is the name of the output file,
\textit{main} is the name of the main file
and \textit{dest} is the name of the main or child file to be processed
(all filenames without extensions).
The optional argument \textit{main} can be omitted
if \textit{main} matches \textit{dest}.
Optionally, compilation \textit{flags} can be defined via |\def| commands.
This command line makes the \TeX{} engine believe
it is compiling the file \textit{target}
whose content is specified as the latter parameter.
The provided code then forwards the processing to
\textit{main} or \textit{dest} as described in \secref{sec:forward}.

%%%%%%%%%%%%%%%%%%%%%%%%%%%%%%%%%%%%%%%%%%%%%%%%%%%%%%%%%%%%%%%%%%%%%%%%%%%%%%%%
\subsection{Include by Input}
\label{sec:input}

Including child documents by |\include| has some restrictions by design.
Most notably, the content of a child document always occupies
its own set of pages; pages cannot be shared between child documents.
Usually, this behaviour makes perfect sense
because each child document contain an essential part of the document.
However, in some situations it may be desirable to compose
a document from a collection of parts
without having mandatory page breaks between then.
For this case, the package
provides a mechanism to include parts
by |\input| which can also be processed individually.
However, by construction this mechanism
requires manual handling of the content to be output.

%%%%%%%%%%%%%%%%%%%%%%%%%%%%%%%%%%%%%%%%
\DescribeMacro{\ifchilddocmanual}
The main file should be prepared as usual, see \secref{sec:include}.
However, the document body must make a distinction
between processing of an individual part and of the main document, e.g.:
%
\begin{center}
\begin{tabular}{l}
|\ifchilddocmanual|\\
|\input{\childdocname}|\\
|\||else|\\
\textit{document body with }|\input{|\textit{part}|}|\\
|\||fi|
\end{tabular}
\end{center}
%
The conditional |\ifchilddocmanual| is true whenever
a part to be included by |\input| is being compiled,
and the name of the part is stored in |\childdocname|.

%%%%%%%%%%%%%%%%%%%%%%%%%%%%%%%%%%%%%%%%
\DescribeMacro{\childdocby}
Each part to be included by |\input| should start with:
%
\begin{center}
\begin{tabular}{l}
|\input{childdoc.def}|\\
|\childdocby{|\textit{main}|}|\\
\end{tabular}
\end{center}
%
The directive |\childdocby| is similar to |\childdocof|
described in \secref{sec:include},
but the subsequent selection of content must be done manually.
To that end, both |\ifchilddoc| and |\ifchilddocmanual|
will be true upon processing of a part,
and the name of the part is stored in |\childdocname|.
Note that |\jobname| will be set to the filename of the current part
so that each part receives an individual |.aux| file
that does not interfere with the |.aux| file(s) of the main document.
This behaviour can be altered by the alternative form
|\childdocby[*]{|\textit{main}|}| (with a non-empty optional argument)
which uses the |.aux| file of the main document
by setting |\jobname| to \textit{main}.

%%%%%%%%%%%%%%%%%%%%%%%%%%%%%%%%%%%%%%%%%%%%%%%%%%%%%%%%%%%%%%%%%%%%%%%%%%%%%%%%
\subsection{Driver Development}
\label{sec:driver}

The \textsf{childdoc} mechanism can also be use for the development
of definition files such as \LaTeX{} styles or classes.
This case differs from the above setup with multiple parts
included by |\include| in that no |\includeonly| should be invoked.
This can be achieved by starting the include file
(before |\ProvidesPackage|) with:
%
\begin{center}
\begin{tabular}{l}
|\input{childdoc.def}|\\
|\childdocforward{|\textit{main}|}|\\
\end{tabular}
\end{center}
%
or alternatively with:
%
\begin{center}
\begin{tabular}{l}
|\input{childdoc.def}|\\
|\childdocby{|\textit{main}|}|\\
\end{tabular}
\end{center}
%
Both forms have slightly different effects as described above.
The main file is prepared as usual, see \secref{sec:include}.

%%%%%%%%%%%%%%%%%%%%%%%%%%%%%%%%%%%%%%%%%%%%%%%%%%%%%%%%%%%%%%%%%%%%%%%%%%%%%%%%
\subsection{Legacy Detection}
\label{sec:detection}

The directive |\childdocmain| in the main file can detect
whether the complete document or merely a child is to be compiled
even without using the directive |\childdocof|.
This method is deprecated because it is less robust
and there is no compelling reason to use it;
it is merely provided for backward compatibility
and it may be removed in future versions.

If the detection mechanism is to be used,
it is mandatory to correctly specify
the filename of the main file as the argument of |\childdocmain|:
%
\begin{center}
\begin{tabular}{l}
|\input{childdoc.def}|\\
|\childdocmain{|\textit{main}|}|\\
\end{tabular}
\end{center}
%
If |\jobname| does not match the argument \textit{main} of |\childdocmain|,
it is assumed that |\jobname| points to the child file to be compiled.
When using |\childdocmain| with the main file specified as argument,
it suffices to start a child file
with just |\input{|\textit{main}|}|
without loading of the package and using |\childdocof|.
If instead all processing is done
with the appropriate \textsf{childdoc} directives,
the argument of \textit{main} of |\childdocmain| can be empty.

An alternative version of the command line processing described
in \secref{sec:commandline} using the detection mechanism reads:
%
\begin{center}
|... -jobname "|\textit{target}|" "|[\textit{flags}]%
[|\def\jobname{|\textit{dest}|}|]|\input{|\textit{main}|}"|
\end{center}

%%%%%%%%%%%%%%%%%%%%%%%%%%%%%%%%%%%%%%%%%%%%%%%%%%%%%%%%%%%%%%%%%%%%%%%%%%%%%%%%
\subsection{Manual Code}
\label{sec:manual}

In case one cannot be certain whether the definitions file |childdoc.def|
is installed on the target \TeX{} distribution
and one prefers not to ship it,
it is conceivable to paste a few relevant commands into the sources.

To that end, drop all statements |\input{childdoc.def}|
and perform the replacements as outlined below.
Instead of |\childdocmain{|\textit{main}|}| add the following code
to the top of the main file:
%
\begin{center}
\begin{tabular}{l}
|\||ifdefined\childdocname\endinput\||fi\newif\ifchilddoc|\\
|\edef\childdocname{\scantokens\expandafter{\jobname\noexpand}}|\\
|\def\childdocmain{|\textit{main}|}\||ifx\childdocmain\childdocname\||else|\\
|\childdoctrue\includeonly{\childdocname}\let\jobname\childdocmain\||fi|\\
\end{tabular}
\end{center}
%
Instead of |\childdocof{|\textit{main}|}| just include the main file
at the top of each child file:
%
\begin{center}
|\input{|\textit{main}|}|
\end{center}
%
A simple redirection |\childdocforward{|\textit{dest}|}| is achieved by:
%
\begin{center}
|\def\jobname{|\textit{dest}|}\input{\jobname}|
\end{center}
%
The redirection with prefix
|\childdocforwardprefix[|\textit{prefix}|]{|\textit{dest}|}|
is accomplished by:
%
\begin{center}
\begin{tabular}{l}
|{\edef\jobname{\scantokens\expandafter{\jobname\noexpand}}|\\
|\def\redirectjob |\textit{prefix}|#1~~~{\gdef\jobname{|\textit{dest}|#1}}|\\
|\expandafter\redirectjob\jobname~~~}\input{\jobname}|
\end{tabular}
\end{center}

In an alternative approach,
child documents can be compiled by a specific command line
without additional code or specific definitions:
%
\begin{center}
|... -jobname "|\textit{target}|" "|[\textit{flags}]%
|\includeonly{|\textit{dest}|}\input{|\textit{main}|}"|
\end{center}
%

%%%%%%%%%%%%%%%%%%%%%%%%%%%%%%%%%%%%%%%%%%%%%%%%%%%%%%%%%%%%%%%%%%%%%%%%%%%%%%%%
%%%%%%%%%%%%%%%%%%%%%%%%%%%%%%%%%%%%%%%%%%%%%%%%%%%%%%%%%%%%%%%%%%%%%%%%%%%%%%%%
\section{Information}

%%%%%%%%%%%%%%%%%%%%%%%%%%%%%%%%%%%%%%%%%%%%%%%%%%%%%%%%%%%%%%%%%%%%%%%%%%%%%%%%
\subsection{Copyright}

Copyright \copyright{} 2017--2018 Niklas Beisert

This work may be distributed and/or modified under the
conditions of the \LaTeX{} Project Public License, either version 1.3
of this license or (at your option) any later version.
The latest version of this license is in
  \url{http://www.latex-project.org/lppl.txt}
and version 1.3 or later is part of all distributions of \LaTeX{}
version 2005/12/01 or later.

This work has the LPPL maintenance status `maintained'.

The Current Maintainer of this work is Niklas Beisert.

This work consists of the files |README.txt|, |childdoc.ins| and |childdoc.dtx|
as well as the derived files |childdoc.def|, |cdocsamp.tex|
with |cdocsch1.tex|, |cdocsch2.tex|, |cdocspt3.tex|, |cdocspt4.tex|,
|cdocsdrf.tex|, |cdocsfn1.tex|, |cdocsfn2.tex|
as well as |childdoc.pdf|.

%%%%%%%%%%%%%%%%%%%%%%%%%%%%%%%%%%%%%%%%%%%%%%%%%%%%%%%%%%%%%%%%%%%%%%%%%%%%%%%%
\subsection{Files and Installation}

The package consists of the files:
%
\begin{center}
\begin{tabular}{ll}
    |README.txt|   & readme file \\
    |childdoc.ins| & installation file \\
    |childdoc.dtx| & source file \\
    |childdoc.def| & definition file \\
    |cdocsamp.tex| & sample main file \\
    |cdocsch1.tex| & sample include file \\
    |cdocsch2.tex| & sample include file \\
    |cdocspt3.tex| & sample part file \\
    |cdocspt4.tex| & sample part file \\
    |cdocsdrf.tex| & sample redirection file \\
    |cdocsfn1.tex| & sample redirection file \\
    |cdocsfn2.tex| & sample redirection file \\
    |childdoc.pdf| & manual
\end{tabular}
\end{center}
%
The distribution consists of the files
|README.txt|, |childdoc.ins| and |childdoc.dtx|.
%
\begin{itemize}
\item
Run (pdf)\LaTeX{} on |childdoc.dtx|
to compile the manual |childdoc.pdf| (this file).
\item
Run \LaTeX{} on |childdoc.ins| to create the definitions file |childdoc.def|
and the sample |cdocsamp.tex| with include files
|cdocsch1.tex|, |cdocsch2.tex|, |cdocspt3.tex|, |cdocspt4.tex|,
|cdocsdrf.tex|, |cdocsfn1.tex|, |cdocsfn2.tex|.
Then copy the file |childdoc.def| to an appropriate directory of your \LaTeX{}
distribution, e.g.\ \textit{texmf-root}|/tex/latex/childdoc|.
\end{itemize}

%%%%%%%%%%%%%%%%%%%%%%%%%%%%%%%%%%%%%%%%%%%%%%%%%%%%%%%%%%%%%%%%%%%%%%%%%%%%%%%%
\subsection{Related CTAN Packages}

There are several other packages which offer a similar functionality:
%
\begin{itemize}
\item
The packages
\href{http://ctan.org/pkg/docmute}{\textsf{docmute}},
\href{http://ctan.org/pkg/includex}{\textsf{includex}} and
\href{http://ctan.org/pkg/standalone}{\textsf{standalone}}
provide commands to include only the document body of
a child file thus allowing both files to be compiled individually.
\item
The packages \href{http://ctan.org/pkg/subdocs}{\textsf{subdocs}}
and \href{http://ctan.org/pkg/subfiles}{\textsf{subfiles}}
provide structures in which the main and child documents can be
encapsulated and allowing them to be compiled individually.
The inclusion mechanism is different from the conventional |\include|.
\item
The package \href{http://ctan.org/pkg/combine}{\textsf{combine}}
is an elaborate solution to combine several documents into one.
\end{itemize}
%
See also the CTAN topic \href{http://ctan.org/topic/subdocs}{\textsf{subdocs}}
for further related packages.
The present package differs from the above solutions in that
a document structure constructed with the conventional |\include| mechanism
just needs two extra commands at the top of every file
such that all constituent files can be compiled individually.

%%%%%%%%%%%%%%%%%%%%%%%%%%%%%%%%%%%%%%%%%%%%%%%%%%%%%%%%%%%%%%%%%%%%%%%%%%%%%%%%
%\subsection{Feature Suggestions}
%
%The following is a list of features which may be useful for future
%versions of this package:
%%
%\begin{itemize}
%\item
%\ldots
%\end{itemize}

%%%%%%%%%%%%%%%%%%%%%%%%%%%%%%%%%%%%%%%%%%%%%%%%%%%%%%%%%%%%%%%%%%%%%%%%%%%%%%%%
\subsection{Revision History}

%%%%%%%%%%%%%%%%%%%%%%%%%%%%%%%%%%%%%%%%
\paragraph{v2.0:} 2018/12/30

\begin{itemize}
\item
immediate forward processing
\item
added |\childdocby| mechanism
\item
manual restructured
\end{itemize}

%%%%%%%%%%%%%%%%%%%%%%%%%%%%%%%%%%%%%%%%
\paragraph{v1.6:} 2018/01/17

\begin{itemize}
\item
application for development of include files
\item
corrections to manual
\end{itemize}

%%%%%%%%%%%%%%%%%%%%%%%%%%%%%%%%%%%%%%%%
\paragraph{v1.5:} 2017/05/21

\begin{itemize}
\item
more complete structuring introduced
\item
|\childdocof| introduced
\item
|\childdoc| renamed to |\childdocmain|
\item
|\childredirect| renamed to |\childdocforward| and |\childdocforwardprefix|
and functionality expanded
\end{itemize}

%%%%%%%%%%%%%%%%%%%%%%%%%%%%%%%%%%%%%%%%
\paragraph{v1.0:} 2017/04/27

\begin{itemize}
\item
manual and install package
\item
first version published on CTAN
\end{itemize}

%%%%%%%%%%%%%%%%%%%%%%%%%%%%%%%%%%%%%%%%
\paragraph{v0.6:} 2017/04/26

\begin{itemize}
\item
redirection mechanism added
\end{itemize}

%%%%%%%%%%%%%%%%%%%%%%%%%%%%%%%%%%%%%%%%
\paragraph{v0.5:} 2017/04/26

\begin{itemize}
\item
functionality in definition file
\end{itemize}


%%%%%%%%%%%%%%%%%%%%%%%%%%%%%%%%%%%%%%%%%%%%%%%%%%%%%%%%%%%%%%%%%%%%%%%%%%%%%%%%
%%%%%%%%%%%%%%%%%%%%%%%%%%%%%%%%%%%%%%%%%%%%%%%%%%%%%%%%%%%%%%%%%%%%%%%%%%%%%%%%
%%%%%%%%%%%%%%%%%%%%%%%%%%%%%%%%%%%%%%%%%%%%%%%%%%%%%%%%%%%%%%%%%%%%%%%%%%%%%%%%
\appendix

\settowidth\MacroIndent{\rmfamily\scriptsize 000\ }

 \DocInput{childdoc.dtx}

\end{document}
%</driver>
% \fi
%
% %%%%%%%%%%%%%%%%%%%%%%%%%%%%%%%%%%%%%%%%%%%%%%%%%%%%%%%%%%%%%%%%%%%%%%%%%%%%%%
% %%%%%%%%%%%%%%%%%%%%%%%%%%%%%%%%%%%%%%%%%%%%%%%%%%%%%%%%%%%%%%%%%%%%%%%%%%%%%%
% \section{Sample}
%\iffalse
%<*samplemain>
%\fi
%
% The following presents a sample document
% with two chapters, two parts, a title page,
% a compile flag as well as three forwarding files to set the flag.
% It consists of eight |.tex| files:
% \begin{center}
% \begin{tabular}{ll}
% |cdocsamp.tex|&main file\\
% |cdocsch1.tex|&include file for chapter 1\\
% |cdocsch2.tex|&include file for chapter 2\\
% |cdocspt3.tex|&include file for part 3\\
% |cdocspt4.tex|&include file for part 4\\
% |cdocsdrf.tex|&forwarding file for main file in draft mode\\
% |cdocsfi1.tex|&forwarding file for final version of chapter 1\\
% |cdocsfi2.tex|&forwarding file for final version of chapter 2\\
% \end{tabular}
% \end{center}
% Each of the eight files can be compiled directly by the \LaTeX{} compiler.
%
% %%%%%%%%%%%%%%%%%%%%%%%%%%%%%%%%%%%%%%
% \paragraph{Main File.}
%
% The main file is called |cdocsamp.tex|.
%
% Load the \textsf{childdoc} definitions and
% declare the filename for the main document:
%    \begin{macrocode}
\input{childdoc.def}
\childdocmain{}
%    \end{macrocode}

% Optional override for |\version| flag:
%    \begin{macrocode}
%%\ifchilddoc\else\providecommand{\version}{draft}\fi
%    \end{macrocode}

% Define the default values for the |\version| flag
% (|final| for the main file and |draft| for childs):
%    \begin{macrocode}
\ifchilddoc
\providecommand{\version}{draft}
\else
\providecommand{\version}{final}
\fi
%    \end{macrocode}

% Load the standard document class:
%    \begin{macrocode}
\documentclass[12pt]{article}
%    \end{macrocode}

% Start the document body:
%    \begin{macrocode}
\begin{document}
%    \end{macrocode}

% Declare a title page.
% Print title, part of document being processed and version flag:
%    \begin{macrocode}
\addtocounter{page}{-1}
\begin{center}
{\LARGE\bfseries{}childdoc example\par}
\vspace{1cm}
\ifchilddoc
\ifchilddocmanual part\else chapter\fi:
`\childdocname' of `\childdocjob'\par
\else
main document: `\childdocjob'\par
\fi
version: \version\par
\end{center}
\newpage
%    \end{macrocode}

% Manually include selected file,
% otherwise process as usual:
%    \begin{macrocode}
\ifchilddocmanual
\section*{part `\childdocname'}
\input{\childdocname}
\else
%    \end{macrocode}

% Include the two chapters:
%    \begin{macrocode}
\include{cdocsch1}
\include{cdocsch2}
%    \end{macrocode}

% Include the two parts unless only chapters should be displayed:
%    \begin{macrocode}
\ifchilddoc\else
\section{part three}
\input{cdocspt3}
\section{part four}
\input{cdocspt4}
\fi
%    \end{macrocode}

% Process as usual until here:
%    \begin{macrocode}
\fi
%    \end{macrocode}

% End of document body:
%    \begin{macrocode}
\end{document}
%    \end{macrocode}
%\iffalse
%</samplemain>
%\fi
%
% %%%%%%%%%%%%%%%%%%%%%%%%%%%%%%%%%%%%%%
% \paragraph{Chapter Include Files.}
%
% The include files are called |cdocsch1.tex| and |cdocsch2.tex|.
%
%\iffalse
%<*samplechap1|samplechap2>
%\fi

% Optional override for |\version| flag:
%    \begin{macrocode}
%%\providecommand{\version}{final}
%    \end{macrocode}

% Include the main document:
%    \begin{macrocode}
\input{childdoc.def}
\childdocof{cdocsamp}
%    \end{macrocode}

%\iffalse
%</samplechap1|samplechap2>
%\fi
%
%\iffalse
%<*samplechap1>
%\fi
% Some text for chapter 1:
%    \begin{macrocode}
\section{one}
some text in chapter one
%    \end{macrocode}

%\iffalse
%</samplechap1>
%\fi
% Some text for chapter 2:
%\iffalse
%<*samplechap2>
%\fi
%    \begin{macrocode}
\section{two}
more text in chapter two
%    \end{macrocode}

%\iffalse
%</samplechap2>
%\fi
%
% %%%%%%%%%%%%%%%%%%%%%%%%%%%%%%%%%%%%%%
% \paragraph{Part Include Files.}
%
% The include files are called |cdocspt3.tex| and |cdocspt4.tex|.
%
%\iffalse
%<*samplepart3|samplepart4>
%\fi

% Optional override for |\version| flag:
%    \begin{macrocode}
%%\providecommand{\version}{final}
%    \end{macrocode}

% Include the main document:
%    \begin{macrocode}
\input{childdoc.def}
\childdocby{cdocsamp}
%    \end{macrocode}

%\iffalse
%</samplepart3|samplepart4>
%\fi
%
%\iffalse
%<*samplepart3>
%\fi
% Some text for part 3:
%    \begin{macrocode}
some text in part three
%    \end{macrocode}

%\iffalse
%</samplepart3>
%\fi
% Some text for part 4:
%\iffalse
%<*samplepart4>
%\fi
%    \begin{macrocode}
more text in part four
%    \end{macrocode}

%\iffalse
%</samplepart4>
%\fi
%
% %%%%%%%%%%%%%%%%%%%%%%%%%%%%%%%%%%%%%%
% \paragraph{Forwarding for a Complete Draft.}
%
% The following forwarding file |cdocsdrf.tex|
% compiles the main document in draft mode:
%\iffalse
%<*sampledraft>
%\fi
%    \begin{macrocode}
\def\version{draft}
\input{childdoc.def}
\childdocforward{cdocsamp}
%    \end{macrocode}

%\iffalse
%</sampledraft>
%\fi
%
% %%%%%%%%%%%%%%%%%%%%%%%%%%%%%%%%%%%%%%
% \paragraph{Forwarding for Final Version of the Chapters.}
%
% The following forwarding files |cdocsfn1.tex| and |cdocsfn2.tex|
% (with identical content)
% compile the final versions of the child documents
% |cdocsch1.tex| and |cdocsch2.tex|, respectively:
%\iffalse
%<*samplefinal>
%\fi
%    \begin{macrocode}
\def\version{final}
\input{childdoc.def}
\childdocforwardprefix[cdocsamp]{cdocsfn}{cdocsch}
%    \end{macrocode}

%\iffalse
%</samplefinal>
%\fi
%
% %%%%%%%%%%%%%%%%%%%%%%%%%%%%%%%%%%%%%%
% \paragraph{Command Line Processing.}
%
% The following three command lines generate the output files
% |cdocscld|, |cdocscl1| and |cdocscl2|
% which should be identical to
% |cdocsdrf|, |cdocsch1| and |cdocsfn2|, respectively:
% \begin{center}
% \begin{tabular}{l}
% |latex -jobname cdocscld \|\\
% |  "\def\version{draft}\input{childdoc.def}\childdocforward{cdocsamp}"|\\
% |latex -jobname cdocscl1 \|\\
% |  "\input{childdoc.def}\childdocforward[cdocsamp]{cdocsch1}"|\\
% |latex -jobname cdocscl2 \|\\
% |  "\def\version{final}\input{childdoc.def}\childdocforward{cdocsch2}"|
% \end{tabular}
% \end{center}
% Note that the trailing backslash on each first line
% merely continues the input to the second line
% (for convenient cut ant paste).
% Furthermore, the command |latex| can be replaced by any
% of its alternative versions such as |pdflatex|.
%
% %%%%%%%%%%%%%%%%%%%%%%%%%%%%%%%%%%%%%%%%%%%%%%%%%%%%%%%%%%%%%%%%%%%%%%%%%%%%%%
% %%%%%%%%%%%%%%%%%%%%%%%%%%%%%%%%%%%%%%%%%%%%%%%%%%%%%%%%%%%%%%%%%%%%%%%%%%%%%%
% \section{Implementation}
%\iffalse
%<*package>
%\fi
%
% This section describes the definitions file |childdoc.def|.

% The definitions cannot be loaded using |\usepackage| or |\RequirePackage|
% which has a mechanism to prevent loading a style file more than once.
% When loading the definitions by means of |\input|
% multiple instances have to be prevented manually:
%\iffalse
%This code needs to be before the `\ProvidesFile' directive
%which is defined at the beginning of this file.
%Therefore it is also placed there and commented out here.
%</package>
%<*discard>
%\fi
%    \begin{macrocode}
\ifdefined\childdocmain\endinput\fi
%    \end{macrocode}
%\iffalse
%</discard>
%<*package>
%\fi
%
% \macro{\ifchilddoc}
% \macro{\ifchilddocmanual}
% The conditional |\ifchilddoc| tells whether a
% child (true) or main (false) document is being compiled.
% The conditional |\ifchilddocmanual| tells whether
% the |\includeonly| mechanism is used (false) or
% the selection of child files must be performed manually (true).
% The definitions initialise to false:
%    \begin{macrocode}
\newif\ifchilddoc
\newif\ifchilddocmanual
%    \end{macrocode}

% \macro{\childdocname}
% \macro{\childdocjob}
% The macro |\childdocname| stores the name of the main document
% to be compiled. The macro |\childdocjob| stores the name of
% the document on which the \LaTeX{} compiler was originally invoked.
% The content of |\jobname| cannot be compared
% to filenames specified in the source due to different catcodes.
% The following code rescans |\jobname|, stores the result
% in |\childdocname| and saves a copy in |\childdocjob|:
%    \begin{macrocode}
\edef\childdocname{\scantokens\expandafter{\jobname\noexpand}}
\let\childdocjob\childdocname
%    \end{macrocode}

% \macro{\childdocdisable}
% The macro |\childdocdisable| prevents the main file
% from being processed more than once.
% At this stage, the main document command |\childdocmain|
% is assumed to be called once again where it should do nothing.
% Any subsequent call to it should prevent
% a secondary processing of the main document
% It overwrites the forwarding commands
% |\childdocof| and |\childdocforward|
% with empty macros to prevent further inclusions of the main document:
%    \begin{macrocode}
\newcommand{\childdocdisable}
{
  \renewcommand{\childdocmain}[1]{\renewcommand{\childdocmain}[1]{\endinput}}
  \renewcommand{\childdocof}[1]{}
  \renewcommand{\childdocby}[2][]{}
  \renewcommand{\childdocforward}[2][]{}
  \renewcommand{\childdocdisable}{}
}
%    \end{macrocode}

% \macro{\childdocmain}
% The macro |\childdocmain| is to be called at the top of the main file
% with nothing or the main filename (without extension) as argument.
% First, it breaks loops.
% If the argument is not empty and does not match |\childdocname|
% (which is set by the first inclusion of |childdoc.def|),
% |\ifchilddoc| is set to true, |\includeonly| is applied to the child file
% and |\jobname| is set to the main file
% (for proper handling of |.aux| files):
%    \begin{macrocode}
\newcommand{\childdocmain}[1]
{
  \childdocdisable\childdocmain{}
  \if?#1?\else
    \begingroup
      \def\childdoctmp{#1}
      \ifx\childdoctmp\childdocname
        \def\childdoctmp{}
      \else
        \def\childdoctmp
        {
          \childdoctrue
          \includeonly{\childdocname}
          \def\childdocjob{#1}
          \def\jobname{#1}
        }
      \fi
      \expandafter
    \endgroup
    \childdoctmp
  \fi
}
%    \end{macrocode}

% \macro{\childdocof}
% The command |\childdocof| redirects
% compilation to the main file |#1|.
%    \begin{macrocode}
\newcommand{\childdocof}[1]
{
  \childdocdisable
  \childdoctrue
  \includeonly{\childdocname}
  \def\jobname{#1}
  \def\childdocjob{#1}
  \input{#1}
}
%    \end{macrocode}

% \macro{\childdocby}
% The command |\childdocby| ....
%    \begin{macrocode}
\newcommand{\childdocby}[2][]
{
  \childdocdisable
  \childdoctrue
  \childdocmanualtrue
  \if?#1?\else
    \def\jobname{#2}
  \fi
  \def\childdocjob{#2}
  \input{#2}
  \endinput
}
%    \end{macrocode}

% \macro{\childdocforward}
% The command |\childdocforward| redirects
% compilation to the main file or
% (if the optional argument is given) a child file.
% Parameters are set as if the main file
% or a child file starting with |\childdocof| was compiled.
% Then compilation is handed over to the main file:
%    \begin{macrocode}
\newcommand{\childdocforward}[2][]
{
  \begingroup
    \if?#1?
      \def\childdoctmp
      {
        \def\childdocname{#2}
        \def\childdocjob{#2}
        \def\jobname{#2}
        \input{#2}
        \endinput
      }
    \else
      \def\childdoctmp
      {
        \childdocdisable
        \def\childdocname{#2}
        \childdoctrue
        \includeonly{#2}
        \def\childdocjob{#1}
        \def\jobname{#1}
        \input{#1}
        \endinput
      }
    \fi
    \expandafter
  \endgroup
  \childdoctmp
}
%    \end{macrocode}

% \macro{\childdocforwardprefix}
% The command |\childdocforwardprefix| redirects
% compilation to the main or a child file by means of a pattern.
% The prefix |#1| in the current filename is replaced by |#2|
% and the suffix of the current filename is kept
% (it is assumed that the filename does not contain the substring `|~~~|'
% which is used as a delimiter).
% Compilation is handed over to the new file by |\childdocforward|:
%    \begin{macrocode}
\newcommand{\childdocforwardprefix}[3][]
{
  \begingroup
    \def\childdocextract #2##1~~~{\def\childdoctmp{\childdocforward[#1]{#3##1}}}
    \expandafter\childdocextract\childdocname~~~
    \expandafter
  \endgroup
  \childdoctmp
}
%    \end{macrocode}

% \macro{\childdoc}
% The deprecated macro |\childdoc| is a legacy version of |\childdocmain|:
%    \begin{macrocode}
\newcommand{\childdoc}{\childdocmain}
%    \end{macrocode}

% \macro{\childdocredirect}
% The deprecated macro |\childdocredirect| is a legacy version
% of |\childdocforward| and |\childdocforwardprefix|:
%    \begin{macrocode}
\newcommand{\childdocredirect}[2][]
{
  \begingroup
    \if?#1?
      \def\childdoctmp{\childdocforward{#2}}
    \else
      \def\childdoctmp{\childdocforwardprefix{#1}{#2}}
    \fi
    \expandafter
  \endgroup
  \childdoctmp
}
%    \end{macrocode}

%\iffalse
%</package>
%\fi
%
\endinput

\childdocforwardprefix[cdocsamp]{cdocsfn}{cdocsch}
%    \end{macrocode}

%\iffalse
%</samplefinal>
%\fi
%
% %%%%%%%%%%%%%%%%%%%%%%%%%%%%%%%%%%%%%%
% \paragraph{Command Line Processing.}
%
% The following three command lines generate the output files
% |cdocscld|, |cdocscl1| and |cdocscl2|
% which should be identical to
% |cdocsdrf|, |cdocsch1| and |cdocsfn2|, respectively:
% \begin{center}
% \begin{tabular}{l}
% |latex -jobname cdocscld \|\\
% |  "\def\version{draft}% \iffalse
%
% childdoc.dtx Copyright (C) 2017-2018 Niklas Beisert
%
% This work may be distributed and/or modified under the
% conditions of the LaTeX Project Public License, either version 1.3
% of this license or (at your option) any later version.
% The latest version of this license is in
%   http://www.latex-project.org/lppl.txt
% and version 1.3 or later is part of all distributions of LaTeX
% version 2005/12/01 or later.
%
% This work has the LPPL maintenance status `maintained'.
%
% The Current Maintainer of this work is Niklas Beisert.
%
% This work consists of the files childdoc.dtx and childdoc.ins
% and the derived files childdoc.def and cdocsamp.tex with
% cdocsch1.tex, cdocsch2.tex, cdocsdrf.tex, cdocsfn1.tex, cdocsfn2.tex.
%
%<package>\ifdefined\childdocmain\endinput\fi
%<package>\ProvidesFile{childdoc.def}[2018/12/30 v2.0 child document driver]
%<samplemain>\ProvidesFile{cdocsamp.tex}[2018/12/30 v2.0 sample for childdoc]
%<*driver>
%\ProvidesFile{childdoc.drv}[2018/12/30 v2.0 childdoc reference manual file]
\PassOptionsToClass{10pt,a4paper}{article}
\documentclass{ltxdoc}

\usepackage[margin=35mm]{geometry}
\usepackage{hyperref}
\usepackage{hyperxmp}
\usepackage[usenames]{color}

\hypersetup{colorlinks=true}
\hypersetup{pdfstartview=FitH}
\hypersetup{pdfpagemode=UseNone}
\hypersetup{pdfsource={}}
\hypersetup{pdflang={en-UK}}
\hypersetup{pdfcopyright={Copyright 2017-2018 Niklas Beisert.
  This work may be distributed and/or modified under the
  conditions of the LaTeX Project Public License, either version 1.3
  of this license or (at your option) any later version.}}
\hypersetup{pdflicenseurl={http://www.latex-project.org/lppl.txt}}
\hypersetup{pdfcontactaddress={ETH Zurich, ITP, HIT K,
  Wolfgang-Pauli-Strasse 27}}
\hypersetup{pdfcontactpostcode={8093}}
\hypersetup{pdfcontactcity={Zurich}}
\hypersetup{pdfcontactcountry={Switzerland}}
\hypersetup{pdfcontactemail={nbeisert@itp.phys.ethz.ch}}
\hypersetup{pdfcontacturl={http://people.phys.ethz.ch/\xmptilde nbeisert/}}

\newcommand{\secref}[1]{\hyperref[#1]{section \ref*{#1}}}

\parskip1ex
\parindent0pt
\let\olditemize\itemize
\def\itemize{\olditemize\parskip0pt}

\begin{document}

\title{The \textsf{childdoc} Package}
\hypersetup{pdftitle={The childdoc Package}}
\author{Niklas Beisert\\[2ex]
  Institut f\"ur Theoretische Physik\\
  Eidgen\"ossische Technische Hochschule Z\"urich\\
  Wolfgang-Pauli-Strasse 27, 8093 Z\"urich, Switzerland\\[1ex]
  \href{mailto:nbeisert@itp.phys.ethz.ch}
  {\texttt{nbeisert@itp.phys.ethz.ch}}}
\hypersetup{pdfauthor={Niklas Beisert}}
\hypersetup{pdfsubject={Manual for the LaTeX2e Package childdoc}}
\date{30 December 2018, \textsf{v2.0}}
\maketitle

\begin{abstract}\noindent
\textsf{childdoc} is a \LaTeXe{} package
that enables the direct compilation
of document sections included by |\include|
to individual files.
\end{abstract}

\begingroup
\parskip0ex
\tableofcontents
\endgroup

%%%%%%%%%%%%%%%%%%%%%%%%%%%%%%%%%%%%%%%%%%%%%%%%%%%%%%%%%%%%%%%%%%%%%%%%%%%%%%%%
%%%%%%%%%%%%%%%%%%%%%%%%%%%%%%%%%%%%%%%%%%%%%%%%%%%%%%%%%%%%%%%%%%%%%%%%%%%%%%%%
\section{Introduction}

\LaTeX{} provides a mechanism to structure a large document (such as a book)
into a main file and several child files (containing the chapters)
using the |\include| command.
This mechanism is beneficial for documents
which span hundreds of pages in order to
make the source file(s) more manageable.
Moreover, compilation can be restricted to
selected child files by means of the |\includeonly| command.
The latter feature can be used to reduce the compilation time while editing
(this was significantly more useful in the earlier days of \LaTeX{})
or to generate a smaller document which is easier to navigate.
Another application of |\includeonly| is to generate
documents consisting of selected parts of the complete document.

However, there are a few drawbacks of the plain |\include| mechanism:
\begin{itemize}
\item
The child files cannot be compiled on their own,
they can only be compiled via the main file.
A naive editing environment
(such as a text editor with an option
to have the current file processed by \LaTeX)
may require one to switch to the main file before compiling;
attempting to compile the child file produces errors.
\item
The main file must be modified (each time)
to adjust the |\includeonly| command
to the present needs. This easily leaves the main file in a messy state.
\item
The generated document will always carry the filename
of the main document. This is inconvenient if
several child files are to be compiled and
to be kept for distribution.
\end{itemize}

The present package provides a simple interface
to make child files individually compilable by \LaTeX{}.
Compiling a child file then has the same effect as compiling
the main file with an |\includeonly| command
to select the appropriate child.
Moreover the generated document will carry the name of the child
rather than the main file.
This resolves all three above issues.

This feature is meant to make the editing of books,
thesis documents and lecture notes somewhat more convenient.
However, the package can also be used efficiently for
composing a series of documents (such as exercise sheets)
which are typically distributed individually.
It then assists the author in generating the individual documents
(potentially in different versions)
as well as a document containing the collected series.
Another application is in developing style files
or other kinds of included material
where compilation of the style file could redirect
to a sample or test file.

%%%%%%%%%%%%%%%%%%%%%%%%%%%%%%%%%%%%%%%%%%%%%%%%%%%%%%%%%%%%%%%%%%%%%%%%%%%%%%%%
%%%%%%%%%%%%%%%%%%%%%%%%%%%%%%%%%%%%%%%%%%%%%%%%%%%%%%%%%%%%%%%%%%%%%%%%%%%%%%%%
\section{Usage}

First of all, the package \textsf{childdoc} is \emph{not} a standard
\LaTeXe{} |.sty| style file! Therefore it needs to be invoked in
a non-standard way.

%%%%%%%%%%%%%%%%%%%%%%%%%%%%%%%%%%%%%%%%%%%%%%%%%%%%%%%%%%%%%%%%%%%%%%%%%%%%%%%%
\subsection{Included Files}
\label{sec:include}

%%%%%%%%%%%%%%%%%%%%%%%%%%%%%%%%%%%%%%%%
\DescribeMacro{\childdocmain}
To use the package, add the commands
\begin{center}
\begin{tabular}{l}
|\input{childdoc.def}|\\
|\childdocmain{}|\\
\end{tabular}
\end{center}
at the very top of the main \LaTeX{} file,
in particular \emph{before} the |\documentclass| statement!
The argument of |\childdocmain| should be left empty
(but it must be present).

%%%%%%%%%%%%%%%%%%%%%%%%%%%%%%%%%%%%%%%%
\DescribeMacro{\childdocof}
Furthermore, add the commands
\begin{center}
\begin{tabular}{l}
|\input{childdoc.def}|\\
|\childdocof{|\textit{main}|}|\\
\end{tabular}
\end{center}
at the top of every child file \textit{child}
which is included by |\include{|\textit{child}|}|
from within the main file
(or at least for those files to be compiled individually).
The argument \textit{main} must be the filename of the main file.

There are a couple of
considerations in setting up the main and child documents:

%%%%%%%%%%%%%%%%%%%%%%%%%%%%%%%%%%%%%%%%
\paragraph{Restrictions.}

Please note the following restrictions:
\begin{itemize}
\item
|\childdocmain| must be called with one argument \textit{main}
to ensure compatibility with earlier version of the package.
It must either be empty (|\childdocmain{}|)
or precisely match the filename of the main file in which it is specified.
See \secref{sec:detection} for further information.
\item
The filename \textit{main} must be specified without the |.tex| extension.
\item
The filename \textit{main} is case sensitive
(even in case-insensitive file systems)
due to internal string comparison.
\item
The argument \textit{main} should be fully expanded, it cannot be a macro.
\item
Subdirectories and special characters should be avoided in filenames.
\item
The command |\childdocmain{|\textit{main}|}| must be followed by a whitespace.
It should not be followed immediately by another command
or by a comment mark `|%|'.
This is because the \TeX{} parser reads the token immediately following
the argument of |\childdocmain| and puts it
at the beginning of every child section;
however, a white\-space is ignored.
\end{itemize}

%%%%%%%%%%%%%%%%%%%%%%%%%%%%%%%%%%%%%%%%
\paragraph{Content of Main File.}

It is advisable to place all content in the child files included by |\include|.
Any output contained in the main file will appear in all child documents
unless suppressed manually;
it cannot be suppressed automatically by the |\includeonly| directive
and thus should normally be avoided.
A method to include some content in the main file
by means of conditional processing is described in \secref{sec:conditional}.

%%%%%%%%%%%%%%%%%%%%%%%%%%%%%%%%%%%%%%%%
\paragraph{Page Numbering.}

When only a part of the document is compiled,
the appropriate numbering of pages
(as well as other status parameters)
is determined from the |.aux| files.
The latter contain information from previous passes.
However this information needs to propagate through
all intermediate child documents.
Therefore the page numbering in child documents may well
be inconsistent until the complete document is compiled at least once.

A useful (if unconventional) way to always ensure a consistent
page numbering is to restart the numbering in each child document
and denote the pages by `\textit{child}|.|\textit{page}'
where \textit{child} represents the chapter/section number of the child file.
This can be achieved by the command
|\numberwithin{page}{|\textit{child}|}|
of the \textsf{amsmath} package
where \textit{child} can be |chapter| or |section|
depending on the chosen structuring.
Alternatively, one can modify the macro |\thepage| appropriately
and reset the counter |page| at the start of each child file.

%%%%%%%%%%%%%%%%%%%%%%%%%%%%%%%%%%%%%%%%%%%%%%%%%%%%%%%%%%%%%%%%%%%%%%%%%%%%%%%%
\subsection{Conditional Processing}
\label{sec:conditional}

The package provides a mechanism to compile different versions
of a document. To customise the versions further some conditional processing
can come in handy to distinguish which version is being compiled.
The package provides two macros to describe the compilation context:

%%%%%%%%%%%%%%%%%%%%%%%%%%%%%%%%%%%%%%%%
\DescribeMacro{\ifchilddoc}
The conditional |\ifchilddoc| distinguishes between the compilation of
child documents and the main document:
%
\begin{center}
|\ifchilddoc |\textit{child-code}| |[|\||else |\textit{main-code}]| \||fi|
\end{center}

%%%%%%%%%%%%%%%%%%%%%%%%%%%%%%%%%%%%%%%%
\DescribeMacro{\childdocname}
\DescribeMacro{\childdocjob}
The macro |\childdocname| contains the filename (without extension)
of the main or child file being processed.
Note that |\childdocjob| will always contain the name of the main file.

%%%%%%%%%%%%%%%%%%%%%%%%%%%%%%%%%%%%%%%%
\paragraph{Title Page.}

Conditional processing can be used to include a title or banner page
in the main document when proper precautions are taken.
Importantly, the code in the main file should ensure that the page counter
(as well as other status parameters which are stored in the |.aux| files)
takes the same value after the conditional processing.
Otherwise the page numbers may take divergent values
depending on which part is compiled.

For example, a title page could be declared by:
%
\begin{center}
\begin{tabular}{l}
|\ifchilddoc\||else|\\
|\addtocounter{page}{-1}|\\
\textit{code for title page}\\
|\newpage|\\
|\||fi|
\end{tabular}
\end{center}
%
A banner page for the child documents can be generated by:
%
\begin{center}
\begin{tabular}{l}
|\ifchilddoc|\\
|\addtocounter{page}{-1}|\\
\textit{code for banner page}\\
|\newpage|\\
|\||fi|
\end{tabular}
\end{center}
%
Here one could write a message such as:
\begin{center}
|This is the part \childdocname{} of \childdocjob{}.|
\end{center}

%%%%%%%%%%%%%%%%%%%%%%%%%%%%%%%%%%%%%%%%%%%%%%%%%%%%%%%%%%%%%%%%%%%%%%%%%%%%%%%%
\subsection{Flags}
\label{sec:flags}

The package makes it easy to generate different versions
of the main or child documents.
To this end compilation flags can be defined
and assigned different default values.
They will be particularly useful in conjunction
with the forwarding mechanism described in \secref{sec:forward}.

For example, it may be useful to have a flag |\version|
which can be set to |draft| or |final|.
The document source will contain some conditional code
depending on the value of |\version|.
Suppose further, the flag should default to |final| for the main file
and to |draft| for child files
which is a natural assignment for editing the document.
This is achieved by placing the following code
in the preamble of the main document
(below the |\childdocmain| directive):
%
\begin{center}
\begin{tabular}{l}
|\ifchilddoc|\\
|\providecommand{\version}{draft}|\\
|\||else|\\
|\providecommand{\version}{final}|\\
|\||fi|
\end{tabular}
\end{center}
%
The definition by |\providecommand| makes sure
that previous definitions are not overwritten.
Further statements |\providecommand{\version}{...}|
can thus be added before the above code to override it.

For the main file, one might add a line
(between |\childdocmain| and the above block)
%
\begin{center}
|%\ifchilddoc\||else\providecommand{\version}{draft}\||fi|
\end{center}
%
which can be uncommented to produce a draft version.
Likewise one can add a line to the very top of a child file
(above the |\childdocof{|\textit{main}|}| directive)
%
\begin{center}
|%\providecommand{\version}{final}|
\end{center}
%
which can be uncommented to produce the final version of this child document.

%%%%%%%%%%%%%%%%%%%%%%%%%%%%%%%%%%%%%%%%%%%%%%%%%%%%%%%%%%%%%%%%%%%%%%%%%%%%%%%%
\subsection{Forwarding}
\label{sec:forward}

Different versions of the main or child documents
using compilation flags as described in \secref{sec:flags}
can be (permanently) stored in different files
for convenient compilation, viewing and distribution.
To this end, the package defines a command
to pass on compilation to a different file:

%%%%%%%%%%%%%%%%%%%%%%%%%%%%%%%%%%%%%%%%
\DescribeMacro{\childdocforward}
The command |\childdocforward| redirects processing to
another source file:
%
\begin{center}
\begin{tabular}{l}
|\input{childdoc.def}|\\
|\childdocforward[|\textit{main}|]{|\textit{dest}|}|\\
\end{tabular}
\end{center}
%
The argument \textit{dest} is the destination file
(without extension).
It should be the main file or one of the child files.
Note that further \textsf{childdoc} directives
such as |\childdocof| and |\childdocforward|
in the indicated file will be processed in this form.
The optional argument \textit{main}
passes on directly to the main file \textit{main}
while pretending to compile the child \textit{dest}.
This form behaves as if \textit{dest}
issues |\childdocof{|\textit{main}|}| right away,
and no further \textsf{childdoc} directives will be processed.

%%%%%%%%%%%%%%%%%%%%%%%%%%%%%%%%%%%%%%%%
\DescribeMacro{\...prefix}
In the alternative form |\childdocforwardprefix|,
%
\begin{center}
\begin{tabular}{l}
|\input{childdoc.def}|\\
|\childdocforwardprefix[|\textit{main}|]{|\textit{prefix}|}{|\textit{dest}|}|
\end{tabular}
\end{center}
%
the destination file is determined by a pattern
depending on the current file:
To make this work, the current file must be called
`{\textit{prefix}\hspace{0.2em}\textit{suffix}}'
with \textit{prefix} matching precisely the argument.
Processing is then passed on to the file
`{\textit{dest}\hspace{0.2em}\textit{suffix}}'.
Surely, the same effect is achieved by
directly specifying the
argument `{\textit{dest}\hspace{0.2em}\textit{suffix}}'
in the first form.
However, that requires to set up a different file
for each child. With the alternative form of the command
all these files can have exactly the same content
which simplifies setting them up and maintaining them.

For example, the following file |draft.tex|
with a compilation flag |\version| as described in \secref{sec:flags}
compiles the main document as a draft:
%
\begin{center}
\begin{tabular}{l}
|\def\version{draft}|\\
|\input{childdoc.def}|\\
|\childdocforward{|\textit{main}|}|
\end{tabular}
\end{center}
%
Likewise, the following files |final|\textit{nn}|.tex|
compile the final version of the child document
|child|\textit{nn}|.tex|:
%
\begin{center}
\begin{tabular}{l}
|\def\version{final}|\\
|\input{childdoc.def}|\\
|\childdocforwardprefix{final}{child}|
\end{tabular}
\end{center}
%

Note that when several versions of a main file and/or of each child file
are to be generated, it may be convenient to set up a |Makefile| or
shell script to automatise the process.

%%%%%%%%%%%%%%%%%%%%%%%%%%%%%%%%%%%%%%%%%%%%%%%%%%%%%%%%%%%%%%%%%%%%%%%%%%%%%%%%
\subsection{Command Line Processing}
\label{sec:commandline}

The effect of redirection files can also be achieved by invoking
the \LaTeX{} compiler with a more elaborate command line.
Most conveniently this should be done as part
of a shell script or a |Makefile|.

When using \textsf{childdoc} in the main file, the following
command lines effectively perform a redirection
(note that depending on the shell being used,
backslashes may have to be doubled: `|\|' $\to$ `|\\|'):
%
\begin{center}
|... -jobname "|\textit{target}|" |\\|"|[\textit{flags}]%
|\input{childdoc.def}\childdocforward[|\textit{main}|]{|\textit{dest}|}"|
\end{center}
%
Here \textit{target} is the name of the output file,
\textit{main} is the name of the main file
and \textit{dest} is the name of the main or child file to be processed
(all filenames without extensions).
The optional argument \textit{main} can be omitted
if \textit{main} matches \textit{dest}.
Optionally, compilation \textit{flags} can be defined via |\def| commands.
This command line makes the \TeX{} engine believe
it is compiling the file \textit{target}
whose content is specified as the latter parameter.
The provided code then forwards the processing to
\textit{main} or \textit{dest} as described in \secref{sec:forward}.

%%%%%%%%%%%%%%%%%%%%%%%%%%%%%%%%%%%%%%%%%%%%%%%%%%%%%%%%%%%%%%%%%%%%%%%%%%%%%%%%
\subsection{Include by Input}
\label{sec:input}

Including child documents by |\include| has some restrictions by design.
Most notably, the content of a child document always occupies
its own set of pages; pages cannot be shared between child documents.
Usually, this behaviour makes perfect sense
because each child document contain an essential part of the document.
However, in some situations it may be desirable to compose
a document from a collection of parts
without having mandatory page breaks between then.
For this case, the package
provides a mechanism to include parts
by |\input| which can also be processed individually.
However, by construction this mechanism
requires manual handling of the content to be output.

%%%%%%%%%%%%%%%%%%%%%%%%%%%%%%%%%%%%%%%%
\DescribeMacro{\ifchilddocmanual}
The main file should be prepared as usual, see \secref{sec:include}.
However, the document body must make a distinction
between processing of an individual part and of the main document, e.g.:
%
\begin{center}
\begin{tabular}{l}
|\ifchilddocmanual|\\
|\input{\childdocname}|\\
|\||else|\\
\textit{document body with }|\input{|\textit{part}|}|\\
|\||fi|
\end{tabular}
\end{center}
%
The conditional |\ifchilddocmanual| is true whenever
a part to be included by |\input| is being compiled,
and the name of the part is stored in |\childdocname|.

%%%%%%%%%%%%%%%%%%%%%%%%%%%%%%%%%%%%%%%%
\DescribeMacro{\childdocby}
Each part to be included by |\input| should start with:
%
\begin{center}
\begin{tabular}{l}
|\input{childdoc.def}|\\
|\childdocby{|\textit{main}|}|\\
\end{tabular}
\end{center}
%
The directive |\childdocby| is similar to |\childdocof|
described in \secref{sec:include},
but the subsequent selection of content must be done manually.
To that end, both |\ifchilddoc| and |\ifchilddocmanual|
will be true upon processing of a part,
and the name of the part is stored in |\childdocname|.
Note that |\jobname| will be set to the filename of the current part
so that each part receives an individual |.aux| file
that does not interfere with the |.aux| file(s) of the main document.
This behaviour can be altered by the alternative form
|\childdocby[*]{|\textit{main}|}| (with a non-empty optional argument)
which uses the |.aux| file of the main document
by setting |\jobname| to \textit{main}.

%%%%%%%%%%%%%%%%%%%%%%%%%%%%%%%%%%%%%%%%%%%%%%%%%%%%%%%%%%%%%%%%%%%%%%%%%%%%%%%%
\subsection{Driver Development}
\label{sec:driver}

The \textsf{childdoc} mechanism can also be use for the development
of definition files such as \LaTeX{} styles or classes.
This case differs from the above setup with multiple parts
included by |\include| in that no |\includeonly| should be invoked.
This can be achieved by starting the include file
(before |\ProvidesPackage|) with:
%
\begin{center}
\begin{tabular}{l}
|\input{childdoc.def}|\\
|\childdocforward{|\textit{main}|}|\\
\end{tabular}
\end{center}
%
or alternatively with:
%
\begin{center}
\begin{tabular}{l}
|\input{childdoc.def}|\\
|\childdocby{|\textit{main}|}|\\
\end{tabular}
\end{center}
%
Both forms have slightly different effects as described above.
The main file is prepared as usual, see \secref{sec:include}.

%%%%%%%%%%%%%%%%%%%%%%%%%%%%%%%%%%%%%%%%%%%%%%%%%%%%%%%%%%%%%%%%%%%%%%%%%%%%%%%%
\subsection{Legacy Detection}
\label{sec:detection}

The directive |\childdocmain| in the main file can detect
whether the complete document or merely a child is to be compiled
even without using the directive |\childdocof|.
This method is deprecated because it is less robust
and there is no compelling reason to use it;
it is merely provided for backward compatibility
and it may be removed in future versions.

If the detection mechanism is to be used,
it is mandatory to correctly specify
the filename of the main file as the argument of |\childdocmain|:
%
\begin{center}
\begin{tabular}{l}
|\input{childdoc.def}|\\
|\childdocmain{|\textit{main}|}|\\
\end{tabular}
\end{center}
%
If |\jobname| does not match the argument \textit{main} of |\childdocmain|,
it is assumed that |\jobname| points to the child file to be compiled.
When using |\childdocmain| with the main file specified as argument,
it suffices to start a child file
with just |\input{|\textit{main}|}|
without loading of the package and using |\childdocof|.
If instead all processing is done
with the appropriate \textsf{childdoc} directives,
the argument of \textit{main} of |\childdocmain| can be empty.

An alternative version of the command line processing described
in \secref{sec:commandline} using the detection mechanism reads:
%
\begin{center}
|... -jobname "|\textit{target}|" "|[\textit{flags}]%
[|\def\jobname{|\textit{dest}|}|]|\input{|\textit{main}|}"|
\end{center}

%%%%%%%%%%%%%%%%%%%%%%%%%%%%%%%%%%%%%%%%%%%%%%%%%%%%%%%%%%%%%%%%%%%%%%%%%%%%%%%%
\subsection{Manual Code}
\label{sec:manual}

In case one cannot be certain whether the definitions file |childdoc.def|
is installed on the target \TeX{} distribution
and one prefers not to ship it,
it is conceivable to paste a few relevant commands into the sources.

To that end, drop all statements |\input{childdoc.def}|
and perform the replacements as outlined below.
Instead of |\childdocmain{|\textit{main}|}| add the following code
to the top of the main file:
%
\begin{center}
\begin{tabular}{l}
|\||ifdefined\childdocname\endinput\||fi\newif\ifchilddoc|\\
|\edef\childdocname{\scantokens\expandafter{\jobname\noexpand}}|\\
|\def\childdocmain{|\textit{main}|}\||ifx\childdocmain\childdocname\||else|\\
|\childdoctrue\includeonly{\childdocname}\let\jobname\childdocmain\||fi|\\
\end{tabular}
\end{center}
%
Instead of |\childdocof{|\textit{main}|}| just include the main file
at the top of each child file:
%
\begin{center}
|\input{|\textit{main}|}|
\end{center}
%
A simple redirection |\childdocforward{|\textit{dest}|}| is achieved by:
%
\begin{center}
|\def\jobname{|\textit{dest}|}\input{\jobname}|
\end{center}
%
The redirection with prefix
|\childdocforwardprefix[|\textit{prefix}|]{|\textit{dest}|}|
is accomplished by:
%
\begin{center}
\begin{tabular}{l}
|{\edef\jobname{\scantokens\expandafter{\jobname\noexpand}}|\\
|\def\redirectjob |\textit{prefix}|#1~~~{\gdef\jobname{|\textit{dest}|#1}}|\\
|\expandafter\redirectjob\jobname~~~}\input{\jobname}|
\end{tabular}
\end{center}

In an alternative approach,
child documents can be compiled by a specific command line
without additional code or specific definitions:
%
\begin{center}
|... -jobname "|\textit{target}|" "|[\textit{flags}]%
|\includeonly{|\textit{dest}|}\input{|\textit{main}|}"|
\end{center}
%

%%%%%%%%%%%%%%%%%%%%%%%%%%%%%%%%%%%%%%%%%%%%%%%%%%%%%%%%%%%%%%%%%%%%%%%%%%%%%%%%
%%%%%%%%%%%%%%%%%%%%%%%%%%%%%%%%%%%%%%%%%%%%%%%%%%%%%%%%%%%%%%%%%%%%%%%%%%%%%%%%
\section{Information}

%%%%%%%%%%%%%%%%%%%%%%%%%%%%%%%%%%%%%%%%%%%%%%%%%%%%%%%%%%%%%%%%%%%%%%%%%%%%%%%%
\subsection{Copyright}

Copyright \copyright{} 2017--2018 Niklas Beisert

This work may be distributed and/or modified under the
conditions of the \LaTeX{} Project Public License, either version 1.3
of this license or (at your option) any later version.
The latest version of this license is in
  \url{http://www.latex-project.org/lppl.txt}
and version 1.3 or later is part of all distributions of \LaTeX{}
version 2005/12/01 or later.

This work has the LPPL maintenance status `maintained'.

The Current Maintainer of this work is Niklas Beisert.

This work consists of the files |README.txt|, |childdoc.ins| and |childdoc.dtx|
as well as the derived files |childdoc.def|, |cdocsamp.tex|
with |cdocsch1.tex|, |cdocsch2.tex|, |cdocspt3.tex|, |cdocspt4.tex|,
|cdocsdrf.tex|, |cdocsfn1.tex|, |cdocsfn2.tex|
as well as |childdoc.pdf|.

%%%%%%%%%%%%%%%%%%%%%%%%%%%%%%%%%%%%%%%%%%%%%%%%%%%%%%%%%%%%%%%%%%%%%%%%%%%%%%%%
\subsection{Files and Installation}

The package consists of the files:
%
\begin{center}
\begin{tabular}{ll}
    |README.txt|   & readme file \\
    |childdoc.ins| & installation file \\
    |childdoc.dtx| & source file \\
    |childdoc.def| & definition file \\
    |cdocsamp.tex| & sample main file \\
    |cdocsch1.tex| & sample include file \\
    |cdocsch2.tex| & sample include file \\
    |cdocspt3.tex| & sample part file \\
    |cdocspt4.tex| & sample part file \\
    |cdocsdrf.tex| & sample redirection file \\
    |cdocsfn1.tex| & sample redirection file \\
    |cdocsfn2.tex| & sample redirection file \\
    |childdoc.pdf| & manual
\end{tabular}
\end{center}
%
The distribution consists of the files
|README.txt|, |childdoc.ins| and |childdoc.dtx|.
%
\begin{itemize}
\item
Run (pdf)\LaTeX{} on |childdoc.dtx|
to compile the manual |childdoc.pdf| (this file).
\item
Run \LaTeX{} on |childdoc.ins| to create the definitions file |childdoc.def|
and the sample |cdocsamp.tex| with include files
|cdocsch1.tex|, |cdocsch2.tex|, |cdocspt3.tex|, |cdocspt4.tex|,
|cdocsdrf.tex|, |cdocsfn1.tex|, |cdocsfn2.tex|.
Then copy the file |childdoc.def| to an appropriate directory of your \LaTeX{}
distribution, e.g.\ \textit{texmf-root}|/tex/latex/childdoc|.
\end{itemize}

%%%%%%%%%%%%%%%%%%%%%%%%%%%%%%%%%%%%%%%%%%%%%%%%%%%%%%%%%%%%%%%%%%%%%%%%%%%%%%%%
\subsection{Related CTAN Packages}

There are several other packages which offer a similar functionality:
%
\begin{itemize}
\item
The packages
\href{http://ctan.org/pkg/docmute}{\textsf{docmute}},
\href{http://ctan.org/pkg/includex}{\textsf{includex}} and
\href{http://ctan.org/pkg/standalone}{\textsf{standalone}}
provide commands to include only the document body of
a child file thus allowing both files to be compiled individually.
\item
The packages \href{http://ctan.org/pkg/subdocs}{\textsf{subdocs}}
and \href{http://ctan.org/pkg/subfiles}{\textsf{subfiles}}
provide structures in which the main and child documents can be
encapsulated and allowing them to be compiled individually.
The inclusion mechanism is different from the conventional |\include|.
\item
The package \href{http://ctan.org/pkg/combine}{\textsf{combine}}
is an elaborate solution to combine several documents into one.
\end{itemize}
%
See also the CTAN topic \href{http://ctan.org/topic/subdocs}{\textsf{subdocs}}
for further related packages.
The present package differs from the above solutions in that
a document structure constructed with the conventional |\include| mechanism
just needs two extra commands at the top of every file
such that all constituent files can be compiled individually.

%%%%%%%%%%%%%%%%%%%%%%%%%%%%%%%%%%%%%%%%%%%%%%%%%%%%%%%%%%%%%%%%%%%%%%%%%%%%%%%%
%\subsection{Feature Suggestions}
%
%The following is a list of features which may be useful for future
%versions of this package:
%%
%\begin{itemize}
%\item
%\ldots
%\end{itemize}

%%%%%%%%%%%%%%%%%%%%%%%%%%%%%%%%%%%%%%%%%%%%%%%%%%%%%%%%%%%%%%%%%%%%%%%%%%%%%%%%
\subsection{Revision History}

%%%%%%%%%%%%%%%%%%%%%%%%%%%%%%%%%%%%%%%%
\paragraph{v2.0:} 2018/12/30

\begin{itemize}
\item
immediate forward processing
\item
added |\childdocby| mechanism
\item
manual restructured
\end{itemize}

%%%%%%%%%%%%%%%%%%%%%%%%%%%%%%%%%%%%%%%%
\paragraph{v1.6:} 2018/01/17

\begin{itemize}
\item
application for development of include files
\item
corrections to manual
\end{itemize}

%%%%%%%%%%%%%%%%%%%%%%%%%%%%%%%%%%%%%%%%
\paragraph{v1.5:} 2017/05/21

\begin{itemize}
\item
more complete structuring introduced
\item
|\childdocof| introduced
\item
|\childdoc| renamed to |\childdocmain|
\item
|\childredirect| renamed to |\childdocforward| and |\childdocforwardprefix|
and functionality expanded
\end{itemize}

%%%%%%%%%%%%%%%%%%%%%%%%%%%%%%%%%%%%%%%%
\paragraph{v1.0:} 2017/04/27

\begin{itemize}
\item
manual and install package
\item
first version published on CTAN
\end{itemize}

%%%%%%%%%%%%%%%%%%%%%%%%%%%%%%%%%%%%%%%%
\paragraph{v0.6:} 2017/04/26

\begin{itemize}
\item
redirection mechanism added
\end{itemize}

%%%%%%%%%%%%%%%%%%%%%%%%%%%%%%%%%%%%%%%%
\paragraph{v0.5:} 2017/04/26

\begin{itemize}
\item
functionality in definition file
\end{itemize}


%%%%%%%%%%%%%%%%%%%%%%%%%%%%%%%%%%%%%%%%%%%%%%%%%%%%%%%%%%%%%%%%%%%%%%%%%%%%%%%%
%%%%%%%%%%%%%%%%%%%%%%%%%%%%%%%%%%%%%%%%%%%%%%%%%%%%%%%%%%%%%%%%%%%%%%%%%%%%%%%%
%%%%%%%%%%%%%%%%%%%%%%%%%%%%%%%%%%%%%%%%%%%%%%%%%%%%%%%%%%%%%%%%%%%%%%%%%%%%%%%%
\appendix

\settowidth\MacroIndent{\rmfamily\scriptsize 000\ }

 \DocInput{childdoc.dtx}

\end{document}
%</driver>
% \fi
%
% %%%%%%%%%%%%%%%%%%%%%%%%%%%%%%%%%%%%%%%%%%%%%%%%%%%%%%%%%%%%%%%%%%%%%%%%%%%%%%
% %%%%%%%%%%%%%%%%%%%%%%%%%%%%%%%%%%%%%%%%%%%%%%%%%%%%%%%%%%%%%%%%%%%%%%%%%%%%%%
% \section{Sample}
%\iffalse
%<*samplemain>
%\fi
%
% The following presents a sample document
% with two chapters, two parts, a title page,
% a compile flag as well as three forwarding files to set the flag.
% It consists of eight |.tex| files:
% \begin{center}
% \begin{tabular}{ll}
% |cdocsamp.tex|&main file\\
% |cdocsch1.tex|&include file for chapter 1\\
% |cdocsch2.tex|&include file for chapter 2\\
% |cdocspt3.tex|&include file for part 3\\
% |cdocspt4.tex|&include file for part 4\\
% |cdocsdrf.tex|&forwarding file for main file in draft mode\\
% |cdocsfi1.tex|&forwarding file for final version of chapter 1\\
% |cdocsfi2.tex|&forwarding file for final version of chapter 2\\
% \end{tabular}
% \end{center}
% Each of the eight files can be compiled directly by the \LaTeX{} compiler.
%
% %%%%%%%%%%%%%%%%%%%%%%%%%%%%%%%%%%%%%%
% \paragraph{Main File.}
%
% The main file is called |cdocsamp.tex|.
%
% Load the \textsf{childdoc} definitions and
% declare the filename for the main document:
%    \begin{macrocode}
\input{childdoc.def}
\childdocmain{}
%    \end{macrocode}

% Optional override for |\version| flag:
%    \begin{macrocode}
%%\ifchilddoc\else\providecommand{\version}{draft}\fi
%    \end{macrocode}

% Define the default values for the |\version| flag
% (|final| for the main file and |draft| for childs):
%    \begin{macrocode}
\ifchilddoc
\providecommand{\version}{draft}
\else
\providecommand{\version}{final}
\fi
%    \end{macrocode}

% Load the standard document class:
%    \begin{macrocode}
\documentclass[12pt]{article}
%    \end{macrocode}

% Start the document body:
%    \begin{macrocode}
\begin{document}
%    \end{macrocode}

% Declare a title page.
% Print title, part of document being processed and version flag:
%    \begin{macrocode}
\addtocounter{page}{-1}
\begin{center}
{\LARGE\bfseries{}childdoc example\par}
\vspace{1cm}
\ifchilddoc
\ifchilddocmanual part\else chapter\fi:
`\childdocname' of `\childdocjob'\par
\else
main document: `\childdocjob'\par
\fi
version: \version\par
\end{center}
\newpage
%    \end{macrocode}

% Manually include selected file,
% otherwise process as usual:
%    \begin{macrocode}
\ifchilddocmanual
\section*{part `\childdocname'}
\input{\childdocname}
\else
%    \end{macrocode}

% Include the two chapters:
%    \begin{macrocode}
\include{cdocsch1}
\include{cdocsch2}
%    \end{macrocode}

% Include the two parts unless only chapters should be displayed:
%    \begin{macrocode}
\ifchilddoc\else
\section{part three}
\input{cdocspt3}
\section{part four}
\input{cdocspt4}
\fi
%    \end{macrocode}

% Process as usual until here:
%    \begin{macrocode}
\fi
%    \end{macrocode}

% End of document body:
%    \begin{macrocode}
\end{document}
%    \end{macrocode}
%\iffalse
%</samplemain>
%\fi
%
% %%%%%%%%%%%%%%%%%%%%%%%%%%%%%%%%%%%%%%
% \paragraph{Chapter Include Files.}
%
% The include files are called |cdocsch1.tex| and |cdocsch2.tex|.
%
%\iffalse
%<*samplechap1|samplechap2>
%\fi

% Optional override for |\version| flag:
%    \begin{macrocode}
%%\providecommand{\version}{final}
%    \end{macrocode}

% Include the main document:
%    \begin{macrocode}
\input{childdoc.def}
\childdocof{cdocsamp}
%    \end{macrocode}

%\iffalse
%</samplechap1|samplechap2>
%\fi
%
%\iffalse
%<*samplechap1>
%\fi
% Some text for chapter 1:
%    \begin{macrocode}
\section{one}
some text in chapter one
%    \end{macrocode}

%\iffalse
%</samplechap1>
%\fi
% Some text for chapter 2:
%\iffalse
%<*samplechap2>
%\fi
%    \begin{macrocode}
\section{two}
more text in chapter two
%    \end{macrocode}

%\iffalse
%</samplechap2>
%\fi
%
% %%%%%%%%%%%%%%%%%%%%%%%%%%%%%%%%%%%%%%
% \paragraph{Part Include Files.}
%
% The include files are called |cdocspt3.tex| and |cdocspt4.tex|.
%
%\iffalse
%<*samplepart3|samplepart4>
%\fi

% Optional override for |\version| flag:
%    \begin{macrocode}
%%\providecommand{\version}{final}
%    \end{macrocode}

% Include the main document:
%    \begin{macrocode}
\input{childdoc.def}
\childdocby{cdocsamp}
%    \end{macrocode}

%\iffalse
%</samplepart3|samplepart4>
%\fi
%
%\iffalse
%<*samplepart3>
%\fi
% Some text for part 3:
%    \begin{macrocode}
some text in part three
%    \end{macrocode}

%\iffalse
%</samplepart3>
%\fi
% Some text for part 4:
%\iffalse
%<*samplepart4>
%\fi
%    \begin{macrocode}
more text in part four
%    \end{macrocode}

%\iffalse
%</samplepart4>
%\fi
%
% %%%%%%%%%%%%%%%%%%%%%%%%%%%%%%%%%%%%%%
% \paragraph{Forwarding for a Complete Draft.}
%
% The following forwarding file |cdocsdrf.tex|
% compiles the main document in draft mode:
%\iffalse
%<*sampledraft>
%\fi
%    \begin{macrocode}
\def\version{draft}
\input{childdoc.def}
\childdocforward{cdocsamp}
%    \end{macrocode}

%\iffalse
%</sampledraft>
%\fi
%
% %%%%%%%%%%%%%%%%%%%%%%%%%%%%%%%%%%%%%%
% \paragraph{Forwarding for Final Version of the Chapters.}
%
% The following forwarding files |cdocsfn1.tex| and |cdocsfn2.tex|
% (with identical content)
% compile the final versions of the child documents
% |cdocsch1.tex| and |cdocsch2.tex|, respectively:
%\iffalse
%<*samplefinal>
%\fi
%    \begin{macrocode}
\def\version{final}
\input{childdoc.def}
\childdocforwardprefix[cdocsamp]{cdocsfn}{cdocsch}
%    \end{macrocode}

%\iffalse
%</samplefinal>
%\fi
%
% %%%%%%%%%%%%%%%%%%%%%%%%%%%%%%%%%%%%%%
% \paragraph{Command Line Processing.}
%
% The following three command lines generate the output files
% |cdocscld|, |cdocscl1| and |cdocscl2|
% which should be identical to
% |cdocsdrf|, |cdocsch1| and |cdocsfn2|, respectively:
% \begin{center}
% \begin{tabular}{l}
% |latex -jobname cdocscld \|\\
% |  "\def\version{draft}\input{childdoc.def}\childdocforward{cdocsamp}"|\\
% |latex -jobname cdocscl1 \|\\
% |  "\input{childdoc.def}\childdocforward[cdocsamp]{cdocsch1}"|\\
% |latex -jobname cdocscl2 \|\\
% |  "\def\version{final}\input{childdoc.def}\childdocforward{cdocsch2}"|
% \end{tabular}
% \end{center}
% Note that the trailing backslash on each first line
% merely continues the input to the second line
% (for convenient cut ant paste).
% Furthermore, the command |latex| can be replaced by any
% of its alternative versions such as |pdflatex|.
%
% %%%%%%%%%%%%%%%%%%%%%%%%%%%%%%%%%%%%%%%%%%%%%%%%%%%%%%%%%%%%%%%%%%%%%%%%%%%%%%
% %%%%%%%%%%%%%%%%%%%%%%%%%%%%%%%%%%%%%%%%%%%%%%%%%%%%%%%%%%%%%%%%%%%%%%%%%%%%%%
% \section{Implementation}
%\iffalse
%<*package>
%\fi
%
% This section describes the definitions file |childdoc.def|.

% The definitions cannot be loaded using |\usepackage| or |\RequirePackage|
% which has a mechanism to prevent loading a style file more than once.
% When loading the definitions by means of |\input|
% multiple instances have to be prevented manually:
%\iffalse
%This code needs to be before the `\ProvidesFile' directive
%which is defined at the beginning of this file.
%Therefore it is also placed there and commented out here.
%</package>
%<*discard>
%\fi
%    \begin{macrocode}
\ifdefined\childdocmain\endinput\fi
%    \end{macrocode}
%\iffalse
%</discard>
%<*package>
%\fi
%
% \macro{\ifchilddoc}
% \macro{\ifchilddocmanual}
% The conditional |\ifchilddoc| tells whether a
% child (true) or main (false) document is being compiled.
% The conditional |\ifchilddocmanual| tells whether
% the |\includeonly| mechanism is used (false) or
% the selection of child files must be performed manually (true).
% The definitions initialise to false:
%    \begin{macrocode}
\newif\ifchilddoc
\newif\ifchilddocmanual
%    \end{macrocode}

% \macro{\childdocname}
% \macro{\childdocjob}
% The macro |\childdocname| stores the name of the main document
% to be compiled. The macro |\childdocjob| stores the name of
% the document on which the \LaTeX{} compiler was originally invoked.
% The content of |\jobname| cannot be compared
% to filenames specified in the source due to different catcodes.
% The following code rescans |\jobname|, stores the result
% in |\childdocname| and saves a copy in |\childdocjob|:
%    \begin{macrocode}
\edef\childdocname{\scantokens\expandafter{\jobname\noexpand}}
\let\childdocjob\childdocname
%    \end{macrocode}

% \macro{\childdocdisable}
% The macro |\childdocdisable| prevents the main file
% from being processed more than once.
% At this stage, the main document command |\childdocmain|
% is assumed to be called once again where it should do nothing.
% Any subsequent call to it should prevent
% a secondary processing of the main document
% It overwrites the forwarding commands
% |\childdocof| and |\childdocforward|
% with empty macros to prevent further inclusions of the main document:
%    \begin{macrocode}
\newcommand{\childdocdisable}
{
  \renewcommand{\childdocmain}[1]{\renewcommand{\childdocmain}[1]{\endinput}}
  \renewcommand{\childdocof}[1]{}
  \renewcommand{\childdocby}[2][]{}
  \renewcommand{\childdocforward}[2][]{}
  \renewcommand{\childdocdisable}{}
}
%    \end{macrocode}

% \macro{\childdocmain}
% The macro |\childdocmain| is to be called at the top of the main file
% with nothing or the main filename (without extension) as argument.
% First, it breaks loops.
% If the argument is not empty and does not match |\childdocname|
% (which is set by the first inclusion of |childdoc.def|),
% |\ifchilddoc| is set to true, |\includeonly| is applied to the child file
% and |\jobname| is set to the main file
% (for proper handling of |.aux| files):
%    \begin{macrocode}
\newcommand{\childdocmain}[1]
{
  \childdocdisable\childdocmain{}
  \if?#1?\else
    \begingroup
      \def\childdoctmp{#1}
      \ifx\childdoctmp\childdocname
        \def\childdoctmp{}
      \else
        \def\childdoctmp
        {
          \childdoctrue
          \includeonly{\childdocname}
          \def\childdocjob{#1}
          \def\jobname{#1}
        }
      \fi
      \expandafter
    \endgroup
    \childdoctmp
  \fi
}
%    \end{macrocode}

% \macro{\childdocof}
% The command |\childdocof| redirects
% compilation to the main file |#1|.
%    \begin{macrocode}
\newcommand{\childdocof}[1]
{
  \childdocdisable
  \childdoctrue
  \includeonly{\childdocname}
  \def\jobname{#1}
  \def\childdocjob{#1}
  \input{#1}
}
%    \end{macrocode}

% \macro{\childdocby}
% The command |\childdocby| ....
%    \begin{macrocode}
\newcommand{\childdocby}[2][]
{
  \childdocdisable
  \childdoctrue
  \childdocmanualtrue
  \if?#1?\else
    \def\jobname{#2}
  \fi
  \def\childdocjob{#2}
  \input{#2}
  \endinput
}
%    \end{macrocode}

% \macro{\childdocforward}
% The command |\childdocforward| redirects
% compilation to the main file or
% (if the optional argument is given) a child file.
% Parameters are set as if the main file
% or a child file starting with |\childdocof| was compiled.
% Then compilation is handed over to the main file:
%    \begin{macrocode}
\newcommand{\childdocforward}[2][]
{
  \begingroup
    \if?#1?
      \def\childdoctmp
      {
        \def\childdocname{#2}
        \def\childdocjob{#2}
        \def\jobname{#2}
        \input{#2}
        \endinput
      }
    \else
      \def\childdoctmp
      {
        \childdocdisable
        \def\childdocname{#2}
        \childdoctrue
        \includeonly{#2}
        \def\childdocjob{#1}
        \def\jobname{#1}
        \input{#1}
        \endinput
      }
    \fi
    \expandafter
  \endgroup
  \childdoctmp
}
%    \end{macrocode}

% \macro{\childdocforwardprefix}
% The command |\childdocforwardprefix| redirects
% compilation to the main or a child file by means of a pattern.
% The prefix |#1| in the current filename is replaced by |#2|
% and the suffix of the current filename is kept
% (it is assumed that the filename does not contain the substring `|~~~|'
% which is used as a delimiter).
% Compilation is handed over to the new file by |\childdocforward|:
%    \begin{macrocode}
\newcommand{\childdocforwardprefix}[3][]
{
  \begingroup
    \def\childdocextract #2##1~~~{\def\childdoctmp{\childdocforward[#1]{#3##1}}}
    \expandafter\childdocextract\childdocname~~~
    \expandafter
  \endgroup
  \childdoctmp
}
%    \end{macrocode}

% \macro{\childdoc}
% The deprecated macro |\childdoc| is a legacy version of |\childdocmain|:
%    \begin{macrocode}
\newcommand{\childdoc}{\childdocmain}
%    \end{macrocode}

% \macro{\childdocredirect}
% The deprecated macro |\childdocredirect| is a legacy version
% of |\childdocforward| and |\childdocforwardprefix|:
%    \begin{macrocode}
\newcommand{\childdocredirect}[2][]
{
  \begingroup
    \if?#1?
      \def\childdoctmp{\childdocforward{#2}}
    \else
      \def\childdoctmp{\childdocforwardprefix{#1}{#2}}
    \fi
    \expandafter
  \endgroup
  \childdoctmp
}
%    \end{macrocode}

%\iffalse
%</package>
%\fi
%
\endinput
\childdocforward{cdocsamp}"|\\
% |latex -jobname cdocscl1 \|\\
% |  "% \iffalse
%
% childdoc.dtx Copyright (C) 2017-2018 Niklas Beisert
%
% This work may be distributed and/or modified under the
% conditions of the LaTeX Project Public License, either version 1.3
% of this license or (at your option) any later version.
% The latest version of this license is in
%   http://www.latex-project.org/lppl.txt
% and version 1.3 or later is part of all distributions of LaTeX
% version 2005/12/01 or later.
%
% This work has the LPPL maintenance status `maintained'.
%
% The Current Maintainer of this work is Niklas Beisert.
%
% This work consists of the files childdoc.dtx and childdoc.ins
% and the derived files childdoc.def and cdocsamp.tex with
% cdocsch1.tex, cdocsch2.tex, cdocsdrf.tex, cdocsfn1.tex, cdocsfn2.tex.
%
%<package>\ifdefined\childdocmain\endinput\fi
%<package>\ProvidesFile{childdoc.def}[2018/12/30 v2.0 child document driver]
%<samplemain>\ProvidesFile{cdocsamp.tex}[2018/12/30 v2.0 sample for childdoc]
%<*driver>
%\ProvidesFile{childdoc.drv}[2018/12/30 v2.0 childdoc reference manual file]
\PassOptionsToClass{10pt,a4paper}{article}
\documentclass{ltxdoc}

\usepackage[margin=35mm]{geometry}
\usepackage{hyperref}
\usepackage{hyperxmp}
\usepackage[usenames]{color}

\hypersetup{colorlinks=true}
\hypersetup{pdfstartview=FitH}
\hypersetup{pdfpagemode=UseNone}
\hypersetup{pdfsource={}}
\hypersetup{pdflang={en-UK}}
\hypersetup{pdfcopyright={Copyright 2017-2018 Niklas Beisert.
  This work may be distributed and/or modified under the
  conditions of the LaTeX Project Public License, either version 1.3
  of this license or (at your option) any later version.}}
\hypersetup{pdflicenseurl={http://www.latex-project.org/lppl.txt}}
\hypersetup{pdfcontactaddress={ETH Zurich, ITP, HIT K,
  Wolfgang-Pauli-Strasse 27}}
\hypersetup{pdfcontactpostcode={8093}}
\hypersetup{pdfcontactcity={Zurich}}
\hypersetup{pdfcontactcountry={Switzerland}}
\hypersetup{pdfcontactemail={nbeisert@itp.phys.ethz.ch}}
\hypersetup{pdfcontacturl={http://people.phys.ethz.ch/\xmptilde nbeisert/}}

\newcommand{\secref}[1]{\hyperref[#1]{section \ref*{#1}}}

\parskip1ex
\parindent0pt
\let\olditemize\itemize
\def\itemize{\olditemize\parskip0pt}

\begin{document}

\title{The \textsf{childdoc} Package}
\hypersetup{pdftitle={The childdoc Package}}
\author{Niklas Beisert\\[2ex]
  Institut f\"ur Theoretische Physik\\
  Eidgen\"ossische Technische Hochschule Z\"urich\\
  Wolfgang-Pauli-Strasse 27, 8093 Z\"urich, Switzerland\\[1ex]
  \href{mailto:nbeisert@itp.phys.ethz.ch}
  {\texttt{nbeisert@itp.phys.ethz.ch}}}
\hypersetup{pdfauthor={Niklas Beisert}}
\hypersetup{pdfsubject={Manual for the LaTeX2e Package childdoc}}
\date{30 December 2018, \textsf{v2.0}}
\maketitle

\begin{abstract}\noindent
\textsf{childdoc} is a \LaTeXe{} package
that enables the direct compilation
of document sections included by |\include|
to individual files.
\end{abstract}

\begingroup
\parskip0ex
\tableofcontents
\endgroup

%%%%%%%%%%%%%%%%%%%%%%%%%%%%%%%%%%%%%%%%%%%%%%%%%%%%%%%%%%%%%%%%%%%%%%%%%%%%%%%%
%%%%%%%%%%%%%%%%%%%%%%%%%%%%%%%%%%%%%%%%%%%%%%%%%%%%%%%%%%%%%%%%%%%%%%%%%%%%%%%%
\section{Introduction}

\LaTeX{} provides a mechanism to structure a large document (such as a book)
into a main file and several child files (containing the chapters)
using the |\include| command.
This mechanism is beneficial for documents
which span hundreds of pages in order to
make the source file(s) more manageable.
Moreover, compilation can be restricted to
selected child files by means of the |\includeonly| command.
The latter feature can be used to reduce the compilation time while editing
(this was significantly more useful in the earlier days of \LaTeX{})
or to generate a smaller document which is easier to navigate.
Another application of |\includeonly| is to generate
documents consisting of selected parts of the complete document.

However, there are a few drawbacks of the plain |\include| mechanism:
\begin{itemize}
\item
The child files cannot be compiled on their own,
they can only be compiled via the main file.
A naive editing environment
(such as a text editor with an option
to have the current file processed by \LaTeX)
may require one to switch to the main file before compiling;
attempting to compile the child file produces errors.
\item
The main file must be modified (each time)
to adjust the |\includeonly| command
to the present needs. This easily leaves the main file in a messy state.
\item
The generated document will always carry the filename
of the main document. This is inconvenient if
several child files are to be compiled and
to be kept for distribution.
\end{itemize}

The present package provides a simple interface
to make child files individually compilable by \LaTeX{}.
Compiling a child file then has the same effect as compiling
the main file with an |\includeonly| command
to select the appropriate child.
Moreover the generated document will carry the name of the child
rather than the main file.
This resolves all three above issues.

This feature is meant to make the editing of books,
thesis documents and lecture notes somewhat more convenient.
However, the package can also be used efficiently for
composing a series of documents (such as exercise sheets)
which are typically distributed individually.
It then assists the author in generating the individual documents
(potentially in different versions)
as well as a document containing the collected series.
Another application is in developing style files
or other kinds of included material
where compilation of the style file could redirect
to a sample or test file.

%%%%%%%%%%%%%%%%%%%%%%%%%%%%%%%%%%%%%%%%%%%%%%%%%%%%%%%%%%%%%%%%%%%%%%%%%%%%%%%%
%%%%%%%%%%%%%%%%%%%%%%%%%%%%%%%%%%%%%%%%%%%%%%%%%%%%%%%%%%%%%%%%%%%%%%%%%%%%%%%%
\section{Usage}

First of all, the package \textsf{childdoc} is \emph{not} a standard
\LaTeXe{} |.sty| style file! Therefore it needs to be invoked in
a non-standard way.

%%%%%%%%%%%%%%%%%%%%%%%%%%%%%%%%%%%%%%%%%%%%%%%%%%%%%%%%%%%%%%%%%%%%%%%%%%%%%%%%
\subsection{Included Files}
\label{sec:include}

%%%%%%%%%%%%%%%%%%%%%%%%%%%%%%%%%%%%%%%%
\DescribeMacro{\childdocmain}
To use the package, add the commands
\begin{center}
\begin{tabular}{l}
|\input{childdoc.def}|\\
|\childdocmain{}|\\
\end{tabular}
\end{center}
at the very top of the main \LaTeX{} file,
in particular \emph{before} the |\documentclass| statement!
The argument of |\childdocmain| should be left empty
(but it must be present).

%%%%%%%%%%%%%%%%%%%%%%%%%%%%%%%%%%%%%%%%
\DescribeMacro{\childdocof}
Furthermore, add the commands
\begin{center}
\begin{tabular}{l}
|\input{childdoc.def}|\\
|\childdocof{|\textit{main}|}|\\
\end{tabular}
\end{center}
at the top of every child file \textit{child}
which is included by |\include{|\textit{child}|}|
from within the main file
(or at least for those files to be compiled individually).
The argument \textit{main} must be the filename of the main file.

There are a couple of
considerations in setting up the main and child documents:

%%%%%%%%%%%%%%%%%%%%%%%%%%%%%%%%%%%%%%%%
\paragraph{Restrictions.}

Please note the following restrictions:
\begin{itemize}
\item
|\childdocmain| must be called with one argument \textit{main}
to ensure compatibility with earlier version of the package.
It must either be empty (|\childdocmain{}|)
or precisely match the filename of the main file in which it is specified.
See \secref{sec:detection} for further information.
\item
The filename \textit{main} must be specified without the |.tex| extension.
\item
The filename \textit{main} is case sensitive
(even in case-insensitive file systems)
due to internal string comparison.
\item
The argument \textit{main} should be fully expanded, it cannot be a macro.
\item
Subdirectories and special characters should be avoided in filenames.
\item
The command |\childdocmain{|\textit{main}|}| must be followed by a whitespace.
It should not be followed immediately by another command
or by a comment mark `|%|'.
This is because the \TeX{} parser reads the token immediately following
the argument of |\childdocmain| and puts it
at the beginning of every child section;
however, a white\-space is ignored.
\end{itemize}

%%%%%%%%%%%%%%%%%%%%%%%%%%%%%%%%%%%%%%%%
\paragraph{Content of Main File.}

It is advisable to place all content in the child files included by |\include|.
Any output contained in the main file will appear in all child documents
unless suppressed manually;
it cannot be suppressed automatically by the |\includeonly| directive
and thus should normally be avoided.
A method to include some content in the main file
by means of conditional processing is described in \secref{sec:conditional}.

%%%%%%%%%%%%%%%%%%%%%%%%%%%%%%%%%%%%%%%%
\paragraph{Page Numbering.}

When only a part of the document is compiled,
the appropriate numbering of pages
(as well as other status parameters)
is determined from the |.aux| files.
The latter contain information from previous passes.
However this information needs to propagate through
all intermediate child documents.
Therefore the page numbering in child documents may well
be inconsistent until the complete document is compiled at least once.

A useful (if unconventional) way to always ensure a consistent
page numbering is to restart the numbering in each child document
and denote the pages by `\textit{child}|.|\textit{page}'
where \textit{child} represents the chapter/section number of the child file.
This can be achieved by the command
|\numberwithin{page}{|\textit{child}|}|
of the \textsf{amsmath} package
where \textit{child} can be |chapter| or |section|
depending on the chosen structuring.
Alternatively, one can modify the macro |\thepage| appropriately
and reset the counter |page| at the start of each child file.

%%%%%%%%%%%%%%%%%%%%%%%%%%%%%%%%%%%%%%%%%%%%%%%%%%%%%%%%%%%%%%%%%%%%%%%%%%%%%%%%
\subsection{Conditional Processing}
\label{sec:conditional}

The package provides a mechanism to compile different versions
of a document. To customise the versions further some conditional processing
can come in handy to distinguish which version is being compiled.
The package provides two macros to describe the compilation context:

%%%%%%%%%%%%%%%%%%%%%%%%%%%%%%%%%%%%%%%%
\DescribeMacro{\ifchilddoc}
The conditional |\ifchilddoc| distinguishes between the compilation of
child documents and the main document:
%
\begin{center}
|\ifchilddoc |\textit{child-code}| |[|\||else |\textit{main-code}]| \||fi|
\end{center}

%%%%%%%%%%%%%%%%%%%%%%%%%%%%%%%%%%%%%%%%
\DescribeMacro{\childdocname}
\DescribeMacro{\childdocjob}
The macro |\childdocname| contains the filename (without extension)
of the main or child file being processed.
Note that |\childdocjob| will always contain the name of the main file.

%%%%%%%%%%%%%%%%%%%%%%%%%%%%%%%%%%%%%%%%
\paragraph{Title Page.}

Conditional processing can be used to include a title or banner page
in the main document when proper precautions are taken.
Importantly, the code in the main file should ensure that the page counter
(as well as other status parameters which are stored in the |.aux| files)
takes the same value after the conditional processing.
Otherwise the page numbers may take divergent values
depending on which part is compiled.

For example, a title page could be declared by:
%
\begin{center}
\begin{tabular}{l}
|\ifchilddoc\||else|\\
|\addtocounter{page}{-1}|\\
\textit{code for title page}\\
|\newpage|\\
|\||fi|
\end{tabular}
\end{center}
%
A banner page for the child documents can be generated by:
%
\begin{center}
\begin{tabular}{l}
|\ifchilddoc|\\
|\addtocounter{page}{-1}|\\
\textit{code for banner page}\\
|\newpage|\\
|\||fi|
\end{tabular}
\end{center}
%
Here one could write a message such as:
\begin{center}
|This is the part \childdocname{} of \childdocjob{}.|
\end{center}

%%%%%%%%%%%%%%%%%%%%%%%%%%%%%%%%%%%%%%%%%%%%%%%%%%%%%%%%%%%%%%%%%%%%%%%%%%%%%%%%
\subsection{Flags}
\label{sec:flags}

The package makes it easy to generate different versions
of the main or child documents.
To this end compilation flags can be defined
and assigned different default values.
They will be particularly useful in conjunction
with the forwarding mechanism described in \secref{sec:forward}.

For example, it may be useful to have a flag |\version|
which can be set to |draft| or |final|.
The document source will contain some conditional code
depending on the value of |\version|.
Suppose further, the flag should default to |final| for the main file
and to |draft| for child files
which is a natural assignment for editing the document.
This is achieved by placing the following code
in the preamble of the main document
(below the |\childdocmain| directive):
%
\begin{center}
\begin{tabular}{l}
|\ifchilddoc|\\
|\providecommand{\version}{draft}|\\
|\||else|\\
|\providecommand{\version}{final}|\\
|\||fi|
\end{tabular}
\end{center}
%
The definition by |\providecommand| makes sure
that previous definitions are not overwritten.
Further statements |\providecommand{\version}{...}|
can thus be added before the above code to override it.

For the main file, one might add a line
(between |\childdocmain| and the above block)
%
\begin{center}
|%\ifchilddoc\||else\providecommand{\version}{draft}\||fi|
\end{center}
%
which can be uncommented to produce a draft version.
Likewise one can add a line to the very top of a child file
(above the |\childdocof{|\textit{main}|}| directive)
%
\begin{center}
|%\providecommand{\version}{final}|
\end{center}
%
which can be uncommented to produce the final version of this child document.

%%%%%%%%%%%%%%%%%%%%%%%%%%%%%%%%%%%%%%%%%%%%%%%%%%%%%%%%%%%%%%%%%%%%%%%%%%%%%%%%
\subsection{Forwarding}
\label{sec:forward}

Different versions of the main or child documents
using compilation flags as described in \secref{sec:flags}
can be (permanently) stored in different files
for convenient compilation, viewing and distribution.
To this end, the package defines a command
to pass on compilation to a different file:

%%%%%%%%%%%%%%%%%%%%%%%%%%%%%%%%%%%%%%%%
\DescribeMacro{\childdocforward}
The command |\childdocforward| redirects processing to
another source file:
%
\begin{center}
\begin{tabular}{l}
|\input{childdoc.def}|\\
|\childdocforward[|\textit{main}|]{|\textit{dest}|}|\\
\end{tabular}
\end{center}
%
The argument \textit{dest} is the destination file
(without extension).
It should be the main file or one of the child files.
Note that further \textsf{childdoc} directives
such as |\childdocof| and |\childdocforward|
in the indicated file will be processed in this form.
The optional argument \textit{main}
passes on directly to the main file \textit{main}
while pretending to compile the child \textit{dest}.
This form behaves as if \textit{dest}
issues |\childdocof{|\textit{main}|}| right away,
and no further \textsf{childdoc} directives will be processed.

%%%%%%%%%%%%%%%%%%%%%%%%%%%%%%%%%%%%%%%%
\DescribeMacro{\...prefix}
In the alternative form |\childdocforwardprefix|,
%
\begin{center}
\begin{tabular}{l}
|\input{childdoc.def}|\\
|\childdocforwardprefix[|\textit{main}|]{|\textit{prefix}|}{|\textit{dest}|}|
\end{tabular}
\end{center}
%
the destination file is determined by a pattern
depending on the current file:
To make this work, the current file must be called
`{\textit{prefix}\hspace{0.2em}\textit{suffix}}'
with \textit{prefix} matching precisely the argument.
Processing is then passed on to the file
`{\textit{dest}\hspace{0.2em}\textit{suffix}}'.
Surely, the same effect is achieved by
directly specifying the
argument `{\textit{dest}\hspace{0.2em}\textit{suffix}}'
in the first form.
However, that requires to set up a different file
for each child. With the alternative form of the command
all these files can have exactly the same content
which simplifies setting them up and maintaining them.

For example, the following file |draft.tex|
with a compilation flag |\version| as described in \secref{sec:flags}
compiles the main document as a draft:
%
\begin{center}
\begin{tabular}{l}
|\def\version{draft}|\\
|\input{childdoc.def}|\\
|\childdocforward{|\textit{main}|}|
\end{tabular}
\end{center}
%
Likewise, the following files |final|\textit{nn}|.tex|
compile the final version of the child document
|child|\textit{nn}|.tex|:
%
\begin{center}
\begin{tabular}{l}
|\def\version{final}|\\
|\input{childdoc.def}|\\
|\childdocforwardprefix{final}{child}|
\end{tabular}
\end{center}
%

Note that when several versions of a main file and/or of each child file
are to be generated, it may be convenient to set up a |Makefile| or
shell script to automatise the process.

%%%%%%%%%%%%%%%%%%%%%%%%%%%%%%%%%%%%%%%%%%%%%%%%%%%%%%%%%%%%%%%%%%%%%%%%%%%%%%%%
\subsection{Command Line Processing}
\label{sec:commandline}

The effect of redirection files can also be achieved by invoking
the \LaTeX{} compiler with a more elaborate command line.
Most conveniently this should be done as part
of a shell script or a |Makefile|.

When using \textsf{childdoc} in the main file, the following
command lines effectively perform a redirection
(note that depending on the shell being used,
backslashes may have to be doubled: `|\|' $\to$ `|\\|'):
%
\begin{center}
|... -jobname "|\textit{target}|" |\\|"|[\textit{flags}]%
|\input{childdoc.def}\childdocforward[|\textit{main}|]{|\textit{dest}|}"|
\end{center}
%
Here \textit{target} is the name of the output file,
\textit{main} is the name of the main file
and \textit{dest} is the name of the main or child file to be processed
(all filenames without extensions).
The optional argument \textit{main} can be omitted
if \textit{main} matches \textit{dest}.
Optionally, compilation \textit{flags} can be defined via |\def| commands.
This command line makes the \TeX{} engine believe
it is compiling the file \textit{target}
whose content is specified as the latter parameter.
The provided code then forwards the processing to
\textit{main} or \textit{dest} as described in \secref{sec:forward}.

%%%%%%%%%%%%%%%%%%%%%%%%%%%%%%%%%%%%%%%%%%%%%%%%%%%%%%%%%%%%%%%%%%%%%%%%%%%%%%%%
\subsection{Include by Input}
\label{sec:input}

Including child documents by |\include| has some restrictions by design.
Most notably, the content of a child document always occupies
its own set of pages; pages cannot be shared between child documents.
Usually, this behaviour makes perfect sense
because each child document contain an essential part of the document.
However, in some situations it may be desirable to compose
a document from a collection of parts
without having mandatory page breaks between then.
For this case, the package
provides a mechanism to include parts
by |\input| which can also be processed individually.
However, by construction this mechanism
requires manual handling of the content to be output.

%%%%%%%%%%%%%%%%%%%%%%%%%%%%%%%%%%%%%%%%
\DescribeMacro{\ifchilddocmanual}
The main file should be prepared as usual, see \secref{sec:include}.
However, the document body must make a distinction
between processing of an individual part and of the main document, e.g.:
%
\begin{center}
\begin{tabular}{l}
|\ifchilddocmanual|\\
|\input{\childdocname}|\\
|\||else|\\
\textit{document body with }|\input{|\textit{part}|}|\\
|\||fi|
\end{tabular}
\end{center}
%
The conditional |\ifchilddocmanual| is true whenever
a part to be included by |\input| is being compiled,
and the name of the part is stored in |\childdocname|.

%%%%%%%%%%%%%%%%%%%%%%%%%%%%%%%%%%%%%%%%
\DescribeMacro{\childdocby}
Each part to be included by |\input| should start with:
%
\begin{center}
\begin{tabular}{l}
|\input{childdoc.def}|\\
|\childdocby{|\textit{main}|}|\\
\end{tabular}
\end{center}
%
The directive |\childdocby| is similar to |\childdocof|
described in \secref{sec:include},
but the subsequent selection of content must be done manually.
To that end, both |\ifchilddoc| and |\ifchilddocmanual|
will be true upon processing of a part,
and the name of the part is stored in |\childdocname|.
Note that |\jobname| will be set to the filename of the current part
so that each part receives an individual |.aux| file
that does not interfere with the |.aux| file(s) of the main document.
This behaviour can be altered by the alternative form
|\childdocby[*]{|\textit{main}|}| (with a non-empty optional argument)
which uses the |.aux| file of the main document
by setting |\jobname| to \textit{main}.

%%%%%%%%%%%%%%%%%%%%%%%%%%%%%%%%%%%%%%%%%%%%%%%%%%%%%%%%%%%%%%%%%%%%%%%%%%%%%%%%
\subsection{Driver Development}
\label{sec:driver}

The \textsf{childdoc} mechanism can also be use for the development
of definition files such as \LaTeX{} styles or classes.
This case differs from the above setup with multiple parts
included by |\include| in that no |\includeonly| should be invoked.
This can be achieved by starting the include file
(before |\ProvidesPackage|) with:
%
\begin{center}
\begin{tabular}{l}
|\input{childdoc.def}|\\
|\childdocforward{|\textit{main}|}|\\
\end{tabular}
\end{center}
%
or alternatively with:
%
\begin{center}
\begin{tabular}{l}
|\input{childdoc.def}|\\
|\childdocby{|\textit{main}|}|\\
\end{tabular}
\end{center}
%
Both forms have slightly different effects as described above.
The main file is prepared as usual, see \secref{sec:include}.

%%%%%%%%%%%%%%%%%%%%%%%%%%%%%%%%%%%%%%%%%%%%%%%%%%%%%%%%%%%%%%%%%%%%%%%%%%%%%%%%
\subsection{Legacy Detection}
\label{sec:detection}

The directive |\childdocmain| in the main file can detect
whether the complete document or merely a child is to be compiled
even without using the directive |\childdocof|.
This method is deprecated because it is less robust
and there is no compelling reason to use it;
it is merely provided for backward compatibility
and it may be removed in future versions.

If the detection mechanism is to be used,
it is mandatory to correctly specify
the filename of the main file as the argument of |\childdocmain|:
%
\begin{center}
\begin{tabular}{l}
|\input{childdoc.def}|\\
|\childdocmain{|\textit{main}|}|\\
\end{tabular}
\end{center}
%
If |\jobname| does not match the argument \textit{main} of |\childdocmain|,
it is assumed that |\jobname| points to the child file to be compiled.
When using |\childdocmain| with the main file specified as argument,
it suffices to start a child file
with just |\input{|\textit{main}|}|
without loading of the package and using |\childdocof|.
If instead all processing is done
with the appropriate \textsf{childdoc} directives,
the argument of \textit{main} of |\childdocmain| can be empty.

An alternative version of the command line processing described
in \secref{sec:commandline} using the detection mechanism reads:
%
\begin{center}
|... -jobname "|\textit{target}|" "|[\textit{flags}]%
[|\def\jobname{|\textit{dest}|}|]|\input{|\textit{main}|}"|
\end{center}

%%%%%%%%%%%%%%%%%%%%%%%%%%%%%%%%%%%%%%%%%%%%%%%%%%%%%%%%%%%%%%%%%%%%%%%%%%%%%%%%
\subsection{Manual Code}
\label{sec:manual}

In case one cannot be certain whether the definitions file |childdoc.def|
is installed on the target \TeX{} distribution
and one prefers not to ship it,
it is conceivable to paste a few relevant commands into the sources.

To that end, drop all statements |\input{childdoc.def}|
and perform the replacements as outlined below.
Instead of |\childdocmain{|\textit{main}|}| add the following code
to the top of the main file:
%
\begin{center}
\begin{tabular}{l}
|\||ifdefined\childdocname\endinput\||fi\newif\ifchilddoc|\\
|\edef\childdocname{\scantokens\expandafter{\jobname\noexpand}}|\\
|\def\childdocmain{|\textit{main}|}\||ifx\childdocmain\childdocname\||else|\\
|\childdoctrue\includeonly{\childdocname}\let\jobname\childdocmain\||fi|\\
\end{tabular}
\end{center}
%
Instead of |\childdocof{|\textit{main}|}| just include the main file
at the top of each child file:
%
\begin{center}
|\input{|\textit{main}|}|
\end{center}
%
A simple redirection |\childdocforward{|\textit{dest}|}| is achieved by:
%
\begin{center}
|\def\jobname{|\textit{dest}|}\input{\jobname}|
\end{center}
%
The redirection with prefix
|\childdocforwardprefix[|\textit{prefix}|]{|\textit{dest}|}|
is accomplished by:
%
\begin{center}
\begin{tabular}{l}
|{\edef\jobname{\scantokens\expandafter{\jobname\noexpand}}|\\
|\def\redirectjob |\textit{prefix}|#1~~~{\gdef\jobname{|\textit{dest}|#1}}|\\
|\expandafter\redirectjob\jobname~~~}\input{\jobname}|
\end{tabular}
\end{center}

In an alternative approach,
child documents can be compiled by a specific command line
without additional code or specific definitions:
%
\begin{center}
|... -jobname "|\textit{target}|" "|[\textit{flags}]%
|\includeonly{|\textit{dest}|}\input{|\textit{main}|}"|
\end{center}
%

%%%%%%%%%%%%%%%%%%%%%%%%%%%%%%%%%%%%%%%%%%%%%%%%%%%%%%%%%%%%%%%%%%%%%%%%%%%%%%%%
%%%%%%%%%%%%%%%%%%%%%%%%%%%%%%%%%%%%%%%%%%%%%%%%%%%%%%%%%%%%%%%%%%%%%%%%%%%%%%%%
\section{Information}

%%%%%%%%%%%%%%%%%%%%%%%%%%%%%%%%%%%%%%%%%%%%%%%%%%%%%%%%%%%%%%%%%%%%%%%%%%%%%%%%
\subsection{Copyright}

Copyright \copyright{} 2017--2018 Niklas Beisert

This work may be distributed and/or modified under the
conditions of the \LaTeX{} Project Public License, either version 1.3
of this license or (at your option) any later version.
The latest version of this license is in
  \url{http://www.latex-project.org/lppl.txt}
and version 1.3 or later is part of all distributions of \LaTeX{}
version 2005/12/01 or later.

This work has the LPPL maintenance status `maintained'.

The Current Maintainer of this work is Niklas Beisert.

This work consists of the files |README.txt|, |childdoc.ins| and |childdoc.dtx|
as well as the derived files |childdoc.def|, |cdocsamp.tex|
with |cdocsch1.tex|, |cdocsch2.tex|, |cdocspt3.tex|, |cdocspt4.tex|,
|cdocsdrf.tex|, |cdocsfn1.tex|, |cdocsfn2.tex|
as well as |childdoc.pdf|.

%%%%%%%%%%%%%%%%%%%%%%%%%%%%%%%%%%%%%%%%%%%%%%%%%%%%%%%%%%%%%%%%%%%%%%%%%%%%%%%%
\subsection{Files and Installation}

The package consists of the files:
%
\begin{center}
\begin{tabular}{ll}
    |README.txt|   & readme file \\
    |childdoc.ins| & installation file \\
    |childdoc.dtx| & source file \\
    |childdoc.def| & definition file \\
    |cdocsamp.tex| & sample main file \\
    |cdocsch1.tex| & sample include file \\
    |cdocsch2.tex| & sample include file \\
    |cdocspt3.tex| & sample part file \\
    |cdocspt4.tex| & sample part file \\
    |cdocsdrf.tex| & sample redirection file \\
    |cdocsfn1.tex| & sample redirection file \\
    |cdocsfn2.tex| & sample redirection file \\
    |childdoc.pdf| & manual
\end{tabular}
\end{center}
%
The distribution consists of the files
|README.txt|, |childdoc.ins| and |childdoc.dtx|.
%
\begin{itemize}
\item
Run (pdf)\LaTeX{} on |childdoc.dtx|
to compile the manual |childdoc.pdf| (this file).
\item
Run \LaTeX{} on |childdoc.ins| to create the definitions file |childdoc.def|
and the sample |cdocsamp.tex| with include files
|cdocsch1.tex|, |cdocsch2.tex|, |cdocspt3.tex|, |cdocspt4.tex|,
|cdocsdrf.tex|, |cdocsfn1.tex|, |cdocsfn2.tex|.
Then copy the file |childdoc.def| to an appropriate directory of your \LaTeX{}
distribution, e.g.\ \textit{texmf-root}|/tex/latex/childdoc|.
\end{itemize}

%%%%%%%%%%%%%%%%%%%%%%%%%%%%%%%%%%%%%%%%%%%%%%%%%%%%%%%%%%%%%%%%%%%%%%%%%%%%%%%%
\subsection{Related CTAN Packages}

There are several other packages which offer a similar functionality:
%
\begin{itemize}
\item
The packages
\href{http://ctan.org/pkg/docmute}{\textsf{docmute}},
\href{http://ctan.org/pkg/includex}{\textsf{includex}} and
\href{http://ctan.org/pkg/standalone}{\textsf{standalone}}
provide commands to include only the document body of
a child file thus allowing both files to be compiled individually.
\item
The packages \href{http://ctan.org/pkg/subdocs}{\textsf{subdocs}}
and \href{http://ctan.org/pkg/subfiles}{\textsf{subfiles}}
provide structures in which the main and child documents can be
encapsulated and allowing them to be compiled individually.
The inclusion mechanism is different from the conventional |\include|.
\item
The package \href{http://ctan.org/pkg/combine}{\textsf{combine}}
is an elaborate solution to combine several documents into one.
\end{itemize}
%
See also the CTAN topic \href{http://ctan.org/topic/subdocs}{\textsf{subdocs}}
for further related packages.
The present package differs from the above solutions in that
a document structure constructed with the conventional |\include| mechanism
just needs two extra commands at the top of every file
such that all constituent files can be compiled individually.

%%%%%%%%%%%%%%%%%%%%%%%%%%%%%%%%%%%%%%%%%%%%%%%%%%%%%%%%%%%%%%%%%%%%%%%%%%%%%%%%
%\subsection{Feature Suggestions}
%
%The following is a list of features which may be useful for future
%versions of this package:
%%
%\begin{itemize}
%\item
%\ldots
%\end{itemize}

%%%%%%%%%%%%%%%%%%%%%%%%%%%%%%%%%%%%%%%%%%%%%%%%%%%%%%%%%%%%%%%%%%%%%%%%%%%%%%%%
\subsection{Revision History}

%%%%%%%%%%%%%%%%%%%%%%%%%%%%%%%%%%%%%%%%
\paragraph{v2.0:} 2018/12/30

\begin{itemize}
\item
immediate forward processing
\item
added |\childdocby| mechanism
\item
manual restructured
\end{itemize}

%%%%%%%%%%%%%%%%%%%%%%%%%%%%%%%%%%%%%%%%
\paragraph{v1.6:} 2018/01/17

\begin{itemize}
\item
application for development of include files
\item
corrections to manual
\end{itemize}

%%%%%%%%%%%%%%%%%%%%%%%%%%%%%%%%%%%%%%%%
\paragraph{v1.5:} 2017/05/21

\begin{itemize}
\item
more complete structuring introduced
\item
|\childdocof| introduced
\item
|\childdoc| renamed to |\childdocmain|
\item
|\childredirect| renamed to |\childdocforward| and |\childdocforwardprefix|
and functionality expanded
\end{itemize}

%%%%%%%%%%%%%%%%%%%%%%%%%%%%%%%%%%%%%%%%
\paragraph{v1.0:} 2017/04/27

\begin{itemize}
\item
manual and install package
\item
first version published on CTAN
\end{itemize}

%%%%%%%%%%%%%%%%%%%%%%%%%%%%%%%%%%%%%%%%
\paragraph{v0.6:} 2017/04/26

\begin{itemize}
\item
redirection mechanism added
\end{itemize}

%%%%%%%%%%%%%%%%%%%%%%%%%%%%%%%%%%%%%%%%
\paragraph{v0.5:} 2017/04/26

\begin{itemize}
\item
functionality in definition file
\end{itemize}


%%%%%%%%%%%%%%%%%%%%%%%%%%%%%%%%%%%%%%%%%%%%%%%%%%%%%%%%%%%%%%%%%%%%%%%%%%%%%%%%
%%%%%%%%%%%%%%%%%%%%%%%%%%%%%%%%%%%%%%%%%%%%%%%%%%%%%%%%%%%%%%%%%%%%%%%%%%%%%%%%
%%%%%%%%%%%%%%%%%%%%%%%%%%%%%%%%%%%%%%%%%%%%%%%%%%%%%%%%%%%%%%%%%%%%%%%%%%%%%%%%
\appendix

\settowidth\MacroIndent{\rmfamily\scriptsize 000\ }

 \DocInput{childdoc.dtx}

\end{document}
%</driver>
% \fi
%
% %%%%%%%%%%%%%%%%%%%%%%%%%%%%%%%%%%%%%%%%%%%%%%%%%%%%%%%%%%%%%%%%%%%%%%%%%%%%%%
% %%%%%%%%%%%%%%%%%%%%%%%%%%%%%%%%%%%%%%%%%%%%%%%%%%%%%%%%%%%%%%%%%%%%%%%%%%%%%%
% \section{Sample}
%\iffalse
%<*samplemain>
%\fi
%
% The following presents a sample document
% with two chapters, two parts, a title page,
% a compile flag as well as three forwarding files to set the flag.
% It consists of eight |.tex| files:
% \begin{center}
% \begin{tabular}{ll}
% |cdocsamp.tex|&main file\\
% |cdocsch1.tex|&include file for chapter 1\\
% |cdocsch2.tex|&include file for chapter 2\\
% |cdocspt3.tex|&include file for part 3\\
% |cdocspt4.tex|&include file for part 4\\
% |cdocsdrf.tex|&forwarding file for main file in draft mode\\
% |cdocsfi1.tex|&forwarding file for final version of chapter 1\\
% |cdocsfi2.tex|&forwarding file for final version of chapter 2\\
% \end{tabular}
% \end{center}
% Each of the eight files can be compiled directly by the \LaTeX{} compiler.
%
% %%%%%%%%%%%%%%%%%%%%%%%%%%%%%%%%%%%%%%
% \paragraph{Main File.}
%
% The main file is called |cdocsamp.tex|.
%
% Load the \textsf{childdoc} definitions and
% declare the filename for the main document:
%    \begin{macrocode}
\input{childdoc.def}
\childdocmain{}
%    \end{macrocode}

% Optional override for |\version| flag:
%    \begin{macrocode}
%%\ifchilddoc\else\providecommand{\version}{draft}\fi
%    \end{macrocode}

% Define the default values for the |\version| flag
% (|final| for the main file and |draft| for childs):
%    \begin{macrocode}
\ifchilddoc
\providecommand{\version}{draft}
\else
\providecommand{\version}{final}
\fi
%    \end{macrocode}

% Load the standard document class:
%    \begin{macrocode}
\documentclass[12pt]{article}
%    \end{macrocode}

% Start the document body:
%    \begin{macrocode}
\begin{document}
%    \end{macrocode}

% Declare a title page.
% Print title, part of document being processed and version flag:
%    \begin{macrocode}
\addtocounter{page}{-1}
\begin{center}
{\LARGE\bfseries{}childdoc example\par}
\vspace{1cm}
\ifchilddoc
\ifchilddocmanual part\else chapter\fi:
`\childdocname' of `\childdocjob'\par
\else
main document: `\childdocjob'\par
\fi
version: \version\par
\end{center}
\newpage
%    \end{macrocode}

% Manually include selected file,
% otherwise process as usual:
%    \begin{macrocode}
\ifchilddocmanual
\section*{part `\childdocname'}
\input{\childdocname}
\else
%    \end{macrocode}

% Include the two chapters:
%    \begin{macrocode}
\include{cdocsch1}
\include{cdocsch2}
%    \end{macrocode}

% Include the two parts unless only chapters should be displayed:
%    \begin{macrocode}
\ifchilddoc\else
\section{part three}
\input{cdocspt3}
\section{part four}
\input{cdocspt4}
\fi
%    \end{macrocode}

% Process as usual until here:
%    \begin{macrocode}
\fi
%    \end{macrocode}

% End of document body:
%    \begin{macrocode}
\end{document}
%    \end{macrocode}
%\iffalse
%</samplemain>
%\fi
%
% %%%%%%%%%%%%%%%%%%%%%%%%%%%%%%%%%%%%%%
% \paragraph{Chapter Include Files.}
%
% The include files are called |cdocsch1.tex| and |cdocsch2.tex|.
%
%\iffalse
%<*samplechap1|samplechap2>
%\fi

% Optional override for |\version| flag:
%    \begin{macrocode}
%%\providecommand{\version}{final}
%    \end{macrocode}

% Include the main document:
%    \begin{macrocode}
\input{childdoc.def}
\childdocof{cdocsamp}
%    \end{macrocode}

%\iffalse
%</samplechap1|samplechap2>
%\fi
%
%\iffalse
%<*samplechap1>
%\fi
% Some text for chapter 1:
%    \begin{macrocode}
\section{one}
some text in chapter one
%    \end{macrocode}

%\iffalse
%</samplechap1>
%\fi
% Some text for chapter 2:
%\iffalse
%<*samplechap2>
%\fi
%    \begin{macrocode}
\section{two}
more text in chapter two
%    \end{macrocode}

%\iffalse
%</samplechap2>
%\fi
%
% %%%%%%%%%%%%%%%%%%%%%%%%%%%%%%%%%%%%%%
% \paragraph{Part Include Files.}
%
% The include files are called |cdocspt3.tex| and |cdocspt4.tex|.
%
%\iffalse
%<*samplepart3|samplepart4>
%\fi

% Optional override for |\version| flag:
%    \begin{macrocode}
%%\providecommand{\version}{final}
%    \end{macrocode}

% Include the main document:
%    \begin{macrocode}
\input{childdoc.def}
\childdocby{cdocsamp}
%    \end{macrocode}

%\iffalse
%</samplepart3|samplepart4>
%\fi
%
%\iffalse
%<*samplepart3>
%\fi
% Some text for part 3:
%    \begin{macrocode}
some text in part three
%    \end{macrocode}

%\iffalse
%</samplepart3>
%\fi
% Some text for part 4:
%\iffalse
%<*samplepart4>
%\fi
%    \begin{macrocode}
more text in part four
%    \end{macrocode}

%\iffalse
%</samplepart4>
%\fi
%
% %%%%%%%%%%%%%%%%%%%%%%%%%%%%%%%%%%%%%%
% \paragraph{Forwarding for a Complete Draft.}
%
% The following forwarding file |cdocsdrf.tex|
% compiles the main document in draft mode:
%\iffalse
%<*sampledraft>
%\fi
%    \begin{macrocode}
\def\version{draft}
\input{childdoc.def}
\childdocforward{cdocsamp}
%    \end{macrocode}

%\iffalse
%</sampledraft>
%\fi
%
% %%%%%%%%%%%%%%%%%%%%%%%%%%%%%%%%%%%%%%
% \paragraph{Forwarding for Final Version of the Chapters.}
%
% The following forwarding files |cdocsfn1.tex| and |cdocsfn2.tex|
% (with identical content)
% compile the final versions of the child documents
% |cdocsch1.tex| and |cdocsch2.tex|, respectively:
%\iffalse
%<*samplefinal>
%\fi
%    \begin{macrocode}
\def\version{final}
\input{childdoc.def}
\childdocforwardprefix[cdocsamp]{cdocsfn}{cdocsch}
%    \end{macrocode}

%\iffalse
%</samplefinal>
%\fi
%
% %%%%%%%%%%%%%%%%%%%%%%%%%%%%%%%%%%%%%%
% \paragraph{Command Line Processing.}
%
% The following three command lines generate the output files
% |cdocscld|, |cdocscl1| and |cdocscl2|
% which should be identical to
% |cdocsdrf|, |cdocsch1| and |cdocsfn2|, respectively:
% \begin{center}
% \begin{tabular}{l}
% |latex -jobname cdocscld \|\\
% |  "\def\version{draft}\input{childdoc.def}\childdocforward{cdocsamp}"|\\
% |latex -jobname cdocscl1 \|\\
% |  "\input{childdoc.def}\childdocforward[cdocsamp]{cdocsch1}"|\\
% |latex -jobname cdocscl2 \|\\
% |  "\def\version{final}\input{childdoc.def}\childdocforward{cdocsch2}"|
% \end{tabular}
% \end{center}
% Note that the trailing backslash on each first line
% merely continues the input to the second line
% (for convenient cut ant paste).
% Furthermore, the command |latex| can be replaced by any
% of its alternative versions such as |pdflatex|.
%
% %%%%%%%%%%%%%%%%%%%%%%%%%%%%%%%%%%%%%%%%%%%%%%%%%%%%%%%%%%%%%%%%%%%%%%%%%%%%%%
% %%%%%%%%%%%%%%%%%%%%%%%%%%%%%%%%%%%%%%%%%%%%%%%%%%%%%%%%%%%%%%%%%%%%%%%%%%%%%%
% \section{Implementation}
%\iffalse
%<*package>
%\fi
%
% This section describes the definitions file |childdoc.def|.

% The definitions cannot be loaded using |\usepackage| or |\RequirePackage|
% which has a mechanism to prevent loading a style file more than once.
% When loading the definitions by means of |\input|
% multiple instances have to be prevented manually:
%\iffalse
%This code needs to be before the `\ProvidesFile' directive
%which is defined at the beginning of this file.
%Therefore it is also placed there and commented out here.
%</package>
%<*discard>
%\fi
%    \begin{macrocode}
\ifdefined\childdocmain\endinput\fi
%    \end{macrocode}
%\iffalse
%</discard>
%<*package>
%\fi
%
% \macro{\ifchilddoc}
% \macro{\ifchilddocmanual}
% The conditional |\ifchilddoc| tells whether a
% child (true) or main (false) document is being compiled.
% The conditional |\ifchilddocmanual| tells whether
% the |\includeonly| mechanism is used (false) or
% the selection of child files must be performed manually (true).
% The definitions initialise to false:
%    \begin{macrocode}
\newif\ifchilddoc
\newif\ifchilddocmanual
%    \end{macrocode}

% \macro{\childdocname}
% \macro{\childdocjob}
% The macro |\childdocname| stores the name of the main document
% to be compiled. The macro |\childdocjob| stores the name of
% the document on which the \LaTeX{} compiler was originally invoked.
% The content of |\jobname| cannot be compared
% to filenames specified in the source due to different catcodes.
% The following code rescans |\jobname|, stores the result
% in |\childdocname| and saves a copy in |\childdocjob|:
%    \begin{macrocode}
\edef\childdocname{\scantokens\expandafter{\jobname\noexpand}}
\let\childdocjob\childdocname
%    \end{macrocode}

% \macro{\childdocdisable}
% The macro |\childdocdisable| prevents the main file
% from being processed more than once.
% At this stage, the main document command |\childdocmain|
% is assumed to be called once again where it should do nothing.
% Any subsequent call to it should prevent
% a secondary processing of the main document
% It overwrites the forwarding commands
% |\childdocof| and |\childdocforward|
% with empty macros to prevent further inclusions of the main document:
%    \begin{macrocode}
\newcommand{\childdocdisable}
{
  \renewcommand{\childdocmain}[1]{\renewcommand{\childdocmain}[1]{\endinput}}
  \renewcommand{\childdocof}[1]{}
  \renewcommand{\childdocby}[2][]{}
  \renewcommand{\childdocforward}[2][]{}
  \renewcommand{\childdocdisable}{}
}
%    \end{macrocode}

% \macro{\childdocmain}
% The macro |\childdocmain| is to be called at the top of the main file
% with nothing or the main filename (without extension) as argument.
% First, it breaks loops.
% If the argument is not empty and does not match |\childdocname|
% (which is set by the first inclusion of |childdoc.def|),
% |\ifchilddoc| is set to true, |\includeonly| is applied to the child file
% and |\jobname| is set to the main file
% (for proper handling of |.aux| files):
%    \begin{macrocode}
\newcommand{\childdocmain}[1]
{
  \childdocdisable\childdocmain{}
  \if?#1?\else
    \begingroup
      \def\childdoctmp{#1}
      \ifx\childdoctmp\childdocname
        \def\childdoctmp{}
      \else
        \def\childdoctmp
        {
          \childdoctrue
          \includeonly{\childdocname}
          \def\childdocjob{#1}
          \def\jobname{#1}
        }
      \fi
      \expandafter
    \endgroup
    \childdoctmp
  \fi
}
%    \end{macrocode}

% \macro{\childdocof}
% The command |\childdocof| redirects
% compilation to the main file |#1|.
%    \begin{macrocode}
\newcommand{\childdocof}[1]
{
  \childdocdisable
  \childdoctrue
  \includeonly{\childdocname}
  \def\jobname{#1}
  \def\childdocjob{#1}
  \input{#1}
}
%    \end{macrocode}

% \macro{\childdocby}
% The command |\childdocby| ....
%    \begin{macrocode}
\newcommand{\childdocby}[2][]
{
  \childdocdisable
  \childdoctrue
  \childdocmanualtrue
  \if?#1?\else
    \def\jobname{#2}
  \fi
  \def\childdocjob{#2}
  \input{#2}
  \endinput
}
%    \end{macrocode}

% \macro{\childdocforward}
% The command |\childdocforward| redirects
% compilation to the main file or
% (if the optional argument is given) a child file.
% Parameters are set as if the main file
% or a child file starting with |\childdocof| was compiled.
% Then compilation is handed over to the main file:
%    \begin{macrocode}
\newcommand{\childdocforward}[2][]
{
  \begingroup
    \if?#1?
      \def\childdoctmp
      {
        \def\childdocname{#2}
        \def\childdocjob{#2}
        \def\jobname{#2}
        \input{#2}
        \endinput
      }
    \else
      \def\childdoctmp
      {
        \childdocdisable
        \def\childdocname{#2}
        \childdoctrue
        \includeonly{#2}
        \def\childdocjob{#1}
        \def\jobname{#1}
        \input{#1}
        \endinput
      }
    \fi
    \expandafter
  \endgroup
  \childdoctmp
}
%    \end{macrocode}

% \macro{\childdocforwardprefix}
% The command |\childdocforwardprefix| redirects
% compilation to the main or a child file by means of a pattern.
% The prefix |#1| in the current filename is replaced by |#2|
% and the suffix of the current filename is kept
% (it is assumed that the filename does not contain the substring `|~~~|'
% which is used as a delimiter).
% Compilation is handed over to the new file by |\childdocforward|:
%    \begin{macrocode}
\newcommand{\childdocforwardprefix}[3][]
{
  \begingroup
    \def\childdocextract #2##1~~~{\def\childdoctmp{\childdocforward[#1]{#3##1}}}
    \expandafter\childdocextract\childdocname~~~
    \expandafter
  \endgroup
  \childdoctmp
}
%    \end{macrocode}

% \macro{\childdoc}
% The deprecated macro |\childdoc| is a legacy version of |\childdocmain|:
%    \begin{macrocode}
\newcommand{\childdoc}{\childdocmain}
%    \end{macrocode}

% \macro{\childdocredirect}
% The deprecated macro |\childdocredirect| is a legacy version
% of |\childdocforward| and |\childdocforwardprefix|:
%    \begin{macrocode}
\newcommand{\childdocredirect}[2][]
{
  \begingroup
    \if?#1?
      \def\childdoctmp{\childdocforward{#2}}
    \else
      \def\childdoctmp{\childdocforwardprefix{#1}{#2}}
    \fi
    \expandafter
  \endgroup
  \childdoctmp
}
%    \end{macrocode}

%\iffalse
%</package>
%\fi
%
\endinput
\childdocforward[cdocsamp]{cdocsch1}"|\\
% |latex -jobname cdocscl2 \|\\
% |  "\def\version{final}% \iffalse
%
% childdoc.dtx Copyright (C) 2017-2018 Niklas Beisert
%
% This work may be distributed and/or modified under the
% conditions of the LaTeX Project Public License, either version 1.3
% of this license or (at your option) any later version.
% The latest version of this license is in
%   http://www.latex-project.org/lppl.txt
% and version 1.3 or later is part of all distributions of LaTeX
% version 2005/12/01 or later.
%
% This work has the LPPL maintenance status `maintained'.
%
% The Current Maintainer of this work is Niklas Beisert.
%
% This work consists of the files childdoc.dtx and childdoc.ins
% and the derived files childdoc.def and cdocsamp.tex with
% cdocsch1.tex, cdocsch2.tex, cdocsdrf.tex, cdocsfn1.tex, cdocsfn2.tex.
%
%<package>\ifdefined\childdocmain\endinput\fi
%<package>\ProvidesFile{childdoc.def}[2018/12/30 v2.0 child document driver]
%<samplemain>\ProvidesFile{cdocsamp.tex}[2018/12/30 v2.0 sample for childdoc]
%<*driver>
%\ProvidesFile{childdoc.drv}[2018/12/30 v2.0 childdoc reference manual file]
\PassOptionsToClass{10pt,a4paper}{article}
\documentclass{ltxdoc}

\usepackage[margin=35mm]{geometry}
\usepackage{hyperref}
\usepackage{hyperxmp}
\usepackage[usenames]{color}

\hypersetup{colorlinks=true}
\hypersetup{pdfstartview=FitH}
\hypersetup{pdfpagemode=UseNone}
\hypersetup{pdfsource={}}
\hypersetup{pdflang={en-UK}}
\hypersetup{pdfcopyright={Copyright 2017-2018 Niklas Beisert.
  This work may be distributed and/or modified under the
  conditions of the LaTeX Project Public License, either version 1.3
  of this license or (at your option) any later version.}}
\hypersetup{pdflicenseurl={http://www.latex-project.org/lppl.txt}}
\hypersetup{pdfcontactaddress={ETH Zurich, ITP, HIT K,
  Wolfgang-Pauli-Strasse 27}}
\hypersetup{pdfcontactpostcode={8093}}
\hypersetup{pdfcontactcity={Zurich}}
\hypersetup{pdfcontactcountry={Switzerland}}
\hypersetup{pdfcontactemail={nbeisert@itp.phys.ethz.ch}}
\hypersetup{pdfcontacturl={http://people.phys.ethz.ch/\xmptilde nbeisert/}}

\newcommand{\secref}[1]{\hyperref[#1]{section \ref*{#1}}}

\parskip1ex
\parindent0pt
\let\olditemize\itemize
\def\itemize{\olditemize\parskip0pt}

\begin{document}

\title{The \textsf{childdoc} Package}
\hypersetup{pdftitle={The childdoc Package}}
\author{Niklas Beisert\\[2ex]
  Institut f\"ur Theoretische Physik\\
  Eidgen\"ossische Technische Hochschule Z\"urich\\
  Wolfgang-Pauli-Strasse 27, 8093 Z\"urich, Switzerland\\[1ex]
  \href{mailto:nbeisert@itp.phys.ethz.ch}
  {\texttt{nbeisert@itp.phys.ethz.ch}}}
\hypersetup{pdfauthor={Niklas Beisert}}
\hypersetup{pdfsubject={Manual for the LaTeX2e Package childdoc}}
\date{30 December 2018, \textsf{v2.0}}
\maketitle

\begin{abstract}\noindent
\textsf{childdoc} is a \LaTeXe{} package
that enables the direct compilation
of document sections included by |\include|
to individual files.
\end{abstract}

\begingroup
\parskip0ex
\tableofcontents
\endgroup

%%%%%%%%%%%%%%%%%%%%%%%%%%%%%%%%%%%%%%%%%%%%%%%%%%%%%%%%%%%%%%%%%%%%%%%%%%%%%%%%
%%%%%%%%%%%%%%%%%%%%%%%%%%%%%%%%%%%%%%%%%%%%%%%%%%%%%%%%%%%%%%%%%%%%%%%%%%%%%%%%
\section{Introduction}

\LaTeX{} provides a mechanism to structure a large document (such as a book)
into a main file and several child files (containing the chapters)
using the |\include| command.
This mechanism is beneficial for documents
which span hundreds of pages in order to
make the source file(s) more manageable.
Moreover, compilation can be restricted to
selected child files by means of the |\includeonly| command.
The latter feature can be used to reduce the compilation time while editing
(this was significantly more useful in the earlier days of \LaTeX{})
or to generate a smaller document which is easier to navigate.
Another application of |\includeonly| is to generate
documents consisting of selected parts of the complete document.

However, there are a few drawbacks of the plain |\include| mechanism:
\begin{itemize}
\item
The child files cannot be compiled on their own,
they can only be compiled via the main file.
A naive editing environment
(such as a text editor with an option
to have the current file processed by \LaTeX)
may require one to switch to the main file before compiling;
attempting to compile the child file produces errors.
\item
The main file must be modified (each time)
to adjust the |\includeonly| command
to the present needs. This easily leaves the main file in a messy state.
\item
The generated document will always carry the filename
of the main document. This is inconvenient if
several child files are to be compiled and
to be kept for distribution.
\end{itemize}

The present package provides a simple interface
to make child files individually compilable by \LaTeX{}.
Compiling a child file then has the same effect as compiling
the main file with an |\includeonly| command
to select the appropriate child.
Moreover the generated document will carry the name of the child
rather than the main file.
This resolves all three above issues.

This feature is meant to make the editing of books,
thesis documents and lecture notes somewhat more convenient.
However, the package can also be used efficiently for
composing a series of documents (such as exercise sheets)
which are typically distributed individually.
It then assists the author in generating the individual documents
(potentially in different versions)
as well as a document containing the collected series.
Another application is in developing style files
or other kinds of included material
where compilation of the style file could redirect
to a sample or test file.

%%%%%%%%%%%%%%%%%%%%%%%%%%%%%%%%%%%%%%%%%%%%%%%%%%%%%%%%%%%%%%%%%%%%%%%%%%%%%%%%
%%%%%%%%%%%%%%%%%%%%%%%%%%%%%%%%%%%%%%%%%%%%%%%%%%%%%%%%%%%%%%%%%%%%%%%%%%%%%%%%
\section{Usage}

First of all, the package \textsf{childdoc} is \emph{not} a standard
\LaTeXe{} |.sty| style file! Therefore it needs to be invoked in
a non-standard way.

%%%%%%%%%%%%%%%%%%%%%%%%%%%%%%%%%%%%%%%%%%%%%%%%%%%%%%%%%%%%%%%%%%%%%%%%%%%%%%%%
\subsection{Included Files}
\label{sec:include}

%%%%%%%%%%%%%%%%%%%%%%%%%%%%%%%%%%%%%%%%
\DescribeMacro{\childdocmain}
To use the package, add the commands
\begin{center}
\begin{tabular}{l}
|\input{childdoc.def}|\\
|\childdocmain{}|\\
\end{tabular}
\end{center}
at the very top of the main \LaTeX{} file,
in particular \emph{before} the |\documentclass| statement!
The argument of |\childdocmain| should be left empty
(but it must be present).

%%%%%%%%%%%%%%%%%%%%%%%%%%%%%%%%%%%%%%%%
\DescribeMacro{\childdocof}
Furthermore, add the commands
\begin{center}
\begin{tabular}{l}
|\input{childdoc.def}|\\
|\childdocof{|\textit{main}|}|\\
\end{tabular}
\end{center}
at the top of every child file \textit{child}
which is included by |\include{|\textit{child}|}|
from within the main file
(or at least for those files to be compiled individually).
The argument \textit{main} must be the filename of the main file.

There are a couple of
considerations in setting up the main and child documents:

%%%%%%%%%%%%%%%%%%%%%%%%%%%%%%%%%%%%%%%%
\paragraph{Restrictions.}

Please note the following restrictions:
\begin{itemize}
\item
|\childdocmain| must be called with one argument \textit{main}
to ensure compatibility with earlier version of the package.
It must either be empty (|\childdocmain{}|)
or precisely match the filename of the main file in which it is specified.
See \secref{sec:detection} for further information.
\item
The filename \textit{main} must be specified without the |.tex| extension.
\item
The filename \textit{main} is case sensitive
(even in case-insensitive file systems)
due to internal string comparison.
\item
The argument \textit{main} should be fully expanded, it cannot be a macro.
\item
Subdirectories and special characters should be avoided in filenames.
\item
The command |\childdocmain{|\textit{main}|}| must be followed by a whitespace.
It should not be followed immediately by another command
or by a comment mark `|%|'.
This is because the \TeX{} parser reads the token immediately following
the argument of |\childdocmain| and puts it
at the beginning of every child section;
however, a white\-space is ignored.
\end{itemize}

%%%%%%%%%%%%%%%%%%%%%%%%%%%%%%%%%%%%%%%%
\paragraph{Content of Main File.}

It is advisable to place all content in the child files included by |\include|.
Any output contained in the main file will appear in all child documents
unless suppressed manually;
it cannot be suppressed automatically by the |\includeonly| directive
and thus should normally be avoided.
A method to include some content in the main file
by means of conditional processing is described in \secref{sec:conditional}.

%%%%%%%%%%%%%%%%%%%%%%%%%%%%%%%%%%%%%%%%
\paragraph{Page Numbering.}

When only a part of the document is compiled,
the appropriate numbering of pages
(as well as other status parameters)
is determined from the |.aux| files.
The latter contain information from previous passes.
However this information needs to propagate through
all intermediate child documents.
Therefore the page numbering in child documents may well
be inconsistent until the complete document is compiled at least once.

A useful (if unconventional) way to always ensure a consistent
page numbering is to restart the numbering in each child document
and denote the pages by `\textit{child}|.|\textit{page}'
where \textit{child} represents the chapter/section number of the child file.
This can be achieved by the command
|\numberwithin{page}{|\textit{child}|}|
of the \textsf{amsmath} package
where \textit{child} can be |chapter| or |section|
depending on the chosen structuring.
Alternatively, one can modify the macro |\thepage| appropriately
and reset the counter |page| at the start of each child file.

%%%%%%%%%%%%%%%%%%%%%%%%%%%%%%%%%%%%%%%%%%%%%%%%%%%%%%%%%%%%%%%%%%%%%%%%%%%%%%%%
\subsection{Conditional Processing}
\label{sec:conditional}

The package provides a mechanism to compile different versions
of a document. To customise the versions further some conditional processing
can come in handy to distinguish which version is being compiled.
The package provides two macros to describe the compilation context:

%%%%%%%%%%%%%%%%%%%%%%%%%%%%%%%%%%%%%%%%
\DescribeMacro{\ifchilddoc}
The conditional |\ifchilddoc| distinguishes between the compilation of
child documents and the main document:
%
\begin{center}
|\ifchilddoc |\textit{child-code}| |[|\||else |\textit{main-code}]| \||fi|
\end{center}

%%%%%%%%%%%%%%%%%%%%%%%%%%%%%%%%%%%%%%%%
\DescribeMacro{\childdocname}
\DescribeMacro{\childdocjob}
The macro |\childdocname| contains the filename (without extension)
of the main or child file being processed.
Note that |\childdocjob| will always contain the name of the main file.

%%%%%%%%%%%%%%%%%%%%%%%%%%%%%%%%%%%%%%%%
\paragraph{Title Page.}

Conditional processing can be used to include a title or banner page
in the main document when proper precautions are taken.
Importantly, the code in the main file should ensure that the page counter
(as well as other status parameters which are stored in the |.aux| files)
takes the same value after the conditional processing.
Otherwise the page numbers may take divergent values
depending on which part is compiled.

For example, a title page could be declared by:
%
\begin{center}
\begin{tabular}{l}
|\ifchilddoc\||else|\\
|\addtocounter{page}{-1}|\\
\textit{code for title page}\\
|\newpage|\\
|\||fi|
\end{tabular}
\end{center}
%
A banner page for the child documents can be generated by:
%
\begin{center}
\begin{tabular}{l}
|\ifchilddoc|\\
|\addtocounter{page}{-1}|\\
\textit{code for banner page}\\
|\newpage|\\
|\||fi|
\end{tabular}
\end{center}
%
Here one could write a message such as:
\begin{center}
|This is the part \childdocname{} of \childdocjob{}.|
\end{center}

%%%%%%%%%%%%%%%%%%%%%%%%%%%%%%%%%%%%%%%%%%%%%%%%%%%%%%%%%%%%%%%%%%%%%%%%%%%%%%%%
\subsection{Flags}
\label{sec:flags}

The package makes it easy to generate different versions
of the main or child documents.
To this end compilation flags can be defined
and assigned different default values.
They will be particularly useful in conjunction
with the forwarding mechanism described in \secref{sec:forward}.

For example, it may be useful to have a flag |\version|
which can be set to |draft| or |final|.
The document source will contain some conditional code
depending on the value of |\version|.
Suppose further, the flag should default to |final| for the main file
and to |draft| for child files
which is a natural assignment for editing the document.
This is achieved by placing the following code
in the preamble of the main document
(below the |\childdocmain| directive):
%
\begin{center}
\begin{tabular}{l}
|\ifchilddoc|\\
|\providecommand{\version}{draft}|\\
|\||else|\\
|\providecommand{\version}{final}|\\
|\||fi|
\end{tabular}
\end{center}
%
The definition by |\providecommand| makes sure
that previous definitions are not overwritten.
Further statements |\providecommand{\version}{...}|
can thus be added before the above code to override it.

For the main file, one might add a line
(between |\childdocmain| and the above block)
%
\begin{center}
|%\ifchilddoc\||else\providecommand{\version}{draft}\||fi|
\end{center}
%
which can be uncommented to produce a draft version.
Likewise one can add a line to the very top of a child file
(above the |\childdocof{|\textit{main}|}| directive)
%
\begin{center}
|%\providecommand{\version}{final}|
\end{center}
%
which can be uncommented to produce the final version of this child document.

%%%%%%%%%%%%%%%%%%%%%%%%%%%%%%%%%%%%%%%%%%%%%%%%%%%%%%%%%%%%%%%%%%%%%%%%%%%%%%%%
\subsection{Forwarding}
\label{sec:forward}

Different versions of the main or child documents
using compilation flags as described in \secref{sec:flags}
can be (permanently) stored in different files
for convenient compilation, viewing and distribution.
To this end, the package defines a command
to pass on compilation to a different file:

%%%%%%%%%%%%%%%%%%%%%%%%%%%%%%%%%%%%%%%%
\DescribeMacro{\childdocforward}
The command |\childdocforward| redirects processing to
another source file:
%
\begin{center}
\begin{tabular}{l}
|\input{childdoc.def}|\\
|\childdocforward[|\textit{main}|]{|\textit{dest}|}|\\
\end{tabular}
\end{center}
%
The argument \textit{dest} is the destination file
(without extension).
It should be the main file or one of the child files.
Note that further \textsf{childdoc} directives
such as |\childdocof| and |\childdocforward|
in the indicated file will be processed in this form.
The optional argument \textit{main}
passes on directly to the main file \textit{main}
while pretending to compile the child \textit{dest}.
This form behaves as if \textit{dest}
issues |\childdocof{|\textit{main}|}| right away,
and no further \textsf{childdoc} directives will be processed.

%%%%%%%%%%%%%%%%%%%%%%%%%%%%%%%%%%%%%%%%
\DescribeMacro{\...prefix}
In the alternative form |\childdocforwardprefix|,
%
\begin{center}
\begin{tabular}{l}
|\input{childdoc.def}|\\
|\childdocforwardprefix[|\textit{main}|]{|\textit{prefix}|}{|\textit{dest}|}|
\end{tabular}
\end{center}
%
the destination file is determined by a pattern
depending on the current file:
To make this work, the current file must be called
`{\textit{prefix}\hspace{0.2em}\textit{suffix}}'
with \textit{prefix} matching precisely the argument.
Processing is then passed on to the file
`{\textit{dest}\hspace{0.2em}\textit{suffix}}'.
Surely, the same effect is achieved by
directly specifying the
argument `{\textit{dest}\hspace{0.2em}\textit{suffix}}'
in the first form.
However, that requires to set up a different file
for each child. With the alternative form of the command
all these files can have exactly the same content
which simplifies setting them up and maintaining them.

For example, the following file |draft.tex|
with a compilation flag |\version| as described in \secref{sec:flags}
compiles the main document as a draft:
%
\begin{center}
\begin{tabular}{l}
|\def\version{draft}|\\
|\input{childdoc.def}|\\
|\childdocforward{|\textit{main}|}|
\end{tabular}
\end{center}
%
Likewise, the following files |final|\textit{nn}|.tex|
compile the final version of the child document
|child|\textit{nn}|.tex|:
%
\begin{center}
\begin{tabular}{l}
|\def\version{final}|\\
|\input{childdoc.def}|\\
|\childdocforwardprefix{final}{child}|
\end{tabular}
\end{center}
%

Note that when several versions of a main file and/or of each child file
are to be generated, it may be convenient to set up a |Makefile| or
shell script to automatise the process.

%%%%%%%%%%%%%%%%%%%%%%%%%%%%%%%%%%%%%%%%%%%%%%%%%%%%%%%%%%%%%%%%%%%%%%%%%%%%%%%%
\subsection{Command Line Processing}
\label{sec:commandline}

The effect of redirection files can also be achieved by invoking
the \LaTeX{} compiler with a more elaborate command line.
Most conveniently this should be done as part
of a shell script or a |Makefile|.

When using \textsf{childdoc} in the main file, the following
command lines effectively perform a redirection
(note that depending on the shell being used,
backslashes may have to be doubled: `|\|' $\to$ `|\\|'):
%
\begin{center}
|... -jobname "|\textit{target}|" |\\|"|[\textit{flags}]%
|\input{childdoc.def}\childdocforward[|\textit{main}|]{|\textit{dest}|}"|
\end{center}
%
Here \textit{target} is the name of the output file,
\textit{main} is the name of the main file
and \textit{dest} is the name of the main or child file to be processed
(all filenames without extensions).
The optional argument \textit{main} can be omitted
if \textit{main} matches \textit{dest}.
Optionally, compilation \textit{flags} can be defined via |\def| commands.
This command line makes the \TeX{} engine believe
it is compiling the file \textit{target}
whose content is specified as the latter parameter.
The provided code then forwards the processing to
\textit{main} or \textit{dest} as described in \secref{sec:forward}.

%%%%%%%%%%%%%%%%%%%%%%%%%%%%%%%%%%%%%%%%%%%%%%%%%%%%%%%%%%%%%%%%%%%%%%%%%%%%%%%%
\subsection{Include by Input}
\label{sec:input}

Including child documents by |\include| has some restrictions by design.
Most notably, the content of a child document always occupies
its own set of pages; pages cannot be shared between child documents.
Usually, this behaviour makes perfect sense
because each child document contain an essential part of the document.
However, in some situations it may be desirable to compose
a document from a collection of parts
without having mandatory page breaks between then.
For this case, the package
provides a mechanism to include parts
by |\input| which can also be processed individually.
However, by construction this mechanism
requires manual handling of the content to be output.

%%%%%%%%%%%%%%%%%%%%%%%%%%%%%%%%%%%%%%%%
\DescribeMacro{\ifchilddocmanual}
The main file should be prepared as usual, see \secref{sec:include}.
However, the document body must make a distinction
between processing of an individual part and of the main document, e.g.:
%
\begin{center}
\begin{tabular}{l}
|\ifchilddocmanual|\\
|\input{\childdocname}|\\
|\||else|\\
\textit{document body with }|\input{|\textit{part}|}|\\
|\||fi|
\end{tabular}
\end{center}
%
The conditional |\ifchilddocmanual| is true whenever
a part to be included by |\input| is being compiled,
and the name of the part is stored in |\childdocname|.

%%%%%%%%%%%%%%%%%%%%%%%%%%%%%%%%%%%%%%%%
\DescribeMacro{\childdocby}
Each part to be included by |\input| should start with:
%
\begin{center}
\begin{tabular}{l}
|\input{childdoc.def}|\\
|\childdocby{|\textit{main}|}|\\
\end{tabular}
\end{center}
%
The directive |\childdocby| is similar to |\childdocof|
described in \secref{sec:include},
but the subsequent selection of content must be done manually.
To that end, both |\ifchilddoc| and |\ifchilddocmanual|
will be true upon processing of a part,
and the name of the part is stored in |\childdocname|.
Note that |\jobname| will be set to the filename of the current part
so that each part receives an individual |.aux| file
that does not interfere with the |.aux| file(s) of the main document.
This behaviour can be altered by the alternative form
|\childdocby[*]{|\textit{main}|}| (with a non-empty optional argument)
which uses the |.aux| file of the main document
by setting |\jobname| to \textit{main}.

%%%%%%%%%%%%%%%%%%%%%%%%%%%%%%%%%%%%%%%%%%%%%%%%%%%%%%%%%%%%%%%%%%%%%%%%%%%%%%%%
\subsection{Driver Development}
\label{sec:driver}

The \textsf{childdoc} mechanism can also be use for the development
of definition files such as \LaTeX{} styles or classes.
This case differs from the above setup with multiple parts
included by |\include| in that no |\includeonly| should be invoked.
This can be achieved by starting the include file
(before |\ProvidesPackage|) with:
%
\begin{center}
\begin{tabular}{l}
|\input{childdoc.def}|\\
|\childdocforward{|\textit{main}|}|\\
\end{tabular}
\end{center}
%
or alternatively with:
%
\begin{center}
\begin{tabular}{l}
|\input{childdoc.def}|\\
|\childdocby{|\textit{main}|}|\\
\end{tabular}
\end{center}
%
Both forms have slightly different effects as described above.
The main file is prepared as usual, see \secref{sec:include}.

%%%%%%%%%%%%%%%%%%%%%%%%%%%%%%%%%%%%%%%%%%%%%%%%%%%%%%%%%%%%%%%%%%%%%%%%%%%%%%%%
\subsection{Legacy Detection}
\label{sec:detection}

The directive |\childdocmain| in the main file can detect
whether the complete document or merely a child is to be compiled
even without using the directive |\childdocof|.
This method is deprecated because it is less robust
and there is no compelling reason to use it;
it is merely provided for backward compatibility
and it may be removed in future versions.

If the detection mechanism is to be used,
it is mandatory to correctly specify
the filename of the main file as the argument of |\childdocmain|:
%
\begin{center}
\begin{tabular}{l}
|\input{childdoc.def}|\\
|\childdocmain{|\textit{main}|}|\\
\end{tabular}
\end{center}
%
If |\jobname| does not match the argument \textit{main} of |\childdocmain|,
it is assumed that |\jobname| points to the child file to be compiled.
When using |\childdocmain| with the main file specified as argument,
it suffices to start a child file
with just |\input{|\textit{main}|}|
without loading of the package and using |\childdocof|.
If instead all processing is done
with the appropriate \textsf{childdoc} directives,
the argument of \textit{main} of |\childdocmain| can be empty.

An alternative version of the command line processing described
in \secref{sec:commandline} using the detection mechanism reads:
%
\begin{center}
|... -jobname "|\textit{target}|" "|[\textit{flags}]%
[|\def\jobname{|\textit{dest}|}|]|\input{|\textit{main}|}"|
\end{center}

%%%%%%%%%%%%%%%%%%%%%%%%%%%%%%%%%%%%%%%%%%%%%%%%%%%%%%%%%%%%%%%%%%%%%%%%%%%%%%%%
\subsection{Manual Code}
\label{sec:manual}

In case one cannot be certain whether the definitions file |childdoc.def|
is installed on the target \TeX{} distribution
and one prefers not to ship it,
it is conceivable to paste a few relevant commands into the sources.

To that end, drop all statements |\input{childdoc.def}|
and perform the replacements as outlined below.
Instead of |\childdocmain{|\textit{main}|}| add the following code
to the top of the main file:
%
\begin{center}
\begin{tabular}{l}
|\||ifdefined\childdocname\endinput\||fi\newif\ifchilddoc|\\
|\edef\childdocname{\scantokens\expandafter{\jobname\noexpand}}|\\
|\def\childdocmain{|\textit{main}|}\||ifx\childdocmain\childdocname\||else|\\
|\childdoctrue\includeonly{\childdocname}\let\jobname\childdocmain\||fi|\\
\end{tabular}
\end{center}
%
Instead of |\childdocof{|\textit{main}|}| just include the main file
at the top of each child file:
%
\begin{center}
|\input{|\textit{main}|}|
\end{center}
%
A simple redirection |\childdocforward{|\textit{dest}|}| is achieved by:
%
\begin{center}
|\def\jobname{|\textit{dest}|}\input{\jobname}|
\end{center}
%
The redirection with prefix
|\childdocforwardprefix[|\textit{prefix}|]{|\textit{dest}|}|
is accomplished by:
%
\begin{center}
\begin{tabular}{l}
|{\edef\jobname{\scantokens\expandafter{\jobname\noexpand}}|\\
|\def\redirectjob |\textit{prefix}|#1~~~{\gdef\jobname{|\textit{dest}|#1}}|\\
|\expandafter\redirectjob\jobname~~~}\input{\jobname}|
\end{tabular}
\end{center}

In an alternative approach,
child documents can be compiled by a specific command line
without additional code or specific definitions:
%
\begin{center}
|... -jobname "|\textit{target}|" "|[\textit{flags}]%
|\includeonly{|\textit{dest}|}\input{|\textit{main}|}"|
\end{center}
%

%%%%%%%%%%%%%%%%%%%%%%%%%%%%%%%%%%%%%%%%%%%%%%%%%%%%%%%%%%%%%%%%%%%%%%%%%%%%%%%%
%%%%%%%%%%%%%%%%%%%%%%%%%%%%%%%%%%%%%%%%%%%%%%%%%%%%%%%%%%%%%%%%%%%%%%%%%%%%%%%%
\section{Information}

%%%%%%%%%%%%%%%%%%%%%%%%%%%%%%%%%%%%%%%%%%%%%%%%%%%%%%%%%%%%%%%%%%%%%%%%%%%%%%%%
\subsection{Copyright}

Copyright \copyright{} 2017--2018 Niklas Beisert

This work may be distributed and/or modified under the
conditions of the \LaTeX{} Project Public License, either version 1.3
of this license or (at your option) any later version.
The latest version of this license is in
  \url{http://www.latex-project.org/lppl.txt}
and version 1.3 or later is part of all distributions of \LaTeX{}
version 2005/12/01 or later.

This work has the LPPL maintenance status `maintained'.

The Current Maintainer of this work is Niklas Beisert.

This work consists of the files |README.txt|, |childdoc.ins| and |childdoc.dtx|
as well as the derived files |childdoc.def|, |cdocsamp.tex|
with |cdocsch1.tex|, |cdocsch2.tex|, |cdocspt3.tex|, |cdocspt4.tex|,
|cdocsdrf.tex|, |cdocsfn1.tex|, |cdocsfn2.tex|
as well as |childdoc.pdf|.

%%%%%%%%%%%%%%%%%%%%%%%%%%%%%%%%%%%%%%%%%%%%%%%%%%%%%%%%%%%%%%%%%%%%%%%%%%%%%%%%
\subsection{Files and Installation}

The package consists of the files:
%
\begin{center}
\begin{tabular}{ll}
    |README.txt|   & readme file \\
    |childdoc.ins| & installation file \\
    |childdoc.dtx| & source file \\
    |childdoc.def| & definition file \\
    |cdocsamp.tex| & sample main file \\
    |cdocsch1.tex| & sample include file \\
    |cdocsch2.tex| & sample include file \\
    |cdocspt3.tex| & sample part file \\
    |cdocspt4.tex| & sample part file \\
    |cdocsdrf.tex| & sample redirection file \\
    |cdocsfn1.tex| & sample redirection file \\
    |cdocsfn2.tex| & sample redirection file \\
    |childdoc.pdf| & manual
\end{tabular}
\end{center}
%
The distribution consists of the files
|README.txt|, |childdoc.ins| and |childdoc.dtx|.
%
\begin{itemize}
\item
Run (pdf)\LaTeX{} on |childdoc.dtx|
to compile the manual |childdoc.pdf| (this file).
\item
Run \LaTeX{} on |childdoc.ins| to create the definitions file |childdoc.def|
and the sample |cdocsamp.tex| with include files
|cdocsch1.tex|, |cdocsch2.tex|, |cdocspt3.tex|, |cdocspt4.tex|,
|cdocsdrf.tex|, |cdocsfn1.tex|, |cdocsfn2.tex|.
Then copy the file |childdoc.def| to an appropriate directory of your \LaTeX{}
distribution, e.g.\ \textit{texmf-root}|/tex/latex/childdoc|.
\end{itemize}

%%%%%%%%%%%%%%%%%%%%%%%%%%%%%%%%%%%%%%%%%%%%%%%%%%%%%%%%%%%%%%%%%%%%%%%%%%%%%%%%
\subsection{Related CTAN Packages}

There are several other packages which offer a similar functionality:
%
\begin{itemize}
\item
The packages
\href{http://ctan.org/pkg/docmute}{\textsf{docmute}},
\href{http://ctan.org/pkg/includex}{\textsf{includex}} and
\href{http://ctan.org/pkg/standalone}{\textsf{standalone}}
provide commands to include only the document body of
a child file thus allowing both files to be compiled individually.
\item
The packages \href{http://ctan.org/pkg/subdocs}{\textsf{subdocs}}
and \href{http://ctan.org/pkg/subfiles}{\textsf{subfiles}}
provide structures in which the main and child documents can be
encapsulated and allowing them to be compiled individually.
The inclusion mechanism is different from the conventional |\include|.
\item
The package \href{http://ctan.org/pkg/combine}{\textsf{combine}}
is an elaborate solution to combine several documents into one.
\end{itemize}
%
See also the CTAN topic \href{http://ctan.org/topic/subdocs}{\textsf{subdocs}}
for further related packages.
The present package differs from the above solutions in that
a document structure constructed with the conventional |\include| mechanism
just needs two extra commands at the top of every file
such that all constituent files can be compiled individually.

%%%%%%%%%%%%%%%%%%%%%%%%%%%%%%%%%%%%%%%%%%%%%%%%%%%%%%%%%%%%%%%%%%%%%%%%%%%%%%%%
%\subsection{Feature Suggestions}
%
%The following is a list of features which may be useful for future
%versions of this package:
%%
%\begin{itemize}
%\item
%\ldots
%\end{itemize}

%%%%%%%%%%%%%%%%%%%%%%%%%%%%%%%%%%%%%%%%%%%%%%%%%%%%%%%%%%%%%%%%%%%%%%%%%%%%%%%%
\subsection{Revision History}

%%%%%%%%%%%%%%%%%%%%%%%%%%%%%%%%%%%%%%%%
\paragraph{v2.0:} 2018/12/30

\begin{itemize}
\item
immediate forward processing
\item
added |\childdocby| mechanism
\item
manual restructured
\end{itemize}

%%%%%%%%%%%%%%%%%%%%%%%%%%%%%%%%%%%%%%%%
\paragraph{v1.6:} 2018/01/17

\begin{itemize}
\item
application for development of include files
\item
corrections to manual
\end{itemize}

%%%%%%%%%%%%%%%%%%%%%%%%%%%%%%%%%%%%%%%%
\paragraph{v1.5:} 2017/05/21

\begin{itemize}
\item
more complete structuring introduced
\item
|\childdocof| introduced
\item
|\childdoc| renamed to |\childdocmain|
\item
|\childredirect| renamed to |\childdocforward| and |\childdocforwardprefix|
and functionality expanded
\end{itemize}

%%%%%%%%%%%%%%%%%%%%%%%%%%%%%%%%%%%%%%%%
\paragraph{v1.0:} 2017/04/27

\begin{itemize}
\item
manual and install package
\item
first version published on CTAN
\end{itemize}

%%%%%%%%%%%%%%%%%%%%%%%%%%%%%%%%%%%%%%%%
\paragraph{v0.6:} 2017/04/26

\begin{itemize}
\item
redirection mechanism added
\end{itemize}

%%%%%%%%%%%%%%%%%%%%%%%%%%%%%%%%%%%%%%%%
\paragraph{v0.5:} 2017/04/26

\begin{itemize}
\item
functionality in definition file
\end{itemize}


%%%%%%%%%%%%%%%%%%%%%%%%%%%%%%%%%%%%%%%%%%%%%%%%%%%%%%%%%%%%%%%%%%%%%%%%%%%%%%%%
%%%%%%%%%%%%%%%%%%%%%%%%%%%%%%%%%%%%%%%%%%%%%%%%%%%%%%%%%%%%%%%%%%%%%%%%%%%%%%%%
%%%%%%%%%%%%%%%%%%%%%%%%%%%%%%%%%%%%%%%%%%%%%%%%%%%%%%%%%%%%%%%%%%%%%%%%%%%%%%%%
\appendix

\settowidth\MacroIndent{\rmfamily\scriptsize 000\ }

 \DocInput{childdoc.dtx}

\end{document}
%</driver>
% \fi
%
% %%%%%%%%%%%%%%%%%%%%%%%%%%%%%%%%%%%%%%%%%%%%%%%%%%%%%%%%%%%%%%%%%%%%%%%%%%%%%%
% %%%%%%%%%%%%%%%%%%%%%%%%%%%%%%%%%%%%%%%%%%%%%%%%%%%%%%%%%%%%%%%%%%%%%%%%%%%%%%
% \section{Sample}
%\iffalse
%<*samplemain>
%\fi
%
% The following presents a sample document
% with two chapters, two parts, a title page,
% a compile flag as well as three forwarding files to set the flag.
% It consists of eight |.tex| files:
% \begin{center}
% \begin{tabular}{ll}
% |cdocsamp.tex|&main file\\
% |cdocsch1.tex|&include file for chapter 1\\
% |cdocsch2.tex|&include file for chapter 2\\
% |cdocspt3.tex|&include file for part 3\\
% |cdocspt4.tex|&include file for part 4\\
% |cdocsdrf.tex|&forwarding file for main file in draft mode\\
% |cdocsfi1.tex|&forwarding file for final version of chapter 1\\
% |cdocsfi2.tex|&forwarding file for final version of chapter 2\\
% \end{tabular}
% \end{center}
% Each of the eight files can be compiled directly by the \LaTeX{} compiler.
%
% %%%%%%%%%%%%%%%%%%%%%%%%%%%%%%%%%%%%%%
% \paragraph{Main File.}
%
% The main file is called |cdocsamp.tex|.
%
% Load the \textsf{childdoc} definitions and
% declare the filename for the main document:
%    \begin{macrocode}
\input{childdoc.def}
\childdocmain{}
%    \end{macrocode}

% Optional override for |\version| flag:
%    \begin{macrocode}
%%\ifchilddoc\else\providecommand{\version}{draft}\fi
%    \end{macrocode}

% Define the default values for the |\version| flag
% (|final| for the main file and |draft| for childs):
%    \begin{macrocode}
\ifchilddoc
\providecommand{\version}{draft}
\else
\providecommand{\version}{final}
\fi
%    \end{macrocode}

% Load the standard document class:
%    \begin{macrocode}
\documentclass[12pt]{article}
%    \end{macrocode}

% Start the document body:
%    \begin{macrocode}
\begin{document}
%    \end{macrocode}

% Declare a title page.
% Print title, part of document being processed and version flag:
%    \begin{macrocode}
\addtocounter{page}{-1}
\begin{center}
{\LARGE\bfseries{}childdoc example\par}
\vspace{1cm}
\ifchilddoc
\ifchilddocmanual part\else chapter\fi:
`\childdocname' of `\childdocjob'\par
\else
main document: `\childdocjob'\par
\fi
version: \version\par
\end{center}
\newpage
%    \end{macrocode}

% Manually include selected file,
% otherwise process as usual:
%    \begin{macrocode}
\ifchilddocmanual
\section*{part `\childdocname'}
\input{\childdocname}
\else
%    \end{macrocode}

% Include the two chapters:
%    \begin{macrocode}
\include{cdocsch1}
\include{cdocsch2}
%    \end{macrocode}

% Include the two parts unless only chapters should be displayed:
%    \begin{macrocode}
\ifchilddoc\else
\section{part three}
\input{cdocspt3}
\section{part four}
\input{cdocspt4}
\fi
%    \end{macrocode}

% Process as usual until here:
%    \begin{macrocode}
\fi
%    \end{macrocode}

% End of document body:
%    \begin{macrocode}
\end{document}
%    \end{macrocode}
%\iffalse
%</samplemain>
%\fi
%
% %%%%%%%%%%%%%%%%%%%%%%%%%%%%%%%%%%%%%%
% \paragraph{Chapter Include Files.}
%
% The include files are called |cdocsch1.tex| and |cdocsch2.tex|.
%
%\iffalse
%<*samplechap1|samplechap2>
%\fi

% Optional override for |\version| flag:
%    \begin{macrocode}
%%\providecommand{\version}{final}
%    \end{macrocode}

% Include the main document:
%    \begin{macrocode}
\input{childdoc.def}
\childdocof{cdocsamp}
%    \end{macrocode}

%\iffalse
%</samplechap1|samplechap2>
%\fi
%
%\iffalse
%<*samplechap1>
%\fi
% Some text for chapter 1:
%    \begin{macrocode}
\section{one}
some text in chapter one
%    \end{macrocode}

%\iffalse
%</samplechap1>
%\fi
% Some text for chapter 2:
%\iffalse
%<*samplechap2>
%\fi
%    \begin{macrocode}
\section{two}
more text in chapter two
%    \end{macrocode}

%\iffalse
%</samplechap2>
%\fi
%
% %%%%%%%%%%%%%%%%%%%%%%%%%%%%%%%%%%%%%%
% \paragraph{Part Include Files.}
%
% The include files are called |cdocspt3.tex| and |cdocspt4.tex|.
%
%\iffalse
%<*samplepart3|samplepart4>
%\fi

% Optional override for |\version| flag:
%    \begin{macrocode}
%%\providecommand{\version}{final}
%    \end{macrocode}

% Include the main document:
%    \begin{macrocode}
\input{childdoc.def}
\childdocby{cdocsamp}
%    \end{macrocode}

%\iffalse
%</samplepart3|samplepart4>
%\fi
%
%\iffalse
%<*samplepart3>
%\fi
% Some text for part 3:
%    \begin{macrocode}
some text in part three
%    \end{macrocode}

%\iffalse
%</samplepart3>
%\fi
% Some text for part 4:
%\iffalse
%<*samplepart4>
%\fi
%    \begin{macrocode}
more text in part four
%    \end{macrocode}

%\iffalse
%</samplepart4>
%\fi
%
% %%%%%%%%%%%%%%%%%%%%%%%%%%%%%%%%%%%%%%
% \paragraph{Forwarding for a Complete Draft.}
%
% The following forwarding file |cdocsdrf.tex|
% compiles the main document in draft mode:
%\iffalse
%<*sampledraft>
%\fi
%    \begin{macrocode}
\def\version{draft}
\input{childdoc.def}
\childdocforward{cdocsamp}
%    \end{macrocode}

%\iffalse
%</sampledraft>
%\fi
%
% %%%%%%%%%%%%%%%%%%%%%%%%%%%%%%%%%%%%%%
% \paragraph{Forwarding for Final Version of the Chapters.}
%
% The following forwarding files |cdocsfn1.tex| and |cdocsfn2.tex|
% (with identical content)
% compile the final versions of the child documents
% |cdocsch1.tex| and |cdocsch2.tex|, respectively:
%\iffalse
%<*samplefinal>
%\fi
%    \begin{macrocode}
\def\version{final}
\input{childdoc.def}
\childdocforwardprefix[cdocsamp]{cdocsfn}{cdocsch}
%    \end{macrocode}

%\iffalse
%</samplefinal>
%\fi
%
% %%%%%%%%%%%%%%%%%%%%%%%%%%%%%%%%%%%%%%
% \paragraph{Command Line Processing.}
%
% The following three command lines generate the output files
% |cdocscld|, |cdocscl1| and |cdocscl2|
% which should be identical to
% |cdocsdrf|, |cdocsch1| and |cdocsfn2|, respectively:
% \begin{center}
% \begin{tabular}{l}
% |latex -jobname cdocscld \|\\
% |  "\def\version{draft}\input{childdoc.def}\childdocforward{cdocsamp}"|\\
% |latex -jobname cdocscl1 \|\\
% |  "\input{childdoc.def}\childdocforward[cdocsamp]{cdocsch1}"|\\
% |latex -jobname cdocscl2 \|\\
% |  "\def\version{final}\input{childdoc.def}\childdocforward{cdocsch2}"|
% \end{tabular}
% \end{center}
% Note that the trailing backslash on each first line
% merely continues the input to the second line
% (for convenient cut ant paste).
% Furthermore, the command |latex| can be replaced by any
% of its alternative versions such as |pdflatex|.
%
% %%%%%%%%%%%%%%%%%%%%%%%%%%%%%%%%%%%%%%%%%%%%%%%%%%%%%%%%%%%%%%%%%%%%%%%%%%%%%%
% %%%%%%%%%%%%%%%%%%%%%%%%%%%%%%%%%%%%%%%%%%%%%%%%%%%%%%%%%%%%%%%%%%%%%%%%%%%%%%
% \section{Implementation}
%\iffalse
%<*package>
%\fi
%
% This section describes the definitions file |childdoc.def|.

% The definitions cannot be loaded using |\usepackage| or |\RequirePackage|
% which has a mechanism to prevent loading a style file more than once.
% When loading the definitions by means of |\input|
% multiple instances have to be prevented manually:
%\iffalse
%This code needs to be before the `\ProvidesFile' directive
%which is defined at the beginning of this file.
%Therefore it is also placed there and commented out here.
%</package>
%<*discard>
%\fi
%    \begin{macrocode}
\ifdefined\childdocmain\endinput\fi
%    \end{macrocode}
%\iffalse
%</discard>
%<*package>
%\fi
%
% \macro{\ifchilddoc}
% \macro{\ifchilddocmanual}
% The conditional |\ifchilddoc| tells whether a
% child (true) or main (false) document is being compiled.
% The conditional |\ifchilddocmanual| tells whether
% the |\includeonly| mechanism is used (false) or
% the selection of child files must be performed manually (true).
% The definitions initialise to false:
%    \begin{macrocode}
\newif\ifchilddoc
\newif\ifchilddocmanual
%    \end{macrocode}

% \macro{\childdocname}
% \macro{\childdocjob}
% The macro |\childdocname| stores the name of the main document
% to be compiled. The macro |\childdocjob| stores the name of
% the document on which the \LaTeX{} compiler was originally invoked.
% The content of |\jobname| cannot be compared
% to filenames specified in the source due to different catcodes.
% The following code rescans |\jobname|, stores the result
% in |\childdocname| and saves a copy in |\childdocjob|:
%    \begin{macrocode}
\edef\childdocname{\scantokens\expandafter{\jobname\noexpand}}
\let\childdocjob\childdocname
%    \end{macrocode}

% \macro{\childdocdisable}
% The macro |\childdocdisable| prevents the main file
% from being processed more than once.
% At this stage, the main document command |\childdocmain|
% is assumed to be called once again where it should do nothing.
% Any subsequent call to it should prevent
% a secondary processing of the main document
% It overwrites the forwarding commands
% |\childdocof| and |\childdocforward|
% with empty macros to prevent further inclusions of the main document:
%    \begin{macrocode}
\newcommand{\childdocdisable}
{
  \renewcommand{\childdocmain}[1]{\renewcommand{\childdocmain}[1]{\endinput}}
  \renewcommand{\childdocof}[1]{}
  \renewcommand{\childdocby}[2][]{}
  \renewcommand{\childdocforward}[2][]{}
  \renewcommand{\childdocdisable}{}
}
%    \end{macrocode}

% \macro{\childdocmain}
% The macro |\childdocmain| is to be called at the top of the main file
% with nothing or the main filename (without extension) as argument.
% First, it breaks loops.
% If the argument is not empty and does not match |\childdocname|
% (which is set by the first inclusion of |childdoc.def|),
% |\ifchilddoc| is set to true, |\includeonly| is applied to the child file
% and |\jobname| is set to the main file
% (for proper handling of |.aux| files):
%    \begin{macrocode}
\newcommand{\childdocmain}[1]
{
  \childdocdisable\childdocmain{}
  \if?#1?\else
    \begingroup
      \def\childdoctmp{#1}
      \ifx\childdoctmp\childdocname
        \def\childdoctmp{}
      \else
        \def\childdoctmp
        {
          \childdoctrue
          \includeonly{\childdocname}
          \def\childdocjob{#1}
          \def\jobname{#1}
        }
      \fi
      \expandafter
    \endgroup
    \childdoctmp
  \fi
}
%    \end{macrocode}

% \macro{\childdocof}
% The command |\childdocof| redirects
% compilation to the main file |#1|.
%    \begin{macrocode}
\newcommand{\childdocof}[1]
{
  \childdocdisable
  \childdoctrue
  \includeonly{\childdocname}
  \def\jobname{#1}
  \def\childdocjob{#1}
  \input{#1}
}
%    \end{macrocode}

% \macro{\childdocby}
% The command |\childdocby| ....
%    \begin{macrocode}
\newcommand{\childdocby}[2][]
{
  \childdocdisable
  \childdoctrue
  \childdocmanualtrue
  \if?#1?\else
    \def\jobname{#2}
  \fi
  \def\childdocjob{#2}
  \input{#2}
  \endinput
}
%    \end{macrocode}

% \macro{\childdocforward}
% The command |\childdocforward| redirects
% compilation to the main file or
% (if the optional argument is given) a child file.
% Parameters are set as if the main file
% or a child file starting with |\childdocof| was compiled.
% Then compilation is handed over to the main file:
%    \begin{macrocode}
\newcommand{\childdocforward}[2][]
{
  \begingroup
    \if?#1?
      \def\childdoctmp
      {
        \def\childdocname{#2}
        \def\childdocjob{#2}
        \def\jobname{#2}
        \input{#2}
        \endinput
      }
    \else
      \def\childdoctmp
      {
        \childdocdisable
        \def\childdocname{#2}
        \childdoctrue
        \includeonly{#2}
        \def\childdocjob{#1}
        \def\jobname{#1}
        \input{#1}
        \endinput
      }
    \fi
    \expandafter
  \endgroup
  \childdoctmp
}
%    \end{macrocode}

% \macro{\childdocforwardprefix}
% The command |\childdocforwardprefix| redirects
% compilation to the main or a child file by means of a pattern.
% The prefix |#1| in the current filename is replaced by |#2|
% and the suffix of the current filename is kept
% (it is assumed that the filename does not contain the substring `|~~~|'
% which is used as a delimiter).
% Compilation is handed over to the new file by |\childdocforward|:
%    \begin{macrocode}
\newcommand{\childdocforwardprefix}[3][]
{
  \begingroup
    \def\childdocextract #2##1~~~{\def\childdoctmp{\childdocforward[#1]{#3##1}}}
    \expandafter\childdocextract\childdocname~~~
    \expandafter
  \endgroup
  \childdoctmp
}
%    \end{macrocode}

% \macro{\childdoc}
% The deprecated macro |\childdoc| is a legacy version of |\childdocmain|:
%    \begin{macrocode}
\newcommand{\childdoc}{\childdocmain}
%    \end{macrocode}

% \macro{\childdocredirect}
% The deprecated macro |\childdocredirect| is a legacy version
% of |\childdocforward| and |\childdocforwardprefix|:
%    \begin{macrocode}
\newcommand{\childdocredirect}[2][]
{
  \begingroup
    \if?#1?
      \def\childdoctmp{\childdocforward{#2}}
    \else
      \def\childdoctmp{\childdocforwardprefix{#1}{#2}}
    \fi
    \expandafter
  \endgroup
  \childdoctmp
}
%    \end{macrocode}

%\iffalse
%</package>
%\fi
%
\endinput
\childdocforward{cdocsch2}"|
% \end{tabular}
% \end{center}
% Note that the trailing backslash on each first line
% merely continues the input to the second line
% (for convenient cut ant paste).
% Furthermore, the command |latex| can be replaced by any
% of its alternative versions such as |pdflatex|.
%
% %%%%%%%%%%%%%%%%%%%%%%%%%%%%%%%%%%%%%%%%%%%%%%%%%%%%%%%%%%%%%%%%%%%%%%%%%%%%%%
% %%%%%%%%%%%%%%%%%%%%%%%%%%%%%%%%%%%%%%%%%%%%%%%%%%%%%%%%%%%%%%%%%%%%%%%%%%%%%%
% \section{Implementation}
%\iffalse
%<*package>
%\fi
%
% This section describes the definitions file |childdoc.def|.

% The definitions cannot be loaded using |\usepackage| or |\RequirePackage|
% which has a mechanism to prevent loading a style file more than once.
% When loading the definitions by means of |\input|
% multiple instances have to be prevented manually:
%\iffalse
%This code needs to be before the `\ProvidesFile' directive
%which is defined at the beginning of this file.
%Therefore it is also placed there and commented out here.
%</package>
%<*discard>
%\fi
%    \begin{macrocode}
\ifdefined\childdocmain\endinput\fi
%    \end{macrocode}
%\iffalse
%</discard>
%<*package>
%\fi
%
% \macro{\ifchilddoc}
% \macro{\ifchilddocmanual}
% The conditional |\ifchilddoc| tells whether a
% child (true) or main (false) document is being compiled.
% The conditional |\ifchilddocmanual| tells whether
% the |\includeonly| mechanism is used (false) or
% the selection of child files must be performed manually (true).
% The definitions initialise to false:
%    \begin{macrocode}
\newif\ifchilddoc
\newif\ifchilddocmanual
%    \end{macrocode}

% \macro{\childdocname}
% \macro{\childdocjob}
% The macro |\childdocname| stores the name of the main document
% to be compiled. The macro |\childdocjob| stores the name of
% the document on which the \LaTeX{} compiler was originally invoked.
% The content of |\jobname| cannot be compared
% to filenames specified in the source due to different catcodes.
% The following code rescans |\jobname|, stores the result
% in |\childdocname| and saves a copy in |\childdocjob|:
%    \begin{macrocode}
\edef\childdocname{\scantokens\expandafter{\jobname\noexpand}}
\let\childdocjob\childdocname
%    \end{macrocode}

% \macro{\childdocdisable}
% The macro |\childdocdisable| prevents the main file
% from being processed more than once.
% At this stage, the main document command |\childdocmain|
% is assumed to be called once again where it should do nothing.
% Any subsequent call to it should prevent
% a secondary processing of the main document
% It overwrites the forwarding commands
% |\childdocof| and |\childdocforward|
% with empty macros to prevent further inclusions of the main document:
%    \begin{macrocode}
\newcommand{\childdocdisable}
{
  \renewcommand{\childdocmain}[1]{\renewcommand{\childdocmain}[1]{\endinput}}
  \renewcommand{\childdocof}[1]{}
  \renewcommand{\childdocby}[2][]{}
  \renewcommand{\childdocforward}[2][]{}
  \renewcommand{\childdocdisable}{}
}
%    \end{macrocode}

% \macro{\childdocmain}
% The macro |\childdocmain| is to be called at the top of the main file
% with nothing or the main filename (without extension) as argument.
% First, it breaks loops.
% If the argument is not empty and does not match |\childdocname|
% (which is set by the first inclusion of |childdoc.def|),
% |\ifchilddoc| is set to true, |\includeonly| is applied to the child file
% and |\jobname| is set to the main file
% (for proper handling of |.aux| files):
%    \begin{macrocode}
\newcommand{\childdocmain}[1]
{
  \childdocdisable\childdocmain{}
  \if?#1?\else
    \begingroup
      \def\childdoctmp{#1}
      \ifx\childdoctmp\childdocname
        \def\childdoctmp{}
      \else
        \def\childdoctmp
        {
          \childdoctrue
          \includeonly{\childdocname}
          \def\childdocjob{#1}
          \def\jobname{#1}
        }
      \fi
      \expandafter
    \endgroup
    \childdoctmp
  \fi
}
%    \end{macrocode}

% \macro{\childdocof}
% The command |\childdocof| redirects
% compilation to the main file |#1|.
%    \begin{macrocode}
\newcommand{\childdocof}[1]
{
  \childdocdisable
  \childdoctrue
  \includeonly{\childdocname}
  \def\jobname{#1}
  \def\childdocjob{#1}
  \input{#1}
}
%    \end{macrocode}

% \macro{\childdocby}
% The command |\childdocby| ....
%    \begin{macrocode}
\newcommand{\childdocby}[2][]
{
  \childdocdisable
  \childdoctrue
  \childdocmanualtrue
  \if?#1?\else
    \def\jobname{#2}
  \fi
  \def\childdocjob{#2}
  \input{#2}
  \endinput
}
%    \end{macrocode}

% \macro{\childdocforward}
% The command |\childdocforward| redirects
% compilation to the main file or
% (if the optional argument is given) a child file.
% Parameters are set as if the main file
% or a child file starting with |\childdocof| was compiled.
% Then compilation is handed over to the main file:
%    \begin{macrocode}
\newcommand{\childdocforward}[2][]
{
  \begingroup
    \if?#1?
      \def\childdoctmp
      {
        \def\childdocname{#2}
        \def\childdocjob{#2}
        \def\jobname{#2}
        \input{#2}
        \endinput
      }
    \else
      \def\childdoctmp
      {
        \childdocdisable
        \def\childdocname{#2}
        \childdoctrue
        \includeonly{#2}
        \def\childdocjob{#1}
        \def\jobname{#1}
        \input{#1}
        \endinput
      }
    \fi
    \expandafter
  \endgroup
  \childdoctmp
}
%    \end{macrocode}

% \macro{\childdocforwardprefix}
% The command |\childdocforwardprefix| redirects
% compilation to the main or a child file by means of a pattern.
% The prefix |#1| in the current filename is replaced by |#2|
% and the suffix of the current filename is kept
% (it is assumed that the filename does not contain the substring `|~~~|'
% which is used as a delimiter).
% Compilation is handed over to the new file by |\childdocforward|:
%    \begin{macrocode}
\newcommand{\childdocforwardprefix}[3][]
{
  \begingroup
    \def\childdocextract #2##1~~~{\def\childdoctmp{\childdocforward[#1]{#3##1}}}
    \expandafter\childdocextract\childdocname~~~
    \expandafter
  \endgroup
  \childdoctmp
}
%    \end{macrocode}

% \macro{\childdoc}
% The deprecated macro |\childdoc| is a legacy version of |\childdocmain|:
%    \begin{macrocode}
\newcommand{\childdoc}{\childdocmain}
%    \end{macrocode}

% \macro{\childdocredirect}
% The deprecated macro |\childdocredirect| is a legacy version
% of |\childdocforward| and |\childdocforwardprefix|:
%    \begin{macrocode}
\newcommand{\childdocredirect}[2][]
{
  \begingroup
    \if?#1?
      \def\childdoctmp{\childdocforward{#2}}
    \else
      \def\childdoctmp{\childdocforwardprefix{#1}{#2}}
    \fi
    \expandafter
  \endgroup
  \childdoctmp
}
%    \end{macrocode}

%\iffalse
%</package>
%\fi
%
\endinput

\childdocforwardprefix[cdocsamp]{cdocsfn}{cdocsch}
%    \end{macrocode}

%\iffalse
%</samplefinal>
%\fi
%
% %%%%%%%%%%%%%%%%%%%%%%%%%%%%%%%%%%%%%%
% \paragraph{Command Line Processing.}
%
% The following three command lines generate the output files
% |cdocscld|, |cdocscl1| and |cdocscl2|
% which should be identical to
% |cdocsdrf|, |cdocsch1| and |cdocsfn2|, respectively:
% \begin{center}
% \begin{tabular}{l}
% |latex -jobname cdocscld \|\\
% |  "\def\version{draft}% \iffalse
%
% childdoc.dtx Copyright (C) 2017-2018 Niklas Beisert
%
% This work may be distributed and/or modified under the
% conditions of the LaTeX Project Public License, either version 1.3
% of this license or (at your option) any later version.
% The latest version of this license is in
%   http://www.latex-project.org/lppl.txt
% and version 1.3 or later is part of all distributions of LaTeX
% version 2005/12/01 or later.
%
% This work has the LPPL maintenance status `maintained'.
%
% The Current Maintainer of this work is Niklas Beisert.
%
% This work consists of the files childdoc.dtx and childdoc.ins
% and the derived files childdoc.def and cdocsamp.tex with
% cdocsch1.tex, cdocsch2.tex, cdocsdrf.tex, cdocsfn1.tex, cdocsfn2.tex.
%
%<package>\ifdefined\childdocmain\endinput\fi
%<package>\ProvidesFile{childdoc.def}[2018/12/30 v2.0 child document driver]
%<samplemain>\ProvidesFile{cdocsamp.tex}[2018/12/30 v2.0 sample for childdoc]
%<*driver>
%\ProvidesFile{childdoc.drv}[2018/12/30 v2.0 childdoc reference manual file]
\PassOptionsToClass{10pt,a4paper}{article}
\documentclass{ltxdoc}

\usepackage[margin=35mm]{geometry}
\usepackage{hyperref}
\usepackage{hyperxmp}
\usepackage[usenames]{color}

\hypersetup{colorlinks=true}
\hypersetup{pdfstartview=FitH}
\hypersetup{pdfpagemode=UseNone}
\hypersetup{pdfsource={}}
\hypersetup{pdflang={en-UK}}
\hypersetup{pdfcopyright={Copyright 2017-2018 Niklas Beisert.
  This work may be distributed and/or modified under the
  conditions of the LaTeX Project Public License, either version 1.3
  of this license or (at your option) any later version.}}
\hypersetup{pdflicenseurl={http://www.latex-project.org/lppl.txt}}
\hypersetup{pdfcontactaddress={ETH Zurich, ITP, HIT K,
  Wolfgang-Pauli-Strasse 27}}
\hypersetup{pdfcontactpostcode={8093}}
\hypersetup{pdfcontactcity={Zurich}}
\hypersetup{pdfcontactcountry={Switzerland}}
\hypersetup{pdfcontactemail={nbeisert@itp.phys.ethz.ch}}
\hypersetup{pdfcontacturl={http://people.phys.ethz.ch/\xmptilde nbeisert/}}

\newcommand{\secref}[1]{\hyperref[#1]{section \ref*{#1}}}

\parskip1ex
\parindent0pt
\let\olditemize\itemize
\def\itemize{\olditemize\parskip0pt}

\begin{document}

\title{The \textsf{childdoc} Package}
\hypersetup{pdftitle={The childdoc Package}}
\author{Niklas Beisert\\[2ex]
  Institut f\"ur Theoretische Physik\\
  Eidgen\"ossische Technische Hochschule Z\"urich\\
  Wolfgang-Pauli-Strasse 27, 8093 Z\"urich, Switzerland\\[1ex]
  \href{mailto:nbeisert@itp.phys.ethz.ch}
  {\texttt{nbeisert@itp.phys.ethz.ch}}}
\hypersetup{pdfauthor={Niklas Beisert}}
\hypersetup{pdfsubject={Manual for the LaTeX2e Package childdoc}}
\date{30 December 2018, \textsf{v2.0}}
\maketitle

\begin{abstract}\noindent
\textsf{childdoc} is a \LaTeXe{} package
that enables the direct compilation
of document sections included by |\include|
to individual files.
\end{abstract}

\begingroup
\parskip0ex
\tableofcontents
\endgroup

%%%%%%%%%%%%%%%%%%%%%%%%%%%%%%%%%%%%%%%%%%%%%%%%%%%%%%%%%%%%%%%%%%%%%%%%%%%%%%%%
%%%%%%%%%%%%%%%%%%%%%%%%%%%%%%%%%%%%%%%%%%%%%%%%%%%%%%%%%%%%%%%%%%%%%%%%%%%%%%%%
\section{Introduction}

\LaTeX{} provides a mechanism to structure a large document (such as a book)
into a main file and several child files (containing the chapters)
using the |\include| command.
This mechanism is beneficial for documents
which span hundreds of pages in order to
make the source file(s) more manageable.
Moreover, compilation can be restricted to
selected child files by means of the |\includeonly| command.
The latter feature can be used to reduce the compilation time while editing
(this was significantly more useful in the earlier days of \LaTeX{})
or to generate a smaller document which is easier to navigate.
Another application of |\includeonly| is to generate
documents consisting of selected parts of the complete document.

However, there are a few drawbacks of the plain |\include| mechanism:
\begin{itemize}
\item
The child files cannot be compiled on their own,
they can only be compiled via the main file.
A naive editing environment
(such as a text editor with an option
to have the current file processed by \LaTeX)
may require one to switch to the main file before compiling;
attempting to compile the child file produces errors.
\item
The main file must be modified (each time)
to adjust the |\includeonly| command
to the present needs. This easily leaves the main file in a messy state.
\item
The generated document will always carry the filename
of the main document. This is inconvenient if
several child files are to be compiled and
to be kept for distribution.
\end{itemize}

The present package provides a simple interface
to make child files individually compilable by \LaTeX{}.
Compiling a child file then has the same effect as compiling
the main file with an |\includeonly| command
to select the appropriate child.
Moreover the generated document will carry the name of the child
rather than the main file.
This resolves all three above issues.

This feature is meant to make the editing of books,
thesis documents and lecture notes somewhat more convenient.
However, the package can also be used efficiently for
composing a series of documents (such as exercise sheets)
which are typically distributed individually.
It then assists the author in generating the individual documents
(potentially in different versions)
as well as a document containing the collected series.
Another application is in developing style files
or other kinds of included material
where compilation of the style file could redirect
to a sample or test file.

%%%%%%%%%%%%%%%%%%%%%%%%%%%%%%%%%%%%%%%%%%%%%%%%%%%%%%%%%%%%%%%%%%%%%%%%%%%%%%%%
%%%%%%%%%%%%%%%%%%%%%%%%%%%%%%%%%%%%%%%%%%%%%%%%%%%%%%%%%%%%%%%%%%%%%%%%%%%%%%%%
\section{Usage}

First of all, the package \textsf{childdoc} is \emph{not} a standard
\LaTeXe{} |.sty| style file! Therefore it needs to be invoked in
a non-standard way.

%%%%%%%%%%%%%%%%%%%%%%%%%%%%%%%%%%%%%%%%%%%%%%%%%%%%%%%%%%%%%%%%%%%%%%%%%%%%%%%%
\subsection{Included Files}
\label{sec:include}

%%%%%%%%%%%%%%%%%%%%%%%%%%%%%%%%%%%%%%%%
\DescribeMacro{\childdocmain}
To use the package, add the commands
\begin{center}
\begin{tabular}{l}
|% \iffalse
%
% childdoc.dtx Copyright (C) 2017-2018 Niklas Beisert
%
% This work may be distributed and/or modified under the
% conditions of the LaTeX Project Public License, either version 1.3
% of this license or (at your option) any later version.
% The latest version of this license is in
%   http://www.latex-project.org/lppl.txt
% and version 1.3 or later is part of all distributions of LaTeX
% version 2005/12/01 or later.
%
% This work has the LPPL maintenance status `maintained'.
%
% The Current Maintainer of this work is Niklas Beisert.
%
% This work consists of the files childdoc.dtx and childdoc.ins
% and the derived files childdoc.def and cdocsamp.tex with
% cdocsch1.tex, cdocsch2.tex, cdocsdrf.tex, cdocsfn1.tex, cdocsfn2.tex.
%
%<package>\ifdefined\childdocmain\endinput\fi
%<package>\ProvidesFile{childdoc.def}[2018/12/30 v2.0 child document driver]
%<samplemain>\ProvidesFile{cdocsamp.tex}[2018/12/30 v2.0 sample for childdoc]
%<*driver>
%\ProvidesFile{childdoc.drv}[2018/12/30 v2.0 childdoc reference manual file]
\PassOptionsToClass{10pt,a4paper}{article}
\documentclass{ltxdoc}

\usepackage[margin=35mm]{geometry}
\usepackage{hyperref}
\usepackage{hyperxmp}
\usepackage[usenames]{color}

\hypersetup{colorlinks=true}
\hypersetup{pdfstartview=FitH}
\hypersetup{pdfpagemode=UseNone}
\hypersetup{pdfsource={}}
\hypersetup{pdflang={en-UK}}
\hypersetup{pdfcopyright={Copyright 2017-2018 Niklas Beisert.
  This work may be distributed and/or modified under the
  conditions of the LaTeX Project Public License, either version 1.3
  of this license or (at your option) any later version.}}
\hypersetup{pdflicenseurl={http://www.latex-project.org/lppl.txt}}
\hypersetup{pdfcontactaddress={ETH Zurich, ITP, HIT K,
  Wolfgang-Pauli-Strasse 27}}
\hypersetup{pdfcontactpostcode={8093}}
\hypersetup{pdfcontactcity={Zurich}}
\hypersetup{pdfcontactcountry={Switzerland}}
\hypersetup{pdfcontactemail={nbeisert@itp.phys.ethz.ch}}
\hypersetup{pdfcontacturl={http://people.phys.ethz.ch/\xmptilde nbeisert/}}

\newcommand{\secref}[1]{\hyperref[#1]{section \ref*{#1}}}

\parskip1ex
\parindent0pt
\let\olditemize\itemize
\def\itemize{\olditemize\parskip0pt}

\begin{document}

\title{The \textsf{childdoc} Package}
\hypersetup{pdftitle={The childdoc Package}}
\author{Niklas Beisert\\[2ex]
  Institut f\"ur Theoretische Physik\\
  Eidgen\"ossische Technische Hochschule Z\"urich\\
  Wolfgang-Pauli-Strasse 27, 8093 Z\"urich, Switzerland\\[1ex]
  \href{mailto:nbeisert@itp.phys.ethz.ch}
  {\texttt{nbeisert@itp.phys.ethz.ch}}}
\hypersetup{pdfauthor={Niklas Beisert}}
\hypersetup{pdfsubject={Manual for the LaTeX2e Package childdoc}}
\date{30 December 2018, \textsf{v2.0}}
\maketitle

\begin{abstract}\noindent
\textsf{childdoc} is a \LaTeXe{} package
that enables the direct compilation
of document sections included by |\include|
to individual files.
\end{abstract}

\begingroup
\parskip0ex
\tableofcontents
\endgroup

%%%%%%%%%%%%%%%%%%%%%%%%%%%%%%%%%%%%%%%%%%%%%%%%%%%%%%%%%%%%%%%%%%%%%%%%%%%%%%%%
%%%%%%%%%%%%%%%%%%%%%%%%%%%%%%%%%%%%%%%%%%%%%%%%%%%%%%%%%%%%%%%%%%%%%%%%%%%%%%%%
\section{Introduction}

\LaTeX{} provides a mechanism to structure a large document (such as a book)
into a main file and several child files (containing the chapters)
using the |\include| command.
This mechanism is beneficial for documents
which span hundreds of pages in order to
make the source file(s) more manageable.
Moreover, compilation can be restricted to
selected child files by means of the |\includeonly| command.
The latter feature can be used to reduce the compilation time while editing
(this was significantly more useful in the earlier days of \LaTeX{})
or to generate a smaller document which is easier to navigate.
Another application of |\includeonly| is to generate
documents consisting of selected parts of the complete document.

However, there are a few drawbacks of the plain |\include| mechanism:
\begin{itemize}
\item
The child files cannot be compiled on their own,
they can only be compiled via the main file.
A naive editing environment
(such as a text editor with an option
to have the current file processed by \LaTeX)
may require one to switch to the main file before compiling;
attempting to compile the child file produces errors.
\item
The main file must be modified (each time)
to adjust the |\includeonly| command
to the present needs. This easily leaves the main file in a messy state.
\item
The generated document will always carry the filename
of the main document. This is inconvenient if
several child files are to be compiled and
to be kept for distribution.
\end{itemize}

The present package provides a simple interface
to make child files individually compilable by \LaTeX{}.
Compiling a child file then has the same effect as compiling
the main file with an |\includeonly| command
to select the appropriate child.
Moreover the generated document will carry the name of the child
rather than the main file.
This resolves all three above issues.

This feature is meant to make the editing of books,
thesis documents and lecture notes somewhat more convenient.
However, the package can also be used efficiently for
composing a series of documents (such as exercise sheets)
which are typically distributed individually.
It then assists the author in generating the individual documents
(potentially in different versions)
as well as a document containing the collected series.
Another application is in developing style files
or other kinds of included material
where compilation of the style file could redirect
to a sample or test file.

%%%%%%%%%%%%%%%%%%%%%%%%%%%%%%%%%%%%%%%%%%%%%%%%%%%%%%%%%%%%%%%%%%%%%%%%%%%%%%%%
%%%%%%%%%%%%%%%%%%%%%%%%%%%%%%%%%%%%%%%%%%%%%%%%%%%%%%%%%%%%%%%%%%%%%%%%%%%%%%%%
\section{Usage}

First of all, the package \textsf{childdoc} is \emph{not} a standard
\LaTeXe{} |.sty| style file! Therefore it needs to be invoked in
a non-standard way.

%%%%%%%%%%%%%%%%%%%%%%%%%%%%%%%%%%%%%%%%%%%%%%%%%%%%%%%%%%%%%%%%%%%%%%%%%%%%%%%%
\subsection{Included Files}
\label{sec:include}

%%%%%%%%%%%%%%%%%%%%%%%%%%%%%%%%%%%%%%%%
\DescribeMacro{\childdocmain}
To use the package, add the commands
\begin{center}
\begin{tabular}{l}
|\input{childdoc.def}|\\
|\childdocmain{}|\\
\end{tabular}
\end{center}
at the very top of the main \LaTeX{} file,
in particular \emph{before} the |\documentclass| statement!
The argument of |\childdocmain| should be left empty
(but it must be present).

%%%%%%%%%%%%%%%%%%%%%%%%%%%%%%%%%%%%%%%%
\DescribeMacro{\childdocof}
Furthermore, add the commands
\begin{center}
\begin{tabular}{l}
|\input{childdoc.def}|\\
|\childdocof{|\textit{main}|}|\\
\end{tabular}
\end{center}
at the top of every child file \textit{child}
which is included by |\include{|\textit{child}|}|
from within the main file
(or at least for those files to be compiled individually).
The argument \textit{main} must be the filename of the main file.

There are a couple of
considerations in setting up the main and child documents:

%%%%%%%%%%%%%%%%%%%%%%%%%%%%%%%%%%%%%%%%
\paragraph{Restrictions.}

Please note the following restrictions:
\begin{itemize}
\item
|\childdocmain| must be called with one argument \textit{main}
to ensure compatibility with earlier version of the package.
It must either be empty (|\childdocmain{}|)
or precisely match the filename of the main file in which it is specified.
See \secref{sec:detection} for further information.
\item
The filename \textit{main} must be specified without the |.tex| extension.
\item
The filename \textit{main} is case sensitive
(even in case-insensitive file systems)
due to internal string comparison.
\item
The argument \textit{main} should be fully expanded, it cannot be a macro.
\item
Subdirectories and special characters should be avoided in filenames.
\item
The command |\childdocmain{|\textit{main}|}| must be followed by a whitespace.
It should not be followed immediately by another command
or by a comment mark `|%|'.
This is because the \TeX{} parser reads the token immediately following
the argument of |\childdocmain| and puts it
at the beginning of every child section;
however, a white\-space is ignored.
\end{itemize}

%%%%%%%%%%%%%%%%%%%%%%%%%%%%%%%%%%%%%%%%
\paragraph{Content of Main File.}

It is advisable to place all content in the child files included by |\include|.
Any output contained in the main file will appear in all child documents
unless suppressed manually;
it cannot be suppressed automatically by the |\includeonly| directive
and thus should normally be avoided.
A method to include some content in the main file
by means of conditional processing is described in \secref{sec:conditional}.

%%%%%%%%%%%%%%%%%%%%%%%%%%%%%%%%%%%%%%%%
\paragraph{Page Numbering.}

When only a part of the document is compiled,
the appropriate numbering of pages
(as well as other status parameters)
is determined from the |.aux| files.
The latter contain information from previous passes.
However this information needs to propagate through
all intermediate child documents.
Therefore the page numbering in child documents may well
be inconsistent until the complete document is compiled at least once.

A useful (if unconventional) way to always ensure a consistent
page numbering is to restart the numbering in each child document
and denote the pages by `\textit{child}|.|\textit{page}'
where \textit{child} represents the chapter/section number of the child file.
This can be achieved by the command
|\numberwithin{page}{|\textit{child}|}|
of the \textsf{amsmath} package
where \textit{child} can be |chapter| or |section|
depending on the chosen structuring.
Alternatively, one can modify the macro |\thepage| appropriately
and reset the counter |page| at the start of each child file.

%%%%%%%%%%%%%%%%%%%%%%%%%%%%%%%%%%%%%%%%%%%%%%%%%%%%%%%%%%%%%%%%%%%%%%%%%%%%%%%%
\subsection{Conditional Processing}
\label{sec:conditional}

The package provides a mechanism to compile different versions
of a document. To customise the versions further some conditional processing
can come in handy to distinguish which version is being compiled.
The package provides two macros to describe the compilation context:

%%%%%%%%%%%%%%%%%%%%%%%%%%%%%%%%%%%%%%%%
\DescribeMacro{\ifchilddoc}
The conditional |\ifchilddoc| distinguishes between the compilation of
child documents and the main document:
%
\begin{center}
|\ifchilddoc |\textit{child-code}| |[|\||else |\textit{main-code}]| \||fi|
\end{center}

%%%%%%%%%%%%%%%%%%%%%%%%%%%%%%%%%%%%%%%%
\DescribeMacro{\childdocname}
\DescribeMacro{\childdocjob}
The macro |\childdocname| contains the filename (without extension)
of the main or child file being processed.
Note that |\childdocjob| will always contain the name of the main file.

%%%%%%%%%%%%%%%%%%%%%%%%%%%%%%%%%%%%%%%%
\paragraph{Title Page.}

Conditional processing can be used to include a title or banner page
in the main document when proper precautions are taken.
Importantly, the code in the main file should ensure that the page counter
(as well as other status parameters which are stored in the |.aux| files)
takes the same value after the conditional processing.
Otherwise the page numbers may take divergent values
depending on which part is compiled.

For example, a title page could be declared by:
%
\begin{center}
\begin{tabular}{l}
|\ifchilddoc\||else|\\
|\addtocounter{page}{-1}|\\
\textit{code for title page}\\
|\newpage|\\
|\||fi|
\end{tabular}
\end{center}
%
A banner page for the child documents can be generated by:
%
\begin{center}
\begin{tabular}{l}
|\ifchilddoc|\\
|\addtocounter{page}{-1}|\\
\textit{code for banner page}\\
|\newpage|\\
|\||fi|
\end{tabular}
\end{center}
%
Here one could write a message such as:
\begin{center}
|This is the part \childdocname{} of \childdocjob{}.|
\end{center}

%%%%%%%%%%%%%%%%%%%%%%%%%%%%%%%%%%%%%%%%%%%%%%%%%%%%%%%%%%%%%%%%%%%%%%%%%%%%%%%%
\subsection{Flags}
\label{sec:flags}

The package makes it easy to generate different versions
of the main or child documents.
To this end compilation flags can be defined
and assigned different default values.
They will be particularly useful in conjunction
with the forwarding mechanism described in \secref{sec:forward}.

For example, it may be useful to have a flag |\version|
which can be set to |draft| or |final|.
The document source will contain some conditional code
depending on the value of |\version|.
Suppose further, the flag should default to |final| for the main file
and to |draft| for child files
which is a natural assignment for editing the document.
This is achieved by placing the following code
in the preamble of the main document
(below the |\childdocmain| directive):
%
\begin{center}
\begin{tabular}{l}
|\ifchilddoc|\\
|\providecommand{\version}{draft}|\\
|\||else|\\
|\providecommand{\version}{final}|\\
|\||fi|
\end{tabular}
\end{center}
%
The definition by |\providecommand| makes sure
that previous definitions are not overwritten.
Further statements |\providecommand{\version}{...}|
can thus be added before the above code to override it.

For the main file, one might add a line
(between |\childdocmain| and the above block)
%
\begin{center}
|%\ifchilddoc\||else\providecommand{\version}{draft}\||fi|
\end{center}
%
which can be uncommented to produce a draft version.
Likewise one can add a line to the very top of a child file
(above the |\childdocof{|\textit{main}|}| directive)
%
\begin{center}
|%\providecommand{\version}{final}|
\end{center}
%
which can be uncommented to produce the final version of this child document.

%%%%%%%%%%%%%%%%%%%%%%%%%%%%%%%%%%%%%%%%%%%%%%%%%%%%%%%%%%%%%%%%%%%%%%%%%%%%%%%%
\subsection{Forwarding}
\label{sec:forward}

Different versions of the main or child documents
using compilation flags as described in \secref{sec:flags}
can be (permanently) stored in different files
for convenient compilation, viewing and distribution.
To this end, the package defines a command
to pass on compilation to a different file:

%%%%%%%%%%%%%%%%%%%%%%%%%%%%%%%%%%%%%%%%
\DescribeMacro{\childdocforward}
The command |\childdocforward| redirects processing to
another source file:
%
\begin{center}
\begin{tabular}{l}
|\input{childdoc.def}|\\
|\childdocforward[|\textit{main}|]{|\textit{dest}|}|\\
\end{tabular}
\end{center}
%
The argument \textit{dest} is the destination file
(without extension).
It should be the main file or one of the child files.
Note that further \textsf{childdoc} directives
such as |\childdocof| and |\childdocforward|
in the indicated file will be processed in this form.
The optional argument \textit{main}
passes on directly to the main file \textit{main}
while pretending to compile the child \textit{dest}.
This form behaves as if \textit{dest}
issues |\childdocof{|\textit{main}|}| right away,
and no further \textsf{childdoc} directives will be processed.

%%%%%%%%%%%%%%%%%%%%%%%%%%%%%%%%%%%%%%%%
\DescribeMacro{\...prefix}
In the alternative form |\childdocforwardprefix|,
%
\begin{center}
\begin{tabular}{l}
|\input{childdoc.def}|\\
|\childdocforwardprefix[|\textit{main}|]{|\textit{prefix}|}{|\textit{dest}|}|
\end{tabular}
\end{center}
%
the destination file is determined by a pattern
depending on the current file:
To make this work, the current file must be called
`{\textit{prefix}\hspace{0.2em}\textit{suffix}}'
with \textit{prefix} matching precisely the argument.
Processing is then passed on to the file
`{\textit{dest}\hspace{0.2em}\textit{suffix}}'.
Surely, the same effect is achieved by
directly specifying the
argument `{\textit{dest}\hspace{0.2em}\textit{suffix}}'
in the first form.
However, that requires to set up a different file
for each child. With the alternative form of the command
all these files can have exactly the same content
which simplifies setting them up and maintaining them.

For example, the following file |draft.tex|
with a compilation flag |\version| as described in \secref{sec:flags}
compiles the main document as a draft:
%
\begin{center}
\begin{tabular}{l}
|\def\version{draft}|\\
|\input{childdoc.def}|\\
|\childdocforward{|\textit{main}|}|
\end{tabular}
\end{center}
%
Likewise, the following files |final|\textit{nn}|.tex|
compile the final version of the child document
|child|\textit{nn}|.tex|:
%
\begin{center}
\begin{tabular}{l}
|\def\version{final}|\\
|\input{childdoc.def}|\\
|\childdocforwardprefix{final}{child}|
\end{tabular}
\end{center}
%

Note that when several versions of a main file and/or of each child file
are to be generated, it may be convenient to set up a |Makefile| or
shell script to automatise the process.

%%%%%%%%%%%%%%%%%%%%%%%%%%%%%%%%%%%%%%%%%%%%%%%%%%%%%%%%%%%%%%%%%%%%%%%%%%%%%%%%
\subsection{Command Line Processing}
\label{sec:commandline}

The effect of redirection files can also be achieved by invoking
the \LaTeX{} compiler with a more elaborate command line.
Most conveniently this should be done as part
of a shell script or a |Makefile|.

When using \textsf{childdoc} in the main file, the following
command lines effectively perform a redirection
(note that depending on the shell being used,
backslashes may have to be doubled: `|\|' $\to$ `|\\|'):
%
\begin{center}
|... -jobname "|\textit{target}|" |\\|"|[\textit{flags}]%
|\input{childdoc.def}\childdocforward[|\textit{main}|]{|\textit{dest}|}"|
\end{center}
%
Here \textit{target} is the name of the output file,
\textit{main} is the name of the main file
and \textit{dest} is the name of the main or child file to be processed
(all filenames without extensions).
The optional argument \textit{main} can be omitted
if \textit{main} matches \textit{dest}.
Optionally, compilation \textit{flags} can be defined via |\def| commands.
This command line makes the \TeX{} engine believe
it is compiling the file \textit{target}
whose content is specified as the latter parameter.
The provided code then forwards the processing to
\textit{main} or \textit{dest} as described in \secref{sec:forward}.

%%%%%%%%%%%%%%%%%%%%%%%%%%%%%%%%%%%%%%%%%%%%%%%%%%%%%%%%%%%%%%%%%%%%%%%%%%%%%%%%
\subsection{Include by Input}
\label{sec:input}

Including child documents by |\include| has some restrictions by design.
Most notably, the content of a child document always occupies
its own set of pages; pages cannot be shared between child documents.
Usually, this behaviour makes perfect sense
because each child document contain an essential part of the document.
However, in some situations it may be desirable to compose
a document from a collection of parts
without having mandatory page breaks between then.
For this case, the package
provides a mechanism to include parts
by |\input| which can also be processed individually.
However, by construction this mechanism
requires manual handling of the content to be output.

%%%%%%%%%%%%%%%%%%%%%%%%%%%%%%%%%%%%%%%%
\DescribeMacro{\ifchilddocmanual}
The main file should be prepared as usual, see \secref{sec:include}.
However, the document body must make a distinction
between processing of an individual part and of the main document, e.g.:
%
\begin{center}
\begin{tabular}{l}
|\ifchilddocmanual|\\
|\input{\childdocname}|\\
|\||else|\\
\textit{document body with }|\input{|\textit{part}|}|\\
|\||fi|
\end{tabular}
\end{center}
%
The conditional |\ifchilddocmanual| is true whenever
a part to be included by |\input| is being compiled,
and the name of the part is stored in |\childdocname|.

%%%%%%%%%%%%%%%%%%%%%%%%%%%%%%%%%%%%%%%%
\DescribeMacro{\childdocby}
Each part to be included by |\input| should start with:
%
\begin{center}
\begin{tabular}{l}
|\input{childdoc.def}|\\
|\childdocby{|\textit{main}|}|\\
\end{tabular}
\end{center}
%
The directive |\childdocby| is similar to |\childdocof|
described in \secref{sec:include},
but the subsequent selection of content must be done manually.
To that end, both |\ifchilddoc| and |\ifchilddocmanual|
will be true upon processing of a part,
and the name of the part is stored in |\childdocname|.
Note that |\jobname| will be set to the filename of the current part
so that each part receives an individual |.aux| file
that does not interfere with the |.aux| file(s) of the main document.
This behaviour can be altered by the alternative form
|\childdocby[*]{|\textit{main}|}| (with a non-empty optional argument)
which uses the |.aux| file of the main document
by setting |\jobname| to \textit{main}.

%%%%%%%%%%%%%%%%%%%%%%%%%%%%%%%%%%%%%%%%%%%%%%%%%%%%%%%%%%%%%%%%%%%%%%%%%%%%%%%%
\subsection{Driver Development}
\label{sec:driver}

The \textsf{childdoc} mechanism can also be use for the development
of definition files such as \LaTeX{} styles or classes.
This case differs from the above setup with multiple parts
included by |\include| in that no |\includeonly| should be invoked.
This can be achieved by starting the include file
(before |\ProvidesPackage|) with:
%
\begin{center}
\begin{tabular}{l}
|\input{childdoc.def}|\\
|\childdocforward{|\textit{main}|}|\\
\end{tabular}
\end{center}
%
or alternatively with:
%
\begin{center}
\begin{tabular}{l}
|\input{childdoc.def}|\\
|\childdocby{|\textit{main}|}|\\
\end{tabular}
\end{center}
%
Both forms have slightly different effects as described above.
The main file is prepared as usual, see \secref{sec:include}.

%%%%%%%%%%%%%%%%%%%%%%%%%%%%%%%%%%%%%%%%%%%%%%%%%%%%%%%%%%%%%%%%%%%%%%%%%%%%%%%%
\subsection{Legacy Detection}
\label{sec:detection}

The directive |\childdocmain| in the main file can detect
whether the complete document or merely a child is to be compiled
even without using the directive |\childdocof|.
This method is deprecated because it is less robust
and there is no compelling reason to use it;
it is merely provided for backward compatibility
and it may be removed in future versions.

If the detection mechanism is to be used,
it is mandatory to correctly specify
the filename of the main file as the argument of |\childdocmain|:
%
\begin{center}
\begin{tabular}{l}
|\input{childdoc.def}|\\
|\childdocmain{|\textit{main}|}|\\
\end{tabular}
\end{center}
%
If |\jobname| does not match the argument \textit{main} of |\childdocmain|,
it is assumed that |\jobname| points to the child file to be compiled.
When using |\childdocmain| with the main file specified as argument,
it suffices to start a child file
with just |\input{|\textit{main}|}|
without loading of the package and using |\childdocof|.
If instead all processing is done
with the appropriate \textsf{childdoc} directives,
the argument of \textit{main} of |\childdocmain| can be empty.

An alternative version of the command line processing described
in \secref{sec:commandline} using the detection mechanism reads:
%
\begin{center}
|... -jobname "|\textit{target}|" "|[\textit{flags}]%
[|\def\jobname{|\textit{dest}|}|]|\input{|\textit{main}|}"|
\end{center}

%%%%%%%%%%%%%%%%%%%%%%%%%%%%%%%%%%%%%%%%%%%%%%%%%%%%%%%%%%%%%%%%%%%%%%%%%%%%%%%%
\subsection{Manual Code}
\label{sec:manual}

In case one cannot be certain whether the definitions file |childdoc.def|
is installed on the target \TeX{} distribution
and one prefers not to ship it,
it is conceivable to paste a few relevant commands into the sources.

To that end, drop all statements |\input{childdoc.def}|
and perform the replacements as outlined below.
Instead of |\childdocmain{|\textit{main}|}| add the following code
to the top of the main file:
%
\begin{center}
\begin{tabular}{l}
|\||ifdefined\childdocname\endinput\||fi\newif\ifchilddoc|\\
|\edef\childdocname{\scantokens\expandafter{\jobname\noexpand}}|\\
|\def\childdocmain{|\textit{main}|}\||ifx\childdocmain\childdocname\||else|\\
|\childdoctrue\includeonly{\childdocname}\let\jobname\childdocmain\||fi|\\
\end{tabular}
\end{center}
%
Instead of |\childdocof{|\textit{main}|}| just include the main file
at the top of each child file:
%
\begin{center}
|\input{|\textit{main}|}|
\end{center}
%
A simple redirection |\childdocforward{|\textit{dest}|}| is achieved by:
%
\begin{center}
|\def\jobname{|\textit{dest}|}\input{\jobname}|
\end{center}
%
The redirection with prefix
|\childdocforwardprefix[|\textit{prefix}|]{|\textit{dest}|}|
is accomplished by:
%
\begin{center}
\begin{tabular}{l}
|{\edef\jobname{\scantokens\expandafter{\jobname\noexpand}}|\\
|\def\redirectjob |\textit{prefix}|#1~~~{\gdef\jobname{|\textit{dest}|#1}}|\\
|\expandafter\redirectjob\jobname~~~}\input{\jobname}|
\end{tabular}
\end{center}

In an alternative approach,
child documents can be compiled by a specific command line
without additional code or specific definitions:
%
\begin{center}
|... -jobname "|\textit{target}|" "|[\textit{flags}]%
|\includeonly{|\textit{dest}|}\input{|\textit{main}|}"|
\end{center}
%

%%%%%%%%%%%%%%%%%%%%%%%%%%%%%%%%%%%%%%%%%%%%%%%%%%%%%%%%%%%%%%%%%%%%%%%%%%%%%%%%
%%%%%%%%%%%%%%%%%%%%%%%%%%%%%%%%%%%%%%%%%%%%%%%%%%%%%%%%%%%%%%%%%%%%%%%%%%%%%%%%
\section{Information}

%%%%%%%%%%%%%%%%%%%%%%%%%%%%%%%%%%%%%%%%%%%%%%%%%%%%%%%%%%%%%%%%%%%%%%%%%%%%%%%%
\subsection{Copyright}

Copyright \copyright{} 2017--2018 Niklas Beisert

This work may be distributed and/or modified under the
conditions of the \LaTeX{} Project Public License, either version 1.3
of this license or (at your option) any later version.
The latest version of this license is in
  \url{http://www.latex-project.org/lppl.txt}
and version 1.3 or later is part of all distributions of \LaTeX{}
version 2005/12/01 or later.

This work has the LPPL maintenance status `maintained'.

The Current Maintainer of this work is Niklas Beisert.

This work consists of the files |README.txt|, |childdoc.ins| and |childdoc.dtx|
as well as the derived files |childdoc.def|, |cdocsamp.tex|
with |cdocsch1.tex|, |cdocsch2.tex|, |cdocspt3.tex|, |cdocspt4.tex|,
|cdocsdrf.tex|, |cdocsfn1.tex|, |cdocsfn2.tex|
as well as |childdoc.pdf|.

%%%%%%%%%%%%%%%%%%%%%%%%%%%%%%%%%%%%%%%%%%%%%%%%%%%%%%%%%%%%%%%%%%%%%%%%%%%%%%%%
\subsection{Files and Installation}

The package consists of the files:
%
\begin{center}
\begin{tabular}{ll}
    |README.txt|   & readme file \\
    |childdoc.ins| & installation file \\
    |childdoc.dtx| & source file \\
    |childdoc.def| & definition file \\
    |cdocsamp.tex| & sample main file \\
    |cdocsch1.tex| & sample include file \\
    |cdocsch2.tex| & sample include file \\
    |cdocspt3.tex| & sample part file \\
    |cdocspt4.tex| & sample part file \\
    |cdocsdrf.tex| & sample redirection file \\
    |cdocsfn1.tex| & sample redirection file \\
    |cdocsfn2.tex| & sample redirection file \\
    |childdoc.pdf| & manual
\end{tabular}
\end{center}
%
The distribution consists of the files
|README.txt|, |childdoc.ins| and |childdoc.dtx|.
%
\begin{itemize}
\item
Run (pdf)\LaTeX{} on |childdoc.dtx|
to compile the manual |childdoc.pdf| (this file).
\item
Run \LaTeX{} on |childdoc.ins| to create the definitions file |childdoc.def|
and the sample |cdocsamp.tex| with include files
|cdocsch1.tex|, |cdocsch2.tex|, |cdocspt3.tex|, |cdocspt4.tex|,
|cdocsdrf.tex|, |cdocsfn1.tex|, |cdocsfn2.tex|.
Then copy the file |childdoc.def| to an appropriate directory of your \LaTeX{}
distribution, e.g.\ \textit{texmf-root}|/tex/latex/childdoc|.
\end{itemize}

%%%%%%%%%%%%%%%%%%%%%%%%%%%%%%%%%%%%%%%%%%%%%%%%%%%%%%%%%%%%%%%%%%%%%%%%%%%%%%%%
\subsection{Related CTAN Packages}

There are several other packages which offer a similar functionality:
%
\begin{itemize}
\item
The packages
\href{http://ctan.org/pkg/docmute}{\textsf{docmute}},
\href{http://ctan.org/pkg/includex}{\textsf{includex}} and
\href{http://ctan.org/pkg/standalone}{\textsf{standalone}}
provide commands to include only the document body of
a child file thus allowing both files to be compiled individually.
\item
The packages \href{http://ctan.org/pkg/subdocs}{\textsf{subdocs}}
and \href{http://ctan.org/pkg/subfiles}{\textsf{subfiles}}
provide structures in which the main and child documents can be
encapsulated and allowing them to be compiled individually.
The inclusion mechanism is different from the conventional |\include|.
\item
The package \href{http://ctan.org/pkg/combine}{\textsf{combine}}
is an elaborate solution to combine several documents into one.
\end{itemize}
%
See also the CTAN topic \href{http://ctan.org/topic/subdocs}{\textsf{subdocs}}
for further related packages.
The present package differs from the above solutions in that
a document structure constructed with the conventional |\include| mechanism
just needs two extra commands at the top of every file
such that all constituent files can be compiled individually.

%%%%%%%%%%%%%%%%%%%%%%%%%%%%%%%%%%%%%%%%%%%%%%%%%%%%%%%%%%%%%%%%%%%%%%%%%%%%%%%%
%\subsection{Feature Suggestions}
%
%The following is a list of features which may be useful for future
%versions of this package:
%%
%\begin{itemize}
%\item
%\ldots
%\end{itemize}

%%%%%%%%%%%%%%%%%%%%%%%%%%%%%%%%%%%%%%%%%%%%%%%%%%%%%%%%%%%%%%%%%%%%%%%%%%%%%%%%
\subsection{Revision History}

%%%%%%%%%%%%%%%%%%%%%%%%%%%%%%%%%%%%%%%%
\paragraph{v2.0:} 2018/12/30

\begin{itemize}
\item
immediate forward processing
\item
added |\childdocby| mechanism
\item
manual restructured
\end{itemize}

%%%%%%%%%%%%%%%%%%%%%%%%%%%%%%%%%%%%%%%%
\paragraph{v1.6:} 2018/01/17

\begin{itemize}
\item
application for development of include files
\item
corrections to manual
\end{itemize}

%%%%%%%%%%%%%%%%%%%%%%%%%%%%%%%%%%%%%%%%
\paragraph{v1.5:} 2017/05/21

\begin{itemize}
\item
more complete structuring introduced
\item
|\childdocof| introduced
\item
|\childdoc| renamed to |\childdocmain|
\item
|\childredirect| renamed to |\childdocforward| and |\childdocforwardprefix|
and functionality expanded
\end{itemize}

%%%%%%%%%%%%%%%%%%%%%%%%%%%%%%%%%%%%%%%%
\paragraph{v1.0:} 2017/04/27

\begin{itemize}
\item
manual and install package
\item
first version published on CTAN
\end{itemize}

%%%%%%%%%%%%%%%%%%%%%%%%%%%%%%%%%%%%%%%%
\paragraph{v0.6:} 2017/04/26

\begin{itemize}
\item
redirection mechanism added
\end{itemize}

%%%%%%%%%%%%%%%%%%%%%%%%%%%%%%%%%%%%%%%%
\paragraph{v0.5:} 2017/04/26

\begin{itemize}
\item
functionality in definition file
\end{itemize}


%%%%%%%%%%%%%%%%%%%%%%%%%%%%%%%%%%%%%%%%%%%%%%%%%%%%%%%%%%%%%%%%%%%%%%%%%%%%%%%%
%%%%%%%%%%%%%%%%%%%%%%%%%%%%%%%%%%%%%%%%%%%%%%%%%%%%%%%%%%%%%%%%%%%%%%%%%%%%%%%%
%%%%%%%%%%%%%%%%%%%%%%%%%%%%%%%%%%%%%%%%%%%%%%%%%%%%%%%%%%%%%%%%%%%%%%%%%%%%%%%%
\appendix

\settowidth\MacroIndent{\rmfamily\scriptsize 000\ }

 \DocInput{childdoc.dtx}

\end{document}
%</driver>
% \fi
%
% %%%%%%%%%%%%%%%%%%%%%%%%%%%%%%%%%%%%%%%%%%%%%%%%%%%%%%%%%%%%%%%%%%%%%%%%%%%%%%
% %%%%%%%%%%%%%%%%%%%%%%%%%%%%%%%%%%%%%%%%%%%%%%%%%%%%%%%%%%%%%%%%%%%%%%%%%%%%%%
% \section{Sample}
%\iffalse
%<*samplemain>
%\fi
%
% The following presents a sample document
% with two chapters, two parts, a title page,
% a compile flag as well as three forwarding files to set the flag.
% It consists of eight |.tex| files:
% \begin{center}
% \begin{tabular}{ll}
% |cdocsamp.tex|&main file\\
% |cdocsch1.tex|&include file for chapter 1\\
% |cdocsch2.tex|&include file for chapter 2\\
% |cdocspt3.tex|&include file for part 3\\
% |cdocspt4.tex|&include file for part 4\\
% |cdocsdrf.tex|&forwarding file for main file in draft mode\\
% |cdocsfi1.tex|&forwarding file for final version of chapter 1\\
% |cdocsfi2.tex|&forwarding file for final version of chapter 2\\
% \end{tabular}
% \end{center}
% Each of the eight files can be compiled directly by the \LaTeX{} compiler.
%
% %%%%%%%%%%%%%%%%%%%%%%%%%%%%%%%%%%%%%%
% \paragraph{Main File.}
%
% The main file is called |cdocsamp.tex|.
%
% Load the \textsf{childdoc} definitions and
% declare the filename for the main document:
%    \begin{macrocode}
\input{childdoc.def}
\childdocmain{}
%    \end{macrocode}

% Optional override for |\version| flag:
%    \begin{macrocode}
%%\ifchilddoc\else\providecommand{\version}{draft}\fi
%    \end{macrocode}

% Define the default values for the |\version| flag
% (|final| for the main file and |draft| for childs):
%    \begin{macrocode}
\ifchilddoc
\providecommand{\version}{draft}
\else
\providecommand{\version}{final}
\fi
%    \end{macrocode}

% Load the standard document class:
%    \begin{macrocode}
\documentclass[12pt]{article}
%    \end{macrocode}

% Start the document body:
%    \begin{macrocode}
\begin{document}
%    \end{macrocode}

% Declare a title page.
% Print title, part of document being processed and version flag:
%    \begin{macrocode}
\addtocounter{page}{-1}
\begin{center}
{\LARGE\bfseries{}childdoc example\par}
\vspace{1cm}
\ifchilddoc
\ifchilddocmanual part\else chapter\fi:
`\childdocname' of `\childdocjob'\par
\else
main document: `\childdocjob'\par
\fi
version: \version\par
\end{center}
\newpage
%    \end{macrocode}

% Manually include selected file,
% otherwise process as usual:
%    \begin{macrocode}
\ifchilddocmanual
\section*{part `\childdocname'}
\input{\childdocname}
\else
%    \end{macrocode}

% Include the two chapters:
%    \begin{macrocode}
\include{cdocsch1}
\include{cdocsch2}
%    \end{macrocode}

% Include the two parts unless only chapters should be displayed:
%    \begin{macrocode}
\ifchilddoc\else
\section{part three}
\input{cdocspt3}
\section{part four}
\input{cdocspt4}
\fi
%    \end{macrocode}

% Process as usual until here:
%    \begin{macrocode}
\fi
%    \end{macrocode}

% End of document body:
%    \begin{macrocode}
\end{document}
%    \end{macrocode}
%\iffalse
%</samplemain>
%\fi
%
% %%%%%%%%%%%%%%%%%%%%%%%%%%%%%%%%%%%%%%
% \paragraph{Chapter Include Files.}
%
% The include files are called |cdocsch1.tex| and |cdocsch2.tex|.
%
%\iffalse
%<*samplechap1|samplechap2>
%\fi

% Optional override for |\version| flag:
%    \begin{macrocode}
%%\providecommand{\version}{final}
%    \end{macrocode}

% Include the main document:
%    \begin{macrocode}
\input{childdoc.def}
\childdocof{cdocsamp}
%    \end{macrocode}

%\iffalse
%</samplechap1|samplechap2>
%\fi
%
%\iffalse
%<*samplechap1>
%\fi
% Some text for chapter 1:
%    \begin{macrocode}
\section{one}
some text in chapter one
%    \end{macrocode}

%\iffalse
%</samplechap1>
%\fi
% Some text for chapter 2:
%\iffalse
%<*samplechap2>
%\fi
%    \begin{macrocode}
\section{two}
more text in chapter two
%    \end{macrocode}

%\iffalse
%</samplechap2>
%\fi
%
% %%%%%%%%%%%%%%%%%%%%%%%%%%%%%%%%%%%%%%
% \paragraph{Part Include Files.}
%
% The include files are called |cdocspt3.tex| and |cdocspt4.tex|.
%
%\iffalse
%<*samplepart3|samplepart4>
%\fi

% Optional override for |\version| flag:
%    \begin{macrocode}
%%\providecommand{\version}{final}
%    \end{macrocode}

% Include the main document:
%    \begin{macrocode}
\input{childdoc.def}
\childdocby{cdocsamp}
%    \end{macrocode}

%\iffalse
%</samplepart3|samplepart4>
%\fi
%
%\iffalse
%<*samplepart3>
%\fi
% Some text for part 3:
%    \begin{macrocode}
some text in part three
%    \end{macrocode}

%\iffalse
%</samplepart3>
%\fi
% Some text for part 4:
%\iffalse
%<*samplepart4>
%\fi
%    \begin{macrocode}
more text in part four
%    \end{macrocode}

%\iffalse
%</samplepart4>
%\fi
%
% %%%%%%%%%%%%%%%%%%%%%%%%%%%%%%%%%%%%%%
% \paragraph{Forwarding for a Complete Draft.}
%
% The following forwarding file |cdocsdrf.tex|
% compiles the main document in draft mode:
%\iffalse
%<*sampledraft>
%\fi
%    \begin{macrocode}
\def\version{draft}
\input{childdoc.def}
\childdocforward{cdocsamp}
%    \end{macrocode}

%\iffalse
%</sampledraft>
%\fi
%
% %%%%%%%%%%%%%%%%%%%%%%%%%%%%%%%%%%%%%%
% \paragraph{Forwarding for Final Version of the Chapters.}
%
% The following forwarding files |cdocsfn1.tex| and |cdocsfn2.tex|
% (with identical content)
% compile the final versions of the child documents
% |cdocsch1.tex| and |cdocsch2.tex|, respectively:
%\iffalse
%<*samplefinal>
%\fi
%    \begin{macrocode}
\def\version{final}
\input{childdoc.def}
\childdocforwardprefix[cdocsamp]{cdocsfn}{cdocsch}
%    \end{macrocode}

%\iffalse
%</samplefinal>
%\fi
%
% %%%%%%%%%%%%%%%%%%%%%%%%%%%%%%%%%%%%%%
% \paragraph{Command Line Processing.}
%
% The following three command lines generate the output files
% |cdocscld|, |cdocscl1| and |cdocscl2|
% which should be identical to
% |cdocsdrf|, |cdocsch1| and |cdocsfn2|, respectively:
% \begin{center}
% \begin{tabular}{l}
% |latex -jobname cdocscld \|\\
% |  "\def\version{draft}\input{childdoc.def}\childdocforward{cdocsamp}"|\\
% |latex -jobname cdocscl1 \|\\
% |  "\input{childdoc.def}\childdocforward[cdocsamp]{cdocsch1}"|\\
% |latex -jobname cdocscl2 \|\\
% |  "\def\version{final}\input{childdoc.def}\childdocforward{cdocsch2}"|
% \end{tabular}
% \end{center}
% Note that the trailing backslash on each first line
% merely continues the input to the second line
% (for convenient cut ant paste).
% Furthermore, the command |latex| can be replaced by any
% of its alternative versions such as |pdflatex|.
%
% %%%%%%%%%%%%%%%%%%%%%%%%%%%%%%%%%%%%%%%%%%%%%%%%%%%%%%%%%%%%%%%%%%%%%%%%%%%%%%
% %%%%%%%%%%%%%%%%%%%%%%%%%%%%%%%%%%%%%%%%%%%%%%%%%%%%%%%%%%%%%%%%%%%%%%%%%%%%%%
% \section{Implementation}
%\iffalse
%<*package>
%\fi
%
% This section describes the definitions file |childdoc.def|.

% The definitions cannot be loaded using |\usepackage| or |\RequirePackage|
% which has a mechanism to prevent loading a style file more than once.
% When loading the definitions by means of |\input|
% multiple instances have to be prevented manually:
%\iffalse
%This code needs to be before the `\ProvidesFile' directive
%which is defined at the beginning of this file.
%Therefore it is also placed there and commented out here.
%</package>
%<*discard>
%\fi
%    \begin{macrocode}
\ifdefined\childdocmain\endinput\fi
%    \end{macrocode}
%\iffalse
%</discard>
%<*package>
%\fi
%
% \macro{\ifchilddoc}
% \macro{\ifchilddocmanual}
% The conditional |\ifchilddoc| tells whether a
% child (true) or main (false) document is being compiled.
% The conditional |\ifchilddocmanual| tells whether
% the |\includeonly| mechanism is used (false) or
% the selection of child files must be performed manually (true).
% The definitions initialise to false:
%    \begin{macrocode}
\newif\ifchilddoc
\newif\ifchilddocmanual
%    \end{macrocode}

% \macro{\childdocname}
% \macro{\childdocjob}
% The macro |\childdocname| stores the name of the main document
% to be compiled. The macro |\childdocjob| stores the name of
% the document on which the \LaTeX{} compiler was originally invoked.
% The content of |\jobname| cannot be compared
% to filenames specified in the source due to different catcodes.
% The following code rescans |\jobname|, stores the result
% in |\childdocname| and saves a copy in |\childdocjob|:
%    \begin{macrocode}
\edef\childdocname{\scantokens\expandafter{\jobname\noexpand}}
\let\childdocjob\childdocname
%    \end{macrocode}

% \macro{\childdocdisable}
% The macro |\childdocdisable| prevents the main file
% from being processed more than once.
% At this stage, the main document command |\childdocmain|
% is assumed to be called once again where it should do nothing.
% Any subsequent call to it should prevent
% a secondary processing of the main document
% It overwrites the forwarding commands
% |\childdocof| and |\childdocforward|
% with empty macros to prevent further inclusions of the main document:
%    \begin{macrocode}
\newcommand{\childdocdisable}
{
  \renewcommand{\childdocmain}[1]{\renewcommand{\childdocmain}[1]{\endinput}}
  \renewcommand{\childdocof}[1]{}
  \renewcommand{\childdocby}[2][]{}
  \renewcommand{\childdocforward}[2][]{}
  \renewcommand{\childdocdisable}{}
}
%    \end{macrocode}

% \macro{\childdocmain}
% The macro |\childdocmain| is to be called at the top of the main file
% with nothing or the main filename (without extension) as argument.
% First, it breaks loops.
% If the argument is not empty and does not match |\childdocname|
% (which is set by the first inclusion of |childdoc.def|),
% |\ifchilddoc| is set to true, |\includeonly| is applied to the child file
% and |\jobname| is set to the main file
% (for proper handling of |.aux| files):
%    \begin{macrocode}
\newcommand{\childdocmain}[1]
{
  \childdocdisable\childdocmain{}
  \if?#1?\else
    \begingroup
      \def\childdoctmp{#1}
      \ifx\childdoctmp\childdocname
        \def\childdoctmp{}
      \else
        \def\childdoctmp
        {
          \childdoctrue
          \includeonly{\childdocname}
          \def\childdocjob{#1}
          \def\jobname{#1}
        }
      \fi
      \expandafter
    \endgroup
    \childdoctmp
  \fi
}
%    \end{macrocode}

% \macro{\childdocof}
% The command |\childdocof| redirects
% compilation to the main file |#1|.
%    \begin{macrocode}
\newcommand{\childdocof}[1]
{
  \childdocdisable
  \childdoctrue
  \includeonly{\childdocname}
  \def\jobname{#1}
  \def\childdocjob{#1}
  \input{#1}
}
%    \end{macrocode}

% \macro{\childdocby}
% The command |\childdocby| ....
%    \begin{macrocode}
\newcommand{\childdocby}[2][]
{
  \childdocdisable
  \childdoctrue
  \childdocmanualtrue
  \if?#1?\else
    \def\jobname{#2}
  \fi
  \def\childdocjob{#2}
  \input{#2}
  \endinput
}
%    \end{macrocode}

% \macro{\childdocforward}
% The command |\childdocforward| redirects
% compilation to the main file or
% (if the optional argument is given) a child file.
% Parameters are set as if the main file
% or a child file starting with |\childdocof| was compiled.
% Then compilation is handed over to the main file:
%    \begin{macrocode}
\newcommand{\childdocforward}[2][]
{
  \begingroup
    \if?#1?
      \def\childdoctmp
      {
        \def\childdocname{#2}
        \def\childdocjob{#2}
        \def\jobname{#2}
        \input{#2}
        \endinput
      }
    \else
      \def\childdoctmp
      {
        \childdocdisable
        \def\childdocname{#2}
        \childdoctrue
        \includeonly{#2}
        \def\childdocjob{#1}
        \def\jobname{#1}
        \input{#1}
        \endinput
      }
    \fi
    \expandafter
  \endgroup
  \childdoctmp
}
%    \end{macrocode}

% \macro{\childdocforwardprefix}
% The command |\childdocforwardprefix| redirects
% compilation to the main or a child file by means of a pattern.
% The prefix |#1| in the current filename is replaced by |#2|
% and the suffix of the current filename is kept
% (it is assumed that the filename does not contain the substring `|~~~|'
% which is used as a delimiter).
% Compilation is handed over to the new file by |\childdocforward|:
%    \begin{macrocode}
\newcommand{\childdocforwardprefix}[3][]
{
  \begingroup
    \def\childdocextract #2##1~~~{\def\childdoctmp{\childdocforward[#1]{#3##1}}}
    \expandafter\childdocextract\childdocname~~~
    \expandafter
  \endgroup
  \childdoctmp
}
%    \end{macrocode}

% \macro{\childdoc}
% The deprecated macro |\childdoc| is a legacy version of |\childdocmain|:
%    \begin{macrocode}
\newcommand{\childdoc}{\childdocmain}
%    \end{macrocode}

% \macro{\childdocredirect}
% The deprecated macro |\childdocredirect| is a legacy version
% of |\childdocforward| and |\childdocforwardprefix|:
%    \begin{macrocode}
\newcommand{\childdocredirect}[2][]
{
  \begingroup
    \if?#1?
      \def\childdoctmp{\childdocforward{#2}}
    \else
      \def\childdoctmp{\childdocforwardprefix{#1}{#2}}
    \fi
    \expandafter
  \endgroup
  \childdoctmp
}
%    \end{macrocode}

%\iffalse
%</package>
%\fi
%
\endinput
|\\
|\childdocmain{}|\\
\end{tabular}
\end{center}
at the very top of the main \LaTeX{} file,
in particular \emph{before} the |\documentclass| statement!
The argument of |\childdocmain| should be left empty
(but it must be present).

%%%%%%%%%%%%%%%%%%%%%%%%%%%%%%%%%%%%%%%%
\DescribeMacro{\childdocof}
Furthermore, add the commands
\begin{center}
\begin{tabular}{l}
|% \iffalse
%
% childdoc.dtx Copyright (C) 2017-2018 Niklas Beisert
%
% This work may be distributed and/or modified under the
% conditions of the LaTeX Project Public License, either version 1.3
% of this license or (at your option) any later version.
% The latest version of this license is in
%   http://www.latex-project.org/lppl.txt
% and version 1.3 or later is part of all distributions of LaTeX
% version 2005/12/01 or later.
%
% This work has the LPPL maintenance status `maintained'.
%
% The Current Maintainer of this work is Niklas Beisert.
%
% This work consists of the files childdoc.dtx and childdoc.ins
% and the derived files childdoc.def and cdocsamp.tex with
% cdocsch1.tex, cdocsch2.tex, cdocsdrf.tex, cdocsfn1.tex, cdocsfn2.tex.
%
%<package>\ifdefined\childdocmain\endinput\fi
%<package>\ProvidesFile{childdoc.def}[2018/12/30 v2.0 child document driver]
%<samplemain>\ProvidesFile{cdocsamp.tex}[2018/12/30 v2.0 sample for childdoc]
%<*driver>
%\ProvidesFile{childdoc.drv}[2018/12/30 v2.0 childdoc reference manual file]
\PassOptionsToClass{10pt,a4paper}{article}
\documentclass{ltxdoc}

\usepackage[margin=35mm]{geometry}
\usepackage{hyperref}
\usepackage{hyperxmp}
\usepackage[usenames]{color}

\hypersetup{colorlinks=true}
\hypersetup{pdfstartview=FitH}
\hypersetup{pdfpagemode=UseNone}
\hypersetup{pdfsource={}}
\hypersetup{pdflang={en-UK}}
\hypersetup{pdfcopyright={Copyright 2017-2018 Niklas Beisert.
  This work may be distributed and/or modified under the
  conditions of the LaTeX Project Public License, either version 1.3
  of this license or (at your option) any later version.}}
\hypersetup{pdflicenseurl={http://www.latex-project.org/lppl.txt}}
\hypersetup{pdfcontactaddress={ETH Zurich, ITP, HIT K,
  Wolfgang-Pauli-Strasse 27}}
\hypersetup{pdfcontactpostcode={8093}}
\hypersetup{pdfcontactcity={Zurich}}
\hypersetup{pdfcontactcountry={Switzerland}}
\hypersetup{pdfcontactemail={nbeisert@itp.phys.ethz.ch}}
\hypersetup{pdfcontacturl={http://people.phys.ethz.ch/\xmptilde nbeisert/}}

\newcommand{\secref}[1]{\hyperref[#1]{section \ref*{#1}}}

\parskip1ex
\parindent0pt
\let\olditemize\itemize
\def\itemize{\olditemize\parskip0pt}

\begin{document}

\title{The \textsf{childdoc} Package}
\hypersetup{pdftitle={The childdoc Package}}
\author{Niklas Beisert\\[2ex]
  Institut f\"ur Theoretische Physik\\
  Eidgen\"ossische Technische Hochschule Z\"urich\\
  Wolfgang-Pauli-Strasse 27, 8093 Z\"urich, Switzerland\\[1ex]
  \href{mailto:nbeisert@itp.phys.ethz.ch}
  {\texttt{nbeisert@itp.phys.ethz.ch}}}
\hypersetup{pdfauthor={Niklas Beisert}}
\hypersetup{pdfsubject={Manual for the LaTeX2e Package childdoc}}
\date{30 December 2018, \textsf{v2.0}}
\maketitle

\begin{abstract}\noindent
\textsf{childdoc} is a \LaTeXe{} package
that enables the direct compilation
of document sections included by |\include|
to individual files.
\end{abstract}

\begingroup
\parskip0ex
\tableofcontents
\endgroup

%%%%%%%%%%%%%%%%%%%%%%%%%%%%%%%%%%%%%%%%%%%%%%%%%%%%%%%%%%%%%%%%%%%%%%%%%%%%%%%%
%%%%%%%%%%%%%%%%%%%%%%%%%%%%%%%%%%%%%%%%%%%%%%%%%%%%%%%%%%%%%%%%%%%%%%%%%%%%%%%%
\section{Introduction}

\LaTeX{} provides a mechanism to structure a large document (such as a book)
into a main file and several child files (containing the chapters)
using the |\include| command.
This mechanism is beneficial for documents
which span hundreds of pages in order to
make the source file(s) more manageable.
Moreover, compilation can be restricted to
selected child files by means of the |\includeonly| command.
The latter feature can be used to reduce the compilation time while editing
(this was significantly more useful in the earlier days of \LaTeX{})
or to generate a smaller document which is easier to navigate.
Another application of |\includeonly| is to generate
documents consisting of selected parts of the complete document.

However, there are a few drawbacks of the plain |\include| mechanism:
\begin{itemize}
\item
The child files cannot be compiled on their own,
they can only be compiled via the main file.
A naive editing environment
(such as a text editor with an option
to have the current file processed by \LaTeX)
may require one to switch to the main file before compiling;
attempting to compile the child file produces errors.
\item
The main file must be modified (each time)
to adjust the |\includeonly| command
to the present needs. This easily leaves the main file in a messy state.
\item
The generated document will always carry the filename
of the main document. This is inconvenient if
several child files are to be compiled and
to be kept for distribution.
\end{itemize}

The present package provides a simple interface
to make child files individually compilable by \LaTeX{}.
Compiling a child file then has the same effect as compiling
the main file with an |\includeonly| command
to select the appropriate child.
Moreover the generated document will carry the name of the child
rather than the main file.
This resolves all three above issues.

This feature is meant to make the editing of books,
thesis documents and lecture notes somewhat more convenient.
However, the package can also be used efficiently for
composing a series of documents (such as exercise sheets)
which are typically distributed individually.
It then assists the author in generating the individual documents
(potentially in different versions)
as well as a document containing the collected series.
Another application is in developing style files
or other kinds of included material
where compilation of the style file could redirect
to a sample or test file.

%%%%%%%%%%%%%%%%%%%%%%%%%%%%%%%%%%%%%%%%%%%%%%%%%%%%%%%%%%%%%%%%%%%%%%%%%%%%%%%%
%%%%%%%%%%%%%%%%%%%%%%%%%%%%%%%%%%%%%%%%%%%%%%%%%%%%%%%%%%%%%%%%%%%%%%%%%%%%%%%%
\section{Usage}

First of all, the package \textsf{childdoc} is \emph{not} a standard
\LaTeXe{} |.sty| style file! Therefore it needs to be invoked in
a non-standard way.

%%%%%%%%%%%%%%%%%%%%%%%%%%%%%%%%%%%%%%%%%%%%%%%%%%%%%%%%%%%%%%%%%%%%%%%%%%%%%%%%
\subsection{Included Files}
\label{sec:include}

%%%%%%%%%%%%%%%%%%%%%%%%%%%%%%%%%%%%%%%%
\DescribeMacro{\childdocmain}
To use the package, add the commands
\begin{center}
\begin{tabular}{l}
|\input{childdoc.def}|\\
|\childdocmain{}|\\
\end{tabular}
\end{center}
at the very top of the main \LaTeX{} file,
in particular \emph{before} the |\documentclass| statement!
The argument of |\childdocmain| should be left empty
(but it must be present).

%%%%%%%%%%%%%%%%%%%%%%%%%%%%%%%%%%%%%%%%
\DescribeMacro{\childdocof}
Furthermore, add the commands
\begin{center}
\begin{tabular}{l}
|\input{childdoc.def}|\\
|\childdocof{|\textit{main}|}|\\
\end{tabular}
\end{center}
at the top of every child file \textit{child}
which is included by |\include{|\textit{child}|}|
from within the main file
(or at least for those files to be compiled individually).
The argument \textit{main} must be the filename of the main file.

There are a couple of
considerations in setting up the main and child documents:

%%%%%%%%%%%%%%%%%%%%%%%%%%%%%%%%%%%%%%%%
\paragraph{Restrictions.}

Please note the following restrictions:
\begin{itemize}
\item
|\childdocmain| must be called with one argument \textit{main}
to ensure compatibility with earlier version of the package.
It must either be empty (|\childdocmain{}|)
or precisely match the filename of the main file in which it is specified.
See \secref{sec:detection} for further information.
\item
The filename \textit{main} must be specified without the |.tex| extension.
\item
The filename \textit{main} is case sensitive
(even in case-insensitive file systems)
due to internal string comparison.
\item
The argument \textit{main} should be fully expanded, it cannot be a macro.
\item
Subdirectories and special characters should be avoided in filenames.
\item
The command |\childdocmain{|\textit{main}|}| must be followed by a whitespace.
It should not be followed immediately by another command
or by a comment mark `|%|'.
This is because the \TeX{} parser reads the token immediately following
the argument of |\childdocmain| and puts it
at the beginning of every child section;
however, a white\-space is ignored.
\end{itemize}

%%%%%%%%%%%%%%%%%%%%%%%%%%%%%%%%%%%%%%%%
\paragraph{Content of Main File.}

It is advisable to place all content in the child files included by |\include|.
Any output contained in the main file will appear in all child documents
unless suppressed manually;
it cannot be suppressed automatically by the |\includeonly| directive
and thus should normally be avoided.
A method to include some content in the main file
by means of conditional processing is described in \secref{sec:conditional}.

%%%%%%%%%%%%%%%%%%%%%%%%%%%%%%%%%%%%%%%%
\paragraph{Page Numbering.}

When only a part of the document is compiled,
the appropriate numbering of pages
(as well as other status parameters)
is determined from the |.aux| files.
The latter contain information from previous passes.
However this information needs to propagate through
all intermediate child documents.
Therefore the page numbering in child documents may well
be inconsistent until the complete document is compiled at least once.

A useful (if unconventional) way to always ensure a consistent
page numbering is to restart the numbering in each child document
and denote the pages by `\textit{child}|.|\textit{page}'
where \textit{child} represents the chapter/section number of the child file.
This can be achieved by the command
|\numberwithin{page}{|\textit{child}|}|
of the \textsf{amsmath} package
where \textit{child} can be |chapter| or |section|
depending on the chosen structuring.
Alternatively, one can modify the macro |\thepage| appropriately
and reset the counter |page| at the start of each child file.

%%%%%%%%%%%%%%%%%%%%%%%%%%%%%%%%%%%%%%%%%%%%%%%%%%%%%%%%%%%%%%%%%%%%%%%%%%%%%%%%
\subsection{Conditional Processing}
\label{sec:conditional}

The package provides a mechanism to compile different versions
of a document. To customise the versions further some conditional processing
can come in handy to distinguish which version is being compiled.
The package provides two macros to describe the compilation context:

%%%%%%%%%%%%%%%%%%%%%%%%%%%%%%%%%%%%%%%%
\DescribeMacro{\ifchilddoc}
The conditional |\ifchilddoc| distinguishes between the compilation of
child documents and the main document:
%
\begin{center}
|\ifchilddoc |\textit{child-code}| |[|\||else |\textit{main-code}]| \||fi|
\end{center}

%%%%%%%%%%%%%%%%%%%%%%%%%%%%%%%%%%%%%%%%
\DescribeMacro{\childdocname}
\DescribeMacro{\childdocjob}
The macro |\childdocname| contains the filename (without extension)
of the main or child file being processed.
Note that |\childdocjob| will always contain the name of the main file.

%%%%%%%%%%%%%%%%%%%%%%%%%%%%%%%%%%%%%%%%
\paragraph{Title Page.}

Conditional processing can be used to include a title or banner page
in the main document when proper precautions are taken.
Importantly, the code in the main file should ensure that the page counter
(as well as other status parameters which are stored in the |.aux| files)
takes the same value after the conditional processing.
Otherwise the page numbers may take divergent values
depending on which part is compiled.

For example, a title page could be declared by:
%
\begin{center}
\begin{tabular}{l}
|\ifchilddoc\||else|\\
|\addtocounter{page}{-1}|\\
\textit{code for title page}\\
|\newpage|\\
|\||fi|
\end{tabular}
\end{center}
%
A banner page for the child documents can be generated by:
%
\begin{center}
\begin{tabular}{l}
|\ifchilddoc|\\
|\addtocounter{page}{-1}|\\
\textit{code for banner page}\\
|\newpage|\\
|\||fi|
\end{tabular}
\end{center}
%
Here one could write a message such as:
\begin{center}
|This is the part \childdocname{} of \childdocjob{}.|
\end{center}

%%%%%%%%%%%%%%%%%%%%%%%%%%%%%%%%%%%%%%%%%%%%%%%%%%%%%%%%%%%%%%%%%%%%%%%%%%%%%%%%
\subsection{Flags}
\label{sec:flags}

The package makes it easy to generate different versions
of the main or child documents.
To this end compilation flags can be defined
and assigned different default values.
They will be particularly useful in conjunction
with the forwarding mechanism described in \secref{sec:forward}.

For example, it may be useful to have a flag |\version|
which can be set to |draft| or |final|.
The document source will contain some conditional code
depending on the value of |\version|.
Suppose further, the flag should default to |final| for the main file
and to |draft| for child files
which is a natural assignment for editing the document.
This is achieved by placing the following code
in the preamble of the main document
(below the |\childdocmain| directive):
%
\begin{center}
\begin{tabular}{l}
|\ifchilddoc|\\
|\providecommand{\version}{draft}|\\
|\||else|\\
|\providecommand{\version}{final}|\\
|\||fi|
\end{tabular}
\end{center}
%
The definition by |\providecommand| makes sure
that previous definitions are not overwritten.
Further statements |\providecommand{\version}{...}|
can thus be added before the above code to override it.

For the main file, one might add a line
(between |\childdocmain| and the above block)
%
\begin{center}
|%\ifchilddoc\||else\providecommand{\version}{draft}\||fi|
\end{center}
%
which can be uncommented to produce a draft version.
Likewise one can add a line to the very top of a child file
(above the |\childdocof{|\textit{main}|}| directive)
%
\begin{center}
|%\providecommand{\version}{final}|
\end{center}
%
which can be uncommented to produce the final version of this child document.

%%%%%%%%%%%%%%%%%%%%%%%%%%%%%%%%%%%%%%%%%%%%%%%%%%%%%%%%%%%%%%%%%%%%%%%%%%%%%%%%
\subsection{Forwarding}
\label{sec:forward}

Different versions of the main or child documents
using compilation flags as described in \secref{sec:flags}
can be (permanently) stored in different files
for convenient compilation, viewing and distribution.
To this end, the package defines a command
to pass on compilation to a different file:

%%%%%%%%%%%%%%%%%%%%%%%%%%%%%%%%%%%%%%%%
\DescribeMacro{\childdocforward}
The command |\childdocforward| redirects processing to
another source file:
%
\begin{center}
\begin{tabular}{l}
|\input{childdoc.def}|\\
|\childdocforward[|\textit{main}|]{|\textit{dest}|}|\\
\end{tabular}
\end{center}
%
The argument \textit{dest} is the destination file
(without extension).
It should be the main file or one of the child files.
Note that further \textsf{childdoc} directives
such as |\childdocof| and |\childdocforward|
in the indicated file will be processed in this form.
The optional argument \textit{main}
passes on directly to the main file \textit{main}
while pretending to compile the child \textit{dest}.
This form behaves as if \textit{dest}
issues |\childdocof{|\textit{main}|}| right away,
and no further \textsf{childdoc} directives will be processed.

%%%%%%%%%%%%%%%%%%%%%%%%%%%%%%%%%%%%%%%%
\DescribeMacro{\...prefix}
In the alternative form |\childdocforwardprefix|,
%
\begin{center}
\begin{tabular}{l}
|\input{childdoc.def}|\\
|\childdocforwardprefix[|\textit{main}|]{|\textit{prefix}|}{|\textit{dest}|}|
\end{tabular}
\end{center}
%
the destination file is determined by a pattern
depending on the current file:
To make this work, the current file must be called
`{\textit{prefix}\hspace{0.2em}\textit{suffix}}'
with \textit{prefix} matching precisely the argument.
Processing is then passed on to the file
`{\textit{dest}\hspace{0.2em}\textit{suffix}}'.
Surely, the same effect is achieved by
directly specifying the
argument `{\textit{dest}\hspace{0.2em}\textit{suffix}}'
in the first form.
However, that requires to set up a different file
for each child. With the alternative form of the command
all these files can have exactly the same content
which simplifies setting them up and maintaining them.

For example, the following file |draft.tex|
with a compilation flag |\version| as described in \secref{sec:flags}
compiles the main document as a draft:
%
\begin{center}
\begin{tabular}{l}
|\def\version{draft}|\\
|\input{childdoc.def}|\\
|\childdocforward{|\textit{main}|}|
\end{tabular}
\end{center}
%
Likewise, the following files |final|\textit{nn}|.tex|
compile the final version of the child document
|child|\textit{nn}|.tex|:
%
\begin{center}
\begin{tabular}{l}
|\def\version{final}|\\
|\input{childdoc.def}|\\
|\childdocforwardprefix{final}{child}|
\end{tabular}
\end{center}
%

Note that when several versions of a main file and/or of each child file
are to be generated, it may be convenient to set up a |Makefile| or
shell script to automatise the process.

%%%%%%%%%%%%%%%%%%%%%%%%%%%%%%%%%%%%%%%%%%%%%%%%%%%%%%%%%%%%%%%%%%%%%%%%%%%%%%%%
\subsection{Command Line Processing}
\label{sec:commandline}

The effect of redirection files can also be achieved by invoking
the \LaTeX{} compiler with a more elaborate command line.
Most conveniently this should be done as part
of a shell script or a |Makefile|.

When using \textsf{childdoc} in the main file, the following
command lines effectively perform a redirection
(note that depending on the shell being used,
backslashes may have to be doubled: `|\|' $\to$ `|\\|'):
%
\begin{center}
|... -jobname "|\textit{target}|" |\\|"|[\textit{flags}]%
|\input{childdoc.def}\childdocforward[|\textit{main}|]{|\textit{dest}|}"|
\end{center}
%
Here \textit{target} is the name of the output file,
\textit{main} is the name of the main file
and \textit{dest} is the name of the main or child file to be processed
(all filenames without extensions).
The optional argument \textit{main} can be omitted
if \textit{main} matches \textit{dest}.
Optionally, compilation \textit{flags} can be defined via |\def| commands.
This command line makes the \TeX{} engine believe
it is compiling the file \textit{target}
whose content is specified as the latter parameter.
The provided code then forwards the processing to
\textit{main} or \textit{dest} as described in \secref{sec:forward}.

%%%%%%%%%%%%%%%%%%%%%%%%%%%%%%%%%%%%%%%%%%%%%%%%%%%%%%%%%%%%%%%%%%%%%%%%%%%%%%%%
\subsection{Include by Input}
\label{sec:input}

Including child documents by |\include| has some restrictions by design.
Most notably, the content of a child document always occupies
its own set of pages; pages cannot be shared between child documents.
Usually, this behaviour makes perfect sense
because each child document contain an essential part of the document.
However, in some situations it may be desirable to compose
a document from a collection of parts
without having mandatory page breaks between then.
For this case, the package
provides a mechanism to include parts
by |\input| which can also be processed individually.
However, by construction this mechanism
requires manual handling of the content to be output.

%%%%%%%%%%%%%%%%%%%%%%%%%%%%%%%%%%%%%%%%
\DescribeMacro{\ifchilddocmanual}
The main file should be prepared as usual, see \secref{sec:include}.
However, the document body must make a distinction
between processing of an individual part and of the main document, e.g.:
%
\begin{center}
\begin{tabular}{l}
|\ifchilddocmanual|\\
|\input{\childdocname}|\\
|\||else|\\
\textit{document body with }|\input{|\textit{part}|}|\\
|\||fi|
\end{tabular}
\end{center}
%
The conditional |\ifchilddocmanual| is true whenever
a part to be included by |\input| is being compiled,
and the name of the part is stored in |\childdocname|.

%%%%%%%%%%%%%%%%%%%%%%%%%%%%%%%%%%%%%%%%
\DescribeMacro{\childdocby}
Each part to be included by |\input| should start with:
%
\begin{center}
\begin{tabular}{l}
|\input{childdoc.def}|\\
|\childdocby{|\textit{main}|}|\\
\end{tabular}
\end{center}
%
The directive |\childdocby| is similar to |\childdocof|
described in \secref{sec:include},
but the subsequent selection of content must be done manually.
To that end, both |\ifchilddoc| and |\ifchilddocmanual|
will be true upon processing of a part,
and the name of the part is stored in |\childdocname|.
Note that |\jobname| will be set to the filename of the current part
so that each part receives an individual |.aux| file
that does not interfere with the |.aux| file(s) of the main document.
This behaviour can be altered by the alternative form
|\childdocby[*]{|\textit{main}|}| (with a non-empty optional argument)
which uses the |.aux| file of the main document
by setting |\jobname| to \textit{main}.

%%%%%%%%%%%%%%%%%%%%%%%%%%%%%%%%%%%%%%%%%%%%%%%%%%%%%%%%%%%%%%%%%%%%%%%%%%%%%%%%
\subsection{Driver Development}
\label{sec:driver}

The \textsf{childdoc} mechanism can also be use for the development
of definition files such as \LaTeX{} styles or classes.
This case differs from the above setup with multiple parts
included by |\include| in that no |\includeonly| should be invoked.
This can be achieved by starting the include file
(before |\ProvidesPackage|) with:
%
\begin{center}
\begin{tabular}{l}
|\input{childdoc.def}|\\
|\childdocforward{|\textit{main}|}|\\
\end{tabular}
\end{center}
%
or alternatively with:
%
\begin{center}
\begin{tabular}{l}
|\input{childdoc.def}|\\
|\childdocby{|\textit{main}|}|\\
\end{tabular}
\end{center}
%
Both forms have slightly different effects as described above.
The main file is prepared as usual, see \secref{sec:include}.

%%%%%%%%%%%%%%%%%%%%%%%%%%%%%%%%%%%%%%%%%%%%%%%%%%%%%%%%%%%%%%%%%%%%%%%%%%%%%%%%
\subsection{Legacy Detection}
\label{sec:detection}

The directive |\childdocmain| in the main file can detect
whether the complete document or merely a child is to be compiled
even without using the directive |\childdocof|.
This method is deprecated because it is less robust
and there is no compelling reason to use it;
it is merely provided for backward compatibility
and it may be removed in future versions.

If the detection mechanism is to be used,
it is mandatory to correctly specify
the filename of the main file as the argument of |\childdocmain|:
%
\begin{center}
\begin{tabular}{l}
|\input{childdoc.def}|\\
|\childdocmain{|\textit{main}|}|\\
\end{tabular}
\end{center}
%
If |\jobname| does not match the argument \textit{main} of |\childdocmain|,
it is assumed that |\jobname| points to the child file to be compiled.
When using |\childdocmain| with the main file specified as argument,
it suffices to start a child file
with just |\input{|\textit{main}|}|
without loading of the package and using |\childdocof|.
If instead all processing is done
with the appropriate \textsf{childdoc} directives,
the argument of \textit{main} of |\childdocmain| can be empty.

An alternative version of the command line processing described
in \secref{sec:commandline} using the detection mechanism reads:
%
\begin{center}
|... -jobname "|\textit{target}|" "|[\textit{flags}]%
[|\def\jobname{|\textit{dest}|}|]|\input{|\textit{main}|}"|
\end{center}

%%%%%%%%%%%%%%%%%%%%%%%%%%%%%%%%%%%%%%%%%%%%%%%%%%%%%%%%%%%%%%%%%%%%%%%%%%%%%%%%
\subsection{Manual Code}
\label{sec:manual}

In case one cannot be certain whether the definitions file |childdoc.def|
is installed on the target \TeX{} distribution
and one prefers not to ship it,
it is conceivable to paste a few relevant commands into the sources.

To that end, drop all statements |\input{childdoc.def}|
and perform the replacements as outlined below.
Instead of |\childdocmain{|\textit{main}|}| add the following code
to the top of the main file:
%
\begin{center}
\begin{tabular}{l}
|\||ifdefined\childdocname\endinput\||fi\newif\ifchilddoc|\\
|\edef\childdocname{\scantokens\expandafter{\jobname\noexpand}}|\\
|\def\childdocmain{|\textit{main}|}\||ifx\childdocmain\childdocname\||else|\\
|\childdoctrue\includeonly{\childdocname}\let\jobname\childdocmain\||fi|\\
\end{tabular}
\end{center}
%
Instead of |\childdocof{|\textit{main}|}| just include the main file
at the top of each child file:
%
\begin{center}
|\input{|\textit{main}|}|
\end{center}
%
A simple redirection |\childdocforward{|\textit{dest}|}| is achieved by:
%
\begin{center}
|\def\jobname{|\textit{dest}|}\input{\jobname}|
\end{center}
%
The redirection with prefix
|\childdocforwardprefix[|\textit{prefix}|]{|\textit{dest}|}|
is accomplished by:
%
\begin{center}
\begin{tabular}{l}
|{\edef\jobname{\scantokens\expandafter{\jobname\noexpand}}|\\
|\def\redirectjob |\textit{prefix}|#1~~~{\gdef\jobname{|\textit{dest}|#1}}|\\
|\expandafter\redirectjob\jobname~~~}\input{\jobname}|
\end{tabular}
\end{center}

In an alternative approach,
child documents can be compiled by a specific command line
without additional code or specific definitions:
%
\begin{center}
|... -jobname "|\textit{target}|" "|[\textit{flags}]%
|\includeonly{|\textit{dest}|}\input{|\textit{main}|}"|
\end{center}
%

%%%%%%%%%%%%%%%%%%%%%%%%%%%%%%%%%%%%%%%%%%%%%%%%%%%%%%%%%%%%%%%%%%%%%%%%%%%%%%%%
%%%%%%%%%%%%%%%%%%%%%%%%%%%%%%%%%%%%%%%%%%%%%%%%%%%%%%%%%%%%%%%%%%%%%%%%%%%%%%%%
\section{Information}

%%%%%%%%%%%%%%%%%%%%%%%%%%%%%%%%%%%%%%%%%%%%%%%%%%%%%%%%%%%%%%%%%%%%%%%%%%%%%%%%
\subsection{Copyright}

Copyright \copyright{} 2017--2018 Niklas Beisert

This work may be distributed and/or modified under the
conditions of the \LaTeX{} Project Public License, either version 1.3
of this license or (at your option) any later version.
The latest version of this license is in
  \url{http://www.latex-project.org/lppl.txt}
and version 1.3 or later is part of all distributions of \LaTeX{}
version 2005/12/01 or later.

This work has the LPPL maintenance status `maintained'.

The Current Maintainer of this work is Niklas Beisert.

This work consists of the files |README.txt|, |childdoc.ins| and |childdoc.dtx|
as well as the derived files |childdoc.def|, |cdocsamp.tex|
with |cdocsch1.tex|, |cdocsch2.tex|, |cdocspt3.tex|, |cdocspt4.tex|,
|cdocsdrf.tex|, |cdocsfn1.tex|, |cdocsfn2.tex|
as well as |childdoc.pdf|.

%%%%%%%%%%%%%%%%%%%%%%%%%%%%%%%%%%%%%%%%%%%%%%%%%%%%%%%%%%%%%%%%%%%%%%%%%%%%%%%%
\subsection{Files and Installation}

The package consists of the files:
%
\begin{center}
\begin{tabular}{ll}
    |README.txt|   & readme file \\
    |childdoc.ins| & installation file \\
    |childdoc.dtx| & source file \\
    |childdoc.def| & definition file \\
    |cdocsamp.tex| & sample main file \\
    |cdocsch1.tex| & sample include file \\
    |cdocsch2.tex| & sample include file \\
    |cdocspt3.tex| & sample part file \\
    |cdocspt4.tex| & sample part file \\
    |cdocsdrf.tex| & sample redirection file \\
    |cdocsfn1.tex| & sample redirection file \\
    |cdocsfn2.tex| & sample redirection file \\
    |childdoc.pdf| & manual
\end{tabular}
\end{center}
%
The distribution consists of the files
|README.txt|, |childdoc.ins| and |childdoc.dtx|.
%
\begin{itemize}
\item
Run (pdf)\LaTeX{} on |childdoc.dtx|
to compile the manual |childdoc.pdf| (this file).
\item
Run \LaTeX{} on |childdoc.ins| to create the definitions file |childdoc.def|
and the sample |cdocsamp.tex| with include files
|cdocsch1.tex|, |cdocsch2.tex|, |cdocspt3.tex|, |cdocspt4.tex|,
|cdocsdrf.tex|, |cdocsfn1.tex|, |cdocsfn2.tex|.
Then copy the file |childdoc.def| to an appropriate directory of your \LaTeX{}
distribution, e.g.\ \textit{texmf-root}|/tex/latex/childdoc|.
\end{itemize}

%%%%%%%%%%%%%%%%%%%%%%%%%%%%%%%%%%%%%%%%%%%%%%%%%%%%%%%%%%%%%%%%%%%%%%%%%%%%%%%%
\subsection{Related CTAN Packages}

There are several other packages which offer a similar functionality:
%
\begin{itemize}
\item
The packages
\href{http://ctan.org/pkg/docmute}{\textsf{docmute}},
\href{http://ctan.org/pkg/includex}{\textsf{includex}} and
\href{http://ctan.org/pkg/standalone}{\textsf{standalone}}
provide commands to include only the document body of
a child file thus allowing both files to be compiled individually.
\item
The packages \href{http://ctan.org/pkg/subdocs}{\textsf{subdocs}}
and \href{http://ctan.org/pkg/subfiles}{\textsf{subfiles}}
provide structures in which the main and child documents can be
encapsulated and allowing them to be compiled individually.
The inclusion mechanism is different from the conventional |\include|.
\item
The package \href{http://ctan.org/pkg/combine}{\textsf{combine}}
is an elaborate solution to combine several documents into one.
\end{itemize}
%
See also the CTAN topic \href{http://ctan.org/topic/subdocs}{\textsf{subdocs}}
for further related packages.
The present package differs from the above solutions in that
a document structure constructed with the conventional |\include| mechanism
just needs two extra commands at the top of every file
such that all constituent files can be compiled individually.

%%%%%%%%%%%%%%%%%%%%%%%%%%%%%%%%%%%%%%%%%%%%%%%%%%%%%%%%%%%%%%%%%%%%%%%%%%%%%%%%
%\subsection{Feature Suggestions}
%
%The following is a list of features which may be useful for future
%versions of this package:
%%
%\begin{itemize}
%\item
%\ldots
%\end{itemize}

%%%%%%%%%%%%%%%%%%%%%%%%%%%%%%%%%%%%%%%%%%%%%%%%%%%%%%%%%%%%%%%%%%%%%%%%%%%%%%%%
\subsection{Revision History}

%%%%%%%%%%%%%%%%%%%%%%%%%%%%%%%%%%%%%%%%
\paragraph{v2.0:} 2018/12/30

\begin{itemize}
\item
immediate forward processing
\item
added |\childdocby| mechanism
\item
manual restructured
\end{itemize}

%%%%%%%%%%%%%%%%%%%%%%%%%%%%%%%%%%%%%%%%
\paragraph{v1.6:} 2018/01/17

\begin{itemize}
\item
application for development of include files
\item
corrections to manual
\end{itemize}

%%%%%%%%%%%%%%%%%%%%%%%%%%%%%%%%%%%%%%%%
\paragraph{v1.5:} 2017/05/21

\begin{itemize}
\item
more complete structuring introduced
\item
|\childdocof| introduced
\item
|\childdoc| renamed to |\childdocmain|
\item
|\childredirect| renamed to |\childdocforward| and |\childdocforwardprefix|
and functionality expanded
\end{itemize}

%%%%%%%%%%%%%%%%%%%%%%%%%%%%%%%%%%%%%%%%
\paragraph{v1.0:} 2017/04/27

\begin{itemize}
\item
manual and install package
\item
first version published on CTAN
\end{itemize}

%%%%%%%%%%%%%%%%%%%%%%%%%%%%%%%%%%%%%%%%
\paragraph{v0.6:} 2017/04/26

\begin{itemize}
\item
redirection mechanism added
\end{itemize}

%%%%%%%%%%%%%%%%%%%%%%%%%%%%%%%%%%%%%%%%
\paragraph{v0.5:} 2017/04/26

\begin{itemize}
\item
functionality in definition file
\end{itemize}


%%%%%%%%%%%%%%%%%%%%%%%%%%%%%%%%%%%%%%%%%%%%%%%%%%%%%%%%%%%%%%%%%%%%%%%%%%%%%%%%
%%%%%%%%%%%%%%%%%%%%%%%%%%%%%%%%%%%%%%%%%%%%%%%%%%%%%%%%%%%%%%%%%%%%%%%%%%%%%%%%
%%%%%%%%%%%%%%%%%%%%%%%%%%%%%%%%%%%%%%%%%%%%%%%%%%%%%%%%%%%%%%%%%%%%%%%%%%%%%%%%
\appendix

\settowidth\MacroIndent{\rmfamily\scriptsize 000\ }

 \DocInput{childdoc.dtx}

\end{document}
%</driver>
% \fi
%
% %%%%%%%%%%%%%%%%%%%%%%%%%%%%%%%%%%%%%%%%%%%%%%%%%%%%%%%%%%%%%%%%%%%%%%%%%%%%%%
% %%%%%%%%%%%%%%%%%%%%%%%%%%%%%%%%%%%%%%%%%%%%%%%%%%%%%%%%%%%%%%%%%%%%%%%%%%%%%%
% \section{Sample}
%\iffalse
%<*samplemain>
%\fi
%
% The following presents a sample document
% with two chapters, two parts, a title page,
% a compile flag as well as three forwarding files to set the flag.
% It consists of eight |.tex| files:
% \begin{center}
% \begin{tabular}{ll}
% |cdocsamp.tex|&main file\\
% |cdocsch1.tex|&include file for chapter 1\\
% |cdocsch2.tex|&include file for chapter 2\\
% |cdocspt3.tex|&include file for part 3\\
% |cdocspt4.tex|&include file for part 4\\
% |cdocsdrf.tex|&forwarding file for main file in draft mode\\
% |cdocsfi1.tex|&forwarding file for final version of chapter 1\\
% |cdocsfi2.tex|&forwarding file for final version of chapter 2\\
% \end{tabular}
% \end{center}
% Each of the eight files can be compiled directly by the \LaTeX{} compiler.
%
% %%%%%%%%%%%%%%%%%%%%%%%%%%%%%%%%%%%%%%
% \paragraph{Main File.}
%
% The main file is called |cdocsamp.tex|.
%
% Load the \textsf{childdoc} definitions and
% declare the filename for the main document:
%    \begin{macrocode}
\input{childdoc.def}
\childdocmain{}
%    \end{macrocode}

% Optional override for |\version| flag:
%    \begin{macrocode}
%%\ifchilddoc\else\providecommand{\version}{draft}\fi
%    \end{macrocode}

% Define the default values for the |\version| flag
% (|final| for the main file and |draft| for childs):
%    \begin{macrocode}
\ifchilddoc
\providecommand{\version}{draft}
\else
\providecommand{\version}{final}
\fi
%    \end{macrocode}

% Load the standard document class:
%    \begin{macrocode}
\documentclass[12pt]{article}
%    \end{macrocode}

% Start the document body:
%    \begin{macrocode}
\begin{document}
%    \end{macrocode}

% Declare a title page.
% Print title, part of document being processed and version flag:
%    \begin{macrocode}
\addtocounter{page}{-1}
\begin{center}
{\LARGE\bfseries{}childdoc example\par}
\vspace{1cm}
\ifchilddoc
\ifchilddocmanual part\else chapter\fi:
`\childdocname' of `\childdocjob'\par
\else
main document: `\childdocjob'\par
\fi
version: \version\par
\end{center}
\newpage
%    \end{macrocode}

% Manually include selected file,
% otherwise process as usual:
%    \begin{macrocode}
\ifchilddocmanual
\section*{part `\childdocname'}
\input{\childdocname}
\else
%    \end{macrocode}

% Include the two chapters:
%    \begin{macrocode}
\include{cdocsch1}
\include{cdocsch2}
%    \end{macrocode}

% Include the two parts unless only chapters should be displayed:
%    \begin{macrocode}
\ifchilddoc\else
\section{part three}
\input{cdocspt3}
\section{part four}
\input{cdocspt4}
\fi
%    \end{macrocode}

% Process as usual until here:
%    \begin{macrocode}
\fi
%    \end{macrocode}

% End of document body:
%    \begin{macrocode}
\end{document}
%    \end{macrocode}
%\iffalse
%</samplemain>
%\fi
%
% %%%%%%%%%%%%%%%%%%%%%%%%%%%%%%%%%%%%%%
% \paragraph{Chapter Include Files.}
%
% The include files are called |cdocsch1.tex| and |cdocsch2.tex|.
%
%\iffalse
%<*samplechap1|samplechap2>
%\fi

% Optional override for |\version| flag:
%    \begin{macrocode}
%%\providecommand{\version}{final}
%    \end{macrocode}

% Include the main document:
%    \begin{macrocode}
\input{childdoc.def}
\childdocof{cdocsamp}
%    \end{macrocode}

%\iffalse
%</samplechap1|samplechap2>
%\fi
%
%\iffalse
%<*samplechap1>
%\fi
% Some text for chapter 1:
%    \begin{macrocode}
\section{one}
some text in chapter one
%    \end{macrocode}

%\iffalse
%</samplechap1>
%\fi
% Some text for chapter 2:
%\iffalse
%<*samplechap2>
%\fi
%    \begin{macrocode}
\section{two}
more text in chapter two
%    \end{macrocode}

%\iffalse
%</samplechap2>
%\fi
%
% %%%%%%%%%%%%%%%%%%%%%%%%%%%%%%%%%%%%%%
% \paragraph{Part Include Files.}
%
% The include files are called |cdocspt3.tex| and |cdocspt4.tex|.
%
%\iffalse
%<*samplepart3|samplepart4>
%\fi

% Optional override for |\version| flag:
%    \begin{macrocode}
%%\providecommand{\version}{final}
%    \end{macrocode}

% Include the main document:
%    \begin{macrocode}
\input{childdoc.def}
\childdocby{cdocsamp}
%    \end{macrocode}

%\iffalse
%</samplepart3|samplepart4>
%\fi
%
%\iffalse
%<*samplepart3>
%\fi
% Some text for part 3:
%    \begin{macrocode}
some text in part three
%    \end{macrocode}

%\iffalse
%</samplepart3>
%\fi
% Some text for part 4:
%\iffalse
%<*samplepart4>
%\fi
%    \begin{macrocode}
more text in part four
%    \end{macrocode}

%\iffalse
%</samplepart4>
%\fi
%
% %%%%%%%%%%%%%%%%%%%%%%%%%%%%%%%%%%%%%%
% \paragraph{Forwarding for a Complete Draft.}
%
% The following forwarding file |cdocsdrf.tex|
% compiles the main document in draft mode:
%\iffalse
%<*sampledraft>
%\fi
%    \begin{macrocode}
\def\version{draft}
\input{childdoc.def}
\childdocforward{cdocsamp}
%    \end{macrocode}

%\iffalse
%</sampledraft>
%\fi
%
% %%%%%%%%%%%%%%%%%%%%%%%%%%%%%%%%%%%%%%
% \paragraph{Forwarding for Final Version of the Chapters.}
%
% The following forwarding files |cdocsfn1.tex| and |cdocsfn2.tex|
% (with identical content)
% compile the final versions of the child documents
% |cdocsch1.tex| and |cdocsch2.tex|, respectively:
%\iffalse
%<*samplefinal>
%\fi
%    \begin{macrocode}
\def\version{final}
\input{childdoc.def}
\childdocforwardprefix[cdocsamp]{cdocsfn}{cdocsch}
%    \end{macrocode}

%\iffalse
%</samplefinal>
%\fi
%
% %%%%%%%%%%%%%%%%%%%%%%%%%%%%%%%%%%%%%%
% \paragraph{Command Line Processing.}
%
% The following three command lines generate the output files
% |cdocscld|, |cdocscl1| and |cdocscl2|
% which should be identical to
% |cdocsdrf|, |cdocsch1| and |cdocsfn2|, respectively:
% \begin{center}
% \begin{tabular}{l}
% |latex -jobname cdocscld \|\\
% |  "\def\version{draft}\input{childdoc.def}\childdocforward{cdocsamp}"|\\
% |latex -jobname cdocscl1 \|\\
% |  "\input{childdoc.def}\childdocforward[cdocsamp]{cdocsch1}"|\\
% |latex -jobname cdocscl2 \|\\
% |  "\def\version{final}\input{childdoc.def}\childdocforward{cdocsch2}"|
% \end{tabular}
% \end{center}
% Note that the trailing backslash on each first line
% merely continues the input to the second line
% (for convenient cut ant paste).
% Furthermore, the command |latex| can be replaced by any
% of its alternative versions such as |pdflatex|.
%
% %%%%%%%%%%%%%%%%%%%%%%%%%%%%%%%%%%%%%%%%%%%%%%%%%%%%%%%%%%%%%%%%%%%%%%%%%%%%%%
% %%%%%%%%%%%%%%%%%%%%%%%%%%%%%%%%%%%%%%%%%%%%%%%%%%%%%%%%%%%%%%%%%%%%%%%%%%%%%%
% \section{Implementation}
%\iffalse
%<*package>
%\fi
%
% This section describes the definitions file |childdoc.def|.

% The definitions cannot be loaded using |\usepackage| or |\RequirePackage|
% which has a mechanism to prevent loading a style file more than once.
% When loading the definitions by means of |\input|
% multiple instances have to be prevented manually:
%\iffalse
%This code needs to be before the `\ProvidesFile' directive
%which is defined at the beginning of this file.
%Therefore it is also placed there and commented out here.
%</package>
%<*discard>
%\fi
%    \begin{macrocode}
\ifdefined\childdocmain\endinput\fi
%    \end{macrocode}
%\iffalse
%</discard>
%<*package>
%\fi
%
% \macro{\ifchilddoc}
% \macro{\ifchilddocmanual}
% The conditional |\ifchilddoc| tells whether a
% child (true) or main (false) document is being compiled.
% The conditional |\ifchilddocmanual| tells whether
% the |\includeonly| mechanism is used (false) or
% the selection of child files must be performed manually (true).
% The definitions initialise to false:
%    \begin{macrocode}
\newif\ifchilddoc
\newif\ifchilddocmanual
%    \end{macrocode}

% \macro{\childdocname}
% \macro{\childdocjob}
% The macro |\childdocname| stores the name of the main document
% to be compiled. The macro |\childdocjob| stores the name of
% the document on which the \LaTeX{} compiler was originally invoked.
% The content of |\jobname| cannot be compared
% to filenames specified in the source due to different catcodes.
% The following code rescans |\jobname|, stores the result
% in |\childdocname| and saves a copy in |\childdocjob|:
%    \begin{macrocode}
\edef\childdocname{\scantokens\expandafter{\jobname\noexpand}}
\let\childdocjob\childdocname
%    \end{macrocode}

% \macro{\childdocdisable}
% The macro |\childdocdisable| prevents the main file
% from being processed more than once.
% At this stage, the main document command |\childdocmain|
% is assumed to be called once again where it should do nothing.
% Any subsequent call to it should prevent
% a secondary processing of the main document
% It overwrites the forwarding commands
% |\childdocof| and |\childdocforward|
% with empty macros to prevent further inclusions of the main document:
%    \begin{macrocode}
\newcommand{\childdocdisable}
{
  \renewcommand{\childdocmain}[1]{\renewcommand{\childdocmain}[1]{\endinput}}
  \renewcommand{\childdocof}[1]{}
  \renewcommand{\childdocby}[2][]{}
  \renewcommand{\childdocforward}[2][]{}
  \renewcommand{\childdocdisable}{}
}
%    \end{macrocode}

% \macro{\childdocmain}
% The macro |\childdocmain| is to be called at the top of the main file
% with nothing or the main filename (without extension) as argument.
% First, it breaks loops.
% If the argument is not empty and does not match |\childdocname|
% (which is set by the first inclusion of |childdoc.def|),
% |\ifchilddoc| is set to true, |\includeonly| is applied to the child file
% and |\jobname| is set to the main file
% (for proper handling of |.aux| files):
%    \begin{macrocode}
\newcommand{\childdocmain}[1]
{
  \childdocdisable\childdocmain{}
  \if?#1?\else
    \begingroup
      \def\childdoctmp{#1}
      \ifx\childdoctmp\childdocname
        \def\childdoctmp{}
      \else
        \def\childdoctmp
        {
          \childdoctrue
          \includeonly{\childdocname}
          \def\childdocjob{#1}
          \def\jobname{#1}
        }
      \fi
      \expandafter
    \endgroup
    \childdoctmp
  \fi
}
%    \end{macrocode}

% \macro{\childdocof}
% The command |\childdocof| redirects
% compilation to the main file |#1|.
%    \begin{macrocode}
\newcommand{\childdocof}[1]
{
  \childdocdisable
  \childdoctrue
  \includeonly{\childdocname}
  \def\jobname{#1}
  \def\childdocjob{#1}
  \input{#1}
}
%    \end{macrocode}

% \macro{\childdocby}
% The command |\childdocby| ....
%    \begin{macrocode}
\newcommand{\childdocby}[2][]
{
  \childdocdisable
  \childdoctrue
  \childdocmanualtrue
  \if?#1?\else
    \def\jobname{#2}
  \fi
  \def\childdocjob{#2}
  \input{#2}
  \endinput
}
%    \end{macrocode}

% \macro{\childdocforward}
% The command |\childdocforward| redirects
% compilation to the main file or
% (if the optional argument is given) a child file.
% Parameters are set as if the main file
% or a child file starting with |\childdocof| was compiled.
% Then compilation is handed over to the main file:
%    \begin{macrocode}
\newcommand{\childdocforward}[2][]
{
  \begingroup
    \if?#1?
      \def\childdoctmp
      {
        \def\childdocname{#2}
        \def\childdocjob{#2}
        \def\jobname{#2}
        \input{#2}
        \endinput
      }
    \else
      \def\childdoctmp
      {
        \childdocdisable
        \def\childdocname{#2}
        \childdoctrue
        \includeonly{#2}
        \def\childdocjob{#1}
        \def\jobname{#1}
        \input{#1}
        \endinput
      }
    \fi
    \expandafter
  \endgroup
  \childdoctmp
}
%    \end{macrocode}

% \macro{\childdocforwardprefix}
% The command |\childdocforwardprefix| redirects
% compilation to the main or a child file by means of a pattern.
% The prefix |#1| in the current filename is replaced by |#2|
% and the suffix of the current filename is kept
% (it is assumed that the filename does not contain the substring `|~~~|'
% which is used as a delimiter).
% Compilation is handed over to the new file by |\childdocforward|:
%    \begin{macrocode}
\newcommand{\childdocforwardprefix}[3][]
{
  \begingroup
    \def\childdocextract #2##1~~~{\def\childdoctmp{\childdocforward[#1]{#3##1}}}
    \expandafter\childdocextract\childdocname~~~
    \expandafter
  \endgroup
  \childdoctmp
}
%    \end{macrocode}

% \macro{\childdoc}
% The deprecated macro |\childdoc| is a legacy version of |\childdocmain|:
%    \begin{macrocode}
\newcommand{\childdoc}{\childdocmain}
%    \end{macrocode}

% \macro{\childdocredirect}
% The deprecated macro |\childdocredirect| is a legacy version
% of |\childdocforward| and |\childdocforwardprefix|:
%    \begin{macrocode}
\newcommand{\childdocredirect}[2][]
{
  \begingroup
    \if?#1?
      \def\childdoctmp{\childdocforward{#2}}
    \else
      \def\childdoctmp{\childdocforwardprefix{#1}{#2}}
    \fi
    \expandafter
  \endgroup
  \childdoctmp
}
%    \end{macrocode}

%\iffalse
%</package>
%\fi
%
\endinput
|\\
|\childdocof{|\textit{main}|}|\\
\end{tabular}
\end{center}
at the top of every child file \textit{child}
which is included by |\include{|\textit{child}|}|
from within the main file
(or at least for those files to be compiled individually).
The argument \textit{main} must be the filename of the main file.

There are a couple of
considerations in setting up the main and child documents:

%%%%%%%%%%%%%%%%%%%%%%%%%%%%%%%%%%%%%%%%
\paragraph{Restrictions.}

Please note the following restrictions:
\begin{itemize}
\item
|\childdocmain| must be called with one argument \textit{main}
to ensure compatibility with earlier version of the package.
It must either be empty (|\childdocmain{}|)
or precisely match the filename of the main file in which it is specified.
See \secref{sec:detection} for further information.
\item
The filename \textit{main} must be specified without the |.tex| extension.
\item
The filename \textit{main} is case sensitive
(even in case-insensitive file systems)
due to internal string comparison.
\item
The argument \textit{main} should be fully expanded, it cannot be a macro.
\item
Subdirectories and special characters should be avoided in filenames.
\item
The command |\childdocmain{|\textit{main}|}| must be followed by a whitespace.
It should not be followed immediately by another command
or by a comment mark `|%|'.
This is because the \TeX{} parser reads the token immediately following
the argument of |\childdocmain| and puts it
at the beginning of every child section;
however, a white\-space is ignored.
\end{itemize}

%%%%%%%%%%%%%%%%%%%%%%%%%%%%%%%%%%%%%%%%
\paragraph{Content of Main File.}

It is advisable to place all content in the child files included by |\include|.
Any output contained in the main file will appear in all child documents
unless suppressed manually;
it cannot be suppressed automatically by the |\includeonly| directive
and thus should normally be avoided.
A method to include some content in the main file
by means of conditional processing is described in \secref{sec:conditional}.

%%%%%%%%%%%%%%%%%%%%%%%%%%%%%%%%%%%%%%%%
\paragraph{Page Numbering.}

When only a part of the document is compiled,
the appropriate numbering of pages
(as well as other status parameters)
is determined from the |.aux| files.
The latter contain information from previous passes.
However this information needs to propagate through
all intermediate child documents.
Therefore the page numbering in child documents may well
be inconsistent until the complete document is compiled at least once.

A useful (if unconventional) way to always ensure a consistent
page numbering is to restart the numbering in each child document
and denote the pages by `\textit{child}|.|\textit{page}'
where \textit{child} represents the chapter/section number of the child file.
This can be achieved by the command
|\numberwithin{page}{|\textit{child}|}|
of the \textsf{amsmath} package
where \textit{child} can be |chapter| or |section|
depending on the chosen structuring.
Alternatively, one can modify the macro |\thepage| appropriately
and reset the counter |page| at the start of each child file.

%%%%%%%%%%%%%%%%%%%%%%%%%%%%%%%%%%%%%%%%%%%%%%%%%%%%%%%%%%%%%%%%%%%%%%%%%%%%%%%%
\subsection{Conditional Processing}
\label{sec:conditional}

The package provides a mechanism to compile different versions
of a document. To customise the versions further some conditional processing
can come in handy to distinguish which version is being compiled.
The package provides two macros to describe the compilation context:

%%%%%%%%%%%%%%%%%%%%%%%%%%%%%%%%%%%%%%%%
\DescribeMacro{\ifchilddoc}
The conditional |\ifchilddoc| distinguishes between the compilation of
child documents and the main document:
%
\begin{center}
|\ifchilddoc |\textit{child-code}| |[|\||else |\textit{main-code}]| \||fi|
\end{center}

%%%%%%%%%%%%%%%%%%%%%%%%%%%%%%%%%%%%%%%%
\DescribeMacro{\childdocname}
\DescribeMacro{\childdocjob}
The macro |\childdocname| contains the filename (without extension)
of the main or child file being processed.
Note that |\childdocjob| will always contain the name of the main file.

%%%%%%%%%%%%%%%%%%%%%%%%%%%%%%%%%%%%%%%%
\paragraph{Title Page.}

Conditional processing can be used to include a title or banner page
in the main document when proper precautions are taken.
Importantly, the code in the main file should ensure that the page counter
(as well as other status parameters which are stored in the |.aux| files)
takes the same value after the conditional processing.
Otherwise the page numbers may take divergent values
depending on which part is compiled.

For example, a title page could be declared by:
%
\begin{center}
\begin{tabular}{l}
|\ifchilddoc\||else|\\
|\addtocounter{page}{-1}|\\
\textit{code for title page}\\
|\newpage|\\
|\||fi|
\end{tabular}
\end{center}
%
A banner page for the child documents can be generated by:
%
\begin{center}
\begin{tabular}{l}
|\ifchilddoc|\\
|\addtocounter{page}{-1}|\\
\textit{code for banner page}\\
|\newpage|\\
|\||fi|
\end{tabular}
\end{center}
%
Here one could write a message such as:
\begin{center}
|This is the part \childdocname{} of \childdocjob{}.|
\end{center}

%%%%%%%%%%%%%%%%%%%%%%%%%%%%%%%%%%%%%%%%%%%%%%%%%%%%%%%%%%%%%%%%%%%%%%%%%%%%%%%%
\subsection{Flags}
\label{sec:flags}

The package makes it easy to generate different versions
of the main or child documents.
To this end compilation flags can be defined
and assigned different default values.
They will be particularly useful in conjunction
with the forwarding mechanism described in \secref{sec:forward}.

For example, it may be useful to have a flag |\version|
which can be set to |draft| or |final|.
The document source will contain some conditional code
depending on the value of |\version|.
Suppose further, the flag should default to |final| for the main file
and to |draft| for child files
which is a natural assignment for editing the document.
This is achieved by placing the following code
in the preamble of the main document
(below the |\childdocmain| directive):
%
\begin{center}
\begin{tabular}{l}
|\ifchilddoc|\\
|\providecommand{\version}{draft}|\\
|\||else|\\
|\providecommand{\version}{final}|\\
|\||fi|
\end{tabular}
\end{center}
%
The definition by |\providecommand| makes sure
that previous definitions are not overwritten.
Further statements |\providecommand{\version}{...}|
can thus be added before the above code to override it.

For the main file, one might add a line
(between |\childdocmain| and the above block)
%
\begin{center}
|%\ifchilddoc\||else\providecommand{\version}{draft}\||fi|
\end{center}
%
which can be uncommented to produce a draft version.
Likewise one can add a line to the very top of a child file
(above the |\childdocof{|\textit{main}|}| directive)
%
\begin{center}
|%\providecommand{\version}{final}|
\end{center}
%
which can be uncommented to produce the final version of this child document.

%%%%%%%%%%%%%%%%%%%%%%%%%%%%%%%%%%%%%%%%%%%%%%%%%%%%%%%%%%%%%%%%%%%%%%%%%%%%%%%%
\subsection{Forwarding}
\label{sec:forward}

Different versions of the main or child documents
using compilation flags as described in \secref{sec:flags}
can be (permanently) stored in different files
for convenient compilation, viewing and distribution.
To this end, the package defines a command
to pass on compilation to a different file:

%%%%%%%%%%%%%%%%%%%%%%%%%%%%%%%%%%%%%%%%
\DescribeMacro{\childdocforward}
The command |\childdocforward| redirects processing to
another source file:
%
\begin{center}
\begin{tabular}{l}
|% \iffalse
%
% childdoc.dtx Copyright (C) 2017-2018 Niklas Beisert
%
% This work may be distributed and/or modified under the
% conditions of the LaTeX Project Public License, either version 1.3
% of this license or (at your option) any later version.
% The latest version of this license is in
%   http://www.latex-project.org/lppl.txt
% and version 1.3 or later is part of all distributions of LaTeX
% version 2005/12/01 or later.
%
% This work has the LPPL maintenance status `maintained'.
%
% The Current Maintainer of this work is Niklas Beisert.
%
% This work consists of the files childdoc.dtx and childdoc.ins
% and the derived files childdoc.def and cdocsamp.tex with
% cdocsch1.tex, cdocsch2.tex, cdocsdrf.tex, cdocsfn1.tex, cdocsfn2.tex.
%
%<package>\ifdefined\childdocmain\endinput\fi
%<package>\ProvidesFile{childdoc.def}[2018/12/30 v2.0 child document driver]
%<samplemain>\ProvidesFile{cdocsamp.tex}[2018/12/30 v2.0 sample for childdoc]
%<*driver>
%\ProvidesFile{childdoc.drv}[2018/12/30 v2.0 childdoc reference manual file]
\PassOptionsToClass{10pt,a4paper}{article}
\documentclass{ltxdoc}

\usepackage[margin=35mm]{geometry}
\usepackage{hyperref}
\usepackage{hyperxmp}
\usepackage[usenames]{color}

\hypersetup{colorlinks=true}
\hypersetup{pdfstartview=FitH}
\hypersetup{pdfpagemode=UseNone}
\hypersetup{pdfsource={}}
\hypersetup{pdflang={en-UK}}
\hypersetup{pdfcopyright={Copyright 2017-2018 Niklas Beisert.
  This work may be distributed and/or modified under the
  conditions of the LaTeX Project Public License, either version 1.3
  of this license or (at your option) any later version.}}
\hypersetup{pdflicenseurl={http://www.latex-project.org/lppl.txt}}
\hypersetup{pdfcontactaddress={ETH Zurich, ITP, HIT K,
  Wolfgang-Pauli-Strasse 27}}
\hypersetup{pdfcontactpostcode={8093}}
\hypersetup{pdfcontactcity={Zurich}}
\hypersetup{pdfcontactcountry={Switzerland}}
\hypersetup{pdfcontactemail={nbeisert@itp.phys.ethz.ch}}
\hypersetup{pdfcontacturl={http://people.phys.ethz.ch/\xmptilde nbeisert/}}

\newcommand{\secref}[1]{\hyperref[#1]{section \ref*{#1}}}

\parskip1ex
\parindent0pt
\let\olditemize\itemize
\def\itemize{\olditemize\parskip0pt}

\begin{document}

\title{The \textsf{childdoc} Package}
\hypersetup{pdftitle={The childdoc Package}}
\author{Niklas Beisert\\[2ex]
  Institut f\"ur Theoretische Physik\\
  Eidgen\"ossische Technische Hochschule Z\"urich\\
  Wolfgang-Pauli-Strasse 27, 8093 Z\"urich, Switzerland\\[1ex]
  \href{mailto:nbeisert@itp.phys.ethz.ch}
  {\texttt{nbeisert@itp.phys.ethz.ch}}}
\hypersetup{pdfauthor={Niklas Beisert}}
\hypersetup{pdfsubject={Manual for the LaTeX2e Package childdoc}}
\date{30 December 2018, \textsf{v2.0}}
\maketitle

\begin{abstract}\noindent
\textsf{childdoc} is a \LaTeXe{} package
that enables the direct compilation
of document sections included by |\include|
to individual files.
\end{abstract}

\begingroup
\parskip0ex
\tableofcontents
\endgroup

%%%%%%%%%%%%%%%%%%%%%%%%%%%%%%%%%%%%%%%%%%%%%%%%%%%%%%%%%%%%%%%%%%%%%%%%%%%%%%%%
%%%%%%%%%%%%%%%%%%%%%%%%%%%%%%%%%%%%%%%%%%%%%%%%%%%%%%%%%%%%%%%%%%%%%%%%%%%%%%%%
\section{Introduction}

\LaTeX{} provides a mechanism to structure a large document (such as a book)
into a main file and several child files (containing the chapters)
using the |\include| command.
This mechanism is beneficial for documents
which span hundreds of pages in order to
make the source file(s) more manageable.
Moreover, compilation can be restricted to
selected child files by means of the |\includeonly| command.
The latter feature can be used to reduce the compilation time while editing
(this was significantly more useful in the earlier days of \LaTeX{})
or to generate a smaller document which is easier to navigate.
Another application of |\includeonly| is to generate
documents consisting of selected parts of the complete document.

However, there are a few drawbacks of the plain |\include| mechanism:
\begin{itemize}
\item
The child files cannot be compiled on their own,
they can only be compiled via the main file.
A naive editing environment
(such as a text editor with an option
to have the current file processed by \LaTeX)
may require one to switch to the main file before compiling;
attempting to compile the child file produces errors.
\item
The main file must be modified (each time)
to adjust the |\includeonly| command
to the present needs. This easily leaves the main file in a messy state.
\item
The generated document will always carry the filename
of the main document. This is inconvenient if
several child files are to be compiled and
to be kept for distribution.
\end{itemize}

The present package provides a simple interface
to make child files individually compilable by \LaTeX{}.
Compiling a child file then has the same effect as compiling
the main file with an |\includeonly| command
to select the appropriate child.
Moreover the generated document will carry the name of the child
rather than the main file.
This resolves all three above issues.

This feature is meant to make the editing of books,
thesis documents and lecture notes somewhat more convenient.
However, the package can also be used efficiently for
composing a series of documents (such as exercise sheets)
which are typically distributed individually.
It then assists the author in generating the individual documents
(potentially in different versions)
as well as a document containing the collected series.
Another application is in developing style files
or other kinds of included material
where compilation of the style file could redirect
to a sample or test file.

%%%%%%%%%%%%%%%%%%%%%%%%%%%%%%%%%%%%%%%%%%%%%%%%%%%%%%%%%%%%%%%%%%%%%%%%%%%%%%%%
%%%%%%%%%%%%%%%%%%%%%%%%%%%%%%%%%%%%%%%%%%%%%%%%%%%%%%%%%%%%%%%%%%%%%%%%%%%%%%%%
\section{Usage}

First of all, the package \textsf{childdoc} is \emph{not} a standard
\LaTeXe{} |.sty| style file! Therefore it needs to be invoked in
a non-standard way.

%%%%%%%%%%%%%%%%%%%%%%%%%%%%%%%%%%%%%%%%%%%%%%%%%%%%%%%%%%%%%%%%%%%%%%%%%%%%%%%%
\subsection{Included Files}
\label{sec:include}

%%%%%%%%%%%%%%%%%%%%%%%%%%%%%%%%%%%%%%%%
\DescribeMacro{\childdocmain}
To use the package, add the commands
\begin{center}
\begin{tabular}{l}
|\input{childdoc.def}|\\
|\childdocmain{}|\\
\end{tabular}
\end{center}
at the very top of the main \LaTeX{} file,
in particular \emph{before} the |\documentclass| statement!
The argument of |\childdocmain| should be left empty
(but it must be present).

%%%%%%%%%%%%%%%%%%%%%%%%%%%%%%%%%%%%%%%%
\DescribeMacro{\childdocof}
Furthermore, add the commands
\begin{center}
\begin{tabular}{l}
|\input{childdoc.def}|\\
|\childdocof{|\textit{main}|}|\\
\end{tabular}
\end{center}
at the top of every child file \textit{child}
which is included by |\include{|\textit{child}|}|
from within the main file
(or at least for those files to be compiled individually).
The argument \textit{main} must be the filename of the main file.

There are a couple of
considerations in setting up the main and child documents:

%%%%%%%%%%%%%%%%%%%%%%%%%%%%%%%%%%%%%%%%
\paragraph{Restrictions.}

Please note the following restrictions:
\begin{itemize}
\item
|\childdocmain| must be called with one argument \textit{main}
to ensure compatibility with earlier version of the package.
It must either be empty (|\childdocmain{}|)
or precisely match the filename of the main file in which it is specified.
See \secref{sec:detection} for further information.
\item
The filename \textit{main} must be specified without the |.tex| extension.
\item
The filename \textit{main} is case sensitive
(even in case-insensitive file systems)
due to internal string comparison.
\item
The argument \textit{main} should be fully expanded, it cannot be a macro.
\item
Subdirectories and special characters should be avoided in filenames.
\item
The command |\childdocmain{|\textit{main}|}| must be followed by a whitespace.
It should not be followed immediately by another command
or by a comment mark `|%|'.
This is because the \TeX{} parser reads the token immediately following
the argument of |\childdocmain| and puts it
at the beginning of every child section;
however, a white\-space is ignored.
\end{itemize}

%%%%%%%%%%%%%%%%%%%%%%%%%%%%%%%%%%%%%%%%
\paragraph{Content of Main File.}

It is advisable to place all content in the child files included by |\include|.
Any output contained in the main file will appear in all child documents
unless suppressed manually;
it cannot be suppressed automatically by the |\includeonly| directive
and thus should normally be avoided.
A method to include some content in the main file
by means of conditional processing is described in \secref{sec:conditional}.

%%%%%%%%%%%%%%%%%%%%%%%%%%%%%%%%%%%%%%%%
\paragraph{Page Numbering.}

When only a part of the document is compiled,
the appropriate numbering of pages
(as well as other status parameters)
is determined from the |.aux| files.
The latter contain information from previous passes.
However this information needs to propagate through
all intermediate child documents.
Therefore the page numbering in child documents may well
be inconsistent until the complete document is compiled at least once.

A useful (if unconventional) way to always ensure a consistent
page numbering is to restart the numbering in each child document
and denote the pages by `\textit{child}|.|\textit{page}'
where \textit{child} represents the chapter/section number of the child file.
This can be achieved by the command
|\numberwithin{page}{|\textit{child}|}|
of the \textsf{amsmath} package
where \textit{child} can be |chapter| or |section|
depending on the chosen structuring.
Alternatively, one can modify the macro |\thepage| appropriately
and reset the counter |page| at the start of each child file.

%%%%%%%%%%%%%%%%%%%%%%%%%%%%%%%%%%%%%%%%%%%%%%%%%%%%%%%%%%%%%%%%%%%%%%%%%%%%%%%%
\subsection{Conditional Processing}
\label{sec:conditional}

The package provides a mechanism to compile different versions
of a document. To customise the versions further some conditional processing
can come in handy to distinguish which version is being compiled.
The package provides two macros to describe the compilation context:

%%%%%%%%%%%%%%%%%%%%%%%%%%%%%%%%%%%%%%%%
\DescribeMacro{\ifchilddoc}
The conditional |\ifchilddoc| distinguishes between the compilation of
child documents and the main document:
%
\begin{center}
|\ifchilddoc |\textit{child-code}| |[|\||else |\textit{main-code}]| \||fi|
\end{center}

%%%%%%%%%%%%%%%%%%%%%%%%%%%%%%%%%%%%%%%%
\DescribeMacro{\childdocname}
\DescribeMacro{\childdocjob}
The macro |\childdocname| contains the filename (without extension)
of the main or child file being processed.
Note that |\childdocjob| will always contain the name of the main file.

%%%%%%%%%%%%%%%%%%%%%%%%%%%%%%%%%%%%%%%%
\paragraph{Title Page.}

Conditional processing can be used to include a title or banner page
in the main document when proper precautions are taken.
Importantly, the code in the main file should ensure that the page counter
(as well as other status parameters which are stored in the |.aux| files)
takes the same value after the conditional processing.
Otherwise the page numbers may take divergent values
depending on which part is compiled.

For example, a title page could be declared by:
%
\begin{center}
\begin{tabular}{l}
|\ifchilddoc\||else|\\
|\addtocounter{page}{-1}|\\
\textit{code for title page}\\
|\newpage|\\
|\||fi|
\end{tabular}
\end{center}
%
A banner page for the child documents can be generated by:
%
\begin{center}
\begin{tabular}{l}
|\ifchilddoc|\\
|\addtocounter{page}{-1}|\\
\textit{code for banner page}\\
|\newpage|\\
|\||fi|
\end{tabular}
\end{center}
%
Here one could write a message such as:
\begin{center}
|This is the part \childdocname{} of \childdocjob{}.|
\end{center}

%%%%%%%%%%%%%%%%%%%%%%%%%%%%%%%%%%%%%%%%%%%%%%%%%%%%%%%%%%%%%%%%%%%%%%%%%%%%%%%%
\subsection{Flags}
\label{sec:flags}

The package makes it easy to generate different versions
of the main or child documents.
To this end compilation flags can be defined
and assigned different default values.
They will be particularly useful in conjunction
with the forwarding mechanism described in \secref{sec:forward}.

For example, it may be useful to have a flag |\version|
which can be set to |draft| or |final|.
The document source will contain some conditional code
depending on the value of |\version|.
Suppose further, the flag should default to |final| for the main file
and to |draft| for child files
which is a natural assignment for editing the document.
This is achieved by placing the following code
in the preamble of the main document
(below the |\childdocmain| directive):
%
\begin{center}
\begin{tabular}{l}
|\ifchilddoc|\\
|\providecommand{\version}{draft}|\\
|\||else|\\
|\providecommand{\version}{final}|\\
|\||fi|
\end{tabular}
\end{center}
%
The definition by |\providecommand| makes sure
that previous definitions are not overwritten.
Further statements |\providecommand{\version}{...}|
can thus be added before the above code to override it.

For the main file, one might add a line
(between |\childdocmain| and the above block)
%
\begin{center}
|%\ifchilddoc\||else\providecommand{\version}{draft}\||fi|
\end{center}
%
which can be uncommented to produce a draft version.
Likewise one can add a line to the very top of a child file
(above the |\childdocof{|\textit{main}|}| directive)
%
\begin{center}
|%\providecommand{\version}{final}|
\end{center}
%
which can be uncommented to produce the final version of this child document.

%%%%%%%%%%%%%%%%%%%%%%%%%%%%%%%%%%%%%%%%%%%%%%%%%%%%%%%%%%%%%%%%%%%%%%%%%%%%%%%%
\subsection{Forwarding}
\label{sec:forward}

Different versions of the main or child documents
using compilation flags as described in \secref{sec:flags}
can be (permanently) stored in different files
for convenient compilation, viewing and distribution.
To this end, the package defines a command
to pass on compilation to a different file:

%%%%%%%%%%%%%%%%%%%%%%%%%%%%%%%%%%%%%%%%
\DescribeMacro{\childdocforward}
The command |\childdocforward| redirects processing to
another source file:
%
\begin{center}
\begin{tabular}{l}
|\input{childdoc.def}|\\
|\childdocforward[|\textit{main}|]{|\textit{dest}|}|\\
\end{tabular}
\end{center}
%
The argument \textit{dest} is the destination file
(without extension).
It should be the main file or one of the child files.
Note that further \textsf{childdoc} directives
such as |\childdocof| and |\childdocforward|
in the indicated file will be processed in this form.
The optional argument \textit{main}
passes on directly to the main file \textit{main}
while pretending to compile the child \textit{dest}.
This form behaves as if \textit{dest}
issues |\childdocof{|\textit{main}|}| right away,
and no further \textsf{childdoc} directives will be processed.

%%%%%%%%%%%%%%%%%%%%%%%%%%%%%%%%%%%%%%%%
\DescribeMacro{\...prefix}
In the alternative form |\childdocforwardprefix|,
%
\begin{center}
\begin{tabular}{l}
|\input{childdoc.def}|\\
|\childdocforwardprefix[|\textit{main}|]{|\textit{prefix}|}{|\textit{dest}|}|
\end{tabular}
\end{center}
%
the destination file is determined by a pattern
depending on the current file:
To make this work, the current file must be called
`{\textit{prefix}\hspace{0.2em}\textit{suffix}}'
with \textit{prefix} matching precisely the argument.
Processing is then passed on to the file
`{\textit{dest}\hspace{0.2em}\textit{suffix}}'.
Surely, the same effect is achieved by
directly specifying the
argument `{\textit{dest}\hspace{0.2em}\textit{suffix}}'
in the first form.
However, that requires to set up a different file
for each child. With the alternative form of the command
all these files can have exactly the same content
which simplifies setting them up and maintaining them.

For example, the following file |draft.tex|
with a compilation flag |\version| as described in \secref{sec:flags}
compiles the main document as a draft:
%
\begin{center}
\begin{tabular}{l}
|\def\version{draft}|\\
|\input{childdoc.def}|\\
|\childdocforward{|\textit{main}|}|
\end{tabular}
\end{center}
%
Likewise, the following files |final|\textit{nn}|.tex|
compile the final version of the child document
|child|\textit{nn}|.tex|:
%
\begin{center}
\begin{tabular}{l}
|\def\version{final}|\\
|\input{childdoc.def}|\\
|\childdocforwardprefix{final}{child}|
\end{tabular}
\end{center}
%

Note that when several versions of a main file and/or of each child file
are to be generated, it may be convenient to set up a |Makefile| or
shell script to automatise the process.

%%%%%%%%%%%%%%%%%%%%%%%%%%%%%%%%%%%%%%%%%%%%%%%%%%%%%%%%%%%%%%%%%%%%%%%%%%%%%%%%
\subsection{Command Line Processing}
\label{sec:commandline}

The effect of redirection files can also be achieved by invoking
the \LaTeX{} compiler with a more elaborate command line.
Most conveniently this should be done as part
of a shell script or a |Makefile|.

When using \textsf{childdoc} in the main file, the following
command lines effectively perform a redirection
(note that depending on the shell being used,
backslashes may have to be doubled: `|\|' $\to$ `|\\|'):
%
\begin{center}
|... -jobname "|\textit{target}|" |\\|"|[\textit{flags}]%
|\input{childdoc.def}\childdocforward[|\textit{main}|]{|\textit{dest}|}"|
\end{center}
%
Here \textit{target} is the name of the output file,
\textit{main} is the name of the main file
and \textit{dest} is the name of the main or child file to be processed
(all filenames without extensions).
The optional argument \textit{main} can be omitted
if \textit{main} matches \textit{dest}.
Optionally, compilation \textit{flags} can be defined via |\def| commands.
This command line makes the \TeX{} engine believe
it is compiling the file \textit{target}
whose content is specified as the latter parameter.
The provided code then forwards the processing to
\textit{main} or \textit{dest} as described in \secref{sec:forward}.

%%%%%%%%%%%%%%%%%%%%%%%%%%%%%%%%%%%%%%%%%%%%%%%%%%%%%%%%%%%%%%%%%%%%%%%%%%%%%%%%
\subsection{Include by Input}
\label{sec:input}

Including child documents by |\include| has some restrictions by design.
Most notably, the content of a child document always occupies
its own set of pages; pages cannot be shared between child documents.
Usually, this behaviour makes perfect sense
because each child document contain an essential part of the document.
However, in some situations it may be desirable to compose
a document from a collection of parts
without having mandatory page breaks between then.
For this case, the package
provides a mechanism to include parts
by |\input| which can also be processed individually.
However, by construction this mechanism
requires manual handling of the content to be output.

%%%%%%%%%%%%%%%%%%%%%%%%%%%%%%%%%%%%%%%%
\DescribeMacro{\ifchilddocmanual}
The main file should be prepared as usual, see \secref{sec:include}.
However, the document body must make a distinction
between processing of an individual part and of the main document, e.g.:
%
\begin{center}
\begin{tabular}{l}
|\ifchilddocmanual|\\
|\input{\childdocname}|\\
|\||else|\\
\textit{document body with }|\input{|\textit{part}|}|\\
|\||fi|
\end{tabular}
\end{center}
%
The conditional |\ifchilddocmanual| is true whenever
a part to be included by |\input| is being compiled,
and the name of the part is stored in |\childdocname|.

%%%%%%%%%%%%%%%%%%%%%%%%%%%%%%%%%%%%%%%%
\DescribeMacro{\childdocby}
Each part to be included by |\input| should start with:
%
\begin{center}
\begin{tabular}{l}
|\input{childdoc.def}|\\
|\childdocby{|\textit{main}|}|\\
\end{tabular}
\end{center}
%
The directive |\childdocby| is similar to |\childdocof|
described in \secref{sec:include},
but the subsequent selection of content must be done manually.
To that end, both |\ifchilddoc| and |\ifchilddocmanual|
will be true upon processing of a part,
and the name of the part is stored in |\childdocname|.
Note that |\jobname| will be set to the filename of the current part
so that each part receives an individual |.aux| file
that does not interfere with the |.aux| file(s) of the main document.
This behaviour can be altered by the alternative form
|\childdocby[*]{|\textit{main}|}| (with a non-empty optional argument)
which uses the |.aux| file of the main document
by setting |\jobname| to \textit{main}.

%%%%%%%%%%%%%%%%%%%%%%%%%%%%%%%%%%%%%%%%%%%%%%%%%%%%%%%%%%%%%%%%%%%%%%%%%%%%%%%%
\subsection{Driver Development}
\label{sec:driver}

The \textsf{childdoc} mechanism can also be use for the development
of definition files such as \LaTeX{} styles or classes.
This case differs from the above setup with multiple parts
included by |\include| in that no |\includeonly| should be invoked.
This can be achieved by starting the include file
(before |\ProvidesPackage|) with:
%
\begin{center}
\begin{tabular}{l}
|\input{childdoc.def}|\\
|\childdocforward{|\textit{main}|}|\\
\end{tabular}
\end{center}
%
or alternatively with:
%
\begin{center}
\begin{tabular}{l}
|\input{childdoc.def}|\\
|\childdocby{|\textit{main}|}|\\
\end{tabular}
\end{center}
%
Both forms have slightly different effects as described above.
The main file is prepared as usual, see \secref{sec:include}.

%%%%%%%%%%%%%%%%%%%%%%%%%%%%%%%%%%%%%%%%%%%%%%%%%%%%%%%%%%%%%%%%%%%%%%%%%%%%%%%%
\subsection{Legacy Detection}
\label{sec:detection}

The directive |\childdocmain| in the main file can detect
whether the complete document or merely a child is to be compiled
even without using the directive |\childdocof|.
This method is deprecated because it is less robust
and there is no compelling reason to use it;
it is merely provided for backward compatibility
and it may be removed in future versions.

If the detection mechanism is to be used,
it is mandatory to correctly specify
the filename of the main file as the argument of |\childdocmain|:
%
\begin{center}
\begin{tabular}{l}
|\input{childdoc.def}|\\
|\childdocmain{|\textit{main}|}|\\
\end{tabular}
\end{center}
%
If |\jobname| does not match the argument \textit{main} of |\childdocmain|,
it is assumed that |\jobname| points to the child file to be compiled.
When using |\childdocmain| with the main file specified as argument,
it suffices to start a child file
with just |\input{|\textit{main}|}|
without loading of the package and using |\childdocof|.
If instead all processing is done
with the appropriate \textsf{childdoc} directives,
the argument of \textit{main} of |\childdocmain| can be empty.

An alternative version of the command line processing described
in \secref{sec:commandline} using the detection mechanism reads:
%
\begin{center}
|... -jobname "|\textit{target}|" "|[\textit{flags}]%
[|\def\jobname{|\textit{dest}|}|]|\input{|\textit{main}|}"|
\end{center}

%%%%%%%%%%%%%%%%%%%%%%%%%%%%%%%%%%%%%%%%%%%%%%%%%%%%%%%%%%%%%%%%%%%%%%%%%%%%%%%%
\subsection{Manual Code}
\label{sec:manual}

In case one cannot be certain whether the definitions file |childdoc.def|
is installed on the target \TeX{} distribution
and one prefers not to ship it,
it is conceivable to paste a few relevant commands into the sources.

To that end, drop all statements |\input{childdoc.def}|
and perform the replacements as outlined below.
Instead of |\childdocmain{|\textit{main}|}| add the following code
to the top of the main file:
%
\begin{center}
\begin{tabular}{l}
|\||ifdefined\childdocname\endinput\||fi\newif\ifchilddoc|\\
|\edef\childdocname{\scantokens\expandafter{\jobname\noexpand}}|\\
|\def\childdocmain{|\textit{main}|}\||ifx\childdocmain\childdocname\||else|\\
|\childdoctrue\includeonly{\childdocname}\let\jobname\childdocmain\||fi|\\
\end{tabular}
\end{center}
%
Instead of |\childdocof{|\textit{main}|}| just include the main file
at the top of each child file:
%
\begin{center}
|\input{|\textit{main}|}|
\end{center}
%
A simple redirection |\childdocforward{|\textit{dest}|}| is achieved by:
%
\begin{center}
|\def\jobname{|\textit{dest}|}\input{\jobname}|
\end{center}
%
The redirection with prefix
|\childdocforwardprefix[|\textit{prefix}|]{|\textit{dest}|}|
is accomplished by:
%
\begin{center}
\begin{tabular}{l}
|{\edef\jobname{\scantokens\expandafter{\jobname\noexpand}}|\\
|\def\redirectjob |\textit{prefix}|#1~~~{\gdef\jobname{|\textit{dest}|#1}}|\\
|\expandafter\redirectjob\jobname~~~}\input{\jobname}|
\end{tabular}
\end{center}

In an alternative approach,
child documents can be compiled by a specific command line
without additional code or specific definitions:
%
\begin{center}
|... -jobname "|\textit{target}|" "|[\textit{flags}]%
|\includeonly{|\textit{dest}|}\input{|\textit{main}|}"|
\end{center}
%

%%%%%%%%%%%%%%%%%%%%%%%%%%%%%%%%%%%%%%%%%%%%%%%%%%%%%%%%%%%%%%%%%%%%%%%%%%%%%%%%
%%%%%%%%%%%%%%%%%%%%%%%%%%%%%%%%%%%%%%%%%%%%%%%%%%%%%%%%%%%%%%%%%%%%%%%%%%%%%%%%
\section{Information}

%%%%%%%%%%%%%%%%%%%%%%%%%%%%%%%%%%%%%%%%%%%%%%%%%%%%%%%%%%%%%%%%%%%%%%%%%%%%%%%%
\subsection{Copyright}

Copyright \copyright{} 2017--2018 Niklas Beisert

This work may be distributed and/or modified under the
conditions of the \LaTeX{} Project Public License, either version 1.3
of this license or (at your option) any later version.
The latest version of this license is in
  \url{http://www.latex-project.org/lppl.txt}
and version 1.3 or later is part of all distributions of \LaTeX{}
version 2005/12/01 or later.

This work has the LPPL maintenance status `maintained'.

The Current Maintainer of this work is Niklas Beisert.

This work consists of the files |README.txt|, |childdoc.ins| and |childdoc.dtx|
as well as the derived files |childdoc.def|, |cdocsamp.tex|
with |cdocsch1.tex|, |cdocsch2.tex|, |cdocspt3.tex|, |cdocspt4.tex|,
|cdocsdrf.tex|, |cdocsfn1.tex|, |cdocsfn2.tex|
as well as |childdoc.pdf|.

%%%%%%%%%%%%%%%%%%%%%%%%%%%%%%%%%%%%%%%%%%%%%%%%%%%%%%%%%%%%%%%%%%%%%%%%%%%%%%%%
\subsection{Files and Installation}

The package consists of the files:
%
\begin{center}
\begin{tabular}{ll}
    |README.txt|   & readme file \\
    |childdoc.ins| & installation file \\
    |childdoc.dtx| & source file \\
    |childdoc.def| & definition file \\
    |cdocsamp.tex| & sample main file \\
    |cdocsch1.tex| & sample include file \\
    |cdocsch2.tex| & sample include file \\
    |cdocspt3.tex| & sample part file \\
    |cdocspt4.tex| & sample part file \\
    |cdocsdrf.tex| & sample redirection file \\
    |cdocsfn1.tex| & sample redirection file \\
    |cdocsfn2.tex| & sample redirection file \\
    |childdoc.pdf| & manual
\end{tabular}
\end{center}
%
The distribution consists of the files
|README.txt|, |childdoc.ins| and |childdoc.dtx|.
%
\begin{itemize}
\item
Run (pdf)\LaTeX{} on |childdoc.dtx|
to compile the manual |childdoc.pdf| (this file).
\item
Run \LaTeX{} on |childdoc.ins| to create the definitions file |childdoc.def|
and the sample |cdocsamp.tex| with include files
|cdocsch1.tex|, |cdocsch2.tex|, |cdocspt3.tex|, |cdocspt4.tex|,
|cdocsdrf.tex|, |cdocsfn1.tex|, |cdocsfn2.tex|.
Then copy the file |childdoc.def| to an appropriate directory of your \LaTeX{}
distribution, e.g.\ \textit{texmf-root}|/tex/latex/childdoc|.
\end{itemize}

%%%%%%%%%%%%%%%%%%%%%%%%%%%%%%%%%%%%%%%%%%%%%%%%%%%%%%%%%%%%%%%%%%%%%%%%%%%%%%%%
\subsection{Related CTAN Packages}

There are several other packages which offer a similar functionality:
%
\begin{itemize}
\item
The packages
\href{http://ctan.org/pkg/docmute}{\textsf{docmute}},
\href{http://ctan.org/pkg/includex}{\textsf{includex}} and
\href{http://ctan.org/pkg/standalone}{\textsf{standalone}}
provide commands to include only the document body of
a child file thus allowing both files to be compiled individually.
\item
The packages \href{http://ctan.org/pkg/subdocs}{\textsf{subdocs}}
and \href{http://ctan.org/pkg/subfiles}{\textsf{subfiles}}
provide structures in which the main and child documents can be
encapsulated and allowing them to be compiled individually.
The inclusion mechanism is different from the conventional |\include|.
\item
The package \href{http://ctan.org/pkg/combine}{\textsf{combine}}
is an elaborate solution to combine several documents into one.
\end{itemize}
%
See also the CTAN topic \href{http://ctan.org/topic/subdocs}{\textsf{subdocs}}
for further related packages.
The present package differs from the above solutions in that
a document structure constructed with the conventional |\include| mechanism
just needs two extra commands at the top of every file
such that all constituent files can be compiled individually.

%%%%%%%%%%%%%%%%%%%%%%%%%%%%%%%%%%%%%%%%%%%%%%%%%%%%%%%%%%%%%%%%%%%%%%%%%%%%%%%%
%\subsection{Feature Suggestions}
%
%The following is a list of features which may be useful for future
%versions of this package:
%%
%\begin{itemize}
%\item
%\ldots
%\end{itemize}

%%%%%%%%%%%%%%%%%%%%%%%%%%%%%%%%%%%%%%%%%%%%%%%%%%%%%%%%%%%%%%%%%%%%%%%%%%%%%%%%
\subsection{Revision History}

%%%%%%%%%%%%%%%%%%%%%%%%%%%%%%%%%%%%%%%%
\paragraph{v2.0:} 2018/12/30

\begin{itemize}
\item
immediate forward processing
\item
added |\childdocby| mechanism
\item
manual restructured
\end{itemize}

%%%%%%%%%%%%%%%%%%%%%%%%%%%%%%%%%%%%%%%%
\paragraph{v1.6:} 2018/01/17

\begin{itemize}
\item
application for development of include files
\item
corrections to manual
\end{itemize}

%%%%%%%%%%%%%%%%%%%%%%%%%%%%%%%%%%%%%%%%
\paragraph{v1.5:} 2017/05/21

\begin{itemize}
\item
more complete structuring introduced
\item
|\childdocof| introduced
\item
|\childdoc| renamed to |\childdocmain|
\item
|\childredirect| renamed to |\childdocforward| and |\childdocforwardprefix|
and functionality expanded
\end{itemize}

%%%%%%%%%%%%%%%%%%%%%%%%%%%%%%%%%%%%%%%%
\paragraph{v1.0:} 2017/04/27

\begin{itemize}
\item
manual and install package
\item
first version published on CTAN
\end{itemize}

%%%%%%%%%%%%%%%%%%%%%%%%%%%%%%%%%%%%%%%%
\paragraph{v0.6:} 2017/04/26

\begin{itemize}
\item
redirection mechanism added
\end{itemize}

%%%%%%%%%%%%%%%%%%%%%%%%%%%%%%%%%%%%%%%%
\paragraph{v0.5:} 2017/04/26

\begin{itemize}
\item
functionality in definition file
\end{itemize}


%%%%%%%%%%%%%%%%%%%%%%%%%%%%%%%%%%%%%%%%%%%%%%%%%%%%%%%%%%%%%%%%%%%%%%%%%%%%%%%%
%%%%%%%%%%%%%%%%%%%%%%%%%%%%%%%%%%%%%%%%%%%%%%%%%%%%%%%%%%%%%%%%%%%%%%%%%%%%%%%%
%%%%%%%%%%%%%%%%%%%%%%%%%%%%%%%%%%%%%%%%%%%%%%%%%%%%%%%%%%%%%%%%%%%%%%%%%%%%%%%%
\appendix

\settowidth\MacroIndent{\rmfamily\scriptsize 000\ }

 \DocInput{childdoc.dtx}

\end{document}
%</driver>
% \fi
%
% %%%%%%%%%%%%%%%%%%%%%%%%%%%%%%%%%%%%%%%%%%%%%%%%%%%%%%%%%%%%%%%%%%%%%%%%%%%%%%
% %%%%%%%%%%%%%%%%%%%%%%%%%%%%%%%%%%%%%%%%%%%%%%%%%%%%%%%%%%%%%%%%%%%%%%%%%%%%%%
% \section{Sample}
%\iffalse
%<*samplemain>
%\fi
%
% The following presents a sample document
% with two chapters, two parts, a title page,
% a compile flag as well as three forwarding files to set the flag.
% It consists of eight |.tex| files:
% \begin{center}
% \begin{tabular}{ll}
% |cdocsamp.tex|&main file\\
% |cdocsch1.tex|&include file for chapter 1\\
% |cdocsch2.tex|&include file for chapter 2\\
% |cdocspt3.tex|&include file for part 3\\
% |cdocspt4.tex|&include file for part 4\\
% |cdocsdrf.tex|&forwarding file for main file in draft mode\\
% |cdocsfi1.tex|&forwarding file for final version of chapter 1\\
% |cdocsfi2.tex|&forwarding file for final version of chapter 2\\
% \end{tabular}
% \end{center}
% Each of the eight files can be compiled directly by the \LaTeX{} compiler.
%
% %%%%%%%%%%%%%%%%%%%%%%%%%%%%%%%%%%%%%%
% \paragraph{Main File.}
%
% The main file is called |cdocsamp.tex|.
%
% Load the \textsf{childdoc} definitions and
% declare the filename for the main document:
%    \begin{macrocode}
\input{childdoc.def}
\childdocmain{}
%    \end{macrocode}

% Optional override for |\version| flag:
%    \begin{macrocode}
%%\ifchilddoc\else\providecommand{\version}{draft}\fi
%    \end{macrocode}

% Define the default values for the |\version| flag
% (|final| for the main file and |draft| for childs):
%    \begin{macrocode}
\ifchilddoc
\providecommand{\version}{draft}
\else
\providecommand{\version}{final}
\fi
%    \end{macrocode}

% Load the standard document class:
%    \begin{macrocode}
\documentclass[12pt]{article}
%    \end{macrocode}

% Start the document body:
%    \begin{macrocode}
\begin{document}
%    \end{macrocode}

% Declare a title page.
% Print title, part of document being processed and version flag:
%    \begin{macrocode}
\addtocounter{page}{-1}
\begin{center}
{\LARGE\bfseries{}childdoc example\par}
\vspace{1cm}
\ifchilddoc
\ifchilddocmanual part\else chapter\fi:
`\childdocname' of `\childdocjob'\par
\else
main document: `\childdocjob'\par
\fi
version: \version\par
\end{center}
\newpage
%    \end{macrocode}

% Manually include selected file,
% otherwise process as usual:
%    \begin{macrocode}
\ifchilddocmanual
\section*{part `\childdocname'}
\input{\childdocname}
\else
%    \end{macrocode}

% Include the two chapters:
%    \begin{macrocode}
\include{cdocsch1}
\include{cdocsch2}
%    \end{macrocode}

% Include the two parts unless only chapters should be displayed:
%    \begin{macrocode}
\ifchilddoc\else
\section{part three}
\input{cdocspt3}
\section{part four}
\input{cdocspt4}
\fi
%    \end{macrocode}

% Process as usual until here:
%    \begin{macrocode}
\fi
%    \end{macrocode}

% End of document body:
%    \begin{macrocode}
\end{document}
%    \end{macrocode}
%\iffalse
%</samplemain>
%\fi
%
% %%%%%%%%%%%%%%%%%%%%%%%%%%%%%%%%%%%%%%
% \paragraph{Chapter Include Files.}
%
% The include files are called |cdocsch1.tex| and |cdocsch2.tex|.
%
%\iffalse
%<*samplechap1|samplechap2>
%\fi

% Optional override for |\version| flag:
%    \begin{macrocode}
%%\providecommand{\version}{final}
%    \end{macrocode}

% Include the main document:
%    \begin{macrocode}
\input{childdoc.def}
\childdocof{cdocsamp}
%    \end{macrocode}

%\iffalse
%</samplechap1|samplechap2>
%\fi
%
%\iffalse
%<*samplechap1>
%\fi
% Some text for chapter 1:
%    \begin{macrocode}
\section{one}
some text in chapter one
%    \end{macrocode}

%\iffalse
%</samplechap1>
%\fi
% Some text for chapter 2:
%\iffalse
%<*samplechap2>
%\fi
%    \begin{macrocode}
\section{two}
more text in chapter two
%    \end{macrocode}

%\iffalse
%</samplechap2>
%\fi
%
% %%%%%%%%%%%%%%%%%%%%%%%%%%%%%%%%%%%%%%
% \paragraph{Part Include Files.}
%
% The include files are called |cdocspt3.tex| and |cdocspt4.tex|.
%
%\iffalse
%<*samplepart3|samplepart4>
%\fi

% Optional override for |\version| flag:
%    \begin{macrocode}
%%\providecommand{\version}{final}
%    \end{macrocode}

% Include the main document:
%    \begin{macrocode}
\input{childdoc.def}
\childdocby{cdocsamp}
%    \end{macrocode}

%\iffalse
%</samplepart3|samplepart4>
%\fi
%
%\iffalse
%<*samplepart3>
%\fi
% Some text for part 3:
%    \begin{macrocode}
some text in part three
%    \end{macrocode}

%\iffalse
%</samplepart3>
%\fi
% Some text for part 4:
%\iffalse
%<*samplepart4>
%\fi
%    \begin{macrocode}
more text in part four
%    \end{macrocode}

%\iffalse
%</samplepart4>
%\fi
%
% %%%%%%%%%%%%%%%%%%%%%%%%%%%%%%%%%%%%%%
% \paragraph{Forwarding for a Complete Draft.}
%
% The following forwarding file |cdocsdrf.tex|
% compiles the main document in draft mode:
%\iffalse
%<*sampledraft>
%\fi
%    \begin{macrocode}
\def\version{draft}
\input{childdoc.def}
\childdocforward{cdocsamp}
%    \end{macrocode}

%\iffalse
%</sampledraft>
%\fi
%
% %%%%%%%%%%%%%%%%%%%%%%%%%%%%%%%%%%%%%%
% \paragraph{Forwarding for Final Version of the Chapters.}
%
% The following forwarding files |cdocsfn1.tex| and |cdocsfn2.tex|
% (with identical content)
% compile the final versions of the child documents
% |cdocsch1.tex| and |cdocsch2.tex|, respectively:
%\iffalse
%<*samplefinal>
%\fi
%    \begin{macrocode}
\def\version{final}
\input{childdoc.def}
\childdocforwardprefix[cdocsamp]{cdocsfn}{cdocsch}
%    \end{macrocode}

%\iffalse
%</samplefinal>
%\fi
%
% %%%%%%%%%%%%%%%%%%%%%%%%%%%%%%%%%%%%%%
% \paragraph{Command Line Processing.}
%
% The following three command lines generate the output files
% |cdocscld|, |cdocscl1| and |cdocscl2|
% which should be identical to
% |cdocsdrf|, |cdocsch1| and |cdocsfn2|, respectively:
% \begin{center}
% \begin{tabular}{l}
% |latex -jobname cdocscld \|\\
% |  "\def\version{draft}\input{childdoc.def}\childdocforward{cdocsamp}"|\\
% |latex -jobname cdocscl1 \|\\
% |  "\input{childdoc.def}\childdocforward[cdocsamp]{cdocsch1}"|\\
% |latex -jobname cdocscl2 \|\\
% |  "\def\version{final}\input{childdoc.def}\childdocforward{cdocsch2}"|
% \end{tabular}
% \end{center}
% Note that the trailing backslash on each first line
% merely continues the input to the second line
% (for convenient cut ant paste).
% Furthermore, the command |latex| can be replaced by any
% of its alternative versions such as |pdflatex|.
%
% %%%%%%%%%%%%%%%%%%%%%%%%%%%%%%%%%%%%%%%%%%%%%%%%%%%%%%%%%%%%%%%%%%%%%%%%%%%%%%
% %%%%%%%%%%%%%%%%%%%%%%%%%%%%%%%%%%%%%%%%%%%%%%%%%%%%%%%%%%%%%%%%%%%%%%%%%%%%%%
% \section{Implementation}
%\iffalse
%<*package>
%\fi
%
% This section describes the definitions file |childdoc.def|.

% The definitions cannot be loaded using |\usepackage| or |\RequirePackage|
% which has a mechanism to prevent loading a style file more than once.
% When loading the definitions by means of |\input|
% multiple instances have to be prevented manually:
%\iffalse
%This code needs to be before the `\ProvidesFile' directive
%which is defined at the beginning of this file.
%Therefore it is also placed there and commented out here.
%</package>
%<*discard>
%\fi
%    \begin{macrocode}
\ifdefined\childdocmain\endinput\fi
%    \end{macrocode}
%\iffalse
%</discard>
%<*package>
%\fi
%
% \macro{\ifchilddoc}
% \macro{\ifchilddocmanual}
% The conditional |\ifchilddoc| tells whether a
% child (true) or main (false) document is being compiled.
% The conditional |\ifchilddocmanual| tells whether
% the |\includeonly| mechanism is used (false) or
% the selection of child files must be performed manually (true).
% The definitions initialise to false:
%    \begin{macrocode}
\newif\ifchilddoc
\newif\ifchilddocmanual
%    \end{macrocode}

% \macro{\childdocname}
% \macro{\childdocjob}
% The macro |\childdocname| stores the name of the main document
% to be compiled. The macro |\childdocjob| stores the name of
% the document on which the \LaTeX{} compiler was originally invoked.
% The content of |\jobname| cannot be compared
% to filenames specified in the source due to different catcodes.
% The following code rescans |\jobname|, stores the result
% in |\childdocname| and saves a copy in |\childdocjob|:
%    \begin{macrocode}
\edef\childdocname{\scantokens\expandafter{\jobname\noexpand}}
\let\childdocjob\childdocname
%    \end{macrocode}

% \macro{\childdocdisable}
% The macro |\childdocdisable| prevents the main file
% from being processed more than once.
% At this stage, the main document command |\childdocmain|
% is assumed to be called once again where it should do nothing.
% Any subsequent call to it should prevent
% a secondary processing of the main document
% It overwrites the forwarding commands
% |\childdocof| and |\childdocforward|
% with empty macros to prevent further inclusions of the main document:
%    \begin{macrocode}
\newcommand{\childdocdisable}
{
  \renewcommand{\childdocmain}[1]{\renewcommand{\childdocmain}[1]{\endinput}}
  \renewcommand{\childdocof}[1]{}
  \renewcommand{\childdocby}[2][]{}
  \renewcommand{\childdocforward}[2][]{}
  \renewcommand{\childdocdisable}{}
}
%    \end{macrocode}

% \macro{\childdocmain}
% The macro |\childdocmain| is to be called at the top of the main file
% with nothing or the main filename (without extension) as argument.
% First, it breaks loops.
% If the argument is not empty and does not match |\childdocname|
% (which is set by the first inclusion of |childdoc.def|),
% |\ifchilddoc| is set to true, |\includeonly| is applied to the child file
% and |\jobname| is set to the main file
% (for proper handling of |.aux| files):
%    \begin{macrocode}
\newcommand{\childdocmain}[1]
{
  \childdocdisable\childdocmain{}
  \if?#1?\else
    \begingroup
      \def\childdoctmp{#1}
      \ifx\childdoctmp\childdocname
        \def\childdoctmp{}
      \else
        \def\childdoctmp
        {
          \childdoctrue
          \includeonly{\childdocname}
          \def\childdocjob{#1}
          \def\jobname{#1}
        }
      \fi
      \expandafter
    \endgroup
    \childdoctmp
  \fi
}
%    \end{macrocode}

% \macro{\childdocof}
% The command |\childdocof| redirects
% compilation to the main file |#1|.
%    \begin{macrocode}
\newcommand{\childdocof}[1]
{
  \childdocdisable
  \childdoctrue
  \includeonly{\childdocname}
  \def\jobname{#1}
  \def\childdocjob{#1}
  \input{#1}
}
%    \end{macrocode}

% \macro{\childdocby}
% The command |\childdocby| ....
%    \begin{macrocode}
\newcommand{\childdocby}[2][]
{
  \childdocdisable
  \childdoctrue
  \childdocmanualtrue
  \if?#1?\else
    \def\jobname{#2}
  \fi
  \def\childdocjob{#2}
  \input{#2}
  \endinput
}
%    \end{macrocode}

% \macro{\childdocforward}
% The command |\childdocforward| redirects
% compilation to the main file or
% (if the optional argument is given) a child file.
% Parameters are set as if the main file
% or a child file starting with |\childdocof| was compiled.
% Then compilation is handed over to the main file:
%    \begin{macrocode}
\newcommand{\childdocforward}[2][]
{
  \begingroup
    \if?#1?
      \def\childdoctmp
      {
        \def\childdocname{#2}
        \def\childdocjob{#2}
        \def\jobname{#2}
        \input{#2}
        \endinput
      }
    \else
      \def\childdoctmp
      {
        \childdocdisable
        \def\childdocname{#2}
        \childdoctrue
        \includeonly{#2}
        \def\childdocjob{#1}
        \def\jobname{#1}
        \input{#1}
        \endinput
      }
    \fi
    \expandafter
  \endgroup
  \childdoctmp
}
%    \end{macrocode}

% \macro{\childdocforwardprefix}
% The command |\childdocforwardprefix| redirects
% compilation to the main or a child file by means of a pattern.
% The prefix |#1| in the current filename is replaced by |#2|
% and the suffix of the current filename is kept
% (it is assumed that the filename does not contain the substring `|~~~|'
% which is used as a delimiter).
% Compilation is handed over to the new file by |\childdocforward|:
%    \begin{macrocode}
\newcommand{\childdocforwardprefix}[3][]
{
  \begingroup
    \def\childdocextract #2##1~~~{\def\childdoctmp{\childdocforward[#1]{#3##1}}}
    \expandafter\childdocextract\childdocname~~~
    \expandafter
  \endgroup
  \childdoctmp
}
%    \end{macrocode}

% \macro{\childdoc}
% The deprecated macro |\childdoc| is a legacy version of |\childdocmain|:
%    \begin{macrocode}
\newcommand{\childdoc}{\childdocmain}
%    \end{macrocode}

% \macro{\childdocredirect}
% The deprecated macro |\childdocredirect| is a legacy version
% of |\childdocforward| and |\childdocforwardprefix|:
%    \begin{macrocode}
\newcommand{\childdocredirect}[2][]
{
  \begingroup
    \if?#1?
      \def\childdoctmp{\childdocforward{#2}}
    \else
      \def\childdoctmp{\childdocforwardprefix{#1}{#2}}
    \fi
    \expandafter
  \endgroup
  \childdoctmp
}
%    \end{macrocode}

%\iffalse
%</package>
%\fi
%
\endinput
|\\
|\childdocforward[|\textit{main}|]{|\textit{dest}|}|\\
\end{tabular}
\end{center}
%
The argument \textit{dest} is the destination file
(without extension).
It should be the main file or one of the child files.
Note that further \textsf{childdoc} directives
such as |\childdocof| and |\childdocforward|
in the indicated file will be processed in this form.
The optional argument \textit{main}
passes on directly to the main file \textit{main}
while pretending to compile the child \textit{dest}.
This form behaves as if \textit{dest}
issues |\childdocof{|\textit{main}|}| right away,
and no further \textsf{childdoc} directives will be processed.

%%%%%%%%%%%%%%%%%%%%%%%%%%%%%%%%%%%%%%%%
\DescribeMacro{\...prefix}
In the alternative form |\childdocforwardprefix|,
%
\begin{center}
\begin{tabular}{l}
|% \iffalse
%
% childdoc.dtx Copyright (C) 2017-2018 Niklas Beisert
%
% This work may be distributed and/or modified under the
% conditions of the LaTeX Project Public License, either version 1.3
% of this license or (at your option) any later version.
% The latest version of this license is in
%   http://www.latex-project.org/lppl.txt
% and version 1.3 or later is part of all distributions of LaTeX
% version 2005/12/01 or later.
%
% This work has the LPPL maintenance status `maintained'.
%
% The Current Maintainer of this work is Niklas Beisert.
%
% This work consists of the files childdoc.dtx and childdoc.ins
% and the derived files childdoc.def and cdocsamp.tex with
% cdocsch1.tex, cdocsch2.tex, cdocsdrf.tex, cdocsfn1.tex, cdocsfn2.tex.
%
%<package>\ifdefined\childdocmain\endinput\fi
%<package>\ProvidesFile{childdoc.def}[2018/12/30 v2.0 child document driver]
%<samplemain>\ProvidesFile{cdocsamp.tex}[2018/12/30 v2.0 sample for childdoc]
%<*driver>
%\ProvidesFile{childdoc.drv}[2018/12/30 v2.0 childdoc reference manual file]
\PassOptionsToClass{10pt,a4paper}{article}
\documentclass{ltxdoc}

\usepackage[margin=35mm]{geometry}
\usepackage{hyperref}
\usepackage{hyperxmp}
\usepackage[usenames]{color}

\hypersetup{colorlinks=true}
\hypersetup{pdfstartview=FitH}
\hypersetup{pdfpagemode=UseNone}
\hypersetup{pdfsource={}}
\hypersetup{pdflang={en-UK}}
\hypersetup{pdfcopyright={Copyright 2017-2018 Niklas Beisert.
  This work may be distributed and/or modified under the
  conditions of the LaTeX Project Public License, either version 1.3
  of this license or (at your option) any later version.}}
\hypersetup{pdflicenseurl={http://www.latex-project.org/lppl.txt}}
\hypersetup{pdfcontactaddress={ETH Zurich, ITP, HIT K,
  Wolfgang-Pauli-Strasse 27}}
\hypersetup{pdfcontactpostcode={8093}}
\hypersetup{pdfcontactcity={Zurich}}
\hypersetup{pdfcontactcountry={Switzerland}}
\hypersetup{pdfcontactemail={nbeisert@itp.phys.ethz.ch}}
\hypersetup{pdfcontacturl={http://people.phys.ethz.ch/\xmptilde nbeisert/}}

\newcommand{\secref}[1]{\hyperref[#1]{section \ref*{#1}}}

\parskip1ex
\parindent0pt
\let\olditemize\itemize
\def\itemize{\olditemize\parskip0pt}

\begin{document}

\title{The \textsf{childdoc} Package}
\hypersetup{pdftitle={The childdoc Package}}
\author{Niklas Beisert\\[2ex]
  Institut f\"ur Theoretische Physik\\
  Eidgen\"ossische Technische Hochschule Z\"urich\\
  Wolfgang-Pauli-Strasse 27, 8093 Z\"urich, Switzerland\\[1ex]
  \href{mailto:nbeisert@itp.phys.ethz.ch}
  {\texttt{nbeisert@itp.phys.ethz.ch}}}
\hypersetup{pdfauthor={Niklas Beisert}}
\hypersetup{pdfsubject={Manual for the LaTeX2e Package childdoc}}
\date{30 December 2018, \textsf{v2.0}}
\maketitle

\begin{abstract}\noindent
\textsf{childdoc} is a \LaTeXe{} package
that enables the direct compilation
of document sections included by |\include|
to individual files.
\end{abstract}

\begingroup
\parskip0ex
\tableofcontents
\endgroup

%%%%%%%%%%%%%%%%%%%%%%%%%%%%%%%%%%%%%%%%%%%%%%%%%%%%%%%%%%%%%%%%%%%%%%%%%%%%%%%%
%%%%%%%%%%%%%%%%%%%%%%%%%%%%%%%%%%%%%%%%%%%%%%%%%%%%%%%%%%%%%%%%%%%%%%%%%%%%%%%%
\section{Introduction}

\LaTeX{} provides a mechanism to structure a large document (such as a book)
into a main file and several child files (containing the chapters)
using the |\include| command.
This mechanism is beneficial for documents
which span hundreds of pages in order to
make the source file(s) more manageable.
Moreover, compilation can be restricted to
selected child files by means of the |\includeonly| command.
The latter feature can be used to reduce the compilation time while editing
(this was significantly more useful in the earlier days of \LaTeX{})
or to generate a smaller document which is easier to navigate.
Another application of |\includeonly| is to generate
documents consisting of selected parts of the complete document.

However, there are a few drawbacks of the plain |\include| mechanism:
\begin{itemize}
\item
The child files cannot be compiled on their own,
they can only be compiled via the main file.
A naive editing environment
(such as a text editor with an option
to have the current file processed by \LaTeX)
may require one to switch to the main file before compiling;
attempting to compile the child file produces errors.
\item
The main file must be modified (each time)
to adjust the |\includeonly| command
to the present needs. This easily leaves the main file in a messy state.
\item
The generated document will always carry the filename
of the main document. This is inconvenient if
several child files are to be compiled and
to be kept for distribution.
\end{itemize}

The present package provides a simple interface
to make child files individually compilable by \LaTeX{}.
Compiling a child file then has the same effect as compiling
the main file with an |\includeonly| command
to select the appropriate child.
Moreover the generated document will carry the name of the child
rather than the main file.
This resolves all three above issues.

This feature is meant to make the editing of books,
thesis documents and lecture notes somewhat more convenient.
However, the package can also be used efficiently for
composing a series of documents (such as exercise sheets)
which are typically distributed individually.
It then assists the author in generating the individual documents
(potentially in different versions)
as well as a document containing the collected series.
Another application is in developing style files
or other kinds of included material
where compilation of the style file could redirect
to a sample or test file.

%%%%%%%%%%%%%%%%%%%%%%%%%%%%%%%%%%%%%%%%%%%%%%%%%%%%%%%%%%%%%%%%%%%%%%%%%%%%%%%%
%%%%%%%%%%%%%%%%%%%%%%%%%%%%%%%%%%%%%%%%%%%%%%%%%%%%%%%%%%%%%%%%%%%%%%%%%%%%%%%%
\section{Usage}

First of all, the package \textsf{childdoc} is \emph{not} a standard
\LaTeXe{} |.sty| style file! Therefore it needs to be invoked in
a non-standard way.

%%%%%%%%%%%%%%%%%%%%%%%%%%%%%%%%%%%%%%%%%%%%%%%%%%%%%%%%%%%%%%%%%%%%%%%%%%%%%%%%
\subsection{Included Files}
\label{sec:include}

%%%%%%%%%%%%%%%%%%%%%%%%%%%%%%%%%%%%%%%%
\DescribeMacro{\childdocmain}
To use the package, add the commands
\begin{center}
\begin{tabular}{l}
|\input{childdoc.def}|\\
|\childdocmain{}|\\
\end{tabular}
\end{center}
at the very top of the main \LaTeX{} file,
in particular \emph{before} the |\documentclass| statement!
The argument of |\childdocmain| should be left empty
(but it must be present).

%%%%%%%%%%%%%%%%%%%%%%%%%%%%%%%%%%%%%%%%
\DescribeMacro{\childdocof}
Furthermore, add the commands
\begin{center}
\begin{tabular}{l}
|\input{childdoc.def}|\\
|\childdocof{|\textit{main}|}|\\
\end{tabular}
\end{center}
at the top of every child file \textit{child}
which is included by |\include{|\textit{child}|}|
from within the main file
(or at least for those files to be compiled individually).
The argument \textit{main} must be the filename of the main file.

There are a couple of
considerations in setting up the main and child documents:

%%%%%%%%%%%%%%%%%%%%%%%%%%%%%%%%%%%%%%%%
\paragraph{Restrictions.}

Please note the following restrictions:
\begin{itemize}
\item
|\childdocmain| must be called with one argument \textit{main}
to ensure compatibility with earlier version of the package.
It must either be empty (|\childdocmain{}|)
or precisely match the filename of the main file in which it is specified.
See \secref{sec:detection} for further information.
\item
The filename \textit{main} must be specified without the |.tex| extension.
\item
The filename \textit{main} is case sensitive
(even in case-insensitive file systems)
due to internal string comparison.
\item
The argument \textit{main} should be fully expanded, it cannot be a macro.
\item
Subdirectories and special characters should be avoided in filenames.
\item
The command |\childdocmain{|\textit{main}|}| must be followed by a whitespace.
It should not be followed immediately by another command
or by a comment mark `|%|'.
This is because the \TeX{} parser reads the token immediately following
the argument of |\childdocmain| and puts it
at the beginning of every child section;
however, a white\-space is ignored.
\end{itemize}

%%%%%%%%%%%%%%%%%%%%%%%%%%%%%%%%%%%%%%%%
\paragraph{Content of Main File.}

It is advisable to place all content in the child files included by |\include|.
Any output contained in the main file will appear in all child documents
unless suppressed manually;
it cannot be suppressed automatically by the |\includeonly| directive
and thus should normally be avoided.
A method to include some content in the main file
by means of conditional processing is described in \secref{sec:conditional}.

%%%%%%%%%%%%%%%%%%%%%%%%%%%%%%%%%%%%%%%%
\paragraph{Page Numbering.}

When only a part of the document is compiled,
the appropriate numbering of pages
(as well as other status parameters)
is determined from the |.aux| files.
The latter contain information from previous passes.
However this information needs to propagate through
all intermediate child documents.
Therefore the page numbering in child documents may well
be inconsistent until the complete document is compiled at least once.

A useful (if unconventional) way to always ensure a consistent
page numbering is to restart the numbering in each child document
and denote the pages by `\textit{child}|.|\textit{page}'
where \textit{child} represents the chapter/section number of the child file.
This can be achieved by the command
|\numberwithin{page}{|\textit{child}|}|
of the \textsf{amsmath} package
where \textit{child} can be |chapter| or |section|
depending on the chosen structuring.
Alternatively, one can modify the macro |\thepage| appropriately
and reset the counter |page| at the start of each child file.

%%%%%%%%%%%%%%%%%%%%%%%%%%%%%%%%%%%%%%%%%%%%%%%%%%%%%%%%%%%%%%%%%%%%%%%%%%%%%%%%
\subsection{Conditional Processing}
\label{sec:conditional}

The package provides a mechanism to compile different versions
of a document. To customise the versions further some conditional processing
can come in handy to distinguish which version is being compiled.
The package provides two macros to describe the compilation context:

%%%%%%%%%%%%%%%%%%%%%%%%%%%%%%%%%%%%%%%%
\DescribeMacro{\ifchilddoc}
The conditional |\ifchilddoc| distinguishes between the compilation of
child documents and the main document:
%
\begin{center}
|\ifchilddoc |\textit{child-code}| |[|\||else |\textit{main-code}]| \||fi|
\end{center}

%%%%%%%%%%%%%%%%%%%%%%%%%%%%%%%%%%%%%%%%
\DescribeMacro{\childdocname}
\DescribeMacro{\childdocjob}
The macro |\childdocname| contains the filename (without extension)
of the main or child file being processed.
Note that |\childdocjob| will always contain the name of the main file.

%%%%%%%%%%%%%%%%%%%%%%%%%%%%%%%%%%%%%%%%
\paragraph{Title Page.}

Conditional processing can be used to include a title or banner page
in the main document when proper precautions are taken.
Importantly, the code in the main file should ensure that the page counter
(as well as other status parameters which are stored in the |.aux| files)
takes the same value after the conditional processing.
Otherwise the page numbers may take divergent values
depending on which part is compiled.

For example, a title page could be declared by:
%
\begin{center}
\begin{tabular}{l}
|\ifchilddoc\||else|\\
|\addtocounter{page}{-1}|\\
\textit{code for title page}\\
|\newpage|\\
|\||fi|
\end{tabular}
\end{center}
%
A banner page for the child documents can be generated by:
%
\begin{center}
\begin{tabular}{l}
|\ifchilddoc|\\
|\addtocounter{page}{-1}|\\
\textit{code for banner page}\\
|\newpage|\\
|\||fi|
\end{tabular}
\end{center}
%
Here one could write a message such as:
\begin{center}
|This is the part \childdocname{} of \childdocjob{}.|
\end{center}

%%%%%%%%%%%%%%%%%%%%%%%%%%%%%%%%%%%%%%%%%%%%%%%%%%%%%%%%%%%%%%%%%%%%%%%%%%%%%%%%
\subsection{Flags}
\label{sec:flags}

The package makes it easy to generate different versions
of the main or child documents.
To this end compilation flags can be defined
and assigned different default values.
They will be particularly useful in conjunction
with the forwarding mechanism described in \secref{sec:forward}.

For example, it may be useful to have a flag |\version|
which can be set to |draft| or |final|.
The document source will contain some conditional code
depending on the value of |\version|.
Suppose further, the flag should default to |final| for the main file
and to |draft| for child files
which is a natural assignment for editing the document.
This is achieved by placing the following code
in the preamble of the main document
(below the |\childdocmain| directive):
%
\begin{center}
\begin{tabular}{l}
|\ifchilddoc|\\
|\providecommand{\version}{draft}|\\
|\||else|\\
|\providecommand{\version}{final}|\\
|\||fi|
\end{tabular}
\end{center}
%
The definition by |\providecommand| makes sure
that previous definitions are not overwritten.
Further statements |\providecommand{\version}{...}|
can thus be added before the above code to override it.

For the main file, one might add a line
(between |\childdocmain| and the above block)
%
\begin{center}
|%\ifchilddoc\||else\providecommand{\version}{draft}\||fi|
\end{center}
%
which can be uncommented to produce a draft version.
Likewise one can add a line to the very top of a child file
(above the |\childdocof{|\textit{main}|}| directive)
%
\begin{center}
|%\providecommand{\version}{final}|
\end{center}
%
which can be uncommented to produce the final version of this child document.

%%%%%%%%%%%%%%%%%%%%%%%%%%%%%%%%%%%%%%%%%%%%%%%%%%%%%%%%%%%%%%%%%%%%%%%%%%%%%%%%
\subsection{Forwarding}
\label{sec:forward}

Different versions of the main or child documents
using compilation flags as described in \secref{sec:flags}
can be (permanently) stored in different files
for convenient compilation, viewing and distribution.
To this end, the package defines a command
to pass on compilation to a different file:

%%%%%%%%%%%%%%%%%%%%%%%%%%%%%%%%%%%%%%%%
\DescribeMacro{\childdocforward}
The command |\childdocforward| redirects processing to
another source file:
%
\begin{center}
\begin{tabular}{l}
|\input{childdoc.def}|\\
|\childdocforward[|\textit{main}|]{|\textit{dest}|}|\\
\end{tabular}
\end{center}
%
The argument \textit{dest} is the destination file
(without extension).
It should be the main file or one of the child files.
Note that further \textsf{childdoc} directives
such as |\childdocof| and |\childdocforward|
in the indicated file will be processed in this form.
The optional argument \textit{main}
passes on directly to the main file \textit{main}
while pretending to compile the child \textit{dest}.
This form behaves as if \textit{dest}
issues |\childdocof{|\textit{main}|}| right away,
and no further \textsf{childdoc} directives will be processed.

%%%%%%%%%%%%%%%%%%%%%%%%%%%%%%%%%%%%%%%%
\DescribeMacro{\...prefix}
In the alternative form |\childdocforwardprefix|,
%
\begin{center}
\begin{tabular}{l}
|\input{childdoc.def}|\\
|\childdocforwardprefix[|\textit{main}|]{|\textit{prefix}|}{|\textit{dest}|}|
\end{tabular}
\end{center}
%
the destination file is determined by a pattern
depending on the current file:
To make this work, the current file must be called
`{\textit{prefix}\hspace{0.2em}\textit{suffix}}'
with \textit{prefix} matching precisely the argument.
Processing is then passed on to the file
`{\textit{dest}\hspace{0.2em}\textit{suffix}}'.
Surely, the same effect is achieved by
directly specifying the
argument `{\textit{dest}\hspace{0.2em}\textit{suffix}}'
in the first form.
However, that requires to set up a different file
for each child. With the alternative form of the command
all these files can have exactly the same content
which simplifies setting them up and maintaining them.

For example, the following file |draft.tex|
with a compilation flag |\version| as described in \secref{sec:flags}
compiles the main document as a draft:
%
\begin{center}
\begin{tabular}{l}
|\def\version{draft}|\\
|\input{childdoc.def}|\\
|\childdocforward{|\textit{main}|}|
\end{tabular}
\end{center}
%
Likewise, the following files |final|\textit{nn}|.tex|
compile the final version of the child document
|child|\textit{nn}|.tex|:
%
\begin{center}
\begin{tabular}{l}
|\def\version{final}|\\
|\input{childdoc.def}|\\
|\childdocforwardprefix{final}{child}|
\end{tabular}
\end{center}
%

Note that when several versions of a main file and/or of each child file
are to be generated, it may be convenient to set up a |Makefile| or
shell script to automatise the process.

%%%%%%%%%%%%%%%%%%%%%%%%%%%%%%%%%%%%%%%%%%%%%%%%%%%%%%%%%%%%%%%%%%%%%%%%%%%%%%%%
\subsection{Command Line Processing}
\label{sec:commandline}

The effect of redirection files can also be achieved by invoking
the \LaTeX{} compiler with a more elaborate command line.
Most conveniently this should be done as part
of a shell script or a |Makefile|.

When using \textsf{childdoc} in the main file, the following
command lines effectively perform a redirection
(note that depending on the shell being used,
backslashes may have to be doubled: `|\|' $\to$ `|\\|'):
%
\begin{center}
|... -jobname "|\textit{target}|" |\\|"|[\textit{flags}]%
|\input{childdoc.def}\childdocforward[|\textit{main}|]{|\textit{dest}|}"|
\end{center}
%
Here \textit{target} is the name of the output file,
\textit{main} is the name of the main file
and \textit{dest} is the name of the main or child file to be processed
(all filenames without extensions).
The optional argument \textit{main} can be omitted
if \textit{main} matches \textit{dest}.
Optionally, compilation \textit{flags} can be defined via |\def| commands.
This command line makes the \TeX{} engine believe
it is compiling the file \textit{target}
whose content is specified as the latter parameter.
The provided code then forwards the processing to
\textit{main} or \textit{dest} as described in \secref{sec:forward}.

%%%%%%%%%%%%%%%%%%%%%%%%%%%%%%%%%%%%%%%%%%%%%%%%%%%%%%%%%%%%%%%%%%%%%%%%%%%%%%%%
\subsection{Include by Input}
\label{sec:input}

Including child documents by |\include| has some restrictions by design.
Most notably, the content of a child document always occupies
its own set of pages; pages cannot be shared between child documents.
Usually, this behaviour makes perfect sense
because each child document contain an essential part of the document.
However, in some situations it may be desirable to compose
a document from a collection of parts
without having mandatory page breaks between then.
For this case, the package
provides a mechanism to include parts
by |\input| which can also be processed individually.
However, by construction this mechanism
requires manual handling of the content to be output.

%%%%%%%%%%%%%%%%%%%%%%%%%%%%%%%%%%%%%%%%
\DescribeMacro{\ifchilddocmanual}
The main file should be prepared as usual, see \secref{sec:include}.
However, the document body must make a distinction
between processing of an individual part and of the main document, e.g.:
%
\begin{center}
\begin{tabular}{l}
|\ifchilddocmanual|\\
|\input{\childdocname}|\\
|\||else|\\
\textit{document body with }|\input{|\textit{part}|}|\\
|\||fi|
\end{tabular}
\end{center}
%
The conditional |\ifchilddocmanual| is true whenever
a part to be included by |\input| is being compiled,
and the name of the part is stored in |\childdocname|.

%%%%%%%%%%%%%%%%%%%%%%%%%%%%%%%%%%%%%%%%
\DescribeMacro{\childdocby}
Each part to be included by |\input| should start with:
%
\begin{center}
\begin{tabular}{l}
|\input{childdoc.def}|\\
|\childdocby{|\textit{main}|}|\\
\end{tabular}
\end{center}
%
The directive |\childdocby| is similar to |\childdocof|
described in \secref{sec:include},
but the subsequent selection of content must be done manually.
To that end, both |\ifchilddoc| and |\ifchilddocmanual|
will be true upon processing of a part,
and the name of the part is stored in |\childdocname|.
Note that |\jobname| will be set to the filename of the current part
so that each part receives an individual |.aux| file
that does not interfere with the |.aux| file(s) of the main document.
This behaviour can be altered by the alternative form
|\childdocby[*]{|\textit{main}|}| (with a non-empty optional argument)
which uses the |.aux| file of the main document
by setting |\jobname| to \textit{main}.

%%%%%%%%%%%%%%%%%%%%%%%%%%%%%%%%%%%%%%%%%%%%%%%%%%%%%%%%%%%%%%%%%%%%%%%%%%%%%%%%
\subsection{Driver Development}
\label{sec:driver}

The \textsf{childdoc} mechanism can also be use for the development
of definition files such as \LaTeX{} styles or classes.
This case differs from the above setup with multiple parts
included by |\include| in that no |\includeonly| should be invoked.
This can be achieved by starting the include file
(before |\ProvidesPackage|) with:
%
\begin{center}
\begin{tabular}{l}
|\input{childdoc.def}|\\
|\childdocforward{|\textit{main}|}|\\
\end{tabular}
\end{center}
%
or alternatively with:
%
\begin{center}
\begin{tabular}{l}
|\input{childdoc.def}|\\
|\childdocby{|\textit{main}|}|\\
\end{tabular}
\end{center}
%
Both forms have slightly different effects as described above.
The main file is prepared as usual, see \secref{sec:include}.

%%%%%%%%%%%%%%%%%%%%%%%%%%%%%%%%%%%%%%%%%%%%%%%%%%%%%%%%%%%%%%%%%%%%%%%%%%%%%%%%
\subsection{Legacy Detection}
\label{sec:detection}

The directive |\childdocmain| in the main file can detect
whether the complete document or merely a child is to be compiled
even without using the directive |\childdocof|.
This method is deprecated because it is less robust
and there is no compelling reason to use it;
it is merely provided for backward compatibility
and it may be removed in future versions.

If the detection mechanism is to be used,
it is mandatory to correctly specify
the filename of the main file as the argument of |\childdocmain|:
%
\begin{center}
\begin{tabular}{l}
|\input{childdoc.def}|\\
|\childdocmain{|\textit{main}|}|\\
\end{tabular}
\end{center}
%
If |\jobname| does not match the argument \textit{main} of |\childdocmain|,
it is assumed that |\jobname| points to the child file to be compiled.
When using |\childdocmain| with the main file specified as argument,
it suffices to start a child file
with just |\input{|\textit{main}|}|
without loading of the package and using |\childdocof|.
If instead all processing is done
with the appropriate \textsf{childdoc} directives,
the argument of \textit{main} of |\childdocmain| can be empty.

An alternative version of the command line processing described
in \secref{sec:commandline} using the detection mechanism reads:
%
\begin{center}
|... -jobname "|\textit{target}|" "|[\textit{flags}]%
[|\def\jobname{|\textit{dest}|}|]|\input{|\textit{main}|}"|
\end{center}

%%%%%%%%%%%%%%%%%%%%%%%%%%%%%%%%%%%%%%%%%%%%%%%%%%%%%%%%%%%%%%%%%%%%%%%%%%%%%%%%
\subsection{Manual Code}
\label{sec:manual}

In case one cannot be certain whether the definitions file |childdoc.def|
is installed on the target \TeX{} distribution
and one prefers not to ship it,
it is conceivable to paste a few relevant commands into the sources.

To that end, drop all statements |\input{childdoc.def}|
and perform the replacements as outlined below.
Instead of |\childdocmain{|\textit{main}|}| add the following code
to the top of the main file:
%
\begin{center}
\begin{tabular}{l}
|\||ifdefined\childdocname\endinput\||fi\newif\ifchilddoc|\\
|\edef\childdocname{\scantokens\expandafter{\jobname\noexpand}}|\\
|\def\childdocmain{|\textit{main}|}\||ifx\childdocmain\childdocname\||else|\\
|\childdoctrue\includeonly{\childdocname}\let\jobname\childdocmain\||fi|\\
\end{tabular}
\end{center}
%
Instead of |\childdocof{|\textit{main}|}| just include the main file
at the top of each child file:
%
\begin{center}
|\input{|\textit{main}|}|
\end{center}
%
A simple redirection |\childdocforward{|\textit{dest}|}| is achieved by:
%
\begin{center}
|\def\jobname{|\textit{dest}|}\input{\jobname}|
\end{center}
%
The redirection with prefix
|\childdocforwardprefix[|\textit{prefix}|]{|\textit{dest}|}|
is accomplished by:
%
\begin{center}
\begin{tabular}{l}
|{\edef\jobname{\scantokens\expandafter{\jobname\noexpand}}|\\
|\def\redirectjob |\textit{prefix}|#1~~~{\gdef\jobname{|\textit{dest}|#1}}|\\
|\expandafter\redirectjob\jobname~~~}\input{\jobname}|
\end{tabular}
\end{center}

In an alternative approach,
child documents can be compiled by a specific command line
without additional code or specific definitions:
%
\begin{center}
|... -jobname "|\textit{target}|" "|[\textit{flags}]%
|\includeonly{|\textit{dest}|}\input{|\textit{main}|}"|
\end{center}
%

%%%%%%%%%%%%%%%%%%%%%%%%%%%%%%%%%%%%%%%%%%%%%%%%%%%%%%%%%%%%%%%%%%%%%%%%%%%%%%%%
%%%%%%%%%%%%%%%%%%%%%%%%%%%%%%%%%%%%%%%%%%%%%%%%%%%%%%%%%%%%%%%%%%%%%%%%%%%%%%%%
\section{Information}

%%%%%%%%%%%%%%%%%%%%%%%%%%%%%%%%%%%%%%%%%%%%%%%%%%%%%%%%%%%%%%%%%%%%%%%%%%%%%%%%
\subsection{Copyright}

Copyright \copyright{} 2017--2018 Niklas Beisert

This work may be distributed and/or modified under the
conditions of the \LaTeX{} Project Public License, either version 1.3
of this license or (at your option) any later version.
The latest version of this license is in
  \url{http://www.latex-project.org/lppl.txt}
and version 1.3 or later is part of all distributions of \LaTeX{}
version 2005/12/01 or later.

This work has the LPPL maintenance status `maintained'.

The Current Maintainer of this work is Niklas Beisert.

This work consists of the files |README.txt|, |childdoc.ins| and |childdoc.dtx|
as well as the derived files |childdoc.def|, |cdocsamp.tex|
with |cdocsch1.tex|, |cdocsch2.tex|, |cdocspt3.tex|, |cdocspt4.tex|,
|cdocsdrf.tex|, |cdocsfn1.tex|, |cdocsfn2.tex|
as well as |childdoc.pdf|.

%%%%%%%%%%%%%%%%%%%%%%%%%%%%%%%%%%%%%%%%%%%%%%%%%%%%%%%%%%%%%%%%%%%%%%%%%%%%%%%%
\subsection{Files and Installation}

The package consists of the files:
%
\begin{center}
\begin{tabular}{ll}
    |README.txt|   & readme file \\
    |childdoc.ins| & installation file \\
    |childdoc.dtx| & source file \\
    |childdoc.def| & definition file \\
    |cdocsamp.tex| & sample main file \\
    |cdocsch1.tex| & sample include file \\
    |cdocsch2.tex| & sample include file \\
    |cdocspt3.tex| & sample part file \\
    |cdocspt4.tex| & sample part file \\
    |cdocsdrf.tex| & sample redirection file \\
    |cdocsfn1.tex| & sample redirection file \\
    |cdocsfn2.tex| & sample redirection file \\
    |childdoc.pdf| & manual
\end{tabular}
\end{center}
%
The distribution consists of the files
|README.txt|, |childdoc.ins| and |childdoc.dtx|.
%
\begin{itemize}
\item
Run (pdf)\LaTeX{} on |childdoc.dtx|
to compile the manual |childdoc.pdf| (this file).
\item
Run \LaTeX{} on |childdoc.ins| to create the definitions file |childdoc.def|
and the sample |cdocsamp.tex| with include files
|cdocsch1.tex|, |cdocsch2.tex|, |cdocspt3.tex|, |cdocspt4.tex|,
|cdocsdrf.tex|, |cdocsfn1.tex|, |cdocsfn2.tex|.
Then copy the file |childdoc.def| to an appropriate directory of your \LaTeX{}
distribution, e.g.\ \textit{texmf-root}|/tex/latex/childdoc|.
\end{itemize}

%%%%%%%%%%%%%%%%%%%%%%%%%%%%%%%%%%%%%%%%%%%%%%%%%%%%%%%%%%%%%%%%%%%%%%%%%%%%%%%%
\subsection{Related CTAN Packages}

There are several other packages which offer a similar functionality:
%
\begin{itemize}
\item
The packages
\href{http://ctan.org/pkg/docmute}{\textsf{docmute}},
\href{http://ctan.org/pkg/includex}{\textsf{includex}} and
\href{http://ctan.org/pkg/standalone}{\textsf{standalone}}
provide commands to include only the document body of
a child file thus allowing both files to be compiled individually.
\item
The packages \href{http://ctan.org/pkg/subdocs}{\textsf{subdocs}}
and \href{http://ctan.org/pkg/subfiles}{\textsf{subfiles}}
provide structures in which the main and child documents can be
encapsulated and allowing them to be compiled individually.
The inclusion mechanism is different from the conventional |\include|.
\item
The package \href{http://ctan.org/pkg/combine}{\textsf{combine}}
is an elaborate solution to combine several documents into one.
\end{itemize}
%
See also the CTAN topic \href{http://ctan.org/topic/subdocs}{\textsf{subdocs}}
for further related packages.
The present package differs from the above solutions in that
a document structure constructed with the conventional |\include| mechanism
just needs two extra commands at the top of every file
such that all constituent files can be compiled individually.

%%%%%%%%%%%%%%%%%%%%%%%%%%%%%%%%%%%%%%%%%%%%%%%%%%%%%%%%%%%%%%%%%%%%%%%%%%%%%%%%
%\subsection{Feature Suggestions}
%
%The following is a list of features which may be useful for future
%versions of this package:
%%
%\begin{itemize}
%\item
%\ldots
%\end{itemize}

%%%%%%%%%%%%%%%%%%%%%%%%%%%%%%%%%%%%%%%%%%%%%%%%%%%%%%%%%%%%%%%%%%%%%%%%%%%%%%%%
\subsection{Revision History}

%%%%%%%%%%%%%%%%%%%%%%%%%%%%%%%%%%%%%%%%
\paragraph{v2.0:} 2018/12/30

\begin{itemize}
\item
immediate forward processing
\item
added |\childdocby| mechanism
\item
manual restructured
\end{itemize}

%%%%%%%%%%%%%%%%%%%%%%%%%%%%%%%%%%%%%%%%
\paragraph{v1.6:} 2018/01/17

\begin{itemize}
\item
application for development of include files
\item
corrections to manual
\end{itemize}

%%%%%%%%%%%%%%%%%%%%%%%%%%%%%%%%%%%%%%%%
\paragraph{v1.5:} 2017/05/21

\begin{itemize}
\item
more complete structuring introduced
\item
|\childdocof| introduced
\item
|\childdoc| renamed to |\childdocmain|
\item
|\childredirect| renamed to |\childdocforward| and |\childdocforwardprefix|
and functionality expanded
\end{itemize}

%%%%%%%%%%%%%%%%%%%%%%%%%%%%%%%%%%%%%%%%
\paragraph{v1.0:} 2017/04/27

\begin{itemize}
\item
manual and install package
\item
first version published on CTAN
\end{itemize}

%%%%%%%%%%%%%%%%%%%%%%%%%%%%%%%%%%%%%%%%
\paragraph{v0.6:} 2017/04/26

\begin{itemize}
\item
redirection mechanism added
\end{itemize}

%%%%%%%%%%%%%%%%%%%%%%%%%%%%%%%%%%%%%%%%
\paragraph{v0.5:} 2017/04/26

\begin{itemize}
\item
functionality in definition file
\end{itemize}


%%%%%%%%%%%%%%%%%%%%%%%%%%%%%%%%%%%%%%%%%%%%%%%%%%%%%%%%%%%%%%%%%%%%%%%%%%%%%%%%
%%%%%%%%%%%%%%%%%%%%%%%%%%%%%%%%%%%%%%%%%%%%%%%%%%%%%%%%%%%%%%%%%%%%%%%%%%%%%%%%
%%%%%%%%%%%%%%%%%%%%%%%%%%%%%%%%%%%%%%%%%%%%%%%%%%%%%%%%%%%%%%%%%%%%%%%%%%%%%%%%
\appendix

\settowidth\MacroIndent{\rmfamily\scriptsize 000\ }

 \DocInput{childdoc.dtx}

\end{document}
%</driver>
% \fi
%
% %%%%%%%%%%%%%%%%%%%%%%%%%%%%%%%%%%%%%%%%%%%%%%%%%%%%%%%%%%%%%%%%%%%%%%%%%%%%%%
% %%%%%%%%%%%%%%%%%%%%%%%%%%%%%%%%%%%%%%%%%%%%%%%%%%%%%%%%%%%%%%%%%%%%%%%%%%%%%%
% \section{Sample}
%\iffalse
%<*samplemain>
%\fi
%
% The following presents a sample document
% with two chapters, two parts, a title page,
% a compile flag as well as three forwarding files to set the flag.
% It consists of eight |.tex| files:
% \begin{center}
% \begin{tabular}{ll}
% |cdocsamp.tex|&main file\\
% |cdocsch1.tex|&include file for chapter 1\\
% |cdocsch2.tex|&include file for chapter 2\\
% |cdocspt3.tex|&include file for part 3\\
% |cdocspt4.tex|&include file for part 4\\
% |cdocsdrf.tex|&forwarding file for main file in draft mode\\
% |cdocsfi1.tex|&forwarding file for final version of chapter 1\\
% |cdocsfi2.tex|&forwarding file for final version of chapter 2\\
% \end{tabular}
% \end{center}
% Each of the eight files can be compiled directly by the \LaTeX{} compiler.
%
% %%%%%%%%%%%%%%%%%%%%%%%%%%%%%%%%%%%%%%
% \paragraph{Main File.}
%
% The main file is called |cdocsamp.tex|.
%
% Load the \textsf{childdoc} definitions and
% declare the filename for the main document:
%    \begin{macrocode}
\input{childdoc.def}
\childdocmain{}
%    \end{macrocode}

% Optional override for |\version| flag:
%    \begin{macrocode}
%%\ifchilddoc\else\providecommand{\version}{draft}\fi
%    \end{macrocode}

% Define the default values for the |\version| flag
% (|final| for the main file and |draft| for childs):
%    \begin{macrocode}
\ifchilddoc
\providecommand{\version}{draft}
\else
\providecommand{\version}{final}
\fi
%    \end{macrocode}

% Load the standard document class:
%    \begin{macrocode}
\documentclass[12pt]{article}
%    \end{macrocode}

% Start the document body:
%    \begin{macrocode}
\begin{document}
%    \end{macrocode}

% Declare a title page.
% Print title, part of document being processed and version flag:
%    \begin{macrocode}
\addtocounter{page}{-1}
\begin{center}
{\LARGE\bfseries{}childdoc example\par}
\vspace{1cm}
\ifchilddoc
\ifchilddocmanual part\else chapter\fi:
`\childdocname' of `\childdocjob'\par
\else
main document: `\childdocjob'\par
\fi
version: \version\par
\end{center}
\newpage
%    \end{macrocode}

% Manually include selected file,
% otherwise process as usual:
%    \begin{macrocode}
\ifchilddocmanual
\section*{part `\childdocname'}
\input{\childdocname}
\else
%    \end{macrocode}

% Include the two chapters:
%    \begin{macrocode}
\include{cdocsch1}
\include{cdocsch2}
%    \end{macrocode}

% Include the two parts unless only chapters should be displayed:
%    \begin{macrocode}
\ifchilddoc\else
\section{part three}
\input{cdocspt3}
\section{part four}
\input{cdocspt4}
\fi
%    \end{macrocode}

% Process as usual until here:
%    \begin{macrocode}
\fi
%    \end{macrocode}

% End of document body:
%    \begin{macrocode}
\end{document}
%    \end{macrocode}
%\iffalse
%</samplemain>
%\fi
%
% %%%%%%%%%%%%%%%%%%%%%%%%%%%%%%%%%%%%%%
% \paragraph{Chapter Include Files.}
%
% The include files are called |cdocsch1.tex| and |cdocsch2.tex|.
%
%\iffalse
%<*samplechap1|samplechap2>
%\fi

% Optional override for |\version| flag:
%    \begin{macrocode}
%%\providecommand{\version}{final}
%    \end{macrocode}

% Include the main document:
%    \begin{macrocode}
\input{childdoc.def}
\childdocof{cdocsamp}
%    \end{macrocode}

%\iffalse
%</samplechap1|samplechap2>
%\fi
%
%\iffalse
%<*samplechap1>
%\fi
% Some text for chapter 1:
%    \begin{macrocode}
\section{one}
some text in chapter one
%    \end{macrocode}

%\iffalse
%</samplechap1>
%\fi
% Some text for chapter 2:
%\iffalse
%<*samplechap2>
%\fi
%    \begin{macrocode}
\section{two}
more text in chapter two
%    \end{macrocode}

%\iffalse
%</samplechap2>
%\fi
%
% %%%%%%%%%%%%%%%%%%%%%%%%%%%%%%%%%%%%%%
% \paragraph{Part Include Files.}
%
% The include files are called |cdocspt3.tex| and |cdocspt4.tex|.
%
%\iffalse
%<*samplepart3|samplepart4>
%\fi

% Optional override for |\version| flag:
%    \begin{macrocode}
%%\providecommand{\version}{final}
%    \end{macrocode}

% Include the main document:
%    \begin{macrocode}
\input{childdoc.def}
\childdocby{cdocsamp}
%    \end{macrocode}

%\iffalse
%</samplepart3|samplepart4>
%\fi
%
%\iffalse
%<*samplepart3>
%\fi
% Some text for part 3:
%    \begin{macrocode}
some text in part three
%    \end{macrocode}

%\iffalse
%</samplepart3>
%\fi
% Some text for part 4:
%\iffalse
%<*samplepart4>
%\fi
%    \begin{macrocode}
more text in part four
%    \end{macrocode}

%\iffalse
%</samplepart4>
%\fi
%
% %%%%%%%%%%%%%%%%%%%%%%%%%%%%%%%%%%%%%%
% \paragraph{Forwarding for a Complete Draft.}
%
% The following forwarding file |cdocsdrf.tex|
% compiles the main document in draft mode:
%\iffalse
%<*sampledraft>
%\fi
%    \begin{macrocode}
\def\version{draft}
\input{childdoc.def}
\childdocforward{cdocsamp}
%    \end{macrocode}

%\iffalse
%</sampledraft>
%\fi
%
% %%%%%%%%%%%%%%%%%%%%%%%%%%%%%%%%%%%%%%
% \paragraph{Forwarding for Final Version of the Chapters.}
%
% The following forwarding files |cdocsfn1.tex| and |cdocsfn2.tex|
% (with identical content)
% compile the final versions of the child documents
% |cdocsch1.tex| and |cdocsch2.tex|, respectively:
%\iffalse
%<*samplefinal>
%\fi
%    \begin{macrocode}
\def\version{final}
\input{childdoc.def}
\childdocforwardprefix[cdocsamp]{cdocsfn}{cdocsch}
%    \end{macrocode}

%\iffalse
%</samplefinal>
%\fi
%
% %%%%%%%%%%%%%%%%%%%%%%%%%%%%%%%%%%%%%%
% \paragraph{Command Line Processing.}
%
% The following three command lines generate the output files
% |cdocscld|, |cdocscl1| and |cdocscl2|
% which should be identical to
% |cdocsdrf|, |cdocsch1| and |cdocsfn2|, respectively:
% \begin{center}
% \begin{tabular}{l}
% |latex -jobname cdocscld \|\\
% |  "\def\version{draft}\input{childdoc.def}\childdocforward{cdocsamp}"|\\
% |latex -jobname cdocscl1 \|\\
% |  "\input{childdoc.def}\childdocforward[cdocsamp]{cdocsch1}"|\\
% |latex -jobname cdocscl2 \|\\
% |  "\def\version{final}\input{childdoc.def}\childdocforward{cdocsch2}"|
% \end{tabular}
% \end{center}
% Note that the trailing backslash on each first line
% merely continues the input to the second line
% (for convenient cut ant paste).
% Furthermore, the command |latex| can be replaced by any
% of its alternative versions such as |pdflatex|.
%
% %%%%%%%%%%%%%%%%%%%%%%%%%%%%%%%%%%%%%%%%%%%%%%%%%%%%%%%%%%%%%%%%%%%%%%%%%%%%%%
% %%%%%%%%%%%%%%%%%%%%%%%%%%%%%%%%%%%%%%%%%%%%%%%%%%%%%%%%%%%%%%%%%%%%%%%%%%%%%%
% \section{Implementation}
%\iffalse
%<*package>
%\fi
%
% This section describes the definitions file |childdoc.def|.

% The definitions cannot be loaded using |\usepackage| or |\RequirePackage|
% which has a mechanism to prevent loading a style file more than once.
% When loading the definitions by means of |\input|
% multiple instances have to be prevented manually:
%\iffalse
%This code needs to be before the `\ProvidesFile' directive
%which is defined at the beginning of this file.
%Therefore it is also placed there and commented out here.
%</package>
%<*discard>
%\fi
%    \begin{macrocode}
\ifdefined\childdocmain\endinput\fi
%    \end{macrocode}
%\iffalse
%</discard>
%<*package>
%\fi
%
% \macro{\ifchilddoc}
% \macro{\ifchilddocmanual}
% The conditional |\ifchilddoc| tells whether a
% child (true) or main (false) document is being compiled.
% The conditional |\ifchilddocmanual| tells whether
% the |\includeonly| mechanism is used (false) or
% the selection of child files must be performed manually (true).
% The definitions initialise to false:
%    \begin{macrocode}
\newif\ifchilddoc
\newif\ifchilddocmanual
%    \end{macrocode}

% \macro{\childdocname}
% \macro{\childdocjob}
% The macro |\childdocname| stores the name of the main document
% to be compiled. The macro |\childdocjob| stores the name of
% the document on which the \LaTeX{} compiler was originally invoked.
% The content of |\jobname| cannot be compared
% to filenames specified in the source due to different catcodes.
% The following code rescans |\jobname|, stores the result
% in |\childdocname| and saves a copy in |\childdocjob|:
%    \begin{macrocode}
\edef\childdocname{\scantokens\expandafter{\jobname\noexpand}}
\let\childdocjob\childdocname
%    \end{macrocode}

% \macro{\childdocdisable}
% The macro |\childdocdisable| prevents the main file
% from being processed more than once.
% At this stage, the main document command |\childdocmain|
% is assumed to be called once again where it should do nothing.
% Any subsequent call to it should prevent
% a secondary processing of the main document
% It overwrites the forwarding commands
% |\childdocof| and |\childdocforward|
% with empty macros to prevent further inclusions of the main document:
%    \begin{macrocode}
\newcommand{\childdocdisable}
{
  \renewcommand{\childdocmain}[1]{\renewcommand{\childdocmain}[1]{\endinput}}
  \renewcommand{\childdocof}[1]{}
  \renewcommand{\childdocby}[2][]{}
  \renewcommand{\childdocforward}[2][]{}
  \renewcommand{\childdocdisable}{}
}
%    \end{macrocode}

% \macro{\childdocmain}
% The macro |\childdocmain| is to be called at the top of the main file
% with nothing or the main filename (without extension) as argument.
% First, it breaks loops.
% If the argument is not empty and does not match |\childdocname|
% (which is set by the first inclusion of |childdoc.def|),
% |\ifchilddoc| is set to true, |\includeonly| is applied to the child file
% and |\jobname| is set to the main file
% (for proper handling of |.aux| files):
%    \begin{macrocode}
\newcommand{\childdocmain}[1]
{
  \childdocdisable\childdocmain{}
  \if?#1?\else
    \begingroup
      \def\childdoctmp{#1}
      \ifx\childdoctmp\childdocname
        \def\childdoctmp{}
      \else
        \def\childdoctmp
        {
          \childdoctrue
          \includeonly{\childdocname}
          \def\childdocjob{#1}
          \def\jobname{#1}
        }
      \fi
      \expandafter
    \endgroup
    \childdoctmp
  \fi
}
%    \end{macrocode}

% \macro{\childdocof}
% The command |\childdocof| redirects
% compilation to the main file |#1|.
%    \begin{macrocode}
\newcommand{\childdocof}[1]
{
  \childdocdisable
  \childdoctrue
  \includeonly{\childdocname}
  \def\jobname{#1}
  \def\childdocjob{#1}
  \input{#1}
}
%    \end{macrocode}

% \macro{\childdocby}
% The command |\childdocby| ....
%    \begin{macrocode}
\newcommand{\childdocby}[2][]
{
  \childdocdisable
  \childdoctrue
  \childdocmanualtrue
  \if?#1?\else
    \def\jobname{#2}
  \fi
  \def\childdocjob{#2}
  \input{#2}
  \endinput
}
%    \end{macrocode}

% \macro{\childdocforward}
% The command |\childdocforward| redirects
% compilation to the main file or
% (if the optional argument is given) a child file.
% Parameters are set as if the main file
% or a child file starting with |\childdocof| was compiled.
% Then compilation is handed over to the main file:
%    \begin{macrocode}
\newcommand{\childdocforward}[2][]
{
  \begingroup
    \if?#1?
      \def\childdoctmp
      {
        \def\childdocname{#2}
        \def\childdocjob{#2}
        \def\jobname{#2}
        \input{#2}
        \endinput
      }
    \else
      \def\childdoctmp
      {
        \childdocdisable
        \def\childdocname{#2}
        \childdoctrue
        \includeonly{#2}
        \def\childdocjob{#1}
        \def\jobname{#1}
        \input{#1}
        \endinput
      }
    \fi
    \expandafter
  \endgroup
  \childdoctmp
}
%    \end{macrocode}

% \macro{\childdocforwardprefix}
% The command |\childdocforwardprefix| redirects
% compilation to the main or a child file by means of a pattern.
% The prefix |#1| in the current filename is replaced by |#2|
% and the suffix of the current filename is kept
% (it is assumed that the filename does not contain the substring `|~~~|'
% which is used as a delimiter).
% Compilation is handed over to the new file by |\childdocforward|:
%    \begin{macrocode}
\newcommand{\childdocforwardprefix}[3][]
{
  \begingroup
    \def\childdocextract #2##1~~~{\def\childdoctmp{\childdocforward[#1]{#3##1}}}
    \expandafter\childdocextract\childdocname~~~
    \expandafter
  \endgroup
  \childdoctmp
}
%    \end{macrocode}

% \macro{\childdoc}
% The deprecated macro |\childdoc| is a legacy version of |\childdocmain|:
%    \begin{macrocode}
\newcommand{\childdoc}{\childdocmain}
%    \end{macrocode}

% \macro{\childdocredirect}
% The deprecated macro |\childdocredirect| is a legacy version
% of |\childdocforward| and |\childdocforwardprefix|:
%    \begin{macrocode}
\newcommand{\childdocredirect}[2][]
{
  \begingroup
    \if?#1?
      \def\childdoctmp{\childdocforward{#2}}
    \else
      \def\childdoctmp{\childdocforwardprefix{#1}{#2}}
    \fi
    \expandafter
  \endgroup
  \childdoctmp
}
%    \end{macrocode}

%\iffalse
%</package>
%\fi
%
\endinput
|\\
|\childdocforwardprefix[|\textit{main}|]{|\textit{prefix}|}{|\textit{dest}|}|
\end{tabular}
\end{center}
%
the destination file is determined by a pattern
depending on the current file:
To make this work, the current file must be called
`{\textit{prefix}\hspace{0.2em}\textit{suffix}}'
with \textit{prefix} matching precisely the argument.
Processing is then passed on to the file
`{\textit{dest}\hspace{0.2em}\textit{suffix}}'.
Surely, the same effect is achieved by
directly specifying the
argument `{\textit{dest}\hspace{0.2em}\textit{suffix}}'
in the first form.
However, that requires to set up a different file
for each child. With the alternative form of the command
all these files can have exactly the same content
which simplifies setting them up and maintaining them.

For example, the following file |draft.tex|
with a compilation flag |\version| as described in \secref{sec:flags}
compiles the main document as a draft:
%
\begin{center}
\begin{tabular}{l}
|\def\version{draft}|\\
|% \iffalse
%
% childdoc.dtx Copyright (C) 2017-2018 Niklas Beisert
%
% This work may be distributed and/or modified under the
% conditions of the LaTeX Project Public License, either version 1.3
% of this license or (at your option) any later version.
% The latest version of this license is in
%   http://www.latex-project.org/lppl.txt
% and version 1.3 or later is part of all distributions of LaTeX
% version 2005/12/01 or later.
%
% This work has the LPPL maintenance status `maintained'.
%
% The Current Maintainer of this work is Niklas Beisert.
%
% This work consists of the files childdoc.dtx and childdoc.ins
% and the derived files childdoc.def and cdocsamp.tex with
% cdocsch1.tex, cdocsch2.tex, cdocsdrf.tex, cdocsfn1.tex, cdocsfn2.tex.
%
%<package>\ifdefined\childdocmain\endinput\fi
%<package>\ProvidesFile{childdoc.def}[2018/12/30 v2.0 child document driver]
%<samplemain>\ProvidesFile{cdocsamp.tex}[2018/12/30 v2.0 sample for childdoc]
%<*driver>
%\ProvidesFile{childdoc.drv}[2018/12/30 v2.0 childdoc reference manual file]
\PassOptionsToClass{10pt,a4paper}{article}
\documentclass{ltxdoc}

\usepackage[margin=35mm]{geometry}
\usepackage{hyperref}
\usepackage{hyperxmp}
\usepackage[usenames]{color}

\hypersetup{colorlinks=true}
\hypersetup{pdfstartview=FitH}
\hypersetup{pdfpagemode=UseNone}
\hypersetup{pdfsource={}}
\hypersetup{pdflang={en-UK}}
\hypersetup{pdfcopyright={Copyright 2017-2018 Niklas Beisert.
  This work may be distributed and/or modified under the
  conditions of the LaTeX Project Public License, either version 1.3
  of this license or (at your option) any later version.}}
\hypersetup{pdflicenseurl={http://www.latex-project.org/lppl.txt}}
\hypersetup{pdfcontactaddress={ETH Zurich, ITP, HIT K,
  Wolfgang-Pauli-Strasse 27}}
\hypersetup{pdfcontactpostcode={8093}}
\hypersetup{pdfcontactcity={Zurich}}
\hypersetup{pdfcontactcountry={Switzerland}}
\hypersetup{pdfcontactemail={nbeisert@itp.phys.ethz.ch}}
\hypersetup{pdfcontacturl={http://people.phys.ethz.ch/\xmptilde nbeisert/}}

\newcommand{\secref}[1]{\hyperref[#1]{section \ref*{#1}}}

\parskip1ex
\parindent0pt
\let\olditemize\itemize
\def\itemize{\olditemize\parskip0pt}

\begin{document}

\title{The \textsf{childdoc} Package}
\hypersetup{pdftitle={The childdoc Package}}
\author{Niklas Beisert\\[2ex]
  Institut f\"ur Theoretische Physik\\
  Eidgen\"ossische Technische Hochschule Z\"urich\\
  Wolfgang-Pauli-Strasse 27, 8093 Z\"urich, Switzerland\\[1ex]
  \href{mailto:nbeisert@itp.phys.ethz.ch}
  {\texttt{nbeisert@itp.phys.ethz.ch}}}
\hypersetup{pdfauthor={Niklas Beisert}}
\hypersetup{pdfsubject={Manual for the LaTeX2e Package childdoc}}
\date{30 December 2018, \textsf{v2.0}}
\maketitle

\begin{abstract}\noindent
\textsf{childdoc} is a \LaTeXe{} package
that enables the direct compilation
of document sections included by |\include|
to individual files.
\end{abstract}

\begingroup
\parskip0ex
\tableofcontents
\endgroup

%%%%%%%%%%%%%%%%%%%%%%%%%%%%%%%%%%%%%%%%%%%%%%%%%%%%%%%%%%%%%%%%%%%%%%%%%%%%%%%%
%%%%%%%%%%%%%%%%%%%%%%%%%%%%%%%%%%%%%%%%%%%%%%%%%%%%%%%%%%%%%%%%%%%%%%%%%%%%%%%%
\section{Introduction}

\LaTeX{} provides a mechanism to structure a large document (such as a book)
into a main file and several child files (containing the chapters)
using the |\include| command.
This mechanism is beneficial for documents
which span hundreds of pages in order to
make the source file(s) more manageable.
Moreover, compilation can be restricted to
selected child files by means of the |\includeonly| command.
The latter feature can be used to reduce the compilation time while editing
(this was significantly more useful in the earlier days of \LaTeX{})
or to generate a smaller document which is easier to navigate.
Another application of |\includeonly| is to generate
documents consisting of selected parts of the complete document.

However, there are a few drawbacks of the plain |\include| mechanism:
\begin{itemize}
\item
The child files cannot be compiled on their own,
they can only be compiled via the main file.
A naive editing environment
(such as a text editor with an option
to have the current file processed by \LaTeX)
may require one to switch to the main file before compiling;
attempting to compile the child file produces errors.
\item
The main file must be modified (each time)
to adjust the |\includeonly| command
to the present needs. This easily leaves the main file in a messy state.
\item
The generated document will always carry the filename
of the main document. This is inconvenient if
several child files are to be compiled and
to be kept for distribution.
\end{itemize}

The present package provides a simple interface
to make child files individually compilable by \LaTeX{}.
Compiling a child file then has the same effect as compiling
the main file with an |\includeonly| command
to select the appropriate child.
Moreover the generated document will carry the name of the child
rather than the main file.
This resolves all three above issues.

This feature is meant to make the editing of books,
thesis documents and lecture notes somewhat more convenient.
However, the package can also be used efficiently for
composing a series of documents (such as exercise sheets)
which are typically distributed individually.
It then assists the author in generating the individual documents
(potentially in different versions)
as well as a document containing the collected series.
Another application is in developing style files
or other kinds of included material
where compilation of the style file could redirect
to a sample or test file.

%%%%%%%%%%%%%%%%%%%%%%%%%%%%%%%%%%%%%%%%%%%%%%%%%%%%%%%%%%%%%%%%%%%%%%%%%%%%%%%%
%%%%%%%%%%%%%%%%%%%%%%%%%%%%%%%%%%%%%%%%%%%%%%%%%%%%%%%%%%%%%%%%%%%%%%%%%%%%%%%%
\section{Usage}

First of all, the package \textsf{childdoc} is \emph{not} a standard
\LaTeXe{} |.sty| style file! Therefore it needs to be invoked in
a non-standard way.

%%%%%%%%%%%%%%%%%%%%%%%%%%%%%%%%%%%%%%%%%%%%%%%%%%%%%%%%%%%%%%%%%%%%%%%%%%%%%%%%
\subsection{Included Files}
\label{sec:include}

%%%%%%%%%%%%%%%%%%%%%%%%%%%%%%%%%%%%%%%%
\DescribeMacro{\childdocmain}
To use the package, add the commands
\begin{center}
\begin{tabular}{l}
|\input{childdoc.def}|\\
|\childdocmain{}|\\
\end{tabular}
\end{center}
at the very top of the main \LaTeX{} file,
in particular \emph{before} the |\documentclass| statement!
The argument of |\childdocmain| should be left empty
(but it must be present).

%%%%%%%%%%%%%%%%%%%%%%%%%%%%%%%%%%%%%%%%
\DescribeMacro{\childdocof}
Furthermore, add the commands
\begin{center}
\begin{tabular}{l}
|\input{childdoc.def}|\\
|\childdocof{|\textit{main}|}|\\
\end{tabular}
\end{center}
at the top of every child file \textit{child}
which is included by |\include{|\textit{child}|}|
from within the main file
(or at least for those files to be compiled individually).
The argument \textit{main} must be the filename of the main file.

There are a couple of
considerations in setting up the main and child documents:

%%%%%%%%%%%%%%%%%%%%%%%%%%%%%%%%%%%%%%%%
\paragraph{Restrictions.}

Please note the following restrictions:
\begin{itemize}
\item
|\childdocmain| must be called with one argument \textit{main}
to ensure compatibility with earlier version of the package.
It must either be empty (|\childdocmain{}|)
or precisely match the filename of the main file in which it is specified.
See \secref{sec:detection} for further information.
\item
The filename \textit{main} must be specified without the |.tex| extension.
\item
The filename \textit{main} is case sensitive
(even in case-insensitive file systems)
due to internal string comparison.
\item
The argument \textit{main} should be fully expanded, it cannot be a macro.
\item
Subdirectories and special characters should be avoided in filenames.
\item
The command |\childdocmain{|\textit{main}|}| must be followed by a whitespace.
It should not be followed immediately by another command
or by a comment mark `|%|'.
This is because the \TeX{} parser reads the token immediately following
the argument of |\childdocmain| and puts it
at the beginning of every child section;
however, a white\-space is ignored.
\end{itemize}

%%%%%%%%%%%%%%%%%%%%%%%%%%%%%%%%%%%%%%%%
\paragraph{Content of Main File.}

It is advisable to place all content in the child files included by |\include|.
Any output contained in the main file will appear in all child documents
unless suppressed manually;
it cannot be suppressed automatically by the |\includeonly| directive
and thus should normally be avoided.
A method to include some content in the main file
by means of conditional processing is described in \secref{sec:conditional}.

%%%%%%%%%%%%%%%%%%%%%%%%%%%%%%%%%%%%%%%%
\paragraph{Page Numbering.}

When only a part of the document is compiled,
the appropriate numbering of pages
(as well as other status parameters)
is determined from the |.aux| files.
The latter contain information from previous passes.
However this information needs to propagate through
all intermediate child documents.
Therefore the page numbering in child documents may well
be inconsistent until the complete document is compiled at least once.

A useful (if unconventional) way to always ensure a consistent
page numbering is to restart the numbering in each child document
and denote the pages by `\textit{child}|.|\textit{page}'
where \textit{child} represents the chapter/section number of the child file.
This can be achieved by the command
|\numberwithin{page}{|\textit{child}|}|
of the \textsf{amsmath} package
where \textit{child} can be |chapter| or |section|
depending on the chosen structuring.
Alternatively, one can modify the macro |\thepage| appropriately
and reset the counter |page| at the start of each child file.

%%%%%%%%%%%%%%%%%%%%%%%%%%%%%%%%%%%%%%%%%%%%%%%%%%%%%%%%%%%%%%%%%%%%%%%%%%%%%%%%
\subsection{Conditional Processing}
\label{sec:conditional}

The package provides a mechanism to compile different versions
of a document. To customise the versions further some conditional processing
can come in handy to distinguish which version is being compiled.
The package provides two macros to describe the compilation context:

%%%%%%%%%%%%%%%%%%%%%%%%%%%%%%%%%%%%%%%%
\DescribeMacro{\ifchilddoc}
The conditional |\ifchilddoc| distinguishes between the compilation of
child documents and the main document:
%
\begin{center}
|\ifchilddoc |\textit{child-code}| |[|\||else |\textit{main-code}]| \||fi|
\end{center}

%%%%%%%%%%%%%%%%%%%%%%%%%%%%%%%%%%%%%%%%
\DescribeMacro{\childdocname}
\DescribeMacro{\childdocjob}
The macro |\childdocname| contains the filename (without extension)
of the main or child file being processed.
Note that |\childdocjob| will always contain the name of the main file.

%%%%%%%%%%%%%%%%%%%%%%%%%%%%%%%%%%%%%%%%
\paragraph{Title Page.}

Conditional processing can be used to include a title or banner page
in the main document when proper precautions are taken.
Importantly, the code in the main file should ensure that the page counter
(as well as other status parameters which are stored in the |.aux| files)
takes the same value after the conditional processing.
Otherwise the page numbers may take divergent values
depending on which part is compiled.

For example, a title page could be declared by:
%
\begin{center}
\begin{tabular}{l}
|\ifchilddoc\||else|\\
|\addtocounter{page}{-1}|\\
\textit{code for title page}\\
|\newpage|\\
|\||fi|
\end{tabular}
\end{center}
%
A banner page for the child documents can be generated by:
%
\begin{center}
\begin{tabular}{l}
|\ifchilddoc|\\
|\addtocounter{page}{-1}|\\
\textit{code for banner page}\\
|\newpage|\\
|\||fi|
\end{tabular}
\end{center}
%
Here one could write a message such as:
\begin{center}
|This is the part \childdocname{} of \childdocjob{}.|
\end{center}

%%%%%%%%%%%%%%%%%%%%%%%%%%%%%%%%%%%%%%%%%%%%%%%%%%%%%%%%%%%%%%%%%%%%%%%%%%%%%%%%
\subsection{Flags}
\label{sec:flags}

The package makes it easy to generate different versions
of the main or child documents.
To this end compilation flags can be defined
and assigned different default values.
They will be particularly useful in conjunction
with the forwarding mechanism described in \secref{sec:forward}.

For example, it may be useful to have a flag |\version|
which can be set to |draft| or |final|.
The document source will contain some conditional code
depending on the value of |\version|.
Suppose further, the flag should default to |final| for the main file
and to |draft| for child files
which is a natural assignment for editing the document.
This is achieved by placing the following code
in the preamble of the main document
(below the |\childdocmain| directive):
%
\begin{center}
\begin{tabular}{l}
|\ifchilddoc|\\
|\providecommand{\version}{draft}|\\
|\||else|\\
|\providecommand{\version}{final}|\\
|\||fi|
\end{tabular}
\end{center}
%
The definition by |\providecommand| makes sure
that previous definitions are not overwritten.
Further statements |\providecommand{\version}{...}|
can thus be added before the above code to override it.

For the main file, one might add a line
(between |\childdocmain| and the above block)
%
\begin{center}
|%\ifchilddoc\||else\providecommand{\version}{draft}\||fi|
\end{center}
%
which can be uncommented to produce a draft version.
Likewise one can add a line to the very top of a child file
(above the |\childdocof{|\textit{main}|}| directive)
%
\begin{center}
|%\providecommand{\version}{final}|
\end{center}
%
which can be uncommented to produce the final version of this child document.

%%%%%%%%%%%%%%%%%%%%%%%%%%%%%%%%%%%%%%%%%%%%%%%%%%%%%%%%%%%%%%%%%%%%%%%%%%%%%%%%
\subsection{Forwarding}
\label{sec:forward}

Different versions of the main or child documents
using compilation flags as described in \secref{sec:flags}
can be (permanently) stored in different files
for convenient compilation, viewing and distribution.
To this end, the package defines a command
to pass on compilation to a different file:

%%%%%%%%%%%%%%%%%%%%%%%%%%%%%%%%%%%%%%%%
\DescribeMacro{\childdocforward}
The command |\childdocforward| redirects processing to
another source file:
%
\begin{center}
\begin{tabular}{l}
|\input{childdoc.def}|\\
|\childdocforward[|\textit{main}|]{|\textit{dest}|}|\\
\end{tabular}
\end{center}
%
The argument \textit{dest} is the destination file
(without extension).
It should be the main file or one of the child files.
Note that further \textsf{childdoc} directives
such as |\childdocof| and |\childdocforward|
in the indicated file will be processed in this form.
The optional argument \textit{main}
passes on directly to the main file \textit{main}
while pretending to compile the child \textit{dest}.
This form behaves as if \textit{dest}
issues |\childdocof{|\textit{main}|}| right away,
and no further \textsf{childdoc} directives will be processed.

%%%%%%%%%%%%%%%%%%%%%%%%%%%%%%%%%%%%%%%%
\DescribeMacro{\...prefix}
In the alternative form |\childdocforwardprefix|,
%
\begin{center}
\begin{tabular}{l}
|\input{childdoc.def}|\\
|\childdocforwardprefix[|\textit{main}|]{|\textit{prefix}|}{|\textit{dest}|}|
\end{tabular}
\end{center}
%
the destination file is determined by a pattern
depending on the current file:
To make this work, the current file must be called
`{\textit{prefix}\hspace{0.2em}\textit{suffix}}'
with \textit{prefix} matching precisely the argument.
Processing is then passed on to the file
`{\textit{dest}\hspace{0.2em}\textit{suffix}}'.
Surely, the same effect is achieved by
directly specifying the
argument `{\textit{dest}\hspace{0.2em}\textit{suffix}}'
in the first form.
However, that requires to set up a different file
for each child. With the alternative form of the command
all these files can have exactly the same content
which simplifies setting them up and maintaining them.

For example, the following file |draft.tex|
with a compilation flag |\version| as described in \secref{sec:flags}
compiles the main document as a draft:
%
\begin{center}
\begin{tabular}{l}
|\def\version{draft}|\\
|\input{childdoc.def}|\\
|\childdocforward{|\textit{main}|}|
\end{tabular}
\end{center}
%
Likewise, the following files |final|\textit{nn}|.tex|
compile the final version of the child document
|child|\textit{nn}|.tex|:
%
\begin{center}
\begin{tabular}{l}
|\def\version{final}|\\
|\input{childdoc.def}|\\
|\childdocforwardprefix{final}{child}|
\end{tabular}
\end{center}
%

Note that when several versions of a main file and/or of each child file
are to be generated, it may be convenient to set up a |Makefile| or
shell script to automatise the process.

%%%%%%%%%%%%%%%%%%%%%%%%%%%%%%%%%%%%%%%%%%%%%%%%%%%%%%%%%%%%%%%%%%%%%%%%%%%%%%%%
\subsection{Command Line Processing}
\label{sec:commandline}

The effect of redirection files can also be achieved by invoking
the \LaTeX{} compiler with a more elaborate command line.
Most conveniently this should be done as part
of a shell script or a |Makefile|.

When using \textsf{childdoc} in the main file, the following
command lines effectively perform a redirection
(note that depending on the shell being used,
backslashes may have to be doubled: `|\|' $\to$ `|\\|'):
%
\begin{center}
|... -jobname "|\textit{target}|" |\\|"|[\textit{flags}]%
|\input{childdoc.def}\childdocforward[|\textit{main}|]{|\textit{dest}|}"|
\end{center}
%
Here \textit{target} is the name of the output file,
\textit{main} is the name of the main file
and \textit{dest} is the name of the main or child file to be processed
(all filenames without extensions).
The optional argument \textit{main} can be omitted
if \textit{main} matches \textit{dest}.
Optionally, compilation \textit{flags} can be defined via |\def| commands.
This command line makes the \TeX{} engine believe
it is compiling the file \textit{target}
whose content is specified as the latter parameter.
The provided code then forwards the processing to
\textit{main} or \textit{dest} as described in \secref{sec:forward}.

%%%%%%%%%%%%%%%%%%%%%%%%%%%%%%%%%%%%%%%%%%%%%%%%%%%%%%%%%%%%%%%%%%%%%%%%%%%%%%%%
\subsection{Include by Input}
\label{sec:input}

Including child documents by |\include| has some restrictions by design.
Most notably, the content of a child document always occupies
its own set of pages; pages cannot be shared between child documents.
Usually, this behaviour makes perfect sense
because each child document contain an essential part of the document.
However, in some situations it may be desirable to compose
a document from a collection of parts
without having mandatory page breaks between then.
For this case, the package
provides a mechanism to include parts
by |\input| which can also be processed individually.
However, by construction this mechanism
requires manual handling of the content to be output.

%%%%%%%%%%%%%%%%%%%%%%%%%%%%%%%%%%%%%%%%
\DescribeMacro{\ifchilddocmanual}
The main file should be prepared as usual, see \secref{sec:include}.
However, the document body must make a distinction
between processing of an individual part and of the main document, e.g.:
%
\begin{center}
\begin{tabular}{l}
|\ifchilddocmanual|\\
|\input{\childdocname}|\\
|\||else|\\
\textit{document body with }|\input{|\textit{part}|}|\\
|\||fi|
\end{tabular}
\end{center}
%
The conditional |\ifchilddocmanual| is true whenever
a part to be included by |\input| is being compiled,
and the name of the part is stored in |\childdocname|.

%%%%%%%%%%%%%%%%%%%%%%%%%%%%%%%%%%%%%%%%
\DescribeMacro{\childdocby}
Each part to be included by |\input| should start with:
%
\begin{center}
\begin{tabular}{l}
|\input{childdoc.def}|\\
|\childdocby{|\textit{main}|}|\\
\end{tabular}
\end{center}
%
The directive |\childdocby| is similar to |\childdocof|
described in \secref{sec:include},
but the subsequent selection of content must be done manually.
To that end, both |\ifchilddoc| and |\ifchilddocmanual|
will be true upon processing of a part,
and the name of the part is stored in |\childdocname|.
Note that |\jobname| will be set to the filename of the current part
so that each part receives an individual |.aux| file
that does not interfere with the |.aux| file(s) of the main document.
This behaviour can be altered by the alternative form
|\childdocby[*]{|\textit{main}|}| (with a non-empty optional argument)
which uses the |.aux| file of the main document
by setting |\jobname| to \textit{main}.

%%%%%%%%%%%%%%%%%%%%%%%%%%%%%%%%%%%%%%%%%%%%%%%%%%%%%%%%%%%%%%%%%%%%%%%%%%%%%%%%
\subsection{Driver Development}
\label{sec:driver}

The \textsf{childdoc} mechanism can also be use for the development
of definition files such as \LaTeX{} styles or classes.
This case differs from the above setup with multiple parts
included by |\include| in that no |\includeonly| should be invoked.
This can be achieved by starting the include file
(before |\ProvidesPackage|) with:
%
\begin{center}
\begin{tabular}{l}
|\input{childdoc.def}|\\
|\childdocforward{|\textit{main}|}|\\
\end{tabular}
\end{center}
%
or alternatively with:
%
\begin{center}
\begin{tabular}{l}
|\input{childdoc.def}|\\
|\childdocby{|\textit{main}|}|\\
\end{tabular}
\end{center}
%
Both forms have slightly different effects as described above.
The main file is prepared as usual, see \secref{sec:include}.

%%%%%%%%%%%%%%%%%%%%%%%%%%%%%%%%%%%%%%%%%%%%%%%%%%%%%%%%%%%%%%%%%%%%%%%%%%%%%%%%
\subsection{Legacy Detection}
\label{sec:detection}

The directive |\childdocmain| in the main file can detect
whether the complete document or merely a child is to be compiled
even without using the directive |\childdocof|.
This method is deprecated because it is less robust
and there is no compelling reason to use it;
it is merely provided for backward compatibility
and it may be removed in future versions.

If the detection mechanism is to be used,
it is mandatory to correctly specify
the filename of the main file as the argument of |\childdocmain|:
%
\begin{center}
\begin{tabular}{l}
|\input{childdoc.def}|\\
|\childdocmain{|\textit{main}|}|\\
\end{tabular}
\end{center}
%
If |\jobname| does not match the argument \textit{main} of |\childdocmain|,
it is assumed that |\jobname| points to the child file to be compiled.
When using |\childdocmain| with the main file specified as argument,
it suffices to start a child file
with just |\input{|\textit{main}|}|
without loading of the package and using |\childdocof|.
If instead all processing is done
with the appropriate \textsf{childdoc} directives,
the argument of \textit{main} of |\childdocmain| can be empty.

An alternative version of the command line processing described
in \secref{sec:commandline} using the detection mechanism reads:
%
\begin{center}
|... -jobname "|\textit{target}|" "|[\textit{flags}]%
[|\def\jobname{|\textit{dest}|}|]|\input{|\textit{main}|}"|
\end{center}

%%%%%%%%%%%%%%%%%%%%%%%%%%%%%%%%%%%%%%%%%%%%%%%%%%%%%%%%%%%%%%%%%%%%%%%%%%%%%%%%
\subsection{Manual Code}
\label{sec:manual}

In case one cannot be certain whether the definitions file |childdoc.def|
is installed on the target \TeX{} distribution
and one prefers not to ship it,
it is conceivable to paste a few relevant commands into the sources.

To that end, drop all statements |\input{childdoc.def}|
and perform the replacements as outlined below.
Instead of |\childdocmain{|\textit{main}|}| add the following code
to the top of the main file:
%
\begin{center}
\begin{tabular}{l}
|\||ifdefined\childdocname\endinput\||fi\newif\ifchilddoc|\\
|\edef\childdocname{\scantokens\expandafter{\jobname\noexpand}}|\\
|\def\childdocmain{|\textit{main}|}\||ifx\childdocmain\childdocname\||else|\\
|\childdoctrue\includeonly{\childdocname}\let\jobname\childdocmain\||fi|\\
\end{tabular}
\end{center}
%
Instead of |\childdocof{|\textit{main}|}| just include the main file
at the top of each child file:
%
\begin{center}
|\input{|\textit{main}|}|
\end{center}
%
A simple redirection |\childdocforward{|\textit{dest}|}| is achieved by:
%
\begin{center}
|\def\jobname{|\textit{dest}|}\input{\jobname}|
\end{center}
%
The redirection with prefix
|\childdocforwardprefix[|\textit{prefix}|]{|\textit{dest}|}|
is accomplished by:
%
\begin{center}
\begin{tabular}{l}
|{\edef\jobname{\scantokens\expandafter{\jobname\noexpand}}|\\
|\def\redirectjob |\textit{prefix}|#1~~~{\gdef\jobname{|\textit{dest}|#1}}|\\
|\expandafter\redirectjob\jobname~~~}\input{\jobname}|
\end{tabular}
\end{center}

In an alternative approach,
child documents can be compiled by a specific command line
without additional code or specific definitions:
%
\begin{center}
|... -jobname "|\textit{target}|" "|[\textit{flags}]%
|\includeonly{|\textit{dest}|}\input{|\textit{main}|}"|
\end{center}
%

%%%%%%%%%%%%%%%%%%%%%%%%%%%%%%%%%%%%%%%%%%%%%%%%%%%%%%%%%%%%%%%%%%%%%%%%%%%%%%%%
%%%%%%%%%%%%%%%%%%%%%%%%%%%%%%%%%%%%%%%%%%%%%%%%%%%%%%%%%%%%%%%%%%%%%%%%%%%%%%%%
\section{Information}

%%%%%%%%%%%%%%%%%%%%%%%%%%%%%%%%%%%%%%%%%%%%%%%%%%%%%%%%%%%%%%%%%%%%%%%%%%%%%%%%
\subsection{Copyright}

Copyright \copyright{} 2017--2018 Niklas Beisert

This work may be distributed and/or modified under the
conditions of the \LaTeX{} Project Public License, either version 1.3
of this license or (at your option) any later version.
The latest version of this license is in
  \url{http://www.latex-project.org/lppl.txt}
and version 1.3 or later is part of all distributions of \LaTeX{}
version 2005/12/01 or later.

This work has the LPPL maintenance status `maintained'.

The Current Maintainer of this work is Niklas Beisert.

This work consists of the files |README.txt|, |childdoc.ins| and |childdoc.dtx|
as well as the derived files |childdoc.def|, |cdocsamp.tex|
with |cdocsch1.tex|, |cdocsch2.tex|, |cdocspt3.tex|, |cdocspt4.tex|,
|cdocsdrf.tex|, |cdocsfn1.tex|, |cdocsfn2.tex|
as well as |childdoc.pdf|.

%%%%%%%%%%%%%%%%%%%%%%%%%%%%%%%%%%%%%%%%%%%%%%%%%%%%%%%%%%%%%%%%%%%%%%%%%%%%%%%%
\subsection{Files and Installation}

The package consists of the files:
%
\begin{center}
\begin{tabular}{ll}
    |README.txt|   & readme file \\
    |childdoc.ins| & installation file \\
    |childdoc.dtx| & source file \\
    |childdoc.def| & definition file \\
    |cdocsamp.tex| & sample main file \\
    |cdocsch1.tex| & sample include file \\
    |cdocsch2.tex| & sample include file \\
    |cdocspt3.tex| & sample part file \\
    |cdocspt4.tex| & sample part file \\
    |cdocsdrf.tex| & sample redirection file \\
    |cdocsfn1.tex| & sample redirection file \\
    |cdocsfn2.tex| & sample redirection file \\
    |childdoc.pdf| & manual
\end{tabular}
\end{center}
%
The distribution consists of the files
|README.txt|, |childdoc.ins| and |childdoc.dtx|.
%
\begin{itemize}
\item
Run (pdf)\LaTeX{} on |childdoc.dtx|
to compile the manual |childdoc.pdf| (this file).
\item
Run \LaTeX{} on |childdoc.ins| to create the definitions file |childdoc.def|
and the sample |cdocsamp.tex| with include files
|cdocsch1.tex|, |cdocsch2.tex|, |cdocspt3.tex|, |cdocspt4.tex|,
|cdocsdrf.tex|, |cdocsfn1.tex|, |cdocsfn2.tex|.
Then copy the file |childdoc.def| to an appropriate directory of your \LaTeX{}
distribution, e.g.\ \textit{texmf-root}|/tex/latex/childdoc|.
\end{itemize}

%%%%%%%%%%%%%%%%%%%%%%%%%%%%%%%%%%%%%%%%%%%%%%%%%%%%%%%%%%%%%%%%%%%%%%%%%%%%%%%%
\subsection{Related CTAN Packages}

There are several other packages which offer a similar functionality:
%
\begin{itemize}
\item
The packages
\href{http://ctan.org/pkg/docmute}{\textsf{docmute}},
\href{http://ctan.org/pkg/includex}{\textsf{includex}} and
\href{http://ctan.org/pkg/standalone}{\textsf{standalone}}
provide commands to include only the document body of
a child file thus allowing both files to be compiled individually.
\item
The packages \href{http://ctan.org/pkg/subdocs}{\textsf{subdocs}}
and \href{http://ctan.org/pkg/subfiles}{\textsf{subfiles}}
provide structures in which the main and child documents can be
encapsulated and allowing them to be compiled individually.
The inclusion mechanism is different from the conventional |\include|.
\item
The package \href{http://ctan.org/pkg/combine}{\textsf{combine}}
is an elaborate solution to combine several documents into one.
\end{itemize}
%
See also the CTAN topic \href{http://ctan.org/topic/subdocs}{\textsf{subdocs}}
for further related packages.
The present package differs from the above solutions in that
a document structure constructed with the conventional |\include| mechanism
just needs two extra commands at the top of every file
such that all constituent files can be compiled individually.

%%%%%%%%%%%%%%%%%%%%%%%%%%%%%%%%%%%%%%%%%%%%%%%%%%%%%%%%%%%%%%%%%%%%%%%%%%%%%%%%
%\subsection{Feature Suggestions}
%
%The following is a list of features which may be useful for future
%versions of this package:
%%
%\begin{itemize}
%\item
%\ldots
%\end{itemize}

%%%%%%%%%%%%%%%%%%%%%%%%%%%%%%%%%%%%%%%%%%%%%%%%%%%%%%%%%%%%%%%%%%%%%%%%%%%%%%%%
\subsection{Revision History}

%%%%%%%%%%%%%%%%%%%%%%%%%%%%%%%%%%%%%%%%
\paragraph{v2.0:} 2018/12/30

\begin{itemize}
\item
immediate forward processing
\item
added |\childdocby| mechanism
\item
manual restructured
\end{itemize}

%%%%%%%%%%%%%%%%%%%%%%%%%%%%%%%%%%%%%%%%
\paragraph{v1.6:} 2018/01/17

\begin{itemize}
\item
application for development of include files
\item
corrections to manual
\end{itemize}

%%%%%%%%%%%%%%%%%%%%%%%%%%%%%%%%%%%%%%%%
\paragraph{v1.5:} 2017/05/21

\begin{itemize}
\item
more complete structuring introduced
\item
|\childdocof| introduced
\item
|\childdoc| renamed to |\childdocmain|
\item
|\childredirect| renamed to |\childdocforward| and |\childdocforwardprefix|
and functionality expanded
\end{itemize}

%%%%%%%%%%%%%%%%%%%%%%%%%%%%%%%%%%%%%%%%
\paragraph{v1.0:} 2017/04/27

\begin{itemize}
\item
manual and install package
\item
first version published on CTAN
\end{itemize}

%%%%%%%%%%%%%%%%%%%%%%%%%%%%%%%%%%%%%%%%
\paragraph{v0.6:} 2017/04/26

\begin{itemize}
\item
redirection mechanism added
\end{itemize}

%%%%%%%%%%%%%%%%%%%%%%%%%%%%%%%%%%%%%%%%
\paragraph{v0.5:} 2017/04/26

\begin{itemize}
\item
functionality in definition file
\end{itemize}


%%%%%%%%%%%%%%%%%%%%%%%%%%%%%%%%%%%%%%%%%%%%%%%%%%%%%%%%%%%%%%%%%%%%%%%%%%%%%%%%
%%%%%%%%%%%%%%%%%%%%%%%%%%%%%%%%%%%%%%%%%%%%%%%%%%%%%%%%%%%%%%%%%%%%%%%%%%%%%%%%
%%%%%%%%%%%%%%%%%%%%%%%%%%%%%%%%%%%%%%%%%%%%%%%%%%%%%%%%%%%%%%%%%%%%%%%%%%%%%%%%
\appendix

\settowidth\MacroIndent{\rmfamily\scriptsize 000\ }

 \DocInput{childdoc.dtx}

\end{document}
%</driver>
% \fi
%
% %%%%%%%%%%%%%%%%%%%%%%%%%%%%%%%%%%%%%%%%%%%%%%%%%%%%%%%%%%%%%%%%%%%%%%%%%%%%%%
% %%%%%%%%%%%%%%%%%%%%%%%%%%%%%%%%%%%%%%%%%%%%%%%%%%%%%%%%%%%%%%%%%%%%%%%%%%%%%%
% \section{Sample}
%\iffalse
%<*samplemain>
%\fi
%
% The following presents a sample document
% with two chapters, two parts, a title page,
% a compile flag as well as three forwarding files to set the flag.
% It consists of eight |.tex| files:
% \begin{center}
% \begin{tabular}{ll}
% |cdocsamp.tex|&main file\\
% |cdocsch1.tex|&include file for chapter 1\\
% |cdocsch2.tex|&include file for chapter 2\\
% |cdocspt3.tex|&include file for part 3\\
% |cdocspt4.tex|&include file for part 4\\
% |cdocsdrf.tex|&forwarding file for main file in draft mode\\
% |cdocsfi1.tex|&forwarding file for final version of chapter 1\\
% |cdocsfi2.tex|&forwarding file for final version of chapter 2\\
% \end{tabular}
% \end{center}
% Each of the eight files can be compiled directly by the \LaTeX{} compiler.
%
% %%%%%%%%%%%%%%%%%%%%%%%%%%%%%%%%%%%%%%
% \paragraph{Main File.}
%
% The main file is called |cdocsamp.tex|.
%
% Load the \textsf{childdoc} definitions and
% declare the filename for the main document:
%    \begin{macrocode}
\input{childdoc.def}
\childdocmain{}
%    \end{macrocode}

% Optional override for |\version| flag:
%    \begin{macrocode}
%%\ifchilddoc\else\providecommand{\version}{draft}\fi
%    \end{macrocode}

% Define the default values for the |\version| flag
% (|final| for the main file and |draft| for childs):
%    \begin{macrocode}
\ifchilddoc
\providecommand{\version}{draft}
\else
\providecommand{\version}{final}
\fi
%    \end{macrocode}

% Load the standard document class:
%    \begin{macrocode}
\documentclass[12pt]{article}
%    \end{macrocode}

% Start the document body:
%    \begin{macrocode}
\begin{document}
%    \end{macrocode}

% Declare a title page.
% Print title, part of document being processed and version flag:
%    \begin{macrocode}
\addtocounter{page}{-1}
\begin{center}
{\LARGE\bfseries{}childdoc example\par}
\vspace{1cm}
\ifchilddoc
\ifchilddocmanual part\else chapter\fi:
`\childdocname' of `\childdocjob'\par
\else
main document: `\childdocjob'\par
\fi
version: \version\par
\end{center}
\newpage
%    \end{macrocode}

% Manually include selected file,
% otherwise process as usual:
%    \begin{macrocode}
\ifchilddocmanual
\section*{part `\childdocname'}
\input{\childdocname}
\else
%    \end{macrocode}

% Include the two chapters:
%    \begin{macrocode}
\include{cdocsch1}
\include{cdocsch2}
%    \end{macrocode}

% Include the two parts unless only chapters should be displayed:
%    \begin{macrocode}
\ifchilddoc\else
\section{part three}
\input{cdocspt3}
\section{part four}
\input{cdocspt4}
\fi
%    \end{macrocode}

% Process as usual until here:
%    \begin{macrocode}
\fi
%    \end{macrocode}

% End of document body:
%    \begin{macrocode}
\end{document}
%    \end{macrocode}
%\iffalse
%</samplemain>
%\fi
%
% %%%%%%%%%%%%%%%%%%%%%%%%%%%%%%%%%%%%%%
% \paragraph{Chapter Include Files.}
%
% The include files are called |cdocsch1.tex| and |cdocsch2.tex|.
%
%\iffalse
%<*samplechap1|samplechap2>
%\fi

% Optional override for |\version| flag:
%    \begin{macrocode}
%%\providecommand{\version}{final}
%    \end{macrocode}

% Include the main document:
%    \begin{macrocode}
\input{childdoc.def}
\childdocof{cdocsamp}
%    \end{macrocode}

%\iffalse
%</samplechap1|samplechap2>
%\fi
%
%\iffalse
%<*samplechap1>
%\fi
% Some text for chapter 1:
%    \begin{macrocode}
\section{one}
some text in chapter one
%    \end{macrocode}

%\iffalse
%</samplechap1>
%\fi
% Some text for chapter 2:
%\iffalse
%<*samplechap2>
%\fi
%    \begin{macrocode}
\section{two}
more text in chapter two
%    \end{macrocode}

%\iffalse
%</samplechap2>
%\fi
%
% %%%%%%%%%%%%%%%%%%%%%%%%%%%%%%%%%%%%%%
% \paragraph{Part Include Files.}
%
% The include files are called |cdocspt3.tex| and |cdocspt4.tex|.
%
%\iffalse
%<*samplepart3|samplepart4>
%\fi

% Optional override for |\version| flag:
%    \begin{macrocode}
%%\providecommand{\version}{final}
%    \end{macrocode}

% Include the main document:
%    \begin{macrocode}
\input{childdoc.def}
\childdocby{cdocsamp}
%    \end{macrocode}

%\iffalse
%</samplepart3|samplepart4>
%\fi
%
%\iffalse
%<*samplepart3>
%\fi
% Some text for part 3:
%    \begin{macrocode}
some text in part three
%    \end{macrocode}

%\iffalse
%</samplepart3>
%\fi
% Some text for part 4:
%\iffalse
%<*samplepart4>
%\fi
%    \begin{macrocode}
more text in part four
%    \end{macrocode}

%\iffalse
%</samplepart4>
%\fi
%
% %%%%%%%%%%%%%%%%%%%%%%%%%%%%%%%%%%%%%%
% \paragraph{Forwarding for a Complete Draft.}
%
% The following forwarding file |cdocsdrf.tex|
% compiles the main document in draft mode:
%\iffalse
%<*sampledraft>
%\fi
%    \begin{macrocode}
\def\version{draft}
\input{childdoc.def}
\childdocforward{cdocsamp}
%    \end{macrocode}

%\iffalse
%</sampledraft>
%\fi
%
% %%%%%%%%%%%%%%%%%%%%%%%%%%%%%%%%%%%%%%
% \paragraph{Forwarding for Final Version of the Chapters.}
%
% The following forwarding files |cdocsfn1.tex| and |cdocsfn2.tex|
% (with identical content)
% compile the final versions of the child documents
% |cdocsch1.tex| and |cdocsch2.tex|, respectively:
%\iffalse
%<*samplefinal>
%\fi
%    \begin{macrocode}
\def\version{final}
\input{childdoc.def}
\childdocforwardprefix[cdocsamp]{cdocsfn}{cdocsch}
%    \end{macrocode}

%\iffalse
%</samplefinal>
%\fi
%
% %%%%%%%%%%%%%%%%%%%%%%%%%%%%%%%%%%%%%%
% \paragraph{Command Line Processing.}
%
% The following three command lines generate the output files
% |cdocscld|, |cdocscl1| and |cdocscl2|
% which should be identical to
% |cdocsdrf|, |cdocsch1| and |cdocsfn2|, respectively:
% \begin{center}
% \begin{tabular}{l}
% |latex -jobname cdocscld \|\\
% |  "\def\version{draft}\input{childdoc.def}\childdocforward{cdocsamp}"|\\
% |latex -jobname cdocscl1 \|\\
% |  "\input{childdoc.def}\childdocforward[cdocsamp]{cdocsch1}"|\\
% |latex -jobname cdocscl2 \|\\
% |  "\def\version{final}\input{childdoc.def}\childdocforward{cdocsch2}"|
% \end{tabular}
% \end{center}
% Note that the trailing backslash on each first line
% merely continues the input to the second line
% (for convenient cut ant paste).
% Furthermore, the command |latex| can be replaced by any
% of its alternative versions such as |pdflatex|.
%
% %%%%%%%%%%%%%%%%%%%%%%%%%%%%%%%%%%%%%%%%%%%%%%%%%%%%%%%%%%%%%%%%%%%%%%%%%%%%%%
% %%%%%%%%%%%%%%%%%%%%%%%%%%%%%%%%%%%%%%%%%%%%%%%%%%%%%%%%%%%%%%%%%%%%%%%%%%%%%%
% \section{Implementation}
%\iffalse
%<*package>
%\fi
%
% This section describes the definitions file |childdoc.def|.

% The definitions cannot be loaded using |\usepackage| or |\RequirePackage|
% which has a mechanism to prevent loading a style file more than once.
% When loading the definitions by means of |\input|
% multiple instances have to be prevented manually:
%\iffalse
%This code needs to be before the `\ProvidesFile' directive
%which is defined at the beginning of this file.
%Therefore it is also placed there and commented out here.
%</package>
%<*discard>
%\fi
%    \begin{macrocode}
\ifdefined\childdocmain\endinput\fi
%    \end{macrocode}
%\iffalse
%</discard>
%<*package>
%\fi
%
% \macro{\ifchilddoc}
% \macro{\ifchilddocmanual}
% The conditional |\ifchilddoc| tells whether a
% child (true) or main (false) document is being compiled.
% The conditional |\ifchilddocmanual| tells whether
% the |\includeonly| mechanism is used (false) or
% the selection of child files must be performed manually (true).
% The definitions initialise to false:
%    \begin{macrocode}
\newif\ifchilddoc
\newif\ifchilddocmanual
%    \end{macrocode}

% \macro{\childdocname}
% \macro{\childdocjob}
% The macro |\childdocname| stores the name of the main document
% to be compiled. The macro |\childdocjob| stores the name of
% the document on which the \LaTeX{} compiler was originally invoked.
% The content of |\jobname| cannot be compared
% to filenames specified in the source due to different catcodes.
% The following code rescans |\jobname|, stores the result
% in |\childdocname| and saves a copy in |\childdocjob|:
%    \begin{macrocode}
\edef\childdocname{\scantokens\expandafter{\jobname\noexpand}}
\let\childdocjob\childdocname
%    \end{macrocode}

% \macro{\childdocdisable}
% The macro |\childdocdisable| prevents the main file
% from being processed more than once.
% At this stage, the main document command |\childdocmain|
% is assumed to be called once again where it should do nothing.
% Any subsequent call to it should prevent
% a secondary processing of the main document
% It overwrites the forwarding commands
% |\childdocof| and |\childdocforward|
% with empty macros to prevent further inclusions of the main document:
%    \begin{macrocode}
\newcommand{\childdocdisable}
{
  \renewcommand{\childdocmain}[1]{\renewcommand{\childdocmain}[1]{\endinput}}
  \renewcommand{\childdocof}[1]{}
  \renewcommand{\childdocby}[2][]{}
  \renewcommand{\childdocforward}[2][]{}
  \renewcommand{\childdocdisable}{}
}
%    \end{macrocode}

% \macro{\childdocmain}
% The macro |\childdocmain| is to be called at the top of the main file
% with nothing or the main filename (without extension) as argument.
% First, it breaks loops.
% If the argument is not empty and does not match |\childdocname|
% (which is set by the first inclusion of |childdoc.def|),
% |\ifchilddoc| is set to true, |\includeonly| is applied to the child file
% and |\jobname| is set to the main file
% (for proper handling of |.aux| files):
%    \begin{macrocode}
\newcommand{\childdocmain}[1]
{
  \childdocdisable\childdocmain{}
  \if?#1?\else
    \begingroup
      \def\childdoctmp{#1}
      \ifx\childdoctmp\childdocname
        \def\childdoctmp{}
      \else
        \def\childdoctmp
        {
          \childdoctrue
          \includeonly{\childdocname}
          \def\childdocjob{#1}
          \def\jobname{#1}
        }
      \fi
      \expandafter
    \endgroup
    \childdoctmp
  \fi
}
%    \end{macrocode}

% \macro{\childdocof}
% The command |\childdocof| redirects
% compilation to the main file |#1|.
%    \begin{macrocode}
\newcommand{\childdocof}[1]
{
  \childdocdisable
  \childdoctrue
  \includeonly{\childdocname}
  \def\jobname{#1}
  \def\childdocjob{#1}
  \input{#1}
}
%    \end{macrocode}

% \macro{\childdocby}
% The command |\childdocby| ....
%    \begin{macrocode}
\newcommand{\childdocby}[2][]
{
  \childdocdisable
  \childdoctrue
  \childdocmanualtrue
  \if?#1?\else
    \def\jobname{#2}
  \fi
  \def\childdocjob{#2}
  \input{#2}
  \endinput
}
%    \end{macrocode}

% \macro{\childdocforward}
% The command |\childdocforward| redirects
% compilation to the main file or
% (if the optional argument is given) a child file.
% Parameters are set as if the main file
% or a child file starting with |\childdocof| was compiled.
% Then compilation is handed over to the main file:
%    \begin{macrocode}
\newcommand{\childdocforward}[2][]
{
  \begingroup
    \if?#1?
      \def\childdoctmp
      {
        \def\childdocname{#2}
        \def\childdocjob{#2}
        \def\jobname{#2}
        \input{#2}
        \endinput
      }
    \else
      \def\childdoctmp
      {
        \childdocdisable
        \def\childdocname{#2}
        \childdoctrue
        \includeonly{#2}
        \def\childdocjob{#1}
        \def\jobname{#1}
        \input{#1}
        \endinput
      }
    \fi
    \expandafter
  \endgroup
  \childdoctmp
}
%    \end{macrocode}

% \macro{\childdocforwardprefix}
% The command |\childdocforwardprefix| redirects
% compilation to the main or a child file by means of a pattern.
% The prefix |#1| in the current filename is replaced by |#2|
% and the suffix of the current filename is kept
% (it is assumed that the filename does not contain the substring `|~~~|'
% which is used as a delimiter).
% Compilation is handed over to the new file by |\childdocforward|:
%    \begin{macrocode}
\newcommand{\childdocforwardprefix}[3][]
{
  \begingroup
    \def\childdocextract #2##1~~~{\def\childdoctmp{\childdocforward[#1]{#3##1}}}
    \expandafter\childdocextract\childdocname~~~
    \expandafter
  \endgroup
  \childdoctmp
}
%    \end{macrocode}

% \macro{\childdoc}
% The deprecated macro |\childdoc| is a legacy version of |\childdocmain|:
%    \begin{macrocode}
\newcommand{\childdoc}{\childdocmain}
%    \end{macrocode}

% \macro{\childdocredirect}
% The deprecated macro |\childdocredirect| is a legacy version
% of |\childdocforward| and |\childdocforwardprefix|:
%    \begin{macrocode}
\newcommand{\childdocredirect}[2][]
{
  \begingroup
    \if?#1?
      \def\childdoctmp{\childdocforward{#2}}
    \else
      \def\childdoctmp{\childdocforwardprefix{#1}{#2}}
    \fi
    \expandafter
  \endgroup
  \childdoctmp
}
%    \end{macrocode}

%\iffalse
%</package>
%\fi
%
\endinput
|\\
|\childdocforward{|\textit{main}|}|
\end{tabular}
\end{center}
%
Likewise, the following files |final|\textit{nn}|.tex|
compile the final version of the child document
|child|\textit{nn}|.tex|:
%
\begin{center}
\begin{tabular}{l}
|\def\version{final}|\\
|% \iffalse
%
% childdoc.dtx Copyright (C) 2017-2018 Niklas Beisert
%
% This work may be distributed and/or modified under the
% conditions of the LaTeX Project Public License, either version 1.3
% of this license or (at your option) any later version.
% The latest version of this license is in
%   http://www.latex-project.org/lppl.txt
% and version 1.3 or later is part of all distributions of LaTeX
% version 2005/12/01 or later.
%
% This work has the LPPL maintenance status `maintained'.
%
% The Current Maintainer of this work is Niklas Beisert.
%
% This work consists of the files childdoc.dtx and childdoc.ins
% and the derived files childdoc.def and cdocsamp.tex with
% cdocsch1.tex, cdocsch2.tex, cdocsdrf.tex, cdocsfn1.tex, cdocsfn2.tex.
%
%<package>\ifdefined\childdocmain\endinput\fi
%<package>\ProvidesFile{childdoc.def}[2018/12/30 v2.0 child document driver]
%<samplemain>\ProvidesFile{cdocsamp.tex}[2018/12/30 v2.0 sample for childdoc]
%<*driver>
%\ProvidesFile{childdoc.drv}[2018/12/30 v2.0 childdoc reference manual file]
\PassOptionsToClass{10pt,a4paper}{article}
\documentclass{ltxdoc}

\usepackage[margin=35mm]{geometry}
\usepackage{hyperref}
\usepackage{hyperxmp}
\usepackage[usenames]{color}

\hypersetup{colorlinks=true}
\hypersetup{pdfstartview=FitH}
\hypersetup{pdfpagemode=UseNone}
\hypersetup{pdfsource={}}
\hypersetup{pdflang={en-UK}}
\hypersetup{pdfcopyright={Copyright 2017-2018 Niklas Beisert.
  This work may be distributed and/or modified under the
  conditions of the LaTeX Project Public License, either version 1.3
  of this license or (at your option) any later version.}}
\hypersetup{pdflicenseurl={http://www.latex-project.org/lppl.txt}}
\hypersetup{pdfcontactaddress={ETH Zurich, ITP, HIT K,
  Wolfgang-Pauli-Strasse 27}}
\hypersetup{pdfcontactpostcode={8093}}
\hypersetup{pdfcontactcity={Zurich}}
\hypersetup{pdfcontactcountry={Switzerland}}
\hypersetup{pdfcontactemail={nbeisert@itp.phys.ethz.ch}}
\hypersetup{pdfcontacturl={http://people.phys.ethz.ch/\xmptilde nbeisert/}}

\newcommand{\secref}[1]{\hyperref[#1]{section \ref*{#1}}}

\parskip1ex
\parindent0pt
\let\olditemize\itemize
\def\itemize{\olditemize\parskip0pt}

\begin{document}

\title{The \textsf{childdoc} Package}
\hypersetup{pdftitle={The childdoc Package}}
\author{Niklas Beisert\\[2ex]
  Institut f\"ur Theoretische Physik\\
  Eidgen\"ossische Technische Hochschule Z\"urich\\
  Wolfgang-Pauli-Strasse 27, 8093 Z\"urich, Switzerland\\[1ex]
  \href{mailto:nbeisert@itp.phys.ethz.ch}
  {\texttt{nbeisert@itp.phys.ethz.ch}}}
\hypersetup{pdfauthor={Niklas Beisert}}
\hypersetup{pdfsubject={Manual for the LaTeX2e Package childdoc}}
\date{30 December 2018, \textsf{v2.0}}
\maketitle

\begin{abstract}\noindent
\textsf{childdoc} is a \LaTeXe{} package
that enables the direct compilation
of document sections included by |\include|
to individual files.
\end{abstract}

\begingroup
\parskip0ex
\tableofcontents
\endgroup

%%%%%%%%%%%%%%%%%%%%%%%%%%%%%%%%%%%%%%%%%%%%%%%%%%%%%%%%%%%%%%%%%%%%%%%%%%%%%%%%
%%%%%%%%%%%%%%%%%%%%%%%%%%%%%%%%%%%%%%%%%%%%%%%%%%%%%%%%%%%%%%%%%%%%%%%%%%%%%%%%
\section{Introduction}

\LaTeX{} provides a mechanism to structure a large document (such as a book)
into a main file and several child files (containing the chapters)
using the |\include| command.
This mechanism is beneficial for documents
which span hundreds of pages in order to
make the source file(s) more manageable.
Moreover, compilation can be restricted to
selected child files by means of the |\includeonly| command.
The latter feature can be used to reduce the compilation time while editing
(this was significantly more useful in the earlier days of \LaTeX{})
or to generate a smaller document which is easier to navigate.
Another application of |\includeonly| is to generate
documents consisting of selected parts of the complete document.

However, there are a few drawbacks of the plain |\include| mechanism:
\begin{itemize}
\item
The child files cannot be compiled on their own,
they can only be compiled via the main file.
A naive editing environment
(such as a text editor with an option
to have the current file processed by \LaTeX)
may require one to switch to the main file before compiling;
attempting to compile the child file produces errors.
\item
The main file must be modified (each time)
to adjust the |\includeonly| command
to the present needs. This easily leaves the main file in a messy state.
\item
The generated document will always carry the filename
of the main document. This is inconvenient if
several child files are to be compiled and
to be kept for distribution.
\end{itemize}

The present package provides a simple interface
to make child files individually compilable by \LaTeX{}.
Compiling a child file then has the same effect as compiling
the main file with an |\includeonly| command
to select the appropriate child.
Moreover the generated document will carry the name of the child
rather than the main file.
This resolves all three above issues.

This feature is meant to make the editing of books,
thesis documents and lecture notes somewhat more convenient.
However, the package can also be used efficiently for
composing a series of documents (such as exercise sheets)
which are typically distributed individually.
It then assists the author in generating the individual documents
(potentially in different versions)
as well as a document containing the collected series.
Another application is in developing style files
or other kinds of included material
where compilation of the style file could redirect
to a sample or test file.

%%%%%%%%%%%%%%%%%%%%%%%%%%%%%%%%%%%%%%%%%%%%%%%%%%%%%%%%%%%%%%%%%%%%%%%%%%%%%%%%
%%%%%%%%%%%%%%%%%%%%%%%%%%%%%%%%%%%%%%%%%%%%%%%%%%%%%%%%%%%%%%%%%%%%%%%%%%%%%%%%
\section{Usage}

First of all, the package \textsf{childdoc} is \emph{not} a standard
\LaTeXe{} |.sty| style file! Therefore it needs to be invoked in
a non-standard way.

%%%%%%%%%%%%%%%%%%%%%%%%%%%%%%%%%%%%%%%%%%%%%%%%%%%%%%%%%%%%%%%%%%%%%%%%%%%%%%%%
\subsection{Included Files}
\label{sec:include}

%%%%%%%%%%%%%%%%%%%%%%%%%%%%%%%%%%%%%%%%
\DescribeMacro{\childdocmain}
To use the package, add the commands
\begin{center}
\begin{tabular}{l}
|\input{childdoc.def}|\\
|\childdocmain{}|\\
\end{tabular}
\end{center}
at the very top of the main \LaTeX{} file,
in particular \emph{before} the |\documentclass| statement!
The argument of |\childdocmain| should be left empty
(but it must be present).

%%%%%%%%%%%%%%%%%%%%%%%%%%%%%%%%%%%%%%%%
\DescribeMacro{\childdocof}
Furthermore, add the commands
\begin{center}
\begin{tabular}{l}
|\input{childdoc.def}|\\
|\childdocof{|\textit{main}|}|\\
\end{tabular}
\end{center}
at the top of every child file \textit{child}
which is included by |\include{|\textit{child}|}|
from within the main file
(or at least for those files to be compiled individually).
The argument \textit{main} must be the filename of the main file.

There are a couple of
considerations in setting up the main and child documents:

%%%%%%%%%%%%%%%%%%%%%%%%%%%%%%%%%%%%%%%%
\paragraph{Restrictions.}

Please note the following restrictions:
\begin{itemize}
\item
|\childdocmain| must be called with one argument \textit{main}
to ensure compatibility with earlier version of the package.
It must either be empty (|\childdocmain{}|)
or precisely match the filename of the main file in which it is specified.
See \secref{sec:detection} for further information.
\item
The filename \textit{main} must be specified without the |.tex| extension.
\item
The filename \textit{main} is case sensitive
(even in case-insensitive file systems)
due to internal string comparison.
\item
The argument \textit{main} should be fully expanded, it cannot be a macro.
\item
Subdirectories and special characters should be avoided in filenames.
\item
The command |\childdocmain{|\textit{main}|}| must be followed by a whitespace.
It should not be followed immediately by another command
or by a comment mark `|%|'.
This is because the \TeX{} parser reads the token immediately following
the argument of |\childdocmain| and puts it
at the beginning of every child section;
however, a white\-space is ignored.
\end{itemize}

%%%%%%%%%%%%%%%%%%%%%%%%%%%%%%%%%%%%%%%%
\paragraph{Content of Main File.}

It is advisable to place all content in the child files included by |\include|.
Any output contained in the main file will appear in all child documents
unless suppressed manually;
it cannot be suppressed automatically by the |\includeonly| directive
and thus should normally be avoided.
A method to include some content in the main file
by means of conditional processing is described in \secref{sec:conditional}.

%%%%%%%%%%%%%%%%%%%%%%%%%%%%%%%%%%%%%%%%
\paragraph{Page Numbering.}

When only a part of the document is compiled,
the appropriate numbering of pages
(as well as other status parameters)
is determined from the |.aux| files.
The latter contain information from previous passes.
However this information needs to propagate through
all intermediate child documents.
Therefore the page numbering in child documents may well
be inconsistent until the complete document is compiled at least once.

A useful (if unconventional) way to always ensure a consistent
page numbering is to restart the numbering in each child document
and denote the pages by `\textit{child}|.|\textit{page}'
where \textit{child} represents the chapter/section number of the child file.
This can be achieved by the command
|\numberwithin{page}{|\textit{child}|}|
of the \textsf{amsmath} package
where \textit{child} can be |chapter| or |section|
depending on the chosen structuring.
Alternatively, one can modify the macro |\thepage| appropriately
and reset the counter |page| at the start of each child file.

%%%%%%%%%%%%%%%%%%%%%%%%%%%%%%%%%%%%%%%%%%%%%%%%%%%%%%%%%%%%%%%%%%%%%%%%%%%%%%%%
\subsection{Conditional Processing}
\label{sec:conditional}

The package provides a mechanism to compile different versions
of a document. To customise the versions further some conditional processing
can come in handy to distinguish which version is being compiled.
The package provides two macros to describe the compilation context:

%%%%%%%%%%%%%%%%%%%%%%%%%%%%%%%%%%%%%%%%
\DescribeMacro{\ifchilddoc}
The conditional |\ifchilddoc| distinguishes between the compilation of
child documents and the main document:
%
\begin{center}
|\ifchilddoc |\textit{child-code}| |[|\||else |\textit{main-code}]| \||fi|
\end{center}

%%%%%%%%%%%%%%%%%%%%%%%%%%%%%%%%%%%%%%%%
\DescribeMacro{\childdocname}
\DescribeMacro{\childdocjob}
The macro |\childdocname| contains the filename (without extension)
of the main or child file being processed.
Note that |\childdocjob| will always contain the name of the main file.

%%%%%%%%%%%%%%%%%%%%%%%%%%%%%%%%%%%%%%%%
\paragraph{Title Page.}

Conditional processing can be used to include a title or banner page
in the main document when proper precautions are taken.
Importantly, the code in the main file should ensure that the page counter
(as well as other status parameters which are stored in the |.aux| files)
takes the same value after the conditional processing.
Otherwise the page numbers may take divergent values
depending on which part is compiled.

For example, a title page could be declared by:
%
\begin{center}
\begin{tabular}{l}
|\ifchilddoc\||else|\\
|\addtocounter{page}{-1}|\\
\textit{code for title page}\\
|\newpage|\\
|\||fi|
\end{tabular}
\end{center}
%
A banner page for the child documents can be generated by:
%
\begin{center}
\begin{tabular}{l}
|\ifchilddoc|\\
|\addtocounter{page}{-1}|\\
\textit{code for banner page}\\
|\newpage|\\
|\||fi|
\end{tabular}
\end{center}
%
Here one could write a message such as:
\begin{center}
|This is the part \childdocname{} of \childdocjob{}.|
\end{center}

%%%%%%%%%%%%%%%%%%%%%%%%%%%%%%%%%%%%%%%%%%%%%%%%%%%%%%%%%%%%%%%%%%%%%%%%%%%%%%%%
\subsection{Flags}
\label{sec:flags}

The package makes it easy to generate different versions
of the main or child documents.
To this end compilation flags can be defined
and assigned different default values.
They will be particularly useful in conjunction
with the forwarding mechanism described in \secref{sec:forward}.

For example, it may be useful to have a flag |\version|
which can be set to |draft| or |final|.
The document source will contain some conditional code
depending on the value of |\version|.
Suppose further, the flag should default to |final| for the main file
and to |draft| for child files
which is a natural assignment for editing the document.
This is achieved by placing the following code
in the preamble of the main document
(below the |\childdocmain| directive):
%
\begin{center}
\begin{tabular}{l}
|\ifchilddoc|\\
|\providecommand{\version}{draft}|\\
|\||else|\\
|\providecommand{\version}{final}|\\
|\||fi|
\end{tabular}
\end{center}
%
The definition by |\providecommand| makes sure
that previous definitions are not overwritten.
Further statements |\providecommand{\version}{...}|
can thus be added before the above code to override it.

For the main file, one might add a line
(between |\childdocmain| and the above block)
%
\begin{center}
|%\ifchilddoc\||else\providecommand{\version}{draft}\||fi|
\end{center}
%
which can be uncommented to produce a draft version.
Likewise one can add a line to the very top of a child file
(above the |\childdocof{|\textit{main}|}| directive)
%
\begin{center}
|%\providecommand{\version}{final}|
\end{center}
%
which can be uncommented to produce the final version of this child document.

%%%%%%%%%%%%%%%%%%%%%%%%%%%%%%%%%%%%%%%%%%%%%%%%%%%%%%%%%%%%%%%%%%%%%%%%%%%%%%%%
\subsection{Forwarding}
\label{sec:forward}

Different versions of the main or child documents
using compilation flags as described in \secref{sec:flags}
can be (permanently) stored in different files
for convenient compilation, viewing and distribution.
To this end, the package defines a command
to pass on compilation to a different file:

%%%%%%%%%%%%%%%%%%%%%%%%%%%%%%%%%%%%%%%%
\DescribeMacro{\childdocforward}
The command |\childdocforward| redirects processing to
another source file:
%
\begin{center}
\begin{tabular}{l}
|\input{childdoc.def}|\\
|\childdocforward[|\textit{main}|]{|\textit{dest}|}|\\
\end{tabular}
\end{center}
%
The argument \textit{dest} is the destination file
(without extension).
It should be the main file or one of the child files.
Note that further \textsf{childdoc} directives
such as |\childdocof| and |\childdocforward|
in the indicated file will be processed in this form.
The optional argument \textit{main}
passes on directly to the main file \textit{main}
while pretending to compile the child \textit{dest}.
This form behaves as if \textit{dest}
issues |\childdocof{|\textit{main}|}| right away,
and no further \textsf{childdoc} directives will be processed.

%%%%%%%%%%%%%%%%%%%%%%%%%%%%%%%%%%%%%%%%
\DescribeMacro{\...prefix}
In the alternative form |\childdocforwardprefix|,
%
\begin{center}
\begin{tabular}{l}
|\input{childdoc.def}|\\
|\childdocforwardprefix[|\textit{main}|]{|\textit{prefix}|}{|\textit{dest}|}|
\end{tabular}
\end{center}
%
the destination file is determined by a pattern
depending on the current file:
To make this work, the current file must be called
`{\textit{prefix}\hspace{0.2em}\textit{suffix}}'
with \textit{prefix} matching precisely the argument.
Processing is then passed on to the file
`{\textit{dest}\hspace{0.2em}\textit{suffix}}'.
Surely, the same effect is achieved by
directly specifying the
argument `{\textit{dest}\hspace{0.2em}\textit{suffix}}'
in the first form.
However, that requires to set up a different file
for each child. With the alternative form of the command
all these files can have exactly the same content
which simplifies setting them up and maintaining them.

For example, the following file |draft.tex|
with a compilation flag |\version| as described in \secref{sec:flags}
compiles the main document as a draft:
%
\begin{center}
\begin{tabular}{l}
|\def\version{draft}|\\
|\input{childdoc.def}|\\
|\childdocforward{|\textit{main}|}|
\end{tabular}
\end{center}
%
Likewise, the following files |final|\textit{nn}|.tex|
compile the final version of the child document
|child|\textit{nn}|.tex|:
%
\begin{center}
\begin{tabular}{l}
|\def\version{final}|\\
|\input{childdoc.def}|\\
|\childdocforwardprefix{final}{child}|
\end{tabular}
\end{center}
%

Note that when several versions of a main file and/or of each child file
are to be generated, it may be convenient to set up a |Makefile| or
shell script to automatise the process.

%%%%%%%%%%%%%%%%%%%%%%%%%%%%%%%%%%%%%%%%%%%%%%%%%%%%%%%%%%%%%%%%%%%%%%%%%%%%%%%%
\subsection{Command Line Processing}
\label{sec:commandline}

The effect of redirection files can also be achieved by invoking
the \LaTeX{} compiler with a more elaborate command line.
Most conveniently this should be done as part
of a shell script or a |Makefile|.

When using \textsf{childdoc} in the main file, the following
command lines effectively perform a redirection
(note that depending on the shell being used,
backslashes may have to be doubled: `|\|' $\to$ `|\\|'):
%
\begin{center}
|... -jobname "|\textit{target}|" |\\|"|[\textit{flags}]%
|\input{childdoc.def}\childdocforward[|\textit{main}|]{|\textit{dest}|}"|
\end{center}
%
Here \textit{target} is the name of the output file,
\textit{main} is the name of the main file
and \textit{dest} is the name of the main or child file to be processed
(all filenames without extensions).
The optional argument \textit{main} can be omitted
if \textit{main} matches \textit{dest}.
Optionally, compilation \textit{flags} can be defined via |\def| commands.
This command line makes the \TeX{} engine believe
it is compiling the file \textit{target}
whose content is specified as the latter parameter.
The provided code then forwards the processing to
\textit{main} or \textit{dest} as described in \secref{sec:forward}.

%%%%%%%%%%%%%%%%%%%%%%%%%%%%%%%%%%%%%%%%%%%%%%%%%%%%%%%%%%%%%%%%%%%%%%%%%%%%%%%%
\subsection{Include by Input}
\label{sec:input}

Including child documents by |\include| has some restrictions by design.
Most notably, the content of a child document always occupies
its own set of pages; pages cannot be shared between child documents.
Usually, this behaviour makes perfect sense
because each child document contain an essential part of the document.
However, in some situations it may be desirable to compose
a document from a collection of parts
without having mandatory page breaks between then.
For this case, the package
provides a mechanism to include parts
by |\input| which can also be processed individually.
However, by construction this mechanism
requires manual handling of the content to be output.

%%%%%%%%%%%%%%%%%%%%%%%%%%%%%%%%%%%%%%%%
\DescribeMacro{\ifchilddocmanual}
The main file should be prepared as usual, see \secref{sec:include}.
However, the document body must make a distinction
between processing of an individual part and of the main document, e.g.:
%
\begin{center}
\begin{tabular}{l}
|\ifchilddocmanual|\\
|\input{\childdocname}|\\
|\||else|\\
\textit{document body with }|\input{|\textit{part}|}|\\
|\||fi|
\end{tabular}
\end{center}
%
The conditional |\ifchilddocmanual| is true whenever
a part to be included by |\input| is being compiled,
and the name of the part is stored in |\childdocname|.

%%%%%%%%%%%%%%%%%%%%%%%%%%%%%%%%%%%%%%%%
\DescribeMacro{\childdocby}
Each part to be included by |\input| should start with:
%
\begin{center}
\begin{tabular}{l}
|\input{childdoc.def}|\\
|\childdocby{|\textit{main}|}|\\
\end{tabular}
\end{center}
%
The directive |\childdocby| is similar to |\childdocof|
described in \secref{sec:include},
but the subsequent selection of content must be done manually.
To that end, both |\ifchilddoc| and |\ifchilddocmanual|
will be true upon processing of a part,
and the name of the part is stored in |\childdocname|.
Note that |\jobname| will be set to the filename of the current part
so that each part receives an individual |.aux| file
that does not interfere with the |.aux| file(s) of the main document.
This behaviour can be altered by the alternative form
|\childdocby[*]{|\textit{main}|}| (with a non-empty optional argument)
which uses the |.aux| file of the main document
by setting |\jobname| to \textit{main}.

%%%%%%%%%%%%%%%%%%%%%%%%%%%%%%%%%%%%%%%%%%%%%%%%%%%%%%%%%%%%%%%%%%%%%%%%%%%%%%%%
\subsection{Driver Development}
\label{sec:driver}

The \textsf{childdoc} mechanism can also be use for the development
of definition files such as \LaTeX{} styles or classes.
This case differs from the above setup with multiple parts
included by |\include| in that no |\includeonly| should be invoked.
This can be achieved by starting the include file
(before |\ProvidesPackage|) with:
%
\begin{center}
\begin{tabular}{l}
|\input{childdoc.def}|\\
|\childdocforward{|\textit{main}|}|\\
\end{tabular}
\end{center}
%
or alternatively with:
%
\begin{center}
\begin{tabular}{l}
|\input{childdoc.def}|\\
|\childdocby{|\textit{main}|}|\\
\end{tabular}
\end{center}
%
Both forms have slightly different effects as described above.
The main file is prepared as usual, see \secref{sec:include}.

%%%%%%%%%%%%%%%%%%%%%%%%%%%%%%%%%%%%%%%%%%%%%%%%%%%%%%%%%%%%%%%%%%%%%%%%%%%%%%%%
\subsection{Legacy Detection}
\label{sec:detection}

The directive |\childdocmain| in the main file can detect
whether the complete document or merely a child is to be compiled
even without using the directive |\childdocof|.
This method is deprecated because it is less robust
and there is no compelling reason to use it;
it is merely provided for backward compatibility
and it may be removed in future versions.

If the detection mechanism is to be used,
it is mandatory to correctly specify
the filename of the main file as the argument of |\childdocmain|:
%
\begin{center}
\begin{tabular}{l}
|\input{childdoc.def}|\\
|\childdocmain{|\textit{main}|}|\\
\end{tabular}
\end{center}
%
If |\jobname| does not match the argument \textit{main} of |\childdocmain|,
it is assumed that |\jobname| points to the child file to be compiled.
When using |\childdocmain| with the main file specified as argument,
it suffices to start a child file
with just |\input{|\textit{main}|}|
without loading of the package and using |\childdocof|.
If instead all processing is done
with the appropriate \textsf{childdoc} directives,
the argument of \textit{main} of |\childdocmain| can be empty.

An alternative version of the command line processing described
in \secref{sec:commandline} using the detection mechanism reads:
%
\begin{center}
|... -jobname "|\textit{target}|" "|[\textit{flags}]%
[|\def\jobname{|\textit{dest}|}|]|\input{|\textit{main}|}"|
\end{center}

%%%%%%%%%%%%%%%%%%%%%%%%%%%%%%%%%%%%%%%%%%%%%%%%%%%%%%%%%%%%%%%%%%%%%%%%%%%%%%%%
\subsection{Manual Code}
\label{sec:manual}

In case one cannot be certain whether the definitions file |childdoc.def|
is installed on the target \TeX{} distribution
and one prefers not to ship it,
it is conceivable to paste a few relevant commands into the sources.

To that end, drop all statements |\input{childdoc.def}|
and perform the replacements as outlined below.
Instead of |\childdocmain{|\textit{main}|}| add the following code
to the top of the main file:
%
\begin{center}
\begin{tabular}{l}
|\||ifdefined\childdocname\endinput\||fi\newif\ifchilddoc|\\
|\edef\childdocname{\scantokens\expandafter{\jobname\noexpand}}|\\
|\def\childdocmain{|\textit{main}|}\||ifx\childdocmain\childdocname\||else|\\
|\childdoctrue\includeonly{\childdocname}\let\jobname\childdocmain\||fi|\\
\end{tabular}
\end{center}
%
Instead of |\childdocof{|\textit{main}|}| just include the main file
at the top of each child file:
%
\begin{center}
|\input{|\textit{main}|}|
\end{center}
%
A simple redirection |\childdocforward{|\textit{dest}|}| is achieved by:
%
\begin{center}
|\def\jobname{|\textit{dest}|}\input{\jobname}|
\end{center}
%
The redirection with prefix
|\childdocforwardprefix[|\textit{prefix}|]{|\textit{dest}|}|
is accomplished by:
%
\begin{center}
\begin{tabular}{l}
|{\edef\jobname{\scantokens\expandafter{\jobname\noexpand}}|\\
|\def\redirectjob |\textit{prefix}|#1~~~{\gdef\jobname{|\textit{dest}|#1}}|\\
|\expandafter\redirectjob\jobname~~~}\input{\jobname}|
\end{tabular}
\end{center}

In an alternative approach,
child documents can be compiled by a specific command line
without additional code or specific definitions:
%
\begin{center}
|... -jobname "|\textit{target}|" "|[\textit{flags}]%
|\includeonly{|\textit{dest}|}\input{|\textit{main}|}"|
\end{center}
%

%%%%%%%%%%%%%%%%%%%%%%%%%%%%%%%%%%%%%%%%%%%%%%%%%%%%%%%%%%%%%%%%%%%%%%%%%%%%%%%%
%%%%%%%%%%%%%%%%%%%%%%%%%%%%%%%%%%%%%%%%%%%%%%%%%%%%%%%%%%%%%%%%%%%%%%%%%%%%%%%%
\section{Information}

%%%%%%%%%%%%%%%%%%%%%%%%%%%%%%%%%%%%%%%%%%%%%%%%%%%%%%%%%%%%%%%%%%%%%%%%%%%%%%%%
\subsection{Copyright}

Copyright \copyright{} 2017--2018 Niklas Beisert

This work may be distributed and/or modified under the
conditions of the \LaTeX{} Project Public License, either version 1.3
of this license or (at your option) any later version.
The latest version of this license is in
  \url{http://www.latex-project.org/lppl.txt}
and version 1.3 or later is part of all distributions of \LaTeX{}
version 2005/12/01 or later.

This work has the LPPL maintenance status `maintained'.

The Current Maintainer of this work is Niklas Beisert.

This work consists of the files |README.txt|, |childdoc.ins| and |childdoc.dtx|
as well as the derived files |childdoc.def|, |cdocsamp.tex|
with |cdocsch1.tex|, |cdocsch2.tex|, |cdocspt3.tex|, |cdocspt4.tex|,
|cdocsdrf.tex|, |cdocsfn1.tex|, |cdocsfn2.tex|
as well as |childdoc.pdf|.

%%%%%%%%%%%%%%%%%%%%%%%%%%%%%%%%%%%%%%%%%%%%%%%%%%%%%%%%%%%%%%%%%%%%%%%%%%%%%%%%
\subsection{Files and Installation}

The package consists of the files:
%
\begin{center}
\begin{tabular}{ll}
    |README.txt|   & readme file \\
    |childdoc.ins| & installation file \\
    |childdoc.dtx| & source file \\
    |childdoc.def| & definition file \\
    |cdocsamp.tex| & sample main file \\
    |cdocsch1.tex| & sample include file \\
    |cdocsch2.tex| & sample include file \\
    |cdocspt3.tex| & sample part file \\
    |cdocspt4.tex| & sample part file \\
    |cdocsdrf.tex| & sample redirection file \\
    |cdocsfn1.tex| & sample redirection file \\
    |cdocsfn2.tex| & sample redirection file \\
    |childdoc.pdf| & manual
\end{tabular}
\end{center}
%
The distribution consists of the files
|README.txt|, |childdoc.ins| and |childdoc.dtx|.
%
\begin{itemize}
\item
Run (pdf)\LaTeX{} on |childdoc.dtx|
to compile the manual |childdoc.pdf| (this file).
\item
Run \LaTeX{} on |childdoc.ins| to create the definitions file |childdoc.def|
and the sample |cdocsamp.tex| with include files
|cdocsch1.tex|, |cdocsch2.tex|, |cdocspt3.tex|, |cdocspt4.tex|,
|cdocsdrf.tex|, |cdocsfn1.tex|, |cdocsfn2.tex|.
Then copy the file |childdoc.def| to an appropriate directory of your \LaTeX{}
distribution, e.g.\ \textit{texmf-root}|/tex/latex/childdoc|.
\end{itemize}

%%%%%%%%%%%%%%%%%%%%%%%%%%%%%%%%%%%%%%%%%%%%%%%%%%%%%%%%%%%%%%%%%%%%%%%%%%%%%%%%
\subsection{Related CTAN Packages}

There are several other packages which offer a similar functionality:
%
\begin{itemize}
\item
The packages
\href{http://ctan.org/pkg/docmute}{\textsf{docmute}},
\href{http://ctan.org/pkg/includex}{\textsf{includex}} and
\href{http://ctan.org/pkg/standalone}{\textsf{standalone}}
provide commands to include only the document body of
a child file thus allowing both files to be compiled individually.
\item
The packages \href{http://ctan.org/pkg/subdocs}{\textsf{subdocs}}
and \href{http://ctan.org/pkg/subfiles}{\textsf{subfiles}}
provide structures in which the main and child documents can be
encapsulated and allowing them to be compiled individually.
The inclusion mechanism is different from the conventional |\include|.
\item
The package \href{http://ctan.org/pkg/combine}{\textsf{combine}}
is an elaborate solution to combine several documents into one.
\end{itemize}
%
See also the CTAN topic \href{http://ctan.org/topic/subdocs}{\textsf{subdocs}}
for further related packages.
The present package differs from the above solutions in that
a document structure constructed with the conventional |\include| mechanism
just needs two extra commands at the top of every file
such that all constituent files can be compiled individually.

%%%%%%%%%%%%%%%%%%%%%%%%%%%%%%%%%%%%%%%%%%%%%%%%%%%%%%%%%%%%%%%%%%%%%%%%%%%%%%%%
%\subsection{Feature Suggestions}
%
%The following is a list of features which may be useful for future
%versions of this package:
%%
%\begin{itemize}
%\item
%\ldots
%\end{itemize}

%%%%%%%%%%%%%%%%%%%%%%%%%%%%%%%%%%%%%%%%%%%%%%%%%%%%%%%%%%%%%%%%%%%%%%%%%%%%%%%%
\subsection{Revision History}

%%%%%%%%%%%%%%%%%%%%%%%%%%%%%%%%%%%%%%%%
\paragraph{v2.0:} 2018/12/30

\begin{itemize}
\item
immediate forward processing
\item
added |\childdocby| mechanism
\item
manual restructured
\end{itemize}

%%%%%%%%%%%%%%%%%%%%%%%%%%%%%%%%%%%%%%%%
\paragraph{v1.6:} 2018/01/17

\begin{itemize}
\item
application for development of include files
\item
corrections to manual
\end{itemize}

%%%%%%%%%%%%%%%%%%%%%%%%%%%%%%%%%%%%%%%%
\paragraph{v1.5:} 2017/05/21

\begin{itemize}
\item
more complete structuring introduced
\item
|\childdocof| introduced
\item
|\childdoc| renamed to |\childdocmain|
\item
|\childredirect| renamed to |\childdocforward| and |\childdocforwardprefix|
and functionality expanded
\end{itemize}

%%%%%%%%%%%%%%%%%%%%%%%%%%%%%%%%%%%%%%%%
\paragraph{v1.0:} 2017/04/27

\begin{itemize}
\item
manual and install package
\item
first version published on CTAN
\end{itemize}

%%%%%%%%%%%%%%%%%%%%%%%%%%%%%%%%%%%%%%%%
\paragraph{v0.6:} 2017/04/26

\begin{itemize}
\item
redirection mechanism added
\end{itemize}

%%%%%%%%%%%%%%%%%%%%%%%%%%%%%%%%%%%%%%%%
\paragraph{v0.5:} 2017/04/26

\begin{itemize}
\item
functionality in definition file
\end{itemize}


%%%%%%%%%%%%%%%%%%%%%%%%%%%%%%%%%%%%%%%%%%%%%%%%%%%%%%%%%%%%%%%%%%%%%%%%%%%%%%%%
%%%%%%%%%%%%%%%%%%%%%%%%%%%%%%%%%%%%%%%%%%%%%%%%%%%%%%%%%%%%%%%%%%%%%%%%%%%%%%%%
%%%%%%%%%%%%%%%%%%%%%%%%%%%%%%%%%%%%%%%%%%%%%%%%%%%%%%%%%%%%%%%%%%%%%%%%%%%%%%%%
\appendix

\settowidth\MacroIndent{\rmfamily\scriptsize 000\ }

 \DocInput{childdoc.dtx}

\end{document}
%</driver>
% \fi
%
% %%%%%%%%%%%%%%%%%%%%%%%%%%%%%%%%%%%%%%%%%%%%%%%%%%%%%%%%%%%%%%%%%%%%%%%%%%%%%%
% %%%%%%%%%%%%%%%%%%%%%%%%%%%%%%%%%%%%%%%%%%%%%%%%%%%%%%%%%%%%%%%%%%%%%%%%%%%%%%
% \section{Sample}
%\iffalse
%<*samplemain>
%\fi
%
% The following presents a sample document
% with two chapters, two parts, a title page,
% a compile flag as well as three forwarding files to set the flag.
% It consists of eight |.tex| files:
% \begin{center}
% \begin{tabular}{ll}
% |cdocsamp.tex|&main file\\
% |cdocsch1.tex|&include file for chapter 1\\
% |cdocsch2.tex|&include file for chapter 2\\
% |cdocspt3.tex|&include file for part 3\\
% |cdocspt4.tex|&include file for part 4\\
% |cdocsdrf.tex|&forwarding file for main file in draft mode\\
% |cdocsfi1.tex|&forwarding file for final version of chapter 1\\
% |cdocsfi2.tex|&forwarding file for final version of chapter 2\\
% \end{tabular}
% \end{center}
% Each of the eight files can be compiled directly by the \LaTeX{} compiler.
%
% %%%%%%%%%%%%%%%%%%%%%%%%%%%%%%%%%%%%%%
% \paragraph{Main File.}
%
% The main file is called |cdocsamp.tex|.
%
% Load the \textsf{childdoc} definitions and
% declare the filename for the main document:
%    \begin{macrocode}
\input{childdoc.def}
\childdocmain{}
%    \end{macrocode}

% Optional override for |\version| flag:
%    \begin{macrocode}
%%\ifchilddoc\else\providecommand{\version}{draft}\fi
%    \end{macrocode}

% Define the default values for the |\version| flag
% (|final| for the main file and |draft| for childs):
%    \begin{macrocode}
\ifchilddoc
\providecommand{\version}{draft}
\else
\providecommand{\version}{final}
\fi
%    \end{macrocode}

% Load the standard document class:
%    \begin{macrocode}
\documentclass[12pt]{article}
%    \end{macrocode}

% Start the document body:
%    \begin{macrocode}
\begin{document}
%    \end{macrocode}

% Declare a title page.
% Print title, part of document being processed and version flag:
%    \begin{macrocode}
\addtocounter{page}{-1}
\begin{center}
{\LARGE\bfseries{}childdoc example\par}
\vspace{1cm}
\ifchilddoc
\ifchilddocmanual part\else chapter\fi:
`\childdocname' of `\childdocjob'\par
\else
main document: `\childdocjob'\par
\fi
version: \version\par
\end{center}
\newpage
%    \end{macrocode}

% Manually include selected file,
% otherwise process as usual:
%    \begin{macrocode}
\ifchilddocmanual
\section*{part `\childdocname'}
\input{\childdocname}
\else
%    \end{macrocode}

% Include the two chapters:
%    \begin{macrocode}
\include{cdocsch1}
\include{cdocsch2}
%    \end{macrocode}

% Include the two parts unless only chapters should be displayed:
%    \begin{macrocode}
\ifchilddoc\else
\section{part three}
\input{cdocspt3}
\section{part four}
\input{cdocspt4}
\fi
%    \end{macrocode}

% Process as usual until here:
%    \begin{macrocode}
\fi
%    \end{macrocode}

% End of document body:
%    \begin{macrocode}
\end{document}
%    \end{macrocode}
%\iffalse
%</samplemain>
%\fi
%
% %%%%%%%%%%%%%%%%%%%%%%%%%%%%%%%%%%%%%%
% \paragraph{Chapter Include Files.}
%
% The include files are called |cdocsch1.tex| and |cdocsch2.tex|.
%
%\iffalse
%<*samplechap1|samplechap2>
%\fi

% Optional override for |\version| flag:
%    \begin{macrocode}
%%\providecommand{\version}{final}
%    \end{macrocode}

% Include the main document:
%    \begin{macrocode}
\input{childdoc.def}
\childdocof{cdocsamp}
%    \end{macrocode}

%\iffalse
%</samplechap1|samplechap2>
%\fi
%
%\iffalse
%<*samplechap1>
%\fi
% Some text for chapter 1:
%    \begin{macrocode}
\section{one}
some text in chapter one
%    \end{macrocode}

%\iffalse
%</samplechap1>
%\fi
% Some text for chapter 2:
%\iffalse
%<*samplechap2>
%\fi
%    \begin{macrocode}
\section{two}
more text in chapter two
%    \end{macrocode}

%\iffalse
%</samplechap2>
%\fi
%
% %%%%%%%%%%%%%%%%%%%%%%%%%%%%%%%%%%%%%%
% \paragraph{Part Include Files.}
%
% The include files are called |cdocspt3.tex| and |cdocspt4.tex|.
%
%\iffalse
%<*samplepart3|samplepart4>
%\fi

% Optional override for |\version| flag:
%    \begin{macrocode}
%%\providecommand{\version}{final}
%    \end{macrocode}

% Include the main document:
%    \begin{macrocode}
\input{childdoc.def}
\childdocby{cdocsamp}
%    \end{macrocode}

%\iffalse
%</samplepart3|samplepart4>
%\fi
%
%\iffalse
%<*samplepart3>
%\fi
% Some text for part 3:
%    \begin{macrocode}
some text in part three
%    \end{macrocode}

%\iffalse
%</samplepart3>
%\fi
% Some text for part 4:
%\iffalse
%<*samplepart4>
%\fi
%    \begin{macrocode}
more text in part four
%    \end{macrocode}

%\iffalse
%</samplepart4>
%\fi
%
% %%%%%%%%%%%%%%%%%%%%%%%%%%%%%%%%%%%%%%
% \paragraph{Forwarding for a Complete Draft.}
%
% The following forwarding file |cdocsdrf.tex|
% compiles the main document in draft mode:
%\iffalse
%<*sampledraft>
%\fi
%    \begin{macrocode}
\def\version{draft}
\input{childdoc.def}
\childdocforward{cdocsamp}
%    \end{macrocode}

%\iffalse
%</sampledraft>
%\fi
%
% %%%%%%%%%%%%%%%%%%%%%%%%%%%%%%%%%%%%%%
% \paragraph{Forwarding for Final Version of the Chapters.}
%
% The following forwarding files |cdocsfn1.tex| and |cdocsfn2.tex|
% (with identical content)
% compile the final versions of the child documents
% |cdocsch1.tex| and |cdocsch2.tex|, respectively:
%\iffalse
%<*samplefinal>
%\fi
%    \begin{macrocode}
\def\version{final}
\input{childdoc.def}
\childdocforwardprefix[cdocsamp]{cdocsfn}{cdocsch}
%    \end{macrocode}

%\iffalse
%</samplefinal>
%\fi
%
% %%%%%%%%%%%%%%%%%%%%%%%%%%%%%%%%%%%%%%
% \paragraph{Command Line Processing.}
%
% The following three command lines generate the output files
% |cdocscld|, |cdocscl1| and |cdocscl2|
% which should be identical to
% |cdocsdrf|, |cdocsch1| and |cdocsfn2|, respectively:
% \begin{center}
% \begin{tabular}{l}
% |latex -jobname cdocscld \|\\
% |  "\def\version{draft}\input{childdoc.def}\childdocforward{cdocsamp}"|\\
% |latex -jobname cdocscl1 \|\\
% |  "\input{childdoc.def}\childdocforward[cdocsamp]{cdocsch1}"|\\
% |latex -jobname cdocscl2 \|\\
% |  "\def\version{final}\input{childdoc.def}\childdocforward{cdocsch2}"|
% \end{tabular}
% \end{center}
% Note that the trailing backslash on each first line
% merely continues the input to the second line
% (for convenient cut ant paste).
% Furthermore, the command |latex| can be replaced by any
% of its alternative versions such as |pdflatex|.
%
% %%%%%%%%%%%%%%%%%%%%%%%%%%%%%%%%%%%%%%%%%%%%%%%%%%%%%%%%%%%%%%%%%%%%%%%%%%%%%%
% %%%%%%%%%%%%%%%%%%%%%%%%%%%%%%%%%%%%%%%%%%%%%%%%%%%%%%%%%%%%%%%%%%%%%%%%%%%%%%
% \section{Implementation}
%\iffalse
%<*package>
%\fi
%
% This section describes the definitions file |childdoc.def|.

% The definitions cannot be loaded using |\usepackage| or |\RequirePackage|
% which has a mechanism to prevent loading a style file more than once.
% When loading the definitions by means of |\input|
% multiple instances have to be prevented manually:
%\iffalse
%This code needs to be before the `\ProvidesFile' directive
%which is defined at the beginning of this file.
%Therefore it is also placed there and commented out here.
%</package>
%<*discard>
%\fi
%    \begin{macrocode}
\ifdefined\childdocmain\endinput\fi
%    \end{macrocode}
%\iffalse
%</discard>
%<*package>
%\fi
%
% \macro{\ifchilddoc}
% \macro{\ifchilddocmanual}
% The conditional |\ifchilddoc| tells whether a
% child (true) or main (false) document is being compiled.
% The conditional |\ifchilddocmanual| tells whether
% the |\includeonly| mechanism is used (false) or
% the selection of child files must be performed manually (true).
% The definitions initialise to false:
%    \begin{macrocode}
\newif\ifchilddoc
\newif\ifchilddocmanual
%    \end{macrocode}

% \macro{\childdocname}
% \macro{\childdocjob}
% The macro |\childdocname| stores the name of the main document
% to be compiled. The macro |\childdocjob| stores the name of
% the document on which the \LaTeX{} compiler was originally invoked.
% The content of |\jobname| cannot be compared
% to filenames specified in the source due to different catcodes.
% The following code rescans |\jobname|, stores the result
% in |\childdocname| and saves a copy in |\childdocjob|:
%    \begin{macrocode}
\edef\childdocname{\scantokens\expandafter{\jobname\noexpand}}
\let\childdocjob\childdocname
%    \end{macrocode}

% \macro{\childdocdisable}
% The macro |\childdocdisable| prevents the main file
% from being processed more than once.
% At this stage, the main document command |\childdocmain|
% is assumed to be called once again where it should do nothing.
% Any subsequent call to it should prevent
% a secondary processing of the main document
% It overwrites the forwarding commands
% |\childdocof| and |\childdocforward|
% with empty macros to prevent further inclusions of the main document:
%    \begin{macrocode}
\newcommand{\childdocdisable}
{
  \renewcommand{\childdocmain}[1]{\renewcommand{\childdocmain}[1]{\endinput}}
  \renewcommand{\childdocof}[1]{}
  \renewcommand{\childdocby}[2][]{}
  \renewcommand{\childdocforward}[2][]{}
  \renewcommand{\childdocdisable}{}
}
%    \end{macrocode}

% \macro{\childdocmain}
% The macro |\childdocmain| is to be called at the top of the main file
% with nothing or the main filename (without extension) as argument.
% First, it breaks loops.
% If the argument is not empty and does not match |\childdocname|
% (which is set by the first inclusion of |childdoc.def|),
% |\ifchilddoc| is set to true, |\includeonly| is applied to the child file
% and |\jobname| is set to the main file
% (for proper handling of |.aux| files):
%    \begin{macrocode}
\newcommand{\childdocmain}[1]
{
  \childdocdisable\childdocmain{}
  \if?#1?\else
    \begingroup
      \def\childdoctmp{#1}
      \ifx\childdoctmp\childdocname
        \def\childdoctmp{}
      \else
        \def\childdoctmp
        {
          \childdoctrue
          \includeonly{\childdocname}
          \def\childdocjob{#1}
          \def\jobname{#1}
        }
      \fi
      \expandafter
    \endgroup
    \childdoctmp
  \fi
}
%    \end{macrocode}

% \macro{\childdocof}
% The command |\childdocof| redirects
% compilation to the main file |#1|.
%    \begin{macrocode}
\newcommand{\childdocof}[1]
{
  \childdocdisable
  \childdoctrue
  \includeonly{\childdocname}
  \def\jobname{#1}
  \def\childdocjob{#1}
  \input{#1}
}
%    \end{macrocode}

% \macro{\childdocby}
% The command |\childdocby| ....
%    \begin{macrocode}
\newcommand{\childdocby}[2][]
{
  \childdocdisable
  \childdoctrue
  \childdocmanualtrue
  \if?#1?\else
    \def\jobname{#2}
  \fi
  \def\childdocjob{#2}
  \input{#2}
  \endinput
}
%    \end{macrocode}

% \macro{\childdocforward}
% The command |\childdocforward| redirects
% compilation to the main file or
% (if the optional argument is given) a child file.
% Parameters are set as if the main file
% or a child file starting with |\childdocof| was compiled.
% Then compilation is handed over to the main file:
%    \begin{macrocode}
\newcommand{\childdocforward}[2][]
{
  \begingroup
    \if?#1?
      \def\childdoctmp
      {
        \def\childdocname{#2}
        \def\childdocjob{#2}
        \def\jobname{#2}
        \input{#2}
        \endinput
      }
    \else
      \def\childdoctmp
      {
        \childdocdisable
        \def\childdocname{#2}
        \childdoctrue
        \includeonly{#2}
        \def\childdocjob{#1}
        \def\jobname{#1}
        \input{#1}
        \endinput
      }
    \fi
    \expandafter
  \endgroup
  \childdoctmp
}
%    \end{macrocode}

% \macro{\childdocforwardprefix}
% The command |\childdocforwardprefix| redirects
% compilation to the main or a child file by means of a pattern.
% The prefix |#1| in the current filename is replaced by |#2|
% and the suffix of the current filename is kept
% (it is assumed that the filename does not contain the substring `|~~~|'
% which is used as a delimiter).
% Compilation is handed over to the new file by |\childdocforward|:
%    \begin{macrocode}
\newcommand{\childdocforwardprefix}[3][]
{
  \begingroup
    \def\childdocextract #2##1~~~{\def\childdoctmp{\childdocforward[#1]{#3##1}}}
    \expandafter\childdocextract\childdocname~~~
    \expandafter
  \endgroup
  \childdoctmp
}
%    \end{macrocode}

% \macro{\childdoc}
% The deprecated macro |\childdoc| is a legacy version of |\childdocmain|:
%    \begin{macrocode}
\newcommand{\childdoc}{\childdocmain}
%    \end{macrocode}

% \macro{\childdocredirect}
% The deprecated macro |\childdocredirect| is a legacy version
% of |\childdocforward| and |\childdocforwardprefix|:
%    \begin{macrocode}
\newcommand{\childdocredirect}[2][]
{
  \begingroup
    \if?#1?
      \def\childdoctmp{\childdocforward{#2}}
    \else
      \def\childdoctmp{\childdocforwardprefix{#1}{#2}}
    \fi
    \expandafter
  \endgroup
  \childdoctmp
}
%    \end{macrocode}

%\iffalse
%</package>
%\fi
%
\endinput
|\\
|\childdocforwardprefix{final}{child}|
\end{tabular}
\end{center}
%

Note that when several versions of a main file and/or of each child file
are to be generated, it may be convenient to set up a |Makefile| or
shell script to automatise the process.

%%%%%%%%%%%%%%%%%%%%%%%%%%%%%%%%%%%%%%%%%%%%%%%%%%%%%%%%%%%%%%%%%%%%%%%%%%%%%%%%
\subsection{Command Line Processing}
\label{sec:commandline}

The effect of redirection files can also be achieved by invoking
the \LaTeX{} compiler with a more elaborate command line.
Most conveniently this should be done as part
of a shell script or a |Makefile|.

When using \textsf{childdoc} in the main file, the following
command lines effectively perform a redirection
(note that depending on the shell being used,
backslashes may have to be doubled: `|\|' $\to$ `|\\|'):
%
\begin{center}
|... -jobname "|\textit{target}|" |\\|"|[\textit{flags}]%
|% \iffalse
%
% childdoc.dtx Copyright (C) 2017-2018 Niklas Beisert
%
% This work may be distributed and/or modified under the
% conditions of the LaTeX Project Public License, either version 1.3
% of this license or (at your option) any later version.
% The latest version of this license is in
%   http://www.latex-project.org/lppl.txt
% and version 1.3 or later is part of all distributions of LaTeX
% version 2005/12/01 or later.
%
% This work has the LPPL maintenance status `maintained'.
%
% The Current Maintainer of this work is Niklas Beisert.
%
% This work consists of the files childdoc.dtx and childdoc.ins
% and the derived files childdoc.def and cdocsamp.tex with
% cdocsch1.tex, cdocsch2.tex, cdocsdrf.tex, cdocsfn1.tex, cdocsfn2.tex.
%
%<package>\ifdefined\childdocmain\endinput\fi
%<package>\ProvidesFile{childdoc.def}[2018/12/30 v2.0 child document driver]
%<samplemain>\ProvidesFile{cdocsamp.tex}[2018/12/30 v2.0 sample for childdoc]
%<*driver>
%\ProvidesFile{childdoc.drv}[2018/12/30 v2.0 childdoc reference manual file]
\PassOptionsToClass{10pt,a4paper}{article}
\documentclass{ltxdoc}

\usepackage[margin=35mm]{geometry}
\usepackage{hyperref}
\usepackage{hyperxmp}
\usepackage[usenames]{color}

\hypersetup{colorlinks=true}
\hypersetup{pdfstartview=FitH}
\hypersetup{pdfpagemode=UseNone}
\hypersetup{pdfsource={}}
\hypersetup{pdflang={en-UK}}
\hypersetup{pdfcopyright={Copyright 2017-2018 Niklas Beisert.
  This work may be distributed and/or modified under the
  conditions of the LaTeX Project Public License, either version 1.3
  of this license or (at your option) any later version.}}
\hypersetup{pdflicenseurl={http://www.latex-project.org/lppl.txt}}
\hypersetup{pdfcontactaddress={ETH Zurich, ITP, HIT K,
  Wolfgang-Pauli-Strasse 27}}
\hypersetup{pdfcontactpostcode={8093}}
\hypersetup{pdfcontactcity={Zurich}}
\hypersetup{pdfcontactcountry={Switzerland}}
\hypersetup{pdfcontactemail={nbeisert@itp.phys.ethz.ch}}
\hypersetup{pdfcontacturl={http://people.phys.ethz.ch/\xmptilde nbeisert/}}

\newcommand{\secref}[1]{\hyperref[#1]{section \ref*{#1}}}

\parskip1ex
\parindent0pt
\let\olditemize\itemize
\def\itemize{\olditemize\parskip0pt}

\begin{document}

\title{The \textsf{childdoc} Package}
\hypersetup{pdftitle={The childdoc Package}}
\author{Niklas Beisert\\[2ex]
  Institut f\"ur Theoretische Physik\\
  Eidgen\"ossische Technische Hochschule Z\"urich\\
  Wolfgang-Pauli-Strasse 27, 8093 Z\"urich, Switzerland\\[1ex]
  \href{mailto:nbeisert@itp.phys.ethz.ch}
  {\texttt{nbeisert@itp.phys.ethz.ch}}}
\hypersetup{pdfauthor={Niklas Beisert}}
\hypersetup{pdfsubject={Manual for the LaTeX2e Package childdoc}}
\date{30 December 2018, \textsf{v2.0}}
\maketitle

\begin{abstract}\noindent
\textsf{childdoc} is a \LaTeXe{} package
that enables the direct compilation
of document sections included by |\include|
to individual files.
\end{abstract}

\begingroup
\parskip0ex
\tableofcontents
\endgroup

%%%%%%%%%%%%%%%%%%%%%%%%%%%%%%%%%%%%%%%%%%%%%%%%%%%%%%%%%%%%%%%%%%%%%%%%%%%%%%%%
%%%%%%%%%%%%%%%%%%%%%%%%%%%%%%%%%%%%%%%%%%%%%%%%%%%%%%%%%%%%%%%%%%%%%%%%%%%%%%%%
\section{Introduction}

\LaTeX{} provides a mechanism to structure a large document (such as a book)
into a main file and several child files (containing the chapters)
using the |\include| command.
This mechanism is beneficial for documents
which span hundreds of pages in order to
make the source file(s) more manageable.
Moreover, compilation can be restricted to
selected child files by means of the |\includeonly| command.
The latter feature can be used to reduce the compilation time while editing
(this was significantly more useful in the earlier days of \LaTeX{})
or to generate a smaller document which is easier to navigate.
Another application of |\includeonly| is to generate
documents consisting of selected parts of the complete document.

However, there are a few drawbacks of the plain |\include| mechanism:
\begin{itemize}
\item
The child files cannot be compiled on their own,
they can only be compiled via the main file.
A naive editing environment
(such as a text editor with an option
to have the current file processed by \LaTeX)
may require one to switch to the main file before compiling;
attempting to compile the child file produces errors.
\item
The main file must be modified (each time)
to adjust the |\includeonly| command
to the present needs. This easily leaves the main file in a messy state.
\item
The generated document will always carry the filename
of the main document. This is inconvenient if
several child files are to be compiled and
to be kept for distribution.
\end{itemize}

The present package provides a simple interface
to make child files individually compilable by \LaTeX{}.
Compiling a child file then has the same effect as compiling
the main file with an |\includeonly| command
to select the appropriate child.
Moreover the generated document will carry the name of the child
rather than the main file.
This resolves all three above issues.

This feature is meant to make the editing of books,
thesis documents and lecture notes somewhat more convenient.
However, the package can also be used efficiently for
composing a series of documents (such as exercise sheets)
which are typically distributed individually.
It then assists the author in generating the individual documents
(potentially in different versions)
as well as a document containing the collected series.
Another application is in developing style files
or other kinds of included material
where compilation of the style file could redirect
to a sample or test file.

%%%%%%%%%%%%%%%%%%%%%%%%%%%%%%%%%%%%%%%%%%%%%%%%%%%%%%%%%%%%%%%%%%%%%%%%%%%%%%%%
%%%%%%%%%%%%%%%%%%%%%%%%%%%%%%%%%%%%%%%%%%%%%%%%%%%%%%%%%%%%%%%%%%%%%%%%%%%%%%%%
\section{Usage}

First of all, the package \textsf{childdoc} is \emph{not} a standard
\LaTeXe{} |.sty| style file! Therefore it needs to be invoked in
a non-standard way.

%%%%%%%%%%%%%%%%%%%%%%%%%%%%%%%%%%%%%%%%%%%%%%%%%%%%%%%%%%%%%%%%%%%%%%%%%%%%%%%%
\subsection{Included Files}
\label{sec:include}

%%%%%%%%%%%%%%%%%%%%%%%%%%%%%%%%%%%%%%%%
\DescribeMacro{\childdocmain}
To use the package, add the commands
\begin{center}
\begin{tabular}{l}
|\input{childdoc.def}|\\
|\childdocmain{}|\\
\end{tabular}
\end{center}
at the very top of the main \LaTeX{} file,
in particular \emph{before} the |\documentclass| statement!
The argument of |\childdocmain| should be left empty
(but it must be present).

%%%%%%%%%%%%%%%%%%%%%%%%%%%%%%%%%%%%%%%%
\DescribeMacro{\childdocof}
Furthermore, add the commands
\begin{center}
\begin{tabular}{l}
|\input{childdoc.def}|\\
|\childdocof{|\textit{main}|}|\\
\end{tabular}
\end{center}
at the top of every child file \textit{child}
which is included by |\include{|\textit{child}|}|
from within the main file
(or at least for those files to be compiled individually).
The argument \textit{main} must be the filename of the main file.

There are a couple of
considerations in setting up the main and child documents:

%%%%%%%%%%%%%%%%%%%%%%%%%%%%%%%%%%%%%%%%
\paragraph{Restrictions.}

Please note the following restrictions:
\begin{itemize}
\item
|\childdocmain| must be called with one argument \textit{main}
to ensure compatibility with earlier version of the package.
It must either be empty (|\childdocmain{}|)
or precisely match the filename of the main file in which it is specified.
See \secref{sec:detection} for further information.
\item
The filename \textit{main} must be specified without the |.tex| extension.
\item
The filename \textit{main} is case sensitive
(even in case-insensitive file systems)
due to internal string comparison.
\item
The argument \textit{main} should be fully expanded, it cannot be a macro.
\item
Subdirectories and special characters should be avoided in filenames.
\item
The command |\childdocmain{|\textit{main}|}| must be followed by a whitespace.
It should not be followed immediately by another command
or by a comment mark `|%|'.
This is because the \TeX{} parser reads the token immediately following
the argument of |\childdocmain| and puts it
at the beginning of every child section;
however, a white\-space is ignored.
\end{itemize}

%%%%%%%%%%%%%%%%%%%%%%%%%%%%%%%%%%%%%%%%
\paragraph{Content of Main File.}

It is advisable to place all content in the child files included by |\include|.
Any output contained in the main file will appear in all child documents
unless suppressed manually;
it cannot be suppressed automatically by the |\includeonly| directive
and thus should normally be avoided.
A method to include some content in the main file
by means of conditional processing is described in \secref{sec:conditional}.

%%%%%%%%%%%%%%%%%%%%%%%%%%%%%%%%%%%%%%%%
\paragraph{Page Numbering.}

When only a part of the document is compiled,
the appropriate numbering of pages
(as well as other status parameters)
is determined from the |.aux| files.
The latter contain information from previous passes.
However this information needs to propagate through
all intermediate child documents.
Therefore the page numbering in child documents may well
be inconsistent until the complete document is compiled at least once.

A useful (if unconventional) way to always ensure a consistent
page numbering is to restart the numbering in each child document
and denote the pages by `\textit{child}|.|\textit{page}'
where \textit{child} represents the chapter/section number of the child file.
This can be achieved by the command
|\numberwithin{page}{|\textit{child}|}|
of the \textsf{amsmath} package
where \textit{child} can be |chapter| or |section|
depending on the chosen structuring.
Alternatively, one can modify the macro |\thepage| appropriately
and reset the counter |page| at the start of each child file.

%%%%%%%%%%%%%%%%%%%%%%%%%%%%%%%%%%%%%%%%%%%%%%%%%%%%%%%%%%%%%%%%%%%%%%%%%%%%%%%%
\subsection{Conditional Processing}
\label{sec:conditional}

The package provides a mechanism to compile different versions
of a document. To customise the versions further some conditional processing
can come in handy to distinguish which version is being compiled.
The package provides two macros to describe the compilation context:

%%%%%%%%%%%%%%%%%%%%%%%%%%%%%%%%%%%%%%%%
\DescribeMacro{\ifchilddoc}
The conditional |\ifchilddoc| distinguishes between the compilation of
child documents and the main document:
%
\begin{center}
|\ifchilddoc |\textit{child-code}| |[|\||else |\textit{main-code}]| \||fi|
\end{center}

%%%%%%%%%%%%%%%%%%%%%%%%%%%%%%%%%%%%%%%%
\DescribeMacro{\childdocname}
\DescribeMacro{\childdocjob}
The macro |\childdocname| contains the filename (without extension)
of the main or child file being processed.
Note that |\childdocjob| will always contain the name of the main file.

%%%%%%%%%%%%%%%%%%%%%%%%%%%%%%%%%%%%%%%%
\paragraph{Title Page.}

Conditional processing can be used to include a title or banner page
in the main document when proper precautions are taken.
Importantly, the code in the main file should ensure that the page counter
(as well as other status parameters which are stored in the |.aux| files)
takes the same value after the conditional processing.
Otherwise the page numbers may take divergent values
depending on which part is compiled.

For example, a title page could be declared by:
%
\begin{center}
\begin{tabular}{l}
|\ifchilddoc\||else|\\
|\addtocounter{page}{-1}|\\
\textit{code for title page}\\
|\newpage|\\
|\||fi|
\end{tabular}
\end{center}
%
A banner page for the child documents can be generated by:
%
\begin{center}
\begin{tabular}{l}
|\ifchilddoc|\\
|\addtocounter{page}{-1}|\\
\textit{code for banner page}\\
|\newpage|\\
|\||fi|
\end{tabular}
\end{center}
%
Here one could write a message such as:
\begin{center}
|This is the part \childdocname{} of \childdocjob{}.|
\end{center}

%%%%%%%%%%%%%%%%%%%%%%%%%%%%%%%%%%%%%%%%%%%%%%%%%%%%%%%%%%%%%%%%%%%%%%%%%%%%%%%%
\subsection{Flags}
\label{sec:flags}

The package makes it easy to generate different versions
of the main or child documents.
To this end compilation flags can be defined
and assigned different default values.
They will be particularly useful in conjunction
with the forwarding mechanism described in \secref{sec:forward}.

For example, it may be useful to have a flag |\version|
which can be set to |draft| or |final|.
The document source will contain some conditional code
depending on the value of |\version|.
Suppose further, the flag should default to |final| for the main file
and to |draft| for child files
which is a natural assignment for editing the document.
This is achieved by placing the following code
in the preamble of the main document
(below the |\childdocmain| directive):
%
\begin{center}
\begin{tabular}{l}
|\ifchilddoc|\\
|\providecommand{\version}{draft}|\\
|\||else|\\
|\providecommand{\version}{final}|\\
|\||fi|
\end{tabular}
\end{center}
%
The definition by |\providecommand| makes sure
that previous definitions are not overwritten.
Further statements |\providecommand{\version}{...}|
can thus be added before the above code to override it.

For the main file, one might add a line
(between |\childdocmain| and the above block)
%
\begin{center}
|%\ifchilddoc\||else\providecommand{\version}{draft}\||fi|
\end{center}
%
which can be uncommented to produce a draft version.
Likewise one can add a line to the very top of a child file
(above the |\childdocof{|\textit{main}|}| directive)
%
\begin{center}
|%\providecommand{\version}{final}|
\end{center}
%
which can be uncommented to produce the final version of this child document.

%%%%%%%%%%%%%%%%%%%%%%%%%%%%%%%%%%%%%%%%%%%%%%%%%%%%%%%%%%%%%%%%%%%%%%%%%%%%%%%%
\subsection{Forwarding}
\label{sec:forward}

Different versions of the main or child documents
using compilation flags as described in \secref{sec:flags}
can be (permanently) stored in different files
for convenient compilation, viewing and distribution.
To this end, the package defines a command
to pass on compilation to a different file:

%%%%%%%%%%%%%%%%%%%%%%%%%%%%%%%%%%%%%%%%
\DescribeMacro{\childdocforward}
The command |\childdocforward| redirects processing to
another source file:
%
\begin{center}
\begin{tabular}{l}
|\input{childdoc.def}|\\
|\childdocforward[|\textit{main}|]{|\textit{dest}|}|\\
\end{tabular}
\end{center}
%
The argument \textit{dest} is the destination file
(without extension).
It should be the main file or one of the child files.
Note that further \textsf{childdoc} directives
such as |\childdocof| and |\childdocforward|
in the indicated file will be processed in this form.
The optional argument \textit{main}
passes on directly to the main file \textit{main}
while pretending to compile the child \textit{dest}.
This form behaves as if \textit{dest}
issues |\childdocof{|\textit{main}|}| right away,
and no further \textsf{childdoc} directives will be processed.

%%%%%%%%%%%%%%%%%%%%%%%%%%%%%%%%%%%%%%%%
\DescribeMacro{\...prefix}
In the alternative form |\childdocforwardprefix|,
%
\begin{center}
\begin{tabular}{l}
|\input{childdoc.def}|\\
|\childdocforwardprefix[|\textit{main}|]{|\textit{prefix}|}{|\textit{dest}|}|
\end{tabular}
\end{center}
%
the destination file is determined by a pattern
depending on the current file:
To make this work, the current file must be called
`{\textit{prefix}\hspace{0.2em}\textit{suffix}}'
with \textit{prefix} matching precisely the argument.
Processing is then passed on to the file
`{\textit{dest}\hspace{0.2em}\textit{suffix}}'.
Surely, the same effect is achieved by
directly specifying the
argument `{\textit{dest}\hspace{0.2em}\textit{suffix}}'
in the first form.
However, that requires to set up a different file
for each child. With the alternative form of the command
all these files can have exactly the same content
which simplifies setting them up and maintaining them.

For example, the following file |draft.tex|
with a compilation flag |\version| as described in \secref{sec:flags}
compiles the main document as a draft:
%
\begin{center}
\begin{tabular}{l}
|\def\version{draft}|\\
|\input{childdoc.def}|\\
|\childdocforward{|\textit{main}|}|
\end{tabular}
\end{center}
%
Likewise, the following files |final|\textit{nn}|.tex|
compile the final version of the child document
|child|\textit{nn}|.tex|:
%
\begin{center}
\begin{tabular}{l}
|\def\version{final}|\\
|\input{childdoc.def}|\\
|\childdocforwardprefix{final}{child}|
\end{tabular}
\end{center}
%

Note that when several versions of a main file and/or of each child file
are to be generated, it may be convenient to set up a |Makefile| or
shell script to automatise the process.

%%%%%%%%%%%%%%%%%%%%%%%%%%%%%%%%%%%%%%%%%%%%%%%%%%%%%%%%%%%%%%%%%%%%%%%%%%%%%%%%
\subsection{Command Line Processing}
\label{sec:commandline}

The effect of redirection files can also be achieved by invoking
the \LaTeX{} compiler with a more elaborate command line.
Most conveniently this should be done as part
of a shell script or a |Makefile|.

When using \textsf{childdoc} in the main file, the following
command lines effectively perform a redirection
(note that depending on the shell being used,
backslashes may have to be doubled: `|\|' $\to$ `|\\|'):
%
\begin{center}
|... -jobname "|\textit{target}|" |\\|"|[\textit{flags}]%
|\input{childdoc.def}\childdocforward[|\textit{main}|]{|\textit{dest}|}"|
\end{center}
%
Here \textit{target} is the name of the output file,
\textit{main} is the name of the main file
and \textit{dest} is the name of the main or child file to be processed
(all filenames without extensions).
The optional argument \textit{main} can be omitted
if \textit{main} matches \textit{dest}.
Optionally, compilation \textit{flags} can be defined via |\def| commands.
This command line makes the \TeX{} engine believe
it is compiling the file \textit{target}
whose content is specified as the latter parameter.
The provided code then forwards the processing to
\textit{main} or \textit{dest} as described in \secref{sec:forward}.

%%%%%%%%%%%%%%%%%%%%%%%%%%%%%%%%%%%%%%%%%%%%%%%%%%%%%%%%%%%%%%%%%%%%%%%%%%%%%%%%
\subsection{Include by Input}
\label{sec:input}

Including child documents by |\include| has some restrictions by design.
Most notably, the content of a child document always occupies
its own set of pages; pages cannot be shared between child documents.
Usually, this behaviour makes perfect sense
because each child document contain an essential part of the document.
However, in some situations it may be desirable to compose
a document from a collection of parts
without having mandatory page breaks between then.
For this case, the package
provides a mechanism to include parts
by |\input| which can also be processed individually.
However, by construction this mechanism
requires manual handling of the content to be output.

%%%%%%%%%%%%%%%%%%%%%%%%%%%%%%%%%%%%%%%%
\DescribeMacro{\ifchilddocmanual}
The main file should be prepared as usual, see \secref{sec:include}.
However, the document body must make a distinction
between processing of an individual part and of the main document, e.g.:
%
\begin{center}
\begin{tabular}{l}
|\ifchilddocmanual|\\
|\input{\childdocname}|\\
|\||else|\\
\textit{document body with }|\input{|\textit{part}|}|\\
|\||fi|
\end{tabular}
\end{center}
%
The conditional |\ifchilddocmanual| is true whenever
a part to be included by |\input| is being compiled,
and the name of the part is stored in |\childdocname|.

%%%%%%%%%%%%%%%%%%%%%%%%%%%%%%%%%%%%%%%%
\DescribeMacro{\childdocby}
Each part to be included by |\input| should start with:
%
\begin{center}
\begin{tabular}{l}
|\input{childdoc.def}|\\
|\childdocby{|\textit{main}|}|\\
\end{tabular}
\end{center}
%
The directive |\childdocby| is similar to |\childdocof|
described in \secref{sec:include},
but the subsequent selection of content must be done manually.
To that end, both |\ifchilddoc| and |\ifchilddocmanual|
will be true upon processing of a part,
and the name of the part is stored in |\childdocname|.
Note that |\jobname| will be set to the filename of the current part
so that each part receives an individual |.aux| file
that does not interfere with the |.aux| file(s) of the main document.
This behaviour can be altered by the alternative form
|\childdocby[*]{|\textit{main}|}| (with a non-empty optional argument)
which uses the |.aux| file of the main document
by setting |\jobname| to \textit{main}.

%%%%%%%%%%%%%%%%%%%%%%%%%%%%%%%%%%%%%%%%%%%%%%%%%%%%%%%%%%%%%%%%%%%%%%%%%%%%%%%%
\subsection{Driver Development}
\label{sec:driver}

The \textsf{childdoc} mechanism can also be use for the development
of definition files such as \LaTeX{} styles or classes.
This case differs from the above setup with multiple parts
included by |\include| in that no |\includeonly| should be invoked.
This can be achieved by starting the include file
(before |\ProvidesPackage|) with:
%
\begin{center}
\begin{tabular}{l}
|\input{childdoc.def}|\\
|\childdocforward{|\textit{main}|}|\\
\end{tabular}
\end{center}
%
or alternatively with:
%
\begin{center}
\begin{tabular}{l}
|\input{childdoc.def}|\\
|\childdocby{|\textit{main}|}|\\
\end{tabular}
\end{center}
%
Both forms have slightly different effects as described above.
The main file is prepared as usual, see \secref{sec:include}.

%%%%%%%%%%%%%%%%%%%%%%%%%%%%%%%%%%%%%%%%%%%%%%%%%%%%%%%%%%%%%%%%%%%%%%%%%%%%%%%%
\subsection{Legacy Detection}
\label{sec:detection}

The directive |\childdocmain| in the main file can detect
whether the complete document or merely a child is to be compiled
even without using the directive |\childdocof|.
This method is deprecated because it is less robust
and there is no compelling reason to use it;
it is merely provided for backward compatibility
and it may be removed in future versions.

If the detection mechanism is to be used,
it is mandatory to correctly specify
the filename of the main file as the argument of |\childdocmain|:
%
\begin{center}
\begin{tabular}{l}
|\input{childdoc.def}|\\
|\childdocmain{|\textit{main}|}|\\
\end{tabular}
\end{center}
%
If |\jobname| does not match the argument \textit{main} of |\childdocmain|,
it is assumed that |\jobname| points to the child file to be compiled.
When using |\childdocmain| with the main file specified as argument,
it suffices to start a child file
with just |\input{|\textit{main}|}|
without loading of the package and using |\childdocof|.
If instead all processing is done
with the appropriate \textsf{childdoc} directives,
the argument of \textit{main} of |\childdocmain| can be empty.

An alternative version of the command line processing described
in \secref{sec:commandline} using the detection mechanism reads:
%
\begin{center}
|... -jobname "|\textit{target}|" "|[\textit{flags}]%
[|\def\jobname{|\textit{dest}|}|]|\input{|\textit{main}|}"|
\end{center}

%%%%%%%%%%%%%%%%%%%%%%%%%%%%%%%%%%%%%%%%%%%%%%%%%%%%%%%%%%%%%%%%%%%%%%%%%%%%%%%%
\subsection{Manual Code}
\label{sec:manual}

In case one cannot be certain whether the definitions file |childdoc.def|
is installed on the target \TeX{} distribution
and one prefers not to ship it,
it is conceivable to paste a few relevant commands into the sources.

To that end, drop all statements |\input{childdoc.def}|
and perform the replacements as outlined below.
Instead of |\childdocmain{|\textit{main}|}| add the following code
to the top of the main file:
%
\begin{center}
\begin{tabular}{l}
|\||ifdefined\childdocname\endinput\||fi\newif\ifchilddoc|\\
|\edef\childdocname{\scantokens\expandafter{\jobname\noexpand}}|\\
|\def\childdocmain{|\textit{main}|}\||ifx\childdocmain\childdocname\||else|\\
|\childdoctrue\includeonly{\childdocname}\let\jobname\childdocmain\||fi|\\
\end{tabular}
\end{center}
%
Instead of |\childdocof{|\textit{main}|}| just include the main file
at the top of each child file:
%
\begin{center}
|\input{|\textit{main}|}|
\end{center}
%
A simple redirection |\childdocforward{|\textit{dest}|}| is achieved by:
%
\begin{center}
|\def\jobname{|\textit{dest}|}\input{\jobname}|
\end{center}
%
The redirection with prefix
|\childdocforwardprefix[|\textit{prefix}|]{|\textit{dest}|}|
is accomplished by:
%
\begin{center}
\begin{tabular}{l}
|{\edef\jobname{\scantokens\expandafter{\jobname\noexpand}}|\\
|\def\redirectjob |\textit{prefix}|#1~~~{\gdef\jobname{|\textit{dest}|#1}}|\\
|\expandafter\redirectjob\jobname~~~}\input{\jobname}|
\end{tabular}
\end{center}

In an alternative approach,
child documents can be compiled by a specific command line
without additional code or specific definitions:
%
\begin{center}
|... -jobname "|\textit{target}|" "|[\textit{flags}]%
|\includeonly{|\textit{dest}|}\input{|\textit{main}|}"|
\end{center}
%

%%%%%%%%%%%%%%%%%%%%%%%%%%%%%%%%%%%%%%%%%%%%%%%%%%%%%%%%%%%%%%%%%%%%%%%%%%%%%%%%
%%%%%%%%%%%%%%%%%%%%%%%%%%%%%%%%%%%%%%%%%%%%%%%%%%%%%%%%%%%%%%%%%%%%%%%%%%%%%%%%
\section{Information}

%%%%%%%%%%%%%%%%%%%%%%%%%%%%%%%%%%%%%%%%%%%%%%%%%%%%%%%%%%%%%%%%%%%%%%%%%%%%%%%%
\subsection{Copyright}

Copyright \copyright{} 2017--2018 Niklas Beisert

This work may be distributed and/or modified under the
conditions of the \LaTeX{} Project Public License, either version 1.3
of this license or (at your option) any later version.
The latest version of this license is in
  \url{http://www.latex-project.org/lppl.txt}
and version 1.3 or later is part of all distributions of \LaTeX{}
version 2005/12/01 or later.

This work has the LPPL maintenance status `maintained'.

The Current Maintainer of this work is Niklas Beisert.

This work consists of the files |README.txt|, |childdoc.ins| and |childdoc.dtx|
as well as the derived files |childdoc.def|, |cdocsamp.tex|
with |cdocsch1.tex|, |cdocsch2.tex|, |cdocspt3.tex|, |cdocspt4.tex|,
|cdocsdrf.tex|, |cdocsfn1.tex|, |cdocsfn2.tex|
as well as |childdoc.pdf|.

%%%%%%%%%%%%%%%%%%%%%%%%%%%%%%%%%%%%%%%%%%%%%%%%%%%%%%%%%%%%%%%%%%%%%%%%%%%%%%%%
\subsection{Files and Installation}

The package consists of the files:
%
\begin{center}
\begin{tabular}{ll}
    |README.txt|   & readme file \\
    |childdoc.ins| & installation file \\
    |childdoc.dtx| & source file \\
    |childdoc.def| & definition file \\
    |cdocsamp.tex| & sample main file \\
    |cdocsch1.tex| & sample include file \\
    |cdocsch2.tex| & sample include file \\
    |cdocspt3.tex| & sample part file \\
    |cdocspt4.tex| & sample part file \\
    |cdocsdrf.tex| & sample redirection file \\
    |cdocsfn1.tex| & sample redirection file \\
    |cdocsfn2.tex| & sample redirection file \\
    |childdoc.pdf| & manual
\end{tabular}
\end{center}
%
The distribution consists of the files
|README.txt|, |childdoc.ins| and |childdoc.dtx|.
%
\begin{itemize}
\item
Run (pdf)\LaTeX{} on |childdoc.dtx|
to compile the manual |childdoc.pdf| (this file).
\item
Run \LaTeX{} on |childdoc.ins| to create the definitions file |childdoc.def|
and the sample |cdocsamp.tex| with include files
|cdocsch1.tex|, |cdocsch2.tex|, |cdocspt3.tex|, |cdocspt4.tex|,
|cdocsdrf.tex|, |cdocsfn1.tex|, |cdocsfn2.tex|.
Then copy the file |childdoc.def| to an appropriate directory of your \LaTeX{}
distribution, e.g.\ \textit{texmf-root}|/tex/latex/childdoc|.
\end{itemize}

%%%%%%%%%%%%%%%%%%%%%%%%%%%%%%%%%%%%%%%%%%%%%%%%%%%%%%%%%%%%%%%%%%%%%%%%%%%%%%%%
\subsection{Related CTAN Packages}

There are several other packages which offer a similar functionality:
%
\begin{itemize}
\item
The packages
\href{http://ctan.org/pkg/docmute}{\textsf{docmute}},
\href{http://ctan.org/pkg/includex}{\textsf{includex}} and
\href{http://ctan.org/pkg/standalone}{\textsf{standalone}}
provide commands to include only the document body of
a child file thus allowing both files to be compiled individually.
\item
The packages \href{http://ctan.org/pkg/subdocs}{\textsf{subdocs}}
and \href{http://ctan.org/pkg/subfiles}{\textsf{subfiles}}
provide structures in which the main and child documents can be
encapsulated and allowing them to be compiled individually.
The inclusion mechanism is different from the conventional |\include|.
\item
The package \href{http://ctan.org/pkg/combine}{\textsf{combine}}
is an elaborate solution to combine several documents into one.
\end{itemize}
%
See also the CTAN topic \href{http://ctan.org/topic/subdocs}{\textsf{subdocs}}
for further related packages.
The present package differs from the above solutions in that
a document structure constructed with the conventional |\include| mechanism
just needs two extra commands at the top of every file
such that all constituent files can be compiled individually.

%%%%%%%%%%%%%%%%%%%%%%%%%%%%%%%%%%%%%%%%%%%%%%%%%%%%%%%%%%%%%%%%%%%%%%%%%%%%%%%%
%\subsection{Feature Suggestions}
%
%The following is a list of features which may be useful for future
%versions of this package:
%%
%\begin{itemize}
%\item
%\ldots
%\end{itemize}

%%%%%%%%%%%%%%%%%%%%%%%%%%%%%%%%%%%%%%%%%%%%%%%%%%%%%%%%%%%%%%%%%%%%%%%%%%%%%%%%
\subsection{Revision History}

%%%%%%%%%%%%%%%%%%%%%%%%%%%%%%%%%%%%%%%%
\paragraph{v2.0:} 2018/12/30

\begin{itemize}
\item
immediate forward processing
\item
added |\childdocby| mechanism
\item
manual restructured
\end{itemize}

%%%%%%%%%%%%%%%%%%%%%%%%%%%%%%%%%%%%%%%%
\paragraph{v1.6:} 2018/01/17

\begin{itemize}
\item
application for development of include files
\item
corrections to manual
\end{itemize}

%%%%%%%%%%%%%%%%%%%%%%%%%%%%%%%%%%%%%%%%
\paragraph{v1.5:} 2017/05/21

\begin{itemize}
\item
more complete structuring introduced
\item
|\childdocof| introduced
\item
|\childdoc| renamed to |\childdocmain|
\item
|\childredirect| renamed to |\childdocforward| and |\childdocforwardprefix|
and functionality expanded
\end{itemize}

%%%%%%%%%%%%%%%%%%%%%%%%%%%%%%%%%%%%%%%%
\paragraph{v1.0:} 2017/04/27

\begin{itemize}
\item
manual and install package
\item
first version published on CTAN
\end{itemize}

%%%%%%%%%%%%%%%%%%%%%%%%%%%%%%%%%%%%%%%%
\paragraph{v0.6:} 2017/04/26

\begin{itemize}
\item
redirection mechanism added
\end{itemize}

%%%%%%%%%%%%%%%%%%%%%%%%%%%%%%%%%%%%%%%%
\paragraph{v0.5:} 2017/04/26

\begin{itemize}
\item
functionality in definition file
\end{itemize}


%%%%%%%%%%%%%%%%%%%%%%%%%%%%%%%%%%%%%%%%%%%%%%%%%%%%%%%%%%%%%%%%%%%%%%%%%%%%%%%%
%%%%%%%%%%%%%%%%%%%%%%%%%%%%%%%%%%%%%%%%%%%%%%%%%%%%%%%%%%%%%%%%%%%%%%%%%%%%%%%%
%%%%%%%%%%%%%%%%%%%%%%%%%%%%%%%%%%%%%%%%%%%%%%%%%%%%%%%%%%%%%%%%%%%%%%%%%%%%%%%%
\appendix

\settowidth\MacroIndent{\rmfamily\scriptsize 000\ }

 \DocInput{childdoc.dtx}

\end{document}
%</driver>
% \fi
%
% %%%%%%%%%%%%%%%%%%%%%%%%%%%%%%%%%%%%%%%%%%%%%%%%%%%%%%%%%%%%%%%%%%%%%%%%%%%%%%
% %%%%%%%%%%%%%%%%%%%%%%%%%%%%%%%%%%%%%%%%%%%%%%%%%%%%%%%%%%%%%%%%%%%%%%%%%%%%%%
% \section{Sample}
%\iffalse
%<*samplemain>
%\fi
%
% The following presents a sample document
% with two chapters, two parts, a title page,
% a compile flag as well as three forwarding files to set the flag.
% It consists of eight |.tex| files:
% \begin{center}
% \begin{tabular}{ll}
% |cdocsamp.tex|&main file\\
% |cdocsch1.tex|&include file for chapter 1\\
% |cdocsch2.tex|&include file for chapter 2\\
% |cdocspt3.tex|&include file for part 3\\
% |cdocspt4.tex|&include file for part 4\\
% |cdocsdrf.tex|&forwarding file for main file in draft mode\\
% |cdocsfi1.tex|&forwarding file for final version of chapter 1\\
% |cdocsfi2.tex|&forwarding file for final version of chapter 2\\
% \end{tabular}
% \end{center}
% Each of the eight files can be compiled directly by the \LaTeX{} compiler.
%
% %%%%%%%%%%%%%%%%%%%%%%%%%%%%%%%%%%%%%%
% \paragraph{Main File.}
%
% The main file is called |cdocsamp.tex|.
%
% Load the \textsf{childdoc} definitions and
% declare the filename for the main document:
%    \begin{macrocode}
\input{childdoc.def}
\childdocmain{}
%    \end{macrocode}

% Optional override for |\version| flag:
%    \begin{macrocode}
%%\ifchilddoc\else\providecommand{\version}{draft}\fi
%    \end{macrocode}

% Define the default values for the |\version| flag
% (|final| for the main file and |draft| for childs):
%    \begin{macrocode}
\ifchilddoc
\providecommand{\version}{draft}
\else
\providecommand{\version}{final}
\fi
%    \end{macrocode}

% Load the standard document class:
%    \begin{macrocode}
\documentclass[12pt]{article}
%    \end{macrocode}

% Start the document body:
%    \begin{macrocode}
\begin{document}
%    \end{macrocode}

% Declare a title page.
% Print title, part of document being processed and version flag:
%    \begin{macrocode}
\addtocounter{page}{-1}
\begin{center}
{\LARGE\bfseries{}childdoc example\par}
\vspace{1cm}
\ifchilddoc
\ifchilddocmanual part\else chapter\fi:
`\childdocname' of `\childdocjob'\par
\else
main document: `\childdocjob'\par
\fi
version: \version\par
\end{center}
\newpage
%    \end{macrocode}

% Manually include selected file,
% otherwise process as usual:
%    \begin{macrocode}
\ifchilddocmanual
\section*{part `\childdocname'}
\input{\childdocname}
\else
%    \end{macrocode}

% Include the two chapters:
%    \begin{macrocode}
\include{cdocsch1}
\include{cdocsch2}
%    \end{macrocode}

% Include the two parts unless only chapters should be displayed:
%    \begin{macrocode}
\ifchilddoc\else
\section{part three}
\input{cdocspt3}
\section{part four}
\input{cdocspt4}
\fi
%    \end{macrocode}

% Process as usual until here:
%    \begin{macrocode}
\fi
%    \end{macrocode}

% End of document body:
%    \begin{macrocode}
\end{document}
%    \end{macrocode}
%\iffalse
%</samplemain>
%\fi
%
% %%%%%%%%%%%%%%%%%%%%%%%%%%%%%%%%%%%%%%
% \paragraph{Chapter Include Files.}
%
% The include files are called |cdocsch1.tex| and |cdocsch2.tex|.
%
%\iffalse
%<*samplechap1|samplechap2>
%\fi

% Optional override for |\version| flag:
%    \begin{macrocode}
%%\providecommand{\version}{final}
%    \end{macrocode}

% Include the main document:
%    \begin{macrocode}
\input{childdoc.def}
\childdocof{cdocsamp}
%    \end{macrocode}

%\iffalse
%</samplechap1|samplechap2>
%\fi
%
%\iffalse
%<*samplechap1>
%\fi
% Some text for chapter 1:
%    \begin{macrocode}
\section{one}
some text in chapter one
%    \end{macrocode}

%\iffalse
%</samplechap1>
%\fi
% Some text for chapter 2:
%\iffalse
%<*samplechap2>
%\fi
%    \begin{macrocode}
\section{two}
more text in chapter two
%    \end{macrocode}

%\iffalse
%</samplechap2>
%\fi
%
% %%%%%%%%%%%%%%%%%%%%%%%%%%%%%%%%%%%%%%
% \paragraph{Part Include Files.}
%
% The include files are called |cdocspt3.tex| and |cdocspt4.tex|.
%
%\iffalse
%<*samplepart3|samplepart4>
%\fi

% Optional override for |\version| flag:
%    \begin{macrocode}
%%\providecommand{\version}{final}
%    \end{macrocode}

% Include the main document:
%    \begin{macrocode}
\input{childdoc.def}
\childdocby{cdocsamp}
%    \end{macrocode}

%\iffalse
%</samplepart3|samplepart4>
%\fi
%
%\iffalse
%<*samplepart3>
%\fi
% Some text for part 3:
%    \begin{macrocode}
some text in part three
%    \end{macrocode}

%\iffalse
%</samplepart3>
%\fi
% Some text for part 4:
%\iffalse
%<*samplepart4>
%\fi
%    \begin{macrocode}
more text in part four
%    \end{macrocode}

%\iffalse
%</samplepart4>
%\fi
%
% %%%%%%%%%%%%%%%%%%%%%%%%%%%%%%%%%%%%%%
% \paragraph{Forwarding for a Complete Draft.}
%
% The following forwarding file |cdocsdrf.tex|
% compiles the main document in draft mode:
%\iffalse
%<*sampledraft>
%\fi
%    \begin{macrocode}
\def\version{draft}
\input{childdoc.def}
\childdocforward{cdocsamp}
%    \end{macrocode}

%\iffalse
%</sampledraft>
%\fi
%
% %%%%%%%%%%%%%%%%%%%%%%%%%%%%%%%%%%%%%%
% \paragraph{Forwarding for Final Version of the Chapters.}
%
% The following forwarding files |cdocsfn1.tex| and |cdocsfn2.tex|
% (with identical content)
% compile the final versions of the child documents
% |cdocsch1.tex| and |cdocsch2.tex|, respectively:
%\iffalse
%<*samplefinal>
%\fi
%    \begin{macrocode}
\def\version{final}
\input{childdoc.def}
\childdocforwardprefix[cdocsamp]{cdocsfn}{cdocsch}
%    \end{macrocode}

%\iffalse
%</samplefinal>
%\fi
%
% %%%%%%%%%%%%%%%%%%%%%%%%%%%%%%%%%%%%%%
% \paragraph{Command Line Processing.}
%
% The following three command lines generate the output files
% |cdocscld|, |cdocscl1| and |cdocscl2|
% which should be identical to
% |cdocsdrf|, |cdocsch1| and |cdocsfn2|, respectively:
% \begin{center}
% \begin{tabular}{l}
% |latex -jobname cdocscld \|\\
% |  "\def\version{draft}\input{childdoc.def}\childdocforward{cdocsamp}"|\\
% |latex -jobname cdocscl1 \|\\
% |  "\input{childdoc.def}\childdocforward[cdocsamp]{cdocsch1}"|\\
% |latex -jobname cdocscl2 \|\\
% |  "\def\version{final}\input{childdoc.def}\childdocforward{cdocsch2}"|
% \end{tabular}
% \end{center}
% Note that the trailing backslash on each first line
% merely continues the input to the second line
% (for convenient cut ant paste).
% Furthermore, the command |latex| can be replaced by any
% of its alternative versions such as |pdflatex|.
%
% %%%%%%%%%%%%%%%%%%%%%%%%%%%%%%%%%%%%%%%%%%%%%%%%%%%%%%%%%%%%%%%%%%%%%%%%%%%%%%
% %%%%%%%%%%%%%%%%%%%%%%%%%%%%%%%%%%%%%%%%%%%%%%%%%%%%%%%%%%%%%%%%%%%%%%%%%%%%%%
% \section{Implementation}
%\iffalse
%<*package>
%\fi
%
% This section describes the definitions file |childdoc.def|.

% The definitions cannot be loaded using |\usepackage| or |\RequirePackage|
% which has a mechanism to prevent loading a style file more than once.
% When loading the definitions by means of |\input|
% multiple instances have to be prevented manually:
%\iffalse
%This code needs to be before the `\ProvidesFile' directive
%which is defined at the beginning of this file.
%Therefore it is also placed there and commented out here.
%</package>
%<*discard>
%\fi
%    \begin{macrocode}
\ifdefined\childdocmain\endinput\fi
%    \end{macrocode}
%\iffalse
%</discard>
%<*package>
%\fi
%
% \macro{\ifchilddoc}
% \macro{\ifchilddocmanual}
% The conditional |\ifchilddoc| tells whether a
% child (true) or main (false) document is being compiled.
% The conditional |\ifchilddocmanual| tells whether
% the |\includeonly| mechanism is used (false) or
% the selection of child files must be performed manually (true).
% The definitions initialise to false:
%    \begin{macrocode}
\newif\ifchilddoc
\newif\ifchilddocmanual
%    \end{macrocode}

% \macro{\childdocname}
% \macro{\childdocjob}
% The macro |\childdocname| stores the name of the main document
% to be compiled. The macro |\childdocjob| stores the name of
% the document on which the \LaTeX{} compiler was originally invoked.
% The content of |\jobname| cannot be compared
% to filenames specified in the source due to different catcodes.
% The following code rescans |\jobname|, stores the result
% in |\childdocname| and saves a copy in |\childdocjob|:
%    \begin{macrocode}
\edef\childdocname{\scantokens\expandafter{\jobname\noexpand}}
\let\childdocjob\childdocname
%    \end{macrocode}

% \macro{\childdocdisable}
% The macro |\childdocdisable| prevents the main file
% from being processed more than once.
% At this stage, the main document command |\childdocmain|
% is assumed to be called once again where it should do nothing.
% Any subsequent call to it should prevent
% a secondary processing of the main document
% It overwrites the forwarding commands
% |\childdocof| and |\childdocforward|
% with empty macros to prevent further inclusions of the main document:
%    \begin{macrocode}
\newcommand{\childdocdisable}
{
  \renewcommand{\childdocmain}[1]{\renewcommand{\childdocmain}[1]{\endinput}}
  \renewcommand{\childdocof}[1]{}
  \renewcommand{\childdocby}[2][]{}
  \renewcommand{\childdocforward}[2][]{}
  \renewcommand{\childdocdisable}{}
}
%    \end{macrocode}

% \macro{\childdocmain}
% The macro |\childdocmain| is to be called at the top of the main file
% with nothing or the main filename (without extension) as argument.
% First, it breaks loops.
% If the argument is not empty and does not match |\childdocname|
% (which is set by the first inclusion of |childdoc.def|),
% |\ifchilddoc| is set to true, |\includeonly| is applied to the child file
% and |\jobname| is set to the main file
% (for proper handling of |.aux| files):
%    \begin{macrocode}
\newcommand{\childdocmain}[1]
{
  \childdocdisable\childdocmain{}
  \if?#1?\else
    \begingroup
      \def\childdoctmp{#1}
      \ifx\childdoctmp\childdocname
        \def\childdoctmp{}
      \else
        \def\childdoctmp
        {
          \childdoctrue
          \includeonly{\childdocname}
          \def\childdocjob{#1}
          \def\jobname{#1}
        }
      \fi
      \expandafter
    \endgroup
    \childdoctmp
  \fi
}
%    \end{macrocode}

% \macro{\childdocof}
% The command |\childdocof| redirects
% compilation to the main file |#1|.
%    \begin{macrocode}
\newcommand{\childdocof}[1]
{
  \childdocdisable
  \childdoctrue
  \includeonly{\childdocname}
  \def\jobname{#1}
  \def\childdocjob{#1}
  \input{#1}
}
%    \end{macrocode}

% \macro{\childdocby}
% The command |\childdocby| ....
%    \begin{macrocode}
\newcommand{\childdocby}[2][]
{
  \childdocdisable
  \childdoctrue
  \childdocmanualtrue
  \if?#1?\else
    \def\jobname{#2}
  \fi
  \def\childdocjob{#2}
  \input{#2}
  \endinput
}
%    \end{macrocode}

% \macro{\childdocforward}
% The command |\childdocforward| redirects
% compilation to the main file or
% (if the optional argument is given) a child file.
% Parameters are set as if the main file
% or a child file starting with |\childdocof| was compiled.
% Then compilation is handed over to the main file:
%    \begin{macrocode}
\newcommand{\childdocforward}[2][]
{
  \begingroup
    \if?#1?
      \def\childdoctmp
      {
        \def\childdocname{#2}
        \def\childdocjob{#2}
        \def\jobname{#2}
        \input{#2}
        \endinput
      }
    \else
      \def\childdoctmp
      {
        \childdocdisable
        \def\childdocname{#2}
        \childdoctrue
        \includeonly{#2}
        \def\childdocjob{#1}
        \def\jobname{#1}
        \input{#1}
        \endinput
      }
    \fi
    \expandafter
  \endgroup
  \childdoctmp
}
%    \end{macrocode}

% \macro{\childdocforwardprefix}
% The command |\childdocforwardprefix| redirects
% compilation to the main or a child file by means of a pattern.
% The prefix |#1| in the current filename is replaced by |#2|
% and the suffix of the current filename is kept
% (it is assumed that the filename does not contain the substring `|~~~|'
% which is used as a delimiter).
% Compilation is handed over to the new file by |\childdocforward|:
%    \begin{macrocode}
\newcommand{\childdocforwardprefix}[3][]
{
  \begingroup
    \def\childdocextract #2##1~~~{\def\childdoctmp{\childdocforward[#1]{#3##1}}}
    \expandafter\childdocextract\childdocname~~~
    \expandafter
  \endgroup
  \childdoctmp
}
%    \end{macrocode}

% \macro{\childdoc}
% The deprecated macro |\childdoc| is a legacy version of |\childdocmain|:
%    \begin{macrocode}
\newcommand{\childdoc}{\childdocmain}
%    \end{macrocode}

% \macro{\childdocredirect}
% The deprecated macro |\childdocredirect| is a legacy version
% of |\childdocforward| and |\childdocforwardprefix|:
%    \begin{macrocode}
\newcommand{\childdocredirect}[2][]
{
  \begingroup
    \if?#1?
      \def\childdoctmp{\childdocforward{#2}}
    \else
      \def\childdoctmp{\childdocforwardprefix{#1}{#2}}
    \fi
    \expandafter
  \endgroup
  \childdoctmp
}
%    \end{macrocode}

%\iffalse
%</package>
%\fi
%
\endinput
\childdocforward[|\textit{main}|]{|\textit{dest}|}"|
\end{center}
%
Here \textit{target} is the name of the output file,
\textit{main} is the name of the main file
and \textit{dest} is the name of the main or child file to be processed
(all filenames without extensions).
The optional argument \textit{main} can be omitted
if \textit{main} matches \textit{dest}.
Optionally, compilation \textit{flags} can be defined via |\def| commands.
This command line makes the \TeX{} engine believe
it is compiling the file \textit{target}
whose content is specified as the latter parameter.
The provided code then forwards the processing to
\textit{main} or \textit{dest} as described in \secref{sec:forward}.

%%%%%%%%%%%%%%%%%%%%%%%%%%%%%%%%%%%%%%%%%%%%%%%%%%%%%%%%%%%%%%%%%%%%%%%%%%%%%%%%
\subsection{Include by Input}
\label{sec:input}

Including child documents by |\include| has some restrictions by design.
Most notably, the content of a child document always occupies
its own set of pages; pages cannot be shared between child documents.
Usually, this behaviour makes perfect sense
because each child document contain an essential part of the document.
However, in some situations it may be desirable to compose
a document from a collection of parts
without having mandatory page breaks between then.
For this case, the package
provides a mechanism to include parts
by |\input| which can also be processed individually.
However, by construction this mechanism
requires manual handling of the content to be output.

%%%%%%%%%%%%%%%%%%%%%%%%%%%%%%%%%%%%%%%%
\DescribeMacro{\ifchilddocmanual}
The main file should be prepared as usual, see \secref{sec:include}.
However, the document body must make a distinction
between processing of an individual part and of the main document, e.g.:
%
\begin{center}
\begin{tabular}{l}
|\ifchilddocmanual|\\
|\input{\childdocname}|\\
|\||else|\\
\textit{document body with }|\input{|\textit{part}|}|\\
|\||fi|
\end{tabular}
\end{center}
%
The conditional |\ifchilddocmanual| is true whenever
a part to be included by |\input| is being compiled,
and the name of the part is stored in |\childdocname|.

%%%%%%%%%%%%%%%%%%%%%%%%%%%%%%%%%%%%%%%%
\DescribeMacro{\childdocby}
Each part to be included by |\input| should start with:
%
\begin{center}
\begin{tabular}{l}
|% \iffalse
%
% childdoc.dtx Copyright (C) 2017-2018 Niklas Beisert
%
% This work may be distributed and/or modified under the
% conditions of the LaTeX Project Public License, either version 1.3
% of this license or (at your option) any later version.
% The latest version of this license is in
%   http://www.latex-project.org/lppl.txt
% and version 1.3 or later is part of all distributions of LaTeX
% version 2005/12/01 or later.
%
% This work has the LPPL maintenance status `maintained'.
%
% The Current Maintainer of this work is Niklas Beisert.
%
% This work consists of the files childdoc.dtx and childdoc.ins
% and the derived files childdoc.def and cdocsamp.tex with
% cdocsch1.tex, cdocsch2.tex, cdocsdrf.tex, cdocsfn1.tex, cdocsfn2.tex.
%
%<package>\ifdefined\childdocmain\endinput\fi
%<package>\ProvidesFile{childdoc.def}[2018/12/30 v2.0 child document driver]
%<samplemain>\ProvidesFile{cdocsamp.tex}[2018/12/30 v2.0 sample for childdoc]
%<*driver>
%\ProvidesFile{childdoc.drv}[2018/12/30 v2.0 childdoc reference manual file]
\PassOptionsToClass{10pt,a4paper}{article}
\documentclass{ltxdoc}

\usepackage[margin=35mm]{geometry}
\usepackage{hyperref}
\usepackage{hyperxmp}
\usepackage[usenames]{color}

\hypersetup{colorlinks=true}
\hypersetup{pdfstartview=FitH}
\hypersetup{pdfpagemode=UseNone}
\hypersetup{pdfsource={}}
\hypersetup{pdflang={en-UK}}
\hypersetup{pdfcopyright={Copyright 2017-2018 Niklas Beisert.
  This work may be distributed and/or modified under the
  conditions of the LaTeX Project Public License, either version 1.3
  of this license or (at your option) any later version.}}
\hypersetup{pdflicenseurl={http://www.latex-project.org/lppl.txt}}
\hypersetup{pdfcontactaddress={ETH Zurich, ITP, HIT K,
  Wolfgang-Pauli-Strasse 27}}
\hypersetup{pdfcontactpostcode={8093}}
\hypersetup{pdfcontactcity={Zurich}}
\hypersetup{pdfcontactcountry={Switzerland}}
\hypersetup{pdfcontactemail={nbeisert@itp.phys.ethz.ch}}
\hypersetup{pdfcontacturl={http://people.phys.ethz.ch/\xmptilde nbeisert/}}

\newcommand{\secref}[1]{\hyperref[#1]{section \ref*{#1}}}

\parskip1ex
\parindent0pt
\let\olditemize\itemize
\def\itemize{\olditemize\parskip0pt}

\begin{document}

\title{The \textsf{childdoc} Package}
\hypersetup{pdftitle={The childdoc Package}}
\author{Niklas Beisert\\[2ex]
  Institut f\"ur Theoretische Physik\\
  Eidgen\"ossische Technische Hochschule Z\"urich\\
  Wolfgang-Pauli-Strasse 27, 8093 Z\"urich, Switzerland\\[1ex]
  \href{mailto:nbeisert@itp.phys.ethz.ch}
  {\texttt{nbeisert@itp.phys.ethz.ch}}}
\hypersetup{pdfauthor={Niklas Beisert}}
\hypersetup{pdfsubject={Manual for the LaTeX2e Package childdoc}}
\date{30 December 2018, \textsf{v2.0}}
\maketitle

\begin{abstract}\noindent
\textsf{childdoc} is a \LaTeXe{} package
that enables the direct compilation
of document sections included by |\include|
to individual files.
\end{abstract}

\begingroup
\parskip0ex
\tableofcontents
\endgroup

%%%%%%%%%%%%%%%%%%%%%%%%%%%%%%%%%%%%%%%%%%%%%%%%%%%%%%%%%%%%%%%%%%%%%%%%%%%%%%%%
%%%%%%%%%%%%%%%%%%%%%%%%%%%%%%%%%%%%%%%%%%%%%%%%%%%%%%%%%%%%%%%%%%%%%%%%%%%%%%%%
\section{Introduction}

\LaTeX{} provides a mechanism to structure a large document (such as a book)
into a main file and several child files (containing the chapters)
using the |\include| command.
This mechanism is beneficial for documents
which span hundreds of pages in order to
make the source file(s) more manageable.
Moreover, compilation can be restricted to
selected child files by means of the |\includeonly| command.
The latter feature can be used to reduce the compilation time while editing
(this was significantly more useful in the earlier days of \LaTeX{})
or to generate a smaller document which is easier to navigate.
Another application of |\includeonly| is to generate
documents consisting of selected parts of the complete document.

However, there are a few drawbacks of the plain |\include| mechanism:
\begin{itemize}
\item
The child files cannot be compiled on their own,
they can only be compiled via the main file.
A naive editing environment
(such as a text editor with an option
to have the current file processed by \LaTeX)
may require one to switch to the main file before compiling;
attempting to compile the child file produces errors.
\item
The main file must be modified (each time)
to adjust the |\includeonly| command
to the present needs. This easily leaves the main file in a messy state.
\item
The generated document will always carry the filename
of the main document. This is inconvenient if
several child files are to be compiled and
to be kept for distribution.
\end{itemize}

The present package provides a simple interface
to make child files individually compilable by \LaTeX{}.
Compiling a child file then has the same effect as compiling
the main file with an |\includeonly| command
to select the appropriate child.
Moreover the generated document will carry the name of the child
rather than the main file.
This resolves all three above issues.

This feature is meant to make the editing of books,
thesis documents and lecture notes somewhat more convenient.
However, the package can also be used efficiently for
composing a series of documents (such as exercise sheets)
which are typically distributed individually.
It then assists the author in generating the individual documents
(potentially in different versions)
as well as a document containing the collected series.
Another application is in developing style files
or other kinds of included material
where compilation of the style file could redirect
to a sample or test file.

%%%%%%%%%%%%%%%%%%%%%%%%%%%%%%%%%%%%%%%%%%%%%%%%%%%%%%%%%%%%%%%%%%%%%%%%%%%%%%%%
%%%%%%%%%%%%%%%%%%%%%%%%%%%%%%%%%%%%%%%%%%%%%%%%%%%%%%%%%%%%%%%%%%%%%%%%%%%%%%%%
\section{Usage}

First of all, the package \textsf{childdoc} is \emph{not} a standard
\LaTeXe{} |.sty| style file! Therefore it needs to be invoked in
a non-standard way.

%%%%%%%%%%%%%%%%%%%%%%%%%%%%%%%%%%%%%%%%%%%%%%%%%%%%%%%%%%%%%%%%%%%%%%%%%%%%%%%%
\subsection{Included Files}
\label{sec:include}

%%%%%%%%%%%%%%%%%%%%%%%%%%%%%%%%%%%%%%%%
\DescribeMacro{\childdocmain}
To use the package, add the commands
\begin{center}
\begin{tabular}{l}
|\input{childdoc.def}|\\
|\childdocmain{}|\\
\end{tabular}
\end{center}
at the very top of the main \LaTeX{} file,
in particular \emph{before} the |\documentclass| statement!
The argument of |\childdocmain| should be left empty
(but it must be present).

%%%%%%%%%%%%%%%%%%%%%%%%%%%%%%%%%%%%%%%%
\DescribeMacro{\childdocof}
Furthermore, add the commands
\begin{center}
\begin{tabular}{l}
|\input{childdoc.def}|\\
|\childdocof{|\textit{main}|}|\\
\end{tabular}
\end{center}
at the top of every child file \textit{child}
which is included by |\include{|\textit{child}|}|
from within the main file
(or at least for those files to be compiled individually).
The argument \textit{main} must be the filename of the main file.

There are a couple of
considerations in setting up the main and child documents:

%%%%%%%%%%%%%%%%%%%%%%%%%%%%%%%%%%%%%%%%
\paragraph{Restrictions.}

Please note the following restrictions:
\begin{itemize}
\item
|\childdocmain| must be called with one argument \textit{main}
to ensure compatibility with earlier version of the package.
It must either be empty (|\childdocmain{}|)
or precisely match the filename of the main file in which it is specified.
See \secref{sec:detection} for further information.
\item
The filename \textit{main} must be specified without the |.tex| extension.
\item
The filename \textit{main} is case sensitive
(even in case-insensitive file systems)
due to internal string comparison.
\item
The argument \textit{main} should be fully expanded, it cannot be a macro.
\item
Subdirectories and special characters should be avoided in filenames.
\item
The command |\childdocmain{|\textit{main}|}| must be followed by a whitespace.
It should not be followed immediately by another command
or by a comment mark `|%|'.
This is because the \TeX{} parser reads the token immediately following
the argument of |\childdocmain| and puts it
at the beginning of every child section;
however, a white\-space is ignored.
\end{itemize}

%%%%%%%%%%%%%%%%%%%%%%%%%%%%%%%%%%%%%%%%
\paragraph{Content of Main File.}

It is advisable to place all content in the child files included by |\include|.
Any output contained in the main file will appear in all child documents
unless suppressed manually;
it cannot be suppressed automatically by the |\includeonly| directive
and thus should normally be avoided.
A method to include some content in the main file
by means of conditional processing is described in \secref{sec:conditional}.

%%%%%%%%%%%%%%%%%%%%%%%%%%%%%%%%%%%%%%%%
\paragraph{Page Numbering.}

When only a part of the document is compiled,
the appropriate numbering of pages
(as well as other status parameters)
is determined from the |.aux| files.
The latter contain information from previous passes.
However this information needs to propagate through
all intermediate child documents.
Therefore the page numbering in child documents may well
be inconsistent until the complete document is compiled at least once.

A useful (if unconventional) way to always ensure a consistent
page numbering is to restart the numbering in each child document
and denote the pages by `\textit{child}|.|\textit{page}'
where \textit{child} represents the chapter/section number of the child file.
This can be achieved by the command
|\numberwithin{page}{|\textit{child}|}|
of the \textsf{amsmath} package
where \textit{child} can be |chapter| or |section|
depending on the chosen structuring.
Alternatively, one can modify the macro |\thepage| appropriately
and reset the counter |page| at the start of each child file.

%%%%%%%%%%%%%%%%%%%%%%%%%%%%%%%%%%%%%%%%%%%%%%%%%%%%%%%%%%%%%%%%%%%%%%%%%%%%%%%%
\subsection{Conditional Processing}
\label{sec:conditional}

The package provides a mechanism to compile different versions
of a document. To customise the versions further some conditional processing
can come in handy to distinguish which version is being compiled.
The package provides two macros to describe the compilation context:

%%%%%%%%%%%%%%%%%%%%%%%%%%%%%%%%%%%%%%%%
\DescribeMacro{\ifchilddoc}
The conditional |\ifchilddoc| distinguishes between the compilation of
child documents and the main document:
%
\begin{center}
|\ifchilddoc |\textit{child-code}| |[|\||else |\textit{main-code}]| \||fi|
\end{center}

%%%%%%%%%%%%%%%%%%%%%%%%%%%%%%%%%%%%%%%%
\DescribeMacro{\childdocname}
\DescribeMacro{\childdocjob}
The macro |\childdocname| contains the filename (without extension)
of the main or child file being processed.
Note that |\childdocjob| will always contain the name of the main file.

%%%%%%%%%%%%%%%%%%%%%%%%%%%%%%%%%%%%%%%%
\paragraph{Title Page.}

Conditional processing can be used to include a title or banner page
in the main document when proper precautions are taken.
Importantly, the code in the main file should ensure that the page counter
(as well as other status parameters which are stored in the |.aux| files)
takes the same value after the conditional processing.
Otherwise the page numbers may take divergent values
depending on which part is compiled.

For example, a title page could be declared by:
%
\begin{center}
\begin{tabular}{l}
|\ifchilddoc\||else|\\
|\addtocounter{page}{-1}|\\
\textit{code for title page}\\
|\newpage|\\
|\||fi|
\end{tabular}
\end{center}
%
A banner page for the child documents can be generated by:
%
\begin{center}
\begin{tabular}{l}
|\ifchilddoc|\\
|\addtocounter{page}{-1}|\\
\textit{code for banner page}\\
|\newpage|\\
|\||fi|
\end{tabular}
\end{center}
%
Here one could write a message such as:
\begin{center}
|This is the part \childdocname{} of \childdocjob{}.|
\end{center}

%%%%%%%%%%%%%%%%%%%%%%%%%%%%%%%%%%%%%%%%%%%%%%%%%%%%%%%%%%%%%%%%%%%%%%%%%%%%%%%%
\subsection{Flags}
\label{sec:flags}

The package makes it easy to generate different versions
of the main or child documents.
To this end compilation flags can be defined
and assigned different default values.
They will be particularly useful in conjunction
with the forwarding mechanism described in \secref{sec:forward}.

For example, it may be useful to have a flag |\version|
which can be set to |draft| or |final|.
The document source will contain some conditional code
depending on the value of |\version|.
Suppose further, the flag should default to |final| for the main file
and to |draft| for child files
which is a natural assignment for editing the document.
This is achieved by placing the following code
in the preamble of the main document
(below the |\childdocmain| directive):
%
\begin{center}
\begin{tabular}{l}
|\ifchilddoc|\\
|\providecommand{\version}{draft}|\\
|\||else|\\
|\providecommand{\version}{final}|\\
|\||fi|
\end{tabular}
\end{center}
%
The definition by |\providecommand| makes sure
that previous definitions are not overwritten.
Further statements |\providecommand{\version}{...}|
can thus be added before the above code to override it.

For the main file, one might add a line
(between |\childdocmain| and the above block)
%
\begin{center}
|%\ifchilddoc\||else\providecommand{\version}{draft}\||fi|
\end{center}
%
which can be uncommented to produce a draft version.
Likewise one can add a line to the very top of a child file
(above the |\childdocof{|\textit{main}|}| directive)
%
\begin{center}
|%\providecommand{\version}{final}|
\end{center}
%
which can be uncommented to produce the final version of this child document.

%%%%%%%%%%%%%%%%%%%%%%%%%%%%%%%%%%%%%%%%%%%%%%%%%%%%%%%%%%%%%%%%%%%%%%%%%%%%%%%%
\subsection{Forwarding}
\label{sec:forward}

Different versions of the main or child documents
using compilation flags as described in \secref{sec:flags}
can be (permanently) stored in different files
for convenient compilation, viewing and distribution.
To this end, the package defines a command
to pass on compilation to a different file:

%%%%%%%%%%%%%%%%%%%%%%%%%%%%%%%%%%%%%%%%
\DescribeMacro{\childdocforward}
The command |\childdocforward| redirects processing to
another source file:
%
\begin{center}
\begin{tabular}{l}
|\input{childdoc.def}|\\
|\childdocforward[|\textit{main}|]{|\textit{dest}|}|\\
\end{tabular}
\end{center}
%
The argument \textit{dest} is the destination file
(without extension).
It should be the main file or one of the child files.
Note that further \textsf{childdoc} directives
such as |\childdocof| and |\childdocforward|
in the indicated file will be processed in this form.
The optional argument \textit{main}
passes on directly to the main file \textit{main}
while pretending to compile the child \textit{dest}.
This form behaves as if \textit{dest}
issues |\childdocof{|\textit{main}|}| right away,
and no further \textsf{childdoc} directives will be processed.

%%%%%%%%%%%%%%%%%%%%%%%%%%%%%%%%%%%%%%%%
\DescribeMacro{\...prefix}
In the alternative form |\childdocforwardprefix|,
%
\begin{center}
\begin{tabular}{l}
|\input{childdoc.def}|\\
|\childdocforwardprefix[|\textit{main}|]{|\textit{prefix}|}{|\textit{dest}|}|
\end{tabular}
\end{center}
%
the destination file is determined by a pattern
depending on the current file:
To make this work, the current file must be called
`{\textit{prefix}\hspace{0.2em}\textit{suffix}}'
with \textit{prefix} matching precisely the argument.
Processing is then passed on to the file
`{\textit{dest}\hspace{0.2em}\textit{suffix}}'.
Surely, the same effect is achieved by
directly specifying the
argument `{\textit{dest}\hspace{0.2em}\textit{suffix}}'
in the first form.
However, that requires to set up a different file
for each child. With the alternative form of the command
all these files can have exactly the same content
which simplifies setting them up and maintaining them.

For example, the following file |draft.tex|
with a compilation flag |\version| as described in \secref{sec:flags}
compiles the main document as a draft:
%
\begin{center}
\begin{tabular}{l}
|\def\version{draft}|\\
|\input{childdoc.def}|\\
|\childdocforward{|\textit{main}|}|
\end{tabular}
\end{center}
%
Likewise, the following files |final|\textit{nn}|.tex|
compile the final version of the child document
|child|\textit{nn}|.tex|:
%
\begin{center}
\begin{tabular}{l}
|\def\version{final}|\\
|\input{childdoc.def}|\\
|\childdocforwardprefix{final}{child}|
\end{tabular}
\end{center}
%

Note that when several versions of a main file and/or of each child file
are to be generated, it may be convenient to set up a |Makefile| or
shell script to automatise the process.

%%%%%%%%%%%%%%%%%%%%%%%%%%%%%%%%%%%%%%%%%%%%%%%%%%%%%%%%%%%%%%%%%%%%%%%%%%%%%%%%
\subsection{Command Line Processing}
\label{sec:commandline}

The effect of redirection files can also be achieved by invoking
the \LaTeX{} compiler with a more elaborate command line.
Most conveniently this should be done as part
of a shell script or a |Makefile|.

When using \textsf{childdoc} in the main file, the following
command lines effectively perform a redirection
(note that depending on the shell being used,
backslashes may have to be doubled: `|\|' $\to$ `|\\|'):
%
\begin{center}
|... -jobname "|\textit{target}|" |\\|"|[\textit{flags}]%
|\input{childdoc.def}\childdocforward[|\textit{main}|]{|\textit{dest}|}"|
\end{center}
%
Here \textit{target} is the name of the output file,
\textit{main} is the name of the main file
and \textit{dest} is the name of the main or child file to be processed
(all filenames without extensions).
The optional argument \textit{main} can be omitted
if \textit{main} matches \textit{dest}.
Optionally, compilation \textit{flags} can be defined via |\def| commands.
This command line makes the \TeX{} engine believe
it is compiling the file \textit{target}
whose content is specified as the latter parameter.
The provided code then forwards the processing to
\textit{main} or \textit{dest} as described in \secref{sec:forward}.

%%%%%%%%%%%%%%%%%%%%%%%%%%%%%%%%%%%%%%%%%%%%%%%%%%%%%%%%%%%%%%%%%%%%%%%%%%%%%%%%
\subsection{Include by Input}
\label{sec:input}

Including child documents by |\include| has some restrictions by design.
Most notably, the content of a child document always occupies
its own set of pages; pages cannot be shared between child documents.
Usually, this behaviour makes perfect sense
because each child document contain an essential part of the document.
However, in some situations it may be desirable to compose
a document from a collection of parts
without having mandatory page breaks between then.
For this case, the package
provides a mechanism to include parts
by |\input| which can also be processed individually.
However, by construction this mechanism
requires manual handling of the content to be output.

%%%%%%%%%%%%%%%%%%%%%%%%%%%%%%%%%%%%%%%%
\DescribeMacro{\ifchilddocmanual}
The main file should be prepared as usual, see \secref{sec:include}.
However, the document body must make a distinction
between processing of an individual part and of the main document, e.g.:
%
\begin{center}
\begin{tabular}{l}
|\ifchilddocmanual|\\
|\input{\childdocname}|\\
|\||else|\\
\textit{document body with }|\input{|\textit{part}|}|\\
|\||fi|
\end{tabular}
\end{center}
%
The conditional |\ifchilddocmanual| is true whenever
a part to be included by |\input| is being compiled,
and the name of the part is stored in |\childdocname|.

%%%%%%%%%%%%%%%%%%%%%%%%%%%%%%%%%%%%%%%%
\DescribeMacro{\childdocby}
Each part to be included by |\input| should start with:
%
\begin{center}
\begin{tabular}{l}
|\input{childdoc.def}|\\
|\childdocby{|\textit{main}|}|\\
\end{tabular}
\end{center}
%
The directive |\childdocby| is similar to |\childdocof|
described in \secref{sec:include},
but the subsequent selection of content must be done manually.
To that end, both |\ifchilddoc| and |\ifchilddocmanual|
will be true upon processing of a part,
and the name of the part is stored in |\childdocname|.
Note that |\jobname| will be set to the filename of the current part
so that each part receives an individual |.aux| file
that does not interfere with the |.aux| file(s) of the main document.
This behaviour can be altered by the alternative form
|\childdocby[*]{|\textit{main}|}| (with a non-empty optional argument)
which uses the |.aux| file of the main document
by setting |\jobname| to \textit{main}.

%%%%%%%%%%%%%%%%%%%%%%%%%%%%%%%%%%%%%%%%%%%%%%%%%%%%%%%%%%%%%%%%%%%%%%%%%%%%%%%%
\subsection{Driver Development}
\label{sec:driver}

The \textsf{childdoc} mechanism can also be use for the development
of definition files such as \LaTeX{} styles or classes.
This case differs from the above setup with multiple parts
included by |\include| in that no |\includeonly| should be invoked.
This can be achieved by starting the include file
(before |\ProvidesPackage|) with:
%
\begin{center}
\begin{tabular}{l}
|\input{childdoc.def}|\\
|\childdocforward{|\textit{main}|}|\\
\end{tabular}
\end{center}
%
or alternatively with:
%
\begin{center}
\begin{tabular}{l}
|\input{childdoc.def}|\\
|\childdocby{|\textit{main}|}|\\
\end{tabular}
\end{center}
%
Both forms have slightly different effects as described above.
The main file is prepared as usual, see \secref{sec:include}.

%%%%%%%%%%%%%%%%%%%%%%%%%%%%%%%%%%%%%%%%%%%%%%%%%%%%%%%%%%%%%%%%%%%%%%%%%%%%%%%%
\subsection{Legacy Detection}
\label{sec:detection}

The directive |\childdocmain| in the main file can detect
whether the complete document or merely a child is to be compiled
even without using the directive |\childdocof|.
This method is deprecated because it is less robust
and there is no compelling reason to use it;
it is merely provided for backward compatibility
and it may be removed in future versions.

If the detection mechanism is to be used,
it is mandatory to correctly specify
the filename of the main file as the argument of |\childdocmain|:
%
\begin{center}
\begin{tabular}{l}
|\input{childdoc.def}|\\
|\childdocmain{|\textit{main}|}|\\
\end{tabular}
\end{center}
%
If |\jobname| does not match the argument \textit{main} of |\childdocmain|,
it is assumed that |\jobname| points to the child file to be compiled.
When using |\childdocmain| with the main file specified as argument,
it suffices to start a child file
with just |\input{|\textit{main}|}|
without loading of the package and using |\childdocof|.
If instead all processing is done
with the appropriate \textsf{childdoc} directives,
the argument of \textit{main} of |\childdocmain| can be empty.

An alternative version of the command line processing described
in \secref{sec:commandline} using the detection mechanism reads:
%
\begin{center}
|... -jobname "|\textit{target}|" "|[\textit{flags}]%
[|\def\jobname{|\textit{dest}|}|]|\input{|\textit{main}|}"|
\end{center}

%%%%%%%%%%%%%%%%%%%%%%%%%%%%%%%%%%%%%%%%%%%%%%%%%%%%%%%%%%%%%%%%%%%%%%%%%%%%%%%%
\subsection{Manual Code}
\label{sec:manual}

In case one cannot be certain whether the definitions file |childdoc.def|
is installed on the target \TeX{} distribution
and one prefers not to ship it,
it is conceivable to paste a few relevant commands into the sources.

To that end, drop all statements |\input{childdoc.def}|
and perform the replacements as outlined below.
Instead of |\childdocmain{|\textit{main}|}| add the following code
to the top of the main file:
%
\begin{center}
\begin{tabular}{l}
|\||ifdefined\childdocname\endinput\||fi\newif\ifchilddoc|\\
|\edef\childdocname{\scantokens\expandafter{\jobname\noexpand}}|\\
|\def\childdocmain{|\textit{main}|}\||ifx\childdocmain\childdocname\||else|\\
|\childdoctrue\includeonly{\childdocname}\let\jobname\childdocmain\||fi|\\
\end{tabular}
\end{center}
%
Instead of |\childdocof{|\textit{main}|}| just include the main file
at the top of each child file:
%
\begin{center}
|\input{|\textit{main}|}|
\end{center}
%
A simple redirection |\childdocforward{|\textit{dest}|}| is achieved by:
%
\begin{center}
|\def\jobname{|\textit{dest}|}\input{\jobname}|
\end{center}
%
The redirection with prefix
|\childdocforwardprefix[|\textit{prefix}|]{|\textit{dest}|}|
is accomplished by:
%
\begin{center}
\begin{tabular}{l}
|{\edef\jobname{\scantokens\expandafter{\jobname\noexpand}}|\\
|\def\redirectjob |\textit{prefix}|#1~~~{\gdef\jobname{|\textit{dest}|#1}}|\\
|\expandafter\redirectjob\jobname~~~}\input{\jobname}|
\end{tabular}
\end{center}

In an alternative approach,
child documents can be compiled by a specific command line
without additional code or specific definitions:
%
\begin{center}
|... -jobname "|\textit{target}|" "|[\textit{flags}]%
|\includeonly{|\textit{dest}|}\input{|\textit{main}|}"|
\end{center}
%

%%%%%%%%%%%%%%%%%%%%%%%%%%%%%%%%%%%%%%%%%%%%%%%%%%%%%%%%%%%%%%%%%%%%%%%%%%%%%%%%
%%%%%%%%%%%%%%%%%%%%%%%%%%%%%%%%%%%%%%%%%%%%%%%%%%%%%%%%%%%%%%%%%%%%%%%%%%%%%%%%
\section{Information}

%%%%%%%%%%%%%%%%%%%%%%%%%%%%%%%%%%%%%%%%%%%%%%%%%%%%%%%%%%%%%%%%%%%%%%%%%%%%%%%%
\subsection{Copyright}

Copyright \copyright{} 2017--2018 Niklas Beisert

This work may be distributed and/or modified under the
conditions of the \LaTeX{} Project Public License, either version 1.3
of this license or (at your option) any later version.
The latest version of this license is in
  \url{http://www.latex-project.org/lppl.txt}
and version 1.3 or later is part of all distributions of \LaTeX{}
version 2005/12/01 or later.

This work has the LPPL maintenance status `maintained'.

The Current Maintainer of this work is Niklas Beisert.

This work consists of the files |README.txt|, |childdoc.ins| and |childdoc.dtx|
as well as the derived files |childdoc.def|, |cdocsamp.tex|
with |cdocsch1.tex|, |cdocsch2.tex|, |cdocspt3.tex|, |cdocspt4.tex|,
|cdocsdrf.tex|, |cdocsfn1.tex|, |cdocsfn2.tex|
as well as |childdoc.pdf|.

%%%%%%%%%%%%%%%%%%%%%%%%%%%%%%%%%%%%%%%%%%%%%%%%%%%%%%%%%%%%%%%%%%%%%%%%%%%%%%%%
\subsection{Files and Installation}

The package consists of the files:
%
\begin{center}
\begin{tabular}{ll}
    |README.txt|   & readme file \\
    |childdoc.ins| & installation file \\
    |childdoc.dtx| & source file \\
    |childdoc.def| & definition file \\
    |cdocsamp.tex| & sample main file \\
    |cdocsch1.tex| & sample include file \\
    |cdocsch2.tex| & sample include file \\
    |cdocspt3.tex| & sample part file \\
    |cdocspt4.tex| & sample part file \\
    |cdocsdrf.tex| & sample redirection file \\
    |cdocsfn1.tex| & sample redirection file \\
    |cdocsfn2.tex| & sample redirection file \\
    |childdoc.pdf| & manual
\end{tabular}
\end{center}
%
The distribution consists of the files
|README.txt|, |childdoc.ins| and |childdoc.dtx|.
%
\begin{itemize}
\item
Run (pdf)\LaTeX{} on |childdoc.dtx|
to compile the manual |childdoc.pdf| (this file).
\item
Run \LaTeX{} on |childdoc.ins| to create the definitions file |childdoc.def|
and the sample |cdocsamp.tex| with include files
|cdocsch1.tex|, |cdocsch2.tex|, |cdocspt3.tex|, |cdocspt4.tex|,
|cdocsdrf.tex|, |cdocsfn1.tex|, |cdocsfn2.tex|.
Then copy the file |childdoc.def| to an appropriate directory of your \LaTeX{}
distribution, e.g.\ \textit{texmf-root}|/tex/latex/childdoc|.
\end{itemize}

%%%%%%%%%%%%%%%%%%%%%%%%%%%%%%%%%%%%%%%%%%%%%%%%%%%%%%%%%%%%%%%%%%%%%%%%%%%%%%%%
\subsection{Related CTAN Packages}

There are several other packages which offer a similar functionality:
%
\begin{itemize}
\item
The packages
\href{http://ctan.org/pkg/docmute}{\textsf{docmute}},
\href{http://ctan.org/pkg/includex}{\textsf{includex}} and
\href{http://ctan.org/pkg/standalone}{\textsf{standalone}}
provide commands to include only the document body of
a child file thus allowing both files to be compiled individually.
\item
The packages \href{http://ctan.org/pkg/subdocs}{\textsf{subdocs}}
and \href{http://ctan.org/pkg/subfiles}{\textsf{subfiles}}
provide structures in which the main and child documents can be
encapsulated and allowing them to be compiled individually.
The inclusion mechanism is different from the conventional |\include|.
\item
The package \href{http://ctan.org/pkg/combine}{\textsf{combine}}
is an elaborate solution to combine several documents into one.
\end{itemize}
%
See also the CTAN topic \href{http://ctan.org/topic/subdocs}{\textsf{subdocs}}
for further related packages.
The present package differs from the above solutions in that
a document structure constructed with the conventional |\include| mechanism
just needs two extra commands at the top of every file
such that all constituent files can be compiled individually.

%%%%%%%%%%%%%%%%%%%%%%%%%%%%%%%%%%%%%%%%%%%%%%%%%%%%%%%%%%%%%%%%%%%%%%%%%%%%%%%%
%\subsection{Feature Suggestions}
%
%The following is a list of features which may be useful for future
%versions of this package:
%%
%\begin{itemize}
%\item
%\ldots
%\end{itemize}

%%%%%%%%%%%%%%%%%%%%%%%%%%%%%%%%%%%%%%%%%%%%%%%%%%%%%%%%%%%%%%%%%%%%%%%%%%%%%%%%
\subsection{Revision History}

%%%%%%%%%%%%%%%%%%%%%%%%%%%%%%%%%%%%%%%%
\paragraph{v2.0:} 2018/12/30

\begin{itemize}
\item
immediate forward processing
\item
added |\childdocby| mechanism
\item
manual restructured
\end{itemize}

%%%%%%%%%%%%%%%%%%%%%%%%%%%%%%%%%%%%%%%%
\paragraph{v1.6:} 2018/01/17

\begin{itemize}
\item
application for development of include files
\item
corrections to manual
\end{itemize}

%%%%%%%%%%%%%%%%%%%%%%%%%%%%%%%%%%%%%%%%
\paragraph{v1.5:} 2017/05/21

\begin{itemize}
\item
more complete structuring introduced
\item
|\childdocof| introduced
\item
|\childdoc| renamed to |\childdocmain|
\item
|\childredirect| renamed to |\childdocforward| and |\childdocforwardprefix|
and functionality expanded
\end{itemize}

%%%%%%%%%%%%%%%%%%%%%%%%%%%%%%%%%%%%%%%%
\paragraph{v1.0:} 2017/04/27

\begin{itemize}
\item
manual and install package
\item
first version published on CTAN
\end{itemize}

%%%%%%%%%%%%%%%%%%%%%%%%%%%%%%%%%%%%%%%%
\paragraph{v0.6:} 2017/04/26

\begin{itemize}
\item
redirection mechanism added
\end{itemize}

%%%%%%%%%%%%%%%%%%%%%%%%%%%%%%%%%%%%%%%%
\paragraph{v0.5:} 2017/04/26

\begin{itemize}
\item
functionality in definition file
\end{itemize}


%%%%%%%%%%%%%%%%%%%%%%%%%%%%%%%%%%%%%%%%%%%%%%%%%%%%%%%%%%%%%%%%%%%%%%%%%%%%%%%%
%%%%%%%%%%%%%%%%%%%%%%%%%%%%%%%%%%%%%%%%%%%%%%%%%%%%%%%%%%%%%%%%%%%%%%%%%%%%%%%%
%%%%%%%%%%%%%%%%%%%%%%%%%%%%%%%%%%%%%%%%%%%%%%%%%%%%%%%%%%%%%%%%%%%%%%%%%%%%%%%%
\appendix

\settowidth\MacroIndent{\rmfamily\scriptsize 000\ }

 \DocInput{childdoc.dtx}

\end{document}
%</driver>
% \fi
%
% %%%%%%%%%%%%%%%%%%%%%%%%%%%%%%%%%%%%%%%%%%%%%%%%%%%%%%%%%%%%%%%%%%%%%%%%%%%%%%
% %%%%%%%%%%%%%%%%%%%%%%%%%%%%%%%%%%%%%%%%%%%%%%%%%%%%%%%%%%%%%%%%%%%%%%%%%%%%%%
% \section{Sample}
%\iffalse
%<*samplemain>
%\fi
%
% The following presents a sample document
% with two chapters, two parts, a title page,
% a compile flag as well as three forwarding files to set the flag.
% It consists of eight |.tex| files:
% \begin{center}
% \begin{tabular}{ll}
% |cdocsamp.tex|&main file\\
% |cdocsch1.tex|&include file for chapter 1\\
% |cdocsch2.tex|&include file for chapter 2\\
% |cdocspt3.tex|&include file for part 3\\
% |cdocspt4.tex|&include file for part 4\\
% |cdocsdrf.tex|&forwarding file for main file in draft mode\\
% |cdocsfi1.tex|&forwarding file for final version of chapter 1\\
% |cdocsfi2.tex|&forwarding file for final version of chapter 2\\
% \end{tabular}
% \end{center}
% Each of the eight files can be compiled directly by the \LaTeX{} compiler.
%
% %%%%%%%%%%%%%%%%%%%%%%%%%%%%%%%%%%%%%%
% \paragraph{Main File.}
%
% The main file is called |cdocsamp.tex|.
%
% Load the \textsf{childdoc} definitions and
% declare the filename for the main document:
%    \begin{macrocode}
\input{childdoc.def}
\childdocmain{}
%    \end{macrocode}

% Optional override for |\version| flag:
%    \begin{macrocode}
%%\ifchilddoc\else\providecommand{\version}{draft}\fi
%    \end{macrocode}

% Define the default values for the |\version| flag
% (|final| for the main file and |draft| for childs):
%    \begin{macrocode}
\ifchilddoc
\providecommand{\version}{draft}
\else
\providecommand{\version}{final}
\fi
%    \end{macrocode}

% Load the standard document class:
%    \begin{macrocode}
\documentclass[12pt]{article}
%    \end{macrocode}

% Start the document body:
%    \begin{macrocode}
\begin{document}
%    \end{macrocode}

% Declare a title page.
% Print title, part of document being processed and version flag:
%    \begin{macrocode}
\addtocounter{page}{-1}
\begin{center}
{\LARGE\bfseries{}childdoc example\par}
\vspace{1cm}
\ifchilddoc
\ifchilddocmanual part\else chapter\fi:
`\childdocname' of `\childdocjob'\par
\else
main document: `\childdocjob'\par
\fi
version: \version\par
\end{center}
\newpage
%    \end{macrocode}

% Manually include selected file,
% otherwise process as usual:
%    \begin{macrocode}
\ifchilddocmanual
\section*{part `\childdocname'}
\input{\childdocname}
\else
%    \end{macrocode}

% Include the two chapters:
%    \begin{macrocode}
\include{cdocsch1}
\include{cdocsch2}
%    \end{macrocode}

% Include the two parts unless only chapters should be displayed:
%    \begin{macrocode}
\ifchilddoc\else
\section{part three}
\input{cdocspt3}
\section{part four}
\input{cdocspt4}
\fi
%    \end{macrocode}

% Process as usual until here:
%    \begin{macrocode}
\fi
%    \end{macrocode}

% End of document body:
%    \begin{macrocode}
\end{document}
%    \end{macrocode}
%\iffalse
%</samplemain>
%\fi
%
% %%%%%%%%%%%%%%%%%%%%%%%%%%%%%%%%%%%%%%
% \paragraph{Chapter Include Files.}
%
% The include files are called |cdocsch1.tex| and |cdocsch2.tex|.
%
%\iffalse
%<*samplechap1|samplechap2>
%\fi

% Optional override for |\version| flag:
%    \begin{macrocode}
%%\providecommand{\version}{final}
%    \end{macrocode}

% Include the main document:
%    \begin{macrocode}
\input{childdoc.def}
\childdocof{cdocsamp}
%    \end{macrocode}

%\iffalse
%</samplechap1|samplechap2>
%\fi
%
%\iffalse
%<*samplechap1>
%\fi
% Some text for chapter 1:
%    \begin{macrocode}
\section{one}
some text in chapter one
%    \end{macrocode}

%\iffalse
%</samplechap1>
%\fi
% Some text for chapter 2:
%\iffalse
%<*samplechap2>
%\fi
%    \begin{macrocode}
\section{two}
more text in chapter two
%    \end{macrocode}

%\iffalse
%</samplechap2>
%\fi
%
% %%%%%%%%%%%%%%%%%%%%%%%%%%%%%%%%%%%%%%
% \paragraph{Part Include Files.}
%
% The include files are called |cdocspt3.tex| and |cdocspt4.tex|.
%
%\iffalse
%<*samplepart3|samplepart4>
%\fi

% Optional override for |\version| flag:
%    \begin{macrocode}
%%\providecommand{\version}{final}
%    \end{macrocode}

% Include the main document:
%    \begin{macrocode}
\input{childdoc.def}
\childdocby{cdocsamp}
%    \end{macrocode}

%\iffalse
%</samplepart3|samplepart4>
%\fi
%
%\iffalse
%<*samplepart3>
%\fi
% Some text for part 3:
%    \begin{macrocode}
some text in part three
%    \end{macrocode}

%\iffalse
%</samplepart3>
%\fi
% Some text for part 4:
%\iffalse
%<*samplepart4>
%\fi
%    \begin{macrocode}
more text in part four
%    \end{macrocode}

%\iffalse
%</samplepart4>
%\fi
%
% %%%%%%%%%%%%%%%%%%%%%%%%%%%%%%%%%%%%%%
% \paragraph{Forwarding for a Complete Draft.}
%
% The following forwarding file |cdocsdrf.tex|
% compiles the main document in draft mode:
%\iffalse
%<*sampledraft>
%\fi
%    \begin{macrocode}
\def\version{draft}
\input{childdoc.def}
\childdocforward{cdocsamp}
%    \end{macrocode}

%\iffalse
%</sampledraft>
%\fi
%
% %%%%%%%%%%%%%%%%%%%%%%%%%%%%%%%%%%%%%%
% \paragraph{Forwarding for Final Version of the Chapters.}
%
% The following forwarding files |cdocsfn1.tex| and |cdocsfn2.tex|
% (with identical content)
% compile the final versions of the child documents
% |cdocsch1.tex| and |cdocsch2.tex|, respectively:
%\iffalse
%<*samplefinal>
%\fi
%    \begin{macrocode}
\def\version{final}
\input{childdoc.def}
\childdocforwardprefix[cdocsamp]{cdocsfn}{cdocsch}
%    \end{macrocode}

%\iffalse
%</samplefinal>
%\fi
%
% %%%%%%%%%%%%%%%%%%%%%%%%%%%%%%%%%%%%%%
% \paragraph{Command Line Processing.}
%
% The following three command lines generate the output files
% |cdocscld|, |cdocscl1| and |cdocscl2|
% which should be identical to
% |cdocsdrf|, |cdocsch1| and |cdocsfn2|, respectively:
% \begin{center}
% \begin{tabular}{l}
% |latex -jobname cdocscld \|\\
% |  "\def\version{draft}\input{childdoc.def}\childdocforward{cdocsamp}"|\\
% |latex -jobname cdocscl1 \|\\
% |  "\input{childdoc.def}\childdocforward[cdocsamp]{cdocsch1}"|\\
% |latex -jobname cdocscl2 \|\\
% |  "\def\version{final}\input{childdoc.def}\childdocforward{cdocsch2}"|
% \end{tabular}
% \end{center}
% Note that the trailing backslash on each first line
% merely continues the input to the second line
% (for convenient cut ant paste).
% Furthermore, the command |latex| can be replaced by any
% of its alternative versions such as |pdflatex|.
%
% %%%%%%%%%%%%%%%%%%%%%%%%%%%%%%%%%%%%%%%%%%%%%%%%%%%%%%%%%%%%%%%%%%%%%%%%%%%%%%
% %%%%%%%%%%%%%%%%%%%%%%%%%%%%%%%%%%%%%%%%%%%%%%%%%%%%%%%%%%%%%%%%%%%%%%%%%%%%%%
% \section{Implementation}
%\iffalse
%<*package>
%\fi
%
% This section describes the definitions file |childdoc.def|.

% The definitions cannot be loaded using |\usepackage| or |\RequirePackage|
% which has a mechanism to prevent loading a style file more than once.
% When loading the definitions by means of |\input|
% multiple instances have to be prevented manually:
%\iffalse
%This code needs to be before the `\ProvidesFile' directive
%which is defined at the beginning of this file.
%Therefore it is also placed there and commented out here.
%</package>
%<*discard>
%\fi
%    \begin{macrocode}
\ifdefined\childdocmain\endinput\fi
%    \end{macrocode}
%\iffalse
%</discard>
%<*package>
%\fi
%
% \macro{\ifchilddoc}
% \macro{\ifchilddocmanual}
% The conditional |\ifchilddoc| tells whether a
% child (true) or main (false) document is being compiled.
% The conditional |\ifchilddocmanual| tells whether
% the |\includeonly| mechanism is used (false) or
% the selection of child files must be performed manually (true).
% The definitions initialise to false:
%    \begin{macrocode}
\newif\ifchilddoc
\newif\ifchilddocmanual
%    \end{macrocode}

% \macro{\childdocname}
% \macro{\childdocjob}
% The macro |\childdocname| stores the name of the main document
% to be compiled. The macro |\childdocjob| stores the name of
% the document on which the \LaTeX{} compiler was originally invoked.
% The content of |\jobname| cannot be compared
% to filenames specified in the source due to different catcodes.
% The following code rescans |\jobname|, stores the result
% in |\childdocname| and saves a copy in |\childdocjob|:
%    \begin{macrocode}
\edef\childdocname{\scantokens\expandafter{\jobname\noexpand}}
\let\childdocjob\childdocname
%    \end{macrocode}

% \macro{\childdocdisable}
% The macro |\childdocdisable| prevents the main file
% from being processed more than once.
% At this stage, the main document command |\childdocmain|
% is assumed to be called once again where it should do nothing.
% Any subsequent call to it should prevent
% a secondary processing of the main document
% It overwrites the forwarding commands
% |\childdocof| and |\childdocforward|
% with empty macros to prevent further inclusions of the main document:
%    \begin{macrocode}
\newcommand{\childdocdisable}
{
  \renewcommand{\childdocmain}[1]{\renewcommand{\childdocmain}[1]{\endinput}}
  \renewcommand{\childdocof}[1]{}
  \renewcommand{\childdocby}[2][]{}
  \renewcommand{\childdocforward}[2][]{}
  \renewcommand{\childdocdisable}{}
}
%    \end{macrocode}

% \macro{\childdocmain}
% The macro |\childdocmain| is to be called at the top of the main file
% with nothing or the main filename (without extension) as argument.
% First, it breaks loops.
% If the argument is not empty and does not match |\childdocname|
% (which is set by the first inclusion of |childdoc.def|),
% |\ifchilddoc| is set to true, |\includeonly| is applied to the child file
% and |\jobname| is set to the main file
% (for proper handling of |.aux| files):
%    \begin{macrocode}
\newcommand{\childdocmain}[1]
{
  \childdocdisable\childdocmain{}
  \if?#1?\else
    \begingroup
      \def\childdoctmp{#1}
      \ifx\childdoctmp\childdocname
        \def\childdoctmp{}
      \else
        \def\childdoctmp
        {
          \childdoctrue
          \includeonly{\childdocname}
          \def\childdocjob{#1}
          \def\jobname{#1}
        }
      \fi
      \expandafter
    \endgroup
    \childdoctmp
  \fi
}
%    \end{macrocode}

% \macro{\childdocof}
% The command |\childdocof| redirects
% compilation to the main file |#1|.
%    \begin{macrocode}
\newcommand{\childdocof}[1]
{
  \childdocdisable
  \childdoctrue
  \includeonly{\childdocname}
  \def\jobname{#1}
  \def\childdocjob{#1}
  \input{#1}
}
%    \end{macrocode}

% \macro{\childdocby}
% The command |\childdocby| ....
%    \begin{macrocode}
\newcommand{\childdocby}[2][]
{
  \childdocdisable
  \childdoctrue
  \childdocmanualtrue
  \if?#1?\else
    \def\jobname{#2}
  \fi
  \def\childdocjob{#2}
  \input{#2}
  \endinput
}
%    \end{macrocode}

% \macro{\childdocforward}
% The command |\childdocforward| redirects
% compilation to the main file or
% (if the optional argument is given) a child file.
% Parameters are set as if the main file
% or a child file starting with |\childdocof| was compiled.
% Then compilation is handed over to the main file:
%    \begin{macrocode}
\newcommand{\childdocforward}[2][]
{
  \begingroup
    \if?#1?
      \def\childdoctmp
      {
        \def\childdocname{#2}
        \def\childdocjob{#2}
        \def\jobname{#2}
        \input{#2}
        \endinput
      }
    \else
      \def\childdoctmp
      {
        \childdocdisable
        \def\childdocname{#2}
        \childdoctrue
        \includeonly{#2}
        \def\childdocjob{#1}
        \def\jobname{#1}
        \input{#1}
        \endinput
      }
    \fi
    \expandafter
  \endgroup
  \childdoctmp
}
%    \end{macrocode}

% \macro{\childdocforwardprefix}
% The command |\childdocforwardprefix| redirects
% compilation to the main or a child file by means of a pattern.
% The prefix |#1| in the current filename is replaced by |#2|
% and the suffix of the current filename is kept
% (it is assumed that the filename does not contain the substring `|~~~|'
% which is used as a delimiter).
% Compilation is handed over to the new file by |\childdocforward|:
%    \begin{macrocode}
\newcommand{\childdocforwardprefix}[3][]
{
  \begingroup
    \def\childdocextract #2##1~~~{\def\childdoctmp{\childdocforward[#1]{#3##1}}}
    \expandafter\childdocextract\childdocname~~~
    \expandafter
  \endgroup
  \childdoctmp
}
%    \end{macrocode}

% \macro{\childdoc}
% The deprecated macro |\childdoc| is a legacy version of |\childdocmain|:
%    \begin{macrocode}
\newcommand{\childdoc}{\childdocmain}
%    \end{macrocode}

% \macro{\childdocredirect}
% The deprecated macro |\childdocredirect| is a legacy version
% of |\childdocforward| and |\childdocforwardprefix|:
%    \begin{macrocode}
\newcommand{\childdocredirect}[2][]
{
  \begingroup
    \if?#1?
      \def\childdoctmp{\childdocforward{#2}}
    \else
      \def\childdoctmp{\childdocforwardprefix{#1}{#2}}
    \fi
    \expandafter
  \endgroup
  \childdoctmp
}
%    \end{macrocode}

%\iffalse
%</package>
%\fi
%
\endinput
|\\
|\childdocby{|\textit{main}|}|\\
\end{tabular}
\end{center}
%
The directive |\childdocby| is similar to |\childdocof|
described in \secref{sec:include},
but the subsequent selection of content must be done manually.
To that end, both |\ifchilddoc| and |\ifchilddocmanual|
will be true upon processing of a part,
and the name of the part is stored in |\childdocname|.
Note that |\jobname| will be set to the filename of the current part
so that each part receives an individual |.aux| file
that does not interfere with the |.aux| file(s) of the main document.
This behaviour can be altered by the alternative form
|\childdocby[*]{|\textit{main}|}| (with a non-empty optional argument)
which uses the |.aux| file of the main document
by setting |\jobname| to \textit{main}.

%%%%%%%%%%%%%%%%%%%%%%%%%%%%%%%%%%%%%%%%%%%%%%%%%%%%%%%%%%%%%%%%%%%%%%%%%%%%%%%%
\subsection{Driver Development}
\label{sec:driver}

The \textsf{childdoc} mechanism can also be use for the development
of definition files such as \LaTeX{} styles or classes.
This case differs from the above setup with multiple parts
included by |\include| in that no |\includeonly| should be invoked.
This can be achieved by starting the include file
(before |\ProvidesPackage|) with:
%
\begin{center}
\begin{tabular}{l}
|% \iffalse
%
% childdoc.dtx Copyright (C) 2017-2018 Niklas Beisert
%
% This work may be distributed and/or modified under the
% conditions of the LaTeX Project Public License, either version 1.3
% of this license or (at your option) any later version.
% The latest version of this license is in
%   http://www.latex-project.org/lppl.txt
% and version 1.3 or later is part of all distributions of LaTeX
% version 2005/12/01 or later.
%
% This work has the LPPL maintenance status `maintained'.
%
% The Current Maintainer of this work is Niklas Beisert.
%
% This work consists of the files childdoc.dtx and childdoc.ins
% and the derived files childdoc.def and cdocsamp.tex with
% cdocsch1.tex, cdocsch2.tex, cdocsdrf.tex, cdocsfn1.tex, cdocsfn2.tex.
%
%<package>\ifdefined\childdocmain\endinput\fi
%<package>\ProvidesFile{childdoc.def}[2018/12/30 v2.0 child document driver]
%<samplemain>\ProvidesFile{cdocsamp.tex}[2018/12/30 v2.0 sample for childdoc]
%<*driver>
%\ProvidesFile{childdoc.drv}[2018/12/30 v2.0 childdoc reference manual file]
\PassOptionsToClass{10pt,a4paper}{article}
\documentclass{ltxdoc}

\usepackage[margin=35mm]{geometry}
\usepackage{hyperref}
\usepackage{hyperxmp}
\usepackage[usenames]{color}

\hypersetup{colorlinks=true}
\hypersetup{pdfstartview=FitH}
\hypersetup{pdfpagemode=UseNone}
\hypersetup{pdfsource={}}
\hypersetup{pdflang={en-UK}}
\hypersetup{pdfcopyright={Copyright 2017-2018 Niklas Beisert.
  This work may be distributed and/or modified under the
  conditions of the LaTeX Project Public License, either version 1.3
  of this license or (at your option) any later version.}}
\hypersetup{pdflicenseurl={http://www.latex-project.org/lppl.txt}}
\hypersetup{pdfcontactaddress={ETH Zurich, ITP, HIT K,
  Wolfgang-Pauli-Strasse 27}}
\hypersetup{pdfcontactpostcode={8093}}
\hypersetup{pdfcontactcity={Zurich}}
\hypersetup{pdfcontactcountry={Switzerland}}
\hypersetup{pdfcontactemail={nbeisert@itp.phys.ethz.ch}}
\hypersetup{pdfcontacturl={http://people.phys.ethz.ch/\xmptilde nbeisert/}}

\newcommand{\secref}[1]{\hyperref[#1]{section \ref*{#1}}}

\parskip1ex
\parindent0pt
\let\olditemize\itemize
\def\itemize{\olditemize\parskip0pt}

\begin{document}

\title{The \textsf{childdoc} Package}
\hypersetup{pdftitle={The childdoc Package}}
\author{Niklas Beisert\\[2ex]
  Institut f\"ur Theoretische Physik\\
  Eidgen\"ossische Technische Hochschule Z\"urich\\
  Wolfgang-Pauli-Strasse 27, 8093 Z\"urich, Switzerland\\[1ex]
  \href{mailto:nbeisert@itp.phys.ethz.ch}
  {\texttt{nbeisert@itp.phys.ethz.ch}}}
\hypersetup{pdfauthor={Niklas Beisert}}
\hypersetup{pdfsubject={Manual for the LaTeX2e Package childdoc}}
\date{30 December 2018, \textsf{v2.0}}
\maketitle

\begin{abstract}\noindent
\textsf{childdoc} is a \LaTeXe{} package
that enables the direct compilation
of document sections included by |\include|
to individual files.
\end{abstract}

\begingroup
\parskip0ex
\tableofcontents
\endgroup

%%%%%%%%%%%%%%%%%%%%%%%%%%%%%%%%%%%%%%%%%%%%%%%%%%%%%%%%%%%%%%%%%%%%%%%%%%%%%%%%
%%%%%%%%%%%%%%%%%%%%%%%%%%%%%%%%%%%%%%%%%%%%%%%%%%%%%%%%%%%%%%%%%%%%%%%%%%%%%%%%
\section{Introduction}

\LaTeX{} provides a mechanism to structure a large document (such as a book)
into a main file and several child files (containing the chapters)
using the |\include| command.
This mechanism is beneficial for documents
which span hundreds of pages in order to
make the source file(s) more manageable.
Moreover, compilation can be restricted to
selected child files by means of the |\includeonly| command.
The latter feature can be used to reduce the compilation time while editing
(this was significantly more useful in the earlier days of \LaTeX{})
or to generate a smaller document which is easier to navigate.
Another application of |\includeonly| is to generate
documents consisting of selected parts of the complete document.

However, there are a few drawbacks of the plain |\include| mechanism:
\begin{itemize}
\item
The child files cannot be compiled on their own,
they can only be compiled via the main file.
A naive editing environment
(such as a text editor with an option
to have the current file processed by \LaTeX)
may require one to switch to the main file before compiling;
attempting to compile the child file produces errors.
\item
The main file must be modified (each time)
to adjust the |\includeonly| command
to the present needs. This easily leaves the main file in a messy state.
\item
The generated document will always carry the filename
of the main document. This is inconvenient if
several child files are to be compiled and
to be kept for distribution.
\end{itemize}

The present package provides a simple interface
to make child files individually compilable by \LaTeX{}.
Compiling a child file then has the same effect as compiling
the main file with an |\includeonly| command
to select the appropriate child.
Moreover the generated document will carry the name of the child
rather than the main file.
This resolves all three above issues.

This feature is meant to make the editing of books,
thesis documents and lecture notes somewhat more convenient.
However, the package can also be used efficiently for
composing a series of documents (such as exercise sheets)
which are typically distributed individually.
It then assists the author in generating the individual documents
(potentially in different versions)
as well as a document containing the collected series.
Another application is in developing style files
or other kinds of included material
where compilation of the style file could redirect
to a sample or test file.

%%%%%%%%%%%%%%%%%%%%%%%%%%%%%%%%%%%%%%%%%%%%%%%%%%%%%%%%%%%%%%%%%%%%%%%%%%%%%%%%
%%%%%%%%%%%%%%%%%%%%%%%%%%%%%%%%%%%%%%%%%%%%%%%%%%%%%%%%%%%%%%%%%%%%%%%%%%%%%%%%
\section{Usage}

First of all, the package \textsf{childdoc} is \emph{not} a standard
\LaTeXe{} |.sty| style file! Therefore it needs to be invoked in
a non-standard way.

%%%%%%%%%%%%%%%%%%%%%%%%%%%%%%%%%%%%%%%%%%%%%%%%%%%%%%%%%%%%%%%%%%%%%%%%%%%%%%%%
\subsection{Included Files}
\label{sec:include}

%%%%%%%%%%%%%%%%%%%%%%%%%%%%%%%%%%%%%%%%
\DescribeMacro{\childdocmain}
To use the package, add the commands
\begin{center}
\begin{tabular}{l}
|\input{childdoc.def}|\\
|\childdocmain{}|\\
\end{tabular}
\end{center}
at the very top of the main \LaTeX{} file,
in particular \emph{before} the |\documentclass| statement!
The argument of |\childdocmain| should be left empty
(but it must be present).

%%%%%%%%%%%%%%%%%%%%%%%%%%%%%%%%%%%%%%%%
\DescribeMacro{\childdocof}
Furthermore, add the commands
\begin{center}
\begin{tabular}{l}
|\input{childdoc.def}|\\
|\childdocof{|\textit{main}|}|\\
\end{tabular}
\end{center}
at the top of every child file \textit{child}
which is included by |\include{|\textit{child}|}|
from within the main file
(or at least for those files to be compiled individually).
The argument \textit{main} must be the filename of the main file.

There are a couple of
considerations in setting up the main and child documents:

%%%%%%%%%%%%%%%%%%%%%%%%%%%%%%%%%%%%%%%%
\paragraph{Restrictions.}

Please note the following restrictions:
\begin{itemize}
\item
|\childdocmain| must be called with one argument \textit{main}
to ensure compatibility with earlier version of the package.
It must either be empty (|\childdocmain{}|)
or precisely match the filename of the main file in which it is specified.
See \secref{sec:detection} for further information.
\item
The filename \textit{main} must be specified without the |.tex| extension.
\item
The filename \textit{main} is case sensitive
(even in case-insensitive file systems)
due to internal string comparison.
\item
The argument \textit{main} should be fully expanded, it cannot be a macro.
\item
Subdirectories and special characters should be avoided in filenames.
\item
The command |\childdocmain{|\textit{main}|}| must be followed by a whitespace.
It should not be followed immediately by another command
or by a comment mark `|%|'.
This is because the \TeX{} parser reads the token immediately following
the argument of |\childdocmain| and puts it
at the beginning of every child section;
however, a white\-space is ignored.
\end{itemize}

%%%%%%%%%%%%%%%%%%%%%%%%%%%%%%%%%%%%%%%%
\paragraph{Content of Main File.}

It is advisable to place all content in the child files included by |\include|.
Any output contained in the main file will appear in all child documents
unless suppressed manually;
it cannot be suppressed automatically by the |\includeonly| directive
and thus should normally be avoided.
A method to include some content in the main file
by means of conditional processing is described in \secref{sec:conditional}.

%%%%%%%%%%%%%%%%%%%%%%%%%%%%%%%%%%%%%%%%
\paragraph{Page Numbering.}

When only a part of the document is compiled,
the appropriate numbering of pages
(as well as other status parameters)
is determined from the |.aux| files.
The latter contain information from previous passes.
However this information needs to propagate through
all intermediate child documents.
Therefore the page numbering in child documents may well
be inconsistent until the complete document is compiled at least once.

A useful (if unconventional) way to always ensure a consistent
page numbering is to restart the numbering in each child document
and denote the pages by `\textit{child}|.|\textit{page}'
where \textit{child} represents the chapter/section number of the child file.
This can be achieved by the command
|\numberwithin{page}{|\textit{child}|}|
of the \textsf{amsmath} package
where \textit{child} can be |chapter| or |section|
depending on the chosen structuring.
Alternatively, one can modify the macro |\thepage| appropriately
and reset the counter |page| at the start of each child file.

%%%%%%%%%%%%%%%%%%%%%%%%%%%%%%%%%%%%%%%%%%%%%%%%%%%%%%%%%%%%%%%%%%%%%%%%%%%%%%%%
\subsection{Conditional Processing}
\label{sec:conditional}

The package provides a mechanism to compile different versions
of a document. To customise the versions further some conditional processing
can come in handy to distinguish which version is being compiled.
The package provides two macros to describe the compilation context:

%%%%%%%%%%%%%%%%%%%%%%%%%%%%%%%%%%%%%%%%
\DescribeMacro{\ifchilddoc}
The conditional |\ifchilddoc| distinguishes between the compilation of
child documents and the main document:
%
\begin{center}
|\ifchilddoc |\textit{child-code}| |[|\||else |\textit{main-code}]| \||fi|
\end{center}

%%%%%%%%%%%%%%%%%%%%%%%%%%%%%%%%%%%%%%%%
\DescribeMacro{\childdocname}
\DescribeMacro{\childdocjob}
The macro |\childdocname| contains the filename (without extension)
of the main or child file being processed.
Note that |\childdocjob| will always contain the name of the main file.

%%%%%%%%%%%%%%%%%%%%%%%%%%%%%%%%%%%%%%%%
\paragraph{Title Page.}

Conditional processing can be used to include a title or banner page
in the main document when proper precautions are taken.
Importantly, the code in the main file should ensure that the page counter
(as well as other status parameters which are stored in the |.aux| files)
takes the same value after the conditional processing.
Otherwise the page numbers may take divergent values
depending on which part is compiled.

For example, a title page could be declared by:
%
\begin{center}
\begin{tabular}{l}
|\ifchilddoc\||else|\\
|\addtocounter{page}{-1}|\\
\textit{code for title page}\\
|\newpage|\\
|\||fi|
\end{tabular}
\end{center}
%
A banner page for the child documents can be generated by:
%
\begin{center}
\begin{tabular}{l}
|\ifchilddoc|\\
|\addtocounter{page}{-1}|\\
\textit{code for banner page}\\
|\newpage|\\
|\||fi|
\end{tabular}
\end{center}
%
Here one could write a message such as:
\begin{center}
|This is the part \childdocname{} of \childdocjob{}.|
\end{center}

%%%%%%%%%%%%%%%%%%%%%%%%%%%%%%%%%%%%%%%%%%%%%%%%%%%%%%%%%%%%%%%%%%%%%%%%%%%%%%%%
\subsection{Flags}
\label{sec:flags}

The package makes it easy to generate different versions
of the main or child documents.
To this end compilation flags can be defined
and assigned different default values.
They will be particularly useful in conjunction
with the forwarding mechanism described in \secref{sec:forward}.

For example, it may be useful to have a flag |\version|
which can be set to |draft| or |final|.
The document source will contain some conditional code
depending on the value of |\version|.
Suppose further, the flag should default to |final| for the main file
and to |draft| for child files
which is a natural assignment for editing the document.
This is achieved by placing the following code
in the preamble of the main document
(below the |\childdocmain| directive):
%
\begin{center}
\begin{tabular}{l}
|\ifchilddoc|\\
|\providecommand{\version}{draft}|\\
|\||else|\\
|\providecommand{\version}{final}|\\
|\||fi|
\end{tabular}
\end{center}
%
The definition by |\providecommand| makes sure
that previous definitions are not overwritten.
Further statements |\providecommand{\version}{...}|
can thus be added before the above code to override it.

For the main file, one might add a line
(between |\childdocmain| and the above block)
%
\begin{center}
|%\ifchilddoc\||else\providecommand{\version}{draft}\||fi|
\end{center}
%
which can be uncommented to produce a draft version.
Likewise one can add a line to the very top of a child file
(above the |\childdocof{|\textit{main}|}| directive)
%
\begin{center}
|%\providecommand{\version}{final}|
\end{center}
%
which can be uncommented to produce the final version of this child document.

%%%%%%%%%%%%%%%%%%%%%%%%%%%%%%%%%%%%%%%%%%%%%%%%%%%%%%%%%%%%%%%%%%%%%%%%%%%%%%%%
\subsection{Forwarding}
\label{sec:forward}

Different versions of the main or child documents
using compilation flags as described in \secref{sec:flags}
can be (permanently) stored in different files
for convenient compilation, viewing and distribution.
To this end, the package defines a command
to pass on compilation to a different file:

%%%%%%%%%%%%%%%%%%%%%%%%%%%%%%%%%%%%%%%%
\DescribeMacro{\childdocforward}
The command |\childdocforward| redirects processing to
another source file:
%
\begin{center}
\begin{tabular}{l}
|\input{childdoc.def}|\\
|\childdocforward[|\textit{main}|]{|\textit{dest}|}|\\
\end{tabular}
\end{center}
%
The argument \textit{dest} is the destination file
(without extension).
It should be the main file or one of the child files.
Note that further \textsf{childdoc} directives
such as |\childdocof| and |\childdocforward|
in the indicated file will be processed in this form.
The optional argument \textit{main}
passes on directly to the main file \textit{main}
while pretending to compile the child \textit{dest}.
This form behaves as if \textit{dest}
issues |\childdocof{|\textit{main}|}| right away,
and no further \textsf{childdoc} directives will be processed.

%%%%%%%%%%%%%%%%%%%%%%%%%%%%%%%%%%%%%%%%
\DescribeMacro{\...prefix}
In the alternative form |\childdocforwardprefix|,
%
\begin{center}
\begin{tabular}{l}
|\input{childdoc.def}|\\
|\childdocforwardprefix[|\textit{main}|]{|\textit{prefix}|}{|\textit{dest}|}|
\end{tabular}
\end{center}
%
the destination file is determined by a pattern
depending on the current file:
To make this work, the current file must be called
`{\textit{prefix}\hspace{0.2em}\textit{suffix}}'
with \textit{prefix} matching precisely the argument.
Processing is then passed on to the file
`{\textit{dest}\hspace{0.2em}\textit{suffix}}'.
Surely, the same effect is achieved by
directly specifying the
argument `{\textit{dest}\hspace{0.2em}\textit{suffix}}'
in the first form.
However, that requires to set up a different file
for each child. With the alternative form of the command
all these files can have exactly the same content
which simplifies setting them up and maintaining them.

For example, the following file |draft.tex|
with a compilation flag |\version| as described in \secref{sec:flags}
compiles the main document as a draft:
%
\begin{center}
\begin{tabular}{l}
|\def\version{draft}|\\
|\input{childdoc.def}|\\
|\childdocforward{|\textit{main}|}|
\end{tabular}
\end{center}
%
Likewise, the following files |final|\textit{nn}|.tex|
compile the final version of the child document
|child|\textit{nn}|.tex|:
%
\begin{center}
\begin{tabular}{l}
|\def\version{final}|\\
|\input{childdoc.def}|\\
|\childdocforwardprefix{final}{child}|
\end{tabular}
\end{center}
%

Note that when several versions of a main file and/or of each child file
are to be generated, it may be convenient to set up a |Makefile| or
shell script to automatise the process.

%%%%%%%%%%%%%%%%%%%%%%%%%%%%%%%%%%%%%%%%%%%%%%%%%%%%%%%%%%%%%%%%%%%%%%%%%%%%%%%%
\subsection{Command Line Processing}
\label{sec:commandline}

The effect of redirection files can also be achieved by invoking
the \LaTeX{} compiler with a more elaborate command line.
Most conveniently this should be done as part
of a shell script or a |Makefile|.

When using \textsf{childdoc} in the main file, the following
command lines effectively perform a redirection
(note that depending on the shell being used,
backslashes may have to be doubled: `|\|' $\to$ `|\\|'):
%
\begin{center}
|... -jobname "|\textit{target}|" |\\|"|[\textit{flags}]%
|\input{childdoc.def}\childdocforward[|\textit{main}|]{|\textit{dest}|}"|
\end{center}
%
Here \textit{target} is the name of the output file,
\textit{main} is the name of the main file
and \textit{dest} is the name of the main or child file to be processed
(all filenames without extensions).
The optional argument \textit{main} can be omitted
if \textit{main} matches \textit{dest}.
Optionally, compilation \textit{flags} can be defined via |\def| commands.
This command line makes the \TeX{} engine believe
it is compiling the file \textit{target}
whose content is specified as the latter parameter.
The provided code then forwards the processing to
\textit{main} or \textit{dest} as described in \secref{sec:forward}.

%%%%%%%%%%%%%%%%%%%%%%%%%%%%%%%%%%%%%%%%%%%%%%%%%%%%%%%%%%%%%%%%%%%%%%%%%%%%%%%%
\subsection{Include by Input}
\label{sec:input}

Including child documents by |\include| has some restrictions by design.
Most notably, the content of a child document always occupies
its own set of pages; pages cannot be shared between child documents.
Usually, this behaviour makes perfect sense
because each child document contain an essential part of the document.
However, in some situations it may be desirable to compose
a document from a collection of parts
without having mandatory page breaks between then.
For this case, the package
provides a mechanism to include parts
by |\input| which can also be processed individually.
However, by construction this mechanism
requires manual handling of the content to be output.

%%%%%%%%%%%%%%%%%%%%%%%%%%%%%%%%%%%%%%%%
\DescribeMacro{\ifchilddocmanual}
The main file should be prepared as usual, see \secref{sec:include}.
However, the document body must make a distinction
between processing of an individual part and of the main document, e.g.:
%
\begin{center}
\begin{tabular}{l}
|\ifchilddocmanual|\\
|\input{\childdocname}|\\
|\||else|\\
\textit{document body with }|\input{|\textit{part}|}|\\
|\||fi|
\end{tabular}
\end{center}
%
The conditional |\ifchilddocmanual| is true whenever
a part to be included by |\input| is being compiled,
and the name of the part is stored in |\childdocname|.

%%%%%%%%%%%%%%%%%%%%%%%%%%%%%%%%%%%%%%%%
\DescribeMacro{\childdocby}
Each part to be included by |\input| should start with:
%
\begin{center}
\begin{tabular}{l}
|\input{childdoc.def}|\\
|\childdocby{|\textit{main}|}|\\
\end{tabular}
\end{center}
%
The directive |\childdocby| is similar to |\childdocof|
described in \secref{sec:include},
but the subsequent selection of content must be done manually.
To that end, both |\ifchilddoc| and |\ifchilddocmanual|
will be true upon processing of a part,
and the name of the part is stored in |\childdocname|.
Note that |\jobname| will be set to the filename of the current part
so that each part receives an individual |.aux| file
that does not interfere with the |.aux| file(s) of the main document.
This behaviour can be altered by the alternative form
|\childdocby[*]{|\textit{main}|}| (with a non-empty optional argument)
which uses the |.aux| file of the main document
by setting |\jobname| to \textit{main}.

%%%%%%%%%%%%%%%%%%%%%%%%%%%%%%%%%%%%%%%%%%%%%%%%%%%%%%%%%%%%%%%%%%%%%%%%%%%%%%%%
\subsection{Driver Development}
\label{sec:driver}

The \textsf{childdoc} mechanism can also be use for the development
of definition files such as \LaTeX{} styles or classes.
This case differs from the above setup with multiple parts
included by |\include| in that no |\includeonly| should be invoked.
This can be achieved by starting the include file
(before |\ProvidesPackage|) with:
%
\begin{center}
\begin{tabular}{l}
|\input{childdoc.def}|\\
|\childdocforward{|\textit{main}|}|\\
\end{tabular}
\end{center}
%
or alternatively with:
%
\begin{center}
\begin{tabular}{l}
|\input{childdoc.def}|\\
|\childdocby{|\textit{main}|}|\\
\end{tabular}
\end{center}
%
Both forms have slightly different effects as described above.
The main file is prepared as usual, see \secref{sec:include}.

%%%%%%%%%%%%%%%%%%%%%%%%%%%%%%%%%%%%%%%%%%%%%%%%%%%%%%%%%%%%%%%%%%%%%%%%%%%%%%%%
\subsection{Legacy Detection}
\label{sec:detection}

The directive |\childdocmain| in the main file can detect
whether the complete document or merely a child is to be compiled
even without using the directive |\childdocof|.
This method is deprecated because it is less robust
and there is no compelling reason to use it;
it is merely provided for backward compatibility
and it may be removed in future versions.

If the detection mechanism is to be used,
it is mandatory to correctly specify
the filename of the main file as the argument of |\childdocmain|:
%
\begin{center}
\begin{tabular}{l}
|\input{childdoc.def}|\\
|\childdocmain{|\textit{main}|}|\\
\end{tabular}
\end{center}
%
If |\jobname| does not match the argument \textit{main} of |\childdocmain|,
it is assumed that |\jobname| points to the child file to be compiled.
When using |\childdocmain| with the main file specified as argument,
it suffices to start a child file
with just |\input{|\textit{main}|}|
without loading of the package and using |\childdocof|.
If instead all processing is done
with the appropriate \textsf{childdoc} directives,
the argument of \textit{main} of |\childdocmain| can be empty.

An alternative version of the command line processing described
in \secref{sec:commandline} using the detection mechanism reads:
%
\begin{center}
|... -jobname "|\textit{target}|" "|[\textit{flags}]%
[|\def\jobname{|\textit{dest}|}|]|\input{|\textit{main}|}"|
\end{center}

%%%%%%%%%%%%%%%%%%%%%%%%%%%%%%%%%%%%%%%%%%%%%%%%%%%%%%%%%%%%%%%%%%%%%%%%%%%%%%%%
\subsection{Manual Code}
\label{sec:manual}

In case one cannot be certain whether the definitions file |childdoc.def|
is installed on the target \TeX{} distribution
and one prefers not to ship it,
it is conceivable to paste a few relevant commands into the sources.

To that end, drop all statements |\input{childdoc.def}|
and perform the replacements as outlined below.
Instead of |\childdocmain{|\textit{main}|}| add the following code
to the top of the main file:
%
\begin{center}
\begin{tabular}{l}
|\||ifdefined\childdocname\endinput\||fi\newif\ifchilddoc|\\
|\edef\childdocname{\scantokens\expandafter{\jobname\noexpand}}|\\
|\def\childdocmain{|\textit{main}|}\||ifx\childdocmain\childdocname\||else|\\
|\childdoctrue\includeonly{\childdocname}\let\jobname\childdocmain\||fi|\\
\end{tabular}
\end{center}
%
Instead of |\childdocof{|\textit{main}|}| just include the main file
at the top of each child file:
%
\begin{center}
|\input{|\textit{main}|}|
\end{center}
%
A simple redirection |\childdocforward{|\textit{dest}|}| is achieved by:
%
\begin{center}
|\def\jobname{|\textit{dest}|}\input{\jobname}|
\end{center}
%
The redirection with prefix
|\childdocforwardprefix[|\textit{prefix}|]{|\textit{dest}|}|
is accomplished by:
%
\begin{center}
\begin{tabular}{l}
|{\edef\jobname{\scantokens\expandafter{\jobname\noexpand}}|\\
|\def\redirectjob |\textit{prefix}|#1~~~{\gdef\jobname{|\textit{dest}|#1}}|\\
|\expandafter\redirectjob\jobname~~~}\input{\jobname}|
\end{tabular}
\end{center}

In an alternative approach,
child documents can be compiled by a specific command line
without additional code or specific definitions:
%
\begin{center}
|... -jobname "|\textit{target}|" "|[\textit{flags}]%
|\includeonly{|\textit{dest}|}\input{|\textit{main}|}"|
\end{center}
%

%%%%%%%%%%%%%%%%%%%%%%%%%%%%%%%%%%%%%%%%%%%%%%%%%%%%%%%%%%%%%%%%%%%%%%%%%%%%%%%%
%%%%%%%%%%%%%%%%%%%%%%%%%%%%%%%%%%%%%%%%%%%%%%%%%%%%%%%%%%%%%%%%%%%%%%%%%%%%%%%%
\section{Information}

%%%%%%%%%%%%%%%%%%%%%%%%%%%%%%%%%%%%%%%%%%%%%%%%%%%%%%%%%%%%%%%%%%%%%%%%%%%%%%%%
\subsection{Copyright}

Copyright \copyright{} 2017--2018 Niklas Beisert

This work may be distributed and/or modified under the
conditions of the \LaTeX{} Project Public License, either version 1.3
of this license or (at your option) any later version.
The latest version of this license is in
  \url{http://www.latex-project.org/lppl.txt}
and version 1.3 or later is part of all distributions of \LaTeX{}
version 2005/12/01 or later.

This work has the LPPL maintenance status `maintained'.

The Current Maintainer of this work is Niklas Beisert.

This work consists of the files |README.txt|, |childdoc.ins| and |childdoc.dtx|
as well as the derived files |childdoc.def|, |cdocsamp.tex|
with |cdocsch1.tex|, |cdocsch2.tex|, |cdocspt3.tex|, |cdocspt4.tex|,
|cdocsdrf.tex|, |cdocsfn1.tex|, |cdocsfn2.tex|
as well as |childdoc.pdf|.

%%%%%%%%%%%%%%%%%%%%%%%%%%%%%%%%%%%%%%%%%%%%%%%%%%%%%%%%%%%%%%%%%%%%%%%%%%%%%%%%
\subsection{Files and Installation}

The package consists of the files:
%
\begin{center}
\begin{tabular}{ll}
    |README.txt|   & readme file \\
    |childdoc.ins| & installation file \\
    |childdoc.dtx| & source file \\
    |childdoc.def| & definition file \\
    |cdocsamp.tex| & sample main file \\
    |cdocsch1.tex| & sample include file \\
    |cdocsch2.tex| & sample include file \\
    |cdocspt3.tex| & sample part file \\
    |cdocspt4.tex| & sample part file \\
    |cdocsdrf.tex| & sample redirection file \\
    |cdocsfn1.tex| & sample redirection file \\
    |cdocsfn2.tex| & sample redirection file \\
    |childdoc.pdf| & manual
\end{tabular}
\end{center}
%
The distribution consists of the files
|README.txt|, |childdoc.ins| and |childdoc.dtx|.
%
\begin{itemize}
\item
Run (pdf)\LaTeX{} on |childdoc.dtx|
to compile the manual |childdoc.pdf| (this file).
\item
Run \LaTeX{} on |childdoc.ins| to create the definitions file |childdoc.def|
and the sample |cdocsamp.tex| with include files
|cdocsch1.tex|, |cdocsch2.tex|, |cdocspt3.tex|, |cdocspt4.tex|,
|cdocsdrf.tex|, |cdocsfn1.tex|, |cdocsfn2.tex|.
Then copy the file |childdoc.def| to an appropriate directory of your \LaTeX{}
distribution, e.g.\ \textit{texmf-root}|/tex/latex/childdoc|.
\end{itemize}

%%%%%%%%%%%%%%%%%%%%%%%%%%%%%%%%%%%%%%%%%%%%%%%%%%%%%%%%%%%%%%%%%%%%%%%%%%%%%%%%
\subsection{Related CTAN Packages}

There are several other packages which offer a similar functionality:
%
\begin{itemize}
\item
The packages
\href{http://ctan.org/pkg/docmute}{\textsf{docmute}},
\href{http://ctan.org/pkg/includex}{\textsf{includex}} and
\href{http://ctan.org/pkg/standalone}{\textsf{standalone}}
provide commands to include only the document body of
a child file thus allowing both files to be compiled individually.
\item
The packages \href{http://ctan.org/pkg/subdocs}{\textsf{subdocs}}
and \href{http://ctan.org/pkg/subfiles}{\textsf{subfiles}}
provide structures in which the main and child documents can be
encapsulated and allowing them to be compiled individually.
The inclusion mechanism is different from the conventional |\include|.
\item
The package \href{http://ctan.org/pkg/combine}{\textsf{combine}}
is an elaborate solution to combine several documents into one.
\end{itemize}
%
See also the CTAN topic \href{http://ctan.org/topic/subdocs}{\textsf{subdocs}}
for further related packages.
The present package differs from the above solutions in that
a document structure constructed with the conventional |\include| mechanism
just needs two extra commands at the top of every file
such that all constituent files can be compiled individually.

%%%%%%%%%%%%%%%%%%%%%%%%%%%%%%%%%%%%%%%%%%%%%%%%%%%%%%%%%%%%%%%%%%%%%%%%%%%%%%%%
%\subsection{Feature Suggestions}
%
%The following is a list of features which may be useful for future
%versions of this package:
%%
%\begin{itemize}
%\item
%\ldots
%\end{itemize}

%%%%%%%%%%%%%%%%%%%%%%%%%%%%%%%%%%%%%%%%%%%%%%%%%%%%%%%%%%%%%%%%%%%%%%%%%%%%%%%%
\subsection{Revision History}

%%%%%%%%%%%%%%%%%%%%%%%%%%%%%%%%%%%%%%%%
\paragraph{v2.0:} 2018/12/30

\begin{itemize}
\item
immediate forward processing
\item
added |\childdocby| mechanism
\item
manual restructured
\end{itemize}

%%%%%%%%%%%%%%%%%%%%%%%%%%%%%%%%%%%%%%%%
\paragraph{v1.6:} 2018/01/17

\begin{itemize}
\item
application for development of include files
\item
corrections to manual
\end{itemize}

%%%%%%%%%%%%%%%%%%%%%%%%%%%%%%%%%%%%%%%%
\paragraph{v1.5:} 2017/05/21

\begin{itemize}
\item
more complete structuring introduced
\item
|\childdocof| introduced
\item
|\childdoc| renamed to |\childdocmain|
\item
|\childredirect| renamed to |\childdocforward| and |\childdocforwardprefix|
and functionality expanded
\end{itemize}

%%%%%%%%%%%%%%%%%%%%%%%%%%%%%%%%%%%%%%%%
\paragraph{v1.0:} 2017/04/27

\begin{itemize}
\item
manual and install package
\item
first version published on CTAN
\end{itemize}

%%%%%%%%%%%%%%%%%%%%%%%%%%%%%%%%%%%%%%%%
\paragraph{v0.6:} 2017/04/26

\begin{itemize}
\item
redirection mechanism added
\end{itemize}

%%%%%%%%%%%%%%%%%%%%%%%%%%%%%%%%%%%%%%%%
\paragraph{v0.5:} 2017/04/26

\begin{itemize}
\item
functionality in definition file
\end{itemize}


%%%%%%%%%%%%%%%%%%%%%%%%%%%%%%%%%%%%%%%%%%%%%%%%%%%%%%%%%%%%%%%%%%%%%%%%%%%%%%%%
%%%%%%%%%%%%%%%%%%%%%%%%%%%%%%%%%%%%%%%%%%%%%%%%%%%%%%%%%%%%%%%%%%%%%%%%%%%%%%%%
%%%%%%%%%%%%%%%%%%%%%%%%%%%%%%%%%%%%%%%%%%%%%%%%%%%%%%%%%%%%%%%%%%%%%%%%%%%%%%%%
\appendix

\settowidth\MacroIndent{\rmfamily\scriptsize 000\ }

 \DocInput{childdoc.dtx}

\end{document}
%</driver>
% \fi
%
% %%%%%%%%%%%%%%%%%%%%%%%%%%%%%%%%%%%%%%%%%%%%%%%%%%%%%%%%%%%%%%%%%%%%%%%%%%%%%%
% %%%%%%%%%%%%%%%%%%%%%%%%%%%%%%%%%%%%%%%%%%%%%%%%%%%%%%%%%%%%%%%%%%%%%%%%%%%%%%
% \section{Sample}
%\iffalse
%<*samplemain>
%\fi
%
% The following presents a sample document
% with two chapters, two parts, a title page,
% a compile flag as well as three forwarding files to set the flag.
% It consists of eight |.tex| files:
% \begin{center}
% \begin{tabular}{ll}
% |cdocsamp.tex|&main file\\
% |cdocsch1.tex|&include file for chapter 1\\
% |cdocsch2.tex|&include file for chapter 2\\
% |cdocspt3.tex|&include file for part 3\\
% |cdocspt4.tex|&include file for part 4\\
% |cdocsdrf.tex|&forwarding file for main file in draft mode\\
% |cdocsfi1.tex|&forwarding file for final version of chapter 1\\
% |cdocsfi2.tex|&forwarding file for final version of chapter 2\\
% \end{tabular}
% \end{center}
% Each of the eight files can be compiled directly by the \LaTeX{} compiler.
%
% %%%%%%%%%%%%%%%%%%%%%%%%%%%%%%%%%%%%%%
% \paragraph{Main File.}
%
% The main file is called |cdocsamp.tex|.
%
% Load the \textsf{childdoc} definitions and
% declare the filename for the main document:
%    \begin{macrocode}
\input{childdoc.def}
\childdocmain{}
%    \end{macrocode}

% Optional override for |\version| flag:
%    \begin{macrocode}
%%\ifchilddoc\else\providecommand{\version}{draft}\fi
%    \end{macrocode}

% Define the default values for the |\version| flag
% (|final| for the main file and |draft| for childs):
%    \begin{macrocode}
\ifchilddoc
\providecommand{\version}{draft}
\else
\providecommand{\version}{final}
\fi
%    \end{macrocode}

% Load the standard document class:
%    \begin{macrocode}
\documentclass[12pt]{article}
%    \end{macrocode}

% Start the document body:
%    \begin{macrocode}
\begin{document}
%    \end{macrocode}

% Declare a title page.
% Print title, part of document being processed and version flag:
%    \begin{macrocode}
\addtocounter{page}{-1}
\begin{center}
{\LARGE\bfseries{}childdoc example\par}
\vspace{1cm}
\ifchilddoc
\ifchilddocmanual part\else chapter\fi:
`\childdocname' of `\childdocjob'\par
\else
main document: `\childdocjob'\par
\fi
version: \version\par
\end{center}
\newpage
%    \end{macrocode}

% Manually include selected file,
% otherwise process as usual:
%    \begin{macrocode}
\ifchilddocmanual
\section*{part `\childdocname'}
\input{\childdocname}
\else
%    \end{macrocode}

% Include the two chapters:
%    \begin{macrocode}
\include{cdocsch1}
\include{cdocsch2}
%    \end{macrocode}

% Include the two parts unless only chapters should be displayed:
%    \begin{macrocode}
\ifchilddoc\else
\section{part three}
\input{cdocspt3}
\section{part four}
\input{cdocspt4}
\fi
%    \end{macrocode}

% Process as usual until here:
%    \begin{macrocode}
\fi
%    \end{macrocode}

% End of document body:
%    \begin{macrocode}
\end{document}
%    \end{macrocode}
%\iffalse
%</samplemain>
%\fi
%
% %%%%%%%%%%%%%%%%%%%%%%%%%%%%%%%%%%%%%%
% \paragraph{Chapter Include Files.}
%
% The include files are called |cdocsch1.tex| and |cdocsch2.tex|.
%
%\iffalse
%<*samplechap1|samplechap2>
%\fi

% Optional override for |\version| flag:
%    \begin{macrocode}
%%\providecommand{\version}{final}
%    \end{macrocode}

% Include the main document:
%    \begin{macrocode}
\input{childdoc.def}
\childdocof{cdocsamp}
%    \end{macrocode}

%\iffalse
%</samplechap1|samplechap2>
%\fi
%
%\iffalse
%<*samplechap1>
%\fi
% Some text for chapter 1:
%    \begin{macrocode}
\section{one}
some text in chapter one
%    \end{macrocode}

%\iffalse
%</samplechap1>
%\fi
% Some text for chapter 2:
%\iffalse
%<*samplechap2>
%\fi
%    \begin{macrocode}
\section{two}
more text in chapter two
%    \end{macrocode}

%\iffalse
%</samplechap2>
%\fi
%
% %%%%%%%%%%%%%%%%%%%%%%%%%%%%%%%%%%%%%%
% \paragraph{Part Include Files.}
%
% The include files are called |cdocspt3.tex| and |cdocspt4.tex|.
%
%\iffalse
%<*samplepart3|samplepart4>
%\fi

% Optional override for |\version| flag:
%    \begin{macrocode}
%%\providecommand{\version}{final}
%    \end{macrocode}

% Include the main document:
%    \begin{macrocode}
\input{childdoc.def}
\childdocby{cdocsamp}
%    \end{macrocode}

%\iffalse
%</samplepart3|samplepart4>
%\fi
%
%\iffalse
%<*samplepart3>
%\fi
% Some text for part 3:
%    \begin{macrocode}
some text in part three
%    \end{macrocode}

%\iffalse
%</samplepart3>
%\fi
% Some text for part 4:
%\iffalse
%<*samplepart4>
%\fi
%    \begin{macrocode}
more text in part four
%    \end{macrocode}

%\iffalse
%</samplepart4>
%\fi
%
% %%%%%%%%%%%%%%%%%%%%%%%%%%%%%%%%%%%%%%
% \paragraph{Forwarding for a Complete Draft.}
%
% The following forwarding file |cdocsdrf.tex|
% compiles the main document in draft mode:
%\iffalse
%<*sampledraft>
%\fi
%    \begin{macrocode}
\def\version{draft}
\input{childdoc.def}
\childdocforward{cdocsamp}
%    \end{macrocode}

%\iffalse
%</sampledraft>
%\fi
%
% %%%%%%%%%%%%%%%%%%%%%%%%%%%%%%%%%%%%%%
% \paragraph{Forwarding for Final Version of the Chapters.}
%
% The following forwarding files |cdocsfn1.tex| and |cdocsfn2.tex|
% (with identical content)
% compile the final versions of the child documents
% |cdocsch1.tex| and |cdocsch2.tex|, respectively:
%\iffalse
%<*samplefinal>
%\fi
%    \begin{macrocode}
\def\version{final}
\input{childdoc.def}
\childdocforwardprefix[cdocsamp]{cdocsfn}{cdocsch}
%    \end{macrocode}

%\iffalse
%</samplefinal>
%\fi
%
% %%%%%%%%%%%%%%%%%%%%%%%%%%%%%%%%%%%%%%
% \paragraph{Command Line Processing.}
%
% The following three command lines generate the output files
% |cdocscld|, |cdocscl1| and |cdocscl2|
% which should be identical to
% |cdocsdrf|, |cdocsch1| and |cdocsfn2|, respectively:
% \begin{center}
% \begin{tabular}{l}
% |latex -jobname cdocscld \|\\
% |  "\def\version{draft}\input{childdoc.def}\childdocforward{cdocsamp}"|\\
% |latex -jobname cdocscl1 \|\\
% |  "\input{childdoc.def}\childdocforward[cdocsamp]{cdocsch1}"|\\
% |latex -jobname cdocscl2 \|\\
% |  "\def\version{final}\input{childdoc.def}\childdocforward{cdocsch2}"|
% \end{tabular}
% \end{center}
% Note that the trailing backslash on each first line
% merely continues the input to the second line
% (for convenient cut ant paste).
% Furthermore, the command |latex| can be replaced by any
% of its alternative versions such as |pdflatex|.
%
% %%%%%%%%%%%%%%%%%%%%%%%%%%%%%%%%%%%%%%%%%%%%%%%%%%%%%%%%%%%%%%%%%%%%%%%%%%%%%%
% %%%%%%%%%%%%%%%%%%%%%%%%%%%%%%%%%%%%%%%%%%%%%%%%%%%%%%%%%%%%%%%%%%%%%%%%%%%%%%
% \section{Implementation}
%\iffalse
%<*package>
%\fi
%
% This section describes the definitions file |childdoc.def|.

% The definitions cannot be loaded using |\usepackage| or |\RequirePackage|
% which has a mechanism to prevent loading a style file more than once.
% When loading the definitions by means of |\input|
% multiple instances have to be prevented manually:
%\iffalse
%This code needs to be before the `\ProvidesFile' directive
%which is defined at the beginning of this file.
%Therefore it is also placed there and commented out here.
%</package>
%<*discard>
%\fi
%    \begin{macrocode}
\ifdefined\childdocmain\endinput\fi
%    \end{macrocode}
%\iffalse
%</discard>
%<*package>
%\fi
%
% \macro{\ifchilddoc}
% \macro{\ifchilddocmanual}
% The conditional |\ifchilddoc| tells whether a
% child (true) or main (false) document is being compiled.
% The conditional |\ifchilddocmanual| tells whether
% the |\includeonly| mechanism is used (false) or
% the selection of child files must be performed manually (true).
% The definitions initialise to false:
%    \begin{macrocode}
\newif\ifchilddoc
\newif\ifchilddocmanual
%    \end{macrocode}

% \macro{\childdocname}
% \macro{\childdocjob}
% The macro |\childdocname| stores the name of the main document
% to be compiled. The macro |\childdocjob| stores the name of
% the document on which the \LaTeX{} compiler was originally invoked.
% The content of |\jobname| cannot be compared
% to filenames specified in the source due to different catcodes.
% The following code rescans |\jobname|, stores the result
% in |\childdocname| and saves a copy in |\childdocjob|:
%    \begin{macrocode}
\edef\childdocname{\scantokens\expandafter{\jobname\noexpand}}
\let\childdocjob\childdocname
%    \end{macrocode}

% \macro{\childdocdisable}
% The macro |\childdocdisable| prevents the main file
% from being processed more than once.
% At this stage, the main document command |\childdocmain|
% is assumed to be called once again where it should do nothing.
% Any subsequent call to it should prevent
% a secondary processing of the main document
% It overwrites the forwarding commands
% |\childdocof| and |\childdocforward|
% with empty macros to prevent further inclusions of the main document:
%    \begin{macrocode}
\newcommand{\childdocdisable}
{
  \renewcommand{\childdocmain}[1]{\renewcommand{\childdocmain}[1]{\endinput}}
  \renewcommand{\childdocof}[1]{}
  \renewcommand{\childdocby}[2][]{}
  \renewcommand{\childdocforward}[2][]{}
  \renewcommand{\childdocdisable}{}
}
%    \end{macrocode}

% \macro{\childdocmain}
% The macro |\childdocmain| is to be called at the top of the main file
% with nothing or the main filename (without extension) as argument.
% First, it breaks loops.
% If the argument is not empty and does not match |\childdocname|
% (which is set by the first inclusion of |childdoc.def|),
% |\ifchilddoc| is set to true, |\includeonly| is applied to the child file
% and |\jobname| is set to the main file
% (for proper handling of |.aux| files):
%    \begin{macrocode}
\newcommand{\childdocmain}[1]
{
  \childdocdisable\childdocmain{}
  \if?#1?\else
    \begingroup
      \def\childdoctmp{#1}
      \ifx\childdoctmp\childdocname
        \def\childdoctmp{}
      \else
        \def\childdoctmp
        {
          \childdoctrue
          \includeonly{\childdocname}
          \def\childdocjob{#1}
          \def\jobname{#1}
        }
      \fi
      \expandafter
    \endgroup
    \childdoctmp
  \fi
}
%    \end{macrocode}

% \macro{\childdocof}
% The command |\childdocof| redirects
% compilation to the main file |#1|.
%    \begin{macrocode}
\newcommand{\childdocof}[1]
{
  \childdocdisable
  \childdoctrue
  \includeonly{\childdocname}
  \def\jobname{#1}
  \def\childdocjob{#1}
  \input{#1}
}
%    \end{macrocode}

% \macro{\childdocby}
% The command |\childdocby| ....
%    \begin{macrocode}
\newcommand{\childdocby}[2][]
{
  \childdocdisable
  \childdoctrue
  \childdocmanualtrue
  \if?#1?\else
    \def\jobname{#2}
  \fi
  \def\childdocjob{#2}
  \input{#2}
  \endinput
}
%    \end{macrocode}

% \macro{\childdocforward}
% The command |\childdocforward| redirects
% compilation to the main file or
% (if the optional argument is given) a child file.
% Parameters are set as if the main file
% or a child file starting with |\childdocof| was compiled.
% Then compilation is handed over to the main file:
%    \begin{macrocode}
\newcommand{\childdocforward}[2][]
{
  \begingroup
    \if?#1?
      \def\childdoctmp
      {
        \def\childdocname{#2}
        \def\childdocjob{#2}
        \def\jobname{#2}
        \input{#2}
        \endinput
      }
    \else
      \def\childdoctmp
      {
        \childdocdisable
        \def\childdocname{#2}
        \childdoctrue
        \includeonly{#2}
        \def\childdocjob{#1}
        \def\jobname{#1}
        \input{#1}
        \endinput
      }
    \fi
    \expandafter
  \endgroup
  \childdoctmp
}
%    \end{macrocode}

% \macro{\childdocforwardprefix}
% The command |\childdocforwardprefix| redirects
% compilation to the main or a child file by means of a pattern.
% The prefix |#1| in the current filename is replaced by |#2|
% and the suffix of the current filename is kept
% (it is assumed that the filename does not contain the substring `|~~~|'
% which is used as a delimiter).
% Compilation is handed over to the new file by |\childdocforward|:
%    \begin{macrocode}
\newcommand{\childdocforwardprefix}[3][]
{
  \begingroup
    \def\childdocextract #2##1~~~{\def\childdoctmp{\childdocforward[#1]{#3##1}}}
    \expandafter\childdocextract\childdocname~~~
    \expandafter
  \endgroup
  \childdoctmp
}
%    \end{macrocode}

% \macro{\childdoc}
% The deprecated macro |\childdoc| is a legacy version of |\childdocmain|:
%    \begin{macrocode}
\newcommand{\childdoc}{\childdocmain}
%    \end{macrocode}

% \macro{\childdocredirect}
% The deprecated macro |\childdocredirect| is a legacy version
% of |\childdocforward| and |\childdocforwardprefix|:
%    \begin{macrocode}
\newcommand{\childdocredirect}[2][]
{
  \begingroup
    \if?#1?
      \def\childdoctmp{\childdocforward{#2}}
    \else
      \def\childdoctmp{\childdocforwardprefix{#1}{#2}}
    \fi
    \expandafter
  \endgroup
  \childdoctmp
}
%    \end{macrocode}

%\iffalse
%</package>
%\fi
%
\endinput
|\\
|\childdocforward{|\textit{main}|}|\\
\end{tabular}
\end{center}
%
or alternatively with:
%
\begin{center}
\begin{tabular}{l}
|% \iffalse
%
% childdoc.dtx Copyright (C) 2017-2018 Niklas Beisert
%
% This work may be distributed and/or modified under the
% conditions of the LaTeX Project Public License, either version 1.3
% of this license or (at your option) any later version.
% The latest version of this license is in
%   http://www.latex-project.org/lppl.txt
% and version 1.3 or later is part of all distributions of LaTeX
% version 2005/12/01 or later.
%
% This work has the LPPL maintenance status `maintained'.
%
% The Current Maintainer of this work is Niklas Beisert.
%
% This work consists of the files childdoc.dtx and childdoc.ins
% and the derived files childdoc.def and cdocsamp.tex with
% cdocsch1.tex, cdocsch2.tex, cdocsdrf.tex, cdocsfn1.tex, cdocsfn2.tex.
%
%<package>\ifdefined\childdocmain\endinput\fi
%<package>\ProvidesFile{childdoc.def}[2018/12/30 v2.0 child document driver]
%<samplemain>\ProvidesFile{cdocsamp.tex}[2018/12/30 v2.0 sample for childdoc]
%<*driver>
%\ProvidesFile{childdoc.drv}[2018/12/30 v2.0 childdoc reference manual file]
\PassOptionsToClass{10pt,a4paper}{article}
\documentclass{ltxdoc}

\usepackage[margin=35mm]{geometry}
\usepackage{hyperref}
\usepackage{hyperxmp}
\usepackage[usenames]{color}

\hypersetup{colorlinks=true}
\hypersetup{pdfstartview=FitH}
\hypersetup{pdfpagemode=UseNone}
\hypersetup{pdfsource={}}
\hypersetup{pdflang={en-UK}}
\hypersetup{pdfcopyright={Copyright 2017-2018 Niklas Beisert.
  This work may be distributed and/or modified under the
  conditions of the LaTeX Project Public License, either version 1.3
  of this license or (at your option) any later version.}}
\hypersetup{pdflicenseurl={http://www.latex-project.org/lppl.txt}}
\hypersetup{pdfcontactaddress={ETH Zurich, ITP, HIT K,
  Wolfgang-Pauli-Strasse 27}}
\hypersetup{pdfcontactpostcode={8093}}
\hypersetup{pdfcontactcity={Zurich}}
\hypersetup{pdfcontactcountry={Switzerland}}
\hypersetup{pdfcontactemail={nbeisert@itp.phys.ethz.ch}}
\hypersetup{pdfcontacturl={http://people.phys.ethz.ch/\xmptilde nbeisert/}}

\newcommand{\secref}[1]{\hyperref[#1]{section \ref*{#1}}}

\parskip1ex
\parindent0pt
\let\olditemize\itemize
\def\itemize{\olditemize\parskip0pt}

\begin{document}

\title{The \textsf{childdoc} Package}
\hypersetup{pdftitle={The childdoc Package}}
\author{Niklas Beisert\\[2ex]
  Institut f\"ur Theoretische Physik\\
  Eidgen\"ossische Technische Hochschule Z\"urich\\
  Wolfgang-Pauli-Strasse 27, 8093 Z\"urich, Switzerland\\[1ex]
  \href{mailto:nbeisert@itp.phys.ethz.ch}
  {\texttt{nbeisert@itp.phys.ethz.ch}}}
\hypersetup{pdfauthor={Niklas Beisert}}
\hypersetup{pdfsubject={Manual for the LaTeX2e Package childdoc}}
\date{30 December 2018, \textsf{v2.0}}
\maketitle

\begin{abstract}\noindent
\textsf{childdoc} is a \LaTeXe{} package
that enables the direct compilation
of document sections included by |\include|
to individual files.
\end{abstract}

\begingroup
\parskip0ex
\tableofcontents
\endgroup

%%%%%%%%%%%%%%%%%%%%%%%%%%%%%%%%%%%%%%%%%%%%%%%%%%%%%%%%%%%%%%%%%%%%%%%%%%%%%%%%
%%%%%%%%%%%%%%%%%%%%%%%%%%%%%%%%%%%%%%%%%%%%%%%%%%%%%%%%%%%%%%%%%%%%%%%%%%%%%%%%
\section{Introduction}

\LaTeX{} provides a mechanism to structure a large document (such as a book)
into a main file and several child files (containing the chapters)
using the |\include| command.
This mechanism is beneficial for documents
which span hundreds of pages in order to
make the source file(s) more manageable.
Moreover, compilation can be restricted to
selected child files by means of the |\includeonly| command.
The latter feature can be used to reduce the compilation time while editing
(this was significantly more useful in the earlier days of \LaTeX{})
or to generate a smaller document which is easier to navigate.
Another application of |\includeonly| is to generate
documents consisting of selected parts of the complete document.

However, there are a few drawbacks of the plain |\include| mechanism:
\begin{itemize}
\item
The child files cannot be compiled on their own,
they can only be compiled via the main file.
A naive editing environment
(such as a text editor with an option
to have the current file processed by \LaTeX)
may require one to switch to the main file before compiling;
attempting to compile the child file produces errors.
\item
The main file must be modified (each time)
to adjust the |\includeonly| command
to the present needs. This easily leaves the main file in a messy state.
\item
The generated document will always carry the filename
of the main document. This is inconvenient if
several child files are to be compiled and
to be kept for distribution.
\end{itemize}

The present package provides a simple interface
to make child files individually compilable by \LaTeX{}.
Compiling a child file then has the same effect as compiling
the main file with an |\includeonly| command
to select the appropriate child.
Moreover the generated document will carry the name of the child
rather than the main file.
This resolves all three above issues.

This feature is meant to make the editing of books,
thesis documents and lecture notes somewhat more convenient.
However, the package can also be used efficiently for
composing a series of documents (such as exercise sheets)
which are typically distributed individually.
It then assists the author in generating the individual documents
(potentially in different versions)
as well as a document containing the collected series.
Another application is in developing style files
or other kinds of included material
where compilation of the style file could redirect
to a sample or test file.

%%%%%%%%%%%%%%%%%%%%%%%%%%%%%%%%%%%%%%%%%%%%%%%%%%%%%%%%%%%%%%%%%%%%%%%%%%%%%%%%
%%%%%%%%%%%%%%%%%%%%%%%%%%%%%%%%%%%%%%%%%%%%%%%%%%%%%%%%%%%%%%%%%%%%%%%%%%%%%%%%
\section{Usage}

First of all, the package \textsf{childdoc} is \emph{not} a standard
\LaTeXe{} |.sty| style file! Therefore it needs to be invoked in
a non-standard way.

%%%%%%%%%%%%%%%%%%%%%%%%%%%%%%%%%%%%%%%%%%%%%%%%%%%%%%%%%%%%%%%%%%%%%%%%%%%%%%%%
\subsection{Included Files}
\label{sec:include}

%%%%%%%%%%%%%%%%%%%%%%%%%%%%%%%%%%%%%%%%
\DescribeMacro{\childdocmain}
To use the package, add the commands
\begin{center}
\begin{tabular}{l}
|\input{childdoc.def}|\\
|\childdocmain{}|\\
\end{tabular}
\end{center}
at the very top of the main \LaTeX{} file,
in particular \emph{before} the |\documentclass| statement!
The argument of |\childdocmain| should be left empty
(but it must be present).

%%%%%%%%%%%%%%%%%%%%%%%%%%%%%%%%%%%%%%%%
\DescribeMacro{\childdocof}
Furthermore, add the commands
\begin{center}
\begin{tabular}{l}
|\input{childdoc.def}|\\
|\childdocof{|\textit{main}|}|\\
\end{tabular}
\end{center}
at the top of every child file \textit{child}
which is included by |\include{|\textit{child}|}|
from within the main file
(or at least for those files to be compiled individually).
The argument \textit{main} must be the filename of the main file.

There are a couple of
considerations in setting up the main and child documents:

%%%%%%%%%%%%%%%%%%%%%%%%%%%%%%%%%%%%%%%%
\paragraph{Restrictions.}

Please note the following restrictions:
\begin{itemize}
\item
|\childdocmain| must be called with one argument \textit{main}
to ensure compatibility with earlier version of the package.
It must either be empty (|\childdocmain{}|)
or precisely match the filename of the main file in which it is specified.
See \secref{sec:detection} for further information.
\item
The filename \textit{main} must be specified without the |.tex| extension.
\item
The filename \textit{main} is case sensitive
(even in case-insensitive file systems)
due to internal string comparison.
\item
The argument \textit{main} should be fully expanded, it cannot be a macro.
\item
Subdirectories and special characters should be avoided in filenames.
\item
The command |\childdocmain{|\textit{main}|}| must be followed by a whitespace.
It should not be followed immediately by another command
or by a comment mark `|%|'.
This is because the \TeX{} parser reads the token immediately following
the argument of |\childdocmain| and puts it
at the beginning of every child section;
however, a white\-space is ignored.
\end{itemize}

%%%%%%%%%%%%%%%%%%%%%%%%%%%%%%%%%%%%%%%%
\paragraph{Content of Main File.}

It is advisable to place all content in the child files included by |\include|.
Any output contained in the main file will appear in all child documents
unless suppressed manually;
it cannot be suppressed automatically by the |\includeonly| directive
and thus should normally be avoided.
A method to include some content in the main file
by means of conditional processing is described in \secref{sec:conditional}.

%%%%%%%%%%%%%%%%%%%%%%%%%%%%%%%%%%%%%%%%
\paragraph{Page Numbering.}

When only a part of the document is compiled,
the appropriate numbering of pages
(as well as other status parameters)
is determined from the |.aux| files.
The latter contain information from previous passes.
However this information needs to propagate through
all intermediate child documents.
Therefore the page numbering in child documents may well
be inconsistent until the complete document is compiled at least once.

A useful (if unconventional) way to always ensure a consistent
page numbering is to restart the numbering in each child document
and denote the pages by `\textit{child}|.|\textit{page}'
where \textit{child} represents the chapter/section number of the child file.
This can be achieved by the command
|\numberwithin{page}{|\textit{child}|}|
of the \textsf{amsmath} package
where \textit{child} can be |chapter| or |section|
depending on the chosen structuring.
Alternatively, one can modify the macro |\thepage| appropriately
and reset the counter |page| at the start of each child file.

%%%%%%%%%%%%%%%%%%%%%%%%%%%%%%%%%%%%%%%%%%%%%%%%%%%%%%%%%%%%%%%%%%%%%%%%%%%%%%%%
\subsection{Conditional Processing}
\label{sec:conditional}

The package provides a mechanism to compile different versions
of a document. To customise the versions further some conditional processing
can come in handy to distinguish which version is being compiled.
The package provides two macros to describe the compilation context:

%%%%%%%%%%%%%%%%%%%%%%%%%%%%%%%%%%%%%%%%
\DescribeMacro{\ifchilddoc}
The conditional |\ifchilddoc| distinguishes between the compilation of
child documents and the main document:
%
\begin{center}
|\ifchilddoc |\textit{child-code}| |[|\||else |\textit{main-code}]| \||fi|
\end{center}

%%%%%%%%%%%%%%%%%%%%%%%%%%%%%%%%%%%%%%%%
\DescribeMacro{\childdocname}
\DescribeMacro{\childdocjob}
The macro |\childdocname| contains the filename (without extension)
of the main or child file being processed.
Note that |\childdocjob| will always contain the name of the main file.

%%%%%%%%%%%%%%%%%%%%%%%%%%%%%%%%%%%%%%%%
\paragraph{Title Page.}

Conditional processing can be used to include a title or banner page
in the main document when proper precautions are taken.
Importantly, the code in the main file should ensure that the page counter
(as well as other status parameters which are stored in the |.aux| files)
takes the same value after the conditional processing.
Otherwise the page numbers may take divergent values
depending on which part is compiled.

For example, a title page could be declared by:
%
\begin{center}
\begin{tabular}{l}
|\ifchilddoc\||else|\\
|\addtocounter{page}{-1}|\\
\textit{code for title page}\\
|\newpage|\\
|\||fi|
\end{tabular}
\end{center}
%
A banner page for the child documents can be generated by:
%
\begin{center}
\begin{tabular}{l}
|\ifchilddoc|\\
|\addtocounter{page}{-1}|\\
\textit{code for banner page}\\
|\newpage|\\
|\||fi|
\end{tabular}
\end{center}
%
Here one could write a message such as:
\begin{center}
|This is the part \childdocname{} of \childdocjob{}.|
\end{center}

%%%%%%%%%%%%%%%%%%%%%%%%%%%%%%%%%%%%%%%%%%%%%%%%%%%%%%%%%%%%%%%%%%%%%%%%%%%%%%%%
\subsection{Flags}
\label{sec:flags}

The package makes it easy to generate different versions
of the main or child documents.
To this end compilation flags can be defined
and assigned different default values.
They will be particularly useful in conjunction
with the forwarding mechanism described in \secref{sec:forward}.

For example, it may be useful to have a flag |\version|
which can be set to |draft| or |final|.
The document source will contain some conditional code
depending on the value of |\version|.
Suppose further, the flag should default to |final| for the main file
and to |draft| for child files
which is a natural assignment for editing the document.
This is achieved by placing the following code
in the preamble of the main document
(below the |\childdocmain| directive):
%
\begin{center}
\begin{tabular}{l}
|\ifchilddoc|\\
|\providecommand{\version}{draft}|\\
|\||else|\\
|\providecommand{\version}{final}|\\
|\||fi|
\end{tabular}
\end{center}
%
The definition by |\providecommand| makes sure
that previous definitions are not overwritten.
Further statements |\providecommand{\version}{...}|
can thus be added before the above code to override it.

For the main file, one might add a line
(between |\childdocmain| and the above block)
%
\begin{center}
|%\ifchilddoc\||else\providecommand{\version}{draft}\||fi|
\end{center}
%
which can be uncommented to produce a draft version.
Likewise one can add a line to the very top of a child file
(above the |\childdocof{|\textit{main}|}| directive)
%
\begin{center}
|%\providecommand{\version}{final}|
\end{center}
%
which can be uncommented to produce the final version of this child document.

%%%%%%%%%%%%%%%%%%%%%%%%%%%%%%%%%%%%%%%%%%%%%%%%%%%%%%%%%%%%%%%%%%%%%%%%%%%%%%%%
\subsection{Forwarding}
\label{sec:forward}

Different versions of the main or child documents
using compilation flags as described in \secref{sec:flags}
can be (permanently) stored in different files
for convenient compilation, viewing and distribution.
To this end, the package defines a command
to pass on compilation to a different file:

%%%%%%%%%%%%%%%%%%%%%%%%%%%%%%%%%%%%%%%%
\DescribeMacro{\childdocforward}
The command |\childdocforward| redirects processing to
another source file:
%
\begin{center}
\begin{tabular}{l}
|\input{childdoc.def}|\\
|\childdocforward[|\textit{main}|]{|\textit{dest}|}|\\
\end{tabular}
\end{center}
%
The argument \textit{dest} is the destination file
(without extension).
It should be the main file or one of the child files.
Note that further \textsf{childdoc} directives
such as |\childdocof| and |\childdocforward|
in the indicated file will be processed in this form.
The optional argument \textit{main}
passes on directly to the main file \textit{main}
while pretending to compile the child \textit{dest}.
This form behaves as if \textit{dest}
issues |\childdocof{|\textit{main}|}| right away,
and no further \textsf{childdoc} directives will be processed.

%%%%%%%%%%%%%%%%%%%%%%%%%%%%%%%%%%%%%%%%
\DescribeMacro{\...prefix}
In the alternative form |\childdocforwardprefix|,
%
\begin{center}
\begin{tabular}{l}
|\input{childdoc.def}|\\
|\childdocforwardprefix[|\textit{main}|]{|\textit{prefix}|}{|\textit{dest}|}|
\end{tabular}
\end{center}
%
the destination file is determined by a pattern
depending on the current file:
To make this work, the current file must be called
`{\textit{prefix}\hspace{0.2em}\textit{suffix}}'
with \textit{prefix} matching precisely the argument.
Processing is then passed on to the file
`{\textit{dest}\hspace{0.2em}\textit{suffix}}'.
Surely, the same effect is achieved by
directly specifying the
argument `{\textit{dest}\hspace{0.2em}\textit{suffix}}'
in the first form.
However, that requires to set up a different file
for each child. With the alternative form of the command
all these files can have exactly the same content
which simplifies setting them up and maintaining them.

For example, the following file |draft.tex|
with a compilation flag |\version| as described in \secref{sec:flags}
compiles the main document as a draft:
%
\begin{center}
\begin{tabular}{l}
|\def\version{draft}|\\
|\input{childdoc.def}|\\
|\childdocforward{|\textit{main}|}|
\end{tabular}
\end{center}
%
Likewise, the following files |final|\textit{nn}|.tex|
compile the final version of the child document
|child|\textit{nn}|.tex|:
%
\begin{center}
\begin{tabular}{l}
|\def\version{final}|\\
|\input{childdoc.def}|\\
|\childdocforwardprefix{final}{child}|
\end{tabular}
\end{center}
%

Note that when several versions of a main file and/or of each child file
are to be generated, it may be convenient to set up a |Makefile| or
shell script to automatise the process.

%%%%%%%%%%%%%%%%%%%%%%%%%%%%%%%%%%%%%%%%%%%%%%%%%%%%%%%%%%%%%%%%%%%%%%%%%%%%%%%%
\subsection{Command Line Processing}
\label{sec:commandline}

The effect of redirection files can also be achieved by invoking
the \LaTeX{} compiler with a more elaborate command line.
Most conveniently this should be done as part
of a shell script or a |Makefile|.

When using \textsf{childdoc} in the main file, the following
command lines effectively perform a redirection
(note that depending on the shell being used,
backslashes may have to be doubled: `|\|' $\to$ `|\\|'):
%
\begin{center}
|... -jobname "|\textit{target}|" |\\|"|[\textit{flags}]%
|\input{childdoc.def}\childdocforward[|\textit{main}|]{|\textit{dest}|}"|
\end{center}
%
Here \textit{target} is the name of the output file,
\textit{main} is the name of the main file
and \textit{dest} is the name of the main or child file to be processed
(all filenames without extensions).
The optional argument \textit{main} can be omitted
if \textit{main} matches \textit{dest}.
Optionally, compilation \textit{flags} can be defined via |\def| commands.
This command line makes the \TeX{} engine believe
it is compiling the file \textit{target}
whose content is specified as the latter parameter.
The provided code then forwards the processing to
\textit{main} or \textit{dest} as described in \secref{sec:forward}.

%%%%%%%%%%%%%%%%%%%%%%%%%%%%%%%%%%%%%%%%%%%%%%%%%%%%%%%%%%%%%%%%%%%%%%%%%%%%%%%%
\subsection{Include by Input}
\label{sec:input}

Including child documents by |\include| has some restrictions by design.
Most notably, the content of a child document always occupies
its own set of pages; pages cannot be shared between child documents.
Usually, this behaviour makes perfect sense
because each child document contain an essential part of the document.
However, in some situations it may be desirable to compose
a document from a collection of parts
without having mandatory page breaks between then.
For this case, the package
provides a mechanism to include parts
by |\input| which can also be processed individually.
However, by construction this mechanism
requires manual handling of the content to be output.

%%%%%%%%%%%%%%%%%%%%%%%%%%%%%%%%%%%%%%%%
\DescribeMacro{\ifchilddocmanual}
The main file should be prepared as usual, see \secref{sec:include}.
However, the document body must make a distinction
between processing of an individual part and of the main document, e.g.:
%
\begin{center}
\begin{tabular}{l}
|\ifchilddocmanual|\\
|\input{\childdocname}|\\
|\||else|\\
\textit{document body with }|\input{|\textit{part}|}|\\
|\||fi|
\end{tabular}
\end{center}
%
The conditional |\ifchilddocmanual| is true whenever
a part to be included by |\input| is being compiled,
and the name of the part is stored in |\childdocname|.

%%%%%%%%%%%%%%%%%%%%%%%%%%%%%%%%%%%%%%%%
\DescribeMacro{\childdocby}
Each part to be included by |\input| should start with:
%
\begin{center}
\begin{tabular}{l}
|\input{childdoc.def}|\\
|\childdocby{|\textit{main}|}|\\
\end{tabular}
\end{center}
%
The directive |\childdocby| is similar to |\childdocof|
described in \secref{sec:include},
but the subsequent selection of content must be done manually.
To that end, both |\ifchilddoc| and |\ifchilddocmanual|
will be true upon processing of a part,
and the name of the part is stored in |\childdocname|.
Note that |\jobname| will be set to the filename of the current part
so that each part receives an individual |.aux| file
that does not interfere with the |.aux| file(s) of the main document.
This behaviour can be altered by the alternative form
|\childdocby[*]{|\textit{main}|}| (with a non-empty optional argument)
which uses the |.aux| file of the main document
by setting |\jobname| to \textit{main}.

%%%%%%%%%%%%%%%%%%%%%%%%%%%%%%%%%%%%%%%%%%%%%%%%%%%%%%%%%%%%%%%%%%%%%%%%%%%%%%%%
\subsection{Driver Development}
\label{sec:driver}

The \textsf{childdoc} mechanism can also be use for the development
of definition files such as \LaTeX{} styles or classes.
This case differs from the above setup with multiple parts
included by |\include| in that no |\includeonly| should be invoked.
This can be achieved by starting the include file
(before |\ProvidesPackage|) with:
%
\begin{center}
\begin{tabular}{l}
|\input{childdoc.def}|\\
|\childdocforward{|\textit{main}|}|\\
\end{tabular}
\end{center}
%
or alternatively with:
%
\begin{center}
\begin{tabular}{l}
|\input{childdoc.def}|\\
|\childdocby{|\textit{main}|}|\\
\end{tabular}
\end{center}
%
Both forms have slightly different effects as described above.
The main file is prepared as usual, see \secref{sec:include}.

%%%%%%%%%%%%%%%%%%%%%%%%%%%%%%%%%%%%%%%%%%%%%%%%%%%%%%%%%%%%%%%%%%%%%%%%%%%%%%%%
\subsection{Legacy Detection}
\label{sec:detection}

The directive |\childdocmain| in the main file can detect
whether the complete document or merely a child is to be compiled
even without using the directive |\childdocof|.
This method is deprecated because it is less robust
and there is no compelling reason to use it;
it is merely provided for backward compatibility
and it may be removed in future versions.

If the detection mechanism is to be used,
it is mandatory to correctly specify
the filename of the main file as the argument of |\childdocmain|:
%
\begin{center}
\begin{tabular}{l}
|\input{childdoc.def}|\\
|\childdocmain{|\textit{main}|}|\\
\end{tabular}
\end{center}
%
If |\jobname| does not match the argument \textit{main} of |\childdocmain|,
it is assumed that |\jobname| points to the child file to be compiled.
When using |\childdocmain| with the main file specified as argument,
it suffices to start a child file
with just |\input{|\textit{main}|}|
without loading of the package and using |\childdocof|.
If instead all processing is done
with the appropriate \textsf{childdoc} directives,
the argument of \textit{main} of |\childdocmain| can be empty.

An alternative version of the command line processing described
in \secref{sec:commandline} using the detection mechanism reads:
%
\begin{center}
|... -jobname "|\textit{target}|" "|[\textit{flags}]%
[|\def\jobname{|\textit{dest}|}|]|\input{|\textit{main}|}"|
\end{center}

%%%%%%%%%%%%%%%%%%%%%%%%%%%%%%%%%%%%%%%%%%%%%%%%%%%%%%%%%%%%%%%%%%%%%%%%%%%%%%%%
\subsection{Manual Code}
\label{sec:manual}

In case one cannot be certain whether the definitions file |childdoc.def|
is installed on the target \TeX{} distribution
and one prefers not to ship it,
it is conceivable to paste a few relevant commands into the sources.

To that end, drop all statements |\input{childdoc.def}|
and perform the replacements as outlined below.
Instead of |\childdocmain{|\textit{main}|}| add the following code
to the top of the main file:
%
\begin{center}
\begin{tabular}{l}
|\||ifdefined\childdocname\endinput\||fi\newif\ifchilddoc|\\
|\edef\childdocname{\scantokens\expandafter{\jobname\noexpand}}|\\
|\def\childdocmain{|\textit{main}|}\||ifx\childdocmain\childdocname\||else|\\
|\childdoctrue\includeonly{\childdocname}\let\jobname\childdocmain\||fi|\\
\end{tabular}
\end{center}
%
Instead of |\childdocof{|\textit{main}|}| just include the main file
at the top of each child file:
%
\begin{center}
|\input{|\textit{main}|}|
\end{center}
%
A simple redirection |\childdocforward{|\textit{dest}|}| is achieved by:
%
\begin{center}
|\def\jobname{|\textit{dest}|}\input{\jobname}|
\end{center}
%
The redirection with prefix
|\childdocforwardprefix[|\textit{prefix}|]{|\textit{dest}|}|
is accomplished by:
%
\begin{center}
\begin{tabular}{l}
|{\edef\jobname{\scantokens\expandafter{\jobname\noexpand}}|\\
|\def\redirectjob |\textit{prefix}|#1~~~{\gdef\jobname{|\textit{dest}|#1}}|\\
|\expandafter\redirectjob\jobname~~~}\input{\jobname}|
\end{tabular}
\end{center}

In an alternative approach,
child documents can be compiled by a specific command line
without additional code or specific definitions:
%
\begin{center}
|... -jobname "|\textit{target}|" "|[\textit{flags}]%
|\includeonly{|\textit{dest}|}\input{|\textit{main}|}"|
\end{center}
%

%%%%%%%%%%%%%%%%%%%%%%%%%%%%%%%%%%%%%%%%%%%%%%%%%%%%%%%%%%%%%%%%%%%%%%%%%%%%%%%%
%%%%%%%%%%%%%%%%%%%%%%%%%%%%%%%%%%%%%%%%%%%%%%%%%%%%%%%%%%%%%%%%%%%%%%%%%%%%%%%%
\section{Information}

%%%%%%%%%%%%%%%%%%%%%%%%%%%%%%%%%%%%%%%%%%%%%%%%%%%%%%%%%%%%%%%%%%%%%%%%%%%%%%%%
\subsection{Copyright}

Copyright \copyright{} 2017--2018 Niklas Beisert

This work may be distributed and/or modified under the
conditions of the \LaTeX{} Project Public License, either version 1.3
of this license or (at your option) any later version.
The latest version of this license is in
  \url{http://www.latex-project.org/lppl.txt}
and version 1.3 or later is part of all distributions of \LaTeX{}
version 2005/12/01 or later.

This work has the LPPL maintenance status `maintained'.

The Current Maintainer of this work is Niklas Beisert.

This work consists of the files |README.txt|, |childdoc.ins| and |childdoc.dtx|
as well as the derived files |childdoc.def|, |cdocsamp.tex|
with |cdocsch1.tex|, |cdocsch2.tex|, |cdocspt3.tex|, |cdocspt4.tex|,
|cdocsdrf.tex|, |cdocsfn1.tex|, |cdocsfn2.tex|
as well as |childdoc.pdf|.

%%%%%%%%%%%%%%%%%%%%%%%%%%%%%%%%%%%%%%%%%%%%%%%%%%%%%%%%%%%%%%%%%%%%%%%%%%%%%%%%
\subsection{Files and Installation}

The package consists of the files:
%
\begin{center}
\begin{tabular}{ll}
    |README.txt|   & readme file \\
    |childdoc.ins| & installation file \\
    |childdoc.dtx| & source file \\
    |childdoc.def| & definition file \\
    |cdocsamp.tex| & sample main file \\
    |cdocsch1.tex| & sample include file \\
    |cdocsch2.tex| & sample include file \\
    |cdocspt3.tex| & sample part file \\
    |cdocspt4.tex| & sample part file \\
    |cdocsdrf.tex| & sample redirection file \\
    |cdocsfn1.tex| & sample redirection file \\
    |cdocsfn2.tex| & sample redirection file \\
    |childdoc.pdf| & manual
\end{tabular}
\end{center}
%
The distribution consists of the files
|README.txt|, |childdoc.ins| and |childdoc.dtx|.
%
\begin{itemize}
\item
Run (pdf)\LaTeX{} on |childdoc.dtx|
to compile the manual |childdoc.pdf| (this file).
\item
Run \LaTeX{} on |childdoc.ins| to create the definitions file |childdoc.def|
and the sample |cdocsamp.tex| with include files
|cdocsch1.tex|, |cdocsch2.tex|, |cdocspt3.tex|, |cdocspt4.tex|,
|cdocsdrf.tex|, |cdocsfn1.tex|, |cdocsfn2.tex|.
Then copy the file |childdoc.def| to an appropriate directory of your \LaTeX{}
distribution, e.g.\ \textit{texmf-root}|/tex/latex/childdoc|.
\end{itemize}

%%%%%%%%%%%%%%%%%%%%%%%%%%%%%%%%%%%%%%%%%%%%%%%%%%%%%%%%%%%%%%%%%%%%%%%%%%%%%%%%
\subsection{Related CTAN Packages}

There are several other packages which offer a similar functionality:
%
\begin{itemize}
\item
The packages
\href{http://ctan.org/pkg/docmute}{\textsf{docmute}},
\href{http://ctan.org/pkg/includex}{\textsf{includex}} and
\href{http://ctan.org/pkg/standalone}{\textsf{standalone}}
provide commands to include only the document body of
a child file thus allowing both files to be compiled individually.
\item
The packages \href{http://ctan.org/pkg/subdocs}{\textsf{subdocs}}
and \href{http://ctan.org/pkg/subfiles}{\textsf{subfiles}}
provide structures in which the main and child documents can be
encapsulated and allowing them to be compiled individually.
The inclusion mechanism is different from the conventional |\include|.
\item
The package \href{http://ctan.org/pkg/combine}{\textsf{combine}}
is an elaborate solution to combine several documents into one.
\end{itemize}
%
See also the CTAN topic \href{http://ctan.org/topic/subdocs}{\textsf{subdocs}}
for further related packages.
The present package differs from the above solutions in that
a document structure constructed with the conventional |\include| mechanism
just needs two extra commands at the top of every file
such that all constituent files can be compiled individually.

%%%%%%%%%%%%%%%%%%%%%%%%%%%%%%%%%%%%%%%%%%%%%%%%%%%%%%%%%%%%%%%%%%%%%%%%%%%%%%%%
%\subsection{Feature Suggestions}
%
%The following is a list of features which may be useful for future
%versions of this package:
%%
%\begin{itemize}
%\item
%\ldots
%\end{itemize}

%%%%%%%%%%%%%%%%%%%%%%%%%%%%%%%%%%%%%%%%%%%%%%%%%%%%%%%%%%%%%%%%%%%%%%%%%%%%%%%%
\subsection{Revision History}

%%%%%%%%%%%%%%%%%%%%%%%%%%%%%%%%%%%%%%%%
\paragraph{v2.0:} 2018/12/30

\begin{itemize}
\item
immediate forward processing
\item
added |\childdocby| mechanism
\item
manual restructured
\end{itemize}

%%%%%%%%%%%%%%%%%%%%%%%%%%%%%%%%%%%%%%%%
\paragraph{v1.6:} 2018/01/17

\begin{itemize}
\item
application for development of include files
\item
corrections to manual
\end{itemize}

%%%%%%%%%%%%%%%%%%%%%%%%%%%%%%%%%%%%%%%%
\paragraph{v1.5:} 2017/05/21

\begin{itemize}
\item
more complete structuring introduced
\item
|\childdocof| introduced
\item
|\childdoc| renamed to |\childdocmain|
\item
|\childredirect| renamed to |\childdocforward| and |\childdocforwardprefix|
and functionality expanded
\end{itemize}

%%%%%%%%%%%%%%%%%%%%%%%%%%%%%%%%%%%%%%%%
\paragraph{v1.0:} 2017/04/27

\begin{itemize}
\item
manual and install package
\item
first version published on CTAN
\end{itemize}

%%%%%%%%%%%%%%%%%%%%%%%%%%%%%%%%%%%%%%%%
\paragraph{v0.6:} 2017/04/26

\begin{itemize}
\item
redirection mechanism added
\end{itemize}

%%%%%%%%%%%%%%%%%%%%%%%%%%%%%%%%%%%%%%%%
\paragraph{v0.5:} 2017/04/26

\begin{itemize}
\item
functionality in definition file
\end{itemize}


%%%%%%%%%%%%%%%%%%%%%%%%%%%%%%%%%%%%%%%%%%%%%%%%%%%%%%%%%%%%%%%%%%%%%%%%%%%%%%%%
%%%%%%%%%%%%%%%%%%%%%%%%%%%%%%%%%%%%%%%%%%%%%%%%%%%%%%%%%%%%%%%%%%%%%%%%%%%%%%%%
%%%%%%%%%%%%%%%%%%%%%%%%%%%%%%%%%%%%%%%%%%%%%%%%%%%%%%%%%%%%%%%%%%%%%%%%%%%%%%%%
\appendix

\settowidth\MacroIndent{\rmfamily\scriptsize 000\ }

 \DocInput{childdoc.dtx}

\end{document}
%</driver>
% \fi
%
% %%%%%%%%%%%%%%%%%%%%%%%%%%%%%%%%%%%%%%%%%%%%%%%%%%%%%%%%%%%%%%%%%%%%%%%%%%%%%%
% %%%%%%%%%%%%%%%%%%%%%%%%%%%%%%%%%%%%%%%%%%%%%%%%%%%%%%%%%%%%%%%%%%%%%%%%%%%%%%
% \section{Sample}
%\iffalse
%<*samplemain>
%\fi
%
% The following presents a sample document
% with two chapters, two parts, a title page,
% a compile flag as well as three forwarding files to set the flag.
% It consists of eight |.tex| files:
% \begin{center}
% \begin{tabular}{ll}
% |cdocsamp.tex|&main file\\
% |cdocsch1.tex|&include file for chapter 1\\
% |cdocsch2.tex|&include file for chapter 2\\
% |cdocspt3.tex|&include file for part 3\\
% |cdocspt4.tex|&include file for part 4\\
% |cdocsdrf.tex|&forwarding file for main file in draft mode\\
% |cdocsfi1.tex|&forwarding file for final version of chapter 1\\
% |cdocsfi2.tex|&forwarding file for final version of chapter 2\\
% \end{tabular}
% \end{center}
% Each of the eight files can be compiled directly by the \LaTeX{} compiler.
%
% %%%%%%%%%%%%%%%%%%%%%%%%%%%%%%%%%%%%%%
% \paragraph{Main File.}
%
% The main file is called |cdocsamp.tex|.
%
% Load the \textsf{childdoc} definitions and
% declare the filename for the main document:
%    \begin{macrocode}
\input{childdoc.def}
\childdocmain{}
%    \end{macrocode}

% Optional override for |\version| flag:
%    \begin{macrocode}
%%\ifchilddoc\else\providecommand{\version}{draft}\fi
%    \end{macrocode}

% Define the default values for the |\version| flag
% (|final| for the main file and |draft| for childs):
%    \begin{macrocode}
\ifchilddoc
\providecommand{\version}{draft}
\else
\providecommand{\version}{final}
\fi
%    \end{macrocode}

% Load the standard document class:
%    \begin{macrocode}
\documentclass[12pt]{article}
%    \end{macrocode}

% Start the document body:
%    \begin{macrocode}
\begin{document}
%    \end{macrocode}

% Declare a title page.
% Print title, part of document being processed and version flag:
%    \begin{macrocode}
\addtocounter{page}{-1}
\begin{center}
{\LARGE\bfseries{}childdoc example\par}
\vspace{1cm}
\ifchilddoc
\ifchilddocmanual part\else chapter\fi:
`\childdocname' of `\childdocjob'\par
\else
main document: `\childdocjob'\par
\fi
version: \version\par
\end{center}
\newpage
%    \end{macrocode}

% Manually include selected file,
% otherwise process as usual:
%    \begin{macrocode}
\ifchilddocmanual
\section*{part `\childdocname'}
\input{\childdocname}
\else
%    \end{macrocode}

% Include the two chapters:
%    \begin{macrocode}
\include{cdocsch1}
\include{cdocsch2}
%    \end{macrocode}

% Include the two parts unless only chapters should be displayed:
%    \begin{macrocode}
\ifchilddoc\else
\section{part three}
\input{cdocspt3}
\section{part four}
\input{cdocspt4}
\fi
%    \end{macrocode}

% Process as usual until here:
%    \begin{macrocode}
\fi
%    \end{macrocode}

% End of document body:
%    \begin{macrocode}
\end{document}
%    \end{macrocode}
%\iffalse
%</samplemain>
%\fi
%
% %%%%%%%%%%%%%%%%%%%%%%%%%%%%%%%%%%%%%%
% \paragraph{Chapter Include Files.}
%
% The include files are called |cdocsch1.tex| and |cdocsch2.tex|.
%
%\iffalse
%<*samplechap1|samplechap2>
%\fi

% Optional override for |\version| flag:
%    \begin{macrocode}
%%\providecommand{\version}{final}
%    \end{macrocode}

% Include the main document:
%    \begin{macrocode}
\input{childdoc.def}
\childdocof{cdocsamp}
%    \end{macrocode}

%\iffalse
%</samplechap1|samplechap2>
%\fi
%
%\iffalse
%<*samplechap1>
%\fi
% Some text for chapter 1:
%    \begin{macrocode}
\section{one}
some text in chapter one
%    \end{macrocode}

%\iffalse
%</samplechap1>
%\fi
% Some text for chapter 2:
%\iffalse
%<*samplechap2>
%\fi
%    \begin{macrocode}
\section{two}
more text in chapter two
%    \end{macrocode}

%\iffalse
%</samplechap2>
%\fi
%
% %%%%%%%%%%%%%%%%%%%%%%%%%%%%%%%%%%%%%%
% \paragraph{Part Include Files.}
%
% The include files are called |cdocspt3.tex| and |cdocspt4.tex|.
%
%\iffalse
%<*samplepart3|samplepart4>
%\fi

% Optional override for |\version| flag:
%    \begin{macrocode}
%%\providecommand{\version}{final}
%    \end{macrocode}

% Include the main document:
%    \begin{macrocode}
\input{childdoc.def}
\childdocby{cdocsamp}
%    \end{macrocode}

%\iffalse
%</samplepart3|samplepart4>
%\fi
%
%\iffalse
%<*samplepart3>
%\fi
% Some text for part 3:
%    \begin{macrocode}
some text in part three
%    \end{macrocode}

%\iffalse
%</samplepart3>
%\fi
% Some text for part 4:
%\iffalse
%<*samplepart4>
%\fi
%    \begin{macrocode}
more text in part four
%    \end{macrocode}

%\iffalse
%</samplepart4>
%\fi
%
% %%%%%%%%%%%%%%%%%%%%%%%%%%%%%%%%%%%%%%
% \paragraph{Forwarding for a Complete Draft.}
%
% The following forwarding file |cdocsdrf.tex|
% compiles the main document in draft mode:
%\iffalse
%<*sampledraft>
%\fi
%    \begin{macrocode}
\def\version{draft}
\input{childdoc.def}
\childdocforward{cdocsamp}
%    \end{macrocode}

%\iffalse
%</sampledraft>
%\fi
%
% %%%%%%%%%%%%%%%%%%%%%%%%%%%%%%%%%%%%%%
% \paragraph{Forwarding for Final Version of the Chapters.}
%
% The following forwarding files |cdocsfn1.tex| and |cdocsfn2.tex|
% (with identical content)
% compile the final versions of the child documents
% |cdocsch1.tex| and |cdocsch2.tex|, respectively:
%\iffalse
%<*samplefinal>
%\fi
%    \begin{macrocode}
\def\version{final}
\input{childdoc.def}
\childdocforwardprefix[cdocsamp]{cdocsfn}{cdocsch}
%    \end{macrocode}

%\iffalse
%</samplefinal>
%\fi
%
% %%%%%%%%%%%%%%%%%%%%%%%%%%%%%%%%%%%%%%
% \paragraph{Command Line Processing.}
%
% The following three command lines generate the output files
% |cdocscld|, |cdocscl1| and |cdocscl2|
% which should be identical to
% |cdocsdrf|, |cdocsch1| and |cdocsfn2|, respectively:
% \begin{center}
% \begin{tabular}{l}
% |latex -jobname cdocscld \|\\
% |  "\def\version{draft}\input{childdoc.def}\childdocforward{cdocsamp}"|\\
% |latex -jobname cdocscl1 \|\\
% |  "\input{childdoc.def}\childdocforward[cdocsamp]{cdocsch1}"|\\
% |latex -jobname cdocscl2 \|\\
% |  "\def\version{final}\input{childdoc.def}\childdocforward{cdocsch2}"|
% \end{tabular}
% \end{center}
% Note that the trailing backslash on each first line
% merely continues the input to the second line
% (for convenient cut ant paste).
% Furthermore, the command |latex| can be replaced by any
% of its alternative versions such as |pdflatex|.
%
% %%%%%%%%%%%%%%%%%%%%%%%%%%%%%%%%%%%%%%%%%%%%%%%%%%%%%%%%%%%%%%%%%%%%%%%%%%%%%%
% %%%%%%%%%%%%%%%%%%%%%%%%%%%%%%%%%%%%%%%%%%%%%%%%%%%%%%%%%%%%%%%%%%%%%%%%%%%%%%
% \section{Implementation}
%\iffalse
%<*package>
%\fi
%
% This section describes the definitions file |childdoc.def|.

% The definitions cannot be loaded using |\usepackage| or |\RequirePackage|
% which has a mechanism to prevent loading a style file more than once.
% When loading the definitions by means of |\input|
% multiple instances have to be prevented manually:
%\iffalse
%This code needs to be before the `\ProvidesFile' directive
%which is defined at the beginning of this file.
%Therefore it is also placed there and commented out here.
%</package>
%<*discard>
%\fi
%    \begin{macrocode}
\ifdefined\childdocmain\endinput\fi
%    \end{macrocode}
%\iffalse
%</discard>
%<*package>
%\fi
%
% \macro{\ifchilddoc}
% \macro{\ifchilddocmanual}
% The conditional |\ifchilddoc| tells whether a
% child (true) or main (false) document is being compiled.
% The conditional |\ifchilddocmanual| tells whether
% the |\includeonly| mechanism is used (false) or
% the selection of child files must be performed manually (true).
% The definitions initialise to false:
%    \begin{macrocode}
\newif\ifchilddoc
\newif\ifchilddocmanual
%    \end{macrocode}

% \macro{\childdocname}
% \macro{\childdocjob}
% The macro |\childdocname| stores the name of the main document
% to be compiled. The macro |\childdocjob| stores the name of
% the document on which the \LaTeX{} compiler was originally invoked.
% The content of |\jobname| cannot be compared
% to filenames specified in the source due to different catcodes.
% The following code rescans |\jobname|, stores the result
% in |\childdocname| and saves a copy in |\childdocjob|:
%    \begin{macrocode}
\edef\childdocname{\scantokens\expandafter{\jobname\noexpand}}
\let\childdocjob\childdocname
%    \end{macrocode}

% \macro{\childdocdisable}
% The macro |\childdocdisable| prevents the main file
% from being processed more than once.
% At this stage, the main document command |\childdocmain|
% is assumed to be called once again where it should do nothing.
% Any subsequent call to it should prevent
% a secondary processing of the main document
% It overwrites the forwarding commands
% |\childdocof| and |\childdocforward|
% with empty macros to prevent further inclusions of the main document:
%    \begin{macrocode}
\newcommand{\childdocdisable}
{
  \renewcommand{\childdocmain}[1]{\renewcommand{\childdocmain}[1]{\endinput}}
  \renewcommand{\childdocof}[1]{}
  \renewcommand{\childdocby}[2][]{}
  \renewcommand{\childdocforward}[2][]{}
  \renewcommand{\childdocdisable}{}
}
%    \end{macrocode}

% \macro{\childdocmain}
% The macro |\childdocmain| is to be called at the top of the main file
% with nothing or the main filename (without extension) as argument.
% First, it breaks loops.
% If the argument is not empty and does not match |\childdocname|
% (which is set by the first inclusion of |childdoc.def|),
% |\ifchilddoc| is set to true, |\includeonly| is applied to the child file
% and |\jobname| is set to the main file
% (for proper handling of |.aux| files):
%    \begin{macrocode}
\newcommand{\childdocmain}[1]
{
  \childdocdisable\childdocmain{}
  \if?#1?\else
    \begingroup
      \def\childdoctmp{#1}
      \ifx\childdoctmp\childdocname
        \def\childdoctmp{}
      \else
        \def\childdoctmp
        {
          \childdoctrue
          \includeonly{\childdocname}
          \def\childdocjob{#1}
          \def\jobname{#1}
        }
      \fi
      \expandafter
    \endgroup
    \childdoctmp
  \fi
}
%    \end{macrocode}

% \macro{\childdocof}
% The command |\childdocof| redirects
% compilation to the main file |#1|.
%    \begin{macrocode}
\newcommand{\childdocof}[1]
{
  \childdocdisable
  \childdoctrue
  \includeonly{\childdocname}
  \def\jobname{#1}
  \def\childdocjob{#1}
  \input{#1}
}
%    \end{macrocode}

% \macro{\childdocby}
% The command |\childdocby| ....
%    \begin{macrocode}
\newcommand{\childdocby}[2][]
{
  \childdocdisable
  \childdoctrue
  \childdocmanualtrue
  \if?#1?\else
    \def\jobname{#2}
  \fi
  \def\childdocjob{#2}
  \input{#2}
  \endinput
}
%    \end{macrocode}

% \macro{\childdocforward}
% The command |\childdocforward| redirects
% compilation to the main file or
% (if the optional argument is given) a child file.
% Parameters are set as if the main file
% or a child file starting with |\childdocof| was compiled.
% Then compilation is handed over to the main file:
%    \begin{macrocode}
\newcommand{\childdocforward}[2][]
{
  \begingroup
    \if?#1?
      \def\childdoctmp
      {
        \def\childdocname{#2}
        \def\childdocjob{#2}
        \def\jobname{#2}
        \input{#2}
        \endinput
      }
    \else
      \def\childdoctmp
      {
        \childdocdisable
        \def\childdocname{#2}
        \childdoctrue
        \includeonly{#2}
        \def\childdocjob{#1}
        \def\jobname{#1}
        \input{#1}
        \endinput
      }
    \fi
    \expandafter
  \endgroup
  \childdoctmp
}
%    \end{macrocode}

% \macro{\childdocforwardprefix}
% The command |\childdocforwardprefix| redirects
% compilation to the main or a child file by means of a pattern.
% The prefix |#1| in the current filename is replaced by |#2|
% and the suffix of the current filename is kept
% (it is assumed that the filename does not contain the substring `|~~~|'
% which is used as a delimiter).
% Compilation is handed over to the new file by |\childdocforward|:
%    \begin{macrocode}
\newcommand{\childdocforwardprefix}[3][]
{
  \begingroup
    \def\childdocextract #2##1~~~{\def\childdoctmp{\childdocforward[#1]{#3##1}}}
    \expandafter\childdocextract\childdocname~~~
    \expandafter
  \endgroup
  \childdoctmp
}
%    \end{macrocode}

% \macro{\childdoc}
% The deprecated macro |\childdoc| is a legacy version of |\childdocmain|:
%    \begin{macrocode}
\newcommand{\childdoc}{\childdocmain}
%    \end{macrocode}

% \macro{\childdocredirect}
% The deprecated macro |\childdocredirect| is a legacy version
% of |\childdocforward| and |\childdocforwardprefix|:
%    \begin{macrocode}
\newcommand{\childdocredirect}[2][]
{
  \begingroup
    \if?#1?
      \def\childdoctmp{\childdocforward{#2}}
    \else
      \def\childdoctmp{\childdocforwardprefix{#1}{#2}}
    \fi
    \expandafter
  \endgroup
  \childdoctmp
}
%    \end{macrocode}

%\iffalse
%</package>
%\fi
%
\endinput
|\\
|\childdocby{|\textit{main}|}|\\
\end{tabular}
\end{center}
%
Both forms have slightly different effects as described above.
The main file is prepared as usual, see \secref{sec:include}.

%%%%%%%%%%%%%%%%%%%%%%%%%%%%%%%%%%%%%%%%%%%%%%%%%%%%%%%%%%%%%%%%%%%%%%%%%%%%%%%%
\subsection{Legacy Detection}
\label{sec:detection}

The directive |\childdocmain| in the main file can detect
whether the complete document or merely a child is to be compiled
even without using the directive |\childdocof|.
This method is deprecated because it is less robust
and there is no compelling reason to use it;
it is merely provided for backward compatibility
and it may be removed in future versions.

If the detection mechanism is to be used,
it is mandatory to correctly specify
the filename of the main file as the argument of |\childdocmain|:
%
\begin{center}
\begin{tabular}{l}
|% \iffalse
%
% childdoc.dtx Copyright (C) 2017-2018 Niklas Beisert
%
% This work may be distributed and/or modified under the
% conditions of the LaTeX Project Public License, either version 1.3
% of this license or (at your option) any later version.
% The latest version of this license is in
%   http://www.latex-project.org/lppl.txt
% and version 1.3 or later is part of all distributions of LaTeX
% version 2005/12/01 or later.
%
% This work has the LPPL maintenance status `maintained'.
%
% The Current Maintainer of this work is Niklas Beisert.
%
% This work consists of the files childdoc.dtx and childdoc.ins
% and the derived files childdoc.def and cdocsamp.tex with
% cdocsch1.tex, cdocsch2.tex, cdocsdrf.tex, cdocsfn1.tex, cdocsfn2.tex.
%
%<package>\ifdefined\childdocmain\endinput\fi
%<package>\ProvidesFile{childdoc.def}[2018/12/30 v2.0 child document driver]
%<samplemain>\ProvidesFile{cdocsamp.tex}[2018/12/30 v2.0 sample for childdoc]
%<*driver>
%\ProvidesFile{childdoc.drv}[2018/12/30 v2.0 childdoc reference manual file]
\PassOptionsToClass{10pt,a4paper}{article}
\documentclass{ltxdoc}

\usepackage[margin=35mm]{geometry}
\usepackage{hyperref}
\usepackage{hyperxmp}
\usepackage[usenames]{color}

\hypersetup{colorlinks=true}
\hypersetup{pdfstartview=FitH}
\hypersetup{pdfpagemode=UseNone}
\hypersetup{pdfsource={}}
\hypersetup{pdflang={en-UK}}
\hypersetup{pdfcopyright={Copyright 2017-2018 Niklas Beisert.
  This work may be distributed and/or modified under the
  conditions of the LaTeX Project Public License, either version 1.3
  of this license or (at your option) any later version.}}
\hypersetup{pdflicenseurl={http://www.latex-project.org/lppl.txt}}
\hypersetup{pdfcontactaddress={ETH Zurich, ITP, HIT K,
  Wolfgang-Pauli-Strasse 27}}
\hypersetup{pdfcontactpostcode={8093}}
\hypersetup{pdfcontactcity={Zurich}}
\hypersetup{pdfcontactcountry={Switzerland}}
\hypersetup{pdfcontactemail={nbeisert@itp.phys.ethz.ch}}
\hypersetup{pdfcontacturl={http://people.phys.ethz.ch/\xmptilde nbeisert/}}

\newcommand{\secref}[1]{\hyperref[#1]{section \ref*{#1}}}

\parskip1ex
\parindent0pt
\let\olditemize\itemize
\def\itemize{\olditemize\parskip0pt}

\begin{document}

\title{The \textsf{childdoc} Package}
\hypersetup{pdftitle={The childdoc Package}}
\author{Niklas Beisert\\[2ex]
  Institut f\"ur Theoretische Physik\\
  Eidgen\"ossische Technische Hochschule Z\"urich\\
  Wolfgang-Pauli-Strasse 27, 8093 Z\"urich, Switzerland\\[1ex]
  \href{mailto:nbeisert@itp.phys.ethz.ch}
  {\texttt{nbeisert@itp.phys.ethz.ch}}}
\hypersetup{pdfauthor={Niklas Beisert}}
\hypersetup{pdfsubject={Manual for the LaTeX2e Package childdoc}}
\date{30 December 2018, \textsf{v2.0}}
\maketitle

\begin{abstract}\noindent
\textsf{childdoc} is a \LaTeXe{} package
that enables the direct compilation
of document sections included by |\include|
to individual files.
\end{abstract}

\begingroup
\parskip0ex
\tableofcontents
\endgroup

%%%%%%%%%%%%%%%%%%%%%%%%%%%%%%%%%%%%%%%%%%%%%%%%%%%%%%%%%%%%%%%%%%%%%%%%%%%%%%%%
%%%%%%%%%%%%%%%%%%%%%%%%%%%%%%%%%%%%%%%%%%%%%%%%%%%%%%%%%%%%%%%%%%%%%%%%%%%%%%%%
\section{Introduction}

\LaTeX{} provides a mechanism to structure a large document (such as a book)
into a main file and several child files (containing the chapters)
using the |\include| command.
This mechanism is beneficial for documents
which span hundreds of pages in order to
make the source file(s) more manageable.
Moreover, compilation can be restricted to
selected child files by means of the |\includeonly| command.
The latter feature can be used to reduce the compilation time while editing
(this was significantly more useful in the earlier days of \LaTeX{})
or to generate a smaller document which is easier to navigate.
Another application of |\includeonly| is to generate
documents consisting of selected parts of the complete document.

However, there are a few drawbacks of the plain |\include| mechanism:
\begin{itemize}
\item
The child files cannot be compiled on their own,
they can only be compiled via the main file.
A naive editing environment
(such as a text editor with an option
to have the current file processed by \LaTeX)
may require one to switch to the main file before compiling;
attempting to compile the child file produces errors.
\item
The main file must be modified (each time)
to adjust the |\includeonly| command
to the present needs. This easily leaves the main file in a messy state.
\item
The generated document will always carry the filename
of the main document. This is inconvenient if
several child files are to be compiled and
to be kept for distribution.
\end{itemize}

The present package provides a simple interface
to make child files individually compilable by \LaTeX{}.
Compiling a child file then has the same effect as compiling
the main file with an |\includeonly| command
to select the appropriate child.
Moreover the generated document will carry the name of the child
rather than the main file.
This resolves all three above issues.

This feature is meant to make the editing of books,
thesis documents and lecture notes somewhat more convenient.
However, the package can also be used efficiently for
composing a series of documents (such as exercise sheets)
which are typically distributed individually.
It then assists the author in generating the individual documents
(potentially in different versions)
as well as a document containing the collected series.
Another application is in developing style files
or other kinds of included material
where compilation of the style file could redirect
to a sample or test file.

%%%%%%%%%%%%%%%%%%%%%%%%%%%%%%%%%%%%%%%%%%%%%%%%%%%%%%%%%%%%%%%%%%%%%%%%%%%%%%%%
%%%%%%%%%%%%%%%%%%%%%%%%%%%%%%%%%%%%%%%%%%%%%%%%%%%%%%%%%%%%%%%%%%%%%%%%%%%%%%%%
\section{Usage}

First of all, the package \textsf{childdoc} is \emph{not} a standard
\LaTeXe{} |.sty| style file! Therefore it needs to be invoked in
a non-standard way.

%%%%%%%%%%%%%%%%%%%%%%%%%%%%%%%%%%%%%%%%%%%%%%%%%%%%%%%%%%%%%%%%%%%%%%%%%%%%%%%%
\subsection{Included Files}
\label{sec:include}

%%%%%%%%%%%%%%%%%%%%%%%%%%%%%%%%%%%%%%%%
\DescribeMacro{\childdocmain}
To use the package, add the commands
\begin{center}
\begin{tabular}{l}
|\input{childdoc.def}|\\
|\childdocmain{}|\\
\end{tabular}
\end{center}
at the very top of the main \LaTeX{} file,
in particular \emph{before} the |\documentclass| statement!
The argument of |\childdocmain| should be left empty
(but it must be present).

%%%%%%%%%%%%%%%%%%%%%%%%%%%%%%%%%%%%%%%%
\DescribeMacro{\childdocof}
Furthermore, add the commands
\begin{center}
\begin{tabular}{l}
|\input{childdoc.def}|\\
|\childdocof{|\textit{main}|}|\\
\end{tabular}
\end{center}
at the top of every child file \textit{child}
which is included by |\include{|\textit{child}|}|
from within the main file
(or at least for those files to be compiled individually).
The argument \textit{main} must be the filename of the main file.

There are a couple of
considerations in setting up the main and child documents:

%%%%%%%%%%%%%%%%%%%%%%%%%%%%%%%%%%%%%%%%
\paragraph{Restrictions.}

Please note the following restrictions:
\begin{itemize}
\item
|\childdocmain| must be called with one argument \textit{main}
to ensure compatibility with earlier version of the package.
It must either be empty (|\childdocmain{}|)
or precisely match the filename of the main file in which it is specified.
See \secref{sec:detection} for further information.
\item
The filename \textit{main} must be specified without the |.tex| extension.
\item
The filename \textit{main} is case sensitive
(even in case-insensitive file systems)
due to internal string comparison.
\item
The argument \textit{main} should be fully expanded, it cannot be a macro.
\item
Subdirectories and special characters should be avoided in filenames.
\item
The command |\childdocmain{|\textit{main}|}| must be followed by a whitespace.
It should not be followed immediately by another command
or by a comment mark `|%|'.
This is because the \TeX{} parser reads the token immediately following
the argument of |\childdocmain| and puts it
at the beginning of every child section;
however, a white\-space is ignored.
\end{itemize}

%%%%%%%%%%%%%%%%%%%%%%%%%%%%%%%%%%%%%%%%
\paragraph{Content of Main File.}

It is advisable to place all content in the child files included by |\include|.
Any output contained in the main file will appear in all child documents
unless suppressed manually;
it cannot be suppressed automatically by the |\includeonly| directive
and thus should normally be avoided.
A method to include some content in the main file
by means of conditional processing is described in \secref{sec:conditional}.

%%%%%%%%%%%%%%%%%%%%%%%%%%%%%%%%%%%%%%%%
\paragraph{Page Numbering.}

When only a part of the document is compiled,
the appropriate numbering of pages
(as well as other status parameters)
is determined from the |.aux| files.
The latter contain information from previous passes.
However this information needs to propagate through
all intermediate child documents.
Therefore the page numbering in child documents may well
be inconsistent until the complete document is compiled at least once.

A useful (if unconventional) way to always ensure a consistent
page numbering is to restart the numbering in each child document
and denote the pages by `\textit{child}|.|\textit{page}'
where \textit{child} represents the chapter/section number of the child file.
This can be achieved by the command
|\numberwithin{page}{|\textit{child}|}|
of the \textsf{amsmath} package
where \textit{child} can be |chapter| or |section|
depending on the chosen structuring.
Alternatively, one can modify the macro |\thepage| appropriately
and reset the counter |page| at the start of each child file.

%%%%%%%%%%%%%%%%%%%%%%%%%%%%%%%%%%%%%%%%%%%%%%%%%%%%%%%%%%%%%%%%%%%%%%%%%%%%%%%%
\subsection{Conditional Processing}
\label{sec:conditional}

The package provides a mechanism to compile different versions
of a document. To customise the versions further some conditional processing
can come in handy to distinguish which version is being compiled.
The package provides two macros to describe the compilation context:

%%%%%%%%%%%%%%%%%%%%%%%%%%%%%%%%%%%%%%%%
\DescribeMacro{\ifchilddoc}
The conditional |\ifchilddoc| distinguishes between the compilation of
child documents and the main document:
%
\begin{center}
|\ifchilddoc |\textit{child-code}| |[|\||else |\textit{main-code}]| \||fi|
\end{center}

%%%%%%%%%%%%%%%%%%%%%%%%%%%%%%%%%%%%%%%%
\DescribeMacro{\childdocname}
\DescribeMacro{\childdocjob}
The macro |\childdocname| contains the filename (without extension)
of the main or child file being processed.
Note that |\childdocjob| will always contain the name of the main file.

%%%%%%%%%%%%%%%%%%%%%%%%%%%%%%%%%%%%%%%%
\paragraph{Title Page.}

Conditional processing can be used to include a title or banner page
in the main document when proper precautions are taken.
Importantly, the code in the main file should ensure that the page counter
(as well as other status parameters which are stored in the |.aux| files)
takes the same value after the conditional processing.
Otherwise the page numbers may take divergent values
depending on which part is compiled.

For example, a title page could be declared by:
%
\begin{center}
\begin{tabular}{l}
|\ifchilddoc\||else|\\
|\addtocounter{page}{-1}|\\
\textit{code for title page}\\
|\newpage|\\
|\||fi|
\end{tabular}
\end{center}
%
A banner page for the child documents can be generated by:
%
\begin{center}
\begin{tabular}{l}
|\ifchilddoc|\\
|\addtocounter{page}{-1}|\\
\textit{code for banner page}\\
|\newpage|\\
|\||fi|
\end{tabular}
\end{center}
%
Here one could write a message such as:
\begin{center}
|This is the part \childdocname{} of \childdocjob{}.|
\end{center}

%%%%%%%%%%%%%%%%%%%%%%%%%%%%%%%%%%%%%%%%%%%%%%%%%%%%%%%%%%%%%%%%%%%%%%%%%%%%%%%%
\subsection{Flags}
\label{sec:flags}

The package makes it easy to generate different versions
of the main or child documents.
To this end compilation flags can be defined
and assigned different default values.
They will be particularly useful in conjunction
with the forwarding mechanism described in \secref{sec:forward}.

For example, it may be useful to have a flag |\version|
which can be set to |draft| or |final|.
The document source will contain some conditional code
depending on the value of |\version|.
Suppose further, the flag should default to |final| for the main file
and to |draft| for child files
which is a natural assignment for editing the document.
This is achieved by placing the following code
in the preamble of the main document
(below the |\childdocmain| directive):
%
\begin{center}
\begin{tabular}{l}
|\ifchilddoc|\\
|\providecommand{\version}{draft}|\\
|\||else|\\
|\providecommand{\version}{final}|\\
|\||fi|
\end{tabular}
\end{center}
%
The definition by |\providecommand| makes sure
that previous definitions are not overwritten.
Further statements |\providecommand{\version}{...}|
can thus be added before the above code to override it.

For the main file, one might add a line
(between |\childdocmain| and the above block)
%
\begin{center}
|%\ifchilddoc\||else\providecommand{\version}{draft}\||fi|
\end{center}
%
which can be uncommented to produce a draft version.
Likewise one can add a line to the very top of a child file
(above the |\childdocof{|\textit{main}|}| directive)
%
\begin{center}
|%\providecommand{\version}{final}|
\end{center}
%
which can be uncommented to produce the final version of this child document.

%%%%%%%%%%%%%%%%%%%%%%%%%%%%%%%%%%%%%%%%%%%%%%%%%%%%%%%%%%%%%%%%%%%%%%%%%%%%%%%%
\subsection{Forwarding}
\label{sec:forward}

Different versions of the main or child documents
using compilation flags as described in \secref{sec:flags}
can be (permanently) stored in different files
for convenient compilation, viewing and distribution.
To this end, the package defines a command
to pass on compilation to a different file:

%%%%%%%%%%%%%%%%%%%%%%%%%%%%%%%%%%%%%%%%
\DescribeMacro{\childdocforward}
The command |\childdocforward| redirects processing to
another source file:
%
\begin{center}
\begin{tabular}{l}
|\input{childdoc.def}|\\
|\childdocforward[|\textit{main}|]{|\textit{dest}|}|\\
\end{tabular}
\end{center}
%
The argument \textit{dest} is the destination file
(without extension).
It should be the main file or one of the child files.
Note that further \textsf{childdoc} directives
such as |\childdocof| and |\childdocforward|
in the indicated file will be processed in this form.
The optional argument \textit{main}
passes on directly to the main file \textit{main}
while pretending to compile the child \textit{dest}.
This form behaves as if \textit{dest}
issues |\childdocof{|\textit{main}|}| right away,
and no further \textsf{childdoc} directives will be processed.

%%%%%%%%%%%%%%%%%%%%%%%%%%%%%%%%%%%%%%%%
\DescribeMacro{\...prefix}
In the alternative form |\childdocforwardprefix|,
%
\begin{center}
\begin{tabular}{l}
|\input{childdoc.def}|\\
|\childdocforwardprefix[|\textit{main}|]{|\textit{prefix}|}{|\textit{dest}|}|
\end{tabular}
\end{center}
%
the destination file is determined by a pattern
depending on the current file:
To make this work, the current file must be called
`{\textit{prefix}\hspace{0.2em}\textit{suffix}}'
with \textit{prefix} matching precisely the argument.
Processing is then passed on to the file
`{\textit{dest}\hspace{0.2em}\textit{suffix}}'.
Surely, the same effect is achieved by
directly specifying the
argument `{\textit{dest}\hspace{0.2em}\textit{suffix}}'
in the first form.
However, that requires to set up a different file
for each child. With the alternative form of the command
all these files can have exactly the same content
which simplifies setting them up and maintaining them.

For example, the following file |draft.tex|
with a compilation flag |\version| as described in \secref{sec:flags}
compiles the main document as a draft:
%
\begin{center}
\begin{tabular}{l}
|\def\version{draft}|\\
|\input{childdoc.def}|\\
|\childdocforward{|\textit{main}|}|
\end{tabular}
\end{center}
%
Likewise, the following files |final|\textit{nn}|.tex|
compile the final version of the child document
|child|\textit{nn}|.tex|:
%
\begin{center}
\begin{tabular}{l}
|\def\version{final}|\\
|\input{childdoc.def}|\\
|\childdocforwardprefix{final}{child}|
\end{tabular}
\end{center}
%

Note that when several versions of a main file and/or of each child file
are to be generated, it may be convenient to set up a |Makefile| or
shell script to automatise the process.

%%%%%%%%%%%%%%%%%%%%%%%%%%%%%%%%%%%%%%%%%%%%%%%%%%%%%%%%%%%%%%%%%%%%%%%%%%%%%%%%
\subsection{Command Line Processing}
\label{sec:commandline}

The effect of redirection files can also be achieved by invoking
the \LaTeX{} compiler with a more elaborate command line.
Most conveniently this should be done as part
of a shell script or a |Makefile|.

When using \textsf{childdoc} in the main file, the following
command lines effectively perform a redirection
(note that depending on the shell being used,
backslashes may have to be doubled: `|\|' $\to$ `|\\|'):
%
\begin{center}
|... -jobname "|\textit{target}|" |\\|"|[\textit{flags}]%
|\input{childdoc.def}\childdocforward[|\textit{main}|]{|\textit{dest}|}"|
\end{center}
%
Here \textit{target} is the name of the output file,
\textit{main} is the name of the main file
and \textit{dest} is the name of the main or child file to be processed
(all filenames without extensions).
The optional argument \textit{main} can be omitted
if \textit{main} matches \textit{dest}.
Optionally, compilation \textit{flags} can be defined via |\def| commands.
This command line makes the \TeX{} engine believe
it is compiling the file \textit{target}
whose content is specified as the latter parameter.
The provided code then forwards the processing to
\textit{main} or \textit{dest} as described in \secref{sec:forward}.

%%%%%%%%%%%%%%%%%%%%%%%%%%%%%%%%%%%%%%%%%%%%%%%%%%%%%%%%%%%%%%%%%%%%%%%%%%%%%%%%
\subsection{Include by Input}
\label{sec:input}

Including child documents by |\include| has some restrictions by design.
Most notably, the content of a child document always occupies
its own set of pages; pages cannot be shared between child documents.
Usually, this behaviour makes perfect sense
because each child document contain an essential part of the document.
However, in some situations it may be desirable to compose
a document from a collection of parts
without having mandatory page breaks between then.
For this case, the package
provides a mechanism to include parts
by |\input| which can also be processed individually.
However, by construction this mechanism
requires manual handling of the content to be output.

%%%%%%%%%%%%%%%%%%%%%%%%%%%%%%%%%%%%%%%%
\DescribeMacro{\ifchilddocmanual}
The main file should be prepared as usual, see \secref{sec:include}.
However, the document body must make a distinction
between processing of an individual part and of the main document, e.g.:
%
\begin{center}
\begin{tabular}{l}
|\ifchilddocmanual|\\
|\input{\childdocname}|\\
|\||else|\\
\textit{document body with }|\input{|\textit{part}|}|\\
|\||fi|
\end{tabular}
\end{center}
%
The conditional |\ifchilddocmanual| is true whenever
a part to be included by |\input| is being compiled,
and the name of the part is stored in |\childdocname|.

%%%%%%%%%%%%%%%%%%%%%%%%%%%%%%%%%%%%%%%%
\DescribeMacro{\childdocby}
Each part to be included by |\input| should start with:
%
\begin{center}
\begin{tabular}{l}
|\input{childdoc.def}|\\
|\childdocby{|\textit{main}|}|\\
\end{tabular}
\end{center}
%
The directive |\childdocby| is similar to |\childdocof|
described in \secref{sec:include},
but the subsequent selection of content must be done manually.
To that end, both |\ifchilddoc| and |\ifchilddocmanual|
will be true upon processing of a part,
and the name of the part is stored in |\childdocname|.
Note that |\jobname| will be set to the filename of the current part
so that each part receives an individual |.aux| file
that does not interfere with the |.aux| file(s) of the main document.
This behaviour can be altered by the alternative form
|\childdocby[*]{|\textit{main}|}| (with a non-empty optional argument)
which uses the |.aux| file of the main document
by setting |\jobname| to \textit{main}.

%%%%%%%%%%%%%%%%%%%%%%%%%%%%%%%%%%%%%%%%%%%%%%%%%%%%%%%%%%%%%%%%%%%%%%%%%%%%%%%%
\subsection{Driver Development}
\label{sec:driver}

The \textsf{childdoc} mechanism can also be use for the development
of definition files such as \LaTeX{} styles or classes.
This case differs from the above setup with multiple parts
included by |\include| in that no |\includeonly| should be invoked.
This can be achieved by starting the include file
(before |\ProvidesPackage|) with:
%
\begin{center}
\begin{tabular}{l}
|\input{childdoc.def}|\\
|\childdocforward{|\textit{main}|}|\\
\end{tabular}
\end{center}
%
or alternatively with:
%
\begin{center}
\begin{tabular}{l}
|\input{childdoc.def}|\\
|\childdocby{|\textit{main}|}|\\
\end{tabular}
\end{center}
%
Both forms have slightly different effects as described above.
The main file is prepared as usual, see \secref{sec:include}.

%%%%%%%%%%%%%%%%%%%%%%%%%%%%%%%%%%%%%%%%%%%%%%%%%%%%%%%%%%%%%%%%%%%%%%%%%%%%%%%%
\subsection{Legacy Detection}
\label{sec:detection}

The directive |\childdocmain| in the main file can detect
whether the complete document or merely a child is to be compiled
even without using the directive |\childdocof|.
This method is deprecated because it is less robust
and there is no compelling reason to use it;
it is merely provided for backward compatibility
and it may be removed in future versions.

If the detection mechanism is to be used,
it is mandatory to correctly specify
the filename of the main file as the argument of |\childdocmain|:
%
\begin{center}
\begin{tabular}{l}
|\input{childdoc.def}|\\
|\childdocmain{|\textit{main}|}|\\
\end{tabular}
\end{center}
%
If |\jobname| does not match the argument \textit{main} of |\childdocmain|,
it is assumed that |\jobname| points to the child file to be compiled.
When using |\childdocmain| with the main file specified as argument,
it suffices to start a child file
with just |\input{|\textit{main}|}|
without loading of the package and using |\childdocof|.
If instead all processing is done
with the appropriate \textsf{childdoc} directives,
the argument of \textit{main} of |\childdocmain| can be empty.

An alternative version of the command line processing described
in \secref{sec:commandline} using the detection mechanism reads:
%
\begin{center}
|... -jobname "|\textit{target}|" "|[\textit{flags}]%
[|\def\jobname{|\textit{dest}|}|]|\input{|\textit{main}|}"|
\end{center}

%%%%%%%%%%%%%%%%%%%%%%%%%%%%%%%%%%%%%%%%%%%%%%%%%%%%%%%%%%%%%%%%%%%%%%%%%%%%%%%%
\subsection{Manual Code}
\label{sec:manual}

In case one cannot be certain whether the definitions file |childdoc.def|
is installed on the target \TeX{} distribution
and one prefers not to ship it,
it is conceivable to paste a few relevant commands into the sources.

To that end, drop all statements |\input{childdoc.def}|
and perform the replacements as outlined below.
Instead of |\childdocmain{|\textit{main}|}| add the following code
to the top of the main file:
%
\begin{center}
\begin{tabular}{l}
|\||ifdefined\childdocname\endinput\||fi\newif\ifchilddoc|\\
|\edef\childdocname{\scantokens\expandafter{\jobname\noexpand}}|\\
|\def\childdocmain{|\textit{main}|}\||ifx\childdocmain\childdocname\||else|\\
|\childdoctrue\includeonly{\childdocname}\let\jobname\childdocmain\||fi|\\
\end{tabular}
\end{center}
%
Instead of |\childdocof{|\textit{main}|}| just include the main file
at the top of each child file:
%
\begin{center}
|\input{|\textit{main}|}|
\end{center}
%
A simple redirection |\childdocforward{|\textit{dest}|}| is achieved by:
%
\begin{center}
|\def\jobname{|\textit{dest}|}\input{\jobname}|
\end{center}
%
The redirection with prefix
|\childdocforwardprefix[|\textit{prefix}|]{|\textit{dest}|}|
is accomplished by:
%
\begin{center}
\begin{tabular}{l}
|{\edef\jobname{\scantokens\expandafter{\jobname\noexpand}}|\\
|\def\redirectjob |\textit{prefix}|#1~~~{\gdef\jobname{|\textit{dest}|#1}}|\\
|\expandafter\redirectjob\jobname~~~}\input{\jobname}|
\end{tabular}
\end{center}

In an alternative approach,
child documents can be compiled by a specific command line
without additional code or specific definitions:
%
\begin{center}
|... -jobname "|\textit{target}|" "|[\textit{flags}]%
|\includeonly{|\textit{dest}|}\input{|\textit{main}|}"|
\end{center}
%

%%%%%%%%%%%%%%%%%%%%%%%%%%%%%%%%%%%%%%%%%%%%%%%%%%%%%%%%%%%%%%%%%%%%%%%%%%%%%%%%
%%%%%%%%%%%%%%%%%%%%%%%%%%%%%%%%%%%%%%%%%%%%%%%%%%%%%%%%%%%%%%%%%%%%%%%%%%%%%%%%
\section{Information}

%%%%%%%%%%%%%%%%%%%%%%%%%%%%%%%%%%%%%%%%%%%%%%%%%%%%%%%%%%%%%%%%%%%%%%%%%%%%%%%%
\subsection{Copyright}

Copyright \copyright{} 2017--2018 Niklas Beisert

This work may be distributed and/or modified under the
conditions of the \LaTeX{} Project Public License, either version 1.3
of this license or (at your option) any later version.
The latest version of this license is in
  \url{http://www.latex-project.org/lppl.txt}
and version 1.3 or later is part of all distributions of \LaTeX{}
version 2005/12/01 or later.

This work has the LPPL maintenance status `maintained'.

The Current Maintainer of this work is Niklas Beisert.

This work consists of the files |README.txt|, |childdoc.ins| and |childdoc.dtx|
as well as the derived files |childdoc.def|, |cdocsamp.tex|
with |cdocsch1.tex|, |cdocsch2.tex|, |cdocspt3.tex|, |cdocspt4.tex|,
|cdocsdrf.tex|, |cdocsfn1.tex|, |cdocsfn2.tex|
as well as |childdoc.pdf|.

%%%%%%%%%%%%%%%%%%%%%%%%%%%%%%%%%%%%%%%%%%%%%%%%%%%%%%%%%%%%%%%%%%%%%%%%%%%%%%%%
\subsection{Files and Installation}

The package consists of the files:
%
\begin{center}
\begin{tabular}{ll}
    |README.txt|   & readme file \\
    |childdoc.ins| & installation file \\
    |childdoc.dtx| & source file \\
    |childdoc.def| & definition file \\
    |cdocsamp.tex| & sample main file \\
    |cdocsch1.tex| & sample include file \\
    |cdocsch2.tex| & sample include file \\
    |cdocspt3.tex| & sample part file \\
    |cdocspt4.tex| & sample part file \\
    |cdocsdrf.tex| & sample redirection file \\
    |cdocsfn1.tex| & sample redirection file \\
    |cdocsfn2.tex| & sample redirection file \\
    |childdoc.pdf| & manual
\end{tabular}
\end{center}
%
The distribution consists of the files
|README.txt|, |childdoc.ins| and |childdoc.dtx|.
%
\begin{itemize}
\item
Run (pdf)\LaTeX{} on |childdoc.dtx|
to compile the manual |childdoc.pdf| (this file).
\item
Run \LaTeX{} on |childdoc.ins| to create the definitions file |childdoc.def|
and the sample |cdocsamp.tex| with include files
|cdocsch1.tex|, |cdocsch2.tex|, |cdocspt3.tex|, |cdocspt4.tex|,
|cdocsdrf.tex|, |cdocsfn1.tex|, |cdocsfn2.tex|.
Then copy the file |childdoc.def| to an appropriate directory of your \LaTeX{}
distribution, e.g.\ \textit{texmf-root}|/tex/latex/childdoc|.
\end{itemize}

%%%%%%%%%%%%%%%%%%%%%%%%%%%%%%%%%%%%%%%%%%%%%%%%%%%%%%%%%%%%%%%%%%%%%%%%%%%%%%%%
\subsection{Related CTAN Packages}

There are several other packages which offer a similar functionality:
%
\begin{itemize}
\item
The packages
\href{http://ctan.org/pkg/docmute}{\textsf{docmute}},
\href{http://ctan.org/pkg/includex}{\textsf{includex}} and
\href{http://ctan.org/pkg/standalone}{\textsf{standalone}}
provide commands to include only the document body of
a child file thus allowing both files to be compiled individually.
\item
The packages \href{http://ctan.org/pkg/subdocs}{\textsf{subdocs}}
and \href{http://ctan.org/pkg/subfiles}{\textsf{subfiles}}
provide structures in which the main and child documents can be
encapsulated and allowing them to be compiled individually.
The inclusion mechanism is different from the conventional |\include|.
\item
The package \href{http://ctan.org/pkg/combine}{\textsf{combine}}
is an elaborate solution to combine several documents into one.
\end{itemize}
%
See also the CTAN topic \href{http://ctan.org/topic/subdocs}{\textsf{subdocs}}
for further related packages.
The present package differs from the above solutions in that
a document structure constructed with the conventional |\include| mechanism
just needs two extra commands at the top of every file
such that all constituent files can be compiled individually.

%%%%%%%%%%%%%%%%%%%%%%%%%%%%%%%%%%%%%%%%%%%%%%%%%%%%%%%%%%%%%%%%%%%%%%%%%%%%%%%%
%\subsection{Feature Suggestions}
%
%The following is a list of features which may be useful for future
%versions of this package:
%%
%\begin{itemize}
%\item
%\ldots
%\end{itemize}

%%%%%%%%%%%%%%%%%%%%%%%%%%%%%%%%%%%%%%%%%%%%%%%%%%%%%%%%%%%%%%%%%%%%%%%%%%%%%%%%
\subsection{Revision History}

%%%%%%%%%%%%%%%%%%%%%%%%%%%%%%%%%%%%%%%%
\paragraph{v2.0:} 2018/12/30

\begin{itemize}
\item
immediate forward processing
\item
added |\childdocby| mechanism
\item
manual restructured
\end{itemize}

%%%%%%%%%%%%%%%%%%%%%%%%%%%%%%%%%%%%%%%%
\paragraph{v1.6:} 2018/01/17

\begin{itemize}
\item
application for development of include files
\item
corrections to manual
\end{itemize}

%%%%%%%%%%%%%%%%%%%%%%%%%%%%%%%%%%%%%%%%
\paragraph{v1.5:} 2017/05/21

\begin{itemize}
\item
more complete structuring introduced
\item
|\childdocof| introduced
\item
|\childdoc| renamed to |\childdocmain|
\item
|\childredirect| renamed to |\childdocforward| and |\childdocforwardprefix|
and functionality expanded
\end{itemize}

%%%%%%%%%%%%%%%%%%%%%%%%%%%%%%%%%%%%%%%%
\paragraph{v1.0:} 2017/04/27

\begin{itemize}
\item
manual and install package
\item
first version published on CTAN
\end{itemize}

%%%%%%%%%%%%%%%%%%%%%%%%%%%%%%%%%%%%%%%%
\paragraph{v0.6:} 2017/04/26

\begin{itemize}
\item
redirection mechanism added
\end{itemize}

%%%%%%%%%%%%%%%%%%%%%%%%%%%%%%%%%%%%%%%%
\paragraph{v0.5:} 2017/04/26

\begin{itemize}
\item
functionality in definition file
\end{itemize}


%%%%%%%%%%%%%%%%%%%%%%%%%%%%%%%%%%%%%%%%%%%%%%%%%%%%%%%%%%%%%%%%%%%%%%%%%%%%%%%%
%%%%%%%%%%%%%%%%%%%%%%%%%%%%%%%%%%%%%%%%%%%%%%%%%%%%%%%%%%%%%%%%%%%%%%%%%%%%%%%%
%%%%%%%%%%%%%%%%%%%%%%%%%%%%%%%%%%%%%%%%%%%%%%%%%%%%%%%%%%%%%%%%%%%%%%%%%%%%%%%%
\appendix

\settowidth\MacroIndent{\rmfamily\scriptsize 000\ }

 \DocInput{childdoc.dtx}

\end{document}
%</driver>
% \fi
%
% %%%%%%%%%%%%%%%%%%%%%%%%%%%%%%%%%%%%%%%%%%%%%%%%%%%%%%%%%%%%%%%%%%%%%%%%%%%%%%
% %%%%%%%%%%%%%%%%%%%%%%%%%%%%%%%%%%%%%%%%%%%%%%%%%%%%%%%%%%%%%%%%%%%%%%%%%%%%%%
% \section{Sample}
%\iffalse
%<*samplemain>
%\fi
%
% The following presents a sample document
% with two chapters, two parts, a title page,
% a compile flag as well as three forwarding files to set the flag.
% It consists of eight |.tex| files:
% \begin{center}
% \begin{tabular}{ll}
% |cdocsamp.tex|&main file\\
% |cdocsch1.tex|&include file for chapter 1\\
% |cdocsch2.tex|&include file for chapter 2\\
% |cdocspt3.tex|&include file for part 3\\
% |cdocspt4.tex|&include file for part 4\\
% |cdocsdrf.tex|&forwarding file for main file in draft mode\\
% |cdocsfi1.tex|&forwarding file for final version of chapter 1\\
% |cdocsfi2.tex|&forwarding file for final version of chapter 2\\
% \end{tabular}
% \end{center}
% Each of the eight files can be compiled directly by the \LaTeX{} compiler.
%
% %%%%%%%%%%%%%%%%%%%%%%%%%%%%%%%%%%%%%%
% \paragraph{Main File.}
%
% The main file is called |cdocsamp.tex|.
%
% Load the \textsf{childdoc} definitions and
% declare the filename for the main document:
%    \begin{macrocode}
\input{childdoc.def}
\childdocmain{}
%    \end{macrocode}

% Optional override for |\version| flag:
%    \begin{macrocode}
%%\ifchilddoc\else\providecommand{\version}{draft}\fi
%    \end{macrocode}

% Define the default values for the |\version| flag
% (|final| for the main file and |draft| for childs):
%    \begin{macrocode}
\ifchilddoc
\providecommand{\version}{draft}
\else
\providecommand{\version}{final}
\fi
%    \end{macrocode}

% Load the standard document class:
%    \begin{macrocode}
\documentclass[12pt]{article}
%    \end{macrocode}

% Start the document body:
%    \begin{macrocode}
\begin{document}
%    \end{macrocode}

% Declare a title page.
% Print title, part of document being processed and version flag:
%    \begin{macrocode}
\addtocounter{page}{-1}
\begin{center}
{\LARGE\bfseries{}childdoc example\par}
\vspace{1cm}
\ifchilddoc
\ifchilddocmanual part\else chapter\fi:
`\childdocname' of `\childdocjob'\par
\else
main document: `\childdocjob'\par
\fi
version: \version\par
\end{center}
\newpage
%    \end{macrocode}

% Manually include selected file,
% otherwise process as usual:
%    \begin{macrocode}
\ifchilddocmanual
\section*{part `\childdocname'}
\input{\childdocname}
\else
%    \end{macrocode}

% Include the two chapters:
%    \begin{macrocode}
\include{cdocsch1}
\include{cdocsch2}
%    \end{macrocode}

% Include the two parts unless only chapters should be displayed:
%    \begin{macrocode}
\ifchilddoc\else
\section{part three}
\input{cdocspt3}
\section{part four}
\input{cdocspt4}
\fi
%    \end{macrocode}

% Process as usual until here:
%    \begin{macrocode}
\fi
%    \end{macrocode}

% End of document body:
%    \begin{macrocode}
\end{document}
%    \end{macrocode}
%\iffalse
%</samplemain>
%\fi
%
% %%%%%%%%%%%%%%%%%%%%%%%%%%%%%%%%%%%%%%
% \paragraph{Chapter Include Files.}
%
% The include files are called |cdocsch1.tex| and |cdocsch2.tex|.
%
%\iffalse
%<*samplechap1|samplechap2>
%\fi

% Optional override for |\version| flag:
%    \begin{macrocode}
%%\providecommand{\version}{final}
%    \end{macrocode}

% Include the main document:
%    \begin{macrocode}
\input{childdoc.def}
\childdocof{cdocsamp}
%    \end{macrocode}

%\iffalse
%</samplechap1|samplechap2>
%\fi
%
%\iffalse
%<*samplechap1>
%\fi
% Some text for chapter 1:
%    \begin{macrocode}
\section{one}
some text in chapter one
%    \end{macrocode}

%\iffalse
%</samplechap1>
%\fi
% Some text for chapter 2:
%\iffalse
%<*samplechap2>
%\fi
%    \begin{macrocode}
\section{two}
more text in chapter two
%    \end{macrocode}

%\iffalse
%</samplechap2>
%\fi
%
% %%%%%%%%%%%%%%%%%%%%%%%%%%%%%%%%%%%%%%
% \paragraph{Part Include Files.}
%
% The include files are called |cdocspt3.tex| and |cdocspt4.tex|.
%
%\iffalse
%<*samplepart3|samplepart4>
%\fi

% Optional override for |\version| flag:
%    \begin{macrocode}
%%\providecommand{\version}{final}
%    \end{macrocode}

% Include the main document:
%    \begin{macrocode}
\input{childdoc.def}
\childdocby{cdocsamp}
%    \end{macrocode}

%\iffalse
%</samplepart3|samplepart4>
%\fi
%
%\iffalse
%<*samplepart3>
%\fi
% Some text for part 3:
%    \begin{macrocode}
some text in part three
%    \end{macrocode}

%\iffalse
%</samplepart3>
%\fi
% Some text for part 4:
%\iffalse
%<*samplepart4>
%\fi
%    \begin{macrocode}
more text in part four
%    \end{macrocode}

%\iffalse
%</samplepart4>
%\fi
%
% %%%%%%%%%%%%%%%%%%%%%%%%%%%%%%%%%%%%%%
% \paragraph{Forwarding for a Complete Draft.}
%
% The following forwarding file |cdocsdrf.tex|
% compiles the main document in draft mode:
%\iffalse
%<*sampledraft>
%\fi
%    \begin{macrocode}
\def\version{draft}
\input{childdoc.def}
\childdocforward{cdocsamp}
%    \end{macrocode}

%\iffalse
%</sampledraft>
%\fi
%
% %%%%%%%%%%%%%%%%%%%%%%%%%%%%%%%%%%%%%%
% \paragraph{Forwarding for Final Version of the Chapters.}
%
% The following forwarding files |cdocsfn1.tex| and |cdocsfn2.tex|
% (with identical content)
% compile the final versions of the child documents
% |cdocsch1.tex| and |cdocsch2.tex|, respectively:
%\iffalse
%<*samplefinal>
%\fi
%    \begin{macrocode}
\def\version{final}
\input{childdoc.def}
\childdocforwardprefix[cdocsamp]{cdocsfn}{cdocsch}
%    \end{macrocode}

%\iffalse
%</samplefinal>
%\fi
%
% %%%%%%%%%%%%%%%%%%%%%%%%%%%%%%%%%%%%%%
% \paragraph{Command Line Processing.}
%
% The following three command lines generate the output files
% |cdocscld|, |cdocscl1| and |cdocscl2|
% which should be identical to
% |cdocsdrf|, |cdocsch1| and |cdocsfn2|, respectively:
% \begin{center}
% \begin{tabular}{l}
% |latex -jobname cdocscld \|\\
% |  "\def\version{draft}\input{childdoc.def}\childdocforward{cdocsamp}"|\\
% |latex -jobname cdocscl1 \|\\
% |  "\input{childdoc.def}\childdocforward[cdocsamp]{cdocsch1}"|\\
% |latex -jobname cdocscl2 \|\\
% |  "\def\version{final}\input{childdoc.def}\childdocforward{cdocsch2}"|
% \end{tabular}
% \end{center}
% Note that the trailing backslash on each first line
% merely continues the input to the second line
% (for convenient cut ant paste).
% Furthermore, the command |latex| can be replaced by any
% of its alternative versions such as |pdflatex|.
%
% %%%%%%%%%%%%%%%%%%%%%%%%%%%%%%%%%%%%%%%%%%%%%%%%%%%%%%%%%%%%%%%%%%%%%%%%%%%%%%
% %%%%%%%%%%%%%%%%%%%%%%%%%%%%%%%%%%%%%%%%%%%%%%%%%%%%%%%%%%%%%%%%%%%%%%%%%%%%%%
% \section{Implementation}
%\iffalse
%<*package>
%\fi
%
% This section describes the definitions file |childdoc.def|.

% The definitions cannot be loaded using |\usepackage| or |\RequirePackage|
% which has a mechanism to prevent loading a style file more than once.
% When loading the definitions by means of |\input|
% multiple instances have to be prevented manually:
%\iffalse
%This code needs to be before the `\ProvidesFile' directive
%which is defined at the beginning of this file.
%Therefore it is also placed there and commented out here.
%</package>
%<*discard>
%\fi
%    \begin{macrocode}
\ifdefined\childdocmain\endinput\fi
%    \end{macrocode}
%\iffalse
%</discard>
%<*package>
%\fi
%
% \macro{\ifchilddoc}
% \macro{\ifchilddocmanual}
% The conditional |\ifchilddoc| tells whether a
% child (true) or main (false) document is being compiled.
% The conditional |\ifchilddocmanual| tells whether
% the |\includeonly| mechanism is used (false) or
% the selection of child files must be performed manually (true).
% The definitions initialise to false:
%    \begin{macrocode}
\newif\ifchilddoc
\newif\ifchilddocmanual
%    \end{macrocode}

% \macro{\childdocname}
% \macro{\childdocjob}
% The macro |\childdocname| stores the name of the main document
% to be compiled. The macro |\childdocjob| stores the name of
% the document on which the \LaTeX{} compiler was originally invoked.
% The content of |\jobname| cannot be compared
% to filenames specified in the source due to different catcodes.
% The following code rescans |\jobname|, stores the result
% in |\childdocname| and saves a copy in |\childdocjob|:
%    \begin{macrocode}
\edef\childdocname{\scantokens\expandafter{\jobname\noexpand}}
\let\childdocjob\childdocname
%    \end{macrocode}

% \macro{\childdocdisable}
% The macro |\childdocdisable| prevents the main file
% from being processed more than once.
% At this stage, the main document command |\childdocmain|
% is assumed to be called once again where it should do nothing.
% Any subsequent call to it should prevent
% a secondary processing of the main document
% It overwrites the forwarding commands
% |\childdocof| and |\childdocforward|
% with empty macros to prevent further inclusions of the main document:
%    \begin{macrocode}
\newcommand{\childdocdisable}
{
  \renewcommand{\childdocmain}[1]{\renewcommand{\childdocmain}[1]{\endinput}}
  \renewcommand{\childdocof}[1]{}
  \renewcommand{\childdocby}[2][]{}
  \renewcommand{\childdocforward}[2][]{}
  \renewcommand{\childdocdisable}{}
}
%    \end{macrocode}

% \macro{\childdocmain}
% The macro |\childdocmain| is to be called at the top of the main file
% with nothing or the main filename (without extension) as argument.
% First, it breaks loops.
% If the argument is not empty and does not match |\childdocname|
% (which is set by the first inclusion of |childdoc.def|),
% |\ifchilddoc| is set to true, |\includeonly| is applied to the child file
% and |\jobname| is set to the main file
% (for proper handling of |.aux| files):
%    \begin{macrocode}
\newcommand{\childdocmain}[1]
{
  \childdocdisable\childdocmain{}
  \if?#1?\else
    \begingroup
      \def\childdoctmp{#1}
      \ifx\childdoctmp\childdocname
        \def\childdoctmp{}
      \else
        \def\childdoctmp
        {
          \childdoctrue
          \includeonly{\childdocname}
          \def\childdocjob{#1}
          \def\jobname{#1}
        }
      \fi
      \expandafter
    \endgroup
    \childdoctmp
  \fi
}
%    \end{macrocode}

% \macro{\childdocof}
% The command |\childdocof| redirects
% compilation to the main file |#1|.
%    \begin{macrocode}
\newcommand{\childdocof}[1]
{
  \childdocdisable
  \childdoctrue
  \includeonly{\childdocname}
  \def\jobname{#1}
  \def\childdocjob{#1}
  \input{#1}
}
%    \end{macrocode}

% \macro{\childdocby}
% The command |\childdocby| ....
%    \begin{macrocode}
\newcommand{\childdocby}[2][]
{
  \childdocdisable
  \childdoctrue
  \childdocmanualtrue
  \if?#1?\else
    \def\jobname{#2}
  \fi
  \def\childdocjob{#2}
  \input{#2}
  \endinput
}
%    \end{macrocode}

% \macro{\childdocforward}
% The command |\childdocforward| redirects
% compilation to the main file or
% (if the optional argument is given) a child file.
% Parameters are set as if the main file
% or a child file starting with |\childdocof| was compiled.
% Then compilation is handed over to the main file:
%    \begin{macrocode}
\newcommand{\childdocforward}[2][]
{
  \begingroup
    \if?#1?
      \def\childdoctmp
      {
        \def\childdocname{#2}
        \def\childdocjob{#2}
        \def\jobname{#2}
        \input{#2}
        \endinput
      }
    \else
      \def\childdoctmp
      {
        \childdocdisable
        \def\childdocname{#2}
        \childdoctrue
        \includeonly{#2}
        \def\childdocjob{#1}
        \def\jobname{#1}
        \input{#1}
        \endinput
      }
    \fi
    \expandafter
  \endgroup
  \childdoctmp
}
%    \end{macrocode}

% \macro{\childdocforwardprefix}
% The command |\childdocforwardprefix| redirects
% compilation to the main or a child file by means of a pattern.
% The prefix |#1| in the current filename is replaced by |#2|
% and the suffix of the current filename is kept
% (it is assumed that the filename does not contain the substring `|~~~|'
% which is used as a delimiter).
% Compilation is handed over to the new file by |\childdocforward|:
%    \begin{macrocode}
\newcommand{\childdocforwardprefix}[3][]
{
  \begingroup
    \def\childdocextract #2##1~~~{\def\childdoctmp{\childdocforward[#1]{#3##1}}}
    \expandafter\childdocextract\childdocname~~~
    \expandafter
  \endgroup
  \childdoctmp
}
%    \end{macrocode}

% \macro{\childdoc}
% The deprecated macro |\childdoc| is a legacy version of |\childdocmain|:
%    \begin{macrocode}
\newcommand{\childdoc}{\childdocmain}
%    \end{macrocode}

% \macro{\childdocredirect}
% The deprecated macro |\childdocredirect| is a legacy version
% of |\childdocforward| and |\childdocforwardprefix|:
%    \begin{macrocode}
\newcommand{\childdocredirect}[2][]
{
  \begingroup
    \if?#1?
      \def\childdoctmp{\childdocforward{#2}}
    \else
      \def\childdoctmp{\childdocforwardprefix{#1}{#2}}
    \fi
    \expandafter
  \endgroup
  \childdoctmp
}
%    \end{macrocode}

%\iffalse
%</package>
%\fi
%
\endinput
|\\
|\childdocmain{|\textit{main}|}|\\
\end{tabular}
\end{center}
%
If |\jobname| does not match the argument \textit{main} of |\childdocmain|,
it is assumed that |\jobname| points to the child file to be compiled.
When using |\childdocmain| with the main file specified as argument,
it suffices to start a child file
with just |\input{|\textit{main}|}|
without loading of the package and using |\childdocof|.
If instead all processing is done
with the appropriate \textsf{childdoc} directives,
the argument of \textit{main} of |\childdocmain| can be empty.

An alternative version of the command line processing described
in \secref{sec:commandline} using the detection mechanism reads:
%
\begin{center}
|... -jobname "|\textit{target}|" "|[\textit{flags}]%
[|\def\jobname{|\textit{dest}|}|]|\input{|\textit{main}|}"|
\end{center}

%%%%%%%%%%%%%%%%%%%%%%%%%%%%%%%%%%%%%%%%%%%%%%%%%%%%%%%%%%%%%%%%%%%%%%%%%%%%%%%%
\subsection{Manual Code}
\label{sec:manual}

In case one cannot be certain whether the definitions file |childdoc.def|
is installed on the target \TeX{} distribution
and one prefers not to ship it,
it is conceivable to paste a few relevant commands into the sources.

To that end, drop all statements |% \iffalse
%
% childdoc.dtx Copyright (C) 2017-2018 Niklas Beisert
%
% This work may be distributed and/or modified under the
% conditions of the LaTeX Project Public License, either version 1.3
% of this license or (at your option) any later version.
% The latest version of this license is in
%   http://www.latex-project.org/lppl.txt
% and version 1.3 or later is part of all distributions of LaTeX
% version 2005/12/01 or later.
%
% This work has the LPPL maintenance status `maintained'.
%
% The Current Maintainer of this work is Niklas Beisert.
%
% This work consists of the files childdoc.dtx and childdoc.ins
% and the derived files childdoc.def and cdocsamp.tex with
% cdocsch1.tex, cdocsch2.tex, cdocsdrf.tex, cdocsfn1.tex, cdocsfn2.tex.
%
%<package>\ifdefined\childdocmain\endinput\fi
%<package>\ProvidesFile{childdoc.def}[2018/12/30 v2.0 child document driver]
%<samplemain>\ProvidesFile{cdocsamp.tex}[2018/12/30 v2.0 sample for childdoc]
%<*driver>
%\ProvidesFile{childdoc.drv}[2018/12/30 v2.0 childdoc reference manual file]
\PassOptionsToClass{10pt,a4paper}{article}
\documentclass{ltxdoc}

\usepackage[margin=35mm]{geometry}
\usepackage{hyperref}
\usepackage{hyperxmp}
\usepackage[usenames]{color}

\hypersetup{colorlinks=true}
\hypersetup{pdfstartview=FitH}
\hypersetup{pdfpagemode=UseNone}
\hypersetup{pdfsource={}}
\hypersetup{pdflang={en-UK}}
\hypersetup{pdfcopyright={Copyright 2017-2018 Niklas Beisert.
  This work may be distributed and/or modified under the
  conditions of the LaTeX Project Public License, either version 1.3
  of this license or (at your option) any later version.}}
\hypersetup{pdflicenseurl={http://www.latex-project.org/lppl.txt}}
\hypersetup{pdfcontactaddress={ETH Zurich, ITP, HIT K,
  Wolfgang-Pauli-Strasse 27}}
\hypersetup{pdfcontactpostcode={8093}}
\hypersetup{pdfcontactcity={Zurich}}
\hypersetup{pdfcontactcountry={Switzerland}}
\hypersetup{pdfcontactemail={nbeisert@itp.phys.ethz.ch}}
\hypersetup{pdfcontacturl={http://people.phys.ethz.ch/\xmptilde nbeisert/}}

\newcommand{\secref}[1]{\hyperref[#1]{section \ref*{#1}}}

\parskip1ex
\parindent0pt
\let\olditemize\itemize
\def\itemize{\olditemize\parskip0pt}

\begin{document}

\title{The \textsf{childdoc} Package}
\hypersetup{pdftitle={The childdoc Package}}
\author{Niklas Beisert\\[2ex]
  Institut f\"ur Theoretische Physik\\
  Eidgen\"ossische Technische Hochschule Z\"urich\\
  Wolfgang-Pauli-Strasse 27, 8093 Z\"urich, Switzerland\\[1ex]
  \href{mailto:nbeisert@itp.phys.ethz.ch}
  {\texttt{nbeisert@itp.phys.ethz.ch}}}
\hypersetup{pdfauthor={Niklas Beisert}}
\hypersetup{pdfsubject={Manual for the LaTeX2e Package childdoc}}
\date{30 December 2018, \textsf{v2.0}}
\maketitle

\begin{abstract}\noindent
\textsf{childdoc} is a \LaTeXe{} package
that enables the direct compilation
of document sections included by |\include|
to individual files.
\end{abstract}

\begingroup
\parskip0ex
\tableofcontents
\endgroup

%%%%%%%%%%%%%%%%%%%%%%%%%%%%%%%%%%%%%%%%%%%%%%%%%%%%%%%%%%%%%%%%%%%%%%%%%%%%%%%%
%%%%%%%%%%%%%%%%%%%%%%%%%%%%%%%%%%%%%%%%%%%%%%%%%%%%%%%%%%%%%%%%%%%%%%%%%%%%%%%%
\section{Introduction}

\LaTeX{} provides a mechanism to structure a large document (such as a book)
into a main file and several child files (containing the chapters)
using the |\include| command.
This mechanism is beneficial for documents
which span hundreds of pages in order to
make the source file(s) more manageable.
Moreover, compilation can be restricted to
selected child files by means of the |\includeonly| command.
The latter feature can be used to reduce the compilation time while editing
(this was significantly more useful in the earlier days of \LaTeX{})
or to generate a smaller document which is easier to navigate.
Another application of |\includeonly| is to generate
documents consisting of selected parts of the complete document.

However, there are a few drawbacks of the plain |\include| mechanism:
\begin{itemize}
\item
The child files cannot be compiled on their own,
they can only be compiled via the main file.
A naive editing environment
(such as a text editor with an option
to have the current file processed by \LaTeX)
may require one to switch to the main file before compiling;
attempting to compile the child file produces errors.
\item
The main file must be modified (each time)
to adjust the |\includeonly| command
to the present needs. This easily leaves the main file in a messy state.
\item
The generated document will always carry the filename
of the main document. This is inconvenient if
several child files are to be compiled and
to be kept for distribution.
\end{itemize}

The present package provides a simple interface
to make child files individually compilable by \LaTeX{}.
Compiling a child file then has the same effect as compiling
the main file with an |\includeonly| command
to select the appropriate child.
Moreover the generated document will carry the name of the child
rather than the main file.
This resolves all three above issues.

This feature is meant to make the editing of books,
thesis documents and lecture notes somewhat more convenient.
However, the package can also be used efficiently for
composing a series of documents (such as exercise sheets)
which are typically distributed individually.
It then assists the author in generating the individual documents
(potentially in different versions)
as well as a document containing the collected series.
Another application is in developing style files
or other kinds of included material
where compilation of the style file could redirect
to a sample or test file.

%%%%%%%%%%%%%%%%%%%%%%%%%%%%%%%%%%%%%%%%%%%%%%%%%%%%%%%%%%%%%%%%%%%%%%%%%%%%%%%%
%%%%%%%%%%%%%%%%%%%%%%%%%%%%%%%%%%%%%%%%%%%%%%%%%%%%%%%%%%%%%%%%%%%%%%%%%%%%%%%%
\section{Usage}

First of all, the package \textsf{childdoc} is \emph{not} a standard
\LaTeXe{} |.sty| style file! Therefore it needs to be invoked in
a non-standard way.

%%%%%%%%%%%%%%%%%%%%%%%%%%%%%%%%%%%%%%%%%%%%%%%%%%%%%%%%%%%%%%%%%%%%%%%%%%%%%%%%
\subsection{Included Files}
\label{sec:include}

%%%%%%%%%%%%%%%%%%%%%%%%%%%%%%%%%%%%%%%%
\DescribeMacro{\childdocmain}
To use the package, add the commands
\begin{center}
\begin{tabular}{l}
|\input{childdoc.def}|\\
|\childdocmain{}|\\
\end{tabular}
\end{center}
at the very top of the main \LaTeX{} file,
in particular \emph{before} the |\documentclass| statement!
The argument of |\childdocmain| should be left empty
(but it must be present).

%%%%%%%%%%%%%%%%%%%%%%%%%%%%%%%%%%%%%%%%
\DescribeMacro{\childdocof}
Furthermore, add the commands
\begin{center}
\begin{tabular}{l}
|\input{childdoc.def}|\\
|\childdocof{|\textit{main}|}|\\
\end{tabular}
\end{center}
at the top of every child file \textit{child}
which is included by |\include{|\textit{child}|}|
from within the main file
(or at least for those files to be compiled individually).
The argument \textit{main} must be the filename of the main file.

There are a couple of
considerations in setting up the main and child documents:

%%%%%%%%%%%%%%%%%%%%%%%%%%%%%%%%%%%%%%%%
\paragraph{Restrictions.}

Please note the following restrictions:
\begin{itemize}
\item
|\childdocmain| must be called with one argument \textit{main}
to ensure compatibility with earlier version of the package.
It must either be empty (|\childdocmain{}|)
or precisely match the filename of the main file in which it is specified.
See \secref{sec:detection} for further information.
\item
The filename \textit{main} must be specified without the |.tex| extension.
\item
The filename \textit{main} is case sensitive
(even in case-insensitive file systems)
due to internal string comparison.
\item
The argument \textit{main} should be fully expanded, it cannot be a macro.
\item
Subdirectories and special characters should be avoided in filenames.
\item
The command |\childdocmain{|\textit{main}|}| must be followed by a whitespace.
It should not be followed immediately by another command
or by a comment mark `|%|'.
This is because the \TeX{} parser reads the token immediately following
the argument of |\childdocmain| and puts it
at the beginning of every child section;
however, a white\-space is ignored.
\end{itemize}

%%%%%%%%%%%%%%%%%%%%%%%%%%%%%%%%%%%%%%%%
\paragraph{Content of Main File.}

It is advisable to place all content in the child files included by |\include|.
Any output contained in the main file will appear in all child documents
unless suppressed manually;
it cannot be suppressed automatically by the |\includeonly| directive
and thus should normally be avoided.
A method to include some content in the main file
by means of conditional processing is described in \secref{sec:conditional}.

%%%%%%%%%%%%%%%%%%%%%%%%%%%%%%%%%%%%%%%%
\paragraph{Page Numbering.}

When only a part of the document is compiled,
the appropriate numbering of pages
(as well as other status parameters)
is determined from the |.aux| files.
The latter contain information from previous passes.
However this information needs to propagate through
all intermediate child documents.
Therefore the page numbering in child documents may well
be inconsistent until the complete document is compiled at least once.

A useful (if unconventional) way to always ensure a consistent
page numbering is to restart the numbering in each child document
and denote the pages by `\textit{child}|.|\textit{page}'
where \textit{child} represents the chapter/section number of the child file.
This can be achieved by the command
|\numberwithin{page}{|\textit{child}|}|
of the \textsf{amsmath} package
where \textit{child} can be |chapter| or |section|
depending on the chosen structuring.
Alternatively, one can modify the macro |\thepage| appropriately
and reset the counter |page| at the start of each child file.

%%%%%%%%%%%%%%%%%%%%%%%%%%%%%%%%%%%%%%%%%%%%%%%%%%%%%%%%%%%%%%%%%%%%%%%%%%%%%%%%
\subsection{Conditional Processing}
\label{sec:conditional}

The package provides a mechanism to compile different versions
of a document. To customise the versions further some conditional processing
can come in handy to distinguish which version is being compiled.
The package provides two macros to describe the compilation context:

%%%%%%%%%%%%%%%%%%%%%%%%%%%%%%%%%%%%%%%%
\DescribeMacro{\ifchilddoc}
The conditional |\ifchilddoc| distinguishes between the compilation of
child documents and the main document:
%
\begin{center}
|\ifchilddoc |\textit{child-code}| |[|\||else |\textit{main-code}]| \||fi|
\end{center}

%%%%%%%%%%%%%%%%%%%%%%%%%%%%%%%%%%%%%%%%
\DescribeMacro{\childdocname}
\DescribeMacro{\childdocjob}
The macro |\childdocname| contains the filename (without extension)
of the main or child file being processed.
Note that |\childdocjob| will always contain the name of the main file.

%%%%%%%%%%%%%%%%%%%%%%%%%%%%%%%%%%%%%%%%
\paragraph{Title Page.}

Conditional processing can be used to include a title or banner page
in the main document when proper precautions are taken.
Importantly, the code in the main file should ensure that the page counter
(as well as other status parameters which are stored in the |.aux| files)
takes the same value after the conditional processing.
Otherwise the page numbers may take divergent values
depending on which part is compiled.

For example, a title page could be declared by:
%
\begin{center}
\begin{tabular}{l}
|\ifchilddoc\||else|\\
|\addtocounter{page}{-1}|\\
\textit{code for title page}\\
|\newpage|\\
|\||fi|
\end{tabular}
\end{center}
%
A banner page for the child documents can be generated by:
%
\begin{center}
\begin{tabular}{l}
|\ifchilddoc|\\
|\addtocounter{page}{-1}|\\
\textit{code for banner page}\\
|\newpage|\\
|\||fi|
\end{tabular}
\end{center}
%
Here one could write a message such as:
\begin{center}
|This is the part \childdocname{} of \childdocjob{}.|
\end{center}

%%%%%%%%%%%%%%%%%%%%%%%%%%%%%%%%%%%%%%%%%%%%%%%%%%%%%%%%%%%%%%%%%%%%%%%%%%%%%%%%
\subsection{Flags}
\label{sec:flags}

The package makes it easy to generate different versions
of the main or child documents.
To this end compilation flags can be defined
and assigned different default values.
They will be particularly useful in conjunction
with the forwarding mechanism described in \secref{sec:forward}.

For example, it may be useful to have a flag |\version|
which can be set to |draft| or |final|.
The document source will contain some conditional code
depending on the value of |\version|.
Suppose further, the flag should default to |final| for the main file
and to |draft| for child files
which is a natural assignment for editing the document.
This is achieved by placing the following code
in the preamble of the main document
(below the |\childdocmain| directive):
%
\begin{center}
\begin{tabular}{l}
|\ifchilddoc|\\
|\providecommand{\version}{draft}|\\
|\||else|\\
|\providecommand{\version}{final}|\\
|\||fi|
\end{tabular}
\end{center}
%
The definition by |\providecommand| makes sure
that previous definitions are not overwritten.
Further statements |\providecommand{\version}{...}|
can thus be added before the above code to override it.

For the main file, one might add a line
(between |\childdocmain| and the above block)
%
\begin{center}
|%\ifchilddoc\||else\providecommand{\version}{draft}\||fi|
\end{center}
%
which can be uncommented to produce a draft version.
Likewise one can add a line to the very top of a child file
(above the |\childdocof{|\textit{main}|}| directive)
%
\begin{center}
|%\providecommand{\version}{final}|
\end{center}
%
which can be uncommented to produce the final version of this child document.

%%%%%%%%%%%%%%%%%%%%%%%%%%%%%%%%%%%%%%%%%%%%%%%%%%%%%%%%%%%%%%%%%%%%%%%%%%%%%%%%
\subsection{Forwarding}
\label{sec:forward}

Different versions of the main or child documents
using compilation flags as described in \secref{sec:flags}
can be (permanently) stored in different files
for convenient compilation, viewing and distribution.
To this end, the package defines a command
to pass on compilation to a different file:

%%%%%%%%%%%%%%%%%%%%%%%%%%%%%%%%%%%%%%%%
\DescribeMacro{\childdocforward}
The command |\childdocforward| redirects processing to
another source file:
%
\begin{center}
\begin{tabular}{l}
|\input{childdoc.def}|\\
|\childdocforward[|\textit{main}|]{|\textit{dest}|}|\\
\end{tabular}
\end{center}
%
The argument \textit{dest} is the destination file
(without extension).
It should be the main file or one of the child files.
Note that further \textsf{childdoc} directives
such as |\childdocof| and |\childdocforward|
in the indicated file will be processed in this form.
The optional argument \textit{main}
passes on directly to the main file \textit{main}
while pretending to compile the child \textit{dest}.
This form behaves as if \textit{dest}
issues |\childdocof{|\textit{main}|}| right away,
and no further \textsf{childdoc} directives will be processed.

%%%%%%%%%%%%%%%%%%%%%%%%%%%%%%%%%%%%%%%%
\DescribeMacro{\...prefix}
In the alternative form |\childdocforwardprefix|,
%
\begin{center}
\begin{tabular}{l}
|\input{childdoc.def}|\\
|\childdocforwardprefix[|\textit{main}|]{|\textit{prefix}|}{|\textit{dest}|}|
\end{tabular}
\end{center}
%
the destination file is determined by a pattern
depending on the current file:
To make this work, the current file must be called
`{\textit{prefix}\hspace{0.2em}\textit{suffix}}'
with \textit{prefix} matching precisely the argument.
Processing is then passed on to the file
`{\textit{dest}\hspace{0.2em}\textit{suffix}}'.
Surely, the same effect is achieved by
directly specifying the
argument `{\textit{dest}\hspace{0.2em}\textit{suffix}}'
in the first form.
However, that requires to set up a different file
for each child. With the alternative form of the command
all these files can have exactly the same content
which simplifies setting them up and maintaining them.

For example, the following file |draft.tex|
with a compilation flag |\version| as described in \secref{sec:flags}
compiles the main document as a draft:
%
\begin{center}
\begin{tabular}{l}
|\def\version{draft}|\\
|\input{childdoc.def}|\\
|\childdocforward{|\textit{main}|}|
\end{tabular}
\end{center}
%
Likewise, the following files |final|\textit{nn}|.tex|
compile the final version of the child document
|child|\textit{nn}|.tex|:
%
\begin{center}
\begin{tabular}{l}
|\def\version{final}|\\
|\input{childdoc.def}|\\
|\childdocforwardprefix{final}{child}|
\end{tabular}
\end{center}
%

Note that when several versions of a main file and/or of each child file
are to be generated, it may be convenient to set up a |Makefile| or
shell script to automatise the process.

%%%%%%%%%%%%%%%%%%%%%%%%%%%%%%%%%%%%%%%%%%%%%%%%%%%%%%%%%%%%%%%%%%%%%%%%%%%%%%%%
\subsection{Command Line Processing}
\label{sec:commandline}

The effect of redirection files can also be achieved by invoking
the \LaTeX{} compiler with a more elaborate command line.
Most conveniently this should be done as part
of a shell script or a |Makefile|.

When using \textsf{childdoc} in the main file, the following
command lines effectively perform a redirection
(note that depending on the shell being used,
backslashes may have to be doubled: `|\|' $\to$ `|\\|'):
%
\begin{center}
|... -jobname "|\textit{target}|" |\\|"|[\textit{flags}]%
|\input{childdoc.def}\childdocforward[|\textit{main}|]{|\textit{dest}|}"|
\end{center}
%
Here \textit{target} is the name of the output file,
\textit{main} is the name of the main file
and \textit{dest} is the name of the main or child file to be processed
(all filenames without extensions).
The optional argument \textit{main} can be omitted
if \textit{main} matches \textit{dest}.
Optionally, compilation \textit{flags} can be defined via |\def| commands.
This command line makes the \TeX{} engine believe
it is compiling the file \textit{target}
whose content is specified as the latter parameter.
The provided code then forwards the processing to
\textit{main} or \textit{dest} as described in \secref{sec:forward}.

%%%%%%%%%%%%%%%%%%%%%%%%%%%%%%%%%%%%%%%%%%%%%%%%%%%%%%%%%%%%%%%%%%%%%%%%%%%%%%%%
\subsection{Include by Input}
\label{sec:input}

Including child documents by |\include| has some restrictions by design.
Most notably, the content of a child document always occupies
its own set of pages; pages cannot be shared between child documents.
Usually, this behaviour makes perfect sense
because each child document contain an essential part of the document.
However, in some situations it may be desirable to compose
a document from a collection of parts
without having mandatory page breaks between then.
For this case, the package
provides a mechanism to include parts
by |\input| which can also be processed individually.
However, by construction this mechanism
requires manual handling of the content to be output.

%%%%%%%%%%%%%%%%%%%%%%%%%%%%%%%%%%%%%%%%
\DescribeMacro{\ifchilddocmanual}
The main file should be prepared as usual, see \secref{sec:include}.
However, the document body must make a distinction
between processing of an individual part and of the main document, e.g.:
%
\begin{center}
\begin{tabular}{l}
|\ifchilddocmanual|\\
|\input{\childdocname}|\\
|\||else|\\
\textit{document body with }|\input{|\textit{part}|}|\\
|\||fi|
\end{tabular}
\end{center}
%
The conditional |\ifchilddocmanual| is true whenever
a part to be included by |\input| is being compiled,
and the name of the part is stored in |\childdocname|.

%%%%%%%%%%%%%%%%%%%%%%%%%%%%%%%%%%%%%%%%
\DescribeMacro{\childdocby}
Each part to be included by |\input| should start with:
%
\begin{center}
\begin{tabular}{l}
|\input{childdoc.def}|\\
|\childdocby{|\textit{main}|}|\\
\end{tabular}
\end{center}
%
The directive |\childdocby| is similar to |\childdocof|
described in \secref{sec:include},
but the subsequent selection of content must be done manually.
To that end, both |\ifchilddoc| and |\ifchilddocmanual|
will be true upon processing of a part,
and the name of the part is stored in |\childdocname|.
Note that |\jobname| will be set to the filename of the current part
so that each part receives an individual |.aux| file
that does not interfere with the |.aux| file(s) of the main document.
This behaviour can be altered by the alternative form
|\childdocby[*]{|\textit{main}|}| (with a non-empty optional argument)
which uses the |.aux| file of the main document
by setting |\jobname| to \textit{main}.

%%%%%%%%%%%%%%%%%%%%%%%%%%%%%%%%%%%%%%%%%%%%%%%%%%%%%%%%%%%%%%%%%%%%%%%%%%%%%%%%
\subsection{Driver Development}
\label{sec:driver}

The \textsf{childdoc} mechanism can also be use for the development
of definition files such as \LaTeX{} styles or classes.
This case differs from the above setup with multiple parts
included by |\include| in that no |\includeonly| should be invoked.
This can be achieved by starting the include file
(before |\ProvidesPackage|) with:
%
\begin{center}
\begin{tabular}{l}
|\input{childdoc.def}|\\
|\childdocforward{|\textit{main}|}|\\
\end{tabular}
\end{center}
%
or alternatively with:
%
\begin{center}
\begin{tabular}{l}
|\input{childdoc.def}|\\
|\childdocby{|\textit{main}|}|\\
\end{tabular}
\end{center}
%
Both forms have slightly different effects as described above.
The main file is prepared as usual, see \secref{sec:include}.

%%%%%%%%%%%%%%%%%%%%%%%%%%%%%%%%%%%%%%%%%%%%%%%%%%%%%%%%%%%%%%%%%%%%%%%%%%%%%%%%
\subsection{Legacy Detection}
\label{sec:detection}

The directive |\childdocmain| in the main file can detect
whether the complete document or merely a child is to be compiled
even without using the directive |\childdocof|.
This method is deprecated because it is less robust
and there is no compelling reason to use it;
it is merely provided for backward compatibility
and it may be removed in future versions.

If the detection mechanism is to be used,
it is mandatory to correctly specify
the filename of the main file as the argument of |\childdocmain|:
%
\begin{center}
\begin{tabular}{l}
|\input{childdoc.def}|\\
|\childdocmain{|\textit{main}|}|\\
\end{tabular}
\end{center}
%
If |\jobname| does not match the argument \textit{main} of |\childdocmain|,
it is assumed that |\jobname| points to the child file to be compiled.
When using |\childdocmain| with the main file specified as argument,
it suffices to start a child file
with just |\input{|\textit{main}|}|
without loading of the package and using |\childdocof|.
If instead all processing is done
with the appropriate \textsf{childdoc} directives,
the argument of \textit{main} of |\childdocmain| can be empty.

An alternative version of the command line processing described
in \secref{sec:commandline} using the detection mechanism reads:
%
\begin{center}
|... -jobname "|\textit{target}|" "|[\textit{flags}]%
[|\def\jobname{|\textit{dest}|}|]|\input{|\textit{main}|}"|
\end{center}

%%%%%%%%%%%%%%%%%%%%%%%%%%%%%%%%%%%%%%%%%%%%%%%%%%%%%%%%%%%%%%%%%%%%%%%%%%%%%%%%
\subsection{Manual Code}
\label{sec:manual}

In case one cannot be certain whether the definitions file |childdoc.def|
is installed on the target \TeX{} distribution
and one prefers not to ship it,
it is conceivable to paste a few relevant commands into the sources.

To that end, drop all statements |\input{childdoc.def}|
and perform the replacements as outlined below.
Instead of |\childdocmain{|\textit{main}|}| add the following code
to the top of the main file:
%
\begin{center}
\begin{tabular}{l}
|\||ifdefined\childdocname\endinput\||fi\newif\ifchilddoc|\\
|\edef\childdocname{\scantokens\expandafter{\jobname\noexpand}}|\\
|\def\childdocmain{|\textit{main}|}\||ifx\childdocmain\childdocname\||else|\\
|\childdoctrue\includeonly{\childdocname}\let\jobname\childdocmain\||fi|\\
\end{tabular}
\end{center}
%
Instead of |\childdocof{|\textit{main}|}| just include the main file
at the top of each child file:
%
\begin{center}
|\input{|\textit{main}|}|
\end{center}
%
A simple redirection |\childdocforward{|\textit{dest}|}| is achieved by:
%
\begin{center}
|\def\jobname{|\textit{dest}|}\input{\jobname}|
\end{center}
%
The redirection with prefix
|\childdocforwardprefix[|\textit{prefix}|]{|\textit{dest}|}|
is accomplished by:
%
\begin{center}
\begin{tabular}{l}
|{\edef\jobname{\scantokens\expandafter{\jobname\noexpand}}|\\
|\def\redirectjob |\textit{prefix}|#1~~~{\gdef\jobname{|\textit{dest}|#1}}|\\
|\expandafter\redirectjob\jobname~~~}\input{\jobname}|
\end{tabular}
\end{center}

In an alternative approach,
child documents can be compiled by a specific command line
without additional code or specific definitions:
%
\begin{center}
|... -jobname "|\textit{target}|" "|[\textit{flags}]%
|\includeonly{|\textit{dest}|}\input{|\textit{main}|}"|
\end{center}
%

%%%%%%%%%%%%%%%%%%%%%%%%%%%%%%%%%%%%%%%%%%%%%%%%%%%%%%%%%%%%%%%%%%%%%%%%%%%%%%%%
%%%%%%%%%%%%%%%%%%%%%%%%%%%%%%%%%%%%%%%%%%%%%%%%%%%%%%%%%%%%%%%%%%%%%%%%%%%%%%%%
\section{Information}

%%%%%%%%%%%%%%%%%%%%%%%%%%%%%%%%%%%%%%%%%%%%%%%%%%%%%%%%%%%%%%%%%%%%%%%%%%%%%%%%
\subsection{Copyright}

Copyright \copyright{} 2017--2018 Niklas Beisert

This work may be distributed and/or modified under the
conditions of the \LaTeX{} Project Public License, either version 1.3
of this license or (at your option) any later version.
The latest version of this license is in
  \url{http://www.latex-project.org/lppl.txt}
and version 1.3 or later is part of all distributions of \LaTeX{}
version 2005/12/01 or later.

This work has the LPPL maintenance status `maintained'.

The Current Maintainer of this work is Niklas Beisert.

This work consists of the files |README.txt|, |childdoc.ins| and |childdoc.dtx|
as well as the derived files |childdoc.def|, |cdocsamp.tex|
with |cdocsch1.tex|, |cdocsch2.tex|, |cdocspt3.tex|, |cdocspt4.tex|,
|cdocsdrf.tex|, |cdocsfn1.tex|, |cdocsfn2.tex|
as well as |childdoc.pdf|.

%%%%%%%%%%%%%%%%%%%%%%%%%%%%%%%%%%%%%%%%%%%%%%%%%%%%%%%%%%%%%%%%%%%%%%%%%%%%%%%%
\subsection{Files and Installation}

The package consists of the files:
%
\begin{center}
\begin{tabular}{ll}
    |README.txt|   & readme file \\
    |childdoc.ins| & installation file \\
    |childdoc.dtx| & source file \\
    |childdoc.def| & definition file \\
    |cdocsamp.tex| & sample main file \\
    |cdocsch1.tex| & sample include file \\
    |cdocsch2.tex| & sample include file \\
    |cdocspt3.tex| & sample part file \\
    |cdocspt4.tex| & sample part file \\
    |cdocsdrf.tex| & sample redirection file \\
    |cdocsfn1.tex| & sample redirection file \\
    |cdocsfn2.tex| & sample redirection file \\
    |childdoc.pdf| & manual
\end{tabular}
\end{center}
%
The distribution consists of the files
|README.txt|, |childdoc.ins| and |childdoc.dtx|.
%
\begin{itemize}
\item
Run (pdf)\LaTeX{} on |childdoc.dtx|
to compile the manual |childdoc.pdf| (this file).
\item
Run \LaTeX{} on |childdoc.ins| to create the definitions file |childdoc.def|
and the sample |cdocsamp.tex| with include files
|cdocsch1.tex|, |cdocsch2.tex|, |cdocspt3.tex|, |cdocspt4.tex|,
|cdocsdrf.tex|, |cdocsfn1.tex|, |cdocsfn2.tex|.
Then copy the file |childdoc.def| to an appropriate directory of your \LaTeX{}
distribution, e.g.\ \textit{texmf-root}|/tex/latex/childdoc|.
\end{itemize}

%%%%%%%%%%%%%%%%%%%%%%%%%%%%%%%%%%%%%%%%%%%%%%%%%%%%%%%%%%%%%%%%%%%%%%%%%%%%%%%%
\subsection{Related CTAN Packages}

There are several other packages which offer a similar functionality:
%
\begin{itemize}
\item
The packages
\href{http://ctan.org/pkg/docmute}{\textsf{docmute}},
\href{http://ctan.org/pkg/includex}{\textsf{includex}} and
\href{http://ctan.org/pkg/standalone}{\textsf{standalone}}
provide commands to include only the document body of
a child file thus allowing both files to be compiled individually.
\item
The packages \href{http://ctan.org/pkg/subdocs}{\textsf{subdocs}}
and \href{http://ctan.org/pkg/subfiles}{\textsf{subfiles}}
provide structures in which the main and child documents can be
encapsulated and allowing them to be compiled individually.
The inclusion mechanism is different from the conventional |\include|.
\item
The package \href{http://ctan.org/pkg/combine}{\textsf{combine}}
is an elaborate solution to combine several documents into one.
\end{itemize}
%
See also the CTAN topic \href{http://ctan.org/topic/subdocs}{\textsf{subdocs}}
for further related packages.
The present package differs from the above solutions in that
a document structure constructed with the conventional |\include| mechanism
just needs two extra commands at the top of every file
such that all constituent files can be compiled individually.

%%%%%%%%%%%%%%%%%%%%%%%%%%%%%%%%%%%%%%%%%%%%%%%%%%%%%%%%%%%%%%%%%%%%%%%%%%%%%%%%
%\subsection{Feature Suggestions}
%
%The following is a list of features which may be useful for future
%versions of this package:
%%
%\begin{itemize}
%\item
%\ldots
%\end{itemize}

%%%%%%%%%%%%%%%%%%%%%%%%%%%%%%%%%%%%%%%%%%%%%%%%%%%%%%%%%%%%%%%%%%%%%%%%%%%%%%%%
\subsection{Revision History}

%%%%%%%%%%%%%%%%%%%%%%%%%%%%%%%%%%%%%%%%
\paragraph{v2.0:} 2018/12/30

\begin{itemize}
\item
immediate forward processing
\item
added |\childdocby| mechanism
\item
manual restructured
\end{itemize}

%%%%%%%%%%%%%%%%%%%%%%%%%%%%%%%%%%%%%%%%
\paragraph{v1.6:} 2018/01/17

\begin{itemize}
\item
application for development of include files
\item
corrections to manual
\end{itemize}

%%%%%%%%%%%%%%%%%%%%%%%%%%%%%%%%%%%%%%%%
\paragraph{v1.5:} 2017/05/21

\begin{itemize}
\item
more complete structuring introduced
\item
|\childdocof| introduced
\item
|\childdoc| renamed to |\childdocmain|
\item
|\childredirect| renamed to |\childdocforward| and |\childdocforwardprefix|
and functionality expanded
\end{itemize}

%%%%%%%%%%%%%%%%%%%%%%%%%%%%%%%%%%%%%%%%
\paragraph{v1.0:} 2017/04/27

\begin{itemize}
\item
manual and install package
\item
first version published on CTAN
\end{itemize}

%%%%%%%%%%%%%%%%%%%%%%%%%%%%%%%%%%%%%%%%
\paragraph{v0.6:} 2017/04/26

\begin{itemize}
\item
redirection mechanism added
\end{itemize}

%%%%%%%%%%%%%%%%%%%%%%%%%%%%%%%%%%%%%%%%
\paragraph{v0.5:} 2017/04/26

\begin{itemize}
\item
functionality in definition file
\end{itemize}


%%%%%%%%%%%%%%%%%%%%%%%%%%%%%%%%%%%%%%%%%%%%%%%%%%%%%%%%%%%%%%%%%%%%%%%%%%%%%%%%
%%%%%%%%%%%%%%%%%%%%%%%%%%%%%%%%%%%%%%%%%%%%%%%%%%%%%%%%%%%%%%%%%%%%%%%%%%%%%%%%
%%%%%%%%%%%%%%%%%%%%%%%%%%%%%%%%%%%%%%%%%%%%%%%%%%%%%%%%%%%%%%%%%%%%%%%%%%%%%%%%
\appendix

\settowidth\MacroIndent{\rmfamily\scriptsize 000\ }

 \DocInput{childdoc.dtx}

\end{document}
%</driver>
% \fi
%
% %%%%%%%%%%%%%%%%%%%%%%%%%%%%%%%%%%%%%%%%%%%%%%%%%%%%%%%%%%%%%%%%%%%%%%%%%%%%%%
% %%%%%%%%%%%%%%%%%%%%%%%%%%%%%%%%%%%%%%%%%%%%%%%%%%%%%%%%%%%%%%%%%%%%%%%%%%%%%%
% \section{Sample}
%\iffalse
%<*samplemain>
%\fi
%
% The following presents a sample document
% with two chapters, two parts, a title page,
% a compile flag as well as three forwarding files to set the flag.
% It consists of eight |.tex| files:
% \begin{center}
% \begin{tabular}{ll}
% |cdocsamp.tex|&main file\\
% |cdocsch1.tex|&include file for chapter 1\\
% |cdocsch2.tex|&include file for chapter 2\\
% |cdocspt3.tex|&include file for part 3\\
% |cdocspt4.tex|&include file for part 4\\
% |cdocsdrf.tex|&forwarding file for main file in draft mode\\
% |cdocsfi1.tex|&forwarding file for final version of chapter 1\\
% |cdocsfi2.tex|&forwarding file for final version of chapter 2\\
% \end{tabular}
% \end{center}
% Each of the eight files can be compiled directly by the \LaTeX{} compiler.
%
% %%%%%%%%%%%%%%%%%%%%%%%%%%%%%%%%%%%%%%
% \paragraph{Main File.}
%
% The main file is called |cdocsamp.tex|.
%
% Load the \textsf{childdoc} definitions and
% declare the filename for the main document:
%    \begin{macrocode}
\input{childdoc.def}
\childdocmain{}
%    \end{macrocode}

% Optional override for |\version| flag:
%    \begin{macrocode}
%%\ifchilddoc\else\providecommand{\version}{draft}\fi
%    \end{macrocode}

% Define the default values for the |\version| flag
% (|final| for the main file and |draft| for childs):
%    \begin{macrocode}
\ifchilddoc
\providecommand{\version}{draft}
\else
\providecommand{\version}{final}
\fi
%    \end{macrocode}

% Load the standard document class:
%    \begin{macrocode}
\documentclass[12pt]{article}
%    \end{macrocode}

% Start the document body:
%    \begin{macrocode}
\begin{document}
%    \end{macrocode}

% Declare a title page.
% Print title, part of document being processed and version flag:
%    \begin{macrocode}
\addtocounter{page}{-1}
\begin{center}
{\LARGE\bfseries{}childdoc example\par}
\vspace{1cm}
\ifchilddoc
\ifchilddocmanual part\else chapter\fi:
`\childdocname' of `\childdocjob'\par
\else
main document: `\childdocjob'\par
\fi
version: \version\par
\end{center}
\newpage
%    \end{macrocode}

% Manually include selected file,
% otherwise process as usual:
%    \begin{macrocode}
\ifchilddocmanual
\section*{part `\childdocname'}
\input{\childdocname}
\else
%    \end{macrocode}

% Include the two chapters:
%    \begin{macrocode}
\include{cdocsch1}
\include{cdocsch2}
%    \end{macrocode}

% Include the two parts unless only chapters should be displayed:
%    \begin{macrocode}
\ifchilddoc\else
\section{part three}
\input{cdocspt3}
\section{part four}
\input{cdocspt4}
\fi
%    \end{macrocode}

% Process as usual until here:
%    \begin{macrocode}
\fi
%    \end{macrocode}

% End of document body:
%    \begin{macrocode}
\end{document}
%    \end{macrocode}
%\iffalse
%</samplemain>
%\fi
%
% %%%%%%%%%%%%%%%%%%%%%%%%%%%%%%%%%%%%%%
% \paragraph{Chapter Include Files.}
%
% The include files are called |cdocsch1.tex| and |cdocsch2.tex|.
%
%\iffalse
%<*samplechap1|samplechap2>
%\fi

% Optional override for |\version| flag:
%    \begin{macrocode}
%%\providecommand{\version}{final}
%    \end{macrocode}

% Include the main document:
%    \begin{macrocode}
\input{childdoc.def}
\childdocof{cdocsamp}
%    \end{macrocode}

%\iffalse
%</samplechap1|samplechap2>
%\fi
%
%\iffalse
%<*samplechap1>
%\fi
% Some text for chapter 1:
%    \begin{macrocode}
\section{one}
some text in chapter one
%    \end{macrocode}

%\iffalse
%</samplechap1>
%\fi
% Some text for chapter 2:
%\iffalse
%<*samplechap2>
%\fi
%    \begin{macrocode}
\section{two}
more text in chapter two
%    \end{macrocode}

%\iffalse
%</samplechap2>
%\fi
%
% %%%%%%%%%%%%%%%%%%%%%%%%%%%%%%%%%%%%%%
% \paragraph{Part Include Files.}
%
% The include files are called |cdocspt3.tex| and |cdocspt4.tex|.
%
%\iffalse
%<*samplepart3|samplepart4>
%\fi

% Optional override for |\version| flag:
%    \begin{macrocode}
%%\providecommand{\version}{final}
%    \end{macrocode}

% Include the main document:
%    \begin{macrocode}
\input{childdoc.def}
\childdocby{cdocsamp}
%    \end{macrocode}

%\iffalse
%</samplepart3|samplepart4>
%\fi
%
%\iffalse
%<*samplepart3>
%\fi
% Some text for part 3:
%    \begin{macrocode}
some text in part three
%    \end{macrocode}

%\iffalse
%</samplepart3>
%\fi
% Some text for part 4:
%\iffalse
%<*samplepart4>
%\fi
%    \begin{macrocode}
more text in part four
%    \end{macrocode}

%\iffalse
%</samplepart4>
%\fi
%
% %%%%%%%%%%%%%%%%%%%%%%%%%%%%%%%%%%%%%%
% \paragraph{Forwarding for a Complete Draft.}
%
% The following forwarding file |cdocsdrf.tex|
% compiles the main document in draft mode:
%\iffalse
%<*sampledraft>
%\fi
%    \begin{macrocode}
\def\version{draft}
\input{childdoc.def}
\childdocforward{cdocsamp}
%    \end{macrocode}

%\iffalse
%</sampledraft>
%\fi
%
% %%%%%%%%%%%%%%%%%%%%%%%%%%%%%%%%%%%%%%
% \paragraph{Forwarding for Final Version of the Chapters.}
%
% The following forwarding files |cdocsfn1.tex| and |cdocsfn2.tex|
% (with identical content)
% compile the final versions of the child documents
% |cdocsch1.tex| and |cdocsch2.tex|, respectively:
%\iffalse
%<*samplefinal>
%\fi
%    \begin{macrocode}
\def\version{final}
\input{childdoc.def}
\childdocforwardprefix[cdocsamp]{cdocsfn}{cdocsch}
%    \end{macrocode}

%\iffalse
%</samplefinal>
%\fi
%
% %%%%%%%%%%%%%%%%%%%%%%%%%%%%%%%%%%%%%%
% \paragraph{Command Line Processing.}
%
% The following three command lines generate the output files
% |cdocscld|, |cdocscl1| and |cdocscl2|
% which should be identical to
% |cdocsdrf|, |cdocsch1| and |cdocsfn2|, respectively:
% \begin{center}
% \begin{tabular}{l}
% |latex -jobname cdocscld \|\\
% |  "\def\version{draft}\input{childdoc.def}\childdocforward{cdocsamp}"|\\
% |latex -jobname cdocscl1 \|\\
% |  "\input{childdoc.def}\childdocforward[cdocsamp]{cdocsch1}"|\\
% |latex -jobname cdocscl2 \|\\
% |  "\def\version{final}\input{childdoc.def}\childdocforward{cdocsch2}"|
% \end{tabular}
% \end{center}
% Note that the trailing backslash on each first line
% merely continues the input to the second line
% (for convenient cut ant paste).
% Furthermore, the command |latex| can be replaced by any
% of its alternative versions such as |pdflatex|.
%
% %%%%%%%%%%%%%%%%%%%%%%%%%%%%%%%%%%%%%%%%%%%%%%%%%%%%%%%%%%%%%%%%%%%%%%%%%%%%%%
% %%%%%%%%%%%%%%%%%%%%%%%%%%%%%%%%%%%%%%%%%%%%%%%%%%%%%%%%%%%%%%%%%%%%%%%%%%%%%%
% \section{Implementation}
%\iffalse
%<*package>
%\fi
%
% This section describes the definitions file |childdoc.def|.

% The definitions cannot be loaded using |\usepackage| or |\RequirePackage|
% which has a mechanism to prevent loading a style file more than once.
% When loading the definitions by means of |\input|
% multiple instances have to be prevented manually:
%\iffalse
%This code needs to be before the `\ProvidesFile' directive
%which is defined at the beginning of this file.
%Therefore it is also placed there and commented out here.
%</package>
%<*discard>
%\fi
%    \begin{macrocode}
\ifdefined\childdocmain\endinput\fi
%    \end{macrocode}
%\iffalse
%</discard>
%<*package>
%\fi
%
% \macro{\ifchilddoc}
% \macro{\ifchilddocmanual}
% The conditional |\ifchilddoc| tells whether a
% child (true) or main (false) document is being compiled.
% The conditional |\ifchilddocmanual| tells whether
% the |\includeonly| mechanism is used (false) or
% the selection of child files must be performed manually (true).
% The definitions initialise to false:
%    \begin{macrocode}
\newif\ifchilddoc
\newif\ifchilddocmanual
%    \end{macrocode}

% \macro{\childdocname}
% \macro{\childdocjob}
% The macro |\childdocname| stores the name of the main document
% to be compiled. The macro |\childdocjob| stores the name of
% the document on which the \LaTeX{} compiler was originally invoked.
% The content of |\jobname| cannot be compared
% to filenames specified in the source due to different catcodes.
% The following code rescans |\jobname|, stores the result
% in |\childdocname| and saves a copy in |\childdocjob|:
%    \begin{macrocode}
\edef\childdocname{\scantokens\expandafter{\jobname\noexpand}}
\let\childdocjob\childdocname
%    \end{macrocode}

% \macro{\childdocdisable}
% The macro |\childdocdisable| prevents the main file
% from being processed more than once.
% At this stage, the main document command |\childdocmain|
% is assumed to be called once again where it should do nothing.
% Any subsequent call to it should prevent
% a secondary processing of the main document
% It overwrites the forwarding commands
% |\childdocof| and |\childdocforward|
% with empty macros to prevent further inclusions of the main document:
%    \begin{macrocode}
\newcommand{\childdocdisable}
{
  \renewcommand{\childdocmain}[1]{\renewcommand{\childdocmain}[1]{\endinput}}
  \renewcommand{\childdocof}[1]{}
  \renewcommand{\childdocby}[2][]{}
  \renewcommand{\childdocforward}[2][]{}
  \renewcommand{\childdocdisable}{}
}
%    \end{macrocode}

% \macro{\childdocmain}
% The macro |\childdocmain| is to be called at the top of the main file
% with nothing or the main filename (without extension) as argument.
% First, it breaks loops.
% If the argument is not empty and does not match |\childdocname|
% (which is set by the first inclusion of |childdoc.def|),
% |\ifchilddoc| is set to true, |\includeonly| is applied to the child file
% and |\jobname| is set to the main file
% (for proper handling of |.aux| files):
%    \begin{macrocode}
\newcommand{\childdocmain}[1]
{
  \childdocdisable\childdocmain{}
  \if?#1?\else
    \begingroup
      \def\childdoctmp{#1}
      \ifx\childdoctmp\childdocname
        \def\childdoctmp{}
      \else
        \def\childdoctmp
        {
          \childdoctrue
          \includeonly{\childdocname}
          \def\childdocjob{#1}
          \def\jobname{#1}
        }
      \fi
      \expandafter
    \endgroup
    \childdoctmp
  \fi
}
%    \end{macrocode}

% \macro{\childdocof}
% The command |\childdocof| redirects
% compilation to the main file |#1|.
%    \begin{macrocode}
\newcommand{\childdocof}[1]
{
  \childdocdisable
  \childdoctrue
  \includeonly{\childdocname}
  \def\jobname{#1}
  \def\childdocjob{#1}
  \input{#1}
}
%    \end{macrocode}

% \macro{\childdocby}
% The command |\childdocby| ....
%    \begin{macrocode}
\newcommand{\childdocby}[2][]
{
  \childdocdisable
  \childdoctrue
  \childdocmanualtrue
  \if?#1?\else
    \def\jobname{#2}
  \fi
  \def\childdocjob{#2}
  \input{#2}
  \endinput
}
%    \end{macrocode}

% \macro{\childdocforward}
% The command |\childdocforward| redirects
% compilation to the main file or
% (if the optional argument is given) a child file.
% Parameters are set as if the main file
% or a child file starting with |\childdocof| was compiled.
% Then compilation is handed over to the main file:
%    \begin{macrocode}
\newcommand{\childdocforward}[2][]
{
  \begingroup
    \if?#1?
      \def\childdoctmp
      {
        \def\childdocname{#2}
        \def\childdocjob{#2}
        \def\jobname{#2}
        \input{#2}
        \endinput
      }
    \else
      \def\childdoctmp
      {
        \childdocdisable
        \def\childdocname{#2}
        \childdoctrue
        \includeonly{#2}
        \def\childdocjob{#1}
        \def\jobname{#1}
        \input{#1}
        \endinput
      }
    \fi
    \expandafter
  \endgroup
  \childdoctmp
}
%    \end{macrocode}

% \macro{\childdocforwardprefix}
% The command |\childdocforwardprefix| redirects
% compilation to the main or a child file by means of a pattern.
% The prefix |#1| in the current filename is replaced by |#2|
% and the suffix of the current filename is kept
% (it is assumed that the filename does not contain the substring `|~~~|'
% which is used as a delimiter).
% Compilation is handed over to the new file by |\childdocforward|:
%    \begin{macrocode}
\newcommand{\childdocforwardprefix}[3][]
{
  \begingroup
    \def\childdocextract #2##1~~~{\def\childdoctmp{\childdocforward[#1]{#3##1}}}
    \expandafter\childdocextract\childdocname~~~
    \expandafter
  \endgroup
  \childdoctmp
}
%    \end{macrocode}

% \macro{\childdoc}
% The deprecated macro |\childdoc| is a legacy version of |\childdocmain|:
%    \begin{macrocode}
\newcommand{\childdoc}{\childdocmain}
%    \end{macrocode}

% \macro{\childdocredirect}
% The deprecated macro |\childdocredirect| is a legacy version
% of |\childdocforward| and |\childdocforwardprefix|:
%    \begin{macrocode}
\newcommand{\childdocredirect}[2][]
{
  \begingroup
    \if?#1?
      \def\childdoctmp{\childdocforward{#2}}
    \else
      \def\childdoctmp{\childdocforwardprefix{#1}{#2}}
    \fi
    \expandafter
  \endgroup
  \childdoctmp
}
%    \end{macrocode}

%\iffalse
%</package>
%\fi
%
\endinput
|
and perform the replacements as outlined below.
Instead of |\childdocmain{|\textit{main}|}| add the following code
to the top of the main file:
%
\begin{center}
\begin{tabular}{l}
|\||ifdefined\childdocname\endinput\||fi\newif\ifchilddoc|\\
|\edef\childdocname{\scantokens\expandafter{\jobname\noexpand}}|\\
|\def\childdocmain{|\textit{main}|}\||ifx\childdocmain\childdocname\||else|\\
|\childdoctrue\includeonly{\childdocname}\let\jobname\childdocmain\||fi|\\
\end{tabular}
\end{center}
%
Instead of |\childdocof{|\textit{main}|}| just include the main file
at the top of each child file:
%
\begin{center}
|\input{|\textit{main}|}|
\end{center}
%
A simple redirection |\childdocforward{|\textit{dest}|}| is achieved by:
%
\begin{center}
|\def\jobname{|\textit{dest}|}\input{\jobname}|
\end{center}
%
The redirection with prefix
|\childdocforwardprefix[|\textit{prefix}|]{|\textit{dest}|}|
is accomplished by:
%
\begin{center}
\begin{tabular}{l}
|{\edef\jobname{\scantokens\expandafter{\jobname\noexpand}}|\\
|\def\redirectjob |\textit{prefix}|#1~~~{\gdef\jobname{|\textit{dest}|#1}}|\\
|\expandafter\redirectjob\jobname~~~}\input{\jobname}|
\end{tabular}
\end{center}

In an alternative approach,
child documents can be compiled by a specific command line
without additional code or specific definitions:
%
\begin{center}
|... -jobname "|\textit{target}|" "|[\textit{flags}]%
|\includeonly{|\textit{dest}|}\input{|\textit{main}|}"|
\end{center}
%

%%%%%%%%%%%%%%%%%%%%%%%%%%%%%%%%%%%%%%%%%%%%%%%%%%%%%%%%%%%%%%%%%%%%%%%%%%%%%%%%
%%%%%%%%%%%%%%%%%%%%%%%%%%%%%%%%%%%%%%%%%%%%%%%%%%%%%%%%%%%%%%%%%%%%%%%%%%%%%%%%
\section{Information}

%%%%%%%%%%%%%%%%%%%%%%%%%%%%%%%%%%%%%%%%%%%%%%%%%%%%%%%%%%%%%%%%%%%%%%%%%%%%%%%%
\subsection{Copyright}

Copyright \copyright{} 2017--2018 Niklas Beisert

This work may be distributed and/or modified under the
conditions of the \LaTeX{} Project Public License, either version 1.3
of this license or (at your option) any later version.
The latest version of this license is in
  \url{http://www.latex-project.org/lppl.txt}
and version 1.3 or later is part of all distributions of \LaTeX{}
version 2005/12/01 or later.

This work has the LPPL maintenance status `maintained'.

The Current Maintainer of this work is Niklas Beisert.

This work consists of the files |README.txt|, |childdoc.ins| and |childdoc.dtx|
as well as the derived files |childdoc.def|, |cdocsamp.tex|
with |cdocsch1.tex|, |cdocsch2.tex|, |cdocspt3.tex|, |cdocspt4.tex|,
|cdocsdrf.tex|, |cdocsfn1.tex|, |cdocsfn2.tex|
as well as |childdoc.pdf|.

%%%%%%%%%%%%%%%%%%%%%%%%%%%%%%%%%%%%%%%%%%%%%%%%%%%%%%%%%%%%%%%%%%%%%%%%%%%%%%%%
\subsection{Files and Installation}

The package consists of the files:
%
\begin{center}
\begin{tabular}{ll}
    |README.txt|   & readme file \\
    |childdoc.ins| & installation file \\
    |childdoc.dtx| & source file \\
    |childdoc.def| & definition file \\
    |cdocsamp.tex| & sample main file \\
    |cdocsch1.tex| & sample include file \\
    |cdocsch2.tex| & sample include file \\
    |cdocspt3.tex| & sample part file \\
    |cdocspt4.tex| & sample part file \\
    |cdocsdrf.tex| & sample redirection file \\
    |cdocsfn1.tex| & sample redirection file \\
    |cdocsfn2.tex| & sample redirection file \\
    |childdoc.pdf| & manual
\end{tabular}
\end{center}
%
The distribution consists of the files
|README.txt|, |childdoc.ins| and |childdoc.dtx|.
%
\begin{itemize}
\item
Run (pdf)\LaTeX{} on |childdoc.dtx|
to compile the manual |childdoc.pdf| (this file).
\item
Run \LaTeX{} on |childdoc.ins| to create the definitions file |childdoc.def|
and the sample |cdocsamp.tex| with include files
|cdocsch1.tex|, |cdocsch2.tex|, |cdocspt3.tex|, |cdocspt4.tex|,
|cdocsdrf.tex|, |cdocsfn1.tex|, |cdocsfn2.tex|.
Then copy the file |childdoc.def| to an appropriate directory of your \LaTeX{}
distribution, e.g.\ \textit{texmf-root}|/tex/latex/childdoc|.
\end{itemize}

%%%%%%%%%%%%%%%%%%%%%%%%%%%%%%%%%%%%%%%%%%%%%%%%%%%%%%%%%%%%%%%%%%%%%%%%%%%%%%%%
\subsection{Related CTAN Packages}

There are several other packages which offer a similar functionality:
%
\begin{itemize}
\item
The packages
\href{http://ctan.org/pkg/docmute}{\textsf{docmute}},
\href{http://ctan.org/pkg/includex}{\textsf{includex}} and
\href{http://ctan.org/pkg/standalone}{\textsf{standalone}}
provide commands to include only the document body of
a child file thus allowing both files to be compiled individually.
\item
The packages \href{http://ctan.org/pkg/subdocs}{\textsf{subdocs}}
and \href{http://ctan.org/pkg/subfiles}{\textsf{subfiles}}
provide structures in which the main and child documents can be
encapsulated and allowing them to be compiled individually.
The inclusion mechanism is different from the conventional |\include|.
\item
The package \href{http://ctan.org/pkg/combine}{\textsf{combine}}
is an elaborate solution to combine several documents into one.
\end{itemize}
%
See also the CTAN topic \href{http://ctan.org/topic/subdocs}{\textsf{subdocs}}
for further related packages.
The present package differs from the above solutions in that
a document structure constructed with the conventional |\include| mechanism
just needs two extra commands at the top of every file
such that all constituent files can be compiled individually.

%%%%%%%%%%%%%%%%%%%%%%%%%%%%%%%%%%%%%%%%%%%%%%%%%%%%%%%%%%%%%%%%%%%%%%%%%%%%%%%%
%\subsection{Feature Suggestions}
%
%The following is a list of features which may be useful for future
%versions of this package:
%%
%\begin{itemize}
%\item
%\ldots
%\end{itemize}

%%%%%%%%%%%%%%%%%%%%%%%%%%%%%%%%%%%%%%%%%%%%%%%%%%%%%%%%%%%%%%%%%%%%%%%%%%%%%%%%
\subsection{Revision History}

%%%%%%%%%%%%%%%%%%%%%%%%%%%%%%%%%%%%%%%%
\paragraph{v2.0:} 2018/12/30

\begin{itemize}
\item
immediate forward processing
\item
added |\childdocby| mechanism
\item
manual restructured
\end{itemize}

%%%%%%%%%%%%%%%%%%%%%%%%%%%%%%%%%%%%%%%%
\paragraph{v1.6:} 2018/01/17

\begin{itemize}
\item
application for development of include files
\item
corrections to manual
\end{itemize}

%%%%%%%%%%%%%%%%%%%%%%%%%%%%%%%%%%%%%%%%
\paragraph{v1.5:} 2017/05/21

\begin{itemize}
\item
more complete structuring introduced
\item
|\childdocof| introduced
\item
|\childdoc| renamed to |\childdocmain|
\item
|\childredirect| renamed to |\childdocforward| and |\childdocforwardprefix|
and functionality expanded
\end{itemize}

%%%%%%%%%%%%%%%%%%%%%%%%%%%%%%%%%%%%%%%%
\paragraph{v1.0:} 2017/04/27

\begin{itemize}
\item
manual and install package
\item
first version published on CTAN
\end{itemize}

%%%%%%%%%%%%%%%%%%%%%%%%%%%%%%%%%%%%%%%%
\paragraph{v0.6:} 2017/04/26

\begin{itemize}
\item
redirection mechanism added
\end{itemize}

%%%%%%%%%%%%%%%%%%%%%%%%%%%%%%%%%%%%%%%%
\paragraph{v0.5:} 2017/04/26

\begin{itemize}
\item
functionality in definition file
\end{itemize}


%%%%%%%%%%%%%%%%%%%%%%%%%%%%%%%%%%%%%%%%%%%%%%%%%%%%%%%%%%%%%%%%%%%%%%%%%%%%%%%%
%%%%%%%%%%%%%%%%%%%%%%%%%%%%%%%%%%%%%%%%%%%%%%%%%%%%%%%%%%%%%%%%%%%%%%%%%%%%%%%%
%%%%%%%%%%%%%%%%%%%%%%%%%%%%%%%%%%%%%%%%%%%%%%%%%%%%%%%%%%%%%%%%%%%%%%%%%%%%%%%%
\appendix

\settowidth\MacroIndent{\rmfamily\scriptsize 000\ }

 \DocInput{childdoc.dtx}

\end{document}
%</driver>
% \fi
%
% %%%%%%%%%%%%%%%%%%%%%%%%%%%%%%%%%%%%%%%%%%%%%%%%%%%%%%%%%%%%%%%%%%%%%%%%%%%%%%
% %%%%%%%%%%%%%%%%%%%%%%%%%%%%%%%%%%%%%%%%%%%%%%%%%%%%%%%%%%%%%%%%%%%%%%%%%%%%%%
% \section{Sample}
%\iffalse
%<*samplemain>
%\fi
%
% The following presents a sample document
% with two chapters, two parts, a title page,
% a compile flag as well as three forwarding files to set the flag.
% It consists of eight |.tex| files:
% \begin{center}
% \begin{tabular}{ll}
% |cdocsamp.tex|&main file\\
% |cdocsch1.tex|&include file for chapter 1\\
% |cdocsch2.tex|&include file for chapter 2\\
% |cdocspt3.tex|&include file for part 3\\
% |cdocspt4.tex|&include file for part 4\\
% |cdocsdrf.tex|&forwarding file for main file in draft mode\\
% |cdocsfi1.tex|&forwarding file for final version of chapter 1\\
% |cdocsfi2.tex|&forwarding file for final version of chapter 2\\
% \end{tabular}
% \end{center}
% Each of the eight files can be compiled directly by the \LaTeX{} compiler.
%
% %%%%%%%%%%%%%%%%%%%%%%%%%%%%%%%%%%%%%%
% \paragraph{Main File.}
%
% The main file is called |cdocsamp.tex|.
%
% Load the \textsf{childdoc} definitions and
% declare the filename for the main document:
%    \begin{macrocode}
% \iffalse
%
% childdoc.dtx Copyright (C) 2017-2018 Niklas Beisert
%
% This work may be distributed and/or modified under the
% conditions of the LaTeX Project Public License, either version 1.3
% of this license or (at your option) any later version.
% The latest version of this license is in
%   http://www.latex-project.org/lppl.txt
% and version 1.3 or later is part of all distributions of LaTeX
% version 2005/12/01 or later.
%
% This work has the LPPL maintenance status `maintained'.
%
% The Current Maintainer of this work is Niklas Beisert.
%
% This work consists of the files childdoc.dtx and childdoc.ins
% and the derived files childdoc.def and cdocsamp.tex with
% cdocsch1.tex, cdocsch2.tex, cdocsdrf.tex, cdocsfn1.tex, cdocsfn2.tex.
%
%<package>\ifdefined\childdocmain\endinput\fi
%<package>\ProvidesFile{childdoc.def}[2018/12/30 v2.0 child document driver]
%<samplemain>\ProvidesFile{cdocsamp.tex}[2018/12/30 v2.0 sample for childdoc]
%<*driver>
%\ProvidesFile{childdoc.drv}[2018/12/30 v2.0 childdoc reference manual file]
\PassOptionsToClass{10pt,a4paper}{article}
\documentclass{ltxdoc}

\usepackage[margin=35mm]{geometry}
\usepackage{hyperref}
\usepackage{hyperxmp}
\usepackage[usenames]{color}

\hypersetup{colorlinks=true}
\hypersetup{pdfstartview=FitH}
\hypersetup{pdfpagemode=UseNone}
\hypersetup{pdfsource={}}
\hypersetup{pdflang={en-UK}}
\hypersetup{pdfcopyright={Copyright 2017-2018 Niklas Beisert.
  This work may be distributed and/or modified under the
  conditions of the LaTeX Project Public License, either version 1.3
  of this license or (at your option) any later version.}}
\hypersetup{pdflicenseurl={http://www.latex-project.org/lppl.txt}}
\hypersetup{pdfcontactaddress={ETH Zurich, ITP, HIT K,
  Wolfgang-Pauli-Strasse 27}}
\hypersetup{pdfcontactpostcode={8093}}
\hypersetup{pdfcontactcity={Zurich}}
\hypersetup{pdfcontactcountry={Switzerland}}
\hypersetup{pdfcontactemail={nbeisert@itp.phys.ethz.ch}}
\hypersetup{pdfcontacturl={http://people.phys.ethz.ch/\xmptilde nbeisert/}}

\newcommand{\secref}[1]{\hyperref[#1]{section \ref*{#1}}}

\parskip1ex
\parindent0pt
\let\olditemize\itemize
\def\itemize{\olditemize\parskip0pt}

\begin{document}

\title{The \textsf{childdoc} Package}
\hypersetup{pdftitle={The childdoc Package}}
\author{Niklas Beisert\\[2ex]
  Institut f\"ur Theoretische Physik\\
  Eidgen\"ossische Technische Hochschule Z\"urich\\
  Wolfgang-Pauli-Strasse 27, 8093 Z\"urich, Switzerland\\[1ex]
  \href{mailto:nbeisert@itp.phys.ethz.ch}
  {\texttt{nbeisert@itp.phys.ethz.ch}}}
\hypersetup{pdfauthor={Niklas Beisert}}
\hypersetup{pdfsubject={Manual for the LaTeX2e Package childdoc}}
\date{30 December 2018, \textsf{v2.0}}
\maketitle

\begin{abstract}\noindent
\textsf{childdoc} is a \LaTeXe{} package
that enables the direct compilation
of document sections included by |\include|
to individual files.
\end{abstract}

\begingroup
\parskip0ex
\tableofcontents
\endgroup

%%%%%%%%%%%%%%%%%%%%%%%%%%%%%%%%%%%%%%%%%%%%%%%%%%%%%%%%%%%%%%%%%%%%%%%%%%%%%%%%
%%%%%%%%%%%%%%%%%%%%%%%%%%%%%%%%%%%%%%%%%%%%%%%%%%%%%%%%%%%%%%%%%%%%%%%%%%%%%%%%
\section{Introduction}

\LaTeX{} provides a mechanism to structure a large document (such as a book)
into a main file and several child files (containing the chapters)
using the |\include| command.
This mechanism is beneficial for documents
which span hundreds of pages in order to
make the source file(s) more manageable.
Moreover, compilation can be restricted to
selected child files by means of the |\includeonly| command.
The latter feature can be used to reduce the compilation time while editing
(this was significantly more useful in the earlier days of \LaTeX{})
or to generate a smaller document which is easier to navigate.
Another application of |\includeonly| is to generate
documents consisting of selected parts of the complete document.

However, there are a few drawbacks of the plain |\include| mechanism:
\begin{itemize}
\item
The child files cannot be compiled on their own,
they can only be compiled via the main file.
A naive editing environment
(such as a text editor with an option
to have the current file processed by \LaTeX)
may require one to switch to the main file before compiling;
attempting to compile the child file produces errors.
\item
The main file must be modified (each time)
to adjust the |\includeonly| command
to the present needs. This easily leaves the main file in a messy state.
\item
The generated document will always carry the filename
of the main document. This is inconvenient if
several child files are to be compiled and
to be kept for distribution.
\end{itemize}

The present package provides a simple interface
to make child files individually compilable by \LaTeX{}.
Compiling a child file then has the same effect as compiling
the main file with an |\includeonly| command
to select the appropriate child.
Moreover the generated document will carry the name of the child
rather than the main file.
This resolves all three above issues.

This feature is meant to make the editing of books,
thesis documents and lecture notes somewhat more convenient.
However, the package can also be used efficiently for
composing a series of documents (such as exercise sheets)
which are typically distributed individually.
It then assists the author in generating the individual documents
(potentially in different versions)
as well as a document containing the collected series.
Another application is in developing style files
or other kinds of included material
where compilation of the style file could redirect
to a sample or test file.

%%%%%%%%%%%%%%%%%%%%%%%%%%%%%%%%%%%%%%%%%%%%%%%%%%%%%%%%%%%%%%%%%%%%%%%%%%%%%%%%
%%%%%%%%%%%%%%%%%%%%%%%%%%%%%%%%%%%%%%%%%%%%%%%%%%%%%%%%%%%%%%%%%%%%%%%%%%%%%%%%
\section{Usage}

First of all, the package \textsf{childdoc} is \emph{not} a standard
\LaTeXe{} |.sty| style file! Therefore it needs to be invoked in
a non-standard way.

%%%%%%%%%%%%%%%%%%%%%%%%%%%%%%%%%%%%%%%%%%%%%%%%%%%%%%%%%%%%%%%%%%%%%%%%%%%%%%%%
\subsection{Included Files}
\label{sec:include}

%%%%%%%%%%%%%%%%%%%%%%%%%%%%%%%%%%%%%%%%
\DescribeMacro{\childdocmain}
To use the package, add the commands
\begin{center}
\begin{tabular}{l}
|\input{childdoc.def}|\\
|\childdocmain{}|\\
\end{tabular}
\end{center}
at the very top of the main \LaTeX{} file,
in particular \emph{before} the |\documentclass| statement!
The argument of |\childdocmain| should be left empty
(but it must be present).

%%%%%%%%%%%%%%%%%%%%%%%%%%%%%%%%%%%%%%%%
\DescribeMacro{\childdocof}
Furthermore, add the commands
\begin{center}
\begin{tabular}{l}
|\input{childdoc.def}|\\
|\childdocof{|\textit{main}|}|\\
\end{tabular}
\end{center}
at the top of every child file \textit{child}
which is included by |\include{|\textit{child}|}|
from within the main file
(or at least for those files to be compiled individually).
The argument \textit{main} must be the filename of the main file.

There are a couple of
considerations in setting up the main and child documents:

%%%%%%%%%%%%%%%%%%%%%%%%%%%%%%%%%%%%%%%%
\paragraph{Restrictions.}

Please note the following restrictions:
\begin{itemize}
\item
|\childdocmain| must be called with one argument \textit{main}
to ensure compatibility with earlier version of the package.
It must either be empty (|\childdocmain{}|)
or precisely match the filename of the main file in which it is specified.
See \secref{sec:detection} for further information.
\item
The filename \textit{main} must be specified without the |.tex| extension.
\item
The filename \textit{main} is case sensitive
(even in case-insensitive file systems)
due to internal string comparison.
\item
The argument \textit{main} should be fully expanded, it cannot be a macro.
\item
Subdirectories and special characters should be avoided in filenames.
\item
The command |\childdocmain{|\textit{main}|}| must be followed by a whitespace.
It should not be followed immediately by another command
or by a comment mark `|%|'.
This is because the \TeX{} parser reads the token immediately following
the argument of |\childdocmain| and puts it
at the beginning of every child section;
however, a white\-space is ignored.
\end{itemize}

%%%%%%%%%%%%%%%%%%%%%%%%%%%%%%%%%%%%%%%%
\paragraph{Content of Main File.}

It is advisable to place all content in the child files included by |\include|.
Any output contained in the main file will appear in all child documents
unless suppressed manually;
it cannot be suppressed automatically by the |\includeonly| directive
and thus should normally be avoided.
A method to include some content in the main file
by means of conditional processing is described in \secref{sec:conditional}.

%%%%%%%%%%%%%%%%%%%%%%%%%%%%%%%%%%%%%%%%
\paragraph{Page Numbering.}

When only a part of the document is compiled,
the appropriate numbering of pages
(as well as other status parameters)
is determined from the |.aux| files.
The latter contain information from previous passes.
However this information needs to propagate through
all intermediate child documents.
Therefore the page numbering in child documents may well
be inconsistent until the complete document is compiled at least once.

A useful (if unconventional) way to always ensure a consistent
page numbering is to restart the numbering in each child document
and denote the pages by `\textit{child}|.|\textit{page}'
where \textit{child} represents the chapter/section number of the child file.
This can be achieved by the command
|\numberwithin{page}{|\textit{child}|}|
of the \textsf{amsmath} package
where \textit{child} can be |chapter| or |section|
depending on the chosen structuring.
Alternatively, one can modify the macro |\thepage| appropriately
and reset the counter |page| at the start of each child file.

%%%%%%%%%%%%%%%%%%%%%%%%%%%%%%%%%%%%%%%%%%%%%%%%%%%%%%%%%%%%%%%%%%%%%%%%%%%%%%%%
\subsection{Conditional Processing}
\label{sec:conditional}

The package provides a mechanism to compile different versions
of a document. To customise the versions further some conditional processing
can come in handy to distinguish which version is being compiled.
The package provides two macros to describe the compilation context:

%%%%%%%%%%%%%%%%%%%%%%%%%%%%%%%%%%%%%%%%
\DescribeMacro{\ifchilddoc}
The conditional |\ifchilddoc| distinguishes between the compilation of
child documents and the main document:
%
\begin{center}
|\ifchilddoc |\textit{child-code}| |[|\||else |\textit{main-code}]| \||fi|
\end{center}

%%%%%%%%%%%%%%%%%%%%%%%%%%%%%%%%%%%%%%%%
\DescribeMacro{\childdocname}
\DescribeMacro{\childdocjob}
The macro |\childdocname| contains the filename (without extension)
of the main or child file being processed.
Note that |\childdocjob| will always contain the name of the main file.

%%%%%%%%%%%%%%%%%%%%%%%%%%%%%%%%%%%%%%%%
\paragraph{Title Page.}

Conditional processing can be used to include a title or banner page
in the main document when proper precautions are taken.
Importantly, the code in the main file should ensure that the page counter
(as well as other status parameters which are stored in the |.aux| files)
takes the same value after the conditional processing.
Otherwise the page numbers may take divergent values
depending on which part is compiled.

For example, a title page could be declared by:
%
\begin{center}
\begin{tabular}{l}
|\ifchilddoc\||else|\\
|\addtocounter{page}{-1}|\\
\textit{code for title page}\\
|\newpage|\\
|\||fi|
\end{tabular}
\end{center}
%
A banner page for the child documents can be generated by:
%
\begin{center}
\begin{tabular}{l}
|\ifchilddoc|\\
|\addtocounter{page}{-1}|\\
\textit{code for banner page}\\
|\newpage|\\
|\||fi|
\end{tabular}
\end{center}
%
Here one could write a message such as:
\begin{center}
|This is the part \childdocname{} of \childdocjob{}.|
\end{center}

%%%%%%%%%%%%%%%%%%%%%%%%%%%%%%%%%%%%%%%%%%%%%%%%%%%%%%%%%%%%%%%%%%%%%%%%%%%%%%%%
\subsection{Flags}
\label{sec:flags}

The package makes it easy to generate different versions
of the main or child documents.
To this end compilation flags can be defined
and assigned different default values.
They will be particularly useful in conjunction
with the forwarding mechanism described in \secref{sec:forward}.

For example, it may be useful to have a flag |\version|
which can be set to |draft| or |final|.
The document source will contain some conditional code
depending on the value of |\version|.
Suppose further, the flag should default to |final| for the main file
and to |draft| for child files
which is a natural assignment for editing the document.
This is achieved by placing the following code
in the preamble of the main document
(below the |\childdocmain| directive):
%
\begin{center}
\begin{tabular}{l}
|\ifchilddoc|\\
|\providecommand{\version}{draft}|\\
|\||else|\\
|\providecommand{\version}{final}|\\
|\||fi|
\end{tabular}
\end{center}
%
The definition by |\providecommand| makes sure
that previous definitions are not overwritten.
Further statements |\providecommand{\version}{...}|
can thus be added before the above code to override it.

For the main file, one might add a line
(between |\childdocmain| and the above block)
%
\begin{center}
|%\ifchilddoc\||else\providecommand{\version}{draft}\||fi|
\end{center}
%
which can be uncommented to produce a draft version.
Likewise one can add a line to the very top of a child file
(above the |\childdocof{|\textit{main}|}| directive)
%
\begin{center}
|%\providecommand{\version}{final}|
\end{center}
%
which can be uncommented to produce the final version of this child document.

%%%%%%%%%%%%%%%%%%%%%%%%%%%%%%%%%%%%%%%%%%%%%%%%%%%%%%%%%%%%%%%%%%%%%%%%%%%%%%%%
\subsection{Forwarding}
\label{sec:forward}

Different versions of the main or child documents
using compilation flags as described in \secref{sec:flags}
can be (permanently) stored in different files
for convenient compilation, viewing and distribution.
To this end, the package defines a command
to pass on compilation to a different file:

%%%%%%%%%%%%%%%%%%%%%%%%%%%%%%%%%%%%%%%%
\DescribeMacro{\childdocforward}
The command |\childdocforward| redirects processing to
another source file:
%
\begin{center}
\begin{tabular}{l}
|\input{childdoc.def}|\\
|\childdocforward[|\textit{main}|]{|\textit{dest}|}|\\
\end{tabular}
\end{center}
%
The argument \textit{dest} is the destination file
(without extension).
It should be the main file or one of the child files.
Note that further \textsf{childdoc} directives
such as |\childdocof| and |\childdocforward|
in the indicated file will be processed in this form.
The optional argument \textit{main}
passes on directly to the main file \textit{main}
while pretending to compile the child \textit{dest}.
This form behaves as if \textit{dest}
issues |\childdocof{|\textit{main}|}| right away,
and no further \textsf{childdoc} directives will be processed.

%%%%%%%%%%%%%%%%%%%%%%%%%%%%%%%%%%%%%%%%
\DescribeMacro{\...prefix}
In the alternative form |\childdocforwardprefix|,
%
\begin{center}
\begin{tabular}{l}
|\input{childdoc.def}|\\
|\childdocforwardprefix[|\textit{main}|]{|\textit{prefix}|}{|\textit{dest}|}|
\end{tabular}
\end{center}
%
the destination file is determined by a pattern
depending on the current file:
To make this work, the current file must be called
`{\textit{prefix}\hspace{0.2em}\textit{suffix}}'
with \textit{prefix} matching precisely the argument.
Processing is then passed on to the file
`{\textit{dest}\hspace{0.2em}\textit{suffix}}'.
Surely, the same effect is achieved by
directly specifying the
argument `{\textit{dest}\hspace{0.2em}\textit{suffix}}'
in the first form.
However, that requires to set up a different file
for each child. With the alternative form of the command
all these files can have exactly the same content
which simplifies setting them up and maintaining them.

For example, the following file |draft.tex|
with a compilation flag |\version| as described in \secref{sec:flags}
compiles the main document as a draft:
%
\begin{center}
\begin{tabular}{l}
|\def\version{draft}|\\
|\input{childdoc.def}|\\
|\childdocforward{|\textit{main}|}|
\end{tabular}
\end{center}
%
Likewise, the following files |final|\textit{nn}|.tex|
compile the final version of the child document
|child|\textit{nn}|.tex|:
%
\begin{center}
\begin{tabular}{l}
|\def\version{final}|\\
|\input{childdoc.def}|\\
|\childdocforwardprefix{final}{child}|
\end{tabular}
\end{center}
%

Note that when several versions of a main file and/or of each child file
are to be generated, it may be convenient to set up a |Makefile| or
shell script to automatise the process.

%%%%%%%%%%%%%%%%%%%%%%%%%%%%%%%%%%%%%%%%%%%%%%%%%%%%%%%%%%%%%%%%%%%%%%%%%%%%%%%%
\subsection{Command Line Processing}
\label{sec:commandline}

The effect of redirection files can also be achieved by invoking
the \LaTeX{} compiler with a more elaborate command line.
Most conveniently this should be done as part
of a shell script or a |Makefile|.

When using \textsf{childdoc} in the main file, the following
command lines effectively perform a redirection
(note that depending on the shell being used,
backslashes may have to be doubled: `|\|' $\to$ `|\\|'):
%
\begin{center}
|... -jobname "|\textit{target}|" |\\|"|[\textit{flags}]%
|\input{childdoc.def}\childdocforward[|\textit{main}|]{|\textit{dest}|}"|
\end{center}
%
Here \textit{target} is the name of the output file,
\textit{main} is the name of the main file
and \textit{dest} is the name of the main or child file to be processed
(all filenames without extensions).
The optional argument \textit{main} can be omitted
if \textit{main} matches \textit{dest}.
Optionally, compilation \textit{flags} can be defined via |\def| commands.
This command line makes the \TeX{} engine believe
it is compiling the file \textit{target}
whose content is specified as the latter parameter.
The provided code then forwards the processing to
\textit{main} or \textit{dest} as described in \secref{sec:forward}.

%%%%%%%%%%%%%%%%%%%%%%%%%%%%%%%%%%%%%%%%%%%%%%%%%%%%%%%%%%%%%%%%%%%%%%%%%%%%%%%%
\subsection{Include by Input}
\label{sec:input}

Including child documents by |\include| has some restrictions by design.
Most notably, the content of a child document always occupies
its own set of pages; pages cannot be shared between child documents.
Usually, this behaviour makes perfect sense
because each child document contain an essential part of the document.
However, in some situations it may be desirable to compose
a document from a collection of parts
without having mandatory page breaks between then.
For this case, the package
provides a mechanism to include parts
by |\input| which can also be processed individually.
However, by construction this mechanism
requires manual handling of the content to be output.

%%%%%%%%%%%%%%%%%%%%%%%%%%%%%%%%%%%%%%%%
\DescribeMacro{\ifchilddocmanual}
The main file should be prepared as usual, see \secref{sec:include}.
However, the document body must make a distinction
between processing of an individual part and of the main document, e.g.:
%
\begin{center}
\begin{tabular}{l}
|\ifchilddocmanual|\\
|\input{\childdocname}|\\
|\||else|\\
\textit{document body with }|\input{|\textit{part}|}|\\
|\||fi|
\end{tabular}
\end{center}
%
The conditional |\ifchilddocmanual| is true whenever
a part to be included by |\input| is being compiled,
and the name of the part is stored in |\childdocname|.

%%%%%%%%%%%%%%%%%%%%%%%%%%%%%%%%%%%%%%%%
\DescribeMacro{\childdocby}
Each part to be included by |\input| should start with:
%
\begin{center}
\begin{tabular}{l}
|\input{childdoc.def}|\\
|\childdocby{|\textit{main}|}|\\
\end{tabular}
\end{center}
%
The directive |\childdocby| is similar to |\childdocof|
described in \secref{sec:include},
but the subsequent selection of content must be done manually.
To that end, both |\ifchilddoc| and |\ifchilddocmanual|
will be true upon processing of a part,
and the name of the part is stored in |\childdocname|.
Note that |\jobname| will be set to the filename of the current part
so that each part receives an individual |.aux| file
that does not interfere with the |.aux| file(s) of the main document.
This behaviour can be altered by the alternative form
|\childdocby[*]{|\textit{main}|}| (with a non-empty optional argument)
which uses the |.aux| file of the main document
by setting |\jobname| to \textit{main}.

%%%%%%%%%%%%%%%%%%%%%%%%%%%%%%%%%%%%%%%%%%%%%%%%%%%%%%%%%%%%%%%%%%%%%%%%%%%%%%%%
\subsection{Driver Development}
\label{sec:driver}

The \textsf{childdoc} mechanism can also be use for the development
of definition files such as \LaTeX{} styles or classes.
This case differs from the above setup with multiple parts
included by |\include| in that no |\includeonly| should be invoked.
This can be achieved by starting the include file
(before |\ProvidesPackage|) with:
%
\begin{center}
\begin{tabular}{l}
|\input{childdoc.def}|\\
|\childdocforward{|\textit{main}|}|\\
\end{tabular}
\end{center}
%
or alternatively with:
%
\begin{center}
\begin{tabular}{l}
|\input{childdoc.def}|\\
|\childdocby{|\textit{main}|}|\\
\end{tabular}
\end{center}
%
Both forms have slightly different effects as described above.
The main file is prepared as usual, see \secref{sec:include}.

%%%%%%%%%%%%%%%%%%%%%%%%%%%%%%%%%%%%%%%%%%%%%%%%%%%%%%%%%%%%%%%%%%%%%%%%%%%%%%%%
\subsection{Legacy Detection}
\label{sec:detection}

The directive |\childdocmain| in the main file can detect
whether the complete document or merely a child is to be compiled
even without using the directive |\childdocof|.
This method is deprecated because it is less robust
and there is no compelling reason to use it;
it is merely provided for backward compatibility
and it may be removed in future versions.

If the detection mechanism is to be used,
it is mandatory to correctly specify
the filename of the main file as the argument of |\childdocmain|:
%
\begin{center}
\begin{tabular}{l}
|\input{childdoc.def}|\\
|\childdocmain{|\textit{main}|}|\\
\end{tabular}
\end{center}
%
If |\jobname| does not match the argument \textit{main} of |\childdocmain|,
it is assumed that |\jobname| points to the child file to be compiled.
When using |\childdocmain| with the main file specified as argument,
it suffices to start a child file
with just |\input{|\textit{main}|}|
without loading of the package and using |\childdocof|.
If instead all processing is done
with the appropriate \textsf{childdoc} directives,
the argument of \textit{main} of |\childdocmain| can be empty.

An alternative version of the command line processing described
in \secref{sec:commandline} using the detection mechanism reads:
%
\begin{center}
|... -jobname "|\textit{target}|" "|[\textit{flags}]%
[|\def\jobname{|\textit{dest}|}|]|\input{|\textit{main}|}"|
\end{center}

%%%%%%%%%%%%%%%%%%%%%%%%%%%%%%%%%%%%%%%%%%%%%%%%%%%%%%%%%%%%%%%%%%%%%%%%%%%%%%%%
\subsection{Manual Code}
\label{sec:manual}

In case one cannot be certain whether the definitions file |childdoc.def|
is installed on the target \TeX{} distribution
and one prefers not to ship it,
it is conceivable to paste a few relevant commands into the sources.

To that end, drop all statements |\input{childdoc.def}|
and perform the replacements as outlined below.
Instead of |\childdocmain{|\textit{main}|}| add the following code
to the top of the main file:
%
\begin{center}
\begin{tabular}{l}
|\||ifdefined\childdocname\endinput\||fi\newif\ifchilddoc|\\
|\edef\childdocname{\scantokens\expandafter{\jobname\noexpand}}|\\
|\def\childdocmain{|\textit{main}|}\||ifx\childdocmain\childdocname\||else|\\
|\childdoctrue\includeonly{\childdocname}\let\jobname\childdocmain\||fi|\\
\end{tabular}
\end{center}
%
Instead of |\childdocof{|\textit{main}|}| just include the main file
at the top of each child file:
%
\begin{center}
|\input{|\textit{main}|}|
\end{center}
%
A simple redirection |\childdocforward{|\textit{dest}|}| is achieved by:
%
\begin{center}
|\def\jobname{|\textit{dest}|}\input{\jobname}|
\end{center}
%
The redirection with prefix
|\childdocforwardprefix[|\textit{prefix}|]{|\textit{dest}|}|
is accomplished by:
%
\begin{center}
\begin{tabular}{l}
|{\edef\jobname{\scantokens\expandafter{\jobname\noexpand}}|\\
|\def\redirectjob |\textit{prefix}|#1~~~{\gdef\jobname{|\textit{dest}|#1}}|\\
|\expandafter\redirectjob\jobname~~~}\input{\jobname}|
\end{tabular}
\end{center}

In an alternative approach,
child documents can be compiled by a specific command line
without additional code or specific definitions:
%
\begin{center}
|... -jobname "|\textit{target}|" "|[\textit{flags}]%
|\includeonly{|\textit{dest}|}\input{|\textit{main}|}"|
\end{center}
%

%%%%%%%%%%%%%%%%%%%%%%%%%%%%%%%%%%%%%%%%%%%%%%%%%%%%%%%%%%%%%%%%%%%%%%%%%%%%%%%%
%%%%%%%%%%%%%%%%%%%%%%%%%%%%%%%%%%%%%%%%%%%%%%%%%%%%%%%%%%%%%%%%%%%%%%%%%%%%%%%%
\section{Information}

%%%%%%%%%%%%%%%%%%%%%%%%%%%%%%%%%%%%%%%%%%%%%%%%%%%%%%%%%%%%%%%%%%%%%%%%%%%%%%%%
\subsection{Copyright}

Copyright \copyright{} 2017--2018 Niklas Beisert

This work may be distributed and/or modified under the
conditions of the \LaTeX{} Project Public License, either version 1.3
of this license or (at your option) any later version.
The latest version of this license is in
  \url{http://www.latex-project.org/lppl.txt}
and version 1.3 or later is part of all distributions of \LaTeX{}
version 2005/12/01 or later.

This work has the LPPL maintenance status `maintained'.

The Current Maintainer of this work is Niklas Beisert.

This work consists of the files |README.txt|, |childdoc.ins| and |childdoc.dtx|
as well as the derived files |childdoc.def|, |cdocsamp.tex|
with |cdocsch1.tex|, |cdocsch2.tex|, |cdocspt3.tex|, |cdocspt4.tex|,
|cdocsdrf.tex|, |cdocsfn1.tex|, |cdocsfn2.tex|
as well as |childdoc.pdf|.

%%%%%%%%%%%%%%%%%%%%%%%%%%%%%%%%%%%%%%%%%%%%%%%%%%%%%%%%%%%%%%%%%%%%%%%%%%%%%%%%
\subsection{Files and Installation}

The package consists of the files:
%
\begin{center}
\begin{tabular}{ll}
    |README.txt|   & readme file \\
    |childdoc.ins| & installation file \\
    |childdoc.dtx| & source file \\
    |childdoc.def| & definition file \\
    |cdocsamp.tex| & sample main file \\
    |cdocsch1.tex| & sample include file \\
    |cdocsch2.tex| & sample include file \\
    |cdocspt3.tex| & sample part file \\
    |cdocspt4.tex| & sample part file \\
    |cdocsdrf.tex| & sample redirection file \\
    |cdocsfn1.tex| & sample redirection file \\
    |cdocsfn2.tex| & sample redirection file \\
    |childdoc.pdf| & manual
\end{tabular}
\end{center}
%
The distribution consists of the files
|README.txt|, |childdoc.ins| and |childdoc.dtx|.
%
\begin{itemize}
\item
Run (pdf)\LaTeX{} on |childdoc.dtx|
to compile the manual |childdoc.pdf| (this file).
\item
Run \LaTeX{} on |childdoc.ins| to create the definitions file |childdoc.def|
and the sample |cdocsamp.tex| with include files
|cdocsch1.tex|, |cdocsch2.tex|, |cdocspt3.tex|, |cdocspt4.tex|,
|cdocsdrf.tex|, |cdocsfn1.tex|, |cdocsfn2.tex|.
Then copy the file |childdoc.def| to an appropriate directory of your \LaTeX{}
distribution, e.g.\ \textit{texmf-root}|/tex/latex/childdoc|.
\end{itemize}

%%%%%%%%%%%%%%%%%%%%%%%%%%%%%%%%%%%%%%%%%%%%%%%%%%%%%%%%%%%%%%%%%%%%%%%%%%%%%%%%
\subsection{Related CTAN Packages}

There are several other packages which offer a similar functionality:
%
\begin{itemize}
\item
The packages
\href{http://ctan.org/pkg/docmute}{\textsf{docmute}},
\href{http://ctan.org/pkg/includex}{\textsf{includex}} and
\href{http://ctan.org/pkg/standalone}{\textsf{standalone}}
provide commands to include only the document body of
a child file thus allowing both files to be compiled individually.
\item
The packages \href{http://ctan.org/pkg/subdocs}{\textsf{subdocs}}
and \href{http://ctan.org/pkg/subfiles}{\textsf{subfiles}}
provide structures in which the main and child documents can be
encapsulated and allowing them to be compiled individually.
The inclusion mechanism is different from the conventional |\include|.
\item
The package \href{http://ctan.org/pkg/combine}{\textsf{combine}}
is an elaborate solution to combine several documents into one.
\end{itemize}
%
See also the CTAN topic \href{http://ctan.org/topic/subdocs}{\textsf{subdocs}}
for further related packages.
The present package differs from the above solutions in that
a document structure constructed with the conventional |\include| mechanism
just needs two extra commands at the top of every file
such that all constituent files can be compiled individually.

%%%%%%%%%%%%%%%%%%%%%%%%%%%%%%%%%%%%%%%%%%%%%%%%%%%%%%%%%%%%%%%%%%%%%%%%%%%%%%%%
%\subsection{Feature Suggestions}
%
%The following is a list of features which may be useful for future
%versions of this package:
%%
%\begin{itemize}
%\item
%\ldots
%\end{itemize}

%%%%%%%%%%%%%%%%%%%%%%%%%%%%%%%%%%%%%%%%%%%%%%%%%%%%%%%%%%%%%%%%%%%%%%%%%%%%%%%%
\subsection{Revision History}

%%%%%%%%%%%%%%%%%%%%%%%%%%%%%%%%%%%%%%%%
\paragraph{v2.0:} 2018/12/30

\begin{itemize}
\item
immediate forward processing
\item
added |\childdocby| mechanism
\item
manual restructured
\end{itemize}

%%%%%%%%%%%%%%%%%%%%%%%%%%%%%%%%%%%%%%%%
\paragraph{v1.6:} 2018/01/17

\begin{itemize}
\item
application for development of include files
\item
corrections to manual
\end{itemize}

%%%%%%%%%%%%%%%%%%%%%%%%%%%%%%%%%%%%%%%%
\paragraph{v1.5:} 2017/05/21

\begin{itemize}
\item
more complete structuring introduced
\item
|\childdocof| introduced
\item
|\childdoc| renamed to |\childdocmain|
\item
|\childredirect| renamed to |\childdocforward| and |\childdocforwardprefix|
and functionality expanded
\end{itemize}

%%%%%%%%%%%%%%%%%%%%%%%%%%%%%%%%%%%%%%%%
\paragraph{v1.0:} 2017/04/27

\begin{itemize}
\item
manual and install package
\item
first version published on CTAN
\end{itemize}

%%%%%%%%%%%%%%%%%%%%%%%%%%%%%%%%%%%%%%%%
\paragraph{v0.6:} 2017/04/26

\begin{itemize}
\item
redirection mechanism added
\end{itemize}

%%%%%%%%%%%%%%%%%%%%%%%%%%%%%%%%%%%%%%%%
\paragraph{v0.5:} 2017/04/26

\begin{itemize}
\item
functionality in definition file
\end{itemize}


%%%%%%%%%%%%%%%%%%%%%%%%%%%%%%%%%%%%%%%%%%%%%%%%%%%%%%%%%%%%%%%%%%%%%%%%%%%%%%%%
%%%%%%%%%%%%%%%%%%%%%%%%%%%%%%%%%%%%%%%%%%%%%%%%%%%%%%%%%%%%%%%%%%%%%%%%%%%%%%%%
%%%%%%%%%%%%%%%%%%%%%%%%%%%%%%%%%%%%%%%%%%%%%%%%%%%%%%%%%%%%%%%%%%%%%%%%%%%%%%%%
\appendix

\settowidth\MacroIndent{\rmfamily\scriptsize 000\ }

 \DocInput{childdoc.dtx}

\end{document}
%</driver>
% \fi
%
% %%%%%%%%%%%%%%%%%%%%%%%%%%%%%%%%%%%%%%%%%%%%%%%%%%%%%%%%%%%%%%%%%%%%%%%%%%%%%%
% %%%%%%%%%%%%%%%%%%%%%%%%%%%%%%%%%%%%%%%%%%%%%%%%%%%%%%%%%%%%%%%%%%%%%%%%%%%%%%
% \section{Sample}
%\iffalse
%<*samplemain>
%\fi
%
% The following presents a sample document
% with two chapters, two parts, a title page,
% a compile flag as well as three forwarding files to set the flag.
% It consists of eight |.tex| files:
% \begin{center}
% \begin{tabular}{ll}
% |cdocsamp.tex|&main file\\
% |cdocsch1.tex|&include file for chapter 1\\
% |cdocsch2.tex|&include file for chapter 2\\
% |cdocspt3.tex|&include file for part 3\\
% |cdocspt4.tex|&include file for part 4\\
% |cdocsdrf.tex|&forwarding file for main file in draft mode\\
% |cdocsfi1.tex|&forwarding file for final version of chapter 1\\
% |cdocsfi2.tex|&forwarding file for final version of chapter 2\\
% \end{tabular}
% \end{center}
% Each of the eight files can be compiled directly by the \LaTeX{} compiler.
%
% %%%%%%%%%%%%%%%%%%%%%%%%%%%%%%%%%%%%%%
% \paragraph{Main File.}
%
% The main file is called |cdocsamp.tex|.
%
% Load the \textsf{childdoc} definitions and
% declare the filename for the main document:
%    \begin{macrocode}
\input{childdoc.def}
\childdocmain{}
%    \end{macrocode}

% Optional override for |\version| flag:
%    \begin{macrocode}
%%\ifchilddoc\else\providecommand{\version}{draft}\fi
%    \end{macrocode}

% Define the default values for the |\version| flag
% (|final| for the main file and |draft| for childs):
%    \begin{macrocode}
\ifchilddoc
\providecommand{\version}{draft}
\else
\providecommand{\version}{final}
\fi
%    \end{macrocode}

% Load the standard document class:
%    \begin{macrocode}
\documentclass[12pt]{article}
%    \end{macrocode}

% Start the document body:
%    \begin{macrocode}
\begin{document}
%    \end{macrocode}

% Declare a title page.
% Print title, part of document being processed and version flag:
%    \begin{macrocode}
\addtocounter{page}{-1}
\begin{center}
{\LARGE\bfseries{}childdoc example\par}
\vspace{1cm}
\ifchilddoc
\ifchilddocmanual part\else chapter\fi:
`\childdocname' of `\childdocjob'\par
\else
main document: `\childdocjob'\par
\fi
version: \version\par
\end{center}
\newpage
%    \end{macrocode}

% Manually include selected file,
% otherwise process as usual:
%    \begin{macrocode}
\ifchilddocmanual
\section*{part `\childdocname'}
\input{\childdocname}
\else
%    \end{macrocode}

% Include the two chapters:
%    \begin{macrocode}
\include{cdocsch1}
\include{cdocsch2}
%    \end{macrocode}

% Include the two parts unless only chapters should be displayed:
%    \begin{macrocode}
\ifchilddoc\else
\section{part three}
\input{cdocspt3}
\section{part four}
\input{cdocspt4}
\fi
%    \end{macrocode}

% Process as usual until here:
%    \begin{macrocode}
\fi
%    \end{macrocode}

% End of document body:
%    \begin{macrocode}
\end{document}
%    \end{macrocode}
%\iffalse
%</samplemain>
%\fi
%
% %%%%%%%%%%%%%%%%%%%%%%%%%%%%%%%%%%%%%%
% \paragraph{Chapter Include Files.}
%
% The include files are called |cdocsch1.tex| and |cdocsch2.tex|.
%
%\iffalse
%<*samplechap1|samplechap2>
%\fi

% Optional override for |\version| flag:
%    \begin{macrocode}
%%\providecommand{\version}{final}
%    \end{macrocode}

% Include the main document:
%    \begin{macrocode}
\input{childdoc.def}
\childdocof{cdocsamp}
%    \end{macrocode}

%\iffalse
%</samplechap1|samplechap2>
%\fi
%
%\iffalse
%<*samplechap1>
%\fi
% Some text for chapter 1:
%    \begin{macrocode}
\section{one}
some text in chapter one
%    \end{macrocode}

%\iffalse
%</samplechap1>
%\fi
% Some text for chapter 2:
%\iffalse
%<*samplechap2>
%\fi
%    \begin{macrocode}
\section{two}
more text in chapter two
%    \end{macrocode}

%\iffalse
%</samplechap2>
%\fi
%
% %%%%%%%%%%%%%%%%%%%%%%%%%%%%%%%%%%%%%%
% \paragraph{Part Include Files.}
%
% The include files are called |cdocspt3.tex| and |cdocspt4.tex|.
%
%\iffalse
%<*samplepart3|samplepart4>
%\fi

% Optional override for |\version| flag:
%    \begin{macrocode}
%%\providecommand{\version}{final}
%    \end{macrocode}

% Include the main document:
%    \begin{macrocode}
\input{childdoc.def}
\childdocby{cdocsamp}
%    \end{macrocode}

%\iffalse
%</samplepart3|samplepart4>
%\fi
%
%\iffalse
%<*samplepart3>
%\fi
% Some text for part 3:
%    \begin{macrocode}
some text in part three
%    \end{macrocode}

%\iffalse
%</samplepart3>
%\fi
% Some text for part 4:
%\iffalse
%<*samplepart4>
%\fi
%    \begin{macrocode}
more text in part four
%    \end{macrocode}

%\iffalse
%</samplepart4>
%\fi
%
% %%%%%%%%%%%%%%%%%%%%%%%%%%%%%%%%%%%%%%
% \paragraph{Forwarding for a Complete Draft.}
%
% The following forwarding file |cdocsdrf.tex|
% compiles the main document in draft mode:
%\iffalse
%<*sampledraft>
%\fi
%    \begin{macrocode}
\def\version{draft}
\input{childdoc.def}
\childdocforward{cdocsamp}
%    \end{macrocode}

%\iffalse
%</sampledraft>
%\fi
%
% %%%%%%%%%%%%%%%%%%%%%%%%%%%%%%%%%%%%%%
% \paragraph{Forwarding for Final Version of the Chapters.}
%
% The following forwarding files |cdocsfn1.tex| and |cdocsfn2.tex|
% (with identical content)
% compile the final versions of the child documents
% |cdocsch1.tex| and |cdocsch2.tex|, respectively:
%\iffalse
%<*samplefinal>
%\fi
%    \begin{macrocode}
\def\version{final}
\input{childdoc.def}
\childdocforwardprefix[cdocsamp]{cdocsfn}{cdocsch}
%    \end{macrocode}

%\iffalse
%</samplefinal>
%\fi
%
% %%%%%%%%%%%%%%%%%%%%%%%%%%%%%%%%%%%%%%
% \paragraph{Command Line Processing.}
%
% The following three command lines generate the output files
% |cdocscld|, |cdocscl1| and |cdocscl2|
% which should be identical to
% |cdocsdrf|, |cdocsch1| and |cdocsfn2|, respectively:
% \begin{center}
% \begin{tabular}{l}
% |latex -jobname cdocscld \|\\
% |  "\def\version{draft}\input{childdoc.def}\childdocforward{cdocsamp}"|\\
% |latex -jobname cdocscl1 \|\\
% |  "\input{childdoc.def}\childdocforward[cdocsamp]{cdocsch1}"|\\
% |latex -jobname cdocscl2 \|\\
% |  "\def\version{final}\input{childdoc.def}\childdocforward{cdocsch2}"|
% \end{tabular}
% \end{center}
% Note that the trailing backslash on each first line
% merely continues the input to the second line
% (for convenient cut ant paste).
% Furthermore, the command |latex| can be replaced by any
% of its alternative versions such as |pdflatex|.
%
% %%%%%%%%%%%%%%%%%%%%%%%%%%%%%%%%%%%%%%%%%%%%%%%%%%%%%%%%%%%%%%%%%%%%%%%%%%%%%%
% %%%%%%%%%%%%%%%%%%%%%%%%%%%%%%%%%%%%%%%%%%%%%%%%%%%%%%%%%%%%%%%%%%%%%%%%%%%%%%
% \section{Implementation}
%\iffalse
%<*package>
%\fi
%
% This section describes the definitions file |childdoc.def|.

% The definitions cannot be loaded using |\usepackage| or |\RequirePackage|
% which has a mechanism to prevent loading a style file more than once.
% When loading the definitions by means of |\input|
% multiple instances have to be prevented manually:
%\iffalse
%This code needs to be before the `\ProvidesFile' directive
%which is defined at the beginning of this file.
%Therefore it is also placed there and commented out here.
%</package>
%<*discard>
%\fi
%    \begin{macrocode}
\ifdefined\childdocmain\endinput\fi
%    \end{macrocode}
%\iffalse
%</discard>
%<*package>
%\fi
%
% \macro{\ifchilddoc}
% \macro{\ifchilddocmanual}
% The conditional |\ifchilddoc| tells whether a
% child (true) or main (false) document is being compiled.
% The conditional |\ifchilddocmanual| tells whether
% the |\includeonly| mechanism is used (false) or
% the selection of child files must be performed manually (true).
% The definitions initialise to false:
%    \begin{macrocode}
\newif\ifchilddoc
\newif\ifchilddocmanual
%    \end{macrocode}

% \macro{\childdocname}
% \macro{\childdocjob}
% The macro |\childdocname| stores the name of the main document
% to be compiled. The macro |\childdocjob| stores the name of
% the document on which the \LaTeX{} compiler was originally invoked.
% The content of |\jobname| cannot be compared
% to filenames specified in the source due to different catcodes.
% The following code rescans |\jobname|, stores the result
% in |\childdocname| and saves a copy in |\childdocjob|:
%    \begin{macrocode}
\edef\childdocname{\scantokens\expandafter{\jobname\noexpand}}
\let\childdocjob\childdocname
%    \end{macrocode}

% \macro{\childdocdisable}
% The macro |\childdocdisable| prevents the main file
% from being processed more than once.
% At this stage, the main document command |\childdocmain|
% is assumed to be called once again where it should do nothing.
% Any subsequent call to it should prevent
% a secondary processing of the main document
% It overwrites the forwarding commands
% |\childdocof| and |\childdocforward|
% with empty macros to prevent further inclusions of the main document:
%    \begin{macrocode}
\newcommand{\childdocdisable}
{
  \renewcommand{\childdocmain}[1]{\renewcommand{\childdocmain}[1]{\endinput}}
  \renewcommand{\childdocof}[1]{}
  \renewcommand{\childdocby}[2][]{}
  \renewcommand{\childdocforward}[2][]{}
  \renewcommand{\childdocdisable}{}
}
%    \end{macrocode}

% \macro{\childdocmain}
% The macro |\childdocmain| is to be called at the top of the main file
% with nothing or the main filename (without extension) as argument.
% First, it breaks loops.
% If the argument is not empty and does not match |\childdocname|
% (which is set by the first inclusion of |childdoc.def|),
% |\ifchilddoc| is set to true, |\includeonly| is applied to the child file
% and |\jobname| is set to the main file
% (for proper handling of |.aux| files):
%    \begin{macrocode}
\newcommand{\childdocmain}[1]
{
  \childdocdisable\childdocmain{}
  \if?#1?\else
    \begingroup
      \def\childdoctmp{#1}
      \ifx\childdoctmp\childdocname
        \def\childdoctmp{}
      \else
        \def\childdoctmp
        {
          \childdoctrue
          \includeonly{\childdocname}
          \def\childdocjob{#1}
          \def\jobname{#1}
        }
      \fi
      \expandafter
    \endgroup
    \childdoctmp
  \fi
}
%    \end{macrocode}

% \macro{\childdocof}
% The command |\childdocof| redirects
% compilation to the main file |#1|.
%    \begin{macrocode}
\newcommand{\childdocof}[1]
{
  \childdocdisable
  \childdoctrue
  \includeonly{\childdocname}
  \def\jobname{#1}
  \def\childdocjob{#1}
  \input{#1}
}
%    \end{macrocode}

% \macro{\childdocby}
% The command |\childdocby| ....
%    \begin{macrocode}
\newcommand{\childdocby}[2][]
{
  \childdocdisable
  \childdoctrue
  \childdocmanualtrue
  \if?#1?\else
    \def\jobname{#2}
  \fi
  \def\childdocjob{#2}
  \input{#2}
  \endinput
}
%    \end{macrocode}

% \macro{\childdocforward}
% The command |\childdocforward| redirects
% compilation to the main file or
% (if the optional argument is given) a child file.
% Parameters are set as if the main file
% or a child file starting with |\childdocof| was compiled.
% Then compilation is handed over to the main file:
%    \begin{macrocode}
\newcommand{\childdocforward}[2][]
{
  \begingroup
    \if?#1?
      \def\childdoctmp
      {
        \def\childdocname{#2}
        \def\childdocjob{#2}
        \def\jobname{#2}
        \input{#2}
        \endinput
      }
    \else
      \def\childdoctmp
      {
        \childdocdisable
        \def\childdocname{#2}
        \childdoctrue
        \includeonly{#2}
        \def\childdocjob{#1}
        \def\jobname{#1}
        \input{#1}
        \endinput
      }
    \fi
    \expandafter
  \endgroup
  \childdoctmp
}
%    \end{macrocode}

% \macro{\childdocforwardprefix}
% The command |\childdocforwardprefix| redirects
% compilation to the main or a child file by means of a pattern.
% The prefix |#1| in the current filename is replaced by |#2|
% and the suffix of the current filename is kept
% (it is assumed that the filename does not contain the substring `|~~~|'
% which is used as a delimiter).
% Compilation is handed over to the new file by |\childdocforward|:
%    \begin{macrocode}
\newcommand{\childdocforwardprefix}[3][]
{
  \begingroup
    \def\childdocextract #2##1~~~{\def\childdoctmp{\childdocforward[#1]{#3##1}}}
    \expandafter\childdocextract\childdocname~~~
    \expandafter
  \endgroup
  \childdoctmp
}
%    \end{macrocode}

% \macro{\childdoc}
% The deprecated macro |\childdoc| is a legacy version of |\childdocmain|:
%    \begin{macrocode}
\newcommand{\childdoc}{\childdocmain}
%    \end{macrocode}

% \macro{\childdocredirect}
% The deprecated macro |\childdocredirect| is a legacy version
% of |\childdocforward| and |\childdocforwardprefix|:
%    \begin{macrocode}
\newcommand{\childdocredirect}[2][]
{
  \begingroup
    \if?#1?
      \def\childdoctmp{\childdocforward{#2}}
    \else
      \def\childdoctmp{\childdocforwardprefix{#1}{#2}}
    \fi
    \expandafter
  \endgroup
  \childdoctmp
}
%    \end{macrocode}

%\iffalse
%</package>
%\fi
%
\endinput

\childdocmain{}
%    \end{macrocode}

% Optional override for |\version| flag:
%    \begin{macrocode}
%%\ifchilddoc\else\providecommand{\version}{draft}\fi
%    \end{macrocode}

% Define the default values for the |\version| flag
% (|final| for the main file and |draft| for childs):
%    \begin{macrocode}
\ifchilddoc
\providecommand{\version}{draft}
\else
\providecommand{\version}{final}
\fi
%    \end{macrocode}

% Load the standard document class:
%    \begin{macrocode}
\documentclass[12pt]{article}
%    \end{macrocode}

% Start the document body:
%    \begin{macrocode}
\begin{document}
%    \end{macrocode}

% Declare a title page.
% Print title, part of document being processed and version flag:
%    \begin{macrocode}
\addtocounter{page}{-1}
\begin{center}
{\LARGE\bfseries{}childdoc example\par}
\vspace{1cm}
\ifchilddoc
\ifchilddocmanual part\else chapter\fi:
`\childdocname' of `\childdocjob'\par
\else
main document: `\childdocjob'\par
\fi
version: \version\par
\end{center}
\newpage
%    \end{macrocode}

% Manually include selected file,
% otherwise process as usual:
%    \begin{macrocode}
\ifchilddocmanual
\section*{part `\childdocname'}
\input{\childdocname}
\else
%    \end{macrocode}

% Include the two chapters:
%    \begin{macrocode}
\include{cdocsch1}
\include{cdocsch2}
%    \end{macrocode}

% Include the two parts unless only chapters should be displayed:
%    \begin{macrocode}
\ifchilddoc\else
\section{part three}
\input{cdocspt3}
\section{part four}
\input{cdocspt4}
\fi
%    \end{macrocode}

% Process as usual until here:
%    \begin{macrocode}
\fi
%    \end{macrocode}

% End of document body:
%    \begin{macrocode}
\end{document}
%    \end{macrocode}
%\iffalse
%</samplemain>
%\fi
%
% %%%%%%%%%%%%%%%%%%%%%%%%%%%%%%%%%%%%%%
% \paragraph{Chapter Include Files.}
%
% The include files are called |cdocsch1.tex| and |cdocsch2.tex|.
%
%\iffalse
%<*samplechap1|samplechap2>
%\fi

% Optional override for |\version| flag:
%    \begin{macrocode}
%%\providecommand{\version}{final}
%    \end{macrocode}

% Include the main document:
%    \begin{macrocode}
% \iffalse
%
% childdoc.dtx Copyright (C) 2017-2018 Niklas Beisert
%
% This work may be distributed and/or modified under the
% conditions of the LaTeX Project Public License, either version 1.3
% of this license or (at your option) any later version.
% The latest version of this license is in
%   http://www.latex-project.org/lppl.txt
% and version 1.3 or later is part of all distributions of LaTeX
% version 2005/12/01 or later.
%
% This work has the LPPL maintenance status `maintained'.
%
% The Current Maintainer of this work is Niklas Beisert.
%
% This work consists of the files childdoc.dtx and childdoc.ins
% and the derived files childdoc.def and cdocsamp.tex with
% cdocsch1.tex, cdocsch2.tex, cdocsdrf.tex, cdocsfn1.tex, cdocsfn2.tex.
%
%<package>\ifdefined\childdocmain\endinput\fi
%<package>\ProvidesFile{childdoc.def}[2018/12/30 v2.0 child document driver]
%<samplemain>\ProvidesFile{cdocsamp.tex}[2018/12/30 v2.0 sample for childdoc]
%<*driver>
%\ProvidesFile{childdoc.drv}[2018/12/30 v2.0 childdoc reference manual file]
\PassOptionsToClass{10pt,a4paper}{article}
\documentclass{ltxdoc}

\usepackage[margin=35mm]{geometry}
\usepackage{hyperref}
\usepackage{hyperxmp}
\usepackage[usenames]{color}

\hypersetup{colorlinks=true}
\hypersetup{pdfstartview=FitH}
\hypersetup{pdfpagemode=UseNone}
\hypersetup{pdfsource={}}
\hypersetup{pdflang={en-UK}}
\hypersetup{pdfcopyright={Copyright 2017-2018 Niklas Beisert.
  This work may be distributed and/or modified under the
  conditions of the LaTeX Project Public License, either version 1.3
  of this license or (at your option) any later version.}}
\hypersetup{pdflicenseurl={http://www.latex-project.org/lppl.txt}}
\hypersetup{pdfcontactaddress={ETH Zurich, ITP, HIT K,
  Wolfgang-Pauli-Strasse 27}}
\hypersetup{pdfcontactpostcode={8093}}
\hypersetup{pdfcontactcity={Zurich}}
\hypersetup{pdfcontactcountry={Switzerland}}
\hypersetup{pdfcontactemail={nbeisert@itp.phys.ethz.ch}}
\hypersetup{pdfcontacturl={http://people.phys.ethz.ch/\xmptilde nbeisert/}}

\newcommand{\secref}[1]{\hyperref[#1]{section \ref*{#1}}}

\parskip1ex
\parindent0pt
\let\olditemize\itemize
\def\itemize{\olditemize\parskip0pt}

\begin{document}

\title{The \textsf{childdoc} Package}
\hypersetup{pdftitle={The childdoc Package}}
\author{Niklas Beisert\\[2ex]
  Institut f\"ur Theoretische Physik\\
  Eidgen\"ossische Technische Hochschule Z\"urich\\
  Wolfgang-Pauli-Strasse 27, 8093 Z\"urich, Switzerland\\[1ex]
  \href{mailto:nbeisert@itp.phys.ethz.ch}
  {\texttt{nbeisert@itp.phys.ethz.ch}}}
\hypersetup{pdfauthor={Niklas Beisert}}
\hypersetup{pdfsubject={Manual for the LaTeX2e Package childdoc}}
\date{30 December 2018, \textsf{v2.0}}
\maketitle

\begin{abstract}\noindent
\textsf{childdoc} is a \LaTeXe{} package
that enables the direct compilation
of document sections included by |\include|
to individual files.
\end{abstract}

\begingroup
\parskip0ex
\tableofcontents
\endgroup

%%%%%%%%%%%%%%%%%%%%%%%%%%%%%%%%%%%%%%%%%%%%%%%%%%%%%%%%%%%%%%%%%%%%%%%%%%%%%%%%
%%%%%%%%%%%%%%%%%%%%%%%%%%%%%%%%%%%%%%%%%%%%%%%%%%%%%%%%%%%%%%%%%%%%%%%%%%%%%%%%
\section{Introduction}

\LaTeX{} provides a mechanism to structure a large document (such as a book)
into a main file and several child files (containing the chapters)
using the |\include| command.
This mechanism is beneficial for documents
which span hundreds of pages in order to
make the source file(s) more manageable.
Moreover, compilation can be restricted to
selected child files by means of the |\includeonly| command.
The latter feature can be used to reduce the compilation time while editing
(this was significantly more useful in the earlier days of \LaTeX{})
or to generate a smaller document which is easier to navigate.
Another application of |\includeonly| is to generate
documents consisting of selected parts of the complete document.

However, there are a few drawbacks of the plain |\include| mechanism:
\begin{itemize}
\item
The child files cannot be compiled on their own,
they can only be compiled via the main file.
A naive editing environment
(such as a text editor with an option
to have the current file processed by \LaTeX)
may require one to switch to the main file before compiling;
attempting to compile the child file produces errors.
\item
The main file must be modified (each time)
to adjust the |\includeonly| command
to the present needs. This easily leaves the main file in a messy state.
\item
The generated document will always carry the filename
of the main document. This is inconvenient if
several child files are to be compiled and
to be kept for distribution.
\end{itemize}

The present package provides a simple interface
to make child files individually compilable by \LaTeX{}.
Compiling a child file then has the same effect as compiling
the main file with an |\includeonly| command
to select the appropriate child.
Moreover the generated document will carry the name of the child
rather than the main file.
This resolves all three above issues.

This feature is meant to make the editing of books,
thesis documents and lecture notes somewhat more convenient.
However, the package can also be used efficiently for
composing a series of documents (such as exercise sheets)
which are typically distributed individually.
It then assists the author in generating the individual documents
(potentially in different versions)
as well as a document containing the collected series.
Another application is in developing style files
or other kinds of included material
where compilation of the style file could redirect
to a sample or test file.

%%%%%%%%%%%%%%%%%%%%%%%%%%%%%%%%%%%%%%%%%%%%%%%%%%%%%%%%%%%%%%%%%%%%%%%%%%%%%%%%
%%%%%%%%%%%%%%%%%%%%%%%%%%%%%%%%%%%%%%%%%%%%%%%%%%%%%%%%%%%%%%%%%%%%%%%%%%%%%%%%
\section{Usage}

First of all, the package \textsf{childdoc} is \emph{not} a standard
\LaTeXe{} |.sty| style file! Therefore it needs to be invoked in
a non-standard way.

%%%%%%%%%%%%%%%%%%%%%%%%%%%%%%%%%%%%%%%%%%%%%%%%%%%%%%%%%%%%%%%%%%%%%%%%%%%%%%%%
\subsection{Included Files}
\label{sec:include}

%%%%%%%%%%%%%%%%%%%%%%%%%%%%%%%%%%%%%%%%
\DescribeMacro{\childdocmain}
To use the package, add the commands
\begin{center}
\begin{tabular}{l}
|\input{childdoc.def}|\\
|\childdocmain{}|\\
\end{tabular}
\end{center}
at the very top of the main \LaTeX{} file,
in particular \emph{before} the |\documentclass| statement!
The argument of |\childdocmain| should be left empty
(but it must be present).

%%%%%%%%%%%%%%%%%%%%%%%%%%%%%%%%%%%%%%%%
\DescribeMacro{\childdocof}
Furthermore, add the commands
\begin{center}
\begin{tabular}{l}
|\input{childdoc.def}|\\
|\childdocof{|\textit{main}|}|\\
\end{tabular}
\end{center}
at the top of every child file \textit{child}
which is included by |\include{|\textit{child}|}|
from within the main file
(or at least for those files to be compiled individually).
The argument \textit{main} must be the filename of the main file.

There are a couple of
considerations in setting up the main and child documents:

%%%%%%%%%%%%%%%%%%%%%%%%%%%%%%%%%%%%%%%%
\paragraph{Restrictions.}

Please note the following restrictions:
\begin{itemize}
\item
|\childdocmain| must be called with one argument \textit{main}
to ensure compatibility with earlier version of the package.
It must either be empty (|\childdocmain{}|)
or precisely match the filename of the main file in which it is specified.
See \secref{sec:detection} for further information.
\item
The filename \textit{main} must be specified without the |.tex| extension.
\item
The filename \textit{main} is case sensitive
(even in case-insensitive file systems)
due to internal string comparison.
\item
The argument \textit{main} should be fully expanded, it cannot be a macro.
\item
Subdirectories and special characters should be avoided in filenames.
\item
The command |\childdocmain{|\textit{main}|}| must be followed by a whitespace.
It should not be followed immediately by another command
or by a comment mark `|%|'.
This is because the \TeX{} parser reads the token immediately following
the argument of |\childdocmain| and puts it
at the beginning of every child section;
however, a white\-space is ignored.
\end{itemize}

%%%%%%%%%%%%%%%%%%%%%%%%%%%%%%%%%%%%%%%%
\paragraph{Content of Main File.}

It is advisable to place all content in the child files included by |\include|.
Any output contained in the main file will appear in all child documents
unless suppressed manually;
it cannot be suppressed automatically by the |\includeonly| directive
and thus should normally be avoided.
A method to include some content in the main file
by means of conditional processing is described in \secref{sec:conditional}.

%%%%%%%%%%%%%%%%%%%%%%%%%%%%%%%%%%%%%%%%
\paragraph{Page Numbering.}

When only a part of the document is compiled,
the appropriate numbering of pages
(as well as other status parameters)
is determined from the |.aux| files.
The latter contain information from previous passes.
However this information needs to propagate through
all intermediate child documents.
Therefore the page numbering in child documents may well
be inconsistent until the complete document is compiled at least once.

A useful (if unconventional) way to always ensure a consistent
page numbering is to restart the numbering in each child document
and denote the pages by `\textit{child}|.|\textit{page}'
where \textit{child} represents the chapter/section number of the child file.
This can be achieved by the command
|\numberwithin{page}{|\textit{child}|}|
of the \textsf{amsmath} package
where \textit{child} can be |chapter| or |section|
depending on the chosen structuring.
Alternatively, one can modify the macro |\thepage| appropriately
and reset the counter |page| at the start of each child file.

%%%%%%%%%%%%%%%%%%%%%%%%%%%%%%%%%%%%%%%%%%%%%%%%%%%%%%%%%%%%%%%%%%%%%%%%%%%%%%%%
\subsection{Conditional Processing}
\label{sec:conditional}

The package provides a mechanism to compile different versions
of a document. To customise the versions further some conditional processing
can come in handy to distinguish which version is being compiled.
The package provides two macros to describe the compilation context:

%%%%%%%%%%%%%%%%%%%%%%%%%%%%%%%%%%%%%%%%
\DescribeMacro{\ifchilddoc}
The conditional |\ifchilddoc| distinguishes between the compilation of
child documents and the main document:
%
\begin{center}
|\ifchilddoc |\textit{child-code}| |[|\||else |\textit{main-code}]| \||fi|
\end{center}

%%%%%%%%%%%%%%%%%%%%%%%%%%%%%%%%%%%%%%%%
\DescribeMacro{\childdocname}
\DescribeMacro{\childdocjob}
The macro |\childdocname| contains the filename (without extension)
of the main or child file being processed.
Note that |\childdocjob| will always contain the name of the main file.

%%%%%%%%%%%%%%%%%%%%%%%%%%%%%%%%%%%%%%%%
\paragraph{Title Page.}

Conditional processing can be used to include a title or banner page
in the main document when proper precautions are taken.
Importantly, the code in the main file should ensure that the page counter
(as well as other status parameters which are stored in the |.aux| files)
takes the same value after the conditional processing.
Otherwise the page numbers may take divergent values
depending on which part is compiled.

For example, a title page could be declared by:
%
\begin{center}
\begin{tabular}{l}
|\ifchilddoc\||else|\\
|\addtocounter{page}{-1}|\\
\textit{code for title page}\\
|\newpage|\\
|\||fi|
\end{tabular}
\end{center}
%
A banner page for the child documents can be generated by:
%
\begin{center}
\begin{tabular}{l}
|\ifchilddoc|\\
|\addtocounter{page}{-1}|\\
\textit{code for banner page}\\
|\newpage|\\
|\||fi|
\end{tabular}
\end{center}
%
Here one could write a message such as:
\begin{center}
|This is the part \childdocname{} of \childdocjob{}.|
\end{center}

%%%%%%%%%%%%%%%%%%%%%%%%%%%%%%%%%%%%%%%%%%%%%%%%%%%%%%%%%%%%%%%%%%%%%%%%%%%%%%%%
\subsection{Flags}
\label{sec:flags}

The package makes it easy to generate different versions
of the main or child documents.
To this end compilation flags can be defined
and assigned different default values.
They will be particularly useful in conjunction
with the forwarding mechanism described in \secref{sec:forward}.

For example, it may be useful to have a flag |\version|
which can be set to |draft| or |final|.
The document source will contain some conditional code
depending on the value of |\version|.
Suppose further, the flag should default to |final| for the main file
and to |draft| for child files
which is a natural assignment for editing the document.
This is achieved by placing the following code
in the preamble of the main document
(below the |\childdocmain| directive):
%
\begin{center}
\begin{tabular}{l}
|\ifchilddoc|\\
|\providecommand{\version}{draft}|\\
|\||else|\\
|\providecommand{\version}{final}|\\
|\||fi|
\end{tabular}
\end{center}
%
The definition by |\providecommand| makes sure
that previous definitions are not overwritten.
Further statements |\providecommand{\version}{...}|
can thus be added before the above code to override it.

For the main file, one might add a line
(between |\childdocmain| and the above block)
%
\begin{center}
|%\ifchilddoc\||else\providecommand{\version}{draft}\||fi|
\end{center}
%
which can be uncommented to produce a draft version.
Likewise one can add a line to the very top of a child file
(above the |\childdocof{|\textit{main}|}| directive)
%
\begin{center}
|%\providecommand{\version}{final}|
\end{center}
%
which can be uncommented to produce the final version of this child document.

%%%%%%%%%%%%%%%%%%%%%%%%%%%%%%%%%%%%%%%%%%%%%%%%%%%%%%%%%%%%%%%%%%%%%%%%%%%%%%%%
\subsection{Forwarding}
\label{sec:forward}

Different versions of the main or child documents
using compilation flags as described in \secref{sec:flags}
can be (permanently) stored in different files
for convenient compilation, viewing and distribution.
To this end, the package defines a command
to pass on compilation to a different file:

%%%%%%%%%%%%%%%%%%%%%%%%%%%%%%%%%%%%%%%%
\DescribeMacro{\childdocforward}
The command |\childdocforward| redirects processing to
another source file:
%
\begin{center}
\begin{tabular}{l}
|\input{childdoc.def}|\\
|\childdocforward[|\textit{main}|]{|\textit{dest}|}|\\
\end{tabular}
\end{center}
%
The argument \textit{dest} is the destination file
(without extension).
It should be the main file or one of the child files.
Note that further \textsf{childdoc} directives
such as |\childdocof| and |\childdocforward|
in the indicated file will be processed in this form.
The optional argument \textit{main}
passes on directly to the main file \textit{main}
while pretending to compile the child \textit{dest}.
This form behaves as if \textit{dest}
issues |\childdocof{|\textit{main}|}| right away,
and no further \textsf{childdoc} directives will be processed.

%%%%%%%%%%%%%%%%%%%%%%%%%%%%%%%%%%%%%%%%
\DescribeMacro{\...prefix}
In the alternative form |\childdocforwardprefix|,
%
\begin{center}
\begin{tabular}{l}
|\input{childdoc.def}|\\
|\childdocforwardprefix[|\textit{main}|]{|\textit{prefix}|}{|\textit{dest}|}|
\end{tabular}
\end{center}
%
the destination file is determined by a pattern
depending on the current file:
To make this work, the current file must be called
`{\textit{prefix}\hspace{0.2em}\textit{suffix}}'
with \textit{prefix} matching precisely the argument.
Processing is then passed on to the file
`{\textit{dest}\hspace{0.2em}\textit{suffix}}'.
Surely, the same effect is achieved by
directly specifying the
argument `{\textit{dest}\hspace{0.2em}\textit{suffix}}'
in the first form.
However, that requires to set up a different file
for each child. With the alternative form of the command
all these files can have exactly the same content
which simplifies setting them up and maintaining them.

For example, the following file |draft.tex|
with a compilation flag |\version| as described in \secref{sec:flags}
compiles the main document as a draft:
%
\begin{center}
\begin{tabular}{l}
|\def\version{draft}|\\
|\input{childdoc.def}|\\
|\childdocforward{|\textit{main}|}|
\end{tabular}
\end{center}
%
Likewise, the following files |final|\textit{nn}|.tex|
compile the final version of the child document
|child|\textit{nn}|.tex|:
%
\begin{center}
\begin{tabular}{l}
|\def\version{final}|\\
|\input{childdoc.def}|\\
|\childdocforwardprefix{final}{child}|
\end{tabular}
\end{center}
%

Note that when several versions of a main file and/or of each child file
are to be generated, it may be convenient to set up a |Makefile| or
shell script to automatise the process.

%%%%%%%%%%%%%%%%%%%%%%%%%%%%%%%%%%%%%%%%%%%%%%%%%%%%%%%%%%%%%%%%%%%%%%%%%%%%%%%%
\subsection{Command Line Processing}
\label{sec:commandline}

The effect of redirection files can also be achieved by invoking
the \LaTeX{} compiler with a more elaborate command line.
Most conveniently this should be done as part
of a shell script or a |Makefile|.

When using \textsf{childdoc} in the main file, the following
command lines effectively perform a redirection
(note that depending on the shell being used,
backslashes may have to be doubled: `|\|' $\to$ `|\\|'):
%
\begin{center}
|... -jobname "|\textit{target}|" |\\|"|[\textit{flags}]%
|\input{childdoc.def}\childdocforward[|\textit{main}|]{|\textit{dest}|}"|
\end{center}
%
Here \textit{target} is the name of the output file,
\textit{main} is the name of the main file
and \textit{dest} is the name of the main or child file to be processed
(all filenames without extensions).
The optional argument \textit{main} can be omitted
if \textit{main} matches \textit{dest}.
Optionally, compilation \textit{flags} can be defined via |\def| commands.
This command line makes the \TeX{} engine believe
it is compiling the file \textit{target}
whose content is specified as the latter parameter.
The provided code then forwards the processing to
\textit{main} or \textit{dest} as described in \secref{sec:forward}.

%%%%%%%%%%%%%%%%%%%%%%%%%%%%%%%%%%%%%%%%%%%%%%%%%%%%%%%%%%%%%%%%%%%%%%%%%%%%%%%%
\subsection{Include by Input}
\label{sec:input}

Including child documents by |\include| has some restrictions by design.
Most notably, the content of a child document always occupies
its own set of pages; pages cannot be shared between child documents.
Usually, this behaviour makes perfect sense
because each child document contain an essential part of the document.
However, in some situations it may be desirable to compose
a document from a collection of parts
without having mandatory page breaks between then.
For this case, the package
provides a mechanism to include parts
by |\input| which can also be processed individually.
However, by construction this mechanism
requires manual handling of the content to be output.

%%%%%%%%%%%%%%%%%%%%%%%%%%%%%%%%%%%%%%%%
\DescribeMacro{\ifchilddocmanual}
The main file should be prepared as usual, see \secref{sec:include}.
However, the document body must make a distinction
between processing of an individual part and of the main document, e.g.:
%
\begin{center}
\begin{tabular}{l}
|\ifchilddocmanual|\\
|\input{\childdocname}|\\
|\||else|\\
\textit{document body with }|\input{|\textit{part}|}|\\
|\||fi|
\end{tabular}
\end{center}
%
The conditional |\ifchilddocmanual| is true whenever
a part to be included by |\input| is being compiled,
and the name of the part is stored in |\childdocname|.

%%%%%%%%%%%%%%%%%%%%%%%%%%%%%%%%%%%%%%%%
\DescribeMacro{\childdocby}
Each part to be included by |\input| should start with:
%
\begin{center}
\begin{tabular}{l}
|\input{childdoc.def}|\\
|\childdocby{|\textit{main}|}|\\
\end{tabular}
\end{center}
%
The directive |\childdocby| is similar to |\childdocof|
described in \secref{sec:include},
but the subsequent selection of content must be done manually.
To that end, both |\ifchilddoc| and |\ifchilddocmanual|
will be true upon processing of a part,
and the name of the part is stored in |\childdocname|.
Note that |\jobname| will be set to the filename of the current part
so that each part receives an individual |.aux| file
that does not interfere with the |.aux| file(s) of the main document.
This behaviour can be altered by the alternative form
|\childdocby[*]{|\textit{main}|}| (with a non-empty optional argument)
which uses the |.aux| file of the main document
by setting |\jobname| to \textit{main}.

%%%%%%%%%%%%%%%%%%%%%%%%%%%%%%%%%%%%%%%%%%%%%%%%%%%%%%%%%%%%%%%%%%%%%%%%%%%%%%%%
\subsection{Driver Development}
\label{sec:driver}

The \textsf{childdoc} mechanism can also be use for the development
of definition files such as \LaTeX{} styles or classes.
This case differs from the above setup with multiple parts
included by |\include| in that no |\includeonly| should be invoked.
This can be achieved by starting the include file
(before |\ProvidesPackage|) with:
%
\begin{center}
\begin{tabular}{l}
|\input{childdoc.def}|\\
|\childdocforward{|\textit{main}|}|\\
\end{tabular}
\end{center}
%
or alternatively with:
%
\begin{center}
\begin{tabular}{l}
|\input{childdoc.def}|\\
|\childdocby{|\textit{main}|}|\\
\end{tabular}
\end{center}
%
Both forms have slightly different effects as described above.
The main file is prepared as usual, see \secref{sec:include}.

%%%%%%%%%%%%%%%%%%%%%%%%%%%%%%%%%%%%%%%%%%%%%%%%%%%%%%%%%%%%%%%%%%%%%%%%%%%%%%%%
\subsection{Legacy Detection}
\label{sec:detection}

The directive |\childdocmain| in the main file can detect
whether the complete document or merely a child is to be compiled
even without using the directive |\childdocof|.
This method is deprecated because it is less robust
and there is no compelling reason to use it;
it is merely provided for backward compatibility
and it may be removed in future versions.

If the detection mechanism is to be used,
it is mandatory to correctly specify
the filename of the main file as the argument of |\childdocmain|:
%
\begin{center}
\begin{tabular}{l}
|\input{childdoc.def}|\\
|\childdocmain{|\textit{main}|}|\\
\end{tabular}
\end{center}
%
If |\jobname| does not match the argument \textit{main} of |\childdocmain|,
it is assumed that |\jobname| points to the child file to be compiled.
When using |\childdocmain| with the main file specified as argument,
it suffices to start a child file
with just |\input{|\textit{main}|}|
without loading of the package and using |\childdocof|.
If instead all processing is done
with the appropriate \textsf{childdoc} directives,
the argument of \textit{main} of |\childdocmain| can be empty.

An alternative version of the command line processing described
in \secref{sec:commandline} using the detection mechanism reads:
%
\begin{center}
|... -jobname "|\textit{target}|" "|[\textit{flags}]%
[|\def\jobname{|\textit{dest}|}|]|\input{|\textit{main}|}"|
\end{center}

%%%%%%%%%%%%%%%%%%%%%%%%%%%%%%%%%%%%%%%%%%%%%%%%%%%%%%%%%%%%%%%%%%%%%%%%%%%%%%%%
\subsection{Manual Code}
\label{sec:manual}

In case one cannot be certain whether the definitions file |childdoc.def|
is installed on the target \TeX{} distribution
and one prefers not to ship it,
it is conceivable to paste a few relevant commands into the sources.

To that end, drop all statements |\input{childdoc.def}|
and perform the replacements as outlined below.
Instead of |\childdocmain{|\textit{main}|}| add the following code
to the top of the main file:
%
\begin{center}
\begin{tabular}{l}
|\||ifdefined\childdocname\endinput\||fi\newif\ifchilddoc|\\
|\edef\childdocname{\scantokens\expandafter{\jobname\noexpand}}|\\
|\def\childdocmain{|\textit{main}|}\||ifx\childdocmain\childdocname\||else|\\
|\childdoctrue\includeonly{\childdocname}\let\jobname\childdocmain\||fi|\\
\end{tabular}
\end{center}
%
Instead of |\childdocof{|\textit{main}|}| just include the main file
at the top of each child file:
%
\begin{center}
|\input{|\textit{main}|}|
\end{center}
%
A simple redirection |\childdocforward{|\textit{dest}|}| is achieved by:
%
\begin{center}
|\def\jobname{|\textit{dest}|}\input{\jobname}|
\end{center}
%
The redirection with prefix
|\childdocforwardprefix[|\textit{prefix}|]{|\textit{dest}|}|
is accomplished by:
%
\begin{center}
\begin{tabular}{l}
|{\edef\jobname{\scantokens\expandafter{\jobname\noexpand}}|\\
|\def\redirectjob |\textit{prefix}|#1~~~{\gdef\jobname{|\textit{dest}|#1}}|\\
|\expandafter\redirectjob\jobname~~~}\input{\jobname}|
\end{tabular}
\end{center}

In an alternative approach,
child documents can be compiled by a specific command line
without additional code or specific definitions:
%
\begin{center}
|... -jobname "|\textit{target}|" "|[\textit{flags}]%
|\includeonly{|\textit{dest}|}\input{|\textit{main}|}"|
\end{center}
%

%%%%%%%%%%%%%%%%%%%%%%%%%%%%%%%%%%%%%%%%%%%%%%%%%%%%%%%%%%%%%%%%%%%%%%%%%%%%%%%%
%%%%%%%%%%%%%%%%%%%%%%%%%%%%%%%%%%%%%%%%%%%%%%%%%%%%%%%%%%%%%%%%%%%%%%%%%%%%%%%%
\section{Information}

%%%%%%%%%%%%%%%%%%%%%%%%%%%%%%%%%%%%%%%%%%%%%%%%%%%%%%%%%%%%%%%%%%%%%%%%%%%%%%%%
\subsection{Copyright}

Copyright \copyright{} 2017--2018 Niklas Beisert

This work may be distributed and/or modified under the
conditions of the \LaTeX{} Project Public License, either version 1.3
of this license or (at your option) any later version.
The latest version of this license is in
  \url{http://www.latex-project.org/lppl.txt}
and version 1.3 or later is part of all distributions of \LaTeX{}
version 2005/12/01 or later.

This work has the LPPL maintenance status `maintained'.

The Current Maintainer of this work is Niklas Beisert.

This work consists of the files |README.txt|, |childdoc.ins| and |childdoc.dtx|
as well as the derived files |childdoc.def|, |cdocsamp.tex|
with |cdocsch1.tex|, |cdocsch2.tex|, |cdocspt3.tex|, |cdocspt4.tex|,
|cdocsdrf.tex|, |cdocsfn1.tex|, |cdocsfn2.tex|
as well as |childdoc.pdf|.

%%%%%%%%%%%%%%%%%%%%%%%%%%%%%%%%%%%%%%%%%%%%%%%%%%%%%%%%%%%%%%%%%%%%%%%%%%%%%%%%
\subsection{Files and Installation}

The package consists of the files:
%
\begin{center}
\begin{tabular}{ll}
    |README.txt|   & readme file \\
    |childdoc.ins| & installation file \\
    |childdoc.dtx| & source file \\
    |childdoc.def| & definition file \\
    |cdocsamp.tex| & sample main file \\
    |cdocsch1.tex| & sample include file \\
    |cdocsch2.tex| & sample include file \\
    |cdocspt3.tex| & sample part file \\
    |cdocspt4.tex| & sample part file \\
    |cdocsdrf.tex| & sample redirection file \\
    |cdocsfn1.tex| & sample redirection file \\
    |cdocsfn2.tex| & sample redirection file \\
    |childdoc.pdf| & manual
\end{tabular}
\end{center}
%
The distribution consists of the files
|README.txt|, |childdoc.ins| and |childdoc.dtx|.
%
\begin{itemize}
\item
Run (pdf)\LaTeX{} on |childdoc.dtx|
to compile the manual |childdoc.pdf| (this file).
\item
Run \LaTeX{} on |childdoc.ins| to create the definitions file |childdoc.def|
and the sample |cdocsamp.tex| with include files
|cdocsch1.tex|, |cdocsch2.tex|, |cdocspt3.tex|, |cdocspt4.tex|,
|cdocsdrf.tex|, |cdocsfn1.tex|, |cdocsfn2.tex|.
Then copy the file |childdoc.def| to an appropriate directory of your \LaTeX{}
distribution, e.g.\ \textit{texmf-root}|/tex/latex/childdoc|.
\end{itemize}

%%%%%%%%%%%%%%%%%%%%%%%%%%%%%%%%%%%%%%%%%%%%%%%%%%%%%%%%%%%%%%%%%%%%%%%%%%%%%%%%
\subsection{Related CTAN Packages}

There are several other packages which offer a similar functionality:
%
\begin{itemize}
\item
The packages
\href{http://ctan.org/pkg/docmute}{\textsf{docmute}},
\href{http://ctan.org/pkg/includex}{\textsf{includex}} and
\href{http://ctan.org/pkg/standalone}{\textsf{standalone}}
provide commands to include only the document body of
a child file thus allowing both files to be compiled individually.
\item
The packages \href{http://ctan.org/pkg/subdocs}{\textsf{subdocs}}
and \href{http://ctan.org/pkg/subfiles}{\textsf{subfiles}}
provide structures in which the main and child documents can be
encapsulated and allowing them to be compiled individually.
The inclusion mechanism is different from the conventional |\include|.
\item
The package \href{http://ctan.org/pkg/combine}{\textsf{combine}}
is an elaborate solution to combine several documents into one.
\end{itemize}
%
See also the CTAN topic \href{http://ctan.org/topic/subdocs}{\textsf{subdocs}}
for further related packages.
The present package differs from the above solutions in that
a document structure constructed with the conventional |\include| mechanism
just needs two extra commands at the top of every file
such that all constituent files can be compiled individually.

%%%%%%%%%%%%%%%%%%%%%%%%%%%%%%%%%%%%%%%%%%%%%%%%%%%%%%%%%%%%%%%%%%%%%%%%%%%%%%%%
%\subsection{Feature Suggestions}
%
%The following is a list of features which may be useful for future
%versions of this package:
%%
%\begin{itemize}
%\item
%\ldots
%\end{itemize}

%%%%%%%%%%%%%%%%%%%%%%%%%%%%%%%%%%%%%%%%%%%%%%%%%%%%%%%%%%%%%%%%%%%%%%%%%%%%%%%%
\subsection{Revision History}

%%%%%%%%%%%%%%%%%%%%%%%%%%%%%%%%%%%%%%%%
\paragraph{v2.0:} 2018/12/30

\begin{itemize}
\item
immediate forward processing
\item
added |\childdocby| mechanism
\item
manual restructured
\end{itemize}

%%%%%%%%%%%%%%%%%%%%%%%%%%%%%%%%%%%%%%%%
\paragraph{v1.6:} 2018/01/17

\begin{itemize}
\item
application for development of include files
\item
corrections to manual
\end{itemize}

%%%%%%%%%%%%%%%%%%%%%%%%%%%%%%%%%%%%%%%%
\paragraph{v1.5:} 2017/05/21

\begin{itemize}
\item
more complete structuring introduced
\item
|\childdocof| introduced
\item
|\childdoc| renamed to |\childdocmain|
\item
|\childredirect| renamed to |\childdocforward| and |\childdocforwardprefix|
and functionality expanded
\end{itemize}

%%%%%%%%%%%%%%%%%%%%%%%%%%%%%%%%%%%%%%%%
\paragraph{v1.0:} 2017/04/27

\begin{itemize}
\item
manual and install package
\item
first version published on CTAN
\end{itemize}

%%%%%%%%%%%%%%%%%%%%%%%%%%%%%%%%%%%%%%%%
\paragraph{v0.6:} 2017/04/26

\begin{itemize}
\item
redirection mechanism added
\end{itemize}

%%%%%%%%%%%%%%%%%%%%%%%%%%%%%%%%%%%%%%%%
\paragraph{v0.5:} 2017/04/26

\begin{itemize}
\item
functionality in definition file
\end{itemize}


%%%%%%%%%%%%%%%%%%%%%%%%%%%%%%%%%%%%%%%%%%%%%%%%%%%%%%%%%%%%%%%%%%%%%%%%%%%%%%%%
%%%%%%%%%%%%%%%%%%%%%%%%%%%%%%%%%%%%%%%%%%%%%%%%%%%%%%%%%%%%%%%%%%%%%%%%%%%%%%%%
%%%%%%%%%%%%%%%%%%%%%%%%%%%%%%%%%%%%%%%%%%%%%%%%%%%%%%%%%%%%%%%%%%%%%%%%%%%%%%%%
\appendix

\settowidth\MacroIndent{\rmfamily\scriptsize 000\ }

 \DocInput{childdoc.dtx}

\end{document}
%</driver>
% \fi
%
% %%%%%%%%%%%%%%%%%%%%%%%%%%%%%%%%%%%%%%%%%%%%%%%%%%%%%%%%%%%%%%%%%%%%%%%%%%%%%%
% %%%%%%%%%%%%%%%%%%%%%%%%%%%%%%%%%%%%%%%%%%%%%%%%%%%%%%%%%%%%%%%%%%%%%%%%%%%%%%
% \section{Sample}
%\iffalse
%<*samplemain>
%\fi
%
% The following presents a sample document
% with two chapters, two parts, a title page,
% a compile flag as well as three forwarding files to set the flag.
% It consists of eight |.tex| files:
% \begin{center}
% \begin{tabular}{ll}
% |cdocsamp.tex|&main file\\
% |cdocsch1.tex|&include file for chapter 1\\
% |cdocsch2.tex|&include file for chapter 2\\
% |cdocspt3.tex|&include file for part 3\\
% |cdocspt4.tex|&include file for part 4\\
% |cdocsdrf.tex|&forwarding file for main file in draft mode\\
% |cdocsfi1.tex|&forwarding file for final version of chapter 1\\
% |cdocsfi2.tex|&forwarding file for final version of chapter 2\\
% \end{tabular}
% \end{center}
% Each of the eight files can be compiled directly by the \LaTeX{} compiler.
%
% %%%%%%%%%%%%%%%%%%%%%%%%%%%%%%%%%%%%%%
% \paragraph{Main File.}
%
% The main file is called |cdocsamp.tex|.
%
% Load the \textsf{childdoc} definitions and
% declare the filename for the main document:
%    \begin{macrocode}
\input{childdoc.def}
\childdocmain{}
%    \end{macrocode}

% Optional override for |\version| flag:
%    \begin{macrocode}
%%\ifchilddoc\else\providecommand{\version}{draft}\fi
%    \end{macrocode}

% Define the default values for the |\version| flag
% (|final| for the main file and |draft| for childs):
%    \begin{macrocode}
\ifchilddoc
\providecommand{\version}{draft}
\else
\providecommand{\version}{final}
\fi
%    \end{macrocode}

% Load the standard document class:
%    \begin{macrocode}
\documentclass[12pt]{article}
%    \end{macrocode}

% Start the document body:
%    \begin{macrocode}
\begin{document}
%    \end{macrocode}

% Declare a title page.
% Print title, part of document being processed and version flag:
%    \begin{macrocode}
\addtocounter{page}{-1}
\begin{center}
{\LARGE\bfseries{}childdoc example\par}
\vspace{1cm}
\ifchilddoc
\ifchilddocmanual part\else chapter\fi:
`\childdocname' of `\childdocjob'\par
\else
main document: `\childdocjob'\par
\fi
version: \version\par
\end{center}
\newpage
%    \end{macrocode}

% Manually include selected file,
% otherwise process as usual:
%    \begin{macrocode}
\ifchilddocmanual
\section*{part `\childdocname'}
\input{\childdocname}
\else
%    \end{macrocode}

% Include the two chapters:
%    \begin{macrocode}
\include{cdocsch1}
\include{cdocsch2}
%    \end{macrocode}

% Include the two parts unless only chapters should be displayed:
%    \begin{macrocode}
\ifchilddoc\else
\section{part three}
\input{cdocspt3}
\section{part four}
\input{cdocspt4}
\fi
%    \end{macrocode}

% Process as usual until here:
%    \begin{macrocode}
\fi
%    \end{macrocode}

% End of document body:
%    \begin{macrocode}
\end{document}
%    \end{macrocode}
%\iffalse
%</samplemain>
%\fi
%
% %%%%%%%%%%%%%%%%%%%%%%%%%%%%%%%%%%%%%%
% \paragraph{Chapter Include Files.}
%
% The include files are called |cdocsch1.tex| and |cdocsch2.tex|.
%
%\iffalse
%<*samplechap1|samplechap2>
%\fi

% Optional override for |\version| flag:
%    \begin{macrocode}
%%\providecommand{\version}{final}
%    \end{macrocode}

% Include the main document:
%    \begin{macrocode}
\input{childdoc.def}
\childdocof{cdocsamp}
%    \end{macrocode}

%\iffalse
%</samplechap1|samplechap2>
%\fi
%
%\iffalse
%<*samplechap1>
%\fi
% Some text for chapter 1:
%    \begin{macrocode}
\section{one}
some text in chapter one
%    \end{macrocode}

%\iffalse
%</samplechap1>
%\fi
% Some text for chapter 2:
%\iffalse
%<*samplechap2>
%\fi
%    \begin{macrocode}
\section{two}
more text in chapter two
%    \end{macrocode}

%\iffalse
%</samplechap2>
%\fi
%
% %%%%%%%%%%%%%%%%%%%%%%%%%%%%%%%%%%%%%%
% \paragraph{Part Include Files.}
%
% The include files are called |cdocspt3.tex| and |cdocspt4.tex|.
%
%\iffalse
%<*samplepart3|samplepart4>
%\fi

% Optional override for |\version| flag:
%    \begin{macrocode}
%%\providecommand{\version}{final}
%    \end{macrocode}

% Include the main document:
%    \begin{macrocode}
\input{childdoc.def}
\childdocby{cdocsamp}
%    \end{macrocode}

%\iffalse
%</samplepart3|samplepart4>
%\fi
%
%\iffalse
%<*samplepart3>
%\fi
% Some text for part 3:
%    \begin{macrocode}
some text in part three
%    \end{macrocode}

%\iffalse
%</samplepart3>
%\fi
% Some text for part 4:
%\iffalse
%<*samplepart4>
%\fi
%    \begin{macrocode}
more text in part four
%    \end{macrocode}

%\iffalse
%</samplepart4>
%\fi
%
% %%%%%%%%%%%%%%%%%%%%%%%%%%%%%%%%%%%%%%
% \paragraph{Forwarding for a Complete Draft.}
%
% The following forwarding file |cdocsdrf.tex|
% compiles the main document in draft mode:
%\iffalse
%<*sampledraft>
%\fi
%    \begin{macrocode}
\def\version{draft}
\input{childdoc.def}
\childdocforward{cdocsamp}
%    \end{macrocode}

%\iffalse
%</sampledraft>
%\fi
%
% %%%%%%%%%%%%%%%%%%%%%%%%%%%%%%%%%%%%%%
% \paragraph{Forwarding for Final Version of the Chapters.}
%
% The following forwarding files |cdocsfn1.tex| and |cdocsfn2.tex|
% (with identical content)
% compile the final versions of the child documents
% |cdocsch1.tex| and |cdocsch2.tex|, respectively:
%\iffalse
%<*samplefinal>
%\fi
%    \begin{macrocode}
\def\version{final}
\input{childdoc.def}
\childdocforwardprefix[cdocsamp]{cdocsfn}{cdocsch}
%    \end{macrocode}

%\iffalse
%</samplefinal>
%\fi
%
% %%%%%%%%%%%%%%%%%%%%%%%%%%%%%%%%%%%%%%
% \paragraph{Command Line Processing.}
%
% The following three command lines generate the output files
% |cdocscld|, |cdocscl1| and |cdocscl2|
% which should be identical to
% |cdocsdrf|, |cdocsch1| and |cdocsfn2|, respectively:
% \begin{center}
% \begin{tabular}{l}
% |latex -jobname cdocscld \|\\
% |  "\def\version{draft}\input{childdoc.def}\childdocforward{cdocsamp}"|\\
% |latex -jobname cdocscl1 \|\\
% |  "\input{childdoc.def}\childdocforward[cdocsamp]{cdocsch1}"|\\
% |latex -jobname cdocscl2 \|\\
% |  "\def\version{final}\input{childdoc.def}\childdocforward{cdocsch2}"|
% \end{tabular}
% \end{center}
% Note that the trailing backslash on each first line
% merely continues the input to the second line
% (for convenient cut ant paste).
% Furthermore, the command |latex| can be replaced by any
% of its alternative versions such as |pdflatex|.
%
% %%%%%%%%%%%%%%%%%%%%%%%%%%%%%%%%%%%%%%%%%%%%%%%%%%%%%%%%%%%%%%%%%%%%%%%%%%%%%%
% %%%%%%%%%%%%%%%%%%%%%%%%%%%%%%%%%%%%%%%%%%%%%%%%%%%%%%%%%%%%%%%%%%%%%%%%%%%%%%
% \section{Implementation}
%\iffalse
%<*package>
%\fi
%
% This section describes the definitions file |childdoc.def|.

% The definitions cannot be loaded using |\usepackage| or |\RequirePackage|
% which has a mechanism to prevent loading a style file more than once.
% When loading the definitions by means of |\input|
% multiple instances have to be prevented manually:
%\iffalse
%This code needs to be before the `\ProvidesFile' directive
%which is defined at the beginning of this file.
%Therefore it is also placed there and commented out here.
%</package>
%<*discard>
%\fi
%    \begin{macrocode}
\ifdefined\childdocmain\endinput\fi
%    \end{macrocode}
%\iffalse
%</discard>
%<*package>
%\fi
%
% \macro{\ifchilddoc}
% \macro{\ifchilddocmanual}
% The conditional |\ifchilddoc| tells whether a
% child (true) or main (false) document is being compiled.
% The conditional |\ifchilddocmanual| tells whether
% the |\includeonly| mechanism is used (false) or
% the selection of child files must be performed manually (true).
% The definitions initialise to false:
%    \begin{macrocode}
\newif\ifchilddoc
\newif\ifchilddocmanual
%    \end{macrocode}

% \macro{\childdocname}
% \macro{\childdocjob}
% The macro |\childdocname| stores the name of the main document
% to be compiled. The macro |\childdocjob| stores the name of
% the document on which the \LaTeX{} compiler was originally invoked.
% The content of |\jobname| cannot be compared
% to filenames specified in the source due to different catcodes.
% The following code rescans |\jobname|, stores the result
% in |\childdocname| and saves a copy in |\childdocjob|:
%    \begin{macrocode}
\edef\childdocname{\scantokens\expandafter{\jobname\noexpand}}
\let\childdocjob\childdocname
%    \end{macrocode}

% \macro{\childdocdisable}
% The macro |\childdocdisable| prevents the main file
% from being processed more than once.
% At this stage, the main document command |\childdocmain|
% is assumed to be called once again where it should do nothing.
% Any subsequent call to it should prevent
% a secondary processing of the main document
% It overwrites the forwarding commands
% |\childdocof| and |\childdocforward|
% with empty macros to prevent further inclusions of the main document:
%    \begin{macrocode}
\newcommand{\childdocdisable}
{
  \renewcommand{\childdocmain}[1]{\renewcommand{\childdocmain}[1]{\endinput}}
  \renewcommand{\childdocof}[1]{}
  \renewcommand{\childdocby}[2][]{}
  \renewcommand{\childdocforward}[2][]{}
  \renewcommand{\childdocdisable}{}
}
%    \end{macrocode}

% \macro{\childdocmain}
% The macro |\childdocmain| is to be called at the top of the main file
% with nothing or the main filename (without extension) as argument.
% First, it breaks loops.
% If the argument is not empty and does not match |\childdocname|
% (which is set by the first inclusion of |childdoc.def|),
% |\ifchilddoc| is set to true, |\includeonly| is applied to the child file
% and |\jobname| is set to the main file
% (for proper handling of |.aux| files):
%    \begin{macrocode}
\newcommand{\childdocmain}[1]
{
  \childdocdisable\childdocmain{}
  \if?#1?\else
    \begingroup
      \def\childdoctmp{#1}
      \ifx\childdoctmp\childdocname
        \def\childdoctmp{}
      \else
        \def\childdoctmp
        {
          \childdoctrue
          \includeonly{\childdocname}
          \def\childdocjob{#1}
          \def\jobname{#1}
        }
      \fi
      \expandafter
    \endgroup
    \childdoctmp
  \fi
}
%    \end{macrocode}

% \macro{\childdocof}
% The command |\childdocof| redirects
% compilation to the main file |#1|.
%    \begin{macrocode}
\newcommand{\childdocof}[1]
{
  \childdocdisable
  \childdoctrue
  \includeonly{\childdocname}
  \def\jobname{#1}
  \def\childdocjob{#1}
  \input{#1}
}
%    \end{macrocode}

% \macro{\childdocby}
% The command |\childdocby| ....
%    \begin{macrocode}
\newcommand{\childdocby}[2][]
{
  \childdocdisable
  \childdoctrue
  \childdocmanualtrue
  \if?#1?\else
    \def\jobname{#2}
  \fi
  \def\childdocjob{#2}
  \input{#2}
  \endinput
}
%    \end{macrocode}

% \macro{\childdocforward}
% The command |\childdocforward| redirects
% compilation to the main file or
% (if the optional argument is given) a child file.
% Parameters are set as if the main file
% or a child file starting with |\childdocof| was compiled.
% Then compilation is handed over to the main file:
%    \begin{macrocode}
\newcommand{\childdocforward}[2][]
{
  \begingroup
    \if?#1?
      \def\childdoctmp
      {
        \def\childdocname{#2}
        \def\childdocjob{#2}
        \def\jobname{#2}
        \input{#2}
        \endinput
      }
    \else
      \def\childdoctmp
      {
        \childdocdisable
        \def\childdocname{#2}
        \childdoctrue
        \includeonly{#2}
        \def\childdocjob{#1}
        \def\jobname{#1}
        \input{#1}
        \endinput
      }
    \fi
    \expandafter
  \endgroup
  \childdoctmp
}
%    \end{macrocode}

% \macro{\childdocforwardprefix}
% The command |\childdocforwardprefix| redirects
% compilation to the main or a child file by means of a pattern.
% The prefix |#1| in the current filename is replaced by |#2|
% and the suffix of the current filename is kept
% (it is assumed that the filename does not contain the substring `|~~~|'
% which is used as a delimiter).
% Compilation is handed over to the new file by |\childdocforward|:
%    \begin{macrocode}
\newcommand{\childdocforwardprefix}[3][]
{
  \begingroup
    \def\childdocextract #2##1~~~{\def\childdoctmp{\childdocforward[#1]{#3##1}}}
    \expandafter\childdocextract\childdocname~~~
    \expandafter
  \endgroup
  \childdoctmp
}
%    \end{macrocode}

% \macro{\childdoc}
% The deprecated macro |\childdoc| is a legacy version of |\childdocmain|:
%    \begin{macrocode}
\newcommand{\childdoc}{\childdocmain}
%    \end{macrocode}

% \macro{\childdocredirect}
% The deprecated macro |\childdocredirect| is a legacy version
% of |\childdocforward| and |\childdocforwardprefix|:
%    \begin{macrocode}
\newcommand{\childdocredirect}[2][]
{
  \begingroup
    \if?#1?
      \def\childdoctmp{\childdocforward{#2}}
    \else
      \def\childdoctmp{\childdocforwardprefix{#1}{#2}}
    \fi
    \expandafter
  \endgroup
  \childdoctmp
}
%    \end{macrocode}

%\iffalse
%</package>
%\fi
%
\endinput

\childdocof{cdocsamp}
%    \end{macrocode}

%\iffalse
%</samplechap1|samplechap2>
%\fi
%
%\iffalse
%<*samplechap1>
%\fi
% Some text for chapter 1:
%    \begin{macrocode}
\section{one}
some text in chapter one
%    \end{macrocode}

%\iffalse
%</samplechap1>
%\fi
% Some text for chapter 2:
%\iffalse
%<*samplechap2>
%\fi
%    \begin{macrocode}
\section{two}
more text in chapter two
%    \end{macrocode}

%\iffalse
%</samplechap2>
%\fi
%
% %%%%%%%%%%%%%%%%%%%%%%%%%%%%%%%%%%%%%%
% \paragraph{Part Include Files.}
%
% The include files are called |cdocspt3.tex| and |cdocspt4.tex|.
%
%\iffalse
%<*samplepart3|samplepart4>
%\fi

% Optional override for |\version| flag:
%    \begin{macrocode}
%%\providecommand{\version}{final}
%    \end{macrocode}

% Include the main document:
%    \begin{macrocode}
% \iffalse
%
% childdoc.dtx Copyright (C) 2017-2018 Niklas Beisert
%
% This work may be distributed and/or modified under the
% conditions of the LaTeX Project Public License, either version 1.3
% of this license or (at your option) any later version.
% The latest version of this license is in
%   http://www.latex-project.org/lppl.txt
% and version 1.3 or later is part of all distributions of LaTeX
% version 2005/12/01 or later.
%
% This work has the LPPL maintenance status `maintained'.
%
% The Current Maintainer of this work is Niklas Beisert.
%
% This work consists of the files childdoc.dtx and childdoc.ins
% and the derived files childdoc.def and cdocsamp.tex with
% cdocsch1.tex, cdocsch2.tex, cdocsdrf.tex, cdocsfn1.tex, cdocsfn2.tex.
%
%<package>\ifdefined\childdocmain\endinput\fi
%<package>\ProvidesFile{childdoc.def}[2018/12/30 v2.0 child document driver]
%<samplemain>\ProvidesFile{cdocsamp.tex}[2018/12/30 v2.0 sample for childdoc]
%<*driver>
%\ProvidesFile{childdoc.drv}[2018/12/30 v2.0 childdoc reference manual file]
\PassOptionsToClass{10pt,a4paper}{article}
\documentclass{ltxdoc}

\usepackage[margin=35mm]{geometry}
\usepackage{hyperref}
\usepackage{hyperxmp}
\usepackage[usenames]{color}

\hypersetup{colorlinks=true}
\hypersetup{pdfstartview=FitH}
\hypersetup{pdfpagemode=UseNone}
\hypersetup{pdfsource={}}
\hypersetup{pdflang={en-UK}}
\hypersetup{pdfcopyright={Copyright 2017-2018 Niklas Beisert.
  This work may be distributed and/or modified under the
  conditions of the LaTeX Project Public License, either version 1.3
  of this license or (at your option) any later version.}}
\hypersetup{pdflicenseurl={http://www.latex-project.org/lppl.txt}}
\hypersetup{pdfcontactaddress={ETH Zurich, ITP, HIT K,
  Wolfgang-Pauli-Strasse 27}}
\hypersetup{pdfcontactpostcode={8093}}
\hypersetup{pdfcontactcity={Zurich}}
\hypersetup{pdfcontactcountry={Switzerland}}
\hypersetup{pdfcontactemail={nbeisert@itp.phys.ethz.ch}}
\hypersetup{pdfcontacturl={http://people.phys.ethz.ch/\xmptilde nbeisert/}}

\newcommand{\secref}[1]{\hyperref[#1]{section \ref*{#1}}}

\parskip1ex
\parindent0pt
\let\olditemize\itemize
\def\itemize{\olditemize\parskip0pt}

\begin{document}

\title{The \textsf{childdoc} Package}
\hypersetup{pdftitle={The childdoc Package}}
\author{Niklas Beisert\\[2ex]
  Institut f\"ur Theoretische Physik\\
  Eidgen\"ossische Technische Hochschule Z\"urich\\
  Wolfgang-Pauli-Strasse 27, 8093 Z\"urich, Switzerland\\[1ex]
  \href{mailto:nbeisert@itp.phys.ethz.ch}
  {\texttt{nbeisert@itp.phys.ethz.ch}}}
\hypersetup{pdfauthor={Niklas Beisert}}
\hypersetup{pdfsubject={Manual for the LaTeX2e Package childdoc}}
\date{30 December 2018, \textsf{v2.0}}
\maketitle

\begin{abstract}\noindent
\textsf{childdoc} is a \LaTeXe{} package
that enables the direct compilation
of document sections included by |\include|
to individual files.
\end{abstract}

\begingroup
\parskip0ex
\tableofcontents
\endgroup

%%%%%%%%%%%%%%%%%%%%%%%%%%%%%%%%%%%%%%%%%%%%%%%%%%%%%%%%%%%%%%%%%%%%%%%%%%%%%%%%
%%%%%%%%%%%%%%%%%%%%%%%%%%%%%%%%%%%%%%%%%%%%%%%%%%%%%%%%%%%%%%%%%%%%%%%%%%%%%%%%
\section{Introduction}

\LaTeX{} provides a mechanism to structure a large document (such as a book)
into a main file and several child files (containing the chapters)
using the |\include| command.
This mechanism is beneficial for documents
which span hundreds of pages in order to
make the source file(s) more manageable.
Moreover, compilation can be restricted to
selected child files by means of the |\includeonly| command.
The latter feature can be used to reduce the compilation time while editing
(this was significantly more useful in the earlier days of \LaTeX{})
or to generate a smaller document which is easier to navigate.
Another application of |\includeonly| is to generate
documents consisting of selected parts of the complete document.

However, there are a few drawbacks of the plain |\include| mechanism:
\begin{itemize}
\item
The child files cannot be compiled on their own,
they can only be compiled via the main file.
A naive editing environment
(such as a text editor with an option
to have the current file processed by \LaTeX)
may require one to switch to the main file before compiling;
attempting to compile the child file produces errors.
\item
The main file must be modified (each time)
to adjust the |\includeonly| command
to the present needs. This easily leaves the main file in a messy state.
\item
The generated document will always carry the filename
of the main document. This is inconvenient if
several child files are to be compiled and
to be kept for distribution.
\end{itemize}

The present package provides a simple interface
to make child files individually compilable by \LaTeX{}.
Compiling a child file then has the same effect as compiling
the main file with an |\includeonly| command
to select the appropriate child.
Moreover the generated document will carry the name of the child
rather than the main file.
This resolves all three above issues.

This feature is meant to make the editing of books,
thesis documents and lecture notes somewhat more convenient.
However, the package can also be used efficiently for
composing a series of documents (such as exercise sheets)
which are typically distributed individually.
It then assists the author in generating the individual documents
(potentially in different versions)
as well as a document containing the collected series.
Another application is in developing style files
or other kinds of included material
where compilation of the style file could redirect
to a sample or test file.

%%%%%%%%%%%%%%%%%%%%%%%%%%%%%%%%%%%%%%%%%%%%%%%%%%%%%%%%%%%%%%%%%%%%%%%%%%%%%%%%
%%%%%%%%%%%%%%%%%%%%%%%%%%%%%%%%%%%%%%%%%%%%%%%%%%%%%%%%%%%%%%%%%%%%%%%%%%%%%%%%
\section{Usage}

First of all, the package \textsf{childdoc} is \emph{not} a standard
\LaTeXe{} |.sty| style file! Therefore it needs to be invoked in
a non-standard way.

%%%%%%%%%%%%%%%%%%%%%%%%%%%%%%%%%%%%%%%%%%%%%%%%%%%%%%%%%%%%%%%%%%%%%%%%%%%%%%%%
\subsection{Included Files}
\label{sec:include}

%%%%%%%%%%%%%%%%%%%%%%%%%%%%%%%%%%%%%%%%
\DescribeMacro{\childdocmain}
To use the package, add the commands
\begin{center}
\begin{tabular}{l}
|\input{childdoc.def}|\\
|\childdocmain{}|\\
\end{tabular}
\end{center}
at the very top of the main \LaTeX{} file,
in particular \emph{before} the |\documentclass| statement!
The argument of |\childdocmain| should be left empty
(but it must be present).

%%%%%%%%%%%%%%%%%%%%%%%%%%%%%%%%%%%%%%%%
\DescribeMacro{\childdocof}
Furthermore, add the commands
\begin{center}
\begin{tabular}{l}
|\input{childdoc.def}|\\
|\childdocof{|\textit{main}|}|\\
\end{tabular}
\end{center}
at the top of every child file \textit{child}
which is included by |\include{|\textit{child}|}|
from within the main file
(or at least for those files to be compiled individually).
The argument \textit{main} must be the filename of the main file.

There are a couple of
considerations in setting up the main and child documents:

%%%%%%%%%%%%%%%%%%%%%%%%%%%%%%%%%%%%%%%%
\paragraph{Restrictions.}

Please note the following restrictions:
\begin{itemize}
\item
|\childdocmain| must be called with one argument \textit{main}
to ensure compatibility with earlier version of the package.
It must either be empty (|\childdocmain{}|)
or precisely match the filename of the main file in which it is specified.
See \secref{sec:detection} for further information.
\item
The filename \textit{main} must be specified without the |.tex| extension.
\item
The filename \textit{main} is case sensitive
(even in case-insensitive file systems)
due to internal string comparison.
\item
The argument \textit{main} should be fully expanded, it cannot be a macro.
\item
Subdirectories and special characters should be avoided in filenames.
\item
The command |\childdocmain{|\textit{main}|}| must be followed by a whitespace.
It should not be followed immediately by another command
or by a comment mark `|%|'.
This is because the \TeX{} parser reads the token immediately following
the argument of |\childdocmain| and puts it
at the beginning of every child section;
however, a white\-space is ignored.
\end{itemize}

%%%%%%%%%%%%%%%%%%%%%%%%%%%%%%%%%%%%%%%%
\paragraph{Content of Main File.}

It is advisable to place all content in the child files included by |\include|.
Any output contained in the main file will appear in all child documents
unless suppressed manually;
it cannot be suppressed automatically by the |\includeonly| directive
and thus should normally be avoided.
A method to include some content in the main file
by means of conditional processing is described in \secref{sec:conditional}.

%%%%%%%%%%%%%%%%%%%%%%%%%%%%%%%%%%%%%%%%
\paragraph{Page Numbering.}

When only a part of the document is compiled,
the appropriate numbering of pages
(as well as other status parameters)
is determined from the |.aux| files.
The latter contain information from previous passes.
However this information needs to propagate through
all intermediate child documents.
Therefore the page numbering in child documents may well
be inconsistent until the complete document is compiled at least once.

A useful (if unconventional) way to always ensure a consistent
page numbering is to restart the numbering in each child document
and denote the pages by `\textit{child}|.|\textit{page}'
where \textit{child} represents the chapter/section number of the child file.
This can be achieved by the command
|\numberwithin{page}{|\textit{child}|}|
of the \textsf{amsmath} package
where \textit{child} can be |chapter| or |section|
depending on the chosen structuring.
Alternatively, one can modify the macro |\thepage| appropriately
and reset the counter |page| at the start of each child file.

%%%%%%%%%%%%%%%%%%%%%%%%%%%%%%%%%%%%%%%%%%%%%%%%%%%%%%%%%%%%%%%%%%%%%%%%%%%%%%%%
\subsection{Conditional Processing}
\label{sec:conditional}

The package provides a mechanism to compile different versions
of a document. To customise the versions further some conditional processing
can come in handy to distinguish which version is being compiled.
The package provides two macros to describe the compilation context:

%%%%%%%%%%%%%%%%%%%%%%%%%%%%%%%%%%%%%%%%
\DescribeMacro{\ifchilddoc}
The conditional |\ifchilddoc| distinguishes between the compilation of
child documents and the main document:
%
\begin{center}
|\ifchilddoc |\textit{child-code}| |[|\||else |\textit{main-code}]| \||fi|
\end{center}

%%%%%%%%%%%%%%%%%%%%%%%%%%%%%%%%%%%%%%%%
\DescribeMacro{\childdocname}
\DescribeMacro{\childdocjob}
The macro |\childdocname| contains the filename (without extension)
of the main or child file being processed.
Note that |\childdocjob| will always contain the name of the main file.

%%%%%%%%%%%%%%%%%%%%%%%%%%%%%%%%%%%%%%%%
\paragraph{Title Page.}

Conditional processing can be used to include a title or banner page
in the main document when proper precautions are taken.
Importantly, the code in the main file should ensure that the page counter
(as well as other status parameters which are stored in the |.aux| files)
takes the same value after the conditional processing.
Otherwise the page numbers may take divergent values
depending on which part is compiled.

For example, a title page could be declared by:
%
\begin{center}
\begin{tabular}{l}
|\ifchilddoc\||else|\\
|\addtocounter{page}{-1}|\\
\textit{code for title page}\\
|\newpage|\\
|\||fi|
\end{tabular}
\end{center}
%
A banner page for the child documents can be generated by:
%
\begin{center}
\begin{tabular}{l}
|\ifchilddoc|\\
|\addtocounter{page}{-1}|\\
\textit{code for banner page}\\
|\newpage|\\
|\||fi|
\end{tabular}
\end{center}
%
Here one could write a message such as:
\begin{center}
|This is the part \childdocname{} of \childdocjob{}.|
\end{center}

%%%%%%%%%%%%%%%%%%%%%%%%%%%%%%%%%%%%%%%%%%%%%%%%%%%%%%%%%%%%%%%%%%%%%%%%%%%%%%%%
\subsection{Flags}
\label{sec:flags}

The package makes it easy to generate different versions
of the main or child documents.
To this end compilation flags can be defined
and assigned different default values.
They will be particularly useful in conjunction
with the forwarding mechanism described in \secref{sec:forward}.

For example, it may be useful to have a flag |\version|
which can be set to |draft| or |final|.
The document source will contain some conditional code
depending on the value of |\version|.
Suppose further, the flag should default to |final| for the main file
and to |draft| for child files
which is a natural assignment for editing the document.
This is achieved by placing the following code
in the preamble of the main document
(below the |\childdocmain| directive):
%
\begin{center}
\begin{tabular}{l}
|\ifchilddoc|\\
|\providecommand{\version}{draft}|\\
|\||else|\\
|\providecommand{\version}{final}|\\
|\||fi|
\end{tabular}
\end{center}
%
The definition by |\providecommand| makes sure
that previous definitions are not overwritten.
Further statements |\providecommand{\version}{...}|
can thus be added before the above code to override it.

For the main file, one might add a line
(between |\childdocmain| and the above block)
%
\begin{center}
|%\ifchilddoc\||else\providecommand{\version}{draft}\||fi|
\end{center}
%
which can be uncommented to produce a draft version.
Likewise one can add a line to the very top of a child file
(above the |\childdocof{|\textit{main}|}| directive)
%
\begin{center}
|%\providecommand{\version}{final}|
\end{center}
%
which can be uncommented to produce the final version of this child document.

%%%%%%%%%%%%%%%%%%%%%%%%%%%%%%%%%%%%%%%%%%%%%%%%%%%%%%%%%%%%%%%%%%%%%%%%%%%%%%%%
\subsection{Forwarding}
\label{sec:forward}

Different versions of the main or child documents
using compilation flags as described in \secref{sec:flags}
can be (permanently) stored in different files
for convenient compilation, viewing and distribution.
To this end, the package defines a command
to pass on compilation to a different file:

%%%%%%%%%%%%%%%%%%%%%%%%%%%%%%%%%%%%%%%%
\DescribeMacro{\childdocforward}
The command |\childdocforward| redirects processing to
another source file:
%
\begin{center}
\begin{tabular}{l}
|\input{childdoc.def}|\\
|\childdocforward[|\textit{main}|]{|\textit{dest}|}|\\
\end{tabular}
\end{center}
%
The argument \textit{dest} is the destination file
(without extension).
It should be the main file or one of the child files.
Note that further \textsf{childdoc} directives
such as |\childdocof| and |\childdocforward|
in the indicated file will be processed in this form.
The optional argument \textit{main}
passes on directly to the main file \textit{main}
while pretending to compile the child \textit{dest}.
This form behaves as if \textit{dest}
issues |\childdocof{|\textit{main}|}| right away,
and no further \textsf{childdoc} directives will be processed.

%%%%%%%%%%%%%%%%%%%%%%%%%%%%%%%%%%%%%%%%
\DescribeMacro{\...prefix}
In the alternative form |\childdocforwardprefix|,
%
\begin{center}
\begin{tabular}{l}
|\input{childdoc.def}|\\
|\childdocforwardprefix[|\textit{main}|]{|\textit{prefix}|}{|\textit{dest}|}|
\end{tabular}
\end{center}
%
the destination file is determined by a pattern
depending on the current file:
To make this work, the current file must be called
`{\textit{prefix}\hspace{0.2em}\textit{suffix}}'
with \textit{prefix} matching precisely the argument.
Processing is then passed on to the file
`{\textit{dest}\hspace{0.2em}\textit{suffix}}'.
Surely, the same effect is achieved by
directly specifying the
argument `{\textit{dest}\hspace{0.2em}\textit{suffix}}'
in the first form.
However, that requires to set up a different file
for each child. With the alternative form of the command
all these files can have exactly the same content
which simplifies setting them up and maintaining them.

For example, the following file |draft.tex|
with a compilation flag |\version| as described in \secref{sec:flags}
compiles the main document as a draft:
%
\begin{center}
\begin{tabular}{l}
|\def\version{draft}|\\
|\input{childdoc.def}|\\
|\childdocforward{|\textit{main}|}|
\end{tabular}
\end{center}
%
Likewise, the following files |final|\textit{nn}|.tex|
compile the final version of the child document
|child|\textit{nn}|.tex|:
%
\begin{center}
\begin{tabular}{l}
|\def\version{final}|\\
|\input{childdoc.def}|\\
|\childdocforwardprefix{final}{child}|
\end{tabular}
\end{center}
%

Note that when several versions of a main file and/or of each child file
are to be generated, it may be convenient to set up a |Makefile| or
shell script to automatise the process.

%%%%%%%%%%%%%%%%%%%%%%%%%%%%%%%%%%%%%%%%%%%%%%%%%%%%%%%%%%%%%%%%%%%%%%%%%%%%%%%%
\subsection{Command Line Processing}
\label{sec:commandline}

The effect of redirection files can also be achieved by invoking
the \LaTeX{} compiler with a more elaborate command line.
Most conveniently this should be done as part
of a shell script or a |Makefile|.

When using \textsf{childdoc} in the main file, the following
command lines effectively perform a redirection
(note that depending on the shell being used,
backslashes may have to be doubled: `|\|' $\to$ `|\\|'):
%
\begin{center}
|... -jobname "|\textit{target}|" |\\|"|[\textit{flags}]%
|\input{childdoc.def}\childdocforward[|\textit{main}|]{|\textit{dest}|}"|
\end{center}
%
Here \textit{target} is the name of the output file,
\textit{main} is the name of the main file
and \textit{dest} is the name of the main or child file to be processed
(all filenames without extensions).
The optional argument \textit{main} can be omitted
if \textit{main} matches \textit{dest}.
Optionally, compilation \textit{flags} can be defined via |\def| commands.
This command line makes the \TeX{} engine believe
it is compiling the file \textit{target}
whose content is specified as the latter parameter.
The provided code then forwards the processing to
\textit{main} or \textit{dest} as described in \secref{sec:forward}.

%%%%%%%%%%%%%%%%%%%%%%%%%%%%%%%%%%%%%%%%%%%%%%%%%%%%%%%%%%%%%%%%%%%%%%%%%%%%%%%%
\subsection{Include by Input}
\label{sec:input}

Including child documents by |\include| has some restrictions by design.
Most notably, the content of a child document always occupies
its own set of pages; pages cannot be shared between child documents.
Usually, this behaviour makes perfect sense
because each child document contain an essential part of the document.
However, in some situations it may be desirable to compose
a document from a collection of parts
without having mandatory page breaks between then.
For this case, the package
provides a mechanism to include parts
by |\input| which can also be processed individually.
However, by construction this mechanism
requires manual handling of the content to be output.

%%%%%%%%%%%%%%%%%%%%%%%%%%%%%%%%%%%%%%%%
\DescribeMacro{\ifchilddocmanual}
The main file should be prepared as usual, see \secref{sec:include}.
However, the document body must make a distinction
between processing of an individual part and of the main document, e.g.:
%
\begin{center}
\begin{tabular}{l}
|\ifchilddocmanual|\\
|\input{\childdocname}|\\
|\||else|\\
\textit{document body with }|\input{|\textit{part}|}|\\
|\||fi|
\end{tabular}
\end{center}
%
The conditional |\ifchilddocmanual| is true whenever
a part to be included by |\input| is being compiled,
and the name of the part is stored in |\childdocname|.

%%%%%%%%%%%%%%%%%%%%%%%%%%%%%%%%%%%%%%%%
\DescribeMacro{\childdocby}
Each part to be included by |\input| should start with:
%
\begin{center}
\begin{tabular}{l}
|\input{childdoc.def}|\\
|\childdocby{|\textit{main}|}|\\
\end{tabular}
\end{center}
%
The directive |\childdocby| is similar to |\childdocof|
described in \secref{sec:include},
but the subsequent selection of content must be done manually.
To that end, both |\ifchilddoc| and |\ifchilddocmanual|
will be true upon processing of a part,
and the name of the part is stored in |\childdocname|.
Note that |\jobname| will be set to the filename of the current part
so that each part receives an individual |.aux| file
that does not interfere with the |.aux| file(s) of the main document.
This behaviour can be altered by the alternative form
|\childdocby[*]{|\textit{main}|}| (with a non-empty optional argument)
which uses the |.aux| file of the main document
by setting |\jobname| to \textit{main}.

%%%%%%%%%%%%%%%%%%%%%%%%%%%%%%%%%%%%%%%%%%%%%%%%%%%%%%%%%%%%%%%%%%%%%%%%%%%%%%%%
\subsection{Driver Development}
\label{sec:driver}

The \textsf{childdoc} mechanism can also be use for the development
of definition files such as \LaTeX{} styles or classes.
This case differs from the above setup with multiple parts
included by |\include| in that no |\includeonly| should be invoked.
This can be achieved by starting the include file
(before |\ProvidesPackage|) with:
%
\begin{center}
\begin{tabular}{l}
|\input{childdoc.def}|\\
|\childdocforward{|\textit{main}|}|\\
\end{tabular}
\end{center}
%
or alternatively with:
%
\begin{center}
\begin{tabular}{l}
|\input{childdoc.def}|\\
|\childdocby{|\textit{main}|}|\\
\end{tabular}
\end{center}
%
Both forms have slightly different effects as described above.
The main file is prepared as usual, see \secref{sec:include}.

%%%%%%%%%%%%%%%%%%%%%%%%%%%%%%%%%%%%%%%%%%%%%%%%%%%%%%%%%%%%%%%%%%%%%%%%%%%%%%%%
\subsection{Legacy Detection}
\label{sec:detection}

The directive |\childdocmain| in the main file can detect
whether the complete document or merely a child is to be compiled
even without using the directive |\childdocof|.
This method is deprecated because it is less robust
and there is no compelling reason to use it;
it is merely provided for backward compatibility
and it may be removed in future versions.

If the detection mechanism is to be used,
it is mandatory to correctly specify
the filename of the main file as the argument of |\childdocmain|:
%
\begin{center}
\begin{tabular}{l}
|\input{childdoc.def}|\\
|\childdocmain{|\textit{main}|}|\\
\end{tabular}
\end{center}
%
If |\jobname| does not match the argument \textit{main} of |\childdocmain|,
it is assumed that |\jobname| points to the child file to be compiled.
When using |\childdocmain| with the main file specified as argument,
it suffices to start a child file
with just |\input{|\textit{main}|}|
without loading of the package and using |\childdocof|.
If instead all processing is done
with the appropriate \textsf{childdoc} directives,
the argument of \textit{main} of |\childdocmain| can be empty.

An alternative version of the command line processing described
in \secref{sec:commandline} using the detection mechanism reads:
%
\begin{center}
|... -jobname "|\textit{target}|" "|[\textit{flags}]%
[|\def\jobname{|\textit{dest}|}|]|\input{|\textit{main}|}"|
\end{center}

%%%%%%%%%%%%%%%%%%%%%%%%%%%%%%%%%%%%%%%%%%%%%%%%%%%%%%%%%%%%%%%%%%%%%%%%%%%%%%%%
\subsection{Manual Code}
\label{sec:manual}

In case one cannot be certain whether the definitions file |childdoc.def|
is installed on the target \TeX{} distribution
and one prefers not to ship it,
it is conceivable to paste a few relevant commands into the sources.

To that end, drop all statements |\input{childdoc.def}|
and perform the replacements as outlined below.
Instead of |\childdocmain{|\textit{main}|}| add the following code
to the top of the main file:
%
\begin{center}
\begin{tabular}{l}
|\||ifdefined\childdocname\endinput\||fi\newif\ifchilddoc|\\
|\edef\childdocname{\scantokens\expandafter{\jobname\noexpand}}|\\
|\def\childdocmain{|\textit{main}|}\||ifx\childdocmain\childdocname\||else|\\
|\childdoctrue\includeonly{\childdocname}\let\jobname\childdocmain\||fi|\\
\end{tabular}
\end{center}
%
Instead of |\childdocof{|\textit{main}|}| just include the main file
at the top of each child file:
%
\begin{center}
|\input{|\textit{main}|}|
\end{center}
%
A simple redirection |\childdocforward{|\textit{dest}|}| is achieved by:
%
\begin{center}
|\def\jobname{|\textit{dest}|}\input{\jobname}|
\end{center}
%
The redirection with prefix
|\childdocforwardprefix[|\textit{prefix}|]{|\textit{dest}|}|
is accomplished by:
%
\begin{center}
\begin{tabular}{l}
|{\edef\jobname{\scantokens\expandafter{\jobname\noexpand}}|\\
|\def\redirectjob |\textit{prefix}|#1~~~{\gdef\jobname{|\textit{dest}|#1}}|\\
|\expandafter\redirectjob\jobname~~~}\input{\jobname}|
\end{tabular}
\end{center}

In an alternative approach,
child documents can be compiled by a specific command line
without additional code or specific definitions:
%
\begin{center}
|... -jobname "|\textit{target}|" "|[\textit{flags}]%
|\includeonly{|\textit{dest}|}\input{|\textit{main}|}"|
\end{center}
%

%%%%%%%%%%%%%%%%%%%%%%%%%%%%%%%%%%%%%%%%%%%%%%%%%%%%%%%%%%%%%%%%%%%%%%%%%%%%%%%%
%%%%%%%%%%%%%%%%%%%%%%%%%%%%%%%%%%%%%%%%%%%%%%%%%%%%%%%%%%%%%%%%%%%%%%%%%%%%%%%%
\section{Information}

%%%%%%%%%%%%%%%%%%%%%%%%%%%%%%%%%%%%%%%%%%%%%%%%%%%%%%%%%%%%%%%%%%%%%%%%%%%%%%%%
\subsection{Copyright}

Copyright \copyright{} 2017--2018 Niklas Beisert

This work may be distributed and/or modified under the
conditions of the \LaTeX{} Project Public License, either version 1.3
of this license or (at your option) any later version.
The latest version of this license is in
  \url{http://www.latex-project.org/lppl.txt}
and version 1.3 or later is part of all distributions of \LaTeX{}
version 2005/12/01 or later.

This work has the LPPL maintenance status `maintained'.

The Current Maintainer of this work is Niklas Beisert.

This work consists of the files |README.txt|, |childdoc.ins| and |childdoc.dtx|
as well as the derived files |childdoc.def|, |cdocsamp.tex|
with |cdocsch1.tex|, |cdocsch2.tex|, |cdocspt3.tex|, |cdocspt4.tex|,
|cdocsdrf.tex|, |cdocsfn1.tex|, |cdocsfn2.tex|
as well as |childdoc.pdf|.

%%%%%%%%%%%%%%%%%%%%%%%%%%%%%%%%%%%%%%%%%%%%%%%%%%%%%%%%%%%%%%%%%%%%%%%%%%%%%%%%
\subsection{Files and Installation}

The package consists of the files:
%
\begin{center}
\begin{tabular}{ll}
    |README.txt|   & readme file \\
    |childdoc.ins| & installation file \\
    |childdoc.dtx| & source file \\
    |childdoc.def| & definition file \\
    |cdocsamp.tex| & sample main file \\
    |cdocsch1.tex| & sample include file \\
    |cdocsch2.tex| & sample include file \\
    |cdocspt3.tex| & sample part file \\
    |cdocspt4.tex| & sample part file \\
    |cdocsdrf.tex| & sample redirection file \\
    |cdocsfn1.tex| & sample redirection file \\
    |cdocsfn2.tex| & sample redirection file \\
    |childdoc.pdf| & manual
\end{tabular}
\end{center}
%
The distribution consists of the files
|README.txt|, |childdoc.ins| and |childdoc.dtx|.
%
\begin{itemize}
\item
Run (pdf)\LaTeX{} on |childdoc.dtx|
to compile the manual |childdoc.pdf| (this file).
\item
Run \LaTeX{} on |childdoc.ins| to create the definitions file |childdoc.def|
and the sample |cdocsamp.tex| with include files
|cdocsch1.tex|, |cdocsch2.tex|, |cdocspt3.tex|, |cdocspt4.tex|,
|cdocsdrf.tex|, |cdocsfn1.tex|, |cdocsfn2.tex|.
Then copy the file |childdoc.def| to an appropriate directory of your \LaTeX{}
distribution, e.g.\ \textit{texmf-root}|/tex/latex/childdoc|.
\end{itemize}

%%%%%%%%%%%%%%%%%%%%%%%%%%%%%%%%%%%%%%%%%%%%%%%%%%%%%%%%%%%%%%%%%%%%%%%%%%%%%%%%
\subsection{Related CTAN Packages}

There are several other packages which offer a similar functionality:
%
\begin{itemize}
\item
The packages
\href{http://ctan.org/pkg/docmute}{\textsf{docmute}},
\href{http://ctan.org/pkg/includex}{\textsf{includex}} and
\href{http://ctan.org/pkg/standalone}{\textsf{standalone}}
provide commands to include only the document body of
a child file thus allowing both files to be compiled individually.
\item
The packages \href{http://ctan.org/pkg/subdocs}{\textsf{subdocs}}
and \href{http://ctan.org/pkg/subfiles}{\textsf{subfiles}}
provide structures in which the main and child documents can be
encapsulated and allowing them to be compiled individually.
The inclusion mechanism is different from the conventional |\include|.
\item
The package \href{http://ctan.org/pkg/combine}{\textsf{combine}}
is an elaborate solution to combine several documents into one.
\end{itemize}
%
See also the CTAN topic \href{http://ctan.org/topic/subdocs}{\textsf{subdocs}}
for further related packages.
The present package differs from the above solutions in that
a document structure constructed with the conventional |\include| mechanism
just needs two extra commands at the top of every file
such that all constituent files can be compiled individually.

%%%%%%%%%%%%%%%%%%%%%%%%%%%%%%%%%%%%%%%%%%%%%%%%%%%%%%%%%%%%%%%%%%%%%%%%%%%%%%%%
%\subsection{Feature Suggestions}
%
%The following is a list of features which may be useful for future
%versions of this package:
%%
%\begin{itemize}
%\item
%\ldots
%\end{itemize}

%%%%%%%%%%%%%%%%%%%%%%%%%%%%%%%%%%%%%%%%%%%%%%%%%%%%%%%%%%%%%%%%%%%%%%%%%%%%%%%%
\subsection{Revision History}

%%%%%%%%%%%%%%%%%%%%%%%%%%%%%%%%%%%%%%%%
\paragraph{v2.0:} 2018/12/30

\begin{itemize}
\item
immediate forward processing
\item
added |\childdocby| mechanism
\item
manual restructured
\end{itemize}

%%%%%%%%%%%%%%%%%%%%%%%%%%%%%%%%%%%%%%%%
\paragraph{v1.6:} 2018/01/17

\begin{itemize}
\item
application for development of include files
\item
corrections to manual
\end{itemize}

%%%%%%%%%%%%%%%%%%%%%%%%%%%%%%%%%%%%%%%%
\paragraph{v1.5:} 2017/05/21

\begin{itemize}
\item
more complete structuring introduced
\item
|\childdocof| introduced
\item
|\childdoc| renamed to |\childdocmain|
\item
|\childredirect| renamed to |\childdocforward| and |\childdocforwardprefix|
and functionality expanded
\end{itemize}

%%%%%%%%%%%%%%%%%%%%%%%%%%%%%%%%%%%%%%%%
\paragraph{v1.0:} 2017/04/27

\begin{itemize}
\item
manual and install package
\item
first version published on CTAN
\end{itemize}

%%%%%%%%%%%%%%%%%%%%%%%%%%%%%%%%%%%%%%%%
\paragraph{v0.6:} 2017/04/26

\begin{itemize}
\item
redirection mechanism added
\end{itemize}

%%%%%%%%%%%%%%%%%%%%%%%%%%%%%%%%%%%%%%%%
\paragraph{v0.5:} 2017/04/26

\begin{itemize}
\item
functionality in definition file
\end{itemize}


%%%%%%%%%%%%%%%%%%%%%%%%%%%%%%%%%%%%%%%%%%%%%%%%%%%%%%%%%%%%%%%%%%%%%%%%%%%%%%%%
%%%%%%%%%%%%%%%%%%%%%%%%%%%%%%%%%%%%%%%%%%%%%%%%%%%%%%%%%%%%%%%%%%%%%%%%%%%%%%%%
%%%%%%%%%%%%%%%%%%%%%%%%%%%%%%%%%%%%%%%%%%%%%%%%%%%%%%%%%%%%%%%%%%%%%%%%%%%%%%%%
\appendix

\settowidth\MacroIndent{\rmfamily\scriptsize 000\ }

 \DocInput{childdoc.dtx}

\end{document}
%</driver>
% \fi
%
% %%%%%%%%%%%%%%%%%%%%%%%%%%%%%%%%%%%%%%%%%%%%%%%%%%%%%%%%%%%%%%%%%%%%%%%%%%%%%%
% %%%%%%%%%%%%%%%%%%%%%%%%%%%%%%%%%%%%%%%%%%%%%%%%%%%%%%%%%%%%%%%%%%%%%%%%%%%%%%
% \section{Sample}
%\iffalse
%<*samplemain>
%\fi
%
% The following presents a sample document
% with two chapters, two parts, a title page,
% a compile flag as well as three forwarding files to set the flag.
% It consists of eight |.tex| files:
% \begin{center}
% \begin{tabular}{ll}
% |cdocsamp.tex|&main file\\
% |cdocsch1.tex|&include file for chapter 1\\
% |cdocsch2.tex|&include file for chapter 2\\
% |cdocspt3.tex|&include file for part 3\\
% |cdocspt4.tex|&include file for part 4\\
% |cdocsdrf.tex|&forwarding file for main file in draft mode\\
% |cdocsfi1.tex|&forwarding file for final version of chapter 1\\
% |cdocsfi2.tex|&forwarding file for final version of chapter 2\\
% \end{tabular}
% \end{center}
% Each of the eight files can be compiled directly by the \LaTeX{} compiler.
%
% %%%%%%%%%%%%%%%%%%%%%%%%%%%%%%%%%%%%%%
% \paragraph{Main File.}
%
% The main file is called |cdocsamp.tex|.
%
% Load the \textsf{childdoc} definitions and
% declare the filename for the main document:
%    \begin{macrocode}
\input{childdoc.def}
\childdocmain{}
%    \end{macrocode}

% Optional override for |\version| flag:
%    \begin{macrocode}
%%\ifchilddoc\else\providecommand{\version}{draft}\fi
%    \end{macrocode}

% Define the default values for the |\version| flag
% (|final| for the main file and |draft| for childs):
%    \begin{macrocode}
\ifchilddoc
\providecommand{\version}{draft}
\else
\providecommand{\version}{final}
\fi
%    \end{macrocode}

% Load the standard document class:
%    \begin{macrocode}
\documentclass[12pt]{article}
%    \end{macrocode}

% Start the document body:
%    \begin{macrocode}
\begin{document}
%    \end{macrocode}

% Declare a title page.
% Print title, part of document being processed and version flag:
%    \begin{macrocode}
\addtocounter{page}{-1}
\begin{center}
{\LARGE\bfseries{}childdoc example\par}
\vspace{1cm}
\ifchilddoc
\ifchilddocmanual part\else chapter\fi:
`\childdocname' of `\childdocjob'\par
\else
main document: `\childdocjob'\par
\fi
version: \version\par
\end{center}
\newpage
%    \end{macrocode}

% Manually include selected file,
% otherwise process as usual:
%    \begin{macrocode}
\ifchilddocmanual
\section*{part `\childdocname'}
\input{\childdocname}
\else
%    \end{macrocode}

% Include the two chapters:
%    \begin{macrocode}
\include{cdocsch1}
\include{cdocsch2}
%    \end{macrocode}

% Include the two parts unless only chapters should be displayed:
%    \begin{macrocode}
\ifchilddoc\else
\section{part three}
\input{cdocspt3}
\section{part four}
\input{cdocspt4}
\fi
%    \end{macrocode}

% Process as usual until here:
%    \begin{macrocode}
\fi
%    \end{macrocode}

% End of document body:
%    \begin{macrocode}
\end{document}
%    \end{macrocode}
%\iffalse
%</samplemain>
%\fi
%
% %%%%%%%%%%%%%%%%%%%%%%%%%%%%%%%%%%%%%%
% \paragraph{Chapter Include Files.}
%
% The include files are called |cdocsch1.tex| and |cdocsch2.tex|.
%
%\iffalse
%<*samplechap1|samplechap2>
%\fi

% Optional override for |\version| flag:
%    \begin{macrocode}
%%\providecommand{\version}{final}
%    \end{macrocode}

% Include the main document:
%    \begin{macrocode}
\input{childdoc.def}
\childdocof{cdocsamp}
%    \end{macrocode}

%\iffalse
%</samplechap1|samplechap2>
%\fi
%
%\iffalse
%<*samplechap1>
%\fi
% Some text for chapter 1:
%    \begin{macrocode}
\section{one}
some text in chapter one
%    \end{macrocode}

%\iffalse
%</samplechap1>
%\fi
% Some text for chapter 2:
%\iffalse
%<*samplechap2>
%\fi
%    \begin{macrocode}
\section{two}
more text in chapter two
%    \end{macrocode}

%\iffalse
%</samplechap2>
%\fi
%
% %%%%%%%%%%%%%%%%%%%%%%%%%%%%%%%%%%%%%%
% \paragraph{Part Include Files.}
%
% The include files are called |cdocspt3.tex| and |cdocspt4.tex|.
%
%\iffalse
%<*samplepart3|samplepart4>
%\fi

% Optional override for |\version| flag:
%    \begin{macrocode}
%%\providecommand{\version}{final}
%    \end{macrocode}

% Include the main document:
%    \begin{macrocode}
\input{childdoc.def}
\childdocby{cdocsamp}
%    \end{macrocode}

%\iffalse
%</samplepart3|samplepart4>
%\fi
%
%\iffalse
%<*samplepart3>
%\fi
% Some text for part 3:
%    \begin{macrocode}
some text in part three
%    \end{macrocode}

%\iffalse
%</samplepart3>
%\fi
% Some text for part 4:
%\iffalse
%<*samplepart4>
%\fi
%    \begin{macrocode}
more text in part four
%    \end{macrocode}

%\iffalse
%</samplepart4>
%\fi
%
% %%%%%%%%%%%%%%%%%%%%%%%%%%%%%%%%%%%%%%
% \paragraph{Forwarding for a Complete Draft.}
%
% The following forwarding file |cdocsdrf.tex|
% compiles the main document in draft mode:
%\iffalse
%<*sampledraft>
%\fi
%    \begin{macrocode}
\def\version{draft}
\input{childdoc.def}
\childdocforward{cdocsamp}
%    \end{macrocode}

%\iffalse
%</sampledraft>
%\fi
%
% %%%%%%%%%%%%%%%%%%%%%%%%%%%%%%%%%%%%%%
% \paragraph{Forwarding for Final Version of the Chapters.}
%
% The following forwarding files |cdocsfn1.tex| and |cdocsfn2.tex|
% (with identical content)
% compile the final versions of the child documents
% |cdocsch1.tex| and |cdocsch2.tex|, respectively:
%\iffalse
%<*samplefinal>
%\fi
%    \begin{macrocode}
\def\version{final}
\input{childdoc.def}
\childdocforwardprefix[cdocsamp]{cdocsfn}{cdocsch}
%    \end{macrocode}

%\iffalse
%</samplefinal>
%\fi
%
% %%%%%%%%%%%%%%%%%%%%%%%%%%%%%%%%%%%%%%
% \paragraph{Command Line Processing.}
%
% The following three command lines generate the output files
% |cdocscld|, |cdocscl1| and |cdocscl2|
% which should be identical to
% |cdocsdrf|, |cdocsch1| and |cdocsfn2|, respectively:
% \begin{center}
% \begin{tabular}{l}
% |latex -jobname cdocscld \|\\
% |  "\def\version{draft}\input{childdoc.def}\childdocforward{cdocsamp}"|\\
% |latex -jobname cdocscl1 \|\\
% |  "\input{childdoc.def}\childdocforward[cdocsamp]{cdocsch1}"|\\
% |latex -jobname cdocscl2 \|\\
% |  "\def\version{final}\input{childdoc.def}\childdocforward{cdocsch2}"|
% \end{tabular}
% \end{center}
% Note that the trailing backslash on each first line
% merely continues the input to the second line
% (for convenient cut ant paste).
% Furthermore, the command |latex| can be replaced by any
% of its alternative versions such as |pdflatex|.
%
% %%%%%%%%%%%%%%%%%%%%%%%%%%%%%%%%%%%%%%%%%%%%%%%%%%%%%%%%%%%%%%%%%%%%%%%%%%%%%%
% %%%%%%%%%%%%%%%%%%%%%%%%%%%%%%%%%%%%%%%%%%%%%%%%%%%%%%%%%%%%%%%%%%%%%%%%%%%%%%
% \section{Implementation}
%\iffalse
%<*package>
%\fi
%
% This section describes the definitions file |childdoc.def|.

% The definitions cannot be loaded using |\usepackage| or |\RequirePackage|
% which has a mechanism to prevent loading a style file more than once.
% When loading the definitions by means of |\input|
% multiple instances have to be prevented manually:
%\iffalse
%This code needs to be before the `\ProvidesFile' directive
%which is defined at the beginning of this file.
%Therefore it is also placed there and commented out here.
%</package>
%<*discard>
%\fi
%    \begin{macrocode}
\ifdefined\childdocmain\endinput\fi
%    \end{macrocode}
%\iffalse
%</discard>
%<*package>
%\fi
%
% \macro{\ifchilddoc}
% \macro{\ifchilddocmanual}
% The conditional |\ifchilddoc| tells whether a
% child (true) or main (false) document is being compiled.
% The conditional |\ifchilddocmanual| tells whether
% the |\includeonly| mechanism is used (false) or
% the selection of child files must be performed manually (true).
% The definitions initialise to false:
%    \begin{macrocode}
\newif\ifchilddoc
\newif\ifchilddocmanual
%    \end{macrocode}

% \macro{\childdocname}
% \macro{\childdocjob}
% The macro |\childdocname| stores the name of the main document
% to be compiled. The macro |\childdocjob| stores the name of
% the document on which the \LaTeX{} compiler was originally invoked.
% The content of |\jobname| cannot be compared
% to filenames specified in the source due to different catcodes.
% The following code rescans |\jobname|, stores the result
% in |\childdocname| and saves a copy in |\childdocjob|:
%    \begin{macrocode}
\edef\childdocname{\scantokens\expandafter{\jobname\noexpand}}
\let\childdocjob\childdocname
%    \end{macrocode}

% \macro{\childdocdisable}
% The macro |\childdocdisable| prevents the main file
% from being processed more than once.
% At this stage, the main document command |\childdocmain|
% is assumed to be called once again where it should do nothing.
% Any subsequent call to it should prevent
% a secondary processing of the main document
% It overwrites the forwarding commands
% |\childdocof| and |\childdocforward|
% with empty macros to prevent further inclusions of the main document:
%    \begin{macrocode}
\newcommand{\childdocdisable}
{
  \renewcommand{\childdocmain}[1]{\renewcommand{\childdocmain}[1]{\endinput}}
  \renewcommand{\childdocof}[1]{}
  \renewcommand{\childdocby}[2][]{}
  \renewcommand{\childdocforward}[2][]{}
  \renewcommand{\childdocdisable}{}
}
%    \end{macrocode}

% \macro{\childdocmain}
% The macro |\childdocmain| is to be called at the top of the main file
% with nothing or the main filename (without extension) as argument.
% First, it breaks loops.
% If the argument is not empty and does not match |\childdocname|
% (which is set by the first inclusion of |childdoc.def|),
% |\ifchilddoc| is set to true, |\includeonly| is applied to the child file
% and |\jobname| is set to the main file
% (for proper handling of |.aux| files):
%    \begin{macrocode}
\newcommand{\childdocmain}[1]
{
  \childdocdisable\childdocmain{}
  \if?#1?\else
    \begingroup
      \def\childdoctmp{#1}
      \ifx\childdoctmp\childdocname
        \def\childdoctmp{}
      \else
        \def\childdoctmp
        {
          \childdoctrue
          \includeonly{\childdocname}
          \def\childdocjob{#1}
          \def\jobname{#1}
        }
      \fi
      \expandafter
    \endgroup
    \childdoctmp
  \fi
}
%    \end{macrocode}

% \macro{\childdocof}
% The command |\childdocof| redirects
% compilation to the main file |#1|.
%    \begin{macrocode}
\newcommand{\childdocof}[1]
{
  \childdocdisable
  \childdoctrue
  \includeonly{\childdocname}
  \def\jobname{#1}
  \def\childdocjob{#1}
  \input{#1}
}
%    \end{macrocode}

% \macro{\childdocby}
% The command |\childdocby| ....
%    \begin{macrocode}
\newcommand{\childdocby}[2][]
{
  \childdocdisable
  \childdoctrue
  \childdocmanualtrue
  \if?#1?\else
    \def\jobname{#2}
  \fi
  \def\childdocjob{#2}
  \input{#2}
  \endinput
}
%    \end{macrocode}

% \macro{\childdocforward}
% The command |\childdocforward| redirects
% compilation to the main file or
% (if the optional argument is given) a child file.
% Parameters are set as if the main file
% or a child file starting with |\childdocof| was compiled.
% Then compilation is handed over to the main file:
%    \begin{macrocode}
\newcommand{\childdocforward}[2][]
{
  \begingroup
    \if?#1?
      \def\childdoctmp
      {
        \def\childdocname{#2}
        \def\childdocjob{#2}
        \def\jobname{#2}
        \input{#2}
        \endinput
      }
    \else
      \def\childdoctmp
      {
        \childdocdisable
        \def\childdocname{#2}
        \childdoctrue
        \includeonly{#2}
        \def\childdocjob{#1}
        \def\jobname{#1}
        \input{#1}
        \endinput
      }
    \fi
    \expandafter
  \endgroup
  \childdoctmp
}
%    \end{macrocode}

% \macro{\childdocforwardprefix}
% The command |\childdocforwardprefix| redirects
% compilation to the main or a child file by means of a pattern.
% The prefix |#1| in the current filename is replaced by |#2|
% and the suffix of the current filename is kept
% (it is assumed that the filename does not contain the substring `|~~~|'
% which is used as a delimiter).
% Compilation is handed over to the new file by |\childdocforward|:
%    \begin{macrocode}
\newcommand{\childdocforwardprefix}[3][]
{
  \begingroup
    \def\childdocextract #2##1~~~{\def\childdoctmp{\childdocforward[#1]{#3##1}}}
    \expandafter\childdocextract\childdocname~~~
    \expandafter
  \endgroup
  \childdoctmp
}
%    \end{macrocode}

% \macro{\childdoc}
% The deprecated macro |\childdoc| is a legacy version of |\childdocmain|:
%    \begin{macrocode}
\newcommand{\childdoc}{\childdocmain}
%    \end{macrocode}

% \macro{\childdocredirect}
% The deprecated macro |\childdocredirect| is a legacy version
% of |\childdocforward| and |\childdocforwardprefix|:
%    \begin{macrocode}
\newcommand{\childdocredirect}[2][]
{
  \begingroup
    \if?#1?
      \def\childdoctmp{\childdocforward{#2}}
    \else
      \def\childdoctmp{\childdocforwardprefix{#1}{#2}}
    \fi
    \expandafter
  \endgroup
  \childdoctmp
}
%    \end{macrocode}

%\iffalse
%</package>
%\fi
%
\endinput

\childdocby{cdocsamp}
%    \end{macrocode}

%\iffalse
%</samplepart3|samplepart4>
%\fi
%
%\iffalse
%<*samplepart3>
%\fi
% Some text for part 3:
%    \begin{macrocode}
some text in part three
%    \end{macrocode}

%\iffalse
%</samplepart3>
%\fi
% Some text for part 4:
%\iffalse
%<*samplepart4>
%\fi
%    \begin{macrocode}
more text in part four
%    \end{macrocode}

%\iffalse
%</samplepart4>
%\fi
%
% %%%%%%%%%%%%%%%%%%%%%%%%%%%%%%%%%%%%%%
% \paragraph{Forwarding for a Complete Draft.}
%
% The following forwarding file |cdocsdrf.tex|
% compiles the main document in draft mode:
%\iffalse
%<*sampledraft>
%\fi
%    \begin{macrocode}
\def\version{draft}
% \iffalse
%
% childdoc.dtx Copyright (C) 2017-2018 Niklas Beisert
%
% This work may be distributed and/or modified under the
% conditions of the LaTeX Project Public License, either version 1.3
% of this license or (at your option) any later version.
% The latest version of this license is in
%   http://www.latex-project.org/lppl.txt
% and version 1.3 or later is part of all distributions of LaTeX
% version 2005/12/01 or later.
%
% This work has the LPPL maintenance status `maintained'.
%
% The Current Maintainer of this work is Niklas Beisert.
%
% This work consists of the files childdoc.dtx and childdoc.ins
% and the derived files childdoc.def and cdocsamp.tex with
% cdocsch1.tex, cdocsch2.tex, cdocsdrf.tex, cdocsfn1.tex, cdocsfn2.tex.
%
%<package>\ifdefined\childdocmain\endinput\fi
%<package>\ProvidesFile{childdoc.def}[2018/12/30 v2.0 child document driver]
%<samplemain>\ProvidesFile{cdocsamp.tex}[2018/12/30 v2.0 sample for childdoc]
%<*driver>
%\ProvidesFile{childdoc.drv}[2018/12/30 v2.0 childdoc reference manual file]
\PassOptionsToClass{10pt,a4paper}{article}
\documentclass{ltxdoc}

\usepackage[margin=35mm]{geometry}
\usepackage{hyperref}
\usepackage{hyperxmp}
\usepackage[usenames]{color}

\hypersetup{colorlinks=true}
\hypersetup{pdfstartview=FitH}
\hypersetup{pdfpagemode=UseNone}
\hypersetup{pdfsource={}}
\hypersetup{pdflang={en-UK}}
\hypersetup{pdfcopyright={Copyright 2017-2018 Niklas Beisert.
  This work may be distributed and/or modified under the
  conditions of the LaTeX Project Public License, either version 1.3
  of this license or (at your option) any later version.}}
\hypersetup{pdflicenseurl={http://www.latex-project.org/lppl.txt}}
\hypersetup{pdfcontactaddress={ETH Zurich, ITP, HIT K,
  Wolfgang-Pauli-Strasse 27}}
\hypersetup{pdfcontactpostcode={8093}}
\hypersetup{pdfcontactcity={Zurich}}
\hypersetup{pdfcontactcountry={Switzerland}}
\hypersetup{pdfcontactemail={nbeisert@itp.phys.ethz.ch}}
\hypersetup{pdfcontacturl={http://people.phys.ethz.ch/\xmptilde nbeisert/}}

\newcommand{\secref}[1]{\hyperref[#1]{section \ref*{#1}}}

\parskip1ex
\parindent0pt
\let\olditemize\itemize
\def\itemize{\olditemize\parskip0pt}

\begin{document}

\title{The \textsf{childdoc} Package}
\hypersetup{pdftitle={The childdoc Package}}
\author{Niklas Beisert\\[2ex]
  Institut f\"ur Theoretische Physik\\
  Eidgen\"ossische Technische Hochschule Z\"urich\\
  Wolfgang-Pauli-Strasse 27, 8093 Z\"urich, Switzerland\\[1ex]
  \href{mailto:nbeisert@itp.phys.ethz.ch}
  {\texttt{nbeisert@itp.phys.ethz.ch}}}
\hypersetup{pdfauthor={Niklas Beisert}}
\hypersetup{pdfsubject={Manual for the LaTeX2e Package childdoc}}
\date{30 December 2018, \textsf{v2.0}}
\maketitle

\begin{abstract}\noindent
\textsf{childdoc} is a \LaTeXe{} package
that enables the direct compilation
of document sections included by |\include|
to individual files.
\end{abstract}

\begingroup
\parskip0ex
\tableofcontents
\endgroup

%%%%%%%%%%%%%%%%%%%%%%%%%%%%%%%%%%%%%%%%%%%%%%%%%%%%%%%%%%%%%%%%%%%%%%%%%%%%%%%%
%%%%%%%%%%%%%%%%%%%%%%%%%%%%%%%%%%%%%%%%%%%%%%%%%%%%%%%%%%%%%%%%%%%%%%%%%%%%%%%%
\section{Introduction}

\LaTeX{} provides a mechanism to structure a large document (such as a book)
into a main file and several child files (containing the chapters)
using the |\include| command.
This mechanism is beneficial for documents
which span hundreds of pages in order to
make the source file(s) more manageable.
Moreover, compilation can be restricted to
selected child files by means of the |\includeonly| command.
The latter feature can be used to reduce the compilation time while editing
(this was significantly more useful in the earlier days of \LaTeX{})
or to generate a smaller document which is easier to navigate.
Another application of |\includeonly| is to generate
documents consisting of selected parts of the complete document.

However, there are a few drawbacks of the plain |\include| mechanism:
\begin{itemize}
\item
The child files cannot be compiled on their own,
they can only be compiled via the main file.
A naive editing environment
(such as a text editor with an option
to have the current file processed by \LaTeX)
may require one to switch to the main file before compiling;
attempting to compile the child file produces errors.
\item
The main file must be modified (each time)
to adjust the |\includeonly| command
to the present needs. This easily leaves the main file in a messy state.
\item
The generated document will always carry the filename
of the main document. This is inconvenient if
several child files are to be compiled and
to be kept for distribution.
\end{itemize}

The present package provides a simple interface
to make child files individually compilable by \LaTeX{}.
Compiling a child file then has the same effect as compiling
the main file with an |\includeonly| command
to select the appropriate child.
Moreover the generated document will carry the name of the child
rather than the main file.
This resolves all three above issues.

This feature is meant to make the editing of books,
thesis documents and lecture notes somewhat more convenient.
However, the package can also be used efficiently for
composing a series of documents (such as exercise sheets)
which are typically distributed individually.
It then assists the author in generating the individual documents
(potentially in different versions)
as well as a document containing the collected series.
Another application is in developing style files
or other kinds of included material
where compilation of the style file could redirect
to a sample or test file.

%%%%%%%%%%%%%%%%%%%%%%%%%%%%%%%%%%%%%%%%%%%%%%%%%%%%%%%%%%%%%%%%%%%%%%%%%%%%%%%%
%%%%%%%%%%%%%%%%%%%%%%%%%%%%%%%%%%%%%%%%%%%%%%%%%%%%%%%%%%%%%%%%%%%%%%%%%%%%%%%%
\section{Usage}

First of all, the package \textsf{childdoc} is \emph{not} a standard
\LaTeXe{} |.sty| style file! Therefore it needs to be invoked in
a non-standard way.

%%%%%%%%%%%%%%%%%%%%%%%%%%%%%%%%%%%%%%%%%%%%%%%%%%%%%%%%%%%%%%%%%%%%%%%%%%%%%%%%
\subsection{Included Files}
\label{sec:include}

%%%%%%%%%%%%%%%%%%%%%%%%%%%%%%%%%%%%%%%%
\DescribeMacro{\childdocmain}
To use the package, add the commands
\begin{center}
\begin{tabular}{l}
|\input{childdoc.def}|\\
|\childdocmain{}|\\
\end{tabular}
\end{center}
at the very top of the main \LaTeX{} file,
in particular \emph{before} the |\documentclass| statement!
The argument of |\childdocmain| should be left empty
(but it must be present).

%%%%%%%%%%%%%%%%%%%%%%%%%%%%%%%%%%%%%%%%
\DescribeMacro{\childdocof}
Furthermore, add the commands
\begin{center}
\begin{tabular}{l}
|\input{childdoc.def}|\\
|\childdocof{|\textit{main}|}|\\
\end{tabular}
\end{center}
at the top of every child file \textit{child}
which is included by |\include{|\textit{child}|}|
from within the main file
(or at least for those files to be compiled individually).
The argument \textit{main} must be the filename of the main file.

There are a couple of
considerations in setting up the main and child documents:

%%%%%%%%%%%%%%%%%%%%%%%%%%%%%%%%%%%%%%%%
\paragraph{Restrictions.}

Please note the following restrictions:
\begin{itemize}
\item
|\childdocmain| must be called with one argument \textit{main}
to ensure compatibility with earlier version of the package.
It must either be empty (|\childdocmain{}|)
or precisely match the filename of the main file in which it is specified.
See \secref{sec:detection} for further information.
\item
The filename \textit{main} must be specified without the |.tex| extension.
\item
The filename \textit{main} is case sensitive
(even in case-insensitive file systems)
due to internal string comparison.
\item
The argument \textit{main} should be fully expanded, it cannot be a macro.
\item
Subdirectories and special characters should be avoided in filenames.
\item
The command |\childdocmain{|\textit{main}|}| must be followed by a whitespace.
It should not be followed immediately by another command
or by a comment mark `|%|'.
This is because the \TeX{} parser reads the token immediately following
the argument of |\childdocmain| and puts it
at the beginning of every child section;
however, a white\-space is ignored.
\end{itemize}

%%%%%%%%%%%%%%%%%%%%%%%%%%%%%%%%%%%%%%%%
\paragraph{Content of Main File.}

It is advisable to place all content in the child files included by |\include|.
Any output contained in the main file will appear in all child documents
unless suppressed manually;
it cannot be suppressed automatically by the |\includeonly| directive
and thus should normally be avoided.
A method to include some content in the main file
by means of conditional processing is described in \secref{sec:conditional}.

%%%%%%%%%%%%%%%%%%%%%%%%%%%%%%%%%%%%%%%%
\paragraph{Page Numbering.}

When only a part of the document is compiled,
the appropriate numbering of pages
(as well as other status parameters)
is determined from the |.aux| files.
The latter contain information from previous passes.
However this information needs to propagate through
all intermediate child documents.
Therefore the page numbering in child documents may well
be inconsistent until the complete document is compiled at least once.

A useful (if unconventional) way to always ensure a consistent
page numbering is to restart the numbering in each child document
and denote the pages by `\textit{child}|.|\textit{page}'
where \textit{child} represents the chapter/section number of the child file.
This can be achieved by the command
|\numberwithin{page}{|\textit{child}|}|
of the \textsf{amsmath} package
where \textit{child} can be |chapter| or |section|
depending on the chosen structuring.
Alternatively, one can modify the macro |\thepage| appropriately
and reset the counter |page| at the start of each child file.

%%%%%%%%%%%%%%%%%%%%%%%%%%%%%%%%%%%%%%%%%%%%%%%%%%%%%%%%%%%%%%%%%%%%%%%%%%%%%%%%
\subsection{Conditional Processing}
\label{sec:conditional}

The package provides a mechanism to compile different versions
of a document. To customise the versions further some conditional processing
can come in handy to distinguish which version is being compiled.
The package provides two macros to describe the compilation context:

%%%%%%%%%%%%%%%%%%%%%%%%%%%%%%%%%%%%%%%%
\DescribeMacro{\ifchilddoc}
The conditional |\ifchilddoc| distinguishes between the compilation of
child documents and the main document:
%
\begin{center}
|\ifchilddoc |\textit{child-code}| |[|\||else |\textit{main-code}]| \||fi|
\end{center}

%%%%%%%%%%%%%%%%%%%%%%%%%%%%%%%%%%%%%%%%
\DescribeMacro{\childdocname}
\DescribeMacro{\childdocjob}
The macro |\childdocname| contains the filename (without extension)
of the main or child file being processed.
Note that |\childdocjob| will always contain the name of the main file.

%%%%%%%%%%%%%%%%%%%%%%%%%%%%%%%%%%%%%%%%
\paragraph{Title Page.}

Conditional processing can be used to include a title or banner page
in the main document when proper precautions are taken.
Importantly, the code in the main file should ensure that the page counter
(as well as other status parameters which are stored in the |.aux| files)
takes the same value after the conditional processing.
Otherwise the page numbers may take divergent values
depending on which part is compiled.

For example, a title page could be declared by:
%
\begin{center}
\begin{tabular}{l}
|\ifchilddoc\||else|\\
|\addtocounter{page}{-1}|\\
\textit{code for title page}\\
|\newpage|\\
|\||fi|
\end{tabular}
\end{center}
%
A banner page for the child documents can be generated by:
%
\begin{center}
\begin{tabular}{l}
|\ifchilddoc|\\
|\addtocounter{page}{-1}|\\
\textit{code for banner page}\\
|\newpage|\\
|\||fi|
\end{tabular}
\end{center}
%
Here one could write a message such as:
\begin{center}
|This is the part \childdocname{} of \childdocjob{}.|
\end{center}

%%%%%%%%%%%%%%%%%%%%%%%%%%%%%%%%%%%%%%%%%%%%%%%%%%%%%%%%%%%%%%%%%%%%%%%%%%%%%%%%
\subsection{Flags}
\label{sec:flags}

The package makes it easy to generate different versions
of the main or child documents.
To this end compilation flags can be defined
and assigned different default values.
They will be particularly useful in conjunction
with the forwarding mechanism described in \secref{sec:forward}.

For example, it may be useful to have a flag |\version|
which can be set to |draft| or |final|.
The document source will contain some conditional code
depending on the value of |\version|.
Suppose further, the flag should default to |final| for the main file
and to |draft| for child files
which is a natural assignment for editing the document.
This is achieved by placing the following code
in the preamble of the main document
(below the |\childdocmain| directive):
%
\begin{center}
\begin{tabular}{l}
|\ifchilddoc|\\
|\providecommand{\version}{draft}|\\
|\||else|\\
|\providecommand{\version}{final}|\\
|\||fi|
\end{tabular}
\end{center}
%
The definition by |\providecommand| makes sure
that previous definitions are not overwritten.
Further statements |\providecommand{\version}{...}|
can thus be added before the above code to override it.

For the main file, one might add a line
(between |\childdocmain| and the above block)
%
\begin{center}
|%\ifchilddoc\||else\providecommand{\version}{draft}\||fi|
\end{center}
%
which can be uncommented to produce a draft version.
Likewise one can add a line to the very top of a child file
(above the |\childdocof{|\textit{main}|}| directive)
%
\begin{center}
|%\providecommand{\version}{final}|
\end{center}
%
which can be uncommented to produce the final version of this child document.

%%%%%%%%%%%%%%%%%%%%%%%%%%%%%%%%%%%%%%%%%%%%%%%%%%%%%%%%%%%%%%%%%%%%%%%%%%%%%%%%
\subsection{Forwarding}
\label{sec:forward}

Different versions of the main or child documents
using compilation flags as described in \secref{sec:flags}
can be (permanently) stored in different files
for convenient compilation, viewing and distribution.
To this end, the package defines a command
to pass on compilation to a different file:

%%%%%%%%%%%%%%%%%%%%%%%%%%%%%%%%%%%%%%%%
\DescribeMacro{\childdocforward}
The command |\childdocforward| redirects processing to
another source file:
%
\begin{center}
\begin{tabular}{l}
|\input{childdoc.def}|\\
|\childdocforward[|\textit{main}|]{|\textit{dest}|}|\\
\end{tabular}
\end{center}
%
The argument \textit{dest} is the destination file
(without extension).
It should be the main file or one of the child files.
Note that further \textsf{childdoc} directives
such as |\childdocof| and |\childdocforward|
in the indicated file will be processed in this form.
The optional argument \textit{main}
passes on directly to the main file \textit{main}
while pretending to compile the child \textit{dest}.
This form behaves as if \textit{dest}
issues |\childdocof{|\textit{main}|}| right away,
and no further \textsf{childdoc} directives will be processed.

%%%%%%%%%%%%%%%%%%%%%%%%%%%%%%%%%%%%%%%%
\DescribeMacro{\...prefix}
In the alternative form |\childdocforwardprefix|,
%
\begin{center}
\begin{tabular}{l}
|\input{childdoc.def}|\\
|\childdocforwardprefix[|\textit{main}|]{|\textit{prefix}|}{|\textit{dest}|}|
\end{tabular}
\end{center}
%
the destination file is determined by a pattern
depending on the current file:
To make this work, the current file must be called
`{\textit{prefix}\hspace{0.2em}\textit{suffix}}'
with \textit{prefix} matching precisely the argument.
Processing is then passed on to the file
`{\textit{dest}\hspace{0.2em}\textit{suffix}}'.
Surely, the same effect is achieved by
directly specifying the
argument `{\textit{dest}\hspace{0.2em}\textit{suffix}}'
in the first form.
However, that requires to set up a different file
for each child. With the alternative form of the command
all these files can have exactly the same content
which simplifies setting them up and maintaining them.

For example, the following file |draft.tex|
with a compilation flag |\version| as described in \secref{sec:flags}
compiles the main document as a draft:
%
\begin{center}
\begin{tabular}{l}
|\def\version{draft}|\\
|\input{childdoc.def}|\\
|\childdocforward{|\textit{main}|}|
\end{tabular}
\end{center}
%
Likewise, the following files |final|\textit{nn}|.tex|
compile the final version of the child document
|child|\textit{nn}|.tex|:
%
\begin{center}
\begin{tabular}{l}
|\def\version{final}|\\
|\input{childdoc.def}|\\
|\childdocforwardprefix{final}{child}|
\end{tabular}
\end{center}
%

Note that when several versions of a main file and/or of each child file
are to be generated, it may be convenient to set up a |Makefile| or
shell script to automatise the process.

%%%%%%%%%%%%%%%%%%%%%%%%%%%%%%%%%%%%%%%%%%%%%%%%%%%%%%%%%%%%%%%%%%%%%%%%%%%%%%%%
\subsection{Command Line Processing}
\label{sec:commandline}

The effect of redirection files can also be achieved by invoking
the \LaTeX{} compiler with a more elaborate command line.
Most conveniently this should be done as part
of a shell script or a |Makefile|.

When using \textsf{childdoc} in the main file, the following
command lines effectively perform a redirection
(note that depending on the shell being used,
backslashes may have to be doubled: `|\|' $\to$ `|\\|'):
%
\begin{center}
|... -jobname "|\textit{target}|" |\\|"|[\textit{flags}]%
|\input{childdoc.def}\childdocforward[|\textit{main}|]{|\textit{dest}|}"|
\end{center}
%
Here \textit{target} is the name of the output file,
\textit{main} is the name of the main file
and \textit{dest} is the name of the main or child file to be processed
(all filenames without extensions).
The optional argument \textit{main} can be omitted
if \textit{main} matches \textit{dest}.
Optionally, compilation \textit{flags} can be defined via |\def| commands.
This command line makes the \TeX{} engine believe
it is compiling the file \textit{target}
whose content is specified as the latter parameter.
The provided code then forwards the processing to
\textit{main} or \textit{dest} as described in \secref{sec:forward}.

%%%%%%%%%%%%%%%%%%%%%%%%%%%%%%%%%%%%%%%%%%%%%%%%%%%%%%%%%%%%%%%%%%%%%%%%%%%%%%%%
\subsection{Include by Input}
\label{sec:input}

Including child documents by |\include| has some restrictions by design.
Most notably, the content of a child document always occupies
its own set of pages; pages cannot be shared between child documents.
Usually, this behaviour makes perfect sense
because each child document contain an essential part of the document.
However, in some situations it may be desirable to compose
a document from a collection of parts
without having mandatory page breaks between then.
For this case, the package
provides a mechanism to include parts
by |\input| which can also be processed individually.
However, by construction this mechanism
requires manual handling of the content to be output.

%%%%%%%%%%%%%%%%%%%%%%%%%%%%%%%%%%%%%%%%
\DescribeMacro{\ifchilddocmanual}
The main file should be prepared as usual, see \secref{sec:include}.
However, the document body must make a distinction
between processing of an individual part and of the main document, e.g.:
%
\begin{center}
\begin{tabular}{l}
|\ifchilddocmanual|\\
|\input{\childdocname}|\\
|\||else|\\
\textit{document body with }|\input{|\textit{part}|}|\\
|\||fi|
\end{tabular}
\end{center}
%
The conditional |\ifchilddocmanual| is true whenever
a part to be included by |\input| is being compiled,
and the name of the part is stored in |\childdocname|.

%%%%%%%%%%%%%%%%%%%%%%%%%%%%%%%%%%%%%%%%
\DescribeMacro{\childdocby}
Each part to be included by |\input| should start with:
%
\begin{center}
\begin{tabular}{l}
|\input{childdoc.def}|\\
|\childdocby{|\textit{main}|}|\\
\end{tabular}
\end{center}
%
The directive |\childdocby| is similar to |\childdocof|
described in \secref{sec:include},
but the subsequent selection of content must be done manually.
To that end, both |\ifchilddoc| and |\ifchilddocmanual|
will be true upon processing of a part,
and the name of the part is stored in |\childdocname|.
Note that |\jobname| will be set to the filename of the current part
so that each part receives an individual |.aux| file
that does not interfere with the |.aux| file(s) of the main document.
This behaviour can be altered by the alternative form
|\childdocby[*]{|\textit{main}|}| (with a non-empty optional argument)
which uses the |.aux| file of the main document
by setting |\jobname| to \textit{main}.

%%%%%%%%%%%%%%%%%%%%%%%%%%%%%%%%%%%%%%%%%%%%%%%%%%%%%%%%%%%%%%%%%%%%%%%%%%%%%%%%
\subsection{Driver Development}
\label{sec:driver}

The \textsf{childdoc} mechanism can also be use for the development
of definition files such as \LaTeX{} styles or classes.
This case differs from the above setup with multiple parts
included by |\include| in that no |\includeonly| should be invoked.
This can be achieved by starting the include file
(before |\ProvidesPackage|) with:
%
\begin{center}
\begin{tabular}{l}
|\input{childdoc.def}|\\
|\childdocforward{|\textit{main}|}|\\
\end{tabular}
\end{center}
%
or alternatively with:
%
\begin{center}
\begin{tabular}{l}
|\input{childdoc.def}|\\
|\childdocby{|\textit{main}|}|\\
\end{tabular}
\end{center}
%
Both forms have slightly different effects as described above.
The main file is prepared as usual, see \secref{sec:include}.

%%%%%%%%%%%%%%%%%%%%%%%%%%%%%%%%%%%%%%%%%%%%%%%%%%%%%%%%%%%%%%%%%%%%%%%%%%%%%%%%
\subsection{Legacy Detection}
\label{sec:detection}

The directive |\childdocmain| in the main file can detect
whether the complete document or merely a child is to be compiled
even without using the directive |\childdocof|.
This method is deprecated because it is less robust
and there is no compelling reason to use it;
it is merely provided for backward compatibility
and it may be removed in future versions.

If the detection mechanism is to be used,
it is mandatory to correctly specify
the filename of the main file as the argument of |\childdocmain|:
%
\begin{center}
\begin{tabular}{l}
|\input{childdoc.def}|\\
|\childdocmain{|\textit{main}|}|\\
\end{tabular}
\end{center}
%
If |\jobname| does not match the argument \textit{main} of |\childdocmain|,
it is assumed that |\jobname| points to the child file to be compiled.
When using |\childdocmain| with the main file specified as argument,
it suffices to start a child file
with just |\input{|\textit{main}|}|
without loading of the package and using |\childdocof|.
If instead all processing is done
with the appropriate \textsf{childdoc} directives,
the argument of \textit{main} of |\childdocmain| can be empty.

An alternative version of the command line processing described
in \secref{sec:commandline} using the detection mechanism reads:
%
\begin{center}
|... -jobname "|\textit{target}|" "|[\textit{flags}]%
[|\def\jobname{|\textit{dest}|}|]|\input{|\textit{main}|}"|
\end{center}

%%%%%%%%%%%%%%%%%%%%%%%%%%%%%%%%%%%%%%%%%%%%%%%%%%%%%%%%%%%%%%%%%%%%%%%%%%%%%%%%
\subsection{Manual Code}
\label{sec:manual}

In case one cannot be certain whether the definitions file |childdoc.def|
is installed on the target \TeX{} distribution
and one prefers not to ship it,
it is conceivable to paste a few relevant commands into the sources.

To that end, drop all statements |\input{childdoc.def}|
and perform the replacements as outlined below.
Instead of |\childdocmain{|\textit{main}|}| add the following code
to the top of the main file:
%
\begin{center}
\begin{tabular}{l}
|\||ifdefined\childdocname\endinput\||fi\newif\ifchilddoc|\\
|\edef\childdocname{\scantokens\expandafter{\jobname\noexpand}}|\\
|\def\childdocmain{|\textit{main}|}\||ifx\childdocmain\childdocname\||else|\\
|\childdoctrue\includeonly{\childdocname}\let\jobname\childdocmain\||fi|\\
\end{tabular}
\end{center}
%
Instead of |\childdocof{|\textit{main}|}| just include the main file
at the top of each child file:
%
\begin{center}
|\input{|\textit{main}|}|
\end{center}
%
A simple redirection |\childdocforward{|\textit{dest}|}| is achieved by:
%
\begin{center}
|\def\jobname{|\textit{dest}|}\input{\jobname}|
\end{center}
%
The redirection with prefix
|\childdocforwardprefix[|\textit{prefix}|]{|\textit{dest}|}|
is accomplished by:
%
\begin{center}
\begin{tabular}{l}
|{\edef\jobname{\scantokens\expandafter{\jobname\noexpand}}|\\
|\def\redirectjob |\textit{prefix}|#1~~~{\gdef\jobname{|\textit{dest}|#1}}|\\
|\expandafter\redirectjob\jobname~~~}\input{\jobname}|
\end{tabular}
\end{center}

In an alternative approach,
child documents can be compiled by a specific command line
without additional code or specific definitions:
%
\begin{center}
|... -jobname "|\textit{target}|" "|[\textit{flags}]%
|\includeonly{|\textit{dest}|}\input{|\textit{main}|}"|
\end{center}
%

%%%%%%%%%%%%%%%%%%%%%%%%%%%%%%%%%%%%%%%%%%%%%%%%%%%%%%%%%%%%%%%%%%%%%%%%%%%%%%%%
%%%%%%%%%%%%%%%%%%%%%%%%%%%%%%%%%%%%%%%%%%%%%%%%%%%%%%%%%%%%%%%%%%%%%%%%%%%%%%%%
\section{Information}

%%%%%%%%%%%%%%%%%%%%%%%%%%%%%%%%%%%%%%%%%%%%%%%%%%%%%%%%%%%%%%%%%%%%%%%%%%%%%%%%
\subsection{Copyright}

Copyright \copyright{} 2017--2018 Niklas Beisert

This work may be distributed and/or modified under the
conditions of the \LaTeX{} Project Public License, either version 1.3
of this license or (at your option) any later version.
The latest version of this license is in
  \url{http://www.latex-project.org/lppl.txt}
and version 1.3 or later is part of all distributions of \LaTeX{}
version 2005/12/01 or later.

This work has the LPPL maintenance status `maintained'.

The Current Maintainer of this work is Niklas Beisert.

This work consists of the files |README.txt|, |childdoc.ins| and |childdoc.dtx|
as well as the derived files |childdoc.def|, |cdocsamp.tex|
with |cdocsch1.tex|, |cdocsch2.tex|, |cdocspt3.tex|, |cdocspt4.tex|,
|cdocsdrf.tex|, |cdocsfn1.tex|, |cdocsfn2.tex|
as well as |childdoc.pdf|.

%%%%%%%%%%%%%%%%%%%%%%%%%%%%%%%%%%%%%%%%%%%%%%%%%%%%%%%%%%%%%%%%%%%%%%%%%%%%%%%%
\subsection{Files and Installation}

The package consists of the files:
%
\begin{center}
\begin{tabular}{ll}
    |README.txt|   & readme file \\
    |childdoc.ins| & installation file \\
    |childdoc.dtx| & source file \\
    |childdoc.def| & definition file \\
    |cdocsamp.tex| & sample main file \\
    |cdocsch1.tex| & sample include file \\
    |cdocsch2.tex| & sample include file \\
    |cdocspt3.tex| & sample part file \\
    |cdocspt4.tex| & sample part file \\
    |cdocsdrf.tex| & sample redirection file \\
    |cdocsfn1.tex| & sample redirection file \\
    |cdocsfn2.tex| & sample redirection file \\
    |childdoc.pdf| & manual
\end{tabular}
\end{center}
%
The distribution consists of the files
|README.txt|, |childdoc.ins| and |childdoc.dtx|.
%
\begin{itemize}
\item
Run (pdf)\LaTeX{} on |childdoc.dtx|
to compile the manual |childdoc.pdf| (this file).
\item
Run \LaTeX{} on |childdoc.ins| to create the definitions file |childdoc.def|
and the sample |cdocsamp.tex| with include files
|cdocsch1.tex|, |cdocsch2.tex|, |cdocspt3.tex|, |cdocspt4.tex|,
|cdocsdrf.tex|, |cdocsfn1.tex|, |cdocsfn2.tex|.
Then copy the file |childdoc.def| to an appropriate directory of your \LaTeX{}
distribution, e.g.\ \textit{texmf-root}|/tex/latex/childdoc|.
\end{itemize}

%%%%%%%%%%%%%%%%%%%%%%%%%%%%%%%%%%%%%%%%%%%%%%%%%%%%%%%%%%%%%%%%%%%%%%%%%%%%%%%%
\subsection{Related CTAN Packages}

There are several other packages which offer a similar functionality:
%
\begin{itemize}
\item
The packages
\href{http://ctan.org/pkg/docmute}{\textsf{docmute}},
\href{http://ctan.org/pkg/includex}{\textsf{includex}} and
\href{http://ctan.org/pkg/standalone}{\textsf{standalone}}
provide commands to include only the document body of
a child file thus allowing both files to be compiled individually.
\item
The packages \href{http://ctan.org/pkg/subdocs}{\textsf{subdocs}}
and \href{http://ctan.org/pkg/subfiles}{\textsf{subfiles}}
provide structures in which the main and child documents can be
encapsulated and allowing them to be compiled individually.
The inclusion mechanism is different from the conventional |\include|.
\item
The package \href{http://ctan.org/pkg/combine}{\textsf{combine}}
is an elaborate solution to combine several documents into one.
\end{itemize}
%
See also the CTAN topic \href{http://ctan.org/topic/subdocs}{\textsf{subdocs}}
for further related packages.
The present package differs from the above solutions in that
a document structure constructed with the conventional |\include| mechanism
just needs two extra commands at the top of every file
such that all constituent files can be compiled individually.

%%%%%%%%%%%%%%%%%%%%%%%%%%%%%%%%%%%%%%%%%%%%%%%%%%%%%%%%%%%%%%%%%%%%%%%%%%%%%%%%
%\subsection{Feature Suggestions}
%
%The following is a list of features which may be useful for future
%versions of this package:
%%
%\begin{itemize}
%\item
%\ldots
%\end{itemize}

%%%%%%%%%%%%%%%%%%%%%%%%%%%%%%%%%%%%%%%%%%%%%%%%%%%%%%%%%%%%%%%%%%%%%%%%%%%%%%%%
\subsection{Revision History}

%%%%%%%%%%%%%%%%%%%%%%%%%%%%%%%%%%%%%%%%
\paragraph{v2.0:} 2018/12/30

\begin{itemize}
\item
immediate forward processing
\item
added |\childdocby| mechanism
\item
manual restructured
\end{itemize}

%%%%%%%%%%%%%%%%%%%%%%%%%%%%%%%%%%%%%%%%
\paragraph{v1.6:} 2018/01/17

\begin{itemize}
\item
application for development of include files
\item
corrections to manual
\end{itemize}

%%%%%%%%%%%%%%%%%%%%%%%%%%%%%%%%%%%%%%%%
\paragraph{v1.5:} 2017/05/21

\begin{itemize}
\item
more complete structuring introduced
\item
|\childdocof| introduced
\item
|\childdoc| renamed to |\childdocmain|
\item
|\childredirect| renamed to |\childdocforward| and |\childdocforwardprefix|
and functionality expanded
\end{itemize}

%%%%%%%%%%%%%%%%%%%%%%%%%%%%%%%%%%%%%%%%
\paragraph{v1.0:} 2017/04/27

\begin{itemize}
\item
manual and install package
\item
first version published on CTAN
\end{itemize}

%%%%%%%%%%%%%%%%%%%%%%%%%%%%%%%%%%%%%%%%
\paragraph{v0.6:} 2017/04/26

\begin{itemize}
\item
redirection mechanism added
\end{itemize}

%%%%%%%%%%%%%%%%%%%%%%%%%%%%%%%%%%%%%%%%
\paragraph{v0.5:} 2017/04/26

\begin{itemize}
\item
functionality in definition file
\end{itemize}


%%%%%%%%%%%%%%%%%%%%%%%%%%%%%%%%%%%%%%%%%%%%%%%%%%%%%%%%%%%%%%%%%%%%%%%%%%%%%%%%
%%%%%%%%%%%%%%%%%%%%%%%%%%%%%%%%%%%%%%%%%%%%%%%%%%%%%%%%%%%%%%%%%%%%%%%%%%%%%%%%
%%%%%%%%%%%%%%%%%%%%%%%%%%%%%%%%%%%%%%%%%%%%%%%%%%%%%%%%%%%%%%%%%%%%%%%%%%%%%%%%
\appendix

\settowidth\MacroIndent{\rmfamily\scriptsize 000\ }

 \DocInput{childdoc.dtx}

\end{document}
%</driver>
% \fi
%
% %%%%%%%%%%%%%%%%%%%%%%%%%%%%%%%%%%%%%%%%%%%%%%%%%%%%%%%%%%%%%%%%%%%%%%%%%%%%%%
% %%%%%%%%%%%%%%%%%%%%%%%%%%%%%%%%%%%%%%%%%%%%%%%%%%%%%%%%%%%%%%%%%%%%%%%%%%%%%%
% \section{Sample}
%\iffalse
%<*samplemain>
%\fi
%
% The following presents a sample document
% with two chapters, two parts, a title page,
% a compile flag as well as three forwarding files to set the flag.
% It consists of eight |.tex| files:
% \begin{center}
% \begin{tabular}{ll}
% |cdocsamp.tex|&main file\\
% |cdocsch1.tex|&include file for chapter 1\\
% |cdocsch2.tex|&include file for chapter 2\\
% |cdocspt3.tex|&include file for part 3\\
% |cdocspt4.tex|&include file for part 4\\
% |cdocsdrf.tex|&forwarding file for main file in draft mode\\
% |cdocsfi1.tex|&forwarding file for final version of chapter 1\\
% |cdocsfi2.tex|&forwarding file for final version of chapter 2\\
% \end{tabular}
% \end{center}
% Each of the eight files can be compiled directly by the \LaTeX{} compiler.
%
% %%%%%%%%%%%%%%%%%%%%%%%%%%%%%%%%%%%%%%
% \paragraph{Main File.}
%
% The main file is called |cdocsamp.tex|.
%
% Load the \textsf{childdoc} definitions and
% declare the filename for the main document:
%    \begin{macrocode}
\input{childdoc.def}
\childdocmain{}
%    \end{macrocode}

% Optional override for |\version| flag:
%    \begin{macrocode}
%%\ifchilddoc\else\providecommand{\version}{draft}\fi
%    \end{macrocode}

% Define the default values for the |\version| flag
% (|final| for the main file and |draft| for childs):
%    \begin{macrocode}
\ifchilddoc
\providecommand{\version}{draft}
\else
\providecommand{\version}{final}
\fi
%    \end{macrocode}

% Load the standard document class:
%    \begin{macrocode}
\documentclass[12pt]{article}
%    \end{macrocode}

% Start the document body:
%    \begin{macrocode}
\begin{document}
%    \end{macrocode}

% Declare a title page.
% Print title, part of document being processed and version flag:
%    \begin{macrocode}
\addtocounter{page}{-1}
\begin{center}
{\LARGE\bfseries{}childdoc example\par}
\vspace{1cm}
\ifchilddoc
\ifchilddocmanual part\else chapter\fi:
`\childdocname' of `\childdocjob'\par
\else
main document: `\childdocjob'\par
\fi
version: \version\par
\end{center}
\newpage
%    \end{macrocode}

% Manually include selected file,
% otherwise process as usual:
%    \begin{macrocode}
\ifchilddocmanual
\section*{part `\childdocname'}
\input{\childdocname}
\else
%    \end{macrocode}

% Include the two chapters:
%    \begin{macrocode}
\include{cdocsch1}
\include{cdocsch2}
%    \end{macrocode}

% Include the two parts unless only chapters should be displayed:
%    \begin{macrocode}
\ifchilddoc\else
\section{part three}
\input{cdocspt3}
\section{part four}
\input{cdocspt4}
\fi
%    \end{macrocode}

% Process as usual until here:
%    \begin{macrocode}
\fi
%    \end{macrocode}

% End of document body:
%    \begin{macrocode}
\end{document}
%    \end{macrocode}
%\iffalse
%</samplemain>
%\fi
%
% %%%%%%%%%%%%%%%%%%%%%%%%%%%%%%%%%%%%%%
% \paragraph{Chapter Include Files.}
%
% The include files are called |cdocsch1.tex| and |cdocsch2.tex|.
%
%\iffalse
%<*samplechap1|samplechap2>
%\fi

% Optional override for |\version| flag:
%    \begin{macrocode}
%%\providecommand{\version}{final}
%    \end{macrocode}

% Include the main document:
%    \begin{macrocode}
\input{childdoc.def}
\childdocof{cdocsamp}
%    \end{macrocode}

%\iffalse
%</samplechap1|samplechap2>
%\fi
%
%\iffalse
%<*samplechap1>
%\fi
% Some text for chapter 1:
%    \begin{macrocode}
\section{one}
some text in chapter one
%    \end{macrocode}

%\iffalse
%</samplechap1>
%\fi
% Some text for chapter 2:
%\iffalse
%<*samplechap2>
%\fi
%    \begin{macrocode}
\section{two}
more text in chapter two
%    \end{macrocode}

%\iffalse
%</samplechap2>
%\fi
%
% %%%%%%%%%%%%%%%%%%%%%%%%%%%%%%%%%%%%%%
% \paragraph{Part Include Files.}
%
% The include files are called |cdocspt3.tex| and |cdocspt4.tex|.
%
%\iffalse
%<*samplepart3|samplepart4>
%\fi

% Optional override for |\version| flag:
%    \begin{macrocode}
%%\providecommand{\version}{final}
%    \end{macrocode}

% Include the main document:
%    \begin{macrocode}
\input{childdoc.def}
\childdocby{cdocsamp}
%    \end{macrocode}

%\iffalse
%</samplepart3|samplepart4>
%\fi
%
%\iffalse
%<*samplepart3>
%\fi
% Some text for part 3:
%    \begin{macrocode}
some text in part three
%    \end{macrocode}

%\iffalse
%</samplepart3>
%\fi
% Some text for part 4:
%\iffalse
%<*samplepart4>
%\fi
%    \begin{macrocode}
more text in part four
%    \end{macrocode}

%\iffalse
%</samplepart4>
%\fi
%
% %%%%%%%%%%%%%%%%%%%%%%%%%%%%%%%%%%%%%%
% \paragraph{Forwarding for a Complete Draft.}
%
% The following forwarding file |cdocsdrf.tex|
% compiles the main document in draft mode:
%\iffalse
%<*sampledraft>
%\fi
%    \begin{macrocode}
\def\version{draft}
\input{childdoc.def}
\childdocforward{cdocsamp}
%    \end{macrocode}

%\iffalse
%</sampledraft>
%\fi
%
% %%%%%%%%%%%%%%%%%%%%%%%%%%%%%%%%%%%%%%
% \paragraph{Forwarding for Final Version of the Chapters.}
%
% The following forwarding files |cdocsfn1.tex| and |cdocsfn2.tex|
% (with identical content)
% compile the final versions of the child documents
% |cdocsch1.tex| and |cdocsch2.tex|, respectively:
%\iffalse
%<*samplefinal>
%\fi
%    \begin{macrocode}
\def\version{final}
\input{childdoc.def}
\childdocforwardprefix[cdocsamp]{cdocsfn}{cdocsch}
%    \end{macrocode}

%\iffalse
%</samplefinal>
%\fi
%
% %%%%%%%%%%%%%%%%%%%%%%%%%%%%%%%%%%%%%%
% \paragraph{Command Line Processing.}
%
% The following three command lines generate the output files
% |cdocscld|, |cdocscl1| and |cdocscl2|
% which should be identical to
% |cdocsdrf|, |cdocsch1| and |cdocsfn2|, respectively:
% \begin{center}
% \begin{tabular}{l}
% |latex -jobname cdocscld \|\\
% |  "\def\version{draft}\input{childdoc.def}\childdocforward{cdocsamp}"|\\
% |latex -jobname cdocscl1 \|\\
% |  "\input{childdoc.def}\childdocforward[cdocsamp]{cdocsch1}"|\\
% |latex -jobname cdocscl2 \|\\
% |  "\def\version{final}\input{childdoc.def}\childdocforward{cdocsch2}"|
% \end{tabular}
% \end{center}
% Note that the trailing backslash on each first line
% merely continues the input to the second line
% (for convenient cut ant paste).
% Furthermore, the command |latex| can be replaced by any
% of its alternative versions such as |pdflatex|.
%
% %%%%%%%%%%%%%%%%%%%%%%%%%%%%%%%%%%%%%%%%%%%%%%%%%%%%%%%%%%%%%%%%%%%%%%%%%%%%%%
% %%%%%%%%%%%%%%%%%%%%%%%%%%%%%%%%%%%%%%%%%%%%%%%%%%%%%%%%%%%%%%%%%%%%%%%%%%%%%%
% \section{Implementation}
%\iffalse
%<*package>
%\fi
%
% This section describes the definitions file |childdoc.def|.

% The definitions cannot be loaded using |\usepackage| or |\RequirePackage|
% which has a mechanism to prevent loading a style file more than once.
% When loading the definitions by means of |\input|
% multiple instances have to be prevented manually:
%\iffalse
%This code needs to be before the `\ProvidesFile' directive
%which is defined at the beginning of this file.
%Therefore it is also placed there and commented out here.
%</package>
%<*discard>
%\fi
%    \begin{macrocode}
\ifdefined\childdocmain\endinput\fi
%    \end{macrocode}
%\iffalse
%</discard>
%<*package>
%\fi
%
% \macro{\ifchilddoc}
% \macro{\ifchilddocmanual}
% The conditional |\ifchilddoc| tells whether a
% child (true) or main (false) document is being compiled.
% The conditional |\ifchilddocmanual| tells whether
% the |\includeonly| mechanism is used (false) or
% the selection of child files must be performed manually (true).
% The definitions initialise to false:
%    \begin{macrocode}
\newif\ifchilddoc
\newif\ifchilddocmanual
%    \end{macrocode}

% \macro{\childdocname}
% \macro{\childdocjob}
% The macro |\childdocname| stores the name of the main document
% to be compiled. The macro |\childdocjob| stores the name of
% the document on which the \LaTeX{} compiler was originally invoked.
% The content of |\jobname| cannot be compared
% to filenames specified in the source due to different catcodes.
% The following code rescans |\jobname|, stores the result
% in |\childdocname| and saves a copy in |\childdocjob|:
%    \begin{macrocode}
\edef\childdocname{\scantokens\expandafter{\jobname\noexpand}}
\let\childdocjob\childdocname
%    \end{macrocode}

% \macro{\childdocdisable}
% The macro |\childdocdisable| prevents the main file
% from being processed more than once.
% At this stage, the main document command |\childdocmain|
% is assumed to be called once again where it should do nothing.
% Any subsequent call to it should prevent
% a secondary processing of the main document
% It overwrites the forwarding commands
% |\childdocof| and |\childdocforward|
% with empty macros to prevent further inclusions of the main document:
%    \begin{macrocode}
\newcommand{\childdocdisable}
{
  \renewcommand{\childdocmain}[1]{\renewcommand{\childdocmain}[1]{\endinput}}
  \renewcommand{\childdocof}[1]{}
  \renewcommand{\childdocby}[2][]{}
  \renewcommand{\childdocforward}[2][]{}
  \renewcommand{\childdocdisable}{}
}
%    \end{macrocode}

% \macro{\childdocmain}
% The macro |\childdocmain| is to be called at the top of the main file
% with nothing or the main filename (without extension) as argument.
% First, it breaks loops.
% If the argument is not empty and does not match |\childdocname|
% (which is set by the first inclusion of |childdoc.def|),
% |\ifchilddoc| is set to true, |\includeonly| is applied to the child file
% and |\jobname| is set to the main file
% (for proper handling of |.aux| files):
%    \begin{macrocode}
\newcommand{\childdocmain}[1]
{
  \childdocdisable\childdocmain{}
  \if?#1?\else
    \begingroup
      \def\childdoctmp{#1}
      \ifx\childdoctmp\childdocname
        \def\childdoctmp{}
      \else
        \def\childdoctmp
        {
          \childdoctrue
          \includeonly{\childdocname}
          \def\childdocjob{#1}
          \def\jobname{#1}
        }
      \fi
      \expandafter
    \endgroup
    \childdoctmp
  \fi
}
%    \end{macrocode}

% \macro{\childdocof}
% The command |\childdocof| redirects
% compilation to the main file |#1|.
%    \begin{macrocode}
\newcommand{\childdocof}[1]
{
  \childdocdisable
  \childdoctrue
  \includeonly{\childdocname}
  \def\jobname{#1}
  \def\childdocjob{#1}
  \input{#1}
}
%    \end{macrocode}

% \macro{\childdocby}
% The command |\childdocby| ....
%    \begin{macrocode}
\newcommand{\childdocby}[2][]
{
  \childdocdisable
  \childdoctrue
  \childdocmanualtrue
  \if?#1?\else
    \def\jobname{#2}
  \fi
  \def\childdocjob{#2}
  \input{#2}
  \endinput
}
%    \end{macrocode}

% \macro{\childdocforward}
% The command |\childdocforward| redirects
% compilation to the main file or
% (if the optional argument is given) a child file.
% Parameters are set as if the main file
% or a child file starting with |\childdocof| was compiled.
% Then compilation is handed over to the main file:
%    \begin{macrocode}
\newcommand{\childdocforward}[2][]
{
  \begingroup
    \if?#1?
      \def\childdoctmp
      {
        \def\childdocname{#2}
        \def\childdocjob{#2}
        \def\jobname{#2}
        \input{#2}
        \endinput
      }
    \else
      \def\childdoctmp
      {
        \childdocdisable
        \def\childdocname{#2}
        \childdoctrue
        \includeonly{#2}
        \def\childdocjob{#1}
        \def\jobname{#1}
        \input{#1}
        \endinput
      }
    \fi
    \expandafter
  \endgroup
  \childdoctmp
}
%    \end{macrocode}

% \macro{\childdocforwardprefix}
% The command |\childdocforwardprefix| redirects
% compilation to the main or a child file by means of a pattern.
% The prefix |#1| in the current filename is replaced by |#2|
% and the suffix of the current filename is kept
% (it is assumed that the filename does not contain the substring `|~~~|'
% which is used as a delimiter).
% Compilation is handed over to the new file by |\childdocforward|:
%    \begin{macrocode}
\newcommand{\childdocforwardprefix}[3][]
{
  \begingroup
    \def\childdocextract #2##1~~~{\def\childdoctmp{\childdocforward[#1]{#3##1}}}
    \expandafter\childdocextract\childdocname~~~
    \expandafter
  \endgroup
  \childdoctmp
}
%    \end{macrocode}

% \macro{\childdoc}
% The deprecated macro |\childdoc| is a legacy version of |\childdocmain|:
%    \begin{macrocode}
\newcommand{\childdoc}{\childdocmain}
%    \end{macrocode}

% \macro{\childdocredirect}
% The deprecated macro |\childdocredirect| is a legacy version
% of |\childdocforward| and |\childdocforwardprefix|:
%    \begin{macrocode}
\newcommand{\childdocredirect}[2][]
{
  \begingroup
    \if?#1?
      \def\childdoctmp{\childdocforward{#2}}
    \else
      \def\childdoctmp{\childdocforwardprefix{#1}{#2}}
    \fi
    \expandafter
  \endgroup
  \childdoctmp
}
%    \end{macrocode}

%\iffalse
%</package>
%\fi
%
\endinput

\childdocforward{cdocsamp}
%    \end{macrocode}

%\iffalse
%</sampledraft>
%\fi
%
% %%%%%%%%%%%%%%%%%%%%%%%%%%%%%%%%%%%%%%
% \paragraph{Forwarding for Final Version of the Chapters.}
%
% The following forwarding files |cdocsfn1.tex| and |cdocsfn2.tex|
% (with identical content)
% compile the final versions of the child documents
% |cdocsch1.tex| and |cdocsch2.tex|, respectively:
%\iffalse
%<*samplefinal>
%\fi
%    \begin{macrocode}
\def\version{final}
% \iffalse
%
% childdoc.dtx Copyright (C) 2017-2018 Niklas Beisert
%
% This work may be distributed and/or modified under the
% conditions of the LaTeX Project Public License, either version 1.3
% of this license or (at your option) any later version.
% The latest version of this license is in
%   http://www.latex-project.org/lppl.txt
% and version 1.3 or later is part of all distributions of LaTeX
% version 2005/12/01 or later.
%
% This work has the LPPL maintenance status `maintained'.
%
% The Current Maintainer of this work is Niklas Beisert.
%
% This work consists of the files childdoc.dtx and childdoc.ins
% and the derived files childdoc.def and cdocsamp.tex with
% cdocsch1.tex, cdocsch2.tex, cdocsdrf.tex, cdocsfn1.tex, cdocsfn2.tex.
%
%<package>\ifdefined\childdocmain\endinput\fi
%<package>\ProvidesFile{childdoc.def}[2018/12/30 v2.0 child document driver]
%<samplemain>\ProvidesFile{cdocsamp.tex}[2018/12/30 v2.0 sample for childdoc]
%<*driver>
%\ProvidesFile{childdoc.drv}[2018/12/30 v2.0 childdoc reference manual file]
\PassOptionsToClass{10pt,a4paper}{article}
\documentclass{ltxdoc}

\usepackage[margin=35mm]{geometry}
\usepackage{hyperref}
\usepackage{hyperxmp}
\usepackage[usenames]{color}

\hypersetup{colorlinks=true}
\hypersetup{pdfstartview=FitH}
\hypersetup{pdfpagemode=UseNone}
\hypersetup{pdfsource={}}
\hypersetup{pdflang={en-UK}}
\hypersetup{pdfcopyright={Copyright 2017-2018 Niklas Beisert.
  This work may be distributed and/or modified under the
  conditions of the LaTeX Project Public License, either version 1.3
  of this license or (at your option) any later version.}}
\hypersetup{pdflicenseurl={http://www.latex-project.org/lppl.txt}}
\hypersetup{pdfcontactaddress={ETH Zurich, ITP, HIT K,
  Wolfgang-Pauli-Strasse 27}}
\hypersetup{pdfcontactpostcode={8093}}
\hypersetup{pdfcontactcity={Zurich}}
\hypersetup{pdfcontactcountry={Switzerland}}
\hypersetup{pdfcontactemail={nbeisert@itp.phys.ethz.ch}}
\hypersetup{pdfcontacturl={http://people.phys.ethz.ch/\xmptilde nbeisert/}}

\newcommand{\secref}[1]{\hyperref[#1]{section \ref*{#1}}}

\parskip1ex
\parindent0pt
\let\olditemize\itemize
\def\itemize{\olditemize\parskip0pt}

\begin{document}

\title{The \textsf{childdoc} Package}
\hypersetup{pdftitle={The childdoc Package}}
\author{Niklas Beisert\\[2ex]
  Institut f\"ur Theoretische Physik\\
  Eidgen\"ossische Technische Hochschule Z\"urich\\
  Wolfgang-Pauli-Strasse 27, 8093 Z\"urich, Switzerland\\[1ex]
  \href{mailto:nbeisert@itp.phys.ethz.ch}
  {\texttt{nbeisert@itp.phys.ethz.ch}}}
\hypersetup{pdfauthor={Niklas Beisert}}
\hypersetup{pdfsubject={Manual for the LaTeX2e Package childdoc}}
\date{30 December 2018, \textsf{v2.0}}
\maketitle

\begin{abstract}\noindent
\textsf{childdoc} is a \LaTeXe{} package
that enables the direct compilation
of document sections included by |\include|
to individual files.
\end{abstract}

\begingroup
\parskip0ex
\tableofcontents
\endgroup

%%%%%%%%%%%%%%%%%%%%%%%%%%%%%%%%%%%%%%%%%%%%%%%%%%%%%%%%%%%%%%%%%%%%%%%%%%%%%%%%
%%%%%%%%%%%%%%%%%%%%%%%%%%%%%%%%%%%%%%%%%%%%%%%%%%%%%%%%%%%%%%%%%%%%%%%%%%%%%%%%
\section{Introduction}

\LaTeX{} provides a mechanism to structure a large document (such as a book)
into a main file and several child files (containing the chapters)
using the |\include| command.
This mechanism is beneficial for documents
which span hundreds of pages in order to
make the source file(s) more manageable.
Moreover, compilation can be restricted to
selected child files by means of the |\includeonly| command.
The latter feature can be used to reduce the compilation time while editing
(this was significantly more useful in the earlier days of \LaTeX{})
or to generate a smaller document which is easier to navigate.
Another application of |\includeonly| is to generate
documents consisting of selected parts of the complete document.

However, there are a few drawbacks of the plain |\include| mechanism:
\begin{itemize}
\item
The child files cannot be compiled on their own,
they can only be compiled via the main file.
A naive editing environment
(such as a text editor with an option
to have the current file processed by \LaTeX)
may require one to switch to the main file before compiling;
attempting to compile the child file produces errors.
\item
The main file must be modified (each time)
to adjust the |\includeonly| command
to the present needs. This easily leaves the main file in a messy state.
\item
The generated document will always carry the filename
of the main document. This is inconvenient if
several child files are to be compiled and
to be kept for distribution.
\end{itemize}

The present package provides a simple interface
to make child files individually compilable by \LaTeX{}.
Compiling a child file then has the same effect as compiling
the main file with an |\includeonly| command
to select the appropriate child.
Moreover the generated document will carry the name of the child
rather than the main file.
This resolves all three above issues.

This feature is meant to make the editing of books,
thesis documents and lecture notes somewhat more convenient.
However, the package can also be used efficiently for
composing a series of documents (such as exercise sheets)
which are typically distributed individually.
It then assists the author in generating the individual documents
(potentially in different versions)
as well as a document containing the collected series.
Another application is in developing style files
or other kinds of included material
where compilation of the style file could redirect
to a sample or test file.

%%%%%%%%%%%%%%%%%%%%%%%%%%%%%%%%%%%%%%%%%%%%%%%%%%%%%%%%%%%%%%%%%%%%%%%%%%%%%%%%
%%%%%%%%%%%%%%%%%%%%%%%%%%%%%%%%%%%%%%%%%%%%%%%%%%%%%%%%%%%%%%%%%%%%%%%%%%%%%%%%
\section{Usage}

First of all, the package \textsf{childdoc} is \emph{not} a standard
\LaTeXe{} |.sty| style file! Therefore it needs to be invoked in
a non-standard way.

%%%%%%%%%%%%%%%%%%%%%%%%%%%%%%%%%%%%%%%%%%%%%%%%%%%%%%%%%%%%%%%%%%%%%%%%%%%%%%%%
\subsection{Included Files}
\label{sec:include}

%%%%%%%%%%%%%%%%%%%%%%%%%%%%%%%%%%%%%%%%
\DescribeMacro{\childdocmain}
To use the package, add the commands
\begin{center}
\begin{tabular}{l}
|\input{childdoc.def}|\\
|\childdocmain{}|\\
\end{tabular}
\end{center}
at the very top of the main \LaTeX{} file,
in particular \emph{before} the |\documentclass| statement!
The argument of |\childdocmain| should be left empty
(but it must be present).

%%%%%%%%%%%%%%%%%%%%%%%%%%%%%%%%%%%%%%%%
\DescribeMacro{\childdocof}
Furthermore, add the commands
\begin{center}
\begin{tabular}{l}
|\input{childdoc.def}|\\
|\childdocof{|\textit{main}|}|\\
\end{tabular}
\end{center}
at the top of every child file \textit{child}
which is included by |\include{|\textit{child}|}|
from within the main file
(or at least for those files to be compiled individually).
The argument \textit{main} must be the filename of the main file.

There are a couple of
considerations in setting up the main and child documents:

%%%%%%%%%%%%%%%%%%%%%%%%%%%%%%%%%%%%%%%%
\paragraph{Restrictions.}

Please note the following restrictions:
\begin{itemize}
\item
|\childdocmain| must be called with one argument \textit{main}
to ensure compatibility with earlier version of the package.
It must either be empty (|\childdocmain{}|)
or precisely match the filename of the main file in which it is specified.
See \secref{sec:detection} for further information.
\item
The filename \textit{main} must be specified without the |.tex| extension.
\item
The filename \textit{main} is case sensitive
(even in case-insensitive file systems)
due to internal string comparison.
\item
The argument \textit{main} should be fully expanded, it cannot be a macro.
\item
Subdirectories and special characters should be avoided in filenames.
\item
The command |\childdocmain{|\textit{main}|}| must be followed by a whitespace.
It should not be followed immediately by another command
or by a comment mark `|%|'.
This is because the \TeX{} parser reads the token immediately following
the argument of |\childdocmain| and puts it
at the beginning of every child section;
however, a white\-space is ignored.
\end{itemize}

%%%%%%%%%%%%%%%%%%%%%%%%%%%%%%%%%%%%%%%%
\paragraph{Content of Main File.}

It is advisable to place all content in the child files included by |\include|.
Any output contained in the main file will appear in all child documents
unless suppressed manually;
it cannot be suppressed automatically by the |\includeonly| directive
and thus should normally be avoided.
A method to include some content in the main file
by means of conditional processing is described in \secref{sec:conditional}.

%%%%%%%%%%%%%%%%%%%%%%%%%%%%%%%%%%%%%%%%
\paragraph{Page Numbering.}

When only a part of the document is compiled,
the appropriate numbering of pages
(as well as other status parameters)
is determined from the |.aux| files.
The latter contain information from previous passes.
However this information needs to propagate through
all intermediate child documents.
Therefore the page numbering in child documents may well
be inconsistent until the complete document is compiled at least once.

A useful (if unconventional) way to always ensure a consistent
page numbering is to restart the numbering in each child document
and denote the pages by `\textit{child}|.|\textit{page}'
where \textit{child} represents the chapter/section number of the child file.
This can be achieved by the command
|\numberwithin{page}{|\textit{child}|}|
of the \textsf{amsmath} package
where \textit{child} can be |chapter| or |section|
depending on the chosen structuring.
Alternatively, one can modify the macro |\thepage| appropriately
and reset the counter |page| at the start of each child file.

%%%%%%%%%%%%%%%%%%%%%%%%%%%%%%%%%%%%%%%%%%%%%%%%%%%%%%%%%%%%%%%%%%%%%%%%%%%%%%%%
\subsection{Conditional Processing}
\label{sec:conditional}

The package provides a mechanism to compile different versions
of a document. To customise the versions further some conditional processing
can come in handy to distinguish which version is being compiled.
The package provides two macros to describe the compilation context:

%%%%%%%%%%%%%%%%%%%%%%%%%%%%%%%%%%%%%%%%
\DescribeMacro{\ifchilddoc}
The conditional |\ifchilddoc| distinguishes between the compilation of
child documents and the main document:
%
\begin{center}
|\ifchilddoc |\textit{child-code}| |[|\||else |\textit{main-code}]| \||fi|
\end{center}

%%%%%%%%%%%%%%%%%%%%%%%%%%%%%%%%%%%%%%%%
\DescribeMacro{\childdocname}
\DescribeMacro{\childdocjob}
The macro |\childdocname| contains the filename (without extension)
of the main or child file being processed.
Note that |\childdocjob| will always contain the name of the main file.

%%%%%%%%%%%%%%%%%%%%%%%%%%%%%%%%%%%%%%%%
\paragraph{Title Page.}

Conditional processing can be used to include a title or banner page
in the main document when proper precautions are taken.
Importantly, the code in the main file should ensure that the page counter
(as well as other status parameters which are stored in the |.aux| files)
takes the same value after the conditional processing.
Otherwise the page numbers may take divergent values
depending on which part is compiled.

For example, a title page could be declared by:
%
\begin{center}
\begin{tabular}{l}
|\ifchilddoc\||else|\\
|\addtocounter{page}{-1}|\\
\textit{code for title page}\\
|\newpage|\\
|\||fi|
\end{tabular}
\end{center}
%
A banner page for the child documents can be generated by:
%
\begin{center}
\begin{tabular}{l}
|\ifchilddoc|\\
|\addtocounter{page}{-1}|\\
\textit{code for banner page}\\
|\newpage|\\
|\||fi|
\end{tabular}
\end{center}
%
Here one could write a message such as:
\begin{center}
|This is the part \childdocname{} of \childdocjob{}.|
\end{center}

%%%%%%%%%%%%%%%%%%%%%%%%%%%%%%%%%%%%%%%%%%%%%%%%%%%%%%%%%%%%%%%%%%%%%%%%%%%%%%%%
\subsection{Flags}
\label{sec:flags}

The package makes it easy to generate different versions
of the main or child documents.
To this end compilation flags can be defined
and assigned different default values.
They will be particularly useful in conjunction
with the forwarding mechanism described in \secref{sec:forward}.

For example, it may be useful to have a flag |\version|
which can be set to |draft| or |final|.
The document source will contain some conditional code
depending on the value of |\version|.
Suppose further, the flag should default to |final| for the main file
and to |draft| for child files
which is a natural assignment for editing the document.
This is achieved by placing the following code
in the preamble of the main document
(below the |\childdocmain| directive):
%
\begin{center}
\begin{tabular}{l}
|\ifchilddoc|\\
|\providecommand{\version}{draft}|\\
|\||else|\\
|\providecommand{\version}{final}|\\
|\||fi|
\end{tabular}
\end{center}
%
The definition by |\providecommand| makes sure
that previous definitions are not overwritten.
Further statements |\providecommand{\version}{...}|
can thus be added before the above code to override it.

For the main file, one might add a line
(between |\childdocmain| and the above block)
%
\begin{center}
|%\ifchilddoc\||else\providecommand{\version}{draft}\||fi|
\end{center}
%
which can be uncommented to produce a draft version.
Likewise one can add a line to the very top of a child file
(above the |\childdocof{|\textit{main}|}| directive)
%
\begin{center}
|%\providecommand{\version}{final}|
\end{center}
%
which can be uncommented to produce the final version of this child document.

%%%%%%%%%%%%%%%%%%%%%%%%%%%%%%%%%%%%%%%%%%%%%%%%%%%%%%%%%%%%%%%%%%%%%%%%%%%%%%%%
\subsection{Forwarding}
\label{sec:forward}

Different versions of the main or child documents
using compilation flags as described in \secref{sec:flags}
can be (permanently) stored in different files
for convenient compilation, viewing and distribution.
To this end, the package defines a command
to pass on compilation to a different file:

%%%%%%%%%%%%%%%%%%%%%%%%%%%%%%%%%%%%%%%%
\DescribeMacro{\childdocforward}
The command |\childdocforward| redirects processing to
another source file:
%
\begin{center}
\begin{tabular}{l}
|\input{childdoc.def}|\\
|\childdocforward[|\textit{main}|]{|\textit{dest}|}|\\
\end{tabular}
\end{center}
%
The argument \textit{dest} is the destination file
(without extension).
It should be the main file or one of the child files.
Note that further \textsf{childdoc} directives
such as |\childdocof| and |\childdocforward|
in the indicated file will be processed in this form.
The optional argument \textit{main}
passes on directly to the main file \textit{main}
while pretending to compile the child \textit{dest}.
This form behaves as if \textit{dest}
issues |\childdocof{|\textit{main}|}| right away,
and no further \textsf{childdoc} directives will be processed.

%%%%%%%%%%%%%%%%%%%%%%%%%%%%%%%%%%%%%%%%
\DescribeMacro{\...prefix}
In the alternative form |\childdocforwardprefix|,
%
\begin{center}
\begin{tabular}{l}
|\input{childdoc.def}|\\
|\childdocforwardprefix[|\textit{main}|]{|\textit{prefix}|}{|\textit{dest}|}|
\end{tabular}
\end{center}
%
the destination file is determined by a pattern
depending on the current file:
To make this work, the current file must be called
`{\textit{prefix}\hspace{0.2em}\textit{suffix}}'
with \textit{prefix} matching precisely the argument.
Processing is then passed on to the file
`{\textit{dest}\hspace{0.2em}\textit{suffix}}'.
Surely, the same effect is achieved by
directly specifying the
argument `{\textit{dest}\hspace{0.2em}\textit{suffix}}'
in the first form.
However, that requires to set up a different file
for each child. With the alternative form of the command
all these files can have exactly the same content
which simplifies setting them up and maintaining them.

For example, the following file |draft.tex|
with a compilation flag |\version| as described in \secref{sec:flags}
compiles the main document as a draft:
%
\begin{center}
\begin{tabular}{l}
|\def\version{draft}|\\
|\input{childdoc.def}|\\
|\childdocforward{|\textit{main}|}|
\end{tabular}
\end{center}
%
Likewise, the following files |final|\textit{nn}|.tex|
compile the final version of the child document
|child|\textit{nn}|.tex|:
%
\begin{center}
\begin{tabular}{l}
|\def\version{final}|\\
|\input{childdoc.def}|\\
|\childdocforwardprefix{final}{child}|
\end{tabular}
\end{center}
%

Note that when several versions of a main file and/or of each child file
are to be generated, it may be convenient to set up a |Makefile| or
shell script to automatise the process.

%%%%%%%%%%%%%%%%%%%%%%%%%%%%%%%%%%%%%%%%%%%%%%%%%%%%%%%%%%%%%%%%%%%%%%%%%%%%%%%%
\subsection{Command Line Processing}
\label{sec:commandline}

The effect of redirection files can also be achieved by invoking
the \LaTeX{} compiler with a more elaborate command line.
Most conveniently this should be done as part
of a shell script or a |Makefile|.

When using \textsf{childdoc} in the main file, the following
command lines effectively perform a redirection
(note that depending on the shell being used,
backslashes may have to be doubled: `|\|' $\to$ `|\\|'):
%
\begin{center}
|... -jobname "|\textit{target}|" |\\|"|[\textit{flags}]%
|\input{childdoc.def}\childdocforward[|\textit{main}|]{|\textit{dest}|}"|
\end{center}
%
Here \textit{target} is the name of the output file,
\textit{main} is the name of the main file
and \textit{dest} is the name of the main or child file to be processed
(all filenames without extensions).
The optional argument \textit{main} can be omitted
if \textit{main} matches \textit{dest}.
Optionally, compilation \textit{flags} can be defined via |\def| commands.
This command line makes the \TeX{} engine believe
it is compiling the file \textit{target}
whose content is specified as the latter parameter.
The provided code then forwards the processing to
\textit{main} or \textit{dest} as described in \secref{sec:forward}.

%%%%%%%%%%%%%%%%%%%%%%%%%%%%%%%%%%%%%%%%%%%%%%%%%%%%%%%%%%%%%%%%%%%%%%%%%%%%%%%%
\subsection{Include by Input}
\label{sec:input}

Including child documents by |\include| has some restrictions by design.
Most notably, the content of a child document always occupies
its own set of pages; pages cannot be shared between child documents.
Usually, this behaviour makes perfect sense
because each child document contain an essential part of the document.
However, in some situations it may be desirable to compose
a document from a collection of parts
without having mandatory page breaks between then.
For this case, the package
provides a mechanism to include parts
by |\input| which can also be processed individually.
However, by construction this mechanism
requires manual handling of the content to be output.

%%%%%%%%%%%%%%%%%%%%%%%%%%%%%%%%%%%%%%%%
\DescribeMacro{\ifchilddocmanual}
The main file should be prepared as usual, see \secref{sec:include}.
However, the document body must make a distinction
between processing of an individual part and of the main document, e.g.:
%
\begin{center}
\begin{tabular}{l}
|\ifchilddocmanual|\\
|\input{\childdocname}|\\
|\||else|\\
\textit{document body with }|\input{|\textit{part}|}|\\
|\||fi|
\end{tabular}
\end{center}
%
The conditional |\ifchilddocmanual| is true whenever
a part to be included by |\input| is being compiled,
and the name of the part is stored in |\childdocname|.

%%%%%%%%%%%%%%%%%%%%%%%%%%%%%%%%%%%%%%%%
\DescribeMacro{\childdocby}
Each part to be included by |\input| should start with:
%
\begin{center}
\begin{tabular}{l}
|\input{childdoc.def}|\\
|\childdocby{|\textit{main}|}|\\
\end{tabular}
\end{center}
%
The directive |\childdocby| is similar to |\childdocof|
described in \secref{sec:include},
but the subsequent selection of content must be done manually.
To that end, both |\ifchilddoc| and |\ifchilddocmanual|
will be true upon processing of a part,
and the name of the part is stored in |\childdocname|.
Note that |\jobname| will be set to the filename of the current part
so that each part receives an individual |.aux| file
that does not interfere with the |.aux| file(s) of the main document.
This behaviour can be altered by the alternative form
|\childdocby[*]{|\textit{main}|}| (with a non-empty optional argument)
which uses the |.aux| file of the main document
by setting |\jobname| to \textit{main}.

%%%%%%%%%%%%%%%%%%%%%%%%%%%%%%%%%%%%%%%%%%%%%%%%%%%%%%%%%%%%%%%%%%%%%%%%%%%%%%%%
\subsection{Driver Development}
\label{sec:driver}

The \textsf{childdoc} mechanism can also be use for the development
of definition files such as \LaTeX{} styles or classes.
This case differs from the above setup with multiple parts
included by |\include| in that no |\includeonly| should be invoked.
This can be achieved by starting the include file
(before |\ProvidesPackage|) with:
%
\begin{center}
\begin{tabular}{l}
|\input{childdoc.def}|\\
|\childdocforward{|\textit{main}|}|\\
\end{tabular}
\end{center}
%
or alternatively with:
%
\begin{center}
\begin{tabular}{l}
|\input{childdoc.def}|\\
|\childdocby{|\textit{main}|}|\\
\end{tabular}
\end{center}
%
Both forms have slightly different effects as described above.
The main file is prepared as usual, see \secref{sec:include}.

%%%%%%%%%%%%%%%%%%%%%%%%%%%%%%%%%%%%%%%%%%%%%%%%%%%%%%%%%%%%%%%%%%%%%%%%%%%%%%%%
\subsection{Legacy Detection}
\label{sec:detection}

The directive |\childdocmain| in the main file can detect
whether the complete document or merely a child is to be compiled
even without using the directive |\childdocof|.
This method is deprecated because it is less robust
and there is no compelling reason to use it;
it is merely provided for backward compatibility
and it may be removed in future versions.

If the detection mechanism is to be used,
it is mandatory to correctly specify
the filename of the main file as the argument of |\childdocmain|:
%
\begin{center}
\begin{tabular}{l}
|\input{childdoc.def}|\\
|\childdocmain{|\textit{main}|}|\\
\end{tabular}
\end{center}
%
If |\jobname| does not match the argument \textit{main} of |\childdocmain|,
it is assumed that |\jobname| points to the child file to be compiled.
When using |\childdocmain| with the main file specified as argument,
it suffices to start a child file
with just |\input{|\textit{main}|}|
without loading of the package and using |\childdocof|.
If instead all processing is done
with the appropriate \textsf{childdoc} directives,
the argument of \textit{main} of |\childdocmain| can be empty.

An alternative version of the command line processing described
in \secref{sec:commandline} using the detection mechanism reads:
%
\begin{center}
|... -jobname "|\textit{target}|" "|[\textit{flags}]%
[|\def\jobname{|\textit{dest}|}|]|\input{|\textit{main}|}"|
\end{center}

%%%%%%%%%%%%%%%%%%%%%%%%%%%%%%%%%%%%%%%%%%%%%%%%%%%%%%%%%%%%%%%%%%%%%%%%%%%%%%%%
\subsection{Manual Code}
\label{sec:manual}

In case one cannot be certain whether the definitions file |childdoc.def|
is installed on the target \TeX{} distribution
and one prefers not to ship it,
it is conceivable to paste a few relevant commands into the sources.

To that end, drop all statements |\input{childdoc.def}|
and perform the replacements as outlined below.
Instead of |\childdocmain{|\textit{main}|}| add the following code
to the top of the main file:
%
\begin{center}
\begin{tabular}{l}
|\||ifdefined\childdocname\endinput\||fi\newif\ifchilddoc|\\
|\edef\childdocname{\scantokens\expandafter{\jobname\noexpand}}|\\
|\def\childdocmain{|\textit{main}|}\||ifx\childdocmain\childdocname\||else|\\
|\childdoctrue\includeonly{\childdocname}\let\jobname\childdocmain\||fi|\\
\end{tabular}
\end{center}
%
Instead of |\childdocof{|\textit{main}|}| just include the main file
at the top of each child file:
%
\begin{center}
|\input{|\textit{main}|}|
\end{center}
%
A simple redirection |\childdocforward{|\textit{dest}|}| is achieved by:
%
\begin{center}
|\def\jobname{|\textit{dest}|}\input{\jobname}|
\end{center}
%
The redirection with prefix
|\childdocforwardprefix[|\textit{prefix}|]{|\textit{dest}|}|
is accomplished by:
%
\begin{center}
\begin{tabular}{l}
|{\edef\jobname{\scantokens\expandafter{\jobname\noexpand}}|\\
|\def\redirectjob |\textit{prefix}|#1~~~{\gdef\jobname{|\textit{dest}|#1}}|\\
|\expandafter\redirectjob\jobname~~~}\input{\jobname}|
\end{tabular}
\end{center}

In an alternative approach,
child documents can be compiled by a specific command line
without additional code or specific definitions:
%
\begin{center}
|... -jobname "|\textit{target}|" "|[\textit{flags}]%
|\includeonly{|\textit{dest}|}\input{|\textit{main}|}"|
\end{center}
%

%%%%%%%%%%%%%%%%%%%%%%%%%%%%%%%%%%%%%%%%%%%%%%%%%%%%%%%%%%%%%%%%%%%%%%%%%%%%%%%%
%%%%%%%%%%%%%%%%%%%%%%%%%%%%%%%%%%%%%%%%%%%%%%%%%%%%%%%%%%%%%%%%%%%%%%%%%%%%%%%%
\section{Information}

%%%%%%%%%%%%%%%%%%%%%%%%%%%%%%%%%%%%%%%%%%%%%%%%%%%%%%%%%%%%%%%%%%%%%%%%%%%%%%%%
\subsection{Copyright}

Copyright \copyright{} 2017--2018 Niklas Beisert

This work may be distributed and/or modified under the
conditions of the \LaTeX{} Project Public License, either version 1.3
of this license or (at your option) any later version.
The latest version of this license is in
  \url{http://www.latex-project.org/lppl.txt}
and version 1.3 or later is part of all distributions of \LaTeX{}
version 2005/12/01 or later.

This work has the LPPL maintenance status `maintained'.

The Current Maintainer of this work is Niklas Beisert.

This work consists of the files |README.txt|, |childdoc.ins| and |childdoc.dtx|
as well as the derived files |childdoc.def|, |cdocsamp.tex|
with |cdocsch1.tex|, |cdocsch2.tex|, |cdocspt3.tex|, |cdocspt4.tex|,
|cdocsdrf.tex|, |cdocsfn1.tex|, |cdocsfn2.tex|
as well as |childdoc.pdf|.

%%%%%%%%%%%%%%%%%%%%%%%%%%%%%%%%%%%%%%%%%%%%%%%%%%%%%%%%%%%%%%%%%%%%%%%%%%%%%%%%
\subsection{Files and Installation}

The package consists of the files:
%
\begin{center}
\begin{tabular}{ll}
    |README.txt|   & readme file \\
    |childdoc.ins| & installation file \\
    |childdoc.dtx| & source file \\
    |childdoc.def| & definition file \\
    |cdocsamp.tex| & sample main file \\
    |cdocsch1.tex| & sample include file \\
    |cdocsch2.tex| & sample include file \\
    |cdocspt3.tex| & sample part file \\
    |cdocspt4.tex| & sample part file \\
    |cdocsdrf.tex| & sample redirection file \\
    |cdocsfn1.tex| & sample redirection file \\
    |cdocsfn2.tex| & sample redirection file \\
    |childdoc.pdf| & manual
\end{tabular}
\end{center}
%
The distribution consists of the files
|README.txt|, |childdoc.ins| and |childdoc.dtx|.
%
\begin{itemize}
\item
Run (pdf)\LaTeX{} on |childdoc.dtx|
to compile the manual |childdoc.pdf| (this file).
\item
Run \LaTeX{} on |childdoc.ins| to create the definitions file |childdoc.def|
and the sample |cdocsamp.tex| with include files
|cdocsch1.tex|, |cdocsch2.tex|, |cdocspt3.tex|, |cdocspt4.tex|,
|cdocsdrf.tex|, |cdocsfn1.tex|, |cdocsfn2.tex|.
Then copy the file |childdoc.def| to an appropriate directory of your \LaTeX{}
distribution, e.g.\ \textit{texmf-root}|/tex/latex/childdoc|.
\end{itemize}

%%%%%%%%%%%%%%%%%%%%%%%%%%%%%%%%%%%%%%%%%%%%%%%%%%%%%%%%%%%%%%%%%%%%%%%%%%%%%%%%
\subsection{Related CTAN Packages}

There are several other packages which offer a similar functionality:
%
\begin{itemize}
\item
The packages
\href{http://ctan.org/pkg/docmute}{\textsf{docmute}},
\href{http://ctan.org/pkg/includex}{\textsf{includex}} and
\href{http://ctan.org/pkg/standalone}{\textsf{standalone}}
provide commands to include only the document body of
a child file thus allowing both files to be compiled individually.
\item
The packages \href{http://ctan.org/pkg/subdocs}{\textsf{subdocs}}
and \href{http://ctan.org/pkg/subfiles}{\textsf{subfiles}}
provide structures in which the main and child documents can be
encapsulated and allowing them to be compiled individually.
The inclusion mechanism is different from the conventional |\include|.
\item
The package \href{http://ctan.org/pkg/combine}{\textsf{combine}}
is an elaborate solution to combine several documents into one.
\end{itemize}
%
See also the CTAN topic \href{http://ctan.org/topic/subdocs}{\textsf{subdocs}}
for further related packages.
The present package differs from the above solutions in that
a document structure constructed with the conventional |\include| mechanism
just needs two extra commands at the top of every file
such that all constituent files can be compiled individually.

%%%%%%%%%%%%%%%%%%%%%%%%%%%%%%%%%%%%%%%%%%%%%%%%%%%%%%%%%%%%%%%%%%%%%%%%%%%%%%%%
%\subsection{Feature Suggestions}
%
%The following is a list of features which may be useful for future
%versions of this package:
%%
%\begin{itemize}
%\item
%\ldots
%\end{itemize}

%%%%%%%%%%%%%%%%%%%%%%%%%%%%%%%%%%%%%%%%%%%%%%%%%%%%%%%%%%%%%%%%%%%%%%%%%%%%%%%%
\subsection{Revision History}

%%%%%%%%%%%%%%%%%%%%%%%%%%%%%%%%%%%%%%%%
\paragraph{v2.0:} 2018/12/30

\begin{itemize}
\item
immediate forward processing
\item
added |\childdocby| mechanism
\item
manual restructured
\end{itemize}

%%%%%%%%%%%%%%%%%%%%%%%%%%%%%%%%%%%%%%%%
\paragraph{v1.6:} 2018/01/17

\begin{itemize}
\item
application for development of include files
\item
corrections to manual
\end{itemize}

%%%%%%%%%%%%%%%%%%%%%%%%%%%%%%%%%%%%%%%%
\paragraph{v1.5:} 2017/05/21

\begin{itemize}
\item
more complete structuring introduced
\item
|\childdocof| introduced
\item
|\childdoc| renamed to |\childdocmain|
\item
|\childredirect| renamed to |\childdocforward| and |\childdocforwardprefix|
and functionality expanded
\end{itemize}

%%%%%%%%%%%%%%%%%%%%%%%%%%%%%%%%%%%%%%%%
\paragraph{v1.0:} 2017/04/27

\begin{itemize}
\item
manual and install package
\item
first version published on CTAN
\end{itemize}

%%%%%%%%%%%%%%%%%%%%%%%%%%%%%%%%%%%%%%%%
\paragraph{v0.6:} 2017/04/26

\begin{itemize}
\item
redirection mechanism added
\end{itemize}

%%%%%%%%%%%%%%%%%%%%%%%%%%%%%%%%%%%%%%%%
\paragraph{v0.5:} 2017/04/26

\begin{itemize}
\item
functionality in definition file
\end{itemize}


%%%%%%%%%%%%%%%%%%%%%%%%%%%%%%%%%%%%%%%%%%%%%%%%%%%%%%%%%%%%%%%%%%%%%%%%%%%%%%%%
%%%%%%%%%%%%%%%%%%%%%%%%%%%%%%%%%%%%%%%%%%%%%%%%%%%%%%%%%%%%%%%%%%%%%%%%%%%%%%%%
%%%%%%%%%%%%%%%%%%%%%%%%%%%%%%%%%%%%%%%%%%%%%%%%%%%%%%%%%%%%%%%%%%%%%%%%%%%%%%%%
\appendix

\settowidth\MacroIndent{\rmfamily\scriptsize 000\ }

 \DocInput{childdoc.dtx}

\end{document}
%</driver>
% \fi
%
% %%%%%%%%%%%%%%%%%%%%%%%%%%%%%%%%%%%%%%%%%%%%%%%%%%%%%%%%%%%%%%%%%%%%%%%%%%%%%%
% %%%%%%%%%%%%%%%%%%%%%%%%%%%%%%%%%%%%%%%%%%%%%%%%%%%%%%%%%%%%%%%%%%%%%%%%%%%%%%
% \section{Sample}
%\iffalse
%<*samplemain>
%\fi
%
% The following presents a sample document
% with two chapters, two parts, a title page,
% a compile flag as well as three forwarding files to set the flag.
% It consists of eight |.tex| files:
% \begin{center}
% \begin{tabular}{ll}
% |cdocsamp.tex|&main file\\
% |cdocsch1.tex|&include file for chapter 1\\
% |cdocsch2.tex|&include file for chapter 2\\
% |cdocspt3.tex|&include file for part 3\\
% |cdocspt4.tex|&include file for part 4\\
% |cdocsdrf.tex|&forwarding file for main file in draft mode\\
% |cdocsfi1.tex|&forwarding file for final version of chapter 1\\
% |cdocsfi2.tex|&forwarding file for final version of chapter 2\\
% \end{tabular}
% \end{center}
% Each of the eight files can be compiled directly by the \LaTeX{} compiler.
%
% %%%%%%%%%%%%%%%%%%%%%%%%%%%%%%%%%%%%%%
% \paragraph{Main File.}
%
% The main file is called |cdocsamp.tex|.
%
% Load the \textsf{childdoc} definitions and
% declare the filename for the main document:
%    \begin{macrocode}
\input{childdoc.def}
\childdocmain{}
%    \end{macrocode}

% Optional override for |\version| flag:
%    \begin{macrocode}
%%\ifchilddoc\else\providecommand{\version}{draft}\fi
%    \end{macrocode}

% Define the default values for the |\version| flag
% (|final| for the main file and |draft| for childs):
%    \begin{macrocode}
\ifchilddoc
\providecommand{\version}{draft}
\else
\providecommand{\version}{final}
\fi
%    \end{macrocode}

% Load the standard document class:
%    \begin{macrocode}
\documentclass[12pt]{article}
%    \end{macrocode}

% Start the document body:
%    \begin{macrocode}
\begin{document}
%    \end{macrocode}

% Declare a title page.
% Print title, part of document being processed and version flag:
%    \begin{macrocode}
\addtocounter{page}{-1}
\begin{center}
{\LARGE\bfseries{}childdoc example\par}
\vspace{1cm}
\ifchilddoc
\ifchilddocmanual part\else chapter\fi:
`\childdocname' of `\childdocjob'\par
\else
main document: `\childdocjob'\par
\fi
version: \version\par
\end{center}
\newpage
%    \end{macrocode}

% Manually include selected file,
% otherwise process as usual:
%    \begin{macrocode}
\ifchilddocmanual
\section*{part `\childdocname'}
\input{\childdocname}
\else
%    \end{macrocode}

% Include the two chapters:
%    \begin{macrocode}
\include{cdocsch1}
\include{cdocsch2}
%    \end{macrocode}

% Include the two parts unless only chapters should be displayed:
%    \begin{macrocode}
\ifchilddoc\else
\section{part three}
\input{cdocspt3}
\section{part four}
\input{cdocspt4}
\fi
%    \end{macrocode}

% Process as usual until here:
%    \begin{macrocode}
\fi
%    \end{macrocode}

% End of document body:
%    \begin{macrocode}
\end{document}
%    \end{macrocode}
%\iffalse
%</samplemain>
%\fi
%
% %%%%%%%%%%%%%%%%%%%%%%%%%%%%%%%%%%%%%%
% \paragraph{Chapter Include Files.}
%
% The include files are called |cdocsch1.tex| and |cdocsch2.tex|.
%
%\iffalse
%<*samplechap1|samplechap2>
%\fi

% Optional override for |\version| flag:
%    \begin{macrocode}
%%\providecommand{\version}{final}
%    \end{macrocode}

% Include the main document:
%    \begin{macrocode}
\input{childdoc.def}
\childdocof{cdocsamp}
%    \end{macrocode}

%\iffalse
%</samplechap1|samplechap2>
%\fi
%
%\iffalse
%<*samplechap1>
%\fi
% Some text for chapter 1:
%    \begin{macrocode}
\section{one}
some text in chapter one
%    \end{macrocode}

%\iffalse
%</samplechap1>
%\fi
% Some text for chapter 2:
%\iffalse
%<*samplechap2>
%\fi
%    \begin{macrocode}
\section{two}
more text in chapter two
%    \end{macrocode}

%\iffalse
%</samplechap2>
%\fi
%
% %%%%%%%%%%%%%%%%%%%%%%%%%%%%%%%%%%%%%%
% \paragraph{Part Include Files.}
%
% The include files are called |cdocspt3.tex| and |cdocspt4.tex|.
%
%\iffalse
%<*samplepart3|samplepart4>
%\fi

% Optional override for |\version| flag:
%    \begin{macrocode}
%%\providecommand{\version}{final}
%    \end{macrocode}

% Include the main document:
%    \begin{macrocode}
\input{childdoc.def}
\childdocby{cdocsamp}
%    \end{macrocode}

%\iffalse
%</samplepart3|samplepart4>
%\fi
%
%\iffalse
%<*samplepart3>
%\fi
% Some text for part 3:
%    \begin{macrocode}
some text in part three
%    \end{macrocode}

%\iffalse
%</samplepart3>
%\fi
% Some text for part 4:
%\iffalse
%<*samplepart4>
%\fi
%    \begin{macrocode}
more text in part four
%    \end{macrocode}

%\iffalse
%</samplepart4>
%\fi
%
% %%%%%%%%%%%%%%%%%%%%%%%%%%%%%%%%%%%%%%
% \paragraph{Forwarding for a Complete Draft.}
%
% The following forwarding file |cdocsdrf.tex|
% compiles the main document in draft mode:
%\iffalse
%<*sampledraft>
%\fi
%    \begin{macrocode}
\def\version{draft}
\input{childdoc.def}
\childdocforward{cdocsamp}
%    \end{macrocode}

%\iffalse
%</sampledraft>
%\fi
%
% %%%%%%%%%%%%%%%%%%%%%%%%%%%%%%%%%%%%%%
% \paragraph{Forwarding for Final Version of the Chapters.}
%
% The following forwarding files |cdocsfn1.tex| and |cdocsfn2.tex|
% (with identical content)
% compile the final versions of the child documents
% |cdocsch1.tex| and |cdocsch2.tex|, respectively:
%\iffalse
%<*samplefinal>
%\fi
%    \begin{macrocode}
\def\version{final}
\input{childdoc.def}
\childdocforwardprefix[cdocsamp]{cdocsfn}{cdocsch}
%    \end{macrocode}

%\iffalse
%</samplefinal>
%\fi
%
% %%%%%%%%%%%%%%%%%%%%%%%%%%%%%%%%%%%%%%
% \paragraph{Command Line Processing.}
%
% The following three command lines generate the output files
% |cdocscld|, |cdocscl1| and |cdocscl2|
% which should be identical to
% |cdocsdrf|, |cdocsch1| and |cdocsfn2|, respectively:
% \begin{center}
% \begin{tabular}{l}
% |latex -jobname cdocscld \|\\
% |  "\def\version{draft}\input{childdoc.def}\childdocforward{cdocsamp}"|\\
% |latex -jobname cdocscl1 \|\\
% |  "\input{childdoc.def}\childdocforward[cdocsamp]{cdocsch1}"|\\
% |latex -jobname cdocscl2 \|\\
% |  "\def\version{final}\input{childdoc.def}\childdocforward{cdocsch2}"|
% \end{tabular}
% \end{center}
% Note that the trailing backslash on each first line
% merely continues the input to the second line
% (for convenient cut ant paste).
% Furthermore, the command |latex| can be replaced by any
% of its alternative versions such as |pdflatex|.
%
% %%%%%%%%%%%%%%%%%%%%%%%%%%%%%%%%%%%%%%%%%%%%%%%%%%%%%%%%%%%%%%%%%%%%%%%%%%%%%%
% %%%%%%%%%%%%%%%%%%%%%%%%%%%%%%%%%%%%%%%%%%%%%%%%%%%%%%%%%%%%%%%%%%%%%%%%%%%%%%
% \section{Implementation}
%\iffalse
%<*package>
%\fi
%
% This section describes the definitions file |childdoc.def|.

% The definitions cannot be loaded using |\usepackage| or |\RequirePackage|
% which has a mechanism to prevent loading a style file more than once.
% When loading the definitions by means of |\input|
% multiple instances have to be prevented manually:
%\iffalse
%This code needs to be before the `\ProvidesFile' directive
%which is defined at the beginning of this file.
%Therefore it is also placed there and commented out here.
%</package>
%<*discard>
%\fi
%    \begin{macrocode}
\ifdefined\childdocmain\endinput\fi
%    \end{macrocode}
%\iffalse
%</discard>
%<*package>
%\fi
%
% \macro{\ifchilddoc}
% \macro{\ifchilddocmanual}
% The conditional |\ifchilddoc| tells whether a
% child (true) or main (false) document is being compiled.
% The conditional |\ifchilddocmanual| tells whether
% the |\includeonly| mechanism is used (false) or
% the selection of child files must be performed manually (true).
% The definitions initialise to false:
%    \begin{macrocode}
\newif\ifchilddoc
\newif\ifchilddocmanual
%    \end{macrocode}

% \macro{\childdocname}
% \macro{\childdocjob}
% The macro |\childdocname| stores the name of the main document
% to be compiled. The macro |\childdocjob| stores the name of
% the document on which the \LaTeX{} compiler was originally invoked.
% The content of |\jobname| cannot be compared
% to filenames specified in the source due to different catcodes.
% The following code rescans |\jobname|, stores the result
% in |\childdocname| and saves a copy in |\childdocjob|:
%    \begin{macrocode}
\edef\childdocname{\scantokens\expandafter{\jobname\noexpand}}
\let\childdocjob\childdocname
%    \end{macrocode}

% \macro{\childdocdisable}
% The macro |\childdocdisable| prevents the main file
% from being processed more than once.
% At this stage, the main document command |\childdocmain|
% is assumed to be called once again where it should do nothing.
% Any subsequent call to it should prevent
% a secondary processing of the main document
% It overwrites the forwarding commands
% |\childdocof| and |\childdocforward|
% with empty macros to prevent further inclusions of the main document:
%    \begin{macrocode}
\newcommand{\childdocdisable}
{
  \renewcommand{\childdocmain}[1]{\renewcommand{\childdocmain}[1]{\endinput}}
  \renewcommand{\childdocof}[1]{}
  \renewcommand{\childdocby}[2][]{}
  \renewcommand{\childdocforward}[2][]{}
  \renewcommand{\childdocdisable}{}
}
%    \end{macrocode}

% \macro{\childdocmain}
% The macro |\childdocmain| is to be called at the top of the main file
% with nothing or the main filename (without extension) as argument.
% First, it breaks loops.
% If the argument is not empty and does not match |\childdocname|
% (which is set by the first inclusion of |childdoc.def|),
% |\ifchilddoc| is set to true, |\includeonly| is applied to the child file
% and |\jobname| is set to the main file
% (for proper handling of |.aux| files):
%    \begin{macrocode}
\newcommand{\childdocmain}[1]
{
  \childdocdisable\childdocmain{}
  \if?#1?\else
    \begingroup
      \def\childdoctmp{#1}
      \ifx\childdoctmp\childdocname
        \def\childdoctmp{}
      \else
        \def\childdoctmp
        {
          \childdoctrue
          \includeonly{\childdocname}
          \def\childdocjob{#1}
          \def\jobname{#1}
        }
      \fi
      \expandafter
    \endgroup
    \childdoctmp
  \fi
}
%    \end{macrocode}

% \macro{\childdocof}
% The command |\childdocof| redirects
% compilation to the main file |#1|.
%    \begin{macrocode}
\newcommand{\childdocof}[1]
{
  \childdocdisable
  \childdoctrue
  \includeonly{\childdocname}
  \def\jobname{#1}
  \def\childdocjob{#1}
  \input{#1}
}
%    \end{macrocode}

% \macro{\childdocby}
% The command |\childdocby| ....
%    \begin{macrocode}
\newcommand{\childdocby}[2][]
{
  \childdocdisable
  \childdoctrue
  \childdocmanualtrue
  \if?#1?\else
    \def\jobname{#2}
  \fi
  \def\childdocjob{#2}
  \input{#2}
  \endinput
}
%    \end{macrocode}

% \macro{\childdocforward}
% The command |\childdocforward| redirects
% compilation to the main file or
% (if the optional argument is given) a child file.
% Parameters are set as if the main file
% or a child file starting with |\childdocof| was compiled.
% Then compilation is handed over to the main file:
%    \begin{macrocode}
\newcommand{\childdocforward}[2][]
{
  \begingroup
    \if?#1?
      \def\childdoctmp
      {
        \def\childdocname{#2}
        \def\childdocjob{#2}
        \def\jobname{#2}
        \input{#2}
        \endinput
      }
    \else
      \def\childdoctmp
      {
        \childdocdisable
        \def\childdocname{#2}
        \childdoctrue
        \includeonly{#2}
        \def\childdocjob{#1}
        \def\jobname{#1}
        \input{#1}
        \endinput
      }
    \fi
    \expandafter
  \endgroup
  \childdoctmp
}
%    \end{macrocode}

% \macro{\childdocforwardprefix}
% The command |\childdocforwardprefix| redirects
% compilation to the main or a child file by means of a pattern.
% The prefix |#1| in the current filename is replaced by |#2|
% and the suffix of the current filename is kept
% (it is assumed that the filename does not contain the substring `|~~~|'
% which is used as a delimiter).
% Compilation is handed over to the new file by |\childdocforward|:
%    \begin{macrocode}
\newcommand{\childdocforwardprefix}[3][]
{
  \begingroup
    \def\childdocextract #2##1~~~{\def\childdoctmp{\childdocforward[#1]{#3##1}}}
    \expandafter\childdocextract\childdocname~~~
    \expandafter
  \endgroup
  \childdoctmp
}
%    \end{macrocode}

% \macro{\childdoc}
% The deprecated macro |\childdoc| is a legacy version of |\childdocmain|:
%    \begin{macrocode}
\newcommand{\childdoc}{\childdocmain}
%    \end{macrocode}

% \macro{\childdocredirect}
% The deprecated macro |\childdocredirect| is a legacy version
% of |\childdocforward| and |\childdocforwardprefix|:
%    \begin{macrocode}
\newcommand{\childdocredirect}[2][]
{
  \begingroup
    \if?#1?
      \def\childdoctmp{\childdocforward{#2}}
    \else
      \def\childdoctmp{\childdocforwardprefix{#1}{#2}}
    \fi
    \expandafter
  \endgroup
  \childdoctmp
}
%    \end{macrocode}

%\iffalse
%</package>
%\fi
%
\endinput

\childdocforwardprefix[cdocsamp]{cdocsfn}{cdocsch}
%    \end{macrocode}

%\iffalse
%</samplefinal>
%\fi
%
% %%%%%%%%%%%%%%%%%%%%%%%%%%%%%%%%%%%%%%
% \paragraph{Command Line Processing.}
%
% The following three command lines generate the output files
% |cdocscld|, |cdocscl1| and |cdocscl2|
% which should be identical to
% |cdocsdrf|, |cdocsch1| and |cdocsfn2|, respectively:
% \begin{center}
% \begin{tabular}{l}
% |latex -jobname cdocscld \|\\
% |  "\def\version{draft}% \iffalse
%
% childdoc.dtx Copyright (C) 2017-2018 Niklas Beisert
%
% This work may be distributed and/or modified under the
% conditions of the LaTeX Project Public License, either version 1.3
% of this license or (at your option) any later version.
% The latest version of this license is in
%   http://www.latex-project.org/lppl.txt
% and version 1.3 or later is part of all distributions of LaTeX
% version 2005/12/01 or later.
%
% This work has the LPPL maintenance status `maintained'.
%
% The Current Maintainer of this work is Niklas Beisert.
%
% This work consists of the files childdoc.dtx and childdoc.ins
% and the derived files childdoc.def and cdocsamp.tex with
% cdocsch1.tex, cdocsch2.tex, cdocsdrf.tex, cdocsfn1.tex, cdocsfn2.tex.
%
%<package>\ifdefined\childdocmain\endinput\fi
%<package>\ProvidesFile{childdoc.def}[2018/12/30 v2.0 child document driver]
%<samplemain>\ProvidesFile{cdocsamp.tex}[2018/12/30 v2.0 sample for childdoc]
%<*driver>
%\ProvidesFile{childdoc.drv}[2018/12/30 v2.0 childdoc reference manual file]
\PassOptionsToClass{10pt,a4paper}{article}
\documentclass{ltxdoc}

\usepackage[margin=35mm]{geometry}
\usepackage{hyperref}
\usepackage{hyperxmp}
\usepackage[usenames]{color}

\hypersetup{colorlinks=true}
\hypersetup{pdfstartview=FitH}
\hypersetup{pdfpagemode=UseNone}
\hypersetup{pdfsource={}}
\hypersetup{pdflang={en-UK}}
\hypersetup{pdfcopyright={Copyright 2017-2018 Niklas Beisert.
  This work may be distributed and/or modified under the
  conditions of the LaTeX Project Public License, either version 1.3
  of this license or (at your option) any later version.}}
\hypersetup{pdflicenseurl={http://www.latex-project.org/lppl.txt}}
\hypersetup{pdfcontactaddress={ETH Zurich, ITP, HIT K,
  Wolfgang-Pauli-Strasse 27}}
\hypersetup{pdfcontactpostcode={8093}}
\hypersetup{pdfcontactcity={Zurich}}
\hypersetup{pdfcontactcountry={Switzerland}}
\hypersetup{pdfcontactemail={nbeisert@itp.phys.ethz.ch}}
\hypersetup{pdfcontacturl={http://people.phys.ethz.ch/\xmptilde nbeisert/}}

\newcommand{\secref}[1]{\hyperref[#1]{section \ref*{#1}}}

\parskip1ex
\parindent0pt
\let\olditemize\itemize
\def\itemize{\olditemize\parskip0pt}

\begin{document}

\title{The \textsf{childdoc} Package}
\hypersetup{pdftitle={The childdoc Package}}
\author{Niklas Beisert\\[2ex]
  Institut f\"ur Theoretische Physik\\
  Eidgen\"ossische Technische Hochschule Z\"urich\\
  Wolfgang-Pauli-Strasse 27, 8093 Z\"urich, Switzerland\\[1ex]
  \href{mailto:nbeisert@itp.phys.ethz.ch}
  {\texttt{nbeisert@itp.phys.ethz.ch}}}
\hypersetup{pdfauthor={Niklas Beisert}}
\hypersetup{pdfsubject={Manual for the LaTeX2e Package childdoc}}
\date{30 December 2018, \textsf{v2.0}}
\maketitle

\begin{abstract}\noindent
\textsf{childdoc} is a \LaTeXe{} package
that enables the direct compilation
of document sections included by |\include|
to individual files.
\end{abstract}

\begingroup
\parskip0ex
\tableofcontents
\endgroup

%%%%%%%%%%%%%%%%%%%%%%%%%%%%%%%%%%%%%%%%%%%%%%%%%%%%%%%%%%%%%%%%%%%%%%%%%%%%%%%%
%%%%%%%%%%%%%%%%%%%%%%%%%%%%%%%%%%%%%%%%%%%%%%%%%%%%%%%%%%%%%%%%%%%%%%%%%%%%%%%%
\section{Introduction}

\LaTeX{} provides a mechanism to structure a large document (such as a book)
into a main file and several child files (containing the chapters)
using the |\include| command.
This mechanism is beneficial for documents
which span hundreds of pages in order to
make the source file(s) more manageable.
Moreover, compilation can be restricted to
selected child files by means of the |\includeonly| command.
The latter feature can be used to reduce the compilation time while editing
(this was significantly more useful in the earlier days of \LaTeX{})
or to generate a smaller document which is easier to navigate.
Another application of |\includeonly| is to generate
documents consisting of selected parts of the complete document.

However, there are a few drawbacks of the plain |\include| mechanism:
\begin{itemize}
\item
The child files cannot be compiled on their own,
they can only be compiled via the main file.
A naive editing environment
(such as a text editor with an option
to have the current file processed by \LaTeX)
may require one to switch to the main file before compiling;
attempting to compile the child file produces errors.
\item
The main file must be modified (each time)
to adjust the |\includeonly| command
to the present needs. This easily leaves the main file in a messy state.
\item
The generated document will always carry the filename
of the main document. This is inconvenient if
several child files are to be compiled and
to be kept for distribution.
\end{itemize}

The present package provides a simple interface
to make child files individually compilable by \LaTeX{}.
Compiling a child file then has the same effect as compiling
the main file with an |\includeonly| command
to select the appropriate child.
Moreover the generated document will carry the name of the child
rather than the main file.
This resolves all three above issues.

This feature is meant to make the editing of books,
thesis documents and lecture notes somewhat more convenient.
However, the package can also be used efficiently for
composing a series of documents (such as exercise sheets)
which are typically distributed individually.
It then assists the author in generating the individual documents
(potentially in different versions)
as well as a document containing the collected series.
Another application is in developing style files
or other kinds of included material
where compilation of the style file could redirect
to a sample or test file.

%%%%%%%%%%%%%%%%%%%%%%%%%%%%%%%%%%%%%%%%%%%%%%%%%%%%%%%%%%%%%%%%%%%%%%%%%%%%%%%%
%%%%%%%%%%%%%%%%%%%%%%%%%%%%%%%%%%%%%%%%%%%%%%%%%%%%%%%%%%%%%%%%%%%%%%%%%%%%%%%%
\section{Usage}

First of all, the package \textsf{childdoc} is \emph{not} a standard
\LaTeXe{} |.sty| style file! Therefore it needs to be invoked in
a non-standard way.

%%%%%%%%%%%%%%%%%%%%%%%%%%%%%%%%%%%%%%%%%%%%%%%%%%%%%%%%%%%%%%%%%%%%%%%%%%%%%%%%
\subsection{Included Files}
\label{sec:include}

%%%%%%%%%%%%%%%%%%%%%%%%%%%%%%%%%%%%%%%%
\DescribeMacro{\childdocmain}
To use the package, add the commands
\begin{center}
\begin{tabular}{l}
|\input{childdoc.def}|\\
|\childdocmain{}|\\
\end{tabular}
\end{center}
at the very top of the main \LaTeX{} file,
in particular \emph{before} the |\documentclass| statement!
The argument of |\childdocmain| should be left empty
(but it must be present).

%%%%%%%%%%%%%%%%%%%%%%%%%%%%%%%%%%%%%%%%
\DescribeMacro{\childdocof}
Furthermore, add the commands
\begin{center}
\begin{tabular}{l}
|\input{childdoc.def}|\\
|\childdocof{|\textit{main}|}|\\
\end{tabular}
\end{center}
at the top of every child file \textit{child}
which is included by |\include{|\textit{child}|}|
from within the main file
(or at least for those files to be compiled individually).
The argument \textit{main} must be the filename of the main file.

There are a couple of
considerations in setting up the main and child documents:

%%%%%%%%%%%%%%%%%%%%%%%%%%%%%%%%%%%%%%%%
\paragraph{Restrictions.}

Please note the following restrictions:
\begin{itemize}
\item
|\childdocmain| must be called with one argument \textit{main}
to ensure compatibility with earlier version of the package.
It must either be empty (|\childdocmain{}|)
or precisely match the filename of the main file in which it is specified.
See \secref{sec:detection} for further information.
\item
The filename \textit{main} must be specified without the |.tex| extension.
\item
The filename \textit{main} is case sensitive
(even in case-insensitive file systems)
due to internal string comparison.
\item
The argument \textit{main} should be fully expanded, it cannot be a macro.
\item
Subdirectories and special characters should be avoided in filenames.
\item
The command |\childdocmain{|\textit{main}|}| must be followed by a whitespace.
It should not be followed immediately by another command
or by a comment mark `|%|'.
This is because the \TeX{} parser reads the token immediately following
the argument of |\childdocmain| and puts it
at the beginning of every child section;
however, a white\-space is ignored.
\end{itemize}

%%%%%%%%%%%%%%%%%%%%%%%%%%%%%%%%%%%%%%%%
\paragraph{Content of Main File.}

It is advisable to place all content in the child files included by |\include|.
Any output contained in the main file will appear in all child documents
unless suppressed manually;
it cannot be suppressed automatically by the |\includeonly| directive
and thus should normally be avoided.
A method to include some content in the main file
by means of conditional processing is described in \secref{sec:conditional}.

%%%%%%%%%%%%%%%%%%%%%%%%%%%%%%%%%%%%%%%%
\paragraph{Page Numbering.}

When only a part of the document is compiled,
the appropriate numbering of pages
(as well as other status parameters)
is determined from the |.aux| files.
The latter contain information from previous passes.
However this information needs to propagate through
all intermediate child documents.
Therefore the page numbering in child documents may well
be inconsistent until the complete document is compiled at least once.

A useful (if unconventional) way to always ensure a consistent
page numbering is to restart the numbering in each child document
and denote the pages by `\textit{child}|.|\textit{page}'
where \textit{child} represents the chapter/section number of the child file.
This can be achieved by the command
|\numberwithin{page}{|\textit{child}|}|
of the \textsf{amsmath} package
where \textit{child} can be |chapter| or |section|
depending on the chosen structuring.
Alternatively, one can modify the macro |\thepage| appropriately
and reset the counter |page| at the start of each child file.

%%%%%%%%%%%%%%%%%%%%%%%%%%%%%%%%%%%%%%%%%%%%%%%%%%%%%%%%%%%%%%%%%%%%%%%%%%%%%%%%
\subsection{Conditional Processing}
\label{sec:conditional}

The package provides a mechanism to compile different versions
of a document. To customise the versions further some conditional processing
can come in handy to distinguish which version is being compiled.
The package provides two macros to describe the compilation context:

%%%%%%%%%%%%%%%%%%%%%%%%%%%%%%%%%%%%%%%%
\DescribeMacro{\ifchilddoc}
The conditional |\ifchilddoc| distinguishes between the compilation of
child documents and the main document:
%
\begin{center}
|\ifchilddoc |\textit{child-code}| |[|\||else |\textit{main-code}]| \||fi|
\end{center}

%%%%%%%%%%%%%%%%%%%%%%%%%%%%%%%%%%%%%%%%
\DescribeMacro{\childdocname}
\DescribeMacro{\childdocjob}
The macro |\childdocname| contains the filename (without extension)
of the main or child file being processed.
Note that |\childdocjob| will always contain the name of the main file.

%%%%%%%%%%%%%%%%%%%%%%%%%%%%%%%%%%%%%%%%
\paragraph{Title Page.}

Conditional processing can be used to include a title or banner page
in the main document when proper precautions are taken.
Importantly, the code in the main file should ensure that the page counter
(as well as other status parameters which are stored in the |.aux| files)
takes the same value after the conditional processing.
Otherwise the page numbers may take divergent values
depending on which part is compiled.

For example, a title page could be declared by:
%
\begin{center}
\begin{tabular}{l}
|\ifchilddoc\||else|\\
|\addtocounter{page}{-1}|\\
\textit{code for title page}\\
|\newpage|\\
|\||fi|
\end{tabular}
\end{center}
%
A banner page for the child documents can be generated by:
%
\begin{center}
\begin{tabular}{l}
|\ifchilddoc|\\
|\addtocounter{page}{-1}|\\
\textit{code for banner page}\\
|\newpage|\\
|\||fi|
\end{tabular}
\end{center}
%
Here one could write a message such as:
\begin{center}
|This is the part \childdocname{} of \childdocjob{}.|
\end{center}

%%%%%%%%%%%%%%%%%%%%%%%%%%%%%%%%%%%%%%%%%%%%%%%%%%%%%%%%%%%%%%%%%%%%%%%%%%%%%%%%
\subsection{Flags}
\label{sec:flags}

The package makes it easy to generate different versions
of the main or child documents.
To this end compilation flags can be defined
and assigned different default values.
They will be particularly useful in conjunction
with the forwarding mechanism described in \secref{sec:forward}.

For example, it may be useful to have a flag |\version|
which can be set to |draft| or |final|.
The document source will contain some conditional code
depending on the value of |\version|.
Suppose further, the flag should default to |final| for the main file
and to |draft| for child files
which is a natural assignment for editing the document.
This is achieved by placing the following code
in the preamble of the main document
(below the |\childdocmain| directive):
%
\begin{center}
\begin{tabular}{l}
|\ifchilddoc|\\
|\providecommand{\version}{draft}|\\
|\||else|\\
|\providecommand{\version}{final}|\\
|\||fi|
\end{tabular}
\end{center}
%
The definition by |\providecommand| makes sure
that previous definitions are not overwritten.
Further statements |\providecommand{\version}{...}|
can thus be added before the above code to override it.

For the main file, one might add a line
(between |\childdocmain| and the above block)
%
\begin{center}
|%\ifchilddoc\||else\providecommand{\version}{draft}\||fi|
\end{center}
%
which can be uncommented to produce a draft version.
Likewise one can add a line to the very top of a child file
(above the |\childdocof{|\textit{main}|}| directive)
%
\begin{center}
|%\providecommand{\version}{final}|
\end{center}
%
which can be uncommented to produce the final version of this child document.

%%%%%%%%%%%%%%%%%%%%%%%%%%%%%%%%%%%%%%%%%%%%%%%%%%%%%%%%%%%%%%%%%%%%%%%%%%%%%%%%
\subsection{Forwarding}
\label{sec:forward}

Different versions of the main or child documents
using compilation flags as described in \secref{sec:flags}
can be (permanently) stored in different files
for convenient compilation, viewing and distribution.
To this end, the package defines a command
to pass on compilation to a different file:

%%%%%%%%%%%%%%%%%%%%%%%%%%%%%%%%%%%%%%%%
\DescribeMacro{\childdocforward}
The command |\childdocforward| redirects processing to
another source file:
%
\begin{center}
\begin{tabular}{l}
|\input{childdoc.def}|\\
|\childdocforward[|\textit{main}|]{|\textit{dest}|}|\\
\end{tabular}
\end{center}
%
The argument \textit{dest} is the destination file
(without extension).
It should be the main file or one of the child files.
Note that further \textsf{childdoc} directives
such as |\childdocof| and |\childdocforward|
in the indicated file will be processed in this form.
The optional argument \textit{main}
passes on directly to the main file \textit{main}
while pretending to compile the child \textit{dest}.
This form behaves as if \textit{dest}
issues |\childdocof{|\textit{main}|}| right away,
and no further \textsf{childdoc} directives will be processed.

%%%%%%%%%%%%%%%%%%%%%%%%%%%%%%%%%%%%%%%%
\DescribeMacro{\...prefix}
In the alternative form |\childdocforwardprefix|,
%
\begin{center}
\begin{tabular}{l}
|\input{childdoc.def}|\\
|\childdocforwardprefix[|\textit{main}|]{|\textit{prefix}|}{|\textit{dest}|}|
\end{tabular}
\end{center}
%
the destination file is determined by a pattern
depending on the current file:
To make this work, the current file must be called
`{\textit{prefix}\hspace{0.2em}\textit{suffix}}'
with \textit{prefix} matching precisely the argument.
Processing is then passed on to the file
`{\textit{dest}\hspace{0.2em}\textit{suffix}}'.
Surely, the same effect is achieved by
directly specifying the
argument `{\textit{dest}\hspace{0.2em}\textit{suffix}}'
in the first form.
However, that requires to set up a different file
for each child. With the alternative form of the command
all these files can have exactly the same content
which simplifies setting them up and maintaining them.

For example, the following file |draft.tex|
with a compilation flag |\version| as described in \secref{sec:flags}
compiles the main document as a draft:
%
\begin{center}
\begin{tabular}{l}
|\def\version{draft}|\\
|\input{childdoc.def}|\\
|\childdocforward{|\textit{main}|}|
\end{tabular}
\end{center}
%
Likewise, the following files |final|\textit{nn}|.tex|
compile the final version of the child document
|child|\textit{nn}|.tex|:
%
\begin{center}
\begin{tabular}{l}
|\def\version{final}|\\
|\input{childdoc.def}|\\
|\childdocforwardprefix{final}{child}|
\end{tabular}
\end{center}
%

Note that when several versions of a main file and/or of each child file
are to be generated, it may be convenient to set up a |Makefile| or
shell script to automatise the process.

%%%%%%%%%%%%%%%%%%%%%%%%%%%%%%%%%%%%%%%%%%%%%%%%%%%%%%%%%%%%%%%%%%%%%%%%%%%%%%%%
\subsection{Command Line Processing}
\label{sec:commandline}

The effect of redirection files can also be achieved by invoking
the \LaTeX{} compiler with a more elaborate command line.
Most conveniently this should be done as part
of a shell script or a |Makefile|.

When using \textsf{childdoc} in the main file, the following
command lines effectively perform a redirection
(note that depending on the shell being used,
backslashes may have to be doubled: `|\|' $\to$ `|\\|'):
%
\begin{center}
|... -jobname "|\textit{target}|" |\\|"|[\textit{flags}]%
|\input{childdoc.def}\childdocforward[|\textit{main}|]{|\textit{dest}|}"|
\end{center}
%
Here \textit{target} is the name of the output file,
\textit{main} is the name of the main file
and \textit{dest} is the name of the main or child file to be processed
(all filenames without extensions).
The optional argument \textit{main} can be omitted
if \textit{main} matches \textit{dest}.
Optionally, compilation \textit{flags} can be defined via |\def| commands.
This command line makes the \TeX{} engine believe
it is compiling the file \textit{target}
whose content is specified as the latter parameter.
The provided code then forwards the processing to
\textit{main} or \textit{dest} as described in \secref{sec:forward}.

%%%%%%%%%%%%%%%%%%%%%%%%%%%%%%%%%%%%%%%%%%%%%%%%%%%%%%%%%%%%%%%%%%%%%%%%%%%%%%%%
\subsection{Include by Input}
\label{sec:input}

Including child documents by |\include| has some restrictions by design.
Most notably, the content of a child document always occupies
its own set of pages; pages cannot be shared between child documents.
Usually, this behaviour makes perfect sense
because each child document contain an essential part of the document.
However, in some situations it may be desirable to compose
a document from a collection of parts
without having mandatory page breaks between then.
For this case, the package
provides a mechanism to include parts
by |\input| which can also be processed individually.
However, by construction this mechanism
requires manual handling of the content to be output.

%%%%%%%%%%%%%%%%%%%%%%%%%%%%%%%%%%%%%%%%
\DescribeMacro{\ifchilddocmanual}
The main file should be prepared as usual, see \secref{sec:include}.
However, the document body must make a distinction
between processing of an individual part and of the main document, e.g.:
%
\begin{center}
\begin{tabular}{l}
|\ifchilddocmanual|\\
|\input{\childdocname}|\\
|\||else|\\
\textit{document body with }|\input{|\textit{part}|}|\\
|\||fi|
\end{tabular}
\end{center}
%
The conditional |\ifchilddocmanual| is true whenever
a part to be included by |\input| is being compiled,
and the name of the part is stored in |\childdocname|.

%%%%%%%%%%%%%%%%%%%%%%%%%%%%%%%%%%%%%%%%
\DescribeMacro{\childdocby}
Each part to be included by |\input| should start with:
%
\begin{center}
\begin{tabular}{l}
|\input{childdoc.def}|\\
|\childdocby{|\textit{main}|}|\\
\end{tabular}
\end{center}
%
The directive |\childdocby| is similar to |\childdocof|
described in \secref{sec:include},
but the subsequent selection of content must be done manually.
To that end, both |\ifchilddoc| and |\ifchilddocmanual|
will be true upon processing of a part,
and the name of the part is stored in |\childdocname|.
Note that |\jobname| will be set to the filename of the current part
so that each part receives an individual |.aux| file
that does not interfere with the |.aux| file(s) of the main document.
This behaviour can be altered by the alternative form
|\childdocby[*]{|\textit{main}|}| (with a non-empty optional argument)
which uses the |.aux| file of the main document
by setting |\jobname| to \textit{main}.

%%%%%%%%%%%%%%%%%%%%%%%%%%%%%%%%%%%%%%%%%%%%%%%%%%%%%%%%%%%%%%%%%%%%%%%%%%%%%%%%
\subsection{Driver Development}
\label{sec:driver}

The \textsf{childdoc} mechanism can also be use for the development
of definition files such as \LaTeX{} styles or classes.
This case differs from the above setup with multiple parts
included by |\include| in that no |\includeonly| should be invoked.
This can be achieved by starting the include file
(before |\ProvidesPackage|) with:
%
\begin{center}
\begin{tabular}{l}
|\input{childdoc.def}|\\
|\childdocforward{|\textit{main}|}|\\
\end{tabular}
\end{center}
%
or alternatively with:
%
\begin{center}
\begin{tabular}{l}
|\input{childdoc.def}|\\
|\childdocby{|\textit{main}|}|\\
\end{tabular}
\end{center}
%
Both forms have slightly different effects as described above.
The main file is prepared as usual, see \secref{sec:include}.

%%%%%%%%%%%%%%%%%%%%%%%%%%%%%%%%%%%%%%%%%%%%%%%%%%%%%%%%%%%%%%%%%%%%%%%%%%%%%%%%
\subsection{Legacy Detection}
\label{sec:detection}

The directive |\childdocmain| in the main file can detect
whether the complete document or merely a child is to be compiled
even without using the directive |\childdocof|.
This method is deprecated because it is less robust
and there is no compelling reason to use it;
it is merely provided for backward compatibility
and it may be removed in future versions.

If the detection mechanism is to be used,
it is mandatory to correctly specify
the filename of the main file as the argument of |\childdocmain|:
%
\begin{center}
\begin{tabular}{l}
|\input{childdoc.def}|\\
|\childdocmain{|\textit{main}|}|\\
\end{tabular}
\end{center}
%
If |\jobname| does not match the argument \textit{main} of |\childdocmain|,
it is assumed that |\jobname| points to the child file to be compiled.
When using |\childdocmain| with the main file specified as argument,
it suffices to start a child file
with just |\input{|\textit{main}|}|
without loading of the package and using |\childdocof|.
If instead all processing is done
with the appropriate \textsf{childdoc} directives,
the argument of \textit{main} of |\childdocmain| can be empty.

An alternative version of the command line processing described
in \secref{sec:commandline} using the detection mechanism reads:
%
\begin{center}
|... -jobname "|\textit{target}|" "|[\textit{flags}]%
[|\def\jobname{|\textit{dest}|}|]|\input{|\textit{main}|}"|
\end{center}

%%%%%%%%%%%%%%%%%%%%%%%%%%%%%%%%%%%%%%%%%%%%%%%%%%%%%%%%%%%%%%%%%%%%%%%%%%%%%%%%
\subsection{Manual Code}
\label{sec:manual}

In case one cannot be certain whether the definitions file |childdoc.def|
is installed on the target \TeX{} distribution
and one prefers not to ship it,
it is conceivable to paste a few relevant commands into the sources.

To that end, drop all statements |\input{childdoc.def}|
and perform the replacements as outlined below.
Instead of |\childdocmain{|\textit{main}|}| add the following code
to the top of the main file:
%
\begin{center}
\begin{tabular}{l}
|\||ifdefined\childdocname\endinput\||fi\newif\ifchilddoc|\\
|\edef\childdocname{\scantokens\expandafter{\jobname\noexpand}}|\\
|\def\childdocmain{|\textit{main}|}\||ifx\childdocmain\childdocname\||else|\\
|\childdoctrue\includeonly{\childdocname}\let\jobname\childdocmain\||fi|\\
\end{tabular}
\end{center}
%
Instead of |\childdocof{|\textit{main}|}| just include the main file
at the top of each child file:
%
\begin{center}
|\input{|\textit{main}|}|
\end{center}
%
A simple redirection |\childdocforward{|\textit{dest}|}| is achieved by:
%
\begin{center}
|\def\jobname{|\textit{dest}|}\input{\jobname}|
\end{center}
%
The redirection with prefix
|\childdocforwardprefix[|\textit{prefix}|]{|\textit{dest}|}|
is accomplished by:
%
\begin{center}
\begin{tabular}{l}
|{\edef\jobname{\scantokens\expandafter{\jobname\noexpand}}|\\
|\def\redirectjob |\textit{prefix}|#1~~~{\gdef\jobname{|\textit{dest}|#1}}|\\
|\expandafter\redirectjob\jobname~~~}\input{\jobname}|
\end{tabular}
\end{center}

In an alternative approach,
child documents can be compiled by a specific command line
without additional code or specific definitions:
%
\begin{center}
|... -jobname "|\textit{target}|" "|[\textit{flags}]%
|\includeonly{|\textit{dest}|}\input{|\textit{main}|}"|
\end{center}
%

%%%%%%%%%%%%%%%%%%%%%%%%%%%%%%%%%%%%%%%%%%%%%%%%%%%%%%%%%%%%%%%%%%%%%%%%%%%%%%%%
%%%%%%%%%%%%%%%%%%%%%%%%%%%%%%%%%%%%%%%%%%%%%%%%%%%%%%%%%%%%%%%%%%%%%%%%%%%%%%%%
\section{Information}

%%%%%%%%%%%%%%%%%%%%%%%%%%%%%%%%%%%%%%%%%%%%%%%%%%%%%%%%%%%%%%%%%%%%%%%%%%%%%%%%
\subsection{Copyright}

Copyright \copyright{} 2017--2018 Niklas Beisert

This work may be distributed and/or modified under the
conditions of the \LaTeX{} Project Public License, either version 1.3
of this license or (at your option) any later version.
The latest version of this license is in
  \url{http://www.latex-project.org/lppl.txt}
and version 1.3 or later is part of all distributions of \LaTeX{}
version 2005/12/01 or later.

This work has the LPPL maintenance status `maintained'.

The Current Maintainer of this work is Niklas Beisert.

This work consists of the files |README.txt|, |childdoc.ins| and |childdoc.dtx|
as well as the derived files |childdoc.def|, |cdocsamp.tex|
with |cdocsch1.tex|, |cdocsch2.tex|, |cdocspt3.tex|, |cdocspt4.tex|,
|cdocsdrf.tex|, |cdocsfn1.tex|, |cdocsfn2.tex|
as well as |childdoc.pdf|.

%%%%%%%%%%%%%%%%%%%%%%%%%%%%%%%%%%%%%%%%%%%%%%%%%%%%%%%%%%%%%%%%%%%%%%%%%%%%%%%%
\subsection{Files and Installation}

The package consists of the files:
%
\begin{center}
\begin{tabular}{ll}
    |README.txt|   & readme file \\
    |childdoc.ins| & installation file \\
    |childdoc.dtx| & source file \\
    |childdoc.def| & definition file \\
    |cdocsamp.tex| & sample main file \\
    |cdocsch1.tex| & sample include file \\
    |cdocsch2.tex| & sample include file \\
    |cdocspt3.tex| & sample part file \\
    |cdocspt4.tex| & sample part file \\
    |cdocsdrf.tex| & sample redirection file \\
    |cdocsfn1.tex| & sample redirection file \\
    |cdocsfn2.tex| & sample redirection file \\
    |childdoc.pdf| & manual
\end{tabular}
\end{center}
%
The distribution consists of the files
|README.txt|, |childdoc.ins| and |childdoc.dtx|.
%
\begin{itemize}
\item
Run (pdf)\LaTeX{} on |childdoc.dtx|
to compile the manual |childdoc.pdf| (this file).
\item
Run \LaTeX{} on |childdoc.ins| to create the definitions file |childdoc.def|
and the sample |cdocsamp.tex| with include files
|cdocsch1.tex|, |cdocsch2.tex|, |cdocspt3.tex|, |cdocspt4.tex|,
|cdocsdrf.tex|, |cdocsfn1.tex|, |cdocsfn2.tex|.
Then copy the file |childdoc.def| to an appropriate directory of your \LaTeX{}
distribution, e.g.\ \textit{texmf-root}|/tex/latex/childdoc|.
\end{itemize}

%%%%%%%%%%%%%%%%%%%%%%%%%%%%%%%%%%%%%%%%%%%%%%%%%%%%%%%%%%%%%%%%%%%%%%%%%%%%%%%%
\subsection{Related CTAN Packages}

There are several other packages which offer a similar functionality:
%
\begin{itemize}
\item
The packages
\href{http://ctan.org/pkg/docmute}{\textsf{docmute}},
\href{http://ctan.org/pkg/includex}{\textsf{includex}} and
\href{http://ctan.org/pkg/standalone}{\textsf{standalone}}
provide commands to include only the document body of
a child file thus allowing both files to be compiled individually.
\item
The packages \href{http://ctan.org/pkg/subdocs}{\textsf{subdocs}}
and \href{http://ctan.org/pkg/subfiles}{\textsf{subfiles}}
provide structures in which the main and child documents can be
encapsulated and allowing them to be compiled individually.
The inclusion mechanism is different from the conventional |\include|.
\item
The package \href{http://ctan.org/pkg/combine}{\textsf{combine}}
is an elaborate solution to combine several documents into one.
\end{itemize}
%
See also the CTAN topic \href{http://ctan.org/topic/subdocs}{\textsf{subdocs}}
for further related packages.
The present package differs from the above solutions in that
a document structure constructed with the conventional |\include| mechanism
just needs two extra commands at the top of every file
such that all constituent files can be compiled individually.

%%%%%%%%%%%%%%%%%%%%%%%%%%%%%%%%%%%%%%%%%%%%%%%%%%%%%%%%%%%%%%%%%%%%%%%%%%%%%%%%
%\subsection{Feature Suggestions}
%
%The following is a list of features which may be useful for future
%versions of this package:
%%
%\begin{itemize}
%\item
%\ldots
%\end{itemize}

%%%%%%%%%%%%%%%%%%%%%%%%%%%%%%%%%%%%%%%%%%%%%%%%%%%%%%%%%%%%%%%%%%%%%%%%%%%%%%%%
\subsection{Revision History}

%%%%%%%%%%%%%%%%%%%%%%%%%%%%%%%%%%%%%%%%
\paragraph{v2.0:} 2018/12/30

\begin{itemize}
\item
immediate forward processing
\item
added |\childdocby| mechanism
\item
manual restructured
\end{itemize}

%%%%%%%%%%%%%%%%%%%%%%%%%%%%%%%%%%%%%%%%
\paragraph{v1.6:} 2018/01/17

\begin{itemize}
\item
application for development of include files
\item
corrections to manual
\end{itemize}

%%%%%%%%%%%%%%%%%%%%%%%%%%%%%%%%%%%%%%%%
\paragraph{v1.5:} 2017/05/21

\begin{itemize}
\item
more complete structuring introduced
\item
|\childdocof| introduced
\item
|\childdoc| renamed to |\childdocmain|
\item
|\childredirect| renamed to |\childdocforward| and |\childdocforwardprefix|
and functionality expanded
\end{itemize}

%%%%%%%%%%%%%%%%%%%%%%%%%%%%%%%%%%%%%%%%
\paragraph{v1.0:} 2017/04/27

\begin{itemize}
\item
manual and install package
\item
first version published on CTAN
\end{itemize}

%%%%%%%%%%%%%%%%%%%%%%%%%%%%%%%%%%%%%%%%
\paragraph{v0.6:} 2017/04/26

\begin{itemize}
\item
redirection mechanism added
\end{itemize}

%%%%%%%%%%%%%%%%%%%%%%%%%%%%%%%%%%%%%%%%
\paragraph{v0.5:} 2017/04/26

\begin{itemize}
\item
functionality in definition file
\end{itemize}


%%%%%%%%%%%%%%%%%%%%%%%%%%%%%%%%%%%%%%%%%%%%%%%%%%%%%%%%%%%%%%%%%%%%%%%%%%%%%%%%
%%%%%%%%%%%%%%%%%%%%%%%%%%%%%%%%%%%%%%%%%%%%%%%%%%%%%%%%%%%%%%%%%%%%%%%%%%%%%%%%
%%%%%%%%%%%%%%%%%%%%%%%%%%%%%%%%%%%%%%%%%%%%%%%%%%%%%%%%%%%%%%%%%%%%%%%%%%%%%%%%
\appendix

\settowidth\MacroIndent{\rmfamily\scriptsize 000\ }

 \DocInput{childdoc.dtx}

\end{document}
%</driver>
% \fi
%
% %%%%%%%%%%%%%%%%%%%%%%%%%%%%%%%%%%%%%%%%%%%%%%%%%%%%%%%%%%%%%%%%%%%%%%%%%%%%%%
% %%%%%%%%%%%%%%%%%%%%%%%%%%%%%%%%%%%%%%%%%%%%%%%%%%%%%%%%%%%%%%%%%%%%%%%%%%%%%%
% \section{Sample}
%\iffalse
%<*samplemain>
%\fi
%
% The following presents a sample document
% with two chapters, two parts, a title page,
% a compile flag as well as three forwarding files to set the flag.
% It consists of eight |.tex| files:
% \begin{center}
% \begin{tabular}{ll}
% |cdocsamp.tex|&main file\\
% |cdocsch1.tex|&include file for chapter 1\\
% |cdocsch2.tex|&include file for chapter 2\\
% |cdocspt3.tex|&include file for part 3\\
% |cdocspt4.tex|&include file for part 4\\
% |cdocsdrf.tex|&forwarding file for main file in draft mode\\
% |cdocsfi1.tex|&forwarding file for final version of chapter 1\\
% |cdocsfi2.tex|&forwarding file for final version of chapter 2\\
% \end{tabular}
% \end{center}
% Each of the eight files can be compiled directly by the \LaTeX{} compiler.
%
% %%%%%%%%%%%%%%%%%%%%%%%%%%%%%%%%%%%%%%
% \paragraph{Main File.}
%
% The main file is called |cdocsamp.tex|.
%
% Load the \textsf{childdoc} definitions and
% declare the filename for the main document:
%    \begin{macrocode}
\input{childdoc.def}
\childdocmain{}
%    \end{macrocode}

% Optional override for |\version| flag:
%    \begin{macrocode}
%%\ifchilddoc\else\providecommand{\version}{draft}\fi
%    \end{macrocode}

% Define the default values for the |\version| flag
% (|final| for the main file and |draft| for childs):
%    \begin{macrocode}
\ifchilddoc
\providecommand{\version}{draft}
\else
\providecommand{\version}{final}
\fi
%    \end{macrocode}

% Load the standard document class:
%    \begin{macrocode}
\documentclass[12pt]{article}
%    \end{macrocode}

% Start the document body:
%    \begin{macrocode}
\begin{document}
%    \end{macrocode}

% Declare a title page.
% Print title, part of document being processed and version flag:
%    \begin{macrocode}
\addtocounter{page}{-1}
\begin{center}
{\LARGE\bfseries{}childdoc example\par}
\vspace{1cm}
\ifchilddoc
\ifchilddocmanual part\else chapter\fi:
`\childdocname' of `\childdocjob'\par
\else
main document: `\childdocjob'\par
\fi
version: \version\par
\end{center}
\newpage
%    \end{macrocode}

% Manually include selected file,
% otherwise process as usual:
%    \begin{macrocode}
\ifchilddocmanual
\section*{part `\childdocname'}
\input{\childdocname}
\else
%    \end{macrocode}

% Include the two chapters:
%    \begin{macrocode}
\include{cdocsch1}
\include{cdocsch2}
%    \end{macrocode}

% Include the two parts unless only chapters should be displayed:
%    \begin{macrocode}
\ifchilddoc\else
\section{part three}
\input{cdocspt3}
\section{part four}
\input{cdocspt4}
\fi
%    \end{macrocode}

% Process as usual until here:
%    \begin{macrocode}
\fi
%    \end{macrocode}

% End of document body:
%    \begin{macrocode}
\end{document}
%    \end{macrocode}
%\iffalse
%</samplemain>
%\fi
%
% %%%%%%%%%%%%%%%%%%%%%%%%%%%%%%%%%%%%%%
% \paragraph{Chapter Include Files.}
%
% The include files are called |cdocsch1.tex| and |cdocsch2.tex|.
%
%\iffalse
%<*samplechap1|samplechap2>
%\fi

% Optional override for |\version| flag:
%    \begin{macrocode}
%%\providecommand{\version}{final}
%    \end{macrocode}

% Include the main document:
%    \begin{macrocode}
\input{childdoc.def}
\childdocof{cdocsamp}
%    \end{macrocode}

%\iffalse
%</samplechap1|samplechap2>
%\fi
%
%\iffalse
%<*samplechap1>
%\fi
% Some text for chapter 1:
%    \begin{macrocode}
\section{one}
some text in chapter one
%    \end{macrocode}

%\iffalse
%</samplechap1>
%\fi
% Some text for chapter 2:
%\iffalse
%<*samplechap2>
%\fi
%    \begin{macrocode}
\section{two}
more text in chapter two
%    \end{macrocode}

%\iffalse
%</samplechap2>
%\fi
%
% %%%%%%%%%%%%%%%%%%%%%%%%%%%%%%%%%%%%%%
% \paragraph{Part Include Files.}
%
% The include files are called |cdocspt3.tex| and |cdocspt4.tex|.
%
%\iffalse
%<*samplepart3|samplepart4>
%\fi

% Optional override for |\version| flag:
%    \begin{macrocode}
%%\providecommand{\version}{final}
%    \end{macrocode}

% Include the main document:
%    \begin{macrocode}
\input{childdoc.def}
\childdocby{cdocsamp}
%    \end{macrocode}

%\iffalse
%</samplepart3|samplepart4>
%\fi
%
%\iffalse
%<*samplepart3>
%\fi
% Some text for part 3:
%    \begin{macrocode}
some text in part three
%    \end{macrocode}

%\iffalse
%</samplepart3>
%\fi
% Some text for part 4:
%\iffalse
%<*samplepart4>
%\fi
%    \begin{macrocode}
more text in part four
%    \end{macrocode}

%\iffalse
%</samplepart4>
%\fi
%
% %%%%%%%%%%%%%%%%%%%%%%%%%%%%%%%%%%%%%%
% \paragraph{Forwarding for a Complete Draft.}
%
% The following forwarding file |cdocsdrf.tex|
% compiles the main document in draft mode:
%\iffalse
%<*sampledraft>
%\fi
%    \begin{macrocode}
\def\version{draft}
\input{childdoc.def}
\childdocforward{cdocsamp}
%    \end{macrocode}

%\iffalse
%</sampledraft>
%\fi
%
% %%%%%%%%%%%%%%%%%%%%%%%%%%%%%%%%%%%%%%
% \paragraph{Forwarding for Final Version of the Chapters.}
%
% The following forwarding files |cdocsfn1.tex| and |cdocsfn2.tex|
% (with identical content)
% compile the final versions of the child documents
% |cdocsch1.tex| and |cdocsch2.tex|, respectively:
%\iffalse
%<*samplefinal>
%\fi
%    \begin{macrocode}
\def\version{final}
\input{childdoc.def}
\childdocforwardprefix[cdocsamp]{cdocsfn}{cdocsch}
%    \end{macrocode}

%\iffalse
%</samplefinal>
%\fi
%
% %%%%%%%%%%%%%%%%%%%%%%%%%%%%%%%%%%%%%%
% \paragraph{Command Line Processing.}
%
% The following three command lines generate the output files
% |cdocscld|, |cdocscl1| and |cdocscl2|
% which should be identical to
% |cdocsdrf|, |cdocsch1| and |cdocsfn2|, respectively:
% \begin{center}
% \begin{tabular}{l}
% |latex -jobname cdocscld \|\\
% |  "\def\version{draft}\input{childdoc.def}\childdocforward{cdocsamp}"|\\
% |latex -jobname cdocscl1 \|\\
% |  "\input{childdoc.def}\childdocforward[cdocsamp]{cdocsch1}"|\\
% |latex -jobname cdocscl2 \|\\
% |  "\def\version{final}\input{childdoc.def}\childdocforward{cdocsch2}"|
% \end{tabular}
% \end{center}
% Note that the trailing backslash on each first line
% merely continues the input to the second line
% (for convenient cut ant paste).
% Furthermore, the command |latex| can be replaced by any
% of its alternative versions such as |pdflatex|.
%
% %%%%%%%%%%%%%%%%%%%%%%%%%%%%%%%%%%%%%%%%%%%%%%%%%%%%%%%%%%%%%%%%%%%%%%%%%%%%%%
% %%%%%%%%%%%%%%%%%%%%%%%%%%%%%%%%%%%%%%%%%%%%%%%%%%%%%%%%%%%%%%%%%%%%%%%%%%%%%%
% \section{Implementation}
%\iffalse
%<*package>
%\fi
%
% This section describes the definitions file |childdoc.def|.

% The definitions cannot be loaded using |\usepackage| or |\RequirePackage|
% which has a mechanism to prevent loading a style file more than once.
% When loading the definitions by means of |\input|
% multiple instances have to be prevented manually:
%\iffalse
%This code needs to be before the `\ProvidesFile' directive
%which is defined at the beginning of this file.
%Therefore it is also placed there and commented out here.
%</package>
%<*discard>
%\fi
%    \begin{macrocode}
\ifdefined\childdocmain\endinput\fi
%    \end{macrocode}
%\iffalse
%</discard>
%<*package>
%\fi
%
% \macro{\ifchilddoc}
% \macro{\ifchilddocmanual}
% The conditional |\ifchilddoc| tells whether a
% child (true) or main (false) document is being compiled.
% The conditional |\ifchilddocmanual| tells whether
% the |\includeonly| mechanism is used (false) or
% the selection of child files must be performed manually (true).
% The definitions initialise to false:
%    \begin{macrocode}
\newif\ifchilddoc
\newif\ifchilddocmanual
%    \end{macrocode}

% \macro{\childdocname}
% \macro{\childdocjob}
% The macro |\childdocname| stores the name of the main document
% to be compiled. The macro |\childdocjob| stores the name of
% the document on which the \LaTeX{} compiler was originally invoked.
% The content of |\jobname| cannot be compared
% to filenames specified in the source due to different catcodes.
% The following code rescans |\jobname|, stores the result
% in |\childdocname| and saves a copy in |\childdocjob|:
%    \begin{macrocode}
\edef\childdocname{\scantokens\expandafter{\jobname\noexpand}}
\let\childdocjob\childdocname
%    \end{macrocode}

% \macro{\childdocdisable}
% The macro |\childdocdisable| prevents the main file
% from being processed more than once.
% At this stage, the main document command |\childdocmain|
% is assumed to be called once again where it should do nothing.
% Any subsequent call to it should prevent
% a secondary processing of the main document
% It overwrites the forwarding commands
% |\childdocof| and |\childdocforward|
% with empty macros to prevent further inclusions of the main document:
%    \begin{macrocode}
\newcommand{\childdocdisable}
{
  \renewcommand{\childdocmain}[1]{\renewcommand{\childdocmain}[1]{\endinput}}
  \renewcommand{\childdocof}[1]{}
  \renewcommand{\childdocby}[2][]{}
  \renewcommand{\childdocforward}[2][]{}
  \renewcommand{\childdocdisable}{}
}
%    \end{macrocode}

% \macro{\childdocmain}
% The macro |\childdocmain| is to be called at the top of the main file
% with nothing or the main filename (without extension) as argument.
% First, it breaks loops.
% If the argument is not empty and does not match |\childdocname|
% (which is set by the first inclusion of |childdoc.def|),
% |\ifchilddoc| is set to true, |\includeonly| is applied to the child file
% and |\jobname| is set to the main file
% (for proper handling of |.aux| files):
%    \begin{macrocode}
\newcommand{\childdocmain}[1]
{
  \childdocdisable\childdocmain{}
  \if?#1?\else
    \begingroup
      \def\childdoctmp{#1}
      \ifx\childdoctmp\childdocname
        \def\childdoctmp{}
      \else
        \def\childdoctmp
        {
          \childdoctrue
          \includeonly{\childdocname}
          \def\childdocjob{#1}
          \def\jobname{#1}
        }
      \fi
      \expandafter
    \endgroup
    \childdoctmp
  \fi
}
%    \end{macrocode}

% \macro{\childdocof}
% The command |\childdocof| redirects
% compilation to the main file |#1|.
%    \begin{macrocode}
\newcommand{\childdocof}[1]
{
  \childdocdisable
  \childdoctrue
  \includeonly{\childdocname}
  \def\jobname{#1}
  \def\childdocjob{#1}
  \input{#1}
}
%    \end{macrocode}

% \macro{\childdocby}
% The command |\childdocby| ....
%    \begin{macrocode}
\newcommand{\childdocby}[2][]
{
  \childdocdisable
  \childdoctrue
  \childdocmanualtrue
  \if?#1?\else
    \def\jobname{#2}
  \fi
  \def\childdocjob{#2}
  \input{#2}
  \endinput
}
%    \end{macrocode}

% \macro{\childdocforward}
% The command |\childdocforward| redirects
% compilation to the main file or
% (if the optional argument is given) a child file.
% Parameters are set as if the main file
% or a child file starting with |\childdocof| was compiled.
% Then compilation is handed over to the main file:
%    \begin{macrocode}
\newcommand{\childdocforward}[2][]
{
  \begingroup
    \if?#1?
      \def\childdoctmp
      {
        \def\childdocname{#2}
        \def\childdocjob{#2}
        \def\jobname{#2}
        \input{#2}
        \endinput
      }
    \else
      \def\childdoctmp
      {
        \childdocdisable
        \def\childdocname{#2}
        \childdoctrue
        \includeonly{#2}
        \def\childdocjob{#1}
        \def\jobname{#1}
        \input{#1}
        \endinput
      }
    \fi
    \expandafter
  \endgroup
  \childdoctmp
}
%    \end{macrocode}

% \macro{\childdocforwardprefix}
% The command |\childdocforwardprefix| redirects
% compilation to the main or a child file by means of a pattern.
% The prefix |#1| in the current filename is replaced by |#2|
% and the suffix of the current filename is kept
% (it is assumed that the filename does not contain the substring `|~~~|'
% which is used as a delimiter).
% Compilation is handed over to the new file by |\childdocforward|:
%    \begin{macrocode}
\newcommand{\childdocforwardprefix}[3][]
{
  \begingroup
    \def\childdocextract #2##1~~~{\def\childdoctmp{\childdocforward[#1]{#3##1}}}
    \expandafter\childdocextract\childdocname~~~
    \expandafter
  \endgroup
  \childdoctmp
}
%    \end{macrocode}

% \macro{\childdoc}
% The deprecated macro |\childdoc| is a legacy version of |\childdocmain|:
%    \begin{macrocode}
\newcommand{\childdoc}{\childdocmain}
%    \end{macrocode}

% \macro{\childdocredirect}
% The deprecated macro |\childdocredirect| is a legacy version
% of |\childdocforward| and |\childdocforwardprefix|:
%    \begin{macrocode}
\newcommand{\childdocredirect}[2][]
{
  \begingroup
    \if?#1?
      \def\childdoctmp{\childdocforward{#2}}
    \else
      \def\childdoctmp{\childdocforwardprefix{#1}{#2}}
    \fi
    \expandafter
  \endgroup
  \childdoctmp
}
%    \end{macrocode}

%\iffalse
%</package>
%\fi
%
\endinput
\childdocforward{cdocsamp}"|\\
% |latex -jobname cdocscl1 \|\\
% |  "% \iffalse
%
% childdoc.dtx Copyright (C) 2017-2018 Niklas Beisert
%
% This work may be distributed and/or modified under the
% conditions of the LaTeX Project Public License, either version 1.3
% of this license or (at your option) any later version.
% The latest version of this license is in
%   http://www.latex-project.org/lppl.txt
% and version 1.3 or later is part of all distributions of LaTeX
% version 2005/12/01 or later.
%
% This work has the LPPL maintenance status `maintained'.
%
% The Current Maintainer of this work is Niklas Beisert.
%
% This work consists of the files childdoc.dtx and childdoc.ins
% and the derived files childdoc.def and cdocsamp.tex with
% cdocsch1.tex, cdocsch2.tex, cdocsdrf.tex, cdocsfn1.tex, cdocsfn2.tex.
%
%<package>\ifdefined\childdocmain\endinput\fi
%<package>\ProvidesFile{childdoc.def}[2018/12/30 v2.0 child document driver]
%<samplemain>\ProvidesFile{cdocsamp.tex}[2018/12/30 v2.0 sample for childdoc]
%<*driver>
%\ProvidesFile{childdoc.drv}[2018/12/30 v2.0 childdoc reference manual file]
\PassOptionsToClass{10pt,a4paper}{article}
\documentclass{ltxdoc}

\usepackage[margin=35mm]{geometry}
\usepackage{hyperref}
\usepackage{hyperxmp}
\usepackage[usenames]{color}

\hypersetup{colorlinks=true}
\hypersetup{pdfstartview=FitH}
\hypersetup{pdfpagemode=UseNone}
\hypersetup{pdfsource={}}
\hypersetup{pdflang={en-UK}}
\hypersetup{pdfcopyright={Copyright 2017-2018 Niklas Beisert.
  This work may be distributed and/or modified under the
  conditions of the LaTeX Project Public License, either version 1.3
  of this license or (at your option) any later version.}}
\hypersetup{pdflicenseurl={http://www.latex-project.org/lppl.txt}}
\hypersetup{pdfcontactaddress={ETH Zurich, ITP, HIT K,
  Wolfgang-Pauli-Strasse 27}}
\hypersetup{pdfcontactpostcode={8093}}
\hypersetup{pdfcontactcity={Zurich}}
\hypersetup{pdfcontactcountry={Switzerland}}
\hypersetup{pdfcontactemail={nbeisert@itp.phys.ethz.ch}}
\hypersetup{pdfcontacturl={http://people.phys.ethz.ch/\xmptilde nbeisert/}}

\newcommand{\secref}[1]{\hyperref[#1]{section \ref*{#1}}}

\parskip1ex
\parindent0pt
\let\olditemize\itemize
\def\itemize{\olditemize\parskip0pt}

\begin{document}

\title{The \textsf{childdoc} Package}
\hypersetup{pdftitle={The childdoc Package}}
\author{Niklas Beisert\\[2ex]
  Institut f\"ur Theoretische Physik\\
  Eidgen\"ossische Technische Hochschule Z\"urich\\
  Wolfgang-Pauli-Strasse 27, 8093 Z\"urich, Switzerland\\[1ex]
  \href{mailto:nbeisert@itp.phys.ethz.ch}
  {\texttt{nbeisert@itp.phys.ethz.ch}}}
\hypersetup{pdfauthor={Niklas Beisert}}
\hypersetup{pdfsubject={Manual for the LaTeX2e Package childdoc}}
\date{30 December 2018, \textsf{v2.0}}
\maketitle

\begin{abstract}\noindent
\textsf{childdoc} is a \LaTeXe{} package
that enables the direct compilation
of document sections included by |\include|
to individual files.
\end{abstract}

\begingroup
\parskip0ex
\tableofcontents
\endgroup

%%%%%%%%%%%%%%%%%%%%%%%%%%%%%%%%%%%%%%%%%%%%%%%%%%%%%%%%%%%%%%%%%%%%%%%%%%%%%%%%
%%%%%%%%%%%%%%%%%%%%%%%%%%%%%%%%%%%%%%%%%%%%%%%%%%%%%%%%%%%%%%%%%%%%%%%%%%%%%%%%
\section{Introduction}

\LaTeX{} provides a mechanism to structure a large document (such as a book)
into a main file and several child files (containing the chapters)
using the |\include| command.
This mechanism is beneficial for documents
which span hundreds of pages in order to
make the source file(s) more manageable.
Moreover, compilation can be restricted to
selected child files by means of the |\includeonly| command.
The latter feature can be used to reduce the compilation time while editing
(this was significantly more useful in the earlier days of \LaTeX{})
or to generate a smaller document which is easier to navigate.
Another application of |\includeonly| is to generate
documents consisting of selected parts of the complete document.

However, there are a few drawbacks of the plain |\include| mechanism:
\begin{itemize}
\item
The child files cannot be compiled on their own,
they can only be compiled via the main file.
A naive editing environment
(such as a text editor with an option
to have the current file processed by \LaTeX)
may require one to switch to the main file before compiling;
attempting to compile the child file produces errors.
\item
The main file must be modified (each time)
to adjust the |\includeonly| command
to the present needs. This easily leaves the main file in a messy state.
\item
The generated document will always carry the filename
of the main document. This is inconvenient if
several child files are to be compiled and
to be kept for distribution.
\end{itemize}

The present package provides a simple interface
to make child files individually compilable by \LaTeX{}.
Compiling a child file then has the same effect as compiling
the main file with an |\includeonly| command
to select the appropriate child.
Moreover the generated document will carry the name of the child
rather than the main file.
This resolves all three above issues.

This feature is meant to make the editing of books,
thesis documents and lecture notes somewhat more convenient.
However, the package can also be used efficiently for
composing a series of documents (such as exercise sheets)
which are typically distributed individually.
It then assists the author in generating the individual documents
(potentially in different versions)
as well as a document containing the collected series.
Another application is in developing style files
or other kinds of included material
where compilation of the style file could redirect
to a sample or test file.

%%%%%%%%%%%%%%%%%%%%%%%%%%%%%%%%%%%%%%%%%%%%%%%%%%%%%%%%%%%%%%%%%%%%%%%%%%%%%%%%
%%%%%%%%%%%%%%%%%%%%%%%%%%%%%%%%%%%%%%%%%%%%%%%%%%%%%%%%%%%%%%%%%%%%%%%%%%%%%%%%
\section{Usage}

First of all, the package \textsf{childdoc} is \emph{not} a standard
\LaTeXe{} |.sty| style file! Therefore it needs to be invoked in
a non-standard way.

%%%%%%%%%%%%%%%%%%%%%%%%%%%%%%%%%%%%%%%%%%%%%%%%%%%%%%%%%%%%%%%%%%%%%%%%%%%%%%%%
\subsection{Included Files}
\label{sec:include}

%%%%%%%%%%%%%%%%%%%%%%%%%%%%%%%%%%%%%%%%
\DescribeMacro{\childdocmain}
To use the package, add the commands
\begin{center}
\begin{tabular}{l}
|\input{childdoc.def}|\\
|\childdocmain{}|\\
\end{tabular}
\end{center}
at the very top of the main \LaTeX{} file,
in particular \emph{before} the |\documentclass| statement!
The argument of |\childdocmain| should be left empty
(but it must be present).

%%%%%%%%%%%%%%%%%%%%%%%%%%%%%%%%%%%%%%%%
\DescribeMacro{\childdocof}
Furthermore, add the commands
\begin{center}
\begin{tabular}{l}
|\input{childdoc.def}|\\
|\childdocof{|\textit{main}|}|\\
\end{tabular}
\end{center}
at the top of every child file \textit{child}
which is included by |\include{|\textit{child}|}|
from within the main file
(or at least for those files to be compiled individually).
The argument \textit{main} must be the filename of the main file.

There are a couple of
considerations in setting up the main and child documents:

%%%%%%%%%%%%%%%%%%%%%%%%%%%%%%%%%%%%%%%%
\paragraph{Restrictions.}

Please note the following restrictions:
\begin{itemize}
\item
|\childdocmain| must be called with one argument \textit{main}
to ensure compatibility with earlier version of the package.
It must either be empty (|\childdocmain{}|)
or precisely match the filename of the main file in which it is specified.
See \secref{sec:detection} for further information.
\item
The filename \textit{main} must be specified without the |.tex| extension.
\item
The filename \textit{main} is case sensitive
(even in case-insensitive file systems)
due to internal string comparison.
\item
The argument \textit{main} should be fully expanded, it cannot be a macro.
\item
Subdirectories and special characters should be avoided in filenames.
\item
The command |\childdocmain{|\textit{main}|}| must be followed by a whitespace.
It should not be followed immediately by another command
or by a comment mark `|%|'.
This is because the \TeX{} parser reads the token immediately following
the argument of |\childdocmain| and puts it
at the beginning of every child section;
however, a white\-space is ignored.
\end{itemize}

%%%%%%%%%%%%%%%%%%%%%%%%%%%%%%%%%%%%%%%%
\paragraph{Content of Main File.}

It is advisable to place all content in the child files included by |\include|.
Any output contained in the main file will appear in all child documents
unless suppressed manually;
it cannot be suppressed automatically by the |\includeonly| directive
and thus should normally be avoided.
A method to include some content in the main file
by means of conditional processing is described in \secref{sec:conditional}.

%%%%%%%%%%%%%%%%%%%%%%%%%%%%%%%%%%%%%%%%
\paragraph{Page Numbering.}

When only a part of the document is compiled,
the appropriate numbering of pages
(as well as other status parameters)
is determined from the |.aux| files.
The latter contain information from previous passes.
However this information needs to propagate through
all intermediate child documents.
Therefore the page numbering in child documents may well
be inconsistent until the complete document is compiled at least once.

A useful (if unconventional) way to always ensure a consistent
page numbering is to restart the numbering in each child document
and denote the pages by `\textit{child}|.|\textit{page}'
where \textit{child} represents the chapter/section number of the child file.
This can be achieved by the command
|\numberwithin{page}{|\textit{child}|}|
of the \textsf{amsmath} package
where \textit{child} can be |chapter| or |section|
depending on the chosen structuring.
Alternatively, one can modify the macro |\thepage| appropriately
and reset the counter |page| at the start of each child file.

%%%%%%%%%%%%%%%%%%%%%%%%%%%%%%%%%%%%%%%%%%%%%%%%%%%%%%%%%%%%%%%%%%%%%%%%%%%%%%%%
\subsection{Conditional Processing}
\label{sec:conditional}

The package provides a mechanism to compile different versions
of a document. To customise the versions further some conditional processing
can come in handy to distinguish which version is being compiled.
The package provides two macros to describe the compilation context:

%%%%%%%%%%%%%%%%%%%%%%%%%%%%%%%%%%%%%%%%
\DescribeMacro{\ifchilddoc}
The conditional |\ifchilddoc| distinguishes between the compilation of
child documents and the main document:
%
\begin{center}
|\ifchilddoc |\textit{child-code}| |[|\||else |\textit{main-code}]| \||fi|
\end{center}

%%%%%%%%%%%%%%%%%%%%%%%%%%%%%%%%%%%%%%%%
\DescribeMacro{\childdocname}
\DescribeMacro{\childdocjob}
The macro |\childdocname| contains the filename (without extension)
of the main or child file being processed.
Note that |\childdocjob| will always contain the name of the main file.

%%%%%%%%%%%%%%%%%%%%%%%%%%%%%%%%%%%%%%%%
\paragraph{Title Page.}

Conditional processing can be used to include a title or banner page
in the main document when proper precautions are taken.
Importantly, the code in the main file should ensure that the page counter
(as well as other status parameters which are stored in the |.aux| files)
takes the same value after the conditional processing.
Otherwise the page numbers may take divergent values
depending on which part is compiled.

For example, a title page could be declared by:
%
\begin{center}
\begin{tabular}{l}
|\ifchilddoc\||else|\\
|\addtocounter{page}{-1}|\\
\textit{code for title page}\\
|\newpage|\\
|\||fi|
\end{tabular}
\end{center}
%
A banner page for the child documents can be generated by:
%
\begin{center}
\begin{tabular}{l}
|\ifchilddoc|\\
|\addtocounter{page}{-1}|\\
\textit{code for banner page}\\
|\newpage|\\
|\||fi|
\end{tabular}
\end{center}
%
Here one could write a message such as:
\begin{center}
|This is the part \childdocname{} of \childdocjob{}.|
\end{center}

%%%%%%%%%%%%%%%%%%%%%%%%%%%%%%%%%%%%%%%%%%%%%%%%%%%%%%%%%%%%%%%%%%%%%%%%%%%%%%%%
\subsection{Flags}
\label{sec:flags}

The package makes it easy to generate different versions
of the main or child documents.
To this end compilation flags can be defined
and assigned different default values.
They will be particularly useful in conjunction
with the forwarding mechanism described in \secref{sec:forward}.

For example, it may be useful to have a flag |\version|
which can be set to |draft| or |final|.
The document source will contain some conditional code
depending on the value of |\version|.
Suppose further, the flag should default to |final| for the main file
and to |draft| for child files
which is a natural assignment for editing the document.
This is achieved by placing the following code
in the preamble of the main document
(below the |\childdocmain| directive):
%
\begin{center}
\begin{tabular}{l}
|\ifchilddoc|\\
|\providecommand{\version}{draft}|\\
|\||else|\\
|\providecommand{\version}{final}|\\
|\||fi|
\end{tabular}
\end{center}
%
The definition by |\providecommand| makes sure
that previous definitions are not overwritten.
Further statements |\providecommand{\version}{...}|
can thus be added before the above code to override it.

For the main file, one might add a line
(between |\childdocmain| and the above block)
%
\begin{center}
|%\ifchilddoc\||else\providecommand{\version}{draft}\||fi|
\end{center}
%
which can be uncommented to produce a draft version.
Likewise one can add a line to the very top of a child file
(above the |\childdocof{|\textit{main}|}| directive)
%
\begin{center}
|%\providecommand{\version}{final}|
\end{center}
%
which can be uncommented to produce the final version of this child document.

%%%%%%%%%%%%%%%%%%%%%%%%%%%%%%%%%%%%%%%%%%%%%%%%%%%%%%%%%%%%%%%%%%%%%%%%%%%%%%%%
\subsection{Forwarding}
\label{sec:forward}

Different versions of the main or child documents
using compilation flags as described in \secref{sec:flags}
can be (permanently) stored in different files
for convenient compilation, viewing and distribution.
To this end, the package defines a command
to pass on compilation to a different file:

%%%%%%%%%%%%%%%%%%%%%%%%%%%%%%%%%%%%%%%%
\DescribeMacro{\childdocforward}
The command |\childdocforward| redirects processing to
another source file:
%
\begin{center}
\begin{tabular}{l}
|\input{childdoc.def}|\\
|\childdocforward[|\textit{main}|]{|\textit{dest}|}|\\
\end{tabular}
\end{center}
%
The argument \textit{dest} is the destination file
(without extension).
It should be the main file or one of the child files.
Note that further \textsf{childdoc} directives
such as |\childdocof| and |\childdocforward|
in the indicated file will be processed in this form.
The optional argument \textit{main}
passes on directly to the main file \textit{main}
while pretending to compile the child \textit{dest}.
This form behaves as if \textit{dest}
issues |\childdocof{|\textit{main}|}| right away,
and no further \textsf{childdoc} directives will be processed.

%%%%%%%%%%%%%%%%%%%%%%%%%%%%%%%%%%%%%%%%
\DescribeMacro{\...prefix}
In the alternative form |\childdocforwardprefix|,
%
\begin{center}
\begin{tabular}{l}
|\input{childdoc.def}|\\
|\childdocforwardprefix[|\textit{main}|]{|\textit{prefix}|}{|\textit{dest}|}|
\end{tabular}
\end{center}
%
the destination file is determined by a pattern
depending on the current file:
To make this work, the current file must be called
`{\textit{prefix}\hspace{0.2em}\textit{suffix}}'
with \textit{prefix} matching precisely the argument.
Processing is then passed on to the file
`{\textit{dest}\hspace{0.2em}\textit{suffix}}'.
Surely, the same effect is achieved by
directly specifying the
argument `{\textit{dest}\hspace{0.2em}\textit{suffix}}'
in the first form.
However, that requires to set up a different file
for each child. With the alternative form of the command
all these files can have exactly the same content
which simplifies setting them up and maintaining them.

For example, the following file |draft.tex|
with a compilation flag |\version| as described in \secref{sec:flags}
compiles the main document as a draft:
%
\begin{center}
\begin{tabular}{l}
|\def\version{draft}|\\
|\input{childdoc.def}|\\
|\childdocforward{|\textit{main}|}|
\end{tabular}
\end{center}
%
Likewise, the following files |final|\textit{nn}|.tex|
compile the final version of the child document
|child|\textit{nn}|.tex|:
%
\begin{center}
\begin{tabular}{l}
|\def\version{final}|\\
|\input{childdoc.def}|\\
|\childdocforwardprefix{final}{child}|
\end{tabular}
\end{center}
%

Note that when several versions of a main file and/or of each child file
are to be generated, it may be convenient to set up a |Makefile| or
shell script to automatise the process.

%%%%%%%%%%%%%%%%%%%%%%%%%%%%%%%%%%%%%%%%%%%%%%%%%%%%%%%%%%%%%%%%%%%%%%%%%%%%%%%%
\subsection{Command Line Processing}
\label{sec:commandline}

The effect of redirection files can also be achieved by invoking
the \LaTeX{} compiler with a more elaborate command line.
Most conveniently this should be done as part
of a shell script or a |Makefile|.

When using \textsf{childdoc} in the main file, the following
command lines effectively perform a redirection
(note that depending on the shell being used,
backslashes may have to be doubled: `|\|' $\to$ `|\\|'):
%
\begin{center}
|... -jobname "|\textit{target}|" |\\|"|[\textit{flags}]%
|\input{childdoc.def}\childdocforward[|\textit{main}|]{|\textit{dest}|}"|
\end{center}
%
Here \textit{target} is the name of the output file,
\textit{main} is the name of the main file
and \textit{dest} is the name of the main or child file to be processed
(all filenames without extensions).
The optional argument \textit{main} can be omitted
if \textit{main} matches \textit{dest}.
Optionally, compilation \textit{flags} can be defined via |\def| commands.
This command line makes the \TeX{} engine believe
it is compiling the file \textit{target}
whose content is specified as the latter parameter.
The provided code then forwards the processing to
\textit{main} or \textit{dest} as described in \secref{sec:forward}.

%%%%%%%%%%%%%%%%%%%%%%%%%%%%%%%%%%%%%%%%%%%%%%%%%%%%%%%%%%%%%%%%%%%%%%%%%%%%%%%%
\subsection{Include by Input}
\label{sec:input}

Including child documents by |\include| has some restrictions by design.
Most notably, the content of a child document always occupies
its own set of pages; pages cannot be shared between child documents.
Usually, this behaviour makes perfect sense
because each child document contain an essential part of the document.
However, in some situations it may be desirable to compose
a document from a collection of parts
without having mandatory page breaks between then.
For this case, the package
provides a mechanism to include parts
by |\input| which can also be processed individually.
However, by construction this mechanism
requires manual handling of the content to be output.

%%%%%%%%%%%%%%%%%%%%%%%%%%%%%%%%%%%%%%%%
\DescribeMacro{\ifchilddocmanual}
The main file should be prepared as usual, see \secref{sec:include}.
However, the document body must make a distinction
between processing of an individual part and of the main document, e.g.:
%
\begin{center}
\begin{tabular}{l}
|\ifchilddocmanual|\\
|\input{\childdocname}|\\
|\||else|\\
\textit{document body with }|\input{|\textit{part}|}|\\
|\||fi|
\end{tabular}
\end{center}
%
The conditional |\ifchilddocmanual| is true whenever
a part to be included by |\input| is being compiled,
and the name of the part is stored in |\childdocname|.

%%%%%%%%%%%%%%%%%%%%%%%%%%%%%%%%%%%%%%%%
\DescribeMacro{\childdocby}
Each part to be included by |\input| should start with:
%
\begin{center}
\begin{tabular}{l}
|\input{childdoc.def}|\\
|\childdocby{|\textit{main}|}|\\
\end{tabular}
\end{center}
%
The directive |\childdocby| is similar to |\childdocof|
described in \secref{sec:include},
but the subsequent selection of content must be done manually.
To that end, both |\ifchilddoc| and |\ifchilddocmanual|
will be true upon processing of a part,
and the name of the part is stored in |\childdocname|.
Note that |\jobname| will be set to the filename of the current part
so that each part receives an individual |.aux| file
that does not interfere with the |.aux| file(s) of the main document.
This behaviour can be altered by the alternative form
|\childdocby[*]{|\textit{main}|}| (with a non-empty optional argument)
which uses the |.aux| file of the main document
by setting |\jobname| to \textit{main}.

%%%%%%%%%%%%%%%%%%%%%%%%%%%%%%%%%%%%%%%%%%%%%%%%%%%%%%%%%%%%%%%%%%%%%%%%%%%%%%%%
\subsection{Driver Development}
\label{sec:driver}

The \textsf{childdoc} mechanism can also be use for the development
of definition files such as \LaTeX{} styles or classes.
This case differs from the above setup with multiple parts
included by |\include| in that no |\includeonly| should be invoked.
This can be achieved by starting the include file
(before |\ProvidesPackage|) with:
%
\begin{center}
\begin{tabular}{l}
|\input{childdoc.def}|\\
|\childdocforward{|\textit{main}|}|\\
\end{tabular}
\end{center}
%
or alternatively with:
%
\begin{center}
\begin{tabular}{l}
|\input{childdoc.def}|\\
|\childdocby{|\textit{main}|}|\\
\end{tabular}
\end{center}
%
Both forms have slightly different effects as described above.
The main file is prepared as usual, see \secref{sec:include}.

%%%%%%%%%%%%%%%%%%%%%%%%%%%%%%%%%%%%%%%%%%%%%%%%%%%%%%%%%%%%%%%%%%%%%%%%%%%%%%%%
\subsection{Legacy Detection}
\label{sec:detection}

The directive |\childdocmain| in the main file can detect
whether the complete document or merely a child is to be compiled
even without using the directive |\childdocof|.
This method is deprecated because it is less robust
and there is no compelling reason to use it;
it is merely provided for backward compatibility
and it may be removed in future versions.

If the detection mechanism is to be used,
it is mandatory to correctly specify
the filename of the main file as the argument of |\childdocmain|:
%
\begin{center}
\begin{tabular}{l}
|\input{childdoc.def}|\\
|\childdocmain{|\textit{main}|}|\\
\end{tabular}
\end{center}
%
If |\jobname| does not match the argument \textit{main} of |\childdocmain|,
it is assumed that |\jobname| points to the child file to be compiled.
When using |\childdocmain| with the main file specified as argument,
it suffices to start a child file
with just |\input{|\textit{main}|}|
without loading of the package and using |\childdocof|.
If instead all processing is done
with the appropriate \textsf{childdoc} directives,
the argument of \textit{main} of |\childdocmain| can be empty.

An alternative version of the command line processing described
in \secref{sec:commandline} using the detection mechanism reads:
%
\begin{center}
|... -jobname "|\textit{target}|" "|[\textit{flags}]%
[|\def\jobname{|\textit{dest}|}|]|\input{|\textit{main}|}"|
\end{center}

%%%%%%%%%%%%%%%%%%%%%%%%%%%%%%%%%%%%%%%%%%%%%%%%%%%%%%%%%%%%%%%%%%%%%%%%%%%%%%%%
\subsection{Manual Code}
\label{sec:manual}

In case one cannot be certain whether the definitions file |childdoc.def|
is installed on the target \TeX{} distribution
and one prefers not to ship it,
it is conceivable to paste a few relevant commands into the sources.

To that end, drop all statements |\input{childdoc.def}|
and perform the replacements as outlined below.
Instead of |\childdocmain{|\textit{main}|}| add the following code
to the top of the main file:
%
\begin{center}
\begin{tabular}{l}
|\||ifdefined\childdocname\endinput\||fi\newif\ifchilddoc|\\
|\edef\childdocname{\scantokens\expandafter{\jobname\noexpand}}|\\
|\def\childdocmain{|\textit{main}|}\||ifx\childdocmain\childdocname\||else|\\
|\childdoctrue\includeonly{\childdocname}\let\jobname\childdocmain\||fi|\\
\end{tabular}
\end{center}
%
Instead of |\childdocof{|\textit{main}|}| just include the main file
at the top of each child file:
%
\begin{center}
|\input{|\textit{main}|}|
\end{center}
%
A simple redirection |\childdocforward{|\textit{dest}|}| is achieved by:
%
\begin{center}
|\def\jobname{|\textit{dest}|}\input{\jobname}|
\end{center}
%
The redirection with prefix
|\childdocforwardprefix[|\textit{prefix}|]{|\textit{dest}|}|
is accomplished by:
%
\begin{center}
\begin{tabular}{l}
|{\edef\jobname{\scantokens\expandafter{\jobname\noexpand}}|\\
|\def\redirectjob |\textit{prefix}|#1~~~{\gdef\jobname{|\textit{dest}|#1}}|\\
|\expandafter\redirectjob\jobname~~~}\input{\jobname}|
\end{tabular}
\end{center}

In an alternative approach,
child documents can be compiled by a specific command line
without additional code or specific definitions:
%
\begin{center}
|... -jobname "|\textit{target}|" "|[\textit{flags}]%
|\includeonly{|\textit{dest}|}\input{|\textit{main}|}"|
\end{center}
%

%%%%%%%%%%%%%%%%%%%%%%%%%%%%%%%%%%%%%%%%%%%%%%%%%%%%%%%%%%%%%%%%%%%%%%%%%%%%%%%%
%%%%%%%%%%%%%%%%%%%%%%%%%%%%%%%%%%%%%%%%%%%%%%%%%%%%%%%%%%%%%%%%%%%%%%%%%%%%%%%%
\section{Information}

%%%%%%%%%%%%%%%%%%%%%%%%%%%%%%%%%%%%%%%%%%%%%%%%%%%%%%%%%%%%%%%%%%%%%%%%%%%%%%%%
\subsection{Copyright}

Copyright \copyright{} 2017--2018 Niklas Beisert

This work may be distributed and/or modified under the
conditions of the \LaTeX{} Project Public License, either version 1.3
of this license or (at your option) any later version.
The latest version of this license is in
  \url{http://www.latex-project.org/lppl.txt}
and version 1.3 or later is part of all distributions of \LaTeX{}
version 2005/12/01 or later.

This work has the LPPL maintenance status `maintained'.

The Current Maintainer of this work is Niklas Beisert.

This work consists of the files |README.txt|, |childdoc.ins| and |childdoc.dtx|
as well as the derived files |childdoc.def|, |cdocsamp.tex|
with |cdocsch1.tex|, |cdocsch2.tex|, |cdocspt3.tex|, |cdocspt4.tex|,
|cdocsdrf.tex|, |cdocsfn1.tex|, |cdocsfn2.tex|
as well as |childdoc.pdf|.

%%%%%%%%%%%%%%%%%%%%%%%%%%%%%%%%%%%%%%%%%%%%%%%%%%%%%%%%%%%%%%%%%%%%%%%%%%%%%%%%
\subsection{Files and Installation}

The package consists of the files:
%
\begin{center}
\begin{tabular}{ll}
    |README.txt|   & readme file \\
    |childdoc.ins| & installation file \\
    |childdoc.dtx| & source file \\
    |childdoc.def| & definition file \\
    |cdocsamp.tex| & sample main file \\
    |cdocsch1.tex| & sample include file \\
    |cdocsch2.tex| & sample include file \\
    |cdocspt3.tex| & sample part file \\
    |cdocspt4.tex| & sample part file \\
    |cdocsdrf.tex| & sample redirection file \\
    |cdocsfn1.tex| & sample redirection file \\
    |cdocsfn2.tex| & sample redirection file \\
    |childdoc.pdf| & manual
\end{tabular}
\end{center}
%
The distribution consists of the files
|README.txt|, |childdoc.ins| and |childdoc.dtx|.
%
\begin{itemize}
\item
Run (pdf)\LaTeX{} on |childdoc.dtx|
to compile the manual |childdoc.pdf| (this file).
\item
Run \LaTeX{} on |childdoc.ins| to create the definitions file |childdoc.def|
and the sample |cdocsamp.tex| with include files
|cdocsch1.tex|, |cdocsch2.tex|, |cdocspt3.tex|, |cdocspt4.tex|,
|cdocsdrf.tex|, |cdocsfn1.tex|, |cdocsfn2.tex|.
Then copy the file |childdoc.def| to an appropriate directory of your \LaTeX{}
distribution, e.g.\ \textit{texmf-root}|/tex/latex/childdoc|.
\end{itemize}

%%%%%%%%%%%%%%%%%%%%%%%%%%%%%%%%%%%%%%%%%%%%%%%%%%%%%%%%%%%%%%%%%%%%%%%%%%%%%%%%
\subsection{Related CTAN Packages}

There are several other packages which offer a similar functionality:
%
\begin{itemize}
\item
The packages
\href{http://ctan.org/pkg/docmute}{\textsf{docmute}},
\href{http://ctan.org/pkg/includex}{\textsf{includex}} and
\href{http://ctan.org/pkg/standalone}{\textsf{standalone}}
provide commands to include only the document body of
a child file thus allowing both files to be compiled individually.
\item
The packages \href{http://ctan.org/pkg/subdocs}{\textsf{subdocs}}
and \href{http://ctan.org/pkg/subfiles}{\textsf{subfiles}}
provide structures in which the main and child documents can be
encapsulated and allowing them to be compiled individually.
The inclusion mechanism is different from the conventional |\include|.
\item
The package \href{http://ctan.org/pkg/combine}{\textsf{combine}}
is an elaborate solution to combine several documents into one.
\end{itemize}
%
See also the CTAN topic \href{http://ctan.org/topic/subdocs}{\textsf{subdocs}}
for further related packages.
The present package differs from the above solutions in that
a document structure constructed with the conventional |\include| mechanism
just needs two extra commands at the top of every file
such that all constituent files can be compiled individually.

%%%%%%%%%%%%%%%%%%%%%%%%%%%%%%%%%%%%%%%%%%%%%%%%%%%%%%%%%%%%%%%%%%%%%%%%%%%%%%%%
%\subsection{Feature Suggestions}
%
%The following is a list of features which may be useful for future
%versions of this package:
%%
%\begin{itemize}
%\item
%\ldots
%\end{itemize}

%%%%%%%%%%%%%%%%%%%%%%%%%%%%%%%%%%%%%%%%%%%%%%%%%%%%%%%%%%%%%%%%%%%%%%%%%%%%%%%%
\subsection{Revision History}

%%%%%%%%%%%%%%%%%%%%%%%%%%%%%%%%%%%%%%%%
\paragraph{v2.0:} 2018/12/30

\begin{itemize}
\item
immediate forward processing
\item
added |\childdocby| mechanism
\item
manual restructured
\end{itemize}

%%%%%%%%%%%%%%%%%%%%%%%%%%%%%%%%%%%%%%%%
\paragraph{v1.6:} 2018/01/17

\begin{itemize}
\item
application for development of include files
\item
corrections to manual
\end{itemize}

%%%%%%%%%%%%%%%%%%%%%%%%%%%%%%%%%%%%%%%%
\paragraph{v1.5:} 2017/05/21

\begin{itemize}
\item
more complete structuring introduced
\item
|\childdocof| introduced
\item
|\childdoc| renamed to |\childdocmain|
\item
|\childredirect| renamed to |\childdocforward| and |\childdocforwardprefix|
and functionality expanded
\end{itemize}

%%%%%%%%%%%%%%%%%%%%%%%%%%%%%%%%%%%%%%%%
\paragraph{v1.0:} 2017/04/27

\begin{itemize}
\item
manual and install package
\item
first version published on CTAN
\end{itemize}

%%%%%%%%%%%%%%%%%%%%%%%%%%%%%%%%%%%%%%%%
\paragraph{v0.6:} 2017/04/26

\begin{itemize}
\item
redirection mechanism added
\end{itemize}

%%%%%%%%%%%%%%%%%%%%%%%%%%%%%%%%%%%%%%%%
\paragraph{v0.5:} 2017/04/26

\begin{itemize}
\item
functionality in definition file
\end{itemize}


%%%%%%%%%%%%%%%%%%%%%%%%%%%%%%%%%%%%%%%%%%%%%%%%%%%%%%%%%%%%%%%%%%%%%%%%%%%%%%%%
%%%%%%%%%%%%%%%%%%%%%%%%%%%%%%%%%%%%%%%%%%%%%%%%%%%%%%%%%%%%%%%%%%%%%%%%%%%%%%%%
%%%%%%%%%%%%%%%%%%%%%%%%%%%%%%%%%%%%%%%%%%%%%%%%%%%%%%%%%%%%%%%%%%%%%%%%%%%%%%%%
\appendix

\settowidth\MacroIndent{\rmfamily\scriptsize 000\ }

 \DocInput{childdoc.dtx}

\end{document}
%</driver>
% \fi
%
% %%%%%%%%%%%%%%%%%%%%%%%%%%%%%%%%%%%%%%%%%%%%%%%%%%%%%%%%%%%%%%%%%%%%%%%%%%%%%%
% %%%%%%%%%%%%%%%%%%%%%%%%%%%%%%%%%%%%%%%%%%%%%%%%%%%%%%%%%%%%%%%%%%%%%%%%%%%%%%
% \section{Sample}
%\iffalse
%<*samplemain>
%\fi
%
% The following presents a sample document
% with two chapters, two parts, a title page,
% a compile flag as well as three forwarding files to set the flag.
% It consists of eight |.tex| files:
% \begin{center}
% \begin{tabular}{ll}
% |cdocsamp.tex|&main file\\
% |cdocsch1.tex|&include file for chapter 1\\
% |cdocsch2.tex|&include file for chapter 2\\
% |cdocspt3.tex|&include file for part 3\\
% |cdocspt4.tex|&include file for part 4\\
% |cdocsdrf.tex|&forwarding file for main file in draft mode\\
% |cdocsfi1.tex|&forwarding file for final version of chapter 1\\
% |cdocsfi2.tex|&forwarding file for final version of chapter 2\\
% \end{tabular}
% \end{center}
% Each of the eight files can be compiled directly by the \LaTeX{} compiler.
%
% %%%%%%%%%%%%%%%%%%%%%%%%%%%%%%%%%%%%%%
% \paragraph{Main File.}
%
% The main file is called |cdocsamp.tex|.
%
% Load the \textsf{childdoc} definitions and
% declare the filename for the main document:
%    \begin{macrocode}
\input{childdoc.def}
\childdocmain{}
%    \end{macrocode}

% Optional override for |\version| flag:
%    \begin{macrocode}
%%\ifchilddoc\else\providecommand{\version}{draft}\fi
%    \end{macrocode}

% Define the default values for the |\version| flag
% (|final| for the main file and |draft| for childs):
%    \begin{macrocode}
\ifchilddoc
\providecommand{\version}{draft}
\else
\providecommand{\version}{final}
\fi
%    \end{macrocode}

% Load the standard document class:
%    \begin{macrocode}
\documentclass[12pt]{article}
%    \end{macrocode}

% Start the document body:
%    \begin{macrocode}
\begin{document}
%    \end{macrocode}

% Declare a title page.
% Print title, part of document being processed and version flag:
%    \begin{macrocode}
\addtocounter{page}{-1}
\begin{center}
{\LARGE\bfseries{}childdoc example\par}
\vspace{1cm}
\ifchilddoc
\ifchilddocmanual part\else chapter\fi:
`\childdocname' of `\childdocjob'\par
\else
main document: `\childdocjob'\par
\fi
version: \version\par
\end{center}
\newpage
%    \end{macrocode}

% Manually include selected file,
% otherwise process as usual:
%    \begin{macrocode}
\ifchilddocmanual
\section*{part `\childdocname'}
\input{\childdocname}
\else
%    \end{macrocode}

% Include the two chapters:
%    \begin{macrocode}
\include{cdocsch1}
\include{cdocsch2}
%    \end{macrocode}

% Include the two parts unless only chapters should be displayed:
%    \begin{macrocode}
\ifchilddoc\else
\section{part three}
\input{cdocspt3}
\section{part four}
\input{cdocspt4}
\fi
%    \end{macrocode}

% Process as usual until here:
%    \begin{macrocode}
\fi
%    \end{macrocode}

% End of document body:
%    \begin{macrocode}
\end{document}
%    \end{macrocode}
%\iffalse
%</samplemain>
%\fi
%
% %%%%%%%%%%%%%%%%%%%%%%%%%%%%%%%%%%%%%%
% \paragraph{Chapter Include Files.}
%
% The include files are called |cdocsch1.tex| and |cdocsch2.tex|.
%
%\iffalse
%<*samplechap1|samplechap2>
%\fi

% Optional override for |\version| flag:
%    \begin{macrocode}
%%\providecommand{\version}{final}
%    \end{macrocode}

% Include the main document:
%    \begin{macrocode}
\input{childdoc.def}
\childdocof{cdocsamp}
%    \end{macrocode}

%\iffalse
%</samplechap1|samplechap2>
%\fi
%
%\iffalse
%<*samplechap1>
%\fi
% Some text for chapter 1:
%    \begin{macrocode}
\section{one}
some text in chapter one
%    \end{macrocode}

%\iffalse
%</samplechap1>
%\fi
% Some text for chapter 2:
%\iffalse
%<*samplechap2>
%\fi
%    \begin{macrocode}
\section{two}
more text in chapter two
%    \end{macrocode}

%\iffalse
%</samplechap2>
%\fi
%
% %%%%%%%%%%%%%%%%%%%%%%%%%%%%%%%%%%%%%%
% \paragraph{Part Include Files.}
%
% The include files are called |cdocspt3.tex| and |cdocspt4.tex|.
%
%\iffalse
%<*samplepart3|samplepart4>
%\fi

% Optional override for |\version| flag:
%    \begin{macrocode}
%%\providecommand{\version}{final}
%    \end{macrocode}

% Include the main document:
%    \begin{macrocode}
\input{childdoc.def}
\childdocby{cdocsamp}
%    \end{macrocode}

%\iffalse
%</samplepart3|samplepart4>
%\fi
%
%\iffalse
%<*samplepart3>
%\fi
% Some text for part 3:
%    \begin{macrocode}
some text in part three
%    \end{macrocode}

%\iffalse
%</samplepart3>
%\fi
% Some text for part 4:
%\iffalse
%<*samplepart4>
%\fi
%    \begin{macrocode}
more text in part four
%    \end{macrocode}

%\iffalse
%</samplepart4>
%\fi
%
% %%%%%%%%%%%%%%%%%%%%%%%%%%%%%%%%%%%%%%
% \paragraph{Forwarding for a Complete Draft.}
%
% The following forwarding file |cdocsdrf.tex|
% compiles the main document in draft mode:
%\iffalse
%<*sampledraft>
%\fi
%    \begin{macrocode}
\def\version{draft}
\input{childdoc.def}
\childdocforward{cdocsamp}
%    \end{macrocode}

%\iffalse
%</sampledraft>
%\fi
%
% %%%%%%%%%%%%%%%%%%%%%%%%%%%%%%%%%%%%%%
% \paragraph{Forwarding for Final Version of the Chapters.}
%
% The following forwarding files |cdocsfn1.tex| and |cdocsfn2.tex|
% (with identical content)
% compile the final versions of the child documents
% |cdocsch1.tex| and |cdocsch2.tex|, respectively:
%\iffalse
%<*samplefinal>
%\fi
%    \begin{macrocode}
\def\version{final}
\input{childdoc.def}
\childdocforwardprefix[cdocsamp]{cdocsfn}{cdocsch}
%    \end{macrocode}

%\iffalse
%</samplefinal>
%\fi
%
% %%%%%%%%%%%%%%%%%%%%%%%%%%%%%%%%%%%%%%
% \paragraph{Command Line Processing.}
%
% The following three command lines generate the output files
% |cdocscld|, |cdocscl1| and |cdocscl2|
% which should be identical to
% |cdocsdrf|, |cdocsch1| and |cdocsfn2|, respectively:
% \begin{center}
% \begin{tabular}{l}
% |latex -jobname cdocscld \|\\
% |  "\def\version{draft}\input{childdoc.def}\childdocforward{cdocsamp}"|\\
% |latex -jobname cdocscl1 \|\\
% |  "\input{childdoc.def}\childdocforward[cdocsamp]{cdocsch1}"|\\
% |latex -jobname cdocscl2 \|\\
% |  "\def\version{final}\input{childdoc.def}\childdocforward{cdocsch2}"|
% \end{tabular}
% \end{center}
% Note that the trailing backslash on each first line
% merely continues the input to the second line
% (for convenient cut ant paste).
% Furthermore, the command |latex| can be replaced by any
% of its alternative versions such as |pdflatex|.
%
% %%%%%%%%%%%%%%%%%%%%%%%%%%%%%%%%%%%%%%%%%%%%%%%%%%%%%%%%%%%%%%%%%%%%%%%%%%%%%%
% %%%%%%%%%%%%%%%%%%%%%%%%%%%%%%%%%%%%%%%%%%%%%%%%%%%%%%%%%%%%%%%%%%%%%%%%%%%%%%
% \section{Implementation}
%\iffalse
%<*package>
%\fi
%
% This section describes the definitions file |childdoc.def|.

% The definitions cannot be loaded using |\usepackage| or |\RequirePackage|
% which has a mechanism to prevent loading a style file more than once.
% When loading the definitions by means of |\input|
% multiple instances have to be prevented manually:
%\iffalse
%This code needs to be before the `\ProvidesFile' directive
%which is defined at the beginning of this file.
%Therefore it is also placed there and commented out here.
%</package>
%<*discard>
%\fi
%    \begin{macrocode}
\ifdefined\childdocmain\endinput\fi
%    \end{macrocode}
%\iffalse
%</discard>
%<*package>
%\fi
%
% \macro{\ifchilddoc}
% \macro{\ifchilddocmanual}
% The conditional |\ifchilddoc| tells whether a
% child (true) or main (false) document is being compiled.
% The conditional |\ifchilddocmanual| tells whether
% the |\includeonly| mechanism is used (false) or
% the selection of child files must be performed manually (true).
% The definitions initialise to false:
%    \begin{macrocode}
\newif\ifchilddoc
\newif\ifchilddocmanual
%    \end{macrocode}

% \macro{\childdocname}
% \macro{\childdocjob}
% The macro |\childdocname| stores the name of the main document
% to be compiled. The macro |\childdocjob| stores the name of
% the document on which the \LaTeX{} compiler was originally invoked.
% The content of |\jobname| cannot be compared
% to filenames specified in the source due to different catcodes.
% The following code rescans |\jobname|, stores the result
% in |\childdocname| and saves a copy in |\childdocjob|:
%    \begin{macrocode}
\edef\childdocname{\scantokens\expandafter{\jobname\noexpand}}
\let\childdocjob\childdocname
%    \end{macrocode}

% \macro{\childdocdisable}
% The macro |\childdocdisable| prevents the main file
% from being processed more than once.
% At this stage, the main document command |\childdocmain|
% is assumed to be called once again where it should do nothing.
% Any subsequent call to it should prevent
% a secondary processing of the main document
% It overwrites the forwarding commands
% |\childdocof| and |\childdocforward|
% with empty macros to prevent further inclusions of the main document:
%    \begin{macrocode}
\newcommand{\childdocdisable}
{
  \renewcommand{\childdocmain}[1]{\renewcommand{\childdocmain}[1]{\endinput}}
  \renewcommand{\childdocof}[1]{}
  \renewcommand{\childdocby}[2][]{}
  \renewcommand{\childdocforward}[2][]{}
  \renewcommand{\childdocdisable}{}
}
%    \end{macrocode}

% \macro{\childdocmain}
% The macro |\childdocmain| is to be called at the top of the main file
% with nothing or the main filename (without extension) as argument.
% First, it breaks loops.
% If the argument is not empty and does not match |\childdocname|
% (which is set by the first inclusion of |childdoc.def|),
% |\ifchilddoc| is set to true, |\includeonly| is applied to the child file
% and |\jobname| is set to the main file
% (for proper handling of |.aux| files):
%    \begin{macrocode}
\newcommand{\childdocmain}[1]
{
  \childdocdisable\childdocmain{}
  \if?#1?\else
    \begingroup
      \def\childdoctmp{#1}
      \ifx\childdoctmp\childdocname
        \def\childdoctmp{}
      \else
        \def\childdoctmp
        {
          \childdoctrue
          \includeonly{\childdocname}
          \def\childdocjob{#1}
          \def\jobname{#1}
        }
      \fi
      \expandafter
    \endgroup
    \childdoctmp
  \fi
}
%    \end{macrocode}

% \macro{\childdocof}
% The command |\childdocof| redirects
% compilation to the main file |#1|.
%    \begin{macrocode}
\newcommand{\childdocof}[1]
{
  \childdocdisable
  \childdoctrue
  \includeonly{\childdocname}
  \def\jobname{#1}
  \def\childdocjob{#1}
  \input{#1}
}
%    \end{macrocode}

% \macro{\childdocby}
% The command |\childdocby| ....
%    \begin{macrocode}
\newcommand{\childdocby}[2][]
{
  \childdocdisable
  \childdoctrue
  \childdocmanualtrue
  \if?#1?\else
    \def\jobname{#2}
  \fi
  \def\childdocjob{#2}
  \input{#2}
  \endinput
}
%    \end{macrocode}

% \macro{\childdocforward}
% The command |\childdocforward| redirects
% compilation to the main file or
% (if the optional argument is given) a child file.
% Parameters are set as if the main file
% or a child file starting with |\childdocof| was compiled.
% Then compilation is handed over to the main file:
%    \begin{macrocode}
\newcommand{\childdocforward}[2][]
{
  \begingroup
    \if?#1?
      \def\childdoctmp
      {
        \def\childdocname{#2}
        \def\childdocjob{#2}
        \def\jobname{#2}
        \input{#2}
        \endinput
      }
    \else
      \def\childdoctmp
      {
        \childdocdisable
        \def\childdocname{#2}
        \childdoctrue
        \includeonly{#2}
        \def\childdocjob{#1}
        \def\jobname{#1}
        \input{#1}
        \endinput
      }
    \fi
    \expandafter
  \endgroup
  \childdoctmp
}
%    \end{macrocode}

% \macro{\childdocforwardprefix}
% The command |\childdocforwardprefix| redirects
% compilation to the main or a child file by means of a pattern.
% The prefix |#1| in the current filename is replaced by |#2|
% and the suffix of the current filename is kept
% (it is assumed that the filename does not contain the substring `|~~~|'
% which is used as a delimiter).
% Compilation is handed over to the new file by |\childdocforward|:
%    \begin{macrocode}
\newcommand{\childdocforwardprefix}[3][]
{
  \begingroup
    \def\childdocextract #2##1~~~{\def\childdoctmp{\childdocforward[#1]{#3##1}}}
    \expandafter\childdocextract\childdocname~~~
    \expandafter
  \endgroup
  \childdoctmp
}
%    \end{macrocode}

% \macro{\childdoc}
% The deprecated macro |\childdoc| is a legacy version of |\childdocmain|:
%    \begin{macrocode}
\newcommand{\childdoc}{\childdocmain}
%    \end{macrocode}

% \macro{\childdocredirect}
% The deprecated macro |\childdocredirect| is a legacy version
% of |\childdocforward| and |\childdocforwardprefix|:
%    \begin{macrocode}
\newcommand{\childdocredirect}[2][]
{
  \begingroup
    \if?#1?
      \def\childdoctmp{\childdocforward{#2}}
    \else
      \def\childdoctmp{\childdocforwardprefix{#1}{#2}}
    \fi
    \expandafter
  \endgroup
  \childdoctmp
}
%    \end{macrocode}

%\iffalse
%</package>
%\fi
%
\endinput
\childdocforward[cdocsamp]{cdocsch1}"|\\
% |latex -jobname cdocscl2 \|\\
% |  "\def\version{final}% \iffalse
%
% childdoc.dtx Copyright (C) 2017-2018 Niklas Beisert
%
% This work may be distributed and/or modified under the
% conditions of the LaTeX Project Public License, either version 1.3
% of this license or (at your option) any later version.
% The latest version of this license is in
%   http://www.latex-project.org/lppl.txt
% and version 1.3 or later is part of all distributions of LaTeX
% version 2005/12/01 or later.
%
% This work has the LPPL maintenance status `maintained'.
%
% The Current Maintainer of this work is Niklas Beisert.
%
% This work consists of the files childdoc.dtx and childdoc.ins
% and the derived files childdoc.def and cdocsamp.tex with
% cdocsch1.tex, cdocsch2.tex, cdocsdrf.tex, cdocsfn1.tex, cdocsfn2.tex.
%
%<package>\ifdefined\childdocmain\endinput\fi
%<package>\ProvidesFile{childdoc.def}[2018/12/30 v2.0 child document driver]
%<samplemain>\ProvidesFile{cdocsamp.tex}[2018/12/30 v2.0 sample for childdoc]
%<*driver>
%\ProvidesFile{childdoc.drv}[2018/12/30 v2.0 childdoc reference manual file]
\PassOptionsToClass{10pt,a4paper}{article}
\documentclass{ltxdoc}

\usepackage[margin=35mm]{geometry}
\usepackage{hyperref}
\usepackage{hyperxmp}
\usepackage[usenames]{color}

\hypersetup{colorlinks=true}
\hypersetup{pdfstartview=FitH}
\hypersetup{pdfpagemode=UseNone}
\hypersetup{pdfsource={}}
\hypersetup{pdflang={en-UK}}
\hypersetup{pdfcopyright={Copyright 2017-2018 Niklas Beisert.
  This work may be distributed and/or modified under the
  conditions of the LaTeX Project Public License, either version 1.3
  of this license or (at your option) any later version.}}
\hypersetup{pdflicenseurl={http://www.latex-project.org/lppl.txt}}
\hypersetup{pdfcontactaddress={ETH Zurich, ITP, HIT K,
  Wolfgang-Pauli-Strasse 27}}
\hypersetup{pdfcontactpostcode={8093}}
\hypersetup{pdfcontactcity={Zurich}}
\hypersetup{pdfcontactcountry={Switzerland}}
\hypersetup{pdfcontactemail={nbeisert@itp.phys.ethz.ch}}
\hypersetup{pdfcontacturl={http://people.phys.ethz.ch/\xmptilde nbeisert/}}

\newcommand{\secref}[1]{\hyperref[#1]{section \ref*{#1}}}

\parskip1ex
\parindent0pt
\let\olditemize\itemize
\def\itemize{\olditemize\parskip0pt}

\begin{document}

\title{The \textsf{childdoc} Package}
\hypersetup{pdftitle={The childdoc Package}}
\author{Niklas Beisert\\[2ex]
  Institut f\"ur Theoretische Physik\\
  Eidgen\"ossische Technische Hochschule Z\"urich\\
  Wolfgang-Pauli-Strasse 27, 8093 Z\"urich, Switzerland\\[1ex]
  \href{mailto:nbeisert@itp.phys.ethz.ch}
  {\texttt{nbeisert@itp.phys.ethz.ch}}}
\hypersetup{pdfauthor={Niklas Beisert}}
\hypersetup{pdfsubject={Manual for the LaTeX2e Package childdoc}}
\date{30 December 2018, \textsf{v2.0}}
\maketitle

\begin{abstract}\noindent
\textsf{childdoc} is a \LaTeXe{} package
that enables the direct compilation
of document sections included by |\include|
to individual files.
\end{abstract}

\begingroup
\parskip0ex
\tableofcontents
\endgroup

%%%%%%%%%%%%%%%%%%%%%%%%%%%%%%%%%%%%%%%%%%%%%%%%%%%%%%%%%%%%%%%%%%%%%%%%%%%%%%%%
%%%%%%%%%%%%%%%%%%%%%%%%%%%%%%%%%%%%%%%%%%%%%%%%%%%%%%%%%%%%%%%%%%%%%%%%%%%%%%%%
\section{Introduction}

\LaTeX{} provides a mechanism to structure a large document (such as a book)
into a main file and several child files (containing the chapters)
using the |\include| command.
This mechanism is beneficial for documents
which span hundreds of pages in order to
make the source file(s) more manageable.
Moreover, compilation can be restricted to
selected child files by means of the |\includeonly| command.
The latter feature can be used to reduce the compilation time while editing
(this was significantly more useful in the earlier days of \LaTeX{})
or to generate a smaller document which is easier to navigate.
Another application of |\includeonly| is to generate
documents consisting of selected parts of the complete document.

However, there are a few drawbacks of the plain |\include| mechanism:
\begin{itemize}
\item
The child files cannot be compiled on their own,
they can only be compiled via the main file.
A naive editing environment
(such as a text editor with an option
to have the current file processed by \LaTeX)
may require one to switch to the main file before compiling;
attempting to compile the child file produces errors.
\item
The main file must be modified (each time)
to adjust the |\includeonly| command
to the present needs. This easily leaves the main file in a messy state.
\item
The generated document will always carry the filename
of the main document. This is inconvenient if
several child files are to be compiled and
to be kept for distribution.
\end{itemize}

The present package provides a simple interface
to make child files individually compilable by \LaTeX{}.
Compiling a child file then has the same effect as compiling
the main file with an |\includeonly| command
to select the appropriate child.
Moreover the generated document will carry the name of the child
rather than the main file.
This resolves all three above issues.

This feature is meant to make the editing of books,
thesis documents and lecture notes somewhat more convenient.
However, the package can also be used efficiently for
composing a series of documents (such as exercise sheets)
which are typically distributed individually.
It then assists the author in generating the individual documents
(potentially in different versions)
as well as a document containing the collected series.
Another application is in developing style files
or other kinds of included material
where compilation of the style file could redirect
to a sample or test file.

%%%%%%%%%%%%%%%%%%%%%%%%%%%%%%%%%%%%%%%%%%%%%%%%%%%%%%%%%%%%%%%%%%%%%%%%%%%%%%%%
%%%%%%%%%%%%%%%%%%%%%%%%%%%%%%%%%%%%%%%%%%%%%%%%%%%%%%%%%%%%%%%%%%%%%%%%%%%%%%%%
\section{Usage}

First of all, the package \textsf{childdoc} is \emph{not} a standard
\LaTeXe{} |.sty| style file! Therefore it needs to be invoked in
a non-standard way.

%%%%%%%%%%%%%%%%%%%%%%%%%%%%%%%%%%%%%%%%%%%%%%%%%%%%%%%%%%%%%%%%%%%%%%%%%%%%%%%%
\subsection{Included Files}
\label{sec:include}

%%%%%%%%%%%%%%%%%%%%%%%%%%%%%%%%%%%%%%%%
\DescribeMacro{\childdocmain}
To use the package, add the commands
\begin{center}
\begin{tabular}{l}
|\input{childdoc.def}|\\
|\childdocmain{}|\\
\end{tabular}
\end{center}
at the very top of the main \LaTeX{} file,
in particular \emph{before} the |\documentclass| statement!
The argument of |\childdocmain| should be left empty
(but it must be present).

%%%%%%%%%%%%%%%%%%%%%%%%%%%%%%%%%%%%%%%%
\DescribeMacro{\childdocof}
Furthermore, add the commands
\begin{center}
\begin{tabular}{l}
|\input{childdoc.def}|\\
|\childdocof{|\textit{main}|}|\\
\end{tabular}
\end{center}
at the top of every child file \textit{child}
which is included by |\include{|\textit{child}|}|
from within the main file
(or at least for those files to be compiled individually).
The argument \textit{main} must be the filename of the main file.

There are a couple of
considerations in setting up the main and child documents:

%%%%%%%%%%%%%%%%%%%%%%%%%%%%%%%%%%%%%%%%
\paragraph{Restrictions.}

Please note the following restrictions:
\begin{itemize}
\item
|\childdocmain| must be called with one argument \textit{main}
to ensure compatibility with earlier version of the package.
It must either be empty (|\childdocmain{}|)
or precisely match the filename of the main file in which it is specified.
See \secref{sec:detection} for further information.
\item
The filename \textit{main} must be specified without the |.tex| extension.
\item
The filename \textit{main} is case sensitive
(even in case-insensitive file systems)
due to internal string comparison.
\item
The argument \textit{main} should be fully expanded, it cannot be a macro.
\item
Subdirectories and special characters should be avoided in filenames.
\item
The command |\childdocmain{|\textit{main}|}| must be followed by a whitespace.
It should not be followed immediately by another command
or by a comment mark `|%|'.
This is because the \TeX{} parser reads the token immediately following
the argument of |\childdocmain| and puts it
at the beginning of every child section;
however, a white\-space is ignored.
\end{itemize}

%%%%%%%%%%%%%%%%%%%%%%%%%%%%%%%%%%%%%%%%
\paragraph{Content of Main File.}

It is advisable to place all content in the child files included by |\include|.
Any output contained in the main file will appear in all child documents
unless suppressed manually;
it cannot be suppressed automatically by the |\includeonly| directive
and thus should normally be avoided.
A method to include some content in the main file
by means of conditional processing is described in \secref{sec:conditional}.

%%%%%%%%%%%%%%%%%%%%%%%%%%%%%%%%%%%%%%%%
\paragraph{Page Numbering.}

When only a part of the document is compiled,
the appropriate numbering of pages
(as well as other status parameters)
is determined from the |.aux| files.
The latter contain information from previous passes.
However this information needs to propagate through
all intermediate child documents.
Therefore the page numbering in child documents may well
be inconsistent until the complete document is compiled at least once.

A useful (if unconventional) way to always ensure a consistent
page numbering is to restart the numbering in each child document
and denote the pages by `\textit{child}|.|\textit{page}'
where \textit{child} represents the chapter/section number of the child file.
This can be achieved by the command
|\numberwithin{page}{|\textit{child}|}|
of the \textsf{amsmath} package
where \textit{child} can be |chapter| or |section|
depending on the chosen structuring.
Alternatively, one can modify the macro |\thepage| appropriately
and reset the counter |page| at the start of each child file.

%%%%%%%%%%%%%%%%%%%%%%%%%%%%%%%%%%%%%%%%%%%%%%%%%%%%%%%%%%%%%%%%%%%%%%%%%%%%%%%%
\subsection{Conditional Processing}
\label{sec:conditional}

The package provides a mechanism to compile different versions
of a document. To customise the versions further some conditional processing
can come in handy to distinguish which version is being compiled.
The package provides two macros to describe the compilation context:

%%%%%%%%%%%%%%%%%%%%%%%%%%%%%%%%%%%%%%%%
\DescribeMacro{\ifchilddoc}
The conditional |\ifchilddoc| distinguishes between the compilation of
child documents and the main document:
%
\begin{center}
|\ifchilddoc |\textit{child-code}| |[|\||else |\textit{main-code}]| \||fi|
\end{center}

%%%%%%%%%%%%%%%%%%%%%%%%%%%%%%%%%%%%%%%%
\DescribeMacro{\childdocname}
\DescribeMacro{\childdocjob}
The macro |\childdocname| contains the filename (without extension)
of the main or child file being processed.
Note that |\childdocjob| will always contain the name of the main file.

%%%%%%%%%%%%%%%%%%%%%%%%%%%%%%%%%%%%%%%%
\paragraph{Title Page.}

Conditional processing can be used to include a title or banner page
in the main document when proper precautions are taken.
Importantly, the code in the main file should ensure that the page counter
(as well as other status parameters which are stored in the |.aux| files)
takes the same value after the conditional processing.
Otherwise the page numbers may take divergent values
depending on which part is compiled.

For example, a title page could be declared by:
%
\begin{center}
\begin{tabular}{l}
|\ifchilddoc\||else|\\
|\addtocounter{page}{-1}|\\
\textit{code for title page}\\
|\newpage|\\
|\||fi|
\end{tabular}
\end{center}
%
A banner page for the child documents can be generated by:
%
\begin{center}
\begin{tabular}{l}
|\ifchilddoc|\\
|\addtocounter{page}{-1}|\\
\textit{code for banner page}\\
|\newpage|\\
|\||fi|
\end{tabular}
\end{center}
%
Here one could write a message such as:
\begin{center}
|This is the part \childdocname{} of \childdocjob{}.|
\end{center}

%%%%%%%%%%%%%%%%%%%%%%%%%%%%%%%%%%%%%%%%%%%%%%%%%%%%%%%%%%%%%%%%%%%%%%%%%%%%%%%%
\subsection{Flags}
\label{sec:flags}

The package makes it easy to generate different versions
of the main or child documents.
To this end compilation flags can be defined
and assigned different default values.
They will be particularly useful in conjunction
with the forwarding mechanism described in \secref{sec:forward}.

For example, it may be useful to have a flag |\version|
which can be set to |draft| or |final|.
The document source will contain some conditional code
depending on the value of |\version|.
Suppose further, the flag should default to |final| for the main file
and to |draft| for child files
which is a natural assignment for editing the document.
This is achieved by placing the following code
in the preamble of the main document
(below the |\childdocmain| directive):
%
\begin{center}
\begin{tabular}{l}
|\ifchilddoc|\\
|\providecommand{\version}{draft}|\\
|\||else|\\
|\providecommand{\version}{final}|\\
|\||fi|
\end{tabular}
\end{center}
%
The definition by |\providecommand| makes sure
that previous definitions are not overwritten.
Further statements |\providecommand{\version}{...}|
can thus be added before the above code to override it.

For the main file, one might add a line
(between |\childdocmain| and the above block)
%
\begin{center}
|%\ifchilddoc\||else\providecommand{\version}{draft}\||fi|
\end{center}
%
which can be uncommented to produce a draft version.
Likewise one can add a line to the very top of a child file
(above the |\childdocof{|\textit{main}|}| directive)
%
\begin{center}
|%\providecommand{\version}{final}|
\end{center}
%
which can be uncommented to produce the final version of this child document.

%%%%%%%%%%%%%%%%%%%%%%%%%%%%%%%%%%%%%%%%%%%%%%%%%%%%%%%%%%%%%%%%%%%%%%%%%%%%%%%%
\subsection{Forwarding}
\label{sec:forward}

Different versions of the main or child documents
using compilation flags as described in \secref{sec:flags}
can be (permanently) stored in different files
for convenient compilation, viewing and distribution.
To this end, the package defines a command
to pass on compilation to a different file:

%%%%%%%%%%%%%%%%%%%%%%%%%%%%%%%%%%%%%%%%
\DescribeMacro{\childdocforward}
The command |\childdocforward| redirects processing to
another source file:
%
\begin{center}
\begin{tabular}{l}
|\input{childdoc.def}|\\
|\childdocforward[|\textit{main}|]{|\textit{dest}|}|\\
\end{tabular}
\end{center}
%
The argument \textit{dest} is the destination file
(without extension).
It should be the main file or one of the child files.
Note that further \textsf{childdoc} directives
such as |\childdocof| and |\childdocforward|
in the indicated file will be processed in this form.
The optional argument \textit{main}
passes on directly to the main file \textit{main}
while pretending to compile the child \textit{dest}.
This form behaves as if \textit{dest}
issues |\childdocof{|\textit{main}|}| right away,
and no further \textsf{childdoc} directives will be processed.

%%%%%%%%%%%%%%%%%%%%%%%%%%%%%%%%%%%%%%%%
\DescribeMacro{\...prefix}
In the alternative form |\childdocforwardprefix|,
%
\begin{center}
\begin{tabular}{l}
|\input{childdoc.def}|\\
|\childdocforwardprefix[|\textit{main}|]{|\textit{prefix}|}{|\textit{dest}|}|
\end{tabular}
\end{center}
%
the destination file is determined by a pattern
depending on the current file:
To make this work, the current file must be called
`{\textit{prefix}\hspace{0.2em}\textit{suffix}}'
with \textit{prefix} matching precisely the argument.
Processing is then passed on to the file
`{\textit{dest}\hspace{0.2em}\textit{suffix}}'.
Surely, the same effect is achieved by
directly specifying the
argument `{\textit{dest}\hspace{0.2em}\textit{suffix}}'
in the first form.
However, that requires to set up a different file
for each child. With the alternative form of the command
all these files can have exactly the same content
which simplifies setting them up and maintaining them.

For example, the following file |draft.tex|
with a compilation flag |\version| as described in \secref{sec:flags}
compiles the main document as a draft:
%
\begin{center}
\begin{tabular}{l}
|\def\version{draft}|\\
|\input{childdoc.def}|\\
|\childdocforward{|\textit{main}|}|
\end{tabular}
\end{center}
%
Likewise, the following files |final|\textit{nn}|.tex|
compile the final version of the child document
|child|\textit{nn}|.tex|:
%
\begin{center}
\begin{tabular}{l}
|\def\version{final}|\\
|\input{childdoc.def}|\\
|\childdocforwardprefix{final}{child}|
\end{tabular}
\end{center}
%

Note that when several versions of a main file and/or of each child file
are to be generated, it may be convenient to set up a |Makefile| or
shell script to automatise the process.

%%%%%%%%%%%%%%%%%%%%%%%%%%%%%%%%%%%%%%%%%%%%%%%%%%%%%%%%%%%%%%%%%%%%%%%%%%%%%%%%
\subsection{Command Line Processing}
\label{sec:commandline}

The effect of redirection files can also be achieved by invoking
the \LaTeX{} compiler with a more elaborate command line.
Most conveniently this should be done as part
of a shell script or a |Makefile|.

When using \textsf{childdoc} in the main file, the following
command lines effectively perform a redirection
(note that depending on the shell being used,
backslashes may have to be doubled: `|\|' $\to$ `|\\|'):
%
\begin{center}
|... -jobname "|\textit{target}|" |\\|"|[\textit{flags}]%
|\input{childdoc.def}\childdocforward[|\textit{main}|]{|\textit{dest}|}"|
\end{center}
%
Here \textit{target} is the name of the output file,
\textit{main} is the name of the main file
and \textit{dest} is the name of the main or child file to be processed
(all filenames without extensions).
The optional argument \textit{main} can be omitted
if \textit{main} matches \textit{dest}.
Optionally, compilation \textit{flags} can be defined via |\def| commands.
This command line makes the \TeX{} engine believe
it is compiling the file \textit{target}
whose content is specified as the latter parameter.
The provided code then forwards the processing to
\textit{main} or \textit{dest} as described in \secref{sec:forward}.

%%%%%%%%%%%%%%%%%%%%%%%%%%%%%%%%%%%%%%%%%%%%%%%%%%%%%%%%%%%%%%%%%%%%%%%%%%%%%%%%
\subsection{Include by Input}
\label{sec:input}

Including child documents by |\include| has some restrictions by design.
Most notably, the content of a child document always occupies
its own set of pages; pages cannot be shared between child documents.
Usually, this behaviour makes perfect sense
because each child document contain an essential part of the document.
However, in some situations it may be desirable to compose
a document from a collection of parts
without having mandatory page breaks between then.
For this case, the package
provides a mechanism to include parts
by |\input| which can also be processed individually.
However, by construction this mechanism
requires manual handling of the content to be output.

%%%%%%%%%%%%%%%%%%%%%%%%%%%%%%%%%%%%%%%%
\DescribeMacro{\ifchilddocmanual}
The main file should be prepared as usual, see \secref{sec:include}.
However, the document body must make a distinction
between processing of an individual part and of the main document, e.g.:
%
\begin{center}
\begin{tabular}{l}
|\ifchilddocmanual|\\
|\input{\childdocname}|\\
|\||else|\\
\textit{document body with }|\input{|\textit{part}|}|\\
|\||fi|
\end{tabular}
\end{center}
%
The conditional |\ifchilddocmanual| is true whenever
a part to be included by |\input| is being compiled,
and the name of the part is stored in |\childdocname|.

%%%%%%%%%%%%%%%%%%%%%%%%%%%%%%%%%%%%%%%%
\DescribeMacro{\childdocby}
Each part to be included by |\input| should start with:
%
\begin{center}
\begin{tabular}{l}
|\input{childdoc.def}|\\
|\childdocby{|\textit{main}|}|\\
\end{tabular}
\end{center}
%
The directive |\childdocby| is similar to |\childdocof|
described in \secref{sec:include},
but the subsequent selection of content must be done manually.
To that end, both |\ifchilddoc| and |\ifchilddocmanual|
will be true upon processing of a part,
and the name of the part is stored in |\childdocname|.
Note that |\jobname| will be set to the filename of the current part
so that each part receives an individual |.aux| file
that does not interfere with the |.aux| file(s) of the main document.
This behaviour can be altered by the alternative form
|\childdocby[*]{|\textit{main}|}| (with a non-empty optional argument)
which uses the |.aux| file of the main document
by setting |\jobname| to \textit{main}.

%%%%%%%%%%%%%%%%%%%%%%%%%%%%%%%%%%%%%%%%%%%%%%%%%%%%%%%%%%%%%%%%%%%%%%%%%%%%%%%%
\subsection{Driver Development}
\label{sec:driver}

The \textsf{childdoc} mechanism can also be use for the development
of definition files such as \LaTeX{} styles or classes.
This case differs from the above setup with multiple parts
included by |\include| in that no |\includeonly| should be invoked.
This can be achieved by starting the include file
(before |\ProvidesPackage|) with:
%
\begin{center}
\begin{tabular}{l}
|\input{childdoc.def}|\\
|\childdocforward{|\textit{main}|}|\\
\end{tabular}
\end{center}
%
or alternatively with:
%
\begin{center}
\begin{tabular}{l}
|\input{childdoc.def}|\\
|\childdocby{|\textit{main}|}|\\
\end{tabular}
\end{center}
%
Both forms have slightly different effects as described above.
The main file is prepared as usual, see \secref{sec:include}.

%%%%%%%%%%%%%%%%%%%%%%%%%%%%%%%%%%%%%%%%%%%%%%%%%%%%%%%%%%%%%%%%%%%%%%%%%%%%%%%%
\subsection{Legacy Detection}
\label{sec:detection}

The directive |\childdocmain| in the main file can detect
whether the complete document or merely a child is to be compiled
even without using the directive |\childdocof|.
This method is deprecated because it is less robust
and there is no compelling reason to use it;
it is merely provided for backward compatibility
and it may be removed in future versions.

If the detection mechanism is to be used,
it is mandatory to correctly specify
the filename of the main file as the argument of |\childdocmain|:
%
\begin{center}
\begin{tabular}{l}
|\input{childdoc.def}|\\
|\childdocmain{|\textit{main}|}|\\
\end{tabular}
\end{center}
%
If |\jobname| does not match the argument \textit{main} of |\childdocmain|,
it is assumed that |\jobname| points to the child file to be compiled.
When using |\childdocmain| with the main file specified as argument,
it suffices to start a child file
with just |\input{|\textit{main}|}|
without loading of the package and using |\childdocof|.
If instead all processing is done
with the appropriate \textsf{childdoc} directives,
the argument of \textit{main} of |\childdocmain| can be empty.

An alternative version of the command line processing described
in \secref{sec:commandline} using the detection mechanism reads:
%
\begin{center}
|... -jobname "|\textit{target}|" "|[\textit{flags}]%
[|\def\jobname{|\textit{dest}|}|]|\input{|\textit{main}|}"|
\end{center}

%%%%%%%%%%%%%%%%%%%%%%%%%%%%%%%%%%%%%%%%%%%%%%%%%%%%%%%%%%%%%%%%%%%%%%%%%%%%%%%%
\subsection{Manual Code}
\label{sec:manual}

In case one cannot be certain whether the definitions file |childdoc.def|
is installed on the target \TeX{} distribution
and one prefers not to ship it,
it is conceivable to paste a few relevant commands into the sources.

To that end, drop all statements |\input{childdoc.def}|
and perform the replacements as outlined below.
Instead of |\childdocmain{|\textit{main}|}| add the following code
to the top of the main file:
%
\begin{center}
\begin{tabular}{l}
|\||ifdefined\childdocname\endinput\||fi\newif\ifchilddoc|\\
|\edef\childdocname{\scantokens\expandafter{\jobname\noexpand}}|\\
|\def\childdocmain{|\textit{main}|}\||ifx\childdocmain\childdocname\||else|\\
|\childdoctrue\includeonly{\childdocname}\let\jobname\childdocmain\||fi|\\
\end{tabular}
\end{center}
%
Instead of |\childdocof{|\textit{main}|}| just include the main file
at the top of each child file:
%
\begin{center}
|\input{|\textit{main}|}|
\end{center}
%
A simple redirection |\childdocforward{|\textit{dest}|}| is achieved by:
%
\begin{center}
|\def\jobname{|\textit{dest}|}\input{\jobname}|
\end{center}
%
The redirection with prefix
|\childdocforwardprefix[|\textit{prefix}|]{|\textit{dest}|}|
is accomplished by:
%
\begin{center}
\begin{tabular}{l}
|{\edef\jobname{\scantokens\expandafter{\jobname\noexpand}}|\\
|\def\redirectjob |\textit{prefix}|#1~~~{\gdef\jobname{|\textit{dest}|#1}}|\\
|\expandafter\redirectjob\jobname~~~}\input{\jobname}|
\end{tabular}
\end{center}

In an alternative approach,
child documents can be compiled by a specific command line
without additional code or specific definitions:
%
\begin{center}
|... -jobname "|\textit{target}|" "|[\textit{flags}]%
|\includeonly{|\textit{dest}|}\input{|\textit{main}|}"|
\end{center}
%

%%%%%%%%%%%%%%%%%%%%%%%%%%%%%%%%%%%%%%%%%%%%%%%%%%%%%%%%%%%%%%%%%%%%%%%%%%%%%%%%
%%%%%%%%%%%%%%%%%%%%%%%%%%%%%%%%%%%%%%%%%%%%%%%%%%%%%%%%%%%%%%%%%%%%%%%%%%%%%%%%
\section{Information}

%%%%%%%%%%%%%%%%%%%%%%%%%%%%%%%%%%%%%%%%%%%%%%%%%%%%%%%%%%%%%%%%%%%%%%%%%%%%%%%%
\subsection{Copyright}

Copyright \copyright{} 2017--2018 Niklas Beisert

This work may be distributed and/or modified under the
conditions of the \LaTeX{} Project Public License, either version 1.3
of this license or (at your option) any later version.
The latest version of this license is in
  \url{http://www.latex-project.org/lppl.txt}
and version 1.3 or later is part of all distributions of \LaTeX{}
version 2005/12/01 or later.

This work has the LPPL maintenance status `maintained'.

The Current Maintainer of this work is Niklas Beisert.

This work consists of the files |README.txt|, |childdoc.ins| and |childdoc.dtx|
as well as the derived files |childdoc.def|, |cdocsamp.tex|
with |cdocsch1.tex|, |cdocsch2.tex|, |cdocspt3.tex|, |cdocspt4.tex|,
|cdocsdrf.tex|, |cdocsfn1.tex|, |cdocsfn2.tex|
as well as |childdoc.pdf|.

%%%%%%%%%%%%%%%%%%%%%%%%%%%%%%%%%%%%%%%%%%%%%%%%%%%%%%%%%%%%%%%%%%%%%%%%%%%%%%%%
\subsection{Files and Installation}

The package consists of the files:
%
\begin{center}
\begin{tabular}{ll}
    |README.txt|   & readme file \\
    |childdoc.ins| & installation file \\
    |childdoc.dtx| & source file \\
    |childdoc.def| & definition file \\
    |cdocsamp.tex| & sample main file \\
    |cdocsch1.tex| & sample include file \\
    |cdocsch2.tex| & sample include file \\
    |cdocspt3.tex| & sample part file \\
    |cdocspt4.tex| & sample part file \\
    |cdocsdrf.tex| & sample redirection file \\
    |cdocsfn1.tex| & sample redirection file \\
    |cdocsfn2.tex| & sample redirection file \\
    |childdoc.pdf| & manual
\end{tabular}
\end{center}
%
The distribution consists of the files
|README.txt|, |childdoc.ins| and |childdoc.dtx|.
%
\begin{itemize}
\item
Run (pdf)\LaTeX{} on |childdoc.dtx|
to compile the manual |childdoc.pdf| (this file).
\item
Run \LaTeX{} on |childdoc.ins| to create the definitions file |childdoc.def|
and the sample |cdocsamp.tex| with include files
|cdocsch1.tex|, |cdocsch2.tex|, |cdocspt3.tex|, |cdocspt4.tex|,
|cdocsdrf.tex|, |cdocsfn1.tex|, |cdocsfn2.tex|.
Then copy the file |childdoc.def| to an appropriate directory of your \LaTeX{}
distribution, e.g.\ \textit{texmf-root}|/tex/latex/childdoc|.
\end{itemize}

%%%%%%%%%%%%%%%%%%%%%%%%%%%%%%%%%%%%%%%%%%%%%%%%%%%%%%%%%%%%%%%%%%%%%%%%%%%%%%%%
\subsection{Related CTAN Packages}

There are several other packages which offer a similar functionality:
%
\begin{itemize}
\item
The packages
\href{http://ctan.org/pkg/docmute}{\textsf{docmute}},
\href{http://ctan.org/pkg/includex}{\textsf{includex}} and
\href{http://ctan.org/pkg/standalone}{\textsf{standalone}}
provide commands to include only the document body of
a child file thus allowing both files to be compiled individually.
\item
The packages \href{http://ctan.org/pkg/subdocs}{\textsf{subdocs}}
and \href{http://ctan.org/pkg/subfiles}{\textsf{subfiles}}
provide structures in which the main and child documents can be
encapsulated and allowing them to be compiled individually.
The inclusion mechanism is different from the conventional |\include|.
\item
The package \href{http://ctan.org/pkg/combine}{\textsf{combine}}
is an elaborate solution to combine several documents into one.
\end{itemize}
%
See also the CTAN topic \href{http://ctan.org/topic/subdocs}{\textsf{subdocs}}
for further related packages.
The present package differs from the above solutions in that
a document structure constructed with the conventional |\include| mechanism
just needs two extra commands at the top of every file
such that all constituent files can be compiled individually.

%%%%%%%%%%%%%%%%%%%%%%%%%%%%%%%%%%%%%%%%%%%%%%%%%%%%%%%%%%%%%%%%%%%%%%%%%%%%%%%%
%\subsection{Feature Suggestions}
%
%The following is a list of features which may be useful for future
%versions of this package:
%%
%\begin{itemize}
%\item
%\ldots
%\end{itemize}

%%%%%%%%%%%%%%%%%%%%%%%%%%%%%%%%%%%%%%%%%%%%%%%%%%%%%%%%%%%%%%%%%%%%%%%%%%%%%%%%
\subsection{Revision History}

%%%%%%%%%%%%%%%%%%%%%%%%%%%%%%%%%%%%%%%%
\paragraph{v2.0:} 2018/12/30

\begin{itemize}
\item
immediate forward processing
\item
added |\childdocby| mechanism
\item
manual restructured
\end{itemize}

%%%%%%%%%%%%%%%%%%%%%%%%%%%%%%%%%%%%%%%%
\paragraph{v1.6:} 2018/01/17

\begin{itemize}
\item
application for development of include files
\item
corrections to manual
\end{itemize}

%%%%%%%%%%%%%%%%%%%%%%%%%%%%%%%%%%%%%%%%
\paragraph{v1.5:} 2017/05/21

\begin{itemize}
\item
more complete structuring introduced
\item
|\childdocof| introduced
\item
|\childdoc| renamed to |\childdocmain|
\item
|\childredirect| renamed to |\childdocforward| and |\childdocforwardprefix|
and functionality expanded
\end{itemize}

%%%%%%%%%%%%%%%%%%%%%%%%%%%%%%%%%%%%%%%%
\paragraph{v1.0:} 2017/04/27

\begin{itemize}
\item
manual and install package
\item
first version published on CTAN
\end{itemize}

%%%%%%%%%%%%%%%%%%%%%%%%%%%%%%%%%%%%%%%%
\paragraph{v0.6:} 2017/04/26

\begin{itemize}
\item
redirection mechanism added
\end{itemize}

%%%%%%%%%%%%%%%%%%%%%%%%%%%%%%%%%%%%%%%%
\paragraph{v0.5:} 2017/04/26

\begin{itemize}
\item
functionality in definition file
\end{itemize}


%%%%%%%%%%%%%%%%%%%%%%%%%%%%%%%%%%%%%%%%%%%%%%%%%%%%%%%%%%%%%%%%%%%%%%%%%%%%%%%%
%%%%%%%%%%%%%%%%%%%%%%%%%%%%%%%%%%%%%%%%%%%%%%%%%%%%%%%%%%%%%%%%%%%%%%%%%%%%%%%%
%%%%%%%%%%%%%%%%%%%%%%%%%%%%%%%%%%%%%%%%%%%%%%%%%%%%%%%%%%%%%%%%%%%%%%%%%%%%%%%%
\appendix

\settowidth\MacroIndent{\rmfamily\scriptsize 000\ }

 \DocInput{childdoc.dtx}

\end{document}
%</driver>
% \fi
%
% %%%%%%%%%%%%%%%%%%%%%%%%%%%%%%%%%%%%%%%%%%%%%%%%%%%%%%%%%%%%%%%%%%%%%%%%%%%%%%
% %%%%%%%%%%%%%%%%%%%%%%%%%%%%%%%%%%%%%%%%%%%%%%%%%%%%%%%%%%%%%%%%%%%%%%%%%%%%%%
% \section{Sample}
%\iffalse
%<*samplemain>
%\fi
%
% The following presents a sample document
% with two chapters, two parts, a title page,
% a compile flag as well as three forwarding files to set the flag.
% It consists of eight |.tex| files:
% \begin{center}
% \begin{tabular}{ll}
% |cdocsamp.tex|&main file\\
% |cdocsch1.tex|&include file for chapter 1\\
% |cdocsch2.tex|&include file for chapter 2\\
% |cdocspt3.tex|&include file for part 3\\
% |cdocspt4.tex|&include file for part 4\\
% |cdocsdrf.tex|&forwarding file for main file in draft mode\\
% |cdocsfi1.tex|&forwarding file for final version of chapter 1\\
% |cdocsfi2.tex|&forwarding file for final version of chapter 2\\
% \end{tabular}
% \end{center}
% Each of the eight files can be compiled directly by the \LaTeX{} compiler.
%
% %%%%%%%%%%%%%%%%%%%%%%%%%%%%%%%%%%%%%%
% \paragraph{Main File.}
%
% The main file is called |cdocsamp.tex|.
%
% Load the \textsf{childdoc} definitions and
% declare the filename for the main document:
%    \begin{macrocode}
\input{childdoc.def}
\childdocmain{}
%    \end{macrocode}

% Optional override for |\version| flag:
%    \begin{macrocode}
%%\ifchilddoc\else\providecommand{\version}{draft}\fi
%    \end{macrocode}

% Define the default values for the |\version| flag
% (|final| for the main file and |draft| for childs):
%    \begin{macrocode}
\ifchilddoc
\providecommand{\version}{draft}
\else
\providecommand{\version}{final}
\fi
%    \end{macrocode}

% Load the standard document class:
%    \begin{macrocode}
\documentclass[12pt]{article}
%    \end{macrocode}

% Start the document body:
%    \begin{macrocode}
\begin{document}
%    \end{macrocode}

% Declare a title page.
% Print title, part of document being processed and version flag:
%    \begin{macrocode}
\addtocounter{page}{-1}
\begin{center}
{\LARGE\bfseries{}childdoc example\par}
\vspace{1cm}
\ifchilddoc
\ifchilddocmanual part\else chapter\fi:
`\childdocname' of `\childdocjob'\par
\else
main document: `\childdocjob'\par
\fi
version: \version\par
\end{center}
\newpage
%    \end{macrocode}

% Manually include selected file,
% otherwise process as usual:
%    \begin{macrocode}
\ifchilddocmanual
\section*{part `\childdocname'}
\input{\childdocname}
\else
%    \end{macrocode}

% Include the two chapters:
%    \begin{macrocode}
\include{cdocsch1}
\include{cdocsch2}
%    \end{macrocode}

% Include the two parts unless only chapters should be displayed:
%    \begin{macrocode}
\ifchilddoc\else
\section{part three}
\input{cdocspt3}
\section{part four}
\input{cdocspt4}
\fi
%    \end{macrocode}

% Process as usual until here:
%    \begin{macrocode}
\fi
%    \end{macrocode}

% End of document body:
%    \begin{macrocode}
\end{document}
%    \end{macrocode}
%\iffalse
%</samplemain>
%\fi
%
% %%%%%%%%%%%%%%%%%%%%%%%%%%%%%%%%%%%%%%
% \paragraph{Chapter Include Files.}
%
% The include files are called |cdocsch1.tex| and |cdocsch2.tex|.
%
%\iffalse
%<*samplechap1|samplechap2>
%\fi

% Optional override for |\version| flag:
%    \begin{macrocode}
%%\providecommand{\version}{final}
%    \end{macrocode}

% Include the main document:
%    \begin{macrocode}
\input{childdoc.def}
\childdocof{cdocsamp}
%    \end{macrocode}

%\iffalse
%</samplechap1|samplechap2>
%\fi
%
%\iffalse
%<*samplechap1>
%\fi
% Some text for chapter 1:
%    \begin{macrocode}
\section{one}
some text in chapter one
%    \end{macrocode}

%\iffalse
%</samplechap1>
%\fi
% Some text for chapter 2:
%\iffalse
%<*samplechap2>
%\fi
%    \begin{macrocode}
\section{two}
more text in chapter two
%    \end{macrocode}

%\iffalse
%</samplechap2>
%\fi
%
% %%%%%%%%%%%%%%%%%%%%%%%%%%%%%%%%%%%%%%
% \paragraph{Part Include Files.}
%
% The include files are called |cdocspt3.tex| and |cdocspt4.tex|.
%
%\iffalse
%<*samplepart3|samplepart4>
%\fi

% Optional override for |\version| flag:
%    \begin{macrocode}
%%\providecommand{\version}{final}
%    \end{macrocode}

% Include the main document:
%    \begin{macrocode}
\input{childdoc.def}
\childdocby{cdocsamp}
%    \end{macrocode}

%\iffalse
%</samplepart3|samplepart4>
%\fi
%
%\iffalse
%<*samplepart3>
%\fi
% Some text for part 3:
%    \begin{macrocode}
some text in part three
%    \end{macrocode}

%\iffalse
%</samplepart3>
%\fi
% Some text for part 4:
%\iffalse
%<*samplepart4>
%\fi
%    \begin{macrocode}
more text in part four
%    \end{macrocode}

%\iffalse
%</samplepart4>
%\fi
%
% %%%%%%%%%%%%%%%%%%%%%%%%%%%%%%%%%%%%%%
% \paragraph{Forwarding for a Complete Draft.}
%
% The following forwarding file |cdocsdrf.tex|
% compiles the main document in draft mode:
%\iffalse
%<*sampledraft>
%\fi
%    \begin{macrocode}
\def\version{draft}
\input{childdoc.def}
\childdocforward{cdocsamp}
%    \end{macrocode}

%\iffalse
%</sampledraft>
%\fi
%
% %%%%%%%%%%%%%%%%%%%%%%%%%%%%%%%%%%%%%%
% \paragraph{Forwarding for Final Version of the Chapters.}
%
% The following forwarding files |cdocsfn1.tex| and |cdocsfn2.tex|
% (with identical content)
% compile the final versions of the child documents
% |cdocsch1.tex| and |cdocsch2.tex|, respectively:
%\iffalse
%<*samplefinal>
%\fi
%    \begin{macrocode}
\def\version{final}
\input{childdoc.def}
\childdocforwardprefix[cdocsamp]{cdocsfn}{cdocsch}
%    \end{macrocode}

%\iffalse
%</samplefinal>
%\fi
%
% %%%%%%%%%%%%%%%%%%%%%%%%%%%%%%%%%%%%%%
% \paragraph{Command Line Processing.}
%
% The following three command lines generate the output files
% |cdocscld|, |cdocscl1| and |cdocscl2|
% which should be identical to
% |cdocsdrf|, |cdocsch1| and |cdocsfn2|, respectively:
% \begin{center}
% \begin{tabular}{l}
% |latex -jobname cdocscld \|\\
% |  "\def\version{draft}\input{childdoc.def}\childdocforward{cdocsamp}"|\\
% |latex -jobname cdocscl1 \|\\
% |  "\input{childdoc.def}\childdocforward[cdocsamp]{cdocsch1}"|\\
% |latex -jobname cdocscl2 \|\\
% |  "\def\version{final}\input{childdoc.def}\childdocforward{cdocsch2}"|
% \end{tabular}
% \end{center}
% Note that the trailing backslash on each first line
% merely continues the input to the second line
% (for convenient cut ant paste).
% Furthermore, the command |latex| can be replaced by any
% of its alternative versions such as |pdflatex|.
%
% %%%%%%%%%%%%%%%%%%%%%%%%%%%%%%%%%%%%%%%%%%%%%%%%%%%%%%%%%%%%%%%%%%%%%%%%%%%%%%
% %%%%%%%%%%%%%%%%%%%%%%%%%%%%%%%%%%%%%%%%%%%%%%%%%%%%%%%%%%%%%%%%%%%%%%%%%%%%%%
% \section{Implementation}
%\iffalse
%<*package>
%\fi
%
% This section describes the definitions file |childdoc.def|.

% The definitions cannot be loaded using |\usepackage| or |\RequirePackage|
% which has a mechanism to prevent loading a style file more than once.
% When loading the definitions by means of |\input|
% multiple instances have to be prevented manually:
%\iffalse
%This code needs to be before the `\ProvidesFile' directive
%which is defined at the beginning of this file.
%Therefore it is also placed there and commented out here.
%</package>
%<*discard>
%\fi
%    \begin{macrocode}
\ifdefined\childdocmain\endinput\fi
%    \end{macrocode}
%\iffalse
%</discard>
%<*package>
%\fi
%
% \macro{\ifchilddoc}
% \macro{\ifchilddocmanual}
% The conditional |\ifchilddoc| tells whether a
% child (true) or main (false) document is being compiled.
% The conditional |\ifchilddocmanual| tells whether
% the |\includeonly| mechanism is used (false) or
% the selection of child files must be performed manually (true).
% The definitions initialise to false:
%    \begin{macrocode}
\newif\ifchilddoc
\newif\ifchilddocmanual
%    \end{macrocode}

% \macro{\childdocname}
% \macro{\childdocjob}
% The macro |\childdocname| stores the name of the main document
% to be compiled. The macro |\childdocjob| stores the name of
% the document on which the \LaTeX{} compiler was originally invoked.
% The content of |\jobname| cannot be compared
% to filenames specified in the source due to different catcodes.
% The following code rescans |\jobname|, stores the result
% in |\childdocname| and saves a copy in |\childdocjob|:
%    \begin{macrocode}
\edef\childdocname{\scantokens\expandafter{\jobname\noexpand}}
\let\childdocjob\childdocname
%    \end{macrocode}

% \macro{\childdocdisable}
% The macro |\childdocdisable| prevents the main file
% from being processed more than once.
% At this stage, the main document command |\childdocmain|
% is assumed to be called once again where it should do nothing.
% Any subsequent call to it should prevent
% a secondary processing of the main document
% It overwrites the forwarding commands
% |\childdocof| and |\childdocforward|
% with empty macros to prevent further inclusions of the main document:
%    \begin{macrocode}
\newcommand{\childdocdisable}
{
  \renewcommand{\childdocmain}[1]{\renewcommand{\childdocmain}[1]{\endinput}}
  \renewcommand{\childdocof}[1]{}
  \renewcommand{\childdocby}[2][]{}
  \renewcommand{\childdocforward}[2][]{}
  \renewcommand{\childdocdisable}{}
}
%    \end{macrocode}

% \macro{\childdocmain}
% The macro |\childdocmain| is to be called at the top of the main file
% with nothing or the main filename (without extension) as argument.
% First, it breaks loops.
% If the argument is not empty and does not match |\childdocname|
% (which is set by the first inclusion of |childdoc.def|),
% |\ifchilddoc| is set to true, |\includeonly| is applied to the child file
% and |\jobname| is set to the main file
% (for proper handling of |.aux| files):
%    \begin{macrocode}
\newcommand{\childdocmain}[1]
{
  \childdocdisable\childdocmain{}
  \if?#1?\else
    \begingroup
      \def\childdoctmp{#1}
      \ifx\childdoctmp\childdocname
        \def\childdoctmp{}
      \else
        \def\childdoctmp
        {
          \childdoctrue
          \includeonly{\childdocname}
          \def\childdocjob{#1}
          \def\jobname{#1}
        }
      \fi
      \expandafter
    \endgroup
    \childdoctmp
  \fi
}
%    \end{macrocode}

% \macro{\childdocof}
% The command |\childdocof| redirects
% compilation to the main file |#1|.
%    \begin{macrocode}
\newcommand{\childdocof}[1]
{
  \childdocdisable
  \childdoctrue
  \includeonly{\childdocname}
  \def\jobname{#1}
  \def\childdocjob{#1}
  \input{#1}
}
%    \end{macrocode}

% \macro{\childdocby}
% The command |\childdocby| ....
%    \begin{macrocode}
\newcommand{\childdocby}[2][]
{
  \childdocdisable
  \childdoctrue
  \childdocmanualtrue
  \if?#1?\else
    \def\jobname{#2}
  \fi
  \def\childdocjob{#2}
  \input{#2}
  \endinput
}
%    \end{macrocode}

% \macro{\childdocforward}
% The command |\childdocforward| redirects
% compilation to the main file or
% (if the optional argument is given) a child file.
% Parameters are set as if the main file
% or a child file starting with |\childdocof| was compiled.
% Then compilation is handed over to the main file:
%    \begin{macrocode}
\newcommand{\childdocforward}[2][]
{
  \begingroup
    \if?#1?
      \def\childdoctmp
      {
        \def\childdocname{#2}
        \def\childdocjob{#2}
        \def\jobname{#2}
        \input{#2}
        \endinput
      }
    \else
      \def\childdoctmp
      {
        \childdocdisable
        \def\childdocname{#2}
        \childdoctrue
        \includeonly{#2}
        \def\childdocjob{#1}
        \def\jobname{#1}
        \input{#1}
        \endinput
      }
    \fi
    \expandafter
  \endgroup
  \childdoctmp
}
%    \end{macrocode}

% \macro{\childdocforwardprefix}
% The command |\childdocforwardprefix| redirects
% compilation to the main or a child file by means of a pattern.
% The prefix |#1| in the current filename is replaced by |#2|
% and the suffix of the current filename is kept
% (it is assumed that the filename does not contain the substring `|~~~|'
% which is used as a delimiter).
% Compilation is handed over to the new file by |\childdocforward|:
%    \begin{macrocode}
\newcommand{\childdocforwardprefix}[3][]
{
  \begingroup
    \def\childdocextract #2##1~~~{\def\childdoctmp{\childdocforward[#1]{#3##1}}}
    \expandafter\childdocextract\childdocname~~~
    \expandafter
  \endgroup
  \childdoctmp
}
%    \end{macrocode}

% \macro{\childdoc}
% The deprecated macro |\childdoc| is a legacy version of |\childdocmain|:
%    \begin{macrocode}
\newcommand{\childdoc}{\childdocmain}
%    \end{macrocode}

% \macro{\childdocredirect}
% The deprecated macro |\childdocredirect| is a legacy version
% of |\childdocforward| and |\childdocforwardprefix|:
%    \begin{macrocode}
\newcommand{\childdocredirect}[2][]
{
  \begingroup
    \if?#1?
      \def\childdoctmp{\childdocforward{#2}}
    \else
      \def\childdoctmp{\childdocforwardprefix{#1}{#2}}
    \fi
    \expandafter
  \endgroup
  \childdoctmp
}
%    \end{macrocode}

%\iffalse
%</package>
%\fi
%
\endinput
\childdocforward{cdocsch2}"|
% \end{tabular}
% \end{center}
% Note that the trailing backslash on each first line
% merely continues the input to the second line
% (for convenient cut ant paste).
% Furthermore, the command |latex| can be replaced by any
% of its alternative versions such as |pdflatex|.
%
% %%%%%%%%%%%%%%%%%%%%%%%%%%%%%%%%%%%%%%%%%%%%%%%%%%%%%%%%%%%%%%%%%%%%%%%%%%%%%%
% %%%%%%%%%%%%%%%%%%%%%%%%%%%%%%%%%%%%%%%%%%%%%%%%%%%%%%%%%%%%%%%%%%%%%%%%%%%%%%
% \section{Implementation}
%\iffalse
%<*package>
%\fi
%
% This section describes the definitions file |childdoc.def|.

% The definitions cannot be loaded using |\usepackage| or |\RequirePackage|
% which has a mechanism to prevent loading a style file more than once.
% When loading the definitions by means of |\input|
% multiple instances have to be prevented manually:
%\iffalse
%This code needs to be before the `\ProvidesFile' directive
%which is defined at the beginning of this file.
%Therefore it is also placed there and commented out here.
%</package>
%<*discard>
%\fi
%    \begin{macrocode}
\ifdefined\childdocmain\endinput\fi
%    \end{macrocode}
%\iffalse
%</discard>
%<*package>
%\fi
%
% \macro{\ifchilddoc}
% \macro{\ifchilddocmanual}
% The conditional |\ifchilddoc| tells whether a
% child (true) or main (false) document is being compiled.
% The conditional |\ifchilddocmanual| tells whether
% the |\includeonly| mechanism is used (false) or
% the selection of child files must be performed manually (true).
% The definitions initialise to false:
%    \begin{macrocode}
\newif\ifchilddoc
\newif\ifchilddocmanual
%    \end{macrocode}

% \macro{\childdocname}
% \macro{\childdocjob}
% The macro |\childdocname| stores the name of the main document
% to be compiled. The macro |\childdocjob| stores the name of
% the document on which the \LaTeX{} compiler was originally invoked.
% The content of |\jobname| cannot be compared
% to filenames specified in the source due to different catcodes.
% The following code rescans |\jobname|, stores the result
% in |\childdocname| and saves a copy in |\childdocjob|:
%    \begin{macrocode}
\edef\childdocname{\scantokens\expandafter{\jobname\noexpand}}
\let\childdocjob\childdocname
%    \end{macrocode}

% \macro{\childdocdisable}
% The macro |\childdocdisable| prevents the main file
% from being processed more than once.
% At this stage, the main document command |\childdocmain|
% is assumed to be called once again where it should do nothing.
% Any subsequent call to it should prevent
% a secondary processing of the main document
% It overwrites the forwarding commands
% |\childdocof| and |\childdocforward|
% with empty macros to prevent further inclusions of the main document:
%    \begin{macrocode}
\newcommand{\childdocdisable}
{
  \renewcommand{\childdocmain}[1]{\renewcommand{\childdocmain}[1]{\endinput}}
  \renewcommand{\childdocof}[1]{}
  \renewcommand{\childdocby}[2][]{}
  \renewcommand{\childdocforward}[2][]{}
  \renewcommand{\childdocdisable}{}
}
%    \end{macrocode}

% \macro{\childdocmain}
% The macro |\childdocmain| is to be called at the top of the main file
% with nothing or the main filename (without extension) as argument.
% First, it breaks loops.
% If the argument is not empty and does not match |\childdocname|
% (which is set by the first inclusion of |childdoc.def|),
% |\ifchilddoc| is set to true, |\includeonly| is applied to the child file
% and |\jobname| is set to the main file
% (for proper handling of |.aux| files):
%    \begin{macrocode}
\newcommand{\childdocmain}[1]
{
  \childdocdisable\childdocmain{}
  \if?#1?\else
    \begingroup
      \def\childdoctmp{#1}
      \ifx\childdoctmp\childdocname
        \def\childdoctmp{}
      \else
        \def\childdoctmp
        {
          \childdoctrue
          \includeonly{\childdocname}
          \def\childdocjob{#1}
          \def\jobname{#1}
        }
      \fi
      \expandafter
    \endgroup
    \childdoctmp
  \fi
}
%    \end{macrocode}

% \macro{\childdocof}
% The command |\childdocof| redirects
% compilation to the main file |#1|.
%    \begin{macrocode}
\newcommand{\childdocof}[1]
{
  \childdocdisable
  \childdoctrue
  \includeonly{\childdocname}
  \def\jobname{#1}
  \def\childdocjob{#1}
  \input{#1}
}
%    \end{macrocode}

% \macro{\childdocby}
% The command |\childdocby| ....
%    \begin{macrocode}
\newcommand{\childdocby}[2][]
{
  \childdocdisable
  \childdoctrue
  \childdocmanualtrue
  \if?#1?\else
    \def\jobname{#2}
  \fi
  \def\childdocjob{#2}
  \input{#2}
  \endinput
}
%    \end{macrocode}

% \macro{\childdocforward}
% The command |\childdocforward| redirects
% compilation to the main file or
% (if the optional argument is given) a child file.
% Parameters are set as if the main file
% or a child file starting with |\childdocof| was compiled.
% Then compilation is handed over to the main file:
%    \begin{macrocode}
\newcommand{\childdocforward}[2][]
{
  \begingroup
    \if?#1?
      \def\childdoctmp
      {
        \def\childdocname{#2}
        \def\childdocjob{#2}
        \def\jobname{#2}
        \input{#2}
        \endinput
      }
    \else
      \def\childdoctmp
      {
        \childdocdisable
        \def\childdocname{#2}
        \childdoctrue
        \includeonly{#2}
        \def\childdocjob{#1}
        \def\jobname{#1}
        \input{#1}
        \endinput
      }
    \fi
    \expandafter
  \endgroup
  \childdoctmp
}
%    \end{macrocode}

% \macro{\childdocforwardprefix}
% The command |\childdocforwardprefix| redirects
% compilation to the main or a child file by means of a pattern.
% The prefix |#1| in the current filename is replaced by |#2|
% and the suffix of the current filename is kept
% (it is assumed that the filename does not contain the substring `|~~~|'
% which is used as a delimiter).
% Compilation is handed over to the new file by |\childdocforward|:
%    \begin{macrocode}
\newcommand{\childdocforwardprefix}[3][]
{
  \begingroup
    \def\childdocextract #2##1~~~{\def\childdoctmp{\childdocforward[#1]{#3##1}}}
    \expandafter\childdocextract\childdocname~~~
    \expandafter
  \endgroup
  \childdoctmp
}
%    \end{macrocode}

% \macro{\childdoc}
% The deprecated macro |\childdoc| is a legacy version of |\childdocmain|:
%    \begin{macrocode}
\newcommand{\childdoc}{\childdocmain}
%    \end{macrocode}

% \macro{\childdocredirect}
% The deprecated macro |\childdocredirect| is a legacy version
% of |\childdocforward| and |\childdocforwardprefix|:
%    \begin{macrocode}
\newcommand{\childdocredirect}[2][]
{
  \begingroup
    \if?#1?
      \def\childdoctmp{\childdocforward{#2}}
    \else
      \def\childdoctmp{\childdocforwardprefix{#1}{#2}}
    \fi
    \expandafter
  \endgroup
  \childdoctmp
}
%    \end{macrocode}

%\iffalse
%</package>
%\fi
%
\endinput
\childdocforward{cdocsamp}"|\\
% |latex -jobname cdocscl1 \|\\
% |  "% \iffalse
%
% childdoc.dtx Copyright (C) 2017-2018 Niklas Beisert
%
% This work may be distributed and/or modified under the
% conditions of the LaTeX Project Public License, either version 1.3
% of this license or (at your option) any later version.
% The latest version of this license is in
%   http://www.latex-project.org/lppl.txt
% and version 1.3 or later is part of all distributions of LaTeX
% version 2005/12/01 or later.
%
% This work has the LPPL maintenance status `maintained'.
%
% The Current Maintainer of this work is Niklas Beisert.
%
% This work consists of the files childdoc.dtx and childdoc.ins
% and the derived files childdoc.def and cdocsamp.tex with
% cdocsch1.tex, cdocsch2.tex, cdocsdrf.tex, cdocsfn1.tex, cdocsfn2.tex.
%
%<package>\ifdefined\childdocmain\endinput\fi
%<package>\ProvidesFile{childdoc.def}[2018/12/30 v2.0 child document driver]
%<samplemain>\ProvidesFile{cdocsamp.tex}[2018/12/30 v2.0 sample for childdoc]
%<*driver>
%\ProvidesFile{childdoc.drv}[2018/12/30 v2.0 childdoc reference manual file]
\PassOptionsToClass{10pt,a4paper}{article}
\documentclass{ltxdoc}

\usepackage[margin=35mm]{geometry}
\usepackage{hyperref}
\usepackage{hyperxmp}
\usepackage[usenames]{color}

\hypersetup{colorlinks=true}
\hypersetup{pdfstartview=FitH}
\hypersetup{pdfpagemode=UseNone}
\hypersetup{pdfsource={}}
\hypersetup{pdflang={en-UK}}
\hypersetup{pdfcopyright={Copyright 2017-2018 Niklas Beisert.
  This work may be distributed and/or modified under the
  conditions of the LaTeX Project Public License, either version 1.3
  of this license or (at your option) any later version.}}
\hypersetup{pdflicenseurl={http://www.latex-project.org/lppl.txt}}
\hypersetup{pdfcontactaddress={ETH Zurich, ITP, HIT K,
  Wolfgang-Pauli-Strasse 27}}
\hypersetup{pdfcontactpostcode={8093}}
\hypersetup{pdfcontactcity={Zurich}}
\hypersetup{pdfcontactcountry={Switzerland}}
\hypersetup{pdfcontactemail={nbeisert@itp.phys.ethz.ch}}
\hypersetup{pdfcontacturl={http://people.phys.ethz.ch/\xmptilde nbeisert/}}

\newcommand{\secref}[1]{\hyperref[#1]{section \ref*{#1}}}

\parskip1ex
\parindent0pt
\let\olditemize\itemize
\def\itemize{\olditemize\parskip0pt}

\begin{document}

\title{The \textsf{childdoc} Package}
\hypersetup{pdftitle={The childdoc Package}}
\author{Niklas Beisert\\[2ex]
  Institut f\"ur Theoretische Physik\\
  Eidgen\"ossische Technische Hochschule Z\"urich\\
  Wolfgang-Pauli-Strasse 27, 8093 Z\"urich, Switzerland\\[1ex]
  \href{mailto:nbeisert@itp.phys.ethz.ch}
  {\texttt{nbeisert@itp.phys.ethz.ch}}}
\hypersetup{pdfauthor={Niklas Beisert}}
\hypersetup{pdfsubject={Manual for the LaTeX2e Package childdoc}}
\date{30 December 2018, \textsf{v2.0}}
\maketitle

\begin{abstract}\noindent
\textsf{childdoc} is a \LaTeXe{} package
that enables the direct compilation
of document sections included by |\include|
to individual files.
\end{abstract}

\begingroup
\parskip0ex
\tableofcontents
\endgroup

%%%%%%%%%%%%%%%%%%%%%%%%%%%%%%%%%%%%%%%%%%%%%%%%%%%%%%%%%%%%%%%%%%%%%%%%%%%%%%%%
%%%%%%%%%%%%%%%%%%%%%%%%%%%%%%%%%%%%%%%%%%%%%%%%%%%%%%%%%%%%%%%%%%%%%%%%%%%%%%%%
\section{Introduction}

\LaTeX{} provides a mechanism to structure a large document (such as a book)
into a main file and several child files (containing the chapters)
using the |\include| command.
This mechanism is beneficial for documents
which span hundreds of pages in order to
make the source file(s) more manageable.
Moreover, compilation can be restricted to
selected child files by means of the |\includeonly| command.
The latter feature can be used to reduce the compilation time while editing
(this was significantly more useful in the earlier days of \LaTeX{})
or to generate a smaller document which is easier to navigate.
Another application of |\includeonly| is to generate
documents consisting of selected parts of the complete document.

However, there are a few drawbacks of the plain |\include| mechanism:
\begin{itemize}
\item
The child files cannot be compiled on their own,
they can only be compiled via the main file.
A naive editing environment
(such as a text editor with an option
to have the current file processed by \LaTeX)
may require one to switch to the main file before compiling;
attempting to compile the child file produces errors.
\item
The main file must be modified (each time)
to adjust the |\includeonly| command
to the present needs. This easily leaves the main file in a messy state.
\item
The generated document will always carry the filename
of the main document. This is inconvenient if
several child files are to be compiled and
to be kept for distribution.
\end{itemize}

The present package provides a simple interface
to make child files individually compilable by \LaTeX{}.
Compiling a child file then has the same effect as compiling
the main file with an |\includeonly| command
to select the appropriate child.
Moreover the generated document will carry the name of the child
rather than the main file.
This resolves all three above issues.

This feature is meant to make the editing of books,
thesis documents and lecture notes somewhat more convenient.
However, the package can also be used efficiently for
composing a series of documents (such as exercise sheets)
which are typically distributed individually.
It then assists the author in generating the individual documents
(potentially in different versions)
as well as a document containing the collected series.
Another application is in developing style files
or other kinds of included material
where compilation of the style file could redirect
to a sample or test file.

%%%%%%%%%%%%%%%%%%%%%%%%%%%%%%%%%%%%%%%%%%%%%%%%%%%%%%%%%%%%%%%%%%%%%%%%%%%%%%%%
%%%%%%%%%%%%%%%%%%%%%%%%%%%%%%%%%%%%%%%%%%%%%%%%%%%%%%%%%%%%%%%%%%%%%%%%%%%%%%%%
\section{Usage}

First of all, the package \textsf{childdoc} is \emph{not} a standard
\LaTeXe{} |.sty| style file! Therefore it needs to be invoked in
a non-standard way.

%%%%%%%%%%%%%%%%%%%%%%%%%%%%%%%%%%%%%%%%%%%%%%%%%%%%%%%%%%%%%%%%%%%%%%%%%%%%%%%%
\subsection{Included Files}
\label{sec:include}

%%%%%%%%%%%%%%%%%%%%%%%%%%%%%%%%%%%%%%%%
\DescribeMacro{\childdocmain}
To use the package, add the commands
\begin{center}
\begin{tabular}{l}
|% \iffalse
%
% childdoc.dtx Copyright (C) 2017-2018 Niklas Beisert
%
% This work may be distributed and/or modified under the
% conditions of the LaTeX Project Public License, either version 1.3
% of this license or (at your option) any later version.
% The latest version of this license is in
%   http://www.latex-project.org/lppl.txt
% and version 1.3 or later is part of all distributions of LaTeX
% version 2005/12/01 or later.
%
% This work has the LPPL maintenance status `maintained'.
%
% The Current Maintainer of this work is Niklas Beisert.
%
% This work consists of the files childdoc.dtx and childdoc.ins
% and the derived files childdoc.def and cdocsamp.tex with
% cdocsch1.tex, cdocsch2.tex, cdocsdrf.tex, cdocsfn1.tex, cdocsfn2.tex.
%
%<package>\ifdefined\childdocmain\endinput\fi
%<package>\ProvidesFile{childdoc.def}[2018/12/30 v2.0 child document driver]
%<samplemain>\ProvidesFile{cdocsamp.tex}[2018/12/30 v2.0 sample for childdoc]
%<*driver>
%\ProvidesFile{childdoc.drv}[2018/12/30 v2.0 childdoc reference manual file]
\PassOptionsToClass{10pt,a4paper}{article}
\documentclass{ltxdoc}

\usepackage[margin=35mm]{geometry}
\usepackage{hyperref}
\usepackage{hyperxmp}
\usepackage[usenames]{color}

\hypersetup{colorlinks=true}
\hypersetup{pdfstartview=FitH}
\hypersetup{pdfpagemode=UseNone}
\hypersetup{pdfsource={}}
\hypersetup{pdflang={en-UK}}
\hypersetup{pdfcopyright={Copyright 2017-2018 Niklas Beisert.
  This work may be distributed and/or modified under the
  conditions of the LaTeX Project Public License, either version 1.3
  of this license or (at your option) any later version.}}
\hypersetup{pdflicenseurl={http://www.latex-project.org/lppl.txt}}
\hypersetup{pdfcontactaddress={ETH Zurich, ITP, HIT K,
  Wolfgang-Pauli-Strasse 27}}
\hypersetup{pdfcontactpostcode={8093}}
\hypersetup{pdfcontactcity={Zurich}}
\hypersetup{pdfcontactcountry={Switzerland}}
\hypersetup{pdfcontactemail={nbeisert@itp.phys.ethz.ch}}
\hypersetup{pdfcontacturl={http://people.phys.ethz.ch/\xmptilde nbeisert/}}

\newcommand{\secref}[1]{\hyperref[#1]{section \ref*{#1}}}

\parskip1ex
\parindent0pt
\let\olditemize\itemize
\def\itemize{\olditemize\parskip0pt}

\begin{document}

\title{The \textsf{childdoc} Package}
\hypersetup{pdftitle={The childdoc Package}}
\author{Niklas Beisert\\[2ex]
  Institut f\"ur Theoretische Physik\\
  Eidgen\"ossische Technische Hochschule Z\"urich\\
  Wolfgang-Pauli-Strasse 27, 8093 Z\"urich, Switzerland\\[1ex]
  \href{mailto:nbeisert@itp.phys.ethz.ch}
  {\texttt{nbeisert@itp.phys.ethz.ch}}}
\hypersetup{pdfauthor={Niklas Beisert}}
\hypersetup{pdfsubject={Manual for the LaTeX2e Package childdoc}}
\date{30 December 2018, \textsf{v2.0}}
\maketitle

\begin{abstract}\noindent
\textsf{childdoc} is a \LaTeXe{} package
that enables the direct compilation
of document sections included by |\include|
to individual files.
\end{abstract}

\begingroup
\parskip0ex
\tableofcontents
\endgroup

%%%%%%%%%%%%%%%%%%%%%%%%%%%%%%%%%%%%%%%%%%%%%%%%%%%%%%%%%%%%%%%%%%%%%%%%%%%%%%%%
%%%%%%%%%%%%%%%%%%%%%%%%%%%%%%%%%%%%%%%%%%%%%%%%%%%%%%%%%%%%%%%%%%%%%%%%%%%%%%%%
\section{Introduction}

\LaTeX{} provides a mechanism to structure a large document (such as a book)
into a main file and several child files (containing the chapters)
using the |\include| command.
This mechanism is beneficial for documents
which span hundreds of pages in order to
make the source file(s) more manageable.
Moreover, compilation can be restricted to
selected child files by means of the |\includeonly| command.
The latter feature can be used to reduce the compilation time while editing
(this was significantly more useful in the earlier days of \LaTeX{})
or to generate a smaller document which is easier to navigate.
Another application of |\includeonly| is to generate
documents consisting of selected parts of the complete document.

However, there are a few drawbacks of the plain |\include| mechanism:
\begin{itemize}
\item
The child files cannot be compiled on their own,
they can only be compiled via the main file.
A naive editing environment
(such as a text editor with an option
to have the current file processed by \LaTeX)
may require one to switch to the main file before compiling;
attempting to compile the child file produces errors.
\item
The main file must be modified (each time)
to adjust the |\includeonly| command
to the present needs. This easily leaves the main file in a messy state.
\item
The generated document will always carry the filename
of the main document. This is inconvenient if
several child files are to be compiled and
to be kept for distribution.
\end{itemize}

The present package provides a simple interface
to make child files individually compilable by \LaTeX{}.
Compiling a child file then has the same effect as compiling
the main file with an |\includeonly| command
to select the appropriate child.
Moreover the generated document will carry the name of the child
rather than the main file.
This resolves all three above issues.

This feature is meant to make the editing of books,
thesis documents and lecture notes somewhat more convenient.
However, the package can also be used efficiently for
composing a series of documents (such as exercise sheets)
which are typically distributed individually.
It then assists the author in generating the individual documents
(potentially in different versions)
as well as a document containing the collected series.
Another application is in developing style files
or other kinds of included material
where compilation of the style file could redirect
to a sample or test file.

%%%%%%%%%%%%%%%%%%%%%%%%%%%%%%%%%%%%%%%%%%%%%%%%%%%%%%%%%%%%%%%%%%%%%%%%%%%%%%%%
%%%%%%%%%%%%%%%%%%%%%%%%%%%%%%%%%%%%%%%%%%%%%%%%%%%%%%%%%%%%%%%%%%%%%%%%%%%%%%%%
\section{Usage}

First of all, the package \textsf{childdoc} is \emph{not} a standard
\LaTeXe{} |.sty| style file! Therefore it needs to be invoked in
a non-standard way.

%%%%%%%%%%%%%%%%%%%%%%%%%%%%%%%%%%%%%%%%%%%%%%%%%%%%%%%%%%%%%%%%%%%%%%%%%%%%%%%%
\subsection{Included Files}
\label{sec:include}

%%%%%%%%%%%%%%%%%%%%%%%%%%%%%%%%%%%%%%%%
\DescribeMacro{\childdocmain}
To use the package, add the commands
\begin{center}
\begin{tabular}{l}
|\input{childdoc.def}|\\
|\childdocmain{}|\\
\end{tabular}
\end{center}
at the very top of the main \LaTeX{} file,
in particular \emph{before} the |\documentclass| statement!
The argument of |\childdocmain| should be left empty
(but it must be present).

%%%%%%%%%%%%%%%%%%%%%%%%%%%%%%%%%%%%%%%%
\DescribeMacro{\childdocof}
Furthermore, add the commands
\begin{center}
\begin{tabular}{l}
|\input{childdoc.def}|\\
|\childdocof{|\textit{main}|}|\\
\end{tabular}
\end{center}
at the top of every child file \textit{child}
which is included by |\include{|\textit{child}|}|
from within the main file
(or at least for those files to be compiled individually).
The argument \textit{main} must be the filename of the main file.

There are a couple of
considerations in setting up the main and child documents:

%%%%%%%%%%%%%%%%%%%%%%%%%%%%%%%%%%%%%%%%
\paragraph{Restrictions.}

Please note the following restrictions:
\begin{itemize}
\item
|\childdocmain| must be called with one argument \textit{main}
to ensure compatibility with earlier version of the package.
It must either be empty (|\childdocmain{}|)
or precisely match the filename of the main file in which it is specified.
See \secref{sec:detection} for further information.
\item
The filename \textit{main} must be specified without the |.tex| extension.
\item
The filename \textit{main} is case sensitive
(even in case-insensitive file systems)
due to internal string comparison.
\item
The argument \textit{main} should be fully expanded, it cannot be a macro.
\item
Subdirectories and special characters should be avoided in filenames.
\item
The command |\childdocmain{|\textit{main}|}| must be followed by a whitespace.
It should not be followed immediately by another command
or by a comment mark `|%|'.
This is because the \TeX{} parser reads the token immediately following
the argument of |\childdocmain| and puts it
at the beginning of every child section;
however, a white\-space is ignored.
\end{itemize}

%%%%%%%%%%%%%%%%%%%%%%%%%%%%%%%%%%%%%%%%
\paragraph{Content of Main File.}

It is advisable to place all content in the child files included by |\include|.
Any output contained in the main file will appear in all child documents
unless suppressed manually;
it cannot be suppressed automatically by the |\includeonly| directive
and thus should normally be avoided.
A method to include some content in the main file
by means of conditional processing is described in \secref{sec:conditional}.

%%%%%%%%%%%%%%%%%%%%%%%%%%%%%%%%%%%%%%%%
\paragraph{Page Numbering.}

When only a part of the document is compiled,
the appropriate numbering of pages
(as well as other status parameters)
is determined from the |.aux| files.
The latter contain information from previous passes.
However this information needs to propagate through
all intermediate child documents.
Therefore the page numbering in child documents may well
be inconsistent until the complete document is compiled at least once.

A useful (if unconventional) way to always ensure a consistent
page numbering is to restart the numbering in each child document
and denote the pages by `\textit{child}|.|\textit{page}'
where \textit{child} represents the chapter/section number of the child file.
This can be achieved by the command
|\numberwithin{page}{|\textit{child}|}|
of the \textsf{amsmath} package
where \textit{child} can be |chapter| or |section|
depending on the chosen structuring.
Alternatively, one can modify the macro |\thepage| appropriately
and reset the counter |page| at the start of each child file.

%%%%%%%%%%%%%%%%%%%%%%%%%%%%%%%%%%%%%%%%%%%%%%%%%%%%%%%%%%%%%%%%%%%%%%%%%%%%%%%%
\subsection{Conditional Processing}
\label{sec:conditional}

The package provides a mechanism to compile different versions
of a document. To customise the versions further some conditional processing
can come in handy to distinguish which version is being compiled.
The package provides two macros to describe the compilation context:

%%%%%%%%%%%%%%%%%%%%%%%%%%%%%%%%%%%%%%%%
\DescribeMacro{\ifchilddoc}
The conditional |\ifchilddoc| distinguishes between the compilation of
child documents and the main document:
%
\begin{center}
|\ifchilddoc |\textit{child-code}| |[|\||else |\textit{main-code}]| \||fi|
\end{center}

%%%%%%%%%%%%%%%%%%%%%%%%%%%%%%%%%%%%%%%%
\DescribeMacro{\childdocname}
\DescribeMacro{\childdocjob}
The macro |\childdocname| contains the filename (without extension)
of the main or child file being processed.
Note that |\childdocjob| will always contain the name of the main file.

%%%%%%%%%%%%%%%%%%%%%%%%%%%%%%%%%%%%%%%%
\paragraph{Title Page.}

Conditional processing can be used to include a title or banner page
in the main document when proper precautions are taken.
Importantly, the code in the main file should ensure that the page counter
(as well as other status parameters which are stored in the |.aux| files)
takes the same value after the conditional processing.
Otherwise the page numbers may take divergent values
depending on which part is compiled.

For example, a title page could be declared by:
%
\begin{center}
\begin{tabular}{l}
|\ifchilddoc\||else|\\
|\addtocounter{page}{-1}|\\
\textit{code for title page}\\
|\newpage|\\
|\||fi|
\end{tabular}
\end{center}
%
A banner page for the child documents can be generated by:
%
\begin{center}
\begin{tabular}{l}
|\ifchilddoc|\\
|\addtocounter{page}{-1}|\\
\textit{code for banner page}\\
|\newpage|\\
|\||fi|
\end{tabular}
\end{center}
%
Here one could write a message such as:
\begin{center}
|This is the part \childdocname{} of \childdocjob{}.|
\end{center}

%%%%%%%%%%%%%%%%%%%%%%%%%%%%%%%%%%%%%%%%%%%%%%%%%%%%%%%%%%%%%%%%%%%%%%%%%%%%%%%%
\subsection{Flags}
\label{sec:flags}

The package makes it easy to generate different versions
of the main or child documents.
To this end compilation flags can be defined
and assigned different default values.
They will be particularly useful in conjunction
with the forwarding mechanism described in \secref{sec:forward}.

For example, it may be useful to have a flag |\version|
which can be set to |draft| or |final|.
The document source will contain some conditional code
depending on the value of |\version|.
Suppose further, the flag should default to |final| for the main file
and to |draft| for child files
which is a natural assignment for editing the document.
This is achieved by placing the following code
in the preamble of the main document
(below the |\childdocmain| directive):
%
\begin{center}
\begin{tabular}{l}
|\ifchilddoc|\\
|\providecommand{\version}{draft}|\\
|\||else|\\
|\providecommand{\version}{final}|\\
|\||fi|
\end{tabular}
\end{center}
%
The definition by |\providecommand| makes sure
that previous definitions are not overwritten.
Further statements |\providecommand{\version}{...}|
can thus be added before the above code to override it.

For the main file, one might add a line
(between |\childdocmain| and the above block)
%
\begin{center}
|%\ifchilddoc\||else\providecommand{\version}{draft}\||fi|
\end{center}
%
which can be uncommented to produce a draft version.
Likewise one can add a line to the very top of a child file
(above the |\childdocof{|\textit{main}|}| directive)
%
\begin{center}
|%\providecommand{\version}{final}|
\end{center}
%
which can be uncommented to produce the final version of this child document.

%%%%%%%%%%%%%%%%%%%%%%%%%%%%%%%%%%%%%%%%%%%%%%%%%%%%%%%%%%%%%%%%%%%%%%%%%%%%%%%%
\subsection{Forwarding}
\label{sec:forward}

Different versions of the main or child documents
using compilation flags as described in \secref{sec:flags}
can be (permanently) stored in different files
for convenient compilation, viewing and distribution.
To this end, the package defines a command
to pass on compilation to a different file:

%%%%%%%%%%%%%%%%%%%%%%%%%%%%%%%%%%%%%%%%
\DescribeMacro{\childdocforward}
The command |\childdocforward| redirects processing to
another source file:
%
\begin{center}
\begin{tabular}{l}
|\input{childdoc.def}|\\
|\childdocforward[|\textit{main}|]{|\textit{dest}|}|\\
\end{tabular}
\end{center}
%
The argument \textit{dest} is the destination file
(without extension).
It should be the main file or one of the child files.
Note that further \textsf{childdoc} directives
such as |\childdocof| and |\childdocforward|
in the indicated file will be processed in this form.
The optional argument \textit{main}
passes on directly to the main file \textit{main}
while pretending to compile the child \textit{dest}.
This form behaves as if \textit{dest}
issues |\childdocof{|\textit{main}|}| right away,
and no further \textsf{childdoc} directives will be processed.

%%%%%%%%%%%%%%%%%%%%%%%%%%%%%%%%%%%%%%%%
\DescribeMacro{\...prefix}
In the alternative form |\childdocforwardprefix|,
%
\begin{center}
\begin{tabular}{l}
|\input{childdoc.def}|\\
|\childdocforwardprefix[|\textit{main}|]{|\textit{prefix}|}{|\textit{dest}|}|
\end{tabular}
\end{center}
%
the destination file is determined by a pattern
depending on the current file:
To make this work, the current file must be called
`{\textit{prefix}\hspace{0.2em}\textit{suffix}}'
with \textit{prefix} matching precisely the argument.
Processing is then passed on to the file
`{\textit{dest}\hspace{0.2em}\textit{suffix}}'.
Surely, the same effect is achieved by
directly specifying the
argument `{\textit{dest}\hspace{0.2em}\textit{suffix}}'
in the first form.
However, that requires to set up a different file
for each child. With the alternative form of the command
all these files can have exactly the same content
which simplifies setting them up and maintaining them.

For example, the following file |draft.tex|
with a compilation flag |\version| as described in \secref{sec:flags}
compiles the main document as a draft:
%
\begin{center}
\begin{tabular}{l}
|\def\version{draft}|\\
|\input{childdoc.def}|\\
|\childdocforward{|\textit{main}|}|
\end{tabular}
\end{center}
%
Likewise, the following files |final|\textit{nn}|.tex|
compile the final version of the child document
|child|\textit{nn}|.tex|:
%
\begin{center}
\begin{tabular}{l}
|\def\version{final}|\\
|\input{childdoc.def}|\\
|\childdocforwardprefix{final}{child}|
\end{tabular}
\end{center}
%

Note that when several versions of a main file and/or of each child file
are to be generated, it may be convenient to set up a |Makefile| or
shell script to automatise the process.

%%%%%%%%%%%%%%%%%%%%%%%%%%%%%%%%%%%%%%%%%%%%%%%%%%%%%%%%%%%%%%%%%%%%%%%%%%%%%%%%
\subsection{Command Line Processing}
\label{sec:commandline}

The effect of redirection files can also be achieved by invoking
the \LaTeX{} compiler with a more elaborate command line.
Most conveniently this should be done as part
of a shell script or a |Makefile|.

When using \textsf{childdoc} in the main file, the following
command lines effectively perform a redirection
(note that depending on the shell being used,
backslashes may have to be doubled: `|\|' $\to$ `|\\|'):
%
\begin{center}
|... -jobname "|\textit{target}|" |\\|"|[\textit{flags}]%
|\input{childdoc.def}\childdocforward[|\textit{main}|]{|\textit{dest}|}"|
\end{center}
%
Here \textit{target} is the name of the output file,
\textit{main} is the name of the main file
and \textit{dest} is the name of the main or child file to be processed
(all filenames without extensions).
The optional argument \textit{main} can be omitted
if \textit{main} matches \textit{dest}.
Optionally, compilation \textit{flags} can be defined via |\def| commands.
This command line makes the \TeX{} engine believe
it is compiling the file \textit{target}
whose content is specified as the latter parameter.
The provided code then forwards the processing to
\textit{main} or \textit{dest} as described in \secref{sec:forward}.

%%%%%%%%%%%%%%%%%%%%%%%%%%%%%%%%%%%%%%%%%%%%%%%%%%%%%%%%%%%%%%%%%%%%%%%%%%%%%%%%
\subsection{Include by Input}
\label{sec:input}

Including child documents by |\include| has some restrictions by design.
Most notably, the content of a child document always occupies
its own set of pages; pages cannot be shared between child documents.
Usually, this behaviour makes perfect sense
because each child document contain an essential part of the document.
However, in some situations it may be desirable to compose
a document from a collection of parts
without having mandatory page breaks between then.
For this case, the package
provides a mechanism to include parts
by |\input| which can also be processed individually.
However, by construction this mechanism
requires manual handling of the content to be output.

%%%%%%%%%%%%%%%%%%%%%%%%%%%%%%%%%%%%%%%%
\DescribeMacro{\ifchilddocmanual}
The main file should be prepared as usual, see \secref{sec:include}.
However, the document body must make a distinction
between processing of an individual part and of the main document, e.g.:
%
\begin{center}
\begin{tabular}{l}
|\ifchilddocmanual|\\
|\input{\childdocname}|\\
|\||else|\\
\textit{document body with }|\input{|\textit{part}|}|\\
|\||fi|
\end{tabular}
\end{center}
%
The conditional |\ifchilddocmanual| is true whenever
a part to be included by |\input| is being compiled,
and the name of the part is stored in |\childdocname|.

%%%%%%%%%%%%%%%%%%%%%%%%%%%%%%%%%%%%%%%%
\DescribeMacro{\childdocby}
Each part to be included by |\input| should start with:
%
\begin{center}
\begin{tabular}{l}
|\input{childdoc.def}|\\
|\childdocby{|\textit{main}|}|\\
\end{tabular}
\end{center}
%
The directive |\childdocby| is similar to |\childdocof|
described in \secref{sec:include},
but the subsequent selection of content must be done manually.
To that end, both |\ifchilddoc| and |\ifchilddocmanual|
will be true upon processing of a part,
and the name of the part is stored in |\childdocname|.
Note that |\jobname| will be set to the filename of the current part
so that each part receives an individual |.aux| file
that does not interfere with the |.aux| file(s) of the main document.
This behaviour can be altered by the alternative form
|\childdocby[*]{|\textit{main}|}| (with a non-empty optional argument)
which uses the |.aux| file of the main document
by setting |\jobname| to \textit{main}.

%%%%%%%%%%%%%%%%%%%%%%%%%%%%%%%%%%%%%%%%%%%%%%%%%%%%%%%%%%%%%%%%%%%%%%%%%%%%%%%%
\subsection{Driver Development}
\label{sec:driver}

The \textsf{childdoc} mechanism can also be use for the development
of definition files such as \LaTeX{} styles or classes.
This case differs from the above setup with multiple parts
included by |\include| in that no |\includeonly| should be invoked.
This can be achieved by starting the include file
(before |\ProvidesPackage|) with:
%
\begin{center}
\begin{tabular}{l}
|\input{childdoc.def}|\\
|\childdocforward{|\textit{main}|}|\\
\end{tabular}
\end{center}
%
or alternatively with:
%
\begin{center}
\begin{tabular}{l}
|\input{childdoc.def}|\\
|\childdocby{|\textit{main}|}|\\
\end{tabular}
\end{center}
%
Both forms have slightly different effects as described above.
The main file is prepared as usual, see \secref{sec:include}.

%%%%%%%%%%%%%%%%%%%%%%%%%%%%%%%%%%%%%%%%%%%%%%%%%%%%%%%%%%%%%%%%%%%%%%%%%%%%%%%%
\subsection{Legacy Detection}
\label{sec:detection}

The directive |\childdocmain| in the main file can detect
whether the complete document or merely a child is to be compiled
even without using the directive |\childdocof|.
This method is deprecated because it is less robust
and there is no compelling reason to use it;
it is merely provided for backward compatibility
and it may be removed in future versions.

If the detection mechanism is to be used,
it is mandatory to correctly specify
the filename of the main file as the argument of |\childdocmain|:
%
\begin{center}
\begin{tabular}{l}
|\input{childdoc.def}|\\
|\childdocmain{|\textit{main}|}|\\
\end{tabular}
\end{center}
%
If |\jobname| does not match the argument \textit{main} of |\childdocmain|,
it is assumed that |\jobname| points to the child file to be compiled.
When using |\childdocmain| with the main file specified as argument,
it suffices to start a child file
with just |\input{|\textit{main}|}|
without loading of the package and using |\childdocof|.
If instead all processing is done
with the appropriate \textsf{childdoc} directives,
the argument of \textit{main} of |\childdocmain| can be empty.

An alternative version of the command line processing described
in \secref{sec:commandline} using the detection mechanism reads:
%
\begin{center}
|... -jobname "|\textit{target}|" "|[\textit{flags}]%
[|\def\jobname{|\textit{dest}|}|]|\input{|\textit{main}|}"|
\end{center}

%%%%%%%%%%%%%%%%%%%%%%%%%%%%%%%%%%%%%%%%%%%%%%%%%%%%%%%%%%%%%%%%%%%%%%%%%%%%%%%%
\subsection{Manual Code}
\label{sec:manual}

In case one cannot be certain whether the definitions file |childdoc.def|
is installed on the target \TeX{} distribution
and one prefers not to ship it,
it is conceivable to paste a few relevant commands into the sources.

To that end, drop all statements |\input{childdoc.def}|
and perform the replacements as outlined below.
Instead of |\childdocmain{|\textit{main}|}| add the following code
to the top of the main file:
%
\begin{center}
\begin{tabular}{l}
|\||ifdefined\childdocname\endinput\||fi\newif\ifchilddoc|\\
|\edef\childdocname{\scantokens\expandafter{\jobname\noexpand}}|\\
|\def\childdocmain{|\textit{main}|}\||ifx\childdocmain\childdocname\||else|\\
|\childdoctrue\includeonly{\childdocname}\let\jobname\childdocmain\||fi|\\
\end{tabular}
\end{center}
%
Instead of |\childdocof{|\textit{main}|}| just include the main file
at the top of each child file:
%
\begin{center}
|\input{|\textit{main}|}|
\end{center}
%
A simple redirection |\childdocforward{|\textit{dest}|}| is achieved by:
%
\begin{center}
|\def\jobname{|\textit{dest}|}\input{\jobname}|
\end{center}
%
The redirection with prefix
|\childdocforwardprefix[|\textit{prefix}|]{|\textit{dest}|}|
is accomplished by:
%
\begin{center}
\begin{tabular}{l}
|{\edef\jobname{\scantokens\expandafter{\jobname\noexpand}}|\\
|\def\redirectjob |\textit{prefix}|#1~~~{\gdef\jobname{|\textit{dest}|#1}}|\\
|\expandafter\redirectjob\jobname~~~}\input{\jobname}|
\end{tabular}
\end{center}

In an alternative approach,
child documents can be compiled by a specific command line
without additional code or specific definitions:
%
\begin{center}
|... -jobname "|\textit{target}|" "|[\textit{flags}]%
|\includeonly{|\textit{dest}|}\input{|\textit{main}|}"|
\end{center}
%

%%%%%%%%%%%%%%%%%%%%%%%%%%%%%%%%%%%%%%%%%%%%%%%%%%%%%%%%%%%%%%%%%%%%%%%%%%%%%%%%
%%%%%%%%%%%%%%%%%%%%%%%%%%%%%%%%%%%%%%%%%%%%%%%%%%%%%%%%%%%%%%%%%%%%%%%%%%%%%%%%
\section{Information}

%%%%%%%%%%%%%%%%%%%%%%%%%%%%%%%%%%%%%%%%%%%%%%%%%%%%%%%%%%%%%%%%%%%%%%%%%%%%%%%%
\subsection{Copyright}

Copyright \copyright{} 2017--2018 Niklas Beisert

This work may be distributed and/or modified under the
conditions of the \LaTeX{} Project Public License, either version 1.3
of this license or (at your option) any later version.
The latest version of this license is in
  \url{http://www.latex-project.org/lppl.txt}
and version 1.3 or later is part of all distributions of \LaTeX{}
version 2005/12/01 or later.

This work has the LPPL maintenance status `maintained'.

The Current Maintainer of this work is Niklas Beisert.

This work consists of the files |README.txt|, |childdoc.ins| and |childdoc.dtx|
as well as the derived files |childdoc.def|, |cdocsamp.tex|
with |cdocsch1.tex|, |cdocsch2.tex|, |cdocspt3.tex|, |cdocspt4.tex|,
|cdocsdrf.tex|, |cdocsfn1.tex|, |cdocsfn2.tex|
as well as |childdoc.pdf|.

%%%%%%%%%%%%%%%%%%%%%%%%%%%%%%%%%%%%%%%%%%%%%%%%%%%%%%%%%%%%%%%%%%%%%%%%%%%%%%%%
\subsection{Files and Installation}

The package consists of the files:
%
\begin{center}
\begin{tabular}{ll}
    |README.txt|   & readme file \\
    |childdoc.ins| & installation file \\
    |childdoc.dtx| & source file \\
    |childdoc.def| & definition file \\
    |cdocsamp.tex| & sample main file \\
    |cdocsch1.tex| & sample include file \\
    |cdocsch2.tex| & sample include file \\
    |cdocspt3.tex| & sample part file \\
    |cdocspt4.tex| & sample part file \\
    |cdocsdrf.tex| & sample redirection file \\
    |cdocsfn1.tex| & sample redirection file \\
    |cdocsfn2.tex| & sample redirection file \\
    |childdoc.pdf| & manual
\end{tabular}
\end{center}
%
The distribution consists of the files
|README.txt|, |childdoc.ins| and |childdoc.dtx|.
%
\begin{itemize}
\item
Run (pdf)\LaTeX{} on |childdoc.dtx|
to compile the manual |childdoc.pdf| (this file).
\item
Run \LaTeX{} on |childdoc.ins| to create the definitions file |childdoc.def|
and the sample |cdocsamp.tex| with include files
|cdocsch1.tex|, |cdocsch2.tex|, |cdocspt3.tex|, |cdocspt4.tex|,
|cdocsdrf.tex|, |cdocsfn1.tex|, |cdocsfn2.tex|.
Then copy the file |childdoc.def| to an appropriate directory of your \LaTeX{}
distribution, e.g.\ \textit{texmf-root}|/tex/latex/childdoc|.
\end{itemize}

%%%%%%%%%%%%%%%%%%%%%%%%%%%%%%%%%%%%%%%%%%%%%%%%%%%%%%%%%%%%%%%%%%%%%%%%%%%%%%%%
\subsection{Related CTAN Packages}

There are several other packages which offer a similar functionality:
%
\begin{itemize}
\item
The packages
\href{http://ctan.org/pkg/docmute}{\textsf{docmute}},
\href{http://ctan.org/pkg/includex}{\textsf{includex}} and
\href{http://ctan.org/pkg/standalone}{\textsf{standalone}}
provide commands to include only the document body of
a child file thus allowing both files to be compiled individually.
\item
The packages \href{http://ctan.org/pkg/subdocs}{\textsf{subdocs}}
and \href{http://ctan.org/pkg/subfiles}{\textsf{subfiles}}
provide structures in which the main and child documents can be
encapsulated and allowing them to be compiled individually.
The inclusion mechanism is different from the conventional |\include|.
\item
The package \href{http://ctan.org/pkg/combine}{\textsf{combine}}
is an elaborate solution to combine several documents into one.
\end{itemize}
%
See also the CTAN topic \href{http://ctan.org/topic/subdocs}{\textsf{subdocs}}
for further related packages.
The present package differs from the above solutions in that
a document structure constructed with the conventional |\include| mechanism
just needs two extra commands at the top of every file
such that all constituent files can be compiled individually.

%%%%%%%%%%%%%%%%%%%%%%%%%%%%%%%%%%%%%%%%%%%%%%%%%%%%%%%%%%%%%%%%%%%%%%%%%%%%%%%%
%\subsection{Feature Suggestions}
%
%The following is a list of features which may be useful for future
%versions of this package:
%%
%\begin{itemize}
%\item
%\ldots
%\end{itemize}

%%%%%%%%%%%%%%%%%%%%%%%%%%%%%%%%%%%%%%%%%%%%%%%%%%%%%%%%%%%%%%%%%%%%%%%%%%%%%%%%
\subsection{Revision History}

%%%%%%%%%%%%%%%%%%%%%%%%%%%%%%%%%%%%%%%%
\paragraph{v2.0:} 2018/12/30

\begin{itemize}
\item
immediate forward processing
\item
added |\childdocby| mechanism
\item
manual restructured
\end{itemize}

%%%%%%%%%%%%%%%%%%%%%%%%%%%%%%%%%%%%%%%%
\paragraph{v1.6:} 2018/01/17

\begin{itemize}
\item
application for development of include files
\item
corrections to manual
\end{itemize}

%%%%%%%%%%%%%%%%%%%%%%%%%%%%%%%%%%%%%%%%
\paragraph{v1.5:} 2017/05/21

\begin{itemize}
\item
more complete structuring introduced
\item
|\childdocof| introduced
\item
|\childdoc| renamed to |\childdocmain|
\item
|\childredirect| renamed to |\childdocforward| and |\childdocforwardprefix|
and functionality expanded
\end{itemize}

%%%%%%%%%%%%%%%%%%%%%%%%%%%%%%%%%%%%%%%%
\paragraph{v1.0:} 2017/04/27

\begin{itemize}
\item
manual and install package
\item
first version published on CTAN
\end{itemize}

%%%%%%%%%%%%%%%%%%%%%%%%%%%%%%%%%%%%%%%%
\paragraph{v0.6:} 2017/04/26

\begin{itemize}
\item
redirection mechanism added
\end{itemize}

%%%%%%%%%%%%%%%%%%%%%%%%%%%%%%%%%%%%%%%%
\paragraph{v0.5:} 2017/04/26

\begin{itemize}
\item
functionality in definition file
\end{itemize}


%%%%%%%%%%%%%%%%%%%%%%%%%%%%%%%%%%%%%%%%%%%%%%%%%%%%%%%%%%%%%%%%%%%%%%%%%%%%%%%%
%%%%%%%%%%%%%%%%%%%%%%%%%%%%%%%%%%%%%%%%%%%%%%%%%%%%%%%%%%%%%%%%%%%%%%%%%%%%%%%%
%%%%%%%%%%%%%%%%%%%%%%%%%%%%%%%%%%%%%%%%%%%%%%%%%%%%%%%%%%%%%%%%%%%%%%%%%%%%%%%%
\appendix

\settowidth\MacroIndent{\rmfamily\scriptsize 000\ }

 \DocInput{childdoc.dtx}

\end{document}
%</driver>
% \fi
%
% %%%%%%%%%%%%%%%%%%%%%%%%%%%%%%%%%%%%%%%%%%%%%%%%%%%%%%%%%%%%%%%%%%%%%%%%%%%%%%
% %%%%%%%%%%%%%%%%%%%%%%%%%%%%%%%%%%%%%%%%%%%%%%%%%%%%%%%%%%%%%%%%%%%%%%%%%%%%%%
% \section{Sample}
%\iffalse
%<*samplemain>
%\fi
%
% The following presents a sample document
% with two chapters, two parts, a title page,
% a compile flag as well as three forwarding files to set the flag.
% It consists of eight |.tex| files:
% \begin{center}
% \begin{tabular}{ll}
% |cdocsamp.tex|&main file\\
% |cdocsch1.tex|&include file for chapter 1\\
% |cdocsch2.tex|&include file for chapter 2\\
% |cdocspt3.tex|&include file for part 3\\
% |cdocspt4.tex|&include file for part 4\\
% |cdocsdrf.tex|&forwarding file for main file in draft mode\\
% |cdocsfi1.tex|&forwarding file for final version of chapter 1\\
% |cdocsfi2.tex|&forwarding file for final version of chapter 2\\
% \end{tabular}
% \end{center}
% Each of the eight files can be compiled directly by the \LaTeX{} compiler.
%
% %%%%%%%%%%%%%%%%%%%%%%%%%%%%%%%%%%%%%%
% \paragraph{Main File.}
%
% The main file is called |cdocsamp.tex|.
%
% Load the \textsf{childdoc} definitions and
% declare the filename for the main document:
%    \begin{macrocode}
\input{childdoc.def}
\childdocmain{}
%    \end{macrocode}

% Optional override for |\version| flag:
%    \begin{macrocode}
%%\ifchilddoc\else\providecommand{\version}{draft}\fi
%    \end{macrocode}

% Define the default values for the |\version| flag
% (|final| for the main file and |draft| for childs):
%    \begin{macrocode}
\ifchilddoc
\providecommand{\version}{draft}
\else
\providecommand{\version}{final}
\fi
%    \end{macrocode}

% Load the standard document class:
%    \begin{macrocode}
\documentclass[12pt]{article}
%    \end{macrocode}

% Start the document body:
%    \begin{macrocode}
\begin{document}
%    \end{macrocode}

% Declare a title page.
% Print title, part of document being processed and version flag:
%    \begin{macrocode}
\addtocounter{page}{-1}
\begin{center}
{\LARGE\bfseries{}childdoc example\par}
\vspace{1cm}
\ifchilddoc
\ifchilddocmanual part\else chapter\fi:
`\childdocname' of `\childdocjob'\par
\else
main document: `\childdocjob'\par
\fi
version: \version\par
\end{center}
\newpage
%    \end{macrocode}

% Manually include selected file,
% otherwise process as usual:
%    \begin{macrocode}
\ifchilddocmanual
\section*{part `\childdocname'}
\input{\childdocname}
\else
%    \end{macrocode}

% Include the two chapters:
%    \begin{macrocode}
\include{cdocsch1}
\include{cdocsch2}
%    \end{macrocode}

% Include the two parts unless only chapters should be displayed:
%    \begin{macrocode}
\ifchilddoc\else
\section{part three}
\input{cdocspt3}
\section{part four}
\input{cdocspt4}
\fi
%    \end{macrocode}

% Process as usual until here:
%    \begin{macrocode}
\fi
%    \end{macrocode}

% End of document body:
%    \begin{macrocode}
\end{document}
%    \end{macrocode}
%\iffalse
%</samplemain>
%\fi
%
% %%%%%%%%%%%%%%%%%%%%%%%%%%%%%%%%%%%%%%
% \paragraph{Chapter Include Files.}
%
% The include files are called |cdocsch1.tex| and |cdocsch2.tex|.
%
%\iffalse
%<*samplechap1|samplechap2>
%\fi

% Optional override for |\version| flag:
%    \begin{macrocode}
%%\providecommand{\version}{final}
%    \end{macrocode}

% Include the main document:
%    \begin{macrocode}
\input{childdoc.def}
\childdocof{cdocsamp}
%    \end{macrocode}

%\iffalse
%</samplechap1|samplechap2>
%\fi
%
%\iffalse
%<*samplechap1>
%\fi
% Some text for chapter 1:
%    \begin{macrocode}
\section{one}
some text in chapter one
%    \end{macrocode}

%\iffalse
%</samplechap1>
%\fi
% Some text for chapter 2:
%\iffalse
%<*samplechap2>
%\fi
%    \begin{macrocode}
\section{two}
more text in chapter two
%    \end{macrocode}

%\iffalse
%</samplechap2>
%\fi
%
% %%%%%%%%%%%%%%%%%%%%%%%%%%%%%%%%%%%%%%
% \paragraph{Part Include Files.}
%
% The include files are called |cdocspt3.tex| and |cdocspt4.tex|.
%
%\iffalse
%<*samplepart3|samplepart4>
%\fi

% Optional override for |\version| flag:
%    \begin{macrocode}
%%\providecommand{\version}{final}
%    \end{macrocode}

% Include the main document:
%    \begin{macrocode}
\input{childdoc.def}
\childdocby{cdocsamp}
%    \end{macrocode}

%\iffalse
%</samplepart3|samplepart4>
%\fi
%
%\iffalse
%<*samplepart3>
%\fi
% Some text for part 3:
%    \begin{macrocode}
some text in part three
%    \end{macrocode}

%\iffalse
%</samplepart3>
%\fi
% Some text for part 4:
%\iffalse
%<*samplepart4>
%\fi
%    \begin{macrocode}
more text in part four
%    \end{macrocode}

%\iffalse
%</samplepart4>
%\fi
%
% %%%%%%%%%%%%%%%%%%%%%%%%%%%%%%%%%%%%%%
% \paragraph{Forwarding for a Complete Draft.}
%
% The following forwarding file |cdocsdrf.tex|
% compiles the main document in draft mode:
%\iffalse
%<*sampledraft>
%\fi
%    \begin{macrocode}
\def\version{draft}
\input{childdoc.def}
\childdocforward{cdocsamp}
%    \end{macrocode}

%\iffalse
%</sampledraft>
%\fi
%
% %%%%%%%%%%%%%%%%%%%%%%%%%%%%%%%%%%%%%%
% \paragraph{Forwarding for Final Version of the Chapters.}
%
% The following forwarding files |cdocsfn1.tex| and |cdocsfn2.tex|
% (with identical content)
% compile the final versions of the child documents
% |cdocsch1.tex| and |cdocsch2.tex|, respectively:
%\iffalse
%<*samplefinal>
%\fi
%    \begin{macrocode}
\def\version{final}
\input{childdoc.def}
\childdocforwardprefix[cdocsamp]{cdocsfn}{cdocsch}
%    \end{macrocode}

%\iffalse
%</samplefinal>
%\fi
%
% %%%%%%%%%%%%%%%%%%%%%%%%%%%%%%%%%%%%%%
% \paragraph{Command Line Processing.}
%
% The following three command lines generate the output files
% |cdocscld|, |cdocscl1| and |cdocscl2|
% which should be identical to
% |cdocsdrf|, |cdocsch1| and |cdocsfn2|, respectively:
% \begin{center}
% \begin{tabular}{l}
% |latex -jobname cdocscld \|\\
% |  "\def\version{draft}\input{childdoc.def}\childdocforward{cdocsamp}"|\\
% |latex -jobname cdocscl1 \|\\
% |  "\input{childdoc.def}\childdocforward[cdocsamp]{cdocsch1}"|\\
% |latex -jobname cdocscl2 \|\\
% |  "\def\version{final}\input{childdoc.def}\childdocforward{cdocsch2}"|
% \end{tabular}
% \end{center}
% Note that the trailing backslash on each first line
% merely continues the input to the second line
% (for convenient cut ant paste).
% Furthermore, the command |latex| can be replaced by any
% of its alternative versions such as |pdflatex|.
%
% %%%%%%%%%%%%%%%%%%%%%%%%%%%%%%%%%%%%%%%%%%%%%%%%%%%%%%%%%%%%%%%%%%%%%%%%%%%%%%
% %%%%%%%%%%%%%%%%%%%%%%%%%%%%%%%%%%%%%%%%%%%%%%%%%%%%%%%%%%%%%%%%%%%%%%%%%%%%%%
% \section{Implementation}
%\iffalse
%<*package>
%\fi
%
% This section describes the definitions file |childdoc.def|.

% The definitions cannot be loaded using |\usepackage| or |\RequirePackage|
% which has a mechanism to prevent loading a style file more than once.
% When loading the definitions by means of |\input|
% multiple instances have to be prevented manually:
%\iffalse
%This code needs to be before the `\ProvidesFile' directive
%which is defined at the beginning of this file.
%Therefore it is also placed there and commented out here.
%</package>
%<*discard>
%\fi
%    \begin{macrocode}
\ifdefined\childdocmain\endinput\fi
%    \end{macrocode}
%\iffalse
%</discard>
%<*package>
%\fi
%
% \macro{\ifchilddoc}
% \macro{\ifchilddocmanual}
% The conditional |\ifchilddoc| tells whether a
% child (true) or main (false) document is being compiled.
% The conditional |\ifchilddocmanual| tells whether
% the |\includeonly| mechanism is used (false) or
% the selection of child files must be performed manually (true).
% The definitions initialise to false:
%    \begin{macrocode}
\newif\ifchilddoc
\newif\ifchilddocmanual
%    \end{macrocode}

% \macro{\childdocname}
% \macro{\childdocjob}
% The macro |\childdocname| stores the name of the main document
% to be compiled. The macro |\childdocjob| stores the name of
% the document on which the \LaTeX{} compiler was originally invoked.
% The content of |\jobname| cannot be compared
% to filenames specified in the source due to different catcodes.
% The following code rescans |\jobname|, stores the result
% in |\childdocname| and saves a copy in |\childdocjob|:
%    \begin{macrocode}
\edef\childdocname{\scantokens\expandafter{\jobname\noexpand}}
\let\childdocjob\childdocname
%    \end{macrocode}

% \macro{\childdocdisable}
% The macro |\childdocdisable| prevents the main file
% from being processed more than once.
% At this stage, the main document command |\childdocmain|
% is assumed to be called once again where it should do nothing.
% Any subsequent call to it should prevent
% a secondary processing of the main document
% It overwrites the forwarding commands
% |\childdocof| and |\childdocforward|
% with empty macros to prevent further inclusions of the main document:
%    \begin{macrocode}
\newcommand{\childdocdisable}
{
  \renewcommand{\childdocmain}[1]{\renewcommand{\childdocmain}[1]{\endinput}}
  \renewcommand{\childdocof}[1]{}
  \renewcommand{\childdocby}[2][]{}
  \renewcommand{\childdocforward}[2][]{}
  \renewcommand{\childdocdisable}{}
}
%    \end{macrocode}

% \macro{\childdocmain}
% The macro |\childdocmain| is to be called at the top of the main file
% with nothing or the main filename (without extension) as argument.
% First, it breaks loops.
% If the argument is not empty and does not match |\childdocname|
% (which is set by the first inclusion of |childdoc.def|),
% |\ifchilddoc| is set to true, |\includeonly| is applied to the child file
% and |\jobname| is set to the main file
% (for proper handling of |.aux| files):
%    \begin{macrocode}
\newcommand{\childdocmain}[1]
{
  \childdocdisable\childdocmain{}
  \if?#1?\else
    \begingroup
      \def\childdoctmp{#1}
      \ifx\childdoctmp\childdocname
        \def\childdoctmp{}
      \else
        \def\childdoctmp
        {
          \childdoctrue
          \includeonly{\childdocname}
          \def\childdocjob{#1}
          \def\jobname{#1}
        }
      \fi
      \expandafter
    \endgroup
    \childdoctmp
  \fi
}
%    \end{macrocode}

% \macro{\childdocof}
% The command |\childdocof| redirects
% compilation to the main file |#1|.
%    \begin{macrocode}
\newcommand{\childdocof}[1]
{
  \childdocdisable
  \childdoctrue
  \includeonly{\childdocname}
  \def\jobname{#1}
  \def\childdocjob{#1}
  \input{#1}
}
%    \end{macrocode}

% \macro{\childdocby}
% The command |\childdocby| ....
%    \begin{macrocode}
\newcommand{\childdocby}[2][]
{
  \childdocdisable
  \childdoctrue
  \childdocmanualtrue
  \if?#1?\else
    \def\jobname{#2}
  \fi
  \def\childdocjob{#2}
  \input{#2}
  \endinput
}
%    \end{macrocode}

% \macro{\childdocforward}
% The command |\childdocforward| redirects
% compilation to the main file or
% (if the optional argument is given) a child file.
% Parameters are set as if the main file
% or a child file starting with |\childdocof| was compiled.
% Then compilation is handed over to the main file:
%    \begin{macrocode}
\newcommand{\childdocforward}[2][]
{
  \begingroup
    \if?#1?
      \def\childdoctmp
      {
        \def\childdocname{#2}
        \def\childdocjob{#2}
        \def\jobname{#2}
        \input{#2}
        \endinput
      }
    \else
      \def\childdoctmp
      {
        \childdocdisable
        \def\childdocname{#2}
        \childdoctrue
        \includeonly{#2}
        \def\childdocjob{#1}
        \def\jobname{#1}
        \input{#1}
        \endinput
      }
    \fi
    \expandafter
  \endgroup
  \childdoctmp
}
%    \end{macrocode}

% \macro{\childdocforwardprefix}
% The command |\childdocforwardprefix| redirects
% compilation to the main or a child file by means of a pattern.
% The prefix |#1| in the current filename is replaced by |#2|
% and the suffix of the current filename is kept
% (it is assumed that the filename does not contain the substring `|~~~|'
% which is used as a delimiter).
% Compilation is handed over to the new file by |\childdocforward|:
%    \begin{macrocode}
\newcommand{\childdocforwardprefix}[3][]
{
  \begingroup
    \def\childdocextract #2##1~~~{\def\childdoctmp{\childdocforward[#1]{#3##1}}}
    \expandafter\childdocextract\childdocname~~~
    \expandafter
  \endgroup
  \childdoctmp
}
%    \end{macrocode}

% \macro{\childdoc}
% The deprecated macro |\childdoc| is a legacy version of |\childdocmain|:
%    \begin{macrocode}
\newcommand{\childdoc}{\childdocmain}
%    \end{macrocode}

% \macro{\childdocredirect}
% The deprecated macro |\childdocredirect| is a legacy version
% of |\childdocforward| and |\childdocforwardprefix|:
%    \begin{macrocode}
\newcommand{\childdocredirect}[2][]
{
  \begingroup
    \if?#1?
      \def\childdoctmp{\childdocforward{#2}}
    \else
      \def\childdoctmp{\childdocforwardprefix{#1}{#2}}
    \fi
    \expandafter
  \endgroup
  \childdoctmp
}
%    \end{macrocode}

%\iffalse
%</package>
%\fi
%
\endinput
|\\
|\childdocmain{}|\\
\end{tabular}
\end{center}
at the very top of the main \LaTeX{} file,
in particular \emph{before} the |\documentclass| statement!
The argument of |\childdocmain| should be left empty
(but it must be present).

%%%%%%%%%%%%%%%%%%%%%%%%%%%%%%%%%%%%%%%%
\DescribeMacro{\childdocof}
Furthermore, add the commands
\begin{center}
\begin{tabular}{l}
|% \iffalse
%
% childdoc.dtx Copyright (C) 2017-2018 Niklas Beisert
%
% This work may be distributed and/or modified under the
% conditions of the LaTeX Project Public License, either version 1.3
% of this license or (at your option) any later version.
% The latest version of this license is in
%   http://www.latex-project.org/lppl.txt
% and version 1.3 or later is part of all distributions of LaTeX
% version 2005/12/01 or later.
%
% This work has the LPPL maintenance status `maintained'.
%
% The Current Maintainer of this work is Niklas Beisert.
%
% This work consists of the files childdoc.dtx and childdoc.ins
% and the derived files childdoc.def and cdocsamp.tex with
% cdocsch1.tex, cdocsch2.tex, cdocsdrf.tex, cdocsfn1.tex, cdocsfn2.tex.
%
%<package>\ifdefined\childdocmain\endinput\fi
%<package>\ProvidesFile{childdoc.def}[2018/12/30 v2.0 child document driver]
%<samplemain>\ProvidesFile{cdocsamp.tex}[2018/12/30 v2.0 sample for childdoc]
%<*driver>
%\ProvidesFile{childdoc.drv}[2018/12/30 v2.0 childdoc reference manual file]
\PassOptionsToClass{10pt,a4paper}{article}
\documentclass{ltxdoc}

\usepackage[margin=35mm]{geometry}
\usepackage{hyperref}
\usepackage{hyperxmp}
\usepackage[usenames]{color}

\hypersetup{colorlinks=true}
\hypersetup{pdfstartview=FitH}
\hypersetup{pdfpagemode=UseNone}
\hypersetup{pdfsource={}}
\hypersetup{pdflang={en-UK}}
\hypersetup{pdfcopyright={Copyright 2017-2018 Niklas Beisert.
  This work may be distributed and/or modified under the
  conditions of the LaTeX Project Public License, either version 1.3
  of this license or (at your option) any later version.}}
\hypersetup{pdflicenseurl={http://www.latex-project.org/lppl.txt}}
\hypersetup{pdfcontactaddress={ETH Zurich, ITP, HIT K,
  Wolfgang-Pauli-Strasse 27}}
\hypersetup{pdfcontactpostcode={8093}}
\hypersetup{pdfcontactcity={Zurich}}
\hypersetup{pdfcontactcountry={Switzerland}}
\hypersetup{pdfcontactemail={nbeisert@itp.phys.ethz.ch}}
\hypersetup{pdfcontacturl={http://people.phys.ethz.ch/\xmptilde nbeisert/}}

\newcommand{\secref}[1]{\hyperref[#1]{section \ref*{#1}}}

\parskip1ex
\parindent0pt
\let\olditemize\itemize
\def\itemize{\olditemize\parskip0pt}

\begin{document}

\title{The \textsf{childdoc} Package}
\hypersetup{pdftitle={The childdoc Package}}
\author{Niklas Beisert\\[2ex]
  Institut f\"ur Theoretische Physik\\
  Eidgen\"ossische Technische Hochschule Z\"urich\\
  Wolfgang-Pauli-Strasse 27, 8093 Z\"urich, Switzerland\\[1ex]
  \href{mailto:nbeisert@itp.phys.ethz.ch}
  {\texttt{nbeisert@itp.phys.ethz.ch}}}
\hypersetup{pdfauthor={Niklas Beisert}}
\hypersetup{pdfsubject={Manual for the LaTeX2e Package childdoc}}
\date{30 December 2018, \textsf{v2.0}}
\maketitle

\begin{abstract}\noindent
\textsf{childdoc} is a \LaTeXe{} package
that enables the direct compilation
of document sections included by |\include|
to individual files.
\end{abstract}

\begingroup
\parskip0ex
\tableofcontents
\endgroup

%%%%%%%%%%%%%%%%%%%%%%%%%%%%%%%%%%%%%%%%%%%%%%%%%%%%%%%%%%%%%%%%%%%%%%%%%%%%%%%%
%%%%%%%%%%%%%%%%%%%%%%%%%%%%%%%%%%%%%%%%%%%%%%%%%%%%%%%%%%%%%%%%%%%%%%%%%%%%%%%%
\section{Introduction}

\LaTeX{} provides a mechanism to structure a large document (such as a book)
into a main file and several child files (containing the chapters)
using the |\include| command.
This mechanism is beneficial for documents
which span hundreds of pages in order to
make the source file(s) more manageable.
Moreover, compilation can be restricted to
selected child files by means of the |\includeonly| command.
The latter feature can be used to reduce the compilation time while editing
(this was significantly more useful in the earlier days of \LaTeX{})
or to generate a smaller document which is easier to navigate.
Another application of |\includeonly| is to generate
documents consisting of selected parts of the complete document.

However, there are a few drawbacks of the plain |\include| mechanism:
\begin{itemize}
\item
The child files cannot be compiled on their own,
they can only be compiled via the main file.
A naive editing environment
(such as a text editor with an option
to have the current file processed by \LaTeX)
may require one to switch to the main file before compiling;
attempting to compile the child file produces errors.
\item
The main file must be modified (each time)
to adjust the |\includeonly| command
to the present needs. This easily leaves the main file in a messy state.
\item
The generated document will always carry the filename
of the main document. This is inconvenient if
several child files are to be compiled and
to be kept for distribution.
\end{itemize}

The present package provides a simple interface
to make child files individually compilable by \LaTeX{}.
Compiling a child file then has the same effect as compiling
the main file with an |\includeonly| command
to select the appropriate child.
Moreover the generated document will carry the name of the child
rather than the main file.
This resolves all three above issues.

This feature is meant to make the editing of books,
thesis documents and lecture notes somewhat more convenient.
However, the package can also be used efficiently for
composing a series of documents (such as exercise sheets)
which are typically distributed individually.
It then assists the author in generating the individual documents
(potentially in different versions)
as well as a document containing the collected series.
Another application is in developing style files
or other kinds of included material
where compilation of the style file could redirect
to a sample or test file.

%%%%%%%%%%%%%%%%%%%%%%%%%%%%%%%%%%%%%%%%%%%%%%%%%%%%%%%%%%%%%%%%%%%%%%%%%%%%%%%%
%%%%%%%%%%%%%%%%%%%%%%%%%%%%%%%%%%%%%%%%%%%%%%%%%%%%%%%%%%%%%%%%%%%%%%%%%%%%%%%%
\section{Usage}

First of all, the package \textsf{childdoc} is \emph{not} a standard
\LaTeXe{} |.sty| style file! Therefore it needs to be invoked in
a non-standard way.

%%%%%%%%%%%%%%%%%%%%%%%%%%%%%%%%%%%%%%%%%%%%%%%%%%%%%%%%%%%%%%%%%%%%%%%%%%%%%%%%
\subsection{Included Files}
\label{sec:include}

%%%%%%%%%%%%%%%%%%%%%%%%%%%%%%%%%%%%%%%%
\DescribeMacro{\childdocmain}
To use the package, add the commands
\begin{center}
\begin{tabular}{l}
|\input{childdoc.def}|\\
|\childdocmain{}|\\
\end{tabular}
\end{center}
at the very top of the main \LaTeX{} file,
in particular \emph{before} the |\documentclass| statement!
The argument of |\childdocmain| should be left empty
(but it must be present).

%%%%%%%%%%%%%%%%%%%%%%%%%%%%%%%%%%%%%%%%
\DescribeMacro{\childdocof}
Furthermore, add the commands
\begin{center}
\begin{tabular}{l}
|\input{childdoc.def}|\\
|\childdocof{|\textit{main}|}|\\
\end{tabular}
\end{center}
at the top of every child file \textit{child}
which is included by |\include{|\textit{child}|}|
from within the main file
(or at least for those files to be compiled individually).
The argument \textit{main} must be the filename of the main file.

There are a couple of
considerations in setting up the main and child documents:

%%%%%%%%%%%%%%%%%%%%%%%%%%%%%%%%%%%%%%%%
\paragraph{Restrictions.}

Please note the following restrictions:
\begin{itemize}
\item
|\childdocmain| must be called with one argument \textit{main}
to ensure compatibility with earlier version of the package.
It must either be empty (|\childdocmain{}|)
or precisely match the filename of the main file in which it is specified.
See \secref{sec:detection} for further information.
\item
The filename \textit{main} must be specified without the |.tex| extension.
\item
The filename \textit{main} is case sensitive
(even in case-insensitive file systems)
due to internal string comparison.
\item
The argument \textit{main} should be fully expanded, it cannot be a macro.
\item
Subdirectories and special characters should be avoided in filenames.
\item
The command |\childdocmain{|\textit{main}|}| must be followed by a whitespace.
It should not be followed immediately by another command
or by a comment mark `|%|'.
This is because the \TeX{} parser reads the token immediately following
the argument of |\childdocmain| and puts it
at the beginning of every child section;
however, a white\-space is ignored.
\end{itemize}

%%%%%%%%%%%%%%%%%%%%%%%%%%%%%%%%%%%%%%%%
\paragraph{Content of Main File.}

It is advisable to place all content in the child files included by |\include|.
Any output contained in the main file will appear in all child documents
unless suppressed manually;
it cannot be suppressed automatically by the |\includeonly| directive
and thus should normally be avoided.
A method to include some content in the main file
by means of conditional processing is described in \secref{sec:conditional}.

%%%%%%%%%%%%%%%%%%%%%%%%%%%%%%%%%%%%%%%%
\paragraph{Page Numbering.}

When only a part of the document is compiled,
the appropriate numbering of pages
(as well as other status parameters)
is determined from the |.aux| files.
The latter contain information from previous passes.
However this information needs to propagate through
all intermediate child documents.
Therefore the page numbering in child documents may well
be inconsistent until the complete document is compiled at least once.

A useful (if unconventional) way to always ensure a consistent
page numbering is to restart the numbering in each child document
and denote the pages by `\textit{child}|.|\textit{page}'
where \textit{child} represents the chapter/section number of the child file.
This can be achieved by the command
|\numberwithin{page}{|\textit{child}|}|
of the \textsf{amsmath} package
where \textit{child} can be |chapter| or |section|
depending on the chosen structuring.
Alternatively, one can modify the macro |\thepage| appropriately
and reset the counter |page| at the start of each child file.

%%%%%%%%%%%%%%%%%%%%%%%%%%%%%%%%%%%%%%%%%%%%%%%%%%%%%%%%%%%%%%%%%%%%%%%%%%%%%%%%
\subsection{Conditional Processing}
\label{sec:conditional}

The package provides a mechanism to compile different versions
of a document. To customise the versions further some conditional processing
can come in handy to distinguish which version is being compiled.
The package provides two macros to describe the compilation context:

%%%%%%%%%%%%%%%%%%%%%%%%%%%%%%%%%%%%%%%%
\DescribeMacro{\ifchilddoc}
The conditional |\ifchilddoc| distinguishes between the compilation of
child documents and the main document:
%
\begin{center}
|\ifchilddoc |\textit{child-code}| |[|\||else |\textit{main-code}]| \||fi|
\end{center}

%%%%%%%%%%%%%%%%%%%%%%%%%%%%%%%%%%%%%%%%
\DescribeMacro{\childdocname}
\DescribeMacro{\childdocjob}
The macro |\childdocname| contains the filename (without extension)
of the main or child file being processed.
Note that |\childdocjob| will always contain the name of the main file.

%%%%%%%%%%%%%%%%%%%%%%%%%%%%%%%%%%%%%%%%
\paragraph{Title Page.}

Conditional processing can be used to include a title or banner page
in the main document when proper precautions are taken.
Importantly, the code in the main file should ensure that the page counter
(as well as other status parameters which are stored in the |.aux| files)
takes the same value after the conditional processing.
Otherwise the page numbers may take divergent values
depending on which part is compiled.

For example, a title page could be declared by:
%
\begin{center}
\begin{tabular}{l}
|\ifchilddoc\||else|\\
|\addtocounter{page}{-1}|\\
\textit{code for title page}\\
|\newpage|\\
|\||fi|
\end{tabular}
\end{center}
%
A banner page for the child documents can be generated by:
%
\begin{center}
\begin{tabular}{l}
|\ifchilddoc|\\
|\addtocounter{page}{-1}|\\
\textit{code for banner page}\\
|\newpage|\\
|\||fi|
\end{tabular}
\end{center}
%
Here one could write a message such as:
\begin{center}
|This is the part \childdocname{} of \childdocjob{}.|
\end{center}

%%%%%%%%%%%%%%%%%%%%%%%%%%%%%%%%%%%%%%%%%%%%%%%%%%%%%%%%%%%%%%%%%%%%%%%%%%%%%%%%
\subsection{Flags}
\label{sec:flags}

The package makes it easy to generate different versions
of the main or child documents.
To this end compilation flags can be defined
and assigned different default values.
They will be particularly useful in conjunction
with the forwarding mechanism described in \secref{sec:forward}.

For example, it may be useful to have a flag |\version|
which can be set to |draft| or |final|.
The document source will contain some conditional code
depending on the value of |\version|.
Suppose further, the flag should default to |final| for the main file
and to |draft| for child files
which is a natural assignment for editing the document.
This is achieved by placing the following code
in the preamble of the main document
(below the |\childdocmain| directive):
%
\begin{center}
\begin{tabular}{l}
|\ifchilddoc|\\
|\providecommand{\version}{draft}|\\
|\||else|\\
|\providecommand{\version}{final}|\\
|\||fi|
\end{tabular}
\end{center}
%
The definition by |\providecommand| makes sure
that previous definitions are not overwritten.
Further statements |\providecommand{\version}{...}|
can thus be added before the above code to override it.

For the main file, one might add a line
(between |\childdocmain| and the above block)
%
\begin{center}
|%\ifchilddoc\||else\providecommand{\version}{draft}\||fi|
\end{center}
%
which can be uncommented to produce a draft version.
Likewise one can add a line to the very top of a child file
(above the |\childdocof{|\textit{main}|}| directive)
%
\begin{center}
|%\providecommand{\version}{final}|
\end{center}
%
which can be uncommented to produce the final version of this child document.

%%%%%%%%%%%%%%%%%%%%%%%%%%%%%%%%%%%%%%%%%%%%%%%%%%%%%%%%%%%%%%%%%%%%%%%%%%%%%%%%
\subsection{Forwarding}
\label{sec:forward}

Different versions of the main or child documents
using compilation flags as described in \secref{sec:flags}
can be (permanently) stored in different files
for convenient compilation, viewing and distribution.
To this end, the package defines a command
to pass on compilation to a different file:

%%%%%%%%%%%%%%%%%%%%%%%%%%%%%%%%%%%%%%%%
\DescribeMacro{\childdocforward}
The command |\childdocforward| redirects processing to
another source file:
%
\begin{center}
\begin{tabular}{l}
|\input{childdoc.def}|\\
|\childdocforward[|\textit{main}|]{|\textit{dest}|}|\\
\end{tabular}
\end{center}
%
The argument \textit{dest} is the destination file
(without extension).
It should be the main file or one of the child files.
Note that further \textsf{childdoc} directives
such as |\childdocof| and |\childdocforward|
in the indicated file will be processed in this form.
The optional argument \textit{main}
passes on directly to the main file \textit{main}
while pretending to compile the child \textit{dest}.
This form behaves as if \textit{dest}
issues |\childdocof{|\textit{main}|}| right away,
and no further \textsf{childdoc} directives will be processed.

%%%%%%%%%%%%%%%%%%%%%%%%%%%%%%%%%%%%%%%%
\DescribeMacro{\...prefix}
In the alternative form |\childdocforwardprefix|,
%
\begin{center}
\begin{tabular}{l}
|\input{childdoc.def}|\\
|\childdocforwardprefix[|\textit{main}|]{|\textit{prefix}|}{|\textit{dest}|}|
\end{tabular}
\end{center}
%
the destination file is determined by a pattern
depending on the current file:
To make this work, the current file must be called
`{\textit{prefix}\hspace{0.2em}\textit{suffix}}'
with \textit{prefix} matching precisely the argument.
Processing is then passed on to the file
`{\textit{dest}\hspace{0.2em}\textit{suffix}}'.
Surely, the same effect is achieved by
directly specifying the
argument `{\textit{dest}\hspace{0.2em}\textit{suffix}}'
in the first form.
However, that requires to set up a different file
for each child. With the alternative form of the command
all these files can have exactly the same content
which simplifies setting them up and maintaining them.

For example, the following file |draft.tex|
with a compilation flag |\version| as described in \secref{sec:flags}
compiles the main document as a draft:
%
\begin{center}
\begin{tabular}{l}
|\def\version{draft}|\\
|\input{childdoc.def}|\\
|\childdocforward{|\textit{main}|}|
\end{tabular}
\end{center}
%
Likewise, the following files |final|\textit{nn}|.tex|
compile the final version of the child document
|child|\textit{nn}|.tex|:
%
\begin{center}
\begin{tabular}{l}
|\def\version{final}|\\
|\input{childdoc.def}|\\
|\childdocforwardprefix{final}{child}|
\end{tabular}
\end{center}
%

Note that when several versions of a main file and/or of each child file
are to be generated, it may be convenient to set up a |Makefile| or
shell script to automatise the process.

%%%%%%%%%%%%%%%%%%%%%%%%%%%%%%%%%%%%%%%%%%%%%%%%%%%%%%%%%%%%%%%%%%%%%%%%%%%%%%%%
\subsection{Command Line Processing}
\label{sec:commandline}

The effect of redirection files can also be achieved by invoking
the \LaTeX{} compiler with a more elaborate command line.
Most conveniently this should be done as part
of a shell script or a |Makefile|.

When using \textsf{childdoc} in the main file, the following
command lines effectively perform a redirection
(note that depending on the shell being used,
backslashes may have to be doubled: `|\|' $\to$ `|\\|'):
%
\begin{center}
|... -jobname "|\textit{target}|" |\\|"|[\textit{flags}]%
|\input{childdoc.def}\childdocforward[|\textit{main}|]{|\textit{dest}|}"|
\end{center}
%
Here \textit{target} is the name of the output file,
\textit{main} is the name of the main file
and \textit{dest} is the name of the main or child file to be processed
(all filenames without extensions).
The optional argument \textit{main} can be omitted
if \textit{main} matches \textit{dest}.
Optionally, compilation \textit{flags} can be defined via |\def| commands.
This command line makes the \TeX{} engine believe
it is compiling the file \textit{target}
whose content is specified as the latter parameter.
The provided code then forwards the processing to
\textit{main} or \textit{dest} as described in \secref{sec:forward}.

%%%%%%%%%%%%%%%%%%%%%%%%%%%%%%%%%%%%%%%%%%%%%%%%%%%%%%%%%%%%%%%%%%%%%%%%%%%%%%%%
\subsection{Include by Input}
\label{sec:input}

Including child documents by |\include| has some restrictions by design.
Most notably, the content of a child document always occupies
its own set of pages; pages cannot be shared between child documents.
Usually, this behaviour makes perfect sense
because each child document contain an essential part of the document.
However, in some situations it may be desirable to compose
a document from a collection of parts
without having mandatory page breaks between then.
For this case, the package
provides a mechanism to include parts
by |\input| which can also be processed individually.
However, by construction this mechanism
requires manual handling of the content to be output.

%%%%%%%%%%%%%%%%%%%%%%%%%%%%%%%%%%%%%%%%
\DescribeMacro{\ifchilddocmanual}
The main file should be prepared as usual, see \secref{sec:include}.
However, the document body must make a distinction
between processing of an individual part and of the main document, e.g.:
%
\begin{center}
\begin{tabular}{l}
|\ifchilddocmanual|\\
|\input{\childdocname}|\\
|\||else|\\
\textit{document body with }|\input{|\textit{part}|}|\\
|\||fi|
\end{tabular}
\end{center}
%
The conditional |\ifchilddocmanual| is true whenever
a part to be included by |\input| is being compiled,
and the name of the part is stored in |\childdocname|.

%%%%%%%%%%%%%%%%%%%%%%%%%%%%%%%%%%%%%%%%
\DescribeMacro{\childdocby}
Each part to be included by |\input| should start with:
%
\begin{center}
\begin{tabular}{l}
|\input{childdoc.def}|\\
|\childdocby{|\textit{main}|}|\\
\end{tabular}
\end{center}
%
The directive |\childdocby| is similar to |\childdocof|
described in \secref{sec:include},
but the subsequent selection of content must be done manually.
To that end, both |\ifchilddoc| and |\ifchilddocmanual|
will be true upon processing of a part,
and the name of the part is stored in |\childdocname|.
Note that |\jobname| will be set to the filename of the current part
so that each part receives an individual |.aux| file
that does not interfere with the |.aux| file(s) of the main document.
This behaviour can be altered by the alternative form
|\childdocby[*]{|\textit{main}|}| (with a non-empty optional argument)
which uses the |.aux| file of the main document
by setting |\jobname| to \textit{main}.

%%%%%%%%%%%%%%%%%%%%%%%%%%%%%%%%%%%%%%%%%%%%%%%%%%%%%%%%%%%%%%%%%%%%%%%%%%%%%%%%
\subsection{Driver Development}
\label{sec:driver}

The \textsf{childdoc} mechanism can also be use for the development
of definition files such as \LaTeX{} styles or classes.
This case differs from the above setup with multiple parts
included by |\include| in that no |\includeonly| should be invoked.
This can be achieved by starting the include file
(before |\ProvidesPackage|) with:
%
\begin{center}
\begin{tabular}{l}
|\input{childdoc.def}|\\
|\childdocforward{|\textit{main}|}|\\
\end{tabular}
\end{center}
%
or alternatively with:
%
\begin{center}
\begin{tabular}{l}
|\input{childdoc.def}|\\
|\childdocby{|\textit{main}|}|\\
\end{tabular}
\end{center}
%
Both forms have slightly different effects as described above.
The main file is prepared as usual, see \secref{sec:include}.

%%%%%%%%%%%%%%%%%%%%%%%%%%%%%%%%%%%%%%%%%%%%%%%%%%%%%%%%%%%%%%%%%%%%%%%%%%%%%%%%
\subsection{Legacy Detection}
\label{sec:detection}

The directive |\childdocmain| in the main file can detect
whether the complete document or merely a child is to be compiled
even without using the directive |\childdocof|.
This method is deprecated because it is less robust
and there is no compelling reason to use it;
it is merely provided for backward compatibility
and it may be removed in future versions.

If the detection mechanism is to be used,
it is mandatory to correctly specify
the filename of the main file as the argument of |\childdocmain|:
%
\begin{center}
\begin{tabular}{l}
|\input{childdoc.def}|\\
|\childdocmain{|\textit{main}|}|\\
\end{tabular}
\end{center}
%
If |\jobname| does not match the argument \textit{main} of |\childdocmain|,
it is assumed that |\jobname| points to the child file to be compiled.
When using |\childdocmain| with the main file specified as argument,
it suffices to start a child file
with just |\input{|\textit{main}|}|
without loading of the package and using |\childdocof|.
If instead all processing is done
with the appropriate \textsf{childdoc} directives,
the argument of \textit{main} of |\childdocmain| can be empty.

An alternative version of the command line processing described
in \secref{sec:commandline} using the detection mechanism reads:
%
\begin{center}
|... -jobname "|\textit{target}|" "|[\textit{flags}]%
[|\def\jobname{|\textit{dest}|}|]|\input{|\textit{main}|}"|
\end{center}

%%%%%%%%%%%%%%%%%%%%%%%%%%%%%%%%%%%%%%%%%%%%%%%%%%%%%%%%%%%%%%%%%%%%%%%%%%%%%%%%
\subsection{Manual Code}
\label{sec:manual}

In case one cannot be certain whether the definitions file |childdoc.def|
is installed on the target \TeX{} distribution
and one prefers not to ship it,
it is conceivable to paste a few relevant commands into the sources.

To that end, drop all statements |\input{childdoc.def}|
and perform the replacements as outlined below.
Instead of |\childdocmain{|\textit{main}|}| add the following code
to the top of the main file:
%
\begin{center}
\begin{tabular}{l}
|\||ifdefined\childdocname\endinput\||fi\newif\ifchilddoc|\\
|\edef\childdocname{\scantokens\expandafter{\jobname\noexpand}}|\\
|\def\childdocmain{|\textit{main}|}\||ifx\childdocmain\childdocname\||else|\\
|\childdoctrue\includeonly{\childdocname}\let\jobname\childdocmain\||fi|\\
\end{tabular}
\end{center}
%
Instead of |\childdocof{|\textit{main}|}| just include the main file
at the top of each child file:
%
\begin{center}
|\input{|\textit{main}|}|
\end{center}
%
A simple redirection |\childdocforward{|\textit{dest}|}| is achieved by:
%
\begin{center}
|\def\jobname{|\textit{dest}|}\input{\jobname}|
\end{center}
%
The redirection with prefix
|\childdocforwardprefix[|\textit{prefix}|]{|\textit{dest}|}|
is accomplished by:
%
\begin{center}
\begin{tabular}{l}
|{\edef\jobname{\scantokens\expandafter{\jobname\noexpand}}|\\
|\def\redirectjob |\textit{prefix}|#1~~~{\gdef\jobname{|\textit{dest}|#1}}|\\
|\expandafter\redirectjob\jobname~~~}\input{\jobname}|
\end{tabular}
\end{center}

In an alternative approach,
child documents can be compiled by a specific command line
without additional code or specific definitions:
%
\begin{center}
|... -jobname "|\textit{target}|" "|[\textit{flags}]%
|\includeonly{|\textit{dest}|}\input{|\textit{main}|}"|
\end{center}
%

%%%%%%%%%%%%%%%%%%%%%%%%%%%%%%%%%%%%%%%%%%%%%%%%%%%%%%%%%%%%%%%%%%%%%%%%%%%%%%%%
%%%%%%%%%%%%%%%%%%%%%%%%%%%%%%%%%%%%%%%%%%%%%%%%%%%%%%%%%%%%%%%%%%%%%%%%%%%%%%%%
\section{Information}

%%%%%%%%%%%%%%%%%%%%%%%%%%%%%%%%%%%%%%%%%%%%%%%%%%%%%%%%%%%%%%%%%%%%%%%%%%%%%%%%
\subsection{Copyright}

Copyright \copyright{} 2017--2018 Niklas Beisert

This work may be distributed and/or modified under the
conditions of the \LaTeX{} Project Public License, either version 1.3
of this license or (at your option) any later version.
The latest version of this license is in
  \url{http://www.latex-project.org/lppl.txt}
and version 1.3 or later is part of all distributions of \LaTeX{}
version 2005/12/01 or later.

This work has the LPPL maintenance status `maintained'.

The Current Maintainer of this work is Niklas Beisert.

This work consists of the files |README.txt|, |childdoc.ins| and |childdoc.dtx|
as well as the derived files |childdoc.def|, |cdocsamp.tex|
with |cdocsch1.tex|, |cdocsch2.tex|, |cdocspt3.tex|, |cdocspt4.tex|,
|cdocsdrf.tex|, |cdocsfn1.tex|, |cdocsfn2.tex|
as well as |childdoc.pdf|.

%%%%%%%%%%%%%%%%%%%%%%%%%%%%%%%%%%%%%%%%%%%%%%%%%%%%%%%%%%%%%%%%%%%%%%%%%%%%%%%%
\subsection{Files and Installation}

The package consists of the files:
%
\begin{center}
\begin{tabular}{ll}
    |README.txt|   & readme file \\
    |childdoc.ins| & installation file \\
    |childdoc.dtx| & source file \\
    |childdoc.def| & definition file \\
    |cdocsamp.tex| & sample main file \\
    |cdocsch1.tex| & sample include file \\
    |cdocsch2.tex| & sample include file \\
    |cdocspt3.tex| & sample part file \\
    |cdocspt4.tex| & sample part file \\
    |cdocsdrf.tex| & sample redirection file \\
    |cdocsfn1.tex| & sample redirection file \\
    |cdocsfn2.tex| & sample redirection file \\
    |childdoc.pdf| & manual
\end{tabular}
\end{center}
%
The distribution consists of the files
|README.txt|, |childdoc.ins| and |childdoc.dtx|.
%
\begin{itemize}
\item
Run (pdf)\LaTeX{} on |childdoc.dtx|
to compile the manual |childdoc.pdf| (this file).
\item
Run \LaTeX{} on |childdoc.ins| to create the definitions file |childdoc.def|
and the sample |cdocsamp.tex| with include files
|cdocsch1.tex|, |cdocsch2.tex|, |cdocspt3.tex|, |cdocspt4.tex|,
|cdocsdrf.tex|, |cdocsfn1.tex|, |cdocsfn2.tex|.
Then copy the file |childdoc.def| to an appropriate directory of your \LaTeX{}
distribution, e.g.\ \textit{texmf-root}|/tex/latex/childdoc|.
\end{itemize}

%%%%%%%%%%%%%%%%%%%%%%%%%%%%%%%%%%%%%%%%%%%%%%%%%%%%%%%%%%%%%%%%%%%%%%%%%%%%%%%%
\subsection{Related CTAN Packages}

There are several other packages which offer a similar functionality:
%
\begin{itemize}
\item
The packages
\href{http://ctan.org/pkg/docmute}{\textsf{docmute}},
\href{http://ctan.org/pkg/includex}{\textsf{includex}} and
\href{http://ctan.org/pkg/standalone}{\textsf{standalone}}
provide commands to include only the document body of
a child file thus allowing both files to be compiled individually.
\item
The packages \href{http://ctan.org/pkg/subdocs}{\textsf{subdocs}}
and \href{http://ctan.org/pkg/subfiles}{\textsf{subfiles}}
provide structures in which the main and child documents can be
encapsulated and allowing them to be compiled individually.
The inclusion mechanism is different from the conventional |\include|.
\item
The package \href{http://ctan.org/pkg/combine}{\textsf{combine}}
is an elaborate solution to combine several documents into one.
\end{itemize}
%
See also the CTAN topic \href{http://ctan.org/topic/subdocs}{\textsf{subdocs}}
for further related packages.
The present package differs from the above solutions in that
a document structure constructed with the conventional |\include| mechanism
just needs two extra commands at the top of every file
such that all constituent files can be compiled individually.

%%%%%%%%%%%%%%%%%%%%%%%%%%%%%%%%%%%%%%%%%%%%%%%%%%%%%%%%%%%%%%%%%%%%%%%%%%%%%%%%
%\subsection{Feature Suggestions}
%
%The following is a list of features which may be useful for future
%versions of this package:
%%
%\begin{itemize}
%\item
%\ldots
%\end{itemize}

%%%%%%%%%%%%%%%%%%%%%%%%%%%%%%%%%%%%%%%%%%%%%%%%%%%%%%%%%%%%%%%%%%%%%%%%%%%%%%%%
\subsection{Revision History}

%%%%%%%%%%%%%%%%%%%%%%%%%%%%%%%%%%%%%%%%
\paragraph{v2.0:} 2018/12/30

\begin{itemize}
\item
immediate forward processing
\item
added |\childdocby| mechanism
\item
manual restructured
\end{itemize}

%%%%%%%%%%%%%%%%%%%%%%%%%%%%%%%%%%%%%%%%
\paragraph{v1.6:} 2018/01/17

\begin{itemize}
\item
application for development of include files
\item
corrections to manual
\end{itemize}

%%%%%%%%%%%%%%%%%%%%%%%%%%%%%%%%%%%%%%%%
\paragraph{v1.5:} 2017/05/21

\begin{itemize}
\item
more complete structuring introduced
\item
|\childdocof| introduced
\item
|\childdoc| renamed to |\childdocmain|
\item
|\childredirect| renamed to |\childdocforward| and |\childdocforwardprefix|
and functionality expanded
\end{itemize}

%%%%%%%%%%%%%%%%%%%%%%%%%%%%%%%%%%%%%%%%
\paragraph{v1.0:} 2017/04/27

\begin{itemize}
\item
manual and install package
\item
first version published on CTAN
\end{itemize}

%%%%%%%%%%%%%%%%%%%%%%%%%%%%%%%%%%%%%%%%
\paragraph{v0.6:} 2017/04/26

\begin{itemize}
\item
redirection mechanism added
\end{itemize}

%%%%%%%%%%%%%%%%%%%%%%%%%%%%%%%%%%%%%%%%
\paragraph{v0.5:} 2017/04/26

\begin{itemize}
\item
functionality in definition file
\end{itemize}


%%%%%%%%%%%%%%%%%%%%%%%%%%%%%%%%%%%%%%%%%%%%%%%%%%%%%%%%%%%%%%%%%%%%%%%%%%%%%%%%
%%%%%%%%%%%%%%%%%%%%%%%%%%%%%%%%%%%%%%%%%%%%%%%%%%%%%%%%%%%%%%%%%%%%%%%%%%%%%%%%
%%%%%%%%%%%%%%%%%%%%%%%%%%%%%%%%%%%%%%%%%%%%%%%%%%%%%%%%%%%%%%%%%%%%%%%%%%%%%%%%
\appendix

\settowidth\MacroIndent{\rmfamily\scriptsize 000\ }

 \DocInput{childdoc.dtx}

\end{document}
%</driver>
% \fi
%
% %%%%%%%%%%%%%%%%%%%%%%%%%%%%%%%%%%%%%%%%%%%%%%%%%%%%%%%%%%%%%%%%%%%%%%%%%%%%%%
% %%%%%%%%%%%%%%%%%%%%%%%%%%%%%%%%%%%%%%%%%%%%%%%%%%%%%%%%%%%%%%%%%%%%%%%%%%%%%%
% \section{Sample}
%\iffalse
%<*samplemain>
%\fi
%
% The following presents a sample document
% with two chapters, two parts, a title page,
% a compile flag as well as three forwarding files to set the flag.
% It consists of eight |.tex| files:
% \begin{center}
% \begin{tabular}{ll}
% |cdocsamp.tex|&main file\\
% |cdocsch1.tex|&include file for chapter 1\\
% |cdocsch2.tex|&include file for chapter 2\\
% |cdocspt3.tex|&include file for part 3\\
% |cdocspt4.tex|&include file for part 4\\
% |cdocsdrf.tex|&forwarding file for main file in draft mode\\
% |cdocsfi1.tex|&forwarding file for final version of chapter 1\\
% |cdocsfi2.tex|&forwarding file for final version of chapter 2\\
% \end{tabular}
% \end{center}
% Each of the eight files can be compiled directly by the \LaTeX{} compiler.
%
% %%%%%%%%%%%%%%%%%%%%%%%%%%%%%%%%%%%%%%
% \paragraph{Main File.}
%
% The main file is called |cdocsamp.tex|.
%
% Load the \textsf{childdoc} definitions and
% declare the filename for the main document:
%    \begin{macrocode}
\input{childdoc.def}
\childdocmain{}
%    \end{macrocode}

% Optional override for |\version| flag:
%    \begin{macrocode}
%%\ifchilddoc\else\providecommand{\version}{draft}\fi
%    \end{macrocode}

% Define the default values for the |\version| flag
% (|final| for the main file and |draft| for childs):
%    \begin{macrocode}
\ifchilddoc
\providecommand{\version}{draft}
\else
\providecommand{\version}{final}
\fi
%    \end{macrocode}

% Load the standard document class:
%    \begin{macrocode}
\documentclass[12pt]{article}
%    \end{macrocode}

% Start the document body:
%    \begin{macrocode}
\begin{document}
%    \end{macrocode}

% Declare a title page.
% Print title, part of document being processed and version flag:
%    \begin{macrocode}
\addtocounter{page}{-1}
\begin{center}
{\LARGE\bfseries{}childdoc example\par}
\vspace{1cm}
\ifchilddoc
\ifchilddocmanual part\else chapter\fi:
`\childdocname' of `\childdocjob'\par
\else
main document: `\childdocjob'\par
\fi
version: \version\par
\end{center}
\newpage
%    \end{macrocode}

% Manually include selected file,
% otherwise process as usual:
%    \begin{macrocode}
\ifchilddocmanual
\section*{part `\childdocname'}
\input{\childdocname}
\else
%    \end{macrocode}

% Include the two chapters:
%    \begin{macrocode}
\include{cdocsch1}
\include{cdocsch2}
%    \end{macrocode}

% Include the two parts unless only chapters should be displayed:
%    \begin{macrocode}
\ifchilddoc\else
\section{part three}
\input{cdocspt3}
\section{part four}
\input{cdocspt4}
\fi
%    \end{macrocode}

% Process as usual until here:
%    \begin{macrocode}
\fi
%    \end{macrocode}

% End of document body:
%    \begin{macrocode}
\end{document}
%    \end{macrocode}
%\iffalse
%</samplemain>
%\fi
%
% %%%%%%%%%%%%%%%%%%%%%%%%%%%%%%%%%%%%%%
% \paragraph{Chapter Include Files.}
%
% The include files are called |cdocsch1.tex| and |cdocsch2.tex|.
%
%\iffalse
%<*samplechap1|samplechap2>
%\fi

% Optional override for |\version| flag:
%    \begin{macrocode}
%%\providecommand{\version}{final}
%    \end{macrocode}

% Include the main document:
%    \begin{macrocode}
\input{childdoc.def}
\childdocof{cdocsamp}
%    \end{macrocode}

%\iffalse
%</samplechap1|samplechap2>
%\fi
%
%\iffalse
%<*samplechap1>
%\fi
% Some text for chapter 1:
%    \begin{macrocode}
\section{one}
some text in chapter one
%    \end{macrocode}

%\iffalse
%</samplechap1>
%\fi
% Some text for chapter 2:
%\iffalse
%<*samplechap2>
%\fi
%    \begin{macrocode}
\section{two}
more text in chapter two
%    \end{macrocode}

%\iffalse
%</samplechap2>
%\fi
%
% %%%%%%%%%%%%%%%%%%%%%%%%%%%%%%%%%%%%%%
% \paragraph{Part Include Files.}
%
% The include files are called |cdocspt3.tex| and |cdocspt4.tex|.
%
%\iffalse
%<*samplepart3|samplepart4>
%\fi

% Optional override for |\version| flag:
%    \begin{macrocode}
%%\providecommand{\version}{final}
%    \end{macrocode}

% Include the main document:
%    \begin{macrocode}
\input{childdoc.def}
\childdocby{cdocsamp}
%    \end{macrocode}

%\iffalse
%</samplepart3|samplepart4>
%\fi
%
%\iffalse
%<*samplepart3>
%\fi
% Some text for part 3:
%    \begin{macrocode}
some text in part three
%    \end{macrocode}

%\iffalse
%</samplepart3>
%\fi
% Some text for part 4:
%\iffalse
%<*samplepart4>
%\fi
%    \begin{macrocode}
more text in part four
%    \end{macrocode}

%\iffalse
%</samplepart4>
%\fi
%
% %%%%%%%%%%%%%%%%%%%%%%%%%%%%%%%%%%%%%%
% \paragraph{Forwarding for a Complete Draft.}
%
% The following forwarding file |cdocsdrf.tex|
% compiles the main document in draft mode:
%\iffalse
%<*sampledraft>
%\fi
%    \begin{macrocode}
\def\version{draft}
\input{childdoc.def}
\childdocforward{cdocsamp}
%    \end{macrocode}

%\iffalse
%</sampledraft>
%\fi
%
% %%%%%%%%%%%%%%%%%%%%%%%%%%%%%%%%%%%%%%
% \paragraph{Forwarding for Final Version of the Chapters.}
%
% The following forwarding files |cdocsfn1.tex| and |cdocsfn2.tex|
% (with identical content)
% compile the final versions of the child documents
% |cdocsch1.tex| and |cdocsch2.tex|, respectively:
%\iffalse
%<*samplefinal>
%\fi
%    \begin{macrocode}
\def\version{final}
\input{childdoc.def}
\childdocforwardprefix[cdocsamp]{cdocsfn}{cdocsch}
%    \end{macrocode}

%\iffalse
%</samplefinal>
%\fi
%
% %%%%%%%%%%%%%%%%%%%%%%%%%%%%%%%%%%%%%%
% \paragraph{Command Line Processing.}
%
% The following three command lines generate the output files
% |cdocscld|, |cdocscl1| and |cdocscl2|
% which should be identical to
% |cdocsdrf|, |cdocsch1| and |cdocsfn2|, respectively:
% \begin{center}
% \begin{tabular}{l}
% |latex -jobname cdocscld \|\\
% |  "\def\version{draft}\input{childdoc.def}\childdocforward{cdocsamp}"|\\
% |latex -jobname cdocscl1 \|\\
% |  "\input{childdoc.def}\childdocforward[cdocsamp]{cdocsch1}"|\\
% |latex -jobname cdocscl2 \|\\
% |  "\def\version{final}\input{childdoc.def}\childdocforward{cdocsch2}"|
% \end{tabular}
% \end{center}
% Note that the trailing backslash on each first line
% merely continues the input to the second line
% (for convenient cut ant paste).
% Furthermore, the command |latex| can be replaced by any
% of its alternative versions such as |pdflatex|.
%
% %%%%%%%%%%%%%%%%%%%%%%%%%%%%%%%%%%%%%%%%%%%%%%%%%%%%%%%%%%%%%%%%%%%%%%%%%%%%%%
% %%%%%%%%%%%%%%%%%%%%%%%%%%%%%%%%%%%%%%%%%%%%%%%%%%%%%%%%%%%%%%%%%%%%%%%%%%%%%%
% \section{Implementation}
%\iffalse
%<*package>
%\fi
%
% This section describes the definitions file |childdoc.def|.

% The definitions cannot be loaded using |\usepackage| or |\RequirePackage|
% which has a mechanism to prevent loading a style file more than once.
% When loading the definitions by means of |\input|
% multiple instances have to be prevented manually:
%\iffalse
%This code needs to be before the `\ProvidesFile' directive
%which is defined at the beginning of this file.
%Therefore it is also placed there and commented out here.
%</package>
%<*discard>
%\fi
%    \begin{macrocode}
\ifdefined\childdocmain\endinput\fi
%    \end{macrocode}
%\iffalse
%</discard>
%<*package>
%\fi
%
% \macro{\ifchilddoc}
% \macro{\ifchilddocmanual}
% The conditional |\ifchilddoc| tells whether a
% child (true) or main (false) document is being compiled.
% The conditional |\ifchilddocmanual| tells whether
% the |\includeonly| mechanism is used (false) or
% the selection of child files must be performed manually (true).
% The definitions initialise to false:
%    \begin{macrocode}
\newif\ifchilddoc
\newif\ifchilddocmanual
%    \end{macrocode}

% \macro{\childdocname}
% \macro{\childdocjob}
% The macro |\childdocname| stores the name of the main document
% to be compiled. The macro |\childdocjob| stores the name of
% the document on which the \LaTeX{} compiler was originally invoked.
% The content of |\jobname| cannot be compared
% to filenames specified in the source due to different catcodes.
% The following code rescans |\jobname|, stores the result
% in |\childdocname| and saves a copy in |\childdocjob|:
%    \begin{macrocode}
\edef\childdocname{\scantokens\expandafter{\jobname\noexpand}}
\let\childdocjob\childdocname
%    \end{macrocode}

% \macro{\childdocdisable}
% The macro |\childdocdisable| prevents the main file
% from being processed more than once.
% At this stage, the main document command |\childdocmain|
% is assumed to be called once again where it should do nothing.
% Any subsequent call to it should prevent
% a secondary processing of the main document
% It overwrites the forwarding commands
% |\childdocof| and |\childdocforward|
% with empty macros to prevent further inclusions of the main document:
%    \begin{macrocode}
\newcommand{\childdocdisable}
{
  \renewcommand{\childdocmain}[1]{\renewcommand{\childdocmain}[1]{\endinput}}
  \renewcommand{\childdocof}[1]{}
  \renewcommand{\childdocby}[2][]{}
  \renewcommand{\childdocforward}[2][]{}
  \renewcommand{\childdocdisable}{}
}
%    \end{macrocode}

% \macro{\childdocmain}
% The macro |\childdocmain| is to be called at the top of the main file
% with nothing or the main filename (without extension) as argument.
% First, it breaks loops.
% If the argument is not empty and does not match |\childdocname|
% (which is set by the first inclusion of |childdoc.def|),
% |\ifchilddoc| is set to true, |\includeonly| is applied to the child file
% and |\jobname| is set to the main file
% (for proper handling of |.aux| files):
%    \begin{macrocode}
\newcommand{\childdocmain}[1]
{
  \childdocdisable\childdocmain{}
  \if?#1?\else
    \begingroup
      \def\childdoctmp{#1}
      \ifx\childdoctmp\childdocname
        \def\childdoctmp{}
      \else
        \def\childdoctmp
        {
          \childdoctrue
          \includeonly{\childdocname}
          \def\childdocjob{#1}
          \def\jobname{#1}
        }
      \fi
      \expandafter
    \endgroup
    \childdoctmp
  \fi
}
%    \end{macrocode}

% \macro{\childdocof}
% The command |\childdocof| redirects
% compilation to the main file |#1|.
%    \begin{macrocode}
\newcommand{\childdocof}[1]
{
  \childdocdisable
  \childdoctrue
  \includeonly{\childdocname}
  \def\jobname{#1}
  \def\childdocjob{#1}
  \input{#1}
}
%    \end{macrocode}

% \macro{\childdocby}
% The command |\childdocby| ....
%    \begin{macrocode}
\newcommand{\childdocby}[2][]
{
  \childdocdisable
  \childdoctrue
  \childdocmanualtrue
  \if?#1?\else
    \def\jobname{#2}
  \fi
  \def\childdocjob{#2}
  \input{#2}
  \endinput
}
%    \end{macrocode}

% \macro{\childdocforward}
% The command |\childdocforward| redirects
% compilation to the main file or
% (if the optional argument is given) a child file.
% Parameters are set as if the main file
% or a child file starting with |\childdocof| was compiled.
% Then compilation is handed over to the main file:
%    \begin{macrocode}
\newcommand{\childdocforward}[2][]
{
  \begingroup
    \if?#1?
      \def\childdoctmp
      {
        \def\childdocname{#2}
        \def\childdocjob{#2}
        \def\jobname{#2}
        \input{#2}
        \endinput
      }
    \else
      \def\childdoctmp
      {
        \childdocdisable
        \def\childdocname{#2}
        \childdoctrue
        \includeonly{#2}
        \def\childdocjob{#1}
        \def\jobname{#1}
        \input{#1}
        \endinput
      }
    \fi
    \expandafter
  \endgroup
  \childdoctmp
}
%    \end{macrocode}

% \macro{\childdocforwardprefix}
% The command |\childdocforwardprefix| redirects
% compilation to the main or a child file by means of a pattern.
% The prefix |#1| in the current filename is replaced by |#2|
% and the suffix of the current filename is kept
% (it is assumed that the filename does not contain the substring `|~~~|'
% which is used as a delimiter).
% Compilation is handed over to the new file by |\childdocforward|:
%    \begin{macrocode}
\newcommand{\childdocforwardprefix}[3][]
{
  \begingroup
    \def\childdocextract #2##1~~~{\def\childdoctmp{\childdocforward[#1]{#3##1}}}
    \expandafter\childdocextract\childdocname~~~
    \expandafter
  \endgroup
  \childdoctmp
}
%    \end{macrocode}

% \macro{\childdoc}
% The deprecated macro |\childdoc| is a legacy version of |\childdocmain|:
%    \begin{macrocode}
\newcommand{\childdoc}{\childdocmain}
%    \end{macrocode}

% \macro{\childdocredirect}
% The deprecated macro |\childdocredirect| is a legacy version
% of |\childdocforward| and |\childdocforwardprefix|:
%    \begin{macrocode}
\newcommand{\childdocredirect}[2][]
{
  \begingroup
    \if?#1?
      \def\childdoctmp{\childdocforward{#2}}
    \else
      \def\childdoctmp{\childdocforwardprefix{#1}{#2}}
    \fi
    \expandafter
  \endgroup
  \childdoctmp
}
%    \end{macrocode}

%\iffalse
%</package>
%\fi
%
\endinput
|\\
|\childdocof{|\textit{main}|}|\\
\end{tabular}
\end{center}
at the top of every child file \textit{child}
which is included by |\include{|\textit{child}|}|
from within the main file
(or at least for those files to be compiled individually).
The argument \textit{main} must be the filename of the main file.

There are a couple of
considerations in setting up the main and child documents:

%%%%%%%%%%%%%%%%%%%%%%%%%%%%%%%%%%%%%%%%
\paragraph{Restrictions.}

Please note the following restrictions:
\begin{itemize}
\item
|\childdocmain| must be called with one argument \textit{main}
to ensure compatibility with earlier version of the package.
It must either be empty (|\childdocmain{}|)
or precisely match the filename of the main file in which it is specified.
See \secref{sec:detection} for further information.
\item
The filename \textit{main} must be specified without the |.tex| extension.
\item
The filename \textit{main} is case sensitive
(even in case-insensitive file systems)
due to internal string comparison.
\item
The argument \textit{main} should be fully expanded, it cannot be a macro.
\item
Subdirectories and special characters should be avoided in filenames.
\item
The command |\childdocmain{|\textit{main}|}| must be followed by a whitespace.
It should not be followed immediately by another command
or by a comment mark `|%|'.
This is because the \TeX{} parser reads the token immediately following
the argument of |\childdocmain| and puts it
at the beginning of every child section;
however, a white\-space is ignored.
\end{itemize}

%%%%%%%%%%%%%%%%%%%%%%%%%%%%%%%%%%%%%%%%
\paragraph{Content of Main File.}

It is advisable to place all content in the child files included by |\include|.
Any output contained in the main file will appear in all child documents
unless suppressed manually;
it cannot be suppressed automatically by the |\includeonly| directive
and thus should normally be avoided.
A method to include some content in the main file
by means of conditional processing is described in \secref{sec:conditional}.

%%%%%%%%%%%%%%%%%%%%%%%%%%%%%%%%%%%%%%%%
\paragraph{Page Numbering.}

When only a part of the document is compiled,
the appropriate numbering of pages
(as well as other status parameters)
is determined from the |.aux| files.
The latter contain information from previous passes.
However this information needs to propagate through
all intermediate child documents.
Therefore the page numbering in child documents may well
be inconsistent until the complete document is compiled at least once.

A useful (if unconventional) way to always ensure a consistent
page numbering is to restart the numbering in each child document
and denote the pages by `\textit{child}|.|\textit{page}'
where \textit{child} represents the chapter/section number of the child file.
This can be achieved by the command
|\numberwithin{page}{|\textit{child}|}|
of the \textsf{amsmath} package
where \textit{child} can be |chapter| or |section|
depending on the chosen structuring.
Alternatively, one can modify the macro |\thepage| appropriately
and reset the counter |page| at the start of each child file.

%%%%%%%%%%%%%%%%%%%%%%%%%%%%%%%%%%%%%%%%%%%%%%%%%%%%%%%%%%%%%%%%%%%%%%%%%%%%%%%%
\subsection{Conditional Processing}
\label{sec:conditional}

The package provides a mechanism to compile different versions
of a document. To customise the versions further some conditional processing
can come in handy to distinguish which version is being compiled.
The package provides two macros to describe the compilation context:

%%%%%%%%%%%%%%%%%%%%%%%%%%%%%%%%%%%%%%%%
\DescribeMacro{\ifchilddoc}
The conditional |\ifchilddoc| distinguishes between the compilation of
child documents and the main document:
%
\begin{center}
|\ifchilddoc |\textit{child-code}| |[|\||else |\textit{main-code}]| \||fi|
\end{center}

%%%%%%%%%%%%%%%%%%%%%%%%%%%%%%%%%%%%%%%%
\DescribeMacro{\childdocname}
\DescribeMacro{\childdocjob}
The macro |\childdocname| contains the filename (without extension)
of the main or child file being processed.
Note that |\childdocjob| will always contain the name of the main file.

%%%%%%%%%%%%%%%%%%%%%%%%%%%%%%%%%%%%%%%%
\paragraph{Title Page.}

Conditional processing can be used to include a title or banner page
in the main document when proper precautions are taken.
Importantly, the code in the main file should ensure that the page counter
(as well as other status parameters which are stored in the |.aux| files)
takes the same value after the conditional processing.
Otherwise the page numbers may take divergent values
depending on which part is compiled.

For example, a title page could be declared by:
%
\begin{center}
\begin{tabular}{l}
|\ifchilddoc\||else|\\
|\addtocounter{page}{-1}|\\
\textit{code for title page}\\
|\newpage|\\
|\||fi|
\end{tabular}
\end{center}
%
A banner page for the child documents can be generated by:
%
\begin{center}
\begin{tabular}{l}
|\ifchilddoc|\\
|\addtocounter{page}{-1}|\\
\textit{code for banner page}\\
|\newpage|\\
|\||fi|
\end{tabular}
\end{center}
%
Here one could write a message such as:
\begin{center}
|This is the part \childdocname{} of \childdocjob{}.|
\end{center}

%%%%%%%%%%%%%%%%%%%%%%%%%%%%%%%%%%%%%%%%%%%%%%%%%%%%%%%%%%%%%%%%%%%%%%%%%%%%%%%%
\subsection{Flags}
\label{sec:flags}

The package makes it easy to generate different versions
of the main or child documents.
To this end compilation flags can be defined
and assigned different default values.
They will be particularly useful in conjunction
with the forwarding mechanism described in \secref{sec:forward}.

For example, it may be useful to have a flag |\version|
which can be set to |draft| or |final|.
The document source will contain some conditional code
depending on the value of |\version|.
Suppose further, the flag should default to |final| for the main file
and to |draft| for child files
which is a natural assignment for editing the document.
This is achieved by placing the following code
in the preamble of the main document
(below the |\childdocmain| directive):
%
\begin{center}
\begin{tabular}{l}
|\ifchilddoc|\\
|\providecommand{\version}{draft}|\\
|\||else|\\
|\providecommand{\version}{final}|\\
|\||fi|
\end{tabular}
\end{center}
%
The definition by |\providecommand| makes sure
that previous definitions are not overwritten.
Further statements |\providecommand{\version}{...}|
can thus be added before the above code to override it.

For the main file, one might add a line
(between |\childdocmain| and the above block)
%
\begin{center}
|%\ifchilddoc\||else\providecommand{\version}{draft}\||fi|
\end{center}
%
which can be uncommented to produce a draft version.
Likewise one can add a line to the very top of a child file
(above the |\childdocof{|\textit{main}|}| directive)
%
\begin{center}
|%\providecommand{\version}{final}|
\end{center}
%
which can be uncommented to produce the final version of this child document.

%%%%%%%%%%%%%%%%%%%%%%%%%%%%%%%%%%%%%%%%%%%%%%%%%%%%%%%%%%%%%%%%%%%%%%%%%%%%%%%%
\subsection{Forwarding}
\label{sec:forward}

Different versions of the main or child documents
using compilation flags as described in \secref{sec:flags}
can be (permanently) stored in different files
for convenient compilation, viewing and distribution.
To this end, the package defines a command
to pass on compilation to a different file:

%%%%%%%%%%%%%%%%%%%%%%%%%%%%%%%%%%%%%%%%
\DescribeMacro{\childdocforward}
The command |\childdocforward| redirects processing to
another source file:
%
\begin{center}
\begin{tabular}{l}
|% \iffalse
%
% childdoc.dtx Copyright (C) 2017-2018 Niklas Beisert
%
% This work may be distributed and/or modified under the
% conditions of the LaTeX Project Public License, either version 1.3
% of this license or (at your option) any later version.
% The latest version of this license is in
%   http://www.latex-project.org/lppl.txt
% and version 1.3 or later is part of all distributions of LaTeX
% version 2005/12/01 or later.
%
% This work has the LPPL maintenance status `maintained'.
%
% The Current Maintainer of this work is Niklas Beisert.
%
% This work consists of the files childdoc.dtx and childdoc.ins
% and the derived files childdoc.def and cdocsamp.tex with
% cdocsch1.tex, cdocsch2.tex, cdocsdrf.tex, cdocsfn1.tex, cdocsfn2.tex.
%
%<package>\ifdefined\childdocmain\endinput\fi
%<package>\ProvidesFile{childdoc.def}[2018/12/30 v2.0 child document driver]
%<samplemain>\ProvidesFile{cdocsamp.tex}[2018/12/30 v2.0 sample for childdoc]
%<*driver>
%\ProvidesFile{childdoc.drv}[2018/12/30 v2.0 childdoc reference manual file]
\PassOptionsToClass{10pt,a4paper}{article}
\documentclass{ltxdoc}

\usepackage[margin=35mm]{geometry}
\usepackage{hyperref}
\usepackage{hyperxmp}
\usepackage[usenames]{color}

\hypersetup{colorlinks=true}
\hypersetup{pdfstartview=FitH}
\hypersetup{pdfpagemode=UseNone}
\hypersetup{pdfsource={}}
\hypersetup{pdflang={en-UK}}
\hypersetup{pdfcopyright={Copyright 2017-2018 Niklas Beisert.
  This work may be distributed and/or modified under the
  conditions of the LaTeX Project Public License, either version 1.3
  of this license or (at your option) any later version.}}
\hypersetup{pdflicenseurl={http://www.latex-project.org/lppl.txt}}
\hypersetup{pdfcontactaddress={ETH Zurich, ITP, HIT K,
  Wolfgang-Pauli-Strasse 27}}
\hypersetup{pdfcontactpostcode={8093}}
\hypersetup{pdfcontactcity={Zurich}}
\hypersetup{pdfcontactcountry={Switzerland}}
\hypersetup{pdfcontactemail={nbeisert@itp.phys.ethz.ch}}
\hypersetup{pdfcontacturl={http://people.phys.ethz.ch/\xmptilde nbeisert/}}

\newcommand{\secref}[1]{\hyperref[#1]{section \ref*{#1}}}

\parskip1ex
\parindent0pt
\let\olditemize\itemize
\def\itemize{\olditemize\parskip0pt}

\begin{document}

\title{The \textsf{childdoc} Package}
\hypersetup{pdftitle={The childdoc Package}}
\author{Niklas Beisert\\[2ex]
  Institut f\"ur Theoretische Physik\\
  Eidgen\"ossische Technische Hochschule Z\"urich\\
  Wolfgang-Pauli-Strasse 27, 8093 Z\"urich, Switzerland\\[1ex]
  \href{mailto:nbeisert@itp.phys.ethz.ch}
  {\texttt{nbeisert@itp.phys.ethz.ch}}}
\hypersetup{pdfauthor={Niklas Beisert}}
\hypersetup{pdfsubject={Manual for the LaTeX2e Package childdoc}}
\date{30 December 2018, \textsf{v2.0}}
\maketitle

\begin{abstract}\noindent
\textsf{childdoc} is a \LaTeXe{} package
that enables the direct compilation
of document sections included by |\include|
to individual files.
\end{abstract}

\begingroup
\parskip0ex
\tableofcontents
\endgroup

%%%%%%%%%%%%%%%%%%%%%%%%%%%%%%%%%%%%%%%%%%%%%%%%%%%%%%%%%%%%%%%%%%%%%%%%%%%%%%%%
%%%%%%%%%%%%%%%%%%%%%%%%%%%%%%%%%%%%%%%%%%%%%%%%%%%%%%%%%%%%%%%%%%%%%%%%%%%%%%%%
\section{Introduction}

\LaTeX{} provides a mechanism to structure a large document (such as a book)
into a main file and several child files (containing the chapters)
using the |\include| command.
This mechanism is beneficial for documents
which span hundreds of pages in order to
make the source file(s) more manageable.
Moreover, compilation can be restricted to
selected child files by means of the |\includeonly| command.
The latter feature can be used to reduce the compilation time while editing
(this was significantly more useful in the earlier days of \LaTeX{})
or to generate a smaller document which is easier to navigate.
Another application of |\includeonly| is to generate
documents consisting of selected parts of the complete document.

However, there are a few drawbacks of the plain |\include| mechanism:
\begin{itemize}
\item
The child files cannot be compiled on their own,
they can only be compiled via the main file.
A naive editing environment
(such as a text editor with an option
to have the current file processed by \LaTeX)
may require one to switch to the main file before compiling;
attempting to compile the child file produces errors.
\item
The main file must be modified (each time)
to adjust the |\includeonly| command
to the present needs. This easily leaves the main file in a messy state.
\item
The generated document will always carry the filename
of the main document. This is inconvenient if
several child files are to be compiled and
to be kept for distribution.
\end{itemize}

The present package provides a simple interface
to make child files individually compilable by \LaTeX{}.
Compiling a child file then has the same effect as compiling
the main file with an |\includeonly| command
to select the appropriate child.
Moreover the generated document will carry the name of the child
rather than the main file.
This resolves all three above issues.

This feature is meant to make the editing of books,
thesis documents and lecture notes somewhat more convenient.
However, the package can also be used efficiently for
composing a series of documents (such as exercise sheets)
which are typically distributed individually.
It then assists the author in generating the individual documents
(potentially in different versions)
as well as a document containing the collected series.
Another application is in developing style files
or other kinds of included material
where compilation of the style file could redirect
to a sample or test file.

%%%%%%%%%%%%%%%%%%%%%%%%%%%%%%%%%%%%%%%%%%%%%%%%%%%%%%%%%%%%%%%%%%%%%%%%%%%%%%%%
%%%%%%%%%%%%%%%%%%%%%%%%%%%%%%%%%%%%%%%%%%%%%%%%%%%%%%%%%%%%%%%%%%%%%%%%%%%%%%%%
\section{Usage}

First of all, the package \textsf{childdoc} is \emph{not} a standard
\LaTeXe{} |.sty| style file! Therefore it needs to be invoked in
a non-standard way.

%%%%%%%%%%%%%%%%%%%%%%%%%%%%%%%%%%%%%%%%%%%%%%%%%%%%%%%%%%%%%%%%%%%%%%%%%%%%%%%%
\subsection{Included Files}
\label{sec:include}

%%%%%%%%%%%%%%%%%%%%%%%%%%%%%%%%%%%%%%%%
\DescribeMacro{\childdocmain}
To use the package, add the commands
\begin{center}
\begin{tabular}{l}
|\input{childdoc.def}|\\
|\childdocmain{}|\\
\end{tabular}
\end{center}
at the very top of the main \LaTeX{} file,
in particular \emph{before} the |\documentclass| statement!
The argument of |\childdocmain| should be left empty
(but it must be present).

%%%%%%%%%%%%%%%%%%%%%%%%%%%%%%%%%%%%%%%%
\DescribeMacro{\childdocof}
Furthermore, add the commands
\begin{center}
\begin{tabular}{l}
|\input{childdoc.def}|\\
|\childdocof{|\textit{main}|}|\\
\end{tabular}
\end{center}
at the top of every child file \textit{child}
which is included by |\include{|\textit{child}|}|
from within the main file
(or at least for those files to be compiled individually).
The argument \textit{main} must be the filename of the main file.

There are a couple of
considerations in setting up the main and child documents:

%%%%%%%%%%%%%%%%%%%%%%%%%%%%%%%%%%%%%%%%
\paragraph{Restrictions.}

Please note the following restrictions:
\begin{itemize}
\item
|\childdocmain| must be called with one argument \textit{main}
to ensure compatibility with earlier version of the package.
It must either be empty (|\childdocmain{}|)
or precisely match the filename of the main file in which it is specified.
See \secref{sec:detection} for further information.
\item
The filename \textit{main} must be specified without the |.tex| extension.
\item
The filename \textit{main} is case sensitive
(even in case-insensitive file systems)
due to internal string comparison.
\item
The argument \textit{main} should be fully expanded, it cannot be a macro.
\item
Subdirectories and special characters should be avoided in filenames.
\item
The command |\childdocmain{|\textit{main}|}| must be followed by a whitespace.
It should not be followed immediately by another command
or by a comment mark `|%|'.
This is because the \TeX{} parser reads the token immediately following
the argument of |\childdocmain| and puts it
at the beginning of every child section;
however, a white\-space is ignored.
\end{itemize}

%%%%%%%%%%%%%%%%%%%%%%%%%%%%%%%%%%%%%%%%
\paragraph{Content of Main File.}

It is advisable to place all content in the child files included by |\include|.
Any output contained in the main file will appear in all child documents
unless suppressed manually;
it cannot be suppressed automatically by the |\includeonly| directive
and thus should normally be avoided.
A method to include some content in the main file
by means of conditional processing is described in \secref{sec:conditional}.

%%%%%%%%%%%%%%%%%%%%%%%%%%%%%%%%%%%%%%%%
\paragraph{Page Numbering.}

When only a part of the document is compiled,
the appropriate numbering of pages
(as well as other status parameters)
is determined from the |.aux| files.
The latter contain information from previous passes.
However this information needs to propagate through
all intermediate child documents.
Therefore the page numbering in child documents may well
be inconsistent until the complete document is compiled at least once.

A useful (if unconventional) way to always ensure a consistent
page numbering is to restart the numbering in each child document
and denote the pages by `\textit{child}|.|\textit{page}'
where \textit{child} represents the chapter/section number of the child file.
This can be achieved by the command
|\numberwithin{page}{|\textit{child}|}|
of the \textsf{amsmath} package
where \textit{child} can be |chapter| or |section|
depending on the chosen structuring.
Alternatively, one can modify the macro |\thepage| appropriately
and reset the counter |page| at the start of each child file.

%%%%%%%%%%%%%%%%%%%%%%%%%%%%%%%%%%%%%%%%%%%%%%%%%%%%%%%%%%%%%%%%%%%%%%%%%%%%%%%%
\subsection{Conditional Processing}
\label{sec:conditional}

The package provides a mechanism to compile different versions
of a document. To customise the versions further some conditional processing
can come in handy to distinguish which version is being compiled.
The package provides two macros to describe the compilation context:

%%%%%%%%%%%%%%%%%%%%%%%%%%%%%%%%%%%%%%%%
\DescribeMacro{\ifchilddoc}
The conditional |\ifchilddoc| distinguishes between the compilation of
child documents and the main document:
%
\begin{center}
|\ifchilddoc |\textit{child-code}| |[|\||else |\textit{main-code}]| \||fi|
\end{center}

%%%%%%%%%%%%%%%%%%%%%%%%%%%%%%%%%%%%%%%%
\DescribeMacro{\childdocname}
\DescribeMacro{\childdocjob}
The macro |\childdocname| contains the filename (without extension)
of the main or child file being processed.
Note that |\childdocjob| will always contain the name of the main file.

%%%%%%%%%%%%%%%%%%%%%%%%%%%%%%%%%%%%%%%%
\paragraph{Title Page.}

Conditional processing can be used to include a title or banner page
in the main document when proper precautions are taken.
Importantly, the code in the main file should ensure that the page counter
(as well as other status parameters which are stored in the |.aux| files)
takes the same value after the conditional processing.
Otherwise the page numbers may take divergent values
depending on which part is compiled.

For example, a title page could be declared by:
%
\begin{center}
\begin{tabular}{l}
|\ifchilddoc\||else|\\
|\addtocounter{page}{-1}|\\
\textit{code for title page}\\
|\newpage|\\
|\||fi|
\end{tabular}
\end{center}
%
A banner page for the child documents can be generated by:
%
\begin{center}
\begin{tabular}{l}
|\ifchilddoc|\\
|\addtocounter{page}{-1}|\\
\textit{code for banner page}\\
|\newpage|\\
|\||fi|
\end{tabular}
\end{center}
%
Here one could write a message such as:
\begin{center}
|This is the part \childdocname{} of \childdocjob{}.|
\end{center}

%%%%%%%%%%%%%%%%%%%%%%%%%%%%%%%%%%%%%%%%%%%%%%%%%%%%%%%%%%%%%%%%%%%%%%%%%%%%%%%%
\subsection{Flags}
\label{sec:flags}

The package makes it easy to generate different versions
of the main or child documents.
To this end compilation flags can be defined
and assigned different default values.
They will be particularly useful in conjunction
with the forwarding mechanism described in \secref{sec:forward}.

For example, it may be useful to have a flag |\version|
which can be set to |draft| or |final|.
The document source will contain some conditional code
depending on the value of |\version|.
Suppose further, the flag should default to |final| for the main file
and to |draft| for child files
which is a natural assignment for editing the document.
This is achieved by placing the following code
in the preamble of the main document
(below the |\childdocmain| directive):
%
\begin{center}
\begin{tabular}{l}
|\ifchilddoc|\\
|\providecommand{\version}{draft}|\\
|\||else|\\
|\providecommand{\version}{final}|\\
|\||fi|
\end{tabular}
\end{center}
%
The definition by |\providecommand| makes sure
that previous definitions are not overwritten.
Further statements |\providecommand{\version}{...}|
can thus be added before the above code to override it.

For the main file, one might add a line
(between |\childdocmain| and the above block)
%
\begin{center}
|%\ifchilddoc\||else\providecommand{\version}{draft}\||fi|
\end{center}
%
which can be uncommented to produce a draft version.
Likewise one can add a line to the very top of a child file
(above the |\childdocof{|\textit{main}|}| directive)
%
\begin{center}
|%\providecommand{\version}{final}|
\end{center}
%
which can be uncommented to produce the final version of this child document.

%%%%%%%%%%%%%%%%%%%%%%%%%%%%%%%%%%%%%%%%%%%%%%%%%%%%%%%%%%%%%%%%%%%%%%%%%%%%%%%%
\subsection{Forwarding}
\label{sec:forward}

Different versions of the main or child documents
using compilation flags as described in \secref{sec:flags}
can be (permanently) stored in different files
for convenient compilation, viewing and distribution.
To this end, the package defines a command
to pass on compilation to a different file:

%%%%%%%%%%%%%%%%%%%%%%%%%%%%%%%%%%%%%%%%
\DescribeMacro{\childdocforward}
The command |\childdocforward| redirects processing to
another source file:
%
\begin{center}
\begin{tabular}{l}
|\input{childdoc.def}|\\
|\childdocforward[|\textit{main}|]{|\textit{dest}|}|\\
\end{tabular}
\end{center}
%
The argument \textit{dest} is the destination file
(without extension).
It should be the main file or one of the child files.
Note that further \textsf{childdoc} directives
such as |\childdocof| and |\childdocforward|
in the indicated file will be processed in this form.
The optional argument \textit{main}
passes on directly to the main file \textit{main}
while pretending to compile the child \textit{dest}.
This form behaves as if \textit{dest}
issues |\childdocof{|\textit{main}|}| right away,
and no further \textsf{childdoc} directives will be processed.

%%%%%%%%%%%%%%%%%%%%%%%%%%%%%%%%%%%%%%%%
\DescribeMacro{\...prefix}
In the alternative form |\childdocforwardprefix|,
%
\begin{center}
\begin{tabular}{l}
|\input{childdoc.def}|\\
|\childdocforwardprefix[|\textit{main}|]{|\textit{prefix}|}{|\textit{dest}|}|
\end{tabular}
\end{center}
%
the destination file is determined by a pattern
depending on the current file:
To make this work, the current file must be called
`{\textit{prefix}\hspace{0.2em}\textit{suffix}}'
with \textit{prefix} matching precisely the argument.
Processing is then passed on to the file
`{\textit{dest}\hspace{0.2em}\textit{suffix}}'.
Surely, the same effect is achieved by
directly specifying the
argument `{\textit{dest}\hspace{0.2em}\textit{suffix}}'
in the first form.
However, that requires to set up a different file
for each child. With the alternative form of the command
all these files can have exactly the same content
which simplifies setting them up and maintaining them.

For example, the following file |draft.tex|
with a compilation flag |\version| as described in \secref{sec:flags}
compiles the main document as a draft:
%
\begin{center}
\begin{tabular}{l}
|\def\version{draft}|\\
|\input{childdoc.def}|\\
|\childdocforward{|\textit{main}|}|
\end{tabular}
\end{center}
%
Likewise, the following files |final|\textit{nn}|.tex|
compile the final version of the child document
|child|\textit{nn}|.tex|:
%
\begin{center}
\begin{tabular}{l}
|\def\version{final}|\\
|\input{childdoc.def}|\\
|\childdocforwardprefix{final}{child}|
\end{tabular}
\end{center}
%

Note that when several versions of a main file and/or of each child file
are to be generated, it may be convenient to set up a |Makefile| or
shell script to automatise the process.

%%%%%%%%%%%%%%%%%%%%%%%%%%%%%%%%%%%%%%%%%%%%%%%%%%%%%%%%%%%%%%%%%%%%%%%%%%%%%%%%
\subsection{Command Line Processing}
\label{sec:commandline}

The effect of redirection files can also be achieved by invoking
the \LaTeX{} compiler with a more elaborate command line.
Most conveniently this should be done as part
of a shell script or a |Makefile|.

When using \textsf{childdoc} in the main file, the following
command lines effectively perform a redirection
(note that depending on the shell being used,
backslashes may have to be doubled: `|\|' $\to$ `|\\|'):
%
\begin{center}
|... -jobname "|\textit{target}|" |\\|"|[\textit{flags}]%
|\input{childdoc.def}\childdocforward[|\textit{main}|]{|\textit{dest}|}"|
\end{center}
%
Here \textit{target} is the name of the output file,
\textit{main} is the name of the main file
and \textit{dest} is the name of the main or child file to be processed
(all filenames without extensions).
The optional argument \textit{main} can be omitted
if \textit{main} matches \textit{dest}.
Optionally, compilation \textit{flags} can be defined via |\def| commands.
This command line makes the \TeX{} engine believe
it is compiling the file \textit{target}
whose content is specified as the latter parameter.
The provided code then forwards the processing to
\textit{main} or \textit{dest} as described in \secref{sec:forward}.

%%%%%%%%%%%%%%%%%%%%%%%%%%%%%%%%%%%%%%%%%%%%%%%%%%%%%%%%%%%%%%%%%%%%%%%%%%%%%%%%
\subsection{Include by Input}
\label{sec:input}

Including child documents by |\include| has some restrictions by design.
Most notably, the content of a child document always occupies
its own set of pages; pages cannot be shared between child documents.
Usually, this behaviour makes perfect sense
because each child document contain an essential part of the document.
However, in some situations it may be desirable to compose
a document from a collection of parts
without having mandatory page breaks between then.
For this case, the package
provides a mechanism to include parts
by |\input| which can also be processed individually.
However, by construction this mechanism
requires manual handling of the content to be output.

%%%%%%%%%%%%%%%%%%%%%%%%%%%%%%%%%%%%%%%%
\DescribeMacro{\ifchilddocmanual}
The main file should be prepared as usual, see \secref{sec:include}.
However, the document body must make a distinction
between processing of an individual part and of the main document, e.g.:
%
\begin{center}
\begin{tabular}{l}
|\ifchilddocmanual|\\
|\input{\childdocname}|\\
|\||else|\\
\textit{document body with }|\input{|\textit{part}|}|\\
|\||fi|
\end{tabular}
\end{center}
%
The conditional |\ifchilddocmanual| is true whenever
a part to be included by |\input| is being compiled,
and the name of the part is stored in |\childdocname|.

%%%%%%%%%%%%%%%%%%%%%%%%%%%%%%%%%%%%%%%%
\DescribeMacro{\childdocby}
Each part to be included by |\input| should start with:
%
\begin{center}
\begin{tabular}{l}
|\input{childdoc.def}|\\
|\childdocby{|\textit{main}|}|\\
\end{tabular}
\end{center}
%
The directive |\childdocby| is similar to |\childdocof|
described in \secref{sec:include},
but the subsequent selection of content must be done manually.
To that end, both |\ifchilddoc| and |\ifchilddocmanual|
will be true upon processing of a part,
and the name of the part is stored in |\childdocname|.
Note that |\jobname| will be set to the filename of the current part
so that each part receives an individual |.aux| file
that does not interfere with the |.aux| file(s) of the main document.
This behaviour can be altered by the alternative form
|\childdocby[*]{|\textit{main}|}| (with a non-empty optional argument)
which uses the |.aux| file of the main document
by setting |\jobname| to \textit{main}.

%%%%%%%%%%%%%%%%%%%%%%%%%%%%%%%%%%%%%%%%%%%%%%%%%%%%%%%%%%%%%%%%%%%%%%%%%%%%%%%%
\subsection{Driver Development}
\label{sec:driver}

The \textsf{childdoc} mechanism can also be use for the development
of definition files such as \LaTeX{} styles or classes.
This case differs from the above setup with multiple parts
included by |\include| in that no |\includeonly| should be invoked.
This can be achieved by starting the include file
(before |\ProvidesPackage|) with:
%
\begin{center}
\begin{tabular}{l}
|\input{childdoc.def}|\\
|\childdocforward{|\textit{main}|}|\\
\end{tabular}
\end{center}
%
or alternatively with:
%
\begin{center}
\begin{tabular}{l}
|\input{childdoc.def}|\\
|\childdocby{|\textit{main}|}|\\
\end{tabular}
\end{center}
%
Both forms have slightly different effects as described above.
The main file is prepared as usual, see \secref{sec:include}.

%%%%%%%%%%%%%%%%%%%%%%%%%%%%%%%%%%%%%%%%%%%%%%%%%%%%%%%%%%%%%%%%%%%%%%%%%%%%%%%%
\subsection{Legacy Detection}
\label{sec:detection}

The directive |\childdocmain| in the main file can detect
whether the complete document or merely a child is to be compiled
even without using the directive |\childdocof|.
This method is deprecated because it is less robust
and there is no compelling reason to use it;
it is merely provided for backward compatibility
and it may be removed in future versions.

If the detection mechanism is to be used,
it is mandatory to correctly specify
the filename of the main file as the argument of |\childdocmain|:
%
\begin{center}
\begin{tabular}{l}
|\input{childdoc.def}|\\
|\childdocmain{|\textit{main}|}|\\
\end{tabular}
\end{center}
%
If |\jobname| does not match the argument \textit{main} of |\childdocmain|,
it is assumed that |\jobname| points to the child file to be compiled.
When using |\childdocmain| with the main file specified as argument,
it suffices to start a child file
with just |\input{|\textit{main}|}|
without loading of the package and using |\childdocof|.
If instead all processing is done
with the appropriate \textsf{childdoc} directives,
the argument of \textit{main} of |\childdocmain| can be empty.

An alternative version of the command line processing described
in \secref{sec:commandline} using the detection mechanism reads:
%
\begin{center}
|... -jobname "|\textit{target}|" "|[\textit{flags}]%
[|\def\jobname{|\textit{dest}|}|]|\input{|\textit{main}|}"|
\end{center}

%%%%%%%%%%%%%%%%%%%%%%%%%%%%%%%%%%%%%%%%%%%%%%%%%%%%%%%%%%%%%%%%%%%%%%%%%%%%%%%%
\subsection{Manual Code}
\label{sec:manual}

In case one cannot be certain whether the definitions file |childdoc.def|
is installed on the target \TeX{} distribution
and one prefers not to ship it,
it is conceivable to paste a few relevant commands into the sources.

To that end, drop all statements |\input{childdoc.def}|
and perform the replacements as outlined below.
Instead of |\childdocmain{|\textit{main}|}| add the following code
to the top of the main file:
%
\begin{center}
\begin{tabular}{l}
|\||ifdefined\childdocname\endinput\||fi\newif\ifchilddoc|\\
|\edef\childdocname{\scantokens\expandafter{\jobname\noexpand}}|\\
|\def\childdocmain{|\textit{main}|}\||ifx\childdocmain\childdocname\||else|\\
|\childdoctrue\includeonly{\childdocname}\let\jobname\childdocmain\||fi|\\
\end{tabular}
\end{center}
%
Instead of |\childdocof{|\textit{main}|}| just include the main file
at the top of each child file:
%
\begin{center}
|\input{|\textit{main}|}|
\end{center}
%
A simple redirection |\childdocforward{|\textit{dest}|}| is achieved by:
%
\begin{center}
|\def\jobname{|\textit{dest}|}\input{\jobname}|
\end{center}
%
The redirection with prefix
|\childdocforwardprefix[|\textit{prefix}|]{|\textit{dest}|}|
is accomplished by:
%
\begin{center}
\begin{tabular}{l}
|{\edef\jobname{\scantokens\expandafter{\jobname\noexpand}}|\\
|\def\redirectjob |\textit{prefix}|#1~~~{\gdef\jobname{|\textit{dest}|#1}}|\\
|\expandafter\redirectjob\jobname~~~}\input{\jobname}|
\end{tabular}
\end{center}

In an alternative approach,
child documents can be compiled by a specific command line
without additional code or specific definitions:
%
\begin{center}
|... -jobname "|\textit{target}|" "|[\textit{flags}]%
|\includeonly{|\textit{dest}|}\input{|\textit{main}|}"|
\end{center}
%

%%%%%%%%%%%%%%%%%%%%%%%%%%%%%%%%%%%%%%%%%%%%%%%%%%%%%%%%%%%%%%%%%%%%%%%%%%%%%%%%
%%%%%%%%%%%%%%%%%%%%%%%%%%%%%%%%%%%%%%%%%%%%%%%%%%%%%%%%%%%%%%%%%%%%%%%%%%%%%%%%
\section{Information}

%%%%%%%%%%%%%%%%%%%%%%%%%%%%%%%%%%%%%%%%%%%%%%%%%%%%%%%%%%%%%%%%%%%%%%%%%%%%%%%%
\subsection{Copyright}

Copyright \copyright{} 2017--2018 Niklas Beisert

This work may be distributed and/or modified under the
conditions of the \LaTeX{} Project Public License, either version 1.3
of this license or (at your option) any later version.
The latest version of this license is in
  \url{http://www.latex-project.org/lppl.txt}
and version 1.3 or later is part of all distributions of \LaTeX{}
version 2005/12/01 or later.

This work has the LPPL maintenance status `maintained'.

The Current Maintainer of this work is Niklas Beisert.

This work consists of the files |README.txt|, |childdoc.ins| and |childdoc.dtx|
as well as the derived files |childdoc.def|, |cdocsamp.tex|
with |cdocsch1.tex|, |cdocsch2.tex|, |cdocspt3.tex|, |cdocspt4.tex|,
|cdocsdrf.tex|, |cdocsfn1.tex|, |cdocsfn2.tex|
as well as |childdoc.pdf|.

%%%%%%%%%%%%%%%%%%%%%%%%%%%%%%%%%%%%%%%%%%%%%%%%%%%%%%%%%%%%%%%%%%%%%%%%%%%%%%%%
\subsection{Files and Installation}

The package consists of the files:
%
\begin{center}
\begin{tabular}{ll}
    |README.txt|   & readme file \\
    |childdoc.ins| & installation file \\
    |childdoc.dtx| & source file \\
    |childdoc.def| & definition file \\
    |cdocsamp.tex| & sample main file \\
    |cdocsch1.tex| & sample include file \\
    |cdocsch2.tex| & sample include file \\
    |cdocspt3.tex| & sample part file \\
    |cdocspt4.tex| & sample part file \\
    |cdocsdrf.tex| & sample redirection file \\
    |cdocsfn1.tex| & sample redirection file \\
    |cdocsfn2.tex| & sample redirection file \\
    |childdoc.pdf| & manual
\end{tabular}
\end{center}
%
The distribution consists of the files
|README.txt|, |childdoc.ins| and |childdoc.dtx|.
%
\begin{itemize}
\item
Run (pdf)\LaTeX{} on |childdoc.dtx|
to compile the manual |childdoc.pdf| (this file).
\item
Run \LaTeX{} on |childdoc.ins| to create the definitions file |childdoc.def|
and the sample |cdocsamp.tex| with include files
|cdocsch1.tex|, |cdocsch2.tex|, |cdocspt3.tex|, |cdocspt4.tex|,
|cdocsdrf.tex|, |cdocsfn1.tex|, |cdocsfn2.tex|.
Then copy the file |childdoc.def| to an appropriate directory of your \LaTeX{}
distribution, e.g.\ \textit{texmf-root}|/tex/latex/childdoc|.
\end{itemize}

%%%%%%%%%%%%%%%%%%%%%%%%%%%%%%%%%%%%%%%%%%%%%%%%%%%%%%%%%%%%%%%%%%%%%%%%%%%%%%%%
\subsection{Related CTAN Packages}

There are several other packages which offer a similar functionality:
%
\begin{itemize}
\item
The packages
\href{http://ctan.org/pkg/docmute}{\textsf{docmute}},
\href{http://ctan.org/pkg/includex}{\textsf{includex}} and
\href{http://ctan.org/pkg/standalone}{\textsf{standalone}}
provide commands to include only the document body of
a child file thus allowing both files to be compiled individually.
\item
The packages \href{http://ctan.org/pkg/subdocs}{\textsf{subdocs}}
and \href{http://ctan.org/pkg/subfiles}{\textsf{subfiles}}
provide structures in which the main and child documents can be
encapsulated and allowing them to be compiled individually.
The inclusion mechanism is different from the conventional |\include|.
\item
The package \href{http://ctan.org/pkg/combine}{\textsf{combine}}
is an elaborate solution to combine several documents into one.
\end{itemize}
%
See also the CTAN topic \href{http://ctan.org/topic/subdocs}{\textsf{subdocs}}
for further related packages.
The present package differs from the above solutions in that
a document structure constructed with the conventional |\include| mechanism
just needs two extra commands at the top of every file
such that all constituent files can be compiled individually.

%%%%%%%%%%%%%%%%%%%%%%%%%%%%%%%%%%%%%%%%%%%%%%%%%%%%%%%%%%%%%%%%%%%%%%%%%%%%%%%%
%\subsection{Feature Suggestions}
%
%The following is a list of features which may be useful for future
%versions of this package:
%%
%\begin{itemize}
%\item
%\ldots
%\end{itemize}

%%%%%%%%%%%%%%%%%%%%%%%%%%%%%%%%%%%%%%%%%%%%%%%%%%%%%%%%%%%%%%%%%%%%%%%%%%%%%%%%
\subsection{Revision History}

%%%%%%%%%%%%%%%%%%%%%%%%%%%%%%%%%%%%%%%%
\paragraph{v2.0:} 2018/12/30

\begin{itemize}
\item
immediate forward processing
\item
added |\childdocby| mechanism
\item
manual restructured
\end{itemize}

%%%%%%%%%%%%%%%%%%%%%%%%%%%%%%%%%%%%%%%%
\paragraph{v1.6:} 2018/01/17

\begin{itemize}
\item
application for development of include files
\item
corrections to manual
\end{itemize}

%%%%%%%%%%%%%%%%%%%%%%%%%%%%%%%%%%%%%%%%
\paragraph{v1.5:} 2017/05/21

\begin{itemize}
\item
more complete structuring introduced
\item
|\childdocof| introduced
\item
|\childdoc| renamed to |\childdocmain|
\item
|\childredirect| renamed to |\childdocforward| and |\childdocforwardprefix|
and functionality expanded
\end{itemize}

%%%%%%%%%%%%%%%%%%%%%%%%%%%%%%%%%%%%%%%%
\paragraph{v1.0:} 2017/04/27

\begin{itemize}
\item
manual and install package
\item
first version published on CTAN
\end{itemize}

%%%%%%%%%%%%%%%%%%%%%%%%%%%%%%%%%%%%%%%%
\paragraph{v0.6:} 2017/04/26

\begin{itemize}
\item
redirection mechanism added
\end{itemize}

%%%%%%%%%%%%%%%%%%%%%%%%%%%%%%%%%%%%%%%%
\paragraph{v0.5:} 2017/04/26

\begin{itemize}
\item
functionality in definition file
\end{itemize}


%%%%%%%%%%%%%%%%%%%%%%%%%%%%%%%%%%%%%%%%%%%%%%%%%%%%%%%%%%%%%%%%%%%%%%%%%%%%%%%%
%%%%%%%%%%%%%%%%%%%%%%%%%%%%%%%%%%%%%%%%%%%%%%%%%%%%%%%%%%%%%%%%%%%%%%%%%%%%%%%%
%%%%%%%%%%%%%%%%%%%%%%%%%%%%%%%%%%%%%%%%%%%%%%%%%%%%%%%%%%%%%%%%%%%%%%%%%%%%%%%%
\appendix

\settowidth\MacroIndent{\rmfamily\scriptsize 000\ }

 \DocInput{childdoc.dtx}

\end{document}
%</driver>
% \fi
%
% %%%%%%%%%%%%%%%%%%%%%%%%%%%%%%%%%%%%%%%%%%%%%%%%%%%%%%%%%%%%%%%%%%%%%%%%%%%%%%
% %%%%%%%%%%%%%%%%%%%%%%%%%%%%%%%%%%%%%%%%%%%%%%%%%%%%%%%%%%%%%%%%%%%%%%%%%%%%%%
% \section{Sample}
%\iffalse
%<*samplemain>
%\fi
%
% The following presents a sample document
% with two chapters, two parts, a title page,
% a compile flag as well as three forwarding files to set the flag.
% It consists of eight |.tex| files:
% \begin{center}
% \begin{tabular}{ll}
% |cdocsamp.tex|&main file\\
% |cdocsch1.tex|&include file for chapter 1\\
% |cdocsch2.tex|&include file for chapter 2\\
% |cdocspt3.tex|&include file for part 3\\
% |cdocspt4.tex|&include file for part 4\\
% |cdocsdrf.tex|&forwarding file for main file in draft mode\\
% |cdocsfi1.tex|&forwarding file for final version of chapter 1\\
% |cdocsfi2.tex|&forwarding file for final version of chapter 2\\
% \end{tabular}
% \end{center}
% Each of the eight files can be compiled directly by the \LaTeX{} compiler.
%
% %%%%%%%%%%%%%%%%%%%%%%%%%%%%%%%%%%%%%%
% \paragraph{Main File.}
%
% The main file is called |cdocsamp.tex|.
%
% Load the \textsf{childdoc} definitions and
% declare the filename for the main document:
%    \begin{macrocode}
\input{childdoc.def}
\childdocmain{}
%    \end{macrocode}

% Optional override for |\version| flag:
%    \begin{macrocode}
%%\ifchilddoc\else\providecommand{\version}{draft}\fi
%    \end{macrocode}

% Define the default values for the |\version| flag
% (|final| for the main file and |draft| for childs):
%    \begin{macrocode}
\ifchilddoc
\providecommand{\version}{draft}
\else
\providecommand{\version}{final}
\fi
%    \end{macrocode}

% Load the standard document class:
%    \begin{macrocode}
\documentclass[12pt]{article}
%    \end{macrocode}

% Start the document body:
%    \begin{macrocode}
\begin{document}
%    \end{macrocode}

% Declare a title page.
% Print title, part of document being processed and version flag:
%    \begin{macrocode}
\addtocounter{page}{-1}
\begin{center}
{\LARGE\bfseries{}childdoc example\par}
\vspace{1cm}
\ifchilddoc
\ifchilddocmanual part\else chapter\fi:
`\childdocname' of `\childdocjob'\par
\else
main document: `\childdocjob'\par
\fi
version: \version\par
\end{center}
\newpage
%    \end{macrocode}

% Manually include selected file,
% otherwise process as usual:
%    \begin{macrocode}
\ifchilddocmanual
\section*{part `\childdocname'}
\input{\childdocname}
\else
%    \end{macrocode}

% Include the two chapters:
%    \begin{macrocode}
\include{cdocsch1}
\include{cdocsch2}
%    \end{macrocode}

% Include the two parts unless only chapters should be displayed:
%    \begin{macrocode}
\ifchilddoc\else
\section{part three}
\input{cdocspt3}
\section{part four}
\input{cdocspt4}
\fi
%    \end{macrocode}

% Process as usual until here:
%    \begin{macrocode}
\fi
%    \end{macrocode}

% End of document body:
%    \begin{macrocode}
\end{document}
%    \end{macrocode}
%\iffalse
%</samplemain>
%\fi
%
% %%%%%%%%%%%%%%%%%%%%%%%%%%%%%%%%%%%%%%
% \paragraph{Chapter Include Files.}
%
% The include files are called |cdocsch1.tex| and |cdocsch2.tex|.
%
%\iffalse
%<*samplechap1|samplechap2>
%\fi

% Optional override for |\version| flag:
%    \begin{macrocode}
%%\providecommand{\version}{final}
%    \end{macrocode}

% Include the main document:
%    \begin{macrocode}
\input{childdoc.def}
\childdocof{cdocsamp}
%    \end{macrocode}

%\iffalse
%</samplechap1|samplechap2>
%\fi
%
%\iffalse
%<*samplechap1>
%\fi
% Some text for chapter 1:
%    \begin{macrocode}
\section{one}
some text in chapter one
%    \end{macrocode}

%\iffalse
%</samplechap1>
%\fi
% Some text for chapter 2:
%\iffalse
%<*samplechap2>
%\fi
%    \begin{macrocode}
\section{two}
more text in chapter two
%    \end{macrocode}

%\iffalse
%</samplechap2>
%\fi
%
% %%%%%%%%%%%%%%%%%%%%%%%%%%%%%%%%%%%%%%
% \paragraph{Part Include Files.}
%
% The include files are called |cdocspt3.tex| and |cdocspt4.tex|.
%
%\iffalse
%<*samplepart3|samplepart4>
%\fi

% Optional override for |\version| flag:
%    \begin{macrocode}
%%\providecommand{\version}{final}
%    \end{macrocode}

% Include the main document:
%    \begin{macrocode}
\input{childdoc.def}
\childdocby{cdocsamp}
%    \end{macrocode}

%\iffalse
%</samplepart3|samplepart4>
%\fi
%
%\iffalse
%<*samplepart3>
%\fi
% Some text for part 3:
%    \begin{macrocode}
some text in part three
%    \end{macrocode}

%\iffalse
%</samplepart3>
%\fi
% Some text for part 4:
%\iffalse
%<*samplepart4>
%\fi
%    \begin{macrocode}
more text in part four
%    \end{macrocode}

%\iffalse
%</samplepart4>
%\fi
%
% %%%%%%%%%%%%%%%%%%%%%%%%%%%%%%%%%%%%%%
% \paragraph{Forwarding for a Complete Draft.}
%
% The following forwarding file |cdocsdrf.tex|
% compiles the main document in draft mode:
%\iffalse
%<*sampledraft>
%\fi
%    \begin{macrocode}
\def\version{draft}
\input{childdoc.def}
\childdocforward{cdocsamp}
%    \end{macrocode}

%\iffalse
%</sampledraft>
%\fi
%
% %%%%%%%%%%%%%%%%%%%%%%%%%%%%%%%%%%%%%%
% \paragraph{Forwarding for Final Version of the Chapters.}
%
% The following forwarding files |cdocsfn1.tex| and |cdocsfn2.tex|
% (with identical content)
% compile the final versions of the child documents
% |cdocsch1.tex| and |cdocsch2.tex|, respectively:
%\iffalse
%<*samplefinal>
%\fi
%    \begin{macrocode}
\def\version{final}
\input{childdoc.def}
\childdocforwardprefix[cdocsamp]{cdocsfn}{cdocsch}
%    \end{macrocode}

%\iffalse
%</samplefinal>
%\fi
%
% %%%%%%%%%%%%%%%%%%%%%%%%%%%%%%%%%%%%%%
% \paragraph{Command Line Processing.}
%
% The following three command lines generate the output files
% |cdocscld|, |cdocscl1| and |cdocscl2|
% which should be identical to
% |cdocsdrf|, |cdocsch1| and |cdocsfn2|, respectively:
% \begin{center}
% \begin{tabular}{l}
% |latex -jobname cdocscld \|\\
% |  "\def\version{draft}\input{childdoc.def}\childdocforward{cdocsamp}"|\\
% |latex -jobname cdocscl1 \|\\
% |  "\input{childdoc.def}\childdocforward[cdocsamp]{cdocsch1}"|\\
% |latex -jobname cdocscl2 \|\\
% |  "\def\version{final}\input{childdoc.def}\childdocforward{cdocsch2}"|
% \end{tabular}
% \end{center}
% Note that the trailing backslash on each first line
% merely continues the input to the second line
% (for convenient cut ant paste).
% Furthermore, the command |latex| can be replaced by any
% of its alternative versions such as |pdflatex|.
%
% %%%%%%%%%%%%%%%%%%%%%%%%%%%%%%%%%%%%%%%%%%%%%%%%%%%%%%%%%%%%%%%%%%%%%%%%%%%%%%
% %%%%%%%%%%%%%%%%%%%%%%%%%%%%%%%%%%%%%%%%%%%%%%%%%%%%%%%%%%%%%%%%%%%%%%%%%%%%%%
% \section{Implementation}
%\iffalse
%<*package>
%\fi
%
% This section describes the definitions file |childdoc.def|.

% The definitions cannot be loaded using |\usepackage| or |\RequirePackage|
% which has a mechanism to prevent loading a style file more than once.
% When loading the definitions by means of |\input|
% multiple instances have to be prevented manually:
%\iffalse
%This code needs to be before the `\ProvidesFile' directive
%which is defined at the beginning of this file.
%Therefore it is also placed there and commented out here.
%</package>
%<*discard>
%\fi
%    \begin{macrocode}
\ifdefined\childdocmain\endinput\fi
%    \end{macrocode}
%\iffalse
%</discard>
%<*package>
%\fi
%
% \macro{\ifchilddoc}
% \macro{\ifchilddocmanual}
% The conditional |\ifchilddoc| tells whether a
% child (true) or main (false) document is being compiled.
% The conditional |\ifchilddocmanual| tells whether
% the |\includeonly| mechanism is used (false) or
% the selection of child files must be performed manually (true).
% The definitions initialise to false:
%    \begin{macrocode}
\newif\ifchilddoc
\newif\ifchilddocmanual
%    \end{macrocode}

% \macro{\childdocname}
% \macro{\childdocjob}
% The macro |\childdocname| stores the name of the main document
% to be compiled. The macro |\childdocjob| stores the name of
% the document on which the \LaTeX{} compiler was originally invoked.
% The content of |\jobname| cannot be compared
% to filenames specified in the source due to different catcodes.
% The following code rescans |\jobname|, stores the result
% in |\childdocname| and saves a copy in |\childdocjob|:
%    \begin{macrocode}
\edef\childdocname{\scantokens\expandafter{\jobname\noexpand}}
\let\childdocjob\childdocname
%    \end{macrocode}

% \macro{\childdocdisable}
% The macro |\childdocdisable| prevents the main file
% from being processed more than once.
% At this stage, the main document command |\childdocmain|
% is assumed to be called once again where it should do nothing.
% Any subsequent call to it should prevent
% a secondary processing of the main document
% It overwrites the forwarding commands
% |\childdocof| and |\childdocforward|
% with empty macros to prevent further inclusions of the main document:
%    \begin{macrocode}
\newcommand{\childdocdisable}
{
  \renewcommand{\childdocmain}[1]{\renewcommand{\childdocmain}[1]{\endinput}}
  \renewcommand{\childdocof}[1]{}
  \renewcommand{\childdocby}[2][]{}
  \renewcommand{\childdocforward}[2][]{}
  \renewcommand{\childdocdisable}{}
}
%    \end{macrocode}

% \macro{\childdocmain}
% The macro |\childdocmain| is to be called at the top of the main file
% with nothing or the main filename (without extension) as argument.
% First, it breaks loops.
% If the argument is not empty and does not match |\childdocname|
% (which is set by the first inclusion of |childdoc.def|),
% |\ifchilddoc| is set to true, |\includeonly| is applied to the child file
% and |\jobname| is set to the main file
% (for proper handling of |.aux| files):
%    \begin{macrocode}
\newcommand{\childdocmain}[1]
{
  \childdocdisable\childdocmain{}
  \if?#1?\else
    \begingroup
      \def\childdoctmp{#1}
      \ifx\childdoctmp\childdocname
        \def\childdoctmp{}
      \else
        \def\childdoctmp
        {
          \childdoctrue
          \includeonly{\childdocname}
          \def\childdocjob{#1}
          \def\jobname{#1}
        }
      \fi
      \expandafter
    \endgroup
    \childdoctmp
  \fi
}
%    \end{macrocode}

% \macro{\childdocof}
% The command |\childdocof| redirects
% compilation to the main file |#1|.
%    \begin{macrocode}
\newcommand{\childdocof}[1]
{
  \childdocdisable
  \childdoctrue
  \includeonly{\childdocname}
  \def\jobname{#1}
  \def\childdocjob{#1}
  \input{#1}
}
%    \end{macrocode}

% \macro{\childdocby}
% The command |\childdocby| ....
%    \begin{macrocode}
\newcommand{\childdocby}[2][]
{
  \childdocdisable
  \childdoctrue
  \childdocmanualtrue
  \if?#1?\else
    \def\jobname{#2}
  \fi
  \def\childdocjob{#2}
  \input{#2}
  \endinput
}
%    \end{macrocode}

% \macro{\childdocforward}
% The command |\childdocforward| redirects
% compilation to the main file or
% (if the optional argument is given) a child file.
% Parameters are set as if the main file
% or a child file starting with |\childdocof| was compiled.
% Then compilation is handed over to the main file:
%    \begin{macrocode}
\newcommand{\childdocforward}[2][]
{
  \begingroup
    \if?#1?
      \def\childdoctmp
      {
        \def\childdocname{#2}
        \def\childdocjob{#2}
        \def\jobname{#2}
        \input{#2}
        \endinput
      }
    \else
      \def\childdoctmp
      {
        \childdocdisable
        \def\childdocname{#2}
        \childdoctrue
        \includeonly{#2}
        \def\childdocjob{#1}
        \def\jobname{#1}
        \input{#1}
        \endinput
      }
    \fi
    \expandafter
  \endgroup
  \childdoctmp
}
%    \end{macrocode}

% \macro{\childdocforwardprefix}
% The command |\childdocforwardprefix| redirects
% compilation to the main or a child file by means of a pattern.
% The prefix |#1| in the current filename is replaced by |#2|
% and the suffix of the current filename is kept
% (it is assumed that the filename does not contain the substring `|~~~|'
% which is used as a delimiter).
% Compilation is handed over to the new file by |\childdocforward|:
%    \begin{macrocode}
\newcommand{\childdocforwardprefix}[3][]
{
  \begingroup
    \def\childdocextract #2##1~~~{\def\childdoctmp{\childdocforward[#1]{#3##1}}}
    \expandafter\childdocextract\childdocname~~~
    \expandafter
  \endgroup
  \childdoctmp
}
%    \end{macrocode}

% \macro{\childdoc}
% The deprecated macro |\childdoc| is a legacy version of |\childdocmain|:
%    \begin{macrocode}
\newcommand{\childdoc}{\childdocmain}
%    \end{macrocode}

% \macro{\childdocredirect}
% The deprecated macro |\childdocredirect| is a legacy version
% of |\childdocforward| and |\childdocforwardprefix|:
%    \begin{macrocode}
\newcommand{\childdocredirect}[2][]
{
  \begingroup
    \if?#1?
      \def\childdoctmp{\childdocforward{#2}}
    \else
      \def\childdoctmp{\childdocforwardprefix{#1}{#2}}
    \fi
    \expandafter
  \endgroup
  \childdoctmp
}
%    \end{macrocode}

%\iffalse
%</package>
%\fi
%
\endinput
|\\
|\childdocforward[|\textit{main}|]{|\textit{dest}|}|\\
\end{tabular}
\end{center}
%
The argument \textit{dest} is the destination file
(without extension).
It should be the main file or one of the child files.
Note that further \textsf{childdoc} directives
such as |\childdocof| and |\childdocforward|
in the indicated file will be processed in this form.
The optional argument \textit{main}
passes on directly to the main file \textit{main}
while pretending to compile the child \textit{dest}.
This form behaves as if \textit{dest}
issues |\childdocof{|\textit{main}|}| right away,
and no further \textsf{childdoc} directives will be processed.

%%%%%%%%%%%%%%%%%%%%%%%%%%%%%%%%%%%%%%%%
\DescribeMacro{\...prefix}
In the alternative form |\childdocforwardprefix|,
%
\begin{center}
\begin{tabular}{l}
|% \iffalse
%
% childdoc.dtx Copyright (C) 2017-2018 Niklas Beisert
%
% This work may be distributed and/or modified under the
% conditions of the LaTeX Project Public License, either version 1.3
% of this license or (at your option) any later version.
% The latest version of this license is in
%   http://www.latex-project.org/lppl.txt
% and version 1.3 or later is part of all distributions of LaTeX
% version 2005/12/01 or later.
%
% This work has the LPPL maintenance status `maintained'.
%
% The Current Maintainer of this work is Niklas Beisert.
%
% This work consists of the files childdoc.dtx and childdoc.ins
% and the derived files childdoc.def and cdocsamp.tex with
% cdocsch1.tex, cdocsch2.tex, cdocsdrf.tex, cdocsfn1.tex, cdocsfn2.tex.
%
%<package>\ifdefined\childdocmain\endinput\fi
%<package>\ProvidesFile{childdoc.def}[2018/12/30 v2.0 child document driver]
%<samplemain>\ProvidesFile{cdocsamp.tex}[2018/12/30 v2.0 sample for childdoc]
%<*driver>
%\ProvidesFile{childdoc.drv}[2018/12/30 v2.0 childdoc reference manual file]
\PassOptionsToClass{10pt,a4paper}{article}
\documentclass{ltxdoc}

\usepackage[margin=35mm]{geometry}
\usepackage{hyperref}
\usepackage{hyperxmp}
\usepackage[usenames]{color}

\hypersetup{colorlinks=true}
\hypersetup{pdfstartview=FitH}
\hypersetup{pdfpagemode=UseNone}
\hypersetup{pdfsource={}}
\hypersetup{pdflang={en-UK}}
\hypersetup{pdfcopyright={Copyright 2017-2018 Niklas Beisert.
  This work may be distributed and/or modified under the
  conditions of the LaTeX Project Public License, either version 1.3
  of this license or (at your option) any later version.}}
\hypersetup{pdflicenseurl={http://www.latex-project.org/lppl.txt}}
\hypersetup{pdfcontactaddress={ETH Zurich, ITP, HIT K,
  Wolfgang-Pauli-Strasse 27}}
\hypersetup{pdfcontactpostcode={8093}}
\hypersetup{pdfcontactcity={Zurich}}
\hypersetup{pdfcontactcountry={Switzerland}}
\hypersetup{pdfcontactemail={nbeisert@itp.phys.ethz.ch}}
\hypersetup{pdfcontacturl={http://people.phys.ethz.ch/\xmptilde nbeisert/}}

\newcommand{\secref}[1]{\hyperref[#1]{section \ref*{#1}}}

\parskip1ex
\parindent0pt
\let\olditemize\itemize
\def\itemize{\olditemize\parskip0pt}

\begin{document}

\title{The \textsf{childdoc} Package}
\hypersetup{pdftitle={The childdoc Package}}
\author{Niklas Beisert\\[2ex]
  Institut f\"ur Theoretische Physik\\
  Eidgen\"ossische Technische Hochschule Z\"urich\\
  Wolfgang-Pauli-Strasse 27, 8093 Z\"urich, Switzerland\\[1ex]
  \href{mailto:nbeisert@itp.phys.ethz.ch}
  {\texttt{nbeisert@itp.phys.ethz.ch}}}
\hypersetup{pdfauthor={Niklas Beisert}}
\hypersetup{pdfsubject={Manual for the LaTeX2e Package childdoc}}
\date{30 December 2018, \textsf{v2.0}}
\maketitle

\begin{abstract}\noindent
\textsf{childdoc} is a \LaTeXe{} package
that enables the direct compilation
of document sections included by |\include|
to individual files.
\end{abstract}

\begingroup
\parskip0ex
\tableofcontents
\endgroup

%%%%%%%%%%%%%%%%%%%%%%%%%%%%%%%%%%%%%%%%%%%%%%%%%%%%%%%%%%%%%%%%%%%%%%%%%%%%%%%%
%%%%%%%%%%%%%%%%%%%%%%%%%%%%%%%%%%%%%%%%%%%%%%%%%%%%%%%%%%%%%%%%%%%%%%%%%%%%%%%%
\section{Introduction}

\LaTeX{} provides a mechanism to structure a large document (such as a book)
into a main file and several child files (containing the chapters)
using the |\include| command.
This mechanism is beneficial for documents
which span hundreds of pages in order to
make the source file(s) more manageable.
Moreover, compilation can be restricted to
selected child files by means of the |\includeonly| command.
The latter feature can be used to reduce the compilation time while editing
(this was significantly more useful in the earlier days of \LaTeX{})
or to generate a smaller document which is easier to navigate.
Another application of |\includeonly| is to generate
documents consisting of selected parts of the complete document.

However, there are a few drawbacks of the plain |\include| mechanism:
\begin{itemize}
\item
The child files cannot be compiled on their own,
they can only be compiled via the main file.
A naive editing environment
(such as a text editor with an option
to have the current file processed by \LaTeX)
may require one to switch to the main file before compiling;
attempting to compile the child file produces errors.
\item
The main file must be modified (each time)
to adjust the |\includeonly| command
to the present needs. This easily leaves the main file in a messy state.
\item
The generated document will always carry the filename
of the main document. This is inconvenient if
several child files are to be compiled and
to be kept for distribution.
\end{itemize}

The present package provides a simple interface
to make child files individually compilable by \LaTeX{}.
Compiling a child file then has the same effect as compiling
the main file with an |\includeonly| command
to select the appropriate child.
Moreover the generated document will carry the name of the child
rather than the main file.
This resolves all three above issues.

This feature is meant to make the editing of books,
thesis documents and lecture notes somewhat more convenient.
However, the package can also be used efficiently for
composing a series of documents (such as exercise sheets)
which are typically distributed individually.
It then assists the author in generating the individual documents
(potentially in different versions)
as well as a document containing the collected series.
Another application is in developing style files
or other kinds of included material
where compilation of the style file could redirect
to a sample or test file.

%%%%%%%%%%%%%%%%%%%%%%%%%%%%%%%%%%%%%%%%%%%%%%%%%%%%%%%%%%%%%%%%%%%%%%%%%%%%%%%%
%%%%%%%%%%%%%%%%%%%%%%%%%%%%%%%%%%%%%%%%%%%%%%%%%%%%%%%%%%%%%%%%%%%%%%%%%%%%%%%%
\section{Usage}

First of all, the package \textsf{childdoc} is \emph{not} a standard
\LaTeXe{} |.sty| style file! Therefore it needs to be invoked in
a non-standard way.

%%%%%%%%%%%%%%%%%%%%%%%%%%%%%%%%%%%%%%%%%%%%%%%%%%%%%%%%%%%%%%%%%%%%%%%%%%%%%%%%
\subsection{Included Files}
\label{sec:include}

%%%%%%%%%%%%%%%%%%%%%%%%%%%%%%%%%%%%%%%%
\DescribeMacro{\childdocmain}
To use the package, add the commands
\begin{center}
\begin{tabular}{l}
|\input{childdoc.def}|\\
|\childdocmain{}|\\
\end{tabular}
\end{center}
at the very top of the main \LaTeX{} file,
in particular \emph{before} the |\documentclass| statement!
The argument of |\childdocmain| should be left empty
(but it must be present).

%%%%%%%%%%%%%%%%%%%%%%%%%%%%%%%%%%%%%%%%
\DescribeMacro{\childdocof}
Furthermore, add the commands
\begin{center}
\begin{tabular}{l}
|\input{childdoc.def}|\\
|\childdocof{|\textit{main}|}|\\
\end{tabular}
\end{center}
at the top of every child file \textit{child}
which is included by |\include{|\textit{child}|}|
from within the main file
(or at least for those files to be compiled individually).
The argument \textit{main} must be the filename of the main file.

There are a couple of
considerations in setting up the main and child documents:

%%%%%%%%%%%%%%%%%%%%%%%%%%%%%%%%%%%%%%%%
\paragraph{Restrictions.}

Please note the following restrictions:
\begin{itemize}
\item
|\childdocmain| must be called with one argument \textit{main}
to ensure compatibility with earlier version of the package.
It must either be empty (|\childdocmain{}|)
or precisely match the filename of the main file in which it is specified.
See \secref{sec:detection} for further information.
\item
The filename \textit{main} must be specified without the |.tex| extension.
\item
The filename \textit{main} is case sensitive
(even in case-insensitive file systems)
due to internal string comparison.
\item
The argument \textit{main} should be fully expanded, it cannot be a macro.
\item
Subdirectories and special characters should be avoided in filenames.
\item
The command |\childdocmain{|\textit{main}|}| must be followed by a whitespace.
It should not be followed immediately by another command
or by a comment mark `|%|'.
This is because the \TeX{} parser reads the token immediately following
the argument of |\childdocmain| and puts it
at the beginning of every child section;
however, a white\-space is ignored.
\end{itemize}

%%%%%%%%%%%%%%%%%%%%%%%%%%%%%%%%%%%%%%%%
\paragraph{Content of Main File.}

It is advisable to place all content in the child files included by |\include|.
Any output contained in the main file will appear in all child documents
unless suppressed manually;
it cannot be suppressed automatically by the |\includeonly| directive
and thus should normally be avoided.
A method to include some content in the main file
by means of conditional processing is described in \secref{sec:conditional}.

%%%%%%%%%%%%%%%%%%%%%%%%%%%%%%%%%%%%%%%%
\paragraph{Page Numbering.}

When only a part of the document is compiled,
the appropriate numbering of pages
(as well as other status parameters)
is determined from the |.aux| files.
The latter contain information from previous passes.
However this information needs to propagate through
all intermediate child documents.
Therefore the page numbering in child documents may well
be inconsistent until the complete document is compiled at least once.

A useful (if unconventional) way to always ensure a consistent
page numbering is to restart the numbering in each child document
and denote the pages by `\textit{child}|.|\textit{page}'
where \textit{child} represents the chapter/section number of the child file.
This can be achieved by the command
|\numberwithin{page}{|\textit{child}|}|
of the \textsf{amsmath} package
where \textit{child} can be |chapter| or |section|
depending on the chosen structuring.
Alternatively, one can modify the macro |\thepage| appropriately
and reset the counter |page| at the start of each child file.

%%%%%%%%%%%%%%%%%%%%%%%%%%%%%%%%%%%%%%%%%%%%%%%%%%%%%%%%%%%%%%%%%%%%%%%%%%%%%%%%
\subsection{Conditional Processing}
\label{sec:conditional}

The package provides a mechanism to compile different versions
of a document. To customise the versions further some conditional processing
can come in handy to distinguish which version is being compiled.
The package provides two macros to describe the compilation context:

%%%%%%%%%%%%%%%%%%%%%%%%%%%%%%%%%%%%%%%%
\DescribeMacro{\ifchilddoc}
The conditional |\ifchilddoc| distinguishes between the compilation of
child documents and the main document:
%
\begin{center}
|\ifchilddoc |\textit{child-code}| |[|\||else |\textit{main-code}]| \||fi|
\end{center}

%%%%%%%%%%%%%%%%%%%%%%%%%%%%%%%%%%%%%%%%
\DescribeMacro{\childdocname}
\DescribeMacro{\childdocjob}
The macro |\childdocname| contains the filename (without extension)
of the main or child file being processed.
Note that |\childdocjob| will always contain the name of the main file.

%%%%%%%%%%%%%%%%%%%%%%%%%%%%%%%%%%%%%%%%
\paragraph{Title Page.}

Conditional processing can be used to include a title or banner page
in the main document when proper precautions are taken.
Importantly, the code in the main file should ensure that the page counter
(as well as other status parameters which are stored in the |.aux| files)
takes the same value after the conditional processing.
Otherwise the page numbers may take divergent values
depending on which part is compiled.

For example, a title page could be declared by:
%
\begin{center}
\begin{tabular}{l}
|\ifchilddoc\||else|\\
|\addtocounter{page}{-1}|\\
\textit{code for title page}\\
|\newpage|\\
|\||fi|
\end{tabular}
\end{center}
%
A banner page for the child documents can be generated by:
%
\begin{center}
\begin{tabular}{l}
|\ifchilddoc|\\
|\addtocounter{page}{-1}|\\
\textit{code for banner page}\\
|\newpage|\\
|\||fi|
\end{tabular}
\end{center}
%
Here one could write a message such as:
\begin{center}
|This is the part \childdocname{} of \childdocjob{}.|
\end{center}

%%%%%%%%%%%%%%%%%%%%%%%%%%%%%%%%%%%%%%%%%%%%%%%%%%%%%%%%%%%%%%%%%%%%%%%%%%%%%%%%
\subsection{Flags}
\label{sec:flags}

The package makes it easy to generate different versions
of the main or child documents.
To this end compilation flags can be defined
and assigned different default values.
They will be particularly useful in conjunction
with the forwarding mechanism described in \secref{sec:forward}.

For example, it may be useful to have a flag |\version|
which can be set to |draft| or |final|.
The document source will contain some conditional code
depending on the value of |\version|.
Suppose further, the flag should default to |final| for the main file
and to |draft| for child files
which is a natural assignment for editing the document.
This is achieved by placing the following code
in the preamble of the main document
(below the |\childdocmain| directive):
%
\begin{center}
\begin{tabular}{l}
|\ifchilddoc|\\
|\providecommand{\version}{draft}|\\
|\||else|\\
|\providecommand{\version}{final}|\\
|\||fi|
\end{tabular}
\end{center}
%
The definition by |\providecommand| makes sure
that previous definitions are not overwritten.
Further statements |\providecommand{\version}{...}|
can thus be added before the above code to override it.

For the main file, one might add a line
(between |\childdocmain| and the above block)
%
\begin{center}
|%\ifchilddoc\||else\providecommand{\version}{draft}\||fi|
\end{center}
%
which can be uncommented to produce a draft version.
Likewise one can add a line to the very top of a child file
(above the |\childdocof{|\textit{main}|}| directive)
%
\begin{center}
|%\providecommand{\version}{final}|
\end{center}
%
which can be uncommented to produce the final version of this child document.

%%%%%%%%%%%%%%%%%%%%%%%%%%%%%%%%%%%%%%%%%%%%%%%%%%%%%%%%%%%%%%%%%%%%%%%%%%%%%%%%
\subsection{Forwarding}
\label{sec:forward}

Different versions of the main or child documents
using compilation flags as described in \secref{sec:flags}
can be (permanently) stored in different files
for convenient compilation, viewing and distribution.
To this end, the package defines a command
to pass on compilation to a different file:

%%%%%%%%%%%%%%%%%%%%%%%%%%%%%%%%%%%%%%%%
\DescribeMacro{\childdocforward}
The command |\childdocforward| redirects processing to
another source file:
%
\begin{center}
\begin{tabular}{l}
|\input{childdoc.def}|\\
|\childdocforward[|\textit{main}|]{|\textit{dest}|}|\\
\end{tabular}
\end{center}
%
The argument \textit{dest} is the destination file
(without extension).
It should be the main file or one of the child files.
Note that further \textsf{childdoc} directives
such as |\childdocof| and |\childdocforward|
in the indicated file will be processed in this form.
The optional argument \textit{main}
passes on directly to the main file \textit{main}
while pretending to compile the child \textit{dest}.
This form behaves as if \textit{dest}
issues |\childdocof{|\textit{main}|}| right away,
and no further \textsf{childdoc} directives will be processed.

%%%%%%%%%%%%%%%%%%%%%%%%%%%%%%%%%%%%%%%%
\DescribeMacro{\...prefix}
In the alternative form |\childdocforwardprefix|,
%
\begin{center}
\begin{tabular}{l}
|\input{childdoc.def}|\\
|\childdocforwardprefix[|\textit{main}|]{|\textit{prefix}|}{|\textit{dest}|}|
\end{tabular}
\end{center}
%
the destination file is determined by a pattern
depending on the current file:
To make this work, the current file must be called
`{\textit{prefix}\hspace{0.2em}\textit{suffix}}'
with \textit{prefix} matching precisely the argument.
Processing is then passed on to the file
`{\textit{dest}\hspace{0.2em}\textit{suffix}}'.
Surely, the same effect is achieved by
directly specifying the
argument `{\textit{dest}\hspace{0.2em}\textit{suffix}}'
in the first form.
However, that requires to set up a different file
for each child. With the alternative form of the command
all these files can have exactly the same content
which simplifies setting them up and maintaining them.

For example, the following file |draft.tex|
with a compilation flag |\version| as described in \secref{sec:flags}
compiles the main document as a draft:
%
\begin{center}
\begin{tabular}{l}
|\def\version{draft}|\\
|\input{childdoc.def}|\\
|\childdocforward{|\textit{main}|}|
\end{tabular}
\end{center}
%
Likewise, the following files |final|\textit{nn}|.tex|
compile the final version of the child document
|child|\textit{nn}|.tex|:
%
\begin{center}
\begin{tabular}{l}
|\def\version{final}|\\
|\input{childdoc.def}|\\
|\childdocforwardprefix{final}{child}|
\end{tabular}
\end{center}
%

Note that when several versions of a main file and/or of each child file
are to be generated, it may be convenient to set up a |Makefile| or
shell script to automatise the process.

%%%%%%%%%%%%%%%%%%%%%%%%%%%%%%%%%%%%%%%%%%%%%%%%%%%%%%%%%%%%%%%%%%%%%%%%%%%%%%%%
\subsection{Command Line Processing}
\label{sec:commandline}

The effect of redirection files can also be achieved by invoking
the \LaTeX{} compiler with a more elaborate command line.
Most conveniently this should be done as part
of a shell script or a |Makefile|.

When using \textsf{childdoc} in the main file, the following
command lines effectively perform a redirection
(note that depending on the shell being used,
backslashes may have to be doubled: `|\|' $\to$ `|\\|'):
%
\begin{center}
|... -jobname "|\textit{target}|" |\\|"|[\textit{flags}]%
|\input{childdoc.def}\childdocforward[|\textit{main}|]{|\textit{dest}|}"|
\end{center}
%
Here \textit{target} is the name of the output file,
\textit{main} is the name of the main file
and \textit{dest} is the name of the main or child file to be processed
(all filenames without extensions).
The optional argument \textit{main} can be omitted
if \textit{main} matches \textit{dest}.
Optionally, compilation \textit{flags} can be defined via |\def| commands.
This command line makes the \TeX{} engine believe
it is compiling the file \textit{target}
whose content is specified as the latter parameter.
The provided code then forwards the processing to
\textit{main} or \textit{dest} as described in \secref{sec:forward}.

%%%%%%%%%%%%%%%%%%%%%%%%%%%%%%%%%%%%%%%%%%%%%%%%%%%%%%%%%%%%%%%%%%%%%%%%%%%%%%%%
\subsection{Include by Input}
\label{sec:input}

Including child documents by |\include| has some restrictions by design.
Most notably, the content of a child document always occupies
its own set of pages; pages cannot be shared between child documents.
Usually, this behaviour makes perfect sense
because each child document contain an essential part of the document.
However, in some situations it may be desirable to compose
a document from a collection of parts
without having mandatory page breaks between then.
For this case, the package
provides a mechanism to include parts
by |\input| which can also be processed individually.
However, by construction this mechanism
requires manual handling of the content to be output.

%%%%%%%%%%%%%%%%%%%%%%%%%%%%%%%%%%%%%%%%
\DescribeMacro{\ifchilddocmanual}
The main file should be prepared as usual, see \secref{sec:include}.
However, the document body must make a distinction
between processing of an individual part and of the main document, e.g.:
%
\begin{center}
\begin{tabular}{l}
|\ifchilddocmanual|\\
|\input{\childdocname}|\\
|\||else|\\
\textit{document body with }|\input{|\textit{part}|}|\\
|\||fi|
\end{tabular}
\end{center}
%
The conditional |\ifchilddocmanual| is true whenever
a part to be included by |\input| is being compiled,
and the name of the part is stored in |\childdocname|.

%%%%%%%%%%%%%%%%%%%%%%%%%%%%%%%%%%%%%%%%
\DescribeMacro{\childdocby}
Each part to be included by |\input| should start with:
%
\begin{center}
\begin{tabular}{l}
|\input{childdoc.def}|\\
|\childdocby{|\textit{main}|}|\\
\end{tabular}
\end{center}
%
The directive |\childdocby| is similar to |\childdocof|
described in \secref{sec:include},
but the subsequent selection of content must be done manually.
To that end, both |\ifchilddoc| and |\ifchilddocmanual|
will be true upon processing of a part,
and the name of the part is stored in |\childdocname|.
Note that |\jobname| will be set to the filename of the current part
so that each part receives an individual |.aux| file
that does not interfere with the |.aux| file(s) of the main document.
This behaviour can be altered by the alternative form
|\childdocby[*]{|\textit{main}|}| (with a non-empty optional argument)
which uses the |.aux| file of the main document
by setting |\jobname| to \textit{main}.

%%%%%%%%%%%%%%%%%%%%%%%%%%%%%%%%%%%%%%%%%%%%%%%%%%%%%%%%%%%%%%%%%%%%%%%%%%%%%%%%
\subsection{Driver Development}
\label{sec:driver}

The \textsf{childdoc} mechanism can also be use for the development
of definition files such as \LaTeX{} styles or classes.
This case differs from the above setup with multiple parts
included by |\include| in that no |\includeonly| should be invoked.
This can be achieved by starting the include file
(before |\ProvidesPackage|) with:
%
\begin{center}
\begin{tabular}{l}
|\input{childdoc.def}|\\
|\childdocforward{|\textit{main}|}|\\
\end{tabular}
\end{center}
%
or alternatively with:
%
\begin{center}
\begin{tabular}{l}
|\input{childdoc.def}|\\
|\childdocby{|\textit{main}|}|\\
\end{tabular}
\end{center}
%
Both forms have slightly different effects as described above.
The main file is prepared as usual, see \secref{sec:include}.

%%%%%%%%%%%%%%%%%%%%%%%%%%%%%%%%%%%%%%%%%%%%%%%%%%%%%%%%%%%%%%%%%%%%%%%%%%%%%%%%
\subsection{Legacy Detection}
\label{sec:detection}

The directive |\childdocmain| in the main file can detect
whether the complete document or merely a child is to be compiled
even without using the directive |\childdocof|.
This method is deprecated because it is less robust
and there is no compelling reason to use it;
it is merely provided for backward compatibility
and it may be removed in future versions.

If the detection mechanism is to be used,
it is mandatory to correctly specify
the filename of the main file as the argument of |\childdocmain|:
%
\begin{center}
\begin{tabular}{l}
|\input{childdoc.def}|\\
|\childdocmain{|\textit{main}|}|\\
\end{tabular}
\end{center}
%
If |\jobname| does not match the argument \textit{main} of |\childdocmain|,
it is assumed that |\jobname| points to the child file to be compiled.
When using |\childdocmain| with the main file specified as argument,
it suffices to start a child file
with just |\input{|\textit{main}|}|
without loading of the package and using |\childdocof|.
If instead all processing is done
with the appropriate \textsf{childdoc} directives,
the argument of \textit{main} of |\childdocmain| can be empty.

An alternative version of the command line processing described
in \secref{sec:commandline} using the detection mechanism reads:
%
\begin{center}
|... -jobname "|\textit{target}|" "|[\textit{flags}]%
[|\def\jobname{|\textit{dest}|}|]|\input{|\textit{main}|}"|
\end{center}

%%%%%%%%%%%%%%%%%%%%%%%%%%%%%%%%%%%%%%%%%%%%%%%%%%%%%%%%%%%%%%%%%%%%%%%%%%%%%%%%
\subsection{Manual Code}
\label{sec:manual}

In case one cannot be certain whether the definitions file |childdoc.def|
is installed on the target \TeX{} distribution
and one prefers not to ship it,
it is conceivable to paste a few relevant commands into the sources.

To that end, drop all statements |\input{childdoc.def}|
and perform the replacements as outlined below.
Instead of |\childdocmain{|\textit{main}|}| add the following code
to the top of the main file:
%
\begin{center}
\begin{tabular}{l}
|\||ifdefined\childdocname\endinput\||fi\newif\ifchilddoc|\\
|\edef\childdocname{\scantokens\expandafter{\jobname\noexpand}}|\\
|\def\childdocmain{|\textit{main}|}\||ifx\childdocmain\childdocname\||else|\\
|\childdoctrue\includeonly{\childdocname}\let\jobname\childdocmain\||fi|\\
\end{tabular}
\end{center}
%
Instead of |\childdocof{|\textit{main}|}| just include the main file
at the top of each child file:
%
\begin{center}
|\input{|\textit{main}|}|
\end{center}
%
A simple redirection |\childdocforward{|\textit{dest}|}| is achieved by:
%
\begin{center}
|\def\jobname{|\textit{dest}|}\input{\jobname}|
\end{center}
%
The redirection with prefix
|\childdocforwardprefix[|\textit{prefix}|]{|\textit{dest}|}|
is accomplished by:
%
\begin{center}
\begin{tabular}{l}
|{\edef\jobname{\scantokens\expandafter{\jobname\noexpand}}|\\
|\def\redirectjob |\textit{prefix}|#1~~~{\gdef\jobname{|\textit{dest}|#1}}|\\
|\expandafter\redirectjob\jobname~~~}\input{\jobname}|
\end{tabular}
\end{center}

In an alternative approach,
child documents can be compiled by a specific command line
without additional code or specific definitions:
%
\begin{center}
|... -jobname "|\textit{target}|" "|[\textit{flags}]%
|\includeonly{|\textit{dest}|}\input{|\textit{main}|}"|
\end{center}
%

%%%%%%%%%%%%%%%%%%%%%%%%%%%%%%%%%%%%%%%%%%%%%%%%%%%%%%%%%%%%%%%%%%%%%%%%%%%%%%%%
%%%%%%%%%%%%%%%%%%%%%%%%%%%%%%%%%%%%%%%%%%%%%%%%%%%%%%%%%%%%%%%%%%%%%%%%%%%%%%%%
\section{Information}

%%%%%%%%%%%%%%%%%%%%%%%%%%%%%%%%%%%%%%%%%%%%%%%%%%%%%%%%%%%%%%%%%%%%%%%%%%%%%%%%
\subsection{Copyright}

Copyright \copyright{} 2017--2018 Niklas Beisert

This work may be distributed and/or modified under the
conditions of the \LaTeX{} Project Public License, either version 1.3
of this license or (at your option) any later version.
The latest version of this license is in
  \url{http://www.latex-project.org/lppl.txt}
and version 1.3 or later is part of all distributions of \LaTeX{}
version 2005/12/01 or later.

This work has the LPPL maintenance status `maintained'.

The Current Maintainer of this work is Niklas Beisert.

This work consists of the files |README.txt|, |childdoc.ins| and |childdoc.dtx|
as well as the derived files |childdoc.def|, |cdocsamp.tex|
with |cdocsch1.tex|, |cdocsch2.tex|, |cdocspt3.tex|, |cdocspt4.tex|,
|cdocsdrf.tex|, |cdocsfn1.tex|, |cdocsfn2.tex|
as well as |childdoc.pdf|.

%%%%%%%%%%%%%%%%%%%%%%%%%%%%%%%%%%%%%%%%%%%%%%%%%%%%%%%%%%%%%%%%%%%%%%%%%%%%%%%%
\subsection{Files and Installation}

The package consists of the files:
%
\begin{center}
\begin{tabular}{ll}
    |README.txt|   & readme file \\
    |childdoc.ins| & installation file \\
    |childdoc.dtx| & source file \\
    |childdoc.def| & definition file \\
    |cdocsamp.tex| & sample main file \\
    |cdocsch1.tex| & sample include file \\
    |cdocsch2.tex| & sample include file \\
    |cdocspt3.tex| & sample part file \\
    |cdocspt4.tex| & sample part file \\
    |cdocsdrf.tex| & sample redirection file \\
    |cdocsfn1.tex| & sample redirection file \\
    |cdocsfn2.tex| & sample redirection file \\
    |childdoc.pdf| & manual
\end{tabular}
\end{center}
%
The distribution consists of the files
|README.txt|, |childdoc.ins| and |childdoc.dtx|.
%
\begin{itemize}
\item
Run (pdf)\LaTeX{} on |childdoc.dtx|
to compile the manual |childdoc.pdf| (this file).
\item
Run \LaTeX{} on |childdoc.ins| to create the definitions file |childdoc.def|
and the sample |cdocsamp.tex| with include files
|cdocsch1.tex|, |cdocsch2.tex|, |cdocspt3.tex|, |cdocspt4.tex|,
|cdocsdrf.tex|, |cdocsfn1.tex|, |cdocsfn2.tex|.
Then copy the file |childdoc.def| to an appropriate directory of your \LaTeX{}
distribution, e.g.\ \textit{texmf-root}|/tex/latex/childdoc|.
\end{itemize}

%%%%%%%%%%%%%%%%%%%%%%%%%%%%%%%%%%%%%%%%%%%%%%%%%%%%%%%%%%%%%%%%%%%%%%%%%%%%%%%%
\subsection{Related CTAN Packages}

There are several other packages which offer a similar functionality:
%
\begin{itemize}
\item
The packages
\href{http://ctan.org/pkg/docmute}{\textsf{docmute}},
\href{http://ctan.org/pkg/includex}{\textsf{includex}} and
\href{http://ctan.org/pkg/standalone}{\textsf{standalone}}
provide commands to include only the document body of
a child file thus allowing both files to be compiled individually.
\item
The packages \href{http://ctan.org/pkg/subdocs}{\textsf{subdocs}}
and \href{http://ctan.org/pkg/subfiles}{\textsf{subfiles}}
provide structures in which the main and child documents can be
encapsulated and allowing them to be compiled individually.
The inclusion mechanism is different from the conventional |\include|.
\item
The package \href{http://ctan.org/pkg/combine}{\textsf{combine}}
is an elaborate solution to combine several documents into one.
\end{itemize}
%
See also the CTAN topic \href{http://ctan.org/topic/subdocs}{\textsf{subdocs}}
for further related packages.
The present package differs from the above solutions in that
a document structure constructed with the conventional |\include| mechanism
just needs two extra commands at the top of every file
such that all constituent files can be compiled individually.

%%%%%%%%%%%%%%%%%%%%%%%%%%%%%%%%%%%%%%%%%%%%%%%%%%%%%%%%%%%%%%%%%%%%%%%%%%%%%%%%
%\subsection{Feature Suggestions}
%
%The following is a list of features which may be useful for future
%versions of this package:
%%
%\begin{itemize}
%\item
%\ldots
%\end{itemize}

%%%%%%%%%%%%%%%%%%%%%%%%%%%%%%%%%%%%%%%%%%%%%%%%%%%%%%%%%%%%%%%%%%%%%%%%%%%%%%%%
\subsection{Revision History}

%%%%%%%%%%%%%%%%%%%%%%%%%%%%%%%%%%%%%%%%
\paragraph{v2.0:} 2018/12/30

\begin{itemize}
\item
immediate forward processing
\item
added |\childdocby| mechanism
\item
manual restructured
\end{itemize}

%%%%%%%%%%%%%%%%%%%%%%%%%%%%%%%%%%%%%%%%
\paragraph{v1.6:} 2018/01/17

\begin{itemize}
\item
application for development of include files
\item
corrections to manual
\end{itemize}

%%%%%%%%%%%%%%%%%%%%%%%%%%%%%%%%%%%%%%%%
\paragraph{v1.5:} 2017/05/21

\begin{itemize}
\item
more complete structuring introduced
\item
|\childdocof| introduced
\item
|\childdoc| renamed to |\childdocmain|
\item
|\childredirect| renamed to |\childdocforward| and |\childdocforwardprefix|
and functionality expanded
\end{itemize}

%%%%%%%%%%%%%%%%%%%%%%%%%%%%%%%%%%%%%%%%
\paragraph{v1.0:} 2017/04/27

\begin{itemize}
\item
manual and install package
\item
first version published on CTAN
\end{itemize}

%%%%%%%%%%%%%%%%%%%%%%%%%%%%%%%%%%%%%%%%
\paragraph{v0.6:} 2017/04/26

\begin{itemize}
\item
redirection mechanism added
\end{itemize}

%%%%%%%%%%%%%%%%%%%%%%%%%%%%%%%%%%%%%%%%
\paragraph{v0.5:} 2017/04/26

\begin{itemize}
\item
functionality in definition file
\end{itemize}


%%%%%%%%%%%%%%%%%%%%%%%%%%%%%%%%%%%%%%%%%%%%%%%%%%%%%%%%%%%%%%%%%%%%%%%%%%%%%%%%
%%%%%%%%%%%%%%%%%%%%%%%%%%%%%%%%%%%%%%%%%%%%%%%%%%%%%%%%%%%%%%%%%%%%%%%%%%%%%%%%
%%%%%%%%%%%%%%%%%%%%%%%%%%%%%%%%%%%%%%%%%%%%%%%%%%%%%%%%%%%%%%%%%%%%%%%%%%%%%%%%
\appendix

\settowidth\MacroIndent{\rmfamily\scriptsize 000\ }

 \DocInput{childdoc.dtx}

\end{document}
%</driver>
% \fi
%
% %%%%%%%%%%%%%%%%%%%%%%%%%%%%%%%%%%%%%%%%%%%%%%%%%%%%%%%%%%%%%%%%%%%%%%%%%%%%%%
% %%%%%%%%%%%%%%%%%%%%%%%%%%%%%%%%%%%%%%%%%%%%%%%%%%%%%%%%%%%%%%%%%%%%%%%%%%%%%%
% \section{Sample}
%\iffalse
%<*samplemain>
%\fi
%
% The following presents a sample document
% with two chapters, two parts, a title page,
% a compile flag as well as three forwarding files to set the flag.
% It consists of eight |.tex| files:
% \begin{center}
% \begin{tabular}{ll}
% |cdocsamp.tex|&main file\\
% |cdocsch1.tex|&include file for chapter 1\\
% |cdocsch2.tex|&include file for chapter 2\\
% |cdocspt3.tex|&include file for part 3\\
% |cdocspt4.tex|&include file for part 4\\
% |cdocsdrf.tex|&forwarding file for main file in draft mode\\
% |cdocsfi1.tex|&forwarding file for final version of chapter 1\\
% |cdocsfi2.tex|&forwarding file for final version of chapter 2\\
% \end{tabular}
% \end{center}
% Each of the eight files can be compiled directly by the \LaTeX{} compiler.
%
% %%%%%%%%%%%%%%%%%%%%%%%%%%%%%%%%%%%%%%
% \paragraph{Main File.}
%
% The main file is called |cdocsamp.tex|.
%
% Load the \textsf{childdoc} definitions and
% declare the filename for the main document:
%    \begin{macrocode}
\input{childdoc.def}
\childdocmain{}
%    \end{macrocode}

% Optional override for |\version| flag:
%    \begin{macrocode}
%%\ifchilddoc\else\providecommand{\version}{draft}\fi
%    \end{macrocode}

% Define the default values for the |\version| flag
% (|final| for the main file and |draft| for childs):
%    \begin{macrocode}
\ifchilddoc
\providecommand{\version}{draft}
\else
\providecommand{\version}{final}
\fi
%    \end{macrocode}

% Load the standard document class:
%    \begin{macrocode}
\documentclass[12pt]{article}
%    \end{macrocode}

% Start the document body:
%    \begin{macrocode}
\begin{document}
%    \end{macrocode}

% Declare a title page.
% Print title, part of document being processed and version flag:
%    \begin{macrocode}
\addtocounter{page}{-1}
\begin{center}
{\LARGE\bfseries{}childdoc example\par}
\vspace{1cm}
\ifchilddoc
\ifchilddocmanual part\else chapter\fi:
`\childdocname' of `\childdocjob'\par
\else
main document: `\childdocjob'\par
\fi
version: \version\par
\end{center}
\newpage
%    \end{macrocode}

% Manually include selected file,
% otherwise process as usual:
%    \begin{macrocode}
\ifchilddocmanual
\section*{part `\childdocname'}
\input{\childdocname}
\else
%    \end{macrocode}

% Include the two chapters:
%    \begin{macrocode}
\include{cdocsch1}
\include{cdocsch2}
%    \end{macrocode}

% Include the two parts unless only chapters should be displayed:
%    \begin{macrocode}
\ifchilddoc\else
\section{part three}
\input{cdocspt3}
\section{part four}
\input{cdocspt4}
\fi
%    \end{macrocode}

% Process as usual until here:
%    \begin{macrocode}
\fi
%    \end{macrocode}

% End of document body:
%    \begin{macrocode}
\end{document}
%    \end{macrocode}
%\iffalse
%</samplemain>
%\fi
%
% %%%%%%%%%%%%%%%%%%%%%%%%%%%%%%%%%%%%%%
% \paragraph{Chapter Include Files.}
%
% The include files are called |cdocsch1.tex| and |cdocsch2.tex|.
%
%\iffalse
%<*samplechap1|samplechap2>
%\fi

% Optional override for |\version| flag:
%    \begin{macrocode}
%%\providecommand{\version}{final}
%    \end{macrocode}

% Include the main document:
%    \begin{macrocode}
\input{childdoc.def}
\childdocof{cdocsamp}
%    \end{macrocode}

%\iffalse
%</samplechap1|samplechap2>
%\fi
%
%\iffalse
%<*samplechap1>
%\fi
% Some text for chapter 1:
%    \begin{macrocode}
\section{one}
some text in chapter one
%    \end{macrocode}

%\iffalse
%</samplechap1>
%\fi
% Some text for chapter 2:
%\iffalse
%<*samplechap2>
%\fi
%    \begin{macrocode}
\section{two}
more text in chapter two
%    \end{macrocode}

%\iffalse
%</samplechap2>
%\fi
%
% %%%%%%%%%%%%%%%%%%%%%%%%%%%%%%%%%%%%%%
% \paragraph{Part Include Files.}
%
% The include files are called |cdocspt3.tex| and |cdocspt4.tex|.
%
%\iffalse
%<*samplepart3|samplepart4>
%\fi

% Optional override for |\version| flag:
%    \begin{macrocode}
%%\providecommand{\version}{final}
%    \end{macrocode}

% Include the main document:
%    \begin{macrocode}
\input{childdoc.def}
\childdocby{cdocsamp}
%    \end{macrocode}

%\iffalse
%</samplepart3|samplepart4>
%\fi
%
%\iffalse
%<*samplepart3>
%\fi
% Some text for part 3:
%    \begin{macrocode}
some text in part three
%    \end{macrocode}

%\iffalse
%</samplepart3>
%\fi
% Some text for part 4:
%\iffalse
%<*samplepart4>
%\fi
%    \begin{macrocode}
more text in part four
%    \end{macrocode}

%\iffalse
%</samplepart4>
%\fi
%
% %%%%%%%%%%%%%%%%%%%%%%%%%%%%%%%%%%%%%%
% \paragraph{Forwarding for a Complete Draft.}
%
% The following forwarding file |cdocsdrf.tex|
% compiles the main document in draft mode:
%\iffalse
%<*sampledraft>
%\fi
%    \begin{macrocode}
\def\version{draft}
\input{childdoc.def}
\childdocforward{cdocsamp}
%    \end{macrocode}

%\iffalse
%</sampledraft>
%\fi
%
% %%%%%%%%%%%%%%%%%%%%%%%%%%%%%%%%%%%%%%
% \paragraph{Forwarding for Final Version of the Chapters.}
%
% The following forwarding files |cdocsfn1.tex| and |cdocsfn2.tex|
% (with identical content)
% compile the final versions of the child documents
% |cdocsch1.tex| and |cdocsch2.tex|, respectively:
%\iffalse
%<*samplefinal>
%\fi
%    \begin{macrocode}
\def\version{final}
\input{childdoc.def}
\childdocforwardprefix[cdocsamp]{cdocsfn}{cdocsch}
%    \end{macrocode}

%\iffalse
%</samplefinal>
%\fi
%
% %%%%%%%%%%%%%%%%%%%%%%%%%%%%%%%%%%%%%%
% \paragraph{Command Line Processing.}
%
% The following three command lines generate the output files
% |cdocscld|, |cdocscl1| and |cdocscl2|
% which should be identical to
% |cdocsdrf|, |cdocsch1| and |cdocsfn2|, respectively:
% \begin{center}
% \begin{tabular}{l}
% |latex -jobname cdocscld \|\\
% |  "\def\version{draft}\input{childdoc.def}\childdocforward{cdocsamp}"|\\
% |latex -jobname cdocscl1 \|\\
% |  "\input{childdoc.def}\childdocforward[cdocsamp]{cdocsch1}"|\\
% |latex -jobname cdocscl2 \|\\
% |  "\def\version{final}\input{childdoc.def}\childdocforward{cdocsch2}"|
% \end{tabular}
% \end{center}
% Note that the trailing backslash on each first line
% merely continues the input to the second line
% (for convenient cut ant paste).
% Furthermore, the command |latex| can be replaced by any
% of its alternative versions such as |pdflatex|.
%
% %%%%%%%%%%%%%%%%%%%%%%%%%%%%%%%%%%%%%%%%%%%%%%%%%%%%%%%%%%%%%%%%%%%%%%%%%%%%%%
% %%%%%%%%%%%%%%%%%%%%%%%%%%%%%%%%%%%%%%%%%%%%%%%%%%%%%%%%%%%%%%%%%%%%%%%%%%%%%%
% \section{Implementation}
%\iffalse
%<*package>
%\fi
%
% This section describes the definitions file |childdoc.def|.

% The definitions cannot be loaded using |\usepackage| or |\RequirePackage|
% which has a mechanism to prevent loading a style file more than once.
% When loading the definitions by means of |\input|
% multiple instances have to be prevented manually:
%\iffalse
%This code needs to be before the `\ProvidesFile' directive
%which is defined at the beginning of this file.
%Therefore it is also placed there and commented out here.
%</package>
%<*discard>
%\fi
%    \begin{macrocode}
\ifdefined\childdocmain\endinput\fi
%    \end{macrocode}
%\iffalse
%</discard>
%<*package>
%\fi
%
% \macro{\ifchilddoc}
% \macro{\ifchilddocmanual}
% The conditional |\ifchilddoc| tells whether a
% child (true) or main (false) document is being compiled.
% The conditional |\ifchilddocmanual| tells whether
% the |\includeonly| mechanism is used (false) or
% the selection of child files must be performed manually (true).
% The definitions initialise to false:
%    \begin{macrocode}
\newif\ifchilddoc
\newif\ifchilddocmanual
%    \end{macrocode}

% \macro{\childdocname}
% \macro{\childdocjob}
% The macro |\childdocname| stores the name of the main document
% to be compiled. The macro |\childdocjob| stores the name of
% the document on which the \LaTeX{} compiler was originally invoked.
% The content of |\jobname| cannot be compared
% to filenames specified in the source due to different catcodes.
% The following code rescans |\jobname|, stores the result
% in |\childdocname| and saves a copy in |\childdocjob|:
%    \begin{macrocode}
\edef\childdocname{\scantokens\expandafter{\jobname\noexpand}}
\let\childdocjob\childdocname
%    \end{macrocode}

% \macro{\childdocdisable}
% The macro |\childdocdisable| prevents the main file
% from being processed more than once.
% At this stage, the main document command |\childdocmain|
% is assumed to be called once again where it should do nothing.
% Any subsequent call to it should prevent
% a secondary processing of the main document
% It overwrites the forwarding commands
% |\childdocof| and |\childdocforward|
% with empty macros to prevent further inclusions of the main document:
%    \begin{macrocode}
\newcommand{\childdocdisable}
{
  \renewcommand{\childdocmain}[1]{\renewcommand{\childdocmain}[1]{\endinput}}
  \renewcommand{\childdocof}[1]{}
  \renewcommand{\childdocby}[2][]{}
  \renewcommand{\childdocforward}[2][]{}
  \renewcommand{\childdocdisable}{}
}
%    \end{macrocode}

% \macro{\childdocmain}
% The macro |\childdocmain| is to be called at the top of the main file
% with nothing or the main filename (without extension) as argument.
% First, it breaks loops.
% If the argument is not empty and does not match |\childdocname|
% (which is set by the first inclusion of |childdoc.def|),
% |\ifchilddoc| is set to true, |\includeonly| is applied to the child file
% and |\jobname| is set to the main file
% (for proper handling of |.aux| files):
%    \begin{macrocode}
\newcommand{\childdocmain}[1]
{
  \childdocdisable\childdocmain{}
  \if?#1?\else
    \begingroup
      \def\childdoctmp{#1}
      \ifx\childdoctmp\childdocname
        \def\childdoctmp{}
      \else
        \def\childdoctmp
        {
          \childdoctrue
          \includeonly{\childdocname}
          \def\childdocjob{#1}
          \def\jobname{#1}
        }
      \fi
      \expandafter
    \endgroup
    \childdoctmp
  \fi
}
%    \end{macrocode}

% \macro{\childdocof}
% The command |\childdocof| redirects
% compilation to the main file |#1|.
%    \begin{macrocode}
\newcommand{\childdocof}[1]
{
  \childdocdisable
  \childdoctrue
  \includeonly{\childdocname}
  \def\jobname{#1}
  \def\childdocjob{#1}
  \input{#1}
}
%    \end{macrocode}

% \macro{\childdocby}
% The command |\childdocby| ....
%    \begin{macrocode}
\newcommand{\childdocby}[2][]
{
  \childdocdisable
  \childdoctrue
  \childdocmanualtrue
  \if?#1?\else
    \def\jobname{#2}
  \fi
  \def\childdocjob{#2}
  \input{#2}
  \endinput
}
%    \end{macrocode}

% \macro{\childdocforward}
% The command |\childdocforward| redirects
% compilation to the main file or
% (if the optional argument is given) a child file.
% Parameters are set as if the main file
% or a child file starting with |\childdocof| was compiled.
% Then compilation is handed over to the main file:
%    \begin{macrocode}
\newcommand{\childdocforward}[2][]
{
  \begingroup
    \if?#1?
      \def\childdoctmp
      {
        \def\childdocname{#2}
        \def\childdocjob{#2}
        \def\jobname{#2}
        \input{#2}
        \endinput
      }
    \else
      \def\childdoctmp
      {
        \childdocdisable
        \def\childdocname{#2}
        \childdoctrue
        \includeonly{#2}
        \def\childdocjob{#1}
        \def\jobname{#1}
        \input{#1}
        \endinput
      }
    \fi
    \expandafter
  \endgroup
  \childdoctmp
}
%    \end{macrocode}

% \macro{\childdocforwardprefix}
% The command |\childdocforwardprefix| redirects
% compilation to the main or a child file by means of a pattern.
% The prefix |#1| in the current filename is replaced by |#2|
% and the suffix of the current filename is kept
% (it is assumed that the filename does not contain the substring `|~~~|'
% which is used as a delimiter).
% Compilation is handed over to the new file by |\childdocforward|:
%    \begin{macrocode}
\newcommand{\childdocforwardprefix}[3][]
{
  \begingroup
    \def\childdocextract #2##1~~~{\def\childdoctmp{\childdocforward[#1]{#3##1}}}
    \expandafter\childdocextract\childdocname~~~
    \expandafter
  \endgroup
  \childdoctmp
}
%    \end{macrocode}

% \macro{\childdoc}
% The deprecated macro |\childdoc| is a legacy version of |\childdocmain|:
%    \begin{macrocode}
\newcommand{\childdoc}{\childdocmain}
%    \end{macrocode}

% \macro{\childdocredirect}
% The deprecated macro |\childdocredirect| is a legacy version
% of |\childdocforward| and |\childdocforwardprefix|:
%    \begin{macrocode}
\newcommand{\childdocredirect}[2][]
{
  \begingroup
    \if?#1?
      \def\childdoctmp{\childdocforward{#2}}
    \else
      \def\childdoctmp{\childdocforwardprefix{#1}{#2}}
    \fi
    \expandafter
  \endgroup
  \childdoctmp
}
%    \end{macrocode}

%\iffalse
%</package>
%\fi
%
\endinput
|\\
|\childdocforwardprefix[|\textit{main}|]{|\textit{prefix}|}{|\textit{dest}|}|
\end{tabular}
\end{center}
%
the destination file is determined by a pattern
depending on the current file:
To make this work, the current file must be called
`{\textit{prefix}\hspace{0.2em}\textit{suffix}}'
with \textit{prefix} matching precisely the argument.
Processing is then passed on to the file
`{\textit{dest}\hspace{0.2em}\textit{suffix}}'.
Surely, the same effect is achieved by
directly specifying the
argument `{\textit{dest}\hspace{0.2em}\textit{suffix}}'
in the first form.
However, that requires to set up a different file
for each child. With the alternative form of the command
all these files can have exactly the same content
which simplifies setting them up and maintaining them.

For example, the following file |draft.tex|
with a compilation flag |\version| as described in \secref{sec:flags}
compiles the main document as a draft:
%
\begin{center}
\begin{tabular}{l}
|\def\version{draft}|\\
|% \iffalse
%
% childdoc.dtx Copyright (C) 2017-2018 Niklas Beisert
%
% This work may be distributed and/or modified under the
% conditions of the LaTeX Project Public License, either version 1.3
% of this license or (at your option) any later version.
% The latest version of this license is in
%   http://www.latex-project.org/lppl.txt
% and version 1.3 or later is part of all distributions of LaTeX
% version 2005/12/01 or later.
%
% This work has the LPPL maintenance status `maintained'.
%
% The Current Maintainer of this work is Niklas Beisert.
%
% This work consists of the files childdoc.dtx and childdoc.ins
% and the derived files childdoc.def and cdocsamp.tex with
% cdocsch1.tex, cdocsch2.tex, cdocsdrf.tex, cdocsfn1.tex, cdocsfn2.tex.
%
%<package>\ifdefined\childdocmain\endinput\fi
%<package>\ProvidesFile{childdoc.def}[2018/12/30 v2.0 child document driver]
%<samplemain>\ProvidesFile{cdocsamp.tex}[2018/12/30 v2.0 sample for childdoc]
%<*driver>
%\ProvidesFile{childdoc.drv}[2018/12/30 v2.0 childdoc reference manual file]
\PassOptionsToClass{10pt,a4paper}{article}
\documentclass{ltxdoc}

\usepackage[margin=35mm]{geometry}
\usepackage{hyperref}
\usepackage{hyperxmp}
\usepackage[usenames]{color}

\hypersetup{colorlinks=true}
\hypersetup{pdfstartview=FitH}
\hypersetup{pdfpagemode=UseNone}
\hypersetup{pdfsource={}}
\hypersetup{pdflang={en-UK}}
\hypersetup{pdfcopyright={Copyright 2017-2018 Niklas Beisert.
  This work may be distributed and/or modified under the
  conditions of the LaTeX Project Public License, either version 1.3
  of this license or (at your option) any later version.}}
\hypersetup{pdflicenseurl={http://www.latex-project.org/lppl.txt}}
\hypersetup{pdfcontactaddress={ETH Zurich, ITP, HIT K,
  Wolfgang-Pauli-Strasse 27}}
\hypersetup{pdfcontactpostcode={8093}}
\hypersetup{pdfcontactcity={Zurich}}
\hypersetup{pdfcontactcountry={Switzerland}}
\hypersetup{pdfcontactemail={nbeisert@itp.phys.ethz.ch}}
\hypersetup{pdfcontacturl={http://people.phys.ethz.ch/\xmptilde nbeisert/}}

\newcommand{\secref}[1]{\hyperref[#1]{section \ref*{#1}}}

\parskip1ex
\parindent0pt
\let\olditemize\itemize
\def\itemize{\olditemize\parskip0pt}

\begin{document}

\title{The \textsf{childdoc} Package}
\hypersetup{pdftitle={The childdoc Package}}
\author{Niklas Beisert\\[2ex]
  Institut f\"ur Theoretische Physik\\
  Eidgen\"ossische Technische Hochschule Z\"urich\\
  Wolfgang-Pauli-Strasse 27, 8093 Z\"urich, Switzerland\\[1ex]
  \href{mailto:nbeisert@itp.phys.ethz.ch}
  {\texttt{nbeisert@itp.phys.ethz.ch}}}
\hypersetup{pdfauthor={Niklas Beisert}}
\hypersetup{pdfsubject={Manual for the LaTeX2e Package childdoc}}
\date{30 December 2018, \textsf{v2.0}}
\maketitle

\begin{abstract}\noindent
\textsf{childdoc} is a \LaTeXe{} package
that enables the direct compilation
of document sections included by |\include|
to individual files.
\end{abstract}

\begingroup
\parskip0ex
\tableofcontents
\endgroup

%%%%%%%%%%%%%%%%%%%%%%%%%%%%%%%%%%%%%%%%%%%%%%%%%%%%%%%%%%%%%%%%%%%%%%%%%%%%%%%%
%%%%%%%%%%%%%%%%%%%%%%%%%%%%%%%%%%%%%%%%%%%%%%%%%%%%%%%%%%%%%%%%%%%%%%%%%%%%%%%%
\section{Introduction}

\LaTeX{} provides a mechanism to structure a large document (such as a book)
into a main file and several child files (containing the chapters)
using the |\include| command.
This mechanism is beneficial for documents
which span hundreds of pages in order to
make the source file(s) more manageable.
Moreover, compilation can be restricted to
selected child files by means of the |\includeonly| command.
The latter feature can be used to reduce the compilation time while editing
(this was significantly more useful in the earlier days of \LaTeX{})
or to generate a smaller document which is easier to navigate.
Another application of |\includeonly| is to generate
documents consisting of selected parts of the complete document.

However, there are a few drawbacks of the plain |\include| mechanism:
\begin{itemize}
\item
The child files cannot be compiled on their own,
they can only be compiled via the main file.
A naive editing environment
(such as a text editor with an option
to have the current file processed by \LaTeX)
may require one to switch to the main file before compiling;
attempting to compile the child file produces errors.
\item
The main file must be modified (each time)
to adjust the |\includeonly| command
to the present needs. This easily leaves the main file in a messy state.
\item
The generated document will always carry the filename
of the main document. This is inconvenient if
several child files are to be compiled and
to be kept for distribution.
\end{itemize}

The present package provides a simple interface
to make child files individually compilable by \LaTeX{}.
Compiling a child file then has the same effect as compiling
the main file with an |\includeonly| command
to select the appropriate child.
Moreover the generated document will carry the name of the child
rather than the main file.
This resolves all three above issues.

This feature is meant to make the editing of books,
thesis documents and lecture notes somewhat more convenient.
However, the package can also be used efficiently for
composing a series of documents (such as exercise sheets)
which are typically distributed individually.
It then assists the author in generating the individual documents
(potentially in different versions)
as well as a document containing the collected series.
Another application is in developing style files
or other kinds of included material
where compilation of the style file could redirect
to a sample or test file.

%%%%%%%%%%%%%%%%%%%%%%%%%%%%%%%%%%%%%%%%%%%%%%%%%%%%%%%%%%%%%%%%%%%%%%%%%%%%%%%%
%%%%%%%%%%%%%%%%%%%%%%%%%%%%%%%%%%%%%%%%%%%%%%%%%%%%%%%%%%%%%%%%%%%%%%%%%%%%%%%%
\section{Usage}

First of all, the package \textsf{childdoc} is \emph{not} a standard
\LaTeXe{} |.sty| style file! Therefore it needs to be invoked in
a non-standard way.

%%%%%%%%%%%%%%%%%%%%%%%%%%%%%%%%%%%%%%%%%%%%%%%%%%%%%%%%%%%%%%%%%%%%%%%%%%%%%%%%
\subsection{Included Files}
\label{sec:include}

%%%%%%%%%%%%%%%%%%%%%%%%%%%%%%%%%%%%%%%%
\DescribeMacro{\childdocmain}
To use the package, add the commands
\begin{center}
\begin{tabular}{l}
|\input{childdoc.def}|\\
|\childdocmain{}|\\
\end{tabular}
\end{center}
at the very top of the main \LaTeX{} file,
in particular \emph{before} the |\documentclass| statement!
The argument of |\childdocmain| should be left empty
(but it must be present).

%%%%%%%%%%%%%%%%%%%%%%%%%%%%%%%%%%%%%%%%
\DescribeMacro{\childdocof}
Furthermore, add the commands
\begin{center}
\begin{tabular}{l}
|\input{childdoc.def}|\\
|\childdocof{|\textit{main}|}|\\
\end{tabular}
\end{center}
at the top of every child file \textit{child}
which is included by |\include{|\textit{child}|}|
from within the main file
(or at least for those files to be compiled individually).
The argument \textit{main} must be the filename of the main file.

There are a couple of
considerations in setting up the main and child documents:

%%%%%%%%%%%%%%%%%%%%%%%%%%%%%%%%%%%%%%%%
\paragraph{Restrictions.}

Please note the following restrictions:
\begin{itemize}
\item
|\childdocmain| must be called with one argument \textit{main}
to ensure compatibility with earlier version of the package.
It must either be empty (|\childdocmain{}|)
or precisely match the filename of the main file in which it is specified.
See \secref{sec:detection} for further information.
\item
The filename \textit{main} must be specified without the |.tex| extension.
\item
The filename \textit{main} is case sensitive
(even in case-insensitive file systems)
due to internal string comparison.
\item
The argument \textit{main} should be fully expanded, it cannot be a macro.
\item
Subdirectories and special characters should be avoided in filenames.
\item
The command |\childdocmain{|\textit{main}|}| must be followed by a whitespace.
It should not be followed immediately by another command
or by a comment mark `|%|'.
This is because the \TeX{} parser reads the token immediately following
the argument of |\childdocmain| and puts it
at the beginning of every child section;
however, a white\-space is ignored.
\end{itemize}

%%%%%%%%%%%%%%%%%%%%%%%%%%%%%%%%%%%%%%%%
\paragraph{Content of Main File.}

It is advisable to place all content in the child files included by |\include|.
Any output contained in the main file will appear in all child documents
unless suppressed manually;
it cannot be suppressed automatically by the |\includeonly| directive
and thus should normally be avoided.
A method to include some content in the main file
by means of conditional processing is described in \secref{sec:conditional}.

%%%%%%%%%%%%%%%%%%%%%%%%%%%%%%%%%%%%%%%%
\paragraph{Page Numbering.}

When only a part of the document is compiled,
the appropriate numbering of pages
(as well as other status parameters)
is determined from the |.aux| files.
The latter contain information from previous passes.
However this information needs to propagate through
all intermediate child documents.
Therefore the page numbering in child documents may well
be inconsistent until the complete document is compiled at least once.

A useful (if unconventional) way to always ensure a consistent
page numbering is to restart the numbering in each child document
and denote the pages by `\textit{child}|.|\textit{page}'
where \textit{child} represents the chapter/section number of the child file.
This can be achieved by the command
|\numberwithin{page}{|\textit{child}|}|
of the \textsf{amsmath} package
where \textit{child} can be |chapter| or |section|
depending on the chosen structuring.
Alternatively, one can modify the macro |\thepage| appropriately
and reset the counter |page| at the start of each child file.

%%%%%%%%%%%%%%%%%%%%%%%%%%%%%%%%%%%%%%%%%%%%%%%%%%%%%%%%%%%%%%%%%%%%%%%%%%%%%%%%
\subsection{Conditional Processing}
\label{sec:conditional}

The package provides a mechanism to compile different versions
of a document. To customise the versions further some conditional processing
can come in handy to distinguish which version is being compiled.
The package provides two macros to describe the compilation context:

%%%%%%%%%%%%%%%%%%%%%%%%%%%%%%%%%%%%%%%%
\DescribeMacro{\ifchilddoc}
The conditional |\ifchilddoc| distinguishes between the compilation of
child documents and the main document:
%
\begin{center}
|\ifchilddoc |\textit{child-code}| |[|\||else |\textit{main-code}]| \||fi|
\end{center}

%%%%%%%%%%%%%%%%%%%%%%%%%%%%%%%%%%%%%%%%
\DescribeMacro{\childdocname}
\DescribeMacro{\childdocjob}
The macro |\childdocname| contains the filename (without extension)
of the main or child file being processed.
Note that |\childdocjob| will always contain the name of the main file.

%%%%%%%%%%%%%%%%%%%%%%%%%%%%%%%%%%%%%%%%
\paragraph{Title Page.}

Conditional processing can be used to include a title or banner page
in the main document when proper precautions are taken.
Importantly, the code in the main file should ensure that the page counter
(as well as other status parameters which are stored in the |.aux| files)
takes the same value after the conditional processing.
Otherwise the page numbers may take divergent values
depending on which part is compiled.

For example, a title page could be declared by:
%
\begin{center}
\begin{tabular}{l}
|\ifchilddoc\||else|\\
|\addtocounter{page}{-1}|\\
\textit{code for title page}\\
|\newpage|\\
|\||fi|
\end{tabular}
\end{center}
%
A banner page for the child documents can be generated by:
%
\begin{center}
\begin{tabular}{l}
|\ifchilddoc|\\
|\addtocounter{page}{-1}|\\
\textit{code for banner page}\\
|\newpage|\\
|\||fi|
\end{tabular}
\end{center}
%
Here one could write a message such as:
\begin{center}
|This is the part \childdocname{} of \childdocjob{}.|
\end{center}

%%%%%%%%%%%%%%%%%%%%%%%%%%%%%%%%%%%%%%%%%%%%%%%%%%%%%%%%%%%%%%%%%%%%%%%%%%%%%%%%
\subsection{Flags}
\label{sec:flags}

The package makes it easy to generate different versions
of the main or child documents.
To this end compilation flags can be defined
and assigned different default values.
They will be particularly useful in conjunction
with the forwarding mechanism described in \secref{sec:forward}.

For example, it may be useful to have a flag |\version|
which can be set to |draft| or |final|.
The document source will contain some conditional code
depending on the value of |\version|.
Suppose further, the flag should default to |final| for the main file
and to |draft| for child files
which is a natural assignment for editing the document.
This is achieved by placing the following code
in the preamble of the main document
(below the |\childdocmain| directive):
%
\begin{center}
\begin{tabular}{l}
|\ifchilddoc|\\
|\providecommand{\version}{draft}|\\
|\||else|\\
|\providecommand{\version}{final}|\\
|\||fi|
\end{tabular}
\end{center}
%
The definition by |\providecommand| makes sure
that previous definitions are not overwritten.
Further statements |\providecommand{\version}{...}|
can thus be added before the above code to override it.

For the main file, one might add a line
(between |\childdocmain| and the above block)
%
\begin{center}
|%\ifchilddoc\||else\providecommand{\version}{draft}\||fi|
\end{center}
%
which can be uncommented to produce a draft version.
Likewise one can add a line to the very top of a child file
(above the |\childdocof{|\textit{main}|}| directive)
%
\begin{center}
|%\providecommand{\version}{final}|
\end{center}
%
which can be uncommented to produce the final version of this child document.

%%%%%%%%%%%%%%%%%%%%%%%%%%%%%%%%%%%%%%%%%%%%%%%%%%%%%%%%%%%%%%%%%%%%%%%%%%%%%%%%
\subsection{Forwarding}
\label{sec:forward}

Different versions of the main or child documents
using compilation flags as described in \secref{sec:flags}
can be (permanently) stored in different files
for convenient compilation, viewing and distribution.
To this end, the package defines a command
to pass on compilation to a different file:

%%%%%%%%%%%%%%%%%%%%%%%%%%%%%%%%%%%%%%%%
\DescribeMacro{\childdocforward}
The command |\childdocforward| redirects processing to
another source file:
%
\begin{center}
\begin{tabular}{l}
|\input{childdoc.def}|\\
|\childdocforward[|\textit{main}|]{|\textit{dest}|}|\\
\end{tabular}
\end{center}
%
The argument \textit{dest} is the destination file
(without extension).
It should be the main file or one of the child files.
Note that further \textsf{childdoc} directives
such as |\childdocof| and |\childdocforward|
in the indicated file will be processed in this form.
The optional argument \textit{main}
passes on directly to the main file \textit{main}
while pretending to compile the child \textit{dest}.
This form behaves as if \textit{dest}
issues |\childdocof{|\textit{main}|}| right away,
and no further \textsf{childdoc} directives will be processed.

%%%%%%%%%%%%%%%%%%%%%%%%%%%%%%%%%%%%%%%%
\DescribeMacro{\...prefix}
In the alternative form |\childdocforwardprefix|,
%
\begin{center}
\begin{tabular}{l}
|\input{childdoc.def}|\\
|\childdocforwardprefix[|\textit{main}|]{|\textit{prefix}|}{|\textit{dest}|}|
\end{tabular}
\end{center}
%
the destination file is determined by a pattern
depending on the current file:
To make this work, the current file must be called
`{\textit{prefix}\hspace{0.2em}\textit{suffix}}'
with \textit{prefix} matching precisely the argument.
Processing is then passed on to the file
`{\textit{dest}\hspace{0.2em}\textit{suffix}}'.
Surely, the same effect is achieved by
directly specifying the
argument `{\textit{dest}\hspace{0.2em}\textit{suffix}}'
in the first form.
However, that requires to set up a different file
for each child. With the alternative form of the command
all these files can have exactly the same content
which simplifies setting them up and maintaining them.

For example, the following file |draft.tex|
with a compilation flag |\version| as described in \secref{sec:flags}
compiles the main document as a draft:
%
\begin{center}
\begin{tabular}{l}
|\def\version{draft}|\\
|\input{childdoc.def}|\\
|\childdocforward{|\textit{main}|}|
\end{tabular}
\end{center}
%
Likewise, the following files |final|\textit{nn}|.tex|
compile the final version of the child document
|child|\textit{nn}|.tex|:
%
\begin{center}
\begin{tabular}{l}
|\def\version{final}|\\
|\input{childdoc.def}|\\
|\childdocforwardprefix{final}{child}|
\end{tabular}
\end{center}
%

Note that when several versions of a main file and/or of each child file
are to be generated, it may be convenient to set up a |Makefile| or
shell script to automatise the process.

%%%%%%%%%%%%%%%%%%%%%%%%%%%%%%%%%%%%%%%%%%%%%%%%%%%%%%%%%%%%%%%%%%%%%%%%%%%%%%%%
\subsection{Command Line Processing}
\label{sec:commandline}

The effect of redirection files can also be achieved by invoking
the \LaTeX{} compiler with a more elaborate command line.
Most conveniently this should be done as part
of a shell script or a |Makefile|.

When using \textsf{childdoc} in the main file, the following
command lines effectively perform a redirection
(note that depending on the shell being used,
backslashes may have to be doubled: `|\|' $\to$ `|\\|'):
%
\begin{center}
|... -jobname "|\textit{target}|" |\\|"|[\textit{flags}]%
|\input{childdoc.def}\childdocforward[|\textit{main}|]{|\textit{dest}|}"|
\end{center}
%
Here \textit{target} is the name of the output file,
\textit{main} is the name of the main file
and \textit{dest} is the name of the main or child file to be processed
(all filenames without extensions).
The optional argument \textit{main} can be omitted
if \textit{main} matches \textit{dest}.
Optionally, compilation \textit{flags} can be defined via |\def| commands.
This command line makes the \TeX{} engine believe
it is compiling the file \textit{target}
whose content is specified as the latter parameter.
The provided code then forwards the processing to
\textit{main} or \textit{dest} as described in \secref{sec:forward}.

%%%%%%%%%%%%%%%%%%%%%%%%%%%%%%%%%%%%%%%%%%%%%%%%%%%%%%%%%%%%%%%%%%%%%%%%%%%%%%%%
\subsection{Include by Input}
\label{sec:input}

Including child documents by |\include| has some restrictions by design.
Most notably, the content of a child document always occupies
its own set of pages; pages cannot be shared between child documents.
Usually, this behaviour makes perfect sense
because each child document contain an essential part of the document.
However, in some situations it may be desirable to compose
a document from a collection of parts
without having mandatory page breaks between then.
For this case, the package
provides a mechanism to include parts
by |\input| which can also be processed individually.
However, by construction this mechanism
requires manual handling of the content to be output.

%%%%%%%%%%%%%%%%%%%%%%%%%%%%%%%%%%%%%%%%
\DescribeMacro{\ifchilddocmanual}
The main file should be prepared as usual, see \secref{sec:include}.
However, the document body must make a distinction
between processing of an individual part and of the main document, e.g.:
%
\begin{center}
\begin{tabular}{l}
|\ifchilddocmanual|\\
|\input{\childdocname}|\\
|\||else|\\
\textit{document body with }|\input{|\textit{part}|}|\\
|\||fi|
\end{tabular}
\end{center}
%
The conditional |\ifchilddocmanual| is true whenever
a part to be included by |\input| is being compiled,
and the name of the part is stored in |\childdocname|.

%%%%%%%%%%%%%%%%%%%%%%%%%%%%%%%%%%%%%%%%
\DescribeMacro{\childdocby}
Each part to be included by |\input| should start with:
%
\begin{center}
\begin{tabular}{l}
|\input{childdoc.def}|\\
|\childdocby{|\textit{main}|}|\\
\end{tabular}
\end{center}
%
The directive |\childdocby| is similar to |\childdocof|
described in \secref{sec:include},
but the subsequent selection of content must be done manually.
To that end, both |\ifchilddoc| and |\ifchilddocmanual|
will be true upon processing of a part,
and the name of the part is stored in |\childdocname|.
Note that |\jobname| will be set to the filename of the current part
so that each part receives an individual |.aux| file
that does not interfere with the |.aux| file(s) of the main document.
This behaviour can be altered by the alternative form
|\childdocby[*]{|\textit{main}|}| (with a non-empty optional argument)
which uses the |.aux| file of the main document
by setting |\jobname| to \textit{main}.

%%%%%%%%%%%%%%%%%%%%%%%%%%%%%%%%%%%%%%%%%%%%%%%%%%%%%%%%%%%%%%%%%%%%%%%%%%%%%%%%
\subsection{Driver Development}
\label{sec:driver}

The \textsf{childdoc} mechanism can also be use for the development
of definition files such as \LaTeX{} styles or classes.
This case differs from the above setup with multiple parts
included by |\include| in that no |\includeonly| should be invoked.
This can be achieved by starting the include file
(before |\ProvidesPackage|) with:
%
\begin{center}
\begin{tabular}{l}
|\input{childdoc.def}|\\
|\childdocforward{|\textit{main}|}|\\
\end{tabular}
\end{center}
%
or alternatively with:
%
\begin{center}
\begin{tabular}{l}
|\input{childdoc.def}|\\
|\childdocby{|\textit{main}|}|\\
\end{tabular}
\end{center}
%
Both forms have slightly different effects as described above.
The main file is prepared as usual, see \secref{sec:include}.

%%%%%%%%%%%%%%%%%%%%%%%%%%%%%%%%%%%%%%%%%%%%%%%%%%%%%%%%%%%%%%%%%%%%%%%%%%%%%%%%
\subsection{Legacy Detection}
\label{sec:detection}

The directive |\childdocmain| in the main file can detect
whether the complete document or merely a child is to be compiled
even without using the directive |\childdocof|.
This method is deprecated because it is less robust
and there is no compelling reason to use it;
it is merely provided for backward compatibility
and it may be removed in future versions.

If the detection mechanism is to be used,
it is mandatory to correctly specify
the filename of the main file as the argument of |\childdocmain|:
%
\begin{center}
\begin{tabular}{l}
|\input{childdoc.def}|\\
|\childdocmain{|\textit{main}|}|\\
\end{tabular}
\end{center}
%
If |\jobname| does not match the argument \textit{main} of |\childdocmain|,
it is assumed that |\jobname| points to the child file to be compiled.
When using |\childdocmain| with the main file specified as argument,
it suffices to start a child file
with just |\input{|\textit{main}|}|
without loading of the package and using |\childdocof|.
If instead all processing is done
with the appropriate \textsf{childdoc} directives,
the argument of \textit{main} of |\childdocmain| can be empty.

An alternative version of the command line processing described
in \secref{sec:commandline} using the detection mechanism reads:
%
\begin{center}
|... -jobname "|\textit{target}|" "|[\textit{flags}]%
[|\def\jobname{|\textit{dest}|}|]|\input{|\textit{main}|}"|
\end{center}

%%%%%%%%%%%%%%%%%%%%%%%%%%%%%%%%%%%%%%%%%%%%%%%%%%%%%%%%%%%%%%%%%%%%%%%%%%%%%%%%
\subsection{Manual Code}
\label{sec:manual}

In case one cannot be certain whether the definitions file |childdoc.def|
is installed on the target \TeX{} distribution
and one prefers not to ship it,
it is conceivable to paste a few relevant commands into the sources.

To that end, drop all statements |\input{childdoc.def}|
and perform the replacements as outlined below.
Instead of |\childdocmain{|\textit{main}|}| add the following code
to the top of the main file:
%
\begin{center}
\begin{tabular}{l}
|\||ifdefined\childdocname\endinput\||fi\newif\ifchilddoc|\\
|\edef\childdocname{\scantokens\expandafter{\jobname\noexpand}}|\\
|\def\childdocmain{|\textit{main}|}\||ifx\childdocmain\childdocname\||else|\\
|\childdoctrue\includeonly{\childdocname}\let\jobname\childdocmain\||fi|\\
\end{tabular}
\end{center}
%
Instead of |\childdocof{|\textit{main}|}| just include the main file
at the top of each child file:
%
\begin{center}
|\input{|\textit{main}|}|
\end{center}
%
A simple redirection |\childdocforward{|\textit{dest}|}| is achieved by:
%
\begin{center}
|\def\jobname{|\textit{dest}|}\input{\jobname}|
\end{center}
%
The redirection with prefix
|\childdocforwardprefix[|\textit{prefix}|]{|\textit{dest}|}|
is accomplished by:
%
\begin{center}
\begin{tabular}{l}
|{\edef\jobname{\scantokens\expandafter{\jobname\noexpand}}|\\
|\def\redirectjob |\textit{prefix}|#1~~~{\gdef\jobname{|\textit{dest}|#1}}|\\
|\expandafter\redirectjob\jobname~~~}\input{\jobname}|
\end{tabular}
\end{center}

In an alternative approach,
child documents can be compiled by a specific command line
without additional code or specific definitions:
%
\begin{center}
|... -jobname "|\textit{target}|" "|[\textit{flags}]%
|\includeonly{|\textit{dest}|}\input{|\textit{main}|}"|
\end{center}
%

%%%%%%%%%%%%%%%%%%%%%%%%%%%%%%%%%%%%%%%%%%%%%%%%%%%%%%%%%%%%%%%%%%%%%%%%%%%%%%%%
%%%%%%%%%%%%%%%%%%%%%%%%%%%%%%%%%%%%%%%%%%%%%%%%%%%%%%%%%%%%%%%%%%%%%%%%%%%%%%%%
\section{Information}

%%%%%%%%%%%%%%%%%%%%%%%%%%%%%%%%%%%%%%%%%%%%%%%%%%%%%%%%%%%%%%%%%%%%%%%%%%%%%%%%
\subsection{Copyright}

Copyright \copyright{} 2017--2018 Niklas Beisert

This work may be distributed and/or modified under the
conditions of the \LaTeX{} Project Public License, either version 1.3
of this license or (at your option) any later version.
The latest version of this license is in
  \url{http://www.latex-project.org/lppl.txt}
and version 1.3 or later is part of all distributions of \LaTeX{}
version 2005/12/01 or later.

This work has the LPPL maintenance status `maintained'.

The Current Maintainer of this work is Niklas Beisert.

This work consists of the files |README.txt|, |childdoc.ins| and |childdoc.dtx|
as well as the derived files |childdoc.def|, |cdocsamp.tex|
with |cdocsch1.tex|, |cdocsch2.tex|, |cdocspt3.tex|, |cdocspt4.tex|,
|cdocsdrf.tex|, |cdocsfn1.tex|, |cdocsfn2.tex|
as well as |childdoc.pdf|.

%%%%%%%%%%%%%%%%%%%%%%%%%%%%%%%%%%%%%%%%%%%%%%%%%%%%%%%%%%%%%%%%%%%%%%%%%%%%%%%%
\subsection{Files and Installation}

The package consists of the files:
%
\begin{center}
\begin{tabular}{ll}
    |README.txt|   & readme file \\
    |childdoc.ins| & installation file \\
    |childdoc.dtx| & source file \\
    |childdoc.def| & definition file \\
    |cdocsamp.tex| & sample main file \\
    |cdocsch1.tex| & sample include file \\
    |cdocsch2.tex| & sample include file \\
    |cdocspt3.tex| & sample part file \\
    |cdocspt4.tex| & sample part file \\
    |cdocsdrf.tex| & sample redirection file \\
    |cdocsfn1.tex| & sample redirection file \\
    |cdocsfn2.tex| & sample redirection file \\
    |childdoc.pdf| & manual
\end{tabular}
\end{center}
%
The distribution consists of the files
|README.txt|, |childdoc.ins| and |childdoc.dtx|.
%
\begin{itemize}
\item
Run (pdf)\LaTeX{} on |childdoc.dtx|
to compile the manual |childdoc.pdf| (this file).
\item
Run \LaTeX{} on |childdoc.ins| to create the definitions file |childdoc.def|
and the sample |cdocsamp.tex| with include files
|cdocsch1.tex|, |cdocsch2.tex|, |cdocspt3.tex|, |cdocspt4.tex|,
|cdocsdrf.tex|, |cdocsfn1.tex|, |cdocsfn2.tex|.
Then copy the file |childdoc.def| to an appropriate directory of your \LaTeX{}
distribution, e.g.\ \textit{texmf-root}|/tex/latex/childdoc|.
\end{itemize}

%%%%%%%%%%%%%%%%%%%%%%%%%%%%%%%%%%%%%%%%%%%%%%%%%%%%%%%%%%%%%%%%%%%%%%%%%%%%%%%%
\subsection{Related CTAN Packages}

There are several other packages which offer a similar functionality:
%
\begin{itemize}
\item
The packages
\href{http://ctan.org/pkg/docmute}{\textsf{docmute}},
\href{http://ctan.org/pkg/includex}{\textsf{includex}} and
\href{http://ctan.org/pkg/standalone}{\textsf{standalone}}
provide commands to include only the document body of
a child file thus allowing both files to be compiled individually.
\item
The packages \href{http://ctan.org/pkg/subdocs}{\textsf{subdocs}}
and \href{http://ctan.org/pkg/subfiles}{\textsf{subfiles}}
provide structures in which the main and child documents can be
encapsulated and allowing them to be compiled individually.
The inclusion mechanism is different from the conventional |\include|.
\item
The package \href{http://ctan.org/pkg/combine}{\textsf{combine}}
is an elaborate solution to combine several documents into one.
\end{itemize}
%
See also the CTAN topic \href{http://ctan.org/topic/subdocs}{\textsf{subdocs}}
for further related packages.
The present package differs from the above solutions in that
a document structure constructed with the conventional |\include| mechanism
just needs two extra commands at the top of every file
such that all constituent files can be compiled individually.

%%%%%%%%%%%%%%%%%%%%%%%%%%%%%%%%%%%%%%%%%%%%%%%%%%%%%%%%%%%%%%%%%%%%%%%%%%%%%%%%
%\subsection{Feature Suggestions}
%
%The following is a list of features which may be useful for future
%versions of this package:
%%
%\begin{itemize}
%\item
%\ldots
%\end{itemize}

%%%%%%%%%%%%%%%%%%%%%%%%%%%%%%%%%%%%%%%%%%%%%%%%%%%%%%%%%%%%%%%%%%%%%%%%%%%%%%%%
\subsection{Revision History}

%%%%%%%%%%%%%%%%%%%%%%%%%%%%%%%%%%%%%%%%
\paragraph{v2.0:} 2018/12/30

\begin{itemize}
\item
immediate forward processing
\item
added |\childdocby| mechanism
\item
manual restructured
\end{itemize}

%%%%%%%%%%%%%%%%%%%%%%%%%%%%%%%%%%%%%%%%
\paragraph{v1.6:} 2018/01/17

\begin{itemize}
\item
application for development of include files
\item
corrections to manual
\end{itemize}

%%%%%%%%%%%%%%%%%%%%%%%%%%%%%%%%%%%%%%%%
\paragraph{v1.5:} 2017/05/21

\begin{itemize}
\item
more complete structuring introduced
\item
|\childdocof| introduced
\item
|\childdoc| renamed to |\childdocmain|
\item
|\childredirect| renamed to |\childdocforward| and |\childdocforwardprefix|
and functionality expanded
\end{itemize}

%%%%%%%%%%%%%%%%%%%%%%%%%%%%%%%%%%%%%%%%
\paragraph{v1.0:} 2017/04/27

\begin{itemize}
\item
manual and install package
\item
first version published on CTAN
\end{itemize}

%%%%%%%%%%%%%%%%%%%%%%%%%%%%%%%%%%%%%%%%
\paragraph{v0.6:} 2017/04/26

\begin{itemize}
\item
redirection mechanism added
\end{itemize}

%%%%%%%%%%%%%%%%%%%%%%%%%%%%%%%%%%%%%%%%
\paragraph{v0.5:} 2017/04/26

\begin{itemize}
\item
functionality in definition file
\end{itemize}


%%%%%%%%%%%%%%%%%%%%%%%%%%%%%%%%%%%%%%%%%%%%%%%%%%%%%%%%%%%%%%%%%%%%%%%%%%%%%%%%
%%%%%%%%%%%%%%%%%%%%%%%%%%%%%%%%%%%%%%%%%%%%%%%%%%%%%%%%%%%%%%%%%%%%%%%%%%%%%%%%
%%%%%%%%%%%%%%%%%%%%%%%%%%%%%%%%%%%%%%%%%%%%%%%%%%%%%%%%%%%%%%%%%%%%%%%%%%%%%%%%
\appendix

\settowidth\MacroIndent{\rmfamily\scriptsize 000\ }

 \DocInput{childdoc.dtx}

\end{document}
%</driver>
% \fi
%
% %%%%%%%%%%%%%%%%%%%%%%%%%%%%%%%%%%%%%%%%%%%%%%%%%%%%%%%%%%%%%%%%%%%%%%%%%%%%%%
% %%%%%%%%%%%%%%%%%%%%%%%%%%%%%%%%%%%%%%%%%%%%%%%%%%%%%%%%%%%%%%%%%%%%%%%%%%%%%%
% \section{Sample}
%\iffalse
%<*samplemain>
%\fi
%
% The following presents a sample document
% with two chapters, two parts, a title page,
% a compile flag as well as three forwarding files to set the flag.
% It consists of eight |.tex| files:
% \begin{center}
% \begin{tabular}{ll}
% |cdocsamp.tex|&main file\\
% |cdocsch1.tex|&include file for chapter 1\\
% |cdocsch2.tex|&include file for chapter 2\\
% |cdocspt3.tex|&include file for part 3\\
% |cdocspt4.tex|&include file for part 4\\
% |cdocsdrf.tex|&forwarding file for main file in draft mode\\
% |cdocsfi1.tex|&forwarding file for final version of chapter 1\\
% |cdocsfi2.tex|&forwarding file for final version of chapter 2\\
% \end{tabular}
% \end{center}
% Each of the eight files can be compiled directly by the \LaTeX{} compiler.
%
% %%%%%%%%%%%%%%%%%%%%%%%%%%%%%%%%%%%%%%
% \paragraph{Main File.}
%
% The main file is called |cdocsamp.tex|.
%
% Load the \textsf{childdoc} definitions and
% declare the filename for the main document:
%    \begin{macrocode}
\input{childdoc.def}
\childdocmain{}
%    \end{macrocode}

% Optional override for |\version| flag:
%    \begin{macrocode}
%%\ifchilddoc\else\providecommand{\version}{draft}\fi
%    \end{macrocode}

% Define the default values for the |\version| flag
% (|final| for the main file and |draft| for childs):
%    \begin{macrocode}
\ifchilddoc
\providecommand{\version}{draft}
\else
\providecommand{\version}{final}
\fi
%    \end{macrocode}

% Load the standard document class:
%    \begin{macrocode}
\documentclass[12pt]{article}
%    \end{macrocode}

% Start the document body:
%    \begin{macrocode}
\begin{document}
%    \end{macrocode}

% Declare a title page.
% Print title, part of document being processed and version flag:
%    \begin{macrocode}
\addtocounter{page}{-1}
\begin{center}
{\LARGE\bfseries{}childdoc example\par}
\vspace{1cm}
\ifchilddoc
\ifchilddocmanual part\else chapter\fi:
`\childdocname' of `\childdocjob'\par
\else
main document: `\childdocjob'\par
\fi
version: \version\par
\end{center}
\newpage
%    \end{macrocode}

% Manually include selected file,
% otherwise process as usual:
%    \begin{macrocode}
\ifchilddocmanual
\section*{part `\childdocname'}
\input{\childdocname}
\else
%    \end{macrocode}

% Include the two chapters:
%    \begin{macrocode}
\include{cdocsch1}
\include{cdocsch2}
%    \end{macrocode}

% Include the two parts unless only chapters should be displayed:
%    \begin{macrocode}
\ifchilddoc\else
\section{part three}
\input{cdocspt3}
\section{part four}
\input{cdocspt4}
\fi
%    \end{macrocode}

% Process as usual until here:
%    \begin{macrocode}
\fi
%    \end{macrocode}

% End of document body:
%    \begin{macrocode}
\end{document}
%    \end{macrocode}
%\iffalse
%</samplemain>
%\fi
%
% %%%%%%%%%%%%%%%%%%%%%%%%%%%%%%%%%%%%%%
% \paragraph{Chapter Include Files.}
%
% The include files are called |cdocsch1.tex| and |cdocsch2.tex|.
%
%\iffalse
%<*samplechap1|samplechap2>
%\fi

% Optional override for |\version| flag:
%    \begin{macrocode}
%%\providecommand{\version}{final}
%    \end{macrocode}

% Include the main document:
%    \begin{macrocode}
\input{childdoc.def}
\childdocof{cdocsamp}
%    \end{macrocode}

%\iffalse
%</samplechap1|samplechap2>
%\fi
%
%\iffalse
%<*samplechap1>
%\fi
% Some text for chapter 1:
%    \begin{macrocode}
\section{one}
some text in chapter one
%    \end{macrocode}

%\iffalse
%</samplechap1>
%\fi
% Some text for chapter 2:
%\iffalse
%<*samplechap2>
%\fi
%    \begin{macrocode}
\section{two}
more text in chapter two
%    \end{macrocode}

%\iffalse
%</samplechap2>
%\fi
%
% %%%%%%%%%%%%%%%%%%%%%%%%%%%%%%%%%%%%%%
% \paragraph{Part Include Files.}
%
% The include files are called |cdocspt3.tex| and |cdocspt4.tex|.
%
%\iffalse
%<*samplepart3|samplepart4>
%\fi

% Optional override for |\version| flag:
%    \begin{macrocode}
%%\providecommand{\version}{final}
%    \end{macrocode}

% Include the main document:
%    \begin{macrocode}
\input{childdoc.def}
\childdocby{cdocsamp}
%    \end{macrocode}

%\iffalse
%</samplepart3|samplepart4>
%\fi
%
%\iffalse
%<*samplepart3>
%\fi
% Some text for part 3:
%    \begin{macrocode}
some text in part three
%    \end{macrocode}

%\iffalse
%</samplepart3>
%\fi
% Some text for part 4:
%\iffalse
%<*samplepart4>
%\fi
%    \begin{macrocode}
more text in part four
%    \end{macrocode}

%\iffalse
%</samplepart4>
%\fi
%
% %%%%%%%%%%%%%%%%%%%%%%%%%%%%%%%%%%%%%%
% \paragraph{Forwarding for a Complete Draft.}
%
% The following forwarding file |cdocsdrf.tex|
% compiles the main document in draft mode:
%\iffalse
%<*sampledraft>
%\fi
%    \begin{macrocode}
\def\version{draft}
\input{childdoc.def}
\childdocforward{cdocsamp}
%    \end{macrocode}

%\iffalse
%</sampledraft>
%\fi
%
% %%%%%%%%%%%%%%%%%%%%%%%%%%%%%%%%%%%%%%
% \paragraph{Forwarding for Final Version of the Chapters.}
%
% The following forwarding files |cdocsfn1.tex| and |cdocsfn2.tex|
% (with identical content)
% compile the final versions of the child documents
% |cdocsch1.tex| and |cdocsch2.tex|, respectively:
%\iffalse
%<*samplefinal>
%\fi
%    \begin{macrocode}
\def\version{final}
\input{childdoc.def}
\childdocforwardprefix[cdocsamp]{cdocsfn}{cdocsch}
%    \end{macrocode}

%\iffalse
%</samplefinal>
%\fi
%
% %%%%%%%%%%%%%%%%%%%%%%%%%%%%%%%%%%%%%%
% \paragraph{Command Line Processing.}
%
% The following three command lines generate the output files
% |cdocscld|, |cdocscl1| and |cdocscl2|
% which should be identical to
% |cdocsdrf|, |cdocsch1| and |cdocsfn2|, respectively:
% \begin{center}
% \begin{tabular}{l}
% |latex -jobname cdocscld \|\\
% |  "\def\version{draft}\input{childdoc.def}\childdocforward{cdocsamp}"|\\
% |latex -jobname cdocscl1 \|\\
% |  "\input{childdoc.def}\childdocforward[cdocsamp]{cdocsch1}"|\\
% |latex -jobname cdocscl2 \|\\
% |  "\def\version{final}\input{childdoc.def}\childdocforward{cdocsch2}"|
% \end{tabular}
% \end{center}
% Note that the trailing backslash on each first line
% merely continues the input to the second line
% (for convenient cut ant paste).
% Furthermore, the command |latex| can be replaced by any
% of its alternative versions such as |pdflatex|.
%
% %%%%%%%%%%%%%%%%%%%%%%%%%%%%%%%%%%%%%%%%%%%%%%%%%%%%%%%%%%%%%%%%%%%%%%%%%%%%%%
% %%%%%%%%%%%%%%%%%%%%%%%%%%%%%%%%%%%%%%%%%%%%%%%%%%%%%%%%%%%%%%%%%%%%%%%%%%%%%%
% \section{Implementation}
%\iffalse
%<*package>
%\fi
%
% This section describes the definitions file |childdoc.def|.

% The definitions cannot be loaded using |\usepackage| or |\RequirePackage|
% which has a mechanism to prevent loading a style file more than once.
% When loading the definitions by means of |\input|
% multiple instances have to be prevented manually:
%\iffalse
%This code needs to be before the `\ProvidesFile' directive
%which is defined at the beginning of this file.
%Therefore it is also placed there and commented out here.
%</package>
%<*discard>
%\fi
%    \begin{macrocode}
\ifdefined\childdocmain\endinput\fi
%    \end{macrocode}
%\iffalse
%</discard>
%<*package>
%\fi
%
% \macro{\ifchilddoc}
% \macro{\ifchilddocmanual}
% The conditional |\ifchilddoc| tells whether a
% child (true) or main (false) document is being compiled.
% The conditional |\ifchilddocmanual| tells whether
% the |\includeonly| mechanism is used (false) or
% the selection of child files must be performed manually (true).
% The definitions initialise to false:
%    \begin{macrocode}
\newif\ifchilddoc
\newif\ifchilddocmanual
%    \end{macrocode}

% \macro{\childdocname}
% \macro{\childdocjob}
% The macro |\childdocname| stores the name of the main document
% to be compiled. The macro |\childdocjob| stores the name of
% the document on which the \LaTeX{} compiler was originally invoked.
% The content of |\jobname| cannot be compared
% to filenames specified in the source due to different catcodes.
% The following code rescans |\jobname|, stores the result
% in |\childdocname| and saves a copy in |\childdocjob|:
%    \begin{macrocode}
\edef\childdocname{\scantokens\expandafter{\jobname\noexpand}}
\let\childdocjob\childdocname
%    \end{macrocode}

% \macro{\childdocdisable}
% The macro |\childdocdisable| prevents the main file
% from being processed more than once.
% At this stage, the main document command |\childdocmain|
% is assumed to be called once again where it should do nothing.
% Any subsequent call to it should prevent
% a secondary processing of the main document
% It overwrites the forwarding commands
% |\childdocof| and |\childdocforward|
% with empty macros to prevent further inclusions of the main document:
%    \begin{macrocode}
\newcommand{\childdocdisable}
{
  \renewcommand{\childdocmain}[1]{\renewcommand{\childdocmain}[1]{\endinput}}
  \renewcommand{\childdocof}[1]{}
  \renewcommand{\childdocby}[2][]{}
  \renewcommand{\childdocforward}[2][]{}
  \renewcommand{\childdocdisable}{}
}
%    \end{macrocode}

% \macro{\childdocmain}
% The macro |\childdocmain| is to be called at the top of the main file
% with nothing or the main filename (without extension) as argument.
% First, it breaks loops.
% If the argument is not empty and does not match |\childdocname|
% (which is set by the first inclusion of |childdoc.def|),
% |\ifchilddoc| is set to true, |\includeonly| is applied to the child file
% and |\jobname| is set to the main file
% (for proper handling of |.aux| files):
%    \begin{macrocode}
\newcommand{\childdocmain}[1]
{
  \childdocdisable\childdocmain{}
  \if?#1?\else
    \begingroup
      \def\childdoctmp{#1}
      \ifx\childdoctmp\childdocname
        \def\childdoctmp{}
      \else
        \def\childdoctmp
        {
          \childdoctrue
          \includeonly{\childdocname}
          \def\childdocjob{#1}
          \def\jobname{#1}
        }
      \fi
      \expandafter
    \endgroup
    \childdoctmp
  \fi
}
%    \end{macrocode}

% \macro{\childdocof}
% The command |\childdocof| redirects
% compilation to the main file |#1|.
%    \begin{macrocode}
\newcommand{\childdocof}[1]
{
  \childdocdisable
  \childdoctrue
  \includeonly{\childdocname}
  \def\jobname{#1}
  \def\childdocjob{#1}
  \input{#1}
}
%    \end{macrocode}

% \macro{\childdocby}
% The command |\childdocby| ....
%    \begin{macrocode}
\newcommand{\childdocby}[2][]
{
  \childdocdisable
  \childdoctrue
  \childdocmanualtrue
  \if?#1?\else
    \def\jobname{#2}
  \fi
  \def\childdocjob{#2}
  \input{#2}
  \endinput
}
%    \end{macrocode}

% \macro{\childdocforward}
% The command |\childdocforward| redirects
% compilation to the main file or
% (if the optional argument is given) a child file.
% Parameters are set as if the main file
% or a child file starting with |\childdocof| was compiled.
% Then compilation is handed over to the main file:
%    \begin{macrocode}
\newcommand{\childdocforward}[2][]
{
  \begingroup
    \if?#1?
      \def\childdoctmp
      {
        \def\childdocname{#2}
        \def\childdocjob{#2}
        \def\jobname{#2}
        \input{#2}
        \endinput
      }
    \else
      \def\childdoctmp
      {
        \childdocdisable
        \def\childdocname{#2}
        \childdoctrue
        \includeonly{#2}
        \def\childdocjob{#1}
        \def\jobname{#1}
        \input{#1}
        \endinput
      }
    \fi
    \expandafter
  \endgroup
  \childdoctmp
}
%    \end{macrocode}

% \macro{\childdocforwardprefix}
% The command |\childdocforwardprefix| redirects
% compilation to the main or a child file by means of a pattern.
% The prefix |#1| in the current filename is replaced by |#2|
% and the suffix of the current filename is kept
% (it is assumed that the filename does not contain the substring `|~~~|'
% which is used as a delimiter).
% Compilation is handed over to the new file by |\childdocforward|:
%    \begin{macrocode}
\newcommand{\childdocforwardprefix}[3][]
{
  \begingroup
    \def\childdocextract #2##1~~~{\def\childdoctmp{\childdocforward[#1]{#3##1}}}
    \expandafter\childdocextract\childdocname~~~
    \expandafter
  \endgroup
  \childdoctmp
}
%    \end{macrocode}

% \macro{\childdoc}
% The deprecated macro |\childdoc| is a legacy version of |\childdocmain|:
%    \begin{macrocode}
\newcommand{\childdoc}{\childdocmain}
%    \end{macrocode}

% \macro{\childdocredirect}
% The deprecated macro |\childdocredirect| is a legacy version
% of |\childdocforward| and |\childdocforwardprefix|:
%    \begin{macrocode}
\newcommand{\childdocredirect}[2][]
{
  \begingroup
    \if?#1?
      \def\childdoctmp{\childdocforward{#2}}
    \else
      \def\childdoctmp{\childdocforwardprefix{#1}{#2}}
    \fi
    \expandafter
  \endgroup
  \childdoctmp
}
%    \end{macrocode}

%\iffalse
%</package>
%\fi
%
\endinput
|\\
|\childdocforward{|\textit{main}|}|
\end{tabular}
\end{center}
%
Likewise, the following files |final|\textit{nn}|.tex|
compile the final version of the child document
|child|\textit{nn}|.tex|:
%
\begin{center}
\begin{tabular}{l}
|\def\version{final}|\\
|% \iffalse
%
% childdoc.dtx Copyright (C) 2017-2018 Niklas Beisert
%
% This work may be distributed and/or modified under the
% conditions of the LaTeX Project Public License, either version 1.3
% of this license or (at your option) any later version.
% The latest version of this license is in
%   http://www.latex-project.org/lppl.txt
% and version 1.3 or later is part of all distributions of LaTeX
% version 2005/12/01 or later.
%
% This work has the LPPL maintenance status `maintained'.
%
% The Current Maintainer of this work is Niklas Beisert.
%
% This work consists of the files childdoc.dtx and childdoc.ins
% and the derived files childdoc.def and cdocsamp.tex with
% cdocsch1.tex, cdocsch2.tex, cdocsdrf.tex, cdocsfn1.tex, cdocsfn2.tex.
%
%<package>\ifdefined\childdocmain\endinput\fi
%<package>\ProvidesFile{childdoc.def}[2018/12/30 v2.0 child document driver]
%<samplemain>\ProvidesFile{cdocsamp.tex}[2018/12/30 v2.0 sample for childdoc]
%<*driver>
%\ProvidesFile{childdoc.drv}[2018/12/30 v2.0 childdoc reference manual file]
\PassOptionsToClass{10pt,a4paper}{article}
\documentclass{ltxdoc}

\usepackage[margin=35mm]{geometry}
\usepackage{hyperref}
\usepackage{hyperxmp}
\usepackage[usenames]{color}

\hypersetup{colorlinks=true}
\hypersetup{pdfstartview=FitH}
\hypersetup{pdfpagemode=UseNone}
\hypersetup{pdfsource={}}
\hypersetup{pdflang={en-UK}}
\hypersetup{pdfcopyright={Copyright 2017-2018 Niklas Beisert.
  This work may be distributed and/or modified under the
  conditions of the LaTeX Project Public License, either version 1.3
  of this license or (at your option) any later version.}}
\hypersetup{pdflicenseurl={http://www.latex-project.org/lppl.txt}}
\hypersetup{pdfcontactaddress={ETH Zurich, ITP, HIT K,
  Wolfgang-Pauli-Strasse 27}}
\hypersetup{pdfcontactpostcode={8093}}
\hypersetup{pdfcontactcity={Zurich}}
\hypersetup{pdfcontactcountry={Switzerland}}
\hypersetup{pdfcontactemail={nbeisert@itp.phys.ethz.ch}}
\hypersetup{pdfcontacturl={http://people.phys.ethz.ch/\xmptilde nbeisert/}}

\newcommand{\secref}[1]{\hyperref[#1]{section \ref*{#1}}}

\parskip1ex
\parindent0pt
\let\olditemize\itemize
\def\itemize{\olditemize\parskip0pt}

\begin{document}

\title{The \textsf{childdoc} Package}
\hypersetup{pdftitle={The childdoc Package}}
\author{Niklas Beisert\\[2ex]
  Institut f\"ur Theoretische Physik\\
  Eidgen\"ossische Technische Hochschule Z\"urich\\
  Wolfgang-Pauli-Strasse 27, 8093 Z\"urich, Switzerland\\[1ex]
  \href{mailto:nbeisert@itp.phys.ethz.ch}
  {\texttt{nbeisert@itp.phys.ethz.ch}}}
\hypersetup{pdfauthor={Niklas Beisert}}
\hypersetup{pdfsubject={Manual for the LaTeX2e Package childdoc}}
\date{30 December 2018, \textsf{v2.0}}
\maketitle

\begin{abstract}\noindent
\textsf{childdoc} is a \LaTeXe{} package
that enables the direct compilation
of document sections included by |\include|
to individual files.
\end{abstract}

\begingroup
\parskip0ex
\tableofcontents
\endgroup

%%%%%%%%%%%%%%%%%%%%%%%%%%%%%%%%%%%%%%%%%%%%%%%%%%%%%%%%%%%%%%%%%%%%%%%%%%%%%%%%
%%%%%%%%%%%%%%%%%%%%%%%%%%%%%%%%%%%%%%%%%%%%%%%%%%%%%%%%%%%%%%%%%%%%%%%%%%%%%%%%
\section{Introduction}

\LaTeX{} provides a mechanism to structure a large document (such as a book)
into a main file and several child files (containing the chapters)
using the |\include| command.
This mechanism is beneficial for documents
which span hundreds of pages in order to
make the source file(s) more manageable.
Moreover, compilation can be restricted to
selected child files by means of the |\includeonly| command.
The latter feature can be used to reduce the compilation time while editing
(this was significantly more useful in the earlier days of \LaTeX{})
or to generate a smaller document which is easier to navigate.
Another application of |\includeonly| is to generate
documents consisting of selected parts of the complete document.

However, there are a few drawbacks of the plain |\include| mechanism:
\begin{itemize}
\item
The child files cannot be compiled on their own,
they can only be compiled via the main file.
A naive editing environment
(such as a text editor with an option
to have the current file processed by \LaTeX)
may require one to switch to the main file before compiling;
attempting to compile the child file produces errors.
\item
The main file must be modified (each time)
to adjust the |\includeonly| command
to the present needs. This easily leaves the main file in a messy state.
\item
The generated document will always carry the filename
of the main document. This is inconvenient if
several child files are to be compiled and
to be kept for distribution.
\end{itemize}

The present package provides a simple interface
to make child files individually compilable by \LaTeX{}.
Compiling a child file then has the same effect as compiling
the main file with an |\includeonly| command
to select the appropriate child.
Moreover the generated document will carry the name of the child
rather than the main file.
This resolves all three above issues.

This feature is meant to make the editing of books,
thesis documents and lecture notes somewhat more convenient.
However, the package can also be used efficiently for
composing a series of documents (such as exercise sheets)
which are typically distributed individually.
It then assists the author in generating the individual documents
(potentially in different versions)
as well as a document containing the collected series.
Another application is in developing style files
or other kinds of included material
where compilation of the style file could redirect
to a sample or test file.

%%%%%%%%%%%%%%%%%%%%%%%%%%%%%%%%%%%%%%%%%%%%%%%%%%%%%%%%%%%%%%%%%%%%%%%%%%%%%%%%
%%%%%%%%%%%%%%%%%%%%%%%%%%%%%%%%%%%%%%%%%%%%%%%%%%%%%%%%%%%%%%%%%%%%%%%%%%%%%%%%
\section{Usage}

First of all, the package \textsf{childdoc} is \emph{not} a standard
\LaTeXe{} |.sty| style file! Therefore it needs to be invoked in
a non-standard way.

%%%%%%%%%%%%%%%%%%%%%%%%%%%%%%%%%%%%%%%%%%%%%%%%%%%%%%%%%%%%%%%%%%%%%%%%%%%%%%%%
\subsection{Included Files}
\label{sec:include}

%%%%%%%%%%%%%%%%%%%%%%%%%%%%%%%%%%%%%%%%
\DescribeMacro{\childdocmain}
To use the package, add the commands
\begin{center}
\begin{tabular}{l}
|\input{childdoc.def}|\\
|\childdocmain{}|\\
\end{tabular}
\end{center}
at the very top of the main \LaTeX{} file,
in particular \emph{before} the |\documentclass| statement!
The argument of |\childdocmain| should be left empty
(but it must be present).

%%%%%%%%%%%%%%%%%%%%%%%%%%%%%%%%%%%%%%%%
\DescribeMacro{\childdocof}
Furthermore, add the commands
\begin{center}
\begin{tabular}{l}
|\input{childdoc.def}|\\
|\childdocof{|\textit{main}|}|\\
\end{tabular}
\end{center}
at the top of every child file \textit{child}
which is included by |\include{|\textit{child}|}|
from within the main file
(or at least for those files to be compiled individually).
The argument \textit{main} must be the filename of the main file.

There are a couple of
considerations in setting up the main and child documents:

%%%%%%%%%%%%%%%%%%%%%%%%%%%%%%%%%%%%%%%%
\paragraph{Restrictions.}

Please note the following restrictions:
\begin{itemize}
\item
|\childdocmain| must be called with one argument \textit{main}
to ensure compatibility with earlier version of the package.
It must either be empty (|\childdocmain{}|)
or precisely match the filename of the main file in which it is specified.
See \secref{sec:detection} for further information.
\item
The filename \textit{main} must be specified without the |.tex| extension.
\item
The filename \textit{main} is case sensitive
(even in case-insensitive file systems)
due to internal string comparison.
\item
The argument \textit{main} should be fully expanded, it cannot be a macro.
\item
Subdirectories and special characters should be avoided in filenames.
\item
The command |\childdocmain{|\textit{main}|}| must be followed by a whitespace.
It should not be followed immediately by another command
or by a comment mark `|%|'.
This is because the \TeX{} parser reads the token immediately following
the argument of |\childdocmain| and puts it
at the beginning of every child section;
however, a white\-space is ignored.
\end{itemize}

%%%%%%%%%%%%%%%%%%%%%%%%%%%%%%%%%%%%%%%%
\paragraph{Content of Main File.}

It is advisable to place all content in the child files included by |\include|.
Any output contained in the main file will appear in all child documents
unless suppressed manually;
it cannot be suppressed automatically by the |\includeonly| directive
and thus should normally be avoided.
A method to include some content in the main file
by means of conditional processing is described in \secref{sec:conditional}.

%%%%%%%%%%%%%%%%%%%%%%%%%%%%%%%%%%%%%%%%
\paragraph{Page Numbering.}

When only a part of the document is compiled,
the appropriate numbering of pages
(as well as other status parameters)
is determined from the |.aux| files.
The latter contain information from previous passes.
However this information needs to propagate through
all intermediate child documents.
Therefore the page numbering in child documents may well
be inconsistent until the complete document is compiled at least once.

A useful (if unconventional) way to always ensure a consistent
page numbering is to restart the numbering in each child document
and denote the pages by `\textit{child}|.|\textit{page}'
where \textit{child} represents the chapter/section number of the child file.
This can be achieved by the command
|\numberwithin{page}{|\textit{child}|}|
of the \textsf{amsmath} package
where \textit{child} can be |chapter| or |section|
depending on the chosen structuring.
Alternatively, one can modify the macro |\thepage| appropriately
and reset the counter |page| at the start of each child file.

%%%%%%%%%%%%%%%%%%%%%%%%%%%%%%%%%%%%%%%%%%%%%%%%%%%%%%%%%%%%%%%%%%%%%%%%%%%%%%%%
\subsection{Conditional Processing}
\label{sec:conditional}

The package provides a mechanism to compile different versions
of a document. To customise the versions further some conditional processing
can come in handy to distinguish which version is being compiled.
The package provides two macros to describe the compilation context:

%%%%%%%%%%%%%%%%%%%%%%%%%%%%%%%%%%%%%%%%
\DescribeMacro{\ifchilddoc}
The conditional |\ifchilddoc| distinguishes between the compilation of
child documents and the main document:
%
\begin{center}
|\ifchilddoc |\textit{child-code}| |[|\||else |\textit{main-code}]| \||fi|
\end{center}

%%%%%%%%%%%%%%%%%%%%%%%%%%%%%%%%%%%%%%%%
\DescribeMacro{\childdocname}
\DescribeMacro{\childdocjob}
The macro |\childdocname| contains the filename (without extension)
of the main or child file being processed.
Note that |\childdocjob| will always contain the name of the main file.

%%%%%%%%%%%%%%%%%%%%%%%%%%%%%%%%%%%%%%%%
\paragraph{Title Page.}

Conditional processing can be used to include a title or banner page
in the main document when proper precautions are taken.
Importantly, the code in the main file should ensure that the page counter
(as well as other status parameters which are stored in the |.aux| files)
takes the same value after the conditional processing.
Otherwise the page numbers may take divergent values
depending on which part is compiled.

For example, a title page could be declared by:
%
\begin{center}
\begin{tabular}{l}
|\ifchilddoc\||else|\\
|\addtocounter{page}{-1}|\\
\textit{code for title page}\\
|\newpage|\\
|\||fi|
\end{tabular}
\end{center}
%
A banner page for the child documents can be generated by:
%
\begin{center}
\begin{tabular}{l}
|\ifchilddoc|\\
|\addtocounter{page}{-1}|\\
\textit{code for banner page}\\
|\newpage|\\
|\||fi|
\end{tabular}
\end{center}
%
Here one could write a message such as:
\begin{center}
|This is the part \childdocname{} of \childdocjob{}.|
\end{center}

%%%%%%%%%%%%%%%%%%%%%%%%%%%%%%%%%%%%%%%%%%%%%%%%%%%%%%%%%%%%%%%%%%%%%%%%%%%%%%%%
\subsection{Flags}
\label{sec:flags}

The package makes it easy to generate different versions
of the main or child documents.
To this end compilation flags can be defined
and assigned different default values.
They will be particularly useful in conjunction
with the forwarding mechanism described in \secref{sec:forward}.

For example, it may be useful to have a flag |\version|
which can be set to |draft| or |final|.
The document source will contain some conditional code
depending on the value of |\version|.
Suppose further, the flag should default to |final| for the main file
and to |draft| for child files
which is a natural assignment for editing the document.
This is achieved by placing the following code
in the preamble of the main document
(below the |\childdocmain| directive):
%
\begin{center}
\begin{tabular}{l}
|\ifchilddoc|\\
|\providecommand{\version}{draft}|\\
|\||else|\\
|\providecommand{\version}{final}|\\
|\||fi|
\end{tabular}
\end{center}
%
The definition by |\providecommand| makes sure
that previous definitions are not overwritten.
Further statements |\providecommand{\version}{...}|
can thus be added before the above code to override it.

For the main file, one might add a line
(between |\childdocmain| and the above block)
%
\begin{center}
|%\ifchilddoc\||else\providecommand{\version}{draft}\||fi|
\end{center}
%
which can be uncommented to produce a draft version.
Likewise one can add a line to the very top of a child file
(above the |\childdocof{|\textit{main}|}| directive)
%
\begin{center}
|%\providecommand{\version}{final}|
\end{center}
%
which can be uncommented to produce the final version of this child document.

%%%%%%%%%%%%%%%%%%%%%%%%%%%%%%%%%%%%%%%%%%%%%%%%%%%%%%%%%%%%%%%%%%%%%%%%%%%%%%%%
\subsection{Forwarding}
\label{sec:forward}

Different versions of the main or child documents
using compilation flags as described in \secref{sec:flags}
can be (permanently) stored in different files
for convenient compilation, viewing and distribution.
To this end, the package defines a command
to pass on compilation to a different file:

%%%%%%%%%%%%%%%%%%%%%%%%%%%%%%%%%%%%%%%%
\DescribeMacro{\childdocforward}
The command |\childdocforward| redirects processing to
another source file:
%
\begin{center}
\begin{tabular}{l}
|\input{childdoc.def}|\\
|\childdocforward[|\textit{main}|]{|\textit{dest}|}|\\
\end{tabular}
\end{center}
%
The argument \textit{dest} is the destination file
(without extension).
It should be the main file or one of the child files.
Note that further \textsf{childdoc} directives
such as |\childdocof| and |\childdocforward|
in the indicated file will be processed in this form.
The optional argument \textit{main}
passes on directly to the main file \textit{main}
while pretending to compile the child \textit{dest}.
This form behaves as if \textit{dest}
issues |\childdocof{|\textit{main}|}| right away,
and no further \textsf{childdoc} directives will be processed.

%%%%%%%%%%%%%%%%%%%%%%%%%%%%%%%%%%%%%%%%
\DescribeMacro{\...prefix}
In the alternative form |\childdocforwardprefix|,
%
\begin{center}
\begin{tabular}{l}
|\input{childdoc.def}|\\
|\childdocforwardprefix[|\textit{main}|]{|\textit{prefix}|}{|\textit{dest}|}|
\end{tabular}
\end{center}
%
the destination file is determined by a pattern
depending on the current file:
To make this work, the current file must be called
`{\textit{prefix}\hspace{0.2em}\textit{suffix}}'
with \textit{prefix} matching precisely the argument.
Processing is then passed on to the file
`{\textit{dest}\hspace{0.2em}\textit{suffix}}'.
Surely, the same effect is achieved by
directly specifying the
argument `{\textit{dest}\hspace{0.2em}\textit{suffix}}'
in the first form.
However, that requires to set up a different file
for each child. With the alternative form of the command
all these files can have exactly the same content
which simplifies setting them up and maintaining them.

For example, the following file |draft.tex|
with a compilation flag |\version| as described in \secref{sec:flags}
compiles the main document as a draft:
%
\begin{center}
\begin{tabular}{l}
|\def\version{draft}|\\
|\input{childdoc.def}|\\
|\childdocforward{|\textit{main}|}|
\end{tabular}
\end{center}
%
Likewise, the following files |final|\textit{nn}|.tex|
compile the final version of the child document
|child|\textit{nn}|.tex|:
%
\begin{center}
\begin{tabular}{l}
|\def\version{final}|\\
|\input{childdoc.def}|\\
|\childdocforwardprefix{final}{child}|
\end{tabular}
\end{center}
%

Note that when several versions of a main file and/or of each child file
are to be generated, it may be convenient to set up a |Makefile| or
shell script to automatise the process.

%%%%%%%%%%%%%%%%%%%%%%%%%%%%%%%%%%%%%%%%%%%%%%%%%%%%%%%%%%%%%%%%%%%%%%%%%%%%%%%%
\subsection{Command Line Processing}
\label{sec:commandline}

The effect of redirection files can also be achieved by invoking
the \LaTeX{} compiler with a more elaborate command line.
Most conveniently this should be done as part
of a shell script or a |Makefile|.

When using \textsf{childdoc} in the main file, the following
command lines effectively perform a redirection
(note that depending on the shell being used,
backslashes may have to be doubled: `|\|' $\to$ `|\\|'):
%
\begin{center}
|... -jobname "|\textit{target}|" |\\|"|[\textit{flags}]%
|\input{childdoc.def}\childdocforward[|\textit{main}|]{|\textit{dest}|}"|
\end{center}
%
Here \textit{target} is the name of the output file,
\textit{main} is the name of the main file
and \textit{dest} is the name of the main or child file to be processed
(all filenames without extensions).
The optional argument \textit{main} can be omitted
if \textit{main} matches \textit{dest}.
Optionally, compilation \textit{flags} can be defined via |\def| commands.
This command line makes the \TeX{} engine believe
it is compiling the file \textit{target}
whose content is specified as the latter parameter.
The provided code then forwards the processing to
\textit{main} or \textit{dest} as described in \secref{sec:forward}.

%%%%%%%%%%%%%%%%%%%%%%%%%%%%%%%%%%%%%%%%%%%%%%%%%%%%%%%%%%%%%%%%%%%%%%%%%%%%%%%%
\subsection{Include by Input}
\label{sec:input}

Including child documents by |\include| has some restrictions by design.
Most notably, the content of a child document always occupies
its own set of pages; pages cannot be shared between child documents.
Usually, this behaviour makes perfect sense
because each child document contain an essential part of the document.
However, in some situations it may be desirable to compose
a document from a collection of parts
without having mandatory page breaks between then.
For this case, the package
provides a mechanism to include parts
by |\input| which can also be processed individually.
However, by construction this mechanism
requires manual handling of the content to be output.

%%%%%%%%%%%%%%%%%%%%%%%%%%%%%%%%%%%%%%%%
\DescribeMacro{\ifchilddocmanual}
The main file should be prepared as usual, see \secref{sec:include}.
However, the document body must make a distinction
between processing of an individual part and of the main document, e.g.:
%
\begin{center}
\begin{tabular}{l}
|\ifchilddocmanual|\\
|\input{\childdocname}|\\
|\||else|\\
\textit{document body with }|\input{|\textit{part}|}|\\
|\||fi|
\end{tabular}
\end{center}
%
The conditional |\ifchilddocmanual| is true whenever
a part to be included by |\input| is being compiled,
and the name of the part is stored in |\childdocname|.

%%%%%%%%%%%%%%%%%%%%%%%%%%%%%%%%%%%%%%%%
\DescribeMacro{\childdocby}
Each part to be included by |\input| should start with:
%
\begin{center}
\begin{tabular}{l}
|\input{childdoc.def}|\\
|\childdocby{|\textit{main}|}|\\
\end{tabular}
\end{center}
%
The directive |\childdocby| is similar to |\childdocof|
described in \secref{sec:include},
but the subsequent selection of content must be done manually.
To that end, both |\ifchilddoc| and |\ifchilddocmanual|
will be true upon processing of a part,
and the name of the part is stored in |\childdocname|.
Note that |\jobname| will be set to the filename of the current part
so that each part receives an individual |.aux| file
that does not interfere with the |.aux| file(s) of the main document.
This behaviour can be altered by the alternative form
|\childdocby[*]{|\textit{main}|}| (with a non-empty optional argument)
which uses the |.aux| file of the main document
by setting |\jobname| to \textit{main}.

%%%%%%%%%%%%%%%%%%%%%%%%%%%%%%%%%%%%%%%%%%%%%%%%%%%%%%%%%%%%%%%%%%%%%%%%%%%%%%%%
\subsection{Driver Development}
\label{sec:driver}

The \textsf{childdoc} mechanism can also be use for the development
of definition files such as \LaTeX{} styles or classes.
This case differs from the above setup with multiple parts
included by |\include| in that no |\includeonly| should be invoked.
This can be achieved by starting the include file
(before |\ProvidesPackage|) with:
%
\begin{center}
\begin{tabular}{l}
|\input{childdoc.def}|\\
|\childdocforward{|\textit{main}|}|\\
\end{tabular}
\end{center}
%
or alternatively with:
%
\begin{center}
\begin{tabular}{l}
|\input{childdoc.def}|\\
|\childdocby{|\textit{main}|}|\\
\end{tabular}
\end{center}
%
Both forms have slightly different effects as described above.
The main file is prepared as usual, see \secref{sec:include}.

%%%%%%%%%%%%%%%%%%%%%%%%%%%%%%%%%%%%%%%%%%%%%%%%%%%%%%%%%%%%%%%%%%%%%%%%%%%%%%%%
\subsection{Legacy Detection}
\label{sec:detection}

The directive |\childdocmain| in the main file can detect
whether the complete document or merely a child is to be compiled
even without using the directive |\childdocof|.
This method is deprecated because it is less robust
and there is no compelling reason to use it;
it is merely provided for backward compatibility
and it may be removed in future versions.

If the detection mechanism is to be used,
it is mandatory to correctly specify
the filename of the main file as the argument of |\childdocmain|:
%
\begin{center}
\begin{tabular}{l}
|\input{childdoc.def}|\\
|\childdocmain{|\textit{main}|}|\\
\end{tabular}
\end{center}
%
If |\jobname| does not match the argument \textit{main} of |\childdocmain|,
it is assumed that |\jobname| points to the child file to be compiled.
When using |\childdocmain| with the main file specified as argument,
it suffices to start a child file
with just |\input{|\textit{main}|}|
without loading of the package and using |\childdocof|.
If instead all processing is done
with the appropriate \textsf{childdoc} directives,
the argument of \textit{main} of |\childdocmain| can be empty.

An alternative version of the command line processing described
in \secref{sec:commandline} using the detection mechanism reads:
%
\begin{center}
|... -jobname "|\textit{target}|" "|[\textit{flags}]%
[|\def\jobname{|\textit{dest}|}|]|\input{|\textit{main}|}"|
\end{center}

%%%%%%%%%%%%%%%%%%%%%%%%%%%%%%%%%%%%%%%%%%%%%%%%%%%%%%%%%%%%%%%%%%%%%%%%%%%%%%%%
\subsection{Manual Code}
\label{sec:manual}

In case one cannot be certain whether the definitions file |childdoc.def|
is installed on the target \TeX{} distribution
and one prefers not to ship it,
it is conceivable to paste a few relevant commands into the sources.

To that end, drop all statements |\input{childdoc.def}|
and perform the replacements as outlined below.
Instead of |\childdocmain{|\textit{main}|}| add the following code
to the top of the main file:
%
\begin{center}
\begin{tabular}{l}
|\||ifdefined\childdocname\endinput\||fi\newif\ifchilddoc|\\
|\edef\childdocname{\scantokens\expandafter{\jobname\noexpand}}|\\
|\def\childdocmain{|\textit{main}|}\||ifx\childdocmain\childdocname\||else|\\
|\childdoctrue\includeonly{\childdocname}\let\jobname\childdocmain\||fi|\\
\end{tabular}
\end{center}
%
Instead of |\childdocof{|\textit{main}|}| just include the main file
at the top of each child file:
%
\begin{center}
|\input{|\textit{main}|}|
\end{center}
%
A simple redirection |\childdocforward{|\textit{dest}|}| is achieved by:
%
\begin{center}
|\def\jobname{|\textit{dest}|}\input{\jobname}|
\end{center}
%
The redirection with prefix
|\childdocforwardprefix[|\textit{prefix}|]{|\textit{dest}|}|
is accomplished by:
%
\begin{center}
\begin{tabular}{l}
|{\edef\jobname{\scantokens\expandafter{\jobname\noexpand}}|\\
|\def\redirectjob |\textit{prefix}|#1~~~{\gdef\jobname{|\textit{dest}|#1}}|\\
|\expandafter\redirectjob\jobname~~~}\input{\jobname}|
\end{tabular}
\end{center}

In an alternative approach,
child documents can be compiled by a specific command line
without additional code or specific definitions:
%
\begin{center}
|... -jobname "|\textit{target}|" "|[\textit{flags}]%
|\includeonly{|\textit{dest}|}\input{|\textit{main}|}"|
\end{center}
%

%%%%%%%%%%%%%%%%%%%%%%%%%%%%%%%%%%%%%%%%%%%%%%%%%%%%%%%%%%%%%%%%%%%%%%%%%%%%%%%%
%%%%%%%%%%%%%%%%%%%%%%%%%%%%%%%%%%%%%%%%%%%%%%%%%%%%%%%%%%%%%%%%%%%%%%%%%%%%%%%%
\section{Information}

%%%%%%%%%%%%%%%%%%%%%%%%%%%%%%%%%%%%%%%%%%%%%%%%%%%%%%%%%%%%%%%%%%%%%%%%%%%%%%%%
\subsection{Copyright}

Copyright \copyright{} 2017--2018 Niklas Beisert

This work may be distributed and/or modified under the
conditions of the \LaTeX{} Project Public License, either version 1.3
of this license or (at your option) any later version.
The latest version of this license is in
  \url{http://www.latex-project.org/lppl.txt}
and version 1.3 or later is part of all distributions of \LaTeX{}
version 2005/12/01 or later.

This work has the LPPL maintenance status `maintained'.

The Current Maintainer of this work is Niklas Beisert.

This work consists of the files |README.txt|, |childdoc.ins| and |childdoc.dtx|
as well as the derived files |childdoc.def|, |cdocsamp.tex|
with |cdocsch1.tex|, |cdocsch2.tex|, |cdocspt3.tex|, |cdocspt4.tex|,
|cdocsdrf.tex|, |cdocsfn1.tex|, |cdocsfn2.tex|
as well as |childdoc.pdf|.

%%%%%%%%%%%%%%%%%%%%%%%%%%%%%%%%%%%%%%%%%%%%%%%%%%%%%%%%%%%%%%%%%%%%%%%%%%%%%%%%
\subsection{Files and Installation}

The package consists of the files:
%
\begin{center}
\begin{tabular}{ll}
    |README.txt|   & readme file \\
    |childdoc.ins| & installation file \\
    |childdoc.dtx| & source file \\
    |childdoc.def| & definition file \\
    |cdocsamp.tex| & sample main file \\
    |cdocsch1.tex| & sample include file \\
    |cdocsch2.tex| & sample include file \\
    |cdocspt3.tex| & sample part file \\
    |cdocspt4.tex| & sample part file \\
    |cdocsdrf.tex| & sample redirection file \\
    |cdocsfn1.tex| & sample redirection file \\
    |cdocsfn2.tex| & sample redirection file \\
    |childdoc.pdf| & manual
\end{tabular}
\end{center}
%
The distribution consists of the files
|README.txt|, |childdoc.ins| and |childdoc.dtx|.
%
\begin{itemize}
\item
Run (pdf)\LaTeX{} on |childdoc.dtx|
to compile the manual |childdoc.pdf| (this file).
\item
Run \LaTeX{} on |childdoc.ins| to create the definitions file |childdoc.def|
and the sample |cdocsamp.tex| with include files
|cdocsch1.tex|, |cdocsch2.tex|, |cdocspt3.tex|, |cdocspt4.tex|,
|cdocsdrf.tex|, |cdocsfn1.tex|, |cdocsfn2.tex|.
Then copy the file |childdoc.def| to an appropriate directory of your \LaTeX{}
distribution, e.g.\ \textit{texmf-root}|/tex/latex/childdoc|.
\end{itemize}

%%%%%%%%%%%%%%%%%%%%%%%%%%%%%%%%%%%%%%%%%%%%%%%%%%%%%%%%%%%%%%%%%%%%%%%%%%%%%%%%
\subsection{Related CTAN Packages}

There are several other packages which offer a similar functionality:
%
\begin{itemize}
\item
The packages
\href{http://ctan.org/pkg/docmute}{\textsf{docmute}},
\href{http://ctan.org/pkg/includex}{\textsf{includex}} and
\href{http://ctan.org/pkg/standalone}{\textsf{standalone}}
provide commands to include only the document body of
a child file thus allowing both files to be compiled individually.
\item
The packages \href{http://ctan.org/pkg/subdocs}{\textsf{subdocs}}
and \href{http://ctan.org/pkg/subfiles}{\textsf{subfiles}}
provide structures in which the main and child documents can be
encapsulated and allowing them to be compiled individually.
The inclusion mechanism is different from the conventional |\include|.
\item
The package \href{http://ctan.org/pkg/combine}{\textsf{combine}}
is an elaborate solution to combine several documents into one.
\end{itemize}
%
See also the CTAN topic \href{http://ctan.org/topic/subdocs}{\textsf{subdocs}}
for further related packages.
The present package differs from the above solutions in that
a document structure constructed with the conventional |\include| mechanism
just needs two extra commands at the top of every file
such that all constituent files can be compiled individually.

%%%%%%%%%%%%%%%%%%%%%%%%%%%%%%%%%%%%%%%%%%%%%%%%%%%%%%%%%%%%%%%%%%%%%%%%%%%%%%%%
%\subsection{Feature Suggestions}
%
%The following is a list of features which may be useful for future
%versions of this package:
%%
%\begin{itemize}
%\item
%\ldots
%\end{itemize}

%%%%%%%%%%%%%%%%%%%%%%%%%%%%%%%%%%%%%%%%%%%%%%%%%%%%%%%%%%%%%%%%%%%%%%%%%%%%%%%%
\subsection{Revision History}

%%%%%%%%%%%%%%%%%%%%%%%%%%%%%%%%%%%%%%%%
\paragraph{v2.0:} 2018/12/30

\begin{itemize}
\item
immediate forward processing
\item
added |\childdocby| mechanism
\item
manual restructured
\end{itemize}

%%%%%%%%%%%%%%%%%%%%%%%%%%%%%%%%%%%%%%%%
\paragraph{v1.6:} 2018/01/17

\begin{itemize}
\item
application for development of include files
\item
corrections to manual
\end{itemize}

%%%%%%%%%%%%%%%%%%%%%%%%%%%%%%%%%%%%%%%%
\paragraph{v1.5:} 2017/05/21

\begin{itemize}
\item
more complete structuring introduced
\item
|\childdocof| introduced
\item
|\childdoc| renamed to |\childdocmain|
\item
|\childredirect| renamed to |\childdocforward| and |\childdocforwardprefix|
and functionality expanded
\end{itemize}

%%%%%%%%%%%%%%%%%%%%%%%%%%%%%%%%%%%%%%%%
\paragraph{v1.0:} 2017/04/27

\begin{itemize}
\item
manual and install package
\item
first version published on CTAN
\end{itemize}

%%%%%%%%%%%%%%%%%%%%%%%%%%%%%%%%%%%%%%%%
\paragraph{v0.6:} 2017/04/26

\begin{itemize}
\item
redirection mechanism added
\end{itemize}

%%%%%%%%%%%%%%%%%%%%%%%%%%%%%%%%%%%%%%%%
\paragraph{v0.5:} 2017/04/26

\begin{itemize}
\item
functionality in definition file
\end{itemize}


%%%%%%%%%%%%%%%%%%%%%%%%%%%%%%%%%%%%%%%%%%%%%%%%%%%%%%%%%%%%%%%%%%%%%%%%%%%%%%%%
%%%%%%%%%%%%%%%%%%%%%%%%%%%%%%%%%%%%%%%%%%%%%%%%%%%%%%%%%%%%%%%%%%%%%%%%%%%%%%%%
%%%%%%%%%%%%%%%%%%%%%%%%%%%%%%%%%%%%%%%%%%%%%%%%%%%%%%%%%%%%%%%%%%%%%%%%%%%%%%%%
\appendix

\settowidth\MacroIndent{\rmfamily\scriptsize 000\ }

 \DocInput{childdoc.dtx}

\end{document}
%</driver>
% \fi
%
% %%%%%%%%%%%%%%%%%%%%%%%%%%%%%%%%%%%%%%%%%%%%%%%%%%%%%%%%%%%%%%%%%%%%%%%%%%%%%%
% %%%%%%%%%%%%%%%%%%%%%%%%%%%%%%%%%%%%%%%%%%%%%%%%%%%%%%%%%%%%%%%%%%%%%%%%%%%%%%
% \section{Sample}
%\iffalse
%<*samplemain>
%\fi
%
% The following presents a sample document
% with two chapters, two parts, a title page,
% a compile flag as well as three forwarding files to set the flag.
% It consists of eight |.tex| files:
% \begin{center}
% \begin{tabular}{ll}
% |cdocsamp.tex|&main file\\
% |cdocsch1.tex|&include file for chapter 1\\
% |cdocsch2.tex|&include file for chapter 2\\
% |cdocspt3.tex|&include file for part 3\\
% |cdocspt4.tex|&include file for part 4\\
% |cdocsdrf.tex|&forwarding file for main file in draft mode\\
% |cdocsfi1.tex|&forwarding file for final version of chapter 1\\
% |cdocsfi2.tex|&forwarding file for final version of chapter 2\\
% \end{tabular}
% \end{center}
% Each of the eight files can be compiled directly by the \LaTeX{} compiler.
%
% %%%%%%%%%%%%%%%%%%%%%%%%%%%%%%%%%%%%%%
% \paragraph{Main File.}
%
% The main file is called |cdocsamp.tex|.
%
% Load the \textsf{childdoc} definitions and
% declare the filename for the main document:
%    \begin{macrocode}
\input{childdoc.def}
\childdocmain{}
%    \end{macrocode}

% Optional override for |\version| flag:
%    \begin{macrocode}
%%\ifchilddoc\else\providecommand{\version}{draft}\fi
%    \end{macrocode}

% Define the default values for the |\version| flag
% (|final| for the main file and |draft| for childs):
%    \begin{macrocode}
\ifchilddoc
\providecommand{\version}{draft}
\else
\providecommand{\version}{final}
\fi
%    \end{macrocode}

% Load the standard document class:
%    \begin{macrocode}
\documentclass[12pt]{article}
%    \end{macrocode}

% Start the document body:
%    \begin{macrocode}
\begin{document}
%    \end{macrocode}

% Declare a title page.
% Print title, part of document being processed and version flag:
%    \begin{macrocode}
\addtocounter{page}{-1}
\begin{center}
{\LARGE\bfseries{}childdoc example\par}
\vspace{1cm}
\ifchilddoc
\ifchilddocmanual part\else chapter\fi:
`\childdocname' of `\childdocjob'\par
\else
main document: `\childdocjob'\par
\fi
version: \version\par
\end{center}
\newpage
%    \end{macrocode}

% Manually include selected file,
% otherwise process as usual:
%    \begin{macrocode}
\ifchilddocmanual
\section*{part `\childdocname'}
\input{\childdocname}
\else
%    \end{macrocode}

% Include the two chapters:
%    \begin{macrocode}
\include{cdocsch1}
\include{cdocsch2}
%    \end{macrocode}

% Include the two parts unless only chapters should be displayed:
%    \begin{macrocode}
\ifchilddoc\else
\section{part three}
\input{cdocspt3}
\section{part four}
\input{cdocspt4}
\fi
%    \end{macrocode}

% Process as usual until here:
%    \begin{macrocode}
\fi
%    \end{macrocode}

% End of document body:
%    \begin{macrocode}
\end{document}
%    \end{macrocode}
%\iffalse
%</samplemain>
%\fi
%
% %%%%%%%%%%%%%%%%%%%%%%%%%%%%%%%%%%%%%%
% \paragraph{Chapter Include Files.}
%
% The include files are called |cdocsch1.tex| and |cdocsch2.tex|.
%
%\iffalse
%<*samplechap1|samplechap2>
%\fi

% Optional override for |\version| flag:
%    \begin{macrocode}
%%\providecommand{\version}{final}
%    \end{macrocode}

% Include the main document:
%    \begin{macrocode}
\input{childdoc.def}
\childdocof{cdocsamp}
%    \end{macrocode}

%\iffalse
%</samplechap1|samplechap2>
%\fi
%
%\iffalse
%<*samplechap1>
%\fi
% Some text for chapter 1:
%    \begin{macrocode}
\section{one}
some text in chapter one
%    \end{macrocode}

%\iffalse
%</samplechap1>
%\fi
% Some text for chapter 2:
%\iffalse
%<*samplechap2>
%\fi
%    \begin{macrocode}
\section{two}
more text in chapter two
%    \end{macrocode}

%\iffalse
%</samplechap2>
%\fi
%
% %%%%%%%%%%%%%%%%%%%%%%%%%%%%%%%%%%%%%%
% \paragraph{Part Include Files.}
%
% The include files are called |cdocspt3.tex| and |cdocspt4.tex|.
%
%\iffalse
%<*samplepart3|samplepart4>
%\fi

% Optional override for |\version| flag:
%    \begin{macrocode}
%%\providecommand{\version}{final}
%    \end{macrocode}

% Include the main document:
%    \begin{macrocode}
\input{childdoc.def}
\childdocby{cdocsamp}
%    \end{macrocode}

%\iffalse
%</samplepart3|samplepart4>
%\fi
%
%\iffalse
%<*samplepart3>
%\fi
% Some text for part 3:
%    \begin{macrocode}
some text in part three
%    \end{macrocode}

%\iffalse
%</samplepart3>
%\fi
% Some text for part 4:
%\iffalse
%<*samplepart4>
%\fi
%    \begin{macrocode}
more text in part four
%    \end{macrocode}

%\iffalse
%</samplepart4>
%\fi
%
% %%%%%%%%%%%%%%%%%%%%%%%%%%%%%%%%%%%%%%
% \paragraph{Forwarding for a Complete Draft.}
%
% The following forwarding file |cdocsdrf.tex|
% compiles the main document in draft mode:
%\iffalse
%<*sampledraft>
%\fi
%    \begin{macrocode}
\def\version{draft}
\input{childdoc.def}
\childdocforward{cdocsamp}
%    \end{macrocode}

%\iffalse
%</sampledraft>
%\fi
%
% %%%%%%%%%%%%%%%%%%%%%%%%%%%%%%%%%%%%%%
% \paragraph{Forwarding for Final Version of the Chapters.}
%
% The following forwarding files |cdocsfn1.tex| and |cdocsfn2.tex|
% (with identical content)
% compile the final versions of the child documents
% |cdocsch1.tex| and |cdocsch2.tex|, respectively:
%\iffalse
%<*samplefinal>
%\fi
%    \begin{macrocode}
\def\version{final}
\input{childdoc.def}
\childdocforwardprefix[cdocsamp]{cdocsfn}{cdocsch}
%    \end{macrocode}

%\iffalse
%</samplefinal>
%\fi
%
% %%%%%%%%%%%%%%%%%%%%%%%%%%%%%%%%%%%%%%
% \paragraph{Command Line Processing.}
%
% The following three command lines generate the output files
% |cdocscld|, |cdocscl1| and |cdocscl2|
% which should be identical to
% |cdocsdrf|, |cdocsch1| and |cdocsfn2|, respectively:
% \begin{center}
% \begin{tabular}{l}
% |latex -jobname cdocscld \|\\
% |  "\def\version{draft}\input{childdoc.def}\childdocforward{cdocsamp}"|\\
% |latex -jobname cdocscl1 \|\\
% |  "\input{childdoc.def}\childdocforward[cdocsamp]{cdocsch1}"|\\
% |latex -jobname cdocscl2 \|\\
% |  "\def\version{final}\input{childdoc.def}\childdocforward{cdocsch2}"|
% \end{tabular}
% \end{center}
% Note that the trailing backslash on each first line
% merely continues the input to the second line
% (for convenient cut ant paste).
% Furthermore, the command |latex| can be replaced by any
% of its alternative versions such as |pdflatex|.
%
% %%%%%%%%%%%%%%%%%%%%%%%%%%%%%%%%%%%%%%%%%%%%%%%%%%%%%%%%%%%%%%%%%%%%%%%%%%%%%%
% %%%%%%%%%%%%%%%%%%%%%%%%%%%%%%%%%%%%%%%%%%%%%%%%%%%%%%%%%%%%%%%%%%%%%%%%%%%%%%
% \section{Implementation}
%\iffalse
%<*package>
%\fi
%
% This section describes the definitions file |childdoc.def|.

% The definitions cannot be loaded using |\usepackage| or |\RequirePackage|
% which has a mechanism to prevent loading a style file more than once.
% When loading the definitions by means of |\input|
% multiple instances have to be prevented manually:
%\iffalse
%This code needs to be before the `\ProvidesFile' directive
%which is defined at the beginning of this file.
%Therefore it is also placed there and commented out here.
%</package>
%<*discard>
%\fi
%    \begin{macrocode}
\ifdefined\childdocmain\endinput\fi
%    \end{macrocode}
%\iffalse
%</discard>
%<*package>
%\fi
%
% \macro{\ifchilddoc}
% \macro{\ifchilddocmanual}
% The conditional |\ifchilddoc| tells whether a
% child (true) or main (false) document is being compiled.
% The conditional |\ifchilddocmanual| tells whether
% the |\includeonly| mechanism is used (false) or
% the selection of child files must be performed manually (true).
% The definitions initialise to false:
%    \begin{macrocode}
\newif\ifchilddoc
\newif\ifchilddocmanual
%    \end{macrocode}

% \macro{\childdocname}
% \macro{\childdocjob}
% The macro |\childdocname| stores the name of the main document
% to be compiled. The macro |\childdocjob| stores the name of
% the document on which the \LaTeX{} compiler was originally invoked.
% The content of |\jobname| cannot be compared
% to filenames specified in the source due to different catcodes.
% The following code rescans |\jobname|, stores the result
% in |\childdocname| and saves a copy in |\childdocjob|:
%    \begin{macrocode}
\edef\childdocname{\scantokens\expandafter{\jobname\noexpand}}
\let\childdocjob\childdocname
%    \end{macrocode}

% \macro{\childdocdisable}
% The macro |\childdocdisable| prevents the main file
% from being processed more than once.
% At this stage, the main document command |\childdocmain|
% is assumed to be called once again where it should do nothing.
% Any subsequent call to it should prevent
% a secondary processing of the main document
% It overwrites the forwarding commands
% |\childdocof| and |\childdocforward|
% with empty macros to prevent further inclusions of the main document:
%    \begin{macrocode}
\newcommand{\childdocdisable}
{
  \renewcommand{\childdocmain}[1]{\renewcommand{\childdocmain}[1]{\endinput}}
  \renewcommand{\childdocof}[1]{}
  \renewcommand{\childdocby}[2][]{}
  \renewcommand{\childdocforward}[2][]{}
  \renewcommand{\childdocdisable}{}
}
%    \end{macrocode}

% \macro{\childdocmain}
% The macro |\childdocmain| is to be called at the top of the main file
% with nothing or the main filename (without extension) as argument.
% First, it breaks loops.
% If the argument is not empty and does not match |\childdocname|
% (which is set by the first inclusion of |childdoc.def|),
% |\ifchilddoc| is set to true, |\includeonly| is applied to the child file
% and |\jobname| is set to the main file
% (for proper handling of |.aux| files):
%    \begin{macrocode}
\newcommand{\childdocmain}[1]
{
  \childdocdisable\childdocmain{}
  \if?#1?\else
    \begingroup
      \def\childdoctmp{#1}
      \ifx\childdoctmp\childdocname
        \def\childdoctmp{}
      \else
        \def\childdoctmp
        {
          \childdoctrue
          \includeonly{\childdocname}
          \def\childdocjob{#1}
          \def\jobname{#1}
        }
      \fi
      \expandafter
    \endgroup
    \childdoctmp
  \fi
}
%    \end{macrocode}

% \macro{\childdocof}
% The command |\childdocof| redirects
% compilation to the main file |#1|.
%    \begin{macrocode}
\newcommand{\childdocof}[1]
{
  \childdocdisable
  \childdoctrue
  \includeonly{\childdocname}
  \def\jobname{#1}
  \def\childdocjob{#1}
  \input{#1}
}
%    \end{macrocode}

% \macro{\childdocby}
% The command |\childdocby| ....
%    \begin{macrocode}
\newcommand{\childdocby}[2][]
{
  \childdocdisable
  \childdoctrue
  \childdocmanualtrue
  \if?#1?\else
    \def\jobname{#2}
  \fi
  \def\childdocjob{#2}
  \input{#2}
  \endinput
}
%    \end{macrocode}

% \macro{\childdocforward}
% The command |\childdocforward| redirects
% compilation to the main file or
% (if the optional argument is given) a child file.
% Parameters are set as if the main file
% or a child file starting with |\childdocof| was compiled.
% Then compilation is handed over to the main file:
%    \begin{macrocode}
\newcommand{\childdocforward}[2][]
{
  \begingroup
    \if?#1?
      \def\childdoctmp
      {
        \def\childdocname{#2}
        \def\childdocjob{#2}
        \def\jobname{#2}
        \input{#2}
        \endinput
      }
    \else
      \def\childdoctmp
      {
        \childdocdisable
        \def\childdocname{#2}
        \childdoctrue
        \includeonly{#2}
        \def\childdocjob{#1}
        \def\jobname{#1}
        \input{#1}
        \endinput
      }
    \fi
    \expandafter
  \endgroup
  \childdoctmp
}
%    \end{macrocode}

% \macro{\childdocforwardprefix}
% The command |\childdocforwardprefix| redirects
% compilation to the main or a child file by means of a pattern.
% The prefix |#1| in the current filename is replaced by |#2|
% and the suffix of the current filename is kept
% (it is assumed that the filename does not contain the substring `|~~~|'
% which is used as a delimiter).
% Compilation is handed over to the new file by |\childdocforward|:
%    \begin{macrocode}
\newcommand{\childdocforwardprefix}[3][]
{
  \begingroup
    \def\childdocextract #2##1~~~{\def\childdoctmp{\childdocforward[#1]{#3##1}}}
    \expandafter\childdocextract\childdocname~~~
    \expandafter
  \endgroup
  \childdoctmp
}
%    \end{macrocode}

% \macro{\childdoc}
% The deprecated macro |\childdoc| is a legacy version of |\childdocmain|:
%    \begin{macrocode}
\newcommand{\childdoc}{\childdocmain}
%    \end{macrocode}

% \macro{\childdocredirect}
% The deprecated macro |\childdocredirect| is a legacy version
% of |\childdocforward| and |\childdocforwardprefix|:
%    \begin{macrocode}
\newcommand{\childdocredirect}[2][]
{
  \begingroup
    \if?#1?
      \def\childdoctmp{\childdocforward{#2}}
    \else
      \def\childdoctmp{\childdocforwardprefix{#1}{#2}}
    \fi
    \expandafter
  \endgroup
  \childdoctmp
}
%    \end{macrocode}

%\iffalse
%</package>
%\fi
%
\endinput
|\\
|\childdocforwardprefix{final}{child}|
\end{tabular}
\end{center}
%

Note that when several versions of a main file and/or of each child file
are to be generated, it may be convenient to set up a |Makefile| or
shell script to automatise the process.

%%%%%%%%%%%%%%%%%%%%%%%%%%%%%%%%%%%%%%%%%%%%%%%%%%%%%%%%%%%%%%%%%%%%%%%%%%%%%%%%
\subsection{Command Line Processing}
\label{sec:commandline}

The effect of redirection files can also be achieved by invoking
the \LaTeX{} compiler with a more elaborate command line.
Most conveniently this should be done as part
of a shell script or a |Makefile|.

When using \textsf{childdoc} in the main file, the following
command lines effectively perform a redirection
(note that depending on the shell being used,
backslashes may have to be doubled: `|\|' $\to$ `|\\|'):
%
\begin{center}
|... -jobname "|\textit{target}|" |\\|"|[\textit{flags}]%
|% \iffalse
%
% childdoc.dtx Copyright (C) 2017-2018 Niklas Beisert
%
% This work may be distributed and/or modified under the
% conditions of the LaTeX Project Public License, either version 1.3
% of this license or (at your option) any later version.
% The latest version of this license is in
%   http://www.latex-project.org/lppl.txt
% and version 1.3 or later is part of all distributions of LaTeX
% version 2005/12/01 or later.
%
% This work has the LPPL maintenance status `maintained'.
%
% The Current Maintainer of this work is Niklas Beisert.
%
% This work consists of the files childdoc.dtx and childdoc.ins
% and the derived files childdoc.def and cdocsamp.tex with
% cdocsch1.tex, cdocsch2.tex, cdocsdrf.tex, cdocsfn1.tex, cdocsfn2.tex.
%
%<package>\ifdefined\childdocmain\endinput\fi
%<package>\ProvidesFile{childdoc.def}[2018/12/30 v2.0 child document driver]
%<samplemain>\ProvidesFile{cdocsamp.tex}[2018/12/30 v2.0 sample for childdoc]
%<*driver>
%\ProvidesFile{childdoc.drv}[2018/12/30 v2.0 childdoc reference manual file]
\PassOptionsToClass{10pt,a4paper}{article}
\documentclass{ltxdoc}

\usepackage[margin=35mm]{geometry}
\usepackage{hyperref}
\usepackage{hyperxmp}
\usepackage[usenames]{color}

\hypersetup{colorlinks=true}
\hypersetup{pdfstartview=FitH}
\hypersetup{pdfpagemode=UseNone}
\hypersetup{pdfsource={}}
\hypersetup{pdflang={en-UK}}
\hypersetup{pdfcopyright={Copyright 2017-2018 Niklas Beisert.
  This work may be distributed and/or modified under the
  conditions of the LaTeX Project Public License, either version 1.3
  of this license or (at your option) any later version.}}
\hypersetup{pdflicenseurl={http://www.latex-project.org/lppl.txt}}
\hypersetup{pdfcontactaddress={ETH Zurich, ITP, HIT K,
  Wolfgang-Pauli-Strasse 27}}
\hypersetup{pdfcontactpostcode={8093}}
\hypersetup{pdfcontactcity={Zurich}}
\hypersetup{pdfcontactcountry={Switzerland}}
\hypersetup{pdfcontactemail={nbeisert@itp.phys.ethz.ch}}
\hypersetup{pdfcontacturl={http://people.phys.ethz.ch/\xmptilde nbeisert/}}

\newcommand{\secref}[1]{\hyperref[#1]{section \ref*{#1}}}

\parskip1ex
\parindent0pt
\let\olditemize\itemize
\def\itemize{\olditemize\parskip0pt}

\begin{document}

\title{The \textsf{childdoc} Package}
\hypersetup{pdftitle={The childdoc Package}}
\author{Niklas Beisert\\[2ex]
  Institut f\"ur Theoretische Physik\\
  Eidgen\"ossische Technische Hochschule Z\"urich\\
  Wolfgang-Pauli-Strasse 27, 8093 Z\"urich, Switzerland\\[1ex]
  \href{mailto:nbeisert@itp.phys.ethz.ch}
  {\texttt{nbeisert@itp.phys.ethz.ch}}}
\hypersetup{pdfauthor={Niklas Beisert}}
\hypersetup{pdfsubject={Manual for the LaTeX2e Package childdoc}}
\date{30 December 2018, \textsf{v2.0}}
\maketitle

\begin{abstract}\noindent
\textsf{childdoc} is a \LaTeXe{} package
that enables the direct compilation
of document sections included by |\include|
to individual files.
\end{abstract}

\begingroup
\parskip0ex
\tableofcontents
\endgroup

%%%%%%%%%%%%%%%%%%%%%%%%%%%%%%%%%%%%%%%%%%%%%%%%%%%%%%%%%%%%%%%%%%%%%%%%%%%%%%%%
%%%%%%%%%%%%%%%%%%%%%%%%%%%%%%%%%%%%%%%%%%%%%%%%%%%%%%%%%%%%%%%%%%%%%%%%%%%%%%%%
\section{Introduction}

\LaTeX{} provides a mechanism to structure a large document (such as a book)
into a main file and several child files (containing the chapters)
using the |\include| command.
This mechanism is beneficial for documents
which span hundreds of pages in order to
make the source file(s) more manageable.
Moreover, compilation can be restricted to
selected child files by means of the |\includeonly| command.
The latter feature can be used to reduce the compilation time while editing
(this was significantly more useful in the earlier days of \LaTeX{})
or to generate a smaller document which is easier to navigate.
Another application of |\includeonly| is to generate
documents consisting of selected parts of the complete document.

However, there are a few drawbacks of the plain |\include| mechanism:
\begin{itemize}
\item
The child files cannot be compiled on their own,
they can only be compiled via the main file.
A naive editing environment
(such as a text editor with an option
to have the current file processed by \LaTeX)
may require one to switch to the main file before compiling;
attempting to compile the child file produces errors.
\item
The main file must be modified (each time)
to adjust the |\includeonly| command
to the present needs. This easily leaves the main file in a messy state.
\item
The generated document will always carry the filename
of the main document. This is inconvenient if
several child files are to be compiled and
to be kept for distribution.
\end{itemize}

The present package provides a simple interface
to make child files individually compilable by \LaTeX{}.
Compiling a child file then has the same effect as compiling
the main file with an |\includeonly| command
to select the appropriate child.
Moreover the generated document will carry the name of the child
rather than the main file.
This resolves all three above issues.

This feature is meant to make the editing of books,
thesis documents and lecture notes somewhat more convenient.
However, the package can also be used efficiently for
composing a series of documents (such as exercise sheets)
which are typically distributed individually.
It then assists the author in generating the individual documents
(potentially in different versions)
as well as a document containing the collected series.
Another application is in developing style files
or other kinds of included material
where compilation of the style file could redirect
to a sample or test file.

%%%%%%%%%%%%%%%%%%%%%%%%%%%%%%%%%%%%%%%%%%%%%%%%%%%%%%%%%%%%%%%%%%%%%%%%%%%%%%%%
%%%%%%%%%%%%%%%%%%%%%%%%%%%%%%%%%%%%%%%%%%%%%%%%%%%%%%%%%%%%%%%%%%%%%%%%%%%%%%%%
\section{Usage}

First of all, the package \textsf{childdoc} is \emph{not} a standard
\LaTeXe{} |.sty| style file! Therefore it needs to be invoked in
a non-standard way.

%%%%%%%%%%%%%%%%%%%%%%%%%%%%%%%%%%%%%%%%%%%%%%%%%%%%%%%%%%%%%%%%%%%%%%%%%%%%%%%%
\subsection{Included Files}
\label{sec:include}

%%%%%%%%%%%%%%%%%%%%%%%%%%%%%%%%%%%%%%%%
\DescribeMacro{\childdocmain}
To use the package, add the commands
\begin{center}
\begin{tabular}{l}
|\input{childdoc.def}|\\
|\childdocmain{}|\\
\end{tabular}
\end{center}
at the very top of the main \LaTeX{} file,
in particular \emph{before} the |\documentclass| statement!
The argument of |\childdocmain| should be left empty
(but it must be present).

%%%%%%%%%%%%%%%%%%%%%%%%%%%%%%%%%%%%%%%%
\DescribeMacro{\childdocof}
Furthermore, add the commands
\begin{center}
\begin{tabular}{l}
|\input{childdoc.def}|\\
|\childdocof{|\textit{main}|}|\\
\end{tabular}
\end{center}
at the top of every child file \textit{child}
which is included by |\include{|\textit{child}|}|
from within the main file
(or at least for those files to be compiled individually).
The argument \textit{main} must be the filename of the main file.

There are a couple of
considerations in setting up the main and child documents:

%%%%%%%%%%%%%%%%%%%%%%%%%%%%%%%%%%%%%%%%
\paragraph{Restrictions.}

Please note the following restrictions:
\begin{itemize}
\item
|\childdocmain| must be called with one argument \textit{main}
to ensure compatibility with earlier version of the package.
It must either be empty (|\childdocmain{}|)
or precisely match the filename of the main file in which it is specified.
See \secref{sec:detection} for further information.
\item
The filename \textit{main} must be specified without the |.tex| extension.
\item
The filename \textit{main} is case sensitive
(even in case-insensitive file systems)
due to internal string comparison.
\item
The argument \textit{main} should be fully expanded, it cannot be a macro.
\item
Subdirectories and special characters should be avoided in filenames.
\item
The command |\childdocmain{|\textit{main}|}| must be followed by a whitespace.
It should not be followed immediately by another command
or by a comment mark `|%|'.
This is because the \TeX{} parser reads the token immediately following
the argument of |\childdocmain| and puts it
at the beginning of every child section;
however, a white\-space is ignored.
\end{itemize}

%%%%%%%%%%%%%%%%%%%%%%%%%%%%%%%%%%%%%%%%
\paragraph{Content of Main File.}

It is advisable to place all content in the child files included by |\include|.
Any output contained in the main file will appear in all child documents
unless suppressed manually;
it cannot be suppressed automatically by the |\includeonly| directive
and thus should normally be avoided.
A method to include some content in the main file
by means of conditional processing is described in \secref{sec:conditional}.

%%%%%%%%%%%%%%%%%%%%%%%%%%%%%%%%%%%%%%%%
\paragraph{Page Numbering.}

When only a part of the document is compiled,
the appropriate numbering of pages
(as well as other status parameters)
is determined from the |.aux| files.
The latter contain information from previous passes.
However this information needs to propagate through
all intermediate child documents.
Therefore the page numbering in child documents may well
be inconsistent until the complete document is compiled at least once.

A useful (if unconventional) way to always ensure a consistent
page numbering is to restart the numbering in each child document
and denote the pages by `\textit{child}|.|\textit{page}'
where \textit{child} represents the chapter/section number of the child file.
This can be achieved by the command
|\numberwithin{page}{|\textit{child}|}|
of the \textsf{amsmath} package
where \textit{child} can be |chapter| or |section|
depending on the chosen structuring.
Alternatively, one can modify the macro |\thepage| appropriately
and reset the counter |page| at the start of each child file.

%%%%%%%%%%%%%%%%%%%%%%%%%%%%%%%%%%%%%%%%%%%%%%%%%%%%%%%%%%%%%%%%%%%%%%%%%%%%%%%%
\subsection{Conditional Processing}
\label{sec:conditional}

The package provides a mechanism to compile different versions
of a document. To customise the versions further some conditional processing
can come in handy to distinguish which version is being compiled.
The package provides two macros to describe the compilation context:

%%%%%%%%%%%%%%%%%%%%%%%%%%%%%%%%%%%%%%%%
\DescribeMacro{\ifchilddoc}
The conditional |\ifchilddoc| distinguishes between the compilation of
child documents and the main document:
%
\begin{center}
|\ifchilddoc |\textit{child-code}| |[|\||else |\textit{main-code}]| \||fi|
\end{center}

%%%%%%%%%%%%%%%%%%%%%%%%%%%%%%%%%%%%%%%%
\DescribeMacro{\childdocname}
\DescribeMacro{\childdocjob}
The macro |\childdocname| contains the filename (without extension)
of the main or child file being processed.
Note that |\childdocjob| will always contain the name of the main file.

%%%%%%%%%%%%%%%%%%%%%%%%%%%%%%%%%%%%%%%%
\paragraph{Title Page.}

Conditional processing can be used to include a title or banner page
in the main document when proper precautions are taken.
Importantly, the code in the main file should ensure that the page counter
(as well as other status parameters which are stored in the |.aux| files)
takes the same value after the conditional processing.
Otherwise the page numbers may take divergent values
depending on which part is compiled.

For example, a title page could be declared by:
%
\begin{center}
\begin{tabular}{l}
|\ifchilddoc\||else|\\
|\addtocounter{page}{-1}|\\
\textit{code for title page}\\
|\newpage|\\
|\||fi|
\end{tabular}
\end{center}
%
A banner page for the child documents can be generated by:
%
\begin{center}
\begin{tabular}{l}
|\ifchilddoc|\\
|\addtocounter{page}{-1}|\\
\textit{code for banner page}\\
|\newpage|\\
|\||fi|
\end{tabular}
\end{center}
%
Here one could write a message such as:
\begin{center}
|This is the part \childdocname{} of \childdocjob{}.|
\end{center}

%%%%%%%%%%%%%%%%%%%%%%%%%%%%%%%%%%%%%%%%%%%%%%%%%%%%%%%%%%%%%%%%%%%%%%%%%%%%%%%%
\subsection{Flags}
\label{sec:flags}

The package makes it easy to generate different versions
of the main or child documents.
To this end compilation flags can be defined
and assigned different default values.
They will be particularly useful in conjunction
with the forwarding mechanism described in \secref{sec:forward}.

For example, it may be useful to have a flag |\version|
which can be set to |draft| or |final|.
The document source will contain some conditional code
depending on the value of |\version|.
Suppose further, the flag should default to |final| for the main file
and to |draft| for child files
which is a natural assignment for editing the document.
This is achieved by placing the following code
in the preamble of the main document
(below the |\childdocmain| directive):
%
\begin{center}
\begin{tabular}{l}
|\ifchilddoc|\\
|\providecommand{\version}{draft}|\\
|\||else|\\
|\providecommand{\version}{final}|\\
|\||fi|
\end{tabular}
\end{center}
%
The definition by |\providecommand| makes sure
that previous definitions are not overwritten.
Further statements |\providecommand{\version}{...}|
can thus be added before the above code to override it.

For the main file, one might add a line
(between |\childdocmain| and the above block)
%
\begin{center}
|%\ifchilddoc\||else\providecommand{\version}{draft}\||fi|
\end{center}
%
which can be uncommented to produce a draft version.
Likewise one can add a line to the very top of a child file
(above the |\childdocof{|\textit{main}|}| directive)
%
\begin{center}
|%\providecommand{\version}{final}|
\end{center}
%
which can be uncommented to produce the final version of this child document.

%%%%%%%%%%%%%%%%%%%%%%%%%%%%%%%%%%%%%%%%%%%%%%%%%%%%%%%%%%%%%%%%%%%%%%%%%%%%%%%%
\subsection{Forwarding}
\label{sec:forward}

Different versions of the main or child documents
using compilation flags as described in \secref{sec:flags}
can be (permanently) stored in different files
for convenient compilation, viewing and distribution.
To this end, the package defines a command
to pass on compilation to a different file:

%%%%%%%%%%%%%%%%%%%%%%%%%%%%%%%%%%%%%%%%
\DescribeMacro{\childdocforward}
The command |\childdocforward| redirects processing to
another source file:
%
\begin{center}
\begin{tabular}{l}
|\input{childdoc.def}|\\
|\childdocforward[|\textit{main}|]{|\textit{dest}|}|\\
\end{tabular}
\end{center}
%
The argument \textit{dest} is the destination file
(without extension).
It should be the main file or one of the child files.
Note that further \textsf{childdoc} directives
such as |\childdocof| and |\childdocforward|
in the indicated file will be processed in this form.
The optional argument \textit{main}
passes on directly to the main file \textit{main}
while pretending to compile the child \textit{dest}.
This form behaves as if \textit{dest}
issues |\childdocof{|\textit{main}|}| right away,
and no further \textsf{childdoc} directives will be processed.

%%%%%%%%%%%%%%%%%%%%%%%%%%%%%%%%%%%%%%%%
\DescribeMacro{\...prefix}
In the alternative form |\childdocforwardprefix|,
%
\begin{center}
\begin{tabular}{l}
|\input{childdoc.def}|\\
|\childdocforwardprefix[|\textit{main}|]{|\textit{prefix}|}{|\textit{dest}|}|
\end{tabular}
\end{center}
%
the destination file is determined by a pattern
depending on the current file:
To make this work, the current file must be called
`{\textit{prefix}\hspace{0.2em}\textit{suffix}}'
with \textit{prefix} matching precisely the argument.
Processing is then passed on to the file
`{\textit{dest}\hspace{0.2em}\textit{suffix}}'.
Surely, the same effect is achieved by
directly specifying the
argument `{\textit{dest}\hspace{0.2em}\textit{suffix}}'
in the first form.
However, that requires to set up a different file
for each child. With the alternative form of the command
all these files can have exactly the same content
which simplifies setting them up and maintaining them.

For example, the following file |draft.tex|
with a compilation flag |\version| as described in \secref{sec:flags}
compiles the main document as a draft:
%
\begin{center}
\begin{tabular}{l}
|\def\version{draft}|\\
|\input{childdoc.def}|\\
|\childdocforward{|\textit{main}|}|
\end{tabular}
\end{center}
%
Likewise, the following files |final|\textit{nn}|.tex|
compile the final version of the child document
|child|\textit{nn}|.tex|:
%
\begin{center}
\begin{tabular}{l}
|\def\version{final}|\\
|\input{childdoc.def}|\\
|\childdocforwardprefix{final}{child}|
\end{tabular}
\end{center}
%

Note that when several versions of a main file and/or of each child file
are to be generated, it may be convenient to set up a |Makefile| or
shell script to automatise the process.

%%%%%%%%%%%%%%%%%%%%%%%%%%%%%%%%%%%%%%%%%%%%%%%%%%%%%%%%%%%%%%%%%%%%%%%%%%%%%%%%
\subsection{Command Line Processing}
\label{sec:commandline}

The effect of redirection files can also be achieved by invoking
the \LaTeX{} compiler with a more elaborate command line.
Most conveniently this should be done as part
of a shell script or a |Makefile|.

When using \textsf{childdoc} in the main file, the following
command lines effectively perform a redirection
(note that depending on the shell being used,
backslashes may have to be doubled: `|\|' $\to$ `|\\|'):
%
\begin{center}
|... -jobname "|\textit{target}|" |\\|"|[\textit{flags}]%
|\input{childdoc.def}\childdocforward[|\textit{main}|]{|\textit{dest}|}"|
\end{center}
%
Here \textit{target} is the name of the output file,
\textit{main} is the name of the main file
and \textit{dest} is the name of the main or child file to be processed
(all filenames without extensions).
The optional argument \textit{main} can be omitted
if \textit{main} matches \textit{dest}.
Optionally, compilation \textit{flags} can be defined via |\def| commands.
This command line makes the \TeX{} engine believe
it is compiling the file \textit{target}
whose content is specified as the latter parameter.
The provided code then forwards the processing to
\textit{main} or \textit{dest} as described in \secref{sec:forward}.

%%%%%%%%%%%%%%%%%%%%%%%%%%%%%%%%%%%%%%%%%%%%%%%%%%%%%%%%%%%%%%%%%%%%%%%%%%%%%%%%
\subsection{Include by Input}
\label{sec:input}

Including child documents by |\include| has some restrictions by design.
Most notably, the content of a child document always occupies
its own set of pages; pages cannot be shared between child documents.
Usually, this behaviour makes perfect sense
because each child document contain an essential part of the document.
However, in some situations it may be desirable to compose
a document from a collection of parts
without having mandatory page breaks between then.
For this case, the package
provides a mechanism to include parts
by |\input| which can also be processed individually.
However, by construction this mechanism
requires manual handling of the content to be output.

%%%%%%%%%%%%%%%%%%%%%%%%%%%%%%%%%%%%%%%%
\DescribeMacro{\ifchilddocmanual}
The main file should be prepared as usual, see \secref{sec:include}.
However, the document body must make a distinction
between processing of an individual part and of the main document, e.g.:
%
\begin{center}
\begin{tabular}{l}
|\ifchilddocmanual|\\
|\input{\childdocname}|\\
|\||else|\\
\textit{document body with }|\input{|\textit{part}|}|\\
|\||fi|
\end{tabular}
\end{center}
%
The conditional |\ifchilddocmanual| is true whenever
a part to be included by |\input| is being compiled,
and the name of the part is stored in |\childdocname|.

%%%%%%%%%%%%%%%%%%%%%%%%%%%%%%%%%%%%%%%%
\DescribeMacro{\childdocby}
Each part to be included by |\input| should start with:
%
\begin{center}
\begin{tabular}{l}
|\input{childdoc.def}|\\
|\childdocby{|\textit{main}|}|\\
\end{tabular}
\end{center}
%
The directive |\childdocby| is similar to |\childdocof|
described in \secref{sec:include},
but the subsequent selection of content must be done manually.
To that end, both |\ifchilddoc| and |\ifchilddocmanual|
will be true upon processing of a part,
and the name of the part is stored in |\childdocname|.
Note that |\jobname| will be set to the filename of the current part
so that each part receives an individual |.aux| file
that does not interfere with the |.aux| file(s) of the main document.
This behaviour can be altered by the alternative form
|\childdocby[*]{|\textit{main}|}| (with a non-empty optional argument)
which uses the |.aux| file of the main document
by setting |\jobname| to \textit{main}.

%%%%%%%%%%%%%%%%%%%%%%%%%%%%%%%%%%%%%%%%%%%%%%%%%%%%%%%%%%%%%%%%%%%%%%%%%%%%%%%%
\subsection{Driver Development}
\label{sec:driver}

The \textsf{childdoc} mechanism can also be use for the development
of definition files such as \LaTeX{} styles or classes.
This case differs from the above setup with multiple parts
included by |\include| in that no |\includeonly| should be invoked.
This can be achieved by starting the include file
(before |\ProvidesPackage|) with:
%
\begin{center}
\begin{tabular}{l}
|\input{childdoc.def}|\\
|\childdocforward{|\textit{main}|}|\\
\end{tabular}
\end{center}
%
or alternatively with:
%
\begin{center}
\begin{tabular}{l}
|\input{childdoc.def}|\\
|\childdocby{|\textit{main}|}|\\
\end{tabular}
\end{center}
%
Both forms have slightly different effects as described above.
The main file is prepared as usual, see \secref{sec:include}.

%%%%%%%%%%%%%%%%%%%%%%%%%%%%%%%%%%%%%%%%%%%%%%%%%%%%%%%%%%%%%%%%%%%%%%%%%%%%%%%%
\subsection{Legacy Detection}
\label{sec:detection}

The directive |\childdocmain| in the main file can detect
whether the complete document or merely a child is to be compiled
even without using the directive |\childdocof|.
This method is deprecated because it is less robust
and there is no compelling reason to use it;
it is merely provided for backward compatibility
and it may be removed in future versions.

If the detection mechanism is to be used,
it is mandatory to correctly specify
the filename of the main file as the argument of |\childdocmain|:
%
\begin{center}
\begin{tabular}{l}
|\input{childdoc.def}|\\
|\childdocmain{|\textit{main}|}|\\
\end{tabular}
\end{center}
%
If |\jobname| does not match the argument \textit{main} of |\childdocmain|,
it is assumed that |\jobname| points to the child file to be compiled.
When using |\childdocmain| with the main file specified as argument,
it suffices to start a child file
with just |\input{|\textit{main}|}|
without loading of the package and using |\childdocof|.
If instead all processing is done
with the appropriate \textsf{childdoc} directives,
the argument of \textit{main} of |\childdocmain| can be empty.

An alternative version of the command line processing described
in \secref{sec:commandline} using the detection mechanism reads:
%
\begin{center}
|... -jobname "|\textit{target}|" "|[\textit{flags}]%
[|\def\jobname{|\textit{dest}|}|]|\input{|\textit{main}|}"|
\end{center}

%%%%%%%%%%%%%%%%%%%%%%%%%%%%%%%%%%%%%%%%%%%%%%%%%%%%%%%%%%%%%%%%%%%%%%%%%%%%%%%%
\subsection{Manual Code}
\label{sec:manual}

In case one cannot be certain whether the definitions file |childdoc.def|
is installed on the target \TeX{} distribution
and one prefers not to ship it,
it is conceivable to paste a few relevant commands into the sources.

To that end, drop all statements |\input{childdoc.def}|
and perform the replacements as outlined below.
Instead of |\childdocmain{|\textit{main}|}| add the following code
to the top of the main file:
%
\begin{center}
\begin{tabular}{l}
|\||ifdefined\childdocname\endinput\||fi\newif\ifchilddoc|\\
|\edef\childdocname{\scantokens\expandafter{\jobname\noexpand}}|\\
|\def\childdocmain{|\textit{main}|}\||ifx\childdocmain\childdocname\||else|\\
|\childdoctrue\includeonly{\childdocname}\let\jobname\childdocmain\||fi|\\
\end{tabular}
\end{center}
%
Instead of |\childdocof{|\textit{main}|}| just include the main file
at the top of each child file:
%
\begin{center}
|\input{|\textit{main}|}|
\end{center}
%
A simple redirection |\childdocforward{|\textit{dest}|}| is achieved by:
%
\begin{center}
|\def\jobname{|\textit{dest}|}\input{\jobname}|
\end{center}
%
The redirection with prefix
|\childdocforwardprefix[|\textit{prefix}|]{|\textit{dest}|}|
is accomplished by:
%
\begin{center}
\begin{tabular}{l}
|{\edef\jobname{\scantokens\expandafter{\jobname\noexpand}}|\\
|\def\redirectjob |\textit{prefix}|#1~~~{\gdef\jobname{|\textit{dest}|#1}}|\\
|\expandafter\redirectjob\jobname~~~}\input{\jobname}|
\end{tabular}
\end{center}

In an alternative approach,
child documents can be compiled by a specific command line
without additional code or specific definitions:
%
\begin{center}
|... -jobname "|\textit{target}|" "|[\textit{flags}]%
|\includeonly{|\textit{dest}|}\input{|\textit{main}|}"|
\end{center}
%

%%%%%%%%%%%%%%%%%%%%%%%%%%%%%%%%%%%%%%%%%%%%%%%%%%%%%%%%%%%%%%%%%%%%%%%%%%%%%%%%
%%%%%%%%%%%%%%%%%%%%%%%%%%%%%%%%%%%%%%%%%%%%%%%%%%%%%%%%%%%%%%%%%%%%%%%%%%%%%%%%
\section{Information}

%%%%%%%%%%%%%%%%%%%%%%%%%%%%%%%%%%%%%%%%%%%%%%%%%%%%%%%%%%%%%%%%%%%%%%%%%%%%%%%%
\subsection{Copyright}

Copyright \copyright{} 2017--2018 Niklas Beisert

This work may be distributed and/or modified under the
conditions of the \LaTeX{} Project Public License, either version 1.3
of this license or (at your option) any later version.
The latest version of this license is in
  \url{http://www.latex-project.org/lppl.txt}
and version 1.3 or later is part of all distributions of \LaTeX{}
version 2005/12/01 or later.

This work has the LPPL maintenance status `maintained'.

The Current Maintainer of this work is Niklas Beisert.

This work consists of the files |README.txt|, |childdoc.ins| and |childdoc.dtx|
as well as the derived files |childdoc.def|, |cdocsamp.tex|
with |cdocsch1.tex|, |cdocsch2.tex|, |cdocspt3.tex|, |cdocspt4.tex|,
|cdocsdrf.tex|, |cdocsfn1.tex|, |cdocsfn2.tex|
as well as |childdoc.pdf|.

%%%%%%%%%%%%%%%%%%%%%%%%%%%%%%%%%%%%%%%%%%%%%%%%%%%%%%%%%%%%%%%%%%%%%%%%%%%%%%%%
\subsection{Files and Installation}

The package consists of the files:
%
\begin{center}
\begin{tabular}{ll}
    |README.txt|   & readme file \\
    |childdoc.ins| & installation file \\
    |childdoc.dtx| & source file \\
    |childdoc.def| & definition file \\
    |cdocsamp.tex| & sample main file \\
    |cdocsch1.tex| & sample include file \\
    |cdocsch2.tex| & sample include file \\
    |cdocspt3.tex| & sample part file \\
    |cdocspt4.tex| & sample part file \\
    |cdocsdrf.tex| & sample redirection file \\
    |cdocsfn1.tex| & sample redirection file \\
    |cdocsfn2.tex| & sample redirection file \\
    |childdoc.pdf| & manual
\end{tabular}
\end{center}
%
The distribution consists of the files
|README.txt|, |childdoc.ins| and |childdoc.dtx|.
%
\begin{itemize}
\item
Run (pdf)\LaTeX{} on |childdoc.dtx|
to compile the manual |childdoc.pdf| (this file).
\item
Run \LaTeX{} on |childdoc.ins| to create the definitions file |childdoc.def|
and the sample |cdocsamp.tex| with include files
|cdocsch1.tex|, |cdocsch2.tex|, |cdocspt3.tex|, |cdocspt4.tex|,
|cdocsdrf.tex|, |cdocsfn1.tex|, |cdocsfn2.tex|.
Then copy the file |childdoc.def| to an appropriate directory of your \LaTeX{}
distribution, e.g.\ \textit{texmf-root}|/tex/latex/childdoc|.
\end{itemize}

%%%%%%%%%%%%%%%%%%%%%%%%%%%%%%%%%%%%%%%%%%%%%%%%%%%%%%%%%%%%%%%%%%%%%%%%%%%%%%%%
\subsection{Related CTAN Packages}

There are several other packages which offer a similar functionality:
%
\begin{itemize}
\item
The packages
\href{http://ctan.org/pkg/docmute}{\textsf{docmute}},
\href{http://ctan.org/pkg/includex}{\textsf{includex}} and
\href{http://ctan.org/pkg/standalone}{\textsf{standalone}}
provide commands to include only the document body of
a child file thus allowing both files to be compiled individually.
\item
The packages \href{http://ctan.org/pkg/subdocs}{\textsf{subdocs}}
and \href{http://ctan.org/pkg/subfiles}{\textsf{subfiles}}
provide structures in which the main and child documents can be
encapsulated and allowing them to be compiled individually.
The inclusion mechanism is different from the conventional |\include|.
\item
The package \href{http://ctan.org/pkg/combine}{\textsf{combine}}
is an elaborate solution to combine several documents into one.
\end{itemize}
%
See also the CTAN topic \href{http://ctan.org/topic/subdocs}{\textsf{subdocs}}
for further related packages.
The present package differs from the above solutions in that
a document structure constructed with the conventional |\include| mechanism
just needs two extra commands at the top of every file
such that all constituent files can be compiled individually.

%%%%%%%%%%%%%%%%%%%%%%%%%%%%%%%%%%%%%%%%%%%%%%%%%%%%%%%%%%%%%%%%%%%%%%%%%%%%%%%%
%\subsection{Feature Suggestions}
%
%The following is a list of features which may be useful for future
%versions of this package:
%%
%\begin{itemize}
%\item
%\ldots
%\end{itemize}

%%%%%%%%%%%%%%%%%%%%%%%%%%%%%%%%%%%%%%%%%%%%%%%%%%%%%%%%%%%%%%%%%%%%%%%%%%%%%%%%
\subsection{Revision History}

%%%%%%%%%%%%%%%%%%%%%%%%%%%%%%%%%%%%%%%%
\paragraph{v2.0:} 2018/12/30

\begin{itemize}
\item
immediate forward processing
\item
added |\childdocby| mechanism
\item
manual restructured
\end{itemize}

%%%%%%%%%%%%%%%%%%%%%%%%%%%%%%%%%%%%%%%%
\paragraph{v1.6:} 2018/01/17

\begin{itemize}
\item
application for development of include files
\item
corrections to manual
\end{itemize}

%%%%%%%%%%%%%%%%%%%%%%%%%%%%%%%%%%%%%%%%
\paragraph{v1.5:} 2017/05/21

\begin{itemize}
\item
more complete structuring introduced
\item
|\childdocof| introduced
\item
|\childdoc| renamed to |\childdocmain|
\item
|\childredirect| renamed to |\childdocforward| and |\childdocforwardprefix|
and functionality expanded
\end{itemize}

%%%%%%%%%%%%%%%%%%%%%%%%%%%%%%%%%%%%%%%%
\paragraph{v1.0:} 2017/04/27

\begin{itemize}
\item
manual and install package
\item
first version published on CTAN
\end{itemize}

%%%%%%%%%%%%%%%%%%%%%%%%%%%%%%%%%%%%%%%%
\paragraph{v0.6:} 2017/04/26

\begin{itemize}
\item
redirection mechanism added
\end{itemize}

%%%%%%%%%%%%%%%%%%%%%%%%%%%%%%%%%%%%%%%%
\paragraph{v0.5:} 2017/04/26

\begin{itemize}
\item
functionality in definition file
\end{itemize}


%%%%%%%%%%%%%%%%%%%%%%%%%%%%%%%%%%%%%%%%%%%%%%%%%%%%%%%%%%%%%%%%%%%%%%%%%%%%%%%%
%%%%%%%%%%%%%%%%%%%%%%%%%%%%%%%%%%%%%%%%%%%%%%%%%%%%%%%%%%%%%%%%%%%%%%%%%%%%%%%%
%%%%%%%%%%%%%%%%%%%%%%%%%%%%%%%%%%%%%%%%%%%%%%%%%%%%%%%%%%%%%%%%%%%%%%%%%%%%%%%%
\appendix

\settowidth\MacroIndent{\rmfamily\scriptsize 000\ }

 \DocInput{childdoc.dtx}

\end{document}
%</driver>
% \fi
%
% %%%%%%%%%%%%%%%%%%%%%%%%%%%%%%%%%%%%%%%%%%%%%%%%%%%%%%%%%%%%%%%%%%%%%%%%%%%%%%
% %%%%%%%%%%%%%%%%%%%%%%%%%%%%%%%%%%%%%%%%%%%%%%%%%%%%%%%%%%%%%%%%%%%%%%%%%%%%%%
% \section{Sample}
%\iffalse
%<*samplemain>
%\fi
%
% The following presents a sample document
% with two chapters, two parts, a title page,
% a compile flag as well as three forwarding files to set the flag.
% It consists of eight |.tex| files:
% \begin{center}
% \begin{tabular}{ll}
% |cdocsamp.tex|&main file\\
% |cdocsch1.tex|&include file for chapter 1\\
% |cdocsch2.tex|&include file for chapter 2\\
% |cdocspt3.tex|&include file for part 3\\
% |cdocspt4.tex|&include file for part 4\\
% |cdocsdrf.tex|&forwarding file for main file in draft mode\\
% |cdocsfi1.tex|&forwarding file for final version of chapter 1\\
% |cdocsfi2.tex|&forwarding file for final version of chapter 2\\
% \end{tabular}
% \end{center}
% Each of the eight files can be compiled directly by the \LaTeX{} compiler.
%
% %%%%%%%%%%%%%%%%%%%%%%%%%%%%%%%%%%%%%%
% \paragraph{Main File.}
%
% The main file is called |cdocsamp.tex|.
%
% Load the \textsf{childdoc} definitions and
% declare the filename for the main document:
%    \begin{macrocode}
\input{childdoc.def}
\childdocmain{}
%    \end{macrocode}

% Optional override for |\version| flag:
%    \begin{macrocode}
%%\ifchilddoc\else\providecommand{\version}{draft}\fi
%    \end{macrocode}

% Define the default values for the |\version| flag
% (|final| for the main file and |draft| for childs):
%    \begin{macrocode}
\ifchilddoc
\providecommand{\version}{draft}
\else
\providecommand{\version}{final}
\fi
%    \end{macrocode}

% Load the standard document class:
%    \begin{macrocode}
\documentclass[12pt]{article}
%    \end{macrocode}

% Start the document body:
%    \begin{macrocode}
\begin{document}
%    \end{macrocode}

% Declare a title page.
% Print title, part of document being processed and version flag:
%    \begin{macrocode}
\addtocounter{page}{-1}
\begin{center}
{\LARGE\bfseries{}childdoc example\par}
\vspace{1cm}
\ifchilddoc
\ifchilddocmanual part\else chapter\fi:
`\childdocname' of `\childdocjob'\par
\else
main document: `\childdocjob'\par
\fi
version: \version\par
\end{center}
\newpage
%    \end{macrocode}

% Manually include selected file,
% otherwise process as usual:
%    \begin{macrocode}
\ifchilddocmanual
\section*{part `\childdocname'}
\input{\childdocname}
\else
%    \end{macrocode}

% Include the two chapters:
%    \begin{macrocode}
\include{cdocsch1}
\include{cdocsch2}
%    \end{macrocode}

% Include the two parts unless only chapters should be displayed:
%    \begin{macrocode}
\ifchilddoc\else
\section{part three}
\input{cdocspt3}
\section{part four}
\input{cdocspt4}
\fi
%    \end{macrocode}

% Process as usual until here:
%    \begin{macrocode}
\fi
%    \end{macrocode}

% End of document body:
%    \begin{macrocode}
\end{document}
%    \end{macrocode}
%\iffalse
%</samplemain>
%\fi
%
% %%%%%%%%%%%%%%%%%%%%%%%%%%%%%%%%%%%%%%
% \paragraph{Chapter Include Files.}
%
% The include files are called |cdocsch1.tex| and |cdocsch2.tex|.
%
%\iffalse
%<*samplechap1|samplechap2>
%\fi

% Optional override for |\version| flag:
%    \begin{macrocode}
%%\providecommand{\version}{final}
%    \end{macrocode}

% Include the main document:
%    \begin{macrocode}
\input{childdoc.def}
\childdocof{cdocsamp}
%    \end{macrocode}

%\iffalse
%</samplechap1|samplechap2>
%\fi
%
%\iffalse
%<*samplechap1>
%\fi
% Some text for chapter 1:
%    \begin{macrocode}
\section{one}
some text in chapter one
%    \end{macrocode}

%\iffalse
%</samplechap1>
%\fi
% Some text for chapter 2:
%\iffalse
%<*samplechap2>
%\fi
%    \begin{macrocode}
\section{two}
more text in chapter two
%    \end{macrocode}

%\iffalse
%</samplechap2>
%\fi
%
% %%%%%%%%%%%%%%%%%%%%%%%%%%%%%%%%%%%%%%
% \paragraph{Part Include Files.}
%
% The include files are called |cdocspt3.tex| and |cdocspt4.tex|.
%
%\iffalse
%<*samplepart3|samplepart4>
%\fi

% Optional override for |\version| flag:
%    \begin{macrocode}
%%\providecommand{\version}{final}
%    \end{macrocode}

% Include the main document:
%    \begin{macrocode}
\input{childdoc.def}
\childdocby{cdocsamp}
%    \end{macrocode}

%\iffalse
%</samplepart3|samplepart4>
%\fi
%
%\iffalse
%<*samplepart3>
%\fi
% Some text for part 3:
%    \begin{macrocode}
some text in part three
%    \end{macrocode}

%\iffalse
%</samplepart3>
%\fi
% Some text for part 4:
%\iffalse
%<*samplepart4>
%\fi
%    \begin{macrocode}
more text in part four
%    \end{macrocode}

%\iffalse
%</samplepart4>
%\fi
%
% %%%%%%%%%%%%%%%%%%%%%%%%%%%%%%%%%%%%%%
% \paragraph{Forwarding for a Complete Draft.}
%
% The following forwarding file |cdocsdrf.tex|
% compiles the main document in draft mode:
%\iffalse
%<*sampledraft>
%\fi
%    \begin{macrocode}
\def\version{draft}
\input{childdoc.def}
\childdocforward{cdocsamp}
%    \end{macrocode}

%\iffalse
%</sampledraft>
%\fi
%
% %%%%%%%%%%%%%%%%%%%%%%%%%%%%%%%%%%%%%%
% \paragraph{Forwarding for Final Version of the Chapters.}
%
% The following forwarding files |cdocsfn1.tex| and |cdocsfn2.tex|
% (with identical content)
% compile the final versions of the child documents
% |cdocsch1.tex| and |cdocsch2.tex|, respectively:
%\iffalse
%<*samplefinal>
%\fi
%    \begin{macrocode}
\def\version{final}
\input{childdoc.def}
\childdocforwardprefix[cdocsamp]{cdocsfn}{cdocsch}
%    \end{macrocode}

%\iffalse
%</samplefinal>
%\fi
%
% %%%%%%%%%%%%%%%%%%%%%%%%%%%%%%%%%%%%%%
% \paragraph{Command Line Processing.}
%
% The following three command lines generate the output files
% |cdocscld|, |cdocscl1| and |cdocscl2|
% which should be identical to
% |cdocsdrf|, |cdocsch1| and |cdocsfn2|, respectively:
% \begin{center}
% \begin{tabular}{l}
% |latex -jobname cdocscld \|\\
% |  "\def\version{draft}\input{childdoc.def}\childdocforward{cdocsamp}"|\\
% |latex -jobname cdocscl1 \|\\
% |  "\input{childdoc.def}\childdocforward[cdocsamp]{cdocsch1}"|\\
% |latex -jobname cdocscl2 \|\\
% |  "\def\version{final}\input{childdoc.def}\childdocforward{cdocsch2}"|
% \end{tabular}
% \end{center}
% Note that the trailing backslash on each first line
% merely continues the input to the second line
% (for convenient cut ant paste).
% Furthermore, the command |latex| can be replaced by any
% of its alternative versions such as |pdflatex|.
%
% %%%%%%%%%%%%%%%%%%%%%%%%%%%%%%%%%%%%%%%%%%%%%%%%%%%%%%%%%%%%%%%%%%%%%%%%%%%%%%
% %%%%%%%%%%%%%%%%%%%%%%%%%%%%%%%%%%%%%%%%%%%%%%%%%%%%%%%%%%%%%%%%%%%%%%%%%%%%%%
% \section{Implementation}
%\iffalse
%<*package>
%\fi
%
% This section describes the definitions file |childdoc.def|.

% The definitions cannot be loaded using |\usepackage| or |\RequirePackage|
% which has a mechanism to prevent loading a style file more than once.
% When loading the definitions by means of |\input|
% multiple instances have to be prevented manually:
%\iffalse
%This code needs to be before the `\ProvidesFile' directive
%which is defined at the beginning of this file.
%Therefore it is also placed there and commented out here.
%</package>
%<*discard>
%\fi
%    \begin{macrocode}
\ifdefined\childdocmain\endinput\fi
%    \end{macrocode}
%\iffalse
%</discard>
%<*package>
%\fi
%
% \macro{\ifchilddoc}
% \macro{\ifchilddocmanual}
% The conditional |\ifchilddoc| tells whether a
% child (true) or main (false) document is being compiled.
% The conditional |\ifchilddocmanual| tells whether
% the |\includeonly| mechanism is used (false) or
% the selection of child files must be performed manually (true).
% The definitions initialise to false:
%    \begin{macrocode}
\newif\ifchilddoc
\newif\ifchilddocmanual
%    \end{macrocode}

% \macro{\childdocname}
% \macro{\childdocjob}
% The macro |\childdocname| stores the name of the main document
% to be compiled. The macro |\childdocjob| stores the name of
% the document on which the \LaTeX{} compiler was originally invoked.
% The content of |\jobname| cannot be compared
% to filenames specified in the source due to different catcodes.
% The following code rescans |\jobname|, stores the result
% in |\childdocname| and saves a copy in |\childdocjob|:
%    \begin{macrocode}
\edef\childdocname{\scantokens\expandafter{\jobname\noexpand}}
\let\childdocjob\childdocname
%    \end{macrocode}

% \macro{\childdocdisable}
% The macro |\childdocdisable| prevents the main file
% from being processed more than once.
% At this stage, the main document command |\childdocmain|
% is assumed to be called once again where it should do nothing.
% Any subsequent call to it should prevent
% a secondary processing of the main document
% It overwrites the forwarding commands
% |\childdocof| and |\childdocforward|
% with empty macros to prevent further inclusions of the main document:
%    \begin{macrocode}
\newcommand{\childdocdisable}
{
  \renewcommand{\childdocmain}[1]{\renewcommand{\childdocmain}[1]{\endinput}}
  \renewcommand{\childdocof}[1]{}
  \renewcommand{\childdocby}[2][]{}
  \renewcommand{\childdocforward}[2][]{}
  \renewcommand{\childdocdisable}{}
}
%    \end{macrocode}

% \macro{\childdocmain}
% The macro |\childdocmain| is to be called at the top of the main file
% with nothing or the main filename (without extension) as argument.
% First, it breaks loops.
% If the argument is not empty and does not match |\childdocname|
% (which is set by the first inclusion of |childdoc.def|),
% |\ifchilddoc| is set to true, |\includeonly| is applied to the child file
% and |\jobname| is set to the main file
% (for proper handling of |.aux| files):
%    \begin{macrocode}
\newcommand{\childdocmain}[1]
{
  \childdocdisable\childdocmain{}
  \if?#1?\else
    \begingroup
      \def\childdoctmp{#1}
      \ifx\childdoctmp\childdocname
        \def\childdoctmp{}
      \else
        \def\childdoctmp
        {
          \childdoctrue
          \includeonly{\childdocname}
          \def\childdocjob{#1}
          \def\jobname{#1}
        }
      \fi
      \expandafter
    \endgroup
    \childdoctmp
  \fi
}
%    \end{macrocode}

% \macro{\childdocof}
% The command |\childdocof| redirects
% compilation to the main file |#1|.
%    \begin{macrocode}
\newcommand{\childdocof}[1]
{
  \childdocdisable
  \childdoctrue
  \includeonly{\childdocname}
  \def\jobname{#1}
  \def\childdocjob{#1}
  \input{#1}
}
%    \end{macrocode}

% \macro{\childdocby}
% The command |\childdocby| ....
%    \begin{macrocode}
\newcommand{\childdocby}[2][]
{
  \childdocdisable
  \childdoctrue
  \childdocmanualtrue
  \if?#1?\else
    \def\jobname{#2}
  \fi
  \def\childdocjob{#2}
  \input{#2}
  \endinput
}
%    \end{macrocode}

% \macro{\childdocforward}
% The command |\childdocforward| redirects
% compilation to the main file or
% (if the optional argument is given) a child file.
% Parameters are set as if the main file
% or a child file starting with |\childdocof| was compiled.
% Then compilation is handed over to the main file:
%    \begin{macrocode}
\newcommand{\childdocforward}[2][]
{
  \begingroup
    \if?#1?
      \def\childdoctmp
      {
        \def\childdocname{#2}
        \def\childdocjob{#2}
        \def\jobname{#2}
        \input{#2}
        \endinput
      }
    \else
      \def\childdoctmp
      {
        \childdocdisable
        \def\childdocname{#2}
        \childdoctrue
        \includeonly{#2}
        \def\childdocjob{#1}
        \def\jobname{#1}
        \input{#1}
        \endinput
      }
    \fi
    \expandafter
  \endgroup
  \childdoctmp
}
%    \end{macrocode}

% \macro{\childdocforwardprefix}
% The command |\childdocforwardprefix| redirects
% compilation to the main or a child file by means of a pattern.
% The prefix |#1| in the current filename is replaced by |#2|
% and the suffix of the current filename is kept
% (it is assumed that the filename does not contain the substring `|~~~|'
% which is used as a delimiter).
% Compilation is handed over to the new file by |\childdocforward|:
%    \begin{macrocode}
\newcommand{\childdocforwardprefix}[3][]
{
  \begingroup
    \def\childdocextract #2##1~~~{\def\childdoctmp{\childdocforward[#1]{#3##1}}}
    \expandafter\childdocextract\childdocname~~~
    \expandafter
  \endgroup
  \childdoctmp
}
%    \end{macrocode}

% \macro{\childdoc}
% The deprecated macro |\childdoc| is a legacy version of |\childdocmain|:
%    \begin{macrocode}
\newcommand{\childdoc}{\childdocmain}
%    \end{macrocode}

% \macro{\childdocredirect}
% The deprecated macro |\childdocredirect| is a legacy version
% of |\childdocforward| and |\childdocforwardprefix|:
%    \begin{macrocode}
\newcommand{\childdocredirect}[2][]
{
  \begingroup
    \if?#1?
      \def\childdoctmp{\childdocforward{#2}}
    \else
      \def\childdoctmp{\childdocforwardprefix{#1}{#2}}
    \fi
    \expandafter
  \endgroup
  \childdoctmp
}
%    \end{macrocode}

%\iffalse
%</package>
%\fi
%
\endinput
\childdocforward[|\textit{main}|]{|\textit{dest}|}"|
\end{center}
%
Here \textit{target} is the name of the output file,
\textit{main} is the name of the main file
and \textit{dest} is the name of the main or child file to be processed
(all filenames without extensions).
The optional argument \textit{main} can be omitted
if \textit{main} matches \textit{dest}.
Optionally, compilation \textit{flags} can be defined via |\def| commands.
This command line makes the \TeX{} engine believe
it is compiling the file \textit{target}
whose content is specified as the latter parameter.
The provided code then forwards the processing to
\textit{main} or \textit{dest} as described in \secref{sec:forward}.

%%%%%%%%%%%%%%%%%%%%%%%%%%%%%%%%%%%%%%%%%%%%%%%%%%%%%%%%%%%%%%%%%%%%%%%%%%%%%%%%
\subsection{Include by Input}
\label{sec:input}

Including child documents by |\include| has some restrictions by design.
Most notably, the content of a child document always occupies
its own set of pages; pages cannot be shared between child documents.
Usually, this behaviour makes perfect sense
because each child document contain an essential part of the document.
However, in some situations it may be desirable to compose
a document from a collection of parts
without having mandatory page breaks between then.
For this case, the package
provides a mechanism to include parts
by |\input| which can also be processed individually.
However, by construction this mechanism
requires manual handling of the content to be output.

%%%%%%%%%%%%%%%%%%%%%%%%%%%%%%%%%%%%%%%%
\DescribeMacro{\ifchilddocmanual}
The main file should be prepared as usual, see \secref{sec:include}.
However, the document body must make a distinction
between processing of an individual part and of the main document, e.g.:
%
\begin{center}
\begin{tabular}{l}
|\ifchilddocmanual|\\
|\input{\childdocname}|\\
|\||else|\\
\textit{document body with }|\input{|\textit{part}|}|\\
|\||fi|
\end{tabular}
\end{center}
%
The conditional |\ifchilddocmanual| is true whenever
a part to be included by |\input| is being compiled,
and the name of the part is stored in |\childdocname|.

%%%%%%%%%%%%%%%%%%%%%%%%%%%%%%%%%%%%%%%%
\DescribeMacro{\childdocby}
Each part to be included by |\input| should start with:
%
\begin{center}
\begin{tabular}{l}
|% \iffalse
%
% childdoc.dtx Copyright (C) 2017-2018 Niklas Beisert
%
% This work may be distributed and/or modified under the
% conditions of the LaTeX Project Public License, either version 1.3
% of this license or (at your option) any later version.
% The latest version of this license is in
%   http://www.latex-project.org/lppl.txt
% and version 1.3 or later is part of all distributions of LaTeX
% version 2005/12/01 or later.
%
% This work has the LPPL maintenance status `maintained'.
%
% The Current Maintainer of this work is Niklas Beisert.
%
% This work consists of the files childdoc.dtx and childdoc.ins
% and the derived files childdoc.def and cdocsamp.tex with
% cdocsch1.tex, cdocsch2.tex, cdocsdrf.tex, cdocsfn1.tex, cdocsfn2.tex.
%
%<package>\ifdefined\childdocmain\endinput\fi
%<package>\ProvidesFile{childdoc.def}[2018/12/30 v2.0 child document driver]
%<samplemain>\ProvidesFile{cdocsamp.tex}[2018/12/30 v2.0 sample for childdoc]
%<*driver>
%\ProvidesFile{childdoc.drv}[2018/12/30 v2.0 childdoc reference manual file]
\PassOptionsToClass{10pt,a4paper}{article}
\documentclass{ltxdoc}

\usepackage[margin=35mm]{geometry}
\usepackage{hyperref}
\usepackage{hyperxmp}
\usepackage[usenames]{color}

\hypersetup{colorlinks=true}
\hypersetup{pdfstartview=FitH}
\hypersetup{pdfpagemode=UseNone}
\hypersetup{pdfsource={}}
\hypersetup{pdflang={en-UK}}
\hypersetup{pdfcopyright={Copyright 2017-2018 Niklas Beisert.
  This work may be distributed and/or modified under the
  conditions of the LaTeX Project Public License, either version 1.3
  of this license or (at your option) any later version.}}
\hypersetup{pdflicenseurl={http://www.latex-project.org/lppl.txt}}
\hypersetup{pdfcontactaddress={ETH Zurich, ITP, HIT K,
  Wolfgang-Pauli-Strasse 27}}
\hypersetup{pdfcontactpostcode={8093}}
\hypersetup{pdfcontactcity={Zurich}}
\hypersetup{pdfcontactcountry={Switzerland}}
\hypersetup{pdfcontactemail={nbeisert@itp.phys.ethz.ch}}
\hypersetup{pdfcontacturl={http://people.phys.ethz.ch/\xmptilde nbeisert/}}

\newcommand{\secref}[1]{\hyperref[#1]{section \ref*{#1}}}

\parskip1ex
\parindent0pt
\let\olditemize\itemize
\def\itemize{\olditemize\parskip0pt}

\begin{document}

\title{The \textsf{childdoc} Package}
\hypersetup{pdftitle={The childdoc Package}}
\author{Niklas Beisert\\[2ex]
  Institut f\"ur Theoretische Physik\\
  Eidgen\"ossische Technische Hochschule Z\"urich\\
  Wolfgang-Pauli-Strasse 27, 8093 Z\"urich, Switzerland\\[1ex]
  \href{mailto:nbeisert@itp.phys.ethz.ch}
  {\texttt{nbeisert@itp.phys.ethz.ch}}}
\hypersetup{pdfauthor={Niklas Beisert}}
\hypersetup{pdfsubject={Manual for the LaTeX2e Package childdoc}}
\date{30 December 2018, \textsf{v2.0}}
\maketitle

\begin{abstract}\noindent
\textsf{childdoc} is a \LaTeXe{} package
that enables the direct compilation
of document sections included by |\include|
to individual files.
\end{abstract}

\begingroup
\parskip0ex
\tableofcontents
\endgroup

%%%%%%%%%%%%%%%%%%%%%%%%%%%%%%%%%%%%%%%%%%%%%%%%%%%%%%%%%%%%%%%%%%%%%%%%%%%%%%%%
%%%%%%%%%%%%%%%%%%%%%%%%%%%%%%%%%%%%%%%%%%%%%%%%%%%%%%%%%%%%%%%%%%%%%%%%%%%%%%%%
\section{Introduction}

\LaTeX{} provides a mechanism to structure a large document (such as a book)
into a main file and several child files (containing the chapters)
using the |\include| command.
This mechanism is beneficial for documents
which span hundreds of pages in order to
make the source file(s) more manageable.
Moreover, compilation can be restricted to
selected child files by means of the |\includeonly| command.
The latter feature can be used to reduce the compilation time while editing
(this was significantly more useful in the earlier days of \LaTeX{})
or to generate a smaller document which is easier to navigate.
Another application of |\includeonly| is to generate
documents consisting of selected parts of the complete document.

However, there are a few drawbacks of the plain |\include| mechanism:
\begin{itemize}
\item
The child files cannot be compiled on their own,
they can only be compiled via the main file.
A naive editing environment
(such as a text editor with an option
to have the current file processed by \LaTeX)
may require one to switch to the main file before compiling;
attempting to compile the child file produces errors.
\item
The main file must be modified (each time)
to adjust the |\includeonly| command
to the present needs. This easily leaves the main file in a messy state.
\item
The generated document will always carry the filename
of the main document. This is inconvenient if
several child files are to be compiled and
to be kept for distribution.
\end{itemize}

The present package provides a simple interface
to make child files individually compilable by \LaTeX{}.
Compiling a child file then has the same effect as compiling
the main file with an |\includeonly| command
to select the appropriate child.
Moreover the generated document will carry the name of the child
rather than the main file.
This resolves all three above issues.

This feature is meant to make the editing of books,
thesis documents and lecture notes somewhat more convenient.
However, the package can also be used efficiently for
composing a series of documents (such as exercise sheets)
which are typically distributed individually.
It then assists the author in generating the individual documents
(potentially in different versions)
as well as a document containing the collected series.
Another application is in developing style files
or other kinds of included material
where compilation of the style file could redirect
to a sample or test file.

%%%%%%%%%%%%%%%%%%%%%%%%%%%%%%%%%%%%%%%%%%%%%%%%%%%%%%%%%%%%%%%%%%%%%%%%%%%%%%%%
%%%%%%%%%%%%%%%%%%%%%%%%%%%%%%%%%%%%%%%%%%%%%%%%%%%%%%%%%%%%%%%%%%%%%%%%%%%%%%%%
\section{Usage}

First of all, the package \textsf{childdoc} is \emph{not} a standard
\LaTeXe{} |.sty| style file! Therefore it needs to be invoked in
a non-standard way.

%%%%%%%%%%%%%%%%%%%%%%%%%%%%%%%%%%%%%%%%%%%%%%%%%%%%%%%%%%%%%%%%%%%%%%%%%%%%%%%%
\subsection{Included Files}
\label{sec:include}

%%%%%%%%%%%%%%%%%%%%%%%%%%%%%%%%%%%%%%%%
\DescribeMacro{\childdocmain}
To use the package, add the commands
\begin{center}
\begin{tabular}{l}
|\input{childdoc.def}|\\
|\childdocmain{}|\\
\end{tabular}
\end{center}
at the very top of the main \LaTeX{} file,
in particular \emph{before} the |\documentclass| statement!
The argument of |\childdocmain| should be left empty
(but it must be present).

%%%%%%%%%%%%%%%%%%%%%%%%%%%%%%%%%%%%%%%%
\DescribeMacro{\childdocof}
Furthermore, add the commands
\begin{center}
\begin{tabular}{l}
|\input{childdoc.def}|\\
|\childdocof{|\textit{main}|}|\\
\end{tabular}
\end{center}
at the top of every child file \textit{child}
which is included by |\include{|\textit{child}|}|
from within the main file
(or at least for those files to be compiled individually).
The argument \textit{main} must be the filename of the main file.

There are a couple of
considerations in setting up the main and child documents:

%%%%%%%%%%%%%%%%%%%%%%%%%%%%%%%%%%%%%%%%
\paragraph{Restrictions.}

Please note the following restrictions:
\begin{itemize}
\item
|\childdocmain| must be called with one argument \textit{main}
to ensure compatibility with earlier version of the package.
It must either be empty (|\childdocmain{}|)
or precisely match the filename of the main file in which it is specified.
See \secref{sec:detection} for further information.
\item
The filename \textit{main} must be specified without the |.tex| extension.
\item
The filename \textit{main} is case sensitive
(even in case-insensitive file systems)
due to internal string comparison.
\item
The argument \textit{main} should be fully expanded, it cannot be a macro.
\item
Subdirectories and special characters should be avoided in filenames.
\item
The command |\childdocmain{|\textit{main}|}| must be followed by a whitespace.
It should not be followed immediately by another command
or by a comment mark `|%|'.
This is because the \TeX{} parser reads the token immediately following
the argument of |\childdocmain| and puts it
at the beginning of every child section;
however, a white\-space is ignored.
\end{itemize}

%%%%%%%%%%%%%%%%%%%%%%%%%%%%%%%%%%%%%%%%
\paragraph{Content of Main File.}

It is advisable to place all content in the child files included by |\include|.
Any output contained in the main file will appear in all child documents
unless suppressed manually;
it cannot be suppressed automatically by the |\includeonly| directive
and thus should normally be avoided.
A method to include some content in the main file
by means of conditional processing is described in \secref{sec:conditional}.

%%%%%%%%%%%%%%%%%%%%%%%%%%%%%%%%%%%%%%%%
\paragraph{Page Numbering.}

When only a part of the document is compiled,
the appropriate numbering of pages
(as well as other status parameters)
is determined from the |.aux| files.
The latter contain information from previous passes.
However this information needs to propagate through
all intermediate child documents.
Therefore the page numbering in child documents may well
be inconsistent until the complete document is compiled at least once.

A useful (if unconventional) way to always ensure a consistent
page numbering is to restart the numbering in each child document
and denote the pages by `\textit{child}|.|\textit{page}'
where \textit{child} represents the chapter/section number of the child file.
This can be achieved by the command
|\numberwithin{page}{|\textit{child}|}|
of the \textsf{amsmath} package
where \textit{child} can be |chapter| or |section|
depending on the chosen structuring.
Alternatively, one can modify the macro |\thepage| appropriately
and reset the counter |page| at the start of each child file.

%%%%%%%%%%%%%%%%%%%%%%%%%%%%%%%%%%%%%%%%%%%%%%%%%%%%%%%%%%%%%%%%%%%%%%%%%%%%%%%%
\subsection{Conditional Processing}
\label{sec:conditional}

The package provides a mechanism to compile different versions
of a document. To customise the versions further some conditional processing
can come in handy to distinguish which version is being compiled.
The package provides two macros to describe the compilation context:

%%%%%%%%%%%%%%%%%%%%%%%%%%%%%%%%%%%%%%%%
\DescribeMacro{\ifchilddoc}
The conditional |\ifchilddoc| distinguishes between the compilation of
child documents and the main document:
%
\begin{center}
|\ifchilddoc |\textit{child-code}| |[|\||else |\textit{main-code}]| \||fi|
\end{center}

%%%%%%%%%%%%%%%%%%%%%%%%%%%%%%%%%%%%%%%%
\DescribeMacro{\childdocname}
\DescribeMacro{\childdocjob}
The macro |\childdocname| contains the filename (without extension)
of the main or child file being processed.
Note that |\childdocjob| will always contain the name of the main file.

%%%%%%%%%%%%%%%%%%%%%%%%%%%%%%%%%%%%%%%%
\paragraph{Title Page.}

Conditional processing can be used to include a title or banner page
in the main document when proper precautions are taken.
Importantly, the code in the main file should ensure that the page counter
(as well as other status parameters which are stored in the |.aux| files)
takes the same value after the conditional processing.
Otherwise the page numbers may take divergent values
depending on which part is compiled.

For example, a title page could be declared by:
%
\begin{center}
\begin{tabular}{l}
|\ifchilddoc\||else|\\
|\addtocounter{page}{-1}|\\
\textit{code for title page}\\
|\newpage|\\
|\||fi|
\end{tabular}
\end{center}
%
A banner page for the child documents can be generated by:
%
\begin{center}
\begin{tabular}{l}
|\ifchilddoc|\\
|\addtocounter{page}{-1}|\\
\textit{code for banner page}\\
|\newpage|\\
|\||fi|
\end{tabular}
\end{center}
%
Here one could write a message such as:
\begin{center}
|This is the part \childdocname{} of \childdocjob{}.|
\end{center}

%%%%%%%%%%%%%%%%%%%%%%%%%%%%%%%%%%%%%%%%%%%%%%%%%%%%%%%%%%%%%%%%%%%%%%%%%%%%%%%%
\subsection{Flags}
\label{sec:flags}

The package makes it easy to generate different versions
of the main or child documents.
To this end compilation flags can be defined
and assigned different default values.
They will be particularly useful in conjunction
with the forwarding mechanism described in \secref{sec:forward}.

For example, it may be useful to have a flag |\version|
which can be set to |draft| or |final|.
The document source will contain some conditional code
depending on the value of |\version|.
Suppose further, the flag should default to |final| for the main file
and to |draft| for child files
which is a natural assignment for editing the document.
This is achieved by placing the following code
in the preamble of the main document
(below the |\childdocmain| directive):
%
\begin{center}
\begin{tabular}{l}
|\ifchilddoc|\\
|\providecommand{\version}{draft}|\\
|\||else|\\
|\providecommand{\version}{final}|\\
|\||fi|
\end{tabular}
\end{center}
%
The definition by |\providecommand| makes sure
that previous definitions are not overwritten.
Further statements |\providecommand{\version}{...}|
can thus be added before the above code to override it.

For the main file, one might add a line
(between |\childdocmain| and the above block)
%
\begin{center}
|%\ifchilddoc\||else\providecommand{\version}{draft}\||fi|
\end{center}
%
which can be uncommented to produce a draft version.
Likewise one can add a line to the very top of a child file
(above the |\childdocof{|\textit{main}|}| directive)
%
\begin{center}
|%\providecommand{\version}{final}|
\end{center}
%
which can be uncommented to produce the final version of this child document.

%%%%%%%%%%%%%%%%%%%%%%%%%%%%%%%%%%%%%%%%%%%%%%%%%%%%%%%%%%%%%%%%%%%%%%%%%%%%%%%%
\subsection{Forwarding}
\label{sec:forward}

Different versions of the main or child documents
using compilation flags as described in \secref{sec:flags}
can be (permanently) stored in different files
for convenient compilation, viewing and distribution.
To this end, the package defines a command
to pass on compilation to a different file:

%%%%%%%%%%%%%%%%%%%%%%%%%%%%%%%%%%%%%%%%
\DescribeMacro{\childdocforward}
The command |\childdocforward| redirects processing to
another source file:
%
\begin{center}
\begin{tabular}{l}
|\input{childdoc.def}|\\
|\childdocforward[|\textit{main}|]{|\textit{dest}|}|\\
\end{tabular}
\end{center}
%
The argument \textit{dest} is the destination file
(without extension).
It should be the main file or one of the child files.
Note that further \textsf{childdoc} directives
such as |\childdocof| and |\childdocforward|
in the indicated file will be processed in this form.
The optional argument \textit{main}
passes on directly to the main file \textit{main}
while pretending to compile the child \textit{dest}.
This form behaves as if \textit{dest}
issues |\childdocof{|\textit{main}|}| right away,
and no further \textsf{childdoc} directives will be processed.

%%%%%%%%%%%%%%%%%%%%%%%%%%%%%%%%%%%%%%%%
\DescribeMacro{\...prefix}
In the alternative form |\childdocforwardprefix|,
%
\begin{center}
\begin{tabular}{l}
|\input{childdoc.def}|\\
|\childdocforwardprefix[|\textit{main}|]{|\textit{prefix}|}{|\textit{dest}|}|
\end{tabular}
\end{center}
%
the destination file is determined by a pattern
depending on the current file:
To make this work, the current file must be called
`{\textit{prefix}\hspace{0.2em}\textit{suffix}}'
with \textit{prefix} matching precisely the argument.
Processing is then passed on to the file
`{\textit{dest}\hspace{0.2em}\textit{suffix}}'.
Surely, the same effect is achieved by
directly specifying the
argument `{\textit{dest}\hspace{0.2em}\textit{suffix}}'
in the first form.
However, that requires to set up a different file
for each child. With the alternative form of the command
all these files can have exactly the same content
which simplifies setting them up and maintaining them.

For example, the following file |draft.tex|
with a compilation flag |\version| as described in \secref{sec:flags}
compiles the main document as a draft:
%
\begin{center}
\begin{tabular}{l}
|\def\version{draft}|\\
|\input{childdoc.def}|\\
|\childdocforward{|\textit{main}|}|
\end{tabular}
\end{center}
%
Likewise, the following files |final|\textit{nn}|.tex|
compile the final version of the child document
|child|\textit{nn}|.tex|:
%
\begin{center}
\begin{tabular}{l}
|\def\version{final}|\\
|\input{childdoc.def}|\\
|\childdocforwardprefix{final}{child}|
\end{tabular}
\end{center}
%

Note that when several versions of a main file and/or of each child file
are to be generated, it may be convenient to set up a |Makefile| or
shell script to automatise the process.

%%%%%%%%%%%%%%%%%%%%%%%%%%%%%%%%%%%%%%%%%%%%%%%%%%%%%%%%%%%%%%%%%%%%%%%%%%%%%%%%
\subsection{Command Line Processing}
\label{sec:commandline}

The effect of redirection files can also be achieved by invoking
the \LaTeX{} compiler with a more elaborate command line.
Most conveniently this should be done as part
of a shell script or a |Makefile|.

When using \textsf{childdoc} in the main file, the following
command lines effectively perform a redirection
(note that depending on the shell being used,
backslashes may have to be doubled: `|\|' $\to$ `|\\|'):
%
\begin{center}
|... -jobname "|\textit{target}|" |\\|"|[\textit{flags}]%
|\input{childdoc.def}\childdocforward[|\textit{main}|]{|\textit{dest}|}"|
\end{center}
%
Here \textit{target} is the name of the output file,
\textit{main} is the name of the main file
and \textit{dest} is the name of the main or child file to be processed
(all filenames without extensions).
The optional argument \textit{main} can be omitted
if \textit{main} matches \textit{dest}.
Optionally, compilation \textit{flags} can be defined via |\def| commands.
This command line makes the \TeX{} engine believe
it is compiling the file \textit{target}
whose content is specified as the latter parameter.
The provided code then forwards the processing to
\textit{main} or \textit{dest} as described in \secref{sec:forward}.

%%%%%%%%%%%%%%%%%%%%%%%%%%%%%%%%%%%%%%%%%%%%%%%%%%%%%%%%%%%%%%%%%%%%%%%%%%%%%%%%
\subsection{Include by Input}
\label{sec:input}

Including child documents by |\include| has some restrictions by design.
Most notably, the content of a child document always occupies
its own set of pages; pages cannot be shared between child documents.
Usually, this behaviour makes perfect sense
because each child document contain an essential part of the document.
However, in some situations it may be desirable to compose
a document from a collection of parts
without having mandatory page breaks between then.
For this case, the package
provides a mechanism to include parts
by |\input| which can also be processed individually.
However, by construction this mechanism
requires manual handling of the content to be output.

%%%%%%%%%%%%%%%%%%%%%%%%%%%%%%%%%%%%%%%%
\DescribeMacro{\ifchilddocmanual}
The main file should be prepared as usual, see \secref{sec:include}.
However, the document body must make a distinction
between processing of an individual part and of the main document, e.g.:
%
\begin{center}
\begin{tabular}{l}
|\ifchilddocmanual|\\
|\input{\childdocname}|\\
|\||else|\\
\textit{document body with }|\input{|\textit{part}|}|\\
|\||fi|
\end{tabular}
\end{center}
%
The conditional |\ifchilddocmanual| is true whenever
a part to be included by |\input| is being compiled,
and the name of the part is stored in |\childdocname|.

%%%%%%%%%%%%%%%%%%%%%%%%%%%%%%%%%%%%%%%%
\DescribeMacro{\childdocby}
Each part to be included by |\input| should start with:
%
\begin{center}
\begin{tabular}{l}
|\input{childdoc.def}|\\
|\childdocby{|\textit{main}|}|\\
\end{tabular}
\end{center}
%
The directive |\childdocby| is similar to |\childdocof|
described in \secref{sec:include},
but the subsequent selection of content must be done manually.
To that end, both |\ifchilddoc| and |\ifchilddocmanual|
will be true upon processing of a part,
and the name of the part is stored in |\childdocname|.
Note that |\jobname| will be set to the filename of the current part
so that each part receives an individual |.aux| file
that does not interfere with the |.aux| file(s) of the main document.
This behaviour can be altered by the alternative form
|\childdocby[*]{|\textit{main}|}| (with a non-empty optional argument)
which uses the |.aux| file of the main document
by setting |\jobname| to \textit{main}.

%%%%%%%%%%%%%%%%%%%%%%%%%%%%%%%%%%%%%%%%%%%%%%%%%%%%%%%%%%%%%%%%%%%%%%%%%%%%%%%%
\subsection{Driver Development}
\label{sec:driver}

The \textsf{childdoc} mechanism can also be use for the development
of definition files such as \LaTeX{} styles or classes.
This case differs from the above setup with multiple parts
included by |\include| in that no |\includeonly| should be invoked.
This can be achieved by starting the include file
(before |\ProvidesPackage|) with:
%
\begin{center}
\begin{tabular}{l}
|\input{childdoc.def}|\\
|\childdocforward{|\textit{main}|}|\\
\end{tabular}
\end{center}
%
or alternatively with:
%
\begin{center}
\begin{tabular}{l}
|\input{childdoc.def}|\\
|\childdocby{|\textit{main}|}|\\
\end{tabular}
\end{center}
%
Both forms have slightly different effects as described above.
The main file is prepared as usual, see \secref{sec:include}.

%%%%%%%%%%%%%%%%%%%%%%%%%%%%%%%%%%%%%%%%%%%%%%%%%%%%%%%%%%%%%%%%%%%%%%%%%%%%%%%%
\subsection{Legacy Detection}
\label{sec:detection}

The directive |\childdocmain| in the main file can detect
whether the complete document or merely a child is to be compiled
even without using the directive |\childdocof|.
This method is deprecated because it is less robust
and there is no compelling reason to use it;
it is merely provided for backward compatibility
and it may be removed in future versions.

If the detection mechanism is to be used,
it is mandatory to correctly specify
the filename of the main file as the argument of |\childdocmain|:
%
\begin{center}
\begin{tabular}{l}
|\input{childdoc.def}|\\
|\childdocmain{|\textit{main}|}|\\
\end{tabular}
\end{center}
%
If |\jobname| does not match the argument \textit{main} of |\childdocmain|,
it is assumed that |\jobname| points to the child file to be compiled.
When using |\childdocmain| with the main file specified as argument,
it suffices to start a child file
with just |\input{|\textit{main}|}|
without loading of the package and using |\childdocof|.
If instead all processing is done
with the appropriate \textsf{childdoc} directives,
the argument of \textit{main} of |\childdocmain| can be empty.

An alternative version of the command line processing described
in \secref{sec:commandline} using the detection mechanism reads:
%
\begin{center}
|... -jobname "|\textit{target}|" "|[\textit{flags}]%
[|\def\jobname{|\textit{dest}|}|]|\input{|\textit{main}|}"|
\end{center}

%%%%%%%%%%%%%%%%%%%%%%%%%%%%%%%%%%%%%%%%%%%%%%%%%%%%%%%%%%%%%%%%%%%%%%%%%%%%%%%%
\subsection{Manual Code}
\label{sec:manual}

In case one cannot be certain whether the definitions file |childdoc.def|
is installed on the target \TeX{} distribution
and one prefers not to ship it,
it is conceivable to paste a few relevant commands into the sources.

To that end, drop all statements |\input{childdoc.def}|
and perform the replacements as outlined below.
Instead of |\childdocmain{|\textit{main}|}| add the following code
to the top of the main file:
%
\begin{center}
\begin{tabular}{l}
|\||ifdefined\childdocname\endinput\||fi\newif\ifchilddoc|\\
|\edef\childdocname{\scantokens\expandafter{\jobname\noexpand}}|\\
|\def\childdocmain{|\textit{main}|}\||ifx\childdocmain\childdocname\||else|\\
|\childdoctrue\includeonly{\childdocname}\let\jobname\childdocmain\||fi|\\
\end{tabular}
\end{center}
%
Instead of |\childdocof{|\textit{main}|}| just include the main file
at the top of each child file:
%
\begin{center}
|\input{|\textit{main}|}|
\end{center}
%
A simple redirection |\childdocforward{|\textit{dest}|}| is achieved by:
%
\begin{center}
|\def\jobname{|\textit{dest}|}\input{\jobname}|
\end{center}
%
The redirection with prefix
|\childdocforwardprefix[|\textit{prefix}|]{|\textit{dest}|}|
is accomplished by:
%
\begin{center}
\begin{tabular}{l}
|{\edef\jobname{\scantokens\expandafter{\jobname\noexpand}}|\\
|\def\redirectjob |\textit{prefix}|#1~~~{\gdef\jobname{|\textit{dest}|#1}}|\\
|\expandafter\redirectjob\jobname~~~}\input{\jobname}|
\end{tabular}
\end{center}

In an alternative approach,
child documents can be compiled by a specific command line
without additional code or specific definitions:
%
\begin{center}
|... -jobname "|\textit{target}|" "|[\textit{flags}]%
|\includeonly{|\textit{dest}|}\input{|\textit{main}|}"|
\end{center}
%

%%%%%%%%%%%%%%%%%%%%%%%%%%%%%%%%%%%%%%%%%%%%%%%%%%%%%%%%%%%%%%%%%%%%%%%%%%%%%%%%
%%%%%%%%%%%%%%%%%%%%%%%%%%%%%%%%%%%%%%%%%%%%%%%%%%%%%%%%%%%%%%%%%%%%%%%%%%%%%%%%
\section{Information}

%%%%%%%%%%%%%%%%%%%%%%%%%%%%%%%%%%%%%%%%%%%%%%%%%%%%%%%%%%%%%%%%%%%%%%%%%%%%%%%%
\subsection{Copyright}

Copyright \copyright{} 2017--2018 Niklas Beisert

This work may be distributed and/or modified under the
conditions of the \LaTeX{} Project Public License, either version 1.3
of this license or (at your option) any later version.
The latest version of this license is in
  \url{http://www.latex-project.org/lppl.txt}
and version 1.3 or later is part of all distributions of \LaTeX{}
version 2005/12/01 or later.

This work has the LPPL maintenance status `maintained'.

The Current Maintainer of this work is Niklas Beisert.

This work consists of the files |README.txt|, |childdoc.ins| and |childdoc.dtx|
as well as the derived files |childdoc.def|, |cdocsamp.tex|
with |cdocsch1.tex|, |cdocsch2.tex|, |cdocspt3.tex|, |cdocspt4.tex|,
|cdocsdrf.tex|, |cdocsfn1.tex|, |cdocsfn2.tex|
as well as |childdoc.pdf|.

%%%%%%%%%%%%%%%%%%%%%%%%%%%%%%%%%%%%%%%%%%%%%%%%%%%%%%%%%%%%%%%%%%%%%%%%%%%%%%%%
\subsection{Files and Installation}

The package consists of the files:
%
\begin{center}
\begin{tabular}{ll}
    |README.txt|   & readme file \\
    |childdoc.ins| & installation file \\
    |childdoc.dtx| & source file \\
    |childdoc.def| & definition file \\
    |cdocsamp.tex| & sample main file \\
    |cdocsch1.tex| & sample include file \\
    |cdocsch2.tex| & sample include file \\
    |cdocspt3.tex| & sample part file \\
    |cdocspt4.tex| & sample part file \\
    |cdocsdrf.tex| & sample redirection file \\
    |cdocsfn1.tex| & sample redirection file \\
    |cdocsfn2.tex| & sample redirection file \\
    |childdoc.pdf| & manual
\end{tabular}
\end{center}
%
The distribution consists of the files
|README.txt|, |childdoc.ins| and |childdoc.dtx|.
%
\begin{itemize}
\item
Run (pdf)\LaTeX{} on |childdoc.dtx|
to compile the manual |childdoc.pdf| (this file).
\item
Run \LaTeX{} on |childdoc.ins| to create the definitions file |childdoc.def|
and the sample |cdocsamp.tex| with include files
|cdocsch1.tex|, |cdocsch2.tex|, |cdocspt3.tex|, |cdocspt4.tex|,
|cdocsdrf.tex|, |cdocsfn1.tex|, |cdocsfn2.tex|.
Then copy the file |childdoc.def| to an appropriate directory of your \LaTeX{}
distribution, e.g.\ \textit{texmf-root}|/tex/latex/childdoc|.
\end{itemize}

%%%%%%%%%%%%%%%%%%%%%%%%%%%%%%%%%%%%%%%%%%%%%%%%%%%%%%%%%%%%%%%%%%%%%%%%%%%%%%%%
\subsection{Related CTAN Packages}

There are several other packages which offer a similar functionality:
%
\begin{itemize}
\item
The packages
\href{http://ctan.org/pkg/docmute}{\textsf{docmute}},
\href{http://ctan.org/pkg/includex}{\textsf{includex}} and
\href{http://ctan.org/pkg/standalone}{\textsf{standalone}}
provide commands to include only the document body of
a child file thus allowing both files to be compiled individually.
\item
The packages \href{http://ctan.org/pkg/subdocs}{\textsf{subdocs}}
and \href{http://ctan.org/pkg/subfiles}{\textsf{subfiles}}
provide structures in which the main and child documents can be
encapsulated and allowing them to be compiled individually.
The inclusion mechanism is different from the conventional |\include|.
\item
The package \href{http://ctan.org/pkg/combine}{\textsf{combine}}
is an elaborate solution to combine several documents into one.
\end{itemize}
%
See also the CTAN topic \href{http://ctan.org/topic/subdocs}{\textsf{subdocs}}
for further related packages.
The present package differs from the above solutions in that
a document structure constructed with the conventional |\include| mechanism
just needs two extra commands at the top of every file
such that all constituent files can be compiled individually.

%%%%%%%%%%%%%%%%%%%%%%%%%%%%%%%%%%%%%%%%%%%%%%%%%%%%%%%%%%%%%%%%%%%%%%%%%%%%%%%%
%\subsection{Feature Suggestions}
%
%The following is a list of features which may be useful for future
%versions of this package:
%%
%\begin{itemize}
%\item
%\ldots
%\end{itemize}

%%%%%%%%%%%%%%%%%%%%%%%%%%%%%%%%%%%%%%%%%%%%%%%%%%%%%%%%%%%%%%%%%%%%%%%%%%%%%%%%
\subsection{Revision History}

%%%%%%%%%%%%%%%%%%%%%%%%%%%%%%%%%%%%%%%%
\paragraph{v2.0:} 2018/12/30

\begin{itemize}
\item
immediate forward processing
\item
added |\childdocby| mechanism
\item
manual restructured
\end{itemize}

%%%%%%%%%%%%%%%%%%%%%%%%%%%%%%%%%%%%%%%%
\paragraph{v1.6:} 2018/01/17

\begin{itemize}
\item
application for development of include files
\item
corrections to manual
\end{itemize}

%%%%%%%%%%%%%%%%%%%%%%%%%%%%%%%%%%%%%%%%
\paragraph{v1.5:} 2017/05/21

\begin{itemize}
\item
more complete structuring introduced
\item
|\childdocof| introduced
\item
|\childdoc| renamed to |\childdocmain|
\item
|\childredirect| renamed to |\childdocforward| and |\childdocforwardprefix|
and functionality expanded
\end{itemize}

%%%%%%%%%%%%%%%%%%%%%%%%%%%%%%%%%%%%%%%%
\paragraph{v1.0:} 2017/04/27

\begin{itemize}
\item
manual and install package
\item
first version published on CTAN
\end{itemize}

%%%%%%%%%%%%%%%%%%%%%%%%%%%%%%%%%%%%%%%%
\paragraph{v0.6:} 2017/04/26

\begin{itemize}
\item
redirection mechanism added
\end{itemize}

%%%%%%%%%%%%%%%%%%%%%%%%%%%%%%%%%%%%%%%%
\paragraph{v0.5:} 2017/04/26

\begin{itemize}
\item
functionality in definition file
\end{itemize}


%%%%%%%%%%%%%%%%%%%%%%%%%%%%%%%%%%%%%%%%%%%%%%%%%%%%%%%%%%%%%%%%%%%%%%%%%%%%%%%%
%%%%%%%%%%%%%%%%%%%%%%%%%%%%%%%%%%%%%%%%%%%%%%%%%%%%%%%%%%%%%%%%%%%%%%%%%%%%%%%%
%%%%%%%%%%%%%%%%%%%%%%%%%%%%%%%%%%%%%%%%%%%%%%%%%%%%%%%%%%%%%%%%%%%%%%%%%%%%%%%%
\appendix

\settowidth\MacroIndent{\rmfamily\scriptsize 000\ }

 \DocInput{childdoc.dtx}

\end{document}
%</driver>
% \fi
%
% %%%%%%%%%%%%%%%%%%%%%%%%%%%%%%%%%%%%%%%%%%%%%%%%%%%%%%%%%%%%%%%%%%%%%%%%%%%%%%
% %%%%%%%%%%%%%%%%%%%%%%%%%%%%%%%%%%%%%%%%%%%%%%%%%%%%%%%%%%%%%%%%%%%%%%%%%%%%%%
% \section{Sample}
%\iffalse
%<*samplemain>
%\fi
%
% The following presents a sample document
% with two chapters, two parts, a title page,
% a compile flag as well as three forwarding files to set the flag.
% It consists of eight |.tex| files:
% \begin{center}
% \begin{tabular}{ll}
% |cdocsamp.tex|&main file\\
% |cdocsch1.tex|&include file for chapter 1\\
% |cdocsch2.tex|&include file for chapter 2\\
% |cdocspt3.tex|&include file for part 3\\
% |cdocspt4.tex|&include file for part 4\\
% |cdocsdrf.tex|&forwarding file for main file in draft mode\\
% |cdocsfi1.tex|&forwarding file for final version of chapter 1\\
% |cdocsfi2.tex|&forwarding file for final version of chapter 2\\
% \end{tabular}
% \end{center}
% Each of the eight files can be compiled directly by the \LaTeX{} compiler.
%
% %%%%%%%%%%%%%%%%%%%%%%%%%%%%%%%%%%%%%%
% \paragraph{Main File.}
%
% The main file is called |cdocsamp.tex|.
%
% Load the \textsf{childdoc} definitions and
% declare the filename for the main document:
%    \begin{macrocode}
\input{childdoc.def}
\childdocmain{}
%    \end{macrocode}

% Optional override for |\version| flag:
%    \begin{macrocode}
%%\ifchilddoc\else\providecommand{\version}{draft}\fi
%    \end{macrocode}

% Define the default values for the |\version| flag
% (|final| for the main file and |draft| for childs):
%    \begin{macrocode}
\ifchilddoc
\providecommand{\version}{draft}
\else
\providecommand{\version}{final}
\fi
%    \end{macrocode}

% Load the standard document class:
%    \begin{macrocode}
\documentclass[12pt]{article}
%    \end{macrocode}

% Start the document body:
%    \begin{macrocode}
\begin{document}
%    \end{macrocode}

% Declare a title page.
% Print title, part of document being processed and version flag:
%    \begin{macrocode}
\addtocounter{page}{-1}
\begin{center}
{\LARGE\bfseries{}childdoc example\par}
\vspace{1cm}
\ifchilddoc
\ifchilddocmanual part\else chapter\fi:
`\childdocname' of `\childdocjob'\par
\else
main document: `\childdocjob'\par
\fi
version: \version\par
\end{center}
\newpage
%    \end{macrocode}

% Manually include selected file,
% otherwise process as usual:
%    \begin{macrocode}
\ifchilddocmanual
\section*{part `\childdocname'}
\input{\childdocname}
\else
%    \end{macrocode}

% Include the two chapters:
%    \begin{macrocode}
\include{cdocsch1}
\include{cdocsch2}
%    \end{macrocode}

% Include the two parts unless only chapters should be displayed:
%    \begin{macrocode}
\ifchilddoc\else
\section{part three}
\input{cdocspt3}
\section{part four}
\input{cdocspt4}
\fi
%    \end{macrocode}

% Process as usual until here:
%    \begin{macrocode}
\fi
%    \end{macrocode}

% End of document body:
%    \begin{macrocode}
\end{document}
%    \end{macrocode}
%\iffalse
%</samplemain>
%\fi
%
% %%%%%%%%%%%%%%%%%%%%%%%%%%%%%%%%%%%%%%
% \paragraph{Chapter Include Files.}
%
% The include files are called |cdocsch1.tex| and |cdocsch2.tex|.
%
%\iffalse
%<*samplechap1|samplechap2>
%\fi

% Optional override for |\version| flag:
%    \begin{macrocode}
%%\providecommand{\version}{final}
%    \end{macrocode}

% Include the main document:
%    \begin{macrocode}
\input{childdoc.def}
\childdocof{cdocsamp}
%    \end{macrocode}

%\iffalse
%</samplechap1|samplechap2>
%\fi
%
%\iffalse
%<*samplechap1>
%\fi
% Some text for chapter 1:
%    \begin{macrocode}
\section{one}
some text in chapter one
%    \end{macrocode}

%\iffalse
%</samplechap1>
%\fi
% Some text for chapter 2:
%\iffalse
%<*samplechap2>
%\fi
%    \begin{macrocode}
\section{two}
more text in chapter two
%    \end{macrocode}

%\iffalse
%</samplechap2>
%\fi
%
% %%%%%%%%%%%%%%%%%%%%%%%%%%%%%%%%%%%%%%
% \paragraph{Part Include Files.}
%
% The include files are called |cdocspt3.tex| and |cdocspt4.tex|.
%
%\iffalse
%<*samplepart3|samplepart4>
%\fi

% Optional override for |\version| flag:
%    \begin{macrocode}
%%\providecommand{\version}{final}
%    \end{macrocode}

% Include the main document:
%    \begin{macrocode}
\input{childdoc.def}
\childdocby{cdocsamp}
%    \end{macrocode}

%\iffalse
%</samplepart3|samplepart4>
%\fi
%
%\iffalse
%<*samplepart3>
%\fi
% Some text for part 3:
%    \begin{macrocode}
some text in part three
%    \end{macrocode}

%\iffalse
%</samplepart3>
%\fi
% Some text for part 4:
%\iffalse
%<*samplepart4>
%\fi
%    \begin{macrocode}
more text in part four
%    \end{macrocode}

%\iffalse
%</samplepart4>
%\fi
%
% %%%%%%%%%%%%%%%%%%%%%%%%%%%%%%%%%%%%%%
% \paragraph{Forwarding for a Complete Draft.}
%
% The following forwarding file |cdocsdrf.tex|
% compiles the main document in draft mode:
%\iffalse
%<*sampledraft>
%\fi
%    \begin{macrocode}
\def\version{draft}
\input{childdoc.def}
\childdocforward{cdocsamp}
%    \end{macrocode}

%\iffalse
%</sampledraft>
%\fi
%
% %%%%%%%%%%%%%%%%%%%%%%%%%%%%%%%%%%%%%%
% \paragraph{Forwarding for Final Version of the Chapters.}
%
% The following forwarding files |cdocsfn1.tex| and |cdocsfn2.tex|
% (with identical content)
% compile the final versions of the child documents
% |cdocsch1.tex| and |cdocsch2.tex|, respectively:
%\iffalse
%<*samplefinal>
%\fi
%    \begin{macrocode}
\def\version{final}
\input{childdoc.def}
\childdocforwardprefix[cdocsamp]{cdocsfn}{cdocsch}
%    \end{macrocode}

%\iffalse
%</samplefinal>
%\fi
%
% %%%%%%%%%%%%%%%%%%%%%%%%%%%%%%%%%%%%%%
% \paragraph{Command Line Processing.}
%
% The following three command lines generate the output files
% |cdocscld|, |cdocscl1| and |cdocscl2|
% which should be identical to
% |cdocsdrf|, |cdocsch1| and |cdocsfn2|, respectively:
% \begin{center}
% \begin{tabular}{l}
% |latex -jobname cdocscld \|\\
% |  "\def\version{draft}\input{childdoc.def}\childdocforward{cdocsamp}"|\\
% |latex -jobname cdocscl1 \|\\
% |  "\input{childdoc.def}\childdocforward[cdocsamp]{cdocsch1}"|\\
% |latex -jobname cdocscl2 \|\\
% |  "\def\version{final}\input{childdoc.def}\childdocforward{cdocsch2}"|
% \end{tabular}
% \end{center}
% Note that the trailing backslash on each first line
% merely continues the input to the second line
% (for convenient cut ant paste).
% Furthermore, the command |latex| can be replaced by any
% of its alternative versions such as |pdflatex|.
%
% %%%%%%%%%%%%%%%%%%%%%%%%%%%%%%%%%%%%%%%%%%%%%%%%%%%%%%%%%%%%%%%%%%%%%%%%%%%%%%
% %%%%%%%%%%%%%%%%%%%%%%%%%%%%%%%%%%%%%%%%%%%%%%%%%%%%%%%%%%%%%%%%%%%%%%%%%%%%%%
% \section{Implementation}
%\iffalse
%<*package>
%\fi
%
% This section describes the definitions file |childdoc.def|.

% The definitions cannot be loaded using |\usepackage| or |\RequirePackage|
% which has a mechanism to prevent loading a style file more than once.
% When loading the definitions by means of |\input|
% multiple instances have to be prevented manually:
%\iffalse
%This code needs to be before the `\ProvidesFile' directive
%which is defined at the beginning of this file.
%Therefore it is also placed there and commented out here.
%</package>
%<*discard>
%\fi
%    \begin{macrocode}
\ifdefined\childdocmain\endinput\fi
%    \end{macrocode}
%\iffalse
%</discard>
%<*package>
%\fi
%
% \macro{\ifchilddoc}
% \macro{\ifchilddocmanual}
% The conditional |\ifchilddoc| tells whether a
% child (true) or main (false) document is being compiled.
% The conditional |\ifchilddocmanual| tells whether
% the |\includeonly| mechanism is used (false) or
% the selection of child files must be performed manually (true).
% The definitions initialise to false:
%    \begin{macrocode}
\newif\ifchilddoc
\newif\ifchilddocmanual
%    \end{macrocode}

% \macro{\childdocname}
% \macro{\childdocjob}
% The macro |\childdocname| stores the name of the main document
% to be compiled. The macro |\childdocjob| stores the name of
% the document on which the \LaTeX{} compiler was originally invoked.
% The content of |\jobname| cannot be compared
% to filenames specified in the source due to different catcodes.
% The following code rescans |\jobname|, stores the result
% in |\childdocname| and saves a copy in |\childdocjob|:
%    \begin{macrocode}
\edef\childdocname{\scantokens\expandafter{\jobname\noexpand}}
\let\childdocjob\childdocname
%    \end{macrocode}

% \macro{\childdocdisable}
% The macro |\childdocdisable| prevents the main file
% from being processed more than once.
% At this stage, the main document command |\childdocmain|
% is assumed to be called once again where it should do nothing.
% Any subsequent call to it should prevent
% a secondary processing of the main document
% It overwrites the forwarding commands
% |\childdocof| and |\childdocforward|
% with empty macros to prevent further inclusions of the main document:
%    \begin{macrocode}
\newcommand{\childdocdisable}
{
  \renewcommand{\childdocmain}[1]{\renewcommand{\childdocmain}[1]{\endinput}}
  \renewcommand{\childdocof}[1]{}
  \renewcommand{\childdocby}[2][]{}
  \renewcommand{\childdocforward}[2][]{}
  \renewcommand{\childdocdisable}{}
}
%    \end{macrocode}

% \macro{\childdocmain}
% The macro |\childdocmain| is to be called at the top of the main file
% with nothing or the main filename (without extension) as argument.
% First, it breaks loops.
% If the argument is not empty and does not match |\childdocname|
% (which is set by the first inclusion of |childdoc.def|),
% |\ifchilddoc| is set to true, |\includeonly| is applied to the child file
% and |\jobname| is set to the main file
% (for proper handling of |.aux| files):
%    \begin{macrocode}
\newcommand{\childdocmain}[1]
{
  \childdocdisable\childdocmain{}
  \if?#1?\else
    \begingroup
      \def\childdoctmp{#1}
      \ifx\childdoctmp\childdocname
        \def\childdoctmp{}
      \else
        \def\childdoctmp
        {
          \childdoctrue
          \includeonly{\childdocname}
          \def\childdocjob{#1}
          \def\jobname{#1}
        }
      \fi
      \expandafter
    \endgroup
    \childdoctmp
  \fi
}
%    \end{macrocode}

% \macro{\childdocof}
% The command |\childdocof| redirects
% compilation to the main file |#1|.
%    \begin{macrocode}
\newcommand{\childdocof}[1]
{
  \childdocdisable
  \childdoctrue
  \includeonly{\childdocname}
  \def\jobname{#1}
  \def\childdocjob{#1}
  \input{#1}
}
%    \end{macrocode}

% \macro{\childdocby}
% The command |\childdocby| ....
%    \begin{macrocode}
\newcommand{\childdocby}[2][]
{
  \childdocdisable
  \childdoctrue
  \childdocmanualtrue
  \if?#1?\else
    \def\jobname{#2}
  \fi
  \def\childdocjob{#2}
  \input{#2}
  \endinput
}
%    \end{macrocode}

% \macro{\childdocforward}
% The command |\childdocforward| redirects
% compilation to the main file or
% (if the optional argument is given) a child file.
% Parameters are set as if the main file
% or a child file starting with |\childdocof| was compiled.
% Then compilation is handed over to the main file:
%    \begin{macrocode}
\newcommand{\childdocforward}[2][]
{
  \begingroup
    \if?#1?
      \def\childdoctmp
      {
        \def\childdocname{#2}
        \def\childdocjob{#2}
        \def\jobname{#2}
        \input{#2}
        \endinput
      }
    \else
      \def\childdoctmp
      {
        \childdocdisable
        \def\childdocname{#2}
        \childdoctrue
        \includeonly{#2}
        \def\childdocjob{#1}
        \def\jobname{#1}
        \input{#1}
        \endinput
      }
    \fi
    \expandafter
  \endgroup
  \childdoctmp
}
%    \end{macrocode}

% \macro{\childdocforwardprefix}
% The command |\childdocforwardprefix| redirects
% compilation to the main or a child file by means of a pattern.
% The prefix |#1| in the current filename is replaced by |#2|
% and the suffix of the current filename is kept
% (it is assumed that the filename does not contain the substring `|~~~|'
% which is used as a delimiter).
% Compilation is handed over to the new file by |\childdocforward|:
%    \begin{macrocode}
\newcommand{\childdocforwardprefix}[3][]
{
  \begingroup
    \def\childdocextract #2##1~~~{\def\childdoctmp{\childdocforward[#1]{#3##1}}}
    \expandafter\childdocextract\childdocname~~~
    \expandafter
  \endgroup
  \childdoctmp
}
%    \end{macrocode}

% \macro{\childdoc}
% The deprecated macro |\childdoc| is a legacy version of |\childdocmain|:
%    \begin{macrocode}
\newcommand{\childdoc}{\childdocmain}
%    \end{macrocode}

% \macro{\childdocredirect}
% The deprecated macro |\childdocredirect| is a legacy version
% of |\childdocforward| and |\childdocforwardprefix|:
%    \begin{macrocode}
\newcommand{\childdocredirect}[2][]
{
  \begingroup
    \if?#1?
      \def\childdoctmp{\childdocforward{#2}}
    \else
      \def\childdoctmp{\childdocforwardprefix{#1}{#2}}
    \fi
    \expandafter
  \endgroup
  \childdoctmp
}
%    \end{macrocode}

%\iffalse
%</package>
%\fi
%
\endinput
|\\
|\childdocby{|\textit{main}|}|\\
\end{tabular}
\end{center}
%
The directive |\childdocby| is similar to |\childdocof|
described in \secref{sec:include},
but the subsequent selection of content must be done manually.
To that end, both |\ifchilddoc| and |\ifchilddocmanual|
will be true upon processing of a part,
and the name of the part is stored in |\childdocname|.
Note that |\jobname| will be set to the filename of the current part
so that each part receives an individual |.aux| file
that does not interfere with the |.aux| file(s) of the main document.
This behaviour can be altered by the alternative form
|\childdocby[*]{|\textit{main}|}| (with a non-empty optional argument)
which uses the |.aux| file of the main document
by setting |\jobname| to \textit{main}.

%%%%%%%%%%%%%%%%%%%%%%%%%%%%%%%%%%%%%%%%%%%%%%%%%%%%%%%%%%%%%%%%%%%%%%%%%%%%%%%%
\subsection{Driver Development}
\label{sec:driver}

The \textsf{childdoc} mechanism can also be use for the development
of definition files such as \LaTeX{} styles or classes.
This case differs from the above setup with multiple parts
included by |\include| in that no |\includeonly| should be invoked.
This can be achieved by starting the include file
(before |\ProvidesPackage|) with:
%
\begin{center}
\begin{tabular}{l}
|% \iffalse
%
% childdoc.dtx Copyright (C) 2017-2018 Niklas Beisert
%
% This work may be distributed and/or modified under the
% conditions of the LaTeX Project Public License, either version 1.3
% of this license or (at your option) any later version.
% The latest version of this license is in
%   http://www.latex-project.org/lppl.txt
% and version 1.3 or later is part of all distributions of LaTeX
% version 2005/12/01 or later.
%
% This work has the LPPL maintenance status `maintained'.
%
% The Current Maintainer of this work is Niklas Beisert.
%
% This work consists of the files childdoc.dtx and childdoc.ins
% and the derived files childdoc.def and cdocsamp.tex with
% cdocsch1.tex, cdocsch2.tex, cdocsdrf.tex, cdocsfn1.tex, cdocsfn2.tex.
%
%<package>\ifdefined\childdocmain\endinput\fi
%<package>\ProvidesFile{childdoc.def}[2018/12/30 v2.0 child document driver]
%<samplemain>\ProvidesFile{cdocsamp.tex}[2018/12/30 v2.0 sample for childdoc]
%<*driver>
%\ProvidesFile{childdoc.drv}[2018/12/30 v2.0 childdoc reference manual file]
\PassOptionsToClass{10pt,a4paper}{article}
\documentclass{ltxdoc}

\usepackage[margin=35mm]{geometry}
\usepackage{hyperref}
\usepackage{hyperxmp}
\usepackage[usenames]{color}

\hypersetup{colorlinks=true}
\hypersetup{pdfstartview=FitH}
\hypersetup{pdfpagemode=UseNone}
\hypersetup{pdfsource={}}
\hypersetup{pdflang={en-UK}}
\hypersetup{pdfcopyright={Copyright 2017-2018 Niklas Beisert.
  This work may be distributed and/or modified under the
  conditions of the LaTeX Project Public License, either version 1.3
  of this license or (at your option) any later version.}}
\hypersetup{pdflicenseurl={http://www.latex-project.org/lppl.txt}}
\hypersetup{pdfcontactaddress={ETH Zurich, ITP, HIT K,
  Wolfgang-Pauli-Strasse 27}}
\hypersetup{pdfcontactpostcode={8093}}
\hypersetup{pdfcontactcity={Zurich}}
\hypersetup{pdfcontactcountry={Switzerland}}
\hypersetup{pdfcontactemail={nbeisert@itp.phys.ethz.ch}}
\hypersetup{pdfcontacturl={http://people.phys.ethz.ch/\xmptilde nbeisert/}}

\newcommand{\secref}[1]{\hyperref[#1]{section \ref*{#1}}}

\parskip1ex
\parindent0pt
\let\olditemize\itemize
\def\itemize{\olditemize\parskip0pt}

\begin{document}

\title{The \textsf{childdoc} Package}
\hypersetup{pdftitle={The childdoc Package}}
\author{Niklas Beisert\\[2ex]
  Institut f\"ur Theoretische Physik\\
  Eidgen\"ossische Technische Hochschule Z\"urich\\
  Wolfgang-Pauli-Strasse 27, 8093 Z\"urich, Switzerland\\[1ex]
  \href{mailto:nbeisert@itp.phys.ethz.ch}
  {\texttt{nbeisert@itp.phys.ethz.ch}}}
\hypersetup{pdfauthor={Niklas Beisert}}
\hypersetup{pdfsubject={Manual for the LaTeX2e Package childdoc}}
\date{30 December 2018, \textsf{v2.0}}
\maketitle

\begin{abstract}\noindent
\textsf{childdoc} is a \LaTeXe{} package
that enables the direct compilation
of document sections included by |\include|
to individual files.
\end{abstract}

\begingroup
\parskip0ex
\tableofcontents
\endgroup

%%%%%%%%%%%%%%%%%%%%%%%%%%%%%%%%%%%%%%%%%%%%%%%%%%%%%%%%%%%%%%%%%%%%%%%%%%%%%%%%
%%%%%%%%%%%%%%%%%%%%%%%%%%%%%%%%%%%%%%%%%%%%%%%%%%%%%%%%%%%%%%%%%%%%%%%%%%%%%%%%
\section{Introduction}

\LaTeX{} provides a mechanism to structure a large document (such as a book)
into a main file and several child files (containing the chapters)
using the |\include| command.
This mechanism is beneficial for documents
which span hundreds of pages in order to
make the source file(s) more manageable.
Moreover, compilation can be restricted to
selected child files by means of the |\includeonly| command.
The latter feature can be used to reduce the compilation time while editing
(this was significantly more useful in the earlier days of \LaTeX{})
or to generate a smaller document which is easier to navigate.
Another application of |\includeonly| is to generate
documents consisting of selected parts of the complete document.

However, there are a few drawbacks of the plain |\include| mechanism:
\begin{itemize}
\item
The child files cannot be compiled on their own,
they can only be compiled via the main file.
A naive editing environment
(such as a text editor with an option
to have the current file processed by \LaTeX)
may require one to switch to the main file before compiling;
attempting to compile the child file produces errors.
\item
The main file must be modified (each time)
to adjust the |\includeonly| command
to the present needs. This easily leaves the main file in a messy state.
\item
The generated document will always carry the filename
of the main document. This is inconvenient if
several child files are to be compiled and
to be kept for distribution.
\end{itemize}

The present package provides a simple interface
to make child files individually compilable by \LaTeX{}.
Compiling a child file then has the same effect as compiling
the main file with an |\includeonly| command
to select the appropriate child.
Moreover the generated document will carry the name of the child
rather than the main file.
This resolves all three above issues.

This feature is meant to make the editing of books,
thesis documents and lecture notes somewhat more convenient.
However, the package can also be used efficiently for
composing a series of documents (such as exercise sheets)
which are typically distributed individually.
It then assists the author in generating the individual documents
(potentially in different versions)
as well as a document containing the collected series.
Another application is in developing style files
or other kinds of included material
where compilation of the style file could redirect
to a sample or test file.

%%%%%%%%%%%%%%%%%%%%%%%%%%%%%%%%%%%%%%%%%%%%%%%%%%%%%%%%%%%%%%%%%%%%%%%%%%%%%%%%
%%%%%%%%%%%%%%%%%%%%%%%%%%%%%%%%%%%%%%%%%%%%%%%%%%%%%%%%%%%%%%%%%%%%%%%%%%%%%%%%
\section{Usage}

First of all, the package \textsf{childdoc} is \emph{not} a standard
\LaTeXe{} |.sty| style file! Therefore it needs to be invoked in
a non-standard way.

%%%%%%%%%%%%%%%%%%%%%%%%%%%%%%%%%%%%%%%%%%%%%%%%%%%%%%%%%%%%%%%%%%%%%%%%%%%%%%%%
\subsection{Included Files}
\label{sec:include}

%%%%%%%%%%%%%%%%%%%%%%%%%%%%%%%%%%%%%%%%
\DescribeMacro{\childdocmain}
To use the package, add the commands
\begin{center}
\begin{tabular}{l}
|\input{childdoc.def}|\\
|\childdocmain{}|\\
\end{tabular}
\end{center}
at the very top of the main \LaTeX{} file,
in particular \emph{before} the |\documentclass| statement!
The argument of |\childdocmain| should be left empty
(but it must be present).

%%%%%%%%%%%%%%%%%%%%%%%%%%%%%%%%%%%%%%%%
\DescribeMacro{\childdocof}
Furthermore, add the commands
\begin{center}
\begin{tabular}{l}
|\input{childdoc.def}|\\
|\childdocof{|\textit{main}|}|\\
\end{tabular}
\end{center}
at the top of every child file \textit{child}
which is included by |\include{|\textit{child}|}|
from within the main file
(or at least for those files to be compiled individually).
The argument \textit{main} must be the filename of the main file.

There are a couple of
considerations in setting up the main and child documents:

%%%%%%%%%%%%%%%%%%%%%%%%%%%%%%%%%%%%%%%%
\paragraph{Restrictions.}

Please note the following restrictions:
\begin{itemize}
\item
|\childdocmain| must be called with one argument \textit{main}
to ensure compatibility with earlier version of the package.
It must either be empty (|\childdocmain{}|)
or precisely match the filename of the main file in which it is specified.
See \secref{sec:detection} for further information.
\item
The filename \textit{main} must be specified without the |.tex| extension.
\item
The filename \textit{main} is case sensitive
(even in case-insensitive file systems)
due to internal string comparison.
\item
The argument \textit{main} should be fully expanded, it cannot be a macro.
\item
Subdirectories and special characters should be avoided in filenames.
\item
The command |\childdocmain{|\textit{main}|}| must be followed by a whitespace.
It should not be followed immediately by another command
or by a comment mark `|%|'.
This is because the \TeX{} parser reads the token immediately following
the argument of |\childdocmain| and puts it
at the beginning of every child section;
however, a white\-space is ignored.
\end{itemize}

%%%%%%%%%%%%%%%%%%%%%%%%%%%%%%%%%%%%%%%%
\paragraph{Content of Main File.}

It is advisable to place all content in the child files included by |\include|.
Any output contained in the main file will appear in all child documents
unless suppressed manually;
it cannot be suppressed automatically by the |\includeonly| directive
and thus should normally be avoided.
A method to include some content in the main file
by means of conditional processing is described in \secref{sec:conditional}.

%%%%%%%%%%%%%%%%%%%%%%%%%%%%%%%%%%%%%%%%
\paragraph{Page Numbering.}

When only a part of the document is compiled,
the appropriate numbering of pages
(as well as other status parameters)
is determined from the |.aux| files.
The latter contain information from previous passes.
However this information needs to propagate through
all intermediate child documents.
Therefore the page numbering in child documents may well
be inconsistent until the complete document is compiled at least once.

A useful (if unconventional) way to always ensure a consistent
page numbering is to restart the numbering in each child document
and denote the pages by `\textit{child}|.|\textit{page}'
where \textit{child} represents the chapter/section number of the child file.
This can be achieved by the command
|\numberwithin{page}{|\textit{child}|}|
of the \textsf{amsmath} package
where \textit{child} can be |chapter| or |section|
depending on the chosen structuring.
Alternatively, one can modify the macro |\thepage| appropriately
and reset the counter |page| at the start of each child file.

%%%%%%%%%%%%%%%%%%%%%%%%%%%%%%%%%%%%%%%%%%%%%%%%%%%%%%%%%%%%%%%%%%%%%%%%%%%%%%%%
\subsection{Conditional Processing}
\label{sec:conditional}

The package provides a mechanism to compile different versions
of a document. To customise the versions further some conditional processing
can come in handy to distinguish which version is being compiled.
The package provides two macros to describe the compilation context:

%%%%%%%%%%%%%%%%%%%%%%%%%%%%%%%%%%%%%%%%
\DescribeMacro{\ifchilddoc}
The conditional |\ifchilddoc| distinguishes between the compilation of
child documents and the main document:
%
\begin{center}
|\ifchilddoc |\textit{child-code}| |[|\||else |\textit{main-code}]| \||fi|
\end{center}

%%%%%%%%%%%%%%%%%%%%%%%%%%%%%%%%%%%%%%%%
\DescribeMacro{\childdocname}
\DescribeMacro{\childdocjob}
The macro |\childdocname| contains the filename (without extension)
of the main or child file being processed.
Note that |\childdocjob| will always contain the name of the main file.

%%%%%%%%%%%%%%%%%%%%%%%%%%%%%%%%%%%%%%%%
\paragraph{Title Page.}

Conditional processing can be used to include a title or banner page
in the main document when proper precautions are taken.
Importantly, the code in the main file should ensure that the page counter
(as well as other status parameters which are stored in the |.aux| files)
takes the same value after the conditional processing.
Otherwise the page numbers may take divergent values
depending on which part is compiled.

For example, a title page could be declared by:
%
\begin{center}
\begin{tabular}{l}
|\ifchilddoc\||else|\\
|\addtocounter{page}{-1}|\\
\textit{code for title page}\\
|\newpage|\\
|\||fi|
\end{tabular}
\end{center}
%
A banner page for the child documents can be generated by:
%
\begin{center}
\begin{tabular}{l}
|\ifchilddoc|\\
|\addtocounter{page}{-1}|\\
\textit{code for banner page}\\
|\newpage|\\
|\||fi|
\end{tabular}
\end{center}
%
Here one could write a message such as:
\begin{center}
|This is the part \childdocname{} of \childdocjob{}.|
\end{center}

%%%%%%%%%%%%%%%%%%%%%%%%%%%%%%%%%%%%%%%%%%%%%%%%%%%%%%%%%%%%%%%%%%%%%%%%%%%%%%%%
\subsection{Flags}
\label{sec:flags}

The package makes it easy to generate different versions
of the main or child documents.
To this end compilation flags can be defined
and assigned different default values.
They will be particularly useful in conjunction
with the forwarding mechanism described in \secref{sec:forward}.

For example, it may be useful to have a flag |\version|
which can be set to |draft| or |final|.
The document source will contain some conditional code
depending on the value of |\version|.
Suppose further, the flag should default to |final| for the main file
and to |draft| for child files
which is a natural assignment for editing the document.
This is achieved by placing the following code
in the preamble of the main document
(below the |\childdocmain| directive):
%
\begin{center}
\begin{tabular}{l}
|\ifchilddoc|\\
|\providecommand{\version}{draft}|\\
|\||else|\\
|\providecommand{\version}{final}|\\
|\||fi|
\end{tabular}
\end{center}
%
The definition by |\providecommand| makes sure
that previous definitions are not overwritten.
Further statements |\providecommand{\version}{...}|
can thus be added before the above code to override it.

For the main file, one might add a line
(between |\childdocmain| and the above block)
%
\begin{center}
|%\ifchilddoc\||else\providecommand{\version}{draft}\||fi|
\end{center}
%
which can be uncommented to produce a draft version.
Likewise one can add a line to the very top of a child file
(above the |\childdocof{|\textit{main}|}| directive)
%
\begin{center}
|%\providecommand{\version}{final}|
\end{center}
%
which can be uncommented to produce the final version of this child document.

%%%%%%%%%%%%%%%%%%%%%%%%%%%%%%%%%%%%%%%%%%%%%%%%%%%%%%%%%%%%%%%%%%%%%%%%%%%%%%%%
\subsection{Forwarding}
\label{sec:forward}

Different versions of the main or child documents
using compilation flags as described in \secref{sec:flags}
can be (permanently) stored in different files
for convenient compilation, viewing and distribution.
To this end, the package defines a command
to pass on compilation to a different file:

%%%%%%%%%%%%%%%%%%%%%%%%%%%%%%%%%%%%%%%%
\DescribeMacro{\childdocforward}
The command |\childdocforward| redirects processing to
another source file:
%
\begin{center}
\begin{tabular}{l}
|\input{childdoc.def}|\\
|\childdocforward[|\textit{main}|]{|\textit{dest}|}|\\
\end{tabular}
\end{center}
%
The argument \textit{dest} is the destination file
(without extension).
It should be the main file or one of the child files.
Note that further \textsf{childdoc} directives
such as |\childdocof| and |\childdocforward|
in the indicated file will be processed in this form.
The optional argument \textit{main}
passes on directly to the main file \textit{main}
while pretending to compile the child \textit{dest}.
This form behaves as if \textit{dest}
issues |\childdocof{|\textit{main}|}| right away,
and no further \textsf{childdoc} directives will be processed.

%%%%%%%%%%%%%%%%%%%%%%%%%%%%%%%%%%%%%%%%
\DescribeMacro{\...prefix}
In the alternative form |\childdocforwardprefix|,
%
\begin{center}
\begin{tabular}{l}
|\input{childdoc.def}|\\
|\childdocforwardprefix[|\textit{main}|]{|\textit{prefix}|}{|\textit{dest}|}|
\end{tabular}
\end{center}
%
the destination file is determined by a pattern
depending on the current file:
To make this work, the current file must be called
`{\textit{prefix}\hspace{0.2em}\textit{suffix}}'
with \textit{prefix} matching precisely the argument.
Processing is then passed on to the file
`{\textit{dest}\hspace{0.2em}\textit{suffix}}'.
Surely, the same effect is achieved by
directly specifying the
argument `{\textit{dest}\hspace{0.2em}\textit{suffix}}'
in the first form.
However, that requires to set up a different file
for each child. With the alternative form of the command
all these files can have exactly the same content
which simplifies setting them up and maintaining them.

For example, the following file |draft.tex|
with a compilation flag |\version| as described in \secref{sec:flags}
compiles the main document as a draft:
%
\begin{center}
\begin{tabular}{l}
|\def\version{draft}|\\
|\input{childdoc.def}|\\
|\childdocforward{|\textit{main}|}|
\end{tabular}
\end{center}
%
Likewise, the following files |final|\textit{nn}|.tex|
compile the final version of the child document
|child|\textit{nn}|.tex|:
%
\begin{center}
\begin{tabular}{l}
|\def\version{final}|\\
|\input{childdoc.def}|\\
|\childdocforwardprefix{final}{child}|
\end{tabular}
\end{center}
%

Note that when several versions of a main file and/or of each child file
are to be generated, it may be convenient to set up a |Makefile| or
shell script to automatise the process.

%%%%%%%%%%%%%%%%%%%%%%%%%%%%%%%%%%%%%%%%%%%%%%%%%%%%%%%%%%%%%%%%%%%%%%%%%%%%%%%%
\subsection{Command Line Processing}
\label{sec:commandline}

The effect of redirection files can also be achieved by invoking
the \LaTeX{} compiler with a more elaborate command line.
Most conveniently this should be done as part
of a shell script or a |Makefile|.

When using \textsf{childdoc} in the main file, the following
command lines effectively perform a redirection
(note that depending on the shell being used,
backslashes may have to be doubled: `|\|' $\to$ `|\\|'):
%
\begin{center}
|... -jobname "|\textit{target}|" |\\|"|[\textit{flags}]%
|\input{childdoc.def}\childdocforward[|\textit{main}|]{|\textit{dest}|}"|
\end{center}
%
Here \textit{target} is the name of the output file,
\textit{main} is the name of the main file
and \textit{dest} is the name of the main or child file to be processed
(all filenames without extensions).
The optional argument \textit{main} can be omitted
if \textit{main} matches \textit{dest}.
Optionally, compilation \textit{flags} can be defined via |\def| commands.
This command line makes the \TeX{} engine believe
it is compiling the file \textit{target}
whose content is specified as the latter parameter.
The provided code then forwards the processing to
\textit{main} or \textit{dest} as described in \secref{sec:forward}.

%%%%%%%%%%%%%%%%%%%%%%%%%%%%%%%%%%%%%%%%%%%%%%%%%%%%%%%%%%%%%%%%%%%%%%%%%%%%%%%%
\subsection{Include by Input}
\label{sec:input}

Including child documents by |\include| has some restrictions by design.
Most notably, the content of a child document always occupies
its own set of pages; pages cannot be shared between child documents.
Usually, this behaviour makes perfect sense
because each child document contain an essential part of the document.
However, in some situations it may be desirable to compose
a document from a collection of parts
without having mandatory page breaks between then.
For this case, the package
provides a mechanism to include parts
by |\input| which can also be processed individually.
However, by construction this mechanism
requires manual handling of the content to be output.

%%%%%%%%%%%%%%%%%%%%%%%%%%%%%%%%%%%%%%%%
\DescribeMacro{\ifchilddocmanual}
The main file should be prepared as usual, see \secref{sec:include}.
However, the document body must make a distinction
between processing of an individual part and of the main document, e.g.:
%
\begin{center}
\begin{tabular}{l}
|\ifchilddocmanual|\\
|\input{\childdocname}|\\
|\||else|\\
\textit{document body with }|\input{|\textit{part}|}|\\
|\||fi|
\end{tabular}
\end{center}
%
The conditional |\ifchilddocmanual| is true whenever
a part to be included by |\input| is being compiled,
and the name of the part is stored in |\childdocname|.

%%%%%%%%%%%%%%%%%%%%%%%%%%%%%%%%%%%%%%%%
\DescribeMacro{\childdocby}
Each part to be included by |\input| should start with:
%
\begin{center}
\begin{tabular}{l}
|\input{childdoc.def}|\\
|\childdocby{|\textit{main}|}|\\
\end{tabular}
\end{center}
%
The directive |\childdocby| is similar to |\childdocof|
described in \secref{sec:include},
but the subsequent selection of content must be done manually.
To that end, both |\ifchilddoc| and |\ifchilddocmanual|
will be true upon processing of a part,
and the name of the part is stored in |\childdocname|.
Note that |\jobname| will be set to the filename of the current part
so that each part receives an individual |.aux| file
that does not interfere with the |.aux| file(s) of the main document.
This behaviour can be altered by the alternative form
|\childdocby[*]{|\textit{main}|}| (with a non-empty optional argument)
which uses the |.aux| file of the main document
by setting |\jobname| to \textit{main}.

%%%%%%%%%%%%%%%%%%%%%%%%%%%%%%%%%%%%%%%%%%%%%%%%%%%%%%%%%%%%%%%%%%%%%%%%%%%%%%%%
\subsection{Driver Development}
\label{sec:driver}

The \textsf{childdoc} mechanism can also be use for the development
of definition files such as \LaTeX{} styles or classes.
This case differs from the above setup with multiple parts
included by |\include| in that no |\includeonly| should be invoked.
This can be achieved by starting the include file
(before |\ProvidesPackage|) with:
%
\begin{center}
\begin{tabular}{l}
|\input{childdoc.def}|\\
|\childdocforward{|\textit{main}|}|\\
\end{tabular}
\end{center}
%
or alternatively with:
%
\begin{center}
\begin{tabular}{l}
|\input{childdoc.def}|\\
|\childdocby{|\textit{main}|}|\\
\end{tabular}
\end{center}
%
Both forms have slightly different effects as described above.
The main file is prepared as usual, see \secref{sec:include}.

%%%%%%%%%%%%%%%%%%%%%%%%%%%%%%%%%%%%%%%%%%%%%%%%%%%%%%%%%%%%%%%%%%%%%%%%%%%%%%%%
\subsection{Legacy Detection}
\label{sec:detection}

The directive |\childdocmain| in the main file can detect
whether the complete document or merely a child is to be compiled
even without using the directive |\childdocof|.
This method is deprecated because it is less robust
and there is no compelling reason to use it;
it is merely provided for backward compatibility
and it may be removed in future versions.

If the detection mechanism is to be used,
it is mandatory to correctly specify
the filename of the main file as the argument of |\childdocmain|:
%
\begin{center}
\begin{tabular}{l}
|\input{childdoc.def}|\\
|\childdocmain{|\textit{main}|}|\\
\end{tabular}
\end{center}
%
If |\jobname| does not match the argument \textit{main} of |\childdocmain|,
it is assumed that |\jobname| points to the child file to be compiled.
When using |\childdocmain| with the main file specified as argument,
it suffices to start a child file
with just |\input{|\textit{main}|}|
without loading of the package and using |\childdocof|.
If instead all processing is done
with the appropriate \textsf{childdoc} directives,
the argument of \textit{main} of |\childdocmain| can be empty.

An alternative version of the command line processing described
in \secref{sec:commandline} using the detection mechanism reads:
%
\begin{center}
|... -jobname "|\textit{target}|" "|[\textit{flags}]%
[|\def\jobname{|\textit{dest}|}|]|\input{|\textit{main}|}"|
\end{center}

%%%%%%%%%%%%%%%%%%%%%%%%%%%%%%%%%%%%%%%%%%%%%%%%%%%%%%%%%%%%%%%%%%%%%%%%%%%%%%%%
\subsection{Manual Code}
\label{sec:manual}

In case one cannot be certain whether the definitions file |childdoc.def|
is installed on the target \TeX{} distribution
and one prefers not to ship it,
it is conceivable to paste a few relevant commands into the sources.

To that end, drop all statements |\input{childdoc.def}|
and perform the replacements as outlined below.
Instead of |\childdocmain{|\textit{main}|}| add the following code
to the top of the main file:
%
\begin{center}
\begin{tabular}{l}
|\||ifdefined\childdocname\endinput\||fi\newif\ifchilddoc|\\
|\edef\childdocname{\scantokens\expandafter{\jobname\noexpand}}|\\
|\def\childdocmain{|\textit{main}|}\||ifx\childdocmain\childdocname\||else|\\
|\childdoctrue\includeonly{\childdocname}\let\jobname\childdocmain\||fi|\\
\end{tabular}
\end{center}
%
Instead of |\childdocof{|\textit{main}|}| just include the main file
at the top of each child file:
%
\begin{center}
|\input{|\textit{main}|}|
\end{center}
%
A simple redirection |\childdocforward{|\textit{dest}|}| is achieved by:
%
\begin{center}
|\def\jobname{|\textit{dest}|}\input{\jobname}|
\end{center}
%
The redirection with prefix
|\childdocforwardprefix[|\textit{prefix}|]{|\textit{dest}|}|
is accomplished by:
%
\begin{center}
\begin{tabular}{l}
|{\edef\jobname{\scantokens\expandafter{\jobname\noexpand}}|\\
|\def\redirectjob |\textit{prefix}|#1~~~{\gdef\jobname{|\textit{dest}|#1}}|\\
|\expandafter\redirectjob\jobname~~~}\input{\jobname}|
\end{tabular}
\end{center}

In an alternative approach,
child documents can be compiled by a specific command line
without additional code or specific definitions:
%
\begin{center}
|... -jobname "|\textit{target}|" "|[\textit{flags}]%
|\includeonly{|\textit{dest}|}\input{|\textit{main}|}"|
\end{center}
%

%%%%%%%%%%%%%%%%%%%%%%%%%%%%%%%%%%%%%%%%%%%%%%%%%%%%%%%%%%%%%%%%%%%%%%%%%%%%%%%%
%%%%%%%%%%%%%%%%%%%%%%%%%%%%%%%%%%%%%%%%%%%%%%%%%%%%%%%%%%%%%%%%%%%%%%%%%%%%%%%%
\section{Information}

%%%%%%%%%%%%%%%%%%%%%%%%%%%%%%%%%%%%%%%%%%%%%%%%%%%%%%%%%%%%%%%%%%%%%%%%%%%%%%%%
\subsection{Copyright}

Copyright \copyright{} 2017--2018 Niklas Beisert

This work may be distributed and/or modified under the
conditions of the \LaTeX{} Project Public License, either version 1.3
of this license or (at your option) any later version.
The latest version of this license is in
  \url{http://www.latex-project.org/lppl.txt}
and version 1.3 or later is part of all distributions of \LaTeX{}
version 2005/12/01 or later.

This work has the LPPL maintenance status `maintained'.

The Current Maintainer of this work is Niklas Beisert.

This work consists of the files |README.txt|, |childdoc.ins| and |childdoc.dtx|
as well as the derived files |childdoc.def|, |cdocsamp.tex|
with |cdocsch1.tex|, |cdocsch2.tex|, |cdocspt3.tex|, |cdocspt4.tex|,
|cdocsdrf.tex|, |cdocsfn1.tex|, |cdocsfn2.tex|
as well as |childdoc.pdf|.

%%%%%%%%%%%%%%%%%%%%%%%%%%%%%%%%%%%%%%%%%%%%%%%%%%%%%%%%%%%%%%%%%%%%%%%%%%%%%%%%
\subsection{Files and Installation}

The package consists of the files:
%
\begin{center}
\begin{tabular}{ll}
    |README.txt|   & readme file \\
    |childdoc.ins| & installation file \\
    |childdoc.dtx| & source file \\
    |childdoc.def| & definition file \\
    |cdocsamp.tex| & sample main file \\
    |cdocsch1.tex| & sample include file \\
    |cdocsch2.tex| & sample include file \\
    |cdocspt3.tex| & sample part file \\
    |cdocspt4.tex| & sample part file \\
    |cdocsdrf.tex| & sample redirection file \\
    |cdocsfn1.tex| & sample redirection file \\
    |cdocsfn2.tex| & sample redirection file \\
    |childdoc.pdf| & manual
\end{tabular}
\end{center}
%
The distribution consists of the files
|README.txt|, |childdoc.ins| and |childdoc.dtx|.
%
\begin{itemize}
\item
Run (pdf)\LaTeX{} on |childdoc.dtx|
to compile the manual |childdoc.pdf| (this file).
\item
Run \LaTeX{} on |childdoc.ins| to create the definitions file |childdoc.def|
and the sample |cdocsamp.tex| with include files
|cdocsch1.tex|, |cdocsch2.tex|, |cdocspt3.tex|, |cdocspt4.tex|,
|cdocsdrf.tex|, |cdocsfn1.tex|, |cdocsfn2.tex|.
Then copy the file |childdoc.def| to an appropriate directory of your \LaTeX{}
distribution, e.g.\ \textit{texmf-root}|/tex/latex/childdoc|.
\end{itemize}

%%%%%%%%%%%%%%%%%%%%%%%%%%%%%%%%%%%%%%%%%%%%%%%%%%%%%%%%%%%%%%%%%%%%%%%%%%%%%%%%
\subsection{Related CTAN Packages}

There are several other packages which offer a similar functionality:
%
\begin{itemize}
\item
The packages
\href{http://ctan.org/pkg/docmute}{\textsf{docmute}},
\href{http://ctan.org/pkg/includex}{\textsf{includex}} and
\href{http://ctan.org/pkg/standalone}{\textsf{standalone}}
provide commands to include only the document body of
a child file thus allowing both files to be compiled individually.
\item
The packages \href{http://ctan.org/pkg/subdocs}{\textsf{subdocs}}
and \href{http://ctan.org/pkg/subfiles}{\textsf{subfiles}}
provide structures in which the main and child documents can be
encapsulated and allowing them to be compiled individually.
The inclusion mechanism is different from the conventional |\include|.
\item
The package \href{http://ctan.org/pkg/combine}{\textsf{combine}}
is an elaborate solution to combine several documents into one.
\end{itemize}
%
See also the CTAN topic \href{http://ctan.org/topic/subdocs}{\textsf{subdocs}}
for further related packages.
The present package differs from the above solutions in that
a document structure constructed with the conventional |\include| mechanism
just needs two extra commands at the top of every file
such that all constituent files can be compiled individually.

%%%%%%%%%%%%%%%%%%%%%%%%%%%%%%%%%%%%%%%%%%%%%%%%%%%%%%%%%%%%%%%%%%%%%%%%%%%%%%%%
%\subsection{Feature Suggestions}
%
%The following is a list of features which may be useful for future
%versions of this package:
%%
%\begin{itemize}
%\item
%\ldots
%\end{itemize}

%%%%%%%%%%%%%%%%%%%%%%%%%%%%%%%%%%%%%%%%%%%%%%%%%%%%%%%%%%%%%%%%%%%%%%%%%%%%%%%%
\subsection{Revision History}

%%%%%%%%%%%%%%%%%%%%%%%%%%%%%%%%%%%%%%%%
\paragraph{v2.0:} 2018/12/30

\begin{itemize}
\item
immediate forward processing
\item
added |\childdocby| mechanism
\item
manual restructured
\end{itemize}

%%%%%%%%%%%%%%%%%%%%%%%%%%%%%%%%%%%%%%%%
\paragraph{v1.6:} 2018/01/17

\begin{itemize}
\item
application for development of include files
\item
corrections to manual
\end{itemize}

%%%%%%%%%%%%%%%%%%%%%%%%%%%%%%%%%%%%%%%%
\paragraph{v1.5:} 2017/05/21

\begin{itemize}
\item
more complete structuring introduced
\item
|\childdocof| introduced
\item
|\childdoc| renamed to |\childdocmain|
\item
|\childredirect| renamed to |\childdocforward| and |\childdocforwardprefix|
and functionality expanded
\end{itemize}

%%%%%%%%%%%%%%%%%%%%%%%%%%%%%%%%%%%%%%%%
\paragraph{v1.0:} 2017/04/27

\begin{itemize}
\item
manual and install package
\item
first version published on CTAN
\end{itemize}

%%%%%%%%%%%%%%%%%%%%%%%%%%%%%%%%%%%%%%%%
\paragraph{v0.6:} 2017/04/26

\begin{itemize}
\item
redirection mechanism added
\end{itemize}

%%%%%%%%%%%%%%%%%%%%%%%%%%%%%%%%%%%%%%%%
\paragraph{v0.5:} 2017/04/26

\begin{itemize}
\item
functionality in definition file
\end{itemize}


%%%%%%%%%%%%%%%%%%%%%%%%%%%%%%%%%%%%%%%%%%%%%%%%%%%%%%%%%%%%%%%%%%%%%%%%%%%%%%%%
%%%%%%%%%%%%%%%%%%%%%%%%%%%%%%%%%%%%%%%%%%%%%%%%%%%%%%%%%%%%%%%%%%%%%%%%%%%%%%%%
%%%%%%%%%%%%%%%%%%%%%%%%%%%%%%%%%%%%%%%%%%%%%%%%%%%%%%%%%%%%%%%%%%%%%%%%%%%%%%%%
\appendix

\settowidth\MacroIndent{\rmfamily\scriptsize 000\ }

 \DocInput{childdoc.dtx}

\end{document}
%</driver>
% \fi
%
% %%%%%%%%%%%%%%%%%%%%%%%%%%%%%%%%%%%%%%%%%%%%%%%%%%%%%%%%%%%%%%%%%%%%%%%%%%%%%%
% %%%%%%%%%%%%%%%%%%%%%%%%%%%%%%%%%%%%%%%%%%%%%%%%%%%%%%%%%%%%%%%%%%%%%%%%%%%%%%
% \section{Sample}
%\iffalse
%<*samplemain>
%\fi
%
% The following presents a sample document
% with two chapters, two parts, a title page,
% a compile flag as well as three forwarding files to set the flag.
% It consists of eight |.tex| files:
% \begin{center}
% \begin{tabular}{ll}
% |cdocsamp.tex|&main file\\
% |cdocsch1.tex|&include file for chapter 1\\
% |cdocsch2.tex|&include file for chapter 2\\
% |cdocspt3.tex|&include file for part 3\\
% |cdocspt4.tex|&include file for part 4\\
% |cdocsdrf.tex|&forwarding file for main file in draft mode\\
% |cdocsfi1.tex|&forwarding file for final version of chapter 1\\
% |cdocsfi2.tex|&forwarding file for final version of chapter 2\\
% \end{tabular}
% \end{center}
% Each of the eight files can be compiled directly by the \LaTeX{} compiler.
%
% %%%%%%%%%%%%%%%%%%%%%%%%%%%%%%%%%%%%%%
% \paragraph{Main File.}
%
% The main file is called |cdocsamp.tex|.
%
% Load the \textsf{childdoc} definitions and
% declare the filename for the main document:
%    \begin{macrocode}
\input{childdoc.def}
\childdocmain{}
%    \end{macrocode}

% Optional override for |\version| flag:
%    \begin{macrocode}
%%\ifchilddoc\else\providecommand{\version}{draft}\fi
%    \end{macrocode}

% Define the default values for the |\version| flag
% (|final| for the main file and |draft| for childs):
%    \begin{macrocode}
\ifchilddoc
\providecommand{\version}{draft}
\else
\providecommand{\version}{final}
\fi
%    \end{macrocode}

% Load the standard document class:
%    \begin{macrocode}
\documentclass[12pt]{article}
%    \end{macrocode}

% Start the document body:
%    \begin{macrocode}
\begin{document}
%    \end{macrocode}

% Declare a title page.
% Print title, part of document being processed and version flag:
%    \begin{macrocode}
\addtocounter{page}{-1}
\begin{center}
{\LARGE\bfseries{}childdoc example\par}
\vspace{1cm}
\ifchilddoc
\ifchilddocmanual part\else chapter\fi:
`\childdocname' of `\childdocjob'\par
\else
main document: `\childdocjob'\par
\fi
version: \version\par
\end{center}
\newpage
%    \end{macrocode}

% Manually include selected file,
% otherwise process as usual:
%    \begin{macrocode}
\ifchilddocmanual
\section*{part `\childdocname'}
\input{\childdocname}
\else
%    \end{macrocode}

% Include the two chapters:
%    \begin{macrocode}
\include{cdocsch1}
\include{cdocsch2}
%    \end{macrocode}

% Include the two parts unless only chapters should be displayed:
%    \begin{macrocode}
\ifchilddoc\else
\section{part three}
\input{cdocspt3}
\section{part four}
\input{cdocspt4}
\fi
%    \end{macrocode}

% Process as usual until here:
%    \begin{macrocode}
\fi
%    \end{macrocode}

% End of document body:
%    \begin{macrocode}
\end{document}
%    \end{macrocode}
%\iffalse
%</samplemain>
%\fi
%
% %%%%%%%%%%%%%%%%%%%%%%%%%%%%%%%%%%%%%%
% \paragraph{Chapter Include Files.}
%
% The include files are called |cdocsch1.tex| and |cdocsch2.tex|.
%
%\iffalse
%<*samplechap1|samplechap2>
%\fi

% Optional override for |\version| flag:
%    \begin{macrocode}
%%\providecommand{\version}{final}
%    \end{macrocode}

% Include the main document:
%    \begin{macrocode}
\input{childdoc.def}
\childdocof{cdocsamp}
%    \end{macrocode}

%\iffalse
%</samplechap1|samplechap2>
%\fi
%
%\iffalse
%<*samplechap1>
%\fi
% Some text for chapter 1:
%    \begin{macrocode}
\section{one}
some text in chapter one
%    \end{macrocode}

%\iffalse
%</samplechap1>
%\fi
% Some text for chapter 2:
%\iffalse
%<*samplechap2>
%\fi
%    \begin{macrocode}
\section{two}
more text in chapter two
%    \end{macrocode}

%\iffalse
%</samplechap2>
%\fi
%
% %%%%%%%%%%%%%%%%%%%%%%%%%%%%%%%%%%%%%%
% \paragraph{Part Include Files.}
%
% The include files are called |cdocspt3.tex| and |cdocspt4.tex|.
%
%\iffalse
%<*samplepart3|samplepart4>
%\fi

% Optional override for |\version| flag:
%    \begin{macrocode}
%%\providecommand{\version}{final}
%    \end{macrocode}

% Include the main document:
%    \begin{macrocode}
\input{childdoc.def}
\childdocby{cdocsamp}
%    \end{macrocode}

%\iffalse
%</samplepart3|samplepart4>
%\fi
%
%\iffalse
%<*samplepart3>
%\fi
% Some text for part 3:
%    \begin{macrocode}
some text in part three
%    \end{macrocode}

%\iffalse
%</samplepart3>
%\fi
% Some text for part 4:
%\iffalse
%<*samplepart4>
%\fi
%    \begin{macrocode}
more text in part four
%    \end{macrocode}

%\iffalse
%</samplepart4>
%\fi
%
% %%%%%%%%%%%%%%%%%%%%%%%%%%%%%%%%%%%%%%
% \paragraph{Forwarding for a Complete Draft.}
%
% The following forwarding file |cdocsdrf.tex|
% compiles the main document in draft mode:
%\iffalse
%<*sampledraft>
%\fi
%    \begin{macrocode}
\def\version{draft}
\input{childdoc.def}
\childdocforward{cdocsamp}
%    \end{macrocode}

%\iffalse
%</sampledraft>
%\fi
%
% %%%%%%%%%%%%%%%%%%%%%%%%%%%%%%%%%%%%%%
% \paragraph{Forwarding for Final Version of the Chapters.}
%
% The following forwarding files |cdocsfn1.tex| and |cdocsfn2.tex|
% (with identical content)
% compile the final versions of the child documents
% |cdocsch1.tex| and |cdocsch2.tex|, respectively:
%\iffalse
%<*samplefinal>
%\fi
%    \begin{macrocode}
\def\version{final}
\input{childdoc.def}
\childdocforwardprefix[cdocsamp]{cdocsfn}{cdocsch}
%    \end{macrocode}

%\iffalse
%</samplefinal>
%\fi
%
% %%%%%%%%%%%%%%%%%%%%%%%%%%%%%%%%%%%%%%
% \paragraph{Command Line Processing.}
%
% The following three command lines generate the output files
% |cdocscld|, |cdocscl1| and |cdocscl2|
% which should be identical to
% |cdocsdrf|, |cdocsch1| and |cdocsfn2|, respectively:
% \begin{center}
% \begin{tabular}{l}
% |latex -jobname cdocscld \|\\
% |  "\def\version{draft}\input{childdoc.def}\childdocforward{cdocsamp}"|\\
% |latex -jobname cdocscl1 \|\\
% |  "\input{childdoc.def}\childdocforward[cdocsamp]{cdocsch1}"|\\
% |latex -jobname cdocscl2 \|\\
% |  "\def\version{final}\input{childdoc.def}\childdocforward{cdocsch2}"|
% \end{tabular}
% \end{center}
% Note that the trailing backslash on each first line
% merely continues the input to the second line
% (for convenient cut ant paste).
% Furthermore, the command |latex| can be replaced by any
% of its alternative versions such as |pdflatex|.
%
% %%%%%%%%%%%%%%%%%%%%%%%%%%%%%%%%%%%%%%%%%%%%%%%%%%%%%%%%%%%%%%%%%%%%%%%%%%%%%%
% %%%%%%%%%%%%%%%%%%%%%%%%%%%%%%%%%%%%%%%%%%%%%%%%%%%%%%%%%%%%%%%%%%%%%%%%%%%%%%
% \section{Implementation}
%\iffalse
%<*package>
%\fi
%
% This section describes the definitions file |childdoc.def|.

% The definitions cannot be loaded using |\usepackage| or |\RequirePackage|
% which has a mechanism to prevent loading a style file more than once.
% When loading the definitions by means of |\input|
% multiple instances have to be prevented manually:
%\iffalse
%This code needs to be before the `\ProvidesFile' directive
%which is defined at the beginning of this file.
%Therefore it is also placed there and commented out here.
%</package>
%<*discard>
%\fi
%    \begin{macrocode}
\ifdefined\childdocmain\endinput\fi
%    \end{macrocode}
%\iffalse
%</discard>
%<*package>
%\fi
%
% \macro{\ifchilddoc}
% \macro{\ifchilddocmanual}
% The conditional |\ifchilddoc| tells whether a
% child (true) or main (false) document is being compiled.
% The conditional |\ifchilddocmanual| tells whether
% the |\includeonly| mechanism is used (false) or
% the selection of child files must be performed manually (true).
% The definitions initialise to false:
%    \begin{macrocode}
\newif\ifchilddoc
\newif\ifchilddocmanual
%    \end{macrocode}

% \macro{\childdocname}
% \macro{\childdocjob}
% The macro |\childdocname| stores the name of the main document
% to be compiled. The macro |\childdocjob| stores the name of
% the document on which the \LaTeX{} compiler was originally invoked.
% The content of |\jobname| cannot be compared
% to filenames specified in the source due to different catcodes.
% The following code rescans |\jobname|, stores the result
% in |\childdocname| and saves a copy in |\childdocjob|:
%    \begin{macrocode}
\edef\childdocname{\scantokens\expandafter{\jobname\noexpand}}
\let\childdocjob\childdocname
%    \end{macrocode}

% \macro{\childdocdisable}
% The macro |\childdocdisable| prevents the main file
% from being processed more than once.
% At this stage, the main document command |\childdocmain|
% is assumed to be called once again where it should do nothing.
% Any subsequent call to it should prevent
% a secondary processing of the main document
% It overwrites the forwarding commands
% |\childdocof| and |\childdocforward|
% with empty macros to prevent further inclusions of the main document:
%    \begin{macrocode}
\newcommand{\childdocdisable}
{
  \renewcommand{\childdocmain}[1]{\renewcommand{\childdocmain}[1]{\endinput}}
  \renewcommand{\childdocof}[1]{}
  \renewcommand{\childdocby}[2][]{}
  \renewcommand{\childdocforward}[2][]{}
  \renewcommand{\childdocdisable}{}
}
%    \end{macrocode}

% \macro{\childdocmain}
% The macro |\childdocmain| is to be called at the top of the main file
% with nothing or the main filename (without extension) as argument.
% First, it breaks loops.
% If the argument is not empty and does not match |\childdocname|
% (which is set by the first inclusion of |childdoc.def|),
% |\ifchilddoc| is set to true, |\includeonly| is applied to the child file
% and |\jobname| is set to the main file
% (for proper handling of |.aux| files):
%    \begin{macrocode}
\newcommand{\childdocmain}[1]
{
  \childdocdisable\childdocmain{}
  \if?#1?\else
    \begingroup
      \def\childdoctmp{#1}
      \ifx\childdoctmp\childdocname
        \def\childdoctmp{}
      \else
        \def\childdoctmp
        {
          \childdoctrue
          \includeonly{\childdocname}
          \def\childdocjob{#1}
          \def\jobname{#1}
        }
      \fi
      \expandafter
    \endgroup
    \childdoctmp
  \fi
}
%    \end{macrocode}

% \macro{\childdocof}
% The command |\childdocof| redirects
% compilation to the main file |#1|.
%    \begin{macrocode}
\newcommand{\childdocof}[1]
{
  \childdocdisable
  \childdoctrue
  \includeonly{\childdocname}
  \def\jobname{#1}
  \def\childdocjob{#1}
  \input{#1}
}
%    \end{macrocode}

% \macro{\childdocby}
% The command |\childdocby| ....
%    \begin{macrocode}
\newcommand{\childdocby}[2][]
{
  \childdocdisable
  \childdoctrue
  \childdocmanualtrue
  \if?#1?\else
    \def\jobname{#2}
  \fi
  \def\childdocjob{#2}
  \input{#2}
  \endinput
}
%    \end{macrocode}

% \macro{\childdocforward}
% The command |\childdocforward| redirects
% compilation to the main file or
% (if the optional argument is given) a child file.
% Parameters are set as if the main file
% or a child file starting with |\childdocof| was compiled.
% Then compilation is handed over to the main file:
%    \begin{macrocode}
\newcommand{\childdocforward}[2][]
{
  \begingroup
    \if?#1?
      \def\childdoctmp
      {
        \def\childdocname{#2}
        \def\childdocjob{#2}
        \def\jobname{#2}
        \input{#2}
        \endinput
      }
    \else
      \def\childdoctmp
      {
        \childdocdisable
        \def\childdocname{#2}
        \childdoctrue
        \includeonly{#2}
        \def\childdocjob{#1}
        \def\jobname{#1}
        \input{#1}
        \endinput
      }
    \fi
    \expandafter
  \endgroup
  \childdoctmp
}
%    \end{macrocode}

% \macro{\childdocforwardprefix}
% The command |\childdocforwardprefix| redirects
% compilation to the main or a child file by means of a pattern.
% The prefix |#1| in the current filename is replaced by |#2|
% and the suffix of the current filename is kept
% (it is assumed that the filename does not contain the substring `|~~~|'
% which is used as a delimiter).
% Compilation is handed over to the new file by |\childdocforward|:
%    \begin{macrocode}
\newcommand{\childdocforwardprefix}[3][]
{
  \begingroup
    \def\childdocextract #2##1~~~{\def\childdoctmp{\childdocforward[#1]{#3##1}}}
    \expandafter\childdocextract\childdocname~~~
    \expandafter
  \endgroup
  \childdoctmp
}
%    \end{macrocode}

% \macro{\childdoc}
% The deprecated macro |\childdoc| is a legacy version of |\childdocmain|:
%    \begin{macrocode}
\newcommand{\childdoc}{\childdocmain}
%    \end{macrocode}

% \macro{\childdocredirect}
% The deprecated macro |\childdocredirect| is a legacy version
% of |\childdocforward| and |\childdocforwardprefix|:
%    \begin{macrocode}
\newcommand{\childdocredirect}[2][]
{
  \begingroup
    \if?#1?
      \def\childdoctmp{\childdocforward{#2}}
    \else
      \def\childdoctmp{\childdocforwardprefix{#1}{#2}}
    \fi
    \expandafter
  \endgroup
  \childdoctmp
}
%    \end{macrocode}

%\iffalse
%</package>
%\fi
%
\endinput
|\\
|\childdocforward{|\textit{main}|}|\\
\end{tabular}
\end{center}
%
or alternatively with:
%
\begin{center}
\begin{tabular}{l}
|% \iffalse
%
% childdoc.dtx Copyright (C) 2017-2018 Niklas Beisert
%
% This work may be distributed and/or modified under the
% conditions of the LaTeX Project Public License, either version 1.3
% of this license or (at your option) any later version.
% The latest version of this license is in
%   http://www.latex-project.org/lppl.txt
% and version 1.3 or later is part of all distributions of LaTeX
% version 2005/12/01 or later.
%
% This work has the LPPL maintenance status `maintained'.
%
% The Current Maintainer of this work is Niklas Beisert.
%
% This work consists of the files childdoc.dtx and childdoc.ins
% and the derived files childdoc.def and cdocsamp.tex with
% cdocsch1.tex, cdocsch2.tex, cdocsdrf.tex, cdocsfn1.tex, cdocsfn2.tex.
%
%<package>\ifdefined\childdocmain\endinput\fi
%<package>\ProvidesFile{childdoc.def}[2018/12/30 v2.0 child document driver]
%<samplemain>\ProvidesFile{cdocsamp.tex}[2018/12/30 v2.0 sample for childdoc]
%<*driver>
%\ProvidesFile{childdoc.drv}[2018/12/30 v2.0 childdoc reference manual file]
\PassOptionsToClass{10pt,a4paper}{article}
\documentclass{ltxdoc}

\usepackage[margin=35mm]{geometry}
\usepackage{hyperref}
\usepackage{hyperxmp}
\usepackage[usenames]{color}

\hypersetup{colorlinks=true}
\hypersetup{pdfstartview=FitH}
\hypersetup{pdfpagemode=UseNone}
\hypersetup{pdfsource={}}
\hypersetup{pdflang={en-UK}}
\hypersetup{pdfcopyright={Copyright 2017-2018 Niklas Beisert.
  This work may be distributed and/or modified under the
  conditions of the LaTeX Project Public License, either version 1.3
  of this license or (at your option) any later version.}}
\hypersetup{pdflicenseurl={http://www.latex-project.org/lppl.txt}}
\hypersetup{pdfcontactaddress={ETH Zurich, ITP, HIT K,
  Wolfgang-Pauli-Strasse 27}}
\hypersetup{pdfcontactpostcode={8093}}
\hypersetup{pdfcontactcity={Zurich}}
\hypersetup{pdfcontactcountry={Switzerland}}
\hypersetup{pdfcontactemail={nbeisert@itp.phys.ethz.ch}}
\hypersetup{pdfcontacturl={http://people.phys.ethz.ch/\xmptilde nbeisert/}}

\newcommand{\secref}[1]{\hyperref[#1]{section \ref*{#1}}}

\parskip1ex
\parindent0pt
\let\olditemize\itemize
\def\itemize{\olditemize\parskip0pt}

\begin{document}

\title{The \textsf{childdoc} Package}
\hypersetup{pdftitle={The childdoc Package}}
\author{Niklas Beisert\\[2ex]
  Institut f\"ur Theoretische Physik\\
  Eidgen\"ossische Technische Hochschule Z\"urich\\
  Wolfgang-Pauli-Strasse 27, 8093 Z\"urich, Switzerland\\[1ex]
  \href{mailto:nbeisert@itp.phys.ethz.ch}
  {\texttt{nbeisert@itp.phys.ethz.ch}}}
\hypersetup{pdfauthor={Niklas Beisert}}
\hypersetup{pdfsubject={Manual for the LaTeX2e Package childdoc}}
\date{30 December 2018, \textsf{v2.0}}
\maketitle

\begin{abstract}\noindent
\textsf{childdoc} is a \LaTeXe{} package
that enables the direct compilation
of document sections included by |\include|
to individual files.
\end{abstract}

\begingroup
\parskip0ex
\tableofcontents
\endgroup

%%%%%%%%%%%%%%%%%%%%%%%%%%%%%%%%%%%%%%%%%%%%%%%%%%%%%%%%%%%%%%%%%%%%%%%%%%%%%%%%
%%%%%%%%%%%%%%%%%%%%%%%%%%%%%%%%%%%%%%%%%%%%%%%%%%%%%%%%%%%%%%%%%%%%%%%%%%%%%%%%
\section{Introduction}

\LaTeX{} provides a mechanism to structure a large document (such as a book)
into a main file and several child files (containing the chapters)
using the |\include| command.
This mechanism is beneficial for documents
which span hundreds of pages in order to
make the source file(s) more manageable.
Moreover, compilation can be restricted to
selected child files by means of the |\includeonly| command.
The latter feature can be used to reduce the compilation time while editing
(this was significantly more useful in the earlier days of \LaTeX{})
or to generate a smaller document which is easier to navigate.
Another application of |\includeonly| is to generate
documents consisting of selected parts of the complete document.

However, there are a few drawbacks of the plain |\include| mechanism:
\begin{itemize}
\item
The child files cannot be compiled on their own,
they can only be compiled via the main file.
A naive editing environment
(such as a text editor with an option
to have the current file processed by \LaTeX)
may require one to switch to the main file before compiling;
attempting to compile the child file produces errors.
\item
The main file must be modified (each time)
to adjust the |\includeonly| command
to the present needs. This easily leaves the main file in a messy state.
\item
The generated document will always carry the filename
of the main document. This is inconvenient if
several child files are to be compiled and
to be kept for distribution.
\end{itemize}

The present package provides a simple interface
to make child files individually compilable by \LaTeX{}.
Compiling a child file then has the same effect as compiling
the main file with an |\includeonly| command
to select the appropriate child.
Moreover the generated document will carry the name of the child
rather than the main file.
This resolves all three above issues.

This feature is meant to make the editing of books,
thesis documents and lecture notes somewhat more convenient.
However, the package can also be used efficiently for
composing a series of documents (such as exercise sheets)
which are typically distributed individually.
It then assists the author in generating the individual documents
(potentially in different versions)
as well as a document containing the collected series.
Another application is in developing style files
or other kinds of included material
where compilation of the style file could redirect
to a sample or test file.

%%%%%%%%%%%%%%%%%%%%%%%%%%%%%%%%%%%%%%%%%%%%%%%%%%%%%%%%%%%%%%%%%%%%%%%%%%%%%%%%
%%%%%%%%%%%%%%%%%%%%%%%%%%%%%%%%%%%%%%%%%%%%%%%%%%%%%%%%%%%%%%%%%%%%%%%%%%%%%%%%
\section{Usage}

First of all, the package \textsf{childdoc} is \emph{not} a standard
\LaTeXe{} |.sty| style file! Therefore it needs to be invoked in
a non-standard way.

%%%%%%%%%%%%%%%%%%%%%%%%%%%%%%%%%%%%%%%%%%%%%%%%%%%%%%%%%%%%%%%%%%%%%%%%%%%%%%%%
\subsection{Included Files}
\label{sec:include}

%%%%%%%%%%%%%%%%%%%%%%%%%%%%%%%%%%%%%%%%
\DescribeMacro{\childdocmain}
To use the package, add the commands
\begin{center}
\begin{tabular}{l}
|\input{childdoc.def}|\\
|\childdocmain{}|\\
\end{tabular}
\end{center}
at the very top of the main \LaTeX{} file,
in particular \emph{before} the |\documentclass| statement!
The argument of |\childdocmain| should be left empty
(but it must be present).

%%%%%%%%%%%%%%%%%%%%%%%%%%%%%%%%%%%%%%%%
\DescribeMacro{\childdocof}
Furthermore, add the commands
\begin{center}
\begin{tabular}{l}
|\input{childdoc.def}|\\
|\childdocof{|\textit{main}|}|\\
\end{tabular}
\end{center}
at the top of every child file \textit{child}
which is included by |\include{|\textit{child}|}|
from within the main file
(or at least for those files to be compiled individually).
The argument \textit{main} must be the filename of the main file.

There are a couple of
considerations in setting up the main and child documents:

%%%%%%%%%%%%%%%%%%%%%%%%%%%%%%%%%%%%%%%%
\paragraph{Restrictions.}

Please note the following restrictions:
\begin{itemize}
\item
|\childdocmain| must be called with one argument \textit{main}
to ensure compatibility with earlier version of the package.
It must either be empty (|\childdocmain{}|)
or precisely match the filename of the main file in which it is specified.
See \secref{sec:detection} for further information.
\item
The filename \textit{main} must be specified without the |.tex| extension.
\item
The filename \textit{main} is case sensitive
(even in case-insensitive file systems)
due to internal string comparison.
\item
The argument \textit{main} should be fully expanded, it cannot be a macro.
\item
Subdirectories and special characters should be avoided in filenames.
\item
The command |\childdocmain{|\textit{main}|}| must be followed by a whitespace.
It should not be followed immediately by another command
or by a comment mark `|%|'.
This is because the \TeX{} parser reads the token immediately following
the argument of |\childdocmain| and puts it
at the beginning of every child section;
however, a white\-space is ignored.
\end{itemize}

%%%%%%%%%%%%%%%%%%%%%%%%%%%%%%%%%%%%%%%%
\paragraph{Content of Main File.}

It is advisable to place all content in the child files included by |\include|.
Any output contained in the main file will appear in all child documents
unless suppressed manually;
it cannot be suppressed automatically by the |\includeonly| directive
and thus should normally be avoided.
A method to include some content in the main file
by means of conditional processing is described in \secref{sec:conditional}.

%%%%%%%%%%%%%%%%%%%%%%%%%%%%%%%%%%%%%%%%
\paragraph{Page Numbering.}

When only a part of the document is compiled,
the appropriate numbering of pages
(as well as other status parameters)
is determined from the |.aux| files.
The latter contain information from previous passes.
However this information needs to propagate through
all intermediate child documents.
Therefore the page numbering in child documents may well
be inconsistent until the complete document is compiled at least once.

A useful (if unconventional) way to always ensure a consistent
page numbering is to restart the numbering in each child document
and denote the pages by `\textit{child}|.|\textit{page}'
where \textit{child} represents the chapter/section number of the child file.
This can be achieved by the command
|\numberwithin{page}{|\textit{child}|}|
of the \textsf{amsmath} package
where \textit{child} can be |chapter| or |section|
depending on the chosen structuring.
Alternatively, one can modify the macro |\thepage| appropriately
and reset the counter |page| at the start of each child file.

%%%%%%%%%%%%%%%%%%%%%%%%%%%%%%%%%%%%%%%%%%%%%%%%%%%%%%%%%%%%%%%%%%%%%%%%%%%%%%%%
\subsection{Conditional Processing}
\label{sec:conditional}

The package provides a mechanism to compile different versions
of a document. To customise the versions further some conditional processing
can come in handy to distinguish which version is being compiled.
The package provides two macros to describe the compilation context:

%%%%%%%%%%%%%%%%%%%%%%%%%%%%%%%%%%%%%%%%
\DescribeMacro{\ifchilddoc}
The conditional |\ifchilddoc| distinguishes between the compilation of
child documents and the main document:
%
\begin{center}
|\ifchilddoc |\textit{child-code}| |[|\||else |\textit{main-code}]| \||fi|
\end{center}

%%%%%%%%%%%%%%%%%%%%%%%%%%%%%%%%%%%%%%%%
\DescribeMacro{\childdocname}
\DescribeMacro{\childdocjob}
The macro |\childdocname| contains the filename (without extension)
of the main or child file being processed.
Note that |\childdocjob| will always contain the name of the main file.

%%%%%%%%%%%%%%%%%%%%%%%%%%%%%%%%%%%%%%%%
\paragraph{Title Page.}

Conditional processing can be used to include a title or banner page
in the main document when proper precautions are taken.
Importantly, the code in the main file should ensure that the page counter
(as well as other status parameters which are stored in the |.aux| files)
takes the same value after the conditional processing.
Otherwise the page numbers may take divergent values
depending on which part is compiled.

For example, a title page could be declared by:
%
\begin{center}
\begin{tabular}{l}
|\ifchilddoc\||else|\\
|\addtocounter{page}{-1}|\\
\textit{code for title page}\\
|\newpage|\\
|\||fi|
\end{tabular}
\end{center}
%
A banner page for the child documents can be generated by:
%
\begin{center}
\begin{tabular}{l}
|\ifchilddoc|\\
|\addtocounter{page}{-1}|\\
\textit{code for banner page}\\
|\newpage|\\
|\||fi|
\end{tabular}
\end{center}
%
Here one could write a message such as:
\begin{center}
|This is the part \childdocname{} of \childdocjob{}.|
\end{center}

%%%%%%%%%%%%%%%%%%%%%%%%%%%%%%%%%%%%%%%%%%%%%%%%%%%%%%%%%%%%%%%%%%%%%%%%%%%%%%%%
\subsection{Flags}
\label{sec:flags}

The package makes it easy to generate different versions
of the main or child documents.
To this end compilation flags can be defined
and assigned different default values.
They will be particularly useful in conjunction
with the forwarding mechanism described in \secref{sec:forward}.

For example, it may be useful to have a flag |\version|
which can be set to |draft| or |final|.
The document source will contain some conditional code
depending on the value of |\version|.
Suppose further, the flag should default to |final| for the main file
and to |draft| for child files
which is a natural assignment for editing the document.
This is achieved by placing the following code
in the preamble of the main document
(below the |\childdocmain| directive):
%
\begin{center}
\begin{tabular}{l}
|\ifchilddoc|\\
|\providecommand{\version}{draft}|\\
|\||else|\\
|\providecommand{\version}{final}|\\
|\||fi|
\end{tabular}
\end{center}
%
The definition by |\providecommand| makes sure
that previous definitions are not overwritten.
Further statements |\providecommand{\version}{...}|
can thus be added before the above code to override it.

For the main file, one might add a line
(between |\childdocmain| and the above block)
%
\begin{center}
|%\ifchilddoc\||else\providecommand{\version}{draft}\||fi|
\end{center}
%
which can be uncommented to produce a draft version.
Likewise one can add a line to the very top of a child file
(above the |\childdocof{|\textit{main}|}| directive)
%
\begin{center}
|%\providecommand{\version}{final}|
\end{center}
%
which can be uncommented to produce the final version of this child document.

%%%%%%%%%%%%%%%%%%%%%%%%%%%%%%%%%%%%%%%%%%%%%%%%%%%%%%%%%%%%%%%%%%%%%%%%%%%%%%%%
\subsection{Forwarding}
\label{sec:forward}

Different versions of the main or child documents
using compilation flags as described in \secref{sec:flags}
can be (permanently) stored in different files
for convenient compilation, viewing and distribution.
To this end, the package defines a command
to pass on compilation to a different file:

%%%%%%%%%%%%%%%%%%%%%%%%%%%%%%%%%%%%%%%%
\DescribeMacro{\childdocforward}
The command |\childdocforward| redirects processing to
another source file:
%
\begin{center}
\begin{tabular}{l}
|\input{childdoc.def}|\\
|\childdocforward[|\textit{main}|]{|\textit{dest}|}|\\
\end{tabular}
\end{center}
%
The argument \textit{dest} is the destination file
(without extension).
It should be the main file or one of the child files.
Note that further \textsf{childdoc} directives
such as |\childdocof| and |\childdocforward|
in the indicated file will be processed in this form.
The optional argument \textit{main}
passes on directly to the main file \textit{main}
while pretending to compile the child \textit{dest}.
This form behaves as if \textit{dest}
issues |\childdocof{|\textit{main}|}| right away,
and no further \textsf{childdoc} directives will be processed.

%%%%%%%%%%%%%%%%%%%%%%%%%%%%%%%%%%%%%%%%
\DescribeMacro{\...prefix}
In the alternative form |\childdocforwardprefix|,
%
\begin{center}
\begin{tabular}{l}
|\input{childdoc.def}|\\
|\childdocforwardprefix[|\textit{main}|]{|\textit{prefix}|}{|\textit{dest}|}|
\end{tabular}
\end{center}
%
the destination file is determined by a pattern
depending on the current file:
To make this work, the current file must be called
`{\textit{prefix}\hspace{0.2em}\textit{suffix}}'
with \textit{prefix} matching precisely the argument.
Processing is then passed on to the file
`{\textit{dest}\hspace{0.2em}\textit{suffix}}'.
Surely, the same effect is achieved by
directly specifying the
argument `{\textit{dest}\hspace{0.2em}\textit{suffix}}'
in the first form.
However, that requires to set up a different file
for each child. With the alternative form of the command
all these files can have exactly the same content
which simplifies setting them up and maintaining them.

For example, the following file |draft.tex|
with a compilation flag |\version| as described in \secref{sec:flags}
compiles the main document as a draft:
%
\begin{center}
\begin{tabular}{l}
|\def\version{draft}|\\
|\input{childdoc.def}|\\
|\childdocforward{|\textit{main}|}|
\end{tabular}
\end{center}
%
Likewise, the following files |final|\textit{nn}|.tex|
compile the final version of the child document
|child|\textit{nn}|.tex|:
%
\begin{center}
\begin{tabular}{l}
|\def\version{final}|\\
|\input{childdoc.def}|\\
|\childdocforwardprefix{final}{child}|
\end{tabular}
\end{center}
%

Note that when several versions of a main file and/or of each child file
are to be generated, it may be convenient to set up a |Makefile| or
shell script to automatise the process.

%%%%%%%%%%%%%%%%%%%%%%%%%%%%%%%%%%%%%%%%%%%%%%%%%%%%%%%%%%%%%%%%%%%%%%%%%%%%%%%%
\subsection{Command Line Processing}
\label{sec:commandline}

The effect of redirection files can also be achieved by invoking
the \LaTeX{} compiler with a more elaborate command line.
Most conveniently this should be done as part
of a shell script or a |Makefile|.

When using \textsf{childdoc} in the main file, the following
command lines effectively perform a redirection
(note that depending on the shell being used,
backslashes may have to be doubled: `|\|' $\to$ `|\\|'):
%
\begin{center}
|... -jobname "|\textit{target}|" |\\|"|[\textit{flags}]%
|\input{childdoc.def}\childdocforward[|\textit{main}|]{|\textit{dest}|}"|
\end{center}
%
Here \textit{target} is the name of the output file,
\textit{main} is the name of the main file
and \textit{dest} is the name of the main or child file to be processed
(all filenames without extensions).
The optional argument \textit{main} can be omitted
if \textit{main} matches \textit{dest}.
Optionally, compilation \textit{flags} can be defined via |\def| commands.
This command line makes the \TeX{} engine believe
it is compiling the file \textit{target}
whose content is specified as the latter parameter.
The provided code then forwards the processing to
\textit{main} or \textit{dest} as described in \secref{sec:forward}.

%%%%%%%%%%%%%%%%%%%%%%%%%%%%%%%%%%%%%%%%%%%%%%%%%%%%%%%%%%%%%%%%%%%%%%%%%%%%%%%%
\subsection{Include by Input}
\label{sec:input}

Including child documents by |\include| has some restrictions by design.
Most notably, the content of a child document always occupies
its own set of pages; pages cannot be shared between child documents.
Usually, this behaviour makes perfect sense
because each child document contain an essential part of the document.
However, in some situations it may be desirable to compose
a document from a collection of parts
without having mandatory page breaks between then.
For this case, the package
provides a mechanism to include parts
by |\input| which can also be processed individually.
However, by construction this mechanism
requires manual handling of the content to be output.

%%%%%%%%%%%%%%%%%%%%%%%%%%%%%%%%%%%%%%%%
\DescribeMacro{\ifchilddocmanual}
The main file should be prepared as usual, see \secref{sec:include}.
However, the document body must make a distinction
between processing of an individual part and of the main document, e.g.:
%
\begin{center}
\begin{tabular}{l}
|\ifchilddocmanual|\\
|\input{\childdocname}|\\
|\||else|\\
\textit{document body with }|\input{|\textit{part}|}|\\
|\||fi|
\end{tabular}
\end{center}
%
The conditional |\ifchilddocmanual| is true whenever
a part to be included by |\input| is being compiled,
and the name of the part is stored in |\childdocname|.

%%%%%%%%%%%%%%%%%%%%%%%%%%%%%%%%%%%%%%%%
\DescribeMacro{\childdocby}
Each part to be included by |\input| should start with:
%
\begin{center}
\begin{tabular}{l}
|\input{childdoc.def}|\\
|\childdocby{|\textit{main}|}|\\
\end{tabular}
\end{center}
%
The directive |\childdocby| is similar to |\childdocof|
described in \secref{sec:include},
but the subsequent selection of content must be done manually.
To that end, both |\ifchilddoc| and |\ifchilddocmanual|
will be true upon processing of a part,
and the name of the part is stored in |\childdocname|.
Note that |\jobname| will be set to the filename of the current part
so that each part receives an individual |.aux| file
that does not interfere with the |.aux| file(s) of the main document.
This behaviour can be altered by the alternative form
|\childdocby[*]{|\textit{main}|}| (with a non-empty optional argument)
which uses the |.aux| file of the main document
by setting |\jobname| to \textit{main}.

%%%%%%%%%%%%%%%%%%%%%%%%%%%%%%%%%%%%%%%%%%%%%%%%%%%%%%%%%%%%%%%%%%%%%%%%%%%%%%%%
\subsection{Driver Development}
\label{sec:driver}

The \textsf{childdoc} mechanism can also be use for the development
of definition files such as \LaTeX{} styles or classes.
This case differs from the above setup with multiple parts
included by |\include| in that no |\includeonly| should be invoked.
This can be achieved by starting the include file
(before |\ProvidesPackage|) with:
%
\begin{center}
\begin{tabular}{l}
|\input{childdoc.def}|\\
|\childdocforward{|\textit{main}|}|\\
\end{tabular}
\end{center}
%
or alternatively with:
%
\begin{center}
\begin{tabular}{l}
|\input{childdoc.def}|\\
|\childdocby{|\textit{main}|}|\\
\end{tabular}
\end{center}
%
Both forms have slightly different effects as described above.
The main file is prepared as usual, see \secref{sec:include}.

%%%%%%%%%%%%%%%%%%%%%%%%%%%%%%%%%%%%%%%%%%%%%%%%%%%%%%%%%%%%%%%%%%%%%%%%%%%%%%%%
\subsection{Legacy Detection}
\label{sec:detection}

The directive |\childdocmain| in the main file can detect
whether the complete document or merely a child is to be compiled
even without using the directive |\childdocof|.
This method is deprecated because it is less robust
and there is no compelling reason to use it;
it is merely provided for backward compatibility
and it may be removed in future versions.

If the detection mechanism is to be used,
it is mandatory to correctly specify
the filename of the main file as the argument of |\childdocmain|:
%
\begin{center}
\begin{tabular}{l}
|\input{childdoc.def}|\\
|\childdocmain{|\textit{main}|}|\\
\end{tabular}
\end{center}
%
If |\jobname| does not match the argument \textit{main} of |\childdocmain|,
it is assumed that |\jobname| points to the child file to be compiled.
When using |\childdocmain| with the main file specified as argument,
it suffices to start a child file
with just |\input{|\textit{main}|}|
without loading of the package and using |\childdocof|.
If instead all processing is done
with the appropriate \textsf{childdoc} directives,
the argument of \textit{main} of |\childdocmain| can be empty.

An alternative version of the command line processing described
in \secref{sec:commandline} using the detection mechanism reads:
%
\begin{center}
|... -jobname "|\textit{target}|" "|[\textit{flags}]%
[|\def\jobname{|\textit{dest}|}|]|\input{|\textit{main}|}"|
\end{center}

%%%%%%%%%%%%%%%%%%%%%%%%%%%%%%%%%%%%%%%%%%%%%%%%%%%%%%%%%%%%%%%%%%%%%%%%%%%%%%%%
\subsection{Manual Code}
\label{sec:manual}

In case one cannot be certain whether the definitions file |childdoc.def|
is installed on the target \TeX{} distribution
and one prefers not to ship it,
it is conceivable to paste a few relevant commands into the sources.

To that end, drop all statements |\input{childdoc.def}|
and perform the replacements as outlined below.
Instead of |\childdocmain{|\textit{main}|}| add the following code
to the top of the main file:
%
\begin{center}
\begin{tabular}{l}
|\||ifdefined\childdocname\endinput\||fi\newif\ifchilddoc|\\
|\edef\childdocname{\scantokens\expandafter{\jobname\noexpand}}|\\
|\def\childdocmain{|\textit{main}|}\||ifx\childdocmain\childdocname\||else|\\
|\childdoctrue\includeonly{\childdocname}\let\jobname\childdocmain\||fi|\\
\end{tabular}
\end{center}
%
Instead of |\childdocof{|\textit{main}|}| just include the main file
at the top of each child file:
%
\begin{center}
|\input{|\textit{main}|}|
\end{center}
%
A simple redirection |\childdocforward{|\textit{dest}|}| is achieved by:
%
\begin{center}
|\def\jobname{|\textit{dest}|}\input{\jobname}|
\end{center}
%
The redirection with prefix
|\childdocforwardprefix[|\textit{prefix}|]{|\textit{dest}|}|
is accomplished by:
%
\begin{center}
\begin{tabular}{l}
|{\edef\jobname{\scantokens\expandafter{\jobname\noexpand}}|\\
|\def\redirectjob |\textit{prefix}|#1~~~{\gdef\jobname{|\textit{dest}|#1}}|\\
|\expandafter\redirectjob\jobname~~~}\input{\jobname}|
\end{tabular}
\end{center}

In an alternative approach,
child documents can be compiled by a specific command line
without additional code or specific definitions:
%
\begin{center}
|... -jobname "|\textit{target}|" "|[\textit{flags}]%
|\includeonly{|\textit{dest}|}\input{|\textit{main}|}"|
\end{center}
%

%%%%%%%%%%%%%%%%%%%%%%%%%%%%%%%%%%%%%%%%%%%%%%%%%%%%%%%%%%%%%%%%%%%%%%%%%%%%%%%%
%%%%%%%%%%%%%%%%%%%%%%%%%%%%%%%%%%%%%%%%%%%%%%%%%%%%%%%%%%%%%%%%%%%%%%%%%%%%%%%%
\section{Information}

%%%%%%%%%%%%%%%%%%%%%%%%%%%%%%%%%%%%%%%%%%%%%%%%%%%%%%%%%%%%%%%%%%%%%%%%%%%%%%%%
\subsection{Copyright}

Copyright \copyright{} 2017--2018 Niklas Beisert

This work may be distributed and/or modified under the
conditions of the \LaTeX{} Project Public License, either version 1.3
of this license or (at your option) any later version.
The latest version of this license is in
  \url{http://www.latex-project.org/lppl.txt}
and version 1.3 or later is part of all distributions of \LaTeX{}
version 2005/12/01 or later.

This work has the LPPL maintenance status `maintained'.

The Current Maintainer of this work is Niklas Beisert.

This work consists of the files |README.txt|, |childdoc.ins| and |childdoc.dtx|
as well as the derived files |childdoc.def|, |cdocsamp.tex|
with |cdocsch1.tex|, |cdocsch2.tex|, |cdocspt3.tex|, |cdocspt4.tex|,
|cdocsdrf.tex|, |cdocsfn1.tex|, |cdocsfn2.tex|
as well as |childdoc.pdf|.

%%%%%%%%%%%%%%%%%%%%%%%%%%%%%%%%%%%%%%%%%%%%%%%%%%%%%%%%%%%%%%%%%%%%%%%%%%%%%%%%
\subsection{Files and Installation}

The package consists of the files:
%
\begin{center}
\begin{tabular}{ll}
    |README.txt|   & readme file \\
    |childdoc.ins| & installation file \\
    |childdoc.dtx| & source file \\
    |childdoc.def| & definition file \\
    |cdocsamp.tex| & sample main file \\
    |cdocsch1.tex| & sample include file \\
    |cdocsch2.tex| & sample include file \\
    |cdocspt3.tex| & sample part file \\
    |cdocspt4.tex| & sample part file \\
    |cdocsdrf.tex| & sample redirection file \\
    |cdocsfn1.tex| & sample redirection file \\
    |cdocsfn2.tex| & sample redirection file \\
    |childdoc.pdf| & manual
\end{tabular}
\end{center}
%
The distribution consists of the files
|README.txt|, |childdoc.ins| and |childdoc.dtx|.
%
\begin{itemize}
\item
Run (pdf)\LaTeX{} on |childdoc.dtx|
to compile the manual |childdoc.pdf| (this file).
\item
Run \LaTeX{} on |childdoc.ins| to create the definitions file |childdoc.def|
and the sample |cdocsamp.tex| with include files
|cdocsch1.tex|, |cdocsch2.tex|, |cdocspt3.tex|, |cdocspt4.tex|,
|cdocsdrf.tex|, |cdocsfn1.tex|, |cdocsfn2.tex|.
Then copy the file |childdoc.def| to an appropriate directory of your \LaTeX{}
distribution, e.g.\ \textit{texmf-root}|/tex/latex/childdoc|.
\end{itemize}

%%%%%%%%%%%%%%%%%%%%%%%%%%%%%%%%%%%%%%%%%%%%%%%%%%%%%%%%%%%%%%%%%%%%%%%%%%%%%%%%
\subsection{Related CTAN Packages}

There are several other packages which offer a similar functionality:
%
\begin{itemize}
\item
The packages
\href{http://ctan.org/pkg/docmute}{\textsf{docmute}},
\href{http://ctan.org/pkg/includex}{\textsf{includex}} and
\href{http://ctan.org/pkg/standalone}{\textsf{standalone}}
provide commands to include only the document body of
a child file thus allowing both files to be compiled individually.
\item
The packages \href{http://ctan.org/pkg/subdocs}{\textsf{subdocs}}
and \href{http://ctan.org/pkg/subfiles}{\textsf{subfiles}}
provide structures in which the main and child documents can be
encapsulated and allowing them to be compiled individually.
The inclusion mechanism is different from the conventional |\include|.
\item
The package \href{http://ctan.org/pkg/combine}{\textsf{combine}}
is an elaborate solution to combine several documents into one.
\end{itemize}
%
See also the CTAN topic \href{http://ctan.org/topic/subdocs}{\textsf{subdocs}}
for further related packages.
The present package differs from the above solutions in that
a document structure constructed with the conventional |\include| mechanism
just needs two extra commands at the top of every file
such that all constituent files can be compiled individually.

%%%%%%%%%%%%%%%%%%%%%%%%%%%%%%%%%%%%%%%%%%%%%%%%%%%%%%%%%%%%%%%%%%%%%%%%%%%%%%%%
%\subsection{Feature Suggestions}
%
%The following is a list of features which may be useful for future
%versions of this package:
%%
%\begin{itemize}
%\item
%\ldots
%\end{itemize}

%%%%%%%%%%%%%%%%%%%%%%%%%%%%%%%%%%%%%%%%%%%%%%%%%%%%%%%%%%%%%%%%%%%%%%%%%%%%%%%%
\subsection{Revision History}

%%%%%%%%%%%%%%%%%%%%%%%%%%%%%%%%%%%%%%%%
\paragraph{v2.0:} 2018/12/30

\begin{itemize}
\item
immediate forward processing
\item
added |\childdocby| mechanism
\item
manual restructured
\end{itemize}

%%%%%%%%%%%%%%%%%%%%%%%%%%%%%%%%%%%%%%%%
\paragraph{v1.6:} 2018/01/17

\begin{itemize}
\item
application for development of include files
\item
corrections to manual
\end{itemize}

%%%%%%%%%%%%%%%%%%%%%%%%%%%%%%%%%%%%%%%%
\paragraph{v1.5:} 2017/05/21

\begin{itemize}
\item
more complete structuring introduced
\item
|\childdocof| introduced
\item
|\childdoc| renamed to |\childdocmain|
\item
|\childredirect| renamed to |\childdocforward| and |\childdocforwardprefix|
and functionality expanded
\end{itemize}

%%%%%%%%%%%%%%%%%%%%%%%%%%%%%%%%%%%%%%%%
\paragraph{v1.0:} 2017/04/27

\begin{itemize}
\item
manual and install package
\item
first version published on CTAN
\end{itemize}

%%%%%%%%%%%%%%%%%%%%%%%%%%%%%%%%%%%%%%%%
\paragraph{v0.6:} 2017/04/26

\begin{itemize}
\item
redirection mechanism added
\end{itemize}

%%%%%%%%%%%%%%%%%%%%%%%%%%%%%%%%%%%%%%%%
\paragraph{v0.5:} 2017/04/26

\begin{itemize}
\item
functionality in definition file
\end{itemize}


%%%%%%%%%%%%%%%%%%%%%%%%%%%%%%%%%%%%%%%%%%%%%%%%%%%%%%%%%%%%%%%%%%%%%%%%%%%%%%%%
%%%%%%%%%%%%%%%%%%%%%%%%%%%%%%%%%%%%%%%%%%%%%%%%%%%%%%%%%%%%%%%%%%%%%%%%%%%%%%%%
%%%%%%%%%%%%%%%%%%%%%%%%%%%%%%%%%%%%%%%%%%%%%%%%%%%%%%%%%%%%%%%%%%%%%%%%%%%%%%%%
\appendix

\settowidth\MacroIndent{\rmfamily\scriptsize 000\ }

 \DocInput{childdoc.dtx}

\end{document}
%</driver>
% \fi
%
% %%%%%%%%%%%%%%%%%%%%%%%%%%%%%%%%%%%%%%%%%%%%%%%%%%%%%%%%%%%%%%%%%%%%%%%%%%%%%%
% %%%%%%%%%%%%%%%%%%%%%%%%%%%%%%%%%%%%%%%%%%%%%%%%%%%%%%%%%%%%%%%%%%%%%%%%%%%%%%
% \section{Sample}
%\iffalse
%<*samplemain>
%\fi
%
% The following presents a sample document
% with two chapters, two parts, a title page,
% a compile flag as well as three forwarding files to set the flag.
% It consists of eight |.tex| files:
% \begin{center}
% \begin{tabular}{ll}
% |cdocsamp.tex|&main file\\
% |cdocsch1.tex|&include file for chapter 1\\
% |cdocsch2.tex|&include file for chapter 2\\
% |cdocspt3.tex|&include file for part 3\\
% |cdocspt4.tex|&include file for part 4\\
% |cdocsdrf.tex|&forwarding file for main file in draft mode\\
% |cdocsfi1.tex|&forwarding file for final version of chapter 1\\
% |cdocsfi2.tex|&forwarding file for final version of chapter 2\\
% \end{tabular}
% \end{center}
% Each of the eight files can be compiled directly by the \LaTeX{} compiler.
%
% %%%%%%%%%%%%%%%%%%%%%%%%%%%%%%%%%%%%%%
% \paragraph{Main File.}
%
% The main file is called |cdocsamp.tex|.
%
% Load the \textsf{childdoc} definitions and
% declare the filename for the main document:
%    \begin{macrocode}
\input{childdoc.def}
\childdocmain{}
%    \end{macrocode}

% Optional override for |\version| flag:
%    \begin{macrocode}
%%\ifchilddoc\else\providecommand{\version}{draft}\fi
%    \end{macrocode}

% Define the default values for the |\version| flag
% (|final| for the main file and |draft| for childs):
%    \begin{macrocode}
\ifchilddoc
\providecommand{\version}{draft}
\else
\providecommand{\version}{final}
\fi
%    \end{macrocode}

% Load the standard document class:
%    \begin{macrocode}
\documentclass[12pt]{article}
%    \end{macrocode}

% Start the document body:
%    \begin{macrocode}
\begin{document}
%    \end{macrocode}

% Declare a title page.
% Print title, part of document being processed and version flag:
%    \begin{macrocode}
\addtocounter{page}{-1}
\begin{center}
{\LARGE\bfseries{}childdoc example\par}
\vspace{1cm}
\ifchilddoc
\ifchilddocmanual part\else chapter\fi:
`\childdocname' of `\childdocjob'\par
\else
main document: `\childdocjob'\par
\fi
version: \version\par
\end{center}
\newpage
%    \end{macrocode}

% Manually include selected file,
% otherwise process as usual:
%    \begin{macrocode}
\ifchilddocmanual
\section*{part `\childdocname'}
\input{\childdocname}
\else
%    \end{macrocode}

% Include the two chapters:
%    \begin{macrocode}
\include{cdocsch1}
\include{cdocsch2}
%    \end{macrocode}

% Include the two parts unless only chapters should be displayed:
%    \begin{macrocode}
\ifchilddoc\else
\section{part three}
\input{cdocspt3}
\section{part four}
\input{cdocspt4}
\fi
%    \end{macrocode}

% Process as usual until here:
%    \begin{macrocode}
\fi
%    \end{macrocode}

% End of document body:
%    \begin{macrocode}
\end{document}
%    \end{macrocode}
%\iffalse
%</samplemain>
%\fi
%
% %%%%%%%%%%%%%%%%%%%%%%%%%%%%%%%%%%%%%%
% \paragraph{Chapter Include Files.}
%
% The include files are called |cdocsch1.tex| and |cdocsch2.tex|.
%
%\iffalse
%<*samplechap1|samplechap2>
%\fi

% Optional override for |\version| flag:
%    \begin{macrocode}
%%\providecommand{\version}{final}
%    \end{macrocode}

% Include the main document:
%    \begin{macrocode}
\input{childdoc.def}
\childdocof{cdocsamp}
%    \end{macrocode}

%\iffalse
%</samplechap1|samplechap2>
%\fi
%
%\iffalse
%<*samplechap1>
%\fi
% Some text for chapter 1:
%    \begin{macrocode}
\section{one}
some text in chapter one
%    \end{macrocode}

%\iffalse
%</samplechap1>
%\fi
% Some text for chapter 2:
%\iffalse
%<*samplechap2>
%\fi
%    \begin{macrocode}
\section{two}
more text in chapter two
%    \end{macrocode}

%\iffalse
%</samplechap2>
%\fi
%
% %%%%%%%%%%%%%%%%%%%%%%%%%%%%%%%%%%%%%%
% \paragraph{Part Include Files.}
%
% The include files are called |cdocspt3.tex| and |cdocspt4.tex|.
%
%\iffalse
%<*samplepart3|samplepart4>
%\fi

% Optional override for |\version| flag:
%    \begin{macrocode}
%%\providecommand{\version}{final}
%    \end{macrocode}

% Include the main document:
%    \begin{macrocode}
\input{childdoc.def}
\childdocby{cdocsamp}
%    \end{macrocode}

%\iffalse
%</samplepart3|samplepart4>
%\fi
%
%\iffalse
%<*samplepart3>
%\fi
% Some text for part 3:
%    \begin{macrocode}
some text in part three
%    \end{macrocode}

%\iffalse
%</samplepart3>
%\fi
% Some text for part 4:
%\iffalse
%<*samplepart4>
%\fi
%    \begin{macrocode}
more text in part four
%    \end{macrocode}

%\iffalse
%</samplepart4>
%\fi
%
% %%%%%%%%%%%%%%%%%%%%%%%%%%%%%%%%%%%%%%
% \paragraph{Forwarding for a Complete Draft.}
%
% The following forwarding file |cdocsdrf.tex|
% compiles the main document in draft mode:
%\iffalse
%<*sampledraft>
%\fi
%    \begin{macrocode}
\def\version{draft}
\input{childdoc.def}
\childdocforward{cdocsamp}
%    \end{macrocode}

%\iffalse
%</sampledraft>
%\fi
%
% %%%%%%%%%%%%%%%%%%%%%%%%%%%%%%%%%%%%%%
% \paragraph{Forwarding for Final Version of the Chapters.}
%
% The following forwarding files |cdocsfn1.tex| and |cdocsfn2.tex|
% (with identical content)
% compile the final versions of the child documents
% |cdocsch1.tex| and |cdocsch2.tex|, respectively:
%\iffalse
%<*samplefinal>
%\fi
%    \begin{macrocode}
\def\version{final}
\input{childdoc.def}
\childdocforwardprefix[cdocsamp]{cdocsfn}{cdocsch}
%    \end{macrocode}

%\iffalse
%</samplefinal>
%\fi
%
% %%%%%%%%%%%%%%%%%%%%%%%%%%%%%%%%%%%%%%
% \paragraph{Command Line Processing.}
%
% The following three command lines generate the output files
% |cdocscld|, |cdocscl1| and |cdocscl2|
% which should be identical to
% |cdocsdrf|, |cdocsch1| and |cdocsfn2|, respectively:
% \begin{center}
% \begin{tabular}{l}
% |latex -jobname cdocscld \|\\
% |  "\def\version{draft}\input{childdoc.def}\childdocforward{cdocsamp}"|\\
% |latex -jobname cdocscl1 \|\\
% |  "\input{childdoc.def}\childdocforward[cdocsamp]{cdocsch1}"|\\
% |latex -jobname cdocscl2 \|\\
% |  "\def\version{final}\input{childdoc.def}\childdocforward{cdocsch2}"|
% \end{tabular}
% \end{center}
% Note that the trailing backslash on each first line
% merely continues the input to the second line
% (for convenient cut ant paste).
% Furthermore, the command |latex| can be replaced by any
% of its alternative versions such as |pdflatex|.
%
% %%%%%%%%%%%%%%%%%%%%%%%%%%%%%%%%%%%%%%%%%%%%%%%%%%%%%%%%%%%%%%%%%%%%%%%%%%%%%%
% %%%%%%%%%%%%%%%%%%%%%%%%%%%%%%%%%%%%%%%%%%%%%%%%%%%%%%%%%%%%%%%%%%%%%%%%%%%%%%
% \section{Implementation}
%\iffalse
%<*package>
%\fi
%
% This section describes the definitions file |childdoc.def|.

% The definitions cannot be loaded using |\usepackage| or |\RequirePackage|
% which has a mechanism to prevent loading a style file more than once.
% When loading the definitions by means of |\input|
% multiple instances have to be prevented manually:
%\iffalse
%This code needs to be before the `\ProvidesFile' directive
%which is defined at the beginning of this file.
%Therefore it is also placed there and commented out here.
%</package>
%<*discard>
%\fi
%    \begin{macrocode}
\ifdefined\childdocmain\endinput\fi
%    \end{macrocode}
%\iffalse
%</discard>
%<*package>
%\fi
%
% \macro{\ifchilddoc}
% \macro{\ifchilddocmanual}
% The conditional |\ifchilddoc| tells whether a
% child (true) or main (false) document is being compiled.
% The conditional |\ifchilddocmanual| tells whether
% the |\includeonly| mechanism is used (false) or
% the selection of child files must be performed manually (true).
% The definitions initialise to false:
%    \begin{macrocode}
\newif\ifchilddoc
\newif\ifchilddocmanual
%    \end{macrocode}

% \macro{\childdocname}
% \macro{\childdocjob}
% The macro |\childdocname| stores the name of the main document
% to be compiled. The macro |\childdocjob| stores the name of
% the document on which the \LaTeX{} compiler was originally invoked.
% The content of |\jobname| cannot be compared
% to filenames specified in the source due to different catcodes.
% The following code rescans |\jobname|, stores the result
% in |\childdocname| and saves a copy in |\childdocjob|:
%    \begin{macrocode}
\edef\childdocname{\scantokens\expandafter{\jobname\noexpand}}
\let\childdocjob\childdocname
%    \end{macrocode}

% \macro{\childdocdisable}
% The macro |\childdocdisable| prevents the main file
% from being processed more than once.
% At this stage, the main document command |\childdocmain|
% is assumed to be called once again where it should do nothing.
% Any subsequent call to it should prevent
% a secondary processing of the main document
% It overwrites the forwarding commands
% |\childdocof| and |\childdocforward|
% with empty macros to prevent further inclusions of the main document:
%    \begin{macrocode}
\newcommand{\childdocdisable}
{
  \renewcommand{\childdocmain}[1]{\renewcommand{\childdocmain}[1]{\endinput}}
  \renewcommand{\childdocof}[1]{}
  \renewcommand{\childdocby}[2][]{}
  \renewcommand{\childdocforward}[2][]{}
  \renewcommand{\childdocdisable}{}
}
%    \end{macrocode}

% \macro{\childdocmain}
% The macro |\childdocmain| is to be called at the top of the main file
% with nothing or the main filename (without extension) as argument.
% First, it breaks loops.
% If the argument is not empty and does not match |\childdocname|
% (which is set by the first inclusion of |childdoc.def|),
% |\ifchilddoc| is set to true, |\includeonly| is applied to the child file
% and |\jobname| is set to the main file
% (for proper handling of |.aux| files):
%    \begin{macrocode}
\newcommand{\childdocmain}[1]
{
  \childdocdisable\childdocmain{}
  \if?#1?\else
    \begingroup
      \def\childdoctmp{#1}
      \ifx\childdoctmp\childdocname
        \def\childdoctmp{}
      \else
        \def\childdoctmp
        {
          \childdoctrue
          \includeonly{\childdocname}
          \def\childdocjob{#1}
          \def\jobname{#1}
        }
      \fi
      \expandafter
    \endgroup
    \childdoctmp
  \fi
}
%    \end{macrocode}

% \macro{\childdocof}
% The command |\childdocof| redirects
% compilation to the main file |#1|.
%    \begin{macrocode}
\newcommand{\childdocof}[1]
{
  \childdocdisable
  \childdoctrue
  \includeonly{\childdocname}
  \def\jobname{#1}
  \def\childdocjob{#1}
  \input{#1}
}
%    \end{macrocode}

% \macro{\childdocby}
% The command |\childdocby| ....
%    \begin{macrocode}
\newcommand{\childdocby}[2][]
{
  \childdocdisable
  \childdoctrue
  \childdocmanualtrue
  \if?#1?\else
    \def\jobname{#2}
  \fi
  \def\childdocjob{#2}
  \input{#2}
  \endinput
}
%    \end{macrocode}

% \macro{\childdocforward}
% The command |\childdocforward| redirects
% compilation to the main file or
% (if the optional argument is given) a child file.
% Parameters are set as if the main file
% or a child file starting with |\childdocof| was compiled.
% Then compilation is handed over to the main file:
%    \begin{macrocode}
\newcommand{\childdocforward}[2][]
{
  \begingroup
    \if?#1?
      \def\childdoctmp
      {
        \def\childdocname{#2}
        \def\childdocjob{#2}
        \def\jobname{#2}
        \input{#2}
        \endinput
      }
    \else
      \def\childdoctmp
      {
        \childdocdisable
        \def\childdocname{#2}
        \childdoctrue
        \includeonly{#2}
        \def\childdocjob{#1}
        \def\jobname{#1}
        \input{#1}
        \endinput
      }
    \fi
    \expandafter
  \endgroup
  \childdoctmp
}
%    \end{macrocode}

% \macro{\childdocforwardprefix}
% The command |\childdocforwardprefix| redirects
% compilation to the main or a child file by means of a pattern.
% The prefix |#1| in the current filename is replaced by |#2|
% and the suffix of the current filename is kept
% (it is assumed that the filename does not contain the substring `|~~~|'
% which is used as a delimiter).
% Compilation is handed over to the new file by |\childdocforward|:
%    \begin{macrocode}
\newcommand{\childdocforwardprefix}[3][]
{
  \begingroup
    \def\childdocextract #2##1~~~{\def\childdoctmp{\childdocforward[#1]{#3##1}}}
    \expandafter\childdocextract\childdocname~~~
    \expandafter
  \endgroup
  \childdoctmp
}
%    \end{macrocode}

% \macro{\childdoc}
% The deprecated macro |\childdoc| is a legacy version of |\childdocmain|:
%    \begin{macrocode}
\newcommand{\childdoc}{\childdocmain}
%    \end{macrocode}

% \macro{\childdocredirect}
% The deprecated macro |\childdocredirect| is a legacy version
% of |\childdocforward| and |\childdocforwardprefix|:
%    \begin{macrocode}
\newcommand{\childdocredirect}[2][]
{
  \begingroup
    \if?#1?
      \def\childdoctmp{\childdocforward{#2}}
    \else
      \def\childdoctmp{\childdocforwardprefix{#1}{#2}}
    \fi
    \expandafter
  \endgroup
  \childdoctmp
}
%    \end{macrocode}

%\iffalse
%</package>
%\fi
%
\endinput
|\\
|\childdocby{|\textit{main}|}|\\
\end{tabular}
\end{center}
%
Both forms have slightly different effects as described above.
The main file is prepared as usual, see \secref{sec:include}.

%%%%%%%%%%%%%%%%%%%%%%%%%%%%%%%%%%%%%%%%%%%%%%%%%%%%%%%%%%%%%%%%%%%%%%%%%%%%%%%%
\subsection{Legacy Detection}
\label{sec:detection}

The directive |\childdocmain| in the main file can detect
whether the complete document or merely a child is to be compiled
even without using the directive |\childdocof|.
This method is deprecated because it is less robust
and there is no compelling reason to use it;
it is merely provided for backward compatibility
and it may be removed in future versions.

If the detection mechanism is to be used,
it is mandatory to correctly specify
the filename of the main file as the argument of |\childdocmain|:
%
\begin{center}
\begin{tabular}{l}
|% \iffalse
%
% childdoc.dtx Copyright (C) 2017-2018 Niklas Beisert
%
% This work may be distributed and/or modified under the
% conditions of the LaTeX Project Public License, either version 1.3
% of this license or (at your option) any later version.
% The latest version of this license is in
%   http://www.latex-project.org/lppl.txt
% and version 1.3 or later is part of all distributions of LaTeX
% version 2005/12/01 or later.
%
% This work has the LPPL maintenance status `maintained'.
%
% The Current Maintainer of this work is Niklas Beisert.
%
% This work consists of the files childdoc.dtx and childdoc.ins
% and the derived files childdoc.def and cdocsamp.tex with
% cdocsch1.tex, cdocsch2.tex, cdocsdrf.tex, cdocsfn1.tex, cdocsfn2.tex.
%
%<package>\ifdefined\childdocmain\endinput\fi
%<package>\ProvidesFile{childdoc.def}[2018/12/30 v2.0 child document driver]
%<samplemain>\ProvidesFile{cdocsamp.tex}[2018/12/30 v2.0 sample for childdoc]
%<*driver>
%\ProvidesFile{childdoc.drv}[2018/12/30 v2.0 childdoc reference manual file]
\PassOptionsToClass{10pt,a4paper}{article}
\documentclass{ltxdoc}

\usepackage[margin=35mm]{geometry}
\usepackage{hyperref}
\usepackage{hyperxmp}
\usepackage[usenames]{color}

\hypersetup{colorlinks=true}
\hypersetup{pdfstartview=FitH}
\hypersetup{pdfpagemode=UseNone}
\hypersetup{pdfsource={}}
\hypersetup{pdflang={en-UK}}
\hypersetup{pdfcopyright={Copyright 2017-2018 Niklas Beisert.
  This work may be distributed and/or modified under the
  conditions of the LaTeX Project Public License, either version 1.3
  of this license or (at your option) any later version.}}
\hypersetup{pdflicenseurl={http://www.latex-project.org/lppl.txt}}
\hypersetup{pdfcontactaddress={ETH Zurich, ITP, HIT K,
  Wolfgang-Pauli-Strasse 27}}
\hypersetup{pdfcontactpostcode={8093}}
\hypersetup{pdfcontactcity={Zurich}}
\hypersetup{pdfcontactcountry={Switzerland}}
\hypersetup{pdfcontactemail={nbeisert@itp.phys.ethz.ch}}
\hypersetup{pdfcontacturl={http://people.phys.ethz.ch/\xmptilde nbeisert/}}

\newcommand{\secref}[1]{\hyperref[#1]{section \ref*{#1}}}

\parskip1ex
\parindent0pt
\let\olditemize\itemize
\def\itemize{\olditemize\parskip0pt}

\begin{document}

\title{The \textsf{childdoc} Package}
\hypersetup{pdftitle={The childdoc Package}}
\author{Niklas Beisert\\[2ex]
  Institut f\"ur Theoretische Physik\\
  Eidgen\"ossische Technische Hochschule Z\"urich\\
  Wolfgang-Pauli-Strasse 27, 8093 Z\"urich, Switzerland\\[1ex]
  \href{mailto:nbeisert@itp.phys.ethz.ch}
  {\texttt{nbeisert@itp.phys.ethz.ch}}}
\hypersetup{pdfauthor={Niklas Beisert}}
\hypersetup{pdfsubject={Manual for the LaTeX2e Package childdoc}}
\date{30 December 2018, \textsf{v2.0}}
\maketitle

\begin{abstract}\noindent
\textsf{childdoc} is a \LaTeXe{} package
that enables the direct compilation
of document sections included by |\include|
to individual files.
\end{abstract}

\begingroup
\parskip0ex
\tableofcontents
\endgroup

%%%%%%%%%%%%%%%%%%%%%%%%%%%%%%%%%%%%%%%%%%%%%%%%%%%%%%%%%%%%%%%%%%%%%%%%%%%%%%%%
%%%%%%%%%%%%%%%%%%%%%%%%%%%%%%%%%%%%%%%%%%%%%%%%%%%%%%%%%%%%%%%%%%%%%%%%%%%%%%%%
\section{Introduction}

\LaTeX{} provides a mechanism to structure a large document (such as a book)
into a main file and several child files (containing the chapters)
using the |\include| command.
This mechanism is beneficial for documents
which span hundreds of pages in order to
make the source file(s) more manageable.
Moreover, compilation can be restricted to
selected child files by means of the |\includeonly| command.
The latter feature can be used to reduce the compilation time while editing
(this was significantly more useful in the earlier days of \LaTeX{})
or to generate a smaller document which is easier to navigate.
Another application of |\includeonly| is to generate
documents consisting of selected parts of the complete document.

However, there are a few drawbacks of the plain |\include| mechanism:
\begin{itemize}
\item
The child files cannot be compiled on their own,
they can only be compiled via the main file.
A naive editing environment
(such as a text editor with an option
to have the current file processed by \LaTeX)
may require one to switch to the main file before compiling;
attempting to compile the child file produces errors.
\item
The main file must be modified (each time)
to adjust the |\includeonly| command
to the present needs. This easily leaves the main file in a messy state.
\item
The generated document will always carry the filename
of the main document. This is inconvenient if
several child files are to be compiled and
to be kept for distribution.
\end{itemize}

The present package provides a simple interface
to make child files individually compilable by \LaTeX{}.
Compiling a child file then has the same effect as compiling
the main file with an |\includeonly| command
to select the appropriate child.
Moreover the generated document will carry the name of the child
rather than the main file.
This resolves all three above issues.

This feature is meant to make the editing of books,
thesis documents and lecture notes somewhat more convenient.
However, the package can also be used efficiently for
composing a series of documents (such as exercise sheets)
which are typically distributed individually.
It then assists the author in generating the individual documents
(potentially in different versions)
as well as a document containing the collected series.
Another application is in developing style files
or other kinds of included material
where compilation of the style file could redirect
to a sample or test file.

%%%%%%%%%%%%%%%%%%%%%%%%%%%%%%%%%%%%%%%%%%%%%%%%%%%%%%%%%%%%%%%%%%%%%%%%%%%%%%%%
%%%%%%%%%%%%%%%%%%%%%%%%%%%%%%%%%%%%%%%%%%%%%%%%%%%%%%%%%%%%%%%%%%%%%%%%%%%%%%%%
\section{Usage}

First of all, the package \textsf{childdoc} is \emph{not} a standard
\LaTeXe{} |.sty| style file! Therefore it needs to be invoked in
a non-standard way.

%%%%%%%%%%%%%%%%%%%%%%%%%%%%%%%%%%%%%%%%%%%%%%%%%%%%%%%%%%%%%%%%%%%%%%%%%%%%%%%%
\subsection{Included Files}
\label{sec:include}

%%%%%%%%%%%%%%%%%%%%%%%%%%%%%%%%%%%%%%%%
\DescribeMacro{\childdocmain}
To use the package, add the commands
\begin{center}
\begin{tabular}{l}
|\input{childdoc.def}|\\
|\childdocmain{}|\\
\end{tabular}
\end{center}
at the very top of the main \LaTeX{} file,
in particular \emph{before} the |\documentclass| statement!
The argument of |\childdocmain| should be left empty
(but it must be present).

%%%%%%%%%%%%%%%%%%%%%%%%%%%%%%%%%%%%%%%%
\DescribeMacro{\childdocof}
Furthermore, add the commands
\begin{center}
\begin{tabular}{l}
|\input{childdoc.def}|\\
|\childdocof{|\textit{main}|}|\\
\end{tabular}
\end{center}
at the top of every child file \textit{child}
which is included by |\include{|\textit{child}|}|
from within the main file
(or at least for those files to be compiled individually).
The argument \textit{main} must be the filename of the main file.

There are a couple of
considerations in setting up the main and child documents:

%%%%%%%%%%%%%%%%%%%%%%%%%%%%%%%%%%%%%%%%
\paragraph{Restrictions.}

Please note the following restrictions:
\begin{itemize}
\item
|\childdocmain| must be called with one argument \textit{main}
to ensure compatibility with earlier version of the package.
It must either be empty (|\childdocmain{}|)
or precisely match the filename of the main file in which it is specified.
See \secref{sec:detection} for further information.
\item
The filename \textit{main} must be specified without the |.tex| extension.
\item
The filename \textit{main} is case sensitive
(even in case-insensitive file systems)
due to internal string comparison.
\item
The argument \textit{main} should be fully expanded, it cannot be a macro.
\item
Subdirectories and special characters should be avoided in filenames.
\item
The command |\childdocmain{|\textit{main}|}| must be followed by a whitespace.
It should not be followed immediately by another command
or by a comment mark `|%|'.
This is because the \TeX{} parser reads the token immediately following
the argument of |\childdocmain| and puts it
at the beginning of every child section;
however, a white\-space is ignored.
\end{itemize}

%%%%%%%%%%%%%%%%%%%%%%%%%%%%%%%%%%%%%%%%
\paragraph{Content of Main File.}

It is advisable to place all content in the child files included by |\include|.
Any output contained in the main file will appear in all child documents
unless suppressed manually;
it cannot be suppressed automatically by the |\includeonly| directive
and thus should normally be avoided.
A method to include some content in the main file
by means of conditional processing is described in \secref{sec:conditional}.

%%%%%%%%%%%%%%%%%%%%%%%%%%%%%%%%%%%%%%%%
\paragraph{Page Numbering.}

When only a part of the document is compiled,
the appropriate numbering of pages
(as well as other status parameters)
is determined from the |.aux| files.
The latter contain information from previous passes.
However this information needs to propagate through
all intermediate child documents.
Therefore the page numbering in child documents may well
be inconsistent until the complete document is compiled at least once.

A useful (if unconventional) way to always ensure a consistent
page numbering is to restart the numbering in each child document
and denote the pages by `\textit{child}|.|\textit{page}'
where \textit{child} represents the chapter/section number of the child file.
This can be achieved by the command
|\numberwithin{page}{|\textit{child}|}|
of the \textsf{amsmath} package
where \textit{child} can be |chapter| or |section|
depending on the chosen structuring.
Alternatively, one can modify the macro |\thepage| appropriately
and reset the counter |page| at the start of each child file.

%%%%%%%%%%%%%%%%%%%%%%%%%%%%%%%%%%%%%%%%%%%%%%%%%%%%%%%%%%%%%%%%%%%%%%%%%%%%%%%%
\subsection{Conditional Processing}
\label{sec:conditional}

The package provides a mechanism to compile different versions
of a document. To customise the versions further some conditional processing
can come in handy to distinguish which version is being compiled.
The package provides two macros to describe the compilation context:

%%%%%%%%%%%%%%%%%%%%%%%%%%%%%%%%%%%%%%%%
\DescribeMacro{\ifchilddoc}
The conditional |\ifchilddoc| distinguishes between the compilation of
child documents and the main document:
%
\begin{center}
|\ifchilddoc |\textit{child-code}| |[|\||else |\textit{main-code}]| \||fi|
\end{center}

%%%%%%%%%%%%%%%%%%%%%%%%%%%%%%%%%%%%%%%%
\DescribeMacro{\childdocname}
\DescribeMacro{\childdocjob}
The macro |\childdocname| contains the filename (without extension)
of the main or child file being processed.
Note that |\childdocjob| will always contain the name of the main file.

%%%%%%%%%%%%%%%%%%%%%%%%%%%%%%%%%%%%%%%%
\paragraph{Title Page.}

Conditional processing can be used to include a title or banner page
in the main document when proper precautions are taken.
Importantly, the code in the main file should ensure that the page counter
(as well as other status parameters which are stored in the |.aux| files)
takes the same value after the conditional processing.
Otherwise the page numbers may take divergent values
depending on which part is compiled.

For example, a title page could be declared by:
%
\begin{center}
\begin{tabular}{l}
|\ifchilddoc\||else|\\
|\addtocounter{page}{-1}|\\
\textit{code for title page}\\
|\newpage|\\
|\||fi|
\end{tabular}
\end{center}
%
A banner page for the child documents can be generated by:
%
\begin{center}
\begin{tabular}{l}
|\ifchilddoc|\\
|\addtocounter{page}{-1}|\\
\textit{code for banner page}\\
|\newpage|\\
|\||fi|
\end{tabular}
\end{center}
%
Here one could write a message such as:
\begin{center}
|This is the part \childdocname{} of \childdocjob{}.|
\end{center}

%%%%%%%%%%%%%%%%%%%%%%%%%%%%%%%%%%%%%%%%%%%%%%%%%%%%%%%%%%%%%%%%%%%%%%%%%%%%%%%%
\subsection{Flags}
\label{sec:flags}

The package makes it easy to generate different versions
of the main or child documents.
To this end compilation flags can be defined
and assigned different default values.
They will be particularly useful in conjunction
with the forwarding mechanism described in \secref{sec:forward}.

For example, it may be useful to have a flag |\version|
which can be set to |draft| or |final|.
The document source will contain some conditional code
depending on the value of |\version|.
Suppose further, the flag should default to |final| for the main file
and to |draft| for child files
which is a natural assignment for editing the document.
This is achieved by placing the following code
in the preamble of the main document
(below the |\childdocmain| directive):
%
\begin{center}
\begin{tabular}{l}
|\ifchilddoc|\\
|\providecommand{\version}{draft}|\\
|\||else|\\
|\providecommand{\version}{final}|\\
|\||fi|
\end{tabular}
\end{center}
%
The definition by |\providecommand| makes sure
that previous definitions are not overwritten.
Further statements |\providecommand{\version}{...}|
can thus be added before the above code to override it.

For the main file, one might add a line
(between |\childdocmain| and the above block)
%
\begin{center}
|%\ifchilddoc\||else\providecommand{\version}{draft}\||fi|
\end{center}
%
which can be uncommented to produce a draft version.
Likewise one can add a line to the very top of a child file
(above the |\childdocof{|\textit{main}|}| directive)
%
\begin{center}
|%\providecommand{\version}{final}|
\end{center}
%
which can be uncommented to produce the final version of this child document.

%%%%%%%%%%%%%%%%%%%%%%%%%%%%%%%%%%%%%%%%%%%%%%%%%%%%%%%%%%%%%%%%%%%%%%%%%%%%%%%%
\subsection{Forwarding}
\label{sec:forward}

Different versions of the main or child documents
using compilation flags as described in \secref{sec:flags}
can be (permanently) stored in different files
for convenient compilation, viewing and distribution.
To this end, the package defines a command
to pass on compilation to a different file:

%%%%%%%%%%%%%%%%%%%%%%%%%%%%%%%%%%%%%%%%
\DescribeMacro{\childdocforward}
The command |\childdocforward| redirects processing to
another source file:
%
\begin{center}
\begin{tabular}{l}
|\input{childdoc.def}|\\
|\childdocforward[|\textit{main}|]{|\textit{dest}|}|\\
\end{tabular}
\end{center}
%
The argument \textit{dest} is the destination file
(without extension).
It should be the main file or one of the child files.
Note that further \textsf{childdoc} directives
such as |\childdocof| and |\childdocforward|
in the indicated file will be processed in this form.
The optional argument \textit{main}
passes on directly to the main file \textit{main}
while pretending to compile the child \textit{dest}.
This form behaves as if \textit{dest}
issues |\childdocof{|\textit{main}|}| right away,
and no further \textsf{childdoc} directives will be processed.

%%%%%%%%%%%%%%%%%%%%%%%%%%%%%%%%%%%%%%%%
\DescribeMacro{\...prefix}
In the alternative form |\childdocforwardprefix|,
%
\begin{center}
\begin{tabular}{l}
|\input{childdoc.def}|\\
|\childdocforwardprefix[|\textit{main}|]{|\textit{prefix}|}{|\textit{dest}|}|
\end{tabular}
\end{center}
%
the destination file is determined by a pattern
depending on the current file:
To make this work, the current file must be called
`{\textit{prefix}\hspace{0.2em}\textit{suffix}}'
with \textit{prefix} matching precisely the argument.
Processing is then passed on to the file
`{\textit{dest}\hspace{0.2em}\textit{suffix}}'.
Surely, the same effect is achieved by
directly specifying the
argument `{\textit{dest}\hspace{0.2em}\textit{suffix}}'
in the first form.
However, that requires to set up a different file
for each child. With the alternative form of the command
all these files can have exactly the same content
which simplifies setting them up and maintaining them.

For example, the following file |draft.tex|
with a compilation flag |\version| as described in \secref{sec:flags}
compiles the main document as a draft:
%
\begin{center}
\begin{tabular}{l}
|\def\version{draft}|\\
|\input{childdoc.def}|\\
|\childdocforward{|\textit{main}|}|
\end{tabular}
\end{center}
%
Likewise, the following files |final|\textit{nn}|.tex|
compile the final version of the child document
|child|\textit{nn}|.tex|:
%
\begin{center}
\begin{tabular}{l}
|\def\version{final}|\\
|\input{childdoc.def}|\\
|\childdocforwardprefix{final}{child}|
\end{tabular}
\end{center}
%

Note that when several versions of a main file and/or of each child file
are to be generated, it may be convenient to set up a |Makefile| or
shell script to automatise the process.

%%%%%%%%%%%%%%%%%%%%%%%%%%%%%%%%%%%%%%%%%%%%%%%%%%%%%%%%%%%%%%%%%%%%%%%%%%%%%%%%
\subsection{Command Line Processing}
\label{sec:commandline}

The effect of redirection files can also be achieved by invoking
the \LaTeX{} compiler with a more elaborate command line.
Most conveniently this should be done as part
of a shell script or a |Makefile|.

When using \textsf{childdoc} in the main file, the following
command lines effectively perform a redirection
(note that depending on the shell being used,
backslashes may have to be doubled: `|\|' $\to$ `|\\|'):
%
\begin{center}
|... -jobname "|\textit{target}|" |\\|"|[\textit{flags}]%
|\input{childdoc.def}\childdocforward[|\textit{main}|]{|\textit{dest}|}"|
\end{center}
%
Here \textit{target} is the name of the output file,
\textit{main} is the name of the main file
and \textit{dest} is the name of the main or child file to be processed
(all filenames without extensions).
The optional argument \textit{main} can be omitted
if \textit{main} matches \textit{dest}.
Optionally, compilation \textit{flags} can be defined via |\def| commands.
This command line makes the \TeX{} engine believe
it is compiling the file \textit{target}
whose content is specified as the latter parameter.
The provided code then forwards the processing to
\textit{main} or \textit{dest} as described in \secref{sec:forward}.

%%%%%%%%%%%%%%%%%%%%%%%%%%%%%%%%%%%%%%%%%%%%%%%%%%%%%%%%%%%%%%%%%%%%%%%%%%%%%%%%
\subsection{Include by Input}
\label{sec:input}

Including child documents by |\include| has some restrictions by design.
Most notably, the content of a child document always occupies
its own set of pages; pages cannot be shared between child documents.
Usually, this behaviour makes perfect sense
because each child document contain an essential part of the document.
However, in some situations it may be desirable to compose
a document from a collection of parts
without having mandatory page breaks between then.
For this case, the package
provides a mechanism to include parts
by |\input| which can also be processed individually.
However, by construction this mechanism
requires manual handling of the content to be output.

%%%%%%%%%%%%%%%%%%%%%%%%%%%%%%%%%%%%%%%%
\DescribeMacro{\ifchilddocmanual}
The main file should be prepared as usual, see \secref{sec:include}.
However, the document body must make a distinction
between processing of an individual part and of the main document, e.g.:
%
\begin{center}
\begin{tabular}{l}
|\ifchilddocmanual|\\
|\input{\childdocname}|\\
|\||else|\\
\textit{document body with }|\input{|\textit{part}|}|\\
|\||fi|
\end{tabular}
\end{center}
%
The conditional |\ifchilddocmanual| is true whenever
a part to be included by |\input| is being compiled,
and the name of the part is stored in |\childdocname|.

%%%%%%%%%%%%%%%%%%%%%%%%%%%%%%%%%%%%%%%%
\DescribeMacro{\childdocby}
Each part to be included by |\input| should start with:
%
\begin{center}
\begin{tabular}{l}
|\input{childdoc.def}|\\
|\childdocby{|\textit{main}|}|\\
\end{tabular}
\end{center}
%
The directive |\childdocby| is similar to |\childdocof|
described in \secref{sec:include},
but the subsequent selection of content must be done manually.
To that end, both |\ifchilddoc| and |\ifchilddocmanual|
will be true upon processing of a part,
and the name of the part is stored in |\childdocname|.
Note that |\jobname| will be set to the filename of the current part
so that each part receives an individual |.aux| file
that does not interfere with the |.aux| file(s) of the main document.
This behaviour can be altered by the alternative form
|\childdocby[*]{|\textit{main}|}| (with a non-empty optional argument)
which uses the |.aux| file of the main document
by setting |\jobname| to \textit{main}.

%%%%%%%%%%%%%%%%%%%%%%%%%%%%%%%%%%%%%%%%%%%%%%%%%%%%%%%%%%%%%%%%%%%%%%%%%%%%%%%%
\subsection{Driver Development}
\label{sec:driver}

The \textsf{childdoc} mechanism can also be use for the development
of definition files such as \LaTeX{} styles or classes.
This case differs from the above setup with multiple parts
included by |\include| in that no |\includeonly| should be invoked.
This can be achieved by starting the include file
(before |\ProvidesPackage|) with:
%
\begin{center}
\begin{tabular}{l}
|\input{childdoc.def}|\\
|\childdocforward{|\textit{main}|}|\\
\end{tabular}
\end{center}
%
or alternatively with:
%
\begin{center}
\begin{tabular}{l}
|\input{childdoc.def}|\\
|\childdocby{|\textit{main}|}|\\
\end{tabular}
\end{center}
%
Both forms have slightly different effects as described above.
The main file is prepared as usual, see \secref{sec:include}.

%%%%%%%%%%%%%%%%%%%%%%%%%%%%%%%%%%%%%%%%%%%%%%%%%%%%%%%%%%%%%%%%%%%%%%%%%%%%%%%%
\subsection{Legacy Detection}
\label{sec:detection}

The directive |\childdocmain| in the main file can detect
whether the complete document or merely a child is to be compiled
even without using the directive |\childdocof|.
This method is deprecated because it is less robust
and there is no compelling reason to use it;
it is merely provided for backward compatibility
and it may be removed in future versions.

If the detection mechanism is to be used,
it is mandatory to correctly specify
the filename of the main file as the argument of |\childdocmain|:
%
\begin{center}
\begin{tabular}{l}
|\input{childdoc.def}|\\
|\childdocmain{|\textit{main}|}|\\
\end{tabular}
\end{center}
%
If |\jobname| does not match the argument \textit{main} of |\childdocmain|,
it is assumed that |\jobname| points to the child file to be compiled.
When using |\childdocmain| with the main file specified as argument,
it suffices to start a child file
with just |\input{|\textit{main}|}|
without loading of the package and using |\childdocof|.
If instead all processing is done
with the appropriate \textsf{childdoc} directives,
the argument of \textit{main} of |\childdocmain| can be empty.

An alternative version of the command line processing described
in \secref{sec:commandline} using the detection mechanism reads:
%
\begin{center}
|... -jobname "|\textit{target}|" "|[\textit{flags}]%
[|\def\jobname{|\textit{dest}|}|]|\input{|\textit{main}|}"|
\end{center}

%%%%%%%%%%%%%%%%%%%%%%%%%%%%%%%%%%%%%%%%%%%%%%%%%%%%%%%%%%%%%%%%%%%%%%%%%%%%%%%%
\subsection{Manual Code}
\label{sec:manual}

In case one cannot be certain whether the definitions file |childdoc.def|
is installed on the target \TeX{} distribution
and one prefers not to ship it,
it is conceivable to paste a few relevant commands into the sources.

To that end, drop all statements |\input{childdoc.def}|
and perform the replacements as outlined below.
Instead of |\childdocmain{|\textit{main}|}| add the following code
to the top of the main file:
%
\begin{center}
\begin{tabular}{l}
|\||ifdefined\childdocname\endinput\||fi\newif\ifchilddoc|\\
|\edef\childdocname{\scantokens\expandafter{\jobname\noexpand}}|\\
|\def\childdocmain{|\textit{main}|}\||ifx\childdocmain\childdocname\||else|\\
|\childdoctrue\includeonly{\childdocname}\let\jobname\childdocmain\||fi|\\
\end{tabular}
\end{center}
%
Instead of |\childdocof{|\textit{main}|}| just include the main file
at the top of each child file:
%
\begin{center}
|\input{|\textit{main}|}|
\end{center}
%
A simple redirection |\childdocforward{|\textit{dest}|}| is achieved by:
%
\begin{center}
|\def\jobname{|\textit{dest}|}\input{\jobname}|
\end{center}
%
The redirection with prefix
|\childdocforwardprefix[|\textit{prefix}|]{|\textit{dest}|}|
is accomplished by:
%
\begin{center}
\begin{tabular}{l}
|{\edef\jobname{\scantokens\expandafter{\jobname\noexpand}}|\\
|\def\redirectjob |\textit{prefix}|#1~~~{\gdef\jobname{|\textit{dest}|#1}}|\\
|\expandafter\redirectjob\jobname~~~}\input{\jobname}|
\end{tabular}
\end{center}

In an alternative approach,
child documents can be compiled by a specific command line
without additional code or specific definitions:
%
\begin{center}
|... -jobname "|\textit{target}|" "|[\textit{flags}]%
|\includeonly{|\textit{dest}|}\input{|\textit{main}|}"|
\end{center}
%

%%%%%%%%%%%%%%%%%%%%%%%%%%%%%%%%%%%%%%%%%%%%%%%%%%%%%%%%%%%%%%%%%%%%%%%%%%%%%%%%
%%%%%%%%%%%%%%%%%%%%%%%%%%%%%%%%%%%%%%%%%%%%%%%%%%%%%%%%%%%%%%%%%%%%%%%%%%%%%%%%
\section{Information}

%%%%%%%%%%%%%%%%%%%%%%%%%%%%%%%%%%%%%%%%%%%%%%%%%%%%%%%%%%%%%%%%%%%%%%%%%%%%%%%%
\subsection{Copyright}

Copyright \copyright{} 2017--2018 Niklas Beisert

This work may be distributed and/or modified under the
conditions of the \LaTeX{} Project Public License, either version 1.3
of this license or (at your option) any later version.
The latest version of this license is in
  \url{http://www.latex-project.org/lppl.txt}
and version 1.3 or later is part of all distributions of \LaTeX{}
version 2005/12/01 or later.

This work has the LPPL maintenance status `maintained'.

The Current Maintainer of this work is Niklas Beisert.

This work consists of the files |README.txt|, |childdoc.ins| and |childdoc.dtx|
as well as the derived files |childdoc.def|, |cdocsamp.tex|
with |cdocsch1.tex|, |cdocsch2.tex|, |cdocspt3.tex|, |cdocspt4.tex|,
|cdocsdrf.tex|, |cdocsfn1.tex|, |cdocsfn2.tex|
as well as |childdoc.pdf|.

%%%%%%%%%%%%%%%%%%%%%%%%%%%%%%%%%%%%%%%%%%%%%%%%%%%%%%%%%%%%%%%%%%%%%%%%%%%%%%%%
\subsection{Files and Installation}

The package consists of the files:
%
\begin{center}
\begin{tabular}{ll}
    |README.txt|   & readme file \\
    |childdoc.ins| & installation file \\
    |childdoc.dtx| & source file \\
    |childdoc.def| & definition file \\
    |cdocsamp.tex| & sample main file \\
    |cdocsch1.tex| & sample include file \\
    |cdocsch2.tex| & sample include file \\
    |cdocspt3.tex| & sample part file \\
    |cdocspt4.tex| & sample part file \\
    |cdocsdrf.tex| & sample redirection file \\
    |cdocsfn1.tex| & sample redirection file \\
    |cdocsfn2.tex| & sample redirection file \\
    |childdoc.pdf| & manual
\end{tabular}
\end{center}
%
The distribution consists of the files
|README.txt|, |childdoc.ins| and |childdoc.dtx|.
%
\begin{itemize}
\item
Run (pdf)\LaTeX{} on |childdoc.dtx|
to compile the manual |childdoc.pdf| (this file).
\item
Run \LaTeX{} on |childdoc.ins| to create the definitions file |childdoc.def|
and the sample |cdocsamp.tex| with include files
|cdocsch1.tex|, |cdocsch2.tex|, |cdocspt3.tex|, |cdocspt4.tex|,
|cdocsdrf.tex|, |cdocsfn1.tex|, |cdocsfn2.tex|.
Then copy the file |childdoc.def| to an appropriate directory of your \LaTeX{}
distribution, e.g.\ \textit{texmf-root}|/tex/latex/childdoc|.
\end{itemize}

%%%%%%%%%%%%%%%%%%%%%%%%%%%%%%%%%%%%%%%%%%%%%%%%%%%%%%%%%%%%%%%%%%%%%%%%%%%%%%%%
\subsection{Related CTAN Packages}

There are several other packages which offer a similar functionality:
%
\begin{itemize}
\item
The packages
\href{http://ctan.org/pkg/docmute}{\textsf{docmute}},
\href{http://ctan.org/pkg/includex}{\textsf{includex}} and
\href{http://ctan.org/pkg/standalone}{\textsf{standalone}}
provide commands to include only the document body of
a child file thus allowing both files to be compiled individually.
\item
The packages \href{http://ctan.org/pkg/subdocs}{\textsf{subdocs}}
and \href{http://ctan.org/pkg/subfiles}{\textsf{subfiles}}
provide structures in which the main and child documents can be
encapsulated and allowing them to be compiled individually.
The inclusion mechanism is different from the conventional |\include|.
\item
The package \href{http://ctan.org/pkg/combine}{\textsf{combine}}
is an elaborate solution to combine several documents into one.
\end{itemize}
%
See also the CTAN topic \href{http://ctan.org/topic/subdocs}{\textsf{subdocs}}
for further related packages.
The present package differs from the above solutions in that
a document structure constructed with the conventional |\include| mechanism
just needs two extra commands at the top of every file
such that all constituent files can be compiled individually.

%%%%%%%%%%%%%%%%%%%%%%%%%%%%%%%%%%%%%%%%%%%%%%%%%%%%%%%%%%%%%%%%%%%%%%%%%%%%%%%%
%\subsection{Feature Suggestions}
%
%The following is a list of features which may be useful for future
%versions of this package:
%%
%\begin{itemize}
%\item
%\ldots
%\end{itemize}

%%%%%%%%%%%%%%%%%%%%%%%%%%%%%%%%%%%%%%%%%%%%%%%%%%%%%%%%%%%%%%%%%%%%%%%%%%%%%%%%
\subsection{Revision History}

%%%%%%%%%%%%%%%%%%%%%%%%%%%%%%%%%%%%%%%%
\paragraph{v2.0:} 2018/12/30

\begin{itemize}
\item
immediate forward processing
\item
added |\childdocby| mechanism
\item
manual restructured
\end{itemize}

%%%%%%%%%%%%%%%%%%%%%%%%%%%%%%%%%%%%%%%%
\paragraph{v1.6:} 2018/01/17

\begin{itemize}
\item
application for development of include files
\item
corrections to manual
\end{itemize}

%%%%%%%%%%%%%%%%%%%%%%%%%%%%%%%%%%%%%%%%
\paragraph{v1.5:} 2017/05/21

\begin{itemize}
\item
more complete structuring introduced
\item
|\childdocof| introduced
\item
|\childdoc| renamed to |\childdocmain|
\item
|\childredirect| renamed to |\childdocforward| and |\childdocforwardprefix|
and functionality expanded
\end{itemize}

%%%%%%%%%%%%%%%%%%%%%%%%%%%%%%%%%%%%%%%%
\paragraph{v1.0:} 2017/04/27

\begin{itemize}
\item
manual and install package
\item
first version published on CTAN
\end{itemize}

%%%%%%%%%%%%%%%%%%%%%%%%%%%%%%%%%%%%%%%%
\paragraph{v0.6:} 2017/04/26

\begin{itemize}
\item
redirection mechanism added
\end{itemize}

%%%%%%%%%%%%%%%%%%%%%%%%%%%%%%%%%%%%%%%%
\paragraph{v0.5:} 2017/04/26

\begin{itemize}
\item
functionality in definition file
\end{itemize}


%%%%%%%%%%%%%%%%%%%%%%%%%%%%%%%%%%%%%%%%%%%%%%%%%%%%%%%%%%%%%%%%%%%%%%%%%%%%%%%%
%%%%%%%%%%%%%%%%%%%%%%%%%%%%%%%%%%%%%%%%%%%%%%%%%%%%%%%%%%%%%%%%%%%%%%%%%%%%%%%%
%%%%%%%%%%%%%%%%%%%%%%%%%%%%%%%%%%%%%%%%%%%%%%%%%%%%%%%%%%%%%%%%%%%%%%%%%%%%%%%%
\appendix

\settowidth\MacroIndent{\rmfamily\scriptsize 000\ }

 \DocInput{childdoc.dtx}

\end{document}
%</driver>
% \fi
%
% %%%%%%%%%%%%%%%%%%%%%%%%%%%%%%%%%%%%%%%%%%%%%%%%%%%%%%%%%%%%%%%%%%%%%%%%%%%%%%
% %%%%%%%%%%%%%%%%%%%%%%%%%%%%%%%%%%%%%%%%%%%%%%%%%%%%%%%%%%%%%%%%%%%%%%%%%%%%%%
% \section{Sample}
%\iffalse
%<*samplemain>
%\fi
%
% The following presents a sample document
% with two chapters, two parts, a title page,
% a compile flag as well as three forwarding files to set the flag.
% It consists of eight |.tex| files:
% \begin{center}
% \begin{tabular}{ll}
% |cdocsamp.tex|&main file\\
% |cdocsch1.tex|&include file for chapter 1\\
% |cdocsch2.tex|&include file for chapter 2\\
% |cdocspt3.tex|&include file for part 3\\
% |cdocspt4.tex|&include file for part 4\\
% |cdocsdrf.tex|&forwarding file for main file in draft mode\\
% |cdocsfi1.tex|&forwarding file for final version of chapter 1\\
% |cdocsfi2.tex|&forwarding file for final version of chapter 2\\
% \end{tabular}
% \end{center}
% Each of the eight files can be compiled directly by the \LaTeX{} compiler.
%
% %%%%%%%%%%%%%%%%%%%%%%%%%%%%%%%%%%%%%%
% \paragraph{Main File.}
%
% The main file is called |cdocsamp.tex|.
%
% Load the \textsf{childdoc} definitions and
% declare the filename for the main document:
%    \begin{macrocode}
\input{childdoc.def}
\childdocmain{}
%    \end{macrocode}

% Optional override for |\version| flag:
%    \begin{macrocode}
%%\ifchilddoc\else\providecommand{\version}{draft}\fi
%    \end{macrocode}

% Define the default values for the |\version| flag
% (|final| for the main file and |draft| for childs):
%    \begin{macrocode}
\ifchilddoc
\providecommand{\version}{draft}
\else
\providecommand{\version}{final}
\fi
%    \end{macrocode}

% Load the standard document class:
%    \begin{macrocode}
\documentclass[12pt]{article}
%    \end{macrocode}

% Start the document body:
%    \begin{macrocode}
\begin{document}
%    \end{macrocode}

% Declare a title page.
% Print title, part of document being processed and version flag:
%    \begin{macrocode}
\addtocounter{page}{-1}
\begin{center}
{\LARGE\bfseries{}childdoc example\par}
\vspace{1cm}
\ifchilddoc
\ifchilddocmanual part\else chapter\fi:
`\childdocname' of `\childdocjob'\par
\else
main document: `\childdocjob'\par
\fi
version: \version\par
\end{center}
\newpage
%    \end{macrocode}

% Manually include selected file,
% otherwise process as usual:
%    \begin{macrocode}
\ifchilddocmanual
\section*{part `\childdocname'}
\input{\childdocname}
\else
%    \end{macrocode}

% Include the two chapters:
%    \begin{macrocode}
\include{cdocsch1}
\include{cdocsch2}
%    \end{macrocode}

% Include the two parts unless only chapters should be displayed:
%    \begin{macrocode}
\ifchilddoc\else
\section{part three}
\input{cdocspt3}
\section{part four}
\input{cdocspt4}
\fi
%    \end{macrocode}

% Process as usual until here:
%    \begin{macrocode}
\fi
%    \end{macrocode}

% End of document body:
%    \begin{macrocode}
\end{document}
%    \end{macrocode}
%\iffalse
%</samplemain>
%\fi
%
% %%%%%%%%%%%%%%%%%%%%%%%%%%%%%%%%%%%%%%
% \paragraph{Chapter Include Files.}
%
% The include files are called |cdocsch1.tex| and |cdocsch2.tex|.
%
%\iffalse
%<*samplechap1|samplechap2>
%\fi

% Optional override for |\version| flag:
%    \begin{macrocode}
%%\providecommand{\version}{final}
%    \end{macrocode}

% Include the main document:
%    \begin{macrocode}
\input{childdoc.def}
\childdocof{cdocsamp}
%    \end{macrocode}

%\iffalse
%</samplechap1|samplechap2>
%\fi
%
%\iffalse
%<*samplechap1>
%\fi
% Some text for chapter 1:
%    \begin{macrocode}
\section{one}
some text in chapter one
%    \end{macrocode}

%\iffalse
%</samplechap1>
%\fi
% Some text for chapter 2:
%\iffalse
%<*samplechap2>
%\fi
%    \begin{macrocode}
\section{two}
more text in chapter two
%    \end{macrocode}

%\iffalse
%</samplechap2>
%\fi
%
% %%%%%%%%%%%%%%%%%%%%%%%%%%%%%%%%%%%%%%
% \paragraph{Part Include Files.}
%
% The include files are called |cdocspt3.tex| and |cdocspt4.tex|.
%
%\iffalse
%<*samplepart3|samplepart4>
%\fi

% Optional override for |\version| flag:
%    \begin{macrocode}
%%\providecommand{\version}{final}
%    \end{macrocode}

% Include the main document:
%    \begin{macrocode}
\input{childdoc.def}
\childdocby{cdocsamp}
%    \end{macrocode}

%\iffalse
%</samplepart3|samplepart4>
%\fi
%
%\iffalse
%<*samplepart3>
%\fi
% Some text for part 3:
%    \begin{macrocode}
some text in part three
%    \end{macrocode}

%\iffalse
%</samplepart3>
%\fi
% Some text for part 4:
%\iffalse
%<*samplepart4>
%\fi
%    \begin{macrocode}
more text in part four
%    \end{macrocode}

%\iffalse
%</samplepart4>
%\fi
%
% %%%%%%%%%%%%%%%%%%%%%%%%%%%%%%%%%%%%%%
% \paragraph{Forwarding for a Complete Draft.}
%
% The following forwarding file |cdocsdrf.tex|
% compiles the main document in draft mode:
%\iffalse
%<*sampledraft>
%\fi
%    \begin{macrocode}
\def\version{draft}
\input{childdoc.def}
\childdocforward{cdocsamp}
%    \end{macrocode}

%\iffalse
%</sampledraft>
%\fi
%
% %%%%%%%%%%%%%%%%%%%%%%%%%%%%%%%%%%%%%%
% \paragraph{Forwarding for Final Version of the Chapters.}
%
% The following forwarding files |cdocsfn1.tex| and |cdocsfn2.tex|
% (with identical content)
% compile the final versions of the child documents
% |cdocsch1.tex| and |cdocsch2.tex|, respectively:
%\iffalse
%<*samplefinal>
%\fi
%    \begin{macrocode}
\def\version{final}
\input{childdoc.def}
\childdocforwardprefix[cdocsamp]{cdocsfn}{cdocsch}
%    \end{macrocode}

%\iffalse
%</samplefinal>
%\fi
%
% %%%%%%%%%%%%%%%%%%%%%%%%%%%%%%%%%%%%%%
% \paragraph{Command Line Processing.}
%
% The following three command lines generate the output files
% |cdocscld|, |cdocscl1| and |cdocscl2|
% which should be identical to
% |cdocsdrf|, |cdocsch1| and |cdocsfn2|, respectively:
% \begin{center}
% \begin{tabular}{l}
% |latex -jobname cdocscld \|\\
% |  "\def\version{draft}\input{childdoc.def}\childdocforward{cdocsamp}"|\\
% |latex -jobname cdocscl1 \|\\
% |  "\input{childdoc.def}\childdocforward[cdocsamp]{cdocsch1}"|\\
% |latex -jobname cdocscl2 \|\\
% |  "\def\version{final}\input{childdoc.def}\childdocforward{cdocsch2}"|
% \end{tabular}
% \end{center}
% Note that the trailing backslash on each first line
% merely continues the input to the second line
% (for convenient cut ant paste).
% Furthermore, the command |latex| can be replaced by any
% of its alternative versions such as |pdflatex|.
%
% %%%%%%%%%%%%%%%%%%%%%%%%%%%%%%%%%%%%%%%%%%%%%%%%%%%%%%%%%%%%%%%%%%%%%%%%%%%%%%
% %%%%%%%%%%%%%%%%%%%%%%%%%%%%%%%%%%%%%%%%%%%%%%%%%%%%%%%%%%%%%%%%%%%%%%%%%%%%%%
% \section{Implementation}
%\iffalse
%<*package>
%\fi
%
% This section describes the definitions file |childdoc.def|.

% The definitions cannot be loaded using |\usepackage| or |\RequirePackage|
% which has a mechanism to prevent loading a style file more than once.
% When loading the definitions by means of |\input|
% multiple instances have to be prevented manually:
%\iffalse
%This code needs to be before the `\ProvidesFile' directive
%which is defined at the beginning of this file.
%Therefore it is also placed there and commented out here.
%</package>
%<*discard>
%\fi
%    \begin{macrocode}
\ifdefined\childdocmain\endinput\fi
%    \end{macrocode}
%\iffalse
%</discard>
%<*package>
%\fi
%
% \macro{\ifchilddoc}
% \macro{\ifchilddocmanual}
% The conditional |\ifchilddoc| tells whether a
% child (true) or main (false) document is being compiled.
% The conditional |\ifchilddocmanual| tells whether
% the |\includeonly| mechanism is used (false) or
% the selection of child files must be performed manually (true).
% The definitions initialise to false:
%    \begin{macrocode}
\newif\ifchilddoc
\newif\ifchilddocmanual
%    \end{macrocode}

% \macro{\childdocname}
% \macro{\childdocjob}
% The macro |\childdocname| stores the name of the main document
% to be compiled. The macro |\childdocjob| stores the name of
% the document on which the \LaTeX{} compiler was originally invoked.
% The content of |\jobname| cannot be compared
% to filenames specified in the source due to different catcodes.
% The following code rescans |\jobname|, stores the result
% in |\childdocname| and saves a copy in |\childdocjob|:
%    \begin{macrocode}
\edef\childdocname{\scantokens\expandafter{\jobname\noexpand}}
\let\childdocjob\childdocname
%    \end{macrocode}

% \macro{\childdocdisable}
% The macro |\childdocdisable| prevents the main file
% from being processed more than once.
% At this stage, the main document command |\childdocmain|
% is assumed to be called once again where it should do nothing.
% Any subsequent call to it should prevent
% a secondary processing of the main document
% It overwrites the forwarding commands
% |\childdocof| and |\childdocforward|
% with empty macros to prevent further inclusions of the main document:
%    \begin{macrocode}
\newcommand{\childdocdisable}
{
  \renewcommand{\childdocmain}[1]{\renewcommand{\childdocmain}[1]{\endinput}}
  \renewcommand{\childdocof}[1]{}
  \renewcommand{\childdocby}[2][]{}
  \renewcommand{\childdocforward}[2][]{}
  \renewcommand{\childdocdisable}{}
}
%    \end{macrocode}

% \macro{\childdocmain}
% The macro |\childdocmain| is to be called at the top of the main file
% with nothing or the main filename (without extension) as argument.
% First, it breaks loops.
% If the argument is not empty and does not match |\childdocname|
% (which is set by the first inclusion of |childdoc.def|),
% |\ifchilddoc| is set to true, |\includeonly| is applied to the child file
% and |\jobname| is set to the main file
% (for proper handling of |.aux| files):
%    \begin{macrocode}
\newcommand{\childdocmain}[1]
{
  \childdocdisable\childdocmain{}
  \if?#1?\else
    \begingroup
      \def\childdoctmp{#1}
      \ifx\childdoctmp\childdocname
        \def\childdoctmp{}
      \else
        \def\childdoctmp
        {
          \childdoctrue
          \includeonly{\childdocname}
          \def\childdocjob{#1}
          \def\jobname{#1}
        }
      \fi
      \expandafter
    \endgroup
    \childdoctmp
  \fi
}
%    \end{macrocode}

% \macro{\childdocof}
% The command |\childdocof| redirects
% compilation to the main file |#1|.
%    \begin{macrocode}
\newcommand{\childdocof}[1]
{
  \childdocdisable
  \childdoctrue
  \includeonly{\childdocname}
  \def\jobname{#1}
  \def\childdocjob{#1}
  \input{#1}
}
%    \end{macrocode}

% \macro{\childdocby}
% The command |\childdocby| ....
%    \begin{macrocode}
\newcommand{\childdocby}[2][]
{
  \childdocdisable
  \childdoctrue
  \childdocmanualtrue
  \if?#1?\else
    \def\jobname{#2}
  \fi
  \def\childdocjob{#2}
  \input{#2}
  \endinput
}
%    \end{macrocode}

% \macro{\childdocforward}
% The command |\childdocforward| redirects
% compilation to the main file or
% (if the optional argument is given) a child file.
% Parameters are set as if the main file
% or a child file starting with |\childdocof| was compiled.
% Then compilation is handed over to the main file:
%    \begin{macrocode}
\newcommand{\childdocforward}[2][]
{
  \begingroup
    \if?#1?
      \def\childdoctmp
      {
        \def\childdocname{#2}
        \def\childdocjob{#2}
        \def\jobname{#2}
        \input{#2}
        \endinput
      }
    \else
      \def\childdoctmp
      {
        \childdocdisable
        \def\childdocname{#2}
        \childdoctrue
        \includeonly{#2}
        \def\childdocjob{#1}
        \def\jobname{#1}
        \input{#1}
        \endinput
      }
    \fi
    \expandafter
  \endgroup
  \childdoctmp
}
%    \end{macrocode}

% \macro{\childdocforwardprefix}
% The command |\childdocforwardprefix| redirects
% compilation to the main or a child file by means of a pattern.
% The prefix |#1| in the current filename is replaced by |#2|
% and the suffix of the current filename is kept
% (it is assumed that the filename does not contain the substring `|~~~|'
% which is used as a delimiter).
% Compilation is handed over to the new file by |\childdocforward|:
%    \begin{macrocode}
\newcommand{\childdocforwardprefix}[3][]
{
  \begingroup
    \def\childdocextract #2##1~~~{\def\childdoctmp{\childdocforward[#1]{#3##1}}}
    \expandafter\childdocextract\childdocname~~~
    \expandafter
  \endgroup
  \childdoctmp
}
%    \end{macrocode}

% \macro{\childdoc}
% The deprecated macro |\childdoc| is a legacy version of |\childdocmain|:
%    \begin{macrocode}
\newcommand{\childdoc}{\childdocmain}
%    \end{macrocode}

% \macro{\childdocredirect}
% The deprecated macro |\childdocredirect| is a legacy version
% of |\childdocforward| and |\childdocforwardprefix|:
%    \begin{macrocode}
\newcommand{\childdocredirect}[2][]
{
  \begingroup
    \if?#1?
      \def\childdoctmp{\childdocforward{#2}}
    \else
      \def\childdoctmp{\childdocforwardprefix{#1}{#2}}
    \fi
    \expandafter
  \endgroup
  \childdoctmp
}
%    \end{macrocode}

%\iffalse
%</package>
%\fi
%
\endinput
|\\
|\childdocmain{|\textit{main}|}|\\
\end{tabular}
\end{center}
%
If |\jobname| does not match the argument \textit{main} of |\childdocmain|,
it is assumed that |\jobname| points to the child file to be compiled.
When using |\childdocmain| with the main file specified as argument,
it suffices to start a child file
with just |\input{|\textit{main}|}|
without loading of the package and using |\childdocof|.
If instead all processing is done
with the appropriate \textsf{childdoc} directives,
the argument of \textit{main} of |\childdocmain| can be empty.

An alternative version of the command line processing described
in \secref{sec:commandline} using the detection mechanism reads:
%
\begin{center}
|... -jobname "|\textit{target}|" "|[\textit{flags}]%
[|\def\jobname{|\textit{dest}|}|]|\input{|\textit{main}|}"|
\end{center}

%%%%%%%%%%%%%%%%%%%%%%%%%%%%%%%%%%%%%%%%%%%%%%%%%%%%%%%%%%%%%%%%%%%%%%%%%%%%%%%%
\subsection{Manual Code}
\label{sec:manual}

In case one cannot be certain whether the definitions file |childdoc.def|
is installed on the target \TeX{} distribution
and one prefers not to ship it,
it is conceivable to paste a few relevant commands into the sources.

To that end, drop all statements |% \iffalse
%
% childdoc.dtx Copyright (C) 2017-2018 Niklas Beisert
%
% This work may be distributed and/or modified under the
% conditions of the LaTeX Project Public License, either version 1.3
% of this license or (at your option) any later version.
% The latest version of this license is in
%   http://www.latex-project.org/lppl.txt
% and version 1.3 or later is part of all distributions of LaTeX
% version 2005/12/01 or later.
%
% This work has the LPPL maintenance status `maintained'.
%
% The Current Maintainer of this work is Niklas Beisert.
%
% This work consists of the files childdoc.dtx and childdoc.ins
% and the derived files childdoc.def and cdocsamp.tex with
% cdocsch1.tex, cdocsch2.tex, cdocsdrf.tex, cdocsfn1.tex, cdocsfn2.tex.
%
%<package>\ifdefined\childdocmain\endinput\fi
%<package>\ProvidesFile{childdoc.def}[2018/12/30 v2.0 child document driver]
%<samplemain>\ProvidesFile{cdocsamp.tex}[2018/12/30 v2.0 sample for childdoc]
%<*driver>
%\ProvidesFile{childdoc.drv}[2018/12/30 v2.0 childdoc reference manual file]
\PassOptionsToClass{10pt,a4paper}{article}
\documentclass{ltxdoc}

\usepackage[margin=35mm]{geometry}
\usepackage{hyperref}
\usepackage{hyperxmp}
\usepackage[usenames]{color}

\hypersetup{colorlinks=true}
\hypersetup{pdfstartview=FitH}
\hypersetup{pdfpagemode=UseNone}
\hypersetup{pdfsource={}}
\hypersetup{pdflang={en-UK}}
\hypersetup{pdfcopyright={Copyright 2017-2018 Niklas Beisert.
  This work may be distributed and/or modified under the
  conditions of the LaTeX Project Public License, either version 1.3
  of this license or (at your option) any later version.}}
\hypersetup{pdflicenseurl={http://www.latex-project.org/lppl.txt}}
\hypersetup{pdfcontactaddress={ETH Zurich, ITP, HIT K,
  Wolfgang-Pauli-Strasse 27}}
\hypersetup{pdfcontactpostcode={8093}}
\hypersetup{pdfcontactcity={Zurich}}
\hypersetup{pdfcontactcountry={Switzerland}}
\hypersetup{pdfcontactemail={nbeisert@itp.phys.ethz.ch}}
\hypersetup{pdfcontacturl={http://people.phys.ethz.ch/\xmptilde nbeisert/}}

\newcommand{\secref}[1]{\hyperref[#1]{section \ref*{#1}}}

\parskip1ex
\parindent0pt
\let\olditemize\itemize
\def\itemize{\olditemize\parskip0pt}

\begin{document}

\title{The \textsf{childdoc} Package}
\hypersetup{pdftitle={The childdoc Package}}
\author{Niklas Beisert\\[2ex]
  Institut f\"ur Theoretische Physik\\
  Eidgen\"ossische Technische Hochschule Z\"urich\\
  Wolfgang-Pauli-Strasse 27, 8093 Z\"urich, Switzerland\\[1ex]
  \href{mailto:nbeisert@itp.phys.ethz.ch}
  {\texttt{nbeisert@itp.phys.ethz.ch}}}
\hypersetup{pdfauthor={Niklas Beisert}}
\hypersetup{pdfsubject={Manual for the LaTeX2e Package childdoc}}
\date{30 December 2018, \textsf{v2.0}}
\maketitle

\begin{abstract}\noindent
\textsf{childdoc} is a \LaTeXe{} package
that enables the direct compilation
of document sections included by |\include|
to individual files.
\end{abstract}

\begingroup
\parskip0ex
\tableofcontents
\endgroup

%%%%%%%%%%%%%%%%%%%%%%%%%%%%%%%%%%%%%%%%%%%%%%%%%%%%%%%%%%%%%%%%%%%%%%%%%%%%%%%%
%%%%%%%%%%%%%%%%%%%%%%%%%%%%%%%%%%%%%%%%%%%%%%%%%%%%%%%%%%%%%%%%%%%%%%%%%%%%%%%%
\section{Introduction}

\LaTeX{} provides a mechanism to structure a large document (such as a book)
into a main file and several child files (containing the chapters)
using the |\include| command.
This mechanism is beneficial for documents
which span hundreds of pages in order to
make the source file(s) more manageable.
Moreover, compilation can be restricted to
selected child files by means of the |\includeonly| command.
The latter feature can be used to reduce the compilation time while editing
(this was significantly more useful in the earlier days of \LaTeX{})
or to generate a smaller document which is easier to navigate.
Another application of |\includeonly| is to generate
documents consisting of selected parts of the complete document.

However, there are a few drawbacks of the plain |\include| mechanism:
\begin{itemize}
\item
The child files cannot be compiled on their own,
they can only be compiled via the main file.
A naive editing environment
(such as a text editor with an option
to have the current file processed by \LaTeX)
may require one to switch to the main file before compiling;
attempting to compile the child file produces errors.
\item
The main file must be modified (each time)
to adjust the |\includeonly| command
to the present needs. This easily leaves the main file in a messy state.
\item
The generated document will always carry the filename
of the main document. This is inconvenient if
several child files are to be compiled and
to be kept for distribution.
\end{itemize}

The present package provides a simple interface
to make child files individually compilable by \LaTeX{}.
Compiling a child file then has the same effect as compiling
the main file with an |\includeonly| command
to select the appropriate child.
Moreover the generated document will carry the name of the child
rather than the main file.
This resolves all three above issues.

This feature is meant to make the editing of books,
thesis documents and lecture notes somewhat more convenient.
However, the package can also be used efficiently for
composing a series of documents (such as exercise sheets)
which are typically distributed individually.
It then assists the author in generating the individual documents
(potentially in different versions)
as well as a document containing the collected series.
Another application is in developing style files
or other kinds of included material
where compilation of the style file could redirect
to a sample or test file.

%%%%%%%%%%%%%%%%%%%%%%%%%%%%%%%%%%%%%%%%%%%%%%%%%%%%%%%%%%%%%%%%%%%%%%%%%%%%%%%%
%%%%%%%%%%%%%%%%%%%%%%%%%%%%%%%%%%%%%%%%%%%%%%%%%%%%%%%%%%%%%%%%%%%%%%%%%%%%%%%%
\section{Usage}

First of all, the package \textsf{childdoc} is \emph{not} a standard
\LaTeXe{} |.sty| style file! Therefore it needs to be invoked in
a non-standard way.

%%%%%%%%%%%%%%%%%%%%%%%%%%%%%%%%%%%%%%%%%%%%%%%%%%%%%%%%%%%%%%%%%%%%%%%%%%%%%%%%
\subsection{Included Files}
\label{sec:include}

%%%%%%%%%%%%%%%%%%%%%%%%%%%%%%%%%%%%%%%%
\DescribeMacro{\childdocmain}
To use the package, add the commands
\begin{center}
\begin{tabular}{l}
|\input{childdoc.def}|\\
|\childdocmain{}|\\
\end{tabular}
\end{center}
at the very top of the main \LaTeX{} file,
in particular \emph{before} the |\documentclass| statement!
The argument of |\childdocmain| should be left empty
(but it must be present).

%%%%%%%%%%%%%%%%%%%%%%%%%%%%%%%%%%%%%%%%
\DescribeMacro{\childdocof}
Furthermore, add the commands
\begin{center}
\begin{tabular}{l}
|\input{childdoc.def}|\\
|\childdocof{|\textit{main}|}|\\
\end{tabular}
\end{center}
at the top of every child file \textit{child}
which is included by |\include{|\textit{child}|}|
from within the main file
(or at least for those files to be compiled individually).
The argument \textit{main} must be the filename of the main file.

There are a couple of
considerations in setting up the main and child documents:

%%%%%%%%%%%%%%%%%%%%%%%%%%%%%%%%%%%%%%%%
\paragraph{Restrictions.}

Please note the following restrictions:
\begin{itemize}
\item
|\childdocmain| must be called with one argument \textit{main}
to ensure compatibility with earlier version of the package.
It must either be empty (|\childdocmain{}|)
or precisely match the filename of the main file in which it is specified.
See \secref{sec:detection} for further information.
\item
The filename \textit{main} must be specified without the |.tex| extension.
\item
The filename \textit{main} is case sensitive
(even in case-insensitive file systems)
due to internal string comparison.
\item
The argument \textit{main} should be fully expanded, it cannot be a macro.
\item
Subdirectories and special characters should be avoided in filenames.
\item
The command |\childdocmain{|\textit{main}|}| must be followed by a whitespace.
It should not be followed immediately by another command
or by a comment mark `|%|'.
This is because the \TeX{} parser reads the token immediately following
the argument of |\childdocmain| and puts it
at the beginning of every child section;
however, a white\-space is ignored.
\end{itemize}

%%%%%%%%%%%%%%%%%%%%%%%%%%%%%%%%%%%%%%%%
\paragraph{Content of Main File.}

It is advisable to place all content in the child files included by |\include|.
Any output contained in the main file will appear in all child documents
unless suppressed manually;
it cannot be suppressed automatically by the |\includeonly| directive
and thus should normally be avoided.
A method to include some content in the main file
by means of conditional processing is described in \secref{sec:conditional}.

%%%%%%%%%%%%%%%%%%%%%%%%%%%%%%%%%%%%%%%%
\paragraph{Page Numbering.}

When only a part of the document is compiled,
the appropriate numbering of pages
(as well as other status parameters)
is determined from the |.aux| files.
The latter contain information from previous passes.
However this information needs to propagate through
all intermediate child documents.
Therefore the page numbering in child documents may well
be inconsistent until the complete document is compiled at least once.

A useful (if unconventional) way to always ensure a consistent
page numbering is to restart the numbering in each child document
and denote the pages by `\textit{child}|.|\textit{page}'
where \textit{child} represents the chapter/section number of the child file.
This can be achieved by the command
|\numberwithin{page}{|\textit{child}|}|
of the \textsf{amsmath} package
where \textit{child} can be |chapter| or |section|
depending on the chosen structuring.
Alternatively, one can modify the macro |\thepage| appropriately
and reset the counter |page| at the start of each child file.

%%%%%%%%%%%%%%%%%%%%%%%%%%%%%%%%%%%%%%%%%%%%%%%%%%%%%%%%%%%%%%%%%%%%%%%%%%%%%%%%
\subsection{Conditional Processing}
\label{sec:conditional}

The package provides a mechanism to compile different versions
of a document. To customise the versions further some conditional processing
can come in handy to distinguish which version is being compiled.
The package provides two macros to describe the compilation context:

%%%%%%%%%%%%%%%%%%%%%%%%%%%%%%%%%%%%%%%%
\DescribeMacro{\ifchilddoc}
The conditional |\ifchilddoc| distinguishes between the compilation of
child documents and the main document:
%
\begin{center}
|\ifchilddoc |\textit{child-code}| |[|\||else |\textit{main-code}]| \||fi|
\end{center}

%%%%%%%%%%%%%%%%%%%%%%%%%%%%%%%%%%%%%%%%
\DescribeMacro{\childdocname}
\DescribeMacro{\childdocjob}
The macro |\childdocname| contains the filename (without extension)
of the main or child file being processed.
Note that |\childdocjob| will always contain the name of the main file.

%%%%%%%%%%%%%%%%%%%%%%%%%%%%%%%%%%%%%%%%
\paragraph{Title Page.}

Conditional processing can be used to include a title or banner page
in the main document when proper precautions are taken.
Importantly, the code in the main file should ensure that the page counter
(as well as other status parameters which are stored in the |.aux| files)
takes the same value after the conditional processing.
Otherwise the page numbers may take divergent values
depending on which part is compiled.

For example, a title page could be declared by:
%
\begin{center}
\begin{tabular}{l}
|\ifchilddoc\||else|\\
|\addtocounter{page}{-1}|\\
\textit{code for title page}\\
|\newpage|\\
|\||fi|
\end{tabular}
\end{center}
%
A banner page for the child documents can be generated by:
%
\begin{center}
\begin{tabular}{l}
|\ifchilddoc|\\
|\addtocounter{page}{-1}|\\
\textit{code for banner page}\\
|\newpage|\\
|\||fi|
\end{tabular}
\end{center}
%
Here one could write a message such as:
\begin{center}
|This is the part \childdocname{} of \childdocjob{}.|
\end{center}

%%%%%%%%%%%%%%%%%%%%%%%%%%%%%%%%%%%%%%%%%%%%%%%%%%%%%%%%%%%%%%%%%%%%%%%%%%%%%%%%
\subsection{Flags}
\label{sec:flags}

The package makes it easy to generate different versions
of the main or child documents.
To this end compilation flags can be defined
and assigned different default values.
They will be particularly useful in conjunction
with the forwarding mechanism described in \secref{sec:forward}.

For example, it may be useful to have a flag |\version|
which can be set to |draft| or |final|.
The document source will contain some conditional code
depending on the value of |\version|.
Suppose further, the flag should default to |final| for the main file
and to |draft| for child files
which is a natural assignment for editing the document.
This is achieved by placing the following code
in the preamble of the main document
(below the |\childdocmain| directive):
%
\begin{center}
\begin{tabular}{l}
|\ifchilddoc|\\
|\providecommand{\version}{draft}|\\
|\||else|\\
|\providecommand{\version}{final}|\\
|\||fi|
\end{tabular}
\end{center}
%
The definition by |\providecommand| makes sure
that previous definitions are not overwritten.
Further statements |\providecommand{\version}{...}|
can thus be added before the above code to override it.

For the main file, one might add a line
(between |\childdocmain| and the above block)
%
\begin{center}
|%\ifchilddoc\||else\providecommand{\version}{draft}\||fi|
\end{center}
%
which can be uncommented to produce a draft version.
Likewise one can add a line to the very top of a child file
(above the |\childdocof{|\textit{main}|}| directive)
%
\begin{center}
|%\providecommand{\version}{final}|
\end{center}
%
which can be uncommented to produce the final version of this child document.

%%%%%%%%%%%%%%%%%%%%%%%%%%%%%%%%%%%%%%%%%%%%%%%%%%%%%%%%%%%%%%%%%%%%%%%%%%%%%%%%
\subsection{Forwarding}
\label{sec:forward}

Different versions of the main or child documents
using compilation flags as described in \secref{sec:flags}
can be (permanently) stored in different files
for convenient compilation, viewing and distribution.
To this end, the package defines a command
to pass on compilation to a different file:

%%%%%%%%%%%%%%%%%%%%%%%%%%%%%%%%%%%%%%%%
\DescribeMacro{\childdocforward}
The command |\childdocforward| redirects processing to
another source file:
%
\begin{center}
\begin{tabular}{l}
|\input{childdoc.def}|\\
|\childdocforward[|\textit{main}|]{|\textit{dest}|}|\\
\end{tabular}
\end{center}
%
The argument \textit{dest} is the destination file
(without extension).
It should be the main file or one of the child files.
Note that further \textsf{childdoc} directives
such as |\childdocof| and |\childdocforward|
in the indicated file will be processed in this form.
The optional argument \textit{main}
passes on directly to the main file \textit{main}
while pretending to compile the child \textit{dest}.
This form behaves as if \textit{dest}
issues |\childdocof{|\textit{main}|}| right away,
and no further \textsf{childdoc} directives will be processed.

%%%%%%%%%%%%%%%%%%%%%%%%%%%%%%%%%%%%%%%%
\DescribeMacro{\...prefix}
In the alternative form |\childdocforwardprefix|,
%
\begin{center}
\begin{tabular}{l}
|\input{childdoc.def}|\\
|\childdocforwardprefix[|\textit{main}|]{|\textit{prefix}|}{|\textit{dest}|}|
\end{tabular}
\end{center}
%
the destination file is determined by a pattern
depending on the current file:
To make this work, the current file must be called
`{\textit{prefix}\hspace{0.2em}\textit{suffix}}'
with \textit{prefix} matching precisely the argument.
Processing is then passed on to the file
`{\textit{dest}\hspace{0.2em}\textit{suffix}}'.
Surely, the same effect is achieved by
directly specifying the
argument `{\textit{dest}\hspace{0.2em}\textit{suffix}}'
in the first form.
However, that requires to set up a different file
for each child. With the alternative form of the command
all these files can have exactly the same content
which simplifies setting them up and maintaining them.

For example, the following file |draft.tex|
with a compilation flag |\version| as described in \secref{sec:flags}
compiles the main document as a draft:
%
\begin{center}
\begin{tabular}{l}
|\def\version{draft}|\\
|\input{childdoc.def}|\\
|\childdocforward{|\textit{main}|}|
\end{tabular}
\end{center}
%
Likewise, the following files |final|\textit{nn}|.tex|
compile the final version of the child document
|child|\textit{nn}|.tex|:
%
\begin{center}
\begin{tabular}{l}
|\def\version{final}|\\
|\input{childdoc.def}|\\
|\childdocforwardprefix{final}{child}|
\end{tabular}
\end{center}
%

Note that when several versions of a main file and/or of each child file
are to be generated, it may be convenient to set up a |Makefile| or
shell script to automatise the process.

%%%%%%%%%%%%%%%%%%%%%%%%%%%%%%%%%%%%%%%%%%%%%%%%%%%%%%%%%%%%%%%%%%%%%%%%%%%%%%%%
\subsection{Command Line Processing}
\label{sec:commandline}

The effect of redirection files can also be achieved by invoking
the \LaTeX{} compiler with a more elaborate command line.
Most conveniently this should be done as part
of a shell script or a |Makefile|.

When using \textsf{childdoc} in the main file, the following
command lines effectively perform a redirection
(note that depending on the shell being used,
backslashes may have to be doubled: `|\|' $\to$ `|\\|'):
%
\begin{center}
|... -jobname "|\textit{target}|" |\\|"|[\textit{flags}]%
|\input{childdoc.def}\childdocforward[|\textit{main}|]{|\textit{dest}|}"|
\end{center}
%
Here \textit{target} is the name of the output file,
\textit{main} is the name of the main file
and \textit{dest} is the name of the main or child file to be processed
(all filenames without extensions).
The optional argument \textit{main} can be omitted
if \textit{main} matches \textit{dest}.
Optionally, compilation \textit{flags} can be defined via |\def| commands.
This command line makes the \TeX{} engine believe
it is compiling the file \textit{target}
whose content is specified as the latter parameter.
The provided code then forwards the processing to
\textit{main} or \textit{dest} as described in \secref{sec:forward}.

%%%%%%%%%%%%%%%%%%%%%%%%%%%%%%%%%%%%%%%%%%%%%%%%%%%%%%%%%%%%%%%%%%%%%%%%%%%%%%%%
\subsection{Include by Input}
\label{sec:input}

Including child documents by |\include| has some restrictions by design.
Most notably, the content of a child document always occupies
its own set of pages; pages cannot be shared between child documents.
Usually, this behaviour makes perfect sense
because each child document contain an essential part of the document.
However, in some situations it may be desirable to compose
a document from a collection of parts
without having mandatory page breaks between then.
For this case, the package
provides a mechanism to include parts
by |\input| which can also be processed individually.
However, by construction this mechanism
requires manual handling of the content to be output.

%%%%%%%%%%%%%%%%%%%%%%%%%%%%%%%%%%%%%%%%
\DescribeMacro{\ifchilddocmanual}
The main file should be prepared as usual, see \secref{sec:include}.
However, the document body must make a distinction
between processing of an individual part and of the main document, e.g.:
%
\begin{center}
\begin{tabular}{l}
|\ifchilddocmanual|\\
|\input{\childdocname}|\\
|\||else|\\
\textit{document body with }|\input{|\textit{part}|}|\\
|\||fi|
\end{tabular}
\end{center}
%
The conditional |\ifchilddocmanual| is true whenever
a part to be included by |\input| is being compiled,
and the name of the part is stored in |\childdocname|.

%%%%%%%%%%%%%%%%%%%%%%%%%%%%%%%%%%%%%%%%
\DescribeMacro{\childdocby}
Each part to be included by |\input| should start with:
%
\begin{center}
\begin{tabular}{l}
|\input{childdoc.def}|\\
|\childdocby{|\textit{main}|}|\\
\end{tabular}
\end{center}
%
The directive |\childdocby| is similar to |\childdocof|
described in \secref{sec:include},
but the subsequent selection of content must be done manually.
To that end, both |\ifchilddoc| and |\ifchilddocmanual|
will be true upon processing of a part,
and the name of the part is stored in |\childdocname|.
Note that |\jobname| will be set to the filename of the current part
so that each part receives an individual |.aux| file
that does not interfere with the |.aux| file(s) of the main document.
This behaviour can be altered by the alternative form
|\childdocby[*]{|\textit{main}|}| (with a non-empty optional argument)
which uses the |.aux| file of the main document
by setting |\jobname| to \textit{main}.

%%%%%%%%%%%%%%%%%%%%%%%%%%%%%%%%%%%%%%%%%%%%%%%%%%%%%%%%%%%%%%%%%%%%%%%%%%%%%%%%
\subsection{Driver Development}
\label{sec:driver}

The \textsf{childdoc} mechanism can also be use for the development
of definition files such as \LaTeX{} styles or classes.
This case differs from the above setup with multiple parts
included by |\include| in that no |\includeonly| should be invoked.
This can be achieved by starting the include file
(before |\ProvidesPackage|) with:
%
\begin{center}
\begin{tabular}{l}
|\input{childdoc.def}|\\
|\childdocforward{|\textit{main}|}|\\
\end{tabular}
\end{center}
%
or alternatively with:
%
\begin{center}
\begin{tabular}{l}
|\input{childdoc.def}|\\
|\childdocby{|\textit{main}|}|\\
\end{tabular}
\end{center}
%
Both forms have slightly different effects as described above.
The main file is prepared as usual, see \secref{sec:include}.

%%%%%%%%%%%%%%%%%%%%%%%%%%%%%%%%%%%%%%%%%%%%%%%%%%%%%%%%%%%%%%%%%%%%%%%%%%%%%%%%
\subsection{Legacy Detection}
\label{sec:detection}

The directive |\childdocmain| in the main file can detect
whether the complete document or merely a child is to be compiled
even without using the directive |\childdocof|.
This method is deprecated because it is less robust
and there is no compelling reason to use it;
it is merely provided for backward compatibility
and it may be removed in future versions.

If the detection mechanism is to be used,
it is mandatory to correctly specify
the filename of the main file as the argument of |\childdocmain|:
%
\begin{center}
\begin{tabular}{l}
|\input{childdoc.def}|\\
|\childdocmain{|\textit{main}|}|\\
\end{tabular}
\end{center}
%
If |\jobname| does not match the argument \textit{main} of |\childdocmain|,
it is assumed that |\jobname| points to the child file to be compiled.
When using |\childdocmain| with the main file specified as argument,
it suffices to start a child file
with just |\input{|\textit{main}|}|
without loading of the package and using |\childdocof|.
If instead all processing is done
with the appropriate \textsf{childdoc} directives,
the argument of \textit{main} of |\childdocmain| can be empty.

An alternative version of the command line processing described
in \secref{sec:commandline} using the detection mechanism reads:
%
\begin{center}
|... -jobname "|\textit{target}|" "|[\textit{flags}]%
[|\def\jobname{|\textit{dest}|}|]|\input{|\textit{main}|}"|
\end{center}

%%%%%%%%%%%%%%%%%%%%%%%%%%%%%%%%%%%%%%%%%%%%%%%%%%%%%%%%%%%%%%%%%%%%%%%%%%%%%%%%
\subsection{Manual Code}
\label{sec:manual}

In case one cannot be certain whether the definitions file |childdoc.def|
is installed on the target \TeX{} distribution
and one prefers not to ship it,
it is conceivable to paste a few relevant commands into the sources.

To that end, drop all statements |\input{childdoc.def}|
and perform the replacements as outlined below.
Instead of |\childdocmain{|\textit{main}|}| add the following code
to the top of the main file:
%
\begin{center}
\begin{tabular}{l}
|\||ifdefined\childdocname\endinput\||fi\newif\ifchilddoc|\\
|\edef\childdocname{\scantokens\expandafter{\jobname\noexpand}}|\\
|\def\childdocmain{|\textit{main}|}\||ifx\childdocmain\childdocname\||else|\\
|\childdoctrue\includeonly{\childdocname}\let\jobname\childdocmain\||fi|\\
\end{tabular}
\end{center}
%
Instead of |\childdocof{|\textit{main}|}| just include the main file
at the top of each child file:
%
\begin{center}
|\input{|\textit{main}|}|
\end{center}
%
A simple redirection |\childdocforward{|\textit{dest}|}| is achieved by:
%
\begin{center}
|\def\jobname{|\textit{dest}|}\input{\jobname}|
\end{center}
%
The redirection with prefix
|\childdocforwardprefix[|\textit{prefix}|]{|\textit{dest}|}|
is accomplished by:
%
\begin{center}
\begin{tabular}{l}
|{\edef\jobname{\scantokens\expandafter{\jobname\noexpand}}|\\
|\def\redirectjob |\textit{prefix}|#1~~~{\gdef\jobname{|\textit{dest}|#1}}|\\
|\expandafter\redirectjob\jobname~~~}\input{\jobname}|
\end{tabular}
\end{center}

In an alternative approach,
child documents can be compiled by a specific command line
without additional code or specific definitions:
%
\begin{center}
|... -jobname "|\textit{target}|" "|[\textit{flags}]%
|\includeonly{|\textit{dest}|}\input{|\textit{main}|}"|
\end{center}
%

%%%%%%%%%%%%%%%%%%%%%%%%%%%%%%%%%%%%%%%%%%%%%%%%%%%%%%%%%%%%%%%%%%%%%%%%%%%%%%%%
%%%%%%%%%%%%%%%%%%%%%%%%%%%%%%%%%%%%%%%%%%%%%%%%%%%%%%%%%%%%%%%%%%%%%%%%%%%%%%%%
\section{Information}

%%%%%%%%%%%%%%%%%%%%%%%%%%%%%%%%%%%%%%%%%%%%%%%%%%%%%%%%%%%%%%%%%%%%%%%%%%%%%%%%
\subsection{Copyright}

Copyright \copyright{} 2017--2018 Niklas Beisert

This work may be distributed and/or modified under the
conditions of the \LaTeX{} Project Public License, either version 1.3
of this license or (at your option) any later version.
The latest version of this license is in
  \url{http://www.latex-project.org/lppl.txt}
and version 1.3 or later is part of all distributions of \LaTeX{}
version 2005/12/01 or later.

This work has the LPPL maintenance status `maintained'.

The Current Maintainer of this work is Niklas Beisert.

This work consists of the files |README.txt|, |childdoc.ins| and |childdoc.dtx|
as well as the derived files |childdoc.def|, |cdocsamp.tex|
with |cdocsch1.tex|, |cdocsch2.tex|, |cdocspt3.tex|, |cdocspt4.tex|,
|cdocsdrf.tex|, |cdocsfn1.tex|, |cdocsfn2.tex|
as well as |childdoc.pdf|.

%%%%%%%%%%%%%%%%%%%%%%%%%%%%%%%%%%%%%%%%%%%%%%%%%%%%%%%%%%%%%%%%%%%%%%%%%%%%%%%%
\subsection{Files and Installation}

The package consists of the files:
%
\begin{center}
\begin{tabular}{ll}
    |README.txt|   & readme file \\
    |childdoc.ins| & installation file \\
    |childdoc.dtx| & source file \\
    |childdoc.def| & definition file \\
    |cdocsamp.tex| & sample main file \\
    |cdocsch1.tex| & sample include file \\
    |cdocsch2.tex| & sample include file \\
    |cdocspt3.tex| & sample part file \\
    |cdocspt4.tex| & sample part file \\
    |cdocsdrf.tex| & sample redirection file \\
    |cdocsfn1.tex| & sample redirection file \\
    |cdocsfn2.tex| & sample redirection file \\
    |childdoc.pdf| & manual
\end{tabular}
\end{center}
%
The distribution consists of the files
|README.txt|, |childdoc.ins| and |childdoc.dtx|.
%
\begin{itemize}
\item
Run (pdf)\LaTeX{} on |childdoc.dtx|
to compile the manual |childdoc.pdf| (this file).
\item
Run \LaTeX{} on |childdoc.ins| to create the definitions file |childdoc.def|
and the sample |cdocsamp.tex| with include files
|cdocsch1.tex|, |cdocsch2.tex|, |cdocspt3.tex|, |cdocspt4.tex|,
|cdocsdrf.tex|, |cdocsfn1.tex|, |cdocsfn2.tex|.
Then copy the file |childdoc.def| to an appropriate directory of your \LaTeX{}
distribution, e.g.\ \textit{texmf-root}|/tex/latex/childdoc|.
\end{itemize}

%%%%%%%%%%%%%%%%%%%%%%%%%%%%%%%%%%%%%%%%%%%%%%%%%%%%%%%%%%%%%%%%%%%%%%%%%%%%%%%%
\subsection{Related CTAN Packages}

There are several other packages which offer a similar functionality:
%
\begin{itemize}
\item
The packages
\href{http://ctan.org/pkg/docmute}{\textsf{docmute}},
\href{http://ctan.org/pkg/includex}{\textsf{includex}} and
\href{http://ctan.org/pkg/standalone}{\textsf{standalone}}
provide commands to include only the document body of
a child file thus allowing both files to be compiled individually.
\item
The packages \href{http://ctan.org/pkg/subdocs}{\textsf{subdocs}}
and \href{http://ctan.org/pkg/subfiles}{\textsf{subfiles}}
provide structures in which the main and child documents can be
encapsulated and allowing them to be compiled individually.
The inclusion mechanism is different from the conventional |\include|.
\item
The package \href{http://ctan.org/pkg/combine}{\textsf{combine}}
is an elaborate solution to combine several documents into one.
\end{itemize}
%
See also the CTAN topic \href{http://ctan.org/topic/subdocs}{\textsf{subdocs}}
for further related packages.
The present package differs from the above solutions in that
a document structure constructed with the conventional |\include| mechanism
just needs two extra commands at the top of every file
such that all constituent files can be compiled individually.

%%%%%%%%%%%%%%%%%%%%%%%%%%%%%%%%%%%%%%%%%%%%%%%%%%%%%%%%%%%%%%%%%%%%%%%%%%%%%%%%
%\subsection{Feature Suggestions}
%
%The following is a list of features which may be useful for future
%versions of this package:
%%
%\begin{itemize}
%\item
%\ldots
%\end{itemize}

%%%%%%%%%%%%%%%%%%%%%%%%%%%%%%%%%%%%%%%%%%%%%%%%%%%%%%%%%%%%%%%%%%%%%%%%%%%%%%%%
\subsection{Revision History}

%%%%%%%%%%%%%%%%%%%%%%%%%%%%%%%%%%%%%%%%
\paragraph{v2.0:} 2018/12/30

\begin{itemize}
\item
immediate forward processing
\item
added |\childdocby| mechanism
\item
manual restructured
\end{itemize}

%%%%%%%%%%%%%%%%%%%%%%%%%%%%%%%%%%%%%%%%
\paragraph{v1.6:} 2018/01/17

\begin{itemize}
\item
application for development of include files
\item
corrections to manual
\end{itemize}

%%%%%%%%%%%%%%%%%%%%%%%%%%%%%%%%%%%%%%%%
\paragraph{v1.5:} 2017/05/21

\begin{itemize}
\item
more complete structuring introduced
\item
|\childdocof| introduced
\item
|\childdoc| renamed to |\childdocmain|
\item
|\childredirect| renamed to |\childdocforward| and |\childdocforwardprefix|
and functionality expanded
\end{itemize}

%%%%%%%%%%%%%%%%%%%%%%%%%%%%%%%%%%%%%%%%
\paragraph{v1.0:} 2017/04/27

\begin{itemize}
\item
manual and install package
\item
first version published on CTAN
\end{itemize}

%%%%%%%%%%%%%%%%%%%%%%%%%%%%%%%%%%%%%%%%
\paragraph{v0.6:} 2017/04/26

\begin{itemize}
\item
redirection mechanism added
\end{itemize}

%%%%%%%%%%%%%%%%%%%%%%%%%%%%%%%%%%%%%%%%
\paragraph{v0.5:} 2017/04/26

\begin{itemize}
\item
functionality in definition file
\end{itemize}


%%%%%%%%%%%%%%%%%%%%%%%%%%%%%%%%%%%%%%%%%%%%%%%%%%%%%%%%%%%%%%%%%%%%%%%%%%%%%%%%
%%%%%%%%%%%%%%%%%%%%%%%%%%%%%%%%%%%%%%%%%%%%%%%%%%%%%%%%%%%%%%%%%%%%%%%%%%%%%%%%
%%%%%%%%%%%%%%%%%%%%%%%%%%%%%%%%%%%%%%%%%%%%%%%%%%%%%%%%%%%%%%%%%%%%%%%%%%%%%%%%
\appendix

\settowidth\MacroIndent{\rmfamily\scriptsize 000\ }

 \DocInput{childdoc.dtx}

\end{document}
%</driver>
% \fi
%
% %%%%%%%%%%%%%%%%%%%%%%%%%%%%%%%%%%%%%%%%%%%%%%%%%%%%%%%%%%%%%%%%%%%%%%%%%%%%%%
% %%%%%%%%%%%%%%%%%%%%%%%%%%%%%%%%%%%%%%%%%%%%%%%%%%%%%%%%%%%%%%%%%%%%%%%%%%%%%%
% \section{Sample}
%\iffalse
%<*samplemain>
%\fi
%
% The following presents a sample document
% with two chapters, two parts, a title page,
% a compile flag as well as three forwarding files to set the flag.
% It consists of eight |.tex| files:
% \begin{center}
% \begin{tabular}{ll}
% |cdocsamp.tex|&main file\\
% |cdocsch1.tex|&include file for chapter 1\\
% |cdocsch2.tex|&include file for chapter 2\\
% |cdocspt3.tex|&include file for part 3\\
% |cdocspt4.tex|&include file for part 4\\
% |cdocsdrf.tex|&forwarding file for main file in draft mode\\
% |cdocsfi1.tex|&forwarding file for final version of chapter 1\\
% |cdocsfi2.tex|&forwarding file for final version of chapter 2\\
% \end{tabular}
% \end{center}
% Each of the eight files can be compiled directly by the \LaTeX{} compiler.
%
% %%%%%%%%%%%%%%%%%%%%%%%%%%%%%%%%%%%%%%
% \paragraph{Main File.}
%
% The main file is called |cdocsamp.tex|.
%
% Load the \textsf{childdoc} definitions and
% declare the filename for the main document:
%    \begin{macrocode}
\input{childdoc.def}
\childdocmain{}
%    \end{macrocode}

% Optional override for |\version| flag:
%    \begin{macrocode}
%%\ifchilddoc\else\providecommand{\version}{draft}\fi
%    \end{macrocode}

% Define the default values for the |\version| flag
% (|final| for the main file and |draft| for childs):
%    \begin{macrocode}
\ifchilddoc
\providecommand{\version}{draft}
\else
\providecommand{\version}{final}
\fi
%    \end{macrocode}

% Load the standard document class:
%    \begin{macrocode}
\documentclass[12pt]{article}
%    \end{macrocode}

% Start the document body:
%    \begin{macrocode}
\begin{document}
%    \end{macrocode}

% Declare a title page.
% Print title, part of document being processed and version flag:
%    \begin{macrocode}
\addtocounter{page}{-1}
\begin{center}
{\LARGE\bfseries{}childdoc example\par}
\vspace{1cm}
\ifchilddoc
\ifchilddocmanual part\else chapter\fi:
`\childdocname' of `\childdocjob'\par
\else
main document: `\childdocjob'\par
\fi
version: \version\par
\end{center}
\newpage
%    \end{macrocode}

% Manually include selected file,
% otherwise process as usual:
%    \begin{macrocode}
\ifchilddocmanual
\section*{part `\childdocname'}
\input{\childdocname}
\else
%    \end{macrocode}

% Include the two chapters:
%    \begin{macrocode}
\include{cdocsch1}
\include{cdocsch2}
%    \end{macrocode}

% Include the two parts unless only chapters should be displayed:
%    \begin{macrocode}
\ifchilddoc\else
\section{part three}
\input{cdocspt3}
\section{part four}
\input{cdocspt4}
\fi
%    \end{macrocode}

% Process as usual until here:
%    \begin{macrocode}
\fi
%    \end{macrocode}

% End of document body:
%    \begin{macrocode}
\end{document}
%    \end{macrocode}
%\iffalse
%</samplemain>
%\fi
%
% %%%%%%%%%%%%%%%%%%%%%%%%%%%%%%%%%%%%%%
% \paragraph{Chapter Include Files.}
%
% The include files are called |cdocsch1.tex| and |cdocsch2.tex|.
%
%\iffalse
%<*samplechap1|samplechap2>
%\fi

% Optional override for |\version| flag:
%    \begin{macrocode}
%%\providecommand{\version}{final}
%    \end{macrocode}

% Include the main document:
%    \begin{macrocode}
\input{childdoc.def}
\childdocof{cdocsamp}
%    \end{macrocode}

%\iffalse
%</samplechap1|samplechap2>
%\fi
%
%\iffalse
%<*samplechap1>
%\fi
% Some text for chapter 1:
%    \begin{macrocode}
\section{one}
some text in chapter one
%    \end{macrocode}

%\iffalse
%</samplechap1>
%\fi
% Some text for chapter 2:
%\iffalse
%<*samplechap2>
%\fi
%    \begin{macrocode}
\section{two}
more text in chapter two
%    \end{macrocode}

%\iffalse
%</samplechap2>
%\fi
%
% %%%%%%%%%%%%%%%%%%%%%%%%%%%%%%%%%%%%%%
% \paragraph{Part Include Files.}
%
% The include files are called |cdocspt3.tex| and |cdocspt4.tex|.
%
%\iffalse
%<*samplepart3|samplepart4>
%\fi

% Optional override for |\version| flag:
%    \begin{macrocode}
%%\providecommand{\version}{final}
%    \end{macrocode}

% Include the main document:
%    \begin{macrocode}
\input{childdoc.def}
\childdocby{cdocsamp}
%    \end{macrocode}

%\iffalse
%</samplepart3|samplepart4>
%\fi
%
%\iffalse
%<*samplepart3>
%\fi
% Some text for part 3:
%    \begin{macrocode}
some text in part three
%    \end{macrocode}

%\iffalse
%</samplepart3>
%\fi
% Some text for part 4:
%\iffalse
%<*samplepart4>
%\fi
%    \begin{macrocode}
more text in part four
%    \end{macrocode}

%\iffalse
%</samplepart4>
%\fi
%
% %%%%%%%%%%%%%%%%%%%%%%%%%%%%%%%%%%%%%%
% \paragraph{Forwarding for a Complete Draft.}
%
% The following forwarding file |cdocsdrf.tex|
% compiles the main document in draft mode:
%\iffalse
%<*sampledraft>
%\fi
%    \begin{macrocode}
\def\version{draft}
\input{childdoc.def}
\childdocforward{cdocsamp}
%    \end{macrocode}

%\iffalse
%</sampledraft>
%\fi
%
% %%%%%%%%%%%%%%%%%%%%%%%%%%%%%%%%%%%%%%
% \paragraph{Forwarding for Final Version of the Chapters.}
%
% The following forwarding files |cdocsfn1.tex| and |cdocsfn2.tex|
% (with identical content)
% compile the final versions of the child documents
% |cdocsch1.tex| and |cdocsch2.tex|, respectively:
%\iffalse
%<*samplefinal>
%\fi
%    \begin{macrocode}
\def\version{final}
\input{childdoc.def}
\childdocforwardprefix[cdocsamp]{cdocsfn}{cdocsch}
%    \end{macrocode}

%\iffalse
%</samplefinal>
%\fi
%
% %%%%%%%%%%%%%%%%%%%%%%%%%%%%%%%%%%%%%%
% \paragraph{Command Line Processing.}
%
% The following three command lines generate the output files
% |cdocscld|, |cdocscl1| and |cdocscl2|
% which should be identical to
% |cdocsdrf|, |cdocsch1| and |cdocsfn2|, respectively:
% \begin{center}
% \begin{tabular}{l}
% |latex -jobname cdocscld \|\\
% |  "\def\version{draft}\input{childdoc.def}\childdocforward{cdocsamp}"|\\
% |latex -jobname cdocscl1 \|\\
% |  "\input{childdoc.def}\childdocforward[cdocsamp]{cdocsch1}"|\\
% |latex -jobname cdocscl2 \|\\
% |  "\def\version{final}\input{childdoc.def}\childdocforward{cdocsch2}"|
% \end{tabular}
% \end{center}
% Note that the trailing backslash on each first line
% merely continues the input to the second line
% (for convenient cut ant paste).
% Furthermore, the command |latex| can be replaced by any
% of its alternative versions such as |pdflatex|.
%
% %%%%%%%%%%%%%%%%%%%%%%%%%%%%%%%%%%%%%%%%%%%%%%%%%%%%%%%%%%%%%%%%%%%%%%%%%%%%%%
% %%%%%%%%%%%%%%%%%%%%%%%%%%%%%%%%%%%%%%%%%%%%%%%%%%%%%%%%%%%%%%%%%%%%%%%%%%%%%%
% \section{Implementation}
%\iffalse
%<*package>
%\fi
%
% This section describes the definitions file |childdoc.def|.

% The definitions cannot be loaded using |\usepackage| or |\RequirePackage|
% which has a mechanism to prevent loading a style file more than once.
% When loading the definitions by means of |\input|
% multiple instances have to be prevented manually:
%\iffalse
%This code needs to be before the `\ProvidesFile' directive
%which is defined at the beginning of this file.
%Therefore it is also placed there and commented out here.
%</package>
%<*discard>
%\fi
%    \begin{macrocode}
\ifdefined\childdocmain\endinput\fi
%    \end{macrocode}
%\iffalse
%</discard>
%<*package>
%\fi
%
% \macro{\ifchilddoc}
% \macro{\ifchilddocmanual}
% The conditional |\ifchilddoc| tells whether a
% child (true) or main (false) document is being compiled.
% The conditional |\ifchilddocmanual| tells whether
% the |\includeonly| mechanism is used (false) or
% the selection of child files must be performed manually (true).
% The definitions initialise to false:
%    \begin{macrocode}
\newif\ifchilddoc
\newif\ifchilddocmanual
%    \end{macrocode}

% \macro{\childdocname}
% \macro{\childdocjob}
% The macro |\childdocname| stores the name of the main document
% to be compiled. The macro |\childdocjob| stores the name of
% the document on which the \LaTeX{} compiler was originally invoked.
% The content of |\jobname| cannot be compared
% to filenames specified in the source due to different catcodes.
% The following code rescans |\jobname|, stores the result
% in |\childdocname| and saves a copy in |\childdocjob|:
%    \begin{macrocode}
\edef\childdocname{\scantokens\expandafter{\jobname\noexpand}}
\let\childdocjob\childdocname
%    \end{macrocode}

% \macro{\childdocdisable}
% The macro |\childdocdisable| prevents the main file
% from being processed more than once.
% At this stage, the main document command |\childdocmain|
% is assumed to be called once again where it should do nothing.
% Any subsequent call to it should prevent
% a secondary processing of the main document
% It overwrites the forwarding commands
% |\childdocof| and |\childdocforward|
% with empty macros to prevent further inclusions of the main document:
%    \begin{macrocode}
\newcommand{\childdocdisable}
{
  \renewcommand{\childdocmain}[1]{\renewcommand{\childdocmain}[1]{\endinput}}
  \renewcommand{\childdocof}[1]{}
  \renewcommand{\childdocby}[2][]{}
  \renewcommand{\childdocforward}[2][]{}
  \renewcommand{\childdocdisable}{}
}
%    \end{macrocode}

% \macro{\childdocmain}
% The macro |\childdocmain| is to be called at the top of the main file
% with nothing or the main filename (without extension) as argument.
% First, it breaks loops.
% If the argument is not empty and does not match |\childdocname|
% (which is set by the first inclusion of |childdoc.def|),
% |\ifchilddoc| is set to true, |\includeonly| is applied to the child file
% and |\jobname| is set to the main file
% (for proper handling of |.aux| files):
%    \begin{macrocode}
\newcommand{\childdocmain}[1]
{
  \childdocdisable\childdocmain{}
  \if?#1?\else
    \begingroup
      \def\childdoctmp{#1}
      \ifx\childdoctmp\childdocname
        \def\childdoctmp{}
      \else
        \def\childdoctmp
        {
          \childdoctrue
          \includeonly{\childdocname}
          \def\childdocjob{#1}
          \def\jobname{#1}
        }
      \fi
      \expandafter
    \endgroup
    \childdoctmp
  \fi
}
%    \end{macrocode}

% \macro{\childdocof}
% The command |\childdocof| redirects
% compilation to the main file |#1|.
%    \begin{macrocode}
\newcommand{\childdocof}[1]
{
  \childdocdisable
  \childdoctrue
  \includeonly{\childdocname}
  \def\jobname{#1}
  \def\childdocjob{#1}
  \input{#1}
}
%    \end{macrocode}

% \macro{\childdocby}
% The command |\childdocby| ....
%    \begin{macrocode}
\newcommand{\childdocby}[2][]
{
  \childdocdisable
  \childdoctrue
  \childdocmanualtrue
  \if?#1?\else
    \def\jobname{#2}
  \fi
  \def\childdocjob{#2}
  \input{#2}
  \endinput
}
%    \end{macrocode}

% \macro{\childdocforward}
% The command |\childdocforward| redirects
% compilation to the main file or
% (if the optional argument is given) a child file.
% Parameters are set as if the main file
% or a child file starting with |\childdocof| was compiled.
% Then compilation is handed over to the main file:
%    \begin{macrocode}
\newcommand{\childdocforward}[2][]
{
  \begingroup
    \if?#1?
      \def\childdoctmp
      {
        \def\childdocname{#2}
        \def\childdocjob{#2}
        \def\jobname{#2}
        \input{#2}
        \endinput
      }
    \else
      \def\childdoctmp
      {
        \childdocdisable
        \def\childdocname{#2}
        \childdoctrue
        \includeonly{#2}
        \def\childdocjob{#1}
        \def\jobname{#1}
        \input{#1}
        \endinput
      }
    \fi
    \expandafter
  \endgroup
  \childdoctmp
}
%    \end{macrocode}

% \macro{\childdocforwardprefix}
% The command |\childdocforwardprefix| redirects
% compilation to the main or a child file by means of a pattern.
% The prefix |#1| in the current filename is replaced by |#2|
% and the suffix of the current filename is kept
% (it is assumed that the filename does not contain the substring `|~~~|'
% which is used as a delimiter).
% Compilation is handed over to the new file by |\childdocforward|:
%    \begin{macrocode}
\newcommand{\childdocforwardprefix}[3][]
{
  \begingroup
    \def\childdocextract #2##1~~~{\def\childdoctmp{\childdocforward[#1]{#3##1}}}
    \expandafter\childdocextract\childdocname~~~
    \expandafter
  \endgroup
  \childdoctmp
}
%    \end{macrocode}

% \macro{\childdoc}
% The deprecated macro |\childdoc| is a legacy version of |\childdocmain|:
%    \begin{macrocode}
\newcommand{\childdoc}{\childdocmain}
%    \end{macrocode}

% \macro{\childdocredirect}
% The deprecated macro |\childdocredirect| is a legacy version
% of |\childdocforward| and |\childdocforwardprefix|:
%    \begin{macrocode}
\newcommand{\childdocredirect}[2][]
{
  \begingroup
    \if?#1?
      \def\childdoctmp{\childdocforward{#2}}
    \else
      \def\childdoctmp{\childdocforwardprefix{#1}{#2}}
    \fi
    \expandafter
  \endgroup
  \childdoctmp
}
%    \end{macrocode}

%\iffalse
%</package>
%\fi
%
\endinput
|
and perform the replacements as outlined below.
Instead of |\childdocmain{|\textit{main}|}| add the following code
to the top of the main file:
%
\begin{center}
\begin{tabular}{l}
|\||ifdefined\childdocname\endinput\||fi\newif\ifchilddoc|\\
|\edef\childdocname{\scantokens\expandafter{\jobname\noexpand}}|\\
|\def\childdocmain{|\textit{main}|}\||ifx\childdocmain\childdocname\||else|\\
|\childdoctrue\includeonly{\childdocname}\let\jobname\childdocmain\||fi|\\
\end{tabular}
\end{center}
%
Instead of |\childdocof{|\textit{main}|}| just include the main file
at the top of each child file:
%
\begin{center}
|\input{|\textit{main}|}|
\end{center}
%
A simple redirection |\childdocforward{|\textit{dest}|}| is achieved by:
%
\begin{center}
|\def\jobname{|\textit{dest}|}\input{\jobname}|
\end{center}
%
The redirection with prefix
|\childdocforwardprefix[|\textit{prefix}|]{|\textit{dest}|}|
is accomplished by:
%
\begin{center}
\begin{tabular}{l}
|{\edef\jobname{\scantokens\expandafter{\jobname\noexpand}}|\\
|\def\redirectjob |\textit{prefix}|#1~~~{\gdef\jobname{|\textit{dest}|#1}}|\\
|\expandafter\redirectjob\jobname~~~}\input{\jobname}|
\end{tabular}
\end{center}

In an alternative approach,
child documents can be compiled by a specific command line
without additional code or specific definitions:
%
\begin{center}
|... -jobname "|\textit{target}|" "|[\textit{flags}]%
|\includeonly{|\textit{dest}|}\input{|\textit{main}|}"|
\end{center}
%

%%%%%%%%%%%%%%%%%%%%%%%%%%%%%%%%%%%%%%%%%%%%%%%%%%%%%%%%%%%%%%%%%%%%%%%%%%%%%%%%
%%%%%%%%%%%%%%%%%%%%%%%%%%%%%%%%%%%%%%%%%%%%%%%%%%%%%%%%%%%%%%%%%%%%%%%%%%%%%%%%
\section{Information}

%%%%%%%%%%%%%%%%%%%%%%%%%%%%%%%%%%%%%%%%%%%%%%%%%%%%%%%%%%%%%%%%%%%%%%%%%%%%%%%%
\subsection{Copyright}

Copyright \copyright{} 2017--2018 Niklas Beisert

This work may be distributed and/or modified under the
conditions of the \LaTeX{} Project Public License, either version 1.3
of this license or (at your option) any later version.
The latest version of this license is in
  \url{http://www.latex-project.org/lppl.txt}
and version 1.3 or later is part of all distributions of \LaTeX{}
version 2005/12/01 or later.

This work has the LPPL maintenance status `maintained'.

The Current Maintainer of this work is Niklas Beisert.

This work consists of the files |README.txt|, |childdoc.ins| and |childdoc.dtx|
as well as the derived files |childdoc.def|, |cdocsamp.tex|
with |cdocsch1.tex|, |cdocsch2.tex|, |cdocspt3.tex|, |cdocspt4.tex|,
|cdocsdrf.tex|, |cdocsfn1.tex|, |cdocsfn2.tex|
as well as |childdoc.pdf|.

%%%%%%%%%%%%%%%%%%%%%%%%%%%%%%%%%%%%%%%%%%%%%%%%%%%%%%%%%%%%%%%%%%%%%%%%%%%%%%%%
\subsection{Files and Installation}

The package consists of the files:
%
\begin{center}
\begin{tabular}{ll}
    |README.txt|   & readme file \\
    |childdoc.ins| & installation file \\
    |childdoc.dtx| & source file \\
    |childdoc.def| & definition file \\
    |cdocsamp.tex| & sample main file \\
    |cdocsch1.tex| & sample include file \\
    |cdocsch2.tex| & sample include file \\
    |cdocspt3.tex| & sample part file \\
    |cdocspt4.tex| & sample part file \\
    |cdocsdrf.tex| & sample redirection file \\
    |cdocsfn1.tex| & sample redirection file \\
    |cdocsfn2.tex| & sample redirection file \\
    |childdoc.pdf| & manual
\end{tabular}
\end{center}
%
The distribution consists of the files
|README.txt|, |childdoc.ins| and |childdoc.dtx|.
%
\begin{itemize}
\item
Run (pdf)\LaTeX{} on |childdoc.dtx|
to compile the manual |childdoc.pdf| (this file).
\item
Run \LaTeX{} on |childdoc.ins| to create the definitions file |childdoc.def|
and the sample |cdocsamp.tex| with include files
|cdocsch1.tex|, |cdocsch2.tex|, |cdocspt3.tex|, |cdocspt4.tex|,
|cdocsdrf.tex|, |cdocsfn1.tex|, |cdocsfn2.tex|.
Then copy the file |childdoc.def| to an appropriate directory of your \LaTeX{}
distribution, e.g.\ \textit{texmf-root}|/tex/latex/childdoc|.
\end{itemize}

%%%%%%%%%%%%%%%%%%%%%%%%%%%%%%%%%%%%%%%%%%%%%%%%%%%%%%%%%%%%%%%%%%%%%%%%%%%%%%%%
\subsection{Related CTAN Packages}

There are several other packages which offer a similar functionality:
%
\begin{itemize}
\item
The packages
\href{http://ctan.org/pkg/docmute}{\textsf{docmute}},
\href{http://ctan.org/pkg/includex}{\textsf{includex}} and
\href{http://ctan.org/pkg/standalone}{\textsf{standalone}}
provide commands to include only the document body of
a child file thus allowing both files to be compiled individually.
\item
The packages \href{http://ctan.org/pkg/subdocs}{\textsf{subdocs}}
and \href{http://ctan.org/pkg/subfiles}{\textsf{subfiles}}
provide structures in which the main and child documents can be
encapsulated and allowing them to be compiled individually.
The inclusion mechanism is different from the conventional |\include|.
\item
The package \href{http://ctan.org/pkg/combine}{\textsf{combine}}
is an elaborate solution to combine several documents into one.
\end{itemize}
%
See also the CTAN topic \href{http://ctan.org/topic/subdocs}{\textsf{subdocs}}
for further related packages.
The present package differs from the above solutions in that
a document structure constructed with the conventional |\include| mechanism
just needs two extra commands at the top of every file
such that all constituent files can be compiled individually.

%%%%%%%%%%%%%%%%%%%%%%%%%%%%%%%%%%%%%%%%%%%%%%%%%%%%%%%%%%%%%%%%%%%%%%%%%%%%%%%%
%\subsection{Feature Suggestions}
%
%The following is a list of features which may be useful for future
%versions of this package:
%%
%\begin{itemize}
%\item
%\ldots
%\end{itemize}

%%%%%%%%%%%%%%%%%%%%%%%%%%%%%%%%%%%%%%%%%%%%%%%%%%%%%%%%%%%%%%%%%%%%%%%%%%%%%%%%
\subsection{Revision History}

%%%%%%%%%%%%%%%%%%%%%%%%%%%%%%%%%%%%%%%%
\paragraph{v2.0:} 2018/12/30

\begin{itemize}
\item
immediate forward processing
\item
added |\childdocby| mechanism
\item
manual restructured
\end{itemize}

%%%%%%%%%%%%%%%%%%%%%%%%%%%%%%%%%%%%%%%%
\paragraph{v1.6:} 2018/01/17

\begin{itemize}
\item
application for development of include files
\item
corrections to manual
\end{itemize}

%%%%%%%%%%%%%%%%%%%%%%%%%%%%%%%%%%%%%%%%
\paragraph{v1.5:} 2017/05/21

\begin{itemize}
\item
more complete structuring introduced
\item
|\childdocof| introduced
\item
|\childdoc| renamed to |\childdocmain|
\item
|\childredirect| renamed to |\childdocforward| and |\childdocforwardprefix|
and functionality expanded
\end{itemize}

%%%%%%%%%%%%%%%%%%%%%%%%%%%%%%%%%%%%%%%%
\paragraph{v1.0:} 2017/04/27

\begin{itemize}
\item
manual and install package
\item
first version published on CTAN
\end{itemize}

%%%%%%%%%%%%%%%%%%%%%%%%%%%%%%%%%%%%%%%%
\paragraph{v0.6:} 2017/04/26

\begin{itemize}
\item
redirection mechanism added
\end{itemize}

%%%%%%%%%%%%%%%%%%%%%%%%%%%%%%%%%%%%%%%%
\paragraph{v0.5:} 2017/04/26

\begin{itemize}
\item
functionality in definition file
\end{itemize}


%%%%%%%%%%%%%%%%%%%%%%%%%%%%%%%%%%%%%%%%%%%%%%%%%%%%%%%%%%%%%%%%%%%%%%%%%%%%%%%%
%%%%%%%%%%%%%%%%%%%%%%%%%%%%%%%%%%%%%%%%%%%%%%%%%%%%%%%%%%%%%%%%%%%%%%%%%%%%%%%%
%%%%%%%%%%%%%%%%%%%%%%%%%%%%%%%%%%%%%%%%%%%%%%%%%%%%%%%%%%%%%%%%%%%%%%%%%%%%%%%%
\appendix

\settowidth\MacroIndent{\rmfamily\scriptsize 000\ }

 \DocInput{childdoc.dtx}

\end{document}
%</driver>
% \fi
%
% %%%%%%%%%%%%%%%%%%%%%%%%%%%%%%%%%%%%%%%%%%%%%%%%%%%%%%%%%%%%%%%%%%%%%%%%%%%%%%
% %%%%%%%%%%%%%%%%%%%%%%%%%%%%%%%%%%%%%%%%%%%%%%%%%%%%%%%%%%%%%%%%%%%%%%%%%%%%%%
% \section{Sample}
%\iffalse
%<*samplemain>
%\fi
%
% The following presents a sample document
% with two chapters, two parts, a title page,
% a compile flag as well as three forwarding files to set the flag.
% It consists of eight |.tex| files:
% \begin{center}
% \begin{tabular}{ll}
% |cdocsamp.tex|&main file\\
% |cdocsch1.tex|&include file for chapter 1\\
% |cdocsch2.tex|&include file for chapter 2\\
% |cdocspt3.tex|&include file for part 3\\
% |cdocspt4.tex|&include file for part 4\\
% |cdocsdrf.tex|&forwarding file for main file in draft mode\\
% |cdocsfi1.tex|&forwarding file for final version of chapter 1\\
% |cdocsfi2.tex|&forwarding file for final version of chapter 2\\
% \end{tabular}
% \end{center}
% Each of the eight files can be compiled directly by the \LaTeX{} compiler.
%
% %%%%%%%%%%%%%%%%%%%%%%%%%%%%%%%%%%%%%%
% \paragraph{Main File.}
%
% The main file is called |cdocsamp.tex|.
%
% Load the \textsf{childdoc} definitions and
% declare the filename for the main document:
%    \begin{macrocode}
% \iffalse
%
% childdoc.dtx Copyright (C) 2017-2018 Niklas Beisert
%
% This work may be distributed and/or modified under the
% conditions of the LaTeX Project Public License, either version 1.3
% of this license or (at your option) any later version.
% The latest version of this license is in
%   http://www.latex-project.org/lppl.txt
% and version 1.3 or later is part of all distributions of LaTeX
% version 2005/12/01 or later.
%
% This work has the LPPL maintenance status `maintained'.
%
% The Current Maintainer of this work is Niklas Beisert.
%
% This work consists of the files childdoc.dtx and childdoc.ins
% and the derived files childdoc.def and cdocsamp.tex with
% cdocsch1.tex, cdocsch2.tex, cdocsdrf.tex, cdocsfn1.tex, cdocsfn2.tex.
%
%<package>\ifdefined\childdocmain\endinput\fi
%<package>\ProvidesFile{childdoc.def}[2018/12/30 v2.0 child document driver]
%<samplemain>\ProvidesFile{cdocsamp.tex}[2018/12/30 v2.0 sample for childdoc]
%<*driver>
%\ProvidesFile{childdoc.drv}[2018/12/30 v2.0 childdoc reference manual file]
\PassOptionsToClass{10pt,a4paper}{article}
\documentclass{ltxdoc}

\usepackage[margin=35mm]{geometry}
\usepackage{hyperref}
\usepackage{hyperxmp}
\usepackage[usenames]{color}

\hypersetup{colorlinks=true}
\hypersetup{pdfstartview=FitH}
\hypersetup{pdfpagemode=UseNone}
\hypersetup{pdfsource={}}
\hypersetup{pdflang={en-UK}}
\hypersetup{pdfcopyright={Copyright 2017-2018 Niklas Beisert.
  This work may be distributed and/or modified under the
  conditions of the LaTeX Project Public License, either version 1.3
  of this license or (at your option) any later version.}}
\hypersetup{pdflicenseurl={http://www.latex-project.org/lppl.txt}}
\hypersetup{pdfcontactaddress={ETH Zurich, ITP, HIT K,
  Wolfgang-Pauli-Strasse 27}}
\hypersetup{pdfcontactpostcode={8093}}
\hypersetup{pdfcontactcity={Zurich}}
\hypersetup{pdfcontactcountry={Switzerland}}
\hypersetup{pdfcontactemail={nbeisert@itp.phys.ethz.ch}}
\hypersetup{pdfcontacturl={http://people.phys.ethz.ch/\xmptilde nbeisert/}}

\newcommand{\secref}[1]{\hyperref[#1]{section \ref*{#1}}}

\parskip1ex
\parindent0pt
\let\olditemize\itemize
\def\itemize{\olditemize\parskip0pt}

\begin{document}

\title{The \textsf{childdoc} Package}
\hypersetup{pdftitle={The childdoc Package}}
\author{Niklas Beisert\\[2ex]
  Institut f\"ur Theoretische Physik\\
  Eidgen\"ossische Technische Hochschule Z\"urich\\
  Wolfgang-Pauli-Strasse 27, 8093 Z\"urich, Switzerland\\[1ex]
  \href{mailto:nbeisert@itp.phys.ethz.ch}
  {\texttt{nbeisert@itp.phys.ethz.ch}}}
\hypersetup{pdfauthor={Niklas Beisert}}
\hypersetup{pdfsubject={Manual for the LaTeX2e Package childdoc}}
\date{30 December 2018, \textsf{v2.0}}
\maketitle

\begin{abstract}\noindent
\textsf{childdoc} is a \LaTeXe{} package
that enables the direct compilation
of document sections included by |\include|
to individual files.
\end{abstract}

\begingroup
\parskip0ex
\tableofcontents
\endgroup

%%%%%%%%%%%%%%%%%%%%%%%%%%%%%%%%%%%%%%%%%%%%%%%%%%%%%%%%%%%%%%%%%%%%%%%%%%%%%%%%
%%%%%%%%%%%%%%%%%%%%%%%%%%%%%%%%%%%%%%%%%%%%%%%%%%%%%%%%%%%%%%%%%%%%%%%%%%%%%%%%
\section{Introduction}

\LaTeX{} provides a mechanism to structure a large document (such as a book)
into a main file and several child files (containing the chapters)
using the |\include| command.
This mechanism is beneficial for documents
which span hundreds of pages in order to
make the source file(s) more manageable.
Moreover, compilation can be restricted to
selected child files by means of the |\includeonly| command.
The latter feature can be used to reduce the compilation time while editing
(this was significantly more useful in the earlier days of \LaTeX{})
or to generate a smaller document which is easier to navigate.
Another application of |\includeonly| is to generate
documents consisting of selected parts of the complete document.

However, there are a few drawbacks of the plain |\include| mechanism:
\begin{itemize}
\item
The child files cannot be compiled on their own,
they can only be compiled via the main file.
A naive editing environment
(such as a text editor with an option
to have the current file processed by \LaTeX)
may require one to switch to the main file before compiling;
attempting to compile the child file produces errors.
\item
The main file must be modified (each time)
to adjust the |\includeonly| command
to the present needs. This easily leaves the main file in a messy state.
\item
The generated document will always carry the filename
of the main document. This is inconvenient if
several child files are to be compiled and
to be kept for distribution.
\end{itemize}

The present package provides a simple interface
to make child files individually compilable by \LaTeX{}.
Compiling a child file then has the same effect as compiling
the main file with an |\includeonly| command
to select the appropriate child.
Moreover the generated document will carry the name of the child
rather than the main file.
This resolves all three above issues.

This feature is meant to make the editing of books,
thesis documents and lecture notes somewhat more convenient.
However, the package can also be used efficiently for
composing a series of documents (such as exercise sheets)
which are typically distributed individually.
It then assists the author in generating the individual documents
(potentially in different versions)
as well as a document containing the collected series.
Another application is in developing style files
or other kinds of included material
where compilation of the style file could redirect
to a sample or test file.

%%%%%%%%%%%%%%%%%%%%%%%%%%%%%%%%%%%%%%%%%%%%%%%%%%%%%%%%%%%%%%%%%%%%%%%%%%%%%%%%
%%%%%%%%%%%%%%%%%%%%%%%%%%%%%%%%%%%%%%%%%%%%%%%%%%%%%%%%%%%%%%%%%%%%%%%%%%%%%%%%
\section{Usage}

First of all, the package \textsf{childdoc} is \emph{not} a standard
\LaTeXe{} |.sty| style file! Therefore it needs to be invoked in
a non-standard way.

%%%%%%%%%%%%%%%%%%%%%%%%%%%%%%%%%%%%%%%%%%%%%%%%%%%%%%%%%%%%%%%%%%%%%%%%%%%%%%%%
\subsection{Included Files}
\label{sec:include}

%%%%%%%%%%%%%%%%%%%%%%%%%%%%%%%%%%%%%%%%
\DescribeMacro{\childdocmain}
To use the package, add the commands
\begin{center}
\begin{tabular}{l}
|\input{childdoc.def}|\\
|\childdocmain{}|\\
\end{tabular}
\end{center}
at the very top of the main \LaTeX{} file,
in particular \emph{before} the |\documentclass| statement!
The argument of |\childdocmain| should be left empty
(but it must be present).

%%%%%%%%%%%%%%%%%%%%%%%%%%%%%%%%%%%%%%%%
\DescribeMacro{\childdocof}
Furthermore, add the commands
\begin{center}
\begin{tabular}{l}
|\input{childdoc.def}|\\
|\childdocof{|\textit{main}|}|\\
\end{tabular}
\end{center}
at the top of every child file \textit{child}
which is included by |\include{|\textit{child}|}|
from within the main file
(or at least for those files to be compiled individually).
The argument \textit{main} must be the filename of the main file.

There are a couple of
considerations in setting up the main and child documents:

%%%%%%%%%%%%%%%%%%%%%%%%%%%%%%%%%%%%%%%%
\paragraph{Restrictions.}

Please note the following restrictions:
\begin{itemize}
\item
|\childdocmain| must be called with one argument \textit{main}
to ensure compatibility with earlier version of the package.
It must either be empty (|\childdocmain{}|)
or precisely match the filename of the main file in which it is specified.
See \secref{sec:detection} for further information.
\item
The filename \textit{main} must be specified without the |.tex| extension.
\item
The filename \textit{main} is case sensitive
(even in case-insensitive file systems)
due to internal string comparison.
\item
The argument \textit{main} should be fully expanded, it cannot be a macro.
\item
Subdirectories and special characters should be avoided in filenames.
\item
The command |\childdocmain{|\textit{main}|}| must be followed by a whitespace.
It should not be followed immediately by another command
or by a comment mark `|%|'.
This is because the \TeX{} parser reads the token immediately following
the argument of |\childdocmain| and puts it
at the beginning of every child section;
however, a white\-space is ignored.
\end{itemize}

%%%%%%%%%%%%%%%%%%%%%%%%%%%%%%%%%%%%%%%%
\paragraph{Content of Main File.}

It is advisable to place all content in the child files included by |\include|.
Any output contained in the main file will appear in all child documents
unless suppressed manually;
it cannot be suppressed automatically by the |\includeonly| directive
and thus should normally be avoided.
A method to include some content in the main file
by means of conditional processing is described in \secref{sec:conditional}.

%%%%%%%%%%%%%%%%%%%%%%%%%%%%%%%%%%%%%%%%
\paragraph{Page Numbering.}

When only a part of the document is compiled,
the appropriate numbering of pages
(as well as other status parameters)
is determined from the |.aux| files.
The latter contain information from previous passes.
However this information needs to propagate through
all intermediate child documents.
Therefore the page numbering in child documents may well
be inconsistent until the complete document is compiled at least once.

A useful (if unconventional) way to always ensure a consistent
page numbering is to restart the numbering in each child document
and denote the pages by `\textit{child}|.|\textit{page}'
where \textit{child} represents the chapter/section number of the child file.
This can be achieved by the command
|\numberwithin{page}{|\textit{child}|}|
of the \textsf{amsmath} package
where \textit{child} can be |chapter| or |section|
depending on the chosen structuring.
Alternatively, one can modify the macro |\thepage| appropriately
and reset the counter |page| at the start of each child file.

%%%%%%%%%%%%%%%%%%%%%%%%%%%%%%%%%%%%%%%%%%%%%%%%%%%%%%%%%%%%%%%%%%%%%%%%%%%%%%%%
\subsection{Conditional Processing}
\label{sec:conditional}

The package provides a mechanism to compile different versions
of a document. To customise the versions further some conditional processing
can come in handy to distinguish which version is being compiled.
The package provides two macros to describe the compilation context:

%%%%%%%%%%%%%%%%%%%%%%%%%%%%%%%%%%%%%%%%
\DescribeMacro{\ifchilddoc}
The conditional |\ifchilddoc| distinguishes between the compilation of
child documents and the main document:
%
\begin{center}
|\ifchilddoc |\textit{child-code}| |[|\||else |\textit{main-code}]| \||fi|
\end{center}

%%%%%%%%%%%%%%%%%%%%%%%%%%%%%%%%%%%%%%%%
\DescribeMacro{\childdocname}
\DescribeMacro{\childdocjob}
The macro |\childdocname| contains the filename (without extension)
of the main or child file being processed.
Note that |\childdocjob| will always contain the name of the main file.

%%%%%%%%%%%%%%%%%%%%%%%%%%%%%%%%%%%%%%%%
\paragraph{Title Page.}

Conditional processing can be used to include a title or banner page
in the main document when proper precautions are taken.
Importantly, the code in the main file should ensure that the page counter
(as well as other status parameters which are stored in the |.aux| files)
takes the same value after the conditional processing.
Otherwise the page numbers may take divergent values
depending on which part is compiled.

For example, a title page could be declared by:
%
\begin{center}
\begin{tabular}{l}
|\ifchilddoc\||else|\\
|\addtocounter{page}{-1}|\\
\textit{code for title page}\\
|\newpage|\\
|\||fi|
\end{tabular}
\end{center}
%
A banner page for the child documents can be generated by:
%
\begin{center}
\begin{tabular}{l}
|\ifchilddoc|\\
|\addtocounter{page}{-1}|\\
\textit{code for banner page}\\
|\newpage|\\
|\||fi|
\end{tabular}
\end{center}
%
Here one could write a message such as:
\begin{center}
|This is the part \childdocname{} of \childdocjob{}.|
\end{center}

%%%%%%%%%%%%%%%%%%%%%%%%%%%%%%%%%%%%%%%%%%%%%%%%%%%%%%%%%%%%%%%%%%%%%%%%%%%%%%%%
\subsection{Flags}
\label{sec:flags}

The package makes it easy to generate different versions
of the main or child documents.
To this end compilation flags can be defined
and assigned different default values.
They will be particularly useful in conjunction
with the forwarding mechanism described in \secref{sec:forward}.

For example, it may be useful to have a flag |\version|
which can be set to |draft| or |final|.
The document source will contain some conditional code
depending on the value of |\version|.
Suppose further, the flag should default to |final| for the main file
and to |draft| for child files
which is a natural assignment for editing the document.
This is achieved by placing the following code
in the preamble of the main document
(below the |\childdocmain| directive):
%
\begin{center}
\begin{tabular}{l}
|\ifchilddoc|\\
|\providecommand{\version}{draft}|\\
|\||else|\\
|\providecommand{\version}{final}|\\
|\||fi|
\end{tabular}
\end{center}
%
The definition by |\providecommand| makes sure
that previous definitions are not overwritten.
Further statements |\providecommand{\version}{...}|
can thus be added before the above code to override it.

For the main file, one might add a line
(between |\childdocmain| and the above block)
%
\begin{center}
|%\ifchilddoc\||else\providecommand{\version}{draft}\||fi|
\end{center}
%
which can be uncommented to produce a draft version.
Likewise one can add a line to the very top of a child file
(above the |\childdocof{|\textit{main}|}| directive)
%
\begin{center}
|%\providecommand{\version}{final}|
\end{center}
%
which can be uncommented to produce the final version of this child document.

%%%%%%%%%%%%%%%%%%%%%%%%%%%%%%%%%%%%%%%%%%%%%%%%%%%%%%%%%%%%%%%%%%%%%%%%%%%%%%%%
\subsection{Forwarding}
\label{sec:forward}

Different versions of the main or child documents
using compilation flags as described in \secref{sec:flags}
can be (permanently) stored in different files
for convenient compilation, viewing and distribution.
To this end, the package defines a command
to pass on compilation to a different file:

%%%%%%%%%%%%%%%%%%%%%%%%%%%%%%%%%%%%%%%%
\DescribeMacro{\childdocforward}
The command |\childdocforward| redirects processing to
another source file:
%
\begin{center}
\begin{tabular}{l}
|\input{childdoc.def}|\\
|\childdocforward[|\textit{main}|]{|\textit{dest}|}|\\
\end{tabular}
\end{center}
%
The argument \textit{dest} is the destination file
(without extension).
It should be the main file or one of the child files.
Note that further \textsf{childdoc} directives
such as |\childdocof| and |\childdocforward|
in the indicated file will be processed in this form.
The optional argument \textit{main}
passes on directly to the main file \textit{main}
while pretending to compile the child \textit{dest}.
This form behaves as if \textit{dest}
issues |\childdocof{|\textit{main}|}| right away,
and no further \textsf{childdoc} directives will be processed.

%%%%%%%%%%%%%%%%%%%%%%%%%%%%%%%%%%%%%%%%
\DescribeMacro{\...prefix}
In the alternative form |\childdocforwardprefix|,
%
\begin{center}
\begin{tabular}{l}
|\input{childdoc.def}|\\
|\childdocforwardprefix[|\textit{main}|]{|\textit{prefix}|}{|\textit{dest}|}|
\end{tabular}
\end{center}
%
the destination file is determined by a pattern
depending on the current file:
To make this work, the current file must be called
`{\textit{prefix}\hspace{0.2em}\textit{suffix}}'
with \textit{prefix} matching precisely the argument.
Processing is then passed on to the file
`{\textit{dest}\hspace{0.2em}\textit{suffix}}'.
Surely, the same effect is achieved by
directly specifying the
argument `{\textit{dest}\hspace{0.2em}\textit{suffix}}'
in the first form.
However, that requires to set up a different file
for each child. With the alternative form of the command
all these files can have exactly the same content
which simplifies setting them up and maintaining them.

For example, the following file |draft.tex|
with a compilation flag |\version| as described in \secref{sec:flags}
compiles the main document as a draft:
%
\begin{center}
\begin{tabular}{l}
|\def\version{draft}|\\
|\input{childdoc.def}|\\
|\childdocforward{|\textit{main}|}|
\end{tabular}
\end{center}
%
Likewise, the following files |final|\textit{nn}|.tex|
compile the final version of the child document
|child|\textit{nn}|.tex|:
%
\begin{center}
\begin{tabular}{l}
|\def\version{final}|\\
|\input{childdoc.def}|\\
|\childdocforwardprefix{final}{child}|
\end{tabular}
\end{center}
%

Note that when several versions of a main file and/or of each child file
are to be generated, it may be convenient to set up a |Makefile| or
shell script to automatise the process.

%%%%%%%%%%%%%%%%%%%%%%%%%%%%%%%%%%%%%%%%%%%%%%%%%%%%%%%%%%%%%%%%%%%%%%%%%%%%%%%%
\subsection{Command Line Processing}
\label{sec:commandline}

The effect of redirection files can also be achieved by invoking
the \LaTeX{} compiler with a more elaborate command line.
Most conveniently this should be done as part
of a shell script or a |Makefile|.

When using \textsf{childdoc} in the main file, the following
command lines effectively perform a redirection
(note that depending on the shell being used,
backslashes may have to be doubled: `|\|' $\to$ `|\\|'):
%
\begin{center}
|... -jobname "|\textit{target}|" |\\|"|[\textit{flags}]%
|\input{childdoc.def}\childdocforward[|\textit{main}|]{|\textit{dest}|}"|
\end{center}
%
Here \textit{target} is the name of the output file,
\textit{main} is the name of the main file
and \textit{dest} is the name of the main or child file to be processed
(all filenames without extensions).
The optional argument \textit{main} can be omitted
if \textit{main} matches \textit{dest}.
Optionally, compilation \textit{flags} can be defined via |\def| commands.
This command line makes the \TeX{} engine believe
it is compiling the file \textit{target}
whose content is specified as the latter parameter.
The provided code then forwards the processing to
\textit{main} or \textit{dest} as described in \secref{sec:forward}.

%%%%%%%%%%%%%%%%%%%%%%%%%%%%%%%%%%%%%%%%%%%%%%%%%%%%%%%%%%%%%%%%%%%%%%%%%%%%%%%%
\subsection{Include by Input}
\label{sec:input}

Including child documents by |\include| has some restrictions by design.
Most notably, the content of a child document always occupies
its own set of pages; pages cannot be shared between child documents.
Usually, this behaviour makes perfect sense
because each child document contain an essential part of the document.
However, in some situations it may be desirable to compose
a document from a collection of parts
without having mandatory page breaks between then.
For this case, the package
provides a mechanism to include parts
by |\input| which can also be processed individually.
However, by construction this mechanism
requires manual handling of the content to be output.

%%%%%%%%%%%%%%%%%%%%%%%%%%%%%%%%%%%%%%%%
\DescribeMacro{\ifchilddocmanual}
The main file should be prepared as usual, see \secref{sec:include}.
However, the document body must make a distinction
between processing of an individual part and of the main document, e.g.:
%
\begin{center}
\begin{tabular}{l}
|\ifchilddocmanual|\\
|\input{\childdocname}|\\
|\||else|\\
\textit{document body with }|\input{|\textit{part}|}|\\
|\||fi|
\end{tabular}
\end{center}
%
The conditional |\ifchilddocmanual| is true whenever
a part to be included by |\input| is being compiled,
and the name of the part is stored in |\childdocname|.

%%%%%%%%%%%%%%%%%%%%%%%%%%%%%%%%%%%%%%%%
\DescribeMacro{\childdocby}
Each part to be included by |\input| should start with:
%
\begin{center}
\begin{tabular}{l}
|\input{childdoc.def}|\\
|\childdocby{|\textit{main}|}|\\
\end{tabular}
\end{center}
%
The directive |\childdocby| is similar to |\childdocof|
described in \secref{sec:include},
but the subsequent selection of content must be done manually.
To that end, both |\ifchilddoc| and |\ifchilddocmanual|
will be true upon processing of a part,
and the name of the part is stored in |\childdocname|.
Note that |\jobname| will be set to the filename of the current part
so that each part receives an individual |.aux| file
that does not interfere with the |.aux| file(s) of the main document.
This behaviour can be altered by the alternative form
|\childdocby[*]{|\textit{main}|}| (with a non-empty optional argument)
which uses the |.aux| file of the main document
by setting |\jobname| to \textit{main}.

%%%%%%%%%%%%%%%%%%%%%%%%%%%%%%%%%%%%%%%%%%%%%%%%%%%%%%%%%%%%%%%%%%%%%%%%%%%%%%%%
\subsection{Driver Development}
\label{sec:driver}

The \textsf{childdoc} mechanism can also be use for the development
of definition files such as \LaTeX{} styles or classes.
This case differs from the above setup with multiple parts
included by |\include| in that no |\includeonly| should be invoked.
This can be achieved by starting the include file
(before |\ProvidesPackage|) with:
%
\begin{center}
\begin{tabular}{l}
|\input{childdoc.def}|\\
|\childdocforward{|\textit{main}|}|\\
\end{tabular}
\end{center}
%
or alternatively with:
%
\begin{center}
\begin{tabular}{l}
|\input{childdoc.def}|\\
|\childdocby{|\textit{main}|}|\\
\end{tabular}
\end{center}
%
Both forms have slightly different effects as described above.
The main file is prepared as usual, see \secref{sec:include}.

%%%%%%%%%%%%%%%%%%%%%%%%%%%%%%%%%%%%%%%%%%%%%%%%%%%%%%%%%%%%%%%%%%%%%%%%%%%%%%%%
\subsection{Legacy Detection}
\label{sec:detection}

The directive |\childdocmain| in the main file can detect
whether the complete document or merely a child is to be compiled
even without using the directive |\childdocof|.
This method is deprecated because it is less robust
and there is no compelling reason to use it;
it is merely provided for backward compatibility
and it may be removed in future versions.

If the detection mechanism is to be used,
it is mandatory to correctly specify
the filename of the main file as the argument of |\childdocmain|:
%
\begin{center}
\begin{tabular}{l}
|\input{childdoc.def}|\\
|\childdocmain{|\textit{main}|}|\\
\end{tabular}
\end{center}
%
If |\jobname| does not match the argument \textit{main} of |\childdocmain|,
it is assumed that |\jobname| points to the child file to be compiled.
When using |\childdocmain| with the main file specified as argument,
it suffices to start a child file
with just |\input{|\textit{main}|}|
without loading of the package and using |\childdocof|.
If instead all processing is done
with the appropriate \textsf{childdoc} directives,
the argument of \textit{main} of |\childdocmain| can be empty.

An alternative version of the command line processing described
in \secref{sec:commandline} using the detection mechanism reads:
%
\begin{center}
|... -jobname "|\textit{target}|" "|[\textit{flags}]%
[|\def\jobname{|\textit{dest}|}|]|\input{|\textit{main}|}"|
\end{center}

%%%%%%%%%%%%%%%%%%%%%%%%%%%%%%%%%%%%%%%%%%%%%%%%%%%%%%%%%%%%%%%%%%%%%%%%%%%%%%%%
\subsection{Manual Code}
\label{sec:manual}

In case one cannot be certain whether the definitions file |childdoc.def|
is installed on the target \TeX{} distribution
and one prefers not to ship it,
it is conceivable to paste a few relevant commands into the sources.

To that end, drop all statements |\input{childdoc.def}|
and perform the replacements as outlined below.
Instead of |\childdocmain{|\textit{main}|}| add the following code
to the top of the main file:
%
\begin{center}
\begin{tabular}{l}
|\||ifdefined\childdocname\endinput\||fi\newif\ifchilddoc|\\
|\edef\childdocname{\scantokens\expandafter{\jobname\noexpand}}|\\
|\def\childdocmain{|\textit{main}|}\||ifx\childdocmain\childdocname\||else|\\
|\childdoctrue\includeonly{\childdocname}\let\jobname\childdocmain\||fi|\\
\end{tabular}
\end{center}
%
Instead of |\childdocof{|\textit{main}|}| just include the main file
at the top of each child file:
%
\begin{center}
|\input{|\textit{main}|}|
\end{center}
%
A simple redirection |\childdocforward{|\textit{dest}|}| is achieved by:
%
\begin{center}
|\def\jobname{|\textit{dest}|}\input{\jobname}|
\end{center}
%
The redirection with prefix
|\childdocforwardprefix[|\textit{prefix}|]{|\textit{dest}|}|
is accomplished by:
%
\begin{center}
\begin{tabular}{l}
|{\edef\jobname{\scantokens\expandafter{\jobname\noexpand}}|\\
|\def\redirectjob |\textit{prefix}|#1~~~{\gdef\jobname{|\textit{dest}|#1}}|\\
|\expandafter\redirectjob\jobname~~~}\input{\jobname}|
\end{tabular}
\end{center}

In an alternative approach,
child documents can be compiled by a specific command line
without additional code or specific definitions:
%
\begin{center}
|... -jobname "|\textit{target}|" "|[\textit{flags}]%
|\includeonly{|\textit{dest}|}\input{|\textit{main}|}"|
\end{center}
%

%%%%%%%%%%%%%%%%%%%%%%%%%%%%%%%%%%%%%%%%%%%%%%%%%%%%%%%%%%%%%%%%%%%%%%%%%%%%%%%%
%%%%%%%%%%%%%%%%%%%%%%%%%%%%%%%%%%%%%%%%%%%%%%%%%%%%%%%%%%%%%%%%%%%%%%%%%%%%%%%%
\section{Information}

%%%%%%%%%%%%%%%%%%%%%%%%%%%%%%%%%%%%%%%%%%%%%%%%%%%%%%%%%%%%%%%%%%%%%%%%%%%%%%%%
\subsection{Copyright}

Copyright \copyright{} 2017--2018 Niklas Beisert

This work may be distributed and/or modified under the
conditions of the \LaTeX{} Project Public License, either version 1.3
of this license or (at your option) any later version.
The latest version of this license is in
  \url{http://www.latex-project.org/lppl.txt}
and version 1.3 or later is part of all distributions of \LaTeX{}
version 2005/12/01 or later.

This work has the LPPL maintenance status `maintained'.

The Current Maintainer of this work is Niklas Beisert.

This work consists of the files |README.txt|, |childdoc.ins| and |childdoc.dtx|
as well as the derived files |childdoc.def|, |cdocsamp.tex|
with |cdocsch1.tex|, |cdocsch2.tex|, |cdocspt3.tex|, |cdocspt4.tex|,
|cdocsdrf.tex|, |cdocsfn1.tex|, |cdocsfn2.tex|
as well as |childdoc.pdf|.

%%%%%%%%%%%%%%%%%%%%%%%%%%%%%%%%%%%%%%%%%%%%%%%%%%%%%%%%%%%%%%%%%%%%%%%%%%%%%%%%
\subsection{Files and Installation}

The package consists of the files:
%
\begin{center}
\begin{tabular}{ll}
    |README.txt|   & readme file \\
    |childdoc.ins| & installation file \\
    |childdoc.dtx| & source file \\
    |childdoc.def| & definition file \\
    |cdocsamp.tex| & sample main file \\
    |cdocsch1.tex| & sample include file \\
    |cdocsch2.tex| & sample include file \\
    |cdocspt3.tex| & sample part file \\
    |cdocspt4.tex| & sample part file \\
    |cdocsdrf.tex| & sample redirection file \\
    |cdocsfn1.tex| & sample redirection file \\
    |cdocsfn2.tex| & sample redirection file \\
    |childdoc.pdf| & manual
\end{tabular}
\end{center}
%
The distribution consists of the files
|README.txt|, |childdoc.ins| and |childdoc.dtx|.
%
\begin{itemize}
\item
Run (pdf)\LaTeX{} on |childdoc.dtx|
to compile the manual |childdoc.pdf| (this file).
\item
Run \LaTeX{} on |childdoc.ins| to create the definitions file |childdoc.def|
and the sample |cdocsamp.tex| with include files
|cdocsch1.tex|, |cdocsch2.tex|, |cdocspt3.tex|, |cdocspt4.tex|,
|cdocsdrf.tex|, |cdocsfn1.tex|, |cdocsfn2.tex|.
Then copy the file |childdoc.def| to an appropriate directory of your \LaTeX{}
distribution, e.g.\ \textit{texmf-root}|/tex/latex/childdoc|.
\end{itemize}

%%%%%%%%%%%%%%%%%%%%%%%%%%%%%%%%%%%%%%%%%%%%%%%%%%%%%%%%%%%%%%%%%%%%%%%%%%%%%%%%
\subsection{Related CTAN Packages}

There are several other packages which offer a similar functionality:
%
\begin{itemize}
\item
The packages
\href{http://ctan.org/pkg/docmute}{\textsf{docmute}},
\href{http://ctan.org/pkg/includex}{\textsf{includex}} and
\href{http://ctan.org/pkg/standalone}{\textsf{standalone}}
provide commands to include only the document body of
a child file thus allowing both files to be compiled individually.
\item
The packages \href{http://ctan.org/pkg/subdocs}{\textsf{subdocs}}
and \href{http://ctan.org/pkg/subfiles}{\textsf{subfiles}}
provide structures in which the main and child documents can be
encapsulated and allowing them to be compiled individually.
The inclusion mechanism is different from the conventional |\include|.
\item
The package \href{http://ctan.org/pkg/combine}{\textsf{combine}}
is an elaborate solution to combine several documents into one.
\end{itemize}
%
See also the CTAN topic \href{http://ctan.org/topic/subdocs}{\textsf{subdocs}}
for further related packages.
The present package differs from the above solutions in that
a document structure constructed with the conventional |\include| mechanism
just needs two extra commands at the top of every file
such that all constituent files can be compiled individually.

%%%%%%%%%%%%%%%%%%%%%%%%%%%%%%%%%%%%%%%%%%%%%%%%%%%%%%%%%%%%%%%%%%%%%%%%%%%%%%%%
%\subsection{Feature Suggestions}
%
%The following is a list of features which may be useful for future
%versions of this package:
%%
%\begin{itemize}
%\item
%\ldots
%\end{itemize}

%%%%%%%%%%%%%%%%%%%%%%%%%%%%%%%%%%%%%%%%%%%%%%%%%%%%%%%%%%%%%%%%%%%%%%%%%%%%%%%%
\subsection{Revision History}

%%%%%%%%%%%%%%%%%%%%%%%%%%%%%%%%%%%%%%%%
\paragraph{v2.0:} 2018/12/30

\begin{itemize}
\item
immediate forward processing
\item
added |\childdocby| mechanism
\item
manual restructured
\end{itemize}

%%%%%%%%%%%%%%%%%%%%%%%%%%%%%%%%%%%%%%%%
\paragraph{v1.6:} 2018/01/17

\begin{itemize}
\item
application for development of include files
\item
corrections to manual
\end{itemize}

%%%%%%%%%%%%%%%%%%%%%%%%%%%%%%%%%%%%%%%%
\paragraph{v1.5:} 2017/05/21

\begin{itemize}
\item
more complete structuring introduced
\item
|\childdocof| introduced
\item
|\childdoc| renamed to |\childdocmain|
\item
|\childredirect| renamed to |\childdocforward| and |\childdocforwardprefix|
and functionality expanded
\end{itemize}

%%%%%%%%%%%%%%%%%%%%%%%%%%%%%%%%%%%%%%%%
\paragraph{v1.0:} 2017/04/27

\begin{itemize}
\item
manual and install package
\item
first version published on CTAN
\end{itemize}

%%%%%%%%%%%%%%%%%%%%%%%%%%%%%%%%%%%%%%%%
\paragraph{v0.6:} 2017/04/26

\begin{itemize}
\item
redirection mechanism added
\end{itemize}

%%%%%%%%%%%%%%%%%%%%%%%%%%%%%%%%%%%%%%%%
\paragraph{v0.5:} 2017/04/26

\begin{itemize}
\item
functionality in definition file
\end{itemize}


%%%%%%%%%%%%%%%%%%%%%%%%%%%%%%%%%%%%%%%%%%%%%%%%%%%%%%%%%%%%%%%%%%%%%%%%%%%%%%%%
%%%%%%%%%%%%%%%%%%%%%%%%%%%%%%%%%%%%%%%%%%%%%%%%%%%%%%%%%%%%%%%%%%%%%%%%%%%%%%%%
%%%%%%%%%%%%%%%%%%%%%%%%%%%%%%%%%%%%%%%%%%%%%%%%%%%%%%%%%%%%%%%%%%%%%%%%%%%%%%%%
\appendix

\settowidth\MacroIndent{\rmfamily\scriptsize 000\ }

 \DocInput{childdoc.dtx}

\end{document}
%</driver>
% \fi
%
% %%%%%%%%%%%%%%%%%%%%%%%%%%%%%%%%%%%%%%%%%%%%%%%%%%%%%%%%%%%%%%%%%%%%%%%%%%%%%%
% %%%%%%%%%%%%%%%%%%%%%%%%%%%%%%%%%%%%%%%%%%%%%%%%%%%%%%%%%%%%%%%%%%%%%%%%%%%%%%
% \section{Sample}
%\iffalse
%<*samplemain>
%\fi
%
% The following presents a sample document
% with two chapters, two parts, a title page,
% a compile flag as well as three forwarding files to set the flag.
% It consists of eight |.tex| files:
% \begin{center}
% \begin{tabular}{ll}
% |cdocsamp.tex|&main file\\
% |cdocsch1.tex|&include file for chapter 1\\
% |cdocsch2.tex|&include file for chapter 2\\
% |cdocspt3.tex|&include file for part 3\\
% |cdocspt4.tex|&include file for part 4\\
% |cdocsdrf.tex|&forwarding file for main file in draft mode\\
% |cdocsfi1.tex|&forwarding file for final version of chapter 1\\
% |cdocsfi2.tex|&forwarding file for final version of chapter 2\\
% \end{tabular}
% \end{center}
% Each of the eight files can be compiled directly by the \LaTeX{} compiler.
%
% %%%%%%%%%%%%%%%%%%%%%%%%%%%%%%%%%%%%%%
% \paragraph{Main File.}
%
% The main file is called |cdocsamp.tex|.
%
% Load the \textsf{childdoc} definitions and
% declare the filename for the main document:
%    \begin{macrocode}
\input{childdoc.def}
\childdocmain{}
%    \end{macrocode}

% Optional override for |\version| flag:
%    \begin{macrocode}
%%\ifchilddoc\else\providecommand{\version}{draft}\fi
%    \end{macrocode}

% Define the default values for the |\version| flag
% (|final| for the main file and |draft| for childs):
%    \begin{macrocode}
\ifchilddoc
\providecommand{\version}{draft}
\else
\providecommand{\version}{final}
\fi
%    \end{macrocode}

% Load the standard document class:
%    \begin{macrocode}
\documentclass[12pt]{article}
%    \end{macrocode}

% Start the document body:
%    \begin{macrocode}
\begin{document}
%    \end{macrocode}

% Declare a title page.
% Print title, part of document being processed and version flag:
%    \begin{macrocode}
\addtocounter{page}{-1}
\begin{center}
{\LARGE\bfseries{}childdoc example\par}
\vspace{1cm}
\ifchilddoc
\ifchilddocmanual part\else chapter\fi:
`\childdocname' of `\childdocjob'\par
\else
main document: `\childdocjob'\par
\fi
version: \version\par
\end{center}
\newpage
%    \end{macrocode}

% Manually include selected file,
% otherwise process as usual:
%    \begin{macrocode}
\ifchilddocmanual
\section*{part `\childdocname'}
\input{\childdocname}
\else
%    \end{macrocode}

% Include the two chapters:
%    \begin{macrocode}
\include{cdocsch1}
\include{cdocsch2}
%    \end{macrocode}

% Include the two parts unless only chapters should be displayed:
%    \begin{macrocode}
\ifchilddoc\else
\section{part three}
\input{cdocspt3}
\section{part four}
\input{cdocspt4}
\fi
%    \end{macrocode}

% Process as usual until here:
%    \begin{macrocode}
\fi
%    \end{macrocode}

% End of document body:
%    \begin{macrocode}
\end{document}
%    \end{macrocode}
%\iffalse
%</samplemain>
%\fi
%
% %%%%%%%%%%%%%%%%%%%%%%%%%%%%%%%%%%%%%%
% \paragraph{Chapter Include Files.}
%
% The include files are called |cdocsch1.tex| and |cdocsch2.tex|.
%
%\iffalse
%<*samplechap1|samplechap2>
%\fi

% Optional override for |\version| flag:
%    \begin{macrocode}
%%\providecommand{\version}{final}
%    \end{macrocode}

% Include the main document:
%    \begin{macrocode}
\input{childdoc.def}
\childdocof{cdocsamp}
%    \end{macrocode}

%\iffalse
%</samplechap1|samplechap2>
%\fi
%
%\iffalse
%<*samplechap1>
%\fi
% Some text for chapter 1:
%    \begin{macrocode}
\section{one}
some text in chapter one
%    \end{macrocode}

%\iffalse
%</samplechap1>
%\fi
% Some text for chapter 2:
%\iffalse
%<*samplechap2>
%\fi
%    \begin{macrocode}
\section{two}
more text in chapter two
%    \end{macrocode}

%\iffalse
%</samplechap2>
%\fi
%
% %%%%%%%%%%%%%%%%%%%%%%%%%%%%%%%%%%%%%%
% \paragraph{Part Include Files.}
%
% The include files are called |cdocspt3.tex| and |cdocspt4.tex|.
%
%\iffalse
%<*samplepart3|samplepart4>
%\fi

% Optional override for |\version| flag:
%    \begin{macrocode}
%%\providecommand{\version}{final}
%    \end{macrocode}

% Include the main document:
%    \begin{macrocode}
\input{childdoc.def}
\childdocby{cdocsamp}
%    \end{macrocode}

%\iffalse
%</samplepart3|samplepart4>
%\fi
%
%\iffalse
%<*samplepart3>
%\fi
% Some text for part 3:
%    \begin{macrocode}
some text in part three
%    \end{macrocode}

%\iffalse
%</samplepart3>
%\fi
% Some text for part 4:
%\iffalse
%<*samplepart4>
%\fi
%    \begin{macrocode}
more text in part four
%    \end{macrocode}

%\iffalse
%</samplepart4>
%\fi
%
% %%%%%%%%%%%%%%%%%%%%%%%%%%%%%%%%%%%%%%
% \paragraph{Forwarding for a Complete Draft.}
%
% The following forwarding file |cdocsdrf.tex|
% compiles the main document in draft mode:
%\iffalse
%<*sampledraft>
%\fi
%    \begin{macrocode}
\def\version{draft}
\input{childdoc.def}
\childdocforward{cdocsamp}
%    \end{macrocode}

%\iffalse
%</sampledraft>
%\fi
%
% %%%%%%%%%%%%%%%%%%%%%%%%%%%%%%%%%%%%%%
% \paragraph{Forwarding for Final Version of the Chapters.}
%
% The following forwarding files |cdocsfn1.tex| and |cdocsfn2.tex|
% (with identical content)
% compile the final versions of the child documents
% |cdocsch1.tex| and |cdocsch2.tex|, respectively:
%\iffalse
%<*samplefinal>
%\fi
%    \begin{macrocode}
\def\version{final}
\input{childdoc.def}
\childdocforwardprefix[cdocsamp]{cdocsfn}{cdocsch}
%    \end{macrocode}

%\iffalse
%</samplefinal>
%\fi
%
% %%%%%%%%%%%%%%%%%%%%%%%%%%%%%%%%%%%%%%
% \paragraph{Command Line Processing.}
%
% The following three command lines generate the output files
% |cdocscld|, |cdocscl1| and |cdocscl2|
% which should be identical to
% |cdocsdrf|, |cdocsch1| and |cdocsfn2|, respectively:
% \begin{center}
% \begin{tabular}{l}
% |latex -jobname cdocscld \|\\
% |  "\def\version{draft}\input{childdoc.def}\childdocforward{cdocsamp}"|\\
% |latex -jobname cdocscl1 \|\\
% |  "\input{childdoc.def}\childdocforward[cdocsamp]{cdocsch1}"|\\
% |latex -jobname cdocscl2 \|\\
% |  "\def\version{final}\input{childdoc.def}\childdocforward{cdocsch2}"|
% \end{tabular}
% \end{center}
% Note that the trailing backslash on each first line
% merely continues the input to the second line
% (for convenient cut ant paste).
% Furthermore, the command |latex| can be replaced by any
% of its alternative versions such as |pdflatex|.
%
% %%%%%%%%%%%%%%%%%%%%%%%%%%%%%%%%%%%%%%%%%%%%%%%%%%%%%%%%%%%%%%%%%%%%%%%%%%%%%%
% %%%%%%%%%%%%%%%%%%%%%%%%%%%%%%%%%%%%%%%%%%%%%%%%%%%%%%%%%%%%%%%%%%%%%%%%%%%%%%
% \section{Implementation}
%\iffalse
%<*package>
%\fi
%
% This section describes the definitions file |childdoc.def|.

% The definitions cannot be loaded using |\usepackage| or |\RequirePackage|
% which has a mechanism to prevent loading a style file more than once.
% When loading the definitions by means of |\input|
% multiple instances have to be prevented manually:
%\iffalse
%This code needs to be before the `\ProvidesFile' directive
%which is defined at the beginning of this file.
%Therefore it is also placed there and commented out here.
%</package>
%<*discard>
%\fi
%    \begin{macrocode}
\ifdefined\childdocmain\endinput\fi
%    \end{macrocode}
%\iffalse
%</discard>
%<*package>
%\fi
%
% \macro{\ifchilddoc}
% \macro{\ifchilddocmanual}
% The conditional |\ifchilddoc| tells whether a
% child (true) or main (false) document is being compiled.
% The conditional |\ifchilddocmanual| tells whether
% the |\includeonly| mechanism is used (false) or
% the selection of child files must be performed manually (true).
% The definitions initialise to false:
%    \begin{macrocode}
\newif\ifchilddoc
\newif\ifchilddocmanual
%    \end{macrocode}

% \macro{\childdocname}
% \macro{\childdocjob}
% The macro |\childdocname| stores the name of the main document
% to be compiled. The macro |\childdocjob| stores the name of
% the document on which the \LaTeX{} compiler was originally invoked.
% The content of |\jobname| cannot be compared
% to filenames specified in the source due to different catcodes.
% The following code rescans |\jobname|, stores the result
% in |\childdocname| and saves a copy in |\childdocjob|:
%    \begin{macrocode}
\edef\childdocname{\scantokens\expandafter{\jobname\noexpand}}
\let\childdocjob\childdocname
%    \end{macrocode}

% \macro{\childdocdisable}
% The macro |\childdocdisable| prevents the main file
% from being processed more than once.
% At this stage, the main document command |\childdocmain|
% is assumed to be called once again where it should do nothing.
% Any subsequent call to it should prevent
% a secondary processing of the main document
% It overwrites the forwarding commands
% |\childdocof| and |\childdocforward|
% with empty macros to prevent further inclusions of the main document:
%    \begin{macrocode}
\newcommand{\childdocdisable}
{
  \renewcommand{\childdocmain}[1]{\renewcommand{\childdocmain}[1]{\endinput}}
  \renewcommand{\childdocof}[1]{}
  \renewcommand{\childdocby}[2][]{}
  \renewcommand{\childdocforward}[2][]{}
  \renewcommand{\childdocdisable}{}
}
%    \end{macrocode}

% \macro{\childdocmain}
% The macro |\childdocmain| is to be called at the top of the main file
% with nothing or the main filename (without extension) as argument.
% First, it breaks loops.
% If the argument is not empty and does not match |\childdocname|
% (which is set by the first inclusion of |childdoc.def|),
% |\ifchilddoc| is set to true, |\includeonly| is applied to the child file
% and |\jobname| is set to the main file
% (for proper handling of |.aux| files):
%    \begin{macrocode}
\newcommand{\childdocmain}[1]
{
  \childdocdisable\childdocmain{}
  \if?#1?\else
    \begingroup
      \def\childdoctmp{#1}
      \ifx\childdoctmp\childdocname
        \def\childdoctmp{}
      \else
        \def\childdoctmp
        {
          \childdoctrue
          \includeonly{\childdocname}
          \def\childdocjob{#1}
          \def\jobname{#1}
        }
      \fi
      \expandafter
    \endgroup
    \childdoctmp
  \fi
}
%    \end{macrocode}

% \macro{\childdocof}
% The command |\childdocof| redirects
% compilation to the main file |#1|.
%    \begin{macrocode}
\newcommand{\childdocof}[1]
{
  \childdocdisable
  \childdoctrue
  \includeonly{\childdocname}
  \def\jobname{#1}
  \def\childdocjob{#1}
  \input{#1}
}
%    \end{macrocode}

% \macro{\childdocby}
% The command |\childdocby| ....
%    \begin{macrocode}
\newcommand{\childdocby}[2][]
{
  \childdocdisable
  \childdoctrue
  \childdocmanualtrue
  \if?#1?\else
    \def\jobname{#2}
  \fi
  \def\childdocjob{#2}
  \input{#2}
  \endinput
}
%    \end{macrocode}

% \macro{\childdocforward}
% The command |\childdocforward| redirects
% compilation to the main file or
% (if the optional argument is given) a child file.
% Parameters are set as if the main file
% or a child file starting with |\childdocof| was compiled.
% Then compilation is handed over to the main file:
%    \begin{macrocode}
\newcommand{\childdocforward}[2][]
{
  \begingroup
    \if?#1?
      \def\childdoctmp
      {
        \def\childdocname{#2}
        \def\childdocjob{#2}
        \def\jobname{#2}
        \input{#2}
        \endinput
      }
    \else
      \def\childdoctmp
      {
        \childdocdisable
        \def\childdocname{#2}
        \childdoctrue
        \includeonly{#2}
        \def\childdocjob{#1}
        \def\jobname{#1}
        \input{#1}
        \endinput
      }
    \fi
    \expandafter
  \endgroup
  \childdoctmp
}
%    \end{macrocode}

% \macro{\childdocforwardprefix}
% The command |\childdocforwardprefix| redirects
% compilation to the main or a child file by means of a pattern.
% The prefix |#1| in the current filename is replaced by |#2|
% and the suffix of the current filename is kept
% (it is assumed that the filename does not contain the substring `|~~~|'
% which is used as a delimiter).
% Compilation is handed over to the new file by |\childdocforward|:
%    \begin{macrocode}
\newcommand{\childdocforwardprefix}[3][]
{
  \begingroup
    \def\childdocextract #2##1~~~{\def\childdoctmp{\childdocforward[#1]{#3##1}}}
    \expandafter\childdocextract\childdocname~~~
    \expandafter
  \endgroup
  \childdoctmp
}
%    \end{macrocode}

% \macro{\childdoc}
% The deprecated macro |\childdoc| is a legacy version of |\childdocmain|:
%    \begin{macrocode}
\newcommand{\childdoc}{\childdocmain}
%    \end{macrocode}

% \macro{\childdocredirect}
% The deprecated macro |\childdocredirect| is a legacy version
% of |\childdocforward| and |\childdocforwardprefix|:
%    \begin{macrocode}
\newcommand{\childdocredirect}[2][]
{
  \begingroup
    \if?#1?
      \def\childdoctmp{\childdocforward{#2}}
    \else
      \def\childdoctmp{\childdocforwardprefix{#1}{#2}}
    \fi
    \expandafter
  \endgroup
  \childdoctmp
}
%    \end{macrocode}

%\iffalse
%</package>
%\fi
%
\endinput

\childdocmain{}
%    \end{macrocode}

% Optional override for |\version| flag:
%    \begin{macrocode}
%%\ifchilddoc\else\providecommand{\version}{draft}\fi
%    \end{macrocode}

% Define the default values for the |\version| flag
% (|final| for the main file and |draft| for childs):
%    \begin{macrocode}
\ifchilddoc
\providecommand{\version}{draft}
\else
\providecommand{\version}{final}
\fi
%    \end{macrocode}

% Load the standard document class:
%    \begin{macrocode}
\documentclass[12pt]{article}
%    \end{macrocode}

% Start the document body:
%    \begin{macrocode}
\begin{document}
%    \end{macrocode}

% Declare a title page.
% Print title, part of document being processed and version flag:
%    \begin{macrocode}
\addtocounter{page}{-1}
\begin{center}
{\LARGE\bfseries{}childdoc example\par}
\vspace{1cm}
\ifchilddoc
\ifchilddocmanual part\else chapter\fi:
`\childdocname' of `\childdocjob'\par
\else
main document: `\childdocjob'\par
\fi
version: \version\par
\end{center}
\newpage
%    \end{macrocode}

% Manually include selected file,
% otherwise process as usual:
%    \begin{macrocode}
\ifchilddocmanual
\section*{part `\childdocname'}
\input{\childdocname}
\else
%    \end{macrocode}

% Include the two chapters:
%    \begin{macrocode}
\include{cdocsch1}
\include{cdocsch2}
%    \end{macrocode}

% Include the two parts unless only chapters should be displayed:
%    \begin{macrocode}
\ifchilddoc\else
\section{part three}
\input{cdocspt3}
\section{part four}
\input{cdocspt4}
\fi
%    \end{macrocode}

% Process as usual until here:
%    \begin{macrocode}
\fi
%    \end{macrocode}

% End of document body:
%    \begin{macrocode}
\end{document}
%    \end{macrocode}
%\iffalse
%</samplemain>
%\fi
%
% %%%%%%%%%%%%%%%%%%%%%%%%%%%%%%%%%%%%%%
% \paragraph{Chapter Include Files.}
%
% The include files are called |cdocsch1.tex| and |cdocsch2.tex|.
%
%\iffalse
%<*samplechap1|samplechap2>
%\fi

% Optional override for |\version| flag:
%    \begin{macrocode}
%%\providecommand{\version}{final}
%    \end{macrocode}

% Include the main document:
%    \begin{macrocode}
% \iffalse
%
% childdoc.dtx Copyright (C) 2017-2018 Niklas Beisert
%
% This work may be distributed and/or modified under the
% conditions of the LaTeX Project Public License, either version 1.3
% of this license or (at your option) any later version.
% The latest version of this license is in
%   http://www.latex-project.org/lppl.txt
% and version 1.3 or later is part of all distributions of LaTeX
% version 2005/12/01 or later.
%
% This work has the LPPL maintenance status `maintained'.
%
% The Current Maintainer of this work is Niklas Beisert.
%
% This work consists of the files childdoc.dtx and childdoc.ins
% and the derived files childdoc.def and cdocsamp.tex with
% cdocsch1.tex, cdocsch2.tex, cdocsdrf.tex, cdocsfn1.tex, cdocsfn2.tex.
%
%<package>\ifdefined\childdocmain\endinput\fi
%<package>\ProvidesFile{childdoc.def}[2018/12/30 v2.0 child document driver]
%<samplemain>\ProvidesFile{cdocsamp.tex}[2018/12/30 v2.0 sample for childdoc]
%<*driver>
%\ProvidesFile{childdoc.drv}[2018/12/30 v2.0 childdoc reference manual file]
\PassOptionsToClass{10pt,a4paper}{article}
\documentclass{ltxdoc}

\usepackage[margin=35mm]{geometry}
\usepackage{hyperref}
\usepackage{hyperxmp}
\usepackage[usenames]{color}

\hypersetup{colorlinks=true}
\hypersetup{pdfstartview=FitH}
\hypersetup{pdfpagemode=UseNone}
\hypersetup{pdfsource={}}
\hypersetup{pdflang={en-UK}}
\hypersetup{pdfcopyright={Copyright 2017-2018 Niklas Beisert.
  This work may be distributed and/or modified under the
  conditions of the LaTeX Project Public License, either version 1.3
  of this license or (at your option) any later version.}}
\hypersetup{pdflicenseurl={http://www.latex-project.org/lppl.txt}}
\hypersetup{pdfcontactaddress={ETH Zurich, ITP, HIT K,
  Wolfgang-Pauli-Strasse 27}}
\hypersetup{pdfcontactpostcode={8093}}
\hypersetup{pdfcontactcity={Zurich}}
\hypersetup{pdfcontactcountry={Switzerland}}
\hypersetup{pdfcontactemail={nbeisert@itp.phys.ethz.ch}}
\hypersetup{pdfcontacturl={http://people.phys.ethz.ch/\xmptilde nbeisert/}}

\newcommand{\secref}[1]{\hyperref[#1]{section \ref*{#1}}}

\parskip1ex
\parindent0pt
\let\olditemize\itemize
\def\itemize{\olditemize\parskip0pt}

\begin{document}

\title{The \textsf{childdoc} Package}
\hypersetup{pdftitle={The childdoc Package}}
\author{Niklas Beisert\\[2ex]
  Institut f\"ur Theoretische Physik\\
  Eidgen\"ossische Technische Hochschule Z\"urich\\
  Wolfgang-Pauli-Strasse 27, 8093 Z\"urich, Switzerland\\[1ex]
  \href{mailto:nbeisert@itp.phys.ethz.ch}
  {\texttt{nbeisert@itp.phys.ethz.ch}}}
\hypersetup{pdfauthor={Niklas Beisert}}
\hypersetup{pdfsubject={Manual for the LaTeX2e Package childdoc}}
\date{30 December 2018, \textsf{v2.0}}
\maketitle

\begin{abstract}\noindent
\textsf{childdoc} is a \LaTeXe{} package
that enables the direct compilation
of document sections included by |\include|
to individual files.
\end{abstract}

\begingroup
\parskip0ex
\tableofcontents
\endgroup

%%%%%%%%%%%%%%%%%%%%%%%%%%%%%%%%%%%%%%%%%%%%%%%%%%%%%%%%%%%%%%%%%%%%%%%%%%%%%%%%
%%%%%%%%%%%%%%%%%%%%%%%%%%%%%%%%%%%%%%%%%%%%%%%%%%%%%%%%%%%%%%%%%%%%%%%%%%%%%%%%
\section{Introduction}

\LaTeX{} provides a mechanism to structure a large document (such as a book)
into a main file and several child files (containing the chapters)
using the |\include| command.
This mechanism is beneficial for documents
which span hundreds of pages in order to
make the source file(s) more manageable.
Moreover, compilation can be restricted to
selected child files by means of the |\includeonly| command.
The latter feature can be used to reduce the compilation time while editing
(this was significantly more useful in the earlier days of \LaTeX{})
or to generate a smaller document which is easier to navigate.
Another application of |\includeonly| is to generate
documents consisting of selected parts of the complete document.

However, there are a few drawbacks of the plain |\include| mechanism:
\begin{itemize}
\item
The child files cannot be compiled on their own,
they can only be compiled via the main file.
A naive editing environment
(such as a text editor with an option
to have the current file processed by \LaTeX)
may require one to switch to the main file before compiling;
attempting to compile the child file produces errors.
\item
The main file must be modified (each time)
to adjust the |\includeonly| command
to the present needs. This easily leaves the main file in a messy state.
\item
The generated document will always carry the filename
of the main document. This is inconvenient if
several child files are to be compiled and
to be kept for distribution.
\end{itemize}

The present package provides a simple interface
to make child files individually compilable by \LaTeX{}.
Compiling a child file then has the same effect as compiling
the main file with an |\includeonly| command
to select the appropriate child.
Moreover the generated document will carry the name of the child
rather than the main file.
This resolves all three above issues.

This feature is meant to make the editing of books,
thesis documents and lecture notes somewhat more convenient.
However, the package can also be used efficiently for
composing a series of documents (such as exercise sheets)
which are typically distributed individually.
It then assists the author in generating the individual documents
(potentially in different versions)
as well as a document containing the collected series.
Another application is in developing style files
or other kinds of included material
where compilation of the style file could redirect
to a sample or test file.

%%%%%%%%%%%%%%%%%%%%%%%%%%%%%%%%%%%%%%%%%%%%%%%%%%%%%%%%%%%%%%%%%%%%%%%%%%%%%%%%
%%%%%%%%%%%%%%%%%%%%%%%%%%%%%%%%%%%%%%%%%%%%%%%%%%%%%%%%%%%%%%%%%%%%%%%%%%%%%%%%
\section{Usage}

First of all, the package \textsf{childdoc} is \emph{not} a standard
\LaTeXe{} |.sty| style file! Therefore it needs to be invoked in
a non-standard way.

%%%%%%%%%%%%%%%%%%%%%%%%%%%%%%%%%%%%%%%%%%%%%%%%%%%%%%%%%%%%%%%%%%%%%%%%%%%%%%%%
\subsection{Included Files}
\label{sec:include}

%%%%%%%%%%%%%%%%%%%%%%%%%%%%%%%%%%%%%%%%
\DescribeMacro{\childdocmain}
To use the package, add the commands
\begin{center}
\begin{tabular}{l}
|\input{childdoc.def}|\\
|\childdocmain{}|\\
\end{tabular}
\end{center}
at the very top of the main \LaTeX{} file,
in particular \emph{before} the |\documentclass| statement!
The argument of |\childdocmain| should be left empty
(but it must be present).

%%%%%%%%%%%%%%%%%%%%%%%%%%%%%%%%%%%%%%%%
\DescribeMacro{\childdocof}
Furthermore, add the commands
\begin{center}
\begin{tabular}{l}
|\input{childdoc.def}|\\
|\childdocof{|\textit{main}|}|\\
\end{tabular}
\end{center}
at the top of every child file \textit{child}
which is included by |\include{|\textit{child}|}|
from within the main file
(or at least for those files to be compiled individually).
The argument \textit{main} must be the filename of the main file.

There are a couple of
considerations in setting up the main and child documents:

%%%%%%%%%%%%%%%%%%%%%%%%%%%%%%%%%%%%%%%%
\paragraph{Restrictions.}

Please note the following restrictions:
\begin{itemize}
\item
|\childdocmain| must be called with one argument \textit{main}
to ensure compatibility with earlier version of the package.
It must either be empty (|\childdocmain{}|)
or precisely match the filename of the main file in which it is specified.
See \secref{sec:detection} for further information.
\item
The filename \textit{main} must be specified without the |.tex| extension.
\item
The filename \textit{main} is case sensitive
(even in case-insensitive file systems)
due to internal string comparison.
\item
The argument \textit{main} should be fully expanded, it cannot be a macro.
\item
Subdirectories and special characters should be avoided in filenames.
\item
The command |\childdocmain{|\textit{main}|}| must be followed by a whitespace.
It should not be followed immediately by another command
or by a comment mark `|%|'.
This is because the \TeX{} parser reads the token immediately following
the argument of |\childdocmain| and puts it
at the beginning of every child section;
however, a white\-space is ignored.
\end{itemize}

%%%%%%%%%%%%%%%%%%%%%%%%%%%%%%%%%%%%%%%%
\paragraph{Content of Main File.}

It is advisable to place all content in the child files included by |\include|.
Any output contained in the main file will appear in all child documents
unless suppressed manually;
it cannot be suppressed automatically by the |\includeonly| directive
and thus should normally be avoided.
A method to include some content in the main file
by means of conditional processing is described in \secref{sec:conditional}.

%%%%%%%%%%%%%%%%%%%%%%%%%%%%%%%%%%%%%%%%
\paragraph{Page Numbering.}

When only a part of the document is compiled,
the appropriate numbering of pages
(as well as other status parameters)
is determined from the |.aux| files.
The latter contain information from previous passes.
However this information needs to propagate through
all intermediate child documents.
Therefore the page numbering in child documents may well
be inconsistent until the complete document is compiled at least once.

A useful (if unconventional) way to always ensure a consistent
page numbering is to restart the numbering in each child document
and denote the pages by `\textit{child}|.|\textit{page}'
where \textit{child} represents the chapter/section number of the child file.
This can be achieved by the command
|\numberwithin{page}{|\textit{child}|}|
of the \textsf{amsmath} package
where \textit{child} can be |chapter| or |section|
depending on the chosen structuring.
Alternatively, one can modify the macro |\thepage| appropriately
and reset the counter |page| at the start of each child file.

%%%%%%%%%%%%%%%%%%%%%%%%%%%%%%%%%%%%%%%%%%%%%%%%%%%%%%%%%%%%%%%%%%%%%%%%%%%%%%%%
\subsection{Conditional Processing}
\label{sec:conditional}

The package provides a mechanism to compile different versions
of a document. To customise the versions further some conditional processing
can come in handy to distinguish which version is being compiled.
The package provides two macros to describe the compilation context:

%%%%%%%%%%%%%%%%%%%%%%%%%%%%%%%%%%%%%%%%
\DescribeMacro{\ifchilddoc}
The conditional |\ifchilddoc| distinguishes between the compilation of
child documents and the main document:
%
\begin{center}
|\ifchilddoc |\textit{child-code}| |[|\||else |\textit{main-code}]| \||fi|
\end{center}

%%%%%%%%%%%%%%%%%%%%%%%%%%%%%%%%%%%%%%%%
\DescribeMacro{\childdocname}
\DescribeMacro{\childdocjob}
The macro |\childdocname| contains the filename (without extension)
of the main or child file being processed.
Note that |\childdocjob| will always contain the name of the main file.

%%%%%%%%%%%%%%%%%%%%%%%%%%%%%%%%%%%%%%%%
\paragraph{Title Page.}

Conditional processing can be used to include a title or banner page
in the main document when proper precautions are taken.
Importantly, the code in the main file should ensure that the page counter
(as well as other status parameters which are stored in the |.aux| files)
takes the same value after the conditional processing.
Otherwise the page numbers may take divergent values
depending on which part is compiled.

For example, a title page could be declared by:
%
\begin{center}
\begin{tabular}{l}
|\ifchilddoc\||else|\\
|\addtocounter{page}{-1}|\\
\textit{code for title page}\\
|\newpage|\\
|\||fi|
\end{tabular}
\end{center}
%
A banner page for the child documents can be generated by:
%
\begin{center}
\begin{tabular}{l}
|\ifchilddoc|\\
|\addtocounter{page}{-1}|\\
\textit{code for banner page}\\
|\newpage|\\
|\||fi|
\end{tabular}
\end{center}
%
Here one could write a message such as:
\begin{center}
|This is the part \childdocname{} of \childdocjob{}.|
\end{center}

%%%%%%%%%%%%%%%%%%%%%%%%%%%%%%%%%%%%%%%%%%%%%%%%%%%%%%%%%%%%%%%%%%%%%%%%%%%%%%%%
\subsection{Flags}
\label{sec:flags}

The package makes it easy to generate different versions
of the main or child documents.
To this end compilation flags can be defined
and assigned different default values.
They will be particularly useful in conjunction
with the forwarding mechanism described in \secref{sec:forward}.

For example, it may be useful to have a flag |\version|
which can be set to |draft| or |final|.
The document source will contain some conditional code
depending on the value of |\version|.
Suppose further, the flag should default to |final| for the main file
and to |draft| for child files
which is a natural assignment for editing the document.
This is achieved by placing the following code
in the preamble of the main document
(below the |\childdocmain| directive):
%
\begin{center}
\begin{tabular}{l}
|\ifchilddoc|\\
|\providecommand{\version}{draft}|\\
|\||else|\\
|\providecommand{\version}{final}|\\
|\||fi|
\end{tabular}
\end{center}
%
The definition by |\providecommand| makes sure
that previous definitions are not overwritten.
Further statements |\providecommand{\version}{...}|
can thus be added before the above code to override it.

For the main file, one might add a line
(between |\childdocmain| and the above block)
%
\begin{center}
|%\ifchilddoc\||else\providecommand{\version}{draft}\||fi|
\end{center}
%
which can be uncommented to produce a draft version.
Likewise one can add a line to the very top of a child file
(above the |\childdocof{|\textit{main}|}| directive)
%
\begin{center}
|%\providecommand{\version}{final}|
\end{center}
%
which can be uncommented to produce the final version of this child document.

%%%%%%%%%%%%%%%%%%%%%%%%%%%%%%%%%%%%%%%%%%%%%%%%%%%%%%%%%%%%%%%%%%%%%%%%%%%%%%%%
\subsection{Forwarding}
\label{sec:forward}

Different versions of the main or child documents
using compilation flags as described in \secref{sec:flags}
can be (permanently) stored in different files
for convenient compilation, viewing and distribution.
To this end, the package defines a command
to pass on compilation to a different file:

%%%%%%%%%%%%%%%%%%%%%%%%%%%%%%%%%%%%%%%%
\DescribeMacro{\childdocforward}
The command |\childdocforward| redirects processing to
another source file:
%
\begin{center}
\begin{tabular}{l}
|\input{childdoc.def}|\\
|\childdocforward[|\textit{main}|]{|\textit{dest}|}|\\
\end{tabular}
\end{center}
%
The argument \textit{dest} is the destination file
(without extension).
It should be the main file or one of the child files.
Note that further \textsf{childdoc} directives
such as |\childdocof| and |\childdocforward|
in the indicated file will be processed in this form.
The optional argument \textit{main}
passes on directly to the main file \textit{main}
while pretending to compile the child \textit{dest}.
This form behaves as if \textit{dest}
issues |\childdocof{|\textit{main}|}| right away,
and no further \textsf{childdoc} directives will be processed.

%%%%%%%%%%%%%%%%%%%%%%%%%%%%%%%%%%%%%%%%
\DescribeMacro{\...prefix}
In the alternative form |\childdocforwardprefix|,
%
\begin{center}
\begin{tabular}{l}
|\input{childdoc.def}|\\
|\childdocforwardprefix[|\textit{main}|]{|\textit{prefix}|}{|\textit{dest}|}|
\end{tabular}
\end{center}
%
the destination file is determined by a pattern
depending on the current file:
To make this work, the current file must be called
`{\textit{prefix}\hspace{0.2em}\textit{suffix}}'
with \textit{prefix} matching precisely the argument.
Processing is then passed on to the file
`{\textit{dest}\hspace{0.2em}\textit{suffix}}'.
Surely, the same effect is achieved by
directly specifying the
argument `{\textit{dest}\hspace{0.2em}\textit{suffix}}'
in the first form.
However, that requires to set up a different file
for each child. With the alternative form of the command
all these files can have exactly the same content
which simplifies setting them up and maintaining them.

For example, the following file |draft.tex|
with a compilation flag |\version| as described in \secref{sec:flags}
compiles the main document as a draft:
%
\begin{center}
\begin{tabular}{l}
|\def\version{draft}|\\
|\input{childdoc.def}|\\
|\childdocforward{|\textit{main}|}|
\end{tabular}
\end{center}
%
Likewise, the following files |final|\textit{nn}|.tex|
compile the final version of the child document
|child|\textit{nn}|.tex|:
%
\begin{center}
\begin{tabular}{l}
|\def\version{final}|\\
|\input{childdoc.def}|\\
|\childdocforwardprefix{final}{child}|
\end{tabular}
\end{center}
%

Note that when several versions of a main file and/or of each child file
are to be generated, it may be convenient to set up a |Makefile| or
shell script to automatise the process.

%%%%%%%%%%%%%%%%%%%%%%%%%%%%%%%%%%%%%%%%%%%%%%%%%%%%%%%%%%%%%%%%%%%%%%%%%%%%%%%%
\subsection{Command Line Processing}
\label{sec:commandline}

The effect of redirection files can also be achieved by invoking
the \LaTeX{} compiler with a more elaborate command line.
Most conveniently this should be done as part
of a shell script or a |Makefile|.

When using \textsf{childdoc} in the main file, the following
command lines effectively perform a redirection
(note that depending on the shell being used,
backslashes may have to be doubled: `|\|' $\to$ `|\\|'):
%
\begin{center}
|... -jobname "|\textit{target}|" |\\|"|[\textit{flags}]%
|\input{childdoc.def}\childdocforward[|\textit{main}|]{|\textit{dest}|}"|
\end{center}
%
Here \textit{target} is the name of the output file,
\textit{main} is the name of the main file
and \textit{dest} is the name of the main or child file to be processed
(all filenames without extensions).
The optional argument \textit{main} can be omitted
if \textit{main} matches \textit{dest}.
Optionally, compilation \textit{flags} can be defined via |\def| commands.
This command line makes the \TeX{} engine believe
it is compiling the file \textit{target}
whose content is specified as the latter parameter.
The provided code then forwards the processing to
\textit{main} or \textit{dest} as described in \secref{sec:forward}.

%%%%%%%%%%%%%%%%%%%%%%%%%%%%%%%%%%%%%%%%%%%%%%%%%%%%%%%%%%%%%%%%%%%%%%%%%%%%%%%%
\subsection{Include by Input}
\label{sec:input}

Including child documents by |\include| has some restrictions by design.
Most notably, the content of a child document always occupies
its own set of pages; pages cannot be shared between child documents.
Usually, this behaviour makes perfect sense
because each child document contain an essential part of the document.
However, in some situations it may be desirable to compose
a document from a collection of parts
without having mandatory page breaks between then.
For this case, the package
provides a mechanism to include parts
by |\input| which can also be processed individually.
However, by construction this mechanism
requires manual handling of the content to be output.

%%%%%%%%%%%%%%%%%%%%%%%%%%%%%%%%%%%%%%%%
\DescribeMacro{\ifchilddocmanual}
The main file should be prepared as usual, see \secref{sec:include}.
However, the document body must make a distinction
between processing of an individual part and of the main document, e.g.:
%
\begin{center}
\begin{tabular}{l}
|\ifchilddocmanual|\\
|\input{\childdocname}|\\
|\||else|\\
\textit{document body with }|\input{|\textit{part}|}|\\
|\||fi|
\end{tabular}
\end{center}
%
The conditional |\ifchilddocmanual| is true whenever
a part to be included by |\input| is being compiled,
and the name of the part is stored in |\childdocname|.

%%%%%%%%%%%%%%%%%%%%%%%%%%%%%%%%%%%%%%%%
\DescribeMacro{\childdocby}
Each part to be included by |\input| should start with:
%
\begin{center}
\begin{tabular}{l}
|\input{childdoc.def}|\\
|\childdocby{|\textit{main}|}|\\
\end{tabular}
\end{center}
%
The directive |\childdocby| is similar to |\childdocof|
described in \secref{sec:include},
but the subsequent selection of content must be done manually.
To that end, both |\ifchilddoc| and |\ifchilddocmanual|
will be true upon processing of a part,
and the name of the part is stored in |\childdocname|.
Note that |\jobname| will be set to the filename of the current part
so that each part receives an individual |.aux| file
that does not interfere with the |.aux| file(s) of the main document.
This behaviour can be altered by the alternative form
|\childdocby[*]{|\textit{main}|}| (with a non-empty optional argument)
which uses the |.aux| file of the main document
by setting |\jobname| to \textit{main}.

%%%%%%%%%%%%%%%%%%%%%%%%%%%%%%%%%%%%%%%%%%%%%%%%%%%%%%%%%%%%%%%%%%%%%%%%%%%%%%%%
\subsection{Driver Development}
\label{sec:driver}

The \textsf{childdoc} mechanism can also be use for the development
of definition files such as \LaTeX{} styles or classes.
This case differs from the above setup with multiple parts
included by |\include| in that no |\includeonly| should be invoked.
This can be achieved by starting the include file
(before |\ProvidesPackage|) with:
%
\begin{center}
\begin{tabular}{l}
|\input{childdoc.def}|\\
|\childdocforward{|\textit{main}|}|\\
\end{tabular}
\end{center}
%
or alternatively with:
%
\begin{center}
\begin{tabular}{l}
|\input{childdoc.def}|\\
|\childdocby{|\textit{main}|}|\\
\end{tabular}
\end{center}
%
Both forms have slightly different effects as described above.
The main file is prepared as usual, see \secref{sec:include}.

%%%%%%%%%%%%%%%%%%%%%%%%%%%%%%%%%%%%%%%%%%%%%%%%%%%%%%%%%%%%%%%%%%%%%%%%%%%%%%%%
\subsection{Legacy Detection}
\label{sec:detection}

The directive |\childdocmain| in the main file can detect
whether the complete document or merely a child is to be compiled
even without using the directive |\childdocof|.
This method is deprecated because it is less robust
and there is no compelling reason to use it;
it is merely provided for backward compatibility
and it may be removed in future versions.

If the detection mechanism is to be used,
it is mandatory to correctly specify
the filename of the main file as the argument of |\childdocmain|:
%
\begin{center}
\begin{tabular}{l}
|\input{childdoc.def}|\\
|\childdocmain{|\textit{main}|}|\\
\end{tabular}
\end{center}
%
If |\jobname| does not match the argument \textit{main} of |\childdocmain|,
it is assumed that |\jobname| points to the child file to be compiled.
When using |\childdocmain| with the main file specified as argument,
it suffices to start a child file
with just |\input{|\textit{main}|}|
without loading of the package and using |\childdocof|.
If instead all processing is done
with the appropriate \textsf{childdoc} directives,
the argument of \textit{main} of |\childdocmain| can be empty.

An alternative version of the command line processing described
in \secref{sec:commandline} using the detection mechanism reads:
%
\begin{center}
|... -jobname "|\textit{target}|" "|[\textit{flags}]%
[|\def\jobname{|\textit{dest}|}|]|\input{|\textit{main}|}"|
\end{center}

%%%%%%%%%%%%%%%%%%%%%%%%%%%%%%%%%%%%%%%%%%%%%%%%%%%%%%%%%%%%%%%%%%%%%%%%%%%%%%%%
\subsection{Manual Code}
\label{sec:manual}

In case one cannot be certain whether the definitions file |childdoc.def|
is installed on the target \TeX{} distribution
and one prefers not to ship it,
it is conceivable to paste a few relevant commands into the sources.

To that end, drop all statements |\input{childdoc.def}|
and perform the replacements as outlined below.
Instead of |\childdocmain{|\textit{main}|}| add the following code
to the top of the main file:
%
\begin{center}
\begin{tabular}{l}
|\||ifdefined\childdocname\endinput\||fi\newif\ifchilddoc|\\
|\edef\childdocname{\scantokens\expandafter{\jobname\noexpand}}|\\
|\def\childdocmain{|\textit{main}|}\||ifx\childdocmain\childdocname\||else|\\
|\childdoctrue\includeonly{\childdocname}\let\jobname\childdocmain\||fi|\\
\end{tabular}
\end{center}
%
Instead of |\childdocof{|\textit{main}|}| just include the main file
at the top of each child file:
%
\begin{center}
|\input{|\textit{main}|}|
\end{center}
%
A simple redirection |\childdocforward{|\textit{dest}|}| is achieved by:
%
\begin{center}
|\def\jobname{|\textit{dest}|}\input{\jobname}|
\end{center}
%
The redirection with prefix
|\childdocforwardprefix[|\textit{prefix}|]{|\textit{dest}|}|
is accomplished by:
%
\begin{center}
\begin{tabular}{l}
|{\edef\jobname{\scantokens\expandafter{\jobname\noexpand}}|\\
|\def\redirectjob |\textit{prefix}|#1~~~{\gdef\jobname{|\textit{dest}|#1}}|\\
|\expandafter\redirectjob\jobname~~~}\input{\jobname}|
\end{tabular}
\end{center}

In an alternative approach,
child documents can be compiled by a specific command line
without additional code or specific definitions:
%
\begin{center}
|... -jobname "|\textit{target}|" "|[\textit{flags}]%
|\includeonly{|\textit{dest}|}\input{|\textit{main}|}"|
\end{center}
%

%%%%%%%%%%%%%%%%%%%%%%%%%%%%%%%%%%%%%%%%%%%%%%%%%%%%%%%%%%%%%%%%%%%%%%%%%%%%%%%%
%%%%%%%%%%%%%%%%%%%%%%%%%%%%%%%%%%%%%%%%%%%%%%%%%%%%%%%%%%%%%%%%%%%%%%%%%%%%%%%%
\section{Information}

%%%%%%%%%%%%%%%%%%%%%%%%%%%%%%%%%%%%%%%%%%%%%%%%%%%%%%%%%%%%%%%%%%%%%%%%%%%%%%%%
\subsection{Copyright}

Copyright \copyright{} 2017--2018 Niklas Beisert

This work may be distributed and/or modified under the
conditions of the \LaTeX{} Project Public License, either version 1.3
of this license or (at your option) any later version.
The latest version of this license is in
  \url{http://www.latex-project.org/lppl.txt}
and version 1.3 or later is part of all distributions of \LaTeX{}
version 2005/12/01 or later.

This work has the LPPL maintenance status `maintained'.

The Current Maintainer of this work is Niklas Beisert.

This work consists of the files |README.txt|, |childdoc.ins| and |childdoc.dtx|
as well as the derived files |childdoc.def|, |cdocsamp.tex|
with |cdocsch1.tex|, |cdocsch2.tex|, |cdocspt3.tex|, |cdocspt4.tex|,
|cdocsdrf.tex|, |cdocsfn1.tex|, |cdocsfn2.tex|
as well as |childdoc.pdf|.

%%%%%%%%%%%%%%%%%%%%%%%%%%%%%%%%%%%%%%%%%%%%%%%%%%%%%%%%%%%%%%%%%%%%%%%%%%%%%%%%
\subsection{Files and Installation}

The package consists of the files:
%
\begin{center}
\begin{tabular}{ll}
    |README.txt|   & readme file \\
    |childdoc.ins| & installation file \\
    |childdoc.dtx| & source file \\
    |childdoc.def| & definition file \\
    |cdocsamp.tex| & sample main file \\
    |cdocsch1.tex| & sample include file \\
    |cdocsch2.tex| & sample include file \\
    |cdocspt3.tex| & sample part file \\
    |cdocspt4.tex| & sample part file \\
    |cdocsdrf.tex| & sample redirection file \\
    |cdocsfn1.tex| & sample redirection file \\
    |cdocsfn2.tex| & sample redirection file \\
    |childdoc.pdf| & manual
\end{tabular}
\end{center}
%
The distribution consists of the files
|README.txt|, |childdoc.ins| and |childdoc.dtx|.
%
\begin{itemize}
\item
Run (pdf)\LaTeX{} on |childdoc.dtx|
to compile the manual |childdoc.pdf| (this file).
\item
Run \LaTeX{} on |childdoc.ins| to create the definitions file |childdoc.def|
and the sample |cdocsamp.tex| with include files
|cdocsch1.tex|, |cdocsch2.tex|, |cdocspt3.tex|, |cdocspt4.tex|,
|cdocsdrf.tex|, |cdocsfn1.tex|, |cdocsfn2.tex|.
Then copy the file |childdoc.def| to an appropriate directory of your \LaTeX{}
distribution, e.g.\ \textit{texmf-root}|/tex/latex/childdoc|.
\end{itemize}

%%%%%%%%%%%%%%%%%%%%%%%%%%%%%%%%%%%%%%%%%%%%%%%%%%%%%%%%%%%%%%%%%%%%%%%%%%%%%%%%
\subsection{Related CTAN Packages}

There are several other packages which offer a similar functionality:
%
\begin{itemize}
\item
The packages
\href{http://ctan.org/pkg/docmute}{\textsf{docmute}},
\href{http://ctan.org/pkg/includex}{\textsf{includex}} and
\href{http://ctan.org/pkg/standalone}{\textsf{standalone}}
provide commands to include only the document body of
a child file thus allowing both files to be compiled individually.
\item
The packages \href{http://ctan.org/pkg/subdocs}{\textsf{subdocs}}
and \href{http://ctan.org/pkg/subfiles}{\textsf{subfiles}}
provide structures in which the main and child documents can be
encapsulated and allowing them to be compiled individually.
The inclusion mechanism is different from the conventional |\include|.
\item
The package \href{http://ctan.org/pkg/combine}{\textsf{combine}}
is an elaborate solution to combine several documents into one.
\end{itemize}
%
See also the CTAN topic \href{http://ctan.org/topic/subdocs}{\textsf{subdocs}}
for further related packages.
The present package differs from the above solutions in that
a document structure constructed with the conventional |\include| mechanism
just needs two extra commands at the top of every file
such that all constituent files can be compiled individually.

%%%%%%%%%%%%%%%%%%%%%%%%%%%%%%%%%%%%%%%%%%%%%%%%%%%%%%%%%%%%%%%%%%%%%%%%%%%%%%%%
%\subsection{Feature Suggestions}
%
%The following is a list of features which may be useful for future
%versions of this package:
%%
%\begin{itemize}
%\item
%\ldots
%\end{itemize}

%%%%%%%%%%%%%%%%%%%%%%%%%%%%%%%%%%%%%%%%%%%%%%%%%%%%%%%%%%%%%%%%%%%%%%%%%%%%%%%%
\subsection{Revision History}

%%%%%%%%%%%%%%%%%%%%%%%%%%%%%%%%%%%%%%%%
\paragraph{v2.0:} 2018/12/30

\begin{itemize}
\item
immediate forward processing
\item
added |\childdocby| mechanism
\item
manual restructured
\end{itemize}

%%%%%%%%%%%%%%%%%%%%%%%%%%%%%%%%%%%%%%%%
\paragraph{v1.6:} 2018/01/17

\begin{itemize}
\item
application for development of include files
\item
corrections to manual
\end{itemize}

%%%%%%%%%%%%%%%%%%%%%%%%%%%%%%%%%%%%%%%%
\paragraph{v1.5:} 2017/05/21

\begin{itemize}
\item
more complete structuring introduced
\item
|\childdocof| introduced
\item
|\childdoc| renamed to |\childdocmain|
\item
|\childredirect| renamed to |\childdocforward| and |\childdocforwardprefix|
and functionality expanded
\end{itemize}

%%%%%%%%%%%%%%%%%%%%%%%%%%%%%%%%%%%%%%%%
\paragraph{v1.0:} 2017/04/27

\begin{itemize}
\item
manual and install package
\item
first version published on CTAN
\end{itemize}

%%%%%%%%%%%%%%%%%%%%%%%%%%%%%%%%%%%%%%%%
\paragraph{v0.6:} 2017/04/26

\begin{itemize}
\item
redirection mechanism added
\end{itemize}

%%%%%%%%%%%%%%%%%%%%%%%%%%%%%%%%%%%%%%%%
\paragraph{v0.5:} 2017/04/26

\begin{itemize}
\item
functionality in definition file
\end{itemize}


%%%%%%%%%%%%%%%%%%%%%%%%%%%%%%%%%%%%%%%%%%%%%%%%%%%%%%%%%%%%%%%%%%%%%%%%%%%%%%%%
%%%%%%%%%%%%%%%%%%%%%%%%%%%%%%%%%%%%%%%%%%%%%%%%%%%%%%%%%%%%%%%%%%%%%%%%%%%%%%%%
%%%%%%%%%%%%%%%%%%%%%%%%%%%%%%%%%%%%%%%%%%%%%%%%%%%%%%%%%%%%%%%%%%%%%%%%%%%%%%%%
\appendix

\settowidth\MacroIndent{\rmfamily\scriptsize 000\ }

 \DocInput{childdoc.dtx}

\end{document}
%</driver>
% \fi
%
% %%%%%%%%%%%%%%%%%%%%%%%%%%%%%%%%%%%%%%%%%%%%%%%%%%%%%%%%%%%%%%%%%%%%%%%%%%%%%%
% %%%%%%%%%%%%%%%%%%%%%%%%%%%%%%%%%%%%%%%%%%%%%%%%%%%%%%%%%%%%%%%%%%%%%%%%%%%%%%
% \section{Sample}
%\iffalse
%<*samplemain>
%\fi
%
% The following presents a sample document
% with two chapters, two parts, a title page,
% a compile flag as well as three forwarding files to set the flag.
% It consists of eight |.tex| files:
% \begin{center}
% \begin{tabular}{ll}
% |cdocsamp.tex|&main file\\
% |cdocsch1.tex|&include file for chapter 1\\
% |cdocsch2.tex|&include file for chapter 2\\
% |cdocspt3.tex|&include file for part 3\\
% |cdocspt4.tex|&include file for part 4\\
% |cdocsdrf.tex|&forwarding file for main file in draft mode\\
% |cdocsfi1.tex|&forwarding file for final version of chapter 1\\
% |cdocsfi2.tex|&forwarding file for final version of chapter 2\\
% \end{tabular}
% \end{center}
% Each of the eight files can be compiled directly by the \LaTeX{} compiler.
%
% %%%%%%%%%%%%%%%%%%%%%%%%%%%%%%%%%%%%%%
% \paragraph{Main File.}
%
% The main file is called |cdocsamp.tex|.
%
% Load the \textsf{childdoc} definitions and
% declare the filename for the main document:
%    \begin{macrocode}
\input{childdoc.def}
\childdocmain{}
%    \end{macrocode}

% Optional override for |\version| flag:
%    \begin{macrocode}
%%\ifchilddoc\else\providecommand{\version}{draft}\fi
%    \end{macrocode}

% Define the default values for the |\version| flag
% (|final| for the main file and |draft| for childs):
%    \begin{macrocode}
\ifchilddoc
\providecommand{\version}{draft}
\else
\providecommand{\version}{final}
\fi
%    \end{macrocode}

% Load the standard document class:
%    \begin{macrocode}
\documentclass[12pt]{article}
%    \end{macrocode}

% Start the document body:
%    \begin{macrocode}
\begin{document}
%    \end{macrocode}

% Declare a title page.
% Print title, part of document being processed and version flag:
%    \begin{macrocode}
\addtocounter{page}{-1}
\begin{center}
{\LARGE\bfseries{}childdoc example\par}
\vspace{1cm}
\ifchilddoc
\ifchilddocmanual part\else chapter\fi:
`\childdocname' of `\childdocjob'\par
\else
main document: `\childdocjob'\par
\fi
version: \version\par
\end{center}
\newpage
%    \end{macrocode}

% Manually include selected file,
% otherwise process as usual:
%    \begin{macrocode}
\ifchilddocmanual
\section*{part `\childdocname'}
\input{\childdocname}
\else
%    \end{macrocode}

% Include the two chapters:
%    \begin{macrocode}
\include{cdocsch1}
\include{cdocsch2}
%    \end{macrocode}

% Include the two parts unless only chapters should be displayed:
%    \begin{macrocode}
\ifchilddoc\else
\section{part three}
\input{cdocspt3}
\section{part four}
\input{cdocspt4}
\fi
%    \end{macrocode}

% Process as usual until here:
%    \begin{macrocode}
\fi
%    \end{macrocode}

% End of document body:
%    \begin{macrocode}
\end{document}
%    \end{macrocode}
%\iffalse
%</samplemain>
%\fi
%
% %%%%%%%%%%%%%%%%%%%%%%%%%%%%%%%%%%%%%%
% \paragraph{Chapter Include Files.}
%
% The include files are called |cdocsch1.tex| and |cdocsch2.tex|.
%
%\iffalse
%<*samplechap1|samplechap2>
%\fi

% Optional override for |\version| flag:
%    \begin{macrocode}
%%\providecommand{\version}{final}
%    \end{macrocode}

% Include the main document:
%    \begin{macrocode}
\input{childdoc.def}
\childdocof{cdocsamp}
%    \end{macrocode}

%\iffalse
%</samplechap1|samplechap2>
%\fi
%
%\iffalse
%<*samplechap1>
%\fi
% Some text for chapter 1:
%    \begin{macrocode}
\section{one}
some text in chapter one
%    \end{macrocode}

%\iffalse
%</samplechap1>
%\fi
% Some text for chapter 2:
%\iffalse
%<*samplechap2>
%\fi
%    \begin{macrocode}
\section{two}
more text in chapter two
%    \end{macrocode}

%\iffalse
%</samplechap2>
%\fi
%
% %%%%%%%%%%%%%%%%%%%%%%%%%%%%%%%%%%%%%%
% \paragraph{Part Include Files.}
%
% The include files are called |cdocspt3.tex| and |cdocspt4.tex|.
%
%\iffalse
%<*samplepart3|samplepart4>
%\fi

% Optional override for |\version| flag:
%    \begin{macrocode}
%%\providecommand{\version}{final}
%    \end{macrocode}

% Include the main document:
%    \begin{macrocode}
\input{childdoc.def}
\childdocby{cdocsamp}
%    \end{macrocode}

%\iffalse
%</samplepart3|samplepart4>
%\fi
%
%\iffalse
%<*samplepart3>
%\fi
% Some text for part 3:
%    \begin{macrocode}
some text in part three
%    \end{macrocode}

%\iffalse
%</samplepart3>
%\fi
% Some text for part 4:
%\iffalse
%<*samplepart4>
%\fi
%    \begin{macrocode}
more text in part four
%    \end{macrocode}

%\iffalse
%</samplepart4>
%\fi
%
% %%%%%%%%%%%%%%%%%%%%%%%%%%%%%%%%%%%%%%
% \paragraph{Forwarding for a Complete Draft.}
%
% The following forwarding file |cdocsdrf.tex|
% compiles the main document in draft mode:
%\iffalse
%<*sampledraft>
%\fi
%    \begin{macrocode}
\def\version{draft}
\input{childdoc.def}
\childdocforward{cdocsamp}
%    \end{macrocode}

%\iffalse
%</sampledraft>
%\fi
%
% %%%%%%%%%%%%%%%%%%%%%%%%%%%%%%%%%%%%%%
% \paragraph{Forwarding for Final Version of the Chapters.}
%
% The following forwarding files |cdocsfn1.tex| and |cdocsfn2.tex|
% (with identical content)
% compile the final versions of the child documents
% |cdocsch1.tex| and |cdocsch2.tex|, respectively:
%\iffalse
%<*samplefinal>
%\fi
%    \begin{macrocode}
\def\version{final}
\input{childdoc.def}
\childdocforwardprefix[cdocsamp]{cdocsfn}{cdocsch}
%    \end{macrocode}

%\iffalse
%</samplefinal>
%\fi
%
% %%%%%%%%%%%%%%%%%%%%%%%%%%%%%%%%%%%%%%
% \paragraph{Command Line Processing.}
%
% The following three command lines generate the output files
% |cdocscld|, |cdocscl1| and |cdocscl2|
% which should be identical to
% |cdocsdrf|, |cdocsch1| and |cdocsfn2|, respectively:
% \begin{center}
% \begin{tabular}{l}
% |latex -jobname cdocscld \|\\
% |  "\def\version{draft}\input{childdoc.def}\childdocforward{cdocsamp}"|\\
% |latex -jobname cdocscl1 \|\\
% |  "\input{childdoc.def}\childdocforward[cdocsamp]{cdocsch1}"|\\
% |latex -jobname cdocscl2 \|\\
% |  "\def\version{final}\input{childdoc.def}\childdocforward{cdocsch2}"|
% \end{tabular}
% \end{center}
% Note that the trailing backslash on each first line
% merely continues the input to the second line
% (for convenient cut ant paste).
% Furthermore, the command |latex| can be replaced by any
% of its alternative versions such as |pdflatex|.
%
% %%%%%%%%%%%%%%%%%%%%%%%%%%%%%%%%%%%%%%%%%%%%%%%%%%%%%%%%%%%%%%%%%%%%%%%%%%%%%%
% %%%%%%%%%%%%%%%%%%%%%%%%%%%%%%%%%%%%%%%%%%%%%%%%%%%%%%%%%%%%%%%%%%%%%%%%%%%%%%
% \section{Implementation}
%\iffalse
%<*package>
%\fi
%
% This section describes the definitions file |childdoc.def|.

% The definitions cannot be loaded using |\usepackage| or |\RequirePackage|
% which has a mechanism to prevent loading a style file more than once.
% When loading the definitions by means of |\input|
% multiple instances have to be prevented manually:
%\iffalse
%This code needs to be before the `\ProvidesFile' directive
%which is defined at the beginning of this file.
%Therefore it is also placed there and commented out here.
%</package>
%<*discard>
%\fi
%    \begin{macrocode}
\ifdefined\childdocmain\endinput\fi
%    \end{macrocode}
%\iffalse
%</discard>
%<*package>
%\fi
%
% \macro{\ifchilddoc}
% \macro{\ifchilddocmanual}
% The conditional |\ifchilddoc| tells whether a
% child (true) or main (false) document is being compiled.
% The conditional |\ifchilddocmanual| tells whether
% the |\includeonly| mechanism is used (false) or
% the selection of child files must be performed manually (true).
% The definitions initialise to false:
%    \begin{macrocode}
\newif\ifchilddoc
\newif\ifchilddocmanual
%    \end{macrocode}

% \macro{\childdocname}
% \macro{\childdocjob}
% The macro |\childdocname| stores the name of the main document
% to be compiled. The macro |\childdocjob| stores the name of
% the document on which the \LaTeX{} compiler was originally invoked.
% The content of |\jobname| cannot be compared
% to filenames specified in the source due to different catcodes.
% The following code rescans |\jobname|, stores the result
% in |\childdocname| and saves a copy in |\childdocjob|:
%    \begin{macrocode}
\edef\childdocname{\scantokens\expandafter{\jobname\noexpand}}
\let\childdocjob\childdocname
%    \end{macrocode}

% \macro{\childdocdisable}
% The macro |\childdocdisable| prevents the main file
% from being processed more than once.
% At this stage, the main document command |\childdocmain|
% is assumed to be called once again where it should do nothing.
% Any subsequent call to it should prevent
% a secondary processing of the main document
% It overwrites the forwarding commands
% |\childdocof| and |\childdocforward|
% with empty macros to prevent further inclusions of the main document:
%    \begin{macrocode}
\newcommand{\childdocdisable}
{
  \renewcommand{\childdocmain}[1]{\renewcommand{\childdocmain}[1]{\endinput}}
  \renewcommand{\childdocof}[1]{}
  \renewcommand{\childdocby}[2][]{}
  \renewcommand{\childdocforward}[2][]{}
  \renewcommand{\childdocdisable}{}
}
%    \end{macrocode}

% \macro{\childdocmain}
% The macro |\childdocmain| is to be called at the top of the main file
% with nothing or the main filename (without extension) as argument.
% First, it breaks loops.
% If the argument is not empty and does not match |\childdocname|
% (which is set by the first inclusion of |childdoc.def|),
% |\ifchilddoc| is set to true, |\includeonly| is applied to the child file
% and |\jobname| is set to the main file
% (for proper handling of |.aux| files):
%    \begin{macrocode}
\newcommand{\childdocmain}[1]
{
  \childdocdisable\childdocmain{}
  \if?#1?\else
    \begingroup
      \def\childdoctmp{#1}
      \ifx\childdoctmp\childdocname
        \def\childdoctmp{}
      \else
        \def\childdoctmp
        {
          \childdoctrue
          \includeonly{\childdocname}
          \def\childdocjob{#1}
          \def\jobname{#1}
        }
      \fi
      \expandafter
    \endgroup
    \childdoctmp
  \fi
}
%    \end{macrocode}

% \macro{\childdocof}
% The command |\childdocof| redirects
% compilation to the main file |#1|.
%    \begin{macrocode}
\newcommand{\childdocof}[1]
{
  \childdocdisable
  \childdoctrue
  \includeonly{\childdocname}
  \def\jobname{#1}
  \def\childdocjob{#1}
  \input{#1}
}
%    \end{macrocode}

% \macro{\childdocby}
% The command |\childdocby| ....
%    \begin{macrocode}
\newcommand{\childdocby}[2][]
{
  \childdocdisable
  \childdoctrue
  \childdocmanualtrue
  \if?#1?\else
    \def\jobname{#2}
  \fi
  \def\childdocjob{#2}
  \input{#2}
  \endinput
}
%    \end{macrocode}

% \macro{\childdocforward}
% The command |\childdocforward| redirects
% compilation to the main file or
% (if the optional argument is given) a child file.
% Parameters are set as if the main file
% or a child file starting with |\childdocof| was compiled.
% Then compilation is handed over to the main file:
%    \begin{macrocode}
\newcommand{\childdocforward}[2][]
{
  \begingroup
    \if?#1?
      \def\childdoctmp
      {
        \def\childdocname{#2}
        \def\childdocjob{#2}
        \def\jobname{#2}
        \input{#2}
        \endinput
      }
    \else
      \def\childdoctmp
      {
        \childdocdisable
        \def\childdocname{#2}
        \childdoctrue
        \includeonly{#2}
        \def\childdocjob{#1}
        \def\jobname{#1}
        \input{#1}
        \endinput
      }
    \fi
    \expandafter
  \endgroup
  \childdoctmp
}
%    \end{macrocode}

% \macro{\childdocforwardprefix}
% The command |\childdocforwardprefix| redirects
% compilation to the main or a child file by means of a pattern.
% The prefix |#1| in the current filename is replaced by |#2|
% and the suffix of the current filename is kept
% (it is assumed that the filename does not contain the substring `|~~~|'
% which is used as a delimiter).
% Compilation is handed over to the new file by |\childdocforward|:
%    \begin{macrocode}
\newcommand{\childdocforwardprefix}[3][]
{
  \begingroup
    \def\childdocextract #2##1~~~{\def\childdoctmp{\childdocforward[#1]{#3##1}}}
    \expandafter\childdocextract\childdocname~~~
    \expandafter
  \endgroup
  \childdoctmp
}
%    \end{macrocode}

% \macro{\childdoc}
% The deprecated macro |\childdoc| is a legacy version of |\childdocmain|:
%    \begin{macrocode}
\newcommand{\childdoc}{\childdocmain}
%    \end{macrocode}

% \macro{\childdocredirect}
% The deprecated macro |\childdocredirect| is a legacy version
% of |\childdocforward| and |\childdocforwardprefix|:
%    \begin{macrocode}
\newcommand{\childdocredirect}[2][]
{
  \begingroup
    \if?#1?
      \def\childdoctmp{\childdocforward{#2}}
    \else
      \def\childdoctmp{\childdocforwardprefix{#1}{#2}}
    \fi
    \expandafter
  \endgroup
  \childdoctmp
}
%    \end{macrocode}

%\iffalse
%</package>
%\fi
%
\endinput

\childdocof{cdocsamp}
%    \end{macrocode}

%\iffalse
%</samplechap1|samplechap2>
%\fi
%
%\iffalse
%<*samplechap1>
%\fi
% Some text for chapter 1:
%    \begin{macrocode}
\section{one}
some text in chapter one
%    \end{macrocode}

%\iffalse
%</samplechap1>
%\fi
% Some text for chapter 2:
%\iffalse
%<*samplechap2>
%\fi
%    \begin{macrocode}
\section{two}
more text in chapter two
%    \end{macrocode}

%\iffalse
%</samplechap2>
%\fi
%
% %%%%%%%%%%%%%%%%%%%%%%%%%%%%%%%%%%%%%%
% \paragraph{Part Include Files.}
%
% The include files are called |cdocspt3.tex| and |cdocspt4.tex|.
%
%\iffalse
%<*samplepart3|samplepart4>
%\fi

% Optional override for |\version| flag:
%    \begin{macrocode}
%%\providecommand{\version}{final}
%    \end{macrocode}

% Include the main document:
%    \begin{macrocode}
% \iffalse
%
% childdoc.dtx Copyright (C) 2017-2018 Niklas Beisert
%
% This work may be distributed and/or modified under the
% conditions of the LaTeX Project Public License, either version 1.3
% of this license or (at your option) any later version.
% The latest version of this license is in
%   http://www.latex-project.org/lppl.txt
% and version 1.3 or later is part of all distributions of LaTeX
% version 2005/12/01 or later.
%
% This work has the LPPL maintenance status `maintained'.
%
% The Current Maintainer of this work is Niklas Beisert.
%
% This work consists of the files childdoc.dtx and childdoc.ins
% and the derived files childdoc.def and cdocsamp.tex with
% cdocsch1.tex, cdocsch2.tex, cdocsdrf.tex, cdocsfn1.tex, cdocsfn2.tex.
%
%<package>\ifdefined\childdocmain\endinput\fi
%<package>\ProvidesFile{childdoc.def}[2018/12/30 v2.0 child document driver]
%<samplemain>\ProvidesFile{cdocsamp.tex}[2018/12/30 v2.0 sample for childdoc]
%<*driver>
%\ProvidesFile{childdoc.drv}[2018/12/30 v2.0 childdoc reference manual file]
\PassOptionsToClass{10pt,a4paper}{article}
\documentclass{ltxdoc}

\usepackage[margin=35mm]{geometry}
\usepackage{hyperref}
\usepackage{hyperxmp}
\usepackage[usenames]{color}

\hypersetup{colorlinks=true}
\hypersetup{pdfstartview=FitH}
\hypersetup{pdfpagemode=UseNone}
\hypersetup{pdfsource={}}
\hypersetup{pdflang={en-UK}}
\hypersetup{pdfcopyright={Copyright 2017-2018 Niklas Beisert.
  This work may be distributed and/or modified under the
  conditions of the LaTeX Project Public License, either version 1.3
  of this license or (at your option) any later version.}}
\hypersetup{pdflicenseurl={http://www.latex-project.org/lppl.txt}}
\hypersetup{pdfcontactaddress={ETH Zurich, ITP, HIT K,
  Wolfgang-Pauli-Strasse 27}}
\hypersetup{pdfcontactpostcode={8093}}
\hypersetup{pdfcontactcity={Zurich}}
\hypersetup{pdfcontactcountry={Switzerland}}
\hypersetup{pdfcontactemail={nbeisert@itp.phys.ethz.ch}}
\hypersetup{pdfcontacturl={http://people.phys.ethz.ch/\xmptilde nbeisert/}}

\newcommand{\secref}[1]{\hyperref[#1]{section \ref*{#1}}}

\parskip1ex
\parindent0pt
\let\olditemize\itemize
\def\itemize{\olditemize\parskip0pt}

\begin{document}

\title{The \textsf{childdoc} Package}
\hypersetup{pdftitle={The childdoc Package}}
\author{Niklas Beisert\\[2ex]
  Institut f\"ur Theoretische Physik\\
  Eidgen\"ossische Technische Hochschule Z\"urich\\
  Wolfgang-Pauli-Strasse 27, 8093 Z\"urich, Switzerland\\[1ex]
  \href{mailto:nbeisert@itp.phys.ethz.ch}
  {\texttt{nbeisert@itp.phys.ethz.ch}}}
\hypersetup{pdfauthor={Niklas Beisert}}
\hypersetup{pdfsubject={Manual for the LaTeX2e Package childdoc}}
\date{30 December 2018, \textsf{v2.0}}
\maketitle

\begin{abstract}\noindent
\textsf{childdoc} is a \LaTeXe{} package
that enables the direct compilation
of document sections included by |\include|
to individual files.
\end{abstract}

\begingroup
\parskip0ex
\tableofcontents
\endgroup

%%%%%%%%%%%%%%%%%%%%%%%%%%%%%%%%%%%%%%%%%%%%%%%%%%%%%%%%%%%%%%%%%%%%%%%%%%%%%%%%
%%%%%%%%%%%%%%%%%%%%%%%%%%%%%%%%%%%%%%%%%%%%%%%%%%%%%%%%%%%%%%%%%%%%%%%%%%%%%%%%
\section{Introduction}

\LaTeX{} provides a mechanism to structure a large document (such as a book)
into a main file and several child files (containing the chapters)
using the |\include| command.
This mechanism is beneficial for documents
which span hundreds of pages in order to
make the source file(s) more manageable.
Moreover, compilation can be restricted to
selected child files by means of the |\includeonly| command.
The latter feature can be used to reduce the compilation time while editing
(this was significantly more useful in the earlier days of \LaTeX{})
or to generate a smaller document which is easier to navigate.
Another application of |\includeonly| is to generate
documents consisting of selected parts of the complete document.

However, there are a few drawbacks of the plain |\include| mechanism:
\begin{itemize}
\item
The child files cannot be compiled on their own,
they can only be compiled via the main file.
A naive editing environment
(such as a text editor with an option
to have the current file processed by \LaTeX)
may require one to switch to the main file before compiling;
attempting to compile the child file produces errors.
\item
The main file must be modified (each time)
to adjust the |\includeonly| command
to the present needs. This easily leaves the main file in a messy state.
\item
The generated document will always carry the filename
of the main document. This is inconvenient if
several child files are to be compiled and
to be kept for distribution.
\end{itemize}

The present package provides a simple interface
to make child files individually compilable by \LaTeX{}.
Compiling a child file then has the same effect as compiling
the main file with an |\includeonly| command
to select the appropriate child.
Moreover the generated document will carry the name of the child
rather than the main file.
This resolves all three above issues.

This feature is meant to make the editing of books,
thesis documents and lecture notes somewhat more convenient.
However, the package can also be used efficiently for
composing a series of documents (such as exercise sheets)
which are typically distributed individually.
It then assists the author in generating the individual documents
(potentially in different versions)
as well as a document containing the collected series.
Another application is in developing style files
or other kinds of included material
where compilation of the style file could redirect
to a sample or test file.

%%%%%%%%%%%%%%%%%%%%%%%%%%%%%%%%%%%%%%%%%%%%%%%%%%%%%%%%%%%%%%%%%%%%%%%%%%%%%%%%
%%%%%%%%%%%%%%%%%%%%%%%%%%%%%%%%%%%%%%%%%%%%%%%%%%%%%%%%%%%%%%%%%%%%%%%%%%%%%%%%
\section{Usage}

First of all, the package \textsf{childdoc} is \emph{not} a standard
\LaTeXe{} |.sty| style file! Therefore it needs to be invoked in
a non-standard way.

%%%%%%%%%%%%%%%%%%%%%%%%%%%%%%%%%%%%%%%%%%%%%%%%%%%%%%%%%%%%%%%%%%%%%%%%%%%%%%%%
\subsection{Included Files}
\label{sec:include}

%%%%%%%%%%%%%%%%%%%%%%%%%%%%%%%%%%%%%%%%
\DescribeMacro{\childdocmain}
To use the package, add the commands
\begin{center}
\begin{tabular}{l}
|\input{childdoc.def}|\\
|\childdocmain{}|\\
\end{tabular}
\end{center}
at the very top of the main \LaTeX{} file,
in particular \emph{before} the |\documentclass| statement!
The argument of |\childdocmain| should be left empty
(but it must be present).

%%%%%%%%%%%%%%%%%%%%%%%%%%%%%%%%%%%%%%%%
\DescribeMacro{\childdocof}
Furthermore, add the commands
\begin{center}
\begin{tabular}{l}
|\input{childdoc.def}|\\
|\childdocof{|\textit{main}|}|\\
\end{tabular}
\end{center}
at the top of every child file \textit{child}
which is included by |\include{|\textit{child}|}|
from within the main file
(or at least for those files to be compiled individually).
The argument \textit{main} must be the filename of the main file.

There are a couple of
considerations in setting up the main and child documents:

%%%%%%%%%%%%%%%%%%%%%%%%%%%%%%%%%%%%%%%%
\paragraph{Restrictions.}

Please note the following restrictions:
\begin{itemize}
\item
|\childdocmain| must be called with one argument \textit{main}
to ensure compatibility with earlier version of the package.
It must either be empty (|\childdocmain{}|)
or precisely match the filename of the main file in which it is specified.
See \secref{sec:detection} for further information.
\item
The filename \textit{main} must be specified without the |.tex| extension.
\item
The filename \textit{main} is case sensitive
(even in case-insensitive file systems)
due to internal string comparison.
\item
The argument \textit{main} should be fully expanded, it cannot be a macro.
\item
Subdirectories and special characters should be avoided in filenames.
\item
The command |\childdocmain{|\textit{main}|}| must be followed by a whitespace.
It should not be followed immediately by another command
or by a comment mark `|%|'.
This is because the \TeX{} parser reads the token immediately following
the argument of |\childdocmain| and puts it
at the beginning of every child section;
however, a white\-space is ignored.
\end{itemize}

%%%%%%%%%%%%%%%%%%%%%%%%%%%%%%%%%%%%%%%%
\paragraph{Content of Main File.}

It is advisable to place all content in the child files included by |\include|.
Any output contained in the main file will appear in all child documents
unless suppressed manually;
it cannot be suppressed automatically by the |\includeonly| directive
and thus should normally be avoided.
A method to include some content in the main file
by means of conditional processing is described in \secref{sec:conditional}.

%%%%%%%%%%%%%%%%%%%%%%%%%%%%%%%%%%%%%%%%
\paragraph{Page Numbering.}

When only a part of the document is compiled,
the appropriate numbering of pages
(as well as other status parameters)
is determined from the |.aux| files.
The latter contain information from previous passes.
However this information needs to propagate through
all intermediate child documents.
Therefore the page numbering in child documents may well
be inconsistent until the complete document is compiled at least once.

A useful (if unconventional) way to always ensure a consistent
page numbering is to restart the numbering in each child document
and denote the pages by `\textit{child}|.|\textit{page}'
where \textit{child} represents the chapter/section number of the child file.
This can be achieved by the command
|\numberwithin{page}{|\textit{child}|}|
of the \textsf{amsmath} package
where \textit{child} can be |chapter| or |section|
depending on the chosen structuring.
Alternatively, one can modify the macro |\thepage| appropriately
and reset the counter |page| at the start of each child file.

%%%%%%%%%%%%%%%%%%%%%%%%%%%%%%%%%%%%%%%%%%%%%%%%%%%%%%%%%%%%%%%%%%%%%%%%%%%%%%%%
\subsection{Conditional Processing}
\label{sec:conditional}

The package provides a mechanism to compile different versions
of a document. To customise the versions further some conditional processing
can come in handy to distinguish which version is being compiled.
The package provides two macros to describe the compilation context:

%%%%%%%%%%%%%%%%%%%%%%%%%%%%%%%%%%%%%%%%
\DescribeMacro{\ifchilddoc}
The conditional |\ifchilddoc| distinguishes between the compilation of
child documents and the main document:
%
\begin{center}
|\ifchilddoc |\textit{child-code}| |[|\||else |\textit{main-code}]| \||fi|
\end{center}

%%%%%%%%%%%%%%%%%%%%%%%%%%%%%%%%%%%%%%%%
\DescribeMacro{\childdocname}
\DescribeMacro{\childdocjob}
The macro |\childdocname| contains the filename (without extension)
of the main or child file being processed.
Note that |\childdocjob| will always contain the name of the main file.

%%%%%%%%%%%%%%%%%%%%%%%%%%%%%%%%%%%%%%%%
\paragraph{Title Page.}

Conditional processing can be used to include a title or banner page
in the main document when proper precautions are taken.
Importantly, the code in the main file should ensure that the page counter
(as well as other status parameters which are stored in the |.aux| files)
takes the same value after the conditional processing.
Otherwise the page numbers may take divergent values
depending on which part is compiled.

For example, a title page could be declared by:
%
\begin{center}
\begin{tabular}{l}
|\ifchilddoc\||else|\\
|\addtocounter{page}{-1}|\\
\textit{code for title page}\\
|\newpage|\\
|\||fi|
\end{tabular}
\end{center}
%
A banner page for the child documents can be generated by:
%
\begin{center}
\begin{tabular}{l}
|\ifchilddoc|\\
|\addtocounter{page}{-1}|\\
\textit{code for banner page}\\
|\newpage|\\
|\||fi|
\end{tabular}
\end{center}
%
Here one could write a message such as:
\begin{center}
|This is the part \childdocname{} of \childdocjob{}.|
\end{center}

%%%%%%%%%%%%%%%%%%%%%%%%%%%%%%%%%%%%%%%%%%%%%%%%%%%%%%%%%%%%%%%%%%%%%%%%%%%%%%%%
\subsection{Flags}
\label{sec:flags}

The package makes it easy to generate different versions
of the main or child documents.
To this end compilation flags can be defined
and assigned different default values.
They will be particularly useful in conjunction
with the forwarding mechanism described in \secref{sec:forward}.

For example, it may be useful to have a flag |\version|
which can be set to |draft| or |final|.
The document source will contain some conditional code
depending on the value of |\version|.
Suppose further, the flag should default to |final| for the main file
and to |draft| for child files
which is a natural assignment for editing the document.
This is achieved by placing the following code
in the preamble of the main document
(below the |\childdocmain| directive):
%
\begin{center}
\begin{tabular}{l}
|\ifchilddoc|\\
|\providecommand{\version}{draft}|\\
|\||else|\\
|\providecommand{\version}{final}|\\
|\||fi|
\end{tabular}
\end{center}
%
The definition by |\providecommand| makes sure
that previous definitions are not overwritten.
Further statements |\providecommand{\version}{...}|
can thus be added before the above code to override it.

For the main file, one might add a line
(between |\childdocmain| and the above block)
%
\begin{center}
|%\ifchilddoc\||else\providecommand{\version}{draft}\||fi|
\end{center}
%
which can be uncommented to produce a draft version.
Likewise one can add a line to the very top of a child file
(above the |\childdocof{|\textit{main}|}| directive)
%
\begin{center}
|%\providecommand{\version}{final}|
\end{center}
%
which can be uncommented to produce the final version of this child document.

%%%%%%%%%%%%%%%%%%%%%%%%%%%%%%%%%%%%%%%%%%%%%%%%%%%%%%%%%%%%%%%%%%%%%%%%%%%%%%%%
\subsection{Forwarding}
\label{sec:forward}

Different versions of the main or child documents
using compilation flags as described in \secref{sec:flags}
can be (permanently) stored in different files
for convenient compilation, viewing and distribution.
To this end, the package defines a command
to pass on compilation to a different file:

%%%%%%%%%%%%%%%%%%%%%%%%%%%%%%%%%%%%%%%%
\DescribeMacro{\childdocforward}
The command |\childdocforward| redirects processing to
another source file:
%
\begin{center}
\begin{tabular}{l}
|\input{childdoc.def}|\\
|\childdocforward[|\textit{main}|]{|\textit{dest}|}|\\
\end{tabular}
\end{center}
%
The argument \textit{dest} is the destination file
(without extension).
It should be the main file or one of the child files.
Note that further \textsf{childdoc} directives
such as |\childdocof| and |\childdocforward|
in the indicated file will be processed in this form.
The optional argument \textit{main}
passes on directly to the main file \textit{main}
while pretending to compile the child \textit{dest}.
This form behaves as if \textit{dest}
issues |\childdocof{|\textit{main}|}| right away,
and no further \textsf{childdoc} directives will be processed.

%%%%%%%%%%%%%%%%%%%%%%%%%%%%%%%%%%%%%%%%
\DescribeMacro{\...prefix}
In the alternative form |\childdocforwardprefix|,
%
\begin{center}
\begin{tabular}{l}
|\input{childdoc.def}|\\
|\childdocforwardprefix[|\textit{main}|]{|\textit{prefix}|}{|\textit{dest}|}|
\end{tabular}
\end{center}
%
the destination file is determined by a pattern
depending on the current file:
To make this work, the current file must be called
`{\textit{prefix}\hspace{0.2em}\textit{suffix}}'
with \textit{prefix} matching precisely the argument.
Processing is then passed on to the file
`{\textit{dest}\hspace{0.2em}\textit{suffix}}'.
Surely, the same effect is achieved by
directly specifying the
argument `{\textit{dest}\hspace{0.2em}\textit{suffix}}'
in the first form.
However, that requires to set up a different file
for each child. With the alternative form of the command
all these files can have exactly the same content
which simplifies setting them up and maintaining them.

For example, the following file |draft.tex|
with a compilation flag |\version| as described in \secref{sec:flags}
compiles the main document as a draft:
%
\begin{center}
\begin{tabular}{l}
|\def\version{draft}|\\
|\input{childdoc.def}|\\
|\childdocforward{|\textit{main}|}|
\end{tabular}
\end{center}
%
Likewise, the following files |final|\textit{nn}|.tex|
compile the final version of the child document
|child|\textit{nn}|.tex|:
%
\begin{center}
\begin{tabular}{l}
|\def\version{final}|\\
|\input{childdoc.def}|\\
|\childdocforwardprefix{final}{child}|
\end{tabular}
\end{center}
%

Note that when several versions of a main file and/or of each child file
are to be generated, it may be convenient to set up a |Makefile| or
shell script to automatise the process.

%%%%%%%%%%%%%%%%%%%%%%%%%%%%%%%%%%%%%%%%%%%%%%%%%%%%%%%%%%%%%%%%%%%%%%%%%%%%%%%%
\subsection{Command Line Processing}
\label{sec:commandline}

The effect of redirection files can also be achieved by invoking
the \LaTeX{} compiler with a more elaborate command line.
Most conveniently this should be done as part
of a shell script or a |Makefile|.

When using \textsf{childdoc} in the main file, the following
command lines effectively perform a redirection
(note that depending on the shell being used,
backslashes may have to be doubled: `|\|' $\to$ `|\\|'):
%
\begin{center}
|... -jobname "|\textit{target}|" |\\|"|[\textit{flags}]%
|\input{childdoc.def}\childdocforward[|\textit{main}|]{|\textit{dest}|}"|
\end{center}
%
Here \textit{target} is the name of the output file,
\textit{main} is the name of the main file
and \textit{dest} is the name of the main or child file to be processed
(all filenames without extensions).
The optional argument \textit{main} can be omitted
if \textit{main} matches \textit{dest}.
Optionally, compilation \textit{flags} can be defined via |\def| commands.
This command line makes the \TeX{} engine believe
it is compiling the file \textit{target}
whose content is specified as the latter parameter.
The provided code then forwards the processing to
\textit{main} or \textit{dest} as described in \secref{sec:forward}.

%%%%%%%%%%%%%%%%%%%%%%%%%%%%%%%%%%%%%%%%%%%%%%%%%%%%%%%%%%%%%%%%%%%%%%%%%%%%%%%%
\subsection{Include by Input}
\label{sec:input}

Including child documents by |\include| has some restrictions by design.
Most notably, the content of a child document always occupies
its own set of pages; pages cannot be shared between child documents.
Usually, this behaviour makes perfect sense
because each child document contain an essential part of the document.
However, in some situations it may be desirable to compose
a document from a collection of parts
without having mandatory page breaks between then.
For this case, the package
provides a mechanism to include parts
by |\input| which can also be processed individually.
However, by construction this mechanism
requires manual handling of the content to be output.

%%%%%%%%%%%%%%%%%%%%%%%%%%%%%%%%%%%%%%%%
\DescribeMacro{\ifchilddocmanual}
The main file should be prepared as usual, see \secref{sec:include}.
However, the document body must make a distinction
between processing of an individual part and of the main document, e.g.:
%
\begin{center}
\begin{tabular}{l}
|\ifchilddocmanual|\\
|\input{\childdocname}|\\
|\||else|\\
\textit{document body with }|\input{|\textit{part}|}|\\
|\||fi|
\end{tabular}
\end{center}
%
The conditional |\ifchilddocmanual| is true whenever
a part to be included by |\input| is being compiled,
and the name of the part is stored in |\childdocname|.

%%%%%%%%%%%%%%%%%%%%%%%%%%%%%%%%%%%%%%%%
\DescribeMacro{\childdocby}
Each part to be included by |\input| should start with:
%
\begin{center}
\begin{tabular}{l}
|\input{childdoc.def}|\\
|\childdocby{|\textit{main}|}|\\
\end{tabular}
\end{center}
%
The directive |\childdocby| is similar to |\childdocof|
described in \secref{sec:include},
but the subsequent selection of content must be done manually.
To that end, both |\ifchilddoc| and |\ifchilddocmanual|
will be true upon processing of a part,
and the name of the part is stored in |\childdocname|.
Note that |\jobname| will be set to the filename of the current part
so that each part receives an individual |.aux| file
that does not interfere with the |.aux| file(s) of the main document.
This behaviour can be altered by the alternative form
|\childdocby[*]{|\textit{main}|}| (with a non-empty optional argument)
which uses the |.aux| file of the main document
by setting |\jobname| to \textit{main}.

%%%%%%%%%%%%%%%%%%%%%%%%%%%%%%%%%%%%%%%%%%%%%%%%%%%%%%%%%%%%%%%%%%%%%%%%%%%%%%%%
\subsection{Driver Development}
\label{sec:driver}

The \textsf{childdoc} mechanism can also be use for the development
of definition files such as \LaTeX{} styles or classes.
This case differs from the above setup with multiple parts
included by |\include| in that no |\includeonly| should be invoked.
This can be achieved by starting the include file
(before |\ProvidesPackage|) with:
%
\begin{center}
\begin{tabular}{l}
|\input{childdoc.def}|\\
|\childdocforward{|\textit{main}|}|\\
\end{tabular}
\end{center}
%
or alternatively with:
%
\begin{center}
\begin{tabular}{l}
|\input{childdoc.def}|\\
|\childdocby{|\textit{main}|}|\\
\end{tabular}
\end{center}
%
Both forms have slightly different effects as described above.
The main file is prepared as usual, see \secref{sec:include}.

%%%%%%%%%%%%%%%%%%%%%%%%%%%%%%%%%%%%%%%%%%%%%%%%%%%%%%%%%%%%%%%%%%%%%%%%%%%%%%%%
\subsection{Legacy Detection}
\label{sec:detection}

The directive |\childdocmain| in the main file can detect
whether the complete document or merely a child is to be compiled
even without using the directive |\childdocof|.
This method is deprecated because it is less robust
and there is no compelling reason to use it;
it is merely provided for backward compatibility
and it may be removed in future versions.

If the detection mechanism is to be used,
it is mandatory to correctly specify
the filename of the main file as the argument of |\childdocmain|:
%
\begin{center}
\begin{tabular}{l}
|\input{childdoc.def}|\\
|\childdocmain{|\textit{main}|}|\\
\end{tabular}
\end{center}
%
If |\jobname| does not match the argument \textit{main} of |\childdocmain|,
it is assumed that |\jobname| points to the child file to be compiled.
When using |\childdocmain| with the main file specified as argument,
it suffices to start a child file
with just |\input{|\textit{main}|}|
without loading of the package and using |\childdocof|.
If instead all processing is done
with the appropriate \textsf{childdoc} directives,
the argument of \textit{main} of |\childdocmain| can be empty.

An alternative version of the command line processing described
in \secref{sec:commandline} using the detection mechanism reads:
%
\begin{center}
|... -jobname "|\textit{target}|" "|[\textit{flags}]%
[|\def\jobname{|\textit{dest}|}|]|\input{|\textit{main}|}"|
\end{center}

%%%%%%%%%%%%%%%%%%%%%%%%%%%%%%%%%%%%%%%%%%%%%%%%%%%%%%%%%%%%%%%%%%%%%%%%%%%%%%%%
\subsection{Manual Code}
\label{sec:manual}

In case one cannot be certain whether the definitions file |childdoc.def|
is installed on the target \TeX{} distribution
and one prefers not to ship it,
it is conceivable to paste a few relevant commands into the sources.

To that end, drop all statements |\input{childdoc.def}|
and perform the replacements as outlined below.
Instead of |\childdocmain{|\textit{main}|}| add the following code
to the top of the main file:
%
\begin{center}
\begin{tabular}{l}
|\||ifdefined\childdocname\endinput\||fi\newif\ifchilddoc|\\
|\edef\childdocname{\scantokens\expandafter{\jobname\noexpand}}|\\
|\def\childdocmain{|\textit{main}|}\||ifx\childdocmain\childdocname\||else|\\
|\childdoctrue\includeonly{\childdocname}\let\jobname\childdocmain\||fi|\\
\end{tabular}
\end{center}
%
Instead of |\childdocof{|\textit{main}|}| just include the main file
at the top of each child file:
%
\begin{center}
|\input{|\textit{main}|}|
\end{center}
%
A simple redirection |\childdocforward{|\textit{dest}|}| is achieved by:
%
\begin{center}
|\def\jobname{|\textit{dest}|}\input{\jobname}|
\end{center}
%
The redirection with prefix
|\childdocforwardprefix[|\textit{prefix}|]{|\textit{dest}|}|
is accomplished by:
%
\begin{center}
\begin{tabular}{l}
|{\edef\jobname{\scantokens\expandafter{\jobname\noexpand}}|\\
|\def\redirectjob |\textit{prefix}|#1~~~{\gdef\jobname{|\textit{dest}|#1}}|\\
|\expandafter\redirectjob\jobname~~~}\input{\jobname}|
\end{tabular}
\end{center}

In an alternative approach,
child documents can be compiled by a specific command line
without additional code or specific definitions:
%
\begin{center}
|... -jobname "|\textit{target}|" "|[\textit{flags}]%
|\includeonly{|\textit{dest}|}\input{|\textit{main}|}"|
\end{center}
%

%%%%%%%%%%%%%%%%%%%%%%%%%%%%%%%%%%%%%%%%%%%%%%%%%%%%%%%%%%%%%%%%%%%%%%%%%%%%%%%%
%%%%%%%%%%%%%%%%%%%%%%%%%%%%%%%%%%%%%%%%%%%%%%%%%%%%%%%%%%%%%%%%%%%%%%%%%%%%%%%%
\section{Information}

%%%%%%%%%%%%%%%%%%%%%%%%%%%%%%%%%%%%%%%%%%%%%%%%%%%%%%%%%%%%%%%%%%%%%%%%%%%%%%%%
\subsection{Copyright}

Copyright \copyright{} 2017--2018 Niklas Beisert

This work may be distributed and/or modified under the
conditions of the \LaTeX{} Project Public License, either version 1.3
of this license or (at your option) any later version.
The latest version of this license is in
  \url{http://www.latex-project.org/lppl.txt}
and version 1.3 or later is part of all distributions of \LaTeX{}
version 2005/12/01 or later.

This work has the LPPL maintenance status `maintained'.

The Current Maintainer of this work is Niklas Beisert.

This work consists of the files |README.txt|, |childdoc.ins| and |childdoc.dtx|
as well as the derived files |childdoc.def|, |cdocsamp.tex|
with |cdocsch1.tex|, |cdocsch2.tex|, |cdocspt3.tex|, |cdocspt4.tex|,
|cdocsdrf.tex|, |cdocsfn1.tex|, |cdocsfn2.tex|
as well as |childdoc.pdf|.

%%%%%%%%%%%%%%%%%%%%%%%%%%%%%%%%%%%%%%%%%%%%%%%%%%%%%%%%%%%%%%%%%%%%%%%%%%%%%%%%
\subsection{Files and Installation}

The package consists of the files:
%
\begin{center}
\begin{tabular}{ll}
    |README.txt|   & readme file \\
    |childdoc.ins| & installation file \\
    |childdoc.dtx| & source file \\
    |childdoc.def| & definition file \\
    |cdocsamp.tex| & sample main file \\
    |cdocsch1.tex| & sample include file \\
    |cdocsch2.tex| & sample include file \\
    |cdocspt3.tex| & sample part file \\
    |cdocspt4.tex| & sample part file \\
    |cdocsdrf.tex| & sample redirection file \\
    |cdocsfn1.tex| & sample redirection file \\
    |cdocsfn2.tex| & sample redirection file \\
    |childdoc.pdf| & manual
\end{tabular}
\end{center}
%
The distribution consists of the files
|README.txt|, |childdoc.ins| and |childdoc.dtx|.
%
\begin{itemize}
\item
Run (pdf)\LaTeX{} on |childdoc.dtx|
to compile the manual |childdoc.pdf| (this file).
\item
Run \LaTeX{} on |childdoc.ins| to create the definitions file |childdoc.def|
and the sample |cdocsamp.tex| with include files
|cdocsch1.tex|, |cdocsch2.tex|, |cdocspt3.tex|, |cdocspt4.tex|,
|cdocsdrf.tex|, |cdocsfn1.tex|, |cdocsfn2.tex|.
Then copy the file |childdoc.def| to an appropriate directory of your \LaTeX{}
distribution, e.g.\ \textit{texmf-root}|/tex/latex/childdoc|.
\end{itemize}

%%%%%%%%%%%%%%%%%%%%%%%%%%%%%%%%%%%%%%%%%%%%%%%%%%%%%%%%%%%%%%%%%%%%%%%%%%%%%%%%
\subsection{Related CTAN Packages}

There are several other packages which offer a similar functionality:
%
\begin{itemize}
\item
The packages
\href{http://ctan.org/pkg/docmute}{\textsf{docmute}},
\href{http://ctan.org/pkg/includex}{\textsf{includex}} and
\href{http://ctan.org/pkg/standalone}{\textsf{standalone}}
provide commands to include only the document body of
a child file thus allowing both files to be compiled individually.
\item
The packages \href{http://ctan.org/pkg/subdocs}{\textsf{subdocs}}
and \href{http://ctan.org/pkg/subfiles}{\textsf{subfiles}}
provide structures in which the main and child documents can be
encapsulated and allowing them to be compiled individually.
The inclusion mechanism is different from the conventional |\include|.
\item
The package \href{http://ctan.org/pkg/combine}{\textsf{combine}}
is an elaborate solution to combine several documents into one.
\end{itemize}
%
See also the CTAN topic \href{http://ctan.org/topic/subdocs}{\textsf{subdocs}}
for further related packages.
The present package differs from the above solutions in that
a document structure constructed with the conventional |\include| mechanism
just needs two extra commands at the top of every file
such that all constituent files can be compiled individually.

%%%%%%%%%%%%%%%%%%%%%%%%%%%%%%%%%%%%%%%%%%%%%%%%%%%%%%%%%%%%%%%%%%%%%%%%%%%%%%%%
%\subsection{Feature Suggestions}
%
%The following is a list of features which may be useful for future
%versions of this package:
%%
%\begin{itemize}
%\item
%\ldots
%\end{itemize}

%%%%%%%%%%%%%%%%%%%%%%%%%%%%%%%%%%%%%%%%%%%%%%%%%%%%%%%%%%%%%%%%%%%%%%%%%%%%%%%%
\subsection{Revision History}

%%%%%%%%%%%%%%%%%%%%%%%%%%%%%%%%%%%%%%%%
\paragraph{v2.0:} 2018/12/30

\begin{itemize}
\item
immediate forward processing
\item
added |\childdocby| mechanism
\item
manual restructured
\end{itemize}

%%%%%%%%%%%%%%%%%%%%%%%%%%%%%%%%%%%%%%%%
\paragraph{v1.6:} 2018/01/17

\begin{itemize}
\item
application for development of include files
\item
corrections to manual
\end{itemize}

%%%%%%%%%%%%%%%%%%%%%%%%%%%%%%%%%%%%%%%%
\paragraph{v1.5:} 2017/05/21

\begin{itemize}
\item
more complete structuring introduced
\item
|\childdocof| introduced
\item
|\childdoc| renamed to |\childdocmain|
\item
|\childredirect| renamed to |\childdocforward| and |\childdocforwardprefix|
and functionality expanded
\end{itemize}

%%%%%%%%%%%%%%%%%%%%%%%%%%%%%%%%%%%%%%%%
\paragraph{v1.0:} 2017/04/27

\begin{itemize}
\item
manual and install package
\item
first version published on CTAN
\end{itemize}

%%%%%%%%%%%%%%%%%%%%%%%%%%%%%%%%%%%%%%%%
\paragraph{v0.6:} 2017/04/26

\begin{itemize}
\item
redirection mechanism added
\end{itemize}

%%%%%%%%%%%%%%%%%%%%%%%%%%%%%%%%%%%%%%%%
\paragraph{v0.5:} 2017/04/26

\begin{itemize}
\item
functionality in definition file
\end{itemize}


%%%%%%%%%%%%%%%%%%%%%%%%%%%%%%%%%%%%%%%%%%%%%%%%%%%%%%%%%%%%%%%%%%%%%%%%%%%%%%%%
%%%%%%%%%%%%%%%%%%%%%%%%%%%%%%%%%%%%%%%%%%%%%%%%%%%%%%%%%%%%%%%%%%%%%%%%%%%%%%%%
%%%%%%%%%%%%%%%%%%%%%%%%%%%%%%%%%%%%%%%%%%%%%%%%%%%%%%%%%%%%%%%%%%%%%%%%%%%%%%%%
\appendix

\settowidth\MacroIndent{\rmfamily\scriptsize 000\ }

 \DocInput{childdoc.dtx}

\end{document}
%</driver>
% \fi
%
% %%%%%%%%%%%%%%%%%%%%%%%%%%%%%%%%%%%%%%%%%%%%%%%%%%%%%%%%%%%%%%%%%%%%%%%%%%%%%%
% %%%%%%%%%%%%%%%%%%%%%%%%%%%%%%%%%%%%%%%%%%%%%%%%%%%%%%%%%%%%%%%%%%%%%%%%%%%%%%
% \section{Sample}
%\iffalse
%<*samplemain>
%\fi
%
% The following presents a sample document
% with two chapters, two parts, a title page,
% a compile flag as well as three forwarding files to set the flag.
% It consists of eight |.tex| files:
% \begin{center}
% \begin{tabular}{ll}
% |cdocsamp.tex|&main file\\
% |cdocsch1.tex|&include file for chapter 1\\
% |cdocsch2.tex|&include file for chapter 2\\
% |cdocspt3.tex|&include file for part 3\\
% |cdocspt4.tex|&include file for part 4\\
% |cdocsdrf.tex|&forwarding file for main file in draft mode\\
% |cdocsfi1.tex|&forwarding file for final version of chapter 1\\
% |cdocsfi2.tex|&forwarding file for final version of chapter 2\\
% \end{tabular}
% \end{center}
% Each of the eight files can be compiled directly by the \LaTeX{} compiler.
%
% %%%%%%%%%%%%%%%%%%%%%%%%%%%%%%%%%%%%%%
% \paragraph{Main File.}
%
% The main file is called |cdocsamp.tex|.
%
% Load the \textsf{childdoc} definitions and
% declare the filename for the main document:
%    \begin{macrocode}
\input{childdoc.def}
\childdocmain{}
%    \end{macrocode}

% Optional override for |\version| flag:
%    \begin{macrocode}
%%\ifchilddoc\else\providecommand{\version}{draft}\fi
%    \end{macrocode}

% Define the default values for the |\version| flag
% (|final| for the main file and |draft| for childs):
%    \begin{macrocode}
\ifchilddoc
\providecommand{\version}{draft}
\else
\providecommand{\version}{final}
\fi
%    \end{macrocode}

% Load the standard document class:
%    \begin{macrocode}
\documentclass[12pt]{article}
%    \end{macrocode}

% Start the document body:
%    \begin{macrocode}
\begin{document}
%    \end{macrocode}

% Declare a title page.
% Print title, part of document being processed and version flag:
%    \begin{macrocode}
\addtocounter{page}{-1}
\begin{center}
{\LARGE\bfseries{}childdoc example\par}
\vspace{1cm}
\ifchilddoc
\ifchilddocmanual part\else chapter\fi:
`\childdocname' of `\childdocjob'\par
\else
main document: `\childdocjob'\par
\fi
version: \version\par
\end{center}
\newpage
%    \end{macrocode}

% Manually include selected file,
% otherwise process as usual:
%    \begin{macrocode}
\ifchilddocmanual
\section*{part `\childdocname'}
\input{\childdocname}
\else
%    \end{macrocode}

% Include the two chapters:
%    \begin{macrocode}
\include{cdocsch1}
\include{cdocsch2}
%    \end{macrocode}

% Include the two parts unless only chapters should be displayed:
%    \begin{macrocode}
\ifchilddoc\else
\section{part three}
\input{cdocspt3}
\section{part four}
\input{cdocspt4}
\fi
%    \end{macrocode}

% Process as usual until here:
%    \begin{macrocode}
\fi
%    \end{macrocode}

% End of document body:
%    \begin{macrocode}
\end{document}
%    \end{macrocode}
%\iffalse
%</samplemain>
%\fi
%
% %%%%%%%%%%%%%%%%%%%%%%%%%%%%%%%%%%%%%%
% \paragraph{Chapter Include Files.}
%
% The include files are called |cdocsch1.tex| and |cdocsch2.tex|.
%
%\iffalse
%<*samplechap1|samplechap2>
%\fi

% Optional override for |\version| flag:
%    \begin{macrocode}
%%\providecommand{\version}{final}
%    \end{macrocode}

% Include the main document:
%    \begin{macrocode}
\input{childdoc.def}
\childdocof{cdocsamp}
%    \end{macrocode}

%\iffalse
%</samplechap1|samplechap2>
%\fi
%
%\iffalse
%<*samplechap1>
%\fi
% Some text for chapter 1:
%    \begin{macrocode}
\section{one}
some text in chapter one
%    \end{macrocode}

%\iffalse
%</samplechap1>
%\fi
% Some text for chapter 2:
%\iffalse
%<*samplechap2>
%\fi
%    \begin{macrocode}
\section{two}
more text in chapter two
%    \end{macrocode}

%\iffalse
%</samplechap2>
%\fi
%
% %%%%%%%%%%%%%%%%%%%%%%%%%%%%%%%%%%%%%%
% \paragraph{Part Include Files.}
%
% The include files are called |cdocspt3.tex| and |cdocspt4.tex|.
%
%\iffalse
%<*samplepart3|samplepart4>
%\fi

% Optional override for |\version| flag:
%    \begin{macrocode}
%%\providecommand{\version}{final}
%    \end{macrocode}

% Include the main document:
%    \begin{macrocode}
\input{childdoc.def}
\childdocby{cdocsamp}
%    \end{macrocode}

%\iffalse
%</samplepart3|samplepart4>
%\fi
%
%\iffalse
%<*samplepart3>
%\fi
% Some text for part 3:
%    \begin{macrocode}
some text in part three
%    \end{macrocode}

%\iffalse
%</samplepart3>
%\fi
% Some text for part 4:
%\iffalse
%<*samplepart4>
%\fi
%    \begin{macrocode}
more text in part four
%    \end{macrocode}

%\iffalse
%</samplepart4>
%\fi
%
% %%%%%%%%%%%%%%%%%%%%%%%%%%%%%%%%%%%%%%
% \paragraph{Forwarding for a Complete Draft.}
%
% The following forwarding file |cdocsdrf.tex|
% compiles the main document in draft mode:
%\iffalse
%<*sampledraft>
%\fi
%    \begin{macrocode}
\def\version{draft}
\input{childdoc.def}
\childdocforward{cdocsamp}
%    \end{macrocode}

%\iffalse
%</sampledraft>
%\fi
%
% %%%%%%%%%%%%%%%%%%%%%%%%%%%%%%%%%%%%%%
% \paragraph{Forwarding for Final Version of the Chapters.}
%
% The following forwarding files |cdocsfn1.tex| and |cdocsfn2.tex|
% (with identical content)
% compile the final versions of the child documents
% |cdocsch1.tex| and |cdocsch2.tex|, respectively:
%\iffalse
%<*samplefinal>
%\fi
%    \begin{macrocode}
\def\version{final}
\input{childdoc.def}
\childdocforwardprefix[cdocsamp]{cdocsfn}{cdocsch}
%    \end{macrocode}

%\iffalse
%</samplefinal>
%\fi
%
% %%%%%%%%%%%%%%%%%%%%%%%%%%%%%%%%%%%%%%
% \paragraph{Command Line Processing.}
%
% The following three command lines generate the output files
% |cdocscld|, |cdocscl1| and |cdocscl2|
% which should be identical to
% |cdocsdrf|, |cdocsch1| and |cdocsfn2|, respectively:
% \begin{center}
% \begin{tabular}{l}
% |latex -jobname cdocscld \|\\
% |  "\def\version{draft}\input{childdoc.def}\childdocforward{cdocsamp}"|\\
% |latex -jobname cdocscl1 \|\\
% |  "\input{childdoc.def}\childdocforward[cdocsamp]{cdocsch1}"|\\
% |latex -jobname cdocscl2 \|\\
% |  "\def\version{final}\input{childdoc.def}\childdocforward{cdocsch2}"|
% \end{tabular}
% \end{center}
% Note that the trailing backslash on each first line
% merely continues the input to the second line
% (for convenient cut ant paste).
% Furthermore, the command |latex| can be replaced by any
% of its alternative versions such as |pdflatex|.
%
% %%%%%%%%%%%%%%%%%%%%%%%%%%%%%%%%%%%%%%%%%%%%%%%%%%%%%%%%%%%%%%%%%%%%%%%%%%%%%%
% %%%%%%%%%%%%%%%%%%%%%%%%%%%%%%%%%%%%%%%%%%%%%%%%%%%%%%%%%%%%%%%%%%%%%%%%%%%%%%
% \section{Implementation}
%\iffalse
%<*package>
%\fi
%
% This section describes the definitions file |childdoc.def|.

% The definitions cannot be loaded using |\usepackage| or |\RequirePackage|
% which has a mechanism to prevent loading a style file more than once.
% When loading the definitions by means of |\input|
% multiple instances have to be prevented manually:
%\iffalse
%This code needs to be before the `\ProvidesFile' directive
%which is defined at the beginning of this file.
%Therefore it is also placed there and commented out here.
%</package>
%<*discard>
%\fi
%    \begin{macrocode}
\ifdefined\childdocmain\endinput\fi
%    \end{macrocode}
%\iffalse
%</discard>
%<*package>
%\fi
%
% \macro{\ifchilddoc}
% \macro{\ifchilddocmanual}
% The conditional |\ifchilddoc| tells whether a
% child (true) or main (false) document is being compiled.
% The conditional |\ifchilddocmanual| tells whether
% the |\includeonly| mechanism is used (false) or
% the selection of child files must be performed manually (true).
% The definitions initialise to false:
%    \begin{macrocode}
\newif\ifchilddoc
\newif\ifchilddocmanual
%    \end{macrocode}

% \macro{\childdocname}
% \macro{\childdocjob}
% The macro |\childdocname| stores the name of the main document
% to be compiled. The macro |\childdocjob| stores the name of
% the document on which the \LaTeX{} compiler was originally invoked.
% The content of |\jobname| cannot be compared
% to filenames specified in the source due to different catcodes.
% The following code rescans |\jobname|, stores the result
% in |\childdocname| and saves a copy in |\childdocjob|:
%    \begin{macrocode}
\edef\childdocname{\scantokens\expandafter{\jobname\noexpand}}
\let\childdocjob\childdocname
%    \end{macrocode}

% \macro{\childdocdisable}
% The macro |\childdocdisable| prevents the main file
% from being processed more than once.
% At this stage, the main document command |\childdocmain|
% is assumed to be called once again where it should do nothing.
% Any subsequent call to it should prevent
% a secondary processing of the main document
% It overwrites the forwarding commands
% |\childdocof| and |\childdocforward|
% with empty macros to prevent further inclusions of the main document:
%    \begin{macrocode}
\newcommand{\childdocdisable}
{
  \renewcommand{\childdocmain}[1]{\renewcommand{\childdocmain}[1]{\endinput}}
  \renewcommand{\childdocof}[1]{}
  \renewcommand{\childdocby}[2][]{}
  \renewcommand{\childdocforward}[2][]{}
  \renewcommand{\childdocdisable}{}
}
%    \end{macrocode}

% \macro{\childdocmain}
% The macro |\childdocmain| is to be called at the top of the main file
% with nothing or the main filename (without extension) as argument.
% First, it breaks loops.
% If the argument is not empty and does not match |\childdocname|
% (which is set by the first inclusion of |childdoc.def|),
% |\ifchilddoc| is set to true, |\includeonly| is applied to the child file
% and |\jobname| is set to the main file
% (for proper handling of |.aux| files):
%    \begin{macrocode}
\newcommand{\childdocmain}[1]
{
  \childdocdisable\childdocmain{}
  \if?#1?\else
    \begingroup
      \def\childdoctmp{#1}
      \ifx\childdoctmp\childdocname
        \def\childdoctmp{}
      \else
        \def\childdoctmp
        {
          \childdoctrue
          \includeonly{\childdocname}
          \def\childdocjob{#1}
          \def\jobname{#1}
        }
      \fi
      \expandafter
    \endgroup
    \childdoctmp
  \fi
}
%    \end{macrocode}

% \macro{\childdocof}
% The command |\childdocof| redirects
% compilation to the main file |#1|.
%    \begin{macrocode}
\newcommand{\childdocof}[1]
{
  \childdocdisable
  \childdoctrue
  \includeonly{\childdocname}
  \def\jobname{#1}
  \def\childdocjob{#1}
  \input{#1}
}
%    \end{macrocode}

% \macro{\childdocby}
% The command |\childdocby| ....
%    \begin{macrocode}
\newcommand{\childdocby}[2][]
{
  \childdocdisable
  \childdoctrue
  \childdocmanualtrue
  \if?#1?\else
    \def\jobname{#2}
  \fi
  \def\childdocjob{#2}
  \input{#2}
  \endinput
}
%    \end{macrocode}

% \macro{\childdocforward}
% The command |\childdocforward| redirects
% compilation to the main file or
% (if the optional argument is given) a child file.
% Parameters are set as if the main file
% or a child file starting with |\childdocof| was compiled.
% Then compilation is handed over to the main file:
%    \begin{macrocode}
\newcommand{\childdocforward}[2][]
{
  \begingroup
    \if?#1?
      \def\childdoctmp
      {
        \def\childdocname{#2}
        \def\childdocjob{#2}
        \def\jobname{#2}
        \input{#2}
        \endinput
      }
    \else
      \def\childdoctmp
      {
        \childdocdisable
        \def\childdocname{#2}
        \childdoctrue
        \includeonly{#2}
        \def\childdocjob{#1}
        \def\jobname{#1}
        \input{#1}
        \endinput
      }
    \fi
    \expandafter
  \endgroup
  \childdoctmp
}
%    \end{macrocode}

% \macro{\childdocforwardprefix}
% The command |\childdocforwardprefix| redirects
% compilation to the main or a child file by means of a pattern.
% The prefix |#1| in the current filename is replaced by |#2|
% and the suffix of the current filename is kept
% (it is assumed that the filename does not contain the substring `|~~~|'
% which is used as a delimiter).
% Compilation is handed over to the new file by |\childdocforward|:
%    \begin{macrocode}
\newcommand{\childdocforwardprefix}[3][]
{
  \begingroup
    \def\childdocextract #2##1~~~{\def\childdoctmp{\childdocforward[#1]{#3##1}}}
    \expandafter\childdocextract\childdocname~~~
    \expandafter
  \endgroup
  \childdoctmp
}
%    \end{macrocode}

% \macro{\childdoc}
% The deprecated macro |\childdoc| is a legacy version of |\childdocmain|:
%    \begin{macrocode}
\newcommand{\childdoc}{\childdocmain}
%    \end{macrocode}

% \macro{\childdocredirect}
% The deprecated macro |\childdocredirect| is a legacy version
% of |\childdocforward| and |\childdocforwardprefix|:
%    \begin{macrocode}
\newcommand{\childdocredirect}[2][]
{
  \begingroup
    \if?#1?
      \def\childdoctmp{\childdocforward{#2}}
    \else
      \def\childdoctmp{\childdocforwardprefix{#1}{#2}}
    \fi
    \expandafter
  \endgroup
  \childdoctmp
}
%    \end{macrocode}

%\iffalse
%</package>
%\fi
%
\endinput

\childdocby{cdocsamp}
%    \end{macrocode}

%\iffalse
%</samplepart3|samplepart4>
%\fi
%
%\iffalse
%<*samplepart3>
%\fi
% Some text for part 3:
%    \begin{macrocode}
some text in part three
%    \end{macrocode}

%\iffalse
%</samplepart3>
%\fi
% Some text for part 4:
%\iffalse
%<*samplepart4>
%\fi
%    \begin{macrocode}
more text in part four
%    \end{macrocode}

%\iffalse
%</samplepart4>
%\fi
%
% %%%%%%%%%%%%%%%%%%%%%%%%%%%%%%%%%%%%%%
% \paragraph{Forwarding for a Complete Draft.}
%
% The following forwarding file |cdocsdrf.tex|
% compiles the main document in draft mode:
%\iffalse
%<*sampledraft>
%\fi
%    \begin{macrocode}
\def\version{draft}
% \iffalse
%
% childdoc.dtx Copyright (C) 2017-2018 Niklas Beisert
%
% This work may be distributed and/or modified under the
% conditions of the LaTeX Project Public License, either version 1.3
% of this license or (at your option) any later version.
% The latest version of this license is in
%   http://www.latex-project.org/lppl.txt
% and version 1.3 or later is part of all distributions of LaTeX
% version 2005/12/01 or later.
%
% This work has the LPPL maintenance status `maintained'.
%
% The Current Maintainer of this work is Niklas Beisert.
%
% This work consists of the files childdoc.dtx and childdoc.ins
% and the derived files childdoc.def and cdocsamp.tex with
% cdocsch1.tex, cdocsch2.tex, cdocsdrf.tex, cdocsfn1.tex, cdocsfn2.tex.
%
%<package>\ifdefined\childdocmain\endinput\fi
%<package>\ProvidesFile{childdoc.def}[2018/12/30 v2.0 child document driver]
%<samplemain>\ProvidesFile{cdocsamp.tex}[2018/12/30 v2.0 sample for childdoc]
%<*driver>
%\ProvidesFile{childdoc.drv}[2018/12/30 v2.0 childdoc reference manual file]
\PassOptionsToClass{10pt,a4paper}{article}
\documentclass{ltxdoc}

\usepackage[margin=35mm]{geometry}
\usepackage{hyperref}
\usepackage{hyperxmp}
\usepackage[usenames]{color}

\hypersetup{colorlinks=true}
\hypersetup{pdfstartview=FitH}
\hypersetup{pdfpagemode=UseNone}
\hypersetup{pdfsource={}}
\hypersetup{pdflang={en-UK}}
\hypersetup{pdfcopyright={Copyright 2017-2018 Niklas Beisert.
  This work may be distributed and/or modified under the
  conditions of the LaTeX Project Public License, either version 1.3
  of this license or (at your option) any later version.}}
\hypersetup{pdflicenseurl={http://www.latex-project.org/lppl.txt}}
\hypersetup{pdfcontactaddress={ETH Zurich, ITP, HIT K,
  Wolfgang-Pauli-Strasse 27}}
\hypersetup{pdfcontactpostcode={8093}}
\hypersetup{pdfcontactcity={Zurich}}
\hypersetup{pdfcontactcountry={Switzerland}}
\hypersetup{pdfcontactemail={nbeisert@itp.phys.ethz.ch}}
\hypersetup{pdfcontacturl={http://people.phys.ethz.ch/\xmptilde nbeisert/}}

\newcommand{\secref}[1]{\hyperref[#1]{section \ref*{#1}}}

\parskip1ex
\parindent0pt
\let\olditemize\itemize
\def\itemize{\olditemize\parskip0pt}

\begin{document}

\title{The \textsf{childdoc} Package}
\hypersetup{pdftitle={The childdoc Package}}
\author{Niklas Beisert\\[2ex]
  Institut f\"ur Theoretische Physik\\
  Eidgen\"ossische Technische Hochschule Z\"urich\\
  Wolfgang-Pauli-Strasse 27, 8093 Z\"urich, Switzerland\\[1ex]
  \href{mailto:nbeisert@itp.phys.ethz.ch}
  {\texttt{nbeisert@itp.phys.ethz.ch}}}
\hypersetup{pdfauthor={Niklas Beisert}}
\hypersetup{pdfsubject={Manual for the LaTeX2e Package childdoc}}
\date{30 December 2018, \textsf{v2.0}}
\maketitle

\begin{abstract}\noindent
\textsf{childdoc} is a \LaTeXe{} package
that enables the direct compilation
of document sections included by |\include|
to individual files.
\end{abstract}

\begingroup
\parskip0ex
\tableofcontents
\endgroup

%%%%%%%%%%%%%%%%%%%%%%%%%%%%%%%%%%%%%%%%%%%%%%%%%%%%%%%%%%%%%%%%%%%%%%%%%%%%%%%%
%%%%%%%%%%%%%%%%%%%%%%%%%%%%%%%%%%%%%%%%%%%%%%%%%%%%%%%%%%%%%%%%%%%%%%%%%%%%%%%%
\section{Introduction}

\LaTeX{} provides a mechanism to structure a large document (such as a book)
into a main file and several child files (containing the chapters)
using the |\include| command.
This mechanism is beneficial for documents
which span hundreds of pages in order to
make the source file(s) more manageable.
Moreover, compilation can be restricted to
selected child files by means of the |\includeonly| command.
The latter feature can be used to reduce the compilation time while editing
(this was significantly more useful in the earlier days of \LaTeX{})
or to generate a smaller document which is easier to navigate.
Another application of |\includeonly| is to generate
documents consisting of selected parts of the complete document.

However, there are a few drawbacks of the plain |\include| mechanism:
\begin{itemize}
\item
The child files cannot be compiled on their own,
they can only be compiled via the main file.
A naive editing environment
(such as a text editor with an option
to have the current file processed by \LaTeX)
may require one to switch to the main file before compiling;
attempting to compile the child file produces errors.
\item
The main file must be modified (each time)
to adjust the |\includeonly| command
to the present needs. This easily leaves the main file in a messy state.
\item
The generated document will always carry the filename
of the main document. This is inconvenient if
several child files are to be compiled and
to be kept for distribution.
\end{itemize}

The present package provides a simple interface
to make child files individually compilable by \LaTeX{}.
Compiling a child file then has the same effect as compiling
the main file with an |\includeonly| command
to select the appropriate child.
Moreover the generated document will carry the name of the child
rather than the main file.
This resolves all three above issues.

This feature is meant to make the editing of books,
thesis documents and lecture notes somewhat more convenient.
However, the package can also be used efficiently for
composing a series of documents (such as exercise sheets)
which are typically distributed individually.
It then assists the author in generating the individual documents
(potentially in different versions)
as well as a document containing the collected series.
Another application is in developing style files
or other kinds of included material
where compilation of the style file could redirect
to a sample or test file.

%%%%%%%%%%%%%%%%%%%%%%%%%%%%%%%%%%%%%%%%%%%%%%%%%%%%%%%%%%%%%%%%%%%%%%%%%%%%%%%%
%%%%%%%%%%%%%%%%%%%%%%%%%%%%%%%%%%%%%%%%%%%%%%%%%%%%%%%%%%%%%%%%%%%%%%%%%%%%%%%%
\section{Usage}

First of all, the package \textsf{childdoc} is \emph{not} a standard
\LaTeXe{} |.sty| style file! Therefore it needs to be invoked in
a non-standard way.

%%%%%%%%%%%%%%%%%%%%%%%%%%%%%%%%%%%%%%%%%%%%%%%%%%%%%%%%%%%%%%%%%%%%%%%%%%%%%%%%
\subsection{Included Files}
\label{sec:include}

%%%%%%%%%%%%%%%%%%%%%%%%%%%%%%%%%%%%%%%%
\DescribeMacro{\childdocmain}
To use the package, add the commands
\begin{center}
\begin{tabular}{l}
|\input{childdoc.def}|\\
|\childdocmain{}|\\
\end{tabular}
\end{center}
at the very top of the main \LaTeX{} file,
in particular \emph{before} the |\documentclass| statement!
The argument of |\childdocmain| should be left empty
(but it must be present).

%%%%%%%%%%%%%%%%%%%%%%%%%%%%%%%%%%%%%%%%
\DescribeMacro{\childdocof}
Furthermore, add the commands
\begin{center}
\begin{tabular}{l}
|\input{childdoc.def}|\\
|\childdocof{|\textit{main}|}|\\
\end{tabular}
\end{center}
at the top of every child file \textit{child}
which is included by |\include{|\textit{child}|}|
from within the main file
(or at least for those files to be compiled individually).
The argument \textit{main} must be the filename of the main file.

There are a couple of
considerations in setting up the main and child documents:

%%%%%%%%%%%%%%%%%%%%%%%%%%%%%%%%%%%%%%%%
\paragraph{Restrictions.}

Please note the following restrictions:
\begin{itemize}
\item
|\childdocmain| must be called with one argument \textit{main}
to ensure compatibility with earlier version of the package.
It must either be empty (|\childdocmain{}|)
or precisely match the filename of the main file in which it is specified.
See \secref{sec:detection} for further information.
\item
The filename \textit{main} must be specified without the |.tex| extension.
\item
The filename \textit{main} is case sensitive
(even in case-insensitive file systems)
due to internal string comparison.
\item
The argument \textit{main} should be fully expanded, it cannot be a macro.
\item
Subdirectories and special characters should be avoided in filenames.
\item
The command |\childdocmain{|\textit{main}|}| must be followed by a whitespace.
It should not be followed immediately by another command
or by a comment mark `|%|'.
This is because the \TeX{} parser reads the token immediately following
the argument of |\childdocmain| and puts it
at the beginning of every child section;
however, a white\-space is ignored.
\end{itemize}

%%%%%%%%%%%%%%%%%%%%%%%%%%%%%%%%%%%%%%%%
\paragraph{Content of Main File.}

It is advisable to place all content in the child files included by |\include|.
Any output contained in the main file will appear in all child documents
unless suppressed manually;
it cannot be suppressed automatically by the |\includeonly| directive
and thus should normally be avoided.
A method to include some content in the main file
by means of conditional processing is described in \secref{sec:conditional}.

%%%%%%%%%%%%%%%%%%%%%%%%%%%%%%%%%%%%%%%%
\paragraph{Page Numbering.}

When only a part of the document is compiled,
the appropriate numbering of pages
(as well as other status parameters)
is determined from the |.aux| files.
The latter contain information from previous passes.
However this information needs to propagate through
all intermediate child documents.
Therefore the page numbering in child documents may well
be inconsistent until the complete document is compiled at least once.

A useful (if unconventional) way to always ensure a consistent
page numbering is to restart the numbering in each child document
and denote the pages by `\textit{child}|.|\textit{page}'
where \textit{child} represents the chapter/section number of the child file.
This can be achieved by the command
|\numberwithin{page}{|\textit{child}|}|
of the \textsf{amsmath} package
where \textit{child} can be |chapter| or |section|
depending on the chosen structuring.
Alternatively, one can modify the macro |\thepage| appropriately
and reset the counter |page| at the start of each child file.

%%%%%%%%%%%%%%%%%%%%%%%%%%%%%%%%%%%%%%%%%%%%%%%%%%%%%%%%%%%%%%%%%%%%%%%%%%%%%%%%
\subsection{Conditional Processing}
\label{sec:conditional}

The package provides a mechanism to compile different versions
of a document. To customise the versions further some conditional processing
can come in handy to distinguish which version is being compiled.
The package provides two macros to describe the compilation context:

%%%%%%%%%%%%%%%%%%%%%%%%%%%%%%%%%%%%%%%%
\DescribeMacro{\ifchilddoc}
The conditional |\ifchilddoc| distinguishes between the compilation of
child documents and the main document:
%
\begin{center}
|\ifchilddoc |\textit{child-code}| |[|\||else |\textit{main-code}]| \||fi|
\end{center}

%%%%%%%%%%%%%%%%%%%%%%%%%%%%%%%%%%%%%%%%
\DescribeMacro{\childdocname}
\DescribeMacro{\childdocjob}
The macro |\childdocname| contains the filename (without extension)
of the main or child file being processed.
Note that |\childdocjob| will always contain the name of the main file.

%%%%%%%%%%%%%%%%%%%%%%%%%%%%%%%%%%%%%%%%
\paragraph{Title Page.}

Conditional processing can be used to include a title or banner page
in the main document when proper precautions are taken.
Importantly, the code in the main file should ensure that the page counter
(as well as other status parameters which are stored in the |.aux| files)
takes the same value after the conditional processing.
Otherwise the page numbers may take divergent values
depending on which part is compiled.

For example, a title page could be declared by:
%
\begin{center}
\begin{tabular}{l}
|\ifchilddoc\||else|\\
|\addtocounter{page}{-1}|\\
\textit{code for title page}\\
|\newpage|\\
|\||fi|
\end{tabular}
\end{center}
%
A banner page for the child documents can be generated by:
%
\begin{center}
\begin{tabular}{l}
|\ifchilddoc|\\
|\addtocounter{page}{-1}|\\
\textit{code for banner page}\\
|\newpage|\\
|\||fi|
\end{tabular}
\end{center}
%
Here one could write a message such as:
\begin{center}
|This is the part \childdocname{} of \childdocjob{}.|
\end{center}

%%%%%%%%%%%%%%%%%%%%%%%%%%%%%%%%%%%%%%%%%%%%%%%%%%%%%%%%%%%%%%%%%%%%%%%%%%%%%%%%
\subsection{Flags}
\label{sec:flags}

The package makes it easy to generate different versions
of the main or child documents.
To this end compilation flags can be defined
and assigned different default values.
They will be particularly useful in conjunction
with the forwarding mechanism described in \secref{sec:forward}.

For example, it may be useful to have a flag |\version|
which can be set to |draft| or |final|.
The document source will contain some conditional code
depending on the value of |\version|.
Suppose further, the flag should default to |final| for the main file
and to |draft| for child files
which is a natural assignment for editing the document.
This is achieved by placing the following code
in the preamble of the main document
(below the |\childdocmain| directive):
%
\begin{center}
\begin{tabular}{l}
|\ifchilddoc|\\
|\providecommand{\version}{draft}|\\
|\||else|\\
|\providecommand{\version}{final}|\\
|\||fi|
\end{tabular}
\end{center}
%
The definition by |\providecommand| makes sure
that previous definitions are not overwritten.
Further statements |\providecommand{\version}{...}|
can thus be added before the above code to override it.

For the main file, one might add a line
(between |\childdocmain| and the above block)
%
\begin{center}
|%\ifchilddoc\||else\providecommand{\version}{draft}\||fi|
\end{center}
%
which can be uncommented to produce a draft version.
Likewise one can add a line to the very top of a child file
(above the |\childdocof{|\textit{main}|}| directive)
%
\begin{center}
|%\providecommand{\version}{final}|
\end{center}
%
which can be uncommented to produce the final version of this child document.

%%%%%%%%%%%%%%%%%%%%%%%%%%%%%%%%%%%%%%%%%%%%%%%%%%%%%%%%%%%%%%%%%%%%%%%%%%%%%%%%
\subsection{Forwarding}
\label{sec:forward}

Different versions of the main or child documents
using compilation flags as described in \secref{sec:flags}
can be (permanently) stored in different files
for convenient compilation, viewing and distribution.
To this end, the package defines a command
to pass on compilation to a different file:

%%%%%%%%%%%%%%%%%%%%%%%%%%%%%%%%%%%%%%%%
\DescribeMacro{\childdocforward}
The command |\childdocforward| redirects processing to
another source file:
%
\begin{center}
\begin{tabular}{l}
|\input{childdoc.def}|\\
|\childdocforward[|\textit{main}|]{|\textit{dest}|}|\\
\end{tabular}
\end{center}
%
The argument \textit{dest} is the destination file
(without extension).
It should be the main file or one of the child files.
Note that further \textsf{childdoc} directives
such as |\childdocof| and |\childdocforward|
in the indicated file will be processed in this form.
The optional argument \textit{main}
passes on directly to the main file \textit{main}
while pretending to compile the child \textit{dest}.
This form behaves as if \textit{dest}
issues |\childdocof{|\textit{main}|}| right away,
and no further \textsf{childdoc} directives will be processed.

%%%%%%%%%%%%%%%%%%%%%%%%%%%%%%%%%%%%%%%%
\DescribeMacro{\...prefix}
In the alternative form |\childdocforwardprefix|,
%
\begin{center}
\begin{tabular}{l}
|\input{childdoc.def}|\\
|\childdocforwardprefix[|\textit{main}|]{|\textit{prefix}|}{|\textit{dest}|}|
\end{tabular}
\end{center}
%
the destination file is determined by a pattern
depending on the current file:
To make this work, the current file must be called
`{\textit{prefix}\hspace{0.2em}\textit{suffix}}'
with \textit{prefix} matching precisely the argument.
Processing is then passed on to the file
`{\textit{dest}\hspace{0.2em}\textit{suffix}}'.
Surely, the same effect is achieved by
directly specifying the
argument `{\textit{dest}\hspace{0.2em}\textit{suffix}}'
in the first form.
However, that requires to set up a different file
for each child. With the alternative form of the command
all these files can have exactly the same content
which simplifies setting them up and maintaining them.

For example, the following file |draft.tex|
with a compilation flag |\version| as described in \secref{sec:flags}
compiles the main document as a draft:
%
\begin{center}
\begin{tabular}{l}
|\def\version{draft}|\\
|\input{childdoc.def}|\\
|\childdocforward{|\textit{main}|}|
\end{tabular}
\end{center}
%
Likewise, the following files |final|\textit{nn}|.tex|
compile the final version of the child document
|child|\textit{nn}|.tex|:
%
\begin{center}
\begin{tabular}{l}
|\def\version{final}|\\
|\input{childdoc.def}|\\
|\childdocforwardprefix{final}{child}|
\end{tabular}
\end{center}
%

Note that when several versions of a main file and/or of each child file
are to be generated, it may be convenient to set up a |Makefile| or
shell script to automatise the process.

%%%%%%%%%%%%%%%%%%%%%%%%%%%%%%%%%%%%%%%%%%%%%%%%%%%%%%%%%%%%%%%%%%%%%%%%%%%%%%%%
\subsection{Command Line Processing}
\label{sec:commandline}

The effect of redirection files can also be achieved by invoking
the \LaTeX{} compiler with a more elaborate command line.
Most conveniently this should be done as part
of a shell script or a |Makefile|.

When using \textsf{childdoc} in the main file, the following
command lines effectively perform a redirection
(note that depending on the shell being used,
backslashes may have to be doubled: `|\|' $\to$ `|\\|'):
%
\begin{center}
|... -jobname "|\textit{target}|" |\\|"|[\textit{flags}]%
|\input{childdoc.def}\childdocforward[|\textit{main}|]{|\textit{dest}|}"|
\end{center}
%
Here \textit{target} is the name of the output file,
\textit{main} is the name of the main file
and \textit{dest} is the name of the main or child file to be processed
(all filenames without extensions).
The optional argument \textit{main} can be omitted
if \textit{main} matches \textit{dest}.
Optionally, compilation \textit{flags} can be defined via |\def| commands.
This command line makes the \TeX{} engine believe
it is compiling the file \textit{target}
whose content is specified as the latter parameter.
The provided code then forwards the processing to
\textit{main} or \textit{dest} as described in \secref{sec:forward}.

%%%%%%%%%%%%%%%%%%%%%%%%%%%%%%%%%%%%%%%%%%%%%%%%%%%%%%%%%%%%%%%%%%%%%%%%%%%%%%%%
\subsection{Include by Input}
\label{sec:input}

Including child documents by |\include| has some restrictions by design.
Most notably, the content of a child document always occupies
its own set of pages; pages cannot be shared between child documents.
Usually, this behaviour makes perfect sense
because each child document contain an essential part of the document.
However, in some situations it may be desirable to compose
a document from a collection of parts
without having mandatory page breaks between then.
For this case, the package
provides a mechanism to include parts
by |\input| which can also be processed individually.
However, by construction this mechanism
requires manual handling of the content to be output.

%%%%%%%%%%%%%%%%%%%%%%%%%%%%%%%%%%%%%%%%
\DescribeMacro{\ifchilddocmanual}
The main file should be prepared as usual, see \secref{sec:include}.
However, the document body must make a distinction
between processing of an individual part and of the main document, e.g.:
%
\begin{center}
\begin{tabular}{l}
|\ifchilddocmanual|\\
|\input{\childdocname}|\\
|\||else|\\
\textit{document body with }|\input{|\textit{part}|}|\\
|\||fi|
\end{tabular}
\end{center}
%
The conditional |\ifchilddocmanual| is true whenever
a part to be included by |\input| is being compiled,
and the name of the part is stored in |\childdocname|.

%%%%%%%%%%%%%%%%%%%%%%%%%%%%%%%%%%%%%%%%
\DescribeMacro{\childdocby}
Each part to be included by |\input| should start with:
%
\begin{center}
\begin{tabular}{l}
|\input{childdoc.def}|\\
|\childdocby{|\textit{main}|}|\\
\end{tabular}
\end{center}
%
The directive |\childdocby| is similar to |\childdocof|
described in \secref{sec:include},
but the subsequent selection of content must be done manually.
To that end, both |\ifchilddoc| and |\ifchilddocmanual|
will be true upon processing of a part,
and the name of the part is stored in |\childdocname|.
Note that |\jobname| will be set to the filename of the current part
so that each part receives an individual |.aux| file
that does not interfere with the |.aux| file(s) of the main document.
This behaviour can be altered by the alternative form
|\childdocby[*]{|\textit{main}|}| (with a non-empty optional argument)
which uses the |.aux| file of the main document
by setting |\jobname| to \textit{main}.

%%%%%%%%%%%%%%%%%%%%%%%%%%%%%%%%%%%%%%%%%%%%%%%%%%%%%%%%%%%%%%%%%%%%%%%%%%%%%%%%
\subsection{Driver Development}
\label{sec:driver}

The \textsf{childdoc} mechanism can also be use for the development
of definition files such as \LaTeX{} styles or classes.
This case differs from the above setup with multiple parts
included by |\include| in that no |\includeonly| should be invoked.
This can be achieved by starting the include file
(before |\ProvidesPackage|) with:
%
\begin{center}
\begin{tabular}{l}
|\input{childdoc.def}|\\
|\childdocforward{|\textit{main}|}|\\
\end{tabular}
\end{center}
%
or alternatively with:
%
\begin{center}
\begin{tabular}{l}
|\input{childdoc.def}|\\
|\childdocby{|\textit{main}|}|\\
\end{tabular}
\end{center}
%
Both forms have slightly different effects as described above.
The main file is prepared as usual, see \secref{sec:include}.

%%%%%%%%%%%%%%%%%%%%%%%%%%%%%%%%%%%%%%%%%%%%%%%%%%%%%%%%%%%%%%%%%%%%%%%%%%%%%%%%
\subsection{Legacy Detection}
\label{sec:detection}

The directive |\childdocmain| in the main file can detect
whether the complete document or merely a child is to be compiled
even without using the directive |\childdocof|.
This method is deprecated because it is less robust
and there is no compelling reason to use it;
it is merely provided for backward compatibility
and it may be removed in future versions.

If the detection mechanism is to be used,
it is mandatory to correctly specify
the filename of the main file as the argument of |\childdocmain|:
%
\begin{center}
\begin{tabular}{l}
|\input{childdoc.def}|\\
|\childdocmain{|\textit{main}|}|\\
\end{tabular}
\end{center}
%
If |\jobname| does not match the argument \textit{main} of |\childdocmain|,
it is assumed that |\jobname| points to the child file to be compiled.
When using |\childdocmain| with the main file specified as argument,
it suffices to start a child file
with just |\input{|\textit{main}|}|
without loading of the package and using |\childdocof|.
If instead all processing is done
with the appropriate \textsf{childdoc} directives,
the argument of \textit{main} of |\childdocmain| can be empty.

An alternative version of the command line processing described
in \secref{sec:commandline} using the detection mechanism reads:
%
\begin{center}
|... -jobname "|\textit{target}|" "|[\textit{flags}]%
[|\def\jobname{|\textit{dest}|}|]|\input{|\textit{main}|}"|
\end{center}

%%%%%%%%%%%%%%%%%%%%%%%%%%%%%%%%%%%%%%%%%%%%%%%%%%%%%%%%%%%%%%%%%%%%%%%%%%%%%%%%
\subsection{Manual Code}
\label{sec:manual}

In case one cannot be certain whether the definitions file |childdoc.def|
is installed on the target \TeX{} distribution
and one prefers not to ship it,
it is conceivable to paste a few relevant commands into the sources.

To that end, drop all statements |\input{childdoc.def}|
and perform the replacements as outlined below.
Instead of |\childdocmain{|\textit{main}|}| add the following code
to the top of the main file:
%
\begin{center}
\begin{tabular}{l}
|\||ifdefined\childdocname\endinput\||fi\newif\ifchilddoc|\\
|\edef\childdocname{\scantokens\expandafter{\jobname\noexpand}}|\\
|\def\childdocmain{|\textit{main}|}\||ifx\childdocmain\childdocname\||else|\\
|\childdoctrue\includeonly{\childdocname}\let\jobname\childdocmain\||fi|\\
\end{tabular}
\end{center}
%
Instead of |\childdocof{|\textit{main}|}| just include the main file
at the top of each child file:
%
\begin{center}
|\input{|\textit{main}|}|
\end{center}
%
A simple redirection |\childdocforward{|\textit{dest}|}| is achieved by:
%
\begin{center}
|\def\jobname{|\textit{dest}|}\input{\jobname}|
\end{center}
%
The redirection with prefix
|\childdocforwardprefix[|\textit{prefix}|]{|\textit{dest}|}|
is accomplished by:
%
\begin{center}
\begin{tabular}{l}
|{\edef\jobname{\scantokens\expandafter{\jobname\noexpand}}|\\
|\def\redirectjob |\textit{prefix}|#1~~~{\gdef\jobname{|\textit{dest}|#1}}|\\
|\expandafter\redirectjob\jobname~~~}\input{\jobname}|
\end{tabular}
\end{center}

In an alternative approach,
child documents can be compiled by a specific command line
without additional code or specific definitions:
%
\begin{center}
|... -jobname "|\textit{target}|" "|[\textit{flags}]%
|\includeonly{|\textit{dest}|}\input{|\textit{main}|}"|
\end{center}
%

%%%%%%%%%%%%%%%%%%%%%%%%%%%%%%%%%%%%%%%%%%%%%%%%%%%%%%%%%%%%%%%%%%%%%%%%%%%%%%%%
%%%%%%%%%%%%%%%%%%%%%%%%%%%%%%%%%%%%%%%%%%%%%%%%%%%%%%%%%%%%%%%%%%%%%%%%%%%%%%%%
\section{Information}

%%%%%%%%%%%%%%%%%%%%%%%%%%%%%%%%%%%%%%%%%%%%%%%%%%%%%%%%%%%%%%%%%%%%%%%%%%%%%%%%
\subsection{Copyright}

Copyright \copyright{} 2017--2018 Niklas Beisert

This work may be distributed and/or modified under the
conditions of the \LaTeX{} Project Public License, either version 1.3
of this license or (at your option) any later version.
The latest version of this license is in
  \url{http://www.latex-project.org/lppl.txt}
and version 1.3 or later is part of all distributions of \LaTeX{}
version 2005/12/01 or later.

This work has the LPPL maintenance status `maintained'.

The Current Maintainer of this work is Niklas Beisert.

This work consists of the files |README.txt|, |childdoc.ins| and |childdoc.dtx|
as well as the derived files |childdoc.def|, |cdocsamp.tex|
with |cdocsch1.tex|, |cdocsch2.tex|, |cdocspt3.tex|, |cdocspt4.tex|,
|cdocsdrf.tex|, |cdocsfn1.tex|, |cdocsfn2.tex|
as well as |childdoc.pdf|.

%%%%%%%%%%%%%%%%%%%%%%%%%%%%%%%%%%%%%%%%%%%%%%%%%%%%%%%%%%%%%%%%%%%%%%%%%%%%%%%%
\subsection{Files and Installation}

The package consists of the files:
%
\begin{center}
\begin{tabular}{ll}
    |README.txt|   & readme file \\
    |childdoc.ins| & installation file \\
    |childdoc.dtx| & source file \\
    |childdoc.def| & definition file \\
    |cdocsamp.tex| & sample main file \\
    |cdocsch1.tex| & sample include file \\
    |cdocsch2.tex| & sample include file \\
    |cdocspt3.tex| & sample part file \\
    |cdocspt4.tex| & sample part file \\
    |cdocsdrf.tex| & sample redirection file \\
    |cdocsfn1.tex| & sample redirection file \\
    |cdocsfn2.tex| & sample redirection file \\
    |childdoc.pdf| & manual
\end{tabular}
\end{center}
%
The distribution consists of the files
|README.txt|, |childdoc.ins| and |childdoc.dtx|.
%
\begin{itemize}
\item
Run (pdf)\LaTeX{} on |childdoc.dtx|
to compile the manual |childdoc.pdf| (this file).
\item
Run \LaTeX{} on |childdoc.ins| to create the definitions file |childdoc.def|
and the sample |cdocsamp.tex| with include files
|cdocsch1.tex|, |cdocsch2.tex|, |cdocspt3.tex|, |cdocspt4.tex|,
|cdocsdrf.tex|, |cdocsfn1.tex|, |cdocsfn2.tex|.
Then copy the file |childdoc.def| to an appropriate directory of your \LaTeX{}
distribution, e.g.\ \textit{texmf-root}|/tex/latex/childdoc|.
\end{itemize}

%%%%%%%%%%%%%%%%%%%%%%%%%%%%%%%%%%%%%%%%%%%%%%%%%%%%%%%%%%%%%%%%%%%%%%%%%%%%%%%%
\subsection{Related CTAN Packages}

There are several other packages which offer a similar functionality:
%
\begin{itemize}
\item
The packages
\href{http://ctan.org/pkg/docmute}{\textsf{docmute}},
\href{http://ctan.org/pkg/includex}{\textsf{includex}} and
\href{http://ctan.org/pkg/standalone}{\textsf{standalone}}
provide commands to include only the document body of
a child file thus allowing both files to be compiled individually.
\item
The packages \href{http://ctan.org/pkg/subdocs}{\textsf{subdocs}}
and \href{http://ctan.org/pkg/subfiles}{\textsf{subfiles}}
provide structures in which the main and child documents can be
encapsulated and allowing them to be compiled individually.
The inclusion mechanism is different from the conventional |\include|.
\item
The package \href{http://ctan.org/pkg/combine}{\textsf{combine}}
is an elaborate solution to combine several documents into one.
\end{itemize}
%
See also the CTAN topic \href{http://ctan.org/topic/subdocs}{\textsf{subdocs}}
for further related packages.
The present package differs from the above solutions in that
a document structure constructed with the conventional |\include| mechanism
just needs two extra commands at the top of every file
such that all constituent files can be compiled individually.

%%%%%%%%%%%%%%%%%%%%%%%%%%%%%%%%%%%%%%%%%%%%%%%%%%%%%%%%%%%%%%%%%%%%%%%%%%%%%%%%
%\subsection{Feature Suggestions}
%
%The following is a list of features which may be useful for future
%versions of this package:
%%
%\begin{itemize}
%\item
%\ldots
%\end{itemize}

%%%%%%%%%%%%%%%%%%%%%%%%%%%%%%%%%%%%%%%%%%%%%%%%%%%%%%%%%%%%%%%%%%%%%%%%%%%%%%%%
\subsection{Revision History}

%%%%%%%%%%%%%%%%%%%%%%%%%%%%%%%%%%%%%%%%
\paragraph{v2.0:} 2018/12/30

\begin{itemize}
\item
immediate forward processing
\item
added |\childdocby| mechanism
\item
manual restructured
\end{itemize}

%%%%%%%%%%%%%%%%%%%%%%%%%%%%%%%%%%%%%%%%
\paragraph{v1.6:} 2018/01/17

\begin{itemize}
\item
application for development of include files
\item
corrections to manual
\end{itemize}

%%%%%%%%%%%%%%%%%%%%%%%%%%%%%%%%%%%%%%%%
\paragraph{v1.5:} 2017/05/21

\begin{itemize}
\item
more complete structuring introduced
\item
|\childdocof| introduced
\item
|\childdoc| renamed to |\childdocmain|
\item
|\childredirect| renamed to |\childdocforward| and |\childdocforwardprefix|
and functionality expanded
\end{itemize}

%%%%%%%%%%%%%%%%%%%%%%%%%%%%%%%%%%%%%%%%
\paragraph{v1.0:} 2017/04/27

\begin{itemize}
\item
manual and install package
\item
first version published on CTAN
\end{itemize}

%%%%%%%%%%%%%%%%%%%%%%%%%%%%%%%%%%%%%%%%
\paragraph{v0.6:} 2017/04/26

\begin{itemize}
\item
redirection mechanism added
\end{itemize}

%%%%%%%%%%%%%%%%%%%%%%%%%%%%%%%%%%%%%%%%
\paragraph{v0.5:} 2017/04/26

\begin{itemize}
\item
functionality in definition file
\end{itemize}


%%%%%%%%%%%%%%%%%%%%%%%%%%%%%%%%%%%%%%%%%%%%%%%%%%%%%%%%%%%%%%%%%%%%%%%%%%%%%%%%
%%%%%%%%%%%%%%%%%%%%%%%%%%%%%%%%%%%%%%%%%%%%%%%%%%%%%%%%%%%%%%%%%%%%%%%%%%%%%%%%
%%%%%%%%%%%%%%%%%%%%%%%%%%%%%%%%%%%%%%%%%%%%%%%%%%%%%%%%%%%%%%%%%%%%%%%%%%%%%%%%
\appendix

\settowidth\MacroIndent{\rmfamily\scriptsize 000\ }

 \DocInput{childdoc.dtx}

\end{document}
%</driver>
% \fi
%
% %%%%%%%%%%%%%%%%%%%%%%%%%%%%%%%%%%%%%%%%%%%%%%%%%%%%%%%%%%%%%%%%%%%%%%%%%%%%%%
% %%%%%%%%%%%%%%%%%%%%%%%%%%%%%%%%%%%%%%%%%%%%%%%%%%%%%%%%%%%%%%%%%%%%%%%%%%%%%%
% \section{Sample}
%\iffalse
%<*samplemain>
%\fi
%
% The following presents a sample document
% with two chapters, two parts, a title page,
% a compile flag as well as three forwarding files to set the flag.
% It consists of eight |.tex| files:
% \begin{center}
% \begin{tabular}{ll}
% |cdocsamp.tex|&main file\\
% |cdocsch1.tex|&include file for chapter 1\\
% |cdocsch2.tex|&include file for chapter 2\\
% |cdocspt3.tex|&include file for part 3\\
% |cdocspt4.tex|&include file for part 4\\
% |cdocsdrf.tex|&forwarding file for main file in draft mode\\
% |cdocsfi1.tex|&forwarding file for final version of chapter 1\\
% |cdocsfi2.tex|&forwarding file for final version of chapter 2\\
% \end{tabular}
% \end{center}
% Each of the eight files can be compiled directly by the \LaTeX{} compiler.
%
% %%%%%%%%%%%%%%%%%%%%%%%%%%%%%%%%%%%%%%
% \paragraph{Main File.}
%
% The main file is called |cdocsamp.tex|.
%
% Load the \textsf{childdoc} definitions and
% declare the filename for the main document:
%    \begin{macrocode}
\input{childdoc.def}
\childdocmain{}
%    \end{macrocode}

% Optional override for |\version| flag:
%    \begin{macrocode}
%%\ifchilddoc\else\providecommand{\version}{draft}\fi
%    \end{macrocode}

% Define the default values for the |\version| flag
% (|final| for the main file and |draft| for childs):
%    \begin{macrocode}
\ifchilddoc
\providecommand{\version}{draft}
\else
\providecommand{\version}{final}
\fi
%    \end{macrocode}

% Load the standard document class:
%    \begin{macrocode}
\documentclass[12pt]{article}
%    \end{macrocode}

% Start the document body:
%    \begin{macrocode}
\begin{document}
%    \end{macrocode}

% Declare a title page.
% Print title, part of document being processed and version flag:
%    \begin{macrocode}
\addtocounter{page}{-1}
\begin{center}
{\LARGE\bfseries{}childdoc example\par}
\vspace{1cm}
\ifchilddoc
\ifchilddocmanual part\else chapter\fi:
`\childdocname' of `\childdocjob'\par
\else
main document: `\childdocjob'\par
\fi
version: \version\par
\end{center}
\newpage
%    \end{macrocode}

% Manually include selected file,
% otherwise process as usual:
%    \begin{macrocode}
\ifchilddocmanual
\section*{part `\childdocname'}
\input{\childdocname}
\else
%    \end{macrocode}

% Include the two chapters:
%    \begin{macrocode}
\include{cdocsch1}
\include{cdocsch2}
%    \end{macrocode}

% Include the two parts unless only chapters should be displayed:
%    \begin{macrocode}
\ifchilddoc\else
\section{part three}
\input{cdocspt3}
\section{part four}
\input{cdocspt4}
\fi
%    \end{macrocode}

% Process as usual until here:
%    \begin{macrocode}
\fi
%    \end{macrocode}

% End of document body:
%    \begin{macrocode}
\end{document}
%    \end{macrocode}
%\iffalse
%</samplemain>
%\fi
%
% %%%%%%%%%%%%%%%%%%%%%%%%%%%%%%%%%%%%%%
% \paragraph{Chapter Include Files.}
%
% The include files are called |cdocsch1.tex| and |cdocsch2.tex|.
%
%\iffalse
%<*samplechap1|samplechap2>
%\fi

% Optional override for |\version| flag:
%    \begin{macrocode}
%%\providecommand{\version}{final}
%    \end{macrocode}

% Include the main document:
%    \begin{macrocode}
\input{childdoc.def}
\childdocof{cdocsamp}
%    \end{macrocode}

%\iffalse
%</samplechap1|samplechap2>
%\fi
%
%\iffalse
%<*samplechap1>
%\fi
% Some text for chapter 1:
%    \begin{macrocode}
\section{one}
some text in chapter one
%    \end{macrocode}

%\iffalse
%</samplechap1>
%\fi
% Some text for chapter 2:
%\iffalse
%<*samplechap2>
%\fi
%    \begin{macrocode}
\section{two}
more text in chapter two
%    \end{macrocode}

%\iffalse
%</samplechap2>
%\fi
%
% %%%%%%%%%%%%%%%%%%%%%%%%%%%%%%%%%%%%%%
% \paragraph{Part Include Files.}
%
% The include files are called |cdocspt3.tex| and |cdocspt4.tex|.
%
%\iffalse
%<*samplepart3|samplepart4>
%\fi

% Optional override for |\version| flag:
%    \begin{macrocode}
%%\providecommand{\version}{final}
%    \end{macrocode}

% Include the main document:
%    \begin{macrocode}
\input{childdoc.def}
\childdocby{cdocsamp}
%    \end{macrocode}

%\iffalse
%</samplepart3|samplepart4>
%\fi
%
%\iffalse
%<*samplepart3>
%\fi
% Some text for part 3:
%    \begin{macrocode}
some text in part three
%    \end{macrocode}

%\iffalse
%</samplepart3>
%\fi
% Some text for part 4:
%\iffalse
%<*samplepart4>
%\fi
%    \begin{macrocode}
more text in part four
%    \end{macrocode}

%\iffalse
%</samplepart4>
%\fi
%
% %%%%%%%%%%%%%%%%%%%%%%%%%%%%%%%%%%%%%%
% \paragraph{Forwarding for a Complete Draft.}
%
% The following forwarding file |cdocsdrf.tex|
% compiles the main document in draft mode:
%\iffalse
%<*sampledraft>
%\fi
%    \begin{macrocode}
\def\version{draft}
\input{childdoc.def}
\childdocforward{cdocsamp}
%    \end{macrocode}

%\iffalse
%</sampledraft>
%\fi
%
% %%%%%%%%%%%%%%%%%%%%%%%%%%%%%%%%%%%%%%
% \paragraph{Forwarding for Final Version of the Chapters.}
%
% The following forwarding files |cdocsfn1.tex| and |cdocsfn2.tex|
% (with identical content)
% compile the final versions of the child documents
% |cdocsch1.tex| and |cdocsch2.tex|, respectively:
%\iffalse
%<*samplefinal>
%\fi
%    \begin{macrocode}
\def\version{final}
\input{childdoc.def}
\childdocforwardprefix[cdocsamp]{cdocsfn}{cdocsch}
%    \end{macrocode}

%\iffalse
%</samplefinal>
%\fi
%
% %%%%%%%%%%%%%%%%%%%%%%%%%%%%%%%%%%%%%%
% \paragraph{Command Line Processing.}
%
% The following three command lines generate the output files
% |cdocscld|, |cdocscl1| and |cdocscl2|
% which should be identical to
% |cdocsdrf|, |cdocsch1| and |cdocsfn2|, respectively:
% \begin{center}
% \begin{tabular}{l}
% |latex -jobname cdocscld \|\\
% |  "\def\version{draft}\input{childdoc.def}\childdocforward{cdocsamp}"|\\
% |latex -jobname cdocscl1 \|\\
% |  "\input{childdoc.def}\childdocforward[cdocsamp]{cdocsch1}"|\\
% |latex -jobname cdocscl2 \|\\
% |  "\def\version{final}\input{childdoc.def}\childdocforward{cdocsch2}"|
% \end{tabular}
% \end{center}
% Note that the trailing backslash on each first line
% merely continues the input to the second line
% (for convenient cut ant paste).
% Furthermore, the command |latex| can be replaced by any
% of its alternative versions such as |pdflatex|.
%
% %%%%%%%%%%%%%%%%%%%%%%%%%%%%%%%%%%%%%%%%%%%%%%%%%%%%%%%%%%%%%%%%%%%%%%%%%%%%%%
% %%%%%%%%%%%%%%%%%%%%%%%%%%%%%%%%%%%%%%%%%%%%%%%%%%%%%%%%%%%%%%%%%%%%%%%%%%%%%%
% \section{Implementation}
%\iffalse
%<*package>
%\fi
%
% This section describes the definitions file |childdoc.def|.

% The definitions cannot be loaded using |\usepackage| or |\RequirePackage|
% which has a mechanism to prevent loading a style file more than once.
% When loading the definitions by means of |\input|
% multiple instances have to be prevented manually:
%\iffalse
%This code needs to be before the `\ProvidesFile' directive
%which is defined at the beginning of this file.
%Therefore it is also placed there and commented out here.
%</package>
%<*discard>
%\fi
%    \begin{macrocode}
\ifdefined\childdocmain\endinput\fi
%    \end{macrocode}
%\iffalse
%</discard>
%<*package>
%\fi
%
% \macro{\ifchilddoc}
% \macro{\ifchilddocmanual}
% The conditional |\ifchilddoc| tells whether a
% child (true) or main (false) document is being compiled.
% The conditional |\ifchilddocmanual| tells whether
% the |\includeonly| mechanism is used (false) or
% the selection of child files must be performed manually (true).
% The definitions initialise to false:
%    \begin{macrocode}
\newif\ifchilddoc
\newif\ifchilddocmanual
%    \end{macrocode}

% \macro{\childdocname}
% \macro{\childdocjob}
% The macro |\childdocname| stores the name of the main document
% to be compiled. The macro |\childdocjob| stores the name of
% the document on which the \LaTeX{} compiler was originally invoked.
% The content of |\jobname| cannot be compared
% to filenames specified in the source due to different catcodes.
% The following code rescans |\jobname|, stores the result
% in |\childdocname| and saves a copy in |\childdocjob|:
%    \begin{macrocode}
\edef\childdocname{\scantokens\expandafter{\jobname\noexpand}}
\let\childdocjob\childdocname
%    \end{macrocode}

% \macro{\childdocdisable}
% The macro |\childdocdisable| prevents the main file
% from being processed more than once.
% At this stage, the main document command |\childdocmain|
% is assumed to be called once again where it should do nothing.
% Any subsequent call to it should prevent
% a secondary processing of the main document
% It overwrites the forwarding commands
% |\childdocof| and |\childdocforward|
% with empty macros to prevent further inclusions of the main document:
%    \begin{macrocode}
\newcommand{\childdocdisable}
{
  \renewcommand{\childdocmain}[1]{\renewcommand{\childdocmain}[1]{\endinput}}
  \renewcommand{\childdocof}[1]{}
  \renewcommand{\childdocby}[2][]{}
  \renewcommand{\childdocforward}[2][]{}
  \renewcommand{\childdocdisable}{}
}
%    \end{macrocode}

% \macro{\childdocmain}
% The macro |\childdocmain| is to be called at the top of the main file
% with nothing or the main filename (without extension) as argument.
% First, it breaks loops.
% If the argument is not empty and does not match |\childdocname|
% (which is set by the first inclusion of |childdoc.def|),
% |\ifchilddoc| is set to true, |\includeonly| is applied to the child file
% and |\jobname| is set to the main file
% (for proper handling of |.aux| files):
%    \begin{macrocode}
\newcommand{\childdocmain}[1]
{
  \childdocdisable\childdocmain{}
  \if?#1?\else
    \begingroup
      \def\childdoctmp{#1}
      \ifx\childdoctmp\childdocname
        \def\childdoctmp{}
      \else
        \def\childdoctmp
        {
          \childdoctrue
          \includeonly{\childdocname}
          \def\childdocjob{#1}
          \def\jobname{#1}
        }
      \fi
      \expandafter
    \endgroup
    \childdoctmp
  \fi
}
%    \end{macrocode}

% \macro{\childdocof}
% The command |\childdocof| redirects
% compilation to the main file |#1|.
%    \begin{macrocode}
\newcommand{\childdocof}[1]
{
  \childdocdisable
  \childdoctrue
  \includeonly{\childdocname}
  \def\jobname{#1}
  \def\childdocjob{#1}
  \input{#1}
}
%    \end{macrocode}

% \macro{\childdocby}
% The command |\childdocby| ....
%    \begin{macrocode}
\newcommand{\childdocby}[2][]
{
  \childdocdisable
  \childdoctrue
  \childdocmanualtrue
  \if?#1?\else
    \def\jobname{#2}
  \fi
  \def\childdocjob{#2}
  \input{#2}
  \endinput
}
%    \end{macrocode}

% \macro{\childdocforward}
% The command |\childdocforward| redirects
% compilation to the main file or
% (if the optional argument is given) a child file.
% Parameters are set as if the main file
% or a child file starting with |\childdocof| was compiled.
% Then compilation is handed over to the main file:
%    \begin{macrocode}
\newcommand{\childdocforward}[2][]
{
  \begingroup
    \if?#1?
      \def\childdoctmp
      {
        \def\childdocname{#2}
        \def\childdocjob{#2}
        \def\jobname{#2}
        \input{#2}
        \endinput
      }
    \else
      \def\childdoctmp
      {
        \childdocdisable
        \def\childdocname{#2}
        \childdoctrue
        \includeonly{#2}
        \def\childdocjob{#1}
        \def\jobname{#1}
        \input{#1}
        \endinput
      }
    \fi
    \expandafter
  \endgroup
  \childdoctmp
}
%    \end{macrocode}

% \macro{\childdocforwardprefix}
% The command |\childdocforwardprefix| redirects
% compilation to the main or a child file by means of a pattern.
% The prefix |#1| in the current filename is replaced by |#2|
% and the suffix of the current filename is kept
% (it is assumed that the filename does not contain the substring `|~~~|'
% which is used as a delimiter).
% Compilation is handed over to the new file by |\childdocforward|:
%    \begin{macrocode}
\newcommand{\childdocforwardprefix}[3][]
{
  \begingroup
    \def\childdocextract #2##1~~~{\def\childdoctmp{\childdocforward[#1]{#3##1}}}
    \expandafter\childdocextract\childdocname~~~
    \expandafter
  \endgroup
  \childdoctmp
}
%    \end{macrocode}

% \macro{\childdoc}
% The deprecated macro |\childdoc| is a legacy version of |\childdocmain|:
%    \begin{macrocode}
\newcommand{\childdoc}{\childdocmain}
%    \end{macrocode}

% \macro{\childdocredirect}
% The deprecated macro |\childdocredirect| is a legacy version
% of |\childdocforward| and |\childdocforwardprefix|:
%    \begin{macrocode}
\newcommand{\childdocredirect}[2][]
{
  \begingroup
    \if?#1?
      \def\childdoctmp{\childdocforward{#2}}
    \else
      \def\childdoctmp{\childdocforwardprefix{#1}{#2}}
    \fi
    \expandafter
  \endgroup
  \childdoctmp
}
%    \end{macrocode}

%\iffalse
%</package>
%\fi
%
\endinput

\childdocforward{cdocsamp}
%    \end{macrocode}

%\iffalse
%</sampledraft>
%\fi
%
% %%%%%%%%%%%%%%%%%%%%%%%%%%%%%%%%%%%%%%
% \paragraph{Forwarding for Final Version of the Chapters.}
%
% The following forwarding files |cdocsfn1.tex| and |cdocsfn2.tex|
% (with identical content)
% compile the final versions of the child documents
% |cdocsch1.tex| and |cdocsch2.tex|, respectively:
%\iffalse
%<*samplefinal>
%\fi
%    \begin{macrocode}
\def\version{final}
% \iffalse
%
% childdoc.dtx Copyright (C) 2017-2018 Niklas Beisert
%
% This work may be distributed and/or modified under the
% conditions of the LaTeX Project Public License, either version 1.3
% of this license or (at your option) any later version.
% The latest version of this license is in
%   http://www.latex-project.org/lppl.txt
% and version 1.3 or later is part of all distributions of LaTeX
% version 2005/12/01 or later.
%
% This work has the LPPL maintenance status `maintained'.
%
% The Current Maintainer of this work is Niklas Beisert.
%
% This work consists of the files childdoc.dtx and childdoc.ins
% and the derived files childdoc.def and cdocsamp.tex with
% cdocsch1.tex, cdocsch2.tex, cdocsdrf.tex, cdocsfn1.tex, cdocsfn2.tex.
%
%<package>\ifdefined\childdocmain\endinput\fi
%<package>\ProvidesFile{childdoc.def}[2018/12/30 v2.0 child document driver]
%<samplemain>\ProvidesFile{cdocsamp.tex}[2018/12/30 v2.0 sample for childdoc]
%<*driver>
%\ProvidesFile{childdoc.drv}[2018/12/30 v2.0 childdoc reference manual file]
\PassOptionsToClass{10pt,a4paper}{article}
\documentclass{ltxdoc}

\usepackage[margin=35mm]{geometry}
\usepackage{hyperref}
\usepackage{hyperxmp}
\usepackage[usenames]{color}

\hypersetup{colorlinks=true}
\hypersetup{pdfstartview=FitH}
\hypersetup{pdfpagemode=UseNone}
\hypersetup{pdfsource={}}
\hypersetup{pdflang={en-UK}}
\hypersetup{pdfcopyright={Copyright 2017-2018 Niklas Beisert.
  This work may be distributed and/or modified under the
  conditions of the LaTeX Project Public License, either version 1.3
  of this license or (at your option) any later version.}}
\hypersetup{pdflicenseurl={http://www.latex-project.org/lppl.txt}}
\hypersetup{pdfcontactaddress={ETH Zurich, ITP, HIT K,
  Wolfgang-Pauli-Strasse 27}}
\hypersetup{pdfcontactpostcode={8093}}
\hypersetup{pdfcontactcity={Zurich}}
\hypersetup{pdfcontactcountry={Switzerland}}
\hypersetup{pdfcontactemail={nbeisert@itp.phys.ethz.ch}}
\hypersetup{pdfcontacturl={http://people.phys.ethz.ch/\xmptilde nbeisert/}}

\newcommand{\secref}[1]{\hyperref[#1]{section \ref*{#1}}}

\parskip1ex
\parindent0pt
\let\olditemize\itemize
\def\itemize{\olditemize\parskip0pt}

\begin{document}

\title{The \textsf{childdoc} Package}
\hypersetup{pdftitle={The childdoc Package}}
\author{Niklas Beisert\\[2ex]
  Institut f\"ur Theoretische Physik\\
  Eidgen\"ossische Technische Hochschule Z\"urich\\
  Wolfgang-Pauli-Strasse 27, 8093 Z\"urich, Switzerland\\[1ex]
  \href{mailto:nbeisert@itp.phys.ethz.ch}
  {\texttt{nbeisert@itp.phys.ethz.ch}}}
\hypersetup{pdfauthor={Niklas Beisert}}
\hypersetup{pdfsubject={Manual for the LaTeX2e Package childdoc}}
\date{30 December 2018, \textsf{v2.0}}
\maketitle

\begin{abstract}\noindent
\textsf{childdoc} is a \LaTeXe{} package
that enables the direct compilation
of document sections included by |\include|
to individual files.
\end{abstract}

\begingroup
\parskip0ex
\tableofcontents
\endgroup

%%%%%%%%%%%%%%%%%%%%%%%%%%%%%%%%%%%%%%%%%%%%%%%%%%%%%%%%%%%%%%%%%%%%%%%%%%%%%%%%
%%%%%%%%%%%%%%%%%%%%%%%%%%%%%%%%%%%%%%%%%%%%%%%%%%%%%%%%%%%%%%%%%%%%%%%%%%%%%%%%
\section{Introduction}

\LaTeX{} provides a mechanism to structure a large document (such as a book)
into a main file and several child files (containing the chapters)
using the |\include| command.
This mechanism is beneficial for documents
which span hundreds of pages in order to
make the source file(s) more manageable.
Moreover, compilation can be restricted to
selected child files by means of the |\includeonly| command.
The latter feature can be used to reduce the compilation time while editing
(this was significantly more useful in the earlier days of \LaTeX{})
or to generate a smaller document which is easier to navigate.
Another application of |\includeonly| is to generate
documents consisting of selected parts of the complete document.

However, there are a few drawbacks of the plain |\include| mechanism:
\begin{itemize}
\item
The child files cannot be compiled on their own,
they can only be compiled via the main file.
A naive editing environment
(such as a text editor with an option
to have the current file processed by \LaTeX)
may require one to switch to the main file before compiling;
attempting to compile the child file produces errors.
\item
The main file must be modified (each time)
to adjust the |\includeonly| command
to the present needs. This easily leaves the main file in a messy state.
\item
The generated document will always carry the filename
of the main document. This is inconvenient if
several child files are to be compiled and
to be kept for distribution.
\end{itemize}

The present package provides a simple interface
to make child files individually compilable by \LaTeX{}.
Compiling a child file then has the same effect as compiling
the main file with an |\includeonly| command
to select the appropriate child.
Moreover the generated document will carry the name of the child
rather than the main file.
This resolves all three above issues.

This feature is meant to make the editing of books,
thesis documents and lecture notes somewhat more convenient.
However, the package can also be used efficiently for
composing a series of documents (such as exercise sheets)
which are typically distributed individually.
It then assists the author in generating the individual documents
(potentially in different versions)
as well as a document containing the collected series.
Another application is in developing style files
or other kinds of included material
where compilation of the style file could redirect
to a sample or test file.

%%%%%%%%%%%%%%%%%%%%%%%%%%%%%%%%%%%%%%%%%%%%%%%%%%%%%%%%%%%%%%%%%%%%%%%%%%%%%%%%
%%%%%%%%%%%%%%%%%%%%%%%%%%%%%%%%%%%%%%%%%%%%%%%%%%%%%%%%%%%%%%%%%%%%%%%%%%%%%%%%
\section{Usage}

First of all, the package \textsf{childdoc} is \emph{not} a standard
\LaTeXe{} |.sty| style file! Therefore it needs to be invoked in
a non-standard way.

%%%%%%%%%%%%%%%%%%%%%%%%%%%%%%%%%%%%%%%%%%%%%%%%%%%%%%%%%%%%%%%%%%%%%%%%%%%%%%%%
\subsection{Included Files}
\label{sec:include}

%%%%%%%%%%%%%%%%%%%%%%%%%%%%%%%%%%%%%%%%
\DescribeMacro{\childdocmain}
To use the package, add the commands
\begin{center}
\begin{tabular}{l}
|\input{childdoc.def}|\\
|\childdocmain{}|\\
\end{tabular}
\end{center}
at the very top of the main \LaTeX{} file,
in particular \emph{before} the |\documentclass| statement!
The argument of |\childdocmain| should be left empty
(but it must be present).

%%%%%%%%%%%%%%%%%%%%%%%%%%%%%%%%%%%%%%%%
\DescribeMacro{\childdocof}
Furthermore, add the commands
\begin{center}
\begin{tabular}{l}
|\input{childdoc.def}|\\
|\childdocof{|\textit{main}|}|\\
\end{tabular}
\end{center}
at the top of every child file \textit{child}
which is included by |\include{|\textit{child}|}|
from within the main file
(or at least for those files to be compiled individually).
The argument \textit{main} must be the filename of the main file.

There are a couple of
considerations in setting up the main and child documents:

%%%%%%%%%%%%%%%%%%%%%%%%%%%%%%%%%%%%%%%%
\paragraph{Restrictions.}

Please note the following restrictions:
\begin{itemize}
\item
|\childdocmain| must be called with one argument \textit{main}
to ensure compatibility with earlier version of the package.
It must either be empty (|\childdocmain{}|)
or precisely match the filename of the main file in which it is specified.
See \secref{sec:detection} for further information.
\item
The filename \textit{main} must be specified without the |.tex| extension.
\item
The filename \textit{main} is case sensitive
(even in case-insensitive file systems)
due to internal string comparison.
\item
The argument \textit{main} should be fully expanded, it cannot be a macro.
\item
Subdirectories and special characters should be avoided in filenames.
\item
The command |\childdocmain{|\textit{main}|}| must be followed by a whitespace.
It should not be followed immediately by another command
or by a comment mark `|%|'.
This is because the \TeX{} parser reads the token immediately following
the argument of |\childdocmain| and puts it
at the beginning of every child section;
however, a white\-space is ignored.
\end{itemize}

%%%%%%%%%%%%%%%%%%%%%%%%%%%%%%%%%%%%%%%%
\paragraph{Content of Main File.}

It is advisable to place all content in the child files included by |\include|.
Any output contained in the main file will appear in all child documents
unless suppressed manually;
it cannot be suppressed automatically by the |\includeonly| directive
and thus should normally be avoided.
A method to include some content in the main file
by means of conditional processing is described in \secref{sec:conditional}.

%%%%%%%%%%%%%%%%%%%%%%%%%%%%%%%%%%%%%%%%
\paragraph{Page Numbering.}

When only a part of the document is compiled,
the appropriate numbering of pages
(as well as other status parameters)
is determined from the |.aux| files.
The latter contain information from previous passes.
However this information needs to propagate through
all intermediate child documents.
Therefore the page numbering in child documents may well
be inconsistent until the complete document is compiled at least once.

A useful (if unconventional) way to always ensure a consistent
page numbering is to restart the numbering in each child document
and denote the pages by `\textit{child}|.|\textit{page}'
where \textit{child} represents the chapter/section number of the child file.
This can be achieved by the command
|\numberwithin{page}{|\textit{child}|}|
of the \textsf{amsmath} package
where \textit{child} can be |chapter| or |section|
depending on the chosen structuring.
Alternatively, one can modify the macro |\thepage| appropriately
and reset the counter |page| at the start of each child file.

%%%%%%%%%%%%%%%%%%%%%%%%%%%%%%%%%%%%%%%%%%%%%%%%%%%%%%%%%%%%%%%%%%%%%%%%%%%%%%%%
\subsection{Conditional Processing}
\label{sec:conditional}

The package provides a mechanism to compile different versions
of a document. To customise the versions further some conditional processing
can come in handy to distinguish which version is being compiled.
The package provides two macros to describe the compilation context:

%%%%%%%%%%%%%%%%%%%%%%%%%%%%%%%%%%%%%%%%
\DescribeMacro{\ifchilddoc}
The conditional |\ifchilddoc| distinguishes between the compilation of
child documents and the main document:
%
\begin{center}
|\ifchilddoc |\textit{child-code}| |[|\||else |\textit{main-code}]| \||fi|
\end{center}

%%%%%%%%%%%%%%%%%%%%%%%%%%%%%%%%%%%%%%%%
\DescribeMacro{\childdocname}
\DescribeMacro{\childdocjob}
The macro |\childdocname| contains the filename (without extension)
of the main or child file being processed.
Note that |\childdocjob| will always contain the name of the main file.

%%%%%%%%%%%%%%%%%%%%%%%%%%%%%%%%%%%%%%%%
\paragraph{Title Page.}

Conditional processing can be used to include a title or banner page
in the main document when proper precautions are taken.
Importantly, the code in the main file should ensure that the page counter
(as well as other status parameters which are stored in the |.aux| files)
takes the same value after the conditional processing.
Otherwise the page numbers may take divergent values
depending on which part is compiled.

For example, a title page could be declared by:
%
\begin{center}
\begin{tabular}{l}
|\ifchilddoc\||else|\\
|\addtocounter{page}{-1}|\\
\textit{code for title page}\\
|\newpage|\\
|\||fi|
\end{tabular}
\end{center}
%
A banner page for the child documents can be generated by:
%
\begin{center}
\begin{tabular}{l}
|\ifchilddoc|\\
|\addtocounter{page}{-1}|\\
\textit{code for banner page}\\
|\newpage|\\
|\||fi|
\end{tabular}
\end{center}
%
Here one could write a message such as:
\begin{center}
|This is the part \childdocname{} of \childdocjob{}.|
\end{center}

%%%%%%%%%%%%%%%%%%%%%%%%%%%%%%%%%%%%%%%%%%%%%%%%%%%%%%%%%%%%%%%%%%%%%%%%%%%%%%%%
\subsection{Flags}
\label{sec:flags}

The package makes it easy to generate different versions
of the main or child documents.
To this end compilation flags can be defined
and assigned different default values.
They will be particularly useful in conjunction
with the forwarding mechanism described in \secref{sec:forward}.

For example, it may be useful to have a flag |\version|
which can be set to |draft| or |final|.
The document source will contain some conditional code
depending on the value of |\version|.
Suppose further, the flag should default to |final| for the main file
and to |draft| for child files
which is a natural assignment for editing the document.
This is achieved by placing the following code
in the preamble of the main document
(below the |\childdocmain| directive):
%
\begin{center}
\begin{tabular}{l}
|\ifchilddoc|\\
|\providecommand{\version}{draft}|\\
|\||else|\\
|\providecommand{\version}{final}|\\
|\||fi|
\end{tabular}
\end{center}
%
The definition by |\providecommand| makes sure
that previous definitions are not overwritten.
Further statements |\providecommand{\version}{...}|
can thus be added before the above code to override it.

For the main file, one might add a line
(between |\childdocmain| and the above block)
%
\begin{center}
|%\ifchilddoc\||else\providecommand{\version}{draft}\||fi|
\end{center}
%
which can be uncommented to produce a draft version.
Likewise one can add a line to the very top of a child file
(above the |\childdocof{|\textit{main}|}| directive)
%
\begin{center}
|%\providecommand{\version}{final}|
\end{center}
%
which can be uncommented to produce the final version of this child document.

%%%%%%%%%%%%%%%%%%%%%%%%%%%%%%%%%%%%%%%%%%%%%%%%%%%%%%%%%%%%%%%%%%%%%%%%%%%%%%%%
\subsection{Forwarding}
\label{sec:forward}

Different versions of the main or child documents
using compilation flags as described in \secref{sec:flags}
can be (permanently) stored in different files
for convenient compilation, viewing and distribution.
To this end, the package defines a command
to pass on compilation to a different file:

%%%%%%%%%%%%%%%%%%%%%%%%%%%%%%%%%%%%%%%%
\DescribeMacro{\childdocforward}
The command |\childdocforward| redirects processing to
another source file:
%
\begin{center}
\begin{tabular}{l}
|\input{childdoc.def}|\\
|\childdocforward[|\textit{main}|]{|\textit{dest}|}|\\
\end{tabular}
\end{center}
%
The argument \textit{dest} is the destination file
(without extension).
It should be the main file or one of the child files.
Note that further \textsf{childdoc} directives
such as |\childdocof| and |\childdocforward|
in the indicated file will be processed in this form.
The optional argument \textit{main}
passes on directly to the main file \textit{main}
while pretending to compile the child \textit{dest}.
This form behaves as if \textit{dest}
issues |\childdocof{|\textit{main}|}| right away,
and no further \textsf{childdoc} directives will be processed.

%%%%%%%%%%%%%%%%%%%%%%%%%%%%%%%%%%%%%%%%
\DescribeMacro{\...prefix}
In the alternative form |\childdocforwardprefix|,
%
\begin{center}
\begin{tabular}{l}
|\input{childdoc.def}|\\
|\childdocforwardprefix[|\textit{main}|]{|\textit{prefix}|}{|\textit{dest}|}|
\end{tabular}
\end{center}
%
the destination file is determined by a pattern
depending on the current file:
To make this work, the current file must be called
`{\textit{prefix}\hspace{0.2em}\textit{suffix}}'
with \textit{prefix} matching precisely the argument.
Processing is then passed on to the file
`{\textit{dest}\hspace{0.2em}\textit{suffix}}'.
Surely, the same effect is achieved by
directly specifying the
argument `{\textit{dest}\hspace{0.2em}\textit{suffix}}'
in the first form.
However, that requires to set up a different file
for each child. With the alternative form of the command
all these files can have exactly the same content
which simplifies setting them up and maintaining them.

For example, the following file |draft.tex|
with a compilation flag |\version| as described in \secref{sec:flags}
compiles the main document as a draft:
%
\begin{center}
\begin{tabular}{l}
|\def\version{draft}|\\
|\input{childdoc.def}|\\
|\childdocforward{|\textit{main}|}|
\end{tabular}
\end{center}
%
Likewise, the following files |final|\textit{nn}|.tex|
compile the final version of the child document
|child|\textit{nn}|.tex|:
%
\begin{center}
\begin{tabular}{l}
|\def\version{final}|\\
|\input{childdoc.def}|\\
|\childdocforwardprefix{final}{child}|
\end{tabular}
\end{center}
%

Note that when several versions of a main file and/or of each child file
are to be generated, it may be convenient to set up a |Makefile| or
shell script to automatise the process.

%%%%%%%%%%%%%%%%%%%%%%%%%%%%%%%%%%%%%%%%%%%%%%%%%%%%%%%%%%%%%%%%%%%%%%%%%%%%%%%%
\subsection{Command Line Processing}
\label{sec:commandline}

The effect of redirection files can also be achieved by invoking
the \LaTeX{} compiler with a more elaborate command line.
Most conveniently this should be done as part
of a shell script or a |Makefile|.

When using \textsf{childdoc} in the main file, the following
command lines effectively perform a redirection
(note that depending on the shell being used,
backslashes may have to be doubled: `|\|' $\to$ `|\\|'):
%
\begin{center}
|... -jobname "|\textit{target}|" |\\|"|[\textit{flags}]%
|\input{childdoc.def}\childdocforward[|\textit{main}|]{|\textit{dest}|}"|
\end{center}
%
Here \textit{target} is the name of the output file,
\textit{main} is the name of the main file
and \textit{dest} is the name of the main or child file to be processed
(all filenames without extensions).
The optional argument \textit{main} can be omitted
if \textit{main} matches \textit{dest}.
Optionally, compilation \textit{flags} can be defined via |\def| commands.
This command line makes the \TeX{} engine believe
it is compiling the file \textit{target}
whose content is specified as the latter parameter.
The provided code then forwards the processing to
\textit{main} or \textit{dest} as described in \secref{sec:forward}.

%%%%%%%%%%%%%%%%%%%%%%%%%%%%%%%%%%%%%%%%%%%%%%%%%%%%%%%%%%%%%%%%%%%%%%%%%%%%%%%%
\subsection{Include by Input}
\label{sec:input}

Including child documents by |\include| has some restrictions by design.
Most notably, the content of a child document always occupies
its own set of pages; pages cannot be shared between child documents.
Usually, this behaviour makes perfect sense
because each child document contain an essential part of the document.
However, in some situations it may be desirable to compose
a document from a collection of parts
without having mandatory page breaks between then.
For this case, the package
provides a mechanism to include parts
by |\input| which can also be processed individually.
However, by construction this mechanism
requires manual handling of the content to be output.

%%%%%%%%%%%%%%%%%%%%%%%%%%%%%%%%%%%%%%%%
\DescribeMacro{\ifchilddocmanual}
The main file should be prepared as usual, see \secref{sec:include}.
However, the document body must make a distinction
between processing of an individual part and of the main document, e.g.:
%
\begin{center}
\begin{tabular}{l}
|\ifchilddocmanual|\\
|\input{\childdocname}|\\
|\||else|\\
\textit{document body with }|\input{|\textit{part}|}|\\
|\||fi|
\end{tabular}
\end{center}
%
The conditional |\ifchilddocmanual| is true whenever
a part to be included by |\input| is being compiled,
and the name of the part is stored in |\childdocname|.

%%%%%%%%%%%%%%%%%%%%%%%%%%%%%%%%%%%%%%%%
\DescribeMacro{\childdocby}
Each part to be included by |\input| should start with:
%
\begin{center}
\begin{tabular}{l}
|\input{childdoc.def}|\\
|\childdocby{|\textit{main}|}|\\
\end{tabular}
\end{center}
%
The directive |\childdocby| is similar to |\childdocof|
described in \secref{sec:include},
but the subsequent selection of content must be done manually.
To that end, both |\ifchilddoc| and |\ifchilddocmanual|
will be true upon processing of a part,
and the name of the part is stored in |\childdocname|.
Note that |\jobname| will be set to the filename of the current part
so that each part receives an individual |.aux| file
that does not interfere with the |.aux| file(s) of the main document.
This behaviour can be altered by the alternative form
|\childdocby[*]{|\textit{main}|}| (with a non-empty optional argument)
which uses the |.aux| file of the main document
by setting |\jobname| to \textit{main}.

%%%%%%%%%%%%%%%%%%%%%%%%%%%%%%%%%%%%%%%%%%%%%%%%%%%%%%%%%%%%%%%%%%%%%%%%%%%%%%%%
\subsection{Driver Development}
\label{sec:driver}

The \textsf{childdoc} mechanism can also be use for the development
of definition files such as \LaTeX{} styles or classes.
This case differs from the above setup with multiple parts
included by |\include| in that no |\includeonly| should be invoked.
This can be achieved by starting the include file
(before |\ProvidesPackage|) with:
%
\begin{center}
\begin{tabular}{l}
|\input{childdoc.def}|\\
|\childdocforward{|\textit{main}|}|\\
\end{tabular}
\end{center}
%
or alternatively with:
%
\begin{center}
\begin{tabular}{l}
|\input{childdoc.def}|\\
|\childdocby{|\textit{main}|}|\\
\end{tabular}
\end{center}
%
Both forms have slightly different effects as described above.
The main file is prepared as usual, see \secref{sec:include}.

%%%%%%%%%%%%%%%%%%%%%%%%%%%%%%%%%%%%%%%%%%%%%%%%%%%%%%%%%%%%%%%%%%%%%%%%%%%%%%%%
\subsection{Legacy Detection}
\label{sec:detection}

The directive |\childdocmain| in the main file can detect
whether the complete document or merely a child is to be compiled
even without using the directive |\childdocof|.
This method is deprecated because it is less robust
and there is no compelling reason to use it;
it is merely provided for backward compatibility
and it may be removed in future versions.

If the detection mechanism is to be used,
it is mandatory to correctly specify
the filename of the main file as the argument of |\childdocmain|:
%
\begin{center}
\begin{tabular}{l}
|\input{childdoc.def}|\\
|\childdocmain{|\textit{main}|}|\\
\end{tabular}
\end{center}
%
If |\jobname| does not match the argument \textit{main} of |\childdocmain|,
it is assumed that |\jobname| points to the child file to be compiled.
When using |\childdocmain| with the main file specified as argument,
it suffices to start a child file
with just |\input{|\textit{main}|}|
without loading of the package and using |\childdocof|.
If instead all processing is done
with the appropriate \textsf{childdoc} directives,
the argument of \textit{main} of |\childdocmain| can be empty.

An alternative version of the command line processing described
in \secref{sec:commandline} using the detection mechanism reads:
%
\begin{center}
|... -jobname "|\textit{target}|" "|[\textit{flags}]%
[|\def\jobname{|\textit{dest}|}|]|\input{|\textit{main}|}"|
\end{center}

%%%%%%%%%%%%%%%%%%%%%%%%%%%%%%%%%%%%%%%%%%%%%%%%%%%%%%%%%%%%%%%%%%%%%%%%%%%%%%%%
\subsection{Manual Code}
\label{sec:manual}

In case one cannot be certain whether the definitions file |childdoc.def|
is installed on the target \TeX{} distribution
and one prefers not to ship it,
it is conceivable to paste a few relevant commands into the sources.

To that end, drop all statements |\input{childdoc.def}|
and perform the replacements as outlined below.
Instead of |\childdocmain{|\textit{main}|}| add the following code
to the top of the main file:
%
\begin{center}
\begin{tabular}{l}
|\||ifdefined\childdocname\endinput\||fi\newif\ifchilddoc|\\
|\edef\childdocname{\scantokens\expandafter{\jobname\noexpand}}|\\
|\def\childdocmain{|\textit{main}|}\||ifx\childdocmain\childdocname\||else|\\
|\childdoctrue\includeonly{\childdocname}\let\jobname\childdocmain\||fi|\\
\end{tabular}
\end{center}
%
Instead of |\childdocof{|\textit{main}|}| just include the main file
at the top of each child file:
%
\begin{center}
|\input{|\textit{main}|}|
\end{center}
%
A simple redirection |\childdocforward{|\textit{dest}|}| is achieved by:
%
\begin{center}
|\def\jobname{|\textit{dest}|}\input{\jobname}|
\end{center}
%
The redirection with prefix
|\childdocforwardprefix[|\textit{prefix}|]{|\textit{dest}|}|
is accomplished by:
%
\begin{center}
\begin{tabular}{l}
|{\edef\jobname{\scantokens\expandafter{\jobname\noexpand}}|\\
|\def\redirectjob |\textit{prefix}|#1~~~{\gdef\jobname{|\textit{dest}|#1}}|\\
|\expandafter\redirectjob\jobname~~~}\input{\jobname}|
\end{tabular}
\end{center}

In an alternative approach,
child documents can be compiled by a specific command line
without additional code or specific definitions:
%
\begin{center}
|... -jobname "|\textit{target}|" "|[\textit{flags}]%
|\includeonly{|\textit{dest}|}\input{|\textit{main}|}"|
\end{center}
%

%%%%%%%%%%%%%%%%%%%%%%%%%%%%%%%%%%%%%%%%%%%%%%%%%%%%%%%%%%%%%%%%%%%%%%%%%%%%%%%%
%%%%%%%%%%%%%%%%%%%%%%%%%%%%%%%%%%%%%%%%%%%%%%%%%%%%%%%%%%%%%%%%%%%%%%%%%%%%%%%%
\section{Information}

%%%%%%%%%%%%%%%%%%%%%%%%%%%%%%%%%%%%%%%%%%%%%%%%%%%%%%%%%%%%%%%%%%%%%%%%%%%%%%%%
\subsection{Copyright}

Copyright \copyright{} 2017--2018 Niklas Beisert

This work may be distributed and/or modified under the
conditions of the \LaTeX{} Project Public License, either version 1.3
of this license or (at your option) any later version.
The latest version of this license is in
  \url{http://www.latex-project.org/lppl.txt}
and version 1.3 or later is part of all distributions of \LaTeX{}
version 2005/12/01 or later.

This work has the LPPL maintenance status `maintained'.

The Current Maintainer of this work is Niklas Beisert.

This work consists of the files |README.txt|, |childdoc.ins| and |childdoc.dtx|
as well as the derived files |childdoc.def|, |cdocsamp.tex|
with |cdocsch1.tex|, |cdocsch2.tex|, |cdocspt3.tex|, |cdocspt4.tex|,
|cdocsdrf.tex|, |cdocsfn1.tex|, |cdocsfn2.tex|
as well as |childdoc.pdf|.

%%%%%%%%%%%%%%%%%%%%%%%%%%%%%%%%%%%%%%%%%%%%%%%%%%%%%%%%%%%%%%%%%%%%%%%%%%%%%%%%
\subsection{Files and Installation}

The package consists of the files:
%
\begin{center}
\begin{tabular}{ll}
    |README.txt|   & readme file \\
    |childdoc.ins| & installation file \\
    |childdoc.dtx| & source file \\
    |childdoc.def| & definition file \\
    |cdocsamp.tex| & sample main file \\
    |cdocsch1.tex| & sample include file \\
    |cdocsch2.tex| & sample include file \\
    |cdocspt3.tex| & sample part file \\
    |cdocspt4.tex| & sample part file \\
    |cdocsdrf.tex| & sample redirection file \\
    |cdocsfn1.tex| & sample redirection file \\
    |cdocsfn2.tex| & sample redirection file \\
    |childdoc.pdf| & manual
\end{tabular}
\end{center}
%
The distribution consists of the files
|README.txt|, |childdoc.ins| and |childdoc.dtx|.
%
\begin{itemize}
\item
Run (pdf)\LaTeX{} on |childdoc.dtx|
to compile the manual |childdoc.pdf| (this file).
\item
Run \LaTeX{} on |childdoc.ins| to create the definitions file |childdoc.def|
and the sample |cdocsamp.tex| with include files
|cdocsch1.tex|, |cdocsch2.tex|, |cdocspt3.tex|, |cdocspt4.tex|,
|cdocsdrf.tex|, |cdocsfn1.tex|, |cdocsfn2.tex|.
Then copy the file |childdoc.def| to an appropriate directory of your \LaTeX{}
distribution, e.g.\ \textit{texmf-root}|/tex/latex/childdoc|.
\end{itemize}

%%%%%%%%%%%%%%%%%%%%%%%%%%%%%%%%%%%%%%%%%%%%%%%%%%%%%%%%%%%%%%%%%%%%%%%%%%%%%%%%
\subsection{Related CTAN Packages}

There are several other packages which offer a similar functionality:
%
\begin{itemize}
\item
The packages
\href{http://ctan.org/pkg/docmute}{\textsf{docmute}},
\href{http://ctan.org/pkg/includex}{\textsf{includex}} and
\href{http://ctan.org/pkg/standalone}{\textsf{standalone}}
provide commands to include only the document body of
a child file thus allowing both files to be compiled individually.
\item
The packages \href{http://ctan.org/pkg/subdocs}{\textsf{subdocs}}
and \href{http://ctan.org/pkg/subfiles}{\textsf{subfiles}}
provide structures in which the main and child documents can be
encapsulated and allowing them to be compiled individually.
The inclusion mechanism is different from the conventional |\include|.
\item
The package \href{http://ctan.org/pkg/combine}{\textsf{combine}}
is an elaborate solution to combine several documents into one.
\end{itemize}
%
See also the CTAN topic \href{http://ctan.org/topic/subdocs}{\textsf{subdocs}}
for further related packages.
The present package differs from the above solutions in that
a document structure constructed with the conventional |\include| mechanism
just needs two extra commands at the top of every file
such that all constituent files can be compiled individually.

%%%%%%%%%%%%%%%%%%%%%%%%%%%%%%%%%%%%%%%%%%%%%%%%%%%%%%%%%%%%%%%%%%%%%%%%%%%%%%%%
%\subsection{Feature Suggestions}
%
%The following is a list of features which may be useful for future
%versions of this package:
%%
%\begin{itemize}
%\item
%\ldots
%\end{itemize}

%%%%%%%%%%%%%%%%%%%%%%%%%%%%%%%%%%%%%%%%%%%%%%%%%%%%%%%%%%%%%%%%%%%%%%%%%%%%%%%%
\subsection{Revision History}

%%%%%%%%%%%%%%%%%%%%%%%%%%%%%%%%%%%%%%%%
\paragraph{v2.0:} 2018/12/30

\begin{itemize}
\item
immediate forward processing
\item
added |\childdocby| mechanism
\item
manual restructured
\end{itemize}

%%%%%%%%%%%%%%%%%%%%%%%%%%%%%%%%%%%%%%%%
\paragraph{v1.6:} 2018/01/17

\begin{itemize}
\item
application for development of include files
\item
corrections to manual
\end{itemize}

%%%%%%%%%%%%%%%%%%%%%%%%%%%%%%%%%%%%%%%%
\paragraph{v1.5:} 2017/05/21

\begin{itemize}
\item
more complete structuring introduced
\item
|\childdocof| introduced
\item
|\childdoc| renamed to |\childdocmain|
\item
|\childredirect| renamed to |\childdocforward| and |\childdocforwardprefix|
and functionality expanded
\end{itemize}

%%%%%%%%%%%%%%%%%%%%%%%%%%%%%%%%%%%%%%%%
\paragraph{v1.0:} 2017/04/27

\begin{itemize}
\item
manual and install package
\item
first version published on CTAN
\end{itemize}

%%%%%%%%%%%%%%%%%%%%%%%%%%%%%%%%%%%%%%%%
\paragraph{v0.6:} 2017/04/26

\begin{itemize}
\item
redirection mechanism added
\end{itemize}

%%%%%%%%%%%%%%%%%%%%%%%%%%%%%%%%%%%%%%%%
\paragraph{v0.5:} 2017/04/26

\begin{itemize}
\item
functionality in definition file
\end{itemize}


%%%%%%%%%%%%%%%%%%%%%%%%%%%%%%%%%%%%%%%%%%%%%%%%%%%%%%%%%%%%%%%%%%%%%%%%%%%%%%%%
%%%%%%%%%%%%%%%%%%%%%%%%%%%%%%%%%%%%%%%%%%%%%%%%%%%%%%%%%%%%%%%%%%%%%%%%%%%%%%%%
%%%%%%%%%%%%%%%%%%%%%%%%%%%%%%%%%%%%%%%%%%%%%%%%%%%%%%%%%%%%%%%%%%%%%%%%%%%%%%%%
\appendix

\settowidth\MacroIndent{\rmfamily\scriptsize 000\ }

 \DocInput{childdoc.dtx}

\end{document}
%</driver>
% \fi
%
% %%%%%%%%%%%%%%%%%%%%%%%%%%%%%%%%%%%%%%%%%%%%%%%%%%%%%%%%%%%%%%%%%%%%%%%%%%%%%%
% %%%%%%%%%%%%%%%%%%%%%%%%%%%%%%%%%%%%%%%%%%%%%%%%%%%%%%%%%%%%%%%%%%%%%%%%%%%%%%
% \section{Sample}
%\iffalse
%<*samplemain>
%\fi
%
% The following presents a sample document
% with two chapters, two parts, a title page,
% a compile flag as well as three forwarding files to set the flag.
% It consists of eight |.tex| files:
% \begin{center}
% \begin{tabular}{ll}
% |cdocsamp.tex|&main file\\
% |cdocsch1.tex|&include file for chapter 1\\
% |cdocsch2.tex|&include file for chapter 2\\
% |cdocspt3.tex|&include file for part 3\\
% |cdocspt4.tex|&include file for part 4\\
% |cdocsdrf.tex|&forwarding file for main file in draft mode\\
% |cdocsfi1.tex|&forwarding file for final version of chapter 1\\
% |cdocsfi2.tex|&forwarding file for final version of chapter 2\\
% \end{tabular}
% \end{center}
% Each of the eight files can be compiled directly by the \LaTeX{} compiler.
%
% %%%%%%%%%%%%%%%%%%%%%%%%%%%%%%%%%%%%%%
% \paragraph{Main File.}
%
% The main file is called |cdocsamp.tex|.
%
% Load the \textsf{childdoc} definitions and
% declare the filename for the main document:
%    \begin{macrocode}
\input{childdoc.def}
\childdocmain{}
%    \end{macrocode}

% Optional override for |\version| flag:
%    \begin{macrocode}
%%\ifchilddoc\else\providecommand{\version}{draft}\fi
%    \end{macrocode}

% Define the default values for the |\version| flag
% (|final| for the main file and |draft| for childs):
%    \begin{macrocode}
\ifchilddoc
\providecommand{\version}{draft}
\else
\providecommand{\version}{final}
\fi
%    \end{macrocode}

% Load the standard document class:
%    \begin{macrocode}
\documentclass[12pt]{article}
%    \end{macrocode}

% Start the document body:
%    \begin{macrocode}
\begin{document}
%    \end{macrocode}

% Declare a title page.
% Print title, part of document being processed and version flag:
%    \begin{macrocode}
\addtocounter{page}{-1}
\begin{center}
{\LARGE\bfseries{}childdoc example\par}
\vspace{1cm}
\ifchilddoc
\ifchilddocmanual part\else chapter\fi:
`\childdocname' of `\childdocjob'\par
\else
main document: `\childdocjob'\par
\fi
version: \version\par
\end{center}
\newpage
%    \end{macrocode}

% Manually include selected file,
% otherwise process as usual:
%    \begin{macrocode}
\ifchilddocmanual
\section*{part `\childdocname'}
\input{\childdocname}
\else
%    \end{macrocode}

% Include the two chapters:
%    \begin{macrocode}
\include{cdocsch1}
\include{cdocsch2}
%    \end{macrocode}

% Include the two parts unless only chapters should be displayed:
%    \begin{macrocode}
\ifchilddoc\else
\section{part three}
\input{cdocspt3}
\section{part four}
\input{cdocspt4}
\fi
%    \end{macrocode}

% Process as usual until here:
%    \begin{macrocode}
\fi
%    \end{macrocode}

% End of document body:
%    \begin{macrocode}
\end{document}
%    \end{macrocode}
%\iffalse
%</samplemain>
%\fi
%
% %%%%%%%%%%%%%%%%%%%%%%%%%%%%%%%%%%%%%%
% \paragraph{Chapter Include Files.}
%
% The include files are called |cdocsch1.tex| and |cdocsch2.tex|.
%
%\iffalse
%<*samplechap1|samplechap2>
%\fi

% Optional override for |\version| flag:
%    \begin{macrocode}
%%\providecommand{\version}{final}
%    \end{macrocode}

% Include the main document:
%    \begin{macrocode}
\input{childdoc.def}
\childdocof{cdocsamp}
%    \end{macrocode}

%\iffalse
%</samplechap1|samplechap2>
%\fi
%
%\iffalse
%<*samplechap1>
%\fi
% Some text for chapter 1:
%    \begin{macrocode}
\section{one}
some text in chapter one
%    \end{macrocode}

%\iffalse
%</samplechap1>
%\fi
% Some text for chapter 2:
%\iffalse
%<*samplechap2>
%\fi
%    \begin{macrocode}
\section{two}
more text in chapter two
%    \end{macrocode}

%\iffalse
%</samplechap2>
%\fi
%
% %%%%%%%%%%%%%%%%%%%%%%%%%%%%%%%%%%%%%%
% \paragraph{Part Include Files.}
%
% The include files are called |cdocspt3.tex| and |cdocspt4.tex|.
%
%\iffalse
%<*samplepart3|samplepart4>
%\fi

% Optional override for |\version| flag:
%    \begin{macrocode}
%%\providecommand{\version}{final}
%    \end{macrocode}

% Include the main document:
%    \begin{macrocode}
\input{childdoc.def}
\childdocby{cdocsamp}
%    \end{macrocode}

%\iffalse
%</samplepart3|samplepart4>
%\fi
%
%\iffalse
%<*samplepart3>
%\fi
% Some text for part 3:
%    \begin{macrocode}
some text in part three
%    \end{macrocode}

%\iffalse
%</samplepart3>
%\fi
% Some text for part 4:
%\iffalse
%<*samplepart4>
%\fi
%    \begin{macrocode}
more text in part four
%    \end{macrocode}

%\iffalse
%</samplepart4>
%\fi
%
% %%%%%%%%%%%%%%%%%%%%%%%%%%%%%%%%%%%%%%
% \paragraph{Forwarding for a Complete Draft.}
%
% The following forwarding file |cdocsdrf.tex|
% compiles the main document in draft mode:
%\iffalse
%<*sampledraft>
%\fi
%    \begin{macrocode}
\def\version{draft}
\input{childdoc.def}
\childdocforward{cdocsamp}
%    \end{macrocode}

%\iffalse
%</sampledraft>
%\fi
%
% %%%%%%%%%%%%%%%%%%%%%%%%%%%%%%%%%%%%%%
% \paragraph{Forwarding for Final Version of the Chapters.}
%
% The following forwarding files |cdocsfn1.tex| and |cdocsfn2.tex|
% (with identical content)
% compile the final versions of the child documents
% |cdocsch1.tex| and |cdocsch2.tex|, respectively:
%\iffalse
%<*samplefinal>
%\fi
%    \begin{macrocode}
\def\version{final}
\input{childdoc.def}
\childdocforwardprefix[cdocsamp]{cdocsfn}{cdocsch}
%    \end{macrocode}

%\iffalse
%</samplefinal>
%\fi
%
% %%%%%%%%%%%%%%%%%%%%%%%%%%%%%%%%%%%%%%
% \paragraph{Command Line Processing.}
%
% The following three command lines generate the output files
% |cdocscld|, |cdocscl1| and |cdocscl2|
% which should be identical to
% |cdocsdrf|, |cdocsch1| and |cdocsfn2|, respectively:
% \begin{center}
% \begin{tabular}{l}
% |latex -jobname cdocscld \|\\
% |  "\def\version{draft}\input{childdoc.def}\childdocforward{cdocsamp}"|\\
% |latex -jobname cdocscl1 \|\\
% |  "\input{childdoc.def}\childdocforward[cdocsamp]{cdocsch1}"|\\
% |latex -jobname cdocscl2 \|\\
% |  "\def\version{final}\input{childdoc.def}\childdocforward{cdocsch2}"|
% \end{tabular}
% \end{center}
% Note that the trailing backslash on each first line
% merely continues the input to the second line
% (for convenient cut ant paste).
% Furthermore, the command |latex| can be replaced by any
% of its alternative versions such as |pdflatex|.
%
% %%%%%%%%%%%%%%%%%%%%%%%%%%%%%%%%%%%%%%%%%%%%%%%%%%%%%%%%%%%%%%%%%%%%%%%%%%%%%%
% %%%%%%%%%%%%%%%%%%%%%%%%%%%%%%%%%%%%%%%%%%%%%%%%%%%%%%%%%%%%%%%%%%%%%%%%%%%%%%
% \section{Implementation}
%\iffalse
%<*package>
%\fi
%
% This section describes the definitions file |childdoc.def|.

% The definitions cannot be loaded using |\usepackage| or |\RequirePackage|
% which has a mechanism to prevent loading a style file more than once.
% When loading the definitions by means of |\input|
% multiple instances have to be prevented manually:
%\iffalse
%This code needs to be before the `\ProvidesFile' directive
%which is defined at the beginning of this file.
%Therefore it is also placed there and commented out here.
%</package>
%<*discard>
%\fi
%    \begin{macrocode}
\ifdefined\childdocmain\endinput\fi
%    \end{macrocode}
%\iffalse
%</discard>
%<*package>
%\fi
%
% \macro{\ifchilddoc}
% \macro{\ifchilddocmanual}
% The conditional |\ifchilddoc| tells whether a
% child (true) or main (false) document is being compiled.
% The conditional |\ifchilddocmanual| tells whether
% the |\includeonly| mechanism is used (false) or
% the selection of child files must be performed manually (true).
% The definitions initialise to false:
%    \begin{macrocode}
\newif\ifchilddoc
\newif\ifchilddocmanual
%    \end{macrocode}

% \macro{\childdocname}
% \macro{\childdocjob}
% The macro |\childdocname| stores the name of the main document
% to be compiled. The macro |\childdocjob| stores the name of
% the document on which the \LaTeX{} compiler was originally invoked.
% The content of |\jobname| cannot be compared
% to filenames specified in the source due to different catcodes.
% The following code rescans |\jobname|, stores the result
% in |\childdocname| and saves a copy in |\childdocjob|:
%    \begin{macrocode}
\edef\childdocname{\scantokens\expandafter{\jobname\noexpand}}
\let\childdocjob\childdocname
%    \end{macrocode}

% \macro{\childdocdisable}
% The macro |\childdocdisable| prevents the main file
% from being processed more than once.
% At this stage, the main document command |\childdocmain|
% is assumed to be called once again where it should do nothing.
% Any subsequent call to it should prevent
% a secondary processing of the main document
% It overwrites the forwarding commands
% |\childdocof| and |\childdocforward|
% with empty macros to prevent further inclusions of the main document:
%    \begin{macrocode}
\newcommand{\childdocdisable}
{
  \renewcommand{\childdocmain}[1]{\renewcommand{\childdocmain}[1]{\endinput}}
  \renewcommand{\childdocof}[1]{}
  \renewcommand{\childdocby}[2][]{}
  \renewcommand{\childdocforward}[2][]{}
  \renewcommand{\childdocdisable}{}
}
%    \end{macrocode}

% \macro{\childdocmain}
% The macro |\childdocmain| is to be called at the top of the main file
% with nothing or the main filename (without extension) as argument.
% First, it breaks loops.
% If the argument is not empty and does not match |\childdocname|
% (which is set by the first inclusion of |childdoc.def|),
% |\ifchilddoc| is set to true, |\includeonly| is applied to the child file
% and |\jobname| is set to the main file
% (for proper handling of |.aux| files):
%    \begin{macrocode}
\newcommand{\childdocmain}[1]
{
  \childdocdisable\childdocmain{}
  \if?#1?\else
    \begingroup
      \def\childdoctmp{#1}
      \ifx\childdoctmp\childdocname
        \def\childdoctmp{}
      \else
        \def\childdoctmp
        {
          \childdoctrue
          \includeonly{\childdocname}
          \def\childdocjob{#1}
          \def\jobname{#1}
        }
      \fi
      \expandafter
    \endgroup
    \childdoctmp
  \fi
}
%    \end{macrocode}

% \macro{\childdocof}
% The command |\childdocof| redirects
% compilation to the main file |#1|.
%    \begin{macrocode}
\newcommand{\childdocof}[1]
{
  \childdocdisable
  \childdoctrue
  \includeonly{\childdocname}
  \def\jobname{#1}
  \def\childdocjob{#1}
  \input{#1}
}
%    \end{macrocode}

% \macro{\childdocby}
% The command |\childdocby| ....
%    \begin{macrocode}
\newcommand{\childdocby}[2][]
{
  \childdocdisable
  \childdoctrue
  \childdocmanualtrue
  \if?#1?\else
    \def\jobname{#2}
  \fi
  \def\childdocjob{#2}
  \input{#2}
  \endinput
}
%    \end{macrocode}

% \macro{\childdocforward}
% The command |\childdocforward| redirects
% compilation to the main file or
% (if the optional argument is given) a child file.
% Parameters are set as if the main file
% or a child file starting with |\childdocof| was compiled.
% Then compilation is handed over to the main file:
%    \begin{macrocode}
\newcommand{\childdocforward}[2][]
{
  \begingroup
    \if?#1?
      \def\childdoctmp
      {
        \def\childdocname{#2}
        \def\childdocjob{#2}
        \def\jobname{#2}
        \input{#2}
        \endinput
      }
    \else
      \def\childdoctmp
      {
        \childdocdisable
        \def\childdocname{#2}
        \childdoctrue
        \includeonly{#2}
        \def\childdocjob{#1}
        \def\jobname{#1}
        \input{#1}
        \endinput
      }
    \fi
    \expandafter
  \endgroup
  \childdoctmp
}
%    \end{macrocode}

% \macro{\childdocforwardprefix}
% The command |\childdocforwardprefix| redirects
% compilation to the main or a child file by means of a pattern.
% The prefix |#1| in the current filename is replaced by |#2|
% and the suffix of the current filename is kept
% (it is assumed that the filename does not contain the substring `|~~~|'
% which is used as a delimiter).
% Compilation is handed over to the new file by |\childdocforward|:
%    \begin{macrocode}
\newcommand{\childdocforwardprefix}[3][]
{
  \begingroup
    \def\childdocextract #2##1~~~{\def\childdoctmp{\childdocforward[#1]{#3##1}}}
    \expandafter\childdocextract\childdocname~~~
    \expandafter
  \endgroup
  \childdoctmp
}
%    \end{macrocode}

% \macro{\childdoc}
% The deprecated macro |\childdoc| is a legacy version of |\childdocmain|:
%    \begin{macrocode}
\newcommand{\childdoc}{\childdocmain}
%    \end{macrocode}

% \macro{\childdocredirect}
% The deprecated macro |\childdocredirect| is a legacy version
% of |\childdocforward| and |\childdocforwardprefix|:
%    \begin{macrocode}
\newcommand{\childdocredirect}[2][]
{
  \begingroup
    \if?#1?
      \def\childdoctmp{\childdocforward{#2}}
    \else
      \def\childdoctmp{\childdocforwardprefix{#1}{#2}}
    \fi
    \expandafter
  \endgroup
  \childdoctmp
}
%    \end{macrocode}

%\iffalse
%</package>
%\fi
%
\endinput

\childdocforwardprefix[cdocsamp]{cdocsfn}{cdocsch}
%    \end{macrocode}

%\iffalse
%</samplefinal>
%\fi
%
% %%%%%%%%%%%%%%%%%%%%%%%%%%%%%%%%%%%%%%
% \paragraph{Command Line Processing.}
%
% The following three command lines generate the output files
% |cdocscld|, |cdocscl1| and |cdocscl2|
% which should be identical to
% |cdocsdrf|, |cdocsch1| and |cdocsfn2|, respectively:
% \begin{center}
% \begin{tabular}{l}
% |latex -jobname cdocscld \|\\
% |  "\def\version{draft}% \iffalse
%
% childdoc.dtx Copyright (C) 2017-2018 Niklas Beisert
%
% This work may be distributed and/or modified under the
% conditions of the LaTeX Project Public License, either version 1.3
% of this license or (at your option) any later version.
% The latest version of this license is in
%   http://www.latex-project.org/lppl.txt
% and version 1.3 or later is part of all distributions of LaTeX
% version 2005/12/01 or later.
%
% This work has the LPPL maintenance status `maintained'.
%
% The Current Maintainer of this work is Niklas Beisert.
%
% This work consists of the files childdoc.dtx and childdoc.ins
% and the derived files childdoc.def and cdocsamp.tex with
% cdocsch1.tex, cdocsch2.tex, cdocsdrf.tex, cdocsfn1.tex, cdocsfn2.tex.
%
%<package>\ifdefined\childdocmain\endinput\fi
%<package>\ProvidesFile{childdoc.def}[2018/12/30 v2.0 child document driver]
%<samplemain>\ProvidesFile{cdocsamp.tex}[2018/12/30 v2.0 sample for childdoc]
%<*driver>
%\ProvidesFile{childdoc.drv}[2018/12/30 v2.0 childdoc reference manual file]
\PassOptionsToClass{10pt,a4paper}{article}
\documentclass{ltxdoc}

\usepackage[margin=35mm]{geometry}
\usepackage{hyperref}
\usepackage{hyperxmp}
\usepackage[usenames]{color}

\hypersetup{colorlinks=true}
\hypersetup{pdfstartview=FitH}
\hypersetup{pdfpagemode=UseNone}
\hypersetup{pdfsource={}}
\hypersetup{pdflang={en-UK}}
\hypersetup{pdfcopyright={Copyright 2017-2018 Niklas Beisert.
  This work may be distributed and/or modified under the
  conditions of the LaTeX Project Public License, either version 1.3
  of this license or (at your option) any later version.}}
\hypersetup{pdflicenseurl={http://www.latex-project.org/lppl.txt}}
\hypersetup{pdfcontactaddress={ETH Zurich, ITP, HIT K,
  Wolfgang-Pauli-Strasse 27}}
\hypersetup{pdfcontactpostcode={8093}}
\hypersetup{pdfcontactcity={Zurich}}
\hypersetup{pdfcontactcountry={Switzerland}}
\hypersetup{pdfcontactemail={nbeisert@itp.phys.ethz.ch}}
\hypersetup{pdfcontacturl={http://people.phys.ethz.ch/\xmptilde nbeisert/}}

\newcommand{\secref}[1]{\hyperref[#1]{section \ref*{#1}}}

\parskip1ex
\parindent0pt
\let\olditemize\itemize
\def\itemize{\olditemize\parskip0pt}

\begin{document}

\title{The \textsf{childdoc} Package}
\hypersetup{pdftitle={The childdoc Package}}
\author{Niklas Beisert\\[2ex]
  Institut f\"ur Theoretische Physik\\
  Eidgen\"ossische Technische Hochschule Z\"urich\\
  Wolfgang-Pauli-Strasse 27, 8093 Z\"urich, Switzerland\\[1ex]
  \href{mailto:nbeisert@itp.phys.ethz.ch}
  {\texttt{nbeisert@itp.phys.ethz.ch}}}
\hypersetup{pdfauthor={Niklas Beisert}}
\hypersetup{pdfsubject={Manual for the LaTeX2e Package childdoc}}
\date{30 December 2018, \textsf{v2.0}}
\maketitle

\begin{abstract}\noindent
\textsf{childdoc} is a \LaTeXe{} package
that enables the direct compilation
of document sections included by |\include|
to individual files.
\end{abstract}

\begingroup
\parskip0ex
\tableofcontents
\endgroup

%%%%%%%%%%%%%%%%%%%%%%%%%%%%%%%%%%%%%%%%%%%%%%%%%%%%%%%%%%%%%%%%%%%%%%%%%%%%%%%%
%%%%%%%%%%%%%%%%%%%%%%%%%%%%%%%%%%%%%%%%%%%%%%%%%%%%%%%%%%%%%%%%%%%%%%%%%%%%%%%%
\section{Introduction}

\LaTeX{} provides a mechanism to structure a large document (such as a book)
into a main file and several child files (containing the chapters)
using the |\include| command.
This mechanism is beneficial for documents
which span hundreds of pages in order to
make the source file(s) more manageable.
Moreover, compilation can be restricted to
selected child files by means of the |\includeonly| command.
The latter feature can be used to reduce the compilation time while editing
(this was significantly more useful in the earlier days of \LaTeX{})
or to generate a smaller document which is easier to navigate.
Another application of |\includeonly| is to generate
documents consisting of selected parts of the complete document.

However, there are a few drawbacks of the plain |\include| mechanism:
\begin{itemize}
\item
The child files cannot be compiled on their own,
they can only be compiled via the main file.
A naive editing environment
(such as a text editor with an option
to have the current file processed by \LaTeX)
may require one to switch to the main file before compiling;
attempting to compile the child file produces errors.
\item
The main file must be modified (each time)
to adjust the |\includeonly| command
to the present needs. This easily leaves the main file in a messy state.
\item
The generated document will always carry the filename
of the main document. This is inconvenient if
several child files are to be compiled and
to be kept for distribution.
\end{itemize}

The present package provides a simple interface
to make child files individually compilable by \LaTeX{}.
Compiling a child file then has the same effect as compiling
the main file with an |\includeonly| command
to select the appropriate child.
Moreover the generated document will carry the name of the child
rather than the main file.
This resolves all three above issues.

This feature is meant to make the editing of books,
thesis documents and lecture notes somewhat more convenient.
However, the package can also be used efficiently for
composing a series of documents (such as exercise sheets)
which are typically distributed individually.
It then assists the author in generating the individual documents
(potentially in different versions)
as well as a document containing the collected series.
Another application is in developing style files
or other kinds of included material
where compilation of the style file could redirect
to a sample or test file.

%%%%%%%%%%%%%%%%%%%%%%%%%%%%%%%%%%%%%%%%%%%%%%%%%%%%%%%%%%%%%%%%%%%%%%%%%%%%%%%%
%%%%%%%%%%%%%%%%%%%%%%%%%%%%%%%%%%%%%%%%%%%%%%%%%%%%%%%%%%%%%%%%%%%%%%%%%%%%%%%%
\section{Usage}

First of all, the package \textsf{childdoc} is \emph{not} a standard
\LaTeXe{} |.sty| style file! Therefore it needs to be invoked in
a non-standard way.

%%%%%%%%%%%%%%%%%%%%%%%%%%%%%%%%%%%%%%%%%%%%%%%%%%%%%%%%%%%%%%%%%%%%%%%%%%%%%%%%
\subsection{Included Files}
\label{sec:include}

%%%%%%%%%%%%%%%%%%%%%%%%%%%%%%%%%%%%%%%%
\DescribeMacro{\childdocmain}
To use the package, add the commands
\begin{center}
\begin{tabular}{l}
|\input{childdoc.def}|\\
|\childdocmain{}|\\
\end{tabular}
\end{center}
at the very top of the main \LaTeX{} file,
in particular \emph{before} the |\documentclass| statement!
The argument of |\childdocmain| should be left empty
(but it must be present).

%%%%%%%%%%%%%%%%%%%%%%%%%%%%%%%%%%%%%%%%
\DescribeMacro{\childdocof}
Furthermore, add the commands
\begin{center}
\begin{tabular}{l}
|\input{childdoc.def}|\\
|\childdocof{|\textit{main}|}|\\
\end{tabular}
\end{center}
at the top of every child file \textit{child}
which is included by |\include{|\textit{child}|}|
from within the main file
(or at least for those files to be compiled individually).
The argument \textit{main} must be the filename of the main file.

There are a couple of
considerations in setting up the main and child documents:

%%%%%%%%%%%%%%%%%%%%%%%%%%%%%%%%%%%%%%%%
\paragraph{Restrictions.}

Please note the following restrictions:
\begin{itemize}
\item
|\childdocmain| must be called with one argument \textit{main}
to ensure compatibility with earlier version of the package.
It must either be empty (|\childdocmain{}|)
or precisely match the filename of the main file in which it is specified.
See \secref{sec:detection} for further information.
\item
The filename \textit{main} must be specified without the |.tex| extension.
\item
The filename \textit{main} is case sensitive
(even in case-insensitive file systems)
due to internal string comparison.
\item
The argument \textit{main} should be fully expanded, it cannot be a macro.
\item
Subdirectories and special characters should be avoided in filenames.
\item
The command |\childdocmain{|\textit{main}|}| must be followed by a whitespace.
It should not be followed immediately by another command
or by a comment mark `|%|'.
This is because the \TeX{} parser reads the token immediately following
the argument of |\childdocmain| and puts it
at the beginning of every child section;
however, a white\-space is ignored.
\end{itemize}

%%%%%%%%%%%%%%%%%%%%%%%%%%%%%%%%%%%%%%%%
\paragraph{Content of Main File.}

It is advisable to place all content in the child files included by |\include|.
Any output contained in the main file will appear in all child documents
unless suppressed manually;
it cannot be suppressed automatically by the |\includeonly| directive
and thus should normally be avoided.
A method to include some content in the main file
by means of conditional processing is described in \secref{sec:conditional}.

%%%%%%%%%%%%%%%%%%%%%%%%%%%%%%%%%%%%%%%%
\paragraph{Page Numbering.}

When only a part of the document is compiled,
the appropriate numbering of pages
(as well as other status parameters)
is determined from the |.aux| files.
The latter contain information from previous passes.
However this information needs to propagate through
all intermediate child documents.
Therefore the page numbering in child documents may well
be inconsistent until the complete document is compiled at least once.

A useful (if unconventional) way to always ensure a consistent
page numbering is to restart the numbering in each child document
and denote the pages by `\textit{child}|.|\textit{page}'
where \textit{child} represents the chapter/section number of the child file.
This can be achieved by the command
|\numberwithin{page}{|\textit{child}|}|
of the \textsf{amsmath} package
where \textit{child} can be |chapter| or |section|
depending on the chosen structuring.
Alternatively, one can modify the macro |\thepage| appropriately
and reset the counter |page| at the start of each child file.

%%%%%%%%%%%%%%%%%%%%%%%%%%%%%%%%%%%%%%%%%%%%%%%%%%%%%%%%%%%%%%%%%%%%%%%%%%%%%%%%
\subsection{Conditional Processing}
\label{sec:conditional}

The package provides a mechanism to compile different versions
of a document. To customise the versions further some conditional processing
can come in handy to distinguish which version is being compiled.
The package provides two macros to describe the compilation context:

%%%%%%%%%%%%%%%%%%%%%%%%%%%%%%%%%%%%%%%%
\DescribeMacro{\ifchilddoc}
The conditional |\ifchilddoc| distinguishes between the compilation of
child documents and the main document:
%
\begin{center}
|\ifchilddoc |\textit{child-code}| |[|\||else |\textit{main-code}]| \||fi|
\end{center}

%%%%%%%%%%%%%%%%%%%%%%%%%%%%%%%%%%%%%%%%
\DescribeMacro{\childdocname}
\DescribeMacro{\childdocjob}
The macro |\childdocname| contains the filename (without extension)
of the main or child file being processed.
Note that |\childdocjob| will always contain the name of the main file.

%%%%%%%%%%%%%%%%%%%%%%%%%%%%%%%%%%%%%%%%
\paragraph{Title Page.}

Conditional processing can be used to include a title or banner page
in the main document when proper precautions are taken.
Importantly, the code in the main file should ensure that the page counter
(as well as other status parameters which are stored in the |.aux| files)
takes the same value after the conditional processing.
Otherwise the page numbers may take divergent values
depending on which part is compiled.

For example, a title page could be declared by:
%
\begin{center}
\begin{tabular}{l}
|\ifchilddoc\||else|\\
|\addtocounter{page}{-1}|\\
\textit{code for title page}\\
|\newpage|\\
|\||fi|
\end{tabular}
\end{center}
%
A banner page for the child documents can be generated by:
%
\begin{center}
\begin{tabular}{l}
|\ifchilddoc|\\
|\addtocounter{page}{-1}|\\
\textit{code for banner page}\\
|\newpage|\\
|\||fi|
\end{tabular}
\end{center}
%
Here one could write a message such as:
\begin{center}
|This is the part \childdocname{} of \childdocjob{}.|
\end{center}

%%%%%%%%%%%%%%%%%%%%%%%%%%%%%%%%%%%%%%%%%%%%%%%%%%%%%%%%%%%%%%%%%%%%%%%%%%%%%%%%
\subsection{Flags}
\label{sec:flags}

The package makes it easy to generate different versions
of the main or child documents.
To this end compilation flags can be defined
and assigned different default values.
They will be particularly useful in conjunction
with the forwarding mechanism described in \secref{sec:forward}.

For example, it may be useful to have a flag |\version|
which can be set to |draft| or |final|.
The document source will contain some conditional code
depending on the value of |\version|.
Suppose further, the flag should default to |final| for the main file
and to |draft| for child files
which is a natural assignment for editing the document.
This is achieved by placing the following code
in the preamble of the main document
(below the |\childdocmain| directive):
%
\begin{center}
\begin{tabular}{l}
|\ifchilddoc|\\
|\providecommand{\version}{draft}|\\
|\||else|\\
|\providecommand{\version}{final}|\\
|\||fi|
\end{tabular}
\end{center}
%
The definition by |\providecommand| makes sure
that previous definitions are not overwritten.
Further statements |\providecommand{\version}{...}|
can thus be added before the above code to override it.

For the main file, one might add a line
(between |\childdocmain| and the above block)
%
\begin{center}
|%\ifchilddoc\||else\providecommand{\version}{draft}\||fi|
\end{center}
%
which can be uncommented to produce a draft version.
Likewise one can add a line to the very top of a child file
(above the |\childdocof{|\textit{main}|}| directive)
%
\begin{center}
|%\providecommand{\version}{final}|
\end{center}
%
which can be uncommented to produce the final version of this child document.

%%%%%%%%%%%%%%%%%%%%%%%%%%%%%%%%%%%%%%%%%%%%%%%%%%%%%%%%%%%%%%%%%%%%%%%%%%%%%%%%
\subsection{Forwarding}
\label{sec:forward}

Different versions of the main or child documents
using compilation flags as described in \secref{sec:flags}
can be (permanently) stored in different files
for convenient compilation, viewing and distribution.
To this end, the package defines a command
to pass on compilation to a different file:

%%%%%%%%%%%%%%%%%%%%%%%%%%%%%%%%%%%%%%%%
\DescribeMacro{\childdocforward}
The command |\childdocforward| redirects processing to
another source file:
%
\begin{center}
\begin{tabular}{l}
|\input{childdoc.def}|\\
|\childdocforward[|\textit{main}|]{|\textit{dest}|}|\\
\end{tabular}
\end{center}
%
The argument \textit{dest} is the destination file
(without extension).
It should be the main file or one of the child files.
Note that further \textsf{childdoc} directives
such as |\childdocof| and |\childdocforward|
in the indicated file will be processed in this form.
The optional argument \textit{main}
passes on directly to the main file \textit{main}
while pretending to compile the child \textit{dest}.
This form behaves as if \textit{dest}
issues |\childdocof{|\textit{main}|}| right away,
and no further \textsf{childdoc} directives will be processed.

%%%%%%%%%%%%%%%%%%%%%%%%%%%%%%%%%%%%%%%%
\DescribeMacro{\...prefix}
In the alternative form |\childdocforwardprefix|,
%
\begin{center}
\begin{tabular}{l}
|\input{childdoc.def}|\\
|\childdocforwardprefix[|\textit{main}|]{|\textit{prefix}|}{|\textit{dest}|}|
\end{tabular}
\end{center}
%
the destination file is determined by a pattern
depending on the current file:
To make this work, the current file must be called
`{\textit{prefix}\hspace{0.2em}\textit{suffix}}'
with \textit{prefix} matching precisely the argument.
Processing is then passed on to the file
`{\textit{dest}\hspace{0.2em}\textit{suffix}}'.
Surely, the same effect is achieved by
directly specifying the
argument `{\textit{dest}\hspace{0.2em}\textit{suffix}}'
in the first form.
However, that requires to set up a different file
for each child. With the alternative form of the command
all these files can have exactly the same content
which simplifies setting them up and maintaining them.

For example, the following file |draft.tex|
with a compilation flag |\version| as described in \secref{sec:flags}
compiles the main document as a draft:
%
\begin{center}
\begin{tabular}{l}
|\def\version{draft}|\\
|\input{childdoc.def}|\\
|\childdocforward{|\textit{main}|}|
\end{tabular}
\end{center}
%
Likewise, the following files |final|\textit{nn}|.tex|
compile the final version of the child document
|child|\textit{nn}|.tex|:
%
\begin{center}
\begin{tabular}{l}
|\def\version{final}|\\
|\input{childdoc.def}|\\
|\childdocforwardprefix{final}{child}|
\end{tabular}
\end{center}
%

Note that when several versions of a main file and/or of each child file
are to be generated, it may be convenient to set up a |Makefile| or
shell script to automatise the process.

%%%%%%%%%%%%%%%%%%%%%%%%%%%%%%%%%%%%%%%%%%%%%%%%%%%%%%%%%%%%%%%%%%%%%%%%%%%%%%%%
\subsection{Command Line Processing}
\label{sec:commandline}

The effect of redirection files can also be achieved by invoking
the \LaTeX{} compiler with a more elaborate command line.
Most conveniently this should be done as part
of a shell script or a |Makefile|.

When using \textsf{childdoc} in the main file, the following
command lines effectively perform a redirection
(note that depending on the shell being used,
backslashes may have to be doubled: `|\|' $\to$ `|\\|'):
%
\begin{center}
|... -jobname "|\textit{target}|" |\\|"|[\textit{flags}]%
|\input{childdoc.def}\childdocforward[|\textit{main}|]{|\textit{dest}|}"|
\end{center}
%
Here \textit{target} is the name of the output file,
\textit{main} is the name of the main file
and \textit{dest} is the name of the main or child file to be processed
(all filenames without extensions).
The optional argument \textit{main} can be omitted
if \textit{main} matches \textit{dest}.
Optionally, compilation \textit{flags} can be defined via |\def| commands.
This command line makes the \TeX{} engine believe
it is compiling the file \textit{target}
whose content is specified as the latter parameter.
The provided code then forwards the processing to
\textit{main} or \textit{dest} as described in \secref{sec:forward}.

%%%%%%%%%%%%%%%%%%%%%%%%%%%%%%%%%%%%%%%%%%%%%%%%%%%%%%%%%%%%%%%%%%%%%%%%%%%%%%%%
\subsection{Include by Input}
\label{sec:input}

Including child documents by |\include| has some restrictions by design.
Most notably, the content of a child document always occupies
its own set of pages; pages cannot be shared between child documents.
Usually, this behaviour makes perfect sense
because each child document contain an essential part of the document.
However, in some situations it may be desirable to compose
a document from a collection of parts
without having mandatory page breaks between then.
For this case, the package
provides a mechanism to include parts
by |\input| which can also be processed individually.
However, by construction this mechanism
requires manual handling of the content to be output.

%%%%%%%%%%%%%%%%%%%%%%%%%%%%%%%%%%%%%%%%
\DescribeMacro{\ifchilddocmanual}
The main file should be prepared as usual, see \secref{sec:include}.
However, the document body must make a distinction
between processing of an individual part and of the main document, e.g.:
%
\begin{center}
\begin{tabular}{l}
|\ifchilddocmanual|\\
|\input{\childdocname}|\\
|\||else|\\
\textit{document body with }|\input{|\textit{part}|}|\\
|\||fi|
\end{tabular}
\end{center}
%
The conditional |\ifchilddocmanual| is true whenever
a part to be included by |\input| is being compiled,
and the name of the part is stored in |\childdocname|.

%%%%%%%%%%%%%%%%%%%%%%%%%%%%%%%%%%%%%%%%
\DescribeMacro{\childdocby}
Each part to be included by |\input| should start with:
%
\begin{center}
\begin{tabular}{l}
|\input{childdoc.def}|\\
|\childdocby{|\textit{main}|}|\\
\end{tabular}
\end{center}
%
The directive |\childdocby| is similar to |\childdocof|
described in \secref{sec:include},
but the subsequent selection of content must be done manually.
To that end, both |\ifchilddoc| and |\ifchilddocmanual|
will be true upon processing of a part,
and the name of the part is stored in |\childdocname|.
Note that |\jobname| will be set to the filename of the current part
so that each part receives an individual |.aux| file
that does not interfere with the |.aux| file(s) of the main document.
This behaviour can be altered by the alternative form
|\childdocby[*]{|\textit{main}|}| (with a non-empty optional argument)
which uses the |.aux| file of the main document
by setting |\jobname| to \textit{main}.

%%%%%%%%%%%%%%%%%%%%%%%%%%%%%%%%%%%%%%%%%%%%%%%%%%%%%%%%%%%%%%%%%%%%%%%%%%%%%%%%
\subsection{Driver Development}
\label{sec:driver}

The \textsf{childdoc} mechanism can also be use for the development
of definition files such as \LaTeX{} styles or classes.
This case differs from the above setup with multiple parts
included by |\include| in that no |\includeonly| should be invoked.
This can be achieved by starting the include file
(before |\ProvidesPackage|) with:
%
\begin{center}
\begin{tabular}{l}
|\input{childdoc.def}|\\
|\childdocforward{|\textit{main}|}|\\
\end{tabular}
\end{center}
%
or alternatively with:
%
\begin{center}
\begin{tabular}{l}
|\input{childdoc.def}|\\
|\childdocby{|\textit{main}|}|\\
\end{tabular}
\end{center}
%
Both forms have slightly different effects as described above.
The main file is prepared as usual, see \secref{sec:include}.

%%%%%%%%%%%%%%%%%%%%%%%%%%%%%%%%%%%%%%%%%%%%%%%%%%%%%%%%%%%%%%%%%%%%%%%%%%%%%%%%
\subsection{Legacy Detection}
\label{sec:detection}

The directive |\childdocmain| in the main file can detect
whether the complete document or merely a child is to be compiled
even without using the directive |\childdocof|.
This method is deprecated because it is less robust
and there is no compelling reason to use it;
it is merely provided for backward compatibility
and it may be removed in future versions.

If the detection mechanism is to be used,
it is mandatory to correctly specify
the filename of the main file as the argument of |\childdocmain|:
%
\begin{center}
\begin{tabular}{l}
|\input{childdoc.def}|\\
|\childdocmain{|\textit{main}|}|\\
\end{tabular}
\end{center}
%
If |\jobname| does not match the argument \textit{main} of |\childdocmain|,
it is assumed that |\jobname| points to the child file to be compiled.
When using |\childdocmain| with the main file specified as argument,
it suffices to start a child file
with just |\input{|\textit{main}|}|
without loading of the package and using |\childdocof|.
If instead all processing is done
with the appropriate \textsf{childdoc} directives,
the argument of \textit{main} of |\childdocmain| can be empty.

An alternative version of the command line processing described
in \secref{sec:commandline} using the detection mechanism reads:
%
\begin{center}
|... -jobname "|\textit{target}|" "|[\textit{flags}]%
[|\def\jobname{|\textit{dest}|}|]|\input{|\textit{main}|}"|
\end{center}

%%%%%%%%%%%%%%%%%%%%%%%%%%%%%%%%%%%%%%%%%%%%%%%%%%%%%%%%%%%%%%%%%%%%%%%%%%%%%%%%
\subsection{Manual Code}
\label{sec:manual}

In case one cannot be certain whether the definitions file |childdoc.def|
is installed on the target \TeX{} distribution
and one prefers not to ship it,
it is conceivable to paste a few relevant commands into the sources.

To that end, drop all statements |\input{childdoc.def}|
and perform the replacements as outlined below.
Instead of |\childdocmain{|\textit{main}|}| add the following code
to the top of the main file:
%
\begin{center}
\begin{tabular}{l}
|\||ifdefined\childdocname\endinput\||fi\newif\ifchilddoc|\\
|\edef\childdocname{\scantokens\expandafter{\jobname\noexpand}}|\\
|\def\childdocmain{|\textit{main}|}\||ifx\childdocmain\childdocname\||else|\\
|\childdoctrue\includeonly{\childdocname}\let\jobname\childdocmain\||fi|\\
\end{tabular}
\end{center}
%
Instead of |\childdocof{|\textit{main}|}| just include the main file
at the top of each child file:
%
\begin{center}
|\input{|\textit{main}|}|
\end{center}
%
A simple redirection |\childdocforward{|\textit{dest}|}| is achieved by:
%
\begin{center}
|\def\jobname{|\textit{dest}|}\input{\jobname}|
\end{center}
%
The redirection with prefix
|\childdocforwardprefix[|\textit{prefix}|]{|\textit{dest}|}|
is accomplished by:
%
\begin{center}
\begin{tabular}{l}
|{\edef\jobname{\scantokens\expandafter{\jobname\noexpand}}|\\
|\def\redirectjob |\textit{prefix}|#1~~~{\gdef\jobname{|\textit{dest}|#1}}|\\
|\expandafter\redirectjob\jobname~~~}\input{\jobname}|
\end{tabular}
\end{center}

In an alternative approach,
child documents can be compiled by a specific command line
without additional code or specific definitions:
%
\begin{center}
|... -jobname "|\textit{target}|" "|[\textit{flags}]%
|\includeonly{|\textit{dest}|}\input{|\textit{main}|}"|
\end{center}
%

%%%%%%%%%%%%%%%%%%%%%%%%%%%%%%%%%%%%%%%%%%%%%%%%%%%%%%%%%%%%%%%%%%%%%%%%%%%%%%%%
%%%%%%%%%%%%%%%%%%%%%%%%%%%%%%%%%%%%%%%%%%%%%%%%%%%%%%%%%%%%%%%%%%%%%%%%%%%%%%%%
\section{Information}

%%%%%%%%%%%%%%%%%%%%%%%%%%%%%%%%%%%%%%%%%%%%%%%%%%%%%%%%%%%%%%%%%%%%%%%%%%%%%%%%
\subsection{Copyright}

Copyright \copyright{} 2017--2018 Niklas Beisert

This work may be distributed and/or modified under the
conditions of the \LaTeX{} Project Public License, either version 1.3
of this license or (at your option) any later version.
The latest version of this license is in
  \url{http://www.latex-project.org/lppl.txt}
and version 1.3 or later is part of all distributions of \LaTeX{}
version 2005/12/01 or later.

This work has the LPPL maintenance status `maintained'.

The Current Maintainer of this work is Niklas Beisert.

This work consists of the files |README.txt|, |childdoc.ins| and |childdoc.dtx|
as well as the derived files |childdoc.def|, |cdocsamp.tex|
with |cdocsch1.tex|, |cdocsch2.tex|, |cdocspt3.tex|, |cdocspt4.tex|,
|cdocsdrf.tex|, |cdocsfn1.tex|, |cdocsfn2.tex|
as well as |childdoc.pdf|.

%%%%%%%%%%%%%%%%%%%%%%%%%%%%%%%%%%%%%%%%%%%%%%%%%%%%%%%%%%%%%%%%%%%%%%%%%%%%%%%%
\subsection{Files and Installation}

The package consists of the files:
%
\begin{center}
\begin{tabular}{ll}
    |README.txt|   & readme file \\
    |childdoc.ins| & installation file \\
    |childdoc.dtx| & source file \\
    |childdoc.def| & definition file \\
    |cdocsamp.tex| & sample main file \\
    |cdocsch1.tex| & sample include file \\
    |cdocsch2.tex| & sample include file \\
    |cdocspt3.tex| & sample part file \\
    |cdocspt4.tex| & sample part file \\
    |cdocsdrf.tex| & sample redirection file \\
    |cdocsfn1.tex| & sample redirection file \\
    |cdocsfn2.tex| & sample redirection file \\
    |childdoc.pdf| & manual
\end{tabular}
\end{center}
%
The distribution consists of the files
|README.txt|, |childdoc.ins| and |childdoc.dtx|.
%
\begin{itemize}
\item
Run (pdf)\LaTeX{} on |childdoc.dtx|
to compile the manual |childdoc.pdf| (this file).
\item
Run \LaTeX{} on |childdoc.ins| to create the definitions file |childdoc.def|
and the sample |cdocsamp.tex| with include files
|cdocsch1.tex|, |cdocsch2.tex|, |cdocspt3.tex|, |cdocspt4.tex|,
|cdocsdrf.tex|, |cdocsfn1.tex|, |cdocsfn2.tex|.
Then copy the file |childdoc.def| to an appropriate directory of your \LaTeX{}
distribution, e.g.\ \textit{texmf-root}|/tex/latex/childdoc|.
\end{itemize}

%%%%%%%%%%%%%%%%%%%%%%%%%%%%%%%%%%%%%%%%%%%%%%%%%%%%%%%%%%%%%%%%%%%%%%%%%%%%%%%%
\subsection{Related CTAN Packages}

There are several other packages which offer a similar functionality:
%
\begin{itemize}
\item
The packages
\href{http://ctan.org/pkg/docmute}{\textsf{docmute}},
\href{http://ctan.org/pkg/includex}{\textsf{includex}} and
\href{http://ctan.org/pkg/standalone}{\textsf{standalone}}
provide commands to include only the document body of
a child file thus allowing both files to be compiled individually.
\item
The packages \href{http://ctan.org/pkg/subdocs}{\textsf{subdocs}}
and \href{http://ctan.org/pkg/subfiles}{\textsf{subfiles}}
provide structures in which the main and child documents can be
encapsulated and allowing them to be compiled individually.
The inclusion mechanism is different from the conventional |\include|.
\item
The package \href{http://ctan.org/pkg/combine}{\textsf{combine}}
is an elaborate solution to combine several documents into one.
\end{itemize}
%
See also the CTAN topic \href{http://ctan.org/topic/subdocs}{\textsf{subdocs}}
for further related packages.
The present package differs from the above solutions in that
a document structure constructed with the conventional |\include| mechanism
just needs two extra commands at the top of every file
such that all constituent files can be compiled individually.

%%%%%%%%%%%%%%%%%%%%%%%%%%%%%%%%%%%%%%%%%%%%%%%%%%%%%%%%%%%%%%%%%%%%%%%%%%%%%%%%
%\subsection{Feature Suggestions}
%
%The following is a list of features which may be useful for future
%versions of this package:
%%
%\begin{itemize}
%\item
%\ldots
%\end{itemize}

%%%%%%%%%%%%%%%%%%%%%%%%%%%%%%%%%%%%%%%%%%%%%%%%%%%%%%%%%%%%%%%%%%%%%%%%%%%%%%%%
\subsection{Revision History}

%%%%%%%%%%%%%%%%%%%%%%%%%%%%%%%%%%%%%%%%
\paragraph{v2.0:} 2018/12/30

\begin{itemize}
\item
immediate forward processing
\item
added |\childdocby| mechanism
\item
manual restructured
\end{itemize}

%%%%%%%%%%%%%%%%%%%%%%%%%%%%%%%%%%%%%%%%
\paragraph{v1.6:} 2018/01/17

\begin{itemize}
\item
application for development of include files
\item
corrections to manual
\end{itemize}

%%%%%%%%%%%%%%%%%%%%%%%%%%%%%%%%%%%%%%%%
\paragraph{v1.5:} 2017/05/21

\begin{itemize}
\item
more complete structuring introduced
\item
|\childdocof| introduced
\item
|\childdoc| renamed to |\childdocmain|
\item
|\childredirect| renamed to |\childdocforward| and |\childdocforwardprefix|
and functionality expanded
\end{itemize}

%%%%%%%%%%%%%%%%%%%%%%%%%%%%%%%%%%%%%%%%
\paragraph{v1.0:} 2017/04/27

\begin{itemize}
\item
manual and install package
\item
first version published on CTAN
\end{itemize}

%%%%%%%%%%%%%%%%%%%%%%%%%%%%%%%%%%%%%%%%
\paragraph{v0.6:} 2017/04/26

\begin{itemize}
\item
redirection mechanism added
\end{itemize}

%%%%%%%%%%%%%%%%%%%%%%%%%%%%%%%%%%%%%%%%
\paragraph{v0.5:} 2017/04/26

\begin{itemize}
\item
functionality in definition file
\end{itemize}


%%%%%%%%%%%%%%%%%%%%%%%%%%%%%%%%%%%%%%%%%%%%%%%%%%%%%%%%%%%%%%%%%%%%%%%%%%%%%%%%
%%%%%%%%%%%%%%%%%%%%%%%%%%%%%%%%%%%%%%%%%%%%%%%%%%%%%%%%%%%%%%%%%%%%%%%%%%%%%%%%
%%%%%%%%%%%%%%%%%%%%%%%%%%%%%%%%%%%%%%%%%%%%%%%%%%%%%%%%%%%%%%%%%%%%%%%%%%%%%%%%
\appendix

\settowidth\MacroIndent{\rmfamily\scriptsize 000\ }

 \DocInput{childdoc.dtx}

\end{document}
%</driver>
% \fi
%
% %%%%%%%%%%%%%%%%%%%%%%%%%%%%%%%%%%%%%%%%%%%%%%%%%%%%%%%%%%%%%%%%%%%%%%%%%%%%%%
% %%%%%%%%%%%%%%%%%%%%%%%%%%%%%%%%%%%%%%%%%%%%%%%%%%%%%%%%%%%%%%%%%%%%%%%%%%%%%%
% \section{Sample}
%\iffalse
%<*samplemain>
%\fi
%
% The following presents a sample document
% with two chapters, two parts, a title page,
% a compile flag as well as three forwarding files to set the flag.
% It consists of eight |.tex| files:
% \begin{center}
% \begin{tabular}{ll}
% |cdocsamp.tex|&main file\\
% |cdocsch1.tex|&include file for chapter 1\\
% |cdocsch2.tex|&include file for chapter 2\\
% |cdocspt3.tex|&include file for part 3\\
% |cdocspt4.tex|&include file for part 4\\
% |cdocsdrf.tex|&forwarding file for main file in draft mode\\
% |cdocsfi1.tex|&forwarding file for final version of chapter 1\\
% |cdocsfi2.tex|&forwarding file for final version of chapter 2\\
% \end{tabular}
% \end{center}
% Each of the eight files can be compiled directly by the \LaTeX{} compiler.
%
% %%%%%%%%%%%%%%%%%%%%%%%%%%%%%%%%%%%%%%
% \paragraph{Main File.}
%
% The main file is called |cdocsamp.tex|.
%
% Load the \textsf{childdoc} definitions and
% declare the filename for the main document:
%    \begin{macrocode}
\input{childdoc.def}
\childdocmain{}
%    \end{macrocode}

% Optional override for |\version| flag:
%    \begin{macrocode}
%%\ifchilddoc\else\providecommand{\version}{draft}\fi
%    \end{macrocode}

% Define the default values for the |\version| flag
% (|final| for the main file and |draft| for childs):
%    \begin{macrocode}
\ifchilddoc
\providecommand{\version}{draft}
\else
\providecommand{\version}{final}
\fi
%    \end{macrocode}

% Load the standard document class:
%    \begin{macrocode}
\documentclass[12pt]{article}
%    \end{macrocode}

% Start the document body:
%    \begin{macrocode}
\begin{document}
%    \end{macrocode}

% Declare a title page.
% Print title, part of document being processed and version flag:
%    \begin{macrocode}
\addtocounter{page}{-1}
\begin{center}
{\LARGE\bfseries{}childdoc example\par}
\vspace{1cm}
\ifchilddoc
\ifchilddocmanual part\else chapter\fi:
`\childdocname' of `\childdocjob'\par
\else
main document: `\childdocjob'\par
\fi
version: \version\par
\end{center}
\newpage
%    \end{macrocode}

% Manually include selected file,
% otherwise process as usual:
%    \begin{macrocode}
\ifchilddocmanual
\section*{part `\childdocname'}
\input{\childdocname}
\else
%    \end{macrocode}

% Include the two chapters:
%    \begin{macrocode}
\include{cdocsch1}
\include{cdocsch2}
%    \end{macrocode}

% Include the two parts unless only chapters should be displayed:
%    \begin{macrocode}
\ifchilddoc\else
\section{part three}
\input{cdocspt3}
\section{part four}
\input{cdocspt4}
\fi
%    \end{macrocode}

% Process as usual until here:
%    \begin{macrocode}
\fi
%    \end{macrocode}

% End of document body:
%    \begin{macrocode}
\end{document}
%    \end{macrocode}
%\iffalse
%</samplemain>
%\fi
%
% %%%%%%%%%%%%%%%%%%%%%%%%%%%%%%%%%%%%%%
% \paragraph{Chapter Include Files.}
%
% The include files are called |cdocsch1.tex| and |cdocsch2.tex|.
%
%\iffalse
%<*samplechap1|samplechap2>
%\fi

% Optional override for |\version| flag:
%    \begin{macrocode}
%%\providecommand{\version}{final}
%    \end{macrocode}

% Include the main document:
%    \begin{macrocode}
\input{childdoc.def}
\childdocof{cdocsamp}
%    \end{macrocode}

%\iffalse
%</samplechap1|samplechap2>
%\fi
%
%\iffalse
%<*samplechap1>
%\fi
% Some text for chapter 1:
%    \begin{macrocode}
\section{one}
some text in chapter one
%    \end{macrocode}

%\iffalse
%</samplechap1>
%\fi
% Some text for chapter 2:
%\iffalse
%<*samplechap2>
%\fi
%    \begin{macrocode}
\section{two}
more text in chapter two
%    \end{macrocode}

%\iffalse
%</samplechap2>
%\fi
%
% %%%%%%%%%%%%%%%%%%%%%%%%%%%%%%%%%%%%%%
% \paragraph{Part Include Files.}
%
% The include files are called |cdocspt3.tex| and |cdocspt4.tex|.
%
%\iffalse
%<*samplepart3|samplepart4>
%\fi

% Optional override for |\version| flag:
%    \begin{macrocode}
%%\providecommand{\version}{final}
%    \end{macrocode}

% Include the main document:
%    \begin{macrocode}
\input{childdoc.def}
\childdocby{cdocsamp}
%    \end{macrocode}

%\iffalse
%</samplepart3|samplepart4>
%\fi
%
%\iffalse
%<*samplepart3>
%\fi
% Some text for part 3:
%    \begin{macrocode}
some text in part three
%    \end{macrocode}

%\iffalse
%</samplepart3>
%\fi
% Some text for part 4:
%\iffalse
%<*samplepart4>
%\fi
%    \begin{macrocode}
more text in part four
%    \end{macrocode}

%\iffalse
%</samplepart4>
%\fi
%
% %%%%%%%%%%%%%%%%%%%%%%%%%%%%%%%%%%%%%%
% \paragraph{Forwarding for a Complete Draft.}
%
% The following forwarding file |cdocsdrf.tex|
% compiles the main document in draft mode:
%\iffalse
%<*sampledraft>
%\fi
%    \begin{macrocode}
\def\version{draft}
\input{childdoc.def}
\childdocforward{cdocsamp}
%    \end{macrocode}

%\iffalse
%</sampledraft>
%\fi
%
% %%%%%%%%%%%%%%%%%%%%%%%%%%%%%%%%%%%%%%
% \paragraph{Forwarding for Final Version of the Chapters.}
%
% The following forwarding files |cdocsfn1.tex| and |cdocsfn2.tex|
% (with identical content)
% compile the final versions of the child documents
% |cdocsch1.tex| and |cdocsch2.tex|, respectively:
%\iffalse
%<*samplefinal>
%\fi
%    \begin{macrocode}
\def\version{final}
\input{childdoc.def}
\childdocforwardprefix[cdocsamp]{cdocsfn}{cdocsch}
%    \end{macrocode}

%\iffalse
%</samplefinal>
%\fi
%
% %%%%%%%%%%%%%%%%%%%%%%%%%%%%%%%%%%%%%%
% \paragraph{Command Line Processing.}
%
% The following three command lines generate the output files
% |cdocscld|, |cdocscl1| and |cdocscl2|
% which should be identical to
% |cdocsdrf|, |cdocsch1| and |cdocsfn2|, respectively:
% \begin{center}
% \begin{tabular}{l}
% |latex -jobname cdocscld \|\\
% |  "\def\version{draft}\input{childdoc.def}\childdocforward{cdocsamp}"|\\
% |latex -jobname cdocscl1 \|\\
% |  "\input{childdoc.def}\childdocforward[cdocsamp]{cdocsch1}"|\\
% |latex -jobname cdocscl2 \|\\
% |  "\def\version{final}\input{childdoc.def}\childdocforward{cdocsch2}"|
% \end{tabular}
% \end{center}
% Note that the trailing backslash on each first line
% merely continues the input to the second line
% (for convenient cut ant paste).
% Furthermore, the command |latex| can be replaced by any
% of its alternative versions such as |pdflatex|.
%
% %%%%%%%%%%%%%%%%%%%%%%%%%%%%%%%%%%%%%%%%%%%%%%%%%%%%%%%%%%%%%%%%%%%%%%%%%%%%%%
% %%%%%%%%%%%%%%%%%%%%%%%%%%%%%%%%%%%%%%%%%%%%%%%%%%%%%%%%%%%%%%%%%%%%%%%%%%%%%%
% \section{Implementation}
%\iffalse
%<*package>
%\fi
%
% This section describes the definitions file |childdoc.def|.

% The definitions cannot be loaded using |\usepackage| or |\RequirePackage|
% which has a mechanism to prevent loading a style file more than once.
% When loading the definitions by means of |\input|
% multiple instances have to be prevented manually:
%\iffalse
%This code needs to be before the `\ProvidesFile' directive
%which is defined at the beginning of this file.
%Therefore it is also placed there and commented out here.
%</package>
%<*discard>
%\fi
%    \begin{macrocode}
\ifdefined\childdocmain\endinput\fi
%    \end{macrocode}
%\iffalse
%</discard>
%<*package>
%\fi
%
% \macro{\ifchilddoc}
% \macro{\ifchilddocmanual}
% The conditional |\ifchilddoc| tells whether a
% child (true) or main (false) document is being compiled.
% The conditional |\ifchilddocmanual| tells whether
% the |\includeonly| mechanism is used (false) or
% the selection of child files must be performed manually (true).
% The definitions initialise to false:
%    \begin{macrocode}
\newif\ifchilddoc
\newif\ifchilddocmanual
%    \end{macrocode}

% \macro{\childdocname}
% \macro{\childdocjob}
% The macro |\childdocname| stores the name of the main document
% to be compiled. The macro |\childdocjob| stores the name of
% the document on which the \LaTeX{} compiler was originally invoked.
% The content of |\jobname| cannot be compared
% to filenames specified in the source due to different catcodes.
% The following code rescans |\jobname|, stores the result
% in |\childdocname| and saves a copy in |\childdocjob|:
%    \begin{macrocode}
\edef\childdocname{\scantokens\expandafter{\jobname\noexpand}}
\let\childdocjob\childdocname
%    \end{macrocode}

% \macro{\childdocdisable}
% The macro |\childdocdisable| prevents the main file
% from being processed more than once.
% At this stage, the main document command |\childdocmain|
% is assumed to be called once again where it should do nothing.
% Any subsequent call to it should prevent
% a secondary processing of the main document
% It overwrites the forwarding commands
% |\childdocof| and |\childdocforward|
% with empty macros to prevent further inclusions of the main document:
%    \begin{macrocode}
\newcommand{\childdocdisable}
{
  \renewcommand{\childdocmain}[1]{\renewcommand{\childdocmain}[1]{\endinput}}
  \renewcommand{\childdocof}[1]{}
  \renewcommand{\childdocby}[2][]{}
  \renewcommand{\childdocforward}[2][]{}
  \renewcommand{\childdocdisable}{}
}
%    \end{macrocode}

% \macro{\childdocmain}
% The macro |\childdocmain| is to be called at the top of the main file
% with nothing or the main filename (without extension) as argument.
% First, it breaks loops.
% If the argument is not empty and does not match |\childdocname|
% (which is set by the first inclusion of |childdoc.def|),
% |\ifchilddoc| is set to true, |\includeonly| is applied to the child file
% and |\jobname| is set to the main file
% (for proper handling of |.aux| files):
%    \begin{macrocode}
\newcommand{\childdocmain}[1]
{
  \childdocdisable\childdocmain{}
  \if?#1?\else
    \begingroup
      \def\childdoctmp{#1}
      \ifx\childdoctmp\childdocname
        \def\childdoctmp{}
      \else
        \def\childdoctmp
        {
          \childdoctrue
          \includeonly{\childdocname}
          \def\childdocjob{#1}
          \def\jobname{#1}
        }
      \fi
      \expandafter
    \endgroup
    \childdoctmp
  \fi
}
%    \end{macrocode}

% \macro{\childdocof}
% The command |\childdocof| redirects
% compilation to the main file |#1|.
%    \begin{macrocode}
\newcommand{\childdocof}[1]
{
  \childdocdisable
  \childdoctrue
  \includeonly{\childdocname}
  \def\jobname{#1}
  \def\childdocjob{#1}
  \input{#1}
}
%    \end{macrocode}

% \macro{\childdocby}
% The command |\childdocby| ....
%    \begin{macrocode}
\newcommand{\childdocby}[2][]
{
  \childdocdisable
  \childdoctrue
  \childdocmanualtrue
  \if?#1?\else
    \def\jobname{#2}
  \fi
  \def\childdocjob{#2}
  \input{#2}
  \endinput
}
%    \end{macrocode}

% \macro{\childdocforward}
% The command |\childdocforward| redirects
% compilation to the main file or
% (if the optional argument is given) a child file.
% Parameters are set as if the main file
% or a child file starting with |\childdocof| was compiled.
% Then compilation is handed over to the main file:
%    \begin{macrocode}
\newcommand{\childdocforward}[2][]
{
  \begingroup
    \if?#1?
      \def\childdoctmp
      {
        \def\childdocname{#2}
        \def\childdocjob{#2}
        \def\jobname{#2}
        \input{#2}
        \endinput
      }
    \else
      \def\childdoctmp
      {
        \childdocdisable
        \def\childdocname{#2}
        \childdoctrue
        \includeonly{#2}
        \def\childdocjob{#1}
        \def\jobname{#1}
        \input{#1}
        \endinput
      }
    \fi
    \expandafter
  \endgroup
  \childdoctmp
}
%    \end{macrocode}

% \macro{\childdocforwardprefix}
% The command |\childdocforwardprefix| redirects
% compilation to the main or a child file by means of a pattern.
% The prefix |#1| in the current filename is replaced by |#2|
% and the suffix of the current filename is kept
% (it is assumed that the filename does not contain the substring `|~~~|'
% which is used as a delimiter).
% Compilation is handed over to the new file by |\childdocforward|:
%    \begin{macrocode}
\newcommand{\childdocforwardprefix}[3][]
{
  \begingroup
    \def\childdocextract #2##1~~~{\def\childdoctmp{\childdocforward[#1]{#3##1}}}
    \expandafter\childdocextract\childdocname~~~
    \expandafter
  \endgroup
  \childdoctmp
}
%    \end{macrocode}

% \macro{\childdoc}
% The deprecated macro |\childdoc| is a legacy version of |\childdocmain|:
%    \begin{macrocode}
\newcommand{\childdoc}{\childdocmain}
%    \end{macrocode}

% \macro{\childdocredirect}
% The deprecated macro |\childdocredirect| is a legacy version
% of |\childdocforward| and |\childdocforwardprefix|:
%    \begin{macrocode}
\newcommand{\childdocredirect}[2][]
{
  \begingroup
    \if?#1?
      \def\childdoctmp{\childdocforward{#2}}
    \else
      \def\childdoctmp{\childdocforwardprefix{#1}{#2}}
    \fi
    \expandafter
  \endgroup
  \childdoctmp
}
%    \end{macrocode}

%\iffalse
%</package>
%\fi
%
\endinput
\childdocforward{cdocsamp}"|\\
% |latex -jobname cdocscl1 \|\\
% |  "% \iffalse
%
% childdoc.dtx Copyright (C) 2017-2018 Niklas Beisert
%
% This work may be distributed and/or modified under the
% conditions of the LaTeX Project Public License, either version 1.3
% of this license or (at your option) any later version.
% The latest version of this license is in
%   http://www.latex-project.org/lppl.txt
% and version 1.3 or later is part of all distributions of LaTeX
% version 2005/12/01 or later.
%
% This work has the LPPL maintenance status `maintained'.
%
% The Current Maintainer of this work is Niklas Beisert.
%
% This work consists of the files childdoc.dtx and childdoc.ins
% and the derived files childdoc.def and cdocsamp.tex with
% cdocsch1.tex, cdocsch2.tex, cdocsdrf.tex, cdocsfn1.tex, cdocsfn2.tex.
%
%<package>\ifdefined\childdocmain\endinput\fi
%<package>\ProvidesFile{childdoc.def}[2018/12/30 v2.0 child document driver]
%<samplemain>\ProvidesFile{cdocsamp.tex}[2018/12/30 v2.0 sample for childdoc]
%<*driver>
%\ProvidesFile{childdoc.drv}[2018/12/30 v2.0 childdoc reference manual file]
\PassOptionsToClass{10pt,a4paper}{article}
\documentclass{ltxdoc}

\usepackage[margin=35mm]{geometry}
\usepackage{hyperref}
\usepackage{hyperxmp}
\usepackage[usenames]{color}

\hypersetup{colorlinks=true}
\hypersetup{pdfstartview=FitH}
\hypersetup{pdfpagemode=UseNone}
\hypersetup{pdfsource={}}
\hypersetup{pdflang={en-UK}}
\hypersetup{pdfcopyright={Copyright 2017-2018 Niklas Beisert.
  This work may be distributed and/or modified under the
  conditions of the LaTeX Project Public License, either version 1.3
  of this license or (at your option) any later version.}}
\hypersetup{pdflicenseurl={http://www.latex-project.org/lppl.txt}}
\hypersetup{pdfcontactaddress={ETH Zurich, ITP, HIT K,
  Wolfgang-Pauli-Strasse 27}}
\hypersetup{pdfcontactpostcode={8093}}
\hypersetup{pdfcontactcity={Zurich}}
\hypersetup{pdfcontactcountry={Switzerland}}
\hypersetup{pdfcontactemail={nbeisert@itp.phys.ethz.ch}}
\hypersetup{pdfcontacturl={http://people.phys.ethz.ch/\xmptilde nbeisert/}}

\newcommand{\secref}[1]{\hyperref[#1]{section \ref*{#1}}}

\parskip1ex
\parindent0pt
\let\olditemize\itemize
\def\itemize{\olditemize\parskip0pt}

\begin{document}

\title{The \textsf{childdoc} Package}
\hypersetup{pdftitle={The childdoc Package}}
\author{Niklas Beisert\\[2ex]
  Institut f\"ur Theoretische Physik\\
  Eidgen\"ossische Technische Hochschule Z\"urich\\
  Wolfgang-Pauli-Strasse 27, 8093 Z\"urich, Switzerland\\[1ex]
  \href{mailto:nbeisert@itp.phys.ethz.ch}
  {\texttt{nbeisert@itp.phys.ethz.ch}}}
\hypersetup{pdfauthor={Niklas Beisert}}
\hypersetup{pdfsubject={Manual for the LaTeX2e Package childdoc}}
\date{30 December 2018, \textsf{v2.0}}
\maketitle

\begin{abstract}\noindent
\textsf{childdoc} is a \LaTeXe{} package
that enables the direct compilation
of document sections included by |\include|
to individual files.
\end{abstract}

\begingroup
\parskip0ex
\tableofcontents
\endgroup

%%%%%%%%%%%%%%%%%%%%%%%%%%%%%%%%%%%%%%%%%%%%%%%%%%%%%%%%%%%%%%%%%%%%%%%%%%%%%%%%
%%%%%%%%%%%%%%%%%%%%%%%%%%%%%%%%%%%%%%%%%%%%%%%%%%%%%%%%%%%%%%%%%%%%%%%%%%%%%%%%
\section{Introduction}

\LaTeX{} provides a mechanism to structure a large document (such as a book)
into a main file and several child files (containing the chapters)
using the |\include| command.
This mechanism is beneficial for documents
which span hundreds of pages in order to
make the source file(s) more manageable.
Moreover, compilation can be restricted to
selected child files by means of the |\includeonly| command.
The latter feature can be used to reduce the compilation time while editing
(this was significantly more useful in the earlier days of \LaTeX{})
or to generate a smaller document which is easier to navigate.
Another application of |\includeonly| is to generate
documents consisting of selected parts of the complete document.

However, there are a few drawbacks of the plain |\include| mechanism:
\begin{itemize}
\item
The child files cannot be compiled on their own,
they can only be compiled via the main file.
A naive editing environment
(such as a text editor with an option
to have the current file processed by \LaTeX)
may require one to switch to the main file before compiling;
attempting to compile the child file produces errors.
\item
The main file must be modified (each time)
to adjust the |\includeonly| command
to the present needs. This easily leaves the main file in a messy state.
\item
The generated document will always carry the filename
of the main document. This is inconvenient if
several child files are to be compiled and
to be kept for distribution.
\end{itemize}

The present package provides a simple interface
to make child files individually compilable by \LaTeX{}.
Compiling a child file then has the same effect as compiling
the main file with an |\includeonly| command
to select the appropriate child.
Moreover the generated document will carry the name of the child
rather than the main file.
This resolves all three above issues.

This feature is meant to make the editing of books,
thesis documents and lecture notes somewhat more convenient.
However, the package can also be used efficiently for
composing a series of documents (such as exercise sheets)
which are typically distributed individually.
It then assists the author in generating the individual documents
(potentially in different versions)
as well as a document containing the collected series.
Another application is in developing style files
or other kinds of included material
where compilation of the style file could redirect
to a sample or test file.

%%%%%%%%%%%%%%%%%%%%%%%%%%%%%%%%%%%%%%%%%%%%%%%%%%%%%%%%%%%%%%%%%%%%%%%%%%%%%%%%
%%%%%%%%%%%%%%%%%%%%%%%%%%%%%%%%%%%%%%%%%%%%%%%%%%%%%%%%%%%%%%%%%%%%%%%%%%%%%%%%
\section{Usage}

First of all, the package \textsf{childdoc} is \emph{not} a standard
\LaTeXe{} |.sty| style file! Therefore it needs to be invoked in
a non-standard way.

%%%%%%%%%%%%%%%%%%%%%%%%%%%%%%%%%%%%%%%%%%%%%%%%%%%%%%%%%%%%%%%%%%%%%%%%%%%%%%%%
\subsection{Included Files}
\label{sec:include}

%%%%%%%%%%%%%%%%%%%%%%%%%%%%%%%%%%%%%%%%
\DescribeMacro{\childdocmain}
To use the package, add the commands
\begin{center}
\begin{tabular}{l}
|\input{childdoc.def}|\\
|\childdocmain{}|\\
\end{tabular}
\end{center}
at the very top of the main \LaTeX{} file,
in particular \emph{before} the |\documentclass| statement!
The argument of |\childdocmain| should be left empty
(but it must be present).

%%%%%%%%%%%%%%%%%%%%%%%%%%%%%%%%%%%%%%%%
\DescribeMacro{\childdocof}
Furthermore, add the commands
\begin{center}
\begin{tabular}{l}
|\input{childdoc.def}|\\
|\childdocof{|\textit{main}|}|\\
\end{tabular}
\end{center}
at the top of every child file \textit{child}
which is included by |\include{|\textit{child}|}|
from within the main file
(or at least for those files to be compiled individually).
The argument \textit{main} must be the filename of the main file.

There are a couple of
considerations in setting up the main and child documents:

%%%%%%%%%%%%%%%%%%%%%%%%%%%%%%%%%%%%%%%%
\paragraph{Restrictions.}

Please note the following restrictions:
\begin{itemize}
\item
|\childdocmain| must be called with one argument \textit{main}
to ensure compatibility with earlier version of the package.
It must either be empty (|\childdocmain{}|)
or precisely match the filename of the main file in which it is specified.
See \secref{sec:detection} for further information.
\item
The filename \textit{main} must be specified without the |.tex| extension.
\item
The filename \textit{main} is case sensitive
(even in case-insensitive file systems)
due to internal string comparison.
\item
The argument \textit{main} should be fully expanded, it cannot be a macro.
\item
Subdirectories and special characters should be avoided in filenames.
\item
The command |\childdocmain{|\textit{main}|}| must be followed by a whitespace.
It should not be followed immediately by another command
or by a comment mark `|%|'.
This is because the \TeX{} parser reads the token immediately following
the argument of |\childdocmain| and puts it
at the beginning of every child section;
however, a white\-space is ignored.
\end{itemize}

%%%%%%%%%%%%%%%%%%%%%%%%%%%%%%%%%%%%%%%%
\paragraph{Content of Main File.}

It is advisable to place all content in the child files included by |\include|.
Any output contained in the main file will appear in all child documents
unless suppressed manually;
it cannot be suppressed automatically by the |\includeonly| directive
and thus should normally be avoided.
A method to include some content in the main file
by means of conditional processing is described in \secref{sec:conditional}.

%%%%%%%%%%%%%%%%%%%%%%%%%%%%%%%%%%%%%%%%
\paragraph{Page Numbering.}

When only a part of the document is compiled,
the appropriate numbering of pages
(as well as other status parameters)
is determined from the |.aux| files.
The latter contain information from previous passes.
However this information needs to propagate through
all intermediate child documents.
Therefore the page numbering in child documents may well
be inconsistent until the complete document is compiled at least once.

A useful (if unconventional) way to always ensure a consistent
page numbering is to restart the numbering in each child document
and denote the pages by `\textit{child}|.|\textit{page}'
where \textit{child} represents the chapter/section number of the child file.
This can be achieved by the command
|\numberwithin{page}{|\textit{child}|}|
of the \textsf{amsmath} package
where \textit{child} can be |chapter| or |section|
depending on the chosen structuring.
Alternatively, one can modify the macro |\thepage| appropriately
and reset the counter |page| at the start of each child file.

%%%%%%%%%%%%%%%%%%%%%%%%%%%%%%%%%%%%%%%%%%%%%%%%%%%%%%%%%%%%%%%%%%%%%%%%%%%%%%%%
\subsection{Conditional Processing}
\label{sec:conditional}

The package provides a mechanism to compile different versions
of a document. To customise the versions further some conditional processing
can come in handy to distinguish which version is being compiled.
The package provides two macros to describe the compilation context:

%%%%%%%%%%%%%%%%%%%%%%%%%%%%%%%%%%%%%%%%
\DescribeMacro{\ifchilddoc}
The conditional |\ifchilddoc| distinguishes between the compilation of
child documents and the main document:
%
\begin{center}
|\ifchilddoc |\textit{child-code}| |[|\||else |\textit{main-code}]| \||fi|
\end{center}

%%%%%%%%%%%%%%%%%%%%%%%%%%%%%%%%%%%%%%%%
\DescribeMacro{\childdocname}
\DescribeMacro{\childdocjob}
The macro |\childdocname| contains the filename (without extension)
of the main or child file being processed.
Note that |\childdocjob| will always contain the name of the main file.

%%%%%%%%%%%%%%%%%%%%%%%%%%%%%%%%%%%%%%%%
\paragraph{Title Page.}

Conditional processing can be used to include a title or banner page
in the main document when proper precautions are taken.
Importantly, the code in the main file should ensure that the page counter
(as well as other status parameters which are stored in the |.aux| files)
takes the same value after the conditional processing.
Otherwise the page numbers may take divergent values
depending on which part is compiled.

For example, a title page could be declared by:
%
\begin{center}
\begin{tabular}{l}
|\ifchilddoc\||else|\\
|\addtocounter{page}{-1}|\\
\textit{code for title page}\\
|\newpage|\\
|\||fi|
\end{tabular}
\end{center}
%
A banner page for the child documents can be generated by:
%
\begin{center}
\begin{tabular}{l}
|\ifchilddoc|\\
|\addtocounter{page}{-1}|\\
\textit{code for banner page}\\
|\newpage|\\
|\||fi|
\end{tabular}
\end{center}
%
Here one could write a message such as:
\begin{center}
|This is the part \childdocname{} of \childdocjob{}.|
\end{center}

%%%%%%%%%%%%%%%%%%%%%%%%%%%%%%%%%%%%%%%%%%%%%%%%%%%%%%%%%%%%%%%%%%%%%%%%%%%%%%%%
\subsection{Flags}
\label{sec:flags}

The package makes it easy to generate different versions
of the main or child documents.
To this end compilation flags can be defined
and assigned different default values.
They will be particularly useful in conjunction
with the forwarding mechanism described in \secref{sec:forward}.

For example, it may be useful to have a flag |\version|
which can be set to |draft| or |final|.
The document source will contain some conditional code
depending on the value of |\version|.
Suppose further, the flag should default to |final| for the main file
and to |draft| for child files
which is a natural assignment for editing the document.
This is achieved by placing the following code
in the preamble of the main document
(below the |\childdocmain| directive):
%
\begin{center}
\begin{tabular}{l}
|\ifchilddoc|\\
|\providecommand{\version}{draft}|\\
|\||else|\\
|\providecommand{\version}{final}|\\
|\||fi|
\end{tabular}
\end{center}
%
The definition by |\providecommand| makes sure
that previous definitions are not overwritten.
Further statements |\providecommand{\version}{...}|
can thus be added before the above code to override it.

For the main file, one might add a line
(between |\childdocmain| and the above block)
%
\begin{center}
|%\ifchilddoc\||else\providecommand{\version}{draft}\||fi|
\end{center}
%
which can be uncommented to produce a draft version.
Likewise one can add a line to the very top of a child file
(above the |\childdocof{|\textit{main}|}| directive)
%
\begin{center}
|%\providecommand{\version}{final}|
\end{center}
%
which can be uncommented to produce the final version of this child document.

%%%%%%%%%%%%%%%%%%%%%%%%%%%%%%%%%%%%%%%%%%%%%%%%%%%%%%%%%%%%%%%%%%%%%%%%%%%%%%%%
\subsection{Forwarding}
\label{sec:forward}

Different versions of the main or child documents
using compilation flags as described in \secref{sec:flags}
can be (permanently) stored in different files
for convenient compilation, viewing and distribution.
To this end, the package defines a command
to pass on compilation to a different file:

%%%%%%%%%%%%%%%%%%%%%%%%%%%%%%%%%%%%%%%%
\DescribeMacro{\childdocforward}
The command |\childdocforward| redirects processing to
another source file:
%
\begin{center}
\begin{tabular}{l}
|\input{childdoc.def}|\\
|\childdocforward[|\textit{main}|]{|\textit{dest}|}|\\
\end{tabular}
\end{center}
%
The argument \textit{dest} is the destination file
(without extension).
It should be the main file or one of the child files.
Note that further \textsf{childdoc} directives
such as |\childdocof| and |\childdocforward|
in the indicated file will be processed in this form.
The optional argument \textit{main}
passes on directly to the main file \textit{main}
while pretending to compile the child \textit{dest}.
This form behaves as if \textit{dest}
issues |\childdocof{|\textit{main}|}| right away,
and no further \textsf{childdoc} directives will be processed.

%%%%%%%%%%%%%%%%%%%%%%%%%%%%%%%%%%%%%%%%
\DescribeMacro{\...prefix}
In the alternative form |\childdocforwardprefix|,
%
\begin{center}
\begin{tabular}{l}
|\input{childdoc.def}|\\
|\childdocforwardprefix[|\textit{main}|]{|\textit{prefix}|}{|\textit{dest}|}|
\end{tabular}
\end{center}
%
the destination file is determined by a pattern
depending on the current file:
To make this work, the current file must be called
`{\textit{prefix}\hspace{0.2em}\textit{suffix}}'
with \textit{prefix} matching precisely the argument.
Processing is then passed on to the file
`{\textit{dest}\hspace{0.2em}\textit{suffix}}'.
Surely, the same effect is achieved by
directly specifying the
argument `{\textit{dest}\hspace{0.2em}\textit{suffix}}'
in the first form.
However, that requires to set up a different file
for each child. With the alternative form of the command
all these files can have exactly the same content
which simplifies setting them up and maintaining them.

For example, the following file |draft.tex|
with a compilation flag |\version| as described in \secref{sec:flags}
compiles the main document as a draft:
%
\begin{center}
\begin{tabular}{l}
|\def\version{draft}|\\
|\input{childdoc.def}|\\
|\childdocforward{|\textit{main}|}|
\end{tabular}
\end{center}
%
Likewise, the following files |final|\textit{nn}|.tex|
compile the final version of the child document
|child|\textit{nn}|.tex|:
%
\begin{center}
\begin{tabular}{l}
|\def\version{final}|\\
|\input{childdoc.def}|\\
|\childdocforwardprefix{final}{child}|
\end{tabular}
\end{center}
%

Note that when several versions of a main file and/or of each child file
are to be generated, it may be convenient to set up a |Makefile| or
shell script to automatise the process.

%%%%%%%%%%%%%%%%%%%%%%%%%%%%%%%%%%%%%%%%%%%%%%%%%%%%%%%%%%%%%%%%%%%%%%%%%%%%%%%%
\subsection{Command Line Processing}
\label{sec:commandline}

The effect of redirection files can also be achieved by invoking
the \LaTeX{} compiler with a more elaborate command line.
Most conveniently this should be done as part
of a shell script or a |Makefile|.

When using \textsf{childdoc} in the main file, the following
command lines effectively perform a redirection
(note that depending on the shell being used,
backslashes may have to be doubled: `|\|' $\to$ `|\\|'):
%
\begin{center}
|... -jobname "|\textit{target}|" |\\|"|[\textit{flags}]%
|\input{childdoc.def}\childdocforward[|\textit{main}|]{|\textit{dest}|}"|
\end{center}
%
Here \textit{target} is the name of the output file,
\textit{main} is the name of the main file
and \textit{dest} is the name of the main or child file to be processed
(all filenames without extensions).
The optional argument \textit{main} can be omitted
if \textit{main} matches \textit{dest}.
Optionally, compilation \textit{flags} can be defined via |\def| commands.
This command line makes the \TeX{} engine believe
it is compiling the file \textit{target}
whose content is specified as the latter parameter.
The provided code then forwards the processing to
\textit{main} or \textit{dest} as described in \secref{sec:forward}.

%%%%%%%%%%%%%%%%%%%%%%%%%%%%%%%%%%%%%%%%%%%%%%%%%%%%%%%%%%%%%%%%%%%%%%%%%%%%%%%%
\subsection{Include by Input}
\label{sec:input}

Including child documents by |\include| has some restrictions by design.
Most notably, the content of a child document always occupies
its own set of pages; pages cannot be shared between child documents.
Usually, this behaviour makes perfect sense
because each child document contain an essential part of the document.
However, in some situations it may be desirable to compose
a document from a collection of parts
without having mandatory page breaks between then.
For this case, the package
provides a mechanism to include parts
by |\input| which can also be processed individually.
However, by construction this mechanism
requires manual handling of the content to be output.

%%%%%%%%%%%%%%%%%%%%%%%%%%%%%%%%%%%%%%%%
\DescribeMacro{\ifchilddocmanual}
The main file should be prepared as usual, see \secref{sec:include}.
However, the document body must make a distinction
between processing of an individual part and of the main document, e.g.:
%
\begin{center}
\begin{tabular}{l}
|\ifchilddocmanual|\\
|\input{\childdocname}|\\
|\||else|\\
\textit{document body with }|\input{|\textit{part}|}|\\
|\||fi|
\end{tabular}
\end{center}
%
The conditional |\ifchilddocmanual| is true whenever
a part to be included by |\input| is being compiled,
and the name of the part is stored in |\childdocname|.

%%%%%%%%%%%%%%%%%%%%%%%%%%%%%%%%%%%%%%%%
\DescribeMacro{\childdocby}
Each part to be included by |\input| should start with:
%
\begin{center}
\begin{tabular}{l}
|\input{childdoc.def}|\\
|\childdocby{|\textit{main}|}|\\
\end{tabular}
\end{center}
%
The directive |\childdocby| is similar to |\childdocof|
described in \secref{sec:include},
but the subsequent selection of content must be done manually.
To that end, both |\ifchilddoc| and |\ifchilddocmanual|
will be true upon processing of a part,
and the name of the part is stored in |\childdocname|.
Note that |\jobname| will be set to the filename of the current part
so that each part receives an individual |.aux| file
that does not interfere with the |.aux| file(s) of the main document.
This behaviour can be altered by the alternative form
|\childdocby[*]{|\textit{main}|}| (with a non-empty optional argument)
which uses the |.aux| file of the main document
by setting |\jobname| to \textit{main}.

%%%%%%%%%%%%%%%%%%%%%%%%%%%%%%%%%%%%%%%%%%%%%%%%%%%%%%%%%%%%%%%%%%%%%%%%%%%%%%%%
\subsection{Driver Development}
\label{sec:driver}

The \textsf{childdoc} mechanism can also be use for the development
of definition files such as \LaTeX{} styles or classes.
This case differs from the above setup with multiple parts
included by |\include| in that no |\includeonly| should be invoked.
This can be achieved by starting the include file
(before |\ProvidesPackage|) with:
%
\begin{center}
\begin{tabular}{l}
|\input{childdoc.def}|\\
|\childdocforward{|\textit{main}|}|\\
\end{tabular}
\end{center}
%
or alternatively with:
%
\begin{center}
\begin{tabular}{l}
|\input{childdoc.def}|\\
|\childdocby{|\textit{main}|}|\\
\end{tabular}
\end{center}
%
Both forms have slightly different effects as described above.
The main file is prepared as usual, see \secref{sec:include}.

%%%%%%%%%%%%%%%%%%%%%%%%%%%%%%%%%%%%%%%%%%%%%%%%%%%%%%%%%%%%%%%%%%%%%%%%%%%%%%%%
\subsection{Legacy Detection}
\label{sec:detection}

The directive |\childdocmain| in the main file can detect
whether the complete document or merely a child is to be compiled
even without using the directive |\childdocof|.
This method is deprecated because it is less robust
and there is no compelling reason to use it;
it is merely provided for backward compatibility
and it may be removed in future versions.

If the detection mechanism is to be used,
it is mandatory to correctly specify
the filename of the main file as the argument of |\childdocmain|:
%
\begin{center}
\begin{tabular}{l}
|\input{childdoc.def}|\\
|\childdocmain{|\textit{main}|}|\\
\end{tabular}
\end{center}
%
If |\jobname| does not match the argument \textit{main} of |\childdocmain|,
it is assumed that |\jobname| points to the child file to be compiled.
When using |\childdocmain| with the main file specified as argument,
it suffices to start a child file
with just |\input{|\textit{main}|}|
without loading of the package and using |\childdocof|.
If instead all processing is done
with the appropriate \textsf{childdoc} directives,
the argument of \textit{main} of |\childdocmain| can be empty.

An alternative version of the command line processing described
in \secref{sec:commandline} using the detection mechanism reads:
%
\begin{center}
|... -jobname "|\textit{target}|" "|[\textit{flags}]%
[|\def\jobname{|\textit{dest}|}|]|\input{|\textit{main}|}"|
\end{center}

%%%%%%%%%%%%%%%%%%%%%%%%%%%%%%%%%%%%%%%%%%%%%%%%%%%%%%%%%%%%%%%%%%%%%%%%%%%%%%%%
\subsection{Manual Code}
\label{sec:manual}

In case one cannot be certain whether the definitions file |childdoc.def|
is installed on the target \TeX{} distribution
and one prefers not to ship it,
it is conceivable to paste a few relevant commands into the sources.

To that end, drop all statements |\input{childdoc.def}|
and perform the replacements as outlined below.
Instead of |\childdocmain{|\textit{main}|}| add the following code
to the top of the main file:
%
\begin{center}
\begin{tabular}{l}
|\||ifdefined\childdocname\endinput\||fi\newif\ifchilddoc|\\
|\edef\childdocname{\scantokens\expandafter{\jobname\noexpand}}|\\
|\def\childdocmain{|\textit{main}|}\||ifx\childdocmain\childdocname\||else|\\
|\childdoctrue\includeonly{\childdocname}\let\jobname\childdocmain\||fi|\\
\end{tabular}
\end{center}
%
Instead of |\childdocof{|\textit{main}|}| just include the main file
at the top of each child file:
%
\begin{center}
|\input{|\textit{main}|}|
\end{center}
%
A simple redirection |\childdocforward{|\textit{dest}|}| is achieved by:
%
\begin{center}
|\def\jobname{|\textit{dest}|}\input{\jobname}|
\end{center}
%
The redirection with prefix
|\childdocforwardprefix[|\textit{prefix}|]{|\textit{dest}|}|
is accomplished by:
%
\begin{center}
\begin{tabular}{l}
|{\edef\jobname{\scantokens\expandafter{\jobname\noexpand}}|\\
|\def\redirectjob |\textit{prefix}|#1~~~{\gdef\jobname{|\textit{dest}|#1}}|\\
|\expandafter\redirectjob\jobname~~~}\input{\jobname}|
\end{tabular}
\end{center}

In an alternative approach,
child documents can be compiled by a specific command line
without additional code or specific definitions:
%
\begin{center}
|... -jobname "|\textit{target}|" "|[\textit{flags}]%
|\includeonly{|\textit{dest}|}\input{|\textit{main}|}"|
\end{center}
%

%%%%%%%%%%%%%%%%%%%%%%%%%%%%%%%%%%%%%%%%%%%%%%%%%%%%%%%%%%%%%%%%%%%%%%%%%%%%%%%%
%%%%%%%%%%%%%%%%%%%%%%%%%%%%%%%%%%%%%%%%%%%%%%%%%%%%%%%%%%%%%%%%%%%%%%%%%%%%%%%%
\section{Information}

%%%%%%%%%%%%%%%%%%%%%%%%%%%%%%%%%%%%%%%%%%%%%%%%%%%%%%%%%%%%%%%%%%%%%%%%%%%%%%%%
\subsection{Copyright}

Copyright \copyright{} 2017--2018 Niklas Beisert

This work may be distributed and/or modified under the
conditions of the \LaTeX{} Project Public License, either version 1.3
of this license or (at your option) any later version.
The latest version of this license is in
  \url{http://www.latex-project.org/lppl.txt}
and version 1.3 or later is part of all distributions of \LaTeX{}
version 2005/12/01 or later.

This work has the LPPL maintenance status `maintained'.

The Current Maintainer of this work is Niklas Beisert.

This work consists of the files |README.txt|, |childdoc.ins| and |childdoc.dtx|
as well as the derived files |childdoc.def|, |cdocsamp.tex|
with |cdocsch1.tex|, |cdocsch2.tex|, |cdocspt3.tex|, |cdocspt4.tex|,
|cdocsdrf.tex|, |cdocsfn1.tex|, |cdocsfn2.tex|
as well as |childdoc.pdf|.

%%%%%%%%%%%%%%%%%%%%%%%%%%%%%%%%%%%%%%%%%%%%%%%%%%%%%%%%%%%%%%%%%%%%%%%%%%%%%%%%
\subsection{Files and Installation}

The package consists of the files:
%
\begin{center}
\begin{tabular}{ll}
    |README.txt|   & readme file \\
    |childdoc.ins| & installation file \\
    |childdoc.dtx| & source file \\
    |childdoc.def| & definition file \\
    |cdocsamp.tex| & sample main file \\
    |cdocsch1.tex| & sample include file \\
    |cdocsch2.tex| & sample include file \\
    |cdocspt3.tex| & sample part file \\
    |cdocspt4.tex| & sample part file \\
    |cdocsdrf.tex| & sample redirection file \\
    |cdocsfn1.tex| & sample redirection file \\
    |cdocsfn2.tex| & sample redirection file \\
    |childdoc.pdf| & manual
\end{tabular}
\end{center}
%
The distribution consists of the files
|README.txt|, |childdoc.ins| and |childdoc.dtx|.
%
\begin{itemize}
\item
Run (pdf)\LaTeX{} on |childdoc.dtx|
to compile the manual |childdoc.pdf| (this file).
\item
Run \LaTeX{} on |childdoc.ins| to create the definitions file |childdoc.def|
and the sample |cdocsamp.tex| with include files
|cdocsch1.tex|, |cdocsch2.tex|, |cdocspt3.tex|, |cdocspt4.tex|,
|cdocsdrf.tex|, |cdocsfn1.tex|, |cdocsfn2.tex|.
Then copy the file |childdoc.def| to an appropriate directory of your \LaTeX{}
distribution, e.g.\ \textit{texmf-root}|/tex/latex/childdoc|.
\end{itemize}

%%%%%%%%%%%%%%%%%%%%%%%%%%%%%%%%%%%%%%%%%%%%%%%%%%%%%%%%%%%%%%%%%%%%%%%%%%%%%%%%
\subsection{Related CTAN Packages}

There are several other packages which offer a similar functionality:
%
\begin{itemize}
\item
The packages
\href{http://ctan.org/pkg/docmute}{\textsf{docmute}},
\href{http://ctan.org/pkg/includex}{\textsf{includex}} and
\href{http://ctan.org/pkg/standalone}{\textsf{standalone}}
provide commands to include only the document body of
a child file thus allowing both files to be compiled individually.
\item
The packages \href{http://ctan.org/pkg/subdocs}{\textsf{subdocs}}
and \href{http://ctan.org/pkg/subfiles}{\textsf{subfiles}}
provide structures in which the main and child documents can be
encapsulated and allowing them to be compiled individually.
The inclusion mechanism is different from the conventional |\include|.
\item
The package \href{http://ctan.org/pkg/combine}{\textsf{combine}}
is an elaborate solution to combine several documents into one.
\end{itemize}
%
See also the CTAN topic \href{http://ctan.org/topic/subdocs}{\textsf{subdocs}}
for further related packages.
The present package differs from the above solutions in that
a document structure constructed with the conventional |\include| mechanism
just needs two extra commands at the top of every file
such that all constituent files can be compiled individually.

%%%%%%%%%%%%%%%%%%%%%%%%%%%%%%%%%%%%%%%%%%%%%%%%%%%%%%%%%%%%%%%%%%%%%%%%%%%%%%%%
%\subsection{Feature Suggestions}
%
%The following is a list of features which may be useful for future
%versions of this package:
%%
%\begin{itemize}
%\item
%\ldots
%\end{itemize}

%%%%%%%%%%%%%%%%%%%%%%%%%%%%%%%%%%%%%%%%%%%%%%%%%%%%%%%%%%%%%%%%%%%%%%%%%%%%%%%%
\subsection{Revision History}

%%%%%%%%%%%%%%%%%%%%%%%%%%%%%%%%%%%%%%%%
\paragraph{v2.0:} 2018/12/30

\begin{itemize}
\item
immediate forward processing
\item
added |\childdocby| mechanism
\item
manual restructured
\end{itemize}

%%%%%%%%%%%%%%%%%%%%%%%%%%%%%%%%%%%%%%%%
\paragraph{v1.6:} 2018/01/17

\begin{itemize}
\item
application for development of include files
\item
corrections to manual
\end{itemize}

%%%%%%%%%%%%%%%%%%%%%%%%%%%%%%%%%%%%%%%%
\paragraph{v1.5:} 2017/05/21

\begin{itemize}
\item
more complete structuring introduced
\item
|\childdocof| introduced
\item
|\childdoc| renamed to |\childdocmain|
\item
|\childredirect| renamed to |\childdocforward| and |\childdocforwardprefix|
and functionality expanded
\end{itemize}

%%%%%%%%%%%%%%%%%%%%%%%%%%%%%%%%%%%%%%%%
\paragraph{v1.0:} 2017/04/27

\begin{itemize}
\item
manual and install package
\item
first version published on CTAN
\end{itemize}

%%%%%%%%%%%%%%%%%%%%%%%%%%%%%%%%%%%%%%%%
\paragraph{v0.6:} 2017/04/26

\begin{itemize}
\item
redirection mechanism added
\end{itemize}

%%%%%%%%%%%%%%%%%%%%%%%%%%%%%%%%%%%%%%%%
\paragraph{v0.5:} 2017/04/26

\begin{itemize}
\item
functionality in definition file
\end{itemize}


%%%%%%%%%%%%%%%%%%%%%%%%%%%%%%%%%%%%%%%%%%%%%%%%%%%%%%%%%%%%%%%%%%%%%%%%%%%%%%%%
%%%%%%%%%%%%%%%%%%%%%%%%%%%%%%%%%%%%%%%%%%%%%%%%%%%%%%%%%%%%%%%%%%%%%%%%%%%%%%%%
%%%%%%%%%%%%%%%%%%%%%%%%%%%%%%%%%%%%%%%%%%%%%%%%%%%%%%%%%%%%%%%%%%%%%%%%%%%%%%%%
\appendix

\settowidth\MacroIndent{\rmfamily\scriptsize 000\ }

 \DocInput{childdoc.dtx}

\end{document}
%</driver>
% \fi
%
% %%%%%%%%%%%%%%%%%%%%%%%%%%%%%%%%%%%%%%%%%%%%%%%%%%%%%%%%%%%%%%%%%%%%%%%%%%%%%%
% %%%%%%%%%%%%%%%%%%%%%%%%%%%%%%%%%%%%%%%%%%%%%%%%%%%%%%%%%%%%%%%%%%%%%%%%%%%%%%
% \section{Sample}
%\iffalse
%<*samplemain>
%\fi
%
% The following presents a sample document
% with two chapters, two parts, a title page,
% a compile flag as well as three forwarding files to set the flag.
% It consists of eight |.tex| files:
% \begin{center}
% \begin{tabular}{ll}
% |cdocsamp.tex|&main file\\
% |cdocsch1.tex|&include file for chapter 1\\
% |cdocsch2.tex|&include file for chapter 2\\
% |cdocspt3.tex|&include file for part 3\\
% |cdocspt4.tex|&include file for part 4\\
% |cdocsdrf.tex|&forwarding file for main file in draft mode\\
% |cdocsfi1.tex|&forwarding file for final version of chapter 1\\
% |cdocsfi2.tex|&forwarding file for final version of chapter 2\\
% \end{tabular}
% \end{center}
% Each of the eight files can be compiled directly by the \LaTeX{} compiler.
%
% %%%%%%%%%%%%%%%%%%%%%%%%%%%%%%%%%%%%%%
% \paragraph{Main File.}
%
% The main file is called |cdocsamp.tex|.
%
% Load the \textsf{childdoc} definitions and
% declare the filename for the main document:
%    \begin{macrocode}
\input{childdoc.def}
\childdocmain{}
%    \end{macrocode}

% Optional override for |\version| flag:
%    \begin{macrocode}
%%\ifchilddoc\else\providecommand{\version}{draft}\fi
%    \end{macrocode}

% Define the default values for the |\version| flag
% (|final| for the main file and |draft| for childs):
%    \begin{macrocode}
\ifchilddoc
\providecommand{\version}{draft}
\else
\providecommand{\version}{final}
\fi
%    \end{macrocode}

% Load the standard document class:
%    \begin{macrocode}
\documentclass[12pt]{article}
%    \end{macrocode}

% Start the document body:
%    \begin{macrocode}
\begin{document}
%    \end{macrocode}

% Declare a title page.
% Print title, part of document being processed and version flag:
%    \begin{macrocode}
\addtocounter{page}{-1}
\begin{center}
{\LARGE\bfseries{}childdoc example\par}
\vspace{1cm}
\ifchilddoc
\ifchilddocmanual part\else chapter\fi:
`\childdocname' of `\childdocjob'\par
\else
main document: `\childdocjob'\par
\fi
version: \version\par
\end{center}
\newpage
%    \end{macrocode}

% Manually include selected file,
% otherwise process as usual:
%    \begin{macrocode}
\ifchilddocmanual
\section*{part `\childdocname'}
\input{\childdocname}
\else
%    \end{macrocode}

% Include the two chapters:
%    \begin{macrocode}
\include{cdocsch1}
\include{cdocsch2}
%    \end{macrocode}

% Include the two parts unless only chapters should be displayed:
%    \begin{macrocode}
\ifchilddoc\else
\section{part three}
\input{cdocspt3}
\section{part four}
\input{cdocspt4}
\fi
%    \end{macrocode}

% Process as usual until here:
%    \begin{macrocode}
\fi
%    \end{macrocode}

% End of document body:
%    \begin{macrocode}
\end{document}
%    \end{macrocode}
%\iffalse
%</samplemain>
%\fi
%
% %%%%%%%%%%%%%%%%%%%%%%%%%%%%%%%%%%%%%%
% \paragraph{Chapter Include Files.}
%
% The include files are called |cdocsch1.tex| and |cdocsch2.tex|.
%
%\iffalse
%<*samplechap1|samplechap2>
%\fi

% Optional override for |\version| flag:
%    \begin{macrocode}
%%\providecommand{\version}{final}
%    \end{macrocode}

% Include the main document:
%    \begin{macrocode}
\input{childdoc.def}
\childdocof{cdocsamp}
%    \end{macrocode}

%\iffalse
%</samplechap1|samplechap2>
%\fi
%
%\iffalse
%<*samplechap1>
%\fi
% Some text for chapter 1:
%    \begin{macrocode}
\section{one}
some text in chapter one
%    \end{macrocode}

%\iffalse
%</samplechap1>
%\fi
% Some text for chapter 2:
%\iffalse
%<*samplechap2>
%\fi
%    \begin{macrocode}
\section{two}
more text in chapter two
%    \end{macrocode}

%\iffalse
%</samplechap2>
%\fi
%
% %%%%%%%%%%%%%%%%%%%%%%%%%%%%%%%%%%%%%%
% \paragraph{Part Include Files.}
%
% The include files are called |cdocspt3.tex| and |cdocspt4.tex|.
%
%\iffalse
%<*samplepart3|samplepart4>
%\fi

% Optional override for |\version| flag:
%    \begin{macrocode}
%%\providecommand{\version}{final}
%    \end{macrocode}

% Include the main document:
%    \begin{macrocode}
\input{childdoc.def}
\childdocby{cdocsamp}
%    \end{macrocode}

%\iffalse
%</samplepart3|samplepart4>
%\fi
%
%\iffalse
%<*samplepart3>
%\fi
% Some text for part 3:
%    \begin{macrocode}
some text in part three
%    \end{macrocode}

%\iffalse
%</samplepart3>
%\fi
% Some text for part 4:
%\iffalse
%<*samplepart4>
%\fi
%    \begin{macrocode}
more text in part four
%    \end{macrocode}

%\iffalse
%</samplepart4>
%\fi
%
% %%%%%%%%%%%%%%%%%%%%%%%%%%%%%%%%%%%%%%
% \paragraph{Forwarding for a Complete Draft.}
%
% The following forwarding file |cdocsdrf.tex|
% compiles the main document in draft mode:
%\iffalse
%<*sampledraft>
%\fi
%    \begin{macrocode}
\def\version{draft}
\input{childdoc.def}
\childdocforward{cdocsamp}
%    \end{macrocode}

%\iffalse
%</sampledraft>
%\fi
%
% %%%%%%%%%%%%%%%%%%%%%%%%%%%%%%%%%%%%%%
% \paragraph{Forwarding for Final Version of the Chapters.}
%
% The following forwarding files |cdocsfn1.tex| and |cdocsfn2.tex|
% (with identical content)
% compile the final versions of the child documents
% |cdocsch1.tex| and |cdocsch2.tex|, respectively:
%\iffalse
%<*samplefinal>
%\fi
%    \begin{macrocode}
\def\version{final}
\input{childdoc.def}
\childdocforwardprefix[cdocsamp]{cdocsfn}{cdocsch}
%    \end{macrocode}

%\iffalse
%</samplefinal>
%\fi
%
% %%%%%%%%%%%%%%%%%%%%%%%%%%%%%%%%%%%%%%
% \paragraph{Command Line Processing.}
%
% The following three command lines generate the output files
% |cdocscld|, |cdocscl1| and |cdocscl2|
% which should be identical to
% |cdocsdrf|, |cdocsch1| and |cdocsfn2|, respectively:
% \begin{center}
% \begin{tabular}{l}
% |latex -jobname cdocscld \|\\
% |  "\def\version{draft}\input{childdoc.def}\childdocforward{cdocsamp}"|\\
% |latex -jobname cdocscl1 \|\\
% |  "\input{childdoc.def}\childdocforward[cdocsamp]{cdocsch1}"|\\
% |latex -jobname cdocscl2 \|\\
% |  "\def\version{final}\input{childdoc.def}\childdocforward{cdocsch2}"|
% \end{tabular}
% \end{center}
% Note that the trailing backslash on each first line
% merely continues the input to the second line
% (for convenient cut ant paste).
% Furthermore, the command |latex| can be replaced by any
% of its alternative versions such as |pdflatex|.
%
% %%%%%%%%%%%%%%%%%%%%%%%%%%%%%%%%%%%%%%%%%%%%%%%%%%%%%%%%%%%%%%%%%%%%%%%%%%%%%%
% %%%%%%%%%%%%%%%%%%%%%%%%%%%%%%%%%%%%%%%%%%%%%%%%%%%%%%%%%%%%%%%%%%%%%%%%%%%%%%
% \section{Implementation}
%\iffalse
%<*package>
%\fi
%
% This section describes the definitions file |childdoc.def|.

% The definitions cannot be loaded using |\usepackage| or |\RequirePackage|
% which has a mechanism to prevent loading a style file more than once.
% When loading the definitions by means of |\input|
% multiple instances have to be prevented manually:
%\iffalse
%This code needs to be before the `\ProvidesFile' directive
%which is defined at the beginning of this file.
%Therefore it is also placed there and commented out here.
%</package>
%<*discard>
%\fi
%    \begin{macrocode}
\ifdefined\childdocmain\endinput\fi
%    \end{macrocode}
%\iffalse
%</discard>
%<*package>
%\fi
%
% \macro{\ifchilddoc}
% \macro{\ifchilddocmanual}
% The conditional |\ifchilddoc| tells whether a
% child (true) or main (false) document is being compiled.
% The conditional |\ifchilddocmanual| tells whether
% the |\includeonly| mechanism is used (false) or
% the selection of child files must be performed manually (true).
% The definitions initialise to false:
%    \begin{macrocode}
\newif\ifchilddoc
\newif\ifchilddocmanual
%    \end{macrocode}

% \macro{\childdocname}
% \macro{\childdocjob}
% The macro |\childdocname| stores the name of the main document
% to be compiled. The macro |\childdocjob| stores the name of
% the document on which the \LaTeX{} compiler was originally invoked.
% The content of |\jobname| cannot be compared
% to filenames specified in the source due to different catcodes.
% The following code rescans |\jobname|, stores the result
% in |\childdocname| and saves a copy in |\childdocjob|:
%    \begin{macrocode}
\edef\childdocname{\scantokens\expandafter{\jobname\noexpand}}
\let\childdocjob\childdocname
%    \end{macrocode}

% \macro{\childdocdisable}
% The macro |\childdocdisable| prevents the main file
% from being processed more than once.
% At this stage, the main document command |\childdocmain|
% is assumed to be called once again where it should do nothing.
% Any subsequent call to it should prevent
% a secondary processing of the main document
% It overwrites the forwarding commands
% |\childdocof| and |\childdocforward|
% with empty macros to prevent further inclusions of the main document:
%    \begin{macrocode}
\newcommand{\childdocdisable}
{
  \renewcommand{\childdocmain}[1]{\renewcommand{\childdocmain}[1]{\endinput}}
  \renewcommand{\childdocof}[1]{}
  \renewcommand{\childdocby}[2][]{}
  \renewcommand{\childdocforward}[2][]{}
  \renewcommand{\childdocdisable}{}
}
%    \end{macrocode}

% \macro{\childdocmain}
% The macro |\childdocmain| is to be called at the top of the main file
% with nothing or the main filename (without extension) as argument.
% First, it breaks loops.
% If the argument is not empty and does not match |\childdocname|
% (which is set by the first inclusion of |childdoc.def|),
% |\ifchilddoc| is set to true, |\includeonly| is applied to the child file
% and |\jobname| is set to the main file
% (for proper handling of |.aux| files):
%    \begin{macrocode}
\newcommand{\childdocmain}[1]
{
  \childdocdisable\childdocmain{}
  \if?#1?\else
    \begingroup
      \def\childdoctmp{#1}
      \ifx\childdoctmp\childdocname
        \def\childdoctmp{}
      \else
        \def\childdoctmp
        {
          \childdoctrue
          \includeonly{\childdocname}
          \def\childdocjob{#1}
          \def\jobname{#1}
        }
      \fi
      \expandafter
    \endgroup
    \childdoctmp
  \fi
}
%    \end{macrocode}

% \macro{\childdocof}
% The command |\childdocof| redirects
% compilation to the main file |#1|.
%    \begin{macrocode}
\newcommand{\childdocof}[1]
{
  \childdocdisable
  \childdoctrue
  \includeonly{\childdocname}
  \def\jobname{#1}
  \def\childdocjob{#1}
  \input{#1}
}
%    \end{macrocode}

% \macro{\childdocby}
% The command |\childdocby| ....
%    \begin{macrocode}
\newcommand{\childdocby}[2][]
{
  \childdocdisable
  \childdoctrue
  \childdocmanualtrue
  \if?#1?\else
    \def\jobname{#2}
  \fi
  \def\childdocjob{#2}
  \input{#2}
  \endinput
}
%    \end{macrocode}

% \macro{\childdocforward}
% The command |\childdocforward| redirects
% compilation to the main file or
% (if the optional argument is given) a child file.
% Parameters are set as if the main file
% or a child file starting with |\childdocof| was compiled.
% Then compilation is handed over to the main file:
%    \begin{macrocode}
\newcommand{\childdocforward}[2][]
{
  \begingroup
    \if?#1?
      \def\childdoctmp
      {
        \def\childdocname{#2}
        \def\childdocjob{#2}
        \def\jobname{#2}
        \input{#2}
        \endinput
      }
    \else
      \def\childdoctmp
      {
        \childdocdisable
        \def\childdocname{#2}
        \childdoctrue
        \includeonly{#2}
        \def\childdocjob{#1}
        \def\jobname{#1}
        \input{#1}
        \endinput
      }
    \fi
    \expandafter
  \endgroup
  \childdoctmp
}
%    \end{macrocode}

% \macro{\childdocforwardprefix}
% The command |\childdocforwardprefix| redirects
% compilation to the main or a child file by means of a pattern.
% The prefix |#1| in the current filename is replaced by |#2|
% and the suffix of the current filename is kept
% (it is assumed that the filename does not contain the substring `|~~~|'
% which is used as a delimiter).
% Compilation is handed over to the new file by |\childdocforward|:
%    \begin{macrocode}
\newcommand{\childdocforwardprefix}[3][]
{
  \begingroup
    \def\childdocextract #2##1~~~{\def\childdoctmp{\childdocforward[#1]{#3##1}}}
    \expandafter\childdocextract\childdocname~~~
    \expandafter
  \endgroup
  \childdoctmp
}
%    \end{macrocode}

% \macro{\childdoc}
% The deprecated macro |\childdoc| is a legacy version of |\childdocmain|:
%    \begin{macrocode}
\newcommand{\childdoc}{\childdocmain}
%    \end{macrocode}

% \macro{\childdocredirect}
% The deprecated macro |\childdocredirect| is a legacy version
% of |\childdocforward| and |\childdocforwardprefix|:
%    \begin{macrocode}
\newcommand{\childdocredirect}[2][]
{
  \begingroup
    \if?#1?
      \def\childdoctmp{\childdocforward{#2}}
    \else
      \def\childdoctmp{\childdocforwardprefix{#1}{#2}}
    \fi
    \expandafter
  \endgroup
  \childdoctmp
}
%    \end{macrocode}

%\iffalse
%</package>
%\fi
%
\endinput
\childdocforward[cdocsamp]{cdocsch1}"|\\
% |latex -jobname cdocscl2 \|\\
% |  "\def\version{final}% \iffalse
%
% childdoc.dtx Copyright (C) 2017-2018 Niklas Beisert
%
% This work may be distributed and/or modified under the
% conditions of the LaTeX Project Public License, either version 1.3
% of this license or (at your option) any later version.
% The latest version of this license is in
%   http://www.latex-project.org/lppl.txt
% and version 1.3 or later is part of all distributions of LaTeX
% version 2005/12/01 or later.
%
% This work has the LPPL maintenance status `maintained'.
%
% The Current Maintainer of this work is Niklas Beisert.
%
% This work consists of the files childdoc.dtx and childdoc.ins
% and the derived files childdoc.def and cdocsamp.tex with
% cdocsch1.tex, cdocsch2.tex, cdocsdrf.tex, cdocsfn1.tex, cdocsfn2.tex.
%
%<package>\ifdefined\childdocmain\endinput\fi
%<package>\ProvidesFile{childdoc.def}[2018/12/30 v2.0 child document driver]
%<samplemain>\ProvidesFile{cdocsamp.tex}[2018/12/30 v2.0 sample for childdoc]
%<*driver>
%\ProvidesFile{childdoc.drv}[2018/12/30 v2.0 childdoc reference manual file]
\PassOptionsToClass{10pt,a4paper}{article}
\documentclass{ltxdoc}

\usepackage[margin=35mm]{geometry}
\usepackage{hyperref}
\usepackage{hyperxmp}
\usepackage[usenames]{color}

\hypersetup{colorlinks=true}
\hypersetup{pdfstartview=FitH}
\hypersetup{pdfpagemode=UseNone}
\hypersetup{pdfsource={}}
\hypersetup{pdflang={en-UK}}
\hypersetup{pdfcopyright={Copyright 2017-2018 Niklas Beisert.
  This work may be distributed and/or modified under the
  conditions of the LaTeX Project Public License, either version 1.3
  of this license or (at your option) any later version.}}
\hypersetup{pdflicenseurl={http://www.latex-project.org/lppl.txt}}
\hypersetup{pdfcontactaddress={ETH Zurich, ITP, HIT K,
  Wolfgang-Pauli-Strasse 27}}
\hypersetup{pdfcontactpostcode={8093}}
\hypersetup{pdfcontactcity={Zurich}}
\hypersetup{pdfcontactcountry={Switzerland}}
\hypersetup{pdfcontactemail={nbeisert@itp.phys.ethz.ch}}
\hypersetup{pdfcontacturl={http://people.phys.ethz.ch/\xmptilde nbeisert/}}

\newcommand{\secref}[1]{\hyperref[#1]{section \ref*{#1}}}

\parskip1ex
\parindent0pt
\let\olditemize\itemize
\def\itemize{\olditemize\parskip0pt}

\begin{document}

\title{The \textsf{childdoc} Package}
\hypersetup{pdftitle={The childdoc Package}}
\author{Niklas Beisert\\[2ex]
  Institut f\"ur Theoretische Physik\\
  Eidgen\"ossische Technische Hochschule Z\"urich\\
  Wolfgang-Pauli-Strasse 27, 8093 Z\"urich, Switzerland\\[1ex]
  \href{mailto:nbeisert@itp.phys.ethz.ch}
  {\texttt{nbeisert@itp.phys.ethz.ch}}}
\hypersetup{pdfauthor={Niklas Beisert}}
\hypersetup{pdfsubject={Manual for the LaTeX2e Package childdoc}}
\date{30 December 2018, \textsf{v2.0}}
\maketitle

\begin{abstract}\noindent
\textsf{childdoc} is a \LaTeXe{} package
that enables the direct compilation
of document sections included by |\include|
to individual files.
\end{abstract}

\begingroup
\parskip0ex
\tableofcontents
\endgroup

%%%%%%%%%%%%%%%%%%%%%%%%%%%%%%%%%%%%%%%%%%%%%%%%%%%%%%%%%%%%%%%%%%%%%%%%%%%%%%%%
%%%%%%%%%%%%%%%%%%%%%%%%%%%%%%%%%%%%%%%%%%%%%%%%%%%%%%%%%%%%%%%%%%%%%%%%%%%%%%%%
\section{Introduction}

\LaTeX{} provides a mechanism to structure a large document (such as a book)
into a main file and several child files (containing the chapters)
using the |\include| command.
This mechanism is beneficial for documents
which span hundreds of pages in order to
make the source file(s) more manageable.
Moreover, compilation can be restricted to
selected child files by means of the |\includeonly| command.
The latter feature can be used to reduce the compilation time while editing
(this was significantly more useful in the earlier days of \LaTeX{})
or to generate a smaller document which is easier to navigate.
Another application of |\includeonly| is to generate
documents consisting of selected parts of the complete document.

However, there are a few drawbacks of the plain |\include| mechanism:
\begin{itemize}
\item
The child files cannot be compiled on their own,
they can only be compiled via the main file.
A naive editing environment
(such as a text editor with an option
to have the current file processed by \LaTeX)
may require one to switch to the main file before compiling;
attempting to compile the child file produces errors.
\item
The main file must be modified (each time)
to adjust the |\includeonly| command
to the present needs. This easily leaves the main file in a messy state.
\item
The generated document will always carry the filename
of the main document. This is inconvenient if
several child files are to be compiled and
to be kept for distribution.
\end{itemize}

The present package provides a simple interface
to make child files individually compilable by \LaTeX{}.
Compiling a child file then has the same effect as compiling
the main file with an |\includeonly| command
to select the appropriate child.
Moreover the generated document will carry the name of the child
rather than the main file.
This resolves all three above issues.

This feature is meant to make the editing of books,
thesis documents and lecture notes somewhat more convenient.
However, the package can also be used efficiently for
composing a series of documents (such as exercise sheets)
which are typically distributed individually.
It then assists the author in generating the individual documents
(potentially in different versions)
as well as a document containing the collected series.
Another application is in developing style files
or other kinds of included material
where compilation of the style file could redirect
to a sample or test file.

%%%%%%%%%%%%%%%%%%%%%%%%%%%%%%%%%%%%%%%%%%%%%%%%%%%%%%%%%%%%%%%%%%%%%%%%%%%%%%%%
%%%%%%%%%%%%%%%%%%%%%%%%%%%%%%%%%%%%%%%%%%%%%%%%%%%%%%%%%%%%%%%%%%%%%%%%%%%%%%%%
\section{Usage}

First of all, the package \textsf{childdoc} is \emph{not} a standard
\LaTeXe{} |.sty| style file! Therefore it needs to be invoked in
a non-standard way.

%%%%%%%%%%%%%%%%%%%%%%%%%%%%%%%%%%%%%%%%%%%%%%%%%%%%%%%%%%%%%%%%%%%%%%%%%%%%%%%%
\subsection{Included Files}
\label{sec:include}

%%%%%%%%%%%%%%%%%%%%%%%%%%%%%%%%%%%%%%%%
\DescribeMacro{\childdocmain}
To use the package, add the commands
\begin{center}
\begin{tabular}{l}
|\input{childdoc.def}|\\
|\childdocmain{}|\\
\end{tabular}
\end{center}
at the very top of the main \LaTeX{} file,
in particular \emph{before} the |\documentclass| statement!
The argument of |\childdocmain| should be left empty
(but it must be present).

%%%%%%%%%%%%%%%%%%%%%%%%%%%%%%%%%%%%%%%%
\DescribeMacro{\childdocof}
Furthermore, add the commands
\begin{center}
\begin{tabular}{l}
|\input{childdoc.def}|\\
|\childdocof{|\textit{main}|}|\\
\end{tabular}
\end{center}
at the top of every child file \textit{child}
which is included by |\include{|\textit{child}|}|
from within the main file
(or at least for those files to be compiled individually).
The argument \textit{main} must be the filename of the main file.

There are a couple of
considerations in setting up the main and child documents:

%%%%%%%%%%%%%%%%%%%%%%%%%%%%%%%%%%%%%%%%
\paragraph{Restrictions.}

Please note the following restrictions:
\begin{itemize}
\item
|\childdocmain| must be called with one argument \textit{main}
to ensure compatibility with earlier version of the package.
It must either be empty (|\childdocmain{}|)
or precisely match the filename of the main file in which it is specified.
See \secref{sec:detection} for further information.
\item
The filename \textit{main} must be specified without the |.tex| extension.
\item
The filename \textit{main} is case sensitive
(even in case-insensitive file systems)
due to internal string comparison.
\item
The argument \textit{main} should be fully expanded, it cannot be a macro.
\item
Subdirectories and special characters should be avoided in filenames.
\item
The command |\childdocmain{|\textit{main}|}| must be followed by a whitespace.
It should not be followed immediately by another command
or by a comment mark `|%|'.
This is because the \TeX{} parser reads the token immediately following
the argument of |\childdocmain| and puts it
at the beginning of every child section;
however, a white\-space is ignored.
\end{itemize}

%%%%%%%%%%%%%%%%%%%%%%%%%%%%%%%%%%%%%%%%
\paragraph{Content of Main File.}

It is advisable to place all content in the child files included by |\include|.
Any output contained in the main file will appear in all child documents
unless suppressed manually;
it cannot be suppressed automatically by the |\includeonly| directive
and thus should normally be avoided.
A method to include some content in the main file
by means of conditional processing is described in \secref{sec:conditional}.

%%%%%%%%%%%%%%%%%%%%%%%%%%%%%%%%%%%%%%%%
\paragraph{Page Numbering.}

When only a part of the document is compiled,
the appropriate numbering of pages
(as well as other status parameters)
is determined from the |.aux| files.
The latter contain information from previous passes.
However this information needs to propagate through
all intermediate child documents.
Therefore the page numbering in child documents may well
be inconsistent until the complete document is compiled at least once.

A useful (if unconventional) way to always ensure a consistent
page numbering is to restart the numbering in each child document
and denote the pages by `\textit{child}|.|\textit{page}'
where \textit{child} represents the chapter/section number of the child file.
This can be achieved by the command
|\numberwithin{page}{|\textit{child}|}|
of the \textsf{amsmath} package
where \textit{child} can be |chapter| or |section|
depending on the chosen structuring.
Alternatively, one can modify the macro |\thepage| appropriately
and reset the counter |page| at the start of each child file.

%%%%%%%%%%%%%%%%%%%%%%%%%%%%%%%%%%%%%%%%%%%%%%%%%%%%%%%%%%%%%%%%%%%%%%%%%%%%%%%%
\subsection{Conditional Processing}
\label{sec:conditional}

The package provides a mechanism to compile different versions
of a document. To customise the versions further some conditional processing
can come in handy to distinguish which version is being compiled.
The package provides two macros to describe the compilation context:

%%%%%%%%%%%%%%%%%%%%%%%%%%%%%%%%%%%%%%%%
\DescribeMacro{\ifchilddoc}
The conditional |\ifchilddoc| distinguishes between the compilation of
child documents and the main document:
%
\begin{center}
|\ifchilddoc |\textit{child-code}| |[|\||else |\textit{main-code}]| \||fi|
\end{center}

%%%%%%%%%%%%%%%%%%%%%%%%%%%%%%%%%%%%%%%%
\DescribeMacro{\childdocname}
\DescribeMacro{\childdocjob}
The macro |\childdocname| contains the filename (without extension)
of the main or child file being processed.
Note that |\childdocjob| will always contain the name of the main file.

%%%%%%%%%%%%%%%%%%%%%%%%%%%%%%%%%%%%%%%%
\paragraph{Title Page.}

Conditional processing can be used to include a title or banner page
in the main document when proper precautions are taken.
Importantly, the code in the main file should ensure that the page counter
(as well as other status parameters which are stored in the |.aux| files)
takes the same value after the conditional processing.
Otherwise the page numbers may take divergent values
depending on which part is compiled.

For example, a title page could be declared by:
%
\begin{center}
\begin{tabular}{l}
|\ifchilddoc\||else|\\
|\addtocounter{page}{-1}|\\
\textit{code for title page}\\
|\newpage|\\
|\||fi|
\end{tabular}
\end{center}
%
A banner page for the child documents can be generated by:
%
\begin{center}
\begin{tabular}{l}
|\ifchilddoc|\\
|\addtocounter{page}{-1}|\\
\textit{code for banner page}\\
|\newpage|\\
|\||fi|
\end{tabular}
\end{center}
%
Here one could write a message such as:
\begin{center}
|This is the part \childdocname{} of \childdocjob{}.|
\end{center}

%%%%%%%%%%%%%%%%%%%%%%%%%%%%%%%%%%%%%%%%%%%%%%%%%%%%%%%%%%%%%%%%%%%%%%%%%%%%%%%%
\subsection{Flags}
\label{sec:flags}

The package makes it easy to generate different versions
of the main or child documents.
To this end compilation flags can be defined
and assigned different default values.
They will be particularly useful in conjunction
with the forwarding mechanism described in \secref{sec:forward}.

For example, it may be useful to have a flag |\version|
which can be set to |draft| or |final|.
The document source will contain some conditional code
depending on the value of |\version|.
Suppose further, the flag should default to |final| for the main file
and to |draft| for child files
which is a natural assignment for editing the document.
This is achieved by placing the following code
in the preamble of the main document
(below the |\childdocmain| directive):
%
\begin{center}
\begin{tabular}{l}
|\ifchilddoc|\\
|\providecommand{\version}{draft}|\\
|\||else|\\
|\providecommand{\version}{final}|\\
|\||fi|
\end{tabular}
\end{center}
%
The definition by |\providecommand| makes sure
that previous definitions are not overwritten.
Further statements |\providecommand{\version}{...}|
can thus be added before the above code to override it.

For the main file, one might add a line
(between |\childdocmain| and the above block)
%
\begin{center}
|%\ifchilddoc\||else\providecommand{\version}{draft}\||fi|
\end{center}
%
which can be uncommented to produce a draft version.
Likewise one can add a line to the very top of a child file
(above the |\childdocof{|\textit{main}|}| directive)
%
\begin{center}
|%\providecommand{\version}{final}|
\end{center}
%
which can be uncommented to produce the final version of this child document.

%%%%%%%%%%%%%%%%%%%%%%%%%%%%%%%%%%%%%%%%%%%%%%%%%%%%%%%%%%%%%%%%%%%%%%%%%%%%%%%%
\subsection{Forwarding}
\label{sec:forward}

Different versions of the main or child documents
using compilation flags as described in \secref{sec:flags}
can be (permanently) stored in different files
for convenient compilation, viewing and distribution.
To this end, the package defines a command
to pass on compilation to a different file:

%%%%%%%%%%%%%%%%%%%%%%%%%%%%%%%%%%%%%%%%
\DescribeMacro{\childdocforward}
The command |\childdocforward| redirects processing to
another source file:
%
\begin{center}
\begin{tabular}{l}
|\input{childdoc.def}|\\
|\childdocforward[|\textit{main}|]{|\textit{dest}|}|\\
\end{tabular}
\end{center}
%
The argument \textit{dest} is the destination file
(without extension).
It should be the main file or one of the child files.
Note that further \textsf{childdoc} directives
such as |\childdocof| and |\childdocforward|
in the indicated file will be processed in this form.
The optional argument \textit{main}
passes on directly to the main file \textit{main}
while pretending to compile the child \textit{dest}.
This form behaves as if \textit{dest}
issues |\childdocof{|\textit{main}|}| right away,
and no further \textsf{childdoc} directives will be processed.

%%%%%%%%%%%%%%%%%%%%%%%%%%%%%%%%%%%%%%%%
\DescribeMacro{\...prefix}
In the alternative form |\childdocforwardprefix|,
%
\begin{center}
\begin{tabular}{l}
|\input{childdoc.def}|\\
|\childdocforwardprefix[|\textit{main}|]{|\textit{prefix}|}{|\textit{dest}|}|
\end{tabular}
\end{center}
%
the destination file is determined by a pattern
depending on the current file:
To make this work, the current file must be called
`{\textit{prefix}\hspace{0.2em}\textit{suffix}}'
with \textit{prefix} matching precisely the argument.
Processing is then passed on to the file
`{\textit{dest}\hspace{0.2em}\textit{suffix}}'.
Surely, the same effect is achieved by
directly specifying the
argument `{\textit{dest}\hspace{0.2em}\textit{suffix}}'
in the first form.
However, that requires to set up a different file
for each child. With the alternative form of the command
all these files can have exactly the same content
which simplifies setting them up and maintaining them.

For example, the following file |draft.tex|
with a compilation flag |\version| as described in \secref{sec:flags}
compiles the main document as a draft:
%
\begin{center}
\begin{tabular}{l}
|\def\version{draft}|\\
|\input{childdoc.def}|\\
|\childdocforward{|\textit{main}|}|
\end{tabular}
\end{center}
%
Likewise, the following files |final|\textit{nn}|.tex|
compile the final version of the child document
|child|\textit{nn}|.tex|:
%
\begin{center}
\begin{tabular}{l}
|\def\version{final}|\\
|\input{childdoc.def}|\\
|\childdocforwardprefix{final}{child}|
\end{tabular}
\end{center}
%

Note that when several versions of a main file and/or of each child file
are to be generated, it may be convenient to set up a |Makefile| or
shell script to automatise the process.

%%%%%%%%%%%%%%%%%%%%%%%%%%%%%%%%%%%%%%%%%%%%%%%%%%%%%%%%%%%%%%%%%%%%%%%%%%%%%%%%
\subsection{Command Line Processing}
\label{sec:commandline}

The effect of redirection files can also be achieved by invoking
the \LaTeX{} compiler with a more elaborate command line.
Most conveniently this should be done as part
of a shell script or a |Makefile|.

When using \textsf{childdoc} in the main file, the following
command lines effectively perform a redirection
(note that depending on the shell being used,
backslashes may have to be doubled: `|\|' $\to$ `|\\|'):
%
\begin{center}
|... -jobname "|\textit{target}|" |\\|"|[\textit{flags}]%
|\input{childdoc.def}\childdocforward[|\textit{main}|]{|\textit{dest}|}"|
\end{center}
%
Here \textit{target} is the name of the output file,
\textit{main} is the name of the main file
and \textit{dest} is the name of the main or child file to be processed
(all filenames without extensions).
The optional argument \textit{main} can be omitted
if \textit{main} matches \textit{dest}.
Optionally, compilation \textit{flags} can be defined via |\def| commands.
This command line makes the \TeX{} engine believe
it is compiling the file \textit{target}
whose content is specified as the latter parameter.
The provided code then forwards the processing to
\textit{main} or \textit{dest} as described in \secref{sec:forward}.

%%%%%%%%%%%%%%%%%%%%%%%%%%%%%%%%%%%%%%%%%%%%%%%%%%%%%%%%%%%%%%%%%%%%%%%%%%%%%%%%
\subsection{Include by Input}
\label{sec:input}

Including child documents by |\include| has some restrictions by design.
Most notably, the content of a child document always occupies
its own set of pages; pages cannot be shared between child documents.
Usually, this behaviour makes perfect sense
because each child document contain an essential part of the document.
However, in some situations it may be desirable to compose
a document from a collection of parts
without having mandatory page breaks between then.
For this case, the package
provides a mechanism to include parts
by |\input| which can also be processed individually.
However, by construction this mechanism
requires manual handling of the content to be output.

%%%%%%%%%%%%%%%%%%%%%%%%%%%%%%%%%%%%%%%%
\DescribeMacro{\ifchilddocmanual}
The main file should be prepared as usual, see \secref{sec:include}.
However, the document body must make a distinction
between processing of an individual part and of the main document, e.g.:
%
\begin{center}
\begin{tabular}{l}
|\ifchilddocmanual|\\
|\input{\childdocname}|\\
|\||else|\\
\textit{document body with }|\input{|\textit{part}|}|\\
|\||fi|
\end{tabular}
\end{center}
%
The conditional |\ifchilddocmanual| is true whenever
a part to be included by |\input| is being compiled,
and the name of the part is stored in |\childdocname|.

%%%%%%%%%%%%%%%%%%%%%%%%%%%%%%%%%%%%%%%%
\DescribeMacro{\childdocby}
Each part to be included by |\input| should start with:
%
\begin{center}
\begin{tabular}{l}
|\input{childdoc.def}|\\
|\childdocby{|\textit{main}|}|\\
\end{tabular}
\end{center}
%
The directive |\childdocby| is similar to |\childdocof|
described in \secref{sec:include},
but the subsequent selection of content must be done manually.
To that end, both |\ifchilddoc| and |\ifchilddocmanual|
will be true upon processing of a part,
and the name of the part is stored in |\childdocname|.
Note that |\jobname| will be set to the filename of the current part
so that each part receives an individual |.aux| file
that does not interfere with the |.aux| file(s) of the main document.
This behaviour can be altered by the alternative form
|\childdocby[*]{|\textit{main}|}| (with a non-empty optional argument)
which uses the |.aux| file of the main document
by setting |\jobname| to \textit{main}.

%%%%%%%%%%%%%%%%%%%%%%%%%%%%%%%%%%%%%%%%%%%%%%%%%%%%%%%%%%%%%%%%%%%%%%%%%%%%%%%%
\subsection{Driver Development}
\label{sec:driver}

The \textsf{childdoc} mechanism can also be use for the development
of definition files such as \LaTeX{} styles or classes.
This case differs from the above setup with multiple parts
included by |\include| in that no |\includeonly| should be invoked.
This can be achieved by starting the include file
(before |\ProvidesPackage|) with:
%
\begin{center}
\begin{tabular}{l}
|\input{childdoc.def}|\\
|\childdocforward{|\textit{main}|}|\\
\end{tabular}
\end{center}
%
or alternatively with:
%
\begin{center}
\begin{tabular}{l}
|\input{childdoc.def}|\\
|\childdocby{|\textit{main}|}|\\
\end{tabular}
\end{center}
%
Both forms have slightly different effects as described above.
The main file is prepared as usual, see \secref{sec:include}.

%%%%%%%%%%%%%%%%%%%%%%%%%%%%%%%%%%%%%%%%%%%%%%%%%%%%%%%%%%%%%%%%%%%%%%%%%%%%%%%%
\subsection{Legacy Detection}
\label{sec:detection}

The directive |\childdocmain| in the main file can detect
whether the complete document or merely a child is to be compiled
even without using the directive |\childdocof|.
This method is deprecated because it is less robust
and there is no compelling reason to use it;
it is merely provided for backward compatibility
and it may be removed in future versions.

If the detection mechanism is to be used,
it is mandatory to correctly specify
the filename of the main file as the argument of |\childdocmain|:
%
\begin{center}
\begin{tabular}{l}
|\input{childdoc.def}|\\
|\childdocmain{|\textit{main}|}|\\
\end{tabular}
\end{center}
%
If |\jobname| does not match the argument \textit{main} of |\childdocmain|,
it is assumed that |\jobname| points to the child file to be compiled.
When using |\childdocmain| with the main file specified as argument,
it suffices to start a child file
with just |\input{|\textit{main}|}|
without loading of the package and using |\childdocof|.
If instead all processing is done
with the appropriate \textsf{childdoc} directives,
the argument of \textit{main} of |\childdocmain| can be empty.

An alternative version of the command line processing described
in \secref{sec:commandline} using the detection mechanism reads:
%
\begin{center}
|... -jobname "|\textit{target}|" "|[\textit{flags}]%
[|\def\jobname{|\textit{dest}|}|]|\input{|\textit{main}|}"|
\end{center}

%%%%%%%%%%%%%%%%%%%%%%%%%%%%%%%%%%%%%%%%%%%%%%%%%%%%%%%%%%%%%%%%%%%%%%%%%%%%%%%%
\subsection{Manual Code}
\label{sec:manual}

In case one cannot be certain whether the definitions file |childdoc.def|
is installed on the target \TeX{} distribution
and one prefers not to ship it,
it is conceivable to paste a few relevant commands into the sources.

To that end, drop all statements |\input{childdoc.def}|
and perform the replacements as outlined below.
Instead of |\childdocmain{|\textit{main}|}| add the following code
to the top of the main file:
%
\begin{center}
\begin{tabular}{l}
|\||ifdefined\childdocname\endinput\||fi\newif\ifchilddoc|\\
|\edef\childdocname{\scantokens\expandafter{\jobname\noexpand}}|\\
|\def\childdocmain{|\textit{main}|}\||ifx\childdocmain\childdocname\||else|\\
|\childdoctrue\includeonly{\childdocname}\let\jobname\childdocmain\||fi|\\
\end{tabular}
\end{center}
%
Instead of |\childdocof{|\textit{main}|}| just include the main file
at the top of each child file:
%
\begin{center}
|\input{|\textit{main}|}|
\end{center}
%
A simple redirection |\childdocforward{|\textit{dest}|}| is achieved by:
%
\begin{center}
|\def\jobname{|\textit{dest}|}\input{\jobname}|
\end{center}
%
The redirection with prefix
|\childdocforwardprefix[|\textit{prefix}|]{|\textit{dest}|}|
is accomplished by:
%
\begin{center}
\begin{tabular}{l}
|{\edef\jobname{\scantokens\expandafter{\jobname\noexpand}}|\\
|\def\redirectjob |\textit{prefix}|#1~~~{\gdef\jobname{|\textit{dest}|#1}}|\\
|\expandafter\redirectjob\jobname~~~}\input{\jobname}|
\end{tabular}
\end{center}

In an alternative approach,
child documents can be compiled by a specific command line
without additional code or specific definitions:
%
\begin{center}
|... -jobname "|\textit{target}|" "|[\textit{flags}]%
|\includeonly{|\textit{dest}|}\input{|\textit{main}|}"|
\end{center}
%

%%%%%%%%%%%%%%%%%%%%%%%%%%%%%%%%%%%%%%%%%%%%%%%%%%%%%%%%%%%%%%%%%%%%%%%%%%%%%%%%
%%%%%%%%%%%%%%%%%%%%%%%%%%%%%%%%%%%%%%%%%%%%%%%%%%%%%%%%%%%%%%%%%%%%%%%%%%%%%%%%
\section{Information}

%%%%%%%%%%%%%%%%%%%%%%%%%%%%%%%%%%%%%%%%%%%%%%%%%%%%%%%%%%%%%%%%%%%%%%%%%%%%%%%%
\subsection{Copyright}

Copyright \copyright{} 2017--2018 Niklas Beisert

This work may be distributed and/or modified under the
conditions of the \LaTeX{} Project Public License, either version 1.3
of this license or (at your option) any later version.
The latest version of this license is in
  \url{http://www.latex-project.org/lppl.txt}
and version 1.3 or later is part of all distributions of \LaTeX{}
version 2005/12/01 or later.

This work has the LPPL maintenance status `maintained'.

The Current Maintainer of this work is Niklas Beisert.

This work consists of the files |README.txt|, |childdoc.ins| and |childdoc.dtx|
as well as the derived files |childdoc.def|, |cdocsamp.tex|
with |cdocsch1.tex|, |cdocsch2.tex|, |cdocspt3.tex|, |cdocspt4.tex|,
|cdocsdrf.tex|, |cdocsfn1.tex|, |cdocsfn2.tex|
as well as |childdoc.pdf|.

%%%%%%%%%%%%%%%%%%%%%%%%%%%%%%%%%%%%%%%%%%%%%%%%%%%%%%%%%%%%%%%%%%%%%%%%%%%%%%%%
\subsection{Files and Installation}

The package consists of the files:
%
\begin{center}
\begin{tabular}{ll}
    |README.txt|   & readme file \\
    |childdoc.ins| & installation file \\
    |childdoc.dtx| & source file \\
    |childdoc.def| & definition file \\
    |cdocsamp.tex| & sample main file \\
    |cdocsch1.tex| & sample include file \\
    |cdocsch2.tex| & sample include file \\
    |cdocspt3.tex| & sample part file \\
    |cdocspt4.tex| & sample part file \\
    |cdocsdrf.tex| & sample redirection file \\
    |cdocsfn1.tex| & sample redirection file \\
    |cdocsfn2.tex| & sample redirection file \\
    |childdoc.pdf| & manual
\end{tabular}
\end{center}
%
The distribution consists of the files
|README.txt|, |childdoc.ins| and |childdoc.dtx|.
%
\begin{itemize}
\item
Run (pdf)\LaTeX{} on |childdoc.dtx|
to compile the manual |childdoc.pdf| (this file).
\item
Run \LaTeX{} on |childdoc.ins| to create the definitions file |childdoc.def|
and the sample |cdocsamp.tex| with include files
|cdocsch1.tex|, |cdocsch2.tex|, |cdocspt3.tex|, |cdocspt4.tex|,
|cdocsdrf.tex|, |cdocsfn1.tex|, |cdocsfn2.tex|.
Then copy the file |childdoc.def| to an appropriate directory of your \LaTeX{}
distribution, e.g.\ \textit{texmf-root}|/tex/latex/childdoc|.
\end{itemize}

%%%%%%%%%%%%%%%%%%%%%%%%%%%%%%%%%%%%%%%%%%%%%%%%%%%%%%%%%%%%%%%%%%%%%%%%%%%%%%%%
\subsection{Related CTAN Packages}

There are several other packages which offer a similar functionality:
%
\begin{itemize}
\item
The packages
\href{http://ctan.org/pkg/docmute}{\textsf{docmute}},
\href{http://ctan.org/pkg/includex}{\textsf{includex}} and
\href{http://ctan.org/pkg/standalone}{\textsf{standalone}}
provide commands to include only the document body of
a child file thus allowing both files to be compiled individually.
\item
The packages \href{http://ctan.org/pkg/subdocs}{\textsf{subdocs}}
and \href{http://ctan.org/pkg/subfiles}{\textsf{subfiles}}
provide structures in which the main and child documents can be
encapsulated and allowing them to be compiled individually.
The inclusion mechanism is different from the conventional |\include|.
\item
The package \href{http://ctan.org/pkg/combine}{\textsf{combine}}
is an elaborate solution to combine several documents into one.
\end{itemize}
%
See also the CTAN topic \href{http://ctan.org/topic/subdocs}{\textsf{subdocs}}
for further related packages.
The present package differs from the above solutions in that
a document structure constructed with the conventional |\include| mechanism
just needs two extra commands at the top of every file
such that all constituent files can be compiled individually.

%%%%%%%%%%%%%%%%%%%%%%%%%%%%%%%%%%%%%%%%%%%%%%%%%%%%%%%%%%%%%%%%%%%%%%%%%%%%%%%%
%\subsection{Feature Suggestions}
%
%The following is a list of features which may be useful for future
%versions of this package:
%%
%\begin{itemize}
%\item
%\ldots
%\end{itemize}

%%%%%%%%%%%%%%%%%%%%%%%%%%%%%%%%%%%%%%%%%%%%%%%%%%%%%%%%%%%%%%%%%%%%%%%%%%%%%%%%
\subsection{Revision History}

%%%%%%%%%%%%%%%%%%%%%%%%%%%%%%%%%%%%%%%%
\paragraph{v2.0:} 2018/12/30

\begin{itemize}
\item
immediate forward processing
\item
added |\childdocby| mechanism
\item
manual restructured
\end{itemize}

%%%%%%%%%%%%%%%%%%%%%%%%%%%%%%%%%%%%%%%%
\paragraph{v1.6:} 2018/01/17

\begin{itemize}
\item
application for development of include files
\item
corrections to manual
\end{itemize}

%%%%%%%%%%%%%%%%%%%%%%%%%%%%%%%%%%%%%%%%
\paragraph{v1.5:} 2017/05/21

\begin{itemize}
\item
more complete structuring introduced
\item
|\childdocof| introduced
\item
|\childdoc| renamed to |\childdocmain|
\item
|\childredirect| renamed to |\childdocforward| and |\childdocforwardprefix|
and functionality expanded
\end{itemize}

%%%%%%%%%%%%%%%%%%%%%%%%%%%%%%%%%%%%%%%%
\paragraph{v1.0:} 2017/04/27

\begin{itemize}
\item
manual and install package
\item
first version published on CTAN
\end{itemize}

%%%%%%%%%%%%%%%%%%%%%%%%%%%%%%%%%%%%%%%%
\paragraph{v0.6:} 2017/04/26

\begin{itemize}
\item
redirection mechanism added
\end{itemize}

%%%%%%%%%%%%%%%%%%%%%%%%%%%%%%%%%%%%%%%%
\paragraph{v0.5:} 2017/04/26

\begin{itemize}
\item
functionality in definition file
\end{itemize}


%%%%%%%%%%%%%%%%%%%%%%%%%%%%%%%%%%%%%%%%%%%%%%%%%%%%%%%%%%%%%%%%%%%%%%%%%%%%%%%%
%%%%%%%%%%%%%%%%%%%%%%%%%%%%%%%%%%%%%%%%%%%%%%%%%%%%%%%%%%%%%%%%%%%%%%%%%%%%%%%%
%%%%%%%%%%%%%%%%%%%%%%%%%%%%%%%%%%%%%%%%%%%%%%%%%%%%%%%%%%%%%%%%%%%%%%%%%%%%%%%%
\appendix

\settowidth\MacroIndent{\rmfamily\scriptsize 000\ }

 \DocInput{childdoc.dtx}

\end{document}
%</driver>
% \fi
%
% %%%%%%%%%%%%%%%%%%%%%%%%%%%%%%%%%%%%%%%%%%%%%%%%%%%%%%%%%%%%%%%%%%%%%%%%%%%%%%
% %%%%%%%%%%%%%%%%%%%%%%%%%%%%%%%%%%%%%%%%%%%%%%%%%%%%%%%%%%%%%%%%%%%%%%%%%%%%%%
% \section{Sample}
%\iffalse
%<*samplemain>
%\fi
%
% The following presents a sample document
% with two chapters, two parts, a title page,
% a compile flag as well as three forwarding files to set the flag.
% It consists of eight |.tex| files:
% \begin{center}
% \begin{tabular}{ll}
% |cdocsamp.tex|&main file\\
% |cdocsch1.tex|&include file for chapter 1\\
% |cdocsch2.tex|&include file for chapter 2\\
% |cdocspt3.tex|&include file for part 3\\
% |cdocspt4.tex|&include file for part 4\\
% |cdocsdrf.tex|&forwarding file for main file in draft mode\\
% |cdocsfi1.tex|&forwarding file for final version of chapter 1\\
% |cdocsfi2.tex|&forwarding file for final version of chapter 2\\
% \end{tabular}
% \end{center}
% Each of the eight files can be compiled directly by the \LaTeX{} compiler.
%
% %%%%%%%%%%%%%%%%%%%%%%%%%%%%%%%%%%%%%%
% \paragraph{Main File.}
%
% The main file is called |cdocsamp.tex|.
%
% Load the \textsf{childdoc} definitions and
% declare the filename for the main document:
%    \begin{macrocode}
\input{childdoc.def}
\childdocmain{}
%    \end{macrocode}

% Optional override for |\version| flag:
%    \begin{macrocode}
%%\ifchilddoc\else\providecommand{\version}{draft}\fi
%    \end{macrocode}

% Define the default values for the |\version| flag
% (|final| for the main file and |draft| for childs):
%    \begin{macrocode}
\ifchilddoc
\providecommand{\version}{draft}
\else
\providecommand{\version}{final}
\fi
%    \end{macrocode}

% Load the standard document class:
%    \begin{macrocode}
\documentclass[12pt]{article}
%    \end{macrocode}

% Start the document body:
%    \begin{macrocode}
\begin{document}
%    \end{macrocode}

% Declare a title page.
% Print title, part of document being processed and version flag:
%    \begin{macrocode}
\addtocounter{page}{-1}
\begin{center}
{\LARGE\bfseries{}childdoc example\par}
\vspace{1cm}
\ifchilddoc
\ifchilddocmanual part\else chapter\fi:
`\childdocname' of `\childdocjob'\par
\else
main document: `\childdocjob'\par
\fi
version: \version\par
\end{center}
\newpage
%    \end{macrocode}

% Manually include selected file,
% otherwise process as usual:
%    \begin{macrocode}
\ifchilddocmanual
\section*{part `\childdocname'}
\input{\childdocname}
\else
%    \end{macrocode}

% Include the two chapters:
%    \begin{macrocode}
\include{cdocsch1}
\include{cdocsch2}
%    \end{macrocode}

% Include the two parts unless only chapters should be displayed:
%    \begin{macrocode}
\ifchilddoc\else
\section{part three}
\input{cdocspt3}
\section{part four}
\input{cdocspt4}
\fi
%    \end{macrocode}

% Process as usual until here:
%    \begin{macrocode}
\fi
%    \end{macrocode}

% End of document body:
%    \begin{macrocode}
\end{document}
%    \end{macrocode}
%\iffalse
%</samplemain>
%\fi
%
% %%%%%%%%%%%%%%%%%%%%%%%%%%%%%%%%%%%%%%
% \paragraph{Chapter Include Files.}
%
% The include files are called |cdocsch1.tex| and |cdocsch2.tex|.
%
%\iffalse
%<*samplechap1|samplechap2>
%\fi

% Optional override for |\version| flag:
%    \begin{macrocode}
%%\providecommand{\version}{final}
%    \end{macrocode}

% Include the main document:
%    \begin{macrocode}
\input{childdoc.def}
\childdocof{cdocsamp}
%    \end{macrocode}

%\iffalse
%</samplechap1|samplechap2>
%\fi
%
%\iffalse
%<*samplechap1>
%\fi
% Some text for chapter 1:
%    \begin{macrocode}
\section{one}
some text in chapter one
%    \end{macrocode}

%\iffalse
%</samplechap1>
%\fi
% Some text for chapter 2:
%\iffalse
%<*samplechap2>
%\fi
%    \begin{macrocode}
\section{two}
more text in chapter two
%    \end{macrocode}

%\iffalse
%</samplechap2>
%\fi
%
% %%%%%%%%%%%%%%%%%%%%%%%%%%%%%%%%%%%%%%
% \paragraph{Part Include Files.}
%
% The include files are called |cdocspt3.tex| and |cdocspt4.tex|.
%
%\iffalse
%<*samplepart3|samplepart4>
%\fi

% Optional override for |\version| flag:
%    \begin{macrocode}
%%\providecommand{\version}{final}
%    \end{macrocode}

% Include the main document:
%    \begin{macrocode}
\input{childdoc.def}
\childdocby{cdocsamp}
%    \end{macrocode}

%\iffalse
%</samplepart3|samplepart4>
%\fi
%
%\iffalse
%<*samplepart3>
%\fi
% Some text for part 3:
%    \begin{macrocode}
some text in part three
%    \end{macrocode}

%\iffalse
%</samplepart3>
%\fi
% Some text for part 4:
%\iffalse
%<*samplepart4>
%\fi
%    \begin{macrocode}
more text in part four
%    \end{macrocode}

%\iffalse
%</samplepart4>
%\fi
%
% %%%%%%%%%%%%%%%%%%%%%%%%%%%%%%%%%%%%%%
% \paragraph{Forwarding for a Complete Draft.}
%
% The following forwarding file |cdocsdrf.tex|
% compiles the main document in draft mode:
%\iffalse
%<*sampledraft>
%\fi
%    \begin{macrocode}
\def\version{draft}
\input{childdoc.def}
\childdocforward{cdocsamp}
%    \end{macrocode}

%\iffalse
%</sampledraft>
%\fi
%
% %%%%%%%%%%%%%%%%%%%%%%%%%%%%%%%%%%%%%%
% \paragraph{Forwarding for Final Version of the Chapters.}
%
% The following forwarding files |cdocsfn1.tex| and |cdocsfn2.tex|
% (with identical content)
% compile the final versions of the child documents
% |cdocsch1.tex| and |cdocsch2.tex|, respectively:
%\iffalse
%<*samplefinal>
%\fi
%    \begin{macrocode}
\def\version{final}
\input{childdoc.def}
\childdocforwardprefix[cdocsamp]{cdocsfn}{cdocsch}
%    \end{macrocode}

%\iffalse
%</samplefinal>
%\fi
%
% %%%%%%%%%%%%%%%%%%%%%%%%%%%%%%%%%%%%%%
% \paragraph{Command Line Processing.}
%
% The following three command lines generate the output files
% |cdocscld|, |cdocscl1| and |cdocscl2|
% which should be identical to
% |cdocsdrf|, |cdocsch1| and |cdocsfn2|, respectively:
% \begin{center}
% \begin{tabular}{l}
% |latex -jobname cdocscld \|\\
% |  "\def\version{draft}\input{childdoc.def}\childdocforward{cdocsamp}"|\\
% |latex -jobname cdocscl1 \|\\
% |  "\input{childdoc.def}\childdocforward[cdocsamp]{cdocsch1}"|\\
% |latex -jobname cdocscl2 \|\\
% |  "\def\version{final}\input{childdoc.def}\childdocforward{cdocsch2}"|
% \end{tabular}
% \end{center}
% Note that the trailing backslash on each first line
% merely continues the input to the second line
% (for convenient cut ant paste).
% Furthermore, the command |latex| can be replaced by any
% of its alternative versions such as |pdflatex|.
%
% %%%%%%%%%%%%%%%%%%%%%%%%%%%%%%%%%%%%%%%%%%%%%%%%%%%%%%%%%%%%%%%%%%%%%%%%%%%%%%
% %%%%%%%%%%%%%%%%%%%%%%%%%%%%%%%%%%%%%%%%%%%%%%%%%%%%%%%%%%%%%%%%%%%%%%%%%%%%%%
% \section{Implementation}
%\iffalse
%<*package>
%\fi
%
% This section describes the definitions file |childdoc.def|.

% The definitions cannot be loaded using |\usepackage| or |\RequirePackage|
% which has a mechanism to prevent loading a style file more than once.
% When loading the definitions by means of |\input|
% multiple instances have to be prevented manually:
%\iffalse
%This code needs to be before the `\ProvidesFile' directive
%which is defined at the beginning of this file.
%Therefore it is also placed there and commented out here.
%</package>
%<*discard>
%\fi
%    \begin{macrocode}
\ifdefined\childdocmain\endinput\fi
%    \end{macrocode}
%\iffalse
%</discard>
%<*package>
%\fi
%
% \macro{\ifchilddoc}
% \macro{\ifchilddocmanual}
% The conditional |\ifchilddoc| tells whether a
% child (true) or main (false) document is being compiled.
% The conditional |\ifchilddocmanual| tells whether
% the |\includeonly| mechanism is used (false) or
% the selection of child files must be performed manually (true).
% The definitions initialise to false:
%    \begin{macrocode}
\newif\ifchilddoc
\newif\ifchilddocmanual
%    \end{macrocode}

% \macro{\childdocname}
% \macro{\childdocjob}
% The macro |\childdocname| stores the name of the main document
% to be compiled. The macro |\childdocjob| stores the name of
% the document on which the \LaTeX{} compiler was originally invoked.
% The content of |\jobname| cannot be compared
% to filenames specified in the source due to different catcodes.
% The following code rescans |\jobname|, stores the result
% in |\childdocname| and saves a copy in |\childdocjob|:
%    \begin{macrocode}
\edef\childdocname{\scantokens\expandafter{\jobname\noexpand}}
\let\childdocjob\childdocname
%    \end{macrocode}

% \macro{\childdocdisable}
% The macro |\childdocdisable| prevents the main file
% from being processed more than once.
% At this stage, the main document command |\childdocmain|
% is assumed to be called once again where it should do nothing.
% Any subsequent call to it should prevent
% a secondary processing of the main document
% It overwrites the forwarding commands
% |\childdocof| and |\childdocforward|
% with empty macros to prevent further inclusions of the main document:
%    \begin{macrocode}
\newcommand{\childdocdisable}
{
  \renewcommand{\childdocmain}[1]{\renewcommand{\childdocmain}[1]{\endinput}}
  \renewcommand{\childdocof}[1]{}
  \renewcommand{\childdocby}[2][]{}
  \renewcommand{\childdocforward}[2][]{}
  \renewcommand{\childdocdisable}{}
}
%    \end{macrocode}

% \macro{\childdocmain}
% The macro |\childdocmain| is to be called at the top of the main file
% with nothing or the main filename (without extension) as argument.
% First, it breaks loops.
% If the argument is not empty and does not match |\childdocname|
% (which is set by the first inclusion of |childdoc.def|),
% |\ifchilddoc| is set to true, |\includeonly| is applied to the child file
% and |\jobname| is set to the main file
% (for proper handling of |.aux| files):
%    \begin{macrocode}
\newcommand{\childdocmain}[1]
{
  \childdocdisable\childdocmain{}
  \if?#1?\else
    \begingroup
      \def\childdoctmp{#1}
      \ifx\childdoctmp\childdocname
        \def\childdoctmp{}
      \else
        \def\childdoctmp
        {
          \childdoctrue
          \includeonly{\childdocname}
          \def\childdocjob{#1}
          \def\jobname{#1}
        }
      \fi
      \expandafter
    \endgroup
    \childdoctmp
  \fi
}
%    \end{macrocode}

% \macro{\childdocof}
% The command |\childdocof| redirects
% compilation to the main file |#1|.
%    \begin{macrocode}
\newcommand{\childdocof}[1]
{
  \childdocdisable
  \childdoctrue
  \includeonly{\childdocname}
  \def\jobname{#1}
  \def\childdocjob{#1}
  \input{#1}
}
%    \end{macrocode}

% \macro{\childdocby}
% The command |\childdocby| ....
%    \begin{macrocode}
\newcommand{\childdocby}[2][]
{
  \childdocdisable
  \childdoctrue
  \childdocmanualtrue
  \if?#1?\else
    \def\jobname{#2}
  \fi
  \def\childdocjob{#2}
  \input{#2}
  \endinput
}
%    \end{macrocode}

% \macro{\childdocforward}
% The command |\childdocforward| redirects
% compilation to the main file or
% (if the optional argument is given) a child file.
% Parameters are set as if the main file
% or a child file starting with |\childdocof| was compiled.
% Then compilation is handed over to the main file:
%    \begin{macrocode}
\newcommand{\childdocforward}[2][]
{
  \begingroup
    \if?#1?
      \def\childdoctmp
      {
        \def\childdocname{#2}
        \def\childdocjob{#2}
        \def\jobname{#2}
        \input{#2}
        \endinput
      }
    \else
      \def\childdoctmp
      {
        \childdocdisable
        \def\childdocname{#2}
        \childdoctrue
        \includeonly{#2}
        \def\childdocjob{#1}
        \def\jobname{#1}
        \input{#1}
        \endinput
      }
    \fi
    \expandafter
  \endgroup
  \childdoctmp
}
%    \end{macrocode}

% \macro{\childdocforwardprefix}
% The command |\childdocforwardprefix| redirects
% compilation to the main or a child file by means of a pattern.
% The prefix |#1| in the current filename is replaced by |#2|
% and the suffix of the current filename is kept
% (it is assumed that the filename does not contain the substring `|~~~|'
% which is used as a delimiter).
% Compilation is handed over to the new file by |\childdocforward|:
%    \begin{macrocode}
\newcommand{\childdocforwardprefix}[3][]
{
  \begingroup
    \def\childdocextract #2##1~~~{\def\childdoctmp{\childdocforward[#1]{#3##1}}}
    \expandafter\childdocextract\childdocname~~~
    \expandafter
  \endgroup
  \childdoctmp
}
%    \end{macrocode}

% \macro{\childdoc}
% The deprecated macro |\childdoc| is a legacy version of |\childdocmain|:
%    \begin{macrocode}
\newcommand{\childdoc}{\childdocmain}
%    \end{macrocode}

% \macro{\childdocredirect}
% The deprecated macro |\childdocredirect| is a legacy version
% of |\childdocforward| and |\childdocforwardprefix|:
%    \begin{macrocode}
\newcommand{\childdocredirect}[2][]
{
  \begingroup
    \if?#1?
      \def\childdoctmp{\childdocforward{#2}}
    \else
      \def\childdoctmp{\childdocforwardprefix{#1}{#2}}
    \fi
    \expandafter
  \endgroup
  \childdoctmp
}
%    \end{macrocode}

%\iffalse
%</package>
%\fi
%
\endinput
\childdocforward{cdocsch2}"|
% \end{tabular}
% \end{center}
% Note that the trailing backslash on each first line
% merely continues the input to the second line
% (for convenient cut ant paste).
% Furthermore, the command |latex| can be replaced by any
% of its alternative versions such as |pdflatex|.
%
% %%%%%%%%%%%%%%%%%%%%%%%%%%%%%%%%%%%%%%%%%%%%%%%%%%%%%%%%%%%%%%%%%%%%%%%%%%%%%%
% %%%%%%%%%%%%%%%%%%%%%%%%%%%%%%%%%%%%%%%%%%%%%%%%%%%%%%%%%%%%%%%%%%%%%%%%%%%%%%
% \section{Implementation}
%\iffalse
%<*package>
%\fi
%
% This section describes the definitions file |childdoc.def|.

% The definitions cannot be loaded using |\usepackage| or |\RequirePackage|
% which has a mechanism to prevent loading a style file more than once.
% When loading the definitions by means of |\input|
% multiple instances have to be prevented manually:
%\iffalse
%This code needs to be before the `\ProvidesFile' directive
%which is defined at the beginning of this file.
%Therefore it is also placed there and commented out here.
%</package>
%<*discard>
%\fi
%    \begin{macrocode}
\ifdefined\childdocmain\endinput\fi
%    \end{macrocode}
%\iffalse
%</discard>
%<*package>
%\fi
%
% \macro{\ifchilddoc}
% \macro{\ifchilddocmanual}
% The conditional |\ifchilddoc| tells whether a
% child (true) or main (false) document is being compiled.
% The conditional |\ifchilddocmanual| tells whether
% the |\includeonly| mechanism is used (false) or
% the selection of child files must be performed manually (true).
% The definitions initialise to false:
%    \begin{macrocode}
\newif\ifchilddoc
\newif\ifchilddocmanual
%    \end{macrocode}

% \macro{\childdocname}
% \macro{\childdocjob}
% The macro |\childdocname| stores the name of the main document
% to be compiled. The macro |\childdocjob| stores the name of
% the document on which the \LaTeX{} compiler was originally invoked.
% The content of |\jobname| cannot be compared
% to filenames specified in the source due to different catcodes.
% The following code rescans |\jobname|, stores the result
% in |\childdocname| and saves a copy in |\childdocjob|:
%    \begin{macrocode}
\edef\childdocname{\scantokens\expandafter{\jobname\noexpand}}
\let\childdocjob\childdocname
%    \end{macrocode}

% \macro{\childdocdisable}
% The macro |\childdocdisable| prevents the main file
% from being processed more than once.
% At this stage, the main document command |\childdocmain|
% is assumed to be called once again where it should do nothing.
% Any subsequent call to it should prevent
% a secondary processing of the main document
% It overwrites the forwarding commands
% |\childdocof| and |\childdocforward|
% with empty macros to prevent further inclusions of the main document:
%    \begin{macrocode}
\newcommand{\childdocdisable}
{
  \renewcommand{\childdocmain}[1]{\renewcommand{\childdocmain}[1]{\endinput}}
  \renewcommand{\childdocof}[1]{}
  \renewcommand{\childdocby}[2][]{}
  \renewcommand{\childdocforward}[2][]{}
  \renewcommand{\childdocdisable}{}
}
%    \end{macrocode}

% \macro{\childdocmain}
% The macro |\childdocmain| is to be called at the top of the main file
% with nothing or the main filename (without extension) as argument.
% First, it breaks loops.
% If the argument is not empty and does not match |\childdocname|
% (which is set by the first inclusion of |childdoc.def|),
% |\ifchilddoc| is set to true, |\includeonly| is applied to the child file
% and |\jobname| is set to the main file
% (for proper handling of |.aux| files):
%    \begin{macrocode}
\newcommand{\childdocmain}[1]
{
  \childdocdisable\childdocmain{}
  \if?#1?\else
    \begingroup
      \def\childdoctmp{#1}
      \ifx\childdoctmp\childdocname
        \def\childdoctmp{}
      \else
        \def\childdoctmp
        {
          \childdoctrue
          \includeonly{\childdocname}
          \def\childdocjob{#1}
          \def\jobname{#1}
        }
      \fi
      \expandafter
    \endgroup
    \childdoctmp
  \fi
}
%    \end{macrocode}

% \macro{\childdocof}
% The command |\childdocof| redirects
% compilation to the main file |#1|.
%    \begin{macrocode}
\newcommand{\childdocof}[1]
{
  \childdocdisable
  \childdoctrue
  \includeonly{\childdocname}
  \def\jobname{#1}
  \def\childdocjob{#1}
  \input{#1}
}
%    \end{macrocode}

% \macro{\childdocby}
% The command |\childdocby| ....
%    \begin{macrocode}
\newcommand{\childdocby}[2][]
{
  \childdocdisable
  \childdoctrue
  \childdocmanualtrue
  \if?#1?\else
    \def\jobname{#2}
  \fi
  \def\childdocjob{#2}
  \input{#2}
  \endinput
}
%    \end{macrocode}

% \macro{\childdocforward}
% The command |\childdocforward| redirects
% compilation to the main file or
% (if the optional argument is given) a child file.
% Parameters are set as if the main file
% or a child file starting with |\childdocof| was compiled.
% Then compilation is handed over to the main file:
%    \begin{macrocode}
\newcommand{\childdocforward}[2][]
{
  \begingroup
    \if?#1?
      \def\childdoctmp
      {
        \def\childdocname{#2}
        \def\childdocjob{#2}
        \def\jobname{#2}
        \input{#2}
        \endinput
      }
    \else
      \def\childdoctmp
      {
        \childdocdisable
        \def\childdocname{#2}
        \childdoctrue
        \includeonly{#2}
        \def\childdocjob{#1}
        \def\jobname{#1}
        \input{#1}
        \endinput
      }
    \fi
    \expandafter
  \endgroup
  \childdoctmp
}
%    \end{macrocode}

% \macro{\childdocforwardprefix}
% The command |\childdocforwardprefix| redirects
% compilation to the main or a child file by means of a pattern.
% The prefix |#1| in the current filename is replaced by |#2|
% and the suffix of the current filename is kept
% (it is assumed that the filename does not contain the substring `|~~~|'
% which is used as a delimiter).
% Compilation is handed over to the new file by |\childdocforward|:
%    \begin{macrocode}
\newcommand{\childdocforwardprefix}[3][]
{
  \begingroup
    \def\childdocextract #2##1~~~{\def\childdoctmp{\childdocforward[#1]{#3##1}}}
    \expandafter\childdocextract\childdocname~~~
    \expandafter
  \endgroup
  \childdoctmp
}
%    \end{macrocode}

% \macro{\childdoc}
% The deprecated macro |\childdoc| is a legacy version of |\childdocmain|:
%    \begin{macrocode}
\newcommand{\childdoc}{\childdocmain}
%    \end{macrocode}

% \macro{\childdocredirect}
% The deprecated macro |\childdocredirect| is a legacy version
% of |\childdocforward| and |\childdocforwardprefix|:
%    \begin{macrocode}
\newcommand{\childdocredirect}[2][]
{
  \begingroup
    \if?#1?
      \def\childdoctmp{\childdocforward{#2}}
    \else
      \def\childdoctmp{\childdocforwardprefix{#1}{#2}}
    \fi
    \expandafter
  \endgroup
  \childdoctmp
}
%    \end{macrocode}

%\iffalse
%</package>
%\fi
%
\endinput
\childdocforward[cdocsamp]{cdocsch1}"|\\
% |latex -jobname cdocscl2 \|\\
% |  "\def\version{final}% \iffalse
%
% childdoc.dtx Copyright (C) 2017-2018 Niklas Beisert
%
% This work may be distributed and/or modified under the
% conditions of the LaTeX Project Public License, either version 1.3
% of this license or (at your option) any later version.
% The latest version of this license is in
%   http://www.latex-project.org/lppl.txt
% and version 1.3 or later is part of all distributions of LaTeX
% version 2005/12/01 or later.
%
% This work has the LPPL maintenance status `maintained'.
%
% The Current Maintainer of this work is Niklas Beisert.
%
% This work consists of the files childdoc.dtx and childdoc.ins
% and the derived files childdoc.def and cdocsamp.tex with
% cdocsch1.tex, cdocsch2.tex, cdocsdrf.tex, cdocsfn1.tex, cdocsfn2.tex.
%
%<package>\ifdefined\childdocmain\endinput\fi
%<package>\ProvidesFile{childdoc.def}[2018/12/30 v2.0 child document driver]
%<samplemain>\ProvidesFile{cdocsamp.tex}[2018/12/30 v2.0 sample for childdoc]
%<*driver>
%\ProvidesFile{childdoc.drv}[2018/12/30 v2.0 childdoc reference manual file]
\PassOptionsToClass{10pt,a4paper}{article}
\documentclass{ltxdoc}

\usepackage[margin=35mm]{geometry}
\usepackage{hyperref}
\usepackage{hyperxmp}
\usepackage[usenames]{color}

\hypersetup{colorlinks=true}
\hypersetup{pdfstartview=FitH}
\hypersetup{pdfpagemode=UseNone}
\hypersetup{pdfsource={}}
\hypersetup{pdflang={en-UK}}
\hypersetup{pdfcopyright={Copyright 2017-2018 Niklas Beisert.
  This work may be distributed and/or modified under the
  conditions of the LaTeX Project Public License, either version 1.3
  of this license or (at your option) any later version.}}
\hypersetup{pdflicenseurl={http://www.latex-project.org/lppl.txt}}
\hypersetup{pdfcontactaddress={ETH Zurich, ITP, HIT K,
  Wolfgang-Pauli-Strasse 27}}
\hypersetup{pdfcontactpostcode={8093}}
\hypersetup{pdfcontactcity={Zurich}}
\hypersetup{pdfcontactcountry={Switzerland}}
\hypersetup{pdfcontactemail={nbeisert@itp.phys.ethz.ch}}
\hypersetup{pdfcontacturl={http://people.phys.ethz.ch/\xmptilde nbeisert/}}

\newcommand{\secref}[1]{\hyperref[#1]{section \ref*{#1}}}

\parskip1ex
\parindent0pt
\let\olditemize\itemize
\def\itemize{\olditemize\parskip0pt}

\begin{document}

\title{The \textsf{childdoc} Package}
\hypersetup{pdftitle={The childdoc Package}}
\author{Niklas Beisert\\[2ex]
  Institut f\"ur Theoretische Physik\\
  Eidgen\"ossische Technische Hochschule Z\"urich\\
  Wolfgang-Pauli-Strasse 27, 8093 Z\"urich, Switzerland\\[1ex]
  \href{mailto:nbeisert@itp.phys.ethz.ch}
  {\texttt{nbeisert@itp.phys.ethz.ch}}}
\hypersetup{pdfauthor={Niklas Beisert}}
\hypersetup{pdfsubject={Manual for the LaTeX2e Package childdoc}}
\date{30 December 2018, \textsf{v2.0}}
\maketitle

\begin{abstract}\noindent
\textsf{childdoc} is a \LaTeXe{} package
that enables the direct compilation
of document sections included by |\include|
to individual files.
\end{abstract}

\begingroup
\parskip0ex
\tableofcontents
\endgroup

%%%%%%%%%%%%%%%%%%%%%%%%%%%%%%%%%%%%%%%%%%%%%%%%%%%%%%%%%%%%%%%%%%%%%%%%%%%%%%%%
%%%%%%%%%%%%%%%%%%%%%%%%%%%%%%%%%%%%%%%%%%%%%%%%%%%%%%%%%%%%%%%%%%%%%%%%%%%%%%%%
\section{Introduction}

\LaTeX{} provides a mechanism to structure a large document (such as a book)
into a main file and several child files (containing the chapters)
using the |\include| command.
This mechanism is beneficial for documents
which span hundreds of pages in order to
make the source file(s) more manageable.
Moreover, compilation can be restricted to
selected child files by means of the |\includeonly| command.
The latter feature can be used to reduce the compilation time while editing
(this was significantly more useful in the earlier days of \LaTeX{})
or to generate a smaller document which is easier to navigate.
Another application of |\includeonly| is to generate
documents consisting of selected parts of the complete document.

However, there are a few drawbacks of the plain |\include| mechanism:
\begin{itemize}
\item
The child files cannot be compiled on their own,
they can only be compiled via the main file.
A naive editing environment
(such as a text editor with an option
to have the current file processed by \LaTeX)
may require one to switch to the main file before compiling;
attempting to compile the child file produces errors.
\item
The main file must be modified (each time)
to adjust the |\includeonly| command
to the present needs. This easily leaves the main file in a messy state.
\item
The generated document will always carry the filename
of the main document. This is inconvenient if
several child files are to be compiled and
to be kept for distribution.
\end{itemize}

The present package provides a simple interface
to make child files individually compilable by \LaTeX{}.
Compiling a child file then has the same effect as compiling
the main file with an |\includeonly| command
to select the appropriate child.
Moreover the generated document will carry the name of the child
rather than the main file.
This resolves all three above issues.

This feature is meant to make the editing of books,
thesis documents and lecture notes somewhat more convenient.
However, the package can also be used efficiently for
composing a series of documents (such as exercise sheets)
which are typically distributed individually.
It then assists the author in generating the individual documents
(potentially in different versions)
as well as a document containing the collected series.
Another application is in developing style files
or other kinds of included material
where compilation of the style file could redirect
to a sample or test file.

%%%%%%%%%%%%%%%%%%%%%%%%%%%%%%%%%%%%%%%%%%%%%%%%%%%%%%%%%%%%%%%%%%%%%%%%%%%%%%%%
%%%%%%%%%%%%%%%%%%%%%%%%%%%%%%%%%%%%%%%%%%%%%%%%%%%%%%%%%%%%%%%%%%%%%%%%%%%%%%%%
\section{Usage}

First of all, the package \textsf{childdoc} is \emph{not} a standard
\LaTeXe{} |.sty| style file! Therefore it needs to be invoked in
a non-standard way.

%%%%%%%%%%%%%%%%%%%%%%%%%%%%%%%%%%%%%%%%%%%%%%%%%%%%%%%%%%%%%%%%%%%%%%%%%%%%%%%%
\subsection{Included Files}
\label{sec:include}

%%%%%%%%%%%%%%%%%%%%%%%%%%%%%%%%%%%%%%%%
\DescribeMacro{\childdocmain}
To use the package, add the commands
\begin{center}
\begin{tabular}{l}
|% \iffalse
%
% childdoc.dtx Copyright (C) 2017-2018 Niklas Beisert
%
% This work may be distributed and/or modified under the
% conditions of the LaTeX Project Public License, either version 1.3
% of this license or (at your option) any later version.
% The latest version of this license is in
%   http://www.latex-project.org/lppl.txt
% and version 1.3 or later is part of all distributions of LaTeX
% version 2005/12/01 or later.
%
% This work has the LPPL maintenance status `maintained'.
%
% The Current Maintainer of this work is Niklas Beisert.
%
% This work consists of the files childdoc.dtx and childdoc.ins
% and the derived files childdoc.def and cdocsamp.tex with
% cdocsch1.tex, cdocsch2.tex, cdocsdrf.tex, cdocsfn1.tex, cdocsfn2.tex.
%
%<package>\ifdefined\childdocmain\endinput\fi
%<package>\ProvidesFile{childdoc.def}[2018/12/30 v2.0 child document driver]
%<samplemain>\ProvidesFile{cdocsamp.tex}[2018/12/30 v2.0 sample for childdoc]
%<*driver>
%\ProvidesFile{childdoc.drv}[2018/12/30 v2.0 childdoc reference manual file]
\PassOptionsToClass{10pt,a4paper}{article}
\documentclass{ltxdoc}

\usepackage[margin=35mm]{geometry}
\usepackage{hyperref}
\usepackage{hyperxmp}
\usepackage[usenames]{color}

\hypersetup{colorlinks=true}
\hypersetup{pdfstartview=FitH}
\hypersetup{pdfpagemode=UseNone}
\hypersetup{pdfsource={}}
\hypersetup{pdflang={en-UK}}
\hypersetup{pdfcopyright={Copyright 2017-2018 Niklas Beisert.
  This work may be distributed and/or modified under the
  conditions of the LaTeX Project Public License, either version 1.3
  of this license or (at your option) any later version.}}
\hypersetup{pdflicenseurl={http://www.latex-project.org/lppl.txt}}
\hypersetup{pdfcontactaddress={ETH Zurich, ITP, HIT K,
  Wolfgang-Pauli-Strasse 27}}
\hypersetup{pdfcontactpostcode={8093}}
\hypersetup{pdfcontactcity={Zurich}}
\hypersetup{pdfcontactcountry={Switzerland}}
\hypersetup{pdfcontactemail={nbeisert@itp.phys.ethz.ch}}
\hypersetup{pdfcontacturl={http://people.phys.ethz.ch/\xmptilde nbeisert/}}

\newcommand{\secref}[1]{\hyperref[#1]{section \ref*{#1}}}

\parskip1ex
\parindent0pt
\let\olditemize\itemize
\def\itemize{\olditemize\parskip0pt}

\begin{document}

\title{The \textsf{childdoc} Package}
\hypersetup{pdftitle={The childdoc Package}}
\author{Niklas Beisert\\[2ex]
  Institut f\"ur Theoretische Physik\\
  Eidgen\"ossische Technische Hochschule Z\"urich\\
  Wolfgang-Pauli-Strasse 27, 8093 Z\"urich, Switzerland\\[1ex]
  \href{mailto:nbeisert@itp.phys.ethz.ch}
  {\texttt{nbeisert@itp.phys.ethz.ch}}}
\hypersetup{pdfauthor={Niklas Beisert}}
\hypersetup{pdfsubject={Manual for the LaTeX2e Package childdoc}}
\date{30 December 2018, \textsf{v2.0}}
\maketitle

\begin{abstract}\noindent
\textsf{childdoc} is a \LaTeXe{} package
that enables the direct compilation
of document sections included by |\include|
to individual files.
\end{abstract}

\begingroup
\parskip0ex
\tableofcontents
\endgroup

%%%%%%%%%%%%%%%%%%%%%%%%%%%%%%%%%%%%%%%%%%%%%%%%%%%%%%%%%%%%%%%%%%%%%%%%%%%%%%%%
%%%%%%%%%%%%%%%%%%%%%%%%%%%%%%%%%%%%%%%%%%%%%%%%%%%%%%%%%%%%%%%%%%%%%%%%%%%%%%%%
\section{Introduction}

\LaTeX{} provides a mechanism to structure a large document (such as a book)
into a main file and several child files (containing the chapters)
using the |\include| command.
This mechanism is beneficial for documents
which span hundreds of pages in order to
make the source file(s) more manageable.
Moreover, compilation can be restricted to
selected child files by means of the |\includeonly| command.
The latter feature can be used to reduce the compilation time while editing
(this was significantly more useful in the earlier days of \LaTeX{})
or to generate a smaller document which is easier to navigate.
Another application of |\includeonly| is to generate
documents consisting of selected parts of the complete document.

However, there are a few drawbacks of the plain |\include| mechanism:
\begin{itemize}
\item
The child files cannot be compiled on their own,
they can only be compiled via the main file.
A naive editing environment
(such as a text editor with an option
to have the current file processed by \LaTeX)
may require one to switch to the main file before compiling;
attempting to compile the child file produces errors.
\item
The main file must be modified (each time)
to adjust the |\includeonly| command
to the present needs. This easily leaves the main file in a messy state.
\item
The generated document will always carry the filename
of the main document. This is inconvenient if
several child files are to be compiled and
to be kept for distribution.
\end{itemize}

The present package provides a simple interface
to make child files individually compilable by \LaTeX{}.
Compiling a child file then has the same effect as compiling
the main file with an |\includeonly| command
to select the appropriate child.
Moreover the generated document will carry the name of the child
rather than the main file.
This resolves all three above issues.

This feature is meant to make the editing of books,
thesis documents and lecture notes somewhat more convenient.
However, the package can also be used efficiently for
composing a series of documents (such as exercise sheets)
which are typically distributed individually.
It then assists the author in generating the individual documents
(potentially in different versions)
as well as a document containing the collected series.
Another application is in developing style files
or other kinds of included material
where compilation of the style file could redirect
to a sample or test file.

%%%%%%%%%%%%%%%%%%%%%%%%%%%%%%%%%%%%%%%%%%%%%%%%%%%%%%%%%%%%%%%%%%%%%%%%%%%%%%%%
%%%%%%%%%%%%%%%%%%%%%%%%%%%%%%%%%%%%%%%%%%%%%%%%%%%%%%%%%%%%%%%%%%%%%%%%%%%%%%%%
\section{Usage}

First of all, the package \textsf{childdoc} is \emph{not} a standard
\LaTeXe{} |.sty| style file! Therefore it needs to be invoked in
a non-standard way.

%%%%%%%%%%%%%%%%%%%%%%%%%%%%%%%%%%%%%%%%%%%%%%%%%%%%%%%%%%%%%%%%%%%%%%%%%%%%%%%%
\subsection{Included Files}
\label{sec:include}

%%%%%%%%%%%%%%%%%%%%%%%%%%%%%%%%%%%%%%%%
\DescribeMacro{\childdocmain}
To use the package, add the commands
\begin{center}
\begin{tabular}{l}
|\input{childdoc.def}|\\
|\childdocmain{}|\\
\end{tabular}
\end{center}
at the very top of the main \LaTeX{} file,
in particular \emph{before} the |\documentclass| statement!
The argument of |\childdocmain| should be left empty
(but it must be present).

%%%%%%%%%%%%%%%%%%%%%%%%%%%%%%%%%%%%%%%%
\DescribeMacro{\childdocof}
Furthermore, add the commands
\begin{center}
\begin{tabular}{l}
|\input{childdoc.def}|\\
|\childdocof{|\textit{main}|}|\\
\end{tabular}
\end{center}
at the top of every child file \textit{child}
which is included by |\include{|\textit{child}|}|
from within the main file
(or at least for those files to be compiled individually).
The argument \textit{main} must be the filename of the main file.

There are a couple of
considerations in setting up the main and child documents:

%%%%%%%%%%%%%%%%%%%%%%%%%%%%%%%%%%%%%%%%
\paragraph{Restrictions.}

Please note the following restrictions:
\begin{itemize}
\item
|\childdocmain| must be called with one argument \textit{main}
to ensure compatibility with earlier version of the package.
It must either be empty (|\childdocmain{}|)
or precisely match the filename of the main file in which it is specified.
See \secref{sec:detection} for further information.
\item
The filename \textit{main} must be specified without the |.tex| extension.
\item
The filename \textit{main} is case sensitive
(even in case-insensitive file systems)
due to internal string comparison.
\item
The argument \textit{main} should be fully expanded, it cannot be a macro.
\item
Subdirectories and special characters should be avoided in filenames.
\item
The command |\childdocmain{|\textit{main}|}| must be followed by a whitespace.
It should not be followed immediately by another command
or by a comment mark `|%|'.
This is because the \TeX{} parser reads the token immediately following
the argument of |\childdocmain| and puts it
at the beginning of every child section;
however, a white\-space is ignored.
\end{itemize}

%%%%%%%%%%%%%%%%%%%%%%%%%%%%%%%%%%%%%%%%
\paragraph{Content of Main File.}

It is advisable to place all content in the child files included by |\include|.
Any output contained in the main file will appear in all child documents
unless suppressed manually;
it cannot be suppressed automatically by the |\includeonly| directive
and thus should normally be avoided.
A method to include some content in the main file
by means of conditional processing is described in \secref{sec:conditional}.

%%%%%%%%%%%%%%%%%%%%%%%%%%%%%%%%%%%%%%%%
\paragraph{Page Numbering.}

When only a part of the document is compiled,
the appropriate numbering of pages
(as well as other status parameters)
is determined from the |.aux| files.
The latter contain information from previous passes.
However this information needs to propagate through
all intermediate child documents.
Therefore the page numbering in child documents may well
be inconsistent until the complete document is compiled at least once.

A useful (if unconventional) way to always ensure a consistent
page numbering is to restart the numbering in each child document
and denote the pages by `\textit{child}|.|\textit{page}'
where \textit{child} represents the chapter/section number of the child file.
This can be achieved by the command
|\numberwithin{page}{|\textit{child}|}|
of the \textsf{amsmath} package
where \textit{child} can be |chapter| or |section|
depending on the chosen structuring.
Alternatively, one can modify the macro |\thepage| appropriately
and reset the counter |page| at the start of each child file.

%%%%%%%%%%%%%%%%%%%%%%%%%%%%%%%%%%%%%%%%%%%%%%%%%%%%%%%%%%%%%%%%%%%%%%%%%%%%%%%%
\subsection{Conditional Processing}
\label{sec:conditional}

The package provides a mechanism to compile different versions
of a document. To customise the versions further some conditional processing
can come in handy to distinguish which version is being compiled.
The package provides two macros to describe the compilation context:

%%%%%%%%%%%%%%%%%%%%%%%%%%%%%%%%%%%%%%%%
\DescribeMacro{\ifchilddoc}
The conditional |\ifchilddoc| distinguishes between the compilation of
child documents and the main document:
%
\begin{center}
|\ifchilddoc |\textit{child-code}| |[|\||else |\textit{main-code}]| \||fi|
\end{center}

%%%%%%%%%%%%%%%%%%%%%%%%%%%%%%%%%%%%%%%%
\DescribeMacro{\childdocname}
\DescribeMacro{\childdocjob}
The macro |\childdocname| contains the filename (without extension)
of the main or child file being processed.
Note that |\childdocjob| will always contain the name of the main file.

%%%%%%%%%%%%%%%%%%%%%%%%%%%%%%%%%%%%%%%%
\paragraph{Title Page.}

Conditional processing can be used to include a title or banner page
in the main document when proper precautions are taken.
Importantly, the code in the main file should ensure that the page counter
(as well as other status parameters which are stored in the |.aux| files)
takes the same value after the conditional processing.
Otherwise the page numbers may take divergent values
depending on which part is compiled.

For example, a title page could be declared by:
%
\begin{center}
\begin{tabular}{l}
|\ifchilddoc\||else|\\
|\addtocounter{page}{-1}|\\
\textit{code for title page}\\
|\newpage|\\
|\||fi|
\end{tabular}
\end{center}
%
A banner page for the child documents can be generated by:
%
\begin{center}
\begin{tabular}{l}
|\ifchilddoc|\\
|\addtocounter{page}{-1}|\\
\textit{code for banner page}\\
|\newpage|\\
|\||fi|
\end{tabular}
\end{center}
%
Here one could write a message such as:
\begin{center}
|This is the part \childdocname{} of \childdocjob{}.|
\end{center}

%%%%%%%%%%%%%%%%%%%%%%%%%%%%%%%%%%%%%%%%%%%%%%%%%%%%%%%%%%%%%%%%%%%%%%%%%%%%%%%%
\subsection{Flags}
\label{sec:flags}

The package makes it easy to generate different versions
of the main or child documents.
To this end compilation flags can be defined
and assigned different default values.
They will be particularly useful in conjunction
with the forwarding mechanism described in \secref{sec:forward}.

For example, it may be useful to have a flag |\version|
which can be set to |draft| or |final|.
The document source will contain some conditional code
depending on the value of |\version|.
Suppose further, the flag should default to |final| for the main file
and to |draft| for child files
which is a natural assignment for editing the document.
This is achieved by placing the following code
in the preamble of the main document
(below the |\childdocmain| directive):
%
\begin{center}
\begin{tabular}{l}
|\ifchilddoc|\\
|\providecommand{\version}{draft}|\\
|\||else|\\
|\providecommand{\version}{final}|\\
|\||fi|
\end{tabular}
\end{center}
%
The definition by |\providecommand| makes sure
that previous definitions are not overwritten.
Further statements |\providecommand{\version}{...}|
can thus be added before the above code to override it.

For the main file, one might add a line
(between |\childdocmain| and the above block)
%
\begin{center}
|%\ifchilddoc\||else\providecommand{\version}{draft}\||fi|
\end{center}
%
which can be uncommented to produce a draft version.
Likewise one can add a line to the very top of a child file
(above the |\childdocof{|\textit{main}|}| directive)
%
\begin{center}
|%\providecommand{\version}{final}|
\end{center}
%
which can be uncommented to produce the final version of this child document.

%%%%%%%%%%%%%%%%%%%%%%%%%%%%%%%%%%%%%%%%%%%%%%%%%%%%%%%%%%%%%%%%%%%%%%%%%%%%%%%%
\subsection{Forwarding}
\label{sec:forward}

Different versions of the main or child documents
using compilation flags as described in \secref{sec:flags}
can be (permanently) stored in different files
for convenient compilation, viewing and distribution.
To this end, the package defines a command
to pass on compilation to a different file:

%%%%%%%%%%%%%%%%%%%%%%%%%%%%%%%%%%%%%%%%
\DescribeMacro{\childdocforward}
The command |\childdocforward| redirects processing to
another source file:
%
\begin{center}
\begin{tabular}{l}
|\input{childdoc.def}|\\
|\childdocforward[|\textit{main}|]{|\textit{dest}|}|\\
\end{tabular}
\end{center}
%
The argument \textit{dest} is the destination file
(without extension).
It should be the main file or one of the child files.
Note that further \textsf{childdoc} directives
such as |\childdocof| and |\childdocforward|
in the indicated file will be processed in this form.
The optional argument \textit{main}
passes on directly to the main file \textit{main}
while pretending to compile the child \textit{dest}.
This form behaves as if \textit{dest}
issues |\childdocof{|\textit{main}|}| right away,
and no further \textsf{childdoc} directives will be processed.

%%%%%%%%%%%%%%%%%%%%%%%%%%%%%%%%%%%%%%%%
\DescribeMacro{\...prefix}
In the alternative form |\childdocforwardprefix|,
%
\begin{center}
\begin{tabular}{l}
|\input{childdoc.def}|\\
|\childdocforwardprefix[|\textit{main}|]{|\textit{prefix}|}{|\textit{dest}|}|
\end{tabular}
\end{center}
%
the destination file is determined by a pattern
depending on the current file:
To make this work, the current file must be called
`{\textit{prefix}\hspace{0.2em}\textit{suffix}}'
with \textit{prefix} matching precisely the argument.
Processing is then passed on to the file
`{\textit{dest}\hspace{0.2em}\textit{suffix}}'.
Surely, the same effect is achieved by
directly specifying the
argument `{\textit{dest}\hspace{0.2em}\textit{suffix}}'
in the first form.
However, that requires to set up a different file
for each child. With the alternative form of the command
all these files can have exactly the same content
which simplifies setting them up and maintaining them.

For example, the following file |draft.tex|
with a compilation flag |\version| as described in \secref{sec:flags}
compiles the main document as a draft:
%
\begin{center}
\begin{tabular}{l}
|\def\version{draft}|\\
|\input{childdoc.def}|\\
|\childdocforward{|\textit{main}|}|
\end{tabular}
\end{center}
%
Likewise, the following files |final|\textit{nn}|.tex|
compile the final version of the child document
|child|\textit{nn}|.tex|:
%
\begin{center}
\begin{tabular}{l}
|\def\version{final}|\\
|\input{childdoc.def}|\\
|\childdocforwardprefix{final}{child}|
\end{tabular}
\end{center}
%

Note that when several versions of a main file and/or of each child file
are to be generated, it may be convenient to set up a |Makefile| or
shell script to automatise the process.

%%%%%%%%%%%%%%%%%%%%%%%%%%%%%%%%%%%%%%%%%%%%%%%%%%%%%%%%%%%%%%%%%%%%%%%%%%%%%%%%
\subsection{Command Line Processing}
\label{sec:commandline}

The effect of redirection files can also be achieved by invoking
the \LaTeX{} compiler with a more elaborate command line.
Most conveniently this should be done as part
of a shell script or a |Makefile|.

When using \textsf{childdoc} in the main file, the following
command lines effectively perform a redirection
(note that depending on the shell being used,
backslashes may have to be doubled: `|\|' $\to$ `|\\|'):
%
\begin{center}
|... -jobname "|\textit{target}|" |\\|"|[\textit{flags}]%
|\input{childdoc.def}\childdocforward[|\textit{main}|]{|\textit{dest}|}"|
\end{center}
%
Here \textit{target} is the name of the output file,
\textit{main} is the name of the main file
and \textit{dest} is the name of the main or child file to be processed
(all filenames without extensions).
The optional argument \textit{main} can be omitted
if \textit{main} matches \textit{dest}.
Optionally, compilation \textit{flags} can be defined via |\def| commands.
This command line makes the \TeX{} engine believe
it is compiling the file \textit{target}
whose content is specified as the latter parameter.
The provided code then forwards the processing to
\textit{main} or \textit{dest} as described in \secref{sec:forward}.

%%%%%%%%%%%%%%%%%%%%%%%%%%%%%%%%%%%%%%%%%%%%%%%%%%%%%%%%%%%%%%%%%%%%%%%%%%%%%%%%
\subsection{Include by Input}
\label{sec:input}

Including child documents by |\include| has some restrictions by design.
Most notably, the content of a child document always occupies
its own set of pages; pages cannot be shared between child documents.
Usually, this behaviour makes perfect sense
because each child document contain an essential part of the document.
However, in some situations it may be desirable to compose
a document from a collection of parts
without having mandatory page breaks between then.
For this case, the package
provides a mechanism to include parts
by |\input| which can also be processed individually.
However, by construction this mechanism
requires manual handling of the content to be output.

%%%%%%%%%%%%%%%%%%%%%%%%%%%%%%%%%%%%%%%%
\DescribeMacro{\ifchilddocmanual}
The main file should be prepared as usual, see \secref{sec:include}.
However, the document body must make a distinction
between processing of an individual part and of the main document, e.g.:
%
\begin{center}
\begin{tabular}{l}
|\ifchilddocmanual|\\
|\input{\childdocname}|\\
|\||else|\\
\textit{document body with }|\input{|\textit{part}|}|\\
|\||fi|
\end{tabular}
\end{center}
%
The conditional |\ifchilddocmanual| is true whenever
a part to be included by |\input| is being compiled,
and the name of the part is stored in |\childdocname|.

%%%%%%%%%%%%%%%%%%%%%%%%%%%%%%%%%%%%%%%%
\DescribeMacro{\childdocby}
Each part to be included by |\input| should start with:
%
\begin{center}
\begin{tabular}{l}
|\input{childdoc.def}|\\
|\childdocby{|\textit{main}|}|\\
\end{tabular}
\end{center}
%
The directive |\childdocby| is similar to |\childdocof|
described in \secref{sec:include},
but the subsequent selection of content must be done manually.
To that end, both |\ifchilddoc| and |\ifchilddocmanual|
will be true upon processing of a part,
and the name of the part is stored in |\childdocname|.
Note that |\jobname| will be set to the filename of the current part
so that each part receives an individual |.aux| file
that does not interfere with the |.aux| file(s) of the main document.
This behaviour can be altered by the alternative form
|\childdocby[*]{|\textit{main}|}| (with a non-empty optional argument)
which uses the |.aux| file of the main document
by setting |\jobname| to \textit{main}.

%%%%%%%%%%%%%%%%%%%%%%%%%%%%%%%%%%%%%%%%%%%%%%%%%%%%%%%%%%%%%%%%%%%%%%%%%%%%%%%%
\subsection{Driver Development}
\label{sec:driver}

The \textsf{childdoc} mechanism can also be use for the development
of definition files such as \LaTeX{} styles or classes.
This case differs from the above setup with multiple parts
included by |\include| in that no |\includeonly| should be invoked.
This can be achieved by starting the include file
(before |\ProvidesPackage|) with:
%
\begin{center}
\begin{tabular}{l}
|\input{childdoc.def}|\\
|\childdocforward{|\textit{main}|}|\\
\end{tabular}
\end{center}
%
or alternatively with:
%
\begin{center}
\begin{tabular}{l}
|\input{childdoc.def}|\\
|\childdocby{|\textit{main}|}|\\
\end{tabular}
\end{center}
%
Both forms have slightly different effects as described above.
The main file is prepared as usual, see \secref{sec:include}.

%%%%%%%%%%%%%%%%%%%%%%%%%%%%%%%%%%%%%%%%%%%%%%%%%%%%%%%%%%%%%%%%%%%%%%%%%%%%%%%%
\subsection{Legacy Detection}
\label{sec:detection}

The directive |\childdocmain| in the main file can detect
whether the complete document or merely a child is to be compiled
even without using the directive |\childdocof|.
This method is deprecated because it is less robust
and there is no compelling reason to use it;
it is merely provided for backward compatibility
and it may be removed in future versions.

If the detection mechanism is to be used,
it is mandatory to correctly specify
the filename of the main file as the argument of |\childdocmain|:
%
\begin{center}
\begin{tabular}{l}
|\input{childdoc.def}|\\
|\childdocmain{|\textit{main}|}|\\
\end{tabular}
\end{center}
%
If |\jobname| does not match the argument \textit{main} of |\childdocmain|,
it is assumed that |\jobname| points to the child file to be compiled.
When using |\childdocmain| with the main file specified as argument,
it suffices to start a child file
with just |\input{|\textit{main}|}|
without loading of the package and using |\childdocof|.
If instead all processing is done
with the appropriate \textsf{childdoc} directives,
the argument of \textit{main} of |\childdocmain| can be empty.

An alternative version of the command line processing described
in \secref{sec:commandline} using the detection mechanism reads:
%
\begin{center}
|... -jobname "|\textit{target}|" "|[\textit{flags}]%
[|\def\jobname{|\textit{dest}|}|]|\input{|\textit{main}|}"|
\end{center}

%%%%%%%%%%%%%%%%%%%%%%%%%%%%%%%%%%%%%%%%%%%%%%%%%%%%%%%%%%%%%%%%%%%%%%%%%%%%%%%%
\subsection{Manual Code}
\label{sec:manual}

In case one cannot be certain whether the definitions file |childdoc.def|
is installed on the target \TeX{} distribution
and one prefers not to ship it,
it is conceivable to paste a few relevant commands into the sources.

To that end, drop all statements |\input{childdoc.def}|
and perform the replacements as outlined below.
Instead of |\childdocmain{|\textit{main}|}| add the following code
to the top of the main file:
%
\begin{center}
\begin{tabular}{l}
|\||ifdefined\childdocname\endinput\||fi\newif\ifchilddoc|\\
|\edef\childdocname{\scantokens\expandafter{\jobname\noexpand}}|\\
|\def\childdocmain{|\textit{main}|}\||ifx\childdocmain\childdocname\||else|\\
|\childdoctrue\includeonly{\childdocname}\let\jobname\childdocmain\||fi|\\
\end{tabular}
\end{center}
%
Instead of |\childdocof{|\textit{main}|}| just include the main file
at the top of each child file:
%
\begin{center}
|\input{|\textit{main}|}|
\end{center}
%
A simple redirection |\childdocforward{|\textit{dest}|}| is achieved by:
%
\begin{center}
|\def\jobname{|\textit{dest}|}\input{\jobname}|
\end{center}
%
The redirection with prefix
|\childdocforwardprefix[|\textit{prefix}|]{|\textit{dest}|}|
is accomplished by:
%
\begin{center}
\begin{tabular}{l}
|{\edef\jobname{\scantokens\expandafter{\jobname\noexpand}}|\\
|\def\redirectjob |\textit{prefix}|#1~~~{\gdef\jobname{|\textit{dest}|#1}}|\\
|\expandafter\redirectjob\jobname~~~}\input{\jobname}|
\end{tabular}
\end{center}

In an alternative approach,
child documents can be compiled by a specific command line
without additional code or specific definitions:
%
\begin{center}
|... -jobname "|\textit{target}|" "|[\textit{flags}]%
|\includeonly{|\textit{dest}|}\input{|\textit{main}|}"|
\end{center}
%

%%%%%%%%%%%%%%%%%%%%%%%%%%%%%%%%%%%%%%%%%%%%%%%%%%%%%%%%%%%%%%%%%%%%%%%%%%%%%%%%
%%%%%%%%%%%%%%%%%%%%%%%%%%%%%%%%%%%%%%%%%%%%%%%%%%%%%%%%%%%%%%%%%%%%%%%%%%%%%%%%
\section{Information}

%%%%%%%%%%%%%%%%%%%%%%%%%%%%%%%%%%%%%%%%%%%%%%%%%%%%%%%%%%%%%%%%%%%%%%%%%%%%%%%%
\subsection{Copyright}

Copyright \copyright{} 2017--2018 Niklas Beisert

This work may be distributed and/or modified under the
conditions of the \LaTeX{} Project Public License, either version 1.3
of this license or (at your option) any later version.
The latest version of this license is in
  \url{http://www.latex-project.org/lppl.txt}
and version 1.3 or later is part of all distributions of \LaTeX{}
version 2005/12/01 or later.

This work has the LPPL maintenance status `maintained'.

The Current Maintainer of this work is Niklas Beisert.

This work consists of the files |README.txt|, |childdoc.ins| and |childdoc.dtx|
as well as the derived files |childdoc.def|, |cdocsamp.tex|
with |cdocsch1.tex|, |cdocsch2.tex|, |cdocspt3.tex|, |cdocspt4.tex|,
|cdocsdrf.tex|, |cdocsfn1.tex|, |cdocsfn2.tex|
as well as |childdoc.pdf|.

%%%%%%%%%%%%%%%%%%%%%%%%%%%%%%%%%%%%%%%%%%%%%%%%%%%%%%%%%%%%%%%%%%%%%%%%%%%%%%%%
\subsection{Files and Installation}

The package consists of the files:
%
\begin{center}
\begin{tabular}{ll}
    |README.txt|   & readme file \\
    |childdoc.ins| & installation file \\
    |childdoc.dtx| & source file \\
    |childdoc.def| & definition file \\
    |cdocsamp.tex| & sample main file \\
    |cdocsch1.tex| & sample include file \\
    |cdocsch2.tex| & sample include file \\
    |cdocspt3.tex| & sample part file \\
    |cdocspt4.tex| & sample part file \\
    |cdocsdrf.tex| & sample redirection file \\
    |cdocsfn1.tex| & sample redirection file \\
    |cdocsfn2.tex| & sample redirection file \\
    |childdoc.pdf| & manual
\end{tabular}
\end{center}
%
The distribution consists of the files
|README.txt|, |childdoc.ins| and |childdoc.dtx|.
%
\begin{itemize}
\item
Run (pdf)\LaTeX{} on |childdoc.dtx|
to compile the manual |childdoc.pdf| (this file).
\item
Run \LaTeX{} on |childdoc.ins| to create the definitions file |childdoc.def|
and the sample |cdocsamp.tex| with include files
|cdocsch1.tex|, |cdocsch2.tex|, |cdocspt3.tex|, |cdocspt4.tex|,
|cdocsdrf.tex|, |cdocsfn1.tex|, |cdocsfn2.tex|.
Then copy the file |childdoc.def| to an appropriate directory of your \LaTeX{}
distribution, e.g.\ \textit{texmf-root}|/tex/latex/childdoc|.
\end{itemize}

%%%%%%%%%%%%%%%%%%%%%%%%%%%%%%%%%%%%%%%%%%%%%%%%%%%%%%%%%%%%%%%%%%%%%%%%%%%%%%%%
\subsection{Related CTAN Packages}

There are several other packages which offer a similar functionality:
%
\begin{itemize}
\item
The packages
\href{http://ctan.org/pkg/docmute}{\textsf{docmute}},
\href{http://ctan.org/pkg/includex}{\textsf{includex}} and
\href{http://ctan.org/pkg/standalone}{\textsf{standalone}}
provide commands to include only the document body of
a child file thus allowing both files to be compiled individually.
\item
The packages \href{http://ctan.org/pkg/subdocs}{\textsf{subdocs}}
and \href{http://ctan.org/pkg/subfiles}{\textsf{subfiles}}
provide structures in which the main and child documents can be
encapsulated and allowing them to be compiled individually.
The inclusion mechanism is different from the conventional |\include|.
\item
The package \href{http://ctan.org/pkg/combine}{\textsf{combine}}
is an elaborate solution to combine several documents into one.
\end{itemize}
%
See also the CTAN topic \href{http://ctan.org/topic/subdocs}{\textsf{subdocs}}
for further related packages.
The present package differs from the above solutions in that
a document structure constructed with the conventional |\include| mechanism
just needs two extra commands at the top of every file
such that all constituent files can be compiled individually.

%%%%%%%%%%%%%%%%%%%%%%%%%%%%%%%%%%%%%%%%%%%%%%%%%%%%%%%%%%%%%%%%%%%%%%%%%%%%%%%%
%\subsection{Feature Suggestions}
%
%The following is a list of features which may be useful for future
%versions of this package:
%%
%\begin{itemize}
%\item
%\ldots
%\end{itemize}

%%%%%%%%%%%%%%%%%%%%%%%%%%%%%%%%%%%%%%%%%%%%%%%%%%%%%%%%%%%%%%%%%%%%%%%%%%%%%%%%
\subsection{Revision History}

%%%%%%%%%%%%%%%%%%%%%%%%%%%%%%%%%%%%%%%%
\paragraph{v2.0:} 2018/12/30

\begin{itemize}
\item
immediate forward processing
\item
added |\childdocby| mechanism
\item
manual restructured
\end{itemize}

%%%%%%%%%%%%%%%%%%%%%%%%%%%%%%%%%%%%%%%%
\paragraph{v1.6:} 2018/01/17

\begin{itemize}
\item
application for development of include files
\item
corrections to manual
\end{itemize}

%%%%%%%%%%%%%%%%%%%%%%%%%%%%%%%%%%%%%%%%
\paragraph{v1.5:} 2017/05/21

\begin{itemize}
\item
more complete structuring introduced
\item
|\childdocof| introduced
\item
|\childdoc| renamed to |\childdocmain|
\item
|\childredirect| renamed to |\childdocforward| and |\childdocforwardprefix|
and functionality expanded
\end{itemize}

%%%%%%%%%%%%%%%%%%%%%%%%%%%%%%%%%%%%%%%%
\paragraph{v1.0:} 2017/04/27

\begin{itemize}
\item
manual and install package
\item
first version published on CTAN
\end{itemize}

%%%%%%%%%%%%%%%%%%%%%%%%%%%%%%%%%%%%%%%%
\paragraph{v0.6:} 2017/04/26

\begin{itemize}
\item
redirection mechanism added
\end{itemize}

%%%%%%%%%%%%%%%%%%%%%%%%%%%%%%%%%%%%%%%%
\paragraph{v0.5:} 2017/04/26

\begin{itemize}
\item
functionality in definition file
\end{itemize}


%%%%%%%%%%%%%%%%%%%%%%%%%%%%%%%%%%%%%%%%%%%%%%%%%%%%%%%%%%%%%%%%%%%%%%%%%%%%%%%%
%%%%%%%%%%%%%%%%%%%%%%%%%%%%%%%%%%%%%%%%%%%%%%%%%%%%%%%%%%%%%%%%%%%%%%%%%%%%%%%%
%%%%%%%%%%%%%%%%%%%%%%%%%%%%%%%%%%%%%%%%%%%%%%%%%%%%%%%%%%%%%%%%%%%%%%%%%%%%%%%%
\appendix

\settowidth\MacroIndent{\rmfamily\scriptsize 000\ }

 \DocInput{childdoc.dtx}

\end{document}
%</driver>
% \fi
%
% %%%%%%%%%%%%%%%%%%%%%%%%%%%%%%%%%%%%%%%%%%%%%%%%%%%%%%%%%%%%%%%%%%%%%%%%%%%%%%
% %%%%%%%%%%%%%%%%%%%%%%%%%%%%%%%%%%%%%%%%%%%%%%%%%%%%%%%%%%%%%%%%%%%%%%%%%%%%%%
% \section{Sample}
%\iffalse
%<*samplemain>
%\fi
%
% The following presents a sample document
% with two chapters, two parts, a title page,
% a compile flag as well as three forwarding files to set the flag.
% It consists of eight |.tex| files:
% \begin{center}
% \begin{tabular}{ll}
% |cdocsamp.tex|&main file\\
% |cdocsch1.tex|&include file for chapter 1\\
% |cdocsch2.tex|&include file for chapter 2\\
% |cdocspt3.tex|&include file for part 3\\
% |cdocspt4.tex|&include file for part 4\\
% |cdocsdrf.tex|&forwarding file for main file in draft mode\\
% |cdocsfi1.tex|&forwarding file for final version of chapter 1\\
% |cdocsfi2.tex|&forwarding file for final version of chapter 2\\
% \end{tabular}
% \end{center}
% Each of the eight files can be compiled directly by the \LaTeX{} compiler.
%
% %%%%%%%%%%%%%%%%%%%%%%%%%%%%%%%%%%%%%%
% \paragraph{Main File.}
%
% The main file is called |cdocsamp.tex|.
%
% Load the \textsf{childdoc} definitions and
% declare the filename for the main document:
%    \begin{macrocode}
\input{childdoc.def}
\childdocmain{}
%    \end{macrocode}

% Optional override for |\version| flag:
%    \begin{macrocode}
%%\ifchilddoc\else\providecommand{\version}{draft}\fi
%    \end{macrocode}

% Define the default values for the |\version| flag
% (|final| for the main file and |draft| for childs):
%    \begin{macrocode}
\ifchilddoc
\providecommand{\version}{draft}
\else
\providecommand{\version}{final}
\fi
%    \end{macrocode}

% Load the standard document class:
%    \begin{macrocode}
\documentclass[12pt]{article}
%    \end{macrocode}

% Start the document body:
%    \begin{macrocode}
\begin{document}
%    \end{macrocode}

% Declare a title page.
% Print title, part of document being processed and version flag:
%    \begin{macrocode}
\addtocounter{page}{-1}
\begin{center}
{\LARGE\bfseries{}childdoc example\par}
\vspace{1cm}
\ifchilddoc
\ifchilddocmanual part\else chapter\fi:
`\childdocname' of `\childdocjob'\par
\else
main document: `\childdocjob'\par
\fi
version: \version\par
\end{center}
\newpage
%    \end{macrocode}

% Manually include selected file,
% otherwise process as usual:
%    \begin{macrocode}
\ifchilddocmanual
\section*{part `\childdocname'}
\input{\childdocname}
\else
%    \end{macrocode}

% Include the two chapters:
%    \begin{macrocode}
\include{cdocsch1}
\include{cdocsch2}
%    \end{macrocode}

% Include the two parts unless only chapters should be displayed:
%    \begin{macrocode}
\ifchilddoc\else
\section{part three}
\input{cdocspt3}
\section{part four}
\input{cdocspt4}
\fi
%    \end{macrocode}

% Process as usual until here:
%    \begin{macrocode}
\fi
%    \end{macrocode}

% End of document body:
%    \begin{macrocode}
\end{document}
%    \end{macrocode}
%\iffalse
%</samplemain>
%\fi
%
% %%%%%%%%%%%%%%%%%%%%%%%%%%%%%%%%%%%%%%
% \paragraph{Chapter Include Files.}
%
% The include files are called |cdocsch1.tex| and |cdocsch2.tex|.
%
%\iffalse
%<*samplechap1|samplechap2>
%\fi

% Optional override for |\version| flag:
%    \begin{macrocode}
%%\providecommand{\version}{final}
%    \end{macrocode}

% Include the main document:
%    \begin{macrocode}
\input{childdoc.def}
\childdocof{cdocsamp}
%    \end{macrocode}

%\iffalse
%</samplechap1|samplechap2>
%\fi
%
%\iffalse
%<*samplechap1>
%\fi
% Some text for chapter 1:
%    \begin{macrocode}
\section{one}
some text in chapter one
%    \end{macrocode}

%\iffalse
%</samplechap1>
%\fi
% Some text for chapter 2:
%\iffalse
%<*samplechap2>
%\fi
%    \begin{macrocode}
\section{two}
more text in chapter two
%    \end{macrocode}

%\iffalse
%</samplechap2>
%\fi
%
% %%%%%%%%%%%%%%%%%%%%%%%%%%%%%%%%%%%%%%
% \paragraph{Part Include Files.}
%
% The include files are called |cdocspt3.tex| and |cdocspt4.tex|.
%
%\iffalse
%<*samplepart3|samplepart4>
%\fi

% Optional override for |\version| flag:
%    \begin{macrocode}
%%\providecommand{\version}{final}
%    \end{macrocode}

% Include the main document:
%    \begin{macrocode}
\input{childdoc.def}
\childdocby{cdocsamp}
%    \end{macrocode}

%\iffalse
%</samplepart3|samplepart4>
%\fi
%
%\iffalse
%<*samplepart3>
%\fi
% Some text for part 3:
%    \begin{macrocode}
some text in part three
%    \end{macrocode}

%\iffalse
%</samplepart3>
%\fi
% Some text for part 4:
%\iffalse
%<*samplepart4>
%\fi
%    \begin{macrocode}
more text in part four
%    \end{macrocode}

%\iffalse
%</samplepart4>
%\fi
%
% %%%%%%%%%%%%%%%%%%%%%%%%%%%%%%%%%%%%%%
% \paragraph{Forwarding for a Complete Draft.}
%
% The following forwarding file |cdocsdrf.tex|
% compiles the main document in draft mode:
%\iffalse
%<*sampledraft>
%\fi
%    \begin{macrocode}
\def\version{draft}
\input{childdoc.def}
\childdocforward{cdocsamp}
%    \end{macrocode}

%\iffalse
%</sampledraft>
%\fi
%
% %%%%%%%%%%%%%%%%%%%%%%%%%%%%%%%%%%%%%%
% \paragraph{Forwarding for Final Version of the Chapters.}
%
% The following forwarding files |cdocsfn1.tex| and |cdocsfn2.tex|
% (with identical content)
% compile the final versions of the child documents
% |cdocsch1.tex| and |cdocsch2.tex|, respectively:
%\iffalse
%<*samplefinal>
%\fi
%    \begin{macrocode}
\def\version{final}
\input{childdoc.def}
\childdocforwardprefix[cdocsamp]{cdocsfn}{cdocsch}
%    \end{macrocode}

%\iffalse
%</samplefinal>
%\fi
%
% %%%%%%%%%%%%%%%%%%%%%%%%%%%%%%%%%%%%%%
% \paragraph{Command Line Processing.}
%
% The following three command lines generate the output files
% |cdocscld|, |cdocscl1| and |cdocscl2|
% which should be identical to
% |cdocsdrf|, |cdocsch1| and |cdocsfn2|, respectively:
% \begin{center}
% \begin{tabular}{l}
% |latex -jobname cdocscld \|\\
% |  "\def\version{draft}\input{childdoc.def}\childdocforward{cdocsamp}"|\\
% |latex -jobname cdocscl1 \|\\
% |  "\input{childdoc.def}\childdocforward[cdocsamp]{cdocsch1}"|\\
% |latex -jobname cdocscl2 \|\\
% |  "\def\version{final}\input{childdoc.def}\childdocforward{cdocsch2}"|
% \end{tabular}
% \end{center}
% Note that the trailing backslash on each first line
% merely continues the input to the second line
% (for convenient cut ant paste).
% Furthermore, the command |latex| can be replaced by any
% of its alternative versions such as |pdflatex|.
%
% %%%%%%%%%%%%%%%%%%%%%%%%%%%%%%%%%%%%%%%%%%%%%%%%%%%%%%%%%%%%%%%%%%%%%%%%%%%%%%
% %%%%%%%%%%%%%%%%%%%%%%%%%%%%%%%%%%%%%%%%%%%%%%%%%%%%%%%%%%%%%%%%%%%%%%%%%%%%%%
% \section{Implementation}
%\iffalse
%<*package>
%\fi
%
% This section describes the definitions file |childdoc.def|.

% The definitions cannot be loaded using |\usepackage| or |\RequirePackage|
% which has a mechanism to prevent loading a style file more than once.
% When loading the definitions by means of |\input|
% multiple instances have to be prevented manually:
%\iffalse
%This code needs to be before the `\ProvidesFile' directive
%which is defined at the beginning of this file.
%Therefore it is also placed there and commented out here.
%</package>
%<*discard>
%\fi
%    \begin{macrocode}
\ifdefined\childdocmain\endinput\fi
%    \end{macrocode}
%\iffalse
%</discard>
%<*package>
%\fi
%
% \macro{\ifchilddoc}
% \macro{\ifchilddocmanual}
% The conditional |\ifchilddoc| tells whether a
% child (true) or main (false) document is being compiled.
% The conditional |\ifchilddocmanual| tells whether
% the |\includeonly| mechanism is used (false) or
% the selection of child files must be performed manually (true).
% The definitions initialise to false:
%    \begin{macrocode}
\newif\ifchilddoc
\newif\ifchilddocmanual
%    \end{macrocode}

% \macro{\childdocname}
% \macro{\childdocjob}
% The macro |\childdocname| stores the name of the main document
% to be compiled. The macro |\childdocjob| stores the name of
% the document on which the \LaTeX{} compiler was originally invoked.
% The content of |\jobname| cannot be compared
% to filenames specified in the source due to different catcodes.
% The following code rescans |\jobname|, stores the result
% in |\childdocname| and saves a copy in |\childdocjob|:
%    \begin{macrocode}
\edef\childdocname{\scantokens\expandafter{\jobname\noexpand}}
\let\childdocjob\childdocname
%    \end{macrocode}

% \macro{\childdocdisable}
% The macro |\childdocdisable| prevents the main file
% from being processed more than once.
% At this stage, the main document command |\childdocmain|
% is assumed to be called once again where it should do nothing.
% Any subsequent call to it should prevent
% a secondary processing of the main document
% It overwrites the forwarding commands
% |\childdocof| and |\childdocforward|
% with empty macros to prevent further inclusions of the main document:
%    \begin{macrocode}
\newcommand{\childdocdisable}
{
  \renewcommand{\childdocmain}[1]{\renewcommand{\childdocmain}[1]{\endinput}}
  \renewcommand{\childdocof}[1]{}
  \renewcommand{\childdocby}[2][]{}
  \renewcommand{\childdocforward}[2][]{}
  \renewcommand{\childdocdisable}{}
}
%    \end{macrocode}

% \macro{\childdocmain}
% The macro |\childdocmain| is to be called at the top of the main file
% with nothing or the main filename (without extension) as argument.
% First, it breaks loops.
% If the argument is not empty and does not match |\childdocname|
% (which is set by the first inclusion of |childdoc.def|),
% |\ifchilddoc| is set to true, |\includeonly| is applied to the child file
% and |\jobname| is set to the main file
% (for proper handling of |.aux| files):
%    \begin{macrocode}
\newcommand{\childdocmain}[1]
{
  \childdocdisable\childdocmain{}
  \if?#1?\else
    \begingroup
      \def\childdoctmp{#1}
      \ifx\childdoctmp\childdocname
        \def\childdoctmp{}
      \else
        \def\childdoctmp
        {
          \childdoctrue
          \includeonly{\childdocname}
          \def\childdocjob{#1}
          \def\jobname{#1}
        }
      \fi
      \expandafter
    \endgroup
    \childdoctmp
  \fi
}
%    \end{macrocode}

% \macro{\childdocof}
% The command |\childdocof| redirects
% compilation to the main file |#1|.
%    \begin{macrocode}
\newcommand{\childdocof}[1]
{
  \childdocdisable
  \childdoctrue
  \includeonly{\childdocname}
  \def\jobname{#1}
  \def\childdocjob{#1}
  \input{#1}
}
%    \end{macrocode}

% \macro{\childdocby}
% The command |\childdocby| ....
%    \begin{macrocode}
\newcommand{\childdocby}[2][]
{
  \childdocdisable
  \childdoctrue
  \childdocmanualtrue
  \if?#1?\else
    \def\jobname{#2}
  \fi
  \def\childdocjob{#2}
  \input{#2}
  \endinput
}
%    \end{macrocode}

% \macro{\childdocforward}
% The command |\childdocforward| redirects
% compilation to the main file or
% (if the optional argument is given) a child file.
% Parameters are set as if the main file
% or a child file starting with |\childdocof| was compiled.
% Then compilation is handed over to the main file:
%    \begin{macrocode}
\newcommand{\childdocforward}[2][]
{
  \begingroup
    \if?#1?
      \def\childdoctmp
      {
        \def\childdocname{#2}
        \def\childdocjob{#2}
        \def\jobname{#2}
        \input{#2}
        \endinput
      }
    \else
      \def\childdoctmp
      {
        \childdocdisable
        \def\childdocname{#2}
        \childdoctrue
        \includeonly{#2}
        \def\childdocjob{#1}
        \def\jobname{#1}
        \input{#1}
        \endinput
      }
    \fi
    \expandafter
  \endgroup
  \childdoctmp
}
%    \end{macrocode}

% \macro{\childdocforwardprefix}
% The command |\childdocforwardprefix| redirects
% compilation to the main or a child file by means of a pattern.
% The prefix |#1| in the current filename is replaced by |#2|
% and the suffix of the current filename is kept
% (it is assumed that the filename does not contain the substring `|~~~|'
% which is used as a delimiter).
% Compilation is handed over to the new file by |\childdocforward|:
%    \begin{macrocode}
\newcommand{\childdocforwardprefix}[3][]
{
  \begingroup
    \def\childdocextract #2##1~~~{\def\childdoctmp{\childdocforward[#1]{#3##1}}}
    \expandafter\childdocextract\childdocname~~~
    \expandafter
  \endgroup
  \childdoctmp
}
%    \end{macrocode}

% \macro{\childdoc}
% The deprecated macro |\childdoc| is a legacy version of |\childdocmain|:
%    \begin{macrocode}
\newcommand{\childdoc}{\childdocmain}
%    \end{macrocode}

% \macro{\childdocredirect}
% The deprecated macro |\childdocredirect| is a legacy version
% of |\childdocforward| and |\childdocforwardprefix|:
%    \begin{macrocode}
\newcommand{\childdocredirect}[2][]
{
  \begingroup
    \if?#1?
      \def\childdoctmp{\childdocforward{#2}}
    \else
      \def\childdoctmp{\childdocforwardprefix{#1}{#2}}
    \fi
    \expandafter
  \endgroup
  \childdoctmp
}
%    \end{macrocode}

%\iffalse
%</package>
%\fi
%
\endinput
|\\
|\childdocmain{}|\\
\end{tabular}
\end{center}
at the very top of the main \LaTeX{} file,
in particular \emph{before} the |\documentclass| statement!
The argument of |\childdocmain| should be left empty
(but it must be present).

%%%%%%%%%%%%%%%%%%%%%%%%%%%%%%%%%%%%%%%%
\DescribeMacro{\childdocof}
Furthermore, add the commands
\begin{center}
\begin{tabular}{l}
|% \iffalse
%
% childdoc.dtx Copyright (C) 2017-2018 Niklas Beisert
%
% This work may be distributed and/or modified under the
% conditions of the LaTeX Project Public License, either version 1.3
% of this license or (at your option) any later version.
% The latest version of this license is in
%   http://www.latex-project.org/lppl.txt
% and version 1.3 or later is part of all distributions of LaTeX
% version 2005/12/01 or later.
%
% This work has the LPPL maintenance status `maintained'.
%
% The Current Maintainer of this work is Niklas Beisert.
%
% This work consists of the files childdoc.dtx and childdoc.ins
% and the derived files childdoc.def and cdocsamp.tex with
% cdocsch1.tex, cdocsch2.tex, cdocsdrf.tex, cdocsfn1.tex, cdocsfn2.tex.
%
%<package>\ifdefined\childdocmain\endinput\fi
%<package>\ProvidesFile{childdoc.def}[2018/12/30 v2.0 child document driver]
%<samplemain>\ProvidesFile{cdocsamp.tex}[2018/12/30 v2.0 sample for childdoc]
%<*driver>
%\ProvidesFile{childdoc.drv}[2018/12/30 v2.0 childdoc reference manual file]
\PassOptionsToClass{10pt,a4paper}{article}
\documentclass{ltxdoc}

\usepackage[margin=35mm]{geometry}
\usepackage{hyperref}
\usepackage{hyperxmp}
\usepackage[usenames]{color}

\hypersetup{colorlinks=true}
\hypersetup{pdfstartview=FitH}
\hypersetup{pdfpagemode=UseNone}
\hypersetup{pdfsource={}}
\hypersetup{pdflang={en-UK}}
\hypersetup{pdfcopyright={Copyright 2017-2018 Niklas Beisert.
  This work may be distributed and/or modified under the
  conditions of the LaTeX Project Public License, either version 1.3
  of this license or (at your option) any later version.}}
\hypersetup{pdflicenseurl={http://www.latex-project.org/lppl.txt}}
\hypersetup{pdfcontactaddress={ETH Zurich, ITP, HIT K,
  Wolfgang-Pauli-Strasse 27}}
\hypersetup{pdfcontactpostcode={8093}}
\hypersetup{pdfcontactcity={Zurich}}
\hypersetup{pdfcontactcountry={Switzerland}}
\hypersetup{pdfcontactemail={nbeisert@itp.phys.ethz.ch}}
\hypersetup{pdfcontacturl={http://people.phys.ethz.ch/\xmptilde nbeisert/}}

\newcommand{\secref}[1]{\hyperref[#1]{section \ref*{#1}}}

\parskip1ex
\parindent0pt
\let\olditemize\itemize
\def\itemize{\olditemize\parskip0pt}

\begin{document}

\title{The \textsf{childdoc} Package}
\hypersetup{pdftitle={The childdoc Package}}
\author{Niklas Beisert\\[2ex]
  Institut f\"ur Theoretische Physik\\
  Eidgen\"ossische Technische Hochschule Z\"urich\\
  Wolfgang-Pauli-Strasse 27, 8093 Z\"urich, Switzerland\\[1ex]
  \href{mailto:nbeisert@itp.phys.ethz.ch}
  {\texttt{nbeisert@itp.phys.ethz.ch}}}
\hypersetup{pdfauthor={Niklas Beisert}}
\hypersetup{pdfsubject={Manual for the LaTeX2e Package childdoc}}
\date{30 December 2018, \textsf{v2.0}}
\maketitle

\begin{abstract}\noindent
\textsf{childdoc} is a \LaTeXe{} package
that enables the direct compilation
of document sections included by |\include|
to individual files.
\end{abstract}

\begingroup
\parskip0ex
\tableofcontents
\endgroup

%%%%%%%%%%%%%%%%%%%%%%%%%%%%%%%%%%%%%%%%%%%%%%%%%%%%%%%%%%%%%%%%%%%%%%%%%%%%%%%%
%%%%%%%%%%%%%%%%%%%%%%%%%%%%%%%%%%%%%%%%%%%%%%%%%%%%%%%%%%%%%%%%%%%%%%%%%%%%%%%%
\section{Introduction}

\LaTeX{} provides a mechanism to structure a large document (such as a book)
into a main file and several child files (containing the chapters)
using the |\include| command.
This mechanism is beneficial for documents
which span hundreds of pages in order to
make the source file(s) more manageable.
Moreover, compilation can be restricted to
selected child files by means of the |\includeonly| command.
The latter feature can be used to reduce the compilation time while editing
(this was significantly more useful in the earlier days of \LaTeX{})
or to generate a smaller document which is easier to navigate.
Another application of |\includeonly| is to generate
documents consisting of selected parts of the complete document.

However, there are a few drawbacks of the plain |\include| mechanism:
\begin{itemize}
\item
The child files cannot be compiled on their own,
they can only be compiled via the main file.
A naive editing environment
(such as a text editor with an option
to have the current file processed by \LaTeX)
may require one to switch to the main file before compiling;
attempting to compile the child file produces errors.
\item
The main file must be modified (each time)
to adjust the |\includeonly| command
to the present needs. This easily leaves the main file in a messy state.
\item
The generated document will always carry the filename
of the main document. This is inconvenient if
several child files are to be compiled and
to be kept for distribution.
\end{itemize}

The present package provides a simple interface
to make child files individually compilable by \LaTeX{}.
Compiling a child file then has the same effect as compiling
the main file with an |\includeonly| command
to select the appropriate child.
Moreover the generated document will carry the name of the child
rather than the main file.
This resolves all three above issues.

This feature is meant to make the editing of books,
thesis documents and lecture notes somewhat more convenient.
However, the package can also be used efficiently for
composing a series of documents (such as exercise sheets)
which are typically distributed individually.
It then assists the author in generating the individual documents
(potentially in different versions)
as well as a document containing the collected series.
Another application is in developing style files
or other kinds of included material
where compilation of the style file could redirect
to a sample or test file.

%%%%%%%%%%%%%%%%%%%%%%%%%%%%%%%%%%%%%%%%%%%%%%%%%%%%%%%%%%%%%%%%%%%%%%%%%%%%%%%%
%%%%%%%%%%%%%%%%%%%%%%%%%%%%%%%%%%%%%%%%%%%%%%%%%%%%%%%%%%%%%%%%%%%%%%%%%%%%%%%%
\section{Usage}

First of all, the package \textsf{childdoc} is \emph{not} a standard
\LaTeXe{} |.sty| style file! Therefore it needs to be invoked in
a non-standard way.

%%%%%%%%%%%%%%%%%%%%%%%%%%%%%%%%%%%%%%%%%%%%%%%%%%%%%%%%%%%%%%%%%%%%%%%%%%%%%%%%
\subsection{Included Files}
\label{sec:include}

%%%%%%%%%%%%%%%%%%%%%%%%%%%%%%%%%%%%%%%%
\DescribeMacro{\childdocmain}
To use the package, add the commands
\begin{center}
\begin{tabular}{l}
|\input{childdoc.def}|\\
|\childdocmain{}|\\
\end{tabular}
\end{center}
at the very top of the main \LaTeX{} file,
in particular \emph{before} the |\documentclass| statement!
The argument of |\childdocmain| should be left empty
(but it must be present).

%%%%%%%%%%%%%%%%%%%%%%%%%%%%%%%%%%%%%%%%
\DescribeMacro{\childdocof}
Furthermore, add the commands
\begin{center}
\begin{tabular}{l}
|\input{childdoc.def}|\\
|\childdocof{|\textit{main}|}|\\
\end{tabular}
\end{center}
at the top of every child file \textit{child}
which is included by |\include{|\textit{child}|}|
from within the main file
(or at least for those files to be compiled individually).
The argument \textit{main} must be the filename of the main file.

There are a couple of
considerations in setting up the main and child documents:

%%%%%%%%%%%%%%%%%%%%%%%%%%%%%%%%%%%%%%%%
\paragraph{Restrictions.}

Please note the following restrictions:
\begin{itemize}
\item
|\childdocmain| must be called with one argument \textit{main}
to ensure compatibility with earlier version of the package.
It must either be empty (|\childdocmain{}|)
or precisely match the filename of the main file in which it is specified.
See \secref{sec:detection} for further information.
\item
The filename \textit{main} must be specified without the |.tex| extension.
\item
The filename \textit{main} is case sensitive
(even in case-insensitive file systems)
due to internal string comparison.
\item
The argument \textit{main} should be fully expanded, it cannot be a macro.
\item
Subdirectories and special characters should be avoided in filenames.
\item
The command |\childdocmain{|\textit{main}|}| must be followed by a whitespace.
It should not be followed immediately by another command
or by a comment mark `|%|'.
This is because the \TeX{} parser reads the token immediately following
the argument of |\childdocmain| and puts it
at the beginning of every child section;
however, a white\-space is ignored.
\end{itemize}

%%%%%%%%%%%%%%%%%%%%%%%%%%%%%%%%%%%%%%%%
\paragraph{Content of Main File.}

It is advisable to place all content in the child files included by |\include|.
Any output contained in the main file will appear in all child documents
unless suppressed manually;
it cannot be suppressed automatically by the |\includeonly| directive
and thus should normally be avoided.
A method to include some content in the main file
by means of conditional processing is described in \secref{sec:conditional}.

%%%%%%%%%%%%%%%%%%%%%%%%%%%%%%%%%%%%%%%%
\paragraph{Page Numbering.}

When only a part of the document is compiled,
the appropriate numbering of pages
(as well as other status parameters)
is determined from the |.aux| files.
The latter contain information from previous passes.
However this information needs to propagate through
all intermediate child documents.
Therefore the page numbering in child documents may well
be inconsistent until the complete document is compiled at least once.

A useful (if unconventional) way to always ensure a consistent
page numbering is to restart the numbering in each child document
and denote the pages by `\textit{child}|.|\textit{page}'
where \textit{child} represents the chapter/section number of the child file.
This can be achieved by the command
|\numberwithin{page}{|\textit{child}|}|
of the \textsf{amsmath} package
where \textit{child} can be |chapter| or |section|
depending on the chosen structuring.
Alternatively, one can modify the macro |\thepage| appropriately
and reset the counter |page| at the start of each child file.

%%%%%%%%%%%%%%%%%%%%%%%%%%%%%%%%%%%%%%%%%%%%%%%%%%%%%%%%%%%%%%%%%%%%%%%%%%%%%%%%
\subsection{Conditional Processing}
\label{sec:conditional}

The package provides a mechanism to compile different versions
of a document. To customise the versions further some conditional processing
can come in handy to distinguish which version is being compiled.
The package provides two macros to describe the compilation context:

%%%%%%%%%%%%%%%%%%%%%%%%%%%%%%%%%%%%%%%%
\DescribeMacro{\ifchilddoc}
The conditional |\ifchilddoc| distinguishes between the compilation of
child documents and the main document:
%
\begin{center}
|\ifchilddoc |\textit{child-code}| |[|\||else |\textit{main-code}]| \||fi|
\end{center}

%%%%%%%%%%%%%%%%%%%%%%%%%%%%%%%%%%%%%%%%
\DescribeMacro{\childdocname}
\DescribeMacro{\childdocjob}
The macro |\childdocname| contains the filename (without extension)
of the main or child file being processed.
Note that |\childdocjob| will always contain the name of the main file.

%%%%%%%%%%%%%%%%%%%%%%%%%%%%%%%%%%%%%%%%
\paragraph{Title Page.}

Conditional processing can be used to include a title or banner page
in the main document when proper precautions are taken.
Importantly, the code in the main file should ensure that the page counter
(as well as other status parameters which are stored in the |.aux| files)
takes the same value after the conditional processing.
Otherwise the page numbers may take divergent values
depending on which part is compiled.

For example, a title page could be declared by:
%
\begin{center}
\begin{tabular}{l}
|\ifchilddoc\||else|\\
|\addtocounter{page}{-1}|\\
\textit{code for title page}\\
|\newpage|\\
|\||fi|
\end{tabular}
\end{center}
%
A banner page for the child documents can be generated by:
%
\begin{center}
\begin{tabular}{l}
|\ifchilddoc|\\
|\addtocounter{page}{-1}|\\
\textit{code for banner page}\\
|\newpage|\\
|\||fi|
\end{tabular}
\end{center}
%
Here one could write a message such as:
\begin{center}
|This is the part \childdocname{} of \childdocjob{}.|
\end{center}

%%%%%%%%%%%%%%%%%%%%%%%%%%%%%%%%%%%%%%%%%%%%%%%%%%%%%%%%%%%%%%%%%%%%%%%%%%%%%%%%
\subsection{Flags}
\label{sec:flags}

The package makes it easy to generate different versions
of the main or child documents.
To this end compilation flags can be defined
and assigned different default values.
They will be particularly useful in conjunction
with the forwarding mechanism described in \secref{sec:forward}.

For example, it may be useful to have a flag |\version|
which can be set to |draft| or |final|.
The document source will contain some conditional code
depending on the value of |\version|.
Suppose further, the flag should default to |final| for the main file
and to |draft| for child files
which is a natural assignment for editing the document.
This is achieved by placing the following code
in the preamble of the main document
(below the |\childdocmain| directive):
%
\begin{center}
\begin{tabular}{l}
|\ifchilddoc|\\
|\providecommand{\version}{draft}|\\
|\||else|\\
|\providecommand{\version}{final}|\\
|\||fi|
\end{tabular}
\end{center}
%
The definition by |\providecommand| makes sure
that previous definitions are not overwritten.
Further statements |\providecommand{\version}{...}|
can thus be added before the above code to override it.

For the main file, one might add a line
(between |\childdocmain| and the above block)
%
\begin{center}
|%\ifchilddoc\||else\providecommand{\version}{draft}\||fi|
\end{center}
%
which can be uncommented to produce a draft version.
Likewise one can add a line to the very top of a child file
(above the |\childdocof{|\textit{main}|}| directive)
%
\begin{center}
|%\providecommand{\version}{final}|
\end{center}
%
which can be uncommented to produce the final version of this child document.

%%%%%%%%%%%%%%%%%%%%%%%%%%%%%%%%%%%%%%%%%%%%%%%%%%%%%%%%%%%%%%%%%%%%%%%%%%%%%%%%
\subsection{Forwarding}
\label{sec:forward}

Different versions of the main or child documents
using compilation flags as described in \secref{sec:flags}
can be (permanently) stored in different files
for convenient compilation, viewing and distribution.
To this end, the package defines a command
to pass on compilation to a different file:

%%%%%%%%%%%%%%%%%%%%%%%%%%%%%%%%%%%%%%%%
\DescribeMacro{\childdocforward}
The command |\childdocforward| redirects processing to
another source file:
%
\begin{center}
\begin{tabular}{l}
|\input{childdoc.def}|\\
|\childdocforward[|\textit{main}|]{|\textit{dest}|}|\\
\end{tabular}
\end{center}
%
The argument \textit{dest} is the destination file
(without extension).
It should be the main file or one of the child files.
Note that further \textsf{childdoc} directives
such as |\childdocof| and |\childdocforward|
in the indicated file will be processed in this form.
The optional argument \textit{main}
passes on directly to the main file \textit{main}
while pretending to compile the child \textit{dest}.
This form behaves as if \textit{dest}
issues |\childdocof{|\textit{main}|}| right away,
and no further \textsf{childdoc} directives will be processed.

%%%%%%%%%%%%%%%%%%%%%%%%%%%%%%%%%%%%%%%%
\DescribeMacro{\...prefix}
In the alternative form |\childdocforwardprefix|,
%
\begin{center}
\begin{tabular}{l}
|\input{childdoc.def}|\\
|\childdocforwardprefix[|\textit{main}|]{|\textit{prefix}|}{|\textit{dest}|}|
\end{tabular}
\end{center}
%
the destination file is determined by a pattern
depending on the current file:
To make this work, the current file must be called
`{\textit{prefix}\hspace{0.2em}\textit{suffix}}'
with \textit{prefix} matching precisely the argument.
Processing is then passed on to the file
`{\textit{dest}\hspace{0.2em}\textit{suffix}}'.
Surely, the same effect is achieved by
directly specifying the
argument `{\textit{dest}\hspace{0.2em}\textit{suffix}}'
in the first form.
However, that requires to set up a different file
for each child. With the alternative form of the command
all these files can have exactly the same content
which simplifies setting them up and maintaining them.

For example, the following file |draft.tex|
with a compilation flag |\version| as described in \secref{sec:flags}
compiles the main document as a draft:
%
\begin{center}
\begin{tabular}{l}
|\def\version{draft}|\\
|\input{childdoc.def}|\\
|\childdocforward{|\textit{main}|}|
\end{tabular}
\end{center}
%
Likewise, the following files |final|\textit{nn}|.tex|
compile the final version of the child document
|child|\textit{nn}|.tex|:
%
\begin{center}
\begin{tabular}{l}
|\def\version{final}|\\
|\input{childdoc.def}|\\
|\childdocforwardprefix{final}{child}|
\end{tabular}
\end{center}
%

Note that when several versions of a main file and/or of each child file
are to be generated, it may be convenient to set up a |Makefile| or
shell script to automatise the process.

%%%%%%%%%%%%%%%%%%%%%%%%%%%%%%%%%%%%%%%%%%%%%%%%%%%%%%%%%%%%%%%%%%%%%%%%%%%%%%%%
\subsection{Command Line Processing}
\label{sec:commandline}

The effect of redirection files can also be achieved by invoking
the \LaTeX{} compiler with a more elaborate command line.
Most conveniently this should be done as part
of a shell script or a |Makefile|.

When using \textsf{childdoc} in the main file, the following
command lines effectively perform a redirection
(note that depending on the shell being used,
backslashes may have to be doubled: `|\|' $\to$ `|\\|'):
%
\begin{center}
|... -jobname "|\textit{target}|" |\\|"|[\textit{flags}]%
|\input{childdoc.def}\childdocforward[|\textit{main}|]{|\textit{dest}|}"|
\end{center}
%
Here \textit{target} is the name of the output file,
\textit{main} is the name of the main file
and \textit{dest} is the name of the main or child file to be processed
(all filenames without extensions).
The optional argument \textit{main} can be omitted
if \textit{main} matches \textit{dest}.
Optionally, compilation \textit{flags} can be defined via |\def| commands.
This command line makes the \TeX{} engine believe
it is compiling the file \textit{target}
whose content is specified as the latter parameter.
The provided code then forwards the processing to
\textit{main} or \textit{dest} as described in \secref{sec:forward}.

%%%%%%%%%%%%%%%%%%%%%%%%%%%%%%%%%%%%%%%%%%%%%%%%%%%%%%%%%%%%%%%%%%%%%%%%%%%%%%%%
\subsection{Include by Input}
\label{sec:input}

Including child documents by |\include| has some restrictions by design.
Most notably, the content of a child document always occupies
its own set of pages; pages cannot be shared between child documents.
Usually, this behaviour makes perfect sense
because each child document contain an essential part of the document.
However, in some situations it may be desirable to compose
a document from a collection of parts
without having mandatory page breaks between then.
For this case, the package
provides a mechanism to include parts
by |\input| which can also be processed individually.
However, by construction this mechanism
requires manual handling of the content to be output.

%%%%%%%%%%%%%%%%%%%%%%%%%%%%%%%%%%%%%%%%
\DescribeMacro{\ifchilddocmanual}
The main file should be prepared as usual, see \secref{sec:include}.
However, the document body must make a distinction
between processing of an individual part and of the main document, e.g.:
%
\begin{center}
\begin{tabular}{l}
|\ifchilddocmanual|\\
|\input{\childdocname}|\\
|\||else|\\
\textit{document body with }|\input{|\textit{part}|}|\\
|\||fi|
\end{tabular}
\end{center}
%
The conditional |\ifchilddocmanual| is true whenever
a part to be included by |\input| is being compiled,
and the name of the part is stored in |\childdocname|.

%%%%%%%%%%%%%%%%%%%%%%%%%%%%%%%%%%%%%%%%
\DescribeMacro{\childdocby}
Each part to be included by |\input| should start with:
%
\begin{center}
\begin{tabular}{l}
|\input{childdoc.def}|\\
|\childdocby{|\textit{main}|}|\\
\end{tabular}
\end{center}
%
The directive |\childdocby| is similar to |\childdocof|
described in \secref{sec:include},
but the subsequent selection of content must be done manually.
To that end, both |\ifchilddoc| and |\ifchilddocmanual|
will be true upon processing of a part,
and the name of the part is stored in |\childdocname|.
Note that |\jobname| will be set to the filename of the current part
so that each part receives an individual |.aux| file
that does not interfere with the |.aux| file(s) of the main document.
This behaviour can be altered by the alternative form
|\childdocby[*]{|\textit{main}|}| (with a non-empty optional argument)
which uses the |.aux| file of the main document
by setting |\jobname| to \textit{main}.

%%%%%%%%%%%%%%%%%%%%%%%%%%%%%%%%%%%%%%%%%%%%%%%%%%%%%%%%%%%%%%%%%%%%%%%%%%%%%%%%
\subsection{Driver Development}
\label{sec:driver}

The \textsf{childdoc} mechanism can also be use for the development
of definition files such as \LaTeX{} styles or classes.
This case differs from the above setup with multiple parts
included by |\include| in that no |\includeonly| should be invoked.
This can be achieved by starting the include file
(before |\ProvidesPackage|) with:
%
\begin{center}
\begin{tabular}{l}
|\input{childdoc.def}|\\
|\childdocforward{|\textit{main}|}|\\
\end{tabular}
\end{center}
%
or alternatively with:
%
\begin{center}
\begin{tabular}{l}
|\input{childdoc.def}|\\
|\childdocby{|\textit{main}|}|\\
\end{tabular}
\end{center}
%
Both forms have slightly different effects as described above.
The main file is prepared as usual, see \secref{sec:include}.

%%%%%%%%%%%%%%%%%%%%%%%%%%%%%%%%%%%%%%%%%%%%%%%%%%%%%%%%%%%%%%%%%%%%%%%%%%%%%%%%
\subsection{Legacy Detection}
\label{sec:detection}

The directive |\childdocmain| in the main file can detect
whether the complete document or merely a child is to be compiled
even without using the directive |\childdocof|.
This method is deprecated because it is less robust
and there is no compelling reason to use it;
it is merely provided for backward compatibility
and it may be removed in future versions.

If the detection mechanism is to be used,
it is mandatory to correctly specify
the filename of the main file as the argument of |\childdocmain|:
%
\begin{center}
\begin{tabular}{l}
|\input{childdoc.def}|\\
|\childdocmain{|\textit{main}|}|\\
\end{tabular}
\end{center}
%
If |\jobname| does not match the argument \textit{main} of |\childdocmain|,
it is assumed that |\jobname| points to the child file to be compiled.
When using |\childdocmain| with the main file specified as argument,
it suffices to start a child file
with just |\input{|\textit{main}|}|
without loading of the package and using |\childdocof|.
If instead all processing is done
with the appropriate \textsf{childdoc} directives,
the argument of \textit{main} of |\childdocmain| can be empty.

An alternative version of the command line processing described
in \secref{sec:commandline} using the detection mechanism reads:
%
\begin{center}
|... -jobname "|\textit{target}|" "|[\textit{flags}]%
[|\def\jobname{|\textit{dest}|}|]|\input{|\textit{main}|}"|
\end{center}

%%%%%%%%%%%%%%%%%%%%%%%%%%%%%%%%%%%%%%%%%%%%%%%%%%%%%%%%%%%%%%%%%%%%%%%%%%%%%%%%
\subsection{Manual Code}
\label{sec:manual}

In case one cannot be certain whether the definitions file |childdoc.def|
is installed on the target \TeX{} distribution
and one prefers not to ship it,
it is conceivable to paste a few relevant commands into the sources.

To that end, drop all statements |\input{childdoc.def}|
and perform the replacements as outlined below.
Instead of |\childdocmain{|\textit{main}|}| add the following code
to the top of the main file:
%
\begin{center}
\begin{tabular}{l}
|\||ifdefined\childdocname\endinput\||fi\newif\ifchilddoc|\\
|\edef\childdocname{\scantokens\expandafter{\jobname\noexpand}}|\\
|\def\childdocmain{|\textit{main}|}\||ifx\childdocmain\childdocname\||else|\\
|\childdoctrue\includeonly{\childdocname}\let\jobname\childdocmain\||fi|\\
\end{tabular}
\end{center}
%
Instead of |\childdocof{|\textit{main}|}| just include the main file
at the top of each child file:
%
\begin{center}
|\input{|\textit{main}|}|
\end{center}
%
A simple redirection |\childdocforward{|\textit{dest}|}| is achieved by:
%
\begin{center}
|\def\jobname{|\textit{dest}|}\input{\jobname}|
\end{center}
%
The redirection with prefix
|\childdocforwardprefix[|\textit{prefix}|]{|\textit{dest}|}|
is accomplished by:
%
\begin{center}
\begin{tabular}{l}
|{\edef\jobname{\scantokens\expandafter{\jobname\noexpand}}|\\
|\def\redirectjob |\textit{prefix}|#1~~~{\gdef\jobname{|\textit{dest}|#1}}|\\
|\expandafter\redirectjob\jobname~~~}\input{\jobname}|
\end{tabular}
\end{center}

In an alternative approach,
child documents can be compiled by a specific command line
without additional code or specific definitions:
%
\begin{center}
|... -jobname "|\textit{target}|" "|[\textit{flags}]%
|\includeonly{|\textit{dest}|}\input{|\textit{main}|}"|
\end{center}
%

%%%%%%%%%%%%%%%%%%%%%%%%%%%%%%%%%%%%%%%%%%%%%%%%%%%%%%%%%%%%%%%%%%%%%%%%%%%%%%%%
%%%%%%%%%%%%%%%%%%%%%%%%%%%%%%%%%%%%%%%%%%%%%%%%%%%%%%%%%%%%%%%%%%%%%%%%%%%%%%%%
\section{Information}

%%%%%%%%%%%%%%%%%%%%%%%%%%%%%%%%%%%%%%%%%%%%%%%%%%%%%%%%%%%%%%%%%%%%%%%%%%%%%%%%
\subsection{Copyright}

Copyright \copyright{} 2017--2018 Niklas Beisert

This work may be distributed and/or modified under the
conditions of the \LaTeX{} Project Public License, either version 1.3
of this license or (at your option) any later version.
The latest version of this license is in
  \url{http://www.latex-project.org/lppl.txt}
and version 1.3 or later is part of all distributions of \LaTeX{}
version 2005/12/01 or later.

This work has the LPPL maintenance status `maintained'.

The Current Maintainer of this work is Niklas Beisert.

This work consists of the files |README.txt|, |childdoc.ins| and |childdoc.dtx|
as well as the derived files |childdoc.def|, |cdocsamp.tex|
with |cdocsch1.tex|, |cdocsch2.tex|, |cdocspt3.tex|, |cdocspt4.tex|,
|cdocsdrf.tex|, |cdocsfn1.tex|, |cdocsfn2.tex|
as well as |childdoc.pdf|.

%%%%%%%%%%%%%%%%%%%%%%%%%%%%%%%%%%%%%%%%%%%%%%%%%%%%%%%%%%%%%%%%%%%%%%%%%%%%%%%%
\subsection{Files and Installation}

The package consists of the files:
%
\begin{center}
\begin{tabular}{ll}
    |README.txt|   & readme file \\
    |childdoc.ins| & installation file \\
    |childdoc.dtx| & source file \\
    |childdoc.def| & definition file \\
    |cdocsamp.tex| & sample main file \\
    |cdocsch1.tex| & sample include file \\
    |cdocsch2.tex| & sample include file \\
    |cdocspt3.tex| & sample part file \\
    |cdocspt4.tex| & sample part file \\
    |cdocsdrf.tex| & sample redirection file \\
    |cdocsfn1.tex| & sample redirection file \\
    |cdocsfn2.tex| & sample redirection file \\
    |childdoc.pdf| & manual
\end{tabular}
\end{center}
%
The distribution consists of the files
|README.txt|, |childdoc.ins| and |childdoc.dtx|.
%
\begin{itemize}
\item
Run (pdf)\LaTeX{} on |childdoc.dtx|
to compile the manual |childdoc.pdf| (this file).
\item
Run \LaTeX{} on |childdoc.ins| to create the definitions file |childdoc.def|
and the sample |cdocsamp.tex| with include files
|cdocsch1.tex|, |cdocsch2.tex|, |cdocspt3.tex|, |cdocspt4.tex|,
|cdocsdrf.tex|, |cdocsfn1.tex|, |cdocsfn2.tex|.
Then copy the file |childdoc.def| to an appropriate directory of your \LaTeX{}
distribution, e.g.\ \textit{texmf-root}|/tex/latex/childdoc|.
\end{itemize}

%%%%%%%%%%%%%%%%%%%%%%%%%%%%%%%%%%%%%%%%%%%%%%%%%%%%%%%%%%%%%%%%%%%%%%%%%%%%%%%%
\subsection{Related CTAN Packages}

There are several other packages which offer a similar functionality:
%
\begin{itemize}
\item
The packages
\href{http://ctan.org/pkg/docmute}{\textsf{docmute}},
\href{http://ctan.org/pkg/includex}{\textsf{includex}} and
\href{http://ctan.org/pkg/standalone}{\textsf{standalone}}
provide commands to include only the document body of
a child file thus allowing both files to be compiled individually.
\item
The packages \href{http://ctan.org/pkg/subdocs}{\textsf{subdocs}}
and \href{http://ctan.org/pkg/subfiles}{\textsf{subfiles}}
provide structures in which the main and child documents can be
encapsulated and allowing them to be compiled individually.
The inclusion mechanism is different from the conventional |\include|.
\item
The package \href{http://ctan.org/pkg/combine}{\textsf{combine}}
is an elaborate solution to combine several documents into one.
\end{itemize}
%
See also the CTAN topic \href{http://ctan.org/topic/subdocs}{\textsf{subdocs}}
for further related packages.
The present package differs from the above solutions in that
a document structure constructed with the conventional |\include| mechanism
just needs two extra commands at the top of every file
such that all constituent files can be compiled individually.

%%%%%%%%%%%%%%%%%%%%%%%%%%%%%%%%%%%%%%%%%%%%%%%%%%%%%%%%%%%%%%%%%%%%%%%%%%%%%%%%
%\subsection{Feature Suggestions}
%
%The following is a list of features which may be useful for future
%versions of this package:
%%
%\begin{itemize}
%\item
%\ldots
%\end{itemize}

%%%%%%%%%%%%%%%%%%%%%%%%%%%%%%%%%%%%%%%%%%%%%%%%%%%%%%%%%%%%%%%%%%%%%%%%%%%%%%%%
\subsection{Revision History}

%%%%%%%%%%%%%%%%%%%%%%%%%%%%%%%%%%%%%%%%
\paragraph{v2.0:} 2018/12/30

\begin{itemize}
\item
immediate forward processing
\item
added |\childdocby| mechanism
\item
manual restructured
\end{itemize}

%%%%%%%%%%%%%%%%%%%%%%%%%%%%%%%%%%%%%%%%
\paragraph{v1.6:} 2018/01/17

\begin{itemize}
\item
application for development of include files
\item
corrections to manual
\end{itemize}

%%%%%%%%%%%%%%%%%%%%%%%%%%%%%%%%%%%%%%%%
\paragraph{v1.5:} 2017/05/21

\begin{itemize}
\item
more complete structuring introduced
\item
|\childdocof| introduced
\item
|\childdoc| renamed to |\childdocmain|
\item
|\childredirect| renamed to |\childdocforward| and |\childdocforwardprefix|
and functionality expanded
\end{itemize}

%%%%%%%%%%%%%%%%%%%%%%%%%%%%%%%%%%%%%%%%
\paragraph{v1.0:} 2017/04/27

\begin{itemize}
\item
manual and install package
\item
first version published on CTAN
\end{itemize}

%%%%%%%%%%%%%%%%%%%%%%%%%%%%%%%%%%%%%%%%
\paragraph{v0.6:} 2017/04/26

\begin{itemize}
\item
redirection mechanism added
\end{itemize}

%%%%%%%%%%%%%%%%%%%%%%%%%%%%%%%%%%%%%%%%
\paragraph{v0.5:} 2017/04/26

\begin{itemize}
\item
functionality in definition file
\end{itemize}


%%%%%%%%%%%%%%%%%%%%%%%%%%%%%%%%%%%%%%%%%%%%%%%%%%%%%%%%%%%%%%%%%%%%%%%%%%%%%%%%
%%%%%%%%%%%%%%%%%%%%%%%%%%%%%%%%%%%%%%%%%%%%%%%%%%%%%%%%%%%%%%%%%%%%%%%%%%%%%%%%
%%%%%%%%%%%%%%%%%%%%%%%%%%%%%%%%%%%%%%%%%%%%%%%%%%%%%%%%%%%%%%%%%%%%%%%%%%%%%%%%
\appendix

\settowidth\MacroIndent{\rmfamily\scriptsize 000\ }

 \DocInput{childdoc.dtx}

\end{document}
%</driver>
% \fi
%
% %%%%%%%%%%%%%%%%%%%%%%%%%%%%%%%%%%%%%%%%%%%%%%%%%%%%%%%%%%%%%%%%%%%%%%%%%%%%%%
% %%%%%%%%%%%%%%%%%%%%%%%%%%%%%%%%%%%%%%%%%%%%%%%%%%%%%%%%%%%%%%%%%%%%%%%%%%%%%%
% \section{Sample}
%\iffalse
%<*samplemain>
%\fi
%
% The following presents a sample document
% with two chapters, two parts, a title page,
% a compile flag as well as three forwarding files to set the flag.
% It consists of eight |.tex| files:
% \begin{center}
% \begin{tabular}{ll}
% |cdocsamp.tex|&main file\\
% |cdocsch1.tex|&include file for chapter 1\\
% |cdocsch2.tex|&include file for chapter 2\\
% |cdocspt3.tex|&include file for part 3\\
% |cdocspt4.tex|&include file for part 4\\
% |cdocsdrf.tex|&forwarding file for main file in draft mode\\
% |cdocsfi1.tex|&forwarding file for final version of chapter 1\\
% |cdocsfi2.tex|&forwarding file for final version of chapter 2\\
% \end{tabular}
% \end{center}
% Each of the eight files can be compiled directly by the \LaTeX{} compiler.
%
% %%%%%%%%%%%%%%%%%%%%%%%%%%%%%%%%%%%%%%
% \paragraph{Main File.}
%
% The main file is called |cdocsamp.tex|.
%
% Load the \textsf{childdoc} definitions and
% declare the filename for the main document:
%    \begin{macrocode}
\input{childdoc.def}
\childdocmain{}
%    \end{macrocode}

% Optional override for |\version| flag:
%    \begin{macrocode}
%%\ifchilddoc\else\providecommand{\version}{draft}\fi
%    \end{macrocode}

% Define the default values for the |\version| flag
% (|final| for the main file and |draft| for childs):
%    \begin{macrocode}
\ifchilddoc
\providecommand{\version}{draft}
\else
\providecommand{\version}{final}
\fi
%    \end{macrocode}

% Load the standard document class:
%    \begin{macrocode}
\documentclass[12pt]{article}
%    \end{macrocode}

% Start the document body:
%    \begin{macrocode}
\begin{document}
%    \end{macrocode}

% Declare a title page.
% Print title, part of document being processed and version flag:
%    \begin{macrocode}
\addtocounter{page}{-1}
\begin{center}
{\LARGE\bfseries{}childdoc example\par}
\vspace{1cm}
\ifchilddoc
\ifchilddocmanual part\else chapter\fi:
`\childdocname' of `\childdocjob'\par
\else
main document: `\childdocjob'\par
\fi
version: \version\par
\end{center}
\newpage
%    \end{macrocode}

% Manually include selected file,
% otherwise process as usual:
%    \begin{macrocode}
\ifchilddocmanual
\section*{part `\childdocname'}
\input{\childdocname}
\else
%    \end{macrocode}

% Include the two chapters:
%    \begin{macrocode}
\include{cdocsch1}
\include{cdocsch2}
%    \end{macrocode}

% Include the two parts unless only chapters should be displayed:
%    \begin{macrocode}
\ifchilddoc\else
\section{part three}
\input{cdocspt3}
\section{part four}
\input{cdocspt4}
\fi
%    \end{macrocode}

% Process as usual until here:
%    \begin{macrocode}
\fi
%    \end{macrocode}

% End of document body:
%    \begin{macrocode}
\end{document}
%    \end{macrocode}
%\iffalse
%</samplemain>
%\fi
%
% %%%%%%%%%%%%%%%%%%%%%%%%%%%%%%%%%%%%%%
% \paragraph{Chapter Include Files.}
%
% The include files are called |cdocsch1.tex| and |cdocsch2.tex|.
%
%\iffalse
%<*samplechap1|samplechap2>
%\fi

% Optional override for |\version| flag:
%    \begin{macrocode}
%%\providecommand{\version}{final}
%    \end{macrocode}

% Include the main document:
%    \begin{macrocode}
\input{childdoc.def}
\childdocof{cdocsamp}
%    \end{macrocode}

%\iffalse
%</samplechap1|samplechap2>
%\fi
%
%\iffalse
%<*samplechap1>
%\fi
% Some text for chapter 1:
%    \begin{macrocode}
\section{one}
some text in chapter one
%    \end{macrocode}

%\iffalse
%</samplechap1>
%\fi
% Some text for chapter 2:
%\iffalse
%<*samplechap2>
%\fi
%    \begin{macrocode}
\section{two}
more text in chapter two
%    \end{macrocode}

%\iffalse
%</samplechap2>
%\fi
%
% %%%%%%%%%%%%%%%%%%%%%%%%%%%%%%%%%%%%%%
% \paragraph{Part Include Files.}
%
% The include files are called |cdocspt3.tex| and |cdocspt4.tex|.
%
%\iffalse
%<*samplepart3|samplepart4>
%\fi

% Optional override for |\version| flag:
%    \begin{macrocode}
%%\providecommand{\version}{final}
%    \end{macrocode}

% Include the main document:
%    \begin{macrocode}
\input{childdoc.def}
\childdocby{cdocsamp}
%    \end{macrocode}

%\iffalse
%</samplepart3|samplepart4>
%\fi
%
%\iffalse
%<*samplepart3>
%\fi
% Some text for part 3:
%    \begin{macrocode}
some text in part three
%    \end{macrocode}

%\iffalse
%</samplepart3>
%\fi
% Some text for part 4:
%\iffalse
%<*samplepart4>
%\fi
%    \begin{macrocode}
more text in part four
%    \end{macrocode}

%\iffalse
%</samplepart4>
%\fi
%
% %%%%%%%%%%%%%%%%%%%%%%%%%%%%%%%%%%%%%%
% \paragraph{Forwarding for a Complete Draft.}
%
% The following forwarding file |cdocsdrf.tex|
% compiles the main document in draft mode:
%\iffalse
%<*sampledraft>
%\fi
%    \begin{macrocode}
\def\version{draft}
\input{childdoc.def}
\childdocforward{cdocsamp}
%    \end{macrocode}

%\iffalse
%</sampledraft>
%\fi
%
% %%%%%%%%%%%%%%%%%%%%%%%%%%%%%%%%%%%%%%
% \paragraph{Forwarding for Final Version of the Chapters.}
%
% The following forwarding files |cdocsfn1.tex| and |cdocsfn2.tex|
% (with identical content)
% compile the final versions of the child documents
% |cdocsch1.tex| and |cdocsch2.tex|, respectively:
%\iffalse
%<*samplefinal>
%\fi
%    \begin{macrocode}
\def\version{final}
\input{childdoc.def}
\childdocforwardprefix[cdocsamp]{cdocsfn}{cdocsch}
%    \end{macrocode}

%\iffalse
%</samplefinal>
%\fi
%
% %%%%%%%%%%%%%%%%%%%%%%%%%%%%%%%%%%%%%%
% \paragraph{Command Line Processing.}
%
% The following three command lines generate the output files
% |cdocscld|, |cdocscl1| and |cdocscl2|
% which should be identical to
% |cdocsdrf|, |cdocsch1| and |cdocsfn2|, respectively:
% \begin{center}
% \begin{tabular}{l}
% |latex -jobname cdocscld \|\\
% |  "\def\version{draft}\input{childdoc.def}\childdocforward{cdocsamp}"|\\
% |latex -jobname cdocscl1 \|\\
% |  "\input{childdoc.def}\childdocforward[cdocsamp]{cdocsch1}"|\\
% |latex -jobname cdocscl2 \|\\
% |  "\def\version{final}\input{childdoc.def}\childdocforward{cdocsch2}"|
% \end{tabular}
% \end{center}
% Note that the trailing backslash on each first line
% merely continues the input to the second line
% (for convenient cut ant paste).
% Furthermore, the command |latex| can be replaced by any
% of its alternative versions such as |pdflatex|.
%
% %%%%%%%%%%%%%%%%%%%%%%%%%%%%%%%%%%%%%%%%%%%%%%%%%%%%%%%%%%%%%%%%%%%%%%%%%%%%%%
% %%%%%%%%%%%%%%%%%%%%%%%%%%%%%%%%%%%%%%%%%%%%%%%%%%%%%%%%%%%%%%%%%%%%%%%%%%%%%%
% \section{Implementation}
%\iffalse
%<*package>
%\fi
%
% This section describes the definitions file |childdoc.def|.

% The definitions cannot be loaded using |\usepackage| or |\RequirePackage|
% which has a mechanism to prevent loading a style file more than once.
% When loading the definitions by means of |\input|
% multiple instances have to be prevented manually:
%\iffalse
%This code needs to be before the `\ProvidesFile' directive
%which is defined at the beginning of this file.
%Therefore it is also placed there and commented out here.
%</package>
%<*discard>
%\fi
%    \begin{macrocode}
\ifdefined\childdocmain\endinput\fi
%    \end{macrocode}
%\iffalse
%</discard>
%<*package>
%\fi
%
% \macro{\ifchilddoc}
% \macro{\ifchilddocmanual}
% The conditional |\ifchilddoc| tells whether a
% child (true) or main (false) document is being compiled.
% The conditional |\ifchilddocmanual| tells whether
% the |\includeonly| mechanism is used (false) or
% the selection of child files must be performed manually (true).
% The definitions initialise to false:
%    \begin{macrocode}
\newif\ifchilddoc
\newif\ifchilddocmanual
%    \end{macrocode}

% \macro{\childdocname}
% \macro{\childdocjob}
% The macro |\childdocname| stores the name of the main document
% to be compiled. The macro |\childdocjob| stores the name of
% the document on which the \LaTeX{} compiler was originally invoked.
% The content of |\jobname| cannot be compared
% to filenames specified in the source due to different catcodes.
% The following code rescans |\jobname|, stores the result
% in |\childdocname| and saves a copy in |\childdocjob|:
%    \begin{macrocode}
\edef\childdocname{\scantokens\expandafter{\jobname\noexpand}}
\let\childdocjob\childdocname
%    \end{macrocode}

% \macro{\childdocdisable}
% The macro |\childdocdisable| prevents the main file
% from being processed more than once.
% At this stage, the main document command |\childdocmain|
% is assumed to be called once again where it should do nothing.
% Any subsequent call to it should prevent
% a secondary processing of the main document
% It overwrites the forwarding commands
% |\childdocof| and |\childdocforward|
% with empty macros to prevent further inclusions of the main document:
%    \begin{macrocode}
\newcommand{\childdocdisable}
{
  \renewcommand{\childdocmain}[1]{\renewcommand{\childdocmain}[1]{\endinput}}
  \renewcommand{\childdocof}[1]{}
  \renewcommand{\childdocby}[2][]{}
  \renewcommand{\childdocforward}[2][]{}
  \renewcommand{\childdocdisable}{}
}
%    \end{macrocode}

% \macro{\childdocmain}
% The macro |\childdocmain| is to be called at the top of the main file
% with nothing or the main filename (without extension) as argument.
% First, it breaks loops.
% If the argument is not empty and does not match |\childdocname|
% (which is set by the first inclusion of |childdoc.def|),
% |\ifchilddoc| is set to true, |\includeonly| is applied to the child file
% and |\jobname| is set to the main file
% (for proper handling of |.aux| files):
%    \begin{macrocode}
\newcommand{\childdocmain}[1]
{
  \childdocdisable\childdocmain{}
  \if?#1?\else
    \begingroup
      \def\childdoctmp{#1}
      \ifx\childdoctmp\childdocname
        \def\childdoctmp{}
      \else
        \def\childdoctmp
        {
          \childdoctrue
          \includeonly{\childdocname}
          \def\childdocjob{#1}
          \def\jobname{#1}
        }
      \fi
      \expandafter
    \endgroup
    \childdoctmp
  \fi
}
%    \end{macrocode}

% \macro{\childdocof}
% The command |\childdocof| redirects
% compilation to the main file |#1|.
%    \begin{macrocode}
\newcommand{\childdocof}[1]
{
  \childdocdisable
  \childdoctrue
  \includeonly{\childdocname}
  \def\jobname{#1}
  \def\childdocjob{#1}
  \input{#1}
}
%    \end{macrocode}

% \macro{\childdocby}
% The command |\childdocby| ....
%    \begin{macrocode}
\newcommand{\childdocby}[2][]
{
  \childdocdisable
  \childdoctrue
  \childdocmanualtrue
  \if?#1?\else
    \def\jobname{#2}
  \fi
  \def\childdocjob{#2}
  \input{#2}
  \endinput
}
%    \end{macrocode}

% \macro{\childdocforward}
% The command |\childdocforward| redirects
% compilation to the main file or
% (if the optional argument is given) a child file.
% Parameters are set as if the main file
% or a child file starting with |\childdocof| was compiled.
% Then compilation is handed over to the main file:
%    \begin{macrocode}
\newcommand{\childdocforward}[2][]
{
  \begingroup
    \if?#1?
      \def\childdoctmp
      {
        \def\childdocname{#2}
        \def\childdocjob{#2}
        \def\jobname{#2}
        \input{#2}
        \endinput
      }
    \else
      \def\childdoctmp
      {
        \childdocdisable
        \def\childdocname{#2}
        \childdoctrue
        \includeonly{#2}
        \def\childdocjob{#1}
        \def\jobname{#1}
        \input{#1}
        \endinput
      }
    \fi
    \expandafter
  \endgroup
  \childdoctmp
}
%    \end{macrocode}

% \macro{\childdocforwardprefix}
% The command |\childdocforwardprefix| redirects
% compilation to the main or a child file by means of a pattern.
% The prefix |#1| in the current filename is replaced by |#2|
% and the suffix of the current filename is kept
% (it is assumed that the filename does not contain the substring `|~~~|'
% which is used as a delimiter).
% Compilation is handed over to the new file by |\childdocforward|:
%    \begin{macrocode}
\newcommand{\childdocforwardprefix}[3][]
{
  \begingroup
    \def\childdocextract #2##1~~~{\def\childdoctmp{\childdocforward[#1]{#3##1}}}
    \expandafter\childdocextract\childdocname~~~
    \expandafter
  \endgroup
  \childdoctmp
}
%    \end{macrocode}

% \macro{\childdoc}
% The deprecated macro |\childdoc| is a legacy version of |\childdocmain|:
%    \begin{macrocode}
\newcommand{\childdoc}{\childdocmain}
%    \end{macrocode}

% \macro{\childdocredirect}
% The deprecated macro |\childdocredirect| is a legacy version
% of |\childdocforward| and |\childdocforwardprefix|:
%    \begin{macrocode}
\newcommand{\childdocredirect}[2][]
{
  \begingroup
    \if?#1?
      \def\childdoctmp{\childdocforward{#2}}
    \else
      \def\childdoctmp{\childdocforwardprefix{#1}{#2}}
    \fi
    \expandafter
  \endgroup
  \childdoctmp
}
%    \end{macrocode}

%\iffalse
%</package>
%\fi
%
\endinput
|\\
|\childdocof{|\textit{main}|}|\\
\end{tabular}
\end{center}
at the top of every child file \textit{child}
which is included by |\include{|\textit{child}|}|
from within the main file
(or at least for those files to be compiled individually).
The argument \textit{main} must be the filename of the main file.

There are a couple of
considerations in setting up the main and child documents:

%%%%%%%%%%%%%%%%%%%%%%%%%%%%%%%%%%%%%%%%
\paragraph{Restrictions.}

Please note the following restrictions:
\begin{itemize}
\item
|\childdocmain| must be called with one argument \textit{main}
to ensure compatibility with earlier version of the package.
It must either be empty (|\childdocmain{}|)
or precisely match the filename of the main file in which it is specified.
See \secref{sec:detection} for further information.
\item
The filename \textit{main} must be specified without the |.tex| extension.
\item
The filename \textit{main} is case sensitive
(even in case-insensitive file systems)
due to internal string comparison.
\item
The argument \textit{main} should be fully expanded, it cannot be a macro.
\item
Subdirectories and special characters should be avoided in filenames.
\item
The command |\childdocmain{|\textit{main}|}| must be followed by a whitespace.
It should not be followed immediately by another command
or by a comment mark `|%|'.
This is because the \TeX{} parser reads the token immediately following
the argument of |\childdocmain| and puts it
at the beginning of every child section;
however, a white\-space is ignored.
\end{itemize}

%%%%%%%%%%%%%%%%%%%%%%%%%%%%%%%%%%%%%%%%
\paragraph{Content of Main File.}

It is advisable to place all content in the child files included by |\include|.
Any output contained in the main file will appear in all child documents
unless suppressed manually;
it cannot be suppressed automatically by the |\includeonly| directive
and thus should normally be avoided.
A method to include some content in the main file
by means of conditional processing is described in \secref{sec:conditional}.

%%%%%%%%%%%%%%%%%%%%%%%%%%%%%%%%%%%%%%%%
\paragraph{Page Numbering.}

When only a part of the document is compiled,
the appropriate numbering of pages
(as well as other status parameters)
is determined from the |.aux| files.
The latter contain information from previous passes.
However this information needs to propagate through
all intermediate child documents.
Therefore the page numbering in child documents may well
be inconsistent until the complete document is compiled at least once.

A useful (if unconventional) way to always ensure a consistent
page numbering is to restart the numbering in each child document
and denote the pages by `\textit{child}|.|\textit{page}'
where \textit{child} represents the chapter/section number of the child file.
This can be achieved by the command
|\numberwithin{page}{|\textit{child}|}|
of the \textsf{amsmath} package
where \textit{child} can be |chapter| or |section|
depending on the chosen structuring.
Alternatively, one can modify the macro |\thepage| appropriately
and reset the counter |page| at the start of each child file.

%%%%%%%%%%%%%%%%%%%%%%%%%%%%%%%%%%%%%%%%%%%%%%%%%%%%%%%%%%%%%%%%%%%%%%%%%%%%%%%%
\subsection{Conditional Processing}
\label{sec:conditional}

The package provides a mechanism to compile different versions
of a document. To customise the versions further some conditional processing
can come in handy to distinguish which version is being compiled.
The package provides two macros to describe the compilation context:

%%%%%%%%%%%%%%%%%%%%%%%%%%%%%%%%%%%%%%%%
\DescribeMacro{\ifchilddoc}
The conditional |\ifchilddoc| distinguishes between the compilation of
child documents and the main document:
%
\begin{center}
|\ifchilddoc |\textit{child-code}| |[|\||else |\textit{main-code}]| \||fi|
\end{center}

%%%%%%%%%%%%%%%%%%%%%%%%%%%%%%%%%%%%%%%%
\DescribeMacro{\childdocname}
\DescribeMacro{\childdocjob}
The macro |\childdocname| contains the filename (without extension)
of the main or child file being processed.
Note that |\childdocjob| will always contain the name of the main file.

%%%%%%%%%%%%%%%%%%%%%%%%%%%%%%%%%%%%%%%%
\paragraph{Title Page.}

Conditional processing can be used to include a title or banner page
in the main document when proper precautions are taken.
Importantly, the code in the main file should ensure that the page counter
(as well as other status parameters which are stored in the |.aux| files)
takes the same value after the conditional processing.
Otherwise the page numbers may take divergent values
depending on which part is compiled.

For example, a title page could be declared by:
%
\begin{center}
\begin{tabular}{l}
|\ifchilddoc\||else|\\
|\addtocounter{page}{-1}|\\
\textit{code for title page}\\
|\newpage|\\
|\||fi|
\end{tabular}
\end{center}
%
A banner page for the child documents can be generated by:
%
\begin{center}
\begin{tabular}{l}
|\ifchilddoc|\\
|\addtocounter{page}{-1}|\\
\textit{code for banner page}\\
|\newpage|\\
|\||fi|
\end{tabular}
\end{center}
%
Here one could write a message such as:
\begin{center}
|This is the part \childdocname{} of \childdocjob{}.|
\end{center}

%%%%%%%%%%%%%%%%%%%%%%%%%%%%%%%%%%%%%%%%%%%%%%%%%%%%%%%%%%%%%%%%%%%%%%%%%%%%%%%%
\subsection{Flags}
\label{sec:flags}

The package makes it easy to generate different versions
of the main or child documents.
To this end compilation flags can be defined
and assigned different default values.
They will be particularly useful in conjunction
with the forwarding mechanism described in \secref{sec:forward}.

For example, it may be useful to have a flag |\version|
which can be set to |draft| or |final|.
The document source will contain some conditional code
depending on the value of |\version|.
Suppose further, the flag should default to |final| for the main file
and to |draft| for child files
which is a natural assignment for editing the document.
This is achieved by placing the following code
in the preamble of the main document
(below the |\childdocmain| directive):
%
\begin{center}
\begin{tabular}{l}
|\ifchilddoc|\\
|\providecommand{\version}{draft}|\\
|\||else|\\
|\providecommand{\version}{final}|\\
|\||fi|
\end{tabular}
\end{center}
%
The definition by |\providecommand| makes sure
that previous definitions are not overwritten.
Further statements |\providecommand{\version}{...}|
can thus be added before the above code to override it.

For the main file, one might add a line
(between |\childdocmain| and the above block)
%
\begin{center}
|%\ifchilddoc\||else\providecommand{\version}{draft}\||fi|
\end{center}
%
which can be uncommented to produce a draft version.
Likewise one can add a line to the very top of a child file
(above the |\childdocof{|\textit{main}|}| directive)
%
\begin{center}
|%\providecommand{\version}{final}|
\end{center}
%
which can be uncommented to produce the final version of this child document.

%%%%%%%%%%%%%%%%%%%%%%%%%%%%%%%%%%%%%%%%%%%%%%%%%%%%%%%%%%%%%%%%%%%%%%%%%%%%%%%%
\subsection{Forwarding}
\label{sec:forward}

Different versions of the main or child documents
using compilation flags as described in \secref{sec:flags}
can be (permanently) stored in different files
for convenient compilation, viewing and distribution.
To this end, the package defines a command
to pass on compilation to a different file:

%%%%%%%%%%%%%%%%%%%%%%%%%%%%%%%%%%%%%%%%
\DescribeMacro{\childdocforward}
The command |\childdocforward| redirects processing to
another source file:
%
\begin{center}
\begin{tabular}{l}
|% \iffalse
%
% childdoc.dtx Copyright (C) 2017-2018 Niklas Beisert
%
% This work may be distributed and/or modified under the
% conditions of the LaTeX Project Public License, either version 1.3
% of this license or (at your option) any later version.
% The latest version of this license is in
%   http://www.latex-project.org/lppl.txt
% and version 1.3 or later is part of all distributions of LaTeX
% version 2005/12/01 or later.
%
% This work has the LPPL maintenance status `maintained'.
%
% The Current Maintainer of this work is Niklas Beisert.
%
% This work consists of the files childdoc.dtx and childdoc.ins
% and the derived files childdoc.def and cdocsamp.tex with
% cdocsch1.tex, cdocsch2.tex, cdocsdrf.tex, cdocsfn1.tex, cdocsfn2.tex.
%
%<package>\ifdefined\childdocmain\endinput\fi
%<package>\ProvidesFile{childdoc.def}[2018/12/30 v2.0 child document driver]
%<samplemain>\ProvidesFile{cdocsamp.tex}[2018/12/30 v2.0 sample for childdoc]
%<*driver>
%\ProvidesFile{childdoc.drv}[2018/12/30 v2.0 childdoc reference manual file]
\PassOptionsToClass{10pt,a4paper}{article}
\documentclass{ltxdoc}

\usepackage[margin=35mm]{geometry}
\usepackage{hyperref}
\usepackage{hyperxmp}
\usepackage[usenames]{color}

\hypersetup{colorlinks=true}
\hypersetup{pdfstartview=FitH}
\hypersetup{pdfpagemode=UseNone}
\hypersetup{pdfsource={}}
\hypersetup{pdflang={en-UK}}
\hypersetup{pdfcopyright={Copyright 2017-2018 Niklas Beisert.
  This work may be distributed and/or modified under the
  conditions of the LaTeX Project Public License, either version 1.3
  of this license or (at your option) any later version.}}
\hypersetup{pdflicenseurl={http://www.latex-project.org/lppl.txt}}
\hypersetup{pdfcontactaddress={ETH Zurich, ITP, HIT K,
  Wolfgang-Pauli-Strasse 27}}
\hypersetup{pdfcontactpostcode={8093}}
\hypersetup{pdfcontactcity={Zurich}}
\hypersetup{pdfcontactcountry={Switzerland}}
\hypersetup{pdfcontactemail={nbeisert@itp.phys.ethz.ch}}
\hypersetup{pdfcontacturl={http://people.phys.ethz.ch/\xmptilde nbeisert/}}

\newcommand{\secref}[1]{\hyperref[#1]{section \ref*{#1}}}

\parskip1ex
\parindent0pt
\let\olditemize\itemize
\def\itemize{\olditemize\parskip0pt}

\begin{document}

\title{The \textsf{childdoc} Package}
\hypersetup{pdftitle={The childdoc Package}}
\author{Niklas Beisert\\[2ex]
  Institut f\"ur Theoretische Physik\\
  Eidgen\"ossische Technische Hochschule Z\"urich\\
  Wolfgang-Pauli-Strasse 27, 8093 Z\"urich, Switzerland\\[1ex]
  \href{mailto:nbeisert@itp.phys.ethz.ch}
  {\texttt{nbeisert@itp.phys.ethz.ch}}}
\hypersetup{pdfauthor={Niklas Beisert}}
\hypersetup{pdfsubject={Manual for the LaTeX2e Package childdoc}}
\date{30 December 2018, \textsf{v2.0}}
\maketitle

\begin{abstract}\noindent
\textsf{childdoc} is a \LaTeXe{} package
that enables the direct compilation
of document sections included by |\include|
to individual files.
\end{abstract}

\begingroup
\parskip0ex
\tableofcontents
\endgroup

%%%%%%%%%%%%%%%%%%%%%%%%%%%%%%%%%%%%%%%%%%%%%%%%%%%%%%%%%%%%%%%%%%%%%%%%%%%%%%%%
%%%%%%%%%%%%%%%%%%%%%%%%%%%%%%%%%%%%%%%%%%%%%%%%%%%%%%%%%%%%%%%%%%%%%%%%%%%%%%%%
\section{Introduction}

\LaTeX{} provides a mechanism to structure a large document (such as a book)
into a main file and several child files (containing the chapters)
using the |\include| command.
This mechanism is beneficial for documents
which span hundreds of pages in order to
make the source file(s) more manageable.
Moreover, compilation can be restricted to
selected child files by means of the |\includeonly| command.
The latter feature can be used to reduce the compilation time while editing
(this was significantly more useful in the earlier days of \LaTeX{})
or to generate a smaller document which is easier to navigate.
Another application of |\includeonly| is to generate
documents consisting of selected parts of the complete document.

However, there are a few drawbacks of the plain |\include| mechanism:
\begin{itemize}
\item
The child files cannot be compiled on their own,
they can only be compiled via the main file.
A naive editing environment
(such as a text editor with an option
to have the current file processed by \LaTeX)
may require one to switch to the main file before compiling;
attempting to compile the child file produces errors.
\item
The main file must be modified (each time)
to adjust the |\includeonly| command
to the present needs. This easily leaves the main file in a messy state.
\item
The generated document will always carry the filename
of the main document. This is inconvenient if
several child files are to be compiled and
to be kept for distribution.
\end{itemize}

The present package provides a simple interface
to make child files individually compilable by \LaTeX{}.
Compiling a child file then has the same effect as compiling
the main file with an |\includeonly| command
to select the appropriate child.
Moreover the generated document will carry the name of the child
rather than the main file.
This resolves all three above issues.

This feature is meant to make the editing of books,
thesis documents and lecture notes somewhat more convenient.
However, the package can also be used efficiently for
composing a series of documents (such as exercise sheets)
which are typically distributed individually.
It then assists the author in generating the individual documents
(potentially in different versions)
as well as a document containing the collected series.
Another application is in developing style files
or other kinds of included material
where compilation of the style file could redirect
to a sample or test file.

%%%%%%%%%%%%%%%%%%%%%%%%%%%%%%%%%%%%%%%%%%%%%%%%%%%%%%%%%%%%%%%%%%%%%%%%%%%%%%%%
%%%%%%%%%%%%%%%%%%%%%%%%%%%%%%%%%%%%%%%%%%%%%%%%%%%%%%%%%%%%%%%%%%%%%%%%%%%%%%%%
\section{Usage}

First of all, the package \textsf{childdoc} is \emph{not} a standard
\LaTeXe{} |.sty| style file! Therefore it needs to be invoked in
a non-standard way.

%%%%%%%%%%%%%%%%%%%%%%%%%%%%%%%%%%%%%%%%%%%%%%%%%%%%%%%%%%%%%%%%%%%%%%%%%%%%%%%%
\subsection{Included Files}
\label{sec:include}

%%%%%%%%%%%%%%%%%%%%%%%%%%%%%%%%%%%%%%%%
\DescribeMacro{\childdocmain}
To use the package, add the commands
\begin{center}
\begin{tabular}{l}
|\input{childdoc.def}|\\
|\childdocmain{}|\\
\end{tabular}
\end{center}
at the very top of the main \LaTeX{} file,
in particular \emph{before} the |\documentclass| statement!
The argument of |\childdocmain| should be left empty
(but it must be present).

%%%%%%%%%%%%%%%%%%%%%%%%%%%%%%%%%%%%%%%%
\DescribeMacro{\childdocof}
Furthermore, add the commands
\begin{center}
\begin{tabular}{l}
|\input{childdoc.def}|\\
|\childdocof{|\textit{main}|}|\\
\end{tabular}
\end{center}
at the top of every child file \textit{child}
which is included by |\include{|\textit{child}|}|
from within the main file
(or at least for those files to be compiled individually).
The argument \textit{main} must be the filename of the main file.

There are a couple of
considerations in setting up the main and child documents:

%%%%%%%%%%%%%%%%%%%%%%%%%%%%%%%%%%%%%%%%
\paragraph{Restrictions.}

Please note the following restrictions:
\begin{itemize}
\item
|\childdocmain| must be called with one argument \textit{main}
to ensure compatibility with earlier version of the package.
It must either be empty (|\childdocmain{}|)
or precisely match the filename of the main file in which it is specified.
See \secref{sec:detection} for further information.
\item
The filename \textit{main} must be specified without the |.tex| extension.
\item
The filename \textit{main} is case sensitive
(even in case-insensitive file systems)
due to internal string comparison.
\item
The argument \textit{main} should be fully expanded, it cannot be a macro.
\item
Subdirectories and special characters should be avoided in filenames.
\item
The command |\childdocmain{|\textit{main}|}| must be followed by a whitespace.
It should not be followed immediately by another command
or by a comment mark `|%|'.
This is because the \TeX{} parser reads the token immediately following
the argument of |\childdocmain| and puts it
at the beginning of every child section;
however, a white\-space is ignored.
\end{itemize}

%%%%%%%%%%%%%%%%%%%%%%%%%%%%%%%%%%%%%%%%
\paragraph{Content of Main File.}

It is advisable to place all content in the child files included by |\include|.
Any output contained in the main file will appear in all child documents
unless suppressed manually;
it cannot be suppressed automatically by the |\includeonly| directive
and thus should normally be avoided.
A method to include some content in the main file
by means of conditional processing is described in \secref{sec:conditional}.

%%%%%%%%%%%%%%%%%%%%%%%%%%%%%%%%%%%%%%%%
\paragraph{Page Numbering.}

When only a part of the document is compiled,
the appropriate numbering of pages
(as well as other status parameters)
is determined from the |.aux| files.
The latter contain information from previous passes.
However this information needs to propagate through
all intermediate child documents.
Therefore the page numbering in child documents may well
be inconsistent until the complete document is compiled at least once.

A useful (if unconventional) way to always ensure a consistent
page numbering is to restart the numbering in each child document
and denote the pages by `\textit{child}|.|\textit{page}'
where \textit{child} represents the chapter/section number of the child file.
This can be achieved by the command
|\numberwithin{page}{|\textit{child}|}|
of the \textsf{amsmath} package
where \textit{child} can be |chapter| or |section|
depending on the chosen structuring.
Alternatively, one can modify the macro |\thepage| appropriately
and reset the counter |page| at the start of each child file.

%%%%%%%%%%%%%%%%%%%%%%%%%%%%%%%%%%%%%%%%%%%%%%%%%%%%%%%%%%%%%%%%%%%%%%%%%%%%%%%%
\subsection{Conditional Processing}
\label{sec:conditional}

The package provides a mechanism to compile different versions
of a document. To customise the versions further some conditional processing
can come in handy to distinguish which version is being compiled.
The package provides two macros to describe the compilation context:

%%%%%%%%%%%%%%%%%%%%%%%%%%%%%%%%%%%%%%%%
\DescribeMacro{\ifchilddoc}
The conditional |\ifchilddoc| distinguishes between the compilation of
child documents and the main document:
%
\begin{center}
|\ifchilddoc |\textit{child-code}| |[|\||else |\textit{main-code}]| \||fi|
\end{center}

%%%%%%%%%%%%%%%%%%%%%%%%%%%%%%%%%%%%%%%%
\DescribeMacro{\childdocname}
\DescribeMacro{\childdocjob}
The macro |\childdocname| contains the filename (without extension)
of the main or child file being processed.
Note that |\childdocjob| will always contain the name of the main file.

%%%%%%%%%%%%%%%%%%%%%%%%%%%%%%%%%%%%%%%%
\paragraph{Title Page.}

Conditional processing can be used to include a title or banner page
in the main document when proper precautions are taken.
Importantly, the code in the main file should ensure that the page counter
(as well as other status parameters which are stored in the |.aux| files)
takes the same value after the conditional processing.
Otherwise the page numbers may take divergent values
depending on which part is compiled.

For example, a title page could be declared by:
%
\begin{center}
\begin{tabular}{l}
|\ifchilddoc\||else|\\
|\addtocounter{page}{-1}|\\
\textit{code for title page}\\
|\newpage|\\
|\||fi|
\end{tabular}
\end{center}
%
A banner page for the child documents can be generated by:
%
\begin{center}
\begin{tabular}{l}
|\ifchilddoc|\\
|\addtocounter{page}{-1}|\\
\textit{code for banner page}\\
|\newpage|\\
|\||fi|
\end{tabular}
\end{center}
%
Here one could write a message such as:
\begin{center}
|This is the part \childdocname{} of \childdocjob{}.|
\end{center}

%%%%%%%%%%%%%%%%%%%%%%%%%%%%%%%%%%%%%%%%%%%%%%%%%%%%%%%%%%%%%%%%%%%%%%%%%%%%%%%%
\subsection{Flags}
\label{sec:flags}

The package makes it easy to generate different versions
of the main or child documents.
To this end compilation flags can be defined
and assigned different default values.
They will be particularly useful in conjunction
with the forwarding mechanism described in \secref{sec:forward}.

For example, it may be useful to have a flag |\version|
which can be set to |draft| or |final|.
The document source will contain some conditional code
depending on the value of |\version|.
Suppose further, the flag should default to |final| for the main file
and to |draft| for child files
which is a natural assignment for editing the document.
This is achieved by placing the following code
in the preamble of the main document
(below the |\childdocmain| directive):
%
\begin{center}
\begin{tabular}{l}
|\ifchilddoc|\\
|\providecommand{\version}{draft}|\\
|\||else|\\
|\providecommand{\version}{final}|\\
|\||fi|
\end{tabular}
\end{center}
%
The definition by |\providecommand| makes sure
that previous definitions are not overwritten.
Further statements |\providecommand{\version}{...}|
can thus be added before the above code to override it.

For the main file, one might add a line
(between |\childdocmain| and the above block)
%
\begin{center}
|%\ifchilddoc\||else\providecommand{\version}{draft}\||fi|
\end{center}
%
which can be uncommented to produce a draft version.
Likewise one can add a line to the very top of a child file
(above the |\childdocof{|\textit{main}|}| directive)
%
\begin{center}
|%\providecommand{\version}{final}|
\end{center}
%
which can be uncommented to produce the final version of this child document.

%%%%%%%%%%%%%%%%%%%%%%%%%%%%%%%%%%%%%%%%%%%%%%%%%%%%%%%%%%%%%%%%%%%%%%%%%%%%%%%%
\subsection{Forwarding}
\label{sec:forward}

Different versions of the main or child documents
using compilation flags as described in \secref{sec:flags}
can be (permanently) stored in different files
for convenient compilation, viewing and distribution.
To this end, the package defines a command
to pass on compilation to a different file:

%%%%%%%%%%%%%%%%%%%%%%%%%%%%%%%%%%%%%%%%
\DescribeMacro{\childdocforward}
The command |\childdocforward| redirects processing to
another source file:
%
\begin{center}
\begin{tabular}{l}
|\input{childdoc.def}|\\
|\childdocforward[|\textit{main}|]{|\textit{dest}|}|\\
\end{tabular}
\end{center}
%
The argument \textit{dest} is the destination file
(without extension).
It should be the main file or one of the child files.
Note that further \textsf{childdoc} directives
such as |\childdocof| and |\childdocforward|
in the indicated file will be processed in this form.
The optional argument \textit{main}
passes on directly to the main file \textit{main}
while pretending to compile the child \textit{dest}.
This form behaves as if \textit{dest}
issues |\childdocof{|\textit{main}|}| right away,
and no further \textsf{childdoc} directives will be processed.

%%%%%%%%%%%%%%%%%%%%%%%%%%%%%%%%%%%%%%%%
\DescribeMacro{\...prefix}
In the alternative form |\childdocforwardprefix|,
%
\begin{center}
\begin{tabular}{l}
|\input{childdoc.def}|\\
|\childdocforwardprefix[|\textit{main}|]{|\textit{prefix}|}{|\textit{dest}|}|
\end{tabular}
\end{center}
%
the destination file is determined by a pattern
depending on the current file:
To make this work, the current file must be called
`{\textit{prefix}\hspace{0.2em}\textit{suffix}}'
with \textit{prefix} matching precisely the argument.
Processing is then passed on to the file
`{\textit{dest}\hspace{0.2em}\textit{suffix}}'.
Surely, the same effect is achieved by
directly specifying the
argument `{\textit{dest}\hspace{0.2em}\textit{suffix}}'
in the first form.
However, that requires to set up a different file
for each child. With the alternative form of the command
all these files can have exactly the same content
which simplifies setting them up and maintaining them.

For example, the following file |draft.tex|
with a compilation flag |\version| as described in \secref{sec:flags}
compiles the main document as a draft:
%
\begin{center}
\begin{tabular}{l}
|\def\version{draft}|\\
|\input{childdoc.def}|\\
|\childdocforward{|\textit{main}|}|
\end{tabular}
\end{center}
%
Likewise, the following files |final|\textit{nn}|.tex|
compile the final version of the child document
|child|\textit{nn}|.tex|:
%
\begin{center}
\begin{tabular}{l}
|\def\version{final}|\\
|\input{childdoc.def}|\\
|\childdocforwardprefix{final}{child}|
\end{tabular}
\end{center}
%

Note that when several versions of a main file and/or of each child file
are to be generated, it may be convenient to set up a |Makefile| or
shell script to automatise the process.

%%%%%%%%%%%%%%%%%%%%%%%%%%%%%%%%%%%%%%%%%%%%%%%%%%%%%%%%%%%%%%%%%%%%%%%%%%%%%%%%
\subsection{Command Line Processing}
\label{sec:commandline}

The effect of redirection files can also be achieved by invoking
the \LaTeX{} compiler with a more elaborate command line.
Most conveniently this should be done as part
of a shell script or a |Makefile|.

When using \textsf{childdoc} in the main file, the following
command lines effectively perform a redirection
(note that depending on the shell being used,
backslashes may have to be doubled: `|\|' $\to$ `|\\|'):
%
\begin{center}
|... -jobname "|\textit{target}|" |\\|"|[\textit{flags}]%
|\input{childdoc.def}\childdocforward[|\textit{main}|]{|\textit{dest}|}"|
\end{center}
%
Here \textit{target} is the name of the output file,
\textit{main} is the name of the main file
and \textit{dest} is the name of the main or child file to be processed
(all filenames without extensions).
The optional argument \textit{main} can be omitted
if \textit{main} matches \textit{dest}.
Optionally, compilation \textit{flags} can be defined via |\def| commands.
This command line makes the \TeX{} engine believe
it is compiling the file \textit{target}
whose content is specified as the latter parameter.
The provided code then forwards the processing to
\textit{main} or \textit{dest} as described in \secref{sec:forward}.

%%%%%%%%%%%%%%%%%%%%%%%%%%%%%%%%%%%%%%%%%%%%%%%%%%%%%%%%%%%%%%%%%%%%%%%%%%%%%%%%
\subsection{Include by Input}
\label{sec:input}

Including child documents by |\include| has some restrictions by design.
Most notably, the content of a child document always occupies
its own set of pages; pages cannot be shared between child documents.
Usually, this behaviour makes perfect sense
because each child document contain an essential part of the document.
However, in some situations it may be desirable to compose
a document from a collection of parts
without having mandatory page breaks between then.
For this case, the package
provides a mechanism to include parts
by |\input| which can also be processed individually.
However, by construction this mechanism
requires manual handling of the content to be output.

%%%%%%%%%%%%%%%%%%%%%%%%%%%%%%%%%%%%%%%%
\DescribeMacro{\ifchilddocmanual}
The main file should be prepared as usual, see \secref{sec:include}.
However, the document body must make a distinction
between processing of an individual part and of the main document, e.g.:
%
\begin{center}
\begin{tabular}{l}
|\ifchilddocmanual|\\
|\input{\childdocname}|\\
|\||else|\\
\textit{document body with }|\input{|\textit{part}|}|\\
|\||fi|
\end{tabular}
\end{center}
%
The conditional |\ifchilddocmanual| is true whenever
a part to be included by |\input| is being compiled,
and the name of the part is stored in |\childdocname|.

%%%%%%%%%%%%%%%%%%%%%%%%%%%%%%%%%%%%%%%%
\DescribeMacro{\childdocby}
Each part to be included by |\input| should start with:
%
\begin{center}
\begin{tabular}{l}
|\input{childdoc.def}|\\
|\childdocby{|\textit{main}|}|\\
\end{tabular}
\end{center}
%
The directive |\childdocby| is similar to |\childdocof|
described in \secref{sec:include},
but the subsequent selection of content must be done manually.
To that end, both |\ifchilddoc| and |\ifchilddocmanual|
will be true upon processing of a part,
and the name of the part is stored in |\childdocname|.
Note that |\jobname| will be set to the filename of the current part
so that each part receives an individual |.aux| file
that does not interfere with the |.aux| file(s) of the main document.
This behaviour can be altered by the alternative form
|\childdocby[*]{|\textit{main}|}| (with a non-empty optional argument)
which uses the |.aux| file of the main document
by setting |\jobname| to \textit{main}.

%%%%%%%%%%%%%%%%%%%%%%%%%%%%%%%%%%%%%%%%%%%%%%%%%%%%%%%%%%%%%%%%%%%%%%%%%%%%%%%%
\subsection{Driver Development}
\label{sec:driver}

The \textsf{childdoc} mechanism can also be use for the development
of definition files such as \LaTeX{} styles or classes.
This case differs from the above setup with multiple parts
included by |\include| in that no |\includeonly| should be invoked.
This can be achieved by starting the include file
(before |\ProvidesPackage|) with:
%
\begin{center}
\begin{tabular}{l}
|\input{childdoc.def}|\\
|\childdocforward{|\textit{main}|}|\\
\end{tabular}
\end{center}
%
or alternatively with:
%
\begin{center}
\begin{tabular}{l}
|\input{childdoc.def}|\\
|\childdocby{|\textit{main}|}|\\
\end{tabular}
\end{center}
%
Both forms have slightly different effects as described above.
The main file is prepared as usual, see \secref{sec:include}.

%%%%%%%%%%%%%%%%%%%%%%%%%%%%%%%%%%%%%%%%%%%%%%%%%%%%%%%%%%%%%%%%%%%%%%%%%%%%%%%%
\subsection{Legacy Detection}
\label{sec:detection}

The directive |\childdocmain| in the main file can detect
whether the complete document or merely a child is to be compiled
even without using the directive |\childdocof|.
This method is deprecated because it is less robust
and there is no compelling reason to use it;
it is merely provided for backward compatibility
and it may be removed in future versions.

If the detection mechanism is to be used,
it is mandatory to correctly specify
the filename of the main file as the argument of |\childdocmain|:
%
\begin{center}
\begin{tabular}{l}
|\input{childdoc.def}|\\
|\childdocmain{|\textit{main}|}|\\
\end{tabular}
\end{center}
%
If |\jobname| does not match the argument \textit{main} of |\childdocmain|,
it is assumed that |\jobname| points to the child file to be compiled.
When using |\childdocmain| with the main file specified as argument,
it suffices to start a child file
with just |\input{|\textit{main}|}|
without loading of the package and using |\childdocof|.
If instead all processing is done
with the appropriate \textsf{childdoc} directives,
the argument of \textit{main} of |\childdocmain| can be empty.

An alternative version of the command line processing described
in \secref{sec:commandline} using the detection mechanism reads:
%
\begin{center}
|... -jobname "|\textit{target}|" "|[\textit{flags}]%
[|\def\jobname{|\textit{dest}|}|]|\input{|\textit{main}|}"|
\end{center}

%%%%%%%%%%%%%%%%%%%%%%%%%%%%%%%%%%%%%%%%%%%%%%%%%%%%%%%%%%%%%%%%%%%%%%%%%%%%%%%%
\subsection{Manual Code}
\label{sec:manual}

In case one cannot be certain whether the definitions file |childdoc.def|
is installed on the target \TeX{} distribution
and one prefers not to ship it,
it is conceivable to paste a few relevant commands into the sources.

To that end, drop all statements |\input{childdoc.def}|
and perform the replacements as outlined below.
Instead of |\childdocmain{|\textit{main}|}| add the following code
to the top of the main file:
%
\begin{center}
\begin{tabular}{l}
|\||ifdefined\childdocname\endinput\||fi\newif\ifchilddoc|\\
|\edef\childdocname{\scantokens\expandafter{\jobname\noexpand}}|\\
|\def\childdocmain{|\textit{main}|}\||ifx\childdocmain\childdocname\||else|\\
|\childdoctrue\includeonly{\childdocname}\let\jobname\childdocmain\||fi|\\
\end{tabular}
\end{center}
%
Instead of |\childdocof{|\textit{main}|}| just include the main file
at the top of each child file:
%
\begin{center}
|\input{|\textit{main}|}|
\end{center}
%
A simple redirection |\childdocforward{|\textit{dest}|}| is achieved by:
%
\begin{center}
|\def\jobname{|\textit{dest}|}\input{\jobname}|
\end{center}
%
The redirection with prefix
|\childdocforwardprefix[|\textit{prefix}|]{|\textit{dest}|}|
is accomplished by:
%
\begin{center}
\begin{tabular}{l}
|{\edef\jobname{\scantokens\expandafter{\jobname\noexpand}}|\\
|\def\redirectjob |\textit{prefix}|#1~~~{\gdef\jobname{|\textit{dest}|#1}}|\\
|\expandafter\redirectjob\jobname~~~}\input{\jobname}|
\end{tabular}
\end{center}

In an alternative approach,
child documents can be compiled by a specific command line
without additional code or specific definitions:
%
\begin{center}
|... -jobname "|\textit{target}|" "|[\textit{flags}]%
|\includeonly{|\textit{dest}|}\input{|\textit{main}|}"|
\end{center}
%

%%%%%%%%%%%%%%%%%%%%%%%%%%%%%%%%%%%%%%%%%%%%%%%%%%%%%%%%%%%%%%%%%%%%%%%%%%%%%%%%
%%%%%%%%%%%%%%%%%%%%%%%%%%%%%%%%%%%%%%%%%%%%%%%%%%%%%%%%%%%%%%%%%%%%%%%%%%%%%%%%
\section{Information}

%%%%%%%%%%%%%%%%%%%%%%%%%%%%%%%%%%%%%%%%%%%%%%%%%%%%%%%%%%%%%%%%%%%%%%%%%%%%%%%%
\subsection{Copyright}

Copyright \copyright{} 2017--2018 Niklas Beisert

This work may be distributed and/or modified under the
conditions of the \LaTeX{} Project Public License, either version 1.3
of this license or (at your option) any later version.
The latest version of this license is in
  \url{http://www.latex-project.org/lppl.txt}
and version 1.3 or later is part of all distributions of \LaTeX{}
version 2005/12/01 or later.

This work has the LPPL maintenance status `maintained'.

The Current Maintainer of this work is Niklas Beisert.

This work consists of the files |README.txt|, |childdoc.ins| and |childdoc.dtx|
as well as the derived files |childdoc.def|, |cdocsamp.tex|
with |cdocsch1.tex|, |cdocsch2.tex|, |cdocspt3.tex|, |cdocspt4.tex|,
|cdocsdrf.tex|, |cdocsfn1.tex|, |cdocsfn2.tex|
as well as |childdoc.pdf|.

%%%%%%%%%%%%%%%%%%%%%%%%%%%%%%%%%%%%%%%%%%%%%%%%%%%%%%%%%%%%%%%%%%%%%%%%%%%%%%%%
\subsection{Files and Installation}

The package consists of the files:
%
\begin{center}
\begin{tabular}{ll}
    |README.txt|   & readme file \\
    |childdoc.ins| & installation file \\
    |childdoc.dtx| & source file \\
    |childdoc.def| & definition file \\
    |cdocsamp.tex| & sample main file \\
    |cdocsch1.tex| & sample include file \\
    |cdocsch2.tex| & sample include file \\
    |cdocspt3.tex| & sample part file \\
    |cdocspt4.tex| & sample part file \\
    |cdocsdrf.tex| & sample redirection file \\
    |cdocsfn1.tex| & sample redirection file \\
    |cdocsfn2.tex| & sample redirection file \\
    |childdoc.pdf| & manual
\end{tabular}
\end{center}
%
The distribution consists of the files
|README.txt|, |childdoc.ins| and |childdoc.dtx|.
%
\begin{itemize}
\item
Run (pdf)\LaTeX{} on |childdoc.dtx|
to compile the manual |childdoc.pdf| (this file).
\item
Run \LaTeX{} on |childdoc.ins| to create the definitions file |childdoc.def|
and the sample |cdocsamp.tex| with include files
|cdocsch1.tex|, |cdocsch2.tex|, |cdocspt3.tex|, |cdocspt4.tex|,
|cdocsdrf.tex|, |cdocsfn1.tex|, |cdocsfn2.tex|.
Then copy the file |childdoc.def| to an appropriate directory of your \LaTeX{}
distribution, e.g.\ \textit{texmf-root}|/tex/latex/childdoc|.
\end{itemize}

%%%%%%%%%%%%%%%%%%%%%%%%%%%%%%%%%%%%%%%%%%%%%%%%%%%%%%%%%%%%%%%%%%%%%%%%%%%%%%%%
\subsection{Related CTAN Packages}

There are several other packages which offer a similar functionality:
%
\begin{itemize}
\item
The packages
\href{http://ctan.org/pkg/docmute}{\textsf{docmute}},
\href{http://ctan.org/pkg/includex}{\textsf{includex}} and
\href{http://ctan.org/pkg/standalone}{\textsf{standalone}}
provide commands to include only the document body of
a child file thus allowing both files to be compiled individually.
\item
The packages \href{http://ctan.org/pkg/subdocs}{\textsf{subdocs}}
and \href{http://ctan.org/pkg/subfiles}{\textsf{subfiles}}
provide structures in which the main and child documents can be
encapsulated and allowing them to be compiled individually.
The inclusion mechanism is different from the conventional |\include|.
\item
The package \href{http://ctan.org/pkg/combine}{\textsf{combine}}
is an elaborate solution to combine several documents into one.
\end{itemize}
%
See also the CTAN topic \href{http://ctan.org/topic/subdocs}{\textsf{subdocs}}
for further related packages.
The present package differs from the above solutions in that
a document structure constructed with the conventional |\include| mechanism
just needs two extra commands at the top of every file
such that all constituent files can be compiled individually.

%%%%%%%%%%%%%%%%%%%%%%%%%%%%%%%%%%%%%%%%%%%%%%%%%%%%%%%%%%%%%%%%%%%%%%%%%%%%%%%%
%\subsection{Feature Suggestions}
%
%The following is a list of features which may be useful for future
%versions of this package:
%%
%\begin{itemize}
%\item
%\ldots
%\end{itemize}

%%%%%%%%%%%%%%%%%%%%%%%%%%%%%%%%%%%%%%%%%%%%%%%%%%%%%%%%%%%%%%%%%%%%%%%%%%%%%%%%
\subsection{Revision History}

%%%%%%%%%%%%%%%%%%%%%%%%%%%%%%%%%%%%%%%%
\paragraph{v2.0:} 2018/12/30

\begin{itemize}
\item
immediate forward processing
\item
added |\childdocby| mechanism
\item
manual restructured
\end{itemize}

%%%%%%%%%%%%%%%%%%%%%%%%%%%%%%%%%%%%%%%%
\paragraph{v1.6:} 2018/01/17

\begin{itemize}
\item
application for development of include files
\item
corrections to manual
\end{itemize}

%%%%%%%%%%%%%%%%%%%%%%%%%%%%%%%%%%%%%%%%
\paragraph{v1.5:} 2017/05/21

\begin{itemize}
\item
more complete structuring introduced
\item
|\childdocof| introduced
\item
|\childdoc| renamed to |\childdocmain|
\item
|\childredirect| renamed to |\childdocforward| and |\childdocforwardprefix|
and functionality expanded
\end{itemize}

%%%%%%%%%%%%%%%%%%%%%%%%%%%%%%%%%%%%%%%%
\paragraph{v1.0:} 2017/04/27

\begin{itemize}
\item
manual and install package
\item
first version published on CTAN
\end{itemize}

%%%%%%%%%%%%%%%%%%%%%%%%%%%%%%%%%%%%%%%%
\paragraph{v0.6:} 2017/04/26

\begin{itemize}
\item
redirection mechanism added
\end{itemize}

%%%%%%%%%%%%%%%%%%%%%%%%%%%%%%%%%%%%%%%%
\paragraph{v0.5:} 2017/04/26

\begin{itemize}
\item
functionality in definition file
\end{itemize}


%%%%%%%%%%%%%%%%%%%%%%%%%%%%%%%%%%%%%%%%%%%%%%%%%%%%%%%%%%%%%%%%%%%%%%%%%%%%%%%%
%%%%%%%%%%%%%%%%%%%%%%%%%%%%%%%%%%%%%%%%%%%%%%%%%%%%%%%%%%%%%%%%%%%%%%%%%%%%%%%%
%%%%%%%%%%%%%%%%%%%%%%%%%%%%%%%%%%%%%%%%%%%%%%%%%%%%%%%%%%%%%%%%%%%%%%%%%%%%%%%%
\appendix

\settowidth\MacroIndent{\rmfamily\scriptsize 000\ }

 \DocInput{childdoc.dtx}

\end{document}
%</driver>
% \fi
%
% %%%%%%%%%%%%%%%%%%%%%%%%%%%%%%%%%%%%%%%%%%%%%%%%%%%%%%%%%%%%%%%%%%%%%%%%%%%%%%
% %%%%%%%%%%%%%%%%%%%%%%%%%%%%%%%%%%%%%%%%%%%%%%%%%%%%%%%%%%%%%%%%%%%%%%%%%%%%%%
% \section{Sample}
%\iffalse
%<*samplemain>
%\fi
%
% The following presents a sample document
% with two chapters, two parts, a title page,
% a compile flag as well as three forwarding files to set the flag.
% It consists of eight |.tex| files:
% \begin{center}
% \begin{tabular}{ll}
% |cdocsamp.tex|&main file\\
% |cdocsch1.tex|&include file for chapter 1\\
% |cdocsch2.tex|&include file for chapter 2\\
% |cdocspt3.tex|&include file for part 3\\
% |cdocspt4.tex|&include file for part 4\\
% |cdocsdrf.tex|&forwarding file for main file in draft mode\\
% |cdocsfi1.tex|&forwarding file for final version of chapter 1\\
% |cdocsfi2.tex|&forwarding file for final version of chapter 2\\
% \end{tabular}
% \end{center}
% Each of the eight files can be compiled directly by the \LaTeX{} compiler.
%
% %%%%%%%%%%%%%%%%%%%%%%%%%%%%%%%%%%%%%%
% \paragraph{Main File.}
%
% The main file is called |cdocsamp.tex|.
%
% Load the \textsf{childdoc} definitions and
% declare the filename for the main document:
%    \begin{macrocode}
\input{childdoc.def}
\childdocmain{}
%    \end{macrocode}

% Optional override for |\version| flag:
%    \begin{macrocode}
%%\ifchilddoc\else\providecommand{\version}{draft}\fi
%    \end{macrocode}

% Define the default values for the |\version| flag
% (|final| for the main file and |draft| for childs):
%    \begin{macrocode}
\ifchilddoc
\providecommand{\version}{draft}
\else
\providecommand{\version}{final}
\fi
%    \end{macrocode}

% Load the standard document class:
%    \begin{macrocode}
\documentclass[12pt]{article}
%    \end{macrocode}

% Start the document body:
%    \begin{macrocode}
\begin{document}
%    \end{macrocode}

% Declare a title page.
% Print title, part of document being processed and version flag:
%    \begin{macrocode}
\addtocounter{page}{-1}
\begin{center}
{\LARGE\bfseries{}childdoc example\par}
\vspace{1cm}
\ifchilddoc
\ifchilddocmanual part\else chapter\fi:
`\childdocname' of `\childdocjob'\par
\else
main document: `\childdocjob'\par
\fi
version: \version\par
\end{center}
\newpage
%    \end{macrocode}

% Manually include selected file,
% otherwise process as usual:
%    \begin{macrocode}
\ifchilddocmanual
\section*{part `\childdocname'}
\input{\childdocname}
\else
%    \end{macrocode}

% Include the two chapters:
%    \begin{macrocode}
\include{cdocsch1}
\include{cdocsch2}
%    \end{macrocode}

% Include the two parts unless only chapters should be displayed:
%    \begin{macrocode}
\ifchilddoc\else
\section{part three}
\input{cdocspt3}
\section{part four}
\input{cdocspt4}
\fi
%    \end{macrocode}

% Process as usual until here:
%    \begin{macrocode}
\fi
%    \end{macrocode}

% End of document body:
%    \begin{macrocode}
\end{document}
%    \end{macrocode}
%\iffalse
%</samplemain>
%\fi
%
% %%%%%%%%%%%%%%%%%%%%%%%%%%%%%%%%%%%%%%
% \paragraph{Chapter Include Files.}
%
% The include files are called |cdocsch1.tex| and |cdocsch2.tex|.
%
%\iffalse
%<*samplechap1|samplechap2>
%\fi

% Optional override for |\version| flag:
%    \begin{macrocode}
%%\providecommand{\version}{final}
%    \end{macrocode}

% Include the main document:
%    \begin{macrocode}
\input{childdoc.def}
\childdocof{cdocsamp}
%    \end{macrocode}

%\iffalse
%</samplechap1|samplechap2>
%\fi
%
%\iffalse
%<*samplechap1>
%\fi
% Some text for chapter 1:
%    \begin{macrocode}
\section{one}
some text in chapter one
%    \end{macrocode}

%\iffalse
%</samplechap1>
%\fi
% Some text for chapter 2:
%\iffalse
%<*samplechap2>
%\fi
%    \begin{macrocode}
\section{two}
more text in chapter two
%    \end{macrocode}

%\iffalse
%</samplechap2>
%\fi
%
% %%%%%%%%%%%%%%%%%%%%%%%%%%%%%%%%%%%%%%
% \paragraph{Part Include Files.}
%
% The include files are called |cdocspt3.tex| and |cdocspt4.tex|.
%
%\iffalse
%<*samplepart3|samplepart4>
%\fi

% Optional override for |\version| flag:
%    \begin{macrocode}
%%\providecommand{\version}{final}
%    \end{macrocode}

% Include the main document:
%    \begin{macrocode}
\input{childdoc.def}
\childdocby{cdocsamp}
%    \end{macrocode}

%\iffalse
%</samplepart3|samplepart4>
%\fi
%
%\iffalse
%<*samplepart3>
%\fi
% Some text for part 3:
%    \begin{macrocode}
some text in part three
%    \end{macrocode}

%\iffalse
%</samplepart3>
%\fi
% Some text for part 4:
%\iffalse
%<*samplepart4>
%\fi
%    \begin{macrocode}
more text in part four
%    \end{macrocode}

%\iffalse
%</samplepart4>
%\fi
%
% %%%%%%%%%%%%%%%%%%%%%%%%%%%%%%%%%%%%%%
% \paragraph{Forwarding for a Complete Draft.}
%
% The following forwarding file |cdocsdrf.tex|
% compiles the main document in draft mode:
%\iffalse
%<*sampledraft>
%\fi
%    \begin{macrocode}
\def\version{draft}
\input{childdoc.def}
\childdocforward{cdocsamp}
%    \end{macrocode}

%\iffalse
%</sampledraft>
%\fi
%
% %%%%%%%%%%%%%%%%%%%%%%%%%%%%%%%%%%%%%%
% \paragraph{Forwarding for Final Version of the Chapters.}
%
% The following forwarding files |cdocsfn1.tex| and |cdocsfn2.tex|
% (with identical content)
% compile the final versions of the child documents
% |cdocsch1.tex| and |cdocsch2.tex|, respectively:
%\iffalse
%<*samplefinal>
%\fi
%    \begin{macrocode}
\def\version{final}
\input{childdoc.def}
\childdocforwardprefix[cdocsamp]{cdocsfn}{cdocsch}
%    \end{macrocode}

%\iffalse
%</samplefinal>
%\fi
%
% %%%%%%%%%%%%%%%%%%%%%%%%%%%%%%%%%%%%%%
% \paragraph{Command Line Processing.}
%
% The following three command lines generate the output files
% |cdocscld|, |cdocscl1| and |cdocscl2|
% which should be identical to
% |cdocsdrf|, |cdocsch1| and |cdocsfn2|, respectively:
% \begin{center}
% \begin{tabular}{l}
% |latex -jobname cdocscld \|\\
% |  "\def\version{draft}\input{childdoc.def}\childdocforward{cdocsamp}"|\\
% |latex -jobname cdocscl1 \|\\
% |  "\input{childdoc.def}\childdocforward[cdocsamp]{cdocsch1}"|\\
% |latex -jobname cdocscl2 \|\\
% |  "\def\version{final}\input{childdoc.def}\childdocforward{cdocsch2}"|
% \end{tabular}
% \end{center}
% Note that the trailing backslash on each first line
% merely continues the input to the second line
% (for convenient cut ant paste).
% Furthermore, the command |latex| can be replaced by any
% of its alternative versions such as |pdflatex|.
%
% %%%%%%%%%%%%%%%%%%%%%%%%%%%%%%%%%%%%%%%%%%%%%%%%%%%%%%%%%%%%%%%%%%%%%%%%%%%%%%
% %%%%%%%%%%%%%%%%%%%%%%%%%%%%%%%%%%%%%%%%%%%%%%%%%%%%%%%%%%%%%%%%%%%%%%%%%%%%%%
% \section{Implementation}
%\iffalse
%<*package>
%\fi
%
% This section describes the definitions file |childdoc.def|.

% The definitions cannot be loaded using |\usepackage| or |\RequirePackage|
% which has a mechanism to prevent loading a style file more than once.
% When loading the definitions by means of |\input|
% multiple instances have to be prevented manually:
%\iffalse
%This code needs to be before the `\ProvidesFile' directive
%which is defined at the beginning of this file.
%Therefore it is also placed there and commented out here.
%</package>
%<*discard>
%\fi
%    \begin{macrocode}
\ifdefined\childdocmain\endinput\fi
%    \end{macrocode}
%\iffalse
%</discard>
%<*package>
%\fi
%
% \macro{\ifchilddoc}
% \macro{\ifchilddocmanual}
% The conditional |\ifchilddoc| tells whether a
% child (true) or main (false) document is being compiled.
% The conditional |\ifchilddocmanual| tells whether
% the |\includeonly| mechanism is used (false) or
% the selection of child files must be performed manually (true).
% The definitions initialise to false:
%    \begin{macrocode}
\newif\ifchilddoc
\newif\ifchilddocmanual
%    \end{macrocode}

% \macro{\childdocname}
% \macro{\childdocjob}
% The macro |\childdocname| stores the name of the main document
% to be compiled. The macro |\childdocjob| stores the name of
% the document on which the \LaTeX{} compiler was originally invoked.
% The content of |\jobname| cannot be compared
% to filenames specified in the source due to different catcodes.
% The following code rescans |\jobname|, stores the result
% in |\childdocname| and saves a copy in |\childdocjob|:
%    \begin{macrocode}
\edef\childdocname{\scantokens\expandafter{\jobname\noexpand}}
\let\childdocjob\childdocname
%    \end{macrocode}

% \macro{\childdocdisable}
% The macro |\childdocdisable| prevents the main file
% from being processed more than once.
% At this stage, the main document command |\childdocmain|
% is assumed to be called once again where it should do nothing.
% Any subsequent call to it should prevent
% a secondary processing of the main document
% It overwrites the forwarding commands
% |\childdocof| and |\childdocforward|
% with empty macros to prevent further inclusions of the main document:
%    \begin{macrocode}
\newcommand{\childdocdisable}
{
  \renewcommand{\childdocmain}[1]{\renewcommand{\childdocmain}[1]{\endinput}}
  \renewcommand{\childdocof}[1]{}
  \renewcommand{\childdocby}[2][]{}
  \renewcommand{\childdocforward}[2][]{}
  \renewcommand{\childdocdisable}{}
}
%    \end{macrocode}

% \macro{\childdocmain}
% The macro |\childdocmain| is to be called at the top of the main file
% with nothing or the main filename (without extension) as argument.
% First, it breaks loops.
% If the argument is not empty and does not match |\childdocname|
% (which is set by the first inclusion of |childdoc.def|),
% |\ifchilddoc| is set to true, |\includeonly| is applied to the child file
% and |\jobname| is set to the main file
% (for proper handling of |.aux| files):
%    \begin{macrocode}
\newcommand{\childdocmain}[1]
{
  \childdocdisable\childdocmain{}
  \if?#1?\else
    \begingroup
      \def\childdoctmp{#1}
      \ifx\childdoctmp\childdocname
        \def\childdoctmp{}
      \else
        \def\childdoctmp
        {
          \childdoctrue
          \includeonly{\childdocname}
          \def\childdocjob{#1}
          \def\jobname{#1}
        }
      \fi
      \expandafter
    \endgroup
    \childdoctmp
  \fi
}
%    \end{macrocode}

% \macro{\childdocof}
% The command |\childdocof| redirects
% compilation to the main file |#1|.
%    \begin{macrocode}
\newcommand{\childdocof}[1]
{
  \childdocdisable
  \childdoctrue
  \includeonly{\childdocname}
  \def\jobname{#1}
  \def\childdocjob{#1}
  \input{#1}
}
%    \end{macrocode}

% \macro{\childdocby}
% The command |\childdocby| ....
%    \begin{macrocode}
\newcommand{\childdocby}[2][]
{
  \childdocdisable
  \childdoctrue
  \childdocmanualtrue
  \if?#1?\else
    \def\jobname{#2}
  \fi
  \def\childdocjob{#2}
  \input{#2}
  \endinput
}
%    \end{macrocode}

% \macro{\childdocforward}
% The command |\childdocforward| redirects
% compilation to the main file or
% (if the optional argument is given) a child file.
% Parameters are set as if the main file
% or a child file starting with |\childdocof| was compiled.
% Then compilation is handed over to the main file:
%    \begin{macrocode}
\newcommand{\childdocforward}[2][]
{
  \begingroup
    \if?#1?
      \def\childdoctmp
      {
        \def\childdocname{#2}
        \def\childdocjob{#2}
        \def\jobname{#2}
        \input{#2}
        \endinput
      }
    \else
      \def\childdoctmp
      {
        \childdocdisable
        \def\childdocname{#2}
        \childdoctrue
        \includeonly{#2}
        \def\childdocjob{#1}
        \def\jobname{#1}
        \input{#1}
        \endinput
      }
    \fi
    \expandafter
  \endgroup
  \childdoctmp
}
%    \end{macrocode}

% \macro{\childdocforwardprefix}
% The command |\childdocforwardprefix| redirects
% compilation to the main or a child file by means of a pattern.
% The prefix |#1| in the current filename is replaced by |#2|
% and the suffix of the current filename is kept
% (it is assumed that the filename does not contain the substring `|~~~|'
% which is used as a delimiter).
% Compilation is handed over to the new file by |\childdocforward|:
%    \begin{macrocode}
\newcommand{\childdocforwardprefix}[3][]
{
  \begingroup
    \def\childdocextract #2##1~~~{\def\childdoctmp{\childdocforward[#1]{#3##1}}}
    \expandafter\childdocextract\childdocname~~~
    \expandafter
  \endgroup
  \childdoctmp
}
%    \end{macrocode}

% \macro{\childdoc}
% The deprecated macro |\childdoc| is a legacy version of |\childdocmain|:
%    \begin{macrocode}
\newcommand{\childdoc}{\childdocmain}
%    \end{macrocode}

% \macro{\childdocredirect}
% The deprecated macro |\childdocredirect| is a legacy version
% of |\childdocforward| and |\childdocforwardprefix|:
%    \begin{macrocode}
\newcommand{\childdocredirect}[2][]
{
  \begingroup
    \if?#1?
      \def\childdoctmp{\childdocforward{#2}}
    \else
      \def\childdoctmp{\childdocforwardprefix{#1}{#2}}
    \fi
    \expandafter
  \endgroup
  \childdoctmp
}
%    \end{macrocode}

%\iffalse
%</package>
%\fi
%
\endinput
|\\
|\childdocforward[|\textit{main}|]{|\textit{dest}|}|\\
\end{tabular}
\end{center}
%
The argument \textit{dest} is the destination file
(without extension).
It should be the main file or one of the child files.
Note that further \textsf{childdoc} directives
such as |\childdocof| and |\childdocforward|
in the indicated file will be processed in this form.
The optional argument \textit{main}
passes on directly to the main file \textit{main}
while pretending to compile the child \textit{dest}.
This form behaves as if \textit{dest}
issues |\childdocof{|\textit{main}|}| right away,
and no further \textsf{childdoc} directives will be processed.

%%%%%%%%%%%%%%%%%%%%%%%%%%%%%%%%%%%%%%%%
\DescribeMacro{\...prefix}
In the alternative form |\childdocforwardprefix|,
%
\begin{center}
\begin{tabular}{l}
|% \iffalse
%
% childdoc.dtx Copyright (C) 2017-2018 Niklas Beisert
%
% This work may be distributed and/or modified under the
% conditions of the LaTeX Project Public License, either version 1.3
% of this license or (at your option) any later version.
% The latest version of this license is in
%   http://www.latex-project.org/lppl.txt
% and version 1.3 or later is part of all distributions of LaTeX
% version 2005/12/01 or later.
%
% This work has the LPPL maintenance status `maintained'.
%
% The Current Maintainer of this work is Niklas Beisert.
%
% This work consists of the files childdoc.dtx and childdoc.ins
% and the derived files childdoc.def and cdocsamp.tex with
% cdocsch1.tex, cdocsch2.tex, cdocsdrf.tex, cdocsfn1.tex, cdocsfn2.tex.
%
%<package>\ifdefined\childdocmain\endinput\fi
%<package>\ProvidesFile{childdoc.def}[2018/12/30 v2.0 child document driver]
%<samplemain>\ProvidesFile{cdocsamp.tex}[2018/12/30 v2.0 sample for childdoc]
%<*driver>
%\ProvidesFile{childdoc.drv}[2018/12/30 v2.0 childdoc reference manual file]
\PassOptionsToClass{10pt,a4paper}{article}
\documentclass{ltxdoc}

\usepackage[margin=35mm]{geometry}
\usepackage{hyperref}
\usepackage{hyperxmp}
\usepackage[usenames]{color}

\hypersetup{colorlinks=true}
\hypersetup{pdfstartview=FitH}
\hypersetup{pdfpagemode=UseNone}
\hypersetup{pdfsource={}}
\hypersetup{pdflang={en-UK}}
\hypersetup{pdfcopyright={Copyright 2017-2018 Niklas Beisert.
  This work may be distributed and/or modified under the
  conditions of the LaTeX Project Public License, either version 1.3
  of this license or (at your option) any later version.}}
\hypersetup{pdflicenseurl={http://www.latex-project.org/lppl.txt}}
\hypersetup{pdfcontactaddress={ETH Zurich, ITP, HIT K,
  Wolfgang-Pauli-Strasse 27}}
\hypersetup{pdfcontactpostcode={8093}}
\hypersetup{pdfcontactcity={Zurich}}
\hypersetup{pdfcontactcountry={Switzerland}}
\hypersetup{pdfcontactemail={nbeisert@itp.phys.ethz.ch}}
\hypersetup{pdfcontacturl={http://people.phys.ethz.ch/\xmptilde nbeisert/}}

\newcommand{\secref}[1]{\hyperref[#1]{section \ref*{#1}}}

\parskip1ex
\parindent0pt
\let\olditemize\itemize
\def\itemize{\olditemize\parskip0pt}

\begin{document}

\title{The \textsf{childdoc} Package}
\hypersetup{pdftitle={The childdoc Package}}
\author{Niklas Beisert\\[2ex]
  Institut f\"ur Theoretische Physik\\
  Eidgen\"ossische Technische Hochschule Z\"urich\\
  Wolfgang-Pauli-Strasse 27, 8093 Z\"urich, Switzerland\\[1ex]
  \href{mailto:nbeisert@itp.phys.ethz.ch}
  {\texttt{nbeisert@itp.phys.ethz.ch}}}
\hypersetup{pdfauthor={Niklas Beisert}}
\hypersetup{pdfsubject={Manual for the LaTeX2e Package childdoc}}
\date{30 December 2018, \textsf{v2.0}}
\maketitle

\begin{abstract}\noindent
\textsf{childdoc} is a \LaTeXe{} package
that enables the direct compilation
of document sections included by |\include|
to individual files.
\end{abstract}

\begingroup
\parskip0ex
\tableofcontents
\endgroup

%%%%%%%%%%%%%%%%%%%%%%%%%%%%%%%%%%%%%%%%%%%%%%%%%%%%%%%%%%%%%%%%%%%%%%%%%%%%%%%%
%%%%%%%%%%%%%%%%%%%%%%%%%%%%%%%%%%%%%%%%%%%%%%%%%%%%%%%%%%%%%%%%%%%%%%%%%%%%%%%%
\section{Introduction}

\LaTeX{} provides a mechanism to structure a large document (such as a book)
into a main file and several child files (containing the chapters)
using the |\include| command.
This mechanism is beneficial for documents
which span hundreds of pages in order to
make the source file(s) more manageable.
Moreover, compilation can be restricted to
selected child files by means of the |\includeonly| command.
The latter feature can be used to reduce the compilation time while editing
(this was significantly more useful in the earlier days of \LaTeX{})
or to generate a smaller document which is easier to navigate.
Another application of |\includeonly| is to generate
documents consisting of selected parts of the complete document.

However, there are a few drawbacks of the plain |\include| mechanism:
\begin{itemize}
\item
The child files cannot be compiled on their own,
they can only be compiled via the main file.
A naive editing environment
(such as a text editor with an option
to have the current file processed by \LaTeX)
may require one to switch to the main file before compiling;
attempting to compile the child file produces errors.
\item
The main file must be modified (each time)
to adjust the |\includeonly| command
to the present needs. This easily leaves the main file in a messy state.
\item
The generated document will always carry the filename
of the main document. This is inconvenient if
several child files are to be compiled and
to be kept for distribution.
\end{itemize}

The present package provides a simple interface
to make child files individually compilable by \LaTeX{}.
Compiling a child file then has the same effect as compiling
the main file with an |\includeonly| command
to select the appropriate child.
Moreover the generated document will carry the name of the child
rather than the main file.
This resolves all three above issues.

This feature is meant to make the editing of books,
thesis documents and lecture notes somewhat more convenient.
However, the package can also be used efficiently for
composing a series of documents (such as exercise sheets)
which are typically distributed individually.
It then assists the author in generating the individual documents
(potentially in different versions)
as well as a document containing the collected series.
Another application is in developing style files
or other kinds of included material
where compilation of the style file could redirect
to a sample or test file.

%%%%%%%%%%%%%%%%%%%%%%%%%%%%%%%%%%%%%%%%%%%%%%%%%%%%%%%%%%%%%%%%%%%%%%%%%%%%%%%%
%%%%%%%%%%%%%%%%%%%%%%%%%%%%%%%%%%%%%%%%%%%%%%%%%%%%%%%%%%%%%%%%%%%%%%%%%%%%%%%%
\section{Usage}

First of all, the package \textsf{childdoc} is \emph{not} a standard
\LaTeXe{} |.sty| style file! Therefore it needs to be invoked in
a non-standard way.

%%%%%%%%%%%%%%%%%%%%%%%%%%%%%%%%%%%%%%%%%%%%%%%%%%%%%%%%%%%%%%%%%%%%%%%%%%%%%%%%
\subsection{Included Files}
\label{sec:include}

%%%%%%%%%%%%%%%%%%%%%%%%%%%%%%%%%%%%%%%%
\DescribeMacro{\childdocmain}
To use the package, add the commands
\begin{center}
\begin{tabular}{l}
|\input{childdoc.def}|\\
|\childdocmain{}|\\
\end{tabular}
\end{center}
at the very top of the main \LaTeX{} file,
in particular \emph{before} the |\documentclass| statement!
The argument of |\childdocmain| should be left empty
(but it must be present).

%%%%%%%%%%%%%%%%%%%%%%%%%%%%%%%%%%%%%%%%
\DescribeMacro{\childdocof}
Furthermore, add the commands
\begin{center}
\begin{tabular}{l}
|\input{childdoc.def}|\\
|\childdocof{|\textit{main}|}|\\
\end{tabular}
\end{center}
at the top of every child file \textit{child}
which is included by |\include{|\textit{child}|}|
from within the main file
(or at least for those files to be compiled individually).
The argument \textit{main} must be the filename of the main file.

There are a couple of
considerations in setting up the main and child documents:

%%%%%%%%%%%%%%%%%%%%%%%%%%%%%%%%%%%%%%%%
\paragraph{Restrictions.}

Please note the following restrictions:
\begin{itemize}
\item
|\childdocmain| must be called with one argument \textit{main}
to ensure compatibility with earlier version of the package.
It must either be empty (|\childdocmain{}|)
or precisely match the filename of the main file in which it is specified.
See \secref{sec:detection} for further information.
\item
The filename \textit{main} must be specified without the |.tex| extension.
\item
The filename \textit{main} is case sensitive
(even in case-insensitive file systems)
due to internal string comparison.
\item
The argument \textit{main} should be fully expanded, it cannot be a macro.
\item
Subdirectories and special characters should be avoided in filenames.
\item
The command |\childdocmain{|\textit{main}|}| must be followed by a whitespace.
It should not be followed immediately by another command
or by a comment mark `|%|'.
This is because the \TeX{} parser reads the token immediately following
the argument of |\childdocmain| and puts it
at the beginning of every child section;
however, a white\-space is ignored.
\end{itemize}

%%%%%%%%%%%%%%%%%%%%%%%%%%%%%%%%%%%%%%%%
\paragraph{Content of Main File.}

It is advisable to place all content in the child files included by |\include|.
Any output contained in the main file will appear in all child documents
unless suppressed manually;
it cannot be suppressed automatically by the |\includeonly| directive
and thus should normally be avoided.
A method to include some content in the main file
by means of conditional processing is described in \secref{sec:conditional}.

%%%%%%%%%%%%%%%%%%%%%%%%%%%%%%%%%%%%%%%%
\paragraph{Page Numbering.}

When only a part of the document is compiled,
the appropriate numbering of pages
(as well as other status parameters)
is determined from the |.aux| files.
The latter contain information from previous passes.
However this information needs to propagate through
all intermediate child documents.
Therefore the page numbering in child documents may well
be inconsistent until the complete document is compiled at least once.

A useful (if unconventional) way to always ensure a consistent
page numbering is to restart the numbering in each child document
and denote the pages by `\textit{child}|.|\textit{page}'
where \textit{child} represents the chapter/section number of the child file.
This can be achieved by the command
|\numberwithin{page}{|\textit{child}|}|
of the \textsf{amsmath} package
where \textit{child} can be |chapter| or |section|
depending on the chosen structuring.
Alternatively, one can modify the macro |\thepage| appropriately
and reset the counter |page| at the start of each child file.

%%%%%%%%%%%%%%%%%%%%%%%%%%%%%%%%%%%%%%%%%%%%%%%%%%%%%%%%%%%%%%%%%%%%%%%%%%%%%%%%
\subsection{Conditional Processing}
\label{sec:conditional}

The package provides a mechanism to compile different versions
of a document. To customise the versions further some conditional processing
can come in handy to distinguish which version is being compiled.
The package provides two macros to describe the compilation context:

%%%%%%%%%%%%%%%%%%%%%%%%%%%%%%%%%%%%%%%%
\DescribeMacro{\ifchilddoc}
The conditional |\ifchilddoc| distinguishes between the compilation of
child documents and the main document:
%
\begin{center}
|\ifchilddoc |\textit{child-code}| |[|\||else |\textit{main-code}]| \||fi|
\end{center}

%%%%%%%%%%%%%%%%%%%%%%%%%%%%%%%%%%%%%%%%
\DescribeMacro{\childdocname}
\DescribeMacro{\childdocjob}
The macro |\childdocname| contains the filename (without extension)
of the main or child file being processed.
Note that |\childdocjob| will always contain the name of the main file.

%%%%%%%%%%%%%%%%%%%%%%%%%%%%%%%%%%%%%%%%
\paragraph{Title Page.}

Conditional processing can be used to include a title or banner page
in the main document when proper precautions are taken.
Importantly, the code in the main file should ensure that the page counter
(as well as other status parameters which are stored in the |.aux| files)
takes the same value after the conditional processing.
Otherwise the page numbers may take divergent values
depending on which part is compiled.

For example, a title page could be declared by:
%
\begin{center}
\begin{tabular}{l}
|\ifchilddoc\||else|\\
|\addtocounter{page}{-1}|\\
\textit{code for title page}\\
|\newpage|\\
|\||fi|
\end{tabular}
\end{center}
%
A banner page for the child documents can be generated by:
%
\begin{center}
\begin{tabular}{l}
|\ifchilddoc|\\
|\addtocounter{page}{-1}|\\
\textit{code for banner page}\\
|\newpage|\\
|\||fi|
\end{tabular}
\end{center}
%
Here one could write a message such as:
\begin{center}
|This is the part \childdocname{} of \childdocjob{}.|
\end{center}

%%%%%%%%%%%%%%%%%%%%%%%%%%%%%%%%%%%%%%%%%%%%%%%%%%%%%%%%%%%%%%%%%%%%%%%%%%%%%%%%
\subsection{Flags}
\label{sec:flags}

The package makes it easy to generate different versions
of the main or child documents.
To this end compilation flags can be defined
and assigned different default values.
They will be particularly useful in conjunction
with the forwarding mechanism described in \secref{sec:forward}.

For example, it may be useful to have a flag |\version|
which can be set to |draft| or |final|.
The document source will contain some conditional code
depending on the value of |\version|.
Suppose further, the flag should default to |final| for the main file
and to |draft| for child files
which is a natural assignment for editing the document.
This is achieved by placing the following code
in the preamble of the main document
(below the |\childdocmain| directive):
%
\begin{center}
\begin{tabular}{l}
|\ifchilddoc|\\
|\providecommand{\version}{draft}|\\
|\||else|\\
|\providecommand{\version}{final}|\\
|\||fi|
\end{tabular}
\end{center}
%
The definition by |\providecommand| makes sure
that previous definitions are not overwritten.
Further statements |\providecommand{\version}{...}|
can thus be added before the above code to override it.

For the main file, one might add a line
(between |\childdocmain| and the above block)
%
\begin{center}
|%\ifchilddoc\||else\providecommand{\version}{draft}\||fi|
\end{center}
%
which can be uncommented to produce a draft version.
Likewise one can add a line to the very top of a child file
(above the |\childdocof{|\textit{main}|}| directive)
%
\begin{center}
|%\providecommand{\version}{final}|
\end{center}
%
which can be uncommented to produce the final version of this child document.

%%%%%%%%%%%%%%%%%%%%%%%%%%%%%%%%%%%%%%%%%%%%%%%%%%%%%%%%%%%%%%%%%%%%%%%%%%%%%%%%
\subsection{Forwarding}
\label{sec:forward}

Different versions of the main or child documents
using compilation flags as described in \secref{sec:flags}
can be (permanently) stored in different files
for convenient compilation, viewing and distribution.
To this end, the package defines a command
to pass on compilation to a different file:

%%%%%%%%%%%%%%%%%%%%%%%%%%%%%%%%%%%%%%%%
\DescribeMacro{\childdocforward}
The command |\childdocforward| redirects processing to
another source file:
%
\begin{center}
\begin{tabular}{l}
|\input{childdoc.def}|\\
|\childdocforward[|\textit{main}|]{|\textit{dest}|}|\\
\end{tabular}
\end{center}
%
The argument \textit{dest} is the destination file
(without extension).
It should be the main file or one of the child files.
Note that further \textsf{childdoc} directives
such as |\childdocof| and |\childdocforward|
in the indicated file will be processed in this form.
The optional argument \textit{main}
passes on directly to the main file \textit{main}
while pretending to compile the child \textit{dest}.
This form behaves as if \textit{dest}
issues |\childdocof{|\textit{main}|}| right away,
and no further \textsf{childdoc} directives will be processed.

%%%%%%%%%%%%%%%%%%%%%%%%%%%%%%%%%%%%%%%%
\DescribeMacro{\...prefix}
In the alternative form |\childdocforwardprefix|,
%
\begin{center}
\begin{tabular}{l}
|\input{childdoc.def}|\\
|\childdocforwardprefix[|\textit{main}|]{|\textit{prefix}|}{|\textit{dest}|}|
\end{tabular}
\end{center}
%
the destination file is determined by a pattern
depending on the current file:
To make this work, the current file must be called
`{\textit{prefix}\hspace{0.2em}\textit{suffix}}'
with \textit{prefix} matching precisely the argument.
Processing is then passed on to the file
`{\textit{dest}\hspace{0.2em}\textit{suffix}}'.
Surely, the same effect is achieved by
directly specifying the
argument `{\textit{dest}\hspace{0.2em}\textit{suffix}}'
in the first form.
However, that requires to set up a different file
for each child. With the alternative form of the command
all these files can have exactly the same content
which simplifies setting them up and maintaining them.

For example, the following file |draft.tex|
with a compilation flag |\version| as described in \secref{sec:flags}
compiles the main document as a draft:
%
\begin{center}
\begin{tabular}{l}
|\def\version{draft}|\\
|\input{childdoc.def}|\\
|\childdocforward{|\textit{main}|}|
\end{tabular}
\end{center}
%
Likewise, the following files |final|\textit{nn}|.tex|
compile the final version of the child document
|child|\textit{nn}|.tex|:
%
\begin{center}
\begin{tabular}{l}
|\def\version{final}|\\
|\input{childdoc.def}|\\
|\childdocforwardprefix{final}{child}|
\end{tabular}
\end{center}
%

Note that when several versions of a main file and/or of each child file
are to be generated, it may be convenient to set up a |Makefile| or
shell script to automatise the process.

%%%%%%%%%%%%%%%%%%%%%%%%%%%%%%%%%%%%%%%%%%%%%%%%%%%%%%%%%%%%%%%%%%%%%%%%%%%%%%%%
\subsection{Command Line Processing}
\label{sec:commandline}

The effect of redirection files can also be achieved by invoking
the \LaTeX{} compiler with a more elaborate command line.
Most conveniently this should be done as part
of a shell script or a |Makefile|.

When using \textsf{childdoc} in the main file, the following
command lines effectively perform a redirection
(note that depending on the shell being used,
backslashes may have to be doubled: `|\|' $\to$ `|\\|'):
%
\begin{center}
|... -jobname "|\textit{target}|" |\\|"|[\textit{flags}]%
|\input{childdoc.def}\childdocforward[|\textit{main}|]{|\textit{dest}|}"|
\end{center}
%
Here \textit{target} is the name of the output file,
\textit{main} is the name of the main file
and \textit{dest} is the name of the main or child file to be processed
(all filenames without extensions).
The optional argument \textit{main} can be omitted
if \textit{main} matches \textit{dest}.
Optionally, compilation \textit{flags} can be defined via |\def| commands.
This command line makes the \TeX{} engine believe
it is compiling the file \textit{target}
whose content is specified as the latter parameter.
The provided code then forwards the processing to
\textit{main} or \textit{dest} as described in \secref{sec:forward}.

%%%%%%%%%%%%%%%%%%%%%%%%%%%%%%%%%%%%%%%%%%%%%%%%%%%%%%%%%%%%%%%%%%%%%%%%%%%%%%%%
\subsection{Include by Input}
\label{sec:input}

Including child documents by |\include| has some restrictions by design.
Most notably, the content of a child document always occupies
its own set of pages; pages cannot be shared between child documents.
Usually, this behaviour makes perfect sense
because each child document contain an essential part of the document.
However, in some situations it may be desirable to compose
a document from a collection of parts
without having mandatory page breaks between then.
For this case, the package
provides a mechanism to include parts
by |\input| which can also be processed individually.
However, by construction this mechanism
requires manual handling of the content to be output.

%%%%%%%%%%%%%%%%%%%%%%%%%%%%%%%%%%%%%%%%
\DescribeMacro{\ifchilddocmanual}
The main file should be prepared as usual, see \secref{sec:include}.
However, the document body must make a distinction
between processing of an individual part and of the main document, e.g.:
%
\begin{center}
\begin{tabular}{l}
|\ifchilddocmanual|\\
|\input{\childdocname}|\\
|\||else|\\
\textit{document body with }|\input{|\textit{part}|}|\\
|\||fi|
\end{tabular}
\end{center}
%
The conditional |\ifchilddocmanual| is true whenever
a part to be included by |\input| is being compiled,
and the name of the part is stored in |\childdocname|.

%%%%%%%%%%%%%%%%%%%%%%%%%%%%%%%%%%%%%%%%
\DescribeMacro{\childdocby}
Each part to be included by |\input| should start with:
%
\begin{center}
\begin{tabular}{l}
|\input{childdoc.def}|\\
|\childdocby{|\textit{main}|}|\\
\end{tabular}
\end{center}
%
The directive |\childdocby| is similar to |\childdocof|
described in \secref{sec:include},
but the subsequent selection of content must be done manually.
To that end, both |\ifchilddoc| and |\ifchilddocmanual|
will be true upon processing of a part,
and the name of the part is stored in |\childdocname|.
Note that |\jobname| will be set to the filename of the current part
so that each part receives an individual |.aux| file
that does not interfere with the |.aux| file(s) of the main document.
This behaviour can be altered by the alternative form
|\childdocby[*]{|\textit{main}|}| (with a non-empty optional argument)
which uses the |.aux| file of the main document
by setting |\jobname| to \textit{main}.

%%%%%%%%%%%%%%%%%%%%%%%%%%%%%%%%%%%%%%%%%%%%%%%%%%%%%%%%%%%%%%%%%%%%%%%%%%%%%%%%
\subsection{Driver Development}
\label{sec:driver}

The \textsf{childdoc} mechanism can also be use for the development
of definition files such as \LaTeX{} styles or classes.
This case differs from the above setup with multiple parts
included by |\include| in that no |\includeonly| should be invoked.
This can be achieved by starting the include file
(before |\ProvidesPackage|) with:
%
\begin{center}
\begin{tabular}{l}
|\input{childdoc.def}|\\
|\childdocforward{|\textit{main}|}|\\
\end{tabular}
\end{center}
%
or alternatively with:
%
\begin{center}
\begin{tabular}{l}
|\input{childdoc.def}|\\
|\childdocby{|\textit{main}|}|\\
\end{tabular}
\end{center}
%
Both forms have slightly different effects as described above.
The main file is prepared as usual, see \secref{sec:include}.

%%%%%%%%%%%%%%%%%%%%%%%%%%%%%%%%%%%%%%%%%%%%%%%%%%%%%%%%%%%%%%%%%%%%%%%%%%%%%%%%
\subsection{Legacy Detection}
\label{sec:detection}

The directive |\childdocmain| in the main file can detect
whether the complete document or merely a child is to be compiled
even without using the directive |\childdocof|.
This method is deprecated because it is less robust
and there is no compelling reason to use it;
it is merely provided for backward compatibility
and it may be removed in future versions.

If the detection mechanism is to be used,
it is mandatory to correctly specify
the filename of the main file as the argument of |\childdocmain|:
%
\begin{center}
\begin{tabular}{l}
|\input{childdoc.def}|\\
|\childdocmain{|\textit{main}|}|\\
\end{tabular}
\end{center}
%
If |\jobname| does not match the argument \textit{main} of |\childdocmain|,
it is assumed that |\jobname| points to the child file to be compiled.
When using |\childdocmain| with the main file specified as argument,
it suffices to start a child file
with just |\input{|\textit{main}|}|
without loading of the package and using |\childdocof|.
If instead all processing is done
with the appropriate \textsf{childdoc} directives,
the argument of \textit{main} of |\childdocmain| can be empty.

An alternative version of the command line processing described
in \secref{sec:commandline} using the detection mechanism reads:
%
\begin{center}
|... -jobname "|\textit{target}|" "|[\textit{flags}]%
[|\def\jobname{|\textit{dest}|}|]|\input{|\textit{main}|}"|
\end{center}

%%%%%%%%%%%%%%%%%%%%%%%%%%%%%%%%%%%%%%%%%%%%%%%%%%%%%%%%%%%%%%%%%%%%%%%%%%%%%%%%
\subsection{Manual Code}
\label{sec:manual}

In case one cannot be certain whether the definitions file |childdoc.def|
is installed on the target \TeX{} distribution
and one prefers not to ship it,
it is conceivable to paste a few relevant commands into the sources.

To that end, drop all statements |\input{childdoc.def}|
and perform the replacements as outlined below.
Instead of |\childdocmain{|\textit{main}|}| add the following code
to the top of the main file:
%
\begin{center}
\begin{tabular}{l}
|\||ifdefined\childdocname\endinput\||fi\newif\ifchilddoc|\\
|\edef\childdocname{\scantokens\expandafter{\jobname\noexpand}}|\\
|\def\childdocmain{|\textit{main}|}\||ifx\childdocmain\childdocname\||else|\\
|\childdoctrue\includeonly{\childdocname}\let\jobname\childdocmain\||fi|\\
\end{tabular}
\end{center}
%
Instead of |\childdocof{|\textit{main}|}| just include the main file
at the top of each child file:
%
\begin{center}
|\input{|\textit{main}|}|
\end{center}
%
A simple redirection |\childdocforward{|\textit{dest}|}| is achieved by:
%
\begin{center}
|\def\jobname{|\textit{dest}|}\input{\jobname}|
\end{center}
%
The redirection with prefix
|\childdocforwardprefix[|\textit{prefix}|]{|\textit{dest}|}|
is accomplished by:
%
\begin{center}
\begin{tabular}{l}
|{\edef\jobname{\scantokens\expandafter{\jobname\noexpand}}|\\
|\def\redirectjob |\textit{prefix}|#1~~~{\gdef\jobname{|\textit{dest}|#1}}|\\
|\expandafter\redirectjob\jobname~~~}\input{\jobname}|
\end{tabular}
\end{center}

In an alternative approach,
child documents can be compiled by a specific command line
without additional code or specific definitions:
%
\begin{center}
|... -jobname "|\textit{target}|" "|[\textit{flags}]%
|\includeonly{|\textit{dest}|}\input{|\textit{main}|}"|
\end{center}
%

%%%%%%%%%%%%%%%%%%%%%%%%%%%%%%%%%%%%%%%%%%%%%%%%%%%%%%%%%%%%%%%%%%%%%%%%%%%%%%%%
%%%%%%%%%%%%%%%%%%%%%%%%%%%%%%%%%%%%%%%%%%%%%%%%%%%%%%%%%%%%%%%%%%%%%%%%%%%%%%%%
\section{Information}

%%%%%%%%%%%%%%%%%%%%%%%%%%%%%%%%%%%%%%%%%%%%%%%%%%%%%%%%%%%%%%%%%%%%%%%%%%%%%%%%
\subsection{Copyright}

Copyright \copyright{} 2017--2018 Niklas Beisert

This work may be distributed and/or modified under the
conditions of the \LaTeX{} Project Public License, either version 1.3
of this license or (at your option) any later version.
The latest version of this license is in
  \url{http://www.latex-project.org/lppl.txt}
and version 1.3 or later is part of all distributions of \LaTeX{}
version 2005/12/01 or later.

This work has the LPPL maintenance status `maintained'.

The Current Maintainer of this work is Niklas Beisert.

This work consists of the files |README.txt|, |childdoc.ins| and |childdoc.dtx|
as well as the derived files |childdoc.def|, |cdocsamp.tex|
with |cdocsch1.tex|, |cdocsch2.tex|, |cdocspt3.tex|, |cdocspt4.tex|,
|cdocsdrf.tex|, |cdocsfn1.tex|, |cdocsfn2.tex|
as well as |childdoc.pdf|.

%%%%%%%%%%%%%%%%%%%%%%%%%%%%%%%%%%%%%%%%%%%%%%%%%%%%%%%%%%%%%%%%%%%%%%%%%%%%%%%%
\subsection{Files and Installation}

The package consists of the files:
%
\begin{center}
\begin{tabular}{ll}
    |README.txt|   & readme file \\
    |childdoc.ins| & installation file \\
    |childdoc.dtx| & source file \\
    |childdoc.def| & definition file \\
    |cdocsamp.tex| & sample main file \\
    |cdocsch1.tex| & sample include file \\
    |cdocsch2.tex| & sample include file \\
    |cdocspt3.tex| & sample part file \\
    |cdocspt4.tex| & sample part file \\
    |cdocsdrf.tex| & sample redirection file \\
    |cdocsfn1.tex| & sample redirection file \\
    |cdocsfn2.tex| & sample redirection file \\
    |childdoc.pdf| & manual
\end{tabular}
\end{center}
%
The distribution consists of the files
|README.txt|, |childdoc.ins| and |childdoc.dtx|.
%
\begin{itemize}
\item
Run (pdf)\LaTeX{} on |childdoc.dtx|
to compile the manual |childdoc.pdf| (this file).
\item
Run \LaTeX{} on |childdoc.ins| to create the definitions file |childdoc.def|
and the sample |cdocsamp.tex| with include files
|cdocsch1.tex|, |cdocsch2.tex|, |cdocspt3.tex|, |cdocspt4.tex|,
|cdocsdrf.tex|, |cdocsfn1.tex|, |cdocsfn2.tex|.
Then copy the file |childdoc.def| to an appropriate directory of your \LaTeX{}
distribution, e.g.\ \textit{texmf-root}|/tex/latex/childdoc|.
\end{itemize}

%%%%%%%%%%%%%%%%%%%%%%%%%%%%%%%%%%%%%%%%%%%%%%%%%%%%%%%%%%%%%%%%%%%%%%%%%%%%%%%%
\subsection{Related CTAN Packages}

There are several other packages which offer a similar functionality:
%
\begin{itemize}
\item
The packages
\href{http://ctan.org/pkg/docmute}{\textsf{docmute}},
\href{http://ctan.org/pkg/includex}{\textsf{includex}} and
\href{http://ctan.org/pkg/standalone}{\textsf{standalone}}
provide commands to include only the document body of
a child file thus allowing both files to be compiled individually.
\item
The packages \href{http://ctan.org/pkg/subdocs}{\textsf{subdocs}}
and \href{http://ctan.org/pkg/subfiles}{\textsf{subfiles}}
provide structures in which the main and child documents can be
encapsulated and allowing them to be compiled individually.
The inclusion mechanism is different from the conventional |\include|.
\item
The package \href{http://ctan.org/pkg/combine}{\textsf{combine}}
is an elaborate solution to combine several documents into one.
\end{itemize}
%
See also the CTAN topic \href{http://ctan.org/topic/subdocs}{\textsf{subdocs}}
for further related packages.
The present package differs from the above solutions in that
a document structure constructed with the conventional |\include| mechanism
just needs two extra commands at the top of every file
such that all constituent files can be compiled individually.

%%%%%%%%%%%%%%%%%%%%%%%%%%%%%%%%%%%%%%%%%%%%%%%%%%%%%%%%%%%%%%%%%%%%%%%%%%%%%%%%
%\subsection{Feature Suggestions}
%
%The following is a list of features which may be useful for future
%versions of this package:
%%
%\begin{itemize}
%\item
%\ldots
%\end{itemize}

%%%%%%%%%%%%%%%%%%%%%%%%%%%%%%%%%%%%%%%%%%%%%%%%%%%%%%%%%%%%%%%%%%%%%%%%%%%%%%%%
\subsection{Revision History}

%%%%%%%%%%%%%%%%%%%%%%%%%%%%%%%%%%%%%%%%
\paragraph{v2.0:} 2018/12/30

\begin{itemize}
\item
immediate forward processing
\item
added |\childdocby| mechanism
\item
manual restructured
\end{itemize}

%%%%%%%%%%%%%%%%%%%%%%%%%%%%%%%%%%%%%%%%
\paragraph{v1.6:} 2018/01/17

\begin{itemize}
\item
application for development of include files
\item
corrections to manual
\end{itemize}

%%%%%%%%%%%%%%%%%%%%%%%%%%%%%%%%%%%%%%%%
\paragraph{v1.5:} 2017/05/21

\begin{itemize}
\item
more complete structuring introduced
\item
|\childdocof| introduced
\item
|\childdoc| renamed to |\childdocmain|
\item
|\childredirect| renamed to |\childdocforward| and |\childdocforwardprefix|
and functionality expanded
\end{itemize}

%%%%%%%%%%%%%%%%%%%%%%%%%%%%%%%%%%%%%%%%
\paragraph{v1.0:} 2017/04/27

\begin{itemize}
\item
manual and install package
\item
first version published on CTAN
\end{itemize}

%%%%%%%%%%%%%%%%%%%%%%%%%%%%%%%%%%%%%%%%
\paragraph{v0.6:} 2017/04/26

\begin{itemize}
\item
redirection mechanism added
\end{itemize}

%%%%%%%%%%%%%%%%%%%%%%%%%%%%%%%%%%%%%%%%
\paragraph{v0.5:} 2017/04/26

\begin{itemize}
\item
functionality in definition file
\end{itemize}


%%%%%%%%%%%%%%%%%%%%%%%%%%%%%%%%%%%%%%%%%%%%%%%%%%%%%%%%%%%%%%%%%%%%%%%%%%%%%%%%
%%%%%%%%%%%%%%%%%%%%%%%%%%%%%%%%%%%%%%%%%%%%%%%%%%%%%%%%%%%%%%%%%%%%%%%%%%%%%%%%
%%%%%%%%%%%%%%%%%%%%%%%%%%%%%%%%%%%%%%%%%%%%%%%%%%%%%%%%%%%%%%%%%%%%%%%%%%%%%%%%
\appendix

\settowidth\MacroIndent{\rmfamily\scriptsize 000\ }

 \DocInput{childdoc.dtx}

\end{document}
%</driver>
% \fi
%
% %%%%%%%%%%%%%%%%%%%%%%%%%%%%%%%%%%%%%%%%%%%%%%%%%%%%%%%%%%%%%%%%%%%%%%%%%%%%%%
% %%%%%%%%%%%%%%%%%%%%%%%%%%%%%%%%%%%%%%%%%%%%%%%%%%%%%%%%%%%%%%%%%%%%%%%%%%%%%%
% \section{Sample}
%\iffalse
%<*samplemain>
%\fi
%
% The following presents a sample document
% with two chapters, two parts, a title page,
% a compile flag as well as three forwarding files to set the flag.
% It consists of eight |.tex| files:
% \begin{center}
% \begin{tabular}{ll}
% |cdocsamp.tex|&main file\\
% |cdocsch1.tex|&include file for chapter 1\\
% |cdocsch2.tex|&include file for chapter 2\\
% |cdocspt3.tex|&include file for part 3\\
% |cdocspt4.tex|&include file for part 4\\
% |cdocsdrf.tex|&forwarding file for main file in draft mode\\
% |cdocsfi1.tex|&forwarding file for final version of chapter 1\\
% |cdocsfi2.tex|&forwarding file for final version of chapter 2\\
% \end{tabular}
% \end{center}
% Each of the eight files can be compiled directly by the \LaTeX{} compiler.
%
% %%%%%%%%%%%%%%%%%%%%%%%%%%%%%%%%%%%%%%
% \paragraph{Main File.}
%
% The main file is called |cdocsamp.tex|.
%
% Load the \textsf{childdoc} definitions and
% declare the filename for the main document:
%    \begin{macrocode}
\input{childdoc.def}
\childdocmain{}
%    \end{macrocode}

% Optional override for |\version| flag:
%    \begin{macrocode}
%%\ifchilddoc\else\providecommand{\version}{draft}\fi
%    \end{macrocode}

% Define the default values for the |\version| flag
% (|final| for the main file and |draft| for childs):
%    \begin{macrocode}
\ifchilddoc
\providecommand{\version}{draft}
\else
\providecommand{\version}{final}
\fi
%    \end{macrocode}

% Load the standard document class:
%    \begin{macrocode}
\documentclass[12pt]{article}
%    \end{macrocode}

% Start the document body:
%    \begin{macrocode}
\begin{document}
%    \end{macrocode}

% Declare a title page.
% Print title, part of document being processed and version flag:
%    \begin{macrocode}
\addtocounter{page}{-1}
\begin{center}
{\LARGE\bfseries{}childdoc example\par}
\vspace{1cm}
\ifchilddoc
\ifchilddocmanual part\else chapter\fi:
`\childdocname' of `\childdocjob'\par
\else
main document: `\childdocjob'\par
\fi
version: \version\par
\end{center}
\newpage
%    \end{macrocode}

% Manually include selected file,
% otherwise process as usual:
%    \begin{macrocode}
\ifchilddocmanual
\section*{part `\childdocname'}
\input{\childdocname}
\else
%    \end{macrocode}

% Include the two chapters:
%    \begin{macrocode}
\include{cdocsch1}
\include{cdocsch2}
%    \end{macrocode}

% Include the two parts unless only chapters should be displayed:
%    \begin{macrocode}
\ifchilddoc\else
\section{part three}
\input{cdocspt3}
\section{part four}
\input{cdocspt4}
\fi
%    \end{macrocode}

% Process as usual until here:
%    \begin{macrocode}
\fi
%    \end{macrocode}

% End of document body:
%    \begin{macrocode}
\end{document}
%    \end{macrocode}
%\iffalse
%</samplemain>
%\fi
%
% %%%%%%%%%%%%%%%%%%%%%%%%%%%%%%%%%%%%%%
% \paragraph{Chapter Include Files.}
%
% The include files are called |cdocsch1.tex| and |cdocsch2.tex|.
%
%\iffalse
%<*samplechap1|samplechap2>
%\fi

% Optional override for |\version| flag:
%    \begin{macrocode}
%%\providecommand{\version}{final}
%    \end{macrocode}

% Include the main document:
%    \begin{macrocode}
\input{childdoc.def}
\childdocof{cdocsamp}
%    \end{macrocode}

%\iffalse
%</samplechap1|samplechap2>
%\fi
%
%\iffalse
%<*samplechap1>
%\fi
% Some text for chapter 1:
%    \begin{macrocode}
\section{one}
some text in chapter one
%    \end{macrocode}

%\iffalse
%</samplechap1>
%\fi
% Some text for chapter 2:
%\iffalse
%<*samplechap2>
%\fi
%    \begin{macrocode}
\section{two}
more text in chapter two
%    \end{macrocode}

%\iffalse
%</samplechap2>
%\fi
%
% %%%%%%%%%%%%%%%%%%%%%%%%%%%%%%%%%%%%%%
% \paragraph{Part Include Files.}
%
% The include files are called |cdocspt3.tex| and |cdocspt4.tex|.
%
%\iffalse
%<*samplepart3|samplepart4>
%\fi

% Optional override for |\version| flag:
%    \begin{macrocode}
%%\providecommand{\version}{final}
%    \end{macrocode}

% Include the main document:
%    \begin{macrocode}
\input{childdoc.def}
\childdocby{cdocsamp}
%    \end{macrocode}

%\iffalse
%</samplepart3|samplepart4>
%\fi
%
%\iffalse
%<*samplepart3>
%\fi
% Some text for part 3:
%    \begin{macrocode}
some text in part three
%    \end{macrocode}

%\iffalse
%</samplepart3>
%\fi
% Some text for part 4:
%\iffalse
%<*samplepart4>
%\fi
%    \begin{macrocode}
more text in part four
%    \end{macrocode}

%\iffalse
%</samplepart4>
%\fi
%
% %%%%%%%%%%%%%%%%%%%%%%%%%%%%%%%%%%%%%%
% \paragraph{Forwarding for a Complete Draft.}
%
% The following forwarding file |cdocsdrf.tex|
% compiles the main document in draft mode:
%\iffalse
%<*sampledraft>
%\fi
%    \begin{macrocode}
\def\version{draft}
\input{childdoc.def}
\childdocforward{cdocsamp}
%    \end{macrocode}

%\iffalse
%</sampledraft>
%\fi
%
% %%%%%%%%%%%%%%%%%%%%%%%%%%%%%%%%%%%%%%
% \paragraph{Forwarding for Final Version of the Chapters.}
%
% The following forwarding files |cdocsfn1.tex| and |cdocsfn2.tex|
% (with identical content)
% compile the final versions of the child documents
% |cdocsch1.tex| and |cdocsch2.tex|, respectively:
%\iffalse
%<*samplefinal>
%\fi
%    \begin{macrocode}
\def\version{final}
\input{childdoc.def}
\childdocforwardprefix[cdocsamp]{cdocsfn}{cdocsch}
%    \end{macrocode}

%\iffalse
%</samplefinal>
%\fi
%
% %%%%%%%%%%%%%%%%%%%%%%%%%%%%%%%%%%%%%%
% \paragraph{Command Line Processing.}
%
% The following three command lines generate the output files
% |cdocscld|, |cdocscl1| and |cdocscl2|
% which should be identical to
% |cdocsdrf|, |cdocsch1| and |cdocsfn2|, respectively:
% \begin{center}
% \begin{tabular}{l}
% |latex -jobname cdocscld \|\\
% |  "\def\version{draft}\input{childdoc.def}\childdocforward{cdocsamp}"|\\
% |latex -jobname cdocscl1 \|\\
% |  "\input{childdoc.def}\childdocforward[cdocsamp]{cdocsch1}"|\\
% |latex -jobname cdocscl2 \|\\
% |  "\def\version{final}\input{childdoc.def}\childdocforward{cdocsch2}"|
% \end{tabular}
% \end{center}
% Note that the trailing backslash on each first line
% merely continues the input to the second line
% (for convenient cut ant paste).
% Furthermore, the command |latex| can be replaced by any
% of its alternative versions such as |pdflatex|.
%
% %%%%%%%%%%%%%%%%%%%%%%%%%%%%%%%%%%%%%%%%%%%%%%%%%%%%%%%%%%%%%%%%%%%%%%%%%%%%%%
% %%%%%%%%%%%%%%%%%%%%%%%%%%%%%%%%%%%%%%%%%%%%%%%%%%%%%%%%%%%%%%%%%%%%%%%%%%%%%%
% \section{Implementation}
%\iffalse
%<*package>
%\fi
%
% This section describes the definitions file |childdoc.def|.

% The definitions cannot be loaded using |\usepackage| or |\RequirePackage|
% which has a mechanism to prevent loading a style file more than once.
% When loading the definitions by means of |\input|
% multiple instances have to be prevented manually:
%\iffalse
%This code needs to be before the `\ProvidesFile' directive
%which is defined at the beginning of this file.
%Therefore it is also placed there and commented out here.
%</package>
%<*discard>
%\fi
%    \begin{macrocode}
\ifdefined\childdocmain\endinput\fi
%    \end{macrocode}
%\iffalse
%</discard>
%<*package>
%\fi
%
% \macro{\ifchilddoc}
% \macro{\ifchilddocmanual}
% The conditional |\ifchilddoc| tells whether a
% child (true) or main (false) document is being compiled.
% The conditional |\ifchilddocmanual| tells whether
% the |\includeonly| mechanism is used (false) or
% the selection of child files must be performed manually (true).
% The definitions initialise to false:
%    \begin{macrocode}
\newif\ifchilddoc
\newif\ifchilddocmanual
%    \end{macrocode}

% \macro{\childdocname}
% \macro{\childdocjob}
% The macro |\childdocname| stores the name of the main document
% to be compiled. The macro |\childdocjob| stores the name of
% the document on which the \LaTeX{} compiler was originally invoked.
% The content of |\jobname| cannot be compared
% to filenames specified in the source due to different catcodes.
% The following code rescans |\jobname|, stores the result
% in |\childdocname| and saves a copy in |\childdocjob|:
%    \begin{macrocode}
\edef\childdocname{\scantokens\expandafter{\jobname\noexpand}}
\let\childdocjob\childdocname
%    \end{macrocode}

% \macro{\childdocdisable}
% The macro |\childdocdisable| prevents the main file
% from being processed more than once.
% At this stage, the main document command |\childdocmain|
% is assumed to be called once again where it should do nothing.
% Any subsequent call to it should prevent
% a secondary processing of the main document
% It overwrites the forwarding commands
% |\childdocof| and |\childdocforward|
% with empty macros to prevent further inclusions of the main document:
%    \begin{macrocode}
\newcommand{\childdocdisable}
{
  \renewcommand{\childdocmain}[1]{\renewcommand{\childdocmain}[1]{\endinput}}
  \renewcommand{\childdocof}[1]{}
  \renewcommand{\childdocby}[2][]{}
  \renewcommand{\childdocforward}[2][]{}
  \renewcommand{\childdocdisable}{}
}
%    \end{macrocode}

% \macro{\childdocmain}
% The macro |\childdocmain| is to be called at the top of the main file
% with nothing or the main filename (without extension) as argument.
% First, it breaks loops.
% If the argument is not empty and does not match |\childdocname|
% (which is set by the first inclusion of |childdoc.def|),
% |\ifchilddoc| is set to true, |\includeonly| is applied to the child file
% and |\jobname| is set to the main file
% (for proper handling of |.aux| files):
%    \begin{macrocode}
\newcommand{\childdocmain}[1]
{
  \childdocdisable\childdocmain{}
  \if?#1?\else
    \begingroup
      \def\childdoctmp{#1}
      \ifx\childdoctmp\childdocname
        \def\childdoctmp{}
      \else
        \def\childdoctmp
        {
          \childdoctrue
          \includeonly{\childdocname}
          \def\childdocjob{#1}
          \def\jobname{#1}
        }
      \fi
      \expandafter
    \endgroup
    \childdoctmp
  \fi
}
%    \end{macrocode}

% \macro{\childdocof}
% The command |\childdocof| redirects
% compilation to the main file |#1|.
%    \begin{macrocode}
\newcommand{\childdocof}[1]
{
  \childdocdisable
  \childdoctrue
  \includeonly{\childdocname}
  \def\jobname{#1}
  \def\childdocjob{#1}
  \input{#1}
}
%    \end{macrocode}

% \macro{\childdocby}
% The command |\childdocby| ....
%    \begin{macrocode}
\newcommand{\childdocby}[2][]
{
  \childdocdisable
  \childdoctrue
  \childdocmanualtrue
  \if?#1?\else
    \def\jobname{#2}
  \fi
  \def\childdocjob{#2}
  \input{#2}
  \endinput
}
%    \end{macrocode}

% \macro{\childdocforward}
% The command |\childdocforward| redirects
% compilation to the main file or
% (if the optional argument is given) a child file.
% Parameters are set as if the main file
% or a child file starting with |\childdocof| was compiled.
% Then compilation is handed over to the main file:
%    \begin{macrocode}
\newcommand{\childdocforward}[2][]
{
  \begingroup
    \if?#1?
      \def\childdoctmp
      {
        \def\childdocname{#2}
        \def\childdocjob{#2}
        \def\jobname{#2}
        \input{#2}
        \endinput
      }
    \else
      \def\childdoctmp
      {
        \childdocdisable
        \def\childdocname{#2}
        \childdoctrue
        \includeonly{#2}
        \def\childdocjob{#1}
        \def\jobname{#1}
        \input{#1}
        \endinput
      }
    \fi
    \expandafter
  \endgroup
  \childdoctmp
}
%    \end{macrocode}

% \macro{\childdocforwardprefix}
% The command |\childdocforwardprefix| redirects
% compilation to the main or a child file by means of a pattern.
% The prefix |#1| in the current filename is replaced by |#2|
% and the suffix of the current filename is kept
% (it is assumed that the filename does not contain the substring `|~~~|'
% which is used as a delimiter).
% Compilation is handed over to the new file by |\childdocforward|:
%    \begin{macrocode}
\newcommand{\childdocforwardprefix}[3][]
{
  \begingroup
    \def\childdocextract #2##1~~~{\def\childdoctmp{\childdocforward[#1]{#3##1}}}
    \expandafter\childdocextract\childdocname~~~
    \expandafter
  \endgroup
  \childdoctmp
}
%    \end{macrocode}

% \macro{\childdoc}
% The deprecated macro |\childdoc| is a legacy version of |\childdocmain|:
%    \begin{macrocode}
\newcommand{\childdoc}{\childdocmain}
%    \end{macrocode}

% \macro{\childdocredirect}
% The deprecated macro |\childdocredirect| is a legacy version
% of |\childdocforward| and |\childdocforwardprefix|:
%    \begin{macrocode}
\newcommand{\childdocredirect}[2][]
{
  \begingroup
    \if?#1?
      \def\childdoctmp{\childdocforward{#2}}
    \else
      \def\childdoctmp{\childdocforwardprefix{#1}{#2}}
    \fi
    \expandafter
  \endgroup
  \childdoctmp
}
%    \end{macrocode}

%\iffalse
%</package>
%\fi
%
\endinput
|\\
|\childdocforwardprefix[|\textit{main}|]{|\textit{prefix}|}{|\textit{dest}|}|
\end{tabular}
\end{center}
%
the destination file is determined by a pattern
depending on the current file:
To make this work, the current file must be called
`{\textit{prefix}\hspace{0.2em}\textit{suffix}}'
with \textit{prefix} matching precisely the argument.
Processing is then passed on to the file
`{\textit{dest}\hspace{0.2em}\textit{suffix}}'.
Surely, the same effect is achieved by
directly specifying the
argument `{\textit{dest}\hspace{0.2em}\textit{suffix}}'
in the first form.
However, that requires to set up a different file
for each child. With the alternative form of the command
all these files can have exactly the same content
which simplifies setting them up and maintaining them.

For example, the following file |draft.tex|
with a compilation flag |\version| as described in \secref{sec:flags}
compiles the main document as a draft:
%
\begin{center}
\begin{tabular}{l}
|\def\version{draft}|\\
|% \iffalse
%
% childdoc.dtx Copyright (C) 2017-2018 Niklas Beisert
%
% This work may be distributed and/or modified under the
% conditions of the LaTeX Project Public License, either version 1.3
% of this license or (at your option) any later version.
% The latest version of this license is in
%   http://www.latex-project.org/lppl.txt
% and version 1.3 or later is part of all distributions of LaTeX
% version 2005/12/01 or later.
%
% This work has the LPPL maintenance status `maintained'.
%
% The Current Maintainer of this work is Niklas Beisert.
%
% This work consists of the files childdoc.dtx and childdoc.ins
% and the derived files childdoc.def and cdocsamp.tex with
% cdocsch1.tex, cdocsch2.tex, cdocsdrf.tex, cdocsfn1.tex, cdocsfn2.tex.
%
%<package>\ifdefined\childdocmain\endinput\fi
%<package>\ProvidesFile{childdoc.def}[2018/12/30 v2.0 child document driver]
%<samplemain>\ProvidesFile{cdocsamp.tex}[2018/12/30 v2.0 sample for childdoc]
%<*driver>
%\ProvidesFile{childdoc.drv}[2018/12/30 v2.0 childdoc reference manual file]
\PassOptionsToClass{10pt,a4paper}{article}
\documentclass{ltxdoc}

\usepackage[margin=35mm]{geometry}
\usepackage{hyperref}
\usepackage{hyperxmp}
\usepackage[usenames]{color}

\hypersetup{colorlinks=true}
\hypersetup{pdfstartview=FitH}
\hypersetup{pdfpagemode=UseNone}
\hypersetup{pdfsource={}}
\hypersetup{pdflang={en-UK}}
\hypersetup{pdfcopyright={Copyright 2017-2018 Niklas Beisert.
  This work may be distributed and/or modified under the
  conditions of the LaTeX Project Public License, either version 1.3
  of this license or (at your option) any later version.}}
\hypersetup{pdflicenseurl={http://www.latex-project.org/lppl.txt}}
\hypersetup{pdfcontactaddress={ETH Zurich, ITP, HIT K,
  Wolfgang-Pauli-Strasse 27}}
\hypersetup{pdfcontactpostcode={8093}}
\hypersetup{pdfcontactcity={Zurich}}
\hypersetup{pdfcontactcountry={Switzerland}}
\hypersetup{pdfcontactemail={nbeisert@itp.phys.ethz.ch}}
\hypersetup{pdfcontacturl={http://people.phys.ethz.ch/\xmptilde nbeisert/}}

\newcommand{\secref}[1]{\hyperref[#1]{section \ref*{#1}}}

\parskip1ex
\parindent0pt
\let\olditemize\itemize
\def\itemize{\olditemize\parskip0pt}

\begin{document}

\title{The \textsf{childdoc} Package}
\hypersetup{pdftitle={The childdoc Package}}
\author{Niklas Beisert\\[2ex]
  Institut f\"ur Theoretische Physik\\
  Eidgen\"ossische Technische Hochschule Z\"urich\\
  Wolfgang-Pauli-Strasse 27, 8093 Z\"urich, Switzerland\\[1ex]
  \href{mailto:nbeisert@itp.phys.ethz.ch}
  {\texttt{nbeisert@itp.phys.ethz.ch}}}
\hypersetup{pdfauthor={Niklas Beisert}}
\hypersetup{pdfsubject={Manual for the LaTeX2e Package childdoc}}
\date{30 December 2018, \textsf{v2.0}}
\maketitle

\begin{abstract}\noindent
\textsf{childdoc} is a \LaTeXe{} package
that enables the direct compilation
of document sections included by |\include|
to individual files.
\end{abstract}

\begingroup
\parskip0ex
\tableofcontents
\endgroup

%%%%%%%%%%%%%%%%%%%%%%%%%%%%%%%%%%%%%%%%%%%%%%%%%%%%%%%%%%%%%%%%%%%%%%%%%%%%%%%%
%%%%%%%%%%%%%%%%%%%%%%%%%%%%%%%%%%%%%%%%%%%%%%%%%%%%%%%%%%%%%%%%%%%%%%%%%%%%%%%%
\section{Introduction}

\LaTeX{} provides a mechanism to structure a large document (such as a book)
into a main file and several child files (containing the chapters)
using the |\include| command.
This mechanism is beneficial for documents
which span hundreds of pages in order to
make the source file(s) more manageable.
Moreover, compilation can be restricted to
selected child files by means of the |\includeonly| command.
The latter feature can be used to reduce the compilation time while editing
(this was significantly more useful in the earlier days of \LaTeX{})
or to generate a smaller document which is easier to navigate.
Another application of |\includeonly| is to generate
documents consisting of selected parts of the complete document.

However, there are a few drawbacks of the plain |\include| mechanism:
\begin{itemize}
\item
The child files cannot be compiled on their own,
they can only be compiled via the main file.
A naive editing environment
(such as a text editor with an option
to have the current file processed by \LaTeX)
may require one to switch to the main file before compiling;
attempting to compile the child file produces errors.
\item
The main file must be modified (each time)
to adjust the |\includeonly| command
to the present needs. This easily leaves the main file in a messy state.
\item
The generated document will always carry the filename
of the main document. This is inconvenient if
several child files are to be compiled and
to be kept for distribution.
\end{itemize}

The present package provides a simple interface
to make child files individually compilable by \LaTeX{}.
Compiling a child file then has the same effect as compiling
the main file with an |\includeonly| command
to select the appropriate child.
Moreover the generated document will carry the name of the child
rather than the main file.
This resolves all three above issues.

This feature is meant to make the editing of books,
thesis documents and lecture notes somewhat more convenient.
However, the package can also be used efficiently for
composing a series of documents (such as exercise sheets)
which are typically distributed individually.
It then assists the author in generating the individual documents
(potentially in different versions)
as well as a document containing the collected series.
Another application is in developing style files
or other kinds of included material
where compilation of the style file could redirect
to a sample or test file.

%%%%%%%%%%%%%%%%%%%%%%%%%%%%%%%%%%%%%%%%%%%%%%%%%%%%%%%%%%%%%%%%%%%%%%%%%%%%%%%%
%%%%%%%%%%%%%%%%%%%%%%%%%%%%%%%%%%%%%%%%%%%%%%%%%%%%%%%%%%%%%%%%%%%%%%%%%%%%%%%%
\section{Usage}

First of all, the package \textsf{childdoc} is \emph{not} a standard
\LaTeXe{} |.sty| style file! Therefore it needs to be invoked in
a non-standard way.

%%%%%%%%%%%%%%%%%%%%%%%%%%%%%%%%%%%%%%%%%%%%%%%%%%%%%%%%%%%%%%%%%%%%%%%%%%%%%%%%
\subsection{Included Files}
\label{sec:include}

%%%%%%%%%%%%%%%%%%%%%%%%%%%%%%%%%%%%%%%%
\DescribeMacro{\childdocmain}
To use the package, add the commands
\begin{center}
\begin{tabular}{l}
|\input{childdoc.def}|\\
|\childdocmain{}|\\
\end{tabular}
\end{center}
at the very top of the main \LaTeX{} file,
in particular \emph{before} the |\documentclass| statement!
The argument of |\childdocmain| should be left empty
(but it must be present).

%%%%%%%%%%%%%%%%%%%%%%%%%%%%%%%%%%%%%%%%
\DescribeMacro{\childdocof}
Furthermore, add the commands
\begin{center}
\begin{tabular}{l}
|\input{childdoc.def}|\\
|\childdocof{|\textit{main}|}|\\
\end{tabular}
\end{center}
at the top of every child file \textit{child}
which is included by |\include{|\textit{child}|}|
from within the main file
(or at least for those files to be compiled individually).
The argument \textit{main} must be the filename of the main file.

There are a couple of
considerations in setting up the main and child documents:

%%%%%%%%%%%%%%%%%%%%%%%%%%%%%%%%%%%%%%%%
\paragraph{Restrictions.}

Please note the following restrictions:
\begin{itemize}
\item
|\childdocmain| must be called with one argument \textit{main}
to ensure compatibility with earlier version of the package.
It must either be empty (|\childdocmain{}|)
or precisely match the filename of the main file in which it is specified.
See \secref{sec:detection} for further information.
\item
The filename \textit{main} must be specified without the |.tex| extension.
\item
The filename \textit{main} is case sensitive
(even in case-insensitive file systems)
due to internal string comparison.
\item
The argument \textit{main} should be fully expanded, it cannot be a macro.
\item
Subdirectories and special characters should be avoided in filenames.
\item
The command |\childdocmain{|\textit{main}|}| must be followed by a whitespace.
It should not be followed immediately by another command
or by a comment mark `|%|'.
This is because the \TeX{} parser reads the token immediately following
the argument of |\childdocmain| and puts it
at the beginning of every child section;
however, a white\-space is ignored.
\end{itemize}

%%%%%%%%%%%%%%%%%%%%%%%%%%%%%%%%%%%%%%%%
\paragraph{Content of Main File.}

It is advisable to place all content in the child files included by |\include|.
Any output contained in the main file will appear in all child documents
unless suppressed manually;
it cannot be suppressed automatically by the |\includeonly| directive
and thus should normally be avoided.
A method to include some content in the main file
by means of conditional processing is described in \secref{sec:conditional}.

%%%%%%%%%%%%%%%%%%%%%%%%%%%%%%%%%%%%%%%%
\paragraph{Page Numbering.}

When only a part of the document is compiled,
the appropriate numbering of pages
(as well as other status parameters)
is determined from the |.aux| files.
The latter contain information from previous passes.
However this information needs to propagate through
all intermediate child documents.
Therefore the page numbering in child documents may well
be inconsistent until the complete document is compiled at least once.

A useful (if unconventional) way to always ensure a consistent
page numbering is to restart the numbering in each child document
and denote the pages by `\textit{child}|.|\textit{page}'
where \textit{child} represents the chapter/section number of the child file.
This can be achieved by the command
|\numberwithin{page}{|\textit{child}|}|
of the \textsf{amsmath} package
where \textit{child} can be |chapter| or |section|
depending on the chosen structuring.
Alternatively, one can modify the macro |\thepage| appropriately
and reset the counter |page| at the start of each child file.

%%%%%%%%%%%%%%%%%%%%%%%%%%%%%%%%%%%%%%%%%%%%%%%%%%%%%%%%%%%%%%%%%%%%%%%%%%%%%%%%
\subsection{Conditional Processing}
\label{sec:conditional}

The package provides a mechanism to compile different versions
of a document. To customise the versions further some conditional processing
can come in handy to distinguish which version is being compiled.
The package provides two macros to describe the compilation context:

%%%%%%%%%%%%%%%%%%%%%%%%%%%%%%%%%%%%%%%%
\DescribeMacro{\ifchilddoc}
The conditional |\ifchilddoc| distinguishes between the compilation of
child documents and the main document:
%
\begin{center}
|\ifchilddoc |\textit{child-code}| |[|\||else |\textit{main-code}]| \||fi|
\end{center}

%%%%%%%%%%%%%%%%%%%%%%%%%%%%%%%%%%%%%%%%
\DescribeMacro{\childdocname}
\DescribeMacro{\childdocjob}
The macro |\childdocname| contains the filename (without extension)
of the main or child file being processed.
Note that |\childdocjob| will always contain the name of the main file.

%%%%%%%%%%%%%%%%%%%%%%%%%%%%%%%%%%%%%%%%
\paragraph{Title Page.}

Conditional processing can be used to include a title or banner page
in the main document when proper precautions are taken.
Importantly, the code in the main file should ensure that the page counter
(as well as other status parameters which are stored in the |.aux| files)
takes the same value after the conditional processing.
Otherwise the page numbers may take divergent values
depending on which part is compiled.

For example, a title page could be declared by:
%
\begin{center}
\begin{tabular}{l}
|\ifchilddoc\||else|\\
|\addtocounter{page}{-1}|\\
\textit{code for title page}\\
|\newpage|\\
|\||fi|
\end{tabular}
\end{center}
%
A banner page for the child documents can be generated by:
%
\begin{center}
\begin{tabular}{l}
|\ifchilddoc|\\
|\addtocounter{page}{-1}|\\
\textit{code for banner page}\\
|\newpage|\\
|\||fi|
\end{tabular}
\end{center}
%
Here one could write a message such as:
\begin{center}
|This is the part \childdocname{} of \childdocjob{}.|
\end{center}

%%%%%%%%%%%%%%%%%%%%%%%%%%%%%%%%%%%%%%%%%%%%%%%%%%%%%%%%%%%%%%%%%%%%%%%%%%%%%%%%
\subsection{Flags}
\label{sec:flags}

The package makes it easy to generate different versions
of the main or child documents.
To this end compilation flags can be defined
and assigned different default values.
They will be particularly useful in conjunction
with the forwarding mechanism described in \secref{sec:forward}.

For example, it may be useful to have a flag |\version|
which can be set to |draft| or |final|.
The document source will contain some conditional code
depending on the value of |\version|.
Suppose further, the flag should default to |final| for the main file
and to |draft| for child files
which is a natural assignment for editing the document.
This is achieved by placing the following code
in the preamble of the main document
(below the |\childdocmain| directive):
%
\begin{center}
\begin{tabular}{l}
|\ifchilddoc|\\
|\providecommand{\version}{draft}|\\
|\||else|\\
|\providecommand{\version}{final}|\\
|\||fi|
\end{tabular}
\end{center}
%
The definition by |\providecommand| makes sure
that previous definitions are not overwritten.
Further statements |\providecommand{\version}{...}|
can thus be added before the above code to override it.

For the main file, one might add a line
(between |\childdocmain| and the above block)
%
\begin{center}
|%\ifchilddoc\||else\providecommand{\version}{draft}\||fi|
\end{center}
%
which can be uncommented to produce a draft version.
Likewise one can add a line to the very top of a child file
(above the |\childdocof{|\textit{main}|}| directive)
%
\begin{center}
|%\providecommand{\version}{final}|
\end{center}
%
which can be uncommented to produce the final version of this child document.

%%%%%%%%%%%%%%%%%%%%%%%%%%%%%%%%%%%%%%%%%%%%%%%%%%%%%%%%%%%%%%%%%%%%%%%%%%%%%%%%
\subsection{Forwarding}
\label{sec:forward}

Different versions of the main or child documents
using compilation flags as described in \secref{sec:flags}
can be (permanently) stored in different files
for convenient compilation, viewing and distribution.
To this end, the package defines a command
to pass on compilation to a different file:

%%%%%%%%%%%%%%%%%%%%%%%%%%%%%%%%%%%%%%%%
\DescribeMacro{\childdocforward}
The command |\childdocforward| redirects processing to
another source file:
%
\begin{center}
\begin{tabular}{l}
|\input{childdoc.def}|\\
|\childdocforward[|\textit{main}|]{|\textit{dest}|}|\\
\end{tabular}
\end{center}
%
The argument \textit{dest} is the destination file
(without extension).
It should be the main file or one of the child files.
Note that further \textsf{childdoc} directives
such as |\childdocof| and |\childdocforward|
in the indicated file will be processed in this form.
The optional argument \textit{main}
passes on directly to the main file \textit{main}
while pretending to compile the child \textit{dest}.
This form behaves as if \textit{dest}
issues |\childdocof{|\textit{main}|}| right away,
and no further \textsf{childdoc} directives will be processed.

%%%%%%%%%%%%%%%%%%%%%%%%%%%%%%%%%%%%%%%%
\DescribeMacro{\...prefix}
In the alternative form |\childdocforwardprefix|,
%
\begin{center}
\begin{tabular}{l}
|\input{childdoc.def}|\\
|\childdocforwardprefix[|\textit{main}|]{|\textit{prefix}|}{|\textit{dest}|}|
\end{tabular}
\end{center}
%
the destination file is determined by a pattern
depending on the current file:
To make this work, the current file must be called
`{\textit{prefix}\hspace{0.2em}\textit{suffix}}'
with \textit{prefix} matching precisely the argument.
Processing is then passed on to the file
`{\textit{dest}\hspace{0.2em}\textit{suffix}}'.
Surely, the same effect is achieved by
directly specifying the
argument `{\textit{dest}\hspace{0.2em}\textit{suffix}}'
in the first form.
However, that requires to set up a different file
for each child. With the alternative form of the command
all these files can have exactly the same content
which simplifies setting them up and maintaining them.

For example, the following file |draft.tex|
with a compilation flag |\version| as described in \secref{sec:flags}
compiles the main document as a draft:
%
\begin{center}
\begin{tabular}{l}
|\def\version{draft}|\\
|\input{childdoc.def}|\\
|\childdocforward{|\textit{main}|}|
\end{tabular}
\end{center}
%
Likewise, the following files |final|\textit{nn}|.tex|
compile the final version of the child document
|child|\textit{nn}|.tex|:
%
\begin{center}
\begin{tabular}{l}
|\def\version{final}|\\
|\input{childdoc.def}|\\
|\childdocforwardprefix{final}{child}|
\end{tabular}
\end{center}
%

Note that when several versions of a main file and/or of each child file
are to be generated, it may be convenient to set up a |Makefile| or
shell script to automatise the process.

%%%%%%%%%%%%%%%%%%%%%%%%%%%%%%%%%%%%%%%%%%%%%%%%%%%%%%%%%%%%%%%%%%%%%%%%%%%%%%%%
\subsection{Command Line Processing}
\label{sec:commandline}

The effect of redirection files can also be achieved by invoking
the \LaTeX{} compiler with a more elaborate command line.
Most conveniently this should be done as part
of a shell script or a |Makefile|.

When using \textsf{childdoc} in the main file, the following
command lines effectively perform a redirection
(note that depending on the shell being used,
backslashes may have to be doubled: `|\|' $\to$ `|\\|'):
%
\begin{center}
|... -jobname "|\textit{target}|" |\\|"|[\textit{flags}]%
|\input{childdoc.def}\childdocforward[|\textit{main}|]{|\textit{dest}|}"|
\end{center}
%
Here \textit{target} is the name of the output file,
\textit{main} is the name of the main file
and \textit{dest} is the name of the main or child file to be processed
(all filenames without extensions).
The optional argument \textit{main} can be omitted
if \textit{main} matches \textit{dest}.
Optionally, compilation \textit{flags} can be defined via |\def| commands.
This command line makes the \TeX{} engine believe
it is compiling the file \textit{target}
whose content is specified as the latter parameter.
The provided code then forwards the processing to
\textit{main} or \textit{dest} as described in \secref{sec:forward}.

%%%%%%%%%%%%%%%%%%%%%%%%%%%%%%%%%%%%%%%%%%%%%%%%%%%%%%%%%%%%%%%%%%%%%%%%%%%%%%%%
\subsection{Include by Input}
\label{sec:input}

Including child documents by |\include| has some restrictions by design.
Most notably, the content of a child document always occupies
its own set of pages; pages cannot be shared between child documents.
Usually, this behaviour makes perfect sense
because each child document contain an essential part of the document.
However, in some situations it may be desirable to compose
a document from a collection of parts
without having mandatory page breaks between then.
For this case, the package
provides a mechanism to include parts
by |\input| which can also be processed individually.
However, by construction this mechanism
requires manual handling of the content to be output.

%%%%%%%%%%%%%%%%%%%%%%%%%%%%%%%%%%%%%%%%
\DescribeMacro{\ifchilddocmanual}
The main file should be prepared as usual, see \secref{sec:include}.
However, the document body must make a distinction
between processing of an individual part and of the main document, e.g.:
%
\begin{center}
\begin{tabular}{l}
|\ifchilddocmanual|\\
|\input{\childdocname}|\\
|\||else|\\
\textit{document body with }|\input{|\textit{part}|}|\\
|\||fi|
\end{tabular}
\end{center}
%
The conditional |\ifchilddocmanual| is true whenever
a part to be included by |\input| is being compiled,
and the name of the part is stored in |\childdocname|.

%%%%%%%%%%%%%%%%%%%%%%%%%%%%%%%%%%%%%%%%
\DescribeMacro{\childdocby}
Each part to be included by |\input| should start with:
%
\begin{center}
\begin{tabular}{l}
|\input{childdoc.def}|\\
|\childdocby{|\textit{main}|}|\\
\end{tabular}
\end{center}
%
The directive |\childdocby| is similar to |\childdocof|
described in \secref{sec:include},
but the subsequent selection of content must be done manually.
To that end, both |\ifchilddoc| and |\ifchilddocmanual|
will be true upon processing of a part,
and the name of the part is stored in |\childdocname|.
Note that |\jobname| will be set to the filename of the current part
so that each part receives an individual |.aux| file
that does not interfere with the |.aux| file(s) of the main document.
This behaviour can be altered by the alternative form
|\childdocby[*]{|\textit{main}|}| (with a non-empty optional argument)
which uses the |.aux| file of the main document
by setting |\jobname| to \textit{main}.

%%%%%%%%%%%%%%%%%%%%%%%%%%%%%%%%%%%%%%%%%%%%%%%%%%%%%%%%%%%%%%%%%%%%%%%%%%%%%%%%
\subsection{Driver Development}
\label{sec:driver}

The \textsf{childdoc} mechanism can also be use for the development
of definition files such as \LaTeX{} styles or classes.
This case differs from the above setup with multiple parts
included by |\include| in that no |\includeonly| should be invoked.
This can be achieved by starting the include file
(before |\ProvidesPackage|) with:
%
\begin{center}
\begin{tabular}{l}
|\input{childdoc.def}|\\
|\childdocforward{|\textit{main}|}|\\
\end{tabular}
\end{center}
%
or alternatively with:
%
\begin{center}
\begin{tabular}{l}
|\input{childdoc.def}|\\
|\childdocby{|\textit{main}|}|\\
\end{tabular}
\end{center}
%
Both forms have slightly different effects as described above.
The main file is prepared as usual, see \secref{sec:include}.

%%%%%%%%%%%%%%%%%%%%%%%%%%%%%%%%%%%%%%%%%%%%%%%%%%%%%%%%%%%%%%%%%%%%%%%%%%%%%%%%
\subsection{Legacy Detection}
\label{sec:detection}

The directive |\childdocmain| in the main file can detect
whether the complete document or merely a child is to be compiled
even without using the directive |\childdocof|.
This method is deprecated because it is less robust
and there is no compelling reason to use it;
it is merely provided for backward compatibility
and it may be removed in future versions.

If the detection mechanism is to be used,
it is mandatory to correctly specify
the filename of the main file as the argument of |\childdocmain|:
%
\begin{center}
\begin{tabular}{l}
|\input{childdoc.def}|\\
|\childdocmain{|\textit{main}|}|\\
\end{tabular}
\end{center}
%
If |\jobname| does not match the argument \textit{main} of |\childdocmain|,
it is assumed that |\jobname| points to the child file to be compiled.
When using |\childdocmain| with the main file specified as argument,
it suffices to start a child file
with just |\input{|\textit{main}|}|
without loading of the package and using |\childdocof|.
If instead all processing is done
with the appropriate \textsf{childdoc} directives,
the argument of \textit{main} of |\childdocmain| can be empty.

An alternative version of the command line processing described
in \secref{sec:commandline} using the detection mechanism reads:
%
\begin{center}
|... -jobname "|\textit{target}|" "|[\textit{flags}]%
[|\def\jobname{|\textit{dest}|}|]|\input{|\textit{main}|}"|
\end{center}

%%%%%%%%%%%%%%%%%%%%%%%%%%%%%%%%%%%%%%%%%%%%%%%%%%%%%%%%%%%%%%%%%%%%%%%%%%%%%%%%
\subsection{Manual Code}
\label{sec:manual}

In case one cannot be certain whether the definitions file |childdoc.def|
is installed on the target \TeX{} distribution
and one prefers not to ship it,
it is conceivable to paste a few relevant commands into the sources.

To that end, drop all statements |\input{childdoc.def}|
and perform the replacements as outlined below.
Instead of |\childdocmain{|\textit{main}|}| add the following code
to the top of the main file:
%
\begin{center}
\begin{tabular}{l}
|\||ifdefined\childdocname\endinput\||fi\newif\ifchilddoc|\\
|\edef\childdocname{\scantokens\expandafter{\jobname\noexpand}}|\\
|\def\childdocmain{|\textit{main}|}\||ifx\childdocmain\childdocname\||else|\\
|\childdoctrue\includeonly{\childdocname}\let\jobname\childdocmain\||fi|\\
\end{tabular}
\end{center}
%
Instead of |\childdocof{|\textit{main}|}| just include the main file
at the top of each child file:
%
\begin{center}
|\input{|\textit{main}|}|
\end{center}
%
A simple redirection |\childdocforward{|\textit{dest}|}| is achieved by:
%
\begin{center}
|\def\jobname{|\textit{dest}|}\input{\jobname}|
\end{center}
%
The redirection with prefix
|\childdocforwardprefix[|\textit{prefix}|]{|\textit{dest}|}|
is accomplished by:
%
\begin{center}
\begin{tabular}{l}
|{\edef\jobname{\scantokens\expandafter{\jobname\noexpand}}|\\
|\def\redirectjob |\textit{prefix}|#1~~~{\gdef\jobname{|\textit{dest}|#1}}|\\
|\expandafter\redirectjob\jobname~~~}\input{\jobname}|
\end{tabular}
\end{center}

In an alternative approach,
child documents can be compiled by a specific command line
without additional code or specific definitions:
%
\begin{center}
|... -jobname "|\textit{target}|" "|[\textit{flags}]%
|\includeonly{|\textit{dest}|}\input{|\textit{main}|}"|
\end{center}
%

%%%%%%%%%%%%%%%%%%%%%%%%%%%%%%%%%%%%%%%%%%%%%%%%%%%%%%%%%%%%%%%%%%%%%%%%%%%%%%%%
%%%%%%%%%%%%%%%%%%%%%%%%%%%%%%%%%%%%%%%%%%%%%%%%%%%%%%%%%%%%%%%%%%%%%%%%%%%%%%%%
\section{Information}

%%%%%%%%%%%%%%%%%%%%%%%%%%%%%%%%%%%%%%%%%%%%%%%%%%%%%%%%%%%%%%%%%%%%%%%%%%%%%%%%
\subsection{Copyright}

Copyright \copyright{} 2017--2018 Niklas Beisert

This work may be distributed and/or modified under the
conditions of the \LaTeX{} Project Public License, either version 1.3
of this license or (at your option) any later version.
The latest version of this license is in
  \url{http://www.latex-project.org/lppl.txt}
and version 1.3 or later is part of all distributions of \LaTeX{}
version 2005/12/01 or later.

This work has the LPPL maintenance status `maintained'.

The Current Maintainer of this work is Niklas Beisert.

This work consists of the files |README.txt|, |childdoc.ins| and |childdoc.dtx|
as well as the derived files |childdoc.def|, |cdocsamp.tex|
with |cdocsch1.tex|, |cdocsch2.tex|, |cdocspt3.tex|, |cdocspt4.tex|,
|cdocsdrf.tex|, |cdocsfn1.tex|, |cdocsfn2.tex|
as well as |childdoc.pdf|.

%%%%%%%%%%%%%%%%%%%%%%%%%%%%%%%%%%%%%%%%%%%%%%%%%%%%%%%%%%%%%%%%%%%%%%%%%%%%%%%%
\subsection{Files and Installation}

The package consists of the files:
%
\begin{center}
\begin{tabular}{ll}
    |README.txt|   & readme file \\
    |childdoc.ins| & installation file \\
    |childdoc.dtx| & source file \\
    |childdoc.def| & definition file \\
    |cdocsamp.tex| & sample main file \\
    |cdocsch1.tex| & sample include file \\
    |cdocsch2.tex| & sample include file \\
    |cdocspt3.tex| & sample part file \\
    |cdocspt4.tex| & sample part file \\
    |cdocsdrf.tex| & sample redirection file \\
    |cdocsfn1.tex| & sample redirection file \\
    |cdocsfn2.tex| & sample redirection file \\
    |childdoc.pdf| & manual
\end{tabular}
\end{center}
%
The distribution consists of the files
|README.txt|, |childdoc.ins| and |childdoc.dtx|.
%
\begin{itemize}
\item
Run (pdf)\LaTeX{} on |childdoc.dtx|
to compile the manual |childdoc.pdf| (this file).
\item
Run \LaTeX{} on |childdoc.ins| to create the definitions file |childdoc.def|
and the sample |cdocsamp.tex| with include files
|cdocsch1.tex|, |cdocsch2.tex|, |cdocspt3.tex|, |cdocspt4.tex|,
|cdocsdrf.tex|, |cdocsfn1.tex|, |cdocsfn2.tex|.
Then copy the file |childdoc.def| to an appropriate directory of your \LaTeX{}
distribution, e.g.\ \textit{texmf-root}|/tex/latex/childdoc|.
\end{itemize}

%%%%%%%%%%%%%%%%%%%%%%%%%%%%%%%%%%%%%%%%%%%%%%%%%%%%%%%%%%%%%%%%%%%%%%%%%%%%%%%%
\subsection{Related CTAN Packages}

There are several other packages which offer a similar functionality:
%
\begin{itemize}
\item
The packages
\href{http://ctan.org/pkg/docmute}{\textsf{docmute}},
\href{http://ctan.org/pkg/includex}{\textsf{includex}} and
\href{http://ctan.org/pkg/standalone}{\textsf{standalone}}
provide commands to include only the document body of
a child file thus allowing both files to be compiled individually.
\item
The packages \href{http://ctan.org/pkg/subdocs}{\textsf{subdocs}}
and \href{http://ctan.org/pkg/subfiles}{\textsf{subfiles}}
provide structures in which the main and child documents can be
encapsulated and allowing them to be compiled individually.
The inclusion mechanism is different from the conventional |\include|.
\item
The package \href{http://ctan.org/pkg/combine}{\textsf{combine}}
is an elaborate solution to combine several documents into one.
\end{itemize}
%
See also the CTAN topic \href{http://ctan.org/topic/subdocs}{\textsf{subdocs}}
for further related packages.
The present package differs from the above solutions in that
a document structure constructed with the conventional |\include| mechanism
just needs two extra commands at the top of every file
such that all constituent files can be compiled individually.

%%%%%%%%%%%%%%%%%%%%%%%%%%%%%%%%%%%%%%%%%%%%%%%%%%%%%%%%%%%%%%%%%%%%%%%%%%%%%%%%
%\subsection{Feature Suggestions}
%
%The following is a list of features which may be useful for future
%versions of this package:
%%
%\begin{itemize}
%\item
%\ldots
%\end{itemize}

%%%%%%%%%%%%%%%%%%%%%%%%%%%%%%%%%%%%%%%%%%%%%%%%%%%%%%%%%%%%%%%%%%%%%%%%%%%%%%%%
\subsection{Revision History}

%%%%%%%%%%%%%%%%%%%%%%%%%%%%%%%%%%%%%%%%
\paragraph{v2.0:} 2018/12/30

\begin{itemize}
\item
immediate forward processing
\item
added |\childdocby| mechanism
\item
manual restructured
\end{itemize}

%%%%%%%%%%%%%%%%%%%%%%%%%%%%%%%%%%%%%%%%
\paragraph{v1.6:} 2018/01/17

\begin{itemize}
\item
application for development of include files
\item
corrections to manual
\end{itemize}

%%%%%%%%%%%%%%%%%%%%%%%%%%%%%%%%%%%%%%%%
\paragraph{v1.5:} 2017/05/21

\begin{itemize}
\item
more complete structuring introduced
\item
|\childdocof| introduced
\item
|\childdoc| renamed to |\childdocmain|
\item
|\childredirect| renamed to |\childdocforward| and |\childdocforwardprefix|
and functionality expanded
\end{itemize}

%%%%%%%%%%%%%%%%%%%%%%%%%%%%%%%%%%%%%%%%
\paragraph{v1.0:} 2017/04/27

\begin{itemize}
\item
manual and install package
\item
first version published on CTAN
\end{itemize}

%%%%%%%%%%%%%%%%%%%%%%%%%%%%%%%%%%%%%%%%
\paragraph{v0.6:} 2017/04/26

\begin{itemize}
\item
redirection mechanism added
\end{itemize}

%%%%%%%%%%%%%%%%%%%%%%%%%%%%%%%%%%%%%%%%
\paragraph{v0.5:} 2017/04/26

\begin{itemize}
\item
functionality in definition file
\end{itemize}


%%%%%%%%%%%%%%%%%%%%%%%%%%%%%%%%%%%%%%%%%%%%%%%%%%%%%%%%%%%%%%%%%%%%%%%%%%%%%%%%
%%%%%%%%%%%%%%%%%%%%%%%%%%%%%%%%%%%%%%%%%%%%%%%%%%%%%%%%%%%%%%%%%%%%%%%%%%%%%%%%
%%%%%%%%%%%%%%%%%%%%%%%%%%%%%%%%%%%%%%%%%%%%%%%%%%%%%%%%%%%%%%%%%%%%%%%%%%%%%%%%
\appendix

\settowidth\MacroIndent{\rmfamily\scriptsize 000\ }

 \DocInput{childdoc.dtx}

\end{document}
%</driver>
% \fi
%
% %%%%%%%%%%%%%%%%%%%%%%%%%%%%%%%%%%%%%%%%%%%%%%%%%%%%%%%%%%%%%%%%%%%%%%%%%%%%%%
% %%%%%%%%%%%%%%%%%%%%%%%%%%%%%%%%%%%%%%%%%%%%%%%%%%%%%%%%%%%%%%%%%%%%%%%%%%%%%%
% \section{Sample}
%\iffalse
%<*samplemain>
%\fi
%
% The following presents a sample document
% with two chapters, two parts, a title page,
% a compile flag as well as three forwarding files to set the flag.
% It consists of eight |.tex| files:
% \begin{center}
% \begin{tabular}{ll}
% |cdocsamp.tex|&main file\\
% |cdocsch1.tex|&include file for chapter 1\\
% |cdocsch2.tex|&include file for chapter 2\\
% |cdocspt3.tex|&include file for part 3\\
% |cdocspt4.tex|&include file for part 4\\
% |cdocsdrf.tex|&forwarding file for main file in draft mode\\
% |cdocsfi1.tex|&forwarding file for final version of chapter 1\\
% |cdocsfi2.tex|&forwarding file for final version of chapter 2\\
% \end{tabular}
% \end{center}
% Each of the eight files can be compiled directly by the \LaTeX{} compiler.
%
% %%%%%%%%%%%%%%%%%%%%%%%%%%%%%%%%%%%%%%
% \paragraph{Main File.}
%
% The main file is called |cdocsamp.tex|.
%
% Load the \textsf{childdoc} definitions and
% declare the filename for the main document:
%    \begin{macrocode}
\input{childdoc.def}
\childdocmain{}
%    \end{macrocode}

% Optional override for |\version| flag:
%    \begin{macrocode}
%%\ifchilddoc\else\providecommand{\version}{draft}\fi
%    \end{macrocode}

% Define the default values for the |\version| flag
% (|final| for the main file and |draft| for childs):
%    \begin{macrocode}
\ifchilddoc
\providecommand{\version}{draft}
\else
\providecommand{\version}{final}
\fi
%    \end{macrocode}

% Load the standard document class:
%    \begin{macrocode}
\documentclass[12pt]{article}
%    \end{macrocode}

% Start the document body:
%    \begin{macrocode}
\begin{document}
%    \end{macrocode}

% Declare a title page.
% Print title, part of document being processed and version flag:
%    \begin{macrocode}
\addtocounter{page}{-1}
\begin{center}
{\LARGE\bfseries{}childdoc example\par}
\vspace{1cm}
\ifchilddoc
\ifchilddocmanual part\else chapter\fi:
`\childdocname' of `\childdocjob'\par
\else
main document: `\childdocjob'\par
\fi
version: \version\par
\end{center}
\newpage
%    \end{macrocode}

% Manually include selected file,
% otherwise process as usual:
%    \begin{macrocode}
\ifchilddocmanual
\section*{part `\childdocname'}
\input{\childdocname}
\else
%    \end{macrocode}

% Include the two chapters:
%    \begin{macrocode}
\include{cdocsch1}
\include{cdocsch2}
%    \end{macrocode}

% Include the two parts unless only chapters should be displayed:
%    \begin{macrocode}
\ifchilddoc\else
\section{part three}
\input{cdocspt3}
\section{part four}
\input{cdocspt4}
\fi
%    \end{macrocode}

% Process as usual until here:
%    \begin{macrocode}
\fi
%    \end{macrocode}

% End of document body:
%    \begin{macrocode}
\end{document}
%    \end{macrocode}
%\iffalse
%</samplemain>
%\fi
%
% %%%%%%%%%%%%%%%%%%%%%%%%%%%%%%%%%%%%%%
% \paragraph{Chapter Include Files.}
%
% The include files are called |cdocsch1.tex| and |cdocsch2.tex|.
%
%\iffalse
%<*samplechap1|samplechap2>
%\fi

% Optional override for |\version| flag:
%    \begin{macrocode}
%%\providecommand{\version}{final}
%    \end{macrocode}

% Include the main document:
%    \begin{macrocode}
\input{childdoc.def}
\childdocof{cdocsamp}
%    \end{macrocode}

%\iffalse
%</samplechap1|samplechap2>
%\fi
%
%\iffalse
%<*samplechap1>
%\fi
% Some text for chapter 1:
%    \begin{macrocode}
\section{one}
some text in chapter one
%    \end{macrocode}

%\iffalse
%</samplechap1>
%\fi
% Some text for chapter 2:
%\iffalse
%<*samplechap2>
%\fi
%    \begin{macrocode}
\section{two}
more text in chapter two
%    \end{macrocode}

%\iffalse
%</samplechap2>
%\fi
%
% %%%%%%%%%%%%%%%%%%%%%%%%%%%%%%%%%%%%%%
% \paragraph{Part Include Files.}
%
% The include files are called |cdocspt3.tex| and |cdocspt4.tex|.
%
%\iffalse
%<*samplepart3|samplepart4>
%\fi

% Optional override for |\version| flag:
%    \begin{macrocode}
%%\providecommand{\version}{final}
%    \end{macrocode}

% Include the main document:
%    \begin{macrocode}
\input{childdoc.def}
\childdocby{cdocsamp}
%    \end{macrocode}

%\iffalse
%</samplepart3|samplepart4>
%\fi
%
%\iffalse
%<*samplepart3>
%\fi
% Some text for part 3:
%    \begin{macrocode}
some text in part three
%    \end{macrocode}

%\iffalse
%</samplepart3>
%\fi
% Some text for part 4:
%\iffalse
%<*samplepart4>
%\fi
%    \begin{macrocode}
more text in part four
%    \end{macrocode}

%\iffalse
%</samplepart4>
%\fi
%
% %%%%%%%%%%%%%%%%%%%%%%%%%%%%%%%%%%%%%%
% \paragraph{Forwarding for a Complete Draft.}
%
% The following forwarding file |cdocsdrf.tex|
% compiles the main document in draft mode:
%\iffalse
%<*sampledraft>
%\fi
%    \begin{macrocode}
\def\version{draft}
\input{childdoc.def}
\childdocforward{cdocsamp}
%    \end{macrocode}

%\iffalse
%</sampledraft>
%\fi
%
% %%%%%%%%%%%%%%%%%%%%%%%%%%%%%%%%%%%%%%
% \paragraph{Forwarding for Final Version of the Chapters.}
%
% The following forwarding files |cdocsfn1.tex| and |cdocsfn2.tex|
% (with identical content)
% compile the final versions of the child documents
% |cdocsch1.tex| and |cdocsch2.tex|, respectively:
%\iffalse
%<*samplefinal>
%\fi
%    \begin{macrocode}
\def\version{final}
\input{childdoc.def}
\childdocforwardprefix[cdocsamp]{cdocsfn}{cdocsch}
%    \end{macrocode}

%\iffalse
%</samplefinal>
%\fi
%
% %%%%%%%%%%%%%%%%%%%%%%%%%%%%%%%%%%%%%%
% \paragraph{Command Line Processing.}
%
% The following three command lines generate the output files
% |cdocscld|, |cdocscl1| and |cdocscl2|
% which should be identical to
% |cdocsdrf|, |cdocsch1| and |cdocsfn2|, respectively:
% \begin{center}
% \begin{tabular}{l}
% |latex -jobname cdocscld \|\\
% |  "\def\version{draft}\input{childdoc.def}\childdocforward{cdocsamp}"|\\
% |latex -jobname cdocscl1 \|\\
% |  "\input{childdoc.def}\childdocforward[cdocsamp]{cdocsch1}"|\\
% |latex -jobname cdocscl2 \|\\
% |  "\def\version{final}\input{childdoc.def}\childdocforward{cdocsch2}"|
% \end{tabular}
% \end{center}
% Note that the trailing backslash on each first line
% merely continues the input to the second line
% (for convenient cut ant paste).
% Furthermore, the command |latex| can be replaced by any
% of its alternative versions such as |pdflatex|.
%
% %%%%%%%%%%%%%%%%%%%%%%%%%%%%%%%%%%%%%%%%%%%%%%%%%%%%%%%%%%%%%%%%%%%%%%%%%%%%%%
% %%%%%%%%%%%%%%%%%%%%%%%%%%%%%%%%%%%%%%%%%%%%%%%%%%%%%%%%%%%%%%%%%%%%%%%%%%%%%%
% \section{Implementation}
%\iffalse
%<*package>
%\fi
%
% This section describes the definitions file |childdoc.def|.

% The definitions cannot be loaded using |\usepackage| or |\RequirePackage|
% which has a mechanism to prevent loading a style file more than once.
% When loading the definitions by means of |\input|
% multiple instances have to be prevented manually:
%\iffalse
%This code needs to be before the `\ProvidesFile' directive
%which is defined at the beginning of this file.
%Therefore it is also placed there and commented out here.
%</package>
%<*discard>
%\fi
%    \begin{macrocode}
\ifdefined\childdocmain\endinput\fi
%    \end{macrocode}
%\iffalse
%</discard>
%<*package>
%\fi
%
% \macro{\ifchilddoc}
% \macro{\ifchilddocmanual}
% The conditional |\ifchilddoc| tells whether a
% child (true) or main (false) document is being compiled.
% The conditional |\ifchilddocmanual| tells whether
% the |\includeonly| mechanism is used (false) or
% the selection of child files must be performed manually (true).
% The definitions initialise to false:
%    \begin{macrocode}
\newif\ifchilddoc
\newif\ifchilddocmanual
%    \end{macrocode}

% \macro{\childdocname}
% \macro{\childdocjob}
% The macro |\childdocname| stores the name of the main document
% to be compiled. The macro |\childdocjob| stores the name of
% the document on which the \LaTeX{} compiler was originally invoked.
% The content of |\jobname| cannot be compared
% to filenames specified in the source due to different catcodes.
% The following code rescans |\jobname|, stores the result
% in |\childdocname| and saves a copy in |\childdocjob|:
%    \begin{macrocode}
\edef\childdocname{\scantokens\expandafter{\jobname\noexpand}}
\let\childdocjob\childdocname
%    \end{macrocode}

% \macro{\childdocdisable}
% The macro |\childdocdisable| prevents the main file
% from being processed more than once.
% At this stage, the main document command |\childdocmain|
% is assumed to be called once again where it should do nothing.
% Any subsequent call to it should prevent
% a secondary processing of the main document
% It overwrites the forwarding commands
% |\childdocof| and |\childdocforward|
% with empty macros to prevent further inclusions of the main document:
%    \begin{macrocode}
\newcommand{\childdocdisable}
{
  \renewcommand{\childdocmain}[1]{\renewcommand{\childdocmain}[1]{\endinput}}
  \renewcommand{\childdocof}[1]{}
  \renewcommand{\childdocby}[2][]{}
  \renewcommand{\childdocforward}[2][]{}
  \renewcommand{\childdocdisable}{}
}
%    \end{macrocode}

% \macro{\childdocmain}
% The macro |\childdocmain| is to be called at the top of the main file
% with nothing or the main filename (without extension) as argument.
% First, it breaks loops.
% If the argument is not empty and does not match |\childdocname|
% (which is set by the first inclusion of |childdoc.def|),
% |\ifchilddoc| is set to true, |\includeonly| is applied to the child file
% and |\jobname| is set to the main file
% (for proper handling of |.aux| files):
%    \begin{macrocode}
\newcommand{\childdocmain}[1]
{
  \childdocdisable\childdocmain{}
  \if?#1?\else
    \begingroup
      \def\childdoctmp{#1}
      \ifx\childdoctmp\childdocname
        \def\childdoctmp{}
      \else
        \def\childdoctmp
        {
          \childdoctrue
          \includeonly{\childdocname}
          \def\childdocjob{#1}
          \def\jobname{#1}
        }
      \fi
      \expandafter
    \endgroup
    \childdoctmp
  \fi
}
%    \end{macrocode}

% \macro{\childdocof}
% The command |\childdocof| redirects
% compilation to the main file |#1|.
%    \begin{macrocode}
\newcommand{\childdocof}[1]
{
  \childdocdisable
  \childdoctrue
  \includeonly{\childdocname}
  \def\jobname{#1}
  \def\childdocjob{#1}
  \input{#1}
}
%    \end{macrocode}

% \macro{\childdocby}
% The command |\childdocby| ....
%    \begin{macrocode}
\newcommand{\childdocby}[2][]
{
  \childdocdisable
  \childdoctrue
  \childdocmanualtrue
  \if?#1?\else
    \def\jobname{#2}
  \fi
  \def\childdocjob{#2}
  \input{#2}
  \endinput
}
%    \end{macrocode}

% \macro{\childdocforward}
% The command |\childdocforward| redirects
% compilation to the main file or
% (if the optional argument is given) a child file.
% Parameters are set as if the main file
% or a child file starting with |\childdocof| was compiled.
% Then compilation is handed over to the main file:
%    \begin{macrocode}
\newcommand{\childdocforward}[2][]
{
  \begingroup
    \if?#1?
      \def\childdoctmp
      {
        \def\childdocname{#2}
        \def\childdocjob{#2}
        \def\jobname{#2}
        \input{#2}
        \endinput
      }
    \else
      \def\childdoctmp
      {
        \childdocdisable
        \def\childdocname{#2}
        \childdoctrue
        \includeonly{#2}
        \def\childdocjob{#1}
        \def\jobname{#1}
        \input{#1}
        \endinput
      }
    \fi
    \expandafter
  \endgroup
  \childdoctmp
}
%    \end{macrocode}

% \macro{\childdocforwardprefix}
% The command |\childdocforwardprefix| redirects
% compilation to the main or a child file by means of a pattern.
% The prefix |#1| in the current filename is replaced by |#2|
% and the suffix of the current filename is kept
% (it is assumed that the filename does not contain the substring `|~~~|'
% which is used as a delimiter).
% Compilation is handed over to the new file by |\childdocforward|:
%    \begin{macrocode}
\newcommand{\childdocforwardprefix}[3][]
{
  \begingroup
    \def\childdocextract #2##1~~~{\def\childdoctmp{\childdocforward[#1]{#3##1}}}
    \expandafter\childdocextract\childdocname~~~
    \expandafter
  \endgroup
  \childdoctmp
}
%    \end{macrocode}

% \macro{\childdoc}
% The deprecated macro |\childdoc| is a legacy version of |\childdocmain|:
%    \begin{macrocode}
\newcommand{\childdoc}{\childdocmain}
%    \end{macrocode}

% \macro{\childdocredirect}
% The deprecated macro |\childdocredirect| is a legacy version
% of |\childdocforward| and |\childdocforwardprefix|:
%    \begin{macrocode}
\newcommand{\childdocredirect}[2][]
{
  \begingroup
    \if?#1?
      \def\childdoctmp{\childdocforward{#2}}
    \else
      \def\childdoctmp{\childdocforwardprefix{#1}{#2}}
    \fi
    \expandafter
  \endgroup
  \childdoctmp
}
%    \end{macrocode}

%\iffalse
%</package>
%\fi
%
\endinput
|\\
|\childdocforward{|\textit{main}|}|
\end{tabular}
\end{center}
%
Likewise, the following files |final|\textit{nn}|.tex|
compile the final version of the child document
|child|\textit{nn}|.tex|:
%
\begin{center}
\begin{tabular}{l}
|\def\version{final}|\\
|% \iffalse
%
% childdoc.dtx Copyright (C) 2017-2018 Niklas Beisert
%
% This work may be distributed and/or modified under the
% conditions of the LaTeX Project Public License, either version 1.3
% of this license or (at your option) any later version.
% The latest version of this license is in
%   http://www.latex-project.org/lppl.txt
% and version 1.3 or later is part of all distributions of LaTeX
% version 2005/12/01 or later.
%
% This work has the LPPL maintenance status `maintained'.
%
% The Current Maintainer of this work is Niklas Beisert.
%
% This work consists of the files childdoc.dtx and childdoc.ins
% and the derived files childdoc.def and cdocsamp.tex with
% cdocsch1.tex, cdocsch2.tex, cdocsdrf.tex, cdocsfn1.tex, cdocsfn2.tex.
%
%<package>\ifdefined\childdocmain\endinput\fi
%<package>\ProvidesFile{childdoc.def}[2018/12/30 v2.0 child document driver]
%<samplemain>\ProvidesFile{cdocsamp.tex}[2018/12/30 v2.0 sample for childdoc]
%<*driver>
%\ProvidesFile{childdoc.drv}[2018/12/30 v2.0 childdoc reference manual file]
\PassOptionsToClass{10pt,a4paper}{article}
\documentclass{ltxdoc}

\usepackage[margin=35mm]{geometry}
\usepackage{hyperref}
\usepackage{hyperxmp}
\usepackage[usenames]{color}

\hypersetup{colorlinks=true}
\hypersetup{pdfstartview=FitH}
\hypersetup{pdfpagemode=UseNone}
\hypersetup{pdfsource={}}
\hypersetup{pdflang={en-UK}}
\hypersetup{pdfcopyright={Copyright 2017-2018 Niklas Beisert.
  This work may be distributed and/or modified under the
  conditions of the LaTeX Project Public License, either version 1.3
  of this license or (at your option) any later version.}}
\hypersetup{pdflicenseurl={http://www.latex-project.org/lppl.txt}}
\hypersetup{pdfcontactaddress={ETH Zurich, ITP, HIT K,
  Wolfgang-Pauli-Strasse 27}}
\hypersetup{pdfcontactpostcode={8093}}
\hypersetup{pdfcontactcity={Zurich}}
\hypersetup{pdfcontactcountry={Switzerland}}
\hypersetup{pdfcontactemail={nbeisert@itp.phys.ethz.ch}}
\hypersetup{pdfcontacturl={http://people.phys.ethz.ch/\xmptilde nbeisert/}}

\newcommand{\secref}[1]{\hyperref[#1]{section \ref*{#1}}}

\parskip1ex
\parindent0pt
\let\olditemize\itemize
\def\itemize{\olditemize\parskip0pt}

\begin{document}

\title{The \textsf{childdoc} Package}
\hypersetup{pdftitle={The childdoc Package}}
\author{Niklas Beisert\\[2ex]
  Institut f\"ur Theoretische Physik\\
  Eidgen\"ossische Technische Hochschule Z\"urich\\
  Wolfgang-Pauli-Strasse 27, 8093 Z\"urich, Switzerland\\[1ex]
  \href{mailto:nbeisert@itp.phys.ethz.ch}
  {\texttt{nbeisert@itp.phys.ethz.ch}}}
\hypersetup{pdfauthor={Niklas Beisert}}
\hypersetup{pdfsubject={Manual for the LaTeX2e Package childdoc}}
\date{30 December 2018, \textsf{v2.0}}
\maketitle

\begin{abstract}\noindent
\textsf{childdoc} is a \LaTeXe{} package
that enables the direct compilation
of document sections included by |\include|
to individual files.
\end{abstract}

\begingroup
\parskip0ex
\tableofcontents
\endgroup

%%%%%%%%%%%%%%%%%%%%%%%%%%%%%%%%%%%%%%%%%%%%%%%%%%%%%%%%%%%%%%%%%%%%%%%%%%%%%%%%
%%%%%%%%%%%%%%%%%%%%%%%%%%%%%%%%%%%%%%%%%%%%%%%%%%%%%%%%%%%%%%%%%%%%%%%%%%%%%%%%
\section{Introduction}

\LaTeX{} provides a mechanism to structure a large document (such as a book)
into a main file and several child files (containing the chapters)
using the |\include| command.
This mechanism is beneficial for documents
which span hundreds of pages in order to
make the source file(s) more manageable.
Moreover, compilation can be restricted to
selected child files by means of the |\includeonly| command.
The latter feature can be used to reduce the compilation time while editing
(this was significantly more useful in the earlier days of \LaTeX{})
or to generate a smaller document which is easier to navigate.
Another application of |\includeonly| is to generate
documents consisting of selected parts of the complete document.

However, there are a few drawbacks of the plain |\include| mechanism:
\begin{itemize}
\item
The child files cannot be compiled on their own,
they can only be compiled via the main file.
A naive editing environment
(such as a text editor with an option
to have the current file processed by \LaTeX)
may require one to switch to the main file before compiling;
attempting to compile the child file produces errors.
\item
The main file must be modified (each time)
to adjust the |\includeonly| command
to the present needs. This easily leaves the main file in a messy state.
\item
The generated document will always carry the filename
of the main document. This is inconvenient if
several child files are to be compiled and
to be kept for distribution.
\end{itemize}

The present package provides a simple interface
to make child files individually compilable by \LaTeX{}.
Compiling a child file then has the same effect as compiling
the main file with an |\includeonly| command
to select the appropriate child.
Moreover the generated document will carry the name of the child
rather than the main file.
This resolves all three above issues.

This feature is meant to make the editing of books,
thesis documents and lecture notes somewhat more convenient.
However, the package can also be used efficiently for
composing a series of documents (such as exercise sheets)
which are typically distributed individually.
It then assists the author in generating the individual documents
(potentially in different versions)
as well as a document containing the collected series.
Another application is in developing style files
or other kinds of included material
where compilation of the style file could redirect
to a sample or test file.

%%%%%%%%%%%%%%%%%%%%%%%%%%%%%%%%%%%%%%%%%%%%%%%%%%%%%%%%%%%%%%%%%%%%%%%%%%%%%%%%
%%%%%%%%%%%%%%%%%%%%%%%%%%%%%%%%%%%%%%%%%%%%%%%%%%%%%%%%%%%%%%%%%%%%%%%%%%%%%%%%
\section{Usage}

First of all, the package \textsf{childdoc} is \emph{not} a standard
\LaTeXe{} |.sty| style file! Therefore it needs to be invoked in
a non-standard way.

%%%%%%%%%%%%%%%%%%%%%%%%%%%%%%%%%%%%%%%%%%%%%%%%%%%%%%%%%%%%%%%%%%%%%%%%%%%%%%%%
\subsection{Included Files}
\label{sec:include}

%%%%%%%%%%%%%%%%%%%%%%%%%%%%%%%%%%%%%%%%
\DescribeMacro{\childdocmain}
To use the package, add the commands
\begin{center}
\begin{tabular}{l}
|\input{childdoc.def}|\\
|\childdocmain{}|\\
\end{tabular}
\end{center}
at the very top of the main \LaTeX{} file,
in particular \emph{before} the |\documentclass| statement!
The argument of |\childdocmain| should be left empty
(but it must be present).

%%%%%%%%%%%%%%%%%%%%%%%%%%%%%%%%%%%%%%%%
\DescribeMacro{\childdocof}
Furthermore, add the commands
\begin{center}
\begin{tabular}{l}
|\input{childdoc.def}|\\
|\childdocof{|\textit{main}|}|\\
\end{tabular}
\end{center}
at the top of every child file \textit{child}
which is included by |\include{|\textit{child}|}|
from within the main file
(or at least for those files to be compiled individually).
The argument \textit{main} must be the filename of the main file.

There are a couple of
considerations in setting up the main and child documents:

%%%%%%%%%%%%%%%%%%%%%%%%%%%%%%%%%%%%%%%%
\paragraph{Restrictions.}

Please note the following restrictions:
\begin{itemize}
\item
|\childdocmain| must be called with one argument \textit{main}
to ensure compatibility with earlier version of the package.
It must either be empty (|\childdocmain{}|)
or precisely match the filename of the main file in which it is specified.
See \secref{sec:detection} for further information.
\item
The filename \textit{main} must be specified without the |.tex| extension.
\item
The filename \textit{main} is case sensitive
(even in case-insensitive file systems)
due to internal string comparison.
\item
The argument \textit{main} should be fully expanded, it cannot be a macro.
\item
Subdirectories and special characters should be avoided in filenames.
\item
The command |\childdocmain{|\textit{main}|}| must be followed by a whitespace.
It should not be followed immediately by another command
or by a comment mark `|%|'.
This is because the \TeX{} parser reads the token immediately following
the argument of |\childdocmain| and puts it
at the beginning of every child section;
however, a white\-space is ignored.
\end{itemize}

%%%%%%%%%%%%%%%%%%%%%%%%%%%%%%%%%%%%%%%%
\paragraph{Content of Main File.}

It is advisable to place all content in the child files included by |\include|.
Any output contained in the main file will appear in all child documents
unless suppressed manually;
it cannot be suppressed automatically by the |\includeonly| directive
and thus should normally be avoided.
A method to include some content in the main file
by means of conditional processing is described in \secref{sec:conditional}.

%%%%%%%%%%%%%%%%%%%%%%%%%%%%%%%%%%%%%%%%
\paragraph{Page Numbering.}

When only a part of the document is compiled,
the appropriate numbering of pages
(as well as other status parameters)
is determined from the |.aux| files.
The latter contain information from previous passes.
However this information needs to propagate through
all intermediate child documents.
Therefore the page numbering in child documents may well
be inconsistent until the complete document is compiled at least once.

A useful (if unconventional) way to always ensure a consistent
page numbering is to restart the numbering in each child document
and denote the pages by `\textit{child}|.|\textit{page}'
where \textit{child} represents the chapter/section number of the child file.
This can be achieved by the command
|\numberwithin{page}{|\textit{child}|}|
of the \textsf{amsmath} package
where \textit{child} can be |chapter| or |section|
depending on the chosen structuring.
Alternatively, one can modify the macro |\thepage| appropriately
and reset the counter |page| at the start of each child file.

%%%%%%%%%%%%%%%%%%%%%%%%%%%%%%%%%%%%%%%%%%%%%%%%%%%%%%%%%%%%%%%%%%%%%%%%%%%%%%%%
\subsection{Conditional Processing}
\label{sec:conditional}

The package provides a mechanism to compile different versions
of a document. To customise the versions further some conditional processing
can come in handy to distinguish which version is being compiled.
The package provides two macros to describe the compilation context:

%%%%%%%%%%%%%%%%%%%%%%%%%%%%%%%%%%%%%%%%
\DescribeMacro{\ifchilddoc}
The conditional |\ifchilddoc| distinguishes between the compilation of
child documents and the main document:
%
\begin{center}
|\ifchilddoc |\textit{child-code}| |[|\||else |\textit{main-code}]| \||fi|
\end{center}

%%%%%%%%%%%%%%%%%%%%%%%%%%%%%%%%%%%%%%%%
\DescribeMacro{\childdocname}
\DescribeMacro{\childdocjob}
The macro |\childdocname| contains the filename (without extension)
of the main or child file being processed.
Note that |\childdocjob| will always contain the name of the main file.

%%%%%%%%%%%%%%%%%%%%%%%%%%%%%%%%%%%%%%%%
\paragraph{Title Page.}

Conditional processing can be used to include a title or banner page
in the main document when proper precautions are taken.
Importantly, the code in the main file should ensure that the page counter
(as well as other status parameters which are stored in the |.aux| files)
takes the same value after the conditional processing.
Otherwise the page numbers may take divergent values
depending on which part is compiled.

For example, a title page could be declared by:
%
\begin{center}
\begin{tabular}{l}
|\ifchilddoc\||else|\\
|\addtocounter{page}{-1}|\\
\textit{code for title page}\\
|\newpage|\\
|\||fi|
\end{tabular}
\end{center}
%
A banner page for the child documents can be generated by:
%
\begin{center}
\begin{tabular}{l}
|\ifchilddoc|\\
|\addtocounter{page}{-1}|\\
\textit{code for banner page}\\
|\newpage|\\
|\||fi|
\end{tabular}
\end{center}
%
Here one could write a message such as:
\begin{center}
|This is the part \childdocname{} of \childdocjob{}.|
\end{center}

%%%%%%%%%%%%%%%%%%%%%%%%%%%%%%%%%%%%%%%%%%%%%%%%%%%%%%%%%%%%%%%%%%%%%%%%%%%%%%%%
\subsection{Flags}
\label{sec:flags}

The package makes it easy to generate different versions
of the main or child documents.
To this end compilation flags can be defined
and assigned different default values.
They will be particularly useful in conjunction
with the forwarding mechanism described in \secref{sec:forward}.

For example, it may be useful to have a flag |\version|
which can be set to |draft| or |final|.
The document source will contain some conditional code
depending on the value of |\version|.
Suppose further, the flag should default to |final| for the main file
and to |draft| for child files
which is a natural assignment for editing the document.
This is achieved by placing the following code
in the preamble of the main document
(below the |\childdocmain| directive):
%
\begin{center}
\begin{tabular}{l}
|\ifchilddoc|\\
|\providecommand{\version}{draft}|\\
|\||else|\\
|\providecommand{\version}{final}|\\
|\||fi|
\end{tabular}
\end{center}
%
The definition by |\providecommand| makes sure
that previous definitions are not overwritten.
Further statements |\providecommand{\version}{...}|
can thus be added before the above code to override it.

For the main file, one might add a line
(between |\childdocmain| and the above block)
%
\begin{center}
|%\ifchilddoc\||else\providecommand{\version}{draft}\||fi|
\end{center}
%
which can be uncommented to produce a draft version.
Likewise one can add a line to the very top of a child file
(above the |\childdocof{|\textit{main}|}| directive)
%
\begin{center}
|%\providecommand{\version}{final}|
\end{center}
%
which can be uncommented to produce the final version of this child document.

%%%%%%%%%%%%%%%%%%%%%%%%%%%%%%%%%%%%%%%%%%%%%%%%%%%%%%%%%%%%%%%%%%%%%%%%%%%%%%%%
\subsection{Forwarding}
\label{sec:forward}

Different versions of the main or child documents
using compilation flags as described in \secref{sec:flags}
can be (permanently) stored in different files
for convenient compilation, viewing and distribution.
To this end, the package defines a command
to pass on compilation to a different file:

%%%%%%%%%%%%%%%%%%%%%%%%%%%%%%%%%%%%%%%%
\DescribeMacro{\childdocforward}
The command |\childdocforward| redirects processing to
another source file:
%
\begin{center}
\begin{tabular}{l}
|\input{childdoc.def}|\\
|\childdocforward[|\textit{main}|]{|\textit{dest}|}|\\
\end{tabular}
\end{center}
%
The argument \textit{dest} is the destination file
(without extension).
It should be the main file or one of the child files.
Note that further \textsf{childdoc} directives
such as |\childdocof| and |\childdocforward|
in the indicated file will be processed in this form.
The optional argument \textit{main}
passes on directly to the main file \textit{main}
while pretending to compile the child \textit{dest}.
This form behaves as if \textit{dest}
issues |\childdocof{|\textit{main}|}| right away,
and no further \textsf{childdoc} directives will be processed.

%%%%%%%%%%%%%%%%%%%%%%%%%%%%%%%%%%%%%%%%
\DescribeMacro{\...prefix}
In the alternative form |\childdocforwardprefix|,
%
\begin{center}
\begin{tabular}{l}
|\input{childdoc.def}|\\
|\childdocforwardprefix[|\textit{main}|]{|\textit{prefix}|}{|\textit{dest}|}|
\end{tabular}
\end{center}
%
the destination file is determined by a pattern
depending on the current file:
To make this work, the current file must be called
`{\textit{prefix}\hspace{0.2em}\textit{suffix}}'
with \textit{prefix} matching precisely the argument.
Processing is then passed on to the file
`{\textit{dest}\hspace{0.2em}\textit{suffix}}'.
Surely, the same effect is achieved by
directly specifying the
argument `{\textit{dest}\hspace{0.2em}\textit{suffix}}'
in the first form.
However, that requires to set up a different file
for each child. With the alternative form of the command
all these files can have exactly the same content
which simplifies setting them up and maintaining them.

For example, the following file |draft.tex|
with a compilation flag |\version| as described in \secref{sec:flags}
compiles the main document as a draft:
%
\begin{center}
\begin{tabular}{l}
|\def\version{draft}|\\
|\input{childdoc.def}|\\
|\childdocforward{|\textit{main}|}|
\end{tabular}
\end{center}
%
Likewise, the following files |final|\textit{nn}|.tex|
compile the final version of the child document
|child|\textit{nn}|.tex|:
%
\begin{center}
\begin{tabular}{l}
|\def\version{final}|\\
|\input{childdoc.def}|\\
|\childdocforwardprefix{final}{child}|
\end{tabular}
\end{center}
%

Note that when several versions of a main file and/or of each child file
are to be generated, it may be convenient to set up a |Makefile| or
shell script to automatise the process.

%%%%%%%%%%%%%%%%%%%%%%%%%%%%%%%%%%%%%%%%%%%%%%%%%%%%%%%%%%%%%%%%%%%%%%%%%%%%%%%%
\subsection{Command Line Processing}
\label{sec:commandline}

The effect of redirection files can also be achieved by invoking
the \LaTeX{} compiler with a more elaborate command line.
Most conveniently this should be done as part
of a shell script or a |Makefile|.

When using \textsf{childdoc} in the main file, the following
command lines effectively perform a redirection
(note that depending on the shell being used,
backslashes may have to be doubled: `|\|' $\to$ `|\\|'):
%
\begin{center}
|... -jobname "|\textit{target}|" |\\|"|[\textit{flags}]%
|\input{childdoc.def}\childdocforward[|\textit{main}|]{|\textit{dest}|}"|
\end{center}
%
Here \textit{target} is the name of the output file,
\textit{main} is the name of the main file
and \textit{dest} is the name of the main or child file to be processed
(all filenames without extensions).
The optional argument \textit{main} can be omitted
if \textit{main} matches \textit{dest}.
Optionally, compilation \textit{flags} can be defined via |\def| commands.
This command line makes the \TeX{} engine believe
it is compiling the file \textit{target}
whose content is specified as the latter parameter.
The provided code then forwards the processing to
\textit{main} or \textit{dest} as described in \secref{sec:forward}.

%%%%%%%%%%%%%%%%%%%%%%%%%%%%%%%%%%%%%%%%%%%%%%%%%%%%%%%%%%%%%%%%%%%%%%%%%%%%%%%%
\subsection{Include by Input}
\label{sec:input}

Including child documents by |\include| has some restrictions by design.
Most notably, the content of a child document always occupies
its own set of pages; pages cannot be shared between child documents.
Usually, this behaviour makes perfect sense
because each child document contain an essential part of the document.
However, in some situations it may be desirable to compose
a document from a collection of parts
without having mandatory page breaks between then.
For this case, the package
provides a mechanism to include parts
by |\input| which can also be processed individually.
However, by construction this mechanism
requires manual handling of the content to be output.

%%%%%%%%%%%%%%%%%%%%%%%%%%%%%%%%%%%%%%%%
\DescribeMacro{\ifchilddocmanual}
The main file should be prepared as usual, see \secref{sec:include}.
However, the document body must make a distinction
between processing of an individual part and of the main document, e.g.:
%
\begin{center}
\begin{tabular}{l}
|\ifchilddocmanual|\\
|\input{\childdocname}|\\
|\||else|\\
\textit{document body with }|\input{|\textit{part}|}|\\
|\||fi|
\end{tabular}
\end{center}
%
The conditional |\ifchilddocmanual| is true whenever
a part to be included by |\input| is being compiled,
and the name of the part is stored in |\childdocname|.

%%%%%%%%%%%%%%%%%%%%%%%%%%%%%%%%%%%%%%%%
\DescribeMacro{\childdocby}
Each part to be included by |\input| should start with:
%
\begin{center}
\begin{tabular}{l}
|\input{childdoc.def}|\\
|\childdocby{|\textit{main}|}|\\
\end{tabular}
\end{center}
%
The directive |\childdocby| is similar to |\childdocof|
described in \secref{sec:include},
but the subsequent selection of content must be done manually.
To that end, both |\ifchilddoc| and |\ifchilddocmanual|
will be true upon processing of a part,
and the name of the part is stored in |\childdocname|.
Note that |\jobname| will be set to the filename of the current part
so that each part receives an individual |.aux| file
that does not interfere with the |.aux| file(s) of the main document.
This behaviour can be altered by the alternative form
|\childdocby[*]{|\textit{main}|}| (with a non-empty optional argument)
which uses the |.aux| file of the main document
by setting |\jobname| to \textit{main}.

%%%%%%%%%%%%%%%%%%%%%%%%%%%%%%%%%%%%%%%%%%%%%%%%%%%%%%%%%%%%%%%%%%%%%%%%%%%%%%%%
\subsection{Driver Development}
\label{sec:driver}

The \textsf{childdoc} mechanism can also be use for the development
of definition files such as \LaTeX{} styles or classes.
This case differs from the above setup with multiple parts
included by |\include| in that no |\includeonly| should be invoked.
This can be achieved by starting the include file
(before |\ProvidesPackage|) with:
%
\begin{center}
\begin{tabular}{l}
|\input{childdoc.def}|\\
|\childdocforward{|\textit{main}|}|\\
\end{tabular}
\end{center}
%
or alternatively with:
%
\begin{center}
\begin{tabular}{l}
|\input{childdoc.def}|\\
|\childdocby{|\textit{main}|}|\\
\end{tabular}
\end{center}
%
Both forms have slightly different effects as described above.
The main file is prepared as usual, see \secref{sec:include}.

%%%%%%%%%%%%%%%%%%%%%%%%%%%%%%%%%%%%%%%%%%%%%%%%%%%%%%%%%%%%%%%%%%%%%%%%%%%%%%%%
\subsection{Legacy Detection}
\label{sec:detection}

The directive |\childdocmain| in the main file can detect
whether the complete document or merely a child is to be compiled
even without using the directive |\childdocof|.
This method is deprecated because it is less robust
and there is no compelling reason to use it;
it is merely provided for backward compatibility
and it may be removed in future versions.

If the detection mechanism is to be used,
it is mandatory to correctly specify
the filename of the main file as the argument of |\childdocmain|:
%
\begin{center}
\begin{tabular}{l}
|\input{childdoc.def}|\\
|\childdocmain{|\textit{main}|}|\\
\end{tabular}
\end{center}
%
If |\jobname| does not match the argument \textit{main} of |\childdocmain|,
it is assumed that |\jobname| points to the child file to be compiled.
When using |\childdocmain| with the main file specified as argument,
it suffices to start a child file
with just |\input{|\textit{main}|}|
without loading of the package and using |\childdocof|.
If instead all processing is done
with the appropriate \textsf{childdoc} directives,
the argument of \textit{main} of |\childdocmain| can be empty.

An alternative version of the command line processing described
in \secref{sec:commandline} using the detection mechanism reads:
%
\begin{center}
|... -jobname "|\textit{target}|" "|[\textit{flags}]%
[|\def\jobname{|\textit{dest}|}|]|\input{|\textit{main}|}"|
\end{center}

%%%%%%%%%%%%%%%%%%%%%%%%%%%%%%%%%%%%%%%%%%%%%%%%%%%%%%%%%%%%%%%%%%%%%%%%%%%%%%%%
\subsection{Manual Code}
\label{sec:manual}

In case one cannot be certain whether the definitions file |childdoc.def|
is installed on the target \TeX{} distribution
and one prefers not to ship it,
it is conceivable to paste a few relevant commands into the sources.

To that end, drop all statements |\input{childdoc.def}|
and perform the replacements as outlined below.
Instead of |\childdocmain{|\textit{main}|}| add the following code
to the top of the main file:
%
\begin{center}
\begin{tabular}{l}
|\||ifdefined\childdocname\endinput\||fi\newif\ifchilddoc|\\
|\edef\childdocname{\scantokens\expandafter{\jobname\noexpand}}|\\
|\def\childdocmain{|\textit{main}|}\||ifx\childdocmain\childdocname\||else|\\
|\childdoctrue\includeonly{\childdocname}\let\jobname\childdocmain\||fi|\\
\end{tabular}
\end{center}
%
Instead of |\childdocof{|\textit{main}|}| just include the main file
at the top of each child file:
%
\begin{center}
|\input{|\textit{main}|}|
\end{center}
%
A simple redirection |\childdocforward{|\textit{dest}|}| is achieved by:
%
\begin{center}
|\def\jobname{|\textit{dest}|}\input{\jobname}|
\end{center}
%
The redirection with prefix
|\childdocforwardprefix[|\textit{prefix}|]{|\textit{dest}|}|
is accomplished by:
%
\begin{center}
\begin{tabular}{l}
|{\edef\jobname{\scantokens\expandafter{\jobname\noexpand}}|\\
|\def\redirectjob |\textit{prefix}|#1~~~{\gdef\jobname{|\textit{dest}|#1}}|\\
|\expandafter\redirectjob\jobname~~~}\input{\jobname}|
\end{tabular}
\end{center}

In an alternative approach,
child documents can be compiled by a specific command line
without additional code or specific definitions:
%
\begin{center}
|... -jobname "|\textit{target}|" "|[\textit{flags}]%
|\includeonly{|\textit{dest}|}\input{|\textit{main}|}"|
\end{center}
%

%%%%%%%%%%%%%%%%%%%%%%%%%%%%%%%%%%%%%%%%%%%%%%%%%%%%%%%%%%%%%%%%%%%%%%%%%%%%%%%%
%%%%%%%%%%%%%%%%%%%%%%%%%%%%%%%%%%%%%%%%%%%%%%%%%%%%%%%%%%%%%%%%%%%%%%%%%%%%%%%%
\section{Information}

%%%%%%%%%%%%%%%%%%%%%%%%%%%%%%%%%%%%%%%%%%%%%%%%%%%%%%%%%%%%%%%%%%%%%%%%%%%%%%%%
\subsection{Copyright}

Copyright \copyright{} 2017--2018 Niklas Beisert

This work may be distributed and/or modified under the
conditions of the \LaTeX{} Project Public License, either version 1.3
of this license or (at your option) any later version.
The latest version of this license is in
  \url{http://www.latex-project.org/lppl.txt}
and version 1.3 or later is part of all distributions of \LaTeX{}
version 2005/12/01 or later.

This work has the LPPL maintenance status `maintained'.

The Current Maintainer of this work is Niklas Beisert.

This work consists of the files |README.txt|, |childdoc.ins| and |childdoc.dtx|
as well as the derived files |childdoc.def|, |cdocsamp.tex|
with |cdocsch1.tex|, |cdocsch2.tex|, |cdocspt3.tex|, |cdocspt4.tex|,
|cdocsdrf.tex|, |cdocsfn1.tex|, |cdocsfn2.tex|
as well as |childdoc.pdf|.

%%%%%%%%%%%%%%%%%%%%%%%%%%%%%%%%%%%%%%%%%%%%%%%%%%%%%%%%%%%%%%%%%%%%%%%%%%%%%%%%
\subsection{Files and Installation}

The package consists of the files:
%
\begin{center}
\begin{tabular}{ll}
    |README.txt|   & readme file \\
    |childdoc.ins| & installation file \\
    |childdoc.dtx| & source file \\
    |childdoc.def| & definition file \\
    |cdocsamp.tex| & sample main file \\
    |cdocsch1.tex| & sample include file \\
    |cdocsch2.tex| & sample include file \\
    |cdocspt3.tex| & sample part file \\
    |cdocspt4.tex| & sample part file \\
    |cdocsdrf.tex| & sample redirection file \\
    |cdocsfn1.tex| & sample redirection file \\
    |cdocsfn2.tex| & sample redirection file \\
    |childdoc.pdf| & manual
\end{tabular}
\end{center}
%
The distribution consists of the files
|README.txt|, |childdoc.ins| and |childdoc.dtx|.
%
\begin{itemize}
\item
Run (pdf)\LaTeX{} on |childdoc.dtx|
to compile the manual |childdoc.pdf| (this file).
\item
Run \LaTeX{} on |childdoc.ins| to create the definitions file |childdoc.def|
and the sample |cdocsamp.tex| with include files
|cdocsch1.tex|, |cdocsch2.tex|, |cdocspt3.tex|, |cdocspt4.tex|,
|cdocsdrf.tex|, |cdocsfn1.tex|, |cdocsfn2.tex|.
Then copy the file |childdoc.def| to an appropriate directory of your \LaTeX{}
distribution, e.g.\ \textit{texmf-root}|/tex/latex/childdoc|.
\end{itemize}

%%%%%%%%%%%%%%%%%%%%%%%%%%%%%%%%%%%%%%%%%%%%%%%%%%%%%%%%%%%%%%%%%%%%%%%%%%%%%%%%
\subsection{Related CTAN Packages}

There are several other packages which offer a similar functionality:
%
\begin{itemize}
\item
The packages
\href{http://ctan.org/pkg/docmute}{\textsf{docmute}},
\href{http://ctan.org/pkg/includex}{\textsf{includex}} and
\href{http://ctan.org/pkg/standalone}{\textsf{standalone}}
provide commands to include only the document body of
a child file thus allowing both files to be compiled individually.
\item
The packages \href{http://ctan.org/pkg/subdocs}{\textsf{subdocs}}
and \href{http://ctan.org/pkg/subfiles}{\textsf{subfiles}}
provide structures in which the main and child documents can be
encapsulated and allowing them to be compiled individually.
The inclusion mechanism is different from the conventional |\include|.
\item
The package \href{http://ctan.org/pkg/combine}{\textsf{combine}}
is an elaborate solution to combine several documents into one.
\end{itemize}
%
See also the CTAN topic \href{http://ctan.org/topic/subdocs}{\textsf{subdocs}}
for further related packages.
The present package differs from the above solutions in that
a document structure constructed with the conventional |\include| mechanism
just needs two extra commands at the top of every file
such that all constituent files can be compiled individually.

%%%%%%%%%%%%%%%%%%%%%%%%%%%%%%%%%%%%%%%%%%%%%%%%%%%%%%%%%%%%%%%%%%%%%%%%%%%%%%%%
%\subsection{Feature Suggestions}
%
%The following is a list of features which may be useful for future
%versions of this package:
%%
%\begin{itemize}
%\item
%\ldots
%\end{itemize}

%%%%%%%%%%%%%%%%%%%%%%%%%%%%%%%%%%%%%%%%%%%%%%%%%%%%%%%%%%%%%%%%%%%%%%%%%%%%%%%%
\subsection{Revision History}

%%%%%%%%%%%%%%%%%%%%%%%%%%%%%%%%%%%%%%%%
\paragraph{v2.0:} 2018/12/30

\begin{itemize}
\item
immediate forward processing
\item
added |\childdocby| mechanism
\item
manual restructured
\end{itemize}

%%%%%%%%%%%%%%%%%%%%%%%%%%%%%%%%%%%%%%%%
\paragraph{v1.6:} 2018/01/17

\begin{itemize}
\item
application for development of include files
\item
corrections to manual
\end{itemize}

%%%%%%%%%%%%%%%%%%%%%%%%%%%%%%%%%%%%%%%%
\paragraph{v1.5:} 2017/05/21

\begin{itemize}
\item
more complete structuring introduced
\item
|\childdocof| introduced
\item
|\childdoc| renamed to |\childdocmain|
\item
|\childredirect| renamed to |\childdocforward| and |\childdocforwardprefix|
and functionality expanded
\end{itemize}

%%%%%%%%%%%%%%%%%%%%%%%%%%%%%%%%%%%%%%%%
\paragraph{v1.0:} 2017/04/27

\begin{itemize}
\item
manual and install package
\item
first version published on CTAN
\end{itemize}

%%%%%%%%%%%%%%%%%%%%%%%%%%%%%%%%%%%%%%%%
\paragraph{v0.6:} 2017/04/26

\begin{itemize}
\item
redirection mechanism added
\end{itemize}

%%%%%%%%%%%%%%%%%%%%%%%%%%%%%%%%%%%%%%%%
\paragraph{v0.5:} 2017/04/26

\begin{itemize}
\item
functionality in definition file
\end{itemize}


%%%%%%%%%%%%%%%%%%%%%%%%%%%%%%%%%%%%%%%%%%%%%%%%%%%%%%%%%%%%%%%%%%%%%%%%%%%%%%%%
%%%%%%%%%%%%%%%%%%%%%%%%%%%%%%%%%%%%%%%%%%%%%%%%%%%%%%%%%%%%%%%%%%%%%%%%%%%%%%%%
%%%%%%%%%%%%%%%%%%%%%%%%%%%%%%%%%%%%%%%%%%%%%%%%%%%%%%%%%%%%%%%%%%%%%%%%%%%%%%%%
\appendix

\settowidth\MacroIndent{\rmfamily\scriptsize 000\ }

 \DocInput{childdoc.dtx}

\end{document}
%</driver>
% \fi
%
% %%%%%%%%%%%%%%%%%%%%%%%%%%%%%%%%%%%%%%%%%%%%%%%%%%%%%%%%%%%%%%%%%%%%%%%%%%%%%%
% %%%%%%%%%%%%%%%%%%%%%%%%%%%%%%%%%%%%%%%%%%%%%%%%%%%%%%%%%%%%%%%%%%%%%%%%%%%%%%
% \section{Sample}
%\iffalse
%<*samplemain>
%\fi
%
% The following presents a sample document
% with two chapters, two parts, a title page,
% a compile flag as well as three forwarding files to set the flag.
% It consists of eight |.tex| files:
% \begin{center}
% \begin{tabular}{ll}
% |cdocsamp.tex|&main file\\
% |cdocsch1.tex|&include file for chapter 1\\
% |cdocsch2.tex|&include file for chapter 2\\
% |cdocspt3.tex|&include file for part 3\\
% |cdocspt4.tex|&include file for part 4\\
% |cdocsdrf.tex|&forwarding file for main file in draft mode\\
% |cdocsfi1.tex|&forwarding file for final version of chapter 1\\
% |cdocsfi2.tex|&forwarding file for final version of chapter 2\\
% \end{tabular}
% \end{center}
% Each of the eight files can be compiled directly by the \LaTeX{} compiler.
%
% %%%%%%%%%%%%%%%%%%%%%%%%%%%%%%%%%%%%%%
% \paragraph{Main File.}
%
% The main file is called |cdocsamp.tex|.
%
% Load the \textsf{childdoc} definitions and
% declare the filename for the main document:
%    \begin{macrocode}
\input{childdoc.def}
\childdocmain{}
%    \end{macrocode}

% Optional override for |\version| flag:
%    \begin{macrocode}
%%\ifchilddoc\else\providecommand{\version}{draft}\fi
%    \end{macrocode}

% Define the default values for the |\version| flag
% (|final| for the main file and |draft| for childs):
%    \begin{macrocode}
\ifchilddoc
\providecommand{\version}{draft}
\else
\providecommand{\version}{final}
\fi
%    \end{macrocode}

% Load the standard document class:
%    \begin{macrocode}
\documentclass[12pt]{article}
%    \end{macrocode}

% Start the document body:
%    \begin{macrocode}
\begin{document}
%    \end{macrocode}

% Declare a title page.
% Print title, part of document being processed and version flag:
%    \begin{macrocode}
\addtocounter{page}{-1}
\begin{center}
{\LARGE\bfseries{}childdoc example\par}
\vspace{1cm}
\ifchilddoc
\ifchilddocmanual part\else chapter\fi:
`\childdocname' of `\childdocjob'\par
\else
main document: `\childdocjob'\par
\fi
version: \version\par
\end{center}
\newpage
%    \end{macrocode}

% Manually include selected file,
% otherwise process as usual:
%    \begin{macrocode}
\ifchilddocmanual
\section*{part `\childdocname'}
\input{\childdocname}
\else
%    \end{macrocode}

% Include the two chapters:
%    \begin{macrocode}
\include{cdocsch1}
\include{cdocsch2}
%    \end{macrocode}

% Include the two parts unless only chapters should be displayed:
%    \begin{macrocode}
\ifchilddoc\else
\section{part three}
\input{cdocspt3}
\section{part four}
\input{cdocspt4}
\fi
%    \end{macrocode}

% Process as usual until here:
%    \begin{macrocode}
\fi
%    \end{macrocode}

% End of document body:
%    \begin{macrocode}
\end{document}
%    \end{macrocode}
%\iffalse
%</samplemain>
%\fi
%
% %%%%%%%%%%%%%%%%%%%%%%%%%%%%%%%%%%%%%%
% \paragraph{Chapter Include Files.}
%
% The include files are called |cdocsch1.tex| and |cdocsch2.tex|.
%
%\iffalse
%<*samplechap1|samplechap2>
%\fi

% Optional override for |\version| flag:
%    \begin{macrocode}
%%\providecommand{\version}{final}
%    \end{macrocode}

% Include the main document:
%    \begin{macrocode}
\input{childdoc.def}
\childdocof{cdocsamp}
%    \end{macrocode}

%\iffalse
%</samplechap1|samplechap2>
%\fi
%
%\iffalse
%<*samplechap1>
%\fi
% Some text for chapter 1:
%    \begin{macrocode}
\section{one}
some text in chapter one
%    \end{macrocode}

%\iffalse
%</samplechap1>
%\fi
% Some text for chapter 2:
%\iffalse
%<*samplechap2>
%\fi
%    \begin{macrocode}
\section{two}
more text in chapter two
%    \end{macrocode}

%\iffalse
%</samplechap2>
%\fi
%
% %%%%%%%%%%%%%%%%%%%%%%%%%%%%%%%%%%%%%%
% \paragraph{Part Include Files.}
%
% The include files are called |cdocspt3.tex| and |cdocspt4.tex|.
%
%\iffalse
%<*samplepart3|samplepart4>
%\fi

% Optional override for |\version| flag:
%    \begin{macrocode}
%%\providecommand{\version}{final}
%    \end{macrocode}

% Include the main document:
%    \begin{macrocode}
\input{childdoc.def}
\childdocby{cdocsamp}
%    \end{macrocode}

%\iffalse
%</samplepart3|samplepart4>
%\fi
%
%\iffalse
%<*samplepart3>
%\fi
% Some text for part 3:
%    \begin{macrocode}
some text in part three
%    \end{macrocode}

%\iffalse
%</samplepart3>
%\fi
% Some text for part 4:
%\iffalse
%<*samplepart4>
%\fi
%    \begin{macrocode}
more text in part four
%    \end{macrocode}

%\iffalse
%</samplepart4>
%\fi
%
% %%%%%%%%%%%%%%%%%%%%%%%%%%%%%%%%%%%%%%
% \paragraph{Forwarding for a Complete Draft.}
%
% The following forwarding file |cdocsdrf.tex|
% compiles the main document in draft mode:
%\iffalse
%<*sampledraft>
%\fi
%    \begin{macrocode}
\def\version{draft}
\input{childdoc.def}
\childdocforward{cdocsamp}
%    \end{macrocode}

%\iffalse
%</sampledraft>
%\fi
%
% %%%%%%%%%%%%%%%%%%%%%%%%%%%%%%%%%%%%%%
% \paragraph{Forwarding for Final Version of the Chapters.}
%
% The following forwarding files |cdocsfn1.tex| and |cdocsfn2.tex|
% (with identical content)
% compile the final versions of the child documents
% |cdocsch1.tex| and |cdocsch2.tex|, respectively:
%\iffalse
%<*samplefinal>
%\fi
%    \begin{macrocode}
\def\version{final}
\input{childdoc.def}
\childdocforwardprefix[cdocsamp]{cdocsfn}{cdocsch}
%    \end{macrocode}

%\iffalse
%</samplefinal>
%\fi
%
% %%%%%%%%%%%%%%%%%%%%%%%%%%%%%%%%%%%%%%
% \paragraph{Command Line Processing.}
%
% The following three command lines generate the output files
% |cdocscld|, |cdocscl1| and |cdocscl2|
% which should be identical to
% |cdocsdrf|, |cdocsch1| and |cdocsfn2|, respectively:
% \begin{center}
% \begin{tabular}{l}
% |latex -jobname cdocscld \|\\
% |  "\def\version{draft}\input{childdoc.def}\childdocforward{cdocsamp}"|\\
% |latex -jobname cdocscl1 \|\\
% |  "\input{childdoc.def}\childdocforward[cdocsamp]{cdocsch1}"|\\
% |latex -jobname cdocscl2 \|\\
% |  "\def\version{final}\input{childdoc.def}\childdocforward{cdocsch2}"|
% \end{tabular}
% \end{center}
% Note that the trailing backslash on each first line
% merely continues the input to the second line
% (for convenient cut ant paste).
% Furthermore, the command |latex| can be replaced by any
% of its alternative versions such as |pdflatex|.
%
% %%%%%%%%%%%%%%%%%%%%%%%%%%%%%%%%%%%%%%%%%%%%%%%%%%%%%%%%%%%%%%%%%%%%%%%%%%%%%%
% %%%%%%%%%%%%%%%%%%%%%%%%%%%%%%%%%%%%%%%%%%%%%%%%%%%%%%%%%%%%%%%%%%%%%%%%%%%%%%
% \section{Implementation}
%\iffalse
%<*package>
%\fi
%
% This section describes the definitions file |childdoc.def|.

% The definitions cannot be loaded using |\usepackage| or |\RequirePackage|
% which has a mechanism to prevent loading a style file more than once.
% When loading the definitions by means of |\input|
% multiple instances have to be prevented manually:
%\iffalse
%This code needs to be before the `\ProvidesFile' directive
%which is defined at the beginning of this file.
%Therefore it is also placed there and commented out here.
%</package>
%<*discard>
%\fi
%    \begin{macrocode}
\ifdefined\childdocmain\endinput\fi
%    \end{macrocode}
%\iffalse
%</discard>
%<*package>
%\fi
%
% \macro{\ifchilddoc}
% \macro{\ifchilddocmanual}
% The conditional |\ifchilddoc| tells whether a
% child (true) or main (false) document is being compiled.
% The conditional |\ifchilddocmanual| tells whether
% the |\includeonly| mechanism is used (false) or
% the selection of child files must be performed manually (true).
% The definitions initialise to false:
%    \begin{macrocode}
\newif\ifchilddoc
\newif\ifchilddocmanual
%    \end{macrocode}

% \macro{\childdocname}
% \macro{\childdocjob}
% The macro |\childdocname| stores the name of the main document
% to be compiled. The macro |\childdocjob| stores the name of
% the document on which the \LaTeX{} compiler was originally invoked.
% The content of |\jobname| cannot be compared
% to filenames specified in the source due to different catcodes.
% The following code rescans |\jobname|, stores the result
% in |\childdocname| and saves a copy in |\childdocjob|:
%    \begin{macrocode}
\edef\childdocname{\scantokens\expandafter{\jobname\noexpand}}
\let\childdocjob\childdocname
%    \end{macrocode}

% \macro{\childdocdisable}
% The macro |\childdocdisable| prevents the main file
% from being processed more than once.
% At this stage, the main document command |\childdocmain|
% is assumed to be called once again where it should do nothing.
% Any subsequent call to it should prevent
% a secondary processing of the main document
% It overwrites the forwarding commands
% |\childdocof| and |\childdocforward|
% with empty macros to prevent further inclusions of the main document:
%    \begin{macrocode}
\newcommand{\childdocdisable}
{
  \renewcommand{\childdocmain}[1]{\renewcommand{\childdocmain}[1]{\endinput}}
  \renewcommand{\childdocof}[1]{}
  \renewcommand{\childdocby}[2][]{}
  \renewcommand{\childdocforward}[2][]{}
  \renewcommand{\childdocdisable}{}
}
%    \end{macrocode}

% \macro{\childdocmain}
% The macro |\childdocmain| is to be called at the top of the main file
% with nothing or the main filename (without extension) as argument.
% First, it breaks loops.
% If the argument is not empty and does not match |\childdocname|
% (which is set by the first inclusion of |childdoc.def|),
% |\ifchilddoc| is set to true, |\includeonly| is applied to the child file
% and |\jobname| is set to the main file
% (for proper handling of |.aux| files):
%    \begin{macrocode}
\newcommand{\childdocmain}[1]
{
  \childdocdisable\childdocmain{}
  \if?#1?\else
    \begingroup
      \def\childdoctmp{#1}
      \ifx\childdoctmp\childdocname
        \def\childdoctmp{}
      \else
        \def\childdoctmp
        {
          \childdoctrue
          \includeonly{\childdocname}
          \def\childdocjob{#1}
          \def\jobname{#1}
        }
      \fi
      \expandafter
    \endgroup
    \childdoctmp
  \fi
}
%    \end{macrocode}

% \macro{\childdocof}
% The command |\childdocof| redirects
% compilation to the main file |#1|.
%    \begin{macrocode}
\newcommand{\childdocof}[1]
{
  \childdocdisable
  \childdoctrue
  \includeonly{\childdocname}
  \def\jobname{#1}
  \def\childdocjob{#1}
  \input{#1}
}
%    \end{macrocode}

% \macro{\childdocby}
% The command |\childdocby| ....
%    \begin{macrocode}
\newcommand{\childdocby}[2][]
{
  \childdocdisable
  \childdoctrue
  \childdocmanualtrue
  \if?#1?\else
    \def\jobname{#2}
  \fi
  \def\childdocjob{#2}
  \input{#2}
  \endinput
}
%    \end{macrocode}

% \macro{\childdocforward}
% The command |\childdocforward| redirects
% compilation to the main file or
% (if the optional argument is given) a child file.
% Parameters are set as if the main file
% or a child file starting with |\childdocof| was compiled.
% Then compilation is handed over to the main file:
%    \begin{macrocode}
\newcommand{\childdocforward}[2][]
{
  \begingroup
    \if?#1?
      \def\childdoctmp
      {
        \def\childdocname{#2}
        \def\childdocjob{#2}
        \def\jobname{#2}
        \input{#2}
        \endinput
      }
    \else
      \def\childdoctmp
      {
        \childdocdisable
        \def\childdocname{#2}
        \childdoctrue
        \includeonly{#2}
        \def\childdocjob{#1}
        \def\jobname{#1}
        \input{#1}
        \endinput
      }
    \fi
    \expandafter
  \endgroup
  \childdoctmp
}
%    \end{macrocode}

% \macro{\childdocforwardprefix}
% The command |\childdocforwardprefix| redirects
% compilation to the main or a child file by means of a pattern.
% The prefix |#1| in the current filename is replaced by |#2|
% and the suffix of the current filename is kept
% (it is assumed that the filename does not contain the substring `|~~~|'
% which is used as a delimiter).
% Compilation is handed over to the new file by |\childdocforward|:
%    \begin{macrocode}
\newcommand{\childdocforwardprefix}[3][]
{
  \begingroup
    \def\childdocextract #2##1~~~{\def\childdoctmp{\childdocforward[#1]{#3##1}}}
    \expandafter\childdocextract\childdocname~~~
    \expandafter
  \endgroup
  \childdoctmp
}
%    \end{macrocode}

% \macro{\childdoc}
% The deprecated macro |\childdoc| is a legacy version of |\childdocmain|:
%    \begin{macrocode}
\newcommand{\childdoc}{\childdocmain}
%    \end{macrocode}

% \macro{\childdocredirect}
% The deprecated macro |\childdocredirect| is a legacy version
% of |\childdocforward| and |\childdocforwardprefix|:
%    \begin{macrocode}
\newcommand{\childdocredirect}[2][]
{
  \begingroup
    \if?#1?
      \def\childdoctmp{\childdocforward{#2}}
    \else
      \def\childdoctmp{\childdocforwardprefix{#1}{#2}}
    \fi
    \expandafter
  \endgroup
  \childdoctmp
}
%    \end{macrocode}

%\iffalse
%</package>
%\fi
%
\endinput
|\\
|\childdocforwardprefix{final}{child}|
\end{tabular}
\end{center}
%

Note that when several versions of a main file and/or of each child file
are to be generated, it may be convenient to set up a |Makefile| or
shell script to automatise the process.

%%%%%%%%%%%%%%%%%%%%%%%%%%%%%%%%%%%%%%%%%%%%%%%%%%%%%%%%%%%%%%%%%%%%%%%%%%%%%%%%
\subsection{Command Line Processing}
\label{sec:commandline}

The effect of redirection files can also be achieved by invoking
the \LaTeX{} compiler with a more elaborate command line.
Most conveniently this should be done as part
of a shell script or a |Makefile|.

When using \textsf{childdoc} in the main file, the following
command lines effectively perform a redirection
(note that depending on the shell being used,
backslashes may have to be doubled: `|\|' $\to$ `|\\|'):
%
\begin{center}
|... -jobname "|\textit{target}|" |\\|"|[\textit{flags}]%
|% \iffalse
%
% childdoc.dtx Copyright (C) 2017-2018 Niklas Beisert
%
% This work may be distributed and/or modified under the
% conditions of the LaTeX Project Public License, either version 1.3
% of this license or (at your option) any later version.
% The latest version of this license is in
%   http://www.latex-project.org/lppl.txt
% and version 1.3 or later is part of all distributions of LaTeX
% version 2005/12/01 or later.
%
% This work has the LPPL maintenance status `maintained'.
%
% The Current Maintainer of this work is Niklas Beisert.
%
% This work consists of the files childdoc.dtx and childdoc.ins
% and the derived files childdoc.def and cdocsamp.tex with
% cdocsch1.tex, cdocsch2.tex, cdocsdrf.tex, cdocsfn1.tex, cdocsfn2.tex.
%
%<package>\ifdefined\childdocmain\endinput\fi
%<package>\ProvidesFile{childdoc.def}[2018/12/30 v2.0 child document driver]
%<samplemain>\ProvidesFile{cdocsamp.tex}[2018/12/30 v2.0 sample for childdoc]
%<*driver>
%\ProvidesFile{childdoc.drv}[2018/12/30 v2.0 childdoc reference manual file]
\PassOptionsToClass{10pt,a4paper}{article}
\documentclass{ltxdoc}

\usepackage[margin=35mm]{geometry}
\usepackage{hyperref}
\usepackage{hyperxmp}
\usepackage[usenames]{color}

\hypersetup{colorlinks=true}
\hypersetup{pdfstartview=FitH}
\hypersetup{pdfpagemode=UseNone}
\hypersetup{pdfsource={}}
\hypersetup{pdflang={en-UK}}
\hypersetup{pdfcopyright={Copyright 2017-2018 Niklas Beisert.
  This work may be distributed and/or modified under the
  conditions of the LaTeX Project Public License, either version 1.3
  of this license or (at your option) any later version.}}
\hypersetup{pdflicenseurl={http://www.latex-project.org/lppl.txt}}
\hypersetup{pdfcontactaddress={ETH Zurich, ITP, HIT K,
  Wolfgang-Pauli-Strasse 27}}
\hypersetup{pdfcontactpostcode={8093}}
\hypersetup{pdfcontactcity={Zurich}}
\hypersetup{pdfcontactcountry={Switzerland}}
\hypersetup{pdfcontactemail={nbeisert@itp.phys.ethz.ch}}
\hypersetup{pdfcontacturl={http://people.phys.ethz.ch/\xmptilde nbeisert/}}

\newcommand{\secref}[1]{\hyperref[#1]{section \ref*{#1}}}

\parskip1ex
\parindent0pt
\let\olditemize\itemize
\def\itemize{\olditemize\parskip0pt}

\begin{document}

\title{The \textsf{childdoc} Package}
\hypersetup{pdftitle={The childdoc Package}}
\author{Niklas Beisert\\[2ex]
  Institut f\"ur Theoretische Physik\\
  Eidgen\"ossische Technische Hochschule Z\"urich\\
  Wolfgang-Pauli-Strasse 27, 8093 Z\"urich, Switzerland\\[1ex]
  \href{mailto:nbeisert@itp.phys.ethz.ch}
  {\texttt{nbeisert@itp.phys.ethz.ch}}}
\hypersetup{pdfauthor={Niklas Beisert}}
\hypersetup{pdfsubject={Manual for the LaTeX2e Package childdoc}}
\date{30 December 2018, \textsf{v2.0}}
\maketitle

\begin{abstract}\noindent
\textsf{childdoc} is a \LaTeXe{} package
that enables the direct compilation
of document sections included by |\include|
to individual files.
\end{abstract}

\begingroup
\parskip0ex
\tableofcontents
\endgroup

%%%%%%%%%%%%%%%%%%%%%%%%%%%%%%%%%%%%%%%%%%%%%%%%%%%%%%%%%%%%%%%%%%%%%%%%%%%%%%%%
%%%%%%%%%%%%%%%%%%%%%%%%%%%%%%%%%%%%%%%%%%%%%%%%%%%%%%%%%%%%%%%%%%%%%%%%%%%%%%%%
\section{Introduction}

\LaTeX{} provides a mechanism to structure a large document (such as a book)
into a main file and several child files (containing the chapters)
using the |\include| command.
This mechanism is beneficial for documents
which span hundreds of pages in order to
make the source file(s) more manageable.
Moreover, compilation can be restricted to
selected child files by means of the |\includeonly| command.
The latter feature can be used to reduce the compilation time while editing
(this was significantly more useful in the earlier days of \LaTeX{})
or to generate a smaller document which is easier to navigate.
Another application of |\includeonly| is to generate
documents consisting of selected parts of the complete document.

However, there are a few drawbacks of the plain |\include| mechanism:
\begin{itemize}
\item
The child files cannot be compiled on their own,
they can only be compiled via the main file.
A naive editing environment
(such as a text editor with an option
to have the current file processed by \LaTeX)
may require one to switch to the main file before compiling;
attempting to compile the child file produces errors.
\item
The main file must be modified (each time)
to adjust the |\includeonly| command
to the present needs. This easily leaves the main file in a messy state.
\item
The generated document will always carry the filename
of the main document. This is inconvenient if
several child files are to be compiled and
to be kept for distribution.
\end{itemize}

The present package provides a simple interface
to make child files individually compilable by \LaTeX{}.
Compiling a child file then has the same effect as compiling
the main file with an |\includeonly| command
to select the appropriate child.
Moreover the generated document will carry the name of the child
rather than the main file.
This resolves all three above issues.

This feature is meant to make the editing of books,
thesis documents and lecture notes somewhat more convenient.
However, the package can also be used efficiently for
composing a series of documents (such as exercise sheets)
which are typically distributed individually.
It then assists the author in generating the individual documents
(potentially in different versions)
as well as a document containing the collected series.
Another application is in developing style files
or other kinds of included material
where compilation of the style file could redirect
to a sample or test file.

%%%%%%%%%%%%%%%%%%%%%%%%%%%%%%%%%%%%%%%%%%%%%%%%%%%%%%%%%%%%%%%%%%%%%%%%%%%%%%%%
%%%%%%%%%%%%%%%%%%%%%%%%%%%%%%%%%%%%%%%%%%%%%%%%%%%%%%%%%%%%%%%%%%%%%%%%%%%%%%%%
\section{Usage}

First of all, the package \textsf{childdoc} is \emph{not} a standard
\LaTeXe{} |.sty| style file! Therefore it needs to be invoked in
a non-standard way.

%%%%%%%%%%%%%%%%%%%%%%%%%%%%%%%%%%%%%%%%%%%%%%%%%%%%%%%%%%%%%%%%%%%%%%%%%%%%%%%%
\subsection{Included Files}
\label{sec:include}

%%%%%%%%%%%%%%%%%%%%%%%%%%%%%%%%%%%%%%%%
\DescribeMacro{\childdocmain}
To use the package, add the commands
\begin{center}
\begin{tabular}{l}
|\input{childdoc.def}|\\
|\childdocmain{}|\\
\end{tabular}
\end{center}
at the very top of the main \LaTeX{} file,
in particular \emph{before} the |\documentclass| statement!
The argument of |\childdocmain| should be left empty
(but it must be present).

%%%%%%%%%%%%%%%%%%%%%%%%%%%%%%%%%%%%%%%%
\DescribeMacro{\childdocof}
Furthermore, add the commands
\begin{center}
\begin{tabular}{l}
|\input{childdoc.def}|\\
|\childdocof{|\textit{main}|}|\\
\end{tabular}
\end{center}
at the top of every child file \textit{child}
which is included by |\include{|\textit{child}|}|
from within the main file
(or at least for those files to be compiled individually).
The argument \textit{main} must be the filename of the main file.

There are a couple of
considerations in setting up the main and child documents:

%%%%%%%%%%%%%%%%%%%%%%%%%%%%%%%%%%%%%%%%
\paragraph{Restrictions.}

Please note the following restrictions:
\begin{itemize}
\item
|\childdocmain| must be called with one argument \textit{main}
to ensure compatibility with earlier version of the package.
It must either be empty (|\childdocmain{}|)
or precisely match the filename of the main file in which it is specified.
See \secref{sec:detection} for further information.
\item
The filename \textit{main} must be specified without the |.tex| extension.
\item
The filename \textit{main} is case sensitive
(even in case-insensitive file systems)
due to internal string comparison.
\item
The argument \textit{main} should be fully expanded, it cannot be a macro.
\item
Subdirectories and special characters should be avoided in filenames.
\item
The command |\childdocmain{|\textit{main}|}| must be followed by a whitespace.
It should not be followed immediately by another command
or by a comment mark `|%|'.
This is because the \TeX{} parser reads the token immediately following
the argument of |\childdocmain| and puts it
at the beginning of every child section;
however, a white\-space is ignored.
\end{itemize}

%%%%%%%%%%%%%%%%%%%%%%%%%%%%%%%%%%%%%%%%
\paragraph{Content of Main File.}

It is advisable to place all content in the child files included by |\include|.
Any output contained in the main file will appear in all child documents
unless suppressed manually;
it cannot be suppressed automatically by the |\includeonly| directive
and thus should normally be avoided.
A method to include some content in the main file
by means of conditional processing is described in \secref{sec:conditional}.

%%%%%%%%%%%%%%%%%%%%%%%%%%%%%%%%%%%%%%%%
\paragraph{Page Numbering.}

When only a part of the document is compiled,
the appropriate numbering of pages
(as well as other status parameters)
is determined from the |.aux| files.
The latter contain information from previous passes.
However this information needs to propagate through
all intermediate child documents.
Therefore the page numbering in child documents may well
be inconsistent until the complete document is compiled at least once.

A useful (if unconventional) way to always ensure a consistent
page numbering is to restart the numbering in each child document
and denote the pages by `\textit{child}|.|\textit{page}'
where \textit{child} represents the chapter/section number of the child file.
This can be achieved by the command
|\numberwithin{page}{|\textit{child}|}|
of the \textsf{amsmath} package
where \textit{child} can be |chapter| or |section|
depending on the chosen structuring.
Alternatively, one can modify the macro |\thepage| appropriately
and reset the counter |page| at the start of each child file.

%%%%%%%%%%%%%%%%%%%%%%%%%%%%%%%%%%%%%%%%%%%%%%%%%%%%%%%%%%%%%%%%%%%%%%%%%%%%%%%%
\subsection{Conditional Processing}
\label{sec:conditional}

The package provides a mechanism to compile different versions
of a document. To customise the versions further some conditional processing
can come in handy to distinguish which version is being compiled.
The package provides two macros to describe the compilation context:

%%%%%%%%%%%%%%%%%%%%%%%%%%%%%%%%%%%%%%%%
\DescribeMacro{\ifchilddoc}
The conditional |\ifchilddoc| distinguishes between the compilation of
child documents and the main document:
%
\begin{center}
|\ifchilddoc |\textit{child-code}| |[|\||else |\textit{main-code}]| \||fi|
\end{center}

%%%%%%%%%%%%%%%%%%%%%%%%%%%%%%%%%%%%%%%%
\DescribeMacro{\childdocname}
\DescribeMacro{\childdocjob}
The macro |\childdocname| contains the filename (without extension)
of the main or child file being processed.
Note that |\childdocjob| will always contain the name of the main file.

%%%%%%%%%%%%%%%%%%%%%%%%%%%%%%%%%%%%%%%%
\paragraph{Title Page.}

Conditional processing can be used to include a title or banner page
in the main document when proper precautions are taken.
Importantly, the code in the main file should ensure that the page counter
(as well as other status parameters which are stored in the |.aux| files)
takes the same value after the conditional processing.
Otherwise the page numbers may take divergent values
depending on which part is compiled.

For example, a title page could be declared by:
%
\begin{center}
\begin{tabular}{l}
|\ifchilddoc\||else|\\
|\addtocounter{page}{-1}|\\
\textit{code for title page}\\
|\newpage|\\
|\||fi|
\end{tabular}
\end{center}
%
A banner page for the child documents can be generated by:
%
\begin{center}
\begin{tabular}{l}
|\ifchilddoc|\\
|\addtocounter{page}{-1}|\\
\textit{code for banner page}\\
|\newpage|\\
|\||fi|
\end{tabular}
\end{center}
%
Here one could write a message such as:
\begin{center}
|This is the part \childdocname{} of \childdocjob{}.|
\end{center}

%%%%%%%%%%%%%%%%%%%%%%%%%%%%%%%%%%%%%%%%%%%%%%%%%%%%%%%%%%%%%%%%%%%%%%%%%%%%%%%%
\subsection{Flags}
\label{sec:flags}

The package makes it easy to generate different versions
of the main or child documents.
To this end compilation flags can be defined
and assigned different default values.
They will be particularly useful in conjunction
with the forwarding mechanism described in \secref{sec:forward}.

For example, it may be useful to have a flag |\version|
which can be set to |draft| or |final|.
The document source will contain some conditional code
depending on the value of |\version|.
Suppose further, the flag should default to |final| for the main file
and to |draft| for child files
which is a natural assignment for editing the document.
This is achieved by placing the following code
in the preamble of the main document
(below the |\childdocmain| directive):
%
\begin{center}
\begin{tabular}{l}
|\ifchilddoc|\\
|\providecommand{\version}{draft}|\\
|\||else|\\
|\providecommand{\version}{final}|\\
|\||fi|
\end{tabular}
\end{center}
%
The definition by |\providecommand| makes sure
that previous definitions are not overwritten.
Further statements |\providecommand{\version}{...}|
can thus be added before the above code to override it.

For the main file, one might add a line
(between |\childdocmain| and the above block)
%
\begin{center}
|%\ifchilddoc\||else\providecommand{\version}{draft}\||fi|
\end{center}
%
which can be uncommented to produce a draft version.
Likewise one can add a line to the very top of a child file
(above the |\childdocof{|\textit{main}|}| directive)
%
\begin{center}
|%\providecommand{\version}{final}|
\end{center}
%
which can be uncommented to produce the final version of this child document.

%%%%%%%%%%%%%%%%%%%%%%%%%%%%%%%%%%%%%%%%%%%%%%%%%%%%%%%%%%%%%%%%%%%%%%%%%%%%%%%%
\subsection{Forwarding}
\label{sec:forward}

Different versions of the main or child documents
using compilation flags as described in \secref{sec:flags}
can be (permanently) stored in different files
for convenient compilation, viewing and distribution.
To this end, the package defines a command
to pass on compilation to a different file:

%%%%%%%%%%%%%%%%%%%%%%%%%%%%%%%%%%%%%%%%
\DescribeMacro{\childdocforward}
The command |\childdocforward| redirects processing to
another source file:
%
\begin{center}
\begin{tabular}{l}
|\input{childdoc.def}|\\
|\childdocforward[|\textit{main}|]{|\textit{dest}|}|\\
\end{tabular}
\end{center}
%
The argument \textit{dest} is the destination file
(without extension).
It should be the main file or one of the child files.
Note that further \textsf{childdoc} directives
such as |\childdocof| and |\childdocforward|
in the indicated file will be processed in this form.
The optional argument \textit{main}
passes on directly to the main file \textit{main}
while pretending to compile the child \textit{dest}.
This form behaves as if \textit{dest}
issues |\childdocof{|\textit{main}|}| right away,
and no further \textsf{childdoc} directives will be processed.

%%%%%%%%%%%%%%%%%%%%%%%%%%%%%%%%%%%%%%%%
\DescribeMacro{\...prefix}
In the alternative form |\childdocforwardprefix|,
%
\begin{center}
\begin{tabular}{l}
|\input{childdoc.def}|\\
|\childdocforwardprefix[|\textit{main}|]{|\textit{prefix}|}{|\textit{dest}|}|
\end{tabular}
\end{center}
%
the destination file is determined by a pattern
depending on the current file:
To make this work, the current file must be called
`{\textit{prefix}\hspace{0.2em}\textit{suffix}}'
with \textit{prefix} matching precisely the argument.
Processing is then passed on to the file
`{\textit{dest}\hspace{0.2em}\textit{suffix}}'.
Surely, the same effect is achieved by
directly specifying the
argument `{\textit{dest}\hspace{0.2em}\textit{suffix}}'
in the first form.
However, that requires to set up a different file
for each child. With the alternative form of the command
all these files can have exactly the same content
which simplifies setting them up and maintaining them.

For example, the following file |draft.tex|
with a compilation flag |\version| as described in \secref{sec:flags}
compiles the main document as a draft:
%
\begin{center}
\begin{tabular}{l}
|\def\version{draft}|\\
|\input{childdoc.def}|\\
|\childdocforward{|\textit{main}|}|
\end{tabular}
\end{center}
%
Likewise, the following files |final|\textit{nn}|.tex|
compile the final version of the child document
|child|\textit{nn}|.tex|:
%
\begin{center}
\begin{tabular}{l}
|\def\version{final}|\\
|\input{childdoc.def}|\\
|\childdocforwardprefix{final}{child}|
\end{tabular}
\end{center}
%

Note that when several versions of a main file and/or of each child file
are to be generated, it may be convenient to set up a |Makefile| or
shell script to automatise the process.

%%%%%%%%%%%%%%%%%%%%%%%%%%%%%%%%%%%%%%%%%%%%%%%%%%%%%%%%%%%%%%%%%%%%%%%%%%%%%%%%
\subsection{Command Line Processing}
\label{sec:commandline}

The effect of redirection files can also be achieved by invoking
the \LaTeX{} compiler with a more elaborate command line.
Most conveniently this should be done as part
of a shell script or a |Makefile|.

When using \textsf{childdoc} in the main file, the following
command lines effectively perform a redirection
(note that depending on the shell being used,
backslashes may have to be doubled: `|\|' $\to$ `|\\|'):
%
\begin{center}
|... -jobname "|\textit{target}|" |\\|"|[\textit{flags}]%
|\input{childdoc.def}\childdocforward[|\textit{main}|]{|\textit{dest}|}"|
\end{center}
%
Here \textit{target} is the name of the output file,
\textit{main} is the name of the main file
and \textit{dest} is the name of the main or child file to be processed
(all filenames without extensions).
The optional argument \textit{main} can be omitted
if \textit{main} matches \textit{dest}.
Optionally, compilation \textit{flags} can be defined via |\def| commands.
This command line makes the \TeX{} engine believe
it is compiling the file \textit{target}
whose content is specified as the latter parameter.
The provided code then forwards the processing to
\textit{main} or \textit{dest} as described in \secref{sec:forward}.

%%%%%%%%%%%%%%%%%%%%%%%%%%%%%%%%%%%%%%%%%%%%%%%%%%%%%%%%%%%%%%%%%%%%%%%%%%%%%%%%
\subsection{Include by Input}
\label{sec:input}

Including child documents by |\include| has some restrictions by design.
Most notably, the content of a child document always occupies
its own set of pages; pages cannot be shared between child documents.
Usually, this behaviour makes perfect sense
because each child document contain an essential part of the document.
However, in some situations it may be desirable to compose
a document from a collection of parts
without having mandatory page breaks between then.
For this case, the package
provides a mechanism to include parts
by |\input| which can also be processed individually.
However, by construction this mechanism
requires manual handling of the content to be output.

%%%%%%%%%%%%%%%%%%%%%%%%%%%%%%%%%%%%%%%%
\DescribeMacro{\ifchilddocmanual}
The main file should be prepared as usual, see \secref{sec:include}.
However, the document body must make a distinction
between processing of an individual part and of the main document, e.g.:
%
\begin{center}
\begin{tabular}{l}
|\ifchilddocmanual|\\
|\input{\childdocname}|\\
|\||else|\\
\textit{document body with }|\input{|\textit{part}|}|\\
|\||fi|
\end{tabular}
\end{center}
%
The conditional |\ifchilddocmanual| is true whenever
a part to be included by |\input| is being compiled,
and the name of the part is stored in |\childdocname|.

%%%%%%%%%%%%%%%%%%%%%%%%%%%%%%%%%%%%%%%%
\DescribeMacro{\childdocby}
Each part to be included by |\input| should start with:
%
\begin{center}
\begin{tabular}{l}
|\input{childdoc.def}|\\
|\childdocby{|\textit{main}|}|\\
\end{tabular}
\end{center}
%
The directive |\childdocby| is similar to |\childdocof|
described in \secref{sec:include},
but the subsequent selection of content must be done manually.
To that end, both |\ifchilddoc| and |\ifchilddocmanual|
will be true upon processing of a part,
and the name of the part is stored in |\childdocname|.
Note that |\jobname| will be set to the filename of the current part
so that each part receives an individual |.aux| file
that does not interfere with the |.aux| file(s) of the main document.
This behaviour can be altered by the alternative form
|\childdocby[*]{|\textit{main}|}| (with a non-empty optional argument)
which uses the |.aux| file of the main document
by setting |\jobname| to \textit{main}.

%%%%%%%%%%%%%%%%%%%%%%%%%%%%%%%%%%%%%%%%%%%%%%%%%%%%%%%%%%%%%%%%%%%%%%%%%%%%%%%%
\subsection{Driver Development}
\label{sec:driver}

The \textsf{childdoc} mechanism can also be use for the development
of definition files such as \LaTeX{} styles or classes.
This case differs from the above setup with multiple parts
included by |\include| in that no |\includeonly| should be invoked.
This can be achieved by starting the include file
(before |\ProvidesPackage|) with:
%
\begin{center}
\begin{tabular}{l}
|\input{childdoc.def}|\\
|\childdocforward{|\textit{main}|}|\\
\end{tabular}
\end{center}
%
or alternatively with:
%
\begin{center}
\begin{tabular}{l}
|\input{childdoc.def}|\\
|\childdocby{|\textit{main}|}|\\
\end{tabular}
\end{center}
%
Both forms have slightly different effects as described above.
The main file is prepared as usual, see \secref{sec:include}.

%%%%%%%%%%%%%%%%%%%%%%%%%%%%%%%%%%%%%%%%%%%%%%%%%%%%%%%%%%%%%%%%%%%%%%%%%%%%%%%%
\subsection{Legacy Detection}
\label{sec:detection}

The directive |\childdocmain| in the main file can detect
whether the complete document or merely a child is to be compiled
even without using the directive |\childdocof|.
This method is deprecated because it is less robust
and there is no compelling reason to use it;
it is merely provided for backward compatibility
and it may be removed in future versions.

If the detection mechanism is to be used,
it is mandatory to correctly specify
the filename of the main file as the argument of |\childdocmain|:
%
\begin{center}
\begin{tabular}{l}
|\input{childdoc.def}|\\
|\childdocmain{|\textit{main}|}|\\
\end{tabular}
\end{center}
%
If |\jobname| does not match the argument \textit{main} of |\childdocmain|,
it is assumed that |\jobname| points to the child file to be compiled.
When using |\childdocmain| with the main file specified as argument,
it suffices to start a child file
with just |\input{|\textit{main}|}|
without loading of the package and using |\childdocof|.
If instead all processing is done
with the appropriate \textsf{childdoc} directives,
the argument of \textit{main} of |\childdocmain| can be empty.

An alternative version of the command line processing described
in \secref{sec:commandline} using the detection mechanism reads:
%
\begin{center}
|... -jobname "|\textit{target}|" "|[\textit{flags}]%
[|\def\jobname{|\textit{dest}|}|]|\input{|\textit{main}|}"|
\end{center}

%%%%%%%%%%%%%%%%%%%%%%%%%%%%%%%%%%%%%%%%%%%%%%%%%%%%%%%%%%%%%%%%%%%%%%%%%%%%%%%%
\subsection{Manual Code}
\label{sec:manual}

In case one cannot be certain whether the definitions file |childdoc.def|
is installed on the target \TeX{} distribution
and one prefers not to ship it,
it is conceivable to paste a few relevant commands into the sources.

To that end, drop all statements |\input{childdoc.def}|
and perform the replacements as outlined below.
Instead of |\childdocmain{|\textit{main}|}| add the following code
to the top of the main file:
%
\begin{center}
\begin{tabular}{l}
|\||ifdefined\childdocname\endinput\||fi\newif\ifchilddoc|\\
|\edef\childdocname{\scantokens\expandafter{\jobname\noexpand}}|\\
|\def\childdocmain{|\textit{main}|}\||ifx\childdocmain\childdocname\||else|\\
|\childdoctrue\includeonly{\childdocname}\let\jobname\childdocmain\||fi|\\
\end{tabular}
\end{center}
%
Instead of |\childdocof{|\textit{main}|}| just include the main file
at the top of each child file:
%
\begin{center}
|\input{|\textit{main}|}|
\end{center}
%
A simple redirection |\childdocforward{|\textit{dest}|}| is achieved by:
%
\begin{center}
|\def\jobname{|\textit{dest}|}\input{\jobname}|
\end{center}
%
The redirection with prefix
|\childdocforwardprefix[|\textit{prefix}|]{|\textit{dest}|}|
is accomplished by:
%
\begin{center}
\begin{tabular}{l}
|{\edef\jobname{\scantokens\expandafter{\jobname\noexpand}}|\\
|\def\redirectjob |\textit{prefix}|#1~~~{\gdef\jobname{|\textit{dest}|#1}}|\\
|\expandafter\redirectjob\jobname~~~}\input{\jobname}|
\end{tabular}
\end{center}

In an alternative approach,
child documents can be compiled by a specific command line
without additional code or specific definitions:
%
\begin{center}
|... -jobname "|\textit{target}|" "|[\textit{flags}]%
|\includeonly{|\textit{dest}|}\input{|\textit{main}|}"|
\end{center}
%

%%%%%%%%%%%%%%%%%%%%%%%%%%%%%%%%%%%%%%%%%%%%%%%%%%%%%%%%%%%%%%%%%%%%%%%%%%%%%%%%
%%%%%%%%%%%%%%%%%%%%%%%%%%%%%%%%%%%%%%%%%%%%%%%%%%%%%%%%%%%%%%%%%%%%%%%%%%%%%%%%
\section{Information}

%%%%%%%%%%%%%%%%%%%%%%%%%%%%%%%%%%%%%%%%%%%%%%%%%%%%%%%%%%%%%%%%%%%%%%%%%%%%%%%%
\subsection{Copyright}

Copyright \copyright{} 2017--2018 Niklas Beisert

This work may be distributed and/or modified under the
conditions of the \LaTeX{} Project Public License, either version 1.3
of this license or (at your option) any later version.
The latest version of this license is in
  \url{http://www.latex-project.org/lppl.txt}
and version 1.3 or later is part of all distributions of \LaTeX{}
version 2005/12/01 or later.

This work has the LPPL maintenance status `maintained'.

The Current Maintainer of this work is Niklas Beisert.

This work consists of the files |README.txt|, |childdoc.ins| and |childdoc.dtx|
as well as the derived files |childdoc.def|, |cdocsamp.tex|
with |cdocsch1.tex|, |cdocsch2.tex|, |cdocspt3.tex|, |cdocspt4.tex|,
|cdocsdrf.tex|, |cdocsfn1.tex|, |cdocsfn2.tex|
as well as |childdoc.pdf|.

%%%%%%%%%%%%%%%%%%%%%%%%%%%%%%%%%%%%%%%%%%%%%%%%%%%%%%%%%%%%%%%%%%%%%%%%%%%%%%%%
\subsection{Files and Installation}

The package consists of the files:
%
\begin{center}
\begin{tabular}{ll}
    |README.txt|   & readme file \\
    |childdoc.ins| & installation file \\
    |childdoc.dtx| & source file \\
    |childdoc.def| & definition file \\
    |cdocsamp.tex| & sample main file \\
    |cdocsch1.tex| & sample include file \\
    |cdocsch2.tex| & sample include file \\
    |cdocspt3.tex| & sample part file \\
    |cdocspt4.tex| & sample part file \\
    |cdocsdrf.tex| & sample redirection file \\
    |cdocsfn1.tex| & sample redirection file \\
    |cdocsfn2.tex| & sample redirection file \\
    |childdoc.pdf| & manual
\end{tabular}
\end{center}
%
The distribution consists of the files
|README.txt|, |childdoc.ins| and |childdoc.dtx|.
%
\begin{itemize}
\item
Run (pdf)\LaTeX{} on |childdoc.dtx|
to compile the manual |childdoc.pdf| (this file).
\item
Run \LaTeX{} on |childdoc.ins| to create the definitions file |childdoc.def|
and the sample |cdocsamp.tex| with include files
|cdocsch1.tex|, |cdocsch2.tex|, |cdocspt3.tex|, |cdocspt4.tex|,
|cdocsdrf.tex|, |cdocsfn1.tex|, |cdocsfn2.tex|.
Then copy the file |childdoc.def| to an appropriate directory of your \LaTeX{}
distribution, e.g.\ \textit{texmf-root}|/tex/latex/childdoc|.
\end{itemize}

%%%%%%%%%%%%%%%%%%%%%%%%%%%%%%%%%%%%%%%%%%%%%%%%%%%%%%%%%%%%%%%%%%%%%%%%%%%%%%%%
\subsection{Related CTAN Packages}

There are several other packages which offer a similar functionality:
%
\begin{itemize}
\item
The packages
\href{http://ctan.org/pkg/docmute}{\textsf{docmute}},
\href{http://ctan.org/pkg/includex}{\textsf{includex}} and
\href{http://ctan.org/pkg/standalone}{\textsf{standalone}}
provide commands to include only the document body of
a child file thus allowing both files to be compiled individually.
\item
The packages \href{http://ctan.org/pkg/subdocs}{\textsf{subdocs}}
and \href{http://ctan.org/pkg/subfiles}{\textsf{subfiles}}
provide structures in which the main and child documents can be
encapsulated and allowing them to be compiled individually.
The inclusion mechanism is different from the conventional |\include|.
\item
The package \href{http://ctan.org/pkg/combine}{\textsf{combine}}
is an elaborate solution to combine several documents into one.
\end{itemize}
%
See also the CTAN topic \href{http://ctan.org/topic/subdocs}{\textsf{subdocs}}
for further related packages.
The present package differs from the above solutions in that
a document structure constructed with the conventional |\include| mechanism
just needs two extra commands at the top of every file
such that all constituent files can be compiled individually.

%%%%%%%%%%%%%%%%%%%%%%%%%%%%%%%%%%%%%%%%%%%%%%%%%%%%%%%%%%%%%%%%%%%%%%%%%%%%%%%%
%\subsection{Feature Suggestions}
%
%The following is a list of features which may be useful for future
%versions of this package:
%%
%\begin{itemize}
%\item
%\ldots
%\end{itemize}

%%%%%%%%%%%%%%%%%%%%%%%%%%%%%%%%%%%%%%%%%%%%%%%%%%%%%%%%%%%%%%%%%%%%%%%%%%%%%%%%
\subsection{Revision History}

%%%%%%%%%%%%%%%%%%%%%%%%%%%%%%%%%%%%%%%%
\paragraph{v2.0:} 2018/12/30

\begin{itemize}
\item
immediate forward processing
\item
added |\childdocby| mechanism
\item
manual restructured
\end{itemize}

%%%%%%%%%%%%%%%%%%%%%%%%%%%%%%%%%%%%%%%%
\paragraph{v1.6:} 2018/01/17

\begin{itemize}
\item
application for development of include files
\item
corrections to manual
\end{itemize}

%%%%%%%%%%%%%%%%%%%%%%%%%%%%%%%%%%%%%%%%
\paragraph{v1.5:} 2017/05/21

\begin{itemize}
\item
more complete structuring introduced
\item
|\childdocof| introduced
\item
|\childdoc| renamed to |\childdocmain|
\item
|\childredirect| renamed to |\childdocforward| and |\childdocforwardprefix|
and functionality expanded
\end{itemize}

%%%%%%%%%%%%%%%%%%%%%%%%%%%%%%%%%%%%%%%%
\paragraph{v1.0:} 2017/04/27

\begin{itemize}
\item
manual and install package
\item
first version published on CTAN
\end{itemize}

%%%%%%%%%%%%%%%%%%%%%%%%%%%%%%%%%%%%%%%%
\paragraph{v0.6:} 2017/04/26

\begin{itemize}
\item
redirection mechanism added
\end{itemize}

%%%%%%%%%%%%%%%%%%%%%%%%%%%%%%%%%%%%%%%%
\paragraph{v0.5:} 2017/04/26

\begin{itemize}
\item
functionality in definition file
\end{itemize}


%%%%%%%%%%%%%%%%%%%%%%%%%%%%%%%%%%%%%%%%%%%%%%%%%%%%%%%%%%%%%%%%%%%%%%%%%%%%%%%%
%%%%%%%%%%%%%%%%%%%%%%%%%%%%%%%%%%%%%%%%%%%%%%%%%%%%%%%%%%%%%%%%%%%%%%%%%%%%%%%%
%%%%%%%%%%%%%%%%%%%%%%%%%%%%%%%%%%%%%%%%%%%%%%%%%%%%%%%%%%%%%%%%%%%%%%%%%%%%%%%%
\appendix

\settowidth\MacroIndent{\rmfamily\scriptsize 000\ }

 \DocInput{childdoc.dtx}

\end{document}
%</driver>
% \fi
%
% %%%%%%%%%%%%%%%%%%%%%%%%%%%%%%%%%%%%%%%%%%%%%%%%%%%%%%%%%%%%%%%%%%%%%%%%%%%%%%
% %%%%%%%%%%%%%%%%%%%%%%%%%%%%%%%%%%%%%%%%%%%%%%%%%%%%%%%%%%%%%%%%%%%%%%%%%%%%%%
% \section{Sample}
%\iffalse
%<*samplemain>
%\fi
%
% The following presents a sample document
% with two chapters, two parts, a title page,
% a compile flag as well as three forwarding files to set the flag.
% It consists of eight |.tex| files:
% \begin{center}
% \begin{tabular}{ll}
% |cdocsamp.tex|&main file\\
% |cdocsch1.tex|&include file for chapter 1\\
% |cdocsch2.tex|&include file for chapter 2\\
% |cdocspt3.tex|&include file for part 3\\
% |cdocspt4.tex|&include file for part 4\\
% |cdocsdrf.tex|&forwarding file for main file in draft mode\\
% |cdocsfi1.tex|&forwarding file for final version of chapter 1\\
% |cdocsfi2.tex|&forwarding file for final version of chapter 2\\
% \end{tabular}
% \end{center}
% Each of the eight files can be compiled directly by the \LaTeX{} compiler.
%
% %%%%%%%%%%%%%%%%%%%%%%%%%%%%%%%%%%%%%%
% \paragraph{Main File.}
%
% The main file is called |cdocsamp.tex|.
%
% Load the \textsf{childdoc} definitions and
% declare the filename for the main document:
%    \begin{macrocode}
\input{childdoc.def}
\childdocmain{}
%    \end{macrocode}

% Optional override for |\version| flag:
%    \begin{macrocode}
%%\ifchilddoc\else\providecommand{\version}{draft}\fi
%    \end{macrocode}

% Define the default values for the |\version| flag
% (|final| for the main file and |draft| for childs):
%    \begin{macrocode}
\ifchilddoc
\providecommand{\version}{draft}
\else
\providecommand{\version}{final}
\fi
%    \end{macrocode}

% Load the standard document class:
%    \begin{macrocode}
\documentclass[12pt]{article}
%    \end{macrocode}

% Start the document body:
%    \begin{macrocode}
\begin{document}
%    \end{macrocode}

% Declare a title page.
% Print title, part of document being processed and version flag:
%    \begin{macrocode}
\addtocounter{page}{-1}
\begin{center}
{\LARGE\bfseries{}childdoc example\par}
\vspace{1cm}
\ifchilddoc
\ifchilddocmanual part\else chapter\fi:
`\childdocname' of `\childdocjob'\par
\else
main document: `\childdocjob'\par
\fi
version: \version\par
\end{center}
\newpage
%    \end{macrocode}

% Manually include selected file,
% otherwise process as usual:
%    \begin{macrocode}
\ifchilddocmanual
\section*{part `\childdocname'}
\input{\childdocname}
\else
%    \end{macrocode}

% Include the two chapters:
%    \begin{macrocode}
\include{cdocsch1}
\include{cdocsch2}
%    \end{macrocode}

% Include the two parts unless only chapters should be displayed:
%    \begin{macrocode}
\ifchilddoc\else
\section{part three}
\input{cdocspt3}
\section{part four}
\input{cdocspt4}
\fi
%    \end{macrocode}

% Process as usual until here:
%    \begin{macrocode}
\fi
%    \end{macrocode}

% End of document body:
%    \begin{macrocode}
\end{document}
%    \end{macrocode}
%\iffalse
%</samplemain>
%\fi
%
% %%%%%%%%%%%%%%%%%%%%%%%%%%%%%%%%%%%%%%
% \paragraph{Chapter Include Files.}
%
% The include files are called |cdocsch1.tex| and |cdocsch2.tex|.
%
%\iffalse
%<*samplechap1|samplechap2>
%\fi

% Optional override for |\version| flag:
%    \begin{macrocode}
%%\providecommand{\version}{final}
%    \end{macrocode}

% Include the main document:
%    \begin{macrocode}
\input{childdoc.def}
\childdocof{cdocsamp}
%    \end{macrocode}

%\iffalse
%</samplechap1|samplechap2>
%\fi
%
%\iffalse
%<*samplechap1>
%\fi
% Some text for chapter 1:
%    \begin{macrocode}
\section{one}
some text in chapter one
%    \end{macrocode}

%\iffalse
%</samplechap1>
%\fi
% Some text for chapter 2:
%\iffalse
%<*samplechap2>
%\fi
%    \begin{macrocode}
\section{two}
more text in chapter two
%    \end{macrocode}

%\iffalse
%</samplechap2>
%\fi
%
% %%%%%%%%%%%%%%%%%%%%%%%%%%%%%%%%%%%%%%
% \paragraph{Part Include Files.}
%
% The include files are called |cdocspt3.tex| and |cdocspt4.tex|.
%
%\iffalse
%<*samplepart3|samplepart4>
%\fi

% Optional override for |\version| flag:
%    \begin{macrocode}
%%\providecommand{\version}{final}
%    \end{macrocode}

% Include the main document:
%    \begin{macrocode}
\input{childdoc.def}
\childdocby{cdocsamp}
%    \end{macrocode}

%\iffalse
%</samplepart3|samplepart4>
%\fi
%
%\iffalse
%<*samplepart3>
%\fi
% Some text for part 3:
%    \begin{macrocode}
some text in part three
%    \end{macrocode}

%\iffalse
%</samplepart3>
%\fi
% Some text for part 4:
%\iffalse
%<*samplepart4>
%\fi
%    \begin{macrocode}
more text in part four
%    \end{macrocode}

%\iffalse
%</samplepart4>
%\fi
%
% %%%%%%%%%%%%%%%%%%%%%%%%%%%%%%%%%%%%%%
% \paragraph{Forwarding for a Complete Draft.}
%
% The following forwarding file |cdocsdrf.tex|
% compiles the main document in draft mode:
%\iffalse
%<*sampledraft>
%\fi
%    \begin{macrocode}
\def\version{draft}
\input{childdoc.def}
\childdocforward{cdocsamp}
%    \end{macrocode}

%\iffalse
%</sampledraft>
%\fi
%
% %%%%%%%%%%%%%%%%%%%%%%%%%%%%%%%%%%%%%%
% \paragraph{Forwarding for Final Version of the Chapters.}
%
% The following forwarding files |cdocsfn1.tex| and |cdocsfn2.tex|
% (with identical content)
% compile the final versions of the child documents
% |cdocsch1.tex| and |cdocsch2.tex|, respectively:
%\iffalse
%<*samplefinal>
%\fi
%    \begin{macrocode}
\def\version{final}
\input{childdoc.def}
\childdocforwardprefix[cdocsamp]{cdocsfn}{cdocsch}
%    \end{macrocode}

%\iffalse
%</samplefinal>
%\fi
%
% %%%%%%%%%%%%%%%%%%%%%%%%%%%%%%%%%%%%%%
% \paragraph{Command Line Processing.}
%
% The following three command lines generate the output files
% |cdocscld|, |cdocscl1| and |cdocscl2|
% which should be identical to
% |cdocsdrf|, |cdocsch1| and |cdocsfn2|, respectively:
% \begin{center}
% \begin{tabular}{l}
% |latex -jobname cdocscld \|\\
% |  "\def\version{draft}\input{childdoc.def}\childdocforward{cdocsamp}"|\\
% |latex -jobname cdocscl1 \|\\
% |  "\input{childdoc.def}\childdocforward[cdocsamp]{cdocsch1}"|\\
% |latex -jobname cdocscl2 \|\\
% |  "\def\version{final}\input{childdoc.def}\childdocforward{cdocsch2}"|
% \end{tabular}
% \end{center}
% Note that the trailing backslash on each first line
% merely continues the input to the second line
% (for convenient cut ant paste).
% Furthermore, the command |latex| can be replaced by any
% of its alternative versions such as |pdflatex|.
%
% %%%%%%%%%%%%%%%%%%%%%%%%%%%%%%%%%%%%%%%%%%%%%%%%%%%%%%%%%%%%%%%%%%%%%%%%%%%%%%
% %%%%%%%%%%%%%%%%%%%%%%%%%%%%%%%%%%%%%%%%%%%%%%%%%%%%%%%%%%%%%%%%%%%%%%%%%%%%%%
% \section{Implementation}
%\iffalse
%<*package>
%\fi
%
% This section describes the definitions file |childdoc.def|.

% The definitions cannot be loaded using |\usepackage| or |\RequirePackage|
% which has a mechanism to prevent loading a style file more than once.
% When loading the definitions by means of |\input|
% multiple instances have to be prevented manually:
%\iffalse
%This code needs to be before the `\ProvidesFile' directive
%which is defined at the beginning of this file.
%Therefore it is also placed there and commented out here.
%</package>
%<*discard>
%\fi
%    \begin{macrocode}
\ifdefined\childdocmain\endinput\fi
%    \end{macrocode}
%\iffalse
%</discard>
%<*package>
%\fi
%
% \macro{\ifchilddoc}
% \macro{\ifchilddocmanual}
% The conditional |\ifchilddoc| tells whether a
% child (true) or main (false) document is being compiled.
% The conditional |\ifchilddocmanual| tells whether
% the |\includeonly| mechanism is used (false) or
% the selection of child files must be performed manually (true).
% The definitions initialise to false:
%    \begin{macrocode}
\newif\ifchilddoc
\newif\ifchilddocmanual
%    \end{macrocode}

% \macro{\childdocname}
% \macro{\childdocjob}
% The macro |\childdocname| stores the name of the main document
% to be compiled. The macro |\childdocjob| stores the name of
% the document on which the \LaTeX{} compiler was originally invoked.
% The content of |\jobname| cannot be compared
% to filenames specified in the source due to different catcodes.
% The following code rescans |\jobname|, stores the result
% in |\childdocname| and saves a copy in |\childdocjob|:
%    \begin{macrocode}
\edef\childdocname{\scantokens\expandafter{\jobname\noexpand}}
\let\childdocjob\childdocname
%    \end{macrocode}

% \macro{\childdocdisable}
% The macro |\childdocdisable| prevents the main file
% from being processed more than once.
% At this stage, the main document command |\childdocmain|
% is assumed to be called once again where it should do nothing.
% Any subsequent call to it should prevent
% a secondary processing of the main document
% It overwrites the forwarding commands
% |\childdocof| and |\childdocforward|
% with empty macros to prevent further inclusions of the main document:
%    \begin{macrocode}
\newcommand{\childdocdisable}
{
  \renewcommand{\childdocmain}[1]{\renewcommand{\childdocmain}[1]{\endinput}}
  \renewcommand{\childdocof}[1]{}
  \renewcommand{\childdocby}[2][]{}
  \renewcommand{\childdocforward}[2][]{}
  \renewcommand{\childdocdisable}{}
}
%    \end{macrocode}

% \macro{\childdocmain}
% The macro |\childdocmain| is to be called at the top of the main file
% with nothing or the main filename (without extension) as argument.
% First, it breaks loops.
% If the argument is not empty and does not match |\childdocname|
% (which is set by the first inclusion of |childdoc.def|),
% |\ifchilddoc| is set to true, |\includeonly| is applied to the child file
% and |\jobname| is set to the main file
% (for proper handling of |.aux| files):
%    \begin{macrocode}
\newcommand{\childdocmain}[1]
{
  \childdocdisable\childdocmain{}
  \if?#1?\else
    \begingroup
      \def\childdoctmp{#1}
      \ifx\childdoctmp\childdocname
        \def\childdoctmp{}
      \else
        \def\childdoctmp
        {
          \childdoctrue
          \includeonly{\childdocname}
          \def\childdocjob{#1}
          \def\jobname{#1}
        }
      \fi
      \expandafter
    \endgroup
    \childdoctmp
  \fi
}
%    \end{macrocode}

% \macro{\childdocof}
% The command |\childdocof| redirects
% compilation to the main file |#1|.
%    \begin{macrocode}
\newcommand{\childdocof}[1]
{
  \childdocdisable
  \childdoctrue
  \includeonly{\childdocname}
  \def\jobname{#1}
  \def\childdocjob{#1}
  \input{#1}
}
%    \end{macrocode}

% \macro{\childdocby}
% The command |\childdocby| ....
%    \begin{macrocode}
\newcommand{\childdocby}[2][]
{
  \childdocdisable
  \childdoctrue
  \childdocmanualtrue
  \if?#1?\else
    \def\jobname{#2}
  \fi
  \def\childdocjob{#2}
  \input{#2}
  \endinput
}
%    \end{macrocode}

% \macro{\childdocforward}
% The command |\childdocforward| redirects
% compilation to the main file or
% (if the optional argument is given) a child file.
% Parameters are set as if the main file
% or a child file starting with |\childdocof| was compiled.
% Then compilation is handed over to the main file:
%    \begin{macrocode}
\newcommand{\childdocforward}[2][]
{
  \begingroup
    \if?#1?
      \def\childdoctmp
      {
        \def\childdocname{#2}
        \def\childdocjob{#2}
        \def\jobname{#2}
        \input{#2}
        \endinput
      }
    \else
      \def\childdoctmp
      {
        \childdocdisable
        \def\childdocname{#2}
        \childdoctrue
        \includeonly{#2}
        \def\childdocjob{#1}
        \def\jobname{#1}
        \input{#1}
        \endinput
      }
    \fi
    \expandafter
  \endgroup
  \childdoctmp
}
%    \end{macrocode}

% \macro{\childdocforwardprefix}
% The command |\childdocforwardprefix| redirects
% compilation to the main or a child file by means of a pattern.
% The prefix |#1| in the current filename is replaced by |#2|
% and the suffix of the current filename is kept
% (it is assumed that the filename does not contain the substring `|~~~|'
% which is used as a delimiter).
% Compilation is handed over to the new file by |\childdocforward|:
%    \begin{macrocode}
\newcommand{\childdocforwardprefix}[3][]
{
  \begingroup
    \def\childdocextract #2##1~~~{\def\childdoctmp{\childdocforward[#1]{#3##1}}}
    \expandafter\childdocextract\childdocname~~~
    \expandafter
  \endgroup
  \childdoctmp
}
%    \end{macrocode}

% \macro{\childdoc}
% The deprecated macro |\childdoc| is a legacy version of |\childdocmain|:
%    \begin{macrocode}
\newcommand{\childdoc}{\childdocmain}
%    \end{macrocode}

% \macro{\childdocredirect}
% The deprecated macro |\childdocredirect| is a legacy version
% of |\childdocforward| and |\childdocforwardprefix|:
%    \begin{macrocode}
\newcommand{\childdocredirect}[2][]
{
  \begingroup
    \if?#1?
      \def\childdoctmp{\childdocforward{#2}}
    \else
      \def\childdoctmp{\childdocforwardprefix{#1}{#2}}
    \fi
    \expandafter
  \endgroup
  \childdoctmp
}
%    \end{macrocode}

%\iffalse
%</package>
%\fi
%
\endinput
\childdocforward[|\textit{main}|]{|\textit{dest}|}"|
\end{center}
%
Here \textit{target} is the name of the output file,
\textit{main} is the name of the main file
and \textit{dest} is the name of the main or child file to be processed
(all filenames without extensions).
The optional argument \textit{main} can be omitted
if \textit{main} matches \textit{dest}.
Optionally, compilation \textit{flags} can be defined via |\def| commands.
This command line makes the \TeX{} engine believe
it is compiling the file \textit{target}
whose content is specified as the latter parameter.
The provided code then forwards the processing to
\textit{main} or \textit{dest} as described in \secref{sec:forward}.

%%%%%%%%%%%%%%%%%%%%%%%%%%%%%%%%%%%%%%%%%%%%%%%%%%%%%%%%%%%%%%%%%%%%%%%%%%%%%%%%
\subsection{Include by Input}
\label{sec:input}

Including child documents by |\include| has some restrictions by design.
Most notably, the content of a child document always occupies
its own set of pages; pages cannot be shared between child documents.
Usually, this behaviour makes perfect sense
because each child document contain an essential part of the document.
However, in some situations it may be desirable to compose
a document from a collection of parts
without having mandatory page breaks between then.
For this case, the package
provides a mechanism to include parts
by |\input| which can also be processed individually.
However, by construction this mechanism
requires manual handling of the content to be output.

%%%%%%%%%%%%%%%%%%%%%%%%%%%%%%%%%%%%%%%%
\DescribeMacro{\ifchilddocmanual}
The main file should be prepared as usual, see \secref{sec:include}.
However, the document body must make a distinction
between processing of an individual part and of the main document, e.g.:
%
\begin{center}
\begin{tabular}{l}
|\ifchilddocmanual|\\
|\input{\childdocname}|\\
|\||else|\\
\textit{document body with }|\input{|\textit{part}|}|\\
|\||fi|
\end{tabular}
\end{center}
%
The conditional |\ifchilddocmanual| is true whenever
a part to be included by |\input| is being compiled,
and the name of the part is stored in |\childdocname|.

%%%%%%%%%%%%%%%%%%%%%%%%%%%%%%%%%%%%%%%%
\DescribeMacro{\childdocby}
Each part to be included by |\input| should start with:
%
\begin{center}
\begin{tabular}{l}
|% \iffalse
%
% childdoc.dtx Copyright (C) 2017-2018 Niklas Beisert
%
% This work may be distributed and/or modified under the
% conditions of the LaTeX Project Public License, either version 1.3
% of this license or (at your option) any later version.
% The latest version of this license is in
%   http://www.latex-project.org/lppl.txt
% and version 1.3 or later is part of all distributions of LaTeX
% version 2005/12/01 or later.
%
% This work has the LPPL maintenance status `maintained'.
%
% The Current Maintainer of this work is Niklas Beisert.
%
% This work consists of the files childdoc.dtx and childdoc.ins
% and the derived files childdoc.def and cdocsamp.tex with
% cdocsch1.tex, cdocsch2.tex, cdocsdrf.tex, cdocsfn1.tex, cdocsfn2.tex.
%
%<package>\ifdefined\childdocmain\endinput\fi
%<package>\ProvidesFile{childdoc.def}[2018/12/30 v2.0 child document driver]
%<samplemain>\ProvidesFile{cdocsamp.tex}[2018/12/30 v2.0 sample for childdoc]
%<*driver>
%\ProvidesFile{childdoc.drv}[2018/12/30 v2.0 childdoc reference manual file]
\PassOptionsToClass{10pt,a4paper}{article}
\documentclass{ltxdoc}

\usepackage[margin=35mm]{geometry}
\usepackage{hyperref}
\usepackage{hyperxmp}
\usepackage[usenames]{color}

\hypersetup{colorlinks=true}
\hypersetup{pdfstartview=FitH}
\hypersetup{pdfpagemode=UseNone}
\hypersetup{pdfsource={}}
\hypersetup{pdflang={en-UK}}
\hypersetup{pdfcopyright={Copyright 2017-2018 Niklas Beisert.
  This work may be distributed and/or modified under the
  conditions of the LaTeX Project Public License, either version 1.3
  of this license or (at your option) any later version.}}
\hypersetup{pdflicenseurl={http://www.latex-project.org/lppl.txt}}
\hypersetup{pdfcontactaddress={ETH Zurich, ITP, HIT K,
  Wolfgang-Pauli-Strasse 27}}
\hypersetup{pdfcontactpostcode={8093}}
\hypersetup{pdfcontactcity={Zurich}}
\hypersetup{pdfcontactcountry={Switzerland}}
\hypersetup{pdfcontactemail={nbeisert@itp.phys.ethz.ch}}
\hypersetup{pdfcontacturl={http://people.phys.ethz.ch/\xmptilde nbeisert/}}

\newcommand{\secref}[1]{\hyperref[#1]{section \ref*{#1}}}

\parskip1ex
\parindent0pt
\let\olditemize\itemize
\def\itemize{\olditemize\parskip0pt}

\begin{document}

\title{The \textsf{childdoc} Package}
\hypersetup{pdftitle={The childdoc Package}}
\author{Niklas Beisert\\[2ex]
  Institut f\"ur Theoretische Physik\\
  Eidgen\"ossische Technische Hochschule Z\"urich\\
  Wolfgang-Pauli-Strasse 27, 8093 Z\"urich, Switzerland\\[1ex]
  \href{mailto:nbeisert@itp.phys.ethz.ch}
  {\texttt{nbeisert@itp.phys.ethz.ch}}}
\hypersetup{pdfauthor={Niklas Beisert}}
\hypersetup{pdfsubject={Manual for the LaTeX2e Package childdoc}}
\date{30 December 2018, \textsf{v2.0}}
\maketitle

\begin{abstract}\noindent
\textsf{childdoc} is a \LaTeXe{} package
that enables the direct compilation
of document sections included by |\include|
to individual files.
\end{abstract}

\begingroup
\parskip0ex
\tableofcontents
\endgroup

%%%%%%%%%%%%%%%%%%%%%%%%%%%%%%%%%%%%%%%%%%%%%%%%%%%%%%%%%%%%%%%%%%%%%%%%%%%%%%%%
%%%%%%%%%%%%%%%%%%%%%%%%%%%%%%%%%%%%%%%%%%%%%%%%%%%%%%%%%%%%%%%%%%%%%%%%%%%%%%%%
\section{Introduction}

\LaTeX{} provides a mechanism to structure a large document (such as a book)
into a main file and several child files (containing the chapters)
using the |\include| command.
This mechanism is beneficial for documents
which span hundreds of pages in order to
make the source file(s) more manageable.
Moreover, compilation can be restricted to
selected child files by means of the |\includeonly| command.
The latter feature can be used to reduce the compilation time while editing
(this was significantly more useful in the earlier days of \LaTeX{})
or to generate a smaller document which is easier to navigate.
Another application of |\includeonly| is to generate
documents consisting of selected parts of the complete document.

However, there are a few drawbacks of the plain |\include| mechanism:
\begin{itemize}
\item
The child files cannot be compiled on their own,
they can only be compiled via the main file.
A naive editing environment
(such as a text editor with an option
to have the current file processed by \LaTeX)
may require one to switch to the main file before compiling;
attempting to compile the child file produces errors.
\item
The main file must be modified (each time)
to adjust the |\includeonly| command
to the present needs. This easily leaves the main file in a messy state.
\item
The generated document will always carry the filename
of the main document. This is inconvenient if
several child files are to be compiled and
to be kept for distribution.
\end{itemize}

The present package provides a simple interface
to make child files individually compilable by \LaTeX{}.
Compiling a child file then has the same effect as compiling
the main file with an |\includeonly| command
to select the appropriate child.
Moreover the generated document will carry the name of the child
rather than the main file.
This resolves all three above issues.

This feature is meant to make the editing of books,
thesis documents and lecture notes somewhat more convenient.
However, the package can also be used efficiently for
composing a series of documents (such as exercise sheets)
which are typically distributed individually.
It then assists the author in generating the individual documents
(potentially in different versions)
as well as a document containing the collected series.
Another application is in developing style files
or other kinds of included material
where compilation of the style file could redirect
to a sample or test file.

%%%%%%%%%%%%%%%%%%%%%%%%%%%%%%%%%%%%%%%%%%%%%%%%%%%%%%%%%%%%%%%%%%%%%%%%%%%%%%%%
%%%%%%%%%%%%%%%%%%%%%%%%%%%%%%%%%%%%%%%%%%%%%%%%%%%%%%%%%%%%%%%%%%%%%%%%%%%%%%%%
\section{Usage}

First of all, the package \textsf{childdoc} is \emph{not} a standard
\LaTeXe{} |.sty| style file! Therefore it needs to be invoked in
a non-standard way.

%%%%%%%%%%%%%%%%%%%%%%%%%%%%%%%%%%%%%%%%%%%%%%%%%%%%%%%%%%%%%%%%%%%%%%%%%%%%%%%%
\subsection{Included Files}
\label{sec:include}

%%%%%%%%%%%%%%%%%%%%%%%%%%%%%%%%%%%%%%%%
\DescribeMacro{\childdocmain}
To use the package, add the commands
\begin{center}
\begin{tabular}{l}
|\input{childdoc.def}|\\
|\childdocmain{}|\\
\end{tabular}
\end{center}
at the very top of the main \LaTeX{} file,
in particular \emph{before} the |\documentclass| statement!
The argument of |\childdocmain| should be left empty
(but it must be present).

%%%%%%%%%%%%%%%%%%%%%%%%%%%%%%%%%%%%%%%%
\DescribeMacro{\childdocof}
Furthermore, add the commands
\begin{center}
\begin{tabular}{l}
|\input{childdoc.def}|\\
|\childdocof{|\textit{main}|}|\\
\end{tabular}
\end{center}
at the top of every child file \textit{child}
which is included by |\include{|\textit{child}|}|
from within the main file
(or at least for those files to be compiled individually).
The argument \textit{main} must be the filename of the main file.

There are a couple of
considerations in setting up the main and child documents:

%%%%%%%%%%%%%%%%%%%%%%%%%%%%%%%%%%%%%%%%
\paragraph{Restrictions.}

Please note the following restrictions:
\begin{itemize}
\item
|\childdocmain| must be called with one argument \textit{main}
to ensure compatibility with earlier version of the package.
It must either be empty (|\childdocmain{}|)
or precisely match the filename of the main file in which it is specified.
See \secref{sec:detection} for further information.
\item
The filename \textit{main} must be specified without the |.tex| extension.
\item
The filename \textit{main} is case sensitive
(even in case-insensitive file systems)
due to internal string comparison.
\item
The argument \textit{main} should be fully expanded, it cannot be a macro.
\item
Subdirectories and special characters should be avoided in filenames.
\item
The command |\childdocmain{|\textit{main}|}| must be followed by a whitespace.
It should not be followed immediately by another command
or by a comment mark `|%|'.
This is because the \TeX{} parser reads the token immediately following
the argument of |\childdocmain| and puts it
at the beginning of every child section;
however, a white\-space is ignored.
\end{itemize}

%%%%%%%%%%%%%%%%%%%%%%%%%%%%%%%%%%%%%%%%
\paragraph{Content of Main File.}

It is advisable to place all content in the child files included by |\include|.
Any output contained in the main file will appear in all child documents
unless suppressed manually;
it cannot be suppressed automatically by the |\includeonly| directive
and thus should normally be avoided.
A method to include some content in the main file
by means of conditional processing is described in \secref{sec:conditional}.

%%%%%%%%%%%%%%%%%%%%%%%%%%%%%%%%%%%%%%%%
\paragraph{Page Numbering.}

When only a part of the document is compiled,
the appropriate numbering of pages
(as well as other status parameters)
is determined from the |.aux| files.
The latter contain information from previous passes.
However this information needs to propagate through
all intermediate child documents.
Therefore the page numbering in child documents may well
be inconsistent until the complete document is compiled at least once.

A useful (if unconventional) way to always ensure a consistent
page numbering is to restart the numbering in each child document
and denote the pages by `\textit{child}|.|\textit{page}'
where \textit{child} represents the chapter/section number of the child file.
This can be achieved by the command
|\numberwithin{page}{|\textit{child}|}|
of the \textsf{amsmath} package
where \textit{child} can be |chapter| or |section|
depending on the chosen structuring.
Alternatively, one can modify the macro |\thepage| appropriately
and reset the counter |page| at the start of each child file.

%%%%%%%%%%%%%%%%%%%%%%%%%%%%%%%%%%%%%%%%%%%%%%%%%%%%%%%%%%%%%%%%%%%%%%%%%%%%%%%%
\subsection{Conditional Processing}
\label{sec:conditional}

The package provides a mechanism to compile different versions
of a document. To customise the versions further some conditional processing
can come in handy to distinguish which version is being compiled.
The package provides two macros to describe the compilation context:

%%%%%%%%%%%%%%%%%%%%%%%%%%%%%%%%%%%%%%%%
\DescribeMacro{\ifchilddoc}
The conditional |\ifchilddoc| distinguishes between the compilation of
child documents and the main document:
%
\begin{center}
|\ifchilddoc |\textit{child-code}| |[|\||else |\textit{main-code}]| \||fi|
\end{center}

%%%%%%%%%%%%%%%%%%%%%%%%%%%%%%%%%%%%%%%%
\DescribeMacro{\childdocname}
\DescribeMacro{\childdocjob}
The macro |\childdocname| contains the filename (without extension)
of the main or child file being processed.
Note that |\childdocjob| will always contain the name of the main file.

%%%%%%%%%%%%%%%%%%%%%%%%%%%%%%%%%%%%%%%%
\paragraph{Title Page.}

Conditional processing can be used to include a title or banner page
in the main document when proper precautions are taken.
Importantly, the code in the main file should ensure that the page counter
(as well as other status parameters which are stored in the |.aux| files)
takes the same value after the conditional processing.
Otherwise the page numbers may take divergent values
depending on which part is compiled.

For example, a title page could be declared by:
%
\begin{center}
\begin{tabular}{l}
|\ifchilddoc\||else|\\
|\addtocounter{page}{-1}|\\
\textit{code for title page}\\
|\newpage|\\
|\||fi|
\end{tabular}
\end{center}
%
A banner page for the child documents can be generated by:
%
\begin{center}
\begin{tabular}{l}
|\ifchilddoc|\\
|\addtocounter{page}{-1}|\\
\textit{code for banner page}\\
|\newpage|\\
|\||fi|
\end{tabular}
\end{center}
%
Here one could write a message such as:
\begin{center}
|This is the part \childdocname{} of \childdocjob{}.|
\end{center}

%%%%%%%%%%%%%%%%%%%%%%%%%%%%%%%%%%%%%%%%%%%%%%%%%%%%%%%%%%%%%%%%%%%%%%%%%%%%%%%%
\subsection{Flags}
\label{sec:flags}

The package makes it easy to generate different versions
of the main or child documents.
To this end compilation flags can be defined
and assigned different default values.
They will be particularly useful in conjunction
with the forwarding mechanism described in \secref{sec:forward}.

For example, it may be useful to have a flag |\version|
which can be set to |draft| or |final|.
The document source will contain some conditional code
depending on the value of |\version|.
Suppose further, the flag should default to |final| for the main file
and to |draft| for child files
which is a natural assignment for editing the document.
This is achieved by placing the following code
in the preamble of the main document
(below the |\childdocmain| directive):
%
\begin{center}
\begin{tabular}{l}
|\ifchilddoc|\\
|\providecommand{\version}{draft}|\\
|\||else|\\
|\providecommand{\version}{final}|\\
|\||fi|
\end{tabular}
\end{center}
%
The definition by |\providecommand| makes sure
that previous definitions are not overwritten.
Further statements |\providecommand{\version}{...}|
can thus be added before the above code to override it.

For the main file, one might add a line
(between |\childdocmain| and the above block)
%
\begin{center}
|%\ifchilddoc\||else\providecommand{\version}{draft}\||fi|
\end{center}
%
which can be uncommented to produce a draft version.
Likewise one can add a line to the very top of a child file
(above the |\childdocof{|\textit{main}|}| directive)
%
\begin{center}
|%\providecommand{\version}{final}|
\end{center}
%
which can be uncommented to produce the final version of this child document.

%%%%%%%%%%%%%%%%%%%%%%%%%%%%%%%%%%%%%%%%%%%%%%%%%%%%%%%%%%%%%%%%%%%%%%%%%%%%%%%%
\subsection{Forwarding}
\label{sec:forward}

Different versions of the main or child documents
using compilation flags as described in \secref{sec:flags}
can be (permanently) stored in different files
for convenient compilation, viewing and distribution.
To this end, the package defines a command
to pass on compilation to a different file:

%%%%%%%%%%%%%%%%%%%%%%%%%%%%%%%%%%%%%%%%
\DescribeMacro{\childdocforward}
The command |\childdocforward| redirects processing to
another source file:
%
\begin{center}
\begin{tabular}{l}
|\input{childdoc.def}|\\
|\childdocforward[|\textit{main}|]{|\textit{dest}|}|\\
\end{tabular}
\end{center}
%
The argument \textit{dest} is the destination file
(without extension).
It should be the main file or one of the child files.
Note that further \textsf{childdoc} directives
such as |\childdocof| and |\childdocforward|
in the indicated file will be processed in this form.
The optional argument \textit{main}
passes on directly to the main file \textit{main}
while pretending to compile the child \textit{dest}.
This form behaves as if \textit{dest}
issues |\childdocof{|\textit{main}|}| right away,
and no further \textsf{childdoc} directives will be processed.

%%%%%%%%%%%%%%%%%%%%%%%%%%%%%%%%%%%%%%%%
\DescribeMacro{\...prefix}
In the alternative form |\childdocforwardprefix|,
%
\begin{center}
\begin{tabular}{l}
|\input{childdoc.def}|\\
|\childdocforwardprefix[|\textit{main}|]{|\textit{prefix}|}{|\textit{dest}|}|
\end{tabular}
\end{center}
%
the destination file is determined by a pattern
depending on the current file:
To make this work, the current file must be called
`{\textit{prefix}\hspace{0.2em}\textit{suffix}}'
with \textit{prefix} matching precisely the argument.
Processing is then passed on to the file
`{\textit{dest}\hspace{0.2em}\textit{suffix}}'.
Surely, the same effect is achieved by
directly specifying the
argument `{\textit{dest}\hspace{0.2em}\textit{suffix}}'
in the first form.
However, that requires to set up a different file
for each child. With the alternative form of the command
all these files can have exactly the same content
which simplifies setting them up and maintaining them.

For example, the following file |draft.tex|
with a compilation flag |\version| as described in \secref{sec:flags}
compiles the main document as a draft:
%
\begin{center}
\begin{tabular}{l}
|\def\version{draft}|\\
|\input{childdoc.def}|\\
|\childdocforward{|\textit{main}|}|
\end{tabular}
\end{center}
%
Likewise, the following files |final|\textit{nn}|.tex|
compile the final version of the child document
|child|\textit{nn}|.tex|:
%
\begin{center}
\begin{tabular}{l}
|\def\version{final}|\\
|\input{childdoc.def}|\\
|\childdocforwardprefix{final}{child}|
\end{tabular}
\end{center}
%

Note that when several versions of a main file and/or of each child file
are to be generated, it may be convenient to set up a |Makefile| or
shell script to automatise the process.

%%%%%%%%%%%%%%%%%%%%%%%%%%%%%%%%%%%%%%%%%%%%%%%%%%%%%%%%%%%%%%%%%%%%%%%%%%%%%%%%
\subsection{Command Line Processing}
\label{sec:commandline}

The effect of redirection files can also be achieved by invoking
the \LaTeX{} compiler with a more elaborate command line.
Most conveniently this should be done as part
of a shell script or a |Makefile|.

When using \textsf{childdoc} in the main file, the following
command lines effectively perform a redirection
(note that depending on the shell being used,
backslashes may have to be doubled: `|\|' $\to$ `|\\|'):
%
\begin{center}
|... -jobname "|\textit{target}|" |\\|"|[\textit{flags}]%
|\input{childdoc.def}\childdocforward[|\textit{main}|]{|\textit{dest}|}"|
\end{center}
%
Here \textit{target} is the name of the output file,
\textit{main} is the name of the main file
and \textit{dest} is the name of the main or child file to be processed
(all filenames without extensions).
The optional argument \textit{main} can be omitted
if \textit{main} matches \textit{dest}.
Optionally, compilation \textit{flags} can be defined via |\def| commands.
This command line makes the \TeX{} engine believe
it is compiling the file \textit{target}
whose content is specified as the latter parameter.
The provided code then forwards the processing to
\textit{main} or \textit{dest} as described in \secref{sec:forward}.

%%%%%%%%%%%%%%%%%%%%%%%%%%%%%%%%%%%%%%%%%%%%%%%%%%%%%%%%%%%%%%%%%%%%%%%%%%%%%%%%
\subsection{Include by Input}
\label{sec:input}

Including child documents by |\include| has some restrictions by design.
Most notably, the content of a child document always occupies
its own set of pages; pages cannot be shared between child documents.
Usually, this behaviour makes perfect sense
because each child document contain an essential part of the document.
However, in some situations it may be desirable to compose
a document from a collection of parts
without having mandatory page breaks between then.
For this case, the package
provides a mechanism to include parts
by |\input| which can also be processed individually.
However, by construction this mechanism
requires manual handling of the content to be output.

%%%%%%%%%%%%%%%%%%%%%%%%%%%%%%%%%%%%%%%%
\DescribeMacro{\ifchilddocmanual}
The main file should be prepared as usual, see \secref{sec:include}.
However, the document body must make a distinction
between processing of an individual part and of the main document, e.g.:
%
\begin{center}
\begin{tabular}{l}
|\ifchilddocmanual|\\
|\input{\childdocname}|\\
|\||else|\\
\textit{document body with }|\input{|\textit{part}|}|\\
|\||fi|
\end{tabular}
\end{center}
%
The conditional |\ifchilddocmanual| is true whenever
a part to be included by |\input| is being compiled,
and the name of the part is stored in |\childdocname|.

%%%%%%%%%%%%%%%%%%%%%%%%%%%%%%%%%%%%%%%%
\DescribeMacro{\childdocby}
Each part to be included by |\input| should start with:
%
\begin{center}
\begin{tabular}{l}
|\input{childdoc.def}|\\
|\childdocby{|\textit{main}|}|\\
\end{tabular}
\end{center}
%
The directive |\childdocby| is similar to |\childdocof|
described in \secref{sec:include},
but the subsequent selection of content must be done manually.
To that end, both |\ifchilddoc| and |\ifchilddocmanual|
will be true upon processing of a part,
and the name of the part is stored in |\childdocname|.
Note that |\jobname| will be set to the filename of the current part
so that each part receives an individual |.aux| file
that does not interfere with the |.aux| file(s) of the main document.
This behaviour can be altered by the alternative form
|\childdocby[*]{|\textit{main}|}| (with a non-empty optional argument)
which uses the |.aux| file of the main document
by setting |\jobname| to \textit{main}.

%%%%%%%%%%%%%%%%%%%%%%%%%%%%%%%%%%%%%%%%%%%%%%%%%%%%%%%%%%%%%%%%%%%%%%%%%%%%%%%%
\subsection{Driver Development}
\label{sec:driver}

The \textsf{childdoc} mechanism can also be use for the development
of definition files such as \LaTeX{} styles or classes.
This case differs from the above setup with multiple parts
included by |\include| in that no |\includeonly| should be invoked.
This can be achieved by starting the include file
(before |\ProvidesPackage|) with:
%
\begin{center}
\begin{tabular}{l}
|\input{childdoc.def}|\\
|\childdocforward{|\textit{main}|}|\\
\end{tabular}
\end{center}
%
or alternatively with:
%
\begin{center}
\begin{tabular}{l}
|\input{childdoc.def}|\\
|\childdocby{|\textit{main}|}|\\
\end{tabular}
\end{center}
%
Both forms have slightly different effects as described above.
The main file is prepared as usual, see \secref{sec:include}.

%%%%%%%%%%%%%%%%%%%%%%%%%%%%%%%%%%%%%%%%%%%%%%%%%%%%%%%%%%%%%%%%%%%%%%%%%%%%%%%%
\subsection{Legacy Detection}
\label{sec:detection}

The directive |\childdocmain| in the main file can detect
whether the complete document or merely a child is to be compiled
even without using the directive |\childdocof|.
This method is deprecated because it is less robust
and there is no compelling reason to use it;
it is merely provided for backward compatibility
and it may be removed in future versions.

If the detection mechanism is to be used,
it is mandatory to correctly specify
the filename of the main file as the argument of |\childdocmain|:
%
\begin{center}
\begin{tabular}{l}
|\input{childdoc.def}|\\
|\childdocmain{|\textit{main}|}|\\
\end{tabular}
\end{center}
%
If |\jobname| does not match the argument \textit{main} of |\childdocmain|,
it is assumed that |\jobname| points to the child file to be compiled.
When using |\childdocmain| with the main file specified as argument,
it suffices to start a child file
with just |\input{|\textit{main}|}|
without loading of the package and using |\childdocof|.
If instead all processing is done
with the appropriate \textsf{childdoc} directives,
the argument of \textit{main} of |\childdocmain| can be empty.

An alternative version of the command line processing described
in \secref{sec:commandline} using the detection mechanism reads:
%
\begin{center}
|... -jobname "|\textit{target}|" "|[\textit{flags}]%
[|\def\jobname{|\textit{dest}|}|]|\input{|\textit{main}|}"|
\end{center}

%%%%%%%%%%%%%%%%%%%%%%%%%%%%%%%%%%%%%%%%%%%%%%%%%%%%%%%%%%%%%%%%%%%%%%%%%%%%%%%%
\subsection{Manual Code}
\label{sec:manual}

In case one cannot be certain whether the definitions file |childdoc.def|
is installed on the target \TeX{} distribution
and one prefers not to ship it,
it is conceivable to paste a few relevant commands into the sources.

To that end, drop all statements |\input{childdoc.def}|
and perform the replacements as outlined below.
Instead of |\childdocmain{|\textit{main}|}| add the following code
to the top of the main file:
%
\begin{center}
\begin{tabular}{l}
|\||ifdefined\childdocname\endinput\||fi\newif\ifchilddoc|\\
|\edef\childdocname{\scantokens\expandafter{\jobname\noexpand}}|\\
|\def\childdocmain{|\textit{main}|}\||ifx\childdocmain\childdocname\||else|\\
|\childdoctrue\includeonly{\childdocname}\let\jobname\childdocmain\||fi|\\
\end{tabular}
\end{center}
%
Instead of |\childdocof{|\textit{main}|}| just include the main file
at the top of each child file:
%
\begin{center}
|\input{|\textit{main}|}|
\end{center}
%
A simple redirection |\childdocforward{|\textit{dest}|}| is achieved by:
%
\begin{center}
|\def\jobname{|\textit{dest}|}\input{\jobname}|
\end{center}
%
The redirection with prefix
|\childdocforwardprefix[|\textit{prefix}|]{|\textit{dest}|}|
is accomplished by:
%
\begin{center}
\begin{tabular}{l}
|{\edef\jobname{\scantokens\expandafter{\jobname\noexpand}}|\\
|\def\redirectjob |\textit{prefix}|#1~~~{\gdef\jobname{|\textit{dest}|#1}}|\\
|\expandafter\redirectjob\jobname~~~}\input{\jobname}|
\end{tabular}
\end{center}

In an alternative approach,
child documents can be compiled by a specific command line
without additional code or specific definitions:
%
\begin{center}
|... -jobname "|\textit{target}|" "|[\textit{flags}]%
|\includeonly{|\textit{dest}|}\input{|\textit{main}|}"|
\end{center}
%

%%%%%%%%%%%%%%%%%%%%%%%%%%%%%%%%%%%%%%%%%%%%%%%%%%%%%%%%%%%%%%%%%%%%%%%%%%%%%%%%
%%%%%%%%%%%%%%%%%%%%%%%%%%%%%%%%%%%%%%%%%%%%%%%%%%%%%%%%%%%%%%%%%%%%%%%%%%%%%%%%
\section{Information}

%%%%%%%%%%%%%%%%%%%%%%%%%%%%%%%%%%%%%%%%%%%%%%%%%%%%%%%%%%%%%%%%%%%%%%%%%%%%%%%%
\subsection{Copyright}

Copyright \copyright{} 2017--2018 Niklas Beisert

This work may be distributed and/or modified under the
conditions of the \LaTeX{} Project Public License, either version 1.3
of this license or (at your option) any later version.
The latest version of this license is in
  \url{http://www.latex-project.org/lppl.txt}
and version 1.3 or later is part of all distributions of \LaTeX{}
version 2005/12/01 or later.

This work has the LPPL maintenance status `maintained'.

The Current Maintainer of this work is Niklas Beisert.

This work consists of the files |README.txt|, |childdoc.ins| and |childdoc.dtx|
as well as the derived files |childdoc.def|, |cdocsamp.tex|
with |cdocsch1.tex|, |cdocsch2.tex|, |cdocspt3.tex|, |cdocspt4.tex|,
|cdocsdrf.tex|, |cdocsfn1.tex|, |cdocsfn2.tex|
as well as |childdoc.pdf|.

%%%%%%%%%%%%%%%%%%%%%%%%%%%%%%%%%%%%%%%%%%%%%%%%%%%%%%%%%%%%%%%%%%%%%%%%%%%%%%%%
\subsection{Files and Installation}

The package consists of the files:
%
\begin{center}
\begin{tabular}{ll}
    |README.txt|   & readme file \\
    |childdoc.ins| & installation file \\
    |childdoc.dtx| & source file \\
    |childdoc.def| & definition file \\
    |cdocsamp.tex| & sample main file \\
    |cdocsch1.tex| & sample include file \\
    |cdocsch2.tex| & sample include file \\
    |cdocspt3.tex| & sample part file \\
    |cdocspt4.tex| & sample part file \\
    |cdocsdrf.tex| & sample redirection file \\
    |cdocsfn1.tex| & sample redirection file \\
    |cdocsfn2.tex| & sample redirection file \\
    |childdoc.pdf| & manual
\end{tabular}
\end{center}
%
The distribution consists of the files
|README.txt|, |childdoc.ins| and |childdoc.dtx|.
%
\begin{itemize}
\item
Run (pdf)\LaTeX{} on |childdoc.dtx|
to compile the manual |childdoc.pdf| (this file).
\item
Run \LaTeX{} on |childdoc.ins| to create the definitions file |childdoc.def|
and the sample |cdocsamp.tex| with include files
|cdocsch1.tex|, |cdocsch2.tex|, |cdocspt3.tex|, |cdocspt4.tex|,
|cdocsdrf.tex|, |cdocsfn1.tex|, |cdocsfn2.tex|.
Then copy the file |childdoc.def| to an appropriate directory of your \LaTeX{}
distribution, e.g.\ \textit{texmf-root}|/tex/latex/childdoc|.
\end{itemize}

%%%%%%%%%%%%%%%%%%%%%%%%%%%%%%%%%%%%%%%%%%%%%%%%%%%%%%%%%%%%%%%%%%%%%%%%%%%%%%%%
\subsection{Related CTAN Packages}

There are several other packages which offer a similar functionality:
%
\begin{itemize}
\item
The packages
\href{http://ctan.org/pkg/docmute}{\textsf{docmute}},
\href{http://ctan.org/pkg/includex}{\textsf{includex}} and
\href{http://ctan.org/pkg/standalone}{\textsf{standalone}}
provide commands to include only the document body of
a child file thus allowing both files to be compiled individually.
\item
The packages \href{http://ctan.org/pkg/subdocs}{\textsf{subdocs}}
and \href{http://ctan.org/pkg/subfiles}{\textsf{subfiles}}
provide structures in which the main and child documents can be
encapsulated and allowing them to be compiled individually.
The inclusion mechanism is different from the conventional |\include|.
\item
The package \href{http://ctan.org/pkg/combine}{\textsf{combine}}
is an elaborate solution to combine several documents into one.
\end{itemize}
%
See also the CTAN topic \href{http://ctan.org/topic/subdocs}{\textsf{subdocs}}
for further related packages.
The present package differs from the above solutions in that
a document structure constructed with the conventional |\include| mechanism
just needs two extra commands at the top of every file
such that all constituent files can be compiled individually.

%%%%%%%%%%%%%%%%%%%%%%%%%%%%%%%%%%%%%%%%%%%%%%%%%%%%%%%%%%%%%%%%%%%%%%%%%%%%%%%%
%\subsection{Feature Suggestions}
%
%The following is a list of features which may be useful for future
%versions of this package:
%%
%\begin{itemize}
%\item
%\ldots
%\end{itemize}

%%%%%%%%%%%%%%%%%%%%%%%%%%%%%%%%%%%%%%%%%%%%%%%%%%%%%%%%%%%%%%%%%%%%%%%%%%%%%%%%
\subsection{Revision History}

%%%%%%%%%%%%%%%%%%%%%%%%%%%%%%%%%%%%%%%%
\paragraph{v2.0:} 2018/12/30

\begin{itemize}
\item
immediate forward processing
\item
added |\childdocby| mechanism
\item
manual restructured
\end{itemize}

%%%%%%%%%%%%%%%%%%%%%%%%%%%%%%%%%%%%%%%%
\paragraph{v1.6:} 2018/01/17

\begin{itemize}
\item
application for development of include files
\item
corrections to manual
\end{itemize}

%%%%%%%%%%%%%%%%%%%%%%%%%%%%%%%%%%%%%%%%
\paragraph{v1.5:} 2017/05/21

\begin{itemize}
\item
more complete structuring introduced
\item
|\childdocof| introduced
\item
|\childdoc| renamed to |\childdocmain|
\item
|\childredirect| renamed to |\childdocforward| and |\childdocforwardprefix|
and functionality expanded
\end{itemize}

%%%%%%%%%%%%%%%%%%%%%%%%%%%%%%%%%%%%%%%%
\paragraph{v1.0:} 2017/04/27

\begin{itemize}
\item
manual and install package
\item
first version published on CTAN
\end{itemize}

%%%%%%%%%%%%%%%%%%%%%%%%%%%%%%%%%%%%%%%%
\paragraph{v0.6:} 2017/04/26

\begin{itemize}
\item
redirection mechanism added
\end{itemize}

%%%%%%%%%%%%%%%%%%%%%%%%%%%%%%%%%%%%%%%%
\paragraph{v0.5:} 2017/04/26

\begin{itemize}
\item
functionality in definition file
\end{itemize}


%%%%%%%%%%%%%%%%%%%%%%%%%%%%%%%%%%%%%%%%%%%%%%%%%%%%%%%%%%%%%%%%%%%%%%%%%%%%%%%%
%%%%%%%%%%%%%%%%%%%%%%%%%%%%%%%%%%%%%%%%%%%%%%%%%%%%%%%%%%%%%%%%%%%%%%%%%%%%%%%%
%%%%%%%%%%%%%%%%%%%%%%%%%%%%%%%%%%%%%%%%%%%%%%%%%%%%%%%%%%%%%%%%%%%%%%%%%%%%%%%%
\appendix

\settowidth\MacroIndent{\rmfamily\scriptsize 000\ }

 \DocInput{childdoc.dtx}

\end{document}
%</driver>
% \fi
%
% %%%%%%%%%%%%%%%%%%%%%%%%%%%%%%%%%%%%%%%%%%%%%%%%%%%%%%%%%%%%%%%%%%%%%%%%%%%%%%
% %%%%%%%%%%%%%%%%%%%%%%%%%%%%%%%%%%%%%%%%%%%%%%%%%%%%%%%%%%%%%%%%%%%%%%%%%%%%%%
% \section{Sample}
%\iffalse
%<*samplemain>
%\fi
%
% The following presents a sample document
% with two chapters, two parts, a title page,
% a compile flag as well as three forwarding files to set the flag.
% It consists of eight |.tex| files:
% \begin{center}
% \begin{tabular}{ll}
% |cdocsamp.tex|&main file\\
% |cdocsch1.tex|&include file for chapter 1\\
% |cdocsch2.tex|&include file for chapter 2\\
% |cdocspt3.tex|&include file for part 3\\
% |cdocspt4.tex|&include file for part 4\\
% |cdocsdrf.tex|&forwarding file for main file in draft mode\\
% |cdocsfi1.tex|&forwarding file for final version of chapter 1\\
% |cdocsfi2.tex|&forwarding file for final version of chapter 2\\
% \end{tabular}
% \end{center}
% Each of the eight files can be compiled directly by the \LaTeX{} compiler.
%
% %%%%%%%%%%%%%%%%%%%%%%%%%%%%%%%%%%%%%%
% \paragraph{Main File.}
%
% The main file is called |cdocsamp.tex|.
%
% Load the \textsf{childdoc} definitions and
% declare the filename for the main document:
%    \begin{macrocode}
\input{childdoc.def}
\childdocmain{}
%    \end{macrocode}

% Optional override for |\version| flag:
%    \begin{macrocode}
%%\ifchilddoc\else\providecommand{\version}{draft}\fi
%    \end{macrocode}

% Define the default values for the |\version| flag
% (|final| for the main file and |draft| for childs):
%    \begin{macrocode}
\ifchilddoc
\providecommand{\version}{draft}
\else
\providecommand{\version}{final}
\fi
%    \end{macrocode}

% Load the standard document class:
%    \begin{macrocode}
\documentclass[12pt]{article}
%    \end{macrocode}

% Start the document body:
%    \begin{macrocode}
\begin{document}
%    \end{macrocode}

% Declare a title page.
% Print title, part of document being processed and version flag:
%    \begin{macrocode}
\addtocounter{page}{-1}
\begin{center}
{\LARGE\bfseries{}childdoc example\par}
\vspace{1cm}
\ifchilddoc
\ifchilddocmanual part\else chapter\fi:
`\childdocname' of `\childdocjob'\par
\else
main document: `\childdocjob'\par
\fi
version: \version\par
\end{center}
\newpage
%    \end{macrocode}

% Manually include selected file,
% otherwise process as usual:
%    \begin{macrocode}
\ifchilddocmanual
\section*{part `\childdocname'}
\input{\childdocname}
\else
%    \end{macrocode}

% Include the two chapters:
%    \begin{macrocode}
\include{cdocsch1}
\include{cdocsch2}
%    \end{macrocode}

% Include the two parts unless only chapters should be displayed:
%    \begin{macrocode}
\ifchilddoc\else
\section{part three}
\input{cdocspt3}
\section{part four}
\input{cdocspt4}
\fi
%    \end{macrocode}

% Process as usual until here:
%    \begin{macrocode}
\fi
%    \end{macrocode}

% End of document body:
%    \begin{macrocode}
\end{document}
%    \end{macrocode}
%\iffalse
%</samplemain>
%\fi
%
% %%%%%%%%%%%%%%%%%%%%%%%%%%%%%%%%%%%%%%
% \paragraph{Chapter Include Files.}
%
% The include files are called |cdocsch1.tex| and |cdocsch2.tex|.
%
%\iffalse
%<*samplechap1|samplechap2>
%\fi

% Optional override for |\version| flag:
%    \begin{macrocode}
%%\providecommand{\version}{final}
%    \end{macrocode}

% Include the main document:
%    \begin{macrocode}
\input{childdoc.def}
\childdocof{cdocsamp}
%    \end{macrocode}

%\iffalse
%</samplechap1|samplechap2>
%\fi
%
%\iffalse
%<*samplechap1>
%\fi
% Some text for chapter 1:
%    \begin{macrocode}
\section{one}
some text in chapter one
%    \end{macrocode}

%\iffalse
%</samplechap1>
%\fi
% Some text for chapter 2:
%\iffalse
%<*samplechap2>
%\fi
%    \begin{macrocode}
\section{two}
more text in chapter two
%    \end{macrocode}

%\iffalse
%</samplechap2>
%\fi
%
% %%%%%%%%%%%%%%%%%%%%%%%%%%%%%%%%%%%%%%
% \paragraph{Part Include Files.}
%
% The include files are called |cdocspt3.tex| and |cdocspt4.tex|.
%
%\iffalse
%<*samplepart3|samplepart4>
%\fi

% Optional override for |\version| flag:
%    \begin{macrocode}
%%\providecommand{\version}{final}
%    \end{macrocode}

% Include the main document:
%    \begin{macrocode}
\input{childdoc.def}
\childdocby{cdocsamp}
%    \end{macrocode}

%\iffalse
%</samplepart3|samplepart4>
%\fi
%
%\iffalse
%<*samplepart3>
%\fi
% Some text for part 3:
%    \begin{macrocode}
some text in part three
%    \end{macrocode}

%\iffalse
%</samplepart3>
%\fi
% Some text for part 4:
%\iffalse
%<*samplepart4>
%\fi
%    \begin{macrocode}
more text in part four
%    \end{macrocode}

%\iffalse
%</samplepart4>
%\fi
%
% %%%%%%%%%%%%%%%%%%%%%%%%%%%%%%%%%%%%%%
% \paragraph{Forwarding for a Complete Draft.}
%
% The following forwarding file |cdocsdrf.tex|
% compiles the main document in draft mode:
%\iffalse
%<*sampledraft>
%\fi
%    \begin{macrocode}
\def\version{draft}
\input{childdoc.def}
\childdocforward{cdocsamp}
%    \end{macrocode}

%\iffalse
%</sampledraft>
%\fi
%
% %%%%%%%%%%%%%%%%%%%%%%%%%%%%%%%%%%%%%%
% \paragraph{Forwarding for Final Version of the Chapters.}
%
% The following forwarding files |cdocsfn1.tex| and |cdocsfn2.tex|
% (with identical content)
% compile the final versions of the child documents
% |cdocsch1.tex| and |cdocsch2.tex|, respectively:
%\iffalse
%<*samplefinal>
%\fi
%    \begin{macrocode}
\def\version{final}
\input{childdoc.def}
\childdocforwardprefix[cdocsamp]{cdocsfn}{cdocsch}
%    \end{macrocode}

%\iffalse
%</samplefinal>
%\fi
%
% %%%%%%%%%%%%%%%%%%%%%%%%%%%%%%%%%%%%%%
% \paragraph{Command Line Processing.}
%
% The following three command lines generate the output files
% |cdocscld|, |cdocscl1| and |cdocscl2|
% which should be identical to
% |cdocsdrf|, |cdocsch1| and |cdocsfn2|, respectively:
% \begin{center}
% \begin{tabular}{l}
% |latex -jobname cdocscld \|\\
% |  "\def\version{draft}\input{childdoc.def}\childdocforward{cdocsamp}"|\\
% |latex -jobname cdocscl1 \|\\
% |  "\input{childdoc.def}\childdocforward[cdocsamp]{cdocsch1}"|\\
% |latex -jobname cdocscl2 \|\\
% |  "\def\version{final}\input{childdoc.def}\childdocforward{cdocsch2}"|
% \end{tabular}
% \end{center}
% Note that the trailing backslash on each first line
% merely continues the input to the second line
% (for convenient cut ant paste).
% Furthermore, the command |latex| can be replaced by any
% of its alternative versions such as |pdflatex|.
%
% %%%%%%%%%%%%%%%%%%%%%%%%%%%%%%%%%%%%%%%%%%%%%%%%%%%%%%%%%%%%%%%%%%%%%%%%%%%%%%
% %%%%%%%%%%%%%%%%%%%%%%%%%%%%%%%%%%%%%%%%%%%%%%%%%%%%%%%%%%%%%%%%%%%%%%%%%%%%%%
% \section{Implementation}
%\iffalse
%<*package>
%\fi
%
% This section describes the definitions file |childdoc.def|.

% The definitions cannot be loaded using |\usepackage| or |\RequirePackage|
% which has a mechanism to prevent loading a style file more than once.
% When loading the definitions by means of |\input|
% multiple instances have to be prevented manually:
%\iffalse
%This code needs to be before the `\ProvidesFile' directive
%which is defined at the beginning of this file.
%Therefore it is also placed there and commented out here.
%</package>
%<*discard>
%\fi
%    \begin{macrocode}
\ifdefined\childdocmain\endinput\fi
%    \end{macrocode}
%\iffalse
%</discard>
%<*package>
%\fi
%
% \macro{\ifchilddoc}
% \macro{\ifchilddocmanual}
% The conditional |\ifchilddoc| tells whether a
% child (true) or main (false) document is being compiled.
% The conditional |\ifchilddocmanual| tells whether
% the |\includeonly| mechanism is used (false) or
% the selection of child files must be performed manually (true).
% The definitions initialise to false:
%    \begin{macrocode}
\newif\ifchilddoc
\newif\ifchilddocmanual
%    \end{macrocode}

% \macro{\childdocname}
% \macro{\childdocjob}
% The macro |\childdocname| stores the name of the main document
% to be compiled. The macro |\childdocjob| stores the name of
% the document on which the \LaTeX{} compiler was originally invoked.
% The content of |\jobname| cannot be compared
% to filenames specified in the source due to different catcodes.
% The following code rescans |\jobname|, stores the result
% in |\childdocname| and saves a copy in |\childdocjob|:
%    \begin{macrocode}
\edef\childdocname{\scantokens\expandafter{\jobname\noexpand}}
\let\childdocjob\childdocname
%    \end{macrocode}

% \macro{\childdocdisable}
% The macro |\childdocdisable| prevents the main file
% from being processed more than once.
% At this stage, the main document command |\childdocmain|
% is assumed to be called once again where it should do nothing.
% Any subsequent call to it should prevent
% a secondary processing of the main document
% It overwrites the forwarding commands
% |\childdocof| and |\childdocforward|
% with empty macros to prevent further inclusions of the main document:
%    \begin{macrocode}
\newcommand{\childdocdisable}
{
  \renewcommand{\childdocmain}[1]{\renewcommand{\childdocmain}[1]{\endinput}}
  \renewcommand{\childdocof}[1]{}
  \renewcommand{\childdocby}[2][]{}
  \renewcommand{\childdocforward}[2][]{}
  \renewcommand{\childdocdisable}{}
}
%    \end{macrocode}

% \macro{\childdocmain}
% The macro |\childdocmain| is to be called at the top of the main file
% with nothing or the main filename (without extension) as argument.
% First, it breaks loops.
% If the argument is not empty and does not match |\childdocname|
% (which is set by the first inclusion of |childdoc.def|),
% |\ifchilddoc| is set to true, |\includeonly| is applied to the child file
% and |\jobname| is set to the main file
% (for proper handling of |.aux| files):
%    \begin{macrocode}
\newcommand{\childdocmain}[1]
{
  \childdocdisable\childdocmain{}
  \if?#1?\else
    \begingroup
      \def\childdoctmp{#1}
      \ifx\childdoctmp\childdocname
        \def\childdoctmp{}
      \else
        \def\childdoctmp
        {
          \childdoctrue
          \includeonly{\childdocname}
          \def\childdocjob{#1}
          \def\jobname{#1}
        }
      \fi
      \expandafter
    \endgroup
    \childdoctmp
  \fi
}
%    \end{macrocode}

% \macro{\childdocof}
% The command |\childdocof| redirects
% compilation to the main file |#1|.
%    \begin{macrocode}
\newcommand{\childdocof}[1]
{
  \childdocdisable
  \childdoctrue
  \includeonly{\childdocname}
  \def\jobname{#1}
  \def\childdocjob{#1}
  \input{#1}
}
%    \end{macrocode}

% \macro{\childdocby}
% The command |\childdocby| ....
%    \begin{macrocode}
\newcommand{\childdocby}[2][]
{
  \childdocdisable
  \childdoctrue
  \childdocmanualtrue
  \if?#1?\else
    \def\jobname{#2}
  \fi
  \def\childdocjob{#2}
  \input{#2}
  \endinput
}
%    \end{macrocode}

% \macro{\childdocforward}
% The command |\childdocforward| redirects
% compilation to the main file or
% (if the optional argument is given) a child file.
% Parameters are set as if the main file
% or a child file starting with |\childdocof| was compiled.
% Then compilation is handed over to the main file:
%    \begin{macrocode}
\newcommand{\childdocforward}[2][]
{
  \begingroup
    \if?#1?
      \def\childdoctmp
      {
        \def\childdocname{#2}
        \def\childdocjob{#2}
        \def\jobname{#2}
        \input{#2}
        \endinput
      }
    \else
      \def\childdoctmp
      {
        \childdocdisable
        \def\childdocname{#2}
        \childdoctrue
        \includeonly{#2}
        \def\childdocjob{#1}
        \def\jobname{#1}
        \input{#1}
        \endinput
      }
    \fi
    \expandafter
  \endgroup
  \childdoctmp
}
%    \end{macrocode}

% \macro{\childdocforwardprefix}
% The command |\childdocforwardprefix| redirects
% compilation to the main or a child file by means of a pattern.
% The prefix |#1| in the current filename is replaced by |#2|
% and the suffix of the current filename is kept
% (it is assumed that the filename does not contain the substring `|~~~|'
% which is used as a delimiter).
% Compilation is handed over to the new file by |\childdocforward|:
%    \begin{macrocode}
\newcommand{\childdocforwardprefix}[3][]
{
  \begingroup
    \def\childdocextract #2##1~~~{\def\childdoctmp{\childdocforward[#1]{#3##1}}}
    \expandafter\childdocextract\childdocname~~~
    \expandafter
  \endgroup
  \childdoctmp
}
%    \end{macrocode}

% \macro{\childdoc}
% The deprecated macro |\childdoc| is a legacy version of |\childdocmain|:
%    \begin{macrocode}
\newcommand{\childdoc}{\childdocmain}
%    \end{macrocode}

% \macro{\childdocredirect}
% The deprecated macro |\childdocredirect| is a legacy version
% of |\childdocforward| and |\childdocforwardprefix|:
%    \begin{macrocode}
\newcommand{\childdocredirect}[2][]
{
  \begingroup
    \if?#1?
      \def\childdoctmp{\childdocforward{#2}}
    \else
      \def\childdoctmp{\childdocforwardprefix{#1}{#2}}
    \fi
    \expandafter
  \endgroup
  \childdoctmp
}
%    \end{macrocode}

%\iffalse
%</package>
%\fi
%
\endinput
|\\
|\childdocby{|\textit{main}|}|\\
\end{tabular}
\end{center}
%
The directive |\childdocby| is similar to |\childdocof|
described in \secref{sec:include},
but the subsequent selection of content must be done manually.
To that end, both |\ifchilddoc| and |\ifchilddocmanual|
will be true upon processing of a part,
and the name of the part is stored in |\childdocname|.
Note that |\jobname| will be set to the filename of the current part
so that each part receives an individual |.aux| file
that does not interfere with the |.aux| file(s) of the main document.
This behaviour can be altered by the alternative form
|\childdocby[*]{|\textit{main}|}| (with a non-empty optional argument)
which uses the |.aux| file of the main document
by setting |\jobname| to \textit{main}.

%%%%%%%%%%%%%%%%%%%%%%%%%%%%%%%%%%%%%%%%%%%%%%%%%%%%%%%%%%%%%%%%%%%%%%%%%%%%%%%%
\subsection{Driver Development}
\label{sec:driver}

The \textsf{childdoc} mechanism can also be use for the development
of definition files such as \LaTeX{} styles or classes.
This case differs from the above setup with multiple parts
included by |\include| in that no |\includeonly| should be invoked.
This can be achieved by starting the include file
(before |\ProvidesPackage|) with:
%
\begin{center}
\begin{tabular}{l}
|% \iffalse
%
% childdoc.dtx Copyright (C) 2017-2018 Niklas Beisert
%
% This work may be distributed and/or modified under the
% conditions of the LaTeX Project Public License, either version 1.3
% of this license or (at your option) any later version.
% The latest version of this license is in
%   http://www.latex-project.org/lppl.txt
% and version 1.3 or later is part of all distributions of LaTeX
% version 2005/12/01 or later.
%
% This work has the LPPL maintenance status `maintained'.
%
% The Current Maintainer of this work is Niklas Beisert.
%
% This work consists of the files childdoc.dtx and childdoc.ins
% and the derived files childdoc.def and cdocsamp.tex with
% cdocsch1.tex, cdocsch2.tex, cdocsdrf.tex, cdocsfn1.tex, cdocsfn2.tex.
%
%<package>\ifdefined\childdocmain\endinput\fi
%<package>\ProvidesFile{childdoc.def}[2018/12/30 v2.0 child document driver]
%<samplemain>\ProvidesFile{cdocsamp.tex}[2018/12/30 v2.0 sample for childdoc]
%<*driver>
%\ProvidesFile{childdoc.drv}[2018/12/30 v2.0 childdoc reference manual file]
\PassOptionsToClass{10pt,a4paper}{article}
\documentclass{ltxdoc}

\usepackage[margin=35mm]{geometry}
\usepackage{hyperref}
\usepackage{hyperxmp}
\usepackage[usenames]{color}

\hypersetup{colorlinks=true}
\hypersetup{pdfstartview=FitH}
\hypersetup{pdfpagemode=UseNone}
\hypersetup{pdfsource={}}
\hypersetup{pdflang={en-UK}}
\hypersetup{pdfcopyright={Copyright 2017-2018 Niklas Beisert.
  This work may be distributed and/or modified under the
  conditions of the LaTeX Project Public License, either version 1.3
  of this license or (at your option) any later version.}}
\hypersetup{pdflicenseurl={http://www.latex-project.org/lppl.txt}}
\hypersetup{pdfcontactaddress={ETH Zurich, ITP, HIT K,
  Wolfgang-Pauli-Strasse 27}}
\hypersetup{pdfcontactpostcode={8093}}
\hypersetup{pdfcontactcity={Zurich}}
\hypersetup{pdfcontactcountry={Switzerland}}
\hypersetup{pdfcontactemail={nbeisert@itp.phys.ethz.ch}}
\hypersetup{pdfcontacturl={http://people.phys.ethz.ch/\xmptilde nbeisert/}}

\newcommand{\secref}[1]{\hyperref[#1]{section \ref*{#1}}}

\parskip1ex
\parindent0pt
\let\olditemize\itemize
\def\itemize{\olditemize\parskip0pt}

\begin{document}

\title{The \textsf{childdoc} Package}
\hypersetup{pdftitle={The childdoc Package}}
\author{Niklas Beisert\\[2ex]
  Institut f\"ur Theoretische Physik\\
  Eidgen\"ossische Technische Hochschule Z\"urich\\
  Wolfgang-Pauli-Strasse 27, 8093 Z\"urich, Switzerland\\[1ex]
  \href{mailto:nbeisert@itp.phys.ethz.ch}
  {\texttt{nbeisert@itp.phys.ethz.ch}}}
\hypersetup{pdfauthor={Niklas Beisert}}
\hypersetup{pdfsubject={Manual for the LaTeX2e Package childdoc}}
\date{30 December 2018, \textsf{v2.0}}
\maketitle

\begin{abstract}\noindent
\textsf{childdoc} is a \LaTeXe{} package
that enables the direct compilation
of document sections included by |\include|
to individual files.
\end{abstract}

\begingroup
\parskip0ex
\tableofcontents
\endgroup

%%%%%%%%%%%%%%%%%%%%%%%%%%%%%%%%%%%%%%%%%%%%%%%%%%%%%%%%%%%%%%%%%%%%%%%%%%%%%%%%
%%%%%%%%%%%%%%%%%%%%%%%%%%%%%%%%%%%%%%%%%%%%%%%%%%%%%%%%%%%%%%%%%%%%%%%%%%%%%%%%
\section{Introduction}

\LaTeX{} provides a mechanism to structure a large document (such as a book)
into a main file and several child files (containing the chapters)
using the |\include| command.
This mechanism is beneficial for documents
which span hundreds of pages in order to
make the source file(s) more manageable.
Moreover, compilation can be restricted to
selected child files by means of the |\includeonly| command.
The latter feature can be used to reduce the compilation time while editing
(this was significantly more useful in the earlier days of \LaTeX{})
or to generate a smaller document which is easier to navigate.
Another application of |\includeonly| is to generate
documents consisting of selected parts of the complete document.

However, there are a few drawbacks of the plain |\include| mechanism:
\begin{itemize}
\item
The child files cannot be compiled on their own,
they can only be compiled via the main file.
A naive editing environment
(such as a text editor with an option
to have the current file processed by \LaTeX)
may require one to switch to the main file before compiling;
attempting to compile the child file produces errors.
\item
The main file must be modified (each time)
to adjust the |\includeonly| command
to the present needs. This easily leaves the main file in a messy state.
\item
The generated document will always carry the filename
of the main document. This is inconvenient if
several child files are to be compiled and
to be kept for distribution.
\end{itemize}

The present package provides a simple interface
to make child files individually compilable by \LaTeX{}.
Compiling a child file then has the same effect as compiling
the main file with an |\includeonly| command
to select the appropriate child.
Moreover the generated document will carry the name of the child
rather than the main file.
This resolves all three above issues.

This feature is meant to make the editing of books,
thesis documents and lecture notes somewhat more convenient.
However, the package can also be used efficiently for
composing a series of documents (such as exercise sheets)
which are typically distributed individually.
It then assists the author in generating the individual documents
(potentially in different versions)
as well as a document containing the collected series.
Another application is in developing style files
or other kinds of included material
where compilation of the style file could redirect
to a sample or test file.

%%%%%%%%%%%%%%%%%%%%%%%%%%%%%%%%%%%%%%%%%%%%%%%%%%%%%%%%%%%%%%%%%%%%%%%%%%%%%%%%
%%%%%%%%%%%%%%%%%%%%%%%%%%%%%%%%%%%%%%%%%%%%%%%%%%%%%%%%%%%%%%%%%%%%%%%%%%%%%%%%
\section{Usage}

First of all, the package \textsf{childdoc} is \emph{not} a standard
\LaTeXe{} |.sty| style file! Therefore it needs to be invoked in
a non-standard way.

%%%%%%%%%%%%%%%%%%%%%%%%%%%%%%%%%%%%%%%%%%%%%%%%%%%%%%%%%%%%%%%%%%%%%%%%%%%%%%%%
\subsection{Included Files}
\label{sec:include}

%%%%%%%%%%%%%%%%%%%%%%%%%%%%%%%%%%%%%%%%
\DescribeMacro{\childdocmain}
To use the package, add the commands
\begin{center}
\begin{tabular}{l}
|\input{childdoc.def}|\\
|\childdocmain{}|\\
\end{tabular}
\end{center}
at the very top of the main \LaTeX{} file,
in particular \emph{before} the |\documentclass| statement!
The argument of |\childdocmain| should be left empty
(but it must be present).

%%%%%%%%%%%%%%%%%%%%%%%%%%%%%%%%%%%%%%%%
\DescribeMacro{\childdocof}
Furthermore, add the commands
\begin{center}
\begin{tabular}{l}
|\input{childdoc.def}|\\
|\childdocof{|\textit{main}|}|\\
\end{tabular}
\end{center}
at the top of every child file \textit{child}
which is included by |\include{|\textit{child}|}|
from within the main file
(or at least for those files to be compiled individually).
The argument \textit{main} must be the filename of the main file.

There are a couple of
considerations in setting up the main and child documents:

%%%%%%%%%%%%%%%%%%%%%%%%%%%%%%%%%%%%%%%%
\paragraph{Restrictions.}

Please note the following restrictions:
\begin{itemize}
\item
|\childdocmain| must be called with one argument \textit{main}
to ensure compatibility with earlier version of the package.
It must either be empty (|\childdocmain{}|)
or precisely match the filename of the main file in which it is specified.
See \secref{sec:detection} for further information.
\item
The filename \textit{main} must be specified without the |.tex| extension.
\item
The filename \textit{main} is case sensitive
(even in case-insensitive file systems)
due to internal string comparison.
\item
The argument \textit{main} should be fully expanded, it cannot be a macro.
\item
Subdirectories and special characters should be avoided in filenames.
\item
The command |\childdocmain{|\textit{main}|}| must be followed by a whitespace.
It should not be followed immediately by another command
or by a comment mark `|%|'.
This is because the \TeX{} parser reads the token immediately following
the argument of |\childdocmain| and puts it
at the beginning of every child section;
however, a white\-space is ignored.
\end{itemize}

%%%%%%%%%%%%%%%%%%%%%%%%%%%%%%%%%%%%%%%%
\paragraph{Content of Main File.}

It is advisable to place all content in the child files included by |\include|.
Any output contained in the main file will appear in all child documents
unless suppressed manually;
it cannot be suppressed automatically by the |\includeonly| directive
and thus should normally be avoided.
A method to include some content in the main file
by means of conditional processing is described in \secref{sec:conditional}.

%%%%%%%%%%%%%%%%%%%%%%%%%%%%%%%%%%%%%%%%
\paragraph{Page Numbering.}

When only a part of the document is compiled,
the appropriate numbering of pages
(as well as other status parameters)
is determined from the |.aux| files.
The latter contain information from previous passes.
However this information needs to propagate through
all intermediate child documents.
Therefore the page numbering in child documents may well
be inconsistent until the complete document is compiled at least once.

A useful (if unconventional) way to always ensure a consistent
page numbering is to restart the numbering in each child document
and denote the pages by `\textit{child}|.|\textit{page}'
where \textit{child} represents the chapter/section number of the child file.
This can be achieved by the command
|\numberwithin{page}{|\textit{child}|}|
of the \textsf{amsmath} package
where \textit{child} can be |chapter| or |section|
depending on the chosen structuring.
Alternatively, one can modify the macro |\thepage| appropriately
and reset the counter |page| at the start of each child file.

%%%%%%%%%%%%%%%%%%%%%%%%%%%%%%%%%%%%%%%%%%%%%%%%%%%%%%%%%%%%%%%%%%%%%%%%%%%%%%%%
\subsection{Conditional Processing}
\label{sec:conditional}

The package provides a mechanism to compile different versions
of a document. To customise the versions further some conditional processing
can come in handy to distinguish which version is being compiled.
The package provides two macros to describe the compilation context:

%%%%%%%%%%%%%%%%%%%%%%%%%%%%%%%%%%%%%%%%
\DescribeMacro{\ifchilddoc}
The conditional |\ifchilddoc| distinguishes between the compilation of
child documents and the main document:
%
\begin{center}
|\ifchilddoc |\textit{child-code}| |[|\||else |\textit{main-code}]| \||fi|
\end{center}

%%%%%%%%%%%%%%%%%%%%%%%%%%%%%%%%%%%%%%%%
\DescribeMacro{\childdocname}
\DescribeMacro{\childdocjob}
The macro |\childdocname| contains the filename (without extension)
of the main or child file being processed.
Note that |\childdocjob| will always contain the name of the main file.

%%%%%%%%%%%%%%%%%%%%%%%%%%%%%%%%%%%%%%%%
\paragraph{Title Page.}

Conditional processing can be used to include a title or banner page
in the main document when proper precautions are taken.
Importantly, the code in the main file should ensure that the page counter
(as well as other status parameters which are stored in the |.aux| files)
takes the same value after the conditional processing.
Otherwise the page numbers may take divergent values
depending on which part is compiled.

For example, a title page could be declared by:
%
\begin{center}
\begin{tabular}{l}
|\ifchilddoc\||else|\\
|\addtocounter{page}{-1}|\\
\textit{code for title page}\\
|\newpage|\\
|\||fi|
\end{tabular}
\end{center}
%
A banner page for the child documents can be generated by:
%
\begin{center}
\begin{tabular}{l}
|\ifchilddoc|\\
|\addtocounter{page}{-1}|\\
\textit{code for banner page}\\
|\newpage|\\
|\||fi|
\end{tabular}
\end{center}
%
Here one could write a message such as:
\begin{center}
|This is the part \childdocname{} of \childdocjob{}.|
\end{center}

%%%%%%%%%%%%%%%%%%%%%%%%%%%%%%%%%%%%%%%%%%%%%%%%%%%%%%%%%%%%%%%%%%%%%%%%%%%%%%%%
\subsection{Flags}
\label{sec:flags}

The package makes it easy to generate different versions
of the main or child documents.
To this end compilation flags can be defined
and assigned different default values.
They will be particularly useful in conjunction
with the forwarding mechanism described in \secref{sec:forward}.

For example, it may be useful to have a flag |\version|
which can be set to |draft| or |final|.
The document source will contain some conditional code
depending on the value of |\version|.
Suppose further, the flag should default to |final| for the main file
and to |draft| for child files
which is a natural assignment for editing the document.
This is achieved by placing the following code
in the preamble of the main document
(below the |\childdocmain| directive):
%
\begin{center}
\begin{tabular}{l}
|\ifchilddoc|\\
|\providecommand{\version}{draft}|\\
|\||else|\\
|\providecommand{\version}{final}|\\
|\||fi|
\end{tabular}
\end{center}
%
The definition by |\providecommand| makes sure
that previous definitions are not overwritten.
Further statements |\providecommand{\version}{...}|
can thus be added before the above code to override it.

For the main file, one might add a line
(between |\childdocmain| and the above block)
%
\begin{center}
|%\ifchilddoc\||else\providecommand{\version}{draft}\||fi|
\end{center}
%
which can be uncommented to produce a draft version.
Likewise one can add a line to the very top of a child file
(above the |\childdocof{|\textit{main}|}| directive)
%
\begin{center}
|%\providecommand{\version}{final}|
\end{center}
%
which can be uncommented to produce the final version of this child document.

%%%%%%%%%%%%%%%%%%%%%%%%%%%%%%%%%%%%%%%%%%%%%%%%%%%%%%%%%%%%%%%%%%%%%%%%%%%%%%%%
\subsection{Forwarding}
\label{sec:forward}

Different versions of the main or child documents
using compilation flags as described in \secref{sec:flags}
can be (permanently) stored in different files
for convenient compilation, viewing and distribution.
To this end, the package defines a command
to pass on compilation to a different file:

%%%%%%%%%%%%%%%%%%%%%%%%%%%%%%%%%%%%%%%%
\DescribeMacro{\childdocforward}
The command |\childdocforward| redirects processing to
another source file:
%
\begin{center}
\begin{tabular}{l}
|\input{childdoc.def}|\\
|\childdocforward[|\textit{main}|]{|\textit{dest}|}|\\
\end{tabular}
\end{center}
%
The argument \textit{dest} is the destination file
(without extension).
It should be the main file or one of the child files.
Note that further \textsf{childdoc} directives
such as |\childdocof| and |\childdocforward|
in the indicated file will be processed in this form.
The optional argument \textit{main}
passes on directly to the main file \textit{main}
while pretending to compile the child \textit{dest}.
This form behaves as if \textit{dest}
issues |\childdocof{|\textit{main}|}| right away,
and no further \textsf{childdoc} directives will be processed.

%%%%%%%%%%%%%%%%%%%%%%%%%%%%%%%%%%%%%%%%
\DescribeMacro{\...prefix}
In the alternative form |\childdocforwardprefix|,
%
\begin{center}
\begin{tabular}{l}
|\input{childdoc.def}|\\
|\childdocforwardprefix[|\textit{main}|]{|\textit{prefix}|}{|\textit{dest}|}|
\end{tabular}
\end{center}
%
the destination file is determined by a pattern
depending on the current file:
To make this work, the current file must be called
`{\textit{prefix}\hspace{0.2em}\textit{suffix}}'
with \textit{prefix} matching precisely the argument.
Processing is then passed on to the file
`{\textit{dest}\hspace{0.2em}\textit{suffix}}'.
Surely, the same effect is achieved by
directly specifying the
argument `{\textit{dest}\hspace{0.2em}\textit{suffix}}'
in the first form.
However, that requires to set up a different file
for each child. With the alternative form of the command
all these files can have exactly the same content
which simplifies setting them up and maintaining them.

For example, the following file |draft.tex|
with a compilation flag |\version| as described in \secref{sec:flags}
compiles the main document as a draft:
%
\begin{center}
\begin{tabular}{l}
|\def\version{draft}|\\
|\input{childdoc.def}|\\
|\childdocforward{|\textit{main}|}|
\end{tabular}
\end{center}
%
Likewise, the following files |final|\textit{nn}|.tex|
compile the final version of the child document
|child|\textit{nn}|.tex|:
%
\begin{center}
\begin{tabular}{l}
|\def\version{final}|\\
|\input{childdoc.def}|\\
|\childdocforwardprefix{final}{child}|
\end{tabular}
\end{center}
%

Note that when several versions of a main file and/or of each child file
are to be generated, it may be convenient to set up a |Makefile| or
shell script to automatise the process.

%%%%%%%%%%%%%%%%%%%%%%%%%%%%%%%%%%%%%%%%%%%%%%%%%%%%%%%%%%%%%%%%%%%%%%%%%%%%%%%%
\subsection{Command Line Processing}
\label{sec:commandline}

The effect of redirection files can also be achieved by invoking
the \LaTeX{} compiler with a more elaborate command line.
Most conveniently this should be done as part
of a shell script or a |Makefile|.

When using \textsf{childdoc} in the main file, the following
command lines effectively perform a redirection
(note that depending on the shell being used,
backslashes may have to be doubled: `|\|' $\to$ `|\\|'):
%
\begin{center}
|... -jobname "|\textit{target}|" |\\|"|[\textit{flags}]%
|\input{childdoc.def}\childdocforward[|\textit{main}|]{|\textit{dest}|}"|
\end{center}
%
Here \textit{target} is the name of the output file,
\textit{main} is the name of the main file
and \textit{dest} is the name of the main or child file to be processed
(all filenames without extensions).
The optional argument \textit{main} can be omitted
if \textit{main} matches \textit{dest}.
Optionally, compilation \textit{flags} can be defined via |\def| commands.
This command line makes the \TeX{} engine believe
it is compiling the file \textit{target}
whose content is specified as the latter parameter.
The provided code then forwards the processing to
\textit{main} or \textit{dest} as described in \secref{sec:forward}.

%%%%%%%%%%%%%%%%%%%%%%%%%%%%%%%%%%%%%%%%%%%%%%%%%%%%%%%%%%%%%%%%%%%%%%%%%%%%%%%%
\subsection{Include by Input}
\label{sec:input}

Including child documents by |\include| has some restrictions by design.
Most notably, the content of a child document always occupies
its own set of pages; pages cannot be shared between child documents.
Usually, this behaviour makes perfect sense
because each child document contain an essential part of the document.
However, in some situations it may be desirable to compose
a document from a collection of parts
without having mandatory page breaks between then.
For this case, the package
provides a mechanism to include parts
by |\input| which can also be processed individually.
However, by construction this mechanism
requires manual handling of the content to be output.

%%%%%%%%%%%%%%%%%%%%%%%%%%%%%%%%%%%%%%%%
\DescribeMacro{\ifchilddocmanual}
The main file should be prepared as usual, see \secref{sec:include}.
However, the document body must make a distinction
between processing of an individual part and of the main document, e.g.:
%
\begin{center}
\begin{tabular}{l}
|\ifchilddocmanual|\\
|\input{\childdocname}|\\
|\||else|\\
\textit{document body with }|\input{|\textit{part}|}|\\
|\||fi|
\end{tabular}
\end{center}
%
The conditional |\ifchilddocmanual| is true whenever
a part to be included by |\input| is being compiled,
and the name of the part is stored in |\childdocname|.

%%%%%%%%%%%%%%%%%%%%%%%%%%%%%%%%%%%%%%%%
\DescribeMacro{\childdocby}
Each part to be included by |\input| should start with:
%
\begin{center}
\begin{tabular}{l}
|\input{childdoc.def}|\\
|\childdocby{|\textit{main}|}|\\
\end{tabular}
\end{center}
%
The directive |\childdocby| is similar to |\childdocof|
described in \secref{sec:include},
but the subsequent selection of content must be done manually.
To that end, both |\ifchilddoc| and |\ifchilddocmanual|
will be true upon processing of a part,
and the name of the part is stored in |\childdocname|.
Note that |\jobname| will be set to the filename of the current part
so that each part receives an individual |.aux| file
that does not interfere with the |.aux| file(s) of the main document.
This behaviour can be altered by the alternative form
|\childdocby[*]{|\textit{main}|}| (with a non-empty optional argument)
which uses the |.aux| file of the main document
by setting |\jobname| to \textit{main}.

%%%%%%%%%%%%%%%%%%%%%%%%%%%%%%%%%%%%%%%%%%%%%%%%%%%%%%%%%%%%%%%%%%%%%%%%%%%%%%%%
\subsection{Driver Development}
\label{sec:driver}

The \textsf{childdoc} mechanism can also be use for the development
of definition files such as \LaTeX{} styles or classes.
This case differs from the above setup with multiple parts
included by |\include| in that no |\includeonly| should be invoked.
This can be achieved by starting the include file
(before |\ProvidesPackage|) with:
%
\begin{center}
\begin{tabular}{l}
|\input{childdoc.def}|\\
|\childdocforward{|\textit{main}|}|\\
\end{tabular}
\end{center}
%
or alternatively with:
%
\begin{center}
\begin{tabular}{l}
|\input{childdoc.def}|\\
|\childdocby{|\textit{main}|}|\\
\end{tabular}
\end{center}
%
Both forms have slightly different effects as described above.
The main file is prepared as usual, see \secref{sec:include}.

%%%%%%%%%%%%%%%%%%%%%%%%%%%%%%%%%%%%%%%%%%%%%%%%%%%%%%%%%%%%%%%%%%%%%%%%%%%%%%%%
\subsection{Legacy Detection}
\label{sec:detection}

The directive |\childdocmain| in the main file can detect
whether the complete document or merely a child is to be compiled
even without using the directive |\childdocof|.
This method is deprecated because it is less robust
and there is no compelling reason to use it;
it is merely provided for backward compatibility
and it may be removed in future versions.

If the detection mechanism is to be used,
it is mandatory to correctly specify
the filename of the main file as the argument of |\childdocmain|:
%
\begin{center}
\begin{tabular}{l}
|\input{childdoc.def}|\\
|\childdocmain{|\textit{main}|}|\\
\end{tabular}
\end{center}
%
If |\jobname| does not match the argument \textit{main} of |\childdocmain|,
it is assumed that |\jobname| points to the child file to be compiled.
When using |\childdocmain| with the main file specified as argument,
it suffices to start a child file
with just |\input{|\textit{main}|}|
without loading of the package and using |\childdocof|.
If instead all processing is done
with the appropriate \textsf{childdoc} directives,
the argument of \textit{main} of |\childdocmain| can be empty.

An alternative version of the command line processing described
in \secref{sec:commandline} using the detection mechanism reads:
%
\begin{center}
|... -jobname "|\textit{target}|" "|[\textit{flags}]%
[|\def\jobname{|\textit{dest}|}|]|\input{|\textit{main}|}"|
\end{center}

%%%%%%%%%%%%%%%%%%%%%%%%%%%%%%%%%%%%%%%%%%%%%%%%%%%%%%%%%%%%%%%%%%%%%%%%%%%%%%%%
\subsection{Manual Code}
\label{sec:manual}

In case one cannot be certain whether the definitions file |childdoc.def|
is installed on the target \TeX{} distribution
and one prefers not to ship it,
it is conceivable to paste a few relevant commands into the sources.

To that end, drop all statements |\input{childdoc.def}|
and perform the replacements as outlined below.
Instead of |\childdocmain{|\textit{main}|}| add the following code
to the top of the main file:
%
\begin{center}
\begin{tabular}{l}
|\||ifdefined\childdocname\endinput\||fi\newif\ifchilddoc|\\
|\edef\childdocname{\scantokens\expandafter{\jobname\noexpand}}|\\
|\def\childdocmain{|\textit{main}|}\||ifx\childdocmain\childdocname\||else|\\
|\childdoctrue\includeonly{\childdocname}\let\jobname\childdocmain\||fi|\\
\end{tabular}
\end{center}
%
Instead of |\childdocof{|\textit{main}|}| just include the main file
at the top of each child file:
%
\begin{center}
|\input{|\textit{main}|}|
\end{center}
%
A simple redirection |\childdocforward{|\textit{dest}|}| is achieved by:
%
\begin{center}
|\def\jobname{|\textit{dest}|}\input{\jobname}|
\end{center}
%
The redirection with prefix
|\childdocforwardprefix[|\textit{prefix}|]{|\textit{dest}|}|
is accomplished by:
%
\begin{center}
\begin{tabular}{l}
|{\edef\jobname{\scantokens\expandafter{\jobname\noexpand}}|\\
|\def\redirectjob |\textit{prefix}|#1~~~{\gdef\jobname{|\textit{dest}|#1}}|\\
|\expandafter\redirectjob\jobname~~~}\input{\jobname}|
\end{tabular}
\end{center}

In an alternative approach,
child documents can be compiled by a specific command line
without additional code or specific definitions:
%
\begin{center}
|... -jobname "|\textit{target}|" "|[\textit{flags}]%
|\includeonly{|\textit{dest}|}\input{|\textit{main}|}"|
\end{center}
%

%%%%%%%%%%%%%%%%%%%%%%%%%%%%%%%%%%%%%%%%%%%%%%%%%%%%%%%%%%%%%%%%%%%%%%%%%%%%%%%%
%%%%%%%%%%%%%%%%%%%%%%%%%%%%%%%%%%%%%%%%%%%%%%%%%%%%%%%%%%%%%%%%%%%%%%%%%%%%%%%%
\section{Information}

%%%%%%%%%%%%%%%%%%%%%%%%%%%%%%%%%%%%%%%%%%%%%%%%%%%%%%%%%%%%%%%%%%%%%%%%%%%%%%%%
\subsection{Copyright}

Copyright \copyright{} 2017--2018 Niklas Beisert

This work may be distributed and/or modified under the
conditions of the \LaTeX{} Project Public License, either version 1.3
of this license or (at your option) any later version.
The latest version of this license is in
  \url{http://www.latex-project.org/lppl.txt}
and version 1.3 or later is part of all distributions of \LaTeX{}
version 2005/12/01 or later.

This work has the LPPL maintenance status `maintained'.

The Current Maintainer of this work is Niklas Beisert.

This work consists of the files |README.txt|, |childdoc.ins| and |childdoc.dtx|
as well as the derived files |childdoc.def|, |cdocsamp.tex|
with |cdocsch1.tex|, |cdocsch2.tex|, |cdocspt3.tex|, |cdocspt4.tex|,
|cdocsdrf.tex|, |cdocsfn1.tex|, |cdocsfn2.tex|
as well as |childdoc.pdf|.

%%%%%%%%%%%%%%%%%%%%%%%%%%%%%%%%%%%%%%%%%%%%%%%%%%%%%%%%%%%%%%%%%%%%%%%%%%%%%%%%
\subsection{Files and Installation}

The package consists of the files:
%
\begin{center}
\begin{tabular}{ll}
    |README.txt|   & readme file \\
    |childdoc.ins| & installation file \\
    |childdoc.dtx| & source file \\
    |childdoc.def| & definition file \\
    |cdocsamp.tex| & sample main file \\
    |cdocsch1.tex| & sample include file \\
    |cdocsch2.tex| & sample include file \\
    |cdocspt3.tex| & sample part file \\
    |cdocspt4.tex| & sample part file \\
    |cdocsdrf.tex| & sample redirection file \\
    |cdocsfn1.tex| & sample redirection file \\
    |cdocsfn2.tex| & sample redirection file \\
    |childdoc.pdf| & manual
\end{tabular}
\end{center}
%
The distribution consists of the files
|README.txt|, |childdoc.ins| and |childdoc.dtx|.
%
\begin{itemize}
\item
Run (pdf)\LaTeX{} on |childdoc.dtx|
to compile the manual |childdoc.pdf| (this file).
\item
Run \LaTeX{} on |childdoc.ins| to create the definitions file |childdoc.def|
and the sample |cdocsamp.tex| with include files
|cdocsch1.tex|, |cdocsch2.tex|, |cdocspt3.tex|, |cdocspt4.tex|,
|cdocsdrf.tex|, |cdocsfn1.tex|, |cdocsfn2.tex|.
Then copy the file |childdoc.def| to an appropriate directory of your \LaTeX{}
distribution, e.g.\ \textit{texmf-root}|/tex/latex/childdoc|.
\end{itemize}

%%%%%%%%%%%%%%%%%%%%%%%%%%%%%%%%%%%%%%%%%%%%%%%%%%%%%%%%%%%%%%%%%%%%%%%%%%%%%%%%
\subsection{Related CTAN Packages}

There are several other packages which offer a similar functionality:
%
\begin{itemize}
\item
The packages
\href{http://ctan.org/pkg/docmute}{\textsf{docmute}},
\href{http://ctan.org/pkg/includex}{\textsf{includex}} and
\href{http://ctan.org/pkg/standalone}{\textsf{standalone}}
provide commands to include only the document body of
a child file thus allowing both files to be compiled individually.
\item
The packages \href{http://ctan.org/pkg/subdocs}{\textsf{subdocs}}
and \href{http://ctan.org/pkg/subfiles}{\textsf{subfiles}}
provide structures in which the main and child documents can be
encapsulated and allowing them to be compiled individually.
The inclusion mechanism is different from the conventional |\include|.
\item
The package \href{http://ctan.org/pkg/combine}{\textsf{combine}}
is an elaborate solution to combine several documents into one.
\end{itemize}
%
See also the CTAN topic \href{http://ctan.org/topic/subdocs}{\textsf{subdocs}}
for further related packages.
The present package differs from the above solutions in that
a document structure constructed with the conventional |\include| mechanism
just needs two extra commands at the top of every file
such that all constituent files can be compiled individually.

%%%%%%%%%%%%%%%%%%%%%%%%%%%%%%%%%%%%%%%%%%%%%%%%%%%%%%%%%%%%%%%%%%%%%%%%%%%%%%%%
%\subsection{Feature Suggestions}
%
%The following is a list of features which may be useful for future
%versions of this package:
%%
%\begin{itemize}
%\item
%\ldots
%\end{itemize}

%%%%%%%%%%%%%%%%%%%%%%%%%%%%%%%%%%%%%%%%%%%%%%%%%%%%%%%%%%%%%%%%%%%%%%%%%%%%%%%%
\subsection{Revision History}

%%%%%%%%%%%%%%%%%%%%%%%%%%%%%%%%%%%%%%%%
\paragraph{v2.0:} 2018/12/30

\begin{itemize}
\item
immediate forward processing
\item
added |\childdocby| mechanism
\item
manual restructured
\end{itemize}

%%%%%%%%%%%%%%%%%%%%%%%%%%%%%%%%%%%%%%%%
\paragraph{v1.6:} 2018/01/17

\begin{itemize}
\item
application for development of include files
\item
corrections to manual
\end{itemize}

%%%%%%%%%%%%%%%%%%%%%%%%%%%%%%%%%%%%%%%%
\paragraph{v1.5:} 2017/05/21

\begin{itemize}
\item
more complete structuring introduced
\item
|\childdocof| introduced
\item
|\childdoc| renamed to |\childdocmain|
\item
|\childredirect| renamed to |\childdocforward| and |\childdocforwardprefix|
and functionality expanded
\end{itemize}

%%%%%%%%%%%%%%%%%%%%%%%%%%%%%%%%%%%%%%%%
\paragraph{v1.0:} 2017/04/27

\begin{itemize}
\item
manual and install package
\item
first version published on CTAN
\end{itemize}

%%%%%%%%%%%%%%%%%%%%%%%%%%%%%%%%%%%%%%%%
\paragraph{v0.6:} 2017/04/26

\begin{itemize}
\item
redirection mechanism added
\end{itemize}

%%%%%%%%%%%%%%%%%%%%%%%%%%%%%%%%%%%%%%%%
\paragraph{v0.5:} 2017/04/26

\begin{itemize}
\item
functionality in definition file
\end{itemize}


%%%%%%%%%%%%%%%%%%%%%%%%%%%%%%%%%%%%%%%%%%%%%%%%%%%%%%%%%%%%%%%%%%%%%%%%%%%%%%%%
%%%%%%%%%%%%%%%%%%%%%%%%%%%%%%%%%%%%%%%%%%%%%%%%%%%%%%%%%%%%%%%%%%%%%%%%%%%%%%%%
%%%%%%%%%%%%%%%%%%%%%%%%%%%%%%%%%%%%%%%%%%%%%%%%%%%%%%%%%%%%%%%%%%%%%%%%%%%%%%%%
\appendix

\settowidth\MacroIndent{\rmfamily\scriptsize 000\ }

 \DocInput{childdoc.dtx}

\end{document}
%</driver>
% \fi
%
% %%%%%%%%%%%%%%%%%%%%%%%%%%%%%%%%%%%%%%%%%%%%%%%%%%%%%%%%%%%%%%%%%%%%%%%%%%%%%%
% %%%%%%%%%%%%%%%%%%%%%%%%%%%%%%%%%%%%%%%%%%%%%%%%%%%%%%%%%%%%%%%%%%%%%%%%%%%%%%
% \section{Sample}
%\iffalse
%<*samplemain>
%\fi
%
% The following presents a sample document
% with two chapters, two parts, a title page,
% a compile flag as well as three forwarding files to set the flag.
% It consists of eight |.tex| files:
% \begin{center}
% \begin{tabular}{ll}
% |cdocsamp.tex|&main file\\
% |cdocsch1.tex|&include file for chapter 1\\
% |cdocsch2.tex|&include file for chapter 2\\
% |cdocspt3.tex|&include file for part 3\\
% |cdocspt4.tex|&include file for part 4\\
% |cdocsdrf.tex|&forwarding file for main file in draft mode\\
% |cdocsfi1.tex|&forwarding file for final version of chapter 1\\
% |cdocsfi2.tex|&forwarding file for final version of chapter 2\\
% \end{tabular}
% \end{center}
% Each of the eight files can be compiled directly by the \LaTeX{} compiler.
%
% %%%%%%%%%%%%%%%%%%%%%%%%%%%%%%%%%%%%%%
% \paragraph{Main File.}
%
% The main file is called |cdocsamp.tex|.
%
% Load the \textsf{childdoc} definitions and
% declare the filename for the main document:
%    \begin{macrocode}
\input{childdoc.def}
\childdocmain{}
%    \end{macrocode}

% Optional override for |\version| flag:
%    \begin{macrocode}
%%\ifchilddoc\else\providecommand{\version}{draft}\fi
%    \end{macrocode}

% Define the default values for the |\version| flag
% (|final| for the main file and |draft| for childs):
%    \begin{macrocode}
\ifchilddoc
\providecommand{\version}{draft}
\else
\providecommand{\version}{final}
\fi
%    \end{macrocode}

% Load the standard document class:
%    \begin{macrocode}
\documentclass[12pt]{article}
%    \end{macrocode}

% Start the document body:
%    \begin{macrocode}
\begin{document}
%    \end{macrocode}

% Declare a title page.
% Print title, part of document being processed and version flag:
%    \begin{macrocode}
\addtocounter{page}{-1}
\begin{center}
{\LARGE\bfseries{}childdoc example\par}
\vspace{1cm}
\ifchilddoc
\ifchilddocmanual part\else chapter\fi:
`\childdocname' of `\childdocjob'\par
\else
main document: `\childdocjob'\par
\fi
version: \version\par
\end{center}
\newpage
%    \end{macrocode}

% Manually include selected file,
% otherwise process as usual:
%    \begin{macrocode}
\ifchilddocmanual
\section*{part `\childdocname'}
\input{\childdocname}
\else
%    \end{macrocode}

% Include the two chapters:
%    \begin{macrocode}
\include{cdocsch1}
\include{cdocsch2}
%    \end{macrocode}

% Include the two parts unless only chapters should be displayed:
%    \begin{macrocode}
\ifchilddoc\else
\section{part three}
\input{cdocspt3}
\section{part four}
\input{cdocspt4}
\fi
%    \end{macrocode}

% Process as usual until here:
%    \begin{macrocode}
\fi
%    \end{macrocode}

% End of document body:
%    \begin{macrocode}
\end{document}
%    \end{macrocode}
%\iffalse
%</samplemain>
%\fi
%
% %%%%%%%%%%%%%%%%%%%%%%%%%%%%%%%%%%%%%%
% \paragraph{Chapter Include Files.}
%
% The include files are called |cdocsch1.tex| and |cdocsch2.tex|.
%
%\iffalse
%<*samplechap1|samplechap2>
%\fi

% Optional override for |\version| flag:
%    \begin{macrocode}
%%\providecommand{\version}{final}
%    \end{macrocode}

% Include the main document:
%    \begin{macrocode}
\input{childdoc.def}
\childdocof{cdocsamp}
%    \end{macrocode}

%\iffalse
%</samplechap1|samplechap2>
%\fi
%
%\iffalse
%<*samplechap1>
%\fi
% Some text for chapter 1:
%    \begin{macrocode}
\section{one}
some text in chapter one
%    \end{macrocode}

%\iffalse
%</samplechap1>
%\fi
% Some text for chapter 2:
%\iffalse
%<*samplechap2>
%\fi
%    \begin{macrocode}
\section{two}
more text in chapter two
%    \end{macrocode}

%\iffalse
%</samplechap2>
%\fi
%
% %%%%%%%%%%%%%%%%%%%%%%%%%%%%%%%%%%%%%%
% \paragraph{Part Include Files.}
%
% The include files are called |cdocspt3.tex| and |cdocspt4.tex|.
%
%\iffalse
%<*samplepart3|samplepart4>
%\fi

% Optional override for |\version| flag:
%    \begin{macrocode}
%%\providecommand{\version}{final}
%    \end{macrocode}

% Include the main document:
%    \begin{macrocode}
\input{childdoc.def}
\childdocby{cdocsamp}
%    \end{macrocode}

%\iffalse
%</samplepart3|samplepart4>
%\fi
%
%\iffalse
%<*samplepart3>
%\fi
% Some text for part 3:
%    \begin{macrocode}
some text in part three
%    \end{macrocode}

%\iffalse
%</samplepart3>
%\fi
% Some text for part 4:
%\iffalse
%<*samplepart4>
%\fi
%    \begin{macrocode}
more text in part four
%    \end{macrocode}

%\iffalse
%</samplepart4>
%\fi
%
% %%%%%%%%%%%%%%%%%%%%%%%%%%%%%%%%%%%%%%
% \paragraph{Forwarding for a Complete Draft.}
%
% The following forwarding file |cdocsdrf.tex|
% compiles the main document in draft mode:
%\iffalse
%<*sampledraft>
%\fi
%    \begin{macrocode}
\def\version{draft}
\input{childdoc.def}
\childdocforward{cdocsamp}
%    \end{macrocode}

%\iffalse
%</sampledraft>
%\fi
%
% %%%%%%%%%%%%%%%%%%%%%%%%%%%%%%%%%%%%%%
% \paragraph{Forwarding for Final Version of the Chapters.}
%
% The following forwarding files |cdocsfn1.tex| and |cdocsfn2.tex|
% (with identical content)
% compile the final versions of the child documents
% |cdocsch1.tex| and |cdocsch2.tex|, respectively:
%\iffalse
%<*samplefinal>
%\fi
%    \begin{macrocode}
\def\version{final}
\input{childdoc.def}
\childdocforwardprefix[cdocsamp]{cdocsfn}{cdocsch}
%    \end{macrocode}

%\iffalse
%</samplefinal>
%\fi
%
% %%%%%%%%%%%%%%%%%%%%%%%%%%%%%%%%%%%%%%
% \paragraph{Command Line Processing.}
%
% The following three command lines generate the output files
% |cdocscld|, |cdocscl1| and |cdocscl2|
% which should be identical to
% |cdocsdrf|, |cdocsch1| and |cdocsfn2|, respectively:
% \begin{center}
% \begin{tabular}{l}
% |latex -jobname cdocscld \|\\
% |  "\def\version{draft}\input{childdoc.def}\childdocforward{cdocsamp}"|\\
% |latex -jobname cdocscl1 \|\\
% |  "\input{childdoc.def}\childdocforward[cdocsamp]{cdocsch1}"|\\
% |latex -jobname cdocscl2 \|\\
% |  "\def\version{final}\input{childdoc.def}\childdocforward{cdocsch2}"|
% \end{tabular}
% \end{center}
% Note that the trailing backslash on each first line
% merely continues the input to the second line
% (for convenient cut ant paste).
% Furthermore, the command |latex| can be replaced by any
% of its alternative versions such as |pdflatex|.
%
% %%%%%%%%%%%%%%%%%%%%%%%%%%%%%%%%%%%%%%%%%%%%%%%%%%%%%%%%%%%%%%%%%%%%%%%%%%%%%%
% %%%%%%%%%%%%%%%%%%%%%%%%%%%%%%%%%%%%%%%%%%%%%%%%%%%%%%%%%%%%%%%%%%%%%%%%%%%%%%
% \section{Implementation}
%\iffalse
%<*package>
%\fi
%
% This section describes the definitions file |childdoc.def|.

% The definitions cannot be loaded using |\usepackage| or |\RequirePackage|
% which has a mechanism to prevent loading a style file more than once.
% When loading the definitions by means of |\input|
% multiple instances have to be prevented manually:
%\iffalse
%This code needs to be before the `\ProvidesFile' directive
%which is defined at the beginning of this file.
%Therefore it is also placed there and commented out here.
%</package>
%<*discard>
%\fi
%    \begin{macrocode}
\ifdefined\childdocmain\endinput\fi
%    \end{macrocode}
%\iffalse
%</discard>
%<*package>
%\fi
%
% \macro{\ifchilddoc}
% \macro{\ifchilddocmanual}
% The conditional |\ifchilddoc| tells whether a
% child (true) or main (false) document is being compiled.
% The conditional |\ifchilddocmanual| tells whether
% the |\includeonly| mechanism is used (false) or
% the selection of child files must be performed manually (true).
% The definitions initialise to false:
%    \begin{macrocode}
\newif\ifchilddoc
\newif\ifchilddocmanual
%    \end{macrocode}

% \macro{\childdocname}
% \macro{\childdocjob}
% The macro |\childdocname| stores the name of the main document
% to be compiled. The macro |\childdocjob| stores the name of
% the document on which the \LaTeX{} compiler was originally invoked.
% The content of |\jobname| cannot be compared
% to filenames specified in the source due to different catcodes.
% The following code rescans |\jobname|, stores the result
% in |\childdocname| and saves a copy in |\childdocjob|:
%    \begin{macrocode}
\edef\childdocname{\scantokens\expandafter{\jobname\noexpand}}
\let\childdocjob\childdocname
%    \end{macrocode}

% \macro{\childdocdisable}
% The macro |\childdocdisable| prevents the main file
% from being processed more than once.
% At this stage, the main document command |\childdocmain|
% is assumed to be called once again where it should do nothing.
% Any subsequent call to it should prevent
% a secondary processing of the main document
% It overwrites the forwarding commands
% |\childdocof| and |\childdocforward|
% with empty macros to prevent further inclusions of the main document:
%    \begin{macrocode}
\newcommand{\childdocdisable}
{
  \renewcommand{\childdocmain}[1]{\renewcommand{\childdocmain}[1]{\endinput}}
  \renewcommand{\childdocof}[1]{}
  \renewcommand{\childdocby}[2][]{}
  \renewcommand{\childdocforward}[2][]{}
  \renewcommand{\childdocdisable}{}
}
%    \end{macrocode}

% \macro{\childdocmain}
% The macro |\childdocmain| is to be called at the top of the main file
% with nothing or the main filename (without extension) as argument.
% First, it breaks loops.
% If the argument is not empty and does not match |\childdocname|
% (which is set by the first inclusion of |childdoc.def|),
% |\ifchilddoc| is set to true, |\includeonly| is applied to the child file
% and |\jobname| is set to the main file
% (for proper handling of |.aux| files):
%    \begin{macrocode}
\newcommand{\childdocmain}[1]
{
  \childdocdisable\childdocmain{}
  \if?#1?\else
    \begingroup
      \def\childdoctmp{#1}
      \ifx\childdoctmp\childdocname
        \def\childdoctmp{}
      \else
        \def\childdoctmp
        {
          \childdoctrue
          \includeonly{\childdocname}
          \def\childdocjob{#1}
          \def\jobname{#1}
        }
      \fi
      \expandafter
    \endgroup
    \childdoctmp
  \fi
}
%    \end{macrocode}

% \macro{\childdocof}
% The command |\childdocof| redirects
% compilation to the main file |#1|.
%    \begin{macrocode}
\newcommand{\childdocof}[1]
{
  \childdocdisable
  \childdoctrue
  \includeonly{\childdocname}
  \def\jobname{#1}
  \def\childdocjob{#1}
  \input{#1}
}
%    \end{macrocode}

% \macro{\childdocby}
% The command |\childdocby| ....
%    \begin{macrocode}
\newcommand{\childdocby}[2][]
{
  \childdocdisable
  \childdoctrue
  \childdocmanualtrue
  \if?#1?\else
    \def\jobname{#2}
  \fi
  \def\childdocjob{#2}
  \input{#2}
  \endinput
}
%    \end{macrocode}

% \macro{\childdocforward}
% The command |\childdocforward| redirects
% compilation to the main file or
% (if the optional argument is given) a child file.
% Parameters are set as if the main file
% or a child file starting with |\childdocof| was compiled.
% Then compilation is handed over to the main file:
%    \begin{macrocode}
\newcommand{\childdocforward}[2][]
{
  \begingroup
    \if?#1?
      \def\childdoctmp
      {
        \def\childdocname{#2}
        \def\childdocjob{#2}
        \def\jobname{#2}
        \input{#2}
        \endinput
      }
    \else
      \def\childdoctmp
      {
        \childdocdisable
        \def\childdocname{#2}
        \childdoctrue
        \includeonly{#2}
        \def\childdocjob{#1}
        \def\jobname{#1}
        \input{#1}
        \endinput
      }
    \fi
    \expandafter
  \endgroup
  \childdoctmp
}
%    \end{macrocode}

% \macro{\childdocforwardprefix}
% The command |\childdocforwardprefix| redirects
% compilation to the main or a child file by means of a pattern.
% The prefix |#1| in the current filename is replaced by |#2|
% and the suffix of the current filename is kept
% (it is assumed that the filename does not contain the substring `|~~~|'
% which is used as a delimiter).
% Compilation is handed over to the new file by |\childdocforward|:
%    \begin{macrocode}
\newcommand{\childdocforwardprefix}[3][]
{
  \begingroup
    \def\childdocextract #2##1~~~{\def\childdoctmp{\childdocforward[#1]{#3##1}}}
    \expandafter\childdocextract\childdocname~~~
    \expandafter
  \endgroup
  \childdoctmp
}
%    \end{macrocode}

% \macro{\childdoc}
% The deprecated macro |\childdoc| is a legacy version of |\childdocmain|:
%    \begin{macrocode}
\newcommand{\childdoc}{\childdocmain}
%    \end{macrocode}

% \macro{\childdocredirect}
% The deprecated macro |\childdocredirect| is a legacy version
% of |\childdocforward| and |\childdocforwardprefix|:
%    \begin{macrocode}
\newcommand{\childdocredirect}[2][]
{
  \begingroup
    \if?#1?
      \def\childdoctmp{\childdocforward{#2}}
    \else
      \def\childdoctmp{\childdocforwardprefix{#1}{#2}}
    \fi
    \expandafter
  \endgroup
  \childdoctmp
}
%    \end{macrocode}

%\iffalse
%</package>
%\fi
%
\endinput
|\\
|\childdocforward{|\textit{main}|}|\\
\end{tabular}
\end{center}
%
or alternatively with:
%
\begin{center}
\begin{tabular}{l}
|% \iffalse
%
% childdoc.dtx Copyright (C) 2017-2018 Niklas Beisert
%
% This work may be distributed and/or modified under the
% conditions of the LaTeX Project Public License, either version 1.3
% of this license or (at your option) any later version.
% The latest version of this license is in
%   http://www.latex-project.org/lppl.txt
% and version 1.3 or later is part of all distributions of LaTeX
% version 2005/12/01 or later.
%
% This work has the LPPL maintenance status `maintained'.
%
% The Current Maintainer of this work is Niklas Beisert.
%
% This work consists of the files childdoc.dtx and childdoc.ins
% and the derived files childdoc.def and cdocsamp.tex with
% cdocsch1.tex, cdocsch2.tex, cdocsdrf.tex, cdocsfn1.tex, cdocsfn2.tex.
%
%<package>\ifdefined\childdocmain\endinput\fi
%<package>\ProvidesFile{childdoc.def}[2018/12/30 v2.0 child document driver]
%<samplemain>\ProvidesFile{cdocsamp.tex}[2018/12/30 v2.0 sample for childdoc]
%<*driver>
%\ProvidesFile{childdoc.drv}[2018/12/30 v2.0 childdoc reference manual file]
\PassOptionsToClass{10pt,a4paper}{article}
\documentclass{ltxdoc}

\usepackage[margin=35mm]{geometry}
\usepackage{hyperref}
\usepackage{hyperxmp}
\usepackage[usenames]{color}

\hypersetup{colorlinks=true}
\hypersetup{pdfstartview=FitH}
\hypersetup{pdfpagemode=UseNone}
\hypersetup{pdfsource={}}
\hypersetup{pdflang={en-UK}}
\hypersetup{pdfcopyright={Copyright 2017-2018 Niklas Beisert.
  This work may be distributed and/or modified under the
  conditions of the LaTeX Project Public License, either version 1.3
  of this license or (at your option) any later version.}}
\hypersetup{pdflicenseurl={http://www.latex-project.org/lppl.txt}}
\hypersetup{pdfcontactaddress={ETH Zurich, ITP, HIT K,
  Wolfgang-Pauli-Strasse 27}}
\hypersetup{pdfcontactpostcode={8093}}
\hypersetup{pdfcontactcity={Zurich}}
\hypersetup{pdfcontactcountry={Switzerland}}
\hypersetup{pdfcontactemail={nbeisert@itp.phys.ethz.ch}}
\hypersetup{pdfcontacturl={http://people.phys.ethz.ch/\xmptilde nbeisert/}}

\newcommand{\secref}[1]{\hyperref[#1]{section \ref*{#1}}}

\parskip1ex
\parindent0pt
\let\olditemize\itemize
\def\itemize{\olditemize\parskip0pt}

\begin{document}

\title{The \textsf{childdoc} Package}
\hypersetup{pdftitle={The childdoc Package}}
\author{Niklas Beisert\\[2ex]
  Institut f\"ur Theoretische Physik\\
  Eidgen\"ossische Technische Hochschule Z\"urich\\
  Wolfgang-Pauli-Strasse 27, 8093 Z\"urich, Switzerland\\[1ex]
  \href{mailto:nbeisert@itp.phys.ethz.ch}
  {\texttt{nbeisert@itp.phys.ethz.ch}}}
\hypersetup{pdfauthor={Niklas Beisert}}
\hypersetup{pdfsubject={Manual for the LaTeX2e Package childdoc}}
\date{30 December 2018, \textsf{v2.0}}
\maketitle

\begin{abstract}\noindent
\textsf{childdoc} is a \LaTeXe{} package
that enables the direct compilation
of document sections included by |\include|
to individual files.
\end{abstract}

\begingroup
\parskip0ex
\tableofcontents
\endgroup

%%%%%%%%%%%%%%%%%%%%%%%%%%%%%%%%%%%%%%%%%%%%%%%%%%%%%%%%%%%%%%%%%%%%%%%%%%%%%%%%
%%%%%%%%%%%%%%%%%%%%%%%%%%%%%%%%%%%%%%%%%%%%%%%%%%%%%%%%%%%%%%%%%%%%%%%%%%%%%%%%
\section{Introduction}

\LaTeX{} provides a mechanism to structure a large document (such as a book)
into a main file and several child files (containing the chapters)
using the |\include| command.
This mechanism is beneficial for documents
which span hundreds of pages in order to
make the source file(s) more manageable.
Moreover, compilation can be restricted to
selected child files by means of the |\includeonly| command.
The latter feature can be used to reduce the compilation time while editing
(this was significantly more useful in the earlier days of \LaTeX{})
or to generate a smaller document which is easier to navigate.
Another application of |\includeonly| is to generate
documents consisting of selected parts of the complete document.

However, there are a few drawbacks of the plain |\include| mechanism:
\begin{itemize}
\item
The child files cannot be compiled on their own,
they can only be compiled via the main file.
A naive editing environment
(such as a text editor with an option
to have the current file processed by \LaTeX)
may require one to switch to the main file before compiling;
attempting to compile the child file produces errors.
\item
The main file must be modified (each time)
to adjust the |\includeonly| command
to the present needs. This easily leaves the main file in a messy state.
\item
The generated document will always carry the filename
of the main document. This is inconvenient if
several child files are to be compiled and
to be kept for distribution.
\end{itemize}

The present package provides a simple interface
to make child files individually compilable by \LaTeX{}.
Compiling a child file then has the same effect as compiling
the main file with an |\includeonly| command
to select the appropriate child.
Moreover the generated document will carry the name of the child
rather than the main file.
This resolves all three above issues.

This feature is meant to make the editing of books,
thesis documents and lecture notes somewhat more convenient.
However, the package can also be used efficiently for
composing a series of documents (such as exercise sheets)
which are typically distributed individually.
It then assists the author in generating the individual documents
(potentially in different versions)
as well as a document containing the collected series.
Another application is in developing style files
or other kinds of included material
where compilation of the style file could redirect
to a sample or test file.

%%%%%%%%%%%%%%%%%%%%%%%%%%%%%%%%%%%%%%%%%%%%%%%%%%%%%%%%%%%%%%%%%%%%%%%%%%%%%%%%
%%%%%%%%%%%%%%%%%%%%%%%%%%%%%%%%%%%%%%%%%%%%%%%%%%%%%%%%%%%%%%%%%%%%%%%%%%%%%%%%
\section{Usage}

First of all, the package \textsf{childdoc} is \emph{not} a standard
\LaTeXe{} |.sty| style file! Therefore it needs to be invoked in
a non-standard way.

%%%%%%%%%%%%%%%%%%%%%%%%%%%%%%%%%%%%%%%%%%%%%%%%%%%%%%%%%%%%%%%%%%%%%%%%%%%%%%%%
\subsection{Included Files}
\label{sec:include}

%%%%%%%%%%%%%%%%%%%%%%%%%%%%%%%%%%%%%%%%
\DescribeMacro{\childdocmain}
To use the package, add the commands
\begin{center}
\begin{tabular}{l}
|\input{childdoc.def}|\\
|\childdocmain{}|\\
\end{tabular}
\end{center}
at the very top of the main \LaTeX{} file,
in particular \emph{before} the |\documentclass| statement!
The argument of |\childdocmain| should be left empty
(but it must be present).

%%%%%%%%%%%%%%%%%%%%%%%%%%%%%%%%%%%%%%%%
\DescribeMacro{\childdocof}
Furthermore, add the commands
\begin{center}
\begin{tabular}{l}
|\input{childdoc.def}|\\
|\childdocof{|\textit{main}|}|\\
\end{tabular}
\end{center}
at the top of every child file \textit{child}
which is included by |\include{|\textit{child}|}|
from within the main file
(or at least for those files to be compiled individually).
The argument \textit{main} must be the filename of the main file.

There are a couple of
considerations in setting up the main and child documents:

%%%%%%%%%%%%%%%%%%%%%%%%%%%%%%%%%%%%%%%%
\paragraph{Restrictions.}

Please note the following restrictions:
\begin{itemize}
\item
|\childdocmain| must be called with one argument \textit{main}
to ensure compatibility with earlier version of the package.
It must either be empty (|\childdocmain{}|)
or precisely match the filename of the main file in which it is specified.
See \secref{sec:detection} for further information.
\item
The filename \textit{main} must be specified without the |.tex| extension.
\item
The filename \textit{main} is case sensitive
(even in case-insensitive file systems)
due to internal string comparison.
\item
The argument \textit{main} should be fully expanded, it cannot be a macro.
\item
Subdirectories and special characters should be avoided in filenames.
\item
The command |\childdocmain{|\textit{main}|}| must be followed by a whitespace.
It should not be followed immediately by another command
or by a comment mark `|%|'.
This is because the \TeX{} parser reads the token immediately following
the argument of |\childdocmain| and puts it
at the beginning of every child section;
however, a white\-space is ignored.
\end{itemize}

%%%%%%%%%%%%%%%%%%%%%%%%%%%%%%%%%%%%%%%%
\paragraph{Content of Main File.}

It is advisable to place all content in the child files included by |\include|.
Any output contained in the main file will appear in all child documents
unless suppressed manually;
it cannot be suppressed automatically by the |\includeonly| directive
and thus should normally be avoided.
A method to include some content in the main file
by means of conditional processing is described in \secref{sec:conditional}.

%%%%%%%%%%%%%%%%%%%%%%%%%%%%%%%%%%%%%%%%
\paragraph{Page Numbering.}

When only a part of the document is compiled,
the appropriate numbering of pages
(as well as other status parameters)
is determined from the |.aux| files.
The latter contain information from previous passes.
However this information needs to propagate through
all intermediate child documents.
Therefore the page numbering in child documents may well
be inconsistent until the complete document is compiled at least once.

A useful (if unconventional) way to always ensure a consistent
page numbering is to restart the numbering in each child document
and denote the pages by `\textit{child}|.|\textit{page}'
where \textit{child} represents the chapter/section number of the child file.
This can be achieved by the command
|\numberwithin{page}{|\textit{child}|}|
of the \textsf{amsmath} package
where \textit{child} can be |chapter| or |section|
depending on the chosen structuring.
Alternatively, one can modify the macro |\thepage| appropriately
and reset the counter |page| at the start of each child file.

%%%%%%%%%%%%%%%%%%%%%%%%%%%%%%%%%%%%%%%%%%%%%%%%%%%%%%%%%%%%%%%%%%%%%%%%%%%%%%%%
\subsection{Conditional Processing}
\label{sec:conditional}

The package provides a mechanism to compile different versions
of a document. To customise the versions further some conditional processing
can come in handy to distinguish which version is being compiled.
The package provides two macros to describe the compilation context:

%%%%%%%%%%%%%%%%%%%%%%%%%%%%%%%%%%%%%%%%
\DescribeMacro{\ifchilddoc}
The conditional |\ifchilddoc| distinguishes between the compilation of
child documents and the main document:
%
\begin{center}
|\ifchilddoc |\textit{child-code}| |[|\||else |\textit{main-code}]| \||fi|
\end{center}

%%%%%%%%%%%%%%%%%%%%%%%%%%%%%%%%%%%%%%%%
\DescribeMacro{\childdocname}
\DescribeMacro{\childdocjob}
The macro |\childdocname| contains the filename (without extension)
of the main or child file being processed.
Note that |\childdocjob| will always contain the name of the main file.

%%%%%%%%%%%%%%%%%%%%%%%%%%%%%%%%%%%%%%%%
\paragraph{Title Page.}

Conditional processing can be used to include a title or banner page
in the main document when proper precautions are taken.
Importantly, the code in the main file should ensure that the page counter
(as well as other status parameters which are stored in the |.aux| files)
takes the same value after the conditional processing.
Otherwise the page numbers may take divergent values
depending on which part is compiled.

For example, a title page could be declared by:
%
\begin{center}
\begin{tabular}{l}
|\ifchilddoc\||else|\\
|\addtocounter{page}{-1}|\\
\textit{code for title page}\\
|\newpage|\\
|\||fi|
\end{tabular}
\end{center}
%
A banner page for the child documents can be generated by:
%
\begin{center}
\begin{tabular}{l}
|\ifchilddoc|\\
|\addtocounter{page}{-1}|\\
\textit{code for banner page}\\
|\newpage|\\
|\||fi|
\end{tabular}
\end{center}
%
Here one could write a message such as:
\begin{center}
|This is the part \childdocname{} of \childdocjob{}.|
\end{center}

%%%%%%%%%%%%%%%%%%%%%%%%%%%%%%%%%%%%%%%%%%%%%%%%%%%%%%%%%%%%%%%%%%%%%%%%%%%%%%%%
\subsection{Flags}
\label{sec:flags}

The package makes it easy to generate different versions
of the main or child documents.
To this end compilation flags can be defined
and assigned different default values.
They will be particularly useful in conjunction
with the forwarding mechanism described in \secref{sec:forward}.

For example, it may be useful to have a flag |\version|
which can be set to |draft| or |final|.
The document source will contain some conditional code
depending on the value of |\version|.
Suppose further, the flag should default to |final| for the main file
and to |draft| for child files
which is a natural assignment for editing the document.
This is achieved by placing the following code
in the preamble of the main document
(below the |\childdocmain| directive):
%
\begin{center}
\begin{tabular}{l}
|\ifchilddoc|\\
|\providecommand{\version}{draft}|\\
|\||else|\\
|\providecommand{\version}{final}|\\
|\||fi|
\end{tabular}
\end{center}
%
The definition by |\providecommand| makes sure
that previous definitions are not overwritten.
Further statements |\providecommand{\version}{...}|
can thus be added before the above code to override it.

For the main file, one might add a line
(between |\childdocmain| and the above block)
%
\begin{center}
|%\ifchilddoc\||else\providecommand{\version}{draft}\||fi|
\end{center}
%
which can be uncommented to produce a draft version.
Likewise one can add a line to the very top of a child file
(above the |\childdocof{|\textit{main}|}| directive)
%
\begin{center}
|%\providecommand{\version}{final}|
\end{center}
%
which can be uncommented to produce the final version of this child document.

%%%%%%%%%%%%%%%%%%%%%%%%%%%%%%%%%%%%%%%%%%%%%%%%%%%%%%%%%%%%%%%%%%%%%%%%%%%%%%%%
\subsection{Forwarding}
\label{sec:forward}

Different versions of the main or child documents
using compilation flags as described in \secref{sec:flags}
can be (permanently) stored in different files
for convenient compilation, viewing and distribution.
To this end, the package defines a command
to pass on compilation to a different file:

%%%%%%%%%%%%%%%%%%%%%%%%%%%%%%%%%%%%%%%%
\DescribeMacro{\childdocforward}
The command |\childdocforward| redirects processing to
another source file:
%
\begin{center}
\begin{tabular}{l}
|\input{childdoc.def}|\\
|\childdocforward[|\textit{main}|]{|\textit{dest}|}|\\
\end{tabular}
\end{center}
%
The argument \textit{dest} is the destination file
(without extension).
It should be the main file or one of the child files.
Note that further \textsf{childdoc} directives
such as |\childdocof| and |\childdocforward|
in the indicated file will be processed in this form.
The optional argument \textit{main}
passes on directly to the main file \textit{main}
while pretending to compile the child \textit{dest}.
This form behaves as if \textit{dest}
issues |\childdocof{|\textit{main}|}| right away,
and no further \textsf{childdoc} directives will be processed.

%%%%%%%%%%%%%%%%%%%%%%%%%%%%%%%%%%%%%%%%
\DescribeMacro{\...prefix}
In the alternative form |\childdocforwardprefix|,
%
\begin{center}
\begin{tabular}{l}
|\input{childdoc.def}|\\
|\childdocforwardprefix[|\textit{main}|]{|\textit{prefix}|}{|\textit{dest}|}|
\end{tabular}
\end{center}
%
the destination file is determined by a pattern
depending on the current file:
To make this work, the current file must be called
`{\textit{prefix}\hspace{0.2em}\textit{suffix}}'
with \textit{prefix} matching precisely the argument.
Processing is then passed on to the file
`{\textit{dest}\hspace{0.2em}\textit{suffix}}'.
Surely, the same effect is achieved by
directly specifying the
argument `{\textit{dest}\hspace{0.2em}\textit{suffix}}'
in the first form.
However, that requires to set up a different file
for each child. With the alternative form of the command
all these files can have exactly the same content
which simplifies setting them up and maintaining them.

For example, the following file |draft.tex|
with a compilation flag |\version| as described in \secref{sec:flags}
compiles the main document as a draft:
%
\begin{center}
\begin{tabular}{l}
|\def\version{draft}|\\
|\input{childdoc.def}|\\
|\childdocforward{|\textit{main}|}|
\end{tabular}
\end{center}
%
Likewise, the following files |final|\textit{nn}|.tex|
compile the final version of the child document
|child|\textit{nn}|.tex|:
%
\begin{center}
\begin{tabular}{l}
|\def\version{final}|\\
|\input{childdoc.def}|\\
|\childdocforwardprefix{final}{child}|
\end{tabular}
\end{center}
%

Note that when several versions of a main file and/or of each child file
are to be generated, it may be convenient to set up a |Makefile| or
shell script to automatise the process.

%%%%%%%%%%%%%%%%%%%%%%%%%%%%%%%%%%%%%%%%%%%%%%%%%%%%%%%%%%%%%%%%%%%%%%%%%%%%%%%%
\subsection{Command Line Processing}
\label{sec:commandline}

The effect of redirection files can also be achieved by invoking
the \LaTeX{} compiler with a more elaborate command line.
Most conveniently this should be done as part
of a shell script or a |Makefile|.

When using \textsf{childdoc} in the main file, the following
command lines effectively perform a redirection
(note that depending on the shell being used,
backslashes may have to be doubled: `|\|' $\to$ `|\\|'):
%
\begin{center}
|... -jobname "|\textit{target}|" |\\|"|[\textit{flags}]%
|\input{childdoc.def}\childdocforward[|\textit{main}|]{|\textit{dest}|}"|
\end{center}
%
Here \textit{target} is the name of the output file,
\textit{main} is the name of the main file
and \textit{dest} is the name of the main or child file to be processed
(all filenames without extensions).
The optional argument \textit{main} can be omitted
if \textit{main} matches \textit{dest}.
Optionally, compilation \textit{flags} can be defined via |\def| commands.
This command line makes the \TeX{} engine believe
it is compiling the file \textit{target}
whose content is specified as the latter parameter.
The provided code then forwards the processing to
\textit{main} or \textit{dest} as described in \secref{sec:forward}.

%%%%%%%%%%%%%%%%%%%%%%%%%%%%%%%%%%%%%%%%%%%%%%%%%%%%%%%%%%%%%%%%%%%%%%%%%%%%%%%%
\subsection{Include by Input}
\label{sec:input}

Including child documents by |\include| has some restrictions by design.
Most notably, the content of a child document always occupies
its own set of pages; pages cannot be shared between child documents.
Usually, this behaviour makes perfect sense
because each child document contain an essential part of the document.
However, in some situations it may be desirable to compose
a document from a collection of parts
without having mandatory page breaks between then.
For this case, the package
provides a mechanism to include parts
by |\input| which can also be processed individually.
However, by construction this mechanism
requires manual handling of the content to be output.

%%%%%%%%%%%%%%%%%%%%%%%%%%%%%%%%%%%%%%%%
\DescribeMacro{\ifchilddocmanual}
The main file should be prepared as usual, see \secref{sec:include}.
However, the document body must make a distinction
between processing of an individual part and of the main document, e.g.:
%
\begin{center}
\begin{tabular}{l}
|\ifchilddocmanual|\\
|\input{\childdocname}|\\
|\||else|\\
\textit{document body with }|\input{|\textit{part}|}|\\
|\||fi|
\end{tabular}
\end{center}
%
The conditional |\ifchilddocmanual| is true whenever
a part to be included by |\input| is being compiled,
and the name of the part is stored in |\childdocname|.

%%%%%%%%%%%%%%%%%%%%%%%%%%%%%%%%%%%%%%%%
\DescribeMacro{\childdocby}
Each part to be included by |\input| should start with:
%
\begin{center}
\begin{tabular}{l}
|\input{childdoc.def}|\\
|\childdocby{|\textit{main}|}|\\
\end{tabular}
\end{center}
%
The directive |\childdocby| is similar to |\childdocof|
described in \secref{sec:include},
but the subsequent selection of content must be done manually.
To that end, both |\ifchilddoc| and |\ifchilddocmanual|
will be true upon processing of a part,
and the name of the part is stored in |\childdocname|.
Note that |\jobname| will be set to the filename of the current part
so that each part receives an individual |.aux| file
that does not interfere with the |.aux| file(s) of the main document.
This behaviour can be altered by the alternative form
|\childdocby[*]{|\textit{main}|}| (with a non-empty optional argument)
which uses the |.aux| file of the main document
by setting |\jobname| to \textit{main}.

%%%%%%%%%%%%%%%%%%%%%%%%%%%%%%%%%%%%%%%%%%%%%%%%%%%%%%%%%%%%%%%%%%%%%%%%%%%%%%%%
\subsection{Driver Development}
\label{sec:driver}

The \textsf{childdoc} mechanism can also be use for the development
of definition files such as \LaTeX{} styles or classes.
This case differs from the above setup with multiple parts
included by |\include| in that no |\includeonly| should be invoked.
This can be achieved by starting the include file
(before |\ProvidesPackage|) with:
%
\begin{center}
\begin{tabular}{l}
|\input{childdoc.def}|\\
|\childdocforward{|\textit{main}|}|\\
\end{tabular}
\end{center}
%
or alternatively with:
%
\begin{center}
\begin{tabular}{l}
|\input{childdoc.def}|\\
|\childdocby{|\textit{main}|}|\\
\end{tabular}
\end{center}
%
Both forms have slightly different effects as described above.
The main file is prepared as usual, see \secref{sec:include}.

%%%%%%%%%%%%%%%%%%%%%%%%%%%%%%%%%%%%%%%%%%%%%%%%%%%%%%%%%%%%%%%%%%%%%%%%%%%%%%%%
\subsection{Legacy Detection}
\label{sec:detection}

The directive |\childdocmain| in the main file can detect
whether the complete document or merely a child is to be compiled
even without using the directive |\childdocof|.
This method is deprecated because it is less robust
and there is no compelling reason to use it;
it is merely provided for backward compatibility
and it may be removed in future versions.

If the detection mechanism is to be used,
it is mandatory to correctly specify
the filename of the main file as the argument of |\childdocmain|:
%
\begin{center}
\begin{tabular}{l}
|\input{childdoc.def}|\\
|\childdocmain{|\textit{main}|}|\\
\end{tabular}
\end{center}
%
If |\jobname| does not match the argument \textit{main} of |\childdocmain|,
it is assumed that |\jobname| points to the child file to be compiled.
When using |\childdocmain| with the main file specified as argument,
it suffices to start a child file
with just |\input{|\textit{main}|}|
without loading of the package and using |\childdocof|.
If instead all processing is done
with the appropriate \textsf{childdoc} directives,
the argument of \textit{main} of |\childdocmain| can be empty.

An alternative version of the command line processing described
in \secref{sec:commandline} using the detection mechanism reads:
%
\begin{center}
|... -jobname "|\textit{target}|" "|[\textit{flags}]%
[|\def\jobname{|\textit{dest}|}|]|\input{|\textit{main}|}"|
\end{center}

%%%%%%%%%%%%%%%%%%%%%%%%%%%%%%%%%%%%%%%%%%%%%%%%%%%%%%%%%%%%%%%%%%%%%%%%%%%%%%%%
\subsection{Manual Code}
\label{sec:manual}

In case one cannot be certain whether the definitions file |childdoc.def|
is installed on the target \TeX{} distribution
and one prefers not to ship it,
it is conceivable to paste a few relevant commands into the sources.

To that end, drop all statements |\input{childdoc.def}|
and perform the replacements as outlined below.
Instead of |\childdocmain{|\textit{main}|}| add the following code
to the top of the main file:
%
\begin{center}
\begin{tabular}{l}
|\||ifdefined\childdocname\endinput\||fi\newif\ifchilddoc|\\
|\edef\childdocname{\scantokens\expandafter{\jobname\noexpand}}|\\
|\def\childdocmain{|\textit{main}|}\||ifx\childdocmain\childdocname\||else|\\
|\childdoctrue\includeonly{\childdocname}\let\jobname\childdocmain\||fi|\\
\end{tabular}
\end{center}
%
Instead of |\childdocof{|\textit{main}|}| just include the main file
at the top of each child file:
%
\begin{center}
|\input{|\textit{main}|}|
\end{center}
%
A simple redirection |\childdocforward{|\textit{dest}|}| is achieved by:
%
\begin{center}
|\def\jobname{|\textit{dest}|}\input{\jobname}|
\end{center}
%
The redirection with prefix
|\childdocforwardprefix[|\textit{prefix}|]{|\textit{dest}|}|
is accomplished by:
%
\begin{center}
\begin{tabular}{l}
|{\edef\jobname{\scantokens\expandafter{\jobname\noexpand}}|\\
|\def\redirectjob |\textit{prefix}|#1~~~{\gdef\jobname{|\textit{dest}|#1}}|\\
|\expandafter\redirectjob\jobname~~~}\input{\jobname}|
\end{tabular}
\end{center}

In an alternative approach,
child documents can be compiled by a specific command line
without additional code or specific definitions:
%
\begin{center}
|... -jobname "|\textit{target}|" "|[\textit{flags}]%
|\includeonly{|\textit{dest}|}\input{|\textit{main}|}"|
\end{center}
%

%%%%%%%%%%%%%%%%%%%%%%%%%%%%%%%%%%%%%%%%%%%%%%%%%%%%%%%%%%%%%%%%%%%%%%%%%%%%%%%%
%%%%%%%%%%%%%%%%%%%%%%%%%%%%%%%%%%%%%%%%%%%%%%%%%%%%%%%%%%%%%%%%%%%%%%%%%%%%%%%%
\section{Information}

%%%%%%%%%%%%%%%%%%%%%%%%%%%%%%%%%%%%%%%%%%%%%%%%%%%%%%%%%%%%%%%%%%%%%%%%%%%%%%%%
\subsection{Copyright}

Copyright \copyright{} 2017--2018 Niklas Beisert

This work may be distributed and/or modified under the
conditions of the \LaTeX{} Project Public License, either version 1.3
of this license or (at your option) any later version.
The latest version of this license is in
  \url{http://www.latex-project.org/lppl.txt}
and version 1.3 or later is part of all distributions of \LaTeX{}
version 2005/12/01 or later.

This work has the LPPL maintenance status `maintained'.

The Current Maintainer of this work is Niklas Beisert.

This work consists of the files |README.txt|, |childdoc.ins| and |childdoc.dtx|
as well as the derived files |childdoc.def|, |cdocsamp.tex|
with |cdocsch1.tex|, |cdocsch2.tex|, |cdocspt3.tex|, |cdocspt4.tex|,
|cdocsdrf.tex|, |cdocsfn1.tex|, |cdocsfn2.tex|
as well as |childdoc.pdf|.

%%%%%%%%%%%%%%%%%%%%%%%%%%%%%%%%%%%%%%%%%%%%%%%%%%%%%%%%%%%%%%%%%%%%%%%%%%%%%%%%
\subsection{Files and Installation}

The package consists of the files:
%
\begin{center}
\begin{tabular}{ll}
    |README.txt|   & readme file \\
    |childdoc.ins| & installation file \\
    |childdoc.dtx| & source file \\
    |childdoc.def| & definition file \\
    |cdocsamp.tex| & sample main file \\
    |cdocsch1.tex| & sample include file \\
    |cdocsch2.tex| & sample include file \\
    |cdocspt3.tex| & sample part file \\
    |cdocspt4.tex| & sample part file \\
    |cdocsdrf.tex| & sample redirection file \\
    |cdocsfn1.tex| & sample redirection file \\
    |cdocsfn2.tex| & sample redirection file \\
    |childdoc.pdf| & manual
\end{tabular}
\end{center}
%
The distribution consists of the files
|README.txt|, |childdoc.ins| and |childdoc.dtx|.
%
\begin{itemize}
\item
Run (pdf)\LaTeX{} on |childdoc.dtx|
to compile the manual |childdoc.pdf| (this file).
\item
Run \LaTeX{} on |childdoc.ins| to create the definitions file |childdoc.def|
and the sample |cdocsamp.tex| with include files
|cdocsch1.tex|, |cdocsch2.tex|, |cdocspt3.tex|, |cdocspt4.tex|,
|cdocsdrf.tex|, |cdocsfn1.tex|, |cdocsfn2.tex|.
Then copy the file |childdoc.def| to an appropriate directory of your \LaTeX{}
distribution, e.g.\ \textit{texmf-root}|/tex/latex/childdoc|.
\end{itemize}

%%%%%%%%%%%%%%%%%%%%%%%%%%%%%%%%%%%%%%%%%%%%%%%%%%%%%%%%%%%%%%%%%%%%%%%%%%%%%%%%
\subsection{Related CTAN Packages}

There are several other packages which offer a similar functionality:
%
\begin{itemize}
\item
The packages
\href{http://ctan.org/pkg/docmute}{\textsf{docmute}},
\href{http://ctan.org/pkg/includex}{\textsf{includex}} and
\href{http://ctan.org/pkg/standalone}{\textsf{standalone}}
provide commands to include only the document body of
a child file thus allowing both files to be compiled individually.
\item
The packages \href{http://ctan.org/pkg/subdocs}{\textsf{subdocs}}
and \href{http://ctan.org/pkg/subfiles}{\textsf{subfiles}}
provide structures in which the main and child documents can be
encapsulated and allowing them to be compiled individually.
The inclusion mechanism is different from the conventional |\include|.
\item
The package \href{http://ctan.org/pkg/combine}{\textsf{combine}}
is an elaborate solution to combine several documents into one.
\end{itemize}
%
See also the CTAN topic \href{http://ctan.org/topic/subdocs}{\textsf{subdocs}}
for further related packages.
The present package differs from the above solutions in that
a document structure constructed with the conventional |\include| mechanism
just needs two extra commands at the top of every file
such that all constituent files can be compiled individually.

%%%%%%%%%%%%%%%%%%%%%%%%%%%%%%%%%%%%%%%%%%%%%%%%%%%%%%%%%%%%%%%%%%%%%%%%%%%%%%%%
%\subsection{Feature Suggestions}
%
%The following is a list of features which may be useful for future
%versions of this package:
%%
%\begin{itemize}
%\item
%\ldots
%\end{itemize}

%%%%%%%%%%%%%%%%%%%%%%%%%%%%%%%%%%%%%%%%%%%%%%%%%%%%%%%%%%%%%%%%%%%%%%%%%%%%%%%%
\subsection{Revision History}

%%%%%%%%%%%%%%%%%%%%%%%%%%%%%%%%%%%%%%%%
\paragraph{v2.0:} 2018/12/30

\begin{itemize}
\item
immediate forward processing
\item
added |\childdocby| mechanism
\item
manual restructured
\end{itemize}

%%%%%%%%%%%%%%%%%%%%%%%%%%%%%%%%%%%%%%%%
\paragraph{v1.6:} 2018/01/17

\begin{itemize}
\item
application for development of include files
\item
corrections to manual
\end{itemize}

%%%%%%%%%%%%%%%%%%%%%%%%%%%%%%%%%%%%%%%%
\paragraph{v1.5:} 2017/05/21

\begin{itemize}
\item
more complete structuring introduced
\item
|\childdocof| introduced
\item
|\childdoc| renamed to |\childdocmain|
\item
|\childredirect| renamed to |\childdocforward| and |\childdocforwardprefix|
and functionality expanded
\end{itemize}

%%%%%%%%%%%%%%%%%%%%%%%%%%%%%%%%%%%%%%%%
\paragraph{v1.0:} 2017/04/27

\begin{itemize}
\item
manual and install package
\item
first version published on CTAN
\end{itemize}

%%%%%%%%%%%%%%%%%%%%%%%%%%%%%%%%%%%%%%%%
\paragraph{v0.6:} 2017/04/26

\begin{itemize}
\item
redirection mechanism added
\end{itemize}

%%%%%%%%%%%%%%%%%%%%%%%%%%%%%%%%%%%%%%%%
\paragraph{v0.5:} 2017/04/26

\begin{itemize}
\item
functionality in definition file
\end{itemize}


%%%%%%%%%%%%%%%%%%%%%%%%%%%%%%%%%%%%%%%%%%%%%%%%%%%%%%%%%%%%%%%%%%%%%%%%%%%%%%%%
%%%%%%%%%%%%%%%%%%%%%%%%%%%%%%%%%%%%%%%%%%%%%%%%%%%%%%%%%%%%%%%%%%%%%%%%%%%%%%%%
%%%%%%%%%%%%%%%%%%%%%%%%%%%%%%%%%%%%%%%%%%%%%%%%%%%%%%%%%%%%%%%%%%%%%%%%%%%%%%%%
\appendix

\settowidth\MacroIndent{\rmfamily\scriptsize 000\ }

 \DocInput{childdoc.dtx}

\end{document}
%</driver>
% \fi
%
% %%%%%%%%%%%%%%%%%%%%%%%%%%%%%%%%%%%%%%%%%%%%%%%%%%%%%%%%%%%%%%%%%%%%%%%%%%%%%%
% %%%%%%%%%%%%%%%%%%%%%%%%%%%%%%%%%%%%%%%%%%%%%%%%%%%%%%%%%%%%%%%%%%%%%%%%%%%%%%
% \section{Sample}
%\iffalse
%<*samplemain>
%\fi
%
% The following presents a sample document
% with two chapters, two parts, a title page,
% a compile flag as well as three forwarding files to set the flag.
% It consists of eight |.tex| files:
% \begin{center}
% \begin{tabular}{ll}
% |cdocsamp.tex|&main file\\
% |cdocsch1.tex|&include file for chapter 1\\
% |cdocsch2.tex|&include file for chapter 2\\
% |cdocspt3.tex|&include file for part 3\\
% |cdocspt4.tex|&include file for part 4\\
% |cdocsdrf.tex|&forwarding file for main file in draft mode\\
% |cdocsfi1.tex|&forwarding file for final version of chapter 1\\
% |cdocsfi2.tex|&forwarding file for final version of chapter 2\\
% \end{tabular}
% \end{center}
% Each of the eight files can be compiled directly by the \LaTeX{} compiler.
%
% %%%%%%%%%%%%%%%%%%%%%%%%%%%%%%%%%%%%%%
% \paragraph{Main File.}
%
% The main file is called |cdocsamp.tex|.
%
% Load the \textsf{childdoc} definitions and
% declare the filename for the main document:
%    \begin{macrocode}
\input{childdoc.def}
\childdocmain{}
%    \end{macrocode}

% Optional override for |\version| flag:
%    \begin{macrocode}
%%\ifchilddoc\else\providecommand{\version}{draft}\fi
%    \end{macrocode}

% Define the default values for the |\version| flag
% (|final| for the main file and |draft| for childs):
%    \begin{macrocode}
\ifchilddoc
\providecommand{\version}{draft}
\else
\providecommand{\version}{final}
\fi
%    \end{macrocode}

% Load the standard document class:
%    \begin{macrocode}
\documentclass[12pt]{article}
%    \end{macrocode}

% Start the document body:
%    \begin{macrocode}
\begin{document}
%    \end{macrocode}

% Declare a title page.
% Print title, part of document being processed and version flag:
%    \begin{macrocode}
\addtocounter{page}{-1}
\begin{center}
{\LARGE\bfseries{}childdoc example\par}
\vspace{1cm}
\ifchilddoc
\ifchilddocmanual part\else chapter\fi:
`\childdocname' of `\childdocjob'\par
\else
main document: `\childdocjob'\par
\fi
version: \version\par
\end{center}
\newpage
%    \end{macrocode}

% Manually include selected file,
% otherwise process as usual:
%    \begin{macrocode}
\ifchilddocmanual
\section*{part `\childdocname'}
\input{\childdocname}
\else
%    \end{macrocode}

% Include the two chapters:
%    \begin{macrocode}
\include{cdocsch1}
\include{cdocsch2}
%    \end{macrocode}

% Include the two parts unless only chapters should be displayed:
%    \begin{macrocode}
\ifchilddoc\else
\section{part three}
\input{cdocspt3}
\section{part four}
\input{cdocspt4}
\fi
%    \end{macrocode}

% Process as usual until here:
%    \begin{macrocode}
\fi
%    \end{macrocode}

% End of document body:
%    \begin{macrocode}
\end{document}
%    \end{macrocode}
%\iffalse
%</samplemain>
%\fi
%
% %%%%%%%%%%%%%%%%%%%%%%%%%%%%%%%%%%%%%%
% \paragraph{Chapter Include Files.}
%
% The include files are called |cdocsch1.tex| and |cdocsch2.tex|.
%
%\iffalse
%<*samplechap1|samplechap2>
%\fi

% Optional override for |\version| flag:
%    \begin{macrocode}
%%\providecommand{\version}{final}
%    \end{macrocode}

% Include the main document:
%    \begin{macrocode}
\input{childdoc.def}
\childdocof{cdocsamp}
%    \end{macrocode}

%\iffalse
%</samplechap1|samplechap2>
%\fi
%
%\iffalse
%<*samplechap1>
%\fi
% Some text for chapter 1:
%    \begin{macrocode}
\section{one}
some text in chapter one
%    \end{macrocode}

%\iffalse
%</samplechap1>
%\fi
% Some text for chapter 2:
%\iffalse
%<*samplechap2>
%\fi
%    \begin{macrocode}
\section{two}
more text in chapter two
%    \end{macrocode}

%\iffalse
%</samplechap2>
%\fi
%
% %%%%%%%%%%%%%%%%%%%%%%%%%%%%%%%%%%%%%%
% \paragraph{Part Include Files.}
%
% The include files are called |cdocspt3.tex| and |cdocspt4.tex|.
%
%\iffalse
%<*samplepart3|samplepart4>
%\fi

% Optional override for |\version| flag:
%    \begin{macrocode}
%%\providecommand{\version}{final}
%    \end{macrocode}

% Include the main document:
%    \begin{macrocode}
\input{childdoc.def}
\childdocby{cdocsamp}
%    \end{macrocode}

%\iffalse
%</samplepart3|samplepart4>
%\fi
%
%\iffalse
%<*samplepart3>
%\fi
% Some text for part 3:
%    \begin{macrocode}
some text in part three
%    \end{macrocode}

%\iffalse
%</samplepart3>
%\fi
% Some text for part 4:
%\iffalse
%<*samplepart4>
%\fi
%    \begin{macrocode}
more text in part four
%    \end{macrocode}

%\iffalse
%</samplepart4>
%\fi
%
% %%%%%%%%%%%%%%%%%%%%%%%%%%%%%%%%%%%%%%
% \paragraph{Forwarding for a Complete Draft.}
%
% The following forwarding file |cdocsdrf.tex|
% compiles the main document in draft mode:
%\iffalse
%<*sampledraft>
%\fi
%    \begin{macrocode}
\def\version{draft}
\input{childdoc.def}
\childdocforward{cdocsamp}
%    \end{macrocode}

%\iffalse
%</sampledraft>
%\fi
%
% %%%%%%%%%%%%%%%%%%%%%%%%%%%%%%%%%%%%%%
% \paragraph{Forwarding for Final Version of the Chapters.}
%
% The following forwarding files |cdocsfn1.tex| and |cdocsfn2.tex|
% (with identical content)
% compile the final versions of the child documents
% |cdocsch1.tex| and |cdocsch2.tex|, respectively:
%\iffalse
%<*samplefinal>
%\fi
%    \begin{macrocode}
\def\version{final}
\input{childdoc.def}
\childdocforwardprefix[cdocsamp]{cdocsfn}{cdocsch}
%    \end{macrocode}

%\iffalse
%</samplefinal>
%\fi
%
% %%%%%%%%%%%%%%%%%%%%%%%%%%%%%%%%%%%%%%
% \paragraph{Command Line Processing.}
%
% The following three command lines generate the output files
% |cdocscld|, |cdocscl1| and |cdocscl2|
% which should be identical to
% |cdocsdrf|, |cdocsch1| and |cdocsfn2|, respectively:
% \begin{center}
% \begin{tabular}{l}
% |latex -jobname cdocscld \|\\
% |  "\def\version{draft}\input{childdoc.def}\childdocforward{cdocsamp}"|\\
% |latex -jobname cdocscl1 \|\\
% |  "\input{childdoc.def}\childdocforward[cdocsamp]{cdocsch1}"|\\
% |latex -jobname cdocscl2 \|\\
% |  "\def\version{final}\input{childdoc.def}\childdocforward{cdocsch2}"|
% \end{tabular}
% \end{center}
% Note that the trailing backslash on each first line
% merely continues the input to the second line
% (for convenient cut ant paste).
% Furthermore, the command |latex| can be replaced by any
% of its alternative versions such as |pdflatex|.
%
% %%%%%%%%%%%%%%%%%%%%%%%%%%%%%%%%%%%%%%%%%%%%%%%%%%%%%%%%%%%%%%%%%%%%%%%%%%%%%%
% %%%%%%%%%%%%%%%%%%%%%%%%%%%%%%%%%%%%%%%%%%%%%%%%%%%%%%%%%%%%%%%%%%%%%%%%%%%%%%
% \section{Implementation}
%\iffalse
%<*package>
%\fi
%
% This section describes the definitions file |childdoc.def|.

% The definitions cannot be loaded using |\usepackage| or |\RequirePackage|
% which has a mechanism to prevent loading a style file more than once.
% When loading the definitions by means of |\input|
% multiple instances have to be prevented manually:
%\iffalse
%This code needs to be before the `\ProvidesFile' directive
%which is defined at the beginning of this file.
%Therefore it is also placed there and commented out here.
%</package>
%<*discard>
%\fi
%    \begin{macrocode}
\ifdefined\childdocmain\endinput\fi
%    \end{macrocode}
%\iffalse
%</discard>
%<*package>
%\fi
%
% \macro{\ifchilddoc}
% \macro{\ifchilddocmanual}
% The conditional |\ifchilddoc| tells whether a
% child (true) or main (false) document is being compiled.
% The conditional |\ifchilddocmanual| tells whether
% the |\includeonly| mechanism is used (false) or
% the selection of child files must be performed manually (true).
% The definitions initialise to false:
%    \begin{macrocode}
\newif\ifchilddoc
\newif\ifchilddocmanual
%    \end{macrocode}

% \macro{\childdocname}
% \macro{\childdocjob}
% The macro |\childdocname| stores the name of the main document
% to be compiled. The macro |\childdocjob| stores the name of
% the document on which the \LaTeX{} compiler was originally invoked.
% The content of |\jobname| cannot be compared
% to filenames specified in the source due to different catcodes.
% The following code rescans |\jobname|, stores the result
% in |\childdocname| and saves a copy in |\childdocjob|:
%    \begin{macrocode}
\edef\childdocname{\scantokens\expandafter{\jobname\noexpand}}
\let\childdocjob\childdocname
%    \end{macrocode}

% \macro{\childdocdisable}
% The macro |\childdocdisable| prevents the main file
% from being processed more than once.
% At this stage, the main document command |\childdocmain|
% is assumed to be called once again where it should do nothing.
% Any subsequent call to it should prevent
% a secondary processing of the main document
% It overwrites the forwarding commands
% |\childdocof| and |\childdocforward|
% with empty macros to prevent further inclusions of the main document:
%    \begin{macrocode}
\newcommand{\childdocdisable}
{
  \renewcommand{\childdocmain}[1]{\renewcommand{\childdocmain}[1]{\endinput}}
  \renewcommand{\childdocof}[1]{}
  \renewcommand{\childdocby}[2][]{}
  \renewcommand{\childdocforward}[2][]{}
  \renewcommand{\childdocdisable}{}
}
%    \end{macrocode}

% \macro{\childdocmain}
% The macro |\childdocmain| is to be called at the top of the main file
% with nothing or the main filename (without extension) as argument.
% First, it breaks loops.
% If the argument is not empty and does not match |\childdocname|
% (which is set by the first inclusion of |childdoc.def|),
% |\ifchilddoc| is set to true, |\includeonly| is applied to the child file
% and |\jobname| is set to the main file
% (for proper handling of |.aux| files):
%    \begin{macrocode}
\newcommand{\childdocmain}[1]
{
  \childdocdisable\childdocmain{}
  \if?#1?\else
    \begingroup
      \def\childdoctmp{#1}
      \ifx\childdoctmp\childdocname
        \def\childdoctmp{}
      \else
        \def\childdoctmp
        {
          \childdoctrue
          \includeonly{\childdocname}
          \def\childdocjob{#1}
          \def\jobname{#1}
        }
      \fi
      \expandafter
    \endgroup
    \childdoctmp
  \fi
}
%    \end{macrocode}

% \macro{\childdocof}
% The command |\childdocof| redirects
% compilation to the main file |#1|.
%    \begin{macrocode}
\newcommand{\childdocof}[1]
{
  \childdocdisable
  \childdoctrue
  \includeonly{\childdocname}
  \def\jobname{#1}
  \def\childdocjob{#1}
  \input{#1}
}
%    \end{macrocode}

% \macro{\childdocby}
% The command |\childdocby| ....
%    \begin{macrocode}
\newcommand{\childdocby}[2][]
{
  \childdocdisable
  \childdoctrue
  \childdocmanualtrue
  \if?#1?\else
    \def\jobname{#2}
  \fi
  \def\childdocjob{#2}
  \input{#2}
  \endinput
}
%    \end{macrocode}

% \macro{\childdocforward}
% The command |\childdocforward| redirects
% compilation to the main file or
% (if the optional argument is given) a child file.
% Parameters are set as if the main file
% or a child file starting with |\childdocof| was compiled.
% Then compilation is handed over to the main file:
%    \begin{macrocode}
\newcommand{\childdocforward}[2][]
{
  \begingroup
    \if?#1?
      \def\childdoctmp
      {
        \def\childdocname{#2}
        \def\childdocjob{#2}
        \def\jobname{#2}
        \input{#2}
        \endinput
      }
    \else
      \def\childdoctmp
      {
        \childdocdisable
        \def\childdocname{#2}
        \childdoctrue
        \includeonly{#2}
        \def\childdocjob{#1}
        \def\jobname{#1}
        \input{#1}
        \endinput
      }
    \fi
    \expandafter
  \endgroup
  \childdoctmp
}
%    \end{macrocode}

% \macro{\childdocforwardprefix}
% The command |\childdocforwardprefix| redirects
% compilation to the main or a child file by means of a pattern.
% The prefix |#1| in the current filename is replaced by |#2|
% and the suffix of the current filename is kept
% (it is assumed that the filename does not contain the substring `|~~~|'
% which is used as a delimiter).
% Compilation is handed over to the new file by |\childdocforward|:
%    \begin{macrocode}
\newcommand{\childdocforwardprefix}[3][]
{
  \begingroup
    \def\childdocextract #2##1~~~{\def\childdoctmp{\childdocforward[#1]{#3##1}}}
    \expandafter\childdocextract\childdocname~~~
    \expandafter
  \endgroup
  \childdoctmp
}
%    \end{macrocode}

% \macro{\childdoc}
% The deprecated macro |\childdoc| is a legacy version of |\childdocmain|:
%    \begin{macrocode}
\newcommand{\childdoc}{\childdocmain}
%    \end{macrocode}

% \macro{\childdocredirect}
% The deprecated macro |\childdocredirect| is a legacy version
% of |\childdocforward| and |\childdocforwardprefix|:
%    \begin{macrocode}
\newcommand{\childdocredirect}[2][]
{
  \begingroup
    \if?#1?
      \def\childdoctmp{\childdocforward{#2}}
    \else
      \def\childdoctmp{\childdocforwardprefix{#1}{#2}}
    \fi
    \expandafter
  \endgroup
  \childdoctmp
}
%    \end{macrocode}

%\iffalse
%</package>
%\fi
%
\endinput
|\\
|\childdocby{|\textit{main}|}|\\
\end{tabular}
\end{center}
%
Both forms have slightly different effects as described above.
The main file is prepared as usual, see \secref{sec:include}.

%%%%%%%%%%%%%%%%%%%%%%%%%%%%%%%%%%%%%%%%%%%%%%%%%%%%%%%%%%%%%%%%%%%%%%%%%%%%%%%%
\subsection{Legacy Detection}
\label{sec:detection}

The directive |\childdocmain| in the main file can detect
whether the complete document or merely a child is to be compiled
even without using the directive |\childdocof|.
This method is deprecated because it is less robust
and there is no compelling reason to use it;
it is merely provided for backward compatibility
and it may be removed in future versions.

If the detection mechanism is to be used,
it is mandatory to correctly specify
the filename of the main file as the argument of |\childdocmain|:
%
\begin{center}
\begin{tabular}{l}
|% \iffalse
%
% childdoc.dtx Copyright (C) 2017-2018 Niklas Beisert
%
% This work may be distributed and/or modified under the
% conditions of the LaTeX Project Public License, either version 1.3
% of this license or (at your option) any later version.
% The latest version of this license is in
%   http://www.latex-project.org/lppl.txt
% and version 1.3 or later is part of all distributions of LaTeX
% version 2005/12/01 or later.
%
% This work has the LPPL maintenance status `maintained'.
%
% The Current Maintainer of this work is Niklas Beisert.
%
% This work consists of the files childdoc.dtx and childdoc.ins
% and the derived files childdoc.def and cdocsamp.tex with
% cdocsch1.tex, cdocsch2.tex, cdocsdrf.tex, cdocsfn1.tex, cdocsfn2.tex.
%
%<package>\ifdefined\childdocmain\endinput\fi
%<package>\ProvidesFile{childdoc.def}[2018/12/30 v2.0 child document driver]
%<samplemain>\ProvidesFile{cdocsamp.tex}[2018/12/30 v2.0 sample for childdoc]
%<*driver>
%\ProvidesFile{childdoc.drv}[2018/12/30 v2.0 childdoc reference manual file]
\PassOptionsToClass{10pt,a4paper}{article}
\documentclass{ltxdoc}

\usepackage[margin=35mm]{geometry}
\usepackage{hyperref}
\usepackage{hyperxmp}
\usepackage[usenames]{color}

\hypersetup{colorlinks=true}
\hypersetup{pdfstartview=FitH}
\hypersetup{pdfpagemode=UseNone}
\hypersetup{pdfsource={}}
\hypersetup{pdflang={en-UK}}
\hypersetup{pdfcopyright={Copyright 2017-2018 Niklas Beisert.
  This work may be distributed and/or modified under the
  conditions of the LaTeX Project Public License, either version 1.3
  of this license or (at your option) any later version.}}
\hypersetup{pdflicenseurl={http://www.latex-project.org/lppl.txt}}
\hypersetup{pdfcontactaddress={ETH Zurich, ITP, HIT K,
  Wolfgang-Pauli-Strasse 27}}
\hypersetup{pdfcontactpostcode={8093}}
\hypersetup{pdfcontactcity={Zurich}}
\hypersetup{pdfcontactcountry={Switzerland}}
\hypersetup{pdfcontactemail={nbeisert@itp.phys.ethz.ch}}
\hypersetup{pdfcontacturl={http://people.phys.ethz.ch/\xmptilde nbeisert/}}

\newcommand{\secref}[1]{\hyperref[#1]{section \ref*{#1}}}

\parskip1ex
\parindent0pt
\let\olditemize\itemize
\def\itemize{\olditemize\parskip0pt}

\begin{document}

\title{The \textsf{childdoc} Package}
\hypersetup{pdftitle={The childdoc Package}}
\author{Niklas Beisert\\[2ex]
  Institut f\"ur Theoretische Physik\\
  Eidgen\"ossische Technische Hochschule Z\"urich\\
  Wolfgang-Pauli-Strasse 27, 8093 Z\"urich, Switzerland\\[1ex]
  \href{mailto:nbeisert@itp.phys.ethz.ch}
  {\texttt{nbeisert@itp.phys.ethz.ch}}}
\hypersetup{pdfauthor={Niklas Beisert}}
\hypersetup{pdfsubject={Manual for the LaTeX2e Package childdoc}}
\date{30 December 2018, \textsf{v2.0}}
\maketitle

\begin{abstract}\noindent
\textsf{childdoc} is a \LaTeXe{} package
that enables the direct compilation
of document sections included by |\include|
to individual files.
\end{abstract}

\begingroup
\parskip0ex
\tableofcontents
\endgroup

%%%%%%%%%%%%%%%%%%%%%%%%%%%%%%%%%%%%%%%%%%%%%%%%%%%%%%%%%%%%%%%%%%%%%%%%%%%%%%%%
%%%%%%%%%%%%%%%%%%%%%%%%%%%%%%%%%%%%%%%%%%%%%%%%%%%%%%%%%%%%%%%%%%%%%%%%%%%%%%%%
\section{Introduction}

\LaTeX{} provides a mechanism to structure a large document (such as a book)
into a main file and several child files (containing the chapters)
using the |\include| command.
This mechanism is beneficial for documents
which span hundreds of pages in order to
make the source file(s) more manageable.
Moreover, compilation can be restricted to
selected child files by means of the |\includeonly| command.
The latter feature can be used to reduce the compilation time while editing
(this was significantly more useful in the earlier days of \LaTeX{})
or to generate a smaller document which is easier to navigate.
Another application of |\includeonly| is to generate
documents consisting of selected parts of the complete document.

However, there are a few drawbacks of the plain |\include| mechanism:
\begin{itemize}
\item
The child files cannot be compiled on their own,
they can only be compiled via the main file.
A naive editing environment
(such as a text editor with an option
to have the current file processed by \LaTeX)
may require one to switch to the main file before compiling;
attempting to compile the child file produces errors.
\item
The main file must be modified (each time)
to adjust the |\includeonly| command
to the present needs. This easily leaves the main file in a messy state.
\item
The generated document will always carry the filename
of the main document. This is inconvenient if
several child files are to be compiled and
to be kept for distribution.
\end{itemize}

The present package provides a simple interface
to make child files individually compilable by \LaTeX{}.
Compiling a child file then has the same effect as compiling
the main file with an |\includeonly| command
to select the appropriate child.
Moreover the generated document will carry the name of the child
rather than the main file.
This resolves all three above issues.

This feature is meant to make the editing of books,
thesis documents and lecture notes somewhat more convenient.
However, the package can also be used efficiently for
composing a series of documents (such as exercise sheets)
which are typically distributed individually.
It then assists the author in generating the individual documents
(potentially in different versions)
as well as a document containing the collected series.
Another application is in developing style files
or other kinds of included material
where compilation of the style file could redirect
to a sample or test file.

%%%%%%%%%%%%%%%%%%%%%%%%%%%%%%%%%%%%%%%%%%%%%%%%%%%%%%%%%%%%%%%%%%%%%%%%%%%%%%%%
%%%%%%%%%%%%%%%%%%%%%%%%%%%%%%%%%%%%%%%%%%%%%%%%%%%%%%%%%%%%%%%%%%%%%%%%%%%%%%%%
\section{Usage}

First of all, the package \textsf{childdoc} is \emph{not} a standard
\LaTeXe{} |.sty| style file! Therefore it needs to be invoked in
a non-standard way.

%%%%%%%%%%%%%%%%%%%%%%%%%%%%%%%%%%%%%%%%%%%%%%%%%%%%%%%%%%%%%%%%%%%%%%%%%%%%%%%%
\subsection{Included Files}
\label{sec:include}

%%%%%%%%%%%%%%%%%%%%%%%%%%%%%%%%%%%%%%%%
\DescribeMacro{\childdocmain}
To use the package, add the commands
\begin{center}
\begin{tabular}{l}
|\input{childdoc.def}|\\
|\childdocmain{}|\\
\end{tabular}
\end{center}
at the very top of the main \LaTeX{} file,
in particular \emph{before} the |\documentclass| statement!
The argument of |\childdocmain| should be left empty
(but it must be present).

%%%%%%%%%%%%%%%%%%%%%%%%%%%%%%%%%%%%%%%%
\DescribeMacro{\childdocof}
Furthermore, add the commands
\begin{center}
\begin{tabular}{l}
|\input{childdoc.def}|\\
|\childdocof{|\textit{main}|}|\\
\end{tabular}
\end{center}
at the top of every child file \textit{child}
which is included by |\include{|\textit{child}|}|
from within the main file
(or at least for those files to be compiled individually).
The argument \textit{main} must be the filename of the main file.

There are a couple of
considerations in setting up the main and child documents:

%%%%%%%%%%%%%%%%%%%%%%%%%%%%%%%%%%%%%%%%
\paragraph{Restrictions.}

Please note the following restrictions:
\begin{itemize}
\item
|\childdocmain| must be called with one argument \textit{main}
to ensure compatibility with earlier version of the package.
It must either be empty (|\childdocmain{}|)
or precisely match the filename of the main file in which it is specified.
See \secref{sec:detection} for further information.
\item
The filename \textit{main} must be specified without the |.tex| extension.
\item
The filename \textit{main} is case sensitive
(even in case-insensitive file systems)
due to internal string comparison.
\item
The argument \textit{main} should be fully expanded, it cannot be a macro.
\item
Subdirectories and special characters should be avoided in filenames.
\item
The command |\childdocmain{|\textit{main}|}| must be followed by a whitespace.
It should not be followed immediately by another command
or by a comment mark `|%|'.
This is because the \TeX{} parser reads the token immediately following
the argument of |\childdocmain| and puts it
at the beginning of every child section;
however, a white\-space is ignored.
\end{itemize}

%%%%%%%%%%%%%%%%%%%%%%%%%%%%%%%%%%%%%%%%
\paragraph{Content of Main File.}

It is advisable to place all content in the child files included by |\include|.
Any output contained in the main file will appear in all child documents
unless suppressed manually;
it cannot be suppressed automatically by the |\includeonly| directive
and thus should normally be avoided.
A method to include some content in the main file
by means of conditional processing is described in \secref{sec:conditional}.

%%%%%%%%%%%%%%%%%%%%%%%%%%%%%%%%%%%%%%%%
\paragraph{Page Numbering.}

When only a part of the document is compiled,
the appropriate numbering of pages
(as well as other status parameters)
is determined from the |.aux| files.
The latter contain information from previous passes.
However this information needs to propagate through
all intermediate child documents.
Therefore the page numbering in child documents may well
be inconsistent until the complete document is compiled at least once.

A useful (if unconventional) way to always ensure a consistent
page numbering is to restart the numbering in each child document
and denote the pages by `\textit{child}|.|\textit{page}'
where \textit{child} represents the chapter/section number of the child file.
This can be achieved by the command
|\numberwithin{page}{|\textit{child}|}|
of the \textsf{amsmath} package
where \textit{child} can be |chapter| or |section|
depending on the chosen structuring.
Alternatively, one can modify the macro |\thepage| appropriately
and reset the counter |page| at the start of each child file.

%%%%%%%%%%%%%%%%%%%%%%%%%%%%%%%%%%%%%%%%%%%%%%%%%%%%%%%%%%%%%%%%%%%%%%%%%%%%%%%%
\subsection{Conditional Processing}
\label{sec:conditional}

The package provides a mechanism to compile different versions
of a document. To customise the versions further some conditional processing
can come in handy to distinguish which version is being compiled.
The package provides two macros to describe the compilation context:

%%%%%%%%%%%%%%%%%%%%%%%%%%%%%%%%%%%%%%%%
\DescribeMacro{\ifchilddoc}
The conditional |\ifchilddoc| distinguishes between the compilation of
child documents and the main document:
%
\begin{center}
|\ifchilddoc |\textit{child-code}| |[|\||else |\textit{main-code}]| \||fi|
\end{center}

%%%%%%%%%%%%%%%%%%%%%%%%%%%%%%%%%%%%%%%%
\DescribeMacro{\childdocname}
\DescribeMacro{\childdocjob}
The macro |\childdocname| contains the filename (without extension)
of the main or child file being processed.
Note that |\childdocjob| will always contain the name of the main file.

%%%%%%%%%%%%%%%%%%%%%%%%%%%%%%%%%%%%%%%%
\paragraph{Title Page.}

Conditional processing can be used to include a title or banner page
in the main document when proper precautions are taken.
Importantly, the code in the main file should ensure that the page counter
(as well as other status parameters which are stored in the |.aux| files)
takes the same value after the conditional processing.
Otherwise the page numbers may take divergent values
depending on which part is compiled.

For example, a title page could be declared by:
%
\begin{center}
\begin{tabular}{l}
|\ifchilddoc\||else|\\
|\addtocounter{page}{-1}|\\
\textit{code for title page}\\
|\newpage|\\
|\||fi|
\end{tabular}
\end{center}
%
A banner page for the child documents can be generated by:
%
\begin{center}
\begin{tabular}{l}
|\ifchilddoc|\\
|\addtocounter{page}{-1}|\\
\textit{code for banner page}\\
|\newpage|\\
|\||fi|
\end{tabular}
\end{center}
%
Here one could write a message such as:
\begin{center}
|This is the part \childdocname{} of \childdocjob{}.|
\end{center}

%%%%%%%%%%%%%%%%%%%%%%%%%%%%%%%%%%%%%%%%%%%%%%%%%%%%%%%%%%%%%%%%%%%%%%%%%%%%%%%%
\subsection{Flags}
\label{sec:flags}

The package makes it easy to generate different versions
of the main or child documents.
To this end compilation flags can be defined
and assigned different default values.
They will be particularly useful in conjunction
with the forwarding mechanism described in \secref{sec:forward}.

For example, it may be useful to have a flag |\version|
which can be set to |draft| or |final|.
The document source will contain some conditional code
depending on the value of |\version|.
Suppose further, the flag should default to |final| for the main file
and to |draft| for child files
which is a natural assignment for editing the document.
This is achieved by placing the following code
in the preamble of the main document
(below the |\childdocmain| directive):
%
\begin{center}
\begin{tabular}{l}
|\ifchilddoc|\\
|\providecommand{\version}{draft}|\\
|\||else|\\
|\providecommand{\version}{final}|\\
|\||fi|
\end{tabular}
\end{center}
%
The definition by |\providecommand| makes sure
that previous definitions are not overwritten.
Further statements |\providecommand{\version}{...}|
can thus be added before the above code to override it.

For the main file, one might add a line
(between |\childdocmain| and the above block)
%
\begin{center}
|%\ifchilddoc\||else\providecommand{\version}{draft}\||fi|
\end{center}
%
which can be uncommented to produce a draft version.
Likewise one can add a line to the very top of a child file
(above the |\childdocof{|\textit{main}|}| directive)
%
\begin{center}
|%\providecommand{\version}{final}|
\end{center}
%
which can be uncommented to produce the final version of this child document.

%%%%%%%%%%%%%%%%%%%%%%%%%%%%%%%%%%%%%%%%%%%%%%%%%%%%%%%%%%%%%%%%%%%%%%%%%%%%%%%%
\subsection{Forwarding}
\label{sec:forward}

Different versions of the main or child documents
using compilation flags as described in \secref{sec:flags}
can be (permanently) stored in different files
for convenient compilation, viewing and distribution.
To this end, the package defines a command
to pass on compilation to a different file:

%%%%%%%%%%%%%%%%%%%%%%%%%%%%%%%%%%%%%%%%
\DescribeMacro{\childdocforward}
The command |\childdocforward| redirects processing to
another source file:
%
\begin{center}
\begin{tabular}{l}
|\input{childdoc.def}|\\
|\childdocforward[|\textit{main}|]{|\textit{dest}|}|\\
\end{tabular}
\end{center}
%
The argument \textit{dest} is the destination file
(without extension).
It should be the main file or one of the child files.
Note that further \textsf{childdoc} directives
such as |\childdocof| and |\childdocforward|
in the indicated file will be processed in this form.
The optional argument \textit{main}
passes on directly to the main file \textit{main}
while pretending to compile the child \textit{dest}.
This form behaves as if \textit{dest}
issues |\childdocof{|\textit{main}|}| right away,
and no further \textsf{childdoc} directives will be processed.

%%%%%%%%%%%%%%%%%%%%%%%%%%%%%%%%%%%%%%%%
\DescribeMacro{\...prefix}
In the alternative form |\childdocforwardprefix|,
%
\begin{center}
\begin{tabular}{l}
|\input{childdoc.def}|\\
|\childdocforwardprefix[|\textit{main}|]{|\textit{prefix}|}{|\textit{dest}|}|
\end{tabular}
\end{center}
%
the destination file is determined by a pattern
depending on the current file:
To make this work, the current file must be called
`{\textit{prefix}\hspace{0.2em}\textit{suffix}}'
with \textit{prefix} matching precisely the argument.
Processing is then passed on to the file
`{\textit{dest}\hspace{0.2em}\textit{suffix}}'.
Surely, the same effect is achieved by
directly specifying the
argument `{\textit{dest}\hspace{0.2em}\textit{suffix}}'
in the first form.
However, that requires to set up a different file
for each child. With the alternative form of the command
all these files can have exactly the same content
which simplifies setting them up and maintaining them.

For example, the following file |draft.tex|
with a compilation flag |\version| as described in \secref{sec:flags}
compiles the main document as a draft:
%
\begin{center}
\begin{tabular}{l}
|\def\version{draft}|\\
|\input{childdoc.def}|\\
|\childdocforward{|\textit{main}|}|
\end{tabular}
\end{center}
%
Likewise, the following files |final|\textit{nn}|.tex|
compile the final version of the child document
|child|\textit{nn}|.tex|:
%
\begin{center}
\begin{tabular}{l}
|\def\version{final}|\\
|\input{childdoc.def}|\\
|\childdocforwardprefix{final}{child}|
\end{tabular}
\end{center}
%

Note that when several versions of a main file and/or of each child file
are to be generated, it may be convenient to set up a |Makefile| or
shell script to automatise the process.

%%%%%%%%%%%%%%%%%%%%%%%%%%%%%%%%%%%%%%%%%%%%%%%%%%%%%%%%%%%%%%%%%%%%%%%%%%%%%%%%
\subsection{Command Line Processing}
\label{sec:commandline}

The effect of redirection files can also be achieved by invoking
the \LaTeX{} compiler with a more elaborate command line.
Most conveniently this should be done as part
of a shell script or a |Makefile|.

When using \textsf{childdoc} in the main file, the following
command lines effectively perform a redirection
(note that depending on the shell being used,
backslashes may have to be doubled: `|\|' $\to$ `|\\|'):
%
\begin{center}
|... -jobname "|\textit{target}|" |\\|"|[\textit{flags}]%
|\input{childdoc.def}\childdocforward[|\textit{main}|]{|\textit{dest}|}"|
\end{center}
%
Here \textit{target} is the name of the output file,
\textit{main} is the name of the main file
and \textit{dest} is the name of the main or child file to be processed
(all filenames without extensions).
The optional argument \textit{main} can be omitted
if \textit{main} matches \textit{dest}.
Optionally, compilation \textit{flags} can be defined via |\def| commands.
This command line makes the \TeX{} engine believe
it is compiling the file \textit{target}
whose content is specified as the latter parameter.
The provided code then forwards the processing to
\textit{main} or \textit{dest} as described in \secref{sec:forward}.

%%%%%%%%%%%%%%%%%%%%%%%%%%%%%%%%%%%%%%%%%%%%%%%%%%%%%%%%%%%%%%%%%%%%%%%%%%%%%%%%
\subsection{Include by Input}
\label{sec:input}

Including child documents by |\include| has some restrictions by design.
Most notably, the content of a child document always occupies
its own set of pages; pages cannot be shared between child documents.
Usually, this behaviour makes perfect sense
because each child document contain an essential part of the document.
However, in some situations it may be desirable to compose
a document from a collection of parts
without having mandatory page breaks between then.
For this case, the package
provides a mechanism to include parts
by |\input| which can also be processed individually.
However, by construction this mechanism
requires manual handling of the content to be output.

%%%%%%%%%%%%%%%%%%%%%%%%%%%%%%%%%%%%%%%%
\DescribeMacro{\ifchilddocmanual}
The main file should be prepared as usual, see \secref{sec:include}.
However, the document body must make a distinction
between processing of an individual part and of the main document, e.g.:
%
\begin{center}
\begin{tabular}{l}
|\ifchilddocmanual|\\
|\input{\childdocname}|\\
|\||else|\\
\textit{document body with }|\input{|\textit{part}|}|\\
|\||fi|
\end{tabular}
\end{center}
%
The conditional |\ifchilddocmanual| is true whenever
a part to be included by |\input| is being compiled,
and the name of the part is stored in |\childdocname|.

%%%%%%%%%%%%%%%%%%%%%%%%%%%%%%%%%%%%%%%%
\DescribeMacro{\childdocby}
Each part to be included by |\input| should start with:
%
\begin{center}
\begin{tabular}{l}
|\input{childdoc.def}|\\
|\childdocby{|\textit{main}|}|\\
\end{tabular}
\end{center}
%
The directive |\childdocby| is similar to |\childdocof|
described in \secref{sec:include},
but the subsequent selection of content must be done manually.
To that end, both |\ifchilddoc| and |\ifchilddocmanual|
will be true upon processing of a part,
and the name of the part is stored in |\childdocname|.
Note that |\jobname| will be set to the filename of the current part
so that each part receives an individual |.aux| file
that does not interfere with the |.aux| file(s) of the main document.
This behaviour can be altered by the alternative form
|\childdocby[*]{|\textit{main}|}| (with a non-empty optional argument)
which uses the |.aux| file of the main document
by setting |\jobname| to \textit{main}.

%%%%%%%%%%%%%%%%%%%%%%%%%%%%%%%%%%%%%%%%%%%%%%%%%%%%%%%%%%%%%%%%%%%%%%%%%%%%%%%%
\subsection{Driver Development}
\label{sec:driver}

The \textsf{childdoc} mechanism can also be use for the development
of definition files such as \LaTeX{} styles or classes.
This case differs from the above setup with multiple parts
included by |\include| in that no |\includeonly| should be invoked.
This can be achieved by starting the include file
(before |\ProvidesPackage|) with:
%
\begin{center}
\begin{tabular}{l}
|\input{childdoc.def}|\\
|\childdocforward{|\textit{main}|}|\\
\end{tabular}
\end{center}
%
or alternatively with:
%
\begin{center}
\begin{tabular}{l}
|\input{childdoc.def}|\\
|\childdocby{|\textit{main}|}|\\
\end{tabular}
\end{center}
%
Both forms have slightly different effects as described above.
The main file is prepared as usual, see \secref{sec:include}.

%%%%%%%%%%%%%%%%%%%%%%%%%%%%%%%%%%%%%%%%%%%%%%%%%%%%%%%%%%%%%%%%%%%%%%%%%%%%%%%%
\subsection{Legacy Detection}
\label{sec:detection}

The directive |\childdocmain| in the main file can detect
whether the complete document or merely a child is to be compiled
even without using the directive |\childdocof|.
This method is deprecated because it is less robust
and there is no compelling reason to use it;
it is merely provided for backward compatibility
and it may be removed in future versions.

If the detection mechanism is to be used,
it is mandatory to correctly specify
the filename of the main file as the argument of |\childdocmain|:
%
\begin{center}
\begin{tabular}{l}
|\input{childdoc.def}|\\
|\childdocmain{|\textit{main}|}|\\
\end{tabular}
\end{center}
%
If |\jobname| does not match the argument \textit{main} of |\childdocmain|,
it is assumed that |\jobname| points to the child file to be compiled.
When using |\childdocmain| with the main file specified as argument,
it suffices to start a child file
with just |\input{|\textit{main}|}|
without loading of the package and using |\childdocof|.
If instead all processing is done
with the appropriate \textsf{childdoc} directives,
the argument of \textit{main} of |\childdocmain| can be empty.

An alternative version of the command line processing described
in \secref{sec:commandline} using the detection mechanism reads:
%
\begin{center}
|... -jobname "|\textit{target}|" "|[\textit{flags}]%
[|\def\jobname{|\textit{dest}|}|]|\input{|\textit{main}|}"|
\end{center}

%%%%%%%%%%%%%%%%%%%%%%%%%%%%%%%%%%%%%%%%%%%%%%%%%%%%%%%%%%%%%%%%%%%%%%%%%%%%%%%%
\subsection{Manual Code}
\label{sec:manual}

In case one cannot be certain whether the definitions file |childdoc.def|
is installed on the target \TeX{} distribution
and one prefers not to ship it,
it is conceivable to paste a few relevant commands into the sources.

To that end, drop all statements |\input{childdoc.def}|
and perform the replacements as outlined below.
Instead of |\childdocmain{|\textit{main}|}| add the following code
to the top of the main file:
%
\begin{center}
\begin{tabular}{l}
|\||ifdefined\childdocname\endinput\||fi\newif\ifchilddoc|\\
|\edef\childdocname{\scantokens\expandafter{\jobname\noexpand}}|\\
|\def\childdocmain{|\textit{main}|}\||ifx\childdocmain\childdocname\||else|\\
|\childdoctrue\includeonly{\childdocname}\let\jobname\childdocmain\||fi|\\
\end{tabular}
\end{center}
%
Instead of |\childdocof{|\textit{main}|}| just include the main file
at the top of each child file:
%
\begin{center}
|\input{|\textit{main}|}|
\end{center}
%
A simple redirection |\childdocforward{|\textit{dest}|}| is achieved by:
%
\begin{center}
|\def\jobname{|\textit{dest}|}\input{\jobname}|
\end{center}
%
The redirection with prefix
|\childdocforwardprefix[|\textit{prefix}|]{|\textit{dest}|}|
is accomplished by:
%
\begin{center}
\begin{tabular}{l}
|{\edef\jobname{\scantokens\expandafter{\jobname\noexpand}}|\\
|\def\redirectjob |\textit{prefix}|#1~~~{\gdef\jobname{|\textit{dest}|#1}}|\\
|\expandafter\redirectjob\jobname~~~}\input{\jobname}|
\end{tabular}
\end{center}

In an alternative approach,
child documents can be compiled by a specific command line
without additional code or specific definitions:
%
\begin{center}
|... -jobname "|\textit{target}|" "|[\textit{flags}]%
|\includeonly{|\textit{dest}|}\input{|\textit{main}|}"|
\end{center}
%

%%%%%%%%%%%%%%%%%%%%%%%%%%%%%%%%%%%%%%%%%%%%%%%%%%%%%%%%%%%%%%%%%%%%%%%%%%%%%%%%
%%%%%%%%%%%%%%%%%%%%%%%%%%%%%%%%%%%%%%%%%%%%%%%%%%%%%%%%%%%%%%%%%%%%%%%%%%%%%%%%
\section{Information}

%%%%%%%%%%%%%%%%%%%%%%%%%%%%%%%%%%%%%%%%%%%%%%%%%%%%%%%%%%%%%%%%%%%%%%%%%%%%%%%%
\subsection{Copyright}

Copyright \copyright{} 2017--2018 Niklas Beisert

This work may be distributed and/or modified under the
conditions of the \LaTeX{} Project Public License, either version 1.3
of this license or (at your option) any later version.
The latest version of this license is in
  \url{http://www.latex-project.org/lppl.txt}
and version 1.3 or later is part of all distributions of \LaTeX{}
version 2005/12/01 or later.

This work has the LPPL maintenance status `maintained'.

The Current Maintainer of this work is Niklas Beisert.

This work consists of the files |README.txt|, |childdoc.ins| and |childdoc.dtx|
as well as the derived files |childdoc.def|, |cdocsamp.tex|
with |cdocsch1.tex|, |cdocsch2.tex|, |cdocspt3.tex|, |cdocspt4.tex|,
|cdocsdrf.tex|, |cdocsfn1.tex|, |cdocsfn2.tex|
as well as |childdoc.pdf|.

%%%%%%%%%%%%%%%%%%%%%%%%%%%%%%%%%%%%%%%%%%%%%%%%%%%%%%%%%%%%%%%%%%%%%%%%%%%%%%%%
\subsection{Files and Installation}

The package consists of the files:
%
\begin{center}
\begin{tabular}{ll}
    |README.txt|   & readme file \\
    |childdoc.ins| & installation file \\
    |childdoc.dtx| & source file \\
    |childdoc.def| & definition file \\
    |cdocsamp.tex| & sample main file \\
    |cdocsch1.tex| & sample include file \\
    |cdocsch2.tex| & sample include file \\
    |cdocspt3.tex| & sample part file \\
    |cdocspt4.tex| & sample part file \\
    |cdocsdrf.tex| & sample redirection file \\
    |cdocsfn1.tex| & sample redirection file \\
    |cdocsfn2.tex| & sample redirection file \\
    |childdoc.pdf| & manual
\end{tabular}
\end{center}
%
The distribution consists of the files
|README.txt|, |childdoc.ins| and |childdoc.dtx|.
%
\begin{itemize}
\item
Run (pdf)\LaTeX{} on |childdoc.dtx|
to compile the manual |childdoc.pdf| (this file).
\item
Run \LaTeX{} on |childdoc.ins| to create the definitions file |childdoc.def|
and the sample |cdocsamp.tex| with include files
|cdocsch1.tex|, |cdocsch2.tex|, |cdocspt3.tex|, |cdocspt4.tex|,
|cdocsdrf.tex|, |cdocsfn1.tex|, |cdocsfn2.tex|.
Then copy the file |childdoc.def| to an appropriate directory of your \LaTeX{}
distribution, e.g.\ \textit{texmf-root}|/tex/latex/childdoc|.
\end{itemize}

%%%%%%%%%%%%%%%%%%%%%%%%%%%%%%%%%%%%%%%%%%%%%%%%%%%%%%%%%%%%%%%%%%%%%%%%%%%%%%%%
\subsection{Related CTAN Packages}

There are several other packages which offer a similar functionality:
%
\begin{itemize}
\item
The packages
\href{http://ctan.org/pkg/docmute}{\textsf{docmute}},
\href{http://ctan.org/pkg/includex}{\textsf{includex}} and
\href{http://ctan.org/pkg/standalone}{\textsf{standalone}}
provide commands to include only the document body of
a child file thus allowing both files to be compiled individually.
\item
The packages \href{http://ctan.org/pkg/subdocs}{\textsf{subdocs}}
and \href{http://ctan.org/pkg/subfiles}{\textsf{subfiles}}
provide structures in which the main and child documents can be
encapsulated and allowing them to be compiled individually.
The inclusion mechanism is different from the conventional |\include|.
\item
The package \href{http://ctan.org/pkg/combine}{\textsf{combine}}
is an elaborate solution to combine several documents into one.
\end{itemize}
%
See also the CTAN topic \href{http://ctan.org/topic/subdocs}{\textsf{subdocs}}
for further related packages.
The present package differs from the above solutions in that
a document structure constructed with the conventional |\include| mechanism
just needs two extra commands at the top of every file
such that all constituent files can be compiled individually.

%%%%%%%%%%%%%%%%%%%%%%%%%%%%%%%%%%%%%%%%%%%%%%%%%%%%%%%%%%%%%%%%%%%%%%%%%%%%%%%%
%\subsection{Feature Suggestions}
%
%The following is a list of features which may be useful for future
%versions of this package:
%%
%\begin{itemize}
%\item
%\ldots
%\end{itemize}

%%%%%%%%%%%%%%%%%%%%%%%%%%%%%%%%%%%%%%%%%%%%%%%%%%%%%%%%%%%%%%%%%%%%%%%%%%%%%%%%
\subsection{Revision History}

%%%%%%%%%%%%%%%%%%%%%%%%%%%%%%%%%%%%%%%%
\paragraph{v2.0:} 2018/12/30

\begin{itemize}
\item
immediate forward processing
\item
added |\childdocby| mechanism
\item
manual restructured
\end{itemize}

%%%%%%%%%%%%%%%%%%%%%%%%%%%%%%%%%%%%%%%%
\paragraph{v1.6:} 2018/01/17

\begin{itemize}
\item
application for development of include files
\item
corrections to manual
\end{itemize}

%%%%%%%%%%%%%%%%%%%%%%%%%%%%%%%%%%%%%%%%
\paragraph{v1.5:} 2017/05/21

\begin{itemize}
\item
more complete structuring introduced
\item
|\childdocof| introduced
\item
|\childdoc| renamed to |\childdocmain|
\item
|\childredirect| renamed to |\childdocforward| and |\childdocforwardprefix|
and functionality expanded
\end{itemize}

%%%%%%%%%%%%%%%%%%%%%%%%%%%%%%%%%%%%%%%%
\paragraph{v1.0:} 2017/04/27

\begin{itemize}
\item
manual and install package
\item
first version published on CTAN
\end{itemize}

%%%%%%%%%%%%%%%%%%%%%%%%%%%%%%%%%%%%%%%%
\paragraph{v0.6:} 2017/04/26

\begin{itemize}
\item
redirection mechanism added
\end{itemize}

%%%%%%%%%%%%%%%%%%%%%%%%%%%%%%%%%%%%%%%%
\paragraph{v0.5:} 2017/04/26

\begin{itemize}
\item
functionality in definition file
\end{itemize}


%%%%%%%%%%%%%%%%%%%%%%%%%%%%%%%%%%%%%%%%%%%%%%%%%%%%%%%%%%%%%%%%%%%%%%%%%%%%%%%%
%%%%%%%%%%%%%%%%%%%%%%%%%%%%%%%%%%%%%%%%%%%%%%%%%%%%%%%%%%%%%%%%%%%%%%%%%%%%%%%%
%%%%%%%%%%%%%%%%%%%%%%%%%%%%%%%%%%%%%%%%%%%%%%%%%%%%%%%%%%%%%%%%%%%%%%%%%%%%%%%%
\appendix

\settowidth\MacroIndent{\rmfamily\scriptsize 000\ }

 \DocInput{childdoc.dtx}

\end{document}
%</driver>
% \fi
%
% %%%%%%%%%%%%%%%%%%%%%%%%%%%%%%%%%%%%%%%%%%%%%%%%%%%%%%%%%%%%%%%%%%%%%%%%%%%%%%
% %%%%%%%%%%%%%%%%%%%%%%%%%%%%%%%%%%%%%%%%%%%%%%%%%%%%%%%%%%%%%%%%%%%%%%%%%%%%%%
% \section{Sample}
%\iffalse
%<*samplemain>
%\fi
%
% The following presents a sample document
% with two chapters, two parts, a title page,
% a compile flag as well as three forwarding files to set the flag.
% It consists of eight |.tex| files:
% \begin{center}
% \begin{tabular}{ll}
% |cdocsamp.tex|&main file\\
% |cdocsch1.tex|&include file for chapter 1\\
% |cdocsch2.tex|&include file for chapter 2\\
% |cdocspt3.tex|&include file for part 3\\
% |cdocspt4.tex|&include file for part 4\\
% |cdocsdrf.tex|&forwarding file for main file in draft mode\\
% |cdocsfi1.tex|&forwarding file for final version of chapter 1\\
% |cdocsfi2.tex|&forwarding file for final version of chapter 2\\
% \end{tabular}
% \end{center}
% Each of the eight files can be compiled directly by the \LaTeX{} compiler.
%
% %%%%%%%%%%%%%%%%%%%%%%%%%%%%%%%%%%%%%%
% \paragraph{Main File.}
%
% The main file is called |cdocsamp.tex|.
%
% Load the \textsf{childdoc} definitions and
% declare the filename for the main document:
%    \begin{macrocode}
\input{childdoc.def}
\childdocmain{}
%    \end{macrocode}

% Optional override for |\version| flag:
%    \begin{macrocode}
%%\ifchilddoc\else\providecommand{\version}{draft}\fi
%    \end{macrocode}

% Define the default values for the |\version| flag
% (|final| for the main file and |draft| for childs):
%    \begin{macrocode}
\ifchilddoc
\providecommand{\version}{draft}
\else
\providecommand{\version}{final}
\fi
%    \end{macrocode}

% Load the standard document class:
%    \begin{macrocode}
\documentclass[12pt]{article}
%    \end{macrocode}

% Start the document body:
%    \begin{macrocode}
\begin{document}
%    \end{macrocode}

% Declare a title page.
% Print title, part of document being processed and version flag:
%    \begin{macrocode}
\addtocounter{page}{-1}
\begin{center}
{\LARGE\bfseries{}childdoc example\par}
\vspace{1cm}
\ifchilddoc
\ifchilddocmanual part\else chapter\fi:
`\childdocname' of `\childdocjob'\par
\else
main document: `\childdocjob'\par
\fi
version: \version\par
\end{center}
\newpage
%    \end{macrocode}

% Manually include selected file,
% otherwise process as usual:
%    \begin{macrocode}
\ifchilddocmanual
\section*{part `\childdocname'}
\input{\childdocname}
\else
%    \end{macrocode}

% Include the two chapters:
%    \begin{macrocode}
\include{cdocsch1}
\include{cdocsch2}
%    \end{macrocode}

% Include the two parts unless only chapters should be displayed:
%    \begin{macrocode}
\ifchilddoc\else
\section{part three}
\input{cdocspt3}
\section{part four}
\input{cdocspt4}
\fi
%    \end{macrocode}

% Process as usual until here:
%    \begin{macrocode}
\fi
%    \end{macrocode}

% End of document body:
%    \begin{macrocode}
\end{document}
%    \end{macrocode}
%\iffalse
%</samplemain>
%\fi
%
% %%%%%%%%%%%%%%%%%%%%%%%%%%%%%%%%%%%%%%
% \paragraph{Chapter Include Files.}
%
% The include files are called |cdocsch1.tex| and |cdocsch2.tex|.
%
%\iffalse
%<*samplechap1|samplechap2>
%\fi

% Optional override for |\version| flag:
%    \begin{macrocode}
%%\providecommand{\version}{final}
%    \end{macrocode}

% Include the main document:
%    \begin{macrocode}
\input{childdoc.def}
\childdocof{cdocsamp}
%    \end{macrocode}

%\iffalse
%</samplechap1|samplechap2>
%\fi
%
%\iffalse
%<*samplechap1>
%\fi
% Some text for chapter 1:
%    \begin{macrocode}
\section{one}
some text in chapter one
%    \end{macrocode}

%\iffalse
%</samplechap1>
%\fi
% Some text for chapter 2:
%\iffalse
%<*samplechap2>
%\fi
%    \begin{macrocode}
\section{two}
more text in chapter two
%    \end{macrocode}

%\iffalse
%</samplechap2>
%\fi
%
% %%%%%%%%%%%%%%%%%%%%%%%%%%%%%%%%%%%%%%
% \paragraph{Part Include Files.}
%
% The include files are called |cdocspt3.tex| and |cdocspt4.tex|.
%
%\iffalse
%<*samplepart3|samplepart4>
%\fi

% Optional override for |\version| flag:
%    \begin{macrocode}
%%\providecommand{\version}{final}
%    \end{macrocode}

% Include the main document:
%    \begin{macrocode}
\input{childdoc.def}
\childdocby{cdocsamp}
%    \end{macrocode}

%\iffalse
%</samplepart3|samplepart4>
%\fi
%
%\iffalse
%<*samplepart3>
%\fi
% Some text for part 3:
%    \begin{macrocode}
some text in part three
%    \end{macrocode}

%\iffalse
%</samplepart3>
%\fi
% Some text for part 4:
%\iffalse
%<*samplepart4>
%\fi
%    \begin{macrocode}
more text in part four
%    \end{macrocode}

%\iffalse
%</samplepart4>
%\fi
%
% %%%%%%%%%%%%%%%%%%%%%%%%%%%%%%%%%%%%%%
% \paragraph{Forwarding for a Complete Draft.}
%
% The following forwarding file |cdocsdrf.tex|
% compiles the main document in draft mode:
%\iffalse
%<*sampledraft>
%\fi
%    \begin{macrocode}
\def\version{draft}
\input{childdoc.def}
\childdocforward{cdocsamp}
%    \end{macrocode}

%\iffalse
%</sampledraft>
%\fi
%
% %%%%%%%%%%%%%%%%%%%%%%%%%%%%%%%%%%%%%%
% \paragraph{Forwarding for Final Version of the Chapters.}
%
% The following forwarding files |cdocsfn1.tex| and |cdocsfn2.tex|
% (with identical content)
% compile the final versions of the child documents
% |cdocsch1.tex| and |cdocsch2.tex|, respectively:
%\iffalse
%<*samplefinal>
%\fi
%    \begin{macrocode}
\def\version{final}
\input{childdoc.def}
\childdocforwardprefix[cdocsamp]{cdocsfn}{cdocsch}
%    \end{macrocode}

%\iffalse
%</samplefinal>
%\fi
%
% %%%%%%%%%%%%%%%%%%%%%%%%%%%%%%%%%%%%%%
% \paragraph{Command Line Processing.}
%
% The following three command lines generate the output files
% |cdocscld|, |cdocscl1| and |cdocscl2|
% which should be identical to
% |cdocsdrf|, |cdocsch1| and |cdocsfn2|, respectively:
% \begin{center}
% \begin{tabular}{l}
% |latex -jobname cdocscld \|\\
% |  "\def\version{draft}\input{childdoc.def}\childdocforward{cdocsamp}"|\\
% |latex -jobname cdocscl1 \|\\
% |  "\input{childdoc.def}\childdocforward[cdocsamp]{cdocsch1}"|\\
% |latex -jobname cdocscl2 \|\\
% |  "\def\version{final}\input{childdoc.def}\childdocforward{cdocsch2}"|
% \end{tabular}
% \end{center}
% Note that the trailing backslash on each first line
% merely continues the input to the second line
% (for convenient cut ant paste).
% Furthermore, the command |latex| can be replaced by any
% of its alternative versions such as |pdflatex|.
%
% %%%%%%%%%%%%%%%%%%%%%%%%%%%%%%%%%%%%%%%%%%%%%%%%%%%%%%%%%%%%%%%%%%%%%%%%%%%%%%
% %%%%%%%%%%%%%%%%%%%%%%%%%%%%%%%%%%%%%%%%%%%%%%%%%%%%%%%%%%%%%%%%%%%%%%%%%%%%%%
% \section{Implementation}
%\iffalse
%<*package>
%\fi
%
% This section describes the definitions file |childdoc.def|.

% The definitions cannot be loaded using |\usepackage| or |\RequirePackage|
% which has a mechanism to prevent loading a style file more than once.
% When loading the definitions by means of |\input|
% multiple instances have to be prevented manually:
%\iffalse
%This code needs to be before the `\ProvidesFile' directive
%which is defined at the beginning of this file.
%Therefore it is also placed there and commented out here.
%</package>
%<*discard>
%\fi
%    \begin{macrocode}
\ifdefined\childdocmain\endinput\fi
%    \end{macrocode}
%\iffalse
%</discard>
%<*package>
%\fi
%
% \macro{\ifchilddoc}
% \macro{\ifchilddocmanual}
% The conditional |\ifchilddoc| tells whether a
% child (true) or main (false) document is being compiled.
% The conditional |\ifchilddocmanual| tells whether
% the |\includeonly| mechanism is used (false) or
% the selection of child files must be performed manually (true).
% The definitions initialise to false:
%    \begin{macrocode}
\newif\ifchilddoc
\newif\ifchilddocmanual
%    \end{macrocode}

% \macro{\childdocname}
% \macro{\childdocjob}
% The macro |\childdocname| stores the name of the main document
% to be compiled. The macro |\childdocjob| stores the name of
% the document on which the \LaTeX{} compiler was originally invoked.
% The content of |\jobname| cannot be compared
% to filenames specified in the source due to different catcodes.
% The following code rescans |\jobname|, stores the result
% in |\childdocname| and saves a copy in |\childdocjob|:
%    \begin{macrocode}
\edef\childdocname{\scantokens\expandafter{\jobname\noexpand}}
\let\childdocjob\childdocname
%    \end{macrocode}

% \macro{\childdocdisable}
% The macro |\childdocdisable| prevents the main file
% from being processed more than once.
% At this stage, the main document command |\childdocmain|
% is assumed to be called once again where it should do nothing.
% Any subsequent call to it should prevent
% a secondary processing of the main document
% It overwrites the forwarding commands
% |\childdocof| and |\childdocforward|
% with empty macros to prevent further inclusions of the main document:
%    \begin{macrocode}
\newcommand{\childdocdisable}
{
  \renewcommand{\childdocmain}[1]{\renewcommand{\childdocmain}[1]{\endinput}}
  \renewcommand{\childdocof}[1]{}
  \renewcommand{\childdocby}[2][]{}
  \renewcommand{\childdocforward}[2][]{}
  \renewcommand{\childdocdisable}{}
}
%    \end{macrocode}

% \macro{\childdocmain}
% The macro |\childdocmain| is to be called at the top of the main file
% with nothing or the main filename (without extension) as argument.
% First, it breaks loops.
% If the argument is not empty and does not match |\childdocname|
% (which is set by the first inclusion of |childdoc.def|),
% |\ifchilddoc| is set to true, |\includeonly| is applied to the child file
% and |\jobname| is set to the main file
% (for proper handling of |.aux| files):
%    \begin{macrocode}
\newcommand{\childdocmain}[1]
{
  \childdocdisable\childdocmain{}
  \if?#1?\else
    \begingroup
      \def\childdoctmp{#1}
      \ifx\childdoctmp\childdocname
        \def\childdoctmp{}
      \else
        \def\childdoctmp
        {
          \childdoctrue
          \includeonly{\childdocname}
          \def\childdocjob{#1}
          \def\jobname{#1}
        }
      \fi
      \expandafter
    \endgroup
    \childdoctmp
  \fi
}
%    \end{macrocode}

% \macro{\childdocof}
% The command |\childdocof| redirects
% compilation to the main file |#1|.
%    \begin{macrocode}
\newcommand{\childdocof}[1]
{
  \childdocdisable
  \childdoctrue
  \includeonly{\childdocname}
  \def\jobname{#1}
  \def\childdocjob{#1}
  \input{#1}
}
%    \end{macrocode}

% \macro{\childdocby}
% The command |\childdocby| ....
%    \begin{macrocode}
\newcommand{\childdocby}[2][]
{
  \childdocdisable
  \childdoctrue
  \childdocmanualtrue
  \if?#1?\else
    \def\jobname{#2}
  \fi
  \def\childdocjob{#2}
  \input{#2}
  \endinput
}
%    \end{macrocode}

% \macro{\childdocforward}
% The command |\childdocforward| redirects
% compilation to the main file or
% (if the optional argument is given) a child file.
% Parameters are set as if the main file
% or a child file starting with |\childdocof| was compiled.
% Then compilation is handed over to the main file:
%    \begin{macrocode}
\newcommand{\childdocforward}[2][]
{
  \begingroup
    \if?#1?
      \def\childdoctmp
      {
        \def\childdocname{#2}
        \def\childdocjob{#2}
        \def\jobname{#2}
        \input{#2}
        \endinput
      }
    \else
      \def\childdoctmp
      {
        \childdocdisable
        \def\childdocname{#2}
        \childdoctrue
        \includeonly{#2}
        \def\childdocjob{#1}
        \def\jobname{#1}
        \input{#1}
        \endinput
      }
    \fi
    \expandafter
  \endgroup
  \childdoctmp
}
%    \end{macrocode}

% \macro{\childdocforwardprefix}
% The command |\childdocforwardprefix| redirects
% compilation to the main or a child file by means of a pattern.
% The prefix |#1| in the current filename is replaced by |#2|
% and the suffix of the current filename is kept
% (it is assumed that the filename does not contain the substring `|~~~|'
% which is used as a delimiter).
% Compilation is handed over to the new file by |\childdocforward|:
%    \begin{macrocode}
\newcommand{\childdocforwardprefix}[3][]
{
  \begingroup
    \def\childdocextract #2##1~~~{\def\childdoctmp{\childdocforward[#1]{#3##1}}}
    \expandafter\childdocextract\childdocname~~~
    \expandafter
  \endgroup
  \childdoctmp
}
%    \end{macrocode}

% \macro{\childdoc}
% The deprecated macro |\childdoc| is a legacy version of |\childdocmain|:
%    \begin{macrocode}
\newcommand{\childdoc}{\childdocmain}
%    \end{macrocode}

% \macro{\childdocredirect}
% The deprecated macro |\childdocredirect| is a legacy version
% of |\childdocforward| and |\childdocforwardprefix|:
%    \begin{macrocode}
\newcommand{\childdocredirect}[2][]
{
  \begingroup
    \if?#1?
      \def\childdoctmp{\childdocforward{#2}}
    \else
      \def\childdoctmp{\childdocforwardprefix{#1}{#2}}
    \fi
    \expandafter
  \endgroup
  \childdoctmp
}
%    \end{macrocode}

%\iffalse
%</package>
%\fi
%
\endinput
|\\
|\childdocmain{|\textit{main}|}|\\
\end{tabular}
\end{center}
%
If |\jobname| does not match the argument \textit{main} of |\childdocmain|,
it is assumed that |\jobname| points to the child file to be compiled.
When using |\childdocmain| with the main file specified as argument,
it suffices to start a child file
with just |\input{|\textit{main}|}|
without loading of the package and using |\childdocof|.
If instead all processing is done
with the appropriate \textsf{childdoc} directives,
the argument of \textit{main} of |\childdocmain| can be empty.

An alternative version of the command line processing described
in \secref{sec:commandline} using the detection mechanism reads:
%
\begin{center}
|... -jobname "|\textit{target}|" "|[\textit{flags}]%
[|\def\jobname{|\textit{dest}|}|]|\input{|\textit{main}|}"|
\end{center}

%%%%%%%%%%%%%%%%%%%%%%%%%%%%%%%%%%%%%%%%%%%%%%%%%%%%%%%%%%%%%%%%%%%%%%%%%%%%%%%%
\subsection{Manual Code}
\label{sec:manual}

In case one cannot be certain whether the definitions file |childdoc.def|
is installed on the target \TeX{} distribution
and one prefers not to ship it,
it is conceivable to paste a few relevant commands into the sources.

To that end, drop all statements |% \iffalse
%
% childdoc.dtx Copyright (C) 2017-2018 Niklas Beisert
%
% This work may be distributed and/or modified under the
% conditions of the LaTeX Project Public License, either version 1.3
% of this license or (at your option) any later version.
% The latest version of this license is in
%   http://www.latex-project.org/lppl.txt
% and version 1.3 or later is part of all distributions of LaTeX
% version 2005/12/01 or later.
%
% This work has the LPPL maintenance status `maintained'.
%
% The Current Maintainer of this work is Niklas Beisert.
%
% This work consists of the files childdoc.dtx and childdoc.ins
% and the derived files childdoc.def and cdocsamp.tex with
% cdocsch1.tex, cdocsch2.tex, cdocsdrf.tex, cdocsfn1.tex, cdocsfn2.tex.
%
%<package>\ifdefined\childdocmain\endinput\fi
%<package>\ProvidesFile{childdoc.def}[2018/12/30 v2.0 child document driver]
%<samplemain>\ProvidesFile{cdocsamp.tex}[2018/12/30 v2.0 sample for childdoc]
%<*driver>
%\ProvidesFile{childdoc.drv}[2018/12/30 v2.0 childdoc reference manual file]
\PassOptionsToClass{10pt,a4paper}{article}
\documentclass{ltxdoc}

\usepackage[margin=35mm]{geometry}
\usepackage{hyperref}
\usepackage{hyperxmp}
\usepackage[usenames]{color}

\hypersetup{colorlinks=true}
\hypersetup{pdfstartview=FitH}
\hypersetup{pdfpagemode=UseNone}
\hypersetup{pdfsource={}}
\hypersetup{pdflang={en-UK}}
\hypersetup{pdfcopyright={Copyright 2017-2018 Niklas Beisert.
  This work may be distributed and/or modified under the
  conditions of the LaTeX Project Public License, either version 1.3
  of this license or (at your option) any later version.}}
\hypersetup{pdflicenseurl={http://www.latex-project.org/lppl.txt}}
\hypersetup{pdfcontactaddress={ETH Zurich, ITP, HIT K,
  Wolfgang-Pauli-Strasse 27}}
\hypersetup{pdfcontactpostcode={8093}}
\hypersetup{pdfcontactcity={Zurich}}
\hypersetup{pdfcontactcountry={Switzerland}}
\hypersetup{pdfcontactemail={nbeisert@itp.phys.ethz.ch}}
\hypersetup{pdfcontacturl={http://people.phys.ethz.ch/\xmptilde nbeisert/}}

\newcommand{\secref}[1]{\hyperref[#1]{section \ref*{#1}}}

\parskip1ex
\parindent0pt
\let\olditemize\itemize
\def\itemize{\olditemize\parskip0pt}

\begin{document}

\title{The \textsf{childdoc} Package}
\hypersetup{pdftitle={The childdoc Package}}
\author{Niklas Beisert\\[2ex]
  Institut f\"ur Theoretische Physik\\
  Eidgen\"ossische Technische Hochschule Z\"urich\\
  Wolfgang-Pauli-Strasse 27, 8093 Z\"urich, Switzerland\\[1ex]
  \href{mailto:nbeisert@itp.phys.ethz.ch}
  {\texttt{nbeisert@itp.phys.ethz.ch}}}
\hypersetup{pdfauthor={Niklas Beisert}}
\hypersetup{pdfsubject={Manual for the LaTeX2e Package childdoc}}
\date{30 December 2018, \textsf{v2.0}}
\maketitle

\begin{abstract}\noindent
\textsf{childdoc} is a \LaTeXe{} package
that enables the direct compilation
of document sections included by |\include|
to individual files.
\end{abstract}

\begingroup
\parskip0ex
\tableofcontents
\endgroup

%%%%%%%%%%%%%%%%%%%%%%%%%%%%%%%%%%%%%%%%%%%%%%%%%%%%%%%%%%%%%%%%%%%%%%%%%%%%%%%%
%%%%%%%%%%%%%%%%%%%%%%%%%%%%%%%%%%%%%%%%%%%%%%%%%%%%%%%%%%%%%%%%%%%%%%%%%%%%%%%%
\section{Introduction}

\LaTeX{} provides a mechanism to structure a large document (such as a book)
into a main file and several child files (containing the chapters)
using the |\include| command.
This mechanism is beneficial for documents
which span hundreds of pages in order to
make the source file(s) more manageable.
Moreover, compilation can be restricted to
selected child files by means of the |\includeonly| command.
The latter feature can be used to reduce the compilation time while editing
(this was significantly more useful in the earlier days of \LaTeX{})
or to generate a smaller document which is easier to navigate.
Another application of |\includeonly| is to generate
documents consisting of selected parts of the complete document.

However, there are a few drawbacks of the plain |\include| mechanism:
\begin{itemize}
\item
The child files cannot be compiled on their own,
they can only be compiled via the main file.
A naive editing environment
(such as a text editor with an option
to have the current file processed by \LaTeX)
may require one to switch to the main file before compiling;
attempting to compile the child file produces errors.
\item
The main file must be modified (each time)
to adjust the |\includeonly| command
to the present needs. This easily leaves the main file in a messy state.
\item
The generated document will always carry the filename
of the main document. This is inconvenient if
several child files are to be compiled and
to be kept for distribution.
\end{itemize}

The present package provides a simple interface
to make child files individually compilable by \LaTeX{}.
Compiling a child file then has the same effect as compiling
the main file with an |\includeonly| command
to select the appropriate child.
Moreover the generated document will carry the name of the child
rather than the main file.
This resolves all three above issues.

This feature is meant to make the editing of books,
thesis documents and lecture notes somewhat more convenient.
However, the package can also be used efficiently for
composing a series of documents (such as exercise sheets)
which are typically distributed individually.
It then assists the author in generating the individual documents
(potentially in different versions)
as well as a document containing the collected series.
Another application is in developing style files
or other kinds of included material
where compilation of the style file could redirect
to a sample or test file.

%%%%%%%%%%%%%%%%%%%%%%%%%%%%%%%%%%%%%%%%%%%%%%%%%%%%%%%%%%%%%%%%%%%%%%%%%%%%%%%%
%%%%%%%%%%%%%%%%%%%%%%%%%%%%%%%%%%%%%%%%%%%%%%%%%%%%%%%%%%%%%%%%%%%%%%%%%%%%%%%%
\section{Usage}

First of all, the package \textsf{childdoc} is \emph{not} a standard
\LaTeXe{} |.sty| style file! Therefore it needs to be invoked in
a non-standard way.

%%%%%%%%%%%%%%%%%%%%%%%%%%%%%%%%%%%%%%%%%%%%%%%%%%%%%%%%%%%%%%%%%%%%%%%%%%%%%%%%
\subsection{Included Files}
\label{sec:include}

%%%%%%%%%%%%%%%%%%%%%%%%%%%%%%%%%%%%%%%%
\DescribeMacro{\childdocmain}
To use the package, add the commands
\begin{center}
\begin{tabular}{l}
|\input{childdoc.def}|\\
|\childdocmain{}|\\
\end{tabular}
\end{center}
at the very top of the main \LaTeX{} file,
in particular \emph{before} the |\documentclass| statement!
The argument of |\childdocmain| should be left empty
(but it must be present).

%%%%%%%%%%%%%%%%%%%%%%%%%%%%%%%%%%%%%%%%
\DescribeMacro{\childdocof}
Furthermore, add the commands
\begin{center}
\begin{tabular}{l}
|\input{childdoc.def}|\\
|\childdocof{|\textit{main}|}|\\
\end{tabular}
\end{center}
at the top of every child file \textit{child}
which is included by |\include{|\textit{child}|}|
from within the main file
(or at least for those files to be compiled individually).
The argument \textit{main} must be the filename of the main file.

There are a couple of
considerations in setting up the main and child documents:

%%%%%%%%%%%%%%%%%%%%%%%%%%%%%%%%%%%%%%%%
\paragraph{Restrictions.}

Please note the following restrictions:
\begin{itemize}
\item
|\childdocmain| must be called with one argument \textit{main}
to ensure compatibility with earlier version of the package.
It must either be empty (|\childdocmain{}|)
or precisely match the filename of the main file in which it is specified.
See \secref{sec:detection} for further information.
\item
The filename \textit{main} must be specified without the |.tex| extension.
\item
The filename \textit{main} is case sensitive
(even in case-insensitive file systems)
due to internal string comparison.
\item
The argument \textit{main} should be fully expanded, it cannot be a macro.
\item
Subdirectories and special characters should be avoided in filenames.
\item
The command |\childdocmain{|\textit{main}|}| must be followed by a whitespace.
It should not be followed immediately by another command
or by a comment mark `|%|'.
This is because the \TeX{} parser reads the token immediately following
the argument of |\childdocmain| and puts it
at the beginning of every child section;
however, a white\-space is ignored.
\end{itemize}

%%%%%%%%%%%%%%%%%%%%%%%%%%%%%%%%%%%%%%%%
\paragraph{Content of Main File.}

It is advisable to place all content in the child files included by |\include|.
Any output contained in the main file will appear in all child documents
unless suppressed manually;
it cannot be suppressed automatically by the |\includeonly| directive
and thus should normally be avoided.
A method to include some content in the main file
by means of conditional processing is described in \secref{sec:conditional}.

%%%%%%%%%%%%%%%%%%%%%%%%%%%%%%%%%%%%%%%%
\paragraph{Page Numbering.}

When only a part of the document is compiled,
the appropriate numbering of pages
(as well as other status parameters)
is determined from the |.aux| files.
The latter contain information from previous passes.
However this information needs to propagate through
all intermediate child documents.
Therefore the page numbering in child documents may well
be inconsistent until the complete document is compiled at least once.

A useful (if unconventional) way to always ensure a consistent
page numbering is to restart the numbering in each child document
and denote the pages by `\textit{child}|.|\textit{page}'
where \textit{child} represents the chapter/section number of the child file.
This can be achieved by the command
|\numberwithin{page}{|\textit{child}|}|
of the \textsf{amsmath} package
where \textit{child} can be |chapter| or |section|
depending on the chosen structuring.
Alternatively, one can modify the macro |\thepage| appropriately
and reset the counter |page| at the start of each child file.

%%%%%%%%%%%%%%%%%%%%%%%%%%%%%%%%%%%%%%%%%%%%%%%%%%%%%%%%%%%%%%%%%%%%%%%%%%%%%%%%
\subsection{Conditional Processing}
\label{sec:conditional}

The package provides a mechanism to compile different versions
of a document. To customise the versions further some conditional processing
can come in handy to distinguish which version is being compiled.
The package provides two macros to describe the compilation context:

%%%%%%%%%%%%%%%%%%%%%%%%%%%%%%%%%%%%%%%%
\DescribeMacro{\ifchilddoc}
The conditional |\ifchilddoc| distinguishes between the compilation of
child documents and the main document:
%
\begin{center}
|\ifchilddoc |\textit{child-code}| |[|\||else |\textit{main-code}]| \||fi|
\end{center}

%%%%%%%%%%%%%%%%%%%%%%%%%%%%%%%%%%%%%%%%
\DescribeMacro{\childdocname}
\DescribeMacro{\childdocjob}
The macro |\childdocname| contains the filename (without extension)
of the main or child file being processed.
Note that |\childdocjob| will always contain the name of the main file.

%%%%%%%%%%%%%%%%%%%%%%%%%%%%%%%%%%%%%%%%
\paragraph{Title Page.}

Conditional processing can be used to include a title or banner page
in the main document when proper precautions are taken.
Importantly, the code in the main file should ensure that the page counter
(as well as other status parameters which are stored in the |.aux| files)
takes the same value after the conditional processing.
Otherwise the page numbers may take divergent values
depending on which part is compiled.

For example, a title page could be declared by:
%
\begin{center}
\begin{tabular}{l}
|\ifchilddoc\||else|\\
|\addtocounter{page}{-1}|\\
\textit{code for title page}\\
|\newpage|\\
|\||fi|
\end{tabular}
\end{center}
%
A banner page for the child documents can be generated by:
%
\begin{center}
\begin{tabular}{l}
|\ifchilddoc|\\
|\addtocounter{page}{-1}|\\
\textit{code for banner page}\\
|\newpage|\\
|\||fi|
\end{tabular}
\end{center}
%
Here one could write a message such as:
\begin{center}
|This is the part \childdocname{} of \childdocjob{}.|
\end{center}

%%%%%%%%%%%%%%%%%%%%%%%%%%%%%%%%%%%%%%%%%%%%%%%%%%%%%%%%%%%%%%%%%%%%%%%%%%%%%%%%
\subsection{Flags}
\label{sec:flags}

The package makes it easy to generate different versions
of the main or child documents.
To this end compilation flags can be defined
and assigned different default values.
They will be particularly useful in conjunction
with the forwarding mechanism described in \secref{sec:forward}.

For example, it may be useful to have a flag |\version|
which can be set to |draft| or |final|.
The document source will contain some conditional code
depending on the value of |\version|.
Suppose further, the flag should default to |final| for the main file
and to |draft| for child files
which is a natural assignment for editing the document.
This is achieved by placing the following code
in the preamble of the main document
(below the |\childdocmain| directive):
%
\begin{center}
\begin{tabular}{l}
|\ifchilddoc|\\
|\providecommand{\version}{draft}|\\
|\||else|\\
|\providecommand{\version}{final}|\\
|\||fi|
\end{tabular}
\end{center}
%
The definition by |\providecommand| makes sure
that previous definitions are not overwritten.
Further statements |\providecommand{\version}{...}|
can thus be added before the above code to override it.

For the main file, one might add a line
(between |\childdocmain| and the above block)
%
\begin{center}
|%\ifchilddoc\||else\providecommand{\version}{draft}\||fi|
\end{center}
%
which can be uncommented to produce a draft version.
Likewise one can add a line to the very top of a child file
(above the |\childdocof{|\textit{main}|}| directive)
%
\begin{center}
|%\providecommand{\version}{final}|
\end{center}
%
which can be uncommented to produce the final version of this child document.

%%%%%%%%%%%%%%%%%%%%%%%%%%%%%%%%%%%%%%%%%%%%%%%%%%%%%%%%%%%%%%%%%%%%%%%%%%%%%%%%
\subsection{Forwarding}
\label{sec:forward}

Different versions of the main or child documents
using compilation flags as described in \secref{sec:flags}
can be (permanently) stored in different files
for convenient compilation, viewing and distribution.
To this end, the package defines a command
to pass on compilation to a different file:

%%%%%%%%%%%%%%%%%%%%%%%%%%%%%%%%%%%%%%%%
\DescribeMacro{\childdocforward}
The command |\childdocforward| redirects processing to
another source file:
%
\begin{center}
\begin{tabular}{l}
|\input{childdoc.def}|\\
|\childdocforward[|\textit{main}|]{|\textit{dest}|}|\\
\end{tabular}
\end{center}
%
The argument \textit{dest} is the destination file
(without extension).
It should be the main file or one of the child files.
Note that further \textsf{childdoc} directives
such as |\childdocof| and |\childdocforward|
in the indicated file will be processed in this form.
The optional argument \textit{main}
passes on directly to the main file \textit{main}
while pretending to compile the child \textit{dest}.
This form behaves as if \textit{dest}
issues |\childdocof{|\textit{main}|}| right away,
and no further \textsf{childdoc} directives will be processed.

%%%%%%%%%%%%%%%%%%%%%%%%%%%%%%%%%%%%%%%%
\DescribeMacro{\...prefix}
In the alternative form |\childdocforwardprefix|,
%
\begin{center}
\begin{tabular}{l}
|\input{childdoc.def}|\\
|\childdocforwardprefix[|\textit{main}|]{|\textit{prefix}|}{|\textit{dest}|}|
\end{tabular}
\end{center}
%
the destination file is determined by a pattern
depending on the current file:
To make this work, the current file must be called
`{\textit{prefix}\hspace{0.2em}\textit{suffix}}'
with \textit{prefix} matching precisely the argument.
Processing is then passed on to the file
`{\textit{dest}\hspace{0.2em}\textit{suffix}}'.
Surely, the same effect is achieved by
directly specifying the
argument `{\textit{dest}\hspace{0.2em}\textit{suffix}}'
in the first form.
However, that requires to set up a different file
for each child. With the alternative form of the command
all these files can have exactly the same content
which simplifies setting them up and maintaining them.

For example, the following file |draft.tex|
with a compilation flag |\version| as described in \secref{sec:flags}
compiles the main document as a draft:
%
\begin{center}
\begin{tabular}{l}
|\def\version{draft}|\\
|\input{childdoc.def}|\\
|\childdocforward{|\textit{main}|}|
\end{tabular}
\end{center}
%
Likewise, the following files |final|\textit{nn}|.tex|
compile the final version of the child document
|child|\textit{nn}|.tex|:
%
\begin{center}
\begin{tabular}{l}
|\def\version{final}|\\
|\input{childdoc.def}|\\
|\childdocforwardprefix{final}{child}|
\end{tabular}
\end{center}
%

Note that when several versions of a main file and/or of each child file
are to be generated, it may be convenient to set up a |Makefile| or
shell script to automatise the process.

%%%%%%%%%%%%%%%%%%%%%%%%%%%%%%%%%%%%%%%%%%%%%%%%%%%%%%%%%%%%%%%%%%%%%%%%%%%%%%%%
\subsection{Command Line Processing}
\label{sec:commandline}

The effect of redirection files can also be achieved by invoking
the \LaTeX{} compiler with a more elaborate command line.
Most conveniently this should be done as part
of a shell script or a |Makefile|.

When using \textsf{childdoc} in the main file, the following
command lines effectively perform a redirection
(note that depending on the shell being used,
backslashes may have to be doubled: `|\|' $\to$ `|\\|'):
%
\begin{center}
|... -jobname "|\textit{target}|" |\\|"|[\textit{flags}]%
|\input{childdoc.def}\childdocforward[|\textit{main}|]{|\textit{dest}|}"|
\end{center}
%
Here \textit{target} is the name of the output file,
\textit{main} is the name of the main file
and \textit{dest} is the name of the main or child file to be processed
(all filenames without extensions).
The optional argument \textit{main} can be omitted
if \textit{main} matches \textit{dest}.
Optionally, compilation \textit{flags} can be defined via |\def| commands.
This command line makes the \TeX{} engine believe
it is compiling the file \textit{target}
whose content is specified as the latter parameter.
The provided code then forwards the processing to
\textit{main} or \textit{dest} as described in \secref{sec:forward}.

%%%%%%%%%%%%%%%%%%%%%%%%%%%%%%%%%%%%%%%%%%%%%%%%%%%%%%%%%%%%%%%%%%%%%%%%%%%%%%%%
\subsection{Include by Input}
\label{sec:input}

Including child documents by |\include| has some restrictions by design.
Most notably, the content of a child document always occupies
its own set of pages; pages cannot be shared between child documents.
Usually, this behaviour makes perfect sense
because each child document contain an essential part of the document.
However, in some situations it may be desirable to compose
a document from a collection of parts
without having mandatory page breaks between then.
For this case, the package
provides a mechanism to include parts
by |\input| which can also be processed individually.
However, by construction this mechanism
requires manual handling of the content to be output.

%%%%%%%%%%%%%%%%%%%%%%%%%%%%%%%%%%%%%%%%
\DescribeMacro{\ifchilddocmanual}
The main file should be prepared as usual, see \secref{sec:include}.
However, the document body must make a distinction
between processing of an individual part and of the main document, e.g.:
%
\begin{center}
\begin{tabular}{l}
|\ifchilddocmanual|\\
|\input{\childdocname}|\\
|\||else|\\
\textit{document body with }|\input{|\textit{part}|}|\\
|\||fi|
\end{tabular}
\end{center}
%
The conditional |\ifchilddocmanual| is true whenever
a part to be included by |\input| is being compiled,
and the name of the part is stored in |\childdocname|.

%%%%%%%%%%%%%%%%%%%%%%%%%%%%%%%%%%%%%%%%
\DescribeMacro{\childdocby}
Each part to be included by |\input| should start with:
%
\begin{center}
\begin{tabular}{l}
|\input{childdoc.def}|\\
|\childdocby{|\textit{main}|}|\\
\end{tabular}
\end{center}
%
The directive |\childdocby| is similar to |\childdocof|
described in \secref{sec:include},
but the subsequent selection of content must be done manually.
To that end, both |\ifchilddoc| and |\ifchilddocmanual|
will be true upon processing of a part,
and the name of the part is stored in |\childdocname|.
Note that |\jobname| will be set to the filename of the current part
so that each part receives an individual |.aux| file
that does not interfere with the |.aux| file(s) of the main document.
This behaviour can be altered by the alternative form
|\childdocby[*]{|\textit{main}|}| (with a non-empty optional argument)
which uses the |.aux| file of the main document
by setting |\jobname| to \textit{main}.

%%%%%%%%%%%%%%%%%%%%%%%%%%%%%%%%%%%%%%%%%%%%%%%%%%%%%%%%%%%%%%%%%%%%%%%%%%%%%%%%
\subsection{Driver Development}
\label{sec:driver}

The \textsf{childdoc} mechanism can also be use for the development
of definition files such as \LaTeX{} styles or classes.
This case differs from the above setup with multiple parts
included by |\include| in that no |\includeonly| should be invoked.
This can be achieved by starting the include file
(before |\ProvidesPackage|) with:
%
\begin{center}
\begin{tabular}{l}
|\input{childdoc.def}|\\
|\childdocforward{|\textit{main}|}|\\
\end{tabular}
\end{center}
%
or alternatively with:
%
\begin{center}
\begin{tabular}{l}
|\input{childdoc.def}|\\
|\childdocby{|\textit{main}|}|\\
\end{tabular}
\end{center}
%
Both forms have slightly different effects as described above.
The main file is prepared as usual, see \secref{sec:include}.

%%%%%%%%%%%%%%%%%%%%%%%%%%%%%%%%%%%%%%%%%%%%%%%%%%%%%%%%%%%%%%%%%%%%%%%%%%%%%%%%
\subsection{Legacy Detection}
\label{sec:detection}

The directive |\childdocmain| in the main file can detect
whether the complete document or merely a child is to be compiled
even without using the directive |\childdocof|.
This method is deprecated because it is less robust
and there is no compelling reason to use it;
it is merely provided for backward compatibility
and it may be removed in future versions.

If the detection mechanism is to be used,
it is mandatory to correctly specify
the filename of the main file as the argument of |\childdocmain|:
%
\begin{center}
\begin{tabular}{l}
|\input{childdoc.def}|\\
|\childdocmain{|\textit{main}|}|\\
\end{tabular}
\end{center}
%
If |\jobname| does not match the argument \textit{main} of |\childdocmain|,
it is assumed that |\jobname| points to the child file to be compiled.
When using |\childdocmain| with the main file specified as argument,
it suffices to start a child file
with just |\input{|\textit{main}|}|
without loading of the package and using |\childdocof|.
If instead all processing is done
with the appropriate \textsf{childdoc} directives,
the argument of \textit{main} of |\childdocmain| can be empty.

An alternative version of the command line processing described
in \secref{sec:commandline} using the detection mechanism reads:
%
\begin{center}
|... -jobname "|\textit{target}|" "|[\textit{flags}]%
[|\def\jobname{|\textit{dest}|}|]|\input{|\textit{main}|}"|
\end{center}

%%%%%%%%%%%%%%%%%%%%%%%%%%%%%%%%%%%%%%%%%%%%%%%%%%%%%%%%%%%%%%%%%%%%%%%%%%%%%%%%
\subsection{Manual Code}
\label{sec:manual}

In case one cannot be certain whether the definitions file |childdoc.def|
is installed on the target \TeX{} distribution
and one prefers not to ship it,
it is conceivable to paste a few relevant commands into the sources.

To that end, drop all statements |\input{childdoc.def}|
and perform the replacements as outlined below.
Instead of |\childdocmain{|\textit{main}|}| add the following code
to the top of the main file:
%
\begin{center}
\begin{tabular}{l}
|\||ifdefined\childdocname\endinput\||fi\newif\ifchilddoc|\\
|\edef\childdocname{\scantokens\expandafter{\jobname\noexpand}}|\\
|\def\childdocmain{|\textit{main}|}\||ifx\childdocmain\childdocname\||else|\\
|\childdoctrue\includeonly{\childdocname}\let\jobname\childdocmain\||fi|\\
\end{tabular}
\end{center}
%
Instead of |\childdocof{|\textit{main}|}| just include the main file
at the top of each child file:
%
\begin{center}
|\input{|\textit{main}|}|
\end{center}
%
A simple redirection |\childdocforward{|\textit{dest}|}| is achieved by:
%
\begin{center}
|\def\jobname{|\textit{dest}|}\input{\jobname}|
\end{center}
%
The redirection with prefix
|\childdocforwardprefix[|\textit{prefix}|]{|\textit{dest}|}|
is accomplished by:
%
\begin{center}
\begin{tabular}{l}
|{\edef\jobname{\scantokens\expandafter{\jobname\noexpand}}|\\
|\def\redirectjob |\textit{prefix}|#1~~~{\gdef\jobname{|\textit{dest}|#1}}|\\
|\expandafter\redirectjob\jobname~~~}\input{\jobname}|
\end{tabular}
\end{center}

In an alternative approach,
child documents can be compiled by a specific command line
without additional code or specific definitions:
%
\begin{center}
|... -jobname "|\textit{target}|" "|[\textit{flags}]%
|\includeonly{|\textit{dest}|}\input{|\textit{main}|}"|
\end{center}
%

%%%%%%%%%%%%%%%%%%%%%%%%%%%%%%%%%%%%%%%%%%%%%%%%%%%%%%%%%%%%%%%%%%%%%%%%%%%%%%%%
%%%%%%%%%%%%%%%%%%%%%%%%%%%%%%%%%%%%%%%%%%%%%%%%%%%%%%%%%%%%%%%%%%%%%%%%%%%%%%%%
\section{Information}

%%%%%%%%%%%%%%%%%%%%%%%%%%%%%%%%%%%%%%%%%%%%%%%%%%%%%%%%%%%%%%%%%%%%%%%%%%%%%%%%
\subsection{Copyright}

Copyright \copyright{} 2017--2018 Niklas Beisert

This work may be distributed and/or modified under the
conditions of the \LaTeX{} Project Public License, either version 1.3
of this license or (at your option) any later version.
The latest version of this license is in
  \url{http://www.latex-project.org/lppl.txt}
and version 1.3 or later is part of all distributions of \LaTeX{}
version 2005/12/01 or later.

This work has the LPPL maintenance status `maintained'.

The Current Maintainer of this work is Niklas Beisert.

This work consists of the files |README.txt|, |childdoc.ins| and |childdoc.dtx|
as well as the derived files |childdoc.def|, |cdocsamp.tex|
with |cdocsch1.tex|, |cdocsch2.tex|, |cdocspt3.tex|, |cdocspt4.tex|,
|cdocsdrf.tex|, |cdocsfn1.tex|, |cdocsfn2.tex|
as well as |childdoc.pdf|.

%%%%%%%%%%%%%%%%%%%%%%%%%%%%%%%%%%%%%%%%%%%%%%%%%%%%%%%%%%%%%%%%%%%%%%%%%%%%%%%%
\subsection{Files and Installation}

The package consists of the files:
%
\begin{center}
\begin{tabular}{ll}
    |README.txt|   & readme file \\
    |childdoc.ins| & installation file \\
    |childdoc.dtx| & source file \\
    |childdoc.def| & definition file \\
    |cdocsamp.tex| & sample main file \\
    |cdocsch1.tex| & sample include file \\
    |cdocsch2.tex| & sample include file \\
    |cdocspt3.tex| & sample part file \\
    |cdocspt4.tex| & sample part file \\
    |cdocsdrf.tex| & sample redirection file \\
    |cdocsfn1.tex| & sample redirection file \\
    |cdocsfn2.tex| & sample redirection file \\
    |childdoc.pdf| & manual
\end{tabular}
\end{center}
%
The distribution consists of the files
|README.txt|, |childdoc.ins| and |childdoc.dtx|.
%
\begin{itemize}
\item
Run (pdf)\LaTeX{} on |childdoc.dtx|
to compile the manual |childdoc.pdf| (this file).
\item
Run \LaTeX{} on |childdoc.ins| to create the definitions file |childdoc.def|
and the sample |cdocsamp.tex| with include files
|cdocsch1.tex|, |cdocsch2.tex|, |cdocspt3.tex|, |cdocspt4.tex|,
|cdocsdrf.tex|, |cdocsfn1.tex|, |cdocsfn2.tex|.
Then copy the file |childdoc.def| to an appropriate directory of your \LaTeX{}
distribution, e.g.\ \textit{texmf-root}|/tex/latex/childdoc|.
\end{itemize}

%%%%%%%%%%%%%%%%%%%%%%%%%%%%%%%%%%%%%%%%%%%%%%%%%%%%%%%%%%%%%%%%%%%%%%%%%%%%%%%%
\subsection{Related CTAN Packages}

There are several other packages which offer a similar functionality:
%
\begin{itemize}
\item
The packages
\href{http://ctan.org/pkg/docmute}{\textsf{docmute}},
\href{http://ctan.org/pkg/includex}{\textsf{includex}} and
\href{http://ctan.org/pkg/standalone}{\textsf{standalone}}
provide commands to include only the document body of
a child file thus allowing both files to be compiled individually.
\item
The packages \href{http://ctan.org/pkg/subdocs}{\textsf{subdocs}}
and \href{http://ctan.org/pkg/subfiles}{\textsf{subfiles}}
provide structures in which the main and child documents can be
encapsulated and allowing them to be compiled individually.
The inclusion mechanism is different from the conventional |\include|.
\item
The package \href{http://ctan.org/pkg/combine}{\textsf{combine}}
is an elaborate solution to combine several documents into one.
\end{itemize}
%
See also the CTAN topic \href{http://ctan.org/topic/subdocs}{\textsf{subdocs}}
for further related packages.
The present package differs from the above solutions in that
a document structure constructed with the conventional |\include| mechanism
just needs two extra commands at the top of every file
such that all constituent files can be compiled individually.

%%%%%%%%%%%%%%%%%%%%%%%%%%%%%%%%%%%%%%%%%%%%%%%%%%%%%%%%%%%%%%%%%%%%%%%%%%%%%%%%
%\subsection{Feature Suggestions}
%
%The following is a list of features which may be useful for future
%versions of this package:
%%
%\begin{itemize}
%\item
%\ldots
%\end{itemize}

%%%%%%%%%%%%%%%%%%%%%%%%%%%%%%%%%%%%%%%%%%%%%%%%%%%%%%%%%%%%%%%%%%%%%%%%%%%%%%%%
\subsection{Revision History}

%%%%%%%%%%%%%%%%%%%%%%%%%%%%%%%%%%%%%%%%
\paragraph{v2.0:} 2018/12/30

\begin{itemize}
\item
immediate forward processing
\item
added |\childdocby| mechanism
\item
manual restructured
\end{itemize}

%%%%%%%%%%%%%%%%%%%%%%%%%%%%%%%%%%%%%%%%
\paragraph{v1.6:} 2018/01/17

\begin{itemize}
\item
application for development of include files
\item
corrections to manual
\end{itemize}

%%%%%%%%%%%%%%%%%%%%%%%%%%%%%%%%%%%%%%%%
\paragraph{v1.5:} 2017/05/21

\begin{itemize}
\item
more complete structuring introduced
\item
|\childdocof| introduced
\item
|\childdoc| renamed to |\childdocmain|
\item
|\childredirect| renamed to |\childdocforward| and |\childdocforwardprefix|
and functionality expanded
\end{itemize}

%%%%%%%%%%%%%%%%%%%%%%%%%%%%%%%%%%%%%%%%
\paragraph{v1.0:} 2017/04/27

\begin{itemize}
\item
manual and install package
\item
first version published on CTAN
\end{itemize}

%%%%%%%%%%%%%%%%%%%%%%%%%%%%%%%%%%%%%%%%
\paragraph{v0.6:} 2017/04/26

\begin{itemize}
\item
redirection mechanism added
\end{itemize}

%%%%%%%%%%%%%%%%%%%%%%%%%%%%%%%%%%%%%%%%
\paragraph{v0.5:} 2017/04/26

\begin{itemize}
\item
functionality in definition file
\end{itemize}


%%%%%%%%%%%%%%%%%%%%%%%%%%%%%%%%%%%%%%%%%%%%%%%%%%%%%%%%%%%%%%%%%%%%%%%%%%%%%%%%
%%%%%%%%%%%%%%%%%%%%%%%%%%%%%%%%%%%%%%%%%%%%%%%%%%%%%%%%%%%%%%%%%%%%%%%%%%%%%%%%
%%%%%%%%%%%%%%%%%%%%%%%%%%%%%%%%%%%%%%%%%%%%%%%%%%%%%%%%%%%%%%%%%%%%%%%%%%%%%%%%
\appendix

\settowidth\MacroIndent{\rmfamily\scriptsize 000\ }

 \DocInput{childdoc.dtx}

\end{document}
%</driver>
% \fi
%
% %%%%%%%%%%%%%%%%%%%%%%%%%%%%%%%%%%%%%%%%%%%%%%%%%%%%%%%%%%%%%%%%%%%%%%%%%%%%%%
% %%%%%%%%%%%%%%%%%%%%%%%%%%%%%%%%%%%%%%%%%%%%%%%%%%%%%%%%%%%%%%%%%%%%%%%%%%%%%%
% \section{Sample}
%\iffalse
%<*samplemain>
%\fi
%
% The following presents a sample document
% with two chapters, two parts, a title page,
% a compile flag as well as three forwarding files to set the flag.
% It consists of eight |.tex| files:
% \begin{center}
% \begin{tabular}{ll}
% |cdocsamp.tex|&main file\\
% |cdocsch1.tex|&include file for chapter 1\\
% |cdocsch2.tex|&include file for chapter 2\\
% |cdocspt3.tex|&include file for part 3\\
% |cdocspt4.tex|&include file for part 4\\
% |cdocsdrf.tex|&forwarding file for main file in draft mode\\
% |cdocsfi1.tex|&forwarding file for final version of chapter 1\\
% |cdocsfi2.tex|&forwarding file for final version of chapter 2\\
% \end{tabular}
% \end{center}
% Each of the eight files can be compiled directly by the \LaTeX{} compiler.
%
% %%%%%%%%%%%%%%%%%%%%%%%%%%%%%%%%%%%%%%
% \paragraph{Main File.}
%
% The main file is called |cdocsamp.tex|.
%
% Load the \textsf{childdoc} definitions and
% declare the filename for the main document:
%    \begin{macrocode}
\input{childdoc.def}
\childdocmain{}
%    \end{macrocode}

% Optional override for |\version| flag:
%    \begin{macrocode}
%%\ifchilddoc\else\providecommand{\version}{draft}\fi
%    \end{macrocode}

% Define the default values for the |\version| flag
% (|final| for the main file and |draft| for childs):
%    \begin{macrocode}
\ifchilddoc
\providecommand{\version}{draft}
\else
\providecommand{\version}{final}
\fi
%    \end{macrocode}

% Load the standard document class:
%    \begin{macrocode}
\documentclass[12pt]{article}
%    \end{macrocode}

% Start the document body:
%    \begin{macrocode}
\begin{document}
%    \end{macrocode}

% Declare a title page.
% Print title, part of document being processed and version flag:
%    \begin{macrocode}
\addtocounter{page}{-1}
\begin{center}
{\LARGE\bfseries{}childdoc example\par}
\vspace{1cm}
\ifchilddoc
\ifchilddocmanual part\else chapter\fi:
`\childdocname' of `\childdocjob'\par
\else
main document: `\childdocjob'\par
\fi
version: \version\par
\end{center}
\newpage
%    \end{macrocode}

% Manually include selected file,
% otherwise process as usual:
%    \begin{macrocode}
\ifchilddocmanual
\section*{part `\childdocname'}
\input{\childdocname}
\else
%    \end{macrocode}

% Include the two chapters:
%    \begin{macrocode}
\include{cdocsch1}
\include{cdocsch2}
%    \end{macrocode}

% Include the two parts unless only chapters should be displayed:
%    \begin{macrocode}
\ifchilddoc\else
\section{part three}
\input{cdocspt3}
\section{part four}
\input{cdocspt4}
\fi
%    \end{macrocode}

% Process as usual until here:
%    \begin{macrocode}
\fi
%    \end{macrocode}

% End of document body:
%    \begin{macrocode}
\end{document}
%    \end{macrocode}
%\iffalse
%</samplemain>
%\fi
%
% %%%%%%%%%%%%%%%%%%%%%%%%%%%%%%%%%%%%%%
% \paragraph{Chapter Include Files.}
%
% The include files are called |cdocsch1.tex| and |cdocsch2.tex|.
%
%\iffalse
%<*samplechap1|samplechap2>
%\fi

% Optional override for |\version| flag:
%    \begin{macrocode}
%%\providecommand{\version}{final}
%    \end{macrocode}

% Include the main document:
%    \begin{macrocode}
\input{childdoc.def}
\childdocof{cdocsamp}
%    \end{macrocode}

%\iffalse
%</samplechap1|samplechap2>
%\fi
%
%\iffalse
%<*samplechap1>
%\fi
% Some text for chapter 1:
%    \begin{macrocode}
\section{one}
some text in chapter one
%    \end{macrocode}

%\iffalse
%</samplechap1>
%\fi
% Some text for chapter 2:
%\iffalse
%<*samplechap2>
%\fi
%    \begin{macrocode}
\section{two}
more text in chapter two
%    \end{macrocode}

%\iffalse
%</samplechap2>
%\fi
%
% %%%%%%%%%%%%%%%%%%%%%%%%%%%%%%%%%%%%%%
% \paragraph{Part Include Files.}
%
% The include files are called |cdocspt3.tex| and |cdocspt4.tex|.
%
%\iffalse
%<*samplepart3|samplepart4>
%\fi

% Optional override for |\version| flag:
%    \begin{macrocode}
%%\providecommand{\version}{final}
%    \end{macrocode}

% Include the main document:
%    \begin{macrocode}
\input{childdoc.def}
\childdocby{cdocsamp}
%    \end{macrocode}

%\iffalse
%</samplepart3|samplepart4>
%\fi
%
%\iffalse
%<*samplepart3>
%\fi
% Some text for part 3:
%    \begin{macrocode}
some text in part three
%    \end{macrocode}

%\iffalse
%</samplepart3>
%\fi
% Some text for part 4:
%\iffalse
%<*samplepart4>
%\fi
%    \begin{macrocode}
more text in part four
%    \end{macrocode}

%\iffalse
%</samplepart4>
%\fi
%
% %%%%%%%%%%%%%%%%%%%%%%%%%%%%%%%%%%%%%%
% \paragraph{Forwarding for a Complete Draft.}
%
% The following forwarding file |cdocsdrf.tex|
% compiles the main document in draft mode:
%\iffalse
%<*sampledraft>
%\fi
%    \begin{macrocode}
\def\version{draft}
\input{childdoc.def}
\childdocforward{cdocsamp}
%    \end{macrocode}

%\iffalse
%</sampledraft>
%\fi
%
% %%%%%%%%%%%%%%%%%%%%%%%%%%%%%%%%%%%%%%
% \paragraph{Forwarding for Final Version of the Chapters.}
%
% The following forwarding files |cdocsfn1.tex| and |cdocsfn2.tex|
% (with identical content)
% compile the final versions of the child documents
% |cdocsch1.tex| and |cdocsch2.tex|, respectively:
%\iffalse
%<*samplefinal>
%\fi
%    \begin{macrocode}
\def\version{final}
\input{childdoc.def}
\childdocforwardprefix[cdocsamp]{cdocsfn}{cdocsch}
%    \end{macrocode}

%\iffalse
%</samplefinal>
%\fi
%
% %%%%%%%%%%%%%%%%%%%%%%%%%%%%%%%%%%%%%%
% \paragraph{Command Line Processing.}
%
% The following three command lines generate the output files
% |cdocscld|, |cdocscl1| and |cdocscl2|
% which should be identical to
% |cdocsdrf|, |cdocsch1| and |cdocsfn2|, respectively:
% \begin{center}
% \begin{tabular}{l}
% |latex -jobname cdocscld \|\\
% |  "\def\version{draft}\input{childdoc.def}\childdocforward{cdocsamp}"|\\
% |latex -jobname cdocscl1 \|\\
% |  "\input{childdoc.def}\childdocforward[cdocsamp]{cdocsch1}"|\\
% |latex -jobname cdocscl2 \|\\
% |  "\def\version{final}\input{childdoc.def}\childdocforward{cdocsch2}"|
% \end{tabular}
% \end{center}
% Note that the trailing backslash on each first line
% merely continues the input to the second line
% (for convenient cut ant paste).
% Furthermore, the command |latex| can be replaced by any
% of its alternative versions such as |pdflatex|.
%
% %%%%%%%%%%%%%%%%%%%%%%%%%%%%%%%%%%%%%%%%%%%%%%%%%%%%%%%%%%%%%%%%%%%%%%%%%%%%%%
% %%%%%%%%%%%%%%%%%%%%%%%%%%%%%%%%%%%%%%%%%%%%%%%%%%%%%%%%%%%%%%%%%%%%%%%%%%%%%%
% \section{Implementation}
%\iffalse
%<*package>
%\fi
%
% This section describes the definitions file |childdoc.def|.

% The definitions cannot be loaded using |\usepackage| or |\RequirePackage|
% which has a mechanism to prevent loading a style file more than once.
% When loading the definitions by means of |\input|
% multiple instances have to be prevented manually:
%\iffalse
%This code needs to be before the `\ProvidesFile' directive
%which is defined at the beginning of this file.
%Therefore it is also placed there and commented out here.
%</package>
%<*discard>
%\fi
%    \begin{macrocode}
\ifdefined\childdocmain\endinput\fi
%    \end{macrocode}
%\iffalse
%</discard>
%<*package>
%\fi
%
% \macro{\ifchilddoc}
% \macro{\ifchilddocmanual}
% The conditional |\ifchilddoc| tells whether a
% child (true) or main (false) document is being compiled.
% The conditional |\ifchilddocmanual| tells whether
% the |\includeonly| mechanism is used (false) or
% the selection of child files must be performed manually (true).
% The definitions initialise to false:
%    \begin{macrocode}
\newif\ifchilddoc
\newif\ifchilddocmanual
%    \end{macrocode}

% \macro{\childdocname}
% \macro{\childdocjob}
% The macro |\childdocname| stores the name of the main document
% to be compiled. The macro |\childdocjob| stores the name of
% the document on which the \LaTeX{} compiler was originally invoked.
% The content of |\jobname| cannot be compared
% to filenames specified in the source due to different catcodes.
% The following code rescans |\jobname|, stores the result
% in |\childdocname| and saves a copy in |\childdocjob|:
%    \begin{macrocode}
\edef\childdocname{\scantokens\expandafter{\jobname\noexpand}}
\let\childdocjob\childdocname
%    \end{macrocode}

% \macro{\childdocdisable}
% The macro |\childdocdisable| prevents the main file
% from being processed more than once.
% At this stage, the main document command |\childdocmain|
% is assumed to be called once again where it should do nothing.
% Any subsequent call to it should prevent
% a secondary processing of the main document
% It overwrites the forwarding commands
% |\childdocof| and |\childdocforward|
% with empty macros to prevent further inclusions of the main document:
%    \begin{macrocode}
\newcommand{\childdocdisable}
{
  \renewcommand{\childdocmain}[1]{\renewcommand{\childdocmain}[1]{\endinput}}
  \renewcommand{\childdocof}[1]{}
  \renewcommand{\childdocby}[2][]{}
  \renewcommand{\childdocforward}[2][]{}
  \renewcommand{\childdocdisable}{}
}
%    \end{macrocode}

% \macro{\childdocmain}
% The macro |\childdocmain| is to be called at the top of the main file
% with nothing or the main filename (without extension) as argument.
% First, it breaks loops.
% If the argument is not empty and does not match |\childdocname|
% (which is set by the first inclusion of |childdoc.def|),
% |\ifchilddoc| is set to true, |\includeonly| is applied to the child file
% and |\jobname| is set to the main file
% (for proper handling of |.aux| files):
%    \begin{macrocode}
\newcommand{\childdocmain}[1]
{
  \childdocdisable\childdocmain{}
  \if?#1?\else
    \begingroup
      \def\childdoctmp{#1}
      \ifx\childdoctmp\childdocname
        \def\childdoctmp{}
      \else
        \def\childdoctmp
        {
          \childdoctrue
          \includeonly{\childdocname}
          \def\childdocjob{#1}
          \def\jobname{#1}
        }
      \fi
      \expandafter
    \endgroup
    \childdoctmp
  \fi
}
%    \end{macrocode}

% \macro{\childdocof}
% The command |\childdocof| redirects
% compilation to the main file |#1|.
%    \begin{macrocode}
\newcommand{\childdocof}[1]
{
  \childdocdisable
  \childdoctrue
  \includeonly{\childdocname}
  \def\jobname{#1}
  \def\childdocjob{#1}
  \input{#1}
}
%    \end{macrocode}

% \macro{\childdocby}
% The command |\childdocby| ....
%    \begin{macrocode}
\newcommand{\childdocby}[2][]
{
  \childdocdisable
  \childdoctrue
  \childdocmanualtrue
  \if?#1?\else
    \def\jobname{#2}
  \fi
  \def\childdocjob{#2}
  \input{#2}
  \endinput
}
%    \end{macrocode}

% \macro{\childdocforward}
% The command |\childdocforward| redirects
% compilation to the main file or
% (if the optional argument is given) a child file.
% Parameters are set as if the main file
% or a child file starting with |\childdocof| was compiled.
% Then compilation is handed over to the main file:
%    \begin{macrocode}
\newcommand{\childdocforward}[2][]
{
  \begingroup
    \if?#1?
      \def\childdoctmp
      {
        \def\childdocname{#2}
        \def\childdocjob{#2}
        \def\jobname{#2}
        \input{#2}
        \endinput
      }
    \else
      \def\childdoctmp
      {
        \childdocdisable
        \def\childdocname{#2}
        \childdoctrue
        \includeonly{#2}
        \def\childdocjob{#1}
        \def\jobname{#1}
        \input{#1}
        \endinput
      }
    \fi
    \expandafter
  \endgroup
  \childdoctmp
}
%    \end{macrocode}

% \macro{\childdocforwardprefix}
% The command |\childdocforwardprefix| redirects
% compilation to the main or a child file by means of a pattern.
% The prefix |#1| in the current filename is replaced by |#2|
% and the suffix of the current filename is kept
% (it is assumed that the filename does not contain the substring `|~~~|'
% which is used as a delimiter).
% Compilation is handed over to the new file by |\childdocforward|:
%    \begin{macrocode}
\newcommand{\childdocforwardprefix}[3][]
{
  \begingroup
    \def\childdocextract #2##1~~~{\def\childdoctmp{\childdocforward[#1]{#3##1}}}
    \expandafter\childdocextract\childdocname~~~
    \expandafter
  \endgroup
  \childdoctmp
}
%    \end{macrocode}

% \macro{\childdoc}
% The deprecated macro |\childdoc| is a legacy version of |\childdocmain|:
%    \begin{macrocode}
\newcommand{\childdoc}{\childdocmain}
%    \end{macrocode}

% \macro{\childdocredirect}
% The deprecated macro |\childdocredirect| is a legacy version
% of |\childdocforward| and |\childdocforwardprefix|:
%    \begin{macrocode}
\newcommand{\childdocredirect}[2][]
{
  \begingroup
    \if?#1?
      \def\childdoctmp{\childdocforward{#2}}
    \else
      \def\childdoctmp{\childdocforwardprefix{#1}{#2}}
    \fi
    \expandafter
  \endgroup
  \childdoctmp
}
%    \end{macrocode}

%\iffalse
%</package>
%\fi
%
\endinput
|
and perform the replacements as outlined below.
Instead of |\childdocmain{|\textit{main}|}| add the following code
to the top of the main file:
%
\begin{center}
\begin{tabular}{l}
|\||ifdefined\childdocname\endinput\||fi\newif\ifchilddoc|\\
|\edef\childdocname{\scantokens\expandafter{\jobname\noexpand}}|\\
|\def\childdocmain{|\textit{main}|}\||ifx\childdocmain\childdocname\||else|\\
|\childdoctrue\includeonly{\childdocname}\let\jobname\childdocmain\||fi|\\
\end{tabular}
\end{center}
%
Instead of |\childdocof{|\textit{main}|}| just include the main file
at the top of each child file:
%
\begin{center}
|\input{|\textit{main}|}|
\end{center}
%
A simple redirection |\childdocforward{|\textit{dest}|}| is achieved by:
%
\begin{center}
|\def\jobname{|\textit{dest}|}\input{\jobname}|
\end{center}
%
The redirection with prefix
|\childdocforwardprefix[|\textit{prefix}|]{|\textit{dest}|}|
is accomplished by:
%
\begin{center}
\begin{tabular}{l}
|{\edef\jobname{\scantokens\expandafter{\jobname\noexpand}}|\\
|\def\redirectjob |\textit{prefix}|#1~~~{\gdef\jobname{|\textit{dest}|#1}}|\\
|\expandafter\redirectjob\jobname~~~}\input{\jobname}|
\end{tabular}
\end{center}

In an alternative approach,
child documents can be compiled by a specific command line
without additional code or specific definitions:
%
\begin{center}
|... -jobname "|\textit{target}|" "|[\textit{flags}]%
|\includeonly{|\textit{dest}|}\input{|\textit{main}|}"|
\end{center}
%

%%%%%%%%%%%%%%%%%%%%%%%%%%%%%%%%%%%%%%%%%%%%%%%%%%%%%%%%%%%%%%%%%%%%%%%%%%%%%%%%
%%%%%%%%%%%%%%%%%%%%%%%%%%%%%%%%%%%%%%%%%%%%%%%%%%%%%%%%%%%%%%%%%%%%%%%%%%%%%%%%
\section{Information}

%%%%%%%%%%%%%%%%%%%%%%%%%%%%%%%%%%%%%%%%%%%%%%%%%%%%%%%%%%%%%%%%%%%%%%%%%%%%%%%%
\subsection{Copyright}

Copyright \copyright{} 2017--2018 Niklas Beisert

This work may be distributed and/or modified under the
conditions of the \LaTeX{} Project Public License, either version 1.3
of this license or (at your option) any later version.
The latest version of this license is in
  \url{http://www.latex-project.org/lppl.txt}
and version 1.3 or later is part of all distributions of \LaTeX{}
version 2005/12/01 or later.

This work has the LPPL maintenance status `maintained'.

The Current Maintainer of this work is Niklas Beisert.

This work consists of the files |README.txt|, |childdoc.ins| and |childdoc.dtx|
as well as the derived files |childdoc.def|, |cdocsamp.tex|
with |cdocsch1.tex|, |cdocsch2.tex|, |cdocspt3.tex|, |cdocspt4.tex|,
|cdocsdrf.tex|, |cdocsfn1.tex|, |cdocsfn2.tex|
as well as |childdoc.pdf|.

%%%%%%%%%%%%%%%%%%%%%%%%%%%%%%%%%%%%%%%%%%%%%%%%%%%%%%%%%%%%%%%%%%%%%%%%%%%%%%%%
\subsection{Files and Installation}

The package consists of the files:
%
\begin{center}
\begin{tabular}{ll}
    |README.txt|   & readme file \\
    |childdoc.ins| & installation file \\
    |childdoc.dtx| & source file \\
    |childdoc.def| & definition file \\
    |cdocsamp.tex| & sample main file \\
    |cdocsch1.tex| & sample include file \\
    |cdocsch2.tex| & sample include file \\
    |cdocspt3.tex| & sample part file \\
    |cdocspt4.tex| & sample part file \\
    |cdocsdrf.tex| & sample redirection file \\
    |cdocsfn1.tex| & sample redirection file \\
    |cdocsfn2.tex| & sample redirection file \\
    |childdoc.pdf| & manual
\end{tabular}
\end{center}
%
The distribution consists of the files
|README.txt|, |childdoc.ins| and |childdoc.dtx|.
%
\begin{itemize}
\item
Run (pdf)\LaTeX{} on |childdoc.dtx|
to compile the manual |childdoc.pdf| (this file).
\item
Run \LaTeX{} on |childdoc.ins| to create the definitions file |childdoc.def|
and the sample |cdocsamp.tex| with include files
|cdocsch1.tex|, |cdocsch2.tex|, |cdocspt3.tex|, |cdocspt4.tex|,
|cdocsdrf.tex|, |cdocsfn1.tex|, |cdocsfn2.tex|.
Then copy the file |childdoc.def| to an appropriate directory of your \LaTeX{}
distribution, e.g.\ \textit{texmf-root}|/tex/latex/childdoc|.
\end{itemize}

%%%%%%%%%%%%%%%%%%%%%%%%%%%%%%%%%%%%%%%%%%%%%%%%%%%%%%%%%%%%%%%%%%%%%%%%%%%%%%%%
\subsection{Related CTAN Packages}

There are several other packages which offer a similar functionality:
%
\begin{itemize}
\item
The packages
\href{http://ctan.org/pkg/docmute}{\textsf{docmute}},
\href{http://ctan.org/pkg/includex}{\textsf{includex}} and
\href{http://ctan.org/pkg/standalone}{\textsf{standalone}}
provide commands to include only the document body of
a child file thus allowing both files to be compiled individually.
\item
The packages \href{http://ctan.org/pkg/subdocs}{\textsf{subdocs}}
and \href{http://ctan.org/pkg/subfiles}{\textsf{subfiles}}
provide structures in which the main and child documents can be
encapsulated and allowing them to be compiled individually.
The inclusion mechanism is different from the conventional |\include|.
\item
The package \href{http://ctan.org/pkg/combine}{\textsf{combine}}
is an elaborate solution to combine several documents into one.
\end{itemize}
%
See also the CTAN topic \href{http://ctan.org/topic/subdocs}{\textsf{subdocs}}
for further related packages.
The present package differs from the above solutions in that
a document structure constructed with the conventional |\include| mechanism
just needs two extra commands at the top of every file
such that all constituent files can be compiled individually.

%%%%%%%%%%%%%%%%%%%%%%%%%%%%%%%%%%%%%%%%%%%%%%%%%%%%%%%%%%%%%%%%%%%%%%%%%%%%%%%%
%\subsection{Feature Suggestions}
%
%The following is a list of features which may be useful for future
%versions of this package:
%%
%\begin{itemize}
%\item
%\ldots
%\end{itemize}

%%%%%%%%%%%%%%%%%%%%%%%%%%%%%%%%%%%%%%%%%%%%%%%%%%%%%%%%%%%%%%%%%%%%%%%%%%%%%%%%
\subsection{Revision History}

%%%%%%%%%%%%%%%%%%%%%%%%%%%%%%%%%%%%%%%%
\paragraph{v2.0:} 2018/12/30

\begin{itemize}
\item
immediate forward processing
\item
added |\childdocby| mechanism
\item
manual restructured
\end{itemize}

%%%%%%%%%%%%%%%%%%%%%%%%%%%%%%%%%%%%%%%%
\paragraph{v1.6:} 2018/01/17

\begin{itemize}
\item
application for development of include files
\item
corrections to manual
\end{itemize}

%%%%%%%%%%%%%%%%%%%%%%%%%%%%%%%%%%%%%%%%
\paragraph{v1.5:} 2017/05/21

\begin{itemize}
\item
more complete structuring introduced
\item
|\childdocof| introduced
\item
|\childdoc| renamed to |\childdocmain|
\item
|\childredirect| renamed to |\childdocforward| and |\childdocforwardprefix|
and functionality expanded
\end{itemize}

%%%%%%%%%%%%%%%%%%%%%%%%%%%%%%%%%%%%%%%%
\paragraph{v1.0:} 2017/04/27

\begin{itemize}
\item
manual and install package
\item
first version published on CTAN
\end{itemize}

%%%%%%%%%%%%%%%%%%%%%%%%%%%%%%%%%%%%%%%%
\paragraph{v0.6:} 2017/04/26

\begin{itemize}
\item
redirection mechanism added
\end{itemize}

%%%%%%%%%%%%%%%%%%%%%%%%%%%%%%%%%%%%%%%%
\paragraph{v0.5:} 2017/04/26

\begin{itemize}
\item
functionality in definition file
\end{itemize}


%%%%%%%%%%%%%%%%%%%%%%%%%%%%%%%%%%%%%%%%%%%%%%%%%%%%%%%%%%%%%%%%%%%%%%%%%%%%%%%%
%%%%%%%%%%%%%%%%%%%%%%%%%%%%%%%%%%%%%%%%%%%%%%%%%%%%%%%%%%%%%%%%%%%%%%%%%%%%%%%%
%%%%%%%%%%%%%%%%%%%%%%%%%%%%%%%%%%%%%%%%%%%%%%%%%%%%%%%%%%%%%%%%%%%%%%%%%%%%%%%%
\appendix

\settowidth\MacroIndent{\rmfamily\scriptsize 000\ }

 \DocInput{childdoc.dtx}

\end{document}
%</driver>
% \fi
%
% %%%%%%%%%%%%%%%%%%%%%%%%%%%%%%%%%%%%%%%%%%%%%%%%%%%%%%%%%%%%%%%%%%%%%%%%%%%%%%
% %%%%%%%%%%%%%%%%%%%%%%%%%%%%%%%%%%%%%%%%%%%%%%%%%%%%%%%%%%%%%%%%%%%%%%%%%%%%%%
% \section{Sample}
%\iffalse
%<*samplemain>
%\fi
%
% The following presents a sample document
% with two chapters, two parts, a title page,
% a compile flag as well as three forwarding files to set the flag.
% It consists of eight |.tex| files:
% \begin{center}
% \begin{tabular}{ll}
% |cdocsamp.tex|&main file\\
% |cdocsch1.tex|&include file for chapter 1\\
% |cdocsch2.tex|&include file for chapter 2\\
% |cdocspt3.tex|&include file for part 3\\
% |cdocspt4.tex|&include file for part 4\\
% |cdocsdrf.tex|&forwarding file for main file in draft mode\\
% |cdocsfi1.tex|&forwarding file for final version of chapter 1\\
% |cdocsfi2.tex|&forwarding file for final version of chapter 2\\
% \end{tabular}
% \end{center}
% Each of the eight files can be compiled directly by the \LaTeX{} compiler.
%
% %%%%%%%%%%%%%%%%%%%%%%%%%%%%%%%%%%%%%%
% \paragraph{Main File.}
%
% The main file is called |cdocsamp.tex|.
%
% Load the \textsf{childdoc} definitions and
% declare the filename for the main document:
%    \begin{macrocode}
% \iffalse
%
% childdoc.dtx Copyright (C) 2017-2018 Niklas Beisert
%
% This work may be distributed and/or modified under the
% conditions of the LaTeX Project Public License, either version 1.3
% of this license or (at your option) any later version.
% The latest version of this license is in
%   http://www.latex-project.org/lppl.txt
% and version 1.3 or later is part of all distributions of LaTeX
% version 2005/12/01 or later.
%
% This work has the LPPL maintenance status `maintained'.
%
% The Current Maintainer of this work is Niklas Beisert.
%
% This work consists of the files childdoc.dtx and childdoc.ins
% and the derived files childdoc.def and cdocsamp.tex with
% cdocsch1.tex, cdocsch2.tex, cdocsdrf.tex, cdocsfn1.tex, cdocsfn2.tex.
%
%<package>\ifdefined\childdocmain\endinput\fi
%<package>\ProvidesFile{childdoc.def}[2018/12/30 v2.0 child document driver]
%<samplemain>\ProvidesFile{cdocsamp.tex}[2018/12/30 v2.0 sample for childdoc]
%<*driver>
%\ProvidesFile{childdoc.drv}[2018/12/30 v2.0 childdoc reference manual file]
\PassOptionsToClass{10pt,a4paper}{article}
\documentclass{ltxdoc}

\usepackage[margin=35mm]{geometry}
\usepackage{hyperref}
\usepackage{hyperxmp}
\usepackage[usenames]{color}

\hypersetup{colorlinks=true}
\hypersetup{pdfstartview=FitH}
\hypersetup{pdfpagemode=UseNone}
\hypersetup{pdfsource={}}
\hypersetup{pdflang={en-UK}}
\hypersetup{pdfcopyright={Copyright 2017-2018 Niklas Beisert.
  This work may be distributed and/or modified under the
  conditions of the LaTeX Project Public License, either version 1.3
  of this license or (at your option) any later version.}}
\hypersetup{pdflicenseurl={http://www.latex-project.org/lppl.txt}}
\hypersetup{pdfcontactaddress={ETH Zurich, ITP, HIT K,
  Wolfgang-Pauli-Strasse 27}}
\hypersetup{pdfcontactpostcode={8093}}
\hypersetup{pdfcontactcity={Zurich}}
\hypersetup{pdfcontactcountry={Switzerland}}
\hypersetup{pdfcontactemail={nbeisert@itp.phys.ethz.ch}}
\hypersetup{pdfcontacturl={http://people.phys.ethz.ch/\xmptilde nbeisert/}}

\newcommand{\secref}[1]{\hyperref[#1]{section \ref*{#1}}}

\parskip1ex
\parindent0pt
\let\olditemize\itemize
\def\itemize{\olditemize\parskip0pt}

\begin{document}

\title{The \textsf{childdoc} Package}
\hypersetup{pdftitle={The childdoc Package}}
\author{Niklas Beisert\\[2ex]
  Institut f\"ur Theoretische Physik\\
  Eidgen\"ossische Technische Hochschule Z\"urich\\
  Wolfgang-Pauli-Strasse 27, 8093 Z\"urich, Switzerland\\[1ex]
  \href{mailto:nbeisert@itp.phys.ethz.ch}
  {\texttt{nbeisert@itp.phys.ethz.ch}}}
\hypersetup{pdfauthor={Niklas Beisert}}
\hypersetup{pdfsubject={Manual for the LaTeX2e Package childdoc}}
\date{30 December 2018, \textsf{v2.0}}
\maketitle

\begin{abstract}\noindent
\textsf{childdoc} is a \LaTeXe{} package
that enables the direct compilation
of document sections included by |\include|
to individual files.
\end{abstract}

\begingroup
\parskip0ex
\tableofcontents
\endgroup

%%%%%%%%%%%%%%%%%%%%%%%%%%%%%%%%%%%%%%%%%%%%%%%%%%%%%%%%%%%%%%%%%%%%%%%%%%%%%%%%
%%%%%%%%%%%%%%%%%%%%%%%%%%%%%%%%%%%%%%%%%%%%%%%%%%%%%%%%%%%%%%%%%%%%%%%%%%%%%%%%
\section{Introduction}

\LaTeX{} provides a mechanism to structure a large document (such as a book)
into a main file and several child files (containing the chapters)
using the |\include| command.
This mechanism is beneficial for documents
which span hundreds of pages in order to
make the source file(s) more manageable.
Moreover, compilation can be restricted to
selected child files by means of the |\includeonly| command.
The latter feature can be used to reduce the compilation time while editing
(this was significantly more useful in the earlier days of \LaTeX{})
or to generate a smaller document which is easier to navigate.
Another application of |\includeonly| is to generate
documents consisting of selected parts of the complete document.

However, there are a few drawbacks of the plain |\include| mechanism:
\begin{itemize}
\item
The child files cannot be compiled on their own,
they can only be compiled via the main file.
A naive editing environment
(such as a text editor with an option
to have the current file processed by \LaTeX)
may require one to switch to the main file before compiling;
attempting to compile the child file produces errors.
\item
The main file must be modified (each time)
to adjust the |\includeonly| command
to the present needs. This easily leaves the main file in a messy state.
\item
The generated document will always carry the filename
of the main document. This is inconvenient if
several child files are to be compiled and
to be kept for distribution.
\end{itemize}

The present package provides a simple interface
to make child files individually compilable by \LaTeX{}.
Compiling a child file then has the same effect as compiling
the main file with an |\includeonly| command
to select the appropriate child.
Moreover the generated document will carry the name of the child
rather than the main file.
This resolves all three above issues.

This feature is meant to make the editing of books,
thesis documents and lecture notes somewhat more convenient.
However, the package can also be used efficiently for
composing a series of documents (such as exercise sheets)
which are typically distributed individually.
It then assists the author in generating the individual documents
(potentially in different versions)
as well as a document containing the collected series.
Another application is in developing style files
or other kinds of included material
where compilation of the style file could redirect
to a sample or test file.

%%%%%%%%%%%%%%%%%%%%%%%%%%%%%%%%%%%%%%%%%%%%%%%%%%%%%%%%%%%%%%%%%%%%%%%%%%%%%%%%
%%%%%%%%%%%%%%%%%%%%%%%%%%%%%%%%%%%%%%%%%%%%%%%%%%%%%%%%%%%%%%%%%%%%%%%%%%%%%%%%
\section{Usage}

First of all, the package \textsf{childdoc} is \emph{not} a standard
\LaTeXe{} |.sty| style file! Therefore it needs to be invoked in
a non-standard way.

%%%%%%%%%%%%%%%%%%%%%%%%%%%%%%%%%%%%%%%%%%%%%%%%%%%%%%%%%%%%%%%%%%%%%%%%%%%%%%%%
\subsection{Included Files}
\label{sec:include}

%%%%%%%%%%%%%%%%%%%%%%%%%%%%%%%%%%%%%%%%
\DescribeMacro{\childdocmain}
To use the package, add the commands
\begin{center}
\begin{tabular}{l}
|\input{childdoc.def}|\\
|\childdocmain{}|\\
\end{tabular}
\end{center}
at the very top of the main \LaTeX{} file,
in particular \emph{before} the |\documentclass| statement!
The argument of |\childdocmain| should be left empty
(but it must be present).

%%%%%%%%%%%%%%%%%%%%%%%%%%%%%%%%%%%%%%%%
\DescribeMacro{\childdocof}
Furthermore, add the commands
\begin{center}
\begin{tabular}{l}
|\input{childdoc.def}|\\
|\childdocof{|\textit{main}|}|\\
\end{tabular}
\end{center}
at the top of every child file \textit{child}
which is included by |\include{|\textit{child}|}|
from within the main file
(or at least for those files to be compiled individually).
The argument \textit{main} must be the filename of the main file.

There are a couple of
considerations in setting up the main and child documents:

%%%%%%%%%%%%%%%%%%%%%%%%%%%%%%%%%%%%%%%%
\paragraph{Restrictions.}

Please note the following restrictions:
\begin{itemize}
\item
|\childdocmain| must be called with one argument \textit{main}
to ensure compatibility with earlier version of the package.
It must either be empty (|\childdocmain{}|)
or precisely match the filename of the main file in which it is specified.
See \secref{sec:detection} for further information.
\item
The filename \textit{main} must be specified without the |.tex| extension.
\item
The filename \textit{main} is case sensitive
(even in case-insensitive file systems)
due to internal string comparison.
\item
The argument \textit{main} should be fully expanded, it cannot be a macro.
\item
Subdirectories and special characters should be avoided in filenames.
\item
The command |\childdocmain{|\textit{main}|}| must be followed by a whitespace.
It should not be followed immediately by another command
or by a comment mark `|%|'.
This is because the \TeX{} parser reads the token immediately following
the argument of |\childdocmain| and puts it
at the beginning of every child section;
however, a white\-space is ignored.
\end{itemize}

%%%%%%%%%%%%%%%%%%%%%%%%%%%%%%%%%%%%%%%%
\paragraph{Content of Main File.}

It is advisable to place all content in the child files included by |\include|.
Any output contained in the main file will appear in all child documents
unless suppressed manually;
it cannot be suppressed automatically by the |\includeonly| directive
and thus should normally be avoided.
A method to include some content in the main file
by means of conditional processing is described in \secref{sec:conditional}.

%%%%%%%%%%%%%%%%%%%%%%%%%%%%%%%%%%%%%%%%
\paragraph{Page Numbering.}

When only a part of the document is compiled,
the appropriate numbering of pages
(as well as other status parameters)
is determined from the |.aux| files.
The latter contain information from previous passes.
However this information needs to propagate through
all intermediate child documents.
Therefore the page numbering in child documents may well
be inconsistent until the complete document is compiled at least once.

A useful (if unconventional) way to always ensure a consistent
page numbering is to restart the numbering in each child document
and denote the pages by `\textit{child}|.|\textit{page}'
where \textit{child} represents the chapter/section number of the child file.
This can be achieved by the command
|\numberwithin{page}{|\textit{child}|}|
of the \textsf{amsmath} package
where \textit{child} can be |chapter| or |section|
depending on the chosen structuring.
Alternatively, one can modify the macro |\thepage| appropriately
and reset the counter |page| at the start of each child file.

%%%%%%%%%%%%%%%%%%%%%%%%%%%%%%%%%%%%%%%%%%%%%%%%%%%%%%%%%%%%%%%%%%%%%%%%%%%%%%%%
\subsection{Conditional Processing}
\label{sec:conditional}

The package provides a mechanism to compile different versions
of a document. To customise the versions further some conditional processing
can come in handy to distinguish which version is being compiled.
The package provides two macros to describe the compilation context:

%%%%%%%%%%%%%%%%%%%%%%%%%%%%%%%%%%%%%%%%
\DescribeMacro{\ifchilddoc}
The conditional |\ifchilddoc| distinguishes between the compilation of
child documents and the main document:
%
\begin{center}
|\ifchilddoc |\textit{child-code}| |[|\||else |\textit{main-code}]| \||fi|
\end{center}

%%%%%%%%%%%%%%%%%%%%%%%%%%%%%%%%%%%%%%%%
\DescribeMacro{\childdocname}
\DescribeMacro{\childdocjob}
The macro |\childdocname| contains the filename (without extension)
of the main or child file being processed.
Note that |\childdocjob| will always contain the name of the main file.

%%%%%%%%%%%%%%%%%%%%%%%%%%%%%%%%%%%%%%%%
\paragraph{Title Page.}

Conditional processing can be used to include a title or banner page
in the main document when proper precautions are taken.
Importantly, the code in the main file should ensure that the page counter
(as well as other status parameters which are stored in the |.aux| files)
takes the same value after the conditional processing.
Otherwise the page numbers may take divergent values
depending on which part is compiled.

For example, a title page could be declared by:
%
\begin{center}
\begin{tabular}{l}
|\ifchilddoc\||else|\\
|\addtocounter{page}{-1}|\\
\textit{code for title page}\\
|\newpage|\\
|\||fi|
\end{tabular}
\end{center}
%
A banner page for the child documents can be generated by:
%
\begin{center}
\begin{tabular}{l}
|\ifchilddoc|\\
|\addtocounter{page}{-1}|\\
\textit{code for banner page}\\
|\newpage|\\
|\||fi|
\end{tabular}
\end{center}
%
Here one could write a message such as:
\begin{center}
|This is the part \childdocname{} of \childdocjob{}.|
\end{center}

%%%%%%%%%%%%%%%%%%%%%%%%%%%%%%%%%%%%%%%%%%%%%%%%%%%%%%%%%%%%%%%%%%%%%%%%%%%%%%%%
\subsection{Flags}
\label{sec:flags}

The package makes it easy to generate different versions
of the main or child documents.
To this end compilation flags can be defined
and assigned different default values.
They will be particularly useful in conjunction
with the forwarding mechanism described in \secref{sec:forward}.

For example, it may be useful to have a flag |\version|
which can be set to |draft| or |final|.
The document source will contain some conditional code
depending on the value of |\version|.
Suppose further, the flag should default to |final| for the main file
and to |draft| for child files
which is a natural assignment for editing the document.
This is achieved by placing the following code
in the preamble of the main document
(below the |\childdocmain| directive):
%
\begin{center}
\begin{tabular}{l}
|\ifchilddoc|\\
|\providecommand{\version}{draft}|\\
|\||else|\\
|\providecommand{\version}{final}|\\
|\||fi|
\end{tabular}
\end{center}
%
The definition by |\providecommand| makes sure
that previous definitions are not overwritten.
Further statements |\providecommand{\version}{...}|
can thus be added before the above code to override it.

For the main file, one might add a line
(between |\childdocmain| and the above block)
%
\begin{center}
|%\ifchilddoc\||else\providecommand{\version}{draft}\||fi|
\end{center}
%
which can be uncommented to produce a draft version.
Likewise one can add a line to the very top of a child file
(above the |\childdocof{|\textit{main}|}| directive)
%
\begin{center}
|%\providecommand{\version}{final}|
\end{center}
%
which can be uncommented to produce the final version of this child document.

%%%%%%%%%%%%%%%%%%%%%%%%%%%%%%%%%%%%%%%%%%%%%%%%%%%%%%%%%%%%%%%%%%%%%%%%%%%%%%%%
\subsection{Forwarding}
\label{sec:forward}

Different versions of the main or child documents
using compilation flags as described in \secref{sec:flags}
can be (permanently) stored in different files
for convenient compilation, viewing and distribution.
To this end, the package defines a command
to pass on compilation to a different file:

%%%%%%%%%%%%%%%%%%%%%%%%%%%%%%%%%%%%%%%%
\DescribeMacro{\childdocforward}
The command |\childdocforward| redirects processing to
another source file:
%
\begin{center}
\begin{tabular}{l}
|\input{childdoc.def}|\\
|\childdocforward[|\textit{main}|]{|\textit{dest}|}|\\
\end{tabular}
\end{center}
%
The argument \textit{dest} is the destination file
(without extension).
It should be the main file or one of the child files.
Note that further \textsf{childdoc} directives
such as |\childdocof| and |\childdocforward|
in the indicated file will be processed in this form.
The optional argument \textit{main}
passes on directly to the main file \textit{main}
while pretending to compile the child \textit{dest}.
This form behaves as if \textit{dest}
issues |\childdocof{|\textit{main}|}| right away,
and no further \textsf{childdoc} directives will be processed.

%%%%%%%%%%%%%%%%%%%%%%%%%%%%%%%%%%%%%%%%
\DescribeMacro{\...prefix}
In the alternative form |\childdocforwardprefix|,
%
\begin{center}
\begin{tabular}{l}
|\input{childdoc.def}|\\
|\childdocforwardprefix[|\textit{main}|]{|\textit{prefix}|}{|\textit{dest}|}|
\end{tabular}
\end{center}
%
the destination file is determined by a pattern
depending on the current file:
To make this work, the current file must be called
`{\textit{prefix}\hspace{0.2em}\textit{suffix}}'
with \textit{prefix} matching precisely the argument.
Processing is then passed on to the file
`{\textit{dest}\hspace{0.2em}\textit{suffix}}'.
Surely, the same effect is achieved by
directly specifying the
argument `{\textit{dest}\hspace{0.2em}\textit{suffix}}'
in the first form.
However, that requires to set up a different file
for each child. With the alternative form of the command
all these files can have exactly the same content
which simplifies setting them up and maintaining them.

For example, the following file |draft.tex|
with a compilation flag |\version| as described in \secref{sec:flags}
compiles the main document as a draft:
%
\begin{center}
\begin{tabular}{l}
|\def\version{draft}|\\
|\input{childdoc.def}|\\
|\childdocforward{|\textit{main}|}|
\end{tabular}
\end{center}
%
Likewise, the following files |final|\textit{nn}|.tex|
compile the final version of the child document
|child|\textit{nn}|.tex|:
%
\begin{center}
\begin{tabular}{l}
|\def\version{final}|\\
|\input{childdoc.def}|\\
|\childdocforwardprefix{final}{child}|
\end{tabular}
\end{center}
%

Note that when several versions of a main file and/or of each child file
are to be generated, it may be convenient to set up a |Makefile| or
shell script to automatise the process.

%%%%%%%%%%%%%%%%%%%%%%%%%%%%%%%%%%%%%%%%%%%%%%%%%%%%%%%%%%%%%%%%%%%%%%%%%%%%%%%%
\subsection{Command Line Processing}
\label{sec:commandline}

The effect of redirection files can also be achieved by invoking
the \LaTeX{} compiler with a more elaborate command line.
Most conveniently this should be done as part
of a shell script or a |Makefile|.

When using \textsf{childdoc} in the main file, the following
command lines effectively perform a redirection
(note that depending on the shell being used,
backslashes may have to be doubled: `|\|' $\to$ `|\\|'):
%
\begin{center}
|... -jobname "|\textit{target}|" |\\|"|[\textit{flags}]%
|\input{childdoc.def}\childdocforward[|\textit{main}|]{|\textit{dest}|}"|
\end{center}
%
Here \textit{target} is the name of the output file,
\textit{main} is the name of the main file
and \textit{dest} is the name of the main or child file to be processed
(all filenames without extensions).
The optional argument \textit{main} can be omitted
if \textit{main} matches \textit{dest}.
Optionally, compilation \textit{flags} can be defined via |\def| commands.
This command line makes the \TeX{} engine believe
it is compiling the file \textit{target}
whose content is specified as the latter parameter.
The provided code then forwards the processing to
\textit{main} or \textit{dest} as described in \secref{sec:forward}.

%%%%%%%%%%%%%%%%%%%%%%%%%%%%%%%%%%%%%%%%%%%%%%%%%%%%%%%%%%%%%%%%%%%%%%%%%%%%%%%%
\subsection{Include by Input}
\label{sec:input}

Including child documents by |\include| has some restrictions by design.
Most notably, the content of a child document always occupies
its own set of pages; pages cannot be shared between child documents.
Usually, this behaviour makes perfect sense
because each child document contain an essential part of the document.
However, in some situations it may be desirable to compose
a document from a collection of parts
without having mandatory page breaks between then.
For this case, the package
provides a mechanism to include parts
by |\input| which can also be processed individually.
However, by construction this mechanism
requires manual handling of the content to be output.

%%%%%%%%%%%%%%%%%%%%%%%%%%%%%%%%%%%%%%%%
\DescribeMacro{\ifchilddocmanual}
The main file should be prepared as usual, see \secref{sec:include}.
However, the document body must make a distinction
between processing of an individual part and of the main document, e.g.:
%
\begin{center}
\begin{tabular}{l}
|\ifchilddocmanual|\\
|\input{\childdocname}|\\
|\||else|\\
\textit{document body with }|\input{|\textit{part}|}|\\
|\||fi|
\end{tabular}
\end{center}
%
The conditional |\ifchilddocmanual| is true whenever
a part to be included by |\input| is being compiled,
and the name of the part is stored in |\childdocname|.

%%%%%%%%%%%%%%%%%%%%%%%%%%%%%%%%%%%%%%%%
\DescribeMacro{\childdocby}
Each part to be included by |\input| should start with:
%
\begin{center}
\begin{tabular}{l}
|\input{childdoc.def}|\\
|\childdocby{|\textit{main}|}|\\
\end{tabular}
\end{center}
%
The directive |\childdocby| is similar to |\childdocof|
described in \secref{sec:include},
but the subsequent selection of content must be done manually.
To that end, both |\ifchilddoc| and |\ifchilddocmanual|
will be true upon processing of a part,
and the name of the part is stored in |\childdocname|.
Note that |\jobname| will be set to the filename of the current part
so that each part receives an individual |.aux| file
that does not interfere with the |.aux| file(s) of the main document.
This behaviour can be altered by the alternative form
|\childdocby[*]{|\textit{main}|}| (with a non-empty optional argument)
which uses the |.aux| file of the main document
by setting |\jobname| to \textit{main}.

%%%%%%%%%%%%%%%%%%%%%%%%%%%%%%%%%%%%%%%%%%%%%%%%%%%%%%%%%%%%%%%%%%%%%%%%%%%%%%%%
\subsection{Driver Development}
\label{sec:driver}

The \textsf{childdoc} mechanism can also be use for the development
of definition files such as \LaTeX{} styles or classes.
This case differs from the above setup with multiple parts
included by |\include| in that no |\includeonly| should be invoked.
This can be achieved by starting the include file
(before |\ProvidesPackage|) with:
%
\begin{center}
\begin{tabular}{l}
|\input{childdoc.def}|\\
|\childdocforward{|\textit{main}|}|\\
\end{tabular}
\end{center}
%
or alternatively with:
%
\begin{center}
\begin{tabular}{l}
|\input{childdoc.def}|\\
|\childdocby{|\textit{main}|}|\\
\end{tabular}
\end{center}
%
Both forms have slightly different effects as described above.
The main file is prepared as usual, see \secref{sec:include}.

%%%%%%%%%%%%%%%%%%%%%%%%%%%%%%%%%%%%%%%%%%%%%%%%%%%%%%%%%%%%%%%%%%%%%%%%%%%%%%%%
\subsection{Legacy Detection}
\label{sec:detection}

The directive |\childdocmain| in the main file can detect
whether the complete document or merely a child is to be compiled
even without using the directive |\childdocof|.
This method is deprecated because it is less robust
and there is no compelling reason to use it;
it is merely provided for backward compatibility
and it may be removed in future versions.

If the detection mechanism is to be used,
it is mandatory to correctly specify
the filename of the main file as the argument of |\childdocmain|:
%
\begin{center}
\begin{tabular}{l}
|\input{childdoc.def}|\\
|\childdocmain{|\textit{main}|}|\\
\end{tabular}
\end{center}
%
If |\jobname| does not match the argument \textit{main} of |\childdocmain|,
it is assumed that |\jobname| points to the child file to be compiled.
When using |\childdocmain| with the main file specified as argument,
it suffices to start a child file
with just |\input{|\textit{main}|}|
without loading of the package and using |\childdocof|.
If instead all processing is done
with the appropriate \textsf{childdoc} directives,
the argument of \textit{main} of |\childdocmain| can be empty.

An alternative version of the command line processing described
in \secref{sec:commandline} using the detection mechanism reads:
%
\begin{center}
|... -jobname "|\textit{target}|" "|[\textit{flags}]%
[|\def\jobname{|\textit{dest}|}|]|\input{|\textit{main}|}"|
\end{center}

%%%%%%%%%%%%%%%%%%%%%%%%%%%%%%%%%%%%%%%%%%%%%%%%%%%%%%%%%%%%%%%%%%%%%%%%%%%%%%%%
\subsection{Manual Code}
\label{sec:manual}

In case one cannot be certain whether the definitions file |childdoc.def|
is installed on the target \TeX{} distribution
and one prefers not to ship it,
it is conceivable to paste a few relevant commands into the sources.

To that end, drop all statements |\input{childdoc.def}|
and perform the replacements as outlined below.
Instead of |\childdocmain{|\textit{main}|}| add the following code
to the top of the main file:
%
\begin{center}
\begin{tabular}{l}
|\||ifdefined\childdocname\endinput\||fi\newif\ifchilddoc|\\
|\edef\childdocname{\scantokens\expandafter{\jobname\noexpand}}|\\
|\def\childdocmain{|\textit{main}|}\||ifx\childdocmain\childdocname\||else|\\
|\childdoctrue\includeonly{\childdocname}\let\jobname\childdocmain\||fi|\\
\end{tabular}
\end{center}
%
Instead of |\childdocof{|\textit{main}|}| just include the main file
at the top of each child file:
%
\begin{center}
|\input{|\textit{main}|}|
\end{center}
%
A simple redirection |\childdocforward{|\textit{dest}|}| is achieved by:
%
\begin{center}
|\def\jobname{|\textit{dest}|}\input{\jobname}|
\end{center}
%
The redirection with prefix
|\childdocforwardprefix[|\textit{prefix}|]{|\textit{dest}|}|
is accomplished by:
%
\begin{center}
\begin{tabular}{l}
|{\edef\jobname{\scantokens\expandafter{\jobname\noexpand}}|\\
|\def\redirectjob |\textit{prefix}|#1~~~{\gdef\jobname{|\textit{dest}|#1}}|\\
|\expandafter\redirectjob\jobname~~~}\input{\jobname}|
\end{tabular}
\end{center}

In an alternative approach,
child documents can be compiled by a specific command line
without additional code or specific definitions:
%
\begin{center}
|... -jobname "|\textit{target}|" "|[\textit{flags}]%
|\includeonly{|\textit{dest}|}\input{|\textit{main}|}"|
\end{center}
%

%%%%%%%%%%%%%%%%%%%%%%%%%%%%%%%%%%%%%%%%%%%%%%%%%%%%%%%%%%%%%%%%%%%%%%%%%%%%%%%%
%%%%%%%%%%%%%%%%%%%%%%%%%%%%%%%%%%%%%%%%%%%%%%%%%%%%%%%%%%%%%%%%%%%%%%%%%%%%%%%%
\section{Information}

%%%%%%%%%%%%%%%%%%%%%%%%%%%%%%%%%%%%%%%%%%%%%%%%%%%%%%%%%%%%%%%%%%%%%%%%%%%%%%%%
\subsection{Copyright}

Copyright \copyright{} 2017--2018 Niklas Beisert

This work may be distributed and/or modified under the
conditions of the \LaTeX{} Project Public License, either version 1.3
of this license or (at your option) any later version.
The latest version of this license is in
  \url{http://www.latex-project.org/lppl.txt}
and version 1.3 or later is part of all distributions of \LaTeX{}
version 2005/12/01 or later.

This work has the LPPL maintenance status `maintained'.

The Current Maintainer of this work is Niklas Beisert.

This work consists of the files |README.txt|, |childdoc.ins| and |childdoc.dtx|
as well as the derived files |childdoc.def|, |cdocsamp.tex|
with |cdocsch1.tex|, |cdocsch2.tex|, |cdocspt3.tex|, |cdocspt4.tex|,
|cdocsdrf.tex|, |cdocsfn1.tex|, |cdocsfn2.tex|
as well as |childdoc.pdf|.

%%%%%%%%%%%%%%%%%%%%%%%%%%%%%%%%%%%%%%%%%%%%%%%%%%%%%%%%%%%%%%%%%%%%%%%%%%%%%%%%
\subsection{Files and Installation}

The package consists of the files:
%
\begin{center}
\begin{tabular}{ll}
    |README.txt|   & readme file \\
    |childdoc.ins| & installation file \\
    |childdoc.dtx| & source file \\
    |childdoc.def| & definition file \\
    |cdocsamp.tex| & sample main file \\
    |cdocsch1.tex| & sample include file \\
    |cdocsch2.tex| & sample include file \\
    |cdocspt3.tex| & sample part file \\
    |cdocspt4.tex| & sample part file \\
    |cdocsdrf.tex| & sample redirection file \\
    |cdocsfn1.tex| & sample redirection file \\
    |cdocsfn2.tex| & sample redirection file \\
    |childdoc.pdf| & manual
\end{tabular}
\end{center}
%
The distribution consists of the files
|README.txt|, |childdoc.ins| and |childdoc.dtx|.
%
\begin{itemize}
\item
Run (pdf)\LaTeX{} on |childdoc.dtx|
to compile the manual |childdoc.pdf| (this file).
\item
Run \LaTeX{} on |childdoc.ins| to create the definitions file |childdoc.def|
and the sample |cdocsamp.tex| with include files
|cdocsch1.tex|, |cdocsch2.tex|, |cdocspt3.tex|, |cdocspt4.tex|,
|cdocsdrf.tex|, |cdocsfn1.tex|, |cdocsfn2.tex|.
Then copy the file |childdoc.def| to an appropriate directory of your \LaTeX{}
distribution, e.g.\ \textit{texmf-root}|/tex/latex/childdoc|.
\end{itemize}

%%%%%%%%%%%%%%%%%%%%%%%%%%%%%%%%%%%%%%%%%%%%%%%%%%%%%%%%%%%%%%%%%%%%%%%%%%%%%%%%
\subsection{Related CTAN Packages}

There are several other packages which offer a similar functionality:
%
\begin{itemize}
\item
The packages
\href{http://ctan.org/pkg/docmute}{\textsf{docmute}},
\href{http://ctan.org/pkg/includex}{\textsf{includex}} and
\href{http://ctan.org/pkg/standalone}{\textsf{standalone}}
provide commands to include only the document body of
a child file thus allowing both files to be compiled individually.
\item
The packages \href{http://ctan.org/pkg/subdocs}{\textsf{subdocs}}
and \href{http://ctan.org/pkg/subfiles}{\textsf{subfiles}}
provide structures in which the main and child documents can be
encapsulated and allowing them to be compiled individually.
The inclusion mechanism is different from the conventional |\include|.
\item
The package \href{http://ctan.org/pkg/combine}{\textsf{combine}}
is an elaborate solution to combine several documents into one.
\end{itemize}
%
See also the CTAN topic \href{http://ctan.org/topic/subdocs}{\textsf{subdocs}}
for further related packages.
The present package differs from the above solutions in that
a document structure constructed with the conventional |\include| mechanism
just needs two extra commands at the top of every file
such that all constituent files can be compiled individually.

%%%%%%%%%%%%%%%%%%%%%%%%%%%%%%%%%%%%%%%%%%%%%%%%%%%%%%%%%%%%%%%%%%%%%%%%%%%%%%%%
%\subsection{Feature Suggestions}
%
%The following is a list of features which may be useful for future
%versions of this package:
%%
%\begin{itemize}
%\item
%\ldots
%\end{itemize}

%%%%%%%%%%%%%%%%%%%%%%%%%%%%%%%%%%%%%%%%%%%%%%%%%%%%%%%%%%%%%%%%%%%%%%%%%%%%%%%%
\subsection{Revision History}

%%%%%%%%%%%%%%%%%%%%%%%%%%%%%%%%%%%%%%%%
\paragraph{v2.0:} 2018/12/30

\begin{itemize}
\item
immediate forward processing
\item
added |\childdocby| mechanism
\item
manual restructured
\end{itemize}

%%%%%%%%%%%%%%%%%%%%%%%%%%%%%%%%%%%%%%%%
\paragraph{v1.6:} 2018/01/17

\begin{itemize}
\item
application for development of include files
\item
corrections to manual
\end{itemize}

%%%%%%%%%%%%%%%%%%%%%%%%%%%%%%%%%%%%%%%%
\paragraph{v1.5:} 2017/05/21

\begin{itemize}
\item
more complete structuring introduced
\item
|\childdocof| introduced
\item
|\childdoc| renamed to |\childdocmain|
\item
|\childredirect| renamed to |\childdocforward| and |\childdocforwardprefix|
and functionality expanded
\end{itemize}

%%%%%%%%%%%%%%%%%%%%%%%%%%%%%%%%%%%%%%%%
\paragraph{v1.0:} 2017/04/27

\begin{itemize}
\item
manual and install package
\item
first version published on CTAN
\end{itemize}

%%%%%%%%%%%%%%%%%%%%%%%%%%%%%%%%%%%%%%%%
\paragraph{v0.6:} 2017/04/26

\begin{itemize}
\item
redirection mechanism added
\end{itemize}

%%%%%%%%%%%%%%%%%%%%%%%%%%%%%%%%%%%%%%%%
\paragraph{v0.5:} 2017/04/26

\begin{itemize}
\item
functionality in definition file
\end{itemize}


%%%%%%%%%%%%%%%%%%%%%%%%%%%%%%%%%%%%%%%%%%%%%%%%%%%%%%%%%%%%%%%%%%%%%%%%%%%%%%%%
%%%%%%%%%%%%%%%%%%%%%%%%%%%%%%%%%%%%%%%%%%%%%%%%%%%%%%%%%%%%%%%%%%%%%%%%%%%%%%%%
%%%%%%%%%%%%%%%%%%%%%%%%%%%%%%%%%%%%%%%%%%%%%%%%%%%%%%%%%%%%%%%%%%%%%%%%%%%%%%%%
\appendix

\settowidth\MacroIndent{\rmfamily\scriptsize 000\ }

 \DocInput{childdoc.dtx}

\end{document}
%</driver>
% \fi
%
% %%%%%%%%%%%%%%%%%%%%%%%%%%%%%%%%%%%%%%%%%%%%%%%%%%%%%%%%%%%%%%%%%%%%%%%%%%%%%%
% %%%%%%%%%%%%%%%%%%%%%%%%%%%%%%%%%%%%%%%%%%%%%%%%%%%%%%%%%%%%%%%%%%%%%%%%%%%%%%
% \section{Sample}
%\iffalse
%<*samplemain>
%\fi
%
% The following presents a sample document
% with two chapters, two parts, a title page,
% a compile flag as well as three forwarding files to set the flag.
% It consists of eight |.tex| files:
% \begin{center}
% \begin{tabular}{ll}
% |cdocsamp.tex|&main file\\
% |cdocsch1.tex|&include file for chapter 1\\
% |cdocsch2.tex|&include file for chapter 2\\
% |cdocspt3.tex|&include file for part 3\\
% |cdocspt4.tex|&include file for part 4\\
% |cdocsdrf.tex|&forwarding file for main file in draft mode\\
% |cdocsfi1.tex|&forwarding file for final version of chapter 1\\
% |cdocsfi2.tex|&forwarding file for final version of chapter 2\\
% \end{tabular}
% \end{center}
% Each of the eight files can be compiled directly by the \LaTeX{} compiler.
%
% %%%%%%%%%%%%%%%%%%%%%%%%%%%%%%%%%%%%%%
% \paragraph{Main File.}
%
% The main file is called |cdocsamp.tex|.
%
% Load the \textsf{childdoc} definitions and
% declare the filename for the main document:
%    \begin{macrocode}
\input{childdoc.def}
\childdocmain{}
%    \end{macrocode}

% Optional override for |\version| flag:
%    \begin{macrocode}
%%\ifchilddoc\else\providecommand{\version}{draft}\fi
%    \end{macrocode}

% Define the default values for the |\version| flag
% (|final| for the main file and |draft| for childs):
%    \begin{macrocode}
\ifchilddoc
\providecommand{\version}{draft}
\else
\providecommand{\version}{final}
\fi
%    \end{macrocode}

% Load the standard document class:
%    \begin{macrocode}
\documentclass[12pt]{article}
%    \end{macrocode}

% Start the document body:
%    \begin{macrocode}
\begin{document}
%    \end{macrocode}

% Declare a title page.
% Print title, part of document being processed and version flag:
%    \begin{macrocode}
\addtocounter{page}{-1}
\begin{center}
{\LARGE\bfseries{}childdoc example\par}
\vspace{1cm}
\ifchilddoc
\ifchilddocmanual part\else chapter\fi:
`\childdocname' of `\childdocjob'\par
\else
main document: `\childdocjob'\par
\fi
version: \version\par
\end{center}
\newpage
%    \end{macrocode}

% Manually include selected file,
% otherwise process as usual:
%    \begin{macrocode}
\ifchilddocmanual
\section*{part `\childdocname'}
\input{\childdocname}
\else
%    \end{macrocode}

% Include the two chapters:
%    \begin{macrocode}
\include{cdocsch1}
\include{cdocsch2}
%    \end{macrocode}

% Include the two parts unless only chapters should be displayed:
%    \begin{macrocode}
\ifchilddoc\else
\section{part three}
\input{cdocspt3}
\section{part four}
\input{cdocspt4}
\fi
%    \end{macrocode}

% Process as usual until here:
%    \begin{macrocode}
\fi
%    \end{macrocode}

% End of document body:
%    \begin{macrocode}
\end{document}
%    \end{macrocode}
%\iffalse
%</samplemain>
%\fi
%
% %%%%%%%%%%%%%%%%%%%%%%%%%%%%%%%%%%%%%%
% \paragraph{Chapter Include Files.}
%
% The include files are called |cdocsch1.tex| and |cdocsch2.tex|.
%
%\iffalse
%<*samplechap1|samplechap2>
%\fi

% Optional override for |\version| flag:
%    \begin{macrocode}
%%\providecommand{\version}{final}
%    \end{macrocode}

% Include the main document:
%    \begin{macrocode}
\input{childdoc.def}
\childdocof{cdocsamp}
%    \end{macrocode}

%\iffalse
%</samplechap1|samplechap2>
%\fi
%
%\iffalse
%<*samplechap1>
%\fi
% Some text for chapter 1:
%    \begin{macrocode}
\section{one}
some text in chapter one
%    \end{macrocode}

%\iffalse
%</samplechap1>
%\fi
% Some text for chapter 2:
%\iffalse
%<*samplechap2>
%\fi
%    \begin{macrocode}
\section{two}
more text in chapter two
%    \end{macrocode}

%\iffalse
%</samplechap2>
%\fi
%
% %%%%%%%%%%%%%%%%%%%%%%%%%%%%%%%%%%%%%%
% \paragraph{Part Include Files.}
%
% The include files are called |cdocspt3.tex| and |cdocspt4.tex|.
%
%\iffalse
%<*samplepart3|samplepart4>
%\fi

% Optional override for |\version| flag:
%    \begin{macrocode}
%%\providecommand{\version}{final}
%    \end{macrocode}

% Include the main document:
%    \begin{macrocode}
\input{childdoc.def}
\childdocby{cdocsamp}
%    \end{macrocode}

%\iffalse
%</samplepart3|samplepart4>
%\fi
%
%\iffalse
%<*samplepart3>
%\fi
% Some text for part 3:
%    \begin{macrocode}
some text in part three
%    \end{macrocode}

%\iffalse
%</samplepart3>
%\fi
% Some text for part 4:
%\iffalse
%<*samplepart4>
%\fi
%    \begin{macrocode}
more text in part four
%    \end{macrocode}

%\iffalse
%</samplepart4>
%\fi
%
% %%%%%%%%%%%%%%%%%%%%%%%%%%%%%%%%%%%%%%
% \paragraph{Forwarding for a Complete Draft.}
%
% The following forwarding file |cdocsdrf.tex|
% compiles the main document in draft mode:
%\iffalse
%<*sampledraft>
%\fi
%    \begin{macrocode}
\def\version{draft}
\input{childdoc.def}
\childdocforward{cdocsamp}
%    \end{macrocode}

%\iffalse
%</sampledraft>
%\fi
%
% %%%%%%%%%%%%%%%%%%%%%%%%%%%%%%%%%%%%%%
% \paragraph{Forwarding for Final Version of the Chapters.}
%
% The following forwarding files |cdocsfn1.tex| and |cdocsfn2.tex|
% (with identical content)
% compile the final versions of the child documents
% |cdocsch1.tex| and |cdocsch2.tex|, respectively:
%\iffalse
%<*samplefinal>
%\fi
%    \begin{macrocode}
\def\version{final}
\input{childdoc.def}
\childdocforwardprefix[cdocsamp]{cdocsfn}{cdocsch}
%    \end{macrocode}

%\iffalse
%</samplefinal>
%\fi
%
% %%%%%%%%%%%%%%%%%%%%%%%%%%%%%%%%%%%%%%
% \paragraph{Command Line Processing.}
%
% The following three command lines generate the output files
% |cdocscld|, |cdocscl1| and |cdocscl2|
% which should be identical to
% |cdocsdrf|, |cdocsch1| and |cdocsfn2|, respectively:
% \begin{center}
% \begin{tabular}{l}
% |latex -jobname cdocscld \|\\
% |  "\def\version{draft}\input{childdoc.def}\childdocforward{cdocsamp}"|\\
% |latex -jobname cdocscl1 \|\\
% |  "\input{childdoc.def}\childdocforward[cdocsamp]{cdocsch1}"|\\
% |latex -jobname cdocscl2 \|\\
% |  "\def\version{final}\input{childdoc.def}\childdocforward{cdocsch2}"|
% \end{tabular}
% \end{center}
% Note that the trailing backslash on each first line
% merely continues the input to the second line
% (for convenient cut ant paste).
% Furthermore, the command |latex| can be replaced by any
% of its alternative versions such as |pdflatex|.
%
% %%%%%%%%%%%%%%%%%%%%%%%%%%%%%%%%%%%%%%%%%%%%%%%%%%%%%%%%%%%%%%%%%%%%%%%%%%%%%%
% %%%%%%%%%%%%%%%%%%%%%%%%%%%%%%%%%%%%%%%%%%%%%%%%%%%%%%%%%%%%%%%%%%%%%%%%%%%%%%
% \section{Implementation}
%\iffalse
%<*package>
%\fi
%
% This section describes the definitions file |childdoc.def|.

% The definitions cannot be loaded using |\usepackage| or |\RequirePackage|
% which has a mechanism to prevent loading a style file more than once.
% When loading the definitions by means of |\input|
% multiple instances have to be prevented manually:
%\iffalse
%This code needs to be before the `\ProvidesFile' directive
%which is defined at the beginning of this file.
%Therefore it is also placed there and commented out here.
%</package>
%<*discard>
%\fi
%    \begin{macrocode}
\ifdefined\childdocmain\endinput\fi
%    \end{macrocode}
%\iffalse
%</discard>
%<*package>
%\fi
%
% \macro{\ifchilddoc}
% \macro{\ifchilddocmanual}
% The conditional |\ifchilddoc| tells whether a
% child (true) or main (false) document is being compiled.
% The conditional |\ifchilddocmanual| tells whether
% the |\includeonly| mechanism is used (false) or
% the selection of child files must be performed manually (true).
% The definitions initialise to false:
%    \begin{macrocode}
\newif\ifchilddoc
\newif\ifchilddocmanual
%    \end{macrocode}

% \macro{\childdocname}
% \macro{\childdocjob}
% The macro |\childdocname| stores the name of the main document
% to be compiled. The macro |\childdocjob| stores the name of
% the document on which the \LaTeX{} compiler was originally invoked.
% The content of |\jobname| cannot be compared
% to filenames specified in the source due to different catcodes.
% The following code rescans |\jobname|, stores the result
% in |\childdocname| and saves a copy in |\childdocjob|:
%    \begin{macrocode}
\edef\childdocname{\scantokens\expandafter{\jobname\noexpand}}
\let\childdocjob\childdocname
%    \end{macrocode}

% \macro{\childdocdisable}
% The macro |\childdocdisable| prevents the main file
% from being processed more than once.
% At this stage, the main document command |\childdocmain|
% is assumed to be called once again where it should do nothing.
% Any subsequent call to it should prevent
% a secondary processing of the main document
% It overwrites the forwarding commands
% |\childdocof| and |\childdocforward|
% with empty macros to prevent further inclusions of the main document:
%    \begin{macrocode}
\newcommand{\childdocdisable}
{
  \renewcommand{\childdocmain}[1]{\renewcommand{\childdocmain}[1]{\endinput}}
  \renewcommand{\childdocof}[1]{}
  \renewcommand{\childdocby}[2][]{}
  \renewcommand{\childdocforward}[2][]{}
  \renewcommand{\childdocdisable}{}
}
%    \end{macrocode}

% \macro{\childdocmain}
% The macro |\childdocmain| is to be called at the top of the main file
% with nothing or the main filename (without extension) as argument.
% First, it breaks loops.
% If the argument is not empty and does not match |\childdocname|
% (which is set by the first inclusion of |childdoc.def|),
% |\ifchilddoc| is set to true, |\includeonly| is applied to the child file
% and |\jobname| is set to the main file
% (for proper handling of |.aux| files):
%    \begin{macrocode}
\newcommand{\childdocmain}[1]
{
  \childdocdisable\childdocmain{}
  \if?#1?\else
    \begingroup
      \def\childdoctmp{#1}
      \ifx\childdoctmp\childdocname
        \def\childdoctmp{}
      \else
        \def\childdoctmp
        {
          \childdoctrue
          \includeonly{\childdocname}
          \def\childdocjob{#1}
          \def\jobname{#1}
        }
      \fi
      \expandafter
    \endgroup
    \childdoctmp
  \fi
}
%    \end{macrocode}

% \macro{\childdocof}
% The command |\childdocof| redirects
% compilation to the main file |#1|.
%    \begin{macrocode}
\newcommand{\childdocof}[1]
{
  \childdocdisable
  \childdoctrue
  \includeonly{\childdocname}
  \def\jobname{#1}
  \def\childdocjob{#1}
  \input{#1}
}
%    \end{macrocode}

% \macro{\childdocby}
% The command |\childdocby| ....
%    \begin{macrocode}
\newcommand{\childdocby}[2][]
{
  \childdocdisable
  \childdoctrue
  \childdocmanualtrue
  \if?#1?\else
    \def\jobname{#2}
  \fi
  \def\childdocjob{#2}
  \input{#2}
  \endinput
}
%    \end{macrocode}

% \macro{\childdocforward}
% The command |\childdocforward| redirects
% compilation to the main file or
% (if the optional argument is given) a child file.
% Parameters are set as if the main file
% or a child file starting with |\childdocof| was compiled.
% Then compilation is handed over to the main file:
%    \begin{macrocode}
\newcommand{\childdocforward}[2][]
{
  \begingroup
    \if?#1?
      \def\childdoctmp
      {
        \def\childdocname{#2}
        \def\childdocjob{#2}
        \def\jobname{#2}
        \input{#2}
        \endinput
      }
    \else
      \def\childdoctmp
      {
        \childdocdisable
        \def\childdocname{#2}
        \childdoctrue
        \includeonly{#2}
        \def\childdocjob{#1}
        \def\jobname{#1}
        \input{#1}
        \endinput
      }
    \fi
    \expandafter
  \endgroup
  \childdoctmp
}
%    \end{macrocode}

% \macro{\childdocforwardprefix}
% The command |\childdocforwardprefix| redirects
% compilation to the main or a child file by means of a pattern.
% The prefix |#1| in the current filename is replaced by |#2|
% and the suffix of the current filename is kept
% (it is assumed that the filename does not contain the substring `|~~~|'
% which is used as a delimiter).
% Compilation is handed over to the new file by |\childdocforward|:
%    \begin{macrocode}
\newcommand{\childdocforwardprefix}[3][]
{
  \begingroup
    \def\childdocextract #2##1~~~{\def\childdoctmp{\childdocforward[#1]{#3##1}}}
    \expandafter\childdocextract\childdocname~~~
    \expandafter
  \endgroup
  \childdoctmp
}
%    \end{macrocode}

% \macro{\childdoc}
% The deprecated macro |\childdoc| is a legacy version of |\childdocmain|:
%    \begin{macrocode}
\newcommand{\childdoc}{\childdocmain}
%    \end{macrocode}

% \macro{\childdocredirect}
% The deprecated macro |\childdocredirect| is a legacy version
% of |\childdocforward| and |\childdocforwardprefix|:
%    \begin{macrocode}
\newcommand{\childdocredirect}[2][]
{
  \begingroup
    \if?#1?
      \def\childdoctmp{\childdocforward{#2}}
    \else
      \def\childdoctmp{\childdocforwardprefix{#1}{#2}}
    \fi
    \expandafter
  \endgroup
  \childdoctmp
}
%    \end{macrocode}

%\iffalse
%</package>
%\fi
%
\endinput

\childdocmain{}
%    \end{macrocode}

% Optional override for |\version| flag:
%    \begin{macrocode}
%%\ifchilddoc\else\providecommand{\version}{draft}\fi
%    \end{macrocode}

% Define the default values for the |\version| flag
% (|final| for the main file and |draft| for childs):
%    \begin{macrocode}
\ifchilddoc
\providecommand{\version}{draft}
\else
\providecommand{\version}{final}
\fi
%    \end{macrocode}

% Load the standard document class:
%    \begin{macrocode}
\documentclass[12pt]{article}
%    \end{macrocode}

% Start the document body:
%    \begin{macrocode}
\begin{document}
%    \end{macrocode}

% Declare a title page.
% Print title, part of document being processed and version flag:
%    \begin{macrocode}
\addtocounter{page}{-1}
\begin{center}
{\LARGE\bfseries{}childdoc example\par}
\vspace{1cm}
\ifchilddoc
\ifchilddocmanual part\else chapter\fi:
`\childdocname' of `\childdocjob'\par
\else
main document: `\childdocjob'\par
\fi
version: \version\par
\end{center}
\newpage
%    \end{macrocode}

% Manually include selected file,
% otherwise process as usual:
%    \begin{macrocode}
\ifchilddocmanual
\section*{part `\childdocname'}
\input{\childdocname}
\else
%    \end{macrocode}

% Include the two chapters:
%    \begin{macrocode}
\include{cdocsch1}
\include{cdocsch2}
%    \end{macrocode}

% Include the two parts unless only chapters should be displayed:
%    \begin{macrocode}
\ifchilddoc\else
\section{part three}
\input{cdocspt3}
\section{part four}
\input{cdocspt4}
\fi
%    \end{macrocode}

% Process as usual until here:
%    \begin{macrocode}
\fi
%    \end{macrocode}

% End of document body:
%    \begin{macrocode}
\end{document}
%    \end{macrocode}
%\iffalse
%</samplemain>
%\fi
%
% %%%%%%%%%%%%%%%%%%%%%%%%%%%%%%%%%%%%%%
% \paragraph{Chapter Include Files.}
%
% The include files are called |cdocsch1.tex| and |cdocsch2.tex|.
%
%\iffalse
%<*samplechap1|samplechap2>
%\fi

% Optional override for |\version| flag:
%    \begin{macrocode}
%%\providecommand{\version}{final}
%    \end{macrocode}

% Include the main document:
%    \begin{macrocode}
% \iffalse
%
% childdoc.dtx Copyright (C) 2017-2018 Niklas Beisert
%
% This work may be distributed and/or modified under the
% conditions of the LaTeX Project Public License, either version 1.3
% of this license or (at your option) any later version.
% The latest version of this license is in
%   http://www.latex-project.org/lppl.txt
% and version 1.3 or later is part of all distributions of LaTeX
% version 2005/12/01 or later.
%
% This work has the LPPL maintenance status `maintained'.
%
% The Current Maintainer of this work is Niklas Beisert.
%
% This work consists of the files childdoc.dtx and childdoc.ins
% and the derived files childdoc.def and cdocsamp.tex with
% cdocsch1.tex, cdocsch2.tex, cdocsdrf.tex, cdocsfn1.tex, cdocsfn2.tex.
%
%<package>\ifdefined\childdocmain\endinput\fi
%<package>\ProvidesFile{childdoc.def}[2018/12/30 v2.0 child document driver]
%<samplemain>\ProvidesFile{cdocsamp.tex}[2018/12/30 v2.0 sample for childdoc]
%<*driver>
%\ProvidesFile{childdoc.drv}[2018/12/30 v2.0 childdoc reference manual file]
\PassOptionsToClass{10pt,a4paper}{article}
\documentclass{ltxdoc}

\usepackage[margin=35mm]{geometry}
\usepackage{hyperref}
\usepackage{hyperxmp}
\usepackage[usenames]{color}

\hypersetup{colorlinks=true}
\hypersetup{pdfstartview=FitH}
\hypersetup{pdfpagemode=UseNone}
\hypersetup{pdfsource={}}
\hypersetup{pdflang={en-UK}}
\hypersetup{pdfcopyright={Copyright 2017-2018 Niklas Beisert.
  This work may be distributed and/or modified under the
  conditions of the LaTeX Project Public License, either version 1.3
  of this license or (at your option) any later version.}}
\hypersetup{pdflicenseurl={http://www.latex-project.org/lppl.txt}}
\hypersetup{pdfcontactaddress={ETH Zurich, ITP, HIT K,
  Wolfgang-Pauli-Strasse 27}}
\hypersetup{pdfcontactpostcode={8093}}
\hypersetup{pdfcontactcity={Zurich}}
\hypersetup{pdfcontactcountry={Switzerland}}
\hypersetup{pdfcontactemail={nbeisert@itp.phys.ethz.ch}}
\hypersetup{pdfcontacturl={http://people.phys.ethz.ch/\xmptilde nbeisert/}}

\newcommand{\secref}[1]{\hyperref[#1]{section \ref*{#1}}}

\parskip1ex
\parindent0pt
\let\olditemize\itemize
\def\itemize{\olditemize\parskip0pt}

\begin{document}

\title{The \textsf{childdoc} Package}
\hypersetup{pdftitle={The childdoc Package}}
\author{Niklas Beisert\\[2ex]
  Institut f\"ur Theoretische Physik\\
  Eidgen\"ossische Technische Hochschule Z\"urich\\
  Wolfgang-Pauli-Strasse 27, 8093 Z\"urich, Switzerland\\[1ex]
  \href{mailto:nbeisert@itp.phys.ethz.ch}
  {\texttt{nbeisert@itp.phys.ethz.ch}}}
\hypersetup{pdfauthor={Niklas Beisert}}
\hypersetup{pdfsubject={Manual for the LaTeX2e Package childdoc}}
\date{30 December 2018, \textsf{v2.0}}
\maketitle

\begin{abstract}\noindent
\textsf{childdoc} is a \LaTeXe{} package
that enables the direct compilation
of document sections included by |\include|
to individual files.
\end{abstract}

\begingroup
\parskip0ex
\tableofcontents
\endgroup

%%%%%%%%%%%%%%%%%%%%%%%%%%%%%%%%%%%%%%%%%%%%%%%%%%%%%%%%%%%%%%%%%%%%%%%%%%%%%%%%
%%%%%%%%%%%%%%%%%%%%%%%%%%%%%%%%%%%%%%%%%%%%%%%%%%%%%%%%%%%%%%%%%%%%%%%%%%%%%%%%
\section{Introduction}

\LaTeX{} provides a mechanism to structure a large document (such as a book)
into a main file and several child files (containing the chapters)
using the |\include| command.
This mechanism is beneficial for documents
which span hundreds of pages in order to
make the source file(s) more manageable.
Moreover, compilation can be restricted to
selected child files by means of the |\includeonly| command.
The latter feature can be used to reduce the compilation time while editing
(this was significantly more useful in the earlier days of \LaTeX{})
or to generate a smaller document which is easier to navigate.
Another application of |\includeonly| is to generate
documents consisting of selected parts of the complete document.

However, there are a few drawbacks of the plain |\include| mechanism:
\begin{itemize}
\item
The child files cannot be compiled on their own,
they can only be compiled via the main file.
A naive editing environment
(such as a text editor with an option
to have the current file processed by \LaTeX)
may require one to switch to the main file before compiling;
attempting to compile the child file produces errors.
\item
The main file must be modified (each time)
to adjust the |\includeonly| command
to the present needs. This easily leaves the main file in a messy state.
\item
The generated document will always carry the filename
of the main document. This is inconvenient if
several child files are to be compiled and
to be kept for distribution.
\end{itemize}

The present package provides a simple interface
to make child files individually compilable by \LaTeX{}.
Compiling a child file then has the same effect as compiling
the main file with an |\includeonly| command
to select the appropriate child.
Moreover the generated document will carry the name of the child
rather than the main file.
This resolves all three above issues.

This feature is meant to make the editing of books,
thesis documents and lecture notes somewhat more convenient.
However, the package can also be used efficiently for
composing a series of documents (such as exercise sheets)
which are typically distributed individually.
It then assists the author in generating the individual documents
(potentially in different versions)
as well as a document containing the collected series.
Another application is in developing style files
or other kinds of included material
where compilation of the style file could redirect
to a sample or test file.

%%%%%%%%%%%%%%%%%%%%%%%%%%%%%%%%%%%%%%%%%%%%%%%%%%%%%%%%%%%%%%%%%%%%%%%%%%%%%%%%
%%%%%%%%%%%%%%%%%%%%%%%%%%%%%%%%%%%%%%%%%%%%%%%%%%%%%%%%%%%%%%%%%%%%%%%%%%%%%%%%
\section{Usage}

First of all, the package \textsf{childdoc} is \emph{not} a standard
\LaTeXe{} |.sty| style file! Therefore it needs to be invoked in
a non-standard way.

%%%%%%%%%%%%%%%%%%%%%%%%%%%%%%%%%%%%%%%%%%%%%%%%%%%%%%%%%%%%%%%%%%%%%%%%%%%%%%%%
\subsection{Included Files}
\label{sec:include}

%%%%%%%%%%%%%%%%%%%%%%%%%%%%%%%%%%%%%%%%
\DescribeMacro{\childdocmain}
To use the package, add the commands
\begin{center}
\begin{tabular}{l}
|\input{childdoc.def}|\\
|\childdocmain{}|\\
\end{tabular}
\end{center}
at the very top of the main \LaTeX{} file,
in particular \emph{before} the |\documentclass| statement!
The argument of |\childdocmain| should be left empty
(but it must be present).

%%%%%%%%%%%%%%%%%%%%%%%%%%%%%%%%%%%%%%%%
\DescribeMacro{\childdocof}
Furthermore, add the commands
\begin{center}
\begin{tabular}{l}
|\input{childdoc.def}|\\
|\childdocof{|\textit{main}|}|\\
\end{tabular}
\end{center}
at the top of every child file \textit{child}
which is included by |\include{|\textit{child}|}|
from within the main file
(or at least for those files to be compiled individually).
The argument \textit{main} must be the filename of the main file.

There are a couple of
considerations in setting up the main and child documents:

%%%%%%%%%%%%%%%%%%%%%%%%%%%%%%%%%%%%%%%%
\paragraph{Restrictions.}

Please note the following restrictions:
\begin{itemize}
\item
|\childdocmain| must be called with one argument \textit{main}
to ensure compatibility with earlier version of the package.
It must either be empty (|\childdocmain{}|)
or precisely match the filename of the main file in which it is specified.
See \secref{sec:detection} for further information.
\item
The filename \textit{main} must be specified without the |.tex| extension.
\item
The filename \textit{main} is case sensitive
(even in case-insensitive file systems)
due to internal string comparison.
\item
The argument \textit{main} should be fully expanded, it cannot be a macro.
\item
Subdirectories and special characters should be avoided in filenames.
\item
The command |\childdocmain{|\textit{main}|}| must be followed by a whitespace.
It should not be followed immediately by another command
or by a comment mark `|%|'.
This is because the \TeX{} parser reads the token immediately following
the argument of |\childdocmain| and puts it
at the beginning of every child section;
however, a white\-space is ignored.
\end{itemize}

%%%%%%%%%%%%%%%%%%%%%%%%%%%%%%%%%%%%%%%%
\paragraph{Content of Main File.}

It is advisable to place all content in the child files included by |\include|.
Any output contained in the main file will appear in all child documents
unless suppressed manually;
it cannot be suppressed automatically by the |\includeonly| directive
and thus should normally be avoided.
A method to include some content in the main file
by means of conditional processing is described in \secref{sec:conditional}.

%%%%%%%%%%%%%%%%%%%%%%%%%%%%%%%%%%%%%%%%
\paragraph{Page Numbering.}

When only a part of the document is compiled,
the appropriate numbering of pages
(as well as other status parameters)
is determined from the |.aux| files.
The latter contain information from previous passes.
However this information needs to propagate through
all intermediate child documents.
Therefore the page numbering in child documents may well
be inconsistent until the complete document is compiled at least once.

A useful (if unconventional) way to always ensure a consistent
page numbering is to restart the numbering in each child document
and denote the pages by `\textit{child}|.|\textit{page}'
where \textit{child} represents the chapter/section number of the child file.
This can be achieved by the command
|\numberwithin{page}{|\textit{child}|}|
of the \textsf{amsmath} package
where \textit{child} can be |chapter| or |section|
depending on the chosen structuring.
Alternatively, one can modify the macro |\thepage| appropriately
and reset the counter |page| at the start of each child file.

%%%%%%%%%%%%%%%%%%%%%%%%%%%%%%%%%%%%%%%%%%%%%%%%%%%%%%%%%%%%%%%%%%%%%%%%%%%%%%%%
\subsection{Conditional Processing}
\label{sec:conditional}

The package provides a mechanism to compile different versions
of a document. To customise the versions further some conditional processing
can come in handy to distinguish which version is being compiled.
The package provides two macros to describe the compilation context:

%%%%%%%%%%%%%%%%%%%%%%%%%%%%%%%%%%%%%%%%
\DescribeMacro{\ifchilddoc}
The conditional |\ifchilddoc| distinguishes between the compilation of
child documents and the main document:
%
\begin{center}
|\ifchilddoc |\textit{child-code}| |[|\||else |\textit{main-code}]| \||fi|
\end{center}

%%%%%%%%%%%%%%%%%%%%%%%%%%%%%%%%%%%%%%%%
\DescribeMacro{\childdocname}
\DescribeMacro{\childdocjob}
The macro |\childdocname| contains the filename (without extension)
of the main or child file being processed.
Note that |\childdocjob| will always contain the name of the main file.

%%%%%%%%%%%%%%%%%%%%%%%%%%%%%%%%%%%%%%%%
\paragraph{Title Page.}

Conditional processing can be used to include a title or banner page
in the main document when proper precautions are taken.
Importantly, the code in the main file should ensure that the page counter
(as well as other status parameters which are stored in the |.aux| files)
takes the same value after the conditional processing.
Otherwise the page numbers may take divergent values
depending on which part is compiled.

For example, a title page could be declared by:
%
\begin{center}
\begin{tabular}{l}
|\ifchilddoc\||else|\\
|\addtocounter{page}{-1}|\\
\textit{code for title page}\\
|\newpage|\\
|\||fi|
\end{tabular}
\end{center}
%
A banner page for the child documents can be generated by:
%
\begin{center}
\begin{tabular}{l}
|\ifchilddoc|\\
|\addtocounter{page}{-1}|\\
\textit{code for banner page}\\
|\newpage|\\
|\||fi|
\end{tabular}
\end{center}
%
Here one could write a message such as:
\begin{center}
|This is the part \childdocname{} of \childdocjob{}.|
\end{center}

%%%%%%%%%%%%%%%%%%%%%%%%%%%%%%%%%%%%%%%%%%%%%%%%%%%%%%%%%%%%%%%%%%%%%%%%%%%%%%%%
\subsection{Flags}
\label{sec:flags}

The package makes it easy to generate different versions
of the main or child documents.
To this end compilation flags can be defined
and assigned different default values.
They will be particularly useful in conjunction
with the forwarding mechanism described in \secref{sec:forward}.

For example, it may be useful to have a flag |\version|
which can be set to |draft| or |final|.
The document source will contain some conditional code
depending on the value of |\version|.
Suppose further, the flag should default to |final| for the main file
and to |draft| for child files
which is a natural assignment for editing the document.
This is achieved by placing the following code
in the preamble of the main document
(below the |\childdocmain| directive):
%
\begin{center}
\begin{tabular}{l}
|\ifchilddoc|\\
|\providecommand{\version}{draft}|\\
|\||else|\\
|\providecommand{\version}{final}|\\
|\||fi|
\end{tabular}
\end{center}
%
The definition by |\providecommand| makes sure
that previous definitions are not overwritten.
Further statements |\providecommand{\version}{...}|
can thus be added before the above code to override it.

For the main file, one might add a line
(between |\childdocmain| and the above block)
%
\begin{center}
|%\ifchilddoc\||else\providecommand{\version}{draft}\||fi|
\end{center}
%
which can be uncommented to produce a draft version.
Likewise one can add a line to the very top of a child file
(above the |\childdocof{|\textit{main}|}| directive)
%
\begin{center}
|%\providecommand{\version}{final}|
\end{center}
%
which can be uncommented to produce the final version of this child document.

%%%%%%%%%%%%%%%%%%%%%%%%%%%%%%%%%%%%%%%%%%%%%%%%%%%%%%%%%%%%%%%%%%%%%%%%%%%%%%%%
\subsection{Forwarding}
\label{sec:forward}

Different versions of the main or child documents
using compilation flags as described in \secref{sec:flags}
can be (permanently) stored in different files
for convenient compilation, viewing and distribution.
To this end, the package defines a command
to pass on compilation to a different file:

%%%%%%%%%%%%%%%%%%%%%%%%%%%%%%%%%%%%%%%%
\DescribeMacro{\childdocforward}
The command |\childdocforward| redirects processing to
another source file:
%
\begin{center}
\begin{tabular}{l}
|\input{childdoc.def}|\\
|\childdocforward[|\textit{main}|]{|\textit{dest}|}|\\
\end{tabular}
\end{center}
%
The argument \textit{dest} is the destination file
(without extension).
It should be the main file or one of the child files.
Note that further \textsf{childdoc} directives
such as |\childdocof| and |\childdocforward|
in the indicated file will be processed in this form.
The optional argument \textit{main}
passes on directly to the main file \textit{main}
while pretending to compile the child \textit{dest}.
This form behaves as if \textit{dest}
issues |\childdocof{|\textit{main}|}| right away,
and no further \textsf{childdoc} directives will be processed.

%%%%%%%%%%%%%%%%%%%%%%%%%%%%%%%%%%%%%%%%
\DescribeMacro{\...prefix}
In the alternative form |\childdocforwardprefix|,
%
\begin{center}
\begin{tabular}{l}
|\input{childdoc.def}|\\
|\childdocforwardprefix[|\textit{main}|]{|\textit{prefix}|}{|\textit{dest}|}|
\end{tabular}
\end{center}
%
the destination file is determined by a pattern
depending on the current file:
To make this work, the current file must be called
`{\textit{prefix}\hspace{0.2em}\textit{suffix}}'
with \textit{prefix} matching precisely the argument.
Processing is then passed on to the file
`{\textit{dest}\hspace{0.2em}\textit{suffix}}'.
Surely, the same effect is achieved by
directly specifying the
argument `{\textit{dest}\hspace{0.2em}\textit{suffix}}'
in the first form.
However, that requires to set up a different file
for each child. With the alternative form of the command
all these files can have exactly the same content
which simplifies setting them up and maintaining them.

For example, the following file |draft.tex|
with a compilation flag |\version| as described in \secref{sec:flags}
compiles the main document as a draft:
%
\begin{center}
\begin{tabular}{l}
|\def\version{draft}|\\
|\input{childdoc.def}|\\
|\childdocforward{|\textit{main}|}|
\end{tabular}
\end{center}
%
Likewise, the following files |final|\textit{nn}|.tex|
compile the final version of the child document
|child|\textit{nn}|.tex|:
%
\begin{center}
\begin{tabular}{l}
|\def\version{final}|\\
|\input{childdoc.def}|\\
|\childdocforwardprefix{final}{child}|
\end{tabular}
\end{center}
%

Note that when several versions of a main file and/or of each child file
are to be generated, it may be convenient to set up a |Makefile| or
shell script to automatise the process.

%%%%%%%%%%%%%%%%%%%%%%%%%%%%%%%%%%%%%%%%%%%%%%%%%%%%%%%%%%%%%%%%%%%%%%%%%%%%%%%%
\subsection{Command Line Processing}
\label{sec:commandline}

The effect of redirection files can also be achieved by invoking
the \LaTeX{} compiler with a more elaborate command line.
Most conveniently this should be done as part
of a shell script or a |Makefile|.

When using \textsf{childdoc} in the main file, the following
command lines effectively perform a redirection
(note that depending on the shell being used,
backslashes may have to be doubled: `|\|' $\to$ `|\\|'):
%
\begin{center}
|... -jobname "|\textit{target}|" |\\|"|[\textit{flags}]%
|\input{childdoc.def}\childdocforward[|\textit{main}|]{|\textit{dest}|}"|
\end{center}
%
Here \textit{target} is the name of the output file,
\textit{main} is the name of the main file
and \textit{dest} is the name of the main or child file to be processed
(all filenames without extensions).
The optional argument \textit{main} can be omitted
if \textit{main} matches \textit{dest}.
Optionally, compilation \textit{flags} can be defined via |\def| commands.
This command line makes the \TeX{} engine believe
it is compiling the file \textit{target}
whose content is specified as the latter parameter.
The provided code then forwards the processing to
\textit{main} or \textit{dest} as described in \secref{sec:forward}.

%%%%%%%%%%%%%%%%%%%%%%%%%%%%%%%%%%%%%%%%%%%%%%%%%%%%%%%%%%%%%%%%%%%%%%%%%%%%%%%%
\subsection{Include by Input}
\label{sec:input}

Including child documents by |\include| has some restrictions by design.
Most notably, the content of a child document always occupies
its own set of pages; pages cannot be shared between child documents.
Usually, this behaviour makes perfect sense
because each child document contain an essential part of the document.
However, in some situations it may be desirable to compose
a document from a collection of parts
without having mandatory page breaks between then.
For this case, the package
provides a mechanism to include parts
by |\input| which can also be processed individually.
However, by construction this mechanism
requires manual handling of the content to be output.

%%%%%%%%%%%%%%%%%%%%%%%%%%%%%%%%%%%%%%%%
\DescribeMacro{\ifchilddocmanual}
The main file should be prepared as usual, see \secref{sec:include}.
However, the document body must make a distinction
between processing of an individual part and of the main document, e.g.:
%
\begin{center}
\begin{tabular}{l}
|\ifchilddocmanual|\\
|\input{\childdocname}|\\
|\||else|\\
\textit{document body with }|\input{|\textit{part}|}|\\
|\||fi|
\end{tabular}
\end{center}
%
The conditional |\ifchilddocmanual| is true whenever
a part to be included by |\input| is being compiled,
and the name of the part is stored in |\childdocname|.

%%%%%%%%%%%%%%%%%%%%%%%%%%%%%%%%%%%%%%%%
\DescribeMacro{\childdocby}
Each part to be included by |\input| should start with:
%
\begin{center}
\begin{tabular}{l}
|\input{childdoc.def}|\\
|\childdocby{|\textit{main}|}|\\
\end{tabular}
\end{center}
%
The directive |\childdocby| is similar to |\childdocof|
described in \secref{sec:include},
but the subsequent selection of content must be done manually.
To that end, both |\ifchilddoc| and |\ifchilddocmanual|
will be true upon processing of a part,
and the name of the part is stored in |\childdocname|.
Note that |\jobname| will be set to the filename of the current part
so that each part receives an individual |.aux| file
that does not interfere with the |.aux| file(s) of the main document.
This behaviour can be altered by the alternative form
|\childdocby[*]{|\textit{main}|}| (with a non-empty optional argument)
which uses the |.aux| file of the main document
by setting |\jobname| to \textit{main}.

%%%%%%%%%%%%%%%%%%%%%%%%%%%%%%%%%%%%%%%%%%%%%%%%%%%%%%%%%%%%%%%%%%%%%%%%%%%%%%%%
\subsection{Driver Development}
\label{sec:driver}

The \textsf{childdoc} mechanism can also be use for the development
of definition files such as \LaTeX{} styles or classes.
This case differs from the above setup with multiple parts
included by |\include| in that no |\includeonly| should be invoked.
This can be achieved by starting the include file
(before |\ProvidesPackage|) with:
%
\begin{center}
\begin{tabular}{l}
|\input{childdoc.def}|\\
|\childdocforward{|\textit{main}|}|\\
\end{tabular}
\end{center}
%
or alternatively with:
%
\begin{center}
\begin{tabular}{l}
|\input{childdoc.def}|\\
|\childdocby{|\textit{main}|}|\\
\end{tabular}
\end{center}
%
Both forms have slightly different effects as described above.
The main file is prepared as usual, see \secref{sec:include}.

%%%%%%%%%%%%%%%%%%%%%%%%%%%%%%%%%%%%%%%%%%%%%%%%%%%%%%%%%%%%%%%%%%%%%%%%%%%%%%%%
\subsection{Legacy Detection}
\label{sec:detection}

The directive |\childdocmain| in the main file can detect
whether the complete document or merely a child is to be compiled
even without using the directive |\childdocof|.
This method is deprecated because it is less robust
and there is no compelling reason to use it;
it is merely provided for backward compatibility
and it may be removed in future versions.

If the detection mechanism is to be used,
it is mandatory to correctly specify
the filename of the main file as the argument of |\childdocmain|:
%
\begin{center}
\begin{tabular}{l}
|\input{childdoc.def}|\\
|\childdocmain{|\textit{main}|}|\\
\end{tabular}
\end{center}
%
If |\jobname| does not match the argument \textit{main} of |\childdocmain|,
it is assumed that |\jobname| points to the child file to be compiled.
When using |\childdocmain| with the main file specified as argument,
it suffices to start a child file
with just |\input{|\textit{main}|}|
without loading of the package and using |\childdocof|.
If instead all processing is done
with the appropriate \textsf{childdoc} directives,
the argument of \textit{main} of |\childdocmain| can be empty.

An alternative version of the command line processing described
in \secref{sec:commandline} using the detection mechanism reads:
%
\begin{center}
|... -jobname "|\textit{target}|" "|[\textit{flags}]%
[|\def\jobname{|\textit{dest}|}|]|\input{|\textit{main}|}"|
\end{center}

%%%%%%%%%%%%%%%%%%%%%%%%%%%%%%%%%%%%%%%%%%%%%%%%%%%%%%%%%%%%%%%%%%%%%%%%%%%%%%%%
\subsection{Manual Code}
\label{sec:manual}

In case one cannot be certain whether the definitions file |childdoc.def|
is installed on the target \TeX{} distribution
and one prefers not to ship it,
it is conceivable to paste a few relevant commands into the sources.

To that end, drop all statements |\input{childdoc.def}|
and perform the replacements as outlined below.
Instead of |\childdocmain{|\textit{main}|}| add the following code
to the top of the main file:
%
\begin{center}
\begin{tabular}{l}
|\||ifdefined\childdocname\endinput\||fi\newif\ifchilddoc|\\
|\edef\childdocname{\scantokens\expandafter{\jobname\noexpand}}|\\
|\def\childdocmain{|\textit{main}|}\||ifx\childdocmain\childdocname\||else|\\
|\childdoctrue\includeonly{\childdocname}\let\jobname\childdocmain\||fi|\\
\end{tabular}
\end{center}
%
Instead of |\childdocof{|\textit{main}|}| just include the main file
at the top of each child file:
%
\begin{center}
|\input{|\textit{main}|}|
\end{center}
%
A simple redirection |\childdocforward{|\textit{dest}|}| is achieved by:
%
\begin{center}
|\def\jobname{|\textit{dest}|}\input{\jobname}|
\end{center}
%
The redirection with prefix
|\childdocforwardprefix[|\textit{prefix}|]{|\textit{dest}|}|
is accomplished by:
%
\begin{center}
\begin{tabular}{l}
|{\edef\jobname{\scantokens\expandafter{\jobname\noexpand}}|\\
|\def\redirectjob |\textit{prefix}|#1~~~{\gdef\jobname{|\textit{dest}|#1}}|\\
|\expandafter\redirectjob\jobname~~~}\input{\jobname}|
\end{tabular}
\end{center}

In an alternative approach,
child documents can be compiled by a specific command line
without additional code or specific definitions:
%
\begin{center}
|... -jobname "|\textit{target}|" "|[\textit{flags}]%
|\includeonly{|\textit{dest}|}\input{|\textit{main}|}"|
\end{center}
%

%%%%%%%%%%%%%%%%%%%%%%%%%%%%%%%%%%%%%%%%%%%%%%%%%%%%%%%%%%%%%%%%%%%%%%%%%%%%%%%%
%%%%%%%%%%%%%%%%%%%%%%%%%%%%%%%%%%%%%%%%%%%%%%%%%%%%%%%%%%%%%%%%%%%%%%%%%%%%%%%%
\section{Information}

%%%%%%%%%%%%%%%%%%%%%%%%%%%%%%%%%%%%%%%%%%%%%%%%%%%%%%%%%%%%%%%%%%%%%%%%%%%%%%%%
\subsection{Copyright}

Copyright \copyright{} 2017--2018 Niklas Beisert

This work may be distributed and/or modified under the
conditions of the \LaTeX{} Project Public License, either version 1.3
of this license or (at your option) any later version.
The latest version of this license is in
  \url{http://www.latex-project.org/lppl.txt}
and version 1.3 or later is part of all distributions of \LaTeX{}
version 2005/12/01 or later.

This work has the LPPL maintenance status `maintained'.

The Current Maintainer of this work is Niklas Beisert.

This work consists of the files |README.txt|, |childdoc.ins| and |childdoc.dtx|
as well as the derived files |childdoc.def|, |cdocsamp.tex|
with |cdocsch1.tex|, |cdocsch2.tex|, |cdocspt3.tex|, |cdocspt4.tex|,
|cdocsdrf.tex|, |cdocsfn1.tex|, |cdocsfn2.tex|
as well as |childdoc.pdf|.

%%%%%%%%%%%%%%%%%%%%%%%%%%%%%%%%%%%%%%%%%%%%%%%%%%%%%%%%%%%%%%%%%%%%%%%%%%%%%%%%
\subsection{Files and Installation}

The package consists of the files:
%
\begin{center}
\begin{tabular}{ll}
    |README.txt|   & readme file \\
    |childdoc.ins| & installation file \\
    |childdoc.dtx| & source file \\
    |childdoc.def| & definition file \\
    |cdocsamp.tex| & sample main file \\
    |cdocsch1.tex| & sample include file \\
    |cdocsch2.tex| & sample include file \\
    |cdocspt3.tex| & sample part file \\
    |cdocspt4.tex| & sample part file \\
    |cdocsdrf.tex| & sample redirection file \\
    |cdocsfn1.tex| & sample redirection file \\
    |cdocsfn2.tex| & sample redirection file \\
    |childdoc.pdf| & manual
\end{tabular}
\end{center}
%
The distribution consists of the files
|README.txt|, |childdoc.ins| and |childdoc.dtx|.
%
\begin{itemize}
\item
Run (pdf)\LaTeX{} on |childdoc.dtx|
to compile the manual |childdoc.pdf| (this file).
\item
Run \LaTeX{} on |childdoc.ins| to create the definitions file |childdoc.def|
and the sample |cdocsamp.tex| with include files
|cdocsch1.tex|, |cdocsch2.tex|, |cdocspt3.tex|, |cdocspt4.tex|,
|cdocsdrf.tex|, |cdocsfn1.tex|, |cdocsfn2.tex|.
Then copy the file |childdoc.def| to an appropriate directory of your \LaTeX{}
distribution, e.g.\ \textit{texmf-root}|/tex/latex/childdoc|.
\end{itemize}

%%%%%%%%%%%%%%%%%%%%%%%%%%%%%%%%%%%%%%%%%%%%%%%%%%%%%%%%%%%%%%%%%%%%%%%%%%%%%%%%
\subsection{Related CTAN Packages}

There are several other packages which offer a similar functionality:
%
\begin{itemize}
\item
The packages
\href{http://ctan.org/pkg/docmute}{\textsf{docmute}},
\href{http://ctan.org/pkg/includex}{\textsf{includex}} and
\href{http://ctan.org/pkg/standalone}{\textsf{standalone}}
provide commands to include only the document body of
a child file thus allowing both files to be compiled individually.
\item
The packages \href{http://ctan.org/pkg/subdocs}{\textsf{subdocs}}
and \href{http://ctan.org/pkg/subfiles}{\textsf{subfiles}}
provide structures in which the main and child documents can be
encapsulated and allowing them to be compiled individually.
The inclusion mechanism is different from the conventional |\include|.
\item
The package \href{http://ctan.org/pkg/combine}{\textsf{combine}}
is an elaborate solution to combine several documents into one.
\end{itemize}
%
See also the CTAN topic \href{http://ctan.org/topic/subdocs}{\textsf{subdocs}}
for further related packages.
The present package differs from the above solutions in that
a document structure constructed with the conventional |\include| mechanism
just needs two extra commands at the top of every file
such that all constituent files can be compiled individually.

%%%%%%%%%%%%%%%%%%%%%%%%%%%%%%%%%%%%%%%%%%%%%%%%%%%%%%%%%%%%%%%%%%%%%%%%%%%%%%%%
%\subsection{Feature Suggestions}
%
%The following is a list of features which may be useful for future
%versions of this package:
%%
%\begin{itemize}
%\item
%\ldots
%\end{itemize}

%%%%%%%%%%%%%%%%%%%%%%%%%%%%%%%%%%%%%%%%%%%%%%%%%%%%%%%%%%%%%%%%%%%%%%%%%%%%%%%%
\subsection{Revision History}

%%%%%%%%%%%%%%%%%%%%%%%%%%%%%%%%%%%%%%%%
\paragraph{v2.0:} 2018/12/30

\begin{itemize}
\item
immediate forward processing
\item
added |\childdocby| mechanism
\item
manual restructured
\end{itemize}

%%%%%%%%%%%%%%%%%%%%%%%%%%%%%%%%%%%%%%%%
\paragraph{v1.6:} 2018/01/17

\begin{itemize}
\item
application for development of include files
\item
corrections to manual
\end{itemize}

%%%%%%%%%%%%%%%%%%%%%%%%%%%%%%%%%%%%%%%%
\paragraph{v1.5:} 2017/05/21

\begin{itemize}
\item
more complete structuring introduced
\item
|\childdocof| introduced
\item
|\childdoc| renamed to |\childdocmain|
\item
|\childredirect| renamed to |\childdocforward| and |\childdocforwardprefix|
and functionality expanded
\end{itemize}

%%%%%%%%%%%%%%%%%%%%%%%%%%%%%%%%%%%%%%%%
\paragraph{v1.0:} 2017/04/27

\begin{itemize}
\item
manual and install package
\item
first version published on CTAN
\end{itemize}

%%%%%%%%%%%%%%%%%%%%%%%%%%%%%%%%%%%%%%%%
\paragraph{v0.6:} 2017/04/26

\begin{itemize}
\item
redirection mechanism added
\end{itemize}

%%%%%%%%%%%%%%%%%%%%%%%%%%%%%%%%%%%%%%%%
\paragraph{v0.5:} 2017/04/26

\begin{itemize}
\item
functionality in definition file
\end{itemize}


%%%%%%%%%%%%%%%%%%%%%%%%%%%%%%%%%%%%%%%%%%%%%%%%%%%%%%%%%%%%%%%%%%%%%%%%%%%%%%%%
%%%%%%%%%%%%%%%%%%%%%%%%%%%%%%%%%%%%%%%%%%%%%%%%%%%%%%%%%%%%%%%%%%%%%%%%%%%%%%%%
%%%%%%%%%%%%%%%%%%%%%%%%%%%%%%%%%%%%%%%%%%%%%%%%%%%%%%%%%%%%%%%%%%%%%%%%%%%%%%%%
\appendix

\settowidth\MacroIndent{\rmfamily\scriptsize 000\ }

 \DocInput{childdoc.dtx}

\end{document}
%</driver>
% \fi
%
% %%%%%%%%%%%%%%%%%%%%%%%%%%%%%%%%%%%%%%%%%%%%%%%%%%%%%%%%%%%%%%%%%%%%%%%%%%%%%%
% %%%%%%%%%%%%%%%%%%%%%%%%%%%%%%%%%%%%%%%%%%%%%%%%%%%%%%%%%%%%%%%%%%%%%%%%%%%%%%
% \section{Sample}
%\iffalse
%<*samplemain>
%\fi
%
% The following presents a sample document
% with two chapters, two parts, a title page,
% a compile flag as well as three forwarding files to set the flag.
% It consists of eight |.tex| files:
% \begin{center}
% \begin{tabular}{ll}
% |cdocsamp.tex|&main file\\
% |cdocsch1.tex|&include file for chapter 1\\
% |cdocsch2.tex|&include file for chapter 2\\
% |cdocspt3.tex|&include file for part 3\\
% |cdocspt4.tex|&include file for part 4\\
% |cdocsdrf.tex|&forwarding file for main file in draft mode\\
% |cdocsfi1.tex|&forwarding file for final version of chapter 1\\
% |cdocsfi2.tex|&forwarding file for final version of chapter 2\\
% \end{tabular}
% \end{center}
% Each of the eight files can be compiled directly by the \LaTeX{} compiler.
%
% %%%%%%%%%%%%%%%%%%%%%%%%%%%%%%%%%%%%%%
% \paragraph{Main File.}
%
% The main file is called |cdocsamp.tex|.
%
% Load the \textsf{childdoc} definitions and
% declare the filename for the main document:
%    \begin{macrocode}
\input{childdoc.def}
\childdocmain{}
%    \end{macrocode}

% Optional override for |\version| flag:
%    \begin{macrocode}
%%\ifchilddoc\else\providecommand{\version}{draft}\fi
%    \end{macrocode}

% Define the default values for the |\version| flag
% (|final| for the main file and |draft| for childs):
%    \begin{macrocode}
\ifchilddoc
\providecommand{\version}{draft}
\else
\providecommand{\version}{final}
\fi
%    \end{macrocode}

% Load the standard document class:
%    \begin{macrocode}
\documentclass[12pt]{article}
%    \end{macrocode}

% Start the document body:
%    \begin{macrocode}
\begin{document}
%    \end{macrocode}

% Declare a title page.
% Print title, part of document being processed and version flag:
%    \begin{macrocode}
\addtocounter{page}{-1}
\begin{center}
{\LARGE\bfseries{}childdoc example\par}
\vspace{1cm}
\ifchilddoc
\ifchilddocmanual part\else chapter\fi:
`\childdocname' of `\childdocjob'\par
\else
main document: `\childdocjob'\par
\fi
version: \version\par
\end{center}
\newpage
%    \end{macrocode}

% Manually include selected file,
% otherwise process as usual:
%    \begin{macrocode}
\ifchilddocmanual
\section*{part `\childdocname'}
\input{\childdocname}
\else
%    \end{macrocode}

% Include the two chapters:
%    \begin{macrocode}
\include{cdocsch1}
\include{cdocsch2}
%    \end{macrocode}

% Include the two parts unless only chapters should be displayed:
%    \begin{macrocode}
\ifchilddoc\else
\section{part three}
\input{cdocspt3}
\section{part four}
\input{cdocspt4}
\fi
%    \end{macrocode}

% Process as usual until here:
%    \begin{macrocode}
\fi
%    \end{macrocode}

% End of document body:
%    \begin{macrocode}
\end{document}
%    \end{macrocode}
%\iffalse
%</samplemain>
%\fi
%
% %%%%%%%%%%%%%%%%%%%%%%%%%%%%%%%%%%%%%%
% \paragraph{Chapter Include Files.}
%
% The include files are called |cdocsch1.tex| and |cdocsch2.tex|.
%
%\iffalse
%<*samplechap1|samplechap2>
%\fi

% Optional override for |\version| flag:
%    \begin{macrocode}
%%\providecommand{\version}{final}
%    \end{macrocode}

% Include the main document:
%    \begin{macrocode}
\input{childdoc.def}
\childdocof{cdocsamp}
%    \end{macrocode}

%\iffalse
%</samplechap1|samplechap2>
%\fi
%
%\iffalse
%<*samplechap1>
%\fi
% Some text for chapter 1:
%    \begin{macrocode}
\section{one}
some text in chapter one
%    \end{macrocode}

%\iffalse
%</samplechap1>
%\fi
% Some text for chapter 2:
%\iffalse
%<*samplechap2>
%\fi
%    \begin{macrocode}
\section{two}
more text in chapter two
%    \end{macrocode}

%\iffalse
%</samplechap2>
%\fi
%
% %%%%%%%%%%%%%%%%%%%%%%%%%%%%%%%%%%%%%%
% \paragraph{Part Include Files.}
%
% The include files are called |cdocspt3.tex| and |cdocspt4.tex|.
%
%\iffalse
%<*samplepart3|samplepart4>
%\fi

% Optional override for |\version| flag:
%    \begin{macrocode}
%%\providecommand{\version}{final}
%    \end{macrocode}

% Include the main document:
%    \begin{macrocode}
\input{childdoc.def}
\childdocby{cdocsamp}
%    \end{macrocode}

%\iffalse
%</samplepart3|samplepart4>
%\fi
%
%\iffalse
%<*samplepart3>
%\fi
% Some text for part 3:
%    \begin{macrocode}
some text in part three
%    \end{macrocode}

%\iffalse
%</samplepart3>
%\fi
% Some text for part 4:
%\iffalse
%<*samplepart4>
%\fi
%    \begin{macrocode}
more text in part four
%    \end{macrocode}

%\iffalse
%</samplepart4>
%\fi
%
% %%%%%%%%%%%%%%%%%%%%%%%%%%%%%%%%%%%%%%
% \paragraph{Forwarding for a Complete Draft.}
%
% The following forwarding file |cdocsdrf.tex|
% compiles the main document in draft mode:
%\iffalse
%<*sampledraft>
%\fi
%    \begin{macrocode}
\def\version{draft}
\input{childdoc.def}
\childdocforward{cdocsamp}
%    \end{macrocode}

%\iffalse
%</sampledraft>
%\fi
%
% %%%%%%%%%%%%%%%%%%%%%%%%%%%%%%%%%%%%%%
% \paragraph{Forwarding for Final Version of the Chapters.}
%
% The following forwarding files |cdocsfn1.tex| and |cdocsfn2.tex|
% (with identical content)
% compile the final versions of the child documents
% |cdocsch1.tex| and |cdocsch2.tex|, respectively:
%\iffalse
%<*samplefinal>
%\fi
%    \begin{macrocode}
\def\version{final}
\input{childdoc.def}
\childdocforwardprefix[cdocsamp]{cdocsfn}{cdocsch}
%    \end{macrocode}

%\iffalse
%</samplefinal>
%\fi
%
% %%%%%%%%%%%%%%%%%%%%%%%%%%%%%%%%%%%%%%
% \paragraph{Command Line Processing.}
%
% The following three command lines generate the output files
% |cdocscld|, |cdocscl1| and |cdocscl2|
% which should be identical to
% |cdocsdrf|, |cdocsch1| and |cdocsfn2|, respectively:
% \begin{center}
% \begin{tabular}{l}
% |latex -jobname cdocscld \|\\
% |  "\def\version{draft}\input{childdoc.def}\childdocforward{cdocsamp}"|\\
% |latex -jobname cdocscl1 \|\\
% |  "\input{childdoc.def}\childdocforward[cdocsamp]{cdocsch1}"|\\
% |latex -jobname cdocscl2 \|\\
% |  "\def\version{final}\input{childdoc.def}\childdocforward{cdocsch2}"|
% \end{tabular}
% \end{center}
% Note that the trailing backslash on each first line
% merely continues the input to the second line
% (for convenient cut ant paste).
% Furthermore, the command |latex| can be replaced by any
% of its alternative versions such as |pdflatex|.
%
% %%%%%%%%%%%%%%%%%%%%%%%%%%%%%%%%%%%%%%%%%%%%%%%%%%%%%%%%%%%%%%%%%%%%%%%%%%%%%%
% %%%%%%%%%%%%%%%%%%%%%%%%%%%%%%%%%%%%%%%%%%%%%%%%%%%%%%%%%%%%%%%%%%%%%%%%%%%%%%
% \section{Implementation}
%\iffalse
%<*package>
%\fi
%
% This section describes the definitions file |childdoc.def|.

% The definitions cannot be loaded using |\usepackage| or |\RequirePackage|
% which has a mechanism to prevent loading a style file more than once.
% When loading the definitions by means of |\input|
% multiple instances have to be prevented manually:
%\iffalse
%This code needs to be before the `\ProvidesFile' directive
%which is defined at the beginning of this file.
%Therefore it is also placed there and commented out here.
%</package>
%<*discard>
%\fi
%    \begin{macrocode}
\ifdefined\childdocmain\endinput\fi
%    \end{macrocode}
%\iffalse
%</discard>
%<*package>
%\fi
%
% \macro{\ifchilddoc}
% \macro{\ifchilddocmanual}
% The conditional |\ifchilddoc| tells whether a
% child (true) or main (false) document is being compiled.
% The conditional |\ifchilddocmanual| tells whether
% the |\includeonly| mechanism is used (false) or
% the selection of child files must be performed manually (true).
% The definitions initialise to false:
%    \begin{macrocode}
\newif\ifchilddoc
\newif\ifchilddocmanual
%    \end{macrocode}

% \macro{\childdocname}
% \macro{\childdocjob}
% The macro |\childdocname| stores the name of the main document
% to be compiled. The macro |\childdocjob| stores the name of
% the document on which the \LaTeX{} compiler was originally invoked.
% The content of |\jobname| cannot be compared
% to filenames specified in the source due to different catcodes.
% The following code rescans |\jobname|, stores the result
% in |\childdocname| and saves a copy in |\childdocjob|:
%    \begin{macrocode}
\edef\childdocname{\scantokens\expandafter{\jobname\noexpand}}
\let\childdocjob\childdocname
%    \end{macrocode}

% \macro{\childdocdisable}
% The macro |\childdocdisable| prevents the main file
% from being processed more than once.
% At this stage, the main document command |\childdocmain|
% is assumed to be called once again where it should do nothing.
% Any subsequent call to it should prevent
% a secondary processing of the main document
% It overwrites the forwarding commands
% |\childdocof| and |\childdocforward|
% with empty macros to prevent further inclusions of the main document:
%    \begin{macrocode}
\newcommand{\childdocdisable}
{
  \renewcommand{\childdocmain}[1]{\renewcommand{\childdocmain}[1]{\endinput}}
  \renewcommand{\childdocof}[1]{}
  \renewcommand{\childdocby}[2][]{}
  \renewcommand{\childdocforward}[2][]{}
  \renewcommand{\childdocdisable}{}
}
%    \end{macrocode}

% \macro{\childdocmain}
% The macro |\childdocmain| is to be called at the top of the main file
% with nothing or the main filename (without extension) as argument.
% First, it breaks loops.
% If the argument is not empty and does not match |\childdocname|
% (which is set by the first inclusion of |childdoc.def|),
% |\ifchilddoc| is set to true, |\includeonly| is applied to the child file
% and |\jobname| is set to the main file
% (for proper handling of |.aux| files):
%    \begin{macrocode}
\newcommand{\childdocmain}[1]
{
  \childdocdisable\childdocmain{}
  \if?#1?\else
    \begingroup
      \def\childdoctmp{#1}
      \ifx\childdoctmp\childdocname
        \def\childdoctmp{}
      \else
        \def\childdoctmp
        {
          \childdoctrue
          \includeonly{\childdocname}
          \def\childdocjob{#1}
          \def\jobname{#1}
        }
      \fi
      \expandafter
    \endgroup
    \childdoctmp
  \fi
}
%    \end{macrocode}

% \macro{\childdocof}
% The command |\childdocof| redirects
% compilation to the main file |#1|.
%    \begin{macrocode}
\newcommand{\childdocof}[1]
{
  \childdocdisable
  \childdoctrue
  \includeonly{\childdocname}
  \def\jobname{#1}
  \def\childdocjob{#1}
  \input{#1}
}
%    \end{macrocode}

% \macro{\childdocby}
% The command |\childdocby| ....
%    \begin{macrocode}
\newcommand{\childdocby}[2][]
{
  \childdocdisable
  \childdoctrue
  \childdocmanualtrue
  \if?#1?\else
    \def\jobname{#2}
  \fi
  \def\childdocjob{#2}
  \input{#2}
  \endinput
}
%    \end{macrocode}

% \macro{\childdocforward}
% The command |\childdocforward| redirects
% compilation to the main file or
% (if the optional argument is given) a child file.
% Parameters are set as if the main file
% or a child file starting with |\childdocof| was compiled.
% Then compilation is handed over to the main file:
%    \begin{macrocode}
\newcommand{\childdocforward}[2][]
{
  \begingroup
    \if?#1?
      \def\childdoctmp
      {
        \def\childdocname{#2}
        \def\childdocjob{#2}
        \def\jobname{#2}
        \input{#2}
        \endinput
      }
    \else
      \def\childdoctmp
      {
        \childdocdisable
        \def\childdocname{#2}
        \childdoctrue
        \includeonly{#2}
        \def\childdocjob{#1}
        \def\jobname{#1}
        \input{#1}
        \endinput
      }
    \fi
    \expandafter
  \endgroup
  \childdoctmp
}
%    \end{macrocode}

% \macro{\childdocforwardprefix}
% The command |\childdocforwardprefix| redirects
% compilation to the main or a child file by means of a pattern.
% The prefix |#1| in the current filename is replaced by |#2|
% and the suffix of the current filename is kept
% (it is assumed that the filename does not contain the substring `|~~~|'
% which is used as a delimiter).
% Compilation is handed over to the new file by |\childdocforward|:
%    \begin{macrocode}
\newcommand{\childdocforwardprefix}[3][]
{
  \begingroup
    \def\childdocextract #2##1~~~{\def\childdoctmp{\childdocforward[#1]{#3##1}}}
    \expandafter\childdocextract\childdocname~~~
    \expandafter
  \endgroup
  \childdoctmp
}
%    \end{macrocode}

% \macro{\childdoc}
% The deprecated macro |\childdoc| is a legacy version of |\childdocmain|:
%    \begin{macrocode}
\newcommand{\childdoc}{\childdocmain}
%    \end{macrocode}

% \macro{\childdocredirect}
% The deprecated macro |\childdocredirect| is a legacy version
% of |\childdocforward| and |\childdocforwardprefix|:
%    \begin{macrocode}
\newcommand{\childdocredirect}[2][]
{
  \begingroup
    \if?#1?
      \def\childdoctmp{\childdocforward{#2}}
    \else
      \def\childdoctmp{\childdocforwardprefix{#1}{#2}}
    \fi
    \expandafter
  \endgroup
  \childdoctmp
}
%    \end{macrocode}

%\iffalse
%</package>
%\fi
%
\endinput

\childdocof{cdocsamp}
%    \end{macrocode}

%\iffalse
%</samplechap1|samplechap2>
%\fi
%
%\iffalse
%<*samplechap1>
%\fi
% Some text for chapter 1:
%    \begin{macrocode}
\section{one}
some text in chapter one
%    \end{macrocode}

%\iffalse
%</samplechap1>
%\fi
% Some text for chapter 2:
%\iffalse
%<*samplechap2>
%\fi
%    \begin{macrocode}
\section{two}
more text in chapter two
%    \end{macrocode}

%\iffalse
%</samplechap2>
%\fi
%
% %%%%%%%%%%%%%%%%%%%%%%%%%%%%%%%%%%%%%%
% \paragraph{Part Include Files.}
%
% The include files are called |cdocspt3.tex| and |cdocspt4.tex|.
%
%\iffalse
%<*samplepart3|samplepart4>
%\fi

% Optional override for |\version| flag:
%    \begin{macrocode}
%%\providecommand{\version}{final}
%    \end{macrocode}

% Include the main document:
%    \begin{macrocode}
% \iffalse
%
% childdoc.dtx Copyright (C) 2017-2018 Niklas Beisert
%
% This work may be distributed and/or modified under the
% conditions of the LaTeX Project Public License, either version 1.3
% of this license or (at your option) any later version.
% The latest version of this license is in
%   http://www.latex-project.org/lppl.txt
% and version 1.3 or later is part of all distributions of LaTeX
% version 2005/12/01 or later.
%
% This work has the LPPL maintenance status `maintained'.
%
% The Current Maintainer of this work is Niklas Beisert.
%
% This work consists of the files childdoc.dtx and childdoc.ins
% and the derived files childdoc.def and cdocsamp.tex with
% cdocsch1.tex, cdocsch2.tex, cdocsdrf.tex, cdocsfn1.tex, cdocsfn2.tex.
%
%<package>\ifdefined\childdocmain\endinput\fi
%<package>\ProvidesFile{childdoc.def}[2018/12/30 v2.0 child document driver]
%<samplemain>\ProvidesFile{cdocsamp.tex}[2018/12/30 v2.0 sample for childdoc]
%<*driver>
%\ProvidesFile{childdoc.drv}[2018/12/30 v2.0 childdoc reference manual file]
\PassOptionsToClass{10pt,a4paper}{article}
\documentclass{ltxdoc}

\usepackage[margin=35mm]{geometry}
\usepackage{hyperref}
\usepackage{hyperxmp}
\usepackage[usenames]{color}

\hypersetup{colorlinks=true}
\hypersetup{pdfstartview=FitH}
\hypersetup{pdfpagemode=UseNone}
\hypersetup{pdfsource={}}
\hypersetup{pdflang={en-UK}}
\hypersetup{pdfcopyright={Copyright 2017-2018 Niklas Beisert.
  This work may be distributed and/or modified under the
  conditions of the LaTeX Project Public License, either version 1.3
  of this license or (at your option) any later version.}}
\hypersetup{pdflicenseurl={http://www.latex-project.org/lppl.txt}}
\hypersetup{pdfcontactaddress={ETH Zurich, ITP, HIT K,
  Wolfgang-Pauli-Strasse 27}}
\hypersetup{pdfcontactpostcode={8093}}
\hypersetup{pdfcontactcity={Zurich}}
\hypersetup{pdfcontactcountry={Switzerland}}
\hypersetup{pdfcontactemail={nbeisert@itp.phys.ethz.ch}}
\hypersetup{pdfcontacturl={http://people.phys.ethz.ch/\xmptilde nbeisert/}}

\newcommand{\secref}[1]{\hyperref[#1]{section \ref*{#1}}}

\parskip1ex
\parindent0pt
\let\olditemize\itemize
\def\itemize{\olditemize\parskip0pt}

\begin{document}

\title{The \textsf{childdoc} Package}
\hypersetup{pdftitle={The childdoc Package}}
\author{Niklas Beisert\\[2ex]
  Institut f\"ur Theoretische Physik\\
  Eidgen\"ossische Technische Hochschule Z\"urich\\
  Wolfgang-Pauli-Strasse 27, 8093 Z\"urich, Switzerland\\[1ex]
  \href{mailto:nbeisert@itp.phys.ethz.ch}
  {\texttt{nbeisert@itp.phys.ethz.ch}}}
\hypersetup{pdfauthor={Niklas Beisert}}
\hypersetup{pdfsubject={Manual for the LaTeX2e Package childdoc}}
\date{30 December 2018, \textsf{v2.0}}
\maketitle

\begin{abstract}\noindent
\textsf{childdoc} is a \LaTeXe{} package
that enables the direct compilation
of document sections included by |\include|
to individual files.
\end{abstract}

\begingroup
\parskip0ex
\tableofcontents
\endgroup

%%%%%%%%%%%%%%%%%%%%%%%%%%%%%%%%%%%%%%%%%%%%%%%%%%%%%%%%%%%%%%%%%%%%%%%%%%%%%%%%
%%%%%%%%%%%%%%%%%%%%%%%%%%%%%%%%%%%%%%%%%%%%%%%%%%%%%%%%%%%%%%%%%%%%%%%%%%%%%%%%
\section{Introduction}

\LaTeX{} provides a mechanism to structure a large document (such as a book)
into a main file and several child files (containing the chapters)
using the |\include| command.
This mechanism is beneficial for documents
which span hundreds of pages in order to
make the source file(s) more manageable.
Moreover, compilation can be restricted to
selected child files by means of the |\includeonly| command.
The latter feature can be used to reduce the compilation time while editing
(this was significantly more useful in the earlier days of \LaTeX{})
or to generate a smaller document which is easier to navigate.
Another application of |\includeonly| is to generate
documents consisting of selected parts of the complete document.

However, there are a few drawbacks of the plain |\include| mechanism:
\begin{itemize}
\item
The child files cannot be compiled on their own,
they can only be compiled via the main file.
A naive editing environment
(such as a text editor with an option
to have the current file processed by \LaTeX)
may require one to switch to the main file before compiling;
attempting to compile the child file produces errors.
\item
The main file must be modified (each time)
to adjust the |\includeonly| command
to the present needs. This easily leaves the main file in a messy state.
\item
The generated document will always carry the filename
of the main document. This is inconvenient if
several child files are to be compiled and
to be kept for distribution.
\end{itemize}

The present package provides a simple interface
to make child files individually compilable by \LaTeX{}.
Compiling a child file then has the same effect as compiling
the main file with an |\includeonly| command
to select the appropriate child.
Moreover the generated document will carry the name of the child
rather than the main file.
This resolves all three above issues.

This feature is meant to make the editing of books,
thesis documents and lecture notes somewhat more convenient.
However, the package can also be used efficiently for
composing a series of documents (such as exercise sheets)
which are typically distributed individually.
It then assists the author in generating the individual documents
(potentially in different versions)
as well as a document containing the collected series.
Another application is in developing style files
or other kinds of included material
where compilation of the style file could redirect
to a sample or test file.

%%%%%%%%%%%%%%%%%%%%%%%%%%%%%%%%%%%%%%%%%%%%%%%%%%%%%%%%%%%%%%%%%%%%%%%%%%%%%%%%
%%%%%%%%%%%%%%%%%%%%%%%%%%%%%%%%%%%%%%%%%%%%%%%%%%%%%%%%%%%%%%%%%%%%%%%%%%%%%%%%
\section{Usage}

First of all, the package \textsf{childdoc} is \emph{not} a standard
\LaTeXe{} |.sty| style file! Therefore it needs to be invoked in
a non-standard way.

%%%%%%%%%%%%%%%%%%%%%%%%%%%%%%%%%%%%%%%%%%%%%%%%%%%%%%%%%%%%%%%%%%%%%%%%%%%%%%%%
\subsection{Included Files}
\label{sec:include}

%%%%%%%%%%%%%%%%%%%%%%%%%%%%%%%%%%%%%%%%
\DescribeMacro{\childdocmain}
To use the package, add the commands
\begin{center}
\begin{tabular}{l}
|\input{childdoc.def}|\\
|\childdocmain{}|\\
\end{tabular}
\end{center}
at the very top of the main \LaTeX{} file,
in particular \emph{before} the |\documentclass| statement!
The argument of |\childdocmain| should be left empty
(but it must be present).

%%%%%%%%%%%%%%%%%%%%%%%%%%%%%%%%%%%%%%%%
\DescribeMacro{\childdocof}
Furthermore, add the commands
\begin{center}
\begin{tabular}{l}
|\input{childdoc.def}|\\
|\childdocof{|\textit{main}|}|\\
\end{tabular}
\end{center}
at the top of every child file \textit{child}
which is included by |\include{|\textit{child}|}|
from within the main file
(or at least for those files to be compiled individually).
The argument \textit{main} must be the filename of the main file.

There are a couple of
considerations in setting up the main and child documents:

%%%%%%%%%%%%%%%%%%%%%%%%%%%%%%%%%%%%%%%%
\paragraph{Restrictions.}

Please note the following restrictions:
\begin{itemize}
\item
|\childdocmain| must be called with one argument \textit{main}
to ensure compatibility with earlier version of the package.
It must either be empty (|\childdocmain{}|)
or precisely match the filename of the main file in which it is specified.
See \secref{sec:detection} for further information.
\item
The filename \textit{main} must be specified without the |.tex| extension.
\item
The filename \textit{main} is case sensitive
(even in case-insensitive file systems)
due to internal string comparison.
\item
The argument \textit{main} should be fully expanded, it cannot be a macro.
\item
Subdirectories and special characters should be avoided in filenames.
\item
The command |\childdocmain{|\textit{main}|}| must be followed by a whitespace.
It should not be followed immediately by another command
or by a comment mark `|%|'.
This is because the \TeX{} parser reads the token immediately following
the argument of |\childdocmain| and puts it
at the beginning of every child section;
however, a white\-space is ignored.
\end{itemize}

%%%%%%%%%%%%%%%%%%%%%%%%%%%%%%%%%%%%%%%%
\paragraph{Content of Main File.}

It is advisable to place all content in the child files included by |\include|.
Any output contained in the main file will appear in all child documents
unless suppressed manually;
it cannot be suppressed automatically by the |\includeonly| directive
and thus should normally be avoided.
A method to include some content in the main file
by means of conditional processing is described in \secref{sec:conditional}.

%%%%%%%%%%%%%%%%%%%%%%%%%%%%%%%%%%%%%%%%
\paragraph{Page Numbering.}

When only a part of the document is compiled,
the appropriate numbering of pages
(as well as other status parameters)
is determined from the |.aux| files.
The latter contain information from previous passes.
However this information needs to propagate through
all intermediate child documents.
Therefore the page numbering in child documents may well
be inconsistent until the complete document is compiled at least once.

A useful (if unconventional) way to always ensure a consistent
page numbering is to restart the numbering in each child document
and denote the pages by `\textit{child}|.|\textit{page}'
where \textit{child} represents the chapter/section number of the child file.
This can be achieved by the command
|\numberwithin{page}{|\textit{child}|}|
of the \textsf{amsmath} package
where \textit{child} can be |chapter| or |section|
depending on the chosen structuring.
Alternatively, one can modify the macro |\thepage| appropriately
and reset the counter |page| at the start of each child file.

%%%%%%%%%%%%%%%%%%%%%%%%%%%%%%%%%%%%%%%%%%%%%%%%%%%%%%%%%%%%%%%%%%%%%%%%%%%%%%%%
\subsection{Conditional Processing}
\label{sec:conditional}

The package provides a mechanism to compile different versions
of a document. To customise the versions further some conditional processing
can come in handy to distinguish which version is being compiled.
The package provides two macros to describe the compilation context:

%%%%%%%%%%%%%%%%%%%%%%%%%%%%%%%%%%%%%%%%
\DescribeMacro{\ifchilddoc}
The conditional |\ifchilddoc| distinguishes between the compilation of
child documents and the main document:
%
\begin{center}
|\ifchilddoc |\textit{child-code}| |[|\||else |\textit{main-code}]| \||fi|
\end{center}

%%%%%%%%%%%%%%%%%%%%%%%%%%%%%%%%%%%%%%%%
\DescribeMacro{\childdocname}
\DescribeMacro{\childdocjob}
The macro |\childdocname| contains the filename (without extension)
of the main or child file being processed.
Note that |\childdocjob| will always contain the name of the main file.

%%%%%%%%%%%%%%%%%%%%%%%%%%%%%%%%%%%%%%%%
\paragraph{Title Page.}

Conditional processing can be used to include a title or banner page
in the main document when proper precautions are taken.
Importantly, the code in the main file should ensure that the page counter
(as well as other status parameters which are stored in the |.aux| files)
takes the same value after the conditional processing.
Otherwise the page numbers may take divergent values
depending on which part is compiled.

For example, a title page could be declared by:
%
\begin{center}
\begin{tabular}{l}
|\ifchilddoc\||else|\\
|\addtocounter{page}{-1}|\\
\textit{code for title page}\\
|\newpage|\\
|\||fi|
\end{tabular}
\end{center}
%
A banner page for the child documents can be generated by:
%
\begin{center}
\begin{tabular}{l}
|\ifchilddoc|\\
|\addtocounter{page}{-1}|\\
\textit{code for banner page}\\
|\newpage|\\
|\||fi|
\end{tabular}
\end{center}
%
Here one could write a message such as:
\begin{center}
|This is the part \childdocname{} of \childdocjob{}.|
\end{center}

%%%%%%%%%%%%%%%%%%%%%%%%%%%%%%%%%%%%%%%%%%%%%%%%%%%%%%%%%%%%%%%%%%%%%%%%%%%%%%%%
\subsection{Flags}
\label{sec:flags}

The package makes it easy to generate different versions
of the main or child documents.
To this end compilation flags can be defined
and assigned different default values.
They will be particularly useful in conjunction
with the forwarding mechanism described in \secref{sec:forward}.

For example, it may be useful to have a flag |\version|
which can be set to |draft| or |final|.
The document source will contain some conditional code
depending on the value of |\version|.
Suppose further, the flag should default to |final| for the main file
and to |draft| for child files
which is a natural assignment for editing the document.
This is achieved by placing the following code
in the preamble of the main document
(below the |\childdocmain| directive):
%
\begin{center}
\begin{tabular}{l}
|\ifchilddoc|\\
|\providecommand{\version}{draft}|\\
|\||else|\\
|\providecommand{\version}{final}|\\
|\||fi|
\end{tabular}
\end{center}
%
The definition by |\providecommand| makes sure
that previous definitions are not overwritten.
Further statements |\providecommand{\version}{...}|
can thus be added before the above code to override it.

For the main file, one might add a line
(between |\childdocmain| and the above block)
%
\begin{center}
|%\ifchilddoc\||else\providecommand{\version}{draft}\||fi|
\end{center}
%
which can be uncommented to produce a draft version.
Likewise one can add a line to the very top of a child file
(above the |\childdocof{|\textit{main}|}| directive)
%
\begin{center}
|%\providecommand{\version}{final}|
\end{center}
%
which can be uncommented to produce the final version of this child document.

%%%%%%%%%%%%%%%%%%%%%%%%%%%%%%%%%%%%%%%%%%%%%%%%%%%%%%%%%%%%%%%%%%%%%%%%%%%%%%%%
\subsection{Forwarding}
\label{sec:forward}

Different versions of the main or child documents
using compilation flags as described in \secref{sec:flags}
can be (permanently) stored in different files
for convenient compilation, viewing and distribution.
To this end, the package defines a command
to pass on compilation to a different file:

%%%%%%%%%%%%%%%%%%%%%%%%%%%%%%%%%%%%%%%%
\DescribeMacro{\childdocforward}
The command |\childdocforward| redirects processing to
another source file:
%
\begin{center}
\begin{tabular}{l}
|\input{childdoc.def}|\\
|\childdocforward[|\textit{main}|]{|\textit{dest}|}|\\
\end{tabular}
\end{center}
%
The argument \textit{dest} is the destination file
(without extension).
It should be the main file or one of the child files.
Note that further \textsf{childdoc} directives
such as |\childdocof| and |\childdocforward|
in the indicated file will be processed in this form.
The optional argument \textit{main}
passes on directly to the main file \textit{main}
while pretending to compile the child \textit{dest}.
This form behaves as if \textit{dest}
issues |\childdocof{|\textit{main}|}| right away,
and no further \textsf{childdoc} directives will be processed.

%%%%%%%%%%%%%%%%%%%%%%%%%%%%%%%%%%%%%%%%
\DescribeMacro{\...prefix}
In the alternative form |\childdocforwardprefix|,
%
\begin{center}
\begin{tabular}{l}
|\input{childdoc.def}|\\
|\childdocforwardprefix[|\textit{main}|]{|\textit{prefix}|}{|\textit{dest}|}|
\end{tabular}
\end{center}
%
the destination file is determined by a pattern
depending on the current file:
To make this work, the current file must be called
`{\textit{prefix}\hspace{0.2em}\textit{suffix}}'
with \textit{prefix} matching precisely the argument.
Processing is then passed on to the file
`{\textit{dest}\hspace{0.2em}\textit{suffix}}'.
Surely, the same effect is achieved by
directly specifying the
argument `{\textit{dest}\hspace{0.2em}\textit{suffix}}'
in the first form.
However, that requires to set up a different file
for each child. With the alternative form of the command
all these files can have exactly the same content
which simplifies setting them up and maintaining them.

For example, the following file |draft.tex|
with a compilation flag |\version| as described in \secref{sec:flags}
compiles the main document as a draft:
%
\begin{center}
\begin{tabular}{l}
|\def\version{draft}|\\
|\input{childdoc.def}|\\
|\childdocforward{|\textit{main}|}|
\end{tabular}
\end{center}
%
Likewise, the following files |final|\textit{nn}|.tex|
compile the final version of the child document
|child|\textit{nn}|.tex|:
%
\begin{center}
\begin{tabular}{l}
|\def\version{final}|\\
|\input{childdoc.def}|\\
|\childdocforwardprefix{final}{child}|
\end{tabular}
\end{center}
%

Note that when several versions of a main file and/or of each child file
are to be generated, it may be convenient to set up a |Makefile| or
shell script to automatise the process.

%%%%%%%%%%%%%%%%%%%%%%%%%%%%%%%%%%%%%%%%%%%%%%%%%%%%%%%%%%%%%%%%%%%%%%%%%%%%%%%%
\subsection{Command Line Processing}
\label{sec:commandline}

The effect of redirection files can also be achieved by invoking
the \LaTeX{} compiler with a more elaborate command line.
Most conveniently this should be done as part
of a shell script or a |Makefile|.

When using \textsf{childdoc} in the main file, the following
command lines effectively perform a redirection
(note that depending on the shell being used,
backslashes may have to be doubled: `|\|' $\to$ `|\\|'):
%
\begin{center}
|... -jobname "|\textit{target}|" |\\|"|[\textit{flags}]%
|\input{childdoc.def}\childdocforward[|\textit{main}|]{|\textit{dest}|}"|
\end{center}
%
Here \textit{target} is the name of the output file,
\textit{main} is the name of the main file
and \textit{dest} is the name of the main or child file to be processed
(all filenames without extensions).
The optional argument \textit{main} can be omitted
if \textit{main} matches \textit{dest}.
Optionally, compilation \textit{flags} can be defined via |\def| commands.
This command line makes the \TeX{} engine believe
it is compiling the file \textit{target}
whose content is specified as the latter parameter.
The provided code then forwards the processing to
\textit{main} or \textit{dest} as described in \secref{sec:forward}.

%%%%%%%%%%%%%%%%%%%%%%%%%%%%%%%%%%%%%%%%%%%%%%%%%%%%%%%%%%%%%%%%%%%%%%%%%%%%%%%%
\subsection{Include by Input}
\label{sec:input}

Including child documents by |\include| has some restrictions by design.
Most notably, the content of a child document always occupies
its own set of pages; pages cannot be shared between child documents.
Usually, this behaviour makes perfect sense
because each child document contain an essential part of the document.
However, in some situations it may be desirable to compose
a document from a collection of parts
without having mandatory page breaks between then.
For this case, the package
provides a mechanism to include parts
by |\input| which can also be processed individually.
However, by construction this mechanism
requires manual handling of the content to be output.

%%%%%%%%%%%%%%%%%%%%%%%%%%%%%%%%%%%%%%%%
\DescribeMacro{\ifchilddocmanual}
The main file should be prepared as usual, see \secref{sec:include}.
However, the document body must make a distinction
between processing of an individual part and of the main document, e.g.:
%
\begin{center}
\begin{tabular}{l}
|\ifchilddocmanual|\\
|\input{\childdocname}|\\
|\||else|\\
\textit{document body with }|\input{|\textit{part}|}|\\
|\||fi|
\end{tabular}
\end{center}
%
The conditional |\ifchilddocmanual| is true whenever
a part to be included by |\input| is being compiled,
and the name of the part is stored in |\childdocname|.

%%%%%%%%%%%%%%%%%%%%%%%%%%%%%%%%%%%%%%%%
\DescribeMacro{\childdocby}
Each part to be included by |\input| should start with:
%
\begin{center}
\begin{tabular}{l}
|\input{childdoc.def}|\\
|\childdocby{|\textit{main}|}|\\
\end{tabular}
\end{center}
%
The directive |\childdocby| is similar to |\childdocof|
described in \secref{sec:include},
but the subsequent selection of content must be done manually.
To that end, both |\ifchilddoc| and |\ifchilddocmanual|
will be true upon processing of a part,
and the name of the part is stored in |\childdocname|.
Note that |\jobname| will be set to the filename of the current part
so that each part receives an individual |.aux| file
that does not interfere with the |.aux| file(s) of the main document.
This behaviour can be altered by the alternative form
|\childdocby[*]{|\textit{main}|}| (with a non-empty optional argument)
which uses the |.aux| file of the main document
by setting |\jobname| to \textit{main}.

%%%%%%%%%%%%%%%%%%%%%%%%%%%%%%%%%%%%%%%%%%%%%%%%%%%%%%%%%%%%%%%%%%%%%%%%%%%%%%%%
\subsection{Driver Development}
\label{sec:driver}

The \textsf{childdoc} mechanism can also be use for the development
of definition files such as \LaTeX{} styles or classes.
This case differs from the above setup with multiple parts
included by |\include| in that no |\includeonly| should be invoked.
This can be achieved by starting the include file
(before |\ProvidesPackage|) with:
%
\begin{center}
\begin{tabular}{l}
|\input{childdoc.def}|\\
|\childdocforward{|\textit{main}|}|\\
\end{tabular}
\end{center}
%
or alternatively with:
%
\begin{center}
\begin{tabular}{l}
|\input{childdoc.def}|\\
|\childdocby{|\textit{main}|}|\\
\end{tabular}
\end{center}
%
Both forms have slightly different effects as described above.
The main file is prepared as usual, see \secref{sec:include}.

%%%%%%%%%%%%%%%%%%%%%%%%%%%%%%%%%%%%%%%%%%%%%%%%%%%%%%%%%%%%%%%%%%%%%%%%%%%%%%%%
\subsection{Legacy Detection}
\label{sec:detection}

The directive |\childdocmain| in the main file can detect
whether the complete document or merely a child is to be compiled
even without using the directive |\childdocof|.
This method is deprecated because it is less robust
and there is no compelling reason to use it;
it is merely provided for backward compatibility
and it may be removed in future versions.

If the detection mechanism is to be used,
it is mandatory to correctly specify
the filename of the main file as the argument of |\childdocmain|:
%
\begin{center}
\begin{tabular}{l}
|\input{childdoc.def}|\\
|\childdocmain{|\textit{main}|}|\\
\end{tabular}
\end{center}
%
If |\jobname| does not match the argument \textit{main} of |\childdocmain|,
it is assumed that |\jobname| points to the child file to be compiled.
When using |\childdocmain| with the main file specified as argument,
it suffices to start a child file
with just |\input{|\textit{main}|}|
without loading of the package and using |\childdocof|.
If instead all processing is done
with the appropriate \textsf{childdoc} directives,
the argument of \textit{main} of |\childdocmain| can be empty.

An alternative version of the command line processing described
in \secref{sec:commandline} using the detection mechanism reads:
%
\begin{center}
|... -jobname "|\textit{target}|" "|[\textit{flags}]%
[|\def\jobname{|\textit{dest}|}|]|\input{|\textit{main}|}"|
\end{center}

%%%%%%%%%%%%%%%%%%%%%%%%%%%%%%%%%%%%%%%%%%%%%%%%%%%%%%%%%%%%%%%%%%%%%%%%%%%%%%%%
\subsection{Manual Code}
\label{sec:manual}

In case one cannot be certain whether the definitions file |childdoc.def|
is installed on the target \TeX{} distribution
and one prefers not to ship it,
it is conceivable to paste a few relevant commands into the sources.

To that end, drop all statements |\input{childdoc.def}|
and perform the replacements as outlined below.
Instead of |\childdocmain{|\textit{main}|}| add the following code
to the top of the main file:
%
\begin{center}
\begin{tabular}{l}
|\||ifdefined\childdocname\endinput\||fi\newif\ifchilddoc|\\
|\edef\childdocname{\scantokens\expandafter{\jobname\noexpand}}|\\
|\def\childdocmain{|\textit{main}|}\||ifx\childdocmain\childdocname\||else|\\
|\childdoctrue\includeonly{\childdocname}\let\jobname\childdocmain\||fi|\\
\end{tabular}
\end{center}
%
Instead of |\childdocof{|\textit{main}|}| just include the main file
at the top of each child file:
%
\begin{center}
|\input{|\textit{main}|}|
\end{center}
%
A simple redirection |\childdocforward{|\textit{dest}|}| is achieved by:
%
\begin{center}
|\def\jobname{|\textit{dest}|}\input{\jobname}|
\end{center}
%
The redirection with prefix
|\childdocforwardprefix[|\textit{prefix}|]{|\textit{dest}|}|
is accomplished by:
%
\begin{center}
\begin{tabular}{l}
|{\edef\jobname{\scantokens\expandafter{\jobname\noexpand}}|\\
|\def\redirectjob |\textit{prefix}|#1~~~{\gdef\jobname{|\textit{dest}|#1}}|\\
|\expandafter\redirectjob\jobname~~~}\input{\jobname}|
\end{tabular}
\end{center}

In an alternative approach,
child documents can be compiled by a specific command line
without additional code or specific definitions:
%
\begin{center}
|... -jobname "|\textit{target}|" "|[\textit{flags}]%
|\includeonly{|\textit{dest}|}\input{|\textit{main}|}"|
\end{center}
%

%%%%%%%%%%%%%%%%%%%%%%%%%%%%%%%%%%%%%%%%%%%%%%%%%%%%%%%%%%%%%%%%%%%%%%%%%%%%%%%%
%%%%%%%%%%%%%%%%%%%%%%%%%%%%%%%%%%%%%%%%%%%%%%%%%%%%%%%%%%%%%%%%%%%%%%%%%%%%%%%%
\section{Information}

%%%%%%%%%%%%%%%%%%%%%%%%%%%%%%%%%%%%%%%%%%%%%%%%%%%%%%%%%%%%%%%%%%%%%%%%%%%%%%%%
\subsection{Copyright}

Copyright \copyright{} 2017--2018 Niklas Beisert

This work may be distributed and/or modified under the
conditions of the \LaTeX{} Project Public License, either version 1.3
of this license or (at your option) any later version.
The latest version of this license is in
  \url{http://www.latex-project.org/lppl.txt}
and version 1.3 or later is part of all distributions of \LaTeX{}
version 2005/12/01 or later.

This work has the LPPL maintenance status `maintained'.

The Current Maintainer of this work is Niklas Beisert.

This work consists of the files |README.txt|, |childdoc.ins| and |childdoc.dtx|
as well as the derived files |childdoc.def|, |cdocsamp.tex|
with |cdocsch1.tex|, |cdocsch2.tex|, |cdocspt3.tex|, |cdocspt4.tex|,
|cdocsdrf.tex|, |cdocsfn1.tex|, |cdocsfn2.tex|
as well as |childdoc.pdf|.

%%%%%%%%%%%%%%%%%%%%%%%%%%%%%%%%%%%%%%%%%%%%%%%%%%%%%%%%%%%%%%%%%%%%%%%%%%%%%%%%
\subsection{Files and Installation}

The package consists of the files:
%
\begin{center}
\begin{tabular}{ll}
    |README.txt|   & readme file \\
    |childdoc.ins| & installation file \\
    |childdoc.dtx| & source file \\
    |childdoc.def| & definition file \\
    |cdocsamp.tex| & sample main file \\
    |cdocsch1.tex| & sample include file \\
    |cdocsch2.tex| & sample include file \\
    |cdocspt3.tex| & sample part file \\
    |cdocspt4.tex| & sample part file \\
    |cdocsdrf.tex| & sample redirection file \\
    |cdocsfn1.tex| & sample redirection file \\
    |cdocsfn2.tex| & sample redirection file \\
    |childdoc.pdf| & manual
\end{tabular}
\end{center}
%
The distribution consists of the files
|README.txt|, |childdoc.ins| and |childdoc.dtx|.
%
\begin{itemize}
\item
Run (pdf)\LaTeX{} on |childdoc.dtx|
to compile the manual |childdoc.pdf| (this file).
\item
Run \LaTeX{} on |childdoc.ins| to create the definitions file |childdoc.def|
and the sample |cdocsamp.tex| with include files
|cdocsch1.tex|, |cdocsch2.tex|, |cdocspt3.tex|, |cdocspt4.tex|,
|cdocsdrf.tex|, |cdocsfn1.tex|, |cdocsfn2.tex|.
Then copy the file |childdoc.def| to an appropriate directory of your \LaTeX{}
distribution, e.g.\ \textit{texmf-root}|/tex/latex/childdoc|.
\end{itemize}

%%%%%%%%%%%%%%%%%%%%%%%%%%%%%%%%%%%%%%%%%%%%%%%%%%%%%%%%%%%%%%%%%%%%%%%%%%%%%%%%
\subsection{Related CTAN Packages}

There are several other packages which offer a similar functionality:
%
\begin{itemize}
\item
The packages
\href{http://ctan.org/pkg/docmute}{\textsf{docmute}},
\href{http://ctan.org/pkg/includex}{\textsf{includex}} and
\href{http://ctan.org/pkg/standalone}{\textsf{standalone}}
provide commands to include only the document body of
a child file thus allowing both files to be compiled individually.
\item
The packages \href{http://ctan.org/pkg/subdocs}{\textsf{subdocs}}
and \href{http://ctan.org/pkg/subfiles}{\textsf{subfiles}}
provide structures in which the main and child documents can be
encapsulated and allowing them to be compiled individually.
The inclusion mechanism is different from the conventional |\include|.
\item
The package \href{http://ctan.org/pkg/combine}{\textsf{combine}}
is an elaborate solution to combine several documents into one.
\end{itemize}
%
See also the CTAN topic \href{http://ctan.org/topic/subdocs}{\textsf{subdocs}}
for further related packages.
The present package differs from the above solutions in that
a document structure constructed with the conventional |\include| mechanism
just needs two extra commands at the top of every file
such that all constituent files can be compiled individually.

%%%%%%%%%%%%%%%%%%%%%%%%%%%%%%%%%%%%%%%%%%%%%%%%%%%%%%%%%%%%%%%%%%%%%%%%%%%%%%%%
%\subsection{Feature Suggestions}
%
%The following is a list of features which may be useful for future
%versions of this package:
%%
%\begin{itemize}
%\item
%\ldots
%\end{itemize}

%%%%%%%%%%%%%%%%%%%%%%%%%%%%%%%%%%%%%%%%%%%%%%%%%%%%%%%%%%%%%%%%%%%%%%%%%%%%%%%%
\subsection{Revision History}

%%%%%%%%%%%%%%%%%%%%%%%%%%%%%%%%%%%%%%%%
\paragraph{v2.0:} 2018/12/30

\begin{itemize}
\item
immediate forward processing
\item
added |\childdocby| mechanism
\item
manual restructured
\end{itemize}

%%%%%%%%%%%%%%%%%%%%%%%%%%%%%%%%%%%%%%%%
\paragraph{v1.6:} 2018/01/17

\begin{itemize}
\item
application for development of include files
\item
corrections to manual
\end{itemize}

%%%%%%%%%%%%%%%%%%%%%%%%%%%%%%%%%%%%%%%%
\paragraph{v1.5:} 2017/05/21

\begin{itemize}
\item
more complete structuring introduced
\item
|\childdocof| introduced
\item
|\childdoc| renamed to |\childdocmain|
\item
|\childredirect| renamed to |\childdocforward| and |\childdocforwardprefix|
and functionality expanded
\end{itemize}

%%%%%%%%%%%%%%%%%%%%%%%%%%%%%%%%%%%%%%%%
\paragraph{v1.0:} 2017/04/27

\begin{itemize}
\item
manual and install package
\item
first version published on CTAN
\end{itemize}

%%%%%%%%%%%%%%%%%%%%%%%%%%%%%%%%%%%%%%%%
\paragraph{v0.6:} 2017/04/26

\begin{itemize}
\item
redirection mechanism added
\end{itemize}

%%%%%%%%%%%%%%%%%%%%%%%%%%%%%%%%%%%%%%%%
\paragraph{v0.5:} 2017/04/26

\begin{itemize}
\item
functionality in definition file
\end{itemize}


%%%%%%%%%%%%%%%%%%%%%%%%%%%%%%%%%%%%%%%%%%%%%%%%%%%%%%%%%%%%%%%%%%%%%%%%%%%%%%%%
%%%%%%%%%%%%%%%%%%%%%%%%%%%%%%%%%%%%%%%%%%%%%%%%%%%%%%%%%%%%%%%%%%%%%%%%%%%%%%%%
%%%%%%%%%%%%%%%%%%%%%%%%%%%%%%%%%%%%%%%%%%%%%%%%%%%%%%%%%%%%%%%%%%%%%%%%%%%%%%%%
\appendix

\settowidth\MacroIndent{\rmfamily\scriptsize 000\ }

 \DocInput{childdoc.dtx}

\end{document}
%</driver>
% \fi
%
% %%%%%%%%%%%%%%%%%%%%%%%%%%%%%%%%%%%%%%%%%%%%%%%%%%%%%%%%%%%%%%%%%%%%%%%%%%%%%%
% %%%%%%%%%%%%%%%%%%%%%%%%%%%%%%%%%%%%%%%%%%%%%%%%%%%%%%%%%%%%%%%%%%%%%%%%%%%%%%
% \section{Sample}
%\iffalse
%<*samplemain>
%\fi
%
% The following presents a sample document
% with two chapters, two parts, a title page,
% a compile flag as well as three forwarding files to set the flag.
% It consists of eight |.tex| files:
% \begin{center}
% \begin{tabular}{ll}
% |cdocsamp.tex|&main file\\
% |cdocsch1.tex|&include file for chapter 1\\
% |cdocsch2.tex|&include file for chapter 2\\
% |cdocspt3.tex|&include file for part 3\\
% |cdocspt4.tex|&include file for part 4\\
% |cdocsdrf.tex|&forwarding file for main file in draft mode\\
% |cdocsfi1.tex|&forwarding file for final version of chapter 1\\
% |cdocsfi2.tex|&forwarding file for final version of chapter 2\\
% \end{tabular}
% \end{center}
% Each of the eight files can be compiled directly by the \LaTeX{} compiler.
%
% %%%%%%%%%%%%%%%%%%%%%%%%%%%%%%%%%%%%%%
% \paragraph{Main File.}
%
% The main file is called |cdocsamp.tex|.
%
% Load the \textsf{childdoc} definitions and
% declare the filename for the main document:
%    \begin{macrocode}
\input{childdoc.def}
\childdocmain{}
%    \end{macrocode}

% Optional override for |\version| flag:
%    \begin{macrocode}
%%\ifchilddoc\else\providecommand{\version}{draft}\fi
%    \end{macrocode}

% Define the default values for the |\version| flag
% (|final| for the main file and |draft| for childs):
%    \begin{macrocode}
\ifchilddoc
\providecommand{\version}{draft}
\else
\providecommand{\version}{final}
\fi
%    \end{macrocode}

% Load the standard document class:
%    \begin{macrocode}
\documentclass[12pt]{article}
%    \end{macrocode}

% Start the document body:
%    \begin{macrocode}
\begin{document}
%    \end{macrocode}

% Declare a title page.
% Print title, part of document being processed and version flag:
%    \begin{macrocode}
\addtocounter{page}{-1}
\begin{center}
{\LARGE\bfseries{}childdoc example\par}
\vspace{1cm}
\ifchilddoc
\ifchilddocmanual part\else chapter\fi:
`\childdocname' of `\childdocjob'\par
\else
main document: `\childdocjob'\par
\fi
version: \version\par
\end{center}
\newpage
%    \end{macrocode}

% Manually include selected file,
% otherwise process as usual:
%    \begin{macrocode}
\ifchilddocmanual
\section*{part `\childdocname'}
\input{\childdocname}
\else
%    \end{macrocode}

% Include the two chapters:
%    \begin{macrocode}
\include{cdocsch1}
\include{cdocsch2}
%    \end{macrocode}

% Include the two parts unless only chapters should be displayed:
%    \begin{macrocode}
\ifchilddoc\else
\section{part three}
\input{cdocspt3}
\section{part four}
\input{cdocspt4}
\fi
%    \end{macrocode}

% Process as usual until here:
%    \begin{macrocode}
\fi
%    \end{macrocode}

% End of document body:
%    \begin{macrocode}
\end{document}
%    \end{macrocode}
%\iffalse
%</samplemain>
%\fi
%
% %%%%%%%%%%%%%%%%%%%%%%%%%%%%%%%%%%%%%%
% \paragraph{Chapter Include Files.}
%
% The include files are called |cdocsch1.tex| and |cdocsch2.tex|.
%
%\iffalse
%<*samplechap1|samplechap2>
%\fi

% Optional override for |\version| flag:
%    \begin{macrocode}
%%\providecommand{\version}{final}
%    \end{macrocode}

% Include the main document:
%    \begin{macrocode}
\input{childdoc.def}
\childdocof{cdocsamp}
%    \end{macrocode}

%\iffalse
%</samplechap1|samplechap2>
%\fi
%
%\iffalse
%<*samplechap1>
%\fi
% Some text for chapter 1:
%    \begin{macrocode}
\section{one}
some text in chapter one
%    \end{macrocode}

%\iffalse
%</samplechap1>
%\fi
% Some text for chapter 2:
%\iffalse
%<*samplechap2>
%\fi
%    \begin{macrocode}
\section{two}
more text in chapter two
%    \end{macrocode}

%\iffalse
%</samplechap2>
%\fi
%
% %%%%%%%%%%%%%%%%%%%%%%%%%%%%%%%%%%%%%%
% \paragraph{Part Include Files.}
%
% The include files are called |cdocspt3.tex| and |cdocspt4.tex|.
%
%\iffalse
%<*samplepart3|samplepart4>
%\fi

% Optional override for |\version| flag:
%    \begin{macrocode}
%%\providecommand{\version}{final}
%    \end{macrocode}

% Include the main document:
%    \begin{macrocode}
\input{childdoc.def}
\childdocby{cdocsamp}
%    \end{macrocode}

%\iffalse
%</samplepart3|samplepart4>
%\fi
%
%\iffalse
%<*samplepart3>
%\fi
% Some text for part 3:
%    \begin{macrocode}
some text in part three
%    \end{macrocode}

%\iffalse
%</samplepart3>
%\fi
% Some text for part 4:
%\iffalse
%<*samplepart4>
%\fi
%    \begin{macrocode}
more text in part four
%    \end{macrocode}

%\iffalse
%</samplepart4>
%\fi
%
% %%%%%%%%%%%%%%%%%%%%%%%%%%%%%%%%%%%%%%
% \paragraph{Forwarding for a Complete Draft.}
%
% The following forwarding file |cdocsdrf.tex|
% compiles the main document in draft mode:
%\iffalse
%<*sampledraft>
%\fi
%    \begin{macrocode}
\def\version{draft}
\input{childdoc.def}
\childdocforward{cdocsamp}
%    \end{macrocode}

%\iffalse
%</sampledraft>
%\fi
%
% %%%%%%%%%%%%%%%%%%%%%%%%%%%%%%%%%%%%%%
% \paragraph{Forwarding for Final Version of the Chapters.}
%
% The following forwarding files |cdocsfn1.tex| and |cdocsfn2.tex|
% (with identical content)
% compile the final versions of the child documents
% |cdocsch1.tex| and |cdocsch2.tex|, respectively:
%\iffalse
%<*samplefinal>
%\fi
%    \begin{macrocode}
\def\version{final}
\input{childdoc.def}
\childdocforwardprefix[cdocsamp]{cdocsfn}{cdocsch}
%    \end{macrocode}

%\iffalse
%</samplefinal>
%\fi
%
% %%%%%%%%%%%%%%%%%%%%%%%%%%%%%%%%%%%%%%
% \paragraph{Command Line Processing.}
%
% The following three command lines generate the output files
% |cdocscld|, |cdocscl1| and |cdocscl2|
% which should be identical to
% |cdocsdrf|, |cdocsch1| and |cdocsfn2|, respectively:
% \begin{center}
% \begin{tabular}{l}
% |latex -jobname cdocscld \|\\
% |  "\def\version{draft}\input{childdoc.def}\childdocforward{cdocsamp}"|\\
% |latex -jobname cdocscl1 \|\\
% |  "\input{childdoc.def}\childdocforward[cdocsamp]{cdocsch1}"|\\
% |latex -jobname cdocscl2 \|\\
% |  "\def\version{final}\input{childdoc.def}\childdocforward{cdocsch2}"|
% \end{tabular}
% \end{center}
% Note that the trailing backslash on each first line
% merely continues the input to the second line
% (for convenient cut ant paste).
% Furthermore, the command |latex| can be replaced by any
% of its alternative versions such as |pdflatex|.
%
% %%%%%%%%%%%%%%%%%%%%%%%%%%%%%%%%%%%%%%%%%%%%%%%%%%%%%%%%%%%%%%%%%%%%%%%%%%%%%%
% %%%%%%%%%%%%%%%%%%%%%%%%%%%%%%%%%%%%%%%%%%%%%%%%%%%%%%%%%%%%%%%%%%%%%%%%%%%%%%
% \section{Implementation}
%\iffalse
%<*package>
%\fi
%
% This section describes the definitions file |childdoc.def|.

% The definitions cannot be loaded using |\usepackage| or |\RequirePackage|
% which has a mechanism to prevent loading a style file more than once.
% When loading the definitions by means of |\input|
% multiple instances have to be prevented manually:
%\iffalse
%This code needs to be before the `\ProvidesFile' directive
%which is defined at the beginning of this file.
%Therefore it is also placed there and commented out here.
%</package>
%<*discard>
%\fi
%    \begin{macrocode}
\ifdefined\childdocmain\endinput\fi
%    \end{macrocode}
%\iffalse
%</discard>
%<*package>
%\fi
%
% \macro{\ifchilddoc}
% \macro{\ifchilddocmanual}
% The conditional |\ifchilddoc| tells whether a
% child (true) or main (false) document is being compiled.
% The conditional |\ifchilddocmanual| tells whether
% the |\includeonly| mechanism is used (false) or
% the selection of child files must be performed manually (true).
% The definitions initialise to false:
%    \begin{macrocode}
\newif\ifchilddoc
\newif\ifchilddocmanual
%    \end{macrocode}

% \macro{\childdocname}
% \macro{\childdocjob}
% The macro |\childdocname| stores the name of the main document
% to be compiled. The macro |\childdocjob| stores the name of
% the document on which the \LaTeX{} compiler was originally invoked.
% The content of |\jobname| cannot be compared
% to filenames specified in the source due to different catcodes.
% The following code rescans |\jobname|, stores the result
% in |\childdocname| and saves a copy in |\childdocjob|:
%    \begin{macrocode}
\edef\childdocname{\scantokens\expandafter{\jobname\noexpand}}
\let\childdocjob\childdocname
%    \end{macrocode}

% \macro{\childdocdisable}
% The macro |\childdocdisable| prevents the main file
% from being processed more than once.
% At this stage, the main document command |\childdocmain|
% is assumed to be called once again where it should do nothing.
% Any subsequent call to it should prevent
% a secondary processing of the main document
% It overwrites the forwarding commands
% |\childdocof| and |\childdocforward|
% with empty macros to prevent further inclusions of the main document:
%    \begin{macrocode}
\newcommand{\childdocdisable}
{
  \renewcommand{\childdocmain}[1]{\renewcommand{\childdocmain}[1]{\endinput}}
  \renewcommand{\childdocof}[1]{}
  \renewcommand{\childdocby}[2][]{}
  \renewcommand{\childdocforward}[2][]{}
  \renewcommand{\childdocdisable}{}
}
%    \end{macrocode}

% \macro{\childdocmain}
% The macro |\childdocmain| is to be called at the top of the main file
% with nothing or the main filename (without extension) as argument.
% First, it breaks loops.
% If the argument is not empty and does not match |\childdocname|
% (which is set by the first inclusion of |childdoc.def|),
% |\ifchilddoc| is set to true, |\includeonly| is applied to the child file
% and |\jobname| is set to the main file
% (for proper handling of |.aux| files):
%    \begin{macrocode}
\newcommand{\childdocmain}[1]
{
  \childdocdisable\childdocmain{}
  \if?#1?\else
    \begingroup
      \def\childdoctmp{#1}
      \ifx\childdoctmp\childdocname
        \def\childdoctmp{}
      \else
        \def\childdoctmp
        {
          \childdoctrue
          \includeonly{\childdocname}
          \def\childdocjob{#1}
          \def\jobname{#1}
        }
      \fi
      \expandafter
    \endgroup
    \childdoctmp
  \fi
}
%    \end{macrocode}

% \macro{\childdocof}
% The command |\childdocof| redirects
% compilation to the main file |#1|.
%    \begin{macrocode}
\newcommand{\childdocof}[1]
{
  \childdocdisable
  \childdoctrue
  \includeonly{\childdocname}
  \def\jobname{#1}
  \def\childdocjob{#1}
  \input{#1}
}
%    \end{macrocode}

% \macro{\childdocby}
% The command |\childdocby| ....
%    \begin{macrocode}
\newcommand{\childdocby}[2][]
{
  \childdocdisable
  \childdoctrue
  \childdocmanualtrue
  \if?#1?\else
    \def\jobname{#2}
  \fi
  \def\childdocjob{#2}
  \input{#2}
  \endinput
}
%    \end{macrocode}

% \macro{\childdocforward}
% The command |\childdocforward| redirects
% compilation to the main file or
% (if the optional argument is given) a child file.
% Parameters are set as if the main file
% or a child file starting with |\childdocof| was compiled.
% Then compilation is handed over to the main file:
%    \begin{macrocode}
\newcommand{\childdocforward}[2][]
{
  \begingroup
    \if?#1?
      \def\childdoctmp
      {
        \def\childdocname{#2}
        \def\childdocjob{#2}
        \def\jobname{#2}
        \input{#2}
        \endinput
      }
    \else
      \def\childdoctmp
      {
        \childdocdisable
        \def\childdocname{#2}
        \childdoctrue
        \includeonly{#2}
        \def\childdocjob{#1}
        \def\jobname{#1}
        \input{#1}
        \endinput
      }
    \fi
    \expandafter
  \endgroup
  \childdoctmp
}
%    \end{macrocode}

% \macro{\childdocforwardprefix}
% The command |\childdocforwardprefix| redirects
% compilation to the main or a child file by means of a pattern.
% The prefix |#1| in the current filename is replaced by |#2|
% and the suffix of the current filename is kept
% (it is assumed that the filename does not contain the substring `|~~~|'
% which is used as a delimiter).
% Compilation is handed over to the new file by |\childdocforward|:
%    \begin{macrocode}
\newcommand{\childdocforwardprefix}[3][]
{
  \begingroup
    \def\childdocextract #2##1~~~{\def\childdoctmp{\childdocforward[#1]{#3##1}}}
    \expandafter\childdocextract\childdocname~~~
    \expandafter
  \endgroup
  \childdoctmp
}
%    \end{macrocode}

% \macro{\childdoc}
% The deprecated macro |\childdoc| is a legacy version of |\childdocmain|:
%    \begin{macrocode}
\newcommand{\childdoc}{\childdocmain}
%    \end{macrocode}

% \macro{\childdocredirect}
% The deprecated macro |\childdocredirect| is a legacy version
% of |\childdocforward| and |\childdocforwardprefix|:
%    \begin{macrocode}
\newcommand{\childdocredirect}[2][]
{
  \begingroup
    \if?#1?
      \def\childdoctmp{\childdocforward{#2}}
    \else
      \def\childdoctmp{\childdocforwardprefix{#1}{#2}}
    \fi
    \expandafter
  \endgroup
  \childdoctmp
}
%    \end{macrocode}

%\iffalse
%</package>
%\fi
%
\endinput

\childdocby{cdocsamp}
%    \end{macrocode}

%\iffalse
%</samplepart3|samplepart4>
%\fi
%
%\iffalse
%<*samplepart3>
%\fi
% Some text for part 3:
%    \begin{macrocode}
some text in part three
%    \end{macrocode}

%\iffalse
%</samplepart3>
%\fi
% Some text for part 4:
%\iffalse
%<*samplepart4>
%\fi
%    \begin{macrocode}
more text in part four
%    \end{macrocode}

%\iffalse
%</samplepart4>
%\fi
%
% %%%%%%%%%%%%%%%%%%%%%%%%%%%%%%%%%%%%%%
% \paragraph{Forwarding for a Complete Draft.}
%
% The following forwarding file |cdocsdrf.tex|
% compiles the main document in draft mode:
%\iffalse
%<*sampledraft>
%\fi
%    \begin{macrocode}
\def\version{draft}
% \iffalse
%
% childdoc.dtx Copyright (C) 2017-2018 Niklas Beisert
%
% This work may be distributed and/or modified under the
% conditions of the LaTeX Project Public License, either version 1.3
% of this license or (at your option) any later version.
% The latest version of this license is in
%   http://www.latex-project.org/lppl.txt
% and version 1.3 or later is part of all distributions of LaTeX
% version 2005/12/01 or later.
%
% This work has the LPPL maintenance status `maintained'.
%
% The Current Maintainer of this work is Niklas Beisert.
%
% This work consists of the files childdoc.dtx and childdoc.ins
% and the derived files childdoc.def and cdocsamp.tex with
% cdocsch1.tex, cdocsch2.tex, cdocsdrf.tex, cdocsfn1.tex, cdocsfn2.tex.
%
%<package>\ifdefined\childdocmain\endinput\fi
%<package>\ProvidesFile{childdoc.def}[2018/12/30 v2.0 child document driver]
%<samplemain>\ProvidesFile{cdocsamp.tex}[2018/12/30 v2.0 sample for childdoc]
%<*driver>
%\ProvidesFile{childdoc.drv}[2018/12/30 v2.0 childdoc reference manual file]
\PassOptionsToClass{10pt,a4paper}{article}
\documentclass{ltxdoc}

\usepackage[margin=35mm]{geometry}
\usepackage{hyperref}
\usepackage{hyperxmp}
\usepackage[usenames]{color}

\hypersetup{colorlinks=true}
\hypersetup{pdfstartview=FitH}
\hypersetup{pdfpagemode=UseNone}
\hypersetup{pdfsource={}}
\hypersetup{pdflang={en-UK}}
\hypersetup{pdfcopyright={Copyright 2017-2018 Niklas Beisert.
  This work may be distributed and/or modified under the
  conditions of the LaTeX Project Public License, either version 1.3
  of this license or (at your option) any later version.}}
\hypersetup{pdflicenseurl={http://www.latex-project.org/lppl.txt}}
\hypersetup{pdfcontactaddress={ETH Zurich, ITP, HIT K,
  Wolfgang-Pauli-Strasse 27}}
\hypersetup{pdfcontactpostcode={8093}}
\hypersetup{pdfcontactcity={Zurich}}
\hypersetup{pdfcontactcountry={Switzerland}}
\hypersetup{pdfcontactemail={nbeisert@itp.phys.ethz.ch}}
\hypersetup{pdfcontacturl={http://people.phys.ethz.ch/\xmptilde nbeisert/}}

\newcommand{\secref}[1]{\hyperref[#1]{section \ref*{#1}}}

\parskip1ex
\parindent0pt
\let\olditemize\itemize
\def\itemize{\olditemize\parskip0pt}

\begin{document}

\title{The \textsf{childdoc} Package}
\hypersetup{pdftitle={The childdoc Package}}
\author{Niklas Beisert\\[2ex]
  Institut f\"ur Theoretische Physik\\
  Eidgen\"ossische Technische Hochschule Z\"urich\\
  Wolfgang-Pauli-Strasse 27, 8093 Z\"urich, Switzerland\\[1ex]
  \href{mailto:nbeisert@itp.phys.ethz.ch}
  {\texttt{nbeisert@itp.phys.ethz.ch}}}
\hypersetup{pdfauthor={Niklas Beisert}}
\hypersetup{pdfsubject={Manual for the LaTeX2e Package childdoc}}
\date{30 December 2018, \textsf{v2.0}}
\maketitle

\begin{abstract}\noindent
\textsf{childdoc} is a \LaTeXe{} package
that enables the direct compilation
of document sections included by |\include|
to individual files.
\end{abstract}

\begingroup
\parskip0ex
\tableofcontents
\endgroup

%%%%%%%%%%%%%%%%%%%%%%%%%%%%%%%%%%%%%%%%%%%%%%%%%%%%%%%%%%%%%%%%%%%%%%%%%%%%%%%%
%%%%%%%%%%%%%%%%%%%%%%%%%%%%%%%%%%%%%%%%%%%%%%%%%%%%%%%%%%%%%%%%%%%%%%%%%%%%%%%%
\section{Introduction}

\LaTeX{} provides a mechanism to structure a large document (such as a book)
into a main file and several child files (containing the chapters)
using the |\include| command.
This mechanism is beneficial for documents
which span hundreds of pages in order to
make the source file(s) more manageable.
Moreover, compilation can be restricted to
selected child files by means of the |\includeonly| command.
The latter feature can be used to reduce the compilation time while editing
(this was significantly more useful in the earlier days of \LaTeX{})
or to generate a smaller document which is easier to navigate.
Another application of |\includeonly| is to generate
documents consisting of selected parts of the complete document.

However, there are a few drawbacks of the plain |\include| mechanism:
\begin{itemize}
\item
The child files cannot be compiled on their own,
they can only be compiled via the main file.
A naive editing environment
(such as a text editor with an option
to have the current file processed by \LaTeX)
may require one to switch to the main file before compiling;
attempting to compile the child file produces errors.
\item
The main file must be modified (each time)
to adjust the |\includeonly| command
to the present needs. This easily leaves the main file in a messy state.
\item
The generated document will always carry the filename
of the main document. This is inconvenient if
several child files are to be compiled and
to be kept for distribution.
\end{itemize}

The present package provides a simple interface
to make child files individually compilable by \LaTeX{}.
Compiling a child file then has the same effect as compiling
the main file with an |\includeonly| command
to select the appropriate child.
Moreover the generated document will carry the name of the child
rather than the main file.
This resolves all three above issues.

This feature is meant to make the editing of books,
thesis documents and lecture notes somewhat more convenient.
However, the package can also be used efficiently for
composing a series of documents (such as exercise sheets)
which are typically distributed individually.
It then assists the author in generating the individual documents
(potentially in different versions)
as well as a document containing the collected series.
Another application is in developing style files
or other kinds of included material
where compilation of the style file could redirect
to a sample or test file.

%%%%%%%%%%%%%%%%%%%%%%%%%%%%%%%%%%%%%%%%%%%%%%%%%%%%%%%%%%%%%%%%%%%%%%%%%%%%%%%%
%%%%%%%%%%%%%%%%%%%%%%%%%%%%%%%%%%%%%%%%%%%%%%%%%%%%%%%%%%%%%%%%%%%%%%%%%%%%%%%%
\section{Usage}

First of all, the package \textsf{childdoc} is \emph{not} a standard
\LaTeXe{} |.sty| style file! Therefore it needs to be invoked in
a non-standard way.

%%%%%%%%%%%%%%%%%%%%%%%%%%%%%%%%%%%%%%%%%%%%%%%%%%%%%%%%%%%%%%%%%%%%%%%%%%%%%%%%
\subsection{Included Files}
\label{sec:include}

%%%%%%%%%%%%%%%%%%%%%%%%%%%%%%%%%%%%%%%%
\DescribeMacro{\childdocmain}
To use the package, add the commands
\begin{center}
\begin{tabular}{l}
|\input{childdoc.def}|\\
|\childdocmain{}|\\
\end{tabular}
\end{center}
at the very top of the main \LaTeX{} file,
in particular \emph{before} the |\documentclass| statement!
The argument of |\childdocmain| should be left empty
(but it must be present).

%%%%%%%%%%%%%%%%%%%%%%%%%%%%%%%%%%%%%%%%
\DescribeMacro{\childdocof}
Furthermore, add the commands
\begin{center}
\begin{tabular}{l}
|\input{childdoc.def}|\\
|\childdocof{|\textit{main}|}|\\
\end{tabular}
\end{center}
at the top of every child file \textit{child}
which is included by |\include{|\textit{child}|}|
from within the main file
(or at least for those files to be compiled individually).
The argument \textit{main} must be the filename of the main file.

There are a couple of
considerations in setting up the main and child documents:

%%%%%%%%%%%%%%%%%%%%%%%%%%%%%%%%%%%%%%%%
\paragraph{Restrictions.}

Please note the following restrictions:
\begin{itemize}
\item
|\childdocmain| must be called with one argument \textit{main}
to ensure compatibility with earlier version of the package.
It must either be empty (|\childdocmain{}|)
or precisely match the filename of the main file in which it is specified.
See \secref{sec:detection} for further information.
\item
The filename \textit{main} must be specified without the |.tex| extension.
\item
The filename \textit{main} is case sensitive
(even in case-insensitive file systems)
due to internal string comparison.
\item
The argument \textit{main} should be fully expanded, it cannot be a macro.
\item
Subdirectories and special characters should be avoided in filenames.
\item
The command |\childdocmain{|\textit{main}|}| must be followed by a whitespace.
It should not be followed immediately by another command
or by a comment mark `|%|'.
This is because the \TeX{} parser reads the token immediately following
the argument of |\childdocmain| and puts it
at the beginning of every child section;
however, a white\-space is ignored.
\end{itemize}

%%%%%%%%%%%%%%%%%%%%%%%%%%%%%%%%%%%%%%%%
\paragraph{Content of Main File.}

It is advisable to place all content in the child files included by |\include|.
Any output contained in the main file will appear in all child documents
unless suppressed manually;
it cannot be suppressed automatically by the |\includeonly| directive
and thus should normally be avoided.
A method to include some content in the main file
by means of conditional processing is described in \secref{sec:conditional}.

%%%%%%%%%%%%%%%%%%%%%%%%%%%%%%%%%%%%%%%%
\paragraph{Page Numbering.}

When only a part of the document is compiled,
the appropriate numbering of pages
(as well as other status parameters)
is determined from the |.aux| files.
The latter contain information from previous passes.
However this information needs to propagate through
all intermediate child documents.
Therefore the page numbering in child documents may well
be inconsistent until the complete document is compiled at least once.

A useful (if unconventional) way to always ensure a consistent
page numbering is to restart the numbering in each child document
and denote the pages by `\textit{child}|.|\textit{page}'
where \textit{child} represents the chapter/section number of the child file.
This can be achieved by the command
|\numberwithin{page}{|\textit{child}|}|
of the \textsf{amsmath} package
where \textit{child} can be |chapter| or |section|
depending on the chosen structuring.
Alternatively, one can modify the macro |\thepage| appropriately
and reset the counter |page| at the start of each child file.

%%%%%%%%%%%%%%%%%%%%%%%%%%%%%%%%%%%%%%%%%%%%%%%%%%%%%%%%%%%%%%%%%%%%%%%%%%%%%%%%
\subsection{Conditional Processing}
\label{sec:conditional}

The package provides a mechanism to compile different versions
of a document. To customise the versions further some conditional processing
can come in handy to distinguish which version is being compiled.
The package provides two macros to describe the compilation context:

%%%%%%%%%%%%%%%%%%%%%%%%%%%%%%%%%%%%%%%%
\DescribeMacro{\ifchilddoc}
The conditional |\ifchilddoc| distinguishes between the compilation of
child documents and the main document:
%
\begin{center}
|\ifchilddoc |\textit{child-code}| |[|\||else |\textit{main-code}]| \||fi|
\end{center}

%%%%%%%%%%%%%%%%%%%%%%%%%%%%%%%%%%%%%%%%
\DescribeMacro{\childdocname}
\DescribeMacro{\childdocjob}
The macro |\childdocname| contains the filename (without extension)
of the main or child file being processed.
Note that |\childdocjob| will always contain the name of the main file.

%%%%%%%%%%%%%%%%%%%%%%%%%%%%%%%%%%%%%%%%
\paragraph{Title Page.}

Conditional processing can be used to include a title or banner page
in the main document when proper precautions are taken.
Importantly, the code in the main file should ensure that the page counter
(as well as other status parameters which are stored in the |.aux| files)
takes the same value after the conditional processing.
Otherwise the page numbers may take divergent values
depending on which part is compiled.

For example, a title page could be declared by:
%
\begin{center}
\begin{tabular}{l}
|\ifchilddoc\||else|\\
|\addtocounter{page}{-1}|\\
\textit{code for title page}\\
|\newpage|\\
|\||fi|
\end{tabular}
\end{center}
%
A banner page for the child documents can be generated by:
%
\begin{center}
\begin{tabular}{l}
|\ifchilddoc|\\
|\addtocounter{page}{-1}|\\
\textit{code for banner page}\\
|\newpage|\\
|\||fi|
\end{tabular}
\end{center}
%
Here one could write a message such as:
\begin{center}
|This is the part \childdocname{} of \childdocjob{}.|
\end{center}

%%%%%%%%%%%%%%%%%%%%%%%%%%%%%%%%%%%%%%%%%%%%%%%%%%%%%%%%%%%%%%%%%%%%%%%%%%%%%%%%
\subsection{Flags}
\label{sec:flags}

The package makes it easy to generate different versions
of the main or child documents.
To this end compilation flags can be defined
and assigned different default values.
They will be particularly useful in conjunction
with the forwarding mechanism described in \secref{sec:forward}.

For example, it may be useful to have a flag |\version|
which can be set to |draft| or |final|.
The document source will contain some conditional code
depending on the value of |\version|.
Suppose further, the flag should default to |final| for the main file
and to |draft| for child files
which is a natural assignment for editing the document.
This is achieved by placing the following code
in the preamble of the main document
(below the |\childdocmain| directive):
%
\begin{center}
\begin{tabular}{l}
|\ifchilddoc|\\
|\providecommand{\version}{draft}|\\
|\||else|\\
|\providecommand{\version}{final}|\\
|\||fi|
\end{tabular}
\end{center}
%
The definition by |\providecommand| makes sure
that previous definitions are not overwritten.
Further statements |\providecommand{\version}{...}|
can thus be added before the above code to override it.

For the main file, one might add a line
(between |\childdocmain| and the above block)
%
\begin{center}
|%\ifchilddoc\||else\providecommand{\version}{draft}\||fi|
\end{center}
%
which can be uncommented to produce a draft version.
Likewise one can add a line to the very top of a child file
(above the |\childdocof{|\textit{main}|}| directive)
%
\begin{center}
|%\providecommand{\version}{final}|
\end{center}
%
which can be uncommented to produce the final version of this child document.

%%%%%%%%%%%%%%%%%%%%%%%%%%%%%%%%%%%%%%%%%%%%%%%%%%%%%%%%%%%%%%%%%%%%%%%%%%%%%%%%
\subsection{Forwarding}
\label{sec:forward}

Different versions of the main or child documents
using compilation flags as described in \secref{sec:flags}
can be (permanently) stored in different files
for convenient compilation, viewing and distribution.
To this end, the package defines a command
to pass on compilation to a different file:

%%%%%%%%%%%%%%%%%%%%%%%%%%%%%%%%%%%%%%%%
\DescribeMacro{\childdocforward}
The command |\childdocforward| redirects processing to
another source file:
%
\begin{center}
\begin{tabular}{l}
|\input{childdoc.def}|\\
|\childdocforward[|\textit{main}|]{|\textit{dest}|}|\\
\end{tabular}
\end{center}
%
The argument \textit{dest} is the destination file
(without extension).
It should be the main file or one of the child files.
Note that further \textsf{childdoc} directives
such as |\childdocof| and |\childdocforward|
in the indicated file will be processed in this form.
The optional argument \textit{main}
passes on directly to the main file \textit{main}
while pretending to compile the child \textit{dest}.
This form behaves as if \textit{dest}
issues |\childdocof{|\textit{main}|}| right away,
and no further \textsf{childdoc} directives will be processed.

%%%%%%%%%%%%%%%%%%%%%%%%%%%%%%%%%%%%%%%%
\DescribeMacro{\...prefix}
In the alternative form |\childdocforwardprefix|,
%
\begin{center}
\begin{tabular}{l}
|\input{childdoc.def}|\\
|\childdocforwardprefix[|\textit{main}|]{|\textit{prefix}|}{|\textit{dest}|}|
\end{tabular}
\end{center}
%
the destination file is determined by a pattern
depending on the current file:
To make this work, the current file must be called
`{\textit{prefix}\hspace{0.2em}\textit{suffix}}'
with \textit{prefix} matching precisely the argument.
Processing is then passed on to the file
`{\textit{dest}\hspace{0.2em}\textit{suffix}}'.
Surely, the same effect is achieved by
directly specifying the
argument `{\textit{dest}\hspace{0.2em}\textit{suffix}}'
in the first form.
However, that requires to set up a different file
for each child. With the alternative form of the command
all these files can have exactly the same content
which simplifies setting them up and maintaining them.

For example, the following file |draft.tex|
with a compilation flag |\version| as described in \secref{sec:flags}
compiles the main document as a draft:
%
\begin{center}
\begin{tabular}{l}
|\def\version{draft}|\\
|\input{childdoc.def}|\\
|\childdocforward{|\textit{main}|}|
\end{tabular}
\end{center}
%
Likewise, the following files |final|\textit{nn}|.tex|
compile the final version of the child document
|child|\textit{nn}|.tex|:
%
\begin{center}
\begin{tabular}{l}
|\def\version{final}|\\
|\input{childdoc.def}|\\
|\childdocforwardprefix{final}{child}|
\end{tabular}
\end{center}
%

Note that when several versions of a main file and/or of each child file
are to be generated, it may be convenient to set up a |Makefile| or
shell script to automatise the process.

%%%%%%%%%%%%%%%%%%%%%%%%%%%%%%%%%%%%%%%%%%%%%%%%%%%%%%%%%%%%%%%%%%%%%%%%%%%%%%%%
\subsection{Command Line Processing}
\label{sec:commandline}

The effect of redirection files can also be achieved by invoking
the \LaTeX{} compiler with a more elaborate command line.
Most conveniently this should be done as part
of a shell script or a |Makefile|.

When using \textsf{childdoc} in the main file, the following
command lines effectively perform a redirection
(note that depending on the shell being used,
backslashes may have to be doubled: `|\|' $\to$ `|\\|'):
%
\begin{center}
|... -jobname "|\textit{target}|" |\\|"|[\textit{flags}]%
|\input{childdoc.def}\childdocforward[|\textit{main}|]{|\textit{dest}|}"|
\end{center}
%
Here \textit{target} is the name of the output file,
\textit{main} is the name of the main file
and \textit{dest} is the name of the main or child file to be processed
(all filenames without extensions).
The optional argument \textit{main} can be omitted
if \textit{main} matches \textit{dest}.
Optionally, compilation \textit{flags} can be defined via |\def| commands.
This command line makes the \TeX{} engine believe
it is compiling the file \textit{target}
whose content is specified as the latter parameter.
The provided code then forwards the processing to
\textit{main} or \textit{dest} as described in \secref{sec:forward}.

%%%%%%%%%%%%%%%%%%%%%%%%%%%%%%%%%%%%%%%%%%%%%%%%%%%%%%%%%%%%%%%%%%%%%%%%%%%%%%%%
\subsection{Include by Input}
\label{sec:input}

Including child documents by |\include| has some restrictions by design.
Most notably, the content of a child document always occupies
its own set of pages; pages cannot be shared between child documents.
Usually, this behaviour makes perfect sense
because each child document contain an essential part of the document.
However, in some situations it may be desirable to compose
a document from a collection of parts
without having mandatory page breaks between then.
For this case, the package
provides a mechanism to include parts
by |\input| which can also be processed individually.
However, by construction this mechanism
requires manual handling of the content to be output.

%%%%%%%%%%%%%%%%%%%%%%%%%%%%%%%%%%%%%%%%
\DescribeMacro{\ifchilddocmanual}
The main file should be prepared as usual, see \secref{sec:include}.
However, the document body must make a distinction
between processing of an individual part and of the main document, e.g.:
%
\begin{center}
\begin{tabular}{l}
|\ifchilddocmanual|\\
|\input{\childdocname}|\\
|\||else|\\
\textit{document body with }|\input{|\textit{part}|}|\\
|\||fi|
\end{tabular}
\end{center}
%
The conditional |\ifchilddocmanual| is true whenever
a part to be included by |\input| is being compiled,
and the name of the part is stored in |\childdocname|.

%%%%%%%%%%%%%%%%%%%%%%%%%%%%%%%%%%%%%%%%
\DescribeMacro{\childdocby}
Each part to be included by |\input| should start with:
%
\begin{center}
\begin{tabular}{l}
|\input{childdoc.def}|\\
|\childdocby{|\textit{main}|}|\\
\end{tabular}
\end{center}
%
The directive |\childdocby| is similar to |\childdocof|
described in \secref{sec:include},
but the subsequent selection of content must be done manually.
To that end, both |\ifchilddoc| and |\ifchilddocmanual|
will be true upon processing of a part,
and the name of the part is stored in |\childdocname|.
Note that |\jobname| will be set to the filename of the current part
so that each part receives an individual |.aux| file
that does not interfere with the |.aux| file(s) of the main document.
This behaviour can be altered by the alternative form
|\childdocby[*]{|\textit{main}|}| (with a non-empty optional argument)
which uses the |.aux| file of the main document
by setting |\jobname| to \textit{main}.

%%%%%%%%%%%%%%%%%%%%%%%%%%%%%%%%%%%%%%%%%%%%%%%%%%%%%%%%%%%%%%%%%%%%%%%%%%%%%%%%
\subsection{Driver Development}
\label{sec:driver}

The \textsf{childdoc} mechanism can also be use for the development
of definition files such as \LaTeX{} styles or classes.
This case differs from the above setup with multiple parts
included by |\include| in that no |\includeonly| should be invoked.
This can be achieved by starting the include file
(before |\ProvidesPackage|) with:
%
\begin{center}
\begin{tabular}{l}
|\input{childdoc.def}|\\
|\childdocforward{|\textit{main}|}|\\
\end{tabular}
\end{center}
%
or alternatively with:
%
\begin{center}
\begin{tabular}{l}
|\input{childdoc.def}|\\
|\childdocby{|\textit{main}|}|\\
\end{tabular}
\end{center}
%
Both forms have slightly different effects as described above.
The main file is prepared as usual, see \secref{sec:include}.

%%%%%%%%%%%%%%%%%%%%%%%%%%%%%%%%%%%%%%%%%%%%%%%%%%%%%%%%%%%%%%%%%%%%%%%%%%%%%%%%
\subsection{Legacy Detection}
\label{sec:detection}

The directive |\childdocmain| in the main file can detect
whether the complete document or merely a child is to be compiled
even without using the directive |\childdocof|.
This method is deprecated because it is less robust
and there is no compelling reason to use it;
it is merely provided for backward compatibility
and it may be removed in future versions.

If the detection mechanism is to be used,
it is mandatory to correctly specify
the filename of the main file as the argument of |\childdocmain|:
%
\begin{center}
\begin{tabular}{l}
|\input{childdoc.def}|\\
|\childdocmain{|\textit{main}|}|\\
\end{tabular}
\end{center}
%
If |\jobname| does not match the argument \textit{main} of |\childdocmain|,
it is assumed that |\jobname| points to the child file to be compiled.
When using |\childdocmain| with the main file specified as argument,
it suffices to start a child file
with just |\input{|\textit{main}|}|
without loading of the package and using |\childdocof|.
If instead all processing is done
with the appropriate \textsf{childdoc} directives,
the argument of \textit{main} of |\childdocmain| can be empty.

An alternative version of the command line processing described
in \secref{sec:commandline} using the detection mechanism reads:
%
\begin{center}
|... -jobname "|\textit{target}|" "|[\textit{flags}]%
[|\def\jobname{|\textit{dest}|}|]|\input{|\textit{main}|}"|
\end{center}

%%%%%%%%%%%%%%%%%%%%%%%%%%%%%%%%%%%%%%%%%%%%%%%%%%%%%%%%%%%%%%%%%%%%%%%%%%%%%%%%
\subsection{Manual Code}
\label{sec:manual}

In case one cannot be certain whether the definitions file |childdoc.def|
is installed on the target \TeX{} distribution
and one prefers not to ship it,
it is conceivable to paste a few relevant commands into the sources.

To that end, drop all statements |\input{childdoc.def}|
and perform the replacements as outlined below.
Instead of |\childdocmain{|\textit{main}|}| add the following code
to the top of the main file:
%
\begin{center}
\begin{tabular}{l}
|\||ifdefined\childdocname\endinput\||fi\newif\ifchilddoc|\\
|\edef\childdocname{\scantokens\expandafter{\jobname\noexpand}}|\\
|\def\childdocmain{|\textit{main}|}\||ifx\childdocmain\childdocname\||else|\\
|\childdoctrue\includeonly{\childdocname}\let\jobname\childdocmain\||fi|\\
\end{tabular}
\end{center}
%
Instead of |\childdocof{|\textit{main}|}| just include the main file
at the top of each child file:
%
\begin{center}
|\input{|\textit{main}|}|
\end{center}
%
A simple redirection |\childdocforward{|\textit{dest}|}| is achieved by:
%
\begin{center}
|\def\jobname{|\textit{dest}|}\input{\jobname}|
\end{center}
%
The redirection with prefix
|\childdocforwardprefix[|\textit{prefix}|]{|\textit{dest}|}|
is accomplished by:
%
\begin{center}
\begin{tabular}{l}
|{\edef\jobname{\scantokens\expandafter{\jobname\noexpand}}|\\
|\def\redirectjob |\textit{prefix}|#1~~~{\gdef\jobname{|\textit{dest}|#1}}|\\
|\expandafter\redirectjob\jobname~~~}\input{\jobname}|
\end{tabular}
\end{center}

In an alternative approach,
child documents can be compiled by a specific command line
without additional code or specific definitions:
%
\begin{center}
|... -jobname "|\textit{target}|" "|[\textit{flags}]%
|\includeonly{|\textit{dest}|}\input{|\textit{main}|}"|
\end{center}
%

%%%%%%%%%%%%%%%%%%%%%%%%%%%%%%%%%%%%%%%%%%%%%%%%%%%%%%%%%%%%%%%%%%%%%%%%%%%%%%%%
%%%%%%%%%%%%%%%%%%%%%%%%%%%%%%%%%%%%%%%%%%%%%%%%%%%%%%%%%%%%%%%%%%%%%%%%%%%%%%%%
\section{Information}

%%%%%%%%%%%%%%%%%%%%%%%%%%%%%%%%%%%%%%%%%%%%%%%%%%%%%%%%%%%%%%%%%%%%%%%%%%%%%%%%
\subsection{Copyright}

Copyright \copyright{} 2017--2018 Niklas Beisert

This work may be distributed and/or modified under the
conditions of the \LaTeX{} Project Public License, either version 1.3
of this license or (at your option) any later version.
The latest version of this license is in
  \url{http://www.latex-project.org/lppl.txt}
and version 1.3 or later is part of all distributions of \LaTeX{}
version 2005/12/01 or later.

This work has the LPPL maintenance status `maintained'.

The Current Maintainer of this work is Niklas Beisert.

This work consists of the files |README.txt|, |childdoc.ins| and |childdoc.dtx|
as well as the derived files |childdoc.def|, |cdocsamp.tex|
with |cdocsch1.tex|, |cdocsch2.tex|, |cdocspt3.tex|, |cdocspt4.tex|,
|cdocsdrf.tex|, |cdocsfn1.tex|, |cdocsfn2.tex|
as well as |childdoc.pdf|.

%%%%%%%%%%%%%%%%%%%%%%%%%%%%%%%%%%%%%%%%%%%%%%%%%%%%%%%%%%%%%%%%%%%%%%%%%%%%%%%%
\subsection{Files and Installation}

The package consists of the files:
%
\begin{center}
\begin{tabular}{ll}
    |README.txt|   & readme file \\
    |childdoc.ins| & installation file \\
    |childdoc.dtx| & source file \\
    |childdoc.def| & definition file \\
    |cdocsamp.tex| & sample main file \\
    |cdocsch1.tex| & sample include file \\
    |cdocsch2.tex| & sample include file \\
    |cdocspt3.tex| & sample part file \\
    |cdocspt4.tex| & sample part file \\
    |cdocsdrf.tex| & sample redirection file \\
    |cdocsfn1.tex| & sample redirection file \\
    |cdocsfn2.tex| & sample redirection file \\
    |childdoc.pdf| & manual
\end{tabular}
\end{center}
%
The distribution consists of the files
|README.txt|, |childdoc.ins| and |childdoc.dtx|.
%
\begin{itemize}
\item
Run (pdf)\LaTeX{} on |childdoc.dtx|
to compile the manual |childdoc.pdf| (this file).
\item
Run \LaTeX{} on |childdoc.ins| to create the definitions file |childdoc.def|
and the sample |cdocsamp.tex| with include files
|cdocsch1.tex|, |cdocsch2.tex|, |cdocspt3.tex|, |cdocspt4.tex|,
|cdocsdrf.tex|, |cdocsfn1.tex|, |cdocsfn2.tex|.
Then copy the file |childdoc.def| to an appropriate directory of your \LaTeX{}
distribution, e.g.\ \textit{texmf-root}|/tex/latex/childdoc|.
\end{itemize}

%%%%%%%%%%%%%%%%%%%%%%%%%%%%%%%%%%%%%%%%%%%%%%%%%%%%%%%%%%%%%%%%%%%%%%%%%%%%%%%%
\subsection{Related CTAN Packages}

There are several other packages which offer a similar functionality:
%
\begin{itemize}
\item
The packages
\href{http://ctan.org/pkg/docmute}{\textsf{docmute}},
\href{http://ctan.org/pkg/includex}{\textsf{includex}} and
\href{http://ctan.org/pkg/standalone}{\textsf{standalone}}
provide commands to include only the document body of
a child file thus allowing both files to be compiled individually.
\item
The packages \href{http://ctan.org/pkg/subdocs}{\textsf{subdocs}}
and \href{http://ctan.org/pkg/subfiles}{\textsf{subfiles}}
provide structures in which the main and child documents can be
encapsulated and allowing them to be compiled individually.
The inclusion mechanism is different from the conventional |\include|.
\item
The package \href{http://ctan.org/pkg/combine}{\textsf{combine}}
is an elaborate solution to combine several documents into one.
\end{itemize}
%
See also the CTAN topic \href{http://ctan.org/topic/subdocs}{\textsf{subdocs}}
for further related packages.
The present package differs from the above solutions in that
a document structure constructed with the conventional |\include| mechanism
just needs two extra commands at the top of every file
such that all constituent files can be compiled individually.

%%%%%%%%%%%%%%%%%%%%%%%%%%%%%%%%%%%%%%%%%%%%%%%%%%%%%%%%%%%%%%%%%%%%%%%%%%%%%%%%
%\subsection{Feature Suggestions}
%
%The following is a list of features which may be useful for future
%versions of this package:
%%
%\begin{itemize}
%\item
%\ldots
%\end{itemize}

%%%%%%%%%%%%%%%%%%%%%%%%%%%%%%%%%%%%%%%%%%%%%%%%%%%%%%%%%%%%%%%%%%%%%%%%%%%%%%%%
\subsection{Revision History}

%%%%%%%%%%%%%%%%%%%%%%%%%%%%%%%%%%%%%%%%
\paragraph{v2.0:} 2018/12/30

\begin{itemize}
\item
immediate forward processing
\item
added |\childdocby| mechanism
\item
manual restructured
\end{itemize}

%%%%%%%%%%%%%%%%%%%%%%%%%%%%%%%%%%%%%%%%
\paragraph{v1.6:} 2018/01/17

\begin{itemize}
\item
application for development of include files
\item
corrections to manual
\end{itemize}

%%%%%%%%%%%%%%%%%%%%%%%%%%%%%%%%%%%%%%%%
\paragraph{v1.5:} 2017/05/21

\begin{itemize}
\item
more complete structuring introduced
\item
|\childdocof| introduced
\item
|\childdoc| renamed to |\childdocmain|
\item
|\childredirect| renamed to |\childdocforward| and |\childdocforwardprefix|
and functionality expanded
\end{itemize}

%%%%%%%%%%%%%%%%%%%%%%%%%%%%%%%%%%%%%%%%
\paragraph{v1.0:} 2017/04/27

\begin{itemize}
\item
manual and install package
\item
first version published on CTAN
\end{itemize}

%%%%%%%%%%%%%%%%%%%%%%%%%%%%%%%%%%%%%%%%
\paragraph{v0.6:} 2017/04/26

\begin{itemize}
\item
redirection mechanism added
\end{itemize}

%%%%%%%%%%%%%%%%%%%%%%%%%%%%%%%%%%%%%%%%
\paragraph{v0.5:} 2017/04/26

\begin{itemize}
\item
functionality in definition file
\end{itemize}


%%%%%%%%%%%%%%%%%%%%%%%%%%%%%%%%%%%%%%%%%%%%%%%%%%%%%%%%%%%%%%%%%%%%%%%%%%%%%%%%
%%%%%%%%%%%%%%%%%%%%%%%%%%%%%%%%%%%%%%%%%%%%%%%%%%%%%%%%%%%%%%%%%%%%%%%%%%%%%%%%
%%%%%%%%%%%%%%%%%%%%%%%%%%%%%%%%%%%%%%%%%%%%%%%%%%%%%%%%%%%%%%%%%%%%%%%%%%%%%%%%
\appendix

\settowidth\MacroIndent{\rmfamily\scriptsize 000\ }

 \DocInput{childdoc.dtx}

\end{document}
%</driver>
% \fi
%
% %%%%%%%%%%%%%%%%%%%%%%%%%%%%%%%%%%%%%%%%%%%%%%%%%%%%%%%%%%%%%%%%%%%%%%%%%%%%%%
% %%%%%%%%%%%%%%%%%%%%%%%%%%%%%%%%%%%%%%%%%%%%%%%%%%%%%%%%%%%%%%%%%%%%%%%%%%%%%%
% \section{Sample}
%\iffalse
%<*samplemain>
%\fi
%
% The following presents a sample document
% with two chapters, two parts, a title page,
% a compile flag as well as three forwarding files to set the flag.
% It consists of eight |.tex| files:
% \begin{center}
% \begin{tabular}{ll}
% |cdocsamp.tex|&main file\\
% |cdocsch1.tex|&include file for chapter 1\\
% |cdocsch2.tex|&include file for chapter 2\\
% |cdocspt3.tex|&include file for part 3\\
% |cdocspt4.tex|&include file for part 4\\
% |cdocsdrf.tex|&forwarding file for main file in draft mode\\
% |cdocsfi1.tex|&forwarding file for final version of chapter 1\\
% |cdocsfi2.tex|&forwarding file for final version of chapter 2\\
% \end{tabular}
% \end{center}
% Each of the eight files can be compiled directly by the \LaTeX{} compiler.
%
% %%%%%%%%%%%%%%%%%%%%%%%%%%%%%%%%%%%%%%
% \paragraph{Main File.}
%
% The main file is called |cdocsamp.tex|.
%
% Load the \textsf{childdoc} definitions and
% declare the filename for the main document:
%    \begin{macrocode}
\input{childdoc.def}
\childdocmain{}
%    \end{macrocode}

% Optional override for |\version| flag:
%    \begin{macrocode}
%%\ifchilddoc\else\providecommand{\version}{draft}\fi
%    \end{macrocode}

% Define the default values for the |\version| flag
% (|final| for the main file and |draft| for childs):
%    \begin{macrocode}
\ifchilddoc
\providecommand{\version}{draft}
\else
\providecommand{\version}{final}
\fi
%    \end{macrocode}

% Load the standard document class:
%    \begin{macrocode}
\documentclass[12pt]{article}
%    \end{macrocode}

% Start the document body:
%    \begin{macrocode}
\begin{document}
%    \end{macrocode}

% Declare a title page.
% Print title, part of document being processed and version flag:
%    \begin{macrocode}
\addtocounter{page}{-1}
\begin{center}
{\LARGE\bfseries{}childdoc example\par}
\vspace{1cm}
\ifchilddoc
\ifchilddocmanual part\else chapter\fi:
`\childdocname' of `\childdocjob'\par
\else
main document: `\childdocjob'\par
\fi
version: \version\par
\end{center}
\newpage
%    \end{macrocode}

% Manually include selected file,
% otherwise process as usual:
%    \begin{macrocode}
\ifchilddocmanual
\section*{part `\childdocname'}
\input{\childdocname}
\else
%    \end{macrocode}

% Include the two chapters:
%    \begin{macrocode}
\include{cdocsch1}
\include{cdocsch2}
%    \end{macrocode}

% Include the two parts unless only chapters should be displayed:
%    \begin{macrocode}
\ifchilddoc\else
\section{part three}
\input{cdocspt3}
\section{part four}
\input{cdocspt4}
\fi
%    \end{macrocode}

% Process as usual until here:
%    \begin{macrocode}
\fi
%    \end{macrocode}

% End of document body:
%    \begin{macrocode}
\end{document}
%    \end{macrocode}
%\iffalse
%</samplemain>
%\fi
%
% %%%%%%%%%%%%%%%%%%%%%%%%%%%%%%%%%%%%%%
% \paragraph{Chapter Include Files.}
%
% The include files are called |cdocsch1.tex| and |cdocsch2.tex|.
%
%\iffalse
%<*samplechap1|samplechap2>
%\fi

% Optional override for |\version| flag:
%    \begin{macrocode}
%%\providecommand{\version}{final}
%    \end{macrocode}

% Include the main document:
%    \begin{macrocode}
\input{childdoc.def}
\childdocof{cdocsamp}
%    \end{macrocode}

%\iffalse
%</samplechap1|samplechap2>
%\fi
%
%\iffalse
%<*samplechap1>
%\fi
% Some text for chapter 1:
%    \begin{macrocode}
\section{one}
some text in chapter one
%    \end{macrocode}

%\iffalse
%</samplechap1>
%\fi
% Some text for chapter 2:
%\iffalse
%<*samplechap2>
%\fi
%    \begin{macrocode}
\section{two}
more text in chapter two
%    \end{macrocode}

%\iffalse
%</samplechap2>
%\fi
%
% %%%%%%%%%%%%%%%%%%%%%%%%%%%%%%%%%%%%%%
% \paragraph{Part Include Files.}
%
% The include files are called |cdocspt3.tex| and |cdocspt4.tex|.
%
%\iffalse
%<*samplepart3|samplepart4>
%\fi

% Optional override for |\version| flag:
%    \begin{macrocode}
%%\providecommand{\version}{final}
%    \end{macrocode}

% Include the main document:
%    \begin{macrocode}
\input{childdoc.def}
\childdocby{cdocsamp}
%    \end{macrocode}

%\iffalse
%</samplepart3|samplepart4>
%\fi
%
%\iffalse
%<*samplepart3>
%\fi
% Some text for part 3:
%    \begin{macrocode}
some text in part three
%    \end{macrocode}

%\iffalse
%</samplepart3>
%\fi
% Some text for part 4:
%\iffalse
%<*samplepart4>
%\fi
%    \begin{macrocode}
more text in part four
%    \end{macrocode}

%\iffalse
%</samplepart4>
%\fi
%
% %%%%%%%%%%%%%%%%%%%%%%%%%%%%%%%%%%%%%%
% \paragraph{Forwarding for a Complete Draft.}
%
% The following forwarding file |cdocsdrf.tex|
% compiles the main document in draft mode:
%\iffalse
%<*sampledraft>
%\fi
%    \begin{macrocode}
\def\version{draft}
\input{childdoc.def}
\childdocforward{cdocsamp}
%    \end{macrocode}

%\iffalse
%</sampledraft>
%\fi
%
% %%%%%%%%%%%%%%%%%%%%%%%%%%%%%%%%%%%%%%
% \paragraph{Forwarding for Final Version of the Chapters.}
%
% The following forwarding files |cdocsfn1.tex| and |cdocsfn2.tex|
% (with identical content)
% compile the final versions of the child documents
% |cdocsch1.tex| and |cdocsch2.tex|, respectively:
%\iffalse
%<*samplefinal>
%\fi
%    \begin{macrocode}
\def\version{final}
\input{childdoc.def}
\childdocforwardprefix[cdocsamp]{cdocsfn}{cdocsch}
%    \end{macrocode}

%\iffalse
%</samplefinal>
%\fi
%
% %%%%%%%%%%%%%%%%%%%%%%%%%%%%%%%%%%%%%%
% \paragraph{Command Line Processing.}
%
% The following three command lines generate the output files
% |cdocscld|, |cdocscl1| and |cdocscl2|
% which should be identical to
% |cdocsdrf|, |cdocsch1| and |cdocsfn2|, respectively:
% \begin{center}
% \begin{tabular}{l}
% |latex -jobname cdocscld \|\\
% |  "\def\version{draft}\input{childdoc.def}\childdocforward{cdocsamp}"|\\
% |latex -jobname cdocscl1 \|\\
% |  "\input{childdoc.def}\childdocforward[cdocsamp]{cdocsch1}"|\\
% |latex -jobname cdocscl2 \|\\
% |  "\def\version{final}\input{childdoc.def}\childdocforward{cdocsch2}"|
% \end{tabular}
% \end{center}
% Note that the trailing backslash on each first line
% merely continues the input to the second line
% (for convenient cut ant paste).
% Furthermore, the command |latex| can be replaced by any
% of its alternative versions such as |pdflatex|.
%
% %%%%%%%%%%%%%%%%%%%%%%%%%%%%%%%%%%%%%%%%%%%%%%%%%%%%%%%%%%%%%%%%%%%%%%%%%%%%%%
% %%%%%%%%%%%%%%%%%%%%%%%%%%%%%%%%%%%%%%%%%%%%%%%%%%%%%%%%%%%%%%%%%%%%%%%%%%%%%%
% \section{Implementation}
%\iffalse
%<*package>
%\fi
%
% This section describes the definitions file |childdoc.def|.

% The definitions cannot be loaded using |\usepackage| or |\RequirePackage|
% which has a mechanism to prevent loading a style file more than once.
% When loading the definitions by means of |\input|
% multiple instances have to be prevented manually:
%\iffalse
%This code needs to be before the `\ProvidesFile' directive
%which is defined at the beginning of this file.
%Therefore it is also placed there and commented out here.
%</package>
%<*discard>
%\fi
%    \begin{macrocode}
\ifdefined\childdocmain\endinput\fi
%    \end{macrocode}
%\iffalse
%</discard>
%<*package>
%\fi
%
% \macro{\ifchilddoc}
% \macro{\ifchilddocmanual}
% The conditional |\ifchilddoc| tells whether a
% child (true) or main (false) document is being compiled.
% The conditional |\ifchilddocmanual| tells whether
% the |\includeonly| mechanism is used (false) or
% the selection of child files must be performed manually (true).
% The definitions initialise to false:
%    \begin{macrocode}
\newif\ifchilddoc
\newif\ifchilddocmanual
%    \end{macrocode}

% \macro{\childdocname}
% \macro{\childdocjob}
% The macro |\childdocname| stores the name of the main document
% to be compiled. The macro |\childdocjob| stores the name of
% the document on which the \LaTeX{} compiler was originally invoked.
% The content of |\jobname| cannot be compared
% to filenames specified in the source due to different catcodes.
% The following code rescans |\jobname|, stores the result
% in |\childdocname| and saves a copy in |\childdocjob|:
%    \begin{macrocode}
\edef\childdocname{\scantokens\expandafter{\jobname\noexpand}}
\let\childdocjob\childdocname
%    \end{macrocode}

% \macro{\childdocdisable}
% The macro |\childdocdisable| prevents the main file
% from being processed more than once.
% At this stage, the main document command |\childdocmain|
% is assumed to be called once again where it should do nothing.
% Any subsequent call to it should prevent
% a secondary processing of the main document
% It overwrites the forwarding commands
% |\childdocof| and |\childdocforward|
% with empty macros to prevent further inclusions of the main document:
%    \begin{macrocode}
\newcommand{\childdocdisable}
{
  \renewcommand{\childdocmain}[1]{\renewcommand{\childdocmain}[1]{\endinput}}
  \renewcommand{\childdocof}[1]{}
  \renewcommand{\childdocby}[2][]{}
  \renewcommand{\childdocforward}[2][]{}
  \renewcommand{\childdocdisable}{}
}
%    \end{macrocode}

% \macro{\childdocmain}
% The macro |\childdocmain| is to be called at the top of the main file
% with nothing or the main filename (without extension) as argument.
% First, it breaks loops.
% If the argument is not empty and does not match |\childdocname|
% (which is set by the first inclusion of |childdoc.def|),
% |\ifchilddoc| is set to true, |\includeonly| is applied to the child file
% and |\jobname| is set to the main file
% (for proper handling of |.aux| files):
%    \begin{macrocode}
\newcommand{\childdocmain}[1]
{
  \childdocdisable\childdocmain{}
  \if?#1?\else
    \begingroup
      \def\childdoctmp{#1}
      \ifx\childdoctmp\childdocname
        \def\childdoctmp{}
      \else
        \def\childdoctmp
        {
          \childdoctrue
          \includeonly{\childdocname}
          \def\childdocjob{#1}
          \def\jobname{#1}
        }
      \fi
      \expandafter
    \endgroup
    \childdoctmp
  \fi
}
%    \end{macrocode}

% \macro{\childdocof}
% The command |\childdocof| redirects
% compilation to the main file |#1|.
%    \begin{macrocode}
\newcommand{\childdocof}[1]
{
  \childdocdisable
  \childdoctrue
  \includeonly{\childdocname}
  \def\jobname{#1}
  \def\childdocjob{#1}
  \input{#1}
}
%    \end{macrocode}

% \macro{\childdocby}
% The command |\childdocby| ....
%    \begin{macrocode}
\newcommand{\childdocby}[2][]
{
  \childdocdisable
  \childdoctrue
  \childdocmanualtrue
  \if?#1?\else
    \def\jobname{#2}
  \fi
  \def\childdocjob{#2}
  \input{#2}
  \endinput
}
%    \end{macrocode}

% \macro{\childdocforward}
% The command |\childdocforward| redirects
% compilation to the main file or
% (if the optional argument is given) a child file.
% Parameters are set as if the main file
% or a child file starting with |\childdocof| was compiled.
% Then compilation is handed over to the main file:
%    \begin{macrocode}
\newcommand{\childdocforward}[2][]
{
  \begingroup
    \if?#1?
      \def\childdoctmp
      {
        \def\childdocname{#2}
        \def\childdocjob{#2}
        \def\jobname{#2}
        \input{#2}
        \endinput
      }
    \else
      \def\childdoctmp
      {
        \childdocdisable
        \def\childdocname{#2}
        \childdoctrue
        \includeonly{#2}
        \def\childdocjob{#1}
        \def\jobname{#1}
        \input{#1}
        \endinput
      }
    \fi
    \expandafter
  \endgroup
  \childdoctmp
}
%    \end{macrocode}

% \macro{\childdocforwardprefix}
% The command |\childdocforwardprefix| redirects
% compilation to the main or a child file by means of a pattern.
% The prefix |#1| in the current filename is replaced by |#2|
% and the suffix of the current filename is kept
% (it is assumed that the filename does not contain the substring `|~~~|'
% which is used as a delimiter).
% Compilation is handed over to the new file by |\childdocforward|:
%    \begin{macrocode}
\newcommand{\childdocforwardprefix}[3][]
{
  \begingroup
    \def\childdocextract #2##1~~~{\def\childdoctmp{\childdocforward[#1]{#3##1}}}
    \expandafter\childdocextract\childdocname~~~
    \expandafter
  \endgroup
  \childdoctmp
}
%    \end{macrocode}

% \macro{\childdoc}
% The deprecated macro |\childdoc| is a legacy version of |\childdocmain|:
%    \begin{macrocode}
\newcommand{\childdoc}{\childdocmain}
%    \end{macrocode}

% \macro{\childdocredirect}
% The deprecated macro |\childdocredirect| is a legacy version
% of |\childdocforward| and |\childdocforwardprefix|:
%    \begin{macrocode}
\newcommand{\childdocredirect}[2][]
{
  \begingroup
    \if?#1?
      \def\childdoctmp{\childdocforward{#2}}
    \else
      \def\childdoctmp{\childdocforwardprefix{#1}{#2}}
    \fi
    \expandafter
  \endgroup
  \childdoctmp
}
%    \end{macrocode}

%\iffalse
%</package>
%\fi
%
\endinput

\childdocforward{cdocsamp}
%    \end{macrocode}

%\iffalse
%</sampledraft>
%\fi
%
% %%%%%%%%%%%%%%%%%%%%%%%%%%%%%%%%%%%%%%
% \paragraph{Forwarding for Final Version of the Chapters.}
%
% The following forwarding files |cdocsfn1.tex| and |cdocsfn2.tex|
% (with identical content)
% compile the final versions of the child documents
% |cdocsch1.tex| and |cdocsch2.tex|, respectively:
%\iffalse
%<*samplefinal>
%\fi
%    \begin{macrocode}
\def\version{final}
% \iffalse
%
% childdoc.dtx Copyright (C) 2017-2018 Niklas Beisert
%
% This work may be distributed and/or modified under the
% conditions of the LaTeX Project Public License, either version 1.3
% of this license or (at your option) any later version.
% The latest version of this license is in
%   http://www.latex-project.org/lppl.txt
% and version 1.3 or later is part of all distributions of LaTeX
% version 2005/12/01 or later.
%
% This work has the LPPL maintenance status `maintained'.
%
% The Current Maintainer of this work is Niklas Beisert.
%
% This work consists of the files childdoc.dtx and childdoc.ins
% and the derived files childdoc.def and cdocsamp.tex with
% cdocsch1.tex, cdocsch2.tex, cdocsdrf.tex, cdocsfn1.tex, cdocsfn2.tex.
%
%<package>\ifdefined\childdocmain\endinput\fi
%<package>\ProvidesFile{childdoc.def}[2018/12/30 v2.0 child document driver]
%<samplemain>\ProvidesFile{cdocsamp.tex}[2018/12/30 v2.0 sample for childdoc]
%<*driver>
%\ProvidesFile{childdoc.drv}[2018/12/30 v2.0 childdoc reference manual file]
\PassOptionsToClass{10pt,a4paper}{article}
\documentclass{ltxdoc}

\usepackage[margin=35mm]{geometry}
\usepackage{hyperref}
\usepackage{hyperxmp}
\usepackage[usenames]{color}

\hypersetup{colorlinks=true}
\hypersetup{pdfstartview=FitH}
\hypersetup{pdfpagemode=UseNone}
\hypersetup{pdfsource={}}
\hypersetup{pdflang={en-UK}}
\hypersetup{pdfcopyright={Copyright 2017-2018 Niklas Beisert.
  This work may be distributed and/or modified under the
  conditions of the LaTeX Project Public License, either version 1.3
  of this license or (at your option) any later version.}}
\hypersetup{pdflicenseurl={http://www.latex-project.org/lppl.txt}}
\hypersetup{pdfcontactaddress={ETH Zurich, ITP, HIT K,
  Wolfgang-Pauli-Strasse 27}}
\hypersetup{pdfcontactpostcode={8093}}
\hypersetup{pdfcontactcity={Zurich}}
\hypersetup{pdfcontactcountry={Switzerland}}
\hypersetup{pdfcontactemail={nbeisert@itp.phys.ethz.ch}}
\hypersetup{pdfcontacturl={http://people.phys.ethz.ch/\xmptilde nbeisert/}}

\newcommand{\secref}[1]{\hyperref[#1]{section \ref*{#1}}}

\parskip1ex
\parindent0pt
\let\olditemize\itemize
\def\itemize{\olditemize\parskip0pt}

\begin{document}

\title{The \textsf{childdoc} Package}
\hypersetup{pdftitle={The childdoc Package}}
\author{Niklas Beisert\\[2ex]
  Institut f\"ur Theoretische Physik\\
  Eidgen\"ossische Technische Hochschule Z\"urich\\
  Wolfgang-Pauli-Strasse 27, 8093 Z\"urich, Switzerland\\[1ex]
  \href{mailto:nbeisert@itp.phys.ethz.ch}
  {\texttt{nbeisert@itp.phys.ethz.ch}}}
\hypersetup{pdfauthor={Niklas Beisert}}
\hypersetup{pdfsubject={Manual for the LaTeX2e Package childdoc}}
\date{30 December 2018, \textsf{v2.0}}
\maketitle

\begin{abstract}\noindent
\textsf{childdoc} is a \LaTeXe{} package
that enables the direct compilation
of document sections included by |\include|
to individual files.
\end{abstract}

\begingroup
\parskip0ex
\tableofcontents
\endgroup

%%%%%%%%%%%%%%%%%%%%%%%%%%%%%%%%%%%%%%%%%%%%%%%%%%%%%%%%%%%%%%%%%%%%%%%%%%%%%%%%
%%%%%%%%%%%%%%%%%%%%%%%%%%%%%%%%%%%%%%%%%%%%%%%%%%%%%%%%%%%%%%%%%%%%%%%%%%%%%%%%
\section{Introduction}

\LaTeX{} provides a mechanism to structure a large document (such as a book)
into a main file and several child files (containing the chapters)
using the |\include| command.
This mechanism is beneficial for documents
which span hundreds of pages in order to
make the source file(s) more manageable.
Moreover, compilation can be restricted to
selected child files by means of the |\includeonly| command.
The latter feature can be used to reduce the compilation time while editing
(this was significantly more useful in the earlier days of \LaTeX{})
or to generate a smaller document which is easier to navigate.
Another application of |\includeonly| is to generate
documents consisting of selected parts of the complete document.

However, there are a few drawbacks of the plain |\include| mechanism:
\begin{itemize}
\item
The child files cannot be compiled on their own,
they can only be compiled via the main file.
A naive editing environment
(such as a text editor with an option
to have the current file processed by \LaTeX)
may require one to switch to the main file before compiling;
attempting to compile the child file produces errors.
\item
The main file must be modified (each time)
to adjust the |\includeonly| command
to the present needs. This easily leaves the main file in a messy state.
\item
The generated document will always carry the filename
of the main document. This is inconvenient if
several child files are to be compiled and
to be kept for distribution.
\end{itemize}

The present package provides a simple interface
to make child files individually compilable by \LaTeX{}.
Compiling a child file then has the same effect as compiling
the main file with an |\includeonly| command
to select the appropriate child.
Moreover the generated document will carry the name of the child
rather than the main file.
This resolves all three above issues.

This feature is meant to make the editing of books,
thesis documents and lecture notes somewhat more convenient.
However, the package can also be used efficiently for
composing a series of documents (such as exercise sheets)
which are typically distributed individually.
It then assists the author in generating the individual documents
(potentially in different versions)
as well as a document containing the collected series.
Another application is in developing style files
or other kinds of included material
where compilation of the style file could redirect
to a sample or test file.

%%%%%%%%%%%%%%%%%%%%%%%%%%%%%%%%%%%%%%%%%%%%%%%%%%%%%%%%%%%%%%%%%%%%%%%%%%%%%%%%
%%%%%%%%%%%%%%%%%%%%%%%%%%%%%%%%%%%%%%%%%%%%%%%%%%%%%%%%%%%%%%%%%%%%%%%%%%%%%%%%
\section{Usage}

First of all, the package \textsf{childdoc} is \emph{not} a standard
\LaTeXe{} |.sty| style file! Therefore it needs to be invoked in
a non-standard way.

%%%%%%%%%%%%%%%%%%%%%%%%%%%%%%%%%%%%%%%%%%%%%%%%%%%%%%%%%%%%%%%%%%%%%%%%%%%%%%%%
\subsection{Included Files}
\label{sec:include}

%%%%%%%%%%%%%%%%%%%%%%%%%%%%%%%%%%%%%%%%
\DescribeMacro{\childdocmain}
To use the package, add the commands
\begin{center}
\begin{tabular}{l}
|\input{childdoc.def}|\\
|\childdocmain{}|\\
\end{tabular}
\end{center}
at the very top of the main \LaTeX{} file,
in particular \emph{before} the |\documentclass| statement!
The argument of |\childdocmain| should be left empty
(but it must be present).

%%%%%%%%%%%%%%%%%%%%%%%%%%%%%%%%%%%%%%%%
\DescribeMacro{\childdocof}
Furthermore, add the commands
\begin{center}
\begin{tabular}{l}
|\input{childdoc.def}|\\
|\childdocof{|\textit{main}|}|\\
\end{tabular}
\end{center}
at the top of every child file \textit{child}
which is included by |\include{|\textit{child}|}|
from within the main file
(or at least for those files to be compiled individually).
The argument \textit{main} must be the filename of the main file.

There are a couple of
considerations in setting up the main and child documents:

%%%%%%%%%%%%%%%%%%%%%%%%%%%%%%%%%%%%%%%%
\paragraph{Restrictions.}

Please note the following restrictions:
\begin{itemize}
\item
|\childdocmain| must be called with one argument \textit{main}
to ensure compatibility with earlier version of the package.
It must either be empty (|\childdocmain{}|)
or precisely match the filename of the main file in which it is specified.
See \secref{sec:detection} for further information.
\item
The filename \textit{main} must be specified without the |.tex| extension.
\item
The filename \textit{main} is case sensitive
(even in case-insensitive file systems)
due to internal string comparison.
\item
The argument \textit{main} should be fully expanded, it cannot be a macro.
\item
Subdirectories and special characters should be avoided in filenames.
\item
The command |\childdocmain{|\textit{main}|}| must be followed by a whitespace.
It should not be followed immediately by another command
or by a comment mark `|%|'.
This is because the \TeX{} parser reads the token immediately following
the argument of |\childdocmain| and puts it
at the beginning of every child section;
however, a white\-space is ignored.
\end{itemize}

%%%%%%%%%%%%%%%%%%%%%%%%%%%%%%%%%%%%%%%%
\paragraph{Content of Main File.}

It is advisable to place all content in the child files included by |\include|.
Any output contained in the main file will appear in all child documents
unless suppressed manually;
it cannot be suppressed automatically by the |\includeonly| directive
and thus should normally be avoided.
A method to include some content in the main file
by means of conditional processing is described in \secref{sec:conditional}.

%%%%%%%%%%%%%%%%%%%%%%%%%%%%%%%%%%%%%%%%
\paragraph{Page Numbering.}

When only a part of the document is compiled,
the appropriate numbering of pages
(as well as other status parameters)
is determined from the |.aux| files.
The latter contain information from previous passes.
However this information needs to propagate through
all intermediate child documents.
Therefore the page numbering in child documents may well
be inconsistent until the complete document is compiled at least once.

A useful (if unconventional) way to always ensure a consistent
page numbering is to restart the numbering in each child document
and denote the pages by `\textit{child}|.|\textit{page}'
where \textit{child} represents the chapter/section number of the child file.
This can be achieved by the command
|\numberwithin{page}{|\textit{child}|}|
of the \textsf{amsmath} package
where \textit{child} can be |chapter| or |section|
depending on the chosen structuring.
Alternatively, one can modify the macro |\thepage| appropriately
and reset the counter |page| at the start of each child file.

%%%%%%%%%%%%%%%%%%%%%%%%%%%%%%%%%%%%%%%%%%%%%%%%%%%%%%%%%%%%%%%%%%%%%%%%%%%%%%%%
\subsection{Conditional Processing}
\label{sec:conditional}

The package provides a mechanism to compile different versions
of a document. To customise the versions further some conditional processing
can come in handy to distinguish which version is being compiled.
The package provides two macros to describe the compilation context:

%%%%%%%%%%%%%%%%%%%%%%%%%%%%%%%%%%%%%%%%
\DescribeMacro{\ifchilddoc}
The conditional |\ifchilddoc| distinguishes between the compilation of
child documents and the main document:
%
\begin{center}
|\ifchilddoc |\textit{child-code}| |[|\||else |\textit{main-code}]| \||fi|
\end{center}

%%%%%%%%%%%%%%%%%%%%%%%%%%%%%%%%%%%%%%%%
\DescribeMacro{\childdocname}
\DescribeMacro{\childdocjob}
The macro |\childdocname| contains the filename (without extension)
of the main or child file being processed.
Note that |\childdocjob| will always contain the name of the main file.

%%%%%%%%%%%%%%%%%%%%%%%%%%%%%%%%%%%%%%%%
\paragraph{Title Page.}

Conditional processing can be used to include a title or banner page
in the main document when proper precautions are taken.
Importantly, the code in the main file should ensure that the page counter
(as well as other status parameters which are stored in the |.aux| files)
takes the same value after the conditional processing.
Otherwise the page numbers may take divergent values
depending on which part is compiled.

For example, a title page could be declared by:
%
\begin{center}
\begin{tabular}{l}
|\ifchilddoc\||else|\\
|\addtocounter{page}{-1}|\\
\textit{code for title page}\\
|\newpage|\\
|\||fi|
\end{tabular}
\end{center}
%
A banner page for the child documents can be generated by:
%
\begin{center}
\begin{tabular}{l}
|\ifchilddoc|\\
|\addtocounter{page}{-1}|\\
\textit{code for banner page}\\
|\newpage|\\
|\||fi|
\end{tabular}
\end{center}
%
Here one could write a message such as:
\begin{center}
|This is the part \childdocname{} of \childdocjob{}.|
\end{center}

%%%%%%%%%%%%%%%%%%%%%%%%%%%%%%%%%%%%%%%%%%%%%%%%%%%%%%%%%%%%%%%%%%%%%%%%%%%%%%%%
\subsection{Flags}
\label{sec:flags}

The package makes it easy to generate different versions
of the main or child documents.
To this end compilation flags can be defined
and assigned different default values.
They will be particularly useful in conjunction
with the forwarding mechanism described in \secref{sec:forward}.

For example, it may be useful to have a flag |\version|
which can be set to |draft| or |final|.
The document source will contain some conditional code
depending on the value of |\version|.
Suppose further, the flag should default to |final| for the main file
and to |draft| for child files
which is a natural assignment for editing the document.
This is achieved by placing the following code
in the preamble of the main document
(below the |\childdocmain| directive):
%
\begin{center}
\begin{tabular}{l}
|\ifchilddoc|\\
|\providecommand{\version}{draft}|\\
|\||else|\\
|\providecommand{\version}{final}|\\
|\||fi|
\end{tabular}
\end{center}
%
The definition by |\providecommand| makes sure
that previous definitions are not overwritten.
Further statements |\providecommand{\version}{...}|
can thus be added before the above code to override it.

For the main file, one might add a line
(between |\childdocmain| and the above block)
%
\begin{center}
|%\ifchilddoc\||else\providecommand{\version}{draft}\||fi|
\end{center}
%
which can be uncommented to produce a draft version.
Likewise one can add a line to the very top of a child file
(above the |\childdocof{|\textit{main}|}| directive)
%
\begin{center}
|%\providecommand{\version}{final}|
\end{center}
%
which can be uncommented to produce the final version of this child document.

%%%%%%%%%%%%%%%%%%%%%%%%%%%%%%%%%%%%%%%%%%%%%%%%%%%%%%%%%%%%%%%%%%%%%%%%%%%%%%%%
\subsection{Forwarding}
\label{sec:forward}

Different versions of the main or child documents
using compilation flags as described in \secref{sec:flags}
can be (permanently) stored in different files
for convenient compilation, viewing and distribution.
To this end, the package defines a command
to pass on compilation to a different file:

%%%%%%%%%%%%%%%%%%%%%%%%%%%%%%%%%%%%%%%%
\DescribeMacro{\childdocforward}
The command |\childdocforward| redirects processing to
another source file:
%
\begin{center}
\begin{tabular}{l}
|\input{childdoc.def}|\\
|\childdocforward[|\textit{main}|]{|\textit{dest}|}|\\
\end{tabular}
\end{center}
%
The argument \textit{dest} is the destination file
(without extension).
It should be the main file or one of the child files.
Note that further \textsf{childdoc} directives
such as |\childdocof| and |\childdocforward|
in the indicated file will be processed in this form.
The optional argument \textit{main}
passes on directly to the main file \textit{main}
while pretending to compile the child \textit{dest}.
This form behaves as if \textit{dest}
issues |\childdocof{|\textit{main}|}| right away,
and no further \textsf{childdoc} directives will be processed.

%%%%%%%%%%%%%%%%%%%%%%%%%%%%%%%%%%%%%%%%
\DescribeMacro{\...prefix}
In the alternative form |\childdocforwardprefix|,
%
\begin{center}
\begin{tabular}{l}
|\input{childdoc.def}|\\
|\childdocforwardprefix[|\textit{main}|]{|\textit{prefix}|}{|\textit{dest}|}|
\end{tabular}
\end{center}
%
the destination file is determined by a pattern
depending on the current file:
To make this work, the current file must be called
`{\textit{prefix}\hspace{0.2em}\textit{suffix}}'
with \textit{prefix} matching precisely the argument.
Processing is then passed on to the file
`{\textit{dest}\hspace{0.2em}\textit{suffix}}'.
Surely, the same effect is achieved by
directly specifying the
argument `{\textit{dest}\hspace{0.2em}\textit{suffix}}'
in the first form.
However, that requires to set up a different file
for each child. With the alternative form of the command
all these files can have exactly the same content
which simplifies setting them up and maintaining them.

For example, the following file |draft.tex|
with a compilation flag |\version| as described in \secref{sec:flags}
compiles the main document as a draft:
%
\begin{center}
\begin{tabular}{l}
|\def\version{draft}|\\
|\input{childdoc.def}|\\
|\childdocforward{|\textit{main}|}|
\end{tabular}
\end{center}
%
Likewise, the following files |final|\textit{nn}|.tex|
compile the final version of the child document
|child|\textit{nn}|.tex|:
%
\begin{center}
\begin{tabular}{l}
|\def\version{final}|\\
|\input{childdoc.def}|\\
|\childdocforwardprefix{final}{child}|
\end{tabular}
\end{center}
%

Note that when several versions of a main file and/or of each child file
are to be generated, it may be convenient to set up a |Makefile| or
shell script to automatise the process.

%%%%%%%%%%%%%%%%%%%%%%%%%%%%%%%%%%%%%%%%%%%%%%%%%%%%%%%%%%%%%%%%%%%%%%%%%%%%%%%%
\subsection{Command Line Processing}
\label{sec:commandline}

The effect of redirection files can also be achieved by invoking
the \LaTeX{} compiler with a more elaborate command line.
Most conveniently this should be done as part
of a shell script or a |Makefile|.

When using \textsf{childdoc} in the main file, the following
command lines effectively perform a redirection
(note that depending on the shell being used,
backslashes may have to be doubled: `|\|' $\to$ `|\\|'):
%
\begin{center}
|... -jobname "|\textit{target}|" |\\|"|[\textit{flags}]%
|\input{childdoc.def}\childdocforward[|\textit{main}|]{|\textit{dest}|}"|
\end{center}
%
Here \textit{target} is the name of the output file,
\textit{main} is the name of the main file
and \textit{dest} is the name of the main or child file to be processed
(all filenames without extensions).
The optional argument \textit{main} can be omitted
if \textit{main} matches \textit{dest}.
Optionally, compilation \textit{flags} can be defined via |\def| commands.
This command line makes the \TeX{} engine believe
it is compiling the file \textit{target}
whose content is specified as the latter parameter.
The provided code then forwards the processing to
\textit{main} or \textit{dest} as described in \secref{sec:forward}.

%%%%%%%%%%%%%%%%%%%%%%%%%%%%%%%%%%%%%%%%%%%%%%%%%%%%%%%%%%%%%%%%%%%%%%%%%%%%%%%%
\subsection{Include by Input}
\label{sec:input}

Including child documents by |\include| has some restrictions by design.
Most notably, the content of a child document always occupies
its own set of pages; pages cannot be shared between child documents.
Usually, this behaviour makes perfect sense
because each child document contain an essential part of the document.
However, in some situations it may be desirable to compose
a document from a collection of parts
without having mandatory page breaks between then.
For this case, the package
provides a mechanism to include parts
by |\input| which can also be processed individually.
However, by construction this mechanism
requires manual handling of the content to be output.

%%%%%%%%%%%%%%%%%%%%%%%%%%%%%%%%%%%%%%%%
\DescribeMacro{\ifchilddocmanual}
The main file should be prepared as usual, see \secref{sec:include}.
However, the document body must make a distinction
between processing of an individual part and of the main document, e.g.:
%
\begin{center}
\begin{tabular}{l}
|\ifchilddocmanual|\\
|\input{\childdocname}|\\
|\||else|\\
\textit{document body with }|\input{|\textit{part}|}|\\
|\||fi|
\end{tabular}
\end{center}
%
The conditional |\ifchilddocmanual| is true whenever
a part to be included by |\input| is being compiled,
and the name of the part is stored in |\childdocname|.

%%%%%%%%%%%%%%%%%%%%%%%%%%%%%%%%%%%%%%%%
\DescribeMacro{\childdocby}
Each part to be included by |\input| should start with:
%
\begin{center}
\begin{tabular}{l}
|\input{childdoc.def}|\\
|\childdocby{|\textit{main}|}|\\
\end{tabular}
\end{center}
%
The directive |\childdocby| is similar to |\childdocof|
described in \secref{sec:include},
but the subsequent selection of content must be done manually.
To that end, both |\ifchilddoc| and |\ifchilddocmanual|
will be true upon processing of a part,
and the name of the part is stored in |\childdocname|.
Note that |\jobname| will be set to the filename of the current part
so that each part receives an individual |.aux| file
that does not interfere with the |.aux| file(s) of the main document.
This behaviour can be altered by the alternative form
|\childdocby[*]{|\textit{main}|}| (with a non-empty optional argument)
which uses the |.aux| file of the main document
by setting |\jobname| to \textit{main}.

%%%%%%%%%%%%%%%%%%%%%%%%%%%%%%%%%%%%%%%%%%%%%%%%%%%%%%%%%%%%%%%%%%%%%%%%%%%%%%%%
\subsection{Driver Development}
\label{sec:driver}

The \textsf{childdoc} mechanism can also be use for the development
of definition files such as \LaTeX{} styles or classes.
This case differs from the above setup with multiple parts
included by |\include| in that no |\includeonly| should be invoked.
This can be achieved by starting the include file
(before |\ProvidesPackage|) with:
%
\begin{center}
\begin{tabular}{l}
|\input{childdoc.def}|\\
|\childdocforward{|\textit{main}|}|\\
\end{tabular}
\end{center}
%
or alternatively with:
%
\begin{center}
\begin{tabular}{l}
|\input{childdoc.def}|\\
|\childdocby{|\textit{main}|}|\\
\end{tabular}
\end{center}
%
Both forms have slightly different effects as described above.
The main file is prepared as usual, see \secref{sec:include}.

%%%%%%%%%%%%%%%%%%%%%%%%%%%%%%%%%%%%%%%%%%%%%%%%%%%%%%%%%%%%%%%%%%%%%%%%%%%%%%%%
\subsection{Legacy Detection}
\label{sec:detection}

The directive |\childdocmain| in the main file can detect
whether the complete document or merely a child is to be compiled
even without using the directive |\childdocof|.
This method is deprecated because it is less robust
and there is no compelling reason to use it;
it is merely provided for backward compatibility
and it may be removed in future versions.

If the detection mechanism is to be used,
it is mandatory to correctly specify
the filename of the main file as the argument of |\childdocmain|:
%
\begin{center}
\begin{tabular}{l}
|\input{childdoc.def}|\\
|\childdocmain{|\textit{main}|}|\\
\end{tabular}
\end{center}
%
If |\jobname| does not match the argument \textit{main} of |\childdocmain|,
it is assumed that |\jobname| points to the child file to be compiled.
When using |\childdocmain| with the main file specified as argument,
it suffices to start a child file
with just |\input{|\textit{main}|}|
without loading of the package and using |\childdocof|.
If instead all processing is done
with the appropriate \textsf{childdoc} directives,
the argument of \textit{main} of |\childdocmain| can be empty.

An alternative version of the command line processing described
in \secref{sec:commandline} using the detection mechanism reads:
%
\begin{center}
|... -jobname "|\textit{target}|" "|[\textit{flags}]%
[|\def\jobname{|\textit{dest}|}|]|\input{|\textit{main}|}"|
\end{center}

%%%%%%%%%%%%%%%%%%%%%%%%%%%%%%%%%%%%%%%%%%%%%%%%%%%%%%%%%%%%%%%%%%%%%%%%%%%%%%%%
\subsection{Manual Code}
\label{sec:manual}

In case one cannot be certain whether the definitions file |childdoc.def|
is installed on the target \TeX{} distribution
and one prefers not to ship it,
it is conceivable to paste a few relevant commands into the sources.

To that end, drop all statements |\input{childdoc.def}|
and perform the replacements as outlined below.
Instead of |\childdocmain{|\textit{main}|}| add the following code
to the top of the main file:
%
\begin{center}
\begin{tabular}{l}
|\||ifdefined\childdocname\endinput\||fi\newif\ifchilddoc|\\
|\edef\childdocname{\scantokens\expandafter{\jobname\noexpand}}|\\
|\def\childdocmain{|\textit{main}|}\||ifx\childdocmain\childdocname\||else|\\
|\childdoctrue\includeonly{\childdocname}\let\jobname\childdocmain\||fi|\\
\end{tabular}
\end{center}
%
Instead of |\childdocof{|\textit{main}|}| just include the main file
at the top of each child file:
%
\begin{center}
|\input{|\textit{main}|}|
\end{center}
%
A simple redirection |\childdocforward{|\textit{dest}|}| is achieved by:
%
\begin{center}
|\def\jobname{|\textit{dest}|}\input{\jobname}|
\end{center}
%
The redirection with prefix
|\childdocforwardprefix[|\textit{prefix}|]{|\textit{dest}|}|
is accomplished by:
%
\begin{center}
\begin{tabular}{l}
|{\edef\jobname{\scantokens\expandafter{\jobname\noexpand}}|\\
|\def\redirectjob |\textit{prefix}|#1~~~{\gdef\jobname{|\textit{dest}|#1}}|\\
|\expandafter\redirectjob\jobname~~~}\input{\jobname}|
\end{tabular}
\end{center}

In an alternative approach,
child documents can be compiled by a specific command line
without additional code or specific definitions:
%
\begin{center}
|... -jobname "|\textit{target}|" "|[\textit{flags}]%
|\includeonly{|\textit{dest}|}\input{|\textit{main}|}"|
\end{center}
%

%%%%%%%%%%%%%%%%%%%%%%%%%%%%%%%%%%%%%%%%%%%%%%%%%%%%%%%%%%%%%%%%%%%%%%%%%%%%%%%%
%%%%%%%%%%%%%%%%%%%%%%%%%%%%%%%%%%%%%%%%%%%%%%%%%%%%%%%%%%%%%%%%%%%%%%%%%%%%%%%%
\section{Information}

%%%%%%%%%%%%%%%%%%%%%%%%%%%%%%%%%%%%%%%%%%%%%%%%%%%%%%%%%%%%%%%%%%%%%%%%%%%%%%%%
\subsection{Copyright}

Copyright \copyright{} 2017--2018 Niklas Beisert

This work may be distributed and/or modified under the
conditions of the \LaTeX{} Project Public License, either version 1.3
of this license or (at your option) any later version.
The latest version of this license is in
  \url{http://www.latex-project.org/lppl.txt}
and version 1.3 or later is part of all distributions of \LaTeX{}
version 2005/12/01 or later.

This work has the LPPL maintenance status `maintained'.

The Current Maintainer of this work is Niklas Beisert.

This work consists of the files |README.txt|, |childdoc.ins| and |childdoc.dtx|
as well as the derived files |childdoc.def|, |cdocsamp.tex|
with |cdocsch1.tex|, |cdocsch2.tex|, |cdocspt3.tex|, |cdocspt4.tex|,
|cdocsdrf.tex|, |cdocsfn1.tex|, |cdocsfn2.tex|
as well as |childdoc.pdf|.

%%%%%%%%%%%%%%%%%%%%%%%%%%%%%%%%%%%%%%%%%%%%%%%%%%%%%%%%%%%%%%%%%%%%%%%%%%%%%%%%
\subsection{Files and Installation}

The package consists of the files:
%
\begin{center}
\begin{tabular}{ll}
    |README.txt|   & readme file \\
    |childdoc.ins| & installation file \\
    |childdoc.dtx| & source file \\
    |childdoc.def| & definition file \\
    |cdocsamp.tex| & sample main file \\
    |cdocsch1.tex| & sample include file \\
    |cdocsch2.tex| & sample include file \\
    |cdocspt3.tex| & sample part file \\
    |cdocspt4.tex| & sample part file \\
    |cdocsdrf.tex| & sample redirection file \\
    |cdocsfn1.tex| & sample redirection file \\
    |cdocsfn2.tex| & sample redirection file \\
    |childdoc.pdf| & manual
\end{tabular}
\end{center}
%
The distribution consists of the files
|README.txt|, |childdoc.ins| and |childdoc.dtx|.
%
\begin{itemize}
\item
Run (pdf)\LaTeX{} on |childdoc.dtx|
to compile the manual |childdoc.pdf| (this file).
\item
Run \LaTeX{} on |childdoc.ins| to create the definitions file |childdoc.def|
and the sample |cdocsamp.tex| with include files
|cdocsch1.tex|, |cdocsch2.tex|, |cdocspt3.tex|, |cdocspt4.tex|,
|cdocsdrf.tex|, |cdocsfn1.tex|, |cdocsfn2.tex|.
Then copy the file |childdoc.def| to an appropriate directory of your \LaTeX{}
distribution, e.g.\ \textit{texmf-root}|/tex/latex/childdoc|.
\end{itemize}

%%%%%%%%%%%%%%%%%%%%%%%%%%%%%%%%%%%%%%%%%%%%%%%%%%%%%%%%%%%%%%%%%%%%%%%%%%%%%%%%
\subsection{Related CTAN Packages}

There are several other packages which offer a similar functionality:
%
\begin{itemize}
\item
The packages
\href{http://ctan.org/pkg/docmute}{\textsf{docmute}},
\href{http://ctan.org/pkg/includex}{\textsf{includex}} and
\href{http://ctan.org/pkg/standalone}{\textsf{standalone}}
provide commands to include only the document body of
a child file thus allowing both files to be compiled individually.
\item
The packages \href{http://ctan.org/pkg/subdocs}{\textsf{subdocs}}
and \href{http://ctan.org/pkg/subfiles}{\textsf{subfiles}}
provide structures in which the main and child documents can be
encapsulated and allowing them to be compiled individually.
The inclusion mechanism is different from the conventional |\include|.
\item
The package \href{http://ctan.org/pkg/combine}{\textsf{combine}}
is an elaborate solution to combine several documents into one.
\end{itemize}
%
See also the CTAN topic \href{http://ctan.org/topic/subdocs}{\textsf{subdocs}}
for further related packages.
The present package differs from the above solutions in that
a document structure constructed with the conventional |\include| mechanism
just needs two extra commands at the top of every file
such that all constituent files can be compiled individually.

%%%%%%%%%%%%%%%%%%%%%%%%%%%%%%%%%%%%%%%%%%%%%%%%%%%%%%%%%%%%%%%%%%%%%%%%%%%%%%%%
%\subsection{Feature Suggestions}
%
%The following is a list of features which may be useful for future
%versions of this package:
%%
%\begin{itemize}
%\item
%\ldots
%\end{itemize}

%%%%%%%%%%%%%%%%%%%%%%%%%%%%%%%%%%%%%%%%%%%%%%%%%%%%%%%%%%%%%%%%%%%%%%%%%%%%%%%%
\subsection{Revision History}

%%%%%%%%%%%%%%%%%%%%%%%%%%%%%%%%%%%%%%%%
\paragraph{v2.0:} 2018/12/30

\begin{itemize}
\item
immediate forward processing
\item
added |\childdocby| mechanism
\item
manual restructured
\end{itemize}

%%%%%%%%%%%%%%%%%%%%%%%%%%%%%%%%%%%%%%%%
\paragraph{v1.6:} 2018/01/17

\begin{itemize}
\item
application for development of include files
\item
corrections to manual
\end{itemize}

%%%%%%%%%%%%%%%%%%%%%%%%%%%%%%%%%%%%%%%%
\paragraph{v1.5:} 2017/05/21

\begin{itemize}
\item
more complete structuring introduced
\item
|\childdocof| introduced
\item
|\childdoc| renamed to |\childdocmain|
\item
|\childredirect| renamed to |\childdocforward| and |\childdocforwardprefix|
and functionality expanded
\end{itemize}

%%%%%%%%%%%%%%%%%%%%%%%%%%%%%%%%%%%%%%%%
\paragraph{v1.0:} 2017/04/27

\begin{itemize}
\item
manual and install package
\item
first version published on CTAN
\end{itemize}

%%%%%%%%%%%%%%%%%%%%%%%%%%%%%%%%%%%%%%%%
\paragraph{v0.6:} 2017/04/26

\begin{itemize}
\item
redirection mechanism added
\end{itemize}

%%%%%%%%%%%%%%%%%%%%%%%%%%%%%%%%%%%%%%%%
\paragraph{v0.5:} 2017/04/26

\begin{itemize}
\item
functionality in definition file
\end{itemize}


%%%%%%%%%%%%%%%%%%%%%%%%%%%%%%%%%%%%%%%%%%%%%%%%%%%%%%%%%%%%%%%%%%%%%%%%%%%%%%%%
%%%%%%%%%%%%%%%%%%%%%%%%%%%%%%%%%%%%%%%%%%%%%%%%%%%%%%%%%%%%%%%%%%%%%%%%%%%%%%%%
%%%%%%%%%%%%%%%%%%%%%%%%%%%%%%%%%%%%%%%%%%%%%%%%%%%%%%%%%%%%%%%%%%%%%%%%%%%%%%%%
\appendix

\settowidth\MacroIndent{\rmfamily\scriptsize 000\ }

 \DocInput{childdoc.dtx}

\end{document}
%</driver>
% \fi
%
% %%%%%%%%%%%%%%%%%%%%%%%%%%%%%%%%%%%%%%%%%%%%%%%%%%%%%%%%%%%%%%%%%%%%%%%%%%%%%%
% %%%%%%%%%%%%%%%%%%%%%%%%%%%%%%%%%%%%%%%%%%%%%%%%%%%%%%%%%%%%%%%%%%%%%%%%%%%%%%
% \section{Sample}
%\iffalse
%<*samplemain>
%\fi
%
% The following presents a sample document
% with two chapters, two parts, a title page,
% a compile flag as well as three forwarding files to set the flag.
% It consists of eight |.tex| files:
% \begin{center}
% \begin{tabular}{ll}
% |cdocsamp.tex|&main file\\
% |cdocsch1.tex|&include file for chapter 1\\
% |cdocsch2.tex|&include file for chapter 2\\
% |cdocspt3.tex|&include file for part 3\\
% |cdocspt4.tex|&include file for part 4\\
% |cdocsdrf.tex|&forwarding file for main file in draft mode\\
% |cdocsfi1.tex|&forwarding file for final version of chapter 1\\
% |cdocsfi2.tex|&forwarding file for final version of chapter 2\\
% \end{tabular}
% \end{center}
% Each of the eight files can be compiled directly by the \LaTeX{} compiler.
%
% %%%%%%%%%%%%%%%%%%%%%%%%%%%%%%%%%%%%%%
% \paragraph{Main File.}
%
% The main file is called |cdocsamp.tex|.
%
% Load the \textsf{childdoc} definitions and
% declare the filename for the main document:
%    \begin{macrocode}
\input{childdoc.def}
\childdocmain{}
%    \end{macrocode}

% Optional override for |\version| flag:
%    \begin{macrocode}
%%\ifchilddoc\else\providecommand{\version}{draft}\fi
%    \end{macrocode}

% Define the default values for the |\version| flag
% (|final| for the main file and |draft| for childs):
%    \begin{macrocode}
\ifchilddoc
\providecommand{\version}{draft}
\else
\providecommand{\version}{final}
\fi
%    \end{macrocode}

% Load the standard document class:
%    \begin{macrocode}
\documentclass[12pt]{article}
%    \end{macrocode}

% Start the document body:
%    \begin{macrocode}
\begin{document}
%    \end{macrocode}

% Declare a title page.
% Print title, part of document being processed and version flag:
%    \begin{macrocode}
\addtocounter{page}{-1}
\begin{center}
{\LARGE\bfseries{}childdoc example\par}
\vspace{1cm}
\ifchilddoc
\ifchilddocmanual part\else chapter\fi:
`\childdocname' of `\childdocjob'\par
\else
main document: `\childdocjob'\par
\fi
version: \version\par
\end{center}
\newpage
%    \end{macrocode}

% Manually include selected file,
% otherwise process as usual:
%    \begin{macrocode}
\ifchilddocmanual
\section*{part `\childdocname'}
\input{\childdocname}
\else
%    \end{macrocode}

% Include the two chapters:
%    \begin{macrocode}
\include{cdocsch1}
\include{cdocsch2}
%    \end{macrocode}

% Include the two parts unless only chapters should be displayed:
%    \begin{macrocode}
\ifchilddoc\else
\section{part three}
\input{cdocspt3}
\section{part four}
\input{cdocspt4}
\fi
%    \end{macrocode}

% Process as usual until here:
%    \begin{macrocode}
\fi
%    \end{macrocode}

% End of document body:
%    \begin{macrocode}
\end{document}
%    \end{macrocode}
%\iffalse
%</samplemain>
%\fi
%
% %%%%%%%%%%%%%%%%%%%%%%%%%%%%%%%%%%%%%%
% \paragraph{Chapter Include Files.}
%
% The include files are called |cdocsch1.tex| and |cdocsch2.tex|.
%
%\iffalse
%<*samplechap1|samplechap2>
%\fi

% Optional override for |\version| flag:
%    \begin{macrocode}
%%\providecommand{\version}{final}
%    \end{macrocode}

% Include the main document:
%    \begin{macrocode}
\input{childdoc.def}
\childdocof{cdocsamp}
%    \end{macrocode}

%\iffalse
%</samplechap1|samplechap2>
%\fi
%
%\iffalse
%<*samplechap1>
%\fi
% Some text for chapter 1:
%    \begin{macrocode}
\section{one}
some text in chapter one
%    \end{macrocode}

%\iffalse
%</samplechap1>
%\fi
% Some text for chapter 2:
%\iffalse
%<*samplechap2>
%\fi
%    \begin{macrocode}
\section{two}
more text in chapter two
%    \end{macrocode}

%\iffalse
%</samplechap2>
%\fi
%
% %%%%%%%%%%%%%%%%%%%%%%%%%%%%%%%%%%%%%%
% \paragraph{Part Include Files.}
%
% The include files are called |cdocspt3.tex| and |cdocspt4.tex|.
%
%\iffalse
%<*samplepart3|samplepart4>
%\fi

% Optional override for |\version| flag:
%    \begin{macrocode}
%%\providecommand{\version}{final}
%    \end{macrocode}

% Include the main document:
%    \begin{macrocode}
\input{childdoc.def}
\childdocby{cdocsamp}
%    \end{macrocode}

%\iffalse
%</samplepart3|samplepart4>
%\fi
%
%\iffalse
%<*samplepart3>
%\fi
% Some text for part 3:
%    \begin{macrocode}
some text in part three
%    \end{macrocode}

%\iffalse
%</samplepart3>
%\fi
% Some text for part 4:
%\iffalse
%<*samplepart4>
%\fi
%    \begin{macrocode}
more text in part four
%    \end{macrocode}

%\iffalse
%</samplepart4>
%\fi
%
% %%%%%%%%%%%%%%%%%%%%%%%%%%%%%%%%%%%%%%
% \paragraph{Forwarding for a Complete Draft.}
%
% The following forwarding file |cdocsdrf.tex|
% compiles the main document in draft mode:
%\iffalse
%<*sampledraft>
%\fi
%    \begin{macrocode}
\def\version{draft}
\input{childdoc.def}
\childdocforward{cdocsamp}
%    \end{macrocode}

%\iffalse
%</sampledraft>
%\fi
%
% %%%%%%%%%%%%%%%%%%%%%%%%%%%%%%%%%%%%%%
% \paragraph{Forwarding for Final Version of the Chapters.}
%
% The following forwarding files |cdocsfn1.tex| and |cdocsfn2.tex|
% (with identical content)
% compile the final versions of the child documents
% |cdocsch1.tex| and |cdocsch2.tex|, respectively:
%\iffalse
%<*samplefinal>
%\fi
%    \begin{macrocode}
\def\version{final}
\input{childdoc.def}
\childdocforwardprefix[cdocsamp]{cdocsfn}{cdocsch}
%    \end{macrocode}

%\iffalse
%</samplefinal>
%\fi
%
% %%%%%%%%%%%%%%%%%%%%%%%%%%%%%%%%%%%%%%
% \paragraph{Command Line Processing.}
%
% The following three command lines generate the output files
% |cdocscld|, |cdocscl1| and |cdocscl2|
% which should be identical to
% |cdocsdrf|, |cdocsch1| and |cdocsfn2|, respectively:
% \begin{center}
% \begin{tabular}{l}
% |latex -jobname cdocscld \|\\
% |  "\def\version{draft}\input{childdoc.def}\childdocforward{cdocsamp}"|\\
% |latex -jobname cdocscl1 \|\\
% |  "\input{childdoc.def}\childdocforward[cdocsamp]{cdocsch1}"|\\
% |latex -jobname cdocscl2 \|\\
% |  "\def\version{final}\input{childdoc.def}\childdocforward{cdocsch2}"|
% \end{tabular}
% \end{center}
% Note that the trailing backslash on each first line
% merely continues the input to the second line
% (for convenient cut ant paste).
% Furthermore, the command |latex| can be replaced by any
% of its alternative versions such as |pdflatex|.
%
% %%%%%%%%%%%%%%%%%%%%%%%%%%%%%%%%%%%%%%%%%%%%%%%%%%%%%%%%%%%%%%%%%%%%%%%%%%%%%%
% %%%%%%%%%%%%%%%%%%%%%%%%%%%%%%%%%%%%%%%%%%%%%%%%%%%%%%%%%%%%%%%%%%%%%%%%%%%%%%
% \section{Implementation}
%\iffalse
%<*package>
%\fi
%
% This section describes the definitions file |childdoc.def|.

% The definitions cannot be loaded using |\usepackage| or |\RequirePackage|
% which has a mechanism to prevent loading a style file more than once.
% When loading the definitions by means of |\input|
% multiple instances have to be prevented manually:
%\iffalse
%This code needs to be before the `\ProvidesFile' directive
%which is defined at the beginning of this file.
%Therefore it is also placed there and commented out here.
%</package>
%<*discard>
%\fi
%    \begin{macrocode}
\ifdefined\childdocmain\endinput\fi
%    \end{macrocode}
%\iffalse
%</discard>
%<*package>
%\fi
%
% \macro{\ifchilddoc}
% \macro{\ifchilddocmanual}
% The conditional |\ifchilddoc| tells whether a
% child (true) or main (false) document is being compiled.
% The conditional |\ifchilddocmanual| tells whether
% the |\includeonly| mechanism is used (false) or
% the selection of child files must be performed manually (true).
% The definitions initialise to false:
%    \begin{macrocode}
\newif\ifchilddoc
\newif\ifchilddocmanual
%    \end{macrocode}

% \macro{\childdocname}
% \macro{\childdocjob}
% The macro |\childdocname| stores the name of the main document
% to be compiled. The macro |\childdocjob| stores the name of
% the document on which the \LaTeX{} compiler was originally invoked.
% The content of |\jobname| cannot be compared
% to filenames specified in the source due to different catcodes.
% The following code rescans |\jobname|, stores the result
% in |\childdocname| and saves a copy in |\childdocjob|:
%    \begin{macrocode}
\edef\childdocname{\scantokens\expandafter{\jobname\noexpand}}
\let\childdocjob\childdocname
%    \end{macrocode}

% \macro{\childdocdisable}
% The macro |\childdocdisable| prevents the main file
% from being processed more than once.
% At this stage, the main document command |\childdocmain|
% is assumed to be called once again where it should do nothing.
% Any subsequent call to it should prevent
% a secondary processing of the main document
% It overwrites the forwarding commands
% |\childdocof| and |\childdocforward|
% with empty macros to prevent further inclusions of the main document:
%    \begin{macrocode}
\newcommand{\childdocdisable}
{
  \renewcommand{\childdocmain}[1]{\renewcommand{\childdocmain}[1]{\endinput}}
  \renewcommand{\childdocof}[1]{}
  \renewcommand{\childdocby}[2][]{}
  \renewcommand{\childdocforward}[2][]{}
  \renewcommand{\childdocdisable}{}
}
%    \end{macrocode}

% \macro{\childdocmain}
% The macro |\childdocmain| is to be called at the top of the main file
% with nothing or the main filename (without extension) as argument.
% First, it breaks loops.
% If the argument is not empty and does not match |\childdocname|
% (which is set by the first inclusion of |childdoc.def|),
% |\ifchilddoc| is set to true, |\includeonly| is applied to the child file
% and |\jobname| is set to the main file
% (for proper handling of |.aux| files):
%    \begin{macrocode}
\newcommand{\childdocmain}[1]
{
  \childdocdisable\childdocmain{}
  \if?#1?\else
    \begingroup
      \def\childdoctmp{#1}
      \ifx\childdoctmp\childdocname
        \def\childdoctmp{}
      \else
        \def\childdoctmp
        {
          \childdoctrue
          \includeonly{\childdocname}
          \def\childdocjob{#1}
          \def\jobname{#1}
        }
      \fi
      \expandafter
    \endgroup
    \childdoctmp
  \fi
}
%    \end{macrocode}

% \macro{\childdocof}
% The command |\childdocof| redirects
% compilation to the main file |#1|.
%    \begin{macrocode}
\newcommand{\childdocof}[1]
{
  \childdocdisable
  \childdoctrue
  \includeonly{\childdocname}
  \def\jobname{#1}
  \def\childdocjob{#1}
  \input{#1}
}
%    \end{macrocode}

% \macro{\childdocby}
% The command |\childdocby| ....
%    \begin{macrocode}
\newcommand{\childdocby}[2][]
{
  \childdocdisable
  \childdoctrue
  \childdocmanualtrue
  \if?#1?\else
    \def\jobname{#2}
  \fi
  \def\childdocjob{#2}
  \input{#2}
  \endinput
}
%    \end{macrocode}

% \macro{\childdocforward}
% The command |\childdocforward| redirects
% compilation to the main file or
% (if the optional argument is given) a child file.
% Parameters are set as if the main file
% or a child file starting with |\childdocof| was compiled.
% Then compilation is handed over to the main file:
%    \begin{macrocode}
\newcommand{\childdocforward}[2][]
{
  \begingroup
    \if?#1?
      \def\childdoctmp
      {
        \def\childdocname{#2}
        \def\childdocjob{#2}
        \def\jobname{#2}
        \input{#2}
        \endinput
      }
    \else
      \def\childdoctmp
      {
        \childdocdisable
        \def\childdocname{#2}
        \childdoctrue
        \includeonly{#2}
        \def\childdocjob{#1}
        \def\jobname{#1}
        \input{#1}
        \endinput
      }
    \fi
    \expandafter
  \endgroup
  \childdoctmp
}
%    \end{macrocode}

% \macro{\childdocforwardprefix}
% The command |\childdocforwardprefix| redirects
% compilation to the main or a child file by means of a pattern.
% The prefix |#1| in the current filename is replaced by |#2|
% and the suffix of the current filename is kept
% (it is assumed that the filename does not contain the substring `|~~~|'
% which is used as a delimiter).
% Compilation is handed over to the new file by |\childdocforward|:
%    \begin{macrocode}
\newcommand{\childdocforwardprefix}[3][]
{
  \begingroup
    \def\childdocextract #2##1~~~{\def\childdoctmp{\childdocforward[#1]{#3##1}}}
    \expandafter\childdocextract\childdocname~~~
    \expandafter
  \endgroup
  \childdoctmp
}
%    \end{macrocode}

% \macro{\childdoc}
% The deprecated macro |\childdoc| is a legacy version of |\childdocmain|:
%    \begin{macrocode}
\newcommand{\childdoc}{\childdocmain}
%    \end{macrocode}

% \macro{\childdocredirect}
% The deprecated macro |\childdocredirect| is a legacy version
% of |\childdocforward| and |\childdocforwardprefix|:
%    \begin{macrocode}
\newcommand{\childdocredirect}[2][]
{
  \begingroup
    \if?#1?
      \def\childdoctmp{\childdocforward{#2}}
    \else
      \def\childdoctmp{\childdocforwardprefix{#1}{#2}}
    \fi
    \expandafter
  \endgroup
  \childdoctmp
}
%    \end{macrocode}

%\iffalse
%</package>
%\fi
%
\endinput

\childdocforwardprefix[cdocsamp]{cdocsfn}{cdocsch}
%    \end{macrocode}

%\iffalse
%</samplefinal>
%\fi
%
% %%%%%%%%%%%%%%%%%%%%%%%%%%%%%%%%%%%%%%
% \paragraph{Command Line Processing.}
%
% The following three command lines generate the output files
% |cdocscld|, |cdocscl1| and |cdocscl2|
% which should be identical to
% |cdocsdrf|, |cdocsch1| and |cdocsfn2|, respectively:
% \begin{center}
% \begin{tabular}{l}
% |latex -jobname cdocscld \|\\
% |  "\def\version{draft}% \iffalse
%
% childdoc.dtx Copyright (C) 2017-2018 Niklas Beisert
%
% This work may be distributed and/or modified under the
% conditions of the LaTeX Project Public License, either version 1.3
% of this license or (at your option) any later version.
% The latest version of this license is in
%   http://www.latex-project.org/lppl.txt
% and version 1.3 or later is part of all distributions of LaTeX
% version 2005/12/01 or later.
%
% This work has the LPPL maintenance status `maintained'.
%
% The Current Maintainer of this work is Niklas Beisert.
%
% This work consists of the files childdoc.dtx and childdoc.ins
% and the derived files childdoc.def and cdocsamp.tex with
% cdocsch1.tex, cdocsch2.tex, cdocsdrf.tex, cdocsfn1.tex, cdocsfn2.tex.
%
%<package>\ifdefined\childdocmain\endinput\fi
%<package>\ProvidesFile{childdoc.def}[2018/12/30 v2.0 child document driver]
%<samplemain>\ProvidesFile{cdocsamp.tex}[2018/12/30 v2.0 sample for childdoc]
%<*driver>
%\ProvidesFile{childdoc.drv}[2018/12/30 v2.0 childdoc reference manual file]
\PassOptionsToClass{10pt,a4paper}{article}
\documentclass{ltxdoc}

\usepackage[margin=35mm]{geometry}
\usepackage{hyperref}
\usepackage{hyperxmp}
\usepackage[usenames]{color}

\hypersetup{colorlinks=true}
\hypersetup{pdfstartview=FitH}
\hypersetup{pdfpagemode=UseNone}
\hypersetup{pdfsource={}}
\hypersetup{pdflang={en-UK}}
\hypersetup{pdfcopyright={Copyright 2017-2018 Niklas Beisert.
  This work may be distributed and/or modified under the
  conditions of the LaTeX Project Public License, either version 1.3
  of this license or (at your option) any later version.}}
\hypersetup{pdflicenseurl={http://www.latex-project.org/lppl.txt}}
\hypersetup{pdfcontactaddress={ETH Zurich, ITP, HIT K,
  Wolfgang-Pauli-Strasse 27}}
\hypersetup{pdfcontactpostcode={8093}}
\hypersetup{pdfcontactcity={Zurich}}
\hypersetup{pdfcontactcountry={Switzerland}}
\hypersetup{pdfcontactemail={nbeisert@itp.phys.ethz.ch}}
\hypersetup{pdfcontacturl={http://people.phys.ethz.ch/\xmptilde nbeisert/}}

\newcommand{\secref}[1]{\hyperref[#1]{section \ref*{#1}}}

\parskip1ex
\parindent0pt
\let\olditemize\itemize
\def\itemize{\olditemize\parskip0pt}

\begin{document}

\title{The \textsf{childdoc} Package}
\hypersetup{pdftitle={The childdoc Package}}
\author{Niklas Beisert\\[2ex]
  Institut f\"ur Theoretische Physik\\
  Eidgen\"ossische Technische Hochschule Z\"urich\\
  Wolfgang-Pauli-Strasse 27, 8093 Z\"urich, Switzerland\\[1ex]
  \href{mailto:nbeisert@itp.phys.ethz.ch}
  {\texttt{nbeisert@itp.phys.ethz.ch}}}
\hypersetup{pdfauthor={Niklas Beisert}}
\hypersetup{pdfsubject={Manual for the LaTeX2e Package childdoc}}
\date{30 December 2018, \textsf{v2.0}}
\maketitle

\begin{abstract}\noindent
\textsf{childdoc} is a \LaTeXe{} package
that enables the direct compilation
of document sections included by |\include|
to individual files.
\end{abstract}

\begingroup
\parskip0ex
\tableofcontents
\endgroup

%%%%%%%%%%%%%%%%%%%%%%%%%%%%%%%%%%%%%%%%%%%%%%%%%%%%%%%%%%%%%%%%%%%%%%%%%%%%%%%%
%%%%%%%%%%%%%%%%%%%%%%%%%%%%%%%%%%%%%%%%%%%%%%%%%%%%%%%%%%%%%%%%%%%%%%%%%%%%%%%%
\section{Introduction}

\LaTeX{} provides a mechanism to structure a large document (such as a book)
into a main file and several child files (containing the chapters)
using the |\include| command.
This mechanism is beneficial for documents
which span hundreds of pages in order to
make the source file(s) more manageable.
Moreover, compilation can be restricted to
selected child files by means of the |\includeonly| command.
The latter feature can be used to reduce the compilation time while editing
(this was significantly more useful in the earlier days of \LaTeX{})
or to generate a smaller document which is easier to navigate.
Another application of |\includeonly| is to generate
documents consisting of selected parts of the complete document.

However, there are a few drawbacks of the plain |\include| mechanism:
\begin{itemize}
\item
The child files cannot be compiled on their own,
they can only be compiled via the main file.
A naive editing environment
(such as a text editor with an option
to have the current file processed by \LaTeX)
may require one to switch to the main file before compiling;
attempting to compile the child file produces errors.
\item
The main file must be modified (each time)
to adjust the |\includeonly| command
to the present needs. This easily leaves the main file in a messy state.
\item
The generated document will always carry the filename
of the main document. This is inconvenient if
several child files are to be compiled and
to be kept for distribution.
\end{itemize}

The present package provides a simple interface
to make child files individually compilable by \LaTeX{}.
Compiling a child file then has the same effect as compiling
the main file with an |\includeonly| command
to select the appropriate child.
Moreover the generated document will carry the name of the child
rather than the main file.
This resolves all three above issues.

This feature is meant to make the editing of books,
thesis documents and lecture notes somewhat more convenient.
However, the package can also be used efficiently for
composing a series of documents (such as exercise sheets)
which are typically distributed individually.
It then assists the author in generating the individual documents
(potentially in different versions)
as well as a document containing the collected series.
Another application is in developing style files
or other kinds of included material
where compilation of the style file could redirect
to a sample or test file.

%%%%%%%%%%%%%%%%%%%%%%%%%%%%%%%%%%%%%%%%%%%%%%%%%%%%%%%%%%%%%%%%%%%%%%%%%%%%%%%%
%%%%%%%%%%%%%%%%%%%%%%%%%%%%%%%%%%%%%%%%%%%%%%%%%%%%%%%%%%%%%%%%%%%%%%%%%%%%%%%%
\section{Usage}

First of all, the package \textsf{childdoc} is \emph{not} a standard
\LaTeXe{} |.sty| style file! Therefore it needs to be invoked in
a non-standard way.

%%%%%%%%%%%%%%%%%%%%%%%%%%%%%%%%%%%%%%%%%%%%%%%%%%%%%%%%%%%%%%%%%%%%%%%%%%%%%%%%
\subsection{Included Files}
\label{sec:include}

%%%%%%%%%%%%%%%%%%%%%%%%%%%%%%%%%%%%%%%%
\DescribeMacro{\childdocmain}
To use the package, add the commands
\begin{center}
\begin{tabular}{l}
|\input{childdoc.def}|\\
|\childdocmain{}|\\
\end{tabular}
\end{center}
at the very top of the main \LaTeX{} file,
in particular \emph{before} the |\documentclass| statement!
The argument of |\childdocmain| should be left empty
(but it must be present).

%%%%%%%%%%%%%%%%%%%%%%%%%%%%%%%%%%%%%%%%
\DescribeMacro{\childdocof}
Furthermore, add the commands
\begin{center}
\begin{tabular}{l}
|\input{childdoc.def}|\\
|\childdocof{|\textit{main}|}|\\
\end{tabular}
\end{center}
at the top of every child file \textit{child}
which is included by |\include{|\textit{child}|}|
from within the main file
(or at least for those files to be compiled individually).
The argument \textit{main} must be the filename of the main file.

There are a couple of
considerations in setting up the main and child documents:

%%%%%%%%%%%%%%%%%%%%%%%%%%%%%%%%%%%%%%%%
\paragraph{Restrictions.}

Please note the following restrictions:
\begin{itemize}
\item
|\childdocmain| must be called with one argument \textit{main}
to ensure compatibility with earlier version of the package.
It must either be empty (|\childdocmain{}|)
or precisely match the filename of the main file in which it is specified.
See \secref{sec:detection} for further information.
\item
The filename \textit{main} must be specified without the |.tex| extension.
\item
The filename \textit{main} is case sensitive
(even in case-insensitive file systems)
due to internal string comparison.
\item
The argument \textit{main} should be fully expanded, it cannot be a macro.
\item
Subdirectories and special characters should be avoided in filenames.
\item
The command |\childdocmain{|\textit{main}|}| must be followed by a whitespace.
It should not be followed immediately by another command
or by a comment mark `|%|'.
This is because the \TeX{} parser reads the token immediately following
the argument of |\childdocmain| and puts it
at the beginning of every child section;
however, a white\-space is ignored.
\end{itemize}

%%%%%%%%%%%%%%%%%%%%%%%%%%%%%%%%%%%%%%%%
\paragraph{Content of Main File.}

It is advisable to place all content in the child files included by |\include|.
Any output contained in the main file will appear in all child documents
unless suppressed manually;
it cannot be suppressed automatically by the |\includeonly| directive
and thus should normally be avoided.
A method to include some content in the main file
by means of conditional processing is described in \secref{sec:conditional}.

%%%%%%%%%%%%%%%%%%%%%%%%%%%%%%%%%%%%%%%%
\paragraph{Page Numbering.}

When only a part of the document is compiled,
the appropriate numbering of pages
(as well as other status parameters)
is determined from the |.aux| files.
The latter contain information from previous passes.
However this information needs to propagate through
all intermediate child documents.
Therefore the page numbering in child documents may well
be inconsistent until the complete document is compiled at least once.

A useful (if unconventional) way to always ensure a consistent
page numbering is to restart the numbering in each child document
and denote the pages by `\textit{child}|.|\textit{page}'
where \textit{child} represents the chapter/section number of the child file.
This can be achieved by the command
|\numberwithin{page}{|\textit{child}|}|
of the \textsf{amsmath} package
where \textit{child} can be |chapter| or |section|
depending on the chosen structuring.
Alternatively, one can modify the macro |\thepage| appropriately
and reset the counter |page| at the start of each child file.

%%%%%%%%%%%%%%%%%%%%%%%%%%%%%%%%%%%%%%%%%%%%%%%%%%%%%%%%%%%%%%%%%%%%%%%%%%%%%%%%
\subsection{Conditional Processing}
\label{sec:conditional}

The package provides a mechanism to compile different versions
of a document. To customise the versions further some conditional processing
can come in handy to distinguish which version is being compiled.
The package provides two macros to describe the compilation context:

%%%%%%%%%%%%%%%%%%%%%%%%%%%%%%%%%%%%%%%%
\DescribeMacro{\ifchilddoc}
The conditional |\ifchilddoc| distinguishes between the compilation of
child documents and the main document:
%
\begin{center}
|\ifchilddoc |\textit{child-code}| |[|\||else |\textit{main-code}]| \||fi|
\end{center}

%%%%%%%%%%%%%%%%%%%%%%%%%%%%%%%%%%%%%%%%
\DescribeMacro{\childdocname}
\DescribeMacro{\childdocjob}
The macro |\childdocname| contains the filename (without extension)
of the main or child file being processed.
Note that |\childdocjob| will always contain the name of the main file.

%%%%%%%%%%%%%%%%%%%%%%%%%%%%%%%%%%%%%%%%
\paragraph{Title Page.}

Conditional processing can be used to include a title or banner page
in the main document when proper precautions are taken.
Importantly, the code in the main file should ensure that the page counter
(as well as other status parameters which are stored in the |.aux| files)
takes the same value after the conditional processing.
Otherwise the page numbers may take divergent values
depending on which part is compiled.

For example, a title page could be declared by:
%
\begin{center}
\begin{tabular}{l}
|\ifchilddoc\||else|\\
|\addtocounter{page}{-1}|\\
\textit{code for title page}\\
|\newpage|\\
|\||fi|
\end{tabular}
\end{center}
%
A banner page for the child documents can be generated by:
%
\begin{center}
\begin{tabular}{l}
|\ifchilddoc|\\
|\addtocounter{page}{-1}|\\
\textit{code for banner page}\\
|\newpage|\\
|\||fi|
\end{tabular}
\end{center}
%
Here one could write a message such as:
\begin{center}
|This is the part \childdocname{} of \childdocjob{}.|
\end{center}

%%%%%%%%%%%%%%%%%%%%%%%%%%%%%%%%%%%%%%%%%%%%%%%%%%%%%%%%%%%%%%%%%%%%%%%%%%%%%%%%
\subsection{Flags}
\label{sec:flags}

The package makes it easy to generate different versions
of the main or child documents.
To this end compilation flags can be defined
and assigned different default values.
They will be particularly useful in conjunction
with the forwarding mechanism described in \secref{sec:forward}.

For example, it may be useful to have a flag |\version|
which can be set to |draft| or |final|.
The document source will contain some conditional code
depending on the value of |\version|.
Suppose further, the flag should default to |final| for the main file
and to |draft| for child files
which is a natural assignment for editing the document.
This is achieved by placing the following code
in the preamble of the main document
(below the |\childdocmain| directive):
%
\begin{center}
\begin{tabular}{l}
|\ifchilddoc|\\
|\providecommand{\version}{draft}|\\
|\||else|\\
|\providecommand{\version}{final}|\\
|\||fi|
\end{tabular}
\end{center}
%
The definition by |\providecommand| makes sure
that previous definitions are not overwritten.
Further statements |\providecommand{\version}{...}|
can thus be added before the above code to override it.

For the main file, one might add a line
(between |\childdocmain| and the above block)
%
\begin{center}
|%\ifchilddoc\||else\providecommand{\version}{draft}\||fi|
\end{center}
%
which can be uncommented to produce a draft version.
Likewise one can add a line to the very top of a child file
(above the |\childdocof{|\textit{main}|}| directive)
%
\begin{center}
|%\providecommand{\version}{final}|
\end{center}
%
which can be uncommented to produce the final version of this child document.

%%%%%%%%%%%%%%%%%%%%%%%%%%%%%%%%%%%%%%%%%%%%%%%%%%%%%%%%%%%%%%%%%%%%%%%%%%%%%%%%
\subsection{Forwarding}
\label{sec:forward}

Different versions of the main or child documents
using compilation flags as described in \secref{sec:flags}
can be (permanently) stored in different files
for convenient compilation, viewing and distribution.
To this end, the package defines a command
to pass on compilation to a different file:

%%%%%%%%%%%%%%%%%%%%%%%%%%%%%%%%%%%%%%%%
\DescribeMacro{\childdocforward}
The command |\childdocforward| redirects processing to
another source file:
%
\begin{center}
\begin{tabular}{l}
|\input{childdoc.def}|\\
|\childdocforward[|\textit{main}|]{|\textit{dest}|}|\\
\end{tabular}
\end{center}
%
The argument \textit{dest} is the destination file
(without extension).
It should be the main file or one of the child files.
Note that further \textsf{childdoc} directives
such as |\childdocof| and |\childdocforward|
in the indicated file will be processed in this form.
The optional argument \textit{main}
passes on directly to the main file \textit{main}
while pretending to compile the child \textit{dest}.
This form behaves as if \textit{dest}
issues |\childdocof{|\textit{main}|}| right away,
and no further \textsf{childdoc} directives will be processed.

%%%%%%%%%%%%%%%%%%%%%%%%%%%%%%%%%%%%%%%%
\DescribeMacro{\...prefix}
In the alternative form |\childdocforwardprefix|,
%
\begin{center}
\begin{tabular}{l}
|\input{childdoc.def}|\\
|\childdocforwardprefix[|\textit{main}|]{|\textit{prefix}|}{|\textit{dest}|}|
\end{tabular}
\end{center}
%
the destination file is determined by a pattern
depending on the current file:
To make this work, the current file must be called
`{\textit{prefix}\hspace{0.2em}\textit{suffix}}'
with \textit{prefix} matching precisely the argument.
Processing is then passed on to the file
`{\textit{dest}\hspace{0.2em}\textit{suffix}}'.
Surely, the same effect is achieved by
directly specifying the
argument `{\textit{dest}\hspace{0.2em}\textit{suffix}}'
in the first form.
However, that requires to set up a different file
for each child. With the alternative form of the command
all these files can have exactly the same content
which simplifies setting them up and maintaining them.

For example, the following file |draft.tex|
with a compilation flag |\version| as described in \secref{sec:flags}
compiles the main document as a draft:
%
\begin{center}
\begin{tabular}{l}
|\def\version{draft}|\\
|\input{childdoc.def}|\\
|\childdocforward{|\textit{main}|}|
\end{tabular}
\end{center}
%
Likewise, the following files |final|\textit{nn}|.tex|
compile the final version of the child document
|child|\textit{nn}|.tex|:
%
\begin{center}
\begin{tabular}{l}
|\def\version{final}|\\
|\input{childdoc.def}|\\
|\childdocforwardprefix{final}{child}|
\end{tabular}
\end{center}
%

Note that when several versions of a main file and/or of each child file
are to be generated, it may be convenient to set up a |Makefile| or
shell script to automatise the process.

%%%%%%%%%%%%%%%%%%%%%%%%%%%%%%%%%%%%%%%%%%%%%%%%%%%%%%%%%%%%%%%%%%%%%%%%%%%%%%%%
\subsection{Command Line Processing}
\label{sec:commandline}

The effect of redirection files can also be achieved by invoking
the \LaTeX{} compiler with a more elaborate command line.
Most conveniently this should be done as part
of a shell script or a |Makefile|.

When using \textsf{childdoc} in the main file, the following
command lines effectively perform a redirection
(note that depending on the shell being used,
backslashes may have to be doubled: `|\|' $\to$ `|\\|'):
%
\begin{center}
|... -jobname "|\textit{target}|" |\\|"|[\textit{flags}]%
|\input{childdoc.def}\childdocforward[|\textit{main}|]{|\textit{dest}|}"|
\end{center}
%
Here \textit{target} is the name of the output file,
\textit{main} is the name of the main file
and \textit{dest} is the name of the main or child file to be processed
(all filenames without extensions).
The optional argument \textit{main} can be omitted
if \textit{main} matches \textit{dest}.
Optionally, compilation \textit{flags} can be defined via |\def| commands.
This command line makes the \TeX{} engine believe
it is compiling the file \textit{target}
whose content is specified as the latter parameter.
The provided code then forwards the processing to
\textit{main} or \textit{dest} as described in \secref{sec:forward}.

%%%%%%%%%%%%%%%%%%%%%%%%%%%%%%%%%%%%%%%%%%%%%%%%%%%%%%%%%%%%%%%%%%%%%%%%%%%%%%%%
\subsection{Include by Input}
\label{sec:input}

Including child documents by |\include| has some restrictions by design.
Most notably, the content of a child document always occupies
its own set of pages; pages cannot be shared between child documents.
Usually, this behaviour makes perfect sense
because each child document contain an essential part of the document.
However, in some situations it may be desirable to compose
a document from a collection of parts
without having mandatory page breaks between then.
For this case, the package
provides a mechanism to include parts
by |\input| which can also be processed individually.
However, by construction this mechanism
requires manual handling of the content to be output.

%%%%%%%%%%%%%%%%%%%%%%%%%%%%%%%%%%%%%%%%
\DescribeMacro{\ifchilddocmanual}
The main file should be prepared as usual, see \secref{sec:include}.
However, the document body must make a distinction
between processing of an individual part and of the main document, e.g.:
%
\begin{center}
\begin{tabular}{l}
|\ifchilddocmanual|\\
|\input{\childdocname}|\\
|\||else|\\
\textit{document body with }|\input{|\textit{part}|}|\\
|\||fi|
\end{tabular}
\end{center}
%
The conditional |\ifchilddocmanual| is true whenever
a part to be included by |\input| is being compiled,
and the name of the part is stored in |\childdocname|.

%%%%%%%%%%%%%%%%%%%%%%%%%%%%%%%%%%%%%%%%
\DescribeMacro{\childdocby}
Each part to be included by |\input| should start with:
%
\begin{center}
\begin{tabular}{l}
|\input{childdoc.def}|\\
|\childdocby{|\textit{main}|}|\\
\end{tabular}
\end{center}
%
The directive |\childdocby| is similar to |\childdocof|
described in \secref{sec:include},
but the subsequent selection of content must be done manually.
To that end, both |\ifchilddoc| and |\ifchilddocmanual|
will be true upon processing of a part,
and the name of the part is stored in |\childdocname|.
Note that |\jobname| will be set to the filename of the current part
so that each part receives an individual |.aux| file
that does not interfere with the |.aux| file(s) of the main document.
This behaviour can be altered by the alternative form
|\childdocby[*]{|\textit{main}|}| (with a non-empty optional argument)
which uses the |.aux| file of the main document
by setting |\jobname| to \textit{main}.

%%%%%%%%%%%%%%%%%%%%%%%%%%%%%%%%%%%%%%%%%%%%%%%%%%%%%%%%%%%%%%%%%%%%%%%%%%%%%%%%
\subsection{Driver Development}
\label{sec:driver}

The \textsf{childdoc} mechanism can also be use for the development
of definition files such as \LaTeX{} styles or classes.
This case differs from the above setup with multiple parts
included by |\include| in that no |\includeonly| should be invoked.
This can be achieved by starting the include file
(before |\ProvidesPackage|) with:
%
\begin{center}
\begin{tabular}{l}
|\input{childdoc.def}|\\
|\childdocforward{|\textit{main}|}|\\
\end{tabular}
\end{center}
%
or alternatively with:
%
\begin{center}
\begin{tabular}{l}
|\input{childdoc.def}|\\
|\childdocby{|\textit{main}|}|\\
\end{tabular}
\end{center}
%
Both forms have slightly different effects as described above.
The main file is prepared as usual, see \secref{sec:include}.

%%%%%%%%%%%%%%%%%%%%%%%%%%%%%%%%%%%%%%%%%%%%%%%%%%%%%%%%%%%%%%%%%%%%%%%%%%%%%%%%
\subsection{Legacy Detection}
\label{sec:detection}

The directive |\childdocmain| in the main file can detect
whether the complete document or merely a child is to be compiled
even without using the directive |\childdocof|.
This method is deprecated because it is less robust
and there is no compelling reason to use it;
it is merely provided for backward compatibility
and it may be removed in future versions.

If the detection mechanism is to be used,
it is mandatory to correctly specify
the filename of the main file as the argument of |\childdocmain|:
%
\begin{center}
\begin{tabular}{l}
|\input{childdoc.def}|\\
|\childdocmain{|\textit{main}|}|\\
\end{tabular}
\end{center}
%
If |\jobname| does not match the argument \textit{main} of |\childdocmain|,
it is assumed that |\jobname| points to the child file to be compiled.
When using |\childdocmain| with the main file specified as argument,
it suffices to start a child file
with just |\input{|\textit{main}|}|
without loading of the package and using |\childdocof|.
If instead all processing is done
with the appropriate \textsf{childdoc} directives,
the argument of \textit{main} of |\childdocmain| can be empty.

An alternative version of the command line processing described
in \secref{sec:commandline} using the detection mechanism reads:
%
\begin{center}
|... -jobname "|\textit{target}|" "|[\textit{flags}]%
[|\def\jobname{|\textit{dest}|}|]|\input{|\textit{main}|}"|
\end{center}

%%%%%%%%%%%%%%%%%%%%%%%%%%%%%%%%%%%%%%%%%%%%%%%%%%%%%%%%%%%%%%%%%%%%%%%%%%%%%%%%
\subsection{Manual Code}
\label{sec:manual}

In case one cannot be certain whether the definitions file |childdoc.def|
is installed on the target \TeX{} distribution
and one prefers not to ship it,
it is conceivable to paste a few relevant commands into the sources.

To that end, drop all statements |\input{childdoc.def}|
and perform the replacements as outlined below.
Instead of |\childdocmain{|\textit{main}|}| add the following code
to the top of the main file:
%
\begin{center}
\begin{tabular}{l}
|\||ifdefined\childdocname\endinput\||fi\newif\ifchilddoc|\\
|\edef\childdocname{\scantokens\expandafter{\jobname\noexpand}}|\\
|\def\childdocmain{|\textit{main}|}\||ifx\childdocmain\childdocname\||else|\\
|\childdoctrue\includeonly{\childdocname}\let\jobname\childdocmain\||fi|\\
\end{tabular}
\end{center}
%
Instead of |\childdocof{|\textit{main}|}| just include the main file
at the top of each child file:
%
\begin{center}
|\input{|\textit{main}|}|
\end{center}
%
A simple redirection |\childdocforward{|\textit{dest}|}| is achieved by:
%
\begin{center}
|\def\jobname{|\textit{dest}|}\input{\jobname}|
\end{center}
%
The redirection with prefix
|\childdocforwardprefix[|\textit{prefix}|]{|\textit{dest}|}|
is accomplished by:
%
\begin{center}
\begin{tabular}{l}
|{\edef\jobname{\scantokens\expandafter{\jobname\noexpand}}|\\
|\def\redirectjob |\textit{prefix}|#1~~~{\gdef\jobname{|\textit{dest}|#1}}|\\
|\expandafter\redirectjob\jobname~~~}\input{\jobname}|
\end{tabular}
\end{center}

In an alternative approach,
child documents can be compiled by a specific command line
without additional code or specific definitions:
%
\begin{center}
|... -jobname "|\textit{target}|" "|[\textit{flags}]%
|\includeonly{|\textit{dest}|}\input{|\textit{main}|}"|
\end{center}
%

%%%%%%%%%%%%%%%%%%%%%%%%%%%%%%%%%%%%%%%%%%%%%%%%%%%%%%%%%%%%%%%%%%%%%%%%%%%%%%%%
%%%%%%%%%%%%%%%%%%%%%%%%%%%%%%%%%%%%%%%%%%%%%%%%%%%%%%%%%%%%%%%%%%%%%%%%%%%%%%%%
\section{Information}

%%%%%%%%%%%%%%%%%%%%%%%%%%%%%%%%%%%%%%%%%%%%%%%%%%%%%%%%%%%%%%%%%%%%%%%%%%%%%%%%
\subsection{Copyright}

Copyright \copyright{} 2017--2018 Niklas Beisert

This work may be distributed and/or modified under the
conditions of the \LaTeX{} Project Public License, either version 1.3
of this license or (at your option) any later version.
The latest version of this license is in
  \url{http://www.latex-project.org/lppl.txt}
and version 1.3 or later is part of all distributions of \LaTeX{}
version 2005/12/01 or later.

This work has the LPPL maintenance status `maintained'.

The Current Maintainer of this work is Niklas Beisert.

This work consists of the files |README.txt|, |childdoc.ins| and |childdoc.dtx|
as well as the derived files |childdoc.def|, |cdocsamp.tex|
with |cdocsch1.tex|, |cdocsch2.tex|, |cdocspt3.tex|, |cdocspt4.tex|,
|cdocsdrf.tex|, |cdocsfn1.tex|, |cdocsfn2.tex|
as well as |childdoc.pdf|.

%%%%%%%%%%%%%%%%%%%%%%%%%%%%%%%%%%%%%%%%%%%%%%%%%%%%%%%%%%%%%%%%%%%%%%%%%%%%%%%%
\subsection{Files and Installation}

The package consists of the files:
%
\begin{center}
\begin{tabular}{ll}
    |README.txt|   & readme file \\
    |childdoc.ins| & installation file \\
    |childdoc.dtx| & source file \\
    |childdoc.def| & definition file \\
    |cdocsamp.tex| & sample main file \\
    |cdocsch1.tex| & sample include file \\
    |cdocsch2.tex| & sample include file \\
    |cdocspt3.tex| & sample part file \\
    |cdocspt4.tex| & sample part file \\
    |cdocsdrf.tex| & sample redirection file \\
    |cdocsfn1.tex| & sample redirection file \\
    |cdocsfn2.tex| & sample redirection file \\
    |childdoc.pdf| & manual
\end{tabular}
\end{center}
%
The distribution consists of the files
|README.txt|, |childdoc.ins| and |childdoc.dtx|.
%
\begin{itemize}
\item
Run (pdf)\LaTeX{} on |childdoc.dtx|
to compile the manual |childdoc.pdf| (this file).
\item
Run \LaTeX{} on |childdoc.ins| to create the definitions file |childdoc.def|
and the sample |cdocsamp.tex| with include files
|cdocsch1.tex|, |cdocsch2.tex|, |cdocspt3.tex|, |cdocspt4.tex|,
|cdocsdrf.tex|, |cdocsfn1.tex|, |cdocsfn2.tex|.
Then copy the file |childdoc.def| to an appropriate directory of your \LaTeX{}
distribution, e.g.\ \textit{texmf-root}|/tex/latex/childdoc|.
\end{itemize}

%%%%%%%%%%%%%%%%%%%%%%%%%%%%%%%%%%%%%%%%%%%%%%%%%%%%%%%%%%%%%%%%%%%%%%%%%%%%%%%%
\subsection{Related CTAN Packages}

There are several other packages which offer a similar functionality:
%
\begin{itemize}
\item
The packages
\href{http://ctan.org/pkg/docmute}{\textsf{docmute}},
\href{http://ctan.org/pkg/includex}{\textsf{includex}} and
\href{http://ctan.org/pkg/standalone}{\textsf{standalone}}
provide commands to include only the document body of
a child file thus allowing both files to be compiled individually.
\item
The packages \href{http://ctan.org/pkg/subdocs}{\textsf{subdocs}}
and \href{http://ctan.org/pkg/subfiles}{\textsf{subfiles}}
provide structures in which the main and child documents can be
encapsulated and allowing them to be compiled individually.
The inclusion mechanism is different from the conventional |\include|.
\item
The package \href{http://ctan.org/pkg/combine}{\textsf{combine}}
is an elaborate solution to combine several documents into one.
\end{itemize}
%
See also the CTAN topic \href{http://ctan.org/topic/subdocs}{\textsf{subdocs}}
for further related packages.
The present package differs from the above solutions in that
a document structure constructed with the conventional |\include| mechanism
just needs two extra commands at the top of every file
such that all constituent files can be compiled individually.

%%%%%%%%%%%%%%%%%%%%%%%%%%%%%%%%%%%%%%%%%%%%%%%%%%%%%%%%%%%%%%%%%%%%%%%%%%%%%%%%
%\subsection{Feature Suggestions}
%
%The following is a list of features which may be useful for future
%versions of this package:
%%
%\begin{itemize}
%\item
%\ldots
%\end{itemize}

%%%%%%%%%%%%%%%%%%%%%%%%%%%%%%%%%%%%%%%%%%%%%%%%%%%%%%%%%%%%%%%%%%%%%%%%%%%%%%%%
\subsection{Revision History}

%%%%%%%%%%%%%%%%%%%%%%%%%%%%%%%%%%%%%%%%
\paragraph{v2.0:} 2018/12/30

\begin{itemize}
\item
immediate forward processing
\item
added |\childdocby| mechanism
\item
manual restructured
\end{itemize}

%%%%%%%%%%%%%%%%%%%%%%%%%%%%%%%%%%%%%%%%
\paragraph{v1.6:} 2018/01/17

\begin{itemize}
\item
application for development of include files
\item
corrections to manual
\end{itemize}

%%%%%%%%%%%%%%%%%%%%%%%%%%%%%%%%%%%%%%%%
\paragraph{v1.5:} 2017/05/21

\begin{itemize}
\item
more complete structuring introduced
\item
|\childdocof| introduced
\item
|\childdoc| renamed to |\childdocmain|
\item
|\childredirect| renamed to |\childdocforward| and |\childdocforwardprefix|
and functionality expanded
\end{itemize}

%%%%%%%%%%%%%%%%%%%%%%%%%%%%%%%%%%%%%%%%
\paragraph{v1.0:} 2017/04/27

\begin{itemize}
\item
manual and install package
\item
first version published on CTAN
\end{itemize}

%%%%%%%%%%%%%%%%%%%%%%%%%%%%%%%%%%%%%%%%
\paragraph{v0.6:} 2017/04/26

\begin{itemize}
\item
redirection mechanism added
\end{itemize}

%%%%%%%%%%%%%%%%%%%%%%%%%%%%%%%%%%%%%%%%
\paragraph{v0.5:} 2017/04/26

\begin{itemize}
\item
functionality in definition file
\end{itemize}


%%%%%%%%%%%%%%%%%%%%%%%%%%%%%%%%%%%%%%%%%%%%%%%%%%%%%%%%%%%%%%%%%%%%%%%%%%%%%%%%
%%%%%%%%%%%%%%%%%%%%%%%%%%%%%%%%%%%%%%%%%%%%%%%%%%%%%%%%%%%%%%%%%%%%%%%%%%%%%%%%
%%%%%%%%%%%%%%%%%%%%%%%%%%%%%%%%%%%%%%%%%%%%%%%%%%%%%%%%%%%%%%%%%%%%%%%%%%%%%%%%
\appendix

\settowidth\MacroIndent{\rmfamily\scriptsize 000\ }

 \DocInput{childdoc.dtx}

\end{document}
%</driver>
% \fi
%
% %%%%%%%%%%%%%%%%%%%%%%%%%%%%%%%%%%%%%%%%%%%%%%%%%%%%%%%%%%%%%%%%%%%%%%%%%%%%%%
% %%%%%%%%%%%%%%%%%%%%%%%%%%%%%%%%%%%%%%%%%%%%%%%%%%%%%%%%%%%%%%%%%%%%%%%%%%%%%%
% \section{Sample}
%\iffalse
%<*samplemain>
%\fi
%
% The following presents a sample document
% with two chapters, two parts, a title page,
% a compile flag as well as three forwarding files to set the flag.
% It consists of eight |.tex| files:
% \begin{center}
% \begin{tabular}{ll}
% |cdocsamp.tex|&main file\\
% |cdocsch1.tex|&include file for chapter 1\\
% |cdocsch2.tex|&include file for chapter 2\\
% |cdocspt3.tex|&include file for part 3\\
% |cdocspt4.tex|&include file for part 4\\
% |cdocsdrf.tex|&forwarding file for main file in draft mode\\
% |cdocsfi1.tex|&forwarding file for final version of chapter 1\\
% |cdocsfi2.tex|&forwarding file for final version of chapter 2\\
% \end{tabular}
% \end{center}
% Each of the eight files can be compiled directly by the \LaTeX{} compiler.
%
% %%%%%%%%%%%%%%%%%%%%%%%%%%%%%%%%%%%%%%
% \paragraph{Main File.}
%
% The main file is called |cdocsamp.tex|.
%
% Load the \textsf{childdoc} definitions and
% declare the filename for the main document:
%    \begin{macrocode}
\input{childdoc.def}
\childdocmain{}
%    \end{macrocode}

% Optional override for |\version| flag:
%    \begin{macrocode}
%%\ifchilddoc\else\providecommand{\version}{draft}\fi
%    \end{macrocode}

% Define the default values for the |\version| flag
% (|final| for the main file and |draft| for childs):
%    \begin{macrocode}
\ifchilddoc
\providecommand{\version}{draft}
\else
\providecommand{\version}{final}
\fi
%    \end{macrocode}

% Load the standard document class:
%    \begin{macrocode}
\documentclass[12pt]{article}
%    \end{macrocode}

% Start the document body:
%    \begin{macrocode}
\begin{document}
%    \end{macrocode}

% Declare a title page.
% Print title, part of document being processed and version flag:
%    \begin{macrocode}
\addtocounter{page}{-1}
\begin{center}
{\LARGE\bfseries{}childdoc example\par}
\vspace{1cm}
\ifchilddoc
\ifchilddocmanual part\else chapter\fi:
`\childdocname' of `\childdocjob'\par
\else
main document: `\childdocjob'\par
\fi
version: \version\par
\end{center}
\newpage
%    \end{macrocode}

% Manually include selected file,
% otherwise process as usual:
%    \begin{macrocode}
\ifchilddocmanual
\section*{part `\childdocname'}
\input{\childdocname}
\else
%    \end{macrocode}

% Include the two chapters:
%    \begin{macrocode}
\include{cdocsch1}
\include{cdocsch2}
%    \end{macrocode}

% Include the two parts unless only chapters should be displayed:
%    \begin{macrocode}
\ifchilddoc\else
\section{part three}
\input{cdocspt3}
\section{part four}
\input{cdocspt4}
\fi
%    \end{macrocode}

% Process as usual until here:
%    \begin{macrocode}
\fi
%    \end{macrocode}

% End of document body:
%    \begin{macrocode}
\end{document}
%    \end{macrocode}
%\iffalse
%</samplemain>
%\fi
%
% %%%%%%%%%%%%%%%%%%%%%%%%%%%%%%%%%%%%%%
% \paragraph{Chapter Include Files.}
%
% The include files are called |cdocsch1.tex| and |cdocsch2.tex|.
%
%\iffalse
%<*samplechap1|samplechap2>
%\fi

% Optional override for |\version| flag:
%    \begin{macrocode}
%%\providecommand{\version}{final}
%    \end{macrocode}

% Include the main document:
%    \begin{macrocode}
\input{childdoc.def}
\childdocof{cdocsamp}
%    \end{macrocode}

%\iffalse
%</samplechap1|samplechap2>
%\fi
%
%\iffalse
%<*samplechap1>
%\fi
% Some text for chapter 1:
%    \begin{macrocode}
\section{one}
some text in chapter one
%    \end{macrocode}

%\iffalse
%</samplechap1>
%\fi
% Some text for chapter 2:
%\iffalse
%<*samplechap2>
%\fi
%    \begin{macrocode}
\section{two}
more text in chapter two
%    \end{macrocode}

%\iffalse
%</samplechap2>
%\fi
%
% %%%%%%%%%%%%%%%%%%%%%%%%%%%%%%%%%%%%%%
% \paragraph{Part Include Files.}
%
% The include files are called |cdocspt3.tex| and |cdocspt4.tex|.
%
%\iffalse
%<*samplepart3|samplepart4>
%\fi

% Optional override for |\version| flag:
%    \begin{macrocode}
%%\providecommand{\version}{final}
%    \end{macrocode}

% Include the main document:
%    \begin{macrocode}
\input{childdoc.def}
\childdocby{cdocsamp}
%    \end{macrocode}

%\iffalse
%</samplepart3|samplepart4>
%\fi
%
%\iffalse
%<*samplepart3>
%\fi
% Some text for part 3:
%    \begin{macrocode}
some text in part three
%    \end{macrocode}

%\iffalse
%</samplepart3>
%\fi
% Some text for part 4:
%\iffalse
%<*samplepart4>
%\fi
%    \begin{macrocode}
more text in part four
%    \end{macrocode}

%\iffalse
%</samplepart4>
%\fi
%
% %%%%%%%%%%%%%%%%%%%%%%%%%%%%%%%%%%%%%%
% \paragraph{Forwarding for a Complete Draft.}
%
% The following forwarding file |cdocsdrf.tex|
% compiles the main document in draft mode:
%\iffalse
%<*sampledraft>
%\fi
%    \begin{macrocode}
\def\version{draft}
\input{childdoc.def}
\childdocforward{cdocsamp}
%    \end{macrocode}

%\iffalse
%</sampledraft>
%\fi
%
% %%%%%%%%%%%%%%%%%%%%%%%%%%%%%%%%%%%%%%
% \paragraph{Forwarding for Final Version of the Chapters.}
%
% The following forwarding files |cdocsfn1.tex| and |cdocsfn2.tex|
% (with identical content)
% compile the final versions of the child documents
% |cdocsch1.tex| and |cdocsch2.tex|, respectively:
%\iffalse
%<*samplefinal>
%\fi
%    \begin{macrocode}
\def\version{final}
\input{childdoc.def}
\childdocforwardprefix[cdocsamp]{cdocsfn}{cdocsch}
%    \end{macrocode}

%\iffalse
%</samplefinal>
%\fi
%
% %%%%%%%%%%%%%%%%%%%%%%%%%%%%%%%%%%%%%%
% \paragraph{Command Line Processing.}
%
% The following three command lines generate the output files
% |cdocscld|, |cdocscl1| and |cdocscl2|
% which should be identical to
% |cdocsdrf|, |cdocsch1| and |cdocsfn2|, respectively:
% \begin{center}
% \begin{tabular}{l}
% |latex -jobname cdocscld \|\\
% |  "\def\version{draft}\input{childdoc.def}\childdocforward{cdocsamp}"|\\
% |latex -jobname cdocscl1 \|\\
% |  "\input{childdoc.def}\childdocforward[cdocsamp]{cdocsch1}"|\\
% |latex -jobname cdocscl2 \|\\
% |  "\def\version{final}\input{childdoc.def}\childdocforward{cdocsch2}"|
% \end{tabular}
% \end{center}
% Note that the trailing backslash on each first line
% merely continues the input to the second line
% (for convenient cut ant paste).
% Furthermore, the command |latex| can be replaced by any
% of its alternative versions such as |pdflatex|.
%
% %%%%%%%%%%%%%%%%%%%%%%%%%%%%%%%%%%%%%%%%%%%%%%%%%%%%%%%%%%%%%%%%%%%%%%%%%%%%%%
% %%%%%%%%%%%%%%%%%%%%%%%%%%%%%%%%%%%%%%%%%%%%%%%%%%%%%%%%%%%%%%%%%%%%%%%%%%%%%%
% \section{Implementation}
%\iffalse
%<*package>
%\fi
%
% This section describes the definitions file |childdoc.def|.

% The definitions cannot be loaded using |\usepackage| or |\RequirePackage|
% which has a mechanism to prevent loading a style file more than once.
% When loading the definitions by means of |\input|
% multiple instances have to be prevented manually:
%\iffalse
%This code needs to be before the `\ProvidesFile' directive
%which is defined at the beginning of this file.
%Therefore it is also placed there and commented out here.
%</package>
%<*discard>
%\fi
%    \begin{macrocode}
\ifdefined\childdocmain\endinput\fi
%    \end{macrocode}
%\iffalse
%</discard>
%<*package>
%\fi
%
% \macro{\ifchilddoc}
% \macro{\ifchilddocmanual}
% The conditional |\ifchilddoc| tells whether a
% child (true) or main (false) document is being compiled.
% The conditional |\ifchilddocmanual| tells whether
% the |\includeonly| mechanism is used (false) or
% the selection of child files must be performed manually (true).
% The definitions initialise to false:
%    \begin{macrocode}
\newif\ifchilddoc
\newif\ifchilddocmanual
%    \end{macrocode}

% \macro{\childdocname}
% \macro{\childdocjob}
% The macro |\childdocname| stores the name of the main document
% to be compiled. The macro |\childdocjob| stores the name of
% the document on which the \LaTeX{} compiler was originally invoked.
% The content of |\jobname| cannot be compared
% to filenames specified in the source due to different catcodes.
% The following code rescans |\jobname|, stores the result
% in |\childdocname| and saves a copy in |\childdocjob|:
%    \begin{macrocode}
\edef\childdocname{\scantokens\expandafter{\jobname\noexpand}}
\let\childdocjob\childdocname
%    \end{macrocode}

% \macro{\childdocdisable}
% The macro |\childdocdisable| prevents the main file
% from being processed more than once.
% At this stage, the main document command |\childdocmain|
% is assumed to be called once again where it should do nothing.
% Any subsequent call to it should prevent
% a secondary processing of the main document
% It overwrites the forwarding commands
% |\childdocof| and |\childdocforward|
% with empty macros to prevent further inclusions of the main document:
%    \begin{macrocode}
\newcommand{\childdocdisable}
{
  \renewcommand{\childdocmain}[1]{\renewcommand{\childdocmain}[1]{\endinput}}
  \renewcommand{\childdocof}[1]{}
  \renewcommand{\childdocby}[2][]{}
  \renewcommand{\childdocforward}[2][]{}
  \renewcommand{\childdocdisable}{}
}
%    \end{macrocode}

% \macro{\childdocmain}
% The macro |\childdocmain| is to be called at the top of the main file
% with nothing or the main filename (without extension) as argument.
% First, it breaks loops.
% If the argument is not empty and does not match |\childdocname|
% (which is set by the first inclusion of |childdoc.def|),
% |\ifchilddoc| is set to true, |\includeonly| is applied to the child file
% and |\jobname| is set to the main file
% (for proper handling of |.aux| files):
%    \begin{macrocode}
\newcommand{\childdocmain}[1]
{
  \childdocdisable\childdocmain{}
  \if?#1?\else
    \begingroup
      \def\childdoctmp{#1}
      \ifx\childdoctmp\childdocname
        \def\childdoctmp{}
      \else
        \def\childdoctmp
        {
          \childdoctrue
          \includeonly{\childdocname}
          \def\childdocjob{#1}
          \def\jobname{#1}
        }
      \fi
      \expandafter
    \endgroup
    \childdoctmp
  \fi
}
%    \end{macrocode}

% \macro{\childdocof}
% The command |\childdocof| redirects
% compilation to the main file |#1|.
%    \begin{macrocode}
\newcommand{\childdocof}[1]
{
  \childdocdisable
  \childdoctrue
  \includeonly{\childdocname}
  \def\jobname{#1}
  \def\childdocjob{#1}
  \input{#1}
}
%    \end{macrocode}

% \macro{\childdocby}
% The command |\childdocby| ....
%    \begin{macrocode}
\newcommand{\childdocby}[2][]
{
  \childdocdisable
  \childdoctrue
  \childdocmanualtrue
  \if?#1?\else
    \def\jobname{#2}
  \fi
  \def\childdocjob{#2}
  \input{#2}
  \endinput
}
%    \end{macrocode}

% \macro{\childdocforward}
% The command |\childdocforward| redirects
% compilation to the main file or
% (if the optional argument is given) a child file.
% Parameters are set as if the main file
% or a child file starting with |\childdocof| was compiled.
% Then compilation is handed over to the main file:
%    \begin{macrocode}
\newcommand{\childdocforward}[2][]
{
  \begingroup
    \if?#1?
      \def\childdoctmp
      {
        \def\childdocname{#2}
        \def\childdocjob{#2}
        \def\jobname{#2}
        \input{#2}
        \endinput
      }
    \else
      \def\childdoctmp
      {
        \childdocdisable
        \def\childdocname{#2}
        \childdoctrue
        \includeonly{#2}
        \def\childdocjob{#1}
        \def\jobname{#1}
        \input{#1}
        \endinput
      }
    \fi
    \expandafter
  \endgroup
  \childdoctmp
}
%    \end{macrocode}

% \macro{\childdocforwardprefix}
% The command |\childdocforwardprefix| redirects
% compilation to the main or a child file by means of a pattern.
% The prefix |#1| in the current filename is replaced by |#2|
% and the suffix of the current filename is kept
% (it is assumed that the filename does not contain the substring `|~~~|'
% which is used as a delimiter).
% Compilation is handed over to the new file by |\childdocforward|:
%    \begin{macrocode}
\newcommand{\childdocforwardprefix}[3][]
{
  \begingroup
    \def\childdocextract #2##1~~~{\def\childdoctmp{\childdocforward[#1]{#3##1}}}
    \expandafter\childdocextract\childdocname~~~
    \expandafter
  \endgroup
  \childdoctmp
}
%    \end{macrocode}

% \macro{\childdoc}
% The deprecated macro |\childdoc| is a legacy version of |\childdocmain|:
%    \begin{macrocode}
\newcommand{\childdoc}{\childdocmain}
%    \end{macrocode}

% \macro{\childdocredirect}
% The deprecated macro |\childdocredirect| is a legacy version
% of |\childdocforward| and |\childdocforwardprefix|:
%    \begin{macrocode}
\newcommand{\childdocredirect}[2][]
{
  \begingroup
    \if?#1?
      \def\childdoctmp{\childdocforward{#2}}
    \else
      \def\childdoctmp{\childdocforwardprefix{#1}{#2}}
    \fi
    \expandafter
  \endgroup
  \childdoctmp
}
%    \end{macrocode}

%\iffalse
%</package>
%\fi
%
\endinput
\childdocforward{cdocsamp}"|\\
% |latex -jobname cdocscl1 \|\\
% |  "% \iffalse
%
% childdoc.dtx Copyright (C) 2017-2018 Niklas Beisert
%
% This work may be distributed and/or modified under the
% conditions of the LaTeX Project Public License, either version 1.3
% of this license or (at your option) any later version.
% The latest version of this license is in
%   http://www.latex-project.org/lppl.txt
% and version 1.3 or later is part of all distributions of LaTeX
% version 2005/12/01 or later.
%
% This work has the LPPL maintenance status `maintained'.
%
% The Current Maintainer of this work is Niklas Beisert.
%
% This work consists of the files childdoc.dtx and childdoc.ins
% and the derived files childdoc.def and cdocsamp.tex with
% cdocsch1.tex, cdocsch2.tex, cdocsdrf.tex, cdocsfn1.tex, cdocsfn2.tex.
%
%<package>\ifdefined\childdocmain\endinput\fi
%<package>\ProvidesFile{childdoc.def}[2018/12/30 v2.0 child document driver]
%<samplemain>\ProvidesFile{cdocsamp.tex}[2018/12/30 v2.0 sample for childdoc]
%<*driver>
%\ProvidesFile{childdoc.drv}[2018/12/30 v2.0 childdoc reference manual file]
\PassOptionsToClass{10pt,a4paper}{article}
\documentclass{ltxdoc}

\usepackage[margin=35mm]{geometry}
\usepackage{hyperref}
\usepackage{hyperxmp}
\usepackage[usenames]{color}

\hypersetup{colorlinks=true}
\hypersetup{pdfstartview=FitH}
\hypersetup{pdfpagemode=UseNone}
\hypersetup{pdfsource={}}
\hypersetup{pdflang={en-UK}}
\hypersetup{pdfcopyright={Copyright 2017-2018 Niklas Beisert.
  This work may be distributed and/or modified under the
  conditions of the LaTeX Project Public License, either version 1.3
  of this license or (at your option) any later version.}}
\hypersetup{pdflicenseurl={http://www.latex-project.org/lppl.txt}}
\hypersetup{pdfcontactaddress={ETH Zurich, ITP, HIT K,
  Wolfgang-Pauli-Strasse 27}}
\hypersetup{pdfcontactpostcode={8093}}
\hypersetup{pdfcontactcity={Zurich}}
\hypersetup{pdfcontactcountry={Switzerland}}
\hypersetup{pdfcontactemail={nbeisert@itp.phys.ethz.ch}}
\hypersetup{pdfcontacturl={http://people.phys.ethz.ch/\xmptilde nbeisert/}}

\newcommand{\secref}[1]{\hyperref[#1]{section \ref*{#1}}}

\parskip1ex
\parindent0pt
\let\olditemize\itemize
\def\itemize{\olditemize\parskip0pt}

\begin{document}

\title{The \textsf{childdoc} Package}
\hypersetup{pdftitle={The childdoc Package}}
\author{Niklas Beisert\\[2ex]
  Institut f\"ur Theoretische Physik\\
  Eidgen\"ossische Technische Hochschule Z\"urich\\
  Wolfgang-Pauli-Strasse 27, 8093 Z\"urich, Switzerland\\[1ex]
  \href{mailto:nbeisert@itp.phys.ethz.ch}
  {\texttt{nbeisert@itp.phys.ethz.ch}}}
\hypersetup{pdfauthor={Niklas Beisert}}
\hypersetup{pdfsubject={Manual for the LaTeX2e Package childdoc}}
\date{30 December 2018, \textsf{v2.0}}
\maketitle

\begin{abstract}\noindent
\textsf{childdoc} is a \LaTeXe{} package
that enables the direct compilation
of document sections included by |\include|
to individual files.
\end{abstract}

\begingroup
\parskip0ex
\tableofcontents
\endgroup

%%%%%%%%%%%%%%%%%%%%%%%%%%%%%%%%%%%%%%%%%%%%%%%%%%%%%%%%%%%%%%%%%%%%%%%%%%%%%%%%
%%%%%%%%%%%%%%%%%%%%%%%%%%%%%%%%%%%%%%%%%%%%%%%%%%%%%%%%%%%%%%%%%%%%%%%%%%%%%%%%
\section{Introduction}

\LaTeX{} provides a mechanism to structure a large document (such as a book)
into a main file and several child files (containing the chapters)
using the |\include| command.
This mechanism is beneficial for documents
which span hundreds of pages in order to
make the source file(s) more manageable.
Moreover, compilation can be restricted to
selected child files by means of the |\includeonly| command.
The latter feature can be used to reduce the compilation time while editing
(this was significantly more useful in the earlier days of \LaTeX{})
or to generate a smaller document which is easier to navigate.
Another application of |\includeonly| is to generate
documents consisting of selected parts of the complete document.

However, there are a few drawbacks of the plain |\include| mechanism:
\begin{itemize}
\item
The child files cannot be compiled on their own,
they can only be compiled via the main file.
A naive editing environment
(such as a text editor with an option
to have the current file processed by \LaTeX)
may require one to switch to the main file before compiling;
attempting to compile the child file produces errors.
\item
The main file must be modified (each time)
to adjust the |\includeonly| command
to the present needs. This easily leaves the main file in a messy state.
\item
The generated document will always carry the filename
of the main document. This is inconvenient if
several child files are to be compiled and
to be kept for distribution.
\end{itemize}

The present package provides a simple interface
to make child files individually compilable by \LaTeX{}.
Compiling a child file then has the same effect as compiling
the main file with an |\includeonly| command
to select the appropriate child.
Moreover the generated document will carry the name of the child
rather than the main file.
This resolves all three above issues.

This feature is meant to make the editing of books,
thesis documents and lecture notes somewhat more convenient.
However, the package can also be used efficiently for
composing a series of documents (such as exercise sheets)
which are typically distributed individually.
It then assists the author in generating the individual documents
(potentially in different versions)
as well as a document containing the collected series.
Another application is in developing style files
or other kinds of included material
where compilation of the style file could redirect
to a sample or test file.

%%%%%%%%%%%%%%%%%%%%%%%%%%%%%%%%%%%%%%%%%%%%%%%%%%%%%%%%%%%%%%%%%%%%%%%%%%%%%%%%
%%%%%%%%%%%%%%%%%%%%%%%%%%%%%%%%%%%%%%%%%%%%%%%%%%%%%%%%%%%%%%%%%%%%%%%%%%%%%%%%
\section{Usage}

First of all, the package \textsf{childdoc} is \emph{not} a standard
\LaTeXe{} |.sty| style file! Therefore it needs to be invoked in
a non-standard way.

%%%%%%%%%%%%%%%%%%%%%%%%%%%%%%%%%%%%%%%%%%%%%%%%%%%%%%%%%%%%%%%%%%%%%%%%%%%%%%%%
\subsection{Included Files}
\label{sec:include}

%%%%%%%%%%%%%%%%%%%%%%%%%%%%%%%%%%%%%%%%
\DescribeMacro{\childdocmain}
To use the package, add the commands
\begin{center}
\begin{tabular}{l}
|\input{childdoc.def}|\\
|\childdocmain{}|\\
\end{tabular}
\end{center}
at the very top of the main \LaTeX{} file,
in particular \emph{before} the |\documentclass| statement!
The argument of |\childdocmain| should be left empty
(but it must be present).

%%%%%%%%%%%%%%%%%%%%%%%%%%%%%%%%%%%%%%%%
\DescribeMacro{\childdocof}
Furthermore, add the commands
\begin{center}
\begin{tabular}{l}
|\input{childdoc.def}|\\
|\childdocof{|\textit{main}|}|\\
\end{tabular}
\end{center}
at the top of every child file \textit{child}
which is included by |\include{|\textit{child}|}|
from within the main file
(or at least for those files to be compiled individually).
The argument \textit{main} must be the filename of the main file.

There are a couple of
considerations in setting up the main and child documents:

%%%%%%%%%%%%%%%%%%%%%%%%%%%%%%%%%%%%%%%%
\paragraph{Restrictions.}

Please note the following restrictions:
\begin{itemize}
\item
|\childdocmain| must be called with one argument \textit{main}
to ensure compatibility with earlier version of the package.
It must either be empty (|\childdocmain{}|)
or precisely match the filename of the main file in which it is specified.
See \secref{sec:detection} for further information.
\item
The filename \textit{main} must be specified without the |.tex| extension.
\item
The filename \textit{main} is case sensitive
(even in case-insensitive file systems)
due to internal string comparison.
\item
The argument \textit{main} should be fully expanded, it cannot be a macro.
\item
Subdirectories and special characters should be avoided in filenames.
\item
The command |\childdocmain{|\textit{main}|}| must be followed by a whitespace.
It should not be followed immediately by another command
or by a comment mark `|%|'.
This is because the \TeX{} parser reads the token immediately following
the argument of |\childdocmain| and puts it
at the beginning of every child section;
however, a white\-space is ignored.
\end{itemize}

%%%%%%%%%%%%%%%%%%%%%%%%%%%%%%%%%%%%%%%%
\paragraph{Content of Main File.}

It is advisable to place all content in the child files included by |\include|.
Any output contained in the main file will appear in all child documents
unless suppressed manually;
it cannot be suppressed automatically by the |\includeonly| directive
and thus should normally be avoided.
A method to include some content in the main file
by means of conditional processing is described in \secref{sec:conditional}.

%%%%%%%%%%%%%%%%%%%%%%%%%%%%%%%%%%%%%%%%
\paragraph{Page Numbering.}

When only a part of the document is compiled,
the appropriate numbering of pages
(as well as other status parameters)
is determined from the |.aux| files.
The latter contain information from previous passes.
However this information needs to propagate through
all intermediate child documents.
Therefore the page numbering in child documents may well
be inconsistent until the complete document is compiled at least once.

A useful (if unconventional) way to always ensure a consistent
page numbering is to restart the numbering in each child document
and denote the pages by `\textit{child}|.|\textit{page}'
where \textit{child} represents the chapter/section number of the child file.
This can be achieved by the command
|\numberwithin{page}{|\textit{child}|}|
of the \textsf{amsmath} package
where \textit{child} can be |chapter| or |section|
depending on the chosen structuring.
Alternatively, one can modify the macro |\thepage| appropriately
and reset the counter |page| at the start of each child file.

%%%%%%%%%%%%%%%%%%%%%%%%%%%%%%%%%%%%%%%%%%%%%%%%%%%%%%%%%%%%%%%%%%%%%%%%%%%%%%%%
\subsection{Conditional Processing}
\label{sec:conditional}

The package provides a mechanism to compile different versions
of a document. To customise the versions further some conditional processing
can come in handy to distinguish which version is being compiled.
The package provides two macros to describe the compilation context:

%%%%%%%%%%%%%%%%%%%%%%%%%%%%%%%%%%%%%%%%
\DescribeMacro{\ifchilddoc}
The conditional |\ifchilddoc| distinguishes between the compilation of
child documents and the main document:
%
\begin{center}
|\ifchilddoc |\textit{child-code}| |[|\||else |\textit{main-code}]| \||fi|
\end{center}

%%%%%%%%%%%%%%%%%%%%%%%%%%%%%%%%%%%%%%%%
\DescribeMacro{\childdocname}
\DescribeMacro{\childdocjob}
The macro |\childdocname| contains the filename (without extension)
of the main or child file being processed.
Note that |\childdocjob| will always contain the name of the main file.

%%%%%%%%%%%%%%%%%%%%%%%%%%%%%%%%%%%%%%%%
\paragraph{Title Page.}

Conditional processing can be used to include a title or banner page
in the main document when proper precautions are taken.
Importantly, the code in the main file should ensure that the page counter
(as well as other status parameters which are stored in the |.aux| files)
takes the same value after the conditional processing.
Otherwise the page numbers may take divergent values
depending on which part is compiled.

For example, a title page could be declared by:
%
\begin{center}
\begin{tabular}{l}
|\ifchilddoc\||else|\\
|\addtocounter{page}{-1}|\\
\textit{code for title page}\\
|\newpage|\\
|\||fi|
\end{tabular}
\end{center}
%
A banner page for the child documents can be generated by:
%
\begin{center}
\begin{tabular}{l}
|\ifchilddoc|\\
|\addtocounter{page}{-1}|\\
\textit{code for banner page}\\
|\newpage|\\
|\||fi|
\end{tabular}
\end{center}
%
Here one could write a message such as:
\begin{center}
|This is the part \childdocname{} of \childdocjob{}.|
\end{center}

%%%%%%%%%%%%%%%%%%%%%%%%%%%%%%%%%%%%%%%%%%%%%%%%%%%%%%%%%%%%%%%%%%%%%%%%%%%%%%%%
\subsection{Flags}
\label{sec:flags}

The package makes it easy to generate different versions
of the main or child documents.
To this end compilation flags can be defined
and assigned different default values.
They will be particularly useful in conjunction
with the forwarding mechanism described in \secref{sec:forward}.

For example, it may be useful to have a flag |\version|
which can be set to |draft| or |final|.
The document source will contain some conditional code
depending on the value of |\version|.
Suppose further, the flag should default to |final| for the main file
and to |draft| for child files
which is a natural assignment for editing the document.
This is achieved by placing the following code
in the preamble of the main document
(below the |\childdocmain| directive):
%
\begin{center}
\begin{tabular}{l}
|\ifchilddoc|\\
|\providecommand{\version}{draft}|\\
|\||else|\\
|\providecommand{\version}{final}|\\
|\||fi|
\end{tabular}
\end{center}
%
The definition by |\providecommand| makes sure
that previous definitions are not overwritten.
Further statements |\providecommand{\version}{...}|
can thus be added before the above code to override it.

For the main file, one might add a line
(between |\childdocmain| and the above block)
%
\begin{center}
|%\ifchilddoc\||else\providecommand{\version}{draft}\||fi|
\end{center}
%
which can be uncommented to produce a draft version.
Likewise one can add a line to the very top of a child file
(above the |\childdocof{|\textit{main}|}| directive)
%
\begin{center}
|%\providecommand{\version}{final}|
\end{center}
%
which can be uncommented to produce the final version of this child document.

%%%%%%%%%%%%%%%%%%%%%%%%%%%%%%%%%%%%%%%%%%%%%%%%%%%%%%%%%%%%%%%%%%%%%%%%%%%%%%%%
\subsection{Forwarding}
\label{sec:forward}

Different versions of the main or child documents
using compilation flags as described in \secref{sec:flags}
can be (permanently) stored in different files
for convenient compilation, viewing and distribution.
To this end, the package defines a command
to pass on compilation to a different file:

%%%%%%%%%%%%%%%%%%%%%%%%%%%%%%%%%%%%%%%%
\DescribeMacro{\childdocforward}
The command |\childdocforward| redirects processing to
another source file:
%
\begin{center}
\begin{tabular}{l}
|\input{childdoc.def}|\\
|\childdocforward[|\textit{main}|]{|\textit{dest}|}|\\
\end{tabular}
\end{center}
%
The argument \textit{dest} is the destination file
(without extension).
It should be the main file or one of the child files.
Note that further \textsf{childdoc} directives
such as |\childdocof| and |\childdocforward|
in the indicated file will be processed in this form.
The optional argument \textit{main}
passes on directly to the main file \textit{main}
while pretending to compile the child \textit{dest}.
This form behaves as if \textit{dest}
issues |\childdocof{|\textit{main}|}| right away,
and no further \textsf{childdoc} directives will be processed.

%%%%%%%%%%%%%%%%%%%%%%%%%%%%%%%%%%%%%%%%
\DescribeMacro{\...prefix}
In the alternative form |\childdocforwardprefix|,
%
\begin{center}
\begin{tabular}{l}
|\input{childdoc.def}|\\
|\childdocforwardprefix[|\textit{main}|]{|\textit{prefix}|}{|\textit{dest}|}|
\end{tabular}
\end{center}
%
the destination file is determined by a pattern
depending on the current file:
To make this work, the current file must be called
`{\textit{prefix}\hspace{0.2em}\textit{suffix}}'
with \textit{prefix} matching precisely the argument.
Processing is then passed on to the file
`{\textit{dest}\hspace{0.2em}\textit{suffix}}'.
Surely, the same effect is achieved by
directly specifying the
argument `{\textit{dest}\hspace{0.2em}\textit{suffix}}'
in the first form.
However, that requires to set up a different file
for each child. With the alternative form of the command
all these files can have exactly the same content
which simplifies setting them up and maintaining them.

For example, the following file |draft.tex|
with a compilation flag |\version| as described in \secref{sec:flags}
compiles the main document as a draft:
%
\begin{center}
\begin{tabular}{l}
|\def\version{draft}|\\
|\input{childdoc.def}|\\
|\childdocforward{|\textit{main}|}|
\end{tabular}
\end{center}
%
Likewise, the following files |final|\textit{nn}|.tex|
compile the final version of the child document
|child|\textit{nn}|.tex|:
%
\begin{center}
\begin{tabular}{l}
|\def\version{final}|\\
|\input{childdoc.def}|\\
|\childdocforwardprefix{final}{child}|
\end{tabular}
\end{center}
%

Note that when several versions of a main file and/or of each child file
are to be generated, it may be convenient to set up a |Makefile| or
shell script to automatise the process.

%%%%%%%%%%%%%%%%%%%%%%%%%%%%%%%%%%%%%%%%%%%%%%%%%%%%%%%%%%%%%%%%%%%%%%%%%%%%%%%%
\subsection{Command Line Processing}
\label{sec:commandline}

The effect of redirection files can also be achieved by invoking
the \LaTeX{} compiler with a more elaborate command line.
Most conveniently this should be done as part
of a shell script or a |Makefile|.

When using \textsf{childdoc} in the main file, the following
command lines effectively perform a redirection
(note that depending on the shell being used,
backslashes may have to be doubled: `|\|' $\to$ `|\\|'):
%
\begin{center}
|... -jobname "|\textit{target}|" |\\|"|[\textit{flags}]%
|\input{childdoc.def}\childdocforward[|\textit{main}|]{|\textit{dest}|}"|
\end{center}
%
Here \textit{target} is the name of the output file,
\textit{main} is the name of the main file
and \textit{dest} is the name of the main or child file to be processed
(all filenames without extensions).
The optional argument \textit{main} can be omitted
if \textit{main} matches \textit{dest}.
Optionally, compilation \textit{flags} can be defined via |\def| commands.
This command line makes the \TeX{} engine believe
it is compiling the file \textit{target}
whose content is specified as the latter parameter.
The provided code then forwards the processing to
\textit{main} or \textit{dest} as described in \secref{sec:forward}.

%%%%%%%%%%%%%%%%%%%%%%%%%%%%%%%%%%%%%%%%%%%%%%%%%%%%%%%%%%%%%%%%%%%%%%%%%%%%%%%%
\subsection{Include by Input}
\label{sec:input}

Including child documents by |\include| has some restrictions by design.
Most notably, the content of a child document always occupies
its own set of pages; pages cannot be shared between child documents.
Usually, this behaviour makes perfect sense
because each child document contain an essential part of the document.
However, in some situations it may be desirable to compose
a document from a collection of parts
without having mandatory page breaks between then.
For this case, the package
provides a mechanism to include parts
by |\input| which can also be processed individually.
However, by construction this mechanism
requires manual handling of the content to be output.

%%%%%%%%%%%%%%%%%%%%%%%%%%%%%%%%%%%%%%%%
\DescribeMacro{\ifchilddocmanual}
The main file should be prepared as usual, see \secref{sec:include}.
However, the document body must make a distinction
between processing of an individual part and of the main document, e.g.:
%
\begin{center}
\begin{tabular}{l}
|\ifchilddocmanual|\\
|\input{\childdocname}|\\
|\||else|\\
\textit{document body with }|\input{|\textit{part}|}|\\
|\||fi|
\end{tabular}
\end{center}
%
The conditional |\ifchilddocmanual| is true whenever
a part to be included by |\input| is being compiled,
and the name of the part is stored in |\childdocname|.

%%%%%%%%%%%%%%%%%%%%%%%%%%%%%%%%%%%%%%%%
\DescribeMacro{\childdocby}
Each part to be included by |\input| should start with:
%
\begin{center}
\begin{tabular}{l}
|\input{childdoc.def}|\\
|\childdocby{|\textit{main}|}|\\
\end{tabular}
\end{center}
%
The directive |\childdocby| is similar to |\childdocof|
described in \secref{sec:include},
but the subsequent selection of content must be done manually.
To that end, both |\ifchilddoc| and |\ifchilddocmanual|
will be true upon processing of a part,
and the name of the part is stored in |\childdocname|.
Note that |\jobname| will be set to the filename of the current part
so that each part receives an individual |.aux| file
that does not interfere with the |.aux| file(s) of the main document.
This behaviour can be altered by the alternative form
|\childdocby[*]{|\textit{main}|}| (with a non-empty optional argument)
which uses the |.aux| file of the main document
by setting |\jobname| to \textit{main}.

%%%%%%%%%%%%%%%%%%%%%%%%%%%%%%%%%%%%%%%%%%%%%%%%%%%%%%%%%%%%%%%%%%%%%%%%%%%%%%%%
\subsection{Driver Development}
\label{sec:driver}

The \textsf{childdoc} mechanism can also be use for the development
of definition files such as \LaTeX{} styles or classes.
This case differs from the above setup with multiple parts
included by |\include| in that no |\includeonly| should be invoked.
This can be achieved by starting the include file
(before |\ProvidesPackage|) with:
%
\begin{center}
\begin{tabular}{l}
|\input{childdoc.def}|\\
|\childdocforward{|\textit{main}|}|\\
\end{tabular}
\end{center}
%
or alternatively with:
%
\begin{center}
\begin{tabular}{l}
|\input{childdoc.def}|\\
|\childdocby{|\textit{main}|}|\\
\end{tabular}
\end{center}
%
Both forms have slightly different effects as described above.
The main file is prepared as usual, see \secref{sec:include}.

%%%%%%%%%%%%%%%%%%%%%%%%%%%%%%%%%%%%%%%%%%%%%%%%%%%%%%%%%%%%%%%%%%%%%%%%%%%%%%%%
\subsection{Legacy Detection}
\label{sec:detection}

The directive |\childdocmain| in the main file can detect
whether the complete document or merely a child is to be compiled
even without using the directive |\childdocof|.
This method is deprecated because it is less robust
and there is no compelling reason to use it;
it is merely provided for backward compatibility
and it may be removed in future versions.

If the detection mechanism is to be used,
it is mandatory to correctly specify
the filename of the main file as the argument of |\childdocmain|:
%
\begin{center}
\begin{tabular}{l}
|\input{childdoc.def}|\\
|\childdocmain{|\textit{main}|}|\\
\end{tabular}
\end{center}
%
If |\jobname| does not match the argument \textit{main} of |\childdocmain|,
it is assumed that |\jobname| points to the child file to be compiled.
When using |\childdocmain| with the main file specified as argument,
it suffices to start a child file
with just |\input{|\textit{main}|}|
without loading of the package and using |\childdocof|.
If instead all processing is done
with the appropriate \textsf{childdoc} directives,
the argument of \textit{main} of |\childdocmain| can be empty.

An alternative version of the command line processing described
in \secref{sec:commandline} using the detection mechanism reads:
%
\begin{center}
|... -jobname "|\textit{target}|" "|[\textit{flags}]%
[|\def\jobname{|\textit{dest}|}|]|\input{|\textit{main}|}"|
\end{center}

%%%%%%%%%%%%%%%%%%%%%%%%%%%%%%%%%%%%%%%%%%%%%%%%%%%%%%%%%%%%%%%%%%%%%%%%%%%%%%%%
\subsection{Manual Code}
\label{sec:manual}

In case one cannot be certain whether the definitions file |childdoc.def|
is installed on the target \TeX{} distribution
and one prefers not to ship it,
it is conceivable to paste a few relevant commands into the sources.

To that end, drop all statements |\input{childdoc.def}|
and perform the replacements as outlined below.
Instead of |\childdocmain{|\textit{main}|}| add the following code
to the top of the main file:
%
\begin{center}
\begin{tabular}{l}
|\||ifdefined\childdocname\endinput\||fi\newif\ifchilddoc|\\
|\edef\childdocname{\scantokens\expandafter{\jobname\noexpand}}|\\
|\def\childdocmain{|\textit{main}|}\||ifx\childdocmain\childdocname\||else|\\
|\childdoctrue\includeonly{\childdocname}\let\jobname\childdocmain\||fi|\\
\end{tabular}
\end{center}
%
Instead of |\childdocof{|\textit{main}|}| just include the main file
at the top of each child file:
%
\begin{center}
|\input{|\textit{main}|}|
\end{center}
%
A simple redirection |\childdocforward{|\textit{dest}|}| is achieved by:
%
\begin{center}
|\def\jobname{|\textit{dest}|}\input{\jobname}|
\end{center}
%
The redirection with prefix
|\childdocforwardprefix[|\textit{prefix}|]{|\textit{dest}|}|
is accomplished by:
%
\begin{center}
\begin{tabular}{l}
|{\edef\jobname{\scantokens\expandafter{\jobname\noexpand}}|\\
|\def\redirectjob |\textit{prefix}|#1~~~{\gdef\jobname{|\textit{dest}|#1}}|\\
|\expandafter\redirectjob\jobname~~~}\input{\jobname}|
\end{tabular}
\end{center}

In an alternative approach,
child documents can be compiled by a specific command line
without additional code or specific definitions:
%
\begin{center}
|... -jobname "|\textit{target}|" "|[\textit{flags}]%
|\includeonly{|\textit{dest}|}\input{|\textit{main}|}"|
\end{center}
%

%%%%%%%%%%%%%%%%%%%%%%%%%%%%%%%%%%%%%%%%%%%%%%%%%%%%%%%%%%%%%%%%%%%%%%%%%%%%%%%%
%%%%%%%%%%%%%%%%%%%%%%%%%%%%%%%%%%%%%%%%%%%%%%%%%%%%%%%%%%%%%%%%%%%%%%%%%%%%%%%%
\section{Information}

%%%%%%%%%%%%%%%%%%%%%%%%%%%%%%%%%%%%%%%%%%%%%%%%%%%%%%%%%%%%%%%%%%%%%%%%%%%%%%%%
\subsection{Copyright}

Copyright \copyright{} 2017--2018 Niklas Beisert

This work may be distributed and/or modified under the
conditions of the \LaTeX{} Project Public License, either version 1.3
of this license or (at your option) any later version.
The latest version of this license is in
  \url{http://www.latex-project.org/lppl.txt}
and version 1.3 or later is part of all distributions of \LaTeX{}
version 2005/12/01 or later.

This work has the LPPL maintenance status `maintained'.

The Current Maintainer of this work is Niklas Beisert.

This work consists of the files |README.txt|, |childdoc.ins| and |childdoc.dtx|
as well as the derived files |childdoc.def|, |cdocsamp.tex|
with |cdocsch1.tex|, |cdocsch2.tex|, |cdocspt3.tex|, |cdocspt4.tex|,
|cdocsdrf.tex|, |cdocsfn1.tex|, |cdocsfn2.tex|
as well as |childdoc.pdf|.

%%%%%%%%%%%%%%%%%%%%%%%%%%%%%%%%%%%%%%%%%%%%%%%%%%%%%%%%%%%%%%%%%%%%%%%%%%%%%%%%
\subsection{Files and Installation}

The package consists of the files:
%
\begin{center}
\begin{tabular}{ll}
    |README.txt|   & readme file \\
    |childdoc.ins| & installation file \\
    |childdoc.dtx| & source file \\
    |childdoc.def| & definition file \\
    |cdocsamp.tex| & sample main file \\
    |cdocsch1.tex| & sample include file \\
    |cdocsch2.tex| & sample include file \\
    |cdocspt3.tex| & sample part file \\
    |cdocspt4.tex| & sample part file \\
    |cdocsdrf.tex| & sample redirection file \\
    |cdocsfn1.tex| & sample redirection file \\
    |cdocsfn2.tex| & sample redirection file \\
    |childdoc.pdf| & manual
\end{tabular}
\end{center}
%
The distribution consists of the files
|README.txt|, |childdoc.ins| and |childdoc.dtx|.
%
\begin{itemize}
\item
Run (pdf)\LaTeX{} on |childdoc.dtx|
to compile the manual |childdoc.pdf| (this file).
\item
Run \LaTeX{} on |childdoc.ins| to create the definitions file |childdoc.def|
and the sample |cdocsamp.tex| with include files
|cdocsch1.tex|, |cdocsch2.tex|, |cdocspt3.tex|, |cdocspt4.tex|,
|cdocsdrf.tex|, |cdocsfn1.tex|, |cdocsfn2.tex|.
Then copy the file |childdoc.def| to an appropriate directory of your \LaTeX{}
distribution, e.g.\ \textit{texmf-root}|/tex/latex/childdoc|.
\end{itemize}

%%%%%%%%%%%%%%%%%%%%%%%%%%%%%%%%%%%%%%%%%%%%%%%%%%%%%%%%%%%%%%%%%%%%%%%%%%%%%%%%
\subsection{Related CTAN Packages}

There are several other packages which offer a similar functionality:
%
\begin{itemize}
\item
The packages
\href{http://ctan.org/pkg/docmute}{\textsf{docmute}},
\href{http://ctan.org/pkg/includex}{\textsf{includex}} and
\href{http://ctan.org/pkg/standalone}{\textsf{standalone}}
provide commands to include only the document body of
a child file thus allowing both files to be compiled individually.
\item
The packages \href{http://ctan.org/pkg/subdocs}{\textsf{subdocs}}
and \href{http://ctan.org/pkg/subfiles}{\textsf{subfiles}}
provide structures in which the main and child documents can be
encapsulated and allowing them to be compiled individually.
The inclusion mechanism is different from the conventional |\include|.
\item
The package \href{http://ctan.org/pkg/combine}{\textsf{combine}}
is an elaborate solution to combine several documents into one.
\end{itemize}
%
See also the CTAN topic \href{http://ctan.org/topic/subdocs}{\textsf{subdocs}}
for further related packages.
The present package differs from the above solutions in that
a document structure constructed with the conventional |\include| mechanism
just needs two extra commands at the top of every file
such that all constituent files can be compiled individually.

%%%%%%%%%%%%%%%%%%%%%%%%%%%%%%%%%%%%%%%%%%%%%%%%%%%%%%%%%%%%%%%%%%%%%%%%%%%%%%%%
%\subsection{Feature Suggestions}
%
%The following is a list of features which may be useful for future
%versions of this package:
%%
%\begin{itemize}
%\item
%\ldots
%\end{itemize}

%%%%%%%%%%%%%%%%%%%%%%%%%%%%%%%%%%%%%%%%%%%%%%%%%%%%%%%%%%%%%%%%%%%%%%%%%%%%%%%%
\subsection{Revision History}

%%%%%%%%%%%%%%%%%%%%%%%%%%%%%%%%%%%%%%%%
\paragraph{v2.0:} 2018/12/30

\begin{itemize}
\item
immediate forward processing
\item
added |\childdocby| mechanism
\item
manual restructured
\end{itemize}

%%%%%%%%%%%%%%%%%%%%%%%%%%%%%%%%%%%%%%%%
\paragraph{v1.6:} 2018/01/17

\begin{itemize}
\item
application for development of include files
\item
corrections to manual
\end{itemize}

%%%%%%%%%%%%%%%%%%%%%%%%%%%%%%%%%%%%%%%%
\paragraph{v1.5:} 2017/05/21

\begin{itemize}
\item
more complete structuring introduced
\item
|\childdocof| introduced
\item
|\childdoc| renamed to |\childdocmain|
\item
|\childredirect| renamed to |\childdocforward| and |\childdocforwardprefix|
and functionality expanded
\end{itemize}

%%%%%%%%%%%%%%%%%%%%%%%%%%%%%%%%%%%%%%%%
\paragraph{v1.0:} 2017/04/27

\begin{itemize}
\item
manual and install package
\item
first version published on CTAN
\end{itemize}

%%%%%%%%%%%%%%%%%%%%%%%%%%%%%%%%%%%%%%%%
\paragraph{v0.6:} 2017/04/26

\begin{itemize}
\item
redirection mechanism added
\end{itemize}

%%%%%%%%%%%%%%%%%%%%%%%%%%%%%%%%%%%%%%%%
\paragraph{v0.5:} 2017/04/26

\begin{itemize}
\item
functionality in definition file
\end{itemize}


%%%%%%%%%%%%%%%%%%%%%%%%%%%%%%%%%%%%%%%%%%%%%%%%%%%%%%%%%%%%%%%%%%%%%%%%%%%%%%%%
%%%%%%%%%%%%%%%%%%%%%%%%%%%%%%%%%%%%%%%%%%%%%%%%%%%%%%%%%%%%%%%%%%%%%%%%%%%%%%%%
%%%%%%%%%%%%%%%%%%%%%%%%%%%%%%%%%%%%%%%%%%%%%%%%%%%%%%%%%%%%%%%%%%%%%%%%%%%%%%%%
\appendix

\settowidth\MacroIndent{\rmfamily\scriptsize 000\ }

 \DocInput{childdoc.dtx}

\end{document}
%</driver>
% \fi
%
% %%%%%%%%%%%%%%%%%%%%%%%%%%%%%%%%%%%%%%%%%%%%%%%%%%%%%%%%%%%%%%%%%%%%%%%%%%%%%%
% %%%%%%%%%%%%%%%%%%%%%%%%%%%%%%%%%%%%%%%%%%%%%%%%%%%%%%%%%%%%%%%%%%%%%%%%%%%%%%
% \section{Sample}
%\iffalse
%<*samplemain>
%\fi
%
% The following presents a sample document
% with two chapters, two parts, a title page,
% a compile flag as well as three forwarding files to set the flag.
% It consists of eight |.tex| files:
% \begin{center}
% \begin{tabular}{ll}
% |cdocsamp.tex|&main file\\
% |cdocsch1.tex|&include file for chapter 1\\
% |cdocsch2.tex|&include file for chapter 2\\
% |cdocspt3.tex|&include file for part 3\\
% |cdocspt4.tex|&include file for part 4\\
% |cdocsdrf.tex|&forwarding file for main file in draft mode\\
% |cdocsfi1.tex|&forwarding file for final version of chapter 1\\
% |cdocsfi2.tex|&forwarding file for final version of chapter 2\\
% \end{tabular}
% \end{center}
% Each of the eight files can be compiled directly by the \LaTeX{} compiler.
%
% %%%%%%%%%%%%%%%%%%%%%%%%%%%%%%%%%%%%%%
% \paragraph{Main File.}
%
% The main file is called |cdocsamp.tex|.
%
% Load the \textsf{childdoc} definitions and
% declare the filename for the main document:
%    \begin{macrocode}
\input{childdoc.def}
\childdocmain{}
%    \end{macrocode}

% Optional override for |\version| flag:
%    \begin{macrocode}
%%\ifchilddoc\else\providecommand{\version}{draft}\fi
%    \end{macrocode}

% Define the default values for the |\version| flag
% (|final| for the main file and |draft| for childs):
%    \begin{macrocode}
\ifchilddoc
\providecommand{\version}{draft}
\else
\providecommand{\version}{final}
\fi
%    \end{macrocode}

% Load the standard document class:
%    \begin{macrocode}
\documentclass[12pt]{article}
%    \end{macrocode}

% Start the document body:
%    \begin{macrocode}
\begin{document}
%    \end{macrocode}

% Declare a title page.
% Print title, part of document being processed and version flag:
%    \begin{macrocode}
\addtocounter{page}{-1}
\begin{center}
{\LARGE\bfseries{}childdoc example\par}
\vspace{1cm}
\ifchilddoc
\ifchilddocmanual part\else chapter\fi:
`\childdocname' of `\childdocjob'\par
\else
main document: `\childdocjob'\par
\fi
version: \version\par
\end{center}
\newpage
%    \end{macrocode}

% Manually include selected file,
% otherwise process as usual:
%    \begin{macrocode}
\ifchilddocmanual
\section*{part `\childdocname'}
\input{\childdocname}
\else
%    \end{macrocode}

% Include the two chapters:
%    \begin{macrocode}
\include{cdocsch1}
\include{cdocsch2}
%    \end{macrocode}

% Include the two parts unless only chapters should be displayed:
%    \begin{macrocode}
\ifchilddoc\else
\section{part three}
\input{cdocspt3}
\section{part four}
\input{cdocspt4}
\fi
%    \end{macrocode}

% Process as usual until here:
%    \begin{macrocode}
\fi
%    \end{macrocode}

% End of document body:
%    \begin{macrocode}
\end{document}
%    \end{macrocode}
%\iffalse
%</samplemain>
%\fi
%
% %%%%%%%%%%%%%%%%%%%%%%%%%%%%%%%%%%%%%%
% \paragraph{Chapter Include Files.}
%
% The include files are called |cdocsch1.tex| and |cdocsch2.tex|.
%
%\iffalse
%<*samplechap1|samplechap2>
%\fi

% Optional override for |\version| flag:
%    \begin{macrocode}
%%\providecommand{\version}{final}
%    \end{macrocode}

% Include the main document:
%    \begin{macrocode}
\input{childdoc.def}
\childdocof{cdocsamp}
%    \end{macrocode}

%\iffalse
%</samplechap1|samplechap2>
%\fi
%
%\iffalse
%<*samplechap1>
%\fi
% Some text for chapter 1:
%    \begin{macrocode}
\section{one}
some text in chapter one
%    \end{macrocode}

%\iffalse
%</samplechap1>
%\fi
% Some text for chapter 2:
%\iffalse
%<*samplechap2>
%\fi
%    \begin{macrocode}
\section{two}
more text in chapter two
%    \end{macrocode}

%\iffalse
%</samplechap2>
%\fi
%
% %%%%%%%%%%%%%%%%%%%%%%%%%%%%%%%%%%%%%%
% \paragraph{Part Include Files.}
%
% The include files are called |cdocspt3.tex| and |cdocspt4.tex|.
%
%\iffalse
%<*samplepart3|samplepart4>
%\fi

% Optional override for |\version| flag:
%    \begin{macrocode}
%%\providecommand{\version}{final}
%    \end{macrocode}

% Include the main document:
%    \begin{macrocode}
\input{childdoc.def}
\childdocby{cdocsamp}
%    \end{macrocode}

%\iffalse
%</samplepart3|samplepart4>
%\fi
%
%\iffalse
%<*samplepart3>
%\fi
% Some text for part 3:
%    \begin{macrocode}
some text in part three
%    \end{macrocode}

%\iffalse
%</samplepart3>
%\fi
% Some text for part 4:
%\iffalse
%<*samplepart4>
%\fi
%    \begin{macrocode}
more text in part four
%    \end{macrocode}

%\iffalse
%</samplepart4>
%\fi
%
% %%%%%%%%%%%%%%%%%%%%%%%%%%%%%%%%%%%%%%
% \paragraph{Forwarding for a Complete Draft.}
%
% The following forwarding file |cdocsdrf.tex|
% compiles the main document in draft mode:
%\iffalse
%<*sampledraft>
%\fi
%    \begin{macrocode}
\def\version{draft}
\input{childdoc.def}
\childdocforward{cdocsamp}
%    \end{macrocode}

%\iffalse
%</sampledraft>
%\fi
%
% %%%%%%%%%%%%%%%%%%%%%%%%%%%%%%%%%%%%%%
% \paragraph{Forwarding for Final Version of the Chapters.}
%
% The following forwarding files |cdocsfn1.tex| and |cdocsfn2.tex|
% (with identical content)
% compile the final versions of the child documents
% |cdocsch1.tex| and |cdocsch2.tex|, respectively:
%\iffalse
%<*samplefinal>
%\fi
%    \begin{macrocode}
\def\version{final}
\input{childdoc.def}
\childdocforwardprefix[cdocsamp]{cdocsfn}{cdocsch}
%    \end{macrocode}

%\iffalse
%</samplefinal>
%\fi
%
% %%%%%%%%%%%%%%%%%%%%%%%%%%%%%%%%%%%%%%
% \paragraph{Command Line Processing.}
%
% The following three command lines generate the output files
% |cdocscld|, |cdocscl1| and |cdocscl2|
% which should be identical to
% |cdocsdrf|, |cdocsch1| and |cdocsfn2|, respectively:
% \begin{center}
% \begin{tabular}{l}
% |latex -jobname cdocscld \|\\
% |  "\def\version{draft}\input{childdoc.def}\childdocforward{cdocsamp}"|\\
% |latex -jobname cdocscl1 \|\\
% |  "\input{childdoc.def}\childdocforward[cdocsamp]{cdocsch1}"|\\
% |latex -jobname cdocscl2 \|\\
% |  "\def\version{final}\input{childdoc.def}\childdocforward{cdocsch2}"|
% \end{tabular}
% \end{center}
% Note that the trailing backslash on each first line
% merely continues the input to the second line
% (for convenient cut ant paste).
% Furthermore, the command |latex| can be replaced by any
% of its alternative versions such as |pdflatex|.
%
% %%%%%%%%%%%%%%%%%%%%%%%%%%%%%%%%%%%%%%%%%%%%%%%%%%%%%%%%%%%%%%%%%%%%%%%%%%%%%%
% %%%%%%%%%%%%%%%%%%%%%%%%%%%%%%%%%%%%%%%%%%%%%%%%%%%%%%%%%%%%%%%%%%%%%%%%%%%%%%
% \section{Implementation}
%\iffalse
%<*package>
%\fi
%
% This section describes the definitions file |childdoc.def|.

% The definitions cannot be loaded using |\usepackage| or |\RequirePackage|
% which has a mechanism to prevent loading a style file more than once.
% When loading the definitions by means of |\input|
% multiple instances have to be prevented manually:
%\iffalse
%This code needs to be before the `\ProvidesFile' directive
%which is defined at the beginning of this file.
%Therefore it is also placed there and commented out here.
%</package>
%<*discard>
%\fi
%    \begin{macrocode}
\ifdefined\childdocmain\endinput\fi
%    \end{macrocode}
%\iffalse
%</discard>
%<*package>
%\fi
%
% \macro{\ifchilddoc}
% \macro{\ifchilddocmanual}
% The conditional |\ifchilddoc| tells whether a
% child (true) or main (false) document is being compiled.
% The conditional |\ifchilddocmanual| tells whether
% the |\includeonly| mechanism is used (false) or
% the selection of child files must be performed manually (true).
% The definitions initialise to false:
%    \begin{macrocode}
\newif\ifchilddoc
\newif\ifchilddocmanual
%    \end{macrocode}

% \macro{\childdocname}
% \macro{\childdocjob}
% The macro |\childdocname| stores the name of the main document
% to be compiled. The macro |\childdocjob| stores the name of
% the document on which the \LaTeX{} compiler was originally invoked.
% The content of |\jobname| cannot be compared
% to filenames specified in the source due to different catcodes.
% The following code rescans |\jobname|, stores the result
% in |\childdocname| and saves a copy in |\childdocjob|:
%    \begin{macrocode}
\edef\childdocname{\scantokens\expandafter{\jobname\noexpand}}
\let\childdocjob\childdocname
%    \end{macrocode}

% \macro{\childdocdisable}
% The macro |\childdocdisable| prevents the main file
% from being processed more than once.
% At this stage, the main document command |\childdocmain|
% is assumed to be called once again where it should do nothing.
% Any subsequent call to it should prevent
% a secondary processing of the main document
% It overwrites the forwarding commands
% |\childdocof| and |\childdocforward|
% with empty macros to prevent further inclusions of the main document:
%    \begin{macrocode}
\newcommand{\childdocdisable}
{
  \renewcommand{\childdocmain}[1]{\renewcommand{\childdocmain}[1]{\endinput}}
  \renewcommand{\childdocof}[1]{}
  \renewcommand{\childdocby}[2][]{}
  \renewcommand{\childdocforward}[2][]{}
  \renewcommand{\childdocdisable}{}
}
%    \end{macrocode}

% \macro{\childdocmain}
% The macro |\childdocmain| is to be called at the top of the main file
% with nothing or the main filename (without extension) as argument.
% First, it breaks loops.
% If the argument is not empty and does not match |\childdocname|
% (which is set by the first inclusion of |childdoc.def|),
% |\ifchilddoc| is set to true, |\includeonly| is applied to the child file
% and |\jobname| is set to the main file
% (for proper handling of |.aux| files):
%    \begin{macrocode}
\newcommand{\childdocmain}[1]
{
  \childdocdisable\childdocmain{}
  \if?#1?\else
    \begingroup
      \def\childdoctmp{#1}
      \ifx\childdoctmp\childdocname
        \def\childdoctmp{}
      \else
        \def\childdoctmp
        {
          \childdoctrue
          \includeonly{\childdocname}
          \def\childdocjob{#1}
          \def\jobname{#1}
        }
      \fi
      \expandafter
    \endgroup
    \childdoctmp
  \fi
}
%    \end{macrocode}

% \macro{\childdocof}
% The command |\childdocof| redirects
% compilation to the main file |#1|.
%    \begin{macrocode}
\newcommand{\childdocof}[1]
{
  \childdocdisable
  \childdoctrue
  \includeonly{\childdocname}
  \def\jobname{#1}
  \def\childdocjob{#1}
  \input{#1}
}
%    \end{macrocode}

% \macro{\childdocby}
% The command |\childdocby| ....
%    \begin{macrocode}
\newcommand{\childdocby}[2][]
{
  \childdocdisable
  \childdoctrue
  \childdocmanualtrue
  \if?#1?\else
    \def\jobname{#2}
  \fi
  \def\childdocjob{#2}
  \input{#2}
  \endinput
}
%    \end{macrocode}

% \macro{\childdocforward}
% The command |\childdocforward| redirects
% compilation to the main file or
% (if the optional argument is given) a child file.
% Parameters are set as if the main file
% or a child file starting with |\childdocof| was compiled.
% Then compilation is handed over to the main file:
%    \begin{macrocode}
\newcommand{\childdocforward}[2][]
{
  \begingroup
    \if?#1?
      \def\childdoctmp
      {
        \def\childdocname{#2}
        \def\childdocjob{#2}
        \def\jobname{#2}
        \input{#2}
        \endinput
      }
    \else
      \def\childdoctmp
      {
        \childdocdisable
        \def\childdocname{#2}
        \childdoctrue
        \includeonly{#2}
        \def\childdocjob{#1}
        \def\jobname{#1}
        \input{#1}
        \endinput
      }
    \fi
    \expandafter
  \endgroup
  \childdoctmp
}
%    \end{macrocode}

% \macro{\childdocforwardprefix}
% The command |\childdocforwardprefix| redirects
% compilation to the main or a child file by means of a pattern.
% The prefix |#1| in the current filename is replaced by |#2|
% and the suffix of the current filename is kept
% (it is assumed that the filename does not contain the substring `|~~~|'
% which is used as a delimiter).
% Compilation is handed over to the new file by |\childdocforward|:
%    \begin{macrocode}
\newcommand{\childdocforwardprefix}[3][]
{
  \begingroup
    \def\childdocextract #2##1~~~{\def\childdoctmp{\childdocforward[#1]{#3##1}}}
    \expandafter\childdocextract\childdocname~~~
    \expandafter
  \endgroup
  \childdoctmp
}
%    \end{macrocode}

% \macro{\childdoc}
% The deprecated macro |\childdoc| is a legacy version of |\childdocmain|:
%    \begin{macrocode}
\newcommand{\childdoc}{\childdocmain}
%    \end{macrocode}

% \macro{\childdocredirect}
% The deprecated macro |\childdocredirect| is a legacy version
% of |\childdocforward| and |\childdocforwardprefix|:
%    \begin{macrocode}
\newcommand{\childdocredirect}[2][]
{
  \begingroup
    \if?#1?
      \def\childdoctmp{\childdocforward{#2}}
    \else
      \def\childdoctmp{\childdocforwardprefix{#1}{#2}}
    \fi
    \expandafter
  \endgroup
  \childdoctmp
}
%    \end{macrocode}

%\iffalse
%</package>
%\fi
%
\endinput
\childdocforward[cdocsamp]{cdocsch1}"|\\
% |latex -jobname cdocscl2 \|\\
% |  "\def\version{final}% \iffalse
%
% childdoc.dtx Copyright (C) 2017-2018 Niklas Beisert
%
% This work may be distributed and/or modified under the
% conditions of the LaTeX Project Public License, either version 1.3
% of this license or (at your option) any later version.
% The latest version of this license is in
%   http://www.latex-project.org/lppl.txt
% and version 1.3 or later is part of all distributions of LaTeX
% version 2005/12/01 or later.
%
% This work has the LPPL maintenance status `maintained'.
%
% The Current Maintainer of this work is Niklas Beisert.
%
% This work consists of the files childdoc.dtx and childdoc.ins
% and the derived files childdoc.def and cdocsamp.tex with
% cdocsch1.tex, cdocsch2.tex, cdocsdrf.tex, cdocsfn1.tex, cdocsfn2.tex.
%
%<package>\ifdefined\childdocmain\endinput\fi
%<package>\ProvidesFile{childdoc.def}[2018/12/30 v2.0 child document driver]
%<samplemain>\ProvidesFile{cdocsamp.tex}[2018/12/30 v2.0 sample for childdoc]
%<*driver>
%\ProvidesFile{childdoc.drv}[2018/12/30 v2.0 childdoc reference manual file]
\PassOptionsToClass{10pt,a4paper}{article}
\documentclass{ltxdoc}

\usepackage[margin=35mm]{geometry}
\usepackage{hyperref}
\usepackage{hyperxmp}
\usepackage[usenames]{color}

\hypersetup{colorlinks=true}
\hypersetup{pdfstartview=FitH}
\hypersetup{pdfpagemode=UseNone}
\hypersetup{pdfsource={}}
\hypersetup{pdflang={en-UK}}
\hypersetup{pdfcopyright={Copyright 2017-2018 Niklas Beisert.
  This work may be distributed and/or modified under the
  conditions of the LaTeX Project Public License, either version 1.3
  of this license or (at your option) any later version.}}
\hypersetup{pdflicenseurl={http://www.latex-project.org/lppl.txt}}
\hypersetup{pdfcontactaddress={ETH Zurich, ITP, HIT K,
  Wolfgang-Pauli-Strasse 27}}
\hypersetup{pdfcontactpostcode={8093}}
\hypersetup{pdfcontactcity={Zurich}}
\hypersetup{pdfcontactcountry={Switzerland}}
\hypersetup{pdfcontactemail={nbeisert@itp.phys.ethz.ch}}
\hypersetup{pdfcontacturl={http://people.phys.ethz.ch/\xmptilde nbeisert/}}

\newcommand{\secref}[1]{\hyperref[#1]{section \ref*{#1}}}

\parskip1ex
\parindent0pt
\let\olditemize\itemize
\def\itemize{\olditemize\parskip0pt}

\begin{document}

\title{The \textsf{childdoc} Package}
\hypersetup{pdftitle={The childdoc Package}}
\author{Niklas Beisert\\[2ex]
  Institut f\"ur Theoretische Physik\\
  Eidgen\"ossische Technische Hochschule Z\"urich\\
  Wolfgang-Pauli-Strasse 27, 8093 Z\"urich, Switzerland\\[1ex]
  \href{mailto:nbeisert@itp.phys.ethz.ch}
  {\texttt{nbeisert@itp.phys.ethz.ch}}}
\hypersetup{pdfauthor={Niklas Beisert}}
\hypersetup{pdfsubject={Manual for the LaTeX2e Package childdoc}}
\date{30 December 2018, \textsf{v2.0}}
\maketitle

\begin{abstract}\noindent
\textsf{childdoc} is a \LaTeXe{} package
that enables the direct compilation
of document sections included by |\include|
to individual files.
\end{abstract}

\begingroup
\parskip0ex
\tableofcontents
\endgroup

%%%%%%%%%%%%%%%%%%%%%%%%%%%%%%%%%%%%%%%%%%%%%%%%%%%%%%%%%%%%%%%%%%%%%%%%%%%%%%%%
%%%%%%%%%%%%%%%%%%%%%%%%%%%%%%%%%%%%%%%%%%%%%%%%%%%%%%%%%%%%%%%%%%%%%%%%%%%%%%%%
\section{Introduction}

\LaTeX{} provides a mechanism to structure a large document (such as a book)
into a main file and several child files (containing the chapters)
using the |\include| command.
This mechanism is beneficial for documents
which span hundreds of pages in order to
make the source file(s) more manageable.
Moreover, compilation can be restricted to
selected child files by means of the |\includeonly| command.
The latter feature can be used to reduce the compilation time while editing
(this was significantly more useful in the earlier days of \LaTeX{})
or to generate a smaller document which is easier to navigate.
Another application of |\includeonly| is to generate
documents consisting of selected parts of the complete document.

However, there are a few drawbacks of the plain |\include| mechanism:
\begin{itemize}
\item
The child files cannot be compiled on their own,
they can only be compiled via the main file.
A naive editing environment
(such as a text editor with an option
to have the current file processed by \LaTeX)
may require one to switch to the main file before compiling;
attempting to compile the child file produces errors.
\item
The main file must be modified (each time)
to adjust the |\includeonly| command
to the present needs. This easily leaves the main file in a messy state.
\item
The generated document will always carry the filename
of the main document. This is inconvenient if
several child files are to be compiled and
to be kept for distribution.
\end{itemize}

The present package provides a simple interface
to make child files individually compilable by \LaTeX{}.
Compiling a child file then has the same effect as compiling
the main file with an |\includeonly| command
to select the appropriate child.
Moreover the generated document will carry the name of the child
rather than the main file.
This resolves all three above issues.

This feature is meant to make the editing of books,
thesis documents and lecture notes somewhat more convenient.
However, the package can also be used efficiently for
composing a series of documents (such as exercise sheets)
which are typically distributed individually.
It then assists the author in generating the individual documents
(potentially in different versions)
as well as a document containing the collected series.
Another application is in developing style files
or other kinds of included material
where compilation of the style file could redirect
to a sample or test file.

%%%%%%%%%%%%%%%%%%%%%%%%%%%%%%%%%%%%%%%%%%%%%%%%%%%%%%%%%%%%%%%%%%%%%%%%%%%%%%%%
%%%%%%%%%%%%%%%%%%%%%%%%%%%%%%%%%%%%%%%%%%%%%%%%%%%%%%%%%%%%%%%%%%%%%%%%%%%%%%%%
\section{Usage}

First of all, the package \textsf{childdoc} is \emph{not} a standard
\LaTeXe{} |.sty| style file! Therefore it needs to be invoked in
a non-standard way.

%%%%%%%%%%%%%%%%%%%%%%%%%%%%%%%%%%%%%%%%%%%%%%%%%%%%%%%%%%%%%%%%%%%%%%%%%%%%%%%%
\subsection{Included Files}
\label{sec:include}

%%%%%%%%%%%%%%%%%%%%%%%%%%%%%%%%%%%%%%%%
\DescribeMacro{\childdocmain}
To use the package, add the commands
\begin{center}
\begin{tabular}{l}
|\input{childdoc.def}|\\
|\childdocmain{}|\\
\end{tabular}
\end{center}
at the very top of the main \LaTeX{} file,
in particular \emph{before} the |\documentclass| statement!
The argument of |\childdocmain| should be left empty
(but it must be present).

%%%%%%%%%%%%%%%%%%%%%%%%%%%%%%%%%%%%%%%%
\DescribeMacro{\childdocof}
Furthermore, add the commands
\begin{center}
\begin{tabular}{l}
|\input{childdoc.def}|\\
|\childdocof{|\textit{main}|}|\\
\end{tabular}
\end{center}
at the top of every child file \textit{child}
which is included by |\include{|\textit{child}|}|
from within the main file
(or at least for those files to be compiled individually).
The argument \textit{main} must be the filename of the main file.

There are a couple of
considerations in setting up the main and child documents:

%%%%%%%%%%%%%%%%%%%%%%%%%%%%%%%%%%%%%%%%
\paragraph{Restrictions.}

Please note the following restrictions:
\begin{itemize}
\item
|\childdocmain| must be called with one argument \textit{main}
to ensure compatibility with earlier version of the package.
It must either be empty (|\childdocmain{}|)
or precisely match the filename of the main file in which it is specified.
See \secref{sec:detection} for further information.
\item
The filename \textit{main} must be specified without the |.tex| extension.
\item
The filename \textit{main} is case sensitive
(even in case-insensitive file systems)
due to internal string comparison.
\item
The argument \textit{main} should be fully expanded, it cannot be a macro.
\item
Subdirectories and special characters should be avoided in filenames.
\item
The command |\childdocmain{|\textit{main}|}| must be followed by a whitespace.
It should not be followed immediately by another command
or by a comment mark `|%|'.
This is because the \TeX{} parser reads the token immediately following
the argument of |\childdocmain| and puts it
at the beginning of every child section;
however, a white\-space is ignored.
\end{itemize}

%%%%%%%%%%%%%%%%%%%%%%%%%%%%%%%%%%%%%%%%
\paragraph{Content of Main File.}

It is advisable to place all content in the child files included by |\include|.
Any output contained in the main file will appear in all child documents
unless suppressed manually;
it cannot be suppressed automatically by the |\includeonly| directive
and thus should normally be avoided.
A method to include some content in the main file
by means of conditional processing is described in \secref{sec:conditional}.

%%%%%%%%%%%%%%%%%%%%%%%%%%%%%%%%%%%%%%%%
\paragraph{Page Numbering.}

When only a part of the document is compiled,
the appropriate numbering of pages
(as well as other status parameters)
is determined from the |.aux| files.
The latter contain information from previous passes.
However this information needs to propagate through
all intermediate child documents.
Therefore the page numbering in child documents may well
be inconsistent until the complete document is compiled at least once.

A useful (if unconventional) way to always ensure a consistent
page numbering is to restart the numbering in each child document
and denote the pages by `\textit{child}|.|\textit{page}'
where \textit{child} represents the chapter/section number of the child file.
This can be achieved by the command
|\numberwithin{page}{|\textit{child}|}|
of the \textsf{amsmath} package
where \textit{child} can be |chapter| or |section|
depending on the chosen structuring.
Alternatively, one can modify the macro |\thepage| appropriately
and reset the counter |page| at the start of each child file.

%%%%%%%%%%%%%%%%%%%%%%%%%%%%%%%%%%%%%%%%%%%%%%%%%%%%%%%%%%%%%%%%%%%%%%%%%%%%%%%%
\subsection{Conditional Processing}
\label{sec:conditional}

The package provides a mechanism to compile different versions
of a document. To customise the versions further some conditional processing
can come in handy to distinguish which version is being compiled.
The package provides two macros to describe the compilation context:

%%%%%%%%%%%%%%%%%%%%%%%%%%%%%%%%%%%%%%%%
\DescribeMacro{\ifchilddoc}
The conditional |\ifchilddoc| distinguishes between the compilation of
child documents and the main document:
%
\begin{center}
|\ifchilddoc |\textit{child-code}| |[|\||else |\textit{main-code}]| \||fi|
\end{center}

%%%%%%%%%%%%%%%%%%%%%%%%%%%%%%%%%%%%%%%%
\DescribeMacro{\childdocname}
\DescribeMacro{\childdocjob}
The macro |\childdocname| contains the filename (without extension)
of the main or child file being processed.
Note that |\childdocjob| will always contain the name of the main file.

%%%%%%%%%%%%%%%%%%%%%%%%%%%%%%%%%%%%%%%%
\paragraph{Title Page.}

Conditional processing can be used to include a title or banner page
in the main document when proper precautions are taken.
Importantly, the code in the main file should ensure that the page counter
(as well as other status parameters which are stored in the |.aux| files)
takes the same value after the conditional processing.
Otherwise the page numbers may take divergent values
depending on which part is compiled.

For example, a title page could be declared by:
%
\begin{center}
\begin{tabular}{l}
|\ifchilddoc\||else|\\
|\addtocounter{page}{-1}|\\
\textit{code for title page}\\
|\newpage|\\
|\||fi|
\end{tabular}
\end{center}
%
A banner page for the child documents can be generated by:
%
\begin{center}
\begin{tabular}{l}
|\ifchilddoc|\\
|\addtocounter{page}{-1}|\\
\textit{code for banner page}\\
|\newpage|\\
|\||fi|
\end{tabular}
\end{center}
%
Here one could write a message such as:
\begin{center}
|This is the part \childdocname{} of \childdocjob{}.|
\end{center}

%%%%%%%%%%%%%%%%%%%%%%%%%%%%%%%%%%%%%%%%%%%%%%%%%%%%%%%%%%%%%%%%%%%%%%%%%%%%%%%%
\subsection{Flags}
\label{sec:flags}

The package makes it easy to generate different versions
of the main or child documents.
To this end compilation flags can be defined
and assigned different default values.
They will be particularly useful in conjunction
with the forwarding mechanism described in \secref{sec:forward}.

For example, it may be useful to have a flag |\version|
which can be set to |draft| or |final|.
The document source will contain some conditional code
depending on the value of |\version|.
Suppose further, the flag should default to |final| for the main file
and to |draft| for child files
which is a natural assignment for editing the document.
This is achieved by placing the following code
in the preamble of the main document
(below the |\childdocmain| directive):
%
\begin{center}
\begin{tabular}{l}
|\ifchilddoc|\\
|\providecommand{\version}{draft}|\\
|\||else|\\
|\providecommand{\version}{final}|\\
|\||fi|
\end{tabular}
\end{center}
%
The definition by |\providecommand| makes sure
that previous definitions are not overwritten.
Further statements |\providecommand{\version}{...}|
can thus be added before the above code to override it.

For the main file, one might add a line
(between |\childdocmain| and the above block)
%
\begin{center}
|%\ifchilddoc\||else\providecommand{\version}{draft}\||fi|
\end{center}
%
which can be uncommented to produce a draft version.
Likewise one can add a line to the very top of a child file
(above the |\childdocof{|\textit{main}|}| directive)
%
\begin{center}
|%\providecommand{\version}{final}|
\end{center}
%
which can be uncommented to produce the final version of this child document.

%%%%%%%%%%%%%%%%%%%%%%%%%%%%%%%%%%%%%%%%%%%%%%%%%%%%%%%%%%%%%%%%%%%%%%%%%%%%%%%%
\subsection{Forwarding}
\label{sec:forward}

Different versions of the main or child documents
using compilation flags as described in \secref{sec:flags}
can be (permanently) stored in different files
for convenient compilation, viewing and distribution.
To this end, the package defines a command
to pass on compilation to a different file:

%%%%%%%%%%%%%%%%%%%%%%%%%%%%%%%%%%%%%%%%
\DescribeMacro{\childdocforward}
The command |\childdocforward| redirects processing to
another source file:
%
\begin{center}
\begin{tabular}{l}
|\input{childdoc.def}|\\
|\childdocforward[|\textit{main}|]{|\textit{dest}|}|\\
\end{tabular}
\end{center}
%
The argument \textit{dest} is the destination file
(without extension).
It should be the main file or one of the child files.
Note that further \textsf{childdoc} directives
such as |\childdocof| and |\childdocforward|
in the indicated file will be processed in this form.
The optional argument \textit{main}
passes on directly to the main file \textit{main}
while pretending to compile the child \textit{dest}.
This form behaves as if \textit{dest}
issues |\childdocof{|\textit{main}|}| right away,
and no further \textsf{childdoc} directives will be processed.

%%%%%%%%%%%%%%%%%%%%%%%%%%%%%%%%%%%%%%%%
\DescribeMacro{\...prefix}
In the alternative form |\childdocforwardprefix|,
%
\begin{center}
\begin{tabular}{l}
|\input{childdoc.def}|\\
|\childdocforwardprefix[|\textit{main}|]{|\textit{prefix}|}{|\textit{dest}|}|
\end{tabular}
\end{center}
%
the destination file is determined by a pattern
depending on the current file:
To make this work, the current file must be called
`{\textit{prefix}\hspace{0.2em}\textit{suffix}}'
with \textit{prefix} matching precisely the argument.
Processing is then passed on to the file
`{\textit{dest}\hspace{0.2em}\textit{suffix}}'.
Surely, the same effect is achieved by
directly specifying the
argument `{\textit{dest}\hspace{0.2em}\textit{suffix}}'
in the first form.
However, that requires to set up a different file
for each child. With the alternative form of the command
all these files can have exactly the same content
which simplifies setting them up and maintaining them.

For example, the following file |draft.tex|
with a compilation flag |\version| as described in \secref{sec:flags}
compiles the main document as a draft:
%
\begin{center}
\begin{tabular}{l}
|\def\version{draft}|\\
|\input{childdoc.def}|\\
|\childdocforward{|\textit{main}|}|
\end{tabular}
\end{center}
%
Likewise, the following files |final|\textit{nn}|.tex|
compile the final version of the child document
|child|\textit{nn}|.tex|:
%
\begin{center}
\begin{tabular}{l}
|\def\version{final}|\\
|\input{childdoc.def}|\\
|\childdocforwardprefix{final}{child}|
\end{tabular}
\end{center}
%

Note that when several versions of a main file and/or of each child file
are to be generated, it may be convenient to set up a |Makefile| or
shell script to automatise the process.

%%%%%%%%%%%%%%%%%%%%%%%%%%%%%%%%%%%%%%%%%%%%%%%%%%%%%%%%%%%%%%%%%%%%%%%%%%%%%%%%
\subsection{Command Line Processing}
\label{sec:commandline}

The effect of redirection files can also be achieved by invoking
the \LaTeX{} compiler with a more elaborate command line.
Most conveniently this should be done as part
of a shell script or a |Makefile|.

When using \textsf{childdoc} in the main file, the following
command lines effectively perform a redirection
(note that depending on the shell being used,
backslashes may have to be doubled: `|\|' $\to$ `|\\|'):
%
\begin{center}
|... -jobname "|\textit{target}|" |\\|"|[\textit{flags}]%
|\input{childdoc.def}\childdocforward[|\textit{main}|]{|\textit{dest}|}"|
\end{center}
%
Here \textit{target} is the name of the output file,
\textit{main} is the name of the main file
and \textit{dest} is the name of the main or child file to be processed
(all filenames without extensions).
The optional argument \textit{main} can be omitted
if \textit{main} matches \textit{dest}.
Optionally, compilation \textit{flags} can be defined via |\def| commands.
This command line makes the \TeX{} engine believe
it is compiling the file \textit{target}
whose content is specified as the latter parameter.
The provided code then forwards the processing to
\textit{main} or \textit{dest} as described in \secref{sec:forward}.

%%%%%%%%%%%%%%%%%%%%%%%%%%%%%%%%%%%%%%%%%%%%%%%%%%%%%%%%%%%%%%%%%%%%%%%%%%%%%%%%
\subsection{Include by Input}
\label{sec:input}

Including child documents by |\include| has some restrictions by design.
Most notably, the content of a child document always occupies
its own set of pages; pages cannot be shared between child documents.
Usually, this behaviour makes perfect sense
because each child document contain an essential part of the document.
However, in some situations it may be desirable to compose
a document from a collection of parts
without having mandatory page breaks between then.
For this case, the package
provides a mechanism to include parts
by |\input| which can also be processed individually.
However, by construction this mechanism
requires manual handling of the content to be output.

%%%%%%%%%%%%%%%%%%%%%%%%%%%%%%%%%%%%%%%%
\DescribeMacro{\ifchilddocmanual}
The main file should be prepared as usual, see \secref{sec:include}.
However, the document body must make a distinction
between processing of an individual part and of the main document, e.g.:
%
\begin{center}
\begin{tabular}{l}
|\ifchilddocmanual|\\
|\input{\childdocname}|\\
|\||else|\\
\textit{document body with }|\input{|\textit{part}|}|\\
|\||fi|
\end{tabular}
\end{center}
%
The conditional |\ifchilddocmanual| is true whenever
a part to be included by |\input| is being compiled,
and the name of the part is stored in |\childdocname|.

%%%%%%%%%%%%%%%%%%%%%%%%%%%%%%%%%%%%%%%%
\DescribeMacro{\childdocby}
Each part to be included by |\input| should start with:
%
\begin{center}
\begin{tabular}{l}
|\input{childdoc.def}|\\
|\childdocby{|\textit{main}|}|\\
\end{tabular}
\end{center}
%
The directive |\childdocby| is similar to |\childdocof|
described in \secref{sec:include},
but the subsequent selection of content must be done manually.
To that end, both |\ifchilddoc| and |\ifchilddocmanual|
will be true upon processing of a part,
and the name of the part is stored in |\childdocname|.
Note that |\jobname| will be set to the filename of the current part
so that each part receives an individual |.aux| file
that does not interfere with the |.aux| file(s) of the main document.
This behaviour can be altered by the alternative form
|\childdocby[*]{|\textit{main}|}| (with a non-empty optional argument)
which uses the |.aux| file of the main document
by setting |\jobname| to \textit{main}.

%%%%%%%%%%%%%%%%%%%%%%%%%%%%%%%%%%%%%%%%%%%%%%%%%%%%%%%%%%%%%%%%%%%%%%%%%%%%%%%%
\subsection{Driver Development}
\label{sec:driver}

The \textsf{childdoc} mechanism can also be use for the development
of definition files such as \LaTeX{} styles or classes.
This case differs from the above setup with multiple parts
included by |\include| in that no |\includeonly| should be invoked.
This can be achieved by starting the include file
(before |\ProvidesPackage|) with:
%
\begin{center}
\begin{tabular}{l}
|\input{childdoc.def}|\\
|\childdocforward{|\textit{main}|}|\\
\end{tabular}
\end{center}
%
or alternatively with:
%
\begin{center}
\begin{tabular}{l}
|\input{childdoc.def}|\\
|\childdocby{|\textit{main}|}|\\
\end{tabular}
\end{center}
%
Both forms have slightly different effects as described above.
The main file is prepared as usual, see \secref{sec:include}.

%%%%%%%%%%%%%%%%%%%%%%%%%%%%%%%%%%%%%%%%%%%%%%%%%%%%%%%%%%%%%%%%%%%%%%%%%%%%%%%%
\subsection{Legacy Detection}
\label{sec:detection}

The directive |\childdocmain| in the main file can detect
whether the complete document or merely a child is to be compiled
even without using the directive |\childdocof|.
This method is deprecated because it is less robust
and there is no compelling reason to use it;
it is merely provided for backward compatibility
and it may be removed in future versions.

If the detection mechanism is to be used,
it is mandatory to correctly specify
the filename of the main file as the argument of |\childdocmain|:
%
\begin{center}
\begin{tabular}{l}
|\input{childdoc.def}|\\
|\childdocmain{|\textit{main}|}|\\
\end{tabular}
\end{center}
%
If |\jobname| does not match the argument \textit{main} of |\childdocmain|,
it is assumed that |\jobname| points to the child file to be compiled.
When using |\childdocmain| with the main file specified as argument,
it suffices to start a child file
with just |\input{|\textit{main}|}|
without loading of the package and using |\childdocof|.
If instead all processing is done
with the appropriate \textsf{childdoc} directives,
the argument of \textit{main} of |\childdocmain| can be empty.

An alternative version of the command line processing described
in \secref{sec:commandline} using the detection mechanism reads:
%
\begin{center}
|... -jobname "|\textit{target}|" "|[\textit{flags}]%
[|\def\jobname{|\textit{dest}|}|]|\input{|\textit{main}|}"|
\end{center}

%%%%%%%%%%%%%%%%%%%%%%%%%%%%%%%%%%%%%%%%%%%%%%%%%%%%%%%%%%%%%%%%%%%%%%%%%%%%%%%%
\subsection{Manual Code}
\label{sec:manual}

In case one cannot be certain whether the definitions file |childdoc.def|
is installed on the target \TeX{} distribution
and one prefers not to ship it,
it is conceivable to paste a few relevant commands into the sources.

To that end, drop all statements |\input{childdoc.def}|
and perform the replacements as outlined below.
Instead of |\childdocmain{|\textit{main}|}| add the following code
to the top of the main file:
%
\begin{center}
\begin{tabular}{l}
|\||ifdefined\childdocname\endinput\||fi\newif\ifchilddoc|\\
|\edef\childdocname{\scantokens\expandafter{\jobname\noexpand}}|\\
|\def\childdocmain{|\textit{main}|}\||ifx\childdocmain\childdocname\||else|\\
|\childdoctrue\includeonly{\childdocname}\let\jobname\childdocmain\||fi|\\
\end{tabular}
\end{center}
%
Instead of |\childdocof{|\textit{main}|}| just include the main file
at the top of each child file:
%
\begin{center}
|\input{|\textit{main}|}|
\end{center}
%
A simple redirection |\childdocforward{|\textit{dest}|}| is achieved by:
%
\begin{center}
|\def\jobname{|\textit{dest}|}\input{\jobname}|
\end{center}
%
The redirection with prefix
|\childdocforwardprefix[|\textit{prefix}|]{|\textit{dest}|}|
is accomplished by:
%
\begin{center}
\begin{tabular}{l}
|{\edef\jobname{\scantokens\expandafter{\jobname\noexpand}}|\\
|\def\redirectjob |\textit{prefix}|#1~~~{\gdef\jobname{|\textit{dest}|#1}}|\\
|\expandafter\redirectjob\jobname~~~}\input{\jobname}|
\end{tabular}
\end{center}

In an alternative approach,
child documents can be compiled by a specific command line
without additional code or specific definitions:
%
\begin{center}
|... -jobname "|\textit{target}|" "|[\textit{flags}]%
|\includeonly{|\textit{dest}|}\input{|\textit{main}|}"|
\end{center}
%

%%%%%%%%%%%%%%%%%%%%%%%%%%%%%%%%%%%%%%%%%%%%%%%%%%%%%%%%%%%%%%%%%%%%%%%%%%%%%%%%
%%%%%%%%%%%%%%%%%%%%%%%%%%%%%%%%%%%%%%%%%%%%%%%%%%%%%%%%%%%%%%%%%%%%%%%%%%%%%%%%
\section{Information}

%%%%%%%%%%%%%%%%%%%%%%%%%%%%%%%%%%%%%%%%%%%%%%%%%%%%%%%%%%%%%%%%%%%%%%%%%%%%%%%%
\subsection{Copyright}

Copyright \copyright{} 2017--2018 Niklas Beisert

This work may be distributed and/or modified under the
conditions of the \LaTeX{} Project Public License, either version 1.3
of this license or (at your option) any later version.
The latest version of this license is in
  \url{http://www.latex-project.org/lppl.txt}
and version 1.3 or later is part of all distributions of \LaTeX{}
version 2005/12/01 or later.

This work has the LPPL maintenance status `maintained'.

The Current Maintainer of this work is Niklas Beisert.

This work consists of the files |README.txt|, |childdoc.ins| and |childdoc.dtx|
as well as the derived files |childdoc.def|, |cdocsamp.tex|
with |cdocsch1.tex|, |cdocsch2.tex|, |cdocspt3.tex|, |cdocspt4.tex|,
|cdocsdrf.tex|, |cdocsfn1.tex|, |cdocsfn2.tex|
as well as |childdoc.pdf|.

%%%%%%%%%%%%%%%%%%%%%%%%%%%%%%%%%%%%%%%%%%%%%%%%%%%%%%%%%%%%%%%%%%%%%%%%%%%%%%%%
\subsection{Files and Installation}

The package consists of the files:
%
\begin{center}
\begin{tabular}{ll}
    |README.txt|   & readme file \\
    |childdoc.ins| & installation file \\
    |childdoc.dtx| & source file \\
    |childdoc.def| & definition file \\
    |cdocsamp.tex| & sample main file \\
    |cdocsch1.tex| & sample include file \\
    |cdocsch2.tex| & sample include file \\
    |cdocspt3.tex| & sample part file \\
    |cdocspt4.tex| & sample part file \\
    |cdocsdrf.tex| & sample redirection file \\
    |cdocsfn1.tex| & sample redirection file \\
    |cdocsfn2.tex| & sample redirection file \\
    |childdoc.pdf| & manual
\end{tabular}
\end{center}
%
The distribution consists of the files
|README.txt|, |childdoc.ins| and |childdoc.dtx|.
%
\begin{itemize}
\item
Run (pdf)\LaTeX{} on |childdoc.dtx|
to compile the manual |childdoc.pdf| (this file).
\item
Run \LaTeX{} on |childdoc.ins| to create the definitions file |childdoc.def|
and the sample |cdocsamp.tex| with include files
|cdocsch1.tex|, |cdocsch2.tex|, |cdocspt3.tex|, |cdocspt4.tex|,
|cdocsdrf.tex|, |cdocsfn1.tex|, |cdocsfn2.tex|.
Then copy the file |childdoc.def| to an appropriate directory of your \LaTeX{}
distribution, e.g.\ \textit{texmf-root}|/tex/latex/childdoc|.
\end{itemize}

%%%%%%%%%%%%%%%%%%%%%%%%%%%%%%%%%%%%%%%%%%%%%%%%%%%%%%%%%%%%%%%%%%%%%%%%%%%%%%%%
\subsection{Related CTAN Packages}

There are several other packages which offer a similar functionality:
%
\begin{itemize}
\item
The packages
\href{http://ctan.org/pkg/docmute}{\textsf{docmute}},
\href{http://ctan.org/pkg/includex}{\textsf{includex}} and
\href{http://ctan.org/pkg/standalone}{\textsf{standalone}}
provide commands to include only the document body of
a child file thus allowing both files to be compiled individually.
\item
The packages \href{http://ctan.org/pkg/subdocs}{\textsf{subdocs}}
and \href{http://ctan.org/pkg/subfiles}{\textsf{subfiles}}
provide structures in which the main and child documents can be
encapsulated and allowing them to be compiled individually.
The inclusion mechanism is different from the conventional |\include|.
\item
The package \href{http://ctan.org/pkg/combine}{\textsf{combine}}
is an elaborate solution to combine several documents into one.
\end{itemize}
%
See also the CTAN topic \href{http://ctan.org/topic/subdocs}{\textsf{subdocs}}
for further related packages.
The present package differs from the above solutions in that
a document structure constructed with the conventional |\include| mechanism
just needs two extra commands at the top of every file
such that all constituent files can be compiled individually.

%%%%%%%%%%%%%%%%%%%%%%%%%%%%%%%%%%%%%%%%%%%%%%%%%%%%%%%%%%%%%%%%%%%%%%%%%%%%%%%%
%\subsection{Feature Suggestions}
%
%The following is a list of features which may be useful for future
%versions of this package:
%%
%\begin{itemize}
%\item
%\ldots
%\end{itemize}

%%%%%%%%%%%%%%%%%%%%%%%%%%%%%%%%%%%%%%%%%%%%%%%%%%%%%%%%%%%%%%%%%%%%%%%%%%%%%%%%
\subsection{Revision History}

%%%%%%%%%%%%%%%%%%%%%%%%%%%%%%%%%%%%%%%%
\paragraph{v2.0:} 2018/12/30

\begin{itemize}
\item
immediate forward processing
\item
added |\childdocby| mechanism
\item
manual restructured
\end{itemize}

%%%%%%%%%%%%%%%%%%%%%%%%%%%%%%%%%%%%%%%%
\paragraph{v1.6:} 2018/01/17

\begin{itemize}
\item
application for development of include files
\item
corrections to manual
\end{itemize}

%%%%%%%%%%%%%%%%%%%%%%%%%%%%%%%%%%%%%%%%
\paragraph{v1.5:} 2017/05/21

\begin{itemize}
\item
more complete structuring introduced
\item
|\childdocof| introduced
\item
|\childdoc| renamed to |\childdocmain|
\item
|\childredirect| renamed to |\childdocforward| and |\childdocforwardprefix|
and functionality expanded
\end{itemize}

%%%%%%%%%%%%%%%%%%%%%%%%%%%%%%%%%%%%%%%%
\paragraph{v1.0:} 2017/04/27

\begin{itemize}
\item
manual and install package
\item
first version published on CTAN
\end{itemize}

%%%%%%%%%%%%%%%%%%%%%%%%%%%%%%%%%%%%%%%%
\paragraph{v0.6:} 2017/04/26

\begin{itemize}
\item
redirection mechanism added
\end{itemize}

%%%%%%%%%%%%%%%%%%%%%%%%%%%%%%%%%%%%%%%%
\paragraph{v0.5:} 2017/04/26

\begin{itemize}
\item
functionality in definition file
\end{itemize}


%%%%%%%%%%%%%%%%%%%%%%%%%%%%%%%%%%%%%%%%%%%%%%%%%%%%%%%%%%%%%%%%%%%%%%%%%%%%%%%%
%%%%%%%%%%%%%%%%%%%%%%%%%%%%%%%%%%%%%%%%%%%%%%%%%%%%%%%%%%%%%%%%%%%%%%%%%%%%%%%%
%%%%%%%%%%%%%%%%%%%%%%%%%%%%%%%%%%%%%%%%%%%%%%%%%%%%%%%%%%%%%%%%%%%%%%%%%%%%%%%%
\appendix

\settowidth\MacroIndent{\rmfamily\scriptsize 000\ }

 \DocInput{childdoc.dtx}

\end{document}
%</driver>
% \fi
%
% %%%%%%%%%%%%%%%%%%%%%%%%%%%%%%%%%%%%%%%%%%%%%%%%%%%%%%%%%%%%%%%%%%%%%%%%%%%%%%
% %%%%%%%%%%%%%%%%%%%%%%%%%%%%%%%%%%%%%%%%%%%%%%%%%%%%%%%%%%%%%%%%%%%%%%%%%%%%%%
% \section{Sample}
%\iffalse
%<*samplemain>
%\fi
%
% The following presents a sample document
% with two chapters, two parts, a title page,
% a compile flag as well as three forwarding files to set the flag.
% It consists of eight |.tex| files:
% \begin{center}
% \begin{tabular}{ll}
% |cdocsamp.tex|&main file\\
% |cdocsch1.tex|&include file for chapter 1\\
% |cdocsch2.tex|&include file for chapter 2\\
% |cdocspt3.tex|&include file for part 3\\
% |cdocspt4.tex|&include file for part 4\\
% |cdocsdrf.tex|&forwarding file for main file in draft mode\\
% |cdocsfi1.tex|&forwarding file for final version of chapter 1\\
% |cdocsfi2.tex|&forwarding file for final version of chapter 2\\
% \end{tabular}
% \end{center}
% Each of the eight files can be compiled directly by the \LaTeX{} compiler.
%
% %%%%%%%%%%%%%%%%%%%%%%%%%%%%%%%%%%%%%%
% \paragraph{Main File.}
%
% The main file is called |cdocsamp.tex|.
%
% Load the \textsf{childdoc} definitions and
% declare the filename for the main document:
%    \begin{macrocode}
\input{childdoc.def}
\childdocmain{}
%    \end{macrocode}

% Optional override for |\version| flag:
%    \begin{macrocode}
%%\ifchilddoc\else\providecommand{\version}{draft}\fi
%    \end{macrocode}

% Define the default values for the |\version| flag
% (|final| for the main file and |draft| for childs):
%    \begin{macrocode}
\ifchilddoc
\providecommand{\version}{draft}
\else
\providecommand{\version}{final}
\fi
%    \end{macrocode}

% Load the standard document class:
%    \begin{macrocode}
\documentclass[12pt]{article}
%    \end{macrocode}

% Start the document body:
%    \begin{macrocode}
\begin{document}
%    \end{macrocode}

% Declare a title page.
% Print title, part of document being processed and version flag:
%    \begin{macrocode}
\addtocounter{page}{-1}
\begin{center}
{\LARGE\bfseries{}childdoc example\par}
\vspace{1cm}
\ifchilddoc
\ifchilddocmanual part\else chapter\fi:
`\childdocname' of `\childdocjob'\par
\else
main document: `\childdocjob'\par
\fi
version: \version\par
\end{center}
\newpage
%    \end{macrocode}

% Manually include selected file,
% otherwise process as usual:
%    \begin{macrocode}
\ifchilddocmanual
\section*{part `\childdocname'}
\input{\childdocname}
\else
%    \end{macrocode}

% Include the two chapters:
%    \begin{macrocode}
\include{cdocsch1}
\include{cdocsch2}
%    \end{macrocode}

% Include the two parts unless only chapters should be displayed:
%    \begin{macrocode}
\ifchilddoc\else
\section{part three}
\input{cdocspt3}
\section{part four}
\input{cdocspt4}
\fi
%    \end{macrocode}

% Process as usual until here:
%    \begin{macrocode}
\fi
%    \end{macrocode}

% End of document body:
%    \begin{macrocode}
\end{document}
%    \end{macrocode}
%\iffalse
%</samplemain>
%\fi
%
% %%%%%%%%%%%%%%%%%%%%%%%%%%%%%%%%%%%%%%
% \paragraph{Chapter Include Files.}
%
% The include files are called |cdocsch1.tex| and |cdocsch2.tex|.
%
%\iffalse
%<*samplechap1|samplechap2>
%\fi

% Optional override for |\version| flag:
%    \begin{macrocode}
%%\providecommand{\version}{final}
%    \end{macrocode}

% Include the main document:
%    \begin{macrocode}
\input{childdoc.def}
\childdocof{cdocsamp}
%    \end{macrocode}

%\iffalse
%</samplechap1|samplechap2>
%\fi
%
%\iffalse
%<*samplechap1>
%\fi
% Some text for chapter 1:
%    \begin{macrocode}
\section{one}
some text in chapter one
%    \end{macrocode}

%\iffalse
%</samplechap1>
%\fi
% Some text for chapter 2:
%\iffalse
%<*samplechap2>
%\fi
%    \begin{macrocode}
\section{two}
more text in chapter two
%    \end{macrocode}

%\iffalse
%</samplechap2>
%\fi
%
% %%%%%%%%%%%%%%%%%%%%%%%%%%%%%%%%%%%%%%
% \paragraph{Part Include Files.}
%
% The include files are called |cdocspt3.tex| and |cdocspt4.tex|.
%
%\iffalse
%<*samplepart3|samplepart4>
%\fi

% Optional override for |\version| flag:
%    \begin{macrocode}
%%\providecommand{\version}{final}
%    \end{macrocode}

% Include the main document:
%    \begin{macrocode}
\input{childdoc.def}
\childdocby{cdocsamp}
%    \end{macrocode}

%\iffalse
%</samplepart3|samplepart4>
%\fi
%
%\iffalse
%<*samplepart3>
%\fi
% Some text for part 3:
%    \begin{macrocode}
some text in part three
%    \end{macrocode}

%\iffalse
%</samplepart3>
%\fi
% Some text for part 4:
%\iffalse
%<*samplepart4>
%\fi
%    \begin{macrocode}
more text in part four
%    \end{macrocode}

%\iffalse
%</samplepart4>
%\fi
%
% %%%%%%%%%%%%%%%%%%%%%%%%%%%%%%%%%%%%%%
% \paragraph{Forwarding for a Complete Draft.}
%
% The following forwarding file |cdocsdrf.tex|
% compiles the main document in draft mode:
%\iffalse
%<*sampledraft>
%\fi
%    \begin{macrocode}
\def\version{draft}
\input{childdoc.def}
\childdocforward{cdocsamp}
%    \end{macrocode}

%\iffalse
%</sampledraft>
%\fi
%
% %%%%%%%%%%%%%%%%%%%%%%%%%%%%%%%%%%%%%%
% \paragraph{Forwarding for Final Version of the Chapters.}
%
% The following forwarding files |cdocsfn1.tex| and |cdocsfn2.tex|
% (with identical content)
% compile the final versions of the child documents
% |cdocsch1.tex| and |cdocsch2.tex|, respectively:
%\iffalse
%<*samplefinal>
%\fi
%    \begin{macrocode}
\def\version{final}
\input{childdoc.def}
\childdocforwardprefix[cdocsamp]{cdocsfn}{cdocsch}
%    \end{macrocode}

%\iffalse
%</samplefinal>
%\fi
%
% %%%%%%%%%%%%%%%%%%%%%%%%%%%%%%%%%%%%%%
% \paragraph{Command Line Processing.}
%
% The following three command lines generate the output files
% |cdocscld|, |cdocscl1| and |cdocscl2|
% which should be identical to
% |cdocsdrf|, |cdocsch1| and |cdocsfn2|, respectively:
% \begin{center}
% \begin{tabular}{l}
% |latex -jobname cdocscld \|\\
% |  "\def\version{draft}\input{childdoc.def}\childdocforward{cdocsamp}"|\\
% |latex -jobname cdocscl1 \|\\
% |  "\input{childdoc.def}\childdocforward[cdocsamp]{cdocsch1}"|\\
% |latex -jobname cdocscl2 \|\\
% |  "\def\version{final}\input{childdoc.def}\childdocforward{cdocsch2}"|
% \end{tabular}
% \end{center}
% Note that the trailing backslash on each first line
% merely continues the input to the second line
% (for convenient cut ant paste).
% Furthermore, the command |latex| can be replaced by any
% of its alternative versions such as |pdflatex|.
%
% %%%%%%%%%%%%%%%%%%%%%%%%%%%%%%%%%%%%%%%%%%%%%%%%%%%%%%%%%%%%%%%%%%%%%%%%%%%%%%
% %%%%%%%%%%%%%%%%%%%%%%%%%%%%%%%%%%%%%%%%%%%%%%%%%%%%%%%%%%%%%%%%%%%%%%%%%%%%%%
% \section{Implementation}
%\iffalse
%<*package>
%\fi
%
% This section describes the definitions file |childdoc.def|.

% The definitions cannot be loaded using |\usepackage| or |\RequirePackage|
% which has a mechanism to prevent loading a style file more than once.
% When loading the definitions by means of |\input|
% multiple instances have to be prevented manually:
%\iffalse
%This code needs to be before the `\ProvidesFile' directive
%which is defined at the beginning of this file.
%Therefore it is also placed there and commented out here.
%</package>
%<*discard>
%\fi
%    \begin{macrocode}
\ifdefined\childdocmain\endinput\fi
%    \end{macrocode}
%\iffalse
%</discard>
%<*package>
%\fi
%
% \macro{\ifchilddoc}
% \macro{\ifchilddocmanual}
% The conditional |\ifchilddoc| tells whether a
% child (true) or main (false) document is being compiled.
% The conditional |\ifchilddocmanual| tells whether
% the |\includeonly| mechanism is used (false) or
% the selection of child files must be performed manually (true).
% The definitions initialise to false:
%    \begin{macrocode}
\newif\ifchilddoc
\newif\ifchilddocmanual
%    \end{macrocode}

% \macro{\childdocname}
% \macro{\childdocjob}
% The macro |\childdocname| stores the name of the main document
% to be compiled. The macro |\childdocjob| stores the name of
% the document on which the \LaTeX{} compiler was originally invoked.
% The content of |\jobname| cannot be compared
% to filenames specified in the source due to different catcodes.
% The following code rescans |\jobname|, stores the result
% in |\childdocname| and saves a copy in |\childdocjob|:
%    \begin{macrocode}
\edef\childdocname{\scantokens\expandafter{\jobname\noexpand}}
\let\childdocjob\childdocname
%    \end{macrocode}

% \macro{\childdocdisable}
% The macro |\childdocdisable| prevents the main file
% from being processed more than once.
% At this stage, the main document command |\childdocmain|
% is assumed to be called once again where it should do nothing.
% Any subsequent call to it should prevent
% a secondary processing of the main document
% It overwrites the forwarding commands
% |\childdocof| and |\childdocforward|
% with empty macros to prevent further inclusions of the main document:
%    \begin{macrocode}
\newcommand{\childdocdisable}
{
  \renewcommand{\childdocmain}[1]{\renewcommand{\childdocmain}[1]{\endinput}}
  \renewcommand{\childdocof}[1]{}
  \renewcommand{\childdocby}[2][]{}
  \renewcommand{\childdocforward}[2][]{}
  \renewcommand{\childdocdisable}{}
}
%    \end{macrocode}

% \macro{\childdocmain}
% The macro |\childdocmain| is to be called at the top of the main file
% with nothing or the main filename (without extension) as argument.
% First, it breaks loops.
% If the argument is not empty and does not match |\childdocname|
% (which is set by the first inclusion of |childdoc.def|),
% |\ifchilddoc| is set to true, |\includeonly| is applied to the child file
% and |\jobname| is set to the main file
% (for proper handling of |.aux| files):
%    \begin{macrocode}
\newcommand{\childdocmain}[1]
{
  \childdocdisable\childdocmain{}
  \if?#1?\else
    \begingroup
      \def\childdoctmp{#1}
      \ifx\childdoctmp\childdocname
        \def\childdoctmp{}
      \else
        \def\childdoctmp
        {
          \childdoctrue
          \includeonly{\childdocname}
          \def\childdocjob{#1}
          \def\jobname{#1}
        }
      \fi
      \expandafter
    \endgroup
    \childdoctmp
  \fi
}
%    \end{macrocode}

% \macro{\childdocof}
% The command |\childdocof| redirects
% compilation to the main file |#1|.
%    \begin{macrocode}
\newcommand{\childdocof}[1]
{
  \childdocdisable
  \childdoctrue
  \includeonly{\childdocname}
  \def\jobname{#1}
  \def\childdocjob{#1}
  \input{#1}
}
%    \end{macrocode}

% \macro{\childdocby}
% The command |\childdocby| ....
%    \begin{macrocode}
\newcommand{\childdocby}[2][]
{
  \childdocdisable
  \childdoctrue
  \childdocmanualtrue
  \if?#1?\else
    \def\jobname{#2}
  \fi
  \def\childdocjob{#2}
  \input{#2}
  \endinput
}
%    \end{macrocode}

% \macro{\childdocforward}
% The command |\childdocforward| redirects
% compilation to the main file or
% (if the optional argument is given) a child file.
% Parameters are set as if the main file
% or a child file starting with |\childdocof| was compiled.
% Then compilation is handed over to the main file:
%    \begin{macrocode}
\newcommand{\childdocforward}[2][]
{
  \begingroup
    \if?#1?
      \def\childdoctmp
      {
        \def\childdocname{#2}
        \def\childdocjob{#2}
        \def\jobname{#2}
        \input{#2}
        \endinput
      }
    \else
      \def\childdoctmp
      {
        \childdocdisable
        \def\childdocname{#2}
        \childdoctrue
        \includeonly{#2}
        \def\childdocjob{#1}
        \def\jobname{#1}
        \input{#1}
        \endinput
      }
    \fi
    \expandafter
  \endgroup
  \childdoctmp
}
%    \end{macrocode}

% \macro{\childdocforwardprefix}
% The command |\childdocforwardprefix| redirects
% compilation to the main or a child file by means of a pattern.
% The prefix |#1| in the current filename is replaced by |#2|
% and the suffix of the current filename is kept
% (it is assumed that the filename does not contain the substring `|~~~|'
% which is used as a delimiter).
% Compilation is handed over to the new file by |\childdocforward|:
%    \begin{macrocode}
\newcommand{\childdocforwardprefix}[3][]
{
  \begingroup
    \def\childdocextract #2##1~~~{\def\childdoctmp{\childdocforward[#1]{#3##1}}}
    \expandafter\childdocextract\childdocname~~~
    \expandafter
  \endgroup
  \childdoctmp
}
%    \end{macrocode}

% \macro{\childdoc}
% The deprecated macro |\childdoc| is a legacy version of |\childdocmain|:
%    \begin{macrocode}
\newcommand{\childdoc}{\childdocmain}
%    \end{macrocode}

% \macro{\childdocredirect}
% The deprecated macro |\childdocredirect| is a legacy version
% of |\childdocforward| and |\childdocforwardprefix|:
%    \begin{macrocode}
\newcommand{\childdocredirect}[2][]
{
  \begingroup
    \if?#1?
      \def\childdoctmp{\childdocforward{#2}}
    \else
      \def\childdoctmp{\childdocforwardprefix{#1}{#2}}
    \fi
    \expandafter
  \endgroup
  \childdoctmp
}
%    \end{macrocode}

%\iffalse
%</package>
%\fi
%
\endinput
\childdocforward{cdocsch2}"|
% \end{tabular}
% \end{center}
% Note that the trailing backslash on each first line
% merely continues the input to the second line
% (for convenient cut ant paste).
% Furthermore, the command |latex| can be replaced by any
% of its alternative versions such as |pdflatex|.
%
% %%%%%%%%%%%%%%%%%%%%%%%%%%%%%%%%%%%%%%%%%%%%%%%%%%%%%%%%%%%%%%%%%%%%%%%%%%%%%%
% %%%%%%%%%%%%%%%%%%%%%%%%%%%%%%%%%%%%%%%%%%%%%%%%%%%%%%%%%%%%%%%%%%%%%%%%%%%%%%
% \section{Implementation}
%\iffalse
%<*package>
%\fi
%
% This section describes the definitions file |childdoc.def|.

% The definitions cannot be loaded using |\usepackage| or |\RequirePackage|
% which has a mechanism to prevent loading a style file more than once.
% When loading the definitions by means of |\input|
% multiple instances have to be prevented manually:
%\iffalse
%This code needs to be before the `\ProvidesFile' directive
%which is defined at the beginning of this file.
%Therefore it is also placed there and commented out here.
%</package>
%<*discard>
%\fi
%    \begin{macrocode}
\ifdefined\childdocmain\endinput\fi
%    \end{macrocode}
%\iffalse
%</discard>
%<*package>
%\fi
%
% \macro{\ifchilddoc}
% \macro{\ifchilddocmanual}
% The conditional |\ifchilddoc| tells whether a
% child (true) or main (false) document is being compiled.
% The conditional |\ifchilddocmanual| tells whether
% the |\includeonly| mechanism is used (false) or
% the selection of child files must be performed manually (true).
% The definitions initialise to false:
%    \begin{macrocode}
\newif\ifchilddoc
\newif\ifchilddocmanual
%    \end{macrocode}

% \macro{\childdocname}
% \macro{\childdocjob}
% The macro |\childdocname| stores the name of the main document
% to be compiled. The macro |\childdocjob| stores the name of
% the document on which the \LaTeX{} compiler was originally invoked.
% The content of |\jobname| cannot be compared
% to filenames specified in the source due to different catcodes.
% The following code rescans |\jobname|, stores the result
% in |\childdocname| and saves a copy in |\childdocjob|:
%    \begin{macrocode}
\edef\childdocname{\scantokens\expandafter{\jobname\noexpand}}
\let\childdocjob\childdocname
%    \end{macrocode}

% \macro{\childdocdisable}
% The macro |\childdocdisable| prevents the main file
% from being processed more than once.
% At this stage, the main document command |\childdocmain|
% is assumed to be called once again where it should do nothing.
% Any subsequent call to it should prevent
% a secondary processing of the main document
% It overwrites the forwarding commands
% |\childdocof| and |\childdocforward|
% with empty macros to prevent further inclusions of the main document:
%    \begin{macrocode}
\newcommand{\childdocdisable}
{
  \renewcommand{\childdocmain}[1]{\renewcommand{\childdocmain}[1]{\endinput}}
  \renewcommand{\childdocof}[1]{}
  \renewcommand{\childdocby}[2][]{}
  \renewcommand{\childdocforward}[2][]{}
  \renewcommand{\childdocdisable}{}
}
%    \end{macrocode}

% \macro{\childdocmain}
% The macro |\childdocmain| is to be called at the top of the main file
% with nothing or the main filename (without extension) as argument.
% First, it breaks loops.
% If the argument is not empty and does not match |\childdocname|
% (which is set by the first inclusion of |childdoc.def|),
% |\ifchilddoc| is set to true, |\includeonly| is applied to the child file
% and |\jobname| is set to the main file
% (for proper handling of |.aux| files):
%    \begin{macrocode}
\newcommand{\childdocmain}[1]
{
  \childdocdisable\childdocmain{}
  \if?#1?\else
    \begingroup
      \def\childdoctmp{#1}
      \ifx\childdoctmp\childdocname
        \def\childdoctmp{}
      \else
        \def\childdoctmp
        {
          \childdoctrue
          \includeonly{\childdocname}
          \def\childdocjob{#1}
          \def\jobname{#1}
        }
      \fi
      \expandafter
    \endgroup
    \childdoctmp
  \fi
}
%    \end{macrocode}

% \macro{\childdocof}
% The command |\childdocof| redirects
% compilation to the main file |#1|.
%    \begin{macrocode}
\newcommand{\childdocof}[1]
{
  \childdocdisable
  \childdoctrue
  \includeonly{\childdocname}
  \def\jobname{#1}
  \def\childdocjob{#1}
  \input{#1}
}
%    \end{macrocode}

% \macro{\childdocby}
% The command |\childdocby| ....
%    \begin{macrocode}
\newcommand{\childdocby}[2][]
{
  \childdocdisable
  \childdoctrue
  \childdocmanualtrue
  \if?#1?\else
    \def\jobname{#2}
  \fi
  \def\childdocjob{#2}
  \input{#2}
  \endinput
}
%    \end{macrocode}

% \macro{\childdocforward}
% The command |\childdocforward| redirects
% compilation to the main file or
% (if the optional argument is given) a child file.
% Parameters are set as if the main file
% or a child file starting with |\childdocof| was compiled.
% Then compilation is handed over to the main file:
%    \begin{macrocode}
\newcommand{\childdocforward}[2][]
{
  \begingroup
    \if?#1?
      \def\childdoctmp
      {
        \def\childdocname{#2}
        \def\childdocjob{#2}
        \def\jobname{#2}
        \input{#2}
        \endinput
      }
    \else
      \def\childdoctmp
      {
        \childdocdisable
        \def\childdocname{#2}
        \childdoctrue
        \includeonly{#2}
        \def\childdocjob{#1}
        \def\jobname{#1}
        \input{#1}
        \endinput
      }
    \fi
    \expandafter
  \endgroup
  \childdoctmp
}
%    \end{macrocode}

% \macro{\childdocforwardprefix}
% The command |\childdocforwardprefix| redirects
% compilation to the main or a child file by means of a pattern.
% The prefix |#1| in the current filename is replaced by |#2|
% and the suffix of the current filename is kept
% (it is assumed that the filename does not contain the substring `|~~~|'
% which is used as a delimiter).
% Compilation is handed over to the new file by |\childdocforward|:
%    \begin{macrocode}
\newcommand{\childdocforwardprefix}[3][]
{
  \begingroup
    \def\childdocextract #2##1~~~{\def\childdoctmp{\childdocforward[#1]{#3##1}}}
    \expandafter\childdocextract\childdocname~~~
    \expandafter
  \endgroup
  \childdoctmp
}
%    \end{macrocode}

% \macro{\childdoc}
% The deprecated macro |\childdoc| is a legacy version of |\childdocmain|:
%    \begin{macrocode}
\newcommand{\childdoc}{\childdocmain}
%    \end{macrocode}

% \macro{\childdocredirect}
% The deprecated macro |\childdocredirect| is a legacy version
% of |\childdocforward| and |\childdocforwardprefix|:
%    \begin{macrocode}
\newcommand{\childdocredirect}[2][]
{
  \begingroup
    \if?#1?
      \def\childdoctmp{\childdocforward{#2}}
    \else
      \def\childdoctmp{\childdocforwardprefix{#1}{#2}}
    \fi
    \expandafter
  \endgroup
  \childdoctmp
}
%    \end{macrocode}

%\iffalse
%</package>
%\fi
%
\endinput
\childdocforward{cdocsch2}"|
% \end{tabular}
% \end{center}
% Note that the trailing backslash on each first line
% merely continues the input to the second line
% (for convenient cut ant paste).
% Furthermore, the command |latex| can be replaced by any
% of its alternative versions such as |pdflatex|.
%
% %%%%%%%%%%%%%%%%%%%%%%%%%%%%%%%%%%%%%%%%%%%%%%%%%%%%%%%%%%%%%%%%%%%%%%%%%%%%%%
% %%%%%%%%%%%%%%%%%%%%%%%%%%%%%%%%%%%%%%%%%%%%%%%%%%%%%%%%%%%%%%%%%%%%%%%%%%%%%%
% \section{Implementation}
%\iffalse
%<*package>
%\fi
%
% This section describes the definitions file |childdoc.def|.

% The definitions cannot be loaded using |\usepackage| or |\RequirePackage|
% which has a mechanism to prevent loading a style file more than once.
% When loading the definitions by means of |\input|
% multiple instances have to be prevented manually:
%\iffalse
%This code needs to be before the `\ProvidesFile' directive
%which is defined at the beginning of this file.
%Therefore it is also placed there and commented out here.
%</package>
%<*discard>
%\fi
%    \begin{macrocode}
\ifdefined\childdocmain\endinput\fi
%    \end{macrocode}
%\iffalse
%</discard>
%<*package>
%\fi
%
% \macro{\ifchilddoc}
% \macro{\ifchilddocmanual}
% The conditional |\ifchilddoc| tells whether a
% child (true) or main (false) document is being compiled.
% The conditional |\ifchilddocmanual| tells whether
% the |\includeonly| mechanism is used (false) or
% the selection of child files must be performed manually (true).
% The definitions initialise to false:
%    \begin{macrocode}
\newif\ifchilddoc
\newif\ifchilddocmanual
%    \end{macrocode}

% \macro{\childdocname}
% \macro{\childdocjob}
% The macro |\childdocname| stores the name of the main document
% to be compiled. The macro |\childdocjob| stores the name of
% the document on which the \LaTeX{} compiler was originally invoked.
% The content of |\jobname| cannot be compared
% to filenames specified in the source due to different catcodes.
% The following code rescans |\jobname|, stores the result
% in |\childdocname| and saves a copy in |\childdocjob|:
%    \begin{macrocode}
\edef\childdocname{\scantokens\expandafter{\jobname\noexpand}}
\let\childdocjob\childdocname
%    \end{macrocode}

% \macro{\childdocdisable}
% The macro |\childdocdisable| prevents the main file
% from being processed more than once.
% At this stage, the main document command |\childdocmain|
% is assumed to be called once again where it should do nothing.
% Any subsequent call to it should prevent
% a secondary processing of the main document
% It overwrites the forwarding commands
% |\childdocof| and |\childdocforward|
% with empty macros to prevent further inclusions of the main document:
%    \begin{macrocode}
\newcommand{\childdocdisable}
{
  \renewcommand{\childdocmain}[1]{\renewcommand{\childdocmain}[1]{\endinput}}
  \renewcommand{\childdocof}[1]{}
  \renewcommand{\childdocby}[2][]{}
  \renewcommand{\childdocforward}[2][]{}
  \renewcommand{\childdocdisable}{}
}
%    \end{macrocode}

% \macro{\childdocmain}
% The macro |\childdocmain| is to be called at the top of the main file
% with nothing or the main filename (without extension) as argument.
% First, it breaks loops.
% If the argument is not empty and does not match |\childdocname|
% (which is set by the first inclusion of |childdoc.def|),
% |\ifchilddoc| is set to true, |\includeonly| is applied to the child file
% and |\jobname| is set to the main file
% (for proper handling of |.aux| files):
%    \begin{macrocode}
\newcommand{\childdocmain}[1]
{
  \childdocdisable\childdocmain{}
  \if?#1?\else
    \begingroup
      \def\childdoctmp{#1}
      \ifx\childdoctmp\childdocname
        \def\childdoctmp{}
      \else
        \def\childdoctmp
        {
          \childdoctrue
          \includeonly{\childdocname}
          \def\childdocjob{#1}
          \def\jobname{#1}
        }
      \fi
      \expandafter
    \endgroup
    \childdoctmp
  \fi
}
%    \end{macrocode}

% \macro{\childdocof}
% The command |\childdocof| redirects
% compilation to the main file |#1|.
%    \begin{macrocode}
\newcommand{\childdocof}[1]
{
  \childdocdisable
  \childdoctrue
  \includeonly{\childdocname}
  \def\jobname{#1}
  \def\childdocjob{#1}
  \input{#1}
}
%    \end{macrocode}

% \macro{\childdocby}
% The command |\childdocby| ....
%    \begin{macrocode}
\newcommand{\childdocby}[2][]
{
  \childdocdisable
  \childdoctrue
  \childdocmanualtrue
  \if?#1?\else
    \def\jobname{#2}
  \fi
  \def\childdocjob{#2}
  \input{#2}
  \endinput
}
%    \end{macrocode}

% \macro{\childdocforward}
% The command |\childdocforward| redirects
% compilation to the main file or
% (if the optional argument is given) a child file.
% Parameters are set as if the main file
% or a child file starting with |\childdocof| was compiled.
% Then compilation is handed over to the main file:
%    \begin{macrocode}
\newcommand{\childdocforward}[2][]
{
  \begingroup
    \if?#1?
      \def\childdoctmp
      {
        \def\childdocname{#2}
        \def\childdocjob{#2}
        \def\jobname{#2}
        \input{#2}
        \endinput
      }
    \else
      \def\childdoctmp
      {
        \childdocdisable
        \def\childdocname{#2}
        \childdoctrue
        \includeonly{#2}
        \def\childdocjob{#1}
        \def\jobname{#1}
        \input{#1}
        \endinput
      }
    \fi
    \expandafter
  \endgroup
  \childdoctmp
}
%    \end{macrocode}

% \macro{\childdocforwardprefix}
% The command |\childdocforwardprefix| redirects
% compilation to the main or a child file by means of a pattern.
% The prefix |#1| in the current filename is replaced by |#2|
% and the suffix of the current filename is kept
% (it is assumed that the filename does not contain the substring `|~~~|'
% which is used as a delimiter).
% Compilation is handed over to the new file by |\childdocforward|:
%    \begin{macrocode}
\newcommand{\childdocforwardprefix}[3][]
{
  \begingroup
    \def\childdocextract #2##1~~~{\def\childdoctmp{\childdocforward[#1]{#3##1}}}
    \expandafter\childdocextract\childdocname~~~
    \expandafter
  \endgroup
  \childdoctmp
}
%    \end{macrocode}

% \macro{\childdoc}
% The deprecated macro |\childdoc| is a legacy version of |\childdocmain|:
%    \begin{macrocode}
\newcommand{\childdoc}{\childdocmain}
%    \end{macrocode}

% \macro{\childdocredirect}
% The deprecated macro |\childdocredirect| is a legacy version
% of |\childdocforward| and |\childdocforwardprefix|:
%    \begin{macrocode}
\newcommand{\childdocredirect}[2][]
{
  \begingroup
    \if?#1?
      \def\childdoctmp{\childdocforward{#2}}
    \else
      \def\childdoctmp{\childdocforwardprefix{#1}{#2}}
    \fi
    \expandafter
  \endgroup
  \childdoctmp
}
%    \end{macrocode}

%\iffalse
%</package>
%\fi
%
\endinput
$fwd"
  for pass in first main
  do
    par="";
    if [[ "$pass" == "first" ]]; then par="-draftmode"; fi
    drop="This is|entering extended mode|\\write18"
    drop="$drop|Preloading the plain mem file|mpost\.mp|plain\.mp"
    pdflatex -shell-escape -interaction=batchmode $par \
      -jobname "$trg" "$body" | grep -vE "$drop"
    if [[ "$pass" != "main" ]]; then continue; fi
    if ! (grep -E -q "may have changed|rerunfilecheck Warning" "$trg.log"); then break; fi
  done
  grep -E "^! |Warning|Error|Undefined|Overfull|Underfull" "$trg.log"
}
%    \end{macrocode}
%\iffalse$\fi

% Function to generate a \textsf{childdoc} compile file with specific options:
%    \begin{macrocode}
function writesource
{
  if [[ -z $num ]]
  then
    fwd="\\childdocforward{$srcmain}"
  else
    fwd="\\childdocforwardprefix[$srcmain]{$target}{$srcsec}"
  fi
  body="$optdef\% \iffalse
%
% childdoc.dtx Copyright (C) 2017-2018 Niklas Beisert
%
% This work may be distributed and/or modified under the
% conditions of the LaTeX Project Public License, either version 1.3
% of this license or (at your option) any later version.
% The latest version of this license is in
%   http://www.latex-project.org/lppl.txt
% and version 1.3 or later is part of all distributions of LaTeX
% version 2005/12/01 or later.
%
% This work has the LPPL maintenance status `maintained'.
%
% The Current Maintainer of this work is Niklas Beisert.
%
% This work consists of the files childdoc.dtx and childdoc.ins
% and the derived files childdoc.def and cdocsamp.tex with
% cdocsch1.tex, cdocsch2.tex, cdocsdrf.tex, cdocsfn1.tex, cdocsfn2.tex.
%
%<package>\ifdefined\childdocmain\endinput\fi
%<package>\ProvidesFile{childdoc.def}[2018/12/30 v2.0 child document driver]
%<samplemain>\ProvidesFile{cdocsamp.tex}[2018/12/30 v2.0 sample for childdoc]
%<*driver>
%\ProvidesFile{childdoc.drv}[2018/12/30 v2.0 childdoc reference manual file]
\PassOptionsToClass{10pt,a4paper}{article}
\documentclass{ltxdoc}

\usepackage[margin=35mm]{geometry}
\usepackage{hyperref}
\usepackage{hyperxmp}
\usepackage[usenames]{color}

\hypersetup{colorlinks=true}
\hypersetup{pdfstartview=FitH}
\hypersetup{pdfpagemode=UseNone}
\hypersetup{pdfsource={}}
\hypersetup{pdflang={en-UK}}
\hypersetup{pdfcopyright={Copyright 2017-2018 Niklas Beisert.
  This work may be distributed and/or modified under the
  conditions of the LaTeX Project Public License, either version 1.3
  of this license or (at your option) any later version.}}
\hypersetup{pdflicenseurl={http://www.latex-project.org/lppl.txt}}
\hypersetup{pdfcontactaddress={ETH Zurich, ITP, HIT K,
  Wolfgang-Pauli-Strasse 27}}
\hypersetup{pdfcontactpostcode={8093}}
\hypersetup{pdfcontactcity={Zurich}}
\hypersetup{pdfcontactcountry={Switzerland}}
\hypersetup{pdfcontactemail={nbeisert@itp.phys.ethz.ch}}
\hypersetup{pdfcontacturl={http://people.phys.ethz.ch/\xmptilde nbeisert/}}

\newcommand{\secref}[1]{\hyperref[#1]{section \ref*{#1}}}

\parskip1ex
\parindent0pt
\let\olditemize\itemize
\def\itemize{\olditemize\parskip0pt}

\begin{document}

\title{The \textsf{childdoc} Package}
\hypersetup{pdftitle={The childdoc Package}}
\author{Niklas Beisert\\[2ex]
  Institut f\"ur Theoretische Physik\\
  Eidgen\"ossische Technische Hochschule Z\"urich\\
  Wolfgang-Pauli-Strasse 27, 8093 Z\"urich, Switzerland\\[1ex]
  \href{mailto:nbeisert@itp.phys.ethz.ch}
  {\texttt{nbeisert@itp.phys.ethz.ch}}}
\hypersetup{pdfauthor={Niklas Beisert}}
\hypersetup{pdfsubject={Manual for the LaTeX2e Package childdoc}}
\date{30 December 2018, \textsf{v2.0}}
\maketitle

\begin{abstract}\noindent
\textsf{childdoc} is a \LaTeXe{} package
that enables the direct compilation
of document sections included by |\include|
to individual files.
\end{abstract}

\begingroup
\parskip0ex
\tableofcontents
\endgroup

%%%%%%%%%%%%%%%%%%%%%%%%%%%%%%%%%%%%%%%%%%%%%%%%%%%%%%%%%%%%%%%%%%%%%%%%%%%%%%%%
%%%%%%%%%%%%%%%%%%%%%%%%%%%%%%%%%%%%%%%%%%%%%%%%%%%%%%%%%%%%%%%%%%%%%%%%%%%%%%%%
\section{Introduction}

\LaTeX{} provides a mechanism to structure a large document (such as a book)
into a main file and several child files (containing the chapters)
using the |\include| command.
This mechanism is beneficial for documents
which span hundreds of pages in order to
make the source file(s) more manageable.
Moreover, compilation can be restricted to
selected child files by means of the |\includeonly| command.
The latter feature can be used to reduce the compilation time while editing
(this was significantly more useful in the earlier days of \LaTeX{})
or to generate a smaller document which is easier to navigate.
Another application of |\includeonly| is to generate
documents consisting of selected parts of the complete document.

However, there are a few drawbacks of the plain |\include| mechanism:
\begin{itemize}
\item
The child files cannot be compiled on their own,
they can only be compiled via the main file.
A naive editing environment
(such as a text editor with an option
to have the current file processed by \LaTeX)
may require one to switch to the main file before compiling;
attempting to compile the child file produces errors.
\item
The main file must be modified (each time)
to adjust the |\includeonly| command
to the present needs. This easily leaves the main file in a messy state.
\item
The generated document will always carry the filename
of the main document. This is inconvenient if
several child files are to be compiled and
to be kept for distribution.
\end{itemize}

The present package provides a simple interface
to make child files individually compilable by \LaTeX{}.
Compiling a child file then has the same effect as compiling
the main file with an |\includeonly| command
to select the appropriate child.
Moreover the generated document will carry the name of the child
rather than the main file.
This resolves all three above issues.

This feature is meant to make the editing of books,
thesis documents and lecture notes somewhat more convenient.
However, the package can also be used efficiently for
composing a series of documents (such as exercise sheets)
which are typically distributed individually.
It then assists the author in generating the individual documents
(potentially in different versions)
as well as a document containing the collected series.
Another application is in developing style files
or other kinds of included material
where compilation of the style file could redirect
to a sample or test file.

%%%%%%%%%%%%%%%%%%%%%%%%%%%%%%%%%%%%%%%%%%%%%%%%%%%%%%%%%%%%%%%%%%%%%%%%%%%%%%%%
%%%%%%%%%%%%%%%%%%%%%%%%%%%%%%%%%%%%%%%%%%%%%%%%%%%%%%%%%%%%%%%%%%%%%%%%%%%%%%%%
\section{Usage}

First of all, the package \textsf{childdoc} is \emph{not} a standard
\LaTeXe{} |.sty| style file! Therefore it needs to be invoked in
a non-standard way.

%%%%%%%%%%%%%%%%%%%%%%%%%%%%%%%%%%%%%%%%%%%%%%%%%%%%%%%%%%%%%%%%%%%%%%%%%%%%%%%%
\subsection{Included Files}
\label{sec:include}

%%%%%%%%%%%%%%%%%%%%%%%%%%%%%%%%%%%%%%%%
\DescribeMacro{\childdocmain}
To use the package, add the commands
\begin{center}
\begin{tabular}{l}
|% \iffalse
%
% childdoc.dtx Copyright (C) 2017-2018 Niklas Beisert
%
% This work may be distributed and/or modified under the
% conditions of the LaTeX Project Public License, either version 1.3
% of this license or (at your option) any later version.
% The latest version of this license is in
%   http://www.latex-project.org/lppl.txt
% and version 1.3 or later is part of all distributions of LaTeX
% version 2005/12/01 or later.
%
% This work has the LPPL maintenance status `maintained'.
%
% The Current Maintainer of this work is Niklas Beisert.
%
% This work consists of the files childdoc.dtx and childdoc.ins
% and the derived files childdoc.def and cdocsamp.tex with
% cdocsch1.tex, cdocsch2.tex, cdocsdrf.tex, cdocsfn1.tex, cdocsfn2.tex.
%
%<package>\ifdefined\childdocmain\endinput\fi
%<package>\ProvidesFile{childdoc.def}[2018/12/30 v2.0 child document driver]
%<samplemain>\ProvidesFile{cdocsamp.tex}[2018/12/30 v2.0 sample for childdoc]
%<*driver>
%\ProvidesFile{childdoc.drv}[2018/12/30 v2.0 childdoc reference manual file]
\PassOptionsToClass{10pt,a4paper}{article}
\documentclass{ltxdoc}

\usepackage[margin=35mm]{geometry}
\usepackage{hyperref}
\usepackage{hyperxmp}
\usepackage[usenames]{color}

\hypersetup{colorlinks=true}
\hypersetup{pdfstartview=FitH}
\hypersetup{pdfpagemode=UseNone}
\hypersetup{pdfsource={}}
\hypersetup{pdflang={en-UK}}
\hypersetup{pdfcopyright={Copyright 2017-2018 Niklas Beisert.
  This work may be distributed and/or modified under the
  conditions of the LaTeX Project Public License, either version 1.3
  of this license or (at your option) any later version.}}
\hypersetup{pdflicenseurl={http://www.latex-project.org/lppl.txt}}
\hypersetup{pdfcontactaddress={ETH Zurich, ITP, HIT K,
  Wolfgang-Pauli-Strasse 27}}
\hypersetup{pdfcontactpostcode={8093}}
\hypersetup{pdfcontactcity={Zurich}}
\hypersetup{pdfcontactcountry={Switzerland}}
\hypersetup{pdfcontactemail={nbeisert@itp.phys.ethz.ch}}
\hypersetup{pdfcontacturl={http://people.phys.ethz.ch/\xmptilde nbeisert/}}

\newcommand{\secref}[1]{\hyperref[#1]{section \ref*{#1}}}

\parskip1ex
\parindent0pt
\let\olditemize\itemize
\def\itemize{\olditemize\parskip0pt}

\begin{document}

\title{The \textsf{childdoc} Package}
\hypersetup{pdftitle={The childdoc Package}}
\author{Niklas Beisert\\[2ex]
  Institut f\"ur Theoretische Physik\\
  Eidgen\"ossische Technische Hochschule Z\"urich\\
  Wolfgang-Pauli-Strasse 27, 8093 Z\"urich, Switzerland\\[1ex]
  \href{mailto:nbeisert@itp.phys.ethz.ch}
  {\texttt{nbeisert@itp.phys.ethz.ch}}}
\hypersetup{pdfauthor={Niklas Beisert}}
\hypersetup{pdfsubject={Manual for the LaTeX2e Package childdoc}}
\date{30 December 2018, \textsf{v2.0}}
\maketitle

\begin{abstract}\noindent
\textsf{childdoc} is a \LaTeXe{} package
that enables the direct compilation
of document sections included by |\include|
to individual files.
\end{abstract}

\begingroup
\parskip0ex
\tableofcontents
\endgroup

%%%%%%%%%%%%%%%%%%%%%%%%%%%%%%%%%%%%%%%%%%%%%%%%%%%%%%%%%%%%%%%%%%%%%%%%%%%%%%%%
%%%%%%%%%%%%%%%%%%%%%%%%%%%%%%%%%%%%%%%%%%%%%%%%%%%%%%%%%%%%%%%%%%%%%%%%%%%%%%%%
\section{Introduction}

\LaTeX{} provides a mechanism to structure a large document (such as a book)
into a main file and several child files (containing the chapters)
using the |\include| command.
This mechanism is beneficial for documents
which span hundreds of pages in order to
make the source file(s) more manageable.
Moreover, compilation can be restricted to
selected child files by means of the |\includeonly| command.
The latter feature can be used to reduce the compilation time while editing
(this was significantly more useful in the earlier days of \LaTeX{})
or to generate a smaller document which is easier to navigate.
Another application of |\includeonly| is to generate
documents consisting of selected parts of the complete document.

However, there are a few drawbacks of the plain |\include| mechanism:
\begin{itemize}
\item
The child files cannot be compiled on their own,
they can only be compiled via the main file.
A naive editing environment
(such as a text editor with an option
to have the current file processed by \LaTeX)
may require one to switch to the main file before compiling;
attempting to compile the child file produces errors.
\item
The main file must be modified (each time)
to adjust the |\includeonly| command
to the present needs. This easily leaves the main file in a messy state.
\item
The generated document will always carry the filename
of the main document. This is inconvenient if
several child files are to be compiled and
to be kept for distribution.
\end{itemize}

The present package provides a simple interface
to make child files individually compilable by \LaTeX{}.
Compiling a child file then has the same effect as compiling
the main file with an |\includeonly| command
to select the appropriate child.
Moreover the generated document will carry the name of the child
rather than the main file.
This resolves all three above issues.

This feature is meant to make the editing of books,
thesis documents and lecture notes somewhat more convenient.
However, the package can also be used efficiently for
composing a series of documents (such as exercise sheets)
which are typically distributed individually.
It then assists the author in generating the individual documents
(potentially in different versions)
as well as a document containing the collected series.
Another application is in developing style files
or other kinds of included material
where compilation of the style file could redirect
to a sample or test file.

%%%%%%%%%%%%%%%%%%%%%%%%%%%%%%%%%%%%%%%%%%%%%%%%%%%%%%%%%%%%%%%%%%%%%%%%%%%%%%%%
%%%%%%%%%%%%%%%%%%%%%%%%%%%%%%%%%%%%%%%%%%%%%%%%%%%%%%%%%%%%%%%%%%%%%%%%%%%%%%%%
\section{Usage}

First of all, the package \textsf{childdoc} is \emph{not} a standard
\LaTeXe{} |.sty| style file! Therefore it needs to be invoked in
a non-standard way.

%%%%%%%%%%%%%%%%%%%%%%%%%%%%%%%%%%%%%%%%%%%%%%%%%%%%%%%%%%%%%%%%%%%%%%%%%%%%%%%%
\subsection{Included Files}
\label{sec:include}

%%%%%%%%%%%%%%%%%%%%%%%%%%%%%%%%%%%%%%%%
\DescribeMacro{\childdocmain}
To use the package, add the commands
\begin{center}
\begin{tabular}{l}
|% \iffalse
%
% childdoc.dtx Copyright (C) 2017-2018 Niklas Beisert
%
% This work may be distributed and/or modified under the
% conditions of the LaTeX Project Public License, either version 1.3
% of this license or (at your option) any later version.
% The latest version of this license is in
%   http://www.latex-project.org/lppl.txt
% and version 1.3 or later is part of all distributions of LaTeX
% version 2005/12/01 or later.
%
% This work has the LPPL maintenance status `maintained'.
%
% The Current Maintainer of this work is Niklas Beisert.
%
% This work consists of the files childdoc.dtx and childdoc.ins
% and the derived files childdoc.def and cdocsamp.tex with
% cdocsch1.tex, cdocsch2.tex, cdocsdrf.tex, cdocsfn1.tex, cdocsfn2.tex.
%
%<package>\ifdefined\childdocmain\endinput\fi
%<package>\ProvidesFile{childdoc.def}[2018/12/30 v2.0 child document driver]
%<samplemain>\ProvidesFile{cdocsamp.tex}[2018/12/30 v2.0 sample for childdoc]
%<*driver>
%\ProvidesFile{childdoc.drv}[2018/12/30 v2.0 childdoc reference manual file]
\PassOptionsToClass{10pt,a4paper}{article}
\documentclass{ltxdoc}

\usepackage[margin=35mm]{geometry}
\usepackage{hyperref}
\usepackage{hyperxmp}
\usepackage[usenames]{color}

\hypersetup{colorlinks=true}
\hypersetup{pdfstartview=FitH}
\hypersetup{pdfpagemode=UseNone}
\hypersetup{pdfsource={}}
\hypersetup{pdflang={en-UK}}
\hypersetup{pdfcopyright={Copyright 2017-2018 Niklas Beisert.
  This work may be distributed and/or modified under the
  conditions of the LaTeX Project Public License, either version 1.3
  of this license or (at your option) any later version.}}
\hypersetup{pdflicenseurl={http://www.latex-project.org/lppl.txt}}
\hypersetup{pdfcontactaddress={ETH Zurich, ITP, HIT K,
  Wolfgang-Pauli-Strasse 27}}
\hypersetup{pdfcontactpostcode={8093}}
\hypersetup{pdfcontactcity={Zurich}}
\hypersetup{pdfcontactcountry={Switzerland}}
\hypersetup{pdfcontactemail={nbeisert@itp.phys.ethz.ch}}
\hypersetup{pdfcontacturl={http://people.phys.ethz.ch/\xmptilde nbeisert/}}

\newcommand{\secref}[1]{\hyperref[#1]{section \ref*{#1}}}

\parskip1ex
\parindent0pt
\let\olditemize\itemize
\def\itemize{\olditemize\parskip0pt}

\begin{document}

\title{The \textsf{childdoc} Package}
\hypersetup{pdftitle={The childdoc Package}}
\author{Niklas Beisert\\[2ex]
  Institut f\"ur Theoretische Physik\\
  Eidgen\"ossische Technische Hochschule Z\"urich\\
  Wolfgang-Pauli-Strasse 27, 8093 Z\"urich, Switzerland\\[1ex]
  \href{mailto:nbeisert@itp.phys.ethz.ch}
  {\texttt{nbeisert@itp.phys.ethz.ch}}}
\hypersetup{pdfauthor={Niklas Beisert}}
\hypersetup{pdfsubject={Manual for the LaTeX2e Package childdoc}}
\date{30 December 2018, \textsf{v2.0}}
\maketitle

\begin{abstract}\noindent
\textsf{childdoc} is a \LaTeXe{} package
that enables the direct compilation
of document sections included by |\include|
to individual files.
\end{abstract}

\begingroup
\parskip0ex
\tableofcontents
\endgroup

%%%%%%%%%%%%%%%%%%%%%%%%%%%%%%%%%%%%%%%%%%%%%%%%%%%%%%%%%%%%%%%%%%%%%%%%%%%%%%%%
%%%%%%%%%%%%%%%%%%%%%%%%%%%%%%%%%%%%%%%%%%%%%%%%%%%%%%%%%%%%%%%%%%%%%%%%%%%%%%%%
\section{Introduction}

\LaTeX{} provides a mechanism to structure a large document (such as a book)
into a main file and several child files (containing the chapters)
using the |\include| command.
This mechanism is beneficial for documents
which span hundreds of pages in order to
make the source file(s) more manageable.
Moreover, compilation can be restricted to
selected child files by means of the |\includeonly| command.
The latter feature can be used to reduce the compilation time while editing
(this was significantly more useful in the earlier days of \LaTeX{})
or to generate a smaller document which is easier to navigate.
Another application of |\includeonly| is to generate
documents consisting of selected parts of the complete document.

However, there are a few drawbacks of the plain |\include| mechanism:
\begin{itemize}
\item
The child files cannot be compiled on their own,
they can only be compiled via the main file.
A naive editing environment
(such as a text editor with an option
to have the current file processed by \LaTeX)
may require one to switch to the main file before compiling;
attempting to compile the child file produces errors.
\item
The main file must be modified (each time)
to adjust the |\includeonly| command
to the present needs. This easily leaves the main file in a messy state.
\item
The generated document will always carry the filename
of the main document. This is inconvenient if
several child files are to be compiled and
to be kept for distribution.
\end{itemize}

The present package provides a simple interface
to make child files individually compilable by \LaTeX{}.
Compiling a child file then has the same effect as compiling
the main file with an |\includeonly| command
to select the appropriate child.
Moreover the generated document will carry the name of the child
rather than the main file.
This resolves all three above issues.

This feature is meant to make the editing of books,
thesis documents and lecture notes somewhat more convenient.
However, the package can also be used efficiently for
composing a series of documents (such as exercise sheets)
which are typically distributed individually.
It then assists the author in generating the individual documents
(potentially in different versions)
as well as a document containing the collected series.
Another application is in developing style files
or other kinds of included material
where compilation of the style file could redirect
to a sample or test file.

%%%%%%%%%%%%%%%%%%%%%%%%%%%%%%%%%%%%%%%%%%%%%%%%%%%%%%%%%%%%%%%%%%%%%%%%%%%%%%%%
%%%%%%%%%%%%%%%%%%%%%%%%%%%%%%%%%%%%%%%%%%%%%%%%%%%%%%%%%%%%%%%%%%%%%%%%%%%%%%%%
\section{Usage}

First of all, the package \textsf{childdoc} is \emph{not} a standard
\LaTeXe{} |.sty| style file! Therefore it needs to be invoked in
a non-standard way.

%%%%%%%%%%%%%%%%%%%%%%%%%%%%%%%%%%%%%%%%%%%%%%%%%%%%%%%%%%%%%%%%%%%%%%%%%%%%%%%%
\subsection{Included Files}
\label{sec:include}

%%%%%%%%%%%%%%%%%%%%%%%%%%%%%%%%%%%%%%%%
\DescribeMacro{\childdocmain}
To use the package, add the commands
\begin{center}
\begin{tabular}{l}
|\input{childdoc.def}|\\
|\childdocmain{}|\\
\end{tabular}
\end{center}
at the very top of the main \LaTeX{} file,
in particular \emph{before} the |\documentclass| statement!
The argument of |\childdocmain| should be left empty
(but it must be present).

%%%%%%%%%%%%%%%%%%%%%%%%%%%%%%%%%%%%%%%%
\DescribeMacro{\childdocof}
Furthermore, add the commands
\begin{center}
\begin{tabular}{l}
|\input{childdoc.def}|\\
|\childdocof{|\textit{main}|}|\\
\end{tabular}
\end{center}
at the top of every child file \textit{child}
which is included by |\include{|\textit{child}|}|
from within the main file
(or at least for those files to be compiled individually).
The argument \textit{main} must be the filename of the main file.

There are a couple of
considerations in setting up the main and child documents:

%%%%%%%%%%%%%%%%%%%%%%%%%%%%%%%%%%%%%%%%
\paragraph{Restrictions.}

Please note the following restrictions:
\begin{itemize}
\item
|\childdocmain| must be called with one argument \textit{main}
to ensure compatibility with earlier version of the package.
It must either be empty (|\childdocmain{}|)
or precisely match the filename of the main file in which it is specified.
See \secref{sec:detection} for further information.
\item
The filename \textit{main} must be specified without the |.tex| extension.
\item
The filename \textit{main} is case sensitive
(even in case-insensitive file systems)
due to internal string comparison.
\item
The argument \textit{main} should be fully expanded, it cannot be a macro.
\item
Subdirectories and special characters should be avoided in filenames.
\item
The command |\childdocmain{|\textit{main}|}| must be followed by a whitespace.
It should not be followed immediately by another command
or by a comment mark `|%|'.
This is because the \TeX{} parser reads the token immediately following
the argument of |\childdocmain| and puts it
at the beginning of every child section;
however, a white\-space is ignored.
\end{itemize}

%%%%%%%%%%%%%%%%%%%%%%%%%%%%%%%%%%%%%%%%
\paragraph{Content of Main File.}

It is advisable to place all content in the child files included by |\include|.
Any output contained in the main file will appear in all child documents
unless suppressed manually;
it cannot be suppressed automatically by the |\includeonly| directive
and thus should normally be avoided.
A method to include some content in the main file
by means of conditional processing is described in \secref{sec:conditional}.

%%%%%%%%%%%%%%%%%%%%%%%%%%%%%%%%%%%%%%%%
\paragraph{Page Numbering.}

When only a part of the document is compiled,
the appropriate numbering of pages
(as well as other status parameters)
is determined from the |.aux| files.
The latter contain information from previous passes.
However this information needs to propagate through
all intermediate child documents.
Therefore the page numbering in child documents may well
be inconsistent until the complete document is compiled at least once.

A useful (if unconventional) way to always ensure a consistent
page numbering is to restart the numbering in each child document
and denote the pages by `\textit{child}|.|\textit{page}'
where \textit{child} represents the chapter/section number of the child file.
This can be achieved by the command
|\numberwithin{page}{|\textit{child}|}|
of the \textsf{amsmath} package
where \textit{child} can be |chapter| or |section|
depending on the chosen structuring.
Alternatively, one can modify the macro |\thepage| appropriately
and reset the counter |page| at the start of each child file.

%%%%%%%%%%%%%%%%%%%%%%%%%%%%%%%%%%%%%%%%%%%%%%%%%%%%%%%%%%%%%%%%%%%%%%%%%%%%%%%%
\subsection{Conditional Processing}
\label{sec:conditional}

The package provides a mechanism to compile different versions
of a document. To customise the versions further some conditional processing
can come in handy to distinguish which version is being compiled.
The package provides two macros to describe the compilation context:

%%%%%%%%%%%%%%%%%%%%%%%%%%%%%%%%%%%%%%%%
\DescribeMacro{\ifchilddoc}
The conditional |\ifchilddoc| distinguishes between the compilation of
child documents and the main document:
%
\begin{center}
|\ifchilddoc |\textit{child-code}| |[|\||else |\textit{main-code}]| \||fi|
\end{center}

%%%%%%%%%%%%%%%%%%%%%%%%%%%%%%%%%%%%%%%%
\DescribeMacro{\childdocname}
\DescribeMacro{\childdocjob}
The macro |\childdocname| contains the filename (without extension)
of the main or child file being processed.
Note that |\childdocjob| will always contain the name of the main file.

%%%%%%%%%%%%%%%%%%%%%%%%%%%%%%%%%%%%%%%%
\paragraph{Title Page.}

Conditional processing can be used to include a title or banner page
in the main document when proper precautions are taken.
Importantly, the code in the main file should ensure that the page counter
(as well as other status parameters which are stored in the |.aux| files)
takes the same value after the conditional processing.
Otherwise the page numbers may take divergent values
depending on which part is compiled.

For example, a title page could be declared by:
%
\begin{center}
\begin{tabular}{l}
|\ifchilddoc\||else|\\
|\addtocounter{page}{-1}|\\
\textit{code for title page}\\
|\newpage|\\
|\||fi|
\end{tabular}
\end{center}
%
A banner page for the child documents can be generated by:
%
\begin{center}
\begin{tabular}{l}
|\ifchilddoc|\\
|\addtocounter{page}{-1}|\\
\textit{code for banner page}\\
|\newpage|\\
|\||fi|
\end{tabular}
\end{center}
%
Here one could write a message such as:
\begin{center}
|This is the part \childdocname{} of \childdocjob{}.|
\end{center}

%%%%%%%%%%%%%%%%%%%%%%%%%%%%%%%%%%%%%%%%%%%%%%%%%%%%%%%%%%%%%%%%%%%%%%%%%%%%%%%%
\subsection{Flags}
\label{sec:flags}

The package makes it easy to generate different versions
of the main or child documents.
To this end compilation flags can be defined
and assigned different default values.
They will be particularly useful in conjunction
with the forwarding mechanism described in \secref{sec:forward}.

For example, it may be useful to have a flag |\version|
which can be set to |draft| or |final|.
The document source will contain some conditional code
depending on the value of |\version|.
Suppose further, the flag should default to |final| for the main file
and to |draft| for child files
which is a natural assignment for editing the document.
This is achieved by placing the following code
in the preamble of the main document
(below the |\childdocmain| directive):
%
\begin{center}
\begin{tabular}{l}
|\ifchilddoc|\\
|\providecommand{\version}{draft}|\\
|\||else|\\
|\providecommand{\version}{final}|\\
|\||fi|
\end{tabular}
\end{center}
%
The definition by |\providecommand| makes sure
that previous definitions are not overwritten.
Further statements |\providecommand{\version}{...}|
can thus be added before the above code to override it.

For the main file, one might add a line
(between |\childdocmain| and the above block)
%
\begin{center}
|%\ifchilddoc\||else\providecommand{\version}{draft}\||fi|
\end{center}
%
which can be uncommented to produce a draft version.
Likewise one can add a line to the very top of a child file
(above the |\childdocof{|\textit{main}|}| directive)
%
\begin{center}
|%\providecommand{\version}{final}|
\end{center}
%
which can be uncommented to produce the final version of this child document.

%%%%%%%%%%%%%%%%%%%%%%%%%%%%%%%%%%%%%%%%%%%%%%%%%%%%%%%%%%%%%%%%%%%%%%%%%%%%%%%%
\subsection{Forwarding}
\label{sec:forward}

Different versions of the main or child documents
using compilation flags as described in \secref{sec:flags}
can be (permanently) stored in different files
for convenient compilation, viewing and distribution.
To this end, the package defines a command
to pass on compilation to a different file:

%%%%%%%%%%%%%%%%%%%%%%%%%%%%%%%%%%%%%%%%
\DescribeMacro{\childdocforward}
The command |\childdocforward| redirects processing to
another source file:
%
\begin{center}
\begin{tabular}{l}
|\input{childdoc.def}|\\
|\childdocforward[|\textit{main}|]{|\textit{dest}|}|\\
\end{tabular}
\end{center}
%
The argument \textit{dest} is the destination file
(without extension).
It should be the main file or one of the child files.
Note that further \textsf{childdoc} directives
such as |\childdocof| and |\childdocforward|
in the indicated file will be processed in this form.
The optional argument \textit{main}
passes on directly to the main file \textit{main}
while pretending to compile the child \textit{dest}.
This form behaves as if \textit{dest}
issues |\childdocof{|\textit{main}|}| right away,
and no further \textsf{childdoc} directives will be processed.

%%%%%%%%%%%%%%%%%%%%%%%%%%%%%%%%%%%%%%%%
\DescribeMacro{\...prefix}
In the alternative form |\childdocforwardprefix|,
%
\begin{center}
\begin{tabular}{l}
|\input{childdoc.def}|\\
|\childdocforwardprefix[|\textit{main}|]{|\textit{prefix}|}{|\textit{dest}|}|
\end{tabular}
\end{center}
%
the destination file is determined by a pattern
depending on the current file:
To make this work, the current file must be called
`{\textit{prefix}\hspace{0.2em}\textit{suffix}}'
with \textit{prefix} matching precisely the argument.
Processing is then passed on to the file
`{\textit{dest}\hspace{0.2em}\textit{suffix}}'.
Surely, the same effect is achieved by
directly specifying the
argument `{\textit{dest}\hspace{0.2em}\textit{suffix}}'
in the first form.
However, that requires to set up a different file
for each child. With the alternative form of the command
all these files can have exactly the same content
which simplifies setting them up and maintaining them.

For example, the following file |draft.tex|
with a compilation flag |\version| as described in \secref{sec:flags}
compiles the main document as a draft:
%
\begin{center}
\begin{tabular}{l}
|\def\version{draft}|\\
|\input{childdoc.def}|\\
|\childdocforward{|\textit{main}|}|
\end{tabular}
\end{center}
%
Likewise, the following files |final|\textit{nn}|.tex|
compile the final version of the child document
|child|\textit{nn}|.tex|:
%
\begin{center}
\begin{tabular}{l}
|\def\version{final}|\\
|\input{childdoc.def}|\\
|\childdocforwardprefix{final}{child}|
\end{tabular}
\end{center}
%

Note that when several versions of a main file and/or of each child file
are to be generated, it may be convenient to set up a |Makefile| or
shell script to automatise the process.

%%%%%%%%%%%%%%%%%%%%%%%%%%%%%%%%%%%%%%%%%%%%%%%%%%%%%%%%%%%%%%%%%%%%%%%%%%%%%%%%
\subsection{Command Line Processing}
\label{sec:commandline}

The effect of redirection files can also be achieved by invoking
the \LaTeX{} compiler with a more elaborate command line.
Most conveniently this should be done as part
of a shell script or a |Makefile|.

When using \textsf{childdoc} in the main file, the following
command lines effectively perform a redirection
(note that depending on the shell being used,
backslashes may have to be doubled: `|\|' $\to$ `|\\|'):
%
\begin{center}
|... -jobname "|\textit{target}|" |\\|"|[\textit{flags}]%
|\input{childdoc.def}\childdocforward[|\textit{main}|]{|\textit{dest}|}"|
\end{center}
%
Here \textit{target} is the name of the output file,
\textit{main} is the name of the main file
and \textit{dest} is the name of the main or child file to be processed
(all filenames without extensions).
The optional argument \textit{main} can be omitted
if \textit{main} matches \textit{dest}.
Optionally, compilation \textit{flags} can be defined via |\def| commands.
This command line makes the \TeX{} engine believe
it is compiling the file \textit{target}
whose content is specified as the latter parameter.
The provided code then forwards the processing to
\textit{main} or \textit{dest} as described in \secref{sec:forward}.

%%%%%%%%%%%%%%%%%%%%%%%%%%%%%%%%%%%%%%%%%%%%%%%%%%%%%%%%%%%%%%%%%%%%%%%%%%%%%%%%
\subsection{Include by Input}
\label{sec:input}

Including child documents by |\include| has some restrictions by design.
Most notably, the content of a child document always occupies
its own set of pages; pages cannot be shared between child documents.
Usually, this behaviour makes perfect sense
because each child document contain an essential part of the document.
However, in some situations it may be desirable to compose
a document from a collection of parts
without having mandatory page breaks between then.
For this case, the package
provides a mechanism to include parts
by |\input| which can also be processed individually.
However, by construction this mechanism
requires manual handling of the content to be output.

%%%%%%%%%%%%%%%%%%%%%%%%%%%%%%%%%%%%%%%%
\DescribeMacro{\ifchilddocmanual}
The main file should be prepared as usual, see \secref{sec:include}.
However, the document body must make a distinction
between processing of an individual part and of the main document, e.g.:
%
\begin{center}
\begin{tabular}{l}
|\ifchilddocmanual|\\
|\input{\childdocname}|\\
|\||else|\\
\textit{document body with }|\input{|\textit{part}|}|\\
|\||fi|
\end{tabular}
\end{center}
%
The conditional |\ifchilddocmanual| is true whenever
a part to be included by |\input| is being compiled,
and the name of the part is stored in |\childdocname|.

%%%%%%%%%%%%%%%%%%%%%%%%%%%%%%%%%%%%%%%%
\DescribeMacro{\childdocby}
Each part to be included by |\input| should start with:
%
\begin{center}
\begin{tabular}{l}
|\input{childdoc.def}|\\
|\childdocby{|\textit{main}|}|\\
\end{tabular}
\end{center}
%
The directive |\childdocby| is similar to |\childdocof|
described in \secref{sec:include},
but the subsequent selection of content must be done manually.
To that end, both |\ifchilddoc| and |\ifchilddocmanual|
will be true upon processing of a part,
and the name of the part is stored in |\childdocname|.
Note that |\jobname| will be set to the filename of the current part
so that each part receives an individual |.aux| file
that does not interfere with the |.aux| file(s) of the main document.
This behaviour can be altered by the alternative form
|\childdocby[*]{|\textit{main}|}| (with a non-empty optional argument)
which uses the |.aux| file of the main document
by setting |\jobname| to \textit{main}.

%%%%%%%%%%%%%%%%%%%%%%%%%%%%%%%%%%%%%%%%%%%%%%%%%%%%%%%%%%%%%%%%%%%%%%%%%%%%%%%%
\subsection{Driver Development}
\label{sec:driver}

The \textsf{childdoc} mechanism can also be use for the development
of definition files such as \LaTeX{} styles or classes.
This case differs from the above setup with multiple parts
included by |\include| in that no |\includeonly| should be invoked.
This can be achieved by starting the include file
(before |\ProvidesPackage|) with:
%
\begin{center}
\begin{tabular}{l}
|\input{childdoc.def}|\\
|\childdocforward{|\textit{main}|}|\\
\end{tabular}
\end{center}
%
or alternatively with:
%
\begin{center}
\begin{tabular}{l}
|\input{childdoc.def}|\\
|\childdocby{|\textit{main}|}|\\
\end{tabular}
\end{center}
%
Both forms have slightly different effects as described above.
The main file is prepared as usual, see \secref{sec:include}.

%%%%%%%%%%%%%%%%%%%%%%%%%%%%%%%%%%%%%%%%%%%%%%%%%%%%%%%%%%%%%%%%%%%%%%%%%%%%%%%%
\subsection{Legacy Detection}
\label{sec:detection}

The directive |\childdocmain| in the main file can detect
whether the complete document or merely a child is to be compiled
even without using the directive |\childdocof|.
This method is deprecated because it is less robust
and there is no compelling reason to use it;
it is merely provided for backward compatibility
and it may be removed in future versions.

If the detection mechanism is to be used,
it is mandatory to correctly specify
the filename of the main file as the argument of |\childdocmain|:
%
\begin{center}
\begin{tabular}{l}
|\input{childdoc.def}|\\
|\childdocmain{|\textit{main}|}|\\
\end{tabular}
\end{center}
%
If |\jobname| does not match the argument \textit{main} of |\childdocmain|,
it is assumed that |\jobname| points to the child file to be compiled.
When using |\childdocmain| with the main file specified as argument,
it suffices to start a child file
with just |\input{|\textit{main}|}|
without loading of the package and using |\childdocof|.
If instead all processing is done
with the appropriate \textsf{childdoc} directives,
the argument of \textit{main} of |\childdocmain| can be empty.

An alternative version of the command line processing described
in \secref{sec:commandline} using the detection mechanism reads:
%
\begin{center}
|... -jobname "|\textit{target}|" "|[\textit{flags}]%
[|\def\jobname{|\textit{dest}|}|]|\input{|\textit{main}|}"|
\end{center}

%%%%%%%%%%%%%%%%%%%%%%%%%%%%%%%%%%%%%%%%%%%%%%%%%%%%%%%%%%%%%%%%%%%%%%%%%%%%%%%%
\subsection{Manual Code}
\label{sec:manual}

In case one cannot be certain whether the definitions file |childdoc.def|
is installed on the target \TeX{} distribution
and one prefers not to ship it,
it is conceivable to paste a few relevant commands into the sources.

To that end, drop all statements |\input{childdoc.def}|
and perform the replacements as outlined below.
Instead of |\childdocmain{|\textit{main}|}| add the following code
to the top of the main file:
%
\begin{center}
\begin{tabular}{l}
|\||ifdefined\childdocname\endinput\||fi\newif\ifchilddoc|\\
|\edef\childdocname{\scantokens\expandafter{\jobname\noexpand}}|\\
|\def\childdocmain{|\textit{main}|}\||ifx\childdocmain\childdocname\||else|\\
|\childdoctrue\includeonly{\childdocname}\let\jobname\childdocmain\||fi|\\
\end{tabular}
\end{center}
%
Instead of |\childdocof{|\textit{main}|}| just include the main file
at the top of each child file:
%
\begin{center}
|\input{|\textit{main}|}|
\end{center}
%
A simple redirection |\childdocforward{|\textit{dest}|}| is achieved by:
%
\begin{center}
|\def\jobname{|\textit{dest}|}\input{\jobname}|
\end{center}
%
The redirection with prefix
|\childdocforwardprefix[|\textit{prefix}|]{|\textit{dest}|}|
is accomplished by:
%
\begin{center}
\begin{tabular}{l}
|{\edef\jobname{\scantokens\expandafter{\jobname\noexpand}}|\\
|\def\redirectjob |\textit{prefix}|#1~~~{\gdef\jobname{|\textit{dest}|#1}}|\\
|\expandafter\redirectjob\jobname~~~}\input{\jobname}|
\end{tabular}
\end{center}

In an alternative approach,
child documents can be compiled by a specific command line
without additional code or specific definitions:
%
\begin{center}
|... -jobname "|\textit{target}|" "|[\textit{flags}]%
|\includeonly{|\textit{dest}|}\input{|\textit{main}|}"|
\end{center}
%

%%%%%%%%%%%%%%%%%%%%%%%%%%%%%%%%%%%%%%%%%%%%%%%%%%%%%%%%%%%%%%%%%%%%%%%%%%%%%%%%
%%%%%%%%%%%%%%%%%%%%%%%%%%%%%%%%%%%%%%%%%%%%%%%%%%%%%%%%%%%%%%%%%%%%%%%%%%%%%%%%
\section{Information}

%%%%%%%%%%%%%%%%%%%%%%%%%%%%%%%%%%%%%%%%%%%%%%%%%%%%%%%%%%%%%%%%%%%%%%%%%%%%%%%%
\subsection{Copyright}

Copyright \copyright{} 2017--2018 Niklas Beisert

This work may be distributed and/or modified under the
conditions of the \LaTeX{} Project Public License, either version 1.3
of this license or (at your option) any later version.
The latest version of this license is in
  \url{http://www.latex-project.org/lppl.txt}
and version 1.3 or later is part of all distributions of \LaTeX{}
version 2005/12/01 or later.

This work has the LPPL maintenance status `maintained'.

The Current Maintainer of this work is Niklas Beisert.

This work consists of the files |README.txt|, |childdoc.ins| and |childdoc.dtx|
as well as the derived files |childdoc.def|, |cdocsamp.tex|
with |cdocsch1.tex|, |cdocsch2.tex|, |cdocspt3.tex|, |cdocspt4.tex|,
|cdocsdrf.tex|, |cdocsfn1.tex|, |cdocsfn2.tex|
as well as |childdoc.pdf|.

%%%%%%%%%%%%%%%%%%%%%%%%%%%%%%%%%%%%%%%%%%%%%%%%%%%%%%%%%%%%%%%%%%%%%%%%%%%%%%%%
\subsection{Files and Installation}

The package consists of the files:
%
\begin{center}
\begin{tabular}{ll}
    |README.txt|   & readme file \\
    |childdoc.ins| & installation file \\
    |childdoc.dtx| & source file \\
    |childdoc.def| & definition file \\
    |cdocsamp.tex| & sample main file \\
    |cdocsch1.tex| & sample include file \\
    |cdocsch2.tex| & sample include file \\
    |cdocspt3.tex| & sample part file \\
    |cdocspt4.tex| & sample part file \\
    |cdocsdrf.tex| & sample redirection file \\
    |cdocsfn1.tex| & sample redirection file \\
    |cdocsfn2.tex| & sample redirection file \\
    |childdoc.pdf| & manual
\end{tabular}
\end{center}
%
The distribution consists of the files
|README.txt|, |childdoc.ins| and |childdoc.dtx|.
%
\begin{itemize}
\item
Run (pdf)\LaTeX{} on |childdoc.dtx|
to compile the manual |childdoc.pdf| (this file).
\item
Run \LaTeX{} on |childdoc.ins| to create the definitions file |childdoc.def|
and the sample |cdocsamp.tex| with include files
|cdocsch1.tex|, |cdocsch2.tex|, |cdocspt3.tex|, |cdocspt4.tex|,
|cdocsdrf.tex|, |cdocsfn1.tex|, |cdocsfn2.tex|.
Then copy the file |childdoc.def| to an appropriate directory of your \LaTeX{}
distribution, e.g.\ \textit{texmf-root}|/tex/latex/childdoc|.
\end{itemize}

%%%%%%%%%%%%%%%%%%%%%%%%%%%%%%%%%%%%%%%%%%%%%%%%%%%%%%%%%%%%%%%%%%%%%%%%%%%%%%%%
\subsection{Related CTAN Packages}

There are several other packages which offer a similar functionality:
%
\begin{itemize}
\item
The packages
\href{http://ctan.org/pkg/docmute}{\textsf{docmute}},
\href{http://ctan.org/pkg/includex}{\textsf{includex}} and
\href{http://ctan.org/pkg/standalone}{\textsf{standalone}}
provide commands to include only the document body of
a child file thus allowing both files to be compiled individually.
\item
The packages \href{http://ctan.org/pkg/subdocs}{\textsf{subdocs}}
and \href{http://ctan.org/pkg/subfiles}{\textsf{subfiles}}
provide structures in which the main and child documents can be
encapsulated and allowing them to be compiled individually.
The inclusion mechanism is different from the conventional |\include|.
\item
The package \href{http://ctan.org/pkg/combine}{\textsf{combine}}
is an elaborate solution to combine several documents into one.
\end{itemize}
%
See also the CTAN topic \href{http://ctan.org/topic/subdocs}{\textsf{subdocs}}
for further related packages.
The present package differs from the above solutions in that
a document structure constructed with the conventional |\include| mechanism
just needs two extra commands at the top of every file
such that all constituent files can be compiled individually.

%%%%%%%%%%%%%%%%%%%%%%%%%%%%%%%%%%%%%%%%%%%%%%%%%%%%%%%%%%%%%%%%%%%%%%%%%%%%%%%%
%\subsection{Feature Suggestions}
%
%The following is a list of features which may be useful for future
%versions of this package:
%%
%\begin{itemize}
%\item
%\ldots
%\end{itemize}

%%%%%%%%%%%%%%%%%%%%%%%%%%%%%%%%%%%%%%%%%%%%%%%%%%%%%%%%%%%%%%%%%%%%%%%%%%%%%%%%
\subsection{Revision History}

%%%%%%%%%%%%%%%%%%%%%%%%%%%%%%%%%%%%%%%%
\paragraph{v2.0:} 2018/12/30

\begin{itemize}
\item
immediate forward processing
\item
added |\childdocby| mechanism
\item
manual restructured
\end{itemize}

%%%%%%%%%%%%%%%%%%%%%%%%%%%%%%%%%%%%%%%%
\paragraph{v1.6:} 2018/01/17

\begin{itemize}
\item
application for development of include files
\item
corrections to manual
\end{itemize}

%%%%%%%%%%%%%%%%%%%%%%%%%%%%%%%%%%%%%%%%
\paragraph{v1.5:} 2017/05/21

\begin{itemize}
\item
more complete structuring introduced
\item
|\childdocof| introduced
\item
|\childdoc| renamed to |\childdocmain|
\item
|\childredirect| renamed to |\childdocforward| and |\childdocforwardprefix|
and functionality expanded
\end{itemize}

%%%%%%%%%%%%%%%%%%%%%%%%%%%%%%%%%%%%%%%%
\paragraph{v1.0:} 2017/04/27

\begin{itemize}
\item
manual and install package
\item
first version published on CTAN
\end{itemize}

%%%%%%%%%%%%%%%%%%%%%%%%%%%%%%%%%%%%%%%%
\paragraph{v0.6:} 2017/04/26

\begin{itemize}
\item
redirection mechanism added
\end{itemize}

%%%%%%%%%%%%%%%%%%%%%%%%%%%%%%%%%%%%%%%%
\paragraph{v0.5:} 2017/04/26

\begin{itemize}
\item
functionality in definition file
\end{itemize}


%%%%%%%%%%%%%%%%%%%%%%%%%%%%%%%%%%%%%%%%%%%%%%%%%%%%%%%%%%%%%%%%%%%%%%%%%%%%%%%%
%%%%%%%%%%%%%%%%%%%%%%%%%%%%%%%%%%%%%%%%%%%%%%%%%%%%%%%%%%%%%%%%%%%%%%%%%%%%%%%%
%%%%%%%%%%%%%%%%%%%%%%%%%%%%%%%%%%%%%%%%%%%%%%%%%%%%%%%%%%%%%%%%%%%%%%%%%%%%%%%%
\appendix

\settowidth\MacroIndent{\rmfamily\scriptsize 000\ }

 \DocInput{childdoc.dtx}

\end{document}
%</driver>
% \fi
%
% %%%%%%%%%%%%%%%%%%%%%%%%%%%%%%%%%%%%%%%%%%%%%%%%%%%%%%%%%%%%%%%%%%%%%%%%%%%%%%
% %%%%%%%%%%%%%%%%%%%%%%%%%%%%%%%%%%%%%%%%%%%%%%%%%%%%%%%%%%%%%%%%%%%%%%%%%%%%%%
% \section{Sample}
%\iffalse
%<*samplemain>
%\fi
%
% The following presents a sample document
% with two chapters, two parts, a title page,
% a compile flag as well as three forwarding files to set the flag.
% It consists of eight |.tex| files:
% \begin{center}
% \begin{tabular}{ll}
% |cdocsamp.tex|&main file\\
% |cdocsch1.tex|&include file for chapter 1\\
% |cdocsch2.tex|&include file for chapter 2\\
% |cdocspt3.tex|&include file for part 3\\
% |cdocspt4.tex|&include file for part 4\\
% |cdocsdrf.tex|&forwarding file for main file in draft mode\\
% |cdocsfi1.tex|&forwarding file for final version of chapter 1\\
% |cdocsfi2.tex|&forwarding file for final version of chapter 2\\
% \end{tabular}
% \end{center}
% Each of the eight files can be compiled directly by the \LaTeX{} compiler.
%
% %%%%%%%%%%%%%%%%%%%%%%%%%%%%%%%%%%%%%%
% \paragraph{Main File.}
%
% The main file is called |cdocsamp.tex|.
%
% Load the \textsf{childdoc} definitions and
% declare the filename for the main document:
%    \begin{macrocode}
\input{childdoc.def}
\childdocmain{}
%    \end{macrocode}

% Optional override for |\version| flag:
%    \begin{macrocode}
%%\ifchilddoc\else\providecommand{\version}{draft}\fi
%    \end{macrocode}

% Define the default values for the |\version| flag
% (|final| for the main file and |draft| for childs):
%    \begin{macrocode}
\ifchilddoc
\providecommand{\version}{draft}
\else
\providecommand{\version}{final}
\fi
%    \end{macrocode}

% Load the standard document class:
%    \begin{macrocode}
\documentclass[12pt]{article}
%    \end{macrocode}

% Start the document body:
%    \begin{macrocode}
\begin{document}
%    \end{macrocode}

% Declare a title page.
% Print title, part of document being processed and version flag:
%    \begin{macrocode}
\addtocounter{page}{-1}
\begin{center}
{\LARGE\bfseries{}childdoc example\par}
\vspace{1cm}
\ifchilddoc
\ifchilddocmanual part\else chapter\fi:
`\childdocname' of `\childdocjob'\par
\else
main document: `\childdocjob'\par
\fi
version: \version\par
\end{center}
\newpage
%    \end{macrocode}

% Manually include selected file,
% otherwise process as usual:
%    \begin{macrocode}
\ifchilddocmanual
\section*{part `\childdocname'}
\input{\childdocname}
\else
%    \end{macrocode}

% Include the two chapters:
%    \begin{macrocode}
\include{cdocsch1}
\include{cdocsch2}
%    \end{macrocode}

% Include the two parts unless only chapters should be displayed:
%    \begin{macrocode}
\ifchilddoc\else
\section{part three}
\input{cdocspt3}
\section{part four}
\input{cdocspt4}
\fi
%    \end{macrocode}

% Process as usual until here:
%    \begin{macrocode}
\fi
%    \end{macrocode}

% End of document body:
%    \begin{macrocode}
\end{document}
%    \end{macrocode}
%\iffalse
%</samplemain>
%\fi
%
% %%%%%%%%%%%%%%%%%%%%%%%%%%%%%%%%%%%%%%
% \paragraph{Chapter Include Files.}
%
% The include files are called |cdocsch1.tex| and |cdocsch2.tex|.
%
%\iffalse
%<*samplechap1|samplechap2>
%\fi

% Optional override for |\version| flag:
%    \begin{macrocode}
%%\providecommand{\version}{final}
%    \end{macrocode}

% Include the main document:
%    \begin{macrocode}
\input{childdoc.def}
\childdocof{cdocsamp}
%    \end{macrocode}

%\iffalse
%</samplechap1|samplechap2>
%\fi
%
%\iffalse
%<*samplechap1>
%\fi
% Some text for chapter 1:
%    \begin{macrocode}
\section{one}
some text in chapter one
%    \end{macrocode}

%\iffalse
%</samplechap1>
%\fi
% Some text for chapter 2:
%\iffalse
%<*samplechap2>
%\fi
%    \begin{macrocode}
\section{two}
more text in chapter two
%    \end{macrocode}

%\iffalse
%</samplechap2>
%\fi
%
% %%%%%%%%%%%%%%%%%%%%%%%%%%%%%%%%%%%%%%
% \paragraph{Part Include Files.}
%
% The include files are called |cdocspt3.tex| and |cdocspt4.tex|.
%
%\iffalse
%<*samplepart3|samplepart4>
%\fi

% Optional override for |\version| flag:
%    \begin{macrocode}
%%\providecommand{\version}{final}
%    \end{macrocode}

% Include the main document:
%    \begin{macrocode}
\input{childdoc.def}
\childdocby{cdocsamp}
%    \end{macrocode}

%\iffalse
%</samplepart3|samplepart4>
%\fi
%
%\iffalse
%<*samplepart3>
%\fi
% Some text for part 3:
%    \begin{macrocode}
some text in part three
%    \end{macrocode}

%\iffalse
%</samplepart3>
%\fi
% Some text for part 4:
%\iffalse
%<*samplepart4>
%\fi
%    \begin{macrocode}
more text in part four
%    \end{macrocode}

%\iffalse
%</samplepart4>
%\fi
%
% %%%%%%%%%%%%%%%%%%%%%%%%%%%%%%%%%%%%%%
% \paragraph{Forwarding for a Complete Draft.}
%
% The following forwarding file |cdocsdrf.tex|
% compiles the main document in draft mode:
%\iffalse
%<*sampledraft>
%\fi
%    \begin{macrocode}
\def\version{draft}
\input{childdoc.def}
\childdocforward{cdocsamp}
%    \end{macrocode}

%\iffalse
%</sampledraft>
%\fi
%
% %%%%%%%%%%%%%%%%%%%%%%%%%%%%%%%%%%%%%%
% \paragraph{Forwarding for Final Version of the Chapters.}
%
% The following forwarding files |cdocsfn1.tex| and |cdocsfn2.tex|
% (with identical content)
% compile the final versions of the child documents
% |cdocsch1.tex| and |cdocsch2.tex|, respectively:
%\iffalse
%<*samplefinal>
%\fi
%    \begin{macrocode}
\def\version{final}
\input{childdoc.def}
\childdocforwardprefix[cdocsamp]{cdocsfn}{cdocsch}
%    \end{macrocode}

%\iffalse
%</samplefinal>
%\fi
%
% %%%%%%%%%%%%%%%%%%%%%%%%%%%%%%%%%%%%%%
% \paragraph{Command Line Processing.}
%
% The following three command lines generate the output files
% |cdocscld|, |cdocscl1| and |cdocscl2|
% which should be identical to
% |cdocsdrf|, |cdocsch1| and |cdocsfn2|, respectively:
% \begin{center}
% \begin{tabular}{l}
% |latex -jobname cdocscld \|\\
% |  "\def\version{draft}\input{childdoc.def}\childdocforward{cdocsamp}"|\\
% |latex -jobname cdocscl1 \|\\
% |  "\input{childdoc.def}\childdocforward[cdocsamp]{cdocsch1}"|\\
% |latex -jobname cdocscl2 \|\\
% |  "\def\version{final}\input{childdoc.def}\childdocforward{cdocsch2}"|
% \end{tabular}
% \end{center}
% Note that the trailing backslash on each first line
% merely continues the input to the second line
% (for convenient cut ant paste).
% Furthermore, the command |latex| can be replaced by any
% of its alternative versions such as |pdflatex|.
%
% %%%%%%%%%%%%%%%%%%%%%%%%%%%%%%%%%%%%%%%%%%%%%%%%%%%%%%%%%%%%%%%%%%%%%%%%%%%%%%
% %%%%%%%%%%%%%%%%%%%%%%%%%%%%%%%%%%%%%%%%%%%%%%%%%%%%%%%%%%%%%%%%%%%%%%%%%%%%%%
% \section{Implementation}
%\iffalse
%<*package>
%\fi
%
% This section describes the definitions file |childdoc.def|.

% The definitions cannot be loaded using |\usepackage| or |\RequirePackage|
% which has a mechanism to prevent loading a style file more than once.
% When loading the definitions by means of |\input|
% multiple instances have to be prevented manually:
%\iffalse
%This code needs to be before the `\ProvidesFile' directive
%which is defined at the beginning of this file.
%Therefore it is also placed there and commented out here.
%</package>
%<*discard>
%\fi
%    \begin{macrocode}
\ifdefined\childdocmain\endinput\fi
%    \end{macrocode}
%\iffalse
%</discard>
%<*package>
%\fi
%
% \macro{\ifchilddoc}
% \macro{\ifchilddocmanual}
% The conditional |\ifchilddoc| tells whether a
% child (true) or main (false) document is being compiled.
% The conditional |\ifchilddocmanual| tells whether
% the |\includeonly| mechanism is used (false) or
% the selection of child files must be performed manually (true).
% The definitions initialise to false:
%    \begin{macrocode}
\newif\ifchilddoc
\newif\ifchilddocmanual
%    \end{macrocode}

% \macro{\childdocname}
% \macro{\childdocjob}
% The macro |\childdocname| stores the name of the main document
% to be compiled. The macro |\childdocjob| stores the name of
% the document on which the \LaTeX{} compiler was originally invoked.
% The content of |\jobname| cannot be compared
% to filenames specified in the source due to different catcodes.
% The following code rescans |\jobname|, stores the result
% in |\childdocname| and saves a copy in |\childdocjob|:
%    \begin{macrocode}
\edef\childdocname{\scantokens\expandafter{\jobname\noexpand}}
\let\childdocjob\childdocname
%    \end{macrocode}

% \macro{\childdocdisable}
% The macro |\childdocdisable| prevents the main file
% from being processed more than once.
% At this stage, the main document command |\childdocmain|
% is assumed to be called once again where it should do nothing.
% Any subsequent call to it should prevent
% a secondary processing of the main document
% It overwrites the forwarding commands
% |\childdocof| and |\childdocforward|
% with empty macros to prevent further inclusions of the main document:
%    \begin{macrocode}
\newcommand{\childdocdisable}
{
  \renewcommand{\childdocmain}[1]{\renewcommand{\childdocmain}[1]{\endinput}}
  \renewcommand{\childdocof}[1]{}
  \renewcommand{\childdocby}[2][]{}
  \renewcommand{\childdocforward}[2][]{}
  \renewcommand{\childdocdisable}{}
}
%    \end{macrocode}

% \macro{\childdocmain}
% The macro |\childdocmain| is to be called at the top of the main file
% with nothing or the main filename (without extension) as argument.
% First, it breaks loops.
% If the argument is not empty and does not match |\childdocname|
% (which is set by the first inclusion of |childdoc.def|),
% |\ifchilddoc| is set to true, |\includeonly| is applied to the child file
% and |\jobname| is set to the main file
% (for proper handling of |.aux| files):
%    \begin{macrocode}
\newcommand{\childdocmain}[1]
{
  \childdocdisable\childdocmain{}
  \if?#1?\else
    \begingroup
      \def\childdoctmp{#1}
      \ifx\childdoctmp\childdocname
        \def\childdoctmp{}
      \else
        \def\childdoctmp
        {
          \childdoctrue
          \includeonly{\childdocname}
          \def\childdocjob{#1}
          \def\jobname{#1}
        }
      \fi
      \expandafter
    \endgroup
    \childdoctmp
  \fi
}
%    \end{macrocode}

% \macro{\childdocof}
% The command |\childdocof| redirects
% compilation to the main file |#1|.
%    \begin{macrocode}
\newcommand{\childdocof}[1]
{
  \childdocdisable
  \childdoctrue
  \includeonly{\childdocname}
  \def\jobname{#1}
  \def\childdocjob{#1}
  \input{#1}
}
%    \end{macrocode}

% \macro{\childdocby}
% The command |\childdocby| ....
%    \begin{macrocode}
\newcommand{\childdocby}[2][]
{
  \childdocdisable
  \childdoctrue
  \childdocmanualtrue
  \if?#1?\else
    \def\jobname{#2}
  \fi
  \def\childdocjob{#2}
  \input{#2}
  \endinput
}
%    \end{macrocode}

% \macro{\childdocforward}
% The command |\childdocforward| redirects
% compilation to the main file or
% (if the optional argument is given) a child file.
% Parameters are set as if the main file
% or a child file starting with |\childdocof| was compiled.
% Then compilation is handed over to the main file:
%    \begin{macrocode}
\newcommand{\childdocforward}[2][]
{
  \begingroup
    \if?#1?
      \def\childdoctmp
      {
        \def\childdocname{#2}
        \def\childdocjob{#2}
        \def\jobname{#2}
        \input{#2}
        \endinput
      }
    \else
      \def\childdoctmp
      {
        \childdocdisable
        \def\childdocname{#2}
        \childdoctrue
        \includeonly{#2}
        \def\childdocjob{#1}
        \def\jobname{#1}
        \input{#1}
        \endinput
      }
    \fi
    \expandafter
  \endgroup
  \childdoctmp
}
%    \end{macrocode}

% \macro{\childdocforwardprefix}
% The command |\childdocforwardprefix| redirects
% compilation to the main or a child file by means of a pattern.
% The prefix |#1| in the current filename is replaced by |#2|
% and the suffix of the current filename is kept
% (it is assumed that the filename does not contain the substring `|~~~|'
% which is used as a delimiter).
% Compilation is handed over to the new file by |\childdocforward|:
%    \begin{macrocode}
\newcommand{\childdocforwardprefix}[3][]
{
  \begingroup
    \def\childdocextract #2##1~~~{\def\childdoctmp{\childdocforward[#1]{#3##1}}}
    \expandafter\childdocextract\childdocname~~~
    \expandafter
  \endgroup
  \childdoctmp
}
%    \end{macrocode}

% \macro{\childdoc}
% The deprecated macro |\childdoc| is a legacy version of |\childdocmain|:
%    \begin{macrocode}
\newcommand{\childdoc}{\childdocmain}
%    \end{macrocode}

% \macro{\childdocredirect}
% The deprecated macro |\childdocredirect| is a legacy version
% of |\childdocforward| and |\childdocforwardprefix|:
%    \begin{macrocode}
\newcommand{\childdocredirect}[2][]
{
  \begingroup
    \if?#1?
      \def\childdoctmp{\childdocforward{#2}}
    \else
      \def\childdoctmp{\childdocforwardprefix{#1}{#2}}
    \fi
    \expandafter
  \endgroup
  \childdoctmp
}
%    \end{macrocode}

%\iffalse
%</package>
%\fi
%
\endinput
|\\
|\childdocmain{}|\\
\end{tabular}
\end{center}
at the very top of the main \LaTeX{} file,
in particular \emph{before} the |\documentclass| statement!
The argument of |\childdocmain| should be left empty
(but it must be present).

%%%%%%%%%%%%%%%%%%%%%%%%%%%%%%%%%%%%%%%%
\DescribeMacro{\childdocof}
Furthermore, add the commands
\begin{center}
\begin{tabular}{l}
|% \iffalse
%
% childdoc.dtx Copyright (C) 2017-2018 Niklas Beisert
%
% This work may be distributed and/or modified under the
% conditions of the LaTeX Project Public License, either version 1.3
% of this license or (at your option) any later version.
% The latest version of this license is in
%   http://www.latex-project.org/lppl.txt
% and version 1.3 or later is part of all distributions of LaTeX
% version 2005/12/01 or later.
%
% This work has the LPPL maintenance status `maintained'.
%
% The Current Maintainer of this work is Niklas Beisert.
%
% This work consists of the files childdoc.dtx and childdoc.ins
% and the derived files childdoc.def and cdocsamp.tex with
% cdocsch1.tex, cdocsch2.tex, cdocsdrf.tex, cdocsfn1.tex, cdocsfn2.tex.
%
%<package>\ifdefined\childdocmain\endinput\fi
%<package>\ProvidesFile{childdoc.def}[2018/12/30 v2.0 child document driver]
%<samplemain>\ProvidesFile{cdocsamp.tex}[2018/12/30 v2.0 sample for childdoc]
%<*driver>
%\ProvidesFile{childdoc.drv}[2018/12/30 v2.0 childdoc reference manual file]
\PassOptionsToClass{10pt,a4paper}{article}
\documentclass{ltxdoc}

\usepackage[margin=35mm]{geometry}
\usepackage{hyperref}
\usepackage{hyperxmp}
\usepackage[usenames]{color}

\hypersetup{colorlinks=true}
\hypersetup{pdfstartview=FitH}
\hypersetup{pdfpagemode=UseNone}
\hypersetup{pdfsource={}}
\hypersetup{pdflang={en-UK}}
\hypersetup{pdfcopyright={Copyright 2017-2018 Niklas Beisert.
  This work may be distributed and/or modified under the
  conditions of the LaTeX Project Public License, either version 1.3
  of this license or (at your option) any later version.}}
\hypersetup{pdflicenseurl={http://www.latex-project.org/lppl.txt}}
\hypersetup{pdfcontactaddress={ETH Zurich, ITP, HIT K,
  Wolfgang-Pauli-Strasse 27}}
\hypersetup{pdfcontactpostcode={8093}}
\hypersetup{pdfcontactcity={Zurich}}
\hypersetup{pdfcontactcountry={Switzerland}}
\hypersetup{pdfcontactemail={nbeisert@itp.phys.ethz.ch}}
\hypersetup{pdfcontacturl={http://people.phys.ethz.ch/\xmptilde nbeisert/}}

\newcommand{\secref}[1]{\hyperref[#1]{section \ref*{#1}}}

\parskip1ex
\parindent0pt
\let\olditemize\itemize
\def\itemize{\olditemize\parskip0pt}

\begin{document}

\title{The \textsf{childdoc} Package}
\hypersetup{pdftitle={The childdoc Package}}
\author{Niklas Beisert\\[2ex]
  Institut f\"ur Theoretische Physik\\
  Eidgen\"ossische Technische Hochschule Z\"urich\\
  Wolfgang-Pauli-Strasse 27, 8093 Z\"urich, Switzerland\\[1ex]
  \href{mailto:nbeisert@itp.phys.ethz.ch}
  {\texttt{nbeisert@itp.phys.ethz.ch}}}
\hypersetup{pdfauthor={Niklas Beisert}}
\hypersetup{pdfsubject={Manual for the LaTeX2e Package childdoc}}
\date{30 December 2018, \textsf{v2.0}}
\maketitle

\begin{abstract}\noindent
\textsf{childdoc} is a \LaTeXe{} package
that enables the direct compilation
of document sections included by |\include|
to individual files.
\end{abstract}

\begingroup
\parskip0ex
\tableofcontents
\endgroup

%%%%%%%%%%%%%%%%%%%%%%%%%%%%%%%%%%%%%%%%%%%%%%%%%%%%%%%%%%%%%%%%%%%%%%%%%%%%%%%%
%%%%%%%%%%%%%%%%%%%%%%%%%%%%%%%%%%%%%%%%%%%%%%%%%%%%%%%%%%%%%%%%%%%%%%%%%%%%%%%%
\section{Introduction}

\LaTeX{} provides a mechanism to structure a large document (such as a book)
into a main file and several child files (containing the chapters)
using the |\include| command.
This mechanism is beneficial for documents
which span hundreds of pages in order to
make the source file(s) more manageable.
Moreover, compilation can be restricted to
selected child files by means of the |\includeonly| command.
The latter feature can be used to reduce the compilation time while editing
(this was significantly more useful in the earlier days of \LaTeX{})
or to generate a smaller document which is easier to navigate.
Another application of |\includeonly| is to generate
documents consisting of selected parts of the complete document.

However, there are a few drawbacks of the plain |\include| mechanism:
\begin{itemize}
\item
The child files cannot be compiled on their own,
they can only be compiled via the main file.
A naive editing environment
(such as a text editor with an option
to have the current file processed by \LaTeX)
may require one to switch to the main file before compiling;
attempting to compile the child file produces errors.
\item
The main file must be modified (each time)
to adjust the |\includeonly| command
to the present needs. This easily leaves the main file in a messy state.
\item
The generated document will always carry the filename
of the main document. This is inconvenient if
several child files are to be compiled and
to be kept for distribution.
\end{itemize}

The present package provides a simple interface
to make child files individually compilable by \LaTeX{}.
Compiling a child file then has the same effect as compiling
the main file with an |\includeonly| command
to select the appropriate child.
Moreover the generated document will carry the name of the child
rather than the main file.
This resolves all three above issues.

This feature is meant to make the editing of books,
thesis documents and lecture notes somewhat more convenient.
However, the package can also be used efficiently for
composing a series of documents (such as exercise sheets)
which are typically distributed individually.
It then assists the author in generating the individual documents
(potentially in different versions)
as well as a document containing the collected series.
Another application is in developing style files
or other kinds of included material
where compilation of the style file could redirect
to a sample or test file.

%%%%%%%%%%%%%%%%%%%%%%%%%%%%%%%%%%%%%%%%%%%%%%%%%%%%%%%%%%%%%%%%%%%%%%%%%%%%%%%%
%%%%%%%%%%%%%%%%%%%%%%%%%%%%%%%%%%%%%%%%%%%%%%%%%%%%%%%%%%%%%%%%%%%%%%%%%%%%%%%%
\section{Usage}

First of all, the package \textsf{childdoc} is \emph{not} a standard
\LaTeXe{} |.sty| style file! Therefore it needs to be invoked in
a non-standard way.

%%%%%%%%%%%%%%%%%%%%%%%%%%%%%%%%%%%%%%%%%%%%%%%%%%%%%%%%%%%%%%%%%%%%%%%%%%%%%%%%
\subsection{Included Files}
\label{sec:include}

%%%%%%%%%%%%%%%%%%%%%%%%%%%%%%%%%%%%%%%%
\DescribeMacro{\childdocmain}
To use the package, add the commands
\begin{center}
\begin{tabular}{l}
|\input{childdoc.def}|\\
|\childdocmain{}|\\
\end{tabular}
\end{center}
at the very top of the main \LaTeX{} file,
in particular \emph{before} the |\documentclass| statement!
The argument of |\childdocmain| should be left empty
(but it must be present).

%%%%%%%%%%%%%%%%%%%%%%%%%%%%%%%%%%%%%%%%
\DescribeMacro{\childdocof}
Furthermore, add the commands
\begin{center}
\begin{tabular}{l}
|\input{childdoc.def}|\\
|\childdocof{|\textit{main}|}|\\
\end{tabular}
\end{center}
at the top of every child file \textit{child}
which is included by |\include{|\textit{child}|}|
from within the main file
(or at least for those files to be compiled individually).
The argument \textit{main} must be the filename of the main file.

There are a couple of
considerations in setting up the main and child documents:

%%%%%%%%%%%%%%%%%%%%%%%%%%%%%%%%%%%%%%%%
\paragraph{Restrictions.}

Please note the following restrictions:
\begin{itemize}
\item
|\childdocmain| must be called with one argument \textit{main}
to ensure compatibility with earlier version of the package.
It must either be empty (|\childdocmain{}|)
or precisely match the filename of the main file in which it is specified.
See \secref{sec:detection} for further information.
\item
The filename \textit{main} must be specified without the |.tex| extension.
\item
The filename \textit{main} is case sensitive
(even in case-insensitive file systems)
due to internal string comparison.
\item
The argument \textit{main} should be fully expanded, it cannot be a macro.
\item
Subdirectories and special characters should be avoided in filenames.
\item
The command |\childdocmain{|\textit{main}|}| must be followed by a whitespace.
It should not be followed immediately by another command
or by a comment mark `|%|'.
This is because the \TeX{} parser reads the token immediately following
the argument of |\childdocmain| and puts it
at the beginning of every child section;
however, a white\-space is ignored.
\end{itemize}

%%%%%%%%%%%%%%%%%%%%%%%%%%%%%%%%%%%%%%%%
\paragraph{Content of Main File.}

It is advisable to place all content in the child files included by |\include|.
Any output contained in the main file will appear in all child documents
unless suppressed manually;
it cannot be suppressed automatically by the |\includeonly| directive
and thus should normally be avoided.
A method to include some content in the main file
by means of conditional processing is described in \secref{sec:conditional}.

%%%%%%%%%%%%%%%%%%%%%%%%%%%%%%%%%%%%%%%%
\paragraph{Page Numbering.}

When only a part of the document is compiled,
the appropriate numbering of pages
(as well as other status parameters)
is determined from the |.aux| files.
The latter contain information from previous passes.
However this information needs to propagate through
all intermediate child documents.
Therefore the page numbering in child documents may well
be inconsistent until the complete document is compiled at least once.

A useful (if unconventional) way to always ensure a consistent
page numbering is to restart the numbering in each child document
and denote the pages by `\textit{child}|.|\textit{page}'
where \textit{child} represents the chapter/section number of the child file.
This can be achieved by the command
|\numberwithin{page}{|\textit{child}|}|
of the \textsf{amsmath} package
where \textit{child} can be |chapter| or |section|
depending on the chosen structuring.
Alternatively, one can modify the macro |\thepage| appropriately
and reset the counter |page| at the start of each child file.

%%%%%%%%%%%%%%%%%%%%%%%%%%%%%%%%%%%%%%%%%%%%%%%%%%%%%%%%%%%%%%%%%%%%%%%%%%%%%%%%
\subsection{Conditional Processing}
\label{sec:conditional}

The package provides a mechanism to compile different versions
of a document. To customise the versions further some conditional processing
can come in handy to distinguish which version is being compiled.
The package provides two macros to describe the compilation context:

%%%%%%%%%%%%%%%%%%%%%%%%%%%%%%%%%%%%%%%%
\DescribeMacro{\ifchilddoc}
The conditional |\ifchilddoc| distinguishes between the compilation of
child documents and the main document:
%
\begin{center}
|\ifchilddoc |\textit{child-code}| |[|\||else |\textit{main-code}]| \||fi|
\end{center}

%%%%%%%%%%%%%%%%%%%%%%%%%%%%%%%%%%%%%%%%
\DescribeMacro{\childdocname}
\DescribeMacro{\childdocjob}
The macro |\childdocname| contains the filename (without extension)
of the main or child file being processed.
Note that |\childdocjob| will always contain the name of the main file.

%%%%%%%%%%%%%%%%%%%%%%%%%%%%%%%%%%%%%%%%
\paragraph{Title Page.}

Conditional processing can be used to include a title or banner page
in the main document when proper precautions are taken.
Importantly, the code in the main file should ensure that the page counter
(as well as other status parameters which are stored in the |.aux| files)
takes the same value after the conditional processing.
Otherwise the page numbers may take divergent values
depending on which part is compiled.

For example, a title page could be declared by:
%
\begin{center}
\begin{tabular}{l}
|\ifchilddoc\||else|\\
|\addtocounter{page}{-1}|\\
\textit{code for title page}\\
|\newpage|\\
|\||fi|
\end{tabular}
\end{center}
%
A banner page for the child documents can be generated by:
%
\begin{center}
\begin{tabular}{l}
|\ifchilddoc|\\
|\addtocounter{page}{-1}|\\
\textit{code for banner page}\\
|\newpage|\\
|\||fi|
\end{tabular}
\end{center}
%
Here one could write a message such as:
\begin{center}
|This is the part \childdocname{} of \childdocjob{}.|
\end{center}

%%%%%%%%%%%%%%%%%%%%%%%%%%%%%%%%%%%%%%%%%%%%%%%%%%%%%%%%%%%%%%%%%%%%%%%%%%%%%%%%
\subsection{Flags}
\label{sec:flags}

The package makes it easy to generate different versions
of the main or child documents.
To this end compilation flags can be defined
and assigned different default values.
They will be particularly useful in conjunction
with the forwarding mechanism described in \secref{sec:forward}.

For example, it may be useful to have a flag |\version|
which can be set to |draft| or |final|.
The document source will contain some conditional code
depending on the value of |\version|.
Suppose further, the flag should default to |final| for the main file
and to |draft| for child files
which is a natural assignment for editing the document.
This is achieved by placing the following code
in the preamble of the main document
(below the |\childdocmain| directive):
%
\begin{center}
\begin{tabular}{l}
|\ifchilddoc|\\
|\providecommand{\version}{draft}|\\
|\||else|\\
|\providecommand{\version}{final}|\\
|\||fi|
\end{tabular}
\end{center}
%
The definition by |\providecommand| makes sure
that previous definitions are not overwritten.
Further statements |\providecommand{\version}{...}|
can thus be added before the above code to override it.

For the main file, one might add a line
(between |\childdocmain| and the above block)
%
\begin{center}
|%\ifchilddoc\||else\providecommand{\version}{draft}\||fi|
\end{center}
%
which can be uncommented to produce a draft version.
Likewise one can add a line to the very top of a child file
(above the |\childdocof{|\textit{main}|}| directive)
%
\begin{center}
|%\providecommand{\version}{final}|
\end{center}
%
which can be uncommented to produce the final version of this child document.

%%%%%%%%%%%%%%%%%%%%%%%%%%%%%%%%%%%%%%%%%%%%%%%%%%%%%%%%%%%%%%%%%%%%%%%%%%%%%%%%
\subsection{Forwarding}
\label{sec:forward}

Different versions of the main or child documents
using compilation flags as described in \secref{sec:flags}
can be (permanently) stored in different files
for convenient compilation, viewing and distribution.
To this end, the package defines a command
to pass on compilation to a different file:

%%%%%%%%%%%%%%%%%%%%%%%%%%%%%%%%%%%%%%%%
\DescribeMacro{\childdocforward}
The command |\childdocforward| redirects processing to
another source file:
%
\begin{center}
\begin{tabular}{l}
|\input{childdoc.def}|\\
|\childdocforward[|\textit{main}|]{|\textit{dest}|}|\\
\end{tabular}
\end{center}
%
The argument \textit{dest} is the destination file
(without extension).
It should be the main file or one of the child files.
Note that further \textsf{childdoc} directives
such as |\childdocof| and |\childdocforward|
in the indicated file will be processed in this form.
The optional argument \textit{main}
passes on directly to the main file \textit{main}
while pretending to compile the child \textit{dest}.
This form behaves as if \textit{dest}
issues |\childdocof{|\textit{main}|}| right away,
and no further \textsf{childdoc} directives will be processed.

%%%%%%%%%%%%%%%%%%%%%%%%%%%%%%%%%%%%%%%%
\DescribeMacro{\...prefix}
In the alternative form |\childdocforwardprefix|,
%
\begin{center}
\begin{tabular}{l}
|\input{childdoc.def}|\\
|\childdocforwardprefix[|\textit{main}|]{|\textit{prefix}|}{|\textit{dest}|}|
\end{tabular}
\end{center}
%
the destination file is determined by a pattern
depending on the current file:
To make this work, the current file must be called
`{\textit{prefix}\hspace{0.2em}\textit{suffix}}'
with \textit{prefix} matching precisely the argument.
Processing is then passed on to the file
`{\textit{dest}\hspace{0.2em}\textit{suffix}}'.
Surely, the same effect is achieved by
directly specifying the
argument `{\textit{dest}\hspace{0.2em}\textit{suffix}}'
in the first form.
However, that requires to set up a different file
for each child. With the alternative form of the command
all these files can have exactly the same content
which simplifies setting them up and maintaining them.

For example, the following file |draft.tex|
with a compilation flag |\version| as described in \secref{sec:flags}
compiles the main document as a draft:
%
\begin{center}
\begin{tabular}{l}
|\def\version{draft}|\\
|\input{childdoc.def}|\\
|\childdocforward{|\textit{main}|}|
\end{tabular}
\end{center}
%
Likewise, the following files |final|\textit{nn}|.tex|
compile the final version of the child document
|child|\textit{nn}|.tex|:
%
\begin{center}
\begin{tabular}{l}
|\def\version{final}|\\
|\input{childdoc.def}|\\
|\childdocforwardprefix{final}{child}|
\end{tabular}
\end{center}
%

Note that when several versions of a main file and/or of each child file
are to be generated, it may be convenient to set up a |Makefile| or
shell script to automatise the process.

%%%%%%%%%%%%%%%%%%%%%%%%%%%%%%%%%%%%%%%%%%%%%%%%%%%%%%%%%%%%%%%%%%%%%%%%%%%%%%%%
\subsection{Command Line Processing}
\label{sec:commandline}

The effect of redirection files can also be achieved by invoking
the \LaTeX{} compiler with a more elaborate command line.
Most conveniently this should be done as part
of a shell script or a |Makefile|.

When using \textsf{childdoc} in the main file, the following
command lines effectively perform a redirection
(note that depending on the shell being used,
backslashes may have to be doubled: `|\|' $\to$ `|\\|'):
%
\begin{center}
|... -jobname "|\textit{target}|" |\\|"|[\textit{flags}]%
|\input{childdoc.def}\childdocforward[|\textit{main}|]{|\textit{dest}|}"|
\end{center}
%
Here \textit{target} is the name of the output file,
\textit{main} is the name of the main file
and \textit{dest} is the name of the main or child file to be processed
(all filenames without extensions).
The optional argument \textit{main} can be omitted
if \textit{main} matches \textit{dest}.
Optionally, compilation \textit{flags} can be defined via |\def| commands.
This command line makes the \TeX{} engine believe
it is compiling the file \textit{target}
whose content is specified as the latter parameter.
The provided code then forwards the processing to
\textit{main} or \textit{dest} as described in \secref{sec:forward}.

%%%%%%%%%%%%%%%%%%%%%%%%%%%%%%%%%%%%%%%%%%%%%%%%%%%%%%%%%%%%%%%%%%%%%%%%%%%%%%%%
\subsection{Include by Input}
\label{sec:input}

Including child documents by |\include| has some restrictions by design.
Most notably, the content of a child document always occupies
its own set of pages; pages cannot be shared between child documents.
Usually, this behaviour makes perfect sense
because each child document contain an essential part of the document.
However, in some situations it may be desirable to compose
a document from a collection of parts
without having mandatory page breaks between then.
For this case, the package
provides a mechanism to include parts
by |\input| which can also be processed individually.
However, by construction this mechanism
requires manual handling of the content to be output.

%%%%%%%%%%%%%%%%%%%%%%%%%%%%%%%%%%%%%%%%
\DescribeMacro{\ifchilddocmanual}
The main file should be prepared as usual, see \secref{sec:include}.
However, the document body must make a distinction
between processing of an individual part and of the main document, e.g.:
%
\begin{center}
\begin{tabular}{l}
|\ifchilddocmanual|\\
|\input{\childdocname}|\\
|\||else|\\
\textit{document body with }|\input{|\textit{part}|}|\\
|\||fi|
\end{tabular}
\end{center}
%
The conditional |\ifchilddocmanual| is true whenever
a part to be included by |\input| is being compiled,
and the name of the part is stored in |\childdocname|.

%%%%%%%%%%%%%%%%%%%%%%%%%%%%%%%%%%%%%%%%
\DescribeMacro{\childdocby}
Each part to be included by |\input| should start with:
%
\begin{center}
\begin{tabular}{l}
|\input{childdoc.def}|\\
|\childdocby{|\textit{main}|}|\\
\end{tabular}
\end{center}
%
The directive |\childdocby| is similar to |\childdocof|
described in \secref{sec:include},
but the subsequent selection of content must be done manually.
To that end, both |\ifchilddoc| and |\ifchilddocmanual|
will be true upon processing of a part,
and the name of the part is stored in |\childdocname|.
Note that |\jobname| will be set to the filename of the current part
so that each part receives an individual |.aux| file
that does not interfere with the |.aux| file(s) of the main document.
This behaviour can be altered by the alternative form
|\childdocby[*]{|\textit{main}|}| (with a non-empty optional argument)
which uses the |.aux| file of the main document
by setting |\jobname| to \textit{main}.

%%%%%%%%%%%%%%%%%%%%%%%%%%%%%%%%%%%%%%%%%%%%%%%%%%%%%%%%%%%%%%%%%%%%%%%%%%%%%%%%
\subsection{Driver Development}
\label{sec:driver}

The \textsf{childdoc} mechanism can also be use for the development
of definition files such as \LaTeX{} styles or classes.
This case differs from the above setup with multiple parts
included by |\include| in that no |\includeonly| should be invoked.
This can be achieved by starting the include file
(before |\ProvidesPackage|) with:
%
\begin{center}
\begin{tabular}{l}
|\input{childdoc.def}|\\
|\childdocforward{|\textit{main}|}|\\
\end{tabular}
\end{center}
%
or alternatively with:
%
\begin{center}
\begin{tabular}{l}
|\input{childdoc.def}|\\
|\childdocby{|\textit{main}|}|\\
\end{tabular}
\end{center}
%
Both forms have slightly different effects as described above.
The main file is prepared as usual, see \secref{sec:include}.

%%%%%%%%%%%%%%%%%%%%%%%%%%%%%%%%%%%%%%%%%%%%%%%%%%%%%%%%%%%%%%%%%%%%%%%%%%%%%%%%
\subsection{Legacy Detection}
\label{sec:detection}

The directive |\childdocmain| in the main file can detect
whether the complete document or merely a child is to be compiled
even without using the directive |\childdocof|.
This method is deprecated because it is less robust
and there is no compelling reason to use it;
it is merely provided for backward compatibility
and it may be removed in future versions.

If the detection mechanism is to be used,
it is mandatory to correctly specify
the filename of the main file as the argument of |\childdocmain|:
%
\begin{center}
\begin{tabular}{l}
|\input{childdoc.def}|\\
|\childdocmain{|\textit{main}|}|\\
\end{tabular}
\end{center}
%
If |\jobname| does not match the argument \textit{main} of |\childdocmain|,
it is assumed that |\jobname| points to the child file to be compiled.
When using |\childdocmain| with the main file specified as argument,
it suffices to start a child file
with just |\input{|\textit{main}|}|
without loading of the package and using |\childdocof|.
If instead all processing is done
with the appropriate \textsf{childdoc} directives,
the argument of \textit{main} of |\childdocmain| can be empty.

An alternative version of the command line processing described
in \secref{sec:commandline} using the detection mechanism reads:
%
\begin{center}
|... -jobname "|\textit{target}|" "|[\textit{flags}]%
[|\def\jobname{|\textit{dest}|}|]|\input{|\textit{main}|}"|
\end{center}

%%%%%%%%%%%%%%%%%%%%%%%%%%%%%%%%%%%%%%%%%%%%%%%%%%%%%%%%%%%%%%%%%%%%%%%%%%%%%%%%
\subsection{Manual Code}
\label{sec:manual}

In case one cannot be certain whether the definitions file |childdoc.def|
is installed on the target \TeX{} distribution
and one prefers not to ship it,
it is conceivable to paste a few relevant commands into the sources.

To that end, drop all statements |\input{childdoc.def}|
and perform the replacements as outlined below.
Instead of |\childdocmain{|\textit{main}|}| add the following code
to the top of the main file:
%
\begin{center}
\begin{tabular}{l}
|\||ifdefined\childdocname\endinput\||fi\newif\ifchilddoc|\\
|\edef\childdocname{\scantokens\expandafter{\jobname\noexpand}}|\\
|\def\childdocmain{|\textit{main}|}\||ifx\childdocmain\childdocname\||else|\\
|\childdoctrue\includeonly{\childdocname}\let\jobname\childdocmain\||fi|\\
\end{tabular}
\end{center}
%
Instead of |\childdocof{|\textit{main}|}| just include the main file
at the top of each child file:
%
\begin{center}
|\input{|\textit{main}|}|
\end{center}
%
A simple redirection |\childdocforward{|\textit{dest}|}| is achieved by:
%
\begin{center}
|\def\jobname{|\textit{dest}|}\input{\jobname}|
\end{center}
%
The redirection with prefix
|\childdocforwardprefix[|\textit{prefix}|]{|\textit{dest}|}|
is accomplished by:
%
\begin{center}
\begin{tabular}{l}
|{\edef\jobname{\scantokens\expandafter{\jobname\noexpand}}|\\
|\def\redirectjob |\textit{prefix}|#1~~~{\gdef\jobname{|\textit{dest}|#1}}|\\
|\expandafter\redirectjob\jobname~~~}\input{\jobname}|
\end{tabular}
\end{center}

In an alternative approach,
child documents can be compiled by a specific command line
without additional code or specific definitions:
%
\begin{center}
|... -jobname "|\textit{target}|" "|[\textit{flags}]%
|\includeonly{|\textit{dest}|}\input{|\textit{main}|}"|
\end{center}
%

%%%%%%%%%%%%%%%%%%%%%%%%%%%%%%%%%%%%%%%%%%%%%%%%%%%%%%%%%%%%%%%%%%%%%%%%%%%%%%%%
%%%%%%%%%%%%%%%%%%%%%%%%%%%%%%%%%%%%%%%%%%%%%%%%%%%%%%%%%%%%%%%%%%%%%%%%%%%%%%%%
\section{Information}

%%%%%%%%%%%%%%%%%%%%%%%%%%%%%%%%%%%%%%%%%%%%%%%%%%%%%%%%%%%%%%%%%%%%%%%%%%%%%%%%
\subsection{Copyright}

Copyright \copyright{} 2017--2018 Niklas Beisert

This work may be distributed and/or modified under the
conditions of the \LaTeX{} Project Public License, either version 1.3
of this license or (at your option) any later version.
The latest version of this license is in
  \url{http://www.latex-project.org/lppl.txt}
and version 1.3 or later is part of all distributions of \LaTeX{}
version 2005/12/01 or later.

This work has the LPPL maintenance status `maintained'.

The Current Maintainer of this work is Niklas Beisert.

This work consists of the files |README.txt|, |childdoc.ins| and |childdoc.dtx|
as well as the derived files |childdoc.def|, |cdocsamp.tex|
with |cdocsch1.tex|, |cdocsch2.tex|, |cdocspt3.tex|, |cdocspt4.tex|,
|cdocsdrf.tex|, |cdocsfn1.tex|, |cdocsfn2.tex|
as well as |childdoc.pdf|.

%%%%%%%%%%%%%%%%%%%%%%%%%%%%%%%%%%%%%%%%%%%%%%%%%%%%%%%%%%%%%%%%%%%%%%%%%%%%%%%%
\subsection{Files and Installation}

The package consists of the files:
%
\begin{center}
\begin{tabular}{ll}
    |README.txt|   & readme file \\
    |childdoc.ins| & installation file \\
    |childdoc.dtx| & source file \\
    |childdoc.def| & definition file \\
    |cdocsamp.tex| & sample main file \\
    |cdocsch1.tex| & sample include file \\
    |cdocsch2.tex| & sample include file \\
    |cdocspt3.tex| & sample part file \\
    |cdocspt4.tex| & sample part file \\
    |cdocsdrf.tex| & sample redirection file \\
    |cdocsfn1.tex| & sample redirection file \\
    |cdocsfn2.tex| & sample redirection file \\
    |childdoc.pdf| & manual
\end{tabular}
\end{center}
%
The distribution consists of the files
|README.txt|, |childdoc.ins| and |childdoc.dtx|.
%
\begin{itemize}
\item
Run (pdf)\LaTeX{} on |childdoc.dtx|
to compile the manual |childdoc.pdf| (this file).
\item
Run \LaTeX{} on |childdoc.ins| to create the definitions file |childdoc.def|
and the sample |cdocsamp.tex| with include files
|cdocsch1.tex|, |cdocsch2.tex|, |cdocspt3.tex|, |cdocspt4.tex|,
|cdocsdrf.tex|, |cdocsfn1.tex|, |cdocsfn2.tex|.
Then copy the file |childdoc.def| to an appropriate directory of your \LaTeX{}
distribution, e.g.\ \textit{texmf-root}|/tex/latex/childdoc|.
\end{itemize}

%%%%%%%%%%%%%%%%%%%%%%%%%%%%%%%%%%%%%%%%%%%%%%%%%%%%%%%%%%%%%%%%%%%%%%%%%%%%%%%%
\subsection{Related CTAN Packages}

There are several other packages which offer a similar functionality:
%
\begin{itemize}
\item
The packages
\href{http://ctan.org/pkg/docmute}{\textsf{docmute}},
\href{http://ctan.org/pkg/includex}{\textsf{includex}} and
\href{http://ctan.org/pkg/standalone}{\textsf{standalone}}
provide commands to include only the document body of
a child file thus allowing both files to be compiled individually.
\item
The packages \href{http://ctan.org/pkg/subdocs}{\textsf{subdocs}}
and \href{http://ctan.org/pkg/subfiles}{\textsf{subfiles}}
provide structures in which the main and child documents can be
encapsulated and allowing them to be compiled individually.
The inclusion mechanism is different from the conventional |\include|.
\item
The package \href{http://ctan.org/pkg/combine}{\textsf{combine}}
is an elaborate solution to combine several documents into one.
\end{itemize}
%
See also the CTAN topic \href{http://ctan.org/topic/subdocs}{\textsf{subdocs}}
for further related packages.
The present package differs from the above solutions in that
a document structure constructed with the conventional |\include| mechanism
just needs two extra commands at the top of every file
such that all constituent files can be compiled individually.

%%%%%%%%%%%%%%%%%%%%%%%%%%%%%%%%%%%%%%%%%%%%%%%%%%%%%%%%%%%%%%%%%%%%%%%%%%%%%%%%
%\subsection{Feature Suggestions}
%
%The following is a list of features which may be useful for future
%versions of this package:
%%
%\begin{itemize}
%\item
%\ldots
%\end{itemize}

%%%%%%%%%%%%%%%%%%%%%%%%%%%%%%%%%%%%%%%%%%%%%%%%%%%%%%%%%%%%%%%%%%%%%%%%%%%%%%%%
\subsection{Revision History}

%%%%%%%%%%%%%%%%%%%%%%%%%%%%%%%%%%%%%%%%
\paragraph{v2.0:} 2018/12/30

\begin{itemize}
\item
immediate forward processing
\item
added |\childdocby| mechanism
\item
manual restructured
\end{itemize}

%%%%%%%%%%%%%%%%%%%%%%%%%%%%%%%%%%%%%%%%
\paragraph{v1.6:} 2018/01/17

\begin{itemize}
\item
application for development of include files
\item
corrections to manual
\end{itemize}

%%%%%%%%%%%%%%%%%%%%%%%%%%%%%%%%%%%%%%%%
\paragraph{v1.5:} 2017/05/21

\begin{itemize}
\item
more complete structuring introduced
\item
|\childdocof| introduced
\item
|\childdoc| renamed to |\childdocmain|
\item
|\childredirect| renamed to |\childdocforward| and |\childdocforwardprefix|
and functionality expanded
\end{itemize}

%%%%%%%%%%%%%%%%%%%%%%%%%%%%%%%%%%%%%%%%
\paragraph{v1.0:} 2017/04/27

\begin{itemize}
\item
manual and install package
\item
first version published on CTAN
\end{itemize}

%%%%%%%%%%%%%%%%%%%%%%%%%%%%%%%%%%%%%%%%
\paragraph{v0.6:} 2017/04/26

\begin{itemize}
\item
redirection mechanism added
\end{itemize}

%%%%%%%%%%%%%%%%%%%%%%%%%%%%%%%%%%%%%%%%
\paragraph{v0.5:} 2017/04/26

\begin{itemize}
\item
functionality in definition file
\end{itemize}


%%%%%%%%%%%%%%%%%%%%%%%%%%%%%%%%%%%%%%%%%%%%%%%%%%%%%%%%%%%%%%%%%%%%%%%%%%%%%%%%
%%%%%%%%%%%%%%%%%%%%%%%%%%%%%%%%%%%%%%%%%%%%%%%%%%%%%%%%%%%%%%%%%%%%%%%%%%%%%%%%
%%%%%%%%%%%%%%%%%%%%%%%%%%%%%%%%%%%%%%%%%%%%%%%%%%%%%%%%%%%%%%%%%%%%%%%%%%%%%%%%
\appendix

\settowidth\MacroIndent{\rmfamily\scriptsize 000\ }

 \DocInput{childdoc.dtx}

\end{document}
%</driver>
% \fi
%
% %%%%%%%%%%%%%%%%%%%%%%%%%%%%%%%%%%%%%%%%%%%%%%%%%%%%%%%%%%%%%%%%%%%%%%%%%%%%%%
% %%%%%%%%%%%%%%%%%%%%%%%%%%%%%%%%%%%%%%%%%%%%%%%%%%%%%%%%%%%%%%%%%%%%%%%%%%%%%%
% \section{Sample}
%\iffalse
%<*samplemain>
%\fi
%
% The following presents a sample document
% with two chapters, two parts, a title page,
% a compile flag as well as three forwarding files to set the flag.
% It consists of eight |.tex| files:
% \begin{center}
% \begin{tabular}{ll}
% |cdocsamp.tex|&main file\\
% |cdocsch1.tex|&include file for chapter 1\\
% |cdocsch2.tex|&include file for chapter 2\\
% |cdocspt3.tex|&include file for part 3\\
% |cdocspt4.tex|&include file for part 4\\
% |cdocsdrf.tex|&forwarding file for main file in draft mode\\
% |cdocsfi1.tex|&forwarding file for final version of chapter 1\\
% |cdocsfi2.tex|&forwarding file for final version of chapter 2\\
% \end{tabular}
% \end{center}
% Each of the eight files can be compiled directly by the \LaTeX{} compiler.
%
% %%%%%%%%%%%%%%%%%%%%%%%%%%%%%%%%%%%%%%
% \paragraph{Main File.}
%
% The main file is called |cdocsamp.tex|.
%
% Load the \textsf{childdoc} definitions and
% declare the filename for the main document:
%    \begin{macrocode}
\input{childdoc.def}
\childdocmain{}
%    \end{macrocode}

% Optional override for |\version| flag:
%    \begin{macrocode}
%%\ifchilddoc\else\providecommand{\version}{draft}\fi
%    \end{macrocode}

% Define the default values for the |\version| flag
% (|final| for the main file and |draft| for childs):
%    \begin{macrocode}
\ifchilddoc
\providecommand{\version}{draft}
\else
\providecommand{\version}{final}
\fi
%    \end{macrocode}

% Load the standard document class:
%    \begin{macrocode}
\documentclass[12pt]{article}
%    \end{macrocode}

% Start the document body:
%    \begin{macrocode}
\begin{document}
%    \end{macrocode}

% Declare a title page.
% Print title, part of document being processed and version flag:
%    \begin{macrocode}
\addtocounter{page}{-1}
\begin{center}
{\LARGE\bfseries{}childdoc example\par}
\vspace{1cm}
\ifchilddoc
\ifchilddocmanual part\else chapter\fi:
`\childdocname' of `\childdocjob'\par
\else
main document: `\childdocjob'\par
\fi
version: \version\par
\end{center}
\newpage
%    \end{macrocode}

% Manually include selected file,
% otherwise process as usual:
%    \begin{macrocode}
\ifchilddocmanual
\section*{part `\childdocname'}
\input{\childdocname}
\else
%    \end{macrocode}

% Include the two chapters:
%    \begin{macrocode}
\include{cdocsch1}
\include{cdocsch2}
%    \end{macrocode}

% Include the two parts unless only chapters should be displayed:
%    \begin{macrocode}
\ifchilddoc\else
\section{part three}
\input{cdocspt3}
\section{part four}
\input{cdocspt4}
\fi
%    \end{macrocode}

% Process as usual until here:
%    \begin{macrocode}
\fi
%    \end{macrocode}

% End of document body:
%    \begin{macrocode}
\end{document}
%    \end{macrocode}
%\iffalse
%</samplemain>
%\fi
%
% %%%%%%%%%%%%%%%%%%%%%%%%%%%%%%%%%%%%%%
% \paragraph{Chapter Include Files.}
%
% The include files are called |cdocsch1.tex| and |cdocsch2.tex|.
%
%\iffalse
%<*samplechap1|samplechap2>
%\fi

% Optional override for |\version| flag:
%    \begin{macrocode}
%%\providecommand{\version}{final}
%    \end{macrocode}

% Include the main document:
%    \begin{macrocode}
\input{childdoc.def}
\childdocof{cdocsamp}
%    \end{macrocode}

%\iffalse
%</samplechap1|samplechap2>
%\fi
%
%\iffalse
%<*samplechap1>
%\fi
% Some text for chapter 1:
%    \begin{macrocode}
\section{one}
some text in chapter one
%    \end{macrocode}

%\iffalse
%</samplechap1>
%\fi
% Some text for chapter 2:
%\iffalse
%<*samplechap2>
%\fi
%    \begin{macrocode}
\section{two}
more text in chapter two
%    \end{macrocode}

%\iffalse
%</samplechap2>
%\fi
%
% %%%%%%%%%%%%%%%%%%%%%%%%%%%%%%%%%%%%%%
% \paragraph{Part Include Files.}
%
% The include files are called |cdocspt3.tex| and |cdocspt4.tex|.
%
%\iffalse
%<*samplepart3|samplepart4>
%\fi

% Optional override for |\version| flag:
%    \begin{macrocode}
%%\providecommand{\version}{final}
%    \end{macrocode}

% Include the main document:
%    \begin{macrocode}
\input{childdoc.def}
\childdocby{cdocsamp}
%    \end{macrocode}

%\iffalse
%</samplepart3|samplepart4>
%\fi
%
%\iffalse
%<*samplepart3>
%\fi
% Some text for part 3:
%    \begin{macrocode}
some text in part three
%    \end{macrocode}

%\iffalse
%</samplepart3>
%\fi
% Some text for part 4:
%\iffalse
%<*samplepart4>
%\fi
%    \begin{macrocode}
more text in part four
%    \end{macrocode}

%\iffalse
%</samplepart4>
%\fi
%
% %%%%%%%%%%%%%%%%%%%%%%%%%%%%%%%%%%%%%%
% \paragraph{Forwarding for a Complete Draft.}
%
% The following forwarding file |cdocsdrf.tex|
% compiles the main document in draft mode:
%\iffalse
%<*sampledraft>
%\fi
%    \begin{macrocode}
\def\version{draft}
\input{childdoc.def}
\childdocforward{cdocsamp}
%    \end{macrocode}

%\iffalse
%</sampledraft>
%\fi
%
% %%%%%%%%%%%%%%%%%%%%%%%%%%%%%%%%%%%%%%
% \paragraph{Forwarding for Final Version of the Chapters.}
%
% The following forwarding files |cdocsfn1.tex| and |cdocsfn2.tex|
% (with identical content)
% compile the final versions of the child documents
% |cdocsch1.tex| and |cdocsch2.tex|, respectively:
%\iffalse
%<*samplefinal>
%\fi
%    \begin{macrocode}
\def\version{final}
\input{childdoc.def}
\childdocforwardprefix[cdocsamp]{cdocsfn}{cdocsch}
%    \end{macrocode}

%\iffalse
%</samplefinal>
%\fi
%
% %%%%%%%%%%%%%%%%%%%%%%%%%%%%%%%%%%%%%%
% \paragraph{Command Line Processing.}
%
% The following three command lines generate the output files
% |cdocscld|, |cdocscl1| and |cdocscl2|
% which should be identical to
% |cdocsdrf|, |cdocsch1| and |cdocsfn2|, respectively:
% \begin{center}
% \begin{tabular}{l}
% |latex -jobname cdocscld \|\\
% |  "\def\version{draft}\input{childdoc.def}\childdocforward{cdocsamp}"|\\
% |latex -jobname cdocscl1 \|\\
% |  "\input{childdoc.def}\childdocforward[cdocsamp]{cdocsch1}"|\\
% |latex -jobname cdocscl2 \|\\
% |  "\def\version{final}\input{childdoc.def}\childdocforward{cdocsch2}"|
% \end{tabular}
% \end{center}
% Note that the trailing backslash on each first line
% merely continues the input to the second line
% (for convenient cut ant paste).
% Furthermore, the command |latex| can be replaced by any
% of its alternative versions such as |pdflatex|.
%
% %%%%%%%%%%%%%%%%%%%%%%%%%%%%%%%%%%%%%%%%%%%%%%%%%%%%%%%%%%%%%%%%%%%%%%%%%%%%%%
% %%%%%%%%%%%%%%%%%%%%%%%%%%%%%%%%%%%%%%%%%%%%%%%%%%%%%%%%%%%%%%%%%%%%%%%%%%%%%%
% \section{Implementation}
%\iffalse
%<*package>
%\fi
%
% This section describes the definitions file |childdoc.def|.

% The definitions cannot be loaded using |\usepackage| or |\RequirePackage|
% which has a mechanism to prevent loading a style file more than once.
% When loading the definitions by means of |\input|
% multiple instances have to be prevented manually:
%\iffalse
%This code needs to be before the `\ProvidesFile' directive
%which is defined at the beginning of this file.
%Therefore it is also placed there and commented out here.
%</package>
%<*discard>
%\fi
%    \begin{macrocode}
\ifdefined\childdocmain\endinput\fi
%    \end{macrocode}
%\iffalse
%</discard>
%<*package>
%\fi
%
% \macro{\ifchilddoc}
% \macro{\ifchilddocmanual}
% The conditional |\ifchilddoc| tells whether a
% child (true) or main (false) document is being compiled.
% The conditional |\ifchilddocmanual| tells whether
% the |\includeonly| mechanism is used (false) or
% the selection of child files must be performed manually (true).
% The definitions initialise to false:
%    \begin{macrocode}
\newif\ifchilddoc
\newif\ifchilddocmanual
%    \end{macrocode}

% \macro{\childdocname}
% \macro{\childdocjob}
% The macro |\childdocname| stores the name of the main document
% to be compiled. The macro |\childdocjob| stores the name of
% the document on which the \LaTeX{} compiler was originally invoked.
% The content of |\jobname| cannot be compared
% to filenames specified in the source due to different catcodes.
% The following code rescans |\jobname|, stores the result
% in |\childdocname| and saves a copy in |\childdocjob|:
%    \begin{macrocode}
\edef\childdocname{\scantokens\expandafter{\jobname\noexpand}}
\let\childdocjob\childdocname
%    \end{macrocode}

% \macro{\childdocdisable}
% The macro |\childdocdisable| prevents the main file
% from being processed more than once.
% At this stage, the main document command |\childdocmain|
% is assumed to be called once again where it should do nothing.
% Any subsequent call to it should prevent
% a secondary processing of the main document
% It overwrites the forwarding commands
% |\childdocof| and |\childdocforward|
% with empty macros to prevent further inclusions of the main document:
%    \begin{macrocode}
\newcommand{\childdocdisable}
{
  \renewcommand{\childdocmain}[1]{\renewcommand{\childdocmain}[1]{\endinput}}
  \renewcommand{\childdocof}[1]{}
  \renewcommand{\childdocby}[2][]{}
  \renewcommand{\childdocforward}[2][]{}
  \renewcommand{\childdocdisable}{}
}
%    \end{macrocode}

% \macro{\childdocmain}
% The macro |\childdocmain| is to be called at the top of the main file
% with nothing or the main filename (without extension) as argument.
% First, it breaks loops.
% If the argument is not empty and does not match |\childdocname|
% (which is set by the first inclusion of |childdoc.def|),
% |\ifchilddoc| is set to true, |\includeonly| is applied to the child file
% and |\jobname| is set to the main file
% (for proper handling of |.aux| files):
%    \begin{macrocode}
\newcommand{\childdocmain}[1]
{
  \childdocdisable\childdocmain{}
  \if?#1?\else
    \begingroup
      \def\childdoctmp{#1}
      \ifx\childdoctmp\childdocname
        \def\childdoctmp{}
      \else
        \def\childdoctmp
        {
          \childdoctrue
          \includeonly{\childdocname}
          \def\childdocjob{#1}
          \def\jobname{#1}
        }
      \fi
      \expandafter
    \endgroup
    \childdoctmp
  \fi
}
%    \end{macrocode}

% \macro{\childdocof}
% The command |\childdocof| redirects
% compilation to the main file |#1|.
%    \begin{macrocode}
\newcommand{\childdocof}[1]
{
  \childdocdisable
  \childdoctrue
  \includeonly{\childdocname}
  \def\jobname{#1}
  \def\childdocjob{#1}
  \input{#1}
}
%    \end{macrocode}

% \macro{\childdocby}
% The command |\childdocby| ....
%    \begin{macrocode}
\newcommand{\childdocby}[2][]
{
  \childdocdisable
  \childdoctrue
  \childdocmanualtrue
  \if?#1?\else
    \def\jobname{#2}
  \fi
  \def\childdocjob{#2}
  \input{#2}
  \endinput
}
%    \end{macrocode}

% \macro{\childdocforward}
% The command |\childdocforward| redirects
% compilation to the main file or
% (if the optional argument is given) a child file.
% Parameters are set as if the main file
% or a child file starting with |\childdocof| was compiled.
% Then compilation is handed over to the main file:
%    \begin{macrocode}
\newcommand{\childdocforward}[2][]
{
  \begingroup
    \if?#1?
      \def\childdoctmp
      {
        \def\childdocname{#2}
        \def\childdocjob{#2}
        \def\jobname{#2}
        \input{#2}
        \endinput
      }
    \else
      \def\childdoctmp
      {
        \childdocdisable
        \def\childdocname{#2}
        \childdoctrue
        \includeonly{#2}
        \def\childdocjob{#1}
        \def\jobname{#1}
        \input{#1}
        \endinput
      }
    \fi
    \expandafter
  \endgroup
  \childdoctmp
}
%    \end{macrocode}

% \macro{\childdocforwardprefix}
% The command |\childdocforwardprefix| redirects
% compilation to the main or a child file by means of a pattern.
% The prefix |#1| in the current filename is replaced by |#2|
% and the suffix of the current filename is kept
% (it is assumed that the filename does not contain the substring `|~~~|'
% which is used as a delimiter).
% Compilation is handed over to the new file by |\childdocforward|:
%    \begin{macrocode}
\newcommand{\childdocforwardprefix}[3][]
{
  \begingroup
    \def\childdocextract #2##1~~~{\def\childdoctmp{\childdocforward[#1]{#3##1}}}
    \expandafter\childdocextract\childdocname~~~
    \expandafter
  \endgroup
  \childdoctmp
}
%    \end{macrocode}

% \macro{\childdoc}
% The deprecated macro |\childdoc| is a legacy version of |\childdocmain|:
%    \begin{macrocode}
\newcommand{\childdoc}{\childdocmain}
%    \end{macrocode}

% \macro{\childdocredirect}
% The deprecated macro |\childdocredirect| is a legacy version
% of |\childdocforward| and |\childdocforwardprefix|:
%    \begin{macrocode}
\newcommand{\childdocredirect}[2][]
{
  \begingroup
    \if?#1?
      \def\childdoctmp{\childdocforward{#2}}
    \else
      \def\childdoctmp{\childdocforwardprefix{#1}{#2}}
    \fi
    \expandafter
  \endgroup
  \childdoctmp
}
%    \end{macrocode}

%\iffalse
%</package>
%\fi
%
\endinput
|\\
|\childdocof{|\textit{main}|}|\\
\end{tabular}
\end{center}
at the top of every child file \textit{child}
which is included by |\include{|\textit{child}|}|
from within the main file
(or at least for those files to be compiled individually).
The argument \textit{main} must be the filename of the main file.

There are a couple of
considerations in setting up the main and child documents:

%%%%%%%%%%%%%%%%%%%%%%%%%%%%%%%%%%%%%%%%
\paragraph{Restrictions.}

Please note the following restrictions:
\begin{itemize}
\item
|\childdocmain| must be called with one argument \textit{main}
to ensure compatibility with earlier version of the package.
It must either be empty (|\childdocmain{}|)
or precisely match the filename of the main file in which it is specified.
See \secref{sec:detection} for further information.
\item
The filename \textit{main} must be specified without the |.tex| extension.
\item
The filename \textit{main} is case sensitive
(even in case-insensitive file systems)
due to internal string comparison.
\item
The argument \textit{main} should be fully expanded, it cannot be a macro.
\item
Subdirectories and special characters should be avoided in filenames.
\item
The command |\childdocmain{|\textit{main}|}| must be followed by a whitespace.
It should not be followed immediately by another command
or by a comment mark `|%|'.
This is because the \TeX{} parser reads the token immediately following
the argument of |\childdocmain| and puts it
at the beginning of every child section;
however, a white\-space is ignored.
\end{itemize}

%%%%%%%%%%%%%%%%%%%%%%%%%%%%%%%%%%%%%%%%
\paragraph{Content of Main File.}

It is advisable to place all content in the child files included by |\include|.
Any output contained in the main file will appear in all child documents
unless suppressed manually;
it cannot be suppressed automatically by the |\includeonly| directive
and thus should normally be avoided.
A method to include some content in the main file
by means of conditional processing is described in \secref{sec:conditional}.

%%%%%%%%%%%%%%%%%%%%%%%%%%%%%%%%%%%%%%%%
\paragraph{Page Numbering.}

When only a part of the document is compiled,
the appropriate numbering of pages
(as well as other status parameters)
is determined from the |.aux| files.
The latter contain information from previous passes.
However this information needs to propagate through
all intermediate child documents.
Therefore the page numbering in child documents may well
be inconsistent until the complete document is compiled at least once.

A useful (if unconventional) way to always ensure a consistent
page numbering is to restart the numbering in each child document
and denote the pages by `\textit{child}|.|\textit{page}'
where \textit{child} represents the chapter/section number of the child file.
This can be achieved by the command
|\numberwithin{page}{|\textit{child}|}|
of the \textsf{amsmath} package
where \textit{child} can be |chapter| or |section|
depending on the chosen structuring.
Alternatively, one can modify the macro |\thepage| appropriately
and reset the counter |page| at the start of each child file.

%%%%%%%%%%%%%%%%%%%%%%%%%%%%%%%%%%%%%%%%%%%%%%%%%%%%%%%%%%%%%%%%%%%%%%%%%%%%%%%%
\subsection{Conditional Processing}
\label{sec:conditional}

The package provides a mechanism to compile different versions
of a document. To customise the versions further some conditional processing
can come in handy to distinguish which version is being compiled.
The package provides two macros to describe the compilation context:

%%%%%%%%%%%%%%%%%%%%%%%%%%%%%%%%%%%%%%%%
\DescribeMacro{\ifchilddoc}
The conditional |\ifchilddoc| distinguishes between the compilation of
child documents and the main document:
%
\begin{center}
|\ifchilddoc |\textit{child-code}| |[|\||else |\textit{main-code}]| \||fi|
\end{center}

%%%%%%%%%%%%%%%%%%%%%%%%%%%%%%%%%%%%%%%%
\DescribeMacro{\childdocname}
\DescribeMacro{\childdocjob}
The macro |\childdocname| contains the filename (without extension)
of the main or child file being processed.
Note that |\childdocjob| will always contain the name of the main file.

%%%%%%%%%%%%%%%%%%%%%%%%%%%%%%%%%%%%%%%%
\paragraph{Title Page.}

Conditional processing can be used to include a title or banner page
in the main document when proper precautions are taken.
Importantly, the code in the main file should ensure that the page counter
(as well as other status parameters which are stored in the |.aux| files)
takes the same value after the conditional processing.
Otherwise the page numbers may take divergent values
depending on which part is compiled.

For example, a title page could be declared by:
%
\begin{center}
\begin{tabular}{l}
|\ifchilddoc\||else|\\
|\addtocounter{page}{-1}|\\
\textit{code for title page}\\
|\newpage|\\
|\||fi|
\end{tabular}
\end{center}
%
A banner page for the child documents can be generated by:
%
\begin{center}
\begin{tabular}{l}
|\ifchilddoc|\\
|\addtocounter{page}{-1}|\\
\textit{code for banner page}\\
|\newpage|\\
|\||fi|
\end{tabular}
\end{center}
%
Here one could write a message such as:
\begin{center}
|This is the part \childdocname{} of \childdocjob{}.|
\end{center}

%%%%%%%%%%%%%%%%%%%%%%%%%%%%%%%%%%%%%%%%%%%%%%%%%%%%%%%%%%%%%%%%%%%%%%%%%%%%%%%%
\subsection{Flags}
\label{sec:flags}

The package makes it easy to generate different versions
of the main or child documents.
To this end compilation flags can be defined
and assigned different default values.
They will be particularly useful in conjunction
with the forwarding mechanism described in \secref{sec:forward}.

For example, it may be useful to have a flag |\version|
which can be set to |draft| or |final|.
The document source will contain some conditional code
depending on the value of |\version|.
Suppose further, the flag should default to |final| for the main file
and to |draft| for child files
which is a natural assignment for editing the document.
This is achieved by placing the following code
in the preamble of the main document
(below the |\childdocmain| directive):
%
\begin{center}
\begin{tabular}{l}
|\ifchilddoc|\\
|\providecommand{\version}{draft}|\\
|\||else|\\
|\providecommand{\version}{final}|\\
|\||fi|
\end{tabular}
\end{center}
%
The definition by |\providecommand| makes sure
that previous definitions are not overwritten.
Further statements |\providecommand{\version}{...}|
can thus be added before the above code to override it.

For the main file, one might add a line
(between |\childdocmain| and the above block)
%
\begin{center}
|%\ifchilddoc\||else\providecommand{\version}{draft}\||fi|
\end{center}
%
which can be uncommented to produce a draft version.
Likewise one can add a line to the very top of a child file
(above the |\childdocof{|\textit{main}|}| directive)
%
\begin{center}
|%\providecommand{\version}{final}|
\end{center}
%
which can be uncommented to produce the final version of this child document.

%%%%%%%%%%%%%%%%%%%%%%%%%%%%%%%%%%%%%%%%%%%%%%%%%%%%%%%%%%%%%%%%%%%%%%%%%%%%%%%%
\subsection{Forwarding}
\label{sec:forward}

Different versions of the main or child documents
using compilation flags as described in \secref{sec:flags}
can be (permanently) stored in different files
for convenient compilation, viewing and distribution.
To this end, the package defines a command
to pass on compilation to a different file:

%%%%%%%%%%%%%%%%%%%%%%%%%%%%%%%%%%%%%%%%
\DescribeMacro{\childdocforward}
The command |\childdocforward| redirects processing to
another source file:
%
\begin{center}
\begin{tabular}{l}
|% \iffalse
%
% childdoc.dtx Copyright (C) 2017-2018 Niklas Beisert
%
% This work may be distributed and/or modified under the
% conditions of the LaTeX Project Public License, either version 1.3
% of this license or (at your option) any later version.
% The latest version of this license is in
%   http://www.latex-project.org/lppl.txt
% and version 1.3 or later is part of all distributions of LaTeX
% version 2005/12/01 or later.
%
% This work has the LPPL maintenance status `maintained'.
%
% The Current Maintainer of this work is Niklas Beisert.
%
% This work consists of the files childdoc.dtx and childdoc.ins
% and the derived files childdoc.def and cdocsamp.tex with
% cdocsch1.tex, cdocsch2.tex, cdocsdrf.tex, cdocsfn1.tex, cdocsfn2.tex.
%
%<package>\ifdefined\childdocmain\endinput\fi
%<package>\ProvidesFile{childdoc.def}[2018/12/30 v2.0 child document driver]
%<samplemain>\ProvidesFile{cdocsamp.tex}[2018/12/30 v2.0 sample for childdoc]
%<*driver>
%\ProvidesFile{childdoc.drv}[2018/12/30 v2.0 childdoc reference manual file]
\PassOptionsToClass{10pt,a4paper}{article}
\documentclass{ltxdoc}

\usepackage[margin=35mm]{geometry}
\usepackage{hyperref}
\usepackage{hyperxmp}
\usepackage[usenames]{color}

\hypersetup{colorlinks=true}
\hypersetup{pdfstartview=FitH}
\hypersetup{pdfpagemode=UseNone}
\hypersetup{pdfsource={}}
\hypersetup{pdflang={en-UK}}
\hypersetup{pdfcopyright={Copyright 2017-2018 Niklas Beisert.
  This work may be distributed and/or modified under the
  conditions of the LaTeX Project Public License, either version 1.3
  of this license or (at your option) any later version.}}
\hypersetup{pdflicenseurl={http://www.latex-project.org/lppl.txt}}
\hypersetup{pdfcontactaddress={ETH Zurich, ITP, HIT K,
  Wolfgang-Pauli-Strasse 27}}
\hypersetup{pdfcontactpostcode={8093}}
\hypersetup{pdfcontactcity={Zurich}}
\hypersetup{pdfcontactcountry={Switzerland}}
\hypersetup{pdfcontactemail={nbeisert@itp.phys.ethz.ch}}
\hypersetup{pdfcontacturl={http://people.phys.ethz.ch/\xmptilde nbeisert/}}

\newcommand{\secref}[1]{\hyperref[#1]{section \ref*{#1}}}

\parskip1ex
\parindent0pt
\let\olditemize\itemize
\def\itemize{\olditemize\parskip0pt}

\begin{document}

\title{The \textsf{childdoc} Package}
\hypersetup{pdftitle={The childdoc Package}}
\author{Niklas Beisert\\[2ex]
  Institut f\"ur Theoretische Physik\\
  Eidgen\"ossische Technische Hochschule Z\"urich\\
  Wolfgang-Pauli-Strasse 27, 8093 Z\"urich, Switzerland\\[1ex]
  \href{mailto:nbeisert@itp.phys.ethz.ch}
  {\texttt{nbeisert@itp.phys.ethz.ch}}}
\hypersetup{pdfauthor={Niklas Beisert}}
\hypersetup{pdfsubject={Manual for the LaTeX2e Package childdoc}}
\date{30 December 2018, \textsf{v2.0}}
\maketitle

\begin{abstract}\noindent
\textsf{childdoc} is a \LaTeXe{} package
that enables the direct compilation
of document sections included by |\include|
to individual files.
\end{abstract}

\begingroup
\parskip0ex
\tableofcontents
\endgroup

%%%%%%%%%%%%%%%%%%%%%%%%%%%%%%%%%%%%%%%%%%%%%%%%%%%%%%%%%%%%%%%%%%%%%%%%%%%%%%%%
%%%%%%%%%%%%%%%%%%%%%%%%%%%%%%%%%%%%%%%%%%%%%%%%%%%%%%%%%%%%%%%%%%%%%%%%%%%%%%%%
\section{Introduction}

\LaTeX{} provides a mechanism to structure a large document (such as a book)
into a main file and several child files (containing the chapters)
using the |\include| command.
This mechanism is beneficial for documents
which span hundreds of pages in order to
make the source file(s) more manageable.
Moreover, compilation can be restricted to
selected child files by means of the |\includeonly| command.
The latter feature can be used to reduce the compilation time while editing
(this was significantly more useful in the earlier days of \LaTeX{})
or to generate a smaller document which is easier to navigate.
Another application of |\includeonly| is to generate
documents consisting of selected parts of the complete document.

However, there are a few drawbacks of the plain |\include| mechanism:
\begin{itemize}
\item
The child files cannot be compiled on their own,
they can only be compiled via the main file.
A naive editing environment
(such as a text editor with an option
to have the current file processed by \LaTeX)
may require one to switch to the main file before compiling;
attempting to compile the child file produces errors.
\item
The main file must be modified (each time)
to adjust the |\includeonly| command
to the present needs. This easily leaves the main file in a messy state.
\item
The generated document will always carry the filename
of the main document. This is inconvenient if
several child files are to be compiled and
to be kept for distribution.
\end{itemize}

The present package provides a simple interface
to make child files individually compilable by \LaTeX{}.
Compiling a child file then has the same effect as compiling
the main file with an |\includeonly| command
to select the appropriate child.
Moreover the generated document will carry the name of the child
rather than the main file.
This resolves all three above issues.

This feature is meant to make the editing of books,
thesis documents and lecture notes somewhat more convenient.
However, the package can also be used efficiently for
composing a series of documents (such as exercise sheets)
which are typically distributed individually.
It then assists the author in generating the individual documents
(potentially in different versions)
as well as a document containing the collected series.
Another application is in developing style files
or other kinds of included material
where compilation of the style file could redirect
to a sample or test file.

%%%%%%%%%%%%%%%%%%%%%%%%%%%%%%%%%%%%%%%%%%%%%%%%%%%%%%%%%%%%%%%%%%%%%%%%%%%%%%%%
%%%%%%%%%%%%%%%%%%%%%%%%%%%%%%%%%%%%%%%%%%%%%%%%%%%%%%%%%%%%%%%%%%%%%%%%%%%%%%%%
\section{Usage}

First of all, the package \textsf{childdoc} is \emph{not} a standard
\LaTeXe{} |.sty| style file! Therefore it needs to be invoked in
a non-standard way.

%%%%%%%%%%%%%%%%%%%%%%%%%%%%%%%%%%%%%%%%%%%%%%%%%%%%%%%%%%%%%%%%%%%%%%%%%%%%%%%%
\subsection{Included Files}
\label{sec:include}

%%%%%%%%%%%%%%%%%%%%%%%%%%%%%%%%%%%%%%%%
\DescribeMacro{\childdocmain}
To use the package, add the commands
\begin{center}
\begin{tabular}{l}
|\input{childdoc.def}|\\
|\childdocmain{}|\\
\end{tabular}
\end{center}
at the very top of the main \LaTeX{} file,
in particular \emph{before} the |\documentclass| statement!
The argument of |\childdocmain| should be left empty
(but it must be present).

%%%%%%%%%%%%%%%%%%%%%%%%%%%%%%%%%%%%%%%%
\DescribeMacro{\childdocof}
Furthermore, add the commands
\begin{center}
\begin{tabular}{l}
|\input{childdoc.def}|\\
|\childdocof{|\textit{main}|}|\\
\end{tabular}
\end{center}
at the top of every child file \textit{child}
which is included by |\include{|\textit{child}|}|
from within the main file
(or at least for those files to be compiled individually).
The argument \textit{main} must be the filename of the main file.

There are a couple of
considerations in setting up the main and child documents:

%%%%%%%%%%%%%%%%%%%%%%%%%%%%%%%%%%%%%%%%
\paragraph{Restrictions.}

Please note the following restrictions:
\begin{itemize}
\item
|\childdocmain| must be called with one argument \textit{main}
to ensure compatibility with earlier version of the package.
It must either be empty (|\childdocmain{}|)
or precisely match the filename of the main file in which it is specified.
See \secref{sec:detection} for further information.
\item
The filename \textit{main} must be specified without the |.tex| extension.
\item
The filename \textit{main} is case sensitive
(even in case-insensitive file systems)
due to internal string comparison.
\item
The argument \textit{main} should be fully expanded, it cannot be a macro.
\item
Subdirectories and special characters should be avoided in filenames.
\item
The command |\childdocmain{|\textit{main}|}| must be followed by a whitespace.
It should not be followed immediately by another command
or by a comment mark `|%|'.
This is because the \TeX{} parser reads the token immediately following
the argument of |\childdocmain| and puts it
at the beginning of every child section;
however, a white\-space is ignored.
\end{itemize}

%%%%%%%%%%%%%%%%%%%%%%%%%%%%%%%%%%%%%%%%
\paragraph{Content of Main File.}

It is advisable to place all content in the child files included by |\include|.
Any output contained in the main file will appear in all child documents
unless suppressed manually;
it cannot be suppressed automatically by the |\includeonly| directive
and thus should normally be avoided.
A method to include some content in the main file
by means of conditional processing is described in \secref{sec:conditional}.

%%%%%%%%%%%%%%%%%%%%%%%%%%%%%%%%%%%%%%%%
\paragraph{Page Numbering.}

When only a part of the document is compiled,
the appropriate numbering of pages
(as well as other status parameters)
is determined from the |.aux| files.
The latter contain information from previous passes.
However this information needs to propagate through
all intermediate child documents.
Therefore the page numbering in child documents may well
be inconsistent until the complete document is compiled at least once.

A useful (if unconventional) way to always ensure a consistent
page numbering is to restart the numbering in each child document
and denote the pages by `\textit{child}|.|\textit{page}'
where \textit{child} represents the chapter/section number of the child file.
This can be achieved by the command
|\numberwithin{page}{|\textit{child}|}|
of the \textsf{amsmath} package
where \textit{child} can be |chapter| or |section|
depending on the chosen structuring.
Alternatively, one can modify the macro |\thepage| appropriately
and reset the counter |page| at the start of each child file.

%%%%%%%%%%%%%%%%%%%%%%%%%%%%%%%%%%%%%%%%%%%%%%%%%%%%%%%%%%%%%%%%%%%%%%%%%%%%%%%%
\subsection{Conditional Processing}
\label{sec:conditional}

The package provides a mechanism to compile different versions
of a document. To customise the versions further some conditional processing
can come in handy to distinguish which version is being compiled.
The package provides two macros to describe the compilation context:

%%%%%%%%%%%%%%%%%%%%%%%%%%%%%%%%%%%%%%%%
\DescribeMacro{\ifchilddoc}
The conditional |\ifchilddoc| distinguishes between the compilation of
child documents and the main document:
%
\begin{center}
|\ifchilddoc |\textit{child-code}| |[|\||else |\textit{main-code}]| \||fi|
\end{center}

%%%%%%%%%%%%%%%%%%%%%%%%%%%%%%%%%%%%%%%%
\DescribeMacro{\childdocname}
\DescribeMacro{\childdocjob}
The macro |\childdocname| contains the filename (without extension)
of the main or child file being processed.
Note that |\childdocjob| will always contain the name of the main file.

%%%%%%%%%%%%%%%%%%%%%%%%%%%%%%%%%%%%%%%%
\paragraph{Title Page.}

Conditional processing can be used to include a title or banner page
in the main document when proper precautions are taken.
Importantly, the code in the main file should ensure that the page counter
(as well as other status parameters which are stored in the |.aux| files)
takes the same value after the conditional processing.
Otherwise the page numbers may take divergent values
depending on which part is compiled.

For example, a title page could be declared by:
%
\begin{center}
\begin{tabular}{l}
|\ifchilddoc\||else|\\
|\addtocounter{page}{-1}|\\
\textit{code for title page}\\
|\newpage|\\
|\||fi|
\end{tabular}
\end{center}
%
A banner page for the child documents can be generated by:
%
\begin{center}
\begin{tabular}{l}
|\ifchilddoc|\\
|\addtocounter{page}{-1}|\\
\textit{code for banner page}\\
|\newpage|\\
|\||fi|
\end{tabular}
\end{center}
%
Here one could write a message such as:
\begin{center}
|This is the part \childdocname{} of \childdocjob{}.|
\end{center}

%%%%%%%%%%%%%%%%%%%%%%%%%%%%%%%%%%%%%%%%%%%%%%%%%%%%%%%%%%%%%%%%%%%%%%%%%%%%%%%%
\subsection{Flags}
\label{sec:flags}

The package makes it easy to generate different versions
of the main or child documents.
To this end compilation flags can be defined
and assigned different default values.
They will be particularly useful in conjunction
with the forwarding mechanism described in \secref{sec:forward}.

For example, it may be useful to have a flag |\version|
which can be set to |draft| or |final|.
The document source will contain some conditional code
depending on the value of |\version|.
Suppose further, the flag should default to |final| for the main file
and to |draft| for child files
which is a natural assignment for editing the document.
This is achieved by placing the following code
in the preamble of the main document
(below the |\childdocmain| directive):
%
\begin{center}
\begin{tabular}{l}
|\ifchilddoc|\\
|\providecommand{\version}{draft}|\\
|\||else|\\
|\providecommand{\version}{final}|\\
|\||fi|
\end{tabular}
\end{center}
%
The definition by |\providecommand| makes sure
that previous definitions are not overwritten.
Further statements |\providecommand{\version}{...}|
can thus be added before the above code to override it.

For the main file, one might add a line
(between |\childdocmain| and the above block)
%
\begin{center}
|%\ifchilddoc\||else\providecommand{\version}{draft}\||fi|
\end{center}
%
which can be uncommented to produce a draft version.
Likewise one can add a line to the very top of a child file
(above the |\childdocof{|\textit{main}|}| directive)
%
\begin{center}
|%\providecommand{\version}{final}|
\end{center}
%
which can be uncommented to produce the final version of this child document.

%%%%%%%%%%%%%%%%%%%%%%%%%%%%%%%%%%%%%%%%%%%%%%%%%%%%%%%%%%%%%%%%%%%%%%%%%%%%%%%%
\subsection{Forwarding}
\label{sec:forward}

Different versions of the main or child documents
using compilation flags as described in \secref{sec:flags}
can be (permanently) stored in different files
for convenient compilation, viewing and distribution.
To this end, the package defines a command
to pass on compilation to a different file:

%%%%%%%%%%%%%%%%%%%%%%%%%%%%%%%%%%%%%%%%
\DescribeMacro{\childdocforward}
The command |\childdocforward| redirects processing to
another source file:
%
\begin{center}
\begin{tabular}{l}
|\input{childdoc.def}|\\
|\childdocforward[|\textit{main}|]{|\textit{dest}|}|\\
\end{tabular}
\end{center}
%
The argument \textit{dest} is the destination file
(without extension).
It should be the main file or one of the child files.
Note that further \textsf{childdoc} directives
such as |\childdocof| and |\childdocforward|
in the indicated file will be processed in this form.
The optional argument \textit{main}
passes on directly to the main file \textit{main}
while pretending to compile the child \textit{dest}.
This form behaves as if \textit{dest}
issues |\childdocof{|\textit{main}|}| right away,
and no further \textsf{childdoc} directives will be processed.

%%%%%%%%%%%%%%%%%%%%%%%%%%%%%%%%%%%%%%%%
\DescribeMacro{\...prefix}
In the alternative form |\childdocforwardprefix|,
%
\begin{center}
\begin{tabular}{l}
|\input{childdoc.def}|\\
|\childdocforwardprefix[|\textit{main}|]{|\textit{prefix}|}{|\textit{dest}|}|
\end{tabular}
\end{center}
%
the destination file is determined by a pattern
depending on the current file:
To make this work, the current file must be called
`{\textit{prefix}\hspace{0.2em}\textit{suffix}}'
with \textit{prefix} matching precisely the argument.
Processing is then passed on to the file
`{\textit{dest}\hspace{0.2em}\textit{suffix}}'.
Surely, the same effect is achieved by
directly specifying the
argument `{\textit{dest}\hspace{0.2em}\textit{suffix}}'
in the first form.
However, that requires to set up a different file
for each child. With the alternative form of the command
all these files can have exactly the same content
which simplifies setting them up and maintaining them.

For example, the following file |draft.tex|
with a compilation flag |\version| as described in \secref{sec:flags}
compiles the main document as a draft:
%
\begin{center}
\begin{tabular}{l}
|\def\version{draft}|\\
|\input{childdoc.def}|\\
|\childdocforward{|\textit{main}|}|
\end{tabular}
\end{center}
%
Likewise, the following files |final|\textit{nn}|.tex|
compile the final version of the child document
|child|\textit{nn}|.tex|:
%
\begin{center}
\begin{tabular}{l}
|\def\version{final}|\\
|\input{childdoc.def}|\\
|\childdocforwardprefix{final}{child}|
\end{tabular}
\end{center}
%

Note that when several versions of a main file and/or of each child file
are to be generated, it may be convenient to set up a |Makefile| or
shell script to automatise the process.

%%%%%%%%%%%%%%%%%%%%%%%%%%%%%%%%%%%%%%%%%%%%%%%%%%%%%%%%%%%%%%%%%%%%%%%%%%%%%%%%
\subsection{Command Line Processing}
\label{sec:commandline}

The effect of redirection files can also be achieved by invoking
the \LaTeX{} compiler with a more elaborate command line.
Most conveniently this should be done as part
of a shell script or a |Makefile|.

When using \textsf{childdoc} in the main file, the following
command lines effectively perform a redirection
(note that depending on the shell being used,
backslashes may have to be doubled: `|\|' $\to$ `|\\|'):
%
\begin{center}
|... -jobname "|\textit{target}|" |\\|"|[\textit{flags}]%
|\input{childdoc.def}\childdocforward[|\textit{main}|]{|\textit{dest}|}"|
\end{center}
%
Here \textit{target} is the name of the output file,
\textit{main} is the name of the main file
and \textit{dest} is the name of the main or child file to be processed
(all filenames without extensions).
The optional argument \textit{main} can be omitted
if \textit{main} matches \textit{dest}.
Optionally, compilation \textit{flags} can be defined via |\def| commands.
This command line makes the \TeX{} engine believe
it is compiling the file \textit{target}
whose content is specified as the latter parameter.
The provided code then forwards the processing to
\textit{main} or \textit{dest} as described in \secref{sec:forward}.

%%%%%%%%%%%%%%%%%%%%%%%%%%%%%%%%%%%%%%%%%%%%%%%%%%%%%%%%%%%%%%%%%%%%%%%%%%%%%%%%
\subsection{Include by Input}
\label{sec:input}

Including child documents by |\include| has some restrictions by design.
Most notably, the content of a child document always occupies
its own set of pages; pages cannot be shared between child documents.
Usually, this behaviour makes perfect sense
because each child document contain an essential part of the document.
However, in some situations it may be desirable to compose
a document from a collection of parts
without having mandatory page breaks between then.
For this case, the package
provides a mechanism to include parts
by |\input| which can also be processed individually.
However, by construction this mechanism
requires manual handling of the content to be output.

%%%%%%%%%%%%%%%%%%%%%%%%%%%%%%%%%%%%%%%%
\DescribeMacro{\ifchilddocmanual}
The main file should be prepared as usual, see \secref{sec:include}.
However, the document body must make a distinction
between processing of an individual part and of the main document, e.g.:
%
\begin{center}
\begin{tabular}{l}
|\ifchilddocmanual|\\
|\input{\childdocname}|\\
|\||else|\\
\textit{document body with }|\input{|\textit{part}|}|\\
|\||fi|
\end{tabular}
\end{center}
%
The conditional |\ifchilddocmanual| is true whenever
a part to be included by |\input| is being compiled,
and the name of the part is stored in |\childdocname|.

%%%%%%%%%%%%%%%%%%%%%%%%%%%%%%%%%%%%%%%%
\DescribeMacro{\childdocby}
Each part to be included by |\input| should start with:
%
\begin{center}
\begin{tabular}{l}
|\input{childdoc.def}|\\
|\childdocby{|\textit{main}|}|\\
\end{tabular}
\end{center}
%
The directive |\childdocby| is similar to |\childdocof|
described in \secref{sec:include},
but the subsequent selection of content must be done manually.
To that end, both |\ifchilddoc| and |\ifchilddocmanual|
will be true upon processing of a part,
and the name of the part is stored in |\childdocname|.
Note that |\jobname| will be set to the filename of the current part
so that each part receives an individual |.aux| file
that does not interfere with the |.aux| file(s) of the main document.
This behaviour can be altered by the alternative form
|\childdocby[*]{|\textit{main}|}| (with a non-empty optional argument)
which uses the |.aux| file of the main document
by setting |\jobname| to \textit{main}.

%%%%%%%%%%%%%%%%%%%%%%%%%%%%%%%%%%%%%%%%%%%%%%%%%%%%%%%%%%%%%%%%%%%%%%%%%%%%%%%%
\subsection{Driver Development}
\label{sec:driver}

The \textsf{childdoc} mechanism can also be use for the development
of definition files such as \LaTeX{} styles or classes.
This case differs from the above setup with multiple parts
included by |\include| in that no |\includeonly| should be invoked.
This can be achieved by starting the include file
(before |\ProvidesPackage|) with:
%
\begin{center}
\begin{tabular}{l}
|\input{childdoc.def}|\\
|\childdocforward{|\textit{main}|}|\\
\end{tabular}
\end{center}
%
or alternatively with:
%
\begin{center}
\begin{tabular}{l}
|\input{childdoc.def}|\\
|\childdocby{|\textit{main}|}|\\
\end{tabular}
\end{center}
%
Both forms have slightly different effects as described above.
The main file is prepared as usual, see \secref{sec:include}.

%%%%%%%%%%%%%%%%%%%%%%%%%%%%%%%%%%%%%%%%%%%%%%%%%%%%%%%%%%%%%%%%%%%%%%%%%%%%%%%%
\subsection{Legacy Detection}
\label{sec:detection}

The directive |\childdocmain| in the main file can detect
whether the complete document or merely a child is to be compiled
even without using the directive |\childdocof|.
This method is deprecated because it is less robust
and there is no compelling reason to use it;
it is merely provided for backward compatibility
and it may be removed in future versions.

If the detection mechanism is to be used,
it is mandatory to correctly specify
the filename of the main file as the argument of |\childdocmain|:
%
\begin{center}
\begin{tabular}{l}
|\input{childdoc.def}|\\
|\childdocmain{|\textit{main}|}|\\
\end{tabular}
\end{center}
%
If |\jobname| does not match the argument \textit{main} of |\childdocmain|,
it is assumed that |\jobname| points to the child file to be compiled.
When using |\childdocmain| with the main file specified as argument,
it suffices to start a child file
with just |\input{|\textit{main}|}|
without loading of the package and using |\childdocof|.
If instead all processing is done
with the appropriate \textsf{childdoc} directives,
the argument of \textit{main} of |\childdocmain| can be empty.

An alternative version of the command line processing described
in \secref{sec:commandline} using the detection mechanism reads:
%
\begin{center}
|... -jobname "|\textit{target}|" "|[\textit{flags}]%
[|\def\jobname{|\textit{dest}|}|]|\input{|\textit{main}|}"|
\end{center}

%%%%%%%%%%%%%%%%%%%%%%%%%%%%%%%%%%%%%%%%%%%%%%%%%%%%%%%%%%%%%%%%%%%%%%%%%%%%%%%%
\subsection{Manual Code}
\label{sec:manual}

In case one cannot be certain whether the definitions file |childdoc.def|
is installed on the target \TeX{} distribution
and one prefers not to ship it,
it is conceivable to paste a few relevant commands into the sources.

To that end, drop all statements |\input{childdoc.def}|
and perform the replacements as outlined below.
Instead of |\childdocmain{|\textit{main}|}| add the following code
to the top of the main file:
%
\begin{center}
\begin{tabular}{l}
|\||ifdefined\childdocname\endinput\||fi\newif\ifchilddoc|\\
|\edef\childdocname{\scantokens\expandafter{\jobname\noexpand}}|\\
|\def\childdocmain{|\textit{main}|}\||ifx\childdocmain\childdocname\||else|\\
|\childdoctrue\includeonly{\childdocname}\let\jobname\childdocmain\||fi|\\
\end{tabular}
\end{center}
%
Instead of |\childdocof{|\textit{main}|}| just include the main file
at the top of each child file:
%
\begin{center}
|\input{|\textit{main}|}|
\end{center}
%
A simple redirection |\childdocforward{|\textit{dest}|}| is achieved by:
%
\begin{center}
|\def\jobname{|\textit{dest}|}\input{\jobname}|
\end{center}
%
The redirection with prefix
|\childdocforwardprefix[|\textit{prefix}|]{|\textit{dest}|}|
is accomplished by:
%
\begin{center}
\begin{tabular}{l}
|{\edef\jobname{\scantokens\expandafter{\jobname\noexpand}}|\\
|\def\redirectjob |\textit{prefix}|#1~~~{\gdef\jobname{|\textit{dest}|#1}}|\\
|\expandafter\redirectjob\jobname~~~}\input{\jobname}|
\end{tabular}
\end{center}

In an alternative approach,
child documents can be compiled by a specific command line
without additional code or specific definitions:
%
\begin{center}
|... -jobname "|\textit{target}|" "|[\textit{flags}]%
|\includeonly{|\textit{dest}|}\input{|\textit{main}|}"|
\end{center}
%

%%%%%%%%%%%%%%%%%%%%%%%%%%%%%%%%%%%%%%%%%%%%%%%%%%%%%%%%%%%%%%%%%%%%%%%%%%%%%%%%
%%%%%%%%%%%%%%%%%%%%%%%%%%%%%%%%%%%%%%%%%%%%%%%%%%%%%%%%%%%%%%%%%%%%%%%%%%%%%%%%
\section{Information}

%%%%%%%%%%%%%%%%%%%%%%%%%%%%%%%%%%%%%%%%%%%%%%%%%%%%%%%%%%%%%%%%%%%%%%%%%%%%%%%%
\subsection{Copyright}

Copyright \copyright{} 2017--2018 Niklas Beisert

This work may be distributed and/or modified under the
conditions of the \LaTeX{} Project Public License, either version 1.3
of this license or (at your option) any later version.
The latest version of this license is in
  \url{http://www.latex-project.org/lppl.txt}
and version 1.3 or later is part of all distributions of \LaTeX{}
version 2005/12/01 or later.

This work has the LPPL maintenance status `maintained'.

The Current Maintainer of this work is Niklas Beisert.

This work consists of the files |README.txt|, |childdoc.ins| and |childdoc.dtx|
as well as the derived files |childdoc.def|, |cdocsamp.tex|
with |cdocsch1.tex|, |cdocsch2.tex|, |cdocspt3.tex|, |cdocspt4.tex|,
|cdocsdrf.tex|, |cdocsfn1.tex|, |cdocsfn2.tex|
as well as |childdoc.pdf|.

%%%%%%%%%%%%%%%%%%%%%%%%%%%%%%%%%%%%%%%%%%%%%%%%%%%%%%%%%%%%%%%%%%%%%%%%%%%%%%%%
\subsection{Files and Installation}

The package consists of the files:
%
\begin{center}
\begin{tabular}{ll}
    |README.txt|   & readme file \\
    |childdoc.ins| & installation file \\
    |childdoc.dtx| & source file \\
    |childdoc.def| & definition file \\
    |cdocsamp.tex| & sample main file \\
    |cdocsch1.tex| & sample include file \\
    |cdocsch2.tex| & sample include file \\
    |cdocspt3.tex| & sample part file \\
    |cdocspt4.tex| & sample part file \\
    |cdocsdrf.tex| & sample redirection file \\
    |cdocsfn1.tex| & sample redirection file \\
    |cdocsfn2.tex| & sample redirection file \\
    |childdoc.pdf| & manual
\end{tabular}
\end{center}
%
The distribution consists of the files
|README.txt|, |childdoc.ins| and |childdoc.dtx|.
%
\begin{itemize}
\item
Run (pdf)\LaTeX{} on |childdoc.dtx|
to compile the manual |childdoc.pdf| (this file).
\item
Run \LaTeX{} on |childdoc.ins| to create the definitions file |childdoc.def|
and the sample |cdocsamp.tex| with include files
|cdocsch1.tex|, |cdocsch2.tex|, |cdocspt3.tex|, |cdocspt4.tex|,
|cdocsdrf.tex|, |cdocsfn1.tex|, |cdocsfn2.tex|.
Then copy the file |childdoc.def| to an appropriate directory of your \LaTeX{}
distribution, e.g.\ \textit{texmf-root}|/tex/latex/childdoc|.
\end{itemize}

%%%%%%%%%%%%%%%%%%%%%%%%%%%%%%%%%%%%%%%%%%%%%%%%%%%%%%%%%%%%%%%%%%%%%%%%%%%%%%%%
\subsection{Related CTAN Packages}

There are several other packages which offer a similar functionality:
%
\begin{itemize}
\item
The packages
\href{http://ctan.org/pkg/docmute}{\textsf{docmute}},
\href{http://ctan.org/pkg/includex}{\textsf{includex}} and
\href{http://ctan.org/pkg/standalone}{\textsf{standalone}}
provide commands to include only the document body of
a child file thus allowing both files to be compiled individually.
\item
The packages \href{http://ctan.org/pkg/subdocs}{\textsf{subdocs}}
and \href{http://ctan.org/pkg/subfiles}{\textsf{subfiles}}
provide structures in which the main and child documents can be
encapsulated and allowing them to be compiled individually.
The inclusion mechanism is different from the conventional |\include|.
\item
The package \href{http://ctan.org/pkg/combine}{\textsf{combine}}
is an elaborate solution to combine several documents into one.
\end{itemize}
%
See also the CTAN topic \href{http://ctan.org/topic/subdocs}{\textsf{subdocs}}
for further related packages.
The present package differs from the above solutions in that
a document structure constructed with the conventional |\include| mechanism
just needs two extra commands at the top of every file
such that all constituent files can be compiled individually.

%%%%%%%%%%%%%%%%%%%%%%%%%%%%%%%%%%%%%%%%%%%%%%%%%%%%%%%%%%%%%%%%%%%%%%%%%%%%%%%%
%\subsection{Feature Suggestions}
%
%The following is a list of features which may be useful for future
%versions of this package:
%%
%\begin{itemize}
%\item
%\ldots
%\end{itemize}

%%%%%%%%%%%%%%%%%%%%%%%%%%%%%%%%%%%%%%%%%%%%%%%%%%%%%%%%%%%%%%%%%%%%%%%%%%%%%%%%
\subsection{Revision History}

%%%%%%%%%%%%%%%%%%%%%%%%%%%%%%%%%%%%%%%%
\paragraph{v2.0:} 2018/12/30

\begin{itemize}
\item
immediate forward processing
\item
added |\childdocby| mechanism
\item
manual restructured
\end{itemize}

%%%%%%%%%%%%%%%%%%%%%%%%%%%%%%%%%%%%%%%%
\paragraph{v1.6:} 2018/01/17

\begin{itemize}
\item
application for development of include files
\item
corrections to manual
\end{itemize}

%%%%%%%%%%%%%%%%%%%%%%%%%%%%%%%%%%%%%%%%
\paragraph{v1.5:} 2017/05/21

\begin{itemize}
\item
more complete structuring introduced
\item
|\childdocof| introduced
\item
|\childdoc| renamed to |\childdocmain|
\item
|\childredirect| renamed to |\childdocforward| and |\childdocforwardprefix|
and functionality expanded
\end{itemize}

%%%%%%%%%%%%%%%%%%%%%%%%%%%%%%%%%%%%%%%%
\paragraph{v1.0:} 2017/04/27

\begin{itemize}
\item
manual and install package
\item
first version published on CTAN
\end{itemize}

%%%%%%%%%%%%%%%%%%%%%%%%%%%%%%%%%%%%%%%%
\paragraph{v0.6:} 2017/04/26

\begin{itemize}
\item
redirection mechanism added
\end{itemize}

%%%%%%%%%%%%%%%%%%%%%%%%%%%%%%%%%%%%%%%%
\paragraph{v0.5:} 2017/04/26

\begin{itemize}
\item
functionality in definition file
\end{itemize}


%%%%%%%%%%%%%%%%%%%%%%%%%%%%%%%%%%%%%%%%%%%%%%%%%%%%%%%%%%%%%%%%%%%%%%%%%%%%%%%%
%%%%%%%%%%%%%%%%%%%%%%%%%%%%%%%%%%%%%%%%%%%%%%%%%%%%%%%%%%%%%%%%%%%%%%%%%%%%%%%%
%%%%%%%%%%%%%%%%%%%%%%%%%%%%%%%%%%%%%%%%%%%%%%%%%%%%%%%%%%%%%%%%%%%%%%%%%%%%%%%%
\appendix

\settowidth\MacroIndent{\rmfamily\scriptsize 000\ }

 \DocInput{childdoc.dtx}

\end{document}
%</driver>
% \fi
%
% %%%%%%%%%%%%%%%%%%%%%%%%%%%%%%%%%%%%%%%%%%%%%%%%%%%%%%%%%%%%%%%%%%%%%%%%%%%%%%
% %%%%%%%%%%%%%%%%%%%%%%%%%%%%%%%%%%%%%%%%%%%%%%%%%%%%%%%%%%%%%%%%%%%%%%%%%%%%%%
% \section{Sample}
%\iffalse
%<*samplemain>
%\fi
%
% The following presents a sample document
% with two chapters, two parts, a title page,
% a compile flag as well as three forwarding files to set the flag.
% It consists of eight |.tex| files:
% \begin{center}
% \begin{tabular}{ll}
% |cdocsamp.tex|&main file\\
% |cdocsch1.tex|&include file for chapter 1\\
% |cdocsch2.tex|&include file for chapter 2\\
% |cdocspt3.tex|&include file for part 3\\
% |cdocspt4.tex|&include file for part 4\\
% |cdocsdrf.tex|&forwarding file for main file in draft mode\\
% |cdocsfi1.tex|&forwarding file for final version of chapter 1\\
% |cdocsfi2.tex|&forwarding file for final version of chapter 2\\
% \end{tabular}
% \end{center}
% Each of the eight files can be compiled directly by the \LaTeX{} compiler.
%
% %%%%%%%%%%%%%%%%%%%%%%%%%%%%%%%%%%%%%%
% \paragraph{Main File.}
%
% The main file is called |cdocsamp.tex|.
%
% Load the \textsf{childdoc} definitions and
% declare the filename for the main document:
%    \begin{macrocode}
\input{childdoc.def}
\childdocmain{}
%    \end{macrocode}

% Optional override for |\version| flag:
%    \begin{macrocode}
%%\ifchilddoc\else\providecommand{\version}{draft}\fi
%    \end{macrocode}

% Define the default values for the |\version| flag
% (|final| for the main file and |draft| for childs):
%    \begin{macrocode}
\ifchilddoc
\providecommand{\version}{draft}
\else
\providecommand{\version}{final}
\fi
%    \end{macrocode}

% Load the standard document class:
%    \begin{macrocode}
\documentclass[12pt]{article}
%    \end{macrocode}

% Start the document body:
%    \begin{macrocode}
\begin{document}
%    \end{macrocode}

% Declare a title page.
% Print title, part of document being processed and version flag:
%    \begin{macrocode}
\addtocounter{page}{-1}
\begin{center}
{\LARGE\bfseries{}childdoc example\par}
\vspace{1cm}
\ifchilddoc
\ifchilddocmanual part\else chapter\fi:
`\childdocname' of `\childdocjob'\par
\else
main document: `\childdocjob'\par
\fi
version: \version\par
\end{center}
\newpage
%    \end{macrocode}

% Manually include selected file,
% otherwise process as usual:
%    \begin{macrocode}
\ifchilddocmanual
\section*{part `\childdocname'}
\input{\childdocname}
\else
%    \end{macrocode}

% Include the two chapters:
%    \begin{macrocode}
\include{cdocsch1}
\include{cdocsch2}
%    \end{macrocode}

% Include the two parts unless only chapters should be displayed:
%    \begin{macrocode}
\ifchilddoc\else
\section{part three}
\input{cdocspt3}
\section{part four}
\input{cdocspt4}
\fi
%    \end{macrocode}

% Process as usual until here:
%    \begin{macrocode}
\fi
%    \end{macrocode}

% End of document body:
%    \begin{macrocode}
\end{document}
%    \end{macrocode}
%\iffalse
%</samplemain>
%\fi
%
% %%%%%%%%%%%%%%%%%%%%%%%%%%%%%%%%%%%%%%
% \paragraph{Chapter Include Files.}
%
% The include files are called |cdocsch1.tex| and |cdocsch2.tex|.
%
%\iffalse
%<*samplechap1|samplechap2>
%\fi

% Optional override for |\version| flag:
%    \begin{macrocode}
%%\providecommand{\version}{final}
%    \end{macrocode}

% Include the main document:
%    \begin{macrocode}
\input{childdoc.def}
\childdocof{cdocsamp}
%    \end{macrocode}

%\iffalse
%</samplechap1|samplechap2>
%\fi
%
%\iffalse
%<*samplechap1>
%\fi
% Some text for chapter 1:
%    \begin{macrocode}
\section{one}
some text in chapter one
%    \end{macrocode}

%\iffalse
%</samplechap1>
%\fi
% Some text for chapter 2:
%\iffalse
%<*samplechap2>
%\fi
%    \begin{macrocode}
\section{two}
more text in chapter two
%    \end{macrocode}

%\iffalse
%</samplechap2>
%\fi
%
% %%%%%%%%%%%%%%%%%%%%%%%%%%%%%%%%%%%%%%
% \paragraph{Part Include Files.}
%
% The include files are called |cdocspt3.tex| and |cdocspt4.tex|.
%
%\iffalse
%<*samplepart3|samplepart4>
%\fi

% Optional override for |\version| flag:
%    \begin{macrocode}
%%\providecommand{\version}{final}
%    \end{macrocode}

% Include the main document:
%    \begin{macrocode}
\input{childdoc.def}
\childdocby{cdocsamp}
%    \end{macrocode}

%\iffalse
%</samplepart3|samplepart4>
%\fi
%
%\iffalse
%<*samplepart3>
%\fi
% Some text for part 3:
%    \begin{macrocode}
some text in part three
%    \end{macrocode}

%\iffalse
%</samplepart3>
%\fi
% Some text for part 4:
%\iffalse
%<*samplepart4>
%\fi
%    \begin{macrocode}
more text in part four
%    \end{macrocode}

%\iffalse
%</samplepart4>
%\fi
%
% %%%%%%%%%%%%%%%%%%%%%%%%%%%%%%%%%%%%%%
% \paragraph{Forwarding for a Complete Draft.}
%
% The following forwarding file |cdocsdrf.tex|
% compiles the main document in draft mode:
%\iffalse
%<*sampledraft>
%\fi
%    \begin{macrocode}
\def\version{draft}
\input{childdoc.def}
\childdocforward{cdocsamp}
%    \end{macrocode}

%\iffalse
%</sampledraft>
%\fi
%
% %%%%%%%%%%%%%%%%%%%%%%%%%%%%%%%%%%%%%%
% \paragraph{Forwarding for Final Version of the Chapters.}
%
% The following forwarding files |cdocsfn1.tex| and |cdocsfn2.tex|
% (with identical content)
% compile the final versions of the child documents
% |cdocsch1.tex| and |cdocsch2.tex|, respectively:
%\iffalse
%<*samplefinal>
%\fi
%    \begin{macrocode}
\def\version{final}
\input{childdoc.def}
\childdocforwardprefix[cdocsamp]{cdocsfn}{cdocsch}
%    \end{macrocode}

%\iffalse
%</samplefinal>
%\fi
%
% %%%%%%%%%%%%%%%%%%%%%%%%%%%%%%%%%%%%%%
% \paragraph{Command Line Processing.}
%
% The following three command lines generate the output files
% |cdocscld|, |cdocscl1| and |cdocscl2|
% which should be identical to
% |cdocsdrf|, |cdocsch1| and |cdocsfn2|, respectively:
% \begin{center}
% \begin{tabular}{l}
% |latex -jobname cdocscld \|\\
% |  "\def\version{draft}\input{childdoc.def}\childdocforward{cdocsamp}"|\\
% |latex -jobname cdocscl1 \|\\
% |  "\input{childdoc.def}\childdocforward[cdocsamp]{cdocsch1}"|\\
% |latex -jobname cdocscl2 \|\\
% |  "\def\version{final}\input{childdoc.def}\childdocforward{cdocsch2}"|
% \end{tabular}
% \end{center}
% Note that the trailing backslash on each first line
% merely continues the input to the second line
% (for convenient cut ant paste).
% Furthermore, the command |latex| can be replaced by any
% of its alternative versions such as |pdflatex|.
%
% %%%%%%%%%%%%%%%%%%%%%%%%%%%%%%%%%%%%%%%%%%%%%%%%%%%%%%%%%%%%%%%%%%%%%%%%%%%%%%
% %%%%%%%%%%%%%%%%%%%%%%%%%%%%%%%%%%%%%%%%%%%%%%%%%%%%%%%%%%%%%%%%%%%%%%%%%%%%%%
% \section{Implementation}
%\iffalse
%<*package>
%\fi
%
% This section describes the definitions file |childdoc.def|.

% The definitions cannot be loaded using |\usepackage| or |\RequirePackage|
% which has a mechanism to prevent loading a style file more than once.
% When loading the definitions by means of |\input|
% multiple instances have to be prevented manually:
%\iffalse
%This code needs to be before the `\ProvidesFile' directive
%which is defined at the beginning of this file.
%Therefore it is also placed there and commented out here.
%</package>
%<*discard>
%\fi
%    \begin{macrocode}
\ifdefined\childdocmain\endinput\fi
%    \end{macrocode}
%\iffalse
%</discard>
%<*package>
%\fi
%
% \macro{\ifchilddoc}
% \macro{\ifchilddocmanual}
% The conditional |\ifchilddoc| tells whether a
% child (true) or main (false) document is being compiled.
% The conditional |\ifchilddocmanual| tells whether
% the |\includeonly| mechanism is used (false) or
% the selection of child files must be performed manually (true).
% The definitions initialise to false:
%    \begin{macrocode}
\newif\ifchilddoc
\newif\ifchilddocmanual
%    \end{macrocode}

% \macro{\childdocname}
% \macro{\childdocjob}
% The macro |\childdocname| stores the name of the main document
% to be compiled. The macro |\childdocjob| stores the name of
% the document on which the \LaTeX{} compiler was originally invoked.
% The content of |\jobname| cannot be compared
% to filenames specified in the source due to different catcodes.
% The following code rescans |\jobname|, stores the result
% in |\childdocname| and saves a copy in |\childdocjob|:
%    \begin{macrocode}
\edef\childdocname{\scantokens\expandafter{\jobname\noexpand}}
\let\childdocjob\childdocname
%    \end{macrocode}

% \macro{\childdocdisable}
% The macro |\childdocdisable| prevents the main file
% from being processed more than once.
% At this stage, the main document command |\childdocmain|
% is assumed to be called once again where it should do nothing.
% Any subsequent call to it should prevent
% a secondary processing of the main document
% It overwrites the forwarding commands
% |\childdocof| and |\childdocforward|
% with empty macros to prevent further inclusions of the main document:
%    \begin{macrocode}
\newcommand{\childdocdisable}
{
  \renewcommand{\childdocmain}[1]{\renewcommand{\childdocmain}[1]{\endinput}}
  \renewcommand{\childdocof}[1]{}
  \renewcommand{\childdocby}[2][]{}
  \renewcommand{\childdocforward}[2][]{}
  \renewcommand{\childdocdisable}{}
}
%    \end{macrocode}

% \macro{\childdocmain}
% The macro |\childdocmain| is to be called at the top of the main file
% with nothing or the main filename (without extension) as argument.
% First, it breaks loops.
% If the argument is not empty and does not match |\childdocname|
% (which is set by the first inclusion of |childdoc.def|),
% |\ifchilddoc| is set to true, |\includeonly| is applied to the child file
% and |\jobname| is set to the main file
% (for proper handling of |.aux| files):
%    \begin{macrocode}
\newcommand{\childdocmain}[1]
{
  \childdocdisable\childdocmain{}
  \if?#1?\else
    \begingroup
      \def\childdoctmp{#1}
      \ifx\childdoctmp\childdocname
        \def\childdoctmp{}
      \else
        \def\childdoctmp
        {
          \childdoctrue
          \includeonly{\childdocname}
          \def\childdocjob{#1}
          \def\jobname{#1}
        }
      \fi
      \expandafter
    \endgroup
    \childdoctmp
  \fi
}
%    \end{macrocode}

% \macro{\childdocof}
% The command |\childdocof| redirects
% compilation to the main file |#1|.
%    \begin{macrocode}
\newcommand{\childdocof}[1]
{
  \childdocdisable
  \childdoctrue
  \includeonly{\childdocname}
  \def\jobname{#1}
  \def\childdocjob{#1}
  \input{#1}
}
%    \end{macrocode}

% \macro{\childdocby}
% The command |\childdocby| ....
%    \begin{macrocode}
\newcommand{\childdocby}[2][]
{
  \childdocdisable
  \childdoctrue
  \childdocmanualtrue
  \if?#1?\else
    \def\jobname{#2}
  \fi
  \def\childdocjob{#2}
  \input{#2}
  \endinput
}
%    \end{macrocode}

% \macro{\childdocforward}
% The command |\childdocforward| redirects
% compilation to the main file or
% (if the optional argument is given) a child file.
% Parameters are set as if the main file
% or a child file starting with |\childdocof| was compiled.
% Then compilation is handed over to the main file:
%    \begin{macrocode}
\newcommand{\childdocforward}[2][]
{
  \begingroup
    \if?#1?
      \def\childdoctmp
      {
        \def\childdocname{#2}
        \def\childdocjob{#2}
        \def\jobname{#2}
        \input{#2}
        \endinput
      }
    \else
      \def\childdoctmp
      {
        \childdocdisable
        \def\childdocname{#2}
        \childdoctrue
        \includeonly{#2}
        \def\childdocjob{#1}
        \def\jobname{#1}
        \input{#1}
        \endinput
      }
    \fi
    \expandafter
  \endgroup
  \childdoctmp
}
%    \end{macrocode}

% \macro{\childdocforwardprefix}
% The command |\childdocforwardprefix| redirects
% compilation to the main or a child file by means of a pattern.
% The prefix |#1| in the current filename is replaced by |#2|
% and the suffix of the current filename is kept
% (it is assumed that the filename does not contain the substring `|~~~|'
% which is used as a delimiter).
% Compilation is handed over to the new file by |\childdocforward|:
%    \begin{macrocode}
\newcommand{\childdocforwardprefix}[3][]
{
  \begingroup
    \def\childdocextract #2##1~~~{\def\childdoctmp{\childdocforward[#1]{#3##1}}}
    \expandafter\childdocextract\childdocname~~~
    \expandafter
  \endgroup
  \childdoctmp
}
%    \end{macrocode}

% \macro{\childdoc}
% The deprecated macro |\childdoc| is a legacy version of |\childdocmain|:
%    \begin{macrocode}
\newcommand{\childdoc}{\childdocmain}
%    \end{macrocode}

% \macro{\childdocredirect}
% The deprecated macro |\childdocredirect| is a legacy version
% of |\childdocforward| and |\childdocforwardprefix|:
%    \begin{macrocode}
\newcommand{\childdocredirect}[2][]
{
  \begingroup
    \if?#1?
      \def\childdoctmp{\childdocforward{#2}}
    \else
      \def\childdoctmp{\childdocforwardprefix{#1}{#2}}
    \fi
    \expandafter
  \endgroup
  \childdoctmp
}
%    \end{macrocode}

%\iffalse
%</package>
%\fi
%
\endinput
|\\
|\childdocforward[|\textit{main}|]{|\textit{dest}|}|\\
\end{tabular}
\end{center}
%
The argument \textit{dest} is the destination file
(without extension).
It should be the main file or one of the child files.
Note that further \textsf{childdoc} directives
such as |\childdocof| and |\childdocforward|
in the indicated file will be processed in this form.
The optional argument \textit{main}
passes on directly to the main file \textit{main}
while pretending to compile the child \textit{dest}.
This form behaves as if \textit{dest}
issues |\childdocof{|\textit{main}|}| right away,
and no further \textsf{childdoc} directives will be processed.

%%%%%%%%%%%%%%%%%%%%%%%%%%%%%%%%%%%%%%%%
\DescribeMacro{\...prefix}
In the alternative form |\childdocforwardprefix|,
%
\begin{center}
\begin{tabular}{l}
|% \iffalse
%
% childdoc.dtx Copyright (C) 2017-2018 Niklas Beisert
%
% This work may be distributed and/or modified under the
% conditions of the LaTeX Project Public License, either version 1.3
% of this license or (at your option) any later version.
% The latest version of this license is in
%   http://www.latex-project.org/lppl.txt
% and version 1.3 or later is part of all distributions of LaTeX
% version 2005/12/01 or later.
%
% This work has the LPPL maintenance status `maintained'.
%
% The Current Maintainer of this work is Niklas Beisert.
%
% This work consists of the files childdoc.dtx and childdoc.ins
% and the derived files childdoc.def and cdocsamp.tex with
% cdocsch1.tex, cdocsch2.tex, cdocsdrf.tex, cdocsfn1.tex, cdocsfn2.tex.
%
%<package>\ifdefined\childdocmain\endinput\fi
%<package>\ProvidesFile{childdoc.def}[2018/12/30 v2.0 child document driver]
%<samplemain>\ProvidesFile{cdocsamp.tex}[2018/12/30 v2.0 sample for childdoc]
%<*driver>
%\ProvidesFile{childdoc.drv}[2018/12/30 v2.0 childdoc reference manual file]
\PassOptionsToClass{10pt,a4paper}{article}
\documentclass{ltxdoc}

\usepackage[margin=35mm]{geometry}
\usepackage{hyperref}
\usepackage{hyperxmp}
\usepackage[usenames]{color}

\hypersetup{colorlinks=true}
\hypersetup{pdfstartview=FitH}
\hypersetup{pdfpagemode=UseNone}
\hypersetup{pdfsource={}}
\hypersetup{pdflang={en-UK}}
\hypersetup{pdfcopyright={Copyright 2017-2018 Niklas Beisert.
  This work may be distributed and/or modified under the
  conditions of the LaTeX Project Public License, either version 1.3
  of this license or (at your option) any later version.}}
\hypersetup{pdflicenseurl={http://www.latex-project.org/lppl.txt}}
\hypersetup{pdfcontactaddress={ETH Zurich, ITP, HIT K,
  Wolfgang-Pauli-Strasse 27}}
\hypersetup{pdfcontactpostcode={8093}}
\hypersetup{pdfcontactcity={Zurich}}
\hypersetup{pdfcontactcountry={Switzerland}}
\hypersetup{pdfcontactemail={nbeisert@itp.phys.ethz.ch}}
\hypersetup{pdfcontacturl={http://people.phys.ethz.ch/\xmptilde nbeisert/}}

\newcommand{\secref}[1]{\hyperref[#1]{section \ref*{#1}}}

\parskip1ex
\parindent0pt
\let\olditemize\itemize
\def\itemize{\olditemize\parskip0pt}

\begin{document}

\title{The \textsf{childdoc} Package}
\hypersetup{pdftitle={The childdoc Package}}
\author{Niklas Beisert\\[2ex]
  Institut f\"ur Theoretische Physik\\
  Eidgen\"ossische Technische Hochschule Z\"urich\\
  Wolfgang-Pauli-Strasse 27, 8093 Z\"urich, Switzerland\\[1ex]
  \href{mailto:nbeisert@itp.phys.ethz.ch}
  {\texttt{nbeisert@itp.phys.ethz.ch}}}
\hypersetup{pdfauthor={Niklas Beisert}}
\hypersetup{pdfsubject={Manual for the LaTeX2e Package childdoc}}
\date{30 December 2018, \textsf{v2.0}}
\maketitle

\begin{abstract}\noindent
\textsf{childdoc} is a \LaTeXe{} package
that enables the direct compilation
of document sections included by |\include|
to individual files.
\end{abstract}

\begingroup
\parskip0ex
\tableofcontents
\endgroup

%%%%%%%%%%%%%%%%%%%%%%%%%%%%%%%%%%%%%%%%%%%%%%%%%%%%%%%%%%%%%%%%%%%%%%%%%%%%%%%%
%%%%%%%%%%%%%%%%%%%%%%%%%%%%%%%%%%%%%%%%%%%%%%%%%%%%%%%%%%%%%%%%%%%%%%%%%%%%%%%%
\section{Introduction}

\LaTeX{} provides a mechanism to structure a large document (such as a book)
into a main file and several child files (containing the chapters)
using the |\include| command.
This mechanism is beneficial for documents
which span hundreds of pages in order to
make the source file(s) more manageable.
Moreover, compilation can be restricted to
selected child files by means of the |\includeonly| command.
The latter feature can be used to reduce the compilation time while editing
(this was significantly more useful in the earlier days of \LaTeX{})
or to generate a smaller document which is easier to navigate.
Another application of |\includeonly| is to generate
documents consisting of selected parts of the complete document.

However, there are a few drawbacks of the plain |\include| mechanism:
\begin{itemize}
\item
The child files cannot be compiled on their own,
they can only be compiled via the main file.
A naive editing environment
(such as a text editor with an option
to have the current file processed by \LaTeX)
may require one to switch to the main file before compiling;
attempting to compile the child file produces errors.
\item
The main file must be modified (each time)
to adjust the |\includeonly| command
to the present needs. This easily leaves the main file in a messy state.
\item
The generated document will always carry the filename
of the main document. This is inconvenient if
several child files are to be compiled and
to be kept for distribution.
\end{itemize}

The present package provides a simple interface
to make child files individually compilable by \LaTeX{}.
Compiling a child file then has the same effect as compiling
the main file with an |\includeonly| command
to select the appropriate child.
Moreover the generated document will carry the name of the child
rather than the main file.
This resolves all three above issues.

This feature is meant to make the editing of books,
thesis documents and lecture notes somewhat more convenient.
However, the package can also be used efficiently for
composing a series of documents (such as exercise sheets)
which are typically distributed individually.
It then assists the author in generating the individual documents
(potentially in different versions)
as well as a document containing the collected series.
Another application is in developing style files
or other kinds of included material
where compilation of the style file could redirect
to a sample or test file.

%%%%%%%%%%%%%%%%%%%%%%%%%%%%%%%%%%%%%%%%%%%%%%%%%%%%%%%%%%%%%%%%%%%%%%%%%%%%%%%%
%%%%%%%%%%%%%%%%%%%%%%%%%%%%%%%%%%%%%%%%%%%%%%%%%%%%%%%%%%%%%%%%%%%%%%%%%%%%%%%%
\section{Usage}

First of all, the package \textsf{childdoc} is \emph{not} a standard
\LaTeXe{} |.sty| style file! Therefore it needs to be invoked in
a non-standard way.

%%%%%%%%%%%%%%%%%%%%%%%%%%%%%%%%%%%%%%%%%%%%%%%%%%%%%%%%%%%%%%%%%%%%%%%%%%%%%%%%
\subsection{Included Files}
\label{sec:include}

%%%%%%%%%%%%%%%%%%%%%%%%%%%%%%%%%%%%%%%%
\DescribeMacro{\childdocmain}
To use the package, add the commands
\begin{center}
\begin{tabular}{l}
|\input{childdoc.def}|\\
|\childdocmain{}|\\
\end{tabular}
\end{center}
at the very top of the main \LaTeX{} file,
in particular \emph{before} the |\documentclass| statement!
The argument of |\childdocmain| should be left empty
(but it must be present).

%%%%%%%%%%%%%%%%%%%%%%%%%%%%%%%%%%%%%%%%
\DescribeMacro{\childdocof}
Furthermore, add the commands
\begin{center}
\begin{tabular}{l}
|\input{childdoc.def}|\\
|\childdocof{|\textit{main}|}|\\
\end{tabular}
\end{center}
at the top of every child file \textit{child}
which is included by |\include{|\textit{child}|}|
from within the main file
(or at least for those files to be compiled individually).
The argument \textit{main} must be the filename of the main file.

There are a couple of
considerations in setting up the main and child documents:

%%%%%%%%%%%%%%%%%%%%%%%%%%%%%%%%%%%%%%%%
\paragraph{Restrictions.}

Please note the following restrictions:
\begin{itemize}
\item
|\childdocmain| must be called with one argument \textit{main}
to ensure compatibility with earlier version of the package.
It must either be empty (|\childdocmain{}|)
or precisely match the filename of the main file in which it is specified.
See \secref{sec:detection} for further information.
\item
The filename \textit{main} must be specified without the |.tex| extension.
\item
The filename \textit{main} is case sensitive
(even in case-insensitive file systems)
due to internal string comparison.
\item
The argument \textit{main} should be fully expanded, it cannot be a macro.
\item
Subdirectories and special characters should be avoided in filenames.
\item
The command |\childdocmain{|\textit{main}|}| must be followed by a whitespace.
It should not be followed immediately by another command
or by a comment mark `|%|'.
This is because the \TeX{} parser reads the token immediately following
the argument of |\childdocmain| and puts it
at the beginning of every child section;
however, a white\-space is ignored.
\end{itemize}

%%%%%%%%%%%%%%%%%%%%%%%%%%%%%%%%%%%%%%%%
\paragraph{Content of Main File.}

It is advisable to place all content in the child files included by |\include|.
Any output contained in the main file will appear in all child documents
unless suppressed manually;
it cannot be suppressed automatically by the |\includeonly| directive
and thus should normally be avoided.
A method to include some content in the main file
by means of conditional processing is described in \secref{sec:conditional}.

%%%%%%%%%%%%%%%%%%%%%%%%%%%%%%%%%%%%%%%%
\paragraph{Page Numbering.}

When only a part of the document is compiled,
the appropriate numbering of pages
(as well as other status parameters)
is determined from the |.aux| files.
The latter contain information from previous passes.
However this information needs to propagate through
all intermediate child documents.
Therefore the page numbering in child documents may well
be inconsistent until the complete document is compiled at least once.

A useful (if unconventional) way to always ensure a consistent
page numbering is to restart the numbering in each child document
and denote the pages by `\textit{child}|.|\textit{page}'
where \textit{child} represents the chapter/section number of the child file.
This can be achieved by the command
|\numberwithin{page}{|\textit{child}|}|
of the \textsf{amsmath} package
where \textit{child} can be |chapter| or |section|
depending on the chosen structuring.
Alternatively, one can modify the macro |\thepage| appropriately
and reset the counter |page| at the start of each child file.

%%%%%%%%%%%%%%%%%%%%%%%%%%%%%%%%%%%%%%%%%%%%%%%%%%%%%%%%%%%%%%%%%%%%%%%%%%%%%%%%
\subsection{Conditional Processing}
\label{sec:conditional}

The package provides a mechanism to compile different versions
of a document. To customise the versions further some conditional processing
can come in handy to distinguish which version is being compiled.
The package provides two macros to describe the compilation context:

%%%%%%%%%%%%%%%%%%%%%%%%%%%%%%%%%%%%%%%%
\DescribeMacro{\ifchilddoc}
The conditional |\ifchilddoc| distinguishes between the compilation of
child documents and the main document:
%
\begin{center}
|\ifchilddoc |\textit{child-code}| |[|\||else |\textit{main-code}]| \||fi|
\end{center}

%%%%%%%%%%%%%%%%%%%%%%%%%%%%%%%%%%%%%%%%
\DescribeMacro{\childdocname}
\DescribeMacro{\childdocjob}
The macro |\childdocname| contains the filename (without extension)
of the main or child file being processed.
Note that |\childdocjob| will always contain the name of the main file.

%%%%%%%%%%%%%%%%%%%%%%%%%%%%%%%%%%%%%%%%
\paragraph{Title Page.}

Conditional processing can be used to include a title or banner page
in the main document when proper precautions are taken.
Importantly, the code in the main file should ensure that the page counter
(as well as other status parameters which are stored in the |.aux| files)
takes the same value after the conditional processing.
Otherwise the page numbers may take divergent values
depending on which part is compiled.

For example, a title page could be declared by:
%
\begin{center}
\begin{tabular}{l}
|\ifchilddoc\||else|\\
|\addtocounter{page}{-1}|\\
\textit{code for title page}\\
|\newpage|\\
|\||fi|
\end{tabular}
\end{center}
%
A banner page for the child documents can be generated by:
%
\begin{center}
\begin{tabular}{l}
|\ifchilddoc|\\
|\addtocounter{page}{-1}|\\
\textit{code for banner page}\\
|\newpage|\\
|\||fi|
\end{tabular}
\end{center}
%
Here one could write a message such as:
\begin{center}
|This is the part \childdocname{} of \childdocjob{}.|
\end{center}

%%%%%%%%%%%%%%%%%%%%%%%%%%%%%%%%%%%%%%%%%%%%%%%%%%%%%%%%%%%%%%%%%%%%%%%%%%%%%%%%
\subsection{Flags}
\label{sec:flags}

The package makes it easy to generate different versions
of the main or child documents.
To this end compilation flags can be defined
and assigned different default values.
They will be particularly useful in conjunction
with the forwarding mechanism described in \secref{sec:forward}.

For example, it may be useful to have a flag |\version|
which can be set to |draft| or |final|.
The document source will contain some conditional code
depending on the value of |\version|.
Suppose further, the flag should default to |final| for the main file
and to |draft| for child files
which is a natural assignment for editing the document.
This is achieved by placing the following code
in the preamble of the main document
(below the |\childdocmain| directive):
%
\begin{center}
\begin{tabular}{l}
|\ifchilddoc|\\
|\providecommand{\version}{draft}|\\
|\||else|\\
|\providecommand{\version}{final}|\\
|\||fi|
\end{tabular}
\end{center}
%
The definition by |\providecommand| makes sure
that previous definitions are not overwritten.
Further statements |\providecommand{\version}{...}|
can thus be added before the above code to override it.

For the main file, one might add a line
(between |\childdocmain| and the above block)
%
\begin{center}
|%\ifchilddoc\||else\providecommand{\version}{draft}\||fi|
\end{center}
%
which can be uncommented to produce a draft version.
Likewise one can add a line to the very top of a child file
(above the |\childdocof{|\textit{main}|}| directive)
%
\begin{center}
|%\providecommand{\version}{final}|
\end{center}
%
which can be uncommented to produce the final version of this child document.

%%%%%%%%%%%%%%%%%%%%%%%%%%%%%%%%%%%%%%%%%%%%%%%%%%%%%%%%%%%%%%%%%%%%%%%%%%%%%%%%
\subsection{Forwarding}
\label{sec:forward}

Different versions of the main or child documents
using compilation flags as described in \secref{sec:flags}
can be (permanently) stored in different files
for convenient compilation, viewing and distribution.
To this end, the package defines a command
to pass on compilation to a different file:

%%%%%%%%%%%%%%%%%%%%%%%%%%%%%%%%%%%%%%%%
\DescribeMacro{\childdocforward}
The command |\childdocforward| redirects processing to
another source file:
%
\begin{center}
\begin{tabular}{l}
|\input{childdoc.def}|\\
|\childdocforward[|\textit{main}|]{|\textit{dest}|}|\\
\end{tabular}
\end{center}
%
The argument \textit{dest} is the destination file
(without extension).
It should be the main file or one of the child files.
Note that further \textsf{childdoc} directives
such as |\childdocof| and |\childdocforward|
in the indicated file will be processed in this form.
The optional argument \textit{main}
passes on directly to the main file \textit{main}
while pretending to compile the child \textit{dest}.
This form behaves as if \textit{dest}
issues |\childdocof{|\textit{main}|}| right away,
and no further \textsf{childdoc} directives will be processed.

%%%%%%%%%%%%%%%%%%%%%%%%%%%%%%%%%%%%%%%%
\DescribeMacro{\...prefix}
In the alternative form |\childdocforwardprefix|,
%
\begin{center}
\begin{tabular}{l}
|\input{childdoc.def}|\\
|\childdocforwardprefix[|\textit{main}|]{|\textit{prefix}|}{|\textit{dest}|}|
\end{tabular}
\end{center}
%
the destination file is determined by a pattern
depending on the current file:
To make this work, the current file must be called
`{\textit{prefix}\hspace{0.2em}\textit{suffix}}'
with \textit{prefix} matching precisely the argument.
Processing is then passed on to the file
`{\textit{dest}\hspace{0.2em}\textit{suffix}}'.
Surely, the same effect is achieved by
directly specifying the
argument `{\textit{dest}\hspace{0.2em}\textit{suffix}}'
in the first form.
However, that requires to set up a different file
for each child. With the alternative form of the command
all these files can have exactly the same content
which simplifies setting them up and maintaining them.

For example, the following file |draft.tex|
with a compilation flag |\version| as described in \secref{sec:flags}
compiles the main document as a draft:
%
\begin{center}
\begin{tabular}{l}
|\def\version{draft}|\\
|\input{childdoc.def}|\\
|\childdocforward{|\textit{main}|}|
\end{tabular}
\end{center}
%
Likewise, the following files |final|\textit{nn}|.tex|
compile the final version of the child document
|child|\textit{nn}|.tex|:
%
\begin{center}
\begin{tabular}{l}
|\def\version{final}|\\
|\input{childdoc.def}|\\
|\childdocforwardprefix{final}{child}|
\end{tabular}
\end{center}
%

Note that when several versions of a main file and/or of each child file
are to be generated, it may be convenient to set up a |Makefile| or
shell script to automatise the process.

%%%%%%%%%%%%%%%%%%%%%%%%%%%%%%%%%%%%%%%%%%%%%%%%%%%%%%%%%%%%%%%%%%%%%%%%%%%%%%%%
\subsection{Command Line Processing}
\label{sec:commandline}

The effect of redirection files can also be achieved by invoking
the \LaTeX{} compiler with a more elaborate command line.
Most conveniently this should be done as part
of a shell script or a |Makefile|.

When using \textsf{childdoc} in the main file, the following
command lines effectively perform a redirection
(note that depending on the shell being used,
backslashes may have to be doubled: `|\|' $\to$ `|\\|'):
%
\begin{center}
|... -jobname "|\textit{target}|" |\\|"|[\textit{flags}]%
|\input{childdoc.def}\childdocforward[|\textit{main}|]{|\textit{dest}|}"|
\end{center}
%
Here \textit{target} is the name of the output file,
\textit{main} is the name of the main file
and \textit{dest} is the name of the main or child file to be processed
(all filenames without extensions).
The optional argument \textit{main} can be omitted
if \textit{main} matches \textit{dest}.
Optionally, compilation \textit{flags} can be defined via |\def| commands.
This command line makes the \TeX{} engine believe
it is compiling the file \textit{target}
whose content is specified as the latter parameter.
The provided code then forwards the processing to
\textit{main} or \textit{dest} as described in \secref{sec:forward}.

%%%%%%%%%%%%%%%%%%%%%%%%%%%%%%%%%%%%%%%%%%%%%%%%%%%%%%%%%%%%%%%%%%%%%%%%%%%%%%%%
\subsection{Include by Input}
\label{sec:input}

Including child documents by |\include| has some restrictions by design.
Most notably, the content of a child document always occupies
its own set of pages; pages cannot be shared between child documents.
Usually, this behaviour makes perfect sense
because each child document contain an essential part of the document.
However, in some situations it may be desirable to compose
a document from a collection of parts
without having mandatory page breaks between then.
For this case, the package
provides a mechanism to include parts
by |\input| which can also be processed individually.
However, by construction this mechanism
requires manual handling of the content to be output.

%%%%%%%%%%%%%%%%%%%%%%%%%%%%%%%%%%%%%%%%
\DescribeMacro{\ifchilddocmanual}
The main file should be prepared as usual, see \secref{sec:include}.
However, the document body must make a distinction
between processing of an individual part and of the main document, e.g.:
%
\begin{center}
\begin{tabular}{l}
|\ifchilddocmanual|\\
|\input{\childdocname}|\\
|\||else|\\
\textit{document body with }|\input{|\textit{part}|}|\\
|\||fi|
\end{tabular}
\end{center}
%
The conditional |\ifchilddocmanual| is true whenever
a part to be included by |\input| is being compiled,
and the name of the part is stored in |\childdocname|.

%%%%%%%%%%%%%%%%%%%%%%%%%%%%%%%%%%%%%%%%
\DescribeMacro{\childdocby}
Each part to be included by |\input| should start with:
%
\begin{center}
\begin{tabular}{l}
|\input{childdoc.def}|\\
|\childdocby{|\textit{main}|}|\\
\end{tabular}
\end{center}
%
The directive |\childdocby| is similar to |\childdocof|
described in \secref{sec:include},
but the subsequent selection of content must be done manually.
To that end, both |\ifchilddoc| and |\ifchilddocmanual|
will be true upon processing of a part,
and the name of the part is stored in |\childdocname|.
Note that |\jobname| will be set to the filename of the current part
so that each part receives an individual |.aux| file
that does not interfere with the |.aux| file(s) of the main document.
This behaviour can be altered by the alternative form
|\childdocby[*]{|\textit{main}|}| (with a non-empty optional argument)
which uses the |.aux| file of the main document
by setting |\jobname| to \textit{main}.

%%%%%%%%%%%%%%%%%%%%%%%%%%%%%%%%%%%%%%%%%%%%%%%%%%%%%%%%%%%%%%%%%%%%%%%%%%%%%%%%
\subsection{Driver Development}
\label{sec:driver}

The \textsf{childdoc} mechanism can also be use for the development
of definition files such as \LaTeX{} styles or classes.
This case differs from the above setup with multiple parts
included by |\include| in that no |\includeonly| should be invoked.
This can be achieved by starting the include file
(before |\ProvidesPackage|) with:
%
\begin{center}
\begin{tabular}{l}
|\input{childdoc.def}|\\
|\childdocforward{|\textit{main}|}|\\
\end{tabular}
\end{center}
%
or alternatively with:
%
\begin{center}
\begin{tabular}{l}
|\input{childdoc.def}|\\
|\childdocby{|\textit{main}|}|\\
\end{tabular}
\end{center}
%
Both forms have slightly different effects as described above.
The main file is prepared as usual, see \secref{sec:include}.

%%%%%%%%%%%%%%%%%%%%%%%%%%%%%%%%%%%%%%%%%%%%%%%%%%%%%%%%%%%%%%%%%%%%%%%%%%%%%%%%
\subsection{Legacy Detection}
\label{sec:detection}

The directive |\childdocmain| in the main file can detect
whether the complete document or merely a child is to be compiled
even without using the directive |\childdocof|.
This method is deprecated because it is less robust
and there is no compelling reason to use it;
it is merely provided for backward compatibility
and it may be removed in future versions.

If the detection mechanism is to be used,
it is mandatory to correctly specify
the filename of the main file as the argument of |\childdocmain|:
%
\begin{center}
\begin{tabular}{l}
|\input{childdoc.def}|\\
|\childdocmain{|\textit{main}|}|\\
\end{tabular}
\end{center}
%
If |\jobname| does not match the argument \textit{main} of |\childdocmain|,
it is assumed that |\jobname| points to the child file to be compiled.
When using |\childdocmain| with the main file specified as argument,
it suffices to start a child file
with just |\input{|\textit{main}|}|
without loading of the package and using |\childdocof|.
If instead all processing is done
with the appropriate \textsf{childdoc} directives,
the argument of \textit{main} of |\childdocmain| can be empty.

An alternative version of the command line processing described
in \secref{sec:commandline} using the detection mechanism reads:
%
\begin{center}
|... -jobname "|\textit{target}|" "|[\textit{flags}]%
[|\def\jobname{|\textit{dest}|}|]|\input{|\textit{main}|}"|
\end{center}

%%%%%%%%%%%%%%%%%%%%%%%%%%%%%%%%%%%%%%%%%%%%%%%%%%%%%%%%%%%%%%%%%%%%%%%%%%%%%%%%
\subsection{Manual Code}
\label{sec:manual}

In case one cannot be certain whether the definitions file |childdoc.def|
is installed on the target \TeX{} distribution
and one prefers not to ship it,
it is conceivable to paste a few relevant commands into the sources.

To that end, drop all statements |\input{childdoc.def}|
and perform the replacements as outlined below.
Instead of |\childdocmain{|\textit{main}|}| add the following code
to the top of the main file:
%
\begin{center}
\begin{tabular}{l}
|\||ifdefined\childdocname\endinput\||fi\newif\ifchilddoc|\\
|\edef\childdocname{\scantokens\expandafter{\jobname\noexpand}}|\\
|\def\childdocmain{|\textit{main}|}\||ifx\childdocmain\childdocname\||else|\\
|\childdoctrue\includeonly{\childdocname}\let\jobname\childdocmain\||fi|\\
\end{tabular}
\end{center}
%
Instead of |\childdocof{|\textit{main}|}| just include the main file
at the top of each child file:
%
\begin{center}
|\input{|\textit{main}|}|
\end{center}
%
A simple redirection |\childdocforward{|\textit{dest}|}| is achieved by:
%
\begin{center}
|\def\jobname{|\textit{dest}|}\input{\jobname}|
\end{center}
%
The redirection with prefix
|\childdocforwardprefix[|\textit{prefix}|]{|\textit{dest}|}|
is accomplished by:
%
\begin{center}
\begin{tabular}{l}
|{\edef\jobname{\scantokens\expandafter{\jobname\noexpand}}|\\
|\def\redirectjob |\textit{prefix}|#1~~~{\gdef\jobname{|\textit{dest}|#1}}|\\
|\expandafter\redirectjob\jobname~~~}\input{\jobname}|
\end{tabular}
\end{center}

In an alternative approach,
child documents can be compiled by a specific command line
without additional code or specific definitions:
%
\begin{center}
|... -jobname "|\textit{target}|" "|[\textit{flags}]%
|\includeonly{|\textit{dest}|}\input{|\textit{main}|}"|
\end{center}
%

%%%%%%%%%%%%%%%%%%%%%%%%%%%%%%%%%%%%%%%%%%%%%%%%%%%%%%%%%%%%%%%%%%%%%%%%%%%%%%%%
%%%%%%%%%%%%%%%%%%%%%%%%%%%%%%%%%%%%%%%%%%%%%%%%%%%%%%%%%%%%%%%%%%%%%%%%%%%%%%%%
\section{Information}

%%%%%%%%%%%%%%%%%%%%%%%%%%%%%%%%%%%%%%%%%%%%%%%%%%%%%%%%%%%%%%%%%%%%%%%%%%%%%%%%
\subsection{Copyright}

Copyright \copyright{} 2017--2018 Niklas Beisert

This work may be distributed and/or modified under the
conditions of the \LaTeX{} Project Public License, either version 1.3
of this license or (at your option) any later version.
The latest version of this license is in
  \url{http://www.latex-project.org/lppl.txt}
and version 1.3 or later is part of all distributions of \LaTeX{}
version 2005/12/01 or later.

This work has the LPPL maintenance status `maintained'.

The Current Maintainer of this work is Niklas Beisert.

This work consists of the files |README.txt|, |childdoc.ins| and |childdoc.dtx|
as well as the derived files |childdoc.def|, |cdocsamp.tex|
with |cdocsch1.tex|, |cdocsch2.tex|, |cdocspt3.tex|, |cdocspt4.tex|,
|cdocsdrf.tex|, |cdocsfn1.tex|, |cdocsfn2.tex|
as well as |childdoc.pdf|.

%%%%%%%%%%%%%%%%%%%%%%%%%%%%%%%%%%%%%%%%%%%%%%%%%%%%%%%%%%%%%%%%%%%%%%%%%%%%%%%%
\subsection{Files and Installation}

The package consists of the files:
%
\begin{center}
\begin{tabular}{ll}
    |README.txt|   & readme file \\
    |childdoc.ins| & installation file \\
    |childdoc.dtx| & source file \\
    |childdoc.def| & definition file \\
    |cdocsamp.tex| & sample main file \\
    |cdocsch1.tex| & sample include file \\
    |cdocsch2.tex| & sample include file \\
    |cdocspt3.tex| & sample part file \\
    |cdocspt4.tex| & sample part file \\
    |cdocsdrf.tex| & sample redirection file \\
    |cdocsfn1.tex| & sample redirection file \\
    |cdocsfn2.tex| & sample redirection file \\
    |childdoc.pdf| & manual
\end{tabular}
\end{center}
%
The distribution consists of the files
|README.txt|, |childdoc.ins| and |childdoc.dtx|.
%
\begin{itemize}
\item
Run (pdf)\LaTeX{} on |childdoc.dtx|
to compile the manual |childdoc.pdf| (this file).
\item
Run \LaTeX{} on |childdoc.ins| to create the definitions file |childdoc.def|
and the sample |cdocsamp.tex| with include files
|cdocsch1.tex|, |cdocsch2.tex|, |cdocspt3.tex|, |cdocspt4.tex|,
|cdocsdrf.tex|, |cdocsfn1.tex|, |cdocsfn2.tex|.
Then copy the file |childdoc.def| to an appropriate directory of your \LaTeX{}
distribution, e.g.\ \textit{texmf-root}|/tex/latex/childdoc|.
\end{itemize}

%%%%%%%%%%%%%%%%%%%%%%%%%%%%%%%%%%%%%%%%%%%%%%%%%%%%%%%%%%%%%%%%%%%%%%%%%%%%%%%%
\subsection{Related CTAN Packages}

There are several other packages which offer a similar functionality:
%
\begin{itemize}
\item
The packages
\href{http://ctan.org/pkg/docmute}{\textsf{docmute}},
\href{http://ctan.org/pkg/includex}{\textsf{includex}} and
\href{http://ctan.org/pkg/standalone}{\textsf{standalone}}
provide commands to include only the document body of
a child file thus allowing both files to be compiled individually.
\item
The packages \href{http://ctan.org/pkg/subdocs}{\textsf{subdocs}}
and \href{http://ctan.org/pkg/subfiles}{\textsf{subfiles}}
provide structures in which the main and child documents can be
encapsulated and allowing them to be compiled individually.
The inclusion mechanism is different from the conventional |\include|.
\item
The package \href{http://ctan.org/pkg/combine}{\textsf{combine}}
is an elaborate solution to combine several documents into one.
\end{itemize}
%
See also the CTAN topic \href{http://ctan.org/topic/subdocs}{\textsf{subdocs}}
for further related packages.
The present package differs from the above solutions in that
a document structure constructed with the conventional |\include| mechanism
just needs two extra commands at the top of every file
such that all constituent files can be compiled individually.

%%%%%%%%%%%%%%%%%%%%%%%%%%%%%%%%%%%%%%%%%%%%%%%%%%%%%%%%%%%%%%%%%%%%%%%%%%%%%%%%
%\subsection{Feature Suggestions}
%
%The following is a list of features which may be useful for future
%versions of this package:
%%
%\begin{itemize}
%\item
%\ldots
%\end{itemize}

%%%%%%%%%%%%%%%%%%%%%%%%%%%%%%%%%%%%%%%%%%%%%%%%%%%%%%%%%%%%%%%%%%%%%%%%%%%%%%%%
\subsection{Revision History}

%%%%%%%%%%%%%%%%%%%%%%%%%%%%%%%%%%%%%%%%
\paragraph{v2.0:} 2018/12/30

\begin{itemize}
\item
immediate forward processing
\item
added |\childdocby| mechanism
\item
manual restructured
\end{itemize}

%%%%%%%%%%%%%%%%%%%%%%%%%%%%%%%%%%%%%%%%
\paragraph{v1.6:} 2018/01/17

\begin{itemize}
\item
application for development of include files
\item
corrections to manual
\end{itemize}

%%%%%%%%%%%%%%%%%%%%%%%%%%%%%%%%%%%%%%%%
\paragraph{v1.5:} 2017/05/21

\begin{itemize}
\item
more complete structuring introduced
\item
|\childdocof| introduced
\item
|\childdoc| renamed to |\childdocmain|
\item
|\childredirect| renamed to |\childdocforward| and |\childdocforwardprefix|
and functionality expanded
\end{itemize}

%%%%%%%%%%%%%%%%%%%%%%%%%%%%%%%%%%%%%%%%
\paragraph{v1.0:} 2017/04/27

\begin{itemize}
\item
manual and install package
\item
first version published on CTAN
\end{itemize}

%%%%%%%%%%%%%%%%%%%%%%%%%%%%%%%%%%%%%%%%
\paragraph{v0.6:} 2017/04/26

\begin{itemize}
\item
redirection mechanism added
\end{itemize}

%%%%%%%%%%%%%%%%%%%%%%%%%%%%%%%%%%%%%%%%
\paragraph{v0.5:} 2017/04/26

\begin{itemize}
\item
functionality in definition file
\end{itemize}


%%%%%%%%%%%%%%%%%%%%%%%%%%%%%%%%%%%%%%%%%%%%%%%%%%%%%%%%%%%%%%%%%%%%%%%%%%%%%%%%
%%%%%%%%%%%%%%%%%%%%%%%%%%%%%%%%%%%%%%%%%%%%%%%%%%%%%%%%%%%%%%%%%%%%%%%%%%%%%%%%
%%%%%%%%%%%%%%%%%%%%%%%%%%%%%%%%%%%%%%%%%%%%%%%%%%%%%%%%%%%%%%%%%%%%%%%%%%%%%%%%
\appendix

\settowidth\MacroIndent{\rmfamily\scriptsize 000\ }

 \DocInput{childdoc.dtx}

\end{document}
%</driver>
% \fi
%
% %%%%%%%%%%%%%%%%%%%%%%%%%%%%%%%%%%%%%%%%%%%%%%%%%%%%%%%%%%%%%%%%%%%%%%%%%%%%%%
% %%%%%%%%%%%%%%%%%%%%%%%%%%%%%%%%%%%%%%%%%%%%%%%%%%%%%%%%%%%%%%%%%%%%%%%%%%%%%%
% \section{Sample}
%\iffalse
%<*samplemain>
%\fi
%
% The following presents a sample document
% with two chapters, two parts, a title page,
% a compile flag as well as three forwarding files to set the flag.
% It consists of eight |.tex| files:
% \begin{center}
% \begin{tabular}{ll}
% |cdocsamp.tex|&main file\\
% |cdocsch1.tex|&include file for chapter 1\\
% |cdocsch2.tex|&include file for chapter 2\\
% |cdocspt3.tex|&include file for part 3\\
% |cdocspt4.tex|&include file for part 4\\
% |cdocsdrf.tex|&forwarding file for main file in draft mode\\
% |cdocsfi1.tex|&forwarding file for final version of chapter 1\\
% |cdocsfi2.tex|&forwarding file for final version of chapter 2\\
% \end{tabular}
% \end{center}
% Each of the eight files can be compiled directly by the \LaTeX{} compiler.
%
% %%%%%%%%%%%%%%%%%%%%%%%%%%%%%%%%%%%%%%
% \paragraph{Main File.}
%
% The main file is called |cdocsamp.tex|.
%
% Load the \textsf{childdoc} definitions and
% declare the filename for the main document:
%    \begin{macrocode}
\input{childdoc.def}
\childdocmain{}
%    \end{macrocode}

% Optional override for |\version| flag:
%    \begin{macrocode}
%%\ifchilddoc\else\providecommand{\version}{draft}\fi
%    \end{macrocode}

% Define the default values for the |\version| flag
% (|final| for the main file and |draft| for childs):
%    \begin{macrocode}
\ifchilddoc
\providecommand{\version}{draft}
\else
\providecommand{\version}{final}
\fi
%    \end{macrocode}

% Load the standard document class:
%    \begin{macrocode}
\documentclass[12pt]{article}
%    \end{macrocode}

% Start the document body:
%    \begin{macrocode}
\begin{document}
%    \end{macrocode}

% Declare a title page.
% Print title, part of document being processed and version flag:
%    \begin{macrocode}
\addtocounter{page}{-1}
\begin{center}
{\LARGE\bfseries{}childdoc example\par}
\vspace{1cm}
\ifchilddoc
\ifchilddocmanual part\else chapter\fi:
`\childdocname' of `\childdocjob'\par
\else
main document: `\childdocjob'\par
\fi
version: \version\par
\end{center}
\newpage
%    \end{macrocode}

% Manually include selected file,
% otherwise process as usual:
%    \begin{macrocode}
\ifchilddocmanual
\section*{part `\childdocname'}
\input{\childdocname}
\else
%    \end{macrocode}

% Include the two chapters:
%    \begin{macrocode}
\include{cdocsch1}
\include{cdocsch2}
%    \end{macrocode}

% Include the two parts unless only chapters should be displayed:
%    \begin{macrocode}
\ifchilddoc\else
\section{part three}
\input{cdocspt3}
\section{part four}
\input{cdocspt4}
\fi
%    \end{macrocode}

% Process as usual until here:
%    \begin{macrocode}
\fi
%    \end{macrocode}

% End of document body:
%    \begin{macrocode}
\end{document}
%    \end{macrocode}
%\iffalse
%</samplemain>
%\fi
%
% %%%%%%%%%%%%%%%%%%%%%%%%%%%%%%%%%%%%%%
% \paragraph{Chapter Include Files.}
%
% The include files are called |cdocsch1.tex| and |cdocsch2.tex|.
%
%\iffalse
%<*samplechap1|samplechap2>
%\fi

% Optional override for |\version| flag:
%    \begin{macrocode}
%%\providecommand{\version}{final}
%    \end{macrocode}

% Include the main document:
%    \begin{macrocode}
\input{childdoc.def}
\childdocof{cdocsamp}
%    \end{macrocode}

%\iffalse
%</samplechap1|samplechap2>
%\fi
%
%\iffalse
%<*samplechap1>
%\fi
% Some text for chapter 1:
%    \begin{macrocode}
\section{one}
some text in chapter one
%    \end{macrocode}

%\iffalse
%</samplechap1>
%\fi
% Some text for chapter 2:
%\iffalse
%<*samplechap2>
%\fi
%    \begin{macrocode}
\section{two}
more text in chapter two
%    \end{macrocode}

%\iffalse
%</samplechap2>
%\fi
%
% %%%%%%%%%%%%%%%%%%%%%%%%%%%%%%%%%%%%%%
% \paragraph{Part Include Files.}
%
% The include files are called |cdocspt3.tex| and |cdocspt4.tex|.
%
%\iffalse
%<*samplepart3|samplepart4>
%\fi

% Optional override for |\version| flag:
%    \begin{macrocode}
%%\providecommand{\version}{final}
%    \end{macrocode}

% Include the main document:
%    \begin{macrocode}
\input{childdoc.def}
\childdocby{cdocsamp}
%    \end{macrocode}

%\iffalse
%</samplepart3|samplepart4>
%\fi
%
%\iffalse
%<*samplepart3>
%\fi
% Some text for part 3:
%    \begin{macrocode}
some text in part three
%    \end{macrocode}

%\iffalse
%</samplepart3>
%\fi
% Some text for part 4:
%\iffalse
%<*samplepart4>
%\fi
%    \begin{macrocode}
more text in part four
%    \end{macrocode}

%\iffalse
%</samplepart4>
%\fi
%
% %%%%%%%%%%%%%%%%%%%%%%%%%%%%%%%%%%%%%%
% \paragraph{Forwarding for a Complete Draft.}
%
% The following forwarding file |cdocsdrf.tex|
% compiles the main document in draft mode:
%\iffalse
%<*sampledraft>
%\fi
%    \begin{macrocode}
\def\version{draft}
\input{childdoc.def}
\childdocforward{cdocsamp}
%    \end{macrocode}

%\iffalse
%</sampledraft>
%\fi
%
% %%%%%%%%%%%%%%%%%%%%%%%%%%%%%%%%%%%%%%
% \paragraph{Forwarding for Final Version of the Chapters.}
%
% The following forwarding files |cdocsfn1.tex| and |cdocsfn2.tex|
% (with identical content)
% compile the final versions of the child documents
% |cdocsch1.tex| and |cdocsch2.tex|, respectively:
%\iffalse
%<*samplefinal>
%\fi
%    \begin{macrocode}
\def\version{final}
\input{childdoc.def}
\childdocforwardprefix[cdocsamp]{cdocsfn}{cdocsch}
%    \end{macrocode}

%\iffalse
%</samplefinal>
%\fi
%
% %%%%%%%%%%%%%%%%%%%%%%%%%%%%%%%%%%%%%%
% \paragraph{Command Line Processing.}
%
% The following three command lines generate the output files
% |cdocscld|, |cdocscl1| and |cdocscl2|
% which should be identical to
% |cdocsdrf|, |cdocsch1| and |cdocsfn2|, respectively:
% \begin{center}
% \begin{tabular}{l}
% |latex -jobname cdocscld \|\\
% |  "\def\version{draft}\input{childdoc.def}\childdocforward{cdocsamp}"|\\
% |latex -jobname cdocscl1 \|\\
% |  "\input{childdoc.def}\childdocforward[cdocsamp]{cdocsch1}"|\\
% |latex -jobname cdocscl2 \|\\
% |  "\def\version{final}\input{childdoc.def}\childdocforward{cdocsch2}"|
% \end{tabular}
% \end{center}
% Note that the trailing backslash on each first line
% merely continues the input to the second line
% (for convenient cut ant paste).
% Furthermore, the command |latex| can be replaced by any
% of its alternative versions such as |pdflatex|.
%
% %%%%%%%%%%%%%%%%%%%%%%%%%%%%%%%%%%%%%%%%%%%%%%%%%%%%%%%%%%%%%%%%%%%%%%%%%%%%%%
% %%%%%%%%%%%%%%%%%%%%%%%%%%%%%%%%%%%%%%%%%%%%%%%%%%%%%%%%%%%%%%%%%%%%%%%%%%%%%%
% \section{Implementation}
%\iffalse
%<*package>
%\fi
%
% This section describes the definitions file |childdoc.def|.

% The definitions cannot be loaded using |\usepackage| or |\RequirePackage|
% which has a mechanism to prevent loading a style file more than once.
% When loading the definitions by means of |\input|
% multiple instances have to be prevented manually:
%\iffalse
%This code needs to be before the `\ProvidesFile' directive
%which is defined at the beginning of this file.
%Therefore it is also placed there and commented out here.
%</package>
%<*discard>
%\fi
%    \begin{macrocode}
\ifdefined\childdocmain\endinput\fi
%    \end{macrocode}
%\iffalse
%</discard>
%<*package>
%\fi
%
% \macro{\ifchilddoc}
% \macro{\ifchilddocmanual}
% The conditional |\ifchilddoc| tells whether a
% child (true) or main (false) document is being compiled.
% The conditional |\ifchilddocmanual| tells whether
% the |\includeonly| mechanism is used (false) or
% the selection of child files must be performed manually (true).
% The definitions initialise to false:
%    \begin{macrocode}
\newif\ifchilddoc
\newif\ifchilddocmanual
%    \end{macrocode}

% \macro{\childdocname}
% \macro{\childdocjob}
% The macro |\childdocname| stores the name of the main document
% to be compiled. The macro |\childdocjob| stores the name of
% the document on which the \LaTeX{} compiler was originally invoked.
% The content of |\jobname| cannot be compared
% to filenames specified in the source due to different catcodes.
% The following code rescans |\jobname|, stores the result
% in |\childdocname| and saves a copy in |\childdocjob|:
%    \begin{macrocode}
\edef\childdocname{\scantokens\expandafter{\jobname\noexpand}}
\let\childdocjob\childdocname
%    \end{macrocode}

% \macro{\childdocdisable}
% The macro |\childdocdisable| prevents the main file
% from being processed more than once.
% At this stage, the main document command |\childdocmain|
% is assumed to be called once again where it should do nothing.
% Any subsequent call to it should prevent
% a secondary processing of the main document
% It overwrites the forwarding commands
% |\childdocof| and |\childdocforward|
% with empty macros to prevent further inclusions of the main document:
%    \begin{macrocode}
\newcommand{\childdocdisable}
{
  \renewcommand{\childdocmain}[1]{\renewcommand{\childdocmain}[1]{\endinput}}
  \renewcommand{\childdocof}[1]{}
  \renewcommand{\childdocby}[2][]{}
  \renewcommand{\childdocforward}[2][]{}
  \renewcommand{\childdocdisable}{}
}
%    \end{macrocode}

% \macro{\childdocmain}
% The macro |\childdocmain| is to be called at the top of the main file
% with nothing or the main filename (without extension) as argument.
% First, it breaks loops.
% If the argument is not empty and does not match |\childdocname|
% (which is set by the first inclusion of |childdoc.def|),
% |\ifchilddoc| is set to true, |\includeonly| is applied to the child file
% and |\jobname| is set to the main file
% (for proper handling of |.aux| files):
%    \begin{macrocode}
\newcommand{\childdocmain}[1]
{
  \childdocdisable\childdocmain{}
  \if?#1?\else
    \begingroup
      \def\childdoctmp{#1}
      \ifx\childdoctmp\childdocname
        \def\childdoctmp{}
      \else
        \def\childdoctmp
        {
          \childdoctrue
          \includeonly{\childdocname}
          \def\childdocjob{#1}
          \def\jobname{#1}
        }
      \fi
      \expandafter
    \endgroup
    \childdoctmp
  \fi
}
%    \end{macrocode}

% \macro{\childdocof}
% The command |\childdocof| redirects
% compilation to the main file |#1|.
%    \begin{macrocode}
\newcommand{\childdocof}[1]
{
  \childdocdisable
  \childdoctrue
  \includeonly{\childdocname}
  \def\jobname{#1}
  \def\childdocjob{#1}
  \input{#1}
}
%    \end{macrocode}

% \macro{\childdocby}
% The command |\childdocby| ....
%    \begin{macrocode}
\newcommand{\childdocby}[2][]
{
  \childdocdisable
  \childdoctrue
  \childdocmanualtrue
  \if?#1?\else
    \def\jobname{#2}
  \fi
  \def\childdocjob{#2}
  \input{#2}
  \endinput
}
%    \end{macrocode}

% \macro{\childdocforward}
% The command |\childdocforward| redirects
% compilation to the main file or
% (if the optional argument is given) a child file.
% Parameters are set as if the main file
% or a child file starting with |\childdocof| was compiled.
% Then compilation is handed over to the main file:
%    \begin{macrocode}
\newcommand{\childdocforward}[2][]
{
  \begingroup
    \if?#1?
      \def\childdoctmp
      {
        \def\childdocname{#2}
        \def\childdocjob{#2}
        \def\jobname{#2}
        \input{#2}
        \endinput
      }
    \else
      \def\childdoctmp
      {
        \childdocdisable
        \def\childdocname{#2}
        \childdoctrue
        \includeonly{#2}
        \def\childdocjob{#1}
        \def\jobname{#1}
        \input{#1}
        \endinput
      }
    \fi
    \expandafter
  \endgroup
  \childdoctmp
}
%    \end{macrocode}

% \macro{\childdocforwardprefix}
% The command |\childdocforwardprefix| redirects
% compilation to the main or a child file by means of a pattern.
% The prefix |#1| in the current filename is replaced by |#2|
% and the suffix of the current filename is kept
% (it is assumed that the filename does not contain the substring `|~~~|'
% which is used as a delimiter).
% Compilation is handed over to the new file by |\childdocforward|:
%    \begin{macrocode}
\newcommand{\childdocforwardprefix}[3][]
{
  \begingroup
    \def\childdocextract #2##1~~~{\def\childdoctmp{\childdocforward[#1]{#3##1}}}
    \expandafter\childdocextract\childdocname~~~
    \expandafter
  \endgroup
  \childdoctmp
}
%    \end{macrocode}

% \macro{\childdoc}
% The deprecated macro |\childdoc| is a legacy version of |\childdocmain|:
%    \begin{macrocode}
\newcommand{\childdoc}{\childdocmain}
%    \end{macrocode}

% \macro{\childdocredirect}
% The deprecated macro |\childdocredirect| is a legacy version
% of |\childdocforward| and |\childdocforwardprefix|:
%    \begin{macrocode}
\newcommand{\childdocredirect}[2][]
{
  \begingroup
    \if?#1?
      \def\childdoctmp{\childdocforward{#2}}
    \else
      \def\childdoctmp{\childdocforwardprefix{#1}{#2}}
    \fi
    \expandafter
  \endgroup
  \childdoctmp
}
%    \end{macrocode}

%\iffalse
%</package>
%\fi
%
\endinput
|\\
|\childdocforwardprefix[|\textit{main}|]{|\textit{prefix}|}{|\textit{dest}|}|
\end{tabular}
\end{center}
%
the destination file is determined by a pattern
depending on the current file:
To make this work, the current file must be called
`{\textit{prefix}\hspace{0.2em}\textit{suffix}}'
with \textit{prefix} matching precisely the argument.
Processing is then passed on to the file
`{\textit{dest}\hspace{0.2em}\textit{suffix}}'.
Surely, the same effect is achieved by
directly specifying the
argument `{\textit{dest}\hspace{0.2em}\textit{suffix}}'
in the first form.
However, that requires to set up a different file
for each child. With the alternative form of the command
all these files can have exactly the same content
which simplifies setting them up and maintaining them.

For example, the following file |draft.tex|
with a compilation flag |\version| as described in \secref{sec:flags}
compiles the main document as a draft:
%
\begin{center}
\begin{tabular}{l}
|\def\version{draft}|\\
|% \iffalse
%
% childdoc.dtx Copyright (C) 2017-2018 Niklas Beisert
%
% This work may be distributed and/or modified under the
% conditions of the LaTeX Project Public License, either version 1.3
% of this license or (at your option) any later version.
% The latest version of this license is in
%   http://www.latex-project.org/lppl.txt
% and version 1.3 or later is part of all distributions of LaTeX
% version 2005/12/01 or later.
%
% This work has the LPPL maintenance status `maintained'.
%
% The Current Maintainer of this work is Niklas Beisert.
%
% This work consists of the files childdoc.dtx and childdoc.ins
% and the derived files childdoc.def and cdocsamp.tex with
% cdocsch1.tex, cdocsch2.tex, cdocsdrf.tex, cdocsfn1.tex, cdocsfn2.tex.
%
%<package>\ifdefined\childdocmain\endinput\fi
%<package>\ProvidesFile{childdoc.def}[2018/12/30 v2.0 child document driver]
%<samplemain>\ProvidesFile{cdocsamp.tex}[2018/12/30 v2.0 sample for childdoc]
%<*driver>
%\ProvidesFile{childdoc.drv}[2018/12/30 v2.0 childdoc reference manual file]
\PassOptionsToClass{10pt,a4paper}{article}
\documentclass{ltxdoc}

\usepackage[margin=35mm]{geometry}
\usepackage{hyperref}
\usepackage{hyperxmp}
\usepackage[usenames]{color}

\hypersetup{colorlinks=true}
\hypersetup{pdfstartview=FitH}
\hypersetup{pdfpagemode=UseNone}
\hypersetup{pdfsource={}}
\hypersetup{pdflang={en-UK}}
\hypersetup{pdfcopyright={Copyright 2017-2018 Niklas Beisert.
  This work may be distributed and/or modified under the
  conditions of the LaTeX Project Public License, either version 1.3
  of this license or (at your option) any later version.}}
\hypersetup{pdflicenseurl={http://www.latex-project.org/lppl.txt}}
\hypersetup{pdfcontactaddress={ETH Zurich, ITP, HIT K,
  Wolfgang-Pauli-Strasse 27}}
\hypersetup{pdfcontactpostcode={8093}}
\hypersetup{pdfcontactcity={Zurich}}
\hypersetup{pdfcontactcountry={Switzerland}}
\hypersetup{pdfcontactemail={nbeisert@itp.phys.ethz.ch}}
\hypersetup{pdfcontacturl={http://people.phys.ethz.ch/\xmptilde nbeisert/}}

\newcommand{\secref}[1]{\hyperref[#1]{section \ref*{#1}}}

\parskip1ex
\parindent0pt
\let\olditemize\itemize
\def\itemize{\olditemize\parskip0pt}

\begin{document}

\title{The \textsf{childdoc} Package}
\hypersetup{pdftitle={The childdoc Package}}
\author{Niklas Beisert\\[2ex]
  Institut f\"ur Theoretische Physik\\
  Eidgen\"ossische Technische Hochschule Z\"urich\\
  Wolfgang-Pauli-Strasse 27, 8093 Z\"urich, Switzerland\\[1ex]
  \href{mailto:nbeisert@itp.phys.ethz.ch}
  {\texttt{nbeisert@itp.phys.ethz.ch}}}
\hypersetup{pdfauthor={Niklas Beisert}}
\hypersetup{pdfsubject={Manual for the LaTeX2e Package childdoc}}
\date{30 December 2018, \textsf{v2.0}}
\maketitle

\begin{abstract}\noindent
\textsf{childdoc} is a \LaTeXe{} package
that enables the direct compilation
of document sections included by |\include|
to individual files.
\end{abstract}

\begingroup
\parskip0ex
\tableofcontents
\endgroup

%%%%%%%%%%%%%%%%%%%%%%%%%%%%%%%%%%%%%%%%%%%%%%%%%%%%%%%%%%%%%%%%%%%%%%%%%%%%%%%%
%%%%%%%%%%%%%%%%%%%%%%%%%%%%%%%%%%%%%%%%%%%%%%%%%%%%%%%%%%%%%%%%%%%%%%%%%%%%%%%%
\section{Introduction}

\LaTeX{} provides a mechanism to structure a large document (such as a book)
into a main file and several child files (containing the chapters)
using the |\include| command.
This mechanism is beneficial for documents
which span hundreds of pages in order to
make the source file(s) more manageable.
Moreover, compilation can be restricted to
selected child files by means of the |\includeonly| command.
The latter feature can be used to reduce the compilation time while editing
(this was significantly more useful in the earlier days of \LaTeX{})
or to generate a smaller document which is easier to navigate.
Another application of |\includeonly| is to generate
documents consisting of selected parts of the complete document.

However, there are a few drawbacks of the plain |\include| mechanism:
\begin{itemize}
\item
The child files cannot be compiled on their own,
they can only be compiled via the main file.
A naive editing environment
(such as a text editor with an option
to have the current file processed by \LaTeX)
may require one to switch to the main file before compiling;
attempting to compile the child file produces errors.
\item
The main file must be modified (each time)
to adjust the |\includeonly| command
to the present needs. This easily leaves the main file in a messy state.
\item
The generated document will always carry the filename
of the main document. This is inconvenient if
several child files are to be compiled and
to be kept for distribution.
\end{itemize}

The present package provides a simple interface
to make child files individually compilable by \LaTeX{}.
Compiling a child file then has the same effect as compiling
the main file with an |\includeonly| command
to select the appropriate child.
Moreover the generated document will carry the name of the child
rather than the main file.
This resolves all three above issues.

This feature is meant to make the editing of books,
thesis documents and lecture notes somewhat more convenient.
However, the package can also be used efficiently for
composing a series of documents (such as exercise sheets)
which are typically distributed individually.
It then assists the author in generating the individual documents
(potentially in different versions)
as well as a document containing the collected series.
Another application is in developing style files
or other kinds of included material
where compilation of the style file could redirect
to a sample or test file.

%%%%%%%%%%%%%%%%%%%%%%%%%%%%%%%%%%%%%%%%%%%%%%%%%%%%%%%%%%%%%%%%%%%%%%%%%%%%%%%%
%%%%%%%%%%%%%%%%%%%%%%%%%%%%%%%%%%%%%%%%%%%%%%%%%%%%%%%%%%%%%%%%%%%%%%%%%%%%%%%%
\section{Usage}

First of all, the package \textsf{childdoc} is \emph{not} a standard
\LaTeXe{} |.sty| style file! Therefore it needs to be invoked in
a non-standard way.

%%%%%%%%%%%%%%%%%%%%%%%%%%%%%%%%%%%%%%%%%%%%%%%%%%%%%%%%%%%%%%%%%%%%%%%%%%%%%%%%
\subsection{Included Files}
\label{sec:include}

%%%%%%%%%%%%%%%%%%%%%%%%%%%%%%%%%%%%%%%%
\DescribeMacro{\childdocmain}
To use the package, add the commands
\begin{center}
\begin{tabular}{l}
|\input{childdoc.def}|\\
|\childdocmain{}|\\
\end{tabular}
\end{center}
at the very top of the main \LaTeX{} file,
in particular \emph{before} the |\documentclass| statement!
The argument of |\childdocmain| should be left empty
(but it must be present).

%%%%%%%%%%%%%%%%%%%%%%%%%%%%%%%%%%%%%%%%
\DescribeMacro{\childdocof}
Furthermore, add the commands
\begin{center}
\begin{tabular}{l}
|\input{childdoc.def}|\\
|\childdocof{|\textit{main}|}|\\
\end{tabular}
\end{center}
at the top of every child file \textit{child}
which is included by |\include{|\textit{child}|}|
from within the main file
(or at least for those files to be compiled individually).
The argument \textit{main} must be the filename of the main file.

There are a couple of
considerations in setting up the main and child documents:

%%%%%%%%%%%%%%%%%%%%%%%%%%%%%%%%%%%%%%%%
\paragraph{Restrictions.}

Please note the following restrictions:
\begin{itemize}
\item
|\childdocmain| must be called with one argument \textit{main}
to ensure compatibility with earlier version of the package.
It must either be empty (|\childdocmain{}|)
or precisely match the filename of the main file in which it is specified.
See \secref{sec:detection} for further information.
\item
The filename \textit{main} must be specified without the |.tex| extension.
\item
The filename \textit{main} is case sensitive
(even in case-insensitive file systems)
due to internal string comparison.
\item
The argument \textit{main} should be fully expanded, it cannot be a macro.
\item
Subdirectories and special characters should be avoided in filenames.
\item
The command |\childdocmain{|\textit{main}|}| must be followed by a whitespace.
It should not be followed immediately by another command
or by a comment mark `|%|'.
This is because the \TeX{} parser reads the token immediately following
the argument of |\childdocmain| and puts it
at the beginning of every child section;
however, a white\-space is ignored.
\end{itemize}

%%%%%%%%%%%%%%%%%%%%%%%%%%%%%%%%%%%%%%%%
\paragraph{Content of Main File.}

It is advisable to place all content in the child files included by |\include|.
Any output contained in the main file will appear in all child documents
unless suppressed manually;
it cannot be suppressed automatically by the |\includeonly| directive
and thus should normally be avoided.
A method to include some content in the main file
by means of conditional processing is described in \secref{sec:conditional}.

%%%%%%%%%%%%%%%%%%%%%%%%%%%%%%%%%%%%%%%%
\paragraph{Page Numbering.}

When only a part of the document is compiled,
the appropriate numbering of pages
(as well as other status parameters)
is determined from the |.aux| files.
The latter contain information from previous passes.
However this information needs to propagate through
all intermediate child documents.
Therefore the page numbering in child documents may well
be inconsistent until the complete document is compiled at least once.

A useful (if unconventional) way to always ensure a consistent
page numbering is to restart the numbering in each child document
and denote the pages by `\textit{child}|.|\textit{page}'
where \textit{child} represents the chapter/section number of the child file.
This can be achieved by the command
|\numberwithin{page}{|\textit{child}|}|
of the \textsf{amsmath} package
where \textit{child} can be |chapter| or |section|
depending on the chosen structuring.
Alternatively, one can modify the macro |\thepage| appropriately
and reset the counter |page| at the start of each child file.

%%%%%%%%%%%%%%%%%%%%%%%%%%%%%%%%%%%%%%%%%%%%%%%%%%%%%%%%%%%%%%%%%%%%%%%%%%%%%%%%
\subsection{Conditional Processing}
\label{sec:conditional}

The package provides a mechanism to compile different versions
of a document. To customise the versions further some conditional processing
can come in handy to distinguish which version is being compiled.
The package provides two macros to describe the compilation context:

%%%%%%%%%%%%%%%%%%%%%%%%%%%%%%%%%%%%%%%%
\DescribeMacro{\ifchilddoc}
The conditional |\ifchilddoc| distinguishes between the compilation of
child documents and the main document:
%
\begin{center}
|\ifchilddoc |\textit{child-code}| |[|\||else |\textit{main-code}]| \||fi|
\end{center}

%%%%%%%%%%%%%%%%%%%%%%%%%%%%%%%%%%%%%%%%
\DescribeMacro{\childdocname}
\DescribeMacro{\childdocjob}
The macro |\childdocname| contains the filename (without extension)
of the main or child file being processed.
Note that |\childdocjob| will always contain the name of the main file.

%%%%%%%%%%%%%%%%%%%%%%%%%%%%%%%%%%%%%%%%
\paragraph{Title Page.}

Conditional processing can be used to include a title or banner page
in the main document when proper precautions are taken.
Importantly, the code in the main file should ensure that the page counter
(as well as other status parameters which are stored in the |.aux| files)
takes the same value after the conditional processing.
Otherwise the page numbers may take divergent values
depending on which part is compiled.

For example, a title page could be declared by:
%
\begin{center}
\begin{tabular}{l}
|\ifchilddoc\||else|\\
|\addtocounter{page}{-1}|\\
\textit{code for title page}\\
|\newpage|\\
|\||fi|
\end{tabular}
\end{center}
%
A banner page for the child documents can be generated by:
%
\begin{center}
\begin{tabular}{l}
|\ifchilddoc|\\
|\addtocounter{page}{-1}|\\
\textit{code for banner page}\\
|\newpage|\\
|\||fi|
\end{tabular}
\end{center}
%
Here one could write a message such as:
\begin{center}
|This is the part \childdocname{} of \childdocjob{}.|
\end{center}

%%%%%%%%%%%%%%%%%%%%%%%%%%%%%%%%%%%%%%%%%%%%%%%%%%%%%%%%%%%%%%%%%%%%%%%%%%%%%%%%
\subsection{Flags}
\label{sec:flags}

The package makes it easy to generate different versions
of the main or child documents.
To this end compilation flags can be defined
and assigned different default values.
They will be particularly useful in conjunction
with the forwarding mechanism described in \secref{sec:forward}.

For example, it may be useful to have a flag |\version|
which can be set to |draft| or |final|.
The document source will contain some conditional code
depending on the value of |\version|.
Suppose further, the flag should default to |final| for the main file
and to |draft| for child files
which is a natural assignment for editing the document.
This is achieved by placing the following code
in the preamble of the main document
(below the |\childdocmain| directive):
%
\begin{center}
\begin{tabular}{l}
|\ifchilddoc|\\
|\providecommand{\version}{draft}|\\
|\||else|\\
|\providecommand{\version}{final}|\\
|\||fi|
\end{tabular}
\end{center}
%
The definition by |\providecommand| makes sure
that previous definitions are not overwritten.
Further statements |\providecommand{\version}{...}|
can thus be added before the above code to override it.

For the main file, one might add a line
(between |\childdocmain| and the above block)
%
\begin{center}
|%\ifchilddoc\||else\providecommand{\version}{draft}\||fi|
\end{center}
%
which can be uncommented to produce a draft version.
Likewise one can add a line to the very top of a child file
(above the |\childdocof{|\textit{main}|}| directive)
%
\begin{center}
|%\providecommand{\version}{final}|
\end{center}
%
which can be uncommented to produce the final version of this child document.

%%%%%%%%%%%%%%%%%%%%%%%%%%%%%%%%%%%%%%%%%%%%%%%%%%%%%%%%%%%%%%%%%%%%%%%%%%%%%%%%
\subsection{Forwarding}
\label{sec:forward}

Different versions of the main or child documents
using compilation flags as described in \secref{sec:flags}
can be (permanently) stored in different files
for convenient compilation, viewing and distribution.
To this end, the package defines a command
to pass on compilation to a different file:

%%%%%%%%%%%%%%%%%%%%%%%%%%%%%%%%%%%%%%%%
\DescribeMacro{\childdocforward}
The command |\childdocforward| redirects processing to
another source file:
%
\begin{center}
\begin{tabular}{l}
|\input{childdoc.def}|\\
|\childdocforward[|\textit{main}|]{|\textit{dest}|}|\\
\end{tabular}
\end{center}
%
The argument \textit{dest} is the destination file
(without extension).
It should be the main file or one of the child files.
Note that further \textsf{childdoc} directives
such as |\childdocof| and |\childdocforward|
in the indicated file will be processed in this form.
The optional argument \textit{main}
passes on directly to the main file \textit{main}
while pretending to compile the child \textit{dest}.
This form behaves as if \textit{dest}
issues |\childdocof{|\textit{main}|}| right away,
and no further \textsf{childdoc} directives will be processed.

%%%%%%%%%%%%%%%%%%%%%%%%%%%%%%%%%%%%%%%%
\DescribeMacro{\...prefix}
In the alternative form |\childdocforwardprefix|,
%
\begin{center}
\begin{tabular}{l}
|\input{childdoc.def}|\\
|\childdocforwardprefix[|\textit{main}|]{|\textit{prefix}|}{|\textit{dest}|}|
\end{tabular}
\end{center}
%
the destination file is determined by a pattern
depending on the current file:
To make this work, the current file must be called
`{\textit{prefix}\hspace{0.2em}\textit{suffix}}'
with \textit{prefix} matching precisely the argument.
Processing is then passed on to the file
`{\textit{dest}\hspace{0.2em}\textit{suffix}}'.
Surely, the same effect is achieved by
directly specifying the
argument `{\textit{dest}\hspace{0.2em}\textit{suffix}}'
in the first form.
However, that requires to set up a different file
for each child. With the alternative form of the command
all these files can have exactly the same content
which simplifies setting them up and maintaining them.

For example, the following file |draft.tex|
with a compilation flag |\version| as described in \secref{sec:flags}
compiles the main document as a draft:
%
\begin{center}
\begin{tabular}{l}
|\def\version{draft}|\\
|\input{childdoc.def}|\\
|\childdocforward{|\textit{main}|}|
\end{tabular}
\end{center}
%
Likewise, the following files |final|\textit{nn}|.tex|
compile the final version of the child document
|child|\textit{nn}|.tex|:
%
\begin{center}
\begin{tabular}{l}
|\def\version{final}|\\
|\input{childdoc.def}|\\
|\childdocforwardprefix{final}{child}|
\end{tabular}
\end{center}
%

Note that when several versions of a main file and/or of each child file
are to be generated, it may be convenient to set up a |Makefile| or
shell script to automatise the process.

%%%%%%%%%%%%%%%%%%%%%%%%%%%%%%%%%%%%%%%%%%%%%%%%%%%%%%%%%%%%%%%%%%%%%%%%%%%%%%%%
\subsection{Command Line Processing}
\label{sec:commandline}

The effect of redirection files can also be achieved by invoking
the \LaTeX{} compiler with a more elaborate command line.
Most conveniently this should be done as part
of a shell script or a |Makefile|.

When using \textsf{childdoc} in the main file, the following
command lines effectively perform a redirection
(note that depending on the shell being used,
backslashes may have to be doubled: `|\|' $\to$ `|\\|'):
%
\begin{center}
|... -jobname "|\textit{target}|" |\\|"|[\textit{flags}]%
|\input{childdoc.def}\childdocforward[|\textit{main}|]{|\textit{dest}|}"|
\end{center}
%
Here \textit{target} is the name of the output file,
\textit{main} is the name of the main file
and \textit{dest} is the name of the main or child file to be processed
(all filenames without extensions).
The optional argument \textit{main} can be omitted
if \textit{main} matches \textit{dest}.
Optionally, compilation \textit{flags} can be defined via |\def| commands.
This command line makes the \TeX{} engine believe
it is compiling the file \textit{target}
whose content is specified as the latter parameter.
The provided code then forwards the processing to
\textit{main} or \textit{dest} as described in \secref{sec:forward}.

%%%%%%%%%%%%%%%%%%%%%%%%%%%%%%%%%%%%%%%%%%%%%%%%%%%%%%%%%%%%%%%%%%%%%%%%%%%%%%%%
\subsection{Include by Input}
\label{sec:input}

Including child documents by |\include| has some restrictions by design.
Most notably, the content of a child document always occupies
its own set of pages; pages cannot be shared between child documents.
Usually, this behaviour makes perfect sense
because each child document contain an essential part of the document.
However, in some situations it may be desirable to compose
a document from a collection of parts
without having mandatory page breaks between then.
For this case, the package
provides a mechanism to include parts
by |\input| which can also be processed individually.
However, by construction this mechanism
requires manual handling of the content to be output.

%%%%%%%%%%%%%%%%%%%%%%%%%%%%%%%%%%%%%%%%
\DescribeMacro{\ifchilddocmanual}
The main file should be prepared as usual, see \secref{sec:include}.
However, the document body must make a distinction
between processing of an individual part and of the main document, e.g.:
%
\begin{center}
\begin{tabular}{l}
|\ifchilddocmanual|\\
|\input{\childdocname}|\\
|\||else|\\
\textit{document body with }|\input{|\textit{part}|}|\\
|\||fi|
\end{tabular}
\end{center}
%
The conditional |\ifchilddocmanual| is true whenever
a part to be included by |\input| is being compiled,
and the name of the part is stored in |\childdocname|.

%%%%%%%%%%%%%%%%%%%%%%%%%%%%%%%%%%%%%%%%
\DescribeMacro{\childdocby}
Each part to be included by |\input| should start with:
%
\begin{center}
\begin{tabular}{l}
|\input{childdoc.def}|\\
|\childdocby{|\textit{main}|}|\\
\end{tabular}
\end{center}
%
The directive |\childdocby| is similar to |\childdocof|
described in \secref{sec:include},
but the subsequent selection of content must be done manually.
To that end, both |\ifchilddoc| and |\ifchilddocmanual|
will be true upon processing of a part,
and the name of the part is stored in |\childdocname|.
Note that |\jobname| will be set to the filename of the current part
so that each part receives an individual |.aux| file
that does not interfere with the |.aux| file(s) of the main document.
This behaviour can be altered by the alternative form
|\childdocby[*]{|\textit{main}|}| (with a non-empty optional argument)
which uses the |.aux| file of the main document
by setting |\jobname| to \textit{main}.

%%%%%%%%%%%%%%%%%%%%%%%%%%%%%%%%%%%%%%%%%%%%%%%%%%%%%%%%%%%%%%%%%%%%%%%%%%%%%%%%
\subsection{Driver Development}
\label{sec:driver}

The \textsf{childdoc} mechanism can also be use for the development
of definition files such as \LaTeX{} styles or classes.
This case differs from the above setup with multiple parts
included by |\include| in that no |\includeonly| should be invoked.
This can be achieved by starting the include file
(before |\ProvidesPackage|) with:
%
\begin{center}
\begin{tabular}{l}
|\input{childdoc.def}|\\
|\childdocforward{|\textit{main}|}|\\
\end{tabular}
\end{center}
%
or alternatively with:
%
\begin{center}
\begin{tabular}{l}
|\input{childdoc.def}|\\
|\childdocby{|\textit{main}|}|\\
\end{tabular}
\end{center}
%
Both forms have slightly different effects as described above.
The main file is prepared as usual, see \secref{sec:include}.

%%%%%%%%%%%%%%%%%%%%%%%%%%%%%%%%%%%%%%%%%%%%%%%%%%%%%%%%%%%%%%%%%%%%%%%%%%%%%%%%
\subsection{Legacy Detection}
\label{sec:detection}

The directive |\childdocmain| in the main file can detect
whether the complete document or merely a child is to be compiled
even without using the directive |\childdocof|.
This method is deprecated because it is less robust
and there is no compelling reason to use it;
it is merely provided for backward compatibility
and it may be removed in future versions.

If the detection mechanism is to be used,
it is mandatory to correctly specify
the filename of the main file as the argument of |\childdocmain|:
%
\begin{center}
\begin{tabular}{l}
|\input{childdoc.def}|\\
|\childdocmain{|\textit{main}|}|\\
\end{tabular}
\end{center}
%
If |\jobname| does not match the argument \textit{main} of |\childdocmain|,
it is assumed that |\jobname| points to the child file to be compiled.
When using |\childdocmain| with the main file specified as argument,
it suffices to start a child file
with just |\input{|\textit{main}|}|
without loading of the package and using |\childdocof|.
If instead all processing is done
with the appropriate \textsf{childdoc} directives,
the argument of \textit{main} of |\childdocmain| can be empty.

An alternative version of the command line processing described
in \secref{sec:commandline} using the detection mechanism reads:
%
\begin{center}
|... -jobname "|\textit{target}|" "|[\textit{flags}]%
[|\def\jobname{|\textit{dest}|}|]|\input{|\textit{main}|}"|
\end{center}

%%%%%%%%%%%%%%%%%%%%%%%%%%%%%%%%%%%%%%%%%%%%%%%%%%%%%%%%%%%%%%%%%%%%%%%%%%%%%%%%
\subsection{Manual Code}
\label{sec:manual}

In case one cannot be certain whether the definitions file |childdoc.def|
is installed on the target \TeX{} distribution
and one prefers not to ship it,
it is conceivable to paste a few relevant commands into the sources.

To that end, drop all statements |\input{childdoc.def}|
and perform the replacements as outlined below.
Instead of |\childdocmain{|\textit{main}|}| add the following code
to the top of the main file:
%
\begin{center}
\begin{tabular}{l}
|\||ifdefined\childdocname\endinput\||fi\newif\ifchilddoc|\\
|\edef\childdocname{\scantokens\expandafter{\jobname\noexpand}}|\\
|\def\childdocmain{|\textit{main}|}\||ifx\childdocmain\childdocname\||else|\\
|\childdoctrue\includeonly{\childdocname}\let\jobname\childdocmain\||fi|\\
\end{tabular}
\end{center}
%
Instead of |\childdocof{|\textit{main}|}| just include the main file
at the top of each child file:
%
\begin{center}
|\input{|\textit{main}|}|
\end{center}
%
A simple redirection |\childdocforward{|\textit{dest}|}| is achieved by:
%
\begin{center}
|\def\jobname{|\textit{dest}|}\input{\jobname}|
\end{center}
%
The redirection with prefix
|\childdocforwardprefix[|\textit{prefix}|]{|\textit{dest}|}|
is accomplished by:
%
\begin{center}
\begin{tabular}{l}
|{\edef\jobname{\scantokens\expandafter{\jobname\noexpand}}|\\
|\def\redirectjob |\textit{prefix}|#1~~~{\gdef\jobname{|\textit{dest}|#1}}|\\
|\expandafter\redirectjob\jobname~~~}\input{\jobname}|
\end{tabular}
\end{center}

In an alternative approach,
child documents can be compiled by a specific command line
without additional code or specific definitions:
%
\begin{center}
|... -jobname "|\textit{target}|" "|[\textit{flags}]%
|\includeonly{|\textit{dest}|}\input{|\textit{main}|}"|
\end{center}
%

%%%%%%%%%%%%%%%%%%%%%%%%%%%%%%%%%%%%%%%%%%%%%%%%%%%%%%%%%%%%%%%%%%%%%%%%%%%%%%%%
%%%%%%%%%%%%%%%%%%%%%%%%%%%%%%%%%%%%%%%%%%%%%%%%%%%%%%%%%%%%%%%%%%%%%%%%%%%%%%%%
\section{Information}

%%%%%%%%%%%%%%%%%%%%%%%%%%%%%%%%%%%%%%%%%%%%%%%%%%%%%%%%%%%%%%%%%%%%%%%%%%%%%%%%
\subsection{Copyright}

Copyright \copyright{} 2017--2018 Niklas Beisert

This work may be distributed and/or modified under the
conditions of the \LaTeX{} Project Public License, either version 1.3
of this license or (at your option) any later version.
The latest version of this license is in
  \url{http://www.latex-project.org/lppl.txt}
and version 1.3 or later is part of all distributions of \LaTeX{}
version 2005/12/01 or later.

This work has the LPPL maintenance status `maintained'.

The Current Maintainer of this work is Niklas Beisert.

This work consists of the files |README.txt|, |childdoc.ins| and |childdoc.dtx|
as well as the derived files |childdoc.def|, |cdocsamp.tex|
with |cdocsch1.tex|, |cdocsch2.tex|, |cdocspt3.tex|, |cdocspt4.tex|,
|cdocsdrf.tex|, |cdocsfn1.tex|, |cdocsfn2.tex|
as well as |childdoc.pdf|.

%%%%%%%%%%%%%%%%%%%%%%%%%%%%%%%%%%%%%%%%%%%%%%%%%%%%%%%%%%%%%%%%%%%%%%%%%%%%%%%%
\subsection{Files and Installation}

The package consists of the files:
%
\begin{center}
\begin{tabular}{ll}
    |README.txt|   & readme file \\
    |childdoc.ins| & installation file \\
    |childdoc.dtx| & source file \\
    |childdoc.def| & definition file \\
    |cdocsamp.tex| & sample main file \\
    |cdocsch1.tex| & sample include file \\
    |cdocsch2.tex| & sample include file \\
    |cdocspt3.tex| & sample part file \\
    |cdocspt4.tex| & sample part file \\
    |cdocsdrf.tex| & sample redirection file \\
    |cdocsfn1.tex| & sample redirection file \\
    |cdocsfn2.tex| & sample redirection file \\
    |childdoc.pdf| & manual
\end{tabular}
\end{center}
%
The distribution consists of the files
|README.txt|, |childdoc.ins| and |childdoc.dtx|.
%
\begin{itemize}
\item
Run (pdf)\LaTeX{} on |childdoc.dtx|
to compile the manual |childdoc.pdf| (this file).
\item
Run \LaTeX{} on |childdoc.ins| to create the definitions file |childdoc.def|
and the sample |cdocsamp.tex| with include files
|cdocsch1.tex|, |cdocsch2.tex|, |cdocspt3.tex|, |cdocspt4.tex|,
|cdocsdrf.tex|, |cdocsfn1.tex|, |cdocsfn2.tex|.
Then copy the file |childdoc.def| to an appropriate directory of your \LaTeX{}
distribution, e.g.\ \textit{texmf-root}|/tex/latex/childdoc|.
\end{itemize}

%%%%%%%%%%%%%%%%%%%%%%%%%%%%%%%%%%%%%%%%%%%%%%%%%%%%%%%%%%%%%%%%%%%%%%%%%%%%%%%%
\subsection{Related CTAN Packages}

There are several other packages which offer a similar functionality:
%
\begin{itemize}
\item
The packages
\href{http://ctan.org/pkg/docmute}{\textsf{docmute}},
\href{http://ctan.org/pkg/includex}{\textsf{includex}} and
\href{http://ctan.org/pkg/standalone}{\textsf{standalone}}
provide commands to include only the document body of
a child file thus allowing both files to be compiled individually.
\item
The packages \href{http://ctan.org/pkg/subdocs}{\textsf{subdocs}}
and \href{http://ctan.org/pkg/subfiles}{\textsf{subfiles}}
provide structures in which the main and child documents can be
encapsulated and allowing them to be compiled individually.
The inclusion mechanism is different from the conventional |\include|.
\item
The package \href{http://ctan.org/pkg/combine}{\textsf{combine}}
is an elaborate solution to combine several documents into one.
\end{itemize}
%
See also the CTAN topic \href{http://ctan.org/topic/subdocs}{\textsf{subdocs}}
for further related packages.
The present package differs from the above solutions in that
a document structure constructed with the conventional |\include| mechanism
just needs two extra commands at the top of every file
such that all constituent files can be compiled individually.

%%%%%%%%%%%%%%%%%%%%%%%%%%%%%%%%%%%%%%%%%%%%%%%%%%%%%%%%%%%%%%%%%%%%%%%%%%%%%%%%
%\subsection{Feature Suggestions}
%
%The following is a list of features which may be useful for future
%versions of this package:
%%
%\begin{itemize}
%\item
%\ldots
%\end{itemize}

%%%%%%%%%%%%%%%%%%%%%%%%%%%%%%%%%%%%%%%%%%%%%%%%%%%%%%%%%%%%%%%%%%%%%%%%%%%%%%%%
\subsection{Revision History}

%%%%%%%%%%%%%%%%%%%%%%%%%%%%%%%%%%%%%%%%
\paragraph{v2.0:} 2018/12/30

\begin{itemize}
\item
immediate forward processing
\item
added |\childdocby| mechanism
\item
manual restructured
\end{itemize}

%%%%%%%%%%%%%%%%%%%%%%%%%%%%%%%%%%%%%%%%
\paragraph{v1.6:} 2018/01/17

\begin{itemize}
\item
application for development of include files
\item
corrections to manual
\end{itemize}

%%%%%%%%%%%%%%%%%%%%%%%%%%%%%%%%%%%%%%%%
\paragraph{v1.5:} 2017/05/21

\begin{itemize}
\item
more complete structuring introduced
\item
|\childdocof| introduced
\item
|\childdoc| renamed to |\childdocmain|
\item
|\childredirect| renamed to |\childdocforward| and |\childdocforwardprefix|
and functionality expanded
\end{itemize}

%%%%%%%%%%%%%%%%%%%%%%%%%%%%%%%%%%%%%%%%
\paragraph{v1.0:} 2017/04/27

\begin{itemize}
\item
manual and install package
\item
first version published on CTAN
\end{itemize}

%%%%%%%%%%%%%%%%%%%%%%%%%%%%%%%%%%%%%%%%
\paragraph{v0.6:} 2017/04/26

\begin{itemize}
\item
redirection mechanism added
\end{itemize}

%%%%%%%%%%%%%%%%%%%%%%%%%%%%%%%%%%%%%%%%
\paragraph{v0.5:} 2017/04/26

\begin{itemize}
\item
functionality in definition file
\end{itemize}


%%%%%%%%%%%%%%%%%%%%%%%%%%%%%%%%%%%%%%%%%%%%%%%%%%%%%%%%%%%%%%%%%%%%%%%%%%%%%%%%
%%%%%%%%%%%%%%%%%%%%%%%%%%%%%%%%%%%%%%%%%%%%%%%%%%%%%%%%%%%%%%%%%%%%%%%%%%%%%%%%
%%%%%%%%%%%%%%%%%%%%%%%%%%%%%%%%%%%%%%%%%%%%%%%%%%%%%%%%%%%%%%%%%%%%%%%%%%%%%%%%
\appendix

\settowidth\MacroIndent{\rmfamily\scriptsize 000\ }

 \DocInput{childdoc.dtx}

\end{document}
%</driver>
% \fi
%
% %%%%%%%%%%%%%%%%%%%%%%%%%%%%%%%%%%%%%%%%%%%%%%%%%%%%%%%%%%%%%%%%%%%%%%%%%%%%%%
% %%%%%%%%%%%%%%%%%%%%%%%%%%%%%%%%%%%%%%%%%%%%%%%%%%%%%%%%%%%%%%%%%%%%%%%%%%%%%%
% \section{Sample}
%\iffalse
%<*samplemain>
%\fi
%
% The following presents a sample document
% with two chapters, two parts, a title page,
% a compile flag as well as three forwarding files to set the flag.
% It consists of eight |.tex| files:
% \begin{center}
% \begin{tabular}{ll}
% |cdocsamp.tex|&main file\\
% |cdocsch1.tex|&include file for chapter 1\\
% |cdocsch2.tex|&include file for chapter 2\\
% |cdocspt3.tex|&include file for part 3\\
% |cdocspt4.tex|&include file for part 4\\
% |cdocsdrf.tex|&forwarding file for main file in draft mode\\
% |cdocsfi1.tex|&forwarding file for final version of chapter 1\\
% |cdocsfi2.tex|&forwarding file for final version of chapter 2\\
% \end{tabular}
% \end{center}
% Each of the eight files can be compiled directly by the \LaTeX{} compiler.
%
% %%%%%%%%%%%%%%%%%%%%%%%%%%%%%%%%%%%%%%
% \paragraph{Main File.}
%
% The main file is called |cdocsamp.tex|.
%
% Load the \textsf{childdoc} definitions and
% declare the filename for the main document:
%    \begin{macrocode}
\input{childdoc.def}
\childdocmain{}
%    \end{macrocode}

% Optional override for |\version| flag:
%    \begin{macrocode}
%%\ifchilddoc\else\providecommand{\version}{draft}\fi
%    \end{macrocode}

% Define the default values for the |\version| flag
% (|final| for the main file and |draft| for childs):
%    \begin{macrocode}
\ifchilddoc
\providecommand{\version}{draft}
\else
\providecommand{\version}{final}
\fi
%    \end{macrocode}

% Load the standard document class:
%    \begin{macrocode}
\documentclass[12pt]{article}
%    \end{macrocode}

% Start the document body:
%    \begin{macrocode}
\begin{document}
%    \end{macrocode}

% Declare a title page.
% Print title, part of document being processed and version flag:
%    \begin{macrocode}
\addtocounter{page}{-1}
\begin{center}
{\LARGE\bfseries{}childdoc example\par}
\vspace{1cm}
\ifchilddoc
\ifchilddocmanual part\else chapter\fi:
`\childdocname' of `\childdocjob'\par
\else
main document: `\childdocjob'\par
\fi
version: \version\par
\end{center}
\newpage
%    \end{macrocode}

% Manually include selected file,
% otherwise process as usual:
%    \begin{macrocode}
\ifchilddocmanual
\section*{part `\childdocname'}
\input{\childdocname}
\else
%    \end{macrocode}

% Include the two chapters:
%    \begin{macrocode}
\include{cdocsch1}
\include{cdocsch2}
%    \end{macrocode}

% Include the two parts unless only chapters should be displayed:
%    \begin{macrocode}
\ifchilddoc\else
\section{part three}
\input{cdocspt3}
\section{part four}
\input{cdocspt4}
\fi
%    \end{macrocode}

% Process as usual until here:
%    \begin{macrocode}
\fi
%    \end{macrocode}

% End of document body:
%    \begin{macrocode}
\end{document}
%    \end{macrocode}
%\iffalse
%</samplemain>
%\fi
%
% %%%%%%%%%%%%%%%%%%%%%%%%%%%%%%%%%%%%%%
% \paragraph{Chapter Include Files.}
%
% The include files are called |cdocsch1.tex| and |cdocsch2.tex|.
%
%\iffalse
%<*samplechap1|samplechap2>
%\fi

% Optional override for |\version| flag:
%    \begin{macrocode}
%%\providecommand{\version}{final}
%    \end{macrocode}

% Include the main document:
%    \begin{macrocode}
\input{childdoc.def}
\childdocof{cdocsamp}
%    \end{macrocode}

%\iffalse
%</samplechap1|samplechap2>
%\fi
%
%\iffalse
%<*samplechap1>
%\fi
% Some text for chapter 1:
%    \begin{macrocode}
\section{one}
some text in chapter one
%    \end{macrocode}

%\iffalse
%</samplechap1>
%\fi
% Some text for chapter 2:
%\iffalse
%<*samplechap2>
%\fi
%    \begin{macrocode}
\section{two}
more text in chapter two
%    \end{macrocode}

%\iffalse
%</samplechap2>
%\fi
%
% %%%%%%%%%%%%%%%%%%%%%%%%%%%%%%%%%%%%%%
% \paragraph{Part Include Files.}
%
% The include files are called |cdocspt3.tex| and |cdocspt4.tex|.
%
%\iffalse
%<*samplepart3|samplepart4>
%\fi

% Optional override for |\version| flag:
%    \begin{macrocode}
%%\providecommand{\version}{final}
%    \end{macrocode}

% Include the main document:
%    \begin{macrocode}
\input{childdoc.def}
\childdocby{cdocsamp}
%    \end{macrocode}

%\iffalse
%</samplepart3|samplepart4>
%\fi
%
%\iffalse
%<*samplepart3>
%\fi
% Some text for part 3:
%    \begin{macrocode}
some text in part three
%    \end{macrocode}

%\iffalse
%</samplepart3>
%\fi
% Some text for part 4:
%\iffalse
%<*samplepart4>
%\fi
%    \begin{macrocode}
more text in part four
%    \end{macrocode}

%\iffalse
%</samplepart4>
%\fi
%
% %%%%%%%%%%%%%%%%%%%%%%%%%%%%%%%%%%%%%%
% \paragraph{Forwarding for a Complete Draft.}
%
% The following forwarding file |cdocsdrf.tex|
% compiles the main document in draft mode:
%\iffalse
%<*sampledraft>
%\fi
%    \begin{macrocode}
\def\version{draft}
\input{childdoc.def}
\childdocforward{cdocsamp}
%    \end{macrocode}

%\iffalse
%</sampledraft>
%\fi
%
% %%%%%%%%%%%%%%%%%%%%%%%%%%%%%%%%%%%%%%
% \paragraph{Forwarding for Final Version of the Chapters.}
%
% The following forwarding files |cdocsfn1.tex| and |cdocsfn2.tex|
% (with identical content)
% compile the final versions of the child documents
% |cdocsch1.tex| and |cdocsch2.tex|, respectively:
%\iffalse
%<*samplefinal>
%\fi
%    \begin{macrocode}
\def\version{final}
\input{childdoc.def}
\childdocforwardprefix[cdocsamp]{cdocsfn}{cdocsch}
%    \end{macrocode}

%\iffalse
%</samplefinal>
%\fi
%
% %%%%%%%%%%%%%%%%%%%%%%%%%%%%%%%%%%%%%%
% \paragraph{Command Line Processing.}
%
% The following three command lines generate the output files
% |cdocscld|, |cdocscl1| and |cdocscl2|
% which should be identical to
% |cdocsdrf|, |cdocsch1| and |cdocsfn2|, respectively:
% \begin{center}
% \begin{tabular}{l}
% |latex -jobname cdocscld \|\\
% |  "\def\version{draft}\input{childdoc.def}\childdocforward{cdocsamp}"|\\
% |latex -jobname cdocscl1 \|\\
% |  "\input{childdoc.def}\childdocforward[cdocsamp]{cdocsch1}"|\\
% |latex -jobname cdocscl2 \|\\
% |  "\def\version{final}\input{childdoc.def}\childdocforward{cdocsch2}"|
% \end{tabular}
% \end{center}
% Note that the trailing backslash on each first line
% merely continues the input to the second line
% (for convenient cut ant paste).
% Furthermore, the command |latex| can be replaced by any
% of its alternative versions such as |pdflatex|.
%
% %%%%%%%%%%%%%%%%%%%%%%%%%%%%%%%%%%%%%%%%%%%%%%%%%%%%%%%%%%%%%%%%%%%%%%%%%%%%%%
% %%%%%%%%%%%%%%%%%%%%%%%%%%%%%%%%%%%%%%%%%%%%%%%%%%%%%%%%%%%%%%%%%%%%%%%%%%%%%%
% \section{Implementation}
%\iffalse
%<*package>
%\fi
%
% This section describes the definitions file |childdoc.def|.

% The definitions cannot be loaded using |\usepackage| or |\RequirePackage|
% which has a mechanism to prevent loading a style file more than once.
% When loading the definitions by means of |\input|
% multiple instances have to be prevented manually:
%\iffalse
%This code needs to be before the `\ProvidesFile' directive
%which is defined at the beginning of this file.
%Therefore it is also placed there and commented out here.
%</package>
%<*discard>
%\fi
%    \begin{macrocode}
\ifdefined\childdocmain\endinput\fi
%    \end{macrocode}
%\iffalse
%</discard>
%<*package>
%\fi
%
% \macro{\ifchilddoc}
% \macro{\ifchilddocmanual}
% The conditional |\ifchilddoc| tells whether a
% child (true) or main (false) document is being compiled.
% The conditional |\ifchilddocmanual| tells whether
% the |\includeonly| mechanism is used (false) or
% the selection of child files must be performed manually (true).
% The definitions initialise to false:
%    \begin{macrocode}
\newif\ifchilddoc
\newif\ifchilddocmanual
%    \end{macrocode}

% \macro{\childdocname}
% \macro{\childdocjob}
% The macro |\childdocname| stores the name of the main document
% to be compiled. The macro |\childdocjob| stores the name of
% the document on which the \LaTeX{} compiler was originally invoked.
% The content of |\jobname| cannot be compared
% to filenames specified in the source due to different catcodes.
% The following code rescans |\jobname|, stores the result
% in |\childdocname| and saves a copy in |\childdocjob|:
%    \begin{macrocode}
\edef\childdocname{\scantokens\expandafter{\jobname\noexpand}}
\let\childdocjob\childdocname
%    \end{macrocode}

% \macro{\childdocdisable}
% The macro |\childdocdisable| prevents the main file
% from being processed more than once.
% At this stage, the main document command |\childdocmain|
% is assumed to be called once again where it should do nothing.
% Any subsequent call to it should prevent
% a secondary processing of the main document
% It overwrites the forwarding commands
% |\childdocof| and |\childdocforward|
% with empty macros to prevent further inclusions of the main document:
%    \begin{macrocode}
\newcommand{\childdocdisable}
{
  \renewcommand{\childdocmain}[1]{\renewcommand{\childdocmain}[1]{\endinput}}
  \renewcommand{\childdocof}[1]{}
  \renewcommand{\childdocby}[2][]{}
  \renewcommand{\childdocforward}[2][]{}
  \renewcommand{\childdocdisable}{}
}
%    \end{macrocode}

% \macro{\childdocmain}
% The macro |\childdocmain| is to be called at the top of the main file
% with nothing or the main filename (without extension) as argument.
% First, it breaks loops.
% If the argument is not empty and does not match |\childdocname|
% (which is set by the first inclusion of |childdoc.def|),
% |\ifchilddoc| is set to true, |\includeonly| is applied to the child file
% and |\jobname| is set to the main file
% (for proper handling of |.aux| files):
%    \begin{macrocode}
\newcommand{\childdocmain}[1]
{
  \childdocdisable\childdocmain{}
  \if?#1?\else
    \begingroup
      \def\childdoctmp{#1}
      \ifx\childdoctmp\childdocname
        \def\childdoctmp{}
      \else
        \def\childdoctmp
        {
          \childdoctrue
          \includeonly{\childdocname}
          \def\childdocjob{#1}
          \def\jobname{#1}
        }
      \fi
      \expandafter
    \endgroup
    \childdoctmp
  \fi
}
%    \end{macrocode}

% \macro{\childdocof}
% The command |\childdocof| redirects
% compilation to the main file |#1|.
%    \begin{macrocode}
\newcommand{\childdocof}[1]
{
  \childdocdisable
  \childdoctrue
  \includeonly{\childdocname}
  \def\jobname{#1}
  \def\childdocjob{#1}
  \input{#1}
}
%    \end{macrocode}

% \macro{\childdocby}
% The command |\childdocby| ....
%    \begin{macrocode}
\newcommand{\childdocby}[2][]
{
  \childdocdisable
  \childdoctrue
  \childdocmanualtrue
  \if?#1?\else
    \def\jobname{#2}
  \fi
  \def\childdocjob{#2}
  \input{#2}
  \endinput
}
%    \end{macrocode}

% \macro{\childdocforward}
% The command |\childdocforward| redirects
% compilation to the main file or
% (if the optional argument is given) a child file.
% Parameters are set as if the main file
% or a child file starting with |\childdocof| was compiled.
% Then compilation is handed over to the main file:
%    \begin{macrocode}
\newcommand{\childdocforward}[2][]
{
  \begingroup
    \if?#1?
      \def\childdoctmp
      {
        \def\childdocname{#2}
        \def\childdocjob{#2}
        \def\jobname{#2}
        \input{#2}
        \endinput
      }
    \else
      \def\childdoctmp
      {
        \childdocdisable
        \def\childdocname{#2}
        \childdoctrue
        \includeonly{#2}
        \def\childdocjob{#1}
        \def\jobname{#1}
        \input{#1}
        \endinput
      }
    \fi
    \expandafter
  \endgroup
  \childdoctmp
}
%    \end{macrocode}

% \macro{\childdocforwardprefix}
% The command |\childdocforwardprefix| redirects
% compilation to the main or a child file by means of a pattern.
% The prefix |#1| in the current filename is replaced by |#2|
% and the suffix of the current filename is kept
% (it is assumed that the filename does not contain the substring `|~~~|'
% which is used as a delimiter).
% Compilation is handed over to the new file by |\childdocforward|:
%    \begin{macrocode}
\newcommand{\childdocforwardprefix}[3][]
{
  \begingroup
    \def\childdocextract #2##1~~~{\def\childdoctmp{\childdocforward[#1]{#3##1}}}
    \expandafter\childdocextract\childdocname~~~
    \expandafter
  \endgroup
  \childdoctmp
}
%    \end{macrocode}

% \macro{\childdoc}
% The deprecated macro |\childdoc| is a legacy version of |\childdocmain|:
%    \begin{macrocode}
\newcommand{\childdoc}{\childdocmain}
%    \end{macrocode}

% \macro{\childdocredirect}
% The deprecated macro |\childdocredirect| is a legacy version
% of |\childdocforward| and |\childdocforwardprefix|:
%    \begin{macrocode}
\newcommand{\childdocredirect}[2][]
{
  \begingroup
    \if?#1?
      \def\childdoctmp{\childdocforward{#2}}
    \else
      \def\childdoctmp{\childdocforwardprefix{#1}{#2}}
    \fi
    \expandafter
  \endgroup
  \childdoctmp
}
%    \end{macrocode}

%\iffalse
%</package>
%\fi
%
\endinput
|\\
|\childdocforward{|\textit{main}|}|
\end{tabular}
\end{center}
%
Likewise, the following files |final|\textit{nn}|.tex|
compile the final version of the child document
|child|\textit{nn}|.tex|:
%
\begin{center}
\begin{tabular}{l}
|\def\version{final}|\\
|% \iffalse
%
% childdoc.dtx Copyright (C) 2017-2018 Niklas Beisert
%
% This work may be distributed and/or modified under the
% conditions of the LaTeX Project Public License, either version 1.3
% of this license or (at your option) any later version.
% The latest version of this license is in
%   http://www.latex-project.org/lppl.txt
% and version 1.3 or later is part of all distributions of LaTeX
% version 2005/12/01 or later.
%
% This work has the LPPL maintenance status `maintained'.
%
% The Current Maintainer of this work is Niklas Beisert.
%
% This work consists of the files childdoc.dtx and childdoc.ins
% and the derived files childdoc.def and cdocsamp.tex with
% cdocsch1.tex, cdocsch2.tex, cdocsdrf.tex, cdocsfn1.tex, cdocsfn2.tex.
%
%<package>\ifdefined\childdocmain\endinput\fi
%<package>\ProvidesFile{childdoc.def}[2018/12/30 v2.0 child document driver]
%<samplemain>\ProvidesFile{cdocsamp.tex}[2018/12/30 v2.0 sample for childdoc]
%<*driver>
%\ProvidesFile{childdoc.drv}[2018/12/30 v2.0 childdoc reference manual file]
\PassOptionsToClass{10pt,a4paper}{article}
\documentclass{ltxdoc}

\usepackage[margin=35mm]{geometry}
\usepackage{hyperref}
\usepackage{hyperxmp}
\usepackage[usenames]{color}

\hypersetup{colorlinks=true}
\hypersetup{pdfstartview=FitH}
\hypersetup{pdfpagemode=UseNone}
\hypersetup{pdfsource={}}
\hypersetup{pdflang={en-UK}}
\hypersetup{pdfcopyright={Copyright 2017-2018 Niklas Beisert.
  This work may be distributed and/or modified under the
  conditions of the LaTeX Project Public License, either version 1.3
  of this license or (at your option) any later version.}}
\hypersetup{pdflicenseurl={http://www.latex-project.org/lppl.txt}}
\hypersetup{pdfcontactaddress={ETH Zurich, ITP, HIT K,
  Wolfgang-Pauli-Strasse 27}}
\hypersetup{pdfcontactpostcode={8093}}
\hypersetup{pdfcontactcity={Zurich}}
\hypersetup{pdfcontactcountry={Switzerland}}
\hypersetup{pdfcontactemail={nbeisert@itp.phys.ethz.ch}}
\hypersetup{pdfcontacturl={http://people.phys.ethz.ch/\xmptilde nbeisert/}}

\newcommand{\secref}[1]{\hyperref[#1]{section \ref*{#1}}}

\parskip1ex
\parindent0pt
\let\olditemize\itemize
\def\itemize{\olditemize\parskip0pt}

\begin{document}

\title{The \textsf{childdoc} Package}
\hypersetup{pdftitle={The childdoc Package}}
\author{Niklas Beisert\\[2ex]
  Institut f\"ur Theoretische Physik\\
  Eidgen\"ossische Technische Hochschule Z\"urich\\
  Wolfgang-Pauli-Strasse 27, 8093 Z\"urich, Switzerland\\[1ex]
  \href{mailto:nbeisert@itp.phys.ethz.ch}
  {\texttt{nbeisert@itp.phys.ethz.ch}}}
\hypersetup{pdfauthor={Niklas Beisert}}
\hypersetup{pdfsubject={Manual for the LaTeX2e Package childdoc}}
\date{30 December 2018, \textsf{v2.0}}
\maketitle

\begin{abstract}\noindent
\textsf{childdoc} is a \LaTeXe{} package
that enables the direct compilation
of document sections included by |\include|
to individual files.
\end{abstract}

\begingroup
\parskip0ex
\tableofcontents
\endgroup

%%%%%%%%%%%%%%%%%%%%%%%%%%%%%%%%%%%%%%%%%%%%%%%%%%%%%%%%%%%%%%%%%%%%%%%%%%%%%%%%
%%%%%%%%%%%%%%%%%%%%%%%%%%%%%%%%%%%%%%%%%%%%%%%%%%%%%%%%%%%%%%%%%%%%%%%%%%%%%%%%
\section{Introduction}

\LaTeX{} provides a mechanism to structure a large document (such as a book)
into a main file and several child files (containing the chapters)
using the |\include| command.
This mechanism is beneficial for documents
which span hundreds of pages in order to
make the source file(s) more manageable.
Moreover, compilation can be restricted to
selected child files by means of the |\includeonly| command.
The latter feature can be used to reduce the compilation time while editing
(this was significantly more useful in the earlier days of \LaTeX{})
or to generate a smaller document which is easier to navigate.
Another application of |\includeonly| is to generate
documents consisting of selected parts of the complete document.

However, there are a few drawbacks of the plain |\include| mechanism:
\begin{itemize}
\item
The child files cannot be compiled on their own,
they can only be compiled via the main file.
A naive editing environment
(such as a text editor with an option
to have the current file processed by \LaTeX)
may require one to switch to the main file before compiling;
attempting to compile the child file produces errors.
\item
The main file must be modified (each time)
to adjust the |\includeonly| command
to the present needs. This easily leaves the main file in a messy state.
\item
The generated document will always carry the filename
of the main document. This is inconvenient if
several child files are to be compiled and
to be kept for distribution.
\end{itemize}

The present package provides a simple interface
to make child files individually compilable by \LaTeX{}.
Compiling a child file then has the same effect as compiling
the main file with an |\includeonly| command
to select the appropriate child.
Moreover the generated document will carry the name of the child
rather than the main file.
This resolves all three above issues.

This feature is meant to make the editing of books,
thesis documents and lecture notes somewhat more convenient.
However, the package can also be used efficiently for
composing a series of documents (such as exercise sheets)
which are typically distributed individually.
It then assists the author in generating the individual documents
(potentially in different versions)
as well as a document containing the collected series.
Another application is in developing style files
or other kinds of included material
where compilation of the style file could redirect
to a sample or test file.

%%%%%%%%%%%%%%%%%%%%%%%%%%%%%%%%%%%%%%%%%%%%%%%%%%%%%%%%%%%%%%%%%%%%%%%%%%%%%%%%
%%%%%%%%%%%%%%%%%%%%%%%%%%%%%%%%%%%%%%%%%%%%%%%%%%%%%%%%%%%%%%%%%%%%%%%%%%%%%%%%
\section{Usage}

First of all, the package \textsf{childdoc} is \emph{not} a standard
\LaTeXe{} |.sty| style file! Therefore it needs to be invoked in
a non-standard way.

%%%%%%%%%%%%%%%%%%%%%%%%%%%%%%%%%%%%%%%%%%%%%%%%%%%%%%%%%%%%%%%%%%%%%%%%%%%%%%%%
\subsection{Included Files}
\label{sec:include}

%%%%%%%%%%%%%%%%%%%%%%%%%%%%%%%%%%%%%%%%
\DescribeMacro{\childdocmain}
To use the package, add the commands
\begin{center}
\begin{tabular}{l}
|\input{childdoc.def}|\\
|\childdocmain{}|\\
\end{tabular}
\end{center}
at the very top of the main \LaTeX{} file,
in particular \emph{before} the |\documentclass| statement!
The argument of |\childdocmain| should be left empty
(but it must be present).

%%%%%%%%%%%%%%%%%%%%%%%%%%%%%%%%%%%%%%%%
\DescribeMacro{\childdocof}
Furthermore, add the commands
\begin{center}
\begin{tabular}{l}
|\input{childdoc.def}|\\
|\childdocof{|\textit{main}|}|\\
\end{tabular}
\end{center}
at the top of every child file \textit{child}
which is included by |\include{|\textit{child}|}|
from within the main file
(or at least for those files to be compiled individually).
The argument \textit{main} must be the filename of the main file.

There are a couple of
considerations in setting up the main and child documents:

%%%%%%%%%%%%%%%%%%%%%%%%%%%%%%%%%%%%%%%%
\paragraph{Restrictions.}

Please note the following restrictions:
\begin{itemize}
\item
|\childdocmain| must be called with one argument \textit{main}
to ensure compatibility with earlier version of the package.
It must either be empty (|\childdocmain{}|)
or precisely match the filename of the main file in which it is specified.
See \secref{sec:detection} for further information.
\item
The filename \textit{main} must be specified without the |.tex| extension.
\item
The filename \textit{main} is case sensitive
(even in case-insensitive file systems)
due to internal string comparison.
\item
The argument \textit{main} should be fully expanded, it cannot be a macro.
\item
Subdirectories and special characters should be avoided in filenames.
\item
The command |\childdocmain{|\textit{main}|}| must be followed by a whitespace.
It should not be followed immediately by another command
or by a comment mark `|%|'.
This is because the \TeX{} parser reads the token immediately following
the argument of |\childdocmain| and puts it
at the beginning of every child section;
however, a white\-space is ignored.
\end{itemize}

%%%%%%%%%%%%%%%%%%%%%%%%%%%%%%%%%%%%%%%%
\paragraph{Content of Main File.}

It is advisable to place all content in the child files included by |\include|.
Any output contained in the main file will appear in all child documents
unless suppressed manually;
it cannot be suppressed automatically by the |\includeonly| directive
and thus should normally be avoided.
A method to include some content in the main file
by means of conditional processing is described in \secref{sec:conditional}.

%%%%%%%%%%%%%%%%%%%%%%%%%%%%%%%%%%%%%%%%
\paragraph{Page Numbering.}

When only a part of the document is compiled,
the appropriate numbering of pages
(as well as other status parameters)
is determined from the |.aux| files.
The latter contain information from previous passes.
However this information needs to propagate through
all intermediate child documents.
Therefore the page numbering in child documents may well
be inconsistent until the complete document is compiled at least once.

A useful (if unconventional) way to always ensure a consistent
page numbering is to restart the numbering in each child document
and denote the pages by `\textit{child}|.|\textit{page}'
where \textit{child} represents the chapter/section number of the child file.
This can be achieved by the command
|\numberwithin{page}{|\textit{child}|}|
of the \textsf{amsmath} package
where \textit{child} can be |chapter| or |section|
depending on the chosen structuring.
Alternatively, one can modify the macro |\thepage| appropriately
and reset the counter |page| at the start of each child file.

%%%%%%%%%%%%%%%%%%%%%%%%%%%%%%%%%%%%%%%%%%%%%%%%%%%%%%%%%%%%%%%%%%%%%%%%%%%%%%%%
\subsection{Conditional Processing}
\label{sec:conditional}

The package provides a mechanism to compile different versions
of a document. To customise the versions further some conditional processing
can come in handy to distinguish which version is being compiled.
The package provides two macros to describe the compilation context:

%%%%%%%%%%%%%%%%%%%%%%%%%%%%%%%%%%%%%%%%
\DescribeMacro{\ifchilddoc}
The conditional |\ifchilddoc| distinguishes between the compilation of
child documents and the main document:
%
\begin{center}
|\ifchilddoc |\textit{child-code}| |[|\||else |\textit{main-code}]| \||fi|
\end{center}

%%%%%%%%%%%%%%%%%%%%%%%%%%%%%%%%%%%%%%%%
\DescribeMacro{\childdocname}
\DescribeMacro{\childdocjob}
The macro |\childdocname| contains the filename (without extension)
of the main or child file being processed.
Note that |\childdocjob| will always contain the name of the main file.

%%%%%%%%%%%%%%%%%%%%%%%%%%%%%%%%%%%%%%%%
\paragraph{Title Page.}

Conditional processing can be used to include a title or banner page
in the main document when proper precautions are taken.
Importantly, the code in the main file should ensure that the page counter
(as well as other status parameters which are stored in the |.aux| files)
takes the same value after the conditional processing.
Otherwise the page numbers may take divergent values
depending on which part is compiled.

For example, a title page could be declared by:
%
\begin{center}
\begin{tabular}{l}
|\ifchilddoc\||else|\\
|\addtocounter{page}{-1}|\\
\textit{code for title page}\\
|\newpage|\\
|\||fi|
\end{tabular}
\end{center}
%
A banner page for the child documents can be generated by:
%
\begin{center}
\begin{tabular}{l}
|\ifchilddoc|\\
|\addtocounter{page}{-1}|\\
\textit{code for banner page}\\
|\newpage|\\
|\||fi|
\end{tabular}
\end{center}
%
Here one could write a message such as:
\begin{center}
|This is the part \childdocname{} of \childdocjob{}.|
\end{center}

%%%%%%%%%%%%%%%%%%%%%%%%%%%%%%%%%%%%%%%%%%%%%%%%%%%%%%%%%%%%%%%%%%%%%%%%%%%%%%%%
\subsection{Flags}
\label{sec:flags}

The package makes it easy to generate different versions
of the main or child documents.
To this end compilation flags can be defined
and assigned different default values.
They will be particularly useful in conjunction
with the forwarding mechanism described in \secref{sec:forward}.

For example, it may be useful to have a flag |\version|
which can be set to |draft| or |final|.
The document source will contain some conditional code
depending on the value of |\version|.
Suppose further, the flag should default to |final| for the main file
and to |draft| for child files
which is a natural assignment for editing the document.
This is achieved by placing the following code
in the preamble of the main document
(below the |\childdocmain| directive):
%
\begin{center}
\begin{tabular}{l}
|\ifchilddoc|\\
|\providecommand{\version}{draft}|\\
|\||else|\\
|\providecommand{\version}{final}|\\
|\||fi|
\end{tabular}
\end{center}
%
The definition by |\providecommand| makes sure
that previous definitions are not overwritten.
Further statements |\providecommand{\version}{...}|
can thus be added before the above code to override it.

For the main file, one might add a line
(between |\childdocmain| and the above block)
%
\begin{center}
|%\ifchilddoc\||else\providecommand{\version}{draft}\||fi|
\end{center}
%
which can be uncommented to produce a draft version.
Likewise one can add a line to the very top of a child file
(above the |\childdocof{|\textit{main}|}| directive)
%
\begin{center}
|%\providecommand{\version}{final}|
\end{center}
%
which can be uncommented to produce the final version of this child document.

%%%%%%%%%%%%%%%%%%%%%%%%%%%%%%%%%%%%%%%%%%%%%%%%%%%%%%%%%%%%%%%%%%%%%%%%%%%%%%%%
\subsection{Forwarding}
\label{sec:forward}

Different versions of the main or child documents
using compilation flags as described in \secref{sec:flags}
can be (permanently) stored in different files
for convenient compilation, viewing and distribution.
To this end, the package defines a command
to pass on compilation to a different file:

%%%%%%%%%%%%%%%%%%%%%%%%%%%%%%%%%%%%%%%%
\DescribeMacro{\childdocforward}
The command |\childdocforward| redirects processing to
another source file:
%
\begin{center}
\begin{tabular}{l}
|\input{childdoc.def}|\\
|\childdocforward[|\textit{main}|]{|\textit{dest}|}|\\
\end{tabular}
\end{center}
%
The argument \textit{dest} is the destination file
(without extension).
It should be the main file or one of the child files.
Note that further \textsf{childdoc} directives
such as |\childdocof| and |\childdocforward|
in the indicated file will be processed in this form.
The optional argument \textit{main}
passes on directly to the main file \textit{main}
while pretending to compile the child \textit{dest}.
This form behaves as if \textit{dest}
issues |\childdocof{|\textit{main}|}| right away,
and no further \textsf{childdoc} directives will be processed.

%%%%%%%%%%%%%%%%%%%%%%%%%%%%%%%%%%%%%%%%
\DescribeMacro{\...prefix}
In the alternative form |\childdocforwardprefix|,
%
\begin{center}
\begin{tabular}{l}
|\input{childdoc.def}|\\
|\childdocforwardprefix[|\textit{main}|]{|\textit{prefix}|}{|\textit{dest}|}|
\end{tabular}
\end{center}
%
the destination file is determined by a pattern
depending on the current file:
To make this work, the current file must be called
`{\textit{prefix}\hspace{0.2em}\textit{suffix}}'
with \textit{prefix} matching precisely the argument.
Processing is then passed on to the file
`{\textit{dest}\hspace{0.2em}\textit{suffix}}'.
Surely, the same effect is achieved by
directly specifying the
argument `{\textit{dest}\hspace{0.2em}\textit{suffix}}'
in the first form.
However, that requires to set up a different file
for each child. With the alternative form of the command
all these files can have exactly the same content
which simplifies setting them up and maintaining them.

For example, the following file |draft.tex|
with a compilation flag |\version| as described in \secref{sec:flags}
compiles the main document as a draft:
%
\begin{center}
\begin{tabular}{l}
|\def\version{draft}|\\
|\input{childdoc.def}|\\
|\childdocforward{|\textit{main}|}|
\end{tabular}
\end{center}
%
Likewise, the following files |final|\textit{nn}|.tex|
compile the final version of the child document
|child|\textit{nn}|.tex|:
%
\begin{center}
\begin{tabular}{l}
|\def\version{final}|\\
|\input{childdoc.def}|\\
|\childdocforwardprefix{final}{child}|
\end{tabular}
\end{center}
%

Note that when several versions of a main file and/or of each child file
are to be generated, it may be convenient to set up a |Makefile| or
shell script to automatise the process.

%%%%%%%%%%%%%%%%%%%%%%%%%%%%%%%%%%%%%%%%%%%%%%%%%%%%%%%%%%%%%%%%%%%%%%%%%%%%%%%%
\subsection{Command Line Processing}
\label{sec:commandline}

The effect of redirection files can also be achieved by invoking
the \LaTeX{} compiler with a more elaborate command line.
Most conveniently this should be done as part
of a shell script or a |Makefile|.

When using \textsf{childdoc} in the main file, the following
command lines effectively perform a redirection
(note that depending on the shell being used,
backslashes may have to be doubled: `|\|' $\to$ `|\\|'):
%
\begin{center}
|... -jobname "|\textit{target}|" |\\|"|[\textit{flags}]%
|\input{childdoc.def}\childdocforward[|\textit{main}|]{|\textit{dest}|}"|
\end{center}
%
Here \textit{target} is the name of the output file,
\textit{main} is the name of the main file
and \textit{dest} is the name of the main or child file to be processed
(all filenames without extensions).
The optional argument \textit{main} can be omitted
if \textit{main} matches \textit{dest}.
Optionally, compilation \textit{flags} can be defined via |\def| commands.
This command line makes the \TeX{} engine believe
it is compiling the file \textit{target}
whose content is specified as the latter parameter.
The provided code then forwards the processing to
\textit{main} or \textit{dest} as described in \secref{sec:forward}.

%%%%%%%%%%%%%%%%%%%%%%%%%%%%%%%%%%%%%%%%%%%%%%%%%%%%%%%%%%%%%%%%%%%%%%%%%%%%%%%%
\subsection{Include by Input}
\label{sec:input}

Including child documents by |\include| has some restrictions by design.
Most notably, the content of a child document always occupies
its own set of pages; pages cannot be shared between child documents.
Usually, this behaviour makes perfect sense
because each child document contain an essential part of the document.
However, in some situations it may be desirable to compose
a document from a collection of parts
without having mandatory page breaks between then.
For this case, the package
provides a mechanism to include parts
by |\input| which can also be processed individually.
However, by construction this mechanism
requires manual handling of the content to be output.

%%%%%%%%%%%%%%%%%%%%%%%%%%%%%%%%%%%%%%%%
\DescribeMacro{\ifchilddocmanual}
The main file should be prepared as usual, see \secref{sec:include}.
However, the document body must make a distinction
between processing of an individual part and of the main document, e.g.:
%
\begin{center}
\begin{tabular}{l}
|\ifchilddocmanual|\\
|\input{\childdocname}|\\
|\||else|\\
\textit{document body with }|\input{|\textit{part}|}|\\
|\||fi|
\end{tabular}
\end{center}
%
The conditional |\ifchilddocmanual| is true whenever
a part to be included by |\input| is being compiled,
and the name of the part is stored in |\childdocname|.

%%%%%%%%%%%%%%%%%%%%%%%%%%%%%%%%%%%%%%%%
\DescribeMacro{\childdocby}
Each part to be included by |\input| should start with:
%
\begin{center}
\begin{tabular}{l}
|\input{childdoc.def}|\\
|\childdocby{|\textit{main}|}|\\
\end{tabular}
\end{center}
%
The directive |\childdocby| is similar to |\childdocof|
described in \secref{sec:include},
but the subsequent selection of content must be done manually.
To that end, both |\ifchilddoc| and |\ifchilddocmanual|
will be true upon processing of a part,
and the name of the part is stored in |\childdocname|.
Note that |\jobname| will be set to the filename of the current part
so that each part receives an individual |.aux| file
that does not interfere with the |.aux| file(s) of the main document.
This behaviour can be altered by the alternative form
|\childdocby[*]{|\textit{main}|}| (with a non-empty optional argument)
which uses the |.aux| file of the main document
by setting |\jobname| to \textit{main}.

%%%%%%%%%%%%%%%%%%%%%%%%%%%%%%%%%%%%%%%%%%%%%%%%%%%%%%%%%%%%%%%%%%%%%%%%%%%%%%%%
\subsection{Driver Development}
\label{sec:driver}

The \textsf{childdoc} mechanism can also be use for the development
of definition files such as \LaTeX{} styles or classes.
This case differs from the above setup with multiple parts
included by |\include| in that no |\includeonly| should be invoked.
This can be achieved by starting the include file
(before |\ProvidesPackage|) with:
%
\begin{center}
\begin{tabular}{l}
|\input{childdoc.def}|\\
|\childdocforward{|\textit{main}|}|\\
\end{tabular}
\end{center}
%
or alternatively with:
%
\begin{center}
\begin{tabular}{l}
|\input{childdoc.def}|\\
|\childdocby{|\textit{main}|}|\\
\end{tabular}
\end{center}
%
Both forms have slightly different effects as described above.
The main file is prepared as usual, see \secref{sec:include}.

%%%%%%%%%%%%%%%%%%%%%%%%%%%%%%%%%%%%%%%%%%%%%%%%%%%%%%%%%%%%%%%%%%%%%%%%%%%%%%%%
\subsection{Legacy Detection}
\label{sec:detection}

The directive |\childdocmain| in the main file can detect
whether the complete document or merely a child is to be compiled
even without using the directive |\childdocof|.
This method is deprecated because it is less robust
and there is no compelling reason to use it;
it is merely provided for backward compatibility
and it may be removed in future versions.

If the detection mechanism is to be used,
it is mandatory to correctly specify
the filename of the main file as the argument of |\childdocmain|:
%
\begin{center}
\begin{tabular}{l}
|\input{childdoc.def}|\\
|\childdocmain{|\textit{main}|}|\\
\end{tabular}
\end{center}
%
If |\jobname| does not match the argument \textit{main} of |\childdocmain|,
it is assumed that |\jobname| points to the child file to be compiled.
When using |\childdocmain| with the main file specified as argument,
it suffices to start a child file
with just |\input{|\textit{main}|}|
without loading of the package and using |\childdocof|.
If instead all processing is done
with the appropriate \textsf{childdoc} directives,
the argument of \textit{main} of |\childdocmain| can be empty.

An alternative version of the command line processing described
in \secref{sec:commandline} using the detection mechanism reads:
%
\begin{center}
|... -jobname "|\textit{target}|" "|[\textit{flags}]%
[|\def\jobname{|\textit{dest}|}|]|\input{|\textit{main}|}"|
\end{center}

%%%%%%%%%%%%%%%%%%%%%%%%%%%%%%%%%%%%%%%%%%%%%%%%%%%%%%%%%%%%%%%%%%%%%%%%%%%%%%%%
\subsection{Manual Code}
\label{sec:manual}

In case one cannot be certain whether the definitions file |childdoc.def|
is installed on the target \TeX{} distribution
and one prefers not to ship it,
it is conceivable to paste a few relevant commands into the sources.

To that end, drop all statements |\input{childdoc.def}|
and perform the replacements as outlined below.
Instead of |\childdocmain{|\textit{main}|}| add the following code
to the top of the main file:
%
\begin{center}
\begin{tabular}{l}
|\||ifdefined\childdocname\endinput\||fi\newif\ifchilddoc|\\
|\edef\childdocname{\scantokens\expandafter{\jobname\noexpand}}|\\
|\def\childdocmain{|\textit{main}|}\||ifx\childdocmain\childdocname\||else|\\
|\childdoctrue\includeonly{\childdocname}\let\jobname\childdocmain\||fi|\\
\end{tabular}
\end{center}
%
Instead of |\childdocof{|\textit{main}|}| just include the main file
at the top of each child file:
%
\begin{center}
|\input{|\textit{main}|}|
\end{center}
%
A simple redirection |\childdocforward{|\textit{dest}|}| is achieved by:
%
\begin{center}
|\def\jobname{|\textit{dest}|}\input{\jobname}|
\end{center}
%
The redirection with prefix
|\childdocforwardprefix[|\textit{prefix}|]{|\textit{dest}|}|
is accomplished by:
%
\begin{center}
\begin{tabular}{l}
|{\edef\jobname{\scantokens\expandafter{\jobname\noexpand}}|\\
|\def\redirectjob |\textit{prefix}|#1~~~{\gdef\jobname{|\textit{dest}|#1}}|\\
|\expandafter\redirectjob\jobname~~~}\input{\jobname}|
\end{tabular}
\end{center}

In an alternative approach,
child documents can be compiled by a specific command line
without additional code or specific definitions:
%
\begin{center}
|... -jobname "|\textit{target}|" "|[\textit{flags}]%
|\includeonly{|\textit{dest}|}\input{|\textit{main}|}"|
\end{center}
%

%%%%%%%%%%%%%%%%%%%%%%%%%%%%%%%%%%%%%%%%%%%%%%%%%%%%%%%%%%%%%%%%%%%%%%%%%%%%%%%%
%%%%%%%%%%%%%%%%%%%%%%%%%%%%%%%%%%%%%%%%%%%%%%%%%%%%%%%%%%%%%%%%%%%%%%%%%%%%%%%%
\section{Information}

%%%%%%%%%%%%%%%%%%%%%%%%%%%%%%%%%%%%%%%%%%%%%%%%%%%%%%%%%%%%%%%%%%%%%%%%%%%%%%%%
\subsection{Copyright}

Copyright \copyright{} 2017--2018 Niklas Beisert

This work may be distributed and/or modified under the
conditions of the \LaTeX{} Project Public License, either version 1.3
of this license or (at your option) any later version.
The latest version of this license is in
  \url{http://www.latex-project.org/lppl.txt}
and version 1.3 or later is part of all distributions of \LaTeX{}
version 2005/12/01 or later.

This work has the LPPL maintenance status `maintained'.

The Current Maintainer of this work is Niklas Beisert.

This work consists of the files |README.txt|, |childdoc.ins| and |childdoc.dtx|
as well as the derived files |childdoc.def|, |cdocsamp.tex|
with |cdocsch1.tex|, |cdocsch2.tex|, |cdocspt3.tex|, |cdocspt4.tex|,
|cdocsdrf.tex|, |cdocsfn1.tex|, |cdocsfn2.tex|
as well as |childdoc.pdf|.

%%%%%%%%%%%%%%%%%%%%%%%%%%%%%%%%%%%%%%%%%%%%%%%%%%%%%%%%%%%%%%%%%%%%%%%%%%%%%%%%
\subsection{Files and Installation}

The package consists of the files:
%
\begin{center}
\begin{tabular}{ll}
    |README.txt|   & readme file \\
    |childdoc.ins| & installation file \\
    |childdoc.dtx| & source file \\
    |childdoc.def| & definition file \\
    |cdocsamp.tex| & sample main file \\
    |cdocsch1.tex| & sample include file \\
    |cdocsch2.tex| & sample include file \\
    |cdocspt3.tex| & sample part file \\
    |cdocspt4.tex| & sample part file \\
    |cdocsdrf.tex| & sample redirection file \\
    |cdocsfn1.tex| & sample redirection file \\
    |cdocsfn2.tex| & sample redirection file \\
    |childdoc.pdf| & manual
\end{tabular}
\end{center}
%
The distribution consists of the files
|README.txt|, |childdoc.ins| and |childdoc.dtx|.
%
\begin{itemize}
\item
Run (pdf)\LaTeX{} on |childdoc.dtx|
to compile the manual |childdoc.pdf| (this file).
\item
Run \LaTeX{} on |childdoc.ins| to create the definitions file |childdoc.def|
and the sample |cdocsamp.tex| with include files
|cdocsch1.tex|, |cdocsch2.tex|, |cdocspt3.tex|, |cdocspt4.tex|,
|cdocsdrf.tex|, |cdocsfn1.tex|, |cdocsfn2.tex|.
Then copy the file |childdoc.def| to an appropriate directory of your \LaTeX{}
distribution, e.g.\ \textit{texmf-root}|/tex/latex/childdoc|.
\end{itemize}

%%%%%%%%%%%%%%%%%%%%%%%%%%%%%%%%%%%%%%%%%%%%%%%%%%%%%%%%%%%%%%%%%%%%%%%%%%%%%%%%
\subsection{Related CTAN Packages}

There are several other packages which offer a similar functionality:
%
\begin{itemize}
\item
The packages
\href{http://ctan.org/pkg/docmute}{\textsf{docmute}},
\href{http://ctan.org/pkg/includex}{\textsf{includex}} and
\href{http://ctan.org/pkg/standalone}{\textsf{standalone}}
provide commands to include only the document body of
a child file thus allowing both files to be compiled individually.
\item
The packages \href{http://ctan.org/pkg/subdocs}{\textsf{subdocs}}
and \href{http://ctan.org/pkg/subfiles}{\textsf{subfiles}}
provide structures in which the main and child documents can be
encapsulated and allowing them to be compiled individually.
The inclusion mechanism is different from the conventional |\include|.
\item
The package \href{http://ctan.org/pkg/combine}{\textsf{combine}}
is an elaborate solution to combine several documents into one.
\end{itemize}
%
See also the CTAN topic \href{http://ctan.org/topic/subdocs}{\textsf{subdocs}}
for further related packages.
The present package differs from the above solutions in that
a document structure constructed with the conventional |\include| mechanism
just needs two extra commands at the top of every file
such that all constituent files can be compiled individually.

%%%%%%%%%%%%%%%%%%%%%%%%%%%%%%%%%%%%%%%%%%%%%%%%%%%%%%%%%%%%%%%%%%%%%%%%%%%%%%%%
%\subsection{Feature Suggestions}
%
%The following is a list of features which may be useful for future
%versions of this package:
%%
%\begin{itemize}
%\item
%\ldots
%\end{itemize}

%%%%%%%%%%%%%%%%%%%%%%%%%%%%%%%%%%%%%%%%%%%%%%%%%%%%%%%%%%%%%%%%%%%%%%%%%%%%%%%%
\subsection{Revision History}

%%%%%%%%%%%%%%%%%%%%%%%%%%%%%%%%%%%%%%%%
\paragraph{v2.0:} 2018/12/30

\begin{itemize}
\item
immediate forward processing
\item
added |\childdocby| mechanism
\item
manual restructured
\end{itemize}

%%%%%%%%%%%%%%%%%%%%%%%%%%%%%%%%%%%%%%%%
\paragraph{v1.6:} 2018/01/17

\begin{itemize}
\item
application for development of include files
\item
corrections to manual
\end{itemize}

%%%%%%%%%%%%%%%%%%%%%%%%%%%%%%%%%%%%%%%%
\paragraph{v1.5:} 2017/05/21

\begin{itemize}
\item
more complete structuring introduced
\item
|\childdocof| introduced
\item
|\childdoc| renamed to |\childdocmain|
\item
|\childredirect| renamed to |\childdocforward| and |\childdocforwardprefix|
and functionality expanded
\end{itemize}

%%%%%%%%%%%%%%%%%%%%%%%%%%%%%%%%%%%%%%%%
\paragraph{v1.0:} 2017/04/27

\begin{itemize}
\item
manual and install package
\item
first version published on CTAN
\end{itemize}

%%%%%%%%%%%%%%%%%%%%%%%%%%%%%%%%%%%%%%%%
\paragraph{v0.6:} 2017/04/26

\begin{itemize}
\item
redirection mechanism added
\end{itemize}

%%%%%%%%%%%%%%%%%%%%%%%%%%%%%%%%%%%%%%%%
\paragraph{v0.5:} 2017/04/26

\begin{itemize}
\item
functionality in definition file
\end{itemize}


%%%%%%%%%%%%%%%%%%%%%%%%%%%%%%%%%%%%%%%%%%%%%%%%%%%%%%%%%%%%%%%%%%%%%%%%%%%%%%%%
%%%%%%%%%%%%%%%%%%%%%%%%%%%%%%%%%%%%%%%%%%%%%%%%%%%%%%%%%%%%%%%%%%%%%%%%%%%%%%%%
%%%%%%%%%%%%%%%%%%%%%%%%%%%%%%%%%%%%%%%%%%%%%%%%%%%%%%%%%%%%%%%%%%%%%%%%%%%%%%%%
\appendix

\settowidth\MacroIndent{\rmfamily\scriptsize 000\ }

 \DocInput{childdoc.dtx}

\end{document}
%</driver>
% \fi
%
% %%%%%%%%%%%%%%%%%%%%%%%%%%%%%%%%%%%%%%%%%%%%%%%%%%%%%%%%%%%%%%%%%%%%%%%%%%%%%%
% %%%%%%%%%%%%%%%%%%%%%%%%%%%%%%%%%%%%%%%%%%%%%%%%%%%%%%%%%%%%%%%%%%%%%%%%%%%%%%
% \section{Sample}
%\iffalse
%<*samplemain>
%\fi
%
% The following presents a sample document
% with two chapters, two parts, a title page,
% a compile flag as well as three forwarding files to set the flag.
% It consists of eight |.tex| files:
% \begin{center}
% \begin{tabular}{ll}
% |cdocsamp.tex|&main file\\
% |cdocsch1.tex|&include file for chapter 1\\
% |cdocsch2.tex|&include file for chapter 2\\
% |cdocspt3.tex|&include file for part 3\\
% |cdocspt4.tex|&include file for part 4\\
% |cdocsdrf.tex|&forwarding file for main file in draft mode\\
% |cdocsfi1.tex|&forwarding file for final version of chapter 1\\
% |cdocsfi2.tex|&forwarding file for final version of chapter 2\\
% \end{tabular}
% \end{center}
% Each of the eight files can be compiled directly by the \LaTeX{} compiler.
%
% %%%%%%%%%%%%%%%%%%%%%%%%%%%%%%%%%%%%%%
% \paragraph{Main File.}
%
% The main file is called |cdocsamp.tex|.
%
% Load the \textsf{childdoc} definitions and
% declare the filename for the main document:
%    \begin{macrocode}
\input{childdoc.def}
\childdocmain{}
%    \end{macrocode}

% Optional override for |\version| flag:
%    \begin{macrocode}
%%\ifchilddoc\else\providecommand{\version}{draft}\fi
%    \end{macrocode}

% Define the default values for the |\version| flag
% (|final| for the main file and |draft| for childs):
%    \begin{macrocode}
\ifchilddoc
\providecommand{\version}{draft}
\else
\providecommand{\version}{final}
\fi
%    \end{macrocode}

% Load the standard document class:
%    \begin{macrocode}
\documentclass[12pt]{article}
%    \end{macrocode}

% Start the document body:
%    \begin{macrocode}
\begin{document}
%    \end{macrocode}

% Declare a title page.
% Print title, part of document being processed and version flag:
%    \begin{macrocode}
\addtocounter{page}{-1}
\begin{center}
{\LARGE\bfseries{}childdoc example\par}
\vspace{1cm}
\ifchilddoc
\ifchilddocmanual part\else chapter\fi:
`\childdocname' of `\childdocjob'\par
\else
main document: `\childdocjob'\par
\fi
version: \version\par
\end{center}
\newpage
%    \end{macrocode}

% Manually include selected file,
% otherwise process as usual:
%    \begin{macrocode}
\ifchilddocmanual
\section*{part `\childdocname'}
\input{\childdocname}
\else
%    \end{macrocode}

% Include the two chapters:
%    \begin{macrocode}
\include{cdocsch1}
\include{cdocsch2}
%    \end{macrocode}

% Include the two parts unless only chapters should be displayed:
%    \begin{macrocode}
\ifchilddoc\else
\section{part three}
\input{cdocspt3}
\section{part four}
\input{cdocspt4}
\fi
%    \end{macrocode}

% Process as usual until here:
%    \begin{macrocode}
\fi
%    \end{macrocode}

% End of document body:
%    \begin{macrocode}
\end{document}
%    \end{macrocode}
%\iffalse
%</samplemain>
%\fi
%
% %%%%%%%%%%%%%%%%%%%%%%%%%%%%%%%%%%%%%%
% \paragraph{Chapter Include Files.}
%
% The include files are called |cdocsch1.tex| and |cdocsch2.tex|.
%
%\iffalse
%<*samplechap1|samplechap2>
%\fi

% Optional override for |\version| flag:
%    \begin{macrocode}
%%\providecommand{\version}{final}
%    \end{macrocode}

% Include the main document:
%    \begin{macrocode}
\input{childdoc.def}
\childdocof{cdocsamp}
%    \end{macrocode}

%\iffalse
%</samplechap1|samplechap2>
%\fi
%
%\iffalse
%<*samplechap1>
%\fi
% Some text for chapter 1:
%    \begin{macrocode}
\section{one}
some text in chapter one
%    \end{macrocode}

%\iffalse
%</samplechap1>
%\fi
% Some text for chapter 2:
%\iffalse
%<*samplechap2>
%\fi
%    \begin{macrocode}
\section{two}
more text in chapter two
%    \end{macrocode}

%\iffalse
%</samplechap2>
%\fi
%
% %%%%%%%%%%%%%%%%%%%%%%%%%%%%%%%%%%%%%%
% \paragraph{Part Include Files.}
%
% The include files are called |cdocspt3.tex| and |cdocspt4.tex|.
%
%\iffalse
%<*samplepart3|samplepart4>
%\fi

% Optional override for |\version| flag:
%    \begin{macrocode}
%%\providecommand{\version}{final}
%    \end{macrocode}

% Include the main document:
%    \begin{macrocode}
\input{childdoc.def}
\childdocby{cdocsamp}
%    \end{macrocode}

%\iffalse
%</samplepart3|samplepart4>
%\fi
%
%\iffalse
%<*samplepart3>
%\fi
% Some text for part 3:
%    \begin{macrocode}
some text in part three
%    \end{macrocode}

%\iffalse
%</samplepart3>
%\fi
% Some text for part 4:
%\iffalse
%<*samplepart4>
%\fi
%    \begin{macrocode}
more text in part four
%    \end{macrocode}

%\iffalse
%</samplepart4>
%\fi
%
% %%%%%%%%%%%%%%%%%%%%%%%%%%%%%%%%%%%%%%
% \paragraph{Forwarding for a Complete Draft.}
%
% The following forwarding file |cdocsdrf.tex|
% compiles the main document in draft mode:
%\iffalse
%<*sampledraft>
%\fi
%    \begin{macrocode}
\def\version{draft}
\input{childdoc.def}
\childdocforward{cdocsamp}
%    \end{macrocode}

%\iffalse
%</sampledraft>
%\fi
%
% %%%%%%%%%%%%%%%%%%%%%%%%%%%%%%%%%%%%%%
% \paragraph{Forwarding for Final Version of the Chapters.}
%
% The following forwarding files |cdocsfn1.tex| and |cdocsfn2.tex|
% (with identical content)
% compile the final versions of the child documents
% |cdocsch1.tex| and |cdocsch2.tex|, respectively:
%\iffalse
%<*samplefinal>
%\fi
%    \begin{macrocode}
\def\version{final}
\input{childdoc.def}
\childdocforwardprefix[cdocsamp]{cdocsfn}{cdocsch}
%    \end{macrocode}

%\iffalse
%</samplefinal>
%\fi
%
% %%%%%%%%%%%%%%%%%%%%%%%%%%%%%%%%%%%%%%
% \paragraph{Command Line Processing.}
%
% The following three command lines generate the output files
% |cdocscld|, |cdocscl1| and |cdocscl2|
% which should be identical to
% |cdocsdrf|, |cdocsch1| and |cdocsfn2|, respectively:
% \begin{center}
% \begin{tabular}{l}
% |latex -jobname cdocscld \|\\
% |  "\def\version{draft}\input{childdoc.def}\childdocforward{cdocsamp}"|\\
% |latex -jobname cdocscl1 \|\\
% |  "\input{childdoc.def}\childdocforward[cdocsamp]{cdocsch1}"|\\
% |latex -jobname cdocscl2 \|\\
% |  "\def\version{final}\input{childdoc.def}\childdocforward{cdocsch2}"|
% \end{tabular}
% \end{center}
% Note that the trailing backslash on each first line
% merely continues the input to the second line
% (for convenient cut ant paste).
% Furthermore, the command |latex| can be replaced by any
% of its alternative versions such as |pdflatex|.
%
% %%%%%%%%%%%%%%%%%%%%%%%%%%%%%%%%%%%%%%%%%%%%%%%%%%%%%%%%%%%%%%%%%%%%%%%%%%%%%%
% %%%%%%%%%%%%%%%%%%%%%%%%%%%%%%%%%%%%%%%%%%%%%%%%%%%%%%%%%%%%%%%%%%%%%%%%%%%%%%
% \section{Implementation}
%\iffalse
%<*package>
%\fi
%
% This section describes the definitions file |childdoc.def|.

% The definitions cannot be loaded using |\usepackage| or |\RequirePackage|
% which has a mechanism to prevent loading a style file more than once.
% When loading the definitions by means of |\input|
% multiple instances have to be prevented manually:
%\iffalse
%This code needs to be before the `\ProvidesFile' directive
%which is defined at the beginning of this file.
%Therefore it is also placed there and commented out here.
%</package>
%<*discard>
%\fi
%    \begin{macrocode}
\ifdefined\childdocmain\endinput\fi
%    \end{macrocode}
%\iffalse
%</discard>
%<*package>
%\fi
%
% \macro{\ifchilddoc}
% \macro{\ifchilddocmanual}
% The conditional |\ifchilddoc| tells whether a
% child (true) or main (false) document is being compiled.
% The conditional |\ifchilddocmanual| tells whether
% the |\includeonly| mechanism is used (false) or
% the selection of child files must be performed manually (true).
% The definitions initialise to false:
%    \begin{macrocode}
\newif\ifchilddoc
\newif\ifchilddocmanual
%    \end{macrocode}

% \macro{\childdocname}
% \macro{\childdocjob}
% The macro |\childdocname| stores the name of the main document
% to be compiled. The macro |\childdocjob| stores the name of
% the document on which the \LaTeX{} compiler was originally invoked.
% The content of |\jobname| cannot be compared
% to filenames specified in the source due to different catcodes.
% The following code rescans |\jobname|, stores the result
% in |\childdocname| and saves a copy in |\childdocjob|:
%    \begin{macrocode}
\edef\childdocname{\scantokens\expandafter{\jobname\noexpand}}
\let\childdocjob\childdocname
%    \end{macrocode}

% \macro{\childdocdisable}
% The macro |\childdocdisable| prevents the main file
% from being processed more than once.
% At this stage, the main document command |\childdocmain|
% is assumed to be called once again where it should do nothing.
% Any subsequent call to it should prevent
% a secondary processing of the main document
% It overwrites the forwarding commands
% |\childdocof| and |\childdocforward|
% with empty macros to prevent further inclusions of the main document:
%    \begin{macrocode}
\newcommand{\childdocdisable}
{
  \renewcommand{\childdocmain}[1]{\renewcommand{\childdocmain}[1]{\endinput}}
  \renewcommand{\childdocof}[1]{}
  \renewcommand{\childdocby}[2][]{}
  \renewcommand{\childdocforward}[2][]{}
  \renewcommand{\childdocdisable}{}
}
%    \end{macrocode}

% \macro{\childdocmain}
% The macro |\childdocmain| is to be called at the top of the main file
% with nothing or the main filename (without extension) as argument.
% First, it breaks loops.
% If the argument is not empty and does not match |\childdocname|
% (which is set by the first inclusion of |childdoc.def|),
% |\ifchilddoc| is set to true, |\includeonly| is applied to the child file
% and |\jobname| is set to the main file
% (for proper handling of |.aux| files):
%    \begin{macrocode}
\newcommand{\childdocmain}[1]
{
  \childdocdisable\childdocmain{}
  \if?#1?\else
    \begingroup
      \def\childdoctmp{#1}
      \ifx\childdoctmp\childdocname
        \def\childdoctmp{}
      \else
        \def\childdoctmp
        {
          \childdoctrue
          \includeonly{\childdocname}
          \def\childdocjob{#1}
          \def\jobname{#1}
        }
      \fi
      \expandafter
    \endgroup
    \childdoctmp
  \fi
}
%    \end{macrocode}

% \macro{\childdocof}
% The command |\childdocof| redirects
% compilation to the main file |#1|.
%    \begin{macrocode}
\newcommand{\childdocof}[1]
{
  \childdocdisable
  \childdoctrue
  \includeonly{\childdocname}
  \def\jobname{#1}
  \def\childdocjob{#1}
  \input{#1}
}
%    \end{macrocode}

% \macro{\childdocby}
% The command |\childdocby| ....
%    \begin{macrocode}
\newcommand{\childdocby}[2][]
{
  \childdocdisable
  \childdoctrue
  \childdocmanualtrue
  \if?#1?\else
    \def\jobname{#2}
  \fi
  \def\childdocjob{#2}
  \input{#2}
  \endinput
}
%    \end{macrocode}

% \macro{\childdocforward}
% The command |\childdocforward| redirects
% compilation to the main file or
% (if the optional argument is given) a child file.
% Parameters are set as if the main file
% or a child file starting with |\childdocof| was compiled.
% Then compilation is handed over to the main file:
%    \begin{macrocode}
\newcommand{\childdocforward}[2][]
{
  \begingroup
    \if?#1?
      \def\childdoctmp
      {
        \def\childdocname{#2}
        \def\childdocjob{#2}
        \def\jobname{#2}
        \input{#2}
        \endinput
      }
    \else
      \def\childdoctmp
      {
        \childdocdisable
        \def\childdocname{#2}
        \childdoctrue
        \includeonly{#2}
        \def\childdocjob{#1}
        \def\jobname{#1}
        \input{#1}
        \endinput
      }
    \fi
    \expandafter
  \endgroup
  \childdoctmp
}
%    \end{macrocode}

% \macro{\childdocforwardprefix}
% The command |\childdocforwardprefix| redirects
% compilation to the main or a child file by means of a pattern.
% The prefix |#1| in the current filename is replaced by |#2|
% and the suffix of the current filename is kept
% (it is assumed that the filename does not contain the substring `|~~~|'
% which is used as a delimiter).
% Compilation is handed over to the new file by |\childdocforward|:
%    \begin{macrocode}
\newcommand{\childdocforwardprefix}[3][]
{
  \begingroup
    \def\childdocextract #2##1~~~{\def\childdoctmp{\childdocforward[#1]{#3##1}}}
    \expandafter\childdocextract\childdocname~~~
    \expandafter
  \endgroup
  \childdoctmp
}
%    \end{macrocode}

% \macro{\childdoc}
% The deprecated macro |\childdoc| is a legacy version of |\childdocmain|:
%    \begin{macrocode}
\newcommand{\childdoc}{\childdocmain}
%    \end{macrocode}

% \macro{\childdocredirect}
% The deprecated macro |\childdocredirect| is a legacy version
% of |\childdocforward| and |\childdocforwardprefix|:
%    \begin{macrocode}
\newcommand{\childdocredirect}[2][]
{
  \begingroup
    \if?#1?
      \def\childdoctmp{\childdocforward{#2}}
    \else
      \def\childdoctmp{\childdocforwardprefix{#1}{#2}}
    \fi
    \expandafter
  \endgroup
  \childdoctmp
}
%    \end{macrocode}

%\iffalse
%</package>
%\fi
%
\endinput
|\\
|\childdocforwardprefix{final}{child}|
\end{tabular}
\end{center}
%

Note that when several versions of a main file and/or of each child file
are to be generated, it may be convenient to set up a |Makefile| or
shell script to automatise the process.

%%%%%%%%%%%%%%%%%%%%%%%%%%%%%%%%%%%%%%%%%%%%%%%%%%%%%%%%%%%%%%%%%%%%%%%%%%%%%%%%
\subsection{Command Line Processing}
\label{sec:commandline}

The effect of redirection files can also be achieved by invoking
the \LaTeX{} compiler with a more elaborate command line.
Most conveniently this should be done as part
of a shell script or a |Makefile|.

When using \textsf{childdoc} in the main file, the following
command lines effectively perform a redirection
(note that depending on the shell being used,
backslashes may have to be doubled: `|\|' $\to$ `|\\|'):
%
\begin{center}
|... -jobname "|\textit{target}|" |\\|"|[\textit{flags}]%
|% \iffalse
%
% childdoc.dtx Copyright (C) 2017-2018 Niklas Beisert
%
% This work may be distributed and/or modified under the
% conditions of the LaTeX Project Public License, either version 1.3
% of this license or (at your option) any later version.
% The latest version of this license is in
%   http://www.latex-project.org/lppl.txt
% and version 1.3 or later is part of all distributions of LaTeX
% version 2005/12/01 or later.
%
% This work has the LPPL maintenance status `maintained'.
%
% The Current Maintainer of this work is Niklas Beisert.
%
% This work consists of the files childdoc.dtx and childdoc.ins
% and the derived files childdoc.def and cdocsamp.tex with
% cdocsch1.tex, cdocsch2.tex, cdocsdrf.tex, cdocsfn1.tex, cdocsfn2.tex.
%
%<package>\ifdefined\childdocmain\endinput\fi
%<package>\ProvidesFile{childdoc.def}[2018/12/30 v2.0 child document driver]
%<samplemain>\ProvidesFile{cdocsamp.tex}[2018/12/30 v2.0 sample for childdoc]
%<*driver>
%\ProvidesFile{childdoc.drv}[2018/12/30 v2.0 childdoc reference manual file]
\PassOptionsToClass{10pt,a4paper}{article}
\documentclass{ltxdoc}

\usepackage[margin=35mm]{geometry}
\usepackage{hyperref}
\usepackage{hyperxmp}
\usepackage[usenames]{color}

\hypersetup{colorlinks=true}
\hypersetup{pdfstartview=FitH}
\hypersetup{pdfpagemode=UseNone}
\hypersetup{pdfsource={}}
\hypersetup{pdflang={en-UK}}
\hypersetup{pdfcopyright={Copyright 2017-2018 Niklas Beisert.
  This work may be distributed and/or modified under the
  conditions of the LaTeX Project Public License, either version 1.3
  of this license or (at your option) any later version.}}
\hypersetup{pdflicenseurl={http://www.latex-project.org/lppl.txt}}
\hypersetup{pdfcontactaddress={ETH Zurich, ITP, HIT K,
  Wolfgang-Pauli-Strasse 27}}
\hypersetup{pdfcontactpostcode={8093}}
\hypersetup{pdfcontactcity={Zurich}}
\hypersetup{pdfcontactcountry={Switzerland}}
\hypersetup{pdfcontactemail={nbeisert@itp.phys.ethz.ch}}
\hypersetup{pdfcontacturl={http://people.phys.ethz.ch/\xmptilde nbeisert/}}

\newcommand{\secref}[1]{\hyperref[#1]{section \ref*{#1}}}

\parskip1ex
\parindent0pt
\let\olditemize\itemize
\def\itemize{\olditemize\parskip0pt}

\begin{document}

\title{The \textsf{childdoc} Package}
\hypersetup{pdftitle={The childdoc Package}}
\author{Niklas Beisert\\[2ex]
  Institut f\"ur Theoretische Physik\\
  Eidgen\"ossische Technische Hochschule Z\"urich\\
  Wolfgang-Pauli-Strasse 27, 8093 Z\"urich, Switzerland\\[1ex]
  \href{mailto:nbeisert@itp.phys.ethz.ch}
  {\texttt{nbeisert@itp.phys.ethz.ch}}}
\hypersetup{pdfauthor={Niklas Beisert}}
\hypersetup{pdfsubject={Manual for the LaTeX2e Package childdoc}}
\date{30 December 2018, \textsf{v2.0}}
\maketitle

\begin{abstract}\noindent
\textsf{childdoc} is a \LaTeXe{} package
that enables the direct compilation
of document sections included by |\include|
to individual files.
\end{abstract}

\begingroup
\parskip0ex
\tableofcontents
\endgroup

%%%%%%%%%%%%%%%%%%%%%%%%%%%%%%%%%%%%%%%%%%%%%%%%%%%%%%%%%%%%%%%%%%%%%%%%%%%%%%%%
%%%%%%%%%%%%%%%%%%%%%%%%%%%%%%%%%%%%%%%%%%%%%%%%%%%%%%%%%%%%%%%%%%%%%%%%%%%%%%%%
\section{Introduction}

\LaTeX{} provides a mechanism to structure a large document (such as a book)
into a main file and several child files (containing the chapters)
using the |\include| command.
This mechanism is beneficial for documents
which span hundreds of pages in order to
make the source file(s) more manageable.
Moreover, compilation can be restricted to
selected child files by means of the |\includeonly| command.
The latter feature can be used to reduce the compilation time while editing
(this was significantly more useful in the earlier days of \LaTeX{})
or to generate a smaller document which is easier to navigate.
Another application of |\includeonly| is to generate
documents consisting of selected parts of the complete document.

However, there are a few drawbacks of the plain |\include| mechanism:
\begin{itemize}
\item
The child files cannot be compiled on their own,
they can only be compiled via the main file.
A naive editing environment
(such as a text editor with an option
to have the current file processed by \LaTeX)
may require one to switch to the main file before compiling;
attempting to compile the child file produces errors.
\item
The main file must be modified (each time)
to adjust the |\includeonly| command
to the present needs. This easily leaves the main file in a messy state.
\item
The generated document will always carry the filename
of the main document. This is inconvenient if
several child files are to be compiled and
to be kept for distribution.
\end{itemize}

The present package provides a simple interface
to make child files individually compilable by \LaTeX{}.
Compiling a child file then has the same effect as compiling
the main file with an |\includeonly| command
to select the appropriate child.
Moreover the generated document will carry the name of the child
rather than the main file.
This resolves all three above issues.

This feature is meant to make the editing of books,
thesis documents and lecture notes somewhat more convenient.
However, the package can also be used efficiently for
composing a series of documents (such as exercise sheets)
which are typically distributed individually.
It then assists the author in generating the individual documents
(potentially in different versions)
as well as a document containing the collected series.
Another application is in developing style files
or other kinds of included material
where compilation of the style file could redirect
to a sample or test file.

%%%%%%%%%%%%%%%%%%%%%%%%%%%%%%%%%%%%%%%%%%%%%%%%%%%%%%%%%%%%%%%%%%%%%%%%%%%%%%%%
%%%%%%%%%%%%%%%%%%%%%%%%%%%%%%%%%%%%%%%%%%%%%%%%%%%%%%%%%%%%%%%%%%%%%%%%%%%%%%%%
\section{Usage}

First of all, the package \textsf{childdoc} is \emph{not} a standard
\LaTeXe{} |.sty| style file! Therefore it needs to be invoked in
a non-standard way.

%%%%%%%%%%%%%%%%%%%%%%%%%%%%%%%%%%%%%%%%%%%%%%%%%%%%%%%%%%%%%%%%%%%%%%%%%%%%%%%%
\subsection{Included Files}
\label{sec:include}

%%%%%%%%%%%%%%%%%%%%%%%%%%%%%%%%%%%%%%%%
\DescribeMacro{\childdocmain}
To use the package, add the commands
\begin{center}
\begin{tabular}{l}
|\input{childdoc.def}|\\
|\childdocmain{}|\\
\end{tabular}
\end{center}
at the very top of the main \LaTeX{} file,
in particular \emph{before} the |\documentclass| statement!
The argument of |\childdocmain| should be left empty
(but it must be present).

%%%%%%%%%%%%%%%%%%%%%%%%%%%%%%%%%%%%%%%%
\DescribeMacro{\childdocof}
Furthermore, add the commands
\begin{center}
\begin{tabular}{l}
|\input{childdoc.def}|\\
|\childdocof{|\textit{main}|}|\\
\end{tabular}
\end{center}
at the top of every child file \textit{child}
which is included by |\include{|\textit{child}|}|
from within the main file
(or at least for those files to be compiled individually).
The argument \textit{main} must be the filename of the main file.

There are a couple of
considerations in setting up the main and child documents:

%%%%%%%%%%%%%%%%%%%%%%%%%%%%%%%%%%%%%%%%
\paragraph{Restrictions.}

Please note the following restrictions:
\begin{itemize}
\item
|\childdocmain| must be called with one argument \textit{main}
to ensure compatibility with earlier version of the package.
It must either be empty (|\childdocmain{}|)
or precisely match the filename of the main file in which it is specified.
See \secref{sec:detection} for further information.
\item
The filename \textit{main} must be specified without the |.tex| extension.
\item
The filename \textit{main} is case sensitive
(even in case-insensitive file systems)
due to internal string comparison.
\item
The argument \textit{main} should be fully expanded, it cannot be a macro.
\item
Subdirectories and special characters should be avoided in filenames.
\item
The command |\childdocmain{|\textit{main}|}| must be followed by a whitespace.
It should not be followed immediately by another command
or by a comment mark `|%|'.
This is because the \TeX{} parser reads the token immediately following
the argument of |\childdocmain| and puts it
at the beginning of every child section;
however, a white\-space is ignored.
\end{itemize}

%%%%%%%%%%%%%%%%%%%%%%%%%%%%%%%%%%%%%%%%
\paragraph{Content of Main File.}

It is advisable to place all content in the child files included by |\include|.
Any output contained in the main file will appear in all child documents
unless suppressed manually;
it cannot be suppressed automatically by the |\includeonly| directive
and thus should normally be avoided.
A method to include some content in the main file
by means of conditional processing is described in \secref{sec:conditional}.

%%%%%%%%%%%%%%%%%%%%%%%%%%%%%%%%%%%%%%%%
\paragraph{Page Numbering.}

When only a part of the document is compiled,
the appropriate numbering of pages
(as well as other status parameters)
is determined from the |.aux| files.
The latter contain information from previous passes.
However this information needs to propagate through
all intermediate child documents.
Therefore the page numbering in child documents may well
be inconsistent until the complete document is compiled at least once.

A useful (if unconventional) way to always ensure a consistent
page numbering is to restart the numbering in each child document
and denote the pages by `\textit{child}|.|\textit{page}'
where \textit{child} represents the chapter/section number of the child file.
This can be achieved by the command
|\numberwithin{page}{|\textit{child}|}|
of the \textsf{amsmath} package
where \textit{child} can be |chapter| or |section|
depending on the chosen structuring.
Alternatively, one can modify the macro |\thepage| appropriately
and reset the counter |page| at the start of each child file.

%%%%%%%%%%%%%%%%%%%%%%%%%%%%%%%%%%%%%%%%%%%%%%%%%%%%%%%%%%%%%%%%%%%%%%%%%%%%%%%%
\subsection{Conditional Processing}
\label{sec:conditional}

The package provides a mechanism to compile different versions
of a document. To customise the versions further some conditional processing
can come in handy to distinguish which version is being compiled.
The package provides two macros to describe the compilation context:

%%%%%%%%%%%%%%%%%%%%%%%%%%%%%%%%%%%%%%%%
\DescribeMacro{\ifchilddoc}
The conditional |\ifchilddoc| distinguishes between the compilation of
child documents and the main document:
%
\begin{center}
|\ifchilddoc |\textit{child-code}| |[|\||else |\textit{main-code}]| \||fi|
\end{center}

%%%%%%%%%%%%%%%%%%%%%%%%%%%%%%%%%%%%%%%%
\DescribeMacro{\childdocname}
\DescribeMacro{\childdocjob}
The macro |\childdocname| contains the filename (without extension)
of the main or child file being processed.
Note that |\childdocjob| will always contain the name of the main file.

%%%%%%%%%%%%%%%%%%%%%%%%%%%%%%%%%%%%%%%%
\paragraph{Title Page.}

Conditional processing can be used to include a title or banner page
in the main document when proper precautions are taken.
Importantly, the code in the main file should ensure that the page counter
(as well as other status parameters which are stored in the |.aux| files)
takes the same value after the conditional processing.
Otherwise the page numbers may take divergent values
depending on which part is compiled.

For example, a title page could be declared by:
%
\begin{center}
\begin{tabular}{l}
|\ifchilddoc\||else|\\
|\addtocounter{page}{-1}|\\
\textit{code for title page}\\
|\newpage|\\
|\||fi|
\end{tabular}
\end{center}
%
A banner page for the child documents can be generated by:
%
\begin{center}
\begin{tabular}{l}
|\ifchilddoc|\\
|\addtocounter{page}{-1}|\\
\textit{code for banner page}\\
|\newpage|\\
|\||fi|
\end{tabular}
\end{center}
%
Here one could write a message such as:
\begin{center}
|This is the part \childdocname{} of \childdocjob{}.|
\end{center}

%%%%%%%%%%%%%%%%%%%%%%%%%%%%%%%%%%%%%%%%%%%%%%%%%%%%%%%%%%%%%%%%%%%%%%%%%%%%%%%%
\subsection{Flags}
\label{sec:flags}

The package makes it easy to generate different versions
of the main or child documents.
To this end compilation flags can be defined
and assigned different default values.
They will be particularly useful in conjunction
with the forwarding mechanism described in \secref{sec:forward}.

For example, it may be useful to have a flag |\version|
which can be set to |draft| or |final|.
The document source will contain some conditional code
depending on the value of |\version|.
Suppose further, the flag should default to |final| for the main file
and to |draft| for child files
which is a natural assignment for editing the document.
This is achieved by placing the following code
in the preamble of the main document
(below the |\childdocmain| directive):
%
\begin{center}
\begin{tabular}{l}
|\ifchilddoc|\\
|\providecommand{\version}{draft}|\\
|\||else|\\
|\providecommand{\version}{final}|\\
|\||fi|
\end{tabular}
\end{center}
%
The definition by |\providecommand| makes sure
that previous definitions are not overwritten.
Further statements |\providecommand{\version}{...}|
can thus be added before the above code to override it.

For the main file, one might add a line
(between |\childdocmain| and the above block)
%
\begin{center}
|%\ifchilddoc\||else\providecommand{\version}{draft}\||fi|
\end{center}
%
which can be uncommented to produce a draft version.
Likewise one can add a line to the very top of a child file
(above the |\childdocof{|\textit{main}|}| directive)
%
\begin{center}
|%\providecommand{\version}{final}|
\end{center}
%
which can be uncommented to produce the final version of this child document.

%%%%%%%%%%%%%%%%%%%%%%%%%%%%%%%%%%%%%%%%%%%%%%%%%%%%%%%%%%%%%%%%%%%%%%%%%%%%%%%%
\subsection{Forwarding}
\label{sec:forward}

Different versions of the main or child documents
using compilation flags as described in \secref{sec:flags}
can be (permanently) stored in different files
for convenient compilation, viewing and distribution.
To this end, the package defines a command
to pass on compilation to a different file:

%%%%%%%%%%%%%%%%%%%%%%%%%%%%%%%%%%%%%%%%
\DescribeMacro{\childdocforward}
The command |\childdocforward| redirects processing to
another source file:
%
\begin{center}
\begin{tabular}{l}
|\input{childdoc.def}|\\
|\childdocforward[|\textit{main}|]{|\textit{dest}|}|\\
\end{tabular}
\end{center}
%
The argument \textit{dest} is the destination file
(without extension).
It should be the main file or one of the child files.
Note that further \textsf{childdoc} directives
such as |\childdocof| and |\childdocforward|
in the indicated file will be processed in this form.
The optional argument \textit{main}
passes on directly to the main file \textit{main}
while pretending to compile the child \textit{dest}.
This form behaves as if \textit{dest}
issues |\childdocof{|\textit{main}|}| right away,
and no further \textsf{childdoc} directives will be processed.

%%%%%%%%%%%%%%%%%%%%%%%%%%%%%%%%%%%%%%%%
\DescribeMacro{\...prefix}
In the alternative form |\childdocforwardprefix|,
%
\begin{center}
\begin{tabular}{l}
|\input{childdoc.def}|\\
|\childdocforwardprefix[|\textit{main}|]{|\textit{prefix}|}{|\textit{dest}|}|
\end{tabular}
\end{center}
%
the destination file is determined by a pattern
depending on the current file:
To make this work, the current file must be called
`{\textit{prefix}\hspace{0.2em}\textit{suffix}}'
with \textit{prefix} matching precisely the argument.
Processing is then passed on to the file
`{\textit{dest}\hspace{0.2em}\textit{suffix}}'.
Surely, the same effect is achieved by
directly specifying the
argument `{\textit{dest}\hspace{0.2em}\textit{suffix}}'
in the first form.
However, that requires to set up a different file
for each child. With the alternative form of the command
all these files can have exactly the same content
which simplifies setting them up and maintaining them.

For example, the following file |draft.tex|
with a compilation flag |\version| as described in \secref{sec:flags}
compiles the main document as a draft:
%
\begin{center}
\begin{tabular}{l}
|\def\version{draft}|\\
|\input{childdoc.def}|\\
|\childdocforward{|\textit{main}|}|
\end{tabular}
\end{center}
%
Likewise, the following files |final|\textit{nn}|.tex|
compile the final version of the child document
|child|\textit{nn}|.tex|:
%
\begin{center}
\begin{tabular}{l}
|\def\version{final}|\\
|\input{childdoc.def}|\\
|\childdocforwardprefix{final}{child}|
\end{tabular}
\end{center}
%

Note that when several versions of a main file and/or of each child file
are to be generated, it may be convenient to set up a |Makefile| or
shell script to automatise the process.

%%%%%%%%%%%%%%%%%%%%%%%%%%%%%%%%%%%%%%%%%%%%%%%%%%%%%%%%%%%%%%%%%%%%%%%%%%%%%%%%
\subsection{Command Line Processing}
\label{sec:commandline}

The effect of redirection files can also be achieved by invoking
the \LaTeX{} compiler with a more elaborate command line.
Most conveniently this should be done as part
of a shell script or a |Makefile|.

When using \textsf{childdoc} in the main file, the following
command lines effectively perform a redirection
(note that depending on the shell being used,
backslashes may have to be doubled: `|\|' $\to$ `|\\|'):
%
\begin{center}
|... -jobname "|\textit{target}|" |\\|"|[\textit{flags}]%
|\input{childdoc.def}\childdocforward[|\textit{main}|]{|\textit{dest}|}"|
\end{center}
%
Here \textit{target} is the name of the output file,
\textit{main} is the name of the main file
and \textit{dest} is the name of the main or child file to be processed
(all filenames without extensions).
The optional argument \textit{main} can be omitted
if \textit{main} matches \textit{dest}.
Optionally, compilation \textit{flags} can be defined via |\def| commands.
This command line makes the \TeX{} engine believe
it is compiling the file \textit{target}
whose content is specified as the latter parameter.
The provided code then forwards the processing to
\textit{main} or \textit{dest} as described in \secref{sec:forward}.

%%%%%%%%%%%%%%%%%%%%%%%%%%%%%%%%%%%%%%%%%%%%%%%%%%%%%%%%%%%%%%%%%%%%%%%%%%%%%%%%
\subsection{Include by Input}
\label{sec:input}

Including child documents by |\include| has some restrictions by design.
Most notably, the content of a child document always occupies
its own set of pages; pages cannot be shared between child documents.
Usually, this behaviour makes perfect sense
because each child document contain an essential part of the document.
However, in some situations it may be desirable to compose
a document from a collection of parts
without having mandatory page breaks between then.
For this case, the package
provides a mechanism to include parts
by |\input| which can also be processed individually.
However, by construction this mechanism
requires manual handling of the content to be output.

%%%%%%%%%%%%%%%%%%%%%%%%%%%%%%%%%%%%%%%%
\DescribeMacro{\ifchilddocmanual}
The main file should be prepared as usual, see \secref{sec:include}.
However, the document body must make a distinction
between processing of an individual part and of the main document, e.g.:
%
\begin{center}
\begin{tabular}{l}
|\ifchilddocmanual|\\
|\input{\childdocname}|\\
|\||else|\\
\textit{document body with }|\input{|\textit{part}|}|\\
|\||fi|
\end{tabular}
\end{center}
%
The conditional |\ifchilddocmanual| is true whenever
a part to be included by |\input| is being compiled,
and the name of the part is stored in |\childdocname|.

%%%%%%%%%%%%%%%%%%%%%%%%%%%%%%%%%%%%%%%%
\DescribeMacro{\childdocby}
Each part to be included by |\input| should start with:
%
\begin{center}
\begin{tabular}{l}
|\input{childdoc.def}|\\
|\childdocby{|\textit{main}|}|\\
\end{tabular}
\end{center}
%
The directive |\childdocby| is similar to |\childdocof|
described in \secref{sec:include},
but the subsequent selection of content must be done manually.
To that end, both |\ifchilddoc| and |\ifchilddocmanual|
will be true upon processing of a part,
and the name of the part is stored in |\childdocname|.
Note that |\jobname| will be set to the filename of the current part
so that each part receives an individual |.aux| file
that does not interfere with the |.aux| file(s) of the main document.
This behaviour can be altered by the alternative form
|\childdocby[*]{|\textit{main}|}| (with a non-empty optional argument)
which uses the |.aux| file of the main document
by setting |\jobname| to \textit{main}.

%%%%%%%%%%%%%%%%%%%%%%%%%%%%%%%%%%%%%%%%%%%%%%%%%%%%%%%%%%%%%%%%%%%%%%%%%%%%%%%%
\subsection{Driver Development}
\label{sec:driver}

The \textsf{childdoc} mechanism can also be use for the development
of definition files such as \LaTeX{} styles or classes.
This case differs from the above setup with multiple parts
included by |\include| in that no |\includeonly| should be invoked.
This can be achieved by starting the include file
(before |\ProvidesPackage|) with:
%
\begin{center}
\begin{tabular}{l}
|\input{childdoc.def}|\\
|\childdocforward{|\textit{main}|}|\\
\end{tabular}
\end{center}
%
or alternatively with:
%
\begin{center}
\begin{tabular}{l}
|\input{childdoc.def}|\\
|\childdocby{|\textit{main}|}|\\
\end{tabular}
\end{center}
%
Both forms have slightly different effects as described above.
The main file is prepared as usual, see \secref{sec:include}.

%%%%%%%%%%%%%%%%%%%%%%%%%%%%%%%%%%%%%%%%%%%%%%%%%%%%%%%%%%%%%%%%%%%%%%%%%%%%%%%%
\subsection{Legacy Detection}
\label{sec:detection}

The directive |\childdocmain| in the main file can detect
whether the complete document or merely a child is to be compiled
even without using the directive |\childdocof|.
This method is deprecated because it is less robust
and there is no compelling reason to use it;
it is merely provided for backward compatibility
and it may be removed in future versions.

If the detection mechanism is to be used,
it is mandatory to correctly specify
the filename of the main file as the argument of |\childdocmain|:
%
\begin{center}
\begin{tabular}{l}
|\input{childdoc.def}|\\
|\childdocmain{|\textit{main}|}|\\
\end{tabular}
\end{center}
%
If |\jobname| does not match the argument \textit{main} of |\childdocmain|,
it is assumed that |\jobname| points to the child file to be compiled.
When using |\childdocmain| with the main file specified as argument,
it suffices to start a child file
with just |\input{|\textit{main}|}|
without loading of the package and using |\childdocof|.
If instead all processing is done
with the appropriate \textsf{childdoc} directives,
the argument of \textit{main} of |\childdocmain| can be empty.

An alternative version of the command line processing described
in \secref{sec:commandline} using the detection mechanism reads:
%
\begin{center}
|... -jobname "|\textit{target}|" "|[\textit{flags}]%
[|\def\jobname{|\textit{dest}|}|]|\input{|\textit{main}|}"|
\end{center}

%%%%%%%%%%%%%%%%%%%%%%%%%%%%%%%%%%%%%%%%%%%%%%%%%%%%%%%%%%%%%%%%%%%%%%%%%%%%%%%%
\subsection{Manual Code}
\label{sec:manual}

In case one cannot be certain whether the definitions file |childdoc.def|
is installed on the target \TeX{} distribution
and one prefers not to ship it,
it is conceivable to paste a few relevant commands into the sources.

To that end, drop all statements |\input{childdoc.def}|
and perform the replacements as outlined below.
Instead of |\childdocmain{|\textit{main}|}| add the following code
to the top of the main file:
%
\begin{center}
\begin{tabular}{l}
|\||ifdefined\childdocname\endinput\||fi\newif\ifchilddoc|\\
|\edef\childdocname{\scantokens\expandafter{\jobname\noexpand}}|\\
|\def\childdocmain{|\textit{main}|}\||ifx\childdocmain\childdocname\||else|\\
|\childdoctrue\includeonly{\childdocname}\let\jobname\childdocmain\||fi|\\
\end{tabular}
\end{center}
%
Instead of |\childdocof{|\textit{main}|}| just include the main file
at the top of each child file:
%
\begin{center}
|\input{|\textit{main}|}|
\end{center}
%
A simple redirection |\childdocforward{|\textit{dest}|}| is achieved by:
%
\begin{center}
|\def\jobname{|\textit{dest}|}\input{\jobname}|
\end{center}
%
The redirection with prefix
|\childdocforwardprefix[|\textit{prefix}|]{|\textit{dest}|}|
is accomplished by:
%
\begin{center}
\begin{tabular}{l}
|{\edef\jobname{\scantokens\expandafter{\jobname\noexpand}}|\\
|\def\redirectjob |\textit{prefix}|#1~~~{\gdef\jobname{|\textit{dest}|#1}}|\\
|\expandafter\redirectjob\jobname~~~}\input{\jobname}|
\end{tabular}
\end{center}

In an alternative approach,
child documents can be compiled by a specific command line
without additional code or specific definitions:
%
\begin{center}
|... -jobname "|\textit{target}|" "|[\textit{flags}]%
|\includeonly{|\textit{dest}|}\input{|\textit{main}|}"|
\end{center}
%

%%%%%%%%%%%%%%%%%%%%%%%%%%%%%%%%%%%%%%%%%%%%%%%%%%%%%%%%%%%%%%%%%%%%%%%%%%%%%%%%
%%%%%%%%%%%%%%%%%%%%%%%%%%%%%%%%%%%%%%%%%%%%%%%%%%%%%%%%%%%%%%%%%%%%%%%%%%%%%%%%
\section{Information}

%%%%%%%%%%%%%%%%%%%%%%%%%%%%%%%%%%%%%%%%%%%%%%%%%%%%%%%%%%%%%%%%%%%%%%%%%%%%%%%%
\subsection{Copyright}

Copyright \copyright{} 2017--2018 Niklas Beisert

This work may be distributed and/or modified under the
conditions of the \LaTeX{} Project Public License, either version 1.3
of this license or (at your option) any later version.
The latest version of this license is in
  \url{http://www.latex-project.org/lppl.txt}
and version 1.3 or later is part of all distributions of \LaTeX{}
version 2005/12/01 or later.

This work has the LPPL maintenance status `maintained'.

The Current Maintainer of this work is Niklas Beisert.

This work consists of the files |README.txt|, |childdoc.ins| and |childdoc.dtx|
as well as the derived files |childdoc.def|, |cdocsamp.tex|
with |cdocsch1.tex|, |cdocsch2.tex|, |cdocspt3.tex|, |cdocspt4.tex|,
|cdocsdrf.tex|, |cdocsfn1.tex|, |cdocsfn2.tex|
as well as |childdoc.pdf|.

%%%%%%%%%%%%%%%%%%%%%%%%%%%%%%%%%%%%%%%%%%%%%%%%%%%%%%%%%%%%%%%%%%%%%%%%%%%%%%%%
\subsection{Files and Installation}

The package consists of the files:
%
\begin{center}
\begin{tabular}{ll}
    |README.txt|   & readme file \\
    |childdoc.ins| & installation file \\
    |childdoc.dtx| & source file \\
    |childdoc.def| & definition file \\
    |cdocsamp.tex| & sample main file \\
    |cdocsch1.tex| & sample include file \\
    |cdocsch2.tex| & sample include file \\
    |cdocspt3.tex| & sample part file \\
    |cdocspt4.tex| & sample part file \\
    |cdocsdrf.tex| & sample redirection file \\
    |cdocsfn1.tex| & sample redirection file \\
    |cdocsfn2.tex| & sample redirection file \\
    |childdoc.pdf| & manual
\end{tabular}
\end{center}
%
The distribution consists of the files
|README.txt|, |childdoc.ins| and |childdoc.dtx|.
%
\begin{itemize}
\item
Run (pdf)\LaTeX{} on |childdoc.dtx|
to compile the manual |childdoc.pdf| (this file).
\item
Run \LaTeX{} on |childdoc.ins| to create the definitions file |childdoc.def|
and the sample |cdocsamp.tex| with include files
|cdocsch1.tex|, |cdocsch2.tex|, |cdocspt3.tex|, |cdocspt4.tex|,
|cdocsdrf.tex|, |cdocsfn1.tex|, |cdocsfn2.tex|.
Then copy the file |childdoc.def| to an appropriate directory of your \LaTeX{}
distribution, e.g.\ \textit{texmf-root}|/tex/latex/childdoc|.
\end{itemize}

%%%%%%%%%%%%%%%%%%%%%%%%%%%%%%%%%%%%%%%%%%%%%%%%%%%%%%%%%%%%%%%%%%%%%%%%%%%%%%%%
\subsection{Related CTAN Packages}

There are several other packages which offer a similar functionality:
%
\begin{itemize}
\item
The packages
\href{http://ctan.org/pkg/docmute}{\textsf{docmute}},
\href{http://ctan.org/pkg/includex}{\textsf{includex}} and
\href{http://ctan.org/pkg/standalone}{\textsf{standalone}}
provide commands to include only the document body of
a child file thus allowing both files to be compiled individually.
\item
The packages \href{http://ctan.org/pkg/subdocs}{\textsf{subdocs}}
and \href{http://ctan.org/pkg/subfiles}{\textsf{subfiles}}
provide structures in which the main and child documents can be
encapsulated and allowing them to be compiled individually.
The inclusion mechanism is different from the conventional |\include|.
\item
The package \href{http://ctan.org/pkg/combine}{\textsf{combine}}
is an elaborate solution to combine several documents into one.
\end{itemize}
%
See also the CTAN topic \href{http://ctan.org/topic/subdocs}{\textsf{subdocs}}
for further related packages.
The present package differs from the above solutions in that
a document structure constructed with the conventional |\include| mechanism
just needs two extra commands at the top of every file
such that all constituent files can be compiled individually.

%%%%%%%%%%%%%%%%%%%%%%%%%%%%%%%%%%%%%%%%%%%%%%%%%%%%%%%%%%%%%%%%%%%%%%%%%%%%%%%%
%\subsection{Feature Suggestions}
%
%The following is a list of features which may be useful for future
%versions of this package:
%%
%\begin{itemize}
%\item
%\ldots
%\end{itemize}

%%%%%%%%%%%%%%%%%%%%%%%%%%%%%%%%%%%%%%%%%%%%%%%%%%%%%%%%%%%%%%%%%%%%%%%%%%%%%%%%
\subsection{Revision History}

%%%%%%%%%%%%%%%%%%%%%%%%%%%%%%%%%%%%%%%%
\paragraph{v2.0:} 2018/12/30

\begin{itemize}
\item
immediate forward processing
\item
added |\childdocby| mechanism
\item
manual restructured
\end{itemize}

%%%%%%%%%%%%%%%%%%%%%%%%%%%%%%%%%%%%%%%%
\paragraph{v1.6:} 2018/01/17

\begin{itemize}
\item
application for development of include files
\item
corrections to manual
\end{itemize}

%%%%%%%%%%%%%%%%%%%%%%%%%%%%%%%%%%%%%%%%
\paragraph{v1.5:} 2017/05/21

\begin{itemize}
\item
more complete structuring introduced
\item
|\childdocof| introduced
\item
|\childdoc| renamed to |\childdocmain|
\item
|\childredirect| renamed to |\childdocforward| and |\childdocforwardprefix|
and functionality expanded
\end{itemize}

%%%%%%%%%%%%%%%%%%%%%%%%%%%%%%%%%%%%%%%%
\paragraph{v1.0:} 2017/04/27

\begin{itemize}
\item
manual and install package
\item
first version published on CTAN
\end{itemize}

%%%%%%%%%%%%%%%%%%%%%%%%%%%%%%%%%%%%%%%%
\paragraph{v0.6:} 2017/04/26

\begin{itemize}
\item
redirection mechanism added
\end{itemize}

%%%%%%%%%%%%%%%%%%%%%%%%%%%%%%%%%%%%%%%%
\paragraph{v0.5:} 2017/04/26

\begin{itemize}
\item
functionality in definition file
\end{itemize}


%%%%%%%%%%%%%%%%%%%%%%%%%%%%%%%%%%%%%%%%%%%%%%%%%%%%%%%%%%%%%%%%%%%%%%%%%%%%%%%%
%%%%%%%%%%%%%%%%%%%%%%%%%%%%%%%%%%%%%%%%%%%%%%%%%%%%%%%%%%%%%%%%%%%%%%%%%%%%%%%%
%%%%%%%%%%%%%%%%%%%%%%%%%%%%%%%%%%%%%%%%%%%%%%%%%%%%%%%%%%%%%%%%%%%%%%%%%%%%%%%%
\appendix

\settowidth\MacroIndent{\rmfamily\scriptsize 000\ }

 \DocInput{childdoc.dtx}

\end{document}
%</driver>
% \fi
%
% %%%%%%%%%%%%%%%%%%%%%%%%%%%%%%%%%%%%%%%%%%%%%%%%%%%%%%%%%%%%%%%%%%%%%%%%%%%%%%
% %%%%%%%%%%%%%%%%%%%%%%%%%%%%%%%%%%%%%%%%%%%%%%%%%%%%%%%%%%%%%%%%%%%%%%%%%%%%%%
% \section{Sample}
%\iffalse
%<*samplemain>
%\fi
%
% The following presents a sample document
% with two chapters, two parts, a title page,
% a compile flag as well as three forwarding files to set the flag.
% It consists of eight |.tex| files:
% \begin{center}
% \begin{tabular}{ll}
% |cdocsamp.tex|&main file\\
% |cdocsch1.tex|&include file for chapter 1\\
% |cdocsch2.tex|&include file for chapter 2\\
% |cdocspt3.tex|&include file for part 3\\
% |cdocspt4.tex|&include file for part 4\\
% |cdocsdrf.tex|&forwarding file for main file in draft mode\\
% |cdocsfi1.tex|&forwarding file for final version of chapter 1\\
% |cdocsfi2.tex|&forwarding file for final version of chapter 2\\
% \end{tabular}
% \end{center}
% Each of the eight files can be compiled directly by the \LaTeX{} compiler.
%
% %%%%%%%%%%%%%%%%%%%%%%%%%%%%%%%%%%%%%%
% \paragraph{Main File.}
%
% The main file is called |cdocsamp.tex|.
%
% Load the \textsf{childdoc} definitions and
% declare the filename for the main document:
%    \begin{macrocode}
\input{childdoc.def}
\childdocmain{}
%    \end{macrocode}

% Optional override for |\version| flag:
%    \begin{macrocode}
%%\ifchilddoc\else\providecommand{\version}{draft}\fi
%    \end{macrocode}

% Define the default values for the |\version| flag
% (|final| for the main file and |draft| for childs):
%    \begin{macrocode}
\ifchilddoc
\providecommand{\version}{draft}
\else
\providecommand{\version}{final}
\fi
%    \end{macrocode}

% Load the standard document class:
%    \begin{macrocode}
\documentclass[12pt]{article}
%    \end{macrocode}

% Start the document body:
%    \begin{macrocode}
\begin{document}
%    \end{macrocode}

% Declare a title page.
% Print title, part of document being processed and version flag:
%    \begin{macrocode}
\addtocounter{page}{-1}
\begin{center}
{\LARGE\bfseries{}childdoc example\par}
\vspace{1cm}
\ifchilddoc
\ifchilddocmanual part\else chapter\fi:
`\childdocname' of `\childdocjob'\par
\else
main document: `\childdocjob'\par
\fi
version: \version\par
\end{center}
\newpage
%    \end{macrocode}

% Manually include selected file,
% otherwise process as usual:
%    \begin{macrocode}
\ifchilddocmanual
\section*{part `\childdocname'}
\input{\childdocname}
\else
%    \end{macrocode}

% Include the two chapters:
%    \begin{macrocode}
\include{cdocsch1}
\include{cdocsch2}
%    \end{macrocode}

% Include the two parts unless only chapters should be displayed:
%    \begin{macrocode}
\ifchilddoc\else
\section{part three}
\input{cdocspt3}
\section{part four}
\input{cdocspt4}
\fi
%    \end{macrocode}

% Process as usual until here:
%    \begin{macrocode}
\fi
%    \end{macrocode}

% End of document body:
%    \begin{macrocode}
\end{document}
%    \end{macrocode}
%\iffalse
%</samplemain>
%\fi
%
% %%%%%%%%%%%%%%%%%%%%%%%%%%%%%%%%%%%%%%
% \paragraph{Chapter Include Files.}
%
% The include files are called |cdocsch1.tex| and |cdocsch2.tex|.
%
%\iffalse
%<*samplechap1|samplechap2>
%\fi

% Optional override for |\version| flag:
%    \begin{macrocode}
%%\providecommand{\version}{final}
%    \end{macrocode}

% Include the main document:
%    \begin{macrocode}
\input{childdoc.def}
\childdocof{cdocsamp}
%    \end{macrocode}

%\iffalse
%</samplechap1|samplechap2>
%\fi
%
%\iffalse
%<*samplechap1>
%\fi
% Some text for chapter 1:
%    \begin{macrocode}
\section{one}
some text in chapter one
%    \end{macrocode}

%\iffalse
%</samplechap1>
%\fi
% Some text for chapter 2:
%\iffalse
%<*samplechap2>
%\fi
%    \begin{macrocode}
\section{two}
more text in chapter two
%    \end{macrocode}

%\iffalse
%</samplechap2>
%\fi
%
% %%%%%%%%%%%%%%%%%%%%%%%%%%%%%%%%%%%%%%
% \paragraph{Part Include Files.}
%
% The include files are called |cdocspt3.tex| and |cdocspt4.tex|.
%
%\iffalse
%<*samplepart3|samplepart4>
%\fi

% Optional override for |\version| flag:
%    \begin{macrocode}
%%\providecommand{\version}{final}
%    \end{macrocode}

% Include the main document:
%    \begin{macrocode}
\input{childdoc.def}
\childdocby{cdocsamp}
%    \end{macrocode}

%\iffalse
%</samplepart3|samplepart4>
%\fi
%
%\iffalse
%<*samplepart3>
%\fi
% Some text for part 3:
%    \begin{macrocode}
some text in part three
%    \end{macrocode}

%\iffalse
%</samplepart3>
%\fi
% Some text for part 4:
%\iffalse
%<*samplepart4>
%\fi
%    \begin{macrocode}
more text in part four
%    \end{macrocode}

%\iffalse
%</samplepart4>
%\fi
%
% %%%%%%%%%%%%%%%%%%%%%%%%%%%%%%%%%%%%%%
% \paragraph{Forwarding for a Complete Draft.}
%
% The following forwarding file |cdocsdrf.tex|
% compiles the main document in draft mode:
%\iffalse
%<*sampledraft>
%\fi
%    \begin{macrocode}
\def\version{draft}
\input{childdoc.def}
\childdocforward{cdocsamp}
%    \end{macrocode}

%\iffalse
%</sampledraft>
%\fi
%
% %%%%%%%%%%%%%%%%%%%%%%%%%%%%%%%%%%%%%%
% \paragraph{Forwarding for Final Version of the Chapters.}
%
% The following forwarding files |cdocsfn1.tex| and |cdocsfn2.tex|
% (with identical content)
% compile the final versions of the child documents
% |cdocsch1.tex| and |cdocsch2.tex|, respectively:
%\iffalse
%<*samplefinal>
%\fi
%    \begin{macrocode}
\def\version{final}
\input{childdoc.def}
\childdocforwardprefix[cdocsamp]{cdocsfn}{cdocsch}
%    \end{macrocode}

%\iffalse
%</samplefinal>
%\fi
%
% %%%%%%%%%%%%%%%%%%%%%%%%%%%%%%%%%%%%%%
% \paragraph{Command Line Processing.}
%
% The following three command lines generate the output files
% |cdocscld|, |cdocscl1| and |cdocscl2|
% which should be identical to
% |cdocsdrf|, |cdocsch1| and |cdocsfn2|, respectively:
% \begin{center}
% \begin{tabular}{l}
% |latex -jobname cdocscld \|\\
% |  "\def\version{draft}\input{childdoc.def}\childdocforward{cdocsamp}"|\\
% |latex -jobname cdocscl1 \|\\
% |  "\input{childdoc.def}\childdocforward[cdocsamp]{cdocsch1}"|\\
% |latex -jobname cdocscl2 \|\\
% |  "\def\version{final}\input{childdoc.def}\childdocforward{cdocsch2}"|
% \end{tabular}
% \end{center}
% Note that the trailing backslash on each first line
% merely continues the input to the second line
% (for convenient cut ant paste).
% Furthermore, the command |latex| can be replaced by any
% of its alternative versions such as |pdflatex|.
%
% %%%%%%%%%%%%%%%%%%%%%%%%%%%%%%%%%%%%%%%%%%%%%%%%%%%%%%%%%%%%%%%%%%%%%%%%%%%%%%
% %%%%%%%%%%%%%%%%%%%%%%%%%%%%%%%%%%%%%%%%%%%%%%%%%%%%%%%%%%%%%%%%%%%%%%%%%%%%%%
% \section{Implementation}
%\iffalse
%<*package>
%\fi
%
% This section describes the definitions file |childdoc.def|.

% The definitions cannot be loaded using |\usepackage| or |\RequirePackage|
% which has a mechanism to prevent loading a style file more than once.
% When loading the definitions by means of |\input|
% multiple instances have to be prevented manually:
%\iffalse
%This code needs to be before the `\ProvidesFile' directive
%which is defined at the beginning of this file.
%Therefore it is also placed there and commented out here.
%</package>
%<*discard>
%\fi
%    \begin{macrocode}
\ifdefined\childdocmain\endinput\fi
%    \end{macrocode}
%\iffalse
%</discard>
%<*package>
%\fi
%
% \macro{\ifchilddoc}
% \macro{\ifchilddocmanual}
% The conditional |\ifchilddoc| tells whether a
% child (true) or main (false) document is being compiled.
% The conditional |\ifchilddocmanual| tells whether
% the |\includeonly| mechanism is used (false) or
% the selection of child files must be performed manually (true).
% The definitions initialise to false:
%    \begin{macrocode}
\newif\ifchilddoc
\newif\ifchilddocmanual
%    \end{macrocode}

% \macro{\childdocname}
% \macro{\childdocjob}
% The macro |\childdocname| stores the name of the main document
% to be compiled. The macro |\childdocjob| stores the name of
% the document on which the \LaTeX{} compiler was originally invoked.
% The content of |\jobname| cannot be compared
% to filenames specified in the source due to different catcodes.
% The following code rescans |\jobname|, stores the result
% in |\childdocname| and saves a copy in |\childdocjob|:
%    \begin{macrocode}
\edef\childdocname{\scantokens\expandafter{\jobname\noexpand}}
\let\childdocjob\childdocname
%    \end{macrocode}

% \macro{\childdocdisable}
% The macro |\childdocdisable| prevents the main file
% from being processed more than once.
% At this stage, the main document command |\childdocmain|
% is assumed to be called once again where it should do nothing.
% Any subsequent call to it should prevent
% a secondary processing of the main document
% It overwrites the forwarding commands
% |\childdocof| and |\childdocforward|
% with empty macros to prevent further inclusions of the main document:
%    \begin{macrocode}
\newcommand{\childdocdisable}
{
  \renewcommand{\childdocmain}[1]{\renewcommand{\childdocmain}[1]{\endinput}}
  \renewcommand{\childdocof}[1]{}
  \renewcommand{\childdocby}[2][]{}
  \renewcommand{\childdocforward}[2][]{}
  \renewcommand{\childdocdisable}{}
}
%    \end{macrocode}

% \macro{\childdocmain}
% The macro |\childdocmain| is to be called at the top of the main file
% with nothing or the main filename (without extension) as argument.
% First, it breaks loops.
% If the argument is not empty and does not match |\childdocname|
% (which is set by the first inclusion of |childdoc.def|),
% |\ifchilddoc| is set to true, |\includeonly| is applied to the child file
% and |\jobname| is set to the main file
% (for proper handling of |.aux| files):
%    \begin{macrocode}
\newcommand{\childdocmain}[1]
{
  \childdocdisable\childdocmain{}
  \if?#1?\else
    \begingroup
      \def\childdoctmp{#1}
      \ifx\childdoctmp\childdocname
        \def\childdoctmp{}
      \else
        \def\childdoctmp
        {
          \childdoctrue
          \includeonly{\childdocname}
          \def\childdocjob{#1}
          \def\jobname{#1}
        }
      \fi
      \expandafter
    \endgroup
    \childdoctmp
  \fi
}
%    \end{macrocode}

% \macro{\childdocof}
% The command |\childdocof| redirects
% compilation to the main file |#1|.
%    \begin{macrocode}
\newcommand{\childdocof}[1]
{
  \childdocdisable
  \childdoctrue
  \includeonly{\childdocname}
  \def\jobname{#1}
  \def\childdocjob{#1}
  \input{#1}
}
%    \end{macrocode}

% \macro{\childdocby}
% The command |\childdocby| ....
%    \begin{macrocode}
\newcommand{\childdocby}[2][]
{
  \childdocdisable
  \childdoctrue
  \childdocmanualtrue
  \if?#1?\else
    \def\jobname{#2}
  \fi
  \def\childdocjob{#2}
  \input{#2}
  \endinput
}
%    \end{macrocode}

% \macro{\childdocforward}
% The command |\childdocforward| redirects
% compilation to the main file or
% (if the optional argument is given) a child file.
% Parameters are set as if the main file
% or a child file starting with |\childdocof| was compiled.
% Then compilation is handed over to the main file:
%    \begin{macrocode}
\newcommand{\childdocforward}[2][]
{
  \begingroup
    \if?#1?
      \def\childdoctmp
      {
        \def\childdocname{#2}
        \def\childdocjob{#2}
        \def\jobname{#2}
        \input{#2}
        \endinput
      }
    \else
      \def\childdoctmp
      {
        \childdocdisable
        \def\childdocname{#2}
        \childdoctrue
        \includeonly{#2}
        \def\childdocjob{#1}
        \def\jobname{#1}
        \input{#1}
        \endinput
      }
    \fi
    \expandafter
  \endgroup
  \childdoctmp
}
%    \end{macrocode}

% \macro{\childdocforwardprefix}
% The command |\childdocforwardprefix| redirects
% compilation to the main or a child file by means of a pattern.
% The prefix |#1| in the current filename is replaced by |#2|
% and the suffix of the current filename is kept
% (it is assumed that the filename does not contain the substring `|~~~|'
% which is used as a delimiter).
% Compilation is handed over to the new file by |\childdocforward|:
%    \begin{macrocode}
\newcommand{\childdocforwardprefix}[3][]
{
  \begingroup
    \def\childdocextract #2##1~~~{\def\childdoctmp{\childdocforward[#1]{#3##1}}}
    \expandafter\childdocextract\childdocname~~~
    \expandafter
  \endgroup
  \childdoctmp
}
%    \end{macrocode}

% \macro{\childdoc}
% The deprecated macro |\childdoc| is a legacy version of |\childdocmain|:
%    \begin{macrocode}
\newcommand{\childdoc}{\childdocmain}
%    \end{macrocode}

% \macro{\childdocredirect}
% The deprecated macro |\childdocredirect| is a legacy version
% of |\childdocforward| and |\childdocforwardprefix|:
%    \begin{macrocode}
\newcommand{\childdocredirect}[2][]
{
  \begingroup
    \if?#1?
      \def\childdoctmp{\childdocforward{#2}}
    \else
      \def\childdoctmp{\childdocforwardprefix{#1}{#2}}
    \fi
    \expandafter
  \endgroup
  \childdoctmp
}
%    \end{macrocode}

%\iffalse
%</package>
%\fi
%
\endinput
\childdocforward[|\textit{main}|]{|\textit{dest}|}"|
\end{center}
%
Here \textit{target} is the name of the output file,
\textit{main} is the name of the main file
and \textit{dest} is the name of the main or child file to be processed
(all filenames without extensions).
The optional argument \textit{main} can be omitted
if \textit{main} matches \textit{dest}.
Optionally, compilation \textit{flags} can be defined via |\def| commands.
This command line makes the \TeX{} engine believe
it is compiling the file \textit{target}
whose content is specified as the latter parameter.
The provided code then forwards the processing to
\textit{main} or \textit{dest} as described in \secref{sec:forward}.

%%%%%%%%%%%%%%%%%%%%%%%%%%%%%%%%%%%%%%%%%%%%%%%%%%%%%%%%%%%%%%%%%%%%%%%%%%%%%%%%
\subsection{Include by Input}
\label{sec:input}

Including child documents by |\include| has some restrictions by design.
Most notably, the content of a child document always occupies
its own set of pages; pages cannot be shared between child documents.
Usually, this behaviour makes perfect sense
because each child document contain an essential part of the document.
However, in some situations it may be desirable to compose
a document from a collection of parts
without having mandatory page breaks between then.
For this case, the package
provides a mechanism to include parts
by |\input| which can also be processed individually.
However, by construction this mechanism
requires manual handling of the content to be output.

%%%%%%%%%%%%%%%%%%%%%%%%%%%%%%%%%%%%%%%%
\DescribeMacro{\ifchilddocmanual}
The main file should be prepared as usual, see \secref{sec:include}.
However, the document body must make a distinction
between processing of an individual part and of the main document, e.g.:
%
\begin{center}
\begin{tabular}{l}
|\ifchilddocmanual|\\
|\input{\childdocname}|\\
|\||else|\\
\textit{document body with }|\input{|\textit{part}|}|\\
|\||fi|
\end{tabular}
\end{center}
%
The conditional |\ifchilddocmanual| is true whenever
a part to be included by |\input| is being compiled,
and the name of the part is stored in |\childdocname|.

%%%%%%%%%%%%%%%%%%%%%%%%%%%%%%%%%%%%%%%%
\DescribeMacro{\childdocby}
Each part to be included by |\input| should start with:
%
\begin{center}
\begin{tabular}{l}
|% \iffalse
%
% childdoc.dtx Copyright (C) 2017-2018 Niklas Beisert
%
% This work may be distributed and/or modified under the
% conditions of the LaTeX Project Public License, either version 1.3
% of this license or (at your option) any later version.
% The latest version of this license is in
%   http://www.latex-project.org/lppl.txt
% and version 1.3 or later is part of all distributions of LaTeX
% version 2005/12/01 or later.
%
% This work has the LPPL maintenance status `maintained'.
%
% The Current Maintainer of this work is Niklas Beisert.
%
% This work consists of the files childdoc.dtx and childdoc.ins
% and the derived files childdoc.def and cdocsamp.tex with
% cdocsch1.tex, cdocsch2.tex, cdocsdrf.tex, cdocsfn1.tex, cdocsfn2.tex.
%
%<package>\ifdefined\childdocmain\endinput\fi
%<package>\ProvidesFile{childdoc.def}[2018/12/30 v2.0 child document driver]
%<samplemain>\ProvidesFile{cdocsamp.tex}[2018/12/30 v2.0 sample for childdoc]
%<*driver>
%\ProvidesFile{childdoc.drv}[2018/12/30 v2.0 childdoc reference manual file]
\PassOptionsToClass{10pt,a4paper}{article}
\documentclass{ltxdoc}

\usepackage[margin=35mm]{geometry}
\usepackage{hyperref}
\usepackage{hyperxmp}
\usepackage[usenames]{color}

\hypersetup{colorlinks=true}
\hypersetup{pdfstartview=FitH}
\hypersetup{pdfpagemode=UseNone}
\hypersetup{pdfsource={}}
\hypersetup{pdflang={en-UK}}
\hypersetup{pdfcopyright={Copyright 2017-2018 Niklas Beisert.
  This work may be distributed and/or modified under the
  conditions of the LaTeX Project Public License, either version 1.3
  of this license or (at your option) any later version.}}
\hypersetup{pdflicenseurl={http://www.latex-project.org/lppl.txt}}
\hypersetup{pdfcontactaddress={ETH Zurich, ITP, HIT K,
  Wolfgang-Pauli-Strasse 27}}
\hypersetup{pdfcontactpostcode={8093}}
\hypersetup{pdfcontactcity={Zurich}}
\hypersetup{pdfcontactcountry={Switzerland}}
\hypersetup{pdfcontactemail={nbeisert@itp.phys.ethz.ch}}
\hypersetup{pdfcontacturl={http://people.phys.ethz.ch/\xmptilde nbeisert/}}

\newcommand{\secref}[1]{\hyperref[#1]{section \ref*{#1}}}

\parskip1ex
\parindent0pt
\let\olditemize\itemize
\def\itemize{\olditemize\parskip0pt}

\begin{document}

\title{The \textsf{childdoc} Package}
\hypersetup{pdftitle={The childdoc Package}}
\author{Niklas Beisert\\[2ex]
  Institut f\"ur Theoretische Physik\\
  Eidgen\"ossische Technische Hochschule Z\"urich\\
  Wolfgang-Pauli-Strasse 27, 8093 Z\"urich, Switzerland\\[1ex]
  \href{mailto:nbeisert@itp.phys.ethz.ch}
  {\texttt{nbeisert@itp.phys.ethz.ch}}}
\hypersetup{pdfauthor={Niklas Beisert}}
\hypersetup{pdfsubject={Manual for the LaTeX2e Package childdoc}}
\date{30 December 2018, \textsf{v2.0}}
\maketitle

\begin{abstract}\noindent
\textsf{childdoc} is a \LaTeXe{} package
that enables the direct compilation
of document sections included by |\include|
to individual files.
\end{abstract}

\begingroup
\parskip0ex
\tableofcontents
\endgroup

%%%%%%%%%%%%%%%%%%%%%%%%%%%%%%%%%%%%%%%%%%%%%%%%%%%%%%%%%%%%%%%%%%%%%%%%%%%%%%%%
%%%%%%%%%%%%%%%%%%%%%%%%%%%%%%%%%%%%%%%%%%%%%%%%%%%%%%%%%%%%%%%%%%%%%%%%%%%%%%%%
\section{Introduction}

\LaTeX{} provides a mechanism to structure a large document (such as a book)
into a main file and several child files (containing the chapters)
using the |\include| command.
This mechanism is beneficial for documents
which span hundreds of pages in order to
make the source file(s) more manageable.
Moreover, compilation can be restricted to
selected child files by means of the |\includeonly| command.
The latter feature can be used to reduce the compilation time while editing
(this was significantly more useful in the earlier days of \LaTeX{})
or to generate a smaller document which is easier to navigate.
Another application of |\includeonly| is to generate
documents consisting of selected parts of the complete document.

However, there are a few drawbacks of the plain |\include| mechanism:
\begin{itemize}
\item
The child files cannot be compiled on their own,
they can only be compiled via the main file.
A naive editing environment
(such as a text editor with an option
to have the current file processed by \LaTeX)
may require one to switch to the main file before compiling;
attempting to compile the child file produces errors.
\item
The main file must be modified (each time)
to adjust the |\includeonly| command
to the present needs. This easily leaves the main file in a messy state.
\item
The generated document will always carry the filename
of the main document. This is inconvenient if
several child files are to be compiled and
to be kept for distribution.
\end{itemize}

The present package provides a simple interface
to make child files individually compilable by \LaTeX{}.
Compiling a child file then has the same effect as compiling
the main file with an |\includeonly| command
to select the appropriate child.
Moreover the generated document will carry the name of the child
rather than the main file.
This resolves all three above issues.

This feature is meant to make the editing of books,
thesis documents and lecture notes somewhat more convenient.
However, the package can also be used efficiently for
composing a series of documents (such as exercise sheets)
which are typically distributed individually.
It then assists the author in generating the individual documents
(potentially in different versions)
as well as a document containing the collected series.
Another application is in developing style files
or other kinds of included material
where compilation of the style file could redirect
to a sample or test file.

%%%%%%%%%%%%%%%%%%%%%%%%%%%%%%%%%%%%%%%%%%%%%%%%%%%%%%%%%%%%%%%%%%%%%%%%%%%%%%%%
%%%%%%%%%%%%%%%%%%%%%%%%%%%%%%%%%%%%%%%%%%%%%%%%%%%%%%%%%%%%%%%%%%%%%%%%%%%%%%%%
\section{Usage}

First of all, the package \textsf{childdoc} is \emph{not} a standard
\LaTeXe{} |.sty| style file! Therefore it needs to be invoked in
a non-standard way.

%%%%%%%%%%%%%%%%%%%%%%%%%%%%%%%%%%%%%%%%%%%%%%%%%%%%%%%%%%%%%%%%%%%%%%%%%%%%%%%%
\subsection{Included Files}
\label{sec:include}

%%%%%%%%%%%%%%%%%%%%%%%%%%%%%%%%%%%%%%%%
\DescribeMacro{\childdocmain}
To use the package, add the commands
\begin{center}
\begin{tabular}{l}
|\input{childdoc.def}|\\
|\childdocmain{}|\\
\end{tabular}
\end{center}
at the very top of the main \LaTeX{} file,
in particular \emph{before} the |\documentclass| statement!
The argument of |\childdocmain| should be left empty
(but it must be present).

%%%%%%%%%%%%%%%%%%%%%%%%%%%%%%%%%%%%%%%%
\DescribeMacro{\childdocof}
Furthermore, add the commands
\begin{center}
\begin{tabular}{l}
|\input{childdoc.def}|\\
|\childdocof{|\textit{main}|}|\\
\end{tabular}
\end{center}
at the top of every child file \textit{child}
which is included by |\include{|\textit{child}|}|
from within the main file
(or at least for those files to be compiled individually).
The argument \textit{main} must be the filename of the main file.

There are a couple of
considerations in setting up the main and child documents:

%%%%%%%%%%%%%%%%%%%%%%%%%%%%%%%%%%%%%%%%
\paragraph{Restrictions.}

Please note the following restrictions:
\begin{itemize}
\item
|\childdocmain| must be called with one argument \textit{main}
to ensure compatibility with earlier version of the package.
It must either be empty (|\childdocmain{}|)
or precisely match the filename of the main file in which it is specified.
See \secref{sec:detection} for further information.
\item
The filename \textit{main} must be specified without the |.tex| extension.
\item
The filename \textit{main} is case sensitive
(even in case-insensitive file systems)
due to internal string comparison.
\item
The argument \textit{main} should be fully expanded, it cannot be a macro.
\item
Subdirectories and special characters should be avoided in filenames.
\item
The command |\childdocmain{|\textit{main}|}| must be followed by a whitespace.
It should not be followed immediately by another command
or by a comment mark `|%|'.
This is because the \TeX{} parser reads the token immediately following
the argument of |\childdocmain| and puts it
at the beginning of every child section;
however, a white\-space is ignored.
\end{itemize}

%%%%%%%%%%%%%%%%%%%%%%%%%%%%%%%%%%%%%%%%
\paragraph{Content of Main File.}

It is advisable to place all content in the child files included by |\include|.
Any output contained in the main file will appear in all child documents
unless suppressed manually;
it cannot be suppressed automatically by the |\includeonly| directive
and thus should normally be avoided.
A method to include some content in the main file
by means of conditional processing is described in \secref{sec:conditional}.

%%%%%%%%%%%%%%%%%%%%%%%%%%%%%%%%%%%%%%%%
\paragraph{Page Numbering.}

When only a part of the document is compiled,
the appropriate numbering of pages
(as well as other status parameters)
is determined from the |.aux| files.
The latter contain information from previous passes.
However this information needs to propagate through
all intermediate child documents.
Therefore the page numbering in child documents may well
be inconsistent until the complete document is compiled at least once.

A useful (if unconventional) way to always ensure a consistent
page numbering is to restart the numbering in each child document
and denote the pages by `\textit{child}|.|\textit{page}'
where \textit{child} represents the chapter/section number of the child file.
This can be achieved by the command
|\numberwithin{page}{|\textit{child}|}|
of the \textsf{amsmath} package
where \textit{child} can be |chapter| or |section|
depending on the chosen structuring.
Alternatively, one can modify the macro |\thepage| appropriately
and reset the counter |page| at the start of each child file.

%%%%%%%%%%%%%%%%%%%%%%%%%%%%%%%%%%%%%%%%%%%%%%%%%%%%%%%%%%%%%%%%%%%%%%%%%%%%%%%%
\subsection{Conditional Processing}
\label{sec:conditional}

The package provides a mechanism to compile different versions
of a document. To customise the versions further some conditional processing
can come in handy to distinguish which version is being compiled.
The package provides two macros to describe the compilation context:

%%%%%%%%%%%%%%%%%%%%%%%%%%%%%%%%%%%%%%%%
\DescribeMacro{\ifchilddoc}
The conditional |\ifchilddoc| distinguishes between the compilation of
child documents and the main document:
%
\begin{center}
|\ifchilddoc |\textit{child-code}| |[|\||else |\textit{main-code}]| \||fi|
\end{center}

%%%%%%%%%%%%%%%%%%%%%%%%%%%%%%%%%%%%%%%%
\DescribeMacro{\childdocname}
\DescribeMacro{\childdocjob}
The macro |\childdocname| contains the filename (without extension)
of the main or child file being processed.
Note that |\childdocjob| will always contain the name of the main file.

%%%%%%%%%%%%%%%%%%%%%%%%%%%%%%%%%%%%%%%%
\paragraph{Title Page.}

Conditional processing can be used to include a title or banner page
in the main document when proper precautions are taken.
Importantly, the code in the main file should ensure that the page counter
(as well as other status parameters which are stored in the |.aux| files)
takes the same value after the conditional processing.
Otherwise the page numbers may take divergent values
depending on which part is compiled.

For example, a title page could be declared by:
%
\begin{center}
\begin{tabular}{l}
|\ifchilddoc\||else|\\
|\addtocounter{page}{-1}|\\
\textit{code for title page}\\
|\newpage|\\
|\||fi|
\end{tabular}
\end{center}
%
A banner page for the child documents can be generated by:
%
\begin{center}
\begin{tabular}{l}
|\ifchilddoc|\\
|\addtocounter{page}{-1}|\\
\textit{code for banner page}\\
|\newpage|\\
|\||fi|
\end{tabular}
\end{center}
%
Here one could write a message such as:
\begin{center}
|This is the part \childdocname{} of \childdocjob{}.|
\end{center}

%%%%%%%%%%%%%%%%%%%%%%%%%%%%%%%%%%%%%%%%%%%%%%%%%%%%%%%%%%%%%%%%%%%%%%%%%%%%%%%%
\subsection{Flags}
\label{sec:flags}

The package makes it easy to generate different versions
of the main or child documents.
To this end compilation flags can be defined
and assigned different default values.
They will be particularly useful in conjunction
with the forwarding mechanism described in \secref{sec:forward}.

For example, it may be useful to have a flag |\version|
which can be set to |draft| or |final|.
The document source will contain some conditional code
depending on the value of |\version|.
Suppose further, the flag should default to |final| for the main file
and to |draft| for child files
which is a natural assignment for editing the document.
This is achieved by placing the following code
in the preamble of the main document
(below the |\childdocmain| directive):
%
\begin{center}
\begin{tabular}{l}
|\ifchilddoc|\\
|\providecommand{\version}{draft}|\\
|\||else|\\
|\providecommand{\version}{final}|\\
|\||fi|
\end{tabular}
\end{center}
%
The definition by |\providecommand| makes sure
that previous definitions are not overwritten.
Further statements |\providecommand{\version}{...}|
can thus be added before the above code to override it.

For the main file, one might add a line
(between |\childdocmain| and the above block)
%
\begin{center}
|%\ifchilddoc\||else\providecommand{\version}{draft}\||fi|
\end{center}
%
which can be uncommented to produce a draft version.
Likewise one can add a line to the very top of a child file
(above the |\childdocof{|\textit{main}|}| directive)
%
\begin{center}
|%\providecommand{\version}{final}|
\end{center}
%
which can be uncommented to produce the final version of this child document.

%%%%%%%%%%%%%%%%%%%%%%%%%%%%%%%%%%%%%%%%%%%%%%%%%%%%%%%%%%%%%%%%%%%%%%%%%%%%%%%%
\subsection{Forwarding}
\label{sec:forward}

Different versions of the main or child documents
using compilation flags as described in \secref{sec:flags}
can be (permanently) stored in different files
for convenient compilation, viewing and distribution.
To this end, the package defines a command
to pass on compilation to a different file:

%%%%%%%%%%%%%%%%%%%%%%%%%%%%%%%%%%%%%%%%
\DescribeMacro{\childdocforward}
The command |\childdocforward| redirects processing to
another source file:
%
\begin{center}
\begin{tabular}{l}
|\input{childdoc.def}|\\
|\childdocforward[|\textit{main}|]{|\textit{dest}|}|\\
\end{tabular}
\end{center}
%
The argument \textit{dest} is the destination file
(without extension).
It should be the main file or one of the child files.
Note that further \textsf{childdoc} directives
such as |\childdocof| and |\childdocforward|
in the indicated file will be processed in this form.
The optional argument \textit{main}
passes on directly to the main file \textit{main}
while pretending to compile the child \textit{dest}.
This form behaves as if \textit{dest}
issues |\childdocof{|\textit{main}|}| right away,
and no further \textsf{childdoc} directives will be processed.

%%%%%%%%%%%%%%%%%%%%%%%%%%%%%%%%%%%%%%%%
\DescribeMacro{\...prefix}
In the alternative form |\childdocforwardprefix|,
%
\begin{center}
\begin{tabular}{l}
|\input{childdoc.def}|\\
|\childdocforwardprefix[|\textit{main}|]{|\textit{prefix}|}{|\textit{dest}|}|
\end{tabular}
\end{center}
%
the destination file is determined by a pattern
depending on the current file:
To make this work, the current file must be called
`{\textit{prefix}\hspace{0.2em}\textit{suffix}}'
with \textit{prefix} matching precisely the argument.
Processing is then passed on to the file
`{\textit{dest}\hspace{0.2em}\textit{suffix}}'.
Surely, the same effect is achieved by
directly specifying the
argument `{\textit{dest}\hspace{0.2em}\textit{suffix}}'
in the first form.
However, that requires to set up a different file
for each child. With the alternative form of the command
all these files can have exactly the same content
which simplifies setting them up and maintaining them.

For example, the following file |draft.tex|
with a compilation flag |\version| as described in \secref{sec:flags}
compiles the main document as a draft:
%
\begin{center}
\begin{tabular}{l}
|\def\version{draft}|\\
|\input{childdoc.def}|\\
|\childdocforward{|\textit{main}|}|
\end{tabular}
\end{center}
%
Likewise, the following files |final|\textit{nn}|.tex|
compile the final version of the child document
|child|\textit{nn}|.tex|:
%
\begin{center}
\begin{tabular}{l}
|\def\version{final}|\\
|\input{childdoc.def}|\\
|\childdocforwardprefix{final}{child}|
\end{tabular}
\end{center}
%

Note that when several versions of a main file and/or of each child file
are to be generated, it may be convenient to set up a |Makefile| or
shell script to automatise the process.

%%%%%%%%%%%%%%%%%%%%%%%%%%%%%%%%%%%%%%%%%%%%%%%%%%%%%%%%%%%%%%%%%%%%%%%%%%%%%%%%
\subsection{Command Line Processing}
\label{sec:commandline}

The effect of redirection files can also be achieved by invoking
the \LaTeX{} compiler with a more elaborate command line.
Most conveniently this should be done as part
of a shell script or a |Makefile|.

When using \textsf{childdoc} in the main file, the following
command lines effectively perform a redirection
(note that depending on the shell being used,
backslashes may have to be doubled: `|\|' $\to$ `|\\|'):
%
\begin{center}
|... -jobname "|\textit{target}|" |\\|"|[\textit{flags}]%
|\input{childdoc.def}\childdocforward[|\textit{main}|]{|\textit{dest}|}"|
\end{center}
%
Here \textit{target} is the name of the output file,
\textit{main} is the name of the main file
and \textit{dest} is the name of the main or child file to be processed
(all filenames without extensions).
The optional argument \textit{main} can be omitted
if \textit{main} matches \textit{dest}.
Optionally, compilation \textit{flags} can be defined via |\def| commands.
This command line makes the \TeX{} engine believe
it is compiling the file \textit{target}
whose content is specified as the latter parameter.
The provided code then forwards the processing to
\textit{main} or \textit{dest} as described in \secref{sec:forward}.

%%%%%%%%%%%%%%%%%%%%%%%%%%%%%%%%%%%%%%%%%%%%%%%%%%%%%%%%%%%%%%%%%%%%%%%%%%%%%%%%
\subsection{Include by Input}
\label{sec:input}

Including child documents by |\include| has some restrictions by design.
Most notably, the content of a child document always occupies
its own set of pages; pages cannot be shared between child documents.
Usually, this behaviour makes perfect sense
because each child document contain an essential part of the document.
However, in some situations it may be desirable to compose
a document from a collection of parts
without having mandatory page breaks between then.
For this case, the package
provides a mechanism to include parts
by |\input| which can also be processed individually.
However, by construction this mechanism
requires manual handling of the content to be output.

%%%%%%%%%%%%%%%%%%%%%%%%%%%%%%%%%%%%%%%%
\DescribeMacro{\ifchilddocmanual}
The main file should be prepared as usual, see \secref{sec:include}.
However, the document body must make a distinction
between processing of an individual part and of the main document, e.g.:
%
\begin{center}
\begin{tabular}{l}
|\ifchilddocmanual|\\
|\input{\childdocname}|\\
|\||else|\\
\textit{document body with }|\input{|\textit{part}|}|\\
|\||fi|
\end{tabular}
\end{center}
%
The conditional |\ifchilddocmanual| is true whenever
a part to be included by |\input| is being compiled,
and the name of the part is stored in |\childdocname|.

%%%%%%%%%%%%%%%%%%%%%%%%%%%%%%%%%%%%%%%%
\DescribeMacro{\childdocby}
Each part to be included by |\input| should start with:
%
\begin{center}
\begin{tabular}{l}
|\input{childdoc.def}|\\
|\childdocby{|\textit{main}|}|\\
\end{tabular}
\end{center}
%
The directive |\childdocby| is similar to |\childdocof|
described in \secref{sec:include},
but the subsequent selection of content must be done manually.
To that end, both |\ifchilddoc| and |\ifchilddocmanual|
will be true upon processing of a part,
and the name of the part is stored in |\childdocname|.
Note that |\jobname| will be set to the filename of the current part
so that each part receives an individual |.aux| file
that does not interfere with the |.aux| file(s) of the main document.
This behaviour can be altered by the alternative form
|\childdocby[*]{|\textit{main}|}| (with a non-empty optional argument)
which uses the |.aux| file of the main document
by setting |\jobname| to \textit{main}.

%%%%%%%%%%%%%%%%%%%%%%%%%%%%%%%%%%%%%%%%%%%%%%%%%%%%%%%%%%%%%%%%%%%%%%%%%%%%%%%%
\subsection{Driver Development}
\label{sec:driver}

The \textsf{childdoc} mechanism can also be use for the development
of definition files such as \LaTeX{} styles or classes.
This case differs from the above setup with multiple parts
included by |\include| in that no |\includeonly| should be invoked.
This can be achieved by starting the include file
(before |\ProvidesPackage|) with:
%
\begin{center}
\begin{tabular}{l}
|\input{childdoc.def}|\\
|\childdocforward{|\textit{main}|}|\\
\end{tabular}
\end{center}
%
or alternatively with:
%
\begin{center}
\begin{tabular}{l}
|\input{childdoc.def}|\\
|\childdocby{|\textit{main}|}|\\
\end{tabular}
\end{center}
%
Both forms have slightly different effects as described above.
The main file is prepared as usual, see \secref{sec:include}.

%%%%%%%%%%%%%%%%%%%%%%%%%%%%%%%%%%%%%%%%%%%%%%%%%%%%%%%%%%%%%%%%%%%%%%%%%%%%%%%%
\subsection{Legacy Detection}
\label{sec:detection}

The directive |\childdocmain| in the main file can detect
whether the complete document or merely a child is to be compiled
even without using the directive |\childdocof|.
This method is deprecated because it is less robust
and there is no compelling reason to use it;
it is merely provided for backward compatibility
and it may be removed in future versions.

If the detection mechanism is to be used,
it is mandatory to correctly specify
the filename of the main file as the argument of |\childdocmain|:
%
\begin{center}
\begin{tabular}{l}
|\input{childdoc.def}|\\
|\childdocmain{|\textit{main}|}|\\
\end{tabular}
\end{center}
%
If |\jobname| does not match the argument \textit{main} of |\childdocmain|,
it is assumed that |\jobname| points to the child file to be compiled.
When using |\childdocmain| with the main file specified as argument,
it suffices to start a child file
with just |\input{|\textit{main}|}|
without loading of the package and using |\childdocof|.
If instead all processing is done
with the appropriate \textsf{childdoc} directives,
the argument of \textit{main} of |\childdocmain| can be empty.

An alternative version of the command line processing described
in \secref{sec:commandline} using the detection mechanism reads:
%
\begin{center}
|... -jobname "|\textit{target}|" "|[\textit{flags}]%
[|\def\jobname{|\textit{dest}|}|]|\input{|\textit{main}|}"|
\end{center}

%%%%%%%%%%%%%%%%%%%%%%%%%%%%%%%%%%%%%%%%%%%%%%%%%%%%%%%%%%%%%%%%%%%%%%%%%%%%%%%%
\subsection{Manual Code}
\label{sec:manual}

In case one cannot be certain whether the definitions file |childdoc.def|
is installed on the target \TeX{} distribution
and one prefers not to ship it,
it is conceivable to paste a few relevant commands into the sources.

To that end, drop all statements |\input{childdoc.def}|
and perform the replacements as outlined below.
Instead of |\childdocmain{|\textit{main}|}| add the following code
to the top of the main file:
%
\begin{center}
\begin{tabular}{l}
|\||ifdefined\childdocname\endinput\||fi\newif\ifchilddoc|\\
|\edef\childdocname{\scantokens\expandafter{\jobname\noexpand}}|\\
|\def\childdocmain{|\textit{main}|}\||ifx\childdocmain\childdocname\||else|\\
|\childdoctrue\includeonly{\childdocname}\let\jobname\childdocmain\||fi|\\
\end{tabular}
\end{center}
%
Instead of |\childdocof{|\textit{main}|}| just include the main file
at the top of each child file:
%
\begin{center}
|\input{|\textit{main}|}|
\end{center}
%
A simple redirection |\childdocforward{|\textit{dest}|}| is achieved by:
%
\begin{center}
|\def\jobname{|\textit{dest}|}\input{\jobname}|
\end{center}
%
The redirection with prefix
|\childdocforwardprefix[|\textit{prefix}|]{|\textit{dest}|}|
is accomplished by:
%
\begin{center}
\begin{tabular}{l}
|{\edef\jobname{\scantokens\expandafter{\jobname\noexpand}}|\\
|\def\redirectjob |\textit{prefix}|#1~~~{\gdef\jobname{|\textit{dest}|#1}}|\\
|\expandafter\redirectjob\jobname~~~}\input{\jobname}|
\end{tabular}
\end{center}

In an alternative approach,
child documents can be compiled by a specific command line
without additional code or specific definitions:
%
\begin{center}
|... -jobname "|\textit{target}|" "|[\textit{flags}]%
|\includeonly{|\textit{dest}|}\input{|\textit{main}|}"|
\end{center}
%

%%%%%%%%%%%%%%%%%%%%%%%%%%%%%%%%%%%%%%%%%%%%%%%%%%%%%%%%%%%%%%%%%%%%%%%%%%%%%%%%
%%%%%%%%%%%%%%%%%%%%%%%%%%%%%%%%%%%%%%%%%%%%%%%%%%%%%%%%%%%%%%%%%%%%%%%%%%%%%%%%
\section{Information}

%%%%%%%%%%%%%%%%%%%%%%%%%%%%%%%%%%%%%%%%%%%%%%%%%%%%%%%%%%%%%%%%%%%%%%%%%%%%%%%%
\subsection{Copyright}

Copyright \copyright{} 2017--2018 Niklas Beisert

This work may be distributed and/or modified under the
conditions of the \LaTeX{} Project Public License, either version 1.3
of this license or (at your option) any later version.
The latest version of this license is in
  \url{http://www.latex-project.org/lppl.txt}
and version 1.3 or later is part of all distributions of \LaTeX{}
version 2005/12/01 or later.

This work has the LPPL maintenance status `maintained'.

The Current Maintainer of this work is Niklas Beisert.

This work consists of the files |README.txt|, |childdoc.ins| and |childdoc.dtx|
as well as the derived files |childdoc.def|, |cdocsamp.tex|
with |cdocsch1.tex|, |cdocsch2.tex|, |cdocspt3.tex|, |cdocspt4.tex|,
|cdocsdrf.tex|, |cdocsfn1.tex|, |cdocsfn2.tex|
as well as |childdoc.pdf|.

%%%%%%%%%%%%%%%%%%%%%%%%%%%%%%%%%%%%%%%%%%%%%%%%%%%%%%%%%%%%%%%%%%%%%%%%%%%%%%%%
\subsection{Files and Installation}

The package consists of the files:
%
\begin{center}
\begin{tabular}{ll}
    |README.txt|   & readme file \\
    |childdoc.ins| & installation file \\
    |childdoc.dtx| & source file \\
    |childdoc.def| & definition file \\
    |cdocsamp.tex| & sample main file \\
    |cdocsch1.tex| & sample include file \\
    |cdocsch2.tex| & sample include file \\
    |cdocspt3.tex| & sample part file \\
    |cdocspt4.tex| & sample part file \\
    |cdocsdrf.tex| & sample redirection file \\
    |cdocsfn1.tex| & sample redirection file \\
    |cdocsfn2.tex| & sample redirection file \\
    |childdoc.pdf| & manual
\end{tabular}
\end{center}
%
The distribution consists of the files
|README.txt|, |childdoc.ins| and |childdoc.dtx|.
%
\begin{itemize}
\item
Run (pdf)\LaTeX{} on |childdoc.dtx|
to compile the manual |childdoc.pdf| (this file).
\item
Run \LaTeX{} on |childdoc.ins| to create the definitions file |childdoc.def|
and the sample |cdocsamp.tex| with include files
|cdocsch1.tex|, |cdocsch2.tex|, |cdocspt3.tex|, |cdocspt4.tex|,
|cdocsdrf.tex|, |cdocsfn1.tex|, |cdocsfn2.tex|.
Then copy the file |childdoc.def| to an appropriate directory of your \LaTeX{}
distribution, e.g.\ \textit{texmf-root}|/tex/latex/childdoc|.
\end{itemize}

%%%%%%%%%%%%%%%%%%%%%%%%%%%%%%%%%%%%%%%%%%%%%%%%%%%%%%%%%%%%%%%%%%%%%%%%%%%%%%%%
\subsection{Related CTAN Packages}

There are several other packages which offer a similar functionality:
%
\begin{itemize}
\item
The packages
\href{http://ctan.org/pkg/docmute}{\textsf{docmute}},
\href{http://ctan.org/pkg/includex}{\textsf{includex}} and
\href{http://ctan.org/pkg/standalone}{\textsf{standalone}}
provide commands to include only the document body of
a child file thus allowing both files to be compiled individually.
\item
The packages \href{http://ctan.org/pkg/subdocs}{\textsf{subdocs}}
and \href{http://ctan.org/pkg/subfiles}{\textsf{subfiles}}
provide structures in which the main and child documents can be
encapsulated and allowing them to be compiled individually.
The inclusion mechanism is different from the conventional |\include|.
\item
The package \href{http://ctan.org/pkg/combine}{\textsf{combine}}
is an elaborate solution to combine several documents into one.
\end{itemize}
%
See also the CTAN topic \href{http://ctan.org/topic/subdocs}{\textsf{subdocs}}
for further related packages.
The present package differs from the above solutions in that
a document structure constructed with the conventional |\include| mechanism
just needs two extra commands at the top of every file
such that all constituent files can be compiled individually.

%%%%%%%%%%%%%%%%%%%%%%%%%%%%%%%%%%%%%%%%%%%%%%%%%%%%%%%%%%%%%%%%%%%%%%%%%%%%%%%%
%\subsection{Feature Suggestions}
%
%The following is a list of features which may be useful for future
%versions of this package:
%%
%\begin{itemize}
%\item
%\ldots
%\end{itemize}

%%%%%%%%%%%%%%%%%%%%%%%%%%%%%%%%%%%%%%%%%%%%%%%%%%%%%%%%%%%%%%%%%%%%%%%%%%%%%%%%
\subsection{Revision History}

%%%%%%%%%%%%%%%%%%%%%%%%%%%%%%%%%%%%%%%%
\paragraph{v2.0:} 2018/12/30

\begin{itemize}
\item
immediate forward processing
\item
added |\childdocby| mechanism
\item
manual restructured
\end{itemize}

%%%%%%%%%%%%%%%%%%%%%%%%%%%%%%%%%%%%%%%%
\paragraph{v1.6:} 2018/01/17

\begin{itemize}
\item
application for development of include files
\item
corrections to manual
\end{itemize}

%%%%%%%%%%%%%%%%%%%%%%%%%%%%%%%%%%%%%%%%
\paragraph{v1.5:} 2017/05/21

\begin{itemize}
\item
more complete structuring introduced
\item
|\childdocof| introduced
\item
|\childdoc| renamed to |\childdocmain|
\item
|\childredirect| renamed to |\childdocforward| and |\childdocforwardprefix|
and functionality expanded
\end{itemize}

%%%%%%%%%%%%%%%%%%%%%%%%%%%%%%%%%%%%%%%%
\paragraph{v1.0:} 2017/04/27

\begin{itemize}
\item
manual and install package
\item
first version published on CTAN
\end{itemize}

%%%%%%%%%%%%%%%%%%%%%%%%%%%%%%%%%%%%%%%%
\paragraph{v0.6:} 2017/04/26

\begin{itemize}
\item
redirection mechanism added
\end{itemize}

%%%%%%%%%%%%%%%%%%%%%%%%%%%%%%%%%%%%%%%%
\paragraph{v0.5:} 2017/04/26

\begin{itemize}
\item
functionality in definition file
\end{itemize}


%%%%%%%%%%%%%%%%%%%%%%%%%%%%%%%%%%%%%%%%%%%%%%%%%%%%%%%%%%%%%%%%%%%%%%%%%%%%%%%%
%%%%%%%%%%%%%%%%%%%%%%%%%%%%%%%%%%%%%%%%%%%%%%%%%%%%%%%%%%%%%%%%%%%%%%%%%%%%%%%%
%%%%%%%%%%%%%%%%%%%%%%%%%%%%%%%%%%%%%%%%%%%%%%%%%%%%%%%%%%%%%%%%%%%%%%%%%%%%%%%%
\appendix

\settowidth\MacroIndent{\rmfamily\scriptsize 000\ }

 \DocInput{childdoc.dtx}

\end{document}
%</driver>
% \fi
%
% %%%%%%%%%%%%%%%%%%%%%%%%%%%%%%%%%%%%%%%%%%%%%%%%%%%%%%%%%%%%%%%%%%%%%%%%%%%%%%
% %%%%%%%%%%%%%%%%%%%%%%%%%%%%%%%%%%%%%%%%%%%%%%%%%%%%%%%%%%%%%%%%%%%%%%%%%%%%%%
% \section{Sample}
%\iffalse
%<*samplemain>
%\fi
%
% The following presents a sample document
% with two chapters, two parts, a title page,
% a compile flag as well as three forwarding files to set the flag.
% It consists of eight |.tex| files:
% \begin{center}
% \begin{tabular}{ll}
% |cdocsamp.tex|&main file\\
% |cdocsch1.tex|&include file for chapter 1\\
% |cdocsch2.tex|&include file for chapter 2\\
% |cdocspt3.tex|&include file for part 3\\
% |cdocspt4.tex|&include file for part 4\\
% |cdocsdrf.tex|&forwarding file for main file in draft mode\\
% |cdocsfi1.tex|&forwarding file for final version of chapter 1\\
% |cdocsfi2.tex|&forwarding file for final version of chapter 2\\
% \end{tabular}
% \end{center}
% Each of the eight files can be compiled directly by the \LaTeX{} compiler.
%
% %%%%%%%%%%%%%%%%%%%%%%%%%%%%%%%%%%%%%%
% \paragraph{Main File.}
%
% The main file is called |cdocsamp.tex|.
%
% Load the \textsf{childdoc} definitions and
% declare the filename for the main document:
%    \begin{macrocode}
\input{childdoc.def}
\childdocmain{}
%    \end{macrocode}

% Optional override for |\version| flag:
%    \begin{macrocode}
%%\ifchilddoc\else\providecommand{\version}{draft}\fi
%    \end{macrocode}

% Define the default values for the |\version| flag
% (|final| for the main file and |draft| for childs):
%    \begin{macrocode}
\ifchilddoc
\providecommand{\version}{draft}
\else
\providecommand{\version}{final}
\fi
%    \end{macrocode}

% Load the standard document class:
%    \begin{macrocode}
\documentclass[12pt]{article}
%    \end{macrocode}

% Start the document body:
%    \begin{macrocode}
\begin{document}
%    \end{macrocode}

% Declare a title page.
% Print title, part of document being processed and version flag:
%    \begin{macrocode}
\addtocounter{page}{-1}
\begin{center}
{\LARGE\bfseries{}childdoc example\par}
\vspace{1cm}
\ifchilddoc
\ifchilddocmanual part\else chapter\fi:
`\childdocname' of `\childdocjob'\par
\else
main document: `\childdocjob'\par
\fi
version: \version\par
\end{center}
\newpage
%    \end{macrocode}

% Manually include selected file,
% otherwise process as usual:
%    \begin{macrocode}
\ifchilddocmanual
\section*{part `\childdocname'}
\input{\childdocname}
\else
%    \end{macrocode}

% Include the two chapters:
%    \begin{macrocode}
\include{cdocsch1}
\include{cdocsch2}
%    \end{macrocode}

% Include the two parts unless only chapters should be displayed:
%    \begin{macrocode}
\ifchilddoc\else
\section{part three}
\input{cdocspt3}
\section{part four}
\input{cdocspt4}
\fi
%    \end{macrocode}

% Process as usual until here:
%    \begin{macrocode}
\fi
%    \end{macrocode}

% End of document body:
%    \begin{macrocode}
\end{document}
%    \end{macrocode}
%\iffalse
%</samplemain>
%\fi
%
% %%%%%%%%%%%%%%%%%%%%%%%%%%%%%%%%%%%%%%
% \paragraph{Chapter Include Files.}
%
% The include files are called |cdocsch1.tex| and |cdocsch2.tex|.
%
%\iffalse
%<*samplechap1|samplechap2>
%\fi

% Optional override for |\version| flag:
%    \begin{macrocode}
%%\providecommand{\version}{final}
%    \end{macrocode}

% Include the main document:
%    \begin{macrocode}
\input{childdoc.def}
\childdocof{cdocsamp}
%    \end{macrocode}

%\iffalse
%</samplechap1|samplechap2>
%\fi
%
%\iffalse
%<*samplechap1>
%\fi
% Some text for chapter 1:
%    \begin{macrocode}
\section{one}
some text in chapter one
%    \end{macrocode}

%\iffalse
%</samplechap1>
%\fi
% Some text for chapter 2:
%\iffalse
%<*samplechap2>
%\fi
%    \begin{macrocode}
\section{two}
more text in chapter two
%    \end{macrocode}

%\iffalse
%</samplechap2>
%\fi
%
% %%%%%%%%%%%%%%%%%%%%%%%%%%%%%%%%%%%%%%
% \paragraph{Part Include Files.}
%
% The include files are called |cdocspt3.tex| and |cdocspt4.tex|.
%
%\iffalse
%<*samplepart3|samplepart4>
%\fi

% Optional override for |\version| flag:
%    \begin{macrocode}
%%\providecommand{\version}{final}
%    \end{macrocode}

% Include the main document:
%    \begin{macrocode}
\input{childdoc.def}
\childdocby{cdocsamp}
%    \end{macrocode}

%\iffalse
%</samplepart3|samplepart4>
%\fi
%
%\iffalse
%<*samplepart3>
%\fi
% Some text for part 3:
%    \begin{macrocode}
some text in part three
%    \end{macrocode}

%\iffalse
%</samplepart3>
%\fi
% Some text for part 4:
%\iffalse
%<*samplepart4>
%\fi
%    \begin{macrocode}
more text in part four
%    \end{macrocode}

%\iffalse
%</samplepart4>
%\fi
%
% %%%%%%%%%%%%%%%%%%%%%%%%%%%%%%%%%%%%%%
% \paragraph{Forwarding for a Complete Draft.}
%
% The following forwarding file |cdocsdrf.tex|
% compiles the main document in draft mode:
%\iffalse
%<*sampledraft>
%\fi
%    \begin{macrocode}
\def\version{draft}
\input{childdoc.def}
\childdocforward{cdocsamp}
%    \end{macrocode}

%\iffalse
%</sampledraft>
%\fi
%
% %%%%%%%%%%%%%%%%%%%%%%%%%%%%%%%%%%%%%%
% \paragraph{Forwarding for Final Version of the Chapters.}
%
% The following forwarding files |cdocsfn1.tex| and |cdocsfn2.tex|
% (with identical content)
% compile the final versions of the child documents
% |cdocsch1.tex| and |cdocsch2.tex|, respectively:
%\iffalse
%<*samplefinal>
%\fi
%    \begin{macrocode}
\def\version{final}
\input{childdoc.def}
\childdocforwardprefix[cdocsamp]{cdocsfn}{cdocsch}
%    \end{macrocode}

%\iffalse
%</samplefinal>
%\fi
%
% %%%%%%%%%%%%%%%%%%%%%%%%%%%%%%%%%%%%%%
% \paragraph{Command Line Processing.}
%
% The following three command lines generate the output files
% |cdocscld|, |cdocscl1| and |cdocscl2|
% which should be identical to
% |cdocsdrf|, |cdocsch1| and |cdocsfn2|, respectively:
% \begin{center}
% \begin{tabular}{l}
% |latex -jobname cdocscld \|\\
% |  "\def\version{draft}\input{childdoc.def}\childdocforward{cdocsamp}"|\\
% |latex -jobname cdocscl1 \|\\
% |  "\input{childdoc.def}\childdocforward[cdocsamp]{cdocsch1}"|\\
% |latex -jobname cdocscl2 \|\\
% |  "\def\version{final}\input{childdoc.def}\childdocforward{cdocsch2}"|
% \end{tabular}
% \end{center}
% Note that the trailing backslash on each first line
% merely continues the input to the second line
% (for convenient cut ant paste).
% Furthermore, the command |latex| can be replaced by any
% of its alternative versions such as |pdflatex|.
%
% %%%%%%%%%%%%%%%%%%%%%%%%%%%%%%%%%%%%%%%%%%%%%%%%%%%%%%%%%%%%%%%%%%%%%%%%%%%%%%
% %%%%%%%%%%%%%%%%%%%%%%%%%%%%%%%%%%%%%%%%%%%%%%%%%%%%%%%%%%%%%%%%%%%%%%%%%%%%%%
% \section{Implementation}
%\iffalse
%<*package>
%\fi
%
% This section describes the definitions file |childdoc.def|.

% The definitions cannot be loaded using |\usepackage| or |\RequirePackage|
% which has a mechanism to prevent loading a style file more than once.
% When loading the definitions by means of |\input|
% multiple instances have to be prevented manually:
%\iffalse
%This code needs to be before the `\ProvidesFile' directive
%which is defined at the beginning of this file.
%Therefore it is also placed there and commented out here.
%</package>
%<*discard>
%\fi
%    \begin{macrocode}
\ifdefined\childdocmain\endinput\fi
%    \end{macrocode}
%\iffalse
%</discard>
%<*package>
%\fi
%
% \macro{\ifchilddoc}
% \macro{\ifchilddocmanual}
% The conditional |\ifchilddoc| tells whether a
% child (true) or main (false) document is being compiled.
% The conditional |\ifchilddocmanual| tells whether
% the |\includeonly| mechanism is used (false) or
% the selection of child files must be performed manually (true).
% The definitions initialise to false:
%    \begin{macrocode}
\newif\ifchilddoc
\newif\ifchilddocmanual
%    \end{macrocode}

% \macro{\childdocname}
% \macro{\childdocjob}
% The macro |\childdocname| stores the name of the main document
% to be compiled. The macro |\childdocjob| stores the name of
% the document on which the \LaTeX{} compiler was originally invoked.
% The content of |\jobname| cannot be compared
% to filenames specified in the source due to different catcodes.
% The following code rescans |\jobname|, stores the result
% in |\childdocname| and saves a copy in |\childdocjob|:
%    \begin{macrocode}
\edef\childdocname{\scantokens\expandafter{\jobname\noexpand}}
\let\childdocjob\childdocname
%    \end{macrocode}

% \macro{\childdocdisable}
% The macro |\childdocdisable| prevents the main file
% from being processed more than once.
% At this stage, the main document command |\childdocmain|
% is assumed to be called once again where it should do nothing.
% Any subsequent call to it should prevent
% a secondary processing of the main document
% It overwrites the forwarding commands
% |\childdocof| and |\childdocforward|
% with empty macros to prevent further inclusions of the main document:
%    \begin{macrocode}
\newcommand{\childdocdisable}
{
  \renewcommand{\childdocmain}[1]{\renewcommand{\childdocmain}[1]{\endinput}}
  \renewcommand{\childdocof}[1]{}
  \renewcommand{\childdocby}[2][]{}
  \renewcommand{\childdocforward}[2][]{}
  \renewcommand{\childdocdisable}{}
}
%    \end{macrocode}

% \macro{\childdocmain}
% The macro |\childdocmain| is to be called at the top of the main file
% with nothing or the main filename (without extension) as argument.
% First, it breaks loops.
% If the argument is not empty and does not match |\childdocname|
% (which is set by the first inclusion of |childdoc.def|),
% |\ifchilddoc| is set to true, |\includeonly| is applied to the child file
% and |\jobname| is set to the main file
% (for proper handling of |.aux| files):
%    \begin{macrocode}
\newcommand{\childdocmain}[1]
{
  \childdocdisable\childdocmain{}
  \if?#1?\else
    \begingroup
      \def\childdoctmp{#1}
      \ifx\childdoctmp\childdocname
        \def\childdoctmp{}
      \else
        \def\childdoctmp
        {
          \childdoctrue
          \includeonly{\childdocname}
          \def\childdocjob{#1}
          \def\jobname{#1}
        }
      \fi
      \expandafter
    \endgroup
    \childdoctmp
  \fi
}
%    \end{macrocode}

% \macro{\childdocof}
% The command |\childdocof| redirects
% compilation to the main file |#1|.
%    \begin{macrocode}
\newcommand{\childdocof}[1]
{
  \childdocdisable
  \childdoctrue
  \includeonly{\childdocname}
  \def\jobname{#1}
  \def\childdocjob{#1}
  \input{#1}
}
%    \end{macrocode}

% \macro{\childdocby}
% The command |\childdocby| ....
%    \begin{macrocode}
\newcommand{\childdocby}[2][]
{
  \childdocdisable
  \childdoctrue
  \childdocmanualtrue
  \if?#1?\else
    \def\jobname{#2}
  \fi
  \def\childdocjob{#2}
  \input{#2}
  \endinput
}
%    \end{macrocode}

% \macro{\childdocforward}
% The command |\childdocforward| redirects
% compilation to the main file or
% (if the optional argument is given) a child file.
% Parameters are set as if the main file
% or a child file starting with |\childdocof| was compiled.
% Then compilation is handed over to the main file:
%    \begin{macrocode}
\newcommand{\childdocforward}[2][]
{
  \begingroup
    \if?#1?
      \def\childdoctmp
      {
        \def\childdocname{#2}
        \def\childdocjob{#2}
        \def\jobname{#2}
        \input{#2}
        \endinput
      }
    \else
      \def\childdoctmp
      {
        \childdocdisable
        \def\childdocname{#2}
        \childdoctrue
        \includeonly{#2}
        \def\childdocjob{#1}
        \def\jobname{#1}
        \input{#1}
        \endinput
      }
    \fi
    \expandafter
  \endgroup
  \childdoctmp
}
%    \end{macrocode}

% \macro{\childdocforwardprefix}
% The command |\childdocforwardprefix| redirects
% compilation to the main or a child file by means of a pattern.
% The prefix |#1| in the current filename is replaced by |#2|
% and the suffix of the current filename is kept
% (it is assumed that the filename does not contain the substring `|~~~|'
% which is used as a delimiter).
% Compilation is handed over to the new file by |\childdocforward|:
%    \begin{macrocode}
\newcommand{\childdocforwardprefix}[3][]
{
  \begingroup
    \def\childdocextract #2##1~~~{\def\childdoctmp{\childdocforward[#1]{#3##1}}}
    \expandafter\childdocextract\childdocname~~~
    \expandafter
  \endgroup
  \childdoctmp
}
%    \end{macrocode}

% \macro{\childdoc}
% The deprecated macro |\childdoc| is a legacy version of |\childdocmain|:
%    \begin{macrocode}
\newcommand{\childdoc}{\childdocmain}
%    \end{macrocode}

% \macro{\childdocredirect}
% The deprecated macro |\childdocredirect| is a legacy version
% of |\childdocforward| and |\childdocforwardprefix|:
%    \begin{macrocode}
\newcommand{\childdocredirect}[2][]
{
  \begingroup
    \if?#1?
      \def\childdoctmp{\childdocforward{#2}}
    \else
      \def\childdoctmp{\childdocforwardprefix{#1}{#2}}
    \fi
    \expandafter
  \endgroup
  \childdoctmp
}
%    \end{macrocode}

%\iffalse
%</package>
%\fi
%
\endinput
|\\
|\childdocby{|\textit{main}|}|\\
\end{tabular}
\end{center}
%
The directive |\childdocby| is similar to |\childdocof|
described in \secref{sec:include},
but the subsequent selection of content must be done manually.
To that end, both |\ifchilddoc| and |\ifchilddocmanual|
will be true upon processing of a part,
and the name of the part is stored in |\childdocname|.
Note that |\jobname| will be set to the filename of the current part
so that each part receives an individual |.aux| file
that does not interfere with the |.aux| file(s) of the main document.
This behaviour can be altered by the alternative form
|\childdocby[*]{|\textit{main}|}| (with a non-empty optional argument)
which uses the |.aux| file of the main document
by setting |\jobname| to \textit{main}.

%%%%%%%%%%%%%%%%%%%%%%%%%%%%%%%%%%%%%%%%%%%%%%%%%%%%%%%%%%%%%%%%%%%%%%%%%%%%%%%%
\subsection{Driver Development}
\label{sec:driver}

The \textsf{childdoc} mechanism can also be use for the development
of definition files such as \LaTeX{} styles or classes.
This case differs from the above setup with multiple parts
included by |\include| in that no |\includeonly| should be invoked.
This can be achieved by starting the include file
(before |\ProvidesPackage|) with:
%
\begin{center}
\begin{tabular}{l}
|% \iffalse
%
% childdoc.dtx Copyright (C) 2017-2018 Niklas Beisert
%
% This work may be distributed and/or modified under the
% conditions of the LaTeX Project Public License, either version 1.3
% of this license or (at your option) any later version.
% The latest version of this license is in
%   http://www.latex-project.org/lppl.txt
% and version 1.3 or later is part of all distributions of LaTeX
% version 2005/12/01 or later.
%
% This work has the LPPL maintenance status `maintained'.
%
% The Current Maintainer of this work is Niklas Beisert.
%
% This work consists of the files childdoc.dtx and childdoc.ins
% and the derived files childdoc.def and cdocsamp.tex with
% cdocsch1.tex, cdocsch2.tex, cdocsdrf.tex, cdocsfn1.tex, cdocsfn2.tex.
%
%<package>\ifdefined\childdocmain\endinput\fi
%<package>\ProvidesFile{childdoc.def}[2018/12/30 v2.0 child document driver]
%<samplemain>\ProvidesFile{cdocsamp.tex}[2018/12/30 v2.0 sample for childdoc]
%<*driver>
%\ProvidesFile{childdoc.drv}[2018/12/30 v2.0 childdoc reference manual file]
\PassOptionsToClass{10pt,a4paper}{article}
\documentclass{ltxdoc}

\usepackage[margin=35mm]{geometry}
\usepackage{hyperref}
\usepackage{hyperxmp}
\usepackage[usenames]{color}

\hypersetup{colorlinks=true}
\hypersetup{pdfstartview=FitH}
\hypersetup{pdfpagemode=UseNone}
\hypersetup{pdfsource={}}
\hypersetup{pdflang={en-UK}}
\hypersetup{pdfcopyright={Copyright 2017-2018 Niklas Beisert.
  This work may be distributed and/or modified under the
  conditions of the LaTeX Project Public License, either version 1.3
  of this license or (at your option) any later version.}}
\hypersetup{pdflicenseurl={http://www.latex-project.org/lppl.txt}}
\hypersetup{pdfcontactaddress={ETH Zurich, ITP, HIT K,
  Wolfgang-Pauli-Strasse 27}}
\hypersetup{pdfcontactpostcode={8093}}
\hypersetup{pdfcontactcity={Zurich}}
\hypersetup{pdfcontactcountry={Switzerland}}
\hypersetup{pdfcontactemail={nbeisert@itp.phys.ethz.ch}}
\hypersetup{pdfcontacturl={http://people.phys.ethz.ch/\xmptilde nbeisert/}}

\newcommand{\secref}[1]{\hyperref[#1]{section \ref*{#1}}}

\parskip1ex
\parindent0pt
\let\olditemize\itemize
\def\itemize{\olditemize\parskip0pt}

\begin{document}

\title{The \textsf{childdoc} Package}
\hypersetup{pdftitle={The childdoc Package}}
\author{Niklas Beisert\\[2ex]
  Institut f\"ur Theoretische Physik\\
  Eidgen\"ossische Technische Hochschule Z\"urich\\
  Wolfgang-Pauli-Strasse 27, 8093 Z\"urich, Switzerland\\[1ex]
  \href{mailto:nbeisert@itp.phys.ethz.ch}
  {\texttt{nbeisert@itp.phys.ethz.ch}}}
\hypersetup{pdfauthor={Niklas Beisert}}
\hypersetup{pdfsubject={Manual for the LaTeX2e Package childdoc}}
\date{30 December 2018, \textsf{v2.0}}
\maketitle

\begin{abstract}\noindent
\textsf{childdoc} is a \LaTeXe{} package
that enables the direct compilation
of document sections included by |\include|
to individual files.
\end{abstract}

\begingroup
\parskip0ex
\tableofcontents
\endgroup

%%%%%%%%%%%%%%%%%%%%%%%%%%%%%%%%%%%%%%%%%%%%%%%%%%%%%%%%%%%%%%%%%%%%%%%%%%%%%%%%
%%%%%%%%%%%%%%%%%%%%%%%%%%%%%%%%%%%%%%%%%%%%%%%%%%%%%%%%%%%%%%%%%%%%%%%%%%%%%%%%
\section{Introduction}

\LaTeX{} provides a mechanism to structure a large document (such as a book)
into a main file and several child files (containing the chapters)
using the |\include| command.
This mechanism is beneficial for documents
which span hundreds of pages in order to
make the source file(s) more manageable.
Moreover, compilation can be restricted to
selected child files by means of the |\includeonly| command.
The latter feature can be used to reduce the compilation time while editing
(this was significantly more useful in the earlier days of \LaTeX{})
or to generate a smaller document which is easier to navigate.
Another application of |\includeonly| is to generate
documents consisting of selected parts of the complete document.

However, there are a few drawbacks of the plain |\include| mechanism:
\begin{itemize}
\item
The child files cannot be compiled on their own,
they can only be compiled via the main file.
A naive editing environment
(such as a text editor with an option
to have the current file processed by \LaTeX)
may require one to switch to the main file before compiling;
attempting to compile the child file produces errors.
\item
The main file must be modified (each time)
to adjust the |\includeonly| command
to the present needs. This easily leaves the main file in a messy state.
\item
The generated document will always carry the filename
of the main document. This is inconvenient if
several child files are to be compiled and
to be kept for distribution.
\end{itemize}

The present package provides a simple interface
to make child files individually compilable by \LaTeX{}.
Compiling a child file then has the same effect as compiling
the main file with an |\includeonly| command
to select the appropriate child.
Moreover the generated document will carry the name of the child
rather than the main file.
This resolves all three above issues.

This feature is meant to make the editing of books,
thesis documents and lecture notes somewhat more convenient.
However, the package can also be used efficiently for
composing a series of documents (such as exercise sheets)
which are typically distributed individually.
It then assists the author in generating the individual documents
(potentially in different versions)
as well as a document containing the collected series.
Another application is in developing style files
or other kinds of included material
where compilation of the style file could redirect
to a sample or test file.

%%%%%%%%%%%%%%%%%%%%%%%%%%%%%%%%%%%%%%%%%%%%%%%%%%%%%%%%%%%%%%%%%%%%%%%%%%%%%%%%
%%%%%%%%%%%%%%%%%%%%%%%%%%%%%%%%%%%%%%%%%%%%%%%%%%%%%%%%%%%%%%%%%%%%%%%%%%%%%%%%
\section{Usage}

First of all, the package \textsf{childdoc} is \emph{not} a standard
\LaTeXe{} |.sty| style file! Therefore it needs to be invoked in
a non-standard way.

%%%%%%%%%%%%%%%%%%%%%%%%%%%%%%%%%%%%%%%%%%%%%%%%%%%%%%%%%%%%%%%%%%%%%%%%%%%%%%%%
\subsection{Included Files}
\label{sec:include}

%%%%%%%%%%%%%%%%%%%%%%%%%%%%%%%%%%%%%%%%
\DescribeMacro{\childdocmain}
To use the package, add the commands
\begin{center}
\begin{tabular}{l}
|\input{childdoc.def}|\\
|\childdocmain{}|\\
\end{tabular}
\end{center}
at the very top of the main \LaTeX{} file,
in particular \emph{before} the |\documentclass| statement!
The argument of |\childdocmain| should be left empty
(but it must be present).

%%%%%%%%%%%%%%%%%%%%%%%%%%%%%%%%%%%%%%%%
\DescribeMacro{\childdocof}
Furthermore, add the commands
\begin{center}
\begin{tabular}{l}
|\input{childdoc.def}|\\
|\childdocof{|\textit{main}|}|\\
\end{tabular}
\end{center}
at the top of every child file \textit{child}
which is included by |\include{|\textit{child}|}|
from within the main file
(or at least for those files to be compiled individually).
The argument \textit{main} must be the filename of the main file.

There are a couple of
considerations in setting up the main and child documents:

%%%%%%%%%%%%%%%%%%%%%%%%%%%%%%%%%%%%%%%%
\paragraph{Restrictions.}

Please note the following restrictions:
\begin{itemize}
\item
|\childdocmain| must be called with one argument \textit{main}
to ensure compatibility with earlier version of the package.
It must either be empty (|\childdocmain{}|)
or precisely match the filename of the main file in which it is specified.
See \secref{sec:detection} for further information.
\item
The filename \textit{main} must be specified without the |.tex| extension.
\item
The filename \textit{main} is case sensitive
(even in case-insensitive file systems)
due to internal string comparison.
\item
The argument \textit{main} should be fully expanded, it cannot be a macro.
\item
Subdirectories and special characters should be avoided in filenames.
\item
The command |\childdocmain{|\textit{main}|}| must be followed by a whitespace.
It should not be followed immediately by another command
or by a comment mark `|%|'.
This is because the \TeX{} parser reads the token immediately following
the argument of |\childdocmain| and puts it
at the beginning of every child section;
however, a white\-space is ignored.
\end{itemize}

%%%%%%%%%%%%%%%%%%%%%%%%%%%%%%%%%%%%%%%%
\paragraph{Content of Main File.}

It is advisable to place all content in the child files included by |\include|.
Any output contained in the main file will appear in all child documents
unless suppressed manually;
it cannot be suppressed automatically by the |\includeonly| directive
and thus should normally be avoided.
A method to include some content in the main file
by means of conditional processing is described in \secref{sec:conditional}.

%%%%%%%%%%%%%%%%%%%%%%%%%%%%%%%%%%%%%%%%
\paragraph{Page Numbering.}

When only a part of the document is compiled,
the appropriate numbering of pages
(as well as other status parameters)
is determined from the |.aux| files.
The latter contain information from previous passes.
However this information needs to propagate through
all intermediate child documents.
Therefore the page numbering in child documents may well
be inconsistent until the complete document is compiled at least once.

A useful (if unconventional) way to always ensure a consistent
page numbering is to restart the numbering in each child document
and denote the pages by `\textit{child}|.|\textit{page}'
where \textit{child} represents the chapter/section number of the child file.
This can be achieved by the command
|\numberwithin{page}{|\textit{child}|}|
of the \textsf{amsmath} package
where \textit{child} can be |chapter| or |section|
depending on the chosen structuring.
Alternatively, one can modify the macro |\thepage| appropriately
and reset the counter |page| at the start of each child file.

%%%%%%%%%%%%%%%%%%%%%%%%%%%%%%%%%%%%%%%%%%%%%%%%%%%%%%%%%%%%%%%%%%%%%%%%%%%%%%%%
\subsection{Conditional Processing}
\label{sec:conditional}

The package provides a mechanism to compile different versions
of a document. To customise the versions further some conditional processing
can come in handy to distinguish which version is being compiled.
The package provides two macros to describe the compilation context:

%%%%%%%%%%%%%%%%%%%%%%%%%%%%%%%%%%%%%%%%
\DescribeMacro{\ifchilddoc}
The conditional |\ifchilddoc| distinguishes between the compilation of
child documents and the main document:
%
\begin{center}
|\ifchilddoc |\textit{child-code}| |[|\||else |\textit{main-code}]| \||fi|
\end{center}

%%%%%%%%%%%%%%%%%%%%%%%%%%%%%%%%%%%%%%%%
\DescribeMacro{\childdocname}
\DescribeMacro{\childdocjob}
The macro |\childdocname| contains the filename (without extension)
of the main or child file being processed.
Note that |\childdocjob| will always contain the name of the main file.

%%%%%%%%%%%%%%%%%%%%%%%%%%%%%%%%%%%%%%%%
\paragraph{Title Page.}

Conditional processing can be used to include a title or banner page
in the main document when proper precautions are taken.
Importantly, the code in the main file should ensure that the page counter
(as well as other status parameters which are stored in the |.aux| files)
takes the same value after the conditional processing.
Otherwise the page numbers may take divergent values
depending on which part is compiled.

For example, a title page could be declared by:
%
\begin{center}
\begin{tabular}{l}
|\ifchilddoc\||else|\\
|\addtocounter{page}{-1}|\\
\textit{code for title page}\\
|\newpage|\\
|\||fi|
\end{tabular}
\end{center}
%
A banner page for the child documents can be generated by:
%
\begin{center}
\begin{tabular}{l}
|\ifchilddoc|\\
|\addtocounter{page}{-1}|\\
\textit{code for banner page}\\
|\newpage|\\
|\||fi|
\end{tabular}
\end{center}
%
Here one could write a message such as:
\begin{center}
|This is the part \childdocname{} of \childdocjob{}.|
\end{center}

%%%%%%%%%%%%%%%%%%%%%%%%%%%%%%%%%%%%%%%%%%%%%%%%%%%%%%%%%%%%%%%%%%%%%%%%%%%%%%%%
\subsection{Flags}
\label{sec:flags}

The package makes it easy to generate different versions
of the main or child documents.
To this end compilation flags can be defined
and assigned different default values.
They will be particularly useful in conjunction
with the forwarding mechanism described in \secref{sec:forward}.

For example, it may be useful to have a flag |\version|
which can be set to |draft| or |final|.
The document source will contain some conditional code
depending on the value of |\version|.
Suppose further, the flag should default to |final| for the main file
and to |draft| for child files
which is a natural assignment for editing the document.
This is achieved by placing the following code
in the preamble of the main document
(below the |\childdocmain| directive):
%
\begin{center}
\begin{tabular}{l}
|\ifchilddoc|\\
|\providecommand{\version}{draft}|\\
|\||else|\\
|\providecommand{\version}{final}|\\
|\||fi|
\end{tabular}
\end{center}
%
The definition by |\providecommand| makes sure
that previous definitions are not overwritten.
Further statements |\providecommand{\version}{...}|
can thus be added before the above code to override it.

For the main file, one might add a line
(between |\childdocmain| and the above block)
%
\begin{center}
|%\ifchilddoc\||else\providecommand{\version}{draft}\||fi|
\end{center}
%
which can be uncommented to produce a draft version.
Likewise one can add a line to the very top of a child file
(above the |\childdocof{|\textit{main}|}| directive)
%
\begin{center}
|%\providecommand{\version}{final}|
\end{center}
%
which can be uncommented to produce the final version of this child document.

%%%%%%%%%%%%%%%%%%%%%%%%%%%%%%%%%%%%%%%%%%%%%%%%%%%%%%%%%%%%%%%%%%%%%%%%%%%%%%%%
\subsection{Forwarding}
\label{sec:forward}

Different versions of the main or child documents
using compilation flags as described in \secref{sec:flags}
can be (permanently) stored in different files
for convenient compilation, viewing and distribution.
To this end, the package defines a command
to pass on compilation to a different file:

%%%%%%%%%%%%%%%%%%%%%%%%%%%%%%%%%%%%%%%%
\DescribeMacro{\childdocforward}
The command |\childdocforward| redirects processing to
another source file:
%
\begin{center}
\begin{tabular}{l}
|\input{childdoc.def}|\\
|\childdocforward[|\textit{main}|]{|\textit{dest}|}|\\
\end{tabular}
\end{center}
%
The argument \textit{dest} is the destination file
(without extension).
It should be the main file or one of the child files.
Note that further \textsf{childdoc} directives
such as |\childdocof| and |\childdocforward|
in the indicated file will be processed in this form.
The optional argument \textit{main}
passes on directly to the main file \textit{main}
while pretending to compile the child \textit{dest}.
This form behaves as if \textit{dest}
issues |\childdocof{|\textit{main}|}| right away,
and no further \textsf{childdoc} directives will be processed.

%%%%%%%%%%%%%%%%%%%%%%%%%%%%%%%%%%%%%%%%
\DescribeMacro{\...prefix}
In the alternative form |\childdocforwardprefix|,
%
\begin{center}
\begin{tabular}{l}
|\input{childdoc.def}|\\
|\childdocforwardprefix[|\textit{main}|]{|\textit{prefix}|}{|\textit{dest}|}|
\end{tabular}
\end{center}
%
the destination file is determined by a pattern
depending on the current file:
To make this work, the current file must be called
`{\textit{prefix}\hspace{0.2em}\textit{suffix}}'
with \textit{prefix} matching precisely the argument.
Processing is then passed on to the file
`{\textit{dest}\hspace{0.2em}\textit{suffix}}'.
Surely, the same effect is achieved by
directly specifying the
argument `{\textit{dest}\hspace{0.2em}\textit{suffix}}'
in the first form.
However, that requires to set up a different file
for each child. With the alternative form of the command
all these files can have exactly the same content
which simplifies setting them up and maintaining them.

For example, the following file |draft.tex|
with a compilation flag |\version| as described in \secref{sec:flags}
compiles the main document as a draft:
%
\begin{center}
\begin{tabular}{l}
|\def\version{draft}|\\
|\input{childdoc.def}|\\
|\childdocforward{|\textit{main}|}|
\end{tabular}
\end{center}
%
Likewise, the following files |final|\textit{nn}|.tex|
compile the final version of the child document
|child|\textit{nn}|.tex|:
%
\begin{center}
\begin{tabular}{l}
|\def\version{final}|\\
|\input{childdoc.def}|\\
|\childdocforwardprefix{final}{child}|
\end{tabular}
\end{center}
%

Note that when several versions of a main file and/or of each child file
are to be generated, it may be convenient to set up a |Makefile| or
shell script to automatise the process.

%%%%%%%%%%%%%%%%%%%%%%%%%%%%%%%%%%%%%%%%%%%%%%%%%%%%%%%%%%%%%%%%%%%%%%%%%%%%%%%%
\subsection{Command Line Processing}
\label{sec:commandline}

The effect of redirection files can also be achieved by invoking
the \LaTeX{} compiler with a more elaborate command line.
Most conveniently this should be done as part
of a shell script or a |Makefile|.

When using \textsf{childdoc} in the main file, the following
command lines effectively perform a redirection
(note that depending on the shell being used,
backslashes may have to be doubled: `|\|' $\to$ `|\\|'):
%
\begin{center}
|... -jobname "|\textit{target}|" |\\|"|[\textit{flags}]%
|\input{childdoc.def}\childdocforward[|\textit{main}|]{|\textit{dest}|}"|
\end{center}
%
Here \textit{target} is the name of the output file,
\textit{main} is the name of the main file
and \textit{dest} is the name of the main or child file to be processed
(all filenames without extensions).
The optional argument \textit{main} can be omitted
if \textit{main} matches \textit{dest}.
Optionally, compilation \textit{flags} can be defined via |\def| commands.
This command line makes the \TeX{} engine believe
it is compiling the file \textit{target}
whose content is specified as the latter parameter.
The provided code then forwards the processing to
\textit{main} or \textit{dest} as described in \secref{sec:forward}.

%%%%%%%%%%%%%%%%%%%%%%%%%%%%%%%%%%%%%%%%%%%%%%%%%%%%%%%%%%%%%%%%%%%%%%%%%%%%%%%%
\subsection{Include by Input}
\label{sec:input}

Including child documents by |\include| has some restrictions by design.
Most notably, the content of a child document always occupies
its own set of pages; pages cannot be shared between child documents.
Usually, this behaviour makes perfect sense
because each child document contain an essential part of the document.
However, in some situations it may be desirable to compose
a document from a collection of parts
without having mandatory page breaks between then.
For this case, the package
provides a mechanism to include parts
by |\input| which can also be processed individually.
However, by construction this mechanism
requires manual handling of the content to be output.

%%%%%%%%%%%%%%%%%%%%%%%%%%%%%%%%%%%%%%%%
\DescribeMacro{\ifchilddocmanual}
The main file should be prepared as usual, see \secref{sec:include}.
However, the document body must make a distinction
between processing of an individual part and of the main document, e.g.:
%
\begin{center}
\begin{tabular}{l}
|\ifchilddocmanual|\\
|\input{\childdocname}|\\
|\||else|\\
\textit{document body with }|\input{|\textit{part}|}|\\
|\||fi|
\end{tabular}
\end{center}
%
The conditional |\ifchilddocmanual| is true whenever
a part to be included by |\input| is being compiled,
and the name of the part is stored in |\childdocname|.

%%%%%%%%%%%%%%%%%%%%%%%%%%%%%%%%%%%%%%%%
\DescribeMacro{\childdocby}
Each part to be included by |\input| should start with:
%
\begin{center}
\begin{tabular}{l}
|\input{childdoc.def}|\\
|\childdocby{|\textit{main}|}|\\
\end{tabular}
\end{center}
%
The directive |\childdocby| is similar to |\childdocof|
described in \secref{sec:include},
but the subsequent selection of content must be done manually.
To that end, both |\ifchilddoc| and |\ifchilddocmanual|
will be true upon processing of a part,
and the name of the part is stored in |\childdocname|.
Note that |\jobname| will be set to the filename of the current part
so that each part receives an individual |.aux| file
that does not interfere with the |.aux| file(s) of the main document.
This behaviour can be altered by the alternative form
|\childdocby[*]{|\textit{main}|}| (with a non-empty optional argument)
which uses the |.aux| file of the main document
by setting |\jobname| to \textit{main}.

%%%%%%%%%%%%%%%%%%%%%%%%%%%%%%%%%%%%%%%%%%%%%%%%%%%%%%%%%%%%%%%%%%%%%%%%%%%%%%%%
\subsection{Driver Development}
\label{sec:driver}

The \textsf{childdoc} mechanism can also be use for the development
of definition files such as \LaTeX{} styles or classes.
This case differs from the above setup with multiple parts
included by |\include| in that no |\includeonly| should be invoked.
This can be achieved by starting the include file
(before |\ProvidesPackage|) with:
%
\begin{center}
\begin{tabular}{l}
|\input{childdoc.def}|\\
|\childdocforward{|\textit{main}|}|\\
\end{tabular}
\end{center}
%
or alternatively with:
%
\begin{center}
\begin{tabular}{l}
|\input{childdoc.def}|\\
|\childdocby{|\textit{main}|}|\\
\end{tabular}
\end{center}
%
Both forms have slightly different effects as described above.
The main file is prepared as usual, see \secref{sec:include}.

%%%%%%%%%%%%%%%%%%%%%%%%%%%%%%%%%%%%%%%%%%%%%%%%%%%%%%%%%%%%%%%%%%%%%%%%%%%%%%%%
\subsection{Legacy Detection}
\label{sec:detection}

The directive |\childdocmain| in the main file can detect
whether the complete document or merely a child is to be compiled
even without using the directive |\childdocof|.
This method is deprecated because it is less robust
and there is no compelling reason to use it;
it is merely provided for backward compatibility
and it may be removed in future versions.

If the detection mechanism is to be used,
it is mandatory to correctly specify
the filename of the main file as the argument of |\childdocmain|:
%
\begin{center}
\begin{tabular}{l}
|\input{childdoc.def}|\\
|\childdocmain{|\textit{main}|}|\\
\end{tabular}
\end{center}
%
If |\jobname| does not match the argument \textit{main} of |\childdocmain|,
it is assumed that |\jobname| points to the child file to be compiled.
When using |\childdocmain| with the main file specified as argument,
it suffices to start a child file
with just |\input{|\textit{main}|}|
without loading of the package and using |\childdocof|.
If instead all processing is done
with the appropriate \textsf{childdoc} directives,
the argument of \textit{main} of |\childdocmain| can be empty.

An alternative version of the command line processing described
in \secref{sec:commandline} using the detection mechanism reads:
%
\begin{center}
|... -jobname "|\textit{target}|" "|[\textit{flags}]%
[|\def\jobname{|\textit{dest}|}|]|\input{|\textit{main}|}"|
\end{center}

%%%%%%%%%%%%%%%%%%%%%%%%%%%%%%%%%%%%%%%%%%%%%%%%%%%%%%%%%%%%%%%%%%%%%%%%%%%%%%%%
\subsection{Manual Code}
\label{sec:manual}

In case one cannot be certain whether the definitions file |childdoc.def|
is installed on the target \TeX{} distribution
and one prefers not to ship it,
it is conceivable to paste a few relevant commands into the sources.

To that end, drop all statements |\input{childdoc.def}|
and perform the replacements as outlined below.
Instead of |\childdocmain{|\textit{main}|}| add the following code
to the top of the main file:
%
\begin{center}
\begin{tabular}{l}
|\||ifdefined\childdocname\endinput\||fi\newif\ifchilddoc|\\
|\edef\childdocname{\scantokens\expandafter{\jobname\noexpand}}|\\
|\def\childdocmain{|\textit{main}|}\||ifx\childdocmain\childdocname\||else|\\
|\childdoctrue\includeonly{\childdocname}\let\jobname\childdocmain\||fi|\\
\end{tabular}
\end{center}
%
Instead of |\childdocof{|\textit{main}|}| just include the main file
at the top of each child file:
%
\begin{center}
|\input{|\textit{main}|}|
\end{center}
%
A simple redirection |\childdocforward{|\textit{dest}|}| is achieved by:
%
\begin{center}
|\def\jobname{|\textit{dest}|}\input{\jobname}|
\end{center}
%
The redirection with prefix
|\childdocforwardprefix[|\textit{prefix}|]{|\textit{dest}|}|
is accomplished by:
%
\begin{center}
\begin{tabular}{l}
|{\edef\jobname{\scantokens\expandafter{\jobname\noexpand}}|\\
|\def\redirectjob |\textit{prefix}|#1~~~{\gdef\jobname{|\textit{dest}|#1}}|\\
|\expandafter\redirectjob\jobname~~~}\input{\jobname}|
\end{tabular}
\end{center}

In an alternative approach,
child documents can be compiled by a specific command line
without additional code or specific definitions:
%
\begin{center}
|... -jobname "|\textit{target}|" "|[\textit{flags}]%
|\includeonly{|\textit{dest}|}\input{|\textit{main}|}"|
\end{center}
%

%%%%%%%%%%%%%%%%%%%%%%%%%%%%%%%%%%%%%%%%%%%%%%%%%%%%%%%%%%%%%%%%%%%%%%%%%%%%%%%%
%%%%%%%%%%%%%%%%%%%%%%%%%%%%%%%%%%%%%%%%%%%%%%%%%%%%%%%%%%%%%%%%%%%%%%%%%%%%%%%%
\section{Information}

%%%%%%%%%%%%%%%%%%%%%%%%%%%%%%%%%%%%%%%%%%%%%%%%%%%%%%%%%%%%%%%%%%%%%%%%%%%%%%%%
\subsection{Copyright}

Copyright \copyright{} 2017--2018 Niklas Beisert

This work may be distributed and/or modified under the
conditions of the \LaTeX{} Project Public License, either version 1.3
of this license or (at your option) any later version.
The latest version of this license is in
  \url{http://www.latex-project.org/lppl.txt}
and version 1.3 or later is part of all distributions of \LaTeX{}
version 2005/12/01 or later.

This work has the LPPL maintenance status `maintained'.

The Current Maintainer of this work is Niklas Beisert.

This work consists of the files |README.txt|, |childdoc.ins| and |childdoc.dtx|
as well as the derived files |childdoc.def|, |cdocsamp.tex|
with |cdocsch1.tex|, |cdocsch2.tex|, |cdocspt3.tex|, |cdocspt4.tex|,
|cdocsdrf.tex|, |cdocsfn1.tex|, |cdocsfn2.tex|
as well as |childdoc.pdf|.

%%%%%%%%%%%%%%%%%%%%%%%%%%%%%%%%%%%%%%%%%%%%%%%%%%%%%%%%%%%%%%%%%%%%%%%%%%%%%%%%
\subsection{Files and Installation}

The package consists of the files:
%
\begin{center}
\begin{tabular}{ll}
    |README.txt|   & readme file \\
    |childdoc.ins| & installation file \\
    |childdoc.dtx| & source file \\
    |childdoc.def| & definition file \\
    |cdocsamp.tex| & sample main file \\
    |cdocsch1.tex| & sample include file \\
    |cdocsch2.tex| & sample include file \\
    |cdocspt3.tex| & sample part file \\
    |cdocspt4.tex| & sample part file \\
    |cdocsdrf.tex| & sample redirection file \\
    |cdocsfn1.tex| & sample redirection file \\
    |cdocsfn2.tex| & sample redirection file \\
    |childdoc.pdf| & manual
\end{tabular}
\end{center}
%
The distribution consists of the files
|README.txt|, |childdoc.ins| and |childdoc.dtx|.
%
\begin{itemize}
\item
Run (pdf)\LaTeX{} on |childdoc.dtx|
to compile the manual |childdoc.pdf| (this file).
\item
Run \LaTeX{} on |childdoc.ins| to create the definitions file |childdoc.def|
and the sample |cdocsamp.tex| with include files
|cdocsch1.tex|, |cdocsch2.tex|, |cdocspt3.tex|, |cdocspt4.tex|,
|cdocsdrf.tex|, |cdocsfn1.tex|, |cdocsfn2.tex|.
Then copy the file |childdoc.def| to an appropriate directory of your \LaTeX{}
distribution, e.g.\ \textit{texmf-root}|/tex/latex/childdoc|.
\end{itemize}

%%%%%%%%%%%%%%%%%%%%%%%%%%%%%%%%%%%%%%%%%%%%%%%%%%%%%%%%%%%%%%%%%%%%%%%%%%%%%%%%
\subsection{Related CTAN Packages}

There are several other packages which offer a similar functionality:
%
\begin{itemize}
\item
The packages
\href{http://ctan.org/pkg/docmute}{\textsf{docmute}},
\href{http://ctan.org/pkg/includex}{\textsf{includex}} and
\href{http://ctan.org/pkg/standalone}{\textsf{standalone}}
provide commands to include only the document body of
a child file thus allowing both files to be compiled individually.
\item
The packages \href{http://ctan.org/pkg/subdocs}{\textsf{subdocs}}
and \href{http://ctan.org/pkg/subfiles}{\textsf{subfiles}}
provide structures in which the main and child documents can be
encapsulated and allowing them to be compiled individually.
The inclusion mechanism is different from the conventional |\include|.
\item
The package \href{http://ctan.org/pkg/combine}{\textsf{combine}}
is an elaborate solution to combine several documents into one.
\end{itemize}
%
See also the CTAN topic \href{http://ctan.org/topic/subdocs}{\textsf{subdocs}}
for further related packages.
The present package differs from the above solutions in that
a document structure constructed with the conventional |\include| mechanism
just needs two extra commands at the top of every file
such that all constituent files can be compiled individually.

%%%%%%%%%%%%%%%%%%%%%%%%%%%%%%%%%%%%%%%%%%%%%%%%%%%%%%%%%%%%%%%%%%%%%%%%%%%%%%%%
%\subsection{Feature Suggestions}
%
%The following is a list of features which may be useful for future
%versions of this package:
%%
%\begin{itemize}
%\item
%\ldots
%\end{itemize}

%%%%%%%%%%%%%%%%%%%%%%%%%%%%%%%%%%%%%%%%%%%%%%%%%%%%%%%%%%%%%%%%%%%%%%%%%%%%%%%%
\subsection{Revision History}

%%%%%%%%%%%%%%%%%%%%%%%%%%%%%%%%%%%%%%%%
\paragraph{v2.0:} 2018/12/30

\begin{itemize}
\item
immediate forward processing
\item
added |\childdocby| mechanism
\item
manual restructured
\end{itemize}

%%%%%%%%%%%%%%%%%%%%%%%%%%%%%%%%%%%%%%%%
\paragraph{v1.6:} 2018/01/17

\begin{itemize}
\item
application for development of include files
\item
corrections to manual
\end{itemize}

%%%%%%%%%%%%%%%%%%%%%%%%%%%%%%%%%%%%%%%%
\paragraph{v1.5:} 2017/05/21

\begin{itemize}
\item
more complete structuring introduced
\item
|\childdocof| introduced
\item
|\childdoc| renamed to |\childdocmain|
\item
|\childredirect| renamed to |\childdocforward| and |\childdocforwardprefix|
and functionality expanded
\end{itemize}

%%%%%%%%%%%%%%%%%%%%%%%%%%%%%%%%%%%%%%%%
\paragraph{v1.0:} 2017/04/27

\begin{itemize}
\item
manual and install package
\item
first version published on CTAN
\end{itemize}

%%%%%%%%%%%%%%%%%%%%%%%%%%%%%%%%%%%%%%%%
\paragraph{v0.6:} 2017/04/26

\begin{itemize}
\item
redirection mechanism added
\end{itemize}

%%%%%%%%%%%%%%%%%%%%%%%%%%%%%%%%%%%%%%%%
\paragraph{v0.5:} 2017/04/26

\begin{itemize}
\item
functionality in definition file
\end{itemize}


%%%%%%%%%%%%%%%%%%%%%%%%%%%%%%%%%%%%%%%%%%%%%%%%%%%%%%%%%%%%%%%%%%%%%%%%%%%%%%%%
%%%%%%%%%%%%%%%%%%%%%%%%%%%%%%%%%%%%%%%%%%%%%%%%%%%%%%%%%%%%%%%%%%%%%%%%%%%%%%%%
%%%%%%%%%%%%%%%%%%%%%%%%%%%%%%%%%%%%%%%%%%%%%%%%%%%%%%%%%%%%%%%%%%%%%%%%%%%%%%%%
\appendix

\settowidth\MacroIndent{\rmfamily\scriptsize 000\ }

 \DocInput{childdoc.dtx}

\end{document}
%</driver>
% \fi
%
% %%%%%%%%%%%%%%%%%%%%%%%%%%%%%%%%%%%%%%%%%%%%%%%%%%%%%%%%%%%%%%%%%%%%%%%%%%%%%%
% %%%%%%%%%%%%%%%%%%%%%%%%%%%%%%%%%%%%%%%%%%%%%%%%%%%%%%%%%%%%%%%%%%%%%%%%%%%%%%
% \section{Sample}
%\iffalse
%<*samplemain>
%\fi
%
% The following presents a sample document
% with two chapters, two parts, a title page,
% a compile flag as well as three forwarding files to set the flag.
% It consists of eight |.tex| files:
% \begin{center}
% \begin{tabular}{ll}
% |cdocsamp.tex|&main file\\
% |cdocsch1.tex|&include file for chapter 1\\
% |cdocsch2.tex|&include file for chapter 2\\
% |cdocspt3.tex|&include file for part 3\\
% |cdocspt4.tex|&include file for part 4\\
% |cdocsdrf.tex|&forwarding file for main file in draft mode\\
% |cdocsfi1.tex|&forwarding file for final version of chapter 1\\
% |cdocsfi2.tex|&forwarding file for final version of chapter 2\\
% \end{tabular}
% \end{center}
% Each of the eight files can be compiled directly by the \LaTeX{} compiler.
%
% %%%%%%%%%%%%%%%%%%%%%%%%%%%%%%%%%%%%%%
% \paragraph{Main File.}
%
% The main file is called |cdocsamp.tex|.
%
% Load the \textsf{childdoc} definitions and
% declare the filename for the main document:
%    \begin{macrocode}
\input{childdoc.def}
\childdocmain{}
%    \end{macrocode}

% Optional override for |\version| flag:
%    \begin{macrocode}
%%\ifchilddoc\else\providecommand{\version}{draft}\fi
%    \end{macrocode}

% Define the default values for the |\version| flag
% (|final| for the main file and |draft| for childs):
%    \begin{macrocode}
\ifchilddoc
\providecommand{\version}{draft}
\else
\providecommand{\version}{final}
\fi
%    \end{macrocode}

% Load the standard document class:
%    \begin{macrocode}
\documentclass[12pt]{article}
%    \end{macrocode}

% Start the document body:
%    \begin{macrocode}
\begin{document}
%    \end{macrocode}

% Declare a title page.
% Print title, part of document being processed and version flag:
%    \begin{macrocode}
\addtocounter{page}{-1}
\begin{center}
{\LARGE\bfseries{}childdoc example\par}
\vspace{1cm}
\ifchilddoc
\ifchilddocmanual part\else chapter\fi:
`\childdocname' of `\childdocjob'\par
\else
main document: `\childdocjob'\par
\fi
version: \version\par
\end{center}
\newpage
%    \end{macrocode}

% Manually include selected file,
% otherwise process as usual:
%    \begin{macrocode}
\ifchilddocmanual
\section*{part `\childdocname'}
\input{\childdocname}
\else
%    \end{macrocode}

% Include the two chapters:
%    \begin{macrocode}
\include{cdocsch1}
\include{cdocsch2}
%    \end{macrocode}

% Include the two parts unless only chapters should be displayed:
%    \begin{macrocode}
\ifchilddoc\else
\section{part three}
\input{cdocspt3}
\section{part four}
\input{cdocspt4}
\fi
%    \end{macrocode}

% Process as usual until here:
%    \begin{macrocode}
\fi
%    \end{macrocode}

% End of document body:
%    \begin{macrocode}
\end{document}
%    \end{macrocode}
%\iffalse
%</samplemain>
%\fi
%
% %%%%%%%%%%%%%%%%%%%%%%%%%%%%%%%%%%%%%%
% \paragraph{Chapter Include Files.}
%
% The include files are called |cdocsch1.tex| and |cdocsch2.tex|.
%
%\iffalse
%<*samplechap1|samplechap2>
%\fi

% Optional override for |\version| flag:
%    \begin{macrocode}
%%\providecommand{\version}{final}
%    \end{macrocode}

% Include the main document:
%    \begin{macrocode}
\input{childdoc.def}
\childdocof{cdocsamp}
%    \end{macrocode}

%\iffalse
%</samplechap1|samplechap2>
%\fi
%
%\iffalse
%<*samplechap1>
%\fi
% Some text for chapter 1:
%    \begin{macrocode}
\section{one}
some text in chapter one
%    \end{macrocode}

%\iffalse
%</samplechap1>
%\fi
% Some text for chapter 2:
%\iffalse
%<*samplechap2>
%\fi
%    \begin{macrocode}
\section{two}
more text in chapter two
%    \end{macrocode}

%\iffalse
%</samplechap2>
%\fi
%
% %%%%%%%%%%%%%%%%%%%%%%%%%%%%%%%%%%%%%%
% \paragraph{Part Include Files.}
%
% The include files are called |cdocspt3.tex| and |cdocspt4.tex|.
%
%\iffalse
%<*samplepart3|samplepart4>
%\fi

% Optional override for |\version| flag:
%    \begin{macrocode}
%%\providecommand{\version}{final}
%    \end{macrocode}

% Include the main document:
%    \begin{macrocode}
\input{childdoc.def}
\childdocby{cdocsamp}
%    \end{macrocode}

%\iffalse
%</samplepart3|samplepart4>
%\fi
%
%\iffalse
%<*samplepart3>
%\fi
% Some text for part 3:
%    \begin{macrocode}
some text in part three
%    \end{macrocode}

%\iffalse
%</samplepart3>
%\fi
% Some text for part 4:
%\iffalse
%<*samplepart4>
%\fi
%    \begin{macrocode}
more text in part four
%    \end{macrocode}

%\iffalse
%</samplepart4>
%\fi
%
% %%%%%%%%%%%%%%%%%%%%%%%%%%%%%%%%%%%%%%
% \paragraph{Forwarding for a Complete Draft.}
%
% The following forwarding file |cdocsdrf.tex|
% compiles the main document in draft mode:
%\iffalse
%<*sampledraft>
%\fi
%    \begin{macrocode}
\def\version{draft}
\input{childdoc.def}
\childdocforward{cdocsamp}
%    \end{macrocode}

%\iffalse
%</sampledraft>
%\fi
%
% %%%%%%%%%%%%%%%%%%%%%%%%%%%%%%%%%%%%%%
% \paragraph{Forwarding for Final Version of the Chapters.}
%
% The following forwarding files |cdocsfn1.tex| and |cdocsfn2.tex|
% (with identical content)
% compile the final versions of the child documents
% |cdocsch1.tex| and |cdocsch2.tex|, respectively:
%\iffalse
%<*samplefinal>
%\fi
%    \begin{macrocode}
\def\version{final}
\input{childdoc.def}
\childdocforwardprefix[cdocsamp]{cdocsfn}{cdocsch}
%    \end{macrocode}

%\iffalse
%</samplefinal>
%\fi
%
% %%%%%%%%%%%%%%%%%%%%%%%%%%%%%%%%%%%%%%
% \paragraph{Command Line Processing.}
%
% The following three command lines generate the output files
% |cdocscld|, |cdocscl1| and |cdocscl2|
% which should be identical to
% |cdocsdrf|, |cdocsch1| and |cdocsfn2|, respectively:
% \begin{center}
% \begin{tabular}{l}
% |latex -jobname cdocscld \|\\
% |  "\def\version{draft}\input{childdoc.def}\childdocforward{cdocsamp}"|\\
% |latex -jobname cdocscl1 \|\\
% |  "\input{childdoc.def}\childdocforward[cdocsamp]{cdocsch1}"|\\
% |latex -jobname cdocscl2 \|\\
% |  "\def\version{final}\input{childdoc.def}\childdocforward{cdocsch2}"|
% \end{tabular}
% \end{center}
% Note that the trailing backslash on each first line
% merely continues the input to the second line
% (for convenient cut ant paste).
% Furthermore, the command |latex| can be replaced by any
% of its alternative versions such as |pdflatex|.
%
% %%%%%%%%%%%%%%%%%%%%%%%%%%%%%%%%%%%%%%%%%%%%%%%%%%%%%%%%%%%%%%%%%%%%%%%%%%%%%%
% %%%%%%%%%%%%%%%%%%%%%%%%%%%%%%%%%%%%%%%%%%%%%%%%%%%%%%%%%%%%%%%%%%%%%%%%%%%%%%
% \section{Implementation}
%\iffalse
%<*package>
%\fi
%
% This section describes the definitions file |childdoc.def|.

% The definitions cannot be loaded using |\usepackage| or |\RequirePackage|
% which has a mechanism to prevent loading a style file more than once.
% When loading the definitions by means of |\input|
% multiple instances have to be prevented manually:
%\iffalse
%This code needs to be before the `\ProvidesFile' directive
%which is defined at the beginning of this file.
%Therefore it is also placed there and commented out here.
%</package>
%<*discard>
%\fi
%    \begin{macrocode}
\ifdefined\childdocmain\endinput\fi
%    \end{macrocode}
%\iffalse
%</discard>
%<*package>
%\fi
%
% \macro{\ifchilddoc}
% \macro{\ifchilddocmanual}
% The conditional |\ifchilddoc| tells whether a
% child (true) or main (false) document is being compiled.
% The conditional |\ifchilddocmanual| tells whether
% the |\includeonly| mechanism is used (false) or
% the selection of child files must be performed manually (true).
% The definitions initialise to false:
%    \begin{macrocode}
\newif\ifchilddoc
\newif\ifchilddocmanual
%    \end{macrocode}

% \macro{\childdocname}
% \macro{\childdocjob}
% The macro |\childdocname| stores the name of the main document
% to be compiled. The macro |\childdocjob| stores the name of
% the document on which the \LaTeX{} compiler was originally invoked.
% The content of |\jobname| cannot be compared
% to filenames specified in the source due to different catcodes.
% The following code rescans |\jobname|, stores the result
% in |\childdocname| and saves a copy in |\childdocjob|:
%    \begin{macrocode}
\edef\childdocname{\scantokens\expandafter{\jobname\noexpand}}
\let\childdocjob\childdocname
%    \end{macrocode}

% \macro{\childdocdisable}
% The macro |\childdocdisable| prevents the main file
% from being processed more than once.
% At this stage, the main document command |\childdocmain|
% is assumed to be called once again where it should do nothing.
% Any subsequent call to it should prevent
% a secondary processing of the main document
% It overwrites the forwarding commands
% |\childdocof| and |\childdocforward|
% with empty macros to prevent further inclusions of the main document:
%    \begin{macrocode}
\newcommand{\childdocdisable}
{
  \renewcommand{\childdocmain}[1]{\renewcommand{\childdocmain}[1]{\endinput}}
  \renewcommand{\childdocof}[1]{}
  \renewcommand{\childdocby}[2][]{}
  \renewcommand{\childdocforward}[2][]{}
  \renewcommand{\childdocdisable}{}
}
%    \end{macrocode}

% \macro{\childdocmain}
% The macro |\childdocmain| is to be called at the top of the main file
% with nothing or the main filename (without extension) as argument.
% First, it breaks loops.
% If the argument is not empty and does not match |\childdocname|
% (which is set by the first inclusion of |childdoc.def|),
% |\ifchilddoc| is set to true, |\includeonly| is applied to the child file
% and |\jobname| is set to the main file
% (for proper handling of |.aux| files):
%    \begin{macrocode}
\newcommand{\childdocmain}[1]
{
  \childdocdisable\childdocmain{}
  \if?#1?\else
    \begingroup
      \def\childdoctmp{#1}
      \ifx\childdoctmp\childdocname
        \def\childdoctmp{}
      \else
        \def\childdoctmp
        {
          \childdoctrue
          \includeonly{\childdocname}
          \def\childdocjob{#1}
          \def\jobname{#1}
        }
      \fi
      \expandafter
    \endgroup
    \childdoctmp
  \fi
}
%    \end{macrocode}

% \macro{\childdocof}
% The command |\childdocof| redirects
% compilation to the main file |#1|.
%    \begin{macrocode}
\newcommand{\childdocof}[1]
{
  \childdocdisable
  \childdoctrue
  \includeonly{\childdocname}
  \def\jobname{#1}
  \def\childdocjob{#1}
  \input{#1}
}
%    \end{macrocode}

% \macro{\childdocby}
% The command |\childdocby| ....
%    \begin{macrocode}
\newcommand{\childdocby}[2][]
{
  \childdocdisable
  \childdoctrue
  \childdocmanualtrue
  \if?#1?\else
    \def\jobname{#2}
  \fi
  \def\childdocjob{#2}
  \input{#2}
  \endinput
}
%    \end{macrocode}

% \macro{\childdocforward}
% The command |\childdocforward| redirects
% compilation to the main file or
% (if the optional argument is given) a child file.
% Parameters are set as if the main file
% or a child file starting with |\childdocof| was compiled.
% Then compilation is handed over to the main file:
%    \begin{macrocode}
\newcommand{\childdocforward}[2][]
{
  \begingroup
    \if?#1?
      \def\childdoctmp
      {
        \def\childdocname{#2}
        \def\childdocjob{#2}
        \def\jobname{#2}
        \input{#2}
        \endinput
      }
    \else
      \def\childdoctmp
      {
        \childdocdisable
        \def\childdocname{#2}
        \childdoctrue
        \includeonly{#2}
        \def\childdocjob{#1}
        \def\jobname{#1}
        \input{#1}
        \endinput
      }
    \fi
    \expandafter
  \endgroup
  \childdoctmp
}
%    \end{macrocode}

% \macro{\childdocforwardprefix}
% The command |\childdocforwardprefix| redirects
% compilation to the main or a child file by means of a pattern.
% The prefix |#1| in the current filename is replaced by |#2|
% and the suffix of the current filename is kept
% (it is assumed that the filename does not contain the substring `|~~~|'
% which is used as a delimiter).
% Compilation is handed over to the new file by |\childdocforward|:
%    \begin{macrocode}
\newcommand{\childdocforwardprefix}[3][]
{
  \begingroup
    \def\childdocextract #2##1~~~{\def\childdoctmp{\childdocforward[#1]{#3##1}}}
    \expandafter\childdocextract\childdocname~~~
    \expandafter
  \endgroup
  \childdoctmp
}
%    \end{macrocode}

% \macro{\childdoc}
% The deprecated macro |\childdoc| is a legacy version of |\childdocmain|:
%    \begin{macrocode}
\newcommand{\childdoc}{\childdocmain}
%    \end{macrocode}

% \macro{\childdocredirect}
% The deprecated macro |\childdocredirect| is a legacy version
% of |\childdocforward| and |\childdocforwardprefix|:
%    \begin{macrocode}
\newcommand{\childdocredirect}[2][]
{
  \begingroup
    \if?#1?
      \def\childdoctmp{\childdocforward{#2}}
    \else
      \def\childdoctmp{\childdocforwardprefix{#1}{#2}}
    \fi
    \expandafter
  \endgroup
  \childdoctmp
}
%    \end{macrocode}

%\iffalse
%</package>
%\fi
%
\endinput
|\\
|\childdocforward{|\textit{main}|}|\\
\end{tabular}
\end{center}
%
or alternatively with:
%
\begin{center}
\begin{tabular}{l}
|% \iffalse
%
% childdoc.dtx Copyright (C) 2017-2018 Niklas Beisert
%
% This work may be distributed and/or modified under the
% conditions of the LaTeX Project Public License, either version 1.3
% of this license or (at your option) any later version.
% The latest version of this license is in
%   http://www.latex-project.org/lppl.txt
% and version 1.3 or later is part of all distributions of LaTeX
% version 2005/12/01 or later.
%
% This work has the LPPL maintenance status `maintained'.
%
% The Current Maintainer of this work is Niklas Beisert.
%
% This work consists of the files childdoc.dtx and childdoc.ins
% and the derived files childdoc.def and cdocsamp.tex with
% cdocsch1.tex, cdocsch2.tex, cdocsdrf.tex, cdocsfn1.tex, cdocsfn2.tex.
%
%<package>\ifdefined\childdocmain\endinput\fi
%<package>\ProvidesFile{childdoc.def}[2018/12/30 v2.0 child document driver]
%<samplemain>\ProvidesFile{cdocsamp.tex}[2018/12/30 v2.0 sample for childdoc]
%<*driver>
%\ProvidesFile{childdoc.drv}[2018/12/30 v2.0 childdoc reference manual file]
\PassOptionsToClass{10pt,a4paper}{article}
\documentclass{ltxdoc}

\usepackage[margin=35mm]{geometry}
\usepackage{hyperref}
\usepackage{hyperxmp}
\usepackage[usenames]{color}

\hypersetup{colorlinks=true}
\hypersetup{pdfstartview=FitH}
\hypersetup{pdfpagemode=UseNone}
\hypersetup{pdfsource={}}
\hypersetup{pdflang={en-UK}}
\hypersetup{pdfcopyright={Copyright 2017-2018 Niklas Beisert.
  This work may be distributed and/or modified under the
  conditions of the LaTeX Project Public License, either version 1.3
  of this license or (at your option) any later version.}}
\hypersetup{pdflicenseurl={http://www.latex-project.org/lppl.txt}}
\hypersetup{pdfcontactaddress={ETH Zurich, ITP, HIT K,
  Wolfgang-Pauli-Strasse 27}}
\hypersetup{pdfcontactpostcode={8093}}
\hypersetup{pdfcontactcity={Zurich}}
\hypersetup{pdfcontactcountry={Switzerland}}
\hypersetup{pdfcontactemail={nbeisert@itp.phys.ethz.ch}}
\hypersetup{pdfcontacturl={http://people.phys.ethz.ch/\xmptilde nbeisert/}}

\newcommand{\secref}[1]{\hyperref[#1]{section \ref*{#1}}}

\parskip1ex
\parindent0pt
\let\olditemize\itemize
\def\itemize{\olditemize\parskip0pt}

\begin{document}

\title{The \textsf{childdoc} Package}
\hypersetup{pdftitle={The childdoc Package}}
\author{Niklas Beisert\\[2ex]
  Institut f\"ur Theoretische Physik\\
  Eidgen\"ossische Technische Hochschule Z\"urich\\
  Wolfgang-Pauli-Strasse 27, 8093 Z\"urich, Switzerland\\[1ex]
  \href{mailto:nbeisert@itp.phys.ethz.ch}
  {\texttt{nbeisert@itp.phys.ethz.ch}}}
\hypersetup{pdfauthor={Niklas Beisert}}
\hypersetup{pdfsubject={Manual for the LaTeX2e Package childdoc}}
\date{30 December 2018, \textsf{v2.0}}
\maketitle

\begin{abstract}\noindent
\textsf{childdoc} is a \LaTeXe{} package
that enables the direct compilation
of document sections included by |\include|
to individual files.
\end{abstract}

\begingroup
\parskip0ex
\tableofcontents
\endgroup

%%%%%%%%%%%%%%%%%%%%%%%%%%%%%%%%%%%%%%%%%%%%%%%%%%%%%%%%%%%%%%%%%%%%%%%%%%%%%%%%
%%%%%%%%%%%%%%%%%%%%%%%%%%%%%%%%%%%%%%%%%%%%%%%%%%%%%%%%%%%%%%%%%%%%%%%%%%%%%%%%
\section{Introduction}

\LaTeX{} provides a mechanism to structure a large document (such as a book)
into a main file and several child files (containing the chapters)
using the |\include| command.
This mechanism is beneficial for documents
which span hundreds of pages in order to
make the source file(s) more manageable.
Moreover, compilation can be restricted to
selected child files by means of the |\includeonly| command.
The latter feature can be used to reduce the compilation time while editing
(this was significantly more useful in the earlier days of \LaTeX{})
or to generate a smaller document which is easier to navigate.
Another application of |\includeonly| is to generate
documents consisting of selected parts of the complete document.

However, there are a few drawbacks of the plain |\include| mechanism:
\begin{itemize}
\item
The child files cannot be compiled on their own,
they can only be compiled via the main file.
A naive editing environment
(such as a text editor with an option
to have the current file processed by \LaTeX)
may require one to switch to the main file before compiling;
attempting to compile the child file produces errors.
\item
The main file must be modified (each time)
to adjust the |\includeonly| command
to the present needs. This easily leaves the main file in a messy state.
\item
The generated document will always carry the filename
of the main document. This is inconvenient if
several child files are to be compiled and
to be kept for distribution.
\end{itemize}

The present package provides a simple interface
to make child files individually compilable by \LaTeX{}.
Compiling a child file then has the same effect as compiling
the main file with an |\includeonly| command
to select the appropriate child.
Moreover the generated document will carry the name of the child
rather than the main file.
This resolves all three above issues.

This feature is meant to make the editing of books,
thesis documents and lecture notes somewhat more convenient.
However, the package can also be used efficiently for
composing a series of documents (such as exercise sheets)
which are typically distributed individually.
It then assists the author in generating the individual documents
(potentially in different versions)
as well as a document containing the collected series.
Another application is in developing style files
or other kinds of included material
where compilation of the style file could redirect
to a sample or test file.

%%%%%%%%%%%%%%%%%%%%%%%%%%%%%%%%%%%%%%%%%%%%%%%%%%%%%%%%%%%%%%%%%%%%%%%%%%%%%%%%
%%%%%%%%%%%%%%%%%%%%%%%%%%%%%%%%%%%%%%%%%%%%%%%%%%%%%%%%%%%%%%%%%%%%%%%%%%%%%%%%
\section{Usage}

First of all, the package \textsf{childdoc} is \emph{not} a standard
\LaTeXe{} |.sty| style file! Therefore it needs to be invoked in
a non-standard way.

%%%%%%%%%%%%%%%%%%%%%%%%%%%%%%%%%%%%%%%%%%%%%%%%%%%%%%%%%%%%%%%%%%%%%%%%%%%%%%%%
\subsection{Included Files}
\label{sec:include}

%%%%%%%%%%%%%%%%%%%%%%%%%%%%%%%%%%%%%%%%
\DescribeMacro{\childdocmain}
To use the package, add the commands
\begin{center}
\begin{tabular}{l}
|\input{childdoc.def}|\\
|\childdocmain{}|\\
\end{tabular}
\end{center}
at the very top of the main \LaTeX{} file,
in particular \emph{before} the |\documentclass| statement!
The argument of |\childdocmain| should be left empty
(but it must be present).

%%%%%%%%%%%%%%%%%%%%%%%%%%%%%%%%%%%%%%%%
\DescribeMacro{\childdocof}
Furthermore, add the commands
\begin{center}
\begin{tabular}{l}
|\input{childdoc.def}|\\
|\childdocof{|\textit{main}|}|\\
\end{tabular}
\end{center}
at the top of every child file \textit{child}
which is included by |\include{|\textit{child}|}|
from within the main file
(or at least for those files to be compiled individually).
The argument \textit{main} must be the filename of the main file.

There are a couple of
considerations in setting up the main and child documents:

%%%%%%%%%%%%%%%%%%%%%%%%%%%%%%%%%%%%%%%%
\paragraph{Restrictions.}

Please note the following restrictions:
\begin{itemize}
\item
|\childdocmain| must be called with one argument \textit{main}
to ensure compatibility with earlier version of the package.
It must either be empty (|\childdocmain{}|)
or precisely match the filename of the main file in which it is specified.
See \secref{sec:detection} for further information.
\item
The filename \textit{main} must be specified without the |.tex| extension.
\item
The filename \textit{main} is case sensitive
(even in case-insensitive file systems)
due to internal string comparison.
\item
The argument \textit{main} should be fully expanded, it cannot be a macro.
\item
Subdirectories and special characters should be avoided in filenames.
\item
The command |\childdocmain{|\textit{main}|}| must be followed by a whitespace.
It should not be followed immediately by another command
or by a comment mark `|%|'.
This is because the \TeX{} parser reads the token immediately following
the argument of |\childdocmain| and puts it
at the beginning of every child section;
however, a white\-space is ignored.
\end{itemize}

%%%%%%%%%%%%%%%%%%%%%%%%%%%%%%%%%%%%%%%%
\paragraph{Content of Main File.}

It is advisable to place all content in the child files included by |\include|.
Any output contained in the main file will appear in all child documents
unless suppressed manually;
it cannot be suppressed automatically by the |\includeonly| directive
and thus should normally be avoided.
A method to include some content in the main file
by means of conditional processing is described in \secref{sec:conditional}.

%%%%%%%%%%%%%%%%%%%%%%%%%%%%%%%%%%%%%%%%
\paragraph{Page Numbering.}

When only a part of the document is compiled,
the appropriate numbering of pages
(as well as other status parameters)
is determined from the |.aux| files.
The latter contain information from previous passes.
However this information needs to propagate through
all intermediate child documents.
Therefore the page numbering in child documents may well
be inconsistent until the complete document is compiled at least once.

A useful (if unconventional) way to always ensure a consistent
page numbering is to restart the numbering in each child document
and denote the pages by `\textit{child}|.|\textit{page}'
where \textit{child} represents the chapter/section number of the child file.
This can be achieved by the command
|\numberwithin{page}{|\textit{child}|}|
of the \textsf{amsmath} package
where \textit{child} can be |chapter| or |section|
depending on the chosen structuring.
Alternatively, one can modify the macro |\thepage| appropriately
and reset the counter |page| at the start of each child file.

%%%%%%%%%%%%%%%%%%%%%%%%%%%%%%%%%%%%%%%%%%%%%%%%%%%%%%%%%%%%%%%%%%%%%%%%%%%%%%%%
\subsection{Conditional Processing}
\label{sec:conditional}

The package provides a mechanism to compile different versions
of a document. To customise the versions further some conditional processing
can come in handy to distinguish which version is being compiled.
The package provides two macros to describe the compilation context:

%%%%%%%%%%%%%%%%%%%%%%%%%%%%%%%%%%%%%%%%
\DescribeMacro{\ifchilddoc}
The conditional |\ifchilddoc| distinguishes between the compilation of
child documents and the main document:
%
\begin{center}
|\ifchilddoc |\textit{child-code}| |[|\||else |\textit{main-code}]| \||fi|
\end{center}

%%%%%%%%%%%%%%%%%%%%%%%%%%%%%%%%%%%%%%%%
\DescribeMacro{\childdocname}
\DescribeMacro{\childdocjob}
The macro |\childdocname| contains the filename (without extension)
of the main or child file being processed.
Note that |\childdocjob| will always contain the name of the main file.

%%%%%%%%%%%%%%%%%%%%%%%%%%%%%%%%%%%%%%%%
\paragraph{Title Page.}

Conditional processing can be used to include a title or banner page
in the main document when proper precautions are taken.
Importantly, the code in the main file should ensure that the page counter
(as well as other status parameters which are stored in the |.aux| files)
takes the same value after the conditional processing.
Otherwise the page numbers may take divergent values
depending on which part is compiled.

For example, a title page could be declared by:
%
\begin{center}
\begin{tabular}{l}
|\ifchilddoc\||else|\\
|\addtocounter{page}{-1}|\\
\textit{code for title page}\\
|\newpage|\\
|\||fi|
\end{tabular}
\end{center}
%
A banner page for the child documents can be generated by:
%
\begin{center}
\begin{tabular}{l}
|\ifchilddoc|\\
|\addtocounter{page}{-1}|\\
\textit{code for banner page}\\
|\newpage|\\
|\||fi|
\end{tabular}
\end{center}
%
Here one could write a message such as:
\begin{center}
|This is the part \childdocname{} of \childdocjob{}.|
\end{center}

%%%%%%%%%%%%%%%%%%%%%%%%%%%%%%%%%%%%%%%%%%%%%%%%%%%%%%%%%%%%%%%%%%%%%%%%%%%%%%%%
\subsection{Flags}
\label{sec:flags}

The package makes it easy to generate different versions
of the main or child documents.
To this end compilation flags can be defined
and assigned different default values.
They will be particularly useful in conjunction
with the forwarding mechanism described in \secref{sec:forward}.

For example, it may be useful to have a flag |\version|
which can be set to |draft| or |final|.
The document source will contain some conditional code
depending on the value of |\version|.
Suppose further, the flag should default to |final| for the main file
and to |draft| for child files
which is a natural assignment for editing the document.
This is achieved by placing the following code
in the preamble of the main document
(below the |\childdocmain| directive):
%
\begin{center}
\begin{tabular}{l}
|\ifchilddoc|\\
|\providecommand{\version}{draft}|\\
|\||else|\\
|\providecommand{\version}{final}|\\
|\||fi|
\end{tabular}
\end{center}
%
The definition by |\providecommand| makes sure
that previous definitions are not overwritten.
Further statements |\providecommand{\version}{...}|
can thus be added before the above code to override it.

For the main file, one might add a line
(between |\childdocmain| and the above block)
%
\begin{center}
|%\ifchilddoc\||else\providecommand{\version}{draft}\||fi|
\end{center}
%
which can be uncommented to produce a draft version.
Likewise one can add a line to the very top of a child file
(above the |\childdocof{|\textit{main}|}| directive)
%
\begin{center}
|%\providecommand{\version}{final}|
\end{center}
%
which can be uncommented to produce the final version of this child document.

%%%%%%%%%%%%%%%%%%%%%%%%%%%%%%%%%%%%%%%%%%%%%%%%%%%%%%%%%%%%%%%%%%%%%%%%%%%%%%%%
\subsection{Forwarding}
\label{sec:forward}

Different versions of the main or child documents
using compilation flags as described in \secref{sec:flags}
can be (permanently) stored in different files
for convenient compilation, viewing and distribution.
To this end, the package defines a command
to pass on compilation to a different file:

%%%%%%%%%%%%%%%%%%%%%%%%%%%%%%%%%%%%%%%%
\DescribeMacro{\childdocforward}
The command |\childdocforward| redirects processing to
another source file:
%
\begin{center}
\begin{tabular}{l}
|\input{childdoc.def}|\\
|\childdocforward[|\textit{main}|]{|\textit{dest}|}|\\
\end{tabular}
\end{center}
%
The argument \textit{dest} is the destination file
(without extension).
It should be the main file or one of the child files.
Note that further \textsf{childdoc} directives
such as |\childdocof| and |\childdocforward|
in the indicated file will be processed in this form.
The optional argument \textit{main}
passes on directly to the main file \textit{main}
while pretending to compile the child \textit{dest}.
This form behaves as if \textit{dest}
issues |\childdocof{|\textit{main}|}| right away,
and no further \textsf{childdoc} directives will be processed.

%%%%%%%%%%%%%%%%%%%%%%%%%%%%%%%%%%%%%%%%
\DescribeMacro{\...prefix}
In the alternative form |\childdocforwardprefix|,
%
\begin{center}
\begin{tabular}{l}
|\input{childdoc.def}|\\
|\childdocforwardprefix[|\textit{main}|]{|\textit{prefix}|}{|\textit{dest}|}|
\end{tabular}
\end{center}
%
the destination file is determined by a pattern
depending on the current file:
To make this work, the current file must be called
`{\textit{prefix}\hspace{0.2em}\textit{suffix}}'
with \textit{prefix} matching precisely the argument.
Processing is then passed on to the file
`{\textit{dest}\hspace{0.2em}\textit{suffix}}'.
Surely, the same effect is achieved by
directly specifying the
argument `{\textit{dest}\hspace{0.2em}\textit{suffix}}'
in the first form.
However, that requires to set up a different file
for each child. With the alternative form of the command
all these files can have exactly the same content
which simplifies setting them up and maintaining them.

For example, the following file |draft.tex|
with a compilation flag |\version| as described in \secref{sec:flags}
compiles the main document as a draft:
%
\begin{center}
\begin{tabular}{l}
|\def\version{draft}|\\
|\input{childdoc.def}|\\
|\childdocforward{|\textit{main}|}|
\end{tabular}
\end{center}
%
Likewise, the following files |final|\textit{nn}|.tex|
compile the final version of the child document
|child|\textit{nn}|.tex|:
%
\begin{center}
\begin{tabular}{l}
|\def\version{final}|\\
|\input{childdoc.def}|\\
|\childdocforwardprefix{final}{child}|
\end{tabular}
\end{center}
%

Note that when several versions of a main file and/or of each child file
are to be generated, it may be convenient to set up a |Makefile| or
shell script to automatise the process.

%%%%%%%%%%%%%%%%%%%%%%%%%%%%%%%%%%%%%%%%%%%%%%%%%%%%%%%%%%%%%%%%%%%%%%%%%%%%%%%%
\subsection{Command Line Processing}
\label{sec:commandline}

The effect of redirection files can also be achieved by invoking
the \LaTeX{} compiler with a more elaborate command line.
Most conveniently this should be done as part
of a shell script or a |Makefile|.

When using \textsf{childdoc} in the main file, the following
command lines effectively perform a redirection
(note that depending on the shell being used,
backslashes may have to be doubled: `|\|' $\to$ `|\\|'):
%
\begin{center}
|... -jobname "|\textit{target}|" |\\|"|[\textit{flags}]%
|\input{childdoc.def}\childdocforward[|\textit{main}|]{|\textit{dest}|}"|
\end{center}
%
Here \textit{target} is the name of the output file,
\textit{main} is the name of the main file
and \textit{dest} is the name of the main or child file to be processed
(all filenames without extensions).
The optional argument \textit{main} can be omitted
if \textit{main} matches \textit{dest}.
Optionally, compilation \textit{flags} can be defined via |\def| commands.
This command line makes the \TeX{} engine believe
it is compiling the file \textit{target}
whose content is specified as the latter parameter.
The provided code then forwards the processing to
\textit{main} or \textit{dest} as described in \secref{sec:forward}.

%%%%%%%%%%%%%%%%%%%%%%%%%%%%%%%%%%%%%%%%%%%%%%%%%%%%%%%%%%%%%%%%%%%%%%%%%%%%%%%%
\subsection{Include by Input}
\label{sec:input}

Including child documents by |\include| has some restrictions by design.
Most notably, the content of a child document always occupies
its own set of pages; pages cannot be shared between child documents.
Usually, this behaviour makes perfect sense
because each child document contain an essential part of the document.
However, in some situations it may be desirable to compose
a document from a collection of parts
without having mandatory page breaks between then.
For this case, the package
provides a mechanism to include parts
by |\input| which can also be processed individually.
However, by construction this mechanism
requires manual handling of the content to be output.

%%%%%%%%%%%%%%%%%%%%%%%%%%%%%%%%%%%%%%%%
\DescribeMacro{\ifchilddocmanual}
The main file should be prepared as usual, see \secref{sec:include}.
However, the document body must make a distinction
between processing of an individual part and of the main document, e.g.:
%
\begin{center}
\begin{tabular}{l}
|\ifchilddocmanual|\\
|\input{\childdocname}|\\
|\||else|\\
\textit{document body with }|\input{|\textit{part}|}|\\
|\||fi|
\end{tabular}
\end{center}
%
The conditional |\ifchilddocmanual| is true whenever
a part to be included by |\input| is being compiled,
and the name of the part is stored in |\childdocname|.

%%%%%%%%%%%%%%%%%%%%%%%%%%%%%%%%%%%%%%%%
\DescribeMacro{\childdocby}
Each part to be included by |\input| should start with:
%
\begin{center}
\begin{tabular}{l}
|\input{childdoc.def}|\\
|\childdocby{|\textit{main}|}|\\
\end{tabular}
\end{center}
%
The directive |\childdocby| is similar to |\childdocof|
described in \secref{sec:include},
but the subsequent selection of content must be done manually.
To that end, both |\ifchilddoc| and |\ifchilddocmanual|
will be true upon processing of a part,
and the name of the part is stored in |\childdocname|.
Note that |\jobname| will be set to the filename of the current part
so that each part receives an individual |.aux| file
that does not interfere with the |.aux| file(s) of the main document.
This behaviour can be altered by the alternative form
|\childdocby[*]{|\textit{main}|}| (with a non-empty optional argument)
which uses the |.aux| file of the main document
by setting |\jobname| to \textit{main}.

%%%%%%%%%%%%%%%%%%%%%%%%%%%%%%%%%%%%%%%%%%%%%%%%%%%%%%%%%%%%%%%%%%%%%%%%%%%%%%%%
\subsection{Driver Development}
\label{sec:driver}

The \textsf{childdoc} mechanism can also be use for the development
of definition files such as \LaTeX{} styles or classes.
This case differs from the above setup with multiple parts
included by |\include| in that no |\includeonly| should be invoked.
This can be achieved by starting the include file
(before |\ProvidesPackage|) with:
%
\begin{center}
\begin{tabular}{l}
|\input{childdoc.def}|\\
|\childdocforward{|\textit{main}|}|\\
\end{tabular}
\end{center}
%
or alternatively with:
%
\begin{center}
\begin{tabular}{l}
|\input{childdoc.def}|\\
|\childdocby{|\textit{main}|}|\\
\end{tabular}
\end{center}
%
Both forms have slightly different effects as described above.
The main file is prepared as usual, see \secref{sec:include}.

%%%%%%%%%%%%%%%%%%%%%%%%%%%%%%%%%%%%%%%%%%%%%%%%%%%%%%%%%%%%%%%%%%%%%%%%%%%%%%%%
\subsection{Legacy Detection}
\label{sec:detection}

The directive |\childdocmain| in the main file can detect
whether the complete document or merely a child is to be compiled
even without using the directive |\childdocof|.
This method is deprecated because it is less robust
and there is no compelling reason to use it;
it is merely provided for backward compatibility
and it may be removed in future versions.

If the detection mechanism is to be used,
it is mandatory to correctly specify
the filename of the main file as the argument of |\childdocmain|:
%
\begin{center}
\begin{tabular}{l}
|\input{childdoc.def}|\\
|\childdocmain{|\textit{main}|}|\\
\end{tabular}
\end{center}
%
If |\jobname| does not match the argument \textit{main} of |\childdocmain|,
it is assumed that |\jobname| points to the child file to be compiled.
When using |\childdocmain| with the main file specified as argument,
it suffices to start a child file
with just |\input{|\textit{main}|}|
without loading of the package and using |\childdocof|.
If instead all processing is done
with the appropriate \textsf{childdoc} directives,
the argument of \textit{main} of |\childdocmain| can be empty.

An alternative version of the command line processing described
in \secref{sec:commandline} using the detection mechanism reads:
%
\begin{center}
|... -jobname "|\textit{target}|" "|[\textit{flags}]%
[|\def\jobname{|\textit{dest}|}|]|\input{|\textit{main}|}"|
\end{center}

%%%%%%%%%%%%%%%%%%%%%%%%%%%%%%%%%%%%%%%%%%%%%%%%%%%%%%%%%%%%%%%%%%%%%%%%%%%%%%%%
\subsection{Manual Code}
\label{sec:manual}

In case one cannot be certain whether the definitions file |childdoc.def|
is installed on the target \TeX{} distribution
and one prefers not to ship it,
it is conceivable to paste a few relevant commands into the sources.

To that end, drop all statements |\input{childdoc.def}|
and perform the replacements as outlined below.
Instead of |\childdocmain{|\textit{main}|}| add the following code
to the top of the main file:
%
\begin{center}
\begin{tabular}{l}
|\||ifdefined\childdocname\endinput\||fi\newif\ifchilddoc|\\
|\edef\childdocname{\scantokens\expandafter{\jobname\noexpand}}|\\
|\def\childdocmain{|\textit{main}|}\||ifx\childdocmain\childdocname\||else|\\
|\childdoctrue\includeonly{\childdocname}\let\jobname\childdocmain\||fi|\\
\end{tabular}
\end{center}
%
Instead of |\childdocof{|\textit{main}|}| just include the main file
at the top of each child file:
%
\begin{center}
|\input{|\textit{main}|}|
\end{center}
%
A simple redirection |\childdocforward{|\textit{dest}|}| is achieved by:
%
\begin{center}
|\def\jobname{|\textit{dest}|}\input{\jobname}|
\end{center}
%
The redirection with prefix
|\childdocforwardprefix[|\textit{prefix}|]{|\textit{dest}|}|
is accomplished by:
%
\begin{center}
\begin{tabular}{l}
|{\edef\jobname{\scantokens\expandafter{\jobname\noexpand}}|\\
|\def\redirectjob |\textit{prefix}|#1~~~{\gdef\jobname{|\textit{dest}|#1}}|\\
|\expandafter\redirectjob\jobname~~~}\input{\jobname}|
\end{tabular}
\end{center}

In an alternative approach,
child documents can be compiled by a specific command line
without additional code or specific definitions:
%
\begin{center}
|... -jobname "|\textit{target}|" "|[\textit{flags}]%
|\includeonly{|\textit{dest}|}\input{|\textit{main}|}"|
\end{center}
%

%%%%%%%%%%%%%%%%%%%%%%%%%%%%%%%%%%%%%%%%%%%%%%%%%%%%%%%%%%%%%%%%%%%%%%%%%%%%%%%%
%%%%%%%%%%%%%%%%%%%%%%%%%%%%%%%%%%%%%%%%%%%%%%%%%%%%%%%%%%%%%%%%%%%%%%%%%%%%%%%%
\section{Information}

%%%%%%%%%%%%%%%%%%%%%%%%%%%%%%%%%%%%%%%%%%%%%%%%%%%%%%%%%%%%%%%%%%%%%%%%%%%%%%%%
\subsection{Copyright}

Copyright \copyright{} 2017--2018 Niklas Beisert

This work may be distributed and/or modified under the
conditions of the \LaTeX{} Project Public License, either version 1.3
of this license or (at your option) any later version.
The latest version of this license is in
  \url{http://www.latex-project.org/lppl.txt}
and version 1.3 or later is part of all distributions of \LaTeX{}
version 2005/12/01 or later.

This work has the LPPL maintenance status `maintained'.

The Current Maintainer of this work is Niklas Beisert.

This work consists of the files |README.txt|, |childdoc.ins| and |childdoc.dtx|
as well as the derived files |childdoc.def|, |cdocsamp.tex|
with |cdocsch1.tex|, |cdocsch2.tex|, |cdocspt3.tex|, |cdocspt4.tex|,
|cdocsdrf.tex|, |cdocsfn1.tex|, |cdocsfn2.tex|
as well as |childdoc.pdf|.

%%%%%%%%%%%%%%%%%%%%%%%%%%%%%%%%%%%%%%%%%%%%%%%%%%%%%%%%%%%%%%%%%%%%%%%%%%%%%%%%
\subsection{Files and Installation}

The package consists of the files:
%
\begin{center}
\begin{tabular}{ll}
    |README.txt|   & readme file \\
    |childdoc.ins| & installation file \\
    |childdoc.dtx| & source file \\
    |childdoc.def| & definition file \\
    |cdocsamp.tex| & sample main file \\
    |cdocsch1.tex| & sample include file \\
    |cdocsch2.tex| & sample include file \\
    |cdocspt3.tex| & sample part file \\
    |cdocspt4.tex| & sample part file \\
    |cdocsdrf.tex| & sample redirection file \\
    |cdocsfn1.tex| & sample redirection file \\
    |cdocsfn2.tex| & sample redirection file \\
    |childdoc.pdf| & manual
\end{tabular}
\end{center}
%
The distribution consists of the files
|README.txt|, |childdoc.ins| and |childdoc.dtx|.
%
\begin{itemize}
\item
Run (pdf)\LaTeX{} on |childdoc.dtx|
to compile the manual |childdoc.pdf| (this file).
\item
Run \LaTeX{} on |childdoc.ins| to create the definitions file |childdoc.def|
and the sample |cdocsamp.tex| with include files
|cdocsch1.tex|, |cdocsch2.tex|, |cdocspt3.tex|, |cdocspt4.tex|,
|cdocsdrf.tex|, |cdocsfn1.tex|, |cdocsfn2.tex|.
Then copy the file |childdoc.def| to an appropriate directory of your \LaTeX{}
distribution, e.g.\ \textit{texmf-root}|/tex/latex/childdoc|.
\end{itemize}

%%%%%%%%%%%%%%%%%%%%%%%%%%%%%%%%%%%%%%%%%%%%%%%%%%%%%%%%%%%%%%%%%%%%%%%%%%%%%%%%
\subsection{Related CTAN Packages}

There are several other packages which offer a similar functionality:
%
\begin{itemize}
\item
The packages
\href{http://ctan.org/pkg/docmute}{\textsf{docmute}},
\href{http://ctan.org/pkg/includex}{\textsf{includex}} and
\href{http://ctan.org/pkg/standalone}{\textsf{standalone}}
provide commands to include only the document body of
a child file thus allowing both files to be compiled individually.
\item
The packages \href{http://ctan.org/pkg/subdocs}{\textsf{subdocs}}
and \href{http://ctan.org/pkg/subfiles}{\textsf{subfiles}}
provide structures in which the main and child documents can be
encapsulated and allowing them to be compiled individually.
The inclusion mechanism is different from the conventional |\include|.
\item
The package \href{http://ctan.org/pkg/combine}{\textsf{combine}}
is an elaborate solution to combine several documents into one.
\end{itemize}
%
See also the CTAN topic \href{http://ctan.org/topic/subdocs}{\textsf{subdocs}}
for further related packages.
The present package differs from the above solutions in that
a document structure constructed with the conventional |\include| mechanism
just needs two extra commands at the top of every file
such that all constituent files can be compiled individually.

%%%%%%%%%%%%%%%%%%%%%%%%%%%%%%%%%%%%%%%%%%%%%%%%%%%%%%%%%%%%%%%%%%%%%%%%%%%%%%%%
%\subsection{Feature Suggestions}
%
%The following is a list of features which may be useful for future
%versions of this package:
%%
%\begin{itemize}
%\item
%\ldots
%\end{itemize}

%%%%%%%%%%%%%%%%%%%%%%%%%%%%%%%%%%%%%%%%%%%%%%%%%%%%%%%%%%%%%%%%%%%%%%%%%%%%%%%%
\subsection{Revision History}

%%%%%%%%%%%%%%%%%%%%%%%%%%%%%%%%%%%%%%%%
\paragraph{v2.0:} 2018/12/30

\begin{itemize}
\item
immediate forward processing
\item
added |\childdocby| mechanism
\item
manual restructured
\end{itemize}

%%%%%%%%%%%%%%%%%%%%%%%%%%%%%%%%%%%%%%%%
\paragraph{v1.6:} 2018/01/17

\begin{itemize}
\item
application for development of include files
\item
corrections to manual
\end{itemize}

%%%%%%%%%%%%%%%%%%%%%%%%%%%%%%%%%%%%%%%%
\paragraph{v1.5:} 2017/05/21

\begin{itemize}
\item
more complete structuring introduced
\item
|\childdocof| introduced
\item
|\childdoc| renamed to |\childdocmain|
\item
|\childredirect| renamed to |\childdocforward| and |\childdocforwardprefix|
and functionality expanded
\end{itemize}

%%%%%%%%%%%%%%%%%%%%%%%%%%%%%%%%%%%%%%%%
\paragraph{v1.0:} 2017/04/27

\begin{itemize}
\item
manual and install package
\item
first version published on CTAN
\end{itemize}

%%%%%%%%%%%%%%%%%%%%%%%%%%%%%%%%%%%%%%%%
\paragraph{v0.6:} 2017/04/26

\begin{itemize}
\item
redirection mechanism added
\end{itemize}

%%%%%%%%%%%%%%%%%%%%%%%%%%%%%%%%%%%%%%%%
\paragraph{v0.5:} 2017/04/26

\begin{itemize}
\item
functionality in definition file
\end{itemize}


%%%%%%%%%%%%%%%%%%%%%%%%%%%%%%%%%%%%%%%%%%%%%%%%%%%%%%%%%%%%%%%%%%%%%%%%%%%%%%%%
%%%%%%%%%%%%%%%%%%%%%%%%%%%%%%%%%%%%%%%%%%%%%%%%%%%%%%%%%%%%%%%%%%%%%%%%%%%%%%%%
%%%%%%%%%%%%%%%%%%%%%%%%%%%%%%%%%%%%%%%%%%%%%%%%%%%%%%%%%%%%%%%%%%%%%%%%%%%%%%%%
\appendix

\settowidth\MacroIndent{\rmfamily\scriptsize 000\ }

 \DocInput{childdoc.dtx}

\end{document}
%</driver>
% \fi
%
% %%%%%%%%%%%%%%%%%%%%%%%%%%%%%%%%%%%%%%%%%%%%%%%%%%%%%%%%%%%%%%%%%%%%%%%%%%%%%%
% %%%%%%%%%%%%%%%%%%%%%%%%%%%%%%%%%%%%%%%%%%%%%%%%%%%%%%%%%%%%%%%%%%%%%%%%%%%%%%
% \section{Sample}
%\iffalse
%<*samplemain>
%\fi
%
% The following presents a sample document
% with two chapters, two parts, a title page,
% a compile flag as well as three forwarding files to set the flag.
% It consists of eight |.tex| files:
% \begin{center}
% \begin{tabular}{ll}
% |cdocsamp.tex|&main file\\
% |cdocsch1.tex|&include file for chapter 1\\
% |cdocsch2.tex|&include file for chapter 2\\
% |cdocspt3.tex|&include file for part 3\\
% |cdocspt4.tex|&include file for part 4\\
% |cdocsdrf.tex|&forwarding file for main file in draft mode\\
% |cdocsfi1.tex|&forwarding file for final version of chapter 1\\
% |cdocsfi2.tex|&forwarding file for final version of chapter 2\\
% \end{tabular}
% \end{center}
% Each of the eight files can be compiled directly by the \LaTeX{} compiler.
%
% %%%%%%%%%%%%%%%%%%%%%%%%%%%%%%%%%%%%%%
% \paragraph{Main File.}
%
% The main file is called |cdocsamp.tex|.
%
% Load the \textsf{childdoc} definitions and
% declare the filename for the main document:
%    \begin{macrocode}
\input{childdoc.def}
\childdocmain{}
%    \end{macrocode}

% Optional override for |\version| flag:
%    \begin{macrocode}
%%\ifchilddoc\else\providecommand{\version}{draft}\fi
%    \end{macrocode}

% Define the default values for the |\version| flag
% (|final| for the main file and |draft| for childs):
%    \begin{macrocode}
\ifchilddoc
\providecommand{\version}{draft}
\else
\providecommand{\version}{final}
\fi
%    \end{macrocode}

% Load the standard document class:
%    \begin{macrocode}
\documentclass[12pt]{article}
%    \end{macrocode}

% Start the document body:
%    \begin{macrocode}
\begin{document}
%    \end{macrocode}

% Declare a title page.
% Print title, part of document being processed and version flag:
%    \begin{macrocode}
\addtocounter{page}{-1}
\begin{center}
{\LARGE\bfseries{}childdoc example\par}
\vspace{1cm}
\ifchilddoc
\ifchilddocmanual part\else chapter\fi:
`\childdocname' of `\childdocjob'\par
\else
main document: `\childdocjob'\par
\fi
version: \version\par
\end{center}
\newpage
%    \end{macrocode}

% Manually include selected file,
% otherwise process as usual:
%    \begin{macrocode}
\ifchilddocmanual
\section*{part `\childdocname'}
\input{\childdocname}
\else
%    \end{macrocode}

% Include the two chapters:
%    \begin{macrocode}
\include{cdocsch1}
\include{cdocsch2}
%    \end{macrocode}

% Include the two parts unless only chapters should be displayed:
%    \begin{macrocode}
\ifchilddoc\else
\section{part three}
\input{cdocspt3}
\section{part four}
\input{cdocspt4}
\fi
%    \end{macrocode}

% Process as usual until here:
%    \begin{macrocode}
\fi
%    \end{macrocode}

% End of document body:
%    \begin{macrocode}
\end{document}
%    \end{macrocode}
%\iffalse
%</samplemain>
%\fi
%
% %%%%%%%%%%%%%%%%%%%%%%%%%%%%%%%%%%%%%%
% \paragraph{Chapter Include Files.}
%
% The include files are called |cdocsch1.tex| and |cdocsch2.tex|.
%
%\iffalse
%<*samplechap1|samplechap2>
%\fi

% Optional override for |\version| flag:
%    \begin{macrocode}
%%\providecommand{\version}{final}
%    \end{macrocode}

% Include the main document:
%    \begin{macrocode}
\input{childdoc.def}
\childdocof{cdocsamp}
%    \end{macrocode}

%\iffalse
%</samplechap1|samplechap2>
%\fi
%
%\iffalse
%<*samplechap1>
%\fi
% Some text for chapter 1:
%    \begin{macrocode}
\section{one}
some text in chapter one
%    \end{macrocode}

%\iffalse
%</samplechap1>
%\fi
% Some text for chapter 2:
%\iffalse
%<*samplechap2>
%\fi
%    \begin{macrocode}
\section{two}
more text in chapter two
%    \end{macrocode}

%\iffalse
%</samplechap2>
%\fi
%
% %%%%%%%%%%%%%%%%%%%%%%%%%%%%%%%%%%%%%%
% \paragraph{Part Include Files.}
%
% The include files are called |cdocspt3.tex| and |cdocspt4.tex|.
%
%\iffalse
%<*samplepart3|samplepart4>
%\fi

% Optional override for |\version| flag:
%    \begin{macrocode}
%%\providecommand{\version}{final}
%    \end{macrocode}

% Include the main document:
%    \begin{macrocode}
\input{childdoc.def}
\childdocby{cdocsamp}
%    \end{macrocode}

%\iffalse
%</samplepart3|samplepart4>
%\fi
%
%\iffalse
%<*samplepart3>
%\fi
% Some text for part 3:
%    \begin{macrocode}
some text in part three
%    \end{macrocode}

%\iffalse
%</samplepart3>
%\fi
% Some text for part 4:
%\iffalse
%<*samplepart4>
%\fi
%    \begin{macrocode}
more text in part four
%    \end{macrocode}

%\iffalse
%</samplepart4>
%\fi
%
% %%%%%%%%%%%%%%%%%%%%%%%%%%%%%%%%%%%%%%
% \paragraph{Forwarding for a Complete Draft.}
%
% The following forwarding file |cdocsdrf.tex|
% compiles the main document in draft mode:
%\iffalse
%<*sampledraft>
%\fi
%    \begin{macrocode}
\def\version{draft}
\input{childdoc.def}
\childdocforward{cdocsamp}
%    \end{macrocode}

%\iffalse
%</sampledraft>
%\fi
%
% %%%%%%%%%%%%%%%%%%%%%%%%%%%%%%%%%%%%%%
% \paragraph{Forwarding for Final Version of the Chapters.}
%
% The following forwarding files |cdocsfn1.tex| and |cdocsfn2.tex|
% (with identical content)
% compile the final versions of the child documents
% |cdocsch1.tex| and |cdocsch2.tex|, respectively:
%\iffalse
%<*samplefinal>
%\fi
%    \begin{macrocode}
\def\version{final}
\input{childdoc.def}
\childdocforwardprefix[cdocsamp]{cdocsfn}{cdocsch}
%    \end{macrocode}

%\iffalse
%</samplefinal>
%\fi
%
% %%%%%%%%%%%%%%%%%%%%%%%%%%%%%%%%%%%%%%
% \paragraph{Command Line Processing.}
%
% The following three command lines generate the output files
% |cdocscld|, |cdocscl1| and |cdocscl2|
% which should be identical to
% |cdocsdrf|, |cdocsch1| and |cdocsfn2|, respectively:
% \begin{center}
% \begin{tabular}{l}
% |latex -jobname cdocscld \|\\
% |  "\def\version{draft}\input{childdoc.def}\childdocforward{cdocsamp}"|\\
% |latex -jobname cdocscl1 \|\\
% |  "\input{childdoc.def}\childdocforward[cdocsamp]{cdocsch1}"|\\
% |latex -jobname cdocscl2 \|\\
% |  "\def\version{final}\input{childdoc.def}\childdocforward{cdocsch2}"|
% \end{tabular}
% \end{center}
% Note that the trailing backslash on each first line
% merely continues the input to the second line
% (for convenient cut ant paste).
% Furthermore, the command |latex| can be replaced by any
% of its alternative versions such as |pdflatex|.
%
% %%%%%%%%%%%%%%%%%%%%%%%%%%%%%%%%%%%%%%%%%%%%%%%%%%%%%%%%%%%%%%%%%%%%%%%%%%%%%%
% %%%%%%%%%%%%%%%%%%%%%%%%%%%%%%%%%%%%%%%%%%%%%%%%%%%%%%%%%%%%%%%%%%%%%%%%%%%%%%
% \section{Implementation}
%\iffalse
%<*package>
%\fi
%
% This section describes the definitions file |childdoc.def|.

% The definitions cannot be loaded using |\usepackage| or |\RequirePackage|
% which has a mechanism to prevent loading a style file more than once.
% When loading the definitions by means of |\input|
% multiple instances have to be prevented manually:
%\iffalse
%This code needs to be before the `\ProvidesFile' directive
%which is defined at the beginning of this file.
%Therefore it is also placed there and commented out here.
%</package>
%<*discard>
%\fi
%    \begin{macrocode}
\ifdefined\childdocmain\endinput\fi
%    \end{macrocode}
%\iffalse
%</discard>
%<*package>
%\fi
%
% \macro{\ifchilddoc}
% \macro{\ifchilddocmanual}
% The conditional |\ifchilddoc| tells whether a
% child (true) or main (false) document is being compiled.
% The conditional |\ifchilddocmanual| tells whether
% the |\includeonly| mechanism is used (false) or
% the selection of child files must be performed manually (true).
% The definitions initialise to false:
%    \begin{macrocode}
\newif\ifchilddoc
\newif\ifchilddocmanual
%    \end{macrocode}

% \macro{\childdocname}
% \macro{\childdocjob}
% The macro |\childdocname| stores the name of the main document
% to be compiled. The macro |\childdocjob| stores the name of
% the document on which the \LaTeX{} compiler was originally invoked.
% The content of |\jobname| cannot be compared
% to filenames specified in the source due to different catcodes.
% The following code rescans |\jobname|, stores the result
% in |\childdocname| and saves a copy in |\childdocjob|:
%    \begin{macrocode}
\edef\childdocname{\scantokens\expandafter{\jobname\noexpand}}
\let\childdocjob\childdocname
%    \end{macrocode}

% \macro{\childdocdisable}
% The macro |\childdocdisable| prevents the main file
% from being processed more than once.
% At this stage, the main document command |\childdocmain|
% is assumed to be called once again where it should do nothing.
% Any subsequent call to it should prevent
% a secondary processing of the main document
% It overwrites the forwarding commands
% |\childdocof| and |\childdocforward|
% with empty macros to prevent further inclusions of the main document:
%    \begin{macrocode}
\newcommand{\childdocdisable}
{
  \renewcommand{\childdocmain}[1]{\renewcommand{\childdocmain}[1]{\endinput}}
  \renewcommand{\childdocof}[1]{}
  \renewcommand{\childdocby}[2][]{}
  \renewcommand{\childdocforward}[2][]{}
  \renewcommand{\childdocdisable}{}
}
%    \end{macrocode}

% \macro{\childdocmain}
% The macro |\childdocmain| is to be called at the top of the main file
% with nothing or the main filename (without extension) as argument.
% First, it breaks loops.
% If the argument is not empty and does not match |\childdocname|
% (which is set by the first inclusion of |childdoc.def|),
% |\ifchilddoc| is set to true, |\includeonly| is applied to the child file
% and |\jobname| is set to the main file
% (for proper handling of |.aux| files):
%    \begin{macrocode}
\newcommand{\childdocmain}[1]
{
  \childdocdisable\childdocmain{}
  \if?#1?\else
    \begingroup
      \def\childdoctmp{#1}
      \ifx\childdoctmp\childdocname
        \def\childdoctmp{}
      \else
        \def\childdoctmp
        {
          \childdoctrue
          \includeonly{\childdocname}
          \def\childdocjob{#1}
          \def\jobname{#1}
        }
      \fi
      \expandafter
    \endgroup
    \childdoctmp
  \fi
}
%    \end{macrocode}

% \macro{\childdocof}
% The command |\childdocof| redirects
% compilation to the main file |#1|.
%    \begin{macrocode}
\newcommand{\childdocof}[1]
{
  \childdocdisable
  \childdoctrue
  \includeonly{\childdocname}
  \def\jobname{#1}
  \def\childdocjob{#1}
  \input{#1}
}
%    \end{macrocode}

% \macro{\childdocby}
% The command |\childdocby| ....
%    \begin{macrocode}
\newcommand{\childdocby}[2][]
{
  \childdocdisable
  \childdoctrue
  \childdocmanualtrue
  \if?#1?\else
    \def\jobname{#2}
  \fi
  \def\childdocjob{#2}
  \input{#2}
  \endinput
}
%    \end{macrocode}

% \macro{\childdocforward}
% The command |\childdocforward| redirects
% compilation to the main file or
% (if the optional argument is given) a child file.
% Parameters are set as if the main file
% or a child file starting with |\childdocof| was compiled.
% Then compilation is handed over to the main file:
%    \begin{macrocode}
\newcommand{\childdocforward}[2][]
{
  \begingroup
    \if?#1?
      \def\childdoctmp
      {
        \def\childdocname{#2}
        \def\childdocjob{#2}
        \def\jobname{#2}
        \input{#2}
        \endinput
      }
    \else
      \def\childdoctmp
      {
        \childdocdisable
        \def\childdocname{#2}
        \childdoctrue
        \includeonly{#2}
        \def\childdocjob{#1}
        \def\jobname{#1}
        \input{#1}
        \endinput
      }
    \fi
    \expandafter
  \endgroup
  \childdoctmp
}
%    \end{macrocode}

% \macro{\childdocforwardprefix}
% The command |\childdocforwardprefix| redirects
% compilation to the main or a child file by means of a pattern.
% The prefix |#1| in the current filename is replaced by |#2|
% and the suffix of the current filename is kept
% (it is assumed that the filename does not contain the substring `|~~~|'
% which is used as a delimiter).
% Compilation is handed over to the new file by |\childdocforward|:
%    \begin{macrocode}
\newcommand{\childdocforwardprefix}[3][]
{
  \begingroup
    \def\childdocextract #2##1~~~{\def\childdoctmp{\childdocforward[#1]{#3##1}}}
    \expandafter\childdocextract\childdocname~~~
    \expandafter
  \endgroup
  \childdoctmp
}
%    \end{macrocode}

% \macro{\childdoc}
% The deprecated macro |\childdoc| is a legacy version of |\childdocmain|:
%    \begin{macrocode}
\newcommand{\childdoc}{\childdocmain}
%    \end{macrocode}

% \macro{\childdocredirect}
% The deprecated macro |\childdocredirect| is a legacy version
% of |\childdocforward| and |\childdocforwardprefix|:
%    \begin{macrocode}
\newcommand{\childdocredirect}[2][]
{
  \begingroup
    \if?#1?
      \def\childdoctmp{\childdocforward{#2}}
    \else
      \def\childdoctmp{\childdocforwardprefix{#1}{#2}}
    \fi
    \expandafter
  \endgroup
  \childdoctmp
}
%    \end{macrocode}

%\iffalse
%</package>
%\fi
%
\endinput
|\\
|\childdocby{|\textit{main}|}|\\
\end{tabular}
\end{center}
%
Both forms have slightly different effects as described above.
The main file is prepared as usual, see \secref{sec:include}.

%%%%%%%%%%%%%%%%%%%%%%%%%%%%%%%%%%%%%%%%%%%%%%%%%%%%%%%%%%%%%%%%%%%%%%%%%%%%%%%%
\subsection{Legacy Detection}
\label{sec:detection}

The directive |\childdocmain| in the main file can detect
whether the complete document or merely a child is to be compiled
even without using the directive |\childdocof|.
This method is deprecated because it is less robust
and there is no compelling reason to use it;
it is merely provided for backward compatibility
and it may be removed in future versions.

If the detection mechanism is to be used,
it is mandatory to correctly specify
the filename of the main file as the argument of |\childdocmain|:
%
\begin{center}
\begin{tabular}{l}
|% \iffalse
%
% childdoc.dtx Copyright (C) 2017-2018 Niklas Beisert
%
% This work may be distributed and/or modified under the
% conditions of the LaTeX Project Public License, either version 1.3
% of this license or (at your option) any later version.
% The latest version of this license is in
%   http://www.latex-project.org/lppl.txt
% and version 1.3 or later is part of all distributions of LaTeX
% version 2005/12/01 or later.
%
% This work has the LPPL maintenance status `maintained'.
%
% The Current Maintainer of this work is Niklas Beisert.
%
% This work consists of the files childdoc.dtx and childdoc.ins
% and the derived files childdoc.def and cdocsamp.tex with
% cdocsch1.tex, cdocsch2.tex, cdocsdrf.tex, cdocsfn1.tex, cdocsfn2.tex.
%
%<package>\ifdefined\childdocmain\endinput\fi
%<package>\ProvidesFile{childdoc.def}[2018/12/30 v2.0 child document driver]
%<samplemain>\ProvidesFile{cdocsamp.tex}[2018/12/30 v2.0 sample for childdoc]
%<*driver>
%\ProvidesFile{childdoc.drv}[2018/12/30 v2.0 childdoc reference manual file]
\PassOptionsToClass{10pt,a4paper}{article}
\documentclass{ltxdoc}

\usepackage[margin=35mm]{geometry}
\usepackage{hyperref}
\usepackage{hyperxmp}
\usepackage[usenames]{color}

\hypersetup{colorlinks=true}
\hypersetup{pdfstartview=FitH}
\hypersetup{pdfpagemode=UseNone}
\hypersetup{pdfsource={}}
\hypersetup{pdflang={en-UK}}
\hypersetup{pdfcopyright={Copyright 2017-2018 Niklas Beisert.
  This work may be distributed and/or modified under the
  conditions of the LaTeX Project Public License, either version 1.3
  of this license or (at your option) any later version.}}
\hypersetup{pdflicenseurl={http://www.latex-project.org/lppl.txt}}
\hypersetup{pdfcontactaddress={ETH Zurich, ITP, HIT K,
  Wolfgang-Pauli-Strasse 27}}
\hypersetup{pdfcontactpostcode={8093}}
\hypersetup{pdfcontactcity={Zurich}}
\hypersetup{pdfcontactcountry={Switzerland}}
\hypersetup{pdfcontactemail={nbeisert@itp.phys.ethz.ch}}
\hypersetup{pdfcontacturl={http://people.phys.ethz.ch/\xmptilde nbeisert/}}

\newcommand{\secref}[1]{\hyperref[#1]{section \ref*{#1}}}

\parskip1ex
\parindent0pt
\let\olditemize\itemize
\def\itemize{\olditemize\parskip0pt}

\begin{document}

\title{The \textsf{childdoc} Package}
\hypersetup{pdftitle={The childdoc Package}}
\author{Niklas Beisert\\[2ex]
  Institut f\"ur Theoretische Physik\\
  Eidgen\"ossische Technische Hochschule Z\"urich\\
  Wolfgang-Pauli-Strasse 27, 8093 Z\"urich, Switzerland\\[1ex]
  \href{mailto:nbeisert@itp.phys.ethz.ch}
  {\texttt{nbeisert@itp.phys.ethz.ch}}}
\hypersetup{pdfauthor={Niklas Beisert}}
\hypersetup{pdfsubject={Manual for the LaTeX2e Package childdoc}}
\date{30 December 2018, \textsf{v2.0}}
\maketitle

\begin{abstract}\noindent
\textsf{childdoc} is a \LaTeXe{} package
that enables the direct compilation
of document sections included by |\include|
to individual files.
\end{abstract}

\begingroup
\parskip0ex
\tableofcontents
\endgroup

%%%%%%%%%%%%%%%%%%%%%%%%%%%%%%%%%%%%%%%%%%%%%%%%%%%%%%%%%%%%%%%%%%%%%%%%%%%%%%%%
%%%%%%%%%%%%%%%%%%%%%%%%%%%%%%%%%%%%%%%%%%%%%%%%%%%%%%%%%%%%%%%%%%%%%%%%%%%%%%%%
\section{Introduction}

\LaTeX{} provides a mechanism to structure a large document (such as a book)
into a main file and several child files (containing the chapters)
using the |\include| command.
This mechanism is beneficial for documents
which span hundreds of pages in order to
make the source file(s) more manageable.
Moreover, compilation can be restricted to
selected child files by means of the |\includeonly| command.
The latter feature can be used to reduce the compilation time while editing
(this was significantly more useful in the earlier days of \LaTeX{})
or to generate a smaller document which is easier to navigate.
Another application of |\includeonly| is to generate
documents consisting of selected parts of the complete document.

However, there are a few drawbacks of the plain |\include| mechanism:
\begin{itemize}
\item
The child files cannot be compiled on their own,
they can only be compiled via the main file.
A naive editing environment
(such as a text editor with an option
to have the current file processed by \LaTeX)
may require one to switch to the main file before compiling;
attempting to compile the child file produces errors.
\item
The main file must be modified (each time)
to adjust the |\includeonly| command
to the present needs. This easily leaves the main file in a messy state.
\item
The generated document will always carry the filename
of the main document. This is inconvenient if
several child files are to be compiled and
to be kept for distribution.
\end{itemize}

The present package provides a simple interface
to make child files individually compilable by \LaTeX{}.
Compiling a child file then has the same effect as compiling
the main file with an |\includeonly| command
to select the appropriate child.
Moreover the generated document will carry the name of the child
rather than the main file.
This resolves all three above issues.

This feature is meant to make the editing of books,
thesis documents and lecture notes somewhat more convenient.
However, the package can also be used efficiently for
composing a series of documents (such as exercise sheets)
which are typically distributed individually.
It then assists the author in generating the individual documents
(potentially in different versions)
as well as a document containing the collected series.
Another application is in developing style files
or other kinds of included material
where compilation of the style file could redirect
to a sample or test file.

%%%%%%%%%%%%%%%%%%%%%%%%%%%%%%%%%%%%%%%%%%%%%%%%%%%%%%%%%%%%%%%%%%%%%%%%%%%%%%%%
%%%%%%%%%%%%%%%%%%%%%%%%%%%%%%%%%%%%%%%%%%%%%%%%%%%%%%%%%%%%%%%%%%%%%%%%%%%%%%%%
\section{Usage}

First of all, the package \textsf{childdoc} is \emph{not} a standard
\LaTeXe{} |.sty| style file! Therefore it needs to be invoked in
a non-standard way.

%%%%%%%%%%%%%%%%%%%%%%%%%%%%%%%%%%%%%%%%%%%%%%%%%%%%%%%%%%%%%%%%%%%%%%%%%%%%%%%%
\subsection{Included Files}
\label{sec:include}

%%%%%%%%%%%%%%%%%%%%%%%%%%%%%%%%%%%%%%%%
\DescribeMacro{\childdocmain}
To use the package, add the commands
\begin{center}
\begin{tabular}{l}
|\input{childdoc.def}|\\
|\childdocmain{}|\\
\end{tabular}
\end{center}
at the very top of the main \LaTeX{} file,
in particular \emph{before} the |\documentclass| statement!
The argument of |\childdocmain| should be left empty
(but it must be present).

%%%%%%%%%%%%%%%%%%%%%%%%%%%%%%%%%%%%%%%%
\DescribeMacro{\childdocof}
Furthermore, add the commands
\begin{center}
\begin{tabular}{l}
|\input{childdoc.def}|\\
|\childdocof{|\textit{main}|}|\\
\end{tabular}
\end{center}
at the top of every child file \textit{child}
which is included by |\include{|\textit{child}|}|
from within the main file
(or at least for those files to be compiled individually).
The argument \textit{main} must be the filename of the main file.

There are a couple of
considerations in setting up the main and child documents:

%%%%%%%%%%%%%%%%%%%%%%%%%%%%%%%%%%%%%%%%
\paragraph{Restrictions.}

Please note the following restrictions:
\begin{itemize}
\item
|\childdocmain| must be called with one argument \textit{main}
to ensure compatibility with earlier version of the package.
It must either be empty (|\childdocmain{}|)
or precisely match the filename of the main file in which it is specified.
See \secref{sec:detection} for further information.
\item
The filename \textit{main} must be specified without the |.tex| extension.
\item
The filename \textit{main} is case sensitive
(even in case-insensitive file systems)
due to internal string comparison.
\item
The argument \textit{main} should be fully expanded, it cannot be a macro.
\item
Subdirectories and special characters should be avoided in filenames.
\item
The command |\childdocmain{|\textit{main}|}| must be followed by a whitespace.
It should not be followed immediately by another command
or by a comment mark `|%|'.
This is because the \TeX{} parser reads the token immediately following
the argument of |\childdocmain| and puts it
at the beginning of every child section;
however, a white\-space is ignored.
\end{itemize}

%%%%%%%%%%%%%%%%%%%%%%%%%%%%%%%%%%%%%%%%
\paragraph{Content of Main File.}

It is advisable to place all content in the child files included by |\include|.
Any output contained in the main file will appear in all child documents
unless suppressed manually;
it cannot be suppressed automatically by the |\includeonly| directive
and thus should normally be avoided.
A method to include some content in the main file
by means of conditional processing is described in \secref{sec:conditional}.

%%%%%%%%%%%%%%%%%%%%%%%%%%%%%%%%%%%%%%%%
\paragraph{Page Numbering.}

When only a part of the document is compiled,
the appropriate numbering of pages
(as well as other status parameters)
is determined from the |.aux| files.
The latter contain information from previous passes.
However this information needs to propagate through
all intermediate child documents.
Therefore the page numbering in child documents may well
be inconsistent until the complete document is compiled at least once.

A useful (if unconventional) way to always ensure a consistent
page numbering is to restart the numbering in each child document
and denote the pages by `\textit{child}|.|\textit{page}'
where \textit{child} represents the chapter/section number of the child file.
This can be achieved by the command
|\numberwithin{page}{|\textit{child}|}|
of the \textsf{amsmath} package
where \textit{child} can be |chapter| or |section|
depending on the chosen structuring.
Alternatively, one can modify the macro |\thepage| appropriately
and reset the counter |page| at the start of each child file.

%%%%%%%%%%%%%%%%%%%%%%%%%%%%%%%%%%%%%%%%%%%%%%%%%%%%%%%%%%%%%%%%%%%%%%%%%%%%%%%%
\subsection{Conditional Processing}
\label{sec:conditional}

The package provides a mechanism to compile different versions
of a document. To customise the versions further some conditional processing
can come in handy to distinguish which version is being compiled.
The package provides two macros to describe the compilation context:

%%%%%%%%%%%%%%%%%%%%%%%%%%%%%%%%%%%%%%%%
\DescribeMacro{\ifchilddoc}
The conditional |\ifchilddoc| distinguishes between the compilation of
child documents and the main document:
%
\begin{center}
|\ifchilddoc |\textit{child-code}| |[|\||else |\textit{main-code}]| \||fi|
\end{center}

%%%%%%%%%%%%%%%%%%%%%%%%%%%%%%%%%%%%%%%%
\DescribeMacro{\childdocname}
\DescribeMacro{\childdocjob}
The macro |\childdocname| contains the filename (without extension)
of the main or child file being processed.
Note that |\childdocjob| will always contain the name of the main file.

%%%%%%%%%%%%%%%%%%%%%%%%%%%%%%%%%%%%%%%%
\paragraph{Title Page.}

Conditional processing can be used to include a title or banner page
in the main document when proper precautions are taken.
Importantly, the code in the main file should ensure that the page counter
(as well as other status parameters which are stored in the |.aux| files)
takes the same value after the conditional processing.
Otherwise the page numbers may take divergent values
depending on which part is compiled.

For example, a title page could be declared by:
%
\begin{center}
\begin{tabular}{l}
|\ifchilddoc\||else|\\
|\addtocounter{page}{-1}|\\
\textit{code for title page}\\
|\newpage|\\
|\||fi|
\end{tabular}
\end{center}
%
A banner page for the child documents can be generated by:
%
\begin{center}
\begin{tabular}{l}
|\ifchilddoc|\\
|\addtocounter{page}{-1}|\\
\textit{code for banner page}\\
|\newpage|\\
|\||fi|
\end{tabular}
\end{center}
%
Here one could write a message such as:
\begin{center}
|This is the part \childdocname{} of \childdocjob{}.|
\end{center}

%%%%%%%%%%%%%%%%%%%%%%%%%%%%%%%%%%%%%%%%%%%%%%%%%%%%%%%%%%%%%%%%%%%%%%%%%%%%%%%%
\subsection{Flags}
\label{sec:flags}

The package makes it easy to generate different versions
of the main or child documents.
To this end compilation flags can be defined
and assigned different default values.
They will be particularly useful in conjunction
with the forwarding mechanism described in \secref{sec:forward}.

For example, it may be useful to have a flag |\version|
which can be set to |draft| or |final|.
The document source will contain some conditional code
depending on the value of |\version|.
Suppose further, the flag should default to |final| for the main file
and to |draft| for child files
which is a natural assignment for editing the document.
This is achieved by placing the following code
in the preamble of the main document
(below the |\childdocmain| directive):
%
\begin{center}
\begin{tabular}{l}
|\ifchilddoc|\\
|\providecommand{\version}{draft}|\\
|\||else|\\
|\providecommand{\version}{final}|\\
|\||fi|
\end{tabular}
\end{center}
%
The definition by |\providecommand| makes sure
that previous definitions are not overwritten.
Further statements |\providecommand{\version}{...}|
can thus be added before the above code to override it.

For the main file, one might add a line
(between |\childdocmain| and the above block)
%
\begin{center}
|%\ifchilddoc\||else\providecommand{\version}{draft}\||fi|
\end{center}
%
which can be uncommented to produce a draft version.
Likewise one can add a line to the very top of a child file
(above the |\childdocof{|\textit{main}|}| directive)
%
\begin{center}
|%\providecommand{\version}{final}|
\end{center}
%
which can be uncommented to produce the final version of this child document.

%%%%%%%%%%%%%%%%%%%%%%%%%%%%%%%%%%%%%%%%%%%%%%%%%%%%%%%%%%%%%%%%%%%%%%%%%%%%%%%%
\subsection{Forwarding}
\label{sec:forward}

Different versions of the main or child documents
using compilation flags as described in \secref{sec:flags}
can be (permanently) stored in different files
for convenient compilation, viewing and distribution.
To this end, the package defines a command
to pass on compilation to a different file:

%%%%%%%%%%%%%%%%%%%%%%%%%%%%%%%%%%%%%%%%
\DescribeMacro{\childdocforward}
The command |\childdocforward| redirects processing to
another source file:
%
\begin{center}
\begin{tabular}{l}
|\input{childdoc.def}|\\
|\childdocforward[|\textit{main}|]{|\textit{dest}|}|\\
\end{tabular}
\end{center}
%
The argument \textit{dest} is the destination file
(without extension).
It should be the main file or one of the child files.
Note that further \textsf{childdoc} directives
such as |\childdocof| and |\childdocforward|
in the indicated file will be processed in this form.
The optional argument \textit{main}
passes on directly to the main file \textit{main}
while pretending to compile the child \textit{dest}.
This form behaves as if \textit{dest}
issues |\childdocof{|\textit{main}|}| right away,
and no further \textsf{childdoc} directives will be processed.

%%%%%%%%%%%%%%%%%%%%%%%%%%%%%%%%%%%%%%%%
\DescribeMacro{\...prefix}
In the alternative form |\childdocforwardprefix|,
%
\begin{center}
\begin{tabular}{l}
|\input{childdoc.def}|\\
|\childdocforwardprefix[|\textit{main}|]{|\textit{prefix}|}{|\textit{dest}|}|
\end{tabular}
\end{center}
%
the destination file is determined by a pattern
depending on the current file:
To make this work, the current file must be called
`{\textit{prefix}\hspace{0.2em}\textit{suffix}}'
with \textit{prefix} matching precisely the argument.
Processing is then passed on to the file
`{\textit{dest}\hspace{0.2em}\textit{suffix}}'.
Surely, the same effect is achieved by
directly specifying the
argument `{\textit{dest}\hspace{0.2em}\textit{suffix}}'
in the first form.
However, that requires to set up a different file
for each child. With the alternative form of the command
all these files can have exactly the same content
which simplifies setting them up and maintaining them.

For example, the following file |draft.tex|
with a compilation flag |\version| as described in \secref{sec:flags}
compiles the main document as a draft:
%
\begin{center}
\begin{tabular}{l}
|\def\version{draft}|\\
|\input{childdoc.def}|\\
|\childdocforward{|\textit{main}|}|
\end{tabular}
\end{center}
%
Likewise, the following files |final|\textit{nn}|.tex|
compile the final version of the child document
|child|\textit{nn}|.tex|:
%
\begin{center}
\begin{tabular}{l}
|\def\version{final}|\\
|\input{childdoc.def}|\\
|\childdocforwardprefix{final}{child}|
\end{tabular}
\end{center}
%

Note that when several versions of a main file and/or of each child file
are to be generated, it may be convenient to set up a |Makefile| or
shell script to automatise the process.

%%%%%%%%%%%%%%%%%%%%%%%%%%%%%%%%%%%%%%%%%%%%%%%%%%%%%%%%%%%%%%%%%%%%%%%%%%%%%%%%
\subsection{Command Line Processing}
\label{sec:commandline}

The effect of redirection files can also be achieved by invoking
the \LaTeX{} compiler with a more elaborate command line.
Most conveniently this should be done as part
of a shell script or a |Makefile|.

When using \textsf{childdoc} in the main file, the following
command lines effectively perform a redirection
(note that depending on the shell being used,
backslashes may have to be doubled: `|\|' $\to$ `|\\|'):
%
\begin{center}
|... -jobname "|\textit{target}|" |\\|"|[\textit{flags}]%
|\input{childdoc.def}\childdocforward[|\textit{main}|]{|\textit{dest}|}"|
\end{center}
%
Here \textit{target} is the name of the output file,
\textit{main} is the name of the main file
and \textit{dest} is the name of the main or child file to be processed
(all filenames without extensions).
The optional argument \textit{main} can be omitted
if \textit{main} matches \textit{dest}.
Optionally, compilation \textit{flags} can be defined via |\def| commands.
This command line makes the \TeX{} engine believe
it is compiling the file \textit{target}
whose content is specified as the latter parameter.
The provided code then forwards the processing to
\textit{main} or \textit{dest} as described in \secref{sec:forward}.

%%%%%%%%%%%%%%%%%%%%%%%%%%%%%%%%%%%%%%%%%%%%%%%%%%%%%%%%%%%%%%%%%%%%%%%%%%%%%%%%
\subsection{Include by Input}
\label{sec:input}

Including child documents by |\include| has some restrictions by design.
Most notably, the content of a child document always occupies
its own set of pages; pages cannot be shared between child documents.
Usually, this behaviour makes perfect sense
because each child document contain an essential part of the document.
However, in some situations it may be desirable to compose
a document from a collection of parts
without having mandatory page breaks between then.
For this case, the package
provides a mechanism to include parts
by |\input| which can also be processed individually.
However, by construction this mechanism
requires manual handling of the content to be output.

%%%%%%%%%%%%%%%%%%%%%%%%%%%%%%%%%%%%%%%%
\DescribeMacro{\ifchilddocmanual}
The main file should be prepared as usual, see \secref{sec:include}.
However, the document body must make a distinction
between processing of an individual part and of the main document, e.g.:
%
\begin{center}
\begin{tabular}{l}
|\ifchilddocmanual|\\
|\input{\childdocname}|\\
|\||else|\\
\textit{document body with }|\input{|\textit{part}|}|\\
|\||fi|
\end{tabular}
\end{center}
%
The conditional |\ifchilddocmanual| is true whenever
a part to be included by |\input| is being compiled,
and the name of the part is stored in |\childdocname|.

%%%%%%%%%%%%%%%%%%%%%%%%%%%%%%%%%%%%%%%%
\DescribeMacro{\childdocby}
Each part to be included by |\input| should start with:
%
\begin{center}
\begin{tabular}{l}
|\input{childdoc.def}|\\
|\childdocby{|\textit{main}|}|\\
\end{tabular}
\end{center}
%
The directive |\childdocby| is similar to |\childdocof|
described in \secref{sec:include},
but the subsequent selection of content must be done manually.
To that end, both |\ifchilddoc| and |\ifchilddocmanual|
will be true upon processing of a part,
and the name of the part is stored in |\childdocname|.
Note that |\jobname| will be set to the filename of the current part
so that each part receives an individual |.aux| file
that does not interfere with the |.aux| file(s) of the main document.
This behaviour can be altered by the alternative form
|\childdocby[*]{|\textit{main}|}| (with a non-empty optional argument)
which uses the |.aux| file of the main document
by setting |\jobname| to \textit{main}.

%%%%%%%%%%%%%%%%%%%%%%%%%%%%%%%%%%%%%%%%%%%%%%%%%%%%%%%%%%%%%%%%%%%%%%%%%%%%%%%%
\subsection{Driver Development}
\label{sec:driver}

The \textsf{childdoc} mechanism can also be use for the development
of definition files such as \LaTeX{} styles or classes.
This case differs from the above setup with multiple parts
included by |\include| in that no |\includeonly| should be invoked.
This can be achieved by starting the include file
(before |\ProvidesPackage|) with:
%
\begin{center}
\begin{tabular}{l}
|\input{childdoc.def}|\\
|\childdocforward{|\textit{main}|}|\\
\end{tabular}
\end{center}
%
or alternatively with:
%
\begin{center}
\begin{tabular}{l}
|\input{childdoc.def}|\\
|\childdocby{|\textit{main}|}|\\
\end{tabular}
\end{center}
%
Both forms have slightly different effects as described above.
The main file is prepared as usual, see \secref{sec:include}.

%%%%%%%%%%%%%%%%%%%%%%%%%%%%%%%%%%%%%%%%%%%%%%%%%%%%%%%%%%%%%%%%%%%%%%%%%%%%%%%%
\subsection{Legacy Detection}
\label{sec:detection}

The directive |\childdocmain| in the main file can detect
whether the complete document or merely a child is to be compiled
even without using the directive |\childdocof|.
This method is deprecated because it is less robust
and there is no compelling reason to use it;
it is merely provided for backward compatibility
and it may be removed in future versions.

If the detection mechanism is to be used,
it is mandatory to correctly specify
the filename of the main file as the argument of |\childdocmain|:
%
\begin{center}
\begin{tabular}{l}
|\input{childdoc.def}|\\
|\childdocmain{|\textit{main}|}|\\
\end{tabular}
\end{center}
%
If |\jobname| does not match the argument \textit{main} of |\childdocmain|,
it is assumed that |\jobname| points to the child file to be compiled.
When using |\childdocmain| with the main file specified as argument,
it suffices to start a child file
with just |\input{|\textit{main}|}|
without loading of the package and using |\childdocof|.
If instead all processing is done
with the appropriate \textsf{childdoc} directives,
the argument of \textit{main} of |\childdocmain| can be empty.

An alternative version of the command line processing described
in \secref{sec:commandline} using the detection mechanism reads:
%
\begin{center}
|... -jobname "|\textit{target}|" "|[\textit{flags}]%
[|\def\jobname{|\textit{dest}|}|]|\input{|\textit{main}|}"|
\end{center}

%%%%%%%%%%%%%%%%%%%%%%%%%%%%%%%%%%%%%%%%%%%%%%%%%%%%%%%%%%%%%%%%%%%%%%%%%%%%%%%%
\subsection{Manual Code}
\label{sec:manual}

In case one cannot be certain whether the definitions file |childdoc.def|
is installed on the target \TeX{} distribution
and one prefers not to ship it,
it is conceivable to paste a few relevant commands into the sources.

To that end, drop all statements |\input{childdoc.def}|
and perform the replacements as outlined below.
Instead of |\childdocmain{|\textit{main}|}| add the following code
to the top of the main file:
%
\begin{center}
\begin{tabular}{l}
|\||ifdefined\childdocname\endinput\||fi\newif\ifchilddoc|\\
|\edef\childdocname{\scantokens\expandafter{\jobname\noexpand}}|\\
|\def\childdocmain{|\textit{main}|}\||ifx\childdocmain\childdocname\||else|\\
|\childdoctrue\includeonly{\childdocname}\let\jobname\childdocmain\||fi|\\
\end{tabular}
\end{center}
%
Instead of |\childdocof{|\textit{main}|}| just include the main file
at the top of each child file:
%
\begin{center}
|\input{|\textit{main}|}|
\end{center}
%
A simple redirection |\childdocforward{|\textit{dest}|}| is achieved by:
%
\begin{center}
|\def\jobname{|\textit{dest}|}\input{\jobname}|
\end{center}
%
The redirection with prefix
|\childdocforwardprefix[|\textit{prefix}|]{|\textit{dest}|}|
is accomplished by:
%
\begin{center}
\begin{tabular}{l}
|{\edef\jobname{\scantokens\expandafter{\jobname\noexpand}}|\\
|\def\redirectjob |\textit{prefix}|#1~~~{\gdef\jobname{|\textit{dest}|#1}}|\\
|\expandafter\redirectjob\jobname~~~}\input{\jobname}|
\end{tabular}
\end{center}

In an alternative approach,
child documents can be compiled by a specific command line
without additional code or specific definitions:
%
\begin{center}
|... -jobname "|\textit{target}|" "|[\textit{flags}]%
|\includeonly{|\textit{dest}|}\input{|\textit{main}|}"|
\end{center}
%

%%%%%%%%%%%%%%%%%%%%%%%%%%%%%%%%%%%%%%%%%%%%%%%%%%%%%%%%%%%%%%%%%%%%%%%%%%%%%%%%
%%%%%%%%%%%%%%%%%%%%%%%%%%%%%%%%%%%%%%%%%%%%%%%%%%%%%%%%%%%%%%%%%%%%%%%%%%%%%%%%
\section{Information}

%%%%%%%%%%%%%%%%%%%%%%%%%%%%%%%%%%%%%%%%%%%%%%%%%%%%%%%%%%%%%%%%%%%%%%%%%%%%%%%%
\subsection{Copyright}

Copyright \copyright{} 2017--2018 Niklas Beisert

This work may be distributed and/or modified under the
conditions of the \LaTeX{} Project Public License, either version 1.3
of this license or (at your option) any later version.
The latest version of this license is in
  \url{http://www.latex-project.org/lppl.txt}
and version 1.3 or later is part of all distributions of \LaTeX{}
version 2005/12/01 or later.

This work has the LPPL maintenance status `maintained'.

The Current Maintainer of this work is Niklas Beisert.

This work consists of the files |README.txt|, |childdoc.ins| and |childdoc.dtx|
as well as the derived files |childdoc.def|, |cdocsamp.tex|
with |cdocsch1.tex|, |cdocsch2.tex|, |cdocspt3.tex|, |cdocspt4.tex|,
|cdocsdrf.tex|, |cdocsfn1.tex|, |cdocsfn2.tex|
as well as |childdoc.pdf|.

%%%%%%%%%%%%%%%%%%%%%%%%%%%%%%%%%%%%%%%%%%%%%%%%%%%%%%%%%%%%%%%%%%%%%%%%%%%%%%%%
\subsection{Files and Installation}

The package consists of the files:
%
\begin{center}
\begin{tabular}{ll}
    |README.txt|   & readme file \\
    |childdoc.ins| & installation file \\
    |childdoc.dtx| & source file \\
    |childdoc.def| & definition file \\
    |cdocsamp.tex| & sample main file \\
    |cdocsch1.tex| & sample include file \\
    |cdocsch2.tex| & sample include file \\
    |cdocspt3.tex| & sample part file \\
    |cdocspt4.tex| & sample part file \\
    |cdocsdrf.tex| & sample redirection file \\
    |cdocsfn1.tex| & sample redirection file \\
    |cdocsfn2.tex| & sample redirection file \\
    |childdoc.pdf| & manual
\end{tabular}
\end{center}
%
The distribution consists of the files
|README.txt|, |childdoc.ins| and |childdoc.dtx|.
%
\begin{itemize}
\item
Run (pdf)\LaTeX{} on |childdoc.dtx|
to compile the manual |childdoc.pdf| (this file).
\item
Run \LaTeX{} on |childdoc.ins| to create the definitions file |childdoc.def|
and the sample |cdocsamp.tex| with include files
|cdocsch1.tex|, |cdocsch2.tex|, |cdocspt3.tex|, |cdocspt4.tex|,
|cdocsdrf.tex|, |cdocsfn1.tex|, |cdocsfn2.tex|.
Then copy the file |childdoc.def| to an appropriate directory of your \LaTeX{}
distribution, e.g.\ \textit{texmf-root}|/tex/latex/childdoc|.
\end{itemize}

%%%%%%%%%%%%%%%%%%%%%%%%%%%%%%%%%%%%%%%%%%%%%%%%%%%%%%%%%%%%%%%%%%%%%%%%%%%%%%%%
\subsection{Related CTAN Packages}

There are several other packages which offer a similar functionality:
%
\begin{itemize}
\item
The packages
\href{http://ctan.org/pkg/docmute}{\textsf{docmute}},
\href{http://ctan.org/pkg/includex}{\textsf{includex}} and
\href{http://ctan.org/pkg/standalone}{\textsf{standalone}}
provide commands to include only the document body of
a child file thus allowing both files to be compiled individually.
\item
The packages \href{http://ctan.org/pkg/subdocs}{\textsf{subdocs}}
and \href{http://ctan.org/pkg/subfiles}{\textsf{subfiles}}
provide structures in which the main and child documents can be
encapsulated and allowing them to be compiled individually.
The inclusion mechanism is different from the conventional |\include|.
\item
The package \href{http://ctan.org/pkg/combine}{\textsf{combine}}
is an elaborate solution to combine several documents into one.
\end{itemize}
%
See also the CTAN topic \href{http://ctan.org/topic/subdocs}{\textsf{subdocs}}
for further related packages.
The present package differs from the above solutions in that
a document structure constructed with the conventional |\include| mechanism
just needs two extra commands at the top of every file
such that all constituent files can be compiled individually.

%%%%%%%%%%%%%%%%%%%%%%%%%%%%%%%%%%%%%%%%%%%%%%%%%%%%%%%%%%%%%%%%%%%%%%%%%%%%%%%%
%\subsection{Feature Suggestions}
%
%The following is a list of features which may be useful for future
%versions of this package:
%%
%\begin{itemize}
%\item
%\ldots
%\end{itemize}

%%%%%%%%%%%%%%%%%%%%%%%%%%%%%%%%%%%%%%%%%%%%%%%%%%%%%%%%%%%%%%%%%%%%%%%%%%%%%%%%
\subsection{Revision History}

%%%%%%%%%%%%%%%%%%%%%%%%%%%%%%%%%%%%%%%%
\paragraph{v2.0:} 2018/12/30

\begin{itemize}
\item
immediate forward processing
\item
added |\childdocby| mechanism
\item
manual restructured
\end{itemize}

%%%%%%%%%%%%%%%%%%%%%%%%%%%%%%%%%%%%%%%%
\paragraph{v1.6:} 2018/01/17

\begin{itemize}
\item
application for development of include files
\item
corrections to manual
\end{itemize}

%%%%%%%%%%%%%%%%%%%%%%%%%%%%%%%%%%%%%%%%
\paragraph{v1.5:} 2017/05/21

\begin{itemize}
\item
more complete structuring introduced
\item
|\childdocof| introduced
\item
|\childdoc| renamed to |\childdocmain|
\item
|\childredirect| renamed to |\childdocforward| and |\childdocforwardprefix|
and functionality expanded
\end{itemize}

%%%%%%%%%%%%%%%%%%%%%%%%%%%%%%%%%%%%%%%%
\paragraph{v1.0:} 2017/04/27

\begin{itemize}
\item
manual and install package
\item
first version published on CTAN
\end{itemize}

%%%%%%%%%%%%%%%%%%%%%%%%%%%%%%%%%%%%%%%%
\paragraph{v0.6:} 2017/04/26

\begin{itemize}
\item
redirection mechanism added
\end{itemize}

%%%%%%%%%%%%%%%%%%%%%%%%%%%%%%%%%%%%%%%%
\paragraph{v0.5:} 2017/04/26

\begin{itemize}
\item
functionality in definition file
\end{itemize}


%%%%%%%%%%%%%%%%%%%%%%%%%%%%%%%%%%%%%%%%%%%%%%%%%%%%%%%%%%%%%%%%%%%%%%%%%%%%%%%%
%%%%%%%%%%%%%%%%%%%%%%%%%%%%%%%%%%%%%%%%%%%%%%%%%%%%%%%%%%%%%%%%%%%%%%%%%%%%%%%%
%%%%%%%%%%%%%%%%%%%%%%%%%%%%%%%%%%%%%%%%%%%%%%%%%%%%%%%%%%%%%%%%%%%%%%%%%%%%%%%%
\appendix

\settowidth\MacroIndent{\rmfamily\scriptsize 000\ }

 \DocInput{childdoc.dtx}

\end{document}
%</driver>
% \fi
%
% %%%%%%%%%%%%%%%%%%%%%%%%%%%%%%%%%%%%%%%%%%%%%%%%%%%%%%%%%%%%%%%%%%%%%%%%%%%%%%
% %%%%%%%%%%%%%%%%%%%%%%%%%%%%%%%%%%%%%%%%%%%%%%%%%%%%%%%%%%%%%%%%%%%%%%%%%%%%%%
% \section{Sample}
%\iffalse
%<*samplemain>
%\fi
%
% The following presents a sample document
% with two chapters, two parts, a title page,
% a compile flag as well as three forwarding files to set the flag.
% It consists of eight |.tex| files:
% \begin{center}
% \begin{tabular}{ll}
% |cdocsamp.tex|&main file\\
% |cdocsch1.tex|&include file for chapter 1\\
% |cdocsch2.tex|&include file for chapter 2\\
% |cdocspt3.tex|&include file for part 3\\
% |cdocspt4.tex|&include file for part 4\\
% |cdocsdrf.tex|&forwarding file for main file in draft mode\\
% |cdocsfi1.tex|&forwarding file for final version of chapter 1\\
% |cdocsfi2.tex|&forwarding file for final version of chapter 2\\
% \end{tabular}
% \end{center}
% Each of the eight files can be compiled directly by the \LaTeX{} compiler.
%
% %%%%%%%%%%%%%%%%%%%%%%%%%%%%%%%%%%%%%%
% \paragraph{Main File.}
%
% The main file is called |cdocsamp.tex|.
%
% Load the \textsf{childdoc} definitions and
% declare the filename for the main document:
%    \begin{macrocode}
\input{childdoc.def}
\childdocmain{}
%    \end{macrocode}

% Optional override for |\version| flag:
%    \begin{macrocode}
%%\ifchilddoc\else\providecommand{\version}{draft}\fi
%    \end{macrocode}

% Define the default values for the |\version| flag
% (|final| for the main file and |draft| for childs):
%    \begin{macrocode}
\ifchilddoc
\providecommand{\version}{draft}
\else
\providecommand{\version}{final}
\fi
%    \end{macrocode}

% Load the standard document class:
%    \begin{macrocode}
\documentclass[12pt]{article}
%    \end{macrocode}

% Start the document body:
%    \begin{macrocode}
\begin{document}
%    \end{macrocode}

% Declare a title page.
% Print title, part of document being processed and version flag:
%    \begin{macrocode}
\addtocounter{page}{-1}
\begin{center}
{\LARGE\bfseries{}childdoc example\par}
\vspace{1cm}
\ifchilddoc
\ifchilddocmanual part\else chapter\fi:
`\childdocname' of `\childdocjob'\par
\else
main document: `\childdocjob'\par
\fi
version: \version\par
\end{center}
\newpage
%    \end{macrocode}

% Manually include selected file,
% otherwise process as usual:
%    \begin{macrocode}
\ifchilddocmanual
\section*{part `\childdocname'}
\input{\childdocname}
\else
%    \end{macrocode}

% Include the two chapters:
%    \begin{macrocode}
\include{cdocsch1}
\include{cdocsch2}
%    \end{macrocode}

% Include the two parts unless only chapters should be displayed:
%    \begin{macrocode}
\ifchilddoc\else
\section{part three}
\input{cdocspt3}
\section{part four}
\input{cdocspt4}
\fi
%    \end{macrocode}

% Process as usual until here:
%    \begin{macrocode}
\fi
%    \end{macrocode}

% End of document body:
%    \begin{macrocode}
\end{document}
%    \end{macrocode}
%\iffalse
%</samplemain>
%\fi
%
% %%%%%%%%%%%%%%%%%%%%%%%%%%%%%%%%%%%%%%
% \paragraph{Chapter Include Files.}
%
% The include files are called |cdocsch1.tex| and |cdocsch2.tex|.
%
%\iffalse
%<*samplechap1|samplechap2>
%\fi

% Optional override for |\version| flag:
%    \begin{macrocode}
%%\providecommand{\version}{final}
%    \end{macrocode}

% Include the main document:
%    \begin{macrocode}
\input{childdoc.def}
\childdocof{cdocsamp}
%    \end{macrocode}

%\iffalse
%</samplechap1|samplechap2>
%\fi
%
%\iffalse
%<*samplechap1>
%\fi
% Some text for chapter 1:
%    \begin{macrocode}
\section{one}
some text in chapter one
%    \end{macrocode}

%\iffalse
%</samplechap1>
%\fi
% Some text for chapter 2:
%\iffalse
%<*samplechap2>
%\fi
%    \begin{macrocode}
\section{two}
more text in chapter two
%    \end{macrocode}

%\iffalse
%</samplechap2>
%\fi
%
% %%%%%%%%%%%%%%%%%%%%%%%%%%%%%%%%%%%%%%
% \paragraph{Part Include Files.}
%
% The include files are called |cdocspt3.tex| and |cdocspt4.tex|.
%
%\iffalse
%<*samplepart3|samplepart4>
%\fi

% Optional override for |\version| flag:
%    \begin{macrocode}
%%\providecommand{\version}{final}
%    \end{macrocode}

% Include the main document:
%    \begin{macrocode}
\input{childdoc.def}
\childdocby{cdocsamp}
%    \end{macrocode}

%\iffalse
%</samplepart3|samplepart4>
%\fi
%
%\iffalse
%<*samplepart3>
%\fi
% Some text for part 3:
%    \begin{macrocode}
some text in part three
%    \end{macrocode}

%\iffalse
%</samplepart3>
%\fi
% Some text for part 4:
%\iffalse
%<*samplepart4>
%\fi
%    \begin{macrocode}
more text in part four
%    \end{macrocode}

%\iffalse
%</samplepart4>
%\fi
%
% %%%%%%%%%%%%%%%%%%%%%%%%%%%%%%%%%%%%%%
% \paragraph{Forwarding for a Complete Draft.}
%
% The following forwarding file |cdocsdrf.tex|
% compiles the main document in draft mode:
%\iffalse
%<*sampledraft>
%\fi
%    \begin{macrocode}
\def\version{draft}
\input{childdoc.def}
\childdocforward{cdocsamp}
%    \end{macrocode}

%\iffalse
%</sampledraft>
%\fi
%
% %%%%%%%%%%%%%%%%%%%%%%%%%%%%%%%%%%%%%%
% \paragraph{Forwarding for Final Version of the Chapters.}
%
% The following forwarding files |cdocsfn1.tex| and |cdocsfn2.tex|
% (with identical content)
% compile the final versions of the child documents
% |cdocsch1.tex| and |cdocsch2.tex|, respectively:
%\iffalse
%<*samplefinal>
%\fi
%    \begin{macrocode}
\def\version{final}
\input{childdoc.def}
\childdocforwardprefix[cdocsamp]{cdocsfn}{cdocsch}
%    \end{macrocode}

%\iffalse
%</samplefinal>
%\fi
%
% %%%%%%%%%%%%%%%%%%%%%%%%%%%%%%%%%%%%%%
% \paragraph{Command Line Processing.}
%
% The following three command lines generate the output files
% |cdocscld|, |cdocscl1| and |cdocscl2|
% which should be identical to
% |cdocsdrf|, |cdocsch1| and |cdocsfn2|, respectively:
% \begin{center}
% \begin{tabular}{l}
% |latex -jobname cdocscld \|\\
% |  "\def\version{draft}\input{childdoc.def}\childdocforward{cdocsamp}"|\\
% |latex -jobname cdocscl1 \|\\
% |  "\input{childdoc.def}\childdocforward[cdocsamp]{cdocsch1}"|\\
% |latex -jobname cdocscl2 \|\\
% |  "\def\version{final}\input{childdoc.def}\childdocforward{cdocsch2}"|
% \end{tabular}
% \end{center}
% Note that the trailing backslash on each first line
% merely continues the input to the second line
% (for convenient cut ant paste).
% Furthermore, the command |latex| can be replaced by any
% of its alternative versions such as |pdflatex|.
%
% %%%%%%%%%%%%%%%%%%%%%%%%%%%%%%%%%%%%%%%%%%%%%%%%%%%%%%%%%%%%%%%%%%%%%%%%%%%%%%
% %%%%%%%%%%%%%%%%%%%%%%%%%%%%%%%%%%%%%%%%%%%%%%%%%%%%%%%%%%%%%%%%%%%%%%%%%%%%%%
% \section{Implementation}
%\iffalse
%<*package>
%\fi
%
% This section describes the definitions file |childdoc.def|.

% The definitions cannot be loaded using |\usepackage| or |\RequirePackage|
% which has a mechanism to prevent loading a style file more than once.
% When loading the definitions by means of |\input|
% multiple instances have to be prevented manually:
%\iffalse
%This code needs to be before the `\ProvidesFile' directive
%which is defined at the beginning of this file.
%Therefore it is also placed there and commented out here.
%</package>
%<*discard>
%\fi
%    \begin{macrocode}
\ifdefined\childdocmain\endinput\fi
%    \end{macrocode}
%\iffalse
%</discard>
%<*package>
%\fi
%
% \macro{\ifchilddoc}
% \macro{\ifchilddocmanual}
% The conditional |\ifchilddoc| tells whether a
% child (true) or main (false) document is being compiled.
% The conditional |\ifchilddocmanual| tells whether
% the |\includeonly| mechanism is used (false) or
% the selection of child files must be performed manually (true).
% The definitions initialise to false:
%    \begin{macrocode}
\newif\ifchilddoc
\newif\ifchilddocmanual
%    \end{macrocode}

% \macro{\childdocname}
% \macro{\childdocjob}
% The macro |\childdocname| stores the name of the main document
% to be compiled. The macro |\childdocjob| stores the name of
% the document on which the \LaTeX{} compiler was originally invoked.
% The content of |\jobname| cannot be compared
% to filenames specified in the source due to different catcodes.
% The following code rescans |\jobname|, stores the result
% in |\childdocname| and saves a copy in |\childdocjob|:
%    \begin{macrocode}
\edef\childdocname{\scantokens\expandafter{\jobname\noexpand}}
\let\childdocjob\childdocname
%    \end{macrocode}

% \macro{\childdocdisable}
% The macro |\childdocdisable| prevents the main file
% from being processed more than once.
% At this stage, the main document command |\childdocmain|
% is assumed to be called once again where it should do nothing.
% Any subsequent call to it should prevent
% a secondary processing of the main document
% It overwrites the forwarding commands
% |\childdocof| and |\childdocforward|
% with empty macros to prevent further inclusions of the main document:
%    \begin{macrocode}
\newcommand{\childdocdisable}
{
  \renewcommand{\childdocmain}[1]{\renewcommand{\childdocmain}[1]{\endinput}}
  \renewcommand{\childdocof}[1]{}
  \renewcommand{\childdocby}[2][]{}
  \renewcommand{\childdocforward}[2][]{}
  \renewcommand{\childdocdisable}{}
}
%    \end{macrocode}

% \macro{\childdocmain}
% The macro |\childdocmain| is to be called at the top of the main file
% with nothing or the main filename (without extension) as argument.
% First, it breaks loops.
% If the argument is not empty and does not match |\childdocname|
% (which is set by the first inclusion of |childdoc.def|),
% |\ifchilddoc| is set to true, |\includeonly| is applied to the child file
% and |\jobname| is set to the main file
% (for proper handling of |.aux| files):
%    \begin{macrocode}
\newcommand{\childdocmain}[1]
{
  \childdocdisable\childdocmain{}
  \if?#1?\else
    \begingroup
      \def\childdoctmp{#1}
      \ifx\childdoctmp\childdocname
        \def\childdoctmp{}
      \else
        \def\childdoctmp
        {
          \childdoctrue
          \includeonly{\childdocname}
          \def\childdocjob{#1}
          \def\jobname{#1}
        }
      \fi
      \expandafter
    \endgroup
    \childdoctmp
  \fi
}
%    \end{macrocode}

% \macro{\childdocof}
% The command |\childdocof| redirects
% compilation to the main file |#1|.
%    \begin{macrocode}
\newcommand{\childdocof}[1]
{
  \childdocdisable
  \childdoctrue
  \includeonly{\childdocname}
  \def\jobname{#1}
  \def\childdocjob{#1}
  \input{#1}
}
%    \end{macrocode}

% \macro{\childdocby}
% The command |\childdocby| ....
%    \begin{macrocode}
\newcommand{\childdocby}[2][]
{
  \childdocdisable
  \childdoctrue
  \childdocmanualtrue
  \if?#1?\else
    \def\jobname{#2}
  \fi
  \def\childdocjob{#2}
  \input{#2}
  \endinput
}
%    \end{macrocode}

% \macro{\childdocforward}
% The command |\childdocforward| redirects
% compilation to the main file or
% (if the optional argument is given) a child file.
% Parameters are set as if the main file
% or a child file starting with |\childdocof| was compiled.
% Then compilation is handed over to the main file:
%    \begin{macrocode}
\newcommand{\childdocforward}[2][]
{
  \begingroup
    \if?#1?
      \def\childdoctmp
      {
        \def\childdocname{#2}
        \def\childdocjob{#2}
        \def\jobname{#2}
        \input{#2}
        \endinput
      }
    \else
      \def\childdoctmp
      {
        \childdocdisable
        \def\childdocname{#2}
        \childdoctrue
        \includeonly{#2}
        \def\childdocjob{#1}
        \def\jobname{#1}
        \input{#1}
        \endinput
      }
    \fi
    \expandafter
  \endgroup
  \childdoctmp
}
%    \end{macrocode}

% \macro{\childdocforwardprefix}
% The command |\childdocforwardprefix| redirects
% compilation to the main or a child file by means of a pattern.
% The prefix |#1| in the current filename is replaced by |#2|
% and the suffix of the current filename is kept
% (it is assumed that the filename does not contain the substring `|~~~|'
% which is used as a delimiter).
% Compilation is handed over to the new file by |\childdocforward|:
%    \begin{macrocode}
\newcommand{\childdocforwardprefix}[3][]
{
  \begingroup
    \def\childdocextract #2##1~~~{\def\childdoctmp{\childdocforward[#1]{#3##1}}}
    \expandafter\childdocextract\childdocname~~~
    \expandafter
  \endgroup
  \childdoctmp
}
%    \end{macrocode}

% \macro{\childdoc}
% The deprecated macro |\childdoc| is a legacy version of |\childdocmain|:
%    \begin{macrocode}
\newcommand{\childdoc}{\childdocmain}
%    \end{macrocode}

% \macro{\childdocredirect}
% The deprecated macro |\childdocredirect| is a legacy version
% of |\childdocforward| and |\childdocforwardprefix|:
%    \begin{macrocode}
\newcommand{\childdocredirect}[2][]
{
  \begingroup
    \if?#1?
      \def\childdoctmp{\childdocforward{#2}}
    \else
      \def\childdoctmp{\childdocforwardprefix{#1}{#2}}
    \fi
    \expandafter
  \endgroup
  \childdoctmp
}
%    \end{macrocode}

%\iffalse
%</package>
%\fi
%
\endinput
|\\
|\childdocmain{|\textit{main}|}|\\
\end{tabular}
\end{center}
%
If |\jobname| does not match the argument \textit{main} of |\childdocmain|,
it is assumed that |\jobname| points to the child file to be compiled.
When using |\childdocmain| with the main file specified as argument,
it suffices to start a child file
with just |\input{|\textit{main}|}|
without loading of the package and using |\childdocof|.
If instead all processing is done
with the appropriate \textsf{childdoc} directives,
the argument of \textit{main} of |\childdocmain| can be empty.

An alternative version of the command line processing described
in \secref{sec:commandline} using the detection mechanism reads:
%
\begin{center}
|... -jobname "|\textit{target}|" "|[\textit{flags}]%
[|\def\jobname{|\textit{dest}|}|]|\input{|\textit{main}|}"|
\end{center}

%%%%%%%%%%%%%%%%%%%%%%%%%%%%%%%%%%%%%%%%%%%%%%%%%%%%%%%%%%%%%%%%%%%%%%%%%%%%%%%%
\subsection{Manual Code}
\label{sec:manual}

In case one cannot be certain whether the definitions file |childdoc.def|
is installed on the target \TeX{} distribution
and one prefers not to ship it,
it is conceivable to paste a few relevant commands into the sources.

To that end, drop all statements |% \iffalse
%
% childdoc.dtx Copyright (C) 2017-2018 Niklas Beisert
%
% This work may be distributed and/or modified under the
% conditions of the LaTeX Project Public License, either version 1.3
% of this license or (at your option) any later version.
% The latest version of this license is in
%   http://www.latex-project.org/lppl.txt
% and version 1.3 or later is part of all distributions of LaTeX
% version 2005/12/01 or later.
%
% This work has the LPPL maintenance status `maintained'.
%
% The Current Maintainer of this work is Niklas Beisert.
%
% This work consists of the files childdoc.dtx and childdoc.ins
% and the derived files childdoc.def and cdocsamp.tex with
% cdocsch1.tex, cdocsch2.tex, cdocsdrf.tex, cdocsfn1.tex, cdocsfn2.tex.
%
%<package>\ifdefined\childdocmain\endinput\fi
%<package>\ProvidesFile{childdoc.def}[2018/12/30 v2.0 child document driver]
%<samplemain>\ProvidesFile{cdocsamp.tex}[2018/12/30 v2.0 sample for childdoc]
%<*driver>
%\ProvidesFile{childdoc.drv}[2018/12/30 v2.0 childdoc reference manual file]
\PassOptionsToClass{10pt,a4paper}{article}
\documentclass{ltxdoc}

\usepackage[margin=35mm]{geometry}
\usepackage{hyperref}
\usepackage{hyperxmp}
\usepackage[usenames]{color}

\hypersetup{colorlinks=true}
\hypersetup{pdfstartview=FitH}
\hypersetup{pdfpagemode=UseNone}
\hypersetup{pdfsource={}}
\hypersetup{pdflang={en-UK}}
\hypersetup{pdfcopyright={Copyright 2017-2018 Niklas Beisert.
  This work may be distributed and/or modified under the
  conditions of the LaTeX Project Public License, either version 1.3
  of this license or (at your option) any later version.}}
\hypersetup{pdflicenseurl={http://www.latex-project.org/lppl.txt}}
\hypersetup{pdfcontactaddress={ETH Zurich, ITP, HIT K,
  Wolfgang-Pauli-Strasse 27}}
\hypersetup{pdfcontactpostcode={8093}}
\hypersetup{pdfcontactcity={Zurich}}
\hypersetup{pdfcontactcountry={Switzerland}}
\hypersetup{pdfcontactemail={nbeisert@itp.phys.ethz.ch}}
\hypersetup{pdfcontacturl={http://people.phys.ethz.ch/\xmptilde nbeisert/}}

\newcommand{\secref}[1]{\hyperref[#1]{section \ref*{#1}}}

\parskip1ex
\parindent0pt
\let\olditemize\itemize
\def\itemize{\olditemize\parskip0pt}

\begin{document}

\title{The \textsf{childdoc} Package}
\hypersetup{pdftitle={The childdoc Package}}
\author{Niklas Beisert\\[2ex]
  Institut f\"ur Theoretische Physik\\
  Eidgen\"ossische Technische Hochschule Z\"urich\\
  Wolfgang-Pauli-Strasse 27, 8093 Z\"urich, Switzerland\\[1ex]
  \href{mailto:nbeisert@itp.phys.ethz.ch}
  {\texttt{nbeisert@itp.phys.ethz.ch}}}
\hypersetup{pdfauthor={Niklas Beisert}}
\hypersetup{pdfsubject={Manual for the LaTeX2e Package childdoc}}
\date{30 December 2018, \textsf{v2.0}}
\maketitle

\begin{abstract}\noindent
\textsf{childdoc} is a \LaTeXe{} package
that enables the direct compilation
of document sections included by |\include|
to individual files.
\end{abstract}

\begingroup
\parskip0ex
\tableofcontents
\endgroup

%%%%%%%%%%%%%%%%%%%%%%%%%%%%%%%%%%%%%%%%%%%%%%%%%%%%%%%%%%%%%%%%%%%%%%%%%%%%%%%%
%%%%%%%%%%%%%%%%%%%%%%%%%%%%%%%%%%%%%%%%%%%%%%%%%%%%%%%%%%%%%%%%%%%%%%%%%%%%%%%%
\section{Introduction}

\LaTeX{} provides a mechanism to structure a large document (such as a book)
into a main file and several child files (containing the chapters)
using the |\include| command.
This mechanism is beneficial for documents
which span hundreds of pages in order to
make the source file(s) more manageable.
Moreover, compilation can be restricted to
selected child files by means of the |\includeonly| command.
The latter feature can be used to reduce the compilation time while editing
(this was significantly more useful in the earlier days of \LaTeX{})
or to generate a smaller document which is easier to navigate.
Another application of |\includeonly| is to generate
documents consisting of selected parts of the complete document.

However, there are a few drawbacks of the plain |\include| mechanism:
\begin{itemize}
\item
The child files cannot be compiled on their own,
they can only be compiled via the main file.
A naive editing environment
(such as a text editor with an option
to have the current file processed by \LaTeX)
may require one to switch to the main file before compiling;
attempting to compile the child file produces errors.
\item
The main file must be modified (each time)
to adjust the |\includeonly| command
to the present needs. This easily leaves the main file in a messy state.
\item
The generated document will always carry the filename
of the main document. This is inconvenient if
several child files are to be compiled and
to be kept for distribution.
\end{itemize}

The present package provides a simple interface
to make child files individually compilable by \LaTeX{}.
Compiling a child file then has the same effect as compiling
the main file with an |\includeonly| command
to select the appropriate child.
Moreover the generated document will carry the name of the child
rather than the main file.
This resolves all three above issues.

This feature is meant to make the editing of books,
thesis documents and lecture notes somewhat more convenient.
However, the package can also be used efficiently for
composing a series of documents (such as exercise sheets)
which are typically distributed individually.
It then assists the author in generating the individual documents
(potentially in different versions)
as well as a document containing the collected series.
Another application is in developing style files
or other kinds of included material
where compilation of the style file could redirect
to a sample or test file.

%%%%%%%%%%%%%%%%%%%%%%%%%%%%%%%%%%%%%%%%%%%%%%%%%%%%%%%%%%%%%%%%%%%%%%%%%%%%%%%%
%%%%%%%%%%%%%%%%%%%%%%%%%%%%%%%%%%%%%%%%%%%%%%%%%%%%%%%%%%%%%%%%%%%%%%%%%%%%%%%%
\section{Usage}

First of all, the package \textsf{childdoc} is \emph{not} a standard
\LaTeXe{} |.sty| style file! Therefore it needs to be invoked in
a non-standard way.

%%%%%%%%%%%%%%%%%%%%%%%%%%%%%%%%%%%%%%%%%%%%%%%%%%%%%%%%%%%%%%%%%%%%%%%%%%%%%%%%
\subsection{Included Files}
\label{sec:include}

%%%%%%%%%%%%%%%%%%%%%%%%%%%%%%%%%%%%%%%%
\DescribeMacro{\childdocmain}
To use the package, add the commands
\begin{center}
\begin{tabular}{l}
|\input{childdoc.def}|\\
|\childdocmain{}|\\
\end{tabular}
\end{center}
at the very top of the main \LaTeX{} file,
in particular \emph{before} the |\documentclass| statement!
The argument of |\childdocmain| should be left empty
(but it must be present).

%%%%%%%%%%%%%%%%%%%%%%%%%%%%%%%%%%%%%%%%
\DescribeMacro{\childdocof}
Furthermore, add the commands
\begin{center}
\begin{tabular}{l}
|\input{childdoc.def}|\\
|\childdocof{|\textit{main}|}|\\
\end{tabular}
\end{center}
at the top of every child file \textit{child}
which is included by |\include{|\textit{child}|}|
from within the main file
(or at least for those files to be compiled individually).
The argument \textit{main} must be the filename of the main file.

There are a couple of
considerations in setting up the main and child documents:

%%%%%%%%%%%%%%%%%%%%%%%%%%%%%%%%%%%%%%%%
\paragraph{Restrictions.}

Please note the following restrictions:
\begin{itemize}
\item
|\childdocmain| must be called with one argument \textit{main}
to ensure compatibility with earlier version of the package.
It must either be empty (|\childdocmain{}|)
or precisely match the filename of the main file in which it is specified.
See \secref{sec:detection} for further information.
\item
The filename \textit{main} must be specified without the |.tex| extension.
\item
The filename \textit{main} is case sensitive
(even in case-insensitive file systems)
due to internal string comparison.
\item
The argument \textit{main} should be fully expanded, it cannot be a macro.
\item
Subdirectories and special characters should be avoided in filenames.
\item
The command |\childdocmain{|\textit{main}|}| must be followed by a whitespace.
It should not be followed immediately by another command
or by a comment mark `|%|'.
This is because the \TeX{} parser reads the token immediately following
the argument of |\childdocmain| and puts it
at the beginning of every child section;
however, a white\-space is ignored.
\end{itemize}

%%%%%%%%%%%%%%%%%%%%%%%%%%%%%%%%%%%%%%%%
\paragraph{Content of Main File.}

It is advisable to place all content in the child files included by |\include|.
Any output contained in the main file will appear in all child documents
unless suppressed manually;
it cannot be suppressed automatically by the |\includeonly| directive
and thus should normally be avoided.
A method to include some content in the main file
by means of conditional processing is described in \secref{sec:conditional}.

%%%%%%%%%%%%%%%%%%%%%%%%%%%%%%%%%%%%%%%%
\paragraph{Page Numbering.}

When only a part of the document is compiled,
the appropriate numbering of pages
(as well as other status parameters)
is determined from the |.aux| files.
The latter contain information from previous passes.
However this information needs to propagate through
all intermediate child documents.
Therefore the page numbering in child documents may well
be inconsistent until the complete document is compiled at least once.

A useful (if unconventional) way to always ensure a consistent
page numbering is to restart the numbering in each child document
and denote the pages by `\textit{child}|.|\textit{page}'
where \textit{child} represents the chapter/section number of the child file.
This can be achieved by the command
|\numberwithin{page}{|\textit{child}|}|
of the \textsf{amsmath} package
where \textit{child} can be |chapter| or |section|
depending on the chosen structuring.
Alternatively, one can modify the macro |\thepage| appropriately
and reset the counter |page| at the start of each child file.

%%%%%%%%%%%%%%%%%%%%%%%%%%%%%%%%%%%%%%%%%%%%%%%%%%%%%%%%%%%%%%%%%%%%%%%%%%%%%%%%
\subsection{Conditional Processing}
\label{sec:conditional}

The package provides a mechanism to compile different versions
of a document. To customise the versions further some conditional processing
can come in handy to distinguish which version is being compiled.
The package provides two macros to describe the compilation context:

%%%%%%%%%%%%%%%%%%%%%%%%%%%%%%%%%%%%%%%%
\DescribeMacro{\ifchilddoc}
The conditional |\ifchilddoc| distinguishes between the compilation of
child documents and the main document:
%
\begin{center}
|\ifchilddoc |\textit{child-code}| |[|\||else |\textit{main-code}]| \||fi|
\end{center}

%%%%%%%%%%%%%%%%%%%%%%%%%%%%%%%%%%%%%%%%
\DescribeMacro{\childdocname}
\DescribeMacro{\childdocjob}
The macro |\childdocname| contains the filename (without extension)
of the main or child file being processed.
Note that |\childdocjob| will always contain the name of the main file.

%%%%%%%%%%%%%%%%%%%%%%%%%%%%%%%%%%%%%%%%
\paragraph{Title Page.}

Conditional processing can be used to include a title or banner page
in the main document when proper precautions are taken.
Importantly, the code in the main file should ensure that the page counter
(as well as other status parameters which are stored in the |.aux| files)
takes the same value after the conditional processing.
Otherwise the page numbers may take divergent values
depending on which part is compiled.

For example, a title page could be declared by:
%
\begin{center}
\begin{tabular}{l}
|\ifchilddoc\||else|\\
|\addtocounter{page}{-1}|\\
\textit{code for title page}\\
|\newpage|\\
|\||fi|
\end{tabular}
\end{center}
%
A banner page for the child documents can be generated by:
%
\begin{center}
\begin{tabular}{l}
|\ifchilddoc|\\
|\addtocounter{page}{-1}|\\
\textit{code for banner page}\\
|\newpage|\\
|\||fi|
\end{tabular}
\end{center}
%
Here one could write a message such as:
\begin{center}
|This is the part \childdocname{} of \childdocjob{}.|
\end{center}

%%%%%%%%%%%%%%%%%%%%%%%%%%%%%%%%%%%%%%%%%%%%%%%%%%%%%%%%%%%%%%%%%%%%%%%%%%%%%%%%
\subsection{Flags}
\label{sec:flags}

The package makes it easy to generate different versions
of the main or child documents.
To this end compilation flags can be defined
and assigned different default values.
They will be particularly useful in conjunction
with the forwarding mechanism described in \secref{sec:forward}.

For example, it may be useful to have a flag |\version|
which can be set to |draft| or |final|.
The document source will contain some conditional code
depending on the value of |\version|.
Suppose further, the flag should default to |final| for the main file
and to |draft| for child files
which is a natural assignment for editing the document.
This is achieved by placing the following code
in the preamble of the main document
(below the |\childdocmain| directive):
%
\begin{center}
\begin{tabular}{l}
|\ifchilddoc|\\
|\providecommand{\version}{draft}|\\
|\||else|\\
|\providecommand{\version}{final}|\\
|\||fi|
\end{tabular}
\end{center}
%
The definition by |\providecommand| makes sure
that previous definitions are not overwritten.
Further statements |\providecommand{\version}{...}|
can thus be added before the above code to override it.

For the main file, one might add a line
(between |\childdocmain| and the above block)
%
\begin{center}
|%\ifchilddoc\||else\providecommand{\version}{draft}\||fi|
\end{center}
%
which can be uncommented to produce a draft version.
Likewise one can add a line to the very top of a child file
(above the |\childdocof{|\textit{main}|}| directive)
%
\begin{center}
|%\providecommand{\version}{final}|
\end{center}
%
which can be uncommented to produce the final version of this child document.

%%%%%%%%%%%%%%%%%%%%%%%%%%%%%%%%%%%%%%%%%%%%%%%%%%%%%%%%%%%%%%%%%%%%%%%%%%%%%%%%
\subsection{Forwarding}
\label{sec:forward}

Different versions of the main or child documents
using compilation flags as described in \secref{sec:flags}
can be (permanently) stored in different files
for convenient compilation, viewing and distribution.
To this end, the package defines a command
to pass on compilation to a different file:

%%%%%%%%%%%%%%%%%%%%%%%%%%%%%%%%%%%%%%%%
\DescribeMacro{\childdocforward}
The command |\childdocforward| redirects processing to
another source file:
%
\begin{center}
\begin{tabular}{l}
|\input{childdoc.def}|\\
|\childdocforward[|\textit{main}|]{|\textit{dest}|}|\\
\end{tabular}
\end{center}
%
The argument \textit{dest} is the destination file
(without extension).
It should be the main file or one of the child files.
Note that further \textsf{childdoc} directives
such as |\childdocof| and |\childdocforward|
in the indicated file will be processed in this form.
The optional argument \textit{main}
passes on directly to the main file \textit{main}
while pretending to compile the child \textit{dest}.
This form behaves as if \textit{dest}
issues |\childdocof{|\textit{main}|}| right away,
and no further \textsf{childdoc} directives will be processed.

%%%%%%%%%%%%%%%%%%%%%%%%%%%%%%%%%%%%%%%%
\DescribeMacro{\...prefix}
In the alternative form |\childdocforwardprefix|,
%
\begin{center}
\begin{tabular}{l}
|\input{childdoc.def}|\\
|\childdocforwardprefix[|\textit{main}|]{|\textit{prefix}|}{|\textit{dest}|}|
\end{tabular}
\end{center}
%
the destination file is determined by a pattern
depending on the current file:
To make this work, the current file must be called
`{\textit{prefix}\hspace{0.2em}\textit{suffix}}'
with \textit{prefix} matching precisely the argument.
Processing is then passed on to the file
`{\textit{dest}\hspace{0.2em}\textit{suffix}}'.
Surely, the same effect is achieved by
directly specifying the
argument `{\textit{dest}\hspace{0.2em}\textit{suffix}}'
in the first form.
However, that requires to set up a different file
for each child. With the alternative form of the command
all these files can have exactly the same content
which simplifies setting them up and maintaining them.

For example, the following file |draft.tex|
with a compilation flag |\version| as described in \secref{sec:flags}
compiles the main document as a draft:
%
\begin{center}
\begin{tabular}{l}
|\def\version{draft}|\\
|\input{childdoc.def}|\\
|\childdocforward{|\textit{main}|}|
\end{tabular}
\end{center}
%
Likewise, the following files |final|\textit{nn}|.tex|
compile the final version of the child document
|child|\textit{nn}|.tex|:
%
\begin{center}
\begin{tabular}{l}
|\def\version{final}|\\
|\input{childdoc.def}|\\
|\childdocforwardprefix{final}{child}|
\end{tabular}
\end{center}
%

Note that when several versions of a main file and/or of each child file
are to be generated, it may be convenient to set up a |Makefile| or
shell script to automatise the process.

%%%%%%%%%%%%%%%%%%%%%%%%%%%%%%%%%%%%%%%%%%%%%%%%%%%%%%%%%%%%%%%%%%%%%%%%%%%%%%%%
\subsection{Command Line Processing}
\label{sec:commandline}

The effect of redirection files can also be achieved by invoking
the \LaTeX{} compiler with a more elaborate command line.
Most conveniently this should be done as part
of a shell script or a |Makefile|.

When using \textsf{childdoc} in the main file, the following
command lines effectively perform a redirection
(note that depending on the shell being used,
backslashes may have to be doubled: `|\|' $\to$ `|\\|'):
%
\begin{center}
|... -jobname "|\textit{target}|" |\\|"|[\textit{flags}]%
|\input{childdoc.def}\childdocforward[|\textit{main}|]{|\textit{dest}|}"|
\end{center}
%
Here \textit{target} is the name of the output file,
\textit{main} is the name of the main file
and \textit{dest} is the name of the main or child file to be processed
(all filenames without extensions).
The optional argument \textit{main} can be omitted
if \textit{main} matches \textit{dest}.
Optionally, compilation \textit{flags} can be defined via |\def| commands.
This command line makes the \TeX{} engine believe
it is compiling the file \textit{target}
whose content is specified as the latter parameter.
The provided code then forwards the processing to
\textit{main} or \textit{dest} as described in \secref{sec:forward}.

%%%%%%%%%%%%%%%%%%%%%%%%%%%%%%%%%%%%%%%%%%%%%%%%%%%%%%%%%%%%%%%%%%%%%%%%%%%%%%%%
\subsection{Include by Input}
\label{sec:input}

Including child documents by |\include| has some restrictions by design.
Most notably, the content of a child document always occupies
its own set of pages; pages cannot be shared between child documents.
Usually, this behaviour makes perfect sense
because each child document contain an essential part of the document.
However, in some situations it may be desirable to compose
a document from a collection of parts
without having mandatory page breaks between then.
For this case, the package
provides a mechanism to include parts
by |\input| which can also be processed individually.
However, by construction this mechanism
requires manual handling of the content to be output.

%%%%%%%%%%%%%%%%%%%%%%%%%%%%%%%%%%%%%%%%
\DescribeMacro{\ifchilddocmanual}
The main file should be prepared as usual, see \secref{sec:include}.
However, the document body must make a distinction
between processing of an individual part and of the main document, e.g.:
%
\begin{center}
\begin{tabular}{l}
|\ifchilddocmanual|\\
|\input{\childdocname}|\\
|\||else|\\
\textit{document body with }|\input{|\textit{part}|}|\\
|\||fi|
\end{tabular}
\end{center}
%
The conditional |\ifchilddocmanual| is true whenever
a part to be included by |\input| is being compiled,
and the name of the part is stored in |\childdocname|.

%%%%%%%%%%%%%%%%%%%%%%%%%%%%%%%%%%%%%%%%
\DescribeMacro{\childdocby}
Each part to be included by |\input| should start with:
%
\begin{center}
\begin{tabular}{l}
|\input{childdoc.def}|\\
|\childdocby{|\textit{main}|}|\\
\end{tabular}
\end{center}
%
The directive |\childdocby| is similar to |\childdocof|
described in \secref{sec:include},
but the subsequent selection of content must be done manually.
To that end, both |\ifchilddoc| and |\ifchilddocmanual|
will be true upon processing of a part,
and the name of the part is stored in |\childdocname|.
Note that |\jobname| will be set to the filename of the current part
so that each part receives an individual |.aux| file
that does not interfere with the |.aux| file(s) of the main document.
This behaviour can be altered by the alternative form
|\childdocby[*]{|\textit{main}|}| (with a non-empty optional argument)
which uses the |.aux| file of the main document
by setting |\jobname| to \textit{main}.

%%%%%%%%%%%%%%%%%%%%%%%%%%%%%%%%%%%%%%%%%%%%%%%%%%%%%%%%%%%%%%%%%%%%%%%%%%%%%%%%
\subsection{Driver Development}
\label{sec:driver}

The \textsf{childdoc} mechanism can also be use for the development
of definition files such as \LaTeX{} styles or classes.
This case differs from the above setup with multiple parts
included by |\include| in that no |\includeonly| should be invoked.
This can be achieved by starting the include file
(before |\ProvidesPackage|) with:
%
\begin{center}
\begin{tabular}{l}
|\input{childdoc.def}|\\
|\childdocforward{|\textit{main}|}|\\
\end{tabular}
\end{center}
%
or alternatively with:
%
\begin{center}
\begin{tabular}{l}
|\input{childdoc.def}|\\
|\childdocby{|\textit{main}|}|\\
\end{tabular}
\end{center}
%
Both forms have slightly different effects as described above.
The main file is prepared as usual, see \secref{sec:include}.

%%%%%%%%%%%%%%%%%%%%%%%%%%%%%%%%%%%%%%%%%%%%%%%%%%%%%%%%%%%%%%%%%%%%%%%%%%%%%%%%
\subsection{Legacy Detection}
\label{sec:detection}

The directive |\childdocmain| in the main file can detect
whether the complete document or merely a child is to be compiled
even without using the directive |\childdocof|.
This method is deprecated because it is less robust
and there is no compelling reason to use it;
it is merely provided for backward compatibility
and it may be removed in future versions.

If the detection mechanism is to be used,
it is mandatory to correctly specify
the filename of the main file as the argument of |\childdocmain|:
%
\begin{center}
\begin{tabular}{l}
|\input{childdoc.def}|\\
|\childdocmain{|\textit{main}|}|\\
\end{tabular}
\end{center}
%
If |\jobname| does not match the argument \textit{main} of |\childdocmain|,
it is assumed that |\jobname| points to the child file to be compiled.
When using |\childdocmain| with the main file specified as argument,
it suffices to start a child file
with just |\input{|\textit{main}|}|
without loading of the package and using |\childdocof|.
If instead all processing is done
with the appropriate \textsf{childdoc} directives,
the argument of \textit{main} of |\childdocmain| can be empty.

An alternative version of the command line processing described
in \secref{sec:commandline} using the detection mechanism reads:
%
\begin{center}
|... -jobname "|\textit{target}|" "|[\textit{flags}]%
[|\def\jobname{|\textit{dest}|}|]|\input{|\textit{main}|}"|
\end{center}

%%%%%%%%%%%%%%%%%%%%%%%%%%%%%%%%%%%%%%%%%%%%%%%%%%%%%%%%%%%%%%%%%%%%%%%%%%%%%%%%
\subsection{Manual Code}
\label{sec:manual}

In case one cannot be certain whether the definitions file |childdoc.def|
is installed on the target \TeX{} distribution
and one prefers not to ship it,
it is conceivable to paste a few relevant commands into the sources.

To that end, drop all statements |\input{childdoc.def}|
and perform the replacements as outlined below.
Instead of |\childdocmain{|\textit{main}|}| add the following code
to the top of the main file:
%
\begin{center}
\begin{tabular}{l}
|\||ifdefined\childdocname\endinput\||fi\newif\ifchilddoc|\\
|\edef\childdocname{\scantokens\expandafter{\jobname\noexpand}}|\\
|\def\childdocmain{|\textit{main}|}\||ifx\childdocmain\childdocname\||else|\\
|\childdoctrue\includeonly{\childdocname}\let\jobname\childdocmain\||fi|\\
\end{tabular}
\end{center}
%
Instead of |\childdocof{|\textit{main}|}| just include the main file
at the top of each child file:
%
\begin{center}
|\input{|\textit{main}|}|
\end{center}
%
A simple redirection |\childdocforward{|\textit{dest}|}| is achieved by:
%
\begin{center}
|\def\jobname{|\textit{dest}|}\input{\jobname}|
\end{center}
%
The redirection with prefix
|\childdocforwardprefix[|\textit{prefix}|]{|\textit{dest}|}|
is accomplished by:
%
\begin{center}
\begin{tabular}{l}
|{\edef\jobname{\scantokens\expandafter{\jobname\noexpand}}|\\
|\def\redirectjob |\textit{prefix}|#1~~~{\gdef\jobname{|\textit{dest}|#1}}|\\
|\expandafter\redirectjob\jobname~~~}\input{\jobname}|
\end{tabular}
\end{center}

In an alternative approach,
child documents can be compiled by a specific command line
without additional code or specific definitions:
%
\begin{center}
|... -jobname "|\textit{target}|" "|[\textit{flags}]%
|\includeonly{|\textit{dest}|}\input{|\textit{main}|}"|
\end{center}
%

%%%%%%%%%%%%%%%%%%%%%%%%%%%%%%%%%%%%%%%%%%%%%%%%%%%%%%%%%%%%%%%%%%%%%%%%%%%%%%%%
%%%%%%%%%%%%%%%%%%%%%%%%%%%%%%%%%%%%%%%%%%%%%%%%%%%%%%%%%%%%%%%%%%%%%%%%%%%%%%%%
\section{Information}

%%%%%%%%%%%%%%%%%%%%%%%%%%%%%%%%%%%%%%%%%%%%%%%%%%%%%%%%%%%%%%%%%%%%%%%%%%%%%%%%
\subsection{Copyright}

Copyright \copyright{} 2017--2018 Niklas Beisert

This work may be distributed and/or modified under the
conditions of the \LaTeX{} Project Public License, either version 1.3
of this license or (at your option) any later version.
The latest version of this license is in
  \url{http://www.latex-project.org/lppl.txt}
and version 1.3 or later is part of all distributions of \LaTeX{}
version 2005/12/01 or later.

This work has the LPPL maintenance status `maintained'.

The Current Maintainer of this work is Niklas Beisert.

This work consists of the files |README.txt|, |childdoc.ins| and |childdoc.dtx|
as well as the derived files |childdoc.def|, |cdocsamp.tex|
with |cdocsch1.tex|, |cdocsch2.tex|, |cdocspt3.tex|, |cdocspt4.tex|,
|cdocsdrf.tex|, |cdocsfn1.tex|, |cdocsfn2.tex|
as well as |childdoc.pdf|.

%%%%%%%%%%%%%%%%%%%%%%%%%%%%%%%%%%%%%%%%%%%%%%%%%%%%%%%%%%%%%%%%%%%%%%%%%%%%%%%%
\subsection{Files and Installation}

The package consists of the files:
%
\begin{center}
\begin{tabular}{ll}
    |README.txt|   & readme file \\
    |childdoc.ins| & installation file \\
    |childdoc.dtx| & source file \\
    |childdoc.def| & definition file \\
    |cdocsamp.tex| & sample main file \\
    |cdocsch1.tex| & sample include file \\
    |cdocsch2.tex| & sample include file \\
    |cdocspt3.tex| & sample part file \\
    |cdocspt4.tex| & sample part file \\
    |cdocsdrf.tex| & sample redirection file \\
    |cdocsfn1.tex| & sample redirection file \\
    |cdocsfn2.tex| & sample redirection file \\
    |childdoc.pdf| & manual
\end{tabular}
\end{center}
%
The distribution consists of the files
|README.txt|, |childdoc.ins| and |childdoc.dtx|.
%
\begin{itemize}
\item
Run (pdf)\LaTeX{} on |childdoc.dtx|
to compile the manual |childdoc.pdf| (this file).
\item
Run \LaTeX{} on |childdoc.ins| to create the definitions file |childdoc.def|
and the sample |cdocsamp.tex| with include files
|cdocsch1.tex|, |cdocsch2.tex|, |cdocspt3.tex|, |cdocspt4.tex|,
|cdocsdrf.tex|, |cdocsfn1.tex|, |cdocsfn2.tex|.
Then copy the file |childdoc.def| to an appropriate directory of your \LaTeX{}
distribution, e.g.\ \textit{texmf-root}|/tex/latex/childdoc|.
\end{itemize}

%%%%%%%%%%%%%%%%%%%%%%%%%%%%%%%%%%%%%%%%%%%%%%%%%%%%%%%%%%%%%%%%%%%%%%%%%%%%%%%%
\subsection{Related CTAN Packages}

There are several other packages which offer a similar functionality:
%
\begin{itemize}
\item
The packages
\href{http://ctan.org/pkg/docmute}{\textsf{docmute}},
\href{http://ctan.org/pkg/includex}{\textsf{includex}} and
\href{http://ctan.org/pkg/standalone}{\textsf{standalone}}
provide commands to include only the document body of
a child file thus allowing both files to be compiled individually.
\item
The packages \href{http://ctan.org/pkg/subdocs}{\textsf{subdocs}}
and \href{http://ctan.org/pkg/subfiles}{\textsf{subfiles}}
provide structures in which the main and child documents can be
encapsulated and allowing them to be compiled individually.
The inclusion mechanism is different from the conventional |\include|.
\item
The package \href{http://ctan.org/pkg/combine}{\textsf{combine}}
is an elaborate solution to combine several documents into one.
\end{itemize}
%
See also the CTAN topic \href{http://ctan.org/topic/subdocs}{\textsf{subdocs}}
for further related packages.
The present package differs from the above solutions in that
a document structure constructed with the conventional |\include| mechanism
just needs two extra commands at the top of every file
such that all constituent files can be compiled individually.

%%%%%%%%%%%%%%%%%%%%%%%%%%%%%%%%%%%%%%%%%%%%%%%%%%%%%%%%%%%%%%%%%%%%%%%%%%%%%%%%
%\subsection{Feature Suggestions}
%
%The following is a list of features which may be useful for future
%versions of this package:
%%
%\begin{itemize}
%\item
%\ldots
%\end{itemize}

%%%%%%%%%%%%%%%%%%%%%%%%%%%%%%%%%%%%%%%%%%%%%%%%%%%%%%%%%%%%%%%%%%%%%%%%%%%%%%%%
\subsection{Revision History}

%%%%%%%%%%%%%%%%%%%%%%%%%%%%%%%%%%%%%%%%
\paragraph{v2.0:} 2018/12/30

\begin{itemize}
\item
immediate forward processing
\item
added |\childdocby| mechanism
\item
manual restructured
\end{itemize}

%%%%%%%%%%%%%%%%%%%%%%%%%%%%%%%%%%%%%%%%
\paragraph{v1.6:} 2018/01/17

\begin{itemize}
\item
application for development of include files
\item
corrections to manual
\end{itemize}

%%%%%%%%%%%%%%%%%%%%%%%%%%%%%%%%%%%%%%%%
\paragraph{v1.5:} 2017/05/21

\begin{itemize}
\item
more complete structuring introduced
\item
|\childdocof| introduced
\item
|\childdoc| renamed to |\childdocmain|
\item
|\childredirect| renamed to |\childdocforward| and |\childdocforwardprefix|
and functionality expanded
\end{itemize}

%%%%%%%%%%%%%%%%%%%%%%%%%%%%%%%%%%%%%%%%
\paragraph{v1.0:} 2017/04/27

\begin{itemize}
\item
manual and install package
\item
first version published on CTAN
\end{itemize}

%%%%%%%%%%%%%%%%%%%%%%%%%%%%%%%%%%%%%%%%
\paragraph{v0.6:} 2017/04/26

\begin{itemize}
\item
redirection mechanism added
\end{itemize}

%%%%%%%%%%%%%%%%%%%%%%%%%%%%%%%%%%%%%%%%
\paragraph{v0.5:} 2017/04/26

\begin{itemize}
\item
functionality in definition file
\end{itemize}


%%%%%%%%%%%%%%%%%%%%%%%%%%%%%%%%%%%%%%%%%%%%%%%%%%%%%%%%%%%%%%%%%%%%%%%%%%%%%%%%
%%%%%%%%%%%%%%%%%%%%%%%%%%%%%%%%%%%%%%%%%%%%%%%%%%%%%%%%%%%%%%%%%%%%%%%%%%%%%%%%
%%%%%%%%%%%%%%%%%%%%%%%%%%%%%%%%%%%%%%%%%%%%%%%%%%%%%%%%%%%%%%%%%%%%%%%%%%%%%%%%
\appendix

\settowidth\MacroIndent{\rmfamily\scriptsize 000\ }

 \DocInput{childdoc.dtx}

\end{document}
%</driver>
% \fi
%
% %%%%%%%%%%%%%%%%%%%%%%%%%%%%%%%%%%%%%%%%%%%%%%%%%%%%%%%%%%%%%%%%%%%%%%%%%%%%%%
% %%%%%%%%%%%%%%%%%%%%%%%%%%%%%%%%%%%%%%%%%%%%%%%%%%%%%%%%%%%%%%%%%%%%%%%%%%%%%%
% \section{Sample}
%\iffalse
%<*samplemain>
%\fi
%
% The following presents a sample document
% with two chapters, two parts, a title page,
% a compile flag as well as three forwarding files to set the flag.
% It consists of eight |.tex| files:
% \begin{center}
% \begin{tabular}{ll}
% |cdocsamp.tex|&main file\\
% |cdocsch1.tex|&include file for chapter 1\\
% |cdocsch2.tex|&include file for chapter 2\\
% |cdocspt3.tex|&include file for part 3\\
% |cdocspt4.tex|&include file for part 4\\
% |cdocsdrf.tex|&forwarding file for main file in draft mode\\
% |cdocsfi1.tex|&forwarding file for final version of chapter 1\\
% |cdocsfi2.tex|&forwarding file for final version of chapter 2\\
% \end{tabular}
% \end{center}
% Each of the eight files can be compiled directly by the \LaTeX{} compiler.
%
% %%%%%%%%%%%%%%%%%%%%%%%%%%%%%%%%%%%%%%
% \paragraph{Main File.}
%
% The main file is called |cdocsamp.tex|.
%
% Load the \textsf{childdoc} definitions and
% declare the filename for the main document:
%    \begin{macrocode}
\input{childdoc.def}
\childdocmain{}
%    \end{macrocode}

% Optional override for |\version| flag:
%    \begin{macrocode}
%%\ifchilddoc\else\providecommand{\version}{draft}\fi
%    \end{macrocode}

% Define the default values for the |\version| flag
% (|final| for the main file and |draft| for childs):
%    \begin{macrocode}
\ifchilddoc
\providecommand{\version}{draft}
\else
\providecommand{\version}{final}
\fi
%    \end{macrocode}

% Load the standard document class:
%    \begin{macrocode}
\documentclass[12pt]{article}
%    \end{macrocode}

% Start the document body:
%    \begin{macrocode}
\begin{document}
%    \end{macrocode}

% Declare a title page.
% Print title, part of document being processed and version flag:
%    \begin{macrocode}
\addtocounter{page}{-1}
\begin{center}
{\LARGE\bfseries{}childdoc example\par}
\vspace{1cm}
\ifchilddoc
\ifchilddocmanual part\else chapter\fi:
`\childdocname' of `\childdocjob'\par
\else
main document: `\childdocjob'\par
\fi
version: \version\par
\end{center}
\newpage
%    \end{macrocode}

% Manually include selected file,
% otherwise process as usual:
%    \begin{macrocode}
\ifchilddocmanual
\section*{part `\childdocname'}
\input{\childdocname}
\else
%    \end{macrocode}

% Include the two chapters:
%    \begin{macrocode}
\include{cdocsch1}
\include{cdocsch2}
%    \end{macrocode}

% Include the two parts unless only chapters should be displayed:
%    \begin{macrocode}
\ifchilddoc\else
\section{part three}
\input{cdocspt3}
\section{part four}
\input{cdocspt4}
\fi
%    \end{macrocode}

% Process as usual until here:
%    \begin{macrocode}
\fi
%    \end{macrocode}

% End of document body:
%    \begin{macrocode}
\end{document}
%    \end{macrocode}
%\iffalse
%</samplemain>
%\fi
%
% %%%%%%%%%%%%%%%%%%%%%%%%%%%%%%%%%%%%%%
% \paragraph{Chapter Include Files.}
%
% The include files are called |cdocsch1.tex| and |cdocsch2.tex|.
%
%\iffalse
%<*samplechap1|samplechap2>
%\fi

% Optional override for |\version| flag:
%    \begin{macrocode}
%%\providecommand{\version}{final}
%    \end{macrocode}

% Include the main document:
%    \begin{macrocode}
\input{childdoc.def}
\childdocof{cdocsamp}
%    \end{macrocode}

%\iffalse
%</samplechap1|samplechap2>
%\fi
%
%\iffalse
%<*samplechap1>
%\fi
% Some text for chapter 1:
%    \begin{macrocode}
\section{one}
some text in chapter one
%    \end{macrocode}

%\iffalse
%</samplechap1>
%\fi
% Some text for chapter 2:
%\iffalse
%<*samplechap2>
%\fi
%    \begin{macrocode}
\section{two}
more text in chapter two
%    \end{macrocode}

%\iffalse
%</samplechap2>
%\fi
%
% %%%%%%%%%%%%%%%%%%%%%%%%%%%%%%%%%%%%%%
% \paragraph{Part Include Files.}
%
% The include files are called |cdocspt3.tex| and |cdocspt4.tex|.
%
%\iffalse
%<*samplepart3|samplepart4>
%\fi

% Optional override for |\version| flag:
%    \begin{macrocode}
%%\providecommand{\version}{final}
%    \end{macrocode}

% Include the main document:
%    \begin{macrocode}
\input{childdoc.def}
\childdocby{cdocsamp}
%    \end{macrocode}

%\iffalse
%</samplepart3|samplepart4>
%\fi
%
%\iffalse
%<*samplepart3>
%\fi
% Some text for part 3:
%    \begin{macrocode}
some text in part three
%    \end{macrocode}

%\iffalse
%</samplepart3>
%\fi
% Some text for part 4:
%\iffalse
%<*samplepart4>
%\fi
%    \begin{macrocode}
more text in part four
%    \end{macrocode}

%\iffalse
%</samplepart4>
%\fi
%
% %%%%%%%%%%%%%%%%%%%%%%%%%%%%%%%%%%%%%%
% \paragraph{Forwarding for a Complete Draft.}
%
% The following forwarding file |cdocsdrf.tex|
% compiles the main document in draft mode:
%\iffalse
%<*sampledraft>
%\fi
%    \begin{macrocode}
\def\version{draft}
\input{childdoc.def}
\childdocforward{cdocsamp}
%    \end{macrocode}

%\iffalse
%</sampledraft>
%\fi
%
% %%%%%%%%%%%%%%%%%%%%%%%%%%%%%%%%%%%%%%
% \paragraph{Forwarding for Final Version of the Chapters.}
%
% The following forwarding files |cdocsfn1.tex| and |cdocsfn2.tex|
% (with identical content)
% compile the final versions of the child documents
% |cdocsch1.tex| and |cdocsch2.tex|, respectively:
%\iffalse
%<*samplefinal>
%\fi
%    \begin{macrocode}
\def\version{final}
\input{childdoc.def}
\childdocforwardprefix[cdocsamp]{cdocsfn}{cdocsch}
%    \end{macrocode}

%\iffalse
%</samplefinal>
%\fi
%
% %%%%%%%%%%%%%%%%%%%%%%%%%%%%%%%%%%%%%%
% \paragraph{Command Line Processing.}
%
% The following three command lines generate the output files
% |cdocscld|, |cdocscl1| and |cdocscl2|
% which should be identical to
% |cdocsdrf|, |cdocsch1| and |cdocsfn2|, respectively:
% \begin{center}
% \begin{tabular}{l}
% |latex -jobname cdocscld \|\\
% |  "\def\version{draft}\input{childdoc.def}\childdocforward{cdocsamp}"|\\
% |latex -jobname cdocscl1 \|\\
% |  "\input{childdoc.def}\childdocforward[cdocsamp]{cdocsch1}"|\\
% |latex -jobname cdocscl2 \|\\
% |  "\def\version{final}\input{childdoc.def}\childdocforward{cdocsch2}"|
% \end{tabular}
% \end{center}
% Note that the trailing backslash on each first line
% merely continues the input to the second line
% (for convenient cut ant paste).
% Furthermore, the command |latex| can be replaced by any
% of its alternative versions such as |pdflatex|.
%
% %%%%%%%%%%%%%%%%%%%%%%%%%%%%%%%%%%%%%%%%%%%%%%%%%%%%%%%%%%%%%%%%%%%%%%%%%%%%%%
% %%%%%%%%%%%%%%%%%%%%%%%%%%%%%%%%%%%%%%%%%%%%%%%%%%%%%%%%%%%%%%%%%%%%%%%%%%%%%%
% \section{Implementation}
%\iffalse
%<*package>
%\fi
%
% This section describes the definitions file |childdoc.def|.

% The definitions cannot be loaded using |\usepackage| or |\RequirePackage|
% which has a mechanism to prevent loading a style file more than once.
% When loading the definitions by means of |\input|
% multiple instances have to be prevented manually:
%\iffalse
%This code needs to be before the `\ProvidesFile' directive
%which is defined at the beginning of this file.
%Therefore it is also placed there and commented out here.
%</package>
%<*discard>
%\fi
%    \begin{macrocode}
\ifdefined\childdocmain\endinput\fi
%    \end{macrocode}
%\iffalse
%</discard>
%<*package>
%\fi
%
% \macro{\ifchilddoc}
% \macro{\ifchilddocmanual}
% The conditional |\ifchilddoc| tells whether a
% child (true) or main (false) document is being compiled.
% The conditional |\ifchilddocmanual| tells whether
% the |\includeonly| mechanism is used (false) or
% the selection of child files must be performed manually (true).
% The definitions initialise to false:
%    \begin{macrocode}
\newif\ifchilddoc
\newif\ifchilddocmanual
%    \end{macrocode}

% \macro{\childdocname}
% \macro{\childdocjob}
% The macro |\childdocname| stores the name of the main document
% to be compiled. The macro |\childdocjob| stores the name of
% the document on which the \LaTeX{} compiler was originally invoked.
% The content of |\jobname| cannot be compared
% to filenames specified in the source due to different catcodes.
% The following code rescans |\jobname|, stores the result
% in |\childdocname| and saves a copy in |\childdocjob|:
%    \begin{macrocode}
\edef\childdocname{\scantokens\expandafter{\jobname\noexpand}}
\let\childdocjob\childdocname
%    \end{macrocode}

% \macro{\childdocdisable}
% The macro |\childdocdisable| prevents the main file
% from being processed more than once.
% At this stage, the main document command |\childdocmain|
% is assumed to be called once again where it should do nothing.
% Any subsequent call to it should prevent
% a secondary processing of the main document
% It overwrites the forwarding commands
% |\childdocof| and |\childdocforward|
% with empty macros to prevent further inclusions of the main document:
%    \begin{macrocode}
\newcommand{\childdocdisable}
{
  \renewcommand{\childdocmain}[1]{\renewcommand{\childdocmain}[1]{\endinput}}
  \renewcommand{\childdocof}[1]{}
  \renewcommand{\childdocby}[2][]{}
  \renewcommand{\childdocforward}[2][]{}
  \renewcommand{\childdocdisable}{}
}
%    \end{macrocode}

% \macro{\childdocmain}
% The macro |\childdocmain| is to be called at the top of the main file
% with nothing or the main filename (without extension) as argument.
% First, it breaks loops.
% If the argument is not empty and does not match |\childdocname|
% (which is set by the first inclusion of |childdoc.def|),
% |\ifchilddoc| is set to true, |\includeonly| is applied to the child file
% and |\jobname| is set to the main file
% (for proper handling of |.aux| files):
%    \begin{macrocode}
\newcommand{\childdocmain}[1]
{
  \childdocdisable\childdocmain{}
  \if?#1?\else
    \begingroup
      \def\childdoctmp{#1}
      \ifx\childdoctmp\childdocname
        \def\childdoctmp{}
      \else
        \def\childdoctmp
        {
          \childdoctrue
          \includeonly{\childdocname}
          \def\childdocjob{#1}
          \def\jobname{#1}
        }
      \fi
      \expandafter
    \endgroup
    \childdoctmp
  \fi
}
%    \end{macrocode}

% \macro{\childdocof}
% The command |\childdocof| redirects
% compilation to the main file |#1|.
%    \begin{macrocode}
\newcommand{\childdocof}[1]
{
  \childdocdisable
  \childdoctrue
  \includeonly{\childdocname}
  \def\jobname{#1}
  \def\childdocjob{#1}
  \input{#1}
}
%    \end{macrocode}

% \macro{\childdocby}
% The command |\childdocby| ....
%    \begin{macrocode}
\newcommand{\childdocby}[2][]
{
  \childdocdisable
  \childdoctrue
  \childdocmanualtrue
  \if?#1?\else
    \def\jobname{#2}
  \fi
  \def\childdocjob{#2}
  \input{#2}
  \endinput
}
%    \end{macrocode}

% \macro{\childdocforward}
% The command |\childdocforward| redirects
% compilation to the main file or
% (if the optional argument is given) a child file.
% Parameters are set as if the main file
% or a child file starting with |\childdocof| was compiled.
% Then compilation is handed over to the main file:
%    \begin{macrocode}
\newcommand{\childdocforward}[2][]
{
  \begingroup
    \if?#1?
      \def\childdoctmp
      {
        \def\childdocname{#2}
        \def\childdocjob{#2}
        \def\jobname{#2}
        \input{#2}
        \endinput
      }
    \else
      \def\childdoctmp
      {
        \childdocdisable
        \def\childdocname{#2}
        \childdoctrue
        \includeonly{#2}
        \def\childdocjob{#1}
        \def\jobname{#1}
        \input{#1}
        \endinput
      }
    \fi
    \expandafter
  \endgroup
  \childdoctmp
}
%    \end{macrocode}

% \macro{\childdocforwardprefix}
% The command |\childdocforwardprefix| redirects
% compilation to the main or a child file by means of a pattern.
% The prefix |#1| in the current filename is replaced by |#2|
% and the suffix of the current filename is kept
% (it is assumed that the filename does not contain the substring `|~~~|'
% which is used as a delimiter).
% Compilation is handed over to the new file by |\childdocforward|:
%    \begin{macrocode}
\newcommand{\childdocforwardprefix}[3][]
{
  \begingroup
    \def\childdocextract #2##1~~~{\def\childdoctmp{\childdocforward[#1]{#3##1}}}
    \expandafter\childdocextract\childdocname~~~
    \expandafter
  \endgroup
  \childdoctmp
}
%    \end{macrocode}

% \macro{\childdoc}
% The deprecated macro |\childdoc| is a legacy version of |\childdocmain|:
%    \begin{macrocode}
\newcommand{\childdoc}{\childdocmain}
%    \end{macrocode}

% \macro{\childdocredirect}
% The deprecated macro |\childdocredirect| is a legacy version
% of |\childdocforward| and |\childdocforwardprefix|:
%    \begin{macrocode}
\newcommand{\childdocredirect}[2][]
{
  \begingroup
    \if?#1?
      \def\childdoctmp{\childdocforward{#2}}
    \else
      \def\childdoctmp{\childdocforwardprefix{#1}{#2}}
    \fi
    \expandafter
  \endgroup
  \childdoctmp
}
%    \end{macrocode}

%\iffalse
%</package>
%\fi
%
\endinput
|
and perform the replacements as outlined below.
Instead of |\childdocmain{|\textit{main}|}| add the following code
to the top of the main file:
%
\begin{center}
\begin{tabular}{l}
|\||ifdefined\childdocname\endinput\||fi\newif\ifchilddoc|\\
|\edef\childdocname{\scantokens\expandafter{\jobname\noexpand}}|\\
|\def\childdocmain{|\textit{main}|}\||ifx\childdocmain\childdocname\||else|\\
|\childdoctrue\includeonly{\childdocname}\let\jobname\childdocmain\||fi|\\
\end{tabular}
\end{center}
%
Instead of |\childdocof{|\textit{main}|}| just include the main file
at the top of each child file:
%
\begin{center}
|\input{|\textit{main}|}|
\end{center}
%
A simple redirection |\childdocforward{|\textit{dest}|}| is achieved by:
%
\begin{center}
|\def\jobname{|\textit{dest}|}\input{\jobname}|
\end{center}
%
The redirection with prefix
|\childdocforwardprefix[|\textit{prefix}|]{|\textit{dest}|}|
is accomplished by:
%
\begin{center}
\begin{tabular}{l}
|{\edef\jobname{\scantokens\expandafter{\jobname\noexpand}}|\\
|\def\redirectjob |\textit{prefix}|#1~~~{\gdef\jobname{|\textit{dest}|#1}}|\\
|\expandafter\redirectjob\jobname~~~}\input{\jobname}|
\end{tabular}
\end{center}

In an alternative approach,
child documents can be compiled by a specific command line
without additional code or specific definitions:
%
\begin{center}
|... -jobname "|\textit{target}|" "|[\textit{flags}]%
|\includeonly{|\textit{dest}|}\input{|\textit{main}|}"|
\end{center}
%

%%%%%%%%%%%%%%%%%%%%%%%%%%%%%%%%%%%%%%%%%%%%%%%%%%%%%%%%%%%%%%%%%%%%%%%%%%%%%%%%
%%%%%%%%%%%%%%%%%%%%%%%%%%%%%%%%%%%%%%%%%%%%%%%%%%%%%%%%%%%%%%%%%%%%%%%%%%%%%%%%
\section{Information}

%%%%%%%%%%%%%%%%%%%%%%%%%%%%%%%%%%%%%%%%%%%%%%%%%%%%%%%%%%%%%%%%%%%%%%%%%%%%%%%%
\subsection{Copyright}

Copyright \copyright{} 2017--2018 Niklas Beisert

This work may be distributed and/or modified under the
conditions of the \LaTeX{} Project Public License, either version 1.3
of this license or (at your option) any later version.
The latest version of this license is in
  \url{http://www.latex-project.org/lppl.txt}
and version 1.3 or later is part of all distributions of \LaTeX{}
version 2005/12/01 or later.

This work has the LPPL maintenance status `maintained'.

The Current Maintainer of this work is Niklas Beisert.

This work consists of the files |README.txt|, |childdoc.ins| and |childdoc.dtx|
as well as the derived files |childdoc.def|, |cdocsamp.tex|
with |cdocsch1.tex|, |cdocsch2.tex|, |cdocspt3.tex|, |cdocspt4.tex|,
|cdocsdrf.tex|, |cdocsfn1.tex|, |cdocsfn2.tex|
as well as |childdoc.pdf|.

%%%%%%%%%%%%%%%%%%%%%%%%%%%%%%%%%%%%%%%%%%%%%%%%%%%%%%%%%%%%%%%%%%%%%%%%%%%%%%%%
\subsection{Files and Installation}

The package consists of the files:
%
\begin{center}
\begin{tabular}{ll}
    |README.txt|   & readme file \\
    |childdoc.ins| & installation file \\
    |childdoc.dtx| & source file \\
    |childdoc.def| & definition file \\
    |cdocsamp.tex| & sample main file \\
    |cdocsch1.tex| & sample include file \\
    |cdocsch2.tex| & sample include file \\
    |cdocspt3.tex| & sample part file \\
    |cdocspt4.tex| & sample part file \\
    |cdocsdrf.tex| & sample redirection file \\
    |cdocsfn1.tex| & sample redirection file \\
    |cdocsfn2.tex| & sample redirection file \\
    |childdoc.pdf| & manual
\end{tabular}
\end{center}
%
The distribution consists of the files
|README.txt|, |childdoc.ins| and |childdoc.dtx|.
%
\begin{itemize}
\item
Run (pdf)\LaTeX{} on |childdoc.dtx|
to compile the manual |childdoc.pdf| (this file).
\item
Run \LaTeX{} on |childdoc.ins| to create the definitions file |childdoc.def|
and the sample |cdocsamp.tex| with include files
|cdocsch1.tex|, |cdocsch2.tex|, |cdocspt3.tex|, |cdocspt4.tex|,
|cdocsdrf.tex|, |cdocsfn1.tex|, |cdocsfn2.tex|.
Then copy the file |childdoc.def| to an appropriate directory of your \LaTeX{}
distribution, e.g.\ \textit{texmf-root}|/tex/latex/childdoc|.
\end{itemize}

%%%%%%%%%%%%%%%%%%%%%%%%%%%%%%%%%%%%%%%%%%%%%%%%%%%%%%%%%%%%%%%%%%%%%%%%%%%%%%%%
\subsection{Related CTAN Packages}

There are several other packages which offer a similar functionality:
%
\begin{itemize}
\item
The packages
\href{http://ctan.org/pkg/docmute}{\textsf{docmute}},
\href{http://ctan.org/pkg/includex}{\textsf{includex}} and
\href{http://ctan.org/pkg/standalone}{\textsf{standalone}}
provide commands to include only the document body of
a child file thus allowing both files to be compiled individually.
\item
The packages \href{http://ctan.org/pkg/subdocs}{\textsf{subdocs}}
and \href{http://ctan.org/pkg/subfiles}{\textsf{subfiles}}
provide structures in which the main and child documents can be
encapsulated and allowing them to be compiled individually.
The inclusion mechanism is different from the conventional |\include|.
\item
The package \href{http://ctan.org/pkg/combine}{\textsf{combine}}
is an elaborate solution to combine several documents into one.
\end{itemize}
%
See also the CTAN topic \href{http://ctan.org/topic/subdocs}{\textsf{subdocs}}
for further related packages.
The present package differs from the above solutions in that
a document structure constructed with the conventional |\include| mechanism
just needs two extra commands at the top of every file
such that all constituent files can be compiled individually.

%%%%%%%%%%%%%%%%%%%%%%%%%%%%%%%%%%%%%%%%%%%%%%%%%%%%%%%%%%%%%%%%%%%%%%%%%%%%%%%%
%\subsection{Feature Suggestions}
%
%The following is a list of features which may be useful for future
%versions of this package:
%%
%\begin{itemize}
%\item
%\ldots
%\end{itemize}

%%%%%%%%%%%%%%%%%%%%%%%%%%%%%%%%%%%%%%%%%%%%%%%%%%%%%%%%%%%%%%%%%%%%%%%%%%%%%%%%
\subsection{Revision History}

%%%%%%%%%%%%%%%%%%%%%%%%%%%%%%%%%%%%%%%%
\paragraph{v2.0:} 2018/12/30

\begin{itemize}
\item
immediate forward processing
\item
added |\childdocby| mechanism
\item
manual restructured
\end{itemize}

%%%%%%%%%%%%%%%%%%%%%%%%%%%%%%%%%%%%%%%%
\paragraph{v1.6:} 2018/01/17

\begin{itemize}
\item
application for development of include files
\item
corrections to manual
\end{itemize}

%%%%%%%%%%%%%%%%%%%%%%%%%%%%%%%%%%%%%%%%
\paragraph{v1.5:} 2017/05/21

\begin{itemize}
\item
more complete structuring introduced
\item
|\childdocof| introduced
\item
|\childdoc| renamed to |\childdocmain|
\item
|\childredirect| renamed to |\childdocforward| and |\childdocforwardprefix|
and functionality expanded
\end{itemize}

%%%%%%%%%%%%%%%%%%%%%%%%%%%%%%%%%%%%%%%%
\paragraph{v1.0:} 2017/04/27

\begin{itemize}
\item
manual and install package
\item
first version published on CTAN
\end{itemize}

%%%%%%%%%%%%%%%%%%%%%%%%%%%%%%%%%%%%%%%%
\paragraph{v0.6:} 2017/04/26

\begin{itemize}
\item
redirection mechanism added
\end{itemize}

%%%%%%%%%%%%%%%%%%%%%%%%%%%%%%%%%%%%%%%%
\paragraph{v0.5:} 2017/04/26

\begin{itemize}
\item
functionality in definition file
\end{itemize}


%%%%%%%%%%%%%%%%%%%%%%%%%%%%%%%%%%%%%%%%%%%%%%%%%%%%%%%%%%%%%%%%%%%%%%%%%%%%%%%%
%%%%%%%%%%%%%%%%%%%%%%%%%%%%%%%%%%%%%%%%%%%%%%%%%%%%%%%%%%%%%%%%%%%%%%%%%%%%%%%%
%%%%%%%%%%%%%%%%%%%%%%%%%%%%%%%%%%%%%%%%%%%%%%%%%%%%%%%%%%%%%%%%%%%%%%%%%%%%%%%%
\appendix

\settowidth\MacroIndent{\rmfamily\scriptsize 000\ }

 \DocInput{childdoc.dtx}

\end{document}
%</driver>
% \fi
%
% %%%%%%%%%%%%%%%%%%%%%%%%%%%%%%%%%%%%%%%%%%%%%%%%%%%%%%%%%%%%%%%%%%%%%%%%%%%%%%
% %%%%%%%%%%%%%%%%%%%%%%%%%%%%%%%%%%%%%%%%%%%%%%%%%%%%%%%%%%%%%%%%%%%%%%%%%%%%%%
% \section{Sample}
%\iffalse
%<*samplemain>
%\fi
%
% The following presents a sample document
% with two chapters, two parts, a title page,
% a compile flag as well as three forwarding files to set the flag.
% It consists of eight |.tex| files:
% \begin{center}
% \begin{tabular}{ll}
% |cdocsamp.tex|&main file\\
% |cdocsch1.tex|&include file for chapter 1\\
% |cdocsch2.tex|&include file for chapter 2\\
% |cdocspt3.tex|&include file for part 3\\
% |cdocspt4.tex|&include file for part 4\\
% |cdocsdrf.tex|&forwarding file for main file in draft mode\\
% |cdocsfi1.tex|&forwarding file for final version of chapter 1\\
% |cdocsfi2.tex|&forwarding file for final version of chapter 2\\
% \end{tabular}
% \end{center}
% Each of the eight files can be compiled directly by the \LaTeX{} compiler.
%
% %%%%%%%%%%%%%%%%%%%%%%%%%%%%%%%%%%%%%%
% \paragraph{Main File.}
%
% The main file is called |cdocsamp.tex|.
%
% Load the \textsf{childdoc} definitions and
% declare the filename for the main document:
%    \begin{macrocode}
% \iffalse
%
% childdoc.dtx Copyright (C) 2017-2018 Niklas Beisert
%
% This work may be distributed and/or modified under the
% conditions of the LaTeX Project Public License, either version 1.3
% of this license or (at your option) any later version.
% The latest version of this license is in
%   http://www.latex-project.org/lppl.txt
% and version 1.3 or later is part of all distributions of LaTeX
% version 2005/12/01 or later.
%
% This work has the LPPL maintenance status `maintained'.
%
% The Current Maintainer of this work is Niklas Beisert.
%
% This work consists of the files childdoc.dtx and childdoc.ins
% and the derived files childdoc.def and cdocsamp.tex with
% cdocsch1.tex, cdocsch2.tex, cdocsdrf.tex, cdocsfn1.tex, cdocsfn2.tex.
%
%<package>\ifdefined\childdocmain\endinput\fi
%<package>\ProvidesFile{childdoc.def}[2018/12/30 v2.0 child document driver]
%<samplemain>\ProvidesFile{cdocsamp.tex}[2018/12/30 v2.0 sample for childdoc]
%<*driver>
%\ProvidesFile{childdoc.drv}[2018/12/30 v2.0 childdoc reference manual file]
\PassOptionsToClass{10pt,a4paper}{article}
\documentclass{ltxdoc}

\usepackage[margin=35mm]{geometry}
\usepackage{hyperref}
\usepackage{hyperxmp}
\usepackage[usenames]{color}

\hypersetup{colorlinks=true}
\hypersetup{pdfstartview=FitH}
\hypersetup{pdfpagemode=UseNone}
\hypersetup{pdfsource={}}
\hypersetup{pdflang={en-UK}}
\hypersetup{pdfcopyright={Copyright 2017-2018 Niklas Beisert.
  This work may be distributed and/or modified under the
  conditions of the LaTeX Project Public License, either version 1.3
  of this license or (at your option) any later version.}}
\hypersetup{pdflicenseurl={http://www.latex-project.org/lppl.txt}}
\hypersetup{pdfcontactaddress={ETH Zurich, ITP, HIT K,
  Wolfgang-Pauli-Strasse 27}}
\hypersetup{pdfcontactpostcode={8093}}
\hypersetup{pdfcontactcity={Zurich}}
\hypersetup{pdfcontactcountry={Switzerland}}
\hypersetup{pdfcontactemail={nbeisert@itp.phys.ethz.ch}}
\hypersetup{pdfcontacturl={http://people.phys.ethz.ch/\xmptilde nbeisert/}}

\newcommand{\secref}[1]{\hyperref[#1]{section \ref*{#1}}}

\parskip1ex
\parindent0pt
\let\olditemize\itemize
\def\itemize{\olditemize\parskip0pt}

\begin{document}

\title{The \textsf{childdoc} Package}
\hypersetup{pdftitle={The childdoc Package}}
\author{Niklas Beisert\\[2ex]
  Institut f\"ur Theoretische Physik\\
  Eidgen\"ossische Technische Hochschule Z\"urich\\
  Wolfgang-Pauli-Strasse 27, 8093 Z\"urich, Switzerland\\[1ex]
  \href{mailto:nbeisert@itp.phys.ethz.ch}
  {\texttt{nbeisert@itp.phys.ethz.ch}}}
\hypersetup{pdfauthor={Niklas Beisert}}
\hypersetup{pdfsubject={Manual for the LaTeX2e Package childdoc}}
\date{30 December 2018, \textsf{v2.0}}
\maketitle

\begin{abstract}\noindent
\textsf{childdoc} is a \LaTeXe{} package
that enables the direct compilation
of document sections included by |\include|
to individual files.
\end{abstract}

\begingroup
\parskip0ex
\tableofcontents
\endgroup

%%%%%%%%%%%%%%%%%%%%%%%%%%%%%%%%%%%%%%%%%%%%%%%%%%%%%%%%%%%%%%%%%%%%%%%%%%%%%%%%
%%%%%%%%%%%%%%%%%%%%%%%%%%%%%%%%%%%%%%%%%%%%%%%%%%%%%%%%%%%%%%%%%%%%%%%%%%%%%%%%
\section{Introduction}

\LaTeX{} provides a mechanism to structure a large document (such as a book)
into a main file and several child files (containing the chapters)
using the |\include| command.
This mechanism is beneficial for documents
which span hundreds of pages in order to
make the source file(s) more manageable.
Moreover, compilation can be restricted to
selected child files by means of the |\includeonly| command.
The latter feature can be used to reduce the compilation time while editing
(this was significantly more useful in the earlier days of \LaTeX{})
or to generate a smaller document which is easier to navigate.
Another application of |\includeonly| is to generate
documents consisting of selected parts of the complete document.

However, there are a few drawbacks of the plain |\include| mechanism:
\begin{itemize}
\item
The child files cannot be compiled on their own,
they can only be compiled via the main file.
A naive editing environment
(such as a text editor with an option
to have the current file processed by \LaTeX)
may require one to switch to the main file before compiling;
attempting to compile the child file produces errors.
\item
The main file must be modified (each time)
to adjust the |\includeonly| command
to the present needs. This easily leaves the main file in a messy state.
\item
The generated document will always carry the filename
of the main document. This is inconvenient if
several child files are to be compiled and
to be kept for distribution.
\end{itemize}

The present package provides a simple interface
to make child files individually compilable by \LaTeX{}.
Compiling a child file then has the same effect as compiling
the main file with an |\includeonly| command
to select the appropriate child.
Moreover the generated document will carry the name of the child
rather than the main file.
This resolves all three above issues.

This feature is meant to make the editing of books,
thesis documents and lecture notes somewhat more convenient.
However, the package can also be used efficiently for
composing a series of documents (such as exercise sheets)
which are typically distributed individually.
It then assists the author in generating the individual documents
(potentially in different versions)
as well as a document containing the collected series.
Another application is in developing style files
or other kinds of included material
where compilation of the style file could redirect
to a sample or test file.

%%%%%%%%%%%%%%%%%%%%%%%%%%%%%%%%%%%%%%%%%%%%%%%%%%%%%%%%%%%%%%%%%%%%%%%%%%%%%%%%
%%%%%%%%%%%%%%%%%%%%%%%%%%%%%%%%%%%%%%%%%%%%%%%%%%%%%%%%%%%%%%%%%%%%%%%%%%%%%%%%
\section{Usage}

First of all, the package \textsf{childdoc} is \emph{not} a standard
\LaTeXe{} |.sty| style file! Therefore it needs to be invoked in
a non-standard way.

%%%%%%%%%%%%%%%%%%%%%%%%%%%%%%%%%%%%%%%%%%%%%%%%%%%%%%%%%%%%%%%%%%%%%%%%%%%%%%%%
\subsection{Included Files}
\label{sec:include}

%%%%%%%%%%%%%%%%%%%%%%%%%%%%%%%%%%%%%%%%
\DescribeMacro{\childdocmain}
To use the package, add the commands
\begin{center}
\begin{tabular}{l}
|\input{childdoc.def}|\\
|\childdocmain{}|\\
\end{tabular}
\end{center}
at the very top of the main \LaTeX{} file,
in particular \emph{before} the |\documentclass| statement!
The argument of |\childdocmain| should be left empty
(but it must be present).

%%%%%%%%%%%%%%%%%%%%%%%%%%%%%%%%%%%%%%%%
\DescribeMacro{\childdocof}
Furthermore, add the commands
\begin{center}
\begin{tabular}{l}
|\input{childdoc.def}|\\
|\childdocof{|\textit{main}|}|\\
\end{tabular}
\end{center}
at the top of every child file \textit{child}
which is included by |\include{|\textit{child}|}|
from within the main file
(or at least for those files to be compiled individually).
The argument \textit{main} must be the filename of the main file.

There are a couple of
considerations in setting up the main and child documents:

%%%%%%%%%%%%%%%%%%%%%%%%%%%%%%%%%%%%%%%%
\paragraph{Restrictions.}

Please note the following restrictions:
\begin{itemize}
\item
|\childdocmain| must be called with one argument \textit{main}
to ensure compatibility with earlier version of the package.
It must either be empty (|\childdocmain{}|)
or precisely match the filename of the main file in which it is specified.
See \secref{sec:detection} for further information.
\item
The filename \textit{main} must be specified without the |.tex| extension.
\item
The filename \textit{main} is case sensitive
(even in case-insensitive file systems)
due to internal string comparison.
\item
The argument \textit{main} should be fully expanded, it cannot be a macro.
\item
Subdirectories and special characters should be avoided in filenames.
\item
The command |\childdocmain{|\textit{main}|}| must be followed by a whitespace.
It should not be followed immediately by another command
or by a comment mark `|%|'.
This is because the \TeX{} parser reads the token immediately following
the argument of |\childdocmain| and puts it
at the beginning of every child section;
however, a white\-space is ignored.
\end{itemize}

%%%%%%%%%%%%%%%%%%%%%%%%%%%%%%%%%%%%%%%%
\paragraph{Content of Main File.}

It is advisable to place all content in the child files included by |\include|.
Any output contained in the main file will appear in all child documents
unless suppressed manually;
it cannot be suppressed automatically by the |\includeonly| directive
and thus should normally be avoided.
A method to include some content in the main file
by means of conditional processing is described in \secref{sec:conditional}.

%%%%%%%%%%%%%%%%%%%%%%%%%%%%%%%%%%%%%%%%
\paragraph{Page Numbering.}

When only a part of the document is compiled,
the appropriate numbering of pages
(as well as other status parameters)
is determined from the |.aux| files.
The latter contain information from previous passes.
However this information needs to propagate through
all intermediate child documents.
Therefore the page numbering in child documents may well
be inconsistent until the complete document is compiled at least once.

A useful (if unconventional) way to always ensure a consistent
page numbering is to restart the numbering in each child document
and denote the pages by `\textit{child}|.|\textit{page}'
where \textit{child} represents the chapter/section number of the child file.
This can be achieved by the command
|\numberwithin{page}{|\textit{child}|}|
of the \textsf{amsmath} package
where \textit{child} can be |chapter| or |section|
depending on the chosen structuring.
Alternatively, one can modify the macro |\thepage| appropriately
and reset the counter |page| at the start of each child file.

%%%%%%%%%%%%%%%%%%%%%%%%%%%%%%%%%%%%%%%%%%%%%%%%%%%%%%%%%%%%%%%%%%%%%%%%%%%%%%%%
\subsection{Conditional Processing}
\label{sec:conditional}

The package provides a mechanism to compile different versions
of a document. To customise the versions further some conditional processing
can come in handy to distinguish which version is being compiled.
The package provides two macros to describe the compilation context:

%%%%%%%%%%%%%%%%%%%%%%%%%%%%%%%%%%%%%%%%
\DescribeMacro{\ifchilddoc}
The conditional |\ifchilddoc| distinguishes between the compilation of
child documents and the main document:
%
\begin{center}
|\ifchilddoc |\textit{child-code}| |[|\||else |\textit{main-code}]| \||fi|
\end{center}

%%%%%%%%%%%%%%%%%%%%%%%%%%%%%%%%%%%%%%%%
\DescribeMacro{\childdocname}
\DescribeMacro{\childdocjob}
The macro |\childdocname| contains the filename (without extension)
of the main or child file being processed.
Note that |\childdocjob| will always contain the name of the main file.

%%%%%%%%%%%%%%%%%%%%%%%%%%%%%%%%%%%%%%%%
\paragraph{Title Page.}

Conditional processing can be used to include a title or banner page
in the main document when proper precautions are taken.
Importantly, the code in the main file should ensure that the page counter
(as well as other status parameters which are stored in the |.aux| files)
takes the same value after the conditional processing.
Otherwise the page numbers may take divergent values
depending on which part is compiled.

For example, a title page could be declared by:
%
\begin{center}
\begin{tabular}{l}
|\ifchilddoc\||else|\\
|\addtocounter{page}{-1}|\\
\textit{code for title page}\\
|\newpage|\\
|\||fi|
\end{tabular}
\end{center}
%
A banner page for the child documents can be generated by:
%
\begin{center}
\begin{tabular}{l}
|\ifchilddoc|\\
|\addtocounter{page}{-1}|\\
\textit{code for banner page}\\
|\newpage|\\
|\||fi|
\end{tabular}
\end{center}
%
Here one could write a message such as:
\begin{center}
|This is the part \childdocname{} of \childdocjob{}.|
\end{center}

%%%%%%%%%%%%%%%%%%%%%%%%%%%%%%%%%%%%%%%%%%%%%%%%%%%%%%%%%%%%%%%%%%%%%%%%%%%%%%%%
\subsection{Flags}
\label{sec:flags}

The package makes it easy to generate different versions
of the main or child documents.
To this end compilation flags can be defined
and assigned different default values.
They will be particularly useful in conjunction
with the forwarding mechanism described in \secref{sec:forward}.

For example, it may be useful to have a flag |\version|
which can be set to |draft| or |final|.
The document source will contain some conditional code
depending on the value of |\version|.
Suppose further, the flag should default to |final| for the main file
and to |draft| for child files
which is a natural assignment for editing the document.
This is achieved by placing the following code
in the preamble of the main document
(below the |\childdocmain| directive):
%
\begin{center}
\begin{tabular}{l}
|\ifchilddoc|\\
|\providecommand{\version}{draft}|\\
|\||else|\\
|\providecommand{\version}{final}|\\
|\||fi|
\end{tabular}
\end{center}
%
The definition by |\providecommand| makes sure
that previous definitions are not overwritten.
Further statements |\providecommand{\version}{...}|
can thus be added before the above code to override it.

For the main file, one might add a line
(between |\childdocmain| and the above block)
%
\begin{center}
|%\ifchilddoc\||else\providecommand{\version}{draft}\||fi|
\end{center}
%
which can be uncommented to produce a draft version.
Likewise one can add a line to the very top of a child file
(above the |\childdocof{|\textit{main}|}| directive)
%
\begin{center}
|%\providecommand{\version}{final}|
\end{center}
%
which can be uncommented to produce the final version of this child document.

%%%%%%%%%%%%%%%%%%%%%%%%%%%%%%%%%%%%%%%%%%%%%%%%%%%%%%%%%%%%%%%%%%%%%%%%%%%%%%%%
\subsection{Forwarding}
\label{sec:forward}

Different versions of the main or child documents
using compilation flags as described in \secref{sec:flags}
can be (permanently) stored in different files
for convenient compilation, viewing and distribution.
To this end, the package defines a command
to pass on compilation to a different file:

%%%%%%%%%%%%%%%%%%%%%%%%%%%%%%%%%%%%%%%%
\DescribeMacro{\childdocforward}
The command |\childdocforward| redirects processing to
another source file:
%
\begin{center}
\begin{tabular}{l}
|\input{childdoc.def}|\\
|\childdocforward[|\textit{main}|]{|\textit{dest}|}|\\
\end{tabular}
\end{center}
%
The argument \textit{dest} is the destination file
(without extension).
It should be the main file or one of the child files.
Note that further \textsf{childdoc} directives
such as |\childdocof| and |\childdocforward|
in the indicated file will be processed in this form.
The optional argument \textit{main}
passes on directly to the main file \textit{main}
while pretending to compile the child \textit{dest}.
This form behaves as if \textit{dest}
issues |\childdocof{|\textit{main}|}| right away,
and no further \textsf{childdoc} directives will be processed.

%%%%%%%%%%%%%%%%%%%%%%%%%%%%%%%%%%%%%%%%
\DescribeMacro{\...prefix}
In the alternative form |\childdocforwardprefix|,
%
\begin{center}
\begin{tabular}{l}
|\input{childdoc.def}|\\
|\childdocforwardprefix[|\textit{main}|]{|\textit{prefix}|}{|\textit{dest}|}|
\end{tabular}
\end{center}
%
the destination file is determined by a pattern
depending on the current file:
To make this work, the current file must be called
`{\textit{prefix}\hspace{0.2em}\textit{suffix}}'
with \textit{prefix} matching precisely the argument.
Processing is then passed on to the file
`{\textit{dest}\hspace{0.2em}\textit{suffix}}'.
Surely, the same effect is achieved by
directly specifying the
argument `{\textit{dest}\hspace{0.2em}\textit{suffix}}'
in the first form.
However, that requires to set up a different file
for each child. With the alternative form of the command
all these files can have exactly the same content
which simplifies setting them up and maintaining them.

For example, the following file |draft.tex|
with a compilation flag |\version| as described in \secref{sec:flags}
compiles the main document as a draft:
%
\begin{center}
\begin{tabular}{l}
|\def\version{draft}|\\
|\input{childdoc.def}|\\
|\childdocforward{|\textit{main}|}|
\end{tabular}
\end{center}
%
Likewise, the following files |final|\textit{nn}|.tex|
compile the final version of the child document
|child|\textit{nn}|.tex|:
%
\begin{center}
\begin{tabular}{l}
|\def\version{final}|\\
|\input{childdoc.def}|\\
|\childdocforwardprefix{final}{child}|
\end{tabular}
\end{center}
%

Note that when several versions of a main file and/or of each child file
are to be generated, it may be convenient to set up a |Makefile| or
shell script to automatise the process.

%%%%%%%%%%%%%%%%%%%%%%%%%%%%%%%%%%%%%%%%%%%%%%%%%%%%%%%%%%%%%%%%%%%%%%%%%%%%%%%%
\subsection{Command Line Processing}
\label{sec:commandline}

The effect of redirection files can also be achieved by invoking
the \LaTeX{} compiler with a more elaborate command line.
Most conveniently this should be done as part
of a shell script or a |Makefile|.

When using \textsf{childdoc} in the main file, the following
command lines effectively perform a redirection
(note that depending on the shell being used,
backslashes may have to be doubled: `|\|' $\to$ `|\\|'):
%
\begin{center}
|... -jobname "|\textit{target}|" |\\|"|[\textit{flags}]%
|\input{childdoc.def}\childdocforward[|\textit{main}|]{|\textit{dest}|}"|
\end{center}
%
Here \textit{target} is the name of the output file,
\textit{main} is the name of the main file
and \textit{dest} is the name of the main or child file to be processed
(all filenames without extensions).
The optional argument \textit{main} can be omitted
if \textit{main} matches \textit{dest}.
Optionally, compilation \textit{flags} can be defined via |\def| commands.
This command line makes the \TeX{} engine believe
it is compiling the file \textit{target}
whose content is specified as the latter parameter.
The provided code then forwards the processing to
\textit{main} or \textit{dest} as described in \secref{sec:forward}.

%%%%%%%%%%%%%%%%%%%%%%%%%%%%%%%%%%%%%%%%%%%%%%%%%%%%%%%%%%%%%%%%%%%%%%%%%%%%%%%%
\subsection{Include by Input}
\label{sec:input}

Including child documents by |\include| has some restrictions by design.
Most notably, the content of a child document always occupies
its own set of pages; pages cannot be shared between child documents.
Usually, this behaviour makes perfect sense
because each child document contain an essential part of the document.
However, in some situations it may be desirable to compose
a document from a collection of parts
without having mandatory page breaks between then.
For this case, the package
provides a mechanism to include parts
by |\input| which can also be processed individually.
However, by construction this mechanism
requires manual handling of the content to be output.

%%%%%%%%%%%%%%%%%%%%%%%%%%%%%%%%%%%%%%%%
\DescribeMacro{\ifchilddocmanual}
The main file should be prepared as usual, see \secref{sec:include}.
However, the document body must make a distinction
between processing of an individual part and of the main document, e.g.:
%
\begin{center}
\begin{tabular}{l}
|\ifchilddocmanual|\\
|\input{\childdocname}|\\
|\||else|\\
\textit{document body with }|\input{|\textit{part}|}|\\
|\||fi|
\end{tabular}
\end{center}
%
The conditional |\ifchilddocmanual| is true whenever
a part to be included by |\input| is being compiled,
and the name of the part is stored in |\childdocname|.

%%%%%%%%%%%%%%%%%%%%%%%%%%%%%%%%%%%%%%%%
\DescribeMacro{\childdocby}
Each part to be included by |\input| should start with:
%
\begin{center}
\begin{tabular}{l}
|\input{childdoc.def}|\\
|\childdocby{|\textit{main}|}|\\
\end{tabular}
\end{center}
%
The directive |\childdocby| is similar to |\childdocof|
described in \secref{sec:include},
but the subsequent selection of content must be done manually.
To that end, both |\ifchilddoc| and |\ifchilddocmanual|
will be true upon processing of a part,
and the name of the part is stored in |\childdocname|.
Note that |\jobname| will be set to the filename of the current part
so that each part receives an individual |.aux| file
that does not interfere with the |.aux| file(s) of the main document.
This behaviour can be altered by the alternative form
|\childdocby[*]{|\textit{main}|}| (with a non-empty optional argument)
which uses the |.aux| file of the main document
by setting |\jobname| to \textit{main}.

%%%%%%%%%%%%%%%%%%%%%%%%%%%%%%%%%%%%%%%%%%%%%%%%%%%%%%%%%%%%%%%%%%%%%%%%%%%%%%%%
\subsection{Driver Development}
\label{sec:driver}

The \textsf{childdoc} mechanism can also be use for the development
of definition files such as \LaTeX{} styles or classes.
This case differs from the above setup with multiple parts
included by |\include| in that no |\includeonly| should be invoked.
This can be achieved by starting the include file
(before |\ProvidesPackage|) with:
%
\begin{center}
\begin{tabular}{l}
|\input{childdoc.def}|\\
|\childdocforward{|\textit{main}|}|\\
\end{tabular}
\end{center}
%
or alternatively with:
%
\begin{center}
\begin{tabular}{l}
|\input{childdoc.def}|\\
|\childdocby{|\textit{main}|}|\\
\end{tabular}
\end{center}
%
Both forms have slightly different effects as described above.
The main file is prepared as usual, see \secref{sec:include}.

%%%%%%%%%%%%%%%%%%%%%%%%%%%%%%%%%%%%%%%%%%%%%%%%%%%%%%%%%%%%%%%%%%%%%%%%%%%%%%%%
\subsection{Legacy Detection}
\label{sec:detection}

The directive |\childdocmain| in the main file can detect
whether the complete document or merely a child is to be compiled
even without using the directive |\childdocof|.
This method is deprecated because it is less robust
and there is no compelling reason to use it;
it is merely provided for backward compatibility
and it may be removed in future versions.

If the detection mechanism is to be used,
it is mandatory to correctly specify
the filename of the main file as the argument of |\childdocmain|:
%
\begin{center}
\begin{tabular}{l}
|\input{childdoc.def}|\\
|\childdocmain{|\textit{main}|}|\\
\end{tabular}
\end{center}
%
If |\jobname| does not match the argument \textit{main} of |\childdocmain|,
it is assumed that |\jobname| points to the child file to be compiled.
When using |\childdocmain| with the main file specified as argument,
it suffices to start a child file
with just |\input{|\textit{main}|}|
without loading of the package and using |\childdocof|.
If instead all processing is done
with the appropriate \textsf{childdoc} directives,
the argument of \textit{main} of |\childdocmain| can be empty.

An alternative version of the command line processing described
in \secref{sec:commandline} using the detection mechanism reads:
%
\begin{center}
|... -jobname "|\textit{target}|" "|[\textit{flags}]%
[|\def\jobname{|\textit{dest}|}|]|\input{|\textit{main}|}"|
\end{center}

%%%%%%%%%%%%%%%%%%%%%%%%%%%%%%%%%%%%%%%%%%%%%%%%%%%%%%%%%%%%%%%%%%%%%%%%%%%%%%%%
\subsection{Manual Code}
\label{sec:manual}

In case one cannot be certain whether the definitions file |childdoc.def|
is installed on the target \TeX{} distribution
and one prefers not to ship it,
it is conceivable to paste a few relevant commands into the sources.

To that end, drop all statements |\input{childdoc.def}|
and perform the replacements as outlined below.
Instead of |\childdocmain{|\textit{main}|}| add the following code
to the top of the main file:
%
\begin{center}
\begin{tabular}{l}
|\||ifdefined\childdocname\endinput\||fi\newif\ifchilddoc|\\
|\edef\childdocname{\scantokens\expandafter{\jobname\noexpand}}|\\
|\def\childdocmain{|\textit{main}|}\||ifx\childdocmain\childdocname\||else|\\
|\childdoctrue\includeonly{\childdocname}\let\jobname\childdocmain\||fi|\\
\end{tabular}
\end{center}
%
Instead of |\childdocof{|\textit{main}|}| just include the main file
at the top of each child file:
%
\begin{center}
|\input{|\textit{main}|}|
\end{center}
%
A simple redirection |\childdocforward{|\textit{dest}|}| is achieved by:
%
\begin{center}
|\def\jobname{|\textit{dest}|}\input{\jobname}|
\end{center}
%
The redirection with prefix
|\childdocforwardprefix[|\textit{prefix}|]{|\textit{dest}|}|
is accomplished by:
%
\begin{center}
\begin{tabular}{l}
|{\edef\jobname{\scantokens\expandafter{\jobname\noexpand}}|\\
|\def\redirectjob |\textit{prefix}|#1~~~{\gdef\jobname{|\textit{dest}|#1}}|\\
|\expandafter\redirectjob\jobname~~~}\input{\jobname}|
\end{tabular}
\end{center}

In an alternative approach,
child documents can be compiled by a specific command line
without additional code or specific definitions:
%
\begin{center}
|... -jobname "|\textit{target}|" "|[\textit{flags}]%
|\includeonly{|\textit{dest}|}\input{|\textit{main}|}"|
\end{center}
%

%%%%%%%%%%%%%%%%%%%%%%%%%%%%%%%%%%%%%%%%%%%%%%%%%%%%%%%%%%%%%%%%%%%%%%%%%%%%%%%%
%%%%%%%%%%%%%%%%%%%%%%%%%%%%%%%%%%%%%%%%%%%%%%%%%%%%%%%%%%%%%%%%%%%%%%%%%%%%%%%%
\section{Information}

%%%%%%%%%%%%%%%%%%%%%%%%%%%%%%%%%%%%%%%%%%%%%%%%%%%%%%%%%%%%%%%%%%%%%%%%%%%%%%%%
\subsection{Copyright}

Copyright \copyright{} 2017--2018 Niklas Beisert

This work may be distributed and/or modified under the
conditions of the \LaTeX{} Project Public License, either version 1.3
of this license or (at your option) any later version.
The latest version of this license is in
  \url{http://www.latex-project.org/lppl.txt}
and version 1.3 or later is part of all distributions of \LaTeX{}
version 2005/12/01 or later.

This work has the LPPL maintenance status `maintained'.

The Current Maintainer of this work is Niklas Beisert.

This work consists of the files |README.txt|, |childdoc.ins| and |childdoc.dtx|
as well as the derived files |childdoc.def|, |cdocsamp.tex|
with |cdocsch1.tex|, |cdocsch2.tex|, |cdocspt3.tex|, |cdocspt4.tex|,
|cdocsdrf.tex|, |cdocsfn1.tex|, |cdocsfn2.tex|
as well as |childdoc.pdf|.

%%%%%%%%%%%%%%%%%%%%%%%%%%%%%%%%%%%%%%%%%%%%%%%%%%%%%%%%%%%%%%%%%%%%%%%%%%%%%%%%
\subsection{Files and Installation}

The package consists of the files:
%
\begin{center}
\begin{tabular}{ll}
    |README.txt|   & readme file \\
    |childdoc.ins| & installation file \\
    |childdoc.dtx| & source file \\
    |childdoc.def| & definition file \\
    |cdocsamp.tex| & sample main file \\
    |cdocsch1.tex| & sample include file \\
    |cdocsch2.tex| & sample include file \\
    |cdocspt3.tex| & sample part file \\
    |cdocspt4.tex| & sample part file \\
    |cdocsdrf.tex| & sample redirection file \\
    |cdocsfn1.tex| & sample redirection file \\
    |cdocsfn2.tex| & sample redirection file \\
    |childdoc.pdf| & manual
\end{tabular}
\end{center}
%
The distribution consists of the files
|README.txt|, |childdoc.ins| and |childdoc.dtx|.
%
\begin{itemize}
\item
Run (pdf)\LaTeX{} on |childdoc.dtx|
to compile the manual |childdoc.pdf| (this file).
\item
Run \LaTeX{} on |childdoc.ins| to create the definitions file |childdoc.def|
and the sample |cdocsamp.tex| with include files
|cdocsch1.tex|, |cdocsch2.tex|, |cdocspt3.tex|, |cdocspt4.tex|,
|cdocsdrf.tex|, |cdocsfn1.tex|, |cdocsfn2.tex|.
Then copy the file |childdoc.def| to an appropriate directory of your \LaTeX{}
distribution, e.g.\ \textit{texmf-root}|/tex/latex/childdoc|.
\end{itemize}

%%%%%%%%%%%%%%%%%%%%%%%%%%%%%%%%%%%%%%%%%%%%%%%%%%%%%%%%%%%%%%%%%%%%%%%%%%%%%%%%
\subsection{Related CTAN Packages}

There are several other packages which offer a similar functionality:
%
\begin{itemize}
\item
The packages
\href{http://ctan.org/pkg/docmute}{\textsf{docmute}},
\href{http://ctan.org/pkg/includex}{\textsf{includex}} and
\href{http://ctan.org/pkg/standalone}{\textsf{standalone}}
provide commands to include only the document body of
a child file thus allowing both files to be compiled individually.
\item
The packages \href{http://ctan.org/pkg/subdocs}{\textsf{subdocs}}
and \href{http://ctan.org/pkg/subfiles}{\textsf{subfiles}}
provide structures in which the main and child documents can be
encapsulated and allowing them to be compiled individually.
The inclusion mechanism is different from the conventional |\include|.
\item
The package \href{http://ctan.org/pkg/combine}{\textsf{combine}}
is an elaborate solution to combine several documents into one.
\end{itemize}
%
See also the CTAN topic \href{http://ctan.org/topic/subdocs}{\textsf{subdocs}}
for further related packages.
The present package differs from the above solutions in that
a document structure constructed with the conventional |\include| mechanism
just needs two extra commands at the top of every file
such that all constituent files can be compiled individually.

%%%%%%%%%%%%%%%%%%%%%%%%%%%%%%%%%%%%%%%%%%%%%%%%%%%%%%%%%%%%%%%%%%%%%%%%%%%%%%%%
%\subsection{Feature Suggestions}
%
%The following is a list of features which may be useful for future
%versions of this package:
%%
%\begin{itemize}
%\item
%\ldots
%\end{itemize}

%%%%%%%%%%%%%%%%%%%%%%%%%%%%%%%%%%%%%%%%%%%%%%%%%%%%%%%%%%%%%%%%%%%%%%%%%%%%%%%%
\subsection{Revision History}

%%%%%%%%%%%%%%%%%%%%%%%%%%%%%%%%%%%%%%%%
\paragraph{v2.0:} 2018/12/30

\begin{itemize}
\item
immediate forward processing
\item
added |\childdocby| mechanism
\item
manual restructured
\end{itemize}

%%%%%%%%%%%%%%%%%%%%%%%%%%%%%%%%%%%%%%%%
\paragraph{v1.6:} 2018/01/17

\begin{itemize}
\item
application for development of include files
\item
corrections to manual
\end{itemize}

%%%%%%%%%%%%%%%%%%%%%%%%%%%%%%%%%%%%%%%%
\paragraph{v1.5:} 2017/05/21

\begin{itemize}
\item
more complete structuring introduced
\item
|\childdocof| introduced
\item
|\childdoc| renamed to |\childdocmain|
\item
|\childredirect| renamed to |\childdocforward| and |\childdocforwardprefix|
and functionality expanded
\end{itemize}

%%%%%%%%%%%%%%%%%%%%%%%%%%%%%%%%%%%%%%%%
\paragraph{v1.0:} 2017/04/27

\begin{itemize}
\item
manual and install package
\item
first version published on CTAN
\end{itemize}

%%%%%%%%%%%%%%%%%%%%%%%%%%%%%%%%%%%%%%%%
\paragraph{v0.6:} 2017/04/26

\begin{itemize}
\item
redirection mechanism added
\end{itemize}

%%%%%%%%%%%%%%%%%%%%%%%%%%%%%%%%%%%%%%%%
\paragraph{v0.5:} 2017/04/26

\begin{itemize}
\item
functionality in definition file
\end{itemize}


%%%%%%%%%%%%%%%%%%%%%%%%%%%%%%%%%%%%%%%%%%%%%%%%%%%%%%%%%%%%%%%%%%%%%%%%%%%%%%%%
%%%%%%%%%%%%%%%%%%%%%%%%%%%%%%%%%%%%%%%%%%%%%%%%%%%%%%%%%%%%%%%%%%%%%%%%%%%%%%%%
%%%%%%%%%%%%%%%%%%%%%%%%%%%%%%%%%%%%%%%%%%%%%%%%%%%%%%%%%%%%%%%%%%%%%%%%%%%%%%%%
\appendix

\settowidth\MacroIndent{\rmfamily\scriptsize 000\ }

 \DocInput{childdoc.dtx}

\end{document}
%</driver>
% \fi
%
% %%%%%%%%%%%%%%%%%%%%%%%%%%%%%%%%%%%%%%%%%%%%%%%%%%%%%%%%%%%%%%%%%%%%%%%%%%%%%%
% %%%%%%%%%%%%%%%%%%%%%%%%%%%%%%%%%%%%%%%%%%%%%%%%%%%%%%%%%%%%%%%%%%%%%%%%%%%%%%
% \section{Sample}
%\iffalse
%<*samplemain>
%\fi
%
% The following presents a sample document
% with two chapters, two parts, a title page,
% a compile flag as well as three forwarding files to set the flag.
% It consists of eight |.tex| files:
% \begin{center}
% \begin{tabular}{ll}
% |cdocsamp.tex|&main file\\
% |cdocsch1.tex|&include file for chapter 1\\
% |cdocsch2.tex|&include file for chapter 2\\
% |cdocspt3.tex|&include file for part 3\\
% |cdocspt4.tex|&include file for part 4\\
% |cdocsdrf.tex|&forwarding file for main file in draft mode\\
% |cdocsfi1.tex|&forwarding file for final version of chapter 1\\
% |cdocsfi2.tex|&forwarding file for final version of chapter 2\\
% \end{tabular}
% \end{center}
% Each of the eight files can be compiled directly by the \LaTeX{} compiler.
%
% %%%%%%%%%%%%%%%%%%%%%%%%%%%%%%%%%%%%%%
% \paragraph{Main File.}
%
% The main file is called |cdocsamp.tex|.
%
% Load the \textsf{childdoc} definitions and
% declare the filename for the main document:
%    \begin{macrocode}
\input{childdoc.def}
\childdocmain{}
%    \end{macrocode}

% Optional override for |\version| flag:
%    \begin{macrocode}
%%\ifchilddoc\else\providecommand{\version}{draft}\fi
%    \end{macrocode}

% Define the default values for the |\version| flag
% (|final| for the main file and |draft| for childs):
%    \begin{macrocode}
\ifchilddoc
\providecommand{\version}{draft}
\else
\providecommand{\version}{final}
\fi
%    \end{macrocode}

% Load the standard document class:
%    \begin{macrocode}
\documentclass[12pt]{article}
%    \end{macrocode}

% Start the document body:
%    \begin{macrocode}
\begin{document}
%    \end{macrocode}

% Declare a title page.
% Print title, part of document being processed and version flag:
%    \begin{macrocode}
\addtocounter{page}{-1}
\begin{center}
{\LARGE\bfseries{}childdoc example\par}
\vspace{1cm}
\ifchilddoc
\ifchilddocmanual part\else chapter\fi:
`\childdocname' of `\childdocjob'\par
\else
main document: `\childdocjob'\par
\fi
version: \version\par
\end{center}
\newpage
%    \end{macrocode}

% Manually include selected file,
% otherwise process as usual:
%    \begin{macrocode}
\ifchilddocmanual
\section*{part `\childdocname'}
\input{\childdocname}
\else
%    \end{macrocode}

% Include the two chapters:
%    \begin{macrocode}
\include{cdocsch1}
\include{cdocsch2}
%    \end{macrocode}

% Include the two parts unless only chapters should be displayed:
%    \begin{macrocode}
\ifchilddoc\else
\section{part three}
\input{cdocspt3}
\section{part four}
\input{cdocspt4}
\fi
%    \end{macrocode}

% Process as usual until here:
%    \begin{macrocode}
\fi
%    \end{macrocode}

% End of document body:
%    \begin{macrocode}
\end{document}
%    \end{macrocode}
%\iffalse
%</samplemain>
%\fi
%
% %%%%%%%%%%%%%%%%%%%%%%%%%%%%%%%%%%%%%%
% \paragraph{Chapter Include Files.}
%
% The include files are called |cdocsch1.tex| and |cdocsch2.tex|.
%
%\iffalse
%<*samplechap1|samplechap2>
%\fi

% Optional override for |\version| flag:
%    \begin{macrocode}
%%\providecommand{\version}{final}
%    \end{macrocode}

% Include the main document:
%    \begin{macrocode}
\input{childdoc.def}
\childdocof{cdocsamp}
%    \end{macrocode}

%\iffalse
%</samplechap1|samplechap2>
%\fi
%
%\iffalse
%<*samplechap1>
%\fi
% Some text for chapter 1:
%    \begin{macrocode}
\section{one}
some text in chapter one
%    \end{macrocode}

%\iffalse
%</samplechap1>
%\fi
% Some text for chapter 2:
%\iffalse
%<*samplechap2>
%\fi
%    \begin{macrocode}
\section{two}
more text in chapter two
%    \end{macrocode}

%\iffalse
%</samplechap2>
%\fi
%
% %%%%%%%%%%%%%%%%%%%%%%%%%%%%%%%%%%%%%%
% \paragraph{Part Include Files.}
%
% The include files are called |cdocspt3.tex| and |cdocspt4.tex|.
%
%\iffalse
%<*samplepart3|samplepart4>
%\fi

% Optional override for |\version| flag:
%    \begin{macrocode}
%%\providecommand{\version}{final}
%    \end{macrocode}

% Include the main document:
%    \begin{macrocode}
\input{childdoc.def}
\childdocby{cdocsamp}
%    \end{macrocode}

%\iffalse
%</samplepart3|samplepart4>
%\fi
%
%\iffalse
%<*samplepart3>
%\fi
% Some text for part 3:
%    \begin{macrocode}
some text in part three
%    \end{macrocode}

%\iffalse
%</samplepart3>
%\fi
% Some text for part 4:
%\iffalse
%<*samplepart4>
%\fi
%    \begin{macrocode}
more text in part four
%    \end{macrocode}

%\iffalse
%</samplepart4>
%\fi
%
% %%%%%%%%%%%%%%%%%%%%%%%%%%%%%%%%%%%%%%
% \paragraph{Forwarding for a Complete Draft.}
%
% The following forwarding file |cdocsdrf.tex|
% compiles the main document in draft mode:
%\iffalse
%<*sampledraft>
%\fi
%    \begin{macrocode}
\def\version{draft}
\input{childdoc.def}
\childdocforward{cdocsamp}
%    \end{macrocode}

%\iffalse
%</sampledraft>
%\fi
%
% %%%%%%%%%%%%%%%%%%%%%%%%%%%%%%%%%%%%%%
% \paragraph{Forwarding for Final Version of the Chapters.}
%
% The following forwarding files |cdocsfn1.tex| and |cdocsfn2.tex|
% (with identical content)
% compile the final versions of the child documents
% |cdocsch1.tex| and |cdocsch2.tex|, respectively:
%\iffalse
%<*samplefinal>
%\fi
%    \begin{macrocode}
\def\version{final}
\input{childdoc.def}
\childdocforwardprefix[cdocsamp]{cdocsfn}{cdocsch}
%    \end{macrocode}

%\iffalse
%</samplefinal>
%\fi
%
% %%%%%%%%%%%%%%%%%%%%%%%%%%%%%%%%%%%%%%
% \paragraph{Command Line Processing.}
%
% The following three command lines generate the output files
% |cdocscld|, |cdocscl1| and |cdocscl2|
% which should be identical to
% |cdocsdrf|, |cdocsch1| and |cdocsfn2|, respectively:
% \begin{center}
% \begin{tabular}{l}
% |latex -jobname cdocscld \|\\
% |  "\def\version{draft}\input{childdoc.def}\childdocforward{cdocsamp}"|\\
% |latex -jobname cdocscl1 \|\\
% |  "\input{childdoc.def}\childdocforward[cdocsamp]{cdocsch1}"|\\
% |latex -jobname cdocscl2 \|\\
% |  "\def\version{final}\input{childdoc.def}\childdocforward{cdocsch2}"|
% \end{tabular}
% \end{center}
% Note that the trailing backslash on each first line
% merely continues the input to the second line
% (for convenient cut ant paste).
% Furthermore, the command |latex| can be replaced by any
% of its alternative versions such as |pdflatex|.
%
% %%%%%%%%%%%%%%%%%%%%%%%%%%%%%%%%%%%%%%%%%%%%%%%%%%%%%%%%%%%%%%%%%%%%%%%%%%%%%%
% %%%%%%%%%%%%%%%%%%%%%%%%%%%%%%%%%%%%%%%%%%%%%%%%%%%%%%%%%%%%%%%%%%%%%%%%%%%%%%
% \section{Implementation}
%\iffalse
%<*package>
%\fi
%
% This section describes the definitions file |childdoc.def|.

% The definitions cannot be loaded using |\usepackage| or |\RequirePackage|
% which has a mechanism to prevent loading a style file more than once.
% When loading the definitions by means of |\input|
% multiple instances have to be prevented manually:
%\iffalse
%This code needs to be before the `\ProvidesFile' directive
%which is defined at the beginning of this file.
%Therefore it is also placed there and commented out here.
%</package>
%<*discard>
%\fi
%    \begin{macrocode}
\ifdefined\childdocmain\endinput\fi
%    \end{macrocode}
%\iffalse
%</discard>
%<*package>
%\fi
%
% \macro{\ifchilddoc}
% \macro{\ifchilddocmanual}
% The conditional |\ifchilddoc| tells whether a
% child (true) or main (false) document is being compiled.
% The conditional |\ifchilddocmanual| tells whether
% the |\includeonly| mechanism is used (false) or
% the selection of child files must be performed manually (true).
% The definitions initialise to false:
%    \begin{macrocode}
\newif\ifchilddoc
\newif\ifchilddocmanual
%    \end{macrocode}

% \macro{\childdocname}
% \macro{\childdocjob}
% The macro |\childdocname| stores the name of the main document
% to be compiled. The macro |\childdocjob| stores the name of
% the document on which the \LaTeX{} compiler was originally invoked.
% The content of |\jobname| cannot be compared
% to filenames specified in the source due to different catcodes.
% The following code rescans |\jobname|, stores the result
% in |\childdocname| and saves a copy in |\childdocjob|:
%    \begin{macrocode}
\edef\childdocname{\scantokens\expandafter{\jobname\noexpand}}
\let\childdocjob\childdocname
%    \end{macrocode}

% \macro{\childdocdisable}
% The macro |\childdocdisable| prevents the main file
% from being processed more than once.
% At this stage, the main document command |\childdocmain|
% is assumed to be called once again where it should do nothing.
% Any subsequent call to it should prevent
% a secondary processing of the main document
% It overwrites the forwarding commands
% |\childdocof| and |\childdocforward|
% with empty macros to prevent further inclusions of the main document:
%    \begin{macrocode}
\newcommand{\childdocdisable}
{
  \renewcommand{\childdocmain}[1]{\renewcommand{\childdocmain}[1]{\endinput}}
  \renewcommand{\childdocof}[1]{}
  \renewcommand{\childdocby}[2][]{}
  \renewcommand{\childdocforward}[2][]{}
  \renewcommand{\childdocdisable}{}
}
%    \end{macrocode}

% \macro{\childdocmain}
% The macro |\childdocmain| is to be called at the top of the main file
% with nothing or the main filename (without extension) as argument.
% First, it breaks loops.
% If the argument is not empty and does not match |\childdocname|
% (which is set by the first inclusion of |childdoc.def|),
% |\ifchilddoc| is set to true, |\includeonly| is applied to the child file
% and |\jobname| is set to the main file
% (for proper handling of |.aux| files):
%    \begin{macrocode}
\newcommand{\childdocmain}[1]
{
  \childdocdisable\childdocmain{}
  \if?#1?\else
    \begingroup
      \def\childdoctmp{#1}
      \ifx\childdoctmp\childdocname
        \def\childdoctmp{}
      \else
        \def\childdoctmp
        {
          \childdoctrue
          \includeonly{\childdocname}
          \def\childdocjob{#1}
          \def\jobname{#1}
        }
      \fi
      \expandafter
    \endgroup
    \childdoctmp
  \fi
}
%    \end{macrocode}

% \macro{\childdocof}
% The command |\childdocof| redirects
% compilation to the main file |#1|.
%    \begin{macrocode}
\newcommand{\childdocof}[1]
{
  \childdocdisable
  \childdoctrue
  \includeonly{\childdocname}
  \def\jobname{#1}
  \def\childdocjob{#1}
  \input{#1}
}
%    \end{macrocode}

% \macro{\childdocby}
% The command |\childdocby| ....
%    \begin{macrocode}
\newcommand{\childdocby}[2][]
{
  \childdocdisable
  \childdoctrue
  \childdocmanualtrue
  \if?#1?\else
    \def\jobname{#2}
  \fi
  \def\childdocjob{#2}
  \input{#2}
  \endinput
}
%    \end{macrocode}

% \macro{\childdocforward}
% The command |\childdocforward| redirects
% compilation to the main file or
% (if the optional argument is given) a child file.
% Parameters are set as if the main file
% or a child file starting with |\childdocof| was compiled.
% Then compilation is handed over to the main file:
%    \begin{macrocode}
\newcommand{\childdocforward}[2][]
{
  \begingroup
    \if?#1?
      \def\childdoctmp
      {
        \def\childdocname{#2}
        \def\childdocjob{#2}
        \def\jobname{#2}
        \input{#2}
        \endinput
      }
    \else
      \def\childdoctmp
      {
        \childdocdisable
        \def\childdocname{#2}
        \childdoctrue
        \includeonly{#2}
        \def\childdocjob{#1}
        \def\jobname{#1}
        \input{#1}
        \endinput
      }
    \fi
    \expandafter
  \endgroup
  \childdoctmp
}
%    \end{macrocode}

% \macro{\childdocforwardprefix}
% The command |\childdocforwardprefix| redirects
% compilation to the main or a child file by means of a pattern.
% The prefix |#1| in the current filename is replaced by |#2|
% and the suffix of the current filename is kept
% (it is assumed that the filename does not contain the substring `|~~~|'
% which is used as a delimiter).
% Compilation is handed over to the new file by |\childdocforward|:
%    \begin{macrocode}
\newcommand{\childdocforwardprefix}[3][]
{
  \begingroup
    \def\childdocextract #2##1~~~{\def\childdoctmp{\childdocforward[#1]{#3##1}}}
    \expandafter\childdocextract\childdocname~~~
    \expandafter
  \endgroup
  \childdoctmp
}
%    \end{macrocode}

% \macro{\childdoc}
% The deprecated macro |\childdoc| is a legacy version of |\childdocmain|:
%    \begin{macrocode}
\newcommand{\childdoc}{\childdocmain}
%    \end{macrocode}

% \macro{\childdocredirect}
% The deprecated macro |\childdocredirect| is a legacy version
% of |\childdocforward| and |\childdocforwardprefix|:
%    \begin{macrocode}
\newcommand{\childdocredirect}[2][]
{
  \begingroup
    \if?#1?
      \def\childdoctmp{\childdocforward{#2}}
    \else
      \def\childdoctmp{\childdocforwardprefix{#1}{#2}}
    \fi
    \expandafter
  \endgroup
  \childdoctmp
}
%    \end{macrocode}

%\iffalse
%</package>
%\fi
%
\endinput

\childdocmain{}
%    \end{macrocode}

% Optional override for |\version| flag:
%    \begin{macrocode}
%%\ifchilddoc\else\providecommand{\version}{draft}\fi
%    \end{macrocode}

% Define the default values for the |\version| flag
% (|final| for the main file and |draft| for childs):
%    \begin{macrocode}
\ifchilddoc
\providecommand{\version}{draft}
\else
\providecommand{\version}{final}
\fi
%    \end{macrocode}

% Load the standard document class:
%    \begin{macrocode}
\documentclass[12pt]{article}
%    \end{macrocode}

% Start the document body:
%    \begin{macrocode}
\begin{document}
%    \end{macrocode}

% Declare a title page.
% Print title, part of document being processed and version flag:
%    \begin{macrocode}
\addtocounter{page}{-1}
\begin{center}
{\LARGE\bfseries{}childdoc example\par}
\vspace{1cm}
\ifchilddoc
\ifchilddocmanual part\else chapter\fi:
`\childdocname' of `\childdocjob'\par
\else
main document: `\childdocjob'\par
\fi
version: \version\par
\end{center}
\newpage
%    \end{macrocode}

% Manually include selected file,
% otherwise process as usual:
%    \begin{macrocode}
\ifchilddocmanual
\section*{part `\childdocname'}
\input{\childdocname}
\else
%    \end{macrocode}

% Include the two chapters:
%    \begin{macrocode}
\include{cdocsch1}
\include{cdocsch2}
%    \end{macrocode}

% Include the two parts unless only chapters should be displayed:
%    \begin{macrocode}
\ifchilddoc\else
\section{part three}
\input{cdocspt3}
\section{part four}
\input{cdocspt4}
\fi
%    \end{macrocode}

% Process as usual until here:
%    \begin{macrocode}
\fi
%    \end{macrocode}

% End of document body:
%    \begin{macrocode}
\end{document}
%    \end{macrocode}
%\iffalse
%</samplemain>
%\fi
%
% %%%%%%%%%%%%%%%%%%%%%%%%%%%%%%%%%%%%%%
% \paragraph{Chapter Include Files.}
%
% The include files are called |cdocsch1.tex| and |cdocsch2.tex|.
%
%\iffalse
%<*samplechap1|samplechap2>
%\fi

% Optional override for |\version| flag:
%    \begin{macrocode}
%%\providecommand{\version}{final}
%    \end{macrocode}

% Include the main document:
%    \begin{macrocode}
% \iffalse
%
% childdoc.dtx Copyright (C) 2017-2018 Niklas Beisert
%
% This work may be distributed and/or modified under the
% conditions of the LaTeX Project Public License, either version 1.3
% of this license or (at your option) any later version.
% The latest version of this license is in
%   http://www.latex-project.org/lppl.txt
% and version 1.3 or later is part of all distributions of LaTeX
% version 2005/12/01 or later.
%
% This work has the LPPL maintenance status `maintained'.
%
% The Current Maintainer of this work is Niklas Beisert.
%
% This work consists of the files childdoc.dtx and childdoc.ins
% and the derived files childdoc.def and cdocsamp.tex with
% cdocsch1.tex, cdocsch2.tex, cdocsdrf.tex, cdocsfn1.tex, cdocsfn2.tex.
%
%<package>\ifdefined\childdocmain\endinput\fi
%<package>\ProvidesFile{childdoc.def}[2018/12/30 v2.0 child document driver]
%<samplemain>\ProvidesFile{cdocsamp.tex}[2018/12/30 v2.0 sample for childdoc]
%<*driver>
%\ProvidesFile{childdoc.drv}[2018/12/30 v2.0 childdoc reference manual file]
\PassOptionsToClass{10pt,a4paper}{article}
\documentclass{ltxdoc}

\usepackage[margin=35mm]{geometry}
\usepackage{hyperref}
\usepackage{hyperxmp}
\usepackage[usenames]{color}

\hypersetup{colorlinks=true}
\hypersetup{pdfstartview=FitH}
\hypersetup{pdfpagemode=UseNone}
\hypersetup{pdfsource={}}
\hypersetup{pdflang={en-UK}}
\hypersetup{pdfcopyright={Copyright 2017-2018 Niklas Beisert.
  This work may be distributed and/or modified under the
  conditions of the LaTeX Project Public License, either version 1.3
  of this license or (at your option) any later version.}}
\hypersetup{pdflicenseurl={http://www.latex-project.org/lppl.txt}}
\hypersetup{pdfcontactaddress={ETH Zurich, ITP, HIT K,
  Wolfgang-Pauli-Strasse 27}}
\hypersetup{pdfcontactpostcode={8093}}
\hypersetup{pdfcontactcity={Zurich}}
\hypersetup{pdfcontactcountry={Switzerland}}
\hypersetup{pdfcontactemail={nbeisert@itp.phys.ethz.ch}}
\hypersetup{pdfcontacturl={http://people.phys.ethz.ch/\xmptilde nbeisert/}}

\newcommand{\secref}[1]{\hyperref[#1]{section \ref*{#1}}}

\parskip1ex
\parindent0pt
\let\olditemize\itemize
\def\itemize{\olditemize\parskip0pt}

\begin{document}

\title{The \textsf{childdoc} Package}
\hypersetup{pdftitle={The childdoc Package}}
\author{Niklas Beisert\\[2ex]
  Institut f\"ur Theoretische Physik\\
  Eidgen\"ossische Technische Hochschule Z\"urich\\
  Wolfgang-Pauli-Strasse 27, 8093 Z\"urich, Switzerland\\[1ex]
  \href{mailto:nbeisert@itp.phys.ethz.ch}
  {\texttt{nbeisert@itp.phys.ethz.ch}}}
\hypersetup{pdfauthor={Niklas Beisert}}
\hypersetup{pdfsubject={Manual for the LaTeX2e Package childdoc}}
\date{30 December 2018, \textsf{v2.0}}
\maketitle

\begin{abstract}\noindent
\textsf{childdoc} is a \LaTeXe{} package
that enables the direct compilation
of document sections included by |\include|
to individual files.
\end{abstract}

\begingroup
\parskip0ex
\tableofcontents
\endgroup

%%%%%%%%%%%%%%%%%%%%%%%%%%%%%%%%%%%%%%%%%%%%%%%%%%%%%%%%%%%%%%%%%%%%%%%%%%%%%%%%
%%%%%%%%%%%%%%%%%%%%%%%%%%%%%%%%%%%%%%%%%%%%%%%%%%%%%%%%%%%%%%%%%%%%%%%%%%%%%%%%
\section{Introduction}

\LaTeX{} provides a mechanism to structure a large document (such as a book)
into a main file and several child files (containing the chapters)
using the |\include| command.
This mechanism is beneficial for documents
which span hundreds of pages in order to
make the source file(s) more manageable.
Moreover, compilation can be restricted to
selected child files by means of the |\includeonly| command.
The latter feature can be used to reduce the compilation time while editing
(this was significantly more useful in the earlier days of \LaTeX{})
or to generate a smaller document which is easier to navigate.
Another application of |\includeonly| is to generate
documents consisting of selected parts of the complete document.

However, there are a few drawbacks of the plain |\include| mechanism:
\begin{itemize}
\item
The child files cannot be compiled on their own,
they can only be compiled via the main file.
A naive editing environment
(such as a text editor with an option
to have the current file processed by \LaTeX)
may require one to switch to the main file before compiling;
attempting to compile the child file produces errors.
\item
The main file must be modified (each time)
to adjust the |\includeonly| command
to the present needs. This easily leaves the main file in a messy state.
\item
The generated document will always carry the filename
of the main document. This is inconvenient if
several child files are to be compiled and
to be kept for distribution.
\end{itemize}

The present package provides a simple interface
to make child files individually compilable by \LaTeX{}.
Compiling a child file then has the same effect as compiling
the main file with an |\includeonly| command
to select the appropriate child.
Moreover the generated document will carry the name of the child
rather than the main file.
This resolves all three above issues.

This feature is meant to make the editing of books,
thesis documents and lecture notes somewhat more convenient.
However, the package can also be used efficiently for
composing a series of documents (such as exercise sheets)
which are typically distributed individually.
It then assists the author in generating the individual documents
(potentially in different versions)
as well as a document containing the collected series.
Another application is in developing style files
or other kinds of included material
where compilation of the style file could redirect
to a sample or test file.

%%%%%%%%%%%%%%%%%%%%%%%%%%%%%%%%%%%%%%%%%%%%%%%%%%%%%%%%%%%%%%%%%%%%%%%%%%%%%%%%
%%%%%%%%%%%%%%%%%%%%%%%%%%%%%%%%%%%%%%%%%%%%%%%%%%%%%%%%%%%%%%%%%%%%%%%%%%%%%%%%
\section{Usage}

First of all, the package \textsf{childdoc} is \emph{not} a standard
\LaTeXe{} |.sty| style file! Therefore it needs to be invoked in
a non-standard way.

%%%%%%%%%%%%%%%%%%%%%%%%%%%%%%%%%%%%%%%%%%%%%%%%%%%%%%%%%%%%%%%%%%%%%%%%%%%%%%%%
\subsection{Included Files}
\label{sec:include}

%%%%%%%%%%%%%%%%%%%%%%%%%%%%%%%%%%%%%%%%
\DescribeMacro{\childdocmain}
To use the package, add the commands
\begin{center}
\begin{tabular}{l}
|\input{childdoc.def}|\\
|\childdocmain{}|\\
\end{tabular}
\end{center}
at the very top of the main \LaTeX{} file,
in particular \emph{before} the |\documentclass| statement!
The argument of |\childdocmain| should be left empty
(but it must be present).

%%%%%%%%%%%%%%%%%%%%%%%%%%%%%%%%%%%%%%%%
\DescribeMacro{\childdocof}
Furthermore, add the commands
\begin{center}
\begin{tabular}{l}
|\input{childdoc.def}|\\
|\childdocof{|\textit{main}|}|\\
\end{tabular}
\end{center}
at the top of every child file \textit{child}
which is included by |\include{|\textit{child}|}|
from within the main file
(or at least for those files to be compiled individually).
The argument \textit{main} must be the filename of the main file.

There are a couple of
considerations in setting up the main and child documents:

%%%%%%%%%%%%%%%%%%%%%%%%%%%%%%%%%%%%%%%%
\paragraph{Restrictions.}

Please note the following restrictions:
\begin{itemize}
\item
|\childdocmain| must be called with one argument \textit{main}
to ensure compatibility with earlier version of the package.
It must either be empty (|\childdocmain{}|)
or precisely match the filename of the main file in which it is specified.
See \secref{sec:detection} for further information.
\item
The filename \textit{main} must be specified without the |.tex| extension.
\item
The filename \textit{main} is case sensitive
(even in case-insensitive file systems)
due to internal string comparison.
\item
The argument \textit{main} should be fully expanded, it cannot be a macro.
\item
Subdirectories and special characters should be avoided in filenames.
\item
The command |\childdocmain{|\textit{main}|}| must be followed by a whitespace.
It should not be followed immediately by another command
or by a comment mark `|%|'.
This is because the \TeX{} parser reads the token immediately following
the argument of |\childdocmain| and puts it
at the beginning of every child section;
however, a white\-space is ignored.
\end{itemize}

%%%%%%%%%%%%%%%%%%%%%%%%%%%%%%%%%%%%%%%%
\paragraph{Content of Main File.}

It is advisable to place all content in the child files included by |\include|.
Any output contained in the main file will appear in all child documents
unless suppressed manually;
it cannot be suppressed automatically by the |\includeonly| directive
and thus should normally be avoided.
A method to include some content in the main file
by means of conditional processing is described in \secref{sec:conditional}.

%%%%%%%%%%%%%%%%%%%%%%%%%%%%%%%%%%%%%%%%
\paragraph{Page Numbering.}

When only a part of the document is compiled,
the appropriate numbering of pages
(as well as other status parameters)
is determined from the |.aux| files.
The latter contain information from previous passes.
However this information needs to propagate through
all intermediate child documents.
Therefore the page numbering in child documents may well
be inconsistent until the complete document is compiled at least once.

A useful (if unconventional) way to always ensure a consistent
page numbering is to restart the numbering in each child document
and denote the pages by `\textit{child}|.|\textit{page}'
where \textit{child} represents the chapter/section number of the child file.
This can be achieved by the command
|\numberwithin{page}{|\textit{child}|}|
of the \textsf{amsmath} package
where \textit{child} can be |chapter| or |section|
depending on the chosen structuring.
Alternatively, one can modify the macro |\thepage| appropriately
and reset the counter |page| at the start of each child file.

%%%%%%%%%%%%%%%%%%%%%%%%%%%%%%%%%%%%%%%%%%%%%%%%%%%%%%%%%%%%%%%%%%%%%%%%%%%%%%%%
\subsection{Conditional Processing}
\label{sec:conditional}

The package provides a mechanism to compile different versions
of a document. To customise the versions further some conditional processing
can come in handy to distinguish which version is being compiled.
The package provides two macros to describe the compilation context:

%%%%%%%%%%%%%%%%%%%%%%%%%%%%%%%%%%%%%%%%
\DescribeMacro{\ifchilddoc}
The conditional |\ifchilddoc| distinguishes between the compilation of
child documents and the main document:
%
\begin{center}
|\ifchilddoc |\textit{child-code}| |[|\||else |\textit{main-code}]| \||fi|
\end{center}

%%%%%%%%%%%%%%%%%%%%%%%%%%%%%%%%%%%%%%%%
\DescribeMacro{\childdocname}
\DescribeMacro{\childdocjob}
The macro |\childdocname| contains the filename (without extension)
of the main or child file being processed.
Note that |\childdocjob| will always contain the name of the main file.

%%%%%%%%%%%%%%%%%%%%%%%%%%%%%%%%%%%%%%%%
\paragraph{Title Page.}

Conditional processing can be used to include a title or banner page
in the main document when proper precautions are taken.
Importantly, the code in the main file should ensure that the page counter
(as well as other status parameters which are stored in the |.aux| files)
takes the same value after the conditional processing.
Otherwise the page numbers may take divergent values
depending on which part is compiled.

For example, a title page could be declared by:
%
\begin{center}
\begin{tabular}{l}
|\ifchilddoc\||else|\\
|\addtocounter{page}{-1}|\\
\textit{code for title page}\\
|\newpage|\\
|\||fi|
\end{tabular}
\end{center}
%
A banner page for the child documents can be generated by:
%
\begin{center}
\begin{tabular}{l}
|\ifchilddoc|\\
|\addtocounter{page}{-1}|\\
\textit{code for banner page}\\
|\newpage|\\
|\||fi|
\end{tabular}
\end{center}
%
Here one could write a message such as:
\begin{center}
|This is the part \childdocname{} of \childdocjob{}.|
\end{center}

%%%%%%%%%%%%%%%%%%%%%%%%%%%%%%%%%%%%%%%%%%%%%%%%%%%%%%%%%%%%%%%%%%%%%%%%%%%%%%%%
\subsection{Flags}
\label{sec:flags}

The package makes it easy to generate different versions
of the main or child documents.
To this end compilation flags can be defined
and assigned different default values.
They will be particularly useful in conjunction
with the forwarding mechanism described in \secref{sec:forward}.

For example, it may be useful to have a flag |\version|
which can be set to |draft| or |final|.
The document source will contain some conditional code
depending on the value of |\version|.
Suppose further, the flag should default to |final| for the main file
and to |draft| for child files
which is a natural assignment for editing the document.
This is achieved by placing the following code
in the preamble of the main document
(below the |\childdocmain| directive):
%
\begin{center}
\begin{tabular}{l}
|\ifchilddoc|\\
|\providecommand{\version}{draft}|\\
|\||else|\\
|\providecommand{\version}{final}|\\
|\||fi|
\end{tabular}
\end{center}
%
The definition by |\providecommand| makes sure
that previous definitions are not overwritten.
Further statements |\providecommand{\version}{...}|
can thus be added before the above code to override it.

For the main file, one might add a line
(between |\childdocmain| and the above block)
%
\begin{center}
|%\ifchilddoc\||else\providecommand{\version}{draft}\||fi|
\end{center}
%
which can be uncommented to produce a draft version.
Likewise one can add a line to the very top of a child file
(above the |\childdocof{|\textit{main}|}| directive)
%
\begin{center}
|%\providecommand{\version}{final}|
\end{center}
%
which can be uncommented to produce the final version of this child document.

%%%%%%%%%%%%%%%%%%%%%%%%%%%%%%%%%%%%%%%%%%%%%%%%%%%%%%%%%%%%%%%%%%%%%%%%%%%%%%%%
\subsection{Forwarding}
\label{sec:forward}

Different versions of the main or child documents
using compilation flags as described in \secref{sec:flags}
can be (permanently) stored in different files
for convenient compilation, viewing and distribution.
To this end, the package defines a command
to pass on compilation to a different file:

%%%%%%%%%%%%%%%%%%%%%%%%%%%%%%%%%%%%%%%%
\DescribeMacro{\childdocforward}
The command |\childdocforward| redirects processing to
another source file:
%
\begin{center}
\begin{tabular}{l}
|\input{childdoc.def}|\\
|\childdocforward[|\textit{main}|]{|\textit{dest}|}|\\
\end{tabular}
\end{center}
%
The argument \textit{dest} is the destination file
(without extension).
It should be the main file or one of the child files.
Note that further \textsf{childdoc} directives
such as |\childdocof| and |\childdocforward|
in the indicated file will be processed in this form.
The optional argument \textit{main}
passes on directly to the main file \textit{main}
while pretending to compile the child \textit{dest}.
This form behaves as if \textit{dest}
issues |\childdocof{|\textit{main}|}| right away,
and no further \textsf{childdoc} directives will be processed.

%%%%%%%%%%%%%%%%%%%%%%%%%%%%%%%%%%%%%%%%
\DescribeMacro{\...prefix}
In the alternative form |\childdocforwardprefix|,
%
\begin{center}
\begin{tabular}{l}
|\input{childdoc.def}|\\
|\childdocforwardprefix[|\textit{main}|]{|\textit{prefix}|}{|\textit{dest}|}|
\end{tabular}
\end{center}
%
the destination file is determined by a pattern
depending on the current file:
To make this work, the current file must be called
`{\textit{prefix}\hspace{0.2em}\textit{suffix}}'
with \textit{prefix} matching precisely the argument.
Processing is then passed on to the file
`{\textit{dest}\hspace{0.2em}\textit{suffix}}'.
Surely, the same effect is achieved by
directly specifying the
argument `{\textit{dest}\hspace{0.2em}\textit{suffix}}'
in the first form.
However, that requires to set up a different file
for each child. With the alternative form of the command
all these files can have exactly the same content
which simplifies setting them up and maintaining them.

For example, the following file |draft.tex|
with a compilation flag |\version| as described in \secref{sec:flags}
compiles the main document as a draft:
%
\begin{center}
\begin{tabular}{l}
|\def\version{draft}|\\
|\input{childdoc.def}|\\
|\childdocforward{|\textit{main}|}|
\end{tabular}
\end{center}
%
Likewise, the following files |final|\textit{nn}|.tex|
compile the final version of the child document
|child|\textit{nn}|.tex|:
%
\begin{center}
\begin{tabular}{l}
|\def\version{final}|\\
|\input{childdoc.def}|\\
|\childdocforwardprefix{final}{child}|
\end{tabular}
\end{center}
%

Note that when several versions of a main file and/or of each child file
are to be generated, it may be convenient to set up a |Makefile| or
shell script to automatise the process.

%%%%%%%%%%%%%%%%%%%%%%%%%%%%%%%%%%%%%%%%%%%%%%%%%%%%%%%%%%%%%%%%%%%%%%%%%%%%%%%%
\subsection{Command Line Processing}
\label{sec:commandline}

The effect of redirection files can also be achieved by invoking
the \LaTeX{} compiler with a more elaborate command line.
Most conveniently this should be done as part
of a shell script or a |Makefile|.

When using \textsf{childdoc} in the main file, the following
command lines effectively perform a redirection
(note that depending on the shell being used,
backslashes may have to be doubled: `|\|' $\to$ `|\\|'):
%
\begin{center}
|... -jobname "|\textit{target}|" |\\|"|[\textit{flags}]%
|\input{childdoc.def}\childdocforward[|\textit{main}|]{|\textit{dest}|}"|
\end{center}
%
Here \textit{target} is the name of the output file,
\textit{main} is the name of the main file
and \textit{dest} is the name of the main or child file to be processed
(all filenames without extensions).
The optional argument \textit{main} can be omitted
if \textit{main} matches \textit{dest}.
Optionally, compilation \textit{flags} can be defined via |\def| commands.
This command line makes the \TeX{} engine believe
it is compiling the file \textit{target}
whose content is specified as the latter parameter.
The provided code then forwards the processing to
\textit{main} or \textit{dest} as described in \secref{sec:forward}.

%%%%%%%%%%%%%%%%%%%%%%%%%%%%%%%%%%%%%%%%%%%%%%%%%%%%%%%%%%%%%%%%%%%%%%%%%%%%%%%%
\subsection{Include by Input}
\label{sec:input}

Including child documents by |\include| has some restrictions by design.
Most notably, the content of a child document always occupies
its own set of pages; pages cannot be shared between child documents.
Usually, this behaviour makes perfect sense
because each child document contain an essential part of the document.
However, in some situations it may be desirable to compose
a document from a collection of parts
without having mandatory page breaks between then.
For this case, the package
provides a mechanism to include parts
by |\input| which can also be processed individually.
However, by construction this mechanism
requires manual handling of the content to be output.

%%%%%%%%%%%%%%%%%%%%%%%%%%%%%%%%%%%%%%%%
\DescribeMacro{\ifchilddocmanual}
The main file should be prepared as usual, see \secref{sec:include}.
However, the document body must make a distinction
between processing of an individual part and of the main document, e.g.:
%
\begin{center}
\begin{tabular}{l}
|\ifchilddocmanual|\\
|\input{\childdocname}|\\
|\||else|\\
\textit{document body with }|\input{|\textit{part}|}|\\
|\||fi|
\end{tabular}
\end{center}
%
The conditional |\ifchilddocmanual| is true whenever
a part to be included by |\input| is being compiled,
and the name of the part is stored in |\childdocname|.

%%%%%%%%%%%%%%%%%%%%%%%%%%%%%%%%%%%%%%%%
\DescribeMacro{\childdocby}
Each part to be included by |\input| should start with:
%
\begin{center}
\begin{tabular}{l}
|\input{childdoc.def}|\\
|\childdocby{|\textit{main}|}|\\
\end{tabular}
\end{center}
%
The directive |\childdocby| is similar to |\childdocof|
described in \secref{sec:include},
but the subsequent selection of content must be done manually.
To that end, both |\ifchilddoc| and |\ifchilddocmanual|
will be true upon processing of a part,
and the name of the part is stored in |\childdocname|.
Note that |\jobname| will be set to the filename of the current part
so that each part receives an individual |.aux| file
that does not interfere with the |.aux| file(s) of the main document.
This behaviour can be altered by the alternative form
|\childdocby[*]{|\textit{main}|}| (with a non-empty optional argument)
which uses the |.aux| file of the main document
by setting |\jobname| to \textit{main}.

%%%%%%%%%%%%%%%%%%%%%%%%%%%%%%%%%%%%%%%%%%%%%%%%%%%%%%%%%%%%%%%%%%%%%%%%%%%%%%%%
\subsection{Driver Development}
\label{sec:driver}

The \textsf{childdoc} mechanism can also be use for the development
of definition files such as \LaTeX{} styles or classes.
This case differs from the above setup with multiple parts
included by |\include| in that no |\includeonly| should be invoked.
This can be achieved by starting the include file
(before |\ProvidesPackage|) with:
%
\begin{center}
\begin{tabular}{l}
|\input{childdoc.def}|\\
|\childdocforward{|\textit{main}|}|\\
\end{tabular}
\end{center}
%
or alternatively with:
%
\begin{center}
\begin{tabular}{l}
|\input{childdoc.def}|\\
|\childdocby{|\textit{main}|}|\\
\end{tabular}
\end{center}
%
Both forms have slightly different effects as described above.
The main file is prepared as usual, see \secref{sec:include}.

%%%%%%%%%%%%%%%%%%%%%%%%%%%%%%%%%%%%%%%%%%%%%%%%%%%%%%%%%%%%%%%%%%%%%%%%%%%%%%%%
\subsection{Legacy Detection}
\label{sec:detection}

The directive |\childdocmain| in the main file can detect
whether the complete document or merely a child is to be compiled
even without using the directive |\childdocof|.
This method is deprecated because it is less robust
and there is no compelling reason to use it;
it is merely provided for backward compatibility
and it may be removed in future versions.

If the detection mechanism is to be used,
it is mandatory to correctly specify
the filename of the main file as the argument of |\childdocmain|:
%
\begin{center}
\begin{tabular}{l}
|\input{childdoc.def}|\\
|\childdocmain{|\textit{main}|}|\\
\end{tabular}
\end{center}
%
If |\jobname| does not match the argument \textit{main} of |\childdocmain|,
it is assumed that |\jobname| points to the child file to be compiled.
When using |\childdocmain| with the main file specified as argument,
it suffices to start a child file
with just |\input{|\textit{main}|}|
without loading of the package and using |\childdocof|.
If instead all processing is done
with the appropriate \textsf{childdoc} directives,
the argument of \textit{main} of |\childdocmain| can be empty.

An alternative version of the command line processing described
in \secref{sec:commandline} using the detection mechanism reads:
%
\begin{center}
|... -jobname "|\textit{target}|" "|[\textit{flags}]%
[|\def\jobname{|\textit{dest}|}|]|\input{|\textit{main}|}"|
\end{center}

%%%%%%%%%%%%%%%%%%%%%%%%%%%%%%%%%%%%%%%%%%%%%%%%%%%%%%%%%%%%%%%%%%%%%%%%%%%%%%%%
\subsection{Manual Code}
\label{sec:manual}

In case one cannot be certain whether the definitions file |childdoc.def|
is installed on the target \TeX{} distribution
and one prefers not to ship it,
it is conceivable to paste a few relevant commands into the sources.

To that end, drop all statements |\input{childdoc.def}|
and perform the replacements as outlined below.
Instead of |\childdocmain{|\textit{main}|}| add the following code
to the top of the main file:
%
\begin{center}
\begin{tabular}{l}
|\||ifdefined\childdocname\endinput\||fi\newif\ifchilddoc|\\
|\edef\childdocname{\scantokens\expandafter{\jobname\noexpand}}|\\
|\def\childdocmain{|\textit{main}|}\||ifx\childdocmain\childdocname\||else|\\
|\childdoctrue\includeonly{\childdocname}\let\jobname\childdocmain\||fi|\\
\end{tabular}
\end{center}
%
Instead of |\childdocof{|\textit{main}|}| just include the main file
at the top of each child file:
%
\begin{center}
|\input{|\textit{main}|}|
\end{center}
%
A simple redirection |\childdocforward{|\textit{dest}|}| is achieved by:
%
\begin{center}
|\def\jobname{|\textit{dest}|}\input{\jobname}|
\end{center}
%
The redirection with prefix
|\childdocforwardprefix[|\textit{prefix}|]{|\textit{dest}|}|
is accomplished by:
%
\begin{center}
\begin{tabular}{l}
|{\edef\jobname{\scantokens\expandafter{\jobname\noexpand}}|\\
|\def\redirectjob |\textit{prefix}|#1~~~{\gdef\jobname{|\textit{dest}|#1}}|\\
|\expandafter\redirectjob\jobname~~~}\input{\jobname}|
\end{tabular}
\end{center}

In an alternative approach,
child documents can be compiled by a specific command line
without additional code or specific definitions:
%
\begin{center}
|... -jobname "|\textit{target}|" "|[\textit{flags}]%
|\includeonly{|\textit{dest}|}\input{|\textit{main}|}"|
\end{center}
%

%%%%%%%%%%%%%%%%%%%%%%%%%%%%%%%%%%%%%%%%%%%%%%%%%%%%%%%%%%%%%%%%%%%%%%%%%%%%%%%%
%%%%%%%%%%%%%%%%%%%%%%%%%%%%%%%%%%%%%%%%%%%%%%%%%%%%%%%%%%%%%%%%%%%%%%%%%%%%%%%%
\section{Information}

%%%%%%%%%%%%%%%%%%%%%%%%%%%%%%%%%%%%%%%%%%%%%%%%%%%%%%%%%%%%%%%%%%%%%%%%%%%%%%%%
\subsection{Copyright}

Copyright \copyright{} 2017--2018 Niklas Beisert

This work may be distributed and/or modified under the
conditions of the \LaTeX{} Project Public License, either version 1.3
of this license or (at your option) any later version.
The latest version of this license is in
  \url{http://www.latex-project.org/lppl.txt}
and version 1.3 or later is part of all distributions of \LaTeX{}
version 2005/12/01 or later.

This work has the LPPL maintenance status `maintained'.

The Current Maintainer of this work is Niklas Beisert.

This work consists of the files |README.txt|, |childdoc.ins| and |childdoc.dtx|
as well as the derived files |childdoc.def|, |cdocsamp.tex|
with |cdocsch1.tex|, |cdocsch2.tex|, |cdocspt3.tex|, |cdocspt4.tex|,
|cdocsdrf.tex|, |cdocsfn1.tex|, |cdocsfn2.tex|
as well as |childdoc.pdf|.

%%%%%%%%%%%%%%%%%%%%%%%%%%%%%%%%%%%%%%%%%%%%%%%%%%%%%%%%%%%%%%%%%%%%%%%%%%%%%%%%
\subsection{Files and Installation}

The package consists of the files:
%
\begin{center}
\begin{tabular}{ll}
    |README.txt|   & readme file \\
    |childdoc.ins| & installation file \\
    |childdoc.dtx| & source file \\
    |childdoc.def| & definition file \\
    |cdocsamp.tex| & sample main file \\
    |cdocsch1.tex| & sample include file \\
    |cdocsch2.tex| & sample include file \\
    |cdocspt3.tex| & sample part file \\
    |cdocspt4.tex| & sample part file \\
    |cdocsdrf.tex| & sample redirection file \\
    |cdocsfn1.tex| & sample redirection file \\
    |cdocsfn2.tex| & sample redirection file \\
    |childdoc.pdf| & manual
\end{tabular}
\end{center}
%
The distribution consists of the files
|README.txt|, |childdoc.ins| and |childdoc.dtx|.
%
\begin{itemize}
\item
Run (pdf)\LaTeX{} on |childdoc.dtx|
to compile the manual |childdoc.pdf| (this file).
\item
Run \LaTeX{} on |childdoc.ins| to create the definitions file |childdoc.def|
and the sample |cdocsamp.tex| with include files
|cdocsch1.tex|, |cdocsch2.tex|, |cdocspt3.tex|, |cdocspt4.tex|,
|cdocsdrf.tex|, |cdocsfn1.tex|, |cdocsfn2.tex|.
Then copy the file |childdoc.def| to an appropriate directory of your \LaTeX{}
distribution, e.g.\ \textit{texmf-root}|/tex/latex/childdoc|.
\end{itemize}

%%%%%%%%%%%%%%%%%%%%%%%%%%%%%%%%%%%%%%%%%%%%%%%%%%%%%%%%%%%%%%%%%%%%%%%%%%%%%%%%
\subsection{Related CTAN Packages}

There are several other packages which offer a similar functionality:
%
\begin{itemize}
\item
The packages
\href{http://ctan.org/pkg/docmute}{\textsf{docmute}},
\href{http://ctan.org/pkg/includex}{\textsf{includex}} and
\href{http://ctan.org/pkg/standalone}{\textsf{standalone}}
provide commands to include only the document body of
a child file thus allowing both files to be compiled individually.
\item
The packages \href{http://ctan.org/pkg/subdocs}{\textsf{subdocs}}
and \href{http://ctan.org/pkg/subfiles}{\textsf{subfiles}}
provide structures in which the main and child documents can be
encapsulated and allowing them to be compiled individually.
The inclusion mechanism is different from the conventional |\include|.
\item
The package \href{http://ctan.org/pkg/combine}{\textsf{combine}}
is an elaborate solution to combine several documents into one.
\end{itemize}
%
See also the CTAN topic \href{http://ctan.org/topic/subdocs}{\textsf{subdocs}}
for further related packages.
The present package differs from the above solutions in that
a document structure constructed with the conventional |\include| mechanism
just needs two extra commands at the top of every file
such that all constituent files can be compiled individually.

%%%%%%%%%%%%%%%%%%%%%%%%%%%%%%%%%%%%%%%%%%%%%%%%%%%%%%%%%%%%%%%%%%%%%%%%%%%%%%%%
%\subsection{Feature Suggestions}
%
%The following is a list of features which may be useful for future
%versions of this package:
%%
%\begin{itemize}
%\item
%\ldots
%\end{itemize}

%%%%%%%%%%%%%%%%%%%%%%%%%%%%%%%%%%%%%%%%%%%%%%%%%%%%%%%%%%%%%%%%%%%%%%%%%%%%%%%%
\subsection{Revision History}

%%%%%%%%%%%%%%%%%%%%%%%%%%%%%%%%%%%%%%%%
\paragraph{v2.0:} 2018/12/30

\begin{itemize}
\item
immediate forward processing
\item
added |\childdocby| mechanism
\item
manual restructured
\end{itemize}

%%%%%%%%%%%%%%%%%%%%%%%%%%%%%%%%%%%%%%%%
\paragraph{v1.6:} 2018/01/17

\begin{itemize}
\item
application for development of include files
\item
corrections to manual
\end{itemize}

%%%%%%%%%%%%%%%%%%%%%%%%%%%%%%%%%%%%%%%%
\paragraph{v1.5:} 2017/05/21

\begin{itemize}
\item
more complete structuring introduced
\item
|\childdocof| introduced
\item
|\childdoc| renamed to |\childdocmain|
\item
|\childredirect| renamed to |\childdocforward| and |\childdocforwardprefix|
and functionality expanded
\end{itemize}

%%%%%%%%%%%%%%%%%%%%%%%%%%%%%%%%%%%%%%%%
\paragraph{v1.0:} 2017/04/27

\begin{itemize}
\item
manual and install package
\item
first version published on CTAN
\end{itemize}

%%%%%%%%%%%%%%%%%%%%%%%%%%%%%%%%%%%%%%%%
\paragraph{v0.6:} 2017/04/26

\begin{itemize}
\item
redirection mechanism added
\end{itemize}

%%%%%%%%%%%%%%%%%%%%%%%%%%%%%%%%%%%%%%%%
\paragraph{v0.5:} 2017/04/26

\begin{itemize}
\item
functionality in definition file
\end{itemize}


%%%%%%%%%%%%%%%%%%%%%%%%%%%%%%%%%%%%%%%%%%%%%%%%%%%%%%%%%%%%%%%%%%%%%%%%%%%%%%%%
%%%%%%%%%%%%%%%%%%%%%%%%%%%%%%%%%%%%%%%%%%%%%%%%%%%%%%%%%%%%%%%%%%%%%%%%%%%%%%%%
%%%%%%%%%%%%%%%%%%%%%%%%%%%%%%%%%%%%%%%%%%%%%%%%%%%%%%%%%%%%%%%%%%%%%%%%%%%%%%%%
\appendix

\settowidth\MacroIndent{\rmfamily\scriptsize 000\ }

 \DocInput{childdoc.dtx}

\end{document}
%</driver>
% \fi
%
% %%%%%%%%%%%%%%%%%%%%%%%%%%%%%%%%%%%%%%%%%%%%%%%%%%%%%%%%%%%%%%%%%%%%%%%%%%%%%%
% %%%%%%%%%%%%%%%%%%%%%%%%%%%%%%%%%%%%%%%%%%%%%%%%%%%%%%%%%%%%%%%%%%%%%%%%%%%%%%
% \section{Sample}
%\iffalse
%<*samplemain>
%\fi
%
% The following presents a sample document
% with two chapters, two parts, a title page,
% a compile flag as well as three forwarding files to set the flag.
% It consists of eight |.tex| files:
% \begin{center}
% \begin{tabular}{ll}
% |cdocsamp.tex|&main file\\
% |cdocsch1.tex|&include file for chapter 1\\
% |cdocsch2.tex|&include file for chapter 2\\
% |cdocspt3.tex|&include file for part 3\\
% |cdocspt4.tex|&include file for part 4\\
% |cdocsdrf.tex|&forwarding file for main file in draft mode\\
% |cdocsfi1.tex|&forwarding file for final version of chapter 1\\
% |cdocsfi2.tex|&forwarding file for final version of chapter 2\\
% \end{tabular}
% \end{center}
% Each of the eight files can be compiled directly by the \LaTeX{} compiler.
%
% %%%%%%%%%%%%%%%%%%%%%%%%%%%%%%%%%%%%%%
% \paragraph{Main File.}
%
% The main file is called |cdocsamp.tex|.
%
% Load the \textsf{childdoc} definitions and
% declare the filename for the main document:
%    \begin{macrocode}
\input{childdoc.def}
\childdocmain{}
%    \end{macrocode}

% Optional override for |\version| flag:
%    \begin{macrocode}
%%\ifchilddoc\else\providecommand{\version}{draft}\fi
%    \end{macrocode}

% Define the default values for the |\version| flag
% (|final| for the main file and |draft| for childs):
%    \begin{macrocode}
\ifchilddoc
\providecommand{\version}{draft}
\else
\providecommand{\version}{final}
\fi
%    \end{macrocode}

% Load the standard document class:
%    \begin{macrocode}
\documentclass[12pt]{article}
%    \end{macrocode}

% Start the document body:
%    \begin{macrocode}
\begin{document}
%    \end{macrocode}

% Declare a title page.
% Print title, part of document being processed and version flag:
%    \begin{macrocode}
\addtocounter{page}{-1}
\begin{center}
{\LARGE\bfseries{}childdoc example\par}
\vspace{1cm}
\ifchilddoc
\ifchilddocmanual part\else chapter\fi:
`\childdocname' of `\childdocjob'\par
\else
main document: `\childdocjob'\par
\fi
version: \version\par
\end{center}
\newpage
%    \end{macrocode}

% Manually include selected file,
% otherwise process as usual:
%    \begin{macrocode}
\ifchilddocmanual
\section*{part `\childdocname'}
\input{\childdocname}
\else
%    \end{macrocode}

% Include the two chapters:
%    \begin{macrocode}
\include{cdocsch1}
\include{cdocsch2}
%    \end{macrocode}

% Include the two parts unless only chapters should be displayed:
%    \begin{macrocode}
\ifchilddoc\else
\section{part three}
\input{cdocspt3}
\section{part four}
\input{cdocspt4}
\fi
%    \end{macrocode}

% Process as usual until here:
%    \begin{macrocode}
\fi
%    \end{macrocode}

% End of document body:
%    \begin{macrocode}
\end{document}
%    \end{macrocode}
%\iffalse
%</samplemain>
%\fi
%
% %%%%%%%%%%%%%%%%%%%%%%%%%%%%%%%%%%%%%%
% \paragraph{Chapter Include Files.}
%
% The include files are called |cdocsch1.tex| and |cdocsch2.tex|.
%
%\iffalse
%<*samplechap1|samplechap2>
%\fi

% Optional override for |\version| flag:
%    \begin{macrocode}
%%\providecommand{\version}{final}
%    \end{macrocode}

% Include the main document:
%    \begin{macrocode}
\input{childdoc.def}
\childdocof{cdocsamp}
%    \end{macrocode}

%\iffalse
%</samplechap1|samplechap2>
%\fi
%
%\iffalse
%<*samplechap1>
%\fi
% Some text for chapter 1:
%    \begin{macrocode}
\section{one}
some text in chapter one
%    \end{macrocode}

%\iffalse
%</samplechap1>
%\fi
% Some text for chapter 2:
%\iffalse
%<*samplechap2>
%\fi
%    \begin{macrocode}
\section{two}
more text in chapter two
%    \end{macrocode}

%\iffalse
%</samplechap2>
%\fi
%
% %%%%%%%%%%%%%%%%%%%%%%%%%%%%%%%%%%%%%%
% \paragraph{Part Include Files.}
%
% The include files are called |cdocspt3.tex| and |cdocspt4.tex|.
%
%\iffalse
%<*samplepart3|samplepart4>
%\fi

% Optional override for |\version| flag:
%    \begin{macrocode}
%%\providecommand{\version}{final}
%    \end{macrocode}

% Include the main document:
%    \begin{macrocode}
\input{childdoc.def}
\childdocby{cdocsamp}
%    \end{macrocode}

%\iffalse
%</samplepart3|samplepart4>
%\fi
%
%\iffalse
%<*samplepart3>
%\fi
% Some text for part 3:
%    \begin{macrocode}
some text in part three
%    \end{macrocode}

%\iffalse
%</samplepart3>
%\fi
% Some text for part 4:
%\iffalse
%<*samplepart4>
%\fi
%    \begin{macrocode}
more text in part four
%    \end{macrocode}

%\iffalse
%</samplepart4>
%\fi
%
% %%%%%%%%%%%%%%%%%%%%%%%%%%%%%%%%%%%%%%
% \paragraph{Forwarding for a Complete Draft.}
%
% The following forwarding file |cdocsdrf.tex|
% compiles the main document in draft mode:
%\iffalse
%<*sampledraft>
%\fi
%    \begin{macrocode}
\def\version{draft}
\input{childdoc.def}
\childdocforward{cdocsamp}
%    \end{macrocode}

%\iffalse
%</sampledraft>
%\fi
%
% %%%%%%%%%%%%%%%%%%%%%%%%%%%%%%%%%%%%%%
% \paragraph{Forwarding for Final Version of the Chapters.}
%
% The following forwarding files |cdocsfn1.tex| and |cdocsfn2.tex|
% (with identical content)
% compile the final versions of the child documents
% |cdocsch1.tex| and |cdocsch2.tex|, respectively:
%\iffalse
%<*samplefinal>
%\fi
%    \begin{macrocode}
\def\version{final}
\input{childdoc.def}
\childdocforwardprefix[cdocsamp]{cdocsfn}{cdocsch}
%    \end{macrocode}

%\iffalse
%</samplefinal>
%\fi
%
% %%%%%%%%%%%%%%%%%%%%%%%%%%%%%%%%%%%%%%
% \paragraph{Command Line Processing.}
%
% The following three command lines generate the output files
% |cdocscld|, |cdocscl1| and |cdocscl2|
% which should be identical to
% |cdocsdrf|, |cdocsch1| and |cdocsfn2|, respectively:
% \begin{center}
% \begin{tabular}{l}
% |latex -jobname cdocscld \|\\
% |  "\def\version{draft}\input{childdoc.def}\childdocforward{cdocsamp}"|\\
% |latex -jobname cdocscl1 \|\\
% |  "\input{childdoc.def}\childdocforward[cdocsamp]{cdocsch1}"|\\
% |latex -jobname cdocscl2 \|\\
% |  "\def\version{final}\input{childdoc.def}\childdocforward{cdocsch2}"|
% \end{tabular}
% \end{center}
% Note that the trailing backslash on each first line
% merely continues the input to the second line
% (for convenient cut ant paste).
% Furthermore, the command |latex| can be replaced by any
% of its alternative versions such as |pdflatex|.
%
% %%%%%%%%%%%%%%%%%%%%%%%%%%%%%%%%%%%%%%%%%%%%%%%%%%%%%%%%%%%%%%%%%%%%%%%%%%%%%%
% %%%%%%%%%%%%%%%%%%%%%%%%%%%%%%%%%%%%%%%%%%%%%%%%%%%%%%%%%%%%%%%%%%%%%%%%%%%%%%
% \section{Implementation}
%\iffalse
%<*package>
%\fi
%
% This section describes the definitions file |childdoc.def|.

% The definitions cannot be loaded using |\usepackage| or |\RequirePackage|
% which has a mechanism to prevent loading a style file more than once.
% When loading the definitions by means of |\input|
% multiple instances have to be prevented manually:
%\iffalse
%This code needs to be before the `\ProvidesFile' directive
%which is defined at the beginning of this file.
%Therefore it is also placed there and commented out here.
%</package>
%<*discard>
%\fi
%    \begin{macrocode}
\ifdefined\childdocmain\endinput\fi
%    \end{macrocode}
%\iffalse
%</discard>
%<*package>
%\fi
%
% \macro{\ifchilddoc}
% \macro{\ifchilddocmanual}
% The conditional |\ifchilddoc| tells whether a
% child (true) or main (false) document is being compiled.
% The conditional |\ifchilddocmanual| tells whether
% the |\includeonly| mechanism is used (false) or
% the selection of child files must be performed manually (true).
% The definitions initialise to false:
%    \begin{macrocode}
\newif\ifchilddoc
\newif\ifchilddocmanual
%    \end{macrocode}

% \macro{\childdocname}
% \macro{\childdocjob}
% The macro |\childdocname| stores the name of the main document
% to be compiled. The macro |\childdocjob| stores the name of
% the document on which the \LaTeX{} compiler was originally invoked.
% The content of |\jobname| cannot be compared
% to filenames specified in the source due to different catcodes.
% The following code rescans |\jobname|, stores the result
% in |\childdocname| and saves a copy in |\childdocjob|:
%    \begin{macrocode}
\edef\childdocname{\scantokens\expandafter{\jobname\noexpand}}
\let\childdocjob\childdocname
%    \end{macrocode}

% \macro{\childdocdisable}
% The macro |\childdocdisable| prevents the main file
% from being processed more than once.
% At this stage, the main document command |\childdocmain|
% is assumed to be called once again where it should do nothing.
% Any subsequent call to it should prevent
% a secondary processing of the main document
% It overwrites the forwarding commands
% |\childdocof| and |\childdocforward|
% with empty macros to prevent further inclusions of the main document:
%    \begin{macrocode}
\newcommand{\childdocdisable}
{
  \renewcommand{\childdocmain}[1]{\renewcommand{\childdocmain}[1]{\endinput}}
  \renewcommand{\childdocof}[1]{}
  \renewcommand{\childdocby}[2][]{}
  \renewcommand{\childdocforward}[2][]{}
  \renewcommand{\childdocdisable}{}
}
%    \end{macrocode}

% \macro{\childdocmain}
% The macro |\childdocmain| is to be called at the top of the main file
% with nothing or the main filename (without extension) as argument.
% First, it breaks loops.
% If the argument is not empty and does not match |\childdocname|
% (which is set by the first inclusion of |childdoc.def|),
% |\ifchilddoc| is set to true, |\includeonly| is applied to the child file
% and |\jobname| is set to the main file
% (for proper handling of |.aux| files):
%    \begin{macrocode}
\newcommand{\childdocmain}[1]
{
  \childdocdisable\childdocmain{}
  \if?#1?\else
    \begingroup
      \def\childdoctmp{#1}
      \ifx\childdoctmp\childdocname
        \def\childdoctmp{}
      \else
        \def\childdoctmp
        {
          \childdoctrue
          \includeonly{\childdocname}
          \def\childdocjob{#1}
          \def\jobname{#1}
        }
      \fi
      \expandafter
    \endgroup
    \childdoctmp
  \fi
}
%    \end{macrocode}

% \macro{\childdocof}
% The command |\childdocof| redirects
% compilation to the main file |#1|.
%    \begin{macrocode}
\newcommand{\childdocof}[1]
{
  \childdocdisable
  \childdoctrue
  \includeonly{\childdocname}
  \def\jobname{#1}
  \def\childdocjob{#1}
  \input{#1}
}
%    \end{macrocode}

% \macro{\childdocby}
% The command |\childdocby| ....
%    \begin{macrocode}
\newcommand{\childdocby}[2][]
{
  \childdocdisable
  \childdoctrue
  \childdocmanualtrue
  \if?#1?\else
    \def\jobname{#2}
  \fi
  \def\childdocjob{#2}
  \input{#2}
  \endinput
}
%    \end{macrocode}

% \macro{\childdocforward}
% The command |\childdocforward| redirects
% compilation to the main file or
% (if the optional argument is given) a child file.
% Parameters are set as if the main file
% or a child file starting with |\childdocof| was compiled.
% Then compilation is handed over to the main file:
%    \begin{macrocode}
\newcommand{\childdocforward}[2][]
{
  \begingroup
    \if?#1?
      \def\childdoctmp
      {
        \def\childdocname{#2}
        \def\childdocjob{#2}
        \def\jobname{#2}
        \input{#2}
        \endinput
      }
    \else
      \def\childdoctmp
      {
        \childdocdisable
        \def\childdocname{#2}
        \childdoctrue
        \includeonly{#2}
        \def\childdocjob{#1}
        \def\jobname{#1}
        \input{#1}
        \endinput
      }
    \fi
    \expandafter
  \endgroup
  \childdoctmp
}
%    \end{macrocode}

% \macro{\childdocforwardprefix}
% The command |\childdocforwardprefix| redirects
% compilation to the main or a child file by means of a pattern.
% The prefix |#1| in the current filename is replaced by |#2|
% and the suffix of the current filename is kept
% (it is assumed that the filename does not contain the substring `|~~~|'
% which is used as a delimiter).
% Compilation is handed over to the new file by |\childdocforward|:
%    \begin{macrocode}
\newcommand{\childdocforwardprefix}[3][]
{
  \begingroup
    \def\childdocextract #2##1~~~{\def\childdoctmp{\childdocforward[#1]{#3##1}}}
    \expandafter\childdocextract\childdocname~~~
    \expandafter
  \endgroup
  \childdoctmp
}
%    \end{macrocode}

% \macro{\childdoc}
% The deprecated macro |\childdoc| is a legacy version of |\childdocmain|:
%    \begin{macrocode}
\newcommand{\childdoc}{\childdocmain}
%    \end{macrocode}

% \macro{\childdocredirect}
% The deprecated macro |\childdocredirect| is a legacy version
% of |\childdocforward| and |\childdocforwardprefix|:
%    \begin{macrocode}
\newcommand{\childdocredirect}[2][]
{
  \begingroup
    \if?#1?
      \def\childdoctmp{\childdocforward{#2}}
    \else
      \def\childdoctmp{\childdocforwardprefix{#1}{#2}}
    \fi
    \expandafter
  \endgroup
  \childdoctmp
}
%    \end{macrocode}

%\iffalse
%</package>
%\fi
%
\endinput

\childdocof{cdocsamp}
%    \end{macrocode}

%\iffalse
%</samplechap1|samplechap2>
%\fi
%
%\iffalse
%<*samplechap1>
%\fi
% Some text for chapter 1:
%    \begin{macrocode}
\section{one}
some text in chapter one
%    \end{macrocode}

%\iffalse
%</samplechap1>
%\fi
% Some text for chapter 2:
%\iffalse
%<*samplechap2>
%\fi
%    \begin{macrocode}
\section{two}
more text in chapter two
%    \end{macrocode}

%\iffalse
%</samplechap2>
%\fi
%
% %%%%%%%%%%%%%%%%%%%%%%%%%%%%%%%%%%%%%%
% \paragraph{Part Include Files.}
%
% The include files are called |cdocspt3.tex| and |cdocspt4.tex|.
%
%\iffalse
%<*samplepart3|samplepart4>
%\fi

% Optional override for |\version| flag:
%    \begin{macrocode}
%%\providecommand{\version}{final}
%    \end{macrocode}

% Include the main document:
%    \begin{macrocode}
% \iffalse
%
% childdoc.dtx Copyright (C) 2017-2018 Niklas Beisert
%
% This work may be distributed and/or modified under the
% conditions of the LaTeX Project Public License, either version 1.3
% of this license or (at your option) any later version.
% The latest version of this license is in
%   http://www.latex-project.org/lppl.txt
% and version 1.3 or later is part of all distributions of LaTeX
% version 2005/12/01 or later.
%
% This work has the LPPL maintenance status `maintained'.
%
% The Current Maintainer of this work is Niklas Beisert.
%
% This work consists of the files childdoc.dtx and childdoc.ins
% and the derived files childdoc.def and cdocsamp.tex with
% cdocsch1.tex, cdocsch2.tex, cdocsdrf.tex, cdocsfn1.tex, cdocsfn2.tex.
%
%<package>\ifdefined\childdocmain\endinput\fi
%<package>\ProvidesFile{childdoc.def}[2018/12/30 v2.0 child document driver]
%<samplemain>\ProvidesFile{cdocsamp.tex}[2018/12/30 v2.0 sample for childdoc]
%<*driver>
%\ProvidesFile{childdoc.drv}[2018/12/30 v2.0 childdoc reference manual file]
\PassOptionsToClass{10pt,a4paper}{article}
\documentclass{ltxdoc}

\usepackage[margin=35mm]{geometry}
\usepackage{hyperref}
\usepackage{hyperxmp}
\usepackage[usenames]{color}

\hypersetup{colorlinks=true}
\hypersetup{pdfstartview=FitH}
\hypersetup{pdfpagemode=UseNone}
\hypersetup{pdfsource={}}
\hypersetup{pdflang={en-UK}}
\hypersetup{pdfcopyright={Copyright 2017-2018 Niklas Beisert.
  This work may be distributed and/or modified under the
  conditions of the LaTeX Project Public License, either version 1.3
  of this license or (at your option) any later version.}}
\hypersetup{pdflicenseurl={http://www.latex-project.org/lppl.txt}}
\hypersetup{pdfcontactaddress={ETH Zurich, ITP, HIT K,
  Wolfgang-Pauli-Strasse 27}}
\hypersetup{pdfcontactpostcode={8093}}
\hypersetup{pdfcontactcity={Zurich}}
\hypersetup{pdfcontactcountry={Switzerland}}
\hypersetup{pdfcontactemail={nbeisert@itp.phys.ethz.ch}}
\hypersetup{pdfcontacturl={http://people.phys.ethz.ch/\xmptilde nbeisert/}}

\newcommand{\secref}[1]{\hyperref[#1]{section \ref*{#1}}}

\parskip1ex
\parindent0pt
\let\olditemize\itemize
\def\itemize{\olditemize\parskip0pt}

\begin{document}

\title{The \textsf{childdoc} Package}
\hypersetup{pdftitle={The childdoc Package}}
\author{Niklas Beisert\\[2ex]
  Institut f\"ur Theoretische Physik\\
  Eidgen\"ossische Technische Hochschule Z\"urich\\
  Wolfgang-Pauli-Strasse 27, 8093 Z\"urich, Switzerland\\[1ex]
  \href{mailto:nbeisert@itp.phys.ethz.ch}
  {\texttt{nbeisert@itp.phys.ethz.ch}}}
\hypersetup{pdfauthor={Niklas Beisert}}
\hypersetup{pdfsubject={Manual for the LaTeX2e Package childdoc}}
\date{30 December 2018, \textsf{v2.0}}
\maketitle

\begin{abstract}\noindent
\textsf{childdoc} is a \LaTeXe{} package
that enables the direct compilation
of document sections included by |\include|
to individual files.
\end{abstract}

\begingroup
\parskip0ex
\tableofcontents
\endgroup

%%%%%%%%%%%%%%%%%%%%%%%%%%%%%%%%%%%%%%%%%%%%%%%%%%%%%%%%%%%%%%%%%%%%%%%%%%%%%%%%
%%%%%%%%%%%%%%%%%%%%%%%%%%%%%%%%%%%%%%%%%%%%%%%%%%%%%%%%%%%%%%%%%%%%%%%%%%%%%%%%
\section{Introduction}

\LaTeX{} provides a mechanism to structure a large document (such as a book)
into a main file and several child files (containing the chapters)
using the |\include| command.
This mechanism is beneficial for documents
which span hundreds of pages in order to
make the source file(s) more manageable.
Moreover, compilation can be restricted to
selected child files by means of the |\includeonly| command.
The latter feature can be used to reduce the compilation time while editing
(this was significantly more useful in the earlier days of \LaTeX{})
or to generate a smaller document which is easier to navigate.
Another application of |\includeonly| is to generate
documents consisting of selected parts of the complete document.

However, there are a few drawbacks of the plain |\include| mechanism:
\begin{itemize}
\item
The child files cannot be compiled on their own,
they can only be compiled via the main file.
A naive editing environment
(such as a text editor with an option
to have the current file processed by \LaTeX)
may require one to switch to the main file before compiling;
attempting to compile the child file produces errors.
\item
The main file must be modified (each time)
to adjust the |\includeonly| command
to the present needs. This easily leaves the main file in a messy state.
\item
The generated document will always carry the filename
of the main document. This is inconvenient if
several child files are to be compiled and
to be kept for distribution.
\end{itemize}

The present package provides a simple interface
to make child files individually compilable by \LaTeX{}.
Compiling a child file then has the same effect as compiling
the main file with an |\includeonly| command
to select the appropriate child.
Moreover the generated document will carry the name of the child
rather than the main file.
This resolves all three above issues.

This feature is meant to make the editing of books,
thesis documents and lecture notes somewhat more convenient.
However, the package can also be used efficiently for
composing a series of documents (such as exercise sheets)
which are typically distributed individually.
It then assists the author in generating the individual documents
(potentially in different versions)
as well as a document containing the collected series.
Another application is in developing style files
or other kinds of included material
where compilation of the style file could redirect
to a sample or test file.

%%%%%%%%%%%%%%%%%%%%%%%%%%%%%%%%%%%%%%%%%%%%%%%%%%%%%%%%%%%%%%%%%%%%%%%%%%%%%%%%
%%%%%%%%%%%%%%%%%%%%%%%%%%%%%%%%%%%%%%%%%%%%%%%%%%%%%%%%%%%%%%%%%%%%%%%%%%%%%%%%
\section{Usage}

First of all, the package \textsf{childdoc} is \emph{not} a standard
\LaTeXe{} |.sty| style file! Therefore it needs to be invoked in
a non-standard way.

%%%%%%%%%%%%%%%%%%%%%%%%%%%%%%%%%%%%%%%%%%%%%%%%%%%%%%%%%%%%%%%%%%%%%%%%%%%%%%%%
\subsection{Included Files}
\label{sec:include}

%%%%%%%%%%%%%%%%%%%%%%%%%%%%%%%%%%%%%%%%
\DescribeMacro{\childdocmain}
To use the package, add the commands
\begin{center}
\begin{tabular}{l}
|\input{childdoc.def}|\\
|\childdocmain{}|\\
\end{tabular}
\end{center}
at the very top of the main \LaTeX{} file,
in particular \emph{before} the |\documentclass| statement!
The argument of |\childdocmain| should be left empty
(but it must be present).

%%%%%%%%%%%%%%%%%%%%%%%%%%%%%%%%%%%%%%%%
\DescribeMacro{\childdocof}
Furthermore, add the commands
\begin{center}
\begin{tabular}{l}
|\input{childdoc.def}|\\
|\childdocof{|\textit{main}|}|\\
\end{tabular}
\end{center}
at the top of every child file \textit{child}
which is included by |\include{|\textit{child}|}|
from within the main file
(or at least for those files to be compiled individually).
The argument \textit{main} must be the filename of the main file.

There are a couple of
considerations in setting up the main and child documents:

%%%%%%%%%%%%%%%%%%%%%%%%%%%%%%%%%%%%%%%%
\paragraph{Restrictions.}

Please note the following restrictions:
\begin{itemize}
\item
|\childdocmain| must be called with one argument \textit{main}
to ensure compatibility with earlier version of the package.
It must either be empty (|\childdocmain{}|)
or precisely match the filename of the main file in which it is specified.
See \secref{sec:detection} for further information.
\item
The filename \textit{main} must be specified without the |.tex| extension.
\item
The filename \textit{main} is case sensitive
(even in case-insensitive file systems)
due to internal string comparison.
\item
The argument \textit{main} should be fully expanded, it cannot be a macro.
\item
Subdirectories and special characters should be avoided in filenames.
\item
The command |\childdocmain{|\textit{main}|}| must be followed by a whitespace.
It should not be followed immediately by another command
or by a comment mark `|%|'.
This is because the \TeX{} parser reads the token immediately following
the argument of |\childdocmain| and puts it
at the beginning of every child section;
however, a white\-space is ignored.
\end{itemize}

%%%%%%%%%%%%%%%%%%%%%%%%%%%%%%%%%%%%%%%%
\paragraph{Content of Main File.}

It is advisable to place all content in the child files included by |\include|.
Any output contained in the main file will appear in all child documents
unless suppressed manually;
it cannot be suppressed automatically by the |\includeonly| directive
and thus should normally be avoided.
A method to include some content in the main file
by means of conditional processing is described in \secref{sec:conditional}.

%%%%%%%%%%%%%%%%%%%%%%%%%%%%%%%%%%%%%%%%
\paragraph{Page Numbering.}

When only a part of the document is compiled,
the appropriate numbering of pages
(as well as other status parameters)
is determined from the |.aux| files.
The latter contain information from previous passes.
However this information needs to propagate through
all intermediate child documents.
Therefore the page numbering in child documents may well
be inconsistent until the complete document is compiled at least once.

A useful (if unconventional) way to always ensure a consistent
page numbering is to restart the numbering in each child document
and denote the pages by `\textit{child}|.|\textit{page}'
where \textit{child} represents the chapter/section number of the child file.
This can be achieved by the command
|\numberwithin{page}{|\textit{child}|}|
of the \textsf{amsmath} package
where \textit{child} can be |chapter| or |section|
depending on the chosen structuring.
Alternatively, one can modify the macro |\thepage| appropriately
and reset the counter |page| at the start of each child file.

%%%%%%%%%%%%%%%%%%%%%%%%%%%%%%%%%%%%%%%%%%%%%%%%%%%%%%%%%%%%%%%%%%%%%%%%%%%%%%%%
\subsection{Conditional Processing}
\label{sec:conditional}

The package provides a mechanism to compile different versions
of a document. To customise the versions further some conditional processing
can come in handy to distinguish which version is being compiled.
The package provides two macros to describe the compilation context:

%%%%%%%%%%%%%%%%%%%%%%%%%%%%%%%%%%%%%%%%
\DescribeMacro{\ifchilddoc}
The conditional |\ifchilddoc| distinguishes between the compilation of
child documents and the main document:
%
\begin{center}
|\ifchilddoc |\textit{child-code}| |[|\||else |\textit{main-code}]| \||fi|
\end{center}

%%%%%%%%%%%%%%%%%%%%%%%%%%%%%%%%%%%%%%%%
\DescribeMacro{\childdocname}
\DescribeMacro{\childdocjob}
The macro |\childdocname| contains the filename (without extension)
of the main or child file being processed.
Note that |\childdocjob| will always contain the name of the main file.

%%%%%%%%%%%%%%%%%%%%%%%%%%%%%%%%%%%%%%%%
\paragraph{Title Page.}

Conditional processing can be used to include a title or banner page
in the main document when proper precautions are taken.
Importantly, the code in the main file should ensure that the page counter
(as well as other status parameters which are stored in the |.aux| files)
takes the same value after the conditional processing.
Otherwise the page numbers may take divergent values
depending on which part is compiled.

For example, a title page could be declared by:
%
\begin{center}
\begin{tabular}{l}
|\ifchilddoc\||else|\\
|\addtocounter{page}{-1}|\\
\textit{code for title page}\\
|\newpage|\\
|\||fi|
\end{tabular}
\end{center}
%
A banner page for the child documents can be generated by:
%
\begin{center}
\begin{tabular}{l}
|\ifchilddoc|\\
|\addtocounter{page}{-1}|\\
\textit{code for banner page}\\
|\newpage|\\
|\||fi|
\end{tabular}
\end{center}
%
Here one could write a message such as:
\begin{center}
|This is the part \childdocname{} of \childdocjob{}.|
\end{center}

%%%%%%%%%%%%%%%%%%%%%%%%%%%%%%%%%%%%%%%%%%%%%%%%%%%%%%%%%%%%%%%%%%%%%%%%%%%%%%%%
\subsection{Flags}
\label{sec:flags}

The package makes it easy to generate different versions
of the main or child documents.
To this end compilation flags can be defined
and assigned different default values.
They will be particularly useful in conjunction
with the forwarding mechanism described in \secref{sec:forward}.

For example, it may be useful to have a flag |\version|
which can be set to |draft| or |final|.
The document source will contain some conditional code
depending on the value of |\version|.
Suppose further, the flag should default to |final| for the main file
and to |draft| for child files
which is a natural assignment for editing the document.
This is achieved by placing the following code
in the preamble of the main document
(below the |\childdocmain| directive):
%
\begin{center}
\begin{tabular}{l}
|\ifchilddoc|\\
|\providecommand{\version}{draft}|\\
|\||else|\\
|\providecommand{\version}{final}|\\
|\||fi|
\end{tabular}
\end{center}
%
The definition by |\providecommand| makes sure
that previous definitions are not overwritten.
Further statements |\providecommand{\version}{...}|
can thus be added before the above code to override it.

For the main file, one might add a line
(between |\childdocmain| and the above block)
%
\begin{center}
|%\ifchilddoc\||else\providecommand{\version}{draft}\||fi|
\end{center}
%
which can be uncommented to produce a draft version.
Likewise one can add a line to the very top of a child file
(above the |\childdocof{|\textit{main}|}| directive)
%
\begin{center}
|%\providecommand{\version}{final}|
\end{center}
%
which can be uncommented to produce the final version of this child document.

%%%%%%%%%%%%%%%%%%%%%%%%%%%%%%%%%%%%%%%%%%%%%%%%%%%%%%%%%%%%%%%%%%%%%%%%%%%%%%%%
\subsection{Forwarding}
\label{sec:forward}

Different versions of the main or child documents
using compilation flags as described in \secref{sec:flags}
can be (permanently) stored in different files
for convenient compilation, viewing and distribution.
To this end, the package defines a command
to pass on compilation to a different file:

%%%%%%%%%%%%%%%%%%%%%%%%%%%%%%%%%%%%%%%%
\DescribeMacro{\childdocforward}
The command |\childdocforward| redirects processing to
another source file:
%
\begin{center}
\begin{tabular}{l}
|\input{childdoc.def}|\\
|\childdocforward[|\textit{main}|]{|\textit{dest}|}|\\
\end{tabular}
\end{center}
%
The argument \textit{dest} is the destination file
(without extension).
It should be the main file or one of the child files.
Note that further \textsf{childdoc} directives
such as |\childdocof| and |\childdocforward|
in the indicated file will be processed in this form.
The optional argument \textit{main}
passes on directly to the main file \textit{main}
while pretending to compile the child \textit{dest}.
This form behaves as if \textit{dest}
issues |\childdocof{|\textit{main}|}| right away,
and no further \textsf{childdoc} directives will be processed.

%%%%%%%%%%%%%%%%%%%%%%%%%%%%%%%%%%%%%%%%
\DescribeMacro{\...prefix}
In the alternative form |\childdocforwardprefix|,
%
\begin{center}
\begin{tabular}{l}
|\input{childdoc.def}|\\
|\childdocforwardprefix[|\textit{main}|]{|\textit{prefix}|}{|\textit{dest}|}|
\end{tabular}
\end{center}
%
the destination file is determined by a pattern
depending on the current file:
To make this work, the current file must be called
`{\textit{prefix}\hspace{0.2em}\textit{suffix}}'
with \textit{prefix} matching precisely the argument.
Processing is then passed on to the file
`{\textit{dest}\hspace{0.2em}\textit{suffix}}'.
Surely, the same effect is achieved by
directly specifying the
argument `{\textit{dest}\hspace{0.2em}\textit{suffix}}'
in the first form.
However, that requires to set up a different file
for each child. With the alternative form of the command
all these files can have exactly the same content
which simplifies setting them up and maintaining them.

For example, the following file |draft.tex|
with a compilation flag |\version| as described in \secref{sec:flags}
compiles the main document as a draft:
%
\begin{center}
\begin{tabular}{l}
|\def\version{draft}|\\
|\input{childdoc.def}|\\
|\childdocforward{|\textit{main}|}|
\end{tabular}
\end{center}
%
Likewise, the following files |final|\textit{nn}|.tex|
compile the final version of the child document
|child|\textit{nn}|.tex|:
%
\begin{center}
\begin{tabular}{l}
|\def\version{final}|\\
|\input{childdoc.def}|\\
|\childdocforwardprefix{final}{child}|
\end{tabular}
\end{center}
%

Note that when several versions of a main file and/or of each child file
are to be generated, it may be convenient to set up a |Makefile| or
shell script to automatise the process.

%%%%%%%%%%%%%%%%%%%%%%%%%%%%%%%%%%%%%%%%%%%%%%%%%%%%%%%%%%%%%%%%%%%%%%%%%%%%%%%%
\subsection{Command Line Processing}
\label{sec:commandline}

The effect of redirection files can also be achieved by invoking
the \LaTeX{} compiler with a more elaborate command line.
Most conveniently this should be done as part
of a shell script or a |Makefile|.

When using \textsf{childdoc} in the main file, the following
command lines effectively perform a redirection
(note that depending on the shell being used,
backslashes may have to be doubled: `|\|' $\to$ `|\\|'):
%
\begin{center}
|... -jobname "|\textit{target}|" |\\|"|[\textit{flags}]%
|\input{childdoc.def}\childdocforward[|\textit{main}|]{|\textit{dest}|}"|
\end{center}
%
Here \textit{target} is the name of the output file,
\textit{main} is the name of the main file
and \textit{dest} is the name of the main or child file to be processed
(all filenames without extensions).
The optional argument \textit{main} can be omitted
if \textit{main} matches \textit{dest}.
Optionally, compilation \textit{flags} can be defined via |\def| commands.
This command line makes the \TeX{} engine believe
it is compiling the file \textit{target}
whose content is specified as the latter parameter.
The provided code then forwards the processing to
\textit{main} or \textit{dest} as described in \secref{sec:forward}.

%%%%%%%%%%%%%%%%%%%%%%%%%%%%%%%%%%%%%%%%%%%%%%%%%%%%%%%%%%%%%%%%%%%%%%%%%%%%%%%%
\subsection{Include by Input}
\label{sec:input}

Including child documents by |\include| has some restrictions by design.
Most notably, the content of a child document always occupies
its own set of pages; pages cannot be shared between child documents.
Usually, this behaviour makes perfect sense
because each child document contain an essential part of the document.
However, in some situations it may be desirable to compose
a document from a collection of parts
without having mandatory page breaks between then.
For this case, the package
provides a mechanism to include parts
by |\input| which can also be processed individually.
However, by construction this mechanism
requires manual handling of the content to be output.

%%%%%%%%%%%%%%%%%%%%%%%%%%%%%%%%%%%%%%%%
\DescribeMacro{\ifchilddocmanual}
The main file should be prepared as usual, see \secref{sec:include}.
However, the document body must make a distinction
between processing of an individual part and of the main document, e.g.:
%
\begin{center}
\begin{tabular}{l}
|\ifchilddocmanual|\\
|\input{\childdocname}|\\
|\||else|\\
\textit{document body with }|\input{|\textit{part}|}|\\
|\||fi|
\end{tabular}
\end{center}
%
The conditional |\ifchilddocmanual| is true whenever
a part to be included by |\input| is being compiled,
and the name of the part is stored in |\childdocname|.

%%%%%%%%%%%%%%%%%%%%%%%%%%%%%%%%%%%%%%%%
\DescribeMacro{\childdocby}
Each part to be included by |\input| should start with:
%
\begin{center}
\begin{tabular}{l}
|\input{childdoc.def}|\\
|\childdocby{|\textit{main}|}|\\
\end{tabular}
\end{center}
%
The directive |\childdocby| is similar to |\childdocof|
described in \secref{sec:include},
but the subsequent selection of content must be done manually.
To that end, both |\ifchilddoc| and |\ifchilddocmanual|
will be true upon processing of a part,
and the name of the part is stored in |\childdocname|.
Note that |\jobname| will be set to the filename of the current part
so that each part receives an individual |.aux| file
that does not interfere with the |.aux| file(s) of the main document.
This behaviour can be altered by the alternative form
|\childdocby[*]{|\textit{main}|}| (with a non-empty optional argument)
which uses the |.aux| file of the main document
by setting |\jobname| to \textit{main}.

%%%%%%%%%%%%%%%%%%%%%%%%%%%%%%%%%%%%%%%%%%%%%%%%%%%%%%%%%%%%%%%%%%%%%%%%%%%%%%%%
\subsection{Driver Development}
\label{sec:driver}

The \textsf{childdoc} mechanism can also be use for the development
of definition files such as \LaTeX{} styles or classes.
This case differs from the above setup with multiple parts
included by |\include| in that no |\includeonly| should be invoked.
This can be achieved by starting the include file
(before |\ProvidesPackage|) with:
%
\begin{center}
\begin{tabular}{l}
|\input{childdoc.def}|\\
|\childdocforward{|\textit{main}|}|\\
\end{tabular}
\end{center}
%
or alternatively with:
%
\begin{center}
\begin{tabular}{l}
|\input{childdoc.def}|\\
|\childdocby{|\textit{main}|}|\\
\end{tabular}
\end{center}
%
Both forms have slightly different effects as described above.
The main file is prepared as usual, see \secref{sec:include}.

%%%%%%%%%%%%%%%%%%%%%%%%%%%%%%%%%%%%%%%%%%%%%%%%%%%%%%%%%%%%%%%%%%%%%%%%%%%%%%%%
\subsection{Legacy Detection}
\label{sec:detection}

The directive |\childdocmain| in the main file can detect
whether the complete document or merely a child is to be compiled
even without using the directive |\childdocof|.
This method is deprecated because it is less robust
and there is no compelling reason to use it;
it is merely provided for backward compatibility
and it may be removed in future versions.

If the detection mechanism is to be used,
it is mandatory to correctly specify
the filename of the main file as the argument of |\childdocmain|:
%
\begin{center}
\begin{tabular}{l}
|\input{childdoc.def}|\\
|\childdocmain{|\textit{main}|}|\\
\end{tabular}
\end{center}
%
If |\jobname| does not match the argument \textit{main} of |\childdocmain|,
it is assumed that |\jobname| points to the child file to be compiled.
When using |\childdocmain| with the main file specified as argument,
it suffices to start a child file
with just |\input{|\textit{main}|}|
without loading of the package and using |\childdocof|.
If instead all processing is done
with the appropriate \textsf{childdoc} directives,
the argument of \textit{main} of |\childdocmain| can be empty.

An alternative version of the command line processing described
in \secref{sec:commandline} using the detection mechanism reads:
%
\begin{center}
|... -jobname "|\textit{target}|" "|[\textit{flags}]%
[|\def\jobname{|\textit{dest}|}|]|\input{|\textit{main}|}"|
\end{center}

%%%%%%%%%%%%%%%%%%%%%%%%%%%%%%%%%%%%%%%%%%%%%%%%%%%%%%%%%%%%%%%%%%%%%%%%%%%%%%%%
\subsection{Manual Code}
\label{sec:manual}

In case one cannot be certain whether the definitions file |childdoc.def|
is installed on the target \TeX{} distribution
and one prefers not to ship it,
it is conceivable to paste a few relevant commands into the sources.

To that end, drop all statements |\input{childdoc.def}|
and perform the replacements as outlined below.
Instead of |\childdocmain{|\textit{main}|}| add the following code
to the top of the main file:
%
\begin{center}
\begin{tabular}{l}
|\||ifdefined\childdocname\endinput\||fi\newif\ifchilddoc|\\
|\edef\childdocname{\scantokens\expandafter{\jobname\noexpand}}|\\
|\def\childdocmain{|\textit{main}|}\||ifx\childdocmain\childdocname\||else|\\
|\childdoctrue\includeonly{\childdocname}\let\jobname\childdocmain\||fi|\\
\end{tabular}
\end{center}
%
Instead of |\childdocof{|\textit{main}|}| just include the main file
at the top of each child file:
%
\begin{center}
|\input{|\textit{main}|}|
\end{center}
%
A simple redirection |\childdocforward{|\textit{dest}|}| is achieved by:
%
\begin{center}
|\def\jobname{|\textit{dest}|}\input{\jobname}|
\end{center}
%
The redirection with prefix
|\childdocforwardprefix[|\textit{prefix}|]{|\textit{dest}|}|
is accomplished by:
%
\begin{center}
\begin{tabular}{l}
|{\edef\jobname{\scantokens\expandafter{\jobname\noexpand}}|\\
|\def\redirectjob |\textit{prefix}|#1~~~{\gdef\jobname{|\textit{dest}|#1}}|\\
|\expandafter\redirectjob\jobname~~~}\input{\jobname}|
\end{tabular}
\end{center}

In an alternative approach,
child documents can be compiled by a specific command line
without additional code or specific definitions:
%
\begin{center}
|... -jobname "|\textit{target}|" "|[\textit{flags}]%
|\includeonly{|\textit{dest}|}\input{|\textit{main}|}"|
\end{center}
%

%%%%%%%%%%%%%%%%%%%%%%%%%%%%%%%%%%%%%%%%%%%%%%%%%%%%%%%%%%%%%%%%%%%%%%%%%%%%%%%%
%%%%%%%%%%%%%%%%%%%%%%%%%%%%%%%%%%%%%%%%%%%%%%%%%%%%%%%%%%%%%%%%%%%%%%%%%%%%%%%%
\section{Information}

%%%%%%%%%%%%%%%%%%%%%%%%%%%%%%%%%%%%%%%%%%%%%%%%%%%%%%%%%%%%%%%%%%%%%%%%%%%%%%%%
\subsection{Copyright}

Copyright \copyright{} 2017--2018 Niklas Beisert

This work may be distributed and/or modified under the
conditions of the \LaTeX{} Project Public License, either version 1.3
of this license or (at your option) any later version.
The latest version of this license is in
  \url{http://www.latex-project.org/lppl.txt}
and version 1.3 or later is part of all distributions of \LaTeX{}
version 2005/12/01 or later.

This work has the LPPL maintenance status `maintained'.

The Current Maintainer of this work is Niklas Beisert.

This work consists of the files |README.txt|, |childdoc.ins| and |childdoc.dtx|
as well as the derived files |childdoc.def|, |cdocsamp.tex|
with |cdocsch1.tex|, |cdocsch2.tex|, |cdocspt3.tex|, |cdocspt4.tex|,
|cdocsdrf.tex|, |cdocsfn1.tex|, |cdocsfn2.tex|
as well as |childdoc.pdf|.

%%%%%%%%%%%%%%%%%%%%%%%%%%%%%%%%%%%%%%%%%%%%%%%%%%%%%%%%%%%%%%%%%%%%%%%%%%%%%%%%
\subsection{Files and Installation}

The package consists of the files:
%
\begin{center}
\begin{tabular}{ll}
    |README.txt|   & readme file \\
    |childdoc.ins| & installation file \\
    |childdoc.dtx| & source file \\
    |childdoc.def| & definition file \\
    |cdocsamp.tex| & sample main file \\
    |cdocsch1.tex| & sample include file \\
    |cdocsch2.tex| & sample include file \\
    |cdocspt3.tex| & sample part file \\
    |cdocspt4.tex| & sample part file \\
    |cdocsdrf.tex| & sample redirection file \\
    |cdocsfn1.tex| & sample redirection file \\
    |cdocsfn2.tex| & sample redirection file \\
    |childdoc.pdf| & manual
\end{tabular}
\end{center}
%
The distribution consists of the files
|README.txt|, |childdoc.ins| and |childdoc.dtx|.
%
\begin{itemize}
\item
Run (pdf)\LaTeX{} on |childdoc.dtx|
to compile the manual |childdoc.pdf| (this file).
\item
Run \LaTeX{} on |childdoc.ins| to create the definitions file |childdoc.def|
and the sample |cdocsamp.tex| with include files
|cdocsch1.tex|, |cdocsch2.tex|, |cdocspt3.tex|, |cdocspt4.tex|,
|cdocsdrf.tex|, |cdocsfn1.tex|, |cdocsfn2.tex|.
Then copy the file |childdoc.def| to an appropriate directory of your \LaTeX{}
distribution, e.g.\ \textit{texmf-root}|/tex/latex/childdoc|.
\end{itemize}

%%%%%%%%%%%%%%%%%%%%%%%%%%%%%%%%%%%%%%%%%%%%%%%%%%%%%%%%%%%%%%%%%%%%%%%%%%%%%%%%
\subsection{Related CTAN Packages}

There are several other packages which offer a similar functionality:
%
\begin{itemize}
\item
The packages
\href{http://ctan.org/pkg/docmute}{\textsf{docmute}},
\href{http://ctan.org/pkg/includex}{\textsf{includex}} and
\href{http://ctan.org/pkg/standalone}{\textsf{standalone}}
provide commands to include only the document body of
a child file thus allowing both files to be compiled individually.
\item
The packages \href{http://ctan.org/pkg/subdocs}{\textsf{subdocs}}
and \href{http://ctan.org/pkg/subfiles}{\textsf{subfiles}}
provide structures in which the main and child documents can be
encapsulated and allowing them to be compiled individually.
The inclusion mechanism is different from the conventional |\include|.
\item
The package \href{http://ctan.org/pkg/combine}{\textsf{combine}}
is an elaborate solution to combine several documents into one.
\end{itemize}
%
See also the CTAN topic \href{http://ctan.org/topic/subdocs}{\textsf{subdocs}}
for further related packages.
The present package differs from the above solutions in that
a document structure constructed with the conventional |\include| mechanism
just needs two extra commands at the top of every file
such that all constituent files can be compiled individually.

%%%%%%%%%%%%%%%%%%%%%%%%%%%%%%%%%%%%%%%%%%%%%%%%%%%%%%%%%%%%%%%%%%%%%%%%%%%%%%%%
%\subsection{Feature Suggestions}
%
%The following is a list of features which may be useful for future
%versions of this package:
%%
%\begin{itemize}
%\item
%\ldots
%\end{itemize}

%%%%%%%%%%%%%%%%%%%%%%%%%%%%%%%%%%%%%%%%%%%%%%%%%%%%%%%%%%%%%%%%%%%%%%%%%%%%%%%%
\subsection{Revision History}

%%%%%%%%%%%%%%%%%%%%%%%%%%%%%%%%%%%%%%%%
\paragraph{v2.0:} 2018/12/30

\begin{itemize}
\item
immediate forward processing
\item
added |\childdocby| mechanism
\item
manual restructured
\end{itemize}

%%%%%%%%%%%%%%%%%%%%%%%%%%%%%%%%%%%%%%%%
\paragraph{v1.6:} 2018/01/17

\begin{itemize}
\item
application for development of include files
\item
corrections to manual
\end{itemize}

%%%%%%%%%%%%%%%%%%%%%%%%%%%%%%%%%%%%%%%%
\paragraph{v1.5:} 2017/05/21

\begin{itemize}
\item
more complete structuring introduced
\item
|\childdocof| introduced
\item
|\childdoc| renamed to |\childdocmain|
\item
|\childredirect| renamed to |\childdocforward| and |\childdocforwardprefix|
and functionality expanded
\end{itemize}

%%%%%%%%%%%%%%%%%%%%%%%%%%%%%%%%%%%%%%%%
\paragraph{v1.0:} 2017/04/27

\begin{itemize}
\item
manual and install package
\item
first version published on CTAN
\end{itemize}

%%%%%%%%%%%%%%%%%%%%%%%%%%%%%%%%%%%%%%%%
\paragraph{v0.6:} 2017/04/26

\begin{itemize}
\item
redirection mechanism added
\end{itemize}

%%%%%%%%%%%%%%%%%%%%%%%%%%%%%%%%%%%%%%%%
\paragraph{v0.5:} 2017/04/26

\begin{itemize}
\item
functionality in definition file
\end{itemize}


%%%%%%%%%%%%%%%%%%%%%%%%%%%%%%%%%%%%%%%%%%%%%%%%%%%%%%%%%%%%%%%%%%%%%%%%%%%%%%%%
%%%%%%%%%%%%%%%%%%%%%%%%%%%%%%%%%%%%%%%%%%%%%%%%%%%%%%%%%%%%%%%%%%%%%%%%%%%%%%%%
%%%%%%%%%%%%%%%%%%%%%%%%%%%%%%%%%%%%%%%%%%%%%%%%%%%%%%%%%%%%%%%%%%%%%%%%%%%%%%%%
\appendix

\settowidth\MacroIndent{\rmfamily\scriptsize 000\ }

 \DocInput{childdoc.dtx}

\end{document}
%</driver>
% \fi
%
% %%%%%%%%%%%%%%%%%%%%%%%%%%%%%%%%%%%%%%%%%%%%%%%%%%%%%%%%%%%%%%%%%%%%%%%%%%%%%%
% %%%%%%%%%%%%%%%%%%%%%%%%%%%%%%%%%%%%%%%%%%%%%%%%%%%%%%%%%%%%%%%%%%%%%%%%%%%%%%
% \section{Sample}
%\iffalse
%<*samplemain>
%\fi
%
% The following presents a sample document
% with two chapters, two parts, a title page,
% a compile flag as well as three forwarding files to set the flag.
% It consists of eight |.tex| files:
% \begin{center}
% \begin{tabular}{ll}
% |cdocsamp.tex|&main file\\
% |cdocsch1.tex|&include file for chapter 1\\
% |cdocsch2.tex|&include file for chapter 2\\
% |cdocspt3.tex|&include file for part 3\\
% |cdocspt4.tex|&include file for part 4\\
% |cdocsdrf.tex|&forwarding file for main file in draft mode\\
% |cdocsfi1.tex|&forwarding file for final version of chapter 1\\
% |cdocsfi2.tex|&forwarding file for final version of chapter 2\\
% \end{tabular}
% \end{center}
% Each of the eight files can be compiled directly by the \LaTeX{} compiler.
%
% %%%%%%%%%%%%%%%%%%%%%%%%%%%%%%%%%%%%%%
% \paragraph{Main File.}
%
% The main file is called |cdocsamp.tex|.
%
% Load the \textsf{childdoc} definitions and
% declare the filename for the main document:
%    \begin{macrocode}
\input{childdoc.def}
\childdocmain{}
%    \end{macrocode}

% Optional override for |\version| flag:
%    \begin{macrocode}
%%\ifchilddoc\else\providecommand{\version}{draft}\fi
%    \end{macrocode}

% Define the default values for the |\version| flag
% (|final| for the main file and |draft| for childs):
%    \begin{macrocode}
\ifchilddoc
\providecommand{\version}{draft}
\else
\providecommand{\version}{final}
\fi
%    \end{macrocode}

% Load the standard document class:
%    \begin{macrocode}
\documentclass[12pt]{article}
%    \end{macrocode}

% Start the document body:
%    \begin{macrocode}
\begin{document}
%    \end{macrocode}

% Declare a title page.
% Print title, part of document being processed and version flag:
%    \begin{macrocode}
\addtocounter{page}{-1}
\begin{center}
{\LARGE\bfseries{}childdoc example\par}
\vspace{1cm}
\ifchilddoc
\ifchilddocmanual part\else chapter\fi:
`\childdocname' of `\childdocjob'\par
\else
main document: `\childdocjob'\par
\fi
version: \version\par
\end{center}
\newpage
%    \end{macrocode}

% Manually include selected file,
% otherwise process as usual:
%    \begin{macrocode}
\ifchilddocmanual
\section*{part `\childdocname'}
\input{\childdocname}
\else
%    \end{macrocode}

% Include the two chapters:
%    \begin{macrocode}
\include{cdocsch1}
\include{cdocsch2}
%    \end{macrocode}

% Include the two parts unless only chapters should be displayed:
%    \begin{macrocode}
\ifchilddoc\else
\section{part three}
\input{cdocspt3}
\section{part four}
\input{cdocspt4}
\fi
%    \end{macrocode}

% Process as usual until here:
%    \begin{macrocode}
\fi
%    \end{macrocode}

% End of document body:
%    \begin{macrocode}
\end{document}
%    \end{macrocode}
%\iffalse
%</samplemain>
%\fi
%
% %%%%%%%%%%%%%%%%%%%%%%%%%%%%%%%%%%%%%%
% \paragraph{Chapter Include Files.}
%
% The include files are called |cdocsch1.tex| and |cdocsch2.tex|.
%
%\iffalse
%<*samplechap1|samplechap2>
%\fi

% Optional override for |\version| flag:
%    \begin{macrocode}
%%\providecommand{\version}{final}
%    \end{macrocode}

% Include the main document:
%    \begin{macrocode}
\input{childdoc.def}
\childdocof{cdocsamp}
%    \end{macrocode}

%\iffalse
%</samplechap1|samplechap2>
%\fi
%
%\iffalse
%<*samplechap1>
%\fi
% Some text for chapter 1:
%    \begin{macrocode}
\section{one}
some text in chapter one
%    \end{macrocode}

%\iffalse
%</samplechap1>
%\fi
% Some text for chapter 2:
%\iffalse
%<*samplechap2>
%\fi
%    \begin{macrocode}
\section{two}
more text in chapter two
%    \end{macrocode}

%\iffalse
%</samplechap2>
%\fi
%
% %%%%%%%%%%%%%%%%%%%%%%%%%%%%%%%%%%%%%%
% \paragraph{Part Include Files.}
%
% The include files are called |cdocspt3.tex| and |cdocspt4.tex|.
%
%\iffalse
%<*samplepart3|samplepart4>
%\fi

% Optional override for |\version| flag:
%    \begin{macrocode}
%%\providecommand{\version}{final}
%    \end{macrocode}

% Include the main document:
%    \begin{macrocode}
\input{childdoc.def}
\childdocby{cdocsamp}
%    \end{macrocode}

%\iffalse
%</samplepart3|samplepart4>
%\fi
%
%\iffalse
%<*samplepart3>
%\fi
% Some text for part 3:
%    \begin{macrocode}
some text in part three
%    \end{macrocode}

%\iffalse
%</samplepart3>
%\fi
% Some text for part 4:
%\iffalse
%<*samplepart4>
%\fi
%    \begin{macrocode}
more text in part four
%    \end{macrocode}

%\iffalse
%</samplepart4>
%\fi
%
% %%%%%%%%%%%%%%%%%%%%%%%%%%%%%%%%%%%%%%
% \paragraph{Forwarding for a Complete Draft.}
%
% The following forwarding file |cdocsdrf.tex|
% compiles the main document in draft mode:
%\iffalse
%<*sampledraft>
%\fi
%    \begin{macrocode}
\def\version{draft}
\input{childdoc.def}
\childdocforward{cdocsamp}
%    \end{macrocode}

%\iffalse
%</sampledraft>
%\fi
%
% %%%%%%%%%%%%%%%%%%%%%%%%%%%%%%%%%%%%%%
% \paragraph{Forwarding for Final Version of the Chapters.}
%
% The following forwarding files |cdocsfn1.tex| and |cdocsfn2.tex|
% (with identical content)
% compile the final versions of the child documents
% |cdocsch1.tex| and |cdocsch2.tex|, respectively:
%\iffalse
%<*samplefinal>
%\fi
%    \begin{macrocode}
\def\version{final}
\input{childdoc.def}
\childdocforwardprefix[cdocsamp]{cdocsfn}{cdocsch}
%    \end{macrocode}

%\iffalse
%</samplefinal>
%\fi
%
% %%%%%%%%%%%%%%%%%%%%%%%%%%%%%%%%%%%%%%
% \paragraph{Command Line Processing.}
%
% The following three command lines generate the output files
% |cdocscld|, |cdocscl1| and |cdocscl2|
% which should be identical to
% |cdocsdrf|, |cdocsch1| and |cdocsfn2|, respectively:
% \begin{center}
% \begin{tabular}{l}
% |latex -jobname cdocscld \|\\
% |  "\def\version{draft}\input{childdoc.def}\childdocforward{cdocsamp}"|\\
% |latex -jobname cdocscl1 \|\\
% |  "\input{childdoc.def}\childdocforward[cdocsamp]{cdocsch1}"|\\
% |latex -jobname cdocscl2 \|\\
% |  "\def\version{final}\input{childdoc.def}\childdocforward{cdocsch2}"|
% \end{tabular}
% \end{center}
% Note that the trailing backslash on each first line
% merely continues the input to the second line
% (for convenient cut ant paste).
% Furthermore, the command |latex| can be replaced by any
% of its alternative versions such as |pdflatex|.
%
% %%%%%%%%%%%%%%%%%%%%%%%%%%%%%%%%%%%%%%%%%%%%%%%%%%%%%%%%%%%%%%%%%%%%%%%%%%%%%%
% %%%%%%%%%%%%%%%%%%%%%%%%%%%%%%%%%%%%%%%%%%%%%%%%%%%%%%%%%%%%%%%%%%%%%%%%%%%%%%
% \section{Implementation}
%\iffalse
%<*package>
%\fi
%
% This section describes the definitions file |childdoc.def|.

% The definitions cannot be loaded using |\usepackage| or |\RequirePackage|
% which has a mechanism to prevent loading a style file more than once.
% When loading the definitions by means of |\input|
% multiple instances have to be prevented manually:
%\iffalse
%This code needs to be before the `\ProvidesFile' directive
%which is defined at the beginning of this file.
%Therefore it is also placed there and commented out here.
%</package>
%<*discard>
%\fi
%    \begin{macrocode}
\ifdefined\childdocmain\endinput\fi
%    \end{macrocode}
%\iffalse
%</discard>
%<*package>
%\fi
%
% \macro{\ifchilddoc}
% \macro{\ifchilddocmanual}
% The conditional |\ifchilddoc| tells whether a
% child (true) or main (false) document is being compiled.
% The conditional |\ifchilddocmanual| tells whether
% the |\includeonly| mechanism is used (false) or
% the selection of child files must be performed manually (true).
% The definitions initialise to false:
%    \begin{macrocode}
\newif\ifchilddoc
\newif\ifchilddocmanual
%    \end{macrocode}

% \macro{\childdocname}
% \macro{\childdocjob}
% The macro |\childdocname| stores the name of the main document
% to be compiled. The macro |\childdocjob| stores the name of
% the document on which the \LaTeX{} compiler was originally invoked.
% The content of |\jobname| cannot be compared
% to filenames specified in the source due to different catcodes.
% The following code rescans |\jobname|, stores the result
% in |\childdocname| and saves a copy in |\childdocjob|:
%    \begin{macrocode}
\edef\childdocname{\scantokens\expandafter{\jobname\noexpand}}
\let\childdocjob\childdocname
%    \end{macrocode}

% \macro{\childdocdisable}
% The macro |\childdocdisable| prevents the main file
% from being processed more than once.
% At this stage, the main document command |\childdocmain|
% is assumed to be called once again where it should do nothing.
% Any subsequent call to it should prevent
% a secondary processing of the main document
% It overwrites the forwarding commands
% |\childdocof| and |\childdocforward|
% with empty macros to prevent further inclusions of the main document:
%    \begin{macrocode}
\newcommand{\childdocdisable}
{
  \renewcommand{\childdocmain}[1]{\renewcommand{\childdocmain}[1]{\endinput}}
  \renewcommand{\childdocof}[1]{}
  \renewcommand{\childdocby}[2][]{}
  \renewcommand{\childdocforward}[2][]{}
  \renewcommand{\childdocdisable}{}
}
%    \end{macrocode}

% \macro{\childdocmain}
% The macro |\childdocmain| is to be called at the top of the main file
% with nothing or the main filename (without extension) as argument.
% First, it breaks loops.
% If the argument is not empty and does not match |\childdocname|
% (which is set by the first inclusion of |childdoc.def|),
% |\ifchilddoc| is set to true, |\includeonly| is applied to the child file
% and |\jobname| is set to the main file
% (for proper handling of |.aux| files):
%    \begin{macrocode}
\newcommand{\childdocmain}[1]
{
  \childdocdisable\childdocmain{}
  \if?#1?\else
    \begingroup
      \def\childdoctmp{#1}
      \ifx\childdoctmp\childdocname
        \def\childdoctmp{}
      \else
        \def\childdoctmp
        {
          \childdoctrue
          \includeonly{\childdocname}
          \def\childdocjob{#1}
          \def\jobname{#1}
        }
      \fi
      \expandafter
    \endgroup
    \childdoctmp
  \fi
}
%    \end{macrocode}

% \macro{\childdocof}
% The command |\childdocof| redirects
% compilation to the main file |#1|.
%    \begin{macrocode}
\newcommand{\childdocof}[1]
{
  \childdocdisable
  \childdoctrue
  \includeonly{\childdocname}
  \def\jobname{#1}
  \def\childdocjob{#1}
  \input{#1}
}
%    \end{macrocode}

% \macro{\childdocby}
% The command |\childdocby| ....
%    \begin{macrocode}
\newcommand{\childdocby}[2][]
{
  \childdocdisable
  \childdoctrue
  \childdocmanualtrue
  \if?#1?\else
    \def\jobname{#2}
  \fi
  \def\childdocjob{#2}
  \input{#2}
  \endinput
}
%    \end{macrocode}

% \macro{\childdocforward}
% The command |\childdocforward| redirects
% compilation to the main file or
% (if the optional argument is given) a child file.
% Parameters are set as if the main file
% or a child file starting with |\childdocof| was compiled.
% Then compilation is handed over to the main file:
%    \begin{macrocode}
\newcommand{\childdocforward}[2][]
{
  \begingroup
    \if?#1?
      \def\childdoctmp
      {
        \def\childdocname{#2}
        \def\childdocjob{#2}
        \def\jobname{#2}
        \input{#2}
        \endinput
      }
    \else
      \def\childdoctmp
      {
        \childdocdisable
        \def\childdocname{#2}
        \childdoctrue
        \includeonly{#2}
        \def\childdocjob{#1}
        \def\jobname{#1}
        \input{#1}
        \endinput
      }
    \fi
    \expandafter
  \endgroup
  \childdoctmp
}
%    \end{macrocode}

% \macro{\childdocforwardprefix}
% The command |\childdocforwardprefix| redirects
% compilation to the main or a child file by means of a pattern.
% The prefix |#1| in the current filename is replaced by |#2|
% and the suffix of the current filename is kept
% (it is assumed that the filename does not contain the substring `|~~~|'
% which is used as a delimiter).
% Compilation is handed over to the new file by |\childdocforward|:
%    \begin{macrocode}
\newcommand{\childdocforwardprefix}[3][]
{
  \begingroup
    \def\childdocextract #2##1~~~{\def\childdoctmp{\childdocforward[#1]{#3##1}}}
    \expandafter\childdocextract\childdocname~~~
    \expandafter
  \endgroup
  \childdoctmp
}
%    \end{macrocode}

% \macro{\childdoc}
% The deprecated macro |\childdoc| is a legacy version of |\childdocmain|:
%    \begin{macrocode}
\newcommand{\childdoc}{\childdocmain}
%    \end{macrocode}

% \macro{\childdocredirect}
% The deprecated macro |\childdocredirect| is a legacy version
% of |\childdocforward| and |\childdocforwardprefix|:
%    \begin{macrocode}
\newcommand{\childdocredirect}[2][]
{
  \begingroup
    \if?#1?
      \def\childdoctmp{\childdocforward{#2}}
    \else
      \def\childdoctmp{\childdocforwardprefix{#1}{#2}}
    \fi
    \expandafter
  \endgroup
  \childdoctmp
}
%    \end{macrocode}

%\iffalse
%</package>
%\fi
%
\endinput

\childdocby{cdocsamp}
%    \end{macrocode}

%\iffalse
%</samplepart3|samplepart4>
%\fi
%
%\iffalse
%<*samplepart3>
%\fi
% Some text for part 3:
%    \begin{macrocode}
some text in part three
%    \end{macrocode}

%\iffalse
%</samplepart3>
%\fi
% Some text for part 4:
%\iffalse
%<*samplepart4>
%\fi
%    \begin{macrocode}
more text in part four
%    \end{macrocode}

%\iffalse
%</samplepart4>
%\fi
%
% %%%%%%%%%%%%%%%%%%%%%%%%%%%%%%%%%%%%%%
% \paragraph{Forwarding for a Complete Draft.}
%
% The following forwarding file |cdocsdrf.tex|
% compiles the main document in draft mode:
%\iffalse
%<*sampledraft>
%\fi
%    \begin{macrocode}
\def\version{draft}
% \iffalse
%
% childdoc.dtx Copyright (C) 2017-2018 Niklas Beisert
%
% This work may be distributed and/or modified under the
% conditions of the LaTeX Project Public License, either version 1.3
% of this license or (at your option) any later version.
% The latest version of this license is in
%   http://www.latex-project.org/lppl.txt
% and version 1.3 or later is part of all distributions of LaTeX
% version 2005/12/01 or later.
%
% This work has the LPPL maintenance status `maintained'.
%
% The Current Maintainer of this work is Niklas Beisert.
%
% This work consists of the files childdoc.dtx and childdoc.ins
% and the derived files childdoc.def and cdocsamp.tex with
% cdocsch1.tex, cdocsch2.tex, cdocsdrf.tex, cdocsfn1.tex, cdocsfn2.tex.
%
%<package>\ifdefined\childdocmain\endinput\fi
%<package>\ProvidesFile{childdoc.def}[2018/12/30 v2.0 child document driver]
%<samplemain>\ProvidesFile{cdocsamp.tex}[2018/12/30 v2.0 sample for childdoc]
%<*driver>
%\ProvidesFile{childdoc.drv}[2018/12/30 v2.0 childdoc reference manual file]
\PassOptionsToClass{10pt,a4paper}{article}
\documentclass{ltxdoc}

\usepackage[margin=35mm]{geometry}
\usepackage{hyperref}
\usepackage{hyperxmp}
\usepackage[usenames]{color}

\hypersetup{colorlinks=true}
\hypersetup{pdfstartview=FitH}
\hypersetup{pdfpagemode=UseNone}
\hypersetup{pdfsource={}}
\hypersetup{pdflang={en-UK}}
\hypersetup{pdfcopyright={Copyright 2017-2018 Niklas Beisert.
  This work may be distributed and/or modified under the
  conditions of the LaTeX Project Public License, either version 1.3
  of this license or (at your option) any later version.}}
\hypersetup{pdflicenseurl={http://www.latex-project.org/lppl.txt}}
\hypersetup{pdfcontactaddress={ETH Zurich, ITP, HIT K,
  Wolfgang-Pauli-Strasse 27}}
\hypersetup{pdfcontactpostcode={8093}}
\hypersetup{pdfcontactcity={Zurich}}
\hypersetup{pdfcontactcountry={Switzerland}}
\hypersetup{pdfcontactemail={nbeisert@itp.phys.ethz.ch}}
\hypersetup{pdfcontacturl={http://people.phys.ethz.ch/\xmptilde nbeisert/}}

\newcommand{\secref}[1]{\hyperref[#1]{section \ref*{#1}}}

\parskip1ex
\parindent0pt
\let\olditemize\itemize
\def\itemize{\olditemize\parskip0pt}

\begin{document}

\title{The \textsf{childdoc} Package}
\hypersetup{pdftitle={The childdoc Package}}
\author{Niklas Beisert\\[2ex]
  Institut f\"ur Theoretische Physik\\
  Eidgen\"ossische Technische Hochschule Z\"urich\\
  Wolfgang-Pauli-Strasse 27, 8093 Z\"urich, Switzerland\\[1ex]
  \href{mailto:nbeisert@itp.phys.ethz.ch}
  {\texttt{nbeisert@itp.phys.ethz.ch}}}
\hypersetup{pdfauthor={Niklas Beisert}}
\hypersetup{pdfsubject={Manual for the LaTeX2e Package childdoc}}
\date{30 December 2018, \textsf{v2.0}}
\maketitle

\begin{abstract}\noindent
\textsf{childdoc} is a \LaTeXe{} package
that enables the direct compilation
of document sections included by |\include|
to individual files.
\end{abstract}

\begingroup
\parskip0ex
\tableofcontents
\endgroup

%%%%%%%%%%%%%%%%%%%%%%%%%%%%%%%%%%%%%%%%%%%%%%%%%%%%%%%%%%%%%%%%%%%%%%%%%%%%%%%%
%%%%%%%%%%%%%%%%%%%%%%%%%%%%%%%%%%%%%%%%%%%%%%%%%%%%%%%%%%%%%%%%%%%%%%%%%%%%%%%%
\section{Introduction}

\LaTeX{} provides a mechanism to structure a large document (such as a book)
into a main file and several child files (containing the chapters)
using the |\include| command.
This mechanism is beneficial for documents
which span hundreds of pages in order to
make the source file(s) more manageable.
Moreover, compilation can be restricted to
selected child files by means of the |\includeonly| command.
The latter feature can be used to reduce the compilation time while editing
(this was significantly more useful in the earlier days of \LaTeX{})
or to generate a smaller document which is easier to navigate.
Another application of |\includeonly| is to generate
documents consisting of selected parts of the complete document.

However, there are a few drawbacks of the plain |\include| mechanism:
\begin{itemize}
\item
The child files cannot be compiled on their own,
they can only be compiled via the main file.
A naive editing environment
(such as a text editor with an option
to have the current file processed by \LaTeX)
may require one to switch to the main file before compiling;
attempting to compile the child file produces errors.
\item
The main file must be modified (each time)
to adjust the |\includeonly| command
to the present needs. This easily leaves the main file in a messy state.
\item
The generated document will always carry the filename
of the main document. This is inconvenient if
several child files are to be compiled and
to be kept for distribution.
\end{itemize}

The present package provides a simple interface
to make child files individually compilable by \LaTeX{}.
Compiling a child file then has the same effect as compiling
the main file with an |\includeonly| command
to select the appropriate child.
Moreover the generated document will carry the name of the child
rather than the main file.
This resolves all three above issues.

This feature is meant to make the editing of books,
thesis documents and lecture notes somewhat more convenient.
However, the package can also be used efficiently for
composing a series of documents (such as exercise sheets)
which are typically distributed individually.
It then assists the author in generating the individual documents
(potentially in different versions)
as well as a document containing the collected series.
Another application is in developing style files
or other kinds of included material
where compilation of the style file could redirect
to a sample or test file.

%%%%%%%%%%%%%%%%%%%%%%%%%%%%%%%%%%%%%%%%%%%%%%%%%%%%%%%%%%%%%%%%%%%%%%%%%%%%%%%%
%%%%%%%%%%%%%%%%%%%%%%%%%%%%%%%%%%%%%%%%%%%%%%%%%%%%%%%%%%%%%%%%%%%%%%%%%%%%%%%%
\section{Usage}

First of all, the package \textsf{childdoc} is \emph{not} a standard
\LaTeXe{} |.sty| style file! Therefore it needs to be invoked in
a non-standard way.

%%%%%%%%%%%%%%%%%%%%%%%%%%%%%%%%%%%%%%%%%%%%%%%%%%%%%%%%%%%%%%%%%%%%%%%%%%%%%%%%
\subsection{Included Files}
\label{sec:include}

%%%%%%%%%%%%%%%%%%%%%%%%%%%%%%%%%%%%%%%%
\DescribeMacro{\childdocmain}
To use the package, add the commands
\begin{center}
\begin{tabular}{l}
|\input{childdoc.def}|\\
|\childdocmain{}|\\
\end{tabular}
\end{center}
at the very top of the main \LaTeX{} file,
in particular \emph{before} the |\documentclass| statement!
The argument of |\childdocmain| should be left empty
(but it must be present).

%%%%%%%%%%%%%%%%%%%%%%%%%%%%%%%%%%%%%%%%
\DescribeMacro{\childdocof}
Furthermore, add the commands
\begin{center}
\begin{tabular}{l}
|\input{childdoc.def}|\\
|\childdocof{|\textit{main}|}|\\
\end{tabular}
\end{center}
at the top of every child file \textit{child}
which is included by |\include{|\textit{child}|}|
from within the main file
(or at least for those files to be compiled individually).
The argument \textit{main} must be the filename of the main file.

There are a couple of
considerations in setting up the main and child documents:

%%%%%%%%%%%%%%%%%%%%%%%%%%%%%%%%%%%%%%%%
\paragraph{Restrictions.}

Please note the following restrictions:
\begin{itemize}
\item
|\childdocmain| must be called with one argument \textit{main}
to ensure compatibility with earlier version of the package.
It must either be empty (|\childdocmain{}|)
or precisely match the filename of the main file in which it is specified.
See \secref{sec:detection} for further information.
\item
The filename \textit{main} must be specified without the |.tex| extension.
\item
The filename \textit{main} is case sensitive
(even in case-insensitive file systems)
due to internal string comparison.
\item
The argument \textit{main} should be fully expanded, it cannot be a macro.
\item
Subdirectories and special characters should be avoided in filenames.
\item
The command |\childdocmain{|\textit{main}|}| must be followed by a whitespace.
It should not be followed immediately by another command
or by a comment mark `|%|'.
This is because the \TeX{} parser reads the token immediately following
the argument of |\childdocmain| and puts it
at the beginning of every child section;
however, a white\-space is ignored.
\end{itemize}

%%%%%%%%%%%%%%%%%%%%%%%%%%%%%%%%%%%%%%%%
\paragraph{Content of Main File.}

It is advisable to place all content in the child files included by |\include|.
Any output contained in the main file will appear in all child documents
unless suppressed manually;
it cannot be suppressed automatically by the |\includeonly| directive
and thus should normally be avoided.
A method to include some content in the main file
by means of conditional processing is described in \secref{sec:conditional}.

%%%%%%%%%%%%%%%%%%%%%%%%%%%%%%%%%%%%%%%%
\paragraph{Page Numbering.}

When only a part of the document is compiled,
the appropriate numbering of pages
(as well as other status parameters)
is determined from the |.aux| files.
The latter contain information from previous passes.
However this information needs to propagate through
all intermediate child documents.
Therefore the page numbering in child documents may well
be inconsistent until the complete document is compiled at least once.

A useful (if unconventional) way to always ensure a consistent
page numbering is to restart the numbering in each child document
and denote the pages by `\textit{child}|.|\textit{page}'
where \textit{child} represents the chapter/section number of the child file.
This can be achieved by the command
|\numberwithin{page}{|\textit{child}|}|
of the \textsf{amsmath} package
where \textit{child} can be |chapter| or |section|
depending on the chosen structuring.
Alternatively, one can modify the macro |\thepage| appropriately
and reset the counter |page| at the start of each child file.

%%%%%%%%%%%%%%%%%%%%%%%%%%%%%%%%%%%%%%%%%%%%%%%%%%%%%%%%%%%%%%%%%%%%%%%%%%%%%%%%
\subsection{Conditional Processing}
\label{sec:conditional}

The package provides a mechanism to compile different versions
of a document. To customise the versions further some conditional processing
can come in handy to distinguish which version is being compiled.
The package provides two macros to describe the compilation context:

%%%%%%%%%%%%%%%%%%%%%%%%%%%%%%%%%%%%%%%%
\DescribeMacro{\ifchilddoc}
The conditional |\ifchilddoc| distinguishes between the compilation of
child documents and the main document:
%
\begin{center}
|\ifchilddoc |\textit{child-code}| |[|\||else |\textit{main-code}]| \||fi|
\end{center}

%%%%%%%%%%%%%%%%%%%%%%%%%%%%%%%%%%%%%%%%
\DescribeMacro{\childdocname}
\DescribeMacro{\childdocjob}
The macro |\childdocname| contains the filename (without extension)
of the main or child file being processed.
Note that |\childdocjob| will always contain the name of the main file.

%%%%%%%%%%%%%%%%%%%%%%%%%%%%%%%%%%%%%%%%
\paragraph{Title Page.}

Conditional processing can be used to include a title or banner page
in the main document when proper precautions are taken.
Importantly, the code in the main file should ensure that the page counter
(as well as other status parameters which are stored in the |.aux| files)
takes the same value after the conditional processing.
Otherwise the page numbers may take divergent values
depending on which part is compiled.

For example, a title page could be declared by:
%
\begin{center}
\begin{tabular}{l}
|\ifchilddoc\||else|\\
|\addtocounter{page}{-1}|\\
\textit{code for title page}\\
|\newpage|\\
|\||fi|
\end{tabular}
\end{center}
%
A banner page for the child documents can be generated by:
%
\begin{center}
\begin{tabular}{l}
|\ifchilddoc|\\
|\addtocounter{page}{-1}|\\
\textit{code for banner page}\\
|\newpage|\\
|\||fi|
\end{tabular}
\end{center}
%
Here one could write a message such as:
\begin{center}
|This is the part \childdocname{} of \childdocjob{}.|
\end{center}

%%%%%%%%%%%%%%%%%%%%%%%%%%%%%%%%%%%%%%%%%%%%%%%%%%%%%%%%%%%%%%%%%%%%%%%%%%%%%%%%
\subsection{Flags}
\label{sec:flags}

The package makes it easy to generate different versions
of the main or child documents.
To this end compilation flags can be defined
and assigned different default values.
They will be particularly useful in conjunction
with the forwarding mechanism described in \secref{sec:forward}.

For example, it may be useful to have a flag |\version|
which can be set to |draft| or |final|.
The document source will contain some conditional code
depending on the value of |\version|.
Suppose further, the flag should default to |final| for the main file
and to |draft| for child files
which is a natural assignment for editing the document.
This is achieved by placing the following code
in the preamble of the main document
(below the |\childdocmain| directive):
%
\begin{center}
\begin{tabular}{l}
|\ifchilddoc|\\
|\providecommand{\version}{draft}|\\
|\||else|\\
|\providecommand{\version}{final}|\\
|\||fi|
\end{tabular}
\end{center}
%
The definition by |\providecommand| makes sure
that previous definitions are not overwritten.
Further statements |\providecommand{\version}{...}|
can thus be added before the above code to override it.

For the main file, one might add a line
(between |\childdocmain| and the above block)
%
\begin{center}
|%\ifchilddoc\||else\providecommand{\version}{draft}\||fi|
\end{center}
%
which can be uncommented to produce a draft version.
Likewise one can add a line to the very top of a child file
(above the |\childdocof{|\textit{main}|}| directive)
%
\begin{center}
|%\providecommand{\version}{final}|
\end{center}
%
which can be uncommented to produce the final version of this child document.

%%%%%%%%%%%%%%%%%%%%%%%%%%%%%%%%%%%%%%%%%%%%%%%%%%%%%%%%%%%%%%%%%%%%%%%%%%%%%%%%
\subsection{Forwarding}
\label{sec:forward}

Different versions of the main or child documents
using compilation flags as described in \secref{sec:flags}
can be (permanently) stored in different files
for convenient compilation, viewing and distribution.
To this end, the package defines a command
to pass on compilation to a different file:

%%%%%%%%%%%%%%%%%%%%%%%%%%%%%%%%%%%%%%%%
\DescribeMacro{\childdocforward}
The command |\childdocforward| redirects processing to
another source file:
%
\begin{center}
\begin{tabular}{l}
|\input{childdoc.def}|\\
|\childdocforward[|\textit{main}|]{|\textit{dest}|}|\\
\end{tabular}
\end{center}
%
The argument \textit{dest} is the destination file
(without extension).
It should be the main file or one of the child files.
Note that further \textsf{childdoc} directives
such as |\childdocof| and |\childdocforward|
in the indicated file will be processed in this form.
The optional argument \textit{main}
passes on directly to the main file \textit{main}
while pretending to compile the child \textit{dest}.
This form behaves as if \textit{dest}
issues |\childdocof{|\textit{main}|}| right away,
and no further \textsf{childdoc} directives will be processed.

%%%%%%%%%%%%%%%%%%%%%%%%%%%%%%%%%%%%%%%%
\DescribeMacro{\...prefix}
In the alternative form |\childdocforwardprefix|,
%
\begin{center}
\begin{tabular}{l}
|\input{childdoc.def}|\\
|\childdocforwardprefix[|\textit{main}|]{|\textit{prefix}|}{|\textit{dest}|}|
\end{tabular}
\end{center}
%
the destination file is determined by a pattern
depending on the current file:
To make this work, the current file must be called
`{\textit{prefix}\hspace{0.2em}\textit{suffix}}'
with \textit{prefix} matching precisely the argument.
Processing is then passed on to the file
`{\textit{dest}\hspace{0.2em}\textit{suffix}}'.
Surely, the same effect is achieved by
directly specifying the
argument `{\textit{dest}\hspace{0.2em}\textit{suffix}}'
in the first form.
However, that requires to set up a different file
for each child. With the alternative form of the command
all these files can have exactly the same content
which simplifies setting them up and maintaining them.

For example, the following file |draft.tex|
with a compilation flag |\version| as described in \secref{sec:flags}
compiles the main document as a draft:
%
\begin{center}
\begin{tabular}{l}
|\def\version{draft}|\\
|\input{childdoc.def}|\\
|\childdocforward{|\textit{main}|}|
\end{tabular}
\end{center}
%
Likewise, the following files |final|\textit{nn}|.tex|
compile the final version of the child document
|child|\textit{nn}|.tex|:
%
\begin{center}
\begin{tabular}{l}
|\def\version{final}|\\
|\input{childdoc.def}|\\
|\childdocforwardprefix{final}{child}|
\end{tabular}
\end{center}
%

Note that when several versions of a main file and/or of each child file
are to be generated, it may be convenient to set up a |Makefile| or
shell script to automatise the process.

%%%%%%%%%%%%%%%%%%%%%%%%%%%%%%%%%%%%%%%%%%%%%%%%%%%%%%%%%%%%%%%%%%%%%%%%%%%%%%%%
\subsection{Command Line Processing}
\label{sec:commandline}

The effect of redirection files can also be achieved by invoking
the \LaTeX{} compiler with a more elaborate command line.
Most conveniently this should be done as part
of a shell script or a |Makefile|.

When using \textsf{childdoc} in the main file, the following
command lines effectively perform a redirection
(note that depending on the shell being used,
backslashes may have to be doubled: `|\|' $\to$ `|\\|'):
%
\begin{center}
|... -jobname "|\textit{target}|" |\\|"|[\textit{flags}]%
|\input{childdoc.def}\childdocforward[|\textit{main}|]{|\textit{dest}|}"|
\end{center}
%
Here \textit{target} is the name of the output file,
\textit{main} is the name of the main file
and \textit{dest} is the name of the main or child file to be processed
(all filenames without extensions).
The optional argument \textit{main} can be omitted
if \textit{main} matches \textit{dest}.
Optionally, compilation \textit{flags} can be defined via |\def| commands.
This command line makes the \TeX{} engine believe
it is compiling the file \textit{target}
whose content is specified as the latter parameter.
The provided code then forwards the processing to
\textit{main} or \textit{dest} as described in \secref{sec:forward}.

%%%%%%%%%%%%%%%%%%%%%%%%%%%%%%%%%%%%%%%%%%%%%%%%%%%%%%%%%%%%%%%%%%%%%%%%%%%%%%%%
\subsection{Include by Input}
\label{sec:input}

Including child documents by |\include| has some restrictions by design.
Most notably, the content of a child document always occupies
its own set of pages; pages cannot be shared between child documents.
Usually, this behaviour makes perfect sense
because each child document contain an essential part of the document.
However, in some situations it may be desirable to compose
a document from a collection of parts
without having mandatory page breaks between then.
For this case, the package
provides a mechanism to include parts
by |\input| which can also be processed individually.
However, by construction this mechanism
requires manual handling of the content to be output.

%%%%%%%%%%%%%%%%%%%%%%%%%%%%%%%%%%%%%%%%
\DescribeMacro{\ifchilddocmanual}
The main file should be prepared as usual, see \secref{sec:include}.
However, the document body must make a distinction
between processing of an individual part and of the main document, e.g.:
%
\begin{center}
\begin{tabular}{l}
|\ifchilddocmanual|\\
|\input{\childdocname}|\\
|\||else|\\
\textit{document body with }|\input{|\textit{part}|}|\\
|\||fi|
\end{tabular}
\end{center}
%
The conditional |\ifchilddocmanual| is true whenever
a part to be included by |\input| is being compiled,
and the name of the part is stored in |\childdocname|.

%%%%%%%%%%%%%%%%%%%%%%%%%%%%%%%%%%%%%%%%
\DescribeMacro{\childdocby}
Each part to be included by |\input| should start with:
%
\begin{center}
\begin{tabular}{l}
|\input{childdoc.def}|\\
|\childdocby{|\textit{main}|}|\\
\end{tabular}
\end{center}
%
The directive |\childdocby| is similar to |\childdocof|
described in \secref{sec:include},
but the subsequent selection of content must be done manually.
To that end, both |\ifchilddoc| and |\ifchilddocmanual|
will be true upon processing of a part,
and the name of the part is stored in |\childdocname|.
Note that |\jobname| will be set to the filename of the current part
so that each part receives an individual |.aux| file
that does not interfere with the |.aux| file(s) of the main document.
This behaviour can be altered by the alternative form
|\childdocby[*]{|\textit{main}|}| (with a non-empty optional argument)
which uses the |.aux| file of the main document
by setting |\jobname| to \textit{main}.

%%%%%%%%%%%%%%%%%%%%%%%%%%%%%%%%%%%%%%%%%%%%%%%%%%%%%%%%%%%%%%%%%%%%%%%%%%%%%%%%
\subsection{Driver Development}
\label{sec:driver}

The \textsf{childdoc} mechanism can also be use for the development
of definition files such as \LaTeX{} styles or classes.
This case differs from the above setup with multiple parts
included by |\include| in that no |\includeonly| should be invoked.
This can be achieved by starting the include file
(before |\ProvidesPackage|) with:
%
\begin{center}
\begin{tabular}{l}
|\input{childdoc.def}|\\
|\childdocforward{|\textit{main}|}|\\
\end{tabular}
\end{center}
%
or alternatively with:
%
\begin{center}
\begin{tabular}{l}
|\input{childdoc.def}|\\
|\childdocby{|\textit{main}|}|\\
\end{tabular}
\end{center}
%
Both forms have slightly different effects as described above.
The main file is prepared as usual, see \secref{sec:include}.

%%%%%%%%%%%%%%%%%%%%%%%%%%%%%%%%%%%%%%%%%%%%%%%%%%%%%%%%%%%%%%%%%%%%%%%%%%%%%%%%
\subsection{Legacy Detection}
\label{sec:detection}

The directive |\childdocmain| in the main file can detect
whether the complete document or merely a child is to be compiled
even without using the directive |\childdocof|.
This method is deprecated because it is less robust
and there is no compelling reason to use it;
it is merely provided for backward compatibility
and it may be removed in future versions.

If the detection mechanism is to be used,
it is mandatory to correctly specify
the filename of the main file as the argument of |\childdocmain|:
%
\begin{center}
\begin{tabular}{l}
|\input{childdoc.def}|\\
|\childdocmain{|\textit{main}|}|\\
\end{tabular}
\end{center}
%
If |\jobname| does not match the argument \textit{main} of |\childdocmain|,
it is assumed that |\jobname| points to the child file to be compiled.
When using |\childdocmain| with the main file specified as argument,
it suffices to start a child file
with just |\input{|\textit{main}|}|
without loading of the package and using |\childdocof|.
If instead all processing is done
with the appropriate \textsf{childdoc} directives,
the argument of \textit{main} of |\childdocmain| can be empty.

An alternative version of the command line processing described
in \secref{sec:commandline} using the detection mechanism reads:
%
\begin{center}
|... -jobname "|\textit{target}|" "|[\textit{flags}]%
[|\def\jobname{|\textit{dest}|}|]|\input{|\textit{main}|}"|
\end{center}

%%%%%%%%%%%%%%%%%%%%%%%%%%%%%%%%%%%%%%%%%%%%%%%%%%%%%%%%%%%%%%%%%%%%%%%%%%%%%%%%
\subsection{Manual Code}
\label{sec:manual}

In case one cannot be certain whether the definitions file |childdoc.def|
is installed on the target \TeX{} distribution
and one prefers not to ship it,
it is conceivable to paste a few relevant commands into the sources.

To that end, drop all statements |\input{childdoc.def}|
and perform the replacements as outlined below.
Instead of |\childdocmain{|\textit{main}|}| add the following code
to the top of the main file:
%
\begin{center}
\begin{tabular}{l}
|\||ifdefined\childdocname\endinput\||fi\newif\ifchilddoc|\\
|\edef\childdocname{\scantokens\expandafter{\jobname\noexpand}}|\\
|\def\childdocmain{|\textit{main}|}\||ifx\childdocmain\childdocname\||else|\\
|\childdoctrue\includeonly{\childdocname}\let\jobname\childdocmain\||fi|\\
\end{tabular}
\end{center}
%
Instead of |\childdocof{|\textit{main}|}| just include the main file
at the top of each child file:
%
\begin{center}
|\input{|\textit{main}|}|
\end{center}
%
A simple redirection |\childdocforward{|\textit{dest}|}| is achieved by:
%
\begin{center}
|\def\jobname{|\textit{dest}|}\input{\jobname}|
\end{center}
%
The redirection with prefix
|\childdocforwardprefix[|\textit{prefix}|]{|\textit{dest}|}|
is accomplished by:
%
\begin{center}
\begin{tabular}{l}
|{\edef\jobname{\scantokens\expandafter{\jobname\noexpand}}|\\
|\def\redirectjob |\textit{prefix}|#1~~~{\gdef\jobname{|\textit{dest}|#1}}|\\
|\expandafter\redirectjob\jobname~~~}\input{\jobname}|
\end{tabular}
\end{center}

In an alternative approach,
child documents can be compiled by a specific command line
without additional code or specific definitions:
%
\begin{center}
|... -jobname "|\textit{target}|" "|[\textit{flags}]%
|\includeonly{|\textit{dest}|}\input{|\textit{main}|}"|
\end{center}
%

%%%%%%%%%%%%%%%%%%%%%%%%%%%%%%%%%%%%%%%%%%%%%%%%%%%%%%%%%%%%%%%%%%%%%%%%%%%%%%%%
%%%%%%%%%%%%%%%%%%%%%%%%%%%%%%%%%%%%%%%%%%%%%%%%%%%%%%%%%%%%%%%%%%%%%%%%%%%%%%%%
\section{Information}

%%%%%%%%%%%%%%%%%%%%%%%%%%%%%%%%%%%%%%%%%%%%%%%%%%%%%%%%%%%%%%%%%%%%%%%%%%%%%%%%
\subsection{Copyright}

Copyright \copyright{} 2017--2018 Niklas Beisert

This work may be distributed and/or modified under the
conditions of the \LaTeX{} Project Public License, either version 1.3
of this license or (at your option) any later version.
The latest version of this license is in
  \url{http://www.latex-project.org/lppl.txt}
and version 1.3 or later is part of all distributions of \LaTeX{}
version 2005/12/01 or later.

This work has the LPPL maintenance status `maintained'.

The Current Maintainer of this work is Niklas Beisert.

This work consists of the files |README.txt|, |childdoc.ins| and |childdoc.dtx|
as well as the derived files |childdoc.def|, |cdocsamp.tex|
with |cdocsch1.tex|, |cdocsch2.tex|, |cdocspt3.tex|, |cdocspt4.tex|,
|cdocsdrf.tex|, |cdocsfn1.tex|, |cdocsfn2.tex|
as well as |childdoc.pdf|.

%%%%%%%%%%%%%%%%%%%%%%%%%%%%%%%%%%%%%%%%%%%%%%%%%%%%%%%%%%%%%%%%%%%%%%%%%%%%%%%%
\subsection{Files and Installation}

The package consists of the files:
%
\begin{center}
\begin{tabular}{ll}
    |README.txt|   & readme file \\
    |childdoc.ins| & installation file \\
    |childdoc.dtx| & source file \\
    |childdoc.def| & definition file \\
    |cdocsamp.tex| & sample main file \\
    |cdocsch1.tex| & sample include file \\
    |cdocsch2.tex| & sample include file \\
    |cdocspt3.tex| & sample part file \\
    |cdocspt4.tex| & sample part file \\
    |cdocsdrf.tex| & sample redirection file \\
    |cdocsfn1.tex| & sample redirection file \\
    |cdocsfn2.tex| & sample redirection file \\
    |childdoc.pdf| & manual
\end{tabular}
\end{center}
%
The distribution consists of the files
|README.txt|, |childdoc.ins| and |childdoc.dtx|.
%
\begin{itemize}
\item
Run (pdf)\LaTeX{} on |childdoc.dtx|
to compile the manual |childdoc.pdf| (this file).
\item
Run \LaTeX{} on |childdoc.ins| to create the definitions file |childdoc.def|
and the sample |cdocsamp.tex| with include files
|cdocsch1.tex|, |cdocsch2.tex|, |cdocspt3.tex|, |cdocspt4.tex|,
|cdocsdrf.tex|, |cdocsfn1.tex|, |cdocsfn2.tex|.
Then copy the file |childdoc.def| to an appropriate directory of your \LaTeX{}
distribution, e.g.\ \textit{texmf-root}|/tex/latex/childdoc|.
\end{itemize}

%%%%%%%%%%%%%%%%%%%%%%%%%%%%%%%%%%%%%%%%%%%%%%%%%%%%%%%%%%%%%%%%%%%%%%%%%%%%%%%%
\subsection{Related CTAN Packages}

There are several other packages which offer a similar functionality:
%
\begin{itemize}
\item
The packages
\href{http://ctan.org/pkg/docmute}{\textsf{docmute}},
\href{http://ctan.org/pkg/includex}{\textsf{includex}} and
\href{http://ctan.org/pkg/standalone}{\textsf{standalone}}
provide commands to include only the document body of
a child file thus allowing both files to be compiled individually.
\item
The packages \href{http://ctan.org/pkg/subdocs}{\textsf{subdocs}}
and \href{http://ctan.org/pkg/subfiles}{\textsf{subfiles}}
provide structures in which the main and child documents can be
encapsulated and allowing them to be compiled individually.
The inclusion mechanism is different from the conventional |\include|.
\item
The package \href{http://ctan.org/pkg/combine}{\textsf{combine}}
is an elaborate solution to combine several documents into one.
\end{itemize}
%
See also the CTAN topic \href{http://ctan.org/topic/subdocs}{\textsf{subdocs}}
for further related packages.
The present package differs from the above solutions in that
a document structure constructed with the conventional |\include| mechanism
just needs two extra commands at the top of every file
such that all constituent files can be compiled individually.

%%%%%%%%%%%%%%%%%%%%%%%%%%%%%%%%%%%%%%%%%%%%%%%%%%%%%%%%%%%%%%%%%%%%%%%%%%%%%%%%
%\subsection{Feature Suggestions}
%
%The following is a list of features which may be useful for future
%versions of this package:
%%
%\begin{itemize}
%\item
%\ldots
%\end{itemize}

%%%%%%%%%%%%%%%%%%%%%%%%%%%%%%%%%%%%%%%%%%%%%%%%%%%%%%%%%%%%%%%%%%%%%%%%%%%%%%%%
\subsection{Revision History}

%%%%%%%%%%%%%%%%%%%%%%%%%%%%%%%%%%%%%%%%
\paragraph{v2.0:} 2018/12/30

\begin{itemize}
\item
immediate forward processing
\item
added |\childdocby| mechanism
\item
manual restructured
\end{itemize}

%%%%%%%%%%%%%%%%%%%%%%%%%%%%%%%%%%%%%%%%
\paragraph{v1.6:} 2018/01/17

\begin{itemize}
\item
application for development of include files
\item
corrections to manual
\end{itemize}

%%%%%%%%%%%%%%%%%%%%%%%%%%%%%%%%%%%%%%%%
\paragraph{v1.5:} 2017/05/21

\begin{itemize}
\item
more complete structuring introduced
\item
|\childdocof| introduced
\item
|\childdoc| renamed to |\childdocmain|
\item
|\childredirect| renamed to |\childdocforward| and |\childdocforwardprefix|
and functionality expanded
\end{itemize}

%%%%%%%%%%%%%%%%%%%%%%%%%%%%%%%%%%%%%%%%
\paragraph{v1.0:} 2017/04/27

\begin{itemize}
\item
manual and install package
\item
first version published on CTAN
\end{itemize}

%%%%%%%%%%%%%%%%%%%%%%%%%%%%%%%%%%%%%%%%
\paragraph{v0.6:} 2017/04/26

\begin{itemize}
\item
redirection mechanism added
\end{itemize}

%%%%%%%%%%%%%%%%%%%%%%%%%%%%%%%%%%%%%%%%
\paragraph{v0.5:} 2017/04/26

\begin{itemize}
\item
functionality in definition file
\end{itemize}


%%%%%%%%%%%%%%%%%%%%%%%%%%%%%%%%%%%%%%%%%%%%%%%%%%%%%%%%%%%%%%%%%%%%%%%%%%%%%%%%
%%%%%%%%%%%%%%%%%%%%%%%%%%%%%%%%%%%%%%%%%%%%%%%%%%%%%%%%%%%%%%%%%%%%%%%%%%%%%%%%
%%%%%%%%%%%%%%%%%%%%%%%%%%%%%%%%%%%%%%%%%%%%%%%%%%%%%%%%%%%%%%%%%%%%%%%%%%%%%%%%
\appendix

\settowidth\MacroIndent{\rmfamily\scriptsize 000\ }

 \DocInput{childdoc.dtx}

\end{document}
%</driver>
% \fi
%
% %%%%%%%%%%%%%%%%%%%%%%%%%%%%%%%%%%%%%%%%%%%%%%%%%%%%%%%%%%%%%%%%%%%%%%%%%%%%%%
% %%%%%%%%%%%%%%%%%%%%%%%%%%%%%%%%%%%%%%%%%%%%%%%%%%%%%%%%%%%%%%%%%%%%%%%%%%%%%%
% \section{Sample}
%\iffalse
%<*samplemain>
%\fi
%
% The following presents a sample document
% with two chapters, two parts, a title page,
% a compile flag as well as three forwarding files to set the flag.
% It consists of eight |.tex| files:
% \begin{center}
% \begin{tabular}{ll}
% |cdocsamp.tex|&main file\\
% |cdocsch1.tex|&include file for chapter 1\\
% |cdocsch2.tex|&include file for chapter 2\\
% |cdocspt3.tex|&include file for part 3\\
% |cdocspt4.tex|&include file for part 4\\
% |cdocsdrf.tex|&forwarding file for main file in draft mode\\
% |cdocsfi1.tex|&forwarding file for final version of chapter 1\\
% |cdocsfi2.tex|&forwarding file for final version of chapter 2\\
% \end{tabular}
% \end{center}
% Each of the eight files can be compiled directly by the \LaTeX{} compiler.
%
% %%%%%%%%%%%%%%%%%%%%%%%%%%%%%%%%%%%%%%
% \paragraph{Main File.}
%
% The main file is called |cdocsamp.tex|.
%
% Load the \textsf{childdoc} definitions and
% declare the filename for the main document:
%    \begin{macrocode}
\input{childdoc.def}
\childdocmain{}
%    \end{macrocode}

% Optional override for |\version| flag:
%    \begin{macrocode}
%%\ifchilddoc\else\providecommand{\version}{draft}\fi
%    \end{macrocode}

% Define the default values for the |\version| flag
% (|final| for the main file and |draft| for childs):
%    \begin{macrocode}
\ifchilddoc
\providecommand{\version}{draft}
\else
\providecommand{\version}{final}
\fi
%    \end{macrocode}

% Load the standard document class:
%    \begin{macrocode}
\documentclass[12pt]{article}
%    \end{macrocode}

% Start the document body:
%    \begin{macrocode}
\begin{document}
%    \end{macrocode}

% Declare a title page.
% Print title, part of document being processed and version flag:
%    \begin{macrocode}
\addtocounter{page}{-1}
\begin{center}
{\LARGE\bfseries{}childdoc example\par}
\vspace{1cm}
\ifchilddoc
\ifchilddocmanual part\else chapter\fi:
`\childdocname' of `\childdocjob'\par
\else
main document: `\childdocjob'\par
\fi
version: \version\par
\end{center}
\newpage
%    \end{macrocode}

% Manually include selected file,
% otherwise process as usual:
%    \begin{macrocode}
\ifchilddocmanual
\section*{part `\childdocname'}
\input{\childdocname}
\else
%    \end{macrocode}

% Include the two chapters:
%    \begin{macrocode}
\include{cdocsch1}
\include{cdocsch2}
%    \end{macrocode}

% Include the two parts unless only chapters should be displayed:
%    \begin{macrocode}
\ifchilddoc\else
\section{part three}
\input{cdocspt3}
\section{part four}
\input{cdocspt4}
\fi
%    \end{macrocode}

% Process as usual until here:
%    \begin{macrocode}
\fi
%    \end{macrocode}

% End of document body:
%    \begin{macrocode}
\end{document}
%    \end{macrocode}
%\iffalse
%</samplemain>
%\fi
%
% %%%%%%%%%%%%%%%%%%%%%%%%%%%%%%%%%%%%%%
% \paragraph{Chapter Include Files.}
%
% The include files are called |cdocsch1.tex| and |cdocsch2.tex|.
%
%\iffalse
%<*samplechap1|samplechap2>
%\fi

% Optional override for |\version| flag:
%    \begin{macrocode}
%%\providecommand{\version}{final}
%    \end{macrocode}

% Include the main document:
%    \begin{macrocode}
\input{childdoc.def}
\childdocof{cdocsamp}
%    \end{macrocode}

%\iffalse
%</samplechap1|samplechap2>
%\fi
%
%\iffalse
%<*samplechap1>
%\fi
% Some text for chapter 1:
%    \begin{macrocode}
\section{one}
some text in chapter one
%    \end{macrocode}

%\iffalse
%</samplechap1>
%\fi
% Some text for chapter 2:
%\iffalse
%<*samplechap2>
%\fi
%    \begin{macrocode}
\section{two}
more text in chapter two
%    \end{macrocode}

%\iffalse
%</samplechap2>
%\fi
%
% %%%%%%%%%%%%%%%%%%%%%%%%%%%%%%%%%%%%%%
% \paragraph{Part Include Files.}
%
% The include files are called |cdocspt3.tex| and |cdocspt4.tex|.
%
%\iffalse
%<*samplepart3|samplepart4>
%\fi

% Optional override for |\version| flag:
%    \begin{macrocode}
%%\providecommand{\version}{final}
%    \end{macrocode}

% Include the main document:
%    \begin{macrocode}
\input{childdoc.def}
\childdocby{cdocsamp}
%    \end{macrocode}

%\iffalse
%</samplepart3|samplepart4>
%\fi
%
%\iffalse
%<*samplepart3>
%\fi
% Some text for part 3:
%    \begin{macrocode}
some text in part three
%    \end{macrocode}

%\iffalse
%</samplepart3>
%\fi
% Some text for part 4:
%\iffalse
%<*samplepart4>
%\fi
%    \begin{macrocode}
more text in part four
%    \end{macrocode}

%\iffalse
%</samplepart4>
%\fi
%
% %%%%%%%%%%%%%%%%%%%%%%%%%%%%%%%%%%%%%%
% \paragraph{Forwarding for a Complete Draft.}
%
% The following forwarding file |cdocsdrf.tex|
% compiles the main document in draft mode:
%\iffalse
%<*sampledraft>
%\fi
%    \begin{macrocode}
\def\version{draft}
\input{childdoc.def}
\childdocforward{cdocsamp}
%    \end{macrocode}

%\iffalse
%</sampledraft>
%\fi
%
% %%%%%%%%%%%%%%%%%%%%%%%%%%%%%%%%%%%%%%
% \paragraph{Forwarding for Final Version of the Chapters.}
%
% The following forwarding files |cdocsfn1.tex| and |cdocsfn2.tex|
% (with identical content)
% compile the final versions of the child documents
% |cdocsch1.tex| and |cdocsch2.tex|, respectively:
%\iffalse
%<*samplefinal>
%\fi
%    \begin{macrocode}
\def\version{final}
\input{childdoc.def}
\childdocforwardprefix[cdocsamp]{cdocsfn}{cdocsch}
%    \end{macrocode}

%\iffalse
%</samplefinal>
%\fi
%
% %%%%%%%%%%%%%%%%%%%%%%%%%%%%%%%%%%%%%%
% \paragraph{Command Line Processing.}
%
% The following three command lines generate the output files
% |cdocscld|, |cdocscl1| and |cdocscl2|
% which should be identical to
% |cdocsdrf|, |cdocsch1| and |cdocsfn2|, respectively:
% \begin{center}
% \begin{tabular}{l}
% |latex -jobname cdocscld \|\\
% |  "\def\version{draft}\input{childdoc.def}\childdocforward{cdocsamp}"|\\
% |latex -jobname cdocscl1 \|\\
% |  "\input{childdoc.def}\childdocforward[cdocsamp]{cdocsch1}"|\\
% |latex -jobname cdocscl2 \|\\
% |  "\def\version{final}\input{childdoc.def}\childdocforward{cdocsch2}"|
% \end{tabular}
% \end{center}
% Note that the trailing backslash on each first line
% merely continues the input to the second line
% (for convenient cut ant paste).
% Furthermore, the command |latex| can be replaced by any
% of its alternative versions such as |pdflatex|.
%
% %%%%%%%%%%%%%%%%%%%%%%%%%%%%%%%%%%%%%%%%%%%%%%%%%%%%%%%%%%%%%%%%%%%%%%%%%%%%%%
% %%%%%%%%%%%%%%%%%%%%%%%%%%%%%%%%%%%%%%%%%%%%%%%%%%%%%%%%%%%%%%%%%%%%%%%%%%%%%%
% \section{Implementation}
%\iffalse
%<*package>
%\fi
%
% This section describes the definitions file |childdoc.def|.

% The definitions cannot be loaded using |\usepackage| or |\RequirePackage|
% which has a mechanism to prevent loading a style file more than once.
% When loading the definitions by means of |\input|
% multiple instances have to be prevented manually:
%\iffalse
%This code needs to be before the `\ProvidesFile' directive
%which is defined at the beginning of this file.
%Therefore it is also placed there and commented out here.
%</package>
%<*discard>
%\fi
%    \begin{macrocode}
\ifdefined\childdocmain\endinput\fi
%    \end{macrocode}
%\iffalse
%</discard>
%<*package>
%\fi
%
% \macro{\ifchilddoc}
% \macro{\ifchilddocmanual}
% The conditional |\ifchilddoc| tells whether a
% child (true) or main (false) document is being compiled.
% The conditional |\ifchilddocmanual| tells whether
% the |\includeonly| mechanism is used (false) or
% the selection of child files must be performed manually (true).
% The definitions initialise to false:
%    \begin{macrocode}
\newif\ifchilddoc
\newif\ifchilddocmanual
%    \end{macrocode}

% \macro{\childdocname}
% \macro{\childdocjob}
% The macro |\childdocname| stores the name of the main document
% to be compiled. The macro |\childdocjob| stores the name of
% the document on which the \LaTeX{} compiler was originally invoked.
% The content of |\jobname| cannot be compared
% to filenames specified in the source due to different catcodes.
% The following code rescans |\jobname|, stores the result
% in |\childdocname| and saves a copy in |\childdocjob|:
%    \begin{macrocode}
\edef\childdocname{\scantokens\expandafter{\jobname\noexpand}}
\let\childdocjob\childdocname
%    \end{macrocode}

% \macro{\childdocdisable}
% The macro |\childdocdisable| prevents the main file
% from being processed more than once.
% At this stage, the main document command |\childdocmain|
% is assumed to be called once again where it should do nothing.
% Any subsequent call to it should prevent
% a secondary processing of the main document
% It overwrites the forwarding commands
% |\childdocof| and |\childdocforward|
% with empty macros to prevent further inclusions of the main document:
%    \begin{macrocode}
\newcommand{\childdocdisable}
{
  \renewcommand{\childdocmain}[1]{\renewcommand{\childdocmain}[1]{\endinput}}
  \renewcommand{\childdocof}[1]{}
  \renewcommand{\childdocby}[2][]{}
  \renewcommand{\childdocforward}[2][]{}
  \renewcommand{\childdocdisable}{}
}
%    \end{macrocode}

% \macro{\childdocmain}
% The macro |\childdocmain| is to be called at the top of the main file
% with nothing or the main filename (without extension) as argument.
% First, it breaks loops.
% If the argument is not empty and does not match |\childdocname|
% (which is set by the first inclusion of |childdoc.def|),
% |\ifchilddoc| is set to true, |\includeonly| is applied to the child file
% and |\jobname| is set to the main file
% (for proper handling of |.aux| files):
%    \begin{macrocode}
\newcommand{\childdocmain}[1]
{
  \childdocdisable\childdocmain{}
  \if?#1?\else
    \begingroup
      \def\childdoctmp{#1}
      \ifx\childdoctmp\childdocname
        \def\childdoctmp{}
      \else
        \def\childdoctmp
        {
          \childdoctrue
          \includeonly{\childdocname}
          \def\childdocjob{#1}
          \def\jobname{#1}
        }
      \fi
      \expandafter
    \endgroup
    \childdoctmp
  \fi
}
%    \end{macrocode}

% \macro{\childdocof}
% The command |\childdocof| redirects
% compilation to the main file |#1|.
%    \begin{macrocode}
\newcommand{\childdocof}[1]
{
  \childdocdisable
  \childdoctrue
  \includeonly{\childdocname}
  \def\jobname{#1}
  \def\childdocjob{#1}
  \input{#1}
}
%    \end{macrocode}

% \macro{\childdocby}
% The command |\childdocby| ....
%    \begin{macrocode}
\newcommand{\childdocby}[2][]
{
  \childdocdisable
  \childdoctrue
  \childdocmanualtrue
  \if?#1?\else
    \def\jobname{#2}
  \fi
  \def\childdocjob{#2}
  \input{#2}
  \endinput
}
%    \end{macrocode}

% \macro{\childdocforward}
% The command |\childdocforward| redirects
% compilation to the main file or
% (if the optional argument is given) a child file.
% Parameters are set as if the main file
% or a child file starting with |\childdocof| was compiled.
% Then compilation is handed over to the main file:
%    \begin{macrocode}
\newcommand{\childdocforward}[2][]
{
  \begingroup
    \if?#1?
      \def\childdoctmp
      {
        \def\childdocname{#2}
        \def\childdocjob{#2}
        \def\jobname{#2}
        \input{#2}
        \endinput
      }
    \else
      \def\childdoctmp
      {
        \childdocdisable
        \def\childdocname{#2}
        \childdoctrue
        \includeonly{#2}
        \def\childdocjob{#1}
        \def\jobname{#1}
        \input{#1}
        \endinput
      }
    \fi
    \expandafter
  \endgroup
  \childdoctmp
}
%    \end{macrocode}

% \macro{\childdocforwardprefix}
% The command |\childdocforwardprefix| redirects
% compilation to the main or a child file by means of a pattern.
% The prefix |#1| in the current filename is replaced by |#2|
% and the suffix of the current filename is kept
% (it is assumed that the filename does not contain the substring `|~~~|'
% which is used as a delimiter).
% Compilation is handed over to the new file by |\childdocforward|:
%    \begin{macrocode}
\newcommand{\childdocforwardprefix}[3][]
{
  \begingroup
    \def\childdocextract #2##1~~~{\def\childdoctmp{\childdocforward[#1]{#3##1}}}
    \expandafter\childdocextract\childdocname~~~
    \expandafter
  \endgroup
  \childdoctmp
}
%    \end{macrocode}

% \macro{\childdoc}
% The deprecated macro |\childdoc| is a legacy version of |\childdocmain|:
%    \begin{macrocode}
\newcommand{\childdoc}{\childdocmain}
%    \end{macrocode}

% \macro{\childdocredirect}
% The deprecated macro |\childdocredirect| is a legacy version
% of |\childdocforward| and |\childdocforwardprefix|:
%    \begin{macrocode}
\newcommand{\childdocredirect}[2][]
{
  \begingroup
    \if?#1?
      \def\childdoctmp{\childdocforward{#2}}
    \else
      \def\childdoctmp{\childdocforwardprefix{#1}{#2}}
    \fi
    \expandafter
  \endgroup
  \childdoctmp
}
%    \end{macrocode}

%\iffalse
%</package>
%\fi
%
\endinput

\childdocforward{cdocsamp}
%    \end{macrocode}

%\iffalse
%</sampledraft>
%\fi
%
% %%%%%%%%%%%%%%%%%%%%%%%%%%%%%%%%%%%%%%
% \paragraph{Forwarding for Final Version of the Chapters.}
%
% The following forwarding files |cdocsfn1.tex| and |cdocsfn2.tex|
% (with identical content)
% compile the final versions of the child documents
% |cdocsch1.tex| and |cdocsch2.tex|, respectively:
%\iffalse
%<*samplefinal>
%\fi
%    \begin{macrocode}
\def\version{final}
% \iffalse
%
% childdoc.dtx Copyright (C) 2017-2018 Niklas Beisert
%
% This work may be distributed and/or modified under the
% conditions of the LaTeX Project Public License, either version 1.3
% of this license or (at your option) any later version.
% The latest version of this license is in
%   http://www.latex-project.org/lppl.txt
% and version 1.3 or later is part of all distributions of LaTeX
% version 2005/12/01 or later.
%
% This work has the LPPL maintenance status `maintained'.
%
% The Current Maintainer of this work is Niklas Beisert.
%
% This work consists of the files childdoc.dtx and childdoc.ins
% and the derived files childdoc.def and cdocsamp.tex with
% cdocsch1.tex, cdocsch2.tex, cdocsdrf.tex, cdocsfn1.tex, cdocsfn2.tex.
%
%<package>\ifdefined\childdocmain\endinput\fi
%<package>\ProvidesFile{childdoc.def}[2018/12/30 v2.0 child document driver]
%<samplemain>\ProvidesFile{cdocsamp.tex}[2018/12/30 v2.0 sample for childdoc]
%<*driver>
%\ProvidesFile{childdoc.drv}[2018/12/30 v2.0 childdoc reference manual file]
\PassOptionsToClass{10pt,a4paper}{article}
\documentclass{ltxdoc}

\usepackage[margin=35mm]{geometry}
\usepackage{hyperref}
\usepackage{hyperxmp}
\usepackage[usenames]{color}

\hypersetup{colorlinks=true}
\hypersetup{pdfstartview=FitH}
\hypersetup{pdfpagemode=UseNone}
\hypersetup{pdfsource={}}
\hypersetup{pdflang={en-UK}}
\hypersetup{pdfcopyright={Copyright 2017-2018 Niklas Beisert.
  This work may be distributed and/or modified under the
  conditions of the LaTeX Project Public License, either version 1.3
  of this license or (at your option) any later version.}}
\hypersetup{pdflicenseurl={http://www.latex-project.org/lppl.txt}}
\hypersetup{pdfcontactaddress={ETH Zurich, ITP, HIT K,
  Wolfgang-Pauli-Strasse 27}}
\hypersetup{pdfcontactpostcode={8093}}
\hypersetup{pdfcontactcity={Zurich}}
\hypersetup{pdfcontactcountry={Switzerland}}
\hypersetup{pdfcontactemail={nbeisert@itp.phys.ethz.ch}}
\hypersetup{pdfcontacturl={http://people.phys.ethz.ch/\xmptilde nbeisert/}}

\newcommand{\secref}[1]{\hyperref[#1]{section \ref*{#1}}}

\parskip1ex
\parindent0pt
\let\olditemize\itemize
\def\itemize{\olditemize\parskip0pt}

\begin{document}

\title{The \textsf{childdoc} Package}
\hypersetup{pdftitle={The childdoc Package}}
\author{Niklas Beisert\\[2ex]
  Institut f\"ur Theoretische Physik\\
  Eidgen\"ossische Technische Hochschule Z\"urich\\
  Wolfgang-Pauli-Strasse 27, 8093 Z\"urich, Switzerland\\[1ex]
  \href{mailto:nbeisert@itp.phys.ethz.ch}
  {\texttt{nbeisert@itp.phys.ethz.ch}}}
\hypersetup{pdfauthor={Niklas Beisert}}
\hypersetup{pdfsubject={Manual for the LaTeX2e Package childdoc}}
\date{30 December 2018, \textsf{v2.0}}
\maketitle

\begin{abstract}\noindent
\textsf{childdoc} is a \LaTeXe{} package
that enables the direct compilation
of document sections included by |\include|
to individual files.
\end{abstract}

\begingroup
\parskip0ex
\tableofcontents
\endgroup

%%%%%%%%%%%%%%%%%%%%%%%%%%%%%%%%%%%%%%%%%%%%%%%%%%%%%%%%%%%%%%%%%%%%%%%%%%%%%%%%
%%%%%%%%%%%%%%%%%%%%%%%%%%%%%%%%%%%%%%%%%%%%%%%%%%%%%%%%%%%%%%%%%%%%%%%%%%%%%%%%
\section{Introduction}

\LaTeX{} provides a mechanism to structure a large document (such as a book)
into a main file and several child files (containing the chapters)
using the |\include| command.
This mechanism is beneficial for documents
which span hundreds of pages in order to
make the source file(s) more manageable.
Moreover, compilation can be restricted to
selected child files by means of the |\includeonly| command.
The latter feature can be used to reduce the compilation time while editing
(this was significantly more useful in the earlier days of \LaTeX{})
or to generate a smaller document which is easier to navigate.
Another application of |\includeonly| is to generate
documents consisting of selected parts of the complete document.

However, there are a few drawbacks of the plain |\include| mechanism:
\begin{itemize}
\item
The child files cannot be compiled on their own,
they can only be compiled via the main file.
A naive editing environment
(such as a text editor with an option
to have the current file processed by \LaTeX)
may require one to switch to the main file before compiling;
attempting to compile the child file produces errors.
\item
The main file must be modified (each time)
to adjust the |\includeonly| command
to the present needs. This easily leaves the main file in a messy state.
\item
The generated document will always carry the filename
of the main document. This is inconvenient if
several child files are to be compiled and
to be kept for distribution.
\end{itemize}

The present package provides a simple interface
to make child files individually compilable by \LaTeX{}.
Compiling a child file then has the same effect as compiling
the main file with an |\includeonly| command
to select the appropriate child.
Moreover the generated document will carry the name of the child
rather than the main file.
This resolves all three above issues.

This feature is meant to make the editing of books,
thesis documents and lecture notes somewhat more convenient.
However, the package can also be used efficiently for
composing a series of documents (such as exercise sheets)
which are typically distributed individually.
It then assists the author in generating the individual documents
(potentially in different versions)
as well as a document containing the collected series.
Another application is in developing style files
or other kinds of included material
where compilation of the style file could redirect
to a sample or test file.

%%%%%%%%%%%%%%%%%%%%%%%%%%%%%%%%%%%%%%%%%%%%%%%%%%%%%%%%%%%%%%%%%%%%%%%%%%%%%%%%
%%%%%%%%%%%%%%%%%%%%%%%%%%%%%%%%%%%%%%%%%%%%%%%%%%%%%%%%%%%%%%%%%%%%%%%%%%%%%%%%
\section{Usage}

First of all, the package \textsf{childdoc} is \emph{not} a standard
\LaTeXe{} |.sty| style file! Therefore it needs to be invoked in
a non-standard way.

%%%%%%%%%%%%%%%%%%%%%%%%%%%%%%%%%%%%%%%%%%%%%%%%%%%%%%%%%%%%%%%%%%%%%%%%%%%%%%%%
\subsection{Included Files}
\label{sec:include}

%%%%%%%%%%%%%%%%%%%%%%%%%%%%%%%%%%%%%%%%
\DescribeMacro{\childdocmain}
To use the package, add the commands
\begin{center}
\begin{tabular}{l}
|\input{childdoc.def}|\\
|\childdocmain{}|\\
\end{tabular}
\end{center}
at the very top of the main \LaTeX{} file,
in particular \emph{before} the |\documentclass| statement!
The argument of |\childdocmain| should be left empty
(but it must be present).

%%%%%%%%%%%%%%%%%%%%%%%%%%%%%%%%%%%%%%%%
\DescribeMacro{\childdocof}
Furthermore, add the commands
\begin{center}
\begin{tabular}{l}
|\input{childdoc.def}|\\
|\childdocof{|\textit{main}|}|\\
\end{tabular}
\end{center}
at the top of every child file \textit{child}
which is included by |\include{|\textit{child}|}|
from within the main file
(or at least for those files to be compiled individually).
The argument \textit{main} must be the filename of the main file.

There are a couple of
considerations in setting up the main and child documents:

%%%%%%%%%%%%%%%%%%%%%%%%%%%%%%%%%%%%%%%%
\paragraph{Restrictions.}

Please note the following restrictions:
\begin{itemize}
\item
|\childdocmain| must be called with one argument \textit{main}
to ensure compatibility with earlier version of the package.
It must either be empty (|\childdocmain{}|)
or precisely match the filename of the main file in which it is specified.
See \secref{sec:detection} for further information.
\item
The filename \textit{main} must be specified without the |.tex| extension.
\item
The filename \textit{main} is case sensitive
(even in case-insensitive file systems)
due to internal string comparison.
\item
The argument \textit{main} should be fully expanded, it cannot be a macro.
\item
Subdirectories and special characters should be avoided in filenames.
\item
The command |\childdocmain{|\textit{main}|}| must be followed by a whitespace.
It should not be followed immediately by another command
or by a comment mark `|%|'.
This is because the \TeX{} parser reads the token immediately following
the argument of |\childdocmain| and puts it
at the beginning of every child section;
however, a white\-space is ignored.
\end{itemize}

%%%%%%%%%%%%%%%%%%%%%%%%%%%%%%%%%%%%%%%%
\paragraph{Content of Main File.}

It is advisable to place all content in the child files included by |\include|.
Any output contained in the main file will appear in all child documents
unless suppressed manually;
it cannot be suppressed automatically by the |\includeonly| directive
and thus should normally be avoided.
A method to include some content in the main file
by means of conditional processing is described in \secref{sec:conditional}.

%%%%%%%%%%%%%%%%%%%%%%%%%%%%%%%%%%%%%%%%
\paragraph{Page Numbering.}

When only a part of the document is compiled,
the appropriate numbering of pages
(as well as other status parameters)
is determined from the |.aux| files.
The latter contain information from previous passes.
However this information needs to propagate through
all intermediate child documents.
Therefore the page numbering in child documents may well
be inconsistent until the complete document is compiled at least once.

A useful (if unconventional) way to always ensure a consistent
page numbering is to restart the numbering in each child document
and denote the pages by `\textit{child}|.|\textit{page}'
where \textit{child} represents the chapter/section number of the child file.
This can be achieved by the command
|\numberwithin{page}{|\textit{child}|}|
of the \textsf{amsmath} package
where \textit{child} can be |chapter| or |section|
depending on the chosen structuring.
Alternatively, one can modify the macro |\thepage| appropriately
and reset the counter |page| at the start of each child file.

%%%%%%%%%%%%%%%%%%%%%%%%%%%%%%%%%%%%%%%%%%%%%%%%%%%%%%%%%%%%%%%%%%%%%%%%%%%%%%%%
\subsection{Conditional Processing}
\label{sec:conditional}

The package provides a mechanism to compile different versions
of a document. To customise the versions further some conditional processing
can come in handy to distinguish which version is being compiled.
The package provides two macros to describe the compilation context:

%%%%%%%%%%%%%%%%%%%%%%%%%%%%%%%%%%%%%%%%
\DescribeMacro{\ifchilddoc}
The conditional |\ifchilddoc| distinguishes between the compilation of
child documents and the main document:
%
\begin{center}
|\ifchilddoc |\textit{child-code}| |[|\||else |\textit{main-code}]| \||fi|
\end{center}

%%%%%%%%%%%%%%%%%%%%%%%%%%%%%%%%%%%%%%%%
\DescribeMacro{\childdocname}
\DescribeMacro{\childdocjob}
The macro |\childdocname| contains the filename (without extension)
of the main or child file being processed.
Note that |\childdocjob| will always contain the name of the main file.

%%%%%%%%%%%%%%%%%%%%%%%%%%%%%%%%%%%%%%%%
\paragraph{Title Page.}

Conditional processing can be used to include a title or banner page
in the main document when proper precautions are taken.
Importantly, the code in the main file should ensure that the page counter
(as well as other status parameters which are stored in the |.aux| files)
takes the same value after the conditional processing.
Otherwise the page numbers may take divergent values
depending on which part is compiled.

For example, a title page could be declared by:
%
\begin{center}
\begin{tabular}{l}
|\ifchilddoc\||else|\\
|\addtocounter{page}{-1}|\\
\textit{code for title page}\\
|\newpage|\\
|\||fi|
\end{tabular}
\end{center}
%
A banner page for the child documents can be generated by:
%
\begin{center}
\begin{tabular}{l}
|\ifchilddoc|\\
|\addtocounter{page}{-1}|\\
\textit{code for banner page}\\
|\newpage|\\
|\||fi|
\end{tabular}
\end{center}
%
Here one could write a message such as:
\begin{center}
|This is the part \childdocname{} of \childdocjob{}.|
\end{center}

%%%%%%%%%%%%%%%%%%%%%%%%%%%%%%%%%%%%%%%%%%%%%%%%%%%%%%%%%%%%%%%%%%%%%%%%%%%%%%%%
\subsection{Flags}
\label{sec:flags}

The package makes it easy to generate different versions
of the main or child documents.
To this end compilation flags can be defined
and assigned different default values.
They will be particularly useful in conjunction
with the forwarding mechanism described in \secref{sec:forward}.

For example, it may be useful to have a flag |\version|
which can be set to |draft| or |final|.
The document source will contain some conditional code
depending on the value of |\version|.
Suppose further, the flag should default to |final| for the main file
and to |draft| for child files
which is a natural assignment for editing the document.
This is achieved by placing the following code
in the preamble of the main document
(below the |\childdocmain| directive):
%
\begin{center}
\begin{tabular}{l}
|\ifchilddoc|\\
|\providecommand{\version}{draft}|\\
|\||else|\\
|\providecommand{\version}{final}|\\
|\||fi|
\end{tabular}
\end{center}
%
The definition by |\providecommand| makes sure
that previous definitions are not overwritten.
Further statements |\providecommand{\version}{...}|
can thus be added before the above code to override it.

For the main file, one might add a line
(between |\childdocmain| and the above block)
%
\begin{center}
|%\ifchilddoc\||else\providecommand{\version}{draft}\||fi|
\end{center}
%
which can be uncommented to produce a draft version.
Likewise one can add a line to the very top of a child file
(above the |\childdocof{|\textit{main}|}| directive)
%
\begin{center}
|%\providecommand{\version}{final}|
\end{center}
%
which can be uncommented to produce the final version of this child document.

%%%%%%%%%%%%%%%%%%%%%%%%%%%%%%%%%%%%%%%%%%%%%%%%%%%%%%%%%%%%%%%%%%%%%%%%%%%%%%%%
\subsection{Forwarding}
\label{sec:forward}

Different versions of the main or child documents
using compilation flags as described in \secref{sec:flags}
can be (permanently) stored in different files
for convenient compilation, viewing and distribution.
To this end, the package defines a command
to pass on compilation to a different file:

%%%%%%%%%%%%%%%%%%%%%%%%%%%%%%%%%%%%%%%%
\DescribeMacro{\childdocforward}
The command |\childdocforward| redirects processing to
another source file:
%
\begin{center}
\begin{tabular}{l}
|\input{childdoc.def}|\\
|\childdocforward[|\textit{main}|]{|\textit{dest}|}|\\
\end{tabular}
\end{center}
%
The argument \textit{dest} is the destination file
(without extension).
It should be the main file or one of the child files.
Note that further \textsf{childdoc} directives
such as |\childdocof| and |\childdocforward|
in the indicated file will be processed in this form.
The optional argument \textit{main}
passes on directly to the main file \textit{main}
while pretending to compile the child \textit{dest}.
This form behaves as if \textit{dest}
issues |\childdocof{|\textit{main}|}| right away,
and no further \textsf{childdoc} directives will be processed.

%%%%%%%%%%%%%%%%%%%%%%%%%%%%%%%%%%%%%%%%
\DescribeMacro{\...prefix}
In the alternative form |\childdocforwardprefix|,
%
\begin{center}
\begin{tabular}{l}
|\input{childdoc.def}|\\
|\childdocforwardprefix[|\textit{main}|]{|\textit{prefix}|}{|\textit{dest}|}|
\end{tabular}
\end{center}
%
the destination file is determined by a pattern
depending on the current file:
To make this work, the current file must be called
`{\textit{prefix}\hspace{0.2em}\textit{suffix}}'
with \textit{prefix} matching precisely the argument.
Processing is then passed on to the file
`{\textit{dest}\hspace{0.2em}\textit{suffix}}'.
Surely, the same effect is achieved by
directly specifying the
argument `{\textit{dest}\hspace{0.2em}\textit{suffix}}'
in the first form.
However, that requires to set up a different file
for each child. With the alternative form of the command
all these files can have exactly the same content
which simplifies setting them up and maintaining them.

For example, the following file |draft.tex|
with a compilation flag |\version| as described in \secref{sec:flags}
compiles the main document as a draft:
%
\begin{center}
\begin{tabular}{l}
|\def\version{draft}|\\
|\input{childdoc.def}|\\
|\childdocforward{|\textit{main}|}|
\end{tabular}
\end{center}
%
Likewise, the following files |final|\textit{nn}|.tex|
compile the final version of the child document
|child|\textit{nn}|.tex|:
%
\begin{center}
\begin{tabular}{l}
|\def\version{final}|\\
|\input{childdoc.def}|\\
|\childdocforwardprefix{final}{child}|
\end{tabular}
\end{center}
%

Note that when several versions of a main file and/or of each child file
are to be generated, it may be convenient to set up a |Makefile| or
shell script to automatise the process.

%%%%%%%%%%%%%%%%%%%%%%%%%%%%%%%%%%%%%%%%%%%%%%%%%%%%%%%%%%%%%%%%%%%%%%%%%%%%%%%%
\subsection{Command Line Processing}
\label{sec:commandline}

The effect of redirection files can also be achieved by invoking
the \LaTeX{} compiler with a more elaborate command line.
Most conveniently this should be done as part
of a shell script or a |Makefile|.

When using \textsf{childdoc} in the main file, the following
command lines effectively perform a redirection
(note that depending on the shell being used,
backslashes may have to be doubled: `|\|' $\to$ `|\\|'):
%
\begin{center}
|... -jobname "|\textit{target}|" |\\|"|[\textit{flags}]%
|\input{childdoc.def}\childdocforward[|\textit{main}|]{|\textit{dest}|}"|
\end{center}
%
Here \textit{target} is the name of the output file,
\textit{main} is the name of the main file
and \textit{dest} is the name of the main or child file to be processed
(all filenames without extensions).
The optional argument \textit{main} can be omitted
if \textit{main} matches \textit{dest}.
Optionally, compilation \textit{flags} can be defined via |\def| commands.
This command line makes the \TeX{} engine believe
it is compiling the file \textit{target}
whose content is specified as the latter parameter.
The provided code then forwards the processing to
\textit{main} or \textit{dest} as described in \secref{sec:forward}.

%%%%%%%%%%%%%%%%%%%%%%%%%%%%%%%%%%%%%%%%%%%%%%%%%%%%%%%%%%%%%%%%%%%%%%%%%%%%%%%%
\subsection{Include by Input}
\label{sec:input}

Including child documents by |\include| has some restrictions by design.
Most notably, the content of a child document always occupies
its own set of pages; pages cannot be shared between child documents.
Usually, this behaviour makes perfect sense
because each child document contain an essential part of the document.
However, in some situations it may be desirable to compose
a document from a collection of parts
without having mandatory page breaks between then.
For this case, the package
provides a mechanism to include parts
by |\input| which can also be processed individually.
However, by construction this mechanism
requires manual handling of the content to be output.

%%%%%%%%%%%%%%%%%%%%%%%%%%%%%%%%%%%%%%%%
\DescribeMacro{\ifchilddocmanual}
The main file should be prepared as usual, see \secref{sec:include}.
However, the document body must make a distinction
between processing of an individual part and of the main document, e.g.:
%
\begin{center}
\begin{tabular}{l}
|\ifchilddocmanual|\\
|\input{\childdocname}|\\
|\||else|\\
\textit{document body with }|\input{|\textit{part}|}|\\
|\||fi|
\end{tabular}
\end{center}
%
The conditional |\ifchilddocmanual| is true whenever
a part to be included by |\input| is being compiled,
and the name of the part is stored in |\childdocname|.

%%%%%%%%%%%%%%%%%%%%%%%%%%%%%%%%%%%%%%%%
\DescribeMacro{\childdocby}
Each part to be included by |\input| should start with:
%
\begin{center}
\begin{tabular}{l}
|\input{childdoc.def}|\\
|\childdocby{|\textit{main}|}|\\
\end{tabular}
\end{center}
%
The directive |\childdocby| is similar to |\childdocof|
described in \secref{sec:include},
but the subsequent selection of content must be done manually.
To that end, both |\ifchilddoc| and |\ifchilddocmanual|
will be true upon processing of a part,
and the name of the part is stored in |\childdocname|.
Note that |\jobname| will be set to the filename of the current part
so that each part receives an individual |.aux| file
that does not interfere with the |.aux| file(s) of the main document.
This behaviour can be altered by the alternative form
|\childdocby[*]{|\textit{main}|}| (with a non-empty optional argument)
which uses the |.aux| file of the main document
by setting |\jobname| to \textit{main}.

%%%%%%%%%%%%%%%%%%%%%%%%%%%%%%%%%%%%%%%%%%%%%%%%%%%%%%%%%%%%%%%%%%%%%%%%%%%%%%%%
\subsection{Driver Development}
\label{sec:driver}

The \textsf{childdoc} mechanism can also be use for the development
of definition files such as \LaTeX{} styles or classes.
This case differs from the above setup with multiple parts
included by |\include| in that no |\includeonly| should be invoked.
This can be achieved by starting the include file
(before |\ProvidesPackage|) with:
%
\begin{center}
\begin{tabular}{l}
|\input{childdoc.def}|\\
|\childdocforward{|\textit{main}|}|\\
\end{tabular}
\end{center}
%
or alternatively with:
%
\begin{center}
\begin{tabular}{l}
|\input{childdoc.def}|\\
|\childdocby{|\textit{main}|}|\\
\end{tabular}
\end{center}
%
Both forms have slightly different effects as described above.
The main file is prepared as usual, see \secref{sec:include}.

%%%%%%%%%%%%%%%%%%%%%%%%%%%%%%%%%%%%%%%%%%%%%%%%%%%%%%%%%%%%%%%%%%%%%%%%%%%%%%%%
\subsection{Legacy Detection}
\label{sec:detection}

The directive |\childdocmain| in the main file can detect
whether the complete document or merely a child is to be compiled
even without using the directive |\childdocof|.
This method is deprecated because it is less robust
and there is no compelling reason to use it;
it is merely provided for backward compatibility
and it may be removed in future versions.

If the detection mechanism is to be used,
it is mandatory to correctly specify
the filename of the main file as the argument of |\childdocmain|:
%
\begin{center}
\begin{tabular}{l}
|\input{childdoc.def}|\\
|\childdocmain{|\textit{main}|}|\\
\end{tabular}
\end{center}
%
If |\jobname| does not match the argument \textit{main} of |\childdocmain|,
it is assumed that |\jobname| points to the child file to be compiled.
When using |\childdocmain| with the main file specified as argument,
it suffices to start a child file
with just |\input{|\textit{main}|}|
without loading of the package and using |\childdocof|.
If instead all processing is done
with the appropriate \textsf{childdoc} directives,
the argument of \textit{main} of |\childdocmain| can be empty.

An alternative version of the command line processing described
in \secref{sec:commandline} using the detection mechanism reads:
%
\begin{center}
|... -jobname "|\textit{target}|" "|[\textit{flags}]%
[|\def\jobname{|\textit{dest}|}|]|\input{|\textit{main}|}"|
\end{center}

%%%%%%%%%%%%%%%%%%%%%%%%%%%%%%%%%%%%%%%%%%%%%%%%%%%%%%%%%%%%%%%%%%%%%%%%%%%%%%%%
\subsection{Manual Code}
\label{sec:manual}

In case one cannot be certain whether the definitions file |childdoc.def|
is installed on the target \TeX{} distribution
and one prefers not to ship it,
it is conceivable to paste a few relevant commands into the sources.

To that end, drop all statements |\input{childdoc.def}|
and perform the replacements as outlined below.
Instead of |\childdocmain{|\textit{main}|}| add the following code
to the top of the main file:
%
\begin{center}
\begin{tabular}{l}
|\||ifdefined\childdocname\endinput\||fi\newif\ifchilddoc|\\
|\edef\childdocname{\scantokens\expandafter{\jobname\noexpand}}|\\
|\def\childdocmain{|\textit{main}|}\||ifx\childdocmain\childdocname\||else|\\
|\childdoctrue\includeonly{\childdocname}\let\jobname\childdocmain\||fi|\\
\end{tabular}
\end{center}
%
Instead of |\childdocof{|\textit{main}|}| just include the main file
at the top of each child file:
%
\begin{center}
|\input{|\textit{main}|}|
\end{center}
%
A simple redirection |\childdocforward{|\textit{dest}|}| is achieved by:
%
\begin{center}
|\def\jobname{|\textit{dest}|}\input{\jobname}|
\end{center}
%
The redirection with prefix
|\childdocforwardprefix[|\textit{prefix}|]{|\textit{dest}|}|
is accomplished by:
%
\begin{center}
\begin{tabular}{l}
|{\edef\jobname{\scantokens\expandafter{\jobname\noexpand}}|\\
|\def\redirectjob |\textit{prefix}|#1~~~{\gdef\jobname{|\textit{dest}|#1}}|\\
|\expandafter\redirectjob\jobname~~~}\input{\jobname}|
\end{tabular}
\end{center}

In an alternative approach,
child documents can be compiled by a specific command line
without additional code or specific definitions:
%
\begin{center}
|... -jobname "|\textit{target}|" "|[\textit{flags}]%
|\includeonly{|\textit{dest}|}\input{|\textit{main}|}"|
\end{center}
%

%%%%%%%%%%%%%%%%%%%%%%%%%%%%%%%%%%%%%%%%%%%%%%%%%%%%%%%%%%%%%%%%%%%%%%%%%%%%%%%%
%%%%%%%%%%%%%%%%%%%%%%%%%%%%%%%%%%%%%%%%%%%%%%%%%%%%%%%%%%%%%%%%%%%%%%%%%%%%%%%%
\section{Information}

%%%%%%%%%%%%%%%%%%%%%%%%%%%%%%%%%%%%%%%%%%%%%%%%%%%%%%%%%%%%%%%%%%%%%%%%%%%%%%%%
\subsection{Copyright}

Copyright \copyright{} 2017--2018 Niklas Beisert

This work may be distributed and/or modified under the
conditions of the \LaTeX{} Project Public License, either version 1.3
of this license or (at your option) any later version.
The latest version of this license is in
  \url{http://www.latex-project.org/lppl.txt}
and version 1.3 or later is part of all distributions of \LaTeX{}
version 2005/12/01 or later.

This work has the LPPL maintenance status `maintained'.

The Current Maintainer of this work is Niklas Beisert.

This work consists of the files |README.txt|, |childdoc.ins| and |childdoc.dtx|
as well as the derived files |childdoc.def|, |cdocsamp.tex|
with |cdocsch1.tex|, |cdocsch2.tex|, |cdocspt3.tex|, |cdocspt4.tex|,
|cdocsdrf.tex|, |cdocsfn1.tex|, |cdocsfn2.tex|
as well as |childdoc.pdf|.

%%%%%%%%%%%%%%%%%%%%%%%%%%%%%%%%%%%%%%%%%%%%%%%%%%%%%%%%%%%%%%%%%%%%%%%%%%%%%%%%
\subsection{Files and Installation}

The package consists of the files:
%
\begin{center}
\begin{tabular}{ll}
    |README.txt|   & readme file \\
    |childdoc.ins| & installation file \\
    |childdoc.dtx| & source file \\
    |childdoc.def| & definition file \\
    |cdocsamp.tex| & sample main file \\
    |cdocsch1.tex| & sample include file \\
    |cdocsch2.tex| & sample include file \\
    |cdocspt3.tex| & sample part file \\
    |cdocspt4.tex| & sample part file \\
    |cdocsdrf.tex| & sample redirection file \\
    |cdocsfn1.tex| & sample redirection file \\
    |cdocsfn2.tex| & sample redirection file \\
    |childdoc.pdf| & manual
\end{tabular}
\end{center}
%
The distribution consists of the files
|README.txt|, |childdoc.ins| and |childdoc.dtx|.
%
\begin{itemize}
\item
Run (pdf)\LaTeX{} on |childdoc.dtx|
to compile the manual |childdoc.pdf| (this file).
\item
Run \LaTeX{} on |childdoc.ins| to create the definitions file |childdoc.def|
and the sample |cdocsamp.tex| with include files
|cdocsch1.tex|, |cdocsch2.tex|, |cdocspt3.tex|, |cdocspt4.tex|,
|cdocsdrf.tex|, |cdocsfn1.tex|, |cdocsfn2.tex|.
Then copy the file |childdoc.def| to an appropriate directory of your \LaTeX{}
distribution, e.g.\ \textit{texmf-root}|/tex/latex/childdoc|.
\end{itemize}

%%%%%%%%%%%%%%%%%%%%%%%%%%%%%%%%%%%%%%%%%%%%%%%%%%%%%%%%%%%%%%%%%%%%%%%%%%%%%%%%
\subsection{Related CTAN Packages}

There are several other packages which offer a similar functionality:
%
\begin{itemize}
\item
The packages
\href{http://ctan.org/pkg/docmute}{\textsf{docmute}},
\href{http://ctan.org/pkg/includex}{\textsf{includex}} and
\href{http://ctan.org/pkg/standalone}{\textsf{standalone}}
provide commands to include only the document body of
a child file thus allowing both files to be compiled individually.
\item
The packages \href{http://ctan.org/pkg/subdocs}{\textsf{subdocs}}
and \href{http://ctan.org/pkg/subfiles}{\textsf{subfiles}}
provide structures in which the main and child documents can be
encapsulated and allowing them to be compiled individually.
The inclusion mechanism is different from the conventional |\include|.
\item
The package \href{http://ctan.org/pkg/combine}{\textsf{combine}}
is an elaborate solution to combine several documents into one.
\end{itemize}
%
See also the CTAN topic \href{http://ctan.org/topic/subdocs}{\textsf{subdocs}}
for further related packages.
The present package differs from the above solutions in that
a document structure constructed with the conventional |\include| mechanism
just needs two extra commands at the top of every file
such that all constituent files can be compiled individually.

%%%%%%%%%%%%%%%%%%%%%%%%%%%%%%%%%%%%%%%%%%%%%%%%%%%%%%%%%%%%%%%%%%%%%%%%%%%%%%%%
%\subsection{Feature Suggestions}
%
%The following is a list of features which may be useful for future
%versions of this package:
%%
%\begin{itemize}
%\item
%\ldots
%\end{itemize}

%%%%%%%%%%%%%%%%%%%%%%%%%%%%%%%%%%%%%%%%%%%%%%%%%%%%%%%%%%%%%%%%%%%%%%%%%%%%%%%%
\subsection{Revision History}

%%%%%%%%%%%%%%%%%%%%%%%%%%%%%%%%%%%%%%%%
\paragraph{v2.0:} 2018/12/30

\begin{itemize}
\item
immediate forward processing
\item
added |\childdocby| mechanism
\item
manual restructured
\end{itemize}

%%%%%%%%%%%%%%%%%%%%%%%%%%%%%%%%%%%%%%%%
\paragraph{v1.6:} 2018/01/17

\begin{itemize}
\item
application for development of include files
\item
corrections to manual
\end{itemize}

%%%%%%%%%%%%%%%%%%%%%%%%%%%%%%%%%%%%%%%%
\paragraph{v1.5:} 2017/05/21

\begin{itemize}
\item
more complete structuring introduced
\item
|\childdocof| introduced
\item
|\childdoc| renamed to |\childdocmain|
\item
|\childredirect| renamed to |\childdocforward| and |\childdocforwardprefix|
and functionality expanded
\end{itemize}

%%%%%%%%%%%%%%%%%%%%%%%%%%%%%%%%%%%%%%%%
\paragraph{v1.0:} 2017/04/27

\begin{itemize}
\item
manual and install package
\item
first version published on CTAN
\end{itemize}

%%%%%%%%%%%%%%%%%%%%%%%%%%%%%%%%%%%%%%%%
\paragraph{v0.6:} 2017/04/26

\begin{itemize}
\item
redirection mechanism added
\end{itemize}

%%%%%%%%%%%%%%%%%%%%%%%%%%%%%%%%%%%%%%%%
\paragraph{v0.5:} 2017/04/26

\begin{itemize}
\item
functionality in definition file
\end{itemize}


%%%%%%%%%%%%%%%%%%%%%%%%%%%%%%%%%%%%%%%%%%%%%%%%%%%%%%%%%%%%%%%%%%%%%%%%%%%%%%%%
%%%%%%%%%%%%%%%%%%%%%%%%%%%%%%%%%%%%%%%%%%%%%%%%%%%%%%%%%%%%%%%%%%%%%%%%%%%%%%%%
%%%%%%%%%%%%%%%%%%%%%%%%%%%%%%%%%%%%%%%%%%%%%%%%%%%%%%%%%%%%%%%%%%%%%%%%%%%%%%%%
\appendix

\settowidth\MacroIndent{\rmfamily\scriptsize 000\ }

 \DocInput{childdoc.dtx}

\end{document}
%</driver>
% \fi
%
% %%%%%%%%%%%%%%%%%%%%%%%%%%%%%%%%%%%%%%%%%%%%%%%%%%%%%%%%%%%%%%%%%%%%%%%%%%%%%%
% %%%%%%%%%%%%%%%%%%%%%%%%%%%%%%%%%%%%%%%%%%%%%%%%%%%%%%%%%%%%%%%%%%%%%%%%%%%%%%
% \section{Sample}
%\iffalse
%<*samplemain>
%\fi
%
% The following presents a sample document
% with two chapters, two parts, a title page,
% a compile flag as well as three forwarding files to set the flag.
% It consists of eight |.tex| files:
% \begin{center}
% \begin{tabular}{ll}
% |cdocsamp.tex|&main file\\
% |cdocsch1.tex|&include file for chapter 1\\
% |cdocsch2.tex|&include file for chapter 2\\
% |cdocspt3.tex|&include file for part 3\\
% |cdocspt4.tex|&include file for part 4\\
% |cdocsdrf.tex|&forwarding file for main file in draft mode\\
% |cdocsfi1.tex|&forwarding file for final version of chapter 1\\
% |cdocsfi2.tex|&forwarding file for final version of chapter 2\\
% \end{tabular}
% \end{center}
% Each of the eight files can be compiled directly by the \LaTeX{} compiler.
%
% %%%%%%%%%%%%%%%%%%%%%%%%%%%%%%%%%%%%%%
% \paragraph{Main File.}
%
% The main file is called |cdocsamp.tex|.
%
% Load the \textsf{childdoc} definitions and
% declare the filename for the main document:
%    \begin{macrocode}
\input{childdoc.def}
\childdocmain{}
%    \end{macrocode}

% Optional override for |\version| flag:
%    \begin{macrocode}
%%\ifchilddoc\else\providecommand{\version}{draft}\fi
%    \end{macrocode}

% Define the default values for the |\version| flag
% (|final| for the main file and |draft| for childs):
%    \begin{macrocode}
\ifchilddoc
\providecommand{\version}{draft}
\else
\providecommand{\version}{final}
\fi
%    \end{macrocode}

% Load the standard document class:
%    \begin{macrocode}
\documentclass[12pt]{article}
%    \end{macrocode}

% Start the document body:
%    \begin{macrocode}
\begin{document}
%    \end{macrocode}

% Declare a title page.
% Print title, part of document being processed and version flag:
%    \begin{macrocode}
\addtocounter{page}{-1}
\begin{center}
{\LARGE\bfseries{}childdoc example\par}
\vspace{1cm}
\ifchilddoc
\ifchilddocmanual part\else chapter\fi:
`\childdocname' of `\childdocjob'\par
\else
main document: `\childdocjob'\par
\fi
version: \version\par
\end{center}
\newpage
%    \end{macrocode}

% Manually include selected file,
% otherwise process as usual:
%    \begin{macrocode}
\ifchilddocmanual
\section*{part `\childdocname'}
\input{\childdocname}
\else
%    \end{macrocode}

% Include the two chapters:
%    \begin{macrocode}
\include{cdocsch1}
\include{cdocsch2}
%    \end{macrocode}

% Include the two parts unless only chapters should be displayed:
%    \begin{macrocode}
\ifchilddoc\else
\section{part three}
\input{cdocspt3}
\section{part four}
\input{cdocspt4}
\fi
%    \end{macrocode}

% Process as usual until here:
%    \begin{macrocode}
\fi
%    \end{macrocode}

% End of document body:
%    \begin{macrocode}
\end{document}
%    \end{macrocode}
%\iffalse
%</samplemain>
%\fi
%
% %%%%%%%%%%%%%%%%%%%%%%%%%%%%%%%%%%%%%%
% \paragraph{Chapter Include Files.}
%
% The include files are called |cdocsch1.tex| and |cdocsch2.tex|.
%
%\iffalse
%<*samplechap1|samplechap2>
%\fi

% Optional override for |\version| flag:
%    \begin{macrocode}
%%\providecommand{\version}{final}
%    \end{macrocode}

% Include the main document:
%    \begin{macrocode}
\input{childdoc.def}
\childdocof{cdocsamp}
%    \end{macrocode}

%\iffalse
%</samplechap1|samplechap2>
%\fi
%
%\iffalse
%<*samplechap1>
%\fi
% Some text for chapter 1:
%    \begin{macrocode}
\section{one}
some text in chapter one
%    \end{macrocode}

%\iffalse
%</samplechap1>
%\fi
% Some text for chapter 2:
%\iffalse
%<*samplechap2>
%\fi
%    \begin{macrocode}
\section{two}
more text in chapter two
%    \end{macrocode}

%\iffalse
%</samplechap2>
%\fi
%
% %%%%%%%%%%%%%%%%%%%%%%%%%%%%%%%%%%%%%%
% \paragraph{Part Include Files.}
%
% The include files are called |cdocspt3.tex| and |cdocspt4.tex|.
%
%\iffalse
%<*samplepart3|samplepart4>
%\fi

% Optional override for |\version| flag:
%    \begin{macrocode}
%%\providecommand{\version}{final}
%    \end{macrocode}

% Include the main document:
%    \begin{macrocode}
\input{childdoc.def}
\childdocby{cdocsamp}
%    \end{macrocode}

%\iffalse
%</samplepart3|samplepart4>
%\fi
%
%\iffalse
%<*samplepart3>
%\fi
% Some text for part 3:
%    \begin{macrocode}
some text in part three
%    \end{macrocode}

%\iffalse
%</samplepart3>
%\fi
% Some text for part 4:
%\iffalse
%<*samplepart4>
%\fi
%    \begin{macrocode}
more text in part four
%    \end{macrocode}

%\iffalse
%</samplepart4>
%\fi
%
% %%%%%%%%%%%%%%%%%%%%%%%%%%%%%%%%%%%%%%
% \paragraph{Forwarding for a Complete Draft.}
%
% The following forwarding file |cdocsdrf.tex|
% compiles the main document in draft mode:
%\iffalse
%<*sampledraft>
%\fi
%    \begin{macrocode}
\def\version{draft}
\input{childdoc.def}
\childdocforward{cdocsamp}
%    \end{macrocode}

%\iffalse
%</sampledraft>
%\fi
%
% %%%%%%%%%%%%%%%%%%%%%%%%%%%%%%%%%%%%%%
% \paragraph{Forwarding for Final Version of the Chapters.}
%
% The following forwarding files |cdocsfn1.tex| and |cdocsfn2.tex|
% (with identical content)
% compile the final versions of the child documents
% |cdocsch1.tex| and |cdocsch2.tex|, respectively:
%\iffalse
%<*samplefinal>
%\fi
%    \begin{macrocode}
\def\version{final}
\input{childdoc.def}
\childdocforwardprefix[cdocsamp]{cdocsfn}{cdocsch}
%    \end{macrocode}

%\iffalse
%</samplefinal>
%\fi
%
% %%%%%%%%%%%%%%%%%%%%%%%%%%%%%%%%%%%%%%
% \paragraph{Command Line Processing.}
%
% The following three command lines generate the output files
% |cdocscld|, |cdocscl1| and |cdocscl2|
% which should be identical to
% |cdocsdrf|, |cdocsch1| and |cdocsfn2|, respectively:
% \begin{center}
% \begin{tabular}{l}
% |latex -jobname cdocscld \|\\
% |  "\def\version{draft}\input{childdoc.def}\childdocforward{cdocsamp}"|\\
% |latex -jobname cdocscl1 \|\\
% |  "\input{childdoc.def}\childdocforward[cdocsamp]{cdocsch1}"|\\
% |latex -jobname cdocscl2 \|\\
% |  "\def\version{final}\input{childdoc.def}\childdocforward{cdocsch2}"|
% \end{tabular}
% \end{center}
% Note that the trailing backslash on each first line
% merely continues the input to the second line
% (for convenient cut ant paste).
% Furthermore, the command |latex| can be replaced by any
% of its alternative versions such as |pdflatex|.
%
% %%%%%%%%%%%%%%%%%%%%%%%%%%%%%%%%%%%%%%%%%%%%%%%%%%%%%%%%%%%%%%%%%%%%%%%%%%%%%%
% %%%%%%%%%%%%%%%%%%%%%%%%%%%%%%%%%%%%%%%%%%%%%%%%%%%%%%%%%%%%%%%%%%%%%%%%%%%%%%
% \section{Implementation}
%\iffalse
%<*package>
%\fi
%
% This section describes the definitions file |childdoc.def|.

% The definitions cannot be loaded using |\usepackage| or |\RequirePackage|
% which has a mechanism to prevent loading a style file more than once.
% When loading the definitions by means of |\input|
% multiple instances have to be prevented manually:
%\iffalse
%This code needs to be before the `\ProvidesFile' directive
%which is defined at the beginning of this file.
%Therefore it is also placed there and commented out here.
%</package>
%<*discard>
%\fi
%    \begin{macrocode}
\ifdefined\childdocmain\endinput\fi
%    \end{macrocode}
%\iffalse
%</discard>
%<*package>
%\fi
%
% \macro{\ifchilddoc}
% \macro{\ifchilddocmanual}
% The conditional |\ifchilddoc| tells whether a
% child (true) or main (false) document is being compiled.
% The conditional |\ifchilddocmanual| tells whether
% the |\includeonly| mechanism is used (false) or
% the selection of child files must be performed manually (true).
% The definitions initialise to false:
%    \begin{macrocode}
\newif\ifchilddoc
\newif\ifchilddocmanual
%    \end{macrocode}

% \macro{\childdocname}
% \macro{\childdocjob}
% The macro |\childdocname| stores the name of the main document
% to be compiled. The macro |\childdocjob| stores the name of
% the document on which the \LaTeX{} compiler was originally invoked.
% The content of |\jobname| cannot be compared
% to filenames specified in the source due to different catcodes.
% The following code rescans |\jobname|, stores the result
% in |\childdocname| and saves a copy in |\childdocjob|:
%    \begin{macrocode}
\edef\childdocname{\scantokens\expandafter{\jobname\noexpand}}
\let\childdocjob\childdocname
%    \end{macrocode}

% \macro{\childdocdisable}
% The macro |\childdocdisable| prevents the main file
% from being processed more than once.
% At this stage, the main document command |\childdocmain|
% is assumed to be called once again where it should do nothing.
% Any subsequent call to it should prevent
% a secondary processing of the main document
% It overwrites the forwarding commands
% |\childdocof| and |\childdocforward|
% with empty macros to prevent further inclusions of the main document:
%    \begin{macrocode}
\newcommand{\childdocdisable}
{
  \renewcommand{\childdocmain}[1]{\renewcommand{\childdocmain}[1]{\endinput}}
  \renewcommand{\childdocof}[1]{}
  \renewcommand{\childdocby}[2][]{}
  \renewcommand{\childdocforward}[2][]{}
  \renewcommand{\childdocdisable}{}
}
%    \end{macrocode}

% \macro{\childdocmain}
% The macro |\childdocmain| is to be called at the top of the main file
% with nothing or the main filename (without extension) as argument.
% First, it breaks loops.
% If the argument is not empty and does not match |\childdocname|
% (which is set by the first inclusion of |childdoc.def|),
% |\ifchilddoc| is set to true, |\includeonly| is applied to the child file
% and |\jobname| is set to the main file
% (for proper handling of |.aux| files):
%    \begin{macrocode}
\newcommand{\childdocmain}[1]
{
  \childdocdisable\childdocmain{}
  \if?#1?\else
    \begingroup
      \def\childdoctmp{#1}
      \ifx\childdoctmp\childdocname
        \def\childdoctmp{}
      \else
        \def\childdoctmp
        {
          \childdoctrue
          \includeonly{\childdocname}
          \def\childdocjob{#1}
          \def\jobname{#1}
        }
      \fi
      \expandafter
    \endgroup
    \childdoctmp
  \fi
}
%    \end{macrocode}

% \macro{\childdocof}
% The command |\childdocof| redirects
% compilation to the main file |#1|.
%    \begin{macrocode}
\newcommand{\childdocof}[1]
{
  \childdocdisable
  \childdoctrue
  \includeonly{\childdocname}
  \def\jobname{#1}
  \def\childdocjob{#1}
  \input{#1}
}
%    \end{macrocode}

% \macro{\childdocby}
% The command |\childdocby| ....
%    \begin{macrocode}
\newcommand{\childdocby}[2][]
{
  \childdocdisable
  \childdoctrue
  \childdocmanualtrue
  \if?#1?\else
    \def\jobname{#2}
  \fi
  \def\childdocjob{#2}
  \input{#2}
  \endinput
}
%    \end{macrocode}

% \macro{\childdocforward}
% The command |\childdocforward| redirects
% compilation to the main file or
% (if the optional argument is given) a child file.
% Parameters are set as if the main file
% or a child file starting with |\childdocof| was compiled.
% Then compilation is handed over to the main file:
%    \begin{macrocode}
\newcommand{\childdocforward}[2][]
{
  \begingroup
    \if?#1?
      \def\childdoctmp
      {
        \def\childdocname{#2}
        \def\childdocjob{#2}
        \def\jobname{#2}
        \input{#2}
        \endinput
      }
    \else
      \def\childdoctmp
      {
        \childdocdisable
        \def\childdocname{#2}
        \childdoctrue
        \includeonly{#2}
        \def\childdocjob{#1}
        \def\jobname{#1}
        \input{#1}
        \endinput
      }
    \fi
    \expandafter
  \endgroup
  \childdoctmp
}
%    \end{macrocode}

% \macro{\childdocforwardprefix}
% The command |\childdocforwardprefix| redirects
% compilation to the main or a child file by means of a pattern.
% The prefix |#1| in the current filename is replaced by |#2|
% and the suffix of the current filename is kept
% (it is assumed that the filename does not contain the substring `|~~~|'
% which is used as a delimiter).
% Compilation is handed over to the new file by |\childdocforward|:
%    \begin{macrocode}
\newcommand{\childdocforwardprefix}[3][]
{
  \begingroup
    \def\childdocextract #2##1~~~{\def\childdoctmp{\childdocforward[#1]{#3##1}}}
    \expandafter\childdocextract\childdocname~~~
    \expandafter
  \endgroup
  \childdoctmp
}
%    \end{macrocode}

% \macro{\childdoc}
% The deprecated macro |\childdoc| is a legacy version of |\childdocmain|:
%    \begin{macrocode}
\newcommand{\childdoc}{\childdocmain}
%    \end{macrocode}

% \macro{\childdocredirect}
% The deprecated macro |\childdocredirect| is a legacy version
% of |\childdocforward| and |\childdocforwardprefix|:
%    \begin{macrocode}
\newcommand{\childdocredirect}[2][]
{
  \begingroup
    \if?#1?
      \def\childdoctmp{\childdocforward{#2}}
    \else
      \def\childdoctmp{\childdocforwardprefix{#1}{#2}}
    \fi
    \expandafter
  \endgroup
  \childdoctmp
}
%    \end{macrocode}

%\iffalse
%</package>
%\fi
%
\endinput

\childdocforwardprefix[cdocsamp]{cdocsfn}{cdocsch}
%    \end{macrocode}

%\iffalse
%</samplefinal>
%\fi
%
% %%%%%%%%%%%%%%%%%%%%%%%%%%%%%%%%%%%%%%
% \paragraph{Command Line Processing.}
%
% The following three command lines generate the output files
% |cdocscld|, |cdocscl1| and |cdocscl2|
% which should be identical to
% |cdocsdrf|, |cdocsch1| and |cdocsfn2|, respectively:
% \begin{center}
% \begin{tabular}{l}
% |latex -jobname cdocscld \|\\
% |  "\def\version{draft}% \iffalse
%
% childdoc.dtx Copyright (C) 2017-2018 Niklas Beisert
%
% This work may be distributed and/or modified under the
% conditions of the LaTeX Project Public License, either version 1.3
% of this license or (at your option) any later version.
% The latest version of this license is in
%   http://www.latex-project.org/lppl.txt
% and version 1.3 or later is part of all distributions of LaTeX
% version 2005/12/01 or later.
%
% This work has the LPPL maintenance status `maintained'.
%
% The Current Maintainer of this work is Niklas Beisert.
%
% This work consists of the files childdoc.dtx and childdoc.ins
% and the derived files childdoc.def and cdocsamp.tex with
% cdocsch1.tex, cdocsch2.tex, cdocsdrf.tex, cdocsfn1.tex, cdocsfn2.tex.
%
%<package>\ifdefined\childdocmain\endinput\fi
%<package>\ProvidesFile{childdoc.def}[2018/12/30 v2.0 child document driver]
%<samplemain>\ProvidesFile{cdocsamp.tex}[2018/12/30 v2.0 sample for childdoc]
%<*driver>
%\ProvidesFile{childdoc.drv}[2018/12/30 v2.0 childdoc reference manual file]
\PassOptionsToClass{10pt,a4paper}{article}
\documentclass{ltxdoc}

\usepackage[margin=35mm]{geometry}
\usepackage{hyperref}
\usepackage{hyperxmp}
\usepackage[usenames]{color}

\hypersetup{colorlinks=true}
\hypersetup{pdfstartview=FitH}
\hypersetup{pdfpagemode=UseNone}
\hypersetup{pdfsource={}}
\hypersetup{pdflang={en-UK}}
\hypersetup{pdfcopyright={Copyright 2017-2018 Niklas Beisert.
  This work may be distributed and/or modified under the
  conditions of the LaTeX Project Public License, either version 1.3
  of this license or (at your option) any later version.}}
\hypersetup{pdflicenseurl={http://www.latex-project.org/lppl.txt}}
\hypersetup{pdfcontactaddress={ETH Zurich, ITP, HIT K,
  Wolfgang-Pauli-Strasse 27}}
\hypersetup{pdfcontactpostcode={8093}}
\hypersetup{pdfcontactcity={Zurich}}
\hypersetup{pdfcontactcountry={Switzerland}}
\hypersetup{pdfcontactemail={nbeisert@itp.phys.ethz.ch}}
\hypersetup{pdfcontacturl={http://people.phys.ethz.ch/\xmptilde nbeisert/}}

\newcommand{\secref}[1]{\hyperref[#1]{section \ref*{#1}}}

\parskip1ex
\parindent0pt
\let\olditemize\itemize
\def\itemize{\olditemize\parskip0pt}

\begin{document}

\title{The \textsf{childdoc} Package}
\hypersetup{pdftitle={The childdoc Package}}
\author{Niklas Beisert\\[2ex]
  Institut f\"ur Theoretische Physik\\
  Eidgen\"ossische Technische Hochschule Z\"urich\\
  Wolfgang-Pauli-Strasse 27, 8093 Z\"urich, Switzerland\\[1ex]
  \href{mailto:nbeisert@itp.phys.ethz.ch}
  {\texttt{nbeisert@itp.phys.ethz.ch}}}
\hypersetup{pdfauthor={Niklas Beisert}}
\hypersetup{pdfsubject={Manual for the LaTeX2e Package childdoc}}
\date{30 December 2018, \textsf{v2.0}}
\maketitle

\begin{abstract}\noindent
\textsf{childdoc} is a \LaTeXe{} package
that enables the direct compilation
of document sections included by |\include|
to individual files.
\end{abstract}

\begingroup
\parskip0ex
\tableofcontents
\endgroup

%%%%%%%%%%%%%%%%%%%%%%%%%%%%%%%%%%%%%%%%%%%%%%%%%%%%%%%%%%%%%%%%%%%%%%%%%%%%%%%%
%%%%%%%%%%%%%%%%%%%%%%%%%%%%%%%%%%%%%%%%%%%%%%%%%%%%%%%%%%%%%%%%%%%%%%%%%%%%%%%%
\section{Introduction}

\LaTeX{} provides a mechanism to structure a large document (such as a book)
into a main file and several child files (containing the chapters)
using the |\include| command.
This mechanism is beneficial for documents
which span hundreds of pages in order to
make the source file(s) more manageable.
Moreover, compilation can be restricted to
selected child files by means of the |\includeonly| command.
The latter feature can be used to reduce the compilation time while editing
(this was significantly more useful in the earlier days of \LaTeX{})
or to generate a smaller document which is easier to navigate.
Another application of |\includeonly| is to generate
documents consisting of selected parts of the complete document.

However, there are a few drawbacks of the plain |\include| mechanism:
\begin{itemize}
\item
The child files cannot be compiled on their own,
they can only be compiled via the main file.
A naive editing environment
(such as a text editor with an option
to have the current file processed by \LaTeX)
may require one to switch to the main file before compiling;
attempting to compile the child file produces errors.
\item
The main file must be modified (each time)
to adjust the |\includeonly| command
to the present needs. This easily leaves the main file in a messy state.
\item
The generated document will always carry the filename
of the main document. This is inconvenient if
several child files are to be compiled and
to be kept for distribution.
\end{itemize}

The present package provides a simple interface
to make child files individually compilable by \LaTeX{}.
Compiling a child file then has the same effect as compiling
the main file with an |\includeonly| command
to select the appropriate child.
Moreover the generated document will carry the name of the child
rather than the main file.
This resolves all three above issues.

This feature is meant to make the editing of books,
thesis documents and lecture notes somewhat more convenient.
However, the package can also be used efficiently for
composing a series of documents (such as exercise sheets)
which are typically distributed individually.
It then assists the author in generating the individual documents
(potentially in different versions)
as well as a document containing the collected series.
Another application is in developing style files
or other kinds of included material
where compilation of the style file could redirect
to a sample or test file.

%%%%%%%%%%%%%%%%%%%%%%%%%%%%%%%%%%%%%%%%%%%%%%%%%%%%%%%%%%%%%%%%%%%%%%%%%%%%%%%%
%%%%%%%%%%%%%%%%%%%%%%%%%%%%%%%%%%%%%%%%%%%%%%%%%%%%%%%%%%%%%%%%%%%%%%%%%%%%%%%%
\section{Usage}

First of all, the package \textsf{childdoc} is \emph{not} a standard
\LaTeXe{} |.sty| style file! Therefore it needs to be invoked in
a non-standard way.

%%%%%%%%%%%%%%%%%%%%%%%%%%%%%%%%%%%%%%%%%%%%%%%%%%%%%%%%%%%%%%%%%%%%%%%%%%%%%%%%
\subsection{Included Files}
\label{sec:include}

%%%%%%%%%%%%%%%%%%%%%%%%%%%%%%%%%%%%%%%%
\DescribeMacro{\childdocmain}
To use the package, add the commands
\begin{center}
\begin{tabular}{l}
|\input{childdoc.def}|\\
|\childdocmain{}|\\
\end{tabular}
\end{center}
at the very top of the main \LaTeX{} file,
in particular \emph{before} the |\documentclass| statement!
The argument of |\childdocmain| should be left empty
(but it must be present).

%%%%%%%%%%%%%%%%%%%%%%%%%%%%%%%%%%%%%%%%
\DescribeMacro{\childdocof}
Furthermore, add the commands
\begin{center}
\begin{tabular}{l}
|\input{childdoc.def}|\\
|\childdocof{|\textit{main}|}|\\
\end{tabular}
\end{center}
at the top of every child file \textit{child}
which is included by |\include{|\textit{child}|}|
from within the main file
(or at least for those files to be compiled individually).
The argument \textit{main} must be the filename of the main file.

There are a couple of
considerations in setting up the main and child documents:

%%%%%%%%%%%%%%%%%%%%%%%%%%%%%%%%%%%%%%%%
\paragraph{Restrictions.}

Please note the following restrictions:
\begin{itemize}
\item
|\childdocmain| must be called with one argument \textit{main}
to ensure compatibility with earlier version of the package.
It must either be empty (|\childdocmain{}|)
or precisely match the filename of the main file in which it is specified.
See \secref{sec:detection} for further information.
\item
The filename \textit{main} must be specified without the |.tex| extension.
\item
The filename \textit{main} is case sensitive
(even in case-insensitive file systems)
due to internal string comparison.
\item
The argument \textit{main} should be fully expanded, it cannot be a macro.
\item
Subdirectories and special characters should be avoided in filenames.
\item
The command |\childdocmain{|\textit{main}|}| must be followed by a whitespace.
It should not be followed immediately by another command
or by a comment mark `|%|'.
This is because the \TeX{} parser reads the token immediately following
the argument of |\childdocmain| and puts it
at the beginning of every child section;
however, a white\-space is ignored.
\end{itemize}

%%%%%%%%%%%%%%%%%%%%%%%%%%%%%%%%%%%%%%%%
\paragraph{Content of Main File.}

It is advisable to place all content in the child files included by |\include|.
Any output contained in the main file will appear in all child documents
unless suppressed manually;
it cannot be suppressed automatically by the |\includeonly| directive
and thus should normally be avoided.
A method to include some content in the main file
by means of conditional processing is described in \secref{sec:conditional}.

%%%%%%%%%%%%%%%%%%%%%%%%%%%%%%%%%%%%%%%%
\paragraph{Page Numbering.}

When only a part of the document is compiled,
the appropriate numbering of pages
(as well as other status parameters)
is determined from the |.aux| files.
The latter contain information from previous passes.
However this information needs to propagate through
all intermediate child documents.
Therefore the page numbering in child documents may well
be inconsistent until the complete document is compiled at least once.

A useful (if unconventional) way to always ensure a consistent
page numbering is to restart the numbering in each child document
and denote the pages by `\textit{child}|.|\textit{page}'
where \textit{child} represents the chapter/section number of the child file.
This can be achieved by the command
|\numberwithin{page}{|\textit{child}|}|
of the \textsf{amsmath} package
where \textit{child} can be |chapter| or |section|
depending on the chosen structuring.
Alternatively, one can modify the macro |\thepage| appropriately
and reset the counter |page| at the start of each child file.

%%%%%%%%%%%%%%%%%%%%%%%%%%%%%%%%%%%%%%%%%%%%%%%%%%%%%%%%%%%%%%%%%%%%%%%%%%%%%%%%
\subsection{Conditional Processing}
\label{sec:conditional}

The package provides a mechanism to compile different versions
of a document. To customise the versions further some conditional processing
can come in handy to distinguish which version is being compiled.
The package provides two macros to describe the compilation context:

%%%%%%%%%%%%%%%%%%%%%%%%%%%%%%%%%%%%%%%%
\DescribeMacro{\ifchilddoc}
The conditional |\ifchilddoc| distinguishes between the compilation of
child documents and the main document:
%
\begin{center}
|\ifchilddoc |\textit{child-code}| |[|\||else |\textit{main-code}]| \||fi|
\end{center}

%%%%%%%%%%%%%%%%%%%%%%%%%%%%%%%%%%%%%%%%
\DescribeMacro{\childdocname}
\DescribeMacro{\childdocjob}
The macro |\childdocname| contains the filename (without extension)
of the main or child file being processed.
Note that |\childdocjob| will always contain the name of the main file.

%%%%%%%%%%%%%%%%%%%%%%%%%%%%%%%%%%%%%%%%
\paragraph{Title Page.}

Conditional processing can be used to include a title or banner page
in the main document when proper precautions are taken.
Importantly, the code in the main file should ensure that the page counter
(as well as other status parameters which are stored in the |.aux| files)
takes the same value after the conditional processing.
Otherwise the page numbers may take divergent values
depending on which part is compiled.

For example, a title page could be declared by:
%
\begin{center}
\begin{tabular}{l}
|\ifchilddoc\||else|\\
|\addtocounter{page}{-1}|\\
\textit{code for title page}\\
|\newpage|\\
|\||fi|
\end{tabular}
\end{center}
%
A banner page for the child documents can be generated by:
%
\begin{center}
\begin{tabular}{l}
|\ifchilddoc|\\
|\addtocounter{page}{-1}|\\
\textit{code for banner page}\\
|\newpage|\\
|\||fi|
\end{tabular}
\end{center}
%
Here one could write a message such as:
\begin{center}
|This is the part \childdocname{} of \childdocjob{}.|
\end{center}

%%%%%%%%%%%%%%%%%%%%%%%%%%%%%%%%%%%%%%%%%%%%%%%%%%%%%%%%%%%%%%%%%%%%%%%%%%%%%%%%
\subsection{Flags}
\label{sec:flags}

The package makes it easy to generate different versions
of the main or child documents.
To this end compilation flags can be defined
and assigned different default values.
They will be particularly useful in conjunction
with the forwarding mechanism described in \secref{sec:forward}.

For example, it may be useful to have a flag |\version|
which can be set to |draft| or |final|.
The document source will contain some conditional code
depending on the value of |\version|.
Suppose further, the flag should default to |final| for the main file
and to |draft| for child files
which is a natural assignment for editing the document.
This is achieved by placing the following code
in the preamble of the main document
(below the |\childdocmain| directive):
%
\begin{center}
\begin{tabular}{l}
|\ifchilddoc|\\
|\providecommand{\version}{draft}|\\
|\||else|\\
|\providecommand{\version}{final}|\\
|\||fi|
\end{tabular}
\end{center}
%
The definition by |\providecommand| makes sure
that previous definitions are not overwritten.
Further statements |\providecommand{\version}{...}|
can thus be added before the above code to override it.

For the main file, one might add a line
(between |\childdocmain| and the above block)
%
\begin{center}
|%\ifchilddoc\||else\providecommand{\version}{draft}\||fi|
\end{center}
%
which can be uncommented to produce a draft version.
Likewise one can add a line to the very top of a child file
(above the |\childdocof{|\textit{main}|}| directive)
%
\begin{center}
|%\providecommand{\version}{final}|
\end{center}
%
which can be uncommented to produce the final version of this child document.

%%%%%%%%%%%%%%%%%%%%%%%%%%%%%%%%%%%%%%%%%%%%%%%%%%%%%%%%%%%%%%%%%%%%%%%%%%%%%%%%
\subsection{Forwarding}
\label{sec:forward}

Different versions of the main or child documents
using compilation flags as described in \secref{sec:flags}
can be (permanently) stored in different files
for convenient compilation, viewing and distribution.
To this end, the package defines a command
to pass on compilation to a different file:

%%%%%%%%%%%%%%%%%%%%%%%%%%%%%%%%%%%%%%%%
\DescribeMacro{\childdocforward}
The command |\childdocforward| redirects processing to
another source file:
%
\begin{center}
\begin{tabular}{l}
|\input{childdoc.def}|\\
|\childdocforward[|\textit{main}|]{|\textit{dest}|}|\\
\end{tabular}
\end{center}
%
The argument \textit{dest} is the destination file
(without extension).
It should be the main file or one of the child files.
Note that further \textsf{childdoc} directives
such as |\childdocof| and |\childdocforward|
in the indicated file will be processed in this form.
The optional argument \textit{main}
passes on directly to the main file \textit{main}
while pretending to compile the child \textit{dest}.
This form behaves as if \textit{dest}
issues |\childdocof{|\textit{main}|}| right away,
and no further \textsf{childdoc} directives will be processed.

%%%%%%%%%%%%%%%%%%%%%%%%%%%%%%%%%%%%%%%%
\DescribeMacro{\...prefix}
In the alternative form |\childdocforwardprefix|,
%
\begin{center}
\begin{tabular}{l}
|\input{childdoc.def}|\\
|\childdocforwardprefix[|\textit{main}|]{|\textit{prefix}|}{|\textit{dest}|}|
\end{tabular}
\end{center}
%
the destination file is determined by a pattern
depending on the current file:
To make this work, the current file must be called
`{\textit{prefix}\hspace{0.2em}\textit{suffix}}'
with \textit{prefix} matching precisely the argument.
Processing is then passed on to the file
`{\textit{dest}\hspace{0.2em}\textit{suffix}}'.
Surely, the same effect is achieved by
directly specifying the
argument `{\textit{dest}\hspace{0.2em}\textit{suffix}}'
in the first form.
However, that requires to set up a different file
for each child. With the alternative form of the command
all these files can have exactly the same content
which simplifies setting them up and maintaining them.

For example, the following file |draft.tex|
with a compilation flag |\version| as described in \secref{sec:flags}
compiles the main document as a draft:
%
\begin{center}
\begin{tabular}{l}
|\def\version{draft}|\\
|\input{childdoc.def}|\\
|\childdocforward{|\textit{main}|}|
\end{tabular}
\end{center}
%
Likewise, the following files |final|\textit{nn}|.tex|
compile the final version of the child document
|child|\textit{nn}|.tex|:
%
\begin{center}
\begin{tabular}{l}
|\def\version{final}|\\
|\input{childdoc.def}|\\
|\childdocforwardprefix{final}{child}|
\end{tabular}
\end{center}
%

Note that when several versions of a main file and/or of each child file
are to be generated, it may be convenient to set up a |Makefile| or
shell script to automatise the process.

%%%%%%%%%%%%%%%%%%%%%%%%%%%%%%%%%%%%%%%%%%%%%%%%%%%%%%%%%%%%%%%%%%%%%%%%%%%%%%%%
\subsection{Command Line Processing}
\label{sec:commandline}

The effect of redirection files can also be achieved by invoking
the \LaTeX{} compiler with a more elaborate command line.
Most conveniently this should be done as part
of a shell script or a |Makefile|.

When using \textsf{childdoc} in the main file, the following
command lines effectively perform a redirection
(note that depending on the shell being used,
backslashes may have to be doubled: `|\|' $\to$ `|\\|'):
%
\begin{center}
|... -jobname "|\textit{target}|" |\\|"|[\textit{flags}]%
|\input{childdoc.def}\childdocforward[|\textit{main}|]{|\textit{dest}|}"|
\end{center}
%
Here \textit{target} is the name of the output file,
\textit{main} is the name of the main file
and \textit{dest} is the name of the main or child file to be processed
(all filenames without extensions).
The optional argument \textit{main} can be omitted
if \textit{main} matches \textit{dest}.
Optionally, compilation \textit{flags} can be defined via |\def| commands.
This command line makes the \TeX{} engine believe
it is compiling the file \textit{target}
whose content is specified as the latter parameter.
The provided code then forwards the processing to
\textit{main} or \textit{dest} as described in \secref{sec:forward}.

%%%%%%%%%%%%%%%%%%%%%%%%%%%%%%%%%%%%%%%%%%%%%%%%%%%%%%%%%%%%%%%%%%%%%%%%%%%%%%%%
\subsection{Include by Input}
\label{sec:input}

Including child documents by |\include| has some restrictions by design.
Most notably, the content of a child document always occupies
its own set of pages; pages cannot be shared between child documents.
Usually, this behaviour makes perfect sense
because each child document contain an essential part of the document.
However, in some situations it may be desirable to compose
a document from a collection of parts
without having mandatory page breaks between then.
For this case, the package
provides a mechanism to include parts
by |\input| which can also be processed individually.
However, by construction this mechanism
requires manual handling of the content to be output.

%%%%%%%%%%%%%%%%%%%%%%%%%%%%%%%%%%%%%%%%
\DescribeMacro{\ifchilddocmanual}
The main file should be prepared as usual, see \secref{sec:include}.
However, the document body must make a distinction
between processing of an individual part and of the main document, e.g.:
%
\begin{center}
\begin{tabular}{l}
|\ifchilddocmanual|\\
|\input{\childdocname}|\\
|\||else|\\
\textit{document body with }|\input{|\textit{part}|}|\\
|\||fi|
\end{tabular}
\end{center}
%
The conditional |\ifchilddocmanual| is true whenever
a part to be included by |\input| is being compiled,
and the name of the part is stored in |\childdocname|.

%%%%%%%%%%%%%%%%%%%%%%%%%%%%%%%%%%%%%%%%
\DescribeMacro{\childdocby}
Each part to be included by |\input| should start with:
%
\begin{center}
\begin{tabular}{l}
|\input{childdoc.def}|\\
|\childdocby{|\textit{main}|}|\\
\end{tabular}
\end{center}
%
The directive |\childdocby| is similar to |\childdocof|
described in \secref{sec:include},
but the subsequent selection of content must be done manually.
To that end, both |\ifchilddoc| and |\ifchilddocmanual|
will be true upon processing of a part,
and the name of the part is stored in |\childdocname|.
Note that |\jobname| will be set to the filename of the current part
so that each part receives an individual |.aux| file
that does not interfere with the |.aux| file(s) of the main document.
This behaviour can be altered by the alternative form
|\childdocby[*]{|\textit{main}|}| (with a non-empty optional argument)
which uses the |.aux| file of the main document
by setting |\jobname| to \textit{main}.

%%%%%%%%%%%%%%%%%%%%%%%%%%%%%%%%%%%%%%%%%%%%%%%%%%%%%%%%%%%%%%%%%%%%%%%%%%%%%%%%
\subsection{Driver Development}
\label{sec:driver}

The \textsf{childdoc} mechanism can also be use for the development
of definition files such as \LaTeX{} styles or classes.
This case differs from the above setup with multiple parts
included by |\include| in that no |\includeonly| should be invoked.
This can be achieved by starting the include file
(before |\ProvidesPackage|) with:
%
\begin{center}
\begin{tabular}{l}
|\input{childdoc.def}|\\
|\childdocforward{|\textit{main}|}|\\
\end{tabular}
\end{center}
%
or alternatively with:
%
\begin{center}
\begin{tabular}{l}
|\input{childdoc.def}|\\
|\childdocby{|\textit{main}|}|\\
\end{tabular}
\end{center}
%
Both forms have slightly different effects as described above.
The main file is prepared as usual, see \secref{sec:include}.

%%%%%%%%%%%%%%%%%%%%%%%%%%%%%%%%%%%%%%%%%%%%%%%%%%%%%%%%%%%%%%%%%%%%%%%%%%%%%%%%
\subsection{Legacy Detection}
\label{sec:detection}

The directive |\childdocmain| in the main file can detect
whether the complete document or merely a child is to be compiled
even without using the directive |\childdocof|.
This method is deprecated because it is less robust
and there is no compelling reason to use it;
it is merely provided for backward compatibility
and it may be removed in future versions.

If the detection mechanism is to be used,
it is mandatory to correctly specify
the filename of the main file as the argument of |\childdocmain|:
%
\begin{center}
\begin{tabular}{l}
|\input{childdoc.def}|\\
|\childdocmain{|\textit{main}|}|\\
\end{tabular}
\end{center}
%
If |\jobname| does not match the argument \textit{main} of |\childdocmain|,
it is assumed that |\jobname| points to the child file to be compiled.
When using |\childdocmain| with the main file specified as argument,
it suffices to start a child file
with just |\input{|\textit{main}|}|
without loading of the package and using |\childdocof|.
If instead all processing is done
with the appropriate \textsf{childdoc} directives,
the argument of \textit{main} of |\childdocmain| can be empty.

An alternative version of the command line processing described
in \secref{sec:commandline} using the detection mechanism reads:
%
\begin{center}
|... -jobname "|\textit{target}|" "|[\textit{flags}]%
[|\def\jobname{|\textit{dest}|}|]|\input{|\textit{main}|}"|
\end{center}

%%%%%%%%%%%%%%%%%%%%%%%%%%%%%%%%%%%%%%%%%%%%%%%%%%%%%%%%%%%%%%%%%%%%%%%%%%%%%%%%
\subsection{Manual Code}
\label{sec:manual}

In case one cannot be certain whether the definitions file |childdoc.def|
is installed on the target \TeX{} distribution
and one prefers not to ship it,
it is conceivable to paste a few relevant commands into the sources.

To that end, drop all statements |\input{childdoc.def}|
and perform the replacements as outlined below.
Instead of |\childdocmain{|\textit{main}|}| add the following code
to the top of the main file:
%
\begin{center}
\begin{tabular}{l}
|\||ifdefined\childdocname\endinput\||fi\newif\ifchilddoc|\\
|\edef\childdocname{\scantokens\expandafter{\jobname\noexpand}}|\\
|\def\childdocmain{|\textit{main}|}\||ifx\childdocmain\childdocname\||else|\\
|\childdoctrue\includeonly{\childdocname}\let\jobname\childdocmain\||fi|\\
\end{tabular}
\end{center}
%
Instead of |\childdocof{|\textit{main}|}| just include the main file
at the top of each child file:
%
\begin{center}
|\input{|\textit{main}|}|
\end{center}
%
A simple redirection |\childdocforward{|\textit{dest}|}| is achieved by:
%
\begin{center}
|\def\jobname{|\textit{dest}|}\input{\jobname}|
\end{center}
%
The redirection with prefix
|\childdocforwardprefix[|\textit{prefix}|]{|\textit{dest}|}|
is accomplished by:
%
\begin{center}
\begin{tabular}{l}
|{\edef\jobname{\scantokens\expandafter{\jobname\noexpand}}|\\
|\def\redirectjob |\textit{prefix}|#1~~~{\gdef\jobname{|\textit{dest}|#1}}|\\
|\expandafter\redirectjob\jobname~~~}\input{\jobname}|
\end{tabular}
\end{center}

In an alternative approach,
child documents can be compiled by a specific command line
without additional code or specific definitions:
%
\begin{center}
|... -jobname "|\textit{target}|" "|[\textit{flags}]%
|\includeonly{|\textit{dest}|}\input{|\textit{main}|}"|
\end{center}
%

%%%%%%%%%%%%%%%%%%%%%%%%%%%%%%%%%%%%%%%%%%%%%%%%%%%%%%%%%%%%%%%%%%%%%%%%%%%%%%%%
%%%%%%%%%%%%%%%%%%%%%%%%%%%%%%%%%%%%%%%%%%%%%%%%%%%%%%%%%%%%%%%%%%%%%%%%%%%%%%%%
\section{Information}

%%%%%%%%%%%%%%%%%%%%%%%%%%%%%%%%%%%%%%%%%%%%%%%%%%%%%%%%%%%%%%%%%%%%%%%%%%%%%%%%
\subsection{Copyright}

Copyright \copyright{} 2017--2018 Niklas Beisert

This work may be distributed and/or modified under the
conditions of the \LaTeX{} Project Public License, either version 1.3
of this license or (at your option) any later version.
The latest version of this license is in
  \url{http://www.latex-project.org/lppl.txt}
and version 1.3 or later is part of all distributions of \LaTeX{}
version 2005/12/01 or later.

This work has the LPPL maintenance status `maintained'.

The Current Maintainer of this work is Niklas Beisert.

This work consists of the files |README.txt|, |childdoc.ins| and |childdoc.dtx|
as well as the derived files |childdoc.def|, |cdocsamp.tex|
with |cdocsch1.tex|, |cdocsch2.tex|, |cdocspt3.tex|, |cdocspt4.tex|,
|cdocsdrf.tex|, |cdocsfn1.tex|, |cdocsfn2.tex|
as well as |childdoc.pdf|.

%%%%%%%%%%%%%%%%%%%%%%%%%%%%%%%%%%%%%%%%%%%%%%%%%%%%%%%%%%%%%%%%%%%%%%%%%%%%%%%%
\subsection{Files and Installation}

The package consists of the files:
%
\begin{center}
\begin{tabular}{ll}
    |README.txt|   & readme file \\
    |childdoc.ins| & installation file \\
    |childdoc.dtx| & source file \\
    |childdoc.def| & definition file \\
    |cdocsamp.tex| & sample main file \\
    |cdocsch1.tex| & sample include file \\
    |cdocsch2.tex| & sample include file \\
    |cdocspt3.tex| & sample part file \\
    |cdocspt4.tex| & sample part file \\
    |cdocsdrf.tex| & sample redirection file \\
    |cdocsfn1.tex| & sample redirection file \\
    |cdocsfn2.tex| & sample redirection file \\
    |childdoc.pdf| & manual
\end{tabular}
\end{center}
%
The distribution consists of the files
|README.txt|, |childdoc.ins| and |childdoc.dtx|.
%
\begin{itemize}
\item
Run (pdf)\LaTeX{} on |childdoc.dtx|
to compile the manual |childdoc.pdf| (this file).
\item
Run \LaTeX{} on |childdoc.ins| to create the definitions file |childdoc.def|
and the sample |cdocsamp.tex| with include files
|cdocsch1.tex|, |cdocsch2.tex|, |cdocspt3.tex|, |cdocspt4.tex|,
|cdocsdrf.tex|, |cdocsfn1.tex|, |cdocsfn2.tex|.
Then copy the file |childdoc.def| to an appropriate directory of your \LaTeX{}
distribution, e.g.\ \textit{texmf-root}|/tex/latex/childdoc|.
\end{itemize}

%%%%%%%%%%%%%%%%%%%%%%%%%%%%%%%%%%%%%%%%%%%%%%%%%%%%%%%%%%%%%%%%%%%%%%%%%%%%%%%%
\subsection{Related CTAN Packages}

There are several other packages which offer a similar functionality:
%
\begin{itemize}
\item
The packages
\href{http://ctan.org/pkg/docmute}{\textsf{docmute}},
\href{http://ctan.org/pkg/includex}{\textsf{includex}} and
\href{http://ctan.org/pkg/standalone}{\textsf{standalone}}
provide commands to include only the document body of
a child file thus allowing both files to be compiled individually.
\item
The packages \href{http://ctan.org/pkg/subdocs}{\textsf{subdocs}}
and \href{http://ctan.org/pkg/subfiles}{\textsf{subfiles}}
provide structures in which the main and child documents can be
encapsulated and allowing them to be compiled individually.
The inclusion mechanism is different from the conventional |\include|.
\item
The package \href{http://ctan.org/pkg/combine}{\textsf{combine}}
is an elaborate solution to combine several documents into one.
\end{itemize}
%
See also the CTAN topic \href{http://ctan.org/topic/subdocs}{\textsf{subdocs}}
for further related packages.
The present package differs from the above solutions in that
a document structure constructed with the conventional |\include| mechanism
just needs two extra commands at the top of every file
such that all constituent files can be compiled individually.

%%%%%%%%%%%%%%%%%%%%%%%%%%%%%%%%%%%%%%%%%%%%%%%%%%%%%%%%%%%%%%%%%%%%%%%%%%%%%%%%
%\subsection{Feature Suggestions}
%
%The following is a list of features which may be useful for future
%versions of this package:
%%
%\begin{itemize}
%\item
%\ldots
%\end{itemize}

%%%%%%%%%%%%%%%%%%%%%%%%%%%%%%%%%%%%%%%%%%%%%%%%%%%%%%%%%%%%%%%%%%%%%%%%%%%%%%%%
\subsection{Revision History}

%%%%%%%%%%%%%%%%%%%%%%%%%%%%%%%%%%%%%%%%
\paragraph{v2.0:} 2018/12/30

\begin{itemize}
\item
immediate forward processing
\item
added |\childdocby| mechanism
\item
manual restructured
\end{itemize}

%%%%%%%%%%%%%%%%%%%%%%%%%%%%%%%%%%%%%%%%
\paragraph{v1.6:} 2018/01/17

\begin{itemize}
\item
application for development of include files
\item
corrections to manual
\end{itemize}

%%%%%%%%%%%%%%%%%%%%%%%%%%%%%%%%%%%%%%%%
\paragraph{v1.5:} 2017/05/21

\begin{itemize}
\item
more complete structuring introduced
\item
|\childdocof| introduced
\item
|\childdoc| renamed to |\childdocmain|
\item
|\childredirect| renamed to |\childdocforward| and |\childdocforwardprefix|
and functionality expanded
\end{itemize}

%%%%%%%%%%%%%%%%%%%%%%%%%%%%%%%%%%%%%%%%
\paragraph{v1.0:} 2017/04/27

\begin{itemize}
\item
manual and install package
\item
first version published on CTAN
\end{itemize}

%%%%%%%%%%%%%%%%%%%%%%%%%%%%%%%%%%%%%%%%
\paragraph{v0.6:} 2017/04/26

\begin{itemize}
\item
redirection mechanism added
\end{itemize}

%%%%%%%%%%%%%%%%%%%%%%%%%%%%%%%%%%%%%%%%
\paragraph{v0.5:} 2017/04/26

\begin{itemize}
\item
functionality in definition file
\end{itemize}


%%%%%%%%%%%%%%%%%%%%%%%%%%%%%%%%%%%%%%%%%%%%%%%%%%%%%%%%%%%%%%%%%%%%%%%%%%%%%%%%
%%%%%%%%%%%%%%%%%%%%%%%%%%%%%%%%%%%%%%%%%%%%%%%%%%%%%%%%%%%%%%%%%%%%%%%%%%%%%%%%
%%%%%%%%%%%%%%%%%%%%%%%%%%%%%%%%%%%%%%%%%%%%%%%%%%%%%%%%%%%%%%%%%%%%%%%%%%%%%%%%
\appendix

\settowidth\MacroIndent{\rmfamily\scriptsize 000\ }

 \DocInput{childdoc.dtx}

\end{document}
%</driver>
% \fi
%
% %%%%%%%%%%%%%%%%%%%%%%%%%%%%%%%%%%%%%%%%%%%%%%%%%%%%%%%%%%%%%%%%%%%%%%%%%%%%%%
% %%%%%%%%%%%%%%%%%%%%%%%%%%%%%%%%%%%%%%%%%%%%%%%%%%%%%%%%%%%%%%%%%%%%%%%%%%%%%%
% \section{Sample}
%\iffalse
%<*samplemain>
%\fi
%
% The following presents a sample document
% with two chapters, two parts, a title page,
% a compile flag as well as three forwarding files to set the flag.
% It consists of eight |.tex| files:
% \begin{center}
% \begin{tabular}{ll}
% |cdocsamp.tex|&main file\\
% |cdocsch1.tex|&include file for chapter 1\\
% |cdocsch2.tex|&include file for chapter 2\\
% |cdocspt3.tex|&include file for part 3\\
% |cdocspt4.tex|&include file for part 4\\
% |cdocsdrf.tex|&forwarding file for main file in draft mode\\
% |cdocsfi1.tex|&forwarding file for final version of chapter 1\\
% |cdocsfi2.tex|&forwarding file for final version of chapter 2\\
% \end{tabular}
% \end{center}
% Each of the eight files can be compiled directly by the \LaTeX{} compiler.
%
% %%%%%%%%%%%%%%%%%%%%%%%%%%%%%%%%%%%%%%
% \paragraph{Main File.}
%
% The main file is called |cdocsamp.tex|.
%
% Load the \textsf{childdoc} definitions and
% declare the filename for the main document:
%    \begin{macrocode}
\input{childdoc.def}
\childdocmain{}
%    \end{macrocode}

% Optional override for |\version| flag:
%    \begin{macrocode}
%%\ifchilddoc\else\providecommand{\version}{draft}\fi
%    \end{macrocode}

% Define the default values for the |\version| flag
% (|final| for the main file and |draft| for childs):
%    \begin{macrocode}
\ifchilddoc
\providecommand{\version}{draft}
\else
\providecommand{\version}{final}
\fi
%    \end{macrocode}

% Load the standard document class:
%    \begin{macrocode}
\documentclass[12pt]{article}
%    \end{macrocode}

% Start the document body:
%    \begin{macrocode}
\begin{document}
%    \end{macrocode}

% Declare a title page.
% Print title, part of document being processed and version flag:
%    \begin{macrocode}
\addtocounter{page}{-1}
\begin{center}
{\LARGE\bfseries{}childdoc example\par}
\vspace{1cm}
\ifchilddoc
\ifchilddocmanual part\else chapter\fi:
`\childdocname' of `\childdocjob'\par
\else
main document: `\childdocjob'\par
\fi
version: \version\par
\end{center}
\newpage
%    \end{macrocode}

% Manually include selected file,
% otherwise process as usual:
%    \begin{macrocode}
\ifchilddocmanual
\section*{part `\childdocname'}
\input{\childdocname}
\else
%    \end{macrocode}

% Include the two chapters:
%    \begin{macrocode}
\include{cdocsch1}
\include{cdocsch2}
%    \end{macrocode}

% Include the two parts unless only chapters should be displayed:
%    \begin{macrocode}
\ifchilddoc\else
\section{part three}
\input{cdocspt3}
\section{part four}
\input{cdocspt4}
\fi
%    \end{macrocode}

% Process as usual until here:
%    \begin{macrocode}
\fi
%    \end{macrocode}

% End of document body:
%    \begin{macrocode}
\end{document}
%    \end{macrocode}
%\iffalse
%</samplemain>
%\fi
%
% %%%%%%%%%%%%%%%%%%%%%%%%%%%%%%%%%%%%%%
% \paragraph{Chapter Include Files.}
%
% The include files are called |cdocsch1.tex| and |cdocsch2.tex|.
%
%\iffalse
%<*samplechap1|samplechap2>
%\fi

% Optional override for |\version| flag:
%    \begin{macrocode}
%%\providecommand{\version}{final}
%    \end{macrocode}

% Include the main document:
%    \begin{macrocode}
\input{childdoc.def}
\childdocof{cdocsamp}
%    \end{macrocode}

%\iffalse
%</samplechap1|samplechap2>
%\fi
%
%\iffalse
%<*samplechap1>
%\fi
% Some text for chapter 1:
%    \begin{macrocode}
\section{one}
some text in chapter one
%    \end{macrocode}

%\iffalse
%</samplechap1>
%\fi
% Some text for chapter 2:
%\iffalse
%<*samplechap2>
%\fi
%    \begin{macrocode}
\section{two}
more text in chapter two
%    \end{macrocode}

%\iffalse
%</samplechap2>
%\fi
%
% %%%%%%%%%%%%%%%%%%%%%%%%%%%%%%%%%%%%%%
% \paragraph{Part Include Files.}
%
% The include files are called |cdocspt3.tex| and |cdocspt4.tex|.
%
%\iffalse
%<*samplepart3|samplepart4>
%\fi

% Optional override for |\version| flag:
%    \begin{macrocode}
%%\providecommand{\version}{final}
%    \end{macrocode}

% Include the main document:
%    \begin{macrocode}
\input{childdoc.def}
\childdocby{cdocsamp}
%    \end{macrocode}

%\iffalse
%</samplepart3|samplepart4>
%\fi
%
%\iffalse
%<*samplepart3>
%\fi
% Some text for part 3:
%    \begin{macrocode}
some text in part three
%    \end{macrocode}

%\iffalse
%</samplepart3>
%\fi
% Some text for part 4:
%\iffalse
%<*samplepart4>
%\fi
%    \begin{macrocode}
more text in part four
%    \end{macrocode}

%\iffalse
%</samplepart4>
%\fi
%
% %%%%%%%%%%%%%%%%%%%%%%%%%%%%%%%%%%%%%%
% \paragraph{Forwarding for a Complete Draft.}
%
% The following forwarding file |cdocsdrf.tex|
% compiles the main document in draft mode:
%\iffalse
%<*sampledraft>
%\fi
%    \begin{macrocode}
\def\version{draft}
\input{childdoc.def}
\childdocforward{cdocsamp}
%    \end{macrocode}

%\iffalse
%</sampledraft>
%\fi
%
% %%%%%%%%%%%%%%%%%%%%%%%%%%%%%%%%%%%%%%
% \paragraph{Forwarding for Final Version of the Chapters.}
%
% The following forwarding files |cdocsfn1.tex| and |cdocsfn2.tex|
% (with identical content)
% compile the final versions of the child documents
% |cdocsch1.tex| and |cdocsch2.tex|, respectively:
%\iffalse
%<*samplefinal>
%\fi
%    \begin{macrocode}
\def\version{final}
\input{childdoc.def}
\childdocforwardprefix[cdocsamp]{cdocsfn}{cdocsch}
%    \end{macrocode}

%\iffalse
%</samplefinal>
%\fi
%
% %%%%%%%%%%%%%%%%%%%%%%%%%%%%%%%%%%%%%%
% \paragraph{Command Line Processing.}
%
% The following three command lines generate the output files
% |cdocscld|, |cdocscl1| and |cdocscl2|
% which should be identical to
% |cdocsdrf|, |cdocsch1| and |cdocsfn2|, respectively:
% \begin{center}
% \begin{tabular}{l}
% |latex -jobname cdocscld \|\\
% |  "\def\version{draft}\input{childdoc.def}\childdocforward{cdocsamp}"|\\
% |latex -jobname cdocscl1 \|\\
% |  "\input{childdoc.def}\childdocforward[cdocsamp]{cdocsch1}"|\\
% |latex -jobname cdocscl2 \|\\
% |  "\def\version{final}\input{childdoc.def}\childdocforward{cdocsch2}"|
% \end{tabular}
% \end{center}
% Note that the trailing backslash on each first line
% merely continues the input to the second line
% (for convenient cut ant paste).
% Furthermore, the command |latex| can be replaced by any
% of its alternative versions such as |pdflatex|.
%
% %%%%%%%%%%%%%%%%%%%%%%%%%%%%%%%%%%%%%%%%%%%%%%%%%%%%%%%%%%%%%%%%%%%%%%%%%%%%%%
% %%%%%%%%%%%%%%%%%%%%%%%%%%%%%%%%%%%%%%%%%%%%%%%%%%%%%%%%%%%%%%%%%%%%%%%%%%%%%%
% \section{Implementation}
%\iffalse
%<*package>
%\fi
%
% This section describes the definitions file |childdoc.def|.

% The definitions cannot be loaded using |\usepackage| or |\RequirePackage|
% which has a mechanism to prevent loading a style file more than once.
% When loading the definitions by means of |\input|
% multiple instances have to be prevented manually:
%\iffalse
%This code needs to be before the `\ProvidesFile' directive
%which is defined at the beginning of this file.
%Therefore it is also placed there and commented out here.
%</package>
%<*discard>
%\fi
%    \begin{macrocode}
\ifdefined\childdocmain\endinput\fi
%    \end{macrocode}
%\iffalse
%</discard>
%<*package>
%\fi
%
% \macro{\ifchilddoc}
% \macro{\ifchilddocmanual}
% The conditional |\ifchilddoc| tells whether a
% child (true) or main (false) document is being compiled.
% The conditional |\ifchilddocmanual| tells whether
% the |\includeonly| mechanism is used (false) or
% the selection of child files must be performed manually (true).
% The definitions initialise to false:
%    \begin{macrocode}
\newif\ifchilddoc
\newif\ifchilddocmanual
%    \end{macrocode}

% \macro{\childdocname}
% \macro{\childdocjob}
% The macro |\childdocname| stores the name of the main document
% to be compiled. The macro |\childdocjob| stores the name of
% the document on which the \LaTeX{} compiler was originally invoked.
% The content of |\jobname| cannot be compared
% to filenames specified in the source due to different catcodes.
% The following code rescans |\jobname|, stores the result
% in |\childdocname| and saves a copy in |\childdocjob|:
%    \begin{macrocode}
\edef\childdocname{\scantokens\expandafter{\jobname\noexpand}}
\let\childdocjob\childdocname
%    \end{macrocode}

% \macro{\childdocdisable}
% The macro |\childdocdisable| prevents the main file
% from being processed more than once.
% At this stage, the main document command |\childdocmain|
% is assumed to be called once again where it should do nothing.
% Any subsequent call to it should prevent
% a secondary processing of the main document
% It overwrites the forwarding commands
% |\childdocof| and |\childdocforward|
% with empty macros to prevent further inclusions of the main document:
%    \begin{macrocode}
\newcommand{\childdocdisable}
{
  \renewcommand{\childdocmain}[1]{\renewcommand{\childdocmain}[1]{\endinput}}
  \renewcommand{\childdocof}[1]{}
  \renewcommand{\childdocby}[2][]{}
  \renewcommand{\childdocforward}[2][]{}
  \renewcommand{\childdocdisable}{}
}
%    \end{macrocode}

% \macro{\childdocmain}
% The macro |\childdocmain| is to be called at the top of the main file
% with nothing or the main filename (without extension) as argument.
% First, it breaks loops.
% If the argument is not empty and does not match |\childdocname|
% (which is set by the first inclusion of |childdoc.def|),
% |\ifchilddoc| is set to true, |\includeonly| is applied to the child file
% and |\jobname| is set to the main file
% (for proper handling of |.aux| files):
%    \begin{macrocode}
\newcommand{\childdocmain}[1]
{
  \childdocdisable\childdocmain{}
  \if?#1?\else
    \begingroup
      \def\childdoctmp{#1}
      \ifx\childdoctmp\childdocname
        \def\childdoctmp{}
      \else
        \def\childdoctmp
        {
          \childdoctrue
          \includeonly{\childdocname}
          \def\childdocjob{#1}
          \def\jobname{#1}
        }
      \fi
      \expandafter
    \endgroup
    \childdoctmp
  \fi
}
%    \end{macrocode}

% \macro{\childdocof}
% The command |\childdocof| redirects
% compilation to the main file |#1|.
%    \begin{macrocode}
\newcommand{\childdocof}[1]
{
  \childdocdisable
  \childdoctrue
  \includeonly{\childdocname}
  \def\jobname{#1}
  \def\childdocjob{#1}
  \input{#1}
}
%    \end{macrocode}

% \macro{\childdocby}
% The command |\childdocby| ....
%    \begin{macrocode}
\newcommand{\childdocby}[2][]
{
  \childdocdisable
  \childdoctrue
  \childdocmanualtrue
  \if?#1?\else
    \def\jobname{#2}
  \fi
  \def\childdocjob{#2}
  \input{#2}
  \endinput
}
%    \end{macrocode}

% \macro{\childdocforward}
% The command |\childdocforward| redirects
% compilation to the main file or
% (if the optional argument is given) a child file.
% Parameters are set as if the main file
% or a child file starting with |\childdocof| was compiled.
% Then compilation is handed over to the main file:
%    \begin{macrocode}
\newcommand{\childdocforward}[2][]
{
  \begingroup
    \if?#1?
      \def\childdoctmp
      {
        \def\childdocname{#2}
        \def\childdocjob{#2}
        \def\jobname{#2}
        \input{#2}
        \endinput
      }
    \else
      \def\childdoctmp
      {
        \childdocdisable
        \def\childdocname{#2}
        \childdoctrue
        \includeonly{#2}
        \def\childdocjob{#1}
        \def\jobname{#1}
        \input{#1}
        \endinput
      }
    \fi
    \expandafter
  \endgroup
  \childdoctmp
}
%    \end{macrocode}

% \macro{\childdocforwardprefix}
% The command |\childdocforwardprefix| redirects
% compilation to the main or a child file by means of a pattern.
% The prefix |#1| in the current filename is replaced by |#2|
% and the suffix of the current filename is kept
% (it is assumed that the filename does not contain the substring `|~~~|'
% which is used as a delimiter).
% Compilation is handed over to the new file by |\childdocforward|:
%    \begin{macrocode}
\newcommand{\childdocforwardprefix}[3][]
{
  \begingroup
    \def\childdocextract #2##1~~~{\def\childdoctmp{\childdocforward[#1]{#3##1}}}
    \expandafter\childdocextract\childdocname~~~
    \expandafter
  \endgroup
  \childdoctmp
}
%    \end{macrocode}

% \macro{\childdoc}
% The deprecated macro |\childdoc| is a legacy version of |\childdocmain|:
%    \begin{macrocode}
\newcommand{\childdoc}{\childdocmain}
%    \end{macrocode}

% \macro{\childdocredirect}
% The deprecated macro |\childdocredirect| is a legacy version
% of |\childdocforward| and |\childdocforwardprefix|:
%    \begin{macrocode}
\newcommand{\childdocredirect}[2][]
{
  \begingroup
    \if?#1?
      \def\childdoctmp{\childdocforward{#2}}
    \else
      \def\childdoctmp{\childdocforwardprefix{#1}{#2}}
    \fi
    \expandafter
  \endgroup
  \childdoctmp
}
%    \end{macrocode}

%\iffalse
%</package>
%\fi
%
\endinput
\childdocforward{cdocsamp}"|\\
% |latex -jobname cdocscl1 \|\\
% |  "% \iffalse
%
% childdoc.dtx Copyright (C) 2017-2018 Niklas Beisert
%
% This work may be distributed and/or modified under the
% conditions of the LaTeX Project Public License, either version 1.3
% of this license or (at your option) any later version.
% The latest version of this license is in
%   http://www.latex-project.org/lppl.txt
% and version 1.3 or later is part of all distributions of LaTeX
% version 2005/12/01 or later.
%
% This work has the LPPL maintenance status `maintained'.
%
% The Current Maintainer of this work is Niklas Beisert.
%
% This work consists of the files childdoc.dtx and childdoc.ins
% and the derived files childdoc.def and cdocsamp.tex with
% cdocsch1.tex, cdocsch2.tex, cdocsdrf.tex, cdocsfn1.tex, cdocsfn2.tex.
%
%<package>\ifdefined\childdocmain\endinput\fi
%<package>\ProvidesFile{childdoc.def}[2018/12/30 v2.0 child document driver]
%<samplemain>\ProvidesFile{cdocsamp.tex}[2018/12/30 v2.0 sample for childdoc]
%<*driver>
%\ProvidesFile{childdoc.drv}[2018/12/30 v2.0 childdoc reference manual file]
\PassOptionsToClass{10pt,a4paper}{article}
\documentclass{ltxdoc}

\usepackage[margin=35mm]{geometry}
\usepackage{hyperref}
\usepackage{hyperxmp}
\usepackage[usenames]{color}

\hypersetup{colorlinks=true}
\hypersetup{pdfstartview=FitH}
\hypersetup{pdfpagemode=UseNone}
\hypersetup{pdfsource={}}
\hypersetup{pdflang={en-UK}}
\hypersetup{pdfcopyright={Copyright 2017-2018 Niklas Beisert.
  This work may be distributed and/or modified under the
  conditions of the LaTeX Project Public License, either version 1.3
  of this license or (at your option) any later version.}}
\hypersetup{pdflicenseurl={http://www.latex-project.org/lppl.txt}}
\hypersetup{pdfcontactaddress={ETH Zurich, ITP, HIT K,
  Wolfgang-Pauli-Strasse 27}}
\hypersetup{pdfcontactpostcode={8093}}
\hypersetup{pdfcontactcity={Zurich}}
\hypersetup{pdfcontactcountry={Switzerland}}
\hypersetup{pdfcontactemail={nbeisert@itp.phys.ethz.ch}}
\hypersetup{pdfcontacturl={http://people.phys.ethz.ch/\xmptilde nbeisert/}}

\newcommand{\secref}[1]{\hyperref[#1]{section \ref*{#1}}}

\parskip1ex
\parindent0pt
\let\olditemize\itemize
\def\itemize{\olditemize\parskip0pt}

\begin{document}

\title{The \textsf{childdoc} Package}
\hypersetup{pdftitle={The childdoc Package}}
\author{Niklas Beisert\\[2ex]
  Institut f\"ur Theoretische Physik\\
  Eidgen\"ossische Technische Hochschule Z\"urich\\
  Wolfgang-Pauli-Strasse 27, 8093 Z\"urich, Switzerland\\[1ex]
  \href{mailto:nbeisert@itp.phys.ethz.ch}
  {\texttt{nbeisert@itp.phys.ethz.ch}}}
\hypersetup{pdfauthor={Niklas Beisert}}
\hypersetup{pdfsubject={Manual for the LaTeX2e Package childdoc}}
\date{30 December 2018, \textsf{v2.0}}
\maketitle

\begin{abstract}\noindent
\textsf{childdoc} is a \LaTeXe{} package
that enables the direct compilation
of document sections included by |\include|
to individual files.
\end{abstract}

\begingroup
\parskip0ex
\tableofcontents
\endgroup

%%%%%%%%%%%%%%%%%%%%%%%%%%%%%%%%%%%%%%%%%%%%%%%%%%%%%%%%%%%%%%%%%%%%%%%%%%%%%%%%
%%%%%%%%%%%%%%%%%%%%%%%%%%%%%%%%%%%%%%%%%%%%%%%%%%%%%%%%%%%%%%%%%%%%%%%%%%%%%%%%
\section{Introduction}

\LaTeX{} provides a mechanism to structure a large document (such as a book)
into a main file and several child files (containing the chapters)
using the |\include| command.
This mechanism is beneficial for documents
which span hundreds of pages in order to
make the source file(s) more manageable.
Moreover, compilation can be restricted to
selected child files by means of the |\includeonly| command.
The latter feature can be used to reduce the compilation time while editing
(this was significantly more useful in the earlier days of \LaTeX{})
or to generate a smaller document which is easier to navigate.
Another application of |\includeonly| is to generate
documents consisting of selected parts of the complete document.

However, there are a few drawbacks of the plain |\include| mechanism:
\begin{itemize}
\item
The child files cannot be compiled on their own,
they can only be compiled via the main file.
A naive editing environment
(such as a text editor with an option
to have the current file processed by \LaTeX)
may require one to switch to the main file before compiling;
attempting to compile the child file produces errors.
\item
The main file must be modified (each time)
to adjust the |\includeonly| command
to the present needs. This easily leaves the main file in a messy state.
\item
The generated document will always carry the filename
of the main document. This is inconvenient if
several child files are to be compiled and
to be kept for distribution.
\end{itemize}

The present package provides a simple interface
to make child files individually compilable by \LaTeX{}.
Compiling a child file then has the same effect as compiling
the main file with an |\includeonly| command
to select the appropriate child.
Moreover the generated document will carry the name of the child
rather than the main file.
This resolves all three above issues.

This feature is meant to make the editing of books,
thesis documents and lecture notes somewhat more convenient.
However, the package can also be used efficiently for
composing a series of documents (such as exercise sheets)
which are typically distributed individually.
It then assists the author in generating the individual documents
(potentially in different versions)
as well as a document containing the collected series.
Another application is in developing style files
or other kinds of included material
where compilation of the style file could redirect
to a sample or test file.

%%%%%%%%%%%%%%%%%%%%%%%%%%%%%%%%%%%%%%%%%%%%%%%%%%%%%%%%%%%%%%%%%%%%%%%%%%%%%%%%
%%%%%%%%%%%%%%%%%%%%%%%%%%%%%%%%%%%%%%%%%%%%%%%%%%%%%%%%%%%%%%%%%%%%%%%%%%%%%%%%
\section{Usage}

First of all, the package \textsf{childdoc} is \emph{not} a standard
\LaTeXe{} |.sty| style file! Therefore it needs to be invoked in
a non-standard way.

%%%%%%%%%%%%%%%%%%%%%%%%%%%%%%%%%%%%%%%%%%%%%%%%%%%%%%%%%%%%%%%%%%%%%%%%%%%%%%%%
\subsection{Included Files}
\label{sec:include}

%%%%%%%%%%%%%%%%%%%%%%%%%%%%%%%%%%%%%%%%
\DescribeMacro{\childdocmain}
To use the package, add the commands
\begin{center}
\begin{tabular}{l}
|\input{childdoc.def}|\\
|\childdocmain{}|\\
\end{tabular}
\end{center}
at the very top of the main \LaTeX{} file,
in particular \emph{before} the |\documentclass| statement!
The argument of |\childdocmain| should be left empty
(but it must be present).

%%%%%%%%%%%%%%%%%%%%%%%%%%%%%%%%%%%%%%%%
\DescribeMacro{\childdocof}
Furthermore, add the commands
\begin{center}
\begin{tabular}{l}
|\input{childdoc.def}|\\
|\childdocof{|\textit{main}|}|\\
\end{tabular}
\end{center}
at the top of every child file \textit{child}
which is included by |\include{|\textit{child}|}|
from within the main file
(or at least for those files to be compiled individually).
The argument \textit{main} must be the filename of the main file.

There are a couple of
considerations in setting up the main and child documents:

%%%%%%%%%%%%%%%%%%%%%%%%%%%%%%%%%%%%%%%%
\paragraph{Restrictions.}

Please note the following restrictions:
\begin{itemize}
\item
|\childdocmain| must be called with one argument \textit{main}
to ensure compatibility with earlier version of the package.
It must either be empty (|\childdocmain{}|)
or precisely match the filename of the main file in which it is specified.
See \secref{sec:detection} for further information.
\item
The filename \textit{main} must be specified without the |.tex| extension.
\item
The filename \textit{main} is case sensitive
(even in case-insensitive file systems)
due to internal string comparison.
\item
The argument \textit{main} should be fully expanded, it cannot be a macro.
\item
Subdirectories and special characters should be avoided in filenames.
\item
The command |\childdocmain{|\textit{main}|}| must be followed by a whitespace.
It should not be followed immediately by another command
or by a comment mark `|%|'.
This is because the \TeX{} parser reads the token immediately following
the argument of |\childdocmain| and puts it
at the beginning of every child section;
however, a white\-space is ignored.
\end{itemize}

%%%%%%%%%%%%%%%%%%%%%%%%%%%%%%%%%%%%%%%%
\paragraph{Content of Main File.}

It is advisable to place all content in the child files included by |\include|.
Any output contained in the main file will appear in all child documents
unless suppressed manually;
it cannot be suppressed automatically by the |\includeonly| directive
and thus should normally be avoided.
A method to include some content in the main file
by means of conditional processing is described in \secref{sec:conditional}.

%%%%%%%%%%%%%%%%%%%%%%%%%%%%%%%%%%%%%%%%
\paragraph{Page Numbering.}

When only a part of the document is compiled,
the appropriate numbering of pages
(as well as other status parameters)
is determined from the |.aux| files.
The latter contain information from previous passes.
However this information needs to propagate through
all intermediate child documents.
Therefore the page numbering in child documents may well
be inconsistent until the complete document is compiled at least once.

A useful (if unconventional) way to always ensure a consistent
page numbering is to restart the numbering in each child document
and denote the pages by `\textit{child}|.|\textit{page}'
where \textit{child} represents the chapter/section number of the child file.
This can be achieved by the command
|\numberwithin{page}{|\textit{child}|}|
of the \textsf{amsmath} package
where \textit{child} can be |chapter| or |section|
depending on the chosen structuring.
Alternatively, one can modify the macro |\thepage| appropriately
and reset the counter |page| at the start of each child file.

%%%%%%%%%%%%%%%%%%%%%%%%%%%%%%%%%%%%%%%%%%%%%%%%%%%%%%%%%%%%%%%%%%%%%%%%%%%%%%%%
\subsection{Conditional Processing}
\label{sec:conditional}

The package provides a mechanism to compile different versions
of a document. To customise the versions further some conditional processing
can come in handy to distinguish which version is being compiled.
The package provides two macros to describe the compilation context:

%%%%%%%%%%%%%%%%%%%%%%%%%%%%%%%%%%%%%%%%
\DescribeMacro{\ifchilddoc}
The conditional |\ifchilddoc| distinguishes between the compilation of
child documents and the main document:
%
\begin{center}
|\ifchilddoc |\textit{child-code}| |[|\||else |\textit{main-code}]| \||fi|
\end{center}

%%%%%%%%%%%%%%%%%%%%%%%%%%%%%%%%%%%%%%%%
\DescribeMacro{\childdocname}
\DescribeMacro{\childdocjob}
The macro |\childdocname| contains the filename (without extension)
of the main or child file being processed.
Note that |\childdocjob| will always contain the name of the main file.

%%%%%%%%%%%%%%%%%%%%%%%%%%%%%%%%%%%%%%%%
\paragraph{Title Page.}

Conditional processing can be used to include a title or banner page
in the main document when proper precautions are taken.
Importantly, the code in the main file should ensure that the page counter
(as well as other status parameters which are stored in the |.aux| files)
takes the same value after the conditional processing.
Otherwise the page numbers may take divergent values
depending on which part is compiled.

For example, a title page could be declared by:
%
\begin{center}
\begin{tabular}{l}
|\ifchilddoc\||else|\\
|\addtocounter{page}{-1}|\\
\textit{code for title page}\\
|\newpage|\\
|\||fi|
\end{tabular}
\end{center}
%
A banner page for the child documents can be generated by:
%
\begin{center}
\begin{tabular}{l}
|\ifchilddoc|\\
|\addtocounter{page}{-1}|\\
\textit{code for banner page}\\
|\newpage|\\
|\||fi|
\end{tabular}
\end{center}
%
Here one could write a message such as:
\begin{center}
|This is the part \childdocname{} of \childdocjob{}.|
\end{center}

%%%%%%%%%%%%%%%%%%%%%%%%%%%%%%%%%%%%%%%%%%%%%%%%%%%%%%%%%%%%%%%%%%%%%%%%%%%%%%%%
\subsection{Flags}
\label{sec:flags}

The package makes it easy to generate different versions
of the main or child documents.
To this end compilation flags can be defined
and assigned different default values.
They will be particularly useful in conjunction
with the forwarding mechanism described in \secref{sec:forward}.

For example, it may be useful to have a flag |\version|
which can be set to |draft| or |final|.
The document source will contain some conditional code
depending on the value of |\version|.
Suppose further, the flag should default to |final| for the main file
and to |draft| for child files
which is a natural assignment for editing the document.
This is achieved by placing the following code
in the preamble of the main document
(below the |\childdocmain| directive):
%
\begin{center}
\begin{tabular}{l}
|\ifchilddoc|\\
|\providecommand{\version}{draft}|\\
|\||else|\\
|\providecommand{\version}{final}|\\
|\||fi|
\end{tabular}
\end{center}
%
The definition by |\providecommand| makes sure
that previous definitions are not overwritten.
Further statements |\providecommand{\version}{...}|
can thus be added before the above code to override it.

For the main file, one might add a line
(between |\childdocmain| and the above block)
%
\begin{center}
|%\ifchilddoc\||else\providecommand{\version}{draft}\||fi|
\end{center}
%
which can be uncommented to produce a draft version.
Likewise one can add a line to the very top of a child file
(above the |\childdocof{|\textit{main}|}| directive)
%
\begin{center}
|%\providecommand{\version}{final}|
\end{center}
%
which can be uncommented to produce the final version of this child document.

%%%%%%%%%%%%%%%%%%%%%%%%%%%%%%%%%%%%%%%%%%%%%%%%%%%%%%%%%%%%%%%%%%%%%%%%%%%%%%%%
\subsection{Forwarding}
\label{sec:forward}

Different versions of the main or child documents
using compilation flags as described in \secref{sec:flags}
can be (permanently) stored in different files
for convenient compilation, viewing and distribution.
To this end, the package defines a command
to pass on compilation to a different file:

%%%%%%%%%%%%%%%%%%%%%%%%%%%%%%%%%%%%%%%%
\DescribeMacro{\childdocforward}
The command |\childdocforward| redirects processing to
another source file:
%
\begin{center}
\begin{tabular}{l}
|\input{childdoc.def}|\\
|\childdocforward[|\textit{main}|]{|\textit{dest}|}|\\
\end{tabular}
\end{center}
%
The argument \textit{dest} is the destination file
(without extension).
It should be the main file or one of the child files.
Note that further \textsf{childdoc} directives
such as |\childdocof| and |\childdocforward|
in the indicated file will be processed in this form.
The optional argument \textit{main}
passes on directly to the main file \textit{main}
while pretending to compile the child \textit{dest}.
This form behaves as if \textit{dest}
issues |\childdocof{|\textit{main}|}| right away,
and no further \textsf{childdoc} directives will be processed.

%%%%%%%%%%%%%%%%%%%%%%%%%%%%%%%%%%%%%%%%
\DescribeMacro{\...prefix}
In the alternative form |\childdocforwardprefix|,
%
\begin{center}
\begin{tabular}{l}
|\input{childdoc.def}|\\
|\childdocforwardprefix[|\textit{main}|]{|\textit{prefix}|}{|\textit{dest}|}|
\end{tabular}
\end{center}
%
the destination file is determined by a pattern
depending on the current file:
To make this work, the current file must be called
`{\textit{prefix}\hspace{0.2em}\textit{suffix}}'
with \textit{prefix} matching precisely the argument.
Processing is then passed on to the file
`{\textit{dest}\hspace{0.2em}\textit{suffix}}'.
Surely, the same effect is achieved by
directly specifying the
argument `{\textit{dest}\hspace{0.2em}\textit{suffix}}'
in the first form.
However, that requires to set up a different file
for each child. With the alternative form of the command
all these files can have exactly the same content
which simplifies setting them up and maintaining them.

For example, the following file |draft.tex|
with a compilation flag |\version| as described in \secref{sec:flags}
compiles the main document as a draft:
%
\begin{center}
\begin{tabular}{l}
|\def\version{draft}|\\
|\input{childdoc.def}|\\
|\childdocforward{|\textit{main}|}|
\end{tabular}
\end{center}
%
Likewise, the following files |final|\textit{nn}|.tex|
compile the final version of the child document
|child|\textit{nn}|.tex|:
%
\begin{center}
\begin{tabular}{l}
|\def\version{final}|\\
|\input{childdoc.def}|\\
|\childdocforwardprefix{final}{child}|
\end{tabular}
\end{center}
%

Note that when several versions of a main file and/or of each child file
are to be generated, it may be convenient to set up a |Makefile| or
shell script to automatise the process.

%%%%%%%%%%%%%%%%%%%%%%%%%%%%%%%%%%%%%%%%%%%%%%%%%%%%%%%%%%%%%%%%%%%%%%%%%%%%%%%%
\subsection{Command Line Processing}
\label{sec:commandline}

The effect of redirection files can also be achieved by invoking
the \LaTeX{} compiler with a more elaborate command line.
Most conveniently this should be done as part
of a shell script or a |Makefile|.

When using \textsf{childdoc} in the main file, the following
command lines effectively perform a redirection
(note that depending on the shell being used,
backslashes may have to be doubled: `|\|' $\to$ `|\\|'):
%
\begin{center}
|... -jobname "|\textit{target}|" |\\|"|[\textit{flags}]%
|\input{childdoc.def}\childdocforward[|\textit{main}|]{|\textit{dest}|}"|
\end{center}
%
Here \textit{target} is the name of the output file,
\textit{main} is the name of the main file
and \textit{dest} is the name of the main or child file to be processed
(all filenames without extensions).
The optional argument \textit{main} can be omitted
if \textit{main} matches \textit{dest}.
Optionally, compilation \textit{flags} can be defined via |\def| commands.
This command line makes the \TeX{} engine believe
it is compiling the file \textit{target}
whose content is specified as the latter parameter.
The provided code then forwards the processing to
\textit{main} or \textit{dest} as described in \secref{sec:forward}.

%%%%%%%%%%%%%%%%%%%%%%%%%%%%%%%%%%%%%%%%%%%%%%%%%%%%%%%%%%%%%%%%%%%%%%%%%%%%%%%%
\subsection{Include by Input}
\label{sec:input}

Including child documents by |\include| has some restrictions by design.
Most notably, the content of a child document always occupies
its own set of pages; pages cannot be shared between child documents.
Usually, this behaviour makes perfect sense
because each child document contain an essential part of the document.
However, in some situations it may be desirable to compose
a document from a collection of parts
without having mandatory page breaks between then.
For this case, the package
provides a mechanism to include parts
by |\input| which can also be processed individually.
However, by construction this mechanism
requires manual handling of the content to be output.

%%%%%%%%%%%%%%%%%%%%%%%%%%%%%%%%%%%%%%%%
\DescribeMacro{\ifchilddocmanual}
The main file should be prepared as usual, see \secref{sec:include}.
However, the document body must make a distinction
between processing of an individual part and of the main document, e.g.:
%
\begin{center}
\begin{tabular}{l}
|\ifchilddocmanual|\\
|\input{\childdocname}|\\
|\||else|\\
\textit{document body with }|\input{|\textit{part}|}|\\
|\||fi|
\end{tabular}
\end{center}
%
The conditional |\ifchilddocmanual| is true whenever
a part to be included by |\input| is being compiled,
and the name of the part is stored in |\childdocname|.

%%%%%%%%%%%%%%%%%%%%%%%%%%%%%%%%%%%%%%%%
\DescribeMacro{\childdocby}
Each part to be included by |\input| should start with:
%
\begin{center}
\begin{tabular}{l}
|\input{childdoc.def}|\\
|\childdocby{|\textit{main}|}|\\
\end{tabular}
\end{center}
%
The directive |\childdocby| is similar to |\childdocof|
described in \secref{sec:include},
but the subsequent selection of content must be done manually.
To that end, both |\ifchilddoc| and |\ifchilddocmanual|
will be true upon processing of a part,
and the name of the part is stored in |\childdocname|.
Note that |\jobname| will be set to the filename of the current part
so that each part receives an individual |.aux| file
that does not interfere with the |.aux| file(s) of the main document.
This behaviour can be altered by the alternative form
|\childdocby[*]{|\textit{main}|}| (with a non-empty optional argument)
which uses the |.aux| file of the main document
by setting |\jobname| to \textit{main}.

%%%%%%%%%%%%%%%%%%%%%%%%%%%%%%%%%%%%%%%%%%%%%%%%%%%%%%%%%%%%%%%%%%%%%%%%%%%%%%%%
\subsection{Driver Development}
\label{sec:driver}

The \textsf{childdoc} mechanism can also be use for the development
of definition files such as \LaTeX{} styles or classes.
This case differs from the above setup with multiple parts
included by |\include| in that no |\includeonly| should be invoked.
This can be achieved by starting the include file
(before |\ProvidesPackage|) with:
%
\begin{center}
\begin{tabular}{l}
|\input{childdoc.def}|\\
|\childdocforward{|\textit{main}|}|\\
\end{tabular}
\end{center}
%
or alternatively with:
%
\begin{center}
\begin{tabular}{l}
|\input{childdoc.def}|\\
|\childdocby{|\textit{main}|}|\\
\end{tabular}
\end{center}
%
Both forms have slightly different effects as described above.
The main file is prepared as usual, see \secref{sec:include}.

%%%%%%%%%%%%%%%%%%%%%%%%%%%%%%%%%%%%%%%%%%%%%%%%%%%%%%%%%%%%%%%%%%%%%%%%%%%%%%%%
\subsection{Legacy Detection}
\label{sec:detection}

The directive |\childdocmain| in the main file can detect
whether the complete document or merely a child is to be compiled
even without using the directive |\childdocof|.
This method is deprecated because it is less robust
and there is no compelling reason to use it;
it is merely provided for backward compatibility
and it may be removed in future versions.

If the detection mechanism is to be used,
it is mandatory to correctly specify
the filename of the main file as the argument of |\childdocmain|:
%
\begin{center}
\begin{tabular}{l}
|\input{childdoc.def}|\\
|\childdocmain{|\textit{main}|}|\\
\end{tabular}
\end{center}
%
If |\jobname| does not match the argument \textit{main} of |\childdocmain|,
it is assumed that |\jobname| points to the child file to be compiled.
When using |\childdocmain| with the main file specified as argument,
it suffices to start a child file
with just |\input{|\textit{main}|}|
without loading of the package and using |\childdocof|.
If instead all processing is done
with the appropriate \textsf{childdoc} directives,
the argument of \textit{main} of |\childdocmain| can be empty.

An alternative version of the command line processing described
in \secref{sec:commandline} using the detection mechanism reads:
%
\begin{center}
|... -jobname "|\textit{target}|" "|[\textit{flags}]%
[|\def\jobname{|\textit{dest}|}|]|\input{|\textit{main}|}"|
\end{center}

%%%%%%%%%%%%%%%%%%%%%%%%%%%%%%%%%%%%%%%%%%%%%%%%%%%%%%%%%%%%%%%%%%%%%%%%%%%%%%%%
\subsection{Manual Code}
\label{sec:manual}

In case one cannot be certain whether the definitions file |childdoc.def|
is installed on the target \TeX{} distribution
and one prefers not to ship it,
it is conceivable to paste a few relevant commands into the sources.

To that end, drop all statements |\input{childdoc.def}|
and perform the replacements as outlined below.
Instead of |\childdocmain{|\textit{main}|}| add the following code
to the top of the main file:
%
\begin{center}
\begin{tabular}{l}
|\||ifdefined\childdocname\endinput\||fi\newif\ifchilddoc|\\
|\edef\childdocname{\scantokens\expandafter{\jobname\noexpand}}|\\
|\def\childdocmain{|\textit{main}|}\||ifx\childdocmain\childdocname\||else|\\
|\childdoctrue\includeonly{\childdocname}\let\jobname\childdocmain\||fi|\\
\end{tabular}
\end{center}
%
Instead of |\childdocof{|\textit{main}|}| just include the main file
at the top of each child file:
%
\begin{center}
|\input{|\textit{main}|}|
\end{center}
%
A simple redirection |\childdocforward{|\textit{dest}|}| is achieved by:
%
\begin{center}
|\def\jobname{|\textit{dest}|}\input{\jobname}|
\end{center}
%
The redirection with prefix
|\childdocforwardprefix[|\textit{prefix}|]{|\textit{dest}|}|
is accomplished by:
%
\begin{center}
\begin{tabular}{l}
|{\edef\jobname{\scantokens\expandafter{\jobname\noexpand}}|\\
|\def\redirectjob |\textit{prefix}|#1~~~{\gdef\jobname{|\textit{dest}|#1}}|\\
|\expandafter\redirectjob\jobname~~~}\input{\jobname}|
\end{tabular}
\end{center}

In an alternative approach,
child documents can be compiled by a specific command line
without additional code or specific definitions:
%
\begin{center}
|... -jobname "|\textit{target}|" "|[\textit{flags}]%
|\includeonly{|\textit{dest}|}\input{|\textit{main}|}"|
\end{center}
%

%%%%%%%%%%%%%%%%%%%%%%%%%%%%%%%%%%%%%%%%%%%%%%%%%%%%%%%%%%%%%%%%%%%%%%%%%%%%%%%%
%%%%%%%%%%%%%%%%%%%%%%%%%%%%%%%%%%%%%%%%%%%%%%%%%%%%%%%%%%%%%%%%%%%%%%%%%%%%%%%%
\section{Information}

%%%%%%%%%%%%%%%%%%%%%%%%%%%%%%%%%%%%%%%%%%%%%%%%%%%%%%%%%%%%%%%%%%%%%%%%%%%%%%%%
\subsection{Copyright}

Copyright \copyright{} 2017--2018 Niklas Beisert

This work may be distributed and/or modified under the
conditions of the \LaTeX{} Project Public License, either version 1.3
of this license or (at your option) any later version.
The latest version of this license is in
  \url{http://www.latex-project.org/lppl.txt}
and version 1.3 or later is part of all distributions of \LaTeX{}
version 2005/12/01 or later.

This work has the LPPL maintenance status `maintained'.

The Current Maintainer of this work is Niklas Beisert.

This work consists of the files |README.txt|, |childdoc.ins| and |childdoc.dtx|
as well as the derived files |childdoc.def|, |cdocsamp.tex|
with |cdocsch1.tex|, |cdocsch2.tex|, |cdocspt3.tex|, |cdocspt4.tex|,
|cdocsdrf.tex|, |cdocsfn1.tex|, |cdocsfn2.tex|
as well as |childdoc.pdf|.

%%%%%%%%%%%%%%%%%%%%%%%%%%%%%%%%%%%%%%%%%%%%%%%%%%%%%%%%%%%%%%%%%%%%%%%%%%%%%%%%
\subsection{Files and Installation}

The package consists of the files:
%
\begin{center}
\begin{tabular}{ll}
    |README.txt|   & readme file \\
    |childdoc.ins| & installation file \\
    |childdoc.dtx| & source file \\
    |childdoc.def| & definition file \\
    |cdocsamp.tex| & sample main file \\
    |cdocsch1.tex| & sample include file \\
    |cdocsch2.tex| & sample include file \\
    |cdocspt3.tex| & sample part file \\
    |cdocspt4.tex| & sample part file \\
    |cdocsdrf.tex| & sample redirection file \\
    |cdocsfn1.tex| & sample redirection file \\
    |cdocsfn2.tex| & sample redirection file \\
    |childdoc.pdf| & manual
\end{tabular}
\end{center}
%
The distribution consists of the files
|README.txt|, |childdoc.ins| and |childdoc.dtx|.
%
\begin{itemize}
\item
Run (pdf)\LaTeX{} on |childdoc.dtx|
to compile the manual |childdoc.pdf| (this file).
\item
Run \LaTeX{} on |childdoc.ins| to create the definitions file |childdoc.def|
and the sample |cdocsamp.tex| with include files
|cdocsch1.tex|, |cdocsch2.tex|, |cdocspt3.tex|, |cdocspt4.tex|,
|cdocsdrf.tex|, |cdocsfn1.tex|, |cdocsfn2.tex|.
Then copy the file |childdoc.def| to an appropriate directory of your \LaTeX{}
distribution, e.g.\ \textit{texmf-root}|/tex/latex/childdoc|.
\end{itemize}

%%%%%%%%%%%%%%%%%%%%%%%%%%%%%%%%%%%%%%%%%%%%%%%%%%%%%%%%%%%%%%%%%%%%%%%%%%%%%%%%
\subsection{Related CTAN Packages}

There are several other packages which offer a similar functionality:
%
\begin{itemize}
\item
The packages
\href{http://ctan.org/pkg/docmute}{\textsf{docmute}},
\href{http://ctan.org/pkg/includex}{\textsf{includex}} and
\href{http://ctan.org/pkg/standalone}{\textsf{standalone}}
provide commands to include only the document body of
a child file thus allowing both files to be compiled individually.
\item
The packages \href{http://ctan.org/pkg/subdocs}{\textsf{subdocs}}
and \href{http://ctan.org/pkg/subfiles}{\textsf{subfiles}}
provide structures in which the main and child documents can be
encapsulated and allowing them to be compiled individually.
The inclusion mechanism is different from the conventional |\include|.
\item
The package \href{http://ctan.org/pkg/combine}{\textsf{combine}}
is an elaborate solution to combine several documents into one.
\end{itemize}
%
See also the CTAN topic \href{http://ctan.org/topic/subdocs}{\textsf{subdocs}}
for further related packages.
The present package differs from the above solutions in that
a document structure constructed with the conventional |\include| mechanism
just needs two extra commands at the top of every file
such that all constituent files can be compiled individually.

%%%%%%%%%%%%%%%%%%%%%%%%%%%%%%%%%%%%%%%%%%%%%%%%%%%%%%%%%%%%%%%%%%%%%%%%%%%%%%%%
%\subsection{Feature Suggestions}
%
%The following is a list of features which may be useful for future
%versions of this package:
%%
%\begin{itemize}
%\item
%\ldots
%\end{itemize}

%%%%%%%%%%%%%%%%%%%%%%%%%%%%%%%%%%%%%%%%%%%%%%%%%%%%%%%%%%%%%%%%%%%%%%%%%%%%%%%%
\subsection{Revision History}

%%%%%%%%%%%%%%%%%%%%%%%%%%%%%%%%%%%%%%%%
\paragraph{v2.0:} 2018/12/30

\begin{itemize}
\item
immediate forward processing
\item
added |\childdocby| mechanism
\item
manual restructured
\end{itemize}

%%%%%%%%%%%%%%%%%%%%%%%%%%%%%%%%%%%%%%%%
\paragraph{v1.6:} 2018/01/17

\begin{itemize}
\item
application for development of include files
\item
corrections to manual
\end{itemize}

%%%%%%%%%%%%%%%%%%%%%%%%%%%%%%%%%%%%%%%%
\paragraph{v1.5:} 2017/05/21

\begin{itemize}
\item
more complete structuring introduced
\item
|\childdocof| introduced
\item
|\childdoc| renamed to |\childdocmain|
\item
|\childredirect| renamed to |\childdocforward| and |\childdocforwardprefix|
and functionality expanded
\end{itemize}

%%%%%%%%%%%%%%%%%%%%%%%%%%%%%%%%%%%%%%%%
\paragraph{v1.0:} 2017/04/27

\begin{itemize}
\item
manual and install package
\item
first version published on CTAN
\end{itemize}

%%%%%%%%%%%%%%%%%%%%%%%%%%%%%%%%%%%%%%%%
\paragraph{v0.6:} 2017/04/26

\begin{itemize}
\item
redirection mechanism added
\end{itemize}

%%%%%%%%%%%%%%%%%%%%%%%%%%%%%%%%%%%%%%%%
\paragraph{v0.5:} 2017/04/26

\begin{itemize}
\item
functionality in definition file
\end{itemize}


%%%%%%%%%%%%%%%%%%%%%%%%%%%%%%%%%%%%%%%%%%%%%%%%%%%%%%%%%%%%%%%%%%%%%%%%%%%%%%%%
%%%%%%%%%%%%%%%%%%%%%%%%%%%%%%%%%%%%%%%%%%%%%%%%%%%%%%%%%%%%%%%%%%%%%%%%%%%%%%%%
%%%%%%%%%%%%%%%%%%%%%%%%%%%%%%%%%%%%%%%%%%%%%%%%%%%%%%%%%%%%%%%%%%%%%%%%%%%%%%%%
\appendix

\settowidth\MacroIndent{\rmfamily\scriptsize 000\ }

 \DocInput{childdoc.dtx}

\end{document}
%</driver>
% \fi
%
% %%%%%%%%%%%%%%%%%%%%%%%%%%%%%%%%%%%%%%%%%%%%%%%%%%%%%%%%%%%%%%%%%%%%%%%%%%%%%%
% %%%%%%%%%%%%%%%%%%%%%%%%%%%%%%%%%%%%%%%%%%%%%%%%%%%%%%%%%%%%%%%%%%%%%%%%%%%%%%
% \section{Sample}
%\iffalse
%<*samplemain>
%\fi
%
% The following presents a sample document
% with two chapters, two parts, a title page,
% a compile flag as well as three forwarding files to set the flag.
% It consists of eight |.tex| files:
% \begin{center}
% \begin{tabular}{ll}
% |cdocsamp.tex|&main file\\
% |cdocsch1.tex|&include file for chapter 1\\
% |cdocsch2.tex|&include file for chapter 2\\
% |cdocspt3.tex|&include file for part 3\\
% |cdocspt4.tex|&include file for part 4\\
% |cdocsdrf.tex|&forwarding file for main file in draft mode\\
% |cdocsfi1.tex|&forwarding file for final version of chapter 1\\
% |cdocsfi2.tex|&forwarding file for final version of chapter 2\\
% \end{tabular}
% \end{center}
% Each of the eight files can be compiled directly by the \LaTeX{} compiler.
%
% %%%%%%%%%%%%%%%%%%%%%%%%%%%%%%%%%%%%%%
% \paragraph{Main File.}
%
% The main file is called |cdocsamp.tex|.
%
% Load the \textsf{childdoc} definitions and
% declare the filename for the main document:
%    \begin{macrocode}
\input{childdoc.def}
\childdocmain{}
%    \end{macrocode}

% Optional override for |\version| flag:
%    \begin{macrocode}
%%\ifchilddoc\else\providecommand{\version}{draft}\fi
%    \end{macrocode}

% Define the default values for the |\version| flag
% (|final| for the main file and |draft| for childs):
%    \begin{macrocode}
\ifchilddoc
\providecommand{\version}{draft}
\else
\providecommand{\version}{final}
\fi
%    \end{macrocode}

% Load the standard document class:
%    \begin{macrocode}
\documentclass[12pt]{article}
%    \end{macrocode}

% Start the document body:
%    \begin{macrocode}
\begin{document}
%    \end{macrocode}

% Declare a title page.
% Print title, part of document being processed and version flag:
%    \begin{macrocode}
\addtocounter{page}{-1}
\begin{center}
{\LARGE\bfseries{}childdoc example\par}
\vspace{1cm}
\ifchilddoc
\ifchilddocmanual part\else chapter\fi:
`\childdocname' of `\childdocjob'\par
\else
main document: `\childdocjob'\par
\fi
version: \version\par
\end{center}
\newpage
%    \end{macrocode}

% Manually include selected file,
% otherwise process as usual:
%    \begin{macrocode}
\ifchilddocmanual
\section*{part `\childdocname'}
\input{\childdocname}
\else
%    \end{macrocode}

% Include the two chapters:
%    \begin{macrocode}
\include{cdocsch1}
\include{cdocsch2}
%    \end{macrocode}

% Include the two parts unless only chapters should be displayed:
%    \begin{macrocode}
\ifchilddoc\else
\section{part three}
\input{cdocspt3}
\section{part four}
\input{cdocspt4}
\fi
%    \end{macrocode}

% Process as usual until here:
%    \begin{macrocode}
\fi
%    \end{macrocode}

% End of document body:
%    \begin{macrocode}
\end{document}
%    \end{macrocode}
%\iffalse
%</samplemain>
%\fi
%
% %%%%%%%%%%%%%%%%%%%%%%%%%%%%%%%%%%%%%%
% \paragraph{Chapter Include Files.}
%
% The include files are called |cdocsch1.tex| and |cdocsch2.tex|.
%
%\iffalse
%<*samplechap1|samplechap2>
%\fi

% Optional override for |\version| flag:
%    \begin{macrocode}
%%\providecommand{\version}{final}
%    \end{macrocode}

% Include the main document:
%    \begin{macrocode}
\input{childdoc.def}
\childdocof{cdocsamp}
%    \end{macrocode}

%\iffalse
%</samplechap1|samplechap2>
%\fi
%
%\iffalse
%<*samplechap1>
%\fi
% Some text for chapter 1:
%    \begin{macrocode}
\section{one}
some text in chapter one
%    \end{macrocode}

%\iffalse
%</samplechap1>
%\fi
% Some text for chapter 2:
%\iffalse
%<*samplechap2>
%\fi
%    \begin{macrocode}
\section{two}
more text in chapter two
%    \end{macrocode}

%\iffalse
%</samplechap2>
%\fi
%
% %%%%%%%%%%%%%%%%%%%%%%%%%%%%%%%%%%%%%%
% \paragraph{Part Include Files.}
%
% The include files are called |cdocspt3.tex| and |cdocspt4.tex|.
%
%\iffalse
%<*samplepart3|samplepart4>
%\fi

% Optional override for |\version| flag:
%    \begin{macrocode}
%%\providecommand{\version}{final}
%    \end{macrocode}

% Include the main document:
%    \begin{macrocode}
\input{childdoc.def}
\childdocby{cdocsamp}
%    \end{macrocode}

%\iffalse
%</samplepart3|samplepart4>
%\fi
%
%\iffalse
%<*samplepart3>
%\fi
% Some text for part 3:
%    \begin{macrocode}
some text in part three
%    \end{macrocode}

%\iffalse
%</samplepart3>
%\fi
% Some text for part 4:
%\iffalse
%<*samplepart4>
%\fi
%    \begin{macrocode}
more text in part four
%    \end{macrocode}

%\iffalse
%</samplepart4>
%\fi
%
% %%%%%%%%%%%%%%%%%%%%%%%%%%%%%%%%%%%%%%
% \paragraph{Forwarding for a Complete Draft.}
%
% The following forwarding file |cdocsdrf.tex|
% compiles the main document in draft mode:
%\iffalse
%<*sampledraft>
%\fi
%    \begin{macrocode}
\def\version{draft}
\input{childdoc.def}
\childdocforward{cdocsamp}
%    \end{macrocode}

%\iffalse
%</sampledraft>
%\fi
%
% %%%%%%%%%%%%%%%%%%%%%%%%%%%%%%%%%%%%%%
% \paragraph{Forwarding for Final Version of the Chapters.}
%
% The following forwarding files |cdocsfn1.tex| and |cdocsfn2.tex|
% (with identical content)
% compile the final versions of the child documents
% |cdocsch1.tex| and |cdocsch2.tex|, respectively:
%\iffalse
%<*samplefinal>
%\fi
%    \begin{macrocode}
\def\version{final}
\input{childdoc.def}
\childdocforwardprefix[cdocsamp]{cdocsfn}{cdocsch}
%    \end{macrocode}

%\iffalse
%</samplefinal>
%\fi
%
% %%%%%%%%%%%%%%%%%%%%%%%%%%%%%%%%%%%%%%
% \paragraph{Command Line Processing.}
%
% The following three command lines generate the output files
% |cdocscld|, |cdocscl1| and |cdocscl2|
% which should be identical to
% |cdocsdrf|, |cdocsch1| and |cdocsfn2|, respectively:
% \begin{center}
% \begin{tabular}{l}
% |latex -jobname cdocscld \|\\
% |  "\def\version{draft}\input{childdoc.def}\childdocforward{cdocsamp}"|\\
% |latex -jobname cdocscl1 \|\\
% |  "\input{childdoc.def}\childdocforward[cdocsamp]{cdocsch1}"|\\
% |latex -jobname cdocscl2 \|\\
% |  "\def\version{final}\input{childdoc.def}\childdocforward{cdocsch2}"|
% \end{tabular}
% \end{center}
% Note that the trailing backslash on each first line
% merely continues the input to the second line
% (for convenient cut ant paste).
% Furthermore, the command |latex| can be replaced by any
% of its alternative versions such as |pdflatex|.
%
% %%%%%%%%%%%%%%%%%%%%%%%%%%%%%%%%%%%%%%%%%%%%%%%%%%%%%%%%%%%%%%%%%%%%%%%%%%%%%%
% %%%%%%%%%%%%%%%%%%%%%%%%%%%%%%%%%%%%%%%%%%%%%%%%%%%%%%%%%%%%%%%%%%%%%%%%%%%%%%
% \section{Implementation}
%\iffalse
%<*package>
%\fi
%
% This section describes the definitions file |childdoc.def|.

% The definitions cannot be loaded using |\usepackage| or |\RequirePackage|
% which has a mechanism to prevent loading a style file more than once.
% When loading the definitions by means of |\input|
% multiple instances have to be prevented manually:
%\iffalse
%This code needs to be before the `\ProvidesFile' directive
%which is defined at the beginning of this file.
%Therefore it is also placed there and commented out here.
%</package>
%<*discard>
%\fi
%    \begin{macrocode}
\ifdefined\childdocmain\endinput\fi
%    \end{macrocode}
%\iffalse
%</discard>
%<*package>
%\fi
%
% \macro{\ifchilddoc}
% \macro{\ifchilddocmanual}
% The conditional |\ifchilddoc| tells whether a
% child (true) or main (false) document is being compiled.
% The conditional |\ifchilddocmanual| tells whether
% the |\includeonly| mechanism is used (false) or
% the selection of child files must be performed manually (true).
% The definitions initialise to false:
%    \begin{macrocode}
\newif\ifchilddoc
\newif\ifchilddocmanual
%    \end{macrocode}

% \macro{\childdocname}
% \macro{\childdocjob}
% The macro |\childdocname| stores the name of the main document
% to be compiled. The macro |\childdocjob| stores the name of
% the document on which the \LaTeX{} compiler was originally invoked.
% The content of |\jobname| cannot be compared
% to filenames specified in the source due to different catcodes.
% The following code rescans |\jobname|, stores the result
% in |\childdocname| and saves a copy in |\childdocjob|:
%    \begin{macrocode}
\edef\childdocname{\scantokens\expandafter{\jobname\noexpand}}
\let\childdocjob\childdocname
%    \end{macrocode}

% \macro{\childdocdisable}
% The macro |\childdocdisable| prevents the main file
% from being processed more than once.
% At this stage, the main document command |\childdocmain|
% is assumed to be called once again where it should do nothing.
% Any subsequent call to it should prevent
% a secondary processing of the main document
% It overwrites the forwarding commands
% |\childdocof| and |\childdocforward|
% with empty macros to prevent further inclusions of the main document:
%    \begin{macrocode}
\newcommand{\childdocdisable}
{
  \renewcommand{\childdocmain}[1]{\renewcommand{\childdocmain}[1]{\endinput}}
  \renewcommand{\childdocof}[1]{}
  \renewcommand{\childdocby}[2][]{}
  \renewcommand{\childdocforward}[2][]{}
  \renewcommand{\childdocdisable}{}
}
%    \end{macrocode}

% \macro{\childdocmain}
% The macro |\childdocmain| is to be called at the top of the main file
% with nothing or the main filename (without extension) as argument.
% First, it breaks loops.
% If the argument is not empty and does not match |\childdocname|
% (which is set by the first inclusion of |childdoc.def|),
% |\ifchilddoc| is set to true, |\includeonly| is applied to the child file
% and |\jobname| is set to the main file
% (for proper handling of |.aux| files):
%    \begin{macrocode}
\newcommand{\childdocmain}[1]
{
  \childdocdisable\childdocmain{}
  \if?#1?\else
    \begingroup
      \def\childdoctmp{#1}
      \ifx\childdoctmp\childdocname
        \def\childdoctmp{}
      \else
        \def\childdoctmp
        {
          \childdoctrue
          \includeonly{\childdocname}
          \def\childdocjob{#1}
          \def\jobname{#1}
        }
      \fi
      \expandafter
    \endgroup
    \childdoctmp
  \fi
}
%    \end{macrocode}

% \macro{\childdocof}
% The command |\childdocof| redirects
% compilation to the main file |#1|.
%    \begin{macrocode}
\newcommand{\childdocof}[1]
{
  \childdocdisable
  \childdoctrue
  \includeonly{\childdocname}
  \def\jobname{#1}
  \def\childdocjob{#1}
  \input{#1}
}
%    \end{macrocode}

% \macro{\childdocby}
% The command |\childdocby| ....
%    \begin{macrocode}
\newcommand{\childdocby}[2][]
{
  \childdocdisable
  \childdoctrue
  \childdocmanualtrue
  \if?#1?\else
    \def\jobname{#2}
  \fi
  \def\childdocjob{#2}
  \input{#2}
  \endinput
}
%    \end{macrocode}

% \macro{\childdocforward}
% The command |\childdocforward| redirects
% compilation to the main file or
% (if the optional argument is given) a child file.
% Parameters are set as if the main file
% or a child file starting with |\childdocof| was compiled.
% Then compilation is handed over to the main file:
%    \begin{macrocode}
\newcommand{\childdocforward}[2][]
{
  \begingroup
    \if?#1?
      \def\childdoctmp
      {
        \def\childdocname{#2}
        \def\childdocjob{#2}
        \def\jobname{#2}
        \input{#2}
        \endinput
      }
    \else
      \def\childdoctmp
      {
        \childdocdisable
        \def\childdocname{#2}
        \childdoctrue
        \includeonly{#2}
        \def\childdocjob{#1}
        \def\jobname{#1}
        \input{#1}
        \endinput
      }
    \fi
    \expandafter
  \endgroup
  \childdoctmp
}
%    \end{macrocode}

% \macro{\childdocforwardprefix}
% The command |\childdocforwardprefix| redirects
% compilation to the main or a child file by means of a pattern.
% The prefix |#1| in the current filename is replaced by |#2|
% and the suffix of the current filename is kept
% (it is assumed that the filename does not contain the substring `|~~~|'
% which is used as a delimiter).
% Compilation is handed over to the new file by |\childdocforward|:
%    \begin{macrocode}
\newcommand{\childdocforwardprefix}[3][]
{
  \begingroup
    \def\childdocextract #2##1~~~{\def\childdoctmp{\childdocforward[#1]{#3##1}}}
    \expandafter\childdocextract\childdocname~~~
    \expandafter
  \endgroup
  \childdoctmp
}
%    \end{macrocode}

% \macro{\childdoc}
% The deprecated macro |\childdoc| is a legacy version of |\childdocmain|:
%    \begin{macrocode}
\newcommand{\childdoc}{\childdocmain}
%    \end{macrocode}

% \macro{\childdocredirect}
% The deprecated macro |\childdocredirect| is a legacy version
% of |\childdocforward| and |\childdocforwardprefix|:
%    \begin{macrocode}
\newcommand{\childdocredirect}[2][]
{
  \begingroup
    \if?#1?
      \def\childdoctmp{\childdocforward{#2}}
    \else
      \def\childdoctmp{\childdocforwardprefix{#1}{#2}}
    \fi
    \expandafter
  \endgroup
  \childdoctmp
}
%    \end{macrocode}

%\iffalse
%</package>
%\fi
%
\endinput
\childdocforward[cdocsamp]{cdocsch1}"|\\
% |latex -jobname cdocscl2 \|\\
% |  "\def\version{final}% \iffalse
%
% childdoc.dtx Copyright (C) 2017-2018 Niklas Beisert
%
% This work may be distributed and/or modified under the
% conditions of the LaTeX Project Public License, either version 1.3
% of this license or (at your option) any later version.
% The latest version of this license is in
%   http://www.latex-project.org/lppl.txt
% and version 1.3 or later is part of all distributions of LaTeX
% version 2005/12/01 or later.
%
% This work has the LPPL maintenance status `maintained'.
%
% The Current Maintainer of this work is Niklas Beisert.
%
% This work consists of the files childdoc.dtx and childdoc.ins
% and the derived files childdoc.def and cdocsamp.tex with
% cdocsch1.tex, cdocsch2.tex, cdocsdrf.tex, cdocsfn1.tex, cdocsfn2.tex.
%
%<package>\ifdefined\childdocmain\endinput\fi
%<package>\ProvidesFile{childdoc.def}[2018/12/30 v2.0 child document driver]
%<samplemain>\ProvidesFile{cdocsamp.tex}[2018/12/30 v2.0 sample for childdoc]
%<*driver>
%\ProvidesFile{childdoc.drv}[2018/12/30 v2.0 childdoc reference manual file]
\PassOptionsToClass{10pt,a4paper}{article}
\documentclass{ltxdoc}

\usepackage[margin=35mm]{geometry}
\usepackage{hyperref}
\usepackage{hyperxmp}
\usepackage[usenames]{color}

\hypersetup{colorlinks=true}
\hypersetup{pdfstartview=FitH}
\hypersetup{pdfpagemode=UseNone}
\hypersetup{pdfsource={}}
\hypersetup{pdflang={en-UK}}
\hypersetup{pdfcopyright={Copyright 2017-2018 Niklas Beisert.
  This work may be distributed and/or modified under the
  conditions of the LaTeX Project Public License, either version 1.3
  of this license or (at your option) any later version.}}
\hypersetup{pdflicenseurl={http://www.latex-project.org/lppl.txt}}
\hypersetup{pdfcontactaddress={ETH Zurich, ITP, HIT K,
  Wolfgang-Pauli-Strasse 27}}
\hypersetup{pdfcontactpostcode={8093}}
\hypersetup{pdfcontactcity={Zurich}}
\hypersetup{pdfcontactcountry={Switzerland}}
\hypersetup{pdfcontactemail={nbeisert@itp.phys.ethz.ch}}
\hypersetup{pdfcontacturl={http://people.phys.ethz.ch/\xmptilde nbeisert/}}

\newcommand{\secref}[1]{\hyperref[#1]{section \ref*{#1}}}

\parskip1ex
\parindent0pt
\let\olditemize\itemize
\def\itemize{\olditemize\parskip0pt}

\begin{document}

\title{The \textsf{childdoc} Package}
\hypersetup{pdftitle={The childdoc Package}}
\author{Niklas Beisert\\[2ex]
  Institut f\"ur Theoretische Physik\\
  Eidgen\"ossische Technische Hochschule Z\"urich\\
  Wolfgang-Pauli-Strasse 27, 8093 Z\"urich, Switzerland\\[1ex]
  \href{mailto:nbeisert@itp.phys.ethz.ch}
  {\texttt{nbeisert@itp.phys.ethz.ch}}}
\hypersetup{pdfauthor={Niklas Beisert}}
\hypersetup{pdfsubject={Manual for the LaTeX2e Package childdoc}}
\date{30 December 2018, \textsf{v2.0}}
\maketitle

\begin{abstract}\noindent
\textsf{childdoc} is a \LaTeXe{} package
that enables the direct compilation
of document sections included by |\include|
to individual files.
\end{abstract}

\begingroup
\parskip0ex
\tableofcontents
\endgroup

%%%%%%%%%%%%%%%%%%%%%%%%%%%%%%%%%%%%%%%%%%%%%%%%%%%%%%%%%%%%%%%%%%%%%%%%%%%%%%%%
%%%%%%%%%%%%%%%%%%%%%%%%%%%%%%%%%%%%%%%%%%%%%%%%%%%%%%%%%%%%%%%%%%%%%%%%%%%%%%%%
\section{Introduction}

\LaTeX{} provides a mechanism to structure a large document (such as a book)
into a main file and several child files (containing the chapters)
using the |\include| command.
This mechanism is beneficial for documents
which span hundreds of pages in order to
make the source file(s) more manageable.
Moreover, compilation can be restricted to
selected child files by means of the |\includeonly| command.
The latter feature can be used to reduce the compilation time while editing
(this was significantly more useful in the earlier days of \LaTeX{})
or to generate a smaller document which is easier to navigate.
Another application of |\includeonly| is to generate
documents consisting of selected parts of the complete document.

However, there are a few drawbacks of the plain |\include| mechanism:
\begin{itemize}
\item
The child files cannot be compiled on their own,
they can only be compiled via the main file.
A naive editing environment
(such as a text editor with an option
to have the current file processed by \LaTeX)
may require one to switch to the main file before compiling;
attempting to compile the child file produces errors.
\item
The main file must be modified (each time)
to adjust the |\includeonly| command
to the present needs. This easily leaves the main file in a messy state.
\item
The generated document will always carry the filename
of the main document. This is inconvenient if
several child files are to be compiled and
to be kept for distribution.
\end{itemize}

The present package provides a simple interface
to make child files individually compilable by \LaTeX{}.
Compiling a child file then has the same effect as compiling
the main file with an |\includeonly| command
to select the appropriate child.
Moreover the generated document will carry the name of the child
rather than the main file.
This resolves all three above issues.

This feature is meant to make the editing of books,
thesis documents and lecture notes somewhat more convenient.
However, the package can also be used efficiently for
composing a series of documents (such as exercise sheets)
which are typically distributed individually.
It then assists the author in generating the individual documents
(potentially in different versions)
as well as a document containing the collected series.
Another application is in developing style files
or other kinds of included material
where compilation of the style file could redirect
to a sample or test file.

%%%%%%%%%%%%%%%%%%%%%%%%%%%%%%%%%%%%%%%%%%%%%%%%%%%%%%%%%%%%%%%%%%%%%%%%%%%%%%%%
%%%%%%%%%%%%%%%%%%%%%%%%%%%%%%%%%%%%%%%%%%%%%%%%%%%%%%%%%%%%%%%%%%%%%%%%%%%%%%%%
\section{Usage}

First of all, the package \textsf{childdoc} is \emph{not} a standard
\LaTeXe{} |.sty| style file! Therefore it needs to be invoked in
a non-standard way.

%%%%%%%%%%%%%%%%%%%%%%%%%%%%%%%%%%%%%%%%%%%%%%%%%%%%%%%%%%%%%%%%%%%%%%%%%%%%%%%%
\subsection{Included Files}
\label{sec:include}

%%%%%%%%%%%%%%%%%%%%%%%%%%%%%%%%%%%%%%%%
\DescribeMacro{\childdocmain}
To use the package, add the commands
\begin{center}
\begin{tabular}{l}
|\input{childdoc.def}|\\
|\childdocmain{}|\\
\end{tabular}
\end{center}
at the very top of the main \LaTeX{} file,
in particular \emph{before} the |\documentclass| statement!
The argument of |\childdocmain| should be left empty
(but it must be present).

%%%%%%%%%%%%%%%%%%%%%%%%%%%%%%%%%%%%%%%%
\DescribeMacro{\childdocof}
Furthermore, add the commands
\begin{center}
\begin{tabular}{l}
|\input{childdoc.def}|\\
|\childdocof{|\textit{main}|}|\\
\end{tabular}
\end{center}
at the top of every child file \textit{child}
which is included by |\include{|\textit{child}|}|
from within the main file
(or at least for those files to be compiled individually).
The argument \textit{main} must be the filename of the main file.

There are a couple of
considerations in setting up the main and child documents:

%%%%%%%%%%%%%%%%%%%%%%%%%%%%%%%%%%%%%%%%
\paragraph{Restrictions.}

Please note the following restrictions:
\begin{itemize}
\item
|\childdocmain| must be called with one argument \textit{main}
to ensure compatibility with earlier version of the package.
It must either be empty (|\childdocmain{}|)
or precisely match the filename of the main file in which it is specified.
See \secref{sec:detection} for further information.
\item
The filename \textit{main} must be specified without the |.tex| extension.
\item
The filename \textit{main} is case sensitive
(even in case-insensitive file systems)
due to internal string comparison.
\item
The argument \textit{main} should be fully expanded, it cannot be a macro.
\item
Subdirectories and special characters should be avoided in filenames.
\item
The command |\childdocmain{|\textit{main}|}| must be followed by a whitespace.
It should not be followed immediately by another command
or by a comment mark `|%|'.
This is because the \TeX{} parser reads the token immediately following
the argument of |\childdocmain| and puts it
at the beginning of every child section;
however, a white\-space is ignored.
\end{itemize}

%%%%%%%%%%%%%%%%%%%%%%%%%%%%%%%%%%%%%%%%
\paragraph{Content of Main File.}

It is advisable to place all content in the child files included by |\include|.
Any output contained in the main file will appear in all child documents
unless suppressed manually;
it cannot be suppressed automatically by the |\includeonly| directive
and thus should normally be avoided.
A method to include some content in the main file
by means of conditional processing is described in \secref{sec:conditional}.

%%%%%%%%%%%%%%%%%%%%%%%%%%%%%%%%%%%%%%%%
\paragraph{Page Numbering.}

When only a part of the document is compiled,
the appropriate numbering of pages
(as well as other status parameters)
is determined from the |.aux| files.
The latter contain information from previous passes.
However this information needs to propagate through
all intermediate child documents.
Therefore the page numbering in child documents may well
be inconsistent until the complete document is compiled at least once.

A useful (if unconventional) way to always ensure a consistent
page numbering is to restart the numbering in each child document
and denote the pages by `\textit{child}|.|\textit{page}'
where \textit{child} represents the chapter/section number of the child file.
This can be achieved by the command
|\numberwithin{page}{|\textit{child}|}|
of the \textsf{amsmath} package
where \textit{child} can be |chapter| or |section|
depending on the chosen structuring.
Alternatively, one can modify the macro |\thepage| appropriately
and reset the counter |page| at the start of each child file.

%%%%%%%%%%%%%%%%%%%%%%%%%%%%%%%%%%%%%%%%%%%%%%%%%%%%%%%%%%%%%%%%%%%%%%%%%%%%%%%%
\subsection{Conditional Processing}
\label{sec:conditional}

The package provides a mechanism to compile different versions
of a document. To customise the versions further some conditional processing
can come in handy to distinguish which version is being compiled.
The package provides two macros to describe the compilation context:

%%%%%%%%%%%%%%%%%%%%%%%%%%%%%%%%%%%%%%%%
\DescribeMacro{\ifchilddoc}
The conditional |\ifchilddoc| distinguishes between the compilation of
child documents and the main document:
%
\begin{center}
|\ifchilddoc |\textit{child-code}| |[|\||else |\textit{main-code}]| \||fi|
\end{center}

%%%%%%%%%%%%%%%%%%%%%%%%%%%%%%%%%%%%%%%%
\DescribeMacro{\childdocname}
\DescribeMacro{\childdocjob}
The macro |\childdocname| contains the filename (without extension)
of the main or child file being processed.
Note that |\childdocjob| will always contain the name of the main file.

%%%%%%%%%%%%%%%%%%%%%%%%%%%%%%%%%%%%%%%%
\paragraph{Title Page.}

Conditional processing can be used to include a title or banner page
in the main document when proper precautions are taken.
Importantly, the code in the main file should ensure that the page counter
(as well as other status parameters which are stored in the |.aux| files)
takes the same value after the conditional processing.
Otherwise the page numbers may take divergent values
depending on which part is compiled.

For example, a title page could be declared by:
%
\begin{center}
\begin{tabular}{l}
|\ifchilddoc\||else|\\
|\addtocounter{page}{-1}|\\
\textit{code for title page}\\
|\newpage|\\
|\||fi|
\end{tabular}
\end{center}
%
A banner page for the child documents can be generated by:
%
\begin{center}
\begin{tabular}{l}
|\ifchilddoc|\\
|\addtocounter{page}{-1}|\\
\textit{code for banner page}\\
|\newpage|\\
|\||fi|
\end{tabular}
\end{center}
%
Here one could write a message such as:
\begin{center}
|This is the part \childdocname{} of \childdocjob{}.|
\end{center}

%%%%%%%%%%%%%%%%%%%%%%%%%%%%%%%%%%%%%%%%%%%%%%%%%%%%%%%%%%%%%%%%%%%%%%%%%%%%%%%%
\subsection{Flags}
\label{sec:flags}

The package makes it easy to generate different versions
of the main or child documents.
To this end compilation flags can be defined
and assigned different default values.
They will be particularly useful in conjunction
with the forwarding mechanism described in \secref{sec:forward}.

For example, it may be useful to have a flag |\version|
which can be set to |draft| or |final|.
The document source will contain some conditional code
depending on the value of |\version|.
Suppose further, the flag should default to |final| for the main file
and to |draft| for child files
which is a natural assignment for editing the document.
This is achieved by placing the following code
in the preamble of the main document
(below the |\childdocmain| directive):
%
\begin{center}
\begin{tabular}{l}
|\ifchilddoc|\\
|\providecommand{\version}{draft}|\\
|\||else|\\
|\providecommand{\version}{final}|\\
|\||fi|
\end{tabular}
\end{center}
%
The definition by |\providecommand| makes sure
that previous definitions are not overwritten.
Further statements |\providecommand{\version}{...}|
can thus be added before the above code to override it.

For the main file, one might add a line
(between |\childdocmain| and the above block)
%
\begin{center}
|%\ifchilddoc\||else\providecommand{\version}{draft}\||fi|
\end{center}
%
which can be uncommented to produce a draft version.
Likewise one can add a line to the very top of a child file
(above the |\childdocof{|\textit{main}|}| directive)
%
\begin{center}
|%\providecommand{\version}{final}|
\end{center}
%
which can be uncommented to produce the final version of this child document.

%%%%%%%%%%%%%%%%%%%%%%%%%%%%%%%%%%%%%%%%%%%%%%%%%%%%%%%%%%%%%%%%%%%%%%%%%%%%%%%%
\subsection{Forwarding}
\label{sec:forward}

Different versions of the main or child documents
using compilation flags as described in \secref{sec:flags}
can be (permanently) stored in different files
for convenient compilation, viewing and distribution.
To this end, the package defines a command
to pass on compilation to a different file:

%%%%%%%%%%%%%%%%%%%%%%%%%%%%%%%%%%%%%%%%
\DescribeMacro{\childdocforward}
The command |\childdocforward| redirects processing to
another source file:
%
\begin{center}
\begin{tabular}{l}
|\input{childdoc.def}|\\
|\childdocforward[|\textit{main}|]{|\textit{dest}|}|\\
\end{tabular}
\end{center}
%
The argument \textit{dest} is the destination file
(without extension).
It should be the main file or one of the child files.
Note that further \textsf{childdoc} directives
such as |\childdocof| and |\childdocforward|
in the indicated file will be processed in this form.
The optional argument \textit{main}
passes on directly to the main file \textit{main}
while pretending to compile the child \textit{dest}.
This form behaves as if \textit{dest}
issues |\childdocof{|\textit{main}|}| right away,
and no further \textsf{childdoc} directives will be processed.

%%%%%%%%%%%%%%%%%%%%%%%%%%%%%%%%%%%%%%%%
\DescribeMacro{\...prefix}
In the alternative form |\childdocforwardprefix|,
%
\begin{center}
\begin{tabular}{l}
|\input{childdoc.def}|\\
|\childdocforwardprefix[|\textit{main}|]{|\textit{prefix}|}{|\textit{dest}|}|
\end{tabular}
\end{center}
%
the destination file is determined by a pattern
depending on the current file:
To make this work, the current file must be called
`{\textit{prefix}\hspace{0.2em}\textit{suffix}}'
with \textit{prefix} matching precisely the argument.
Processing is then passed on to the file
`{\textit{dest}\hspace{0.2em}\textit{suffix}}'.
Surely, the same effect is achieved by
directly specifying the
argument `{\textit{dest}\hspace{0.2em}\textit{suffix}}'
in the first form.
However, that requires to set up a different file
for each child. With the alternative form of the command
all these files can have exactly the same content
which simplifies setting them up and maintaining them.

For example, the following file |draft.tex|
with a compilation flag |\version| as described in \secref{sec:flags}
compiles the main document as a draft:
%
\begin{center}
\begin{tabular}{l}
|\def\version{draft}|\\
|\input{childdoc.def}|\\
|\childdocforward{|\textit{main}|}|
\end{tabular}
\end{center}
%
Likewise, the following files |final|\textit{nn}|.tex|
compile the final version of the child document
|child|\textit{nn}|.tex|:
%
\begin{center}
\begin{tabular}{l}
|\def\version{final}|\\
|\input{childdoc.def}|\\
|\childdocforwardprefix{final}{child}|
\end{tabular}
\end{center}
%

Note that when several versions of a main file and/or of each child file
are to be generated, it may be convenient to set up a |Makefile| or
shell script to automatise the process.

%%%%%%%%%%%%%%%%%%%%%%%%%%%%%%%%%%%%%%%%%%%%%%%%%%%%%%%%%%%%%%%%%%%%%%%%%%%%%%%%
\subsection{Command Line Processing}
\label{sec:commandline}

The effect of redirection files can also be achieved by invoking
the \LaTeX{} compiler with a more elaborate command line.
Most conveniently this should be done as part
of a shell script or a |Makefile|.

When using \textsf{childdoc} in the main file, the following
command lines effectively perform a redirection
(note that depending on the shell being used,
backslashes may have to be doubled: `|\|' $\to$ `|\\|'):
%
\begin{center}
|... -jobname "|\textit{target}|" |\\|"|[\textit{flags}]%
|\input{childdoc.def}\childdocforward[|\textit{main}|]{|\textit{dest}|}"|
\end{center}
%
Here \textit{target} is the name of the output file,
\textit{main} is the name of the main file
and \textit{dest} is the name of the main or child file to be processed
(all filenames without extensions).
The optional argument \textit{main} can be omitted
if \textit{main} matches \textit{dest}.
Optionally, compilation \textit{flags} can be defined via |\def| commands.
This command line makes the \TeX{} engine believe
it is compiling the file \textit{target}
whose content is specified as the latter parameter.
The provided code then forwards the processing to
\textit{main} or \textit{dest} as described in \secref{sec:forward}.

%%%%%%%%%%%%%%%%%%%%%%%%%%%%%%%%%%%%%%%%%%%%%%%%%%%%%%%%%%%%%%%%%%%%%%%%%%%%%%%%
\subsection{Include by Input}
\label{sec:input}

Including child documents by |\include| has some restrictions by design.
Most notably, the content of a child document always occupies
its own set of pages; pages cannot be shared between child documents.
Usually, this behaviour makes perfect sense
because each child document contain an essential part of the document.
However, in some situations it may be desirable to compose
a document from a collection of parts
without having mandatory page breaks between then.
For this case, the package
provides a mechanism to include parts
by |\input| which can also be processed individually.
However, by construction this mechanism
requires manual handling of the content to be output.

%%%%%%%%%%%%%%%%%%%%%%%%%%%%%%%%%%%%%%%%
\DescribeMacro{\ifchilddocmanual}
The main file should be prepared as usual, see \secref{sec:include}.
However, the document body must make a distinction
between processing of an individual part and of the main document, e.g.:
%
\begin{center}
\begin{tabular}{l}
|\ifchilddocmanual|\\
|\input{\childdocname}|\\
|\||else|\\
\textit{document body with }|\input{|\textit{part}|}|\\
|\||fi|
\end{tabular}
\end{center}
%
The conditional |\ifchilddocmanual| is true whenever
a part to be included by |\input| is being compiled,
and the name of the part is stored in |\childdocname|.

%%%%%%%%%%%%%%%%%%%%%%%%%%%%%%%%%%%%%%%%
\DescribeMacro{\childdocby}
Each part to be included by |\input| should start with:
%
\begin{center}
\begin{tabular}{l}
|\input{childdoc.def}|\\
|\childdocby{|\textit{main}|}|\\
\end{tabular}
\end{center}
%
The directive |\childdocby| is similar to |\childdocof|
described in \secref{sec:include},
but the subsequent selection of content must be done manually.
To that end, both |\ifchilddoc| and |\ifchilddocmanual|
will be true upon processing of a part,
and the name of the part is stored in |\childdocname|.
Note that |\jobname| will be set to the filename of the current part
so that each part receives an individual |.aux| file
that does not interfere with the |.aux| file(s) of the main document.
This behaviour can be altered by the alternative form
|\childdocby[*]{|\textit{main}|}| (with a non-empty optional argument)
which uses the |.aux| file of the main document
by setting |\jobname| to \textit{main}.

%%%%%%%%%%%%%%%%%%%%%%%%%%%%%%%%%%%%%%%%%%%%%%%%%%%%%%%%%%%%%%%%%%%%%%%%%%%%%%%%
\subsection{Driver Development}
\label{sec:driver}

The \textsf{childdoc} mechanism can also be use for the development
of definition files such as \LaTeX{} styles or classes.
This case differs from the above setup with multiple parts
included by |\include| in that no |\includeonly| should be invoked.
This can be achieved by starting the include file
(before |\ProvidesPackage|) with:
%
\begin{center}
\begin{tabular}{l}
|\input{childdoc.def}|\\
|\childdocforward{|\textit{main}|}|\\
\end{tabular}
\end{center}
%
or alternatively with:
%
\begin{center}
\begin{tabular}{l}
|\input{childdoc.def}|\\
|\childdocby{|\textit{main}|}|\\
\end{tabular}
\end{center}
%
Both forms have slightly different effects as described above.
The main file is prepared as usual, see \secref{sec:include}.

%%%%%%%%%%%%%%%%%%%%%%%%%%%%%%%%%%%%%%%%%%%%%%%%%%%%%%%%%%%%%%%%%%%%%%%%%%%%%%%%
\subsection{Legacy Detection}
\label{sec:detection}

The directive |\childdocmain| in the main file can detect
whether the complete document or merely a child is to be compiled
even without using the directive |\childdocof|.
This method is deprecated because it is less robust
and there is no compelling reason to use it;
it is merely provided for backward compatibility
and it may be removed in future versions.

If the detection mechanism is to be used,
it is mandatory to correctly specify
the filename of the main file as the argument of |\childdocmain|:
%
\begin{center}
\begin{tabular}{l}
|\input{childdoc.def}|\\
|\childdocmain{|\textit{main}|}|\\
\end{tabular}
\end{center}
%
If |\jobname| does not match the argument \textit{main} of |\childdocmain|,
it is assumed that |\jobname| points to the child file to be compiled.
When using |\childdocmain| with the main file specified as argument,
it suffices to start a child file
with just |\input{|\textit{main}|}|
without loading of the package and using |\childdocof|.
If instead all processing is done
with the appropriate \textsf{childdoc} directives,
the argument of \textit{main} of |\childdocmain| can be empty.

An alternative version of the command line processing described
in \secref{sec:commandline} using the detection mechanism reads:
%
\begin{center}
|... -jobname "|\textit{target}|" "|[\textit{flags}]%
[|\def\jobname{|\textit{dest}|}|]|\input{|\textit{main}|}"|
\end{center}

%%%%%%%%%%%%%%%%%%%%%%%%%%%%%%%%%%%%%%%%%%%%%%%%%%%%%%%%%%%%%%%%%%%%%%%%%%%%%%%%
\subsection{Manual Code}
\label{sec:manual}

In case one cannot be certain whether the definitions file |childdoc.def|
is installed on the target \TeX{} distribution
and one prefers not to ship it,
it is conceivable to paste a few relevant commands into the sources.

To that end, drop all statements |\input{childdoc.def}|
and perform the replacements as outlined below.
Instead of |\childdocmain{|\textit{main}|}| add the following code
to the top of the main file:
%
\begin{center}
\begin{tabular}{l}
|\||ifdefined\childdocname\endinput\||fi\newif\ifchilddoc|\\
|\edef\childdocname{\scantokens\expandafter{\jobname\noexpand}}|\\
|\def\childdocmain{|\textit{main}|}\||ifx\childdocmain\childdocname\||else|\\
|\childdoctrue\includeonly{\childdocname}\let\jobname\childdocmain\||fi|\\
\end{tabular}
\end{center}
%
Instead of |\childdocof{|\textit{main}|}| just include the main file
at the top of each child file:
%
\begin{center}
|\input{|\textit{main}|}|
\end{center}
%
A simple redirection |\childdocforward{|\textit{dest}|}| is achieved by:
%
\begin{center}
|\def\jobname{|\textit{dest}|}\input{\jobname}|
\end{center}
%
The redirection with prefix
|\childdocforwardprefix[|\textit{prefix}|]{|\textit{dest}|}|
is accomplished by:
%
\begin{center}
\begin{tabular}{l}
|{\edef\jobname{\scantokens\expandafter{\jobname\noexpand}}|\\
|\def\redirectjob |\textit{prefix}|#1~~~{\gdef\jobname{|\textit{dest}|#1}}|\\
|\expandafter\redirectjob\jobname~~~}\input{\jobname}|
\end{tabular}
\end{center}

In an alternative approach,
child documents can be compiled by a specific command line
without additional code or specific definitions:
%
\begin{center}
|... -jobname "|\textit{target}|" "|[\textit{flags}]%
|\includeonly{|\textit{dest}|}\input{|\textit{main}|}"|
\end{center}
%

%%%%%%%%%%%%%%%%%%%%%%%%%%%%%%%%%%%%%%%%%%%%%%%%%%%%%%%%%%%%%%%%%%%%%%%%%%%%%%%%
%%%%%%%%%%%%%%%%%%%%%%%%%%%%%%%%%%%%%%%%%%%%%%%%%%%%%%%%%%%%%%%%%%%%%%%%%%%%%%%%
\section{Information}

%%%%%%%%%%%%%%%%%%%%%%%%%%%%%%%%%%%%%%%%%%%%%%%%%%%%%%%%%%%%%%%%%%%%%%%%%%%%%%%%
\subsection{Copyright}

Copyright \copyright{} 2017--2018 Niklas Beisert

This work may be distributed and/or modified under the
conditions of the \LaTeX{} Project Public License, either version 1.3
of this license or (at your option) any later version.
The latest version of this license is in
  \url{http://www.latex-project.org/lppl.txt}
and version 1.3 or later is part of all distributions of \LaTeX{}
version 2005/12/01 or later.

This work has the LPPL maintenance status `maintained'.

The Current Maintainer of this work is Niklas Beisert.

This work consists of the files |README.txt|, |childdoc.ins| and |childdoc.dtx|
as well as the derived files |childdoc.def|, |cdocsamp.tex|
with |cdocsch1.tex|, |cdocsch2.tex|, |cdocspt3.tex|, |cdocspt4.tex|,
|cdocsdrf.tex|, |cdocsfn1.tex|, |cdocsfn2.tex|
as well as |childdoc.pdf|.

%%%%%%%%%%%%%%%%%%%%%%%%%%%%%%%%%%%%%%%%%%%%%%%%%%%%%%%%%%%%%%%%%%%%%%%%%%%%%%%%
\subsection{Files and Installation}

The package consists of the files:
%
\begin{center}
\begin{tabular}{ll}
    |README.txt|   & readme file \\
    |childdoc.ins| & installation file \\
    |childdoc.dtx| & source file \\
    |childdoc.def| & definition file \\
    |cdocsamp.tex| & sample main file \\
    |cdocsch1.tex| & sample include file \\
    |cdocsch2.tex| & sample include file \\
    |cdocspt3.tex| & sample part file \\
    |cdocspt4.tex| & sample part file \\
    |cdocsdrf.tex| & sample redirection file \\
    |cdocsfn1.tex| & sample redirection file \\
    |cdocsfn2.tex| & sample redirection file \\
    |childdoc.pdf| & manual
\end{tabular}
\end{center}
%
The distribution consists of the files
|README.txt|, |childdoc.ins| and |childdoc.dtx|.
%
\begin{itemize}
\item
Run (pdf)\LaTeX{} on |childdoc.dtx|
to compile the manual |childdoc.pdf| (this file).
\item
Run \LaTeX{} on |childdoc.ins| to create the definitions file |childdoc.def|
and the sample |cdocsamp.tex| with include files
|cdocsch1.tex|, |cdocsch2.tex|, |cdocspt3.tex|, |cdocspt4.tex|,
|cdocsdrf.tex|, |cdocsfn1.tex|, |cdocsfn2.tex|.
Then copy the file |childdoc.def| to an appropriate directory of your \LaTeX{}
distribution, e.g.\ \textit{texmf-root}|/tex/latex/childdoc|.
\end{itemize}

%%%%%%%%%%%%%%%%%%%%%%%%%%%%%%%%%%%%%%%%%%%%%%%%%%%%%%%%%%%%%%%%%%%%%%%%%%%%%%%%
\subsection{Related CTAN Packages}

There are several other packages which offer a similar functionality:
%
\begin{itemize}
\item
The packages
\href{http://ctan.org/pkg/docmute}{\textsf{docmute}},
\href{http://ctan.org/pkg/includex}{\textsf{includex}} and
\href{http://ctan.org/pkg/standalone}{\textsf{standalone}}
provide commands to include only the document body of
a child file thus allowing both files to be compiled individually.
\item
The packages \href{http://ctan.org/pkg/subdocs}{\textsf{subdocs}}
and \href{http://ctan.org/pkg/subfiles}{\textsf{subfiles}}
provide structures in which the main and child documents can be
encapsulated and allowing them to be compiled individually.
The inclusion mechanism is different from the conventional |\include|.
\item
The package \href{http://ctan.org/pkg/combine}{\textsf{combine}}
is an elaborate solution to combine several documents into one.
\end{itemize}
%
See also the CTAN topic \href{http://ctan.org/topic/subdocs}{\textsf{subdocs}}
for further related packages.
The present package differs from the above solutions in that
a document structure constructed with the conventional |\include| mechanism
just needs two extra commands at the top of every file
such that all constituent files can be compiled individually.

%%%%%%%%%%%%%%%%%%%%%%%%%%%%%%%%%%%%%%%%%%%%%%%%%%%%%%%%%%%%%%%%%%%%%%%%%%%%%%%%
%\subsection{Feature Suggestions}
%
%The following is a list of features which may be useful for future
%versions of this package:
%%
%\begin{itemize}
%\item
%\ldots
%\end{itemize}

%%%%%%%%%%%%%%%%%%%%%%%%%%%%%%%%%%%%%%%%%%%%%%%%%%%%%%%%%%%%%%%%%%%%%%%%%%%%%%%%
\subsection{Revision History}

%%%%%%%%%%%%%%%%%%%%%%%%%%%%%%%%%%%%%%%%
\paragraph{v2.0:} 2018/12/30

\begin{itemize}
\item
immediate forward processing
\item
added |\childdocby| mechanism
\item
manual restructured
\end{itemize}

%%%%%%%%%%%%%%%%%%%%%%%%%%%%%%%%%%%%%%%%
\paragraph{v1.6:} 2018/01/17

\begin{itemize}
\item
application for development of include files
\item
corrections to manual
\end{itemize}

%%%%%%%%%%%%%%%%%%%%%%%%%%%%%%%%%%%%%%%%
\paragraph{v1.5:} 2017/05/21

\begin{itemize}
\item
more complete structuring introduced
\item
|\childdocof| introduced
\item
|\childdoc| renamed to |\childdocmain|
\item
|\childredirect| renamed to |\childdocforward| and |\childdocforwardprefix|
and functionality expanded
\end{itemize}

%%%%%%%%%%%%%%%%%%%%%%%%%%%%%%%%%%%%%%%%
\paragraph{v1.0:} 2017/04/27

\begin{itemize}
\item
manual and install package
\item
first version published on CTAN
\end{itemize}

%%%%%%%%%%%%%%%%%%%%%%%%%%%%%%%%%%%%%%%%
\paragraph{v0.6:} 2017/04/26

\begin{itemize}
\item
redirection mechanism added
\end{itemize}

%%%%%%%%%%%%%%%%%%%%%%%%%%%%%%%%%%%%%%%%
\paragraph{v0.5:} 2017/04/26

\begin{itemize}
\item
functionality in definition file
\end{itemize}


%%%%%%%%%%%%%%%%%%%%%%%%%%%%%%%%%%%%%%%%%%%%%%%%%%%%%%%%%%%%%%%%%%%%%%%%%%%%%%%%
%%%%%%%%%%%%%%%%%%%%%%%%%%%%%%%%%%%%%%%%%%%%%%%%%%%%%%%%%%%%%%%%%%%%%%%%%%%%%%%%
%%%%%%%%%%%%%%%%%%%%%%%%%%%%%%%%%%%%%%%%%%%%%%%%%%%%%%%%%%%%%%%%%%%%%%%%%%%%%%%%
\appendix

\settowidth\MacroIndent{\rmfamily\scriptsize 000\ }

 \DocInput{childdoc.dtx}

\end{document}
%</driver>
% \fi
%
% %%%%%%%%%%%%%%%%%%%%%%%%%%%%%%%%%%%%%%%%%%%%%%%%%%%%%%%%%%%%%%%%%%%%%%%%%%%%%%
% %%%%%%%%%%%%%%%%%%%%%%%%%%%%%%%%%%%%%%%%%%%%%%%%%%%%%%%%%%%%%%%%%%%%%%%%%%%%%%
% \section{Sample}
%\iffalse
%<*samplemain>
%\fi
%
% The following presents a sample document
% with two chapters, two parts, a title page,
% a compile flag as well as three forwarding files to set the flag.
% It consists of eight |.tex| files:
% \begin{center}
% \begin{tabular}{ll}
% |cdocsamp.tex|&main file\\
% |cdocsch1.tex|&include file for chapter 1\\
% |cdocsch2.tex|&include file for chapter 2\\
% |cdocspt3.tex|&include file for part 3\\
% |cdocspt4.tex|&include file for part 4\\
% |cdocsdrf.tex|&forwarding file for main file in draft mode\\
% |cdocsfi1.tex|&forwarding file for final version of chapter 1\\
% |cdocsfi2.tex|&forwarding file for final version of chapter 2\\
% \end{tabular}
% \end{center}
% Each of the eight files can be compiled directly by the \LaTeX{} compiler.
%
% %%%%%%%%%%%%%%%%%%%%%%%%%%%%%%%%%%%%%%
% \paragraph{Main File.}
%
% The main file is called |cdocsamp.tex|.
%
% Load the \textsf{childdoc} definitions and
% declare the filename for the main document:
%    \begin{macrocode}
\input{childdoc.def}
\childdocmain{}
%    \end{macrocode}

% Optional override for |\version| flag:
%    \begin{macrocode}
%%\ifchilddoc\else\providecommand{\version}{draft}\fi
%    \end{macrocode}

% Define the default values for the |\version| flag
% (|final| for the main file and |draft| for childs):
%    \begin{macrocode}
\ifchilddoc
\providecommand{\version}{draft}
\else
\providecommand{\version}{final}
\fi
%    \end{macrocode}

% Load the standard document class:
%    \begin{macrocode}
\documentclass[12pt]{article}
%    \end{macrocode}

% Start the document body:
%    \begin{macrocode}
\begin{document}
%    \end{macrocode}

% Declare a title page.
% Print title, part of document being processed and version flag:
%    \begin{macrocode}
\addtocounter{page}{-1}
\begin{center}
{\LARGE\bfseries{}childdoc example\par}
\vspace{1cm}
\ifchilddoc
\ifchilddocmanual part\else chapter\fi:
`\childdocname' of `\childdocjob'\par
\else
main document: `\childdocjob'\par
\fi
version: \version\par
\end{center}
\newpage
%    \end{macrocode}

% Manually include selected file,
% otherwise process as usual:
%    \begin{macrocode}
\ifchilddocmanual
\section*{part `\childdocname'}
\input{\childdocname}
\else
%    \end{macrocode}

% Include the two chapters:
%    \begin{macrocode}
\include{cdocsch1}
\include{cdocsch2}
%    \end{macrocode}

% Include the two parts unless only chapters should be displayed:
%    \begin{macrocode}
\ifchilddoc\else
\section{part three}
\input{cdocspt3}
\section{part four}
\input{cdocspt4}
\fi
%    \end{macrocode}

% Process as usual until here:
%    \begin{macrocode}
\fi
%    \end{macrocode}

% End of document body:
%    \begin{macrocode}
\end{document}
%    \end{macrocode}
%\iffalse
%</samplemain>
%\fi
%
% %%%%%%%%%%%%%%%%%%%%%%%%%%%%%%%%%%%%%%
% \paragraph{Chapter Include Files.}
%
% The include files are called |cdocsch1.tex| and |cdocsch2.tex|.
%
%\iffalse
%<*samplechap1|samplechap2>
%\fi

% Optional override for |\version| flag:
%    \begin{macrocode}
%%\providecommand{\version}{final}
%    \end{macrocode}

% Include the main document:
%    \begin{macrocode}
\input{childdoc.def}
\childdocof{cdocsamp}
%    \end{macrocode}

%\iffalse
%</samplechap1|samplechap2>
%\fi
%
%\iffalse
%<*samplechap1>
%\fi
% Some text for chapter 1:
%    \begin{macrocode}
\section{one}
some text in chapter one
%    \end{macrocode}

%\iffalse
%</samplechap1>
%\fi
% Some text for chapter 2:
%\iffalse
%<*samplechap2>
%\fi
%    \begin{macrocode}
\section{two}
more text in chapter two
%    \end{macrocode}

%\iffalse
%</samplechap2>
%\fi
%
% %%%%%%%%%%%%%%%%%%%%%%%%%%%%%%%%%%%%%%
% \paragraph{Part Include Files.}
%
% The include files are called |cdocspt3.tex| and |cdocspt4.tex|.
%
%\iffalse
%<*samplepart3|samplepart4>
%\fi

% Optional override for |\version| flag:
%    \begin{macrocode}
%%\providecommand{\version}{final}
%    \end{macrocode}

% Include the main document:
%    \begin{macrocode}
\input{childdoc.def}
\childdocby{cdocsamp}
%    \end{macrocode}

%\iffalse
%</samplepart3|samplepart4>
%\fi
%
%\iffalse
%<*samplepart3>
%\fi
% Some text for part 3:
%    \begin{macrocode}
some text in part three
%    \end{macrocode}

%\iffalse
%</samplepart3>
%\fi
% Some text for part 4:
%\iffalse
%<*samplepart4>
%\fi
%    \begin{macrocode}
more text in part four
%    \end{macrocode}

%\iffalse
%</samplepart4>
%\fi
%
% %%%%%%%%%%%%%%%%%%%%%%%%%%%%%%%%%%%%%%
% \paragraph{Forwarding for a Complete Draft.}
%
% The following forwarding file |cdocsdrf.tex|
% compiles the main document in draft mode:
%\iffalse
%<*sampledraft>
%\fi
%    \begin{macrocode}
\def\version{draft}
\input{childdoc.def}
\childdocforward{cdocsamp}
%    \end{macrocode}

%\iffalse
%</sampledraft>
%\fi
%
% %%%%%%%%%%%%%%%%%%%%%%%%%%%%%%%%%%%%%%
% \paragraph{Forwarding for Final Version of the Chapters.}
%
% The following forwarding files |cdocsfn1.tex| and |cdocsfn2.tex|
% (with identical content)
% compile the final versions of the child documents
% |cdocsch1.tex| and |cdocsch2.tex|, respectively:
%\iffalse
%<*samplefinal>
%\fi
%    \begin{macrocode}
\def\version{final}
\input{childdoc.def}
\childdocforwardprefix[cdocsamp]{cdocsfn}{cdocsch}
%    \end{macrocode}

%\iffalse
%</samplefinal>
%\fi
%
% %%%%%%%%%%%%%%%%%%%%%%%%%%%%%%%%%%%%%%
% \paragraph{Command Line Processing.}
%
% The following three command lines generate the output files
% |cdocscld|, |cdocscl1| and |cdocscl2|
% which should be identical to
% |cdocsdrf|, |cdocsch1| and |cdocsfn2|, respectively:
% \begin{center}
% \begin{tabular}{l}
% |latex -jobname cdocscld \|\\
% |  "\def\version{draft}\input{childdoc.def}\childdocforward{cdocsamp}"|\\
% |latex -jobname cdocscl1 \|\\
% |  "\input{childdoc.def}\childdocforward[cdocsamp]{cdocsch1}"|\\
% |latex -jobname cdocscl2 \|\\
% |  "\def\version{final}\input{childdoc.def}\childdocforward{cdocsch2}"|
% \end{tabular}
% \end{center}
% Note that the trailing backslash on each first line
% merely continues the input to the second line
% (for convenient cut ant paste).
% Furthermore, the command |latex| can be replaced by any
% of its alternative versions such as |pdflatex|.
%
% %%%%%%%%%%%%%%%%%%%%%%%%%%%%%%%%%%%%%%%%%%%%%%%%%%%%%%%%%%%%%%%%%%%%%%%%%%%%%%
% %%%%%%%%%%%%%%%%%%%%%%%%%%%%%%%%%%%%%%%%%%%%%%%%%%%%%%%%%%%%%%%%%%%%%%%%%%%%%%
% \section{Implementation}
%\iffalse
%<*package>
%\fi
%
% This section describes the definitions file |childdoc.def|.

% The definitions cannot be loaded using |\usepackage| or |\RequirePackage|
% which has a mechanism to prevent loading a style file more than once.
% When loading the definitions by means of |\input|
% multiple instances have to be prevented manually:
%\iffalse
%This code needs to be before the `\ProvidesFile' directive
%which is defined at the beginning of this file.
%Therefore it is also placed there and commented out here.
%</package>
%<*discard>
%\fi
%    \begin{macrocode}
\ifdefined\childdocmain\endinput\fi
%    \end{macrocode}
%\iffalse
%</discard>
%<*package>
%\fi
%
% \macro{\ifchilddoc}
% \macro{\ifchilddocmanual}
% The conditional |\ifchilddoc| tells whether a
% child (true) or main (false) document is being compiled.
% The conditional |\ifchilddocmanual| tells whether
% the |\includeonly| mechanism is used (false) or
% the selection of child files must be performed manually (true).
% The definitions initialise to false:
%    \begin{macrocode}
\newif\ifchilddoc
\newif\ifchilddocmanual
%    \end{macrocode}

% \macro{\childdocname}
% \macro{\childdocjob}
% The macro |\childdocname| stores the name of the main document
% to be compiled. The macro |\childdocjob| stores the name of
% the document on which the \LaTeX{} compiler was originally invoked.
% The content of |\jobname| cannot be compared
% to filenames specified in the source due to different catcodes.
% The following code rescans |\jobname|, stores the result
% in |\childdocname| and saves a copy in |\childdocjob|:
%    \begin{macrocode}
\edef\childdocname{\scantokens\expandafter{\jobname\noexpand}}
\let\childdocjob\childdocname
%    \end{macrocode}

% \macro{\childdocdisable}
% The macro |\childdocdisable| prevents the main file
% from being processed more than once.
% At this stage, the main document command |\childdocmain|
% is assumed to be called once again where it should do nothing.
% Any subsequent call to it should prevent
% a secondary processing of the main document
% It overwrites the forwarding commands
% |\childdocof| and |\childdocforward|
% with empty macros to prevent further inclusions of the main document:
%    \begin{macrocode}
\newcommand{\childdocdisable}
{
  \renewcommand{\childdocmain}[1]{\renewcommand{\childdocmain}[1]{\endinput}}
  \renewcommand{\childdocof}[1]{}
  \renewcommand{\childdocby}[2][]{}
  \renewcommand{\childdocforward}[2][]{}
  \renewcommand{\childdocdisable}{}
}
%    \end{macrocode}

% \macro{\childdocmain}
% The macro |\childdocmain| is to be called at the top of the main file
% with nothing or the main filename (without extension) as argument.
% First, it breaks loops.
% If the argument is not empty and does not match |\childdocname|
% (which is set by the first inclusion of |childdoc.def|),
% |\ifchilddoc| is set to true, |\includeonly| is applied to the child file
% and |\jobname| is set to the main file
% (for proper handling of |.aux| files):
%    \begin{macrocode}
\newcommand{\childdocmain}[1]
{
  \childdocdisable\childdocmain{}
  \if?#1?\else
    \begingroup
      \def\childdoctmp{#1}
      \ifx\childdoctmp\childdocname
        \def\childdoctmp{}
      \else
        \def\childdoctmp
        {
          \childdoctrue
          \includeonly{\childdocname}
          \def\childdocjob{#1}
          \def\jobname{#1}
        }
      \fi
      \expandafter
    \endgroup
    \childdoctmp
  \fi
}
%    \end{macrocode}

% \macro{\childdocof}
% The command |\childdocof| redirects
% compilation to the main file |#1|.
%    \begin{macrocode}
\newcommand{\childdocof}[1]
{
  \childdocdisable
  \childdoctrue
  \includeonly{\childdocname}
  \def\jobname{#1}
  \def\childdocjob{#1}
  \input{#1}
}
%    \end{macrocode}

% \macro{\childdocby}
% The command |\childdocby| ....
%    \begin{macrocode}
\newcommand{\childdocby}[2][]
{
  \childdocdisable
  \childdoctrue
  \childdocmanualtrue
  \if?#1?\else
    \def\jobname{#2}
  \fi
  \def\childdocjob{#2}
  \input{#2}
  \endinput
}
%    \end{macrocode}

% \macro{\childdocforward}
% The command |\childdocforward| redirects
% compilation to the main file or
% (if the optional argument is given) a child file.
% Parameters are set as if the main file
% or a child file starting with |\childdocof| was compiled.
% Then compilation is handed over to the main file:
%    \begin{macrocode}
\newcommand{\childdocforward}[2][]
{
  \begingroup
    \if?#1?
      \def\childdoctmp
      {
        \def\childdocname{#2}
        \def\childdocjob{#2}
        \def\jobname{#2}
        \input{#2}
        \endinput
      }
    \else
      \def\childdoctmp
      {
        \childdocdisable
        \def\childdocname{#2}
        \childdoctrue
        \includeonly{#2}
        \def\childdocjob{#1}
        \def\jobname{#1}
        \input{#1}
        \endinput
      }
    \fi
    \expandafter
  \endgroup
  \childdoctmp
}
%    \end{macrocode}

% \macro{\childdocforwardprefix}
% The command |\childdocforwardprefix| redirects
% compilation to the main or a child file by means of a pattern.
% The prefix |#1| in the current filename is replaced by |#2|
% and the suffix of the current filename is kept
% (it is assumed that the filename does not contain the substring `|~~~|'
% which is used as a delimiter).
% Compilation is handed over to the new file by |\childdocforward|:
%    \begin{macrocode}
\newcommand{\childdocforwardprefix}[3][]
{
  \begingroup
    \def\childdocextract #2##1~~~{\def\childdoctmp{\childdocforward[#1]{#3##1}}}
    \expandafter\childdocextract\childdocname~~~
    \expandafter
  \endgroup
  \childdoctmp
}
%    \end{macrocode}

% \macro{\childdoc}
% The deprecated macro |\childdoc| is a legacy version of |\childdocmain|:
%    \begin{macrocode}
\newcommand{\childdoc}{\childdocmain}
%    \end{macrocode}

% \macro{\childdocredirect}
% The deprecated macro |\childdocredirect| is a legacy version
% of |\childdocforward| and |\childdocforwardprefix|:
%    \begin{macrocode}
\newcommand{\childdocredirect}[2][]
{
  \begingroup
    \if?#1?
      \def\childdoctmp{\childdocforward{#2}}
    \else
      \def\childdoctmp{\childdocforwardprefix{#1}{#2}}
    \fi
    \expandafter
  \endgroup
  \childdoctmp
}
%    \end{macrocode}

%\iffalse
%</package>
%\fi
%
\endinput
\childdocforward{cdocsch2}"|
% \end{tabular}
% \end{center}
% Note that the trailing backslash on each first line
% merely continues the input to the second line
% (for convenient cut ant paste).
% Furthermore, the command |latex| can be replaced by any
% of its alternative versions such as |pdflatex|.
%
% %%%%%%%%%%%%%%%%%%%%%%%%%%%%%%%%%%%%%%%%%%%%%%%%%%%%%%%%%%%%%%%%%%%%%%%%%%%%%%
% %%%%%%%%%%%%%%%%%%%%%%%%%%%%%%%%%%%%%%%%%%%%%%%%%%%%%%%%%%%%%%%%%%%%%%%%%%%%%%
% \section{Implementation}
%\iffalse
%<*package>
%\fi
%
% This section describes the definitions file |childdoc.def|.

% The definitions cannot be loaded using |\usepackage| or |\RequirePackage|
% which has a mechanism to prevent loading a style file more than once.
% When loading the definitions by means of |\input|
% multiple instances have to be prevented manually:
%\iffalse
%This code needs to be before the `\ProvidesFile' directive
%which is defined at the beginning of this file.
%Therefore it is also placed there and commented out here.
%</package>
%<*discard>
%\fi
%    \begin{macrocode}
\ifdefined\childdocmain\endinput\fi
%    \end{macrocode}
%\iffalse
%</discard>
%<*package>
%\fi
%
% \macro{\ifchilddoc}
% \macro{\ifchilddocmanual}
% The conditional |\ifchilddoc| tells whether a
% child (true) or main (false) document is being compiled.
% The conditional |\ifchilddocmanual| tells whether
% the |\includeonly| mechanism is used (false) or
% the selection of child files must be performed manually (true).
% The definitions initialise to false:
%    \begin{macrocode}
\newif\ifchilddoc
\newif\ifchilddocmanual
%    \end{macrocode}

% \macro{\childdocname}
% \macro{\childdocjob}
% The macro |\childdocname| stores the name of the main document
% to be compiled. The macro |\childdocjob| stores the name of
% the document on which the \LaTeX{} compiler was originally invoked.
% The content of |\jobname| cannot be compared
% to filenames specified in the source due to different catcodes.
% The following code rescans |\jobname|, stores the result
% in |\childdocname| and saves a copy in |\childdocjob|:
%    \begin{macrocode}
\edef\childdocname{\scantokens\expandafter{\jobname\noexpand}}
\let\childdocjob\childdocname
%    \end{macrocode}

% \macro{\childdocdisable}
% The macro |\childdocdisable| prevents the main file
% from being processed more than once.
% At this stage, the main document command |\childdocmain|
% is assumed to be called once again where it should do nothing.
% Any subsequent call to it should prevent
% a secondary processing of the main document
% It overwrites the forwarding commands
% |\childdocof| and |\childdocforward|
% with empty macros to prevent further inclusions of the main document:
%    \begin{macrocode}
\newcommand{\childdocdisable}
{
  \renewcommand{\childdocmain}[1]{\renewcommand{\childdocmain}[1]{\endinput}}
  \renewcommand{\childdocof}[1]{}
  \renewcommand{\childdocby}[2][]{}
  \renewcommand{\childdocforward}[2][]{}
  \renewcommand{\childdocdisable}{}
}
%    \end{macrocode}

% \macro{\childdocmain}
% The macro |\childdocmain| is to be called at the top of the main file
% with nothing or the main filename (without extension) as argument.
% First, it breaks loops.
% If the argument is not empty and does not match |\childdocname|
% (which is set by the first inclusion of |childdoc.def|),
% |\ifchilddoc| is set to true, |\includeonly| is applied to the child file
% and |\jobname| is set to the main file
% (for proper handling of |.aux| files):
%    \begin{macrocode}
\newcommand{\childdocmain}[1]
{
  \childdocdisable\childdocmain{}
  \if?#1?\else
    \begingroup
      \def\childdoctmp{#1}
      \ifx\childdoctmp\childdocname
        \def\childdoctmp{}
      \else
        \def\childdoctmp
        {
          \childdoctrue
          \includeonly{\childdocname}
          \def\childdocjob{#1}
          \def\jobname{#1}
        }
      \fi
      \expandafter
    \endgroup
    \childdoctmp
  \fi
}
%    \end{macrocode}

% \macro{\childdocof}
% The command |\childdocof| redirects
% compilation to the main file |#1|.
%    \begin{macrocode}
\newcommand{\childdocof}[1]
{
  \childdocdisable
  \childdoctrue
  \includeonly{\childdocname}
  \def\jobname{#1}
  \def\childdocjob{#1}
  \input{#1}
}
%    \end{macrocode}

% \macro{\childdocby}
% The command |\childdocby| ....
%    \begin{macrocode}
\newcommand{\childdocby}[2][]
{
  \childdocdisable
  \childdoctrue
  \childdocmanualtrue
  \if?#1?\else
    \def\jobname{#2}
  \fi
  \def\childdocjob{#2}
  \input{#2}
  \endinput
}
%    \end{macrocode}

% \macro{\childdocforward}
% The command |\childdocforward| redirects
% compilation to the main file or
% (if the optional argument is given) a child file.
% Parameters are set as if the main file
% or a child file starting with |\childdocof| was compiled.
% Then compilation is handed over to the main file:
%    \begin{macrocode}
\newcommand{\childdocforward}[2][]
{
  \begingroup
    \if?#1?
      \def\childdoctmp
      {
        \def\childdocname{#2}
        \def\childdocjob{#2}
        \def\jobname{#2}
        \input{#2}
        \endinput
      }
    \else
      \def\childdoctmp
      {
        \childdocdisable
        \def\childdocname{#2}
        \childdoctrue
        \includeonly{#2}
        \def\childdocjob{#1}
        \def\jobname{#1}
        \input{#1}
        \endinput
      }
    \fi
    \expandafter
  \endgroup
  \childdoctmp
}
%    \end{macrocode}

% \macro{\childdocforwardprefix}
% The command |\childdocforwardprefix| redirects
% compilation to the main or a child file by means of a pattern.
% The prefix |#1| in the current filename is replaced by |#2|
% and the suffix of the current filename is kept
% (it is assumed that the filename does not contain the substring `|~~~|'
% which is used as a delimiter).
% Compilation is handed over to the new file by |\childdocforward|:
%    \begin{macrocode}
\newcommand{\childdocforwardprefix}[3][]
{
  \begingroup
    \def\childdocextract #2##1~~~{\def\childdoctmp{\childdocforward[#1]{#3##1}}}
    \expandafter\childdocextract\childdocname~~~
    \expandafter
  \endgroup
  \childdoctmp
}
%    \end{macrocode}

% \macro{\childdoc}
% The deprecated macro |\childdoc| is a legacy version of |\childdocmain|:
%    \begin{macrocode}
\newcommand{\childdoc}{\childdocmain}
%    \end{macrocode}

% \macro{\childdocredirect}
% The deprecated macro |\childdocredirect| is a legacy version
% of |\childdocforward| and |\childdocforwardprefix|:
%    \begin{macrocode}
\newcommand{\childdocredirect}[2][]
{
  \begingroup
    \if?#1?
      \def\childdoctmp{\childdocforward{#2}}
    \else
      \def\childdoctmp{\childdocforwardprefix{#1}{#2}}
    \fi
    \expandafter
  \endgroup
  \childdoctmp
}
%    \end{macrocode}

%\iffalse
%</package>
%\fi
%
\endinput
|\\
|\childdocmain{}|\\
\end{tabular}
\end{center}
at the very top of the main \LaTeX{} file,
in particular \emph{before} the |\documentclass| statement!
The argument of |\childdocmain| should be left empty
(but it must be present).

%%%%%%%%%%%%%%%%%%%%%%%%%%%%%%%%%%%%%%%%
\DescribeMacro{\childdocof}
Furthermore, add the commands
\begin{center}
\begin{tabular}{l}
|% \iffalse
%
% childdoc.dtx Copyright (C) 2017-2018 Niklas Beisert
%
% This work may be distributed and/or modified under the
% conditions of the LaTeX Project Public License, either version 1.3
% of this license or (at your option) any later version.
% The latest version of this license is in
%   http://www.latex-project.org/lppl.txt
% and version 1.3 or later is part of all distributions of LaTeX
% version 2005/12/01 or later.
%
% This work has the LPPL maintenance status `maintained'.
%
% The Current Maintainer of this work is Niklas Beisert.
%
% This work consists of the files childdoc.dtx and childdoc.ins
% and the derived files childdoc.def and cdocsamp.tex with
% cdocsch1.tex, cdocsch2.tex, cdocsdrf.tex, cdocsfn1.tex, cdocsfn2.tex.
%
%<package>\ifdefined\childdocmain\endinput\fi
%<package>\ProvidesFile{childdoc.def}[2018/12/30 v2.0 child document driver]
%<samplemain>\ProvidesFile{cdocsamp.tex}[2018/12/30 v2.0 sample for childdoc]
%<*driver>
%\ProvidesFile{childdoc.drv}[2018/12/30 v2.0 childdoc reference manual file]
\PassOptionsToClass{10pt,a4paper}{article}
\documentclass{ltxdoc}

\usepackage[margin=35mm]{geometry}
\usepackage{hyperref}
\usepackage{hyperxmp}
\usepackage[usenames]{color}

\hypersetup{colorlinks=true}
\hypersetup{pdfstartview=FitH}
\hypersetup{pdfpagemode=UseNone}
\hypersetup{pdfsource={}}
\hypersetup{pdflang={en-UK}}
\hypersetup{pdfcopyright={Copyright 2017-2018 Niklas Beisert.
  This work may be distributed and/or modified under the
  conditions of the LaTeX Project Public License, either version 1.3
  of this license or (at your option) any later version.}}
\hypersetup{pdflicenseurl={http://www.latex-project.org/lppl.txt}}
\hypersetup{pdfcontactaddress={ETH Zurich, ITP, HIT K,
  Wolfgang-Pauli-Strasse 27}}
\hypersetup{pdfcontactpostcode={8093}}
\hypersetup{pdfcontactcity={Zurich}}
\hypersetup{pdfcontactcountry={Switzerland}}
\hypersetup{pdfcontactemail={nbeisert@itp.phys.ethz.ch}}
\hypersetup{pdfcontacturl={http://people.phys.ethz.ch/\xmptilde nbeisert/}}

\newcommand{\secref}[1]{\hyperref[#1]{section \ref*{#1}}}

\parskip1ex
\parindent0pt
\let\olditemize\itemize
\def\itemize{\olditemize\parskip0pt}

\begin{document}

\title{The \textsf{childdoc} Package}
\hypersetup{pdftitle={The childdoc Package}}
\author{Niklas Beisert\\[2ex]
  Institut f\"ur Theoretische Physik\\
  Eidgen\"ossische Technische Hochschule Z\"urich\\
  Wolfgang-Pauli-Strasse 27, 8093 Z\"urich, Switzerland\\[1ex]
  \href{mailto:nbeisert@itp.phys.ethz.ch}
  {\texttt{nbeisert@itp.phys.ethz.ch}}}
\hypersetup{pdfauthor={Niklas Beisert}}
\hypersetup{pdfsubject={Manual for the LaTeX2e Package childdoc}}
\date{30 December 2018, \textsf{v2.0}}
\maketitle

\begin{abstract}\noindent
\textsf{childdoc} is a \LaTeXe{} package
that enables the direct compilation
of document sections included by |\include|
to individual files.
\end{abstract}

\begingroup
\parskip0ex
\tableofcontents
\endgroup

%%%%%%%%%%%%%%%%%%%%%%%%%%%%%%%%%%%%%%%%%%%%%%%%%%%%%%%%%%%%%%%%%%%%%%%%%%%%%%%%
%%%%%%%%%%%%%%%%%%%%%%%%%%%%%%%%%%%%%%%%%%%%%%%%%%%%%%%%%%%%%%%%%%%%%%%%%%%%%%%%
\section{Introduction}

\LaTeX{} provides a mechanism to structure a large document (such as a book)
into a main file and several child files (containing the chapters)
using the |\include| command.
This mechanism is beneficial for documents
which span hundreds of pages in order to
make the source file(s) more manageable.
Moreover, compilation can be restricted to
selected child files by means of the |\includeonly| command.
The latter feature can be used to reduce the compilation time while editing
(this was significantly more useful in the earlier days of \LaTeX{})
or to generate a smaller document which is easier to navigate.
Another application of |\includeonly| is to generate
documents consisting of selected parts of the complete document.

However, there are a few drawbacks of the plain |\include| mechanism:
\begin{itemize}
\item
The child files cannot be compiled on their own,
they can only be compiled via the main file.
A naive editing environment
(such as a text editor with an option
to have the current file processed by \LaTeX)
may require one to switch to the main file before compiling;
attempting to compile the child file produces errors.
\item
The main file must be modified (each time)
to adjust the |\includeonly| command
to the present needs. This easily leaves the main file in a messy state.
\item
The generated document will always carry the filename
of the main document. This is inconvenient if
several child files are to be compiled and
to be kept for distribution.
\end{itemize}

The present package provides a simple interface
to make child files individually compilable by \LaTeX{}.
Compiling a child file then has the same effect as compiling
the main file with an |\includeonly| command
to select the appropriate child.
Moreover the generated document will carry the name of the child
rather than the main file.
This resolves all three above issues.

This feature is meant to make the editing of books,
thesis documents and lecture notes somewhat more convenient.
However, the package can also be used efficiently for
composing a series of documents (such as exercise sheets)
which are typically distributed individually.
It then assists the author in generating the individual documents
(potentially in different versions)
as well as a document containing the collected series.
Another application is in developing style files
or other kinds of included material
where compilation of the style file could redirect
to a sample or test file.

%%%%%%%%%%%%%%%%%%%%%%%%%%%%%%%%%%%%%%%%%%%%%%%%%%%%%%%%%%%%%%%%%%%%%%%%%%%%%%%%
%%%%%%%%%%%%%%%%%%%%%%%%%%%%%%%%%%%%%%%%%%%%%%%%%%%%%%%%%%%%%%%%%%%%%%%%%%%%%%%%
\section{Usage}

First of all, the package \textsf{childdoc} is \emph{not} a standard
\LaTeXe{} |.sty| style file! Therefore it needs to be invoked in
a non-standard way.

%%%%%%%%%%%%%%%%%%%%%%%%%%%%%%%%%%%%%%%%%%%%%%%%%%%%%%%%%%%%%%%%%%%%%%%%%%%%%%%%
\subsection{Included Files}
\label{sec:include}

%%%%%%%%%%%%%%%%%%%%%%%%%%%%%%%%%%%%%%%%
\DescribeMacro{\childdocmain}
To use the package, add the commands
\begin{center}
\begin{tabular}{l}
|% \iffalse
%
% childdoc.dtx Copyright (C) 2017-2018 Niklas Beisert
%
% This work may be distributed and/or modified under the
% conditions of the LaTeX Project Public License, either version 1.3
% of this license or (at your option) any later version.
% The latest version of this license is in
%   http://www.latex-project.org/lppl.txt
% and version 1.3 or later is part of all distributions of LaTeX
% version 2005/12/01 or later.
%
% This work has the LPPL maintenance status `maintained'.
%
% The Current Maintainer of this work is Niklas Beisert.
%
% This work consists of the files childdoc.dtx and childdoc.ins
% and the derived files childdoc.def and cdocsamp.tex with
% cdocsch1.tex, cdocsch2.tex, cdocsdrf.tex, cdocsfn1.tex, cdocsfn2.tex.
%
%<package>\ifdefined\childdocmain\endinput\fi
%<package>\ProvidesFile{childdoc.def}[2018/12/30 v2.0 child document driver]
%<samplemain>\ProvidesFile{cdocsamp.tex}[2018/12/30 v2.0 sample for childdoc]
%<*driver>
%\ProvidesFile{childdoc.drv}[2018/12/30 v2.0 childdoc reference manual file]
\PassOptionsToClass{10pt,a4paper}{article}
\documentclass{ltxdoc}

\usepackage[margin=35mm]{geometry}
\usepackage{hyperref}
\usepackage{hyperxmp}
\usepackage[usenames]{color}

\hypersetup{colorlinks=true}
\hypersetup{pdfstartview=FitH}
\hypersetup{pdfpagemode=UseNone}
\hypersetup{pdfsource={}}
\hypersetup{pdflang={en-UK}}
\hypersetup{pdfcopyright={Copyright 2017-2018 Niklas Beisert.
  This work may be distributed and/or modified under the
  conditions of the LaTeX Project Public License, either version 1.3
  of this license or (at your option) any later version.}}
\hypersetup{pdflicenseurl={http://www.latex-project.org/lppl.txt}}
\hypersetup{pdfcontactaddress={ETH Zurich, ITP, HIT K,
  Wolfgang-Pauli-Strasse 27}}
\hypersetup{pdfcontactpostcode={8093}}
\hypersetup{pdfcontactcity={Zurich}}
\hypersetup{pdfcontactcountry={Switzerland}}
\hypersetup{pdfcontactemail={nbeisert@itp.phys.ethz.ch}}
\hypersetup{pdfcontacturl={http://people.phys.ethz.ch/\xmptilde nbeisert/}}

\newcommand{\secref}[1]{\hyperref[#1]{section \ref*{#1}}}

\parskip1ex
\parindent0pt
\let\olditemize\itemize
\def\itemize{\olditemize\parskip0pt}

\begin{document}

\title{The \textsf{childdoc} Package}
\hypersetup{pdftitle={The childdoc Package}}
\author{Niklas Beisert\\[2ex]
  Institut f\"ur Theoretische Physik\\
  Eidgen\"ossische Technische Hochschule Z\"urich\\
  Wolfgang-Pauli-Strasse 27, 8093 Z\"urich, Switzerland\\[1ex]
  \href{mailto:nbeisert@itp.phys.ethz.ch}
  {\texttt{nbeisert@itp.phys.ethz.ch}}}
\hypersetup{pdfauthor={Niklas Beisert}}
\hypersetup{pdfsubject={Manual for the LaTeX2e Package childdoc}}
\date{30 December 2018, \textsf{v2.0}}
\maketitle

\begin{abstract}\noindent
\textsf{childdoc} is a \LaTeXe{} package
that enables the direct compilation
of document sections included by |\include|
to individual files.
\end{abstract}

\begingroup
\parskip0ex
\tableofcontents
\endgroup

%%%%%%%%%%%%%%%%%%%%%%%%%%%%%%%%%%%%%%%%%%%%%%%%%%%%%%%%%%%%%%%%%%%%%%%%%%%%%%%%
%%%%%%%%%%%%%%%%%%%%%%%%%%%%%%%%%%%%%%%%%%%%%%%%%%%%%%%%%%%%%%%%%%%%%%%%%%%%%%%%
\section{Introduction}

\LaTeX{} provides a mechanism to structure a large document (such as a book)
into a main file and several child files (containing the chapters)
using the |\include| command.
This mechanism is beneficial for documents
which span hundreds of pages in order to
make the source file(s) more manageable.
Moreover, compilation can be restricted to
selected child files by means of the |\includeonly| command.
The latter feature can be used to reduce the compilation time while editing
(this was significantly more useful in the earlier days of \LaTeX{})
or to generate a smaller document which is easier to navigate.
Another application of |\includeonly| is to generate
documents consisting of selected parts of the complete document.

However, there are a few drawbacks of the plain |\include| mechanism:
\begin{itemize}
\item
The child files cannot be compiled on their own,
they can only be compiled via the main file.
A naive editing environment
(such as a text editor with an option
to have the current file processed by \LaTeX)
may require one to switch to the main file before compiling;
attempting to compile the child file produces errors.
\item
The main file must be modified (each time)
to adjust the |\includeonly| command
to the present needs. This easily leaves the main file in a messy state.
\item
The generated document will always carry the filename
of the main document. This is inconvenient if
several child files are to be compiled and
to be kept for distribution.
\end{itemize}

The present package provides a simple interface
to make child files individually compilable by \LaTeX{}.
Compiling a child file then has the same effect as compiling
the main file with an |\includeonly| command
to select the appropriate child.
Moreover the generated document will carry the name of the child
rather than the main file.
This resolves all three above issues.

This feature is meant to make the editing of books,
thesis documents and lecture notes somewhat more convenient.
However, the package can also be used efficiently for
composing a series of documents (such as exercise sheets)
which are typically distributed individually.
It then assists the author in generating the individual documents
(potentially in different versions)
as well as a document containing the collected series.
Another application is in developing style files
or other kinds of included material
where compilation of the style file could redirect
to a sample or test file.

%%%%%%%%%%%%%%%%%%%%%%%%%%%%%%%%%%%%%%%%%%%%%%%%%%%%%%%%%%%%%%%%%%%%%%%%%%%%%%%%
%%%%%%%%%%%%%%%%%%%%%%%%%%%%%%%%%%%%%%%%%%%%%%%%%%%%%%%%%%%%%%%%%%%%%%%%%%%%%%%%
\section{Usage}

First of all, the package \textsf{childdoc} is \emph{not} a standard
\LaTeXe{} |.sty| style file! Therefore it needs to be invoked in
a non-standard way.

%%%%%%%%%%%%%%%%%%%%%%%%%%%%%%%%%%%%%%%%%%%%%%%%%%%%%%%%%%%%%%%%%%%%%%%%%%%%%%%%
\subsection{Included Files}
\label{sec:include}

%%%%%%%%%%%%%%%%%%%%%%%%%%%%%%%%%%%%%%%%
\DescribeMacro{\childdocmain}
To use the package, add the commands
\begin{center}
\begin{tabular}{l}
|\input{childdoc.def}|\\
|\childdocmain{}|\\
\end{tabular}
\end{center}
at the very top of the main \LaTeX{} file,
in particular \emph{before} the |\documentclass| statement!
The argument of |\childdocmain| should be left empty
(but it must be present).

%%%%%%%%%%%%%%%%%%%%%%%%%%%%%%%%%%%%%%%%
\DescribeMacro{\childdocof}
Furthermore, add the commands
\begin{center}
\begin{tabular}{l}
|\input{childdoc.def}|\\
|\childdocof{|\textit{main}|}|\\
\end{tabular}
\end{center}
at the top of every child file \textit{child}
which is included by |\include{|\textit{child}|}|
from within the main file
(or at least for those files to be compiled individually).
The argument \textit{main} must be the filename of the main file.

There are a couple of
considerations in setting up the main and child documents:

%%%%%%%%%%%%%%%%%%%%%%%%%%%%%%%%%%%%%%%%
\paragraph{Restrictions.}

Please note the following restrictions:
\begin{itemize}
\item
|\childdocmain| must be called with one argument \textit{main}
to ensure compatibility with earlier version of the package.
It must either be empty (|\childdocmain{}|)
or precisely match the filename of the main file in which it is specified.
See \secref{sec:detection} for further information.
\item
The filename \textit{main} must be specified without the |.tex| extension.
\item
The filename \textit{main} is case sensitive
(even in case-insensitive file systems)
due to internal string comparison.
\item
The argument \textit{main} should be fully expanded, it cannot be a macro.
\item
Subdirectories and special characters should be avoided in filenames.
\item
The command |\childdocmain{|\textit{main}|}| must be followed by a whitespace.
It should not be followed immediately by another command
or by a comment mark `|%|'.
This is because the \TeX{} parser reads the token immediately following
the argument of |\childdocmain| and puts it
at the beginning of every child section;
however, a white\-space is ignored.
\end{itemize}

%%%%%%%%%%%%%%%%%%%%%%%%%%%%%%%%%%%%%%%%
\paragraph{Content of Main File.}

It is advisable to place all content in the child files included by |\include|.
Any output contained in the main file will appear in all child documents
unless suppressed manually;
it cannot be suppressed automatically by the |\includeonly| directive
and thus should normally be avoided.
A method to include some content in the main file
by means of conditional processing is described in \secref{sec:conditional}.

%%%%%%%%%%%%%%%%%%%%%%%%%%%%%%%%%%%%%%%%
\paragraph{Page Numbering.}

When only a part of the document is compiled,
the appropriate numbering of pages
(as well as other status parameters)
is determined from the |.aux| files.
The latter contain information from previous passes.
However this information needs to propagate through
all intermediate child documents.
Therefore the page numbering in child documents may well
be inconsistent until the complete document is compiled at least once.

A useful (if unconventional) way to always ensure a consistent
page numbering is to restart the numbering in each child document
and denote the pages by `\textit{child}|.|\textit{page}'
where \textit{child} represents the chapter/section number of the child file.
This can be achieved by the command
|\numberwithin{page}{|\textit{child}|}|
of the \textsf{amsmath} package
where \textit{child} can be |chapter| or |section|
depending on the chosen structuring.
Alternatively, one can modify the macro |\thepage| appropriately
and reset the counter |page| at the start of each child file.

%%%%%%%%%%%%%%%%%%%%%%%%%%%%%%%%%%%%%%%%%%%%%%%%%%%%%%%%%%%%%%%%%%%%%%%%%%%%%%%%
\subsection{Conditional Processing}
\label{sec:conditional}

The package provides a mechanism to compile different versions
of a document. To customise the versions further some conditional processing
can come in handy to distinguish which version is being compiled.
The package provides two macros to describe the compilation context:

%%%%%%%%%%%%%%%%%%%%%%%%%%%%%%%%%%%%%%%%
\DescribeMacro{\ifchilddoc}
The conditional |\ifchilddoc| distinguishes between the compilation of
child documents and the main document:
%
\begin{center}
|\ifchilddoc |\textit{child-code}| |[|\||else |\textit{main-code}]| \||fi|
\end{center}

%%%%%%%%%%%%%%%%%%%%%%%%%%%%%%%%%%%%%%%%
\DescribeMacro{\childdocname}
\DescribeMacro{\childdocjob}
The macro |\childdocname| contains the filename (without extension)
of the main or child file being processed.
Note that |\childdocjob| will always contain the name of the main file.

%%%%%%%%%%%%%%%%%%%%%%%%%%%%%%%%%%%%%%%%
\paragraph{Title Page.}

Conditional processing can be used to include a title or banner page
in the main document when proper precautions are taken.
Importantly, the code in the main file should ensure that the page counter
(as well as other status parameters which are stored in the |.aux| files)
takes the same value after the conditional processing.
Otherwise the page numbers may take divergent values
depending on which part is compiled.

For example, a title page could be declared by:
%
\begin{center}
\begin{tabular}{l}
|\ifchilddoc\||else|\\
|\addtocounter{page}{-1}|\\
\textit{code for title page}\\
|\newpage|\\
|\||fi|
\end{tabular}
\end{center}
%
A banner page for the child documents can be generated by:
%
\begin{center}
\begin{tabular}{l}
|\ifchilddoc|\\
|\addtocounter{page}{-1}|\\
\textit{code for banner page}\\
|\newpage|\\
|\||fi|
\end{tabular}
\end{center}
%
Here one could write a message such as:
\begin{center}
|This is the part \childdocname{} of \childdocjob{}.|
\end{center}

%%%%%%%%%%%%%%%%%%%%%%%%%%%%%%%%%%%%%%%%%%%%%%%%%%%%%%%%%%%%%%%%%%%%%%%%%%%%%%%%
\subsection{Flags}
\label{sec:flags}

The package makes it easy to generate different versions
of the main or child documents.
To this end compilation flags can be defined
and assigned different default values.
They will be particularly useful in conjunction
with the forwarding mechanism described in \secref{sec:forward}.

For example, it may be useful to have a flag |\version|
which can be set to |draft| or |final|.
The document source will contain some conditional code
depending on the value of |\version|.
Suppose further, the flag should default to |final| for the main file
and to |draft| for child files
which is a natural assignment for editing the document.
This is achieved by placing the following code
in the preamble of the main document
(below the |\childdocmain| directive):
%
\begin{center}
\begin{tabular}{l}
|\ifchilddoc|\\
|\providecommand{\version}{draft}|\\
|\||else|\\
|\providecommand{\version}{final}|\\
|\||fi|
\end{tabular}
\end{center}
%
The definition by |\providecommand| makes sure
that previous definitions are not overwritten.
Further statements |\providecommand{\version}{...}|
can thus be added before the above code to override it.

For the main file, one might add a line
(between |\childdocmain| and the above block)
%
\begin{center}
|%\ifchilddoc\||else\providecommand{\version}{draft}\||fi|
\end{center}
%
which can be uncommented to produce a draft version.
Likewise one can add a line to the very top of a child file
(above the |\childdocof{|\textit{main}|}| directive)
%
\begin{center}
|%\providecommand{\version}{final}|
\end{center}
%
which can be uncommented to produce the final version of this child document.

%%%%%%%%%%%%%%%%%%%%%%%%%%%%%%%%%%%%%%%%%%%%%%%%%%%%%%%%%%%%%%%%%%%%%%%%%%%%%%%%
\subsection{Forwarding}
\label{sec:forward}

Different versions of the main or child documents
using compilation flags as described in \secref{sec:flags}
can be (permanently) stored in different files
for convenient compilation, viewing and distribution.
To this end, the package defines a command
to pass on compilation to a different file:

%%%%%%%%%%%%%%%%%%%%%%%%%%%%%%%%%%%%%%%%
\DescribeMacro{\childdocforward}
The command |\childdocforward| redirects processing to
another source file:
%
\begin{center}
\begin{tabular}{l}
|\input{childdoc.def}|\\
|\childdocforward[|\textit{main}|]{|\textit{dest}|}|\\
\end{tabular}
\end{center}
%
The argument \textit{dest} is the destination file
(without extension).
It should be the main file or one of the child files.
Note that further \textsf{childdoc} directives
such as |\childdocof| and |\childdocforward|
in the indicated file will be processed in this form.
The optional argument \textit{main}
passes on directly to the main file \textit{main}
while pretending to compile the child \textit{dest}.
This form behaves as if \textit{dest}
issues |\childdocof{|\textit{main}|}| right away,
and no further \textsf{childdoc} directives will be processed.

%%%%%%%%%%%%%%%%%%%%%%%%%%%%%%%%%%%%%%%%
\DescribeMacro{\...prefix}
In the alternative form |\childdocforwardprefix|,
%
\begin{center}
\begin{tabular}{l}
|\input{childdoc.def}|\\
|\childdocforwardprefix[|\textit{main}|]{|\textit{prefix}|}{|\textit{dest}|}|
\end{tabular}
\end{center}
%
the destination file is determined by a pattern
depending on the current file:
To make this work, the current file must be called
`{\textit{prefix}\hspace{0.2em}\textit{suffix}}'
with \textit{prefix} matching precisely the argument.
Processing is then passed on to the file
`{\textit{dest}\hspace{0.2em}\textit{suffix}}'.
Surely, the same effect is achieved by
directly specifying the
argument `{\textit{dest}\hspace{0.2em}\textit{suffix}}'
in the first form.
However, that requires to set up a different file
for each child. With the alternative form of the command
all these files can have exactly the same content
which simplifies setting them up and maintaining them.

For example, the following file |draft.tex|
with a compilation flag |\version| as described in \secref{sec:flags}
compiles the main document as a draft:
%
\begin{center}
\begin{tabular}{l}
|\def\version{draft}|\\
|\input{childdoc.def}|\\
|\childdocforward{|\textit{main}|}|
\end{tabular}
\end{center}
%
Likewise, the following files |final|\textit{nn}|.tex|
compile the final version of the child document
|child|\textit{nn}|.tex|:
%
\begin{center}
\begin{tabular}{l}
|\def\version{final}|\\
|\input{childdoc.def}|\\
|\childdocforwardprefix{final}{child}|
\end{tabular}
\end{center}
%

Note that when several versions of a main file and/or of each child file
are to be generated, it may be convenient to set up a |Makefile| or
shell script to automatise the process.

%%%%%%%%%%%%%%%%%%%%%%%%%%%%%%%%%%%%%%%%%%%%%%%%%%%%%%%%%%%%%%%%%%%%%%%%%%%%%%%%
\subsection{Command Line Processing}
\label{sec:commandline}

The effect of redirection files can also be achieved by invoking
the \LaTeX{} compiler with a more elaborate command line.
Most conveniently this should be done as part
of a shell script or a |Makefile|.

When using \textsf{childdoc} in the main file, the following
command lines effectively perform a redirection
(note that depending on the shell being used,
backslashes may have to be doubled: `|\|' $\to$ `|\\|'):
%
\begin{center}
|... -jobname "|\textit{target}|" |\\|"|[\textit{flags}]%
|\input{childdoc.def}\childdocforward[|\textit{main}|]{|\textit{dest}|}"|
\end{center}
%
Here \textit{target} is the name of the output file,
\textit{main} is the name of the main file
and \textit{dest} is the name of the main or child file to be processed
(all filenames without extensions).
The optional argument \textit{main} can be omitted
if \textit{main} matches \textit{dest}.
Optionally, compilation \textit{flags} can be defined via |\def| commands.
This command line makes the \TeX{} engine believe
it is compiling the file \textit{target}
whose content is specified as the latter parameter.
The provided code then forwards the processing to
\textit{main} or \textit{dest} as described in \secref{sec:forward}.

%%%%%%%%%%%%%%%%%%%%%%%%%%%%%%%%%%%%%%%%%%%%%%%%%%%%%%%%%%%%%%%%%%%%%%%%%%%%%%%%
\subsection{Include by Input}
\label{sec:input}

Including child documents by |\include| has some restrictions by design.
Most notably, the content of a child document always occupies
its own set of pages; pages cannot be shared between child documents.
Usually, this behaviour makes perfect sense
because each child document contain an essential part of the document.
However, in some situations it may be desirable to compose
a document from a collection of parts
without having mandatory page breaks between then.
For this case, the package
provides a mechanism to include parts
by |\input| which can also be processed individually.
However, by construction this mechanism
requires manual handling of the content to be output.

%%%%%%%%%%%%%%%%%%%%%%%%%%%%%%%%%%%%%%%%
\DescribeMacro{\ifchilddocmanual}
The main file should be prepared as usual, see \secref{sec:include}.
However, the document body must make a distinction
between processing of an individual part and of the main document, e.g.:
%
\begin{center}
\begin{tabular}{l}
|\ifchilddocmanual|\\
|\input{\childdocname}|\\
|\||else|\\
\textit{document body with }|\input{|\textit{part}|}|\\
|\||fi|
\end{tabular}
\end{center}
%
The conditional |\ifchilddocmanual| is true whenever
a part to be included by |\input| is being compiled,
and the name of the part is stored in |\childdocname|.

%%%%%%%%%%%%%%%%%%%%%%%%%%%%%%%%%%%%%%%%
\DescribeMacro{\childdocby}
Each part to be included by |\input| should start with:
%
\begin{center}
\begin{tabular}{l}
|\input{childdoc.def}|\\
|\childdocby{|\textit{main}|}|\\
\end{tabular}
\end{center}
%
The directive |\childdocby| is similar to |\childdocof|
described in \secref{sec:include},
but the subsequent selection of content must be done manually.
To that end, both |\ifchilddoc| and |\ifchilddocmanual|
will be true upon processing of a part,
and the name of the part is stored in |\childdocname|.
Note that |\jobname| will be set to the filename of the current part
so that each part receives an individual |.aux| file
that does not interfere with the |.aux| file(s) of the main document.
This behaviour can be altered by the alternative form
|\childdocby[*]{|\textit{main}|}| (with a non-empty optional argument)
which uses the |.aux| file of the main document
by setting |\jobname| to \textit{main}.

%%%%%%%%%%%%%%%%%%%%%%%%%%%%%%%%%%%%%%%%%%%%%%%%%%%%%%%%%%%%%%%%%%%%%%%%%%%%%%%%
\subsection{Driver Development}
\label{sec:driver}

The \textsf{childdoc} mechanism can also be use for the development
of definition files such as \LaTeX{} styles or classes.
This case differs from the above setup with multiple parts
included by |\include| in that no |\includeonly| should be invoked.
This can be achieved by starting the include file
(before |\ProvidesPackage|) with:
%
\begin{center}
\begin{tabular}{l}
|\input{childdoc.def}|\\
|\childdocforward{|\textit{main}|}|\\
\end{tabular}
\end{center}
%
or alternatively with:
%
\begin{center}
\begin{tabular}{l}
|\input{childdoc.def}|\\
|\childdocby{|\textit{main}|}|\\
\end{tabular}
\end{center}
%
Both forms have slightly different effects as described above.
The main file is prepared as usual, see \secref{sec:include}.

%%%%%%%%%%%%%%%%%%%%%%%%%%%%%%%%%%%%%%%%%%%%%%%%%%%%%%%%%%%%%%%%%%%%%%%%%%%%%%%%
\subsection{Legacy Detection}
\label{sec:detection}

The directive |\childdocmain| in the main file can detect
whether the complete document or merely a child is to be compiled
even without using the directive |\childdocof|.
This method is deprecated because it is less robust
and there is no compelling reason to use it;
it is merely provided for backward compatibility
and it may be removed in future versions.

If the detection mechanism is to be used,
it is mandatory to correctly specify
the filename of the main file as the argument of |\childdocmain|:
%
\begin{center}
\begin{tabular}{l}
|\input{childdoc.def}|\\
|\childdocmain{|\textit{main}|}|\\
\end{tabular}
\end{center}
%
If |\jobname| does not match the argument \textit{main} of |\childdocmain|,
it is assumed that |\jobname| points to the child file to be compiled.
When using |\childdocmain| with the main file specified as argument,
it suffices to start a child file
with just |\input{|\textit{main}|}|
without loading of the package and using |\childdocof|.
If instead all processing is done
with the appropriate \textsf{childdoc} directives,
the argument of \textit{main} of |\childdocmain| can be empty.

An alternative version of the command line processing described
in \secref{sec:commandline} using the detection mechanism reads:
%
\begin{center}
|... -jobname "|\textit{target}|" "|[\textit{flags}]%
[|\def\jobname{|\textit{dest}|}|]|\input{|\textit{main}|}"|
\end{center}

%%%%%%%%%%%%%%%%%%%%%%%%%%%%%%%%%%%%%%%%%%%%%%%%%%%%%%%%%%%%%%%%%%%%%%%%%%%%%%%%
\subsection{Manual Code}
\label{sec:manual}

In case one cannot be certain whether the definitions file |childdoc.def|
is installed on the target \TeX{} distribution
and one prefers not to ship it,
it is conceivable to paste a few relevant commands into the sources.

To that end, drop all statements |\input{childdoc.def}|
and perform the replacements as outlined below.
Instead of |\childdocmain{|\textit{main}|}| add the following code
to the top of the main file:
%
\begin{center}
\begin{tabular}{l}
|\||ifdefined\childdocname\endinput\||fi\newif\ifchilddoc|\\
|\edef\childdocname{\scantokens\expandafter{\jobname\noexpand}}|\\
|\def\childdocmain{|\textit{main}|}\||ifx\childdocmain\childdocname\||else|\\
|\childdoctrue\includeonly{\childdocname}\let\jobname\childdocmain\||fi|\\
\end{tabular}
\end{center}
%
Instead of |\childdocof{|\textit{main}|}| just include the main file
at the top of each child file:
%
\begin{center}
|\input{|\textit{main}|}|
\end{center}
%
A simple redirection |\childdocforward{|\textit{dest}|}| is achieved by:
%
\begin{center}
|\def\jobname{|\textit{dest}|}\input{\jobname}|
\end{center}
%
The redirection with prefix
|\childdocforwardprefix[|\textit{prefix}|]{|\textit{dest}|}|
is accomplished by:
%
\begin{center}
\begin{tabular}{l}
|{\edef\jobname{\scantokens\expandafter{\jobname\noexpand}}|\\
|\def\redirectjob |\textit{prefix}|#1~~~{\gdef\jobname{|\textit{dest}|#1}}|\\
|\expandafter\redirectjob\jobname~~~}\input{\jobname}|
\end{tabular}
\end{center}

In an alternative approach,
child documents can be compiled by a specific command line
without additional code or specific definitions:
%
\begin{center}
|... -jobname "|\textit{target}|" "|[\textit{flags}]%
|\includeonly{|\textit{dest}|}\input{|\textit{main}|}"|
\end{center}
%

%%%%%%%%%%%%%%%%%%%%%%%%%%%%%%%%%%%%%%%%%%%%%%%%%%%%%%%%%%%%%%%%%%%%%%%%%%%%%%%%
%%%%%%%%%%%%%%%%%%%%%%%%%%%%%%%%%%%%%%%%%%%%%%%%%%%%%%%%%%%%%%%%%%%%%%%%%%%%%%%%
\section{Information}

%%%%%%%%%%%%%%%%%%%%%%%%%%%%%%%%%%%%%%%%%%%%%%%%%%%%%%%%%%%%%%%%%%%%%%%%%%%%%%%%
\subsection{Copyright}

Copyright \copyright{} 2017--2018 Niklas Beisert

This work may be distributed and/or modified under the
conditions of the \LaTeX{} Project Public License, either version 1.3
of this license or (at your option) any later version.
The latest version of this license is in
  \url{http://www.latex-project.org/lppl.txt}
and version 1.3 or later is part of all distributions of \LaTeX{}
version 2005/12/01 or later.

This work has the LPPL maintenance status `maintained'.

The Current Maintainer of this work is Niklas Beisert.

This work consists of the files |README.txt|, |childdoc.ins| and |childdoc.dtx|
as well as the derived files |childdoc.def|, |cdocsamp.tex|
with |cdocsch1.tex|, |cdocsch2.tex|, |cdocspt3.tex|, |cdocspt4.tex|,
|cdocsdrf.tex|, |cdocsfn1.tex|, |cdocsfn2.tex|
as well as |childdoc.pdf|.

%%%%%%%%%%%%%%%%%%%%%%%%%%%%%%%%%%%%%%%%%%%%%%%%%%%%%%%%%%%%%%%%%%%%%%%%%%%%%%%%
\subsection{Files and Installation}

The package consists of the files:
%
\begin{center}
\begin{tabular}{ll}
    |README.txt|   & readme file \\
    |childdoc.ins| & installation file \\
    |childdoc.dtx| & source file \\
    |childdoc.def| & definition file \\
    |cdocsamp.tex| & sample main file \\
    |cdocsch1.tex| & sample include file \\
    |cdocsch2.tex| & sample include file \\
    |cdocspt3.tex| & sample part file \\
    |cdocspt4.tex| & sample part file \\
    |cdocsdrf.tex| & sample redirection file \\
    |cdocsfn1.tex| & sample redirection file \\
    |cdocsfn2.tex| & sample redirection file \\
    |childdoc.pdf| & manual
\end{tabular}
\end{center}
%
The distribution consists of the files
|README.txt|, |childdoc.ins| and |childdoc.dtx|.
%
\begin{itemize}
\item
Run (pdf)\LaTeX{} on |childdoc.dtx|
to compile the manual |childdoc.pdf| (this file).
\item
Run \LaTeX{} on |childdoc.ins| to create the definitions file |childdoc.def|
and the sample |cdocsamp.tex| with include files
|cdocsch1.tex|, |cdocsch2.tex|, |cdocspt3.tex|, |cdocspt4.tex|,
|cdocsdrf.tex|, |cdocsfn1.tex|, |cdocsfn2.tex|.
Then copy the file |childdoc.def| to an appropriate directory of your \LaTeX{}
distribution, e.g.\ \textit{texmf-root}|/tex/latex/childdoc|.
\end{itemize}

%%%%%%%%%%%%%%%%%%%%%%%%%%%%%%%%%%%%%%%%%%%%%%%%%%%%%%%%%%%%%%%%%%%%%%%%%%%%%%%%
\subsection{Related CTAN Packages}

There are several other packages which offer a similar functionality:
%
\begin{itemize}
\item
The packages
\href{http://ctan.org/pkg/docmute}{\textsf{docmute}},
\href{http://ctan.org/pkg/includex}{\textsf{includex}} and
\href{http://ctan.org/pkg/standalone}{\textsf{standalone}}
provide commands to include only the document body of
a child file thus allowing both files to be compiled individually.
\item
The packages \href{http://ctan.org/pkg/subdocs}{\textsf{subdocs}}
and \href{http://ctan.org/pkg/subfiles}{\textsf{subfiles}}
provide structures in which the main and child documents can be
encapsulated and allowing them to be compiled individually.
The inclusion mechanism is different from the conventional |\include|.
\item
The package \href{http://ctan.org/pkg/combine}{\textsf{combine}}
is an elaborate solution to combine several documents into one.
\end{itemize}
%
See also the CTAN topic \href{http://ctan.org/topic/subdocs}{\textsf{subdocs}}
for further related packages.
The present package differs from the above solutions in that
a document structure constructed with the conventional |\include| mechanism
just needs two extra commands at the top of every file
such that all constituent files can be compiled individually.

%%%%%%%%%%%%%%%%%%%%%%%%%%%%%%%%%%%%%%%%%%%%%%%%%%%%%%%%%%%%%%%%%%%%%%%%%%%%%%%%
%\subsection{Feature Suggestions}
%
%The following is a list of features which may be useful for future
%versions of this package:
%%
%\begin{itemize}
%\item
%\ldots
%\end{itemize}

%%%%%%%%%%%%%%%%%%%%%%%%%%%%%%%%%%%%%%%%%%%%%%%%%%%%%%%%%%%%%%%%%%%%%%%%%%%%%%%%
\subsection{Revision History}

%%%%%%%%%%%%%%%%%%%%%%%%%%%%%%%%%%%%%%%%
\paragraph{v2.0:} 2018/12/30

\begin{itemize}
\item
immediate forward processing
\item
added |\childdocby| mechanism
\item
manual restructured
\end{itemize}

%%%%%%%%%%%%%%%%%%%%%%%%%%%%%%%%%%%%%%%%
\paragraph{v1.6:} 2018/01/17

\begin{itemize}
\item
application for development of include files
\item
corrections to manual
\end{itemize}

%%%%%%%%%%%%%%%%%%%%%%%%%%%%%%%%%%%%%%%%
\paragraph{v1.5:} 2017/05/21

\begin{itemize}
\item
more complete structuring introduced
\item
|\childdocof| introduced
\item
|\childdoc| renamed to |\childdocmain|
\item
|\childredirect| renamed to |\childdocforward| and |\childdocforwardprefix|
and functionality expanded
\end{itemize}

%%%%%%%%%%%%%%%%%%%%%%%%%%%%%%%%%%%%%%%%
\paragraph{v1.0:} 2017/04/27

\begin{itemize}
\item
manual and install package
\item
first version published on CTAN
\end{itemize}

%%%%%%%%%%%%%%%%%%%%%%%%%%%%%%%%%%%%%%%%
\paragraph{v0.6:} 2017/04/26

\begin{itemize}
\item
redirection mechanism added
\end{itemize}

%%%%%%%%%%%%%%%%%%%%%%%%%%%%%%%%%%%%%%%%
\paragraph{v0.5:} 2017/04/26

\begin{itemize}
\item
functionality in definition file
\end{itemize}


%%%%%%%%%%%%%%%%%%%%%%%%%%%%%%%%%%%%%%%%%%%%%%%%%%%%%%%%%%%%%%%%%%%%%%%%%%%%%%%%
%%%%%%%%%%%%%%%%%%%%%%%%%%%%%%%%%%%%%%%%%%%%%%%%%%%%%%%%%%%%%%%%%%%%%%%%%%%%%%%%
%%%%%%%%%%%%%%%%%%%%%%%%%%%%%%%%%%%%%%%%%%%%%%%%%%%%%%%%%%%%%%%%%%%%%%%%%%%%%%%%
\appendix

\settowidth\MacroIndent{\rmfamily\scriptsize 000\ }

 \DocInput{childdoc.dtx}

\end{document}
%</driver>
% \fi
%
% %%%%%%%%%%%%%%%%%%%%%%%%%%%%%%%%%%%%%%%%%%%%%%%%%%%%%%%%%%%%%%%%%%%%%%%%%%%%%%
% %%%%%%%%%%%%%%%%%%%%%%%%%%%%%%%%%%%%%%%%%%%%%%%%%%%%%%%%%%%%%%%%%%%%%%%%%%%%%%
% \section{Sample}
%\iffalse
%<*samplemain>
%\fi
%
% The following presents a sample document
% with two chapters, two parts, a title page,
% a compile flag as well as three forwarding files to set the flag.
% It consists of eight |.tex| files:
% \begin{center}
% \begin{tabular}{ll}
% |cdocsamp.tex|&main file\\
% |cdocsch1.tex|&include file for chapter 1\\
% |cdocsch2.tex|&include file for chapter 2\\
% |cdocspt3.tex|&include file for part 3\\
% |cdocspt4.tex|&include file for part 4\\
% |cdocsdrf.tex|&forwarding file for main file in draft mode\\
% |cdocsfi1.tex|&forwarding file for final version of chapter 1\\
% |cdocsfi2.tex|&forwarding file for final version of chapter 2\\
% \end{tabular}
% \end{center}
% Each of the eight files can be compiled directly by the \LaTeX{} compiler.
%
% %%%%%%%%%%%%%%%%%%%%%%%%%%%%%%%%%%%%%%
% \paragraph{Main File.}
%
% The main file is called |cdocsamp.tex|.
%
% Load the \textsf{childdoc} definitions and
% declare the filename for the main document:
%    \begin{macrocode}
\input{childdoc.def}
\childdocmain{}
%    \end{macrocode}

% Optional override for |\version| flag:
%    \begin{macrocode}
%%\ifchilddoc\else\providecommand{\version}{draft}\fi
%    \end{macrocode}

% Define the default values for the |\version| flag
% (|final| for the main file and |draft| for childs):
%    \begin{macrocode}
\ifchilddoc
\providecommand{\version}{draft}
\else
\providecommand{\version}{final}
\fi
%    \end{macrocode}

% Load the standard document class:
%    \begin{macrocode}
\documentclass[12pt]{article}
%    \end{macrocode}

% Start the document body:
%    \begin{macrocode}
\begin{document}
%    \end{macrocode}

% Declare a title page.
% Print title, part of document being processed and version flag:
%    \begin{macrocode}
\addtocounter{page}{-1}
\begin{center}
{\LARGE\bfseries{}childdoc example\par}
\vspace{1cm}
\ifchilddoc
\ifchilddocmanual part\else chapter\fi:
`\childdocname' of `\childdocjob'\par
\else
main document: `\childdocjob'\par
\fi
version: \version\par
\end{center}
\newpage
%    \end{macrocode}

% Manually include selected file,
% otherwise process as usual:
%    \begin{macrocode}
\ifchilddocmanual
\section*{part `\childdocname'}
\input{\childdocname}
\else
%    \end{macrocode}

% Include the two chapters:
%    \begin{macrocode}
\include{cdocsch1}
\include{cdocsch2}
%    \end{macrocode}

% Include the two parts unless only chapters should be displayed:
%    \begin{macrocode}
\ifchilddoc\else
\section{part three}
\input{cdocspt3}
\section{part four}
\input{cdocspt4}
\fi
%    \end{macrocode}

% Process as usual until here:
%    \begin{macrocode}
\fi
%    \end{macrocode}

% End of document body:
%    \begin{macrocode}
\end{document}
%    \end{macrocode}
%\iffalse
%</samplemain>
%\fi
%
% %%%%%%%%%%%%%%%%%%%%%%%%%%%%%%%%%%%%%%
% \paragraph{Chapter Include Files.}
%
% The include files are called |cdocsch1.tex| and |cdocsch2.tex|.
%
%\iffalse
%<*samplechap1|samplechap2>
%\fi

% Optional override for |\version| flag:
%    \begin{macrocode}
%%\providecommand{\version}{final}
%    \end{macrocode}

% Include the main document:
%    \begin{macrocode}
\input{childdoc.def}
\childdocof{cdocsamp}
%    \end{macrocode}

%\iffalse
%</samplechap1|samplechap2>
%\fi
%
%\iffalse
%<*samplechap1>
%\fi
% Some text for chapter 1:
%    \begin{macrocode}
\section{one}
some text in chapter one
%    \end{macrocode}

%\iffalse
%</samplechap1>
%\fi
% Some text for chapter 2:
%\iffalse
%<*samplechap2>
%\fi
%    \begin{macrocode}
\section{two}
more text in chapter two
%    \end{macrocode}

%\iffalse
%</samplechap2>
%\fi
%
% %%%%%%%%%%%%%%%%%%%%%%%%%%%%%%%%%%%%%%
% \paragraph{Part Include Files.}
%
% The include files are called |cdocspt3.tex| and |cdocspt4.tex|.
%
%\iffalse
%<*samplepart3|samplepart4>
%\fi

% Optional override for |\version| flag:
%    \begin{macrocode}
%%\providecommand{\version}{final}
%    \end{macrocode}

% Include the main document:
%    \begin{macrocode}
\input{childdoc.def}
\childdocby{cdocsamp}
%    \end{macrocode}

%\iffalse
%</samplepart3|samplepart4>
%\fi
%
%\iffalse
%<*samplepart3>
%\fi
% Some text for part 3:
%    \begin{macrocode}
some text in part three
%    \end{macrocode}

%\iffalse
%</samplepart3>
%\fi
% Some text for part 4:
%\iffalse
%<*samplepart4>
%\fi
%    \begin{macrocode}
more text in part four
%    \end{macrocode}

%\iffalse
%</samplepart4>
%\fi
%
% %%%%%%%%%%%%%%%%%%%%%%%%%%%%%%%%%%%%%%
% \paragraph{Forwarding for a Complete Draft.}
%
% The following forwarding file |cdocsdrf.tex|
% compiles the main document in draft mode:
%\iffalse
%<*sampledraft>
%\fi
%    \begin{macrocode}
\def\version{draft}
\input{childdoc.def}
\childdocforward{cdocsamp}
%    \end{macrocode}

%\iffalse
%</sampledraft>
%\fi
%
% %%%%%%%%%%%%%%%%%%%%%%%%%%%%%%%%%%%%%%
% \paragraph{Forwarding for Final Version of the Chapters.}
%
% The following forwarding files |cdocsfn1.tex| and |cdocsfn2.tex|
% (with identical content)
% compile the final versions of the child documents
% |cdocsch1.tex| and |cdocsch2.tex|, respectively:
%\iffalse
%<*samplefinal>
%\fi
%    \begin{macrocode}
\def\version{final}
\input{childdoc.def}
\childdocforwardprefix[cdocsamp]{cdocsfn}{cdocsch}
%    \end{macrocode}

%\iffalse
%</samplefinal>
%\fi
%
% %%%%%%%%%%%%%%%%%%%%%%%%%%%%%%%%%%%%%%
% \paragraph{Command Line Processing.}
%
% The following three command lines generate the output files
% |cdocscld|, |cdocscl1| and |cdocscl2|
% which should be identical to
% |cdocsdrf|, |cdocsch1| and |cdocsfn2|, respectively:
% \begin{center}
% \begin{tabular}{l}
% |latex -jobname cdocscld \|\\
% |  "\def\version{draft}\input{childdoc.def}\childdocforward{cdocsamp}"|\\
% |latex -jobname cdocscl1 \|\\
% |  "\input{childdoc.def}\childdocforward[cdocsamp]{cdocsch1}"|\\
% |latex -jobname cdocscl2 \|\\
% |  "\def\version{final}\input{childdoc.def}\childdocforward{cdocsch2}"|
% \end{tabular}
% \end{center}
% Note that the trailing backslash on each first line
% merely continues the input to the second line
% (for convenient cut ant paste).
% Furthermore, the command |latex| can be replaced by any
% of its alternative versions such as |pdflatex|.
%
% %%%%%%%%%%%%%%%%%%%%%%%%%%%%%%%%%%%%%%%%%%%%%%%%%%%%%%%%%%%%%%%%%%%%%%%%%%%%%%
% %%%%%%%%%%%%%%%%%%%%%%%%%%%%%%%%%%%%%%%%%%%%%%%%%%%%%%%%%%%%%%%%%%%%%%%%%%%%%%
% \section{Implementation}
%\iffalse
%<*package>
%\fi
%
% This section describes the definitions file |childdoc.def|.

% The definitions cannot be loaded using |\usepackage| or |\RequirePackage|
% which has a mechanism to prevent loading a style file more than once.
% When loading the definitions by means of |\input|
% multiple instances have to be prevented manually:
%\iffalse
%This code needs to be before the `\ProvidesFile' directive
%which is defined at the beginning of this file.
%Therefore it is also placed there and commented out here.
%</package>
%<*discard>
%\fi
%    \begin{macrocode}
\ifdefined\childdocmain\endinput\fi
%    \end{macrocode}
%\iffalse
%</discard>
%<*package>
%\fi
%
% \macro{\ifchilddoc}
% \macro{\ifchilddocmanual}
% The conditional |\ifchilddoc| tells whether a
% child (true) or main (false) document is being compiled.
% The conditional |\ifchilddocmanual| tells whether
% the |\includeonly| mechanism is used (false) or
% the selection of child files must be performed manually (true).
% The definitions initialise to false:
%    \begin{macrocode}
\newif\ifchilddoc
\newif\ifchilddocmanual
%    \end{macrocode}

% \macro{\childdocname}
% \macro{\childdocjob}
% The macro |\childdocname| stores the name of the main document
% to be compiled. The macro |\childdocjob| stores the name of
% the document on which the \LaTeX{} compiler was originally invoked.
% The content of |\jobname| cannot be compared
% to filenames specified in the source due to different catcodes.
% The following code rescans |\jobname|, stores the result
% in |\childdocname| and saves a copy in |\childdocjob|:
%    \begin{macrocode}
\edef\childdocname{\scantokens\expandafter{\jobname\noexpand}}
\let\childdocjob\childdocname
%    \end{macrocode}

% \macro{\childdocdisable}
% The macro |\childdocdisable| prevents the main file
% from being processed more than once.
% At this stage, the main document command |\childdocmain|
% is assumed to be called once again where it should do nothing.
% Any subsequent call to it should prevent
% a secondary processing of the main document
% It overwrites the forwarding commands
% |\childdocof| and |\childdocforward|
% with empty macros to prevent further inclusions of the main document:
%    \begin{macrocode}
\newcommand{\childdocdisable}
{
  \renewcommand{\childdocmain}[1]{\renewcommand{\childdocmain}[1]{\endinput}}
  \renewcommand{\childdocof}[1]{}
  \renewcommand{\childdocby}[2][]{}
  \renewcommand{\childdocforward}[2][]{}
  \renewcommand{\childdocdisable}{}
}
%    \end{macrocode}

% \macro{\childdocmain}
% The macro |\childdocmain| is to be called at the top of the main file
% with nothing or the main filename (without extension) as argument.
% First, it breaks loops.
% If the argument is not empty and does not match |\childdocname|
% (which is set by the first inclusion of |childdoc.def|),
% |\ifchilddoc| is set to true, |\includeonly| is applied to the child file
% and |\jobname| is set to the main file
% (for proper handling of |.aux| files):
%    \begin{macrocode}
\newcommand{\childdocmain}[1]
{
  \childdocdisable\childdocmain{}
  \if?#1?\else
    \begingroup
      \def\childdoctmp{#1}
      \ifx\childdoctmp\childdocname
        \def\childdoctmp{}
      \else
        \def\childdoctmp
        {
          \childdoctrue
          \includeonly{\childdocname}
          \def\childdocjob{#1}
          \def\jobname{#1}
        }
      \fi
      \expandafter
    \endgroup
    \childdoctmp
  \fi
}
%    \end{macrocode}

% \macro{\childdocof}
% The command |\childdocof| redirects
% compilation to the main file |#1|.
%    \begin{macrocode}
\newcommand{\childdocof}[1]
{
  \childdocdisable
  \childdoctrue
  \includeonly{\childdocname}
  \def\jobname{#1}
  \def\childdocjob{#1}
  \input{#1}
}
%    \end{macrocode}

% \macro{\childdocby}
% The command |\childdocby| ....
%    \begin{macrocode}
\newcommand{\childdocby}[2][]
{
  \childdocdisable
  \childdoctrue
  \childdocmanualtrue
  \if?#1?\else
    \def\jobname{#2}
  \fi
  \def\childdocjob{#2}
  \input{#2}
  \endinput
}
%    \end{macrocode}

% \macro{\childdocforward}
% The command |\childdocforward| redirects
% compilation to the main file or
% (if the optional argument is given) a child file.
% Parameters are set as if the main file
% or a child file starting with |\childdocof| was compiled.
% Then compilation is handed over to the main file:
%    \begin{macrocode}
\newcommand{\childdocforward}[2][]
{
  \begingroup
    \if?#1?
      \def\childdoctmp
      {
        \def\childdocname{#2}
        \def\childdocjob{#2}
        \def\jobname{#2}
        \input{#2}
        \endinput
      }
    \else
      \def\childdoctmp
      {
        \childdocdisable
        \def\childdocname{#2}
        \childdoctrue
        \includeonly{#2}
        \def\childdocjob{#1}
        \def\jobname{#1}
        \input{#1}
        \endinput
      }
    \fi
    \expandafter
  \endgroup
  \childdoctmp
}
%    \end{macrocode}

% \macro{\childdocforwardprefix}
% The command |\childdocforwardprefix| redirects
% compilation to the main or a child file by means of a pattern.
% The prefix |#1| in the current filename is replaced by |#2|
% and the suffix of the current filename is kept
% (it is assumed that the filename does not contain the substring `|~~~|'
% which is used as a delimiter).
% Compilation is handed over to the new file by |\childdocforward|:
%    \begin{macrocode}
\newcommand{\childdocforwardprefix}[3][]
{
  \begingroup
    \def\childdocextract #2##1~~~{\def\childdoctmp{\childdocforward[#1]{#3##1}}}
    \expandafter\childdocextract\childdocname~~~
    \expandafter
  \endgroup
  \childdoctmp
}
%    \end{macrocode}

% \macro{\childdoc}
% The deprecated macro |\childdoc| is a legacy version of |\childdocmain|:
%    \begin{macrocode}
\newcommand{\childdoc}{\childdocmain}
%    \end{macrocode}

% \macro{\childdocredirect}
% The deprecated macro |\childdocredirect| is a legacy version
% of |\childdocforward| and |\childdocforwardprefix|:
%    \begin{macrocode}
\newcommand{\childdocredirect}[2][]
{
  \begingroup
    \if?#1?
      \def\childdoctmp{\childdocforward{#2}}
    \else
      \def\childdoctmp{\childdocforwardprefix{#1}{#2}}
    \fi
    \expandafter
  \endgroup
  \childdoctmp
}
%    \end{macrocode}

%\iffalse
%</package>
%\fi
%
\endinput
|\\
|\childdocmain{}|\\
\end{tabular}
\end{center}
at the very top of the main \LaTeX{} file,
in particular \emph{before} the |\documentclass| statement!
The argument of |\childdocmain| should be left empty
(but it must be present).

%%%%%%%%%%%%%%%%%%%%%%%%%%%%%%%%%%%%%%%%
\DescribeMacro{\childdocof}
Furthermore, add the commands
\begin{center}
\begin{tabular}{l}
|% \iffalse
%
% childdoc.dtx Copyright (C) 2017-2018 Niklas Beisert
%
% This work may be distributed and/or modified under the
% conditions of the LaTeX Project Public License, either version 1.3
% of this license or (at your option) any later version.
% The latest version of this license is in
%   http://www.latex-project.org/lppl.txt
% and version 1.3 or later is part of all distributions of LaTeX
% version 2005/12/01 or later.
%
% This work has the LPPL maintenance status `maintained'.
%
% The Current Maintainer of this work is Niklas Beisert.
%
% This work consists of the files childdoc.dtx and childdoc.ins
% and the derived files childdoc.def and cdocsamp.tex with
% cdocsch1.tex, cdocsch2.tex, cdocsdrf.tex, cdocsfn1.tex, cdocsfn2.tex.
%
%<package>\ifdefined\childdocmain\endinput\fi
%<package>\ProvidesFile{childdoc.def}[2018/12/30 v2.0 child document driver]
%<samplemain>\ProvidesFile{cdocsamp.tex}[2018/12/30 v2.0 sample for childdoc]
%<*driver>
%\ProvidesFile{childdoc.drv}[2018/12/30 v2.0 childdoc reference manual file]
\PassOptionsToClass{10pt,a4paper}{article}
\documentclass{ltxdoc}

\usepackage[margin=35mm]{geometry}
\usepackage{hyperref}
\usepackage{hyperxmp}
\usepackage[usenames]{color}

\hypersetup{colorlinks=true}
\hypersetup{pdfstartview=FitH}
\hypersetup{pdfpagemode=UseNone}
\hypersetup{pdfsource={}}
\hypersetup{pdflang={en-UK}}
\hypersetup{pdfcopyright={Copyright 2017-2018 Niklas Beisert.
  This work may be distributed and/or modified under the
  conditions of the LaTeX Project Public License, either version 1.3
  of this license or (at your option) any later version.}}
\hypersetup{pdflicenseurl={http://www.latex-project.org/lppl.txt}}
\hypersetup{pdfcontactaddress={ETH Zurich, ITP, HIT K,
  Wolfgang-Pauli-Strasse 27}}
\hypersetup{pdfcontactpostcode={8093}}
\hypersetup{pdfcontactcity={Zurich}}
\hypersetup{pdfcontactcountry={Switzerland}}
\hypersetup{pdfcontactemail={nbeisert@itp.phys.ethz.ch}}
\hypersetup{pdfcontacturl={http://people.phys.ethz.ch/\xmptilde nbeisert/}}

\newcommand{\secref}[1]{\hyperref[#1]{section \ref*{#1}}}

\parskip1ex
\parindent0pt
\let\olditemize\itemize
\def\itemize{\olditemize\parskip0pt}

\begin{document}

\title{The \textsf{childdoc} Package}
\hypersetup{pdftitle={The childdoc Package}}
\author{Niklas Beisert\\[2ex]
  Institut f\"ur Theoretische Physik\\
  Eidgen\"ossische Technische Hochschule Z\"urich\\
  Wolfgang-Pauli-Strasse 27, 8093 Z\"urich, Switzerland\\[1ex]
  \href{mailto:nbeisert@itp.phys.ethz.ch}
  {\texttt{nbeisert@itp.phys.ethz.ch}}}
\hypersetup{pdfauthor={Niklas Beisert}}
\hypersetup{pdfsubject={Manual for the LaTeX2e Package childdoc}}
\date{30 December 2018, \textsf{v2.0}}
\maketitle

\begin{abstract}\noindent
\textsf{childdoc} is a \LaTeXe{} package
that enables the direct compilation
of document sections included by |\include|
to individual files.
\end{abstract}

\begingroup
\parskip0ex
\tableofcontents
\endgroup

%%%%%%%%%%%%%%%%%%%%%%%%%%%%%%%%%%%%%%%%%%%%%%%%%%%%%%%%%%%%%%%%%%%%%%%%%%%%%%%%
%%%%%%%%%%%%%%%%%%%%%%%%%%%%%%%%%%%%%%%%%%%%%%%%%%%%%%%%%%%%%%%%%%%%%%%%%%%%%%%%
\section{Introduction}

\LaTeX{} provides a mechanism to structure a large document (such as a book)
into a main file and several child files (containing the chapters)
using the |\include| command.
This mechanism is beneficial for documents
which span hundreds of pages in order to
make the source file(s) more manageable.
Moreover, compilation can be restricted to
selected child files by means of the |\includeonly| command.
The latter feature can be used to reduce the compilation time while editing
(this was significantly more useful in the earlier days of \LaTeX{})
or to generate a smaller document which is easier to navigate.
Another application of |\includeonly| is to generate
documents consisting of selected parts of the complete document.

However, there are a few drawbacks of the plain |\include| mechanism:
\begin{itemize}
\item
The child files cannot be compiled on their own,
they can only be compiled via the main file.
A naive editing environment
(such as a text editor with an option
to have the current file processed by \LaTeX)
may require one to switch to the main file before compiling;
attempting to compile the child file produces errors.
\item
The main file must be modified (each time)
to adjust the |\includeonly| command
to the present needs. This easily leaves the main file in a messy state.
\item
The generated document will always carry the filename
of the main document. This is inconvenient if
several child files are to be compiled and
to be kept for distribution.
\end{itemize}

The present package provides a simple interface
to make child files individually compilable by \LaTeX{}.
Compiling a child file then has the same effect as compiling
the main file with an |\includeonly| command
to select the appropriate child.
Moreover the generated document will carry the name of the child
rather than the main file.
This resolves all three above issues.

This feature is meant to make the editing of books,
thesis documents and lecture notes somewhat more convenient.
However, the package can also be used efficiently for
composing a series of documents (such as exercise sheets)
which are typically distributed individually.
It then assists the author in generating the individual documents
(potentially in different versions)
as well as a document containing the collected series.
Another application is in developing style files
or other kinds of included material
where compilation of the style file could redirect
to a sample or test file.

%%%%%%%%%%%%%%%%%%%%%%%%%%%%%%%%%%%%%%%%%%%%%%%%%%%%%%%%%%%%%%%%%%%%%%%%%%%%%%%%
%%%%%%%%%%%%%%%%%%%%%%%%%%%%%%%%%%%%%%%%%%%%%%%%%%%%%%%%%%%%%%%%%%%%%%%%%%%%%%%%
\section{Usage}

First of all, the package \textsf{childdoc} is \emph{not} a standard
\LaTeXe{} |.sty| style file! Therefore it needs to be invoked in
a non-standard way.

%%%%%%%%%%%%%%%%%%%%%%%%%%%%%%%%%%%%%%%%%%%%%%%%%%%%%%%%%%%%%%%%%%%%%%%%%%%%%%%%
\subsection{Included Files}
\label{sec:include}

%%%%%%%%%%%%%%%%%%%%%%%%%%%%%%%%%%%%%%%%
\DescribeMacro{\childdocmain}
To use the package, add the commands
\begin{center}
\begin{tabular}{l}
|\input{childdoc.def}|\\
|\childdocmain{}|\\
\end{tabular}
\end{center}
at the very top of the main \LaTeX{} file,
in particular \emph{before} the |\documentclass| statement!
The argument of |\childdocmain| should be left empty
(but it must be present).

%%%%%%%%%%%%%%%%%%%%%%%%%%%%%%%%%%%%%%%%
\DescribeMacro{\childdocof}
Furthermore, add the commands
\begin{center}
\begin{tabular}{l}
|\input{childdoc.def}|\\
|\childdocof{|\textit{main}|}|\\
\end{tabular}
\end{center}
at the top of every child file \textit{child}
which is included by |\include{|\textit{child}|}|
from within the main file
(or at least for those files to be compiled individually).
The argument \textit{main} must be the filename of the main file.

There are a couple of
considerations in setting up the main and child documents:

%%%%%%%%%%%%%%%%%%%%%%%%%%%%%%%%%%%%%%%%
\paragraph{Restrictions.}

Please note the following restrictions:
\begin{itemize}
\item
|\childdocmain| must be called with one argument \textit{main}
to ensure compatibility with earlier version of the package.
It must either be empty (|\childdocmain{}|)
or precisely match the filename of the main file in which it is specified.
See \secref{sec:detection} for further information.
\item
The filename \textit{main} must be specified without the |.tex| extension.
\item
The filename \textit{main} is case sensitive
(even in case-insensitive file systems)
due to internal string comparison.
\item
The argument \textit{main} should be fully expanded, it cannot be a macro.
\item
Subdirectories and special characters should be avoided in filenames.
\item
The command |\childdocmain{|\textit{main}|}| must be followed by a whitespace.
It should not be followed immediately by another command
or by a comment mark `|%|'.
This is because the \TeX{} parser reads the token immediately following
the argument of |\childdocmain| and puts it
at the beginning of every child section;
however, a white\-space is ignored.
\end{itemize}

%%%%%%%%%%%%%%%%%%%%%%%%%%%%%%%%%%%%%%%%
\paragraph{Content of Main File.}

It is advisable to place all content in the child files included by |\include|.
Any output contained in the main file will appear in all child documents
unless suppressed manually;
it cannot be suppressed automatically by the |\includeonly| directive
and thus should normally be avoided.
A method to include some content in the main file
by means of conditional processing is described in \secref{sec:conditional}.

%%%%%%%%%%%%%%%%%%%%%%%%%%%%%%%%%%%%%%%%
\paragraph{Page Numbering.}

When only a part of the document is compiled,
the appropriate numbering of pages
(as well as other status parameters)
is determined from the |.aux| files.
The latter contain information from previous passes.
However this information needs to propagate through
all intermediate child documents.
Therefore the page numbering in child documents may well
be inconsistent until the complete document is compiled at least once.

A useful (if unconventional) way to always ensure a consistent
page numbering is to restart the numbering in each child document
and denote the pages by `\textit{child}|.|\textit{page}'
where \textit{child} represents the chapter/section number of the child file.
This can be achieved by the command
|\numberwithin{page}{|\textit{child}|}|
of the \textsf{amsmath} package
where \textit{child} can be |chapter| or |section|
depending on the chosen structuring.
Alternatively, one can modify the macro |\thepage| appropriately
and reset the counter |page| at the start of each child file.

%%%%%%%%%%%%%%%%%%%%%%%%%%%%%%%%%%%%%%%%%%%%%%%%%%%%%%%%%%%%%%%%%%%%%%%%%%%%%%%%
\subsection{Conditional Processing}
\label{sec:conditional}

The package provides a mechanism to compile different versions
of a document. To customise the versions further some conditional processing
can come in handy to distinguish which version is being compiled.
The package provides two macros to describe the compilation context:

%%%%%%%%%%%%%%%%%%%%%%%%%%%%%%%%%%%%%%%%
\DescribeMacro{\ifchilddoc}
The conditional |\ifchilddoc| distinguishes between the compilation of
child documents and the main document:
%
\begin{center}
|\ifchilddoc |\textit{child-code}| |[|\||else |\textit{main-code}]| \||fi|
\end{center}

%%%%%%%%%%%%%%%%%%%%%%%%%%%%%%%%%%%%%%%%
\DescribeMacro{\childdocname}
\DescribeMacro{\childdocjob}
The macro |\childdocname| contains the filename (without extension)
of the main or child file being processed.
Note that |\childdocjob| will always contain the name of the main file.

%%%%%%%%%%%%%%%%%%%%%%%%%%%%%%%%%%%%%%%%
\paragraph{Title Page.}

Conditional processing can be used to include a title or banner page
in the main document when proper precautions are taken.
Importantly, the code in the main file should ensure that the page counter
(as well as other status parameters which are stored in the |.aux| files)
takes the same value after the conditional processing.
Otherwise the page numbers may take divergent values
depending on which part is compiled.

For example, a title page could be declared by:
%
\begin{center}
\begin{tabular}{l}
|\ifchilddoc\||else|\\
|\addtocounter{page}{-1}|\\
\textit{code for title page}\\
|\newpage|\\
|\||fi|
\end{tabular}
\end{center}
%
A banner page for the child documents can be generated by:
%
\begin{center}
\begin{tabular}{l}
|\ifchilddoc|\\
|\addtocounter{page}{-1}|\\
\textit{code for banner page}\\
|\newpage|\\
|\||fi|
\end{tabular}
\end{center}
%
Here one could write a message such as:
\begin{center}
|This is the part \childdocname{} of \childdocjob{}.|
\end{center}

%%%%%%%%%%%%%%%%%%%%%%%%%%%%%%%%%%%%%%%%%%%%%%%%%%%%%%%%%%%%%%%%%%%%%%%%%%%%%%%%
\subsection{Flags}
\label{sec:flags}

The package makes it easy to generate different versions
of the main or child documents.
To this end compilation flags can be defined
and assigned different default values.
They will be particularly useful in conjunction
with the forwarding mechanism described in \secref{sec:forward}.

For example, it may be useful to have a flag |\version|
which can be set to |draft| or |final|.
The document source will contain some conditional code
depending on the value of |\version|.
Suppose further, the flag should default to |final| for the main file
and to |draft| for child files
which is a natural assignment for editing the document.
This is achieved by placing the following code
in the preamble of the main document
(below the |\childdocmain| directive):
%
\begin{center}
\begin{tabular}{l}
|\ifchilddoc|\\
|\providecommand{\version}{draft}|\\
|\||else|\\
|\providecommand{\version}{final}|\\
|\||fi|
\end{tabular}
\end{center}
%
The definition by |\providecommand| makes sure
that previous definitions are not overwritten.
Further statements |\providecommand{\version}{...}|
can thus be added before the above code to override it.

For the main file, one might add a line
(between |\childdocmain| and the above block)
%
\begin{center}
|%\ifchilddoc\||else\providecommand{\version}{draft}\||fi|
\end{center}
%
which can be uncommented to produce a draft version.
Likewise one can add a line to the very top of a child file
(above the |\childdocof{|\textit{main}|}| directive)
%
\begin{center}
|%\providecommand{\version}{final}|
\end{center}
%
which can be uncommented to produce the final version of this child document.

%%%%%%%%%%%%%%%%%%%%%%%%%%%%%%%%%%%%%%%%%%%%%%%%%%%%%%%%%%%%%%%%%%%%%%%%%%%%%%%%
\subsection{Forwarding}
\label{sec:forward}

Different versions of the main or child documents
using compilation flags as described in \secref{sec:flags}
can be (permanently) stored in different files
for convenient compilation, viewing and distribution.
To this end, the package defines a command
to pass on compilation to a different file:

%%%%%%%%%%%%%%%%%%%%%%%%%%%%%%%%%%%%%%%%
\DescribeMacro{\childdocforward}
The command |\childdocforward| redirects processing to
another source file:
%
\begin{center}
\begin{tabular}{l}
|\input{childdoc.def}|\\
|\childdocforward[|\textit{main}|]{|\textit{dest}|}|\\
\end{tabular}
\end{center}
%
The argument \textit{dest} is the destination file
(without extension).
It should be the main file or one of the child files.
Note that further \textsf{childdoc} directives
such as |\childdocof| and |\childdocforward|
in the indicated file will be processed in this form.
The optional argument \textit{main}
passes on directly to the main file \textit{main}
while pretending to compile the child \textit{dest}.
This form behaves as if \textit{dest}
issues |\childdocof{|\textit{main}|}| right away,
and no further \textsf{childdoc} directives will be processed.

%%%%%%%%%%%%%%%%%%%%%%%%%%%%%%%%%%%%%%%%
\DescribeMacro{\...prefix}
In the alternative form |\childdocforwardprefix|,
%
\begin{center}
\begin{tabular}{l}
|\input{childdoc.def}|\\
|\childdocforwardprefix[|\textit{main}|]{|\textit{prefix}|}{|\textit{dest}|}|
\end{tabular}
\end{center}
%
the destination file is determined by a pattern
depending on the current file:
To make this work, the current file must be called
`{\textit{prefix}\hspace{0.2em}\textit{suffix}}'
with \textit{prefix} matching precisely the argument.
Processing is then passed on to the file
`{\textit{dest}\hspace{0.2em}\textit{suffix}}'.
Surely, the same effect is achieved by
directly specifying the
argument `{\textit{dest}\hspace{0.2em}\textit{suffix}}'
in the first form.
However, that requires to set up a different file
for each child. With the alternative form of the command
all these files can have exactly the same content
which simplifies setting them up and maintaining them.

For example, the following file |draft.tex|
with a compilation flag |\version| as described in \secref{sec:flags}
compiles the main document as a draft:
%
\begin{center}
\begin{tabular}{l}
|\def\version{draft}|\\
|\input{childdoc.def}|\\
|\childdocforward{|\textit{main}|}|
\end{tabular}
\end{center}
%
Likewise, the following files |final|\textit{nn}|.tex|
compile the final version of the child document
|child|\textit{nn}|.tex|:
%
\begin{center}
\begin{tabular}{l}
|\def\version{final}|\\
|\input{childdoc.def}|\\
|\childdocforwardprefix{final}{child}|
\end{tabular}
\end{center}
%

Note that when several versions of a main file and/or of each child file
are to be generated, it may be convenient to set up a |Makefile| or
shell script to automatise the process.

%%%%%%%%%%%%%%%%%%%%%%%%%%%%%%%%%%%%%%%%%%%%%%%%%%%%%%%%%%%%%%%%%%%%%%%%%%%%%%%%
\subsection{Command Line Processing}
\label{sec:commandline}

The effect of redirection files can also be achieved by invoking
the \LaTeX{} compiler with a more elaborate command line.
Most conveniently this should be done as part
of a shell script or a |Makefile|.

When using \textsf{childdoc} in the main file, the following
command lines effectively perform a redirection
(note that depending on the shell being used,
backslashes may have to be doubled: `|\|' $\to$ `|\\|'):
%
\begin{center}
|... -jobname "|\textit{target}|" |\\|"|[\textit{flags}]%
|\input{childdoc.def}\childdocforward[|\textit{main}|]{|\textit{dest}|}"|
\end{center}
%
Here \textit{target} is the name of the output file,
\textit{main} is the name of the main file
and \textit{dest} is the name of the main or child file to be processed
(all filenames without extensions).
The optional argument \textit{main} can be omitted
if \textit{main} matches \textit{dest}.
Optionally, compilation \textit{flags} can be defined via |\def| commands.
This command line makes the \TeX{} engine believe
it is compiling the file \textit{target}
whose content is specified as the latter parameter.
The provided code then forwards the processing to
\textit{main} or \textit{dest} as described in \secref{sec:forward}.

%%%%%%%%%%%%%%%%%%%%%%%%%%%%%%%%%%%%%%%%%%%%%%%%%%%%%%%%%%%%%%%%%%%%%%%%%%%%%%%%
\subsection{Include by Input}
\label{sec:input}

Including child documents by |\include| has some restrictions by design.
Most notably, the content of a child document always occupies
its own set of pages; pages cannot be shared between child documents.
Usually, this behaviour makes perfect sense
because each child document contain an essential part of the document.
However, in some situations it may be desirable to compose
a document from a collection of parts
without having mandatory page breaks between then.
For this case, the package
provides a mechanism to include parts
by |\input| which can also be processed individually.
However, by construction this mechanism
requires manual handling of the content to be output.

%%%%%%%%%%%%%%%%%%%%%%%%%%%%%%%%%%%%%%%%
\DescribeMacro{\ifchilddocmanual}
The main file should be prepared as usual, see \secref{sec:include}.
However, the document body must make a distinction
between processing of an individual part and of the main document, e.g.:
%
\begin{center}
\begin{tabular}{l}
|\ifchilddocmanual|\\
|\input{\childdocname}|\\
|\||else|\\
\textit{document body with }|\input{|\textit{part}|}|\\
|\||fi|
\end{tabular}
\end{center}
%
The conditional |\ifchilddocmanual| is true whenever
a part to be included by |\input| is being compiled,
and the name of the part is stored in |\childdocname|.

%%%%%%%%%%%%%%%%%%%%%%%%%%%%%%%%%%%%%%%%
\DescribeMacro{\childdocby}
Each part to be included by |\input| should start with:
%
\begin{center}
\begin{tabular}{l}
|\input{childdoc.def}|\\
|\childdocby{|\textit{main}|}|\\
\end{tabular}
\end{center}
%
The directive |\childdocby| is similar to |\childdocof|
described in \secref{sec:include},
but the subsequent selection of content must be done manually.
To that end, both |\ifchilddoc| and |\ifchilddocmanual|
will be true upon processing of a part,
and the name of the part is stored in |\childdocname|.
Note that |\jobname| will be set to the filename of the current part
so that each part receives an individual |.aux| file
that does not interfere with the |.aux| file(s) of the main document.
This behaviour can be altered by the alternative form
|\childdocby[*]{|\textit{main}|}| (with a non-empty optional argument)
which uses the |.aux| file of the main document
by setting |\jobname| to \textit{main}.

%%%%%%%%%%%%%%%%%%%%%%%%%%%%%%%%%%%%%%%%%%%%%%%%%%%%%%%%%%%%%%%%%%%%%%%%%%%%%%%%
\subsection{Driver Development}
\label{sec:driver}

The \textsf{childdoc} mechanism can also be use for the development
of definition files such as \LaTeX{} styles or classes.
This case differs from the above setup with multiple parts
included by |\include| in that no |\includeonly| should be invoked.
This can be achieved by starting the include file
(before |\ProvidesPackage|) with:
%
\begin{center}
\begin{tabular}{l}
|\input{childdoc.def}|\\
|\childdocforward{|\textit{main}|}|\\
\end{tabular}
\end{center}
%
or alternatively with:
%
\begin{center}
\begin{tabular}{l}
|\input{childdoc.def}|\\
|\childdocby{|\textit{main}|}|\\
\end{tabular}
\end{center}
%
Both forms have slightly different effects as described above.
The main file is prepared as usual, see \secref{sec:include}.

%%%%%%%%%%%%%%%%%%%%%%%%%%%%%%%%%%%%%%%%%%%%%%%%%%%%%%%%%%%%%%%%%%%%%%%%%%%%%%%%
\subsection{Legacy Detection}
\label{sec:detection}

The directive |\childdocmain| in the main file can detect
whether the complete document or merely a child is to be compiled
even without using the directive |\childdocof|.
This method is deprecated because it is less robust
and there is no compelling reason to use it;
it is merely provided for backward compatibility
and it may be removed in future versions.

If the detection mechanism is to be used,
it is mandatory to correctly specify
the filename of the main file as the argument of |\childdocmain|:
%
\begin{center}
\begin{tabular}{l}
|\input{childdoc.def}|\\
|\childdocmain{|\textit{main}|}|\\
\end{tabular}
\end{center}
%
If |\jobname| does not match the argument \textit{main} of |\childdocmain|,
it is assumed that |\jobname| points to the child file to be compiled.
When using |\childdocmain| with the main file specified as argument,
it suffices to start a child file
with just |\input{|\textit{main}|}|
without loading of the package and using |\childdocof|.
If instead all processing is done
with the appropriate \textsf{childdoc} directives,
the argument of \textit{main} of |\childdocmain| can be empty.

An alternative version of the command line processing described
in \secref{sec:commandline} using the detection mechanism reads:
%
\begin{center}
|... -jobname "|\textit{target}|" "|[\textit{flags}]%
[|\def\jobname{|\textit{dest}|}|]|\input{|\textit{main}|}"|
\end{center}

%%%%%%%%%%%%%%%%%%%%%%%%%%%%%%%%%%%%%%%%%%%%%%%%%%%%%%%%%%%%%%%%%%%%%%%%%%%%%%%%
\subsection{Manual Code}
\label{sec:manual}

In case one cannot be certain whether the definitions file |childdoc.def|
is installed on the target \TeX{} distribution
and one prefers not to ship it,
it is conceivable to paste a few relevant commands into the sources.

To that end, drop all statements |\input{childdoc.def}|
and perform the replacements as outlined below.
Instead of |\childdocmain{|\textit{main}|}| add the following code
to the top of the main file:
%
\begin{center}
\begin{tabular}{l}
|\||ifdefined\childdocname\endinput\||fi\newif\ifchilddoc|\\
|\edef\childdocname{\scantokens\expandafter{\jobname\noexpand}}|\\
|\def\childdocmain{|\textit{main}|}\||ifx\childdocmain\childdocname\||else|\\
|\childdoctrue\includeonly{\childdocname}\let\jobname\childdocmain\||fi|\\
\end{tabular}
\end{center}
%
Instead of |\childdocof{|\textit{main}|}| just include the main file
at the top of each child file:
%
\begin{center}
|\input{|\textit{main}|}|
\end{center}
%
A simple redirection |\childdocforward{|\textit{dest}|}| is achieved by:
%
\begin{center}
|\def\jobname{|\textit{dest}|}\input{\jobname}|
\end{center}
%
The redirection with prefix
|\childdocforwardprefix[|\textit{prefix}|]{|\textit{dest}|}|
is accomplished by:
%
\begin{center}
\begin{tabular}{l}
|{\edef\jobname{\scantokens\expandafter{\jobname\noexpand}}|\\
|\def\redirectjob |\textit{prefix}|#1~~~{\gdef\jobname{|\textit{dest}|#1}}|\\
|\expandafter\redirectjob\jobname~~~}\input{\jobname}|
\end{tabular}
\end{center}

In an alternative approach,
child documents can be compiled by a specific command line
without additional code or specific definitions:
%
\begin{center}
|... -jobname "|\textit{target}|" "|[\textit{flags}]%
|\includeonly{|\textit{dest}|}\input{|\textit{main}|}"|
\end{center}
%

%%%%%%%%%%%%%%%%%%%%%%%%%%%%%%%%%%%%%%%%%%%%%%%%%%%%%%%%%%%%%%%%%%%%%%%%%%%%%%%%
%%%%%%%%%%%%%%%%%%%%%%%%%%%%%%%%%%%%%%%%%%%%%%%%%%%%%%%%%%%%%%%%%%%%%%%%%%%%%%%%
\section{Information}

%%%%%%%%%%%%%%%%%%%%%%%%%%%%%%%%%%%%%%%%%%%%%%%%%%%%%%%%%%%%%%%%%%%%%%%%%%%%%%%%
\subsection{Copyright}

Copyright \copyright{} 2017--2018 Niklas Beisert

This work may be distributed and/or modified under the
conditions of the \LaTeX{} Project Public License, either version 1.3
of this license or (at your option) any later version.
The latest version of this license is in
  \url{http://www.latex-project.org/lppl.txt}
and version 1.3 or later is part of all distributions of \LaTeX{}
version 2005/12/01 or later.

This work has the LPPL maintenance status `maintained'.

The Current Maintainer of this work is Niklas Beisert.

This work consists of the files |README.txt|, |childdoc.ins| and |childdoc.dtx|
as well as the derived files |childdoc.def|, |cdocsamp.tex|
with |cdocsch1.tex|, |cdocsch2.tex|, |cdocspt3.tex|, |cdocspt4.tex|,
|cdocsdrf.tex|, |cdocsfn1.tex|, |cdocsfn2.tex|
as well as |childdoc.pdf|.

%%%%%%%%%%%%%%%%%%%%%%%%%%%%%%%%%%%%%%%%%%%%%%%%%%%%%%%%%%%%%%%%%%%%%%%%%%%%%%%%
\subsection{Files and Installation}

The package consists of the files:
%
\begin{center}
\begin{tabular}{ll}
    |README.txt|   & readme file \\
    |childdoc.ins| & installation file \\
    |childdoc.dtx| & source file \\
    |childdoc.def| & definition file \\
    |cdocsamp.tex| & sample main file \\
    |cdocsch1.tex| & sample include file \\
    |cdocsch2.tex| & sample include file \\
    |cdocspt3.tex| & sample part file \\
    |cdocspt4.tex| & sample part file \\
    |cdocsdrf.tex| & sample redirection file \\
    |cdocsfn1.tex| & sample redirection file \\
    |cdocsfn2.tex| & sample redirection file \\
    |childdoc.pdf| & manual
\end{tabular}
\end{center}
%
The distribution consists of the files
|README.txt|, |childdoc.ins| and |childdoc.dtx|.
%
\begin{itemize}
\item
Run (pdf)\LaTeX{} on |childdoc.dtx|
to compile the manual |childdoc.pdf| (this file).
\item
Run \LaTeX{} on |childdoc.ins| to create the definitions file |childdoc.def|
and the sample |cdocsamp.tex| with include files
|cdocsch1.tex|, |cdocsch2.tex|, |cdocspt3.tex|, |cdocspt4.tex|,
|cdocsdrf.tex|, |cdocsfn1.tex|, |cdocsfn2.tex|.
Then copy the file |childdoc.def| to an appropriate directory of your \LaTeX{}
distribution, e.g.\ \textit{texmf-root}|/tex/latex/childdoc|.
\end{itemize}

%%%%%%%%%%%%%%%%%%%%%%%%%%%%%%%%%%%%%%%%%%%%%%%%%%%%%%%%%%%%%%%%%%%%%%%%%%%%%%%%
\subsection{Related CTAN Packages}

There are several other packages which offer a similar functionality:
%
\begin{itemize}
\item
The packages
\href{http://ctan.org/pkg/docmute}{\textsf{docmute}},
\href{http://ctan.org/pkg/includex}{\textsf{includex}} and
\href{http://ctan.org/pkg/standalone}{\textsf{standalone}}
provide commands to include only the document body of
a child file thus allowing both files to be compiled individually.
\item
The packages \href{http://ctan.org/pkg/subdocs}{\textsf{subdocs}}
and \href{http://ctan.org/pkg/subfiles}{\textsf{subfiles}}
provide structures in which the main and child documents can be
encapsulated and allowing them to be compiled individually.
The inclusion mechanism is different from the conventional |\include|.
\item
The package \href{http://ctan.org/pkg/combine}{\textsf{combine}}
is an elaborate solution to combine several documents into one.
\end{itemize}
%
See also the CTAN topic \href{http://ctan.org/topic/subdocs}{\textsf{subdocs}}
for further related packages.
The present package differs from the above solutions in that
a document structure constructed with the conventional |\include| mechanism
just needs two extra commands at the top of every file
such that all constituent files can be compiled individually.

%%%%%%%%%%%%%%%%%%%%%%%%%%%%%%%%%%%%%%%%%%%%%%%%%%%%%%%%%%%%%%%%%%%%%%%%%%%%%%%%
%\subsection{Feature Suggestions}
%
%The following is a list of features which may be useful for future
%versions of this package:
%%
%\begin{itemize}
%\item
%\ldots
%\end{itemize}

%%%%%%%%%%%%%%%%%%%%%%%%%%%%%%%%%%%%%%%%%%%%%%%%%%%%%%%%%%%%%%%%%%%%%%%%%%%%%%%%
\subsection{Revision History}

%%%%%%%%%%%%%%%%%%%%%%%%%%%%%%%%%%%%%%%%
\paragraph{v2.0:} 2018/12/30

\begin{itemize}
\item
immediate forward processing
\item
added |\childdocby| mechanism
\item
manual restructured
\end{itemize}

%%%%%%%%%%%%%%%%%%%%%%%%%%%%%%%%%%%%%%%%
\paragraph{v1.6:} 2018/01/17

\begin{itemize}
\item
application for development of include files
\item
corrections to manual
\end{itemize}

%%%%%%%%%%%%%%%%%%%%%%%%%%%%%%%%%%%%%%%%
\paragraph{v1.5:} 2017/05/21

\begin{itemize}
\item
more complete structuring introduced
\item
|\childdocof| introduced
\item
|\childdoc| renamed to |\childdocmain|
\item
|\childredirect| renamed to |\childdocforward| and |\childdocforwardprefix|
and functionality expanded
\end{itemize}

%%%%%%%%%%%%%%%%%%%%%%%%%%%%%%%%%%%%%%%%
\paragraph{v1.0:} 2017/04/27

\begin{itemize}
\item
manual and install package
\item
first version published on CTAN
\end{itemize}

%%%%%%%%%%%%%%%%%%%%%%%%%%%%%%%%%%%%%%%%
\paragraph{v0.6:} 2017/04/26

\begin{itemize}
\item
redirection mechanism added
\end{itemize}

%%%%%%%%%%%%%%%%%%%%%%%%%%%%%%%%%%%%%%%%
\paragraph{v0.5:} 2017/04/26

\begin{itemize}
\item
functionality in definition file
\end{itemize}


%%%%%%%%%%%%%%%%%%%%%%%%%%%%%%%%%%%%%%%%%%%%%%%%%%%%%%%%%%%%%%%%%%%%%%%%%%%%%%%%
%%%%%%%%%%%%%%%%%%%%%%%%%%%%%%%%%%%%%%%%%%%%%%%%%%%%%%%%%%%%%%%%%%%%%%%%%%%%%%%%
%%%%%%%%%%%%%%%%%%%%%%%%%%%%%%%%%%%%%%%%%%%%%%%%%%%%%%%%%%%%%%%%%%%%%%%%%%%%%%%%
\appendix

\settowidth\MacroIndent{\rmfamily\scriptsize 000\ }

 \DocInput{childdoc.dtx}

\end{document}
%</driver>
% \fi
%
% %%%%%%%%%%%%%%%%%%%%%%%%%%%%%%%%%%%%%%%%%%%%%%%%%%%%%%%%%%%%%%%%%%%%%%%%%%%%%%
% %%%%%%%%%%%%%%%%%%%%%%%%%%%%%%%%%%%%%%%%%%%%%%%%%%%%%%%%%%%%%%%%%%%%%%%%%%%%%%
% \section{Sample}
%\iffalse
%<*samplemain>
%\fi
%
% The following presents a sample document
% with two chapters, two parts, a title page,
% a compile flag as well as three forwarding files to set the flag.
% It consists of eight |.tex| files:
% \begin{center}
% \begin{tabular}{ll}
% |cdocsamp.tex|&main file\\
% |cdocsch1.tex|&include file for chapter 1\\
% |cdocsch2.tex|&include file for chapter 2\\
% |cdocspt3.tex|&include file for part 3\\
% |cdocspt4.tex|&include file for part 4\\
% |cdocsdrf.tex|&forwarding file for main file in draft mode\\
% |cdocsfi1.tex|&forwarding file for final version of chapter 1\\
% |cdocsfi2.tex|&forwarding file for final version of chapter 2\\
% \end{tabular}
% \end{center}
% Each of the eight files can be compiled directly by the \LaTeX{} compiler.
%
% %%%%%%%%%%%%%%%%%%%%%%%%%%%%%%%%%%%%%%
% \paragraph{Main File.}
%
% The main file is called |cdocsamp.tex|.
%
% Load the \textsf{childdoc} definitions and
% declare the filename for the main document:
%    \begin{macrocode}
\input{childdoc.def}
\childdocmain{}
%    \end{macrocode}

% Optional override for |\version| flag:
%    \begin{macrocode}
%%\ifchilddoc\else\providecommand{\version}{draft}\fi
%    \end{macrocode}

% Define the default values for the |\version| flag
% (|final| for the main file and |draft| for childs):
%    \begin{macrocode}
\ifchilddoc
\providecommand{\version}{draft}
\else
\providecommand{\version}{final}
\fi
%    \end{macrocode}

% Load the standard document class:
%    \begin{macrocode}
\documentclass[12pt]{article}
%    \end{macrocode}

% Start the document body:
%    \begin{macrocode}
\begin{document}
%    \end{macrocode}

% Declare a title page.
% Print title, part of document being processed and version flag:
%    \begin{macrocode}
\addtocounter{page}{-1}
\begin{center}
{\LARGE\bfseries{}childdoc example\par}
\vspace{1cm}
\ifchilddoc
\ifchilddocmanual part\else chapter\fi:
`\childdocname' of `\childdocjob'\par
\else
main document: `\childdocjob'\par
\fi
version: \version\par
\end{center}
\newpage
%    \end{macrocode}

% Manually include selected file,
% otherwise process as usual:
%    \begin{macrocode}
\ifchilddocmanual
\section*{part `\childdocname'}
\input{\childdocname}
\else
%    \end{macrocode}

% Include the two chapters:
%    \begin{macrocode}
\include{cdocsch1}
\include{cdocsch2}
%    \end{macrocode}

% Include the two parts unless only chapters should be displayed:
%    \begin{macrocode}
\ifchilddoc\else
\section{part three}
\input{cdocspt3}
\section{part four}
\input{cdocspt4}
\fi
%    \end{macrocode}

% Process as usual until here:
%    \begin{macrocode}
\fi
%    \end{macrocode}

% End of document body:
%    \begin{macrocode}
\end{document}
%    \end{macrocode}
%\iffalse
%</samplemain>
%\fi
%
% %%%%%%%%%%%%%%%%%%%%%%%%%%%%%%%%%%%%%%
% \paragraph{Chapter Include Files.}
%
% The include files are called |cdocsch1.tex| and |cdocsch2.tex|.
%
%\iffalse
%<*samplechap1|samplechap2>
%\fi

% Optional override for |\version| flag:
%    \begin{macrocode}
%%\providecommand{\version}{final}
%    \end{macrocode}

% Include the main document:
%    \begin{macrocode}
\input{childdoc.def}
\childdocof{cdocsamp}
%    \end{macrocode}

%\iffalse
%</samplechap1|samplechap2>
%\fi
%
%\iffalse
%<*samplechap1>
%\fi
% Some text for chapter 1:
%    \begin{macrocode}
\section{one}
some text in chapter one
%    \end{macrocode}

%\iffalse
%</samplechap1>
%\fi
% Some text for chapter 2:
%\iffalse
%<*samplechap2>
%\fi
%    \begin{macrocode}
\section{two}
more text in chapter two
%    \end{macrocode}

%\iffalse
%</samplechap2>
%\fi
%
% %%%%%%%%%%%%%%%%%%%%%%%%%%%%%%%%%%%%%%
% \paragraph{Part Include Files.}
%
% The include files are called |cdocspt3.tex| and |cdocspt4.tex|.
%
%\iffalse
%<*samplepart3|samplepart4>
%\fi

% Optional override for |\version| flag:
%    \begin{macrocode}
%%\providecommand{\version}{final}
%    \end{macrocode}

% Include the main document:
%    \begin{macrocode}
\input{childdoc.def}
\childdocby{cdocsamp}
%    \end{macrocode}

%\iffalse
%</samplepart3|samplepart4>
%\fi
%
%\iffalse
%<*samplepart3>
%\fi
% Some text for part 3:
%    \begin{macrocode}
some text in part three
%    \end{macrocode}

%\iffalse
%</samplepart3>
%\fi
% Some text for part 4:
%\iffalse
%<*samplepart4>
%\fi
%    \begin{macrocode}
more text in part four
%    \end{macrocode}

%\iffalse
%</samplepart4>
%\fi
%
% %%%%%%%%%%%%%%%%%%%%%%%%%%%%%%%%%%%%%%
% \paragraph{Forwarding for a Complete Draft.}
%
% The following forwarding file |cdocsdrf.tex|
% compiles the main document in draft mode:
%\iffalse
%<*sampledraft>
%\fi
%    \begin{macrocode}
\def\version{draft}
\input{childdoc.def}
\childdocforward{cdocsamp}
%    \end{macrocode}

%\iffalse
%</sampledraft>
%\fi
%
% %%%%%%%%%%%%%%%%%%%%%%%%%%%%%%%%%%%%%%
% \paragraph{Forwarding for Final Version of the Chapters.}
%
% The following forwarding files |cdocsfn1.tex| and |cdocsfn2.tex|
% (with identical content)
% compile the final versions of the child documents
% |cdocsch1.tex| and |cdocsch2.tex|, respectively:
%\iffalse
%<*samplefinal>
%\fi
%    \begin{macrocode}
\def\version{final}
\input{childdoc.def}
\childdocforwardprefix[cdocsamp]{cdocsfn}{cdocsch}
%    \end{macrocode}

%\iffalse
%</samplefinal>
%\fi
%
% %%%%%%%%%%%%%%%%%%%%%%%%%%%%%%%%%%%%%%
% \paragraph{Command Line Processing.}
%
% The following three command lines generate the output files
% |cdocscld|, |cdocscl1| and |cdocscl2|
% which should be identical to
% |cdocsdrf|, |cdocsch1| and |cdocsfn2|, respectively:
% \begin{center}
% \begin{tabular}{l}
% |latex -jobname cdocscld \|\\
% |  "\def\version{draft}\input{childdoc.def}\childdocforward{cdocsamp}"|\\
% |latex -jobname cdocscl1 \|\\
% |  "\input{childdoc.def}\childdocforward[cdocsamp]{cdocsch1}"|\\
% |latex -jobname cdocscl2 \|\\
% |  "\def\version{final}\input{childdoc.def}\childdocforward{cdocsch2}"|
% \end{tabular}
% \end{center}
% Note that the trailing backslash on each first line
% merely continues the input to the second line
% (for convenient cut ant paste).
% Furthermore, the command |latex| can be replaced by any
% of its alternative versions such as |pdflatex|.
%
% %%%%%%%%%%%%%%%%%%%%%%%%%%%%%%%%%%%%%%%%%%%%%%%%%%%%%%%%%%%%%%%%%%%%%%%%%%%%%%
% %%%%%%%%%%%%%%%%%%%%%%%%%%%%%%%%%%%%%%%%%%%%%%%%%%%%%%%%%%%%%%%%%%%%%%%%%%%%%%
% \section{Implementation}
%\iffalse
%<*package>
%\fi
%
% This section describes the definitions file |childdoc.def|.

% The definitions cannot be loaded using |\usepackage| or |\RequirePackage|
% which has a mechanism to prevent loading a style file more than once.
% When loading the definitions by means of |\input|
% multiple instances have to be prevented manually:
%\iffalse
%This code needs to be before the `\ProvidesFile' directive
%which is defined at the beginning of this file.
%Therefore it is also placed there and commented out here.
%</package>
%<*discard>
%\fi
%    \begin{macrocode}
\ifdefined\childdocmain\endinput\fi
%    \end{macrocode}
%\iffalse
%</discard>
%<*package>
%\fi
%
% \macro{\ifchilddoc}
% \macro{\ifchilddocmanual}
% The conditional |\ifchilddoc| tells whether a
% child (true) or main (false) document is being compiled.
% The conditional |\ifchilddocmanual| tells whether
% the |\includeonly| mechanism is used (false) or
% the selection of child files must be performed manually (true).
% The definitions initialise to false:
%    \begin{macrocode}
\newif\ifchilddoc
\newif\ifchilddocmanual
%    \end{macrocode}

% \macro{\childdocname}
% \macro{\childdocjob}
% The macro |\childdocname| stores the name of the main document
% to be compiled. The macro |\childdocjob| stores the name of
% the document on which the \LaTeX{} compiler was originally invoked.
% The content of |\jobname| cannot be compared
% to filenames specified in the source due to different catcodes.
% The following code rescans |\jobname|, stores the result
% in |\childdocname| and saves a copy in |\childdocjob|:
%    \begin{macrocode}
\edef\childdocname{\scantokens\expandafter{\jobname\noexpand}}
\let\childdocjob\childdocname
%    \end{macrocode}

% \macro{\childdocdisable}
% The macro |\childdocdisable| prevents the main file
% from being processed more than once.
% At this stage, the main document command |\childdocmain|
% is assumed to be called once again where it should do nothing.
% Any subsequent call to it should prevent
% a secondary processing of the main document
% It overwrites the forwarding commands
% |\childdocof| and |\childdocforward|
% with empty macros to prevent further inclusions of the main document:
%    \begin{macrocode}
\newcommand{\childdocdisable}
{
  \renewcommand{\childdocmain}[1]{\renewcommand{\childdocmain}[1]{\endinput}}
  \renewcommand{\childdocof}[1]{}
  \renewcommand{\childdocby}[2][]{}
  \renewcommand{\childdocforward}[2][]{}
  \renewcommand{\childdocdisable}{}
}
%    \end{macrocode}

% \macro{\childdocmain}
% The macro |\childdocmain| is to be called at the top of the main file
% with nothing or the main filename (without extension) as argument.
% First, it breaks loops.
% If the argument is not empty and does not match |\childdocname|
% (which is set by the first inclusion of |childdoc.def|),
% |\ifchilddoc| is set to true, |\includeonly| is applied to the child file
% and |\jobname| is set to the main file
% (for proper handling of |.aux| files):
%    \begin{macrocode}
\newcommand{\childdocmain}[1]
{
  \childdocdisable\childdocmain{}
  \if?#1?\else
    \begingroup
      \def\childdoctmp{#1}
      \ifx\childdoctmp\childdocname
        \def\childdoctmp{}
      \else
        \def\childdoctmp
        {
          \childdoctrue
          \includeonly{\childdocname}
          \def\childdocjob{#1}
          \def\jobname{#1}
        }
      \fi
      \expandafter
    \endgroup
    \childdoctmp
  \fi
}
%    \end{macrocode}

% \macro{\childdocof}
% The command |\childdocof| redirects
% compilation to the main file |#1|.
%    \begin{macrocode}
\newcommand{\childdocof}[1]
{
  \childdocdisable
  \childdoctrue
  \includeonly{\childdocname}
  \def\jobname{#1}
  \def\childdocjob{#1}
  \input{#1}
}
%    \end{macrocode}

% \macro{\childdocby}
% The command |\childdocby| ....
%    \begin{macrocode}
\newcommand{\childdocby}[2][]
{
  \childdocdisable
  \childdoctrue
  \childdocmanualtrue
  \if?#1?\else
    \def\jobname{#2}
  \fi
  \def\childdocjob{#2}
  \input{#2}
  \endinput
}
%    \end{macrocode}

% \macro{\childdocforward}
% The command |\childdocforward| redirects
% compilation to the main file or
% (if the optional argument is given) a child file.
% Parameters are set as if the main file
% or a child file starting with |\childdocof| was compiled.
% Then compilation is handed over to the main file:
%    \begin{macrocode}
\newcommand{\childdocforward}[2][]
{
  \begingroup
    \if?#1?
      \def\childdoctmp
      {
        \def\childdocname{#2}
        \def\childdocjob{#2}
        \def\jobname{#2}
        \input{#2}
        \endinput
      }
    \else
      \def\childdoctmp
      {
        \childdocdisable
        \def\childdocname{#2}
        \childdoctrue
        \includeonly{#2}
        \def\childdocjob{#1}
        \def\jobname{#1}
        \input{#1}
        \endinput
      }
    \fi
    \expandafter
  \endgroup
  \childdoctmp
}
%    \end{macrocode}

% \macro{\childdocforwardprefix}
% The command |\childdocforwardprefix| redirects
% compilation to the main or a child file by means of a pattern.
% The prefix |#1| in the current filename is replaced by |#2|
% and the suffix of the current filename is kept
% (it is assumed that the filename does not contain the substring `|~~~|'
% which is used as a delimiter).
% Compilation is handed over to the new file by |\childdocforward|:
%    \begin{macrocode}
\newcommand{\childdocforwardprefix}[3][]
{
  \begingroup
    \def\childdocextract #2##1~~~{\def\childdoctmp{\childdocforward[#1]{#3##1}}}
    \expandafter\childdocextract\childdocname~~~
    \expandafter
  \endgroup
  \childdoctmp
}
%    \end{macrocode}

% \macro{\childdoc}
% The deprecated macro |\childdoc| is a legacy version of |\childdocmain|:
%    \begin{macrocode}
\newcommand{\childdoc}{\childdocmain}
%    \end{macrocode}

% \macro{\childdocredirect}
% The deprecated macro |\childdocredirect| is a legacy version
% of |\childdocforward| and |\childdocforwardprefix|:
%    \begin{macrocode}
\newcommand{\childdocredirect}[2][]
{
  \begingroup
    \if?#1?
      \def\childdoctmp{\childdocforward{#2}}
    \else
      \def\childdoctmp{\childdocforwardprefix{#1}{#2}}
    \fi
    \expandafter
  \endgroup
  \childdoctmp
}
%    \end{macrocode}

%\iffalse
%</package>
%\fi
%
\endinput
|\\
|\childdocof{|\textit{main}|}|\\
\end{tabular}
\end{center}
at the top of every child file \textit{child}
which is included by |\include{|\textit{child}|}|
from within the main file
(or at least for those files to be compiled individually).
The argument \textit{main} must be the filename of the main file.

There are a couple of
considerations in setting up the main and child documents:

%%%%%%%%%%%%%%%%%%%%%%%%%%%%%%%%%%%%%%%%
\paragraph{Restrictions.}

Please note the following restrictions:
\begin{itemize}
\item
|\childdocmain| must be called with one argument \textit{main}
to ensure compatibility with earlier version of the package.
It must either be empty (|\childdocmain{}|)
or precisely match the filename of the main file in which it is specified.
See \secref{sec:detection} for further information.
\item
The filename \textit{main} must be specified without the |.tex| extension.
\item
The filename \textit{main} is case sensitive
(even in case-insensitive file systems)
due to internal string comparison.
\item
The argument \textit{main} should be fully expanded, it cannot be a macro.
\item
Subdirectories and special characters should be avoided in filenames.
\item
The command |\childdocmain{|\textit{main}|}| must be followed by a whitespace.
It should not be followed immediately by another command
or by a comment mark `|%|'.
This is because the \TeX{} parser reads the token immediately following
the argument of |\childdocmain| and puts it
at the beginning of every child section;
however, a white\-space is ignored.
\end{itemize}

%%%%%%%%%%%%%%%%%%%%%%%%%%%%%%%%%%%%%%%%
\paragraph{Content of Main File.}

It is advisable to place all content in the child files included by |\include|.
Any output contained in the main file will appear in all child documents
unless suppressed manually;
it cannot be suppressed automatically by the |\includeonly| directive
and thus should normally be avoided.
A method to include some content in the main file
by means of conditional processing is described in \secref{sec:conditional}.

%%%%%%%%%%%%%%%%%%%%%%%%%%%%%%%%%%%%%%%%
\paragraph{Page Numbering.}

When only a part of the document is compiled,
the appropriate numbering of pages
(as well as other status parameters)
is determined from the |.aux| files.
The latter contain information from previous passes.
However this information needs to propagate through
all intermediate child documents.
Therefore the page numbering in child documents may well
be inconsistent until the complete document is compiled at least once.

A useful (if unconventional) way to always ensure a consistent
page numbering is to restart the numbering in each child document
and denote the pages by `\textit{child}|.|\textit{page}'
where \textit{child} represents the chapter/section number of the child file.
This can be achieved by the command
|\numberwithin{page}{|\textit{child}|}|
of the \textsf{amsmath} package
where \textit{child} can be |chapter| or |section|
depending on the chosen structuring.
Alternatively, one can modify the macro |\thepage| appropriately
and reset the counter |page| at the start of each child file.

%%%%%%%%%%%%%%%%%%%%%%%%%%%%%%%%%%%%%%%%%%%%%%%%%%%%%%%%%%%%%%%%%%%%%%%%%%%%%%%%
\subsection{Conditional Processing}
\label{sec:conditional}

The package provides a mechanism to compile different versions
of a document. To customise the versions further some conditional processing
can come in handy to distinguish which version is being compiled.
The package provides two macros to describe the compilation context:

%%%%%%%%%%%%%%%%%%%%%%%%%%%%%%%%%%%%%%%%
\DescribeMacro{\ifchilddoc}
The conditional |\ifchilddoc| distinguishes between the compilation of
child documents and the main document:
%
\begin{center}
|\ifchilddoc |\textit{child-code}| |[|\||else |\textit{main-code}]| \||fi|
\end{center}

%%%%%%%%%%%%%%%%%%%%%%%%%%%%%%%%%%%%%%%%
\DescribeMacro{\childdocname}
\DescribeMacro{\childdocjob}
The macro |\childdocname| contains the filename (without extension)
of the main or child file being processed.
Note that |\childdocjob| will always contain the name of the main file.

%%%%%%%%%%%%%%%%%%%%%%%%%%%%%%%%%%%%%%%%
\paragraph{Title Page.}

Conditional processing can be used to include a title or banner page
in the main document when proper precautions are taken.
Importantly, the code in the main file should ensure that the page counter
(as well as other status parameters which are stored in the |.aux| files)
takes the same value after the conditional processing.
Otherwise the page numbers may take divergent values
depending on which part is compiled.

For example, a title page could be declared by:
%
\begin{center}
\begin{tabular}{l}
|\ifchilddoc\||else|\\
|\addtocounter{page}{-1}|\\
\textit{code for title page}\\
|\newpage|\\
|\||fi|
\end{tabular}
\end{center}
%
A banner page for the child documents can be generated by:
%
\begin{center}
\begin{tabular}{l}
|\ifchilddoc|\\
|\addtocounter{page}{-1}|\\
\textit{code for banner page}\\
|\newpage|\\
|\||fi|
\end{tabular}
\end{center}
%
Here one could write a message such as:
\begin{center}
|This is the part \childdocname{} of \childdocjob{}.|
\end{center}

%%%%%%%%%%%%%%%%%%%%%%%%%%%%%%%%%%%%%%%%%%%%%%%%%%%%%%%%%%%%%%%%%%%%%%%%%%%%%%%%
\subsection{Flags}
\label{sec:flags}

The package makes it easy to generate different versions
of the main or child documents.
To this end compilation flags can be defined
and assigned different default values.
They will be particularly useful in conjunction
with the forwarding mechanism described in \secref{sec:forward}.

For example, it may be useful to have a flag |\version|
which can be set to |draft| or |final|.
The document source will contain some conditional code
depending on the value of |\version|.
Suppose further, the flag should default to |final| for the main file
and to |draft| for child files
which is a natural assignment for editing the document.
This is achieved by placing the following code
in the preamble of the main document
(below the |\childdocmain| directive):
%
\begin{center}
\begin{tabular}{l}
|\ifchilddoc|\\
|\providecommand{\version}{draft}|\\
|\||else|\\
|\providecommand{\version}{final}|\\
|\||fi|
\end{tabular}
\end{center}
%
The definition by |\providecommand| makes sure
that previous definitions are not overwritten.
Further statements |\providecommand{\version}{...}|
can thus be added before the above code to override it.

For the main file, one might add a line
(between |\childdocmain| and the above block)
%
\begin{center}
|%\ifchilddoc\||else\providecommand{\version}{draft}\||fi|
\end{center}
%
which can be uncommented to produce a draft version.
Likewise one can add a line to the very top of a child file
(above the |\childdocof{|\textit{main}|}| directive)
%
\begin{center}
|%\providecommand{\version}{final}|
\end{center}
%
which can be uncommented to produce the final version of this child document.

%%%%%%%%%%%%%%%%%%%%%%%%%%%%%%%%%%%%%%%%%%%%%%%%%%%%%%%%%%%%%%%%%%%%%%%%%%%%%%%%
\subsection{Forwarding}
\label{sec:forward}

Different versions of the main or child documents
using compilation flags as described in \secref{sec:flags}
can be (permanently) stored in different files
for convenient compilation, viewing and distribution.
To this end, the package defines a command
to pass on compilation to a different file:

%%%%%%%%%%%%%%%%%%%%%%%%%%%%%%%%%%%%%%%%
\DescribeMacro{\childdocforward}
The command |\childdocforward| redirects processing to
another source file:
%
\begin{center}
\begin{tabular}{l}
|% \iffalse
%
% childdoc.dtx Copyright (C) 2017-2018 Niklas Beisert
%
% This work may be distributed and/or modified under the
% conditions of the LaTeX Project Public License, either version 1.3
% of this license or (at your option) any later version.
% The latest version of this license is in
%   http://www.latex-project.org/lppl.txt
% and version 1.3 or later is part of all distributions of LaTeX
% version 2005/12/01 or later.
%
% This work has the LPPL maintenance status `maintained'.
%
% The Current Maintainer of this work is Niklas Beisert.
%
% This work consists of the files childdoc.dtx and childdoc.ins
% and the derived files childdoc.def and cdocsamp.tex with
% cdocsch1.tex, cdocsch2.tex, cdocsdrf.tex, cdocsfn1.tex, cdocsfn2.tex.
%
%<package>\ifdefined\childdocmain\endinput\fi
%<package>\ProvidesFile{childdoc.def}[2018/12/30 v2.0 child document driver]
%<samplemain>\ProvidesFile{cdocsamp.tex}[2018/12/30 v2.0 sample for childdoc]
%<*driver>
%\ProvidesFile{childdoc.drv}[2018/12/30 v2.0 childdoc reference manual file]
\PassOptionsToClass{10pt,a4paper}{article}
\documentclass{ltxdoc}

\usepackage[margin=35mm]{geometry}
\usepackage{hyperref}
\usepackage{hyperxmp}
\usepackage[usenames]{color}

\hypersetup{colorlinks=true}
\hypersetup{pdfstartview=FitH}
\hypersetup{pdfpagemode=UseNone}
\hypersetup{pdfsource={}}
\hypersetup{pdflang={en-UK}}
\hypersetup{pdfcopyright={Copyright 2017-2018 Niklas Beisert.
  This work may be distributed and/or modified under the
  conditions of the LaTeX Project Public License, either version 1.3
  of this license or (at your option) any later version.}}
\hypersetup{pdflicenseurl={http://www.latex-project.org/lppl.txt}}
\hypersetup{pdfcontactaddress={ETH Zurich, ITP, HIT K,
  Wolfgang-Pauli-Strasse 27}}
\hypersetup{pdfcontactpostcode={8093}}
\hypersetup{pdfcontactcity={Zurich}}
\hypersetup{pdfcontactcountry={Switzerland}}
\hypersetup{pdfcontactemail={nbeisert@itp.phys.ethz.ch}}
\hypersetup{pdfcontacturl={http://people.phys.ethz.ch/\xmptilde nbeisert/}}

\newcommand{\secref}[1]{\hyperref[#1]{section \ref*{#1}}}

\parskip1ex
\parindent0pt
\let\olditemize\itemize
\def\itemize{\olditemize\parskip0pt}

\begin{document}

\title{The \textsf{childdoc} Package}
\hypersetup{pdftitle={The childdoc Package}}
\author{Niklas Beisert\\[2ex]
  Institut f\"ur Theoretische Physik\\
  Eidgen\"ossische Technische Hochschule Z\"urich\\
  Wolfgang-Pauli-Strasse 27, 8093 Z\"urich, Switzerland\\[1ex]
  \href{mailto:nbeisert@itp.phys.ethz.ch}
  {\texttt{nbeisert@itp.phys.ethz.ch}}}
\hypersetup{pdfauthor={Niklas Beisert}}
\hypersetup{pdfsubject={Manual for the LaTeX2e Package childdoc}}
\date{30 December 2018, \textsf{v2.0}}
\maketitle

\begin{abstract}\noindent
\textsf{childdoc} is a \LaTeXe{} package
that enables the direct compilation
of document sections included by |\include|
to individual files.
\end{abstract}

\begingroup
\parskip0ex
\tableofcontents
\endgroup

%%%%%%%%%%%%%%%%%%%%%%%%%%%%%%%%%%%%%%%%%%%%%%%%%%%%%%%%%%%%%%%%%%%%%%%%%%%%%%%%
%%%%%%%%%%%%%%%%%%%%%%%%%%%%%%%%%%%%%%%%%%%%%%%%%%%%%%%%%%%%%%%%%%%%%%%%%%%%%%%%
\section{Introduction}

\LaTeX{} provides a mechanism to structure a large document (such as a book)
into a main file and several child files (containing the chapters)
using the |\include| command.
This mechanism is beneficial for documents
which span hundreds of pages in order to
make the source file(s) more manageable.
Moreover, compilation can be restricted to
selected child files by means of the |\includeonly| command.
The latter feature can be used to reduce the compilation time while editing
(this was significantly more useful in the earlier days of \LaTeX{})
or to generate a smaller document which is easier to navigate.
Another application of |\includeonly| is to generate
documents consisting of selected parts of the complete document.

However, there are a few drawbacks of the plain |\include| mechanism:
\begin{itemize}
\item
The child files cannot be compiled on their own,
they can only be compiled via the main file.
A naive editing environment
(such as a text editor with an option
to have the current file processed by \LaTeX)
may require one to switch to the main file before compiling;
attempting to compile the child file produces errors.
\item
The main file must be modified (each time)
to adjust the |\includeonly| command
to the present needs. This easily leaves the main file in a messy state.
\item
The generated document will always carry the filename
of the main document. This is inconvenient if
several child files are to be compiled and
to be kept for distribution.
\end{itemize}

The present package provides a simple interface
to make child files individually compilable by \LaTeX{}.
Compiling a child file then has the same effect as compiling
the main file with an |\includeonly| command
to select the appropriate child.
Moreover the generated document will carry the name of the child
rather than the main file.
This resolves all three above issues.

This feature is meant to make the editing of books,
thesis documents and lecture notes somewhat more convenient.
However, the package can also be used efficiently for
composing a series of documents (such as exercise sheets)
which are typically distributed individually.
It then assists the author in generating the individual documents
(potentially in different versions)
as well as a document containing the collected series.
Another application is in developing style files
or other kinds of included material
where compilation of the style file could redirect
to a sample or test file.

%%%%%%%%%%%%%%%%%%%%%%%%%%%%%%%%%%%%%%%%%%%%%%%%%%%%%%%%%%%%%%%%%%%%%%%%%%%%%%%%
%%%%%%%%%%%%%%%%%%%%%%%%%%%%%%%%%%%%%%%%%%%%%%%%%%%%%%%%%%%%%%%%%%%%%%%%%%%%%%%%
\section{Usage}

First of all, the package \textsf{childdoc} is \emph{not} a standard
\LaTeXe{} |.sty| style file! Therefore it needs to be invoked in
a non-standard way.

%%%%%%%%%%%%%%%%%%%%%%%%%%%%%%%%%%%%%%%%%%%%%%%%%%%%%%%%%%%%%%%%%%%%%%%%%%%%%%%%
\subsection{Included Files}
\label{sec:include}

%%%%%%%%%%%%%%%%%%%%%%%%%%%%%%%%%%%%%%%%
\DescribeMacro{\childdocmain}
To use the package, add the commands
\begin{center}
\begin{tabular}{l}
|\input{childdoc.def}|\\
|\childdocmain{}|\\
\end{tabular}
\end{center}
at the very top of the main \LaTeX{} file,
in particular \emph{before} the |\documentclass| statement!
The argument of |\childdocmain| should be left empty
(but it must be present).

%%%%%%%%%%%%%%%%%%%%%%%%%%%%%%%%%%%%%%%%
\DescribeMacro{\childdocof}
Furthermore, add the commands
\begin{center}
\begin{tabular}{l}
|\input{childdoc.def}|\\
|\childdocof{|\textit{main}|}|\\
\end{tabular}
\end{center}
at the top of every child file \textit{child}
which is included by |\include{|\textit{child}|}|
from within the main file
(or at least for those files to be compiled individually).
The argument \textit{main} must be the filename of the main file.

There are a couple of
considerations in setting up the main and child documents:

%%%%%%%%%%%%%%%%%%%%%%%%%%%%%%%%%%%%%%%%
\paragraph{Restrictions.}

Please note the following restrictions:
\begin{itemize}
\item
|\childdocmain| must be called with one argument \textit{main}
to ensure compatibility with earlier version of the package.
It must either be empty (|\childdocmain{}|)
or precisely match the filename of the main file in which it is specified.
See \secref{sec:detection} for further information.
\item
The filename \textit{main} must be specified without the |.tex| extension.
\item
The filename \textit{main} is case sensitive
(even in case-insensitive file systems)
due to internal string comparison.
\item
The argument \textit{main} should be fully expanded, it cannot be a macro.
\item
Subdirectories and special characters should be avoided in filenames.
\item
The command |\childdocmain{|\textit{main}|}| must be followed by a whitespace.
It should not be followed immediately by another command
or by a comment mark `|%|'.
This is because the \TeX{} parser reads the token immediately following
the argument of |\childdocmain| and puts it
at the beginning of every child section;
however, a white\-space is ignored.
\end{itemize}

%%%%%%%%%%%%%%%%%%%%%%%%%%%%%%%%%%%%%%%%
\paragraph{Content of Main File.}

It is advisable to place all content in the child files included by |\include|.
Any output contained in the main file will appear in all child documents
unless suppressed manually;
it cannot be suppressed automatically by the |\includeonly| directive
and thus should normally be avoided.
A method to include some content in the main file
by means of conditional processing is described in \secref{sec:conditional}.

%%%%%%%%%%%%%%%%%%%%%%%%%%%%%%%%%%%%%%%%
\paragraph{Page Numbering.}

When only a part of the document is compiled,
the appropriate numbering of pages
(as well as other status parameters)
is determined from the |.aux| files.
The latter contain information from previous passes.
However this information needs to propagate through
all intermediate child documents.
Therefore the page numbering in child documents may well
be inconsistent until the complete document is compiled at least once.

A useful (if unconventional) way to always ensure a consistent
page numbering is to restart the numbering in each child document
and denote the pages by `\textit{child}|.|\textit{page}'
where \textit{child} represents the chapter/section number of the child file.
This can be achieved by the command
|\numberwithin{page}{|\textit{child}|}|
of the \textsf{amsmath} package
where \textit{child} can be |chapter| or |section|
depending on the chosen structuring.
Alternatively, one can modify the macro |\thepage| appropriately
and reset the counter |page| at the start of each child file.

%%%%%%%%%%%%%%%%%%%%%%%%%%%%%%%%%%%%%%%%%%%%%%%%%%%%%%%%%%%%%%%%%%%%%%%%%%%%%%%%
\subsection{Conditional Processing}
\label{sec:conditional}

The package provides a mechanism to compile different versions
of a document. To customise the versions further some conditional processing
can come in handy to distinguish which version is being compiled.
The package provides two macros to describe the compilation context:

%%%%%%%%%%%%%%%%%%%%%%%%%%%%%%%%%%%%%%%%
\DescribeMacro{\ifchilddoc}
The conditional |\ifchilddoc| distinguishes between the compilation of
child documents and the main document:
%
\begin{center}
|\ifchilddoc |\textit{child-code}| |[|\||else |\textit{main-code}]| \||fi|
\end{center}

%%%%%%%%%%%%%%%%%%%%%%%%%%%%%%%%%%%%%%%%
\DescribeMacro{\childdocname}
\DescribeMacro{\childdocjob}
The macro |\childdocname| contains the filename (without extension)
of the main or child file being processed.
Note that |\childdocjob| will always contain the name of the main file.

%%%%%%%%%%%%%%%%%%%%%%%%%%%%%%%%%%%%%%%%
\paragraph{Title Page.}

Conditional processing can be used to include a title or banner page
in the main document when proper precautions are taken.
Importantly, the code in the main file should ensure that the page counter
(as well as other status parameters which are stored in the |.aux| files)
takes the same value after the conditional processing.
Otherwise the page numbers may take divergent values
depending on which part is compiled.

For example, a title page could be declared by:
%
\begin{center}
\begin{tabular}{l}
|\ifchilddoc\||else|\\
|\addtocounter{page}{-1}|\\
\textit{code for title page}\\
|\newpage|\\
|\||fi|
\end{tabular}
\end{center}
%
A banner page for the child documents can be generated by:
%
\begin{center}
\begin{tabular}{l}
|\ifchilddoc|\\
|\addtocounter{page}{-1}|\\
\textit{code for banner page}\\
|\newpage|\\
|\||fi|
\end{tabular}
\end{center}
%
Here one could write a message such as:
\begin{center}
|This is the part \childdocname{} of \childdocjob{}.|
\end{center}

%%%%%%%%%%%%%%%%%%%%%%%%%%%%%%%%%%%%%%%%%%%%%%%%%%%%%%%%%%%%%%%%%%%%%%%%%%%%%%%%
\subsection{Flags}
\label{sec:flags}

The package makes it easy to generate different versions
of the main or child documents.
To this end compilation flags can be defined
and assigned different default values.
They will be particularly useful in conjunction
with the forwarding mechanism described in \secref{sec:forward}.

For example, it may be useful to have a flag |\version|
which can be set to |draft| or |final|.
The document source will contain some conditional code
depending on the value of |\version|.
Suppose further, the flag should default to |final| for the main file
and to |draft| for child files
which is a natural assignment for editing the document.
This is achieved by placing the following code
in the preamble of the main document
(below the |\childdocmain| directive):
%
\begin{center}
\begin{tabular}{l}
|\ifchilddoc|\\
|\providecommand{\version}{draft}|\\
|\||else|\\
|\providecommand{\version}{final}|\\
|\||fi|
\end{tabular}
\end{center}
%
The definition by |\providecommand| makes sure
that previous definitions are not overwritten.
Further statements |\providecommand{\version}{...}|
can thus be added before the above code to override it.

For the main file, one might add a line
(between |\childdocmain| and the above block)
%
\begin{center}
|%\ifchilddoc\||else\providecommand{\version}{draft}\||fi|
\end{center}
%
which can be uncommented to produce a draft version.
Likewise one can add a line to the very top of a child file
(above the |\childdocof{|\textit{main}|}| directive)
%
\begin{center}
|%\providecommand{\version}{final}|
\end{center}
%
which can be uncommented to produce the final version of this child document.

%%%%%%%%%%%%%%%%%%%%%%%%%%%%%%%%%%%%%%%%%%%%%%%%%%%%%%%%%%%%%%%%%%%%%%%%%%%%%%%%
\subsection{Forwarding}
\label{sec:forward}

Different versions of the main or child documents
using compilation flags as described in \secref{sec:flags}
can be (permanently) stored in different files
for convenient compilation, viewing and distribution.
To this end, the package defines a command
to pass on compilation to a different file:

%%%%%%%%%%%%%%%%%%%%%%%%%%%%%%%%%%%%%%%%
\DescribeMacro{\childdocforward}
The command |\childdocforward| redirects processing to
another source file:
%
\begin{center}
\begin{tabular}{l}
|\input{childdoc.def}|\\
|\childdocforward[|\textit{main}|]{|\textit{dest}|}|\\
\end{tabular}
\end{center}
%
The argument \textit{dest} is the destination file
(without extension).
It should be the main file or one of the child files.
Note that further \textsf{childdoc} directives
such as |\childdocof| and |\childdocforward|
in the indicated file will be processed in this form.
The optional argument \textit{main}
passes on directly to the main file \textit{main}
while pretending to compile the child \textit{dest}.
This form behaves as if \textit{dest}
issues |\childdocof{|\textit{main}|}| right away,
and no further \textsf{childdoc} directives will be processed.

%%%%%%%%%%%%%%%%%%%%%%%%%%%%%%%%%%%%%%%%
\DescribeMacro{\...prefix}
In the alternative form |\childdocforwardprefix|,
%
\begin{center}
\begin{tabular}{l}
|\input{childdoc.def}|\\
|\childdocforwardprefix[|\textit{main}|]{|\textit{prefix}|}{|\textit{dest}|}|
\end{tabular}
\end{center}
%
the destination file is determined by a pattern
depending on the current file:
To make this work, the current file must be called
`{\textit{prefix}\hspace{0.2em}\textit{suffix}}'
with \textit{prefix} matching precisely the argument.
Processing is then passed on to the file
`{\textit{dest}\hspace{0.2em}\textit{suffix}}'.
Surely, the same effect is achieved by
directly specifying the
argument `{\textit{dest}\hspace{0.2em}\textit{suffix}}'
in the first form.
However, that requires to set up a different file
for each child. With the alternative form of the command
all these files can have exactly the same content
which simplifies setting them up and maintaining them.

For example, the following file |draft.tex|
with a compilation flag |\version| as described in \secref{sec:flags}
compiles the main document as a draft:
%
\begin{center}
\begin{tabular}{l}
|\def\version{draft}|\\
|\input{childdoc.def}|\\
|\childdocforward{|\textit{main}|}|
\end{tabular}
\end{center}
%
Likewise, the following files |final|\textit{nn}|.tex|
compile the final version of the child document
|child|\textit{nn}|.tex|:
%
\begin{center}
\begin{tabular}{l}
|\def\version{final}|\\
|\input{childdoc.def}|\\
|\childdocforwardprefix{final}{child}|
\end{tabular}
\end{center}
%

Note that when several versions of a main file and/or of each child file
are to be generated, it may be convenient to set up a |Makefile| or
shell script to automatise the process.

%%%%%%%%%%%%%%%%%%%%%%%%%%%%%%%%%%%%%%%%%%%%%%%%%%%%%%%%%%%%%%%%%%%%%%%%%%%%%%%%
\subsection{Command Line Processing}
\label{sec:commandline}

The effect of redirection files can also be achieved by invoking
the \LaTeX{} compiler with a more elaborate command line.
Most conveniently this should be done as part
of a shell script or a |Makefile|.

When using \textsf{childdoc} in the main file, the following
command lines effectively perform a redirection
(note that depending on the shell being used,
backslashes may have to be doubled: `|\|' $\to$ `|\\|'):
%
\begin{center}
|... -jobname "|\textit{target}|" |\\|"|[\textit{flags}]%
|\input{childdoc.def}\childdocforward[|\textit{main}|]{|\textit{dest}|}"|
\end{center}
%
Here \textit{target} is the name of the output file,
\textit{main} is the name of the main file
and \textit{dest} is the name of the main or child file to be processed
(all filenames without extensions).
The optional argument \textit{main} can be omitted
if \textit{main} matches \textit{dest}.
Optionally, compilation \textit{flags} can be defined via |\def| commands.
This command line makes the \TeX{} engine believe
it is compiling the file \textit{target}
whose content is specified as the latter parameter.
The provided code then forwards the processing to
\textit{main} or \textit{dest} as described in \secref{sec:forward}.

%%%%%%%%%%%%%%%%%%%%%%%%%%%%%%%%%%%%%%%%%%%%%%%%%%%%%%%%%%%%%%%%%%%%%%%%%%%%%%%%
\subsection{Include by Input}
\label{sec:input}

Including child documents by |\include| has some restrictions by design.
Most notably, the content of a child document always occupies
its own set of pages; pages cannot be shared between child documents.
Usually, this behaviour makes perfect sense
because each child document contain an essential part of the document.
However, in some situations it may be desirable to compose
a document from a collection of parts
without having mandatory page breaks between then.
For this case, the package
provides a mechanism to include parts
by |\input| which can also be processed individually.
However, by construction this mechanism
requires manual handling of the content to be output.

%%%%%%%%%%%%%%%%%%%%%%%%%%%%%%%%%%%%%%%%
\DescribeMacro{\ifchilddocmanual}
The main file should be prepared as usual, see \secref{sec:include}.
However, the document body must make a distinction
between processing of an individual part and of the main document, e.g.:
%
\begin{center}
\begin{tabular}{l}
|\ifchilddocmanual|\\
|\input{\childdocname}|\\
|\||else|\\
\textit{document body with }|\input{|\textit{part}|}|\\
|\||fi|
\end{tabular}
\end{center}
%
The conditional |\ifchilddocmanual| is true whenever
a part to be included by |\input| is being compiled,
and the name of the part is stored in |\childdocname|.

%%%%%%%%%%%%%%%%%%%%%%%%%%%%%%%%%%%%%%%%
\DescribeMacro{\childdocby}
Each part to be included by |\input| should start with:
%
\begin{center}
\begin{tabular}{l}
|\input{childdoc.def}|\\
|\childdocby{|\textit{main}|}|\\
\end{tabular}
\end{center}
%
The directive |\childdocby| is similar to |\childdocof|
described in \secref{sec:include},
but the subsequent selection of content must be done manually.
To that end, both |\ifchilddoc| and |\ifchilddocmanual|
will be true upon processing of a part,
and the name of the part is stored in |\childdocname|.
Note that |\jobname| will be set to the filename of the current part
so that each part receives an individual |.aux| file
that does not interfere with the |.aux| file(s) of the main document.
This behaviour can be altered by the alternative form
|\childdocby[*]{|\textit{main}|}| (with a non-empty optional argument)
which uses the |.aux| file of the main document
by setting |\jobname| to \textit{main}.

%%%%%%%%%%%%%%%%%%%%%%%%%%%%%%%%%%%%%%%%%%%%%%%%%%%%%%%%%%%%%%%%%%%%%%%%%%%%%%%%
\subsection{Driver Development}
\label{sec:driver}

The \textsf{childdoc} mechanism can also be use for the development
of definition files such as \LaTeX{} styles or classes.
This case differs from the above setup with multiple parts
included by |\include| in that no |\includeonly| should be invoked.
This can be achieved by starting the include file
(before |\ProvidesPackage|) with:
%
\begin{center}
\begin{tabular}{l}
|\input{childdoc.def}|\\
|\childdocforward{|\textit{main}|}|\\
\end{tabular}
\end{center}
%
or alternatively with:
%
\begin{center}
\begin{tabular}{l}
|\input{childdoc.def}|\\
|\childdocby{|\textit{main}|}|\\
\end{tabular}
\end{center}
%
Both forms have slightly different effects as described above.
The main file is prepared as usual, see \secref{sec:include}.

%%%%%%%%%%%%%%%%%%%%%%%%%%%%%%%%%%%%%%%%%%%%%%%%%%%%%%%%%%%%%%%%%%%%%%%%%%%%%%%%
\subsection{Legacy Detection}
\label{sec:detection}

The directive |\childdocmain| in the main file can detect
whether the complete document or merely a child is to be compiled
even without using the directive |\childdocof|.
This method is deprecated because it is less robust
and there is no compelling reason to use it;
it is merely provided for backward compatibility
and it may be removed in future versions.

If the detection mechanism is to be used,
it is mandatory to correctly specify
the filename of the main file as the argument of |\childdocmain|:
%
\begin{center}
\begin{tabular}{l}
|\input{childdoc.def}|\\
|\childdocmain{|\textit{main}|}|\\
\end{tabular}
\end{center}
%
If |\jobname| does not match the argument \textit{main} of |\childdocmain|,
it is assumed that |\jobname| points to the child file to be compiled.
When using |\childdocmain| with the main file specified as argument,
it suffices to start a child file
with just |\input{|\textit{main}|}|
without loading of the package and using |\childdocof|.
If instead all processing is done
with the appropriate \textsf{childdoc} directives,
the argument of \textit{main} of |\childdocmain| can be empty.

An alternative version of the command line processing described
in \secref{sec:commandline} using the detection mechanism reads:
%
\begin{center}
|... -jobname "|\textit{target}|" "|[\textit{flags}]%
[|\def\jobname{|\textit{dest}|}|]|\input{|\textit{main}|}"|
\end{center}

%%%%%%%%%%%%%%%%%%%%%%%%%%%%%%%%%%%%%%%%%%%%%%%%%%%%%%%%%%%%%%%%%%%%%%%%%%%%%%%%
\subsection{Manual Code}
\label{sec:manual}

In case one cannot be certain whether the definitions file |childdoc.def|
is installed on the target \TeX{} distribution
and one prefers not to ship it,
it is conceivable to paste a few relevant commands into the sources.

To that end, drop all statements |\input{childdoc.def}|
and perform the replacements as outlined below.
Instead of |\childdocmain{|\textit{main}|}| add the following code
to the top of the main file:
%
\begin{center}
\begin{tabular}{l}
|\||ifdefined\childdocname\endinput\||fi\newif\ifchilddoc|\\
|\edef\childdocname{\scantokens\expandafter{\jobname\noexpand}}|\\
|\def\childdocmain{|\textit{main}|}\||ifx\childdocmain\childdocname\||else|\\
|\childdoctrue\includeonly{\childdocname}\let\jobname\childdocmain\||fi|\\
\end{tabular}
\end{center}
%
Instead of |\childdocof{|\textit{main}|}| just include the main file
at the top of each child file:
%
\begin{center}
|\input{|\textit{main}|}|
\end{center}
%
A simple redirection |\childdocforward{|\textit{dest}|}| is achieved by:
%
\begin{center}
|\def\jobname{|\textit{dest}|}\input{\jobname}|
\end{center}
%
The redirection with prefix
|\childdocforwardprefix[|\textit{prefix}|]{|\textit{dest}|}|
is accomplished by:
%
\begin{center}
\begin{tabular}{l}
|{\edef\jobname{\scantokens\expandafter{\jobname\noexpand}}|\\
|\def\redirectjob |\textit{prefix}|#1~~~{\gdef\jobname{|\textit{dest}|#1}}|\\
|\expandafter\redirectjob\jobname~~~}\input{\jobname}|
\end{tabular}
\end{center}

In an alternative approach,
child documents can be compiled by a specific command line
without additional code or specific definitions:
%
\begin{center}
|... -jobname "|\textit{target}|" "|[\textit{flags}]%
|\includeonly{|\textit{dest}|}\input{|\textit{main}|}"|
\end{center}
%

%%%%%%%%%%%%%%%%%%%%%%%%%%%%%%%%%%%%%%%%%%%%%%%%%%%%%%%%%%%%%%%%%%%%%%%%%%%%%%%%
%%%%%%%%%%%%%%%%%%%%%%%%%%%%%%%%%%%%%%%%%%%%%%%%%%%%%%%%%%%%%%%%%%%%%%%%%%%%%%%%
\section{Information}

%%%%%%%%%%%%%%%%%%%%%%%%%%%%%%%%%%%%%%%%%%%%%%%%%%%%%%%%%%%%%%%%%%%%%%%%%%%%%%%%
\subsection{Copyright}

Copyright \copyright{} 2017--2018 Niklas Beisert

This work may be distributed and/or modified under the
conditions of the \LaTeX{} Project Public License, either version 1.3
of this license or (at your option) any later version.
The latest version of this license is in
  \url{http://www.latex-project.org/lppl.txt}
and version 1.3 or later is part of all distributions of \LaTeX{}
version 2005/12/01 or later.

This work has the LPPL maintenance status `maintained'.

The Current Maintainer of this work is Niklas Beisert.

This work consists of the files |README.txt|, |childdoc.ins| and |childdoc.dtx|
as well as the derived files |childdoc.def|, |cdocsamp.tex|
with |cdocsch1.tex|, |cdocsch2.tex|, |cdocspt3.tex|, |cdocspt4.tex|,
|cdocsdrf.tex|, |cdocsfn1.tex|, |cdocsfn2.tex|
as well as |childdoc.pdf|.

%%%%%%%%%%%%%%%%%%%%%%%%%%%%%%%%%%%%%%%%%%%%%%%%%%%%%%%%%%%%%%%%%%%%%%%%%%%%%%%%
\subsection{Files and Installation}

The package consists of the files:
%
\begin{center}
\begin{tabular}{ll}
    |README.txt|   & readme file \\
    |childdoc.ins| & installation file \\
    |childdoc.dtx| & source file \\
    |childdoc.def| & definition file \\
    |cdocsamp.tex| & sample main file \\
    |cdocsch1.tex| & sample include file \\
    |cdocsch2.tex| & sample include file \\
    |cdocspt3.tex| & sample part file \\
    |cdocspt4.tex| & sample part file \\
    |cdocsdrf.tex| & sample redirection file \\
    |cdocsfn1.tex| & sample redirection file \\
    |cdocsfn2.tex| & sample redirection file \\
    |childdoc.pdf| & manual
\end{tabular}
\end{center}
%
The distribution consists of the files
|README.txt|, |childdoc.ins| and |childdoc.dtx|.
%
\begin{itemize}
\item
Run (pdf)\LaTeX{} on |childdoc.dtx|
to compile the manual |childdoc.pdf| (this file).
\item
Run \LaTeX{} on |childdoc.ins| to create the definitions file |childdoc.def|
and the sample |cdocsamp.tex| with include files
|cdocsch1.tex|, |cdocsch2.tex|, |cdocspt3.tex|, |cdocspt4.tex|,
|cdocsdrf.tex|, |cdocsfn1.tex|, |cdocsfn2.tex|.
Then copy the file |childdoc.def| to an appropriate directory of your \LaTeX{}
distribution, e.g.\ \textit{texmf-root}|/tex/latex/childdoc|.
\end{itemize}

%%%%%%%%%%%%%%%%%%%%%%%%%%%%%%%%%%%%%%%%%%%%%%%%%%%%%%%%%%%%%%%%%%%%%%%%%%%%%%%%
\subsection{Related CTAN Packages}

There are several other packages which offer a similar functionality:
%
\begin{itemize}
\item
The packages
\href{http://ctan.org/pkg/docmute}{\textsf{docmute}},
\href{http://ctan.org/pkg/includex}{\textsf{includex}} and
\href{http://ctan.org/pkg/standalone}{\textsf{standalone}}
provide commands to include only the document body of
a child file thus allowing both files to be compiled individually.
\item
The packages \href{http://ctan.org/pkg/subdocs}{\textsf{subdocs}}
and \href{http://ctan.org/pkg/subfiles}{\textsf{subfiles}}
provide structures in which the main and child documents can be
encapsulated and allowing them to be compiled individually.
The inclusion mechanism is different from the conventional |\include|.
\item
The package \href{http://ctan.org/pkg/combine}{\textsf{combine}}
is an elaborate solution to combine several documents into one.
\end{itemize}
%
See also the CTAN topic \href{http://ctan.org/topic/subdocs}{\textsf{subdocs}}
for further related packages.
The present package differs from the above solutions in that
a document structure constructed with the conventional |\include| mechanism
just needs two extra commands at the top of every file
such that all constituent files can be compiled individually.

%%%%%%%%%%%%%%%%%%%%%%%%%%%%%%%%%%%%%%%%%%%%%%%%%%%%%%%%%%%%%%%%%%%%%%%%%%%%%%%%
%\subsection{Feature Suggestions}
%
%The following is a list of features which may be useful for future
%versions of this package:
%%
%\begin{itemize}
%\item
%\ldots
%\end{itemize}

%%%%%%%%%%%%%%%%%%%%%%%%%%%%%%%%%%%%%%%%%%%%%%%%%%%%%%%%%%%%%%%%%%%%%%%%%%%%%%%%
\subsection{Revision History}

%%%%%%%%%%%%%%%%%%%%%%%%%%%%%%%%%%%%%%%%
\paragraph{v2.0:} 2018/12/30

\begin{itemize}
\item
immediate forward processing
\item
added |\childdocby| mechanism
\item
manual restructured
\end{itemize}

%%%%%%%%%%%%%%%%%%%%%%%%%%%%%%%%%%%%%%%%
\paragraph{v1.6:} 2018/01/17

\begin{itemize}
\item
application for development of include files
\item
corrections to manual
\end{itemize}

%%%%%%%%%%%%%%%%%%%%%%%%%%%%%%%%%%%%%%%%
\paragraph{v1.5:} 2017/05/21

\begin{itemize}
\item
more complete structuring introduced
\item
|\childdocof| introduced
\item
|\childdoc| renamed to |\childdocmain|
\item
|\childredirect| renamed to |\childdocforward| and |\childdocforwardprefix|
and functionality expanded
\end{itemize}

%%%%%%%%%%%%%%%%%%%%%%%%%%%%%%%%%%%%%%%%
\paragraph{v1.0:} 2017/04/27

\begin{itemize}
\item
manual and install package
\item
first version published on CTAN
\end{itemize}

%%%%%%%%%%%%%%%%%%%%%%%%%%%%%%%%%%%%%%%%
\paragraph{v0.6:} 2017/04/26

\begin{itemize}
\item
redirection mechanism added
\end{itemize}

%%%%%%%%%%%%%%%%%%%%%%%%%%%%%%%%%%%%%%%%
\paragraph{v0.5:} 2017/04/26

\begin{itemize}
\item
functionality in definition file
\end{itemize}


%%%%%%%%%%%%%%%%%%%%%%%%%%%%%%%%%%%%%%%%%%%%%%%%%%%%%%%%%%%%%%%%%%%%%%%%%%%%%%%%
%%%%%%%%%%%%%%%%%%%%%%%%%%%%%%%%%%%%%%%%%%%%%%%%%%%%%%%%%%%%%%%%%%%%%%%%%%%%%%%%
%%%%%%%%%%%%%%%%%%%%%%%%%%%%%%%%%%%%%%%%%%%%%%%%%%%%%%%%%%%%%%%%%%%%%%%%%%%%%%%%
\appendix

\settowidth\MacroIndent{\rmfamily\scriptsize 000\ }

 \DocInput{childdoc.dtx}

\end{document}
%</driver>
% \fi
%
% %%%%%%%%%%%%%%%%%%%%%%%%%%%%%%%%%%%%%%%%%%%%%%%%%%%%%%%%%%%%%%%%%%%%%%%%%%%%%%
% %%%%%%%%%%%%%%%%%%%%%%%%%%%%%%%%%%%%%%%%%%%%%%%%%%%%%%%%%%%%%%%%%%%%%%%%%%%%%%
% \section{Sample}
%\iffalse
%<*samplemain>
%\fi
%
% The following presents a sample document
% with two chapters, two parts, a title page,
% a compile flag as well as three forwarding files to set the flag.
% It consists of eight |.tex| files:
% \begin{center}
% \begin{tabular}{ll}
% |cdocsamp.tex|&main file\\
% |cdocsch1.tex|&include file for chapter 1\\
% |cdocsch2.tex|&include file for chapter 2\\
% |cdocspt3.tex|&include file for part 3\\
% |cdocspt4.tex|&include file for part 4\\
% |cdocsdrf.tex|&forwarding file for main file in draft mode\\
% |cdocsfi1.tex|&forwarding file for final version of chapter 1\\
% |cdocsfi2.tex|&forwarding file for final version of chapter 2\\
% \end{tabular}
% \end{center}
% Each of the eight files can be compiled directly by the \LaTeX{} compiler.
%
% %%%%%%%%%%%%%%%%%%%%%%%%%%%%%%%%%%%%%%
% \paragraph{Main File.}
%
% The main file is called |cdocsamp.tex|.
%
% Load the \textsf{childdoc} definitions and
% declare the filename for the main document:
%    \begin{macrocode}
\input{childdoc.def}
\childdocmain{}
%    \end{macrocode}

% Optional override for |\version| flag:
%    \begin{macrocode}
%%\ifchilddoc\else\providecommand{\version}{draft}\fi
%    \end{macrocode}

% Define the default values for the |\version| flag
% (|final| for the main file and |draft| for childs):
%    \begin{macrocode}
\ifchilddoc
\providecommand{\version}{draft}
\else
\providecommand{\version}{final}
\fi
%    \end{macrocode}

% Load the standard document class:
%    \begin{macrocode}
\documentclass[12pt]{article}
%    \end{macrocode}

% Start the document body:
%    \begin{macrocode}
\begin{document}
%    \end{macrocode}

% Declare a title page.
% Print title, part of document being processed and version flag:
%    \begin{macrocode}
\addtocounter{page}{-1}
\begin{center}
{\LARGE\bfseries{}childdoc example\par}
\vspace{1cm}
\ifchilddoc
\ifchilddocmanual part\else chapter\fi:
`\childdocname' of `\childdocjob'\par
\else
main document: `\childdocjob'\par
\fi
version: \version\par
\end{center}
\newpage
%    \end{macrocode}

% Manually include selected file,
% otherwise process as usual:
%    \begin{macrocode}
\ifchilddocmanual
\section*{part `\childdocname'}
\input{\childdocname}
\else
%    \end{macrocode}

% Include the two chapters:
%    \begin{macrocode}
\include{cdocsch1}
\include{cdocsch2}
%    \end{macrocode}

% Include the two parts unless only chapters should be displayed:
%    \begin{macrocode}
\ifchilddoc\else
\section{part three}
\input{cdocspt3}
\section{part four}
\input{cdocspt4}
\fi
%    \end{macrocode}

% Process as usual until here:
%    \begin{macrocode}
\fi
%    \end{macrocode}

% End of document body:
%    \begin{macrocode}
\end{document}
%    \end{macrocode}
%\iffalse
%</samplemain>
%\fi
%
% %%%%%%%%%%%%%%%%%%%%%%%%%%%%%%%%%%%%%%
% \paragraph{Chapter Include Files.}
%
% The include files are called |cdocsch1.tex| and |cdocsch2.tex|.
%
%\iffalse
%<*samplechap1|samplechap2>
%\fi

% Optional override for |\version| flag:
%    \begin{macrocode}
%%\providecommand{\version}{final}
%    \end{macrocode}

% Include the main document:
%    \begin{macrocode}
\input{childdoc.def}
\childdocof{cdocsamp}
%    \end{macrocode}

%\iffalse
%</samplechap1|samplechap2>
%\fi
%
%\iffalse
%<*samplechap1>
%\fi
% Some text for chapter 1:
%    \begin{macrocode}
\section{one}
some text in chapter one
%    \end{macrocode}

%\iffalse
%</samplechap1>
%\fi
% Some text for chapter 2:
%\iffalse
%<*samplechap2>
%\fi
%    \begin{macrocode}
\section{two}
more text in chapter two
%    \end{macrocode}

%\iffalse
%</samplechap2>
%\fi
%
% %%%%%%%%%%%%%%%%%%%%%%%%%%%%%%%%%%%%%%
% \paragraph{Part Include Files.}
%
% The include files are called |cdocspt3.tex| and |cdocspt4.tex|.
%
%\iffalse
%<*samplepart3|samplepart4>
%\fi

% Optional override for |\version| flag:
%    \begin{macrocode}
%%\providecommand{\version}{final}
%    \end{macrocode}

% Include the main document:
%    \begin{macrocode}
\input{childdoc.def}
\childdocby{cdocsamp}
%    \end{macrocode}

%\iffalse
%</samplepart3|samplepart4>
%\fi
%
%\iffalse
%<*samplepart3>
%\fi
% Some text for part 3:
%    \begin{macrocode}
some text in part three
%    \end{macrocode}

%\iffalse
%</samplepart3>
%\fi
% Some text for part 4:
%\iffalse
%<*samplepart4>
%\fi
%    \begin{macrocode}
more text in part four
%    \end{macrocode}

%\iffalse
%</samplepart4>
%\fi
%
% %%%%%%%%%%%%%%%%%%%%%%%%%%%%%%%%%%%%%%
% \paragraph{Forwarding for a Complete Draft.}
%
% The following forwarding file |cdocsdrf.tex|
% compiles the main document in draft mode:
%\iffalse
%<*sampledraft>
%\fi
%    \begin{macrocode}
\def\version{draft}
\input{childdoc.def}
\childdocforward{cdocsamp}
%    \end{macrocode}

%\iffalse
%</sampledraft>
%\fi
%
% %%%%%%%%%%%%%%%%%%%%%%%%%%%%%%%%%%%%%%
% \paragraph{Forwarding for Final Version of the Chapters.}
%
% The following forwarding files |cdocsfn1.tex| and |cdocsfn2.tex|
% (with identical content)
% compile the final versions of the child documents
% |cdocsch1.tex| and |cdocsch2.tex|, respectively:
%\iffalse
%<*samplefinal>
%\fi
%    \begin{macrocode}
\def\version{final}
\input{childdoc.def}
\childdocforwardprefix[cdocsamp]{cdocsfn}{cdocsch}
%    \end{macrocode}

%\iffalse
%</samplefinal>
%\fi
%
% %%%%%%%%%%%%%%%%%%%%%%%%%%%%%%%%%%%%%%
% \paragraph{Command Line Processing.}
%
% The following three command lines generate the output files
% |cdocscld|, |cdocscl1| and |cdocscl2|
% which should be identical to
% |cdocsdrf|, |cdocsch1| and |cdocsfn2|, respectively:
% \begin{center}
% \begin{tabular}{l}
% |latex -jobname cdocscld \|\\
% |  "\def\version{draft}\input{childdoc.def}\childdocforward{cdocsamp}"|\\
% |latex -jobname cdocscl1 \|\\
% |  "\input{childdoc.def}\childdocforward[cdocsamp]{cdocsch1}"|\\
% |latex -jobname cdocscl2 \|\\
% |  "\def\version{final}\input{childdoc.def}\childdocforward{cdocsch2}"|
% \end{tabular}
% \end{center}
% Note that the trailing backslash on each first line
% merely continues the input to the second line
% (for convenient cut ant paste).
% Furthermore, the command |latex| can be replaced by any
% of its alternative versions such as |pdflatex|.
%
% %%%%%%%%%%%%%%%%%%%%%%%%%%%%%%%%%%%%%%%%%%%%%%%%%%%%%%%%%%%%%%%%%%%%%%%%%%%%%%
% %%%%%%%%%%%%%%%%%%%%%%%%%%%%%%%%%%%%%%%%%%%%%%%%%%%%%%%%%%%%%%%%%%%%%%%%%%%%%%
% \section{Implementation}
%\iffalse
%<*package>
%\fi
%
% This section describes the definitions file |childdoc.def|.

% The definitions cannot be loaded using |\usepackage| or |\RequirePackage|
% which has a mechanism to prevent loading a style file more than once.
% When loading the definitions by means of |\input|
% multiple instances have to be prevented manually:
%\iffalse
%This code needs to be before the `\ProvidesFile' directive
%which is defined at the beginning of this file.
%Therefore it is also placed there and commented out here.
%</package>
%<*discard>
%\fi
%    \begin{macrocode}
\ifdefined\childdocmain\endinput\fi
%    \end{macrocode}
%\iffalse
%</discard>
%<*package>
%\fi
%
% \macro{\ifchilddoc}
% \macro{\ifchilddocmanual}
% The conditional |\ifchilddoc| tells whether a
% child (true) or main (false) document is being compiled.
% The conditional |\ifchilddocmanual| tells whether
% the |\includeonly| mechanism is used (false) or
% the selection of child files must be performed manually (true).
% The definitions initialise to false:
%    \begin{macrocode}
\newif\ifchilddoc
\newif\ifchilddocmanual
%    \end{macrocode}

% \macro{\childdocname}
% \macro{\childdocjob}
% The macro |\childdocname| stores the name of the main document
% to be compiled. The macro |\childdocjob| stores the name of
% the document on which the \LaTeX{} compiler was originally invoked.
% The content of |\jobname| cannot be compared
% to filenames specified in the source due to different catcodes.
% The following code rescans |\jobname|, stores the result
% in |\childdocname| and saves a copy in |\childdocjob|:
%    \begin{macrocode}
\edef\childdocname{\scantokens\expandafter{\jobname\noexpand}}
\let\childdocjob\childdocname
%    \end{macrocode}

% \macro{\childdocdisable}
% The macro |\childdocdisable| prevents the main file
% from being processed more than once.
% At this stage, the main document command |\childdocmain|
% is assumed to be called once again where it should do nothing.
% Any subsequent call to it should prevent
% a secondary processing of the main document
% It overwrites the forwarding commands
% |\childdocof| and |\childdocforward|
% with empty macros to prevent further inclusions of the main document:
%    \begin{macrocode}
\newcommand{\childdocdisable}
{
  \renewcommand{\childdocmain}[1]{\renewcommand{\childdocmain}[1]{\endinput}}
  \renewcommand{\childdocof}[1]{}
  \renewcommand{\childdocby}[2][]{}
  \renewcommand{\childdocforward}[2][]{}
  \renewcommand{\childdocdisable}{}
}
%    \end{macrocode}

% \macro{\childdocmain}
% The macro |\childdocmain| is to be called at the top of the main file
% with nothing or the main filename (without extension) as argument.
% First, it breaks loops.
% If the argument is not empty and does not match |\childdocname|
% (which is set by the first inclusion of |childdoc.def|),
% |\ifchilddoc| is set to true, |\includeonly| is applied to the child file
% and |\jobname| is set to the main file
% (for proper handling of |.aux| files):
%    \begin{macrocode}
\newcommand{\childdocmain}[1]
{
  \childdocdisable\childdocmain{}
  \if?#1?\else
    \begingroup
      \def\childdoctmp{#1}
      \ifx\childdoctmp\childdocname
        \def\childdoctmp{}
      \else
        \def\childdoctmp
        {
          \childdoctrue
          \includeonly{\childdocname}
          \def\childdocjob{#1}
          \def\jobname{#1}
        }
      \fi
      \expandafter
    \endgroup
    \childdoctmp
  \fi
}
%    \end{macrocode}

% \macro{\childdocof}
% The command |\childdocof| redirects
% compilation to the main file |#1|.
%    \begin{macrocode}
\newcommand{\childdocof}[1]
{
  \childdocdisable
  \childdoctrue
  \includeonly{\childdocname}
  \def\jobname{#1}
  \def\childdocjob{#1}
  \input{#1}
}
%    \end{macrocode}

% \macro{\childdocby}
% The command |\childdocby| ....
%    \begin{macrocode}
\newcommand{\childdocby}[2][]
{
  \childdocdisable
  \childdoctrue
  \childdocmanualtrue
  \if?#1?\else
    \def\jobname{#2}
  \fi
  \def\childdocjob{#2}
  \input{#2}
  \endinput
}
%    \end{macrocode}

% \macro{\childdocforward}
% The command |\childdocforward| redirects
% compilation to the main file or
% (if the optional argument is given) a child file.
% Parameters are set as if the main file
% or a child file starting with |\childdocof| was compiled.
% Then compilation is handed over to the main file:
%    \begin{macrocode}
\newcommand{\childdocforward}[2][]
{
  \begingroup
    \if?#1?
      \def\childdoctmp
      {
        \def\childdocname{#2}
        \def\childdocjob{#2}
        \def\jobname{#2}
        \input{#2}
        \endinput
      }
    \else
      \def\childdoctmp
      {
        \childdocdisable
        \def\childdocname{#2}
        \childdoctrue
        \includeonly{#2}
        \def\childdocjob{#1}
        \def\jobname{#1}
        \input{#1}
        \endinput
      }
    \fi
    \expandafter
  \endgroup
  \childdoctmp
}
%    \end{macrocode}

% \macro{\childdocforwardprefix}
% The command |\childdocforwardprefix| redirects
% compilation to the main or a child file by means of a pattern.
% The prefix |#1| in the current filename is replaced by |#2|
% and the suffix of the current filename is kept
% (it is assumed that the filename does not contain the substring `|~~~|'
% which is used as a delimiter).
% Compilation is handed over to the new file by |\childdocforward|:
%    \begin{macrocode}
\newcommand{\childdocforwardprefix}[3][]
{
  \begingroup
    \def\childdocextract #2##1~~~{\def\childdoctmp{\childdocforward[#1]{#3##1}}}
    \expandafter\childdocextract\childdocname~~~
    \expandafter
  \endgroup
  \childdoctmp
}
%    \end{macrocode}

% \macro{\childdoc}
% The deprecated macro |\childdoc| is a legacy version of |\childdocmain|:
%    \begin{macrocode}
\newcommand{\childdoc}{\childdocmain}
%    \end{macrocode}

% \macro{\childdocredirect}
% The deprecated macro |\childdocredirect| is a legacy version
% of |\childdocforward| and |\childdocforwardprefix|:
%    \begin{macrocode}
\newcommand{\childdocredirect}[2][]
{
  \begingroup
    \if?#1?
      \def\childdoctmp{\childdocforward{#2}}
    \else
      \def\childdoctmp{\childdocforwardprefix{#1}{#2}}
    \fi
    \expandafter
  \endgroup
  \childdoctmp
}
%    \end{macrocode}

%\iffalse
%</package>
%\fi
%
\endinput
|\\
|\childdocforward[|\textit{main}|]{|\textit{dest}|}|\\
\end{tabular}
\end{center}
%
The argument \textit{dest} is the destination file
(without extension).
It should be the main file or one of the child files.
Note that further \textsf{childdoc} directives
such as |\childdocof| and |\childdocforward|
in the indicated file will be processed in this form.
The optional argument \textit{main}
passes on directly to the main file \textit{main}
while pretending to compile the child \textit{dest}.
This form behaves as if \textit{dest}
issues |\childdocof{|\textit{main}|}| right away,
and no further \textsf{childdoc} directives will be processed.

%%%%%%%%%%%%%%%%%%%%%%%%%%%%%%%%%%%%%%%%
\DescribeMacro{\...prefix}
In the alternative form |\childdocforwardprefix|,
%
\begin{center}
\begin{tabular}{l}
|% \iffalse
%
% childdoc.dtx Copyright (C) 2017-2018 Niklas Beisert
%
% This work may be distributed and/or modified under the
% conditions of the LaTeX Project Public License, either version 1.3
% of this license or (at your option) any later version.
% The latest version of this license is in
%   http://www.latex-project.org/lppl.txt
% and version 1.3 or later is part of all distributions of LaTeX
% version 2005/12/01 or later.
%
% This work has the LPPL maintenance status `maintained'.
%
% The Current Maintainer of this work is Niklas Beisert.
%
% This work consists of the files childdoc.dtx and childdoc.ins
% and the derived files childdoc.def and cdocsamp.tex with
% cdocsch1.tex, cdocsch2.tex, cdocsdrf.tex, cdocsfn1.tex, cdocsfn2.tex.
%
%<package>\ifdefined\childdocmain\endinput\fi
%<package>\ProvidesFile{childdoc.def}[2018/12/30 v2.0 child document driver]
%<samplemain>\ProvidesFile{cdocsamp.tex}[2018/12/30 v2.0 sample for childdoc]
%<*driver>
%\ProvidesFile{childdoc.drv}[2018/12/30 v2.0 childdoc reference manual file]
\PassOptionsToClass{10pt,a4paper}{article}
\documentclass{ltxdoc}

\usepackage[margin=35mm]{geometry}
\usepackage{hyperref}
\usepackage{hyperxmp}
\usepackage[usenames]{color}

\hypersetup{colorlinks=true}
\hypersetup{pdfstartview=FitH}
\hypersetup{pdfpagemode=UseNone}
\hypersetup{pdfsource={}}
\hypersetup{pdflang={en-UK}}
\hypersetup{pdfcopyright={Copyright 2017-2018 Niklas Beisert.
  This work may be distributed and/or modified under the
  conditions of the LaTeX Project Public License, either version 1.3
  of this license or (at your option) any later version.}}
\hypersetup{pdflicenseurl={http://www.latex-project.org/lppl.txt}}
\hypersetup{pdfcontactaddress={ETH Zurich, ITP, HIT K,
  Wolfgang-Pauli-Strasse 27}}
\hypersetup{pdfcontactpostcode={8093}}
\hypersetup{pdfcontactcity={Zurich}}
\hypersetup{pdfcontactcountry={Switzerland}}
\hypersetup{pdfcontactemail={nbeisert@itp.phys.ethz.ch}}
\hypersetup{pdfcontacturl={http://people.phys.ethz.ch/\xmptilde nbeisert/}}

\newcommand{\secref}[1]{\hyperref[#1]{section \ref*{#1}}}

\parskip1ex
\parindent0pt
\let\olditemize\itemize
\def\itemize{\olditemize\parskip0pt}

\begin{document}

\title{The \textsf{childdoc} Package}
\hypersetup{pdftitle={The childdoc Package}}
\author{Niklas Beisert\\[2ex]
  Institut f\"ur Theoretische Physik\\
  Eidgen\"ossische Technische Hochschule Z\"urich\\
  Wolfgang-Pauli-Strasse 27, 8093 Z\"urich, Switzerland\\[1ex]
  \href{mailto:nbeisert@itp.phys.ethz.ch}
  {\texttt{nbeisert@itp.phys.ethz.ch}}}
\hypersetup{pdfauthor={Niklas Beisert}}
\hypersetup{pdfsubject={Manual for the LaTeX2e Package childdoc}}
\date{30 December 2018, \textsf{v2.0}}
\maketitle

\begin{abstract}\noindent
\textsf{childdoc} is a \LaTeXe{} package
that enables the direct compilation
of document sections included by |\include|
to individual files.
\end{abstract}

\begingroup
\parskip0ex
\tableofcontents
\endgroup

%%%%%%%%%%%%%%%%%%%%%%%%%%%%%%%%%%%%%%%%%%%%%%%%%%%%%%%%%%%%%%%%%%%%%%%%%%%%%%%%
%%%%%%%%%%%%%%%%%%%%%%%%%%%%%%%%%%%%%%%%%%%%%%%%%%%%%%%%%%%%%%%%%%%%%%%%%%%%%%%%
\section{Introduction}

\LaTeX{} provides a mechanism to structure a large document (such as a book)
into a main file and several child files (containing the chapters)
using the |\include| command.
This mechanism is beneficial for documents
which span hundreds of pages in order to
make the source file(s) more manageable.
Moreover, compilation can be restricted to
selected child files by means of the |\includeonly| command.
The latter feature can be used to reduce the compilation time while editing
(this was significantly more useful in the earlier days of \LaTeX{})
or to generate a smaller document which is easier to navigate.
Another application of |\includeonly| is to generate
documents consisting of selected parts of the complete document.

However, there are a few drawbacks of the plain |\include| mechanism:
\begin{itemize}
\item
The child files cannot be compiled on their own,
they can only be compiled via the main file.
A naive editing environment
(such as a text editor with an option
to have the current file processed by \LaTeX)
may require one to switch to the main file before compiling;
attempting to compile the child file produces errors.
\item
The main file must be modified (each time)
to adjust the |\includeonly| command
to the present needs. This easily leaves the main file in a messy state.
\item
The generated document will always carry the filename
of the main document. This is inconvenient if
several child files are to be compiled and
to be kept for distribution.
\end{itemize}

The present package provides a simple interface
to make child files individually compilable by \LaTeX{}.
Compiling a child file then has the same effect as compiling
the main file with an |\includeonly| command
to select the appropriate child.
Moreover the generated document will carry the name of the child
rather than the main file.
This resolves all three above issues.

This feature is meant to make the editing of books,
thesis documents and lecture notes somewhat more convenient.
However, the package can also be used efficiently for
composing a series of documents (such as exercise sheets)
which are typically distributed individually.
It then assists the author in generating the individual documents
(potentially in different versions)
as well as a document containing the collected series.
Another application is in developing style files
or other kinds of included material
where compilation of the style file could redirect
to a sample or test file.

%%%%%%%%%%%%%%%%%%%%%%%%%%%%%%%%%%%%%%%%%%%%%%%%%%%%%%%%%%%%%%%%%%%%%%%%%%%%%%%%
%%%%%%%%%%%%%%%%%%%%%%%%%%%%%%%%%%%%%%%%%%%%%%%%%%%%%%%%%%%%%%%%%%%%%%%%%%%%%%%%
\section{Usage}

First of all, the package \textsf{childdoc} is \emph{not} a standard
\LaTeXe{} |.sty| style file! Therefore it needs to be invoked in
a non-standard way.

%%%%%%%%%%%%%%%%%%%%%%%%%%%%%%%%%%%%%%%%%%%%%%%%%%%%%%%%%%%%%%%%%%%%%%%%%%%%%%%%
\subsection{Included Files}
\label{sec:include}

%%%%%%%%%%%%%%%%%%%%%%%%%%%%%%%%%%%%%%%%
\DescribeMacro{\childdocmain}
To use the package, add the commands
\begin{center}
\begin{tabular}{l}
|\input{childdoc.def}|\\
|\childdocmain{}|\\
\end{tabular}
\end{center}
at the very top of the main \LaTeX{} file,
in particular \emph{before} the |\documentclass| statement!
The argument of |\childdocmain| should be left empty
(but it must be present).

%%%%%%%%%%%%%%%%%%%%%%%%%%%%%%%%%%%%%%%%
\DescribeMacro{\childdocof}
Furthermore, add the commands
\begin{center}
\begin{tabular}{l}
|\input{childdoc.def}|\\
|\childdocof{|\textit{main}|}|\\
\end{tabular}
\end{center}
at the top of every child file \textit{child}
which is included by |\include{|\textit{child}|}|
from within the main file
(or at least for those files to be compiled individually).
The argument \textit{main} must be the filename of the main file.

There are a couple of
considerations in setting up the main and child documents:

%%%%%%%%%%%%%%%%%%%%%%%%%%%%%%%%%%%%%%%%
\paragraph{Restrictions.}

Please note the following restrictions:
\begin{itemize}
\item
|\childdocmain| must be called with one argument \textit{main}
to ensure compatibility with earlier version of the package.
It must either be empty (|\childdocmain{}|)
or precisely match the filename of the main file in which it is specified.
See \secref{sec:detection} for further information.
\item
The filename \textit{main} must be specified without the |.tex| extension.
\item
The filename \textit{main} is case sensitive
(even in case-insensitive file systems)
due to internal string comparison.
\item
The argument \textit{main} should be fully expanded, it cannot be a macro.
\item
Subdirectories and special characters should be avoided in filenames.
\item
The command |\childdocmain{|\textit{main}|}| must be followed by a whitespace.
It should not be followed immediately by another command
or by a comment mark `|%|'.
This is because the \TeX{} parser reads the token immediately following
the argument of |\childdocmain| and puts it
at the beginning of every child section;
however, a white\-space is ignored.
\end{itemize}

%%%%%%%%%%%%%%%%%%%%%%%%%%%%%%%%%%%%%%%%
\paragraph{Content of Main File.}

It is advisable to place all content in the child files included by |\include|.
Any output contained in the main file will appear in all child documents
unless suppressed manually;
it cannot be suppressed automatically by the |\includeonly| directive
and thus should normally be avoided.
A method to include some content in the main file
by means of conditional processing is described in \secref{sec:conditional}.

%%%%%%%%%%%%%%%%%%%%%%%%%%%%%%%%%%%%%%%%
\paragraph{Page Numbering.}

When only a part of the document is compiled,
the appropriate numbering of pages
(as well as other status parameters)
is determined from the |.aux| files.
The latter contain information from previous passes.
However this information needs to propagate through
all intermediate child documents.
Therefore the page numbering in child documents may well
be inconsistent until the complete document is compiled at least once.

A useful (if unconventional) way to always ensure a consistent
page numbering is to restart the numbering in each child document
and denote the pages by `\textit{child}|.|\textit{page}'
where \textit{child} represents the chapter/section number of the child file.
This can be achieved by the command
|\numberwithin{page}{|\textit{child}|}|
of the \textsf{amsmath} package
where \textit{child} can be |chapter| or |section|
depending on the chosen structuring.
Alternatively, one can modify the macro |\thepage| appropriately
and reset the counter |page| at the start of each child file.

%%%%%%%%%%%%%%%%%%%%%%%%%%%%%%%%%%%%%%%%%%%%%%%%%%%%%%%%%%%%%%%%%%%%%%%%%%%%%%%%
\subsection{Conditional Processing}
\label{sec:conditional}

The package provides a mechanism to compile different versions
of a document. To customise the versions further some conditional processing
can come in handy to distinguish which version is being compiled.
The package provides two macros to describe the compilation context:

%%%%%%%%%%%%%%%%%%%%%%%%%%%%%%%%%%%%%%%%
\DescribeMacro{\ifchilddoc}
The conditional |\ifchilddoc| distinguishes between the compilation of
child documents and the main document:
%
\begin{center}
|\ifchilddoc |\textit{child-code}| |[|\||else |\textit{main-code}]| \||fi|
\end{center}

%%%%%%%%%%%%%%%%%%%%%%%%%%%%%%%%%%%%%%%%
\DescribeMacro{\childdocname}
\DescribeMacro{\childdocjob}
The macro |\childdocname| contains the filename (without extension)
of the main or child file being processed.
Note that |\childdocjob| will always contain the name of the main file.

%%%%%%%%%%%%%%%%%%%%%%%%%%%%%%%%%%%%%%%%
\paragraph{Title Page.}

Conditional processing can be used to include a title or banner page
in the main document when proper precautions are taken.
Importantly, the code in the main file should ensure that the page counter
(as well as other status parameters which are stored in the |.aux| files)
takes the same value after the conditional processing.
Otherwise the page numbers may take divergent values
depending on which part is compiled.

For example, a title page could be declared by:
%
\begin{center}
\begin{tabular}{l}
|\ifchilddoc\||else|\\
|\addtocounter{page}{-1}|\\
\textit{code for title page}\\
|\newpage|\\
|\||fi|
\end{tabular}
\end{center}
%
A banner page for the child documents can be generated by:
%
\begin{center}
\begin{tabular}{l}
|\ifchilddoc|\\
|\addtocounter{page}{-1}|\\
\textit{code for banner page}\\
|\newpage|\\
|\||fi|
\end{tabular}
\end{center}
%
Here one could write a message such as:
\begin{center}
|This is the part \childdocname{} of \childdocjob{}.|
\end{center}

%%%%%%%%%%%%%%%%%%%%%%%%%%%%%%%%%%%%%%%%%%%%%%%%%%%%%%%%%%%%%%%%%%%%%%%%%%%%%%%%
\subsection{Flags}
\label{sec:flags}

The package makes it easy to generate different versions
of the main or child documents.
To this end compilation flags can be defined
and assigned different default values.
They will be particularly useful in conjunction
with the forwarding mechanism described in \secref{sec:forward}.

For example, it may be useful to have a flag |\version|
which can be set to |draft| or |final|.
The document source will contain some conditional code
depending on the value of |\version|.
Suppose further, the flag should default to |final| for the main file
and to |draft| for child files
which is a natural assignment for editing the document.
This is achieved by placing the following code
in the preamble of the main document
(below the |\childdocmain| directive):
%
\begin{center}
\begin{tabular}{l}
|\ifchilddoc|\\
|\providecommand{\version}{draft}|\\
|\||else|\\
|\providecommand{\version}{final}|\\
|\||fi|
\end{tabular}
\end{center}
%
The definition by |\providecommand| makes sure
that previous definitions are not overwritten.
Further statements |\providecommand{\version}{...}|
can thus be added before the above code to override it.

For the main file, one might add a line
(between |\childdocmain| and the above block)
%
\begin{center}
|%\ifchilddoc\||else\providecommand{\version}{draft}\||fi|
\end{center}
%
which can be uncommented to produce a draft version.
Likewise one can add a line to the very top of a child file
(above the |\childdocof{|\textit{main}|}| directive)
%
\begin{center}
|%\providecommand{\version}{final}|
\end{center}
%
which can be uncommented to produce the final version of this child document.

%%%%%%%%%%%%%%%%%%%%%%%%%%%%%%%%%%%%%%%%%%%%%%%%%%%%%%%%%%%%%%%%%%%%%%%%%%%%%%%%
\subsection{Forwarding}
\label{sec:forward}

Different versions of the main or child documents
using compilation flags as described in \secref{sec:flags}
can be (permanently) stored in different files
for convenient compilation, viewing and distribution.
To this end, the package defines a command
to pass on compilation to a different file:

%%%%%%%%%%%%%%%%%%%%%%%%%%%%%%%%%%%%%%%%
\DescribeMacro{\childdocforward}
The command |\childdocforward| redirects processing to
another source file:
%
\begin{center}
\begin{tabular}{l}
|\input{childdoc.def}|\\
|\childdocforward[|\textit{main}|]{|\textit{dest}|}|\\
\end{tabular}
\end{center}
%
The argument \textit{dest} is the destination file
(without extension).
It should be the main file or one of the child files.
Note that further \textsf{childdoc} directives
such as |\childdocof| and |\childdocforward|
in the indicated file will be processed in this form.
The optional argument \textit{main}
passes on directly to the main file \textit{main}
while pretending to compile the child \textit{dest}.
This form behaves as if \textit{dest}
issues |\childdocof{|\textit{main}|}| right away,
and no further \textsf{childdoc} directives will be processed.

%%%%%%%%%%%%%%%%%%%%%%%%%%%%%%%%%%%%%%%%
\DescribeMacro{\...prefix}
In the alternative form |\childdocforwardprefix|,
%
\begin{center}
\begin{tabular}{l}
|\input{childdoc.def}|\\
|\childdocforwardprefix[|\textit{main}|]{|\textit{prefix}|}{|\textit{dest}|}|
\end{tabular}
\end{center}
%
the destination file is determined by a pattern
depending on the current file:
To make this work, the current file must be called
`{\textit{prefix}\hspace{0.2em}\textit{suffix}}'
with \textit{prefix} matching precisely the argument.
Processing is then passed on to the file
`{\textit{dest}\hspace{0.2em}\textit{suffix}}'.
Surely, the same effect is achieved by
directly specifying the
argument `{\textit{dest}\hspace{0.2em}\textit{suffix}}'
in the first form.
However, that requires to set up a different file
for each child. With the alternative form of the command
all these files can have exactly the same content
which simplifies setting them up and maintaining them.

For example, the following file |draft.tex|
with a compilation flag |\version| as described in \secref{sec:flags}
compiles the main document as a draft:
%
\begin{center}
\begin{tabular}{l}
|\def\version{draft}|\\
|\input{childdoc.def}|\\
|\childdocforward{|\textit{main}|}|
\end{tabular}
\end{center}
%
Likewise, the following files |final|\textit{nn}|.tex|
compile the final version of the child document
|child|\textit{nn}|.tex|:
%
\begin{center}
\begin{tabular}{l}
|\def\version{final}|\\
|\input{childdoc.def}|\\
|\childdocforwardprefix{final}{child}|
\end{tabular}
\end{center}
%

Note that when several versions of a main file and/or of each child file
are to be generated, it may be convenient to set up a |Makefile| or
shell script to automatise the process.

%%%%%%%%%%%%%%%%%%%%%%%%%%%%%%%%%%%%%%%%%%%%%%%%%%%%%%%%%%%%%%%%%%%%%%%%%%%%%%%%
\subsection{Command Line Processing}
\label{sec:commandline}

The effect of redirection files can also be achieved by invoking
the \LaTeX{} compiler with a more elaborate command line.
Most conveniently this should be done as part
of a shell script or a |Makefile|.

When using \textsf{childdoc} in the main file, the following
command lines effectively perform a redirection
(note that depending on the shell being used,
backslashes may have to be doubled: `|\|' $\to$ `|\\|'):
%
\begin{center}
|... -jobname "|\textit{target}|" |\\|"|[\textit{flags}]%
|\input{childdoc.def}\childdocforward[|\textit{main}|]{|\textit{dest}|}"|
\end{center}
%
Here \textit{target} is the name of the output file,
\textit{main} is the name of the main file
and \textit{dest} is the name of the main or child file to be processed
(all filenames without extensions).
The optional argument \textit{main} can be omitted
if \textit{main} matches \textit{dest}.
Optionally, compilation \textit{flags} can be defined via |\def| commands.
This command line makes the \TeX{} engine believe
it is compiling the file \textit{target}
whose content is specified as the latter parameter.
The provided code then forwards the processing to
\textit{main} or \textit{dest} as described in \secref{sec:forward}.

%%%%%%%%%%%%%%%%%%%%%%%%%%%%%%%%%%%%%%%%%%%%%%%%%%%%%%%%%%%%%%%%%%%%%%%%%%%%%%%%
\subsection{Include by Input}
\label{sec:input}

Including child documents by |\include| has some restrictions by design.
Most notably, the content of a child document always occupies
its own set of pages; pages cannot be shared between child documents.
Usually, this behaviour makes perfect sense
because each child document contain an essential part of the document.
However, in some situations it may be desirable to compose
a document from a collection of parts
without having mandatory page breaks between then.
For this case, the package
provides a mechanism to include parts
by |\input| which can also be processed individually.
However, by construction this mechanism
requires manual handling of the content to be output.

%%%%%%%%%%%%%%%%%%%%%%%%%%%%%%%%%%%%%%%%
\DescribeMacro{\ifchilddocmanual}
The main file should be prepared as usual, see \secref{sec:include}.
However, the document body must make a distinction
between processing of an individual part and of the main document, e.g.:
%
\begin{center}
\begin{tabular}{l}
|\ifchilddocmanual|\\
|\input{\childdocname}|\\
|\||else|\\
\textit{document body with }|\input{|\textit{part}|}|\\
|\||fi|
\end{tabular}
\end{center}
%
The conditional |\ifchilddocmanual| is true whenever
a part to be included by |\input| is being compiled,
and the name of the part is stored in |\childdocname|.

%%%%%%%%%%%%%%%%%%%%%%%%%%%%%%%%%%%%%%%%
\DescribeMacro{\childdocby}
Each part to be included by |\input| should start with:
%
\begin{center}
\begin{tabular}{l}
|\input{childdoc.def}|\\
|\childdocby{|\textit{main}|}|\\
\end{tabular}
\end{center}
%
The directive |\childdocby| is similar to |\childdocof|
described in \secref{sec:include},
but the subsequent selection of content must be done manually.
To that end, both |\ifchilddoc| and |\ifchilddocmanual|
will be true upon processing of a part,
and the name of the part is stored in |\childdocname|.
Note that |\jobname| will be set to the filename of the current part
so that each part receives an individual |.aux| file
that does not interfere with the |.aux| file(s) of the main document.
This behaviour can be altered by the alternative form
|\childdocby[*]{|\textit{main}|}| (with a non-empty optional argument)
which uses the |.aux| file of the main document
by setting |\jobname| to \textit{main}.

%%%%%%%%%%%%%%%%%%%%%%%%%%%%%%%%%%%%%%%%%%%%%%%%%%%%%%%%%%%%%%%%%%%%%%%%%%%%%%%%
\subsection{Driver Development}
\label{sec:driver}

The \textsf{childdoc} mechanism can also be use for the development
of definition files such as \LaTeX{} styles or classes.
This case differs from the above setup with multiple parts
included by |\include| in that no |\includeonly| should be invoked.
This can be achieved by starting the include file
(before |\ProvidesPackage|) with:
%
\begin{center}
\begin{tabular}{l}
|\input{childdoc.def}|\\
|\childdocforward{|\textit{main}|}|\\
\end{tabular}
\end{center}
%
or alternatively with:
%
\begin{center}
\begin{tabular}{l}
|\input{childdoc.def}|\\
|\childdocby{|\textit{main}|}|\\
\end{tabular}
\end{center}
%
Both forms have slightly different effects as described above.
The main file is prepared as usual, see \secref{sec:include}.

%%%%%%%%%%%%%%%%%%%%%%%%%%%%%%%%%%%%%%%%%%%%%%%%%%%%%%%%%%%%%%%%%%%%%%%%%%%%%%%%
\subsection{Legacy Detection}
\label{sec:detection}

The directive |\childdocmain| in the main file can detect
whether the complete document or merely a child is to be compiled
even without using the directive |\childdocof|.
This method is deprecated because it is less robust
and there is no compelling reason to use it;
it is merely provided for backward compatibility
and it may be removed in future versions.

If the detection mechanism is to be used,
it is mandatory to correctly specify
the filename of the main file as the argument of |\childdocmain|:
%
\begin{center}
\begin{tabular}{l}
|\input{childdoc.def}|\\
|\childdocmain{|\textit{main}|}|\\
\end{tabular}
\end{center}
%
If |\jobname| does not match the argument \textit{main} of |\childdocmain|,
it is assumed that |\jobname| points to the child file to be compiled.
When using |\childdocmain| with the main file specified as argument,
it suffices to start a child file
with just |\input{|\textit{main}|}|
without loading of the package and using |\childdocof|.
If instead all processing is done
with the appropriate \textsf{childdoc} directives,
the argument of \textit{main} of |\childdocmain| can be empty.

An alternative version of the command line processing described
in \secref{sec:commandline} using the detection mechanism reads:
%
\begin{center}
|... -jobname "|\textit{target}|" "|[\textit{flags}]%
[|\def\jobname{|\textit{dest}|}|]|\input{|\textit{main}|}"|
\end{center}

%%%%%%%%%%%%%%%%%%%%%%%%%%%%%%%%%%%%%%%%%%%%%%%%%%%%%%%%%%%%%%%%%%%%%%%%%%%%%%%%
\subsection{Manual Code}
\label{sec:manual}

In case one cannot be certain whether the definitions file |childdoc.def|
is installed on the target \TeX{} distribution
and one prefers not to ship it,
it is conceivable to paste a few relevant commands into the sources.

To that end, drop all statements |\input{childdoc.def}|
and perform the replacements as outlined below.
Instead of |\childdocmain{|\textit{main}|}| add the following code
to the top of the main file:
%
\begin{center}
\begin{tabular}{l}
|\||ifdefined\childdocname\endinput\||fi\newif\ifchilddoc|\\
|\edef\childdocname{\scantokens\expandafter{\jobname\noexpand}}|\\
|\def\childdocmain{|\textit{main}|}\||ifx\childdocmain\childdocname\||else|\\
|\childdoctrue\includeonly{\childdocname}\let\jobname\childdocmain\||fi|\\
\end{tabular}
\end{center}
%
Instead of |\childdocof{|\textit{main}|}| just include the main file
at the top of each child file:
%
\begin{center}
|\input{|\textit{main}|}|
\end{center}
%
A simple redirection |\childdocforward{|\textit{dest}|}| is achieved by:
%
\begin{center}
|\def\jobname{|\textit{dest}|}\input{\jobname}|
\end{center}
%
The redirection with prefix
|\childdocforwardprefix[|\textit{prefix}|]{|\textit{dest}|}|
is accomplished by:
%
\begin{center}
\begin{tabular}{l}
|{\edef\jobname{\scantokens\expandafter{\jobname\noexpand}}|\\
|\def\redirectjob |\textit{prefix}|#1~~~{\gdef\jobname{|\textit{dest}|#1}}|\\
|\expandafter\redirectjob\jobname~~~}\input{\jobname}|
\end{tabular}
\end{center}

In an alternative approach,
child documents can be compiled by a specific command line
without additional code or specific definitions:
%
\begin{center}
|... -jobname "|\textit{target}|" "|[\textit{flags}]%
|\includeonly{|\textit{dest}|}\input{|\textit{main}|}"|
\end{center}
%

%%%%%%%%%%%%%%%%%%%%%%%%%%%%%%%%%%%%%%%%%%%%%%%%%%%%%%%%%%%%%%%%%%%%%%%%%%%%%%%%
%%%%%%%%%%%%%%%%%%%%%%%%%%%%%%%%%%%%%%%%%%%%%%%%%%%%%%%%%%%%%%%%%%%%%%%%%%%%%%%%
\section{Information}

%%%%%%%%%%%%%%%%%%%%%%%%%%%%%%%%%%%%%%%%%%%%%%%%%%%%%%%%%%%%%%%%%%%%%%%%%%%%%%%%
\subsection{Copyright}

Copyright \copyright{} 2017--2018 Niklas Beisert

This work may be distributed and/or modified under the
conditions of the \LaTeX{} Project Public License, either version 1.3
of this license or (at your option) any later version.
The latest version of this license is in
  \url{http://www.latex-project.org/lppl.txt}
and version 1.3 or later is part of all distributions of \LaTeX{}
version 2005/12/01 or later.

This work has the LPPL maintenance status `maintained'.

The Current Maintainer of this work is Niklas Beisert.

This work consists of the files |README.txt|, |childdoc.ins| and |childdoc.dtx|
as well as the derived files |childdoc.def|, |cdocsamp.tex|
with |cdocsch1.tex|, |cdocsch2.tex|, |cdocspt3.tex|, |cdocspt4.tex|,
|cdocsdrf.tex|, |cdocsfn1.tex|, |cdocsfn2.tex|
as well as |childdoc.pdf|.

%%%%%%%%%%%%%%%%%%%%%%%%%%%%%%%%%%%%%%%%%%%%%%%%%%%%%%%%%%%%%%%%%%%%%%%%%%%%%%%%
\subsection{Files and Installation}

The package consists of the files:
%
\begin{center}
\begin{tabular}{ll}
    |README.txt|   & readme file \\
    |childdoc.ins| & installation file \\
    |childdoc.dtx| & source file \\
    |childdoc.def| & definition file \\
    |cdocsamp.tex| & sample main file \\
    |cdocsch1.tex| & sample include file \\
    |cdocsch2.tex| & sample include file \\
    |cdocspt3.tex| & sample part file \\
    |cdocspt4.tex| & sample part file \\
    |cdocsdrf.tex| & sample redirection file \\
    |cdocsfn1.tex| & sample redirection file \\
    |cdocsfn2.tex| & sample redirection file \\
    |childdoc.pdf| & manual
\end{tabular}
\end{center}
%
The distribution consists of the files
|README.txt|, |childdoc.ins| and |childdoc.dtx|.
%
\begin{itemize}
\item
Run (pdf)\LaTeX{} on |childdoc.dtx|
to compile the manual |childdoc.pdf| (this file).
\item
Run \LaTeX{} on |childdoc.ins| to create the definitions file |childdoc.def|
and the sample |cdocsamp.tex| with include files
|cdocsch1.tex|, |cdocsch2.tex|, |cdocspt3.tex|, |cdocspt4.tex|,
|cdocsdrf.tex|, |cdocsfn1.tex|, |cdocsfn2.tex|.
Then copy the file |childdoc.def| to an appropriate directory of your \LaTeX{}
distribution, e.g.\ \textit{texmf-root}|/tex/latex/childdoc|.
\end{itemize}

%%%%%%%%%%%%%%%%%%%%%%%%%%%%%%%%%%%%%%%%%%%%%%%%%%%%%%%%%%%%%%%%%%%%%%%%%%%%%%%%
\subsection{Related CTAN Packages}

There are several other packages which offer a similar functionality:
%
\begin{itemize}
\item
The packages
\href{http://ctan.org/pkg/docmute}{\textsf{docmute}},
\href{http://ctan.org/pkg/includex}{\textsf{includex}} and
\href{http://ctan.org/pkg/standalone}{\textsf{standalone}}
provide commands to include only the document body of
a child file thus allowing both files to be compiled individually.
\item
The packages \href{http://ctan.org/pkg/subdocs}{\textsf{subdocs}}
and \href{http://ctan.org/pkg/subfiles}{\textsf{subfiles}}
provide structures in which the main and child documents can be
encapsulated and allowing them to be compiled individually.
The inclusion mechanism is different from the conventional |\include|.
\item
The package \href{http://ctan.org/pkg/combine}{\textsf{combine}}
is an elaborate solution to combine several documents into one.
\end{itemize}
%
See also the CTAN topic \href{http://ctan.org/topic/subdocs}{\textsf{subdocs}}
for further related packages.
The present package differs from the above solutions in that
a document structure constructed with the conventional |\include| mechanism
just needs two extra commands at the top of every file
such that all constituent files can be compiled individually.

%%%%%%%%%%%%%%%%%%%%%%%%%%%%%%%%%%%%%%%%%%%%%%%%%%%%%%%%%%%%%%%%%%%%%%%%%%%%%%%%
%\subsection{Feature Suggestions}
%
%The following is a list of features which may be useful for future
%versions of this package:
%%
%\begin{itemize}
%\item
%\ldots
%\end{itemize}

%%%%%%%%%%%%%%%%%%%%%%%%%%%%%%%%%%%%%%%%%%%%%%%%%%%%%%%%%%%%%%%%%%%%%%%%%%%%%%%%
\subsection{Revision History}

%%%%%%%%%%%%%%%%%%%%%%%%%%%%%%%%%%%%%%%%
\paragraph{v2.0:} 2018/12/30

\begin{itemize}
\item
immediate forward processing
\item
added |\childdocby| mechanism
\item
manual restructured
\end{itemize}

%%%%%%%%%%%%%%%%%%%%%%%%%%%%%%%%%%%%%%%%
\paragraph{v1.6:} 2018/01/17

\begin{itemize}
\item
application for development of include files
\item
corrections to manual
\end{itemize}

%%%%%%%%%%%%%%%%%%%%%%%%%%%%%%%%%%%%%%%%
\paragraph{v1.5:} 2017/05/21

\begin{itemize}
\item
more complete structuring introduced
\item
|\childdocof| introduced
\item
|\childdoc| renamed to |\childdocmain|
\item
|\childredirect| renamed to |\childdocforward| and |\childdocforwardprefix|
and functionality expanded
\end{itemize}

%%%%%%%%%%%%%%%%%%%%%%%%%%%%%%%%%%%%%%%%
\paragraph{v1.0:} 2017/04/27

\begin{itemize}
\item
manual and install package
\item
first version published on CTAN
\end{itemize}

%%%%%%%%%%%%%%%%%%%%%%%%%%%%%%%%%%%%%%%%
\paragraph{v0.6:} 2017/04/26

\begin{itemize}
\item
redirection mechanism added
\end{itemize}

%%%%%%%%%%%%%%%%%%%%%%%%%%%%%%%%%%%%%%%%
\paragraph{v0.5:} 2017/04/26

\begin{itemize}
\item
functionality in definition file
\end{itemize}


%%%%%%%%%%%%%%%%%%%%%%%%%%%%%%%%%%%%%%%%%%%%%%%%%%%%%%%%%%%%%%%%%%%%%%%%%%%%%%%%
%%%%%%%%%%%%%%%%%%%%%%%%%%%%%%%%%%%%%%%%%%%%%%%%%%%%%%%%%%%%%%%%%%%%%%%%%%%%%%%%
%%%%%%%%%%%%%%%%%%%%%%%%%%%%%%%%%%%%%%%%%%%%%%%%%%%%%%%%%%%%%%%%%%%%%%%%%%%%%%%%
\appendix

\settowidth\MacroIndent{\rmfamily\scriptsize 000\ }

 \DocInput{childdoc.dtx}

\end{document}
%</driver>
% \fi
%
% %%%%%%%%%%%%%%%%%%%%%%%%%%%%%%%%%%%%%%%%%%%%%%%%%%%%%%%%%%%%%%%%%%%%%%%%%%%%%%
% %%%%%%%%%%%%%%%%%%%%%%%%%%%%%%%%%%%%%%%%%%%%%%%%%%%%%%%%%%%%%%%%%%%%%%%%%%%%%%
% \section{Sample}
%\iffalse
%<*samplemain>
%\fi
%
% The following presents a sample document
% with two chapters, two parts, a title page,
% a compile flag as well as three forwarding files to set the flag.
% It consists of eight |.tex| files:
% \begin{center}
% \begin{tabular}{ll}
% |cdocsamp.tex|&main file\\
% |cdocsch1.tex|&include file for chapter 1\\
% |cdocsch2.tex|&include file for chapter 2\\
% |cdocspt3.tex|&include file for part 3\\
% |cdocspt4.tex|&include file for part 4\\
% |cdocsdrf.tex|&forwarding file for main file in draft mode\\
% |cdocsfi1.tex|&forwarding file for final version of chapter 1\\
% |cdocsfi2.tex|&forwarding file for final version of chapter 2\\
% \end{tabular}
% \end{center}
% Each of the eight files can be compiled directly by the \LaTeX{} compiler.
%
% %%%%%%%%%%%%%%%%%%%%%%%%%%%%%%%%%%%%%%
% \paragraph{Main File.}
%
% The main file is called |cdocsamp.tex|.
%
% Load the \textsf{childdoc} definitions and
% declare the filename for the main document:
%    \begin{macrocode}
\input{childdoc.def}
\childdocmain{}
%    \end{macrocode}

% Optional override for |\version| flag:
%    \begin{macrocode}
%%\ifchilddoc\else\providecommand{\version}{draft}\fi
%    \end{macrocode}

% Define the default values for the |\version| flag
% (|final| for the main file and |draft| for childs):
%    \begin{macrocode}
\ifchilddoc
\providecommand{\version}{draft}
\else
\providecommand{\version}{final}
\fi
%    \end{macrocode}

% Load the standard document class:
%    \begin{macrocode}
\documentclass[12pt]{article}
%    \end{macrocode}

% Start the document body:
%    \begin{macrocode}
\begin{document}
%    \end{macrocode}

% Declare a title page.
% Print title, part of document being processed and version flag:
%    \begin{macrocode}
\addtocounter{page}{-1}
\begin{center}
{\LARGE\bfseries{}childdoc example\par}
\vspace{1cm}
\ifchilddoc
\ifchilddocmanual part\else chapter\fi:
`\childdocname' of `\childdocjob'\par
\else
main document: `\childdocjob'\par
\fi
version: \version\par
\end{center}
\newpage
%    \end{macrocode}

% Manually include selected file,
% otherwise process as usual:
%    \begin{macrocode}
\ifchilddocmanual
\section*{part `\childdocname'}
\input{\childdocname}
\else
%    \end{macrocode}

% Include the two chapters:
%    \begin{macrocode}
\include{cdocsch1}
\include{cdocsch2}
%    \end{macrocode}

% Include the two parts unless only chapters should be displayed:
%    \begin{macrocode}
\ifchilddoc\else
\section{part three}
\input{cdocspt3}
\section{part four}
\input{cdocspt4}
\fi
%    \end{macrocode}

% Process as usual until here:
%    \begin{macrocode}
\fi
%    \end{macrocode}

% End of document body:
%    \begin{macrocode}
\end{document}
%    \end{macrocode}
%\iffalse
%</samplemain>
%\fi
%
% %%%%%%%%%%%%%%%%%%%%%%%%%%%%%%%%%%%%%%
% \paragraph{Chapter Include Files.}
%
% The include files are called |cdocsch1.tex| and |cdocsch2.tex|.
%
%\iffalse
%<*samplechap1|samplechap2>
%\fi

% Optional override for |\version| flag:
%    \begin{macrocode}
%%\providecommand{\version}{final}
%    \end{macrocode}

% Include the main document:
%    \begin{macrocode}
\input{childdoc.def}
\childdocof{cdocsamp}
%    \end{macrocode}

%\iffalse
%</samplechap1|samplechap2>
%\fi
%
%\iffalse
%<*samplechap1>
%\fi
% Some text for chapter 1:
%    \begin{macrocode}
\section{one}
some text in chapter one
%    \end{macrocode}

%\iffalse
%</samplechap1>
%\fi
% Some text for chapter 2:
%\iffalse
%<*samplechap2>
%\fi
%    \begin{macrocode}
\section{two}
more text in chapter two
%    \end{macrocode}

%\iffalse
%</samplechap2>
%\fi
%
% %%%%%%%%%%%%%%%%%%%%%%%%%%%%%%%%%%%%%%
% \paragraph{Part Include Files.}
%
% The include files are called |cdocspt3.tex| and |cdocspt4.tex|.
%
%\iffalse
%<*samplepart3|samplepart4>
%\fi

% Optional override for |\version| flag:
%    \begin{macrocode}
%%\providecommand{\version}{final}
%    \end{macrocode}

% Include the main document:
%    \begin{macrocode}
\input{childdoc.def}
\childdocby{cdocsamp}
%    \end{macrocode}

%\iffalse
%</samplepart3|samplepart4>
%\fi
%
%\iffalse
%<*samplepart3>
%\fi
% Some text for part 3:
%    \begin{macrocode}
some text in part three
%    \end{macrocode}

%\iffalse
%</samplepart3>
%\fi
% Some text for part 4:
%\iffalse
%<*samplepart4>
%\fi
%    \begin{macrocode}
more text in part four
%    \end{macrocode}

%\iffalse
%</samplepart4>
%\fi
%
% %%%%%%%%%%%%%%%%%%%%%%%%%%%%%%%%%%%%%%
% \paragraph{Forwarding for a Complete Draft.}
%
% The following forwarding file |cdocsdrf.tex|
% compiles the main document in draft mode:
%\iffalse
%<*sampledraft>
%\fi
%    \begin{macrocode}
\def\version{draft}
\input{childdoc.def}
\childdocforward{cdocsamp}
%    \end{macrocode}

%\iffalse
%</sampledraft>
%\fi
%
% %%%%%%%%%%%%%%%%%%%%%%%%%%%%%%%%%%%%%%
% \paragraph{Forwarding for Final Version of the Chapters.}
%
% The following forwarding files |cdocsfn1.tex| and |cdocsfn2.tex|
% (with identical content)
% compile the final versions of the child documents
% |cdocsch1.tex| and |cdocsch2.tex|, respectively:
%\iffalse
%<*samplefinal>
%\fi
%    \begin{macrocode}
\def\version{final}
\input{childdoc.def}
\childdocforwardprefix[cdocsamp]{cdocsfn}{cdocsch}
%    \end{macrocode}

%\iffalse
%</samplefinal>
%\fi
%
% %%%%%%%%%%%%%%%%%%%%%%%%%%%%%%%%%%%%%%
% \paragraph{Command Line Processing.}
%
% The following three command lines generate the output files
% |cdocscld|, |cdocscl1| and |cdocscl2|
% which should be identical to
% |cdocsdrf|, |cdocsch1| and |cdocsfn2|, respectively:
% \begin{center}
% \begin{tabular}{l}
% |latex -jobname cdocscld \|\\
% |  "\def\version{draft}\input{childdoc.def}\childdocforward{cdocsamp}"|\\
% |latex -jobname cdocscl1 \|\\
% |  "\input{childdoc.def}\childdocforward[cdocsamp]{cdocsch1}"|\\
% |latex -jobname cdocscl2 \|\\
% |  "\def\version{final}\input{childdoc.def}\childdocforward{cdocsch2}"|
% \end{tabular}
% \end{center}
% Note that the trailing backslash on each first line
% merely continues the input to the second line
% (for convenient cut ant paste).
% Furthermore, the command |latex| can be replaced by any
% of its alternative versions such as |pdflatex|.
%
% %%%%%%%%%%%%%%%%%%%%%%%%%%%%%%%%%%%%%%%%%%%%%%%%%%%%%%%%%%%%%%%%%%%%%%%%%%%%%%
% %%%%%%%%%%%%%%%%%%%%%%%%%%%%%%%%%%%%%%%%%%%%%%%%%%%%%%%%%%%%%%%%%%%%%%%%%%%%%%
% \section{Implementation}
%\iffalse
%<*package>
%\fi
%
% This section describes the definitions file |childdoc.def|.

% The definitions cannot be loaded using |\usepackage| or |\RequirePackage|
% which has a mechanism to prevent loading a style file more than once.
% When loading the definitions by means of |\input|
% multiple instances have to be prevented manually:
%\iffalse
%This code needs to be before the `\ProvidesFile' directive
%which is defined at the beginning of this file.
%Therefore it is also placed there and commented out here.
%</package>
%<*discard>
%\fi
%    \begin{macrocode}
\ifdefined\childdocmain\endinput\fi
%    \end{macrocode}
%\iffalse
%</discard>
%<*package>
%\fi
%
% \macro{\ifchilddoc}
% \macro{\ifchilddocmanual}
% The conditional |\ifchilddoc| tells whether a
% child (true) or main (false) document is being compiled.
% The conditional |\ifchilddocmanual| tells whether
% the |\includeonly| mechanism is used (false) or
% the selection of child files must be performed manually (true).
% The definitions initialise to false:
%    \begin{macrocode}
\newif\ifchilddoc
\newif\ifchilddocmanual
%    \end{macrocode}

% \macro{\childdocname}
% \macro{\childdocjob}
% The macro |\childdocname| stores the name of the main document
% to be compiled. The macro |\childdocjob| stores the name of
% the document on which the \LaTeX{} compiler was originally invoked.
% The content of |\jobname| cannot be compared
% to filenames specified in the source due to different catcodes.
% The following code rescans |\jobname|, stores the result
% in |\childdocname| and saves a copy in |\childdocjob|:
%    \begin{macrocode}
\edef\childdocname{\scantokens\expandafter{\jobname\noexpand}}
\let\childdocjob\childdocname
%    \end{macrocode}

% \macro{\childdocdisable}
% The macro |\childdocdisable| prevents the main file
% from being processed more than once.
% At this stage, the main document command |\childdocmain|
% is assumed to be called once again where it should do nothing.
% Any subsequent call to it should prevent
% a secondary processing of the main document
% It overwrites the forwarding commands
% |\childdocof| and |\childdocforward|
% with empty macros to prevent further inclusions of the main document:
%    \begin{macrocode}
\newcommand{\childdocdisable}
{
  \renewcommand{\childdocmain}[1]{\renewcommand{\childdocmain}[1]{\endinput}}
  \renewcommand{\childdocof}[1]{}
  \renewcommand{\childdocby}[2][]{}
  \renewcommand{\childdocforward}[2][]{}
  \renewcommand{\childdocdisable}{}
}
%    \end{macrocode}

% \macro{\childdocmain}
% The macro |\childdocmain| is to be called at the top of the main file
% with nothing or the main filename (without extension) as argument.
% First, it breaks loops.
% If the argument is not empty and does not match |\childdocname|
% (which is set by the first inclusion of |childdoc.def|),
% |\ifchilddoc| is set to true, |\includeonly| is applied to the child file
% and |\jobname| is set to the main file
% (for proper handling of |.aux| files):
%    \begin{macrocode}
\newcommand{\childdocmain}[1]
{
  \childdocdisable\childdocmain{}
  \if?#1?\else
    \begingroup
      \def\childdoctmp{#1}
      \ifx\childdoctmp\childdocname
        \def\childdoctmp{}
      \else
        \def\childdoctmp
        {
          \childdoctrue
          \includeonly{\childdocname}
          \def\childdocjob{#1}
          \def\jobname{#1}
        }
      \fi
      \expandafter
    \endgroup
    \childdoctmp
  \fi
}
%    \end{macrocode}

% \macro{\childdocof}
% The command |\childdocof| redirects
% compilation to the main file |#1|.
%    \begin{macrocode}
\newcommand{\childdocof}[1]
{
  \childdocdisable
  \childdoctrue
  \includeonly{\childdocname}
  \def\jobname{#1}
  \def\childdocjob{#1}
  \input{#1}
}
%    \end{macrocode}

% \macro{\childdocby}
% The command |\childdocby| ....
%    \begin{macrocode}
\newcommand{\childdocby}[2][]
{
  \childdocdisable
  \childdoctrue
  \childdocmanualtrue
  \if?#1?\else
    \def\jobname{#2}
  \fi
  \def\childdocjob{#2}
  \input{#2}
  \endinput
}
%    \end{macrocode}

% \macro{\childdocforward}
% The command |\childdocforward| redirects
% compilation to the main file or
% (if the optional argument is given) a child file.
% Parameters are set as if the main file
% or a child file starting with |\childdocof| was compiled.
% Then compilation is handed over to the main file:
%    \begin{macrocode}
\newcommand{\childdocforward}[2][]
{
  \begingroup
    \if?#1?
      \def\childdoctmp
      {
        \def\childdocname{#2}
        \def\childdocjob{#2}
        \def\jobname{#2}
        \input{#2}
        \endinput
      }
    \else
      \def\childdoctmp
      {
        \childdocdisable
        \def\childdocname{#2}
        \childdoctrue
        \includeonly{#2}
        \def\childdocjob{#1}
        \def\jobname{#1}
        \input{#1}
        \endinput
      }
    \fi
    \expandafter
  \endgroup
  \childdoctmp
}
%    \end{macrocode}

% \macro{\childdocforwardprefix}
% The command |\childdocforwardprefix| redirects
% compilation to the main or a child file by means of a pattern.
% The prefix |#1| in the current filename is replaced by |#2|
% and the suffix of the current filename is kept
% (it is assumed that the filename does not contain the substring `|~~~|'
% which is used as a delimiter).
% Compilation is handed over to the new file by |\childdocforward|:
%    \begin{macrocode}
\newcommand{\childdocforwardprefix}[3][]
{
  \begingroup
    \def\childdocextract #2##1~~~{\def\childdoctmp{\childdocforward[#1]{#3##1}}}
    \expandafter\childdocextract\childdocname~~~
    \expandafter
  \endgroup
  \childdoctmp
}
%    \end{macrocode}

% \macro{\childdoc}
% The deprecated macro |\childdoc| is a legacy version of |\childdocmain|:
%    \begin{macrocode}
\newcommand{\childdoc}{\childdocmain}
%    \end{macrocode}

% \macro{\childdocredirect}
% The deprecated macro |\childdocredirect| is a legacy version
% of |\childdocforward| and |\childdocforwardprefix|:
%    \begin{macrocode}
\newcommand{\childdocredirect}[2][]
{
  \begingroup
    \if?#1?
      \def\childdoctmp{\childdocforward{#2}}
    \else
      \def\childdoctmp{\childdocforwardprefix{#1}{#2}}
    \fi
    \expandafter
  \endgroup
  \childdoctmp
}
%    \end{macrocode}

%\iffalse
%</package>
%\fi
%
\endinput
|\\
|\childdocforwardprefix[|\textit{main}|]{|\textit{prefix}|}{|\textit{dest}|}|
\end{tabular}
\end{center}
%
the destination file is determined by a pattern
depending on the current file:
To make this work, the current file must be called
`{\textit{prefix}\hspace{0.2em}\textit{suffix}}'
with \textit{prefix} matching precisely the argument.
Processing is then passed on to the file
`{\textit{dest}\hspace{0.2em}\textit{suffix}}'.
Surely, the same effect is achieved by
directly specifying the
argument `{\textit{dest}\hspace{0.2em}\textit{suffix}}'
in the first form.
However, that requires to set up a different file
for each child. With the alternative form of the command
all these files can have exactly the same content
which simplifies setting them up and maintaining them.

For example, the following file |draft.tex|
with a compilation flag |\version| as described in \secref{sec:flags}
compiles the main document as a draft:
%
\begin{center}
\begin{tabular}{l}
|\def\version{draft}|\\
|% \iffalse
%
% childdoc.dtx Copyright (C) 2017-2018 Niklas Beisert
%
% This work may be distributed and/or modified under the
% conditions of the LaTeX Project Public License, either version 1.3
% of this license or (at your option) any later version.
% The latest version of this license is in
%   http://www.latex-project.org/lppl.txt
% and version 1.3 or later is part of all distributions of LaTeX
% version 2005/12/01 or later.
%
% This work has the LPPL maintenance status `maintained'.
%
% The Current Maintainer of this work is Niklas Beisert.
%
% This work consists of the files childdoc.dtx and childdoc.ins
% and the derived files childdoc.def and cdocsamp.tex with
% cdocsch1.tex, cdocsch2.tex, cdocsdrf.tex, cdocsfn1.tex, cdocsfn2.tex.
%
%<package>\ifdefined\childdocmain\endinput\fi
%<package>\ProvidesFile{childdoc.def}[2018/12/30 v2.0 child document driver]
%<samplemain>\ProvidesFile{cdocsamp.tex}[2018/12/30 v2.0 sample for childdoc]
%<*driver>
%\ProvidesFile{childdoc.drv}[2018/12/30 v2.0 childdoc reference manual file]
\PassOptionsToClass{10pt,a4paper}{article}
\documentclass{ltxdoc}

\usepackage[margin=35mm]{geometry}
\usepackage{hyperref}
\usepackage{hyperxmp}
\usepackage[usenames]{color}

\hypersetup{colorlinks=true}
\hypersetup{pdfstartview=FitH}
\hypersetup{pdfpagemode=UseNone}
\hypersetup{pdfsource={}}
\hypersetup{pdflang={en-UK}}
\hypersetup{pdfcopyright={Copyright 2017-2018 Niklas Beisert.
  This work may be distributed and/or modified under the
  conditions of the LaTeX Project Public License, either version 1.3
  of this license or (at your option) any later version.}}
\hypersetup{pdflicenseurl={http://www.latex-project.org/lppl.txt}}
\hypersetup{pdfcontactaddress={ETH Zurich, ITP, HIT K,
  Wolfgang-Pauli-Strasse 27}}
\hypersetup{pdfcontactpostcode={8093}}
\hypersetup{pdfcontactcity={Zurich}}
\hypersetup{pdfcontactcountry={Switzerland}}
\hypersetup{pdfcontactemail={nbeisert@itp.phys.ethz.ch}}
\hypersetup{pdfcontacturl={http://people.phys.ethz.ch/\xmptilde nbeisert/}}

\newcommand{\secref}[1]{\hyperref[#1]{section \ref*{#1}}}

\parskip1ex
\parindent0pt
\let\olditemize\itemize
\def\itemize{\olditemize\parskip0pt}

\begin{document}

\title{The \textsf{childdoc} Package}
\hypersetup{pdftitle={The childdoc Package}}
\author{Niklas Beisert\\[2ex]
  Institut f\"ur Theoretische Physik\\
  Eidgen\"ossische Technische Hochschule Z\"urich\\
  Wolfgang-Pauli-Strasse 27, 8093 Z\"urich, Switzerland\\[1ex]
  \href{mailto:nbeisert@itp.phys.ethz.ch}
  {\texttt{nbeisert@itp.phys.ethz.ch}}}
\hypersetup{pdfauthor={Niklas Beisert}}
\hypersetup{pdfsubject={Manual for the LaTeX2e Package childdoc}}
\date{30 December 2018, \textsf{v2.0}}
\maketitle

\begin{abstract}\noindent
\textsf{childdoc} is a \LaTeXe{} package
that enables the direct compilation
of document sections included by |\include|
to individual files.
\end{abstract}

\begingroup
\parskip0ex
\tableofcontents
\endgroup

%%%%%%%%%%%%%%%%%%%%%%%%%%%%%%%%%%%%%%%%%%%%%%%%%%%%%%%%%%%%%%%%%%%%%%%%%%%%%%%%
%%%%%%%%%%%%%%%%%%%%%%%%%%%%%%%%%%%%%%%%%%%%%%%%%%%%%%%%%%%%%%%%%%%%%%%%%%%%%%%%
\section{Introduction}

\LaTeX{} provides a mechanism to structure a large document (such as a book)
into a main file and several child files (containing the chapters)
using the |\include| command.
This mechanism is beneficial for documents
which span hundreds of pages in order to
make the source file(s) more manageable.
Moreover, compilation can be restricted to
selected child files by means of the |\includeonly| command.
The latter feature can be used to reduce the compilation time while editing
(this was significantly more useful in the earlier days of \LaTeX{})
or to generate a smaller document which is easier to navigate.
Another application of |\includeonly| is to generate
documents consisting of selected parts of the complete document.

However, there are a few drawbacks of the plain |\include| mechanism:
\begin{itemize}
\item
The child files cannot be compiled on their own,
they can only be compiled via the main file.
A naive editing environment
(such as a text editor with an option
to have the current file processed by \LaTeX)
may require one to switch to the main file before compiling;
attempting to compile the child file produces errors.
\item
The main file must be modified (each time)
to adjust the |\includeonly| command
to the present needs. This easily leaves the main file in a messy state.
\item
The generated document will always carry the filename
of the main document. This is inconvenient if
several child files are to be compiled and
to be kept for distribution.
\end{itemize}

The present package provides a simple interface
to make child files individually compilable by \LaTeX{}.
Compiling a child file then has the same effect as compiling
the main file with an |\includeonly| command
to select the appropriate child.
Moreover the generated document will carry the name of the child
rather than the main file.
This resolves all three above issues.

This feature is meant to make the editing of books,
thesis documents and lecture notes somewhat more convenient.
However, the package can also be used efficiently for
composing a series of documents (such as exercise sheets)
which are typically distributed individually.
It then assists the author in generating the individual documents
(potentially in different versions)
as well as a document containing the collected series.
Another application is in developing style files
or other kinds of included material
where compilation of the style file could redirect
to a sample or test file.

%%%%%%%%%%%%%%%%%%%%%%%%%%%%%%%%%%%%%%%%%%%%%%%%%%%%%%%%%%%%%%%%%%%%%%%%%%%%%%%%
%%%%%%%%%%%%%%%%%%%%%%%%%%%%%%%%%%%%%%%%%%%%%%%%%%%%%%%%%%%%%%%%%%%%%%%%%%%%%%%%
\section{Usage}

First of all, the package \textsf{childdoc} is \emph{not} a standard
\LaTeXe{} |.sty| style file! Therefore it needs to be invoked in
a non-standard way.

%%%%%%%%%%%%%%%%%%%%%%%%%%%%%%%%%%%%%%%%%%%%%%%%%%%%%%%%%%%%%%%%%%%%%%%%%%%%%%%%
\subsection{Included Files}
\label{sec:include}

%%%%%%%%%%%%%%%%%%%%%%%%%%%%%%%%%%%%%%%%
\DescribeMacro{\childdocmain}
To use the package, add the commands
\begin{center}
\begin{tabular}{l}
|\input{childdoc.def}|\\
|\childdocmain{}|\\
\end{tabular}
\end{center}
at the very top of the main \LaTeX{} file,
in particular \emph{before} the |\documentclass| statement!
The argument of |\childdocmain| should be left empty
(but it must be present).

%%%%%%%%%%%%%%%%%%%%%%%%%%%%%%%%%%%%%%%%
\DescribeMacro{\childdocof}
Furthermore, add the commands
\begin{center}
\begin{tabular}{l}
|\input{childdoc.def}|\\
|\childdocof{|\textit{main}|}|\\
\end{tabular}
\end{center}
at the top of every child file \textit{child}
which is included by |\include{|\textit{child}|}|
from within the main file
(or at least for those files to be compiled individually).
The argument \textit{main} must be the filename of the main file.

There are a couple of
considerations in setting up the main and child documents:

%%%%%%%%%%%%%%%%%%%%%%%%%%%%%%%%%%%%%%%%
\paragraph{Restrictions.}

Please note the following restrictions:
\begin{itemize}
\item
|\childdocmain| must be called with one argument \textit{main}
to ensure compatibility with earlier version of the package.
It must either be empty (|\childdocmain{}|)
or precisely match the filename of the main file in which it is specified.
See \secref{sec:detection} for further information.
\item
The filename \textit{main} must be specified without the |.tex| extension.
\item
The filename \textit{main} is case sensitive
(even in case-insensitive file systems)
due to internal string comparison.
\item
The argument \textit{main} should be fully expanded, it cannot be a macro.
\item
Subdirectories and special characters should be avoided in filenames.
\item
The command |\childdocmain{|\textit{main}|}| must be followed by a whitespace.
It should not be followed immediately by another command
or by a comment mark `|%|'.
This is because the \TeX{} parser reads the token immediately following
the argument of |\childdocmain| and puts it
at the beginning of every child section;
however, a white\-space is ignored.
\end{itemize}

%%%%%%%%%%%%%%%%%%%%%%%%%%%%%%%%%%%%%%%%
\paragraph{Content of Main File.}

It is advisable to place all content in the child files included by |\include|.
Any output contained in the main file will appear in all child documents
unless suppressed manually;
it cannot be suppressed automatically by the |\includeonly| directive
and thus should normally be avoided.
A method to include some content in the main file
by means of conditional processing is described in \secref{sec:conditional}.

%%%%%%%%%%%%%%%%%%%%%%%%%%%%%%%%%%%%%%%%
\paragraph{Page Numbering.}

When only a part of the document is compiled,
the appropriate numbering of pages
(as well as other status parameters)
is determined from the |.aux| files.
The latter contain information from previous passes.
However this information needs to propagate through
all intermediate child documents.
Therefore the page numbering in child documents may well
be inconsistent until the complete document is compiled at least once.

A useful (if unconventional) way to always ensure a consistent
page numbering is to restart the numbering in each child document
and denote the pages by `\textit{child}|.|\textit{page}'
where \textit{child} represents the chapter/section number of the child file.
This can be achieved by the command
|\numberwithin{page}{|\textit{child}|}|
of the \textsf{amsmath} package
where \textit{child} can be |chapter| or |section|
depending on the chosen structuring.
Alternatively, one can modify the macro |\thepage| appropriately
and reset the counter |page| at the start of each child file.

%%%%%%%%%%%%%%%%%%%%%%%%%%%%%%%%%%%%%%%%%%%%%%%%%%%%%%%%%%%%%%%%%%%%%%%%%%%%%%%%
\subsection{Conditional Processing}
\label{sec:conditional}

The package provides a mechanism to compile different versions
of a document. To customise the versions further some conditional processing
can come in handy to distinguish which version is being compiled.
The package provides two macros to describe the compilation context:

%%%%%%%%%%%%%%%%%%%%%%%%%%%%%%%%%%%%%%%%
\DescribeMacro{\ifchilddoc}
The conditional |\ifchilddoc| distinguishes between the compilation of
child documents and the main document:
%
\begin{center}
|\ifchilddoc |\textit{child-code}| |[|\||else |\textit{main-code}]| \||fi|
\end{center}

%%%%%%%%%%%%%%%%%%%%%%%%%%%%%%%%%%%%%%%%
\DescribeMacro{\childdocname}
\DescribeMacro{\childdocjob}
The macro |\childdocname| contains the filename (without extension)
of the main or child file being processed.
Note that |\childdocjob| will always contain the name of the main file.

%%%%%%%%%%%%%%%%%%%%%%%%%%%%%%%%%%%%%%%%
\paragraph{Title Page.}

Conditional processing can be used to include a title or banner page
in the main document when proper precautions are taken.
Importantly, the code in the main file should ensure that the page counter
(as well as other status parameters which are stored in the |.aux| files)
takes the same value after the conditional processing.
Otherwise the page numbers may take divergent values
depending on which part is compiled.

For example, a title page could be declared by:
%
\begin{center}
\begin{tabular}{l}
|\ifchilddoc\||else|\\
|\addtocounter{page}{-1}|\\
\textit{code for title page}\\
|\newpage|\\
|\||fi|
\end{tabular}
\end{center}
%
A banner page for the child documents can be generated by:
%
\begin{center}
\begin{tabular}{l}
|\ifchilddoc|\\
|\addtocounter{page}{-1}|\\
\textit{code for banner page}\\
|\newpage|\\
|\||fi|
\end{tabular}
\end{center}
%
Here one could write a message such as:
\begin{center}
|This is the part \childdocname{} of \childdocjob{}.|
\end{center}

%%%%%%%%%%%%%%%%%%%%%%%%%%%%%%%%%%%%%%%%%%%%%%%%%%%%%%%%%%%%%%%%%%%%%%%%%%%%%%%%
\subsection{Flags}
\label{sec:flags}

The package makes it easy to generate different versions
of the main or child documents.
To this end compilation flags can be defined
and assigned different default values.
They will be particularly useful in conjunction
with the forwarding mechanism described in \secref{sec:forward}.

For example, it may be useful to have a flag |\version|
which can be set to |draft| or |final|.
The document source will contain some conditional code
depending on the value of |\version|.
Suppose further, the flag should default to |final| for the main file
and to |draft| for child files
which is a natural assignment for editing the document.
This is achieved by placing the following code
in the preamble of the main document
(below the |\childdocmain| directive):
%
\begin{center}
\begin{tabular}{l}
|\ifchilddoc|\\
|\providecommand{\version}{draft}|\\
|\||else|\\
|\providecommand{\version}{final}|\\
|\||fi|
\end{tabular}
\end{center}
%
The definition by |\providecommand| makes sure
that previous definitions are not overwritten.
Further statements |\providecommand{\version}{...}|
can thus be added before the above code to override it.

For the main file, one might add a line
(between |\childdocmain| and the above block)
%
\begin{center}
|%\ifchilddoc\||else\providecommand{\version}{draft}\||fi|
\end{center}
%
which can be uncommented to produce a draft version.
Likewise one can add a line to the very top of a child file
(above the |\childdocof{|\textit{main}|}| directive)
%
\begin{center}
|%\providecommand{\version}{final}|
\end{center}
%
which can be uncommented to produce the final version of this child document.

%%%%%%%%%%%%%%%%%%%%%%%%%%%%%%%%%%%%%%%%%%%%%%%%%%%%%%%%%%%%%%%%%%%%%%%%%%%%%%%%
\subsection{Forwarding}
\label{sec:forward}

Different versions of the main or child documents
using compilation flags as described in \secref{sec:flags}
can be (permanently) stored in different files
for convenient compilation, viewing and distribution.
To this end, the package defines a command
to pass on compilation to a different file:

%%%%%%%%%%%%%%%%%%%%%%%%%%%%%%%%%%%%%%%%
\DescribeMacro{\childdocforward}
The command |\childdocforward| redirects processing to
another source file:
%
\begin{center}
\begin{tabular}{l}
|\input{childdoc.def}|\\
|\childdocforward[|\textit{main}|]{|\textit{dest}|}|\\
\end{tabular}
\end{center}
%
The argument \textit{dest} is the destination file
(without extension).
It should be the main file or one of the child files.
Note that further \textsf{childdoc} directives
such as |\childdocof| and |\childdocforward|
in the indicated file will be processed in this form.
The optional argument \textit{main}
passes on directly to the main file \textit{main}
while pretending to compile the child \textit{dest}.
This form behaves as if \textit{dest}
issues |\childdocof{|\textit{main}|}| right away,
and no further \textsf{childdoc} directives will be processed.

%%%%%%%%%%%%%%%%%%%%%%%%%%%%%%%%%%%%%%%%
\DescribeMacro{\...prefix}
In the alternative form |\childdocforwardprefix|,
%
\begin{center}
\begin{tabular}{l}
|\input{childdoc.def}|\\
|\childdocforwardprefix[|\textit{main}|]{|\textit{prefix}|}{|\textit{dest}|}|
\end{tabular}
\end{center}
%
the destination file is determined by a pattern
depending on the current file:
To make this work, the current file must be called
`{\textit{prefix}\hspace{0.2em}\textit{suffix}}'
with \textit{prefix} matching precisely the argument.
Processing is then passed on to the file
`{\textit{dest}\hspace{0.2em}\textit{suffix}}'.
Surely, the same effect is achieved by
directly specifying the
argument `{\textit{dest}\hspace{0.2em}\textit{suffix}}'
in the first form.
However, that requires to set up a different file
for each child. With the alternative form of the command
all these files can have exactly the same content
which simplifies setting them up and maintaining them.

For example, the following file |draft.tex|
with a compilation flag |\version| as described in \secref{sec:flags}
compiles the main document as a draft:
%
\begin{center}
\begin{tabular}{l}
|\def\version{draft}|\\
|\input{childdoc.def}|\\
|\childdocforward{|\textit{main}|}|
\end{tabular}
\end{center}
%
Likewise, the following files |final|\textit{nn}|.tex|
compile the final version of the child document
|child|\textit{nn}|.tex|:
%
\begin{center}
\begin{tabular}{l}
|\def\version{final}|\\
|\input{childdoc.def}|\\
|\childdocforwardprefix{final}{child}|
\end{tabular}
\end{center}
%

Note that when several versions of a main file and/or of each child file
are to be generated, it may be convenient to set up a |Makefile| or
shell script to automatise the process.

%%%%%%%%%%%%%%%%%%%%%%%%%%%%%%%%%%%%%%%%%%%%%%%%%%%%%%%%%%%%%%%%%%%%%%%%%%%%%%%%
\subsection{Command Line Processing}
\label{sec:commandline}

The effect of redirection files can also be achieved by invoking
the \LaTeX{} compiler with a more elaborate command line.
Most conveniently this should be done as part
of a shell script or a |Makefile|.

When using \textsf{childdoc} in the main file, the following
command lines effectively perform a redirection
(note that depending on the shell being used,
backslashes may have to be doubled: `|\|' $\to$ `|\\|'):
%
\begin{center}
|... -jobname "|\textit{target}|" |\\|"|[\textit{flags}]%
|\input{childdoc.def}\childdocforward[|\textit{main}|]{|\textit{dest}|}"|
\end{center}
%
Here \textit{target} is the name of the output file,
\textit{main} is the name of the main file
and \textit{dest} is the name of the main or child file to be processed
(all filenames without extensions).
The optional argument \textit{main} can be omitted
if \textit{main} matches \textit{dest}.
Optionally, compilation \textit{flags} can be defined via |\def| commands.
This command line makes the \TeX{} engine believe
it is compiling the file \textit{target}
whose content is specified as the latter parameter.
The provided code then forwards the processing to
\textit{main} or \textit{dest} as described in \secref{sec:forward}.

%%%%%%%%%%%%%%%%%%%%%%%%%%%%%%%%%%%%%%%%%%%%%%%%%%%%%%%%%%%%%%%%%%%%%%%%%%%%%%%%
\subsection{Include by Input}
\label{sec:input}

Including child documents by |\include| has some restrictions by design.
Most notably, the content of a child document always occupies
its own set of pages; pages cannot be shared between child documents.
Usually, this behaviour makes perfect sense
because each child document contain an essential part of the document.
However, in some situations it may be desirable to compose
a document from a collection of parts
without having mandatory page breaks between then.
For this case, the package
provides a mechanism to include parts
by |\input| which can also be processed individually.
However, by construction this mechanism
requires manual handling of the content to be output.

%%%%%%%%%%%%%%%%%%%%%%%%%%%%%%%%%%%%%%%%
\DescribeMacro{\ifchilddocmanual}
The main file should be prepared as usual, see \secref{sec:include}.
However, the document body must make a distinction
between processing of an individual part and of the main document, e.g.:
%
\begin{center}
\begin{tabular}{l}
|\ifchilddocmanual|\\
|\input{\childdocname}|\\
|\||else|\\
\textit{document body with }|\input{|\textit{part}|}|\\
|\||fi|
\end{tabular}
\end{center}
%
The conditional |\ifchilddocmanual| is true whenever
a part to be included by |\input| is being compiled,
and the name of the part is stored in |\childdocname|.

%%%%%%%%%%%%%%%%%%%%%%%%%%%%%%%%%%%%%%%%
\DescribeMacro{\childdocby}
Each part to be included by |\input| should start with:
%
\begin{center}
\begin{tabular}{l}
|\input{childdoc.def}|\\
|\childdocby{|\textit{main}|}|\\
\end{tabular}
\end{center}
%
The directive |\childdocby| is similar to |\childdocof|
described in \secref{sec:include},
but the subsequent selection of content must be done manually.
To that end, both |\ifchilddoc| and |\ifchilddocmanual|
will be true upon processing of a part,
and the name of the part is stored in |\childdocname|.
Note that |\jobname| will be set to the filename of the current part
so that each part receives an individual |.aux| file
that does not interfere with the |.aux| file(s) of the main document.
This behaviour can be altered by the alternative form
|\childdocby[*]{|\textit{main}|}| (with a non-empty optional argument)
which uses the |.aux| file of the main document
by setting |\jobname| to \textit{main}.

%%%%%%%%%%%%%%%%%%%%%%%%%%%%%%%%%%%%%%%%%%%%%%%%%%%%%%%%%%%%%%%%%%%%%%%%%%%%%%%%
\subsection{Driver Development}
\label{sec:driver}

The \textsf{childdoc} mechanism can also be use for the development
of definition files such as \LaTeX{} styles or classes.
This case differs from the above setup with multiple parts
included by |\include| in that no |\includeonly| should be invoked.
This can be achieved by starting the include file
(before |\ProvidesPackage|) with:
%
\begin{center}
\begin{tabular}{l}
|\input{childdoc.def}|\\
|\childdocforward{|\textit{main}|}|\\
\end{tabular}
\end{center}
%
or alternatively with:
%
\begin{center}
\begin{tabular}{l}
|\input{childdoc.def}|\\
|\childdocby{|\textit{main}|}|\\
\end{tabular}
\end{center}
%
Both forms have slightly different effects as described above.
The main file is prepared as usual, see \secref{sec:include}.

%%%%%%%%%%%%%%%%%%%%%%%%%%%%%%%%%%%%%%%%%%%%%%%%%%%%%%%%%%%%%%%%%%%%%%%%%%%%%%%%
\subsection{Legacy Detection}
\label{sec:detection}

The directive |\childdocmain| in the main file can detect
whether the complete document or merely a child is to be compiled
even without using the directive |\childdocof|.
This method is deprecated because it is less robust
and there is no compelling reason to use it;
it is merely provided for backward compatibility
and it may be removed in future versions.

If the detection mechanism is to be used,
it is mandatory to correctly specify
the filename of the main file as the argument of |\childdocmain|:
%
\begin{center}
\begin{tabular}{l}
|\input{childdoc.def}|\\
|\childdocmain{|\textit{main}|}|\\
\end{tabular}
\end{center}
%
If |\jobname| does not match the argument \textit{main} of |\childdocmain|,
it is assumed that |\jobname| points to the child file to be compiled.
When using |\childdocmain| with the main file specified as argument,
it suffices to start a child file
with just |\input{|\textit{main}|}|
without loading of the package and using |\childdocof|.
If instead all processing is done
with the appropriate \textsf{childdoc} directives,
the argument of \textit{main} of |\childdocmain| can be empty.

An alternative version of the command line processing described
in \secref{sec:commandline} using the detection mechanism reads:
%
\begin{center}
|... -jobname "|\textit{target}|" "|[\textit{flags}]%
[|\def\jobname{|\textit{dest}|}|]|\input{|\textit{main}|}"|
\end{center}

%%%%%%%%%%%%%%%%%%%%%%%%%%%%%%%%%%%%%%%%%%%%%%%%%%%%%%%%%%%%%%%%%%%%%%%%%%%%%%%%
\subsection{Manual Code}
\label{sec:manual}

In case one cannot be certain whether the definitions file |childdoc.def|
is installed on the target \TeX{} distribution
and one prefers not to ship it,
it is conceivable to paste a few relevant commands into the sources.

To that end, drop all statements |\input{childdoc.def}|
and perform the replacements as outlined below.
Instead of |\childdocmain{|\textit{main}|}| add the following code
to the top of the main file:
%
\begin{center}
\begin{tabular}{l}
|\||ifdefined\childdocname\endinput\||fi\newif\ifchilddoc|\\
|\edef\childdocname{\scantokens\expandafter{\jobname\noexpand}}|\\
|\def\childdocmain{|\textit{main}|}\||ifx\childdocmain\childdocname\||else|\\
|\childdoctrue\includeonly{\childdocname}\let\jobname\childdocmain\||fi|\\
\end{tabular}
\end{center}
%
Instead of |\childdocof{|\textit{main}|}| just include the main file
at the top of each child file:
%
\begin{center}
|\input{|\textit{main}|}|
\end{center}
%
A simple redirection |\childdocforward{|\textit{dest}|}| is achieved by:
%
\begin{center}
|\def\jobname{|\textit{dest}|}\input{\jobname}|
\end{center}
%
The redirection with prefix
|\childdocforwardprefix[|\textit{prefix}|]{|\textit{dest}|}|
is accomplished by:
%
\begin{center}
\begin{tabular}{l}
|{\edef\jobname{\scantokens\expandafter{\jobname\noexpand}}|\\
|\def\redirectjob |\textit{prefix}|#1~~~{\gdef\jobname{|\textit{dest}|#1}}|\\
|\expandafter\redirectjob\jobname~~~}\input{\jobname}|
\end{tabular}
\end{center}

In an alternative approach,
child documents can be compiled by a specific command line
without additional code or specific definitions:
%
\begin{center}
|... -jobname "|\textit{target}|" "|[\textit{flags}]%
|\includeonly{|\textit{dest}|}\input{|\textit{main}|}"|
\end{center}
%

%%%%%%%%%%%%%%%%%%%%%%%%%%%%%%%%%%%%%%%%%%%%%%%%%%%%%%%%%%%%%%%%%%%%%%%%%%%%%%%%
%%%%%%%%%%%%%%%%%%%%%%%%%%%%%%%%%%%%%%%%%%%%%%%%%%%%%%%%%%%%%%%%%%%%%%%%%%%%%%%%
\section{Information}

%%%%%%%%%%%%%%%%%%%%%%%%%%%%%%%%%%%%%%%%%%%%%%%%%%%%%%%%%%%%%%%%%%%%%%%%%%%%%%%%
\subsection{Copyright}

Copyright \copyright{} 2017--2018 Niklas Beisert

This work may be distributed and/or modified under the
conditions of the \LaTeX{} Project Public License, either version 1.3
of this license or (at your option) any later version.
The latest version of this license is in
  \url{http://www.latex-project.org/lppl.txt}
and version 1.3 or later is part of all distributions of \LaTeX{}
version 2005/12/01 or later.

This work has the LPPL maintenance status `maintained'.

The Current Maintainer of this work is Niklas Beisert.

This work consists of the files |README.txt|, |childdoc.ins| and |childdoc.dtx|
as well as the derived files |childdoc.def|, |cdocsamp.tex|
with |cdocsch1.tex|, |cdocsch2.tex|, |cdocspt3.tex|, |cdocspt4.tex|,
|cdocsdrf.tex|, |cdocsfn1.tex|, |cdocsfn2.tex|
as well as |childdoc.pdf|.

%%%%%%%%%%%%%%%%%%%%%%%%%%%%%%%%%%%%%%%%%%%%%%%%%%%%%%%%%%%%%%%%%%%%%%%%%%%%%%%%
\subsection{Files and Installation}

The package consists of the files:
%
\begin{center}
\begin{tabular}{ll}
    |README.txt|   & readme file \\
    |childdoc.ins| & installation file \\
    |childdoc.dtx| & source file \\
    |childdoc.def| & definition file \\
    |cdocsamp.tex| & sample main file \\
    |cdocsch1.tex| & sample include file \\
    |cdocsch2.tex| & sample include file \\
    |cdocspt3.tex| & sample part file \\
    |cdocspt4.tex| & sample part file \\
    |cdocsdrf.tex| & sample redirection file \\
    |cdocsfn1.tex| & sample redirection file \\
    |cdocsfn2.tex| & sample redirection file \\
    |childdoc.pdf| & manual
\end{tabular}
\end{center}
%
The distribution consists of the files
|README.txt|, |childdoc.ins| and |childdoc.dtx|.
%
\begin{itemize}
\item
Run (pdf)\LaTeX{} on |childdoc.dtx|
to compile the manual |childdoc.pdf| (this file).
\item
Run \LaTeX{} on |childdoc.ins| to create the definitions file |childdoc.def|
and the sample |cdocsamp.tex| with include files
|cdocsch1.tex|, |cdocsch2.tex|, |cdocspt3.tex|, |cdocspt4.tex|,
|cdocsdrf.tex|, |cdocsfn1.tex|, |cdocsfn2.tex|.
Then copy the file |childdoc.def| to an appropriate directory of your \LaTeX{}
distribution, e.g.\ \textit{texmf-root}|/tex/latex/childdoc|.
\end{itemize}

%%%%%%%%%%%%%%%%%%%%%%%%%%%%%%%%%%%%%%%%%%%%%%%%%%%%%%%%%%%%%%%%%%%%%%%%%%%%%%%%
\subsection{Related CTAN Packages}

There are several other packages which offer a similar functionality:
%
\begin{itemize}
\item
The packages
\href{http://ctan.org/pkg/docmute}{\textsf{docmute}},
\href{http://ctan.org/pkg/includex}{\textsf{includex}} and
\href{http://ctan.org/pkg/standalone}{\textsf{standalone}}
provide commands to include only the document body of
a child file thus allowing both files to be compiled individually.
\item
The packages \href{http://ctan.org/pkg/subdocs}{\textsf{subdocs}}
and \href{http://ctan.org/pkg/subfiles}{\textsf{subfiles}}
provide structures in which the main and child documents can be
encapsulated and allowing them to be compiled individually.
The inclusion mechanism is different from the conventional |\include|.
\item
The package \href{http://ctan.org/pkg/combine}{\textsf{combine}}
is an elaborate solution to combine several documents into one.
\end{itemize}
%
See also the CTAN topic \href{http://ctan.org/topic/subdocs}{\textsf{subdocs}}
for further related packages.
The present package differs from the above solutions in that
a document structure constructed with the conventional |\include| mechanism
just needs two extra commands at the top of every file
such that all constituent files can be compiled individually.

%%%%%%%%%%%%%%%%%%%%%%%%%%%%%%%%%%%%%%%%%%%%%%%%%%%%%%%%%%%%%%%%%%%%%%%%%%%%%%%%
%\subsection{Feature Suggestions}
%
%The following is a list of features which may be useful for future
%versions of this package:
%%
%\begin{itemize}
%\item
%\ldots
%\end{itemize}

%%%%%%%%%%%%%%%%%%%%%%%%%%%%%%%%%%%%%%%%%%%%%%%%%%%%%%%%%%%%%%%%%%%%%%%%%%%%%%%%
\subsection{Revision History}

%%%%%%%%%%%%%%%%%%%%%%%%%%%%%%%%%%%%%%%%
\paragraph{v2.0:} 2018/12/30

\begin{itemize}
\item
immediate forward processing
\item
added |\childdocby| mechanism
\item
manual restructured
\end{itemize}

%%%%%%%%%%%%%%%%%%%%%%%%%%%%%%%%%%%%%%%%
\paragraph{v1.6:} 2018/01/17

\begin{itemize}
\item
application for development of include files
\item
corrections to manual
\end{itemize}

%%%%%%%%%%%%%%%%%%%%%%%%%%%%%%%%%%%%%%%%
\paragraph{v1.5:} 2017/05/21

\begin{itemize}
\item
more complete structuring introduced
\item
|\childdocof| introduced
\item
|\childdoc| renamed to |\childdocmain|
\item
|\childredirect| renamed to |\childdocforward| and |\childdocforwardprefix|
and functionality expanded
\end{itemize}

%%%%%%%%%%%%%%%%%%%%%%%%%%%%%%%%%%%%%%%%
\paragraph{v1.0:} 2017/04/27

\begin{itemize}
\item
manual and install package
\item
first version published on CTAN
\end{itemize}

%%%%%%%%%%%%%%%%%%%%%%%%%%%%%%%%%%%%%%%%
\paragraph{v0.6:} 2017/04/26

\begin{itemize}
\item
redirection mechanism added
\end{itemize}

%%%%%%%%%%%%%%%%%%%%%%%%%%%%%%%%%%%%%%%%
\paragraph{v0.5:} 2017/04/26

\begin{itemize}
\item
functionality in definition file
\end{itemize}


%%%%%%%%%%%%%%%%%%%%%%%%%%%%%%%%%%%%%%%%%%%%%%%%%%%%%%%%%%%%%%%%%%%%%%%%%%%%%%%%
%%%%%%%%%%%%%%%%%%%%%%%%%%%%%%%%%%%%%%%%%%%%%%%%%%%%%%%%%%%%%%%%%%%%%%%%%%%%%%%%
%%%%%%%%%%%%%%%%%%%%%%%%%%%%%%%%%%%%%%%%%%%%%%%%%%%%%%%%%%%%%%%%%%%%%%%%%%%%%%%%
\appendix

\settowidth\MacroIndent{\rmfamily\scriptsize 000\ }

 \DocInput{childdoc.dtx}

\end{document}
%</driver>
% \fi
%
% %%%%%%%%%%%%%%%%%%%%%%%%%%%%%%%%%%%%%%%%%%%%%%%%%%%%%%%%%%%%%%%%%%%%%%%%%%%%%%
% %%%%%%%%%%%%%%%%%%%%%%%%%%%%%%%%%%%%%%%%%%%%%%%%%%%%%%%%%%%%%%%%%%%%%%%%%%%%%%
% \section{Sample}
%\iffalse
%<*samplemain>
%\fi
%
% The following presents a sample document
% with two chapters, two parts, a title page,
% a compile flag as well as three forwarding files to set the flag.
% It consists of eight |.tex| files:
% \begin{center}
% \begin{tabular}{ll}
% |cdocsamp.tex|&main file\\
% |cdocsch1.tex|&include file for chapter 1\\
% |cdocsch2.tex|&include file for chapter 2\\
% |cdocspt3.tex|&include file for part 3\\
% |cdocspt4.tex|&include file for part 4\\
% |cdocsdrf.tex|&forwarding file for main file in draft mode\\
% |cdocsfi1.tex|&forwarding file for final version of chapter 1\\
% |cdocsfi2.tex|&forwarding file for final version of chapter 2\\
% \end{tabular}
% \end{center}
% Each of the eight files can be compiled directly by the \LaTeX{} compiler.
%
% %%%%%%%%%%%%%%%%%%%%%%%%%%%%%%%%%%%%%%
% \paragraph{Main File.}
%
% The main file is called |cdocsamp.tex|.
%
% Load the \textsf{childdoc} definitions and
% declare the filename for the main document:
%    \begin{macrocode}
\input{childdoc.def}
\childdocmain{}
%    \end{macrocode}

% Optional override for |\version| flag:
%    \begin{macrocode}
%%\ifchilddoc\else\providecommand{\version}{draft}\fi
%    \end{macrocode}

% Define the default values for the |\version| flag
% (|final| for the main file and |draft| for childs):
%    \begin{macrocode}
\ifchilddoc
\providecommand{\version}{draft}
\else
\providecommand{\version}{final}
\fi
%    \end{macrocode}

% Load the standard document class:
%    \begin{macrocode}
\documentclass[12pt]{article}
%    \end{macrocode}

% Start the document body:
%    \begin{macrocode}
\begin{document}
%    \end{macrocode}

% Declare a title page.
% Print title, part of document being processed and version flag:
%    \begin{macrocode}
\addtocounter{page}{-1}
\begin{center}
{\LARGE\bfseries{}childdoc example\par}
\vspace{1cm}
\ifchilddoc
\ifchilddocmanual part\else chapter\fi:
`\childdocname' of `\childdocjob'\par
\else
main document: `\childdocjob'\par
\fi
version: \version\par
\end{center}
\newpage
%    \end{macrocode}

% Manually include selected file,
% otherwise process as usual:
%    \begin{macrocode}
\ifchilddocmanual
\section*{part `\childdocname'}
\input{\childdocname}
\else
%    \end{macrocode}

% Include the two chapters:
%    \begin{macrocode}
\include{cdocsch1}
\include{cdocsch2}
%    \end{macrocode}

% Include the two parts unless only chapters should be displayed:
%    \begin{macrocode}
\ifchilddoc\else
\section{part three}
\input{cdocspt3}
\section{part four}
\input{cdocspt4}
\fi
%    \end{macrocode}

% Process as usual until here:
%    \begin{macrocode}
\fi
%    \end{macrocode}

% End of document body:
%    \begin{macrocode}
\end{document}
%    \end{macrocode}
%\iffalse
%</samplemain>
%\fi
%
% %%%%%%%%%%%%%%%%%%%%%%%%%%%%%%%%%%%%%%
% \paragraph{Chapter Include Files.}
%
% The include files are called |cdocsch1.tex| and |cdocsch2.tex|.
%
%\iffalse
%<*samplechap1|samplechap2>
%\fi

% Optional override for |\version| flag:
%    \begin{macrocode}
%%\providecommand{\version}{final}
%    \end{macrocode}

% Include the main document:
%    \begin{macrocode}
\input{childdoc.def}
\childdocof{cdocsamp}
%    \end{macrocode}

%\iffalse
%</samplechap1|samplechap2>
%\fi
%
%\iffalse
%<*samplechap1>
%\fi
% Some text for chapter 1:
%    \begin{macrocode}
\section{one}
some text in chapter one
%    \end{macrocode}

%\iffalse
%</samplechap1>
%\fi
% Some text for chapter 2:
%\iffalse
%<*samplechap2>
%\fi
%    \begin{macrocode}
\section{two}
more text in chapter two
%    \end{macrocode}

%\iffalse
%</samplechap2>
%\fi
%
% %%%%%%%%%%%%%%%%%%%%%%%%%%%%%%%%%%%%%%
% \paragraph{Part Include Files.}
%
% The include files are called |cdocspt3.tex| and |cdocspt4.tex|.
%
%\iffalse
%<*samplepart3|samplepart4>
%\fi

% Optional override for |\version| flag:
%    \begin{macrocode}
%%\providecommand{\version}{final}
%    \end{macrocode}

% Include the main document:
%    \begin{macrocode}
\input{childdoc.def}
\childdocby{cdocsamp}
%    \end{macrocode}

%\iffalse
%</samplepart3|samplepart4>
%\fi
%
%\iffalse
%<*samplepart3>
%\fi
% Some text for part 3:
%    \begin{macrocode}
some text in part three
%    \end{macrocode}

%\iffalse
%</samplepart3>
%\fi
% Some text for part 4:
%\iffalse
%<*samplepart4>
%\fi
%    \begin{macrocode}
more text in part four
%    \end{macrocode}

%\iffalse
%</samplepart4>
%\fi
%
% %%%%%%%%%%%%%%%%%%%%%%%%%%%%%%%%%%%%%%
% \paragraph{Forwarding for a Complete Draft.}
%
% The following forwarding file |cdocsdrf.tex|
% compiles the main document in draft mode:
%\iffalse
%<*sampledraft>
%\fi
%    \begin{macrocode}
\def\version{draft}
\input{childdoc.def}
\childdocforward{cdocsamp}
%    \end{macrocode}

%\iffalse
%</sampledraft>
%\fi
%
% %%%%%%%%%%%%%%%%%%%%%%%%%%%%%%%%%%%%%%
% \paragraph{Forwarding for Final Version of the Chapters.}
%
% The following forwarding files |cdocsfn1.tex| and |cdocsfn2.tex|
% (with identical content)
% compile the final versions of the child documents
% |cdocsch1.tex| and |cdocsch2.tex|, respectively:
%\iffalse
%<*samplefinal>
%\fi
%    \begin{macrocode}
\def\version{final}
\input{childdoc.def}
\childdocforwardprefix[cdocsamp]{cdocsfn}{cdocsch}
%    \end{macrocode}

%\iffalse
%</samplefinal>
%\fi
%
% %%%%%%%%%%%%%%%%%%%%%%%%%%%%%%%%%%%%%%
% \paragraph{Command Line Processing.}
%
% The following three command lines generate the output files
% |cdocscld|, |cdocscl1| and |cdocscl2|
% which should be identical to
% |cdocsdrf|, |cdocsch1| and |cdocsfn2|, respectively:
% \begin{center}
% \begin{tabular}{l}
% |latex -jobname cdocscld \|\\
% |  "\def\version{draft}\input{childdoc.def}\childdocforward{cdocsamp}"|\\
% |latex -jobname cdocscl1 \|\\
% |  "\input{childdoc.def}\childdocforward[cdocsamp]{cdocsch1}"|\\
% |latex -jobname cdocscl2 \|\\
% |  "\def\version{final}\input{childdoc.def}\childdocforward{cdocsch2}"|
% \end{tabular}
% \end{center}
% Note that the trailing backslash on each first line
% merely continues the input to the second line
% (for convenient cut ant paste).
% Furthermore, the command |latex| can be replaced by any
% of its alternative versions such as |pdflatex|.
%
% %%%%%%%%%%%%%%%%%%%%%%%%%%%%%%%%%%%%%%%%%%%%%%%%%%%%%%%%%%%%%%%%%%%%%%%%%%%%%%
% %%%%%%%%%%%%%%%%%%%%%%%%%%%%%%%%%%%%%%%%%%%%%%%%%%%%%%%%%%%%%%%%%%%%%%%%%%%%%%
% \section{Implementation}
%\iffalse
%<*package>
%\fi
%
% This section describes the definitions file |childdoc.def|.

% The definitions cannot be loaded using |\usepackage| or |\RequirePackage|
% which has a mechanism to prevent loading a style file more than once.
% When loading the definitions by means of |\input|
% multiple instances have to be prevented manually:
%\iffalse
%This code needs to be before the `\ProvidesFile' directive
%which is defined at the beginning of this file.
%Therefore it is also placed there and commented out here.
%</package>
%<*discard>
%\fi
%    \begin{macrocode}
\ifdefined\childdocmain\endinput\fi
%    \end{macrocode}
%\iffalse
%</discard>
%<*package>
%\fi
%
% \macro{\ifchilddoc}
% \macro{\ifchilddocmanual}
% The conditional |\ifchilddoc| tells whether a
% child (true) or main (false) document is being compiled.
% The conditional |\ifchilddocmanual| tells whether
% the |\includeonly| mechanism is used (false) or
% the selection of child files must be performed manually (true).
% The definitions initialise to false:
%    \begin{macrocode}
\newif\ifchilddoc
\newif\ifchilddocmanual
%    \end{macrocode}

% \macro{\childdocname}
% \macro{\childdocjob}
% The macro |\childdocname| stores the name of the main document
% to be compiled. The macro |\childdocjob| stores the name of
% the document on which the \LaTeX{} compiler was originally invoked.
% The content of |\jobname| cannot be compared
% to filenames specified in the source due to different catcodes.
% The following code rescans |\jobname|, stores the result
% in |\childdocname| and saves a copy in |\childdocjob|:
%    \begin{macrocode}
\edef\childdocname{\scantokens\expandafter{\jobname\noexpand}}
\let\childdocjob\childdocname
%    \end{macrocode}

% \macro{\childdocdisable}
% The macro |\childdocdisable| prevents the main file
% from being processed more than once.
% At this stage, the main document command |\childdocmain|
% is assumed to be called once again where it should do nothing.
% Any subsequent call to it should prevent
% a secondary processing of the main document
% It overwrites the forwarding commands
% |\childdocof| and |\childdocforward|
% with empty macros to prevent further inclusions of the main document:
%    \begin{macrocode}
\newcommand{\childdocdisable}
{
  \renewcommand{\childdocmain}[1]{\renewcommand{\childdocmain}[1]{\endinput}}
  \renewcommand{\childdocof}[1]{}
  \renewcommand{\childdocby}[2][]{}
  \renewcommand{\childdocforward}[2][]{}
  \renewcommand{\childdocdisable}{}
}
%    \end{macrocode}

% \macro{\childdocmain}
% The macro |\childdocmain| is to be called at the top of the main file
% with nothing or the main filename (without extension) as argument.
% First, it breaks loops.
% If the argument is not empty and does not match |\childdocname|
% (which is set by the first inclusion of |childdoc.def|),
% |\ifchilddoc| is set to true, |\includeonly| is applied to the child file
% and |\jobname| is set to the main file
% (for proper handling of |.aux| files):
%    \begin{macrocode}
\newcommand{\childdocmain}[1]
{
  \childdocdisable\childdocmain{}
  \if?#1?\else
    \begingroup
      \def\childdoctmp{#1}
      \ifx\childdoctmp\childdocname
        \def\childdoctmp{}
      \else
        \def\childdoctmp
        {
          \childdoctrue
          \includeonly{\childdocname}
          \def\childdocjob{#1}
          \def\jobname{#1}
        }
      \fi
      \expandafter
    \endgroup
    \childdoctmp
  \fi
}
%    \end{macrocode}

% \macro{\childdocof}
% The command |\childdocof| redirects
% compilation to the main file |#1|.
%    \begin{macrocode}
\newcommand{\childdocof}[1]
{
  \childdocdisable
  \childdoctrue
  \includeonly{\childdocname}
  \def\jobname{#1}
  \def\childdocjob{#1}
  \input{#1}
}
%    \end{macrocode}

% \macro{\childdocby}
% The command |\childdocby| ....
%    \begin{macrocode}
\newcommand{\childdocby}[2][]
{
  \childdocdisable
  \childdoctrue
  \childdocmanualtrue
  \if?#1?\else
    \def\jobname{#2}
  \fi
  \def\childdocjob{#2}
  \input{#2}
  \endinput
}
%    \end{macrocode}

% \macro{\childdocforward}
% The command |\childdocforward| redirects
% compilation to the main file or
% (if the optional argument is given) a child file.
% Parameters are set as if the main file
% or a child file starting with |\childdocof| was compiled.
% Then compilation is handed over to the main file:
%    \begin{macrocode}
\newcommand{\childdocforward}[2][]
{
  \begingroup
    \if?#1?
      \def\childdoctmp
      {
        \def\childdocname{#2}
        \def\childdocjob{#2}
        \def\jobname{#2}
        \input{#2}
        \endinput
      }
    \else
      \def\childdoctmp
      {
        \childdocdisable
        \def\childdocname{#2}
        \childdoctrue
        \includeonly{#2}
        \def\childdocjob{#1}
        \def\jobname{#1}
        \input{#1}
        \endinput
      }
    \fi
    \expandafter
  \endgroup
  \childdoctmp
}
%    \end{macrocode}

% \macro{\childdocforwardprefix}
% The command |\childdocforwardprefix| redirects
% compilation to the main or a child file by means of a pattern.
% The prefix |#1| in the current filename is replaced by |#2|
% and the suffix of the current filename is kept
% (it is assumed that the filename does not contain the substring `|~~~|'
% which is used as a delimiter).
% Compilation is handed over to the new file by |\childdocforward|:
%    \begin{macrocode}
\newcommand{\childdocforwardprefix}[3][]
{
  \begingroup
    \def\childdocextract #2##1~~~{\def\childdoctmp{\childdocforward[#1]{#3##1}}}
    \expandafter\childdocextract\childdocname~~~
    \expandafter
  \endgroup
  \childdoctmp
}
%    \end{macrocode}

% \macro{\childdoc}
% The deprecated macro |\childdoc| is a legacy version of |\childdocmain|:
%    \begin{macrocode}
\newcommand{\childdoc}{\childdocmain}
%    \end{macrocode}

% \macro{\childdocredirect}
% The deprecated macro |\childdocredirect| is a legacy version
% of |\childdocforward| and |\childdocforwardprefix|:
%    \begin{macrocode}
\newcommand{\childdocredirect}[2][]
{
  \begingroup
    \if?#1?
      \def\childdoctmp{\childdocforward{#2}}
    \else
      \def\childdoctmp{\childdocforwardprefix{#1}{#2}}
    \fi
    \expandafter
  \endgroup
  \childdoctmp
}
%    \end{macrocode}

%\iffalse
%</package>
%\fi
%
\endinput
|\\
|\childdocforward{|\textit{main}|}|
\end{tabular}
\end{center}
%
Likewise, the following files |final|\textit{nn}|.tex|
compile the final version of the child document
|child|\textit{nn}|.tex|:
%
\begin{center}
\begin{tabular}{l}
|\def\version{final}|\\
|% \iffalse
%
% childdoc.dtx Copyright (C) 2017-2018 Niklas Beisert
%
% This work may be distributed and/or modified under the
% conditions of the LaTeX Project Public License, either version 1.3
% of this license or (at your option) any later version.
% The latest version of this license is in
%   http://www.latex-project.org/lppl.txt
% and version 1.3 or later is part of all distributions of LaTeX
% version 2005/12/01 or later.
%
% This work has the LPPL maintenance status `maintained'.
%
% The Current Maintainer of this work is Niklas Beisert.
%
% This work consists of the files childdoc.dtx and childdoc.ins
% and the derived files childdoc.def and cdocsamp.tex with
% cdocsch1.tex, cdocsch2.tex, cdocsdrf.tex, cdocsfn1.tex, cdocsfn2.tex.
%
%<package>\ifdefined\childdocmain\endinput\fi
%<package>\ProvidesFile{childdoc.def}[2018/12/30 v2.0 child document driver]
%<samplemain>\ProvidesFile{cdocsamp.tex}[2018/12/30 v2.0 sample for childdoc]
%<*driver>
%\ProvidesFile{childdoc.drv}[2018/12/30 v2.0 childdoc reference manual file]
\PassOptionsToClass{10pt,a4paper}{article}
\documentclass{ltxdoc}

\usepackage[margin=35mm]{geometry}
\usepackage{hyperref}
\usepackage{hyperxmp}
\usepackage[usenames]{color}

\hypersetup{colorlinks=true}
\hypersetup{pdfstartview=FitH}
\hypersetup{pdfpagemode=UseNone}
\hypersetup{pdfsource={}}
\hypersetup{pdflang={en-UK}}
\hypersetup{pdfcopyright={Copyright 2017-2018 Niklas Beisert.
  This work may be distributed and/or modified under the
  conditions of the LaTeX Project Public License, either version 1.3
  of this license or (at your option) any later version.}}
\hypersetup{pdflicenseurl={http://www.latex-project.org/lppl.txt}}
\hypersetup{pdfcontactaddress={ETH Zurich, ITP, HIT K,
  Wolfgang-Pauli-Strasse 27}}
\hypersetup{pdfcontactpostcode={8093}}
\hypersetup{pdfcontactcity={Zurich}}
\hypersetup{pdfcontactcountry={Switzerland}}
\hypersetup{pdfcontactemail={nbeisert@itp.phys.ethz.ch}}
\hypersetup{pdfcontacturl={http://people.phys.ethz.ch/\xmptilde nbeisert/}}

\newcommand{\secref}[1]{\hyperref[#1]{section \ref*{#1}}}

\parskip1ex
\parindent0pt
\let\olditemize\itemize
\def\itemize{\olditemize\parskip0pt}

\begin{document}

\title{The \textsf{childdoc} Package}
\hypersetup{pdftitle={The childdoc Package}}
\author{Niklas Beisert\\[2ex]
  Institut f\"ur Theoretische Physik\\
  Eidgen\"ossische Technische Hochschule Z\"urich\\
  Wolfgang-Pauli-Strasse 27, 8093 Z\"urich, Switzerland\\[1ex]
  \href{mailto:nbeisert@itp.phys.ethz.ch}
  {\texttt{nbeisert@itp.phys.ethz.ch}}}
\hypersetup{pdfauthor={Niklas Beisert}}
\hypersetup{pdfsubject={Manual for the LaTeX2e Package childdoc}}
\date{30 December 2018, \textsf{v2.0}}
\maketitle

\begin{abstract}\noindent
\textsf{childdoc} is a \LaTeXe{} package
that enables the direct compilation
of document sections included by |\include|
to individual files.
\end{abstract}

\begingroup
\parskip0ex
\tableofcontents
\endgroup

%%%%%%%%%%%%%%%%%%%%%%%%%%%%%%%%%%%%%%%%%%%%%%%%%%%%%%%%%%%%%%%%%%%%%%%%%%%%%%%%
%%%%%%%%%%%%%%%%%%%%%%%%%%%%%%%%%%%%%%%%%%%%%%%%%%%%%%%%%%%%%%%%%%%%%%%%%%%%%%%%
\section{Introduction}

\LaTeX{} provides a mechanism to structure a large document (such as a book)
into a main file and several child files (containing the chapters)
using the |\include| command.
This mechanism is beneficial for documents
which span hundreds of pages in order to
make the source file(s) more manageable.
Moreover, compilation can be restricted to
selected child files by means of the |\includeonly| command.
The latter feature can be used to reduce the compilation time while editing
(this was significantly more useful in the earlier days of \LaTeX{})
or to generate a smaller document which is easier to navigate.
Another application of |\includeonly| is to generate
documents consisting of selected parts of the complete document.

However, there are a few drawbacks of the plain |\include| mechanism:
\begin{itemize}
\item
The child files cannot be compiled on their own,
they can only be compiled via the main file.
A naive editing environment
(such as a text editor with an option
to have the current file processed by \LaTeX)
may require one to switch to the main file before compiling;
attempting to compile the child file produces errors.
\item
The main file must be modified (each time)
to adjust the |\includeonly| command
to the present needs. This easily leaves the main file in a messy state.
\item
The generated document will always carry the filename
of the main document. This is inconvenient if
several child files are to be compiled and
to be kept for distribution.
\end{itemize}

The present package provides a simple interface
to make child files individually compilable by \LaTeX{}.
Compiling a child file then has the same effect as compiling
the main file with an |\includeonly| command
to select the appropriate child.
Moreover the generated document will carry the name of the child
rather than the main file.
This resolves all three above issues.

This feature is meant to make the editing of books,
thesis documents and lecture notes somewhat more convenient.
However, the package can also be used efficiently for
composing a series of documents (such as exercise sheets)
which are typically distributed individually.
It then assists the author in generating the individual documents
(potentially in different versions)
as well as a document containing the collected series.
Another application is in developing style files
or other kinds of included material
where compilation of the style file could redirect
to a sample or test file.

%%%%%%%%%%%%%%%%%%%%%%%%%%%%%%%%%%%%%%%%%%%%%%%%%%%%%%%%%%%%%%%%%%%%%%%%%%%%%%%%
%%%%%%%%%%%%%%%%%%%%%%%%%%%%%%%%%%%%%%%%%%%%%%%%%%%%%%%%%%%%%%%%%%%%%%%%%%%%%%%%
\section{Usage}

First of all, the package \textsf{childdoc} is \emph{not} a standard
\LaTeXe{} |.sty| style file! Therefore it needs to be invoked in
a non-standard way.

%%%%%%%%%%%%%%%%%%%%%%%%%%%%%%%%%%%%%%%%%%%%%%%%%%%%%%%%%%%%%%%%%%%%%%%%%%%%%%%%
\subsection{Included Files}
\label{sec:include}

%%%%%%%%%%%%%%%%%%%%%%%%%%%%%%%%%%%%%%%%
\DescribeMacro{\childdocmain}
To use the package, add the commands
\begin{center}
\begin{tabular}{l}
|\input{childdoc.def}|\\
|\childdocmain{}|\\
\end{tabular}
\end{center}
at the very top of the main \LaTeX{} file,
in particular \emph{before} the |\documentclass| statement!
The argument of |\childdocmain| should be left empty
(but it must be present).

%%%%%%%%%%%%%%%%%%%%%%%%%%%%%%%%%%%%%%%%
\DescribeMacro{\childdocof}
Furthermore, add the commands
\begin{center}
\begin{tabular}{l}
|\input{childdoc.def}|\\
|\childdocof{|\textit{main}|}|\\
\end{tabular}
\end{center}
at the top of every child file \textit{child}
which is included by |\include{|\textit{child}|}|
from within the main file
(or at least for those files to be compiled individually).
The argument \textit{main} must be the filename of the main file.

There are a couple of
considerations in setting up the main and child documents:

%%%%%%%%%%%%%%%%%%%%%%%%%%%%%%%%%%%%%%%%
\paragraph{Restrictions.}

Please note the following restrictions:
\begin{itemize}
\item
|\childdocmain| must be called with one argument \textit{main}
to ensure compatibility with earlier version of the package.
It must either be empty (|\childdocmain{}|)
or precisely match the filename of the main file in which it is specified.
See \secref{sec:detection} for further information.
\item
The filename \textit{main} must be specified without the |.tex| extension.
\item
The filename \textit{main} is case sensitive
(even in case-insensitive file systems)
due to internal string comparison.
\item
The argument \textit{main} should be fully expanded, it cannot be a macro.
\item
Subdirectories and special characters should be avoided in filenames.
\item
The command |\childdocmain{|\textit{main}|}| must be followed by a whitespace.
It should not be followed immediately by another command
or by a comment mark `|%|'.
This is because the \TeX{} parser reads the token immediately following
the argument of |\childdocmain| and puts it
at the beginning of every child section;
however, a white\-space is ignored.
\end{itemize}

%%%%%%%%%%%%%%%%%%%%%%%%%%%%%%%%%%%%%%%%
\paragraph{Content of Main File.}

It is advisable to place all content in the child files included by |\include|.
Any output contained in the main file will appear in all child documents
unless suppressed manually;
it cannot be suppressed automatically by the |\includeonly| directive
and thus should normally be avoided.
A method to include some content in the main file
by means of conditional processing is described in \secref{sec:conditional}.

%%%%%%%%%%%%%%%%%%%%%%%%%%%%%%%%%%%%%%%%
\paragraph{Page Numbering.}

When only a part of the document is compiled,
the appropriate numbering of pages
(as well as other status parameters)
is determined from the |.aux| files.
The latter contain information from previous passes.
However this information needs to propagate through
all intermediate child documents.
Therefore the page numbering in child documents may well
be inconsistent until the complete document is compiled at least once.

A useful (if unconventional) way to always ensure a consistent
page numbering is to restart the numbering in each child document
and denote the pages by `\textit{child}|.|\textit{page}'
where \textit{child} represents the chapter/section number of the child file.
This can be achieved by the command
|\numberwithin{page}{|\textit{child}|}|
of the \textsf{amsmath} package
where \textit{child} can be |chapter| or |section|
depending on the chosen structuring.
Alternatively, one can modify the macro |\thepage| appropriately
and reset the counter |page| at the start of each child file.

%%%%%%%%%%%%%%%%%%%%%%%%%%%%%%%%%%%%%%%%%%%%%%%%%%%%%%%%%%%%%%%%%%%%%%%%%%%%%%%%
\subsection{Conditional Processing}
\label{sec:conditional}

The package provides a mechanism to compile different versions
of a document. To customise the versions further some conditional processing
can come in handy to distinguish which version is being compiled.
The package provides two macros to describe the compilation context:

%%%%%%%%%%%%%%%%%%%%%%%%%%%%%%%%%%%%%%%%
\DescribeMacro{\ifchilddoc}
The conditional |\ifchilddoc| distinguishes between the compilation of
child documents and the main document:
%
\begin{center}
|\ifchilddoc |\textit{child-code}| |[|\||else |\textit{main-code}]| \||fi|
\end{center}

%%%%%%%%%%%%%%%%%%%%%%%%%%%%%%%%%%%%%%%%
\DescribeMacro{\childdocname}
\DescribeMacro{\childdocjob}
The macro |\childdocname| contains the filename (without extension)
of the main or child file being processed.
Note that |\childdocjob| will always contain the name of the main file.

%%%%%%%%%%%%%%%%%%%%%%%%%%%%%%%%%%%%%%%%
\paragraph{Title Page.}

Conditional processing can be used to include a title or banner page
in the main document when proper precautions are taken.
Importantly, the code in the main file should ensure that the page counter
(as well as other status parameters which are stored in the |.aux| files)
takes the same value after the conditional processing.
Otherwise the page numbers may take divergent values
depending on which part is compiled.

For example, a title page could be declared by:
%
\begin{center}
\begin{tabular}{l}
|\ifchilddoc\||else|\\
|\addtocounter{page}{-1}|\\
\textit{code for title page}\\
|\newpage|\\
|\||fi|
\end{tabular}
\end{center}
%
A banner page for the child documents can be generated by:
%
\begin{center}
\begin{tabular}{l}
|\ifchilddoc|\\
|\addtocounter{page}{-1}|\\
\textit{code for banner page}\\
|\newpage|\\
|\||fi|
\end{tabular}
\end{center}
%
Here one could write a message such as:
\begin{center}
|This is the part \childdocname{} of \childdocjob{}.|
\end{center}

%%%%%%%%%%%%%%%%%%%%%%%%%%%%%%%%%%%%%%%%%%%%%%%%%%%%%%%%%%%%%%%%%%%%%%%%%%%%%%%%
\subsection{Flags}
\label{sec:flags}

The package makes it easy to generate different versions
of the main or child documents.
To this end compilation flags can be defined
and assigned different default values.
They will be particularly useful in conjunction
with the forwarding mechanism described in \secref{sec:forward}.

For example, it may be useful to have a flag |\version|
which can be set to |draft| or |final|.
The document source will contain some conditional code
depending on the value of |\version|.
Suppose further, the flag should default to |final| for the main file
and to |draft| for child files
which is a natural assignment for editing the document.
This is achieved by placing the following code
in the preamble of the main document
(below the |\childdocmain| directive):
%
\begin{center}
\begin{tabular}{l}
|\ifchilddoc|\\
|\providecommand{\version}{draft}|\\
|\||else|\\
|\providecommand{\version}{final}|\\
|\||fi|
\end{tabular}
\end{center}
%
The definition by |\providecommand| makes sure
that previous definitions are not overwritten.
Further statements |\providecommand{\version}{...}|
can thus be added before the above code to override it.

For the main file, one might add a line
(between |\childdocmain| and the above block)
%
\begin{center}
|%\ifchilddoc\||else\providecommand{\version}{draft}\||fi|
\end{center}
%
which can be uncommented to produce a draft version.
Likewise one can add a line to the very top of a child file
(above the |\childdocof{|\textit{main}|}| directive)
%
\begin{center}
|%\providecommand{\version}{final}|
\end{center}
%
which can be uncommented to produce the final version of this child document.

%%%%%%%%%%%%%%%%%%%%%%%%%%%%%%%%%%%%%%%%%%%%%%%%%%%%%%%%%%%%%%%%%%%%%%%%%%%%%%%%
\subsection{Forwarding}
\label{sec:forward}

Different versions of the main or child documents
using compilation flags as described in \secref{sec:flags}
can be (permanently) stored in different files
for convenient compilation, viewing and distribution.
To this end, the package defines a command
to pass on compilation to a different file:

%%%%%%%%%%%%%%%%%%%%%%%%%%%%%%%%%%%%%%%%
\DescribeMacro{\childdocforward}
The command |\childdocforward| redirects processing to
another source file:
%
\begin{center}
\begin{tabular}{l}
|\input{childdoc.def}|\\
|\childdocforward[|\textit{main}|]{|\textit{dest}|}|\\
\end{tabular}
\end{center}
%
The argument \textit{dest} is the destination file
(without extension).
It should be the main file or one of the child files.
Note that further \textsf{childdoc} directives
such as |\childdocof| and |\childdocforward|
in the indicated file will be processed in this form.
The optional argument \textit{main}
passes on directly to the main file \textit{main}
while pretending to compile the child \textit{dest}.
This form behaves as if \textit{dest}
issues |\childdocof{|\textit{main}|}| right away,
and no further \textsf{childdoc} directives will be processed.

%%%%%%%%%%%%%%%%%%%%%%%%%%%%%%%%%%%%%%%%
\DescribeMacro{\...prefix}
In the alternative form |\childdocforwardprefix|,
%
\begin{center}
\begin{tabular}{l}
|\input{childdoc.def}|\\
|\childdocforwardprefix[|\textit{main}|]{|\textit{prefix}|}{|\textit{dest}|}|
\end{tabular}
\end{center}
%
the destination file is determined by a pattern
depending on the current file:
To make this work, the current file must be called
`{\textit{prefix}\hspace{0.2em}\textit{suffix}}'
with \textit{prefix} matching precisely the argument.
Processing is then passed on to the file
`{\textit{dest}\hspace{0.2em}\textit{suffix}}'.
Surely, the same effect is achieved by
directly specifying the
argument `{\textit{dest}\hspace{0.2em}\textit{suffix}}'
in the first form.
However, that requires to set up a different file
for each child. With the alternative form of the command
all these files can have exactly the same content
which simplifies setting them up and maintaining them.

For example, the following file |draft.tex|
with a compilation flag |\version| as described in \secref{sec:flags}
compiles the main document as a draft:
%
\begin{center}
\begin{tabular}{l}
|\def\version{draft}|\\
|\input{childdoc.def}|\\
|\childdocforward{|\textit{main}|}|
\end{tabular}
\end{center}
%
Likewise, the following files |final|\textit{nn}|.tex|
compile the final version of the child document
|child|\textit{nn}|.tex|:
%
\begin{center}
\begin{tabular}{l}
|\def\version{final}|\\
|\input{childdoc.def}|\\
|\childdocforwardprefix{final}{child}|
\end{tabular}
\end{center}
%

Note that when several versions of a main file and/or of each child file
are to be generated, it may be convenient to set up a |Makefile| or
shell script to automatise the process.

%%%%%%%%%%%%%%%%%%%%%%%%%%%%%%%%%%%%%%%%%%%%%%%%%%%%%%%%%%%%%%%%%%%%%%%%%%%%%%%%
\subsection{Command Line Processing}
\label{sec:commandline}

The effect of redirection files can also be achieved by invoking
the \LaTeX{} compiler with a more elaborate command line.
Most conveniently this should be done as part
of a shell script or a |Makefile|.

When using \textsf{childdoc} in the main file, the following
command lines effectively perform a redirection
(note that depending on the shell being used,
backslashes may have to be doubled: `|\|' $\to$ `|\\|'):
%
\begin{center}
|... -jobname "|\textit{target}|" |\\|"|[\textit{flags}]%
|\input{childdoc.def}\childdocforward[|\textit{main}|]{|\textit{dest}|}"|
\end{center}
%
Here \textit{target} is the name of the output file,
\textit{main} is the name of the main file
and \textit{dest} is the name of the main or child file to be processed
(all filenames without extensions).
The optional argument \textit{main} can be omitted
if \textit{main} matches \textit{dest}.
Optionally, compilation \textit{flags} can be defined via |\def| commands.
This command line makes the \TeX{} engine believe
it is compiling the file \textit{target}
whose content is specified as the latter parameter.
The provided code then forwards the processing to
\textit{main} or \textit{dest} as described in \secref{sec:forward}.

%%%%%%%%%%%%%%%%%%%%%%%%%%%%%%%%%%%%%%%%%%%%%%%%%%%%%%%%%%%%%%%%%%%%%%%%%%%%%%%%
\subsection{Include by Input}
\label{sec:input}

Including child documents by |\include| has some restrictions by design.
Most notably, the content of a child document always occupies
its own set of pages; pages cannot be shared between child documents.
Usually, this behaviour makes perfect sense
because each child document contain an essential part of the document.
However, in some situations it may be desirable to compose
a document from a collection of parts
without having mandatory page breaks between then.
For this case, the package
provides a mechanism to include parts
by |\input| which can also be processed individually.
However, by construction this mechanism
requires manual handling of the content to be output.

%%%%%%%%%%%%%%%%%%%%%%%%%%%%%%%%%%%%%%%%
\DescribeMacro{\ifchilddocmanual}
The main file should be prepared as usual, see \secref{sec:include}.
However, the document body must make a distinction
between processing of an individual part and of the main document, e.g.:
%
\begin{center}
\begin{tabular}{l}
|\ifchilddocmanual|\\
|\input{\childdocname}|\\
|\||else|\\
\textit{document body with }|\input{|\textit{part}|}|\\
|\||fi|
\end{tabular}
\end{center}
%
The conditional |\ifchilddocmanual| is true whenever
a part to be included by |\input| is being compiled,
and the name of the part is stored in |\childdocname|.

%%%%%%%%%%%%%%%%%%%%%%%%%%%%%%%%%%%%%%%%
\DescribeMacro{\childdocby}
Each part to be included by |\input| should start with:
%
\begin{center}
\begin{tabular}{l}
|\input{childdoc.def}|\\
|\childdocby{|\textit{main}|}|\\
\end{tabular}
\end{center}
%
The directive |\childdocby| is similar to |\childdocof|
described in \secref{sec:include},
but the subsequent selection of content must be done manually.
To that end, both |\ifchilddoc| and |\ifchilddocmanual|
will be true upon processing of a part,
and the name of the part is stored in |\childdocname|.
Note that |\jobname| will be set to the filename of the current part
so that each part receives an individual |.aux| file
that does not interfere with the |.aux| file(s) of the main document.
This behaviour can be altered by the alternative form
|\childdocby[*]{|\textit{main}|}| (with a non-empty optional argument)
which uses the |.aux| file of the main document
by setting |\jobname| to \textit{main}.

%%%%%%%%%%%%%%%%%%%%%%%%%%%%%%%%%%%%%%%%%%%%%%%%%%%%%%%%%%%%%%%%%%%%%%%%%%%%%%%%
\subsection{Driver Development}
\label{sec:driver}

The \textsf{childdoc} mechanism can also be use for the development
of definition files such as \LaTeX{} styles or classes.
This case differs from the above setup with multiple parts
included by |\include| in that no |\includeonly| should be invoked.
This can be achieved by starting the include file
(before |\ProvidesPackage|) with:
%
\begin{center}
\begin{tabular}{l}
|\input{childdoc.def}|\\
|\childdocforward{|\textit{main}|}|\\
\end{tabular}
\end{center}
%
or alternatively with:
%
\begin{center}
\begin{tabular}{l}
|\input{childdoc.def}|\\
|\childdocby{|\textit{main}|}|\\
\end{tabular}
\end{center}
%
Both forms have slightly different effects as described above.
The main file is prepared as usual, see \secref{sec:include}.

%%%%%%%%%%%%%%%%%%%%%%%%%%%%%%%%%%%%%%%%%%%%%%%%%%%%%%%%%%%%%%%%%%%%%%%%%%%%%%%%
\subsection{Legacy Detection}
\label{sec:detection}

The directive |\childdocmain| in the main file can detect
whether the complete document or merely a child is to be compiled
even without using the directive |\childdocof|.
This method is deprecated because it is less robust
and there is no compelling reason to use it;
it is merely provided for backward compatibility
and it may be removed in future versions.

If the detection mechanism is to be used,
it is mandatory to correctly specify
the filename of the main file as the argument of |\childdocmain|:
%
\begin{center}
\begin{tabular}{l}
|\input{childdoc.def}|\\
|\childdocmain{|\textit{main}|}|\\
\end{tabular}
\end{center}
%
If |\jobname| does not match the argument \textit{main} of |\childdocmain|,
it is assumed that |\jobname| points to the child file to be compiled.
When using |\childdocmain| with the main file specified as argument,
it suffices to start a child file
with just |\input{|\textit{main}|}|
without loading of the package and using |\childdocof|.
If instead all processing is done
with the appropriate \textsf{childdoc} directives,
the argument of \textit{main} of |\childdocmain| can be empty.

An alternative version of the command line processing described
in \secref{sec:commandline} using the detection mechanism reads:
%
\begin{center}
|... -jobname "|\textit{target}|" "|[\textit{flags}]%
[|\def\jobname{|\textit{dest}|}|]|\input{|\textit{main}|}"|
\end{center}

%%%%%%%%%%%%%%%%%%%%%%%%%%%%%%%%%%%%%%%%%%%%%%%%%%%%%%%%%%%%%%%%%%%%%%%%%%%%%%%%
\subsection{Manual Code}
\label{sec:manual}

In case one cannot be certain whether the definitions file |childdoc.def|
is installed on the target \TeX{} distribution
and one prefers not to ship it,
it is conceivable to paste a few relevant commands into the sources.

To that end, drop all statements |\input{childdoc.def}|
and perform the replacements as outlined below.
Instead of |\childdocmain{|\textit{main}|}| add the following code
to the top of the main file:
%
\begin{center}
\begin{tabular}{l}
|\||ifdefined\childdocname\endinput\||fi\newif\ifchilddoc|\\
|\edef\childdocname{\scantokens\expandafter{\jobname\noexpand}}|\\
|\def\childdocmain{|\textit{main}|}\||ifx\childdocmain\childdocname\||else|\\
|\childdoctrue\includeonly{\childdocname}\let\jobname\childdocmain\||fi|\\
\end{tabular}
\end{center}
%
Instead of |\childdocof{|\textit{main}|}| just include the main file
at the top of each child file:
%
\begin{center}
|\input{|\textit{main}|}|
\end{center}
%
A simple redirection |\childdocforward{|\textit{dest}|}| is achieved by:
%
\begin{center}
|\def\jobname{|\textit{dest}|}\input{\jobname}|
\end{center}
%
The redirection with prefix
|\childdocforwardprefix[|\textit{prefix}|]{|\textit{dest}|}|
is accomplished by:
%
\begin{center}
\begin{tabular}{l}
|{\edef\jobname{\scantokens\expandafter{\jobname\noexpand}}|\\
|\def\redirectjob |\textit{prefix}|#1~~~{\gdef\jobname{|\textit{dest}|#1}}|\\
|\expandafter\redirectjob\jobname~~~}\input{\jobname}|
\end{tabular}
\end{center}

In an alternative approach,
child documents can be compiled by a specific command line
without additional code or specific definitions:
%
\begin{center}
|... -jobname "|\textit{target}|" "|[\textit{flags}]%
|\includeonly{|\textit{dest}|}\input{|\textit{main}|}"|
\end{center}
%

%%%%%%%%%%%%%%%%%%%%%%%%%%%%%%%%%%%%%%%%%%%%%%%%%%%%%%%%%%%%%%%%%%%%%%%%%%%%%%%%
%%%%%%%%%%%%%%%%%%%%%%%%%%%%%%%%%%%%%%%%%%%%%%%%%%%%%%%%%%%%%%%%%%%%%%%%%%%%%%%%
\section{Information}

%%%%%%%%%%%%%%%%%%%%%%%%%%%%%%%%%%%%%%%%%%%%%%%%%%%%%%%%%%%%%%%%%%%%%%%%%%%%%%%%
\subsection{Copyright}

Copyright \copyright{} 2017--2018 Niklas Beisert

This work may be distributed and/or modified under the
conditions of the \LaTeX{} Project Public License, either version 1.3
of this license or (at your option) any later version.
The latest version of this license is in
  \url{http://www.latex-project.org/lppl.txt}
and version 1.3 or later is part of all distributions of \LaTeX{}
version 2005/12/01 or later.

This work has the LPPL maintenance status `maintained'.

The Current Maintainer of this work is Niklas Beisert.

This work consists of the files |README.txt|, |childdoc.ins| and |childdoc.dtx|
as well as the derived files |childdoc.def|, |cdocsamp.tex|
with |cdocsch1.tex|, |cdocsch2.tex|, |cdocspt3.tex|, |cdocspt4.tex|,
|cdocsdrf.tex|, |cdocsfn1.tex|, |cdocsfn2.tex|
as well as |childdoc.pdf|.

%%%%%%%%%%%%%%%%%%%%%%%%%%%%%%%%%%%%%%%%%%%%%%%%%%%%%%%%%%%%%%%%%%%%%%%%%%%%%%%%
\subsection{Files and Installation}

The package consists of the files:
%
\begin{center}
\begin{tabular}{ll}
    |README.txt|   & readme file \\
    |childdoc.ins| & installation file \\
    |childdoc.dtx| & source file \\
    |childdoc.def| & definition file \\
    |cdocsamp.tex| & sample main file \\
    |cdocsch1.tex| & sample include file \\
    |cdocsch2.tex| & sample include file \\
    |cdocspt3.tex| & sample part file \\
    |cdocspt4.tex| & sample part file \\
    |cdocsdrf.tex| & sample redirection file \\
    |cdocsfn1.tex| & sample redirection file \\
    |cdocsfn2.tex| & sample redirection file \\
    |childdoc.pdf| & manual
\end{tabular}
\end{center}
%
The distribution consists of the files
|README.txt|, |childdoc.ins| and |childdoc.dtx|.
%
\begin{itemize}
\item
Run (pdf)\LaTeX{} on |childdoc.dtx|
to compile the manual |childdoc.pdf| (this file).
\item
Run \LaTeX{} on |childdoc.ins| to create the definitions file |childdoc.def|
and the sample |cdocsamp.tex| with include files
|cdocsch1.tex|, |cdocsch2.tex|, |cdocspt3.tex|, |cdocspt4.tex|,
|cdocsdrf.tex|, |cdocsfn1.tex|, |cdocsfn2.tex|.
Then copy the file |childdoc.def| to an appropriate directory of your \LaTeX{}
distribution, e.g.\ \textit{texmf-root}|/tex/latex/childdoc|.
\end{itemize}

%%%%%%%%%%%%%%%%%%%%%%%%%%%%%%%%%%%%%%%%%%%%%%%%%%%%%%%%%%%%%%%%%%%%%%%%%%%%%%%%
\subsection{Related CTAN Packages}

There are several other packages which offer a similar functionality:
%
\begin{itemize}
\item
The packages
\href{http://ctan.org/pkg/docmute}{\textsf{docmute}},
\href{http://ctan.org/pkg/includex}{\textsf{includex}} and
\href{http://ctan.org/pkg/standalone}{\textsf{standalone}}
provide commands to include only the document body of
a child file thus allowing both files to be compiled individually.
\item
The packages \href{http://ctan.org/pkg/subdocs}{\textsf{subdocs}}
and \href{http://ctan.org/pkg/subfiles}{\textsf{subfiles}}
provide structures in which the main and child documents can be
encapsulated and allowing them to be compiled individually.
The inclusion mechanism is different from the conventional |\include|.
\item
The package \href{http://ctan.org/pkg/combine}{\textsf{combine}}
is an elaborate solution to combine several documents into one.
\end{itemize}
%
See also the CTAN topic \href{http://ctan.org/topic/subdocs}{\textsf{subdocs}}
for further related packages.
The present package differs from the above solutions in that
a document structure constructed with the conventional |\include| mechanism
just needs two extra commands at the top of every file
such that all constituent files can be compiled individually.

%%%%%%%%%%%%%%%%%%%%%%%%%%%%%%%%%%%%%%%%%%%%%%%%%%%%%%%%%%%%%%%%%%%%%%%%%%%%%%%%
%\subsection{Feature Suggestions}
%
%The following is a list of features which may be useful for future
%versions of this package:
%%
%\begin{itemize}
%\item
%\ldots
%\end{itemize}

%%%%%%%%%%%%%%%%%%%%%%%%%%%%%%%%%%%%%%%%%%%%%%%%%%%%%%%%%%%%%%%%%%%%%%%%%%%%%%%%
\subsection{Revision History}

%%%%%%%%%%%%%%%%%%%%%%%%%%%%%%%%%%%%%%%%
\paragraph{v2.0:} 2018/12/30

\begin{itemize}
\item
immediate forward processing
\item
added |\childdocby| mechanism
\item
manual restructured
\end{itemize}

%%%%%%%%%%%%%%%%%%%%%%%%%%%%%%%%%%%%%%%%
\paragraph{v1.6:} 2018/01/17

\begin{itemize}
\item
application for development of include files
\item
corrections to manual
\end{itemize}

%%%%%%%%%%%%%%%%%%%%%%%%%%%%%%%%%%%%%%%%
\paragraph{v1.5:} 2017/05/21

\begin{itemize}
\item
more complete structuring introduced
\item
|\childdocof| introduced
\item
|\childdoc| renamed to |\childdocmain|
\item
|\childredirect| renamed to |\childdocforward| and |\childdocforwardprefix|
and functionality expanded
\end{itemize}

%%%%%%%%%%%%%%%%%%%%%%%%%%%%%%%%%%%%%%%%
\paragraph{v1.0:} 2017/04/27

\begin{itemize}
\item
manual and install package
\item
first version published on CTAN
\end{itemize}

%%%%%%%%%%%%%%%%%%%%%%%%%%%%%%%%%%%%%%%%
\paragraph{v0.6:} 2017/04/26

\begin{itemize}
\item
redirection mechanism added
\end{itemize}

%%%%%%%%%%%%%%%%%%%%%%%%%%%%%%%%%%%%%%%%
\paragraph{v0.5:} 2017/04/26

\begin{itemize}
\item
functionality in definition file
\end{itemize}


%%%%%%%%%%%%%%%%%%%%%%%%%%%%%%%%%%%%%%%%%%%%%%%%%%%%%%%%%%%%%%%%%%%%%%%%%%%%%%%%
%%%%%%%%%%%%%%%%%%%%%%%%%%%%%%%%%%%%%%%%%%%%%%%%%%%%%%%%%%%%%%%%%%%%%%%%%%%%%%%%
%%%%%%%%%%%%%%%%%%%%%%%%%%%%%%%%%%%%%%%%%%%%%%%%%%%%%%%%%%%%%%%%%%%%%%%%%%%%%%%%
\appendix

\settowidth\MacroIndent{\rmfamily\scriptsize 000\ }

 \DocInput{childdoc.dtx}

\end{document}
%</driver>
% \fi
%
% %%%%%%%%%%%%%%%%%%%%%%%%%%%%%%%%%%%%%%%%%%%%%%%%%%%%%%%%%%%%%%%%%%%%%%%%%%%%%%
% %%%%%%%%%%%%%%%%%%%%%%%%%%%%%%%%%%%%%%%%%%%%%%%%%%%%%%%%%%%%%%%%%%%%%%%%%%%%%%
% \section{Sample}
%\iffalse
%<*samplemain>
%\fi
%
% The following presents a sample document
% with two chapters, two parts, a title page,
% a compile flag as well as three forwarding files to set the flag.
% It consists of eight |.tex| files:
% \begin{center}
% \begin{tabular}{ll}
% |cdocsamp.tex|&main file\\
% |cdocsch1.tex|&include file for chapter 1\\
% |cdocsch2.tex|&include file for chapter 2\\
% |cdocspt3.tex|&include file for part 3\\
% |cdocspt4.tex|&include file for part 4\\
% |cdocsdrf.tex|&forwarding file for main file in draft mode\\
% |cdocsfi1.tex|&forwarding file for final version of chapter 1\\
% |cdocsfi2.tex|&forwarding file for final version of chapter 2\\
% \end{tabular}
% \end{center}
% Each of the eight files can be compiled directly by the \LaTeX{} compiler.
%
% %%%%%%%%%%%%%%%%%%%%%%%%%%%%%%%%%%%%%%
% \paragraph{Main File.}
%
% The main file is called |cdocsamp.tex|.
%
% Load the \textsf{childdoc} definitions and
% declare the filename for the main document:
%    \begin{macrocode}
\input{childdoc.def}
\childdocmain{}
%    \end{macrocode}

% Optional override for |\version| flag:
%    \begin{macrocode}
%%\ifchilddoc\else\providecommand{\version}{draft}\fi
%    \end{macrocode}

% Define the default values for the |\version| flag
% (|final| for the main file and |draft| for childs):
%    \begin{macrocode}
\ifchilddoc
\providecommand{\version}{draft}
\else
\providecommand{\version}{final}
\fi
%    \end{macrocode}

% Load the standard document class:
%    \begin{macrocode}
\documentclass[12pt]{article}
%    \end{macrocode}

% Start the document body:
%    \begin{macrocode}
\begin{document}
%    \end{macrocode}

% Declare a title page.
% Print title, part of document being processed and version flag:
%    \begin{macrocode}
\addtocounter{page}{-1}
\begin{center}
{\LARGE\bfseries{}childdoc example\par}
\vspace{1cm}
\ifchilddoc
\ifchilddocmanual part\else chapter\fi:
`\childdocname' of `\childdocjob'\par
\else
main document: `\childdocjob'\par
\fi
version: \version\par
\end{center}
\newpage
%    \end{macrocode}

% Manually include selected file,
% otherwise process as usual:
%    \begin{macrocode}
\ifchilddocmanual
\section*{part `\childdocname'}
\input{\childdocname}
\else
%    \end{macrocode}

% Include the two chapters:
%    \begin{macrocode}
\include{cdocsch1}
\include{cdocsch2}
%    \end{macrocode}

% Include the two parts unless only chapters should be displayed:
%    \begin{macrocode}
\ifchilddoc\else
\section{part three}
\input{cdocspt3}
\section{part four}
\input{cdocspt4}
\fi
%    \end{macrocode}

% Process as usual until here:
%    \begin{macrocode}
\fi
%    \end{macrocode}

% End of document body:
%    \begin{macrocode}
\end{document}
%    \end{macrocode}
%\iffalse
%</samplemain>
%\fi
%
% %%%%%%%%%%%%%%%%%%%%%%%%%%%%%%%%%%%%%%
% \paragraph{Chapter Include Files.}
%
% The include files are called |cdocsch1.tex| and |cdocsch2.tex|.
%
%\iffalse
%<*samplechap1|samplechap2>
%\fi

% Optional override for |\version| flag:
%    \begin{macrocode}
%%\providecommand{\version}{final}
%    \end{macrocode}

% Include the main document:
%    \begin{macrocode}
\input{childdoc.def}
\childdocof{cdocsamp}
%    \end{macrocode}

%\iffalse
%</samplechap1|samplechap2>
%\fi
%
%\iffalse
%<*samplechap1>
%\fi
% Some text for chapter 1:
%    \begin{macrocode}
\section{one}
some text in chapter one
%    \end{macrocode}

%\iffalse
%</samplechap1>
%\fi
% Some text for chapter 2:
%\iffalse
%<*samplechap2>
%\fi
%    \begin{macrocode}
\section{two}
more text in chapter two
%    \end{macrocode}

%\iffalse
%</samplechap2>
%\fi
%
% %%%%%%%%%%%%%%%%%%%%%%%%%%%%%%%%%%%%%%
% \paragraph{Part Include Files.}
%
% The include files are called |cdocspt3.tex| and |cdocspt4.tex|.
%
%\iffalse
%<*samplepart3|samplepart4>
%\fi

% Optional override for |\version| flag:
%    \begin{macrocode}
%%\providecommand{\version}{final}
%    \end{macrocode}

% Include the main document:
%    \begin{macrocode}
\input{childdoc.def}
\childdocby{cdocsamp}
%    \end{macrocode}

%\iffalse
%</samplepart3|samplepart4>
%\fi
%
%\iffalse
%<*samplepart3>
%\fi
% Some text for part 3:
%    \begin{macrocode}
some text in part three
%    \end{macrocode}

%\iffalse
%</samplepart3>
%\fi
% Some text for part 4:
%\iffalse
%<*samplepart4>
%\fi
%    \begin{macrocode}
more text in part four
%    \end{macrocode}

%\iffalse
%</samplepart4>
%\fi
%
% %%%%%%%%%%%%%%%%%%%%%%%%%%%%%%%%%%%%%%
% \paragraph{Forwarding for a Complete Draft.}
%
% The following forwarding file |cdocsdrf.tex|
% compiles the main document in draft mode:
%\iffalse
%<*sampledraft>
%\fi
%    \begin{macrocode}
\def\version{draft}
\input{childdoc.def}
\childdocforward{cdocsamp}
%    \end{macrocode}

%\iffalse
%</sampledraft>
%\fi
%
% %%%%%%%%%%%%%%%%%%%%%%%%%%%%%%%%%%%%%%
% \paragraph{Forwarding for Final Version of the Chapters.}
%
% The following forwarding files |cdocsfn1.tex| and |cdocsfn2.tex|
% (with identical content)
% compile the final versions of the child documents
% |cdocsch1.tex| and |cdocsch2.tex|, respectively:
%\iffalse
%<*samplefinal>
%\fi
%    \begin{macrocode}
\def\version{final}
\input{childdoc.def}
\childdocforwardprefix[cdocsamp]{cdocsfn}{cdocsch}
%    \end{macrocode}

%\iffalse
%</samplefinal>
%\fi
%
% %%%%%%%%%%%%%%%%%%%%%%%%%%%%%%%%%%%%%%
% \paragraph{Command Line Processing.}
%
% The following three command lines generate the output files
% |cdocscld|, |cdocscl1| and |cdocscl2|
% which should be identical to
% |cdocsdrf|, |cdocsch1| and |cdocsfn2|, respectively:
% \begin{center}
% \begin{tabular}{l}
% |latex -jobname cdocscld \|\\
% |  "\def\version{draft}\input{childdoc.def}\childdocforward{cdocsamp}"|\\
% |latex -jobname cdocscl1 \|\\
% |  "\input{childdoc.def}\childdocforward[cdocsamp]{cdocsch1}"|\\
% |latex -jobname cdocscl2 \|\\
% |  "\def\version{final}\input{childdoc.def}\childdocforward{cdocsch2}"|
% \end{tabular}
% \end{center}
% Note that the trailing backslash on each first line
% merely continues the input to the second line
% (for convenient cut ant paste).
% Furthermore, the command |latex| can be replaced by any
% of its alternative versions such as |pdflatex|.
%
% %%%%%%%%%%%%%%%%%%%%%%%%%%%%%%%%%%%%%%%%%%%%%%%%%%%%%%%%%%%%%%%%%%%%%%%%%%%%%%
% %%%%%%%%%%%%%%%%%%%%%%%%%%%%%%%%%%%%%%%%%%%%%%%%%%%%%%%%%%%%%%%%%%%%%%%%%%%%%%
% \section{Implementation}
%\iffalse
%<*package>
%\fi
%
% This section describes the definitions file |childdoc.def|.

% The definitions cannot be loaded using |\usepackage| or |\RequirePackage|
% which has a mechanism to prevent loading a style file more than once.
% When loading the definitions by means of |\input|
% multiple instances have to be prevented manually:
%\iffalse
%This code needs to be before the `\ProvidesFile' directive
%which is defined at the beginning of this file.
%Therefore it is also placed there and commented out here.
%</package>
%<*discard>
%\fi
%    \begin{macrocode}
\ifdefined\childdocmain\endinput\fi
%    \end{macrocode}
%\iffalse
%</discard>
%<*package>
%\fi
%
% \macro{\ifchilddoc}
% \macro{\ifchilddocmanual}
% The conditional |\ifchilddoc| tells whether a
% child (true) or main (false) document is being compiled.
% The conditional |\ifchilddocmanual| tells whether
% the |\includeonly| mechanism is used (false) or
% the selection of child files must be performed manually (true).
% The definitions initialise to false:
%    \begin{macrocode}
\newif\ifchilddoc
\newif\ifchilddocmanual
%    \end{macrocode}

% \macro{\childdocname}
% \macro{\childdocjob}
% The macro |\childdocname| stores the name of the main document
% to be compiled. The macro |\childdocjob| stores the name of
% the document on which the \LaTeX{} compiler was originally invoked.
% The content of |\jobname| cannot be compared
% to filenames specified in the source due to different catcodes.
% The following code rescans |\jobname|, stores the result
% in |\childdocname| and saves a copy in |\childdocjob|:
%    \begin{macrocode}
\edef\childdocname{\scantokens\expandafter{\jobname\noexpand}}
\let\childdocjob\childdocname
%    \end{macrocode}

% \macro{\childdocdisable}
% The macro |\childdocdisable| prevents the main file
% from being processed more than once.
% At this stage, the main document command |\childdocmain|
% is assumed to be called once again where it should do nothing.
% Any subsequent call to it should prevent
% a secondary processing of the main document
% It overwrites the forwarding commands
% |\childdocof| and |\childdocforward|
% with empty macros to prevent further inclusions of the main document:
%    \begin{macrocode}
\newcommand{\childdocdisable}
{
  \renewcommand{\childdocmain}[1]{\renewcommand{\childdocmain}[1]{\endinput}}
  \renewcommand{\childdocof}[1]{}
  \renewcommand{\childdocby}[2][]{}
  \renewcommand{\childdocforward}[2][]{}
  \renewcommand{\childdocdisable}{}
}
%    \end{macrocode}

% \macro{\childdocmain}
% The macro |\childdocmain| is to be called at the top of the main file
% with nothing or the main filename (without extension) as argument.
% First, it breaks loops.
% If the argument is not empty and does not match |\childdocname|
% (which is set by the first inclusion of |childdoc.def|),
% |\ifchilddoc| is set to true, |\includeonly| is applied to the child file
% and |\jobname| is set to the main file
% (for proper handling of |.aux| files):
%    \begin{macrocode}
\newcommand{\childdocmain}[1]
{
  \childdocdisable\childdocmain{}
  \if?#1?\else
    \begingroup
      \def\childdoctmp{#1}
      \ifx\childdoctmp\childdocname
        \def\childdoctmp{}
      \else
        \def\childdoctmp
        {
          \childdoctrue
          \includeonly{\childdocname}
          \def\childdocjob{#1}
          \def\jobname{#1}
        }
      \fi
      \expandafter
    \endgroup
    \childdoctmp
  \fi
}
%    \end{macrocode}

% \macro{\childdocof}
% The command |\childdocof| redirects
% compilation to the main file |#1|.
%    \begin{macrocode}
\newcommand{\childdocof}[1]
{
  \childdocdisable
  \childdoctrue
  \includeonly{\childdocname}
  \def\jobname{#1}
  \def\childdocjob{#1}
  \input{#1}
}
%    \end{macrocode}

% \macro{\childdocby}
% The command |\childdocby| ....
%    \begin{macrocode}
\newcommand{\childdocby}[2][]
{
  \childdocdisable
  \childdoctrue
  \childdocmanualtrue
  \if?#1?\else
    \def\jobname{#2}
  \fi
  \def\childdocjob{#2}
  \input{#2}
  \endinput
}
%    \end{macrocode}

% \macro{\childdocforward}
% The command |\childdocforward| redirects
% compilation to the main file or
% (if the optional argument is given) a child file.
% Parameters are set as if the main file
% or a child file starting with |\childdocof| was compiled.
% Then compilation is handed over to the main file:
%    \begin{macrocode}
\newcommand{\childdocforward}[2][]
{
  \begingroup
    \if?#1?
      \def\childdoctmp
      {
        \def\childdocname{#2}
        \def\childdocjob{#2}
        \def\jobname{#2}
        \input{#2}
        \endinput
      }
    \else
      \def\childdoctmp
      {
        \childdocdisable
        \def\childdocname{#2}
        \childdoctrue
        \includeonly{#2}
        \def\childdocjob{#1}
        \def\jobname{#1}
        \input{#1}
        \endinput
      }
    \fi
    \expandafter
  \endgroup
  \childdoctmp
}
%    \end{macrocode}

% \macro{\childdocforwardprefix}
% The command |\childdocforwardprefix| redirects
% compilation to the main or a child file by means of a pattern.
% The prefix |#1| in the current filename is replaced by |#2|
% and the suffix of the current filename is kept
% (it is assumed that the filename does not contain the substring `|~~~|'
% which is used as a delimiter).
% Compilation is handed over to the new file by |\childdocforward|:
%    \begin{macrocode}
\newcommand{\childdocforwardprefix}[3][]
{
  \begingroup
    \def\childdocextract #2##1~~~{\def\childdoctmp{\childdocforward[#1]{#3##1}}}
    \expandafter\childdocextract\childdocname~~~
    \expandafter
  \endgroup
  \childdoctmp
}
%    \end{macrocode}

% \macro{\childdoc}
% The deprecated macro |\childdoc| is a legacy version of |\childdocmain|:
%    \begin{macrocode}
\newcommand{\childdoc}{\childdocmain}
%    \end{macrocode}

% \macro{\childdocredirect}
% The deprecated macro |\childdocredirect| is a legacy version
% of |\childdocforward| and |\childdocforwardprefix|:
%    \begin{macrocode}
\newcommand{\childdocredirect}[2][]
{
  \begingroup
    \if?#1?
      \def\childdoctmp{\childdocforward{#2}}
    \else
      \def\childdoctmp{\childdocforwardprefix{#1}{#2}}
    \fi
    \expandafter
  \endgroup
  \childdoctmp
}
%    \end{macrocode}

%\iffalse
%</package>
%\fi
%
\endinput
|\\
|\childdocforwardprefix{final}{child}|
\end{tabular}
\end{center}
%

Note that when several versions of a main file and/or of each child file
are to be generated, it may be convenient to set up a |Makefile| or
shell script to automatise the process.

%%%%%%%%%%%%%%%%%%%%%%%%%%%%%%%%%%%%%%%%%%%%%%%%%%%%%%%%%%%%%%%%%%%%%%%%%%%%%%%%
\subsection{Command Line Processing}
\label{sec:commandline}

The effect of redirection files can also be achieved by invoking
the \LaTeX{} compiler with a more elaborate command line.
Most conveniently this should be done as part
of a shell script or a |Makefile|.

When using \textsf{childdoc} in the main file, the following
command lines effectively perform a redirection
(note that depending on the shell being used,
backslashes may have to be doubled: `|\|' $\to$ `|\\|'):
%
\begin{center}
|... -jobname "|\textit{target}|" |\\|"|[\textit{flags}]%
|% \iffalse
%
% childdoc.dtx Copyright (C) 2017-2018 Niklas Beisert
%
% This work may be distributed and/or modified under the
% conditions of the LaTeX Project Public License, either version 1.3
% of this license or (at your option) any later version.
% The latest version of this license is in
%   http://www.latex-project.org/lppl.txt
% and version 1.3 or later is part of all distributions of LaTeX
% version 2005/12/01 or later.
%
% This work has the LPPL maintenance status `maintained'.
%
% The Current Maintainer of this work is Niklas Beisert.
%
% This work consists of the files childdoc.dtx and childdoc.ins
% and the derived files childdoc.def and cdocsamp.tex with
% cdocsch1.tex, cdocsch2.tex, cdocsdrf.tex, cdocsfn1.tex, cdocsfn2.tex.
%
%<package>\ifdefined\childdocmain\endinput\fi
%<package>\ProvidesFile{childdoc.def}[2018/12/30 v2.0 child document driver]
%<samplemain>\ProvidesFile{cdocsamp.tex}[2018/12/30 v2.0 sample for childdoc]
%<*driver>
%\ProvidesFile{childdoc.drv}[2018/12/30 v2.0 childdoc reference manual file]
\PassOptionsToClass{10pt,a4paper}{article}
\documentclass{ltxdoc}

\usepackage[margin=35mm]{geometry}
\usepackage{hyperref}
\usepackage{hyperxmp}
\usepackage[usenames]{color}

\hypersetup{colorlinks=true}
\hypersetup{pdfstartview=FitH}
\hypersetup{pdfpagemode=UseNone}
\hypersetup{pdfsource={}}
\hypersetup{pdflang={en-UK}}
\hypersetup{pdfcopyright={Copyright 2017-2018 Niklas Beisert.
  This work may be distributed and/or modified under the
  conditions of the LaTeX Project Public License, either version 1.3
  of this license or (at your option) any later version.}}
\hypersetup{pdflicenseurl={http://www.latex-project.org/lppl.txt}}
\hypersetup{pdfcontactaddress={ETH Zurich, ITP, HIT K,
  Wolfgang-Pauli-Strasse 27}}
\hypersetup{pdfcontactpostcode={8093}}
\hypersetup{pdfcontactcity={Zurich}}
\hypersetup{pdfcontactcountry={Switzerland}}
\hypersetup{pdfcontactemail={nbeisert@itp.phys.ethz.ch}}
\hypersetup{pdfcontacturl={http://people.phys.ethz.ch/\xmptilde nbeisert/}}

\newcommand{\secref}[1]{\hyperref[#1]{section \ref*{#1}}}

\parskip1ex
\parindent0pt
\let\olditemize\itemize
\def\itemize{\olditemize\parskip0pt}

\begin{document}

\title{The \textsf{childdoc} Package}
\hypersetup{pdftitle={The childdoc Package}}
\author{Niklas Beisert\\[2ex]
  Institut f\"ur Theoretische Physik\\
  Eidgen\"ossische Technische Hochschule Z\"urich\\
  Wolfgang-Pauli-Strasse 27, 8093 Z\"urich, Switzerland\\[1ex]
  \href{mailto:nbeisert@itp.phys.ethz.ch}
  {\texttt{nbeisert@itp.phys.ethz.ch}}}
\hypersetup{pdfauthor={Niklas Beisert}}
\hypersetup{pdfsubject={Manual for the LaTeX2e Package childdoc}}
\date{30 December 2018, \textsf{v2.0}}
\maketitle

\begin{abstract}\noindent
\textsf{childdoc} is a \LaTeXe{} package
that enables the direct compilation
of document sections included by |\include|
to individual files.
\end{abstract}

\begingroup
\parskip0ex
\tableofcontents
\endgroup

%%%%%%%%%%%%%%%%%%%%%%%%%%%%%%%%%%%%%%%%%%%%%%%%%%%%%%%%%%%%%%%%%%%%%%%%%%%%%%%%
%%%%%%%%%%%%%%%%%%%%%%%%%%%%%%%%%%%%%%%%%%%%%%%%%%%%%%%%%%%%%%%%%%%%%%%%%%%%%%%%
\section{Introduction}

\LaTeX{} provides a mechanism to structure a large document (such as a book)
into a main file and several child files (containing the chapters)
using the |\include| command.
This mechanism is beneficial for documents
which span hundreds of pages in order to
make the source file(s) more manageable.
Moreover, compilation can be restricted to
selected child files by means of the |\includeonly| command.
The latter feature can be used to reduce the compilation time while editing
(this was significantly more useful in the earlier days of \LaTeX{})
or to generate a smaller document which is easier to navigate.
Another application of |\includeonly| is to generate
documents consisting of selected parts of the complete document.

However, there are a few drawbacks of the plain |\include| mechanism:
\begin{itemize}
\item
The child files cannot be compiled on their own,
they can only be compiled via the main file.
A naive editing environment
(such as a text editor with an option
to have the current file processed by \LaTeX)
may require one to switch to the main file before compiling;
attempting to compile the child file produces errors.
\item
The main file must be modified (each time)
to adjust the |\includeonly| command
to the present needs. This easily leaves the main file in a messy state.
\item
The generated document will always carry the filename
of the main document. This is inconvenient if
several child files are to be compiled and
to be kept for distribution.
\end{itemize}

The present package provides a simple interface
to make child files individually compilable by \LaTeX{}.
Compiling a child file then has the same effect as compiling
the main file with an |\includeonly| command
to select the appropriate child.
Moreover the generated document will carry the name of the child
rather than the main file.
This resolves all three above issues.

This feature is meant to make the editing of books,
thesis documents and lecture notes somewhat more convenient.
However, the package can also be used efficiently for
composing a series of documents (such as exercise sheets)
which are typically distributed individually.
It then assists the author in generating the individual documents
(potentially in different versions)
as well as a document containing the collected series.
Another application is in developing style files
or other kinds of included material
where compilation of the style file could redirect
to a sample or test file.

%%%%%%%%%%%%%%%%%%%%%%%%%%%%%%%%%%%%%%%%%%%%%%%%%%%%%%%%%%%%%%%%%%%%%%%%%%%%%%%%
%%%%%%%%%%%%%%%%%%%%%%%%%%%%%%%%%%%%%%%%%%%%%%%%%%%%%%%%%%%%%%%%%%%%%%%%%%%%%%%%
\section{Usage}

First of all, the package \textsf{childdoc} is \emph{not} a standard
\LaTeXe{} |.sty| style file! Therefore it needs to be invoked in
a non-standard way.

%%%%%%%%%%%%%%%%%%%%%%%%%%%%%%%%%%%%%%%%%%%%%%%%%%%%%%%%%%%%%%%%%%%%%%%%%%%%%%%%
\subsection{Included Files}
\label{sec:include}

%%%%%%%%%%%%%%%%%%%%%%%%%%%%%%%%%%%%%%%%
\DescribeMacro{\childdocmain}
To use the package, add the commands
\begin{center}
\begin{tabular}{l}
|\input{childdoc.def}|\\
|\childdocmain{}|\\
\end{tabular}
\end{center}
at the very top of the main \LaTeX{} file,
in particular \emph{before} the |\documentclass| statement!
The argument of |\childdocmain| should be left empty
(but it must be present).

%%%%%%%%%%%%%%%%%%%%%%%%%%%%%%%%%%%%%%%%
\DescribeMacro{\childdocof}
Furthermore, add the commands
\begin{center}
\begin{tabular}{l}
|\input{childdoc.def}|\\
|\childdocof{|\textit{main}|}|\\
\end{tabular}
\end{center}
at the top of every child file \textit{child}
which is included by |\include{|\textit{child}|}|
from within the main file
(or at least for those files to be compiled individually).
The argument \textit{main} must be the filename of the main file.

There are a couple of
considerations in setting up the main and child documents:

%%%%%%%%%%%%%%%%%%%%%%%%%%%%%%%%%%%%%%%%
\paragraph{Restrictions.}

Please note the following restrictions:
\begin{itemize}
\item
|\childdocmain| must be called with one argument \textit{main}
to ensure compatibility with earlier version of the package.
It must either be empty (|\childdocmain{}|)
or precisely match the filename of the main file in which it is specified.
See \secref{sec:detection} for further information.
\item
The filename \textit{main} must be specified without the |.tex| extension.
\item
The filename \textit{main} is case sensitive
(even in case-insensitive file systems)
due to internal string comparison.
\item
The argument \textit{main} should be fully expanded, it cannot be a macro.
\item
Subdirectories and special characters should be avoided in filenames.
\item
The command |\childdocmain{|\textit{main}|}| must be followed by a whitespace.
It should not be followed immediately by another command
or by a comment mark `|%|'.
This is because the \TeX{} parser reads the token immediately following
the argument of |\childdocmain| and puts it
at the beginning of every child section;
however, a white\-space is ignored.
\end{itemize}

%%%%%%%%%%%%%%%%%%%%%%%%%%%%%%%%%%%%%%%%
\paragraph{Content of Main File.}

It is advisable to place all content in the child files included by |\include|.
Any output contained in the main file will appear in all child documents
unless suppressed manually;
it cannot be suppressed automatically by the |\includeonly| directive
and thus should normally be avoided.
A method to include some content in the main file
by means of conditional processing is described in \secref{sec:conditional}.

%%%%%%%%%%%%%%%%%%%%%%%%%%%%%%%%%%%%%%%%
\paragraph{Page Numbering.}

When only a part of the document is compiled,
the appropriate numbering of pages
(as well as other status parameters)
is determined from the |.aux| files.
The latter contain information from previous passes.
However this information needs to propagate through
all intermediate child documents.
Therefore the page numbering in child documents may well
be inconsistent until the complete document is compiled at least once.

A useful (if unconventional) way to always ensure a consistent
page numbering is to restart the numbering in each child document
and denote the pages by `\textit{child}|.|\textit{page}'
where \textit{child} represents the chapter/section number of the child file.
This can be achieved by the command
|\numberwithin{page}{|\textit{child}|}|
of the \textsf{amsmath} package
where \textit{child} can be |chapter| or |section|
depending on the chosen structuring.
Alternatively, one can modify the macro |\thepage| appropriately
and reset the counter |page| at the start of each child file.

%%%%%%%%%%%%%%%%%%%%%%%%%%%%%%%%%%%%%%%%%%%%%%%%%%%%%%%%%%%%%%%%%%%%%%%%%%%%%%%%
\subsection{Conditional Processing}
\label{sec:conditional}

The package provides a mechanism to compile different versions
of a document. To customise the versions further some conditional processing
can come in handy to distinguish which version is being compiled.
The package provides two macros to describe the compilation context:

%%%%%%%%%%%%%%%%%%%%%%%%%%%%%%%%%%%%%%%%
\DescribeMacro{\ifchilddoc}
The conditional |\ifchilddoc| distinguishes between the compilation of
child documents and the main document:
%
\begin{center}
|\ifchilddoc |\textit{child-code}| |[|\||else |\textit{main-code}]| \||fi|
\end{center}

%%%%%%%%%%%%%%%%%%%%%%%%%%%%%%%%%%%%%%%%
\DescribeMacro{\childdocname}
\DescribeMacro{\childdocjob}
The macro |\childdocname| contains the filename (without extension)
of the main or child file being processed.
Note that |\childdocjob| will always contain the name of the main file.

%%%%%%%%%%%%%%%%%%%%%%%%%%%%%%%%%%%%%%%%
\paragraph{Title Page.}

Conditional processing can be used to include a title or banner page
in the main document when proper precautions are taken.
Importantly, the code in the main file should ensure that the page counter
(as well as other status parameters which are stored in the |.aux| files)
takes the same value after the conditional processing.
Otherwise the page numbers may take divergent values
depending on which part is compiled.

For example, a title page could be declared by:
%
\begin{center}
\begin{tabular}{l}
|\ifchilddoc\||else|\\
|\addtocounter{page}{-1}|\\
\textit{code for title page}\\
|\newpage|\\
|\||fi|
\end{tabular}
\end{center}
%
A banner page for the child documents can be generated by:
%
\begin{center}
\begin{tabular}{l}
|\ifchilddoc|\\
|\addtocounter{page}{-1}|\\
\textit{code for banner page}\\
|\newpage|\\
|\||fi|
\end{tabular}
\end{center}
%
Here one could write a message such as:
\begin{center}
|This is the part \childdocname{} of \childdocjob{}.|
\end{center}

%%%%%%%%%%%%%%%%%%%%%%%%%%%%%%%%%%%%%%%%%%%%%%%%%%%%%%%%%%%%%%%%%%%%%%%%%%%%%%%%
\subsection{Flags}
\label{sec:flags}

The package makes it easy to generate different versions
of the main or child documents.
To this end compilation flags can be defined
and assigned different default values.
They will be particularly useful in conjunction
with the forwarding mechanism described in \secref{sec:forward}.

For example, it may be useful to have a flag |\version|
which can be set to |draft| or |final|.
The document source will contain some conditional code
depending on the value of |\version|.
Suppose further, the flag should default to |final| for the main file
and to |draft| for child files
which is a natural assignment for editing the document.
This is achieved by placing the following code
in the preamble of the main document
(below the |\childdocmain| directive):
%
\begin{center}
\begin{tabular}{l}
|\ifchilddoc|\\
|\providecommand{\version}{draft}|\\
|\||else|\\
|\providecommand{\version}{final}|\\
|\||fi|
\end{tabular}
\end{center}
%
The definition by |\providecommand| makes sure
that previous definitions are not overwritten.
Further statements |\providecommand{\version}{...}|
can thus be added before the above code to override it.

For the main file, one might add a line
(between |\childdocmain| and the above block)
%
\begin{center}
|%\ifchilddoc\||else\providecommand{\version}{draft}\||fi|
\end{center}
%
which can be uncommented to produce a draft version.
Likewise one can add a line to the very top of a child file
(above the |\childdocof{|\textit{main}|}| directive)
%
\begin{center}
|%\providecommand{\version}{final}|
\end{center}
%
which can be uncommented to produce the final version of this child document.

%%%%%%%%%%%%%%%%%%%%%%%%%%%%%%%%%%%%%%%%%%%%%%%%%%%%%%%%%%%%%%%%%%%%%%%%%%%%%%%%
\subsection{Forwarding}
\label{sec:forward}

Different versions of the main or child documents
using compilation flags as described in \secref{sec:flags}
can be (permanently) stored in different files
for convenient compilation, viewing and distribution.
To this end, the package defines a command
to pass on compilation to a different file:

%%%%%%%%%%%%%%%%%%%%%%%%%%%%%%%%%%%%%%%%
\DescribeMacro{\childdocforward}
The command |\childdocforward| redirects processing to
another source file:
%
\begin{center}
\begin{tabular}{l}
|\input{childdoc.def}|\\
|\childdocforward[|\textit{main}|]{|\textit{dest}|}|\\
\end{tabular}
\end{center}
%
The argument \textit{dest} is the destination file
(without extension).
It should be the main file or one of the child files.
Note that further \textsf{childdoc} directives
such as |\childdocof| and |\childdocforward|
in the indicated file will be processed in this form.
The optional argument \textit{main}
passes on directly to the main file \textit{main}
while pretending to compile the child \textit{dest}.
This form behaves as if \textit{dest}
issues |\childdocof{|\textit{main}|}| right away,
and no further \textsf{childdoc} directives will be processed.

%%%%%%%%%%%%%%%%%%%%%%%%%%%%%%%%%%%%%%%%
\DescribeMacro{\...prefix}
In the alternative form |\childdocforwardprefix|,
%
\begin{center}
\begin{tabular}{l}
|\input{childdoc.def}|\\
|\childdocforwardprefix[|\textit{main}|]{|\textit{prefix}|}{|\textit{dest}|}|
\end{tabular}
\end{center}
%
the destination file is determined by a pattern
depending on the current file:
To make this work, the current file must be called
`{\textit{prefix}\hspace{0.2em}\textit{suffix}}'
with \textit{prefix} matching precisely the argument.
Processing is then passed on to the file
`{\textit{dest}\hspace{0.2em}\textit{suffix}}'.
Surely, the same effect is achieved by
directly specifying the
argument `{\textit{dest}\hspace{0.2em}\textit{suffix}}'
in the first form.
However, that requires to set up a different file
for each child. With the alternative form of the command
all these files can have exactly the same content
which simplifies setting them up and maintaining them.

For example, the following file |draft.tex|
with a compilation flag |\version| as described in \secref{sec:flags}
compiles the main document as a draft:
%
\begin{center}
\begin{tabular}{l}
|\def\version{draft}|\\
|\input{childdoc.def}|\\
|\childdocforward{|\textit{main}|}|
\end{tabular}
\end{center}
%
Likewise, the following files |final|\textit{nn}|.tex|
compile the final version of the child document
|child|\textit{nn}|.tex|:
%
\begin{center}
\begin{tabular}{l}
|\def\version{final}|\\
|\input{childdoc.def}|\\
|\childdocforwardprefix{final}{child}|
\end{tabular}
\end{center}
%

Note that when several versions of a main file and/or of each child file
are to be generated, it may be convenient to set up a |Makefile| or
shell script to automatise the process.

%%%%%%%%%%%%%%%%%%%%%%%%%%%%%%%%%%%%%%%%%%%%%%%%%%%%%%%%%%%%%%%%%%%%%%%%%%%%%%%%
\subsection{Command Line Processing}
\label{sec:commandline}

The effect of redirection files can also be achieved by invoking
the \LaTeX{} compiler with a more elaborate command line.
Most conveniently this should be done as part
of a shell script or a |Makefile|.

When using \textsf{childdoc} in the main file, the following
command lines effectively perform a redirection
(note that depending on the shell being used,
backslashes may have to be doubled: `|\|' $\to$ `|\\|'):
%
\begin{center}
|... -jobname "|\textit{target}|" |\\|"|[\textit{flags}]%
|\input{childdoc.def}\childdocforward[|\textit{main}|]{|\textit{dest}|}"|
\end{center}
%
Here \textit{target} is the name of the output file,
\textit{main} is the name of the main file
and \textit{dest} is the name of the main or child file to be processed
(all filenames without extensions).
The optional argument \textit{main} can be omitted
if \textit{main} matches \textit{dest}.
Optionally, compilation \textit{flags} can be defined via |\def| commands.
This command line makes the \TeX{} engine believe
it is compiling the file \textit{target}
whose content is specified as the latter parameter.
The provided code then forwards the processing to
\textit{main} or \textit{dest} as described in \secref{sec:forward}.

%%%%%%%%%%%%%%%%%%%%%%%%%%%%%%%%%%%%%%%%%%%%%%%%%%%%%%%%%%%%%%%%%%%%%%%%%%%%%%%%
\subsection{Include by Input}
\label{sec:input}

Including child documents by |\include| has some restrictions by design.
Most notably, the content of a child document always occupies
its own set of pages; pages cannot be shared between child documents.
Usually, this behaviour makes perfect sense
because each child document contain an essential part of the document.
However, in some situations it may be desirable to compose
a document from a collection of parts
without having mandatory page breaks between then.
For this case, the package
provides a mechanism to include parts
by |\input| which can also be processed individually.
However, by construction this mechanism
requires manual handling of the content to be output.

%%%%%%%%%%%%%%%%%%%%%%%%%%%%%%%%%%%%%%%%
\DescribeMacro{\ifchilddocmanual}
The main file should be prepared as usual, see \secref{sec:include}.
However, the document body must make a distinction
between processing of an individual part and of the main document, e.g.:
%
\begin{center}
\begin{tabular}{l}
|\ifchilddocmanual|\\
|\input{\childdocname}|\\
|\||else|\\
\textit{document body with }|\input{|\textit{part}|}|\\
|\||fi|
\end{tabular}
\end{center}
%
The conditional |\ifchilddocmanual| is true whenever
a part to be included by |\input| is being compiled,
and the name of the part is stored in |\childdocname|.

%%%%%%%%%%%%%%%%%%%%%%%%%%%%%%%%%%%%%%%%
\DescribeMacro{\childdocby}
Each part to be included by |\input| should start with:
%
\begin{center}
\begin{tabular}{l}
|\input{childdoc.def}|\\
|\childdocby{|\textit{main}|}|\\
\end{tabular}
\end{center}
%
The directive |\childdocby| is similar to |\childdocof|
described in \secref{sec:include},
but the subsequent selection of content must be done manually.
To that end, both |\ifchilddoc| and |\ifchilddocmanual|
will be true upon processing of a part,
and the name of the part is stored in |\childdocname|.
Note that |\jobname| will be set to the filename of the current part
so that each part receives an individual |.aux| file
that does not interfere with the |.aux| file(s) of the main document.
This behaviour can be altered by the alternative form
|\childdocby[*]{|\textit{main}|}| (with a non-empty optional argument)
which uses the |.aux| file of the main document
by setting |\jobname| to \textit{main}.

%%%%%%%%%%%%%%%%%%%%%%%%%%%%%%%%%%%%%%%%%%%%%%%%%%%%%%%%%%%%%%%%%%%%%%%%%%%%%%%%
\subsection{Driver Development}
\label{sec:driver}

The \textsf{childdoc} mechanism can also be use for the development
of definition files such as \LaTeX{} styles or classes.
This case differs from the above setup with multiple parts
included by |\include| in that no |\includeonly| should be invoked.
This can be achieved by starting the include file
(before |\ProvidesPackage|) with:
%
\begin{center}
\begin{tabular}{l}
|\input{childdoc.def}|\\
|\childdocforward{|\textit{main}|}|\\
\end{tabular}
\end{center}
%
or alternatively with:
%
\begin{center}
\begin{tabular}{l}
|\input{childdoc.def}|\\
|\childdocby{|\textit{main}|}|\\
\end{tabular}
\end{center}
%
Both forms have slightly different effects as described above.
The main file is prepared as usual, see \secref{sec:include}.

%%%%%%%%%%%%%%%%%%%%%%%%%%%%%%%%%%%%%%%%%%%%%%%%%%%%%%%%%%%%%%%%%%%%%%%%%%%%%%%%
\subsection{Legacy Detection}
\label{sec:detection}

The directive |\childdocmain| in the main file can detect
whether the complete document or merely a child is to be compiled
even without using the directive |\childdocof|.
This method is deprecated because it is less robust
and there is no compelling reason to use it;
it is merely provided for backward compatibility
and it may be removed in future versions.

If the detection mechanism is to be used,
it is mandatory to correctly specify
the filename of the main file as the argument of |\childdocmain|:
%
\begin{center}
\begin{tabular}{l}
|\input{childdoc.def}|\\
|\childdocmain{|\textit{main}|}|\\
\end{tabular}
\end{center}
%
If |\jobname| does not match the argument \textit{main} of |\childdocmain|,
it is assumed that |\jobname| points to the child file to be compiled.
When using |\childdocmain| with the main file specified as argument,
it suffices to start a child file
with just |\input{|\textit{main}|}|
without loading of the package and using |\childdocof|.
If instead all processing is done
with the appropriate \textsf{childdoc} directives,
the argument of \textit{main} of |\childdocmain| can be empty.

An alternative version of the command line processing described
in \secref{sec:commandline} using the detection mechanism reads:
%
\begin{center}
|... -jobname "|\textit{target}|" "|[\textit{flags}]%
[|\def\jobname{|\textit{dest}|}|]|\input{|\textit{main}|}"|
\end{center}

%%%%%%%%%%%%%%%%%%%%%%%%%%%%%%%%%%%%%%%%%%%%%%%%%%%%%%%%%%%%%%%%%%%%%%%%%%%%%%%%
\subsection{Manual Code}
\label{sec:manual}

In case one cannot be certain whether the definitions file |childdoc.def|
is installed on the target \TeX{} distribution
and one prefers not to ship it,
it is conceivable to paste a few relevant commands into the sources.

To that end, drop all statements |\input{childdoc.def}|
and perform the replacements as outlined below.
Instead of |\childdocmain{|\textit{main}|}| add the following code
to the top of the main file:
%
\begin{center}
\begin{tabular}{l}
|\||ifdefined\childdocname\endinput\||fi\newif\ifchilddoc|\\
|\edef\childdocname{\scantokens\expandafter{\jobname\noexpand}}|\\
|\def\childdocmain{|\textit{main}|}\||ifx\childdocmain\childdocname\||else|\\
|\childdoctrue\includeonly{\childdocname}\let\jobname\childdocmain\||fi|\\
\end{tabular}
\end{center}
%
Instead of |\childdocof{|\textit{main}|}| just include the main file
at the top of each child file:
%
\begin{center}
|\input{|\textit{main}|}|
\end{center}
%
A simple redirection |\childdocforward{|\textit{dest}|}| is achieved by:
%
\begin{center}
|\def\jobname{|\textit{dest}|}\input{\jobname}|
\end{center}
%
The redirection with prefix
|\childdocforwardprefix[|\textit{prefix}|]{|\textit{dest}|}|
is accomplished by:
%
\begin{center}
\begin{tabular}{l}
|{\edef\jobname{\scantokens\expandafter{\jobname\noexpand}}|\\
|\def\redirectjob |\textit{prefix}|#1~~~{\gdef\jobname{|\textit{dest}|#1}}|\\
|\expandafter\redirectjob\jobname~~~}\input{\jobname}|
\end{tabular}
\end{center}

In an alternative approach,
child documents can be compiled by a specific command line
without additional code or specific definitions:
%
\begin{center}
|... -jobname "|\textit{target}|" "|[\textit{flags}]%
|\includeonly{|\textit{dest}|}\input{|\textit{main}|}"|
\end{center}
%

%%%%%%%%%%%%%%%%%%%%%%%%%%%%%%%%%%%%%%%%%%%%%%%%%%%%%%%%%%%%%%%%%%%%%%%%%%%%%%%%
%%%%%%%%%%%%%%%%%%%%%%%%%%%%%%%%%%%%%%%%%%%%%%%%%%%%%%%%%%%%%%%%%%%%%%%%%%%%%%%%
\section{Information}

%%%%%%%%%%%%%%%%%%%%%%%%%%%%%%%%%%%%%%%%%%%%%%%%%%%%%%%%%%%%%%%%%%%%%%%%%%%%%%%%
\subsection{Copyright}

Copyright \copyright{} 2017--2018 Niklas Beisert

This work may be distributed and/or modified under the
conditions of the \LaTeX{} Project Public License, either version 1.3
of this license or (at your option) any later version.
The latest version of this license is in
  \url{http://www.latex-project.org/lppl.txt}
and version 1.3 or later is part of all distributions of \LaTeX{}
version 2005/12/01 or later.

This work has the LPPL maintenance status `maintained'.

The Current Maintainer of this work is Niklas Beisert.

This work consists of the files |README.txt|, |childdoc.ins| and |childdoc.dtx|
as well as the derived files |childdoc.def|, |cdocsamp.tex|
with |cdocsch1.tex|, |cdocsch2.tex|, |cdocspt3.tex|, |cdocspt4.tex|,
|cdocsdrf.tex|, |cdocsfn1.tex|, |cdocsfn2.tex|
as well as |childdoc.pdf|.

%%%%%%%%%%%%%%%%%%%%%%%%%%%%%%%%%%%%%%%%%%%%%%%%%%%%%%%%%%%%%%%%%%%%%%%%%%%%%%%%
\subsection{Files and Installation}

The package consists of the files:
%
\begin{center}
\begin{tabular}{ll}
    |README.txt|   & readme file \\
    |childdoc.ins| & installation file \\
    |childdoc.dtx| & source file \\
    |childdoc.def| & definition file \\
    |cdocsamp.tex| & sample main file \\
    |cdocsch1.tex| & sample include file \\
    |cdocsch2.tex| & sample include file \\
    |cdocspt3.tex| & sample part file \\
    |cdocspt4.tex| & sample part file \\
    |cdocsdrf.tex| & sample redirection file \\
    |cdocsfn1.tex| & sample redirection file \\
    |cdocsfn2.tex| & sample redirection file \\
    |childdoc.pdf| & manual
\end{tabular}
\end{center}
%
The distribution consists of the files
|README.txt|, |childdoc.ins| and |childdoc.dtx|.
%
\begin{itemize}
\item
Run (pdf)\LaTeX{} on |childdoc.dtx|
to compile the manual |childdoc.pdf| (this file).
\item
Run \LaTeX{} on |childdoc.ins| to create the definitions file |childdoc.def|
and the sample |cdocsamp.tex| with include files
|cdocsch1.tex|, |cdocsch2.tex|, |cdocspt3.tex|, |cdocspt4.tex|,
|cdocsdrf.tex|, |cdocsfn1.tex|, |cdocsfn2.tex|.
Then copy the file |childdoc.def| to an appropriate directory of your \LaTeX{}
distribution, e.g.\ \textit{texmf-root}|/tex/latex/childdoc|.
\end{itemize}

%%%%%%%%%%%%%%%%%%%%%%%%%%%%%%%%%%%%%%%%%%%%%%%%%%%%%%%%%%%%%%%%%%%%%%%%%%%%%%%%
\subsection{Related CTAN Packages}

There are several other packages which offer a similar functionality:
%
\begin{itemize}
\item
The packages
\href{http://ctan.org/pkg/docmute}{\textsf{docmute}},
\href{http://ctan.org/pkg/includex}{\textsf{includex}} and
\href{http://ctan.org/pkg/standalone}{\textsf{standalone}}
provide commands to include only the document body of
a child file thus allowing both files to be compiled individually.
\item
The packages \href{http://ctan.org/pkg/subdocs}{\textsf{subdocs}}
and \href{http://ctan.org/pkg/subfiles}{\textsf{subfiles}}
provide structures in which the main and child documents can be
encapsulated and allowing them to be compiled individually.
The inclusion mechanism is different from the conventional |\include|.
\item
The package \href{http://ctan.org/pkg/combine}{\textsf{combine}}
is an elaborate solution to combine several documents into one.
\end{itemize}
%
See also the CTAN topic \href{http://ctan.org/topic/subdocs}{\textsf{subdocs}}
for further related packages.
The present package differs from the above solutions in that
a document structure constructed with the conventional |\include| mechanism
just needs two extra commands at the top of every file
such that all constituent files can be compiled individually.

%%%%%%%%%%%%%%%%%%%%%%%%%%%%%%%%%%%%%%%%%%%%%%%%%%%%%%%%%%%%%%%%%%%%%%%%%%%%%%%%
%\subsection{Feature Suggestions}
%
%The following is a list of features which may be useful for future
%versions of this package:
%%
%\begin{itemize}
%\item
%\ldots
%\end{itemize}

%%%%%%%%%%%%%%%%%%%%%%%%%%%%%%%%%%%%%%%%%%%%%%%%%%%%%%%%%%%%%%%%%%%%%%%%%%%%%%%%
\subsection{Revision History}

%%%%%%%%%%%%%%%%%%%%%%%%%%%%%%%%%%%%%%%%
\paragraph{v2.0:} 2018/12/30

\begin{itemize}
\item
immediate forward processing
\item
added |\childdocby| mechanism
\item
manual restructured
\end{itemize}

%%%%%%%%%%%%%%%%%%%%%%%%%%%%%%%%%%%%%%%%
\paragraph{v1.6:} 2018/01/17

\begin{itemize}
\item
application for development of include files
\item
corrections to manual
\end{itemize}

%%%%%%%%%%%%%%%%%%%%%%%%%%%%%%%%%%%%%%%%
\paragraph{v1.5:} 2017/05/21

\begin{itemize}
\item
more complete structuring introduced
\item
|\childdocof| introduced
\item
|\childdoc| renamed to |\childdocmain|
\item
|\childredirect| renamed to |\childdocforward| and |\childdocforwardprefix|
and functionality expanded
\end{itemize}

%%%%%%%%%%%%%%%%%%%%%%%%%%%%%%%%%%%%%%%%
\paragraph{v1.0:} 2017/04/27

\begin{itemize}
\item
manual and install package
\item
first version published on CTAN
\end{itemize}

%%%%%%%%%%%%%%%%%%%%%%%%%%%%%%%%%%%%%%%%
\paragraph{v0.6:} 2017/04/26

\begin{itemize}
\item
redirection mechanism added
\end{itemize}

%%%%%%%%%%%%%%%%%%%%%%%%%%%%%%%%%%%%%%%%
\paragraph{v0.5:} 2017/04/26

\begin{itemize}
\item
functionality in definition file
\end{itemize}


%%%%%%%%%%%%%%%%%%%%%%%%%%%%%%%%%%%%%%%%%%%%%%%%%%%%%%%%%%%%%%%%%%%%%%%%%%%%%%%%
%%%%%%%%%%%%%%%%%%%%%%%%%%%%%%%%%%%%%%%%%%%%%%%%%%%%%%%%%%%%%%%%%%%%%%%%%%%%%%%%
%%%%%%%%%%%%%%%%%%%%%%%%%%%%%%%%%%%%%%%%%%%%%%%%%%%%%%%%%%%%%%%%%%%%%%%%%%%%%%%%
\appendix

\settowidth\MacroIndent{\rmfamily\scriptsize 000\ }

 \DocInput{childdoc.dtx}

\end{document}
%</driver>
% \fi
%
% %%%%%%%%%%%%%%%%%%%%%%%%%%%%%%%%%%%%%%%%%%%%%%%%%%%%%%%%%%%%%%%%%%%%%%%%%%%%%%
% %%%%%%%%%%%%%%%%%%%%%%%%%%%%%%%%%%%%%%%%%%%%%%%%%%%%%%%%%%%%%%%%%%%%%%%%%%%%%%
% \section{Sample}
%\iffalse
%<*samplemain>
%\fi
%
% The following presents a sample document
% with two chapters, two parts, a title page,
% a compile flag as well as three forwarding files to set the flag.
% It consists of eight |.tex| files:
% \begin{center}
% \begin{tabular}{ll}
% |cdocsamp.tex|&main file\\
% |cdocsch1.tex|&include file for chapter 1\\
% |cdocsch2.tex|&include file for chapter 2\\
% |cdocspt3.tex|&include file for part 3\\
% |cdocspt4.tex|&include file for part 4\\
% |cdocsdrf.tex|&forwarding file for main file in draft mode\\
% |cdocsfi1.tex|&forwarding file for final version of chapter 1\\
% |cdocsfi2.tex|&forwarding file for final version of chapter 2\\
% \end{tabular}
% \end{center}
% Each of the eight files can be compiled directly by the \LaTeX{} compiler.
%
% %%%%%%%%%%%%%%%%%%%%%%%%%%%%%%%%%%%%%%
% \paragraph{Main File.}
%
% The main file is called |cdocsamp.tex|.
%
% Load the \textsf{childdoc} definitions and
% declare the filename for the main document:
%    \begin{macrocode}
\input{childdoc.def}
\childdocmain{}
%    \end{macrocode}

% Optional override for |\version| flag:
%    \begin{macrocode}
%%\ifchilddoc\else\providecommand{\version}{draft}\fi
%    \end{macrocode}

% Define the default values for the |\version| flag
% (|final| for the main file and |draft| for childs):
%    \begin{macrocode}
\ifchilddoc
\providecommand{\version}{draft}
\else
\providecommand{\version}{final}
\fi
%    \end{macrocode}

% Load the standard document class:
%    \begin{macrocode}
\documentclass[12pt]{article}
%    \end{macrocode}

% Start the document body:
%    \begin{macrocode}
\begin{document}
%    \end{macrocode}

% Declare a title page.
% Print title, part of document being processed and version flag:
%    \begin{macrocode}
\addtocounter{page}{-1}
\begin{center}
{\LARGE\bfseries{}childdoc example\par}
\vspace{1cm}
\ifchilddoc
\ifchilddocmanual part\else chapter\fi:
`\childdocname' of `\childdocjob'\par
\else
main document: `\childdocjob'\par
\fi
version: \version\par
\end{center}
\newpage
%    \end{macrocode}

% Manually include selected file,
% otherwise process as usual:
%    \begin{macrocode}
\ifchilddocmanual
\section*{part `\childdocname'}
\input{\childdocname}
\else
%    \end{macrocode}

% Include the two chapters:
%    \begin{macrocode}
\include{cdocsch1}
\include{cdocsch2}
%    \end{macrocode}

% Include the two parts unless only chapters should be displayed:
%    \begin{macrocode}
\ifchilddoc\else
\section{part three}
\input{cdocspt3}
\section{part four}
\input{cdocspt4}
\fi
%    \end{macrocode}

% Process as usual until here:
%    \begin{macrocode}
\fi
%    \end{macrocode}

% End of document body:
%    \begin{macrocode}
\end{document}
%    \end{macrocode}
%\iffalse
%</samplemain>
%\fi
%
% %%%%%%%%%%%%%%%%%%%%%%%%%%%%%%%%%%%%%%
% \paragraph{Chapter Include Files.}
%
% The include files are called |cdocsch1.tex| and |cdocsch2.tex|.
%
%\iffalse
%<*samplechap1|samplechap2>
%\fi

% Optional override for |\version| flag:
%    \begin{macrocode}
%%\providecommand{\version}{final}
%    \end{macrocode}

% Include the main document:
%    \begin{macrocode}
\input{childdoc.def}
\childdocof{cdocsamp}
%    \end{macrocode}

%\iffalse
%</samplechap1|samplechap2>
%\fi
%
%\iffalse
%<*samplechap1>
%\fi
% Some text for chapter 1:
%    \begin{macrocode}
\section{one}
some text in chapter one
%    \end{macrocode}

%\iffalse
%</samplechap1>
%\fi
% Some text for chapter 2:
%\iffalse
%<*samplechap2>
%\fi
%    \begin{macrocode}
\section{two}
more text in chapter two
%    \end{macrocode}

%\iffalse
%</samplechap2>
%\fi
%
% %%%%%%%%%%%%%%%%%%%%%%%%%%%%%%%%%%%%%%
% \paragraph{Part Include Files.}
%
% The include files are called |cdocspt3.tex| and |cdocspt4.tex|.
%
%\iffalse
%<*samplepart3|samplepart4>
%\fi

% Optional override for |\version| flag:
%    \begin{macrocode}
%%\providecommand{\version}{final}
%    \end{macrocode}

% Include the main document:
%    \begin{macrocode}
\input{childdoc.def}
\childdocby{cdocsamp}
%    \end{macrocode}

%\iffalse
%</samplepart3|samplepart4>
%\fi
%
%\iffalse
%<*samplepart3>
%\fi
% Some text for part 3:
%    \begin{macrocode}
some text in part three
%    \end{macrocode}

%\iffalse
%</samplepart3>
%\fi
% Some text for part 4:
%\iffalse
%<*samplepart4>
%\fi
%    \begin{macrocode}
more text in part four
%    \end{macrocode}

%\iffalse
%</samplepart4>
%\fi
%
% %%%%%%%%%%%%%%%%%%%%%%%%%%%%%%%%%%%%%%
% \paragraph{Forwarding for a Complete Draft.}
%
% The following forwarding file |cdocsdrf.tex|
% compiles the main document in draft mode:
%\iffalse
%<*sampledraft>
%\fi
%    \begin{macrocode}
\def\version{draft}
\input{childdoc.def}
\childdocforward{cdocsamp}
%    \end{macrocode}

%\iffalse
%</sampledraft>
%\fi
%
% %%%%%%%%%%%%%%%%%%%%%%%%%%%%%%%%%%%%%%
% \paragraph{Forwarding for Final Version of the Chapters.}
%
% The following forwarding files |cdocsfn1.tex| and |cdocsfn2.tex|
% (with identical content)
% compile the final versions of the child documents
% |cdocsch1.tex| and |cdocsch2.tex|, respectively:
%\iffalse
%<*samplefinal>
%\fi
%    \begin{macrocode}
\def\version{final}
\input{childdoc.def}
\childdocforwardprefix[cdocsamp]{cdocsfn}{cdocsch}
%    \end{macrocode}

%\iffalse
%</samplefinal>
%\fi
%
% %%%%%%%%%%%%%%%%%%%%%%%%%%%%%%%%%%%%%%
% \paragraph{Command Line Processing.}
%
% The following three command lines generate the output files
% |cdocscld|, |cdocscl1| and |cdocscl2|
% which should be identical to
% |cdocsdrf|, |cdocsch1| and |cdocsfn2|, respectively:
% \begin{center}
% \begin{tabular}{l}
% |latex -jobname cdocscld \|\\
% |  "\def\version{draft}\input{childdoc.def}\childdocforward{cdocsamp}"|\\
% |latex -jobname cdocscl1 \|\\
% |  "\input{childdoc.def}\childdocforward[cdocsamp]{cdocsch1}"|\\
% |latex -jobname cdocscl2 \|\\
% |  "\def\version{final}\input{childdoc.def}\childdocforward{cdocsch2}"|
% \end{tabular}
% \end{center}
% Note that the trailing backslash on each first line
% merely continues the input to the second line
% (for convenient cut ant paste).
% Furthermore, the command |latex| can be replaced by any
% of its alternative versions such as |pdflatex|.
%
% %%%%%%%%%%%%%%%%%%%%%%%%%%%%%%%%%%%%%%%%%%%%%%%%%%%%%%%%%%%%%%%%%%%%%%%%%%%%%%
% %%%%%%%%%%%%%%%%%%%%%%%%%%%%%%%%%%%%%%%%%%%%%%%%%%%%%%%%%%%%%%%%%%%%%%%%%%%%%%
% \section{Implementation}
%\iffalse
%<*package>
%\fi
%
% This section describes the definitions file |childdoc.def|.

% The definitions cannot be loaded using |\usepackage| or |\RequirePackage|
% which has a mechanism to prevent loading a style file more than once.
% When loading the definitions by means of |\input|
% multiple instances have to be prevented manually:
%\iffalse
%This code needs to be before the `\ProvidesFile' directive
%which is defined at the beginning of this file.
%Therefore it is also placed there and commented out here.
%</package>
%<*discard>
%\fi
%    \begin{macrocode}
\ifdefined\childdocmain\endinput\fi
%    \end{macrocode}
%\iffalse
%</discard>
%<*package>
%\fi
%
% \macro{\ifchilddoc}
% \macro{\ifchilddocmanual}
% The conditional |\ifchilddoc| tells whether a
% child (true) or main (false) document is being compiled.
% The conditional |\ifchilddocmanual| tells whether
% the |\includeonly| mechanism is used (false) or
% the selection of child files must be performed manually (true).
% The definitions initialise to false:
%    \begin{macrocode}
\newif\ifchilddoc
\newif\ifchilddocmanual
%    \end{macrocode}

% \macro{\childdocname}
% \macro{\childdocjob}
% The macro |\childdocname| stores the name of the main document
% to be compiled. The macro |\childdocjob| stores the name of
% the document on which the \LaTeX{} compiler was originally invoked.
% The content of |\jobname| cannot be compared
% to filenames specified in the source due to different catcodes.
% The following code rescans |\jobname|, stores the result
% in |\childdocname| and saves a copy in |\childdocjob|:
%    \begin{macrocode}
\edef\childdocname{\scantokens\expandafter{\jobname\noexpand}}
\let\childdocjob\childdocname
%    \end{macrocode}

% \macro{\childdocdisable}
% The macro |\childdocdisable| prevents the main file
% from being processed more than once.
% At this stage, the main document command |\childdocmain|
% is assumed to be called once again where it should do nothing.
% Any subsequent call to it should prevent
% a secondary processing of the main document
% It overwrites the forwarding commands
% |\childdocof| and |\childdocforward|
% with empty macros to prevent further inclusions of the main document:
%    \begin{macrocode}
\newcommand{\childdocdisable}
{
  \renewcommand{\childdocmain}[1]{\renewcommand{\childdocmain}[1]{\endinput}}
  \renewcommand{\childdocof}[1]{}
  \renewcommand{\childdocby}[2][]{}
  \renewcommand{\childdocforward}[2][]{}
  \renewcommand{\childdocdisable}{}
}
%    \end{macrocode}

% \macro{\childdocmain}
% The macro |\childdocmain| is to be called at the top of the main file
% with nothing or the main filename (without extension) as argument.
% First, it breaks loops.
% If the argument is not empty and does not match |\childdocname|
% (which is set by the first inclusion of |childdoc.def|),
% |\ifchilddoc| is set to true, |\includeonly| is applied to the child file
% and |\jobname| is set to the main file
% (for proper handling of |.aux| files):
%    \begin{macrocode}
\newcommand{\childdocmain}[1]
{
  \childdocdisable\childdocmain{}
  \if?#1?\else
    \begingroup
      \def\childdoctmp{#1}
      \ifx\childdoctmp\childdocname
        \def\childdoctmp{}
      \else
        \def\childdoctmp
        {
          \childdoctrue
          \includeonly{\childdocname}
          \def\childdocjob{#1}
          \def\jobname{#1}
        }
      \fi
      \expandafter
    \endgroup
    \childdoctmp
  \fi
}
%    \end{macrocode}

% \macro{\childdocof}
% The command |\childdocof| redirects
% compilation to the main file |#1|.
%    \begin{macrocode}
\newcommand{\childdocof}[1]
{
  \childdocdisable
  \childdoctrue
  \includeonly{\childdocname}
  \def\jobname{#1}
  \def\childdocjob{#1}
  \input{#1}
}
%    \end{macrocode}

% \macro{\childdocby}
% The command |\childdocby| ....
%    \begin{macrocode}
\newcommand{\childdocby}[2][]
{
  \childdocdisable
  \childdoctrue
  \childdocmanualtrue
  \if?#1?\else
    \def\jobname{#2}
  \fi
  \def\childdocjob{#2}
  \input{#2}
  \endinput
}
%    \end{macrocode}

% \macro{\childdocforward}
% The command |\childdocforward| redirects
% compilation to the main file or
% (if the optional argument is given) a child file.
% Parameters are set as if the main file
% or a child file starting with |\childdocof| was compiled.
% Then compilation is handed over to the main file:
%    \begin{macrocode}
\newcommand{\childdocforward}[2][]
{
  \begingroup
    \if?#1?
      \def\childdoctmp
      {
        \def\childdocname{#2}
        \def\childdocjob{#2}
        \def\jobname{#2}
        \input{#2}
        \endinput
      }
    \else
      \def\childdoctmp
      {
        \childdocdisable
        \def\childdocname{#2}
        \childdoctrue
        \includeonly{#2}
        \def\childdocjob{#1}
        \def\jobname{#1}
        \input{#1}
        \endinput
      }
    \fi
    \expandafter
  \endgroup
  \childdoctmp
}
%    \end{macrocode}

% \macro{\childdocforwardprefix}
% The command |\childdocforwardprefix| redirects
% compilation to the main or a child file by means of a pattern.
% The prefix |#1| in the current filename is replaced by |#2|
% and the suffix of the current filename is kept
% (it is assumed that the filename does not contain the substring `|~~~|'
% which is used as a delimiter).
% Compilation is handed over to the new file by |\childdocforward|:
%    \begin{macrocode}
\newcommand{\childdocforwardprefix}[3][]
{
  \begingroup
    \def\childdocextract #2##1~~~{\def\childdoctmp{\childdocforward[#1]{#3##1}}}
    \expandafter\childdocextract\childdocname~~~
    \expandafter
  \endgroup
  \childdoctmp
}
%    \end{macrocode}

% \macro{\childdoc}
% The deprecated macro |\childdoc| is a legacy version of |\childdocmain|:
%    \begin{macrocode}
\newcommand{\childdoc}{\childdocmain}
%    \end{macrocode}

% \macro{\childdocredirect}
% The deprecated macro |\childdocredirect| is a legacy version
% of |\childdocforward| and |\childdocforwardprefix|:
%    \begin{macrocode}
\newcommand{\childdocredirect}[2][]
{
  \begingroup
    \if?#1?
      \def\childdoctmp{\childdocforward{#2}}
    \else
      \def\childdoctmp{\childdocforwardprefix{#1}{#2}}
    \fi
    \expandafter
  \endgroup
  \childdoctmp
}
%    \end{macrocode}

%\iffalse
%</package>
%\fi
%
\endinput
\childdocforward[|\textit{main}|]{|\textit{dest}|}"|
\end{center}
%
Here \textit{target} is the name of the output file,
\textit{main} is the name of the main file
and \textit{dest} is the name of the main or child file to be processed
(all filenames without extensions).
The optional argument \textit{main} can be omitted
if \textit{main} matches \textit{dest}.
Optionally, compilation \textit{flags} can be defined via |\def| commands.
This command line makes the \TeX{} engine believe
it is compiling the file \textit{target}
whose content is specified as the latter parameter.
The provided code then forwards the processing to
\textit{main} or \textit{dest} as described in \secref{sec:forward}.

%%%%%%%%%%%%%%%%%%%%%%%%%%%%%%%%%%%%%%%%%%%%%%%%%%%%%%%%%%%%%%%%%%%%%%%%%%%%%%%%
\subsection{Include by Input}
\label{sec:input}

Including child documents by |\include| has some restrictions by design.
Most notably, the content of a child document always occupies
its own set of pages; pages cannot be shared between child documents.
Usually, this behaviour makes perfect sense
because each child document contain an essential part of the document.
However, in some situations it may be desirable to compose
a document from a collection of parts
without having mandatory page breaks between then.
For this case, the package
provides a mechanism to include parts
by |\input| which can also be processed individually.
However, by construction this mechanism
requires manual handling of the content to be output.

%%%%%%%%%%%%%%%%%%%%%%%%%%%%%%%%%%%%%%%%
\DescribeMacro{\ifchilddocmanual}
The main file should be prepared as usual, see \secref{sec:include}.
However, the document body must make a distinction
between processing of an individual part and of the main document, e.g.:
%
\begin{center}
\begin{tabular}{l}
|\ifchilddocmanual|\\
|\input{\childdocname}|\\
|\||else|\\
\textit{document body with }|\input{|\textit{part}|}|\\
|\||fi|
\end{tabular}
\end{center}
%
The conditional |\ifchilddocmanual| is true whenever
a part to be included by |\input| is being compiled,
and the name of the part is stored in |\childdocname|.

%%%%%%%%%%%%%%%%%%%%%%%%%%%%%%%%%%%%%%%%
\DescribeMacro{\childdocby}
Each part to be included by |\input| should start with:
%
\begin{center}
\begin{tabular}{l}
|% \iffalse
%
% childdoc.dtx Copyright (C) 2017-2018 Niklas Beisert
%
% This work may be distributed and/or modified under the
% conditions of the LaTeX Project Public License, either version 1.3
% of this license or (at your option) any later version.
% The latest version of this license is in
%   http://www.latex-project.org/lppl.txt
% and version 1.3 or later is part of all distributions of LaTeX
% version 2005/12/01 or later.
%
% This work has the LPPL maintenance status `maintained'.
%
% The Current Maintainer of this work is Niklas Beisert.
%
% This work consists of the files childdoc.dtx and childdoc.ins
% and the derived files childdoc.def and cdocsamp.tex with
% cdocsch1.tex, cdocsch2.tex, cdocsdrf.tex, cdocsfn1.tex, cdocsfn2.tex.
%
%<package>\ifdefined\childdocmain\endinput\fi
%<package>\ProvidesFile{childdoc.def}[2018/12/30 v2.0 child document driver]
%<samplemain>\ProvidesFile{cdocsamp.tex}[2018/12/30 v2.0 sample for childdoc]
%<*driver>
%\ProvidesFile{childdoc.drv}[2018/12/30 v2.0 childdoc reference manual file]
\PassOptionsToClass{10pt,a4paper}{article}
\documentclass{ltxdoc}

\usepackage[margin=35mm]{geometry}
\usepackage{hyperref}
\usepackage{hyperxmp}
\usepackage[usenames]{color}

\hypersetup{colorlinks=true}
\hypersetup{pdfstartview=FitH}
\hypersetup{pdfpagemode=UseNone}
\hypersetup{pdfsource={}}
\hypersetup{pdflang={en-UK}}
\hypersetup{pdfcopyright={Copyright 2017-2018 Niklas Beisert.
  This work may be distributed and/or modified under the
  conditions of the LaTeX Project Public License, either version 1.3
  of this license or (at your option) any later version.}}
\hypersetup{pdflicenseurl={http://www.latex-project.org/lppl.txt}}
\hypersetup{pdfcontactaddress={ETH Zurich, ITP, HIT K,
  Wolfgang-Pauli-Strasse 27}}
\hypersetup{pdfcontactpostcode={8093}}
\hypersetup{pdfcontactcity={Zurich}}
\hypersetup{pdfcontactcountry={Switzerland}}
\hypersetup{pdfcontactemail={nbeisert@itp.phys.ethz.ch}}
\hypersetup{pdfcontacturl={http://people.phys.ethz.ch/\xmptilde nbeisert/}}

\newcommand{\secref}[1]{\hyperref[#1]{section \ref*{#1}}}

\parskip1ex
\parindent0pt
\let\olditemize\itemize
\def\itemize{\olditemize\parskip0pt}

\begin{document}

\title{The \textsf{childdoc} Package}
\hypersetup{pdftitle={The childdoc Package}}
\author{Niklas Beisert\\[2ex]
  Institut f\"ur Theoretische Physik\\
  Eidgen\"ossische Technische Hochschule Z\"urich\\
  Wolfgang-Pauli-Strasse 27, 8093 Z\"urich, Switzerland\\[1ex]
  \href{mailto:nbeisert@itp.phys.ethz.ch}
  {\texttt{nbeisert@itp.phys.ethz.ch}}}
\hypersetup{pdfauthor={Niklas Beisert}}
\hypersetup{pdfsubject={Manual for the LaTeX2e Package childdoc}}
\date{30 December 2018, \textsf{v2.0}}
\maketitle

\begin{abstract}\noindent
\textsf{childdoc} is a \LaTeXe{} package
that enables the direct compilation
of document sections included by |\include|
to individual files.
\end{abstract}

\begingroup
\parskip0ex
\tableofcontents
\endgroup

%%%%%%%%%%%%%%%%%%%%%%%%%%%%%%%%%%%%%%%%%%%%%%%%%%%%%%%%%%%%%%%%%%%%%%%%%%%%%%%%
%%%%%%%%%%%%%%%%%%%%%%%%%%%%%%%%%%%%%%%%%%%%%%%%%%%%%%%%%%%%%%%%%%%%%%%%%%%%%%%%
\section{Introduction}

\LaTeX{} provides a mechanism to structure a large document (such as a book)
into a main file and several child files (containing the chapters)
using the |\include| command.
This mechanism is beneficial for documents
which span hundreds of pages in order to
make the source file(s) more manageable.
Moreover, compilation can be restricted to
selected child files by means of the |\includeonly| command.
The latter feature can be used to reduce the compilation time while editing
(this was significantly more useful in the earlier days of \LaTeX{})
or to generate a smaller document which is easier to navigate.
Another application of |\includeonly| is to generate
documents consisting of selected parts of the complete document.

However, there are a few drawbacks of the plain |\include| mechanism:
\begin{itemize}
\item
The child files cannot be compiled on their own,
they can only be compiled via the main file.
A naive editing environment
(such as a text editor with an option
to have the current file processed by \LaTeX)
may require one to switch to the main file before compiling;
attempting to compile the child file produces errors.
\item
The main file must be modified (each time)
to adjust the |\includeonly| command
to the present needs. This easily leaves the main file in a messy state.
\item
The generated document will always carry the filename
of the main document. This is inconvenient if
several child files are to be compiled and
to be kept for distribution.
\end{itemize}

The present package provides a simple interface
to make child files individually compilable by \LaTeX{}.
Compiling a child file then has the same effect as compiling
the main file with an |\includeonly| command
to select the appropriate child.
Moreover the generated document will carry the name of the child
rather than the main file.
This resolves all three above issues.

This feature is meant to make the editing of books,
thesis documents and lecture notes somewhat more convenient.
However, the package can also be used efficiently for
composing a series of documents (such as exercise sheets)
which are typically distributed individually.
It then assists the author in generating the individual documents
(potentially in different versions)
as well as a document containing the collected series.
Another application is in developing style files
or other kinds of included material
where compilation of the style file could redirect
to a sample or test file.

%%%%%%%%%%%%%%%%%%%%%%%%%%%%%%%%%%%%%%%%%%%%%%%%%%%%%%%%%%%%%%%%%%%%%%%%%%%%%%%%
%%%%%%%%%%%%%%%%%%%%%%%%%%%%%%%%%%%%%%%%%%%%%%%%%%%%%%%%%%%%%%%%%%%%%%%%%%%%%%%%
\section{Usage}

First of all, the package \textsf{childdoc} is \emph{not} a standard
\LaTeXe{} |.sty| style file! Therefore it needs to be invoked in
a non-standard way.

%%%%%%%%%%%%%%%%%%%%%%%%%%%%%%%%%%%%%%%%%%%%%%%%%%%%%%%%%%%%%%%%%%%%%%%%%%%%%%%%
\subsection{Included Files}
\label{sec:include}

%%%%%%%%%%%%%%%%%%%%%%%%%%%%%%%%%%%%%%%%
\DescribeMacro{\childdocmain}
To use the package, add the commands
\begin{center}
\begin{tabular}{l}
|\input{childdoc.def}|\\
|\childdocmain{}|\\
\end{tabular}
\end{center}
at the very top of the main \LaTeX{} file,
in particular \emph{before} the |\documentclass| statement!
The argument of |\childdocmain| should be left empty
(but it must be present).

%%%%%%%%%%%%%%%%%%%%%%%%%%%%%%%%%%%%%%%%
\DescribeMacro{\childdocof}
Furthermore, add the commands
\begin{center}
\begin{tabular}{l}
|\input{childdoc.def}|\\
|\childdocof{|\textit{main}|}|\\
\end{tabular}
\end{center}
at the top of every child file \textit{child}
which is included by |\include{|\textit{child}|}|
from within the main file
(or at least for those files to be compiled individually).
The argument \textit{main} must be the filename of the main file.

There are a couple of
considerations in setting up the main and child documents:

%%%%%%%%%%%%%%%%%%%%%%%%%%%%%%%%%%%%%%%%
\paragraph{Restrictions.}

Please note the following restrictions:
\begin{itemize}
\item
|\childdocmain| must be called with one argument \textit{main}
to ensure compatibility with earlier version of the package.
It must either be empty (|\childdocmain{}|)
or precisely match the filename of the main file in which it is specified.
See \secref{sec:detection} for further information.
\item
The filename \textit{main} must be specified without the |.tex| extension.
\item
The filename \textit{main} is case sensitive
(even in case-insensitive file systems)
due to internal string comparison.
\item
The argument \textit{main} should be fully expanded, it cannot be a macro.
\item
Subdirectories and special characters should be avoided in filenames.
\item
The command |\childdocmain{|\textit{main}|}| must be followed by a whitespace.
It should not be followed immediately by another command
or by a comment mark `|%|'.
This is because the \TeX{} parser reads the token immediately following
the argument of |\childdocmain| and puts it
at the beginning of every child section;
however, a white\-space is ignored.
\end{itemize}

%%%%%%%%%%%%%%%%%%%%%%%%%%%%%%%%%%%%%%%%
\paragraph{Content of Main File.}

It is advisable to place all content in the child files included by |\include|.
Any output contained in the main file will appear in all child documents
unless suppressed manually;
it cannot be suppressed automatically by the |\includeonly| directive
and thus should normally be avoided.
A method to include some content in the main file
by means of conditional processing is described in \secref{sec:conditional}.

%%%%%%%%%%%%%%%%%%%%%%%%%%%%%%%%%%%%%%%%
\paragraph{Page Numbering.}

When only a part of the document is compiled,
the appropriate numbering of pages
(as well as other status parameters)
is determined from the |.aux| files.
The latter contain information from previous passes.
However this information needs to propagate through
all intermediate child documents.
Therefore the page numbering in child documents may well
be inconsistent until the complete document is compiled at least once.

A useful (if unconventional) way to always ensure a consistent
page numbering is to restart the numbering in each child document
and denote the pages by `\textit{child}|.|\textit{page}'
where \textit{child} represents the chapter/section number of the child file.
This can be achieved by the command
|\numberwithin{page}{|\textit{child}|}|
of the \textsf{amsmath} package
where \textit{child} can be |chapter| or |section|
depending on the chosen structuring.
Alternatively, one can modify the macro |\thepage| appropriately
and reset the counter |page| at the start of each child file.

%%%%%%%%%%%%%%%%%%%%%%%%%%%%%%%%%%%%%%%%%%%%%%%%%%%%%%%%%%%%%%%%%%%%%%%%%%%%%%%%
\subsection{Conditional Processing}
\label{sec:conditional}

The package provides a mechanism to compile different versions
of a document. To customise the versions further some conditional processing
can come in handy to distinguish which version is being compiled.
The package provides two macros to describe the compilation context:

%%%%%%%%%%%%%%%%%%%%%%%%%%%%%%%%%%%%%%%%
\DescribeMacro{\ifchilddoc}
The conditional |\ifchilddoc| distinguishes between the compilation of
child documents and the main document:
%
\begin{center}
|\ifchilddoc |\textit{child-code}| |[|\||else |\textit{main-code}]| \||fi|
\end{center}

%%%%%%%%%%%%%%%%%%%%%%%%%%%%%%%%%%%%%%%%
\DescribeMacro{\childdocname}
\DescribeMacro{\childdocjob}
The macro |\childdocname| contains the filename (without extension)
of the main or child file being processed.
Note that |\childdocjob| will always contain the name of the main file.

%%%%%%%%%%%%%%%%%%%%%%%%%%%%%%%%%%%%%%%%
\paragraph{Title Page.}

Conditional processing can be used to include a title or banner page
in the main document when proper precautions are taken.
Importantly, the code in the main file should ensure that the page counter
(as well as other status parameters which are stored in the |.aux| files)
takes the same value after the conditional processing.
Otherwise the page numbers may take divergent values
depending on which part is compiled.

For example, a title page could be declared by:
%
\begin{center}
\begin{tabular}{l}
|\ifchilddoc\||else|\\
|\addtocounter{page}{-1}|\\
\textit{code for title page}\\
|\newpage|\\
|\||fi|
\end{tabular}
\end{center}
%
A banner page for the child documents can be generated by:
%
\begin{center}
\begin{tabular}{l}
|\ifchilddoc|\\
|\addtocounter{page}{-1}|\\
\textit{code for banner page}\\
|\newpage|\\
|\||fi|
\end{tabular}
\end{center}
%
Here one could write a message such as:
\begin{center}
|This is the part \childdocname{} of \childdocjob{}.|
\end{center}

%%%%%%%%%%%%%%%%%%%%%%%%%%%%%%%%%%%%%%%%%%%%%%%%%%%%%%%%%%%%%%%%%%%%%%%%%%%%%%%%
\subsection{Flags}
\label{sec:flags}

The package makes it easy to generate different versions
of the main or child documents.
To this end compilation flags can be defined
and assigned different default values.
They will be particularly useful in conjunction
with the forwarding mechanism described in \secref{sec:forward}.

For example, it may be useful to have a flag |\version|
which can be set to |draft| or |final|.
The document source will contain some conditional code
depending on the value of |\version|.
Suppose further, the flag should default to |final| for the main file
and to |draft| for child files
which is a natural assignment for editing the document.
This is achieved by placing the following code
in the preamble of the main document
(below the |\childdocmain| directive):
%
\begin{center}
\begin{tabular}{l}
|\ifchilddoc|\\
|\providecommand{\version}{draft}|\\
|\||else|\\
|\providecommand{\version}{final}|\\
|\||fi|
\end{tabular}
\end{center}
%
The definition by |\providecommand| makes sure
that previous definitions are not overwritten.
Further statements |\providecommand{\version}{...}|
can thus be added before the above code to override it.

For the main file, one might add a line
(between |\childdocmain| and the above block)
%
\begin{center}
|%\ifchilddoc\||else\providecommand{\version}{draft}\||fi|
\end{center}
%
which can be uncommented to produce a draft version.
Likewise one can add a line to the very top of a child file
(above the |\childdocof{|\textit{main}|}| directive)
%
\begin{center}
|%\providecommand{\version}{final}|
\end{center}
%
which can be uncommented to produce the final version of this child document.

%%%%%%%%%%%%%%%%%%%%%%%%%%%%%%%%%%%%%%%%%%%%%%%%%%%%%%%%%%%%%%%%%%%%%%%%%%%%%%%%
\subsection{Forwarding}
\label{sec:forward}

Different versions of the main or child documents
using compilation flags as described in \secref{sec:flags}
can be (permanently) stored in different files
for convenient compilation, viewing and distribution.
To this end, the package defines a command
to pass on compilation to a different file:

%%%%%%%%%%%%%%%%%%%%%%%%%%%%%%%%%%%%%%%%
\DescribeMacro{\childdocforward}
The command |\childdocforward| redirects processing to
another source file:
%
\begin{center}
\begin{tabular}{l}
|\input{childdoc.def}|\\
|\childdocforward[|\textit{main}|]{|\textit{dest}|}|\\
\end{tabular}
\end{center}
%
The argument \textit{dest} is the destination file
(without extension).
It should be the main file or one of the child files.
Note that further \textsf{childdoc} directives
such as |\childdocof| and |\childdocforward|
in the indicated file will be processed in this form.
The optional argument \textit{main}
passes on directly to the main file \textit{main}
while pretending to compile the child \textit{dest}.
This form behaves as if \textit{dest}
issues |\childdocof{|\textit{main}|}| right away,
and no further \textsf{childdoc} directives will be processed.

%%%%%%%%%%%%%%%%%%%%%%%%%%%%%%%%%%%%%%%%
\DescribeMacro{\...prefix}
In the alternative form |\childdocforwardprefix|,
%
\begin{center}
\begin{tabular}{l}
|\input{childdoc.def}|\\
|\childdocforwardprefix[|\textit{main}|]{|\textit{prefix}|}{|\textit{dest}|}|
\end{tabular}
\end{center}
%
the destination file is determined by a pattern
depending on the current file:
To make this work, the current file must be called
`{\textit{prefix}\hspace{0.2em}\textit{suffix}}'
with \textit{prefix} matching precisely the argument.
Processing is then passed on to the file
`{\textit{dest}\hspace{0.2em}\textit{suffix}}'.
Surely, the same effect is achieved by
directly specifying the
argument `{\textit{dest}\hspace{0.2em}\textit{suffix}}'
in the first form.
However, that requires to set up a different file
for each child. With the alternative form of the command
all these files can have exactly the same content
which simplifies setting them up and maintaining them.

For example, the following file |draft.tex|
with a compilation flag |\version| as described in \secref{sec:flags}
compiles the main document as a draft:
%
\begin{center}
\begin{tabular}{l}
|\def\version{draft}|\\
|\input{childdoc.def}|\\
|\childdocforward{|\textit{main}|}|
\end{tabular}
\end{center}
%
Likewise, the following files |final|\textit{nn}|.tex|
compile the final version of the child document
|child|\textit{nn}|.tex|:
%
\begin{center}
\begin{tabular}{l}
|\def\version{final}|\\
|\input{childdoc.def}|\\
|\childdocforwardprefix{final}{child}|
\end{tabular}
\end{center}
%

Note that when several versions of a main file and/or of each child file
are to be generated, it may be convenient to set up a |Makefile| or
shell script to automatise the process.

%%%%%%%%%%%%%%%%%%%%%%%%%%%%%%%%%%%%%%%%%%%%%%%%%%%%%%%%%%%%%%%%%%%%%%%%%%%%%%%%
\subsection{Command Line Processing}
\label{sec:commandline}

The effect of redirection files can also be achieved by invoking
the \LaTeX{} compiler with a more elaborate command line.
Most conveniently this should be done as part
of a shell script or a |Makefile|.

When using \textsf{childdoc} in the main file, the following
command lines effectively perform a redirection
(note that depending on the shell being used,
backslashes may have to be doubled: `|\|' $\to$ `|\\|'):
%
\begin{center}
|... -jobname "|\textit{target}|" |\\|"|[\textit{flags}]%
|\input{childdoc.def}\childdocforward[|\textit{main}|]{|\textit{dest}|}"|
\end{center}
%
Here \textit{target} is the name of the output file,
\textit{main} is the name of the main file
and \textit{dest} is the name of the main or child file to be processed
(all filenames without extensions).
The optional argument \textit{main} can be omitted
if \textit{main} matches \textit{dest}.
Optionally, compilation \textit{flags} can be defined via |\def| commands.
This command line makes the \TeX{} engine believe
it is compiling the file \textit{target}
whose content is specified as the latter parameter.
The provided code then forwards the processing to
\textit{main} or \textit{dest} as described in \secref{sec:forward}.

%%%%%%%%%%%%%%%%%%%%%%%%%%%%%%%%%%%%%%%%%%%%%%%%%%%%%%%%%%%%%%%%%%%%%%%%%%%%%%%%
\subsection{Include by Input}
\label{sec:input}

Including child documents by |\include| has some restrictions by design.
Most notably, the content of a child document always occupies
its own set of pages; pages cannot be shared between child documents.
Usually, this behaviour makes perfect sense
because each child document contain an essential part of the document.
However, in some situations it may be desirable to compose
a document from a collection of parts
without having mandatory page breaks between then.
For this case, the package
provides a mechanism to include parts
by |\input| which can also be processed individually.
However, by construction this mechanism
requires manual handling of the content to be output.

%%%%%%%%%%%%%%%%%%%%%%%%%%%%%%%%%%%%%%%%
\DescribeMacro{\ifchilddocmanual}
The main file should be prepared as usual, see \secref{sec:include}.
However, the document body must make a distinction
between processing of an individual part and of the main document, e.g.:
%
\begin{center}
\begin{tabular}{l}
|\ifchilddocmanual|\\
|\input{\childdocname}|\\
|\||else|\\
\textit{document body with }|\input{|\textit{part}|}|\\
|\||fi|
\end{tabular}
\end{center}
%
The conditional |\ifchilddocmanual| is true whenever
a part to be included by |\input| is being compiled,
and the name of the part is stored in |\childdocname|.

%%%%%%%%%%%%%%%%%%%%%%%%%%%%%%%%%%%%%%%%
\DescribeMacro{\childdocby}
Each part to be included by |\input| should start with:
%
\begin{center}
\begin{tabular}{l}
|\input{childdoc.def}|\\
|\childdocby{|\textit{main}|}|\\
\end{tabular}
\end{center}
%
The directive |\childdocby| is similar to |\childdocof|
described in \secref{sec:include},
but the subsequent selection of content must be done manually.
To that end, both |\ifchilddoc| and |\ifchilddocmanual|
will be true upon processing of a part,
and the name of the part is stored in |\childdocname|.
Note that |\jobname| will be set to the filename of the current part
so that each part receives an individual |.aux| file
that does not interfere with the |.aux| file(s) of the main document.
This behaviour can be altered by the alternative form
|\childdocby[*]{|\textit{main}|}| (with a non-empty optional argument)
which uses the |.aux| file of the main document
by setting |\jobname| to \textit{main}.

%%%%%%%%%%%%%%%%%%%%%%%%%%%%%%%%%%%%%%%%%%%%%%%%%%%%%%%%%%%%%%%%%%%%%%%%%%%%%%%%
\subsection{Driver Development}
\label{sec:driver}

The \textsf{childdoc} mechanism can also be use for the development
of definition files such as \LaTeX{} styles or classes.
This case differs from the above setup with multiple parts
included by |\include| in that no |\includeonly| should be invoked.
This can be achieved by starting the include file
(before |\ProvidesPackage|) with:
%
\begin{center}
\begin{tabular}{l}
|\input{childdoc.def}|\\
|\childdocforward{|\textit{main}|}|\\
\end{tabular}
\end{center}
%
or alternatively with:
%
\begin{center}
\begin{tabular}{l}
|\input{childdoc.def}|\\
|\childdocby{|\textit{main}|}|\\
\end{tabular}
\end{center}
%
Both forms have slightly different effects as described above.
The main file is prepared as usual, see \secref{sec:include}.

%%%%%%%%%%%%%%%%%%%%%%%%%%%%%%%%%%%%%%%%%%%%%%%%%%%%%%%%%%%%%%%%%%%%%%%%%%%%%%%%
\subsection{Legacy Detection}
\label{sec:detection}

The directive |\childdocmain| in the main file can detect
whether the complete document or merely a child is to be compiled
even without using the directive |\childdocof|.
This method is deprecated because it is less robust
and there is no compelling reason to use it;
it is merely provided for backward compatibility
and it may be removed in future versions.

If the detection mechanism is to be used,
it is mandatory to correctly specify
the filename of the main file as the argument of |\childdocmain|:
%
\begin{center}
\begin{tabular}{l}
|\input{childdoc.def}|\\
|\childdocmain{|\textit{main}|}|\\
\end{tabular}
\end{center}
%
If |\jobname| does not match the argument \textit{main} of |\childdocmain|,
it is assumed that |\jobname| points to the child file to be compiled.
When using |\childdocmain| with the main file specified as argument,
it suffices to start a child file
with just |\input{|\textit{main}|}|
without loading of the package and using |\childdocof|.
If instead all processing is done
with the appropriate \textsf{childdoc} directives,
the argument of \textit{main} of |\childdocmain| can be empty.

An alternative version of the command line processing described
in \secref{sec:commandline} using the detection mechanism reads:
%
\begin{center}
|... -jobname "|\textit{target}|" "|[\textit{flags}]%
[|\def\jobname{|\textit{dest}|}|]|\input{|\textit{main}|}"|
\end{center}

%%%%%%%%%%%%%%%%%%%%%%%%%%%%%%%%%%%%%%%%%%%%%%%%%%%%%%%%%%%%%%%%%%%%%%%%%%%%%%%%
\subsection{Manual Code}
\label{sec:manual}

In case one cannot be certain whether the definitions file |childdoc.def|
is installed on the target \TeX{} distribution
and one prefers not to ship it,
it is conceivable to paste a few relevant commands into the sources.

To that end, drop all statements |\input{childdoc.def}|
and perform the replacements as outlined below.
Instead of |\childdocmain{|\textit{main}|}| add the following code
to the top of the main file:
%
\begin{center}
\begin{tabular}{l}
|\||ifdefined\childdocname\endinput\||fi\newif\ifchilddoc|\\
|\edef\childdocname{\scantokens\expandafter{\jobname\noexpand}}|\\
|\def\childdocmain{|\textit{main}|}\||ifx\childdocmain\childdocname\||else|\\
|\childdoctrue\includeonly{\childdocname}\let\jobname\childdocmain\||fi|\\
\end{tabular}
\end{center}
%
Instead of |\childdocof{|\textit{main}|}| just include the main file
at the top of each child file:
%
\begin{center}
|\input{|\textit{main}|}|
\end{center}
%
A simple redirection |\childdocforward{|\textit{dest}|}| is achieved by:
%
\begin{center}
|\def\jobname{|\textit{dest}|}\input{\jobname}|
\end{center}
%
The redirection with prefix
|\childdocforwardprefix[|\textit{prefix}|]{|\textit{dest}|}|
is accomplished by:
%
\begin{center}
\begin{tabular}{l}
|{\edef\jobname{\scantokens\expandafter{\jobname\noexpand}}|\\
|\def\redirectjob |\textit{prefix}|#1~~~{\gdef\jobname{|\textit{dest}|#1}}|\\
|\expandafter\redirectjob\jobname~~~}\input{\jobname}|
\end{tabular}
\end{center}

In an alternative approach,
child documents can be compiled by a specific command line
without additional code or specific definitions:
%
\begin{center}
|... -jobname "|\textit{target}|" "|[\textit{flags}]%
|\includeonly{|\textit{dest}|}\input{|\textit{main}|}"|
\end{center}
%

%%%%%%%%%%%%%%%%%%%%%%%%%%%%%%%%%%%%%%%%%%%%%%%%%%%%%%%%%%%%%%%%%%%%%%%%%%%%%%%%
%%%%%%%%%%%%%%%%%%%%%%%%%%%%%%%%%%%%%%%%%%%%%%%%%%%%%%%%%%%%%%%%%%%%%%%%%%%%%%%%
\section{Information}

%%%%%%%%%%%%%%%%%%%%%%%%%%%%%%%%%%%%%%%%%%%%%%%%%%%%%%%%%%%%%%%%%%%%%%%%%%%%%%%%
\subsection{Copyright}

Copyright \copyright{} 2017--2018 Niklas Beisert

This work may be distributed and/or modified under the
conditions of the \LaTeX{} Project Public License, either version 1.3
of this license or (at your option) any later version.
The latest version of this license is in
  \url{http://www.latex-project.org/lppl.txt}
and version 1.3 or later is part of all distributions of \LaTeX{}
version 2005/12/01 or later.

This work has the LPPL maintenance status `maintained'.

The Current Maintainer of this work is Niklas Beisert.

This work consists of the files |README.txt|, |childdoc.ins| and |childdoc.dtx|
as well as the derived files |childdoc.def|, |cdocsamp.tex|
with |cdocsch1.tex|, |cdocsch2.tex|, |cdocspt3.tex|, |cdocspt4.tex|,
|cdocsdrf.tex|, |cdocsfn1.tex|, |cdocsfn2.tex|
as well as |childdoc.pdf|.

%%%%%%%%%%%%%%%%%%%%%%%%%%%%%%%%%%%%%%%%%%%%%%%%%%%%%%%%%%%%%%%%%%%%%%%%%%%%%%%%
\subsection{Files and Installation}

The package consists of the files:
%
\begin{center}
\begin{tabular}{ll}
    |README.txt|   & readme file \\
    |childdoc.ins| & installation file \\
    |childdoc.dtx| & source file \\
    |childdoc.def| & definition file \\
    |cdocsamp.tex| & sample main file \\
    |cdocsch1.tex| & sample include file \\
    |cdocsch2.tex| & sample include file \\
    |cdocspt3.tex| & sample part file \\
    |cdocspt4.tex| & sample part file \\
    |cdocsdrf.tex| & sample redirection file \\
    |cdocsfn1.tex| & sample redirection file \\
    |cdocsfn2.tex| & sample redirection file \\
    |childdoc.pdf| & manual
\end{tabular}
\end{center}
%
The distribution consists of the files
|README.txt|, |childdoc.ins| and |childdoc.dtx|.
%
\begin{itemize}
\item
Run (pdf)\LaTeX{} on |childdoc.dtx|
to compile the manual |childdoc.pdf| (this file).
\item
Run \LaTeX{} on |childdoc.ins| to create the definitions file |childdoc.def|
and the sample |cdocsamp.tex| with include files
|cdocsch1.tex|, |cdocsch2.tex|, |cdocspt3.tex|, |cdocspt4.tex|,
|cdocsdrf.tex|, |cdocsfn1.tex|, |cdocsfn2.tex|.
Then copy the file |childdoc.def| to an appropriate directory of your \LaTeX{}
distribution, e.g.\ \textit{texmf-root}|/tex/latex/childdoc|.
\end{itemize}

%%%%%%%%%%%%%%%%%%%%%%%%%%%%%%%%%%%%%%%%%%%%%%%%%%%%%%%%%%%%%%%%%%%%%%%%%%%%%%%%
\subsection{Related CTAN Packages}

There are several other packages which offer a similar functionality:
%
\begin{itemize}
\item
The packages
\href{http://ctan.org/pkg/docmute}{\textsf{docmute}},
\href{http://ctan.org/pkg/includex}{\textsf{includex}} and
\href{http://ctan.org/pkg/standalone}{\textsf{standalone}}
provide commands to include only the document body of
a child file thus allowing both files to be compiled individually.
\item
The packages \href{http://ctan.org/pkg/subdocs}{\textsf{subdocs}}
and \href{http://ctan.org/pkg/subfiles}{\textsf{subfiles}}
provide structures in which the main and child documents can be
encapsulated and allowing them to be compiled individually.
The inclusion mechanism is different from the conventional |\include|.
\item
The package \href{http://ctan.org/pkg/combine}{\textsf{combine}}
is an elaborate solution to combine several documents into one.
\end{itemize}
%
See also the CTAN topic \href{http://ctan.org/topic/subdocs}{\textsf{subdocs}}
for further related packages.
The present package differs from the above solutions in that
a document structure constructed with the conventional |\include| mechanism
just needs two extra commands at the top of every file
such that all constituent files can be compiled individually.

%%%%%%%%%%%%%%%%%%%%%%%%%%%%%%%%%%%%%%%%%%%%%%%%%%%%%%%%%%%%%%%%%%%%%%%%%%%%%%%%
%\subsection{Feature Suggestions}
%
%The following is a list of features which may be useful for future
%versions of this package:
%%
%\begin{itemize}
%\item
%\ldots
%\end{itemize}

%%%%%%%%%%%%%%%%%%%%%%%%%%%%%%%%%%%%%%%%%%%%%%%%%%%%%%%%%%%%%%%%%%%%%%%%%%%%%%%%
\subsection{Revision History}

%%%%%%%%%%%%%%%%%%%%%%%%%%%%%%%%%%%%%%%%
\paragraph{v2.0:} 2018/12/30

\begin{itemize}
\item
immediate forward processing
\item
added |\childdocby| mechanism
\item
manual restructured
\end{itemize}

%%%%%%%%%%%%%%%%%%%%%%%%%%%%%%%%%%%%%%%%
\paragraph{v1.6:} 2018/01/17

\begin{itemize}
\item
application for development of include files
\item
corrections to manual
\end{itemize}

%%%%%%%%%%%%%%%%%%%%%%%%%%%%%%%%%%%%%%%%
\paragraph{v1.5:} 2017/05/21

\begin{itemize}
\item
more complete structuring introduced
\item
|\childdocof| introduced
\item
|\childdoc| renamed to |\childdocmain|
\item
|\childredirect| renamed to |\childdocforward| and |\childdocforwardprefix|
and functionality expanded
\end{itemize}

%%%%%%%%%%%%%%%%%%%%%%%%%%%%%%%%%%%%%%%%
\paragraph{v1.0:} 2017/04/27

\begin{itemize}
\item
manual and install package
\item
first version published on CTAN
\end{itemize}

%%%%%%%%%%%%%%%%%%%%%%%%%%%%%%%%%%%%%%%%
\paragraph{v0.6:} 2017/04/26

\begin{itemize}
\item
redirection mechanism added
\end{itemize}

%%%%%%%%%%%%%%%%%%%%%%%%%%%%%%%%%%%%%%%%
\paragraph{v0.5:} 2017/04/26

\begin{itemize}
\item
functionality in definition file
\end{itemize}


%%%%%%%%%%%%%%%%%%%%%%%%%%%%%%%%%%%%%%%%%%%%%%%%%%%%%%%%%%%%%%%%%%%%%%%%%%%%%%%%
%%%%%%%%%%%%%%%%%%%%%%%%%%%%%%%%%%%%%%%%%%%%%%%%%%%%%%%%%%%%%%%%%%%%%%%%%%%%%%%%
%%%%%%%%%%%%%%%%%%%%%%%%%%%%%%%%%%%%%%%%%%%%%%%%%%%%%%%%%%%%%%%%%%%%%%%%%%%%%%%%
\appendix

\settowidth\MacroIndent{\rmfamily\scriptsize 000\ }

 \DocInput{childdoc.dtx}

\end{document}
%</driver>
% \fi
%
% %%%%%%%%%%%%%%%%%%%%%%%%%%%%%%%%%%%%%%%%%%%%%%%%%%%%%%%%%%%%%%%%%%%%%%%%%%%%%%
% %%%%%%%%%%%%%%%%%%%%%%%%%%%%%%%%%%%%%%%%%%%%%%%%%%%%%%%%%%%%%%%%%%%%%%%%%%%%%%
% \section{Sample}
%\iffalse
%<*samplemain>
%\fi
%
% The following presents a sample document
% with two chapters, two parts, a title page,
% a compile flag as well as three forwarding files to set the flag.
% It consists of eight |.tex| files:
% \begin{center}
% \begin{tabular}{ll}
% |cdocsamp.tex|&main file\\
% |cdocsch1.tex|&include file for chapter 1\\
% |cdocsch2.tex|&include file for chapter 2\\
% |cdocspt3.tex|&include file for part 3\\
% |cdocspt4.tex|&include file for part 4\\
% |cdocsdrf.tex|&forwarding file for main file in draft mode\\
% |cdocsfi1.tex|&forwarding file for final version of chapter 1\\
% |cdocsfi2.tex|&forwarding file for final version of chapter 2\\
% \end{tabular}
% \end{center}
% Each of the eight files can be compiled directly by the \LaTeX{} compiler.
%
% %%%%%%%%%%%%%%%%%%%%%%%%%%%%%%%%%%%%%%
% \paragraph{Main File.}
%
% The main file is called |cdocsamp.tex|.
%
% Load the \textsf{childdoc} definitions and
% declare the filename for the main document:
%    \begin{macrocode}
\input{childdoc.def}
\childdocmain{}
%    \end{macrocode}

% Optional override for |\version| flag:
%    \begin{macrocode}
%%\ifchilddoc\else\providecommand{\version}{draft}\fi
%    \end{macrocode}

% Define the default values for the |\version| flag
% (|final| for the main file and |draft| for childs):
%    \begin{macrocode}
\ifchilddoc
\providecommand{\version}{draft}
\else
\providecommand{\version}{final}
\fi
%    \end{macrocode}

% Load the standard document class:
%    \begin{macrocode}
\documentclass[12pt]{article}
%    \end{macrocode}

% Start the document body:
%    \begin{macrocode}
\begin{document}
%    \end{macrocode}

% Declare a title page.
% Print title, part of document being processed and version flag:
%    \begin{macrocode}
\addtocounter{page}{-1}
\begin{center}
{\LARGE\bfseries{}childdoc example\par}
\vspace{1cm}
\ifchilddoc
\ifchilddocmanual part\else chapter\fi:
`\childdocname' of `\childdocjob'\par
\else
main document: `\childdocjob'\par
\fi
version: \version\par
\end{center}
\newpage
%    \end{macrocode}

% Manually include selected file,
% otherwise process as usual:
%    \begin{macrocode}
\ifchilddocmanual
\section*{part `\childdocname'}
\input{\childdocname}
\else
%    \end{macrocode}

% Include the two chapters:
%    \begin{macrocode}
\include{cdocsch1}
\include{cdocsch2}
%    \end{macrocode}

% Include the two parts unless only chapters should be displayed:
%    \begin{macrocode}
\ifchilddoc\else
\section{part three}
\input{cdocspt3}
\section{part four}
\input{cdocspt4}
\fi
%    \end{macrocode}

% Process as usual until here:
%    \begin{macrocode}
\fi
%    \end{macrocode}

% End of document body:
%    \begin{macrocode}
\end{document}
%    \end{macrocode}
%\iffalse
%</samplemain>
%\fi
%
% %%%%%%%%%%%%%%%%%%%%%%%%%%%%%%%%%%%%%%
% \paragraph{Chapter Include Files.}
%
% The include files are called |cdocsch1.tex| and |cdocsch2.tex|.
%
%\iffalse
%<*samplechap1|samplechap2>
%\fi

% Optional override for |\version| flag:
%    \begin{macrocode}
%%\providecommand{\version}{final}
%    \end{macrocode}

% Include the main document:
%    \begin{macrocode}
\input{childdoc.def}
\childdocof{cdocsamp}
%    \end{macrocode}

%\iffalse
%</samplechap1|samplechap2>
%\fi
%
%\iffalse
%<*samplechap1>
%\fi
% Some text for chapter 1:
%    \begin{macrocode}
\section{one}
some text in chapter one
%    \end{macrocode}

%\iffalse
%</samplechap1>
%\fi
% Some text for chapter 2:
%\iffalse
%<*samplechap2>
%\fi
%    \begin{macrocode}
\section{two}
more text in chapter two
%    \end{macrocode}

%\iffalse
%</samplechap2>
%\fi
%
% %%%%%%%%%%%%%%%%%%%%%%%%%%%%%%%%%%%%%%
% \paragraph{Part Include Files.}
%
% The include files are called |cdocspt3.tex| and |cdocspt4.tex|.
%
%\iffalse
%<*samplepart3|samplepart4>
%\fi

% Optional override for |\version| flag:
%    \begin{macrocode}
%%\providecommand{\version}{final}
%    \end{macrocode}

% Include the main document:
%    \begin{macrocode}
\input{childdoc.def}
\childdocby{cdocsamp}
%    \end{macrocode}

%\iffalse
%</samplepart3|samplepart4>
%\fi
%
%\iffalse
%<*samplepart3>
%\fi
% Some text for part 3:
%    \begin{macrocode}
some text in part three
%    \end{macrocode}

%\iffalse
%</samplepart3>
%\fi
% Some text for part 4:
%\iffalse
%<*samplepart4>
%\fi
%    \begin{macrocode}
more text in part four
%    \end{macrocode}

%\iffalse
%</samplepart4>
%\fi
%
% %%%%%%%%%%%%%%%%%%%%%%%%%%%%%%%%%%%%%%
% \paragraph{Forwarding for a Complete Draft.}
%
% The following forwarding file |cdocsdrf.tex|
% compiles the main document in draft mode:
%\iffalse
%<*sampledraft>
%\fi
%    \begin{macrocode}
\def\version{draft}
\input{childdoc.def}
\childdocforward{cdocsamp}
%    \end{macrocode}

%\iffalse
%</sampledraft>
%\fi
%
% %%%%%%%%%%%%%%%%%%%%%%%%%%%%%%%%%%%%%%
% \paragraph{Forwarding for Final Version of the Chapters.}
%
% The following forwarding files |cdocsfn1.tex| and |cdocsfn2.tex|
% (with identical content)
% compile the final versions of the child documents
% |cdocsch1.tex| and |cdocsch2.tex|, respectively:
%\iffalse
%<*samplefinal>
%\fi
%    \begin{macrocode}
\def\version{final}
\input{childdoc.def}
\childdocforwardprefix[cdocsamp]{cdocsfn}{cdocsch}
%    \end{macrocode}

%\iffalse
%</samplefinal>
%\fi
%
% %%%%%%%%%%%%%%%%%%%%%%%%%%%%%%%%%%%%%%
% \paragraph{Command Line Processing.}
%
% The following three command lines generate the output files
% |cdocscld|, |cdocscl1| and |cdocscl2|
% which should be identical to
% |cdocsdrf|, |cdocsch1| and |cdocsfn2|, respectively:
% \begin{center}
% \begin{tabular}{l}
% |latex -jobname cdocscld \|\\
% |  "\def\version{draft}\input{childdoc.def}\childdocforward{cdocsamp}"|\\
% |latex -jobname cdocscl1 \|\\
% |  "\input{childdoc.def}\childdocforward[cdocsamp]{cdocsch1}"|\\
% |latex -jobname cdocscl2 \|\\
% |  "\def\version{final}\input{childdoc.def}\childdocforward{cdocsch2}"|
% \end{tabular}
% \end{center}
% Note that the trailing backslash on each first line
% merely continues the input to the second line
% (for convenient cut ant paste).
% Furthermore, the command |latex| can be replaced by any
% of its alternative versions such as |pdflatex|.
%
% %%%%%%%%%%%%%%%%%%%%%%%%%%%%%%%%%%%%%%%%%%%%%%%%%%%%%%%%%%%%%%%%%%%%%%%%%%%%%%
% %%%%%%%%%%%%%%%%%%%%%%%%%%%%%%%%%%%%%%%%%%%%%%%%%%%%%%%%%%%%%%%%%%%%%%%%%%%%%%
% \section{Implementation}
%\iffalse
%<*package>
%\fi
%
% This section describes the definitions file |childdoc.def|.

% The definitions cannot be loaded using |\usepackage| or |\RequirePackage|
% which has a mechanism to prevent loading a style file more than once.
% When loading the definitions by means of |\input|
% multiple instances have to be prevented manually:
%\iffalse
%This code needs to be before the `\ProvidesFile' directive
%which is defined at the beginning of this file.
%Therefore it is also placed there and commented out here.
%</package>
%<*discard>
%\fi
%    \begin{macrocode}
\ifdefined\childdocmain\endinput\fi
%    \end{macrocode}
%\iffalse
%</discard>
%<*package>
%\fi
%
% \macro{\ifchilddoc}
% \macro{\ifchilddocmanual}
% The conditional |\ifchilddoc| tells whether a
% child (true) or main (false) document is being compiled.
% The conditional |\ifchilddocmanual| tells whether
% the |\includeonly| mechanism is used (false) or
% the selection of child files must be performed manually (true).
% The definitions initialise to false:
%    \begin{macrocode}
\newif\ifchilddoc
\newif\ifchilddocmanual
%    \end{macrocode}

% \macro{\childdocname}
% \macro{\childdocjob}
% The macro |\childdocname| stores the name of the main document
% to be compiled. The macro |\childdocjob| stores the name of
% the document on which the \LaTeX{} compiler was originally invoked.
% The content of |\jobname| cannot be compared
% to filenames specified in the source due to different catcodes.
% The following code rescans |\jobname|, stores the result
% in |\childdocname| and saves a copy in |\childdocjob|:
%    \begin{macrocode}
\edef\childdocname{\scantokens\expandafter{\jobname\noexpand}}
\let\childdocjob\childdocname
%    \end{macrocode}

% \macro{\childdocdisable}
% The macro |\childdocdisable| prevents the main file
% from being processed more than once.
% At this stage, the main document command |\childdocmain|
% is assumed to be called once again where it should do nothing.
% Any subsequent call to it should prevent
% a secondary processing of the main document
% It overwrites the forwarding commands
% |\childdocof| and |\childdocforward|
% with empty macros to prevent further inclusions of the main document:
%    \begin{macrocode}
\newcommand{\childdocdisable}
{
  \renewcommand{\childdocmain}[1]{\renewcommand{\childdocmain}[1]{\endinput}}
  \renewcommand{\childdocof}[1]{}
  \renewcommand{\childdocby}[2][]{}
  \renewcommand{\childdocforward}[2][]{}
  \renewcommand{\childdocdisable}{}
}
%    \end{macrocode}

% \macro{\childdocmain}
% The macro |\childdocmain| is to be called at the top of the main file
% with nothing or the main filename (without extension) as argument.
% First, it breaks loops.
% If the argument is not empty and does not match |\childdocname|
% (which is set by the first inclusion of |childdoc.def|),
% |\ifchilddoc| is set to true, |\includeonly| is applied to the child file
% and |\jobname| is set to the main file
% (for proper handling of |.aux| files):
%    \begin{macrocode}
\newcommand{\childdocmain}[1]
{
  \childdocdisable\childdocmain{}
  \if?#1?\else
    \begingroup
      \def\childdoctmp{#1}
      \ifx\childdoctmp\childdocname
        \def\childdoctmp{}
      \else
        \def\childdoctmp
        {
          \childdoctrue
          \includeonly{\childdocname}
          \def\childdocjob{#1}
          \def\jobname{#1}
        }
      \fi
      \expandafter
    \endgroup
    \childdoctmp
  \fi
}
%    \end{macrocode}

% \macro{\childdocof}
% The command |\childdocof| redirects
% compilation to the main file |#1|.
%    \begin{macrocode}
\newcommand{\childdocof}[1]
{
  \childdocdisable
  \childdoctrue
  \includeonly{\childdocname}
  \def\jobname{#1}
  \def\childdocjob{#1}
  \input{#1}
}
%    \end{macrocode}

% \macro{\childdocby}
% The command |\childdocby| ....
%    \begin{macrocode}
\newcommand{\childdocby}[2][]
{
  \childdocdisable
  \childdoctrue
  \childdocmanualtrue
  \if?#1?\else
    \def\jobname{#2}
  \fi
  \def\childdocjob{#2}
  \input{#2}
  \endinput
}
%    \end{macrocode}

% \macro{\childdocforward}
% The command |\childdocforward| redirects
% compilation to the main file or
% (if the optional argument is given) a child file.
% Parameters are set as if the main file
% or a child file starting with |\childdocof| was compiled.
% Then compilation is handed over to the main file:
%    \begin{macrocode}
\newcommand{\childdocforward}[2][]
{
  \begingroup
    \if?#1?
      \def\childdoctmp
      {
        \def\childdocname{#2}
        \def\childdocjob{#2}
        \def\jobname{#2}
        \input{#2}
        \endinput
      }
    \else
      \def\childdoctmp
      {
        \childdocdisable
        \def\childdocname{#2}
        \childdoctrue
        \includeonly{#2}
        \def\childdocjob{#1}
        \def\jobname{#1}
        \input{#1}
        \endinput
      }
    \fi
    \expandafter
  \endgroup
  \childdoctmp
}
%    \end{macrocode}

% \macro{\childdocforwardprefix}
% The command |\childdocforwardprefix| redirects
% compilation to the main or a child file by means of a pattern.
% The prefix |#1| in the current filename is replaced by |#2|
% and the suffix of the current filename is kept
% (it is assumed that the filename does not contain the substring `|~~~|'
% which is used as a delimiter).
% Compilation is handed over to the new file by |\childdocforward|:
%    \begin{macrocode}
\newcommand{\childdocforwardprefix}[3][]
{
  \begingroup
    \def\childdocextract #2##1~~~{\def\childdoctmp{\childdocforward[#1]{#3##1}}}
    \expandafter\childdocextract\childdocname~~~
    \expandafter
  \endgroup
  \childdoctmp
}
%    \end{macrocode}

% \macro{\childdoc}
% The deprecated macro |\childdoc| is a legacy version of |\childdocmain|:
%    \begin{macrocode}
\newcommand{\childdoc}{\childdocmain}
%    \end{macrocode}

% \macro{\childdocredirect}
% The deprecated macro |\childdocredirect| is a legacy version
% of |\childdocforward| and |\childdocforwardprefix|:
%    \begin{macrocode}
\newcommand{\childdocredirect}[2][]
{
  \begingroup
    \if?#1?
      \def\childdoctmp{\childdocforward{#2}}
    \else
      \def\childdoctmp{\childdocforwardprefix{#1}{#2}}
    \fi
    \expandafter
  \endgroup
  \childdoctmp
}
%    \end{macrocode}

%\iffalse
%</package>
%\fi
%
\endinput
|\\
|\childdocby{|\textit{main}|}|\\
\end{tabular}
\end{center}
%
The directive |\childdocby| is similar to |\childdocof|
described in \secref{sec:include},
but the subsequent selection of content must be done manually.
To that end, both |\ifchilddoc| and |\ifchilddocmanual|
will be true upon processing of a part,
and the name of the part is stored in |\childdocname|.
Note that |\jobname| will be set to the filename of the current part
so that each part receives an individual |.aux| file
that does not interfere with the |.aux| file(s) of the main document.
This behaviour can be altered by the alternative form
|\childdocby[*]{|\textit{main}|}| (with a non-empty optional argument)
which uses the |.aux| file of the main document
by setting |\jobname| to \textit{main}.

%%%%%%%%%%%%%%%%%%%%%%%%%%%%%%%%%%%%%%%%%%%%%%%%%%%%%%%%%%%%%%%%%%%%%%%%%%%%%%%%
\subsection{Driver Development}
\label{sec:driver}

The \textsf{childdoc} mechanism can also be use for the development
of definition files such as \LaTeX{} styles or classes.
This case differs from the above setup with multiple parts
included by |\include| in that no |\includeonly| should be invoked.
This can be achieved by starting the include file
(before |\ProvidesPackage|) with:
%
\begin{center}
\begin{tabular}{l}
|% \iffalse
%
% childdoc.dtx Copyright (C) 2017-2018 Niklas Beisert
%
% This work may be distributed and/or modified under the
% conditions of the LaTeX Project Public License, either version 1.3
% of this license or (at your option) any later version.
% The latest version of this license is in
%   http://www.latex-project.org/lppl.txt
% and version 1.3 or later is part of all distributions of LaTeX
% version 2005/12/01 or later.
%
% This work has the LPPL maintenance status `maintained'.
%
% The Current Maintainer of this work is Niklas Beisert.
%
% This work consists of the files childdoc.dtx and childdoc.ins
% and the derived files childdoc.def and cdocsamp.tex with
% cdocsch1.tex, cdocsch2.tex, cdocsdrf.tex, cdocsfn1.tex, cdocsfn2.tex.
%
%<package>\ifdefined\childdocmain\endinput\fi
%<package>\ProvidesFile{childdoc.def}[2018/12/30 v2.0 child document driver]
%<samplemain>\ProvidesFile{cdocsamp.tex}[2018/12/30 v2.0 sample for childdoc]
%<*driver>
%\ProvidesFile{childdoc.drv}[2018/12/30 v2.0 childdoc reference manual file]
\PassOptionsToClass{10pt,a4paper}{article}
\documentclass{ltxdoc}

\usepackage[margin=35mm]{geometry}
\usepackage{hyperref}
\usepackage{hyperxmp}
\usepackage[usenames]{color}

\hypersetup{colorlinks=true}
\hypersetup{pdfstartview=FitH}
\hypersetup{pdfpagemode=UseNone}
\hypersetup{pdfsource={}}
\hypersetup{pdflang={en-UK}}
\hypersetup{pdfcopyright={Copyright 2017-2018 Niklas Beisert.
  This work may be distributed and/or modified under the
  conditions of the LaTeX Project Public License, either version 1.3
  of this license or (at your option) any later version.}}
\hypersetup{pdflicenseurl={http://www.latex-project.org/lppl.txt}}
\hypersetup{pdfcontactaddress={ETH Zurich, ITP, HIT K,
  Wolfgang-Pauli-Strasse 27}}
\hypersetup{pdfcontactpostcode={8093}}
\hypersetup{pdfcontactcity={Zurich}}
\hypersetup{pdfcontactcountry={Switzerland}}
\hypersetup{pdfcontactemail={nbeisert@itp.phys.ethz.ch}}
\hypersetup{pdfcontacturl={http://people.phys.ethz.ch/\xmptilde nbeisert/}}

\newcommand{\secref}[1]{\hyperref[#1]{section \ref*{#1}}}

\parskip1ex
\parindent0pt
\let\olditemize\itemize
\def\itemize{\olditemize\parskip0pt}

\begin{document}

\title{The \textsf{childdoc} Package}
\hypersetup{pdftitle={The childdoc Package}}
\author{Niklas Beisert\\[2ex]
  Institut f\"ur Theoretische Physik\\
  Eidgen\"ossische Technische Hochschule Z\"urich\\
  Wolfgang-Pauli-Strasse 27, 8093 Z\"urich, Switzerland\\[1ex]
  \href{mailto:nbeisert@itp.phys.ethz.ch}
  {\texttt{nbeisert@itp.phys.ethz.ch}}}
\hypersetup{pdfauthor={Niklas Beisert}}
\hypersetup{pdfsubject={Manual for the LaTeX2e Package childdoc}}
\date{30 December 2018, \textsf{v2.0}}
\maketitle

\begin{abstract}\noindent
\textsf{childdoc} is a \LaTeXe{} package
that enables the direct compilation
of document sections included by |\include|
to individual files.
\end{abstract}

\begingroup
\parskip0ex
\tableofcontents
\endgroup

%%%%%%%%%%%%%%%%%%%%%%%%%%%%%%%%%%%%%%%%%%%%%%%%%%%%%%%%%%%%%%%%%%%%%%%%%%%%%%%%
%%%%%%%%%%%%%%%%%%%%%%%%%%%%%%%%%%%%%%%%%%%%%%%%%%%%%%%%%%%%%%%%%%%%%%%%%%%%%%%%
\section{Introduction}

\LaTeX{} provides a mechanism to structure a large document (such as a book)
into a main file and several child files (containing the chapters)
using the |\include| command.
This mechanism is beneficial for documents
which span hundreds of pages in order to
make the source file(s) more manageable.
Moreover, compilation can be restricted to
selected child files by means of the |\includeonly| command.
The latter feature can be used to reduce the compilation time while editing
(this was significantly more useful in the earlier days of \LaTeX{})
or to generate a smaller document which is easier to navigate.
Another application of |\includeonly| is to generate
documents consisting of selected parts of the complete document.

However, there are a few drawbacks of the plain |\include| mechanism:
\begin{itemize}
\item
The child files cannot be compiled on their own,
they can only be compiled via the main file.
A naive editing environment
(such as a text editor with an option
to have the current file processed by \LaTeX)
may require one to switch to the main file before compiling;
attempting to compile the child file produces errors.
\item
The main file must be modified (each time)
to adjust the |\includeonly| command
to the present needs. This easily leaves the main file in a messy state.
\item
The generated document will always carry the filename
of the main document. This is inconvenient if
several child files are to be compiled and
to be kept for distribution.
\end{itemize}

The present package provides a simple interface
to make child files individually compilable by \LaTeX{}.
Compiling a child file then has the same effect as compiling
the main file with an |\includeonly| command
to select the appropriate child.
Moreover the generated document will carry the name of the child
rather than the main file.
This resolves all three above issues.

This feature is meant to make the editing of books,
thesis documents and lecture notes somewhat more convenient.
However, the package can also be used efficiently for
composing a series of documents (such as exercise sheets)
which are typically distributed individually.
It then assists the author in generating the individual documents
(potentially in different versions)
as well as a document containing the collected series.
Another application is in developing style files
or other kinds of included material
where compilation of the style file could redirect
to a sample or test file.

%%%%%%%%%%%%%%%%%%%%%%%%%%%%%%%%%%%%%%%%%%%%%%%%%%%%%%%%%%%%%%%%%%%%%%%%%%%%%%%%
%%%%%%%%%%%%%%%%%%%%%%%%%%%%%%%%%%%%%%%%%%%%%%%%%%%%%%%%%%%%%%%%%%%%%%%%%%%%%%%%
\section{Usage}

First of all, the package \textsf{childdoc} is \emph{not} a standard
\LaTeXe{} |.sty| style file! Therefore it needs to be invoked in
a non-standard way.

%%%%%%%%%%%%%%%%%%%%%%%%%%%%%%%%%%%%%%%%%%%%%%%%%%%%%%%%%%%%%%%%%%%%%%%%%%%%%%%%
\subsection{Included Files}
\label{sec:include}

%%%%%%%%%%%%%%%%%%%%%%%%%%%%%%%%%%%%%%%%
\DescribeMacro{\childdocmain}
To use the package, add the commands
\begin{center}
\begin{tabular}{l}
|\input{childdoc.def}|\\
|\childdocmain{}|\\
\end{tabular}
\end{center}
at the very top of the main \LaTeX{} file,
in particular \emph{before} the |\documentclass| statement!
The argument of |\childdocmain| should be left empty
(but it must be present).

%%%%%%%%%%%%%%%%%%%%%%%%%%%%%%%%%%%%%%%%
\DescribeMacro{\childdocof}
Furthermore, add the commands
\begin{center}
\begin{tabular}{l}
|\input{childdoc.def}|\\
|\childdocof{|\textit{main}|}|\\
\end{tabular}
\end{center}
at the top of every child file \textit{child}
which is included by |\include{|\textit{child}|}|
from within the main file
(or at least for those files to be compiled individually).
The argument \textit{main} must be the filename of the main file.

There are a couple of
considerations in setting up the main and child documents:

%%%%%%%%%%%%%%%%%%%%%%%%%%%%%%%%%%%%%%%%
\paragraph{Restrictions.}

Please note the following restrictions:
\begin{itemize}
\item
|\childdocmain| must be called with one argument \textit{main}
to ensure compatibility with earlier version of the package.
It must either be empty (|\childdocmain{}|)
or precisely match the filename of the main file in which it is specified.
See \secref{sec:detection} for further information.
\item
The filename \textit{main} must be specified without the |.tex| extension.
\item
The filename \textit{main} is case sensitive
(even in case-insensitive file systems)
due to internal string comparison.
\item
The argument \textit{main} should be fully expanded, it cannot be a macro.
\item
Subdirectories and special characters should be avoided in filenames.
\item
The command |\childdocmain{|\textit{main}|}| must be followed by a whitespace.
It should not be followed immediately by another command
or by a comment mark `|%|'.
This is because the \TeX{} parser reads the token immediately following
the argument of |\childdocmain| and puts it
at the beginning of every child section;
however, a white\-space is ignored.
\end{itemize}

%%%%%%%%%%%%%%%%%%%%%%%%%%%%%%%%%%%%%%%%
\paragraph{Content of Main File.}

It is advisable to place all content in the child files included by |\include|.
Any output contained in the main file will appear in all child documents
unless suppressed manually;
it cannot be suppressed automatically by the |\includeonly| directive
and thus should normally be avoided.
A method to include some content in the main file
by means of conditional processing is described in \secref{sec:conditional}.

%%%%%%%%%%%%%%%%%%%%%%%%%%%%%%%%%%%%%%%%
\paragraph{Page Numbering.}

When only a part of the document is compiled,
the appropriate numbering of pages
(as well as other status parameters)
is determined from the |.aux| files.
The latter contain information from previous passes.
However this information needs to propagate through
all intermediate child documents.
Therefore the page numbering in child documents may well
be inconsistent until the complete document is compiled at least once.

A useful (if unconventional) way to always ensure a consistent
page numbering is to restart the numbering in each child document
and denote the pages by `\textit{child}|.|\textit{page}'
where \textit{child} represents the chapter/section number of the child file.
This can be achieved by the command
|\numberwithin{page}{|\textit{child}|}|
of the \textsf{amsmath} package
where \textit{child} can be |chapter| or |section|
depending on the chosen structuring.
Alternatively, one can modify the macro |\thepage| appropriately
and reset the counter |page| at the start of each child file.

%%%%%%%%%%%%%%%%%%%%%%%%%%%%%%%%%%%%%%%%%%%%%%%%%%%%%%%%%%%%%%%%%%%%%%%%%%%%%%%%
\subsection{Conditional Processing}
\label{sec:conditional}

The package provides a mechanism to compile different versions
of a document. To customise the versions further some conditional processing
can come in handy to distinguish which version is being compiled.
The package provides two macros to describe the compilation context:

%%%%%%%%%%%%%%%%%%%%%%%%%%%%%%%%%%%%%%%%
\DescribeMacro{\ifchilddoc}
The conditional |\ifchilddoc| distinguishes between the compilation of
child documents and the main document:
%
\begin{center}
|\ifchilddoc |\textit{child-code}| |[|\||else |\textit{main-code}]| \||fi|
\end{center}

%%%%%%%%%%%%%%%%%%%%%%%%%%%%%%%%%%%%%%%%
\DescribeMacro{\childdocname}
\DescribeMacro{\childdocjob}
The macro |\childdocname| contains the filename (without extension)
of the main or child file being processed.
Note that |\childdocjob| will always contain the name of the main file.

%%%%%%%%%%%%%%%%%%%%%%%%%%%%%%%%%%%%%%%%
\paragraph{Title Page.}

Conditional processing can be used to include a title or banner page
in the main document when proper precautions are taken.
Importantly, the code in the main file should ensure that the page counter
(as well as other status parameters which are stored in the |.aux| files)
takes the same value after the conditional processing.
Otherwise the page numbers may take divergent values
depending on which part is compiled.

For example, a title page could be declared by:
%
\begin{center}
\begin{tabular}{l}
|\ifchilddoc\||else|\\
|\addtocounter{page}{-1}|\\
\textit{code for title page}\\
|\newpage|\\
|\||fi|
\end{tabular}
\end{center}
%
A banner page for the child documents can be generated by:
%
\begin{center}
\begin{tabular}{l}
|\ifchilddoc|\\
|\addtocounter{page}{-1}|\\
\textit{code for banner page}\\
|\newpage|\\
|\||fi|
\end{tabular}
\end{center}
%
Here one could write a message such as:
\begin{center}
|This is the part \childdocname{} of \childdocjob{}.|
\end{center}

%%%%%%%%%%%%%%%%%%%%%%%%%%%%%%%%%%%%%%%%%%%%%%%%%%%%%%%%%%%%%%%%%%%%%%%%%%%%%%%%
\subsection{Flags}
\label{sec:flags}

The package makes it easy to generate different versions
of the main or child documents.
To this end compilation flags can be defined
and assigned different default values.
They will be particularly useful in conjunction
with the forwarding mechanism described in \secref{sec:forward}.

For example, it may be useful to have a flag |\version|
which can be set to |draft| or |final|.
The document source will contain some conditional code
depending on the value of |\version|.
Suppose further, the flag should default to |final| for the main file
and to |draft| for child files
which is a natural assignment for editing the document.
This is achieved by placing the following code
in the preamble of the main document
(below the |\childdocmain| directive):
%
\begin{center}
\begin{tabular}{l}
|\ifchilddoc|\\
|\providecommand{\version}{draft}|\\
|\||else|\\
|\providecommand{\version}{final}|\\
|\||fi|
\end{tabular}
\end{center}
%
The definition by |\providecommand| makes sure
that previous definitions are not overwritten.
Further statements |\providecommand{\version}{...}|
can thus be added before the above code to override it.

For the main file, one might add a line
(between |\childdocmain| and the above block)
%
\begin{center}
|%\ifchilddoc\||else\providecommand{\version}{draft}\||fi|
\end{center}
%
which can be uncommented to produce a draft version.
Likewise one can add a line to the very top of a child file
(above the |\childdocof{|\textit{main}|}| directive)
%
\begin{center}
|%\providecommand{\version}{final}|
\end{center}
%
which can be uncommented to produce the final version of this child document.

%%%%%%%%%%%%%%%%%%%%%%%%%%%%%%%%%%%%%%%%%%%%%%%%%%%%%%%%%%%%%%%%%%%%%%%%%%%%%%%%
\subsection{Forwarding}
\label{sec:forward}

Different versions of the main or child documents
using compilation flags as described in \secref{sec:flags}
can be (permanently) stored in different files
for convenient compilation, viewing and distribution.
To this end, the package defines a command
to pass on compilation to a different file:

%%%%%%%%%%%%%%%%%%%%%%%%%%%%%%%%%%%%%%%%
\DescribeMacro{\childdocforward}
The command |\childdocforward| redirects processing to
another source file:
%
\begin{center}
\begin{tabular}{l}
|\input{childdoc.def}|\\
|\childdocforward[|\textit{main}|]{|\textit{dest}|}|\\
\end{tabular}
\end{center}
%
The argument \textit{dest} is the destination file
(without extension).
It should be the main file or one of the child files.
Note that further \textsf{childdoc} directives
such as |\childdocof| and |\childdocforward|
in the indicated file will be processed in this form.
The optional argument \textit{main}
passes on directly to the main file \textit{main}
while pretending to compile the child \textit{dest}.
This form behaves as if \textit{dest}
issues |\childdocof{|\textit{main}|}| right away,
and no further \textsf{childdoc} directives will be processed.

%%%%%%%%%%%%%%%%%%%%%%%%%%%%%%%%%%%%%%%%
\DescribeMacro{\...prefix}
In the alternative form |\childdocforwardprefix|,
%
\begin{center}
\begin{tabular}{l}
|\input{childdoc.def}|\\
|\childdocforwardprefix[|\textit{main}|]{|\textit{prefix}|}{|\textit{dest}|}|
\end{tabular}
\end{center}
%
the destination file is determined by a pattern
depending on the current file:
To make this work, the current file must be called
`{\textit{prefix}\hspace{0.2em}\textit{suffix}}'
with \textit{prefix} matching precisely the argument.
Processing is then passed on to the file
`{\textit{dest}\hspace{0.2em}\textit{suffix}}'.
Surely, the same effect is achieved by
directly specifying the
argument `{\textit{dest}\hspace{0.2em}\textit{suffix}}'
in the first form.
However, that requires to set up a different file
for each child. With the alternative form of the command
all these files can have exactly the same content
which simplifies setting them up and maintaining them.

For example, the following file |draft.tex|
with a compilation flag |\version| as described in \secref{sec:flags}
compiles the main document as a draft:
%
\begin{center}
\begin{tabular}{l}
|\def\version{draft}|\\
|\input{childdoc.def}|\\
|\childdocforward{|\textit{main}|}|
\end{tabular}
\end{center}
%
Likewise, the following files |final|\textit{nn}|.tex|
compile the final version of the child document
|child|\textit{nn}|.tex|:
%
\begin{center}
\begin{tabular}{l}
|\def\version{final}|\\
|\input{childdoc.def}|\\
|\childdocforwardprefix{final}{child}|
\end{tabular}
\end{center}
%

Note that when several versions of a main file and/or of each child file
are to be generated, it may be convenient to set up a |Makefile| or
shell script to automatise the process.

%%%%%%%%%%%%%%%%%%%%%%%%%%%%%%%%%%%%%%%%%%%%%%%%%%%%%%%%%%%%%%%%%%%%%%%%%%%%%%%%
\subsection{Command Line Processing}
\label{sec:commandline}

The effect of redirection files can also be achieved by invoking
the \LaTeX{} compiler with a more elaborate command line.
Most conveniently this should be done as part
of a shell script or a |Makefile|.

When using \textsf{childdoc} in the main file, the following
command lines effectively perform a redirection
(note that depending on the shell being used,
backslashes may have to be doubled: `|\|' $\to$ `|\\|'):
%
\begin{center}
|... -jobname "|\textit{target}|" |\\|"|[\textit{flags}]%
|\input{childdoc.def}\childdocforward[|\textit{main}|]{|\textit{dest}|}"|
\end{center}
%
Here \textit{target} is the name of the output file,
\textit{main} is the name of the main file
and \textit{dest} is the name of the main or child file to be processed
(all filenames without extensions).
The optional argument \textit{main} can be omitted
if \textit{main} matches \textit{dest}.
Optionally, compilation \textit{flags} can be defined via |\def| commands.
This command line makes the \TeX{} engine believe
it is compiling the file \textit{target}
whose content is specified as the latter parameter.
The provided code then forwards the processing to
\textit{main} or \textit{dest} as described in \secref{sec:forward}.

%%%%%%%%%%%%%%%%%%%%%%%%%%%%%%%%%%%%%%%%%%%%%%%%%%%%%%%%%%%%%%%%%%%%%%%%%%%%%%%%
\subsection{Include by Input}
\label{sec:input}

Including child documents by |\include| has some restrictions by design.
Most notably, the content of a child document always occupies
its own set of pages; pages cannot be shared between child documents.
Usually, this behaviour makes perfect sense
because each child document contain an essential part of the document.
However, in some situations it may be desirable to compose
a document from a collection of parts
without having mandatory page breaks between then.
For this case, the package
provides a mechanism to include parts
by |\input| which can also be processed individually.
However, by construction this mechanism
requires manual handling of the content to be output.

%%%%%%%%%%%%%%%%%%%%%%%%%%%%%%%%%%%%%%%%
\DescribeMacro{\ifchilddocmanual}
The main file should be prepared as usual, see \secref{sec:include}.
However, the document body must make a distinction
between processing of an individual part and of the main document, e.g.:
%
\begin{center}
\begin{tabular}{l}
|\ifchilddocmanual|\\
|\input{\childdocname}|\\
|\||else|\\
\textit{document body with }|\input{|\textit{part}|}|\\
|\||fi|
\end{tabular}
\end{center}
%
The conditional |\ifchilddocmanual| is true whenever
a part to be included by |\input| is being compiled,
and the name of the part is stored in |\childdocname|.

%%%%%%%%%%%%%%%%%%%%%%%%%%%%%%%%%%%%%%%%
\DescribeMacro{\childdocby}
Each part to be included by |\input| should start with:
%
\begin{center}
\begin{tabular}{l}
|\input{childdoc.def}|\\
|\childdocby{|\textit{main}|}|\\
\end{tabular}
\end{center}
%
The directive |\childdocby| is similar to |\childdocof|
described in \secref{sec:include},
but the subsequent selection of content must be done manually.
To that end, both |\ifchilddoc| and |\ifchilddocmanual|
will be true upon processing of a part,
and the name of the part is stored in |\childdocname|.
Note that |\jobname| will be set to the filename of the current part
so that each part receives an individual |.aux| file
that does not interfere with the |.aux| file(s) of the main document.
This behaviour can be altered by the alternative form
|\childdocby[*]{|\textit{main}|}| (with a non-empty optional argument)
which uses the |.aux| file of the main document
by setting |\jobname| to \textit{main}.

%%%%%%%%%%%%%%%%%%%%%%%%%%%%%%%%%%%%%%%%%%%%%%%%%%%%%%%%%%%%%%%%%%%%%%%%%%%%%%%%
\subsection{Driver Development}
\label{sec:driver}

The \textsf{childdoc} mechanism can also be use for the development
of definition files such as \LaTeX{} styles or classes.
This case differs from the above setup with multiple parts
included by |\include| in that no |\includeonly| should be invoked.
This can be achieved by starting the include file
(before |\ProvidesPackage|) with:
%
\begin{center}
\begin{tabular}{l}
|\input{childdoc.def}|\\
|\childdocforward{|\textit{main}|}|\\
\end{tabular}
\end{center}
%
or alternatively with:
%
\begin{center}
\begin{tabular}{l}
|\input{childdoc.def}|\\
|\childdocby{|\textit{main}|}|\\
\end{tabular}
\end{center}
%
Both forms have slightly different effects as described above.
The main file is prepared as usual, see \secref{sec:include}.

%%%%%%%%%%%%%%%%%%%%%%%%%%%%%%%%%%%%%%%%%%%%%%%%%%%%%%%%%%%%%%%%%%%%%%%%%%%%%%%%
\subsection{Legacy Detection}
\label{sec:detection}

The directive |\childdocmain| in the main file can detect
whether the complete document or merely a child is to be compiled
even without using the directive |\childdocof|.
This method is deprecated because it is less robust
and there is no compelling reason to use it;
it is merely provided for backward compatibility
and it may be removed in future versions.

If the detection mechanism is to be used,
it is mandatory to correctly specify
the filename of the main file as the argument of |\childdocmain|:
%
\begin{center}
\begin{tabular}{l}
|\input{childdoc.def}|\\
|\childdocmain{|\textit{main}|}|\\
\end{tabular}
\end{center}
%
If |\jobname| does not match the argument \textit{main} of |\childdocmain|,
it is assumed that |\jobname| points to the child file to be compiled.
When using |\childdocmain| with the main file specified as argument,
it suffices to start a child file
with just |\input{|\textit{main}|}|
without loading of the package and using |\childdocof|.
If instead all processing is done
with the appropriate \textsf{childdoc} directives,
the argument of \textit{main} of |\childdocmain| can be empty.

An alternative version of the command line processing described
in \secref{sec:commandline} using the detection mechanism reads:
%
\begin{center}
|... -jobname "|\textit{target}|" "|[\textit{flags}]%
[|\def\jobname{|\textit{dest}|}|]|\input{|\textit{main}|}"|
\end{center}

%%%%%%%%%%%%%%%%%%%%%%%%%%%%%%%%%%%%%%%%%%%%%%%%%%%%%%%%%%%%%%%%%%%%%%%%%%%%%%%%
\subsection{Manual Code}
\label{sec:manual}

In case one cannot be certain whether the definitions file |childdoc.def|
is installed on the target \TeX{} distribution
and one prefers not to ship it,
it is conceivable to paste a few relevant commands into the sources.

To that end, drop all statements |\input{childdoc.def}|
and perform the replacements as outlined below.
Instead of |\childdocmain{|\textit{main}|}| add the following code
to the top of the main file:
%
\begin{center}
\begin{tabular}{l}
|\||ifdefined\childdocname\endinput\||fi\newif\ifchilddoc|\\
|\edef\childdocname{\scantokens\expandafter{\jobname\noexpand}}|\\
|\def\childdocmain{|\textit{main}|}\||ifx\childdocmain\childdocname\||else|\\
|\childdoctrue\includeonly{\childdocname}\let\jobname\childdocmain\||fi|\\
\end{tabular}
\end{center}
%
Instead of |\childdocof{|\textit{main}|}| just include the main file
at the top of each child file:
%
\begin{center}
|\input{|\textit{main}|}|
\end{center}
%
A simple redirection |\childdocforward{|\textit{dest}|}| is achieved by:
%
\begin{center}
|\def\jobname{|\textit{dest}|}\input{\jobname}|
\end{center}
%
The redirection with prefix
|\childdocforwardprefix[|\textit{prefix}|]{|\textit{dest}|}|
is accomplished by:
%
\begin{center}
\begin{tabular}{l}
|{\edef\jobname{\scantokens\expandafter{\jobname\noexpand}}|\\
|\def\redirectjob |\textit{prefix}|#1~~~{\gdef\jobname{|\textit{dest}|#1}}|\\
|\expandafter\redirectjob\jobname~~~}\input{\jobname}|
\end{tabular}
\end{center}

In an alternative approach,
child documents can be compiled by a specific command line
without additional code or specific definitions:
%
\begin{center}
|... -jobname "|\textit{target}|" "|[\textit{flags}]%
|\includeonly{|\textit{dest}|}\input{|\textit{main}|}"|
\end{center}
%

%%%%%%%%%%%%%%%%%%%%%%%%%%%%%%%%%%%%%%%%%%%%%%%%%%%%%%%%%%%%%%%%%%%%%%%%%%%%%%%%
%%%%%%%%%%%%%%%%%%%%%%%%%%%%%%%%%%%%%%%%%%%%%%%%%%%%%%%%%%%%%%%%%%%%%%%%%%%%%%%%
\section{Information}

%%%%%%%%%%%%%%%%%%%%%%%%%%%%%%%%%%%%%%%%%%%%%%%%%%%%%%%%%%%%%%%%%%%%%%%%%%%%%%%%
\subsection{Copyright}

Copyright \copyright{} 2017--2018 Niklas Beisert

This work may be distributed and/or modified under the
conditions of the \LaTeX{} Project Public License, either version 1.3
of this license or (at your option) any later version.
The latest version of this license is in
  \url{http://www.latex-project.org/lppl.txt}
and version 1.3 or later is part of all distributions of \LaTeX{}
version 2005/12/01 or later.

This work has the LPPL maintenance status `maintained'.

The Current Maintainer of this work is Niklas Beisert.

This work consists of the files |README.txt|, |childdoc.ins| and |childdoc.dtx|
as well as the derived files |childdoc.def|, |cdocsamp.tex|
with |cdocsch1.tex|, |cdocsch2.tex|, |cdocspt3.tex|, |cdocspt4.tex|,
|cdocsdrf.tex|, |cdocsfn1.tex|, |cdocsfn2.tex|
as well as |childdoc.pdf|.

%%%%%%%%%%%%%%%%%%%%%%%%%%%%%%%%%%%%%%%%%%%%%%%%%%%%%%%%%%%%%%%%%%%%%%%%%%%%%%%%
\subsection{Files and Installation}

The package consists of the files:
%
\begin{center}
\begin{tabular}{ll}
    |README.txt|   & readme file \\
    |childdoc.ins| & installation file \\
    |childdoc.dtx| & source file \\
    |childdoc.def| & definition file \\
    |cdocsamp.tex| & sample main file \\
    |cdocsch1.tex| & sample include file \\
    |cdocsch2.tex| & sample include file \\
    |cdocspt3.tex| & sample part file \\
    |cdocspt4.tex| & sample part file \\
    |cdocsdrf.tex| & sample redirection file \\
    |cdocsfn1.tex| & sample redirection file \\
    |cdocsfn2.tex| & sample redirection file \\
    |childdoc.pdf| & manual
\end{tabular}
\end{center}
%
The distribution consists of the files
|README.txt|, |childdoc.ins| and |childdoc.dtx|.
%
\begin{itemize}
\item
Run (pdf)\LaTeX{} on |childdoc.dtx|
to compile the manual |childdoc.pdf| (this file).
\item
Run \LaTeX{} on |childdoc.ins| to create the definitions file |childdoc.def|
and the sample |cdocsamp.tex| with include files
|cdocsch1.tex|, |cdocsch2.tex|, |cdocspt3.tex|, |cdocspt4.tex|,
|cdocsdrf.tex|, |cdocsfn1.tex|, |cdocsfn2.tex|.
Then copy the file |childdoc.def| to an appropriate directory of your \LaTeX{}
distribution, e.g.\ \textit{texmf-root}|/tex/latex/childdoc|.
\end{itemize}

%%%%%%%%%%%%%%%%%%%%%%%%%%%%%%%%%%%%%%%%%%%%%%%%%%%%%%%%%%%%%%%%%%%%%%%%%%%%%%%%
\subsection{Related CTAN Packages}

There are several other packages which offer a similar functionality:
%
\begin{itemize}
\item
The packages
\href{http://ctan.org/pkg/docmute}{\textsf{docmute}},
\href{http://ctan.org/pkg/includex}{\textsf{includex}} and
\href{http://ctan.org/pkg/standalone}{\textsf{standalone}}
provide commands to include only the document body of
a child file thus allowing both files to be compiled individually.
\item
The packages \href{http://ctan.org/pkg/subdocs}{\textsf{subdocs}}
and \href{http://ctan.org/pkg/subfiles}{\textsf{subfiles}}
provide structures in which the main and child documents can be
encapsulated and allowing them to be compiled individually.
The inclusion mechanism is different from the conventional |\include|.
\item
The package \href{http://ctan.org/pkg/combine}{\textsf{combine}}
is an elaborate solution to combine several documents into one.
\end{itemize}
%
See also the CTAN topic \href{http://ctan.org/topic/subdocs}{\textsf{subdocs}}
for further related packages.
The present package differs from the above solutions in that
a document structure constructed with the conventional |\include| mechanism
just needs two extra commands at the top of every file
such that all constituent files can be compiled individually.

%%%%%%%%%%%%%%%%%%%%%%%%%%%%%%%%%%%%%%%%%%%%%%%%%%%%%%%%%%%%%%%%%%%%%%%%%%%%%%%%
%\subsection{Feature Suggestions}
%
%The following is a list of features which may be useful for future
%versions of this package:
%%
%\begin{itemize}
%\item
%\ldots
%\end{itemize}

%%%%%%%%%%%%%%%%%%%%%%%%%%%%%%%%%%%%%%%%%%%%%%%%%%%%%%%%%%%%%%%%%%%%%%%%%%%%%%%%
\subsection{Revision History}

%%%%%%%%%%%%%%%%%%%%%%%%%%%%%%%%%%%%%%%%
\paragraph{v2.0:} 2018/12/30

\begin{itemize}
\item
immediate forward processing
\item
added |\childdocby| mechanism
\item
manual restructured
\end{itemize}

%%%%%%%%%%%%%%%%%%%%%%%%%%%%%%%%%%%%%%%%
\paragraph{v1.6:} 2018/01/17

\begin{itemize}
\item
application for development of include files
\item
corrections to manual
\end{itemize}

%%%%%%%%%%%%%%%%%%%%%%%%%%%%%%%%%%%%%%%%
\paragraph{v1.5:} 2017/05/21

\begin{itemize}
\item
more complete structuring introduced
\item
|\childdocof| introduced
\item
|\childdoc| renamed to |\childdocmain|
\item
|\childredirect| renamed to |\childdocforward| and |\childdocforwardprefix|
and functionality expanded
\end{itemize}

%%%%%%%%%%%%%%%%%%%%%%%%%%%%%%%%%%%%%%%%
\paragraph{v1.0:} 2017/04/27

\begin{itemize}
\item
manual and install package
\item
first version published on CTAN
\end{itemize}

%%%%%%%%%%%%%%%%%%%%%%%%%%%%%%%%%%%%%%%%
\paragraph{v0.6:} 2017/04/26

\begin{itemize}
\item
redirection mechanism added
\end{itemize}

%%%%%%%%%%%%%%%%%%%%%%%%%%%%%%%%%%%%%%%%
\paragraph{v0.5:} 2017/04/26

\begin{itemize}
\item
functionality in definition file
\end{itemize}


%%%%%%%%%%%%%%%%%%%%%%%%%%%%%%%%%%%%%%%%%%%%%%%%%%%%%%%%%%%%%%%%%%%%%%%%%%%%%%%%
%%%%%%%%%%%%%%%%%%%%%%%%%%%%%%%%%%%%%%%%%%%%%%%%%%%%%%%%%%%%%%%%%%%%%%%%%%%%%%%%
%%%%%%%%%%%%%%%%%%%%%%%%%%%%%%%%%%%%%%%%%%%%%%%%%%%%%%%%%%%%%%%%%%%%%%%%%%%%%%%%
\appendix

\settowidth\MacroIndent{\rmfamily\scriptsize 000\ }

 \DocInput{childdoc.dtx}

\end{document}
%</driver>
% \fi
%
% %%%%%%%%%%%%%%%%%%%%%%%%%%%%%%%%%%%%%%%%%%%%%%%%%%%%%%%%%%%%%%%%%%%%%%%%%%%%%%
% %%%%%%%%%%%%%%%%%%%%%%%%%%%%%%%%%%%%%%%%%%%%%%%%%%%%%%%%%%%%%%%%%%%%%%%%%%%%%%
% \section{Sample}
%\iffalse
%<*samplemain>
%\fi
%
% The following presents a sample document
% with two chapters, two parts, a title page,
% a compile flag as well as three forwarding files to set the flag.
% It consists of eight |.tex| files:
% \begin{center}
% \begin{tabular}{ll}
% |cdocsamp.tex|&main file\\
% |cdocsch1.tex|&include file for chapter 1\\
% |cdocsch2.tex|&include file for chapter 2\\
% |cdocspt3.tex|&include file for part 3\\
% |cdocspt4.tex|&include file for part 4\\
% |cdocsdrf.tex|&forwarding file for main file in draft mode\\
% |cdocsfi1.tex|&forwarding file for final version of chapter 1\\
% |cdocsfi2.tex|&forwarding file for final version of chapter 2\\
% \end{tabular}
% \end{center}
% Each of the eight files can be compiled directly by the \LaTeX{} compiler.
%
% %%%%%%%%%%%%%%%%%%%%%%%%%%%%%%%%%%%%%%
% \paragraph{Main File.}
%
% The main file is called |cdocsamp.tex|.
%
% Load the \textsf{childdoc} definitions and
% declare the filename for the main document:
%    \begin{macrocode}
\input{childdoc.def}
\childdocmain{}
%    \end{macrocode}

% Optional override for |\version| flag:
%    \begin{macrocode}
%%\ifchilddoc\else\providecommand{\version}{draft}\fi
%    \end{macrocode}

% Define the default values for the |\version| flag
% (|final| for the main file and |draft| for childs):
%    \begin{macrocode}
\ifchilddoc
\providecommand{\version}{draft}
\else
\providecommand{\version}{final}
\fi
%    \end{macrocode}

% Load the standard document class:
%    \begin{macrocode}
\documentclass[12pt]{article}
%    \end{macrocode}

% Start the document body:
%    \begin{macrocode}
\begin{document}
%    \end{macrocode}

% Declare a title page.
% Print title, part of document being processed and version flag:
%    \begin{macrocode}
\addtocounter{page}{-1}
\begin{center}
{\LARGE\bfseries{}childdoc example\par}
\vspace{1cm}
\ifchilddoc
\ifchilddocmanual part\else chapter\fi:
`\childdocname' of `\childdocjob'\par
\else
main document: `\childdocjob'\par
\fi
version: \version\par
\end{center}
\newpage
%    \end{macrocode}

% Manually include selected file,
% otherwise process as usual:
%    \begin{macrocode}
\ifchilddocmanual
\section*{part `\childdocname'}
\input{\childdocname}
\else
%    \end{macrocode}

% Include the two chapters:
%    \begin{macrocode}
\include{cdocsch1}
\include{cdocsch2}
%    \end{macrocode}

% Include the two parts unless only chapters should be displayed:
%    \begin{macrocode}
\ifchilddoc\else
\section{part three}
\input{cdocspt3}
\section{part four}
\input{cdocspt4}
\fi
%    \end{macrocode}

% Process as usual until here:
%    \begin{macrocode}
\fi
%    \end{macrocode}

% End of document body:
%    \begin{macrocode}
\end{document}
%    \end{macrocode}
%\iffalse
%</samplemain>
%\fi
%
% %%%%%%%%%%%%%%%%%%%%%%%%%%%%%%%%%%%%%%
% \paragraph{Chapter Include Files.}
%
% The include files are called |cdocsch1.tex| and |cdocsch2.tex|.
%
%\iffalse
%<*samplechap1|samplechap2>
%\fi

% Optional override for |\version| flag:
%    \begin{macrocode}
%%\providecommand{\version}{final}
%    \end{macrocode}

% Include the main document:
%    \begin{macrocode}
\input{childdoc.def}
\childdocof{cdocsamp}
%    \end{macrocode}

%\iffalse
%</samplechap1|samplechap2>
%\fi
%
%\iffalse
%<*samplechap1>
%\fi
% Some text for chapter 1:
%    \begin{macrocode}
\section{one}
some text in chapter one
%    \end{macrocode}

%\iffalse
%</samplechap1>
%\fi
% Some text for chapter 2:
%\iffalse
%<*samplechap2>
%\fi
%    \begin{macrocode}
\section{two}
more text in chapter two
%    \end{macrocode}

%\iffalse
%</samplechap2>
%\fi
%
% %%%%%%%%%%%%%%%%%%%%%%%%%%%%%%%%%%%%%%
% \paragraph{Part Include Files.}
%
% The include files are called |cdocspt3.tex| and |cdocspt4.tex|.
%
%\iffalse
%<*samplepart3|samplepart4>
%\fi

% Optional override for |\version| flag:
%    \begin{macrocode}
%%\providecommand{\version}{final}
%    \end{macrocode}

% Include the main document:
%    \begin{macrocode}
\input{childdoc.def}
\childdocby{cdocsamp}
%    \end{macrocode}

%\iffalse
%</samplepart3|samplepart4>
%\fi
%
%\iffalse
%<*samplepart3>
%\fi
% Some text for part 3:
%    \begin{macrocode}
some text in part three
%    \end{macrocode}

%\iffalse
%</samplepart3>
%\fi
% Some text for part 4:
%\iffalse
%<*samplepart4>
%\fi
%    \begin{macrocode}
more text in part four
%    \end{macrocode}

%\iffalse
%</samplepart4>
%\fi
%
% %%%%%%%%%%%%%%%%%%%%%%%%%%%%%%%%%%%%%%
% \paragraph{Forwarding for a Complete Draft.}
%
% The following forwarding file |cdocsdrf.tex|
% compiles the main document in draft mode:
%\iffalse
%<*sampledraft>
%\fi
%    \begin{macrocode}
\def\version{draft}
\input{childdoc.def}
\childdocforward{cdocsamp}
%    \end{macrocode}

%\iffalse
%</sampledraft>
%\fi
%
% %%%%%%%%%%%%%%%%%%%%%%%%%%%%%%%%%%%%%%
% \paragraph{Forwarding for Final Version of the Chapters.}
%
% The following forwarding files |cdocsfn1.tex| and |cdocsfn2.tex|
% (with identical content)
% compile the final versions of the child documents
% |cdocsch1.tex| and |cdocsch2.tex|, respectively:
%\iffalse
%<*samplefinal>
%\fi
%    \begin{macrocode}
\def\version{final}
\input{childdoc.def}
\childdocforwardprefix[cdocsamp]{cdocsfn}{cdocsch}
%    \end{macrocode}

%\iffalse
%</samplefinal>
%\fi
%
% %%%%%%%%%%%%%%%%%%%%%%%%%%%%%%%%%%%%%%
% \paragraph{Command Line Processing.}
%
% The following three command lines generate the output files
% |cdocscld|, |cdocscl1| and |cdocscl2|
% which should be identical to
% |cdocsdrf|, |cdocsch1| and |cdocsfn2|, respectively:
% \begin{center}
% \begin{tabular}{l}
% |latex -jobname cdocscld \|\\
% |  "\def\version{draft}\input{childdoc.def}\childdocforward{cdocsamp}"|\\
% |latex -jobname cdocscl1 \|\\
% |  "\input{childdoc.def}\childdocforward[cdocsamp]{cdocsch1}"|\\
% |latex -jobname cdocscl2 \|\\
% |  "\def\version{final}\input{childdoc.def}\childdocforward{cdocsch2}"|
% \end{tabular}
% \end{center}
% Note that the trailing backslash on each first line
% merely continues the input to the second line
% (for convenient cut ant paste).
% Furthermore, the command |latex| can be replaced by any
% of its alternative versions such as |pdflatex|.
%
% %%%%%%%%%%%%%%%%%%%%%%%%%%%%%%%%%%%%%%%%%%%%%%%%%%%%%%%%%%%%%%%%%%%%%%%%%%%%%%
% %%%%%%%%%%%%%%%%%%%%%%%%%%%%%%%%%%%%%%%%%%%%%%%%%%%%%%%%%%%%%%%%%%%%%%%%%%%%%%
% \section{Implementation}
%\iffalse
%<*package>
%\fi
%
% This section describes the definitions file |childdoc.def|.

% The definitions cannot be loaded using |\usepackage| or |\RequirePackage|
% which has a mechanism to prevent loading a style file more than once.
% When loading the definitions by means of |\input|
% multiple instances have to be prevented manually:
%\iffalse
%This code needs to be before the `\ProvidesFile' directive
%which is defined at the beginning of this file.
%Therefore it is also placed there and commented out here.
%</package>
%<*discard>
%\fi
%    \begin{macrocode}
\ifdefined\childdocmain\endinput\fi
%    \end{macrocode}
%\iffalse
%</discard>
%<*package>
%\fi
%
% \macro{\ifchilddoc}
% \macro{\ifchilddocmanual}
% The conditional |\ifchilddoc| tells whether a
% child (true) or main (false) document is being compiled.
% The conditional |\ifchilddocmanual| tells whether
% the |\includeonly| mechanism is used (false) or
% the selection of child files must be performed manually (true).
% The definitions initialise to false:
%    \begin{macrocode}
\newif\ifchilddoc
\newif\ifchilddocmanual
%    \end{macrocode}

% \macro{\childdocname}
% \macro{\childdocjob}
% The macro |\childdocname| stores the name of the main document
% to be compiled. The macro |\childdocjob| stores the name of
% the document on which the \LaTeX{} compiler was originally invoked.
% The content of |\jobname| cannot be compared
% to filenames specified in the source due to different catcodes.
% The following code rescans |\jobname|, stores the result
% in |\childdocname| and saves a copy in |\childdocjob|:
%    \begin{macrocode}
\edef\childdocname{\scantokens\expandafter{\jobname\noexpand}}
\let\childdocjob\childdocname
%    \end{macrocode}

% \macro{\childdocdisable}
% The macro |\childdocdisable| prevents the main file
% from being processed more than once.
% At this stage, the main document command |\childdocmain|
% is assumed to be called once again where it should do nothing.
% Any subsequent call to it should prevent
% a secondary processing of the main document
% It overwrites the forwarding commands
% |\childdocof| and |\childdocforward|
% with empty macros to prevent further inclusions of the main document:
%    \begin{macrocode}
\newcommand{\childdocdisable}
{
  \renewcommand{\childdocmain}[1]{\renewcommand{\childdocmain}[1]{\endinput}}
  \renewcommand{\childdocof}[1]{}
  \renewcommand{\childdocby}[2][]{}
  \renewcommand{\childdocforward}[2][]{}
  \renewcommand{\childdocdisable}{}
}
%    \end{macrocode}

% \macro{\childdocmain}
% The macro |\childdocmain| is to be called at the top of the main file
% with nothing or the main filename (without extension) as argument.
% First, it breaks loops.
% If the argument is not empty and does not match |\childdocname|
% (which is set by the first inclusion of |childdoc.def|),
% |\ifchilddoc| is set to true, |\includeonly| is applied to the child file
% and |\jobname| is set to the main file
% (for proper handling of |.aux| files):
%    \begin{macrocode}
\newcommand{\childdocmain}[1]
{
  \childdocdisable\childdocmain{}
  \if?#1?\else
    \begingroup
      \def\childdoctmp{#1}
      \ifx\childdoctmp\childdocname
        \def\childdoctmp{}
      \else
        \def\childdoctmp
        {
          \childdoctrue
          \includeonly{\childdocname}
          \def\childdocjob{#1}
          \def\jobname{#1}
        }
      \fi
      \expandafter
    \endgroup
    \childdoctmp
  \fi
}
%    \end{macrocode}

% \macro{\childdocof}
% The command |\childdocof| redirects
% compilation to the main file |#1|.
%    \begin{macrocode}
\newcommand{\childdocof}[1]
{
  \childdocdisable
  \childdoctrue
  \includeonly{\childdocname}
  \def\jobname{#1}
  \def\childdocjob{#1}
  \input{#1}
}
%    \end{macrocode}

% \macro{\childdocby}
% The command |\childdocby| ....
%    \begin{macrocode}
\newcommand{\childdocby}[2][]
{
  \childdocdisable
  \childdoctrue
  \childdocmanualtrue
  \if?#1?\else
    \def\jobname{#2}
  \fi
  \def\childdocjob{#2}
  \input{#2}
  \endinput
}
%    \end{macrocode}

% \macro{\childdocforward}
% The command |\childdocforward| redirects
% compilation to the main file or
% (if the optional argument is given) a child file.
% Parameters are set as if the main file
% or a child file starting with |\childdocof| was compiled.
% Then compilation is handed over to the main file:
%    \begin{macrocode}
\newcommand{\childdocforward}[2][]
{
  \begingroup
    \if?#1?
      \def\childdoctmp
      {
        \def\childdocname{#2}
        \def\childdocjob{#2}
        \def\jobname{#2}
        \input{#2}
        \endinput
      }
    \else
      \def\childdoctmp
      {
        \childdocdisable
        \def\childdocname{#2}
        \childdoctrue
        \includeonly{#2}
        \def\childdocjob{#1}
        \def\jobname{#1}
        \input{#1}
        \endinput
      }
    \fi
    \expandafter
  \endgroup
  \childdoctmp
}
%    \end{macrocode}

% \macro{\childdocforwardprefix}
% The command |\childdocforwardprefix| redirects
% compilation to the main or a child file by means of a pattern.
% The prefix |#1| in the current filename is replaced by |#2|
% and the suffix of the current filename is kept
% (it is assumed that the filename does not contain the substring `|~~~|'
% which is used as a delimiter).
% Compilation is handed over to the new file by |\childdocforward|:
%    \begin{macrocode}
\newcommand{\childdocforwardprefix}[3][]
{
  \begingroup
    \def\childdocextract #2##1~~~{\def\childdoctmp{\childdocforward[#1]{#3##1}}}
    \expandafter\childdocextract\childdocname~~~
    \expandafter
  \endgroup
  \childdoctmp
}
%    \end{macrocode}

% \macro{\childdoc}
% The deprecated macro |\childdoc| is a legacy version of |\childdocmain|:
%    \begin{macrocode}
\newcommand{\childdoc}{\childdocmain}
%    \end{macrocode}

% \macro{\childdocredirect}
% The deprecated macro |\childdocredirect| is a legacy version
% of |\childdocforward| and |\childdocforwardprefix|:
%    \begin{macrocode}
\newcommand{\childdocredirect}[2][]
{
  \begingroup
    \if?#1?
      \def\childdoctmp{\childdocforward{#2}}
    \else
      \def\childdoctmp{\childdocforwardprefix{#1}{#2}}
    \fi
    \expandafter
  \endgroup
  \childdoctmp
}
%    \end{macrocode}

%\iffalse
%</package>
%\fi
%
\endinput
|\\
|\childdocforward{|\textit{main}|}|\\
\end{tabular}
\end{center}
%
or alternatively with:
%
\begin{center}
\begin{tabular}{l}
|% \iffalse
%
% childdoc.dtx Copyright (C) 2017-2018 Niklas Beisert
%
% This work may be distributed and/or modified under the
% conditions of the LaTeX Project Public License, either version 1.3
% of this license or (at your option) any later version.
% The latest version of this license is in
%   http://www.latex-project.org/lppl.txt
% and version 1.3 or later is part of all distributions of LaTeX
% version 2005/12/01 or later.
%
% This work has the LPPL maintenance status `maintained'.
%
% The Current Maintainer of this work is Niklas Beisert.
%
% This work consists of the files childdoc.dtx and childdoc.ins
% and the derived files childdoc.def and cdocsamp.tex with
% cdocsch1.tex, cdocsch2.tex, cdocsdrf.tex, cdocsfn1.tex, cdocsfn2.tex.
%
%<package>\ifdefined\childdocmain\endinput\fi
%<package>\ProvidesFile{childdoc.def}[2018/12/30 v2.0 child document driver]
%<samplemain>\ProvidesFile{cdocsamp.tex}[2018/12/30 v2.0 sample for childdoc]
%<*driver>
%\ProvidesFile{childdoc.drv}[2018/12/30 v2.0 childdoc reference manual file]
\PassOptionsToClass{10pt,a4paper}{article}
\documentclass{ltxdoc}

\usepackage[margin=35mm]{geometry}
\usepackage{hyperref}
\usepackage{hyperxmp}
\usepackage[usenames]{color}

\hypersetup{colorlinks=true}
\hypersetup{pdfstartview=FitH}
\hypersetup{pdfpagemode=UseNone}
\hypersetup{pdfsource={}}
\hypersetup{pdflang={en-UK}}
\hypersetup{pdfcopyright={Copyright 2017-2018 Niklas Beisert.
  This work may be distributed and/or modified under the
  conditions of the LaTeX Project Public License, either version 1.3
  of this license or (at your option) any later version.}}
\hypersetup{pdflicenseurl={http://www.latex-project.org/lppl.txt}}
\hypersetup{pdfcontactaddress={ETH Zurich, ITP, HIT K,
  Wolfgang-Pauli-Strasse 27}}
\hypersetup{pdfcontactpostcode={8093}}
\hypersetup{pdfcontactcity={Zurich}}
\hypersetup{pdfcontactcountry={Switzerland}}
\hypersetup{pdfcontactemail={nbeisert@itp.phys.ethz.ch}}
\hypersetup{pdfcontacturl={http://people.phys.ethz.ch/\xmptilde nbeisert/}}

\newcommand{\secref}[1]{\hyperref[#1]{section \ref*{#1}}}

\parskip1ex
\parindent0pt
\let\olditemize\itemize
\def\itemize{\olditemize\parskip0pt}

\begin{document}

\title{The \textsf{childdoc} Package}
\hypersetup{pdftitle={The childdoc Package}}
\author{Niklas Beisert\\[2ex]
  Institut f\"ur Theoretische Physik\\
  Eidgen\"ossische Technische Hochschule Z\"urich\\
  Wolfgang-Pauli-Strasse 27, 8093 Z\"urich, Switzerland\\[1ex]
  \href{mailto:nbeisert@itp.phys.ethz.ch}
  {\texttt{nbeisert@itp.phys.ethz.ch}}}
\hypersetup{pdfauthor={Niklas Beisert}}
\hypersetup{pdfsubject={Manual for the LaTeX2e Package childdoc}}
\date{30 December 2018, \textsf{v2.0}}
\maketitle

\begin{abstract}\noindent
\textsf{childdoc} is a \LaTeXe{} package
that enables the direct compilation
of document sections included by |\include|
to individual files.
\end{abstract}

\begingroup
\parskip0ex
\tableofcontents
\endgroup

%%%%%%%%%%%%%%%%%%%%%%%%%%%%%%%%%%%%%%%%%%%%%%%%%%%%%%%%%%%%%%%%%%%%%%%%%%%%%%%%
%%%%%%%%%%%%%%%%%%%%%%%%%%%%%%%%%%%%%%%%%%%%%%%%%%%%%%%%%%%%%%%%%%%%%%%%%%%%%%%%
\section{Introduction}

\LaTeX{} provides a mechanism to structure a large document (such as a book)
into a main file and several child files (containing the chapters)
using the |\include| command.
This mechanism is beneficial for documents
which span hundreds of pages in order to
make the source file(s) more manageable.
Moreover, compilation can be restricted to
selected child files by means of the |\includeonly| command.
The latter feature can be used to reduce the compilation time while editing
(this was significantly more useful in the earlier days of \LaTeX{})
or to generate a smaller document which is easier to navigate.
Another application of |\includeonly| is to generate
documents consisting of selected parts of the complete document.

However, there are a few drawbacks of the plain |\include| mechanism:
\begin{itemize}
\item
The child files cannot be compiled on their own,
they can only be compiled via the main file.
A naive editing environment
(such as a text editor with an option
to have the current file processed by \LaTeX)
may require one to switch to the main file before compiling;
attempting to compile the child file produces errors.
\item
The main file must be modified (each time)
to adjust the |\includeonly| command
to the present needs. This easily leaves the main file in a messy state.
\item
The generated document will always carry the filename
of the main document. This is inconvenient if
several child files are to be compiled and
to be kept for distribution.
\end{itemize}

The present package provides a simple interface
to make child files individually compilable by \LaTeX{}.
Compiling a child file then has the same effect as compiling
the main file with an |\includeonly| command
to select the appropriate child.
Moreover the generated document will carry the name of the child
rather than the main file.
This resolves all three above issues.

This feature is meant to make the editing of books,
thesis documents and lecture notes somewhat more convenient.
However, the package can also be used efficiently for
composing a series of documents (such as exercise sheets)
which are typically distributed individually.
It then assists the author in generating the individual documents
(potentially in different versions)
as well as a document containing the collected series.
Another application is in developing style files
or other kinds of included material
where compilation of the style file could redirect
to a sample or test file.

%%%%%%%%%%%%%%%%%%%%%%%%%%%%%%%%%%%%%%%%%%%%%%%%%%%%%%%%%%%%%%%%%%%%%%%%%%%%%%%%
%%%%%%%%%%%%%%%%%%%%%%%%%%%%%%%%%%%%%%%%%%%%%%%%%%%%%%%%%%%%%%%%%%%%%%%%%%%%%%%%
\section{Usage}

First of all, the package \textsf{childdoc} is \emph{not} a standard
\LaTeXe{} |.sty| style file! Therefore it needs to be invoked in
a non-standard way.

%%%%%%%%%%%%%%%%%%%%%%%%%%%%%%%%%%%%%%%%%%%%%%%%%%%%%%%%%%%%%%%%%%%%%%%%%%%%%%%%
\subsection{Included Files}
\label{sec:include}

%%%%%%%%%%%%%%%%%%%%%%%%%%%%%%%%%%%%%%%%
\DescribeMacro{\childdocmain}
To use the package, add the commands
\begin{center}
\begin{tabular}{l}
|\input{childdoc.def}|\\
|\childdocmain{}|\\
\end{tabular}
\end{center}
at the very top of the main \LaTeX{} file,
in particular \emph{before} the |\documentclass| statement!
The argument of |\childdocmain| should be left empty
(but it must be present).

%%%%%%%%%%%%%%%%%%%%%%%%%%%%%%%%%%%%%%%%
\DescribeMacro{\childdocof}
Furthermore, add the commands
\begin{center}
\begin{tabular}{l}
|\input{childdoc.def}|\\
|\childdocof{|\textit{main}|}|\\
\end{tabular}
\end{center}
at the top of every child file \textit{child}
which is included by |\include{|\textit{child}|}|
from within the main file
(or at least for those files to be compiled individually).
The argument \textit{main} must be the filename of the main file.

There are a couple of
considerations in setting up the main and child documents:

%%%%%%%%%%%%%%%%%%%%%%%%%%%%%%%%%%%%%%%%
\paragraph{Restrictions.}

Please note the following restrictions:
\begin{itemize}
\item
|\childdocmain| must be called with one argument \textit{main}
to ensure compatibility with earlier version of the package.
It must either be empty (|\childdocmain{}|)
or precisely match the filename of the main file in which it is specified.
See \secref{sec:detection} for further information.
\item
The filename \textit{main} must be specified without the |.tex| extension.
\item
The filename \textit{main} is case sensitive
(even in case-insensitive file systems)
due to internal string comparison.
\item
The argument \textit{main} should be fully expanded, it cannot be a macro.
\item
Subdirectories and special characters should be avoided in filenames.
\item
The command |\childdocmain{|\textit{main}|}| must be followed by a whitespace.
It should not be followed immediately by another command
or by a comment mark `|%|'.
This is because the \TeX{} parser reads the token immediately following
the argument of |\childdocmain| and puts it
at the beginning of every child section;
however, a white\-space is ignored.
\end{itemize}

%%%%%%%%%%%%%%%%%%%%%%%%%%%%%%%%%%%%%%%%
\paragraph{Content of Main File.}

It is advisable to place all content in the child files included by |\include|.
Any output contained in the main file will appear in all child documents
unless suppressed manually;
it cannot be suppressed automatically by the |\includeonly| directive
and thus should normally be avoided.
A method to include some content in the main file
by means of conditional processing is described in \secref{sec:conditional}.

%%%%%%%%%%%%%%%%%%%%%%%%%%%%%%%%%%%%%%%%
\paragraph{Page Numbering.}

When only a part of the document is compiled,
the appropriate numbering of pages
(as well as other status parameters)
is determined from the |.aux| files.
The latter contain information from previous passes.
However this information needs to propagate through
all intermediate child documents.
Therefore the page numbering in child documents may well
be inconsistent until the complete document is compiled at least once.

A useful (if unconventional) way to always ensure a consistent
page numbering is to restart the numbering in each child document
and denote the pages by `\textit{child}|.|\textit{page}'
where \textit{child} represents the chapter/section number of the child file.
This can be achieved by the command
|\numberwithin{page}{|\textit{child}|}|
of the \textsf{amsmath} package
where \textit{child} can be |chapter| or |section|
depending on the chosen structuring.
Alternatively, one can modify the macro |\thepage| appropriately
and reset the counter |page| at the start of each child file.

%%%%%%%%%%%%%%%%%%%%%%%%%%%%%%%%%%%%%%%%%%%%%%%%%%%%%%%%%%%%%%%%%%%%%%%%%%%%%%%%
\subsection{Conditional Processing}
\label{sec:conditional}

The package provides a mechanism to compile different versions
of a document. To customise the versions further some conditional processing
can come in handy to distinguish which version is being compiled.
The package provides two macros to describe the compilation context:

%%%%%%%%%%%%%%%%%%%%%%%%%%%%%%%%%%%%%%%%
\DescribeMacro{\ifchilddoc}
The conditional |\ifchilddoc| distinguishes between the compilation of
child documents and the main document:
%
\begin{center}
|\ifchilddoc |\textit{child-code}| |[|\||else |\textit{main-code}]| \||fi|
\end{center}

%%%%%%%%%%%%%%%%%%%%%%%%%%%%%%%%%%%%%%%%
\DescribeMacro{\childdocname}
\DescribeMacro{\childdocjob}
The macro |\childdocname| contains the filename (without extension)
of the main or child file being processed.
Note that |\childdocjob| will always contain the name of the main file.

%%%%%%%%%%%%%%%%%%%%%%%%%%%%%%%%%%%%%%%%
\paragraph{Title Page.}

Conditional processing can be used to include a title or banner page
in the main document when proper precautions are taken.
Importantly, the code in the main file should ensure that the page counter
(as well as other status parameters which are stored in the |.aux| files)
takes the same value after the conditional processing.
Otherwise the page numbers may take divergent values
depending on which part is compiled.

For example, a title page could be declared by:
%
\begin{center}
\begin{tabular}{l}
|\ifchilddoc\||else|\\
|\addtocounter{page}{-1}|\\
\textit{code for title page}\\
|\newpage|\\
|\||fi|
\end{tabular}
\end{center}
%
A banner page for the child documents can be generated by:
%
\begin{center}
\begin{tabular}{l}
|\ifchilddoc|\\
|\addtocounter{page}{-1}|\\
\textit{code for banner page}\\
|\newpage|\\
|\||fi|
\end{tabular}
\end{center}
%
Here one could write a message such as:
\begin{center}
|This is the part \childdocname{} of \childdocjob{}.|
\end{center}

%%%%%%%%%%%%%%%%%%%%%%%%%%%%%%%%%%%%%%%%%%%%%%%%%%%%%%%%%%%%%%%%%%%%%%%%%%%%%%%%
\subsection{Flags}
\label{sec:flags}

The package makes it easy to generate different versions
of the main or child documents.
To this end compilation flags can be defined
and assigned different default values.
They will be particularly useful in conjunction
with the forwarding mechanism described in \secref{sec:forward}.

For example, it may be useful to have a flag |\version|
which can be set to |draft| or |final|.
The document source will contain some conditional code
depending on the value of |\version|.
Suppose further, the flag should default to |final| for the main file
and to |draft| for child files
which is a natural assignment for editing the document.
This is achieved by placing the following code
in the preamble of the main document
(below the |\childdocmain| directive):
%
\begin{center}
\begin{tabular}{l}
|\ifchilddoc|\\
|\providecommand{\version}{draft}|\\
|\||else|\\
|\providecommand{\version}{final}|\\
|\||fi|
\end{tabular}
\end{center}
%
The definition by |\providecommand| makes sure
that previous definitions are not overwritten.
Further statements |\providecommand{\version}{...}|
can thus be added before the above code to override it.

For the main file, one might add a line
(between |\childdocmain| and the above block)
%
\begin{center}
|%\ifchilddoc\||else\providecommand{\version}{draft}\||fi|
\end{center}
%
which can be uncommented to produce a draft version.
Likewise one can add a line to the very top of a child file
(above the |\childdocof{|\textit{main}|}| directive)
%
\begin{center}
|%\providecommand{\version}{final}|
\end{center}
%
which can be uncommented to produce the final version of this child document.

%%%%%%%%%%%%%%%%%%%%%%%%%%%%%%%%%%%%%%%%%%%%%%%%%%%%%%%%%%%%%%%%%%%%%%%%%%%%%%%%
\subsection{Forwarding}
\label{sec:forward}

Different versions of the main or child documents
using compilation flags as described in \secref{sec:flags}
can be (permanently) stored in different files
for convenient compilation, viewing and distribution.
To this end, the package defines a command
to pass on compilation to a different file:

%%%%%%%%%%%%%%%%%%%%%%%%%%%%%%%%%%%%%%%%
\DescribeMacro{\childdocforward}
The command |\childdocforward| redirects processing to
another source file:
%
\begin{center}
\begin{tabular}{l}
|\input{childdoc.def}|\\
|\childdocforward[|\textit{main}|]{|\textit{dest}|}|\\
\end{tabular}
\end{center}
%
The argument \textit{dest} is the destination file
(without extension).
It should be the main file or one of the child files.
Note that further \textsf{childdoc} directives
such as |\childdocof| and |\childdocforward|
in the indicated file will be processed in this form.
The optional argument \textit{main}
passes on directly to the main file \textit{main}
while pretending to compile the child \textit{dest}.
This form behaves as if \textit{dest}
issues |\childdocof{|\textit{main}|}| right away,
and no further \textsf{childdoc} directives will be processed.

%%%%%%%%%%%%%%%%%%%%%%%%%%%%%%%%%%%%%%%%
\DescribeMacro{\...prefix}
In the alternative form |\childdocforwardprefix|,
%
\begin{center}
\begin{tabular}{l}
|\input{childdoc.def}|\\
|\childdocforwardprefix[|\textit{main}|]{|\textit{prefix}|}{|\textit{dest}|}|
\end{tabular}
\end{center}
%
the destination file is determined by a pattern
depending on the current file:
To make this work, the current file must be called
`{\textit{prefix}\hspace{0.2em}\textit{suffix}}'
with \textit{prefix} matching precisely the argument.
Processing is then passed on to the file
`{\textit{dest}\hspace{0.2em}\textit{suffix}}'.
Surely, the same effect is achieved by
directly specifying the
argument `{\textit{dest}\hspace{0.2em}\textit{suffix}}'
in the first form.
However, that requires to set up a different file
for each child. With the alternative form of the command
all these files can have exactly the same content
which simplifies setting them up and maintaining them.

For example, the following file |draft.tex|
with a compilation flag |\version| as described in \secref{sec:flags}
compiles the main document as a draft:
%
\begin{center}
\begin{tabular}{l}
|\def\version{draft}|\\
|\input{childdoc.def}|\\
|\childdocforward{|\textit{main}|}|
\end{tabular}
\end{center}
%
Likewise, the following files |final|\textit{nn}|.tex|
compile the final version of the child document
|child|\textit{nn}|.tex|:
%
\begin{center}
\begin{tabular}{l}
|\def\version{final}|\\
|\input{childdoc.def}|\\
|\childdocforwardprefix{final}{child}|
\end{tabular}
\end{center}
%

Note that when several versions of a main file and/or of each child file
are to be generated, it may be convenient to set up a |Makefile| or
shell script to automatise the process.

%%%%%%%%%%%%%%%%%%%%%%%%%%%%%%%%%%%%%%%%%%%%%%%%%%%%%%%%%%%%%%%%%%%%%%%%%%%%%%%%
\subsection{Command Line Processing}
\label{sec:commandline}

The effect of redirection files can also be achieved by invoking
the \LaTeX{} compiler with a more elaborate command line.
Most conveniently this should be done as part
of a shell script or a |Makefile|.

When using \textsf{childdoc} in the main file, the following
command lines effectively perform a redirection
(note that depending on the shell being used,
backslashes may have to be doubled: `|\|' $\to$ `|\\|'):
%
\begin{center}
|... -jobname "|\textit{target}|" |\\|"|[\textit{flags}]%
|\input{childdoc.def}\childdocforward[|\textit{main}|]{|\textit{dest}|}"|
\end{center}
%
Here \textit{target} is the name of the output file,
\textit{main} is the name of the main file
and \textit{dest} is the name of the main or child file to be processed
(all filenames without extensions).
The optional argument \textit{main} can be omitted
if \textit{main} matches \textit{dest}.
Optionally, compilation \textit{flags} can be defined via |\def| commands.
This command line makes the \TeX{} engine believe
it is compiling the file \textit{target}
whose content is specified as the latter parameter.
The provided code then forwards the processing to
\textit{main} or \textit{dest} as described in \secref{sec:forward}.

%%%%%%%%%%%%%%%%%%%%%%%%%%%%%%%%%%%%%%%%%%%%%%%%%%%%%%%%%%%%%%%%%%%%%%%%%%%%%%%%
\subsection{Include by Input}
\label{sec:input}

Including child documents by |\include| has some restrictions by design.
Most notably, the content of a child document always occupies
its own set of pages; pages cannot be shared between child documents.
Usually, this behaviour makes perfect sense
because each child document contain an essential part of the document.
However, in some situations it may be desirable to compose
a document from a collection of parts
without having mandatory page breaks between then.
For this case, the package
provides a mechanism to include parts
by |\input| which can also be processed individually.
However, by construction this mechanism
requires manual handling of the content to be output.

%%%%%%%%%%%%%%%%%%%%%%%%%%%%%%%%%%%%%%%%
\DescribeMacro{\ifchilddocmanual}
The main file should be prepared as usual, see \secref{sec:include}.
However, the document body must make a distinction
between processing of an individual part and of the main document, e.g.:
%
\begin{center}
\begin{tabular}{l}
|\ifchilddocmanual|\\
|\input{\childdocname}|\\
|\||else|\\
\textit{document body with }|\input{|\textit{part}|}|\\
|\||fi|
\end{tabular}
\end{center}
%
The conditional |\ifchilddocmanual| is true whenever
a part to be included by |\input| is being compiled,
and the name of the part is stored in |\childdocname|.

%%%%%%%%%%%%%%%%%%%%%%%%%%%%%%%%%%%%%%%%
\DescribeMacro{\childdocby}
Each part to be included by |\input| should start with:
%
\begin{center}
\begin{tabular}{l}
|\input{childdoc.def}|\\
|\childdocby{|\textit{main}|}|\\
\end{tabular}
\end{center}
%
The directive |\childdocby| is similar to |\childdocof|
described in \secref{sec:include},
but the subsequent selection of content must be done manually.
To that end, both |\ifchilddoc| and |\ifchilddocmanual|
will be true upon processing of a part,
and the name of the part is stored in |\childdocname|.
Note that |\jobname| will be set to the filename of the current part
so that each part receives an individual |.aux| file
that does not interfere with the |.aux| file(s) of the main document.
This behaviour can be altered by the alternative form
|\childdocby[*]{|\textit{main}|}| (with a non-empty optional argument)
which uses the |.aux| file of the main document
by setting |\jobname| to \textit{main}.

%%%%%%%%%%%%%%%%%%%%%%%%%%%%%%%%%%%%%%%%%%%%%%%%%%%%%%%%%%%%%%%%%%%%%%%%%%%%%%%%
\subsection{Driver Development}
\label{sec:driver}

The \textsf{childdoc} mechanism can also be use for the development
of definition files such as \LaTeX{} styles or classes.
This case differs from the above setup with multiple parts
included by |\include| in that no |\includeonly| should be invoked.
This can be achieved by starting the include file
(before |\ProvidesPackage|) with:
%
\begin{center}
\begin{tabular}{l}
|\input{childdoc.def}|\\
|\childdocforward{|\textit{main}|}|\\
\end{tabular}
\end{center}
%
or alternatively with:
%
\begin{center}
\begin{tabular}{l}
|\input{childdoc.def}|\\
|\childdocby{|\textit{main}|}|\\
\end{tabular}
\end{center}
%
Both forms have slightly different effects as described above.
The main file is prepared as usual, see \secref{sec:include}.

%%%%%%%%%%%%%%%%%%%%%%%%%%%%%%%%%%%%%%%%%%%%%%%%%%%%%%%%%%%%%%%%%%%%%%%%%%%%%%%%
\subsection{Legacy Detection}
\label{sec:detection}

The directive |\childdocmain| in the main file can detect
whether the complete document or merely a child is to be compiled
even without using the directive |\childdocof|.
This method is deprecated because it is less robust
and there is no compelling reason to use it;
it is merely provided for backward compatibility
and it may be removed in future versions.

If the detection mechanism is to be used,
it is mandatory to correctly specify
the filename of the main file as the argument of |\childdocmain|:
%
\begin{center}
\begin{tabular}{l}
|\input{childdoc.def}|\\
|\childdocmain{|\textit{main}|}|\\
\end{tabular}
\end{center}
%
If |\jobname| does not match the argument \textit{main} of |\childdocmain|,
it is assumed that |\jobname| points to the child file to be compiled.
When using |\childdocmain| with the main file specified as argument,
it suffices to start a child file
with just |\input{|\textit{main}|}|
without loading of the package and using |\childdocof|.
If instead all processing is done
with the appropriate \textsf{childdoc} directives,
the argument of \textit{main} of |\childdocmain| can be empty.

An alternative version of the command line processing described
in \secref{sec:commandline} using the detection mechanism reads:
%
\begin{center}
|... -jobname "|\textit{target}|" "|[\textit{flags}]%
[|\def\jobname{|\textit{dest}|}|]|\input{|\textit{main}|}"|
\end{center}

%%%%%%%%%%%%%%%%%%%%%%%%%%%%%%%%%%%%%%%%%%%%%%%%%%%%%%%%%%%%%%%%%%%%%%%%%%%%%%%%
\subsection{Manual Code}
\label{sec:manual}

In case one cannot be certain whether the definitions file |childdoc.def|
is installed on the target \TeX{} distribution
and one prefers not to ship it,
it is conceivable to paste a few relevant commands into the sources.

To that end, drop all statements |\input{childdoc.def}|
and perform the replacements as outlined below.
Instead of |\childdocmain{|\textit{main}|}| add the following code
to the top of the main file:
%
\begin{center}
\begin{tabular}{l}
|\||ifdefined\childdocname\endinput\||fi\newif\ifchilddoc|\\
|\edef\childdocname{\scantokens\expandafter{\jobname\noexpand}}|\\
|\def\childdocmain{|\textit{main}|}\||ifx\childdocmain\childdocname\||else|\\
|\childdoctrue\includeonly{\childdocname}\let\jobname\childdocmain\||fi|\\
\end{tabular}
\end{center}
%
Instead of |\childdocof{|\textit{main}|}| just include the main file
at the top of each child file:
%
\begin{center}
|\input{|\textit{main}|}|
\end{center}
%
A simple redirection |\childdocforward{|\textit{dest}|}| is achieved by:
%
\begin{center}
|\def\jobname{|\textit{dest}|}\input{\jobname}|
\end{center}
%
The redirection with prefix
|\childdocforwardprefix[|\textit{prefix}|]{|\textit{dest}|}|
is accomplished by:
%
\begin{center}
\begin{tabular}{l}
|{\edef\jobname{\scantokens\expandafter{\jobname\noexpand}}|\\
|\def\redirectjob |\textit{prefix}|#1~~~{\gdef\jobname{|\textit{dest}|#1}}|\\
|\expandafter\redirectjob\jobname~~~}\input{\jobname}|
\end{tabular}
\end{center}

In an alternative approach,
child documents can be compiled by a specific command line
without additional code or specific definitions:
%
\begin{center}
|... -jobname "|\textit{target}|" "|[\textit{flags}]%
|\includeonly{|\textit{dest}|}\input{|\textit{main}|}"|
\end{center}
%

%%%%%%%%%%%%%%%%%%%%%%%%%%%%%%%%%%%%%%%%%%%%%%%%%%%%%%%%%%%%%%%%%%%%%%%%%%%%%%%%
%%%%%%%%%%%%%%%%%%%%%%%%%%%%%%%%%%%%%%%%%%%%%%%%%%%%%%%%%%%%%%%%%%%%%%%%%%%%%%%%
\section{Information}

%%%%%%%%%%%%%%%%%%%%%%%%%%%%%%%%%%%%%%%%%%%%%%%%%%%%%%%%%%%%%%%%%%%%%%%%%%%%%%%%
\subsection{Copyright}

Copyright \copyright{} 2017--2018 Niklas Beisert

This work may be distributed and/or modified under the
conditions of the \LaTeX{} Project Public License, either version 1.3
of this license or (at your option) any later version.
The latest version of this license is in
  \url{http://www.latex-project.org/lppl.txt}
and version 1.3 or later is part of all distributions of \LaTeX{}
version 2005/12/01 or later.

This work has the LPPL maintenance status `maintained'.

The Current Maintainer of this work is Niklas Beisert.

This work consists of the files |README.txt|, |childdoc.ins| and |childdoc.dtx|
as well as the derived files |childdoc.def|, |cdocsamp.tex|
with |cdocsch1.tex|, |cdocsch2.tex|, |cdocspt3.tex|, |cdocspt4.tex|,
|cdocsdrf.tex|, |cdocsfn1.tex|, |cdocsfn2.tex|
as well as |childdoc.pdf|.

%%%%%%%%%%%%%%%%%%%%%%%%%%%%%%%%%%%%%%%%%%%%%%%%%%%%%%%%%%%%%%%%%%%%%%%%%%%%%%%%
\subsection{Files and Installation}

The package consists of the files:
%
\begin{center}
\begin{tabular}{ll}
    |README.txt|   & readme file \\
    |childdoc.ins| & installation file \\
    |childdoc.dtx| & source file \\
    |childdoc.def| & definition file \\
    |cdocsamp.tex| & sample main file \\
    |cdocsch1.tex| & sample include file \\
    |cdocsch2.tex| & sample include file \\
    |cdocspt3.tex| & sample part file \\
    |cdocspt4.tex| & sample part file \\
    |cdocsdrf.tex| & sample redirection file \\
    |cdocsfn1.tex| & sample redirection file \\
    |cdocsfn2.tex| & sample redirection file \\
    |childdoc.pdf| & manual
\end{tabular}
\end{center}
%
The distribution consists of the files
|README.txt|, |childdoc.ins| and |childdoc.dtx|.
%
\begin{itemize}
\item
Run (pdf)\LaTeX{} on |childdoc.dtx|
to compile the manual |childdoc.pdf| (this file).
\item
Run \LaTeX{} on |childdoc.ins| to create the definitions file |childdoc.def|
and the sample |cdocsamp.tex| with include files
|cdocsch1.tex|, |cdocsch2.tex|, |cdocspt3.tex|, |cdocspt4.tex|,
|cdocsdrf.tex|, |cdocsfn1.tex|, |cdocsfn2.tex|.
Then copy the file |childdoc.def| to an appropriate directory of your \LaTeX{}
distribution, e.g.\ \textit{texmf-root}|/tex/latex/childdoc|.
\end{itemize}

%%%%%%%%%%%%%%%%%%%%%%%%%%%%%%%%%%%%%%%%%%%%%%%%%%%%%%%%%%%%%%%%%%%%%%%%%%%%%%%%
\subsection{Related CTAN Packages}

There are several other packages which offer a similar functionality:
%
\begin{itemize}
\item
The packages
\href{http://ctan.org/pkg/docmute}{\textsf{docmute}},
\href{http://ctan.org/pkg/includex}{\textsf{includex}} and
\href{http://ctan.org/pkg/standalone}{\textsf{standalone}}
provide commands to include only the document body of
a child file thus allowing both files to be compiled individually.
\item
The packages \href{http://ctan.org/pkg/subdocs}{\textsf{subdocs}}
and \href{http://ctan.org/pkg/subfiles}{\textsf{subfiles}}
provide structures in which the main and child documents can be
encapsulated and allowing them to be compiled individually.
The inclusion mechanism is different from the conventional |\include|.
\item
The package \href{http://ctan.org/pkg/combine}{\textsf{combine}}
is an elaborate solution to combine several documents into one.
\end{itemize}
%
See also the CTAN topic \href{http://ctan.org/topic/subdocs}{\textsf{subdocs}}
for further related packages.
The present package differs from the above solutions in that
a document structure constructed with the conventional |\include| mechanism
just needs two extra commands at the top of every file
such that all constituent files can be compiled individually.

%%%%%%%%%%%%%%%%%%%%%%%%%%%%%%%%%%%%%%%%%%%%%%%%%%%%%%%%%%%%%%%%%%%%%%%%%%%%%%%%
%\subsection{Feature Suggestions}
%
%The following is a list of features which may be useful for future
%versions of this package:
%%
%\begin{itemize}
%\item
%\ldots
%\end{itemize}

%%%%%%%%%%%%%%%%%%%%%%%%%%%%%%%%%%%%%%%%%%%%%%%%%%%%%%%%%%%%%%%%%%%%%%%%%%%%%%%%
\subsection{Revision History}

%%%%%%%%%%%%%%%%%%%%%%%%%%%%%%%%%%%%%%%%
\paragraph{v2.0:} 2018/12/30

\begin{itemize}
\item
immediate forward processing
\item
added |\childdocby| mechanism
\item
manual restructured
\end{itemize}

%%%%%%%%%%%%%%%%%%%%%%%%%%%%%%%%%%%%%%%%
\paragraph{v1.6:} 2018/01/17

\begin{itemize}
\item
application for development of include files
\item
corrections to manual
\end{itemize}

%%%%%%%%%%%%%%%%%%%%%%%%%%%%%%%%%%%%%%%%
\paragraph{v1.5:} 2017/05/21

\begin{itemize}
\item
more complete structuring introduced
\item
|\childdocof| introduced
\item
|\childdoc| renamed to |\childdocmain|
\item
|\childredirect| renamed to |\childdocforward| and |\childdocforwardprefix|
and functionality expanded
\end{itemize}

%%%%%%%%%%%%%%%%%%%%%%%%%%%%%%%%%%%%%%%%
\paragraph{v1.0:} 2017/04/27

\begin{itemize}
\item
manual and install package
\item
first version published on CTAN
\end{itemize}

%%%%%%%%%%%%%%%%%%%%%%%%%%%%%%%%%%%%%%%%
\paragraph{v0.6:} 2017/04/26

\begin{itemize}
\item
redirection mechanism added
\end{itemize}

%%%%%%%%%%%%%%%%%%%%%%%%%%%%%%%%%%%%%%%%
\paragraph{v0.5:} 2017/04/26

\begin{itemize}
\item
functionality in definition file
\end{itemize}


%%%%%%%%%%%%%%%%%%%%%%%%%%%%%%%%%%%%%%%%%%%%%%%%%%%%%%%%%%%%%%%%%%%%%%%%%%%%%%%%
%%%%%%%%%%%%%%%%%%%%%%%%%%%%%%%%%%%%%%%%%%%%%%%%%%%%%%%%%%%%%%%%%%%%%%%%%%%%%%%%
%%%%%%%%%%%%%%%%%%%%%%%%%%%%%%%%%%%%%%%%%%%%%%%%%%%%%%%%%%%%%%%%%%%%%%%%%%%%%%%%
\appendix

\settowidth\MacroIndent{\rmfamily\scriptsize 000\ }

 \DocInput{childdoc.dtx}

\end{document}
%</driver>
% \fi
%
% %%%%%%%%%%%%%%%%%%%%%%%%%%%%%%%%%%%%%%%%%%%%%%%%%%%%%%%%%%%%%%%%%%%%%%%%%%%%%%
% %%%%%%%%%%%%%%%%%%%%%%%%%%%%%%%%%%%%%%%%%%%%%%%%%%%%%%%%%%%%%%%%%%%%%%%%%%%%%%
% \section{Sample}
%\iffalse
%<*samplemain>
%\fi
%
% The following presents a sample document
% with two chapters, two parts, a title page,
% a compile flag as well as three forwarding files to set the flag.
% It consists of eight |.tex| files:
% \begin{center}
% \begin{tabular}{ll}
% |cdocsamp.tex|&main file\\
% |cdocsch1.tex|&include file for chapter 1\\
% |cdocsch2.tex|&include file for chapter 2\\
% |cdocspt3.tex|&include file for part 3\\
% |cdocspt4.tex|&include file for part 4\\
% |cdocsdrf.tex|&forwarding file for main file in draft mode\\
% |cdocsfi1.tex|&forwarding file for final version of chapter 1\\
% |cdocsfi2.tex|&forwarding file for final version of chapter 2\\
% \end{tabular}
% \end{center}
% Each of the eight files can be compiled directly by the \LaTeX{} compiler.
%
% %%%%%%%%%%%%%%%%%%%%%%%%%%%%%%%%%%%%%%
% \paragraph{Main File.}
%
% The main file is called |cdocsamp.tex|.
%
% Load the \textsf{childdoc} definitions and
% declare the filename for the main document:
%    \begin{macrocode}
\input{childdoc.def}
\childdocmain{}
%    \end{macrocode}

% Optional override for |\version| flag:
%    \begin{macrocode}
%%\ifchilddoc\else\providecommand{\version}{draft}\fi
%    \end{macrocode}

% Define the default values for the |\version| flag
% (|final| for the main file and |draft| for childs):
%    \begin{macrocode}
\ifchilddoc
\providecommand{\version}{draft}
\else
\providecommand{\version}{final}
\fi
%    \end{macrocode}

% Load the standard document class:
%    \begin{macrocode}
\documentclass[12pt]{article}
%    \end{macrocode}

% Start the document body:
%    \begin{macrocode}
\begin{document}
%    \end{macrocode}

% Declare a title page.
% Print title, part of document being processed and version flag:
%    \begin{macrocode}
\addtocounter{page}{-1}
\begin{center}
{\LARGE\bfseries{}childdoc example\par}
\vspace{1cm}
\ifchilddoc
\ifchilddocmanual part\else chapter\fi:
`\childdocname' of `\childdocjob'\par
\else
main document: `\childdocjob'\par
\fi
version: \version\par
\end{center}
\newpage
%    \end{macrocode}

% Manually include selected file,
% otherwise process as usual:
%    \begin{macrocode}
\ifchilddocmanual
\section*{part `\childdocname'}
\input{\childdocname}
\else
%    \end{macrocode}

% Include the two chapters:
%    \begin{macrocode}
\include{cdocsch1}
\include{cdocsch2}
%    \end{macrocode}

% Include the two parts unless only chapters should be displayed:
%    \begin{macrocode}
\ifchilddoc\else
\section{part three}
\input{cdocspt3}
\section{part four}
\input{cdocspt4}
\fi
%    \end{macrocode}

% Process as usual until here:
%    \begin{macrocode}
\fi
%    \end{macrocode}

% End of document body:
%    \begin{macrocode}
\end{document}
%    \end{macrocode}
%\iffalse
%</samplemain>
%\fi
%
% %%%%%%%%%%%%%%%%%%%%%%%%%%%%%%%%%%%%%%
% \paragraph{Chapter Include Files.}
%
% The include files are called |cdocsch1.tex| and |cdocsch2.tex|.
%
%\iffalse
%<*samplechap1|samplechap2>
%\fi

% Optional override for |\version| flag:
%    \begin{macrocode}
%%\providecommand{\version}{final}
%    \end{macrocode}

% Include the main document:
%    \begin{macrocode}
\input{childdoc.def}
\childdocof{cdocsamp}
%    \end{macrocode}

%\iffalse
%</samplechap1|samplechap2>
%\fi
%
%\iffalse
%<*samplechap1>
%\fi
% Some text for chapter 1:
%    \begin{macrocode}
\section{one}
some text in chapter one
%    \end{macrocode}

%\iffalse
%</samplechap1>
%\fi
% Some text for chapter 2:
%\iffalse
%<*samplechap2>
%\fi
%    \begin{macrocode}
\section{two}
more text in chapter two
%    \end{macrocode}

%\iffalse
%</samplechap2>
%\fi
%
% %%%%%%%%%%%%%%%%%%%%%%%%%%%%%%%%%%%%%%
% \paragraph{Part Include Files.}
%
% The include files are called |cdocspt3.tex| and |cdocspt4.tex|.
%
%\iffalse
%<*samplepart3|samplepart4>
%\fi

% Optional override for |\version| flag:
%    \begin{macrocode}
%%\providecommand{\version}{final}
%    \end{macrocode}

% Include the main document:
%    \begin{macrocode}
\input{childdoc.def}
\childdocby{cdocsamp}
%    \end{macrocode}

%\iffalse
%</samplepart3|samplepart4>
%\fi
%
%\iffalse
%<*samplepart3>
%\fi
% Some text for part 3:
%    \begin{macrocode}
some text in part three
%    \end{macrocode}

%\iffalse
%</samplepart3>
%\fi
% Some text for part 4:
%\iffalse
%<*samplepart4>
%\fi
%    \begin{macrocode}
more text in part four
%    \end{macrocode}

%\iffalse
%</samplepart4>
%\fi
%
% %%%%%%%%%%%%%%%%%%%%%%%%%%%%%%%%%%%%%%
% \paragraph{Forwarding for a Complete Draft.}
%
% The following forwarding file |cdocsdrf.tex|
% compiles the main document in draft mode:
%\iffalse
%<*sampledraft>
%\fi
%    \begin{macrocode}
\def\version{draft}
\input{childdoc.def}
\childdocforward{cdocsamp}
%    \end{macrocode}

%\iffalse
%</sampledraft>
%\fi
%
% %%%%%%%%%%%%%%%%%%%%%%%%%%%%%%%%%%%%%%
% \paragraph{Forwarding for Final Version of the Chapters.}
%
% The following forwarding files |cdocsfn1.tex| and |cdocsfn2.tex|
% (with identical content)
% compile the final versions of the child documents
% |cdocsch1.tex| and |cdocsch2.tex|, respectively:
%\iffalse
%<*samplefinal>
%\fi
%    \begin{macrocode}
\def\version{final}
\input{childdoc.def}
\childdocforwardprefix[cdocsamp]{cdocsfn}{cdocsch}
%    \end{macrocode}

%\iffalse
%</samplefinal>
%\fi
%
% %%%%%%%%%%%%%%%%%%%%%%%%%%%%%%%%%%%%%%
% \paragraph{Command Line Processing.}
%
% The following three command lines generate the output files
% |cdocscld|, |cdocscl1| and |cdocscl2|
% which should be identical to
% |cdocsdrf|, |cdocsch1| and |cdocsfn2|, respectively:
% \begin{center}
% \begin{tabular}{l}
% |latex -jobname cdocscld \|\\
% |  "\def\version{draft}\input{childdoc.def}\childdocforward{cdocsamp}"|\\
% |latex -jobname cdocscl1 \|\\
% |  "\input{childdoc.def}\childdocforward[cdocsamp]{cdocsch1}"|\\
% |latex -jobname cdocscl2 \|\\
% |  "\def\version{final}\input{childdoc.def}\childdocforward{cdocsch2}"|
% \end{tabular}
% \end{center}
% Note that the trailing backslash on each first line
% merely continues the input to the second line
% (for convenient cut ant paste).
% Furthermore, the command |latex| can be replaced by any
% of its alternative versions such as |pdflatex|.
%
% %%%%%%%%%%%%%%%%%%%%%%%%%%%%%%%%%%%%%%%%%%%%%%%%%%%%%%%%%%%%%%%%%%%%%%%%%%%%%%
% %%%%%%%%%%%%%%%%%%%%%%%%%%%%%%%%%%%%%%%%%%%%%%%%%%%%%%%%%%%%%%%%%%%%%%%%%%%%%%
% \section{Implementation}
%\iffalse
%<*package>
%\fi
%
% This section describes the definitions file |childdoc.def|.

% The definitions cannot be loaded using |\usepackage| or |\RequirePackage|
% which has a mechanism to prevent loading a style file more than once.
% When loading the definitions by means of |\input|
% multiple instances have to be prevented manually:
%\iffalse
%This code needs to be before the `\ProvidesFile' directive
%which is defined at the beginning of this file.
%Therefore it is also placed there and commented out here.
%</package>
%<*discard>
%\fi
%    \begin{macrocode}
\ifdefined\childdocmain\endinput\fi
%    \end{macrocode}
%\iffalse
%</discard>
%<*package>
%\fi
%
% \macro{\ifchilddoc}
% \macro{\ifchilddocmanual}
% The conditional |\ifchilddoc| tells whether a
% child (true) or main (false) document is being compiled.
% The conditional |\ifchilddocmanual| tells whether
% the |\includeonly| mechanism is used (false) or
% the selection of child files must be performed manually (true).
% The definitions initialise to false:
%    \begin{macrocode}
\newif\ifchilddoc
\newif\ifchilddocmanual
%    \end{macrocode}

% \macro{\childdocname}
% \macro{\childdocjob}
% The macro |\childdocname| stores the name of the main document
% to be compiled. The macro |\childdocjob| stores the name of
% the document on which the \LaTeX{} compiler was originally invoked.
% The content of |\jobname| cannot be compared
% to filenames specified in the source due to different catcodes.
% The following code rescans |\jobname|, stores the result
% in |\childdocname| and saves a copy in |\childdocjob|:
%    \begin{macrocode}
\edef\childdocname{\scantokens\expandafter{\jobname\noexpand}}
\let\childdocjob\childdocname
%    \end{macrocode}

% \macro{\childdocdisable}
% The macro |\childdocdisable| prevents the main file
% from being processed more than once.
% At this stage, the main document command |\childdocmain|
% is assumed to be called once again where it should do nothing.
% Any subsequent call to it should prevent
% a secondary processing of the main document
% It overwrites the forwarding commands
% |\childdocof| and |\childdocforward|
% with empty macros to prevent further inclusions of the main document:
%    \begin{macrocode}
\newcommand{\childdocdisable}
{
  \renewcommand{\childdocmain}[1]{\renewcommand{\childdocmain}[1]{\endinput}}
  \renewcommand{\childdocof}[1]{}
  \renewcommand{\childdocby}[2][]{}
  \renewcommand{\childdocforward}[2][]{}
  \renewcommand{\childdocdisable}{}
}
%    \end{macrocode}

% \macro{\childdocmain}
% The macro |\childdocmain| is to be called at the top of the main file
% with nothing or the main filename (without extension) as argument.
% First, it breaks loops.
% If the argument is not empty and does not match |\childdocname|
% (which is set by the first inclusion of |childdoc.def|),
% |\ifchilddoc| is set to true, |\includeonly| is applied to the child file
% and |\jobname| is set to the main file
% (for proper handling of |.aux| files):
%    \begin{macrocode}
\newcommand{\childdocmain}[1]
{
  \childdocdisable\childdocmain{}
  \if?#1?\else
    \begingroup
      \def\childdoctmp{#1}
      \ifx\childdoctmp\childdocname
        \def\childdoctmp{}
      \else
        \def\childdoctmp
        {
          \childdoctrue
          \includeonly{\childdocname}
          \def\childdocjob{#1}
          \def\jobname{#1}
        }
      \fi
      \expandafter
    \endgroup
    \childdoctmp
  \fi
}
%    \end{macrocode}

% \macro{\childdocof}
% The command |\childdocof| redirects
% compilation to the main file |#1|.
%    \begin{macrocode}
\newcommand{\childdocof}[1]
{
  \childdocdisable
  \childdoctrue
  \includeonly{\childdocname}
  \def\jobname{#1}
  \def\childdocjob{#1}
  \input{#1}
}
%    \end{macrocode}

% \macro{\childdocby}
% The command |\childdocby| ....
%    \begin{macrocode}
\newcommand{\childdocby}[2][]
{
  \childdocdisable
  \childdoctrue
  \childdocmanualtrue
  \if?#1?\else
    \def\jobname{#2}
  \fi
  \def\childdocjob{#2}
  \input{#2}
  \endinput
}
%    \end{macrocode}

% \macro{\childdocforward}
% The command |\childdocforward| redirects
% compilation to the main file or
% (if the optional argument is given) a child file.
% Parameters are set as if the main file
% or a child file starting with |\childdocof| was compiled.
% Then compilation is handed over to the main file:
%    \begin{macrocode}
\newcommand{\childdocforward}[2][]
{
  \begingroup
    \if?#1?
      \def\childdoctmp
      {
        \def\childdocname{#2}
        \def\childdocjob{#2}
        \def\jobname{#2}
        \input{#2}
        \endinput
      }
    \else
      \def\childdoctmp
      {
        \childdocdisable
        \def\childdocname{#2}
        \childdoctrue
        \includeonly{#2}
        \def\childdocjob{#1}
        \def\jobname{#1}
        \input{#1}
        \endinput
      }
    \fi
    \expandafter
  \endgroup
  \childdoctmp
}
%    \end{macrocode}

% \macro{\childdocforwardprefix}
% The command |\childdocforwardprefix| redirects
% compilation to the main or a child file by means of a pattern.
% The prefix |#1| in the current filename is replaced by |#2|
% and the suffix of the current filename is kept
% (it is assumed that the filename does not contain the substring `|~~~|'
% which is used as a delimiter).
% Compilation is handed over to the new file by |\childdocforward|:
%    \begin{macrocode}
\newcommand{\childdocforwardprefix}[3][]
{
  \begingroup
    \def\childdocextract #2##1~~~{\def\childdoctmp{\childdocforward[#1]{#3##1}}}
    \expandafter\childdocextract\childdocname~~~
    \expandafter
  \endgroup
  \childdoctmp
}
%    \end{macrocode}

% \macro{\childdoc}
% The deprecated macro |\childdoc| is a legacy version of |\childdocmain|:
%    \begin{macrocode}
\newcommand{\childdoc}{\childdocmain}
%    \end{macrocode}

% \macro{\childdocredirect}
% The deprecated macro |\childdocredirect| is a legacy version
% of |\childdocforward| and |\childdocforwardprefix|:
%    \begin{macrocode}
\newcommand{\childdocredirect}[2][]
{
  \begingroup
    \if?#1?
      \def\childdoctmp{\childdocforward{#2}}
    \else
      \def\childdoctmp{\childdocforwardprefix{#1}{#2}}
    \fi
    \expandafter
  \endgroup
  \childdoctmp
}
%    \end{macrocode}

%\iffalse
%</package>
%\fi
%
\endinput
|\\
|\childdocby{|\textit{main}|}|\\
\end{tabular}
\end{center}
%
Both forms have slightly different effects as described above.
The main file is prepared as usual, see \secref{sec:include}.

%%%%%%%%%%%%%%%%%%%%%%%%%%%%%%%%%%%%%%%%%%%%%%%%%%%%%%%%%%%%%%%%%%%%%%%%%%%%%%%%
\subsection{Legacy Detection}
\label{sec:detection}

The directive |\childdocmain| in the main file can detect
whether the complete document or merely a child is to be compiled
even without using the directive |\childdocof|.
This method is deprecated because it is less robust
and there is no compelling reason to use it;
it is merely provided for backward compatibility
and it may be removed in future versions.

If the detection mechanism is to be used,
it is mandatory to correctly specify
the filename of the main file as the argument of |\childdocmain|:
%
\begin{center}
\begin{tabular}{l}
|% \iffalse
%
% childdoc.dtx Copyright (C) 2017-2018 Niklas Beisert
%
% This work may be distributed and/or modified under the
% conditions of the LaTeX Project Public License, either version 1.3
% of this license or (at your option) any later version.
% The latest version of this license is in
%   http://www.latex-project.org/lppl.txt
% and version 1.3 or later is part of all distributions of LaTeX
% version 2005/12/01 or later.
%
% This work has the LPPL maintenance status `maintained'.
%
% The Current Maintainer of this work is Niklas Beisert.
%
% This work consists of the files childdoc.dtx and childdoc.ins
% and the derived files childdoc.def and cdocsamp.tex with
% cdocsch1.tex, cdocsch2.tex, cdocsdrf.tex, cdocsfn1.tex, cdocsfn2.tex.
%
%<package>\ifdefined\childdocmain\endinput\fi
%<package>\ProvidesFile{childdoc.def}[2018/12/30 v2.0 child document driver]
%<samplemain>\ProvidesFile{cdocsamp.tex}[2018/12/30 v2.0 sample for childdoc]
%<*driver>
%\ProvidesFile{childdoc.drv}[2018/12/30 v2.0 childdoc reference manual file]
\PassOptionsToClass{10pt,a4paper}{article}
\documentclass{ltxdoc}

\usepackage[margin=35mm]{geometry}
\usepackage{hyperref}
\usepackage{hyperxmp}
\usepackage[usenames]{color}

\hypersetup{colorlinks=true}
\hypersetup{pdfstartview=FitH}
\hypersetup{pdfpagemode=UseNone}
\hypersetup{pdfsource={}}
\hypersetup{pdflang={en-UK}}
\hypersetup{pdfcopyright={Copyright 2017-2018 Niklas Beisert.
  This work may be distributed and/or modified under the
  conditions of the LaTeX Project Public License, either version 1.3
  of this license or (at your option) any later version.}}
\hypersetup{pdflicenseurl={http://www.latex-project.org/lppl.txt}}
\hypersetup{pdfcontactaddress={ETH Zurich, ITP, HIT K,
  Wolfgang-Pauli-Strasse 27}}
\hypersetup{pdfcontactpostcode={8093}}
\hypersetup{pdfcontactcity={Zurich}}
\hypersetup{pdfcontactcountry={Switzerland}}
\hypersetup{pdfcontactemail={nbeisert@itp.phys.ethz.ch}}
\hypersetup{pdfcontacturl={http://people.phys.ethz.ch/\xmptilde nbeisert/}}

\newcommand{\secref}[1]{\hyperref[#1]{section \ref*{#1}}}

\parskip1ex
\parindent0pt
\let\olditemize\itemize
\def\itemize{\olditemize\parskip0pt}

\begin{document}

\title{The \textsf{childdoc} Package}
\hypersetup{pdftitle={The childdoc Package}}
\author{Niklas Beisert\\[2ex]
  Institut f\"ur Theoretische Physik\\
  Eidgen\"ossische Technische Hochschule Z\"urich\\
  Wolfgang-Pauli-Strasse 27, 8093 Z\"urich, Switzerland\\[1ex]
  \href{mailto:nbeisert@itp.phys.ethz.ch}
  {\texttt{nbeisert@itp.phys.ethz.ch}}}
\hypersetup{pdfauthor={Niklas Beisert}}
\hypersetup{pdfsubject={Manual for the LaTeX2e Package childdoc}}
\date{30 December 2018, \textsf{v2.0}}
\maketitle

\begin{abstract}\noindent
\textsf{childdoc} is a \LaTeXe{} package
that enables the direct compilation
of document sections included by |\include|
to individual files.
\end{abstract}

\begingroup
\parskip0ex
\tableofcontents
\endgroup

%%%%%%%%%%%%%%%%%%%%%%%%%%%%%%%%%%%%%%%%%%%%%%%%%%%%%%%%%%%%%%%%%%%%%%%%%%%%%%%%
%%%%%%%%%%%%%%%%%%%%%%%%%%%%%%%%%%%%%%%%%%%%%%%%%%%%%%%%%%%%%%%%%%%%%%%%%%%%%%%%
\section{Introduction}

\LaTeX{} provides a mechanism to structure a large document (such as a book)
into a main file and several child files (containing the chapters)
using the |\include| command.
This mechanism is beneficial for documents
which span hundreds of pages in order to
make the source file(s) more manageable.
Moreover, compilation can be restricted to
selected child files by means of the |\includeonly| command.
The latter feature can be used to reduce the compilation time while editing
(this was significantly more useful in the earlier days of \LaTeX{})
or to generate a smaller document which is easier to navigate.
Another application of |\includeonly| is to generate
documents consisting of selected parts of the complete document.

However, there are a few drawbacks of the plain |\include| mechanism:
\begin{itemize}
\item
The child files cannot be compiled on their own,
they can only be compiled via the main file.
A naive editing environment
(such as a text editor with an option
to have the current file processed by \LaTeX)
may require one to switch to the main file before compiling;
attempting to compile the child file produces errors.
\item
The main file must be modified (each time)
to adjust the |\includeonly| command
to the present needs. This easily leaves the main file in a messy state.
\item
The generated document will always carry the filename
of the main document. This is inconvenient if
several child files are to be compiled and
to be kept for distribution.
\end{itemize}

The present package provides a simple interface
to make child files individually compilable by \LaTeX{}.
Compiling a child file then has the same effect as compiling
the main file with an |\includeonly| command
to select the appropriate child.
Moreover the generated document will carry the name of the child
rather than the main file.
This resolves all three above issues.

This feature is meant to make the editing of books,
thesis documents and lecture notes somewhat more convenient.
However, the package can also be used efficiently for
composing a series of documents (such as exercise sheets)
which are typically distributed individually.
It then assists the author in generating the individual documents
(potentially in different versions)
as well as a document containing the collected series.
Another application is in developing style files
or other kinds of included material
where compilation of the style file could redirect
to a sample or test file.

%%%%%%%%%%%%%%%%%%%%%%%%%%%%%%%%%%%%%%%%%%%%%%%%%%%%%%%%%%%%%%%%%%%%%%%%%%%%%%%%
%%%%%%%%%%%%%%%%%%%%%%%%%%%%%%%%%%%%%%%%%%%%%%%%%%%%%%%%%%%%%%%%%%%%%%%%%%%%%%%%
\section{Usage}

First of all, the package \textsf{childdoc} is \emph{not} a standard
\LaTeXe{} |.sty| style file! Therefore it needs to be invoked in
a non-standard way.

%%%%%%%%%%%%%%%%%%%%%%%%%%%%%%%%%%%%%%%%%%%%%%%%%%%%%%%%%%%%%%%%%%%%%%%%%%%%%%%%
\subsection{Included Files}
\label{sec:include}

%%%%%%%%%%%%%%%%%%%%%%%%%%%%%%%%%%%%%%%%
\DescribeMacro{\childdocmain}
To use the package, add the commands
\begin{center}
\begin{tabular}{l}
|\input{childdoc.def}|\\
|\childdocmain{}|\\
\end{tabular}
\end{center}
at the very top of the main \LaTeX{} file,
in particular \emph{before} the |\documentclass| statement!
The argument of |\childdocmain| should be left empty
(but it must be present).

%%%%%%%%%%%%%%%%%%%%%%%%%%%%%%%%%%%%%%%%
\DescribeMacro{\childdocof}
Furthermore, add the commands
\begin{center}
\begin{tabular}{l}
|\input{childdoc.def}|\\
|\childdocof{|\textit{main}|}|\\
\end{tabular}
\end{center}
at the top of every child file \textit{child}
which is included by |\include{|\textit{child}|}|
from within the main file
(or at least for those files to be compiled individually).
The argument \textit{main} must be the filename of the main file.

There are a couple of
considerations in setting up the main and child documents:

%%%%%%%%%%%%%%%%%%%%%%%%%%%%%%%%%%%%%%%%
\paragraph{Restrictions.}

Please note the following restrictions:
\begin{itemize}
\item
|\childdocmain| must be called with one argument \textit{main}
to ensure compatibility with earlier version of the package.
It must either be empty (|\childdocmain{}|)
or precisely match the filename of the main file in which it is specified.
See \secref{sec:detection} for further information.
\item
The filename \textit{main} must be specified without the |.tex| extension.
\item
The filename \textit{main} is case sensitive
(even in case-insensitive file systems)
due to internal string comparison.
\item
The argument \textit{main} should be fully expanded, it cannot be a macro.
\item
Subdirectories and special characters should be avoided in filenames.
\item
The command |\childdocmain{|\textit{main}|}| must be followed by a whitespace.
It should not be followed immediately by another command
or by a comment mark `|%|'.
This is because the \TeX{} parser reads the token immediately following
the argument of |\childdocmain| and puts it
at the beginning of every child section;
however, a white\-space is ignored.
\end{itemize}

%%%%%%%%%%%%%%%%%%%%%%%%%%%%%%%%%%%%%%%%
\paragraph{Content of Main File.}

It is advisable to place all content in the child files included by |\include|.
Any output contained in the main file will appear in all child documents
unless suppressed manually;
it cannot be suppressed automatically by the |\includeonly| directive
and thus should normally be avoided.
A method to include some content in the main file
by means of conditional processing is described in \secref{sec:conditional}.

%%%%%%%%%%%%%%%%%%%%%%%%%%%%%%%%%%%%%%%%
\paragraph{Page Numbering.}

When only a part of the document is compiled,
the appropriate numbering of pages
(as well as other status parameters)
is determined from the |.aux| files.
The latter contain information from previous passes.
However this information needs to propagate through
all intermediate child documents.
Therefore the page numbering in child documents may well
be inconsistent until the complete document is compiled at least once.

A useful (if unconventional) way to always ensure a consistent
page numbering is to restart the numbering in each child document
and denote the pages by `\textit{child}|.|\textit{page}'
where \textit{child} represents the chapter/section number of the child file.
This can be achieved by the command
|\numberwithin{page}{|\textit{child}|}|
of the \textsf{amsmath} package
where \textit{child} can be |chapter| or |section|
depending on the chosen structuring.
Alternatively, one can modify the macro |\thepage| appropriately
and reset the counter |page| at the start of each child file.

%%%%%%%%%%%%%%%%%%%%%%%%%%%%%%%%%%%%%%%%%%%%%%%%%%%%%%%%%%%%%%%%%%%%%%%%%%%%%%%%
\subsection{Conditional Processing}
\label{sec:conditional}

The package provides a mechanism to compile different versions
of a document. To customise the versions further some conditional processing
can come in handy to distinguish which version is being compiled.
The package provides two macros to describe the compilation context:

%%%%%%%%%%%%%%%%%%%%%%%%%%%%%%%%%%%%%%%%
\DescribeMacro{\ifchilddoc}
The conditional |\ifchilddoc| distinguishes between the compilation of
child documents and the main document:
%
\begin{center}
|\ifchilddoc |\textit{child-code}| |[|\||else |\textit{main-code}]| \||fi|
\end{center}

%%%%%%%%%%%%%%%%%%%%%%%%%%%%%%%%%%%%%%%%
\DescribeMacro{\childdocname}
\DescribeMacro{\childdocjob}
The macro |\childdocname| contains the filename (without extension)
of the main or child file being processed.
Note that |\childdocjob| will always contain the name of the main file.

%%%%%%%%%%%%%%%%%%%%%%%%%%%%%%%%%%%%%%%%
\paragraph{Title Page.}

Conditional processing can be used to include a title or banner page
in the main document when proper precautions are taken.
Importantly, the code in the main file should ensure that the page counter
(as well as other status parameters which are stored in the |.aux| files)
takes the same value after the conditional processing.
Otherwise the page numbers may take divergent values
depending on which part is compiled.

For example, a title page could be declared by:
%
\begin{center}
\begin{tabular}{l}
|\ifchilddoc\||else|\\
|\addtocounter{page}{-1}|\\
\textit{code for title page}\\
|\newpage|\\
|\||fi|
\end{tabular}
\end{center}
%
A banner page for the child documents can be generated by:
%
\begin{center}
\begin{tabular}{l}
|\ifchilddoc|\\
|\addtocounter{page}{-1}|\\
\textit{code for banner page}\\
|\newpage|\\
|\||fi|
\end{tabular}
\end{center}
%
Here one could write a message such as:
\begin{center}
|This is the part \childdocname{} of \childdocjob{}.|
\end{center}

%%%%%%%%%%%%%%%%%%%%%%%%%%%%%%%%%%%%%%%%%%%%%%%%%%%%%%%%%%%%%%%%%%%%%%%%%%%%%%%%
\subsection{Flags}
\label{sec:flags}

The package makes it easy to generate different versions
of the main or child documents.
To this end compilation flags can be defined
and assigned different default values.
They will be particularly useful in conjunction
with the forwarding mechanism described in \secref{sec:forward}.

For example, it may be useful to have a flag |\version|
which can be set to |draft| or |final|.
The document source will contain some conditional code
depending on the value of |\version|.
Suppose further, the flag should default to |final| for the main file
and to |draft| for child files
which is a natural assignment for editing the document.
This is achieved by placing the following code
in the preamble of the main document
(below the |\childdocmain| directive):
%
\begin{center}
\begin{tabular}{l}
|\ifchilddoc|\\
|\providecommand{\version}{draft}|\\
|\||else|\\
|\providecommand{\version}{final}|\\
|\||fi|
\end{tabular}
\end{center}
%
The definition by |\providecommand| makes sure
that previous definitions are not overwritten.
Further statements |\providecommand{\version}{...}|
can thus be added before the above code to override it.

For the main file, one might add a line
(between |\childdocmain| and the above block)
%
\begin{center}
|%\ifchilddoc\||else\providecommand{\version}{draft}\||fi|
\end{center}
%
which can be uncommented to produce a draft version.
Likewise one can add a line to the very top of a child file
(above the |\childdocof{|\textit{main}|}| directive)
%
\begin{center}
|%\providecommand{\version}{final}|
\end{center}
%
which can be uncommented to produce the final version of this child document.

%%%%%%%%%%%%%%%%%%%%%%%%%%%%%%%%%%%%%%%%%%%%%%%%%%%%%%%%%%%%%%%%%%%%%%%%%%%%%%%%
\subsection{Forwarding}
\label{sec:forward}

Different versions of the main or child documents
using compilation flags as described in \secref{sec:flags}
can be (permanently) stored in different files
for convenient compilation, viewing and distribution.
To this end, the package defines a command
to pass on compilation to a different file:

%%%%%%%%%%%%%%%%%%%%%%%%%%%%%%%%%%%%%%%%
\DescribeMacro{\childdocforward}
The command |\childdocforward| redirects processing to
another source file:
%
\begin{center}
\begin{tabular}{l}
|\input{childdoc.def}|\\
|\childdocforward[|\textit{main}|]{|\textit{dest}|}|\\
\end{tabular}
\end{center}
%
The argument \textit{dest} is the destination file
(without extension).
It should be the main file or one of the child files.
Note that further \textsf{childdoc} directives
such as |\childdocof| and |\childdocforward|
in the indicated file will be processed in this form.
The optional argument \textit{main}
passes on directly to the main file \textit{main}
while pretending to compile the child \textit{dest}.
This form behaves as if \textit{dest}
issues |\childdocof{|\textit{main}|}| right away,
and no further \textsf{childdoc} directives will be processed.

%%%%%%%%%%%%%%%%%%%%%%%%%%%%%%%%%%%%%%%%
\DescribeMacro{\...prefix}
In the alternative form |\childdocforwardprefix|,
%
\begin{center}
\begin{tabular}{l}
|\input{childdoc.def}|\\
|\childdocforwardprefix[|\textit{main}|]{|\textit{prefix}|}{|\textit{dest}|}|
\end{tabular}
\end{center}
%
the destination file is determined by a pattern
depending on the current file:
To make this work, the current file must be called
`{\textit{prefix}\hspace{0.2em}\textit{suffix}}'
with \textit{prefix} matching precisely the argument.
Processing is then passed on to the file
`{\textit{dest}\hspace{0.2em}\textit{suffix}}'.
Surely, the same effect is achieved by
directly specifying the
argument `{\textit{dest}\hspace{0.2em}\textit{suffix}}'
in the first form.
However, that requires to set up a different file
for each child. With the alternative form of the command
all these files can have exactly the same content
which simplifies setting them up and maintaining them.

For example, the following file |draft.tex|
with a compilation flag |\version| as described in \secref{sec:flags}
compiles the main document as a draft:
%
\begin{center}
\begin{tabular}{l}
|\def\version{draft}|\\
|\input{childdoc.def}|\\
|\childdocforward{|\textit{main}|}|
\end{tabular}
\end{center}
%
Likewise, the following files |final|\textit{nn}|.tex|
compile the final version of the child document
|child|\textit{nn}|.tex|:
%
\begin{center}
\begin{tabular}{l}
|\def\version{final}|\\
|\input{childdoc.def}|\\
|\childdocforwardprefix{final}{child}|
\end{tabular}
\end{center}
%

Note that when several versions of a main file and/or of each child file
are to be generated, it may be convenient to set up a |Makefile| or
shell script to automatise the process.

%%%%%%%%%%%%%%%%%%%%%%%%%%%%%%%%%%%%%%%%%%%%%%%%%%%%%%%%%%%%%%%%%%%%%%%%%%%%%%%%
\subsection{Command Line Processing}
\label{sec:commandline}

The effect of redirection files can also be achieved by invoking
the \LaTeX{} compiler with a more elaborate command line.
Most conveniently this should be done as part
of a shell script or a |Makefile|.

When using \textsf{childdoc} in the main file, the following
command lines effectively perform a redirection
(note that depending on the shell being used,
backslashes may have to be doubled: `|\|' $\to$ `|\\|'):
%
\begin{center}
|... -jobname "|\textit{target}|" |\\|"|[\textit{flags}]%
|\input{childdoc.def}\childdocforward[|\textit{main}|]{|\textit{dest}|}"|
\end{center}
%
Here \textit{target} is the name of the output file,
\textit{main} is the name of the main file
and \textit{dest} is the name of the main or child file to be processed
(all filenames without extensions).
The optional argument \textit{main} can be omitted
if \textit{main} matches \textit{dest}.
Optionally, compilation \textit{flags} can be defined via |\def| commands.
This command line makes the \TeX{} engine believe
it is compiling the file \textit{target}
whose content is specified as the latter parameter.
The provided code then forwards the processing to
\textit{main} or \textit{dest} as described in \secref{sec:forward}.

%%%%%%%%%%%%%%%%%%%%%%%%%%%%%%%%%%%%%%%%%%%%%%%%%%%%%%%%%%%%%%%%%%%%%%%%%%%%%%%%
\subsection{Include by Input}
\label{sec:input}

Including child documents by |\include| has some restrictions by design.
Most notably, the content of a child document always occupies
its own set of pages; pages cannot be shared between child documents.
Usually, this behaviour makes perfect sense
because each child document contain an essential part of the document.
However, in some situations it may be desirable to compose
a document from a collection of parts
without having mandatory page breaks between then.
For this case, the package
provides a mechanism to include parts
by |\input| which can also be processed individually.
However, by construction this mechanism
requires manual handling of the content to be output.

%%%%%%%%%%%%%%%%%%%%%%%%%%%%%%%%%%%%%%%%
\DescribeMacro{\ifchilddocmanual}
The main file should be prepared as usual, see \secref{sec:include}.
However, the document body must make a distinction
between processing of an individual part and of the main document, e.g.:
%
\begin{center}
\begin{tabular}{l}
|\ifchilddocmanual|\\
|\input{\childdocname}|\\
|\||else|\\
\textit{document body with }|\input{|\textit{part}|}|\\
|\||fi|
\end{tabular}
\end{center}
%
The conditional |\ifchilddocmanual| is true whenever
a part to be included by |\input| is being compiled,
and the name of the part is stored in |\childdocname|.

%%%%%%%%%%%%%%%%%%%%%%%%%%%%%%%%%%%%%%%%
\DescribeMacro{\childdocby}
Each part to be included by |\input| should start with:
%
\begin{center}
\begin{tabular}{l}
|\input{childdoc.def}|\\
|\childdocby{|\textit{main}|}|\\
\end{tabular}
\end{center}
%
The directive |\childdocby| is similar to |\childdocof|
described in \secref{sec:include},
but the subsequent selection of content must be done manually.
To that end, both |\ifchilddoc| and |\ifchilddocmanual|
will be true upon processing of a part,
and the name of the part is stored in |\childdocname|.
Note that |\jobname| will be set to the filename of the current part
so that each part receives an individual |.aux| file
that does not interfere with the |.aux| file(s) of the main document.
This behaviour can be altered by the alternative form
|\childdocby[*]{|\textit{main}|}| (with a non-empty optional argument)
which uses the |.aux| file of the main document
by setting |\jobname| to \textit{main}.

%%%%%%%%%%%%%%%%%%%%%%%%%%%%%%%%%%%%%%%%%%%%%%%%%%%%%%%%%%%%%%%%%%%%%%%%%%%%%%%%
\subsection{Driver Development}
\label{sec:driver}

The \textsf{childdoc} mechanism can also be use for the development
of definition files such as \LaTeX{} styles or classes.
This case differs from the above setup with multiple parts
included by |\include| in that no |\includeonly| should be invoked.
This can be achieved by starting the include file
(before |\ProvidesPackage|) with:
%
\begin{center}
\begin{tabular}{l}
|\input{childdoc.def}|\\
|\childdocforward{|\textit{main}|}|\\
\end{tabular}
\end{center}
%
or alternatively with:
%
\begin{center}
\begin{tabular}{l}
|\input{childdoc.def}|\\
|\childdocby{|\textit{main}|}|\\
\end{tabular}
\end{center}
%
Both forms have slightly different effects as described above.
The main file is prepared as usual, see \secref{sec:include}.

%%%%%%%%%%%%%%%%%%%%%%%%%%%%%%%%%%%%%%%%%%%%%%%%%%%%%%%%%%%%%%%%%%%%%%%%%%%%%%%%
\subsection{Legacy Detection}
\label{sec:detection}

The directive |\childdocmain| in the main file can detect
whether the complete document or merely a child is to be compiled
even without using the directive |\childdocof|.
This method is deprecated because it is less robust
and there is no compelling reason to use it;
it is merely provided for backward compatibility
and it may be removed in future versions.

If the detection mechanism is to be used,
it is mandatory to correctly specify
the filename of the main file as the argument of |\childdocmain|:
%
\begin{center}
\begin{tabular}{l}
|\input{childdoc.def}|\\
|\childdocmain{|\textit{main}|}|\\
\end{tabular}
\end{center}
%
If |\jobname| does not match the argument \textit{main} of |\childdocmain|,
it is assumed that |\jobname| points to the child file to be compiled.
When using |\childdocmain| with the main file specified as argument,
it suffices to start a child file
with just |\input{|\textit{main}|}|
without loading of the package and using |\childdocof|.
If instead all processing is done
with the appropriate \textsf{childdoc} directives,
the argument of \textit{main} of |\childdocmain| can be empty.

An alternative version of the command line processing described
in \secref{sec:commandline} using the detection mechanism reads:
%
\begin{center}
|... -jobname "|\textit{target}|" "|[\textit{flags}]%
[|\def\jobname{|\textit{dest}|}|]|\input{|\textit{main}|}"|
\end{center}

%%%%%%%%%%%%%%%%%%%%%%%%%%%%%%%%%%%%%%%%%%%%%%%%%%%%%%%%%%%%%%%%%%%%%%%%%%%%%%%%
\subsection{Manual Code}
\label{sec:manual}

In case one cannot be certain whether the definitions file |childdoc.def|
is installed on the target \TeX{} distribution
and one prefers not to ship it,
it is conceivable to paste a few relevant commands into the sources.

To that end, drop all statements |\input{childdoc.def}|
and perform the replacements as outlined below.
Instead of |\childdocmain{|\textit{main}|}| add the following code
to the top of the main file:
%
\begin{center}
\begin{tabular}{l}
|\||ifdefined\childdocname\endinput\||fi\newif\ifchilddoc|\\
|\edef\childdocname{\scantokens\expandafter{\jobname\noexpand}}|\\
|\def\childdocmain{|\textit{main}|}\||ifx\childdocmain\childdocname\||else|\\
|\childdoctrue\includeonly{\childdocname}\let\jobname\childdocmain\||fi|\\
\end{tabular}
\end{center}
%
Instead of |\childdocof{|\textit{main}|}| just include the main file
at the top of each child file:
%
\begin{center}
|\input{|\textit{main}|}|
\end{center}
%
A simple redirection |\childdocforward{|\textit{dest}|}| is achieved by:
%
\begin{center}
|\def\jobname{|\textit{dest}|}\input{\jobname}|
\end{center}
%
The redirection with prefix
|\childdocforwardprefix[|\textit{prefix}|]{|\textit{dest}|}|
is accomplished by:
%
\begin{center}
\begin{tabular}{l}
|{\edef\jobname{\scantokens\expandafter{\jobname\noexpand}}|\\
|\def\redirectjob |\textit{prefix}|#1~~~{\gdef\jobname{|\textit{dest}|#1}}|\\
|\expandafter\redirectjob\jobname~~~}\input{\jobname}|
\end{tabular}
\end{center}

In an alternative approach,
child documents can be compiled by a specific command line
without additional code or specific definitions:
%
\begin{center}
|... -jobname "|\textit{target}|" "|[\textit{flags}]%
|\includeonly{|\textit{dest}|}\input{|\textit{main}|}"|
\end{center}
%

%%%%%%%%%%%%%%%%%%%%%%%%%%%%%%%%%%%%%%%%%%%%%%%%%%%%%%%%%%%%%%%%%%%%%%%%%%%%%%%%
%%%%%%%%%%%%%%%%%%%%%%%%%%%%%%%%%%%%%%%%%%%%%%%%%%%%%%%%%%%%%%%%%%%%%%%%%%%%%%%%
\section{Information}

%%%%%%%%%%%%%%%%%%%%%%%%%%%%%%%%%%%%%%%%%%%%%%%%%%%%%%%%%%%%%%%%%%%%%%%%%%%%%%%%
\subsection{Copyright}

Copyright \copyright{} 2017--2018 Niklas Beisert

This work may be distributed and/or modified under the
conditions of the \LaTeX{} Project Public License, either version 1.3
of this license or (at your option) any later version.
The latest version of this license is in
  \url{http://www.latex-project.org/lppl.txt}
and version 1.3 or later is part of all distributions of \LaTeX{}
version 2005/12/01 or later.

This work has the LPPL maintenance status `maintained'.

The Current Maintainer of this work is Niklas Beisert.

This work consists of the files |README.txt|, |childdoc.ins| and |childdoc.dtx|
as well as the derived files |childdoc.def|, |cdocsamp.tex|
with |cdocsch1.tex|, |cdocsch2.tex|, |cdocspt3.tex|, |cdocspt4.tex|,
|cdocsdrf.tex|, |cdocsfn1.tex|, |cdocsfn2.tex|
as well as |childdoc.pdf|.

%%%%%%%%%%%%%%%%%%%%%%%%%%%%%%%%%%%%%%%%%%%%%%%%%%%%%%%%%%%%%%%%%%%%%%%%%%%%%%%%
\subsection{Files and Installation}

The package consists of the files:
%
\begin{center}
\begin{tabular}{ll}
    |README.txt|   & readme file \\
    |childdoc.ins| & installation file \\
    |childdoc.dtx| & source file \\
    |childdoc.def| & definition file \\
    |cdocsamp.tex| & sample main file \\
    |cdocsch1.tex| & sample include file \\
    |cdocsch2.tex| & sample include file \\
    |cdocspt3.tex| & sample part file \\
    |cdocspt4.tex| & sample part file \\
    |cdocsdrf.tex| & sample redirection file \\
    |cdocsfn1.tex| & sample redirection file \\
    |cdocsfn2.tex| & sample redirection file \\
    |childdoc.pdf| & manual
\end{tabular}
\end{center}
%
The distribution consists of the files
|README.txt|, |childdoc.ins| and |childdoc.dtx|.
%
\begin{itemize}
\item
Run (pdf)\LaTeX{} on |childdoc.dtx|
to compile the manual |childdoc.pdf| (this file).
\item
Run \LaTeX{} on |childdoc.ins| to create the definitions file |childdoc.def|
and the sample |cdocsamp.tex| with include files
|cdocsch1.tex|, |cdocsch2.tex|, |cdocspt3.tex|, |cdocspt4.tex|,
|cdocsdrf.tex|, |cdocsfn1.tex|, |cdocsfn2.tex|.
Then copy the file |childdoc.def| to an appropriate directory of your \LaTeX{}
distribution, e.g.\ \textit{texmf-root}|/tex/latex/childdoc|.
\end{itemize}

%%%%%%%%%%%%%%%%%%%%%%%%%%%%%%%%%%%%%%%%%%%%%%%%%%%%%%%%%%%%%%%%%%%%%%%%%%%%%%%%
\subsection{Related CTAN Packages}

There are several other packages which offer a similar functionality:
%
\begin{itemize}
\item
The packages
\href{http://ctan.org/pkg/docmute}{\textsf{docmute}},
\href{http://ctan.org/pkg/includex}{\textsf{includex}} and
\href{http://ctan.org/pkg/standalone}{\textsf{standalone}}
provide commands to include only the document body of
a child file thus allowing both files to be compiled individually.
\item
The packages \href{http://ctan.org/pkg/subdocs}{\textsf{subdocs}}
and \href{http://ctan.org/pkg/subfiles}{\textsf{subfiles}}
provide structures in which the main and child documents can be
encapsulated and allowing them to be compiled individually.
The inclusion mechanism is different from the conventional |\include|.
\item
The package \href{http://ctan.org/pkg/combine}{\textsf{combine}}
is an elaborate solution to combine several documents into one.
\end{itemize}
%
See also the CTAN topic \href{http://ctan.org/topic/subdocs}{\textsf{subdocs}}
for further related packages.
The present package differs from the above solutions in that
a document structure constructed with the conventional |\include| mechanism
just needs two extra commands at the top of every file
such that all constituent files can be compiled individually.

%%%%%%%%%%%%%%%%%%%%%%%%%%%%%%%%%%%%%%%%%%%%%%%%%%%%%%%%%%%%%%%%%%%%%%%%%%%%%%%%
%\subsection{Feature Suggestions}
%
%The following is a list of features which may be useful for future
%versions of this package:
%%
%\begin{itemize}
%\item
%\ldots
%\end{itemize}

%%%%%%%%%%%%%%%%%%%%%%%%%%%%%%%%%%%%%%%%%%%%%%%%%%%%%%%%%%%%%%%%%%%%%%%%%%%%%%%%
\subsection{Revision History}

%%%%%%%%%%%%%%%%%%%%%%%%%%%%%%%%%%%%%%%%
\paragraph{v2.0:} 2018/12/30

\begin{itemize}
\item
immediate forward processing
\item
added |\childdocby| mechanism
\item
manual restructured
\end{itemize}

%%%%%%%%%%%%%%%%%%%%%%%%%%%%%%%%%%%%%%%%
\paragraph{v1.6:} 2018/01/17

\begin{itemize}
\item
application for development of include files
\item
corrections to manual
\end{itemize}

%%%%%%%%%%%%%%%%%%%%%%%%%%%%%%%%%%%%%%%%
\paragraph{v1.5:} 2017/05/21

\begin{itemize}
\item
more complete structuring introduced
\item
|\childdocof| introduced
\item
|\childdoc| renamed to |\childdocmain|
\item
|\childredirect| renamed to |\childdocforward| and |\childdocforwardprefix|
and functionality expanded
\end{itemize}

%%%%%%%%%%%%%%%%%%%%%%%%%%%%%%%%%%%%%%%%
\paragraph{v1.0:} 2017/04/27

\begin{itemize}
\item
manual and install package
\item
first version published on CTAN
\end{itemize}

%%%%%%%%%%%%%%%%%%%%%%%%%%%%%%%%%%%%%%%%
\paragraph{v0.6:} 2017/04/26

\begin{itemize}
\item
redirection mechanism added
\end{itemize}

%%%%%%%%%%%%%%%%%%%%%%%%%%%%%%%%%%%%%%%%
\paragraph{v0.5:} 2017/04/26

\begin{itemize}
\item
functionality in definition file
\end{itemize}


%%%%%%%%%%%%%%%%%%%%%%%%%%%%%%%%%%%%%%%%%%%%%%%%%%%%%%%%%%%%%%%%%%%%%%%%%%%%%%%%
%%%%%%%%%%%%%%%%%%%%%%%%%%%%%%%%%%%%%%%%%%%%%%%%%%%%%%%%%%%%%%%%%%%%%%%%%%%%%%%%
%%%%%%%%%%%%%%%%%%%%%%%%%%%%%%%%%%%%%%%%%%%%%%%%%%%%%%%%%%%%%%%%%%%%%%%%%%%%%%%%
\appendix

\settowidth\MacroIndent{\rmfamily\scriptsize 000\ }

 \DocInput{childdoc.dtx}

\end{document}
%</driver>
% \fi
%
% %%%%%%%%%%%%%%%%%%%%%%%%%%%%%%%%%%%%%%%%%%%%%%%%%%%%%%%%%%%%%%%%%%%%%%%%%%%%%%
% %%%%%%%%%%%%%%%%%%%%%%%%%%%%%%%%%%%%%%%%%%%%%%%%%%%%%%%%%%%%%%%%%%%%%%%%%%%%%%
% \section{Sample}
%\iffalse
%<*samplemain>
%\fi
%
% The following presents a sample document
% with two chapters, two parts, a title page,
% a compile flag as well as three forwarding files to set the flag.
% It consists of eight |.tex| files:
% \begin{center}
% \begin{tabular}{ll}
% |cdocsamp.tex|&main file\\
% |cdocsch1.tex|&include file for chapter 1\\
% |cdocsch2.tex|&include file for chapter 2\\
% |cdocspt3.tex|&include file for part 3\\
% |cdocspt4.tex|&include file for part 4\\
% |cdocsdrf.tex|&forwarding file for main file in draft mode\\
% |cdocsfi1.tex|&forwarding file for final version of chapter 1\\
% |cdocsfi2.tex|&forwarding file for final version of chapter 2\\
% \end{tabular}
% \end{center}
% Each of the eight files can be compiled directly by the \LaTeX{} compiler.
%
% %%%%%%%%%%%%%%%%%%%%%%%%%%%%%%%%%%%%%%
% \paragraph{Main File.}
%
% The main file is called |cdocsamp.tex|.
%
% Load the \textsf{childdoc} definitions and
% declare the filename for the main document:
%    \begin{macrocode}
\input{childdoc.def}
\childdocmain{}
%    \end{macrocode}

% Optional override for |\version| flag:
%    \begin{macrocode}
%%\ifchilddoc\else\providecommand{\version}{draft}\fi
%    \end{macrocode}

% Define the default values for the |\version| flag
% (|final| for the main file and |draft| for childs):
%    \begin{macrocode}
\ifchilddoc
\providecommand{\version}{draft}
\else
\providecommand{\version}{final}
\fi
%    \end{macrocode}

% Load the standard document class:
%    \begin{macrocode}
\documentclass[12pt]{article}
%    \end{macrocode}

% Start the document body:
%    \begin{macrocode}
\begin{document}
%    \end{macrocode}

% Declare a title page.
% Print title, part of document being processed and version flag:
%    \begin{macrocode}
\addtocounter{page}{-1}
\begin{center}
{\LARGE\bfseries{}childdoc example\par}
\vspace{1cm}
\ifchilddoc
\ifchilddocmanual part\else chapter\fi:
`\childdocname' of `\childdocjob'\par
\else
main document: `\childdocjob'\par
\fi
version: \version\par
\end{center}
\newpage
%    \end{macrocode}

% Manually include selected file,
% otherwise process as usual:
%    \begin{macrocode}
\ifchilddocmanual
\section*{part `\childdocname'}
\input{\childdocname}
\else
%    \end{macrocode}

% Include the two chapters:
%    \begin{macrocode}
\include{cdocsch1}
\include{cdocsch2}
%    \end{macrocode}

% Include the two parts unless only chapters should be displayed:
%    \begin{macrocode}
\ifchilddoc\else
\section{part three}
\input{cdocspt3}
\section{part four}
\input{cdocspt4}
\fi
%    \end{macrocode}

% Process as usual until here:
%    \begin{macrocode}
\fi
%    \end{macrocode}

% End of document body:
%    \begin{macrocode}
\end{document}
%    \end{macrocode}
%\iffalse
%</samplemain>
%\fi
%
% %%%%%%%%%%%%%%%%%%%%%%%%%%%%%%%%%%%%%%
% \paragraph{Chapter Include Files.}
%
% The include files are called |cdocsch1.tex| and |cdocsch2.tex|.
%
%\iffalse
%<*samplechap1|samplechap2>
%\fi

% Optional override for |\version| flag:
%    \begin{macrocode}
%%\providecommand{\version}{final}
%    \end{macrocode}

% Include the main document:
%    \begin{macrocode}
\input{childdoc.def}
\childdocof{cdocsamp}
%    \end{macrocode}

%\iffalse
%</samplechap1|samplechap2>
%\fi
%
%\iffalse
%<*samplechap1>
%\fi
% Some text for chapter 1:
%    \begin{macrocode}
\section{one}
some text in chapter one
%    \end{macrocode}

%\iffalse
%</samplechap1>
%\fi
% Some text for chapter 2:
%\iffalse
%<*samplechap2>
%\fi
%    \begin{macrocode}
\section{two}
more text in chapter two
%    \end{macrocode}

%\iffalse
%</samplechap2>
%\fi
%
% %%%%%%%%%%%%%%%%%%%%%%%%%%%%%%%%%%%%%%
% \paragraph{Part Include Files.}
%
% The include files are called |cdocspt3.tex| and |cdocspt4.tex|.
%
%\iffalse
%<*samplepart3|samplepart4>
%\fi

% Optional override for |\version| flag:
%    \begin{macrocode}
%%\providecommand{\version}{final}
%    \end{macrocode}

% Include the main document:
%    \begin{macrocode}
\input{childdoc.def}
\childdocby{cdocsamp}
%    \end{macrocode}

%\iffalse
%</samplepart3|samplepart4>
%\fi
%
%\iffalse
%<*samplepart3>
%\fi
% Some text for part 3:
%    \begin{macrocode}
some text in part three
%    \end{macrocode}

%\iffalse
%</samplepart3>
%\fi
% Some text for part 4:
%\iffalse
%<*samplepart4>
%\fi
%    \begin{macrocode}
more text in part four
%    \end{macrocode}

%\iffalse
%</samplepart4>
%\fi
%
% %%%%%%%%%%%%%%%%%%%%%%%%%%%%%%%%%%%%%%
% \paragraph{Forwarding for a Complete Draft.}
%
% The following forwarding file |cdocsdrf.tex|
% compiles the main document in draft mode:
%\iffalse
%<*sampledraft>
%\fi
%    \begin{macrocode}
\def\version{draft}
\input{childdoc.def}
\childdocforward{cdocsamp}
%    \end{macrocode}

%\iffalse
%</sampledraft>
%\fi
%
% %%%%%%%%%%%%%%%%%%%%%%%%%%%%%%%%%%%%%%
% \paragraph{Forwarding for Final Version of the Chapters.}
%
% The following forwarding files |cdocsfn1.tex| and |cdocsfn2.tex|
% (with identical content)
% compile the final versions of the child documents
% |cdocsch1.tex| and |cdocsch2.tex|, respectively:
%\iffalse
%<*samplefinal>
%\fi
%    \begin{macrocode}
\def\version{final}
\input{childdoc.def}
\childdocforwardprefix[cdocsamp]{cdocsfn}{cdocsch}
%    \end{macrocode}

%\iffalse
%</samplefinal>
%\fi
%
% %%%%%%%%%%%%%%%%%%%%%%%%%%%%%%%%%%%%%%
% \paragraph{Command Line Processing.}
%
% The following three command lines generate the output files
% |cdocscld|, |cdocscl1| and |cdocscl2|
% which should be identical to
% |cdocsdrf|, |cdocsch1| and |cdocsfn2|, respectively:
% \begin{center}
% \begin{tabular}{l}
% |latex -jobname cdocscld \|\\
% |  "\def\version{draft}\input{childdoc.def}\childdocforward{cdocsamp}"|\\
% |latex -jobname cdocscl1 \|\\
% |  "\input{childdoc.def}\childdocforward[cdocsamp]{cdocsch1}"|\\
% |latex -jobname cdocscl2 \|\\
% |  "\def\version{final}\input{childdoc.def}\childdocforward{cdocsch2}"|
% \end{tabular}
% \end{center}
% Note that the trailing backslash on each first line
% merely continues the input to the second line
% (for convenient cut ant paste).
% Furthermore, the command |latex| can be replaced by any
% of its alternative versions such as |pdflatex|.
%
% %%%%%%%%%%%%%%%%%%%%%%%%%%%%%%%%%%%%%%%%%%%%%%%%%%%%%%%%%%%%%%%%%%%%%%%%%%%%%%
% %%%%%%%%%%%%%%%%%%%%%%%%%%%%%%%%%%%%%%%%%%%%%%%%%%%%%%%%%%%%%%%%%%%%%%%%%%%%%%
% \section{Implementation}
%\iffalse
%<*package>
%\fi
%
% This section describes the definitions file |childdoc.def|.

% The definitions cannot be loaded using |\usepackage| or |\RequirePackage|
% which has a mechanism to prevent loading a style file more than once.
% When loading the definitions by means of |\input|
% multiple instances have to be prevented manually:
%\iffalse
%This code needs to be before the `\ProvidesFile' directive
%which is defined at the beginning of this file.
%Therefore it is also placed there and commented out here.
%</package>
%<*discard>
%\fi
%    \begin{macrocode}
\ifdefined\childdocmain\endinput\fi
%    \end{macrocode}
%\iffalse
%</discard>
%<*package>
%\fi
%
% \macro{\ifchilddoc}
% \macro{\ifchilddocmanual}
% The conditional |\ifchilddoc| tells whether a
% child (true) or main (false) document is being compiled.
% The conditional |\ifchilddocmanual| tells whether
% the |\includeonly| mechanism is used (false) or
% the selection of child files must be performed manually (true).
% The definitions initialise to false:
%    \begin{macrocode}
\newif\ifchilddoc
\newif\ifchilddocmanual
%    \end{macrocode}

% \macro{\childdocname}
% \macro{\childdocjob}
% The macro |\childdocname| stores the name of the main document
% to be compiled. The macro |\childdocjob| stores the name of
% the document on which the \LaTeX{} compiler was originally invoked.
% The content of |\jobname| cannot be compared
% to filenames specified in the source due to different catcodes.
% The following code rescans |\jobname|, stores the result
% in |\childdocname| and saves a copy in |\childdocjob|:
%    \begin{macrocode}
\edef\childdocname{\scantokens\expandafter{\jobname\noexpand}}
\let\childdocjob\childdocname
%    \end{macrocode}

% \macro{\childdocdisable}
% The macro |\childdocdisable| prevents the main file
% from being processed more than once.
% At this stage, the main document command |\childdocmain|
% is assumed to be called once again where it should do nothing.
% Any subsequent call to it should prevent
% a secondary processing of the main document
% It overwrites the forwarding commands
% |\childdocof| and |\childdocforward|
% with empty macros to prevent further inclusions of the main document:
%    \begin{macrocode}
\newcommand{\childdocdisable}
{
  \renewcommand{\childdocmain}[1]{\renewcommand{\childdocmain}[1]{\endinput}}
  \renewcommand{\childdocof}[1]{}
  \renewcommand{\childdocby}[2][]{}
  \renewcommand{\childdocforward}[2][]{}
  \renewcommand{\childdocdisable}{}
}
%    \end{macrocode}

% \macro{\childdocmain}
% The macro |\childdocmain| is to be called at the top of the main file
% with nothing or the main filename (without extension) as argument.
% First, it breaks loops.
% If the argument is not empty and does not match |\childdocname|
% (which is set by the first inclusion of |childdoc.def|),
% |\ifchilddoc| is set to true, |\includeonly| is applied to the child file
% and |\jobname| is set to the main file
% (for proper handling of |.aux| files):
%    \begin{macrocode}
\newcommand{\childdocmain}[1]
{
  \childdocdisable\childdocmain{}
  \if?#1?\else
    \begingroup
      \def\childdoctmp{#1}
      \ifx\childdoctmp\childdocname
        \def\childdoctmp{}
      \else
        \def\childdoctmp
        {
          \childdoctrue
          \includeonly{\childdocname}
          \def\childdocjob{#1}
          \def\jobname{#1}
        }
      \fi
      \expandafter
    \endgroup
    \childdoctmp
  \fi
}
%    \end{macrocode}

% \macro{\childdocof}
% The command |\childdocof| redirects
% compilation to the main file |#1|.
%    \begin{macrocode}
\newcommand{\childdocof}[1]
{
  \childdocdisable
  \childdoctrue
  \includeonly{\childdocname}
  \def\jobname{#1}
  \def\childdocjob{#1}
  \input{#1}
}
%    \end{macrocode}

% \macro{\childdocby}
% The command |\childdocby| ....
%    \begin{macrocode}
\newcommand{\childdocby}[2][]
{
  \childdocdisable
  \childdoctrue
  \childdocmanualtrue
  \if?#1?\else
    \def\jobname{#2}
  \fi
  \def\childdocjob{#2}
  \input{#2}
  \endinput
}
%    \end{macrocode}

% \macro{\childdocforward}
% The command |\childdocforward| redirects
% compilation to the main file or
% (if the optional argument is given) a child file.
% Parameters are set as if the main file
% or a child file starting with |\childdocof| was compiled.
% Then compilation is handed over to the main file:
%    \begin{macrocode}
\newcommand{\childdocforward}[2][]
{
  \begingroup
    \if?#1?
      \def\childdoctmp
      {
        \def\childdocname{#2}
        \def\childdocjob{#2}
        \def\jobname{#2}
        \input{#2}
        \endinput
      }
    \else
      \def\childdoctmp
      {
        \childdocdisable
        \def\childdocname{#2}
        \childdoctrue
        \includeonly{#2}
        \def\childdocjob{#1}
        \def\jobname{#1}
        \input{#1}
        \endinput
      }
    \fi
    \expandafter
  \endgroup
  \childdoctmp
}
%    \end{macrocode}

% \macro{\childdocforwardprefix}
% The command |\childdocforwardprefix| redirects
% compilation to the main or a child file by means of a pattern.
% The prefix |#1| in the current filename is replaced by |#2|
% and the suffix of the current filename is kept
% (it is assumed that the filename does not contain the substring `|~~~|'
% which is used as a delimiter).
% Compilation is handed over to the new file by |\childdocforward|:
%    \begin{macrocode}
\newcommand{\childdocforwardprefix}[3][]
{
  \begingroup
    \def\childdocextract #2##1~~~{\def\childdoctmp{\childdocforward[#1]{#3##1}}}
    \expandafter\childdocextract\childdocname~~~
    \expandafter
  \endgroup
  \childdoctmp
}
%    \end{macrocode}

% \macro{\childdoc}
% The deprecated macro |\childdoc| is a legacy version of |\childdocmain|:
%    \begin{macrocode}
\newcommand{\childdoc}{\childdocmain}
%    \end{macrocode}

% \macro{\childdocredirect}
% The deprecated macro |\childdocredirect| is a legacy version
% of |\childdocforward| and |\childdocforwardprefix|:
%    \begin{macrocode}
\newcommand{\childdocredirect}[2][]
{
  \begingroup
    \if?#1?
      \def\childdoctmp{\childdocforward{#2}}
    \else
      \def\childdoctmp{\childdocforwardprefix{#1}{#2}}
    \fi
    \expandafter
  \endgroup
  \childdoctmp
}
%    \end{macrocode}

%\iffalse
%</package>
%\fi
%
\endinput
|\\
|\childdocmain{|\textit{main}|}|\\
\end{tabular}
\end{center}
%
If |\jobname| does not match the argument \textit{main} of |\childdocmain|,
it is assumed that |\jobname| points to the child file to be compiled.
When using |\childdocmain| with the main file specified as argument,
it suffices to start a child file
with just |\input{|\textit{main}|}|
without loading of the package and using |\childdocof|.
If instead all processing is done
with the appropriate \textsf{childdoc} directives,
the argument of \textit{main} of |\childdocmain| can be empty.

An alternative version of the command line processing described
in \secref{sec:commandline} using the detection mechanism reads:
%
\begin{center}
|... -jobname "|\textit{target}|" "|[\textit{flags}]%
[|\def\jobname{|\textit{dest}|}|]|\input{|\textit{main}|}"|
\end{center}

%%%%%%%%%%%%%%%%%%%%%%%%%%%%%%%%%%%%%%%%%%%%%%%%%%%%%%%%%%%%%%%%%%%%%%%%%%%%%%%%
\subsection{Manual Code}
\label{sec:manual}

In case one cannot be certain whether the definitions file |childdoc.def|
is installed on the target \TeX{} distribution
and one prefers not to ship it,
it is conceivable to paste a few relevant commands into the sources.

To that end, drop all statements |% \iffalse
%
% childdoc.dtx Copyright (C) 2017-2018 Niklas Beisert
%
% This work may be distributed and/or modified under the
% conditions of the LaTeX Project Public License, either version 1.3
% of this license or (at your option) any later version.
% The latest version of this license is in
%   http://www.latex-project.org/lppl.txt
% and version 1.3 or later is part of all distributions of LaTeX
% version 2005/12/01 or later.
%
% This work has the LPPL maintenance status `maintained'.
%
% The Current Maintainer of this work is Niklas Beisert.
%
% This work consists of the files childdoc.dtx and childdoc.ins
% and the derived files childdoc.def and cdocsamp.tex with
% cdocsch1.tex, cdocsch2.tex, cdocsdrf.tex, cdocsfn1.tex, cdocsfn2.tex.
%
%<package>\ifdefined\childdocmain\endinput\fi
%<package>\ProvidesFile{childdoc.def}[2018/12/30 v2.0 child document driver]
%<samplemain>\ProvidesFile{cdocsamp.tex}[2018/12/30 v2.0 sample for childdoc]
%<*driver>
%\ProvidesFile{childdoc.drv}[2018/12/30 v2.0 childdoc reference manual file]
\PassOptionsToClass{10pt,a4paper}{article}
\documentclass{ltxdoc}

\usepackage[margin=35mm]{geometry}
\usepackage{hyperref}
\usepackage{hyperxmp}
\usepackage[usenames]{color}

\hypersetup{colorlinks=true}
\hypersetup{pdfstartview=FitH}
\hypersetup{pdfpagemode=UseNone}
\hypersetup{pdfsource={}}
\hypersetup{pdflang={en-UK}}
\hypersetup{pdfcopyright={Copyright 2017-2018 Niklas Beisert.
  This work may be distributed and/or modified under the
  conditions of the LaTeX Project Public License, either version 1.3
  of this license or (at your option) any later version.}}
\hypersetup{pdflicenseurl={http://www.latex-project.org/lppl.txt}}
\hypersetup{pdfcontactaddress={ETH Zurich, ITP, HIT K,
  Wolfgang-Pauli-Strasse 27}}
\hypersetup{pdfcontactpostcode={8093}}
\hypersetup{pdfcontactcity={Zurich}}
\hypersetup{pdfcontactcountry={Switzerland}}
\hypersetup{pdfcontactemail={nbeisert@itp.phys.ethz.ch}}
\hypersetup{pdfcontacturl={http://people.phys.ethz.ch/\xmptilde nbeisert/}}

\newcommand{\secref}[1]{\hyperref[#1]{section \ref*{#1}}}

\parskip1ex
\parindent0pt
\let\olditemize\itemize
\def\itemize{\olditemize\parskip0pt}

\begin{document}

\title{The \textsf{childdoc} Package}
\hypersetup{pdftitle={The childdoc Package}}
\author{Niklas Beisert\\[2ex]
  Institut f\"ur Theoretische Physik\\
  Eidgen\"ossische Technische Hochschule Z\"urich\\
  Wolfgang-Pauli-Strasse 27, 8093 Z\"urich, Switzerland\\[1ex]
  \href{mailto:nbeisert@itp.phys.ethz.ch}
  {\texttt{nbeisert@itp.phys.ethz.ch}}}
\hypersetup{pdfauthor={Niklas Beisert}}
\hypersetup{pdfsubject={Manual for the LaTeX2e Package childdoc}}
\date{30 December 2018, \textsf{v2.0}}
\maketitle

\begin{abstract}\noindent
\textsf{childdoc} is a \LaTeXe{} package
that enables the direct compilation
of document sections included by |\include|
to individual files.
\end{abstract}

\begingroup
\parskip0ex
\tableofcontents
\endgroup

%%%%%%%%%%%%%%%%%%%%%%%%%%%%%%%%%%%%%%%%%%%%%%%%%%%%%%%%%%%%%%%%%%%%%%%%%%%%%%%%
%%%%%%%%%%%%%%%%%%%%%%%%%%%%%%%%%%%%%%%%%%%%%%%%%%%%%%%%%%%%%%%%%%%%%%%%%%%%%%%%
\section{Introduction}

\LaTeX{} provides a mechanism to structure a large document (such as a book)
into a main file and several child files (containing the chapters)
using the |\include| command.
This mechanism is beneficial for documents
which span hundreds of pages in order to
make the source file(s) more manageable.
Moreover, compilation can be restricted to
selected child files by means of the |\includeonly| command.
The latter feature can be used to reduce the compilation time while editing
(this was significantly more useful in the earlier days of \LaTeX{})
or to generate a smaller document which is easier to navigate.
Another application of |\includeonly| is to generate
documents consisting of selected parts of the complete document.

However, there are a few drawbacks of the plain |\include| mechanism:
\begin{itemize}
\item
The child files cannot be compiled on their own,
they can only be compiled via the main file.
A naive editing environment
(such as a text editor with an option
to have the current file processed by \LaTeX)
may require one to switch to the main file before compiling;
attempting to compile the child file produces errors.
\item
The main file must be modified (each time)
to adjust the |\includeonly| command
to the present needs. This easily leaves the main file in a messy state.
\item
The generated document will always carry the filename
of the main document. This is inconvenient if
several child files are to be compiled and
to be kept for distribution.
\end{itemize}

The present package provides a simple interface
to make child files individually compilable by \LaTeX{}.
Compiling a child file then has the same effect as compiling
the main file with an |\includeonly| command
to select the appropriate child.
Moreover the generated document will carry the name of the child
rather than the main file.
This resolves all three above issues.

This feature is meant to make the editing of books,
thesis documents and lecture notes somewhat more convenient.
However, the package can also be used efficiently for
composing a series of documents (such as exercise sheets)
which are typically distributed individually.
It then assists the author in generating the individual documents
(potentially in different versions)
as well as a document containing the collected series.
Another application is in developing style files
or other kinds of included material
where compilation of the style file could redirect
to a sample or test file.

%%%%%%%%%%%%%%%%%%%%%%%%%%%%%%%%%%%%%%%%%%%%%%%%%%%%%%%%%%%%%%%%%%%%%%%%%%%%%%%%
%%%%%%%%%%%%%%%%%%%%%%%%%%%%%%%%%%%%%%%%%%%%%%%%%%%%%%%%%%%%%%%%%%%%%%%%%%%%%%%%
\section{Usage}

First of all, the package \textsf{childdoc} is \emph{not} a standard
\LaTeXe{} |.sty| style file! Therefore it needs to be invoked in
a non-standard way.

%%%%%%%%%%%%%%%%%%%%%%%%%%%%%%%%%%%%%%%%%%%%%%%%%%%%%%%%%%%%%%%%%%%%%%%%%%%%%%%%
\subsection{Included Files}
\label{sec:include}

%%%%%%%%%%%%%%%%%%%%%%%%%%%%%%%%%%%%%%%%
\DescribeMacro{\childdocmain}
To use the package, add the commands
\begin{center}
\begin{tabular}{l}
|\input{childdoc.def}|\\
|\childdocmain{}|\\
\end{tabular}
\end{center}
at the very top of the main \LaTeX{} file,
in particular \emph{before} the |\documentclass| statement!
The argument of |\childdocmain| should be left empty
(but it must be present).

%%%%%%%%%%%%%%%%%%%%%%%%%%%%%%%%%%%%%%%%
\DescribeMacro{\childdocof}
Furthermore, add the commands
\begin{center}
\begin{tabular}{l}
|\input{childdoc.def}|\\
|\childdocof{|\textit{main}|}|\\
\end{tabular}
\end{center}
at the top of every child file \textit{child}
which is included by |\include{|\textit{child}|}|
from within the main file
(or at least for those files to be compiled individually).
The argument \textit{main} must be the filename of the main file.

There are a couple of
considerations in setting up the main and child documents:

%%%%%%%%%%%%%%%%%%%%%%%%%%%%%%%%%%%%%%%%
\paragraph{Restrictions.}

Please note the following restrictions:
\begin{itemize}
\item
|\childdocmain| must be called with one argument \textit{main}
to ensure compatibility with earlier version of the package.
It must either be empty (|\childdocmain{}|)
or precisely match the filename of the main file in which it is specified.
See \secref{sec:detection} for further information.
\item
The filename \textit{main} must be specified without the |.tex| extension.
\item
The filename \textit{main} is case sensitive
(even in case-insensitive file systems)
due to internal string comparison.
\item
The argument \textit{main} should be fully expanded, it cannot be a macro.
\item
Subdirectories and special characters should be avoided in filenames.
\item
The command |\childdocmain{|\textit{main}|}| must be followed by a whitespace.
It should not be followed immediately by another command
or by a comment mark `|%|'.
This is because the \TeX{} parser reads the token immediately following
the argument of |\childdocmain| and puts it
at the beginning of every child section;
however, a white\-space is ignored.
\end{itemize}

%%%%%%%%%%%%%%%%%%%%%%%%%%%%%%%%%%%%%%%%
\paragraph{Content of Main File.}

It is advisable to place all content in the child files included by |\include|.
Any output contained in the main file will appear in all child documents
unless suppressed manually;
it cannot be suppressed automatically by the |\includeonly| directive
and thus should normally be avoided.
A method to include some content in the main file
by means of conditional processing is described in \secref{sec:conditional}.

%%%%%%%%%%%%%%%%%%%%%%%%%%%%%%%%%%%%%%%%
\paragraph{Page Numbering.}

When only a part of the document is compiled,
the appropriate numbering of pages
(as well as other status parameters)
is determined from the |.aux| files.
The latter contain information from previous passes.
However this information needs to propagate through
all intermediate child documents.
Therefore the page numbering in child documents may well
be inconsistent until the complete document is compiled at least once.

A useful (if unconventional) way to always ensure a consistent
page numbering is to restart the numbering in each child document
and denote the pages by `\textit{child}|.|\textit{page}'
where \textit{child} represents the chapter/section number of the child file.
This can be achieved by the command
|\numberwithin{page}{|\textit{child}|}|
of the \textsf{amsmath} package
where \textit{child} can be |chapter| or |section|
depending on the chosen structuring.
Alternatively, one can modify the macro |\thepage| appropriately
and reset the counter |page| at the start of each child file.

%%%%%%%%%%%%%%%%%%%%%%%%%%%%%%%%%%%%%%%%%%%%%%%%%%%%%%%%%%%%%%%%%%%%%%%%%%%%%%%%
\subsection{Conditional Processing}
\label{sec:conditional}

The package provides a mechanism to compile different versions
of a document. To customise the versions further some conditional processing
can come in handy to distinguish which version is being compiled.
The package provides two macros to describe the compilation context:

%%%%%%%%%%%%%%%%%%%%%%%%%%%%%%%%%%%%%%%%
\DescribeMacro{\ifchilddoc}
The conditional |\ifchilddoc| distinguishes between the compilation of
child documents and the main document:
%
\begin{center}
|\ifchilddoc |\textit{child-code}| |[|\||else |\textit{main-code}]| \||fi|
\end{center}

%%%%%%%%%%%%%%%%%%%%%%%%%%%%%%%%%%%%%%%%
\DescribeMacro{\childdocname}
\DescribeMacro{\childdocjob}
The macro |\childdocname| contains the filename (without extension)
of the main or child file being processed.
Note that |\childdocjob| will always contain the name of the main file.

%%%%%%%%%%%%%%%%%%%%%%%%%%%%%%%%%%%%%%%%
\paragraph{Title Page.}

Conditional processing can be used to include a title or banner page
in the main document when proper precautions are taken.
Importantly, the code in the main file should ensure that the page counter
(as well as other status parameters which are stored in the |.aux| files)
takes the same value after the conditional processing.
Otherwise the page numbers may take divergent values
depending on which part is compiled.

For example, a title page could be declared by:
%
\begin{center}
\begin{tabular}{l}
|\ifchilddoc\||else|\\
|\addtocounter{page}{-1}|\\
\textit{code for title page}\\
|\newpage|\\
|\||fi|
\end{tabular}
\end{center}
%
A banner page for the child documents can be generated by:
%
\begin{center}
\begin{tabular}{l}
|\ifchilddoc|\\
|\addtocounter{page}{-1}|\\
\textit{code for banner page}\\
|\newpage|\\
|\||fi|
\end{tabular}
\end{center}
%
Here one could write a message such as:
\begin{center}
|This is the part \childdocname{} of \childdocjob{}.|
\end{center}

%%%%%%%%%%%%%%%%%%%%%%%%%%%%%%%%%%%%%%%%%%%%%%%%%%%%%%%%%%%%%%%%%%%%%%%%%%%%%%%%
\subsection{Flags}
\label{sec:flags}

The package makes it easy to generate different versions
of the main or child documents.
To this end compilation flags can be defined
and assigned different default values.
They will be particularly useful in conjunction
with the forwarding mechanism described in \secref{sec:forward}.

For example, it may be useful to have a flag |\version|
which can be set to |draft| or |final|.
The document source will contain some conditional code
depending on the value of |\version|.
Suppose further, the flag should default to |final| for the main file
and to |draft| for child files
which is a natural assignment for editing the document.
This is achieved by placing the following code
in the preamble of the main document
(below the |\childdocmain| directive):
%
\begin{center}
\begin{tabular}{l}
|\ifchilddoc|\\
|\providecommand{\version}{draft}|\\
|\||else|\\
|\providecommand{\version}{final}|\\
|\||fi|
\end{tabular}
\end{center}
%
The definition by |\providecommand| makes sure
that previous definitions are not overwritten.
Further statements |\providecommand{\version}{...}|
can thus be added before the above code to override it.

For the main file, one might add a line
(between |\childdocmain| and the above block)
%
\begin{center}
|%\ifchilddoc\||else\providecommand{\version}{draft}\||fi|
\end{center}
%
which can be uncommented to produce a draft version.
Likewise one can add a line to the very top of a child file
(above the |\childdocof{|\textit{main}|}| directive)
%
\begin{center}
|%\providecommand{\version}{final}|
\end{center}
%
which can be uncommented to produce the final version of this child document.

%%%%%%%%%%%%%%%%%%%%%%%%%%%%%%%%%%%%%%%%%%%%%%%%%%%%%%%%%%%%%%%%%%%%%%%%%%%%%%%%
\subsection{Forwarding}
\label{sec:forward}

Different versions of the main or child documents
using compilation flags as described in \secref{sec:flags}
can be (permanently) stored in different files
for convenient compilation, viewing and distribution.
To this end, the package defines a command
to pass on compilation to a different file:

%%%%%%%%%%%%%%%%%%%%%%%%%%%%%%%%%%%%%%%%
\DescribeMacro{\childdocforward}
The command |\childdocforward| redirects processing to
another source file:
%
\begin{center}
\begin{tabular}{l}
|\input{childdoc.def}|\\
|\childdocforward[|\textit{main}|]{|\textit{dest}|}|\\
\end{tabular}
\end{center}
%
The argument \textit{dest} is the destination file
(without extension).
It should be the main file or one of the child files.
Note that further \textsf{childdoc} directives
such as |\childdocof| and |\childdocforward|
in the indicated file will be processed in this form.
The optional argument \textit{main}
passes on directly to the main file \textit{main}
while pretending to compile the child \textit{dest}.
This form behaves as if \textit{dest}
issues |\childdocof{|\textit{main}|}| right away,
and no further \textsf{childdoc} directives will be processed.

%%%%%%%%%%%%%%%%%%%%%%%%%%%%%%%%%%%%%%%%
\DescribeMacro{\...prefix}
In the alternative form |\childdocforwardprefix|,
%
\begin{center}
\begin{tabular}{l}
|\input{childdoc.def}|\\
|\childdocforwardprefix[|\textit{main}|]{|\textit{prefix}|}{|\textit{dest}|}|
\end{tabular}
\end{center}
%
the destination file is determined by a pattern
depending on the current file:
To make this work, the current file must be called
`{\textit{prefix}\hspace{0.2em}\textit{suffix}}'
with \textit{prefix} matching precisely the argument.
Processing is then passed on to the file
`{\textit{dest}\hspace{0.2em}\textit{suffix}}'.
Surely, the same effect is achieved by
directly specifying the
argument `{\textit{dest}\hspace{0.2em}\textit{suffix}}'
in the first form.
However, that requires to set up a different file
for each child. With the alternative form of the command
all these files can have exactly the same content
which simplifies setting them up and maintaining them.

For example, the following file |draft.tex|
with a compilation flag |\version| as described in \secref{sec:flags}
compiles the main document as a draft:
%
\begin{center}
\begin{tabular}{l}
|\def\version{draft}|\\
|\input{childdoc.def}|\\
|\childdocforward{|\textit{main}|}|
\end{tabular}
\end{center}
%
Likewise, the following files |final|\textit{nn}|.tex|
compile the final version of the child document
|child|\textit{nn}|.tex|:
%
\begin{center}
\begin{tabular}{l}
|\def\version{final}|\\
|\input{childdoc.def}|\\
|\childdocforwardprefix{final}{child}|
\end{tabular}
\end{center}
%

Note that when several versions of a main file and/or of each child file
are to be generated, it may be convenient to set up a |Makefile| or
shell script to automatise the process.

%%%%%%%%%%%%%%%%%%%%%%%%%%%%%%%%%%%%%%%%%%%%%%%%%%%%%%%%%%%%%%%%%%%%%%%%%%%%%%%%
\subsection{Command Line Processing}
\label{sec:commandline}

The effect of redirection files can also be achieved by invoking
the \LaTeX{} compiler with a more elaborate command line.
Most conveniently this should be done as part
of a shell script or a |Makefile|.

When using \textsf{childdoc} in the main file, the following
command lines effectively perform a redirection
(note that depending on the shell being used,
backslashes may have to be doubled: `|\|' $\to$ `|\\|'):
%
\begin{center}
|... -jobname "|\textit{target}|" |\\|"|[\textit{flags}]%
|\input{childdoc.def}\childdocforward[|\textit{main}|]{|\textit{dest}|}"|
\end{center}
%
Here \textit{target} is the name of the output file,
\textit{main} is the name of the main file
and \textit{dest} is the name of the main or child file to be processed
(all filenames without extensions).
The optional argument \textit{main} can be omitted
if \textit{main} matches \textit{dest}.
Optionally, compilation \textit{flags} can be defined via |\def| commands.
This command line makes the \TeX{} engine believe
it is compiling the file \textit{target}
whose content is specified as the latter parameter.
The provided code then forwards the processing to
\textit{main} or \textit{dest} as described in \secref{sec:forward}.

%%%%%%%%%%%%%%%%%%%%%%%%%%%%%%%%%%%%%%%%%%%%%%%%%%%%%%%%%%%%%%%%%%%%%%%%%%%%%%%%
\subsection{Include by Input}
\label{sec:input}

Including child documents by |\include| has some restrictions by design.
Most notably, the content of a child document always occupies
its own set of pages; pages cannot be shared between child documents.
Usually, this behaviour makes perfect sense
because each child document contain an essential part of the document.
However, in some situations it may be desirable to compose
a document from a collection of parts
without having mandatory page breaks between then.
For this case, the package
provides a mechanism to include parts
by |\input| which can also be processed individually.
However, by construction this mechanism
requires manual handling of the content to be output.

%%%%%%%%%%%%%%%%%%%%%%%%%%%%%%%%%%%%%%%%
\DescribeMacro{\ifchilddocmanual}
The main file should be prepared as usual, see \secref{sec:include}.
However, the document body must make a distinction
between processing of an individual part and of the main document, e.g.:
%
\begin{center}
\begin{tabular}{l}
|\ifchilddocmanual|\\
|\input{\childdocname}|\\
|\||else|\\
\textit{document body with }|\input{|\textit{part}|}|\\
|\||fi|
\end{tabular}
\end{center}
%
The conditional |\ifchilddocmanual| is true whenever
a part to be included by |\input| is being compiled,
and the name of the part is stored in |\childdocname|.

%%%%%%%%%%%%%%%%%%%%%%%%%%%%%%%%%%%%%%%%
\DescribeMacro{\childdocby}
Each part to be included by |\input| should start with:
%
\begin{center}
\begin{tabular}{l}
|\input{childdoc.def}|\\
|\childdocby{|\textit{main}|}|\\
\end{tabular}
\end{center}
%
The directive |\childdocby| is similar to |\childdocof|
described in \secref{sec:include},
but the subsequent selection of content must be done manually.
To that end, both |\ifchilddoc| and |\ifchilddocmanual|
will be true upon processing of a part,
and the name of the part is stored in |\childdocname|.
Note that |\jobname| will be set to the filename of the current part
so that each part receives an individual |.aux| file
that does not interfere with the |.aux| file(s) of the main document.
This behaviour can be altered by the alternative form
|\childdocby[*]{|\textit{main}|}| (with a non-empty optional argument)
which uses the |.aux| file of the main document
by setting |\jobname| to \textit{main}.

%%%%%%%%%%%%%%%%%%%%%%%%%%%%%%%%%%%%%%%%%%%%%%%%%%%%%%%%%%%%%%%%%%%%%%%%%%%%%%%%
\subsection{Driver Development}
\label{sec:driver}

The \textsf{childdoc} mechanism can also be use for the development
of definition files such as \LaTeX{} styles or classes.
This case differs from the above setup with multiple parts
included by |\include| in that no |\includeonly| should be invoked.
This can be achieved by starting the include file
(before |\ProvidesPackage|) with:
%
\begin{center}
\begin{tabular}{l}
|\input{childdoc.def}|\\
|\childdocforward{|\textit{main}|}|\\
\end{tabular}
\end{center}
%
or alternatively with:
%
\begin{center}
\begin{tabular}{l}
|\input{childdoc.def}|\\
|\childdocby{|\textit{main}|}|\\
\end{tabular}
\end{center}
%
Both forms have slightly different effects as described above.
The main file is prepared as usual, see \secref{sec:include}.

%%%%%%%%%%%%%%%%%%%%%%%%%%%%%%%%%%%%%%%%%%%%%%%%%%%%%%%%%%%%%%%%%%%%%%%%%%%%%%%%
\subsection{Legacy Detection}
\label{sec:detection}

The directive |\childdocmain| in the main file can detect
whether the complete document or merely a child is to be compiled
even without using the directive |\childdocof|.
This method is deprecated because it is less robust
and there is no compelling reason to use it;
it is merely provided for backward compatibility
and it may be removed in future versions.

If the detection mechanism is to be used,
it is mandatory to correctly specify
the filename of the main file as the argument of |\childdocmain|:
%
\begin{center}
\begin{tabular}{l}
|\input{childdoc.def}|\\
|\childdocmain{|\textit{main}|}|\\
\end{tabular}
\end{center}
%
If |\jobname| does not match the argument \textit{main} of |\childdocmain|,
it is assumed that |\jobname| points to the child file to be compiled.
When using |\childdocmain| with the main file specified as argument,
it suffices to start a child file
with just |\input{|\textit{main}|}|
without loading of the package and using |\childdocof|.
If instead all processing is done
with the appropriate \textsf{childdoc} directives,
the argument of \textit{main} of |\childdocmain| can be empty.

An alternative version of the command line processing described
in \secref{sec:commandline} using the detection mechanism reads:
%
\begin{center}
|... -jobname "|\textit{target}|" "|[\textit{flags}]%
[|\def\jobname{|\textit{dest}|}|]|\input{|\textit{main}|}"|
\end{center}

%%%%%%%%%%%%%%%%%%%%%%%%%%%%%%%%%%%%%%%%%%%%%%%%%%%%%%%%%%%%%%%%%%%%%%%%%%%%%%%%
\subsection{Manual Code}
\label{sec:manual}

In case one cannot be certain whether the definitions file |childdoc.def|
is installed on the target \TeX{} distribution
and one prefers not to ship it,
it is conceivable to paste a few relevant commands into the sources.

To that end, drop all statements |\input{childdoc.def}|
and perform the replacements as outlined below.
Instead of |\childdocmain{|\textit{main}|}| add the following code
to the top of the main file:
%
\begin{center}
\begin{tabular}{l}
|\||ifdefined\childdocname\endinput\||fi\newif\ifchilddoc|\\
|\edef\childdocname{\scantokens\expandafter{\jobname\noexpand}}|\\
|\def\childdocmain{|\textit{main}|}\||ifx\childdocmain\childdocname\||else|\\
|\childdoctrue\includeonly{\childdocname}\let\jobname\childdocmain\||fi|\\
\end{tabular}
\end{center}
%
Instead of |\childdocof{|\textit{main}|}| just include the main file
at the top of each child file:
%
\begin{center}
|\input{|\textit{main}|}|
\end{center}
%
A simple redirection |\childdocforward{|\textit{dest}|}| is achieved by:
%
\begin{center}
|\def\jobname{|\textit{dest}|}\input{\jobname}|
\end{center}
%
The redirection with prefix
|\childdocforwardprefix[|\textit{prefix}|]{|\textit{dest}|}|
is accomplished by:
%
\begin{center}
\begin{tabular}{l}
|{\edef\jobname{\scantokens\expandafter{\jobname\noexpand}}|\\
|\def\redirectjob |\textit{prefix}|#1~~~{\gdef\jobname{|\textit{dest}|#1}}|\\
|\expandafter\redirectjob\jobname~~~}\input{\jobname}|
\end{tabular}
\end{center}

In an alternative approach,
child documents can be compiled by a specific command line
without additional code or specific definitions:
%
\begin{center}
|... -jobname "|\textit{target}|" "|[\textit{flags}]%
|\includeonly{|\textit{dest}|}\input{|\textit{main}|}"|
\end{center}
%

%%%%%%%%%%%%%%%%%%%%%%%%%%%%%%%%%%%%%%%%%%%%%%%%%%%%%%%%%%%%%%%%%%%%%%%%%%%%%%%%
%%%%%%%%%%%%%%%%%%%%%%%%%%%%%%%%%%%%%%%%%%%%%%%%%%%%%%%%%%%%%%%%%%%%%%%%%%%%%%%%
\section{Information}

%%%%%%%%%%%%%%%%%%%%%%%%%%%%%%%%%%%%%%%%%%%%%%%%%%%%%%%%%%%%%%%%%%%%%%%%%%%%%%%%
\subsection{Copyright}

Copyright \copyright{} 2017--2018 Niklas Beisert

This work may be distributed and/or modified under the
conditions of the \LaTeX{} Project Public License, either version 1.3
of this license or (at your option) any later version.
The latest version of this license is in
  \url{http://www.latex-project.org/lppl.txt}
and version 1.3 or later is part of all distributions of \LaTeX{}
version 2005/12/01 or later.

This work has the LPPL maintenance status `maintained'.

The Current Maintainer of this work is Niklas Beisert.

This work consists of the files |README.txt|, |childdoc.ins| and |childdoc.dtx|
as well as the derived files |childdoc.def|, |cdocsamp.tex|
with |cdocsch1.tex|, |cdocsch2.tex|, |cdocspt3.tex|, |cdocspt4.tex|,
|cdocsdrf.tex|, |cdocsfn1.tex|, |cdocsfn2.tex|
as well as |childdoc.pdf|.

%%%%%%%%%%%%%%%%%%%%%%%%%%%%%%%%%%%%%%%%%%%%%%%%%%%%%%%%%%%%%%%%%%%%%%%%%%%%%%%%
\subsection{Files and Installation}

The package consists of the files:
%
\begin{center}
\begin{tabular}{ll}
    |README.txt|   & readme file \\
    |childdoc.ins| & installation file \\
    |childdoc.dtx| & source file \\
    |childdoc.def| & definition file \\
    |cdocsamp.tex| & sample main file \\
    |cdocsch1.tex| & sample include file \\
    |cdocsch2.tex| & sample include file \\
    |cdocspt3.tex| & sample part file \\
    |cdocspt4.tex| & sample part file \\
    |cdocsdrf.tex| & sample redirection file \\
    |cdocsfn1.tex| & sample redirection file \\
    |cdocsfn2.tex| & sample redirection file \\
    |childdoc.pdf| & manual
\end{tabular}
\end{center}
%
The distribution consists of the files
|README.txt|, |childdoc.ins| and |childdoc.dtx|.
%
\begin{itemize}
\item
Run (pdf)\LaTeX{} on |childdoc.dtx|
to compile the manual |childdoc.pdf| (this file).
\item
Run \LaTeX{} on |childdoc.ins| to create the definitions file |childdoc.def|
and the sample |cdocsamp.tex| with include files
|cdocsch1.tex|, |cdocsch2.tex|, |cdocspt3.tex|, |cdocspt4.tex|,
|cdocsdrf.tex|, |cdocsfn1.tex|, |cdocsfn2.tex|.
Then copy the file |childdoc.def| to an appropriate directory of your \LaTeX{}
distribution, e.g.\ \textit{texmf-root}|/tex/latex/childdoc|.
\end{itemize}

%%%%%%%%%%%%%%%%%%%%%%%%%%%%%%%%%%%%%%%%%%%%%%%%%%%%%%%%%%%%%%%%%%%%%%%%%%%%%%%%
\subsection{Related CTAN Packages}

There are several other packages which offer a similar functionality:
%
\begin{itemize}
\item
The packages
\href{http://ctan.org/pkg/docmute}{\textsf{docmute}},
\href{http://ctan.org/pkg/includex}{\textsf{includex}} and
\href{http://ctan.org/pkg/standalone}{\textsf{standalone}}
provide commands to include only the document body of
a child file thus allowing both files to be compiled individually.
\item
The packages \href{http://ctan.org/pkg/subdocs}{\textsf{subdocs}}
and \href{http://ctan.org/pkg/subfiles}{\textsf{subfiles}}
provide structures in which the main and child documents can be
encapsulated and allowing them to be compiled individually.
The inclusion mechanism is different from the conventional |\include|.
\item
The package \href{http://ctan.org/pkg/combine}{\textsf{combine}}
is an elaborate solution to combine several documents into one.
\end{itemize}
%
See also the CTAN topic \href{http://ctan.org/topic/subdocs}{\textsf{subdocs}}
for further related packages.
The present package differs from the above solutions in that
a document structure constructed with the conventional |\include| mechanism
just needs two extra commands at the top of every file
such that all constituent files can be compiled individually.

%%%%%%%%%%%%%%%%%%%%%%%%%%%%%%%%%%%%%%%%%%%%%%%%%%%%%%%%%%%%%%%%%%%%%%%%%%%%%%%%
%\subsection{Feature Suggestions}
%
%The following is a list of features which may be useful for future
%versions of this package:
%%
%\begin{itemize}
%\item
%\ldots
%\end{itemize}

%%%%%%%%%%%%%%%%%%%%%%%%%%%%%%%%%%%%%%%%%%%%%%%%%%%%%%%%%%%%%%%%%%%%%%%%%%%%%%%%
\subsection{Revision History}

%%%%%%%%%%%%%%%%%%%%%%%%%%%%%%%%%%%%%%%%
\paragraph{v2.0:} 2018/12/30

\begin{itemize}
\item
immediate forward processing
\item
added |\childdocby| mechanism
\item
manual restructured
\end{itemize}

%%%%%%%%%%%%%%%%%%%%%%%%%%%%%%%%%%%%%%%%
\paragraph{v1.6:} 2018/01/17

\begin{itemize}
\item
application for development of include files
\item
corrections to manual
\end{itemize}

%%%%%%%%%%%%%%%%%%%%%%%%%%%%%%%%%%%%%%%%
\paragraph{v1.5:} 2017/05/21

\begin{itemize}
\item
more complete structuring introduced
\item
|\childdocof| introduced
\item
|\childdoc| renamed to |\childdocmain|
\item
|\childredirect| renamed to |\childdocforward| and |\childdocforwardprefix|
and functionality expanded
\end{itemize}

%%%%%%%%%%%%%%%%%%%%%%%%%%%%%%%%%%%%%%%%
\paragraph{v1.0:} 2017/04/27

\begin{itemize}
\item
manual and install package
\item
first version published on CTAN
\end{itemize}

%%%%%%%%%%%%%%%%%%%%%%%%%%%%%%%%%%%%%%%%
\paragraph{v0.6:} 2017/04/26

\begin{itemize}
\item
redirection mechanism added
\end{itemize}

%%%%%%%%%%%%%%%%%%%%%%%%%%%%%%%%%%%%%%%%
\paragraph{v0.5:} 2017/04/26

\begin{itemize}
\item
functionality in definition file
\end{itemize}


%%%%%%%%%%%%%%%%%%%%%%%%%%%%%%%%%%%%%%%%%%%%%%%%%%%%%%%%%%%%%%%%%%%%%%%%%%%%%%%%
%%%%%%%%%%%%%%%%%%%%%%%%%%%%%%%%%%%%%%%%%%%%%%%%%%%%%%%%%%%%%%%%%%%%%%%%%%%%%%%%
%%%%%%%%%%%%%%%%%%%%%%%%%%%%%%%%%%%%%%%%%%%%%%%%%%%%%%%%%%%%%%%%%%%%%%%%%%%%%%%%
\appendix

\settowidth\MacroIndent{\rmfamily\scriptsize 000\ }

 \DocInput{childdoc.dtx}

\end{document}
%</driver>
% \fi
%
% %%%%%%%%%%%%%%%%%%%%%%%%%%%%%%%%%%%%%%%%%%%%%%%%%%%%%%%%%%%%%%%%%%%%%%%%%%%%%%
% %%%%%%%%%%%%%%%%%%%%%%%%%%%%%%%%%%%%%%%%%%%%%%%%%%%%%%%%%%%%%%%%%%%%%%%%%%%%%%
% \section{Sample}
%\iffalse
%<*samplemain>
%\fi
%
% The following presents a sample document
% with two chapters, two parts, a title page,
% a compile flag as well as three forwarding files to set the flag.
% It consists of eight |.tex| files:
% \begin{center}
% \begin{tabular}{ll}
% |cdocsamp.tex|&main file\\
% |cdocsch1.tex|&include file for chapter 1\\
% |cdocsch2.tex|&include file for chapter 2\\
% |cdocspt3.tex|&include file for part 3\\
% |cdocspt4.tex|&include file for part 4\\
% |cdocsdrf.tex|&forwarding file for main file in draft mode\\
% |cdocsfi1.tex|&forwarding file for final version of chapter 1\\
% |cdocsfi2.tex|&forwarding file for final version of chapter 2\\
% \end{tabular}
% \end{center}
% Each of the eight files can be compiled directly by the \LaTeX{} compiler.
%
% %%%%%%%%%%%%%%%%%%%%%%%%%%%%%%%%%%%%%%
% \paragraph{Main File.}
%
% The main file is called |cdocsamp.tex|.
%
% Load the \textsf{childdoc} definitions and
% declare the filename for the main document:
%    \begin{macrocode}
\input{childdoc.def}
\childdocmain{}
%    \end{macrocode}

% Optional override for |\version| flag:
%    \begin{macrocode}
%%\ifchilddoc\else\providecommand{\version}{draft}\fi
%    \end{macrocode}

% Define the default values for the |\version| flag
% (|final| for the main file and |draft| for childs):
%    \begin{macrocode}
\ifchilddoc
\providecommand{\version}{draft}
\else
\providecommand{\version}{final}
\fi
%    \end{macrocode}

% Load the standard document class:
%    \begin{macrocode}
\documentclass[12pt]{article}
%    \end{macrocode}

% Start the document body:
%    \begin{macrocode}
\begin{document}
%    \end{macrocode}

% Declare a title page.
% Print title, part of document being processed and version flag:
%    \begin{macrocode}
\addtocounter{page}{-1}
\begin{center}
{\LARGE\bfseries{}childdoc example\par}
\vspace{1cm}
\ifchilddoc
\ifchilddocmanual part\else chapter\fi:
`\childdocname' of `\childdocjob'\par
\else
main document: `\childdocjob'\par
\fi
version: \version\par
\end{center}
\newpage
%    \end{macrocode}

% Manually include selected file,
% otherwise process as usual:
%    \begin{macrocode}
\ifchilddocmanual
\section*{part `\childdocname'}
\input{\childdocname}
\else
%    \end{macrocode}

% Include the two chapters:
%    \begin{macrocode}
\include{cdocsch1}
\include{cdocsch2}
%    \end{macrocode}

% Include the two parts unless only chapters should be displayed:
%    \begin{macrocode}
\ifchilddoc\else
\section{part three}
\input{cdocspt3}
\section{part four}
\input{cdocspt4}
\fi
%    \end{macrocode}

% Process as usual until here:
%    \begin{macrocode}
\fi
%    \end{macrocode}

% End of document body:
%    \begin{macrocode}
\end{document}
%    \end{macrocode}
%\iffalse
%</samplemain>
%\fi
%
% %%%%%%%%%%%%%%%%%%%%%%%%%%%%%%%%%%%%%%
% \paragraph{Chapter Include Files.}
%
% The include files are called |cdocsch1.tex| and |cdocsch2.tex|.
%
%\iffalse
%<*samplechap1|samplechap2>
%\fi

% Optional override for |\version| flag:
%    \begin{macrocode}
%%\providecommand{\version}{final}
%    \end{macrocode}

% Include the main document:
%    \begin{macrocode}
\input{childdoc.def}
\childdocof{cdocsamp}
%    \end{macrocode}

%\iffalse
%</samplechap1|samplechap2>
%\fi
%
%\iffalse
%<*samplechap1>
%\fi
% Some text for chapter 1:
%    \begin{macrocode}
\section{one}
some text in chapter one
%    \end{macrocode}

%\iffalse
%</samplechap1>
%\fi
% Some text for chapter 2:
%\iffalse
%<*samplechap2>
%\fi
%    \begin{macrocode}
\section{two}
more text in chapter two
%    \end{macrocode}

%\iffalse
%</samplechap2>
%\fi
%
% %%%%%%%%%%%%%%%%%%%%%%%%%%%%%%%%%%%%%%
% \paragraph{Part Include Files.}
%
% The include files are called |cdocspt3.tex| and |cdocspt4.tex|.
%
%\iffalse
%<*samplepart3|samplepart4>
%\fi

% Optional override for |\version| flag:
%    \begin{macrocode}
%%\providecommand{\version}{final}
%    \end{macrocode}

% Include the main document:
%    \begin{macrocode}
\input{childdoc.def}
\childdocby{cdocsamp}
%    \end{macrocode}

%\iffalse
%</samplepart3|samplepart4>
%\fi
%
%\iffalse
%<*samplepart3>
%\fi
% Some text for part 3:
%    \begin{macrocode}
some text in part three
%    \end{macrocode}

%\iffalse
%</samplepart3>
%\fi
% Some text for part 4:
%\iffalse
%<*samplepart4>
%\fi
%    \begin{macrocode}
more text in part four
%    \end{macrocode}

%\iffalse
%</samplepart4>
%\fi
%
% %%%%%%%%%%%%%%%%%%%%%%%%%%%%%%%%%%%%%%
% \paragraph{Forwarding for a Complete Draft.}
%
% The following forwarding file |cdocsdrf.tex|
% compiles the main document in draft mode:
%\iffalse
%<*sampledraft>
%\fi
%    \begin{macrocode}
\def\version{draft}
\input{childdoc.def}
\childdocforward{cdocsamp}
%    \end{macrocode}

%\iffalse
%</sampledraft>
%\fi
%
% %%%%%%%%%%%%%%%%%%%%%%%%%%%%%%%%%%%%%%
% \paragraph{Forwarding for Final Version of the Chapters.}
%
% The following forwarding files |cdocsfn1.tex| and |cdocsfn2.tex|
% (with identical content)
% compile the final versions of the child documents
% |cdocsch1.tex| and |cdocsch2.tex|, respectively:
%\iffalse
%<*samplefinal>
%\fi
%    \begin{macrocode}
\def\version{final}
\input{childdoc.def}
\childdocforwardprefix[cdocsamp]{cdocsfn}{cdocsch}
%    \end{macrocode}

%\iffalse
%</samplefinal>
%\fi
%
% %%%%%%%%%%%%%%%%%%%%%%%%%%%%%%%%%%%%%%
% \paragraph{Command Line Processing.}
%
% The following three command lines generate the output files
% |cdocscld|, |cdocscl1| and |cdocscl2|
% which should be identical to
% |cdocsdrf|, |cdocsch1| and |cdocsfn2|, respectively:
% \begin{center}
% \begin{tabular}{l}
% |latex -jobname cdocscld \|\\
% |  "\def\version{draft}\input{childdoc.def}\childdocforward{cdocsamp}"|\\
% |latex -jobname cdocscl1 \|\\
% |  "\input{childdoc.def}\childdocforward[cdocsamp]{cdocsch1}"|\\
% |latex -jobname cdocscl2 \|\\
% |  "\def\version{final}\input{childdoc.def}\childdocforward{cdocsch2}"|
% \end{tabular}
% \end{center}
% Note that the trailing backslash on each first line
% merely continues the input to the second line
% (for convenient cut ant paste).
% Furthermore, the command |latex| can be replaced by any
% of its alternative versions such as |pdflatex|.
%
% %%%%%%%%%%%%%%%%%%%%%%%%%%%%%%%%%%%%%%%%%%%%%%%%%%%%%%%%%%%%%%%%%%%%%%%%%%%%%%
% %%%%%%%%%%%%%%%%%%%%%%%%%%%%%%%%%%%%%%%%%%%%%%%%%%%%%%%%%%%%%%%%%%%%%%%%%%%%%%
% \section{Implementation}
%\iffalse
%<*package>
%\fi
%
% This section describes the definitions file |childdoc.def|.

% The definitions cannot be loaded using |\usepackage| or |\RequirePackage|
% which has a mechanism to prevent loading a style file more than once.
% When loading the definitions by means of |\input|
% multiple instances have to be prevented manually:
%\iffalse
%This code needs to be before the `\ProvidesFile' directive
%which is defined at the beginning of this file.
%Therefore it is also placed there and commented out here.
%</package>
%<*discard>
%\fi
%    \begin{macrocode}
\ifdefined\childdocmain\endinput\fi
%    \end{macrocode}
%\iffalse
%</discard>
%<*package>
%\fi
%
% \macro{\ifchilddoc}
% \macro{\ifchilddocmanual}
% The conditional |\ifchilddoc| tells whether a
% child (true) or main (false) document is being compiled.
% The conditional |\ifchilddocmanual| tells whether
% the |\includeonly| mechanism is used (false) or
% the selection of child files must be performed manually (true).
% The definitions initialise to false:
%    \begin{macrocode}
\newif\ifchilddoc
\newif\ifchilddocmanual
%    \end{macrocode}

% \macro{\childdocname}
% \macro{\childdocjob}
% The macro |\childdocname| stores the name of the main document
% to be compiled. The macro |\childdocjob| stores the name of
% the document on which the \LaTeX{} compiler was originally invoked.
% The content of |\jobname| cannot be compared
% to filenames specified in the source due to different catcodes.
% The following code rescans |\jobname|, stores the result
% in |\childdocname| and saves a copy in |\childdocjob|:
%    \begin{macrocode}
\edef\childdocname{\scantokens\expandafter{\jobname\noexpand}}
\let\childdocjob\childdocname
%    \end{macrocode}

% \macro{\childdocdisable}
% The macro |\childdocdisable| prevents the main file
% from being processed more than once.
% At this stage, the main document command |\childdocmain|
% is assumed to be called once again where it should do nothing.
% Any subsequent call to it should prevent
% a secondary processing of the main document
% It overwrites the forwarding commands
% |\childdocof| and |\childdocforward|
% with empty macros to prevent further inclusions of the main document:
%    \begin{macrocode}
\newcommand{\childdocdisable}
{
  \renewcommand{\childdocmain}[1]{\renewcommand{\childdocmain}[1]{\endinput}}
  \renewcommand{\childdocof}[1]{}
  \renewcommand{\childdocby}[2][]{}
  \renewcommand{\childdocforward}[2][]{}
  \renewcommand{\childdocdisable}{}
}
%    \end{macrocode}

% \macro{\childdocmain}
% The macro |\childdocmain| is to be called at the top of the main file
% with nothing or the main filename (without extension) as argument.
% First, it breaks loops.
% If the argument is not empty and does not match |\childdocname|
% (which is set by the first inclusion of |childdoc.def|),
% |\ifchilddoc| is set to true, |\includeonly| is applied to the child file
% and |\jobname| is set to the main file
% (for proper handling of |.aux| files):
%    \begin{macrocode}
\newcommand{\childdocmain}[1]
{
  \childdocdisable\childdocmain{}
  \if?#1?\else
    \begingroup
      \def\childdoctmp{#1}
      \ifx\childdoctmp\childdocname
        \def\childdoctmp{}
      \else
        \def\childdoctmp
        {
          \childdoctrue
          \includeonly{\childdocname}
          \def\childdocjob{#1}
          \def\jobname{#1}
        }
      \fi
      \expandafter
    \endgroup
    \childdoctmp
  \fi
}
%    \end{macrocode}

% \macro{\childdocof}
% The command |\childdocof| redirects
% compilation to the main file |#1|.
%    \begin{macrocode}
\newcommand{\childdocof}[1]
{
  \childdocdisable
  \childdoctrue
  \includeonly{\childdocname}
  \def\jobname{#1}
  \def\childdocjob{#1}
  \input{#1}
}
%    \end{macrocode}

% \macro{\childdocby}
% The command |\childdocby| ....
%    \begin{macrocode}
\newcommand{\childdocby}[2][]
{
  \childdocdisable
  \childdoctrue
  \childdocmanualtrue
  \if?#1?\else
    \def\jobname{#2}
  \fi
  \def\childdocjob{#2}
  \input{#2}
  \endinput
}
%    \end{macrocode}

% \macro{\childdocforward}
% The command |\childdocforward| redirects
% compilation to the main file or
% (if the optional argument is given) a child file.
% Parameters are set as if the main file
% or a child file starting with |\childdocof| was compiled.
% Then compilation is handed over to the main file:
%    \begin{macrocode}
\newcommand{\childdocforward}[2][]
{
  \begingroup
    \if?#1?
      \def\childdoctmp
      {
        \def\childdocname{#2}
        \def\childdocjob{#2}
        \def\jobname{#2}
        \input{#2}
        \endinput
      }
    \else
      \def\childdoctmp
      {
        \childdocdisable
        \def\childdocname{#2}
        \childdoctrue
        \includeonly{#2}
        \def\childdocjob{#1}
        \def\jobname{#1}
        \input{#1}
        \endinput
      }
    \fi
    \expandafter
  \endgroup
  \childdoctmp
}
%    \end{macrocode}

% \macro{\childdocforwardprefix}
% The command |\childdocforwardprefix| redirects
% compilation to the main or a child file by means of a pattern.
% The prefix |#1| in the current filename is replaced by |#2|
% and the suffix of the current filename is kept
% (it is assumed that the filename does not contain the substring `|~~~|'
% which is used as a delimiter).
% Compilation is handed over to the new file by |\childdocforward|:
%    \begin{macrocode}
\newcommand{\childdocforwardprefix}[3][]
{
  \begingroup
    \def\childdocextract #2##1~~~{\def\childdoctmp{\childdocforward[#1]{#3##1}}}
    \expandafter\childdocextract\childdocname~~~
    \expandafter
  \endgroup
  \childdoctmp
}
%    \end{macrocode}

% \macro{\childdoc}
% The deprecated macro |\childdoc| is a legacy version of |\childdocmain|:
%    \begin{macrocode}
\newcommand{\childdoc}{\childdocmain}
%    \end{macrocode}

% \macro{\childdocredirect}
% The deprecated macro |\childdocredirect| is a legacy version
% of |\childdocforward| and |\childdocforwardprefix|:
%    \begin{macrocode}
\newcommand{\childdocredirect}[2][]
{
  \begingroup
    \if?#1?
      \def\childdoctmp{\childdocforward{#2}}
    \else
      \def\childdoctmp{\childdocforwardprefix{#1}{#2}}
    \fi
    \expandafter
  \endgroup
  \childdoctmp
}
%    \end{macrocode}

%\iffalse
%</package>
%\fi
%
\endinput
|
and perform the replacements as outlined below.
Instead of |\childdocmain{|\textit{main}|}| add the following code
to the top of the main file:
%
\begin{center}
\begin{tabular}{l}
|\||ifdefined\childdocname\endinput\||fi\newif\ifchilddoc|\\
|\edef\childdocname{\scantokens\expandafter{\jobname\noexpand}}|\\
|\def\childdocmain{|\textit{main}|}\||ifx\childdocmain\childdocname\||else|\\
|\childdoctrue\includeonly{\childdocname}\let\jobname\childdocmain\||fi|\\
\end{tabular}
\end{center}
%
Instead of |\childdocof{|\textit{main}|}| just include the main file
at the top of each child file:
%
\begin{center}
|\input{|\textit{main}|}|
\end{center}
%
A simple redirection |\childdocforward{|\textit{dest}|}| is achieved by:
%
\begin{center}
|\def\jobname{|\textit{dest}|}\input{\jobname}|
\end{center}
%
The redirection with prefix
|\childdocforwardprefix[|\textit{prefix}|]{|\textit{dest}|}|
is accomplished by:
%
\begin{center}
\begin{tabular}{l}
|{\edef\jobname{\scantokens\expandafter{\jobname\noexpand}}|\\
|\def\redirectjob |\textit{prefix}|#1~~~{\gdef\jobname{|\textit{dest}|#1}}|\\
|\expandafter\redirectjob\jobname~~~}\input{\jobname}|
\end{tabular}
\end{center}

In an alternative approach,
child documents can be compiled by a specific command line
without additional code or specific definitions:
%
\begin{center}
|... -jobname "|\textit{target}|" "|[\textit{flags}]%
|\includeonly{|\textit{dest}|}\input{|\textit{main}|}"|
\end{center}
%

%%%%%%%%%%%%%%%%%%%%%%%%%%%%%%%%%%%%%%%%%%%%%%%%%%%%%%%%%%%%%%%%%%%%%%%%%%%%%%%%
%%%%%%%%%%%%%%%%%%%%%%%%%%%%%%%%%%%%%%%%%%%%%%%%%%%%%%%%%%%%%%%%%%%%%%%%%%%%%%%%
\section{Information}

%%%%%%%%%%%%%%%%%%%%%%%%%%%%%%%%%%%%%%%%%%%%%%%%%%%%%%%%%%%%%%%%%%%%%%%%%%%%%%%%
\subsection{Copyright}

Copyright \copyright{} 2017--2018 Niklas Beisert

This work may be distributed and/or modified under the
conditions of the \LaTeX{} Project Public License, either version 1.3
of this license or (at your option) any later version.
The latest version of this license is in
  \url{http://www.latex-project.org/lppl.txt}
and version 1.3 or later is part of all distributions of \LaTeX{}
version 2005/12/01 or later.

This work has the LPPL maintenance status `maintained'.

The Current Maintainer of this work is Niklas Beisert.

This work consists of the files |README.txt|, |childdoc.ins| and |childdoc.dtx|
as well as the derived files |childdoc.def|, |cdocsamp.tex|
with |cdocsch1.tex|, |cdocsch2.tex|, |cdocspt3.tex|, |cdocspt4.tex|,
|cdocsdrf.tex|, |cdocsfn1.tex|, |cdocsfn2.tex|
as well as |childdoc.pdf|.

%%%%%%%%%%%%%%%%%%%%%%%%%%%%%%%%%%%%%%%%%%%%%%%%%%%%%%%%%%%%%%%%%%%%%%%%%%%%%%%%
\subsection{Files and Installation}

The package consists of the files:
%
\begin{center}
\begin{tabular}{ll}
    |README.txt|   & readme file \\
    |childdoc.ins| & installation file \\
    |childdoc.dtx| & source file \\
    |childdoc.def| & definition file \\
    |cdocsamp.tex| & sample main file \\
    |cdocsch1.tex| & sample include file \\
    |cdocsch2.tex| & sample include file \\
    |cdocspt3.tex| & sample part file \\
    |cdocspt4.tex| & sample part file \\
    |cdocsdrf.tex| & sample redirection file \\
    |cdocsfn1.tex| & sample redirection file \\
    |cdocsfn2.tex| & sample redirection file \\
    |childdoc.pdf| & manual
\end{tabular}
\end{center}
%
The distribution consists of the files
|README.txt|, |childdoc.ins| and |childdoc.dtx|.
%
\begin{itemize}
\item
Run (pdf)\LaTeX{} on |childdoc.dtx|
to compile the manual |childdoc.pdf| (this file).
\item
Run \LaTeX{} on |childdoc.ins| to create the definitions file |childdoc.def|
and the sample |cdocsamp.tex| with include files
|cdocsch1.tex|, |cdocsch2.tex|, |cdocspt3.tex|, |cdocspt4.tex|,
|cdocsdrf.tex|, |cdocsfn1.tex|, |cdocsfn2.tex|.
Then copy the file |childdoc.def| to an appropriate directory of your \LaTeX{}
distribution, e.g.\ \textit{texmf-root}|/tex/latex/childdoc|.
\end{itemize}

%%%%%%%%%%%%%%%%%%%%%%%%%%%%%%%%%%%%%%%%%%%%%%%%%%%%%%%%%%%%%%%%%%%%%%%%%%%%%%%%
\subsection{Related CTAN Packages}

There are several other packages which offer a similar functionality:
%
\begin{itemize}
\item
The packages
\href{http://ctan.org/pkg/docmute}{\textsf{docmute}},
\href{http://ctan.org/pkg/includex}{\textsf{includex}} and
\href{http://ctan.org/pkg/standalone}{\textsf{standalone}}
provide commands to include only the document body of
a child file thus allowing both files to be compiled individually.
\item
The packages \href{http://ctan.org/pkg/subdocs}{\textsf{subdocs}}
and \href{http://ctan.org/pkg/subfiles}{\textsf{subfiles}}
provide structures in which the main and child documents can be
encapsulated and allowing them to be compiled individually.
The inclusion mechanism is different from the conventional |\include|.
\item
The package \href{http://ctan.org/pkg/combine}{\textsf{combine}}
is an elaborate solution to combine several documents into one.
\end{itemize}
%
See also the CTAN topic \href{http://ctan.org/topic/subdocs}{\textsf{subdocs}}
for further related packages.
The present package differs from the above solutions in that
a document structure constructed with the conventional |\include| mechanism
just needs two extra commands at the top of every file
such that all constituent files can be compiled individually.

%%%%%%%%%%%%%%%%%%%%%%%%%%%%%%%%%%%%%%%%%%%%%%%%%%%%%%%%%%%%%%%%%%%%%%%%%%%%%%%%
%\subsection{Feature Suggestions}
%
%The following is a list of features which may be useful for future
%versions of this package:
%%
%\begin{itemize}
%\item
%\ldots
%\end{itemize}

%%%%%%%%%%%%%%%%%%%%%%%%%%%%%%%%%%%%%%%%%%%%%%%%%%%%%%%%%%%%%%%%%%%%%%%%%%%%%%%%
\subsection{Revision History}

%%%%%%%%%%%%%%%%%%%%%%%%%%%%%%%%%%%%%%%%
\paragraph{v2.0:} 2018/12/30

\begin{itemize}
\item
immediate forward processing
\item
added |\childdocby| mechanism
\item
manual restructured
\end{itemize}

%%%%%%%%%%%%%%%%%%%%%%%%%%%%%%%%%%%%%%%%
\paragraph{v1.6:} 2018/01/17

\begin{itemize}
\item
application for development of include files
\item
corrections to manual
\end{itemize}

%%%%%%%%%%%%%%%%%%%%%%%%%%%%%%%%%%%%%%%%
\paragraph{v1.5:} 2017/05/21

\begin{itemize}
\item
more complete structuring introduced
\item
|\childdocof| introduced
\item
|\childdoc| renamed to |\childdocmain|
\item
|\childredirect| renamed to |\childdocforward| and |\childdocforwardprefix|
and functionality expanded
\end{itemize}

%%%%%%%%%%%%%%%%%%%%%%%%%%%%%%%%%%%%%%%%
\paragraph{v1.0:} 2017/04/27

\begin{itemize}
\item
manual and install package
\item
first version published on CTAN
\end{itemize}

%%%%%%%%%%%%%%%%%%%%%%%%%%%%%%%%%%%%%%%%
\paragraph{v0.6:} 2017/04/26

\begin{itemize}
\item
redirection mechanism added
\end{itemize}

%%%%%%%%%%%%%%%%%%%%%%%%%%%%%%%%%%%%%%%%
\paragraph{v0.5:} 2017/04/26

\begin{itemize}
\item
functionality in definition file
\end{itemize}


%%%%%%%%%%%%%%%%%%%%%%%%%%%%%%%%%%%%%%%%%%%%%%%%%%%%%%%%%%%%%%%%%%%%%%%%%%%%%%%%
%%%%%%%%%%%%%%%%%%%%%%%%%%%%%%%%%%%%%%%%%%%%%%%%%%%%%%%%%%%%%%%%%%%%%%%%%%%%%%%%
%%%%%%%%%%%%%%%%%%%%%%%%%%%%%%%%%%%%%%%%%%%%%%%%%%%%%%%%%%%%%%%%%%%%%%%%%%%%%%%%
\appendix

\settowidth\MacroIndent{\rmfamily\scriptsize 000\ }

 \DocInput{childdoc.dtx}

\end{document}
%</driver>
% \fi
%
% %%%%%%%%%%%%%%%%%%%%%%%%%%%%%%%%%%%%%%%%%%%%%%%%%%%%%%%%%%%%%%%%%%%%%%%%%%%%%%
% %%%%%%%%%%%%%%%%%%%%%%%%%%%%%%%%%%%%%%%%%%%%%%%%%%%%%%%%%%%%%%%%%%%%%%%%%%%%%%
% \section{Sample}
%\iffalse
%<*samplemain>
%\fi
%
% The following presents a sample document
% with two chapters, two parts, a title page,
% a compile flag as well as three forwarding files to set the flag.
% It consists of eight |.tex| files:
% \begin{center}
% \begin{tabular}{ll}
% |cdocsamp.tex|&main file\\
% |cdocsch1.tex|&include file for chapter 1\\
% |cdocsch2.tex|&include file for chapter 2\\
% |cdocspt3.tex|&include file for part 3\\
% |cdocspt4.tex|&include file for part 4\\
% |cdocsdrf.tex|&forwarding file for main file in draft mode\\
% |cdocsfi1.tex|&forwarding file for final version of chapter 1\\
% |cdocsfi2.tex|&forwarding file for final version of chapter 2\\
% \end{tabular}
% \end{center}
% Each of the eight files can be compiled directly by the \LaTeX{} compiler.
%
% %%%%%%%%%%%%%%%%%%%%%%%%%%%%%%%%%%%%%%
% \paragraph{Main File.}
%
% The main file is called |cdocsamp.tex|.
%
% Load the \textsf{childdoc} definitions and
% declare the filename for the main document:
%    \begin{macrocode}
% \iffalse
%
% childdoc.dtx Copyright (C) 2017-2018 Niklas Beisert
%
% This work may be distributed and/or modified under the
% conditions of the LaTeX Project Public License, either version 1.3
% of this license or (at your option) any later version.
% The latest version of this license is in
%   http://www.latex-project.org/lppl.txt
% and version 1.3 or later is part of all distributions of LaTeX
% version 2005/12/01 or later.
%
% This work has the LPPL maintenance status `maintained'.
%
% The Current Maintainer of this work is Niklas Beisert.
%
% This work consists of the files childdoc.dtx and childdoc.ins
% and the derived files childdoc.def and cdocsamp.tex with
% cdocsch1.tex, cdocsch2.tex, cdocsdrf.tex, cdocsfn1.tex, cdocsfn2.tex.
%
%<package>\ifdefined\childdocmain\endinput\fi
%<package>\ProvidesFile{childdoc.def}[2018/12/30 v2.0 child document driver]
%<samplemain>\ProvidesFile{cdocsamp.tex}[2018/12/30 v2.0 sample for childdoc]
%<*driver>
%\ProvidesFile{childdoc.drv}[2018/12/30 v2.0 childdoc reference manual file]
\PassOptionsToClass{10pt,a4paper}{article}
\documentclass{ltxdoc}

\usepackage[margin=35mm]{geometry}
\usepackage{hyperref}
\usepackage{hyperxmp}
\usepackage[usenames]{color}

\hypersetup{colorlinks=true}
\hypersetup{pdfstartview=FitH}
\hypersetup{pdfpagemode=UseNone}
\hypersetup{pdfsource={}}
\hypersetup{pdflang={en-UK}}
\hypersetup{pdfcopyright={Copyright 2017-2018 Niklas Beisert.
  This work may be distributed and/or modified under the
  conditions of the LaTeX Project Public License, either version 1.3
  of this license or (at your option) any later version.}}
\hypersetup{pdflicenseurl={http://www.latex-project.org/lppl.txt}}
\hypersetup{pdfcontactaddress={ETH Zurich, ITP, HIT K,
  Wolfgang-Pauli-Strasse 27}}
\hypersetup{pdfcontactpostcode={8093}}
\hypersetup{pdfcontactcity={Zurich}}
\hypersetup{pdfcontactcountry={Switzerland}}
\hypersetup{pdfcontactemail={nbeisert@itp.phys.ethz.ch}}
\hypersetup{pdfcontacturl={http://people.phys.ethz.ch/\xmptilde nbeisert/}}

\newcommand{\secref}[1]{\hyperref[#1]{section \ref*{#1}}}

\parskip1ex
\parindent0pt
\let\olditemize\itemize
\def\itemize{\olditemize\parskip0pt}

\begin{document}

\title{The \textsf{childdoc} Package}
\hypersetup{pdftitle={The childdoc Package}}
\author{Niklas Beisert\\[2ex]
  Institut f\"ur Theoretische Physik\\
  Eidgen\"ossische Technische Hochschule Z\"urich\\
  Wolfgang-Pauli-Strasse 27, 8093 Z\"urich, Switzerland\\[1ex]
  \href{mailto:nbeisert@itp.phys.ethz.ch}
  {\texttt{nbeisert@itp.phys.ethz.ch}}}
\hypersetup{pdfauthor={Niklas Beisert}}
\hypersetup{pdfsubject={Manual for the LaTeX2e Package childdoc}}
\date{30 December 2018, \textsf{v2.0}}
\maketitle

\begin{abstract}\noindent
\textsf{childdoc} is a \LaTeXe{} package
that enables the direct compilation
of document sections included by |\include|
to individual files.
\end{abstract}

\begingroup
\parskip0ex
\tableofcontents
\endgroup

%%%%%%%%%%%%%%%%%%%%%%%%%%%%%%%%%%%%%%%%%%%%%%%%%%%%%%%%%%%%%%%%%%%%%%%%%%%%%%%%
%%%%%%%%%%%%%%%%%%%%%%%%%%%%%%%%%%%%%%%%%%%%%%%%%%%%%%%%%%%%%%%%%%%%%%%%%%%%%%%%
\section{Introduction}

\LaTeX{} provides a mechanism to structure a large document (such as a book)
into a main file and several child files (containing the chapters)
using the |\include| command.
This mechanism is beneficial for documents
which span hundreds of pages in order to
make the source file(s) more manageable.
Moreover, compilation can be restricted to
selected child files by means of the |\includeonly| command.
The latter feature can be used to reduce the compilation time while editing
(this was significantly more useful in the earlier days of \LaTeX{})
or to generate a smaller document which is easier to navigate.
Another application of |\includeonly| is to generate
documents consisting of selected parts of the complete document.

However, there are a few drawbacks of the plain |\include| mechanism:
\begin{itemize}
\item
The child files cannot be compiled on their own,
they can only be compiled via the main file.
A naive editing environment
(such as a text editor with an option
to have the current file processed by \LaTeX)
may require one to switch to the main file before compiling;
attempting to compile the child file produces errors.
\item
The main file must be modified (each time)
to adjust the |\includeonly| command
to the present needs. This easily leaves the main file in a messy state.
\item
The generated document will always carry the filename
of the main document. This is inconvenient if
several child files are to be compiled and
to be kept for distribution.
\end{itemize}

The present package provides a simple interface
to make child files individually compilable by \LaTeX{}.
Compiling a child file then has the same effect as compiling
the main file with an |\includeonly| command
to select the appropriate child.
Moreover the generated document will carry the name of the child
rather than the main file.
This resolves all three above issues.

This feature is meant to make the editing of books,
thesis documents and lecture notes somewhat more convenient.
However, the package can also be used efficiently for
composing a series of documents (such as exercise sheets)
which are typically distributed individually.
It then assists the author in generating the individual documents
(potentially in different versions)
as well as a document containing the collected series.
Another application is in developing style files
or other kinds of included material
where compilation of the style file could redirect
to a sample or test file.

%%%%%%%%%%%%%%%%%%%%%%%%%%%%%%%%%%%%%%%%%%%%%%%%%%%%%%%%%%%%%%%%%%%%%%%%%%%%%%%%
%%%%%%%%%%%%%%%%%%%%%%%%%%%%%%%%%%%%%%%%%%%%%%%%%%%%%%%%%%%%%%%%%%%%%%%%%%%%%%%%
\section{Usage}

First of all, the package \textsf{childdoc} is \emph{not} a standard
\LaTeXe{} |.sty| style file! Therefore it needs to be invoked in
a non-standard way.

%%%%%%%%%%%%%%%%%%%%%%%%%%%%%%%%%%%%%%%%%%%%%%%%%%%%%%%%%%%%%%%%%%%%%%%%%%%%%%%%
\subsection{Included Files}
\label{sec:include}

%%%%%%%%%%%%%%%%%%%%%%%%%%%%%%%%%%%%%%%%
\DescribeMacro{\childdocmain}
To use the package, add the commands
\begin{center}
\begin{tabular}{l}
|\input{childdoc.def}|\\
|\childdocmain{}|\\
\end{tabular}
\end{center}
at the very top of the main \LaTeX{} file,
in particular \emph{before} the |\documentclass| statement!
The argument of |\childdocmain| should be left empty
(but it must be present).

%%%%%%%%%%%%%%%%%%%%%%%%%%%%%%%%%%%%%%%%
\DescribeMacro{\childdocof}
Furthermore, add the commands
\begin{center}
\begin{tabular}{l}
|\input{childdoc.def}|\\
|\childdocof{|\textit{main}|}|\\
\end{tabular}
\end{center}
at the top of every child file \textit{child}
which is included by |\include{|\textit{child}|}|
from within the main file
(or at least for those files to be compiled individually).
The argument \textit{main} must be the filename of the main file.

There are a couple of
considerations in setting up the main and child documents:

%%%%%%%%%%%%%%%%%%%%%%%%%%%%%%%%%%%%%%%%
\paragraph{Restrictions.}

Please note the following restrictions:
\begin{itemize}
\item
|\childdocmain| must be called with one argument \textit{main}
to ensure compatibility with earlier version of the package.
It must either be empty (|\childdocmain{}|)
or precisely match the filename of the main file in which it is specified.
See \secref{sec:detection} for further information.
\item
The filename \textit{main} must be specified without the |.tex| extension.
\item
The filename \textit{main} is case sensitive
(even in case-insensitive file systems)
due to internal string comparison.
\item
The argument \textit{main} should be fully expanded, it cannot be a macro.
\item
Subdirectories and special characters should be avoided in filenames.
\item
The command |\childdocmain{|\textit{main}|}| must be followed by a whitespace.
It should not be followed immediately by another command
or by a comment mark `|%|'.
This is because the \TeX{} parser reads the token immediately following
the argument of |\childdocmain| and puts it
at the beginning of every child section;
however, a white\-space is ignored.
\end{itemize}

%%%%%%%%%%%%%%%%%%%%%%%%%%%%%%%%%%%%%%%%
\paragraph{Content of Main File.}

It is advisable to place all content in the child files included by |\include|.
Any output contained in the main file will appear in all child documents
unless suppressed manually;
it cannot be suppressed automatically by the |\includeonly| directive
and thus should normally be avoided.
A method to include some content in the main file
by means of conditional processing is described in \secref{sec:conditional}.

%%%%%%%%%%%%%%%%%%%%%%%%%%%%%%%%%%%%%%%%
\paragraph{Page Numbering.}

When only a part of the document is compiled,
the appropriate numbering of pages
(as well as other status parameters)
is determined from the |.aux| files.
The latter contain information from previous passes.
However this information needs to propagate through
all intermediate child documents.
Therefore the page numbering in child documents may well
be inconsistent until the complete document is compiled at least once.

A useful (if unconventional) way to always ensure a consistent
page numbering is to restart the numbering in each child document
and denote the pages by `\textit{child}|.|\textit{page}'
where \textit{child} represents the chapter/section number of the child file.
This can be achieved by the command
|\numberwithin{page}{|\textit{child}|}|
of the \textsf{amsmath} package
where \textit{child} can be |chapter| or |section|
depending on the chosen structuring.
Alternatively, one can modify the macro |\thepage| appropriately
and reset the counter |page| at the start of each child file.

%%%%%%%%%%%%%%%%%%%%%%%%%%%%%%%%%%%%%%%%%%%%%%%%%%%%%%%%%%%%%%%%%%%%%%%%%%%%%%%%
\subsection{Conditional Processing}
\label{sec:conditional}

The package provides a mechanism to compile different versions
of a document. To customise the versions further some conditional processing
can come in handy to distinguish which version is being compiled.
The package provides two macros to describe the compilation context:

%%%%%%%%%%%%%%%%%%%%%%%%%%%%%%%%%%%%%%%%
\DescribeMacro{\ifchilddoc}
The conditional |\ifchilddoc| distinguishes between the compilation of
child documents and the main document:
%
\begin{center}
|\ifchilddoc |\textit{child-code}| |[|\||else |\textit{main-code}]| \||fi|
\end{center}

%%%%%%%%%%%%%%%%%%%%%%%%%%%%%%%%%%%%%%%%
\DescribeMacro{\childdocname}
\DescribeMacro{\childdocjob}
The macro |\childdocname| contains the filename (without extension)
of the main or child file being processed.
Note that |\childdocjob| will always contain the name of the main file.

%%%%%%%%%%%%%%%%%%%%%%%%%%%%%%%%%%%%%%%%
\paragraph{Title Page.}

Conditional processing can be used to include a title or banner page
in the main document when proper precautions are taken.
Importantly, the code in the main file should ensure that the page counter
(as well as other status parameters which are stored in the |.aux| files)
takes the same value after the conditional processing.
Otherwise the page numbers may take divergent values
depending on which part is compiled.

For example, a title page could be declared by:
%
\begin{center}
\begin{tabular}{l}
|\ifchilddoc\||else|\\
|\addtocounter{page}{-1}|\\
\textit{code for title page}\\
|\newpage|\\
|\||fi|
\end{tabular}
\end{center}
%
A banner page for the child documents can be generated by:
%
\begin{center}
\begin{tabular}{l}
|\ifchilddoc|\\
|\addtocounter{page}{-1}|\\
\textit{code for banner page}\\
|\newpage|\\
|\||fi|
\end{tabular}
\end{center}
%
Here one could write a message such as:
\begin{center}
|This is the part \childdocname{} of \childdocjob{}.|
\end{center}

%%%%%%%%%%%%%%%%%%%%%%%%%%%%%%%%%%%%%%%%%%%%%%%%%%%%%%%%%%%%%%%%%%%%%%%%%%%%%%%%
\subsection{Flags}
\label{sec:flags}

The package makes it easy to generate different versions
of the main or child documents.
To this end compilation flags can be defined
and assigned different default values.
They will be particularly useful in conjunction
with the forwarding mechanism described in \secref{sec:forward}.

For example, it may be useful to have a flag |\version|
which can be set to |draft| or |final|.
The document source will contain some conditional code
depending on the value of |\version|.
Suppose further, the flag should default to |final| for the main file
and to |draft| for child files
which is a natural assignment for editing the document.
This is achieved by placing the following code
in the preamble of the main document
(below the |\childdocmain| directive):
%
\begin{center}
\begin{tabular}{l}
|\ifchilddoc|\\
|\providecommand{\version}{draft}|\\
|\||else|\\
|\providecommand{\version}{final}|\\
|\||fi|
\end{tabular}
\end{center}
%
The definition by |\providecommand| makes sure
that previous definitions are not overwritten.
Further statements |\providecommand{\version}{...}|
can thus be added before the above code to override it.

For the main file, one might add a line
(between |\childdocmain| and the above block)
%
\begin{center}
|%\ifchilddoc\||else\providecommand{\version}{draft}\||fi|
\end{center}
%
which can be uncommented to produce a draft version.
Likewise one can add a line to the very top of a child file
(above the |\childdocof{|\textit{main}|}| directive)
%
\begin{center}
|%\providecommand{\version}{final}|
\end{center}
%
which can be uncommented to produce the final version of this child document.

%%%%%%%%%%%%%%%%%%%%%%%%%%%%%%%%%%%%%%%%%%%%%%%%%%%%%%%%%%%%%%%%%%%%%%%%%%%%%%%%
\subsection{Forwarding}
\label{sec:forward}

Different versions of the main or child documents
using compilation flags as described in \secref{sec:flags}
can be (permanently) stored in different files
for convenient compilation, viewing and distribution.
To this end, the package defines a command
to pass on compilation to a different file:

%%%%%%%%%%%%%%%%%%%%%%%%%%%%%%%%%%%%%%%%
\DescribeMacro{\childdocforward}
The command |\childdocforward| redirects processing to
another source file:
%
\begin{center}
\begin{tabular}{l}
|\input{childdoc.def}|\\
|\childdocforward[|\textit{main}|]{|\textit{dest}|}|\\
\end{tabular}
\end{center}
%
The argument \textit{dest} is the destination file
(without extension).
It should be the main file or one of the child files.
Note that further \textsf{childdoc} directives
such as |\childdocof| and |\childdocforward|
in the indicated file will be processed in this form.
The optional argument \textit{main}
passes on directly to the main file \textit{main}
while pretending to compile the child \textit{dest}.
This form behaves as if \textit{dest}
issues |\childdocof{|\textit{main}|}| right away,
and no further \textsf{childdoc} directives will be processed.

%%%%%%%%%%%%%%%%%%%%%%%%%%%%%%%%%%%%%%%%
\DescribeMacro{\...prefix}
In the alternative form |\childdocforwardprefix|,
%
\begin{center}
\begin{tabular}{l}
|\input{childdoc.def}|\\
|\childdocforwardprefix[|\textit{main}|]{|\textit{prefix}|}{|\textit{dest}|}|
\end{tabular}
\end{center}
%
the destination file is determined by a pattern
depending on the current file:
To make this work, the current file must be called
`{\textit{prefix}\hspace{0.2em}\textit{suffix}}'
with \textit{prefix} matching precisely the argument.
Processing is then passed on to the file
`{\textit{dest}\hspace{0.2em}\textit{suffix}}'.
Surely, the same effect is achieved by
directly specifying the
argument `{\textit{dest}\hspace{0.2em}\textit{suffix}}'
in the first form.
However, that requires to set up a different file
for each child. With the alternative form of the command
all these files can have exactly the same content
which simplifies setting them up and maintaining them.

For example, the following file |draft.tex|
with a compilation flag |\version| as described in \secref{sec:flags}
compiles the main document as a draft:
%
\begin{center}
\begin{tabular}{l}
|\def\version{draft}|\\
|\input{childdoc.def}|\\
|\childdocforward{|\textit{main}|}|
\end{tabular}
\end{center}
%
Likewise, the following files |final|\textit{nn}|.tex|
compile the final version of the child document
|child|\textit{nn}|.tex|:
%
\begin{center}
\begin{tabular}{l}
|\def\version{final}|\\
|\input{childdoc.def}|\\
|\childdocforwardprefix{final}{child}|
\end{tabular}
\end{center}
%

Note that when several versions of a main file and/or of each child file
are to be generated, it may be convenient to set up a |Makefile| or
shell script to automatise the process.

%%%%%%%%%%%%%%%%%%%%%%%%%%%%%%%%%%%%%%%%%%%%%%%%%%%%%%%%%%%%%%%%%%%%%%%%%%%%%%%%
\subsection{Command Line Processing}
\label{sec:commandline}

The effect of redirection files can also be achieved by invoking
the \LaTeX{} compiler with a more elaborate command line.
Most conveniently this should be done as part
of a shell script or a |Makefile|.

When using \textsf{childdoc} in the main file, the following
command lines effectively perform a redirection
(note that depending on the shell being used,
backslashes may have to be doubled: `|\|' $\to$ `|\\|'):
%
\begin{center}
|... -jobname "|\textit{target}|" |\\|"|[\textit{flags}]%
|\input{childdoc.def}\childdocforward[|\textit{main}|]{|\textit{dest}|}"|
\end{center}
%
Here \textit{target} is the name of the output file,
\textit{main} is the name of the main file
and \textit{dest} is the name of the main or child file to be processed
(all filenames without extensions).
The optional argument \textit{main} can be omitted
if \textit{main} matches \textit{dest}.
Optionally, compilation \textit{flags} can be defined via |\def| commands.
This command line makes the \TeX{} engine believe
it is compiling the file \textit{target}
whose content is specified as the latter parameter.
The provided code then forwards the processing to
\textit{main} or \textit{dest} as described in \secref{sec:forward}.

%%%%%%%%%%%%%%%%%%%%%%%%%%%%%%%%%%%%%%%%%%%%%%%%%%%%%%%%%%%%%%%%%%%%%%%%%%%%%%%%
\subsection{Include by Input}
\label{sec:input}

Including child documents by |\include| has some restrictions by design.
Most notably, the content of a child document always occupies
its own set of pages; pages cannot be shared between child documents.
Usually, this behaviour makes perfect sense
because each child document contain an essential part of the document.
However, in some situations it may be desirable to compose
a document from a collection of parts
without having mandatory page breaks between then.
For this case, the package
provides a mechanism to include parts
by |\input| which can also be processed individually.
However, by construction this mechanism
requires manual handling of the content to be output.

%%%%%%%%%%%%%%%%%%%%%%%%%%%%%%%%%%%%%%%%
\DescribeMacro{\ifchilddocmanual}
The main file should be prepared as usual, see \secref{sec:include}.
However, the document body must make a distinction
between processing of an individual part and of the main document, e.g.:
%
\begin{center}
\begin{tabular}{l}
|\ifchilddocmanual|\\
|\input{\childdocname}|\\
|\||else|\\
\textit{document body with }|\input{|\textit{part}|}|\\
|\||fi|
\end{tabular}
\end{center}
%
The conditional |\ifchilddocmanual| is true whenever
a part to be included by |\input| is being compiled,
and the name of the part is stored in |\childdocname|.

%%%%%%%%%%%%%%%%%%%%%%%%%%%%%%%%%%%%%%%%
\DescribeMacro{\childdocby}
Each part to be included by |\input| should start with:
%
\begin{center}
\begin{tabular}{l}
|\input{childdoc.def}|\\
|\childdocby{|\textit{main}|}|\\
\end{tabular}
\end{center}
%
The directive |\childdocby| is similar to |\childdocof|
described in \secref{sec:include},
but the subsequent selection of content must be done manually.
To that end, both |\ifchilddoc| and |\ifchilddocmanual|
will be true upon processing of a part,
and the name of the part is stored in |\childdocname|.
Note that |\jobname| will be set to the filename of the current part
so that each part receives an individual |.aux| file
that does not interfere with the |.aux| file(s) of the main document.
This behaviour can be altered by the alternative form
|\childdocby[*]{|\textit{main}|}| (with a non-empty optional argument)
which uses the |.aux| file of the main document
by setting |\jobname| to \textit{main}.

%%%%%%%%%%%%%%%%%%%%%%%%%%%%%%%%%%%%%%%%%%%%%%%%%%%%%%%%%%%%%%%%%%%%%%%%%%%%%%%%
\subsection{Driver Development}
\label{sec:driver}

The \textsf{childdoc} mechanism can also be use for the development
of definition files such as \LaTeX{} styles or classes.
This case differs from the above setup with multiple parts
included by |\include| in that no |\includeonly| should be invoked.
This can be achieved by starting the include file
(before |\ProvidesPackage|) with:
%
\begin{center}
\begin{tabular}{l}
|\input{childdoc.def}|\\
|\childdocforward{|\textit{main}|}|\\
\end{tabular}
\end{center}
%
or alternatively with:
%
\begin{center}
\begin{tabular}{l}
|\input{childdoc.def}|\\
|\childdocby{|\textit{main}|}|\\
\end{tabular}
\end{center}
%
Both forms have slightly different effects as described above.
The main file is prepared as usual, see \secref{sec:include}.

%%%%%%%%%%%%%%%%%%%%%%%%%%%%%%%%%%%%%%%%%%%%%%%%%%%%%%%%%%%%%%%%%%%%%%%%%%%%%%%%
\subsection{Legacy Detection}
\label{sec:detection}

The directive |\childdocmain| in the main file can detect
whether the complete document or merely a child is to be compiled
even without using the directive |\childdocof|.
This method is deprecated because it is less robust
and there is no compelling reason to use it;
it is merely provided for backward compatibility
and it may be removed in future versions.

If the detection mechanism is to be used,
it is mandatory to correctly specify
the filename of the main file as the argument of |\childdocmain|:
%
\begin{center}
\begin{tabular}{l}
|\input{childdoc.def}|\\
|\childdocmain{|\textit{main}|}|\\
\end{tabular}
\end{center}
%
If |\jobname| does not match the argument \textit{main} of |\childdocmain|,
it is assumed that |\jobname| points to the child file to be compiled.
When using |\childdocmain| with the main file specified as argument,
it suffices to start a child file
with just |\input{|\textit{main}|}|
without loading of the package and using |\childdocof|.
If instead all processing is done
with the appropriate \textsf{childdoc} directives,
the argument of \textit{main} of |\childdocmain| can be empty.

An alternative version of the command line processing described
in \secref{sec:commandline} using the detection mechanism reads:
%
\begin{center}
|... -jobname "|\textit{target}|" "|[\textit{flags}]%
[|\def\jobname{|\textit{dest}|}|]|\input{|\textit{main}|}"|
\end{center}

%%%%%%%%%%%%%%%%%%%%%%%%%%%%%%%%%%%%%%%%%%%%%%%%%%%%%%%%%%%%%%%%%%%%%%%%%%%%%%%%
\subsection{Manual Code}
\label{sec:manual}

In case one cannot be certain whether the definitions file |childdoc.def|
is installed on the target \TeX{} distribution
and one prefers not to ship it,
it is conceivable to paste a few relevant commands into the sources.

To that end, drop all statements |\input{childdoc.def}|
and perform the replacements as outlined below.
Instead of |\childdocmain{|\textit{main}|}| add the following code
to the top of the main file:
%
\begin{center}
\begin{tabular}{l}
|\||ifdefined\childdocname\endinput\||fi\newif\ifchilddoc|\\
|\edef\childdocname{\scantokens\expandafter{\jobname\noexpand}}|\\
|\def\childdocmain{|\textit{main}|}\||ifx\childdocmain\childdocname\||else|\\
|\childdoctrue\includeonly{\childdocname}\let\jobname\childdocmain\||fi|\\
\end{tabular}
\end{center}
%
Instead of |\childdocof{|\textit{main}|}| just include the main file
at the top of each child file:
%
\begin{center}
|\input{|\textit{main}|}|
\end{center}
%
A simple redirection |\childdocforward{|\textit{dest}|}| is achieved by:
%
\begin{center}
|\def\jobname{|\textit{dest}|}\input{\jobname}|
\end{center}
%
The redirection with prefix
|\childdocforwardprefix[|\textit{prefix}|]{|\textit{dest}|}|
is accomplished by:
%
\begin{center}
\begin{tabular}{l}
|{\edef\jobname{\scantokens\expandafter{\jobname\noexpand}}|\\
|\def\redirectjob |\textit{prefix}|#1~~~{\gdef\jobname{|\textit{dest}|#1}}|\\
|\expandafter\redirectjob\jobname~~~}\input{\jobname}|
\end{tabular}
\end{center}

In an alternative approach,
child documents can be compiled by a specific command line
without additional code or specific definitions:
%
\begin{center}
|... -jobname "|\textit{target}|" "|[\textit{flags}]%
|\includeonly{|\textit{dest}|}\input{|\textit{main}|}"|
\end{center}
%

%%%%%%%%%%%%%%%%%%%%%%%%%%%%%%%%%%%%%%%%%%%%%%%%%%%%%%%%%%%%%%%%%%%%%%%%%%%%%%%%
%%%%%%%%%%%%%%%%%%%%%%%%%%%%%%%%%%%%%%%%%%%%%%%%%%%%%%%%%%%%%%%%%%%%%%%%%%%%%%%%
\section{Information}

%%%%%%%%%%%%%%%%%%%%%%%%%%%%%%%%%%%%%%%%%%%%%%%%%%%%%%%%%%%%%%%%%%%%%%%%%%%%%%%%
\subsection{Copyright}

Copyright \copyright{} 2017--2018 Niklas Beisert

This work may be distributed and/or modified under the
conditions of the \LaTeX{} Project Public License, either version 1.3
of this license or (at your option) any later version.
The latest version of this license is in
  \url{http://www.latex-project.org/lppl.txt}
and version 1.3 or later is part of all distributions of \LaTeX{}
version 2005/12/01 or later.

This work has the LPPL maintenance status `maintained'.

The Current Maintainer of this work is Niklas Beisert.

This work consists of the files |README.txt|, |childdoc.ins| and |childdoc.dtx|
as well as the derived files |childdoc.def|, |cdocsamp.tex|
with |cdocsch1.tex|, |cdocsch2.tex|, |cdocspt3.tex|, |cdocspt4.tex|,
|cdocsdrf.tex|, |cdocsfn1.tex|, |cdocsfn2.tex|
as well as |childdoc.pdf|.

%%%%%%%%%%%%%%%%%%%%%%%%%%%%%%%%%%%%%%%%%%%%%%%%%%%%%%%%%%%%%%%%%%%%%%%%%%%%%%%%
\subsection{Files and Installation}

The package consists of the files:
%
\begin{center}
\begin{tabular}{ll}
    |README.txt|   & readme file \\
    |childdoc.ins| & installation file \\
    |childdoc.dtx| & source file \\
    |childdoc.def| & definition file \\
    |cdocsamp.tex| & sample main file \\
    |cdocsch1.tex| & sample include file \\
    |cdocsch2.tex| & sample include file \\
    |cdocspt3.tex| & sample part file \\
    |cdocspt4.tex| & sample part file \\
    |cdocsdrf.tex| & sample redirection file \\
    |cdocsfn1.tex| & sample redirection file \\
    |cdocsfn2.tex| & sample redirection file \\
    |childdoc.pdf| & manual
\end{tabular}
\end{center}
%
The distribution consists of the files
|README.txt|, |childdoc.ins| and |childdoc.dtx|.
%
\begin{itemize}
\item
Run (pdf)\LaTeX{} on |childdoc.dtx|
to compile the manual |childdoc.pdf| (this file).
\item
Run \LaTeX{} on |childdoc.ins| to create the definitions file |childdoc.def|
and the sample |cdocsamp.tex| with include files
|cdocsch1.tex|, |cdocsch2.tex|, |cdocspt3.tex|, |cdocspt4.tex|,
|cdocsdrf.tex|, |cdocsfn1.tex|, |cdocsfn2.tex|.
Then copy the file |childdoc.def| to an appropriate directory of your \LaTeX{}
distribution, e.g.\ \textit{texmf-root}|/tex/latex/childdoc|.
\end{itemize}

%%%%%%%%%%%%%%%%%%%%%%%%%%%%%%%%%%%%%%%%%%%%%%%%%%%%%%%%%%%%%%%%%%%%%%%%%%%%%%%%
\subsection{Related CTAN Packages}

There are several other packages which offer a similar functionality:
%
\begin{itemize}
\item
The packages
\href{http://ctan.org/pkg/docmute}{\textsf{docmute}},
\href{http://ctan.org/pkg/includex}{\textsf{includex}} and
\href{http://ctan.org/pkg/standalone}{\textsf{standalone}}
provide commands to include only the document body of
a child file thus allowing both files to be compiled individually.
\item
The packages \href{http://ctan.org/pkg/subdocs}{\textsf{subdocs}}
and \href{http://ctan.org/pkg/subfiles}{\textsf{subfiles}}
provide structures in which the main and child documents can be
encapsulated and allowing them to be compiled individually.
The inclusion mechanism is different from the conventional |\include|.
\item
The package \href{http://ctan.org/pkg/combine}{\textsf{combine}}
is an elaborate solution to combine several documents into one.
\end{itemize}
%
See also the CTAN topic \href{http://ctan.org/topic/subdocs}{\textsf{subdocs}}
for further related packages.
The present package differs from the above solutions in that
a document structure constructed with the conventional |\include| mechanism
just needs two extra commands at the top of every file
such that all constituent files can be compiled individually.

%%%%%%%%%%%%%%%%%%%%%%%%%%%%%%%%%%%%%%%%%%%%%%%%%%%%%%%%%%%%%%%%%%%%%%%%%%%%%%%%
%\subsection{Feature Suggestions}
%
%The following is a list of features which may be useful for future
%versions of this package:
%%
%\begin{itemize}
%\item
%\ldots
%\end{itemize}

%%%%%%%%%%%%%%%%%%%%%%%%%%%%%%%%%%%%%%%%%%%%%%%%%%%%%%%%%%%%%%%%%%%%%%%%%%%%%%%%
\subsection{Revision History}

%%%%%%%%%%%%%%%%%%%%%%%%%%%%%%%%%%%%%%%%
\paragraph{v2.0:} 2018/12/30

\begin{itemize}
\item
immediate forward processing
\item
added |\childdocby| mechanism
\item
manual restructured
\end{itemize}

%%%%%%%%%%%%%%%%%%%%%%%%%%%%%%%%%%%%%%%%
\paragraph{v1.6:} 2018/01/17

\begin{itemize}
\item
application for development of include files
\item
corrections to manual
\end{itemize}

%%%%%%%%%%%%%%%%%%%%%%%%%%%%%%%%%%%%%%%%
\paragraph{v1.5:} 2017/05/21

\begin{itemize}
\item
more complete structuring introduced
\item
|\childdocof| introduced
\item
|\childdoc| renamed to |\childdocmain|
\item
|\childredirect| renamed to |\childdocforward| and |\childdocforwardprefix|
and functionality expanded
\end{itemize}

%%%%%%%%%%%%%%%%%%%%%%%%%%%%%%%%%%%%%%%%
\paragraph{v1.0:} 2017/04/27

\begin{itemize}
\item
manual and install package
\item
first version published on CTAN
\end{itemize}

%%%%%%%%%%%%%%%%%%%%%%%%%%%%%%%%%%%%%%%%
\paragraph{v0.6:} 2017/04/26

\begin{itemize}
\item
redirection mechanism added
\end{itemize}

%%%%%%%%%%%%%%%%%%%%%%%%%%%%%%%%%%%%%%%%
\paragraph{v0.5:} 2017/04/26

\begin{itemize}
\item
functionality in definition file
\end{itemize}


%%%%%%%%%%%%%%%%%%%%%%%%%%%%%%%%%%%%%%%%%%%%%%%%%%%%%%%%%%%%%%%%%%%%%%%%%%%%%%%%
%%%%%%%%%%%%%%%%%%%%%%%%%%%%%%%%%%%%%%%%%%%%%%%%%%%%%%%%%%%%%%%%%%%%%%%%%%%%%%%%
%%%%%%%%%%%%%%%%%%%%%%%%%%%%%%%%%%%%%%%%%%%%%%%%%%%%%%%%%%%%%%%%%%%%%%%%%%%%%%%%
\appendix

\settowidth\MacroIndent{\rmfamily\scriptsize 000\ }

 \DocInput{childdoc.dtx}

\end{document}
%</driver>
% \fi
%
% %%%%%%%%%%%%%%%%%%%%%%%%%%%%%%%%%%%%%%%%%%%%%%%%%%%%%%%%%%%%%%%%%%%%%%%%%%%%%%
% %%%%%%%%%%%%%%%%%%%%%%%%%%%%%%%%%%%%%%%%%%%%%%%%%%%%%%%%%%%%%%%%%%%%%%%%%%%%%%
% \section{Sample}
%\iffalse
%<*samplemain>
%\fi
%
% The following presents a sample document
% with two chapters, two parts, a title page,
% a compile flag as well as three forwarding files to set the flag.
% It consists of eight |.tex| files:
% \begin{center}
% \begin{tabular}{ll}
% |cdocsamp.tex|&main file\\
% |cdocsch1.tex|&include file for chapter 1\\
% |cdocsch2.tex|&include file for chapter 2\\
% |cdocspt3.tex|&include file for part 3\\
% |cdocspt4.tex|&include file for part 4\\
% |cdocsdrf.tex|&forwarding file for main file in draft mode\\
% |cdocsfi1.tex|&forwarding file for final version of chapter 1\\
% |cdocsfi2.tex|&forwarding file for final version of chapter 2\\
% \end{tabular}
% \end{center}
% Each of the eight files can be compiled directly by the \LaTeX{} compiler.
%
% %%%%%%%%%%%%%%%%%%%%%%%%%%%%%%%%%%%%%%
% \paragraph{Main File.}
%
% The main file is called |cdocsamp.tex|.
%
% Load the \textsf{childdoc} definitions and
% declare the filename for the main document:
%    \begin{macrocode}
\input{childdoc.def}
\childdocmain{}
%    \end{macrocode}

% Optional override for |\version| flag:
%    \begin{macrocode}
%%\ifchilddoc\else\providecommand{\version}{draft}\fi
%    \end{macrocode}

% Define the default values for the |\version| flag
% (|final| for the main file and |draft| for childs):
%    \begin{macrocode}
\ifchilddoc
\providecommand{\version}{draft}
\else
\providecommand{\version}{final}
\fi
%    \end{macrocode}

% Load the standard document class:
%    \begin{macrocode}
\documentclass[12pt]{article}
%    \end{macrocode}

% Start the document body:
%    \begin{macrocode}
\begin{document}
%    \end{macrocode}

% Declare a title page.
% Print title, part of document being processed and version flag:
%    \begin{macrocode}
\addtocounter{page}{-1}
\begin{center}
{\LARGE\bfseries{}childdoc example\par}
\vspace{1cm}
\ifchilddoc
\ifchilddocmanual part\else chapter\fi:
`\childdocname' of `\childdocjob'\par
\else
main document: `\childdocjob'\par
\fi
version: \version\par
\end{center}
\newpage
%    \end{macrocode}

% Manually include selected file,
% otherwise process as usual:
%    \begin{macrocode}
\ifchilddocmanual
\section*{part `\childdocname'}
\input{\childdocname}
\else
%    \end{macrocode}

% Include the two chapters:
%    \begin{macrocode}
\include{cdocsch1}
\include{cdocsch2}
%    \end{macrocode}

% Include the two parts unless only chapters should be displayed:
%    \begin{macrocode}
\ifchilddoc\else
\section{part three}
\input{cdocspt3}
\section{part four}
\input{cdocspt4}
\fi
%    \end{macrocode}

% Process as usual until here:
%    \begin{macrocode}
\fi
%    \end{macrocode}

% End of document body:
%    \begin{macrocode}
\end{document}
%    \end{macrocode}
%\iffalse
%</samplemain>
%\fi
%
% %%%%%%%%%%%%%%%%%%%%%%%%%%%%%%%%%%%%%%
% \paragraph{Chapter Include Files.}
%
% The include files are called |cdocsch1.tex| and |cdocsch2.tex|.
%
%\iffalse
%<*samplechap1|samplechap2>
%\fi

% Optional override for |\version| flag:
%    \begin{macrocode}
%%\providecommand{\version}{final}
%    \end{macrocode}

% Include the main document:
%    \begin{macrocode}
\input{childdoc.def}
\childdocof{cdocsamp}
%    \end{macrocode}

%\iffalse
%</samplechap1|samplechap2>
%\fi
%
%\iffalse
%<*samplechap1>
%\fi
% Some text for chapter 1:
%    \begin{macrocode}
\section{one}
some text in chapter one
%    \end{macrocode}

%\iffalse
%</samplechap1>
%\fi
% Some text for chapter 2:
%\iffalse
%<*samplechap2>
%\fi
%    \begin{macrocode}
\section{two}
more text in chapter two
%    \end{macrocode}

%\iffalse
%</samplechap2>
%\fi
%
% %%%%%%%%%%%%%%%%%%%%%%%%%%%%%%%%%%%%%%
% \paragraph{Part Include Files.}
%
% The include files are called |cdocspt3.tex| and |cdocspt4.tex|.
%
%\iffalse
%<*samplepart3|samplepart4>
%\fi

% Optional override for |\version| flag:
%    \begin{macrocode}
%%\providecommand{\version}{final}
%    \end{macrocode}

% Include the main document:
%    \begin{macrocode}
\input{childdoc.def}
\childdocby{cdocsamp}
%    \end{macrocode}

%\iffalse
%</samplepart3|samplepart4>
%\fi
%
%\iffalse
%<*samplepart3>
%\fi
% Some text for part 3:
%    \begin{macrocode}
some text in part three
%    \end{macrocode}

%\iffalse
%</samplepart3>
%\fi
% Some text for part 4:
%\iffalse
%<*samplepart4>
%\fi
%    \begin{macrocode}
more text in part four
%    \end{macrocode}

%\iffalse
%</samplepart4>
%\fi
%
% %%%%%%%%%%%%%%%%%%%%%%%%%%%%%%%%%%%%%%
% \paragraph{Forwarding for a Complete Draft.}
%
% The following forwarding file |cdocsdrf.tex|
% compiles the main document in draft mode:
%\iffalse
%<*sampledraft>
%\fi
%    \begin{macrocode}
\def\version{draft}
\input{childdoc.def}
\childdocforward{cdocsamp}
%    \end{macrocode}

%\iffalse
%</sampledraft>
%\fi
%
% %%%%%%%%%%%%%%%%%%%%%%%%%%%%%%%%%%%%%%
% \paragraph{Forwarding for Final Version of the Chapters.}
%
% The following forwarding files |cdocsfn1.tex| and |cdocsfn2.tex|
% (with identical content)
% compile the final versions of the child documents
% |cdocsch1.tex| and |cdocsch2.tex|, respectively:
%\iffalse
%<*samplefinal>
%\fi
%    \begin{macrocode}
\def\version{final}
\input{childdoc.def}
\childdocforwardprefix[cdocsamp]{cdocsfn}{cdocsch}
%    \end{macrocode}

%\iffalse
%</samplefinal>
%\fi
%
% %%%%%%%%%%%%%%%%%%%%%%%%%%%%%%%%%%%%%%
% \paragraph{Command Line Processing.}
%
% The following three command lines generate the output files
% |cdocscld|, |cdocscl1| and |cdocscl2|
% which should be identical to
% |cdocsdrf|, |cdocsch1| and |cdocsfn2|, respectively:
% \begin{center}
% \begin{tabular}{l}
% |latex -jobname cdocscld \|\\
% |  "\def\version{draft}\input{childdoc.def}\childdocforward{cdocsamp}"|\\
% |latex -jobname cdocscl1 \|\\
% |  "\input{childdoc.def}\childdocforward[cdocsamp]{cdocsch1}"|\\
% |latex -jobname cdocscl2 \|\\
% |  "\def\version{final}\input{childdoc.def}\childdocforward{cdocsch2}"|
% \end{tabular}
% \end{center}
% Note that the trailing backslash on each first line
% merely continues the input to the second line
% (for convenient cut ant paste).
% Furthermore, the command |latex| can be replaced by any
% of its alternative versions such as |pdflatex|.
%
% %%%%%%%%%%%%%%%%%%%%%%%%%%%%%%%%%%%%%%%%%%%%%%%%%%%%%%%%%%%%%%%%%%%%%%%%%%%%%%
% %%%%%%%%%%%%%%%%%%%%%%%%%%%%%%%%%%%%%%%%%%%%%%%%%%%%%%%%%%%%%%%%%%%%%%%%%%%%%%
% \section{Implementation}
%\iffalse
%<*package>
%\fi
%
% This section describes the definitions file |childdoc.def|.

% The definitions cannot be loaded using |\usepackage| or |\RequirePackage|
% which has a mechanism to prevent loading a style file more than once.
% When loading the definitions by means of |\input|
% multiple instances have to be prevented manually:
%\iffalse
%This code needs to be before the `\ProvidesFile' directive
%which is defined at the beginning of this file.
%Therefore it is also placed there and commented out here.
%</package>
%<*discard>
%\fi
%    \begin{macrocode}
\ifdefined\childdocmain\endinput\fi
%    \end{macrocode}
%\iffalse
%</discard>
%<*package>
%\fi
%
% \macro{\ifchilddoc}
% \macro{\ifchilddocmanual}
% The conditional |\ifchilddoc| tells whether a
% child (true) or main (false) document is being compiled.
% The conditional |\ifchilddocmanual| tells whether
% the |\includeonly| mechanism is used (false) or
% the selection of child files must be performed manually (true).
% The definitions initialise to false:
%    \begin{macrocode}
\newif\ifchilddoc
\newif\ifchilddocmanual
%    \end{macrocode}

% \macro{\childdocname}
% \macro{\childdocjob}
% The macro |\childdocname| stores the name of the main document
% to be compiled. The macro |\childdocjob| stores the name of
% the document on which the \LaTeX{} compiler was originally invoked.
% The content of |\jobname| cannot be compared
% to filenames specified in the source due to different catcodes.
% The following code rescans |\jobname|, stores the result
% in |\childdocname| and saves a copy in |\childdocjob|:
%    \begin{macrocode}
\edef\childdocname{\scantokens\expandafter{\jobname\noexpand}}
\let\childdocjob\childdocname
%    \end{macrocode}

% \macro{\childdocdisable}
% The macro |\childdocdisable| prevents the main file
% from being processed more than once.
% At this stage, the main document command |\childdocmain|
% is assumed to be called once again where it should do nothing.
% Any subsequent call to it should prevent
% a secondary processing of the main document
% It overwrites the forwarding commands
% |\childdocof| and |\childdocforward|
% with empty macros to prevent further inclusions of the main document:
%    \begin{macrocode}
\newcommand{\childdocdisable}
{
  \renewcommand{\childdocmain}[1]{\renewcommand{\childdocmain}[1]{\endinput}}
  \renewcommand{\childdocof}[1]{}
  \renewcommand{\childdocby}[2][]{}
  \renewcommand{\childdocforward}[2][]{}
  \renewcommand{\childdocdisable}{}
}
%    \end{macrocode}

% \macro{\childdocmain}
% The macro |\childdocmain| is to be called at the top of the main file
% with nothing or the main filename (without extension) as argument.
% First, it breaks loops.
% If the argument is not empty and does not match |\childdocname|
% (which is set by the first inclusion of |childdoc.def|),
% |\ifchilddoc| is set to true, |\includeonly| is applied to the child file
% and |\jobname| is set to the main file
% (for proper handling of |.aux| files):
%    \begin{macrocode}
\newcommand{\childdocmain}[1]
{
  \childdocdisable\childdocmain{}
  \if?#1?\else
    \begingroup
      \def\childdoctmp{#1}
      \ifx\childdoctmp\childdocname
        \def\childdoctmp{}
      \else
        \def\childdoctmp
        {
          \childdoctrue
          \includeonly{\childdocname}
          \def\childdocjob{#1}
          \def\jobname{#1}
        }
      \fi
      \expandafter
    \endgroup
    \childdoctmp
  \fi
}
%    \end{macrocode}

% \macro{\childdocof}
% The command |\childdocof| redirects
% compilation to the main file |#1|.
%    \begin{macrocode}
\newcommand{\childdocof}[1]
{
  \childdocdisable
  \childdoctrue
  \includeonly{\childdocname}
  \def\jobname{#1}
  \def\childdocjob{#1}
  \input{#1}
}
%    \end{macrocode}

% \macro{\childdocby}
% The command |\childdocby| ....
%    \begin{macrocode}
\newcommand{\childdocby}[2][]
{
  \childdocdisable
  \childdoctrue
  \childdocmanualtrue
  \if?#1?\else
    \def\jobname{#2}
  \fi
  \def\childdocjob{#2}
  \input{#2}
  \endinput
}
%    \end{macrocode}

% \macro{\childdocforward}
% The command |\childdocforward| redirects
% compilation to the main file or
% (if the optional argument is given) a child file.
% Parameters are set as if the main file
% or a child file starting with |\childdocof| was compiled.
% Then compilation is handed over to the main file:
%    \begin{macrocode}
\newcommand{\childdocforward}[2][]
{
  \begingroup
    \if?#1?
      \def\childdoctmp
      {
        \def\childdocname{#2}
        \def\childdocjob{#2}
        \def\jobname{#2}
        \input{#2}
        \endinput
      }
    \else
      \def\childdoctmp
      {
        \childdocdisable
        \def\childdocname{#2}
        \childdoctrue
        \includeonly{#2}
        \def\childdocjob{#1}
        \def\jobname{#1}
        \input{#1}
        \endinput
      }
    \fi
    \expandafter
  \endgroup
  \childdoctmp
}
%    \end{macrocode}

% \macro{\childdocforwardprefix}
% The command |\childdocforwardprefix| redirects
% compilation to the main or a child file by means of a pattern.
% The prefix |#1| in the current filename is replaced by |#2|
% and the suffix of the current filename is kept
% (it is assumed that the filename does not contain the substring `|~~~|'
% which is used as a delimiter).
% Compilation is handed over to the new file by |\childdocforward|:
%    \begin{macrocode}
\newcommand{\childdocforwardprefix}[3][]
{
  \begingroup
    \def\childdocextract #2##1~~~{\def\childdoctmp{\childdocforward[#1]{#3##1}}}
    \expandafter\childdocextract\childdocname~~~
    \expandafter
  \endgroup
  \childdoctmp
}
%    \end{macrocode}

% \macro{\childdoc}
% The deprecated macro |\childdoc| is a legacy version of |\childdocmain|:
%    \begin{macrocode}
\newcommand{\childdoc}{\childdocmain}
%    \end{macrocode}

% \macro{\childdocredirect}
% The deprecated macro |\childdocredirect| is a legacy version
% of |\childdocforward| and |\childdocforwardprefix|:
%    \begin{macrocode}
\newcommand{\childdocredirect}[2][]
{
  \begingroup
    \if?#1?
      \def\childdoctmp{\childdocforward{#2}}
    \else
      \def\childdoctmp{\childdocforwardprefix{#1}{#2}}
    \fi
    \expandafter
  \endgroup
  \childdoctmp
}
%    \end{macrocode}

%\iffalse
%</package>
%\fi
%
\endinput

\childdocmain{}
%    \end{macrocode}

% Optional override for |\version| flag:
%    \begin{macrocode}
%%\ifchilddoc\else\providecommand{\version}{draft}\fi
%    \end{macrocode}

% Define the default values for the |\version| flag
% (|final| for the main file and |draft| for childs):
%    \begin{macrocode}
\ifchilddoc
\providecommand{\version}{draft}
\else
\providecommand{\version}{final}
\fi
%    \end{macrocode}

% Load the standard document class:
%    \begin{macrocode}
\documentclass[12pt]{article}
%    \end{macrocode}

% Start the document body:
%    \begin{macrocode}
\begin{document}
%    \end{macrocode}

% Declare a title page.
% Print title, part of document being processed and version flag:
%    \begin{macrocode}
\addtocounter{page}{-1}
\begin{center}
{\LARGE\bfseries{}childdoc example\par}
\vspace{1cm}
\ifchilddoc
\ifchilddocmanual part\else chapter\fi:
`\childdocname' of `\childdocjob'\par
\else
main document: `\childdocjob'\par
\fi
version: \version\par
\end{center}
\newpage
%    \end{macrocode}

% Manually include selected file,
% otherwise process as usual:
%    \begin{macrocode}
\ifchilddocmanual
\section*{part `\childdocname'}
\input{\childdocname}
\else
%    \end{macrocode}

% Include the two chapters:
%    \begin{macrocode}
\include{cdocsch1}
\include{cdocsch2}
%    \end{macrocode}

% Include the two parts unless only chapters should be displayed:
%    \begin{macrocode}
\ifchilddoc\else
\section{part three}
\input{cdocspt3}
\section{part four}
\input{cdocspt4}
\fi
%    \end{macrocode}

% Process as usual until here:
%    \begin{macrocode}
\fi
%    \end{macrocode}

% End of document body:
%    \begin{macrocode}
\end{document}
%    \end{macrocode}
%\iffalse
%</samplemain>
%\fi
%
% %%%%%%%%%%%%%%%%%%%%%%%%%%%%%%%%%%%%%%
% \paragraph{Chapter Include Files.}
%
% The include files are called |cdocsch1.tex| and |cdocsch2.tex|.
%
%\iffalse
%<*samplechap1|samplechap2>
%\fi

% Optional override for |\version| flag:
%    \begin{macrocode}
%%\providecommand{\version}{final}
%    \end{macrocode}

% Include the main document:
%    \begin{macrocode}
% \iffalse
%
% childdoc.dtx Copyright (C) 2017-2018 Niklas Beisert
%
% This work may be distributed and/or modified under the
% conditions of the LaTeX Project Public License, either version 1.3
% of this license or (at your option) any later version.
% The latest version of this license is in
%   http://www.latex-project.org/lppl.txt
% and version 1.3 or later is part of all distributions of LaTeX
% version 2005/12/01 or later.
%
% This work has the LPPL maintenance status `maintained'.
%
% The Current Maintainer of this work is Niklas Beisert.
%
% This work consists of the files childdoc.dtx and childdoc.ins
% and the derived files childdoc.def and cdocsamp.tex with
% cdocsch1.tex, cdocsch2.tex, cdocsdrf.tex, cdocsfn1.tex, cdocsfn2.tex.
%
%<package>\ifdefined\childdocmain\endinput\fi
%<package>\ProvidesFile{childdoc.def}[2018/12/30 v2.0 child document driver]
%<samplemain>\ProvidesFile{cdocsamp.tex}[2018/12/30 v2.0 sample for childdoc]
%<*driver>
%\ProvidesFile{childdoc.drv}[2018/12/30 v2.0 childdoc reference manual file]
\PassOptionsToClass{10pt,a4paper}{article}
\documentclass{ltxdoc}

\usepackage[margin=35mm]{geometry}
\usepackage{hyperref}
\usepackage{hyperxmp}
\usepackage[usenames]{color}

\hypersetup{colorlinks=true}
\hypersetup{pdfstartview=FitH}
\hypersetup{pdfpagemode=UseNone}
\hypersetup{pdfsource={}}
\hypersetup{pdflang={en-UK}}
\hypersetup{pdfcopyright={Copyright 2017-2018 Niklas Beisert.
  This work may be distributed and/or modified under the
  conditions of the LaTeX Project Public License, either version 1.3
  of this license or (at your option) any later version.}}
\hypersetup{pdflicenseurl={http://www.latex-project.org/lppl.txt}}
\hypersetup{pdfcontactaddress={ETH Zurich, ITP, HIT K,
  Wolfgang-Pauli-Strasse 27}}
\hypersetup{pdfcontactpostcode={8093}}
\hypersetup{pdfcontactcity={Zurich}}
\hypersetup{pdfcontactcountry={Switzerland}}
\hypersetup{pdfcontactemail={nbeisert@itp.phys.ethz.ch}}
\hypersetup{pdfcontacturl={http://people.phys.ethz.ch/\xmptilde nbeisert/}}

\newcommand{\secref}[1]{\hyperref[#1]{section \ref*{#1}}}

\parskip1ex
\parindent0pt
\let\olditemize\itemize
\def\itemize{\olditemize\parskip0pt}

\begin{document}

\title{The \textsf{childdoc} Package}
\hypersetup{pdftitle={The childdoc Package}}
\author{Niklas Beisert\\[2ex]
  Institut f\"ur Theoretische Physik\\
  Eidgen\"ossische Technische Hochschule Z\"urich\\
  Wolfgang-Pauli-Strasse 27, 8093 Z\"urich, Switzerland\\[1ex]
  \href{mailto:nbeisert@itp.phys.ethz.ch}
  {\texttt{nbeisert@itp.phys.ethz.ch}}}
\hypersetup{pdfauthor={Niklas Beisert}}
\hypersetup{pdfsubject={Manual for the LaTeX2e Package childdoc}}
\date{30 December 2018, \textsf{v2.0}}
\maketitle

\begin{abstract}\noindent
\textsf{childdoc} is a \LaTeXe{} package
that enables the direct compilation
of document sections included by |\include|
to individual files.
\end{abstract}

\begingroup
\parskip0ex
\tableofcontents
\endgroup

%%%%%%%%%%%%%%%%%%%%%%%%%%%%%%%%%%%%%%%%%%%%%%%%%%%%%%%%%%%%%%%%%%%%%%%%%%%%%%%%
%%%%%%%%%%%%%%%%%%%%%%%%%%%%%%%%%%%%%%%%%%%%%%%%%%%%%%%%%%%%%%%%%%%%%%%%%%%%%%%%
\section{Introduction}

\LaTeX{} provides a mechanism to structure a large document (such as a book)
into a main file and several child files (containing the chapters)
using the |\include| command.
This mechanism is beneficial for documents
which span hundreds of pages in order to
make the source file(s) more manageable.
Moreover, compilation can be restricted to
selected child files by means of the |\includeonly| command.
The latter feature can be used to reduce the compilation time while editing
(this was significantly more useful in the earlier days of \LaTeX{})
or to generate a smaller document which is easier to navigate.
Another application of |\includeonly| is to generate
documents consisting of selected parts of the complete document.

However, there are a few drawbacks of the plain |\include| mechanism:
\begin{itemize}
\item
The child files cannot be compiled on their own,
they can only be compiled via the main file.
A naive editing environment
(such as a text editor with an option
to have the current file processed by \LaTeX)
may require one to switch to the main file before compiling;
attempting to compile the child file produces errors.
\item
The main file must be modified (each time)
to adjust the |\includeonly| command
to the present needs. This easily leaves the main file in a messy state.
\item
The generated document will always carry the filename
of the main document. This is inconvenient if
several child files are to be compiled and
to be kept for distribution.
\end{itemize}

The present package provides a simple interface
to make child files individually compilable by \LaTeX{}.
Compiling a child file then has the same effect as compiling
the main file with an |\includeonly| command
to select the appropriate child.
Moreover the generated document will carry the name of the child
rather than the main file.
This resolves all three above issues.

This feature is meant to make the editing of books,
thesis documents and lecture notes somewhat more convenient.
However, the package can also be used efficiently for
composing a series of documents (such as exercise sheets)
which are typically distributed individually.
It then assists the author in generating the individual documents
(potentially in different versions)
as well as a document containing the collected series.
Another application is in developing style files
or other kinds of included material
where compilation of the style file could redirect
to a sample or test file.

%%%%%%%%%%%%%%%%%%%%%%%%%%%%%%%%%%%%%%%%%%%%%%%%%%%%%%%%%%%%%%%%%%%%%%%%%%%%%%%%
%%%%%%%%%%%%%%%%%%%%%%%%%%%%%%%%%%%%%%%%%%%%%%%%%%%%%%%%%%%%%%%%%%%%%%%%%%%%%%%%
\section{Usage}

First of all, the package \textsf{childdoc} is \emph{not} a standard
\LaTeXe{} |.sty| style file! Therefore it needs to be invoked in
a non-standard way.

%%%%%%%%%%%%%%%%%%%%%%%%%%%%%%%%%%%%%%%%%%%%%%%%%%%%%%%%%%%%%%%%%%%%%%%%%%%%%%%%
\subsection{Included Files}
\label{sec:include}

%%%%%%%%%%%%%%%%%%%%%%%%%%%%%%%%%%%%%%%%
\DescribeMacro{\childdocmain}
To use the package, add the commands
\begin{center}
\begin{tabular}{l}
|\input{childdoc.def}|\\
|\childdocmain{}|\\
\end{tabular}
\end{center}
at the very top of the main \LaTeX{} file,
in particular \emph{before} the |\documentclass| statement!
The argument of |\childdocmain| should be left empty
(but it must be present).

%%%%%%%%%%%%%%%%%%%%%%%%%%%%%%%%%%%%%%%%
\DescribeMacro{\childdocof}
Furthermore, add the commands
\begin{center}
\begin{tabular}{l}
|\input{childdoc.def}|\\
|\childdocof{|\textit{main}|}|\\
\end{tabular}
\end{center}
at the top of every child file \textit{child}
which is included by |\include{|\textit{child}|}|
from within the main file
(or at least for those files to be compiled individually).
The argument \textit{main} must be the filename of the main file.

There are a couple of
considerations in setting up the main and child documents:

%%%%%%%%%%%%%%%%%%%%%%%%%%%%%%%%%%%%%%%%
\paragraph{Restrictions.}

Please note the following restrictions:
\begin{itemize}
\item
|\childdocmain| must be called with one argument \textit{main}
to ensure compatibility with earlier version of the package.
It must either be empty (|\childdocmain{}|)
or precisely match the filename of the main file in which it is specified.
See \secref{sec:detection} for further information.
\item
The filename \textit{main} must be specified without the |.tex| extension.
\item
The filename \textit{main} is case sensitive
(even in case-insensitive file systems)
due to internal string comparison.
\item
The argument \textit{main} should be fully expanded, it cannot be a macro.
\item
Subdirectories and special characters should be avoided in filenames.
\item
The command |\childdocmain{|\textit{main}|}| must be followed by a whitespace.
It should not be followed immediately by another command
or by a comment mark `|%|'.
This is because the \TeX{} parser reads the token immediately following
the argument of |\childdocmain| and puts it
at the beginning of every child section;
however, a white\-space is ignored.
\end{itemize}

%%%%%%%%%%%%%%%%%%%%%%%%%%%%%%%%%%%%%%%%
\paragraph{Content of Main File.}

It is advisable to place all content in the child files included by |\include|.
Any output contained in the main file will appear in all child documents
unless suppressed manually;
it cannot be suppressed automatically by the |\includeonly| directive
and thus should normally be avoided.
A method to include some content in the main file
by means of conditional processing is described in \secref{sec:conditional}.

%%%%%%%%%%%%%%%%%%%%%%%%%%%%%%%%%%%%%%%%
\paragraph{Page Numbering.}

When only a part of the document is compiled,
the appropriate numbering of pages
(as well as other status parameters)
is determined from the |.aux| files.
The latter contain information from previous passes.
However this information needs to propagate through
all intermediate child documents.
Therefore the page numbering in child documents may well
be inconsistent until the complete document is compiled at least once.

A useful (if unconventional) way to always ensure a consistent
page numbering is to restart the numbering in each child document
and denote the pages by `\textit{child}|.|\textit{page}'
where \textit{child} represents the chapter/section number of the child file.
This can be achieved by the command
|\numberwithin{page}{|\textit{child}|}|
of the \textsf{amsmath} package
where \textit{child} can be |chapter| or |section|
depending on the chosen structuring.
Alternatively, one can modify the macro |\thepage| appropriately
and reset the counter |page| at the start of each child file.

%%%%%%%%%%%%%%%%%%%%%%%%%%%%%%%%%%%%%%%%%%%%%%%%%%%%%%%%%%%%%%%%%%%%%%%%%%%%%%%%
\subsection{Conditional Processing}
\label{sec:conditional}

The package provides a mechanism to compile different versions
of a document. To customise the versions further some conditional processing
can come in handy to distinguish which version is being compiled.
The package provides two macros to describe the compilation context:

%%%%%%%%%%%%%%%%%%%%%%%%%%%%%%%%%%%%%%%%
\DescribeMacro{\ifchilddoc}
The conditional |\ifchilddoc| distinguishes between the compilation of
child documents and the main document:
%
\begin{center}
|\ifchilddoc |\textit{child-code}| |[|\||else |\textit{main-code}]| \||fi|
\end{center}

%%%%%%%%%%%%%%%%%%%%%%%%%%%%%%%%%%%%%%%%
\DescribeMacro{\childdocname}
\DescribeMacro{\childdocjob}
The macro |\childdocname| contains the filename (without extension)
of the main or child file being processed.
Note that |\childdocjob| will always contain the name of the main file.

%%%%%%%%%%%%%%%%%%%%%%%%%%%%%%%%%%%%%%%%
\paragraph{Title Page.}

Conditional processing can be used to include a title or banner page
in the main document when proper precautions are taken.
Importantly, the code in the main file should ensure that the page counter
(as well as other status parameters which are stored in the |.aux| files)
takes the same value after the conditional processing.
Otherwise the page numbers may take divergent values
depending on which part is compiled.

For example, a title page could be declared by:
%
\begin{center}
\begin{tabular}{l}
|\ifchilddoc\||else|\\
|\addtocounter{page}{-1}|\\
\textit{code for title page}\\
|\newpage|\\
|\||fi|
\end{tabular}
\end{center}
%
A banner page for the child documents can be generated by:
%
\begin{center}
\begin{tabular}{l}
|\ifchilddoc|\\
|\addtocounter{page}{-1}|\\
\textit{code for banner page}\\
|\newpage|\\
|\||fi|
\end{tabular}
\end{center}
%
Here one could write a message such as:
\begin{center}
|This is the part \childdocname{} of \childdocjob{}.|
\end{center}

%%%%%%%%%%%%%%%%%%%%%%%%%%%%%%%%%%%%%%%%%%%%%%%%%%%%%%%%%%%%%%%%%%%%%%%%%%%%%%%%
\subsection{Flags}
\label{sec:flags}

The package makes it easy to generate different versions
of the main or child documents.
To this end compilation flags can be defined
and assigned different default values.
They will be particularly useful in conjunction
with the forwarding mechanism described in \secref{sec:forward}.

For example, it may be useful to have a flag |\version|
which can be set to |draft| or |final|.
The document source will contain some conditional code
depending on the value of |\version|.
Suppose further, the flag should default to |final| for the main file
and to |draft| for child files
which is a natural assignment for editing the document.
This is achieved by placing the following code
in the preamble of the main document
(below the |\childdocmain| directive):
%
\begin{center}
\begin{tabular}{l}
|\ifchilddoc|\\
|\providecommand{\version}{draft}|\\
|\||else|\\
|\providecommand{\version}{final}|\\
|\||fi|
\end{tabular}
\end{center}
%
The definition by |\providecommand| makes sure
that previous definitions are not overwritten.
Further statements |\providecommand{\version}{...}|
can thus be added before the above code to override it.

For the main file, one might add a line
(between |\childdocmain| and the above block)
%
\begin{center}
|%\ifchilddoc\||else\providecommand{\version}{draft}\||fi|
\end{center}
%
which can be uncommented to produce a draft version.
Likewise one can add a line to the very top of a child file
(above the |\childdocof{|\textit{main}|}| directive)
%
\begin{center}
|%\providecommand{\version}{final}|
\end{center}
%
which can be uncommented to produce the final version of this child document.

%%%%%%%%%%%%%%%%%%%%%%%%%%%%%%%%%%%%%%%%%%%%%%%%%%%%%%%%%%%%%%%%%%%%%%%%%%%%%%%%
\subsection{Forwarding}
\label{sec:forward}

Different versions of the main or child documents
using compilation flags as described in \secref{sec:flags}
can be (permanently) stored in different files
for convenient compilation, viewing and distribution.
To this end, the package defines a command
to pass on compilation to a different file:

%%%%%%%%%%%%%%%%%%%%%%%%%%%%%%%%%%%%%%%%
\DescribeMacro{\childdocforward}
The command |\childdocforward| redirects processing to
another source file:
%
\begin{center}
\begin{tabular}{l}
|\input{childdoc.def}|\\
|\childdocforward[|\textit{main}|]{|\textit{dest}|}|\\
\end{tabular}
\end{center}
%
The argument \textit{dest} is the destination file
(without extension).
It should be the main file or one of the child files.
Note that further \textsf{childdoc} directives
such as |\childdocof| and |\childdocforward|
in the indicated file will be processed in this form.
The optional argument \textit{main}
passes on directly to the main file \textit{main}
while pretending to compile the child \textit{dest}.
This form behaves as if \textit{dest}
issues |\childdocof{|\textit{main}|}| right away,
and no further \textsf{childdoc} directives will be processed.

%%%%%%%%%%%%%%%%%%%%%%%%%%%%%%%%%%%%%%%%
\DescribeMacro{\...prefix}
In the alternative form |\childdocforwardprefix|,
%
\begin{center}
\begin{tabular}{l}
|\input{childdoc.def}|\\
|\childdocforwardprefix[|\textit{main}|]{|\textit{prefix}|}{|\textit{dest}|}|
\end{tabular}
\end{center}
%
the destination file is determined by a pattern
depending on the current file:
To make this work, the current file must be called
`{\textit{prefix}\hspace{0.2em}\textit{suffix}}'
with \textit{prefix} matching precisely the argument.
Processing is then passed on to the file
`{\textit{dest}\hspace{0.2em}\textit{suffix}}'.
Surely, the same effect is achieved by
directly specifying the
argument `{\textit{dest}\hspace{0.2em}\textit{suffix}}'
in the first form.
However, that requires to set up a different file
for each child. With the alternative form of the command
all these files can have exactly the same content
which simplifies setting them up and maintaining them.

For example, the following file |draft.tex|
with a compilation flag |\version| as described in \secref{sec:flags}
compiles the main document as a draft:
%
\begin{center}
\begin{tabular}{l}
|\def\version{draft}|\\
|\input{childdoc.def}|\\
|\childdocforward{|\textit{main}|}|
\end{tabular}
\end{center}
%
Likewise, the following files |final|\textit{nn}|.tex|
compile the final version of the child document
|child|\textit{nn}|.tex|:
%
\begin{center}
\begin{tabular}{l}
|\def\version{final}|\\
|\input{childdoc.def}|\\
|\childdocforwardprefix{final}{child}|
\end{tabular}
\end{center}
%

Note that when several versions of a main file and/or of each child file
are to be generated, it may be convenient to set up a |Makefile| or
shell script to automatise the process.

%%%%%%%%%%%%%%%%%%%%%%%%%%%%%%%%%%%%%%%%%%%%%%%%%%%%%%%%%%%%%%%%%%%%%%%%%%%%%%%%
\subsection{Command Line Processing}
\label{sec:commandline}

The effect of redirection files can also be achieved by invoking
the \LaTeX{} compiler with a more elaborate command line.
Most conveniently this should be done as part
of a shell script or a |Makefile|.

When using \textsf{childdoc} in the main file, the following
command lines effectively perform a redirection
(note that depending on the shell being used,
backslashes may have to be doubled: `|\|' $\to$ `|\\|'):
%
\begin{center}
|... -jobname "|\textit{target}|" |\\|"|[\textit{flags}]%
|\input{childdoc.def}\childdocforward[|\textit{main}|]{|\textit{dest}|}"|
\end{center}
%
Here \textit{target} is the name of the output file,
\textit{main} is the name of the main file
and \textit{dest} is the name of the main or child file to be processed
(all filenames without extensions).
The optional argument \textit{main} can be omitted
if \textit{main} matches \textit{dest}.
Optionally, compilation \textit{flags} can be defined via |\def| commands.
This command line makes the \TeX{} engine believe
it is compiling the file \textit{target}
whose content is specified as the latter parameter.
The provided code then forwards the processing to
\textit{main} or \textit{dest} as described in \secref{sec:forward}.

%%%%%%%%%%%%%%%%%%%%%%%%%%%%%%%%%%%%%%%%%%%%%%%%%%%%%%%%%%%%%%%%%%%%%%%%%%%%%%%%
\subsection{Include by Input}
\label{sec:input}

Including child documents by |\include| has some restrictions by design.
Most notably, the content of a child document always occupies
its own set of pages; pages cannot be shared between child documents.
Usually, this behaviour makes perfect sense
because each child document contain an essential part of the document.
However, in some situations it may be desirable to compose
a document from a collection of parts
without having mandatory page breaks between then.
For this case, the package
provides a mechanism to include parts
by |\input| which can also be processed individually.
However, by construction this mechanism
requires manual handling of the content to be output.

%%%%%%%%%%%%%%%%%%%%%%%%%%%%%%%%%%%%%%%%
\DescribeMacro{\ifchilddocmanual}
The main file should be prepared as usual, see \secref{sec:include}.
However, the document body must make a distinction
between processing of an individual part and of the main document, e.g.:
%
\begin{center}
\begin{tabular}{l}
|\ifchilddocmanual|\\
|\input{\childdocname}|\\
|\||else|\\
\textit{document body with }|\input{|\textit{part}|}|\\
|\||fi|
\end{tabular}
\end{center}
%
The conditional |\ifchilddocmanual| is true whenever
a part to be included by |\input| is being compiled,
and the name of the part is stored in |\childdocname|.

%%%%%%%%%%%%%%%%%%%%%%%%%%%%%%%%%%%%%%%%
\DescribeMacro{\childdocby}
Each part to be included by |\input| should start with:
%
\begin{center}
\begin{tabular}{l}
|\input{childdoc.def}|\\
|\childdocby{|\textit{main}|}|\\
\end{tabular}
\end{center}
%
The directive |\childdocby| is similar to |\childdocof|
described in \secref{sec:include},
but the subsequent selection of content must be done manually.
To that end, both |\ifchilddoc| and |\ifchilddocmanual|
will be true upon processing of a part,
and the name of the part is stored in |\childdocname|.
Note that |\jobname| will be set to the filename of the current part
so that each part receives an individual |.aux| file
that does not interfere with the |.aux| file(s) of the main document.
This behaviour can be altered by the alternative form
|\childdocby[*]{|\textit{main}|}| (with a non-empty optional argument)
which uses the |.aux| file of the main document
by setting |\jobname| to \textit{main}.

%%%%%%%%%%%%%%%%%%%%%%%%%%%%%%%%%%%%%%%%%%%%%%%%%%%%%%%%%%%%%%%%%%%%%%%%%%%%%%%%
\subsection{Driver Development}
\label{sec:driver}

The \textsf{childdoc} mechanism can also be use for the development
of definition files such as \LaTeX{} styles or classes.
This case differs from the above setup with multiple parts
included by |\include| in that no |\includeonly| should be invoked.
This can be achieved by starting the include file
(before |\ProvidesPackage|) with:
%
\begin{center}
\begin{tabular}{l}
|\input{childdoc.def}|\\
|\childdocforward{|\textit{main}|}|\\
\end{tabular}
\end{center}
%
or alternatively with:
%
\begin{center}
\begin{tabular}{l}
|\input{childdoc.def}|\\
|\childdocby{|\textit{main}|}|\\
\end{tabular}
\end{center}
%
Both forms have slightly different effects as described above.
The main file is prepared as usual, see \secref{sec:include}.

%%%%%%%%%%%%%%%%%%%%%%%%%%%%%%%%%%%%%%%%%%%%%%%%%%%%%%%%%%%%%%%%%%%%%%%%%%%%%%%%
\subsection{Legacy Detection}
\label{sec:detection}

The directive |\childdocmain| in the main file can detect
whether the complete document or merely a child is to be compiled
even without using the directive |\childdocof|.
This method is deprecated because it is less robust
and there is no compelling reason to use it;
it is merely provided for backward compatibility
and it may be removed in future versions.

If the detection mechanism is to be used,
it is mandatory to correctly specify
the filename of the main file as the argument of |\childdocmain|:
%
\begin{center}
\begin{tabular}{l}
|\input{childdoc.def}|\\
|\childdocmain{|\textit{main}|}|\\
\end{tabular}
\end{center}
%
If |\jobname| does not match the argument \textit{main} of |\childdocmain|,
it is assumed that |\jobname| points to the child file to be compiled.
When using |\childdocmain| with the main file specified as argument,
it suffices to start a child file
with just |\input{|\textit{main}|}|
without loading of the package and using |\childdocof|.
If instead all processing is done
with the appropriate \textsf{childdoc} directives,
the argument of \textit{main} of |\childdocmain| can be empty.

An alternative version of the command line processing described
in \secref{sec:commandline} using the detection mechanism reads:
%
\begin{center}
|... -jobname "|\textit{target}|" "|[\textit{flags}]%
[|\def\jobname{|\textit{dest}|}|]|\input{|\textit{main}|}"|
\end{center}

%%%%%%%%%%%%%%%%%%%%%%%%%%%%%%%%%%%%%%%%%%%%%%%%%%%%%%%%%%%%%%%%%%%%%%%%%%%%%%%%
\subsection{Manual Code}
\label{sec:manual}

In case one cannot be certain whether the definitions file |childdoc.def|
is installed on the target \TeX{} distribution
and one prefers not to ship it,
it is conceivable to paste a few relevant commands into the sources.

To that end, drop all statements |\input{childdoc.def}|
and perform the replacements as outlined below.
Instead of |\childdocmain{|\textit{main}|}| add the following code
to the top of the main file:
%
\begin{center}
\begin{tabular}{l}
|\||ifdefined\childdocname\endinput\||fi\newif\ifchilddoc|\\
|\edef\childdocname{\scantokens\expandafter{\jobname\noexpand}}|\\
|\def\childdocmain{|\textit{main}|}\||ifx\childdocmain\childdocname\||else|\\
|\childdoctrue\includeonly{\childdocname}\let\jobname\childdocmain\||fi|\\
\end{tabular}
\end{center}
%
Instead of |\childdocof{|\textit{main}|}| just include the main file
at the top of each child file:
%
\begin{center}
|\input{|\textit{main}|}|
\end{center}
%
A simple redirection |\childdocforward{|\textit{dest}|}| is achieved by:
%
\begin{center}
|\def\jobname{|\textit{dest}|}\input{\jobname}|
\end{center}
%
The redirection with prefix
|\childdocforwardprefix[|\textit{prefix}|]{|\textit{dest}|}|
is accomplished by:
%
\begin{center}
\begin{tabular}{l}
|{\edef\jobname{\scantokens\expandafter{\jobname\noexpand}}|\\
|\def\redirectjob |\textit{prefix}|#1~~~{\gdef\jobname{|\textit{dest}|#1}}|\\
|\expandafter\redirectjob\jobname~~~}\input{\jobname}|
\end{tabular}
\end{center}

In an alternative approach,
child documents can be compiled by a specific command line
without additional code or specific definitions:
%
\begin{center}
|... -jobname "|\textit{target}|" "|[\textit{flags}]%
|\includeonly{|\textit{dest}|}\input{|\textit{main}|}"|
\end{center}
%

%%%%%%%%%%%%%%%%%%%%%%%%%%%%%%%%%%%%%%%%%%%%%%%%%%%%%%%%%%%%%%%%%%%%%%%%%%%%%%%%
%%%%%%%%%%%%%%%%%%%%%%%%%%%%%%%%%%%%%%%%%%%%%%%%%%%%%%%%%%%%%%%%%%%%%%%%%%%%%%%%
\section{Information}

%%%%%%%%%%%%%%%%%%%%%%%%%%%%%%%%%%%%%%%%%%%%%%%%%%%%%%%%%%%%%%%%%%%%%%%%%%%%%%%%
\subsection{Copyright}

Copyright \copyright{} 2017--2018 Niklas Beisert

This work may be distributed and/or modified under the
conditions of the \LaTeX{} Project Public License, either version 1.3
of this license or (at your option) any later version.
The latest version of this license is in
  \url{http://www.latex-project.org/lppl.txt}
and version 1.3 or later is part of all distributions of \LaTeX{}
version 2005/12/01 or later.

This work has the LPPL maintenance status `maintained'.

The Current Maintainer of this work is Niklas Beisert.

This work consists of the files |README.txt|, |childdoc.ins| and |childdoc.dtx|
as well as the derived files |childdoc.def|, |cdocsamp.tex|
with |cdocsch1.tex|, |cdocsch2.tex|, |cdocspt3.tex|, |cdocspt4.tex|,
|cdocsdrf.tex|, |cdocsfn1.tex|, |cdocsfn2.tex|
as well as |childdoc.pdf|.

%%%%%%%%%%%%%%%%%%%%%%%%%%%%%%%%%%%%%%%%%%%%%%%%%%%%%%%%%%%%%%%%%%%%%%%%%%%%%%%%
\subsection{Files and Installation}

The package consists of the files:
%
\begin{center}
\begin{tabular}{ll}
    |README.txt|   & readme file \\
    |childdoc.ins| & installation file \\
    |childdoc.dtx| & source file \\
    |childdoc.def| & definition file \\
    |cdocsamp.tex| & sample main file \\
    |cdocsch1.tex| & sample include file \\
    |cdocsch2.tex| & sample include file \\
    |cdocspt3.tex| & sample part file \\
    |cdocspt4.tex| & sample part file \\
    |cdocsdrf.tex| & sample redirection file \\
    |cdocsfn1.tex| & sample redirection file \\
    |cdocsfn2.tex| & sample redirection file \\
    |childdoc.pdf| & manual
\end{tabular}
\end{center}
%
The distribution consists of the files
|README.txt|, |childdoc.ins| and |childdoc.dtx|.
%
\begin{itemize}
\item
Run (pdf)\LaTeX{} on |childdoc.dtx|
to compile the manual |childdoc.pdf| (this file).
\item
Run \LaTeX{} on |childdoc.ins| to create the definitions file |childdoc.def|
and the sample |cdocsamp.tex| with include files
|cdocsch1.tex|, |cdocsch2.tex|, |cdocspt3.tex|, |cdocspt4.tex|,
|cdocsdrf.tex|, |cdocsfn1.tex|, |cdocsfn2.tex|.
Then copy the file |childdoc.def| to an appropriate directory of your \LaTeX{}
distribution, e.g.\ \textit{texmf-root}|/tex/latex/childdoc|.
\end{itemize}

%%%%%%%%%%%%%%%%%%%%%%%%%%%%%%%%%%%%%%%%%%%%%%%%%%%%%%%%%%%%%%%%%%%%%%%%%%%%%%%%
\subsection{Related CTAN Packages}

There are several other packages which offer a similar functionality:
%
\begin{itemize}
\item
The packages
\href{http://ctan.org/pkg/docmute}{\textsf{docmute}},
\href{http://ctan.org/pkg/includex}{\textsf{includex}} and
\href{http://ctan.org/pkg/standalone}{\textsf{standalone}}
provide commands to include only the document body of
a child file thus allowing both files to be compiled individually.
\item
The packages \href{http://ctan.org/pkg/subdocs}{\textsf{subdocs}}
and \href{http://ctan.org/pkg/subfiles}{\textsf{subfiles}}
provide structures in which the main and child documents can be
encapsulated and allowing them to be compiled individually.
The inclusion mechanism is different from the conventional |\include|.
\item
The package \href{http://ctan.org/pkg/combine}{\textsf{combine}}
is an elaborate solution to combine several documents into one.
\end{itemize}
%
See also the CTAN topic \href{http://ctan.org/topic/subdocs}{\textsf{subdocs}}
for further related packages.
The present package differs from the above solutions in that
a document structure constructed with the conventional |\include| mechanism
just needs two extra commands at the top of every file
such that all constituent files can be compiled individually.

%%%%%%%%%%%%%%%%%%%%%%%%%%%%%%%%%%%%%%%%%%%%%%%%%%%%%%%%%%%%%%%%%%%%%%%%%%%%%%%%
%\subsection{Feature Suggestions}
%
%The following is a list of features which may be useful for future
%versions of this package:
%%
%\begin{itemize}
%\item
%\ldots
%\end{itemize}

%%%%%%%%%%%%%%%%%%%%%%%%%%%%%%%%%%%%%%%%%%%%%%%%%%%%%%%%%%%%%%%%%%%%%%%%%%%%%%%%
\subsection{Revision History}

%%%%%%%%%%%%%%%%%%%%%%%%%%%%%%%%%%%%%%%%
\paragraph{v2.0:} 2018/12/30

\begin{itemize}
\item
immediate forward processing
\item
added |\childdocby| mechanism
\item
manual restructured
\end{itemize}

%%%%%%%%%%%%%%%%%%%%%%%%%%%%%%%%%%%%%%%%
\paragraph{v1.6:} 2018/01/17

\begin{itemize}
\item
application for development of include files
\item
corrections to manual
\end{itemize}

%%%%%%%%%%%%%%%%%%%%%%%%%%%%%%%%%%%%%%%%
\paragraph{v1.5:} 2017/05/21

\begin{itemize}
\item
more complete structuring introduced
\item
|\childdocof| introduced
\item
|\childdoc| renamed to |\childdocmain|
\item
|\childredirect| renamed to |\childdocforward| and |\childdocforwardprefix|
and functionality expanded
\end{itemize}

%%%%%%%%%%%%%%%%%%%%%%%%%%%%%%%%%%%%%%%%
\paragraph{v1.0:} 2017/04/27

\begin{itemize}
\item
manual and install package
\item
first version published on CTAN
\end{itemize}

%%%%%%%%%%%%%%%%%%%%%%%%%%%%%%%%%%%%%%%%
\paragraph{v0.6:} 2017/04/26

\begin{itemize}
\item
redirection mechanism added
\end{itemize}

%%%%%%%%%%%%%%%%%%%%%%%%%%%%%%%%%%%%%%%%
\paragraph{v0.5:} 2017/04/26

\begin{itemize}
\item
functionality in definition file
\end{itemize}


%%%%%%%%%%%%%%%%%%%%%%%%%%%%%%%%%%%%%%%%%%%%%%%%%%%%%%%%%%%%%%%%%%%%%%%%%%%%%%%%
%%%%%%%%%%%%%%%%%%%%%%%%%%%%%%%%%%%%%%%%%%%%%%%%%%%%%%%%%%%%%%%%%%%%%%%%%%%%%%%%
%%%%%%%%%%%%%%%%%%%%%%%%%%%%%%%%%%%%%%%%%%%%%%%%%%%%%%%%%%%%%%%%%%%%%%%%%%%%%%%%
\appendix

\settowidth\MacroIndent{\rmfamily\scriptsize 000\ }

 \DocInput{childdoc.dtx}

\end{document}
%</driver>
% \fi
%
% %%%%%%%%%%%%%%%%%%%%%%%%%%%%%%%%%%%%%%%%%%%%%%%%%%%%%%%%%%%%%%%%%%%%%%%%%%%%%%
% %%%%%%%%%%%%%%%%%%%%%%%%%%%%%%%%%%%%%%%%%%%%%%%%%%%%%%%%%%%%%%%%%%%%%%%%%%%%%%
% \section{Sample}
%\iffalse
%<*samplemain>
%\fi
%
% The following presents a sample document
% with two chapters, two parts, a title page,
% a compile flag as well as three forwarding files to set the flag.
% It consists of eight |.tex| files:
% \begin{center}
% \begin{tabular}{ll}
% |cdocsamp.tex|&main file\\
% |cdocsch1.tex|&include file for chapter 1\\
% |cdocsch2.tex|&include file for chapter 2\\
% |cdocspt3.tex|&include file for part 3\\
% |cdocspt4.tex|&include file for part 4\\
% |cdocsdrf.tex|&forwarding file for main file in draft mode\\
% |cdocsfi1.tex|&forwarding file for final version of chapter 1\\
% |cdocsfi2.tex|&forwarding file for final version of chapter 2\\
% \end{tabular}
% \end{center}
% Each of the eight files can be compiled directly by the \LaTeX{} compiler.
%
% %%%%%%%%%%%%%%%%%%%%%%%%%%%%%%%%%%%%%%
% \paragraph{Main File.}
%
% The main file is called |cdocsamp.tex|.
%
% Load the \textsf{childdoc} definitions and
% declare the filename for the main document:
%    \begin{macrocode}
\input{childdoc.def}
\childdocmain{}
%    \end{macrocode}

% Optional override for |\version| flag:
%    \begin{macrocode}
%%\ifchilddoc\else\providecommand{\version}{draft}\fi
%    \end{macrocode}

% Define the default values for the |\version| flag
% (|final| for the main file and |draft| for childs):
%    \begin{macrocode}
\ifchilddoc
\providecommand{\version}{draft}
\else
\providecommand{\version}{final}
\fi
%    \end{macrocode}

% Load the standard document class:
%    \begin{macrocode}
\documentclass[12pt]{article}
%    \end{macrocode}

% Start the document body:
%    \begin{macrocode}
\begin{document}
%    \end{macrocode}

% Declare a title page.
% Print title, part of document being processed and version flag:
%    \begin{macrocode}
\addtocounter{page}{-1}
\begin{center}
{\LARGE\bfseries{}childdoc example\par}
\vspace{1cm}
\ifchilddoc
\ifchilddocmanual part\else chapter\fi:
`\childdocname' of `\childdocjob'\par
\else
main document: `\childdocjob'\par
\fi
version: \version\par
\end{center}
\newpage
%    \end{macrocode}

% Manually include selected file,
% otherwise process as usual:
%    \begin{macrocode}
\ifchilddocmanual
\section*{part `\childdocname'}
\input{\childdocname}
\else
%    \end{macrocode}

% Include the two chapters:
%    \begin{macrocode}
\include{cdocsch1}
\include{cdocsch2}
%    \end{macrocode}

% Include the two parts unless only chapters should be displayed:
%    \begin{macrocode}
\ifchilddoc\else
\section{part three}
\input{cdocspt3}
\section{part four}
\input{cdocspt4}
\fi
%    \end{macrocode}

% Process as usual until here:
%    \begin{macrocode}
\fi
%    \end{macrocode}

% End of document body:
%    \begin{macrocode}
\end{document}
%    \end{macrocode}
%\iffalse
%</samplemain>
%\fi
%
% %%%%%%%%%%%%%%%%%%%%%%%%%%%%%%%%%%%%%%
% \paragraph{Chapter Include Files.}
%
% The include files are called |cdocsch1.tex| and |cdocsch2.tex|.
%
%\iffalse
%<*samplechap1|samplechap2>
%\fi

% Optional override for |\version| flag:
%    \begin{macrocode}
%%\providecommand{\version}{final}
%    \end{macrocode}

% Include the main document:
%    \begin{macrocode}
\input{childdoc.def}
\childdocof{cdocsamp}
%    \end{macrocode}

%\iffalse
%</samplechap1|samplechap2>
%\fi
%
%\iffalse
%<*samplechap1>
%\fi
% Some text for chapter 1:
%    \begin{macrocode}
\section{one}
some text in chapter one
%    \end{macrocode}

%\iffalse
%</samplechap1>
%\fi
% Some text for chapter 2:
%\iffalse
%<*samplechap2>
%\fi
%    \begin{macrocode}
\section{two}
more text in chapter two
%    \end{macrocode}

%\iffalse
%</samplechap2>
%\fi
%
% %%%%%%%%%%%%%%%%%%%%%%%%%%%%%%%%%%%%%%
% \paragraph{Part Include Files.}
%
% The include files are called |cdocspt3.tex| and |cdocspt4.tex|.
%
%\iffalse
%<*samplepart3|samplepart4>
%\fi

% Optional override for |\version| flag:
%    \begin{macrocode}
%%\providecommand{\version}{final}
%    \end{macrocode}

% Include the main document:
%    \begin{macrocode}
\input{childdoc.def}
\childdocby{cdocsamp}
%    \end{macrocode}

%\iffalse
%</samplepart3|samplepart4>
%\fi
%
%\iffalse
%<*samplepart3>
%\fi
% Some text for part 3:
%    \begin{macrocode}
some text in part three
%    \end{macrocode}

%\iffalse
%</samplepart3>
%\fi
% Some text for part 4:
%\iffalse
%<*samplepart4>
%\fi
%    \begin{macrocode}
more text in part four
%    \end{macrocode}

%\iffalse
%</samplepart4>
%\fi
%
% %%%%%%%%%%%%%%%%%%%%%%%%%%%%%%%%%%%%%%
% \paragraph{Forwarding for a Complete Draft.}
%
% The following forwarding file |cdocsdrf.tex|
% compiles the main document in draft mode:
%\iffalse
%<*sampledraft>
%\fi
%    \begin{macrocode}
\def\version{draft}
\input{childdoc.def}
\childdocforward{cdocsamp}
%    \end{macrocode}

%\iffalse
%</sampledraft>
%\fi
%
% %%%%%%%%%%%%%%%%%%%%%%%%%%%%%%%%%%%%%%
% \paragraph{Forwarding for Final Version of the Chapters.}
%
% The following forwarding files |cdocsfn1.tex| and |cdocsfn2.tex|
% (with identical content)
% compile the final versions of the child documents
% |cdocsch1.tex| and |cdocsch2.tex|, respectively:
%\iffalse
%<*samplefinal>
%\fi
%    \begin{macrocode}
\def\version{final}
\input{childdoc.def}
\childdocforwardprefix[cdocsamp]{cdocsfn}{cdocsch}
%    \end{macrocode}

%\iffalse
%</samplefinal>
%\fi
%
% %%%%%%%%%%%%%%%%%%%%%%%%%%%%%%%%%%%%%%
% \paragraph{Command Line Processing.}
%
% The following three command lines generate the output files
% |cdocscld|, |cdocscl1| and |cdocscl2|
% which should be identical to
% |cdocsdrf|, |cdocsch1| and |cdocsfn2|, respectively:
% \begin{center}
% \begin{tabular}{l}
% |latex -jobname cdocscld \|\\
% |  "\def\version{draft}\input{childdoc.def}\childdocforward{cdocsamp}"|\\
% |latex -jobname cdocscl1 \|\\
% |  "\input{childdoc.def}\childdocforward[cdocsamp]{cdocsch1}"|\\
% |latex -jobname cdocscl2 \|\\
% |  "\def\version{final}\input{childdoc.def}\childdocforward{cdocsch2}"|
% \end{tabular}
% \end{center}
% Note that the trailing backslash on each first line
% merely continues the input to the second line
% (for convenient cut ant paste).
% Furthermore, the command |latex| can be replaced by any
% of its alternative versions such as |pdflatex|.
%
% %%%%%%%%%%%%%%%%%%%%%%%%%%%%%%%%%%%%%%%%%%%%%%%%%%%%%%%%%%%%%%%%%%%%%%%%%%%%%%
% %%%%%%%%%%%%%%%%%%%%%%%%%%%%%%%%%%%%%%%%%%%%%%%%%%%%%%%%%%%%%%%%%%%%%%%%%%%%%%
% \section{Implementation}
%\iffalse
%<*package>
%\fi
%
% This section describes the definitions file |childdoc.def|.

% The definitions cannot be loaded using |\usepackage| or |\RequirePackage|
% which has a mechanism to prevent loading a style file more than once.
% When loading the definitions by means of |\input|
% multiple instances have to be prevented manually:
%\iffalse
%This code needs to be before the `\ProvidesFile' directive
%which is defined at the beginning of this file.
%Therefore it is also placed there and commented out here.
%</package>
%<*discard>
%\fi
%    \begin{macrocode}
\ifdefined\childdocmain\endinput\fi
%    \end{macrocode}
%\iffalse
%</discard>
%<*package>
%\fi
%
% \macro{\ifchilddoc}
% \macro{\ifchilddocmanual}
% The conditional |\ifchilddoc| tells whether a
% child (true) or main (false) document is being compiled.
% The conditional |\ifchilddocmanual| tells whether
% the |\includeonly| mechanism is used (false) or
% the selection of child files must be performed manually (true).
% The definitions initialise to false:
%    \begin{macrocode}
\newif\ifchilddoc
\newif\ifchilddocmanual
%    \end{macrocode}

% \macro{\childdocname}
% \macro{\childdocjob}
% The macro |\childdocname| stores the name of the main document
% to be compiled. The macro |\childdocjob| stores the name of
% the document on which the \LaTeX{} compiler was originally invoked.
% The content of |\jobname| cannot be compared
% to filenames specified in the source due to different catcodes.
% The following code rescans |\jobname|, stores the result
% in |\childdocname| and saves a copy in |\childdocjob|:
%    \begin{macrocode}
\edef\childdocname{\scantokens\expandafter{\jobname\noexpand}}
\let\childdocjob\childdocname
%    \end{macrocode}

% \macro{\childdocdisable}
% The macro |\childdocdisable| prevents the main file
% from being processed more than once.
% At this stage, the main document command |\childdocmain|
% is assumed to be called once again where it should do nothing.
% Any subsequent call to it should prevent
% a secondary processing of the main document
% It overwrites the forwarding commands
% |\childdocof| and |\childdocforward|
% with empty macros to prevent further inclusions of the main document:
%    \begin{macrocode}
\newcommand{\childdocdisable}
{
  \renewcommand{\childdocmain}[1]{\renewcommand{\childdocmain}[1]{\endinput}}
  \renewcommand{\childdocof}[1]{}
  \renewcommand{\childdocby}[2][]{}
  \renewcommand{\childdocforward}[2][]{}
  \renewcommand{\childdocdisable}{}
}
%    \end{macrocode}

% \macro{\childdocmain}
% The macro |\childdocmain| is to be called at the top of the main file
% with nothing or the main filename (without extension) as argument.
% First, it breaks loops.
% If the argument is not empty and does not match |\childdocname|
% (which is set by the first inclusion of |childdoc.def|),
% |\ifchilddoc| is set to true, |\includeonly| is applied to the child file
% and |\jobname| is set to the main file
% (for proper handling of |.aux| files):
%    \begin{macrocode}
\newcommand{\childdocmain}[1]
{
  \childdocdisable\childdocmain{}
  \if?#1?\else
    \begingroup
      \def\childdoctmp{#1}
      \ifx\childdoctmp\childdocname
        \def\childdoctmp{}
      \else
        \def\childdoctmp
        {
          \childdoctrue
          \includeonly{\childdocname}
          \def\childdocjob{#1}
          \def\jobname{#1}
        }
      \fi
      \expandafter
    \endgroup
    \childdoctmp
  \fi
}
%    \end{macrocode}

% \macro{\childdocof}
% The command |\childdocof| redirects
% compilation to the main file |#1|.
%    \begin{macrocode}
\newcommand{\childdocof}[1]
{
  \childdocdisable
  \childdoctrue
  \includeonly{\childdocname}
  \def\jobname{#1}
  \def\childdocjob{#1}
  \input{#1}
}
%    \end{macrocode}

% \macro{\childdocby}
% The command |\childdocby| ....
%    \begin{macrocode}
\newcommand{\childdocby}[2][]
{
  \childdocdisable
  \childdoctrue
  \childdocmanualtrue
  \if?#1?\else
    \def\jobname{#2}
  \fi
  \def\childdocjob{#2}
  \input{#2}
  \endinput
}
%    \end{macrocode}

% \macro{\childdocforward}
% The command |\childdocforward| redirects
% compilation to the main file or
% (if the optional argument is given) a child file.
% Parameters are set as if the main file
% or a child file starting with |\childdocof| was compiled.
% Then compilation is handed over to the main file:
%    \begin{macrocode}
\newcommand{\childdocforward}[2][]
{
  \begingroup
    \if?#1?
      \def\childdoctmp
      {
        \def\childdocname{#2}
        \def\childdocjob{#2}
        \def\jobname{#2}
        \input{#2}
        \endinput
      }
    \else
      \def\childdoctmp
      {
        \childdocdisable
        \def\childdocname{#2}
        \childdoctrue
        \includeonly{#2}
        \def\childdocjob{#1}
        \def\jobname{#1}
        \input{#1}
        \endinput
      }
    \fi
    \expandafter
  \endgroup
  \childdoctmp
}
%    \end{macrocode}

% \macro{\childdocforwardprefix}
% The command |\childdocforwardprefix| redirects
% compilation to the main or a child file by means of a pattern.
% The prefix |#1| in the current filename is replaced by |#2|
% and the suffix of the current filename is kept
% (it is assumed that the filename does not contain the substring `|~~~|'
% which is used as a delimiter).
% Compilation is handed over to the new file by |\childdocforward|:
%    \begin{macrocode}
\newcommand{\childdocforwardprefix}[3][]
{
  \begingroup
    \def\childdocextract #2##1~~~{\def\childdoctmp{\childdocforward[#1]{#3##1}}}
    \expandafter\childdocextract\childdocname~~~
    \expandafter
  \endgroup
  \childdoctmp
}
%    \end{macrocode}

% \macro{\childdoc}
% The deprecated macro |\childdoc| is a legacy version of |\childdocmain|:
%    \begin{macrocode}
\newcommand{\childdoc}{\childdocmain}
%    \end{macrocode}

% \macro{\childdocredirect}
% The deprecated macro |\childdocredirect| is a legacy version
% of |\childdocforward| and |\childdocforwardprefix|:
%    \begin{macrocode}
\newcommand{\childdocredirect}[2][]
{
  \begingroup
    \if?#1?
      \def\childdoctmp{\childdocforward{#2}}
    \else
      \def\childdoctmp{\childdocforwardprefix{#1}{#2}}
    \fi
    \expandafter
  \endgroup
  \childdoctmp
}
%    \end{macrocode}

%\iffalse
%</package>
%\fi
%
\endinput

\childdocof{cdocsamp}
%    \end{macrocode}

%\iffalse
%</samplechap1|samplechap2>
%\fi
%
%\iffalse
%<*samplechap1>
%\fi
% Some text for chapter 1:
%    \begin{macrocode}
\section{one}
some text in chapter one
%    \end{macrocode}

%\iffalse
%</samplechap1>
%\fi
% Some text for chapter 2:
%\iffalse
%<*samplechap2>
%\fi
%    \begin{macrocode}
\section{two}
more text in chapter two
%    \end{macrocode}

%\iffalse
%</samplechap2>
%\fi
%
% %%%%%%%%%%%%%%%%%%%%%%%%%%%%%%%%%%%%%%
% \paragraph{Part Include Files.}
%
% The include files are called |cdocspt3.tex| and |cdocspt4.tex|.
%
%\iffalse
%<*samplepart3|samplepart4>
%\fi

% Optional override for |\version| flag:
%    \begin{macrocode}
%%\providecommand{\version}{final}
%    \end{macrocode}

% Include the main document:
%    \begin{macrocode}
% \iffalse
%
% childdoc.dtx Copyright (C) 2017-2018 Niklas Beisert
%
% This work may be distributed and/or modified under the
% conditions of the LaTeX Project Public License, either version 1.3
% of this license or (at your option) any later version.
% The latest version of this license is in
%   http://www.latex-project.org/lppl.txt
% and version 1.3 or later is part of all distributions of LaTeX
% version 2005/12/01 or later.
%
% This work has the LPPL maintenance status `maintained'.
%
% The Current Maintainer of this work is Niklas Beisert.
%
% This work consists of the files childdoc.dtx and childdoc.ins
% and the derived files childdoc.def and cdocsamp.tex with
% cdocsch1.tex, cdocsch2.tex, cdocsdrf.tex, cdocsfn1.tex, cdocsfn2.tex.
%
%<package>\ifdefined\childdocmain\endinput\fi
%<package>\ProvidesFile{childdoc.def}[2018/12/30 v2.0 child document driver]
%<samplemain>\ProvidesFile{cdocsamp.tex}[2018/12/30 v2.0 sample for childdoc]
%<*driver>
%\ProvidesFile{childdoc.drv}[2018/12/30 v2.0 childdoc reference manual file]
\PassOptionsToClass{10pt,a4paper}{article}
\documentclass{ltxdoc}

\usepackage[margin=35mm]{geometry}
\usepackage{hyperref}
\usepackage{hyperxmp}
\usepackage[usenames]{color}

\hypersetup{colorlinks=true}
\hypersetup{pdfstartview=FitH}
\hypersetup{pdfpagemode=UseNone}
\hypersetup{pdfsource={}}
\hypersetup{pdflang={en-UK}}
\hypersetup{pdfcopyright={Copyright 2017-2018 Niklas Beisert.
  This work may be distributed and/or modified under the
  conditions of the LaTeX Project Public License, either version 1.3
  of this license or (at your option) any later version.}}
\hypersetup{pdflicenseurl={http://www.latex-project.org/lppl.txt}}
\hypersetup{pdfcontactaddress={ETH Zurich, ITP, HIT K,
  Wolfgang-Pauli-Strasse 27}}
\hypersetup{pdfcontactpostcode={8093}}
\hypersetup{pdfcontactcity={Zurich}}
\hypersetup{pdfcontactcountry={Switzerland}}
\hypersetup{pdfcontactemail={nbeisert@itp.phys.ethz.ch}}
\hypersetup{pdfcontacturl={http://people.phys.ethz.ch/\xmptilde nbeisert/}}

\newcommand{\secref}[1]{\hyperref[#1]{section \ref*{#1}}}

\parskip1ex
\parindent0pt
\let\olditemize\itemize
\def\itemize{\olditemize\parskip0pt}

\begin{document}

\title{The \textsf{childdoc} Package}
\hypersetup{pdftitle={The childdoc Package}}
\author{Niklas Beisert\\[2ex]
  Institut f\"ur Theoretische Physik\\
  Eidgen\"ossische Technische Hochschule Z\"urich\\
  Wolfgang-Pauli-Strasse 27, 8093 Z\"urich, Switzerland\\[1ex]
  \href{mailto:nbeisert@itp.phys.ethz.ch}
  {\texttt{nbeisert@itp.phys.ethz.ch}}}
\hypersetup{pdfauthor={Niklas Beisert}}
\hypersetup{pdfsubject={Manual for the LaTeX2e Package childdoc}}
\date{30 December 2018, \textsf{v2.0}}
\maketitle

\begin{abstract}\noindent
\textsf{childdoc} is a \LaTeXe{} package
that enables the direct compilation
of document sections included by |\include|
to individual files.
\end{abstract}

\begingroup
\parskip0ex
\tableofcontents
\endgroup

%%%%%%%%%%%%%%%%%%%%%%%%%%%%%%%%%%%%%%%%%%%%%%%%%%%%%%%%%%%%%%%%%%%%%%%%%%%%%%%%
%%%%%%%%%%%%%%%%%%%%%%%%%%%%%%%%%%%%%%%%%%%%%%%%%%%%%%%%%%%%%%%%%%%%%%%%%%%%%%%%
\section{Introduction}

\LaTeX{} provides a mechanism to structure a large document (such as a book)
into a main file and several child files (containing the chapters)
using the |\include| command.
This mechanism is beneficial for documents
which span hundreds of pages in order to
make the source file(s) more manageable.
Moreover, compilation can be restricted to
selected child files by means of the |\includeonly| command.
The latter feature can be used to reduce the compilation time while editing
(this was significantly more useful in the earlier days of \LaTeX{})
or to generate a smaller document which is easier to navigate.
Another application of |\includeonly| is to generate
documents consisting of selected parts of the complete document.

However, there are a few drawbacks of the plain |\include| mechanism:
\begin{itemize}
\item
The child files cannot be compiled on their own,
they can only be compiled via the main file.
A naive editing environment
(such as a text editor with an option
to have the current file processed by \LaTeX)
may require one to switch to the main file before compiling;
attempting to compile the child file produces errors.
\item
The main file must be modified (each time)
to adjust the |\includeonly| command
to the present needs. This easily leaves the main file in a messy state.
\item
The generated document will always carry the filename
of the main document. This is inconvenient if
several child files are to be compiled and
to be kept for distribution.
\end{itemize}

The present package provides a simple interface
to make child files individually compilable by \LaTeX{}.
Compiling a child file then has the same effect as compiling
the main file with an |\includeonly| command
to select the appropriate child.
Moreover the generated document will carry the name of the child
rather than the main file.
This resolves all three above issues.

This feature is meant to make the editing of books,
thesis documents and lecture notes somewhat more convenient.
However, the package can also be used efficiently for
composing a series of documents (such as exercise sheets)
which are typically distributed individually.
It then assists the author in generating the individual documents
(potentially in different versions)
as well as a document containing the collected series.
Another application is in developing style files
or other kinds of included material
where compilation of the style file could redirect
to a sample or test file.

%%%%%%%%%%%%%%%%%%%%%%%%%%%%%%%%%%%%%%%%%%%%%%%%%%%%%%%%%%%%%%%%%%%%%%%%%%%%%%%%
%%%%%%%%%%%%%%%%%%%%%%%%%%%%%%%%%%%%%%%%%%%%%%%%%%%%%%%%%%%%%%%%%%%%%%%%%%%%%%%%
\section{Usage}

First of all, the package \textsf{childdoc} is \emph{not} a standard
\LaTeXe{} |.sty| style file! Therefore it needs to be invoked in
a non-standard way.

%%%%%%%%%%%%%%%%%%%%%%%%%%%%%%%%%%%%%%%%%%%%%%%%%%%%%%%%%%%%%%%%%%%%%%%%%%%%%%%%
\subsection{Included Files}
\label{sec:include}

%%%%%%%%%%%%%%%%%%%%%%%%%%%%%%%%%%%%%%%%
\DescribeMacro{\childdocmain}
To use the package, add the commands
\begin{center}
\begin{tabular}{l}
|\input{childdoc.def}|\\
|\childdocmain{}|\\
\end{tabular}
\end{center}
at the very top of the main \LaTeX{} file,
in particular \emph{before} the |\documentclass| statement!
The argument of |\childdocmain| should be left empty
(but it must be present).

%%%%%%%%%%%%%%%%%%%%%%%%%%%%%%%%%%%%%%%%
\DescribeMacro{\childdocof}
Furthermore, add the commands
\begin{center}
\begin{tabular}{l}
|\input{childdoc.def}|\\
|\childdocof{|\textit{main}|}|\\
\end{tabular}
\end{center}
at the top of every child file \textit{child}
which is included by |\include{|\textit{child}|}|
from within the main file
(or at least for those files to be compiled individually).
The argument \textit{main} must be the filename of the main file.

There are a couple of
considerations in setting up the main and child documents:

%%%%%%%%%%%%%%%%%%%%%%%%%%%%%%%%%%%%%%%%
\paragraph{Restrictions.}

Please note the following restrictions:
\begin{itemize}
\item
|\childdocmain| must be called with one argument \textit{main}
to ensure compatibility with earlier version of the package.
It must either be empty (|\childdocmain{}|)
or precisely match the filename of the main file in which it is specified.
See \secref{sec:detection} for further information.
\item
The filename \textit{main} must be specified without the |.tex| extension.
\item
The filename \textit{main} is case sensitive
(even in case-insensitive file systems)
due to internal string comparison.
\item
The argument \textit{main} should be fully expanded, it cannot be a macro.
\item
Subdirectories and special characters should be avoided in filenames.
\item
The command |\childdocmain{|\textit{main}|}| must be followed by a whitespace.
It should not be followed immediately by another command
or by a comment mark `|%|'.
This is because the \TeX{} parser reads the token immediately following
the argument of |\childdocmain| and puts it
at the beginning of every child section;
however, a white\-space is ignored.
\end{itemize}

%%%%%%%%%%%%%%%%%%%%%%%%%%%%%%%%%%%%%%%%
\paragraph{Content of Main File.}

It is advisable to place all content in the child files included by |\include|.
Any output contained in the main file will appear in all child documents
unless suppressed manually;
it cannot be suppressed automatically by the |\includeonly| directive
and thus should normally be avoided.
A method to include some content in the main file
by means of conditional processing is described in \secref{sec:conditional}.

%%%%%%%%%%%%%%%%%%%%%%%%%%%%%%%%%%%%%%%%
\paragraph{Page Numbering.}

When only a part of the document is compiled,
the appropriate numbering of pages
(as well as other status parameters)
is determined from the |.aux| files.
The latter contain information from previous passes.
However this information needs to propagate through
all intermediate child documents.
Therefore the page numbering in child documents may well
be inconsistent until the complete document is compiled at least once.

A useful (if unconventional) way to always ensure a consistent
page numbering is to restart the numbering in each child document
and denote the pages by `\textit{child}|.|\textit{page}'
where \textit{child} represents the chapter/section number of the child file.
This can be achieved by the command
|\numberwithin{page}{|\textit{child}|}|
of the \textsf{amsmath} package
where \textit{child} can be |chapter| or |section|
depending on the chosen structuring.
Alternatively, one can modify the macro |\thepage| appropriately
and reset the counter |page| at the start of each child file.

%%%%%%%%%%%%%%%%%%%%%%%%%%%%%%%%%%%%%%%%%%%%%%%%%%%%%%%%%%%%%%%%%%%%%%%%%%%%%%%%
\subsection{Conditional Processing}
\label{sec:conditional}

The package provides a mechanism to compile different versions
of a document. To customise the versions further some conditional processing
can come in handy to distinguish which version is being compiled.
The package provides two macros to describe the compilation context:

%%%%%%%%%%%%%%%%%%%%%%%%%%%%%%%%%%%%%%%%
\DescribeMacro{\ifchilddoc}
The conditional |\ifchilddoc| distinguishes between the compilation of
child documents and the main document:
%
\begin{center}
|\ifchilddoc |\textit{child-code}| |[|\||else |\textit{main-code}]| \||fi|
\end{center}

%%%%%%%%%%%%%%%%%%%%%%%%%%%%%%%%%%%%%%%%
\DescribeMacro{\childdocname}
\DescribeMacro{\childdocjob}
The macro |\childdocname| contains the filename (without extension)
of the main or child file being processed.
Note that |\childdocjob| will always contain the name of the main file.

%%%%%%%%%%%%%%%%%%%%%%%%%%%%%%%%%%%%%%%%
\paragraph{Title Page.}

Conditional processing can be used to include a title or banner page
in the main document when proper precautions are taken.
Importantly, the code in the main file should ensure that the page counter
(as well as other status parameters which are stored in the |.aux| files)
takes the same value after the conditional processing.
Otherwise the page numbers may take divergent values
depending on which part is compiled.

For example, a title page could be declared by:
%
\begin{center}
\begin{tabular}{l}
|\ifchilddoc\||else|\\
|\addtocounter{page}{-1}|\\
\textit{code for title page}\\
|\newpage|\\
|\||fi|
\end{tabular}
\end{center}
%
A banner page for the child documents can be generated by:
%
\begin{center}
\begin{tabular}{l}
|\ifchilddoc|\\
|\addtocounter{page}{-1}|\\
\textit{code for banner page}\\
|\newpage|\\
|\||fi|
\end{tabular}
\end{center}
%
Here one could write a message such as:
\begin{center}
|This is the part \childdocname{} of \childdocjob{}.|
\end{center}

%%%%%%%%%%%%%%%%%%%%%%%%%%%%%%%%%%%%%%%%%%%%%%%%%%%%%%%%%%%%%%%%%%%%%%%%%%%%%%%%
\subsection{Flags}
\label{sec:flags}

The package makes it easy to generate different versions
of the main or child documents.
To this end compilation flags can be defined
and assigned different default values.
They will be particularly useful in conjunction
with the forwarding mechanism described in \secref{sec:forward}.

For example, it may be useful to have a flag |\version|
which can be set to |draft| or |final|.
The document source will contain some conditional code
depending on the value of |\version|.
Suppose further, the flag should default to |final| for the main file
and to |draft| for child files
which is a natural assignment for editing the document.
This is achieved by placing the following code
in the preamble of the main document
(below the |\childdocmain| directive):
%
\begin{center}
\begin{tabular}{l}
|\ifchilddoc|\\
|\providecommand{\version}{draft}|\\
|\||else|\\
|\providecommand{\version}{final}|\\
|\||fi|
\end{tabular}
\end{center}
%
The definition by |\providecommand| makes sure
that previous definitions are not overwritten.
Further statements |\providecommand{\version}{...}|
can thus be added before the above code to override it.

For the main file, one might add a line
(between |\childdocmain| and the above block)
%
\begin{center}
|%\ifchilddoc\||else\providecommand{\version}{draft}\||fi|
\end{center}
%
which can be uncommented to produce a draft version.
Likewise one can add a line to the very top of a child file
(above the |\childdocof{|\textit{main}|}| directive)
%
\begin{center}
|%\providecommand{\version}{final}|
\end{center}
%
which can be uncommented to produce the final version of this child document.

%%%%%%%%%%%%%%%%%%%%%%%%%%%%%%%%%%%%%%%%%%%%%%%%%%%%%%%%%%%%%%%%%%%%%%%%%%%%%%%%
\subsection{Forwarding}
\label{sec:forward}

Different versions of the main or child documents
using compilation flags as described in \secref{sec:flags}
can be (permanently) stored in different files
for convenient compilation, viewing and distribution.
To this end, the package defines a command
to pass on compilation to a different file:

%%%%%%%%%%%%%%%%%%%%%%%%%%%%%%%%%%%%%%%%
\DescribeMacro{\childdocforward}
The command |\childdocforward| redirects processing to
another source file:
%
\begin{center}
\begin{tabular}{l}
|\input{childdoc.def}|\\
|\childdocforward[|\textit{main}|]{|\textit{dest}|}|\\
\end{tabular}
\end{center}
%
The argument \textit{dest} is the destination file
(without extension).
It should be the main file or one of the child files.
Note that further \textsf{childdoc} directives
such as |\childdocof| and |\childdocforward|
in the indicated file will be processed in this form.
The optional argument \textit{main}
passes on directly to the main file \textit{main}
while pretending to compile the child \textit{dest}.
This form behaves as if \textit{dest}
issues |\childdocof{|\textit{main}|}| right away,
and no further \textsf{childdoc} directives will be processed.

%%%%%%%%%%%%%%%%%%%%%%%%%%%%%%%%%%%%%%%%
\DescribeMacro{\...prefix}
In the alternative form |\childdocforwardprefix|,
%
\begin{center}
\begin{tabular}{l}
|\input{childdoc.def}|\\
|\childdocforwardprefix[|\textit{main}|]{|\textit{prefix}|}{|\textit{dest}|}|
\end{tabular}
\end{center}
%
the destination file is determined by a pattern
depending on the current file:
To make this work, the current file must be called
`{\textit{prefix}\hspace{0.2em}\textit{suffix}}'
with \textit{prefix} matching precisely the argument.
Processing is then passed on to the file
`{\textit{dest}\hspace{0.2em}\textit{suffix}}'.
Surely, the same effect is achieved by
directly specifying the
argument `{\textit{dest}\hspace{0.2em}\textit{suffix}}'
in the first form.
However, that requires to set up a different file
for each child. With the alternative form of the command
all these files can have exactly the same content
which simplifies setting them up and maintaining them.

For example, the following file |draft.tex|
with a compilation flag |\version| as described in \secref{sec:flags}
compiles the main document as a draft:
%
\begin{center}
\begin{tabular}{l}
|\def\version{draft}|\\
|\input{childdoc.def}|\\
|\childdocforward{|\textit{main}|}|
\end{tabular}
\end{center}
%
Likewise, the following files |final|\textit{nn}|.tex|
compile the final version of the child document
|child|\textit{nn}|.tex|:
%
\begin{center}
\begin{tabular}{l}
|\def\version{final}|\\
|\input{childdoc.def}|\\
|\childdocforwardprefix{final}{child}|
\end{tabular}
\end{center}
%

Note that when several versions of a main file and/or of each child file
are to be generated, it may be convenient to set up a |Makefile| or
shell script to automatise the process.

%%%%%%%%%%%%%%%%%%%%%%%%%%%%%%%%%%%%%%%%%%%%%%%%%%%%%%%%%%%%%%%%%%%%%%%%%%%%%%%%
\subsection{Command Line Processing}
\label{sec:commandline}

The effect of redirection files can also be achieved by invoking
the \LaTeX{} compiler with a more elaborate command line.
Most conveniently this should be done as part
of a shell script or a |Makefile|.

When using \textsf{childdoc} in the main file, the following
command lines effectively perform a redirection
(note that depending on the shell being used,
backslashes may have to be doubled: `|\|' $\to$ `|\\|'):
%
\begin{center}
|... -jobname "|\textit{target}|" |\\|"|[\textit{flags}]%
|\input{childdoc.def}\childdocforward[|\textit{main}|]{|\textit{dest}|}"|
\end{center}
%
Here \textit{target} is the name of the output file,
\textit{main} is the name of the main file
and \textit{dest} is the name of the main or child file to be processed
(all filenames without extensions).
The optional argument \textit{main} can be omitted
if \textit{main} matches \textit{dest}.
Optionally, compilation \textit{flags} can be defined via |\def| commands.
This command line makes the \TeX{} engine believe
it is compiling the file \textit{target}
whose content is specified as the latter parameter.
The provided code then forwards the processing to
\textit{main} or \textit{dest} as described in \secref{sec:forward}.

%%%%%%%%%%%%%%%%%%%%%%%%%%%%%%%%%%%%%%%%%%%%%%%%%%%%%%%%%%%%%%%%%%%%%%%%%%%%%%%%
\subsection{Include by Input}
\label{sec:input}

Including child documents by |\include| has some restrictions by design.
Most notably, the content of a child document always occupies
its own set of pages; pages cannot be shared between child documents.
Usually, this behaviour makes perfect sense
because each child document contain an essential part of the document.
However, in some situations it may be desirable to compose
a document from a collection of parts
without having mandatory page breaks between then.
For this case, the package
provides a mechanism to include parts
by |\input| which can also be processed individually.
However, by construction this mechanism
requires manual handling of the content to be output.

%%%%%%%%%%%%%%%%%%%%%%%%%%%%%%%%%%%%%%%%
\DescribeMacro{\ifchilddocmanual}
The main file should be prepared as usual, see \secref{sec:include}.
However, the document body must make a distinction
between processing of an individual part and of the main document, e.g.:
%
\begin{center}
\begin{tabular}{l}
|\ifchilddocmanual|\\
|\input{\childdocname}|\\
|\||else|\\
\textit{document body with }|\input{|\textit{part}|}|\\
|\||fi|
\end{tabular}
\end{center}
%
The conditional |\ifchilddocmanual| is true whenever
a part to be included by |\input| is being compiled,
and the name of the part is stored in |\childdocname|.

%%%%%%%%%%%%%%%%%%%%%%%%%%%%%%%%%%%%%%%%
\DescribeMacro{\childdocby}
Each part to be included by |\input| should start with:
%
\begin{center}
\begin{tabular}{l}
|\input{childdoc.def}|\\
|\childdocby{|\textit{main}|}|\\
\end{tabular}
\end{center}
%
The directive |\childdocby| is similar to |\childdocof|
described in \secref{sec:include},
but the subsequent selection of content must be done manually.
To that end, both |\ifchilddoc| and |\ifchilddocmanual|
will be true upon processing of a part,
and the name of the part is stored in |\childdocname|.
Note that |\jobname| will be set to the filename of the current part
so that each part receives an individual |.aux| file
that does not interfere with the |.aux| file(s) of the main document.
This behaviour can be altered by the alternative form
|\childdocby[*]{|\textit{main}|}| (with a non-empty optional argument)
which uses the |.aux| file of the main document
by setting |\jobname| to \textit{main}.

%%%%%%%%%%%%%%%%%%%%%%%%%%%%%%%%%%%%%%%%%%%%%%%%%%%%%%%%%%%%%%%%%%%%%%%%%%%%%%%%
\subsection{Driver Development}
\label{sec:driver}

The \textsf{childdoc} mechanism can also be use for the development
of definition files such as \LaTeX{} styles or classes.
This case differs from the above setup with multiple parts
included by |\include| in that no |\includeonly| should be invoked.
This can be achieved by starting the include file
(before |\ProvidesPackage|) with:
%
\begin{center}
\begin{tabular}{l}
|\input{childdoc.def}|\\
|\childdocforward{|\textit{main}|}|\\
\end{tabular}
\end{center}
%
or alternatively with:
%
\begin{center}
\begin{tabular}{l}
|\input{childdoc.def}|\\
|\childdocby{|\textit{main}|}|\\
\end{tabular}
\end{center}
%
Both forms have slightly different effects as described above.
The main file is prepared as usual, see \secref{sec:include}.

%%%%%%%%%%%%%%%%%%%%%%%%%%%%%%%%%%%%%%%%%%%%%%%%%%%%%%%%%%%%%%%%%%%%%%%%%%%%%%%%
\subsection{Legacy Detection}
\label{sec:detection}

The directive |\childdocmain| in the main file can detect
whether the complete document or merely a child is to be compiled
even without using the directive |\childdocof|.
This method is deprecated because it is less robust
and there is no compelling reason to use it;
it is merely provided for backward compatibility
and it may be removed in future versions.

If the detection mechanism is to be used,
it is mandatory to correctly specify
the filename of the main file as the argument of |\childdocmain|:
%
\begin{center}
\begin{tabular}{l}
|\input{childdoc.def}|\\
|\childdocmain{|\textit{main}|}|\\
\end{tabular}
\end{center}
%
If |\jobname| does not match the argument \textit{main} of |\childdocmain|,
it is assumed that |\jobname| points to the child file to be compiled.
When using |\childdocmain| with the main file specified as argument,
it suffices to start a child file
with just |\input{|\textit{main}|}|
without loading of the package and using |\childdocof|.
If instead all processing is done
with the appropriate \textsf{childdoc} directives,
the argument of \textit{main} of |\childdocmain| can be empty.

An alternative version of the command line processing described
in \secref{sec:commandline} using the detection mechanism reads:
%
\begin{center}
|... -jobname "|\textit{target}|" "|[\textit{flags}]%
[|\def\jobname{|\textit{dest}|}|]|\input{|\textit{main}|}"|
\end{center}

%%%%%%%%%%%%%%%%%%%%%%%%%%%%%%%%%%%%%%%%%%%%%%%%%%%%%%%%%%%%%%%%%%%%%%%%%%%%%%%%
\subsection{Manual Code}
\label{sec:manual}

In case one cannot be certain whether the definitions file |childdoc.def|
is installed on the target \TeX{} distribution
and one prefers not to ship it,
it is conceivable to paste a few relevant commands into the sources.

To that end, drop all statements |\input{childdoc.def}|
and perform the replacements as outlined below.
Instead of |\childdocmain{|\textit{main}|}| add the following code
to the top of the main file:
%
\begin{center}
\begin{tabular}{l}
|\||ifdefined\childdocname\endinput\||fi\newif\ifchilddoc|\\
|\edef\childdocname{\scantokens\expandafter{\jobname\noexpand}}|\\
|\def\childdocmain{|\textit{main}|}\||ifx\childdocmain\childdocname\||else|\\
|\childdoctrue\includeonly{\childdocname}\let\jobname\childdocmain\||fi|\\
\end{tabular}
\end{center}
%
Instead of |\childdocof{|\textit{main}|}| just include the main file
at the top of each child file:
%
\begin{center}
|\input{|\textit{main}|}|
\end{center}
%
A simple redirection |\childdocforward{|\textit{dest}|}| is achieved by:
%
\begin{center}
|\def\jobname{|\textit{dest}|}\input{\jobname}|
\end{center}
%
The redirection with prefix
|\childdocforwardprefix[|\textit{prefix}|]{|\textit{dest}|}|
is accomplished by:
%
\begin{center}
\begin{tabular}{l}
|{\edef\jobname{\scantokens\expandafter{\jobname\noexpand}}|\\
|\def\redirectjob |\textit{prefix}|#1~~~{\gdef\jobname{|\textit{dest}|#1}}|\\
|\expandafter\redirectjob\jobname~~~}\input{\jobname}|
\end{tabular}
\end{center}

In an alternative approach,
child documents can be compiled by a specific command line
without additional code or specific definitions:
%
\begin{center}
|... -jobname "|\textit{target}|" "|[\textit{flags}]%
|\includeonly{|\textit{dest}|}\input{|\textit{main}|}"|
\end{center}
%

%%%%%%%%%%%%%%%%%%%%%%%%%%%%%%%%%%%%%%%%%%%%%%%%%%%%%%%%%%%%%%%%%%%%%%%%%%%%%%%%
%%%%%%%%%%%%%%%%%%%%%%%%%%%%%%%%%%%%%%%%%%%%%%%%%%%%%%%%%%%%%%%%%%%%%%%%%%%%%%%%
\section{Information}

%%%%%%%%%%%%%%%%%%%%%%%%%%%%%%%%%%%%%%%%%%%%%%%%%%%%%%%%%%%%%%%%%%%%%%%%%%%%%%%%
\subsection{Copyright}

Copyright \copyright{} 2017--2018 Niklas Beisert

This work may be distributed and/or modified under the
conditions of the \LaTeX{} Project Public License, either version 1.3
of this license or (at your option) any later version.
The latest version of this license is in
  \url{http://www.latex-project.org/lppl.txt}
and version 1.3 or later is part of all distributions of \LaTeX{}
version 2005/12/01 or later.

This work has the LPPL maintenance status `maintained'.

The Current Maintainer of this work is Niklas Beisert.

This work consists of the files |README.txt|, |childdoc.ins| and |childdoc.dtx|
as well as the derived files |childdoc.def|, |cdocsamp.tex|
with |cdocsch1.tex|, |cdocsch2.tex|, |cdocspt3.tex|, |cdocspt4.tex|,
|cdocsdrf.tex|, |cdocsfn1.tex|, |cdocsfn2.tex|
as well as |childdoc.pdf|.

%%%%%%%%%%%%%%%%%%%%%%%%%%%%%%%%%%%%%%%%%%%%%%%%%%%%%%%%%%%%%%%%%%%%%%%%%%%%%%%%
\subsection{Files and Installation}

The package consists of the files:
%
\begin{center}
\begin{tabular}{ll}
    |README.txt|   & readme file \\
    |childdoc.ins| & installation file \\
    |childdoc.dtx| & source file \\
    |childdoc.def| & definition file \\
    |cdocsamp.tex| & sample main file \\
    |cdocsch1.tex| & sample include file \\
    |cdocsch2.tex| & sample include file \\
    |cdocspt3.tex| & sample part file \\
    |cdocspt4.tex| & sample part file \\
    |cdocsdrf.tex| & sample redirection file \\
    |cdocsfn1.tex| & sample redirection file \\
    |cdocsfn2.tex| & sample redirection file \\
    |childdoc.pdf| & manual
\end{tabular}
\end{center}
%
The distribution consists of the files
|README.txt|, |childdoc.ins| and |childdoc.dtx|.
%
\begin{itemize}
\item
Run (pdf)\LaTeX{} on |childdoc.dtx|
to compile the manual |childdoc.pdf| (this file).
\item
Run \LaTeX{} on |childdoc.ins| to create the definitions file |childdoc.def|
and the sample |cdocsamp.tex| with include files
|cdocsch1.tex|, |cdocsch2.tex|, |cdocspt3.tex|, |cdocspt4.tex|,
|cdocsdrf.tex|, |cdocsfn1.tex|, |cdocsfn2.tex|.
Then copy the file |childdoc.def| to an appropriate directory of your \LaTeX{}
distribution, e.g.\ \textit{texmf-root}|/tex/latex/childdoc|.
\end{itemize}

%%%%%%%%%%%%%%%%%%%%%%%%%%%%%%%%%%%%%%%%%%%%%%%%%%%%%%%%%%%%%%%%%%%%%%%%%%%%%%%%
\subsection{Related CTAN Packages}

There are several other packages which offer a similar functionality:
%
\begin{itemize}
\item
The packages
\href{http://ctan.org/pkg/docmute}{\textsf{docmute}},
\href{http://ctan.org/pkg/includex}{\textsf{includex}} and
\href{http://ctan.org/pkg/standalone}{\textsf{standalone}}
provide commands to include only the document body of
a child file thus allowing both files to be compiled individually.
\item
The packages \href{http://ctan.org/pkg/subdocs}{\textsf{subdocs}}
and \href{http://ctan.org/pkg/subfiles}{\textsf{subfiles}}
provide structures in which the main and child documents can be
encapsulated and allowing them to be compiled individually.
The inclusion mechanism is different from the conventional |\include|.
\item
The package \href{http://ctan.org/pkg/combine}{\textsf{combine}}
is an elaborate solution to combine several documents into one.
\end{itemize}
%
See also the CTAN topic \href{http://ctan.org/topic/subdocs}{\textsf{subdocs}}
for further related packages.
The present package differs from the above solutions in that
a document structure constructed with the conventional |\include| mechanism
just needs two extra commands at the top of every file
such that all constituent files can be compiled individually.

%%%%%%%%%%%%%%%%%%%%%%%%%%%%%%%%%%%%%%%%%%%%%%%%%%%%%%%%%%%%%%%%%%%%%%%%%%%%%%%%
%\subsection{Feature Suggestions}
%
%The following is a list of features which may be useful for future
%versions of this package:
%%
%\begin{itemize}
%\item
%\ldots
%\end{itemize}

%%%%%%%%%%%%%%%%%%%%%%%%%%%%%%%%%%%%%%%%%%%%%%%%%%%%%%%%%%%%%%%%%%%%%%%%%%%%%%%%
\subsection{Revision History}

%%%%%%%%%%%%%%%%%%%%%%%%%%%%%%%%%%%%%%%%
\paragraph{v2.0:} 2018/12/30

\begin{itemize}
\item
immediate forward processing
\item
added |\childdocby| mechanism
\item
manual restructured
\end{itemize}

%%%%%%%%%%%%%%%%%%%%%%%%%%%%%%%%%%%%%%%%
\paragraph{v1.6:} 2018/01/17

\begin{itemize}
\item
application for development of include files
\item
corrections to manual
\end{itemize}

%%%%%%%%%%%%%%%%%%%%%%%%%%%%%%%%%%%%%%%%
\paragraph{v1.5:} 2017/05/21

\begin{itemize}
\item
more complete structuring introduced
\item
|\childdocof| introduced
\item
|\childdoc| renamed to |\childdocmain|
\item
|\childredirect| renamed to |\childdocforward| and |\childdocforwardprefix|
and functionality expanded
\end{itemize}

%%%%%%%%%%%%%%%%%%%%%%%%%%%%%%%%%%%%%%%%
\paragraph{v1.0:} 2017/04/27

\begin{itemize}
\item
manual and install package
\item
first version published on CTAN
\end{itemize}

%%%%%%%%%%%%%%%%%%%%%%%%%%%%%%%%%%%%%%%%
\paragraph{v0.6:} 2017/04/26

\begin{itemize}
\item
redirection mechanism added
\end{itemize}

%%%%%%%%%%%%%%%%%%%%%%%%%%%%%%%%%%%%%%%%
\paragraph{v0.5:} 2017/04/26

\begin{itemize}
\item
functionality in definition file
\end{itemize}


%%%%%%%%%%%%%%%%%%%%%%%%%%%%%%%%%%%%%%%%%%%%%%%%%%%%%%%%%%%%%%%%%%%%%%%%%%%%%%%%
%%%%%%%%%%%%%%%%%%%%%%%%%%%%%%%%%%%%%%%%%%%%%%%%%%%%%%%%%%%%%%%%%%%%%%%%%%%%%%%%
%%%%%%%%%%%%%%%%%%%%%%%%%%%%%%%%%%%%%%%%%%%%%%%%%%%%%%%%%%%%%%%%%%%%%%%%%%%%%%%%
\appendix

\settowidth\MacroIndent{\rmfamily\scriptsize 000\ }

 \DocInput{childdoc.dtx}

\end{document}
%</driver>
% \fi
%
% %%%%%%%%%%%%%%%%%%%%%%%%%%%%%%%%%%%%%%%%%%%%%%%%%%%%%%%%%%%%%%%%%%%%%%%%%%%%%%
% %%%%%%%%%%%%%%%%%%%%%%%%%%%%%%%%%%%%%%%%%%%%%%%%%%%%%%%%%%%%%%%%%%%%%%%%%%%%%%
% \section{Sample}
%\iffalse
%<*samplemain>
%\fi
%
% The following presents a sample document
% with two chapters, two parts, a title page,
% a compile flag as well as three forwarding files to set the flag.
% It consists of eight |.tex| files:
% \begin{center}
% \begin{tabular}{ll}
% |cdocsamp.tex|&main file\\
% |cdocsch1.tex|&include file for chapter 1\\
% |cdocsch2.tex|&include file for chapter 2\\
% |cdocspt3.tex|&include file for part 3\\
% |cdocspt4.tex|&include file for part 4\\
% |cdocsdrf.tex|&forwarding file for main file in draft mode\\
% |cdocsfi1.tex|&forwarding file for final version of chapter 1\\
% |cdocsfi2.tex|&forwarding file for final version of chapter 2\\
% \end{tabular}
% \end{center}
% Each of the eight files can be compiled directly by the \LaTeX{} compiler.
%
% %%%%%%%%%%%%%%%%%%%%%%%%%%%%%%%%%%%%%%
% \paragraph{Main File.}
%
% The main file is called |cdocsamp.tex|.
%
% Load the \textsf{childdoc} definitions and
% declare the filename for the main document:
%    \begin{macrocode}
\input{childdoc.def}
\childdocmain{}
%    \end{macrocode}

% Optional override for |\version| flag:
%    \begin{macrocode}
%%\ifchilddoc\else\providecommand{\version}{draft}\fi
%    \end{macrocode}

% Define the default values for the |\version| flag
% (|final| for the main file and |draft| for childs):
%    \begin{macrocode}
\ifchilddoc
\providecommand{\version}{draft}
\else
\providecommand{\version}{final}
\fi
%    \end{macrocode}

% Load the standard document class:
%    \begin{macrocode}
\documentclass[12pt]{article}
%    \end{macrocode}

% Start the document body:
%    \begin{macrocode}
\begin{document}
%    \end{macrocode}

% Declare a title page.
% Print title, part of document being processed and version flag:
%    \begin{macrocode}
\addtocounter{page}{-1}
\begin{center}
{\LARGE\bfseries{}childdoc example\par}
\vspace{1cm}
\ifchilddoc
\ifchilddocmanual part\else chapter\fi:
`\childdocname' of `\childdocjob'\par
\else
main document: `\childdocjob'\par
\fi
version: \version\par
\end{center}
\newpage
%    \end{macrocode}

% Manually include selected file,
% otherwise process as usual:
%    \begin{macrocode}
\ifchilddocmanual
\section*{part `\childdocname'}
\input{\childdocname}
\else
%    \end{macrocode}

% Include the two chapters:
%    \begin{macrocode}
\include{cdocsch1}
\include{cdocsch2}
%    \end{macrocode}

% Include the two parts unless only chapters should be displayed:
%    \begin{macrocode}
\ifchilddoc\else
\section{part three}
\input{cdocspt3}
\section{part four}
\input{cdocspt4}
\fi
%    \end{macrocode}

% Process as usual until here:
%    \begin{macrocode}
\fi
%    \end{macrocode}

% End of document body:
%    \begin{macrocode}
\end{document}
%    \end{macrocode}
%\iffalse
%</samplemain>
%\fi
%
% %%%%%%%%%%%%%%%%%%%%%%%%%%%%%%%%%%%%%%
% \paragraph{Chapter Include Files.}
%
% The include files are called |cdocsch1.tex| and |cdocsch2.tex|.
%
%\iffalse
%<*samplechap1|samplechap2>
%\fi

% Optional override for |\version| flag:
%    \begin{macrocode}
%%\providecommand{\version}{final}
%    \end{macrocode}

% Include the main document:
%    \begin{macrocode}
\input{childdoc.def}
\childdocof{cdocsamp}
%    \end{macrocode}

%\iffalse
%</samplechap1|samplechap2>
%\fi
%
%\iffalse
%<*samplechap1>
%\fi
% Some text for chapter 1:
%    \begin{macrocode}
\section{one}
some text in chapter one
%    \end{macrocode}

%\iffalse
%</samplechap1>
%\fi
% Some text for chapter 2:
%\iffalse
%<*samplechap2>
%\fi
%    \begin{macrocode}
\section{two}
more text in chapter two
%    \end{macrocode}

%\iffalse
%</samplechap2>
%\fi
%
% %%%%%%%%%%%%%%%%%%%%%%%%%%%%%%%%%%%%%%
% \paragraph{Part Include Files.}
%
% The include files are called |cdocspt3.tex| and |cdocspt4.tex|.
%
%\iffalse
%<*samplepart3|samplepart4>
%\fi

% Optional override for |\version| flag:
%    \begin{macrocode}
%%\providecommand{\version}{final}
%    \end{macrocode}

% Include the main document:
%    \begin{macrocode}
\input{childdoc.def}
\childdocby{cdocsamp}
%    \end{macrocode}

%\iffalse
%</samplepart3|samplepart4>
%\fi
%
%\iffalse
%<*samplepart3>
%\fi
% Some text for part 3:
%    \begin{macrocode}
some text in part three
%    \end{macrocode}

%\iffalse
%</samplepart3>
%\fi
% Some text for part 4:
%\iffalse
%<*samplepart4>
%\fi
%    \begin{macrocode}
more text in part four
%    \end{macrocode}

%\iffalse
%</samplepart4>
%\fi
%
% %%%%%%%%%%%%%%%%%%%%%%%%%%%%%%%%%%%%%%
% \paragraph{Forwarding for a Complete Draft.}
%
% The following forwarding file |cdocsdrf.tex|
% compiles the main document in draft mode:
%\iffalse
%<*sampledraft>
%\fi
%    \begin{macrocode}
\def\version{draft}
\input{childdoc.def}
\childdocforward{cdocsamp}
%    \end{macrocode}

%\iffalse
%</sampledraft>
%\fi
%
% %%%%%%%%%%%%%%%%%%%%%%%%%%%%%%%%%%%%%%
% \paragraph{Forwarding for Final Version of the Chapters.}
%
% The following forwarding files |cdocsfn1.tex| and |cdocsfn2.tex|
% (with identical content)
% compile the final versions of the child documents
% |cdocsch1.tex| and |cdocsch2.tex|, respectively:
%\iffalse
%<*samplefinal>
%\fi
%    \begin{macrocode}
\def\version{final}
\input{childdoc.def}
\childdocforwardprefix[cdocsamp]{cdocsfn}{cdocsch}
%    \end{macrocode}

%\iffalse
%</samplefinal>
%\fi
%
% %%%%%%%%%%%%%%%%%%%%%%%%%%%%%%%%%%%%%%
% \paragraph{Command Line Processing.}
%
% The following three command lines generate the output files
% |cdocscld|, |cdocscl1| and |cdocscl2|
% which should be identical to
% |cdocsdrf|, |cdocsch1| and |cdocsfn2|, respectively:
% \begin{center}
% \begin{tabular}{l}
% |latex -jobname cdocscld \|\\
% |  "\def\version{draft}\input{childdoc.def}\childdocforward{cdocsamp}"|\\
% |latex -jobname cdocscl1 \|\\
% |  "\input{childdoc.def}\childdocforward[cdocsamp]{cdocsch1}"|\\
% |latex -jobname cdocscl2 \|\\
% |  "\def\version{final}\input{childdoc.def}\childdocforward{cdocsch2}"|
% \end{tabular}
% \end{center}
% Note that the trailing backslash on each first line
% merely continues the input to the second line
% (for convenient cut ant paste).
% Furthermore, the command |latex| can be replaced by any
% of its alternative versions such as |pdflatex|.
%
% %%%%%%%%%%%%%%%%%%%%%%%%%%%%%%%%%%%%%%%%%%%%%%%%%%%%%%%%%%%%%%%%%%%%%%%%%%%%%%
% %%%%%%%%%%%%%%%%%%%%%%%%%%%%%%%%%%%%%%%%%%%%%%%%%%%%%%%%%%%%%%%%%%%%%%%%%%%%%%
% \section{Implementation}
%\iffalse
%<*package>
%\fi
%
% This section describes the definitions file |childdoc.def|.

% The definitions cannot be loaded using |\usepackage| or |\RequirePackage|
% which has a mechanism to prevent loading a style file more than once.
% When loading the definitions by means of |\input|
% multiple instances have to be prevented manually:
%\iffalse
%This code needs to be before the `\ProvidesFile' directive
%which is defined at the beginning of this file.
%Therefore it is also placed there and commented out here.
%</package>
%<*discard>
%\fi
%    \begin{macrocode}
\ifdefined\childdocmain\endinput\fi
%    \end{macrocode}
%\iffalse
%</discard>
%<*package>
%\fi
%
% \macro{\ifchilddoc}
% \macro{\ifchilddocmanual}
% The conditional |\ifchilddoc| tells whether a
% child (true) or main (false) document is being compiled.
% The conditional |\ifchilddocmanual| tells whether
% the |\includeonly| mechanism is used (false) or
% the selection of child files must be performed manually (true).
% The definitions initialise to false:
%    \begin{macrocode}
\newif\ifchilddoc
\newif\ifchilddocmanual
%    \end{macrocode}

% \macro{\childdocname}
% \macro{\childdocjob}
% The macro |\childdocname| stores the name of the main document
% to be compiled. The macro |\childdocjob| stores the name of
% the document on which the \LaTeX{} compiler was originally invoked.
% The content of |\jobname| cannot be compared
% to filenames specified in the source due to different catcodes.
% The following code rescans |\jobname|, stores the result
% in |\childdocname| and saves a copy in |\childdocjob|:
%    \begin{macrocode}
\edef\childdocname{\scantokens\expandafter{\jobname\noexpand}}
\let\childdocjob\childdocname
%    \end{macrocode}

% \macro{\childdocdisable}
% The macro |\childdocdisable| prevents the main file
% from being processed more than once.
% At this stage, the main document command |\childdocmain|
% is assumed to be called once again where it should do nothing.
% Any subsequent call to it should prevent
% a secondary processing of the main document
% It overwrites the forwarding commands
% |\childdocof| and |\childdocforward|
% with empty macros to prevent further inclusions of the main document:
%    \begin{macrocode}
\newcommand{\childdocdisable}
{
  \renewcommand{\childdocmain}[1]{\renewcommand{\childdocmain}[1]{\endinput}}
  \renewcommand{\childdocof}[1]{}
  \renewcommand{\childdocby}[2][]{}
  \renewcommand{\childdocforward}[2][]{}
  \renewcommand{\childdocdisable}{}
}
%    \end{macrocode}

% \macro{\childdocmain}
% The macro |\childdocmain| is to be called at the top of the main file
% with nothing or the main filename (without extension) as argument.
% First, it breaks loops.
% If the argument is not empty and does not match |\childdocname|
% (which is set by the first inclusion of |childdoc.def|),
% |\ifchilddoc| is set to true, |\includeonly| is applied to the child file
% and |\jobname| is set to the main file
% (for proper handling of |.aux| files):
%    \begin{macrocode}
\newcommand{\childdocmain}[1]
{
  \childdocdisable\childdocmain{}
  \if?#1?\else
    \begingroup
      \def\childdoctmp{#1}
      \ifx\childdoctmp\childdocname
        \def\childdoctmp{}
      \else
        \def\childdoctmp
        {
          \childdoctrue
          \includeonly{\childdocname}
          \def\childdocjob{#1}
          \def\jobname{#1}
        }
      \fi
      \expandafter
    \endgroup
    \childdoctmp
  \fi
}
%    \end{macrocode}

% \macro{\childdocof}
% The command |\childdocof| redirects
% compilation to the main file |#1|.
%    \begin{macrocode}
\newcommand{\childdocof}[1]
{
  \childdocdisable
  \childdoctrue
  \includeonly{\childdocname}
  \def\jobname{#1}
  \def\childdocjob{#1}
  \input{#1}
}
%    \end{macrocode}

% \macro{\childdocby}
% The command |\childdocby| ....
%    \begin{macrocode}
\newcommand{\childdocby}[2][]
{
  \childdocdisable
  \childdoctrue
  \childdocmanualtrue
  \if?#1?\else
    \def\jobname{#2}
  \fi
  \def\childdocjob{#2}
  \input{#2}
  \endinput
}
%    \end{macrocode}

% \macro{\childdocforward}
% The command |\childdocforward| redirects
% compilation to the main file or
% (if the optional argument is given) a child file.
% Parameters are set as if the main file
% or a child file starting with |\childdocof| was compiled.
% Then compilation is handed over to the main file:
%    \begin{macrocode}
\newcommand{\childdocforward}[2][]
{
  \begingroup
    \if?#1?
      \def\childdoctmp
      {
        \def\childdocname{#2}
        \def\childdocjob{#2}
        \def\jobname{#2}
        \input{#2}
        \endinput
      }
    \else
      \def\childdoctmp
      {
        \childdocdisable
        \def\childdocname{#2}
        \childdoctrue
        \includeonly{#2}
        \def\childdocjob{#1}
        \def\jobname{#1}
        \input{#1}
        \endinput
      }
    \fi
    \expandafter
  \endgroup
  \childdoctmp
}
%    \end{macrocode}

% \macro{\childdocforwardprefix}
% The command |\childdocforwardprefix| redirects
% compilation to the main or a child file by means of a pattern.
% The prefix |#1| in the current filename is replaced by |#2|
% and the suffix of the current filename is kept
% (it is assumed that the filename does not contain the substring `|~~~|'
% which is used as a delimiter).
% Compilation is handed over to the new file by |\childdocforward|:
%    \begin{macrocode}
\newcommand{\childdocforwardprefix}[3][]
{
  \begingroup
    \def\childdocextract #2##1~~~{\def\childdoctmp{\childdocforward[#1]{#3##1}}}
    \expandafter\childdocextract\childdocname~~~
    \expandafter
  \endgroup
  \childdoctmp
}
%    \end{macrocode}

% \macro{\childdoc}
% The deprecated macro |\childdoc| is a legacy version of |\childdocmain|:
%    \begin{macrocode}
\newcommand{\childdoc}{\childdocmain}
%    \end{macrocode}

% \macro{\childdocredirect}
% The deprecated macro |\childdocredirect| is a legacy version
% of |\childdocforward| and |\childdocforwardprefix|:
%    \begin{macrocode}
\newcommand{\childdocredirect}[2][]
{
  \begingroup
    \if?#1?
      \def\childdoctmp{\childdocforward{#2}}
    \else
      \def\childdoctmp{\childdocforwardprefix{#1}{#2}}
    \fi
    \expandafter
  \endgroup
  \childdoctmp
}
%    \end{macrocode}

%\iffalse
%</package>
%\fi
%
\endinput

\childdocby{cdocsamp}
%    \end{macrocode}

%\iffalse
%</samplepart3|samplepart4>
%\fi
%
%\iffalse
%<*samplepart3>
%\fi
% Some text for part 3:
%    \begin{macrocode}
some text in part three
%    \end{macrocode}

%\iffalse
%</samplepart3>
%\fi
% Some text for part 4:
%\iffalse
%<*samplepart4>
%\fi
%    \begin{macrocode}
more text in part four
%    \end{macrocode}

%\iffalse
%</samplepart4>
%\fi
%
% %%%%%%%%%%%%%%%%%%%%%%%%%%%%%%%%%%%%%%
% \paragraph{Forwarding for a Complete Draft.}
%
% The following forwarding file |cdocsdrf.tex|
% compiles the main document in draft mode:
%\iffalse
%<*sampledraft>
%\fi
%    \begin{macrocode}
\def\version{draft}
% \iffalse
%
% childdoc.dtx Copyright (C) 2017-2018 Niklas Beisert
%
% This work may be distributed and/or modified under the
% conditions of the LaTeX Project Public License, either version 1.3
% of this license or (at your option) any later version.
% The latest version of this license is in
%   http://www.latex-project.org/lppl.txt
% and version 1.3 or later is part of all distributions of LaTeX
% version 2005/12/01 or later.
%
% This work has the LPPL maintenance status `maintained'.
%
% The Current Maintainer of this work is Niklas Beisert.
%
% This work consists of the files childdoc.dtx and childdoc.ins
% and the derived files childdoc.def and cdocsamp.tex with
% cdocsch1.tex, cdocsch2.tex, cdocsdrf.tex, cdocsfn1.tex, cdocsfn2.tex.
%
%<package>\ifdefined\childdocmain\endinput\fi
%<package>\ProvidesFile{childdoc.def}[2018/12/30 v2.0 child document driver]
%<samplemain>\ProvidesFile{cdocsamp.tex}[2018/12/30 v2.0 sample for childdoc]
%<*driver>
%\ProvidesFile{childdoc.drv}[2018/12/30 v2.0 childdoc reference manual file]
\PassOptionsToClass{10pt,a4paper}{article}
\documentclass{ltxdoc}

\usepackage[margin=35mm]{geometry}
\usepackage{hyperref}
\usepackage{hyperxmp}
\usepackage[usenames]{color}

\hypersetup{colorlinks=true}
\hypersetup{pdfstartview=FitH}
\hypersetup{pdfpagemode=UseNone}
\hypersetup{pdfsource={}}
\hypersetup{pdflang={en-UK}}
\hypersetup{pdfcopyright={Copyright 2017-2018 Niklas Beisert.
  This work may be distributed and/or modified under the
  conditions of the LaTeX Project Public License, either version 1.3
  of this license or (at your option) any later version.}}
\hypersetup{pdflicenseurl={http://www.latex-project.org/lppl.txt}}
\hypersetup{pdfcontactaddress={ETH Zurich, ITP, HIT K,
  Wolfgang-Pauli-Strasse 27}}
\hypersetup{pdfcontactpostcode={8093}}
\hypersetup{pdfcontactcity={Zurich}}
\hypersetup{pdfcontactcountry={Switzerland}}
\hypersetup{pdfcontactemail={nbeisert@itp.phys.ethz.ch}}
\hypersetup{pdfcontacturl={http://people.phys.ethz.ch/\xmptilde nbeisert/}}

\newcommand{\secref}[1]{\hyperref[#1]{section \ref*{#1}}}

\parskip1ex
\parindent0pt
\let\olditemize\itemize
\def\itemize{\olditemize\parskip0pt}

\begin{document}

\title{The \textsf{childdoc} Package}
\hypersetup{pdftitle={The childdoc Package}}
\author{Niklas Beisert\\[2ex]
  Institut f\"ur Theoretische Physik\\
  Eidgen\"ossische Technische Hochschule Z\"urich\\
  Wolfgang-Pauli-Strasse 27, 8093 Z\"urich, Switzerland\\[1ex]
  \href{mailto:nbeisert@itp.phys.ethz.ch}
  {\texttt{nbeisert@itp.phys.ethz.ch}}}
\hypersetup{pdfauthor={Niklas Beisert}}
\hypersetup{pdfsubject={Manual for the LaTeX2e Package childdoc}}
\date{30 December 2018, \textsf{v2.0}}
\maketitle

\begin{abstract}\noindent
\textsf{childdoc} is a \LaTeXe{} package
that enables the direct compilation
of document sections included by |\include|
to individual files.
\end{abstract}

\begingroup
\parskip0ex
\tableofcontents
\endgroup

%%%%%%%%%%%%%%%%%%%%%%%%%%%%%%%%%%%%%%%%%%%%%%%%%%%%%%%%%%%%%%%%%%%%%%%%%%%%%%%%
%%%%%%%%%%%%%%%%%%%%%%%%%%%%%%%%%%%%%%%%%%%%%%%%%%%%%%%%%%%%%%%%%%%%%%%%%%%%%%%%
\section{Introduction}

\LaTeX{} provides a mechanism to structure a large document (such as a book)
into a main file and several child files (containing the chapters)
using the |\include| command.
This mechanism is beneficial for documents
which span hundreds of pages in order to
make the source file(s) more manageable.
Moreover, compilation can be restricted to
selected child files by means of the |\includeonly| command.
The latter feature can be used to reduce the compilation time while editing
(this was significantly more useful in the earlier days of \LaTeX{})
or to generate a smaller document which is easier to navigate.
Another application of |\includeonly| is to generate
documents consisting of selected parts of the complete document.

However, there are a few drawbacks of the plain |\include| mechanism:
\begin{itemize}
\item
The child files cannot be compiled on their own,
they can only be compiled via the main file.
A naive editing environment
(such as a text editor with an option
to have the current file processed by \LaTeX)
may require one to switch to the main file before compiling;
attempting to compile the child file produces errors.
\item
The main file must be modified (each time)
to adjust the |\includeonly| command
to the present needs. This easily leaves the main file in a messy state.
\item
The generated document will always carry the filename
of the main document. This is inconvenient if
several child files are to be compiled and
to be kept for distribution.
\end{itemize}

The present package provides a simple interface
to make child files individually compilable by \LaTeX{}.
Compiling a child file then has the same effect as compiling
the main file with an |\includeonly| command
to select the appropriate child.
Moreover the generated document will carry the name of the child
rather than the main file.
This resolves all three above issues.

This feature is meant to make the editing of books,
thesis documents and lecture notes somewhat more convenient.
However, the package can also be used efficiently for
composing a series of documents (such as exercise sheets)
which are typically distributed individually.
It then assists the author in generating the individual documents
(potentially in different versions)
as well as a document containing the collected series.
Another application is in developing style files
or other kinds of included material
where compilation of the style file could redirect
to a sample or test file.

%%%%%%%%%%%%%%%%%%%%%%%%%%%%%%%%%%%%%%%%%%%%%%%%%%%%%%%%%%%%%%%%%%%%%%%%%%%%%%%%
%%%%%%%%%%%%%%%%%%%%%%%%%%%%%%%%%%%%%%%%%%%%%%%%%%%%%%%%%%%%%%%%%%%%%%%%%%%%%%%%
\section{Usage}

First of all, the package \textsf{childdoc} is \emph{not} a standard
\LaTeXe{} |.sty| style file! Therefore it needs to be invoked in
a non-standard way.

%%%%%%%%%%%%%%%%%%%%%%%%%%%%%%%%%%%%%%%%%%%%%%%%%%%%%%%%%%%%%%%%%%%%%%%%%%%%%%%%
\subsection{Included Files}
\label{sec:include}

%%%%%%%%%%%%%%%%%%%%%%%%%%%%%%%%%%%%%%%%
\DescribeMacro{\childdocmain}
To use the package, add the commands
\begin{center}
\begin{tabular}{l}
|\input{childdoc.def}|\\
|\childdocmain{}|\\
\end{tabular}
\end{center}
at the very top of the main \LaTeX{} file,
in particular \emph{before} the |\documentclass| statement!
The argument of |\childdocmain| should be left empty
(but it must be present).

%%%%%%%%%%%%%%%%%%%%%%%%%%%%%%%%%%%%%%%%
\DescribeMacro{\childdocof}
Furthermore, add the commands
\begin{center}
\begin{tabular}{l}
|\input{childdoc.def}|\\
|\childdocof{|\textit{main}|}|\\
\end{tabular}
\end{center}
at the top of every child file \textit{child}
which is included by |\include{|\textit{child}|}|
from within the main file
(or at least for those files to be compiled individually).
The argument \textit{main} must be the filename of the main file.

There are a couple of
considerations in setting up the main and child documents:

%%%%%%%%%%%%%%%%%%%%%%%%%%%%%%%%%%%%%%%%
\paragraph{Restrictions.}

Please note the following restrictions:
\begin{itemize}
\item
|\childdocmain| must be called with one argument \textit{main}
to ensure compatibility with earlier version of the package.
It must either be empty (|\childdocmain{}|)
or precisely match the filename of the main file in which it is specified.
See \secref{sec:detection} for further information.
\item
The filename \textit{main} must be specified without the |.tex| extension.
\item
The filename \textit{main} is case sensitive
(even in case-insensitive file systems)
due to internal string comparison.
\item
The argument \textit{main} should be fully expanded, it cannot be a macro.
\item
Subdirectories and special characters should be avoided in filenames.
\item
The command |\childdocmain{|\textit{main}|}| must be followed by a whitespace.
It should not be followed immediately by another command
or by a comment mark `|%|'.
This is because the \TeX{} parser reads the token immediately following
the argument of |\childdocmain| and puts it
at the beginning of every child section;
however, a white\-space is ignored.
\end{itemize}

%%%%%%%%%%%%%%%%%%%%%%%%%%%%%%%%%%%%%%%%
\paragraph{Content of Main File.}

It is advisable to place all content in the child files included by |\include|.
Any output contained in the main file will appear in all child documents
unless suppressed manually;
it cannot be suppressed automatically by the |\includeonly| directive
and thus should normally be avoided.
A method to include some content in the main file
by means of conditional processing is described in \secref{sec:conditional}.

%%%%%%%%%%%%%%%%%%%%%%%%%%%%%%%%%%%%%%%%
\paragraph{Page Numbering.}

When only a part of the document is compiled,
the appropriate numbering of pages
(as well as other status parameters)
is determined from the |.aux| files.
The latter contain information from previous passes.
However this information needs to propagate through
all intermediate child documents.
Therefore the page numbering in child documents may well
be inconsistent until the complete document is compiled at least once.

A useful (if unconventional) way to always ensure a consistent
page numbering is to restart the numbering in each child document
and denote the pages by `\textit{child}|.|\textit{page}'
where \textit{child} represents the chapter/section number of the child file.
This can be achieved by the command
|\numberwithin{page}{|\textit{child}|}|
of the \textsf{amsmath} package
where \textit{child} can be |chapter| or |section|
depending on the chosen structuring.
Alternatively, one can modify the macro |\thepage| appropriately
and reset the counter |page| at the start of each child file.

%%%%%%%%%%%%%%%%%%%%%%%%%%%%%%%%%%%%%%%%%%%%%%%%%%%%%%%%%%%%%%%%%%%%%%%%%%%%%%%%
\subsection{Conditional Processing}
\label{sec:conditional}

The package provides a mechanism to compile different versions
of a document. To customise the versions further some conditional processing
can come in handy to distinguish which version is being compiled.
The package provides two macros to describe the compilation context:

%%%%%%%%%%%%%%%%%%%%%%%%%%%%%%%%%%%%%%%%
\DescribeMacro{\ifchilddoc}
The conditional |\ifchilddoc| distinguishes between the compilation of
child documents and the main document:
%
\begin{center}
|\ifchilddoc |\textit{child-code}| |[|\||else |\textit{main-code}]| \||fi|
\end{center}

%%%%%%%%%%%%%%%%%%%%%%%%%%%%%%%%%%%%%%%%
\DescribeMacro{\childdocname}
\DescribeMacro{\childdocjob}
The macro |\childdocname| contains the filename (without extension)
of the main or child file being processed.
Note that |\childdocjob| will always contain the name of the main file.

%%%%%%%%%%%%%%%%%%%%%%%%%%%%%%%%%%%%%%%%
\paragraph{Title Page.}

Conditional processing can be used to include a title or banner page
in the main document when proper precautions are taken.
Importantly, the code in the main file should ensure that the page counter
(as well as other status parameters which are stored in the |.aux| files)
takes the same value after the conditional processing.
Otherwise the page numbers may take divergent values
depending on which part is compiled.

For example, a title page could be declared by:
%
\begin{center}
\begin{tabular}{l}
|\ifchilddoc\||else|\\
|\addtocounter{page}{-1}|\\
\textit{code for title page}\\
|\newpage|\\
|\||fi|
\end{tabular}
\end{center}
%
A banner page for the child documents can be generated by:
%
\begin{center}
\begin{tabular}{l}
|\ifchilddoc|\\
|\addtocounter{page}{-1}|\\
\textit{code for banner page}\\
|\newpage|\\
|\||fi|
\end{tabular}
\end{center}
%
Here one could write a message such as:
\begin{center}
|This is the part \childdocname{} of \childdocjob{}.|
\end{center}

%%%%%%%%%%%%%%%%%%%%%%%%%%%%%%%%%%%%%%%%%%%%%%%%%%%%%%%%%%%%%%%%%%%%%%%%%%%%%%%%
\subsection{Flags}
\label{sec:flags}

The package makes it easy to generate different versions
of the main or child documents.
To this end compilation flags can be defined
and assigned different default values.
They will be particularly useful in conjunction
with the forwarding mechanism described in \secref{sec:forward}.

For example, it may be useful to have a flag |\version|
which can be set to |draft| or |final|.
The document source will contain some conditional code
depending on the value of |\version|.
Suppose further, the flag should default to |final| for the main file
and to |draft| for child files
which is a natural assignment for editing the document.
This is achieved by placing the following code
in the preamble of the main document
(below the |\childdocmain| directive):
%
\begin{center}
\begin{tabular}{l}
|\ifchilddoc|\\
|\providecommand{\version}{draft}|\\
|\||else|\\
|\providecommand{\version}{final}|\\
|\||fi|
\end{tabular}
\end{center}
%
The definition by |\providecommand| makes sure
that previous definitions are not overwritten.
Further statements |\providecommand{\version}{...}|
can thus be added before the above code to override it.

For the main file, one might add a line
(between |\childdocmain| and the above block)
%
\begin{center}
|%\ifchilddoc\||else\providecommand{\version}{draft}\||fi|
\end{center}
%
which can be uncommented to produce a draft version.
Likewise one can add a line to the very top of a child file
(above the |\childdocof{|\textit{main}|}| directive)
%
\begin{center}
|%\providecommand{\version}{final}|
\end{center}
%
which can be uncommented to produce the final version of this child document.

%%%%%%%%%%%%%%%%%%%%%%%%%%%%%%%%%%%%%%%%%%%%%%%%%%%%%%%%%%%%%%%%%%%%%%%%%%%%%%%%
\subsection{Forwarding}
\label{sec:forward}

Different versions of the main or child documents
using compilation flags as described in \secref{sec:flags}
can be (permanently) stored in different files
for convenient compilation, viewing and distribution.
To this end, the package defines a command
to pass on compilation to a different file:

%%%%%%%%%%%%%%%%%%%%%%%%%%%%%%%%%%%%%%%%
\DescribeMacro{\childdocforward}
The command |\childdocforward| redirects processing to
another source file:
%
\begin{center}
\begin{tabular}{l}
|\input{childdoc.def}|\\
|\childdocforward[|\textit{main}|]{|\textit{dest}|}|\\
\end{tabular}
\end{center}
%
The argument \textit{dest} is the destination file
(without extension).
It should be the main file or one of the child files.
Note that further \textsf{childdoc} directives
such as |\childdocof| and |\childdocforward|
in the indicated file will be processed in this form.
The optional argument \textit{main}
passes on directly to the main file \textit{main}
while pretending to compile the child \textit{dest}.
This form behaves as if \textit{dest}
issues |\childdocof{|\textit{main}|}| right away,
and no further \textsf{childdoc} directives will be processed.

%%%%%%%%%%%%%%%%%%%%%%%%%%%%%%%%%%%%%%%%
\DescribeMacro{\...prefix}
In the alternative form |\childdocforwardprefix|,
%
\begin{center}
\begin{tabular}{l}
|\input{childdoc.def}|\\
|\childdocforwardprefix[|\textit{main}|]{|\textit{prefix}|}{|\textit{dest}|}|
\end{tabular}
\end{center}
%
the destination file is determined by a pattern
depending on the current file:
To make this work, the current file must be called
`{\textit{prefix}\hspace{0.2em}\textit{suffix}}'
with \textit{prefix} matching precisely the argument.
Processing is then passed on to the file
`{\textit{dest}\hspace{0.2em}\textit{suffix}}'.
Surely, the same effect is achieved by
directly specifying the
argument `{\textit{dest}\hspace{0.2em}\textit{suffix}}'
in the first form.
However, that requires to set up a different file
for each child. With the alternative form of the command
all these files can have exactly the same content
which simplifies setting them up and maintaining them.

For example, the following file |draft.tex|
with a compilation flag |\version| as described in \secref{sec:flags}
compiles the main document as a draft:
%
\begin{center}
\begin{tabular}{l}
|\def\version{draft}|\\
|\input{childdoc.def}|\\
|\childdocforward{|\textit{main}|}|
\end{tabular}
\end{center}
%
Likewise, the following files |final|\textit{nn}|.tex|
compile the final version of the child document
|child|\textit{nn}|.tex|:
%
\begin{center}
\begin{tabular}{l}
|\def\version{final}|\\
|\input{childdoc.def}|\\
|\childdocforwardprefix{final}{child}|
\end{tabular}
\end{center}
%

Note that when several versions of a main file and/or of each child file
are to be generated, it may be convenient to set up a |Makefile| or
shell script to automatise the process.

%%%%%%%%%%%%%%%%%%%%%%%%%%%%%%%%%%%%%%%%%%%%%%%%%%%%%%%%%%%%%%%%%%%%%%%%%%%%%%%%
\subsection{Command Line Processing}
\label{sec:commandline}

The effect of redirection files can also be achieved by invoking
the \LaTeX{} compiler with a more elaborate command line.
Most conveniently this should be done as part
of a shell script or a |Makefile|.

When using \textsf{childdoc} in the main file, the following
command lines effectively perform a redirection
(note that depending on the shell being used,
backslashes may have to be doubled: `|\|' $\to$ `|\\|'):
%
\begin{center}
|... -jobname "|\textit{target}|" |\\|"|[\textit{flags}]%
|\input{childdoc.def}\childdocforward[|\textit{main}|]{|\textit{dest}|}"|
\end{center}
%
Here \textit{target} is the name of the output file,
\textit{main} is the name of the main file
and \textit{dest} is the name of the main or child file to be processed
(all filenames without extensions).
The optional argument \textit{main} can be omitted
if \textit{main} matches \textit{dest}.
Optionally, compilation \textit{flags} can be defined via |\def| commands.
This command line makes the \TeX{} engine believe
it is compiling the file \textit{target}
whose content is specified as the latter parameter.
The provided code then forwards the processing to
\textit{main} or \textit{dest} as described in \secref{sec:forward}.

%%%%%%%%%%%%%%%%%%%%%%%%%%%%%%%%%%%%%%%%%%%%%%%%%%%%%%%%%%%%%%%%%%%%%%%%%%%%%%%%
\subsection{Include by Input}
\label{sec:input}

Including child documents by |\include| has some restrictions by design.
Most notably, the content of a child document always occupies
its own set of pages; pages cannot be shared between child documents.
Usually, this behaviour makes perfect sense
because each child document contain an essential part of the document.
However, in some situations it may be desirable to compose
a document from a collection of parts
without having mandatory page breaks between then.
For this case, the package
provides a mechanism to include parts
by |\input| which can also be processed individually.
However, by construction this mechanism
requires manual handling of the content to be output.

%%%%%%%%%%%%%%%%%%%%%%%%%%%%%%%%%%%%%%%%
\DescribeMacro{\ifchilddocmanual}
The main file should be prepared as usual, see \secref{sec:include}.
However, the document body must make a distinction
between processing of an individual part and of the main document, e.g.:
%
\begin{center}
\begin{tabular}{l}
|\ifchilddocmanual|\\
|\input{\childdocname}|\\
|\||else|\\
\textit{document body with }|\input{|\textit{part}|}|\\
|\||fi|
\end{tabular}
\end{center}
%
The conditional |\ifchilddocmanual| is true whenever
a part to be included by |\input| is being compiled,
and the name of the part is stored in |\childdocname|.

%%%%%%%%%%%%%%%%%%%%%%%%%%%%%%%%%%%%%%%%
\DescribeMacro{\childdocby}
Each part to be included by |\input| should start with:
%
\begin{center}
\begin{tabular}{l}
|\input{childdoc.def}|\\
|\childdocby{|\textit{main}|}|\\
\end{tabular}
\end{center}
%
The directive |\childdocby| is similar to |\childdocof|
described in \secref{sec:include},
but the subsequent selection of content must be done manually.
To that end, both |\ifchilddoc| and |\ifchilddocmanual|
will be true upon processing of a part,
and the name of the part is stored in |\childdocname|.
Note that |\jobname| will be set to the filename of the current part
so that each part receives an individual |.aux| file
that does not interfere with the |.aux| file(s) of the main document.
This behaviour can be altered by the alternative form
|\childdocby[*]{|\textit{main}|}| (with a non-empty optional argument)
which uses the |.aux| file of the main document
by setting |\jobname| to \textit{main}.

%%%%%%%%%%%%%%%%%%%%%%%%%%%%%%%%%%%%%%%%%%%%%%%%%%%%%%%%%%%%%%%%%%%%%%%%%%%%%%%%
\subsection{Driver Development}
\label{sec:driver}

The \textsf{childdoc} mechanism can also be use for the development
of definition files such as \LaTeX{} styles or classes.
This case differs from the above setup with multiple parts
included by |\include| in that no |\includeonly| should be invoked.
This can be achieved by starting the include file
(before |\ProvidesPackage|) with:
%
\begin{center}
\begin{tabular}{l}
|\input{childdoc.def}|\\
|\childdocforward{|\textit{main}|}|\\
\end{tabular}
\end{center}
%
or alternatively with:
%
\begin{center}
\begin{tabular}{l}
|\input{childdoc.def}|\\
|\childdocby{|\textit{main}|}|\\
\end{tabular}
\end{center}
%
Both forms have slightly different effects as described above.
The main file is prepared as usual, see \secref{sec:include}.

%%%%%%%%%%%%%%%%%%%%%%%%%%%%%%%%%%%%%%%%%%%%%%%%%%%%%%%%%%%%%%%%%%%%%%%%%%%%%%%%
\subsection{Legacy Detection}
\label{sec:detection}

The directive |\childdocmain| in the main file can detect
whether the complete document or merely a child is to be compiled
even without using the directive |\childdocof|.
This method is deprecated because it is less robust
and there is no compelling reason to use it;
it is merely provided for backward compatibility
and it may be removed in future versions.

If the detection mechanism is to be used,
it is mandatory to correctly specify
the filename of the main file as the argument of |\childdocmain|:
%
\begin{center}
\begin{tabular}{l}
|\input{childdoc.def}|\\
|\childdocmain{|\textit{main}|}|\\
\end{tabular}
\end{center}
%
If |\jobname| does not match the argument \textit{main} of |\childdocmain|,
it is assumed that |\jobname| points to the child file to be compiled.
When using |\childdocmain| with the main file specified as argument,
it suffices to start a child file
with just |\input{|\textit{main}|}|
without loading of the package and using |\childdocof|.
If instead all processing is done
with the appropriate \textsf{childdoc} directives,
the argument of \textit{main} of |\childdocmain| can be empty.

An alternative version of the command line processing described
in \secref{sec:commandline} using the detection mechanism reads:
%
\begin{center}
|... -jobname "|\textit{target}|" "|[\textit{flags}]%
[|\def\jobname{|\textit{dest}|}|]|\input{|\textit{main}|}"|
\end{center}

%%%%%%%%%%%%%%%%%%%%%%%%%%%%%%%%%%%%%%%%%%%%%%%%%%%%%%%%%%%%%%%%%%%%%%%%%%%%%%%%
\subsection{Manual Code}
\label{sec:manual}

In case one cannot be certain whether the definitions file |childdoc.def|
is installed on the target \TeX{} distribution
and one prefers not to ship it,
it is conceivable to paste a few relevant commands into the sources.

To that end, drop all statements |\input{childdoc.def}|
and perform the replacements as outlined below.
Instead of |\childdocmain{|\textit{main}|}| add the following code
to the top of the main file:
%
\begin{center}
\begin{tabular}{l}
|\||ifdefined\childdocname\endinput\||fi\newif\ifchilddoc|\\
|\edef\childdocname{\scantokens\expandafter{\jobname\noexpand}}|\\
|\def\childdocmain{|\textit{main}|}\||ifx\childdocmain\childdocname\||else|\\
|\childdoctrue\includeonly{\childdocname}\let\jobname\childdocmain\||fi|\\
\end{tabular}
\end{center}
%
Instead of |\childdocof{|\textit{main}|}| just include the main file
at the top of each child file:
%
\begin{center}
|\input{|\textit{main}|}|
\end{center}
%
A simple redirection |\childdocforward{|\textit{dest}|}| is achieved by:
%
\begin{center}
|\def\jobname{|\textit{dest}|}\input{\jobname}|
\end{center}
%
The redirection with prefix
|\childdocforwardprefix[|\textit{prefix}|]{|\textit{dest}|}|
is accomplished by:
%
\begin{center}
\begin{tabular}{l}
|{\edef\jobname{\scantokens\expandafter{\jobname\noexpand}}|\\
|\def\redirectjob |\textit{prefix}|#1~~~{\gdef\jobname{|\textit{dest}|#1}}|\\
|\expandafter\redirectjob\jobname~~~}\input{\jobname}|
\end{tabular}
\end{center}

In an alternative approach,
child documents can be compiled by a specific command line
without additional code or specific definitions:
%
\begin{center}
|... -jobname "|\textit{target}|" "|[\textit{flags}]%
|\includeonly{|\textit{dest}|}\input{|\textit{main}|}"|
\end{center}
%

%%%%%%%%%%%%%%%%%%%%%%%%%%%%%%%%%%%%%%%%%%%%%%%%%%%%%%%%%%%%%%%%%%%%%%%%%%%%%%%%
%%%%%%%%%%%%%%%%%%%%%%%%%%%%%%%%%%%%%%%%%%%%%%%%%%%%%%%%%%%%%%%%%%%%%%%%%%%%%%%%
\section{Information}

%%%%%%%%%%%%%%%%%%%%%%%%%%%%%%%%%%%%%%%%%%%%%%%%%%%%%%%%%%%%%%%%%%%%%%%%%%%%%%%%
\subsection{Copyright}

Copyright \copyright{} 2017--2018 Niklas Beisert

This work may be distributed and/or modified under the
conditions of the \LaTeX{} Project Public License, either version 1.3
of this license or (at your option) any later version.
The latest version of this license is in
  \url{http://www.latex-project.org/lppl.txt}
and version 1.3 or later is part of all distributions of \LaTeX{}
version 2005/12/01 or later.

This work has the LPPL maintenance status `maintained'.

The Current Maintainer of this work is Niklas Beisert.

This work consists of the files |README.txt|, |childdoc.ins| and |childdoc.dtx|
as well as the derived files |childdoc.def|, |cdocsamp.tex|
with |cdocsch1.tex|, |cdocsch2.tex|, |cdocspt3.tex|, |cdocspt4.tex|,
|cdocsdrf.tex|, |cdocsfn1.tex|, |cdocsfn2.tex|
as well as |childdoc.pdf|.

%%%%%%%%%%%%%%%%%%%%%%%%%%%%%%%%%%%%%%%%%%%%%%%%%%%%%%%%%%%%%%%%%%%%%%%%%%%%%%%%
\subsection{Files and Installation}

The package consists of the files:
%
\begin{center}
\begin{tabular}{ll}
    |README.txt|   & readme file \\
    |childdoc.ins| & installation file \\
    |childdoc.dtx| & source file \\
    |childdoc.def| & definition file \\
    |cdocsamp.tex| & sample main file \\
    |cdocsch1.tex| & sample include file \\
    |cdocsch2.tex| & sample include file \\
    |cdocspt3.tex| & sample part file \\
    |cdocspt4.tex| & sample part file \\
    |cdocsdrf.tex| & sample redirection file \\
    |cdocsfn1.tex| & sample redirection file \\
    |cdocsfn2.tex| & sample redirection file \\
    |childdoc.pdf| & manual
\end{tabular}
\end{center}
%
The distribution consists of the files
|README.txt|, |childdoc.ins| and |childdoc.dtx|.
%
\begin{itemize}
\item
Run (pdf)\LaTeX{} on |childdoc.dtx|
to compile the manual |childdoc.pdf| (this file).
\item
Run \LaTeX{} on |childdoc.ins| to create the definitions file |childdoc.def|
and the sample |cdocsamp.tex| with include files
|cdocsch1.tex|, |cdocsch2.tex|, |cdocspt3.tex|, |cdocspt4.tex|,
|cdocsdrf.tex|, |cdocsfn1.tex|, |cdocsfn2.tex|.
Then copy the file |childdoc.def| to an appropriate directory of your \LaTeX{}
distribution, e.g.\ \textit{texmf-root}|/tex/latex/childdoc|.
\end{itemize}

%%%%%%%%%%%%%%%%%%%%%%%%%%%%%%%%%%%%%%%%%%%%%%%%%%%%%%%%%%%%%%%%%%%%%%%%%%%%%%%%
\subsection{Related CTAN Packages}

There are several other packages which offer a similar functionality:
%
\begin{itemize}
\item
The packages
\href{http://ctan.org/pkg/docmute}{\textsf{docmute}},
\href{http://ctan.org/pkg/includex}{\textsf{includex}} and
\href{http://ctan.org/pkg/standalone}{\textsf{standalone}}
provide commands to include only the document body of
a child file thus allowing both files to be compiled individually.
\item
The packages \href{http://ctan.org/pkg/subdocs}{\textsf{subdocs}}
and \href{http://ctan.org/pkg/subfiles}{\textsf{subfiles}}
provide structures in which the main and child documents can be
encapsulated and allowing them to be compiled individually.
The inclusion mechanism is different from the conventional |\include|.
\item
The package \href{http://ctan.org/pkg/combine}{\textsf{combine}}
is an elaborate solution to combine several documents into one.
\end{itemize}
%
See also the CTAN topic \href{http://ctan.org/topic/subdocs}{\textsf{subdocs}}
for further related packages.
The present package differs from the above solutions in that
a document structure constructed with the conventional |\include| mechanism
just needs two extra commands at the top of every file
such that all constituent files can be compiled individually.

%%%%%%%%%%%%%%%%%%%%%%%%%%%%%%%%%%%%%%%%%%%%%%%%%%%%%%%%%%%%%%%%%%%%%%%%%%%%%%%%
%\subsection{Feature Suggestions}
%
%The following is a list of features which may be useful for future
%versions of this package:
%%
%\begin{itemize}
%\item
%\ldots
%\end{itemize}

%%%%%%%%%%%%%%%%%%%%%%%%%%%%%%%%%%%%%%%%%%%%%%%%%%%%%%%%%%%%%%%%%%%%%%%%%%%%%%%%
\subsection{Revision History}

%%%%%%%%%%%%%%%%%%%%%%%%%%%%%%%%%%%%%%%%
\paragraph{v2.0:} 2018/12/30

\begin{itemize}
\item
immediate forward processing
\item
added |\childdocby| mechanism
\item
manual restructured
\end{itemize}

%%%%%%%%%%%%%%%%%%%%%%%%%%%%%%%%%%%%%%%%
\paragraph{v1.6:} 2018/01/17

\begin{itemize}
\item
application for development of include files
\item
corrections to manual
\end{itemize}

%%%%%%%%%%%%%%%%%%%%%%%%%%%%%%%%%%%%%%%%
\paragraph{v1.5:} 2017/05/21

\begin{itemize}
\item
more complete structuring introduced
\item
|\childdocof| introduced
\item
|\childdoc| renamed to |\childdocmain|
\item
|\childredirect| renamed to |\childdocforward| and |\childdocforwardprefix|
and functionality expanded
\end{itemize}

%%%%%%%%%%%%%%%%%%%%%%%%%%%%%%%%%%%%%%%%
\paragraph{v1.0:} 2017/04/27

\begin{itemize}
\item
manual and install package
\item
first version published on CTAN
\end{itemize}

%%%%%%%%%%%%%%%%%%%%%%%%%%%%%%%%%%%%%%%%
\paragraph{v0.6:} 2017/04/26

\begin{itemize}
\item
redirection mechanism added
\end{itemize}

%%%%%%%%%%%%%%%%%%%%%%%%%%%%%%%%%%%%%%%%
\paragraph{v0.5:} 2017/04/26

\begin{itemize}
\item
functionality in definition file
\end{itemize}


%%%%%%%%%%%%%%%%%%%%%%%%%%%%%%%%%%%%%%%%%%%%%%%%%%%%%%%%%%%%%%%%%%%%%%%%%%%%%%%%
%%%%%%%%%%%%%%%%%%%%%%%%%%%%%%%%%%%%%%%%%%%%%%%%%%%%%%%%%%%%%%%%%%%%%%%%%%%%%%%%
%%%%%%%%%%%%%%%%%%%%%%%%%%%%%%%%%%%%%%%%%%%%%%%%%%%%%%%%%%%%%%%%%%%%%%%%%%%%%%%%
\appendix

\settowidth\MacroIndent{\rmfamily\scriptsize 000\ }

 \DocInput{childdoc.dtx}

\end{document}
%</driver>
% \fi
%
% %%%%%%%%%%%%%%%%%%%%%%%%%%%%%%%%%%%%%%%%%%%%%%%%%%%%%%%%%%%%%%%%%%%%%%%%%%%%%%
% %%%%%%%%%%%%%%%%%%%%%%%%%%%%%%%%%%%%%%%%%%%%%%%%%%%%%%%%%%%%%%%%%%%%%%%%%%%%%%
% \section{Sample}
%\iffalse
%<*samplemain>
%\fi
%
% The following presents a sample document
% with two chapters, two parts, a title page,
% a compile flag as well as three forwarding files to set the flag.
% It consists of eight |.tex| files:
% \begin{center}
% \begin{tabular}{ll}
% |cdocsamp.tex|&main file\\
% |cdocsch1.tex|&include file for chapter 1\\
% |cdocsch2.tex|&include file for chapter 2\\
% |cdocspt3.tex|&include file for part 3\\
% |cdocspt4.tex|&include file for part 4\\
% |cdocsdrf.tex|&forwarding file for main file in draft mode\\
% |cdocsfi1.tex|&forwarding file for final version of chapter 1\\
% |cdocsfi2.tex|&forwarding file for final version of chapter 2\\
% \end{tabular}
% \end{center}
% Each of the eight files can be compiled directly by the \LaTeX{} compiler.
%
% %%%%%%%%%%%%%%%%%%%%%%%%%%%%%%%%%%%%%%
% \paragraph{Main File.}
%
% The main file is called |cdocsamp.tex|.
%
% Load the \textsf{childdoc} definitions and
% declare the filename for the main document:
%    \begin{macrocode}
\input{childdoc.def}
\childdocmain{}
%    \end{macrocode}

% Optional override for |\version| flag:
%    \begin{macrocode}
%%\ifchilddoc\else\providecommand{\version}{draft}\fi
%    \end{macrocode}

% Define the default values for the |\version| flag
% (|final| for the main file and |draft| for childs):
%    \begin{macrocode}
\ifchilddoc
\providecommand{\version}{draft}
\else
\providecommand{\version}{final}
\fi
%    \end{macrocode}

% Load the standard document class:
%    \begin{macrocode}
\documentclass[12pt]{article}
%    \end{macrocode}

% Start the document body:
%    \begin{macrocode}
\begin{document}
%    \end{macrocode}

% Declare a title page.
% Print title, part of document being processed and version flag:
%    \begin{macrocode}
\addtocounter{page}{-1}
\begin{center}
{\LARGE\bfseries{}childdoc example\par}
\vspace{1cm}
\ifchilddoc
\ifchilddocmanual part\else chapter\fi:
`\childdocname' of `\childdocjob'\par
\else
main document: `\childdocjob'\par
\fi
version: \version\par
\end{center}
\newpage
%    \end{macrocode}

% Manually include selected file,
% otherwise process as usual:
%    \begin{macrocode}
\ifchilddocmanual
\section*{part `\childdocname'}
\input{\childdocname}
\else
%    \end{macrocode}

% Include the two chapters:
%    \begin{macrocode}
\include{cdocsch1}
\include{cdocsch2}
%    \end{macrocode}

% Include the two parts unless only chapters should be displayed:
%    \begin{macrocode}
\ifchilddoc\else
\section{part three}
\input{cdocspt3}
\section{part four}
\input{cdocspt4}
\fi
%    \end{macrocode}

% Process as usual until here:
%    \begin{macrocode}
\fi
%    \end{macrocode}

% End of document body:
%    \begin{macrocode}
\end{document}
%    \end{macrocode}
%\iffalse
%</samplemain>
%\fi
%
% %%%%%%%%%%%%%%%%%%%%%%%%%%%%%%%%%%%%%%
% \paragraph{Chapter Include Files.}
%
% The include files are called |cdocsch1.tex| and |cdocsch2.tex|.
%
%\iffalse
%<*samplechap1|samplechap2>
%\fi

% Optional override for |\version| flag:
%    \begin{macrocode}
%%\providecommand{\version}{final}
%    \end{macrocode}

% Include the main document:
%    \begin{macrocode}
\input{childdoc.def}
\childdocof{cdocsamp}
%    \end{macrocode}

%\iffalse
%</samplechap1|samplechap2>
%\fi
%
%\iffalse
%<*samplechap1>
%\fi
% Some text for chapter 1:
%    \begin{macrocode}
\section{one}
some text in chapter one
%    \end{macrocode}

%\iffalse
%</samplechap1>
%\fi
% Some text for chapter 2:
%\iffalse
%<*samplechap2>
%\fi
%    \begin{macrocode}
\section{two}
more text in chapter two
%    \end{macrocode}

%\iffalse
%</samplechap2>
%\fi
%
% %%%%%%%%%%%%%%%%%%%%%%%%%%%%%%%%%%%%%%
% \paragraph{Part Include Files.}
%
% The include files are called |cdocspt3.tex| and |cdocspt4.tex|.
%
%\iffalse
%<*samplepart3|samplepart4>
%\fi

% Optional override for |\version| flag:
%    \begin{macrocode}
%%\providecommand{\version}{final}
%    \end{macrocode}

% Include the main document:
%    \begin{macrocode}
\input{childdoc.def}
\childdocby{cdocsamp}
%    \end{macrocode}

%\iffalse
%</samplepart3|samplepart4>
%\fi
%
%\iffalse
%<*samplepart3>
%\fi
% Some text for part 3:
%    \begin{macrocode}
some text in part three
%    \end{macrocode}

%\iffalse
%</samplepart3>
%\fi
% Some text for part 4:
%\iffalse
%<*samplepart4>
%\fi
%    \begin{macrocode}
more text in part four
%    \end{macrocode}

%\iffalse
%</samplepart4>
%\fi
%
% %%%%%%%%%%%%%%%%%%%%%%%%%%%%%%%%%%%%%%
% \paragraph{Forwarding for a Complete Draft.}
%
% The following forwarding file |cdocsdrf.tex|
% compiles the main document in draft mode:
%\iffalse
%<*sampledraft>
%\fi
%    \begin{macrocode}
\def\version{draft}
\input{childdoc.def}
\childdocforward{cdocsamp}
%    \end{macrocode}

%\iffalse
%</sampledraft>
%\fi
%
% %%%%%%%%%%%%%%%%%%%%%%%%%%%%%%%%%%%%%%
% \paragraph{Forwarding for Final Version of the Chapters.}
%
% The following forwarding files |cdocsfn1.tex| and |cdocsfn2.tex|
% (with identical content)
% compile the final versions of the child documents
% |cdocsch1.tex| and |cdocsch2.tex|, respectively:
%\iffalse
%<*samplefinal>
%\fi
%    \begin{macrocode}
\def\version{final}
\input{childdoc.def}
\childdocforwardprefix[cdocsamp]{cdocsfn}{cdocsch}
%    \end{macrocode}

%\iffalse
%</samplefinal>
%\fi
%
% %%%%%%%%%%%%%%%%%%%%%%%%%%%%%%%%%%%%%%
% \paragraph{Command Line Processing.}
%
% The following three command lines generate the output files
% |cdocscld|, |cdocscl1| and |cdocscl2|
% which should be identical to
% |cdocsdrf|, |cdocsch1| and |cdocsfn2|, respectively:
% \begin{center}
% \begin{tabular}{l}
% |latex -jobname cdocscld \|\\
% |  "\def\version{draft}\input{childdoc.def}\childdocforward{cdocsamp}"|\\
% |latex -jobname cdocscl1 \|\\
% |  "\input{childdoc.def}\childdocforward[cdocsamp]{cdocsch1}"|\\
% |latex -jobname cdocscl2 \|\\
% |  "\def\version{final}\input{childdoc.def}\childdocforward{cdocsch2}"|
% \end{tabular}
% \end{center}
% Note that the trailing backslash on each first line
% merely continues the input to the second line
% (for convenient cut ant paste).
% Furthermore, the command |latex| can be replaced by any
% of its alternative versions such as |pdflatex|.
%
% %%%%%%%%%%%%%%%%%%%%%%%%%%%%%%%%%%%%%%%%%%%%%%%%%%%%%%%%%%%%%%%%%%%%%%%%%%%%%%
% %%%%%%%%%%%%%%%%%%%%%%%%%%%%%%%%%%%%%%%%%%%%%%%%%%%%%%%%%%%%%%%%%%%%%%%%%%%%%%
% \section{Implementation}
%\iffalse
%<*package>
%\fi
%
% This section describes the definitions file |childdoc.def|.

% The definitions cannot be loaded using |\usepackage| or |\RequirePackage|
% which has a mechanism to prevent loading a style file more than once.
% When loading the definitions by means of |\input|
% multiple instances have to be prevented manually:
%\iffalse
%This code needs to be before the `\ProvidesFile' directive
%which is defined at the beginning of this file.
%Therefore it is also placed there and commented out here.
%</package>
%<*discard>
%\fi
%    \begin{macrocode}
\ifdefined\childdocmain\endinput\fi
%    \end{macrocode}
%\iffalse
%</discard>
%<*package>
%\fi
%
% \macro{\ifchilddoc}
% \macro{\ifchilddocmanual}
% The conditional |\ifchilddoc| tells whether a
% child (true) or main (false) document is being compiled.
% The conditional |\ifchilddocmanual| tells whether
% the |\includeonly| mechanism is used (false) or
% the selection of child files must be performed manually (true).
% The definitions initialise to false:
%    \begin{macrocode}
\newif\ifchilddoc
\newif\ifchilddocmanual
%    \end{macrocode}

% \macro{\childdocname}
% \macro{\childdocjob}
% The macro |\childdocname| stores the name of the main document
% to be compiled. The macro |\childdocjob| stores the name of
% the document on which the \LaTeX{} compiler was originally invoked.
% The content of |\jobname| cannot be compared
% to filenames specified in the source due to different catcodes.
% The following code rescans |\jobname|, stores the result
% in |\childdocname| and saves a copy in |\childdocjob|:
%    \begin{macrocode}
\edef\childdocname{\scantokens\expandafter{\jobname\noexpand}}
\let\childdocjob\childdocname
%    \end{macrocode}

% \macro{\childdocdisable}
% The macro |\childdocdisable| prevents the main file
% from being processed more than once.
% At this stage, the main document command |\childdocmain|
% is assumed to be called once again where it should do nothing.
% Any subsequent call to it should prevent
% a secondary processing of the main document
% It overwrites the forwarding commands
% |\childdocof| and |\childdocforward|
% with empty macros to prevent further inclusions of the main document:
%    \begin{macrocode}
\newcommand{\childdocdisable}
{
  \renewcommand{\childdocmain}[1]{\renewcommand{\childdocmain}[1]{\endinput}}
  \renewcommand{\childdocof}[1]{}
  \renewcommand{\childdocby}[2][]{}
  \renewcommand{\childdocforward}[2][]{}
  \renewcommand{\childdocdisable}{}
}
%    \end{macrocode}

% \macro{\childdocmain}
% The macro |\childdocmain| is to be called at the top of the main file
% with nothing or the main filename (without extension) as argument.
% First, it breaks loops.
% If the argument is not empty and does not match |\childdocname|
% (which is set by the first inclusion of |childdoc.def|),
% |\ifchilddoc| is set to true, |\includeonly| is applied to the child file
% and |\jobname| is set to the main file
% (for proper handling of |.aux| files):
%    \begin{macrocode}
\newcommand{\childdocmain}[1]
{
  \childdocdisable\childdocmain{}
  \if?#1?\else
    \begingroup
      \def\childdoctmp{#1}
      \ifx\childdoctmp\childdocname
        \def\childdoctmp{}
      \else
        \def\childdoctmp
        {
          \childdoctrue
          \includeonly{\childdocname}
          \def\childdocjob{#1}
          \def\jobname{#1}
        }
      \fi
      \expandafter
    \endgroup
    \childdoctmp
  \fi
}
%    \end{macrocode}

% \macro{\childdocof}
% The command |\childdocof| redirects
% compilation to the main file |#1|.
%    \begin{macrocode}
\newcommand{\childdocof}[1]
{
  \childdocdisable
  \childdoctrue
  \includeonly{\childdocname}
  \def\jobname{#1}
  \def\childdocjob{#1}
  \input{#1}
}
%    \end{macrocode}

% \macro{\childdocby}
% The command |\childdocby| ....
%    \begin{macrocode}
\newcommand{\childdocby}[2][]
{
  \childdocdisable
  \childdoctrue
  \childdocmanualtrue
  \if?#1?\else
    \def\jobname{#2}
  \fi
  \def\childdocjob{#2}
  \input{#2}
  \endinput
}
%    \end{macrocode}

% \macro{\childdocforward}
% The command |\childdocforward| redirects
% compilation to the main file or
% (if the optional argument is given) a child file.
% Parameters are set as if the main file
% or a child file starting with |\childdocof| was compiled.
% Then compilation is handed over to the main file:
%    \begin{macrocode}
\newcommand{\childdocforward}[2][]
{
  \begingroup
    \if?#1?
      \def\childdoctmp
      {
        \def\childdocname{#2}
        \def\childdocjob{#2}
        \def\jobname{#2}
        \input{#2}
        \endinput
      }
    \else
      \def\childdoctmp
      {
        \childdocdisable
        \def\childdocname{#2}
        \childdoctrue
        \includeonly{#2}
        \def\childdocjob{#1}
        \def\jobname{#1}
        \input{#1}
        \endinput
      }
    \fi
    \expandafter
  \endgroup
  \childdoctmp
}
%    \end{macrocode}

% \macro{\childdocforwardprefix}
% The command |\childdocforwardprefix| redirects
% compilation to the main or a child file by means of a pattern.
% The prefix |#1| in the current filename is replaced by |#2|
% and the suffix of the current filename is kept
% (it is assumed that the filename does not contain the substring `|~~~|'
% which is used as a delimiter).
% Compilation is handed over to the new file by |\childdocforward|:
%    \begin{macrocode}
\newcommand{\childdocforwardprefix}[3][]
{
  \begingroup
    \def\childdocextract #2##1~~~{\def\childdoctmp{\childdocforward[#1]{#3##1}}}
    \expandafter\childdocextract\childdocname~~~
    \expandafter
  \endgroup
  \childdoctmp
}
%    \end{macrocode}

% \macro{\childdoc}
% The deprecated macro |\childdoc| is a legacy version of |\childdocmain|:
%    \begin{macrocode}
\newcommand{\childdoc}{\childdocmain}
%    \end{macrocode}

% \macro{\childdocredirect}
% The deprecated macro |\childdocredirect| is a legacy version
% of |\childdocforward| and |\childdocforwardprefix|:
%    \begin{macrocode}
\newcommand{\childdocredirect}[2][]
{
  \begingroup
    \if?#1?
      \def\childdoctmp{\childdocforward{#2}}
    \else
      \def\childdoctmp{\childdocforwardprefix{#1}{#2}}
    \fi
    \expandafter
  \endgroup
  \childdoctmp
}
%    \end{macrocode}

%\iffalse
%</package>
%\fi
%
\endinput

\childdocforward{cdocsamp}
%    \end{macrocode}

%\iffalse
%</sampledraft>
%\fi
%
% %%%%%%%%%%%%%%%%%%%%%%%%%%%%%%%%%%%%%%
% \paragraph{Forwarding for Final Version of the Chapters.}
%
% The following forwarding files |cdocsfn1.tex| and |cdocsfn2.tex|
% (with identical content)
% compile the final versions of the child documents
% |cdocsch1.tex| and |cdocsch2.tex|, respectively:
%\iffalse
%<*samplefinal>
%\fi
%    \begin{macrocode}
\def\version{final}
% \iffalse
%
% childdoc.dtx Copyright (C) 2017-2018 Niklas Beisert
%
% This work may be distributed and/or modified under the
% conditions of the LaTeX Project Public License, either version 1.3
% of this license or (at your option) any later version.
% The latest version of this license is in
%   http://www.latex-project.org/lppl.txt
% and version 1.3 or later is part of all distributions of LaTeX
% version 2005/12/01 or later.
%
% This work has the LPPL maintenance status `maintained'.
%
% The Current Maintainer of this work is Niklas Beisert.
%
% This work consists of the files childdoc.dtx and childdoc.ins
% and the derived files childdoc.def and cdocsamp.tex with
% cdocsch1.tex, cdocsch2.tex, cdocsdrf.tex, cdocsfn1.tex, cdocsfn2.tex.
%
%<package>\ifdefined\childdocmain\endinput\fi
%<package>\ProvidesFile{childdoc.def}[2018/12/30 v2.0 child document driver]
%<samplemain>\ProvidesFile{cdocsamp.tex}[2018/12/30 v2.0 sample for childdoc]
%<*driver>
%\ProvidesFile{childdoc.drv}[2018/12/30 v2.0 childdoc reference manual file]
\PassOptionsToClass{10pt,a4paper}{article}
\documentclass{ltxdoc}

\usepackage[margin=35mm]{geometry}
\usepackage{hyperref}
\usepackage{hyperxmp}
\usepackage[usenames]{color}

\hypersetup{colorlinks=true}
\hypersetup{pdfstartview=FitH}
\hypersetup{pdfpagemode=UseNone}
\hypersetup{pdfsource={}}
\hypersetup{pdflang={en-UK}}
\hypersetup{pdfcopyright={Copyright 2017-2018 Niklas Beisert.
  This work may be distributed and/or modified under the
  conditions of the LaTeX Project Public License, either version 1.3
  of this license or (at your option) any later version.}}
\hypersetup{pdflicenseurl={http://www.latex-project.org/lppl.txt}}
\hypersetup{pdfcontactaddress={ETH Zurich, ITP, HIT K,
  Wolfgang-Pauli-Strasse 27}}
\hypersetup{pdfcontactpostcode={8093}}
\hypersetup{pdfcontactcity={Zurich}}
\hypersetup{pdfcontactcountry={Switzerland}}
\hypersetup{pdfcontactemail={nbeisert@itp.phys.ethz.ch}}
\hypersetup{pdfcontacturl={http://people.phys.ethz.ch/\xmptilde nbeisert/}}

\newcommand{\secref}[1]{\hyperref[#1]{section \ref*{#1}}}

\parskip1ex
\parindent0pt
\let\olditemize\itemize
\def\itemize{\olditemize\parskip0pt}

\begin{document}

\title{The \textsf{childdoc} Package}
\hypersetup{pdftitle={The childdoc Package}}
\author{Niklas Beisert\\[2ex]
  Institut f\"ur Theoretische Physik\\
  Eidgen\"ossische Technische Hochschule Z\"urich\\
  Wolfgang-Pauli-Strasse 27, 8093 Z\"urich, Switzerland\\[1ex]
  \href{mailto:nbeisert@itp.phys.ethz.ch}
  {\texttt{nbeisert@itp.phys.ethz.ch}}}
\hypersetup{pdfauthor={Niklas Beisert}}
\hypersetup{pdfsubject={Manual for the LaTeX2e Package childdoc}}
\date{30 December 2018, \textsf{v2.0}}
\maketitle

\begin{abstract}\noindent
\textsf{childdoc} is a \LaTeXe{} package
that enables the direct compilation
of document sections included by |\include|
to individual files.
\end{abstract}

\begingroup
\parskip0ex
\tableofcontents
\endgroup

%%%%%%%%%%%%%%%%%%%%%%%%%%%%%%%%%%%%%%%%%%%%%%%%%%%%%%%%%%%%%%%%%%%%%%%%%%%%%%%%
%%%%%%%%%%%%%%%%%%%%%%%%%%%%%%%%%%%%%%%%%%%%%%%%%%%%%%%%%%%%%%%%%%%%%%%%%%%%%%%%
\section{Introduction}

\LaTeX{} provides a mechanism to structure a large document (such as a book)
into a main file and several child files (containing the chapters)
using the |\include| command.
This mechanism is beneficial for documents
which span hundreds of pages in order to
make the source file(s) more manageable.
Moreover, compilation can be restricted to
selected child files by means of the |\includeonly| command.
The latter feature can be used to reduce the compilation time while editing
(this was significantly more useful in the earlier days of \LaTeX{})
or to generate a smaller document which is easier to navigate.
Another application of |\includeonly| is to generate
documents consisting of selected parts of the complete document.

However, there are a few drawbacks of the plain |\include| mechanism:
\begin{itemize}
\item
The child files cannot be compiled on their own,
they can only be compiled via the main file.
A naive editing environment
(such as a text editor with an option
to have the current file processed by \LaTeX)
may require one to switch to the main file before compiling;
attempting to compile the child file produces errors.
\item
The main file must be modified (each time)
to adjust the |\includeonly| command
to the present needs. This easily leaves the main file in a messy state.
\item
The generated document will always carry the filename
of the main document. This is inconvenient if
several child files are to be compiled and
to be kept for distribution.
\end{itemize}

The present package provides a simple interface
to make child files individually compilable by \LaTeX{}.
Compiling a child file then has the same effect as compiling
the main file with an |\includeonly| command
to select the appropriate child.
Moreover the generated document will carry the name of the child
rather than the main file.
This resolves all three above issues.

This feature is meant to make the editing of books,
thesis documents and lecture notes somewhat more convenient.
However, the package can also be used efficiently for
composing a series of documents (such as exercise sheets)
which are typically distributed individually.
It then assists the author in generating the individual documents
(potentially in different versions)
as well as a document containing the collected series.
Another application is in developing style files
or other kinds of included material
where compilation of the style file could redirect
to a sample or test file.

%%%%%%%%%%%%%%%%%%%%%%%%%%%%%%%%%%%%%%%%%%%%%%%%%%%%%%%%%%%%%%%%%%%%%%%%%%%%%%%%
%%%%%%%%%%%%%%%%%%%%%%%%%%%%%%%%%%%%%%%%%%%%%%%%%%%%%%%%%%%%%%%%%%%%%%%%%%%%%%%%
\section{Usage}

First of all, the package \textsf{childdoc} is \emph{not} a standard
\LaTeXe{} |.sty| style file! Therefore it needs to be invoked in
a non-standard way.

%%%%%%%%%%%%%%%%%%%%%%%%%%%%%%%%%%%%%%%%%%%%%%%%%%%%%%%%%%%%%%%%%%%%%%%%%%%%%%%%
\subsection{Included Files}
\label{sec:include}

%%%%%%%%%%%%%%%%%%%%%%%%%%%%%%%%%%%%%%%%
\DescribeMacro{\childdocmain}
To use the package, add the commands
\begin{center}
\begin{tabular}{l}
|\input{childdoc.def}|\\
|\childdocmain{}|\\
\end{tabular}
\end{center}
at the very top of the main \LaTeX{} file,
in particular \emph{before} the |\documentclass| statement!
The argument of |\childdocmain| should be left empty
(but it must be present).

%%%%%%%%%%%%%%%%%%%%%%%%%%%%%%%%%%%%%%%%
\DescribeMacro{\childdocof}
Furthermore, add the commands
\begin{center}
\begin{tabular}{l}
|\input{childdoc.def}|\\
|\childdocof{|\textit{main}|}|\\
\end{tabular}
\end{center}
at the top of every child file \textit{child}
which is included by |\include{|\textit{child}|}|
from within the main file
(or at least for those files to be compiled individually).
The argument \textit{main} must be the filename of the main file.

There are a couple of
considerations in setting up the main and child documents:

%%%%%%%%%%%%%%%%%%%%%%%%%%%%%%%%%%%%%%%%
\paragraph{Restrictions.}

Please note the following restrictions:
\begin{itemize}
\item
|\childdocmain| must be called with one argument \textit{main}
to ensure compatibility with earlier version of the package.
It must either be empty (|\childdocmain{}|)
or precisely match the filename of the main file in which it is specified.
See \secref{sec:detection} for further information.
\item
The filename \textit{main} must be specified without the |.tex| extension.
\item
The filename \textit{main} is case sensitive
(even in case-insensitive file systems)
due to internal string comparison.
\item
The argument \textit{main} should be fully expanded, it cannot be a macro.
\item
Subdirectories and special characters should be avoided in filenames.
\item
The command |\childdocmain{|\textit{main}|}| must be followed by a whitespace.
It should not be followed immediately by another command
or by a comment mark `|%|'.
This is because the \TeX{} parser reads the token immediately following
the argument of |\childdocmain| and puts it
at the beginning of every child section;
however, a white\-space is ignored.
\end{itemize}

%%%%%%%%%%%%%%%%%%%%%%%%%%%%%%%%%%%%%%%%
\paragraph{Content of Main File.}

It is advisable to place all content in the child files included by |\include|.
Any output contained in the main file will appear in all child documents
unless suppressed manually;
it cannot be suppressed automatically by the |\includeonly| directive
and thus should normally be avoided.
A method to include some content in the main file
by means of conditional processing is described in \secref{sec:conditional}.

%%%%%%%%%%%%%%%%%%%%%%%%%%%%%%%%%%%%%%%%
\paragraph{Page Numbering.}

When only a part of the document is compiled,
the appropriate numbering of pages
(as well as other status parameters)
is determined from the |.aux| files.
The latter contain information from previous passes.
However this information needs to propagate through
all intermediate child documents.
Therefore the page numbering in child documents may well
be inconsistent until the complete document is compiled at least once.

A useful (if unconventional) way to always ensure a consistent
page numbering is to restart the numbering in each child document
and denote the pages by `\textit{child}|.|\textit{page}'
where \textit{child} represents the chapter/section number of the child file.
This can be achieved by the command
|\numberwithin{page}{|\textit{child}|}|
of the \textsf{amsmath} package
where \textit{child} can be |chapter| or |section|
depending on the chosen structuring.
Alternatively, one can modify the macro |\thepage| appropriately
and reset the counter |page| at the start of each child file.

%%%%%%%%%%%%%%%%%%%%%%%%%%%%%%%%%%%%%%%%%%%%%%%%%%%%%%%%%%%%%%%%%%%%%%%%%%%%%%%%
\subsection{Conditional Processing}
\label{sec:conditional}

The package provides a mechanism to compile different versions
of a document. To customise the versions further some conditional processing
can come in handy to distinguish which version is being compiled.
The package provides two macros to describe the compilation context:

%%%%%%%%%%%%%%%%%%%%%%%%%%%%%%%%%%%%%%%%
\DescribeMacro{\ifchilddoc}
The conditional |\ifchilddoc| distinguishes between the compilation of
child documents and the main document:
%
\begin{center}
|\ifchilddoc |\textit{child-code}| |[|\||else |\textit{main-code}]| \||fi|
\end{center}

%%%%%%%%%%%%%%%%%%%%%%%%%%%%%%%%%%%%%%%%
\DescribeMacro{\childdocname}
\DescribeMacro{\childdocjob}
The macro |\childdocname| contains the filename (without extension)
of the main or child file being processed.
Note that |\childdocjob| will always contain the name of the main file.

%%%%%%%%%%%%%%%%%%%%%%%%%%%%%%%%%%%%%%%%
\paragraph{Title Page.}

Conditional processing can be used to include a title or banner page
in the main document when proper precautions are taken.
Importantly, the code in the main file should ensure that the page counter
(as well as other status parameters which are stored in the |.aux| files)
takes the same value after the conditional processing.
Otherwise the page numbers may take divergent values
depending on which part is compiled.

For example, a title page could be declared by:
%
\begin{center}
\begin{tabular}{l}
|\ifchilddoc\||else|\\
|\addtocounter{page}{-1}|\\
\textit{code for title page}\\
|\newpage|\\
|\||fi|
\end{tabular}
\end{center}
%
A banner page for the child documents can be generated by:
%
\begin{center}
\begin{tabular}{l}
|\ifchilddoc|\\
|\addtocounter{page}{-1}|\\
\textit{code for banner page}\\
|\newpage|\\
|\||fi|
\end{tabular}
\end{center}
%
Here one could write a message such as:
\begin{center}
|This is the part \childdocname{} of \childdocjob{}.|
\end{center}

%%%%%%%%%%%%%%%%%%%%%%%%%%%%%%%%%%%%%%%%%%%%%%%%%%%%%%%%%%%%%%%%%%%%%%%%%%%%%%%%
\subsection{Flags}
\label{sec:flags}

The package makes it easy to generate different versions
of the main or child documents.
To this end compilation flags can be defined
and assigned different default values.
They will be particularly useful in conjunction
with the forwarding mechanism described in \secref{sec:forward}.

For example, it may be useful to have a flag |\version|
which can be set to |draft| or |final|.
The document source will contain some conditional code
depending on the value of |\version|.
Suppose further, the flag should default to |final| for the main file
and to |draft| for child files
which is a natural assignment for editing the document.
This is achieved by placing the following code
in the preamble of the main document
(below the |\childdocmain| directive):
%
\begin{center}
\begin{tabular}{l}
|\ifchilddoc|\\
|\providecommand{\version}{draft}|\\
|\||else|\\
|\providecommand{\version}{final}|\\
|\||fi|
\end{tabular}
\end{center}
%
The definition by |\providecommand| makes sure
that previous definitions are not overwritten.
Further statements |\providecommand{\version}{...}|
can thus be added before the above code to override it.

For the main file, one might add a line
(between |\childdocmain| and the above block)
%
\begin{center}
|%\ifchilddoc\||else\providecommand{\version}{draft}\||fi|
\end{center}
%
which can be uncommented to produce a draft version.
Likewise one can add a line to the very top of a child file
(above the |\childdocof{|\textit{main}|}| directive)
%
\begin{center}
|%\providecommand{\version}{final}|
\end{center}
%
which can be uncommented to produce the final version of this child document.

%%%%%%%%%%%%%%%%%%%%%%%%%%%%%%%%%%%%%%%%%%%%%%%%%%%%%%%%%%%%%%%%%%%%%%%%%%%%%%%%
\subsection{Forwarding}
\label{sec:forward}

Different versions of the main or child documents
using compilation flags as described in \secref{sec:flags}
can be (permanently) stored in different files
for convenient compilation, viewing and distribution.
To this end, the package defines a command
to pass on compilation to a different file:

%%%%%%%%%%%%%%%%%%%%%%%%%%%%%%%%%%%%%%%%
\DescribeMacro{\childdocforward}
The command |\childdocforward| redirects processing to
another source file:
%
\begin{center}
\begin{tabular}{l}
|\input{childdoc.def}|\\
|\childdocforward[|\textit{main}|]{|\textit{dest}|}|\\
\end{tabular}
\end{center}
%
The argument \textit{dest} is the destination file
(without extension).
It should be the main file or one of the child files.
Note that further \textsf{childdoc} directives
such as |\childdocof| and |\childdocforward|
in the indicated file will be processed in this form.
The optional argument \textit{main}
passes on directly to the main file \textit{main}
while pretending to compile the child \textit{dest}.
This form behaves as if \textit{dest}
issues |\childdocof{|\textit{main}|}| right away,
and no further \textsf{childdoc} directives will be processed.

%%%%%%%%%%%%%%%%%%%%%%%%%%%%%%%%%%%%%%%%
\DescribeMacro{\...prefix}
In the alternative form |\childdocforwardprefix|,
%
\begin{center}
\begin{tabular}{l}
|\input{childdoc.def}|\\
|\childdocforwardprefix[|\textit{main}|]{|\textit{prefix}|}{|\textit{dest}|}|
\end{tabular}
\end{center}
%
the destination file is determined by a pattern
depending on the current file:
To make this work, the current file must be called
`{\textit{prefix}\hspace{0.2em}\textit{suffix}}'
with \textit{prefix} matching precisely the argument.
Processing is then passed on to the file
`{\textit{dest}\hspace{0.2em}\textit{suffix}}'.
Surely, the same effect is achieved by
directly specifying the
argument `{\textit{dest}\hspace{0.2em}\textit{suffix}}'
in the first form.
However, that requires to set up a different file
for each child. With the alternative form of the command
all these files can have exactly the same content
which simplifies setting them up and maintaining them.

For example, the following file |draft.tex|
with a compilation flag |\version| as described in \secref{sec:flags}
compiles the main document as a draft:
%
\begin{center}
\begin{tabular}{l}
|\def\version{draft}|\\
|\input{childdoc.def}|\\
|\childdocforward{|\textit{main}|}|
\end{tabular}
\end{center}
%
Likewise, the following files |final|\textit{nn}|.tex|
compile the final version of the child document
|child|\textit{nn}|.tex|:
%
\begin{center}
\begin{tabular}{l}
|\def\version{final}|\\
|\input{childdoc.def}|\\
|\childdocforwardprefix{final}{child}|
\end{tabular}
\end{center}
%

Note that when several versions of a main file and/or of each child file
are to be generated, it may be convenient to set up a |Makefile| or
shell script to automatise the process.

%%%%%%%%%%%%%%%%%%%%%%%%%%%%%%%%%%%%%%%%%%%%%%%%%%%%%%%%%%%%%%%%%%%%%%%%%%%%%%%%
\subsection{Command Line Processing}
\label{sec:commandline}

The effect of redirection files can also be achieved by invoking
the \LaTeX{} compiler with a more elaborate command line.
Most conveniently this should be done as part
of a shell script or a |Makefile|.

When using \textsf{childdoc} in the main file, the following
command lines effectively perform a redirection
(note that depending on the shell being used,
backslashes may have to be doubled: `|\|' $\to$ `|\\|'):
%
\begin{center}
|... -jobname "|\textit{target}|" |\\|"|[\textit{flags}]%
|\input{childdoc.def}\childdocforward[|\textit{main}|]{|\textit{dest}|}"|
\end{center}
%
Here \textit{target} is the name of the output file,
\textit{main} is the name of the main file
and \textit{dest} is the name of the main or child file to be processed
(all filenames without extensions).
The optional argument \textit{main} can be omitted
if \textit{main} matches \textit{dest}.
Optionally, compilation \textit{flags} can be defined via |\def| commands.
This command line makes the \TeX{} engine believe
it is compiling the file \textit{target}
whose content is specified as the latter parameter.
The provided code then forwards the processing to
\textit{main} or \textit{dest} as described in \secref{sec:forward}.

%%%%%%%%%%%%%%%%%%%%%%%%%%%%%%%%%%%%%%%%%%%%%%%%%%%%%%%%%%%%%%%%%%%%%%%%%%%%%%%%
\subsection{Include by Input}
\label{sec:input}

Including child documents by |\include| has some restrictions by design.
Most notably, the content of a child document always occupies
its own set of pages; pages cannot be shared between child documents.
Usually, this behaviour makes perfect sense
because each child document contain an essential part of the document.
However, in some situations it may be desirable to compose
a document from a collection of parts
without having mandatory page breaks between then.
For this case, the package
provides a mechanism to include parts
by |\input| which can also be processed individually.
However, by construction this mechanism
requires manual handling of the content to be output.

%%%%%%%%%%%%%%%%%%%%%%%%%%%%%%%%%%%%%%%%
\DescribeMacro{\ifchilddocmanual}
The main file should be prepared as usual, see \secref{sec:include}.
However, the document body must make a distinction
between processing of an individual part and of the main document, e.g.:
%
\begin{center}
\begin{tabular}{l}
|\ifchilddocmanual|\\
|\input{\childdocname}|\\
|\||else|\\
\textit{document body with }|\input{|\textit{part}|}|\\
|\||fi|
\end{tabular}
\end{center}
%
The conditional |\ifchilddocmanual| is true whenever
a part to be included by |\input| is being compiled,
and the name of the part is stored in |\childdocname|.

%%%%%%%%%%%%%%%%%%%%%%%%%%%%%%%%%%%%%%%%
\DescribeMacro{\childdocby}
Each part to be included by |\input| should start with:
%
\begin{center}
\begin{tabular}{l}
|\input{childdoc.def}|\\
|\childdocby{|\textit{main}|}|\\
\end{tabular}
\end{center}
%
The directive |\childdocby| is similar to |\childdocof|
described in \secref{sec:include},
but the subsequent selection of content must be done manually.
To that end, both |\ifchilddoc| and |\ifchilddocmanual|
will be true upon processing of a part,
and the name of the part is stored in |\childdocname|.
Note that |\jobname| will be set to the filename of the current part
so that each part receives an individual |.aux| file
that does not interfere with the |.aux| file(s) of the main document.
This behaviour can be altered by the alternative form
|\childdocby[*]{|\textit{main}|}| (with a non-empty optional argument)
which uses the |.aux| file of the main document
by setting |\jobname| to \textit{main}.

%%%%%%%%%%%%%%%%%%%%%%%%%%%%%%%%%%%%%%%%%%%%%%%%%%%%%%%%%%%%%%%%%%%%%%%%%%%%%%%%
\subsection{Driver Development}
\label{sec:driver}

The \textsf{childdoc} mechanism can also be use for the development
of definition files such as \LaTeX{} styles or classes.
This case differs from the above setup with multiple parts
included by |\include| in that no |\includeonly| should be invoked.
This can be achieved by starting the include file
(before |\ProvidesPackage|) with:
%
\begin{center}
\begin{tabular}{l}
|\input{childdoc.def}|\\
|\childdocforward{|\textit{main}|}|\\
\end{tabular}
\end{center}
%
or alternatively with:
%
\begin{center}
\begin{tabular}{l}
|\input{childdoc.def}|\\
|\childdocby{|\textit{main}|}|\\
\end{tabular}
\end{center}
%
Both forms have slightly different effects as described above.
The main file is prepared as usual, see \secref{sec:include}.

%%%%%%%%%%%%%%%%%%%%%%%%%%%%%%%%%%%%%%%%%%%%%%%%%%%%%%%%%%%%%%%%%%%%%%%%%%%%%%%%
\subsection{Legacy Detection}
\label{sec:detection}

The directive |\childdocmain| in the main file can detect
whether the complete document or merely a child is to be compiled
even without using the directive |\childdocof|.
This method is deprecated because it is less robust
and there is no compelling reason to use it;
it is merely provided for backward compatibility
and it may be removed in future versions.

If the detection mechanism is to be used,
it is mandatory to correctly specify
the filename of the main file as the argument of |\childdocmain|:
%
\begin{center}
\begin{tabular}{l}
|\input{childdoc.def}|\\
|\childdocmain{|\textit{main}|}|\\
\end{tabular}
\end{center}
%
If |\jobname| does not match the argument \textit{main} of |\childdocmain|,
it is assumed that |\jobname| points to the child file to be compiled.
When using |\childdocmain| with the main file specified as argument,
it suffices to start a child file
with just |\input{|\textit{main}|}|
without loading of the package and using |\childdocof|.
If instead all processing is done
with the appropriate \textsf{childdoc} directives,
the argument of \textit{main} of |\childdocmain| can be empty.

An alternative version of the command line processing described
in \secref{sec:commandline} using the detection mechanism reads:
%
\begin{center}
|... -jobname "|\textit{target}|" "|[\textit{flags}]%
[|\def\jobname{|\textit{dest}|}|]|\input{|\textit{main}|}"|
\end{center}

%%%%%%%%%%%%%%%%%%%%%%%%%%%%%%%%%%%%%%%%%%%%%%%%%%%%%%%%%%%%%%%%%%%%%%%%%%%%%%%%
\subsection{Manual Code}
\label{sec:manual}

In case one cannot be certain whether the definitions file |childdoc.def|
is installed on the target \TeX{} distribution
and one prefers not to ship it,
it is conceivable to paste a few relevant commands into the sources.

To that end, drop all statements |\input{childdoc.def}|
and perform the replacements as outlined below.
Instead of |\childdocmain{|\textit{main}|}| add the following code
to the top of the main file:
%
\begin{center}
\begin{tabular}{l}
|\||ifdefined\childdocname\endinput\||fi\newif\ifchilddoc|\\
|\edef\childdocname{\scantokens\expandafter{\jobname\noexpand}}|\\
|\def\childdocmain{|\textit{main}|}\||ifx\childdocmain\childdocname\||else|\\
|\childdoctrue\includeonly{\childdocname}\let\jobname\childdocmain\||fi|\\
\end{tabular}
\end{center}
%
Instead of |\childdocof{|\textit{main}|}| just include the main file
at the top of each child file:
%
\begin{center}
|\input{|\textit{main}|}|
\end{center}
%
A simple redirection |\childdocforward{|\textit{dest}|}| is achieved by:
%
\begin{center}
|\def\jobname{|\textit{dest}|}\input{\jobname}|
\end{center}
%
The redirection with prefix
|\childdocforwardprefix[|\textit{prefix}|]{|\textit{dest}|}|
is accomplished by:
%
\begin{center}
\begin{tabular}{l}
|{\edef\jobname{\scantokens\expandafter{\jobname\noexpand}}|\\
|\def\redirectjob |\textit{prefix}|#1~~~{\gdef\jobname{|\textit{dest}|#1}}|\\
|\expandafter\redirectjob\jobname~~~}\input{\jobname}|
\end{tabular}
\end{center}

In an alternative approach,
child documents can be compiled by a specific command line
without additional code or specific definitions:
%
\begin{center}
|... -jobname "|\textit{target}|" "|[\textit{flags}]%
|\includeonly{|\textit{dest}|}\input{|\textit{main}|}"|
\end{center}
%

%%%%%%%%%%%%%%%%%%%%%%%%%%%%%%%%%%%%%%%%%%%%%%%%%%%%%%%%%%%%%%%%%%%%%%%%%%%%%%%%
%%%%%%%%%%%%%%%%%%%%%%%%%%%%%%%%%%%%%%%%%%%%%%%%%%%%%%%%%%%%%%%%%%%%%%%%%%%%%%%%
\section{Information}

%%%%%%%%%%%%%%%%%%%%%%%%%%%%%%%%%%%%%%%%%%%%%%%%%%%%%%%%%%%%%%%%%%%%%%%%%%%%%%%%
\subsection{Copyright}

Copyright \copyright{} 2017--2018 Niklas Beisert

This work may be distributed and/or modified under the
conditions of the \LaTeX{} Project Public License, either version 1.3
of this license or (at your option) any later version.
The latest version of this license is in
  \url{http://www.latex-project.org/lppl.txt}
and version 1.3 or later is part of all distributions of \LaTeX{}
version 2005/12/01 or later.

This work has the LPPL maintenance status `maintained'.

The Current Maintainer of this work is Niklas Beisert.

This work consists of the files |README.txt|, |childdoc.ins| and |childdoc.dtx|
as well as the derived files |childdoc.def|, |cdocsamp.tex|
with |cdocsch1.tex|, |cdocsch2.tex|, |cdocspt3.tex|, |cdocspt4.tex|,
|cdocsdrf.tex|, |cdocsfn1.tex|, |cdocsfn2.tex|
as well as |childdoc.pdf|.

%%%%%%%%%%%%%%%%%%%%%%%%%%%%%%%%%%%%%%%%%%%%%%%%%%%%%%%%%%%%%%%%%%%%%%%%%%%%%%%%
\subsection{Files and Installation}

The package consists of the files:
%
\begin{center}
\begin{tabular}{ll}
    |README.txt|   & readme file \\
    |childdoc.ins| & installation file \\
    |childdoc.dtx| & source file \\
    |childdoc.def| & definition file \\
    |cdocsamp.tex| & sample main file \\
    |cdocsch1.tex| & sample include file \\
    |cdocsch2.tex| & sample include file \\
    |cdocspt3.tex| & sample part file \\
    |cdocspt4.tex| & sample part file \\
    |cdocsdrf.tex| & sample redirection file \\
    |cdocsfn1.tex| & sample redirection file \\
    |cdocsfn2.tex| & sample redirection file \\
    |childdoc.pdf| & manual
\end{tabular}
\end{center}
%
The distribution consists of the files
|README.txt|, |childdoc.ins| and |childdoc.dtx|.
%
\begin{itemize}
\item
Run (pdf)\LaTeX{} on |childdoc.dtx|
to compile the manual |childdoc.pdf| (this file).
\item
Run \LaTeX{} on |childdoc.ins| to create the definitions file |childdoc.def|
and the sample |cdocsamp.tex| with include files
|cdocsch1.tex|, |cdocsch2.tex|, |cdocspt3.tex|, |cdocspt4.tex|,
|cdocsdrf.tex|, |cdocsfn1.tex|, |cdocsfn2.tex|.
Then copy the file |childdoc.def| to an appropriate directory of your \LaTeX{}
distribution, e.g.\ \textit{texmf-root}|/tex/latex/childdoc|.
\end{itemize}

%%%%%%%%%%%%%%%%%%%%%%%%%%%%%%%%%%%%%%%%%%%%%%%%%%%%%%%%%%%%%%%%%%%%%%%%%%%%%%%%
\subsection{Related CTAN Packages}

There are several other packages which offer a similar functionality:
%
\begin{itemize}
\item
The packages
\href{http://ctan.org/pkg/docmute}{\textsf{docmute}},
\href{http://ctan.org/pkg/includex}{\textsf{includex}} and
\href{http://ctan.org/pkg/standalone}{\textsf{standalone}}
provide commands to include only the document body of
a child file thus allowing both files to be compiled individually.
\item
The packages \href{http://ctan.org/pkg/subdocs}{\textsf{subdocs}}
and \href{http://ctan.org/pkg/subfiles}{\textsf{subfiles}}
provide structures in which the main and child documents can be
encapsulated and allowing them to be compiled individually.
The inclusion mechanism is different from the conventional |\include|.
\item
The package \href{http://ctan.org/pkg/combine}{\textsf{combine}}
is an elaborate solution to combine several documents into one.
\end{itemize}
%
See also the CTAN topic \href{http://ctan.org/topic/subdocs}{\textsf{subdocs}}
for further related packages.
The present package differs from the above solutions in that
a document structure constructed with the conventional |\include| mechanism
just needs two extra commands at the top of every file
such that all constituent files can be compiled individually.

%%%%%%%%%%%%%%%%%%%%%%%%%%%%%%%%%%%%%%%%%%%%%%%%%%%%%%%%%%%%%%%%%%%%%%%%%%%%%%%%
%\subsection{Feature Suggestions}
%
%The following is a list of features which may be useful for future
%versions of this package:
%%
%\begin{itemize}
%\item
%\ldots
%\end{itemize}

%%%%%%%%%%%%%%%%%%%%%%%%%%%%%%%%%%%%%%%%%%%%%%%%%%%%%%%%%%%%%%%%%%%%%%%%%%%%%%%%
\subsection{Revision History}

%%%%%%%%%%%%%%%%%%%%%%%%%%%%%%%%%%%%%%%%
\paragraph{v2.0:} 2018/12/30

\begin{itemize}
\item
immediate forward processing
\item
added |\childdocby| mechanism
\item
manual restructured
\end{itemize}

%%%%%%%%%%%%%%%%%%%%%%%%%%%%%%%%%%%%%%%%
\paragraph{v1.6:} 2018/01/17

\begin{itemize}
\item
application for development of include files
\item
corrections to manual
\end{itemize}

%%%%%%%%%%%%%%%%%%%%%%%%%%%%%%%%%%%%%%%%
\paragraph{v1.5:} 2017/05/21

\begin{itemize}
\item
more complete structuring introduced
\item
|\childdocof| introduced
\item
|\childdoc| renamed to |\childdocmain|
\item
|\childredirect| renamed to |\childdocforward| and |\childdocforwardprefix|
and functionality expanded
\end{itemize}

%%%%%%%%%%%%%%%%%%%%%%%%%%%%%%%%%%%%%%%%
\paragraph{v1.0:} 2017/04/27

\begin{itemize}
\item
manual and install package
\item
first version published on CTAN
\end{itemize}

%%%%%%%%%%%%%%%%%%%%%%%%%%%%%%%%%%%%%%%%
\paragraph{v0.6:} 2017/04/26

\begin{itemize}
\item
redirection mechanism added
\end{itemize}

%%%%%%%%%%%%%%%%%%%%%%%%%%%%%%%%%%%%%%%%
\paragraph{v0.5:} 2017/04/26

\begin{itemize}
\item
functionality in definition file
\end{itemize}


%%%%%%%%%%%%%%%%%%%%%%%%%%%%%%%%%%%%%%%%%%%%%%%%%%%%%%%%%%%%%%%%%%%%%%%%%%%%%%%%
%%%%%%%%%%%%%%%%%%%%%%%%%%%%%%%%%%%%%%%%%%%%%%%%%%%%%%%%%%%%%%%%%%%%%%%%%%%%%%%%
%%%%%%%%%%%%%%%%%%%%%%%%%%%%%%%%%%%%%%%%%%%%%%%%%%%%%%%%%%%%%%%%%%%%%%%%%%%%%%%%
\appendix

\settowidth\MacroIndent{\rmfamily\scriptsize 000\ }

 \DocInput{childdoc.dtx}

\end{document}
%</driver>
% \fi
%
% %%%%%%%%%%%%%%%%%%%%%%%%%%%%%%%%%%%%%%%%%%%%%%%%%%%%%%%%%%%%%%%%%%%%%%%%%%%%%%
% %%%%%%%%%%%%%%%%%%%%%%%%%%%%%%%%%%%%%%%%%%%%%%%%%%%%%%%%%%%%%%%%%%%%%%%%%%%%%%
% \section{Sample}
%\iffalse
%<*samplemain>
%\fi
%
% The following presents a sample document
% with two chapters, two parts, a title page,
% a compile flag as well as three forwarding files to set the flag.
% It consists of eight |.tex| files:
% \begin{center}
% \begin{tabular}{ll}
% |cdocsamp.tex|&main file\\
% |cdocsch1.tex|&include file for chapter 1\\
% |cdocsch2.tex|&include file for chapter 2\\
% |cdocspt3.tex|&include file for part 3\\
% |cdocspt4.tex|&include file for part 4\\
% |cdocsdrf.tex|&forwarding file for main file in draft mode\\
% |cdocsfi1.tex|&forwarding file for final version of chapter 1\\
% |cdocsfi2.tex|&forwarding file for final version of chapter 2\\
% \end{tabular}
% \end{center}
% Each of the eight files can be compiled directly by the \LaTeX{} compiler.
%
% %%%%%%%%%%%%%%%%%%%%%%%%%%%%%%%%%%%%%%
% \paragraph{Main File.}
%
% The main file is called |cdocsamp.tex|.
%
% Load the \textsf{childdoc} definitions and
% declare the filename for the main document:
%    \begin{macrocode}
\input{childdoc.def}
\childdocmain{}
%    \end{macrocode}

% Optional override for |\version| flag:
%    \begin{macrocode}
%%\ifchilddoc\else\providecommand{\version}{draft}\fi
%    \end{macrocode}

% Define the default values for the |\version| flag
% (|final| for the main file and |draft| for childs):
%    \begin{macrocode}
\ifchilddoc
\providecommand{\version}{draft}
\else
\providecommand{\version}{final}
\fi
%    \end{macrocode}

% Load the standard document class:
%    \begin{macrocode}
\documentclass[12pt]{article}
%    \end{macrocode}

% Start the document body:
%    \begin{macrocode}
\begin{document}
%    \end{macrocode}

% Declare a title page.
% Print title, part of document being processed and version flag:
%    \begin{macrocode}
\addtocounter{page}{-1}
\begin{center}
{\LARGE\bfseries{}childdoc example\par}
\vspace{1cm}
\ifchilddoc
\ifchilddocmanual part\else chapter\fi:
`\childdocname' of `\childdocjob'\par
\else
main document: `\childdocjob'\par
\fi
version: \version\par
\end{center}
\newpage
%    \end{macrocode}

% Manually include selected file,
% otherwise process as usual:
%    \begin{macrocode}
\ifchilddocmanual
\section*{part `\childdocname'}
\input{\childdocname}
\else
%    \end{macrocode}

% Include the two chapters:
%    \begin{macrocode}
\include{cdocsch1}
\include{cdocsch2}
%    \end{macrocode}

% Include the two parts unless only chapters should be displayed:
%    \begin{macrocode}
\ifchilddoc\else
\section{part three}
\input{cdocspt3}
\section{part four}
\input{cdocspt4}
\fi
%    \end{macrocode}

% Process as usual until here:
%    \begin{macrocode}
\fi
%    \end{macrocode}

% End of document body:
%    \begin{macrocode}
\end{document}
%    \end{macrocode}
%\iffalse
%</samplemain>
%\fi
%
% %%%%%%%%%%%%%%%%%%%%%%%%%%%%%%%%%%%%%%
% \paragraph{Chapter Include Files.}
%
% The include files are called |cdocsch1.tex| and |cdocsch2.tex|.
%
%\iffalse
%<*samplechap1|samplechap2>
%\fi

% Optional override for |\version| flag:
%    \begin{macrocode}
%%\providecommand{\version}{final}
%    \end{macrocode}

% Include the main document:
%    \begin{macrocode}
\input{childdoc.def}
\childdocof{cdocsamp}
%    \end{macrocode}

%\iffalse
%</samplechap1|samplechap2>
%\fi
%
%\iffalse
%<*samplechap1>
%\fi
% Some text for chapter 1:
%    \begin{macrocode}
\section{one}
some text in chapter one
%    \end{macrocode}

%\iffalse
%</samplechap1>
%\fi
% Some text for chapter 2:
%\iffalse
%<*samplechap2>
%\fi
%    \begin{macrocode}
\section{two}
more text in chapter two
%    \end{macrocode}

%\iffalse
%</samplechap2>
%\fi
%
% %%%%%%%%%%%%%%%%%%%%%%%%%%%%%%%%%%%%%%
% \paragraph{Part Include Files.}
%
% The include files are called |cdocspt3.tex| and |cdocspt4.tex|.
%
%\iffalse
%<*samplepart3|samplepart4>
%\fi

% Optional override for |\version| flag:
%    \begin{macrocode}
%%\providecommand{\version}{final}
%    \end{macrocode}

% Include the main document:
%    \begin{macrocode}
\input{childdoc.def}
\childdocby{cdocsamp}
%    \end{macrocode}

%\iffalse
%</samplepart3|samplepart4>
%\fi
%
%\iffalse
%<*samplepart3>
%\fi
% Some text for part 3:
%    \begin{macrocode}
some text in part three
%    \end{macrocode}

%\iffalse
%</samplepart3>
%\fi
% Some text for part 4:
%\iffalse
%<*samplepart4>
%\fi
%    \begin{macrocode}
more text in part four
%    \end{macrocode}

%\iffalse
%</samplepart4>
%\fi
%
% %%%%%%%%%%%%%%%%%%%%%%%%%%%%%%%%%%%%%%
% \paragraph{Forwarding for a Complete Draft.}
%
% The following forwarding file |cdocsdrf.tex|
% compiles the main document in draft mode:
%\iffalse
%<*sampledraft>
%\fi
%    \begin{macrocode}
\def\version{draft}
\input{childdoc.def}
\childdocforward{cdocsamp}
%    \end{macrocode}

%\iffalse
%</sampledraft>
%\fi
%
% %%%%%%%%%%%%%%%%%%%%%%%%%%%%%%%%%%%%%%
% \paragraph{Forwarding for Final Version of the Chapters.}
%
% The following forwarding files |cdocsfn1.tex| and |cdocsfn2.tex|
% (with identical content)
% compile the final versions of the child documents
% |cdocsch1.tex| and |cdocsch2.tex|, respectively:
%\iffalse
%<*samplefinal>
%\fi
%    \begin{macrocode}
\def\version{final}
\input{childdoc.def}
\childdocforwardprefix[cdocsamp]{cdocsfn}{cdocsch}
%    \end{macrocode}

%\iffalse
%</samplefinal>
%\fi
%
% %%%%%%%%%%%%%%%%%%%%%%%%%%%%%%%%%%%%%%
% \paragraph{Command Line Processing.}
%
% The following three command lines generate the output files
% |cdocscld|, |cdocscl1| and |cdocscl2|
% which should be identical to
% |cdocsdrf|, |cdocsch1| and |cdocsfn2|, respectively:
% \begin{center}
% \begin{tabular}{l}
% |latex -jobname cdocscld \|\\
% |  "\def\version{draft}\input{childdoc.def}\childdocforward{cdocsamp}"|\\
% |latex -jobname cdocscl1 \|\\
% |  "\input{childdoc.def}\childdocforward[cdocsamp]{cdocsch1}"|\\
% |latex -jobname cdocscl2 \|\\
% |  "\def\version{final}\input{childdoc.def}\childdocforward{cdocsch2}"|
% \end{tabular}
% \end{center}
% Note that the trailing backslash on each first line
% merely continues the input to the second line
% (for convenient cut ant paste).
% Furthermore, the command |latex| can be replaced by any
% of its alternative versions such as |pdflatex|.
%
% %%%%%%%%%%%%%%%%%%%%%%%%%%%%%%%%%%%%%%%%%%%%%%%%%%%%%%%%%%%%%%%%%%%%%%%%%%%%%%
% %%%%%%%%%%%%%%%%%%%%%%%%%%%%%%%%%%%%%%%%%%%%%%%%%%%%%%%%%%%%%%%%%%%%%%%%%%%%%%
% \section{Implementation}
%\iffalse
%<*package>
%\fi
%
% This section describes the definitions file |childdoc.def|.

% The definitions cannot be loaded using |\usepackage| or |\RequirePackage|
% which has a mechanism to prevent loading a style file more than once.
% When loading the definitions by means of |\input|
% multiple instances have to be prevented manually:
%\iffalse
%This code needs to be before the `\ProvidesFile' directive
%which is defined at the beginning of this file.
%Therefore it is also placed there and commented out here.
%</package>
%<*discard>
%\fi
%    \begin{macrocode}
\ifdefined\childdocmain\endinput\fi
%    \end{macrocode}
%\iffalse
%</discard>
%<*package>
%\fi
%
% \macro{\ifchilddoc}
% \macro{\ifchilddocmanual}
% The conditional |\ifchilddoc| tells whether a
% child (true) or main (false) document is being compiled.
% The conditional |\ifchilddocmanual| tells whether
% the |\includeonly| mechanism is used (false) or
% the selection of child files must be performed manually (true).
% The definitions initialise to false:
%    \begin{macrocode}
\newif\ifchilddoc
\newif\ifchilddocmanual
%    \end{macrocode}

% \macro{\childdocname}
% \macro{\childdocjob}
% The macro |\childdocname| stores the name of the main document
% to be compiled. The macro |\childdocjob| stores the name of
% the document on which the \LaTeX{} compiler was originally invoked.
% The content of |\jobname| cannot be compared
% to filenames specified in the source due to different catcodes.
% The following code rescans |\jobname|, stores the result
% in |\childdocname| and saves a copy in |\childdocjob|:
%    \begin{macrocode}
\edef\childdocname{\scantokens\expandafter{\jobname\noexpand}}
\let\childdocjob\childdocname
%    \end{macrocode}

% \macro{\childdocdisable}
% The macro |\childdocdisable| prevents the main file
% from being processed more than once.
% At this stage, the main document command |\childdocmain|
% is assumed to be called once again where it should do nothing.
% Any subsequent call to it should prevent
% a secondary processing of the main document
% It overwrites the forwarding commands
% |\childdocof| and |\childdocforward|
% with empty macros to prevent further inclusions of the main document:
%    \begin{macrocode}
\newcommand{\childdocdisable}
{
  \renewcommand{\childdocmain}[1]{\renewcommand{\childdocmain}[1]{\endinput}}
  \renewcommand{\childdocof}[1]{}
  \renewcommand{\childdocby}[2][]{}
  \renewcommand{\childdocforward}[2][]{}
  \renewcommand{\childdocdisable}{}
}
%    \end{macrocode}

% \macro{\childdocmain}
% The macro |\childdocmain| is to be called at the top of the main file
% with nothing or the main filename (without extension) as argument.
% First, it breaks loops.
% If the argument is not empty and does not match |\childdocname|
% (which is set by the first inclusion of |childdoc.def|),
% |\ifchilddoc| is set to true, |\includeonly| is applied to the child file
% and |\jobname| is set to the main file
% (for proper handling of |.aux| files):
%    \begin{macrocode}
\newcommand{\childdocmain}[1]
{
  \childdocdisable\childdocmain{}
  \if?#1?\else
    \begingroup
      \def\childdoctmp{#1}
      \ifx\childdoctmp\childdocname
        \def\childdoctmp{}
      \else
        \def\childdoctmp
        {
          \childdoctrue
          \includeonly{\childdocname}
          \def\childdocjob{#1}
          \def\jobname{#1}
        }
      \fi
      \expandafter
    \endgroup
    \childdoctmp
  \fi
}
%    \end{macrocode}

% \macro{\childdocof}
% The command |\childdocof| redirects
% compilation to the main file |#1|.
%    \begin{macrocode}
\newcommand{\childdocof}[1]
{
  \childdocdisable
  \childdoctrue
  \includeonly{\childdocname}
  \def\jobname{#1}
  \def\childdocjob{#1}
  \input{#1}
}
%    \end{macrocode}

% \macro{\childdocby}
% The command |\childdocby| ....
%    \begin{macrocode}
\newcommand{\childdocby}[2][]
{
  \childdocdisable
  \childdoctrue
  \childdocmanualtrue
  \if?#1?\else
    \def\jobname{#2}
  \fi
  \def\childdocjob{#2}
  \input{#2}
  \endinput
}
%    \end{macrocode}

% \macro{\childdocforward}
% The command |\childdocforward| redirects
% compilation to the main file or
% (if the optional argument is given) a child file.
% Parameters are set as if the main file
% or a child file starting with |\childdocof| was compiled.
% Then compilation is handed over to the main file:
%    \begin{macrocode}
\newcommand{\childdocforward}[2][]
{
  \begingroup
    \if?#1?
      \def\childdoctmp
      {
        \def\childdocname{#2}
        \def\childdocjob{#2}
        \def\jobname{#2}
        \input{#2}
        \endinput
      }
    \else
      \def\childdoctmp
      {
        \childdocdisable
        \def\childdocname{#2}
        \childdoctrue
        \includeonly{#2}
        \def\childdocjob{#1}
        \def\jobname{#1}
        \input{#1}
        \endinput
      }
    \fi
    \expandafter
  \endgroup
  \childdoctmp
}
%    \end{macrocode}

% \macro{\childdocforwardprefix}
% The command |\childdocforwardprefix| redirects
% compilation to the main or a child file by means of a pattern.
% The prefix |#1| in the current filename is replaced by |#2|
% and the suffix of the current filename is kept
% (it is assumed that the filename does not contain the substring `|~~~|'
% which is used as a delimiter).
% Compilation is handed over to the new file by |\childdocforward|:
%    \begin{macrocode}
\newcommand{\childdocforwardprefix}[3][]
{
  \begingroup
    \def\childdocextract #2##1~~~{\def\childdoctmp{\childdocforward[#1]{#3##1}}}
    \expandafter\childdocextract\childdocname~~~
    \expandafter
  \endgroup
  \childdoctmp
}
%    \end{macrocode}

% \macro{\childdoc}
% The deprecated macro |\childdoc| is a legacy version of |\childdocmain|:
%    \begin{macrocode}
\newcommand{\childdoc}{\childdocmain}
%    \end{macrocode}

% \macro{\childdocredirect}
% The deprecated macro |\childdocredirect| is a legacy version
% of |\childdocforward| and |\childdocforwardprefix|:
%    \begin{macrocode}
\newcommand{\childdocredirect}[2][]
{
  \begingroup
    \if?#1?
      \def\childdoctmp{\childdocforward{#2}}
    \else
      \def\childdoctmp{\childdocforwardprefix{#1}{#2}}
    \fi
    \expandafter
  \endgroup
  \childdoctmp
}
%    \end{macrocode}

%\iffalse
%</package>
%\fi
%
\endinput

\childdocforwardprefix[cdocsamp]{cdocsfn}{cdocsch}
%    \end{macrocode}

%\iffalse
%</samplefinal>
%\fi
%
% %%%%%%%%%%%%%%%%%%%%%%%%%%%%%%%%%%%%%%
% \paragraph{Command Line Processing.}
%
% The following three command lines generate the output files
% |cdocscld|, |cdocscl1| and |cdocscl2|
% which should be identical to
% |cdocsdrf|, |cdocsch1| and |cdocsfn2|, respectively:
% \begin{center}
% \begin{tabular}{l}
% |latex -jobname cdocscld \|\\
% |  "\def\version{draft}% \iffalse
%
% childdoc.dtx Copyright (C) 2017-2018 Niklas Beisert
%
% This work may be distributed and/or modified under the
% conditions of the LaTeX Project Public License, either version 1.3
% of this license or (at your option) any later version.
% The latest version of this license is in
%   http://www.latex-project.org/lppl.txt
% and version 1.3 or later is part of all distributions of LaTeX
% version 2005/12/01 or later.
%
% This work has the LPPL maintenance status `maintained'.
%
% The Current Maintainer of this work is Niklas Beisert.
%
% This work consists of the files childdoc.dtx and childdoc.ins
% and the derived files childdoc.def and cdocsamp.tex with
% cdocsch1.tex, cdocsch2.tex, cdocsdrf.tex, cdocsfn1.tex, cdocsfn2.tex.
%
%<package>\ifdefined\childdocmain\endinput\fi
%<package>\ProvidesFile{childdoc.def}[2018/12/30 v2.0 child document driver]
%<samplemain>\ProvidesFile{cdocsamp.tex}[2018/12/30 v2.0 sample for childdoc]
%<*driver>
%\ProvidesFile{childdoc.drv}[2018/12/30 v2.0 childdoc reference manual file]
\PassOptionsToClass{10pt,a4paper}{article}
\documentclass{ltxdoc}

\usepackage[margin=35mm]{geometry}
\usepackage{hyperref}
\usepackage{hyperxmp}
\usepackage[usenames]{color}

\hypersetup{colorlinks=true}
\hypersetup{pdfstartview=FitH}
\hypersetup{pdfpagemode=UseNone}
\hypersetup{pdfsource={}}
\hypersetup{pdflang={en-UK}}
\hypersetup{pdfcopyright={Copyright 2017-2018 Niklas Beisert.
  This work may be distributed and/or modified under the
  conditions of the LaTeX Project Public License, either version 1.3
  of this license or (at your option) any later version.}}
\hypersetup{pdflicenseurl={http://www.latex-project.org/lppl.txt}}
\hypersetup{pdfcontactaddress={ETH Zurich, ITP, HIT K,
  Wolfgang-Pauli-Strasse 27}}
\hypersetup{pdfcontactpostcode={8093}}
\hypersetup{pdfcontactcity={Zurich}}
\hypersetup{pdfcontactcountry={Switzerland}}
\hypersetup{pdfcontactemail={nbeisert@itp.phys.ethz.ch}}
\hypersetup{pdfcontacturl={http://people.phys.ethz.ch/\xmptilde nbeisert/}}

\newcommand{\secref}[1]{\hyperref[#1]{section \ref*{#1}}}

\parskip1ex
\parindent0pt
\let\olditemize\itemize
\def\itemize{\olditemize\parskip0pt}

\begin{document}

\title{The \textsf{childdoc} Package}
\hypersetup{pdftitle={The childdoc Package}}
\author{Niklas Beisert\\[2ex]
  Institut f\"ur Theoretische Physik\\
  Eidgen\"ossische Technische Hochschule Z\"urich\\
  Wolfgang-Pauli-Strasse 27, 8093 Z\"urich, Switzerland\\[1ex]
  \href{mailto:nbeisert@itp.phys.ethz.ch}
  {\texttt{nbeisert@itp.phys.ethz.ch}}}
\hypersetup{pdfauthor={Niklas Beisert}}
\hypersetup{pdfsubject={Manual for the LaTeX2e Package childdoc}}
\date{30 December 2018, \textsf{v2.0}}
\maketitle

\begin{abstract}\noindent
\textsf{childdoc} is a \LaTeXe{} package
that enables the direct compilation
of document sections included by |\include|
to individual files.
\end{abstract}

\begingroup
\parskip0ex
\tableofcontents
\endgroup

%%%%%%%%%%%%%%%%%%%%%%%%%%%%%%%%%%%%%%%%%%%%%%%%%%%%%%%%%%%%%%%%%%%%%%%%%%%%%%%%
%%%%%%%%%%%%%%%%%%%%%%%%%%%%%%%%%%%%%%%%%%%%%%%%%%%%%%%%%%%%%%%%%%%%%%%%%%%%%%%%
\section{Introduction}

\LaTeX{} provides a mechanism to structure a large document (such as a book)
into a main file and several child files (containing the chapters)
using the |\include| command.
This mechanism is beneficial for documents
which span hundreds of pages in order to
make the source file(s) more manageable.
Moreover, compilation can be restricted to
selected child files by means of the |\includeonly| command.
The latter feature can be used to reduce the compilation time while editing
(this was significantly more useful in the earlier days of \LaTeX{})
or to generate a smaller document which is easier to navigate.
Another application of |\includeonly| is to generate
documents consisting of selected parts of the complete document.

However, there are a few drawbacks of the plain |\include| mechanism:
\begin{itemize}
\item
The child files cannot be compiled on their own,
they can only be compiled via the main file.
A naive editing environment
(such as a text editor with an option
to have the current file processed by \LaTeX)
may require one to switch to the main file before compiling;
attempting to compile the child file produces errors.
\item
The main file must be modified (each time)
to adjust the |\includeonly| command
to the present needs. This easily leaves the main file in a messy state.
\item
The generated document will always carry the filename
of the main document. This is inconvenient if
several child files are to be compiled and
to be kept for distribution.
\end{itemize}

The present package provides a simple interface
to make child files individually compilable by \LaTeX{}.
Compiling a child file then has the same effect as compiling
the main file with an |\includeonly| command
to select the appropriate child.
Moreover the generated document will carry the name of the child
rather than the main file.
This resolves all three above issues.

This feature is meant to make the editing of books,
thesis documents and lecture notes somewhat more convenient.
However, the package can also be used efficiently for
composing a series of documents (such as exercise sheets)
which are typically distributed individually.
It then assists the author in generating the individual documents
(potentially in different versions)
as well as a document containing the collected series.
Another application is in developing style files
or other kinds of included material
where compilation of the style file could redirect
to a sample or test file.

%%%%%%%%%%%%%%%%%%%%%%%%%%%%%%%%%%%%%%%%%%%%%%%%%%%%%%%%%%%%%%%%%%%%%%%%%%%%%%%%
%%%%%%%%%%%%%%%%%%%%%%%%%%%%%%%%%%%%%%%%%%%%%%%%%%%%%%%%%%%%%%%%%%%%%%%%%%%%%%%%
\section{Usage}

First of all, the package \textsf{childdoc} is \emph{not} a standard
\LaTeXe{} |.sty| style file! Therefore it needs to be invoked in
a non-standard way.

%%%%%%%%%%%%%%%%%%%%%%%%%%%%%%%%%%%%%%%%%%%%%%%%%%%%%%%%%%%%%%%%%%%%%%%%%%%%%%%%
\subsection{Included Files}
\label{sec:include}

%%%%%%%%%%%%%%%%%%%%%%%%%%%%%%%%%%%%%%%%
\DescribeMacro{\childdocmain}
To use the package, add the commands
\begin{center}
\begin{tabular}{l}
|\input{childdoc.def}|\\
|\childdocmain{}|\\
\end{tabular}
\end{center}
at the very top of the main \LaTeX{} file,
in particular \emph{before} the |\documentclass| statement!
The argument of |\childdocmain| should be left empty
(but it must be present).

%%%%%%%%%%%%%%%%%%%%%%%%%%%%%%%%%%%%%%%%
\DescribeMacro{\childdocof}
Furthermore, add the commands
\begin{center}
\begin{tabular}{l}
|\input{childdoc.def}|\\
|\childdocof{|\textit{main}|}|\\
\end{tabular}
\end{center}
at the top of every child file \textit{child}
which is included by |\include{|\textit{child}|}|
from within the main file
(or at least for those files to be compiled individually).
The argument \textit{main} must be the filename of the main file.

There are a couple of
considerations in setting up the main and child documents:

%%%%%%%%%%%%%%%%%%%%%%%%%%%%%%%%%%%%%%%%
\paragraph{Restrictions.}

Please note the following restrictions:
\begin{itemize}
\item
|\childdocmain| must be called with one argument \textit{main}
to ensure compatibility with earlier version of the package.
It must either be empty (|\childdocmain{}|)
or precisely match the filename of the main file in which it is specified.
See \secref{sec:detection} for further information.
\item
The filename \textit{main} must be specified without the |.tex| extension.
\item
The filename \textit{main} is case sensitive
(even in case-insensitive file systems)
due to internal string comparison.
\item
The argument \textit{main} should be fully expanded, it cannot be a macro.
\item
Subdirectories and special characters should be avoided in filenames.
\item
The command |\childdocmain{|\textit{main}|}| must be followed by a whitespace.
It should not be followed immediately by another command
or by a comment mark `|%|'.
This is because the \TeX{} parser reads the token immediately following
the argument of |\childdocmain| and puts it
at the beginning of every child section;
however, a white\-space is ignored.
\end{itemize}

%%%%%%%%%%%%%%%%%%%%%%%%%%%%%%%%%%%%%%%%
\paragraph{Content of Main File.}

It is advisable to place all content in the child files included by |\include|.
Any output contained in the main file will appear in all child documents
unless suppressed manually;
it cannot be suppressed automatically by the |\includeonly| directive
and thus should normally be avoided.
A method to include some content in the main file
by means of conditional processing is described in \secref{sec:conditional}.

%%%%%%%%%%%%%%%%%%%%%%%%%%%%%%%%%%%%%%%%
\paragraph{Page Numbering.}

When only a part of the document is compiled,
the appropriate numbering of pages
(as well as other status parameters)
is determined from the |.aux| files.
The latter contain information from previous passes.
However this information needs to propagate through
all intermediate child documents.
Therefore the page numbering in child documents may well
be inconsistent until the complete document is compiled at least once.

A useful (if unconventional) way to always ensure a consistent
page numbering is to restart the numbering in each child document
and denote the pages by `\textit{child}|.|\textit{page}'
where \textit{child} represents the chapter/section number of the child file.
This can be achieved by the command
|\numberwithin{page}{|\textit{child}|}|
of the \textsf{amsmath} package
where \textit{child} can be |chapter| or |section|
depending on the chosen structuring.
Alternatively, one can modify the macro |\thepage| appropriately
and reset the counter |page| at the start of each child file.

%%%%%%%%%%%%%%%%%%%%%%%%%%%%%%%%%%%%%%%%%%%%%%%%%%%%%%%%%%%%%%%%%%%%%%%%%%%%%%%%
\subsection{Conditional Processing}
\label{sec:conditional}

The package provides a mechanism to compile different versions
of a document. To customise the versions further some conditional processing
can come in handy to distinguish which version is being compiled.
The package provides two macros to describe the compilation context:

%%%%%%%%%%%%%%%%%%%%%%%%%%%%%%%%%%%%%%%%
\DescribeMacro{\ifchilddoc}
The conditional |\ifchilddoc| distinguishes between the compilation of
child documents and the main document:
%
\begin{center}
|\ifchilddoc |\textit{child-code}| |[|\||else |\textit{main-code}]| \||fi|
\end{center}

%%%%%%%%%%%%%%%%%%%%%%%%%%%%%%%%%%%%%%%%
\DescribeMacro{\childdocname}
\DescribeMacro{\childdocjob}
The macro |\childdocname| contains the filename (without extension)
of the main or child file being processed.
Note that |\childdocjob| will always contain the name of the main file.

%%%%%%%%%%%%%%%%%%%%%%%%%%%%%%%%%%%%%%%%
\paragraph{Title Page.}

Conditional processing can be used to include a title or banner page
in the main document when proper precautions are taken.
Importantly, the code in the main file should ensure that the page counter
(as well as other status parameters which are stored in the |.aux| files)
takes the same value after the conditional processing.
Otherwise the page numbers may take divergent values
depending on which part is compiled.

For example, a title page could be declared by:
%
\begin{center}
\begin{tabular}{l}
|\ifchilddoc\||else|\\
|\addtocounter{page}{-1}|\\
\textit{code for title page}\\
|\newpage|\\
|\||fi|
\end{tabular}
\end{center}
%
A banner page for the child documents can be generated by:
%
\begin{center}
\begin{tabular}{l}
|\ifchilddoc|\\
|\addtocounter{page}{-1}|\\
\textit{code for banner page}\\
|\newpage|\\
|\||fi|
\end{tabular}
\end{center}
%
Here one could write a message such as:
\begin{center}
|This is the part \childdocname{} of \childdocjob{}.|
\end{center}

%%%%%%%%%%%%%%%%%%%%%%%%%%%%%%%%%%%%%%%%%%%%%%%%%%%%%%%%%%%%%%%%%%%%%%%%%%%%%%%%
\subsection{Flags}
\label{sec:flags}

The package makes it easy to generate different versions
of the main or child documents.
To this end compilation flags can be defined
and assigned different default values.
They will be particularly useful in conjunction
with the forwarding mechanism described in \secref{sec:forward}.

For example, it may be useful to have a flag |\version|
which can be set to |draft| or |final|.
The document source will contain some conditional code
depending on the value of |\version|.
Suppose further, the flag should default to |final| for the main file
and to |draft| for child files
which is a natural assignment for editing the document.
This is achieved by placing the following code
in the preamble of the main document
(below the |\childdocmain| directive):
%
\begin{center}
\begin{tabular}{l}
|\ifchilddoc|\\
|\providecommand{\version}{draft}|\\
|\||else|\\
|\providecommand{\version}{final}|\\
|\||fi|
\end{tabular}
\end{center}
%
The definition by |\providecommand| makes sure
that previous definitions are not overwritten.
Further statements |\providecommand{\version}{...}|
can thus be added before the above code to override it.

For the main file, one might add a line
(between |\childdocmain| and the above block)
%
\begin{center}
|%\ifchilddoc\||else\providecommand{\version}{draft}\||fi|
\end{center}
%
which can be uncommented to produce a draft version.
Likewise one can add a line to the very top of a child file
(above the |\childdocof{|\textit{main}|}| directive)
%
\begin{center}
|%\providecommand{\version}{final}|
\end{center}
%
which can be uncommented to produce the final version of this child document.

%%%%%%%%%%%%%%%%%%%%%%%%%%%%%%%%%%%%%%%%%%%%%%%%%%%%%%%%%%%%%%%%%%%%%%%%%%%%%%%%
\subsection{Forwarding}
\label{sec:forward}

Different versions of the main or child documents
using compilation flags as described in \secref{sec:flags}
can be (permanently) stored in different files
for convenient compilation, viewing and distribution.
To this end, the package defines a command
to pass on compilation to a different file:

%%%%%%%%%%%%%%%%%%%%%%%%%%%%%%%%%%%%%%%%
\DescribeMacro{\childdocforward}
The command |\childdocforward| redirects processing to
another source file:
%
\begin{center}
\begin{tabular}{l}
|\input{childdoc.def}|\\
|\childdocforward[|\textit{main}|]{|\textit{dest}|}|\\
\end{tabular}
\end{center}
%
The argument \textit{dest} is the destination file
(without extension).
It should be the main file or one of the child files.
Note that further \textsf{childdoc} directives
such as |\childdocof| and |\childdocforward|
in the indicated file will be processed in this form.
The optional argument \textit{main}
passes on directly to the main file \textit{main}
while pretending to compile the child \textit{dest}.
This form behaves as if \textit{dest}
issues |\childdocof{|\textit{main}|}| right away,
and no further \textsf{childdoc} directives will be processed.

%%%%%%%%%%%%%%%%%%%%%%%%%%%%%%%%%%%%%%%%
\DescribeMacro{\...prefix}
In the alternative form |\childdocforwardprefix|,
%
\begin{center}
\begin{tabular}{l}
|\input{childdoc.def}|\\
|\childdocforwardprefix[|\textit{main}|]{|\textit{prefix}|}{|\textit{dest}|}|
\end{tabular}
\end{center}
%
the destination file is determined by a pattern
depending on the current file:
To make this work, the current file must be called
`{\textit{prefix}\hspace{0.2em}\textit{suffix}}'
with \textit{prefix} matching precisely the argument.
Processing is then passed on to the file
`{\textit{dest}\hspace{0.2em}\textit{suffix}}'.
Surely, the same effect is achieved by
directly specifying the
argument `{\textit{dest}\hspace{0.2em}\textit{suffix}}'
in the first form.
However, that requires to set up a different file
for each child. With the alternative form of the command
all these files can have exactly the same content
which simplifies setting them up and maintaining them.

For example, the following file |draft.tex|
with a compilation flag |\version| as described in \secref{sec:flags}
compiles the main document as a draft:
%
\begin{center}
\begin{tabular}{l}
|\def\version{draft}|\\
|\input{childdoc.def}|\\
|\childdocforward{|\textit{main}|}|
\end{tabular}
\end{center}
%
Likewise, the following files |final|\textit{nn}|.tex|
compile the final version of the child document
|child|\textit{nn}|.tex|:
%
\begin{center}
\begin{tabular}{l}
|\def\version{final}|\\
|\input{childdoc.def}|\\
|\childdocforwardprefix{final}{child}|
\end{tabular}
\end{center}
%

Note that when several versions of a main file and/or of each child file
are to be generated, it may be convenient to set up a |Makefile| or
shell script to automatise the process.

%%%%%%%%%%%%%%%%%%%%%%%%%%%%%%%%%%%%%%%%%%%%%%%%%%%%%%%%%%%%%%%%%%%%%%%%%%%%%%%%
\subsection{Command Line Processing}
\label{sec:commandline}

The effect of redirection files can also be achieved by invoking
the \LaTeX{} compiler with a more elaborate command line.
Most conveniently this should be done as part
of a shell script or a |Makefile|.

When using \textsf{childdoc} in the main file, the following
command lines effectively perform a redirection
(note that depending on the shell being used,
backslashes may have to be doubled: `|\|' $\to$ `|\\|'):
%
\begin{center}
|... -jobname "|\textit{target}|" |\\|"|[\textit{flags}]%
|\input{childdoc.def}\childdocforward[|\textit{main}|]{|\textit{dest}|}"|
\end{center}
%
Here \textit{target} is the name of the output file,
\textit{main} is the name of the main file
and \textit{dest} is the name of the main or child file to be processed
(all filenames without extensions).
The optional argument \textit{main} can be omitted
if \textit{main} matches \textit{dest}.
Optionally, compilation \textit{flags} can be defined via |\def| commands.
This command line makes the \TeX{} engine believe
it is compiling the file \textit{target}
whose content is specified as the latter parameter.
The provided code then forwards the processing to
\textit{main} or \textit{dest} as described in \secref{sec:forward}.

%%%%%%%%%%%%%%%%%%%%%%%%%%%%%%%%%%%%%%%%%%%%%%%%%%%%%%%%%%%%%%%%%%%%%%%%%%%%%%%%
\subsection{Include by Input}
\label{sec:input}

Including child documents by |\include| has some restrictions by design.
Most notably, the content of a child document always occupies
its own set of pages; pages cannot be shared between child documents.
Usually, this behaviour makes perfect sense
because each child document contain an essential part of the document.
However, in some situations it may be desirable to compose
a document from a collection of parts
without having mandatory page breaks between then.
For this case, the package
provides a mechanism to include parts
by |\input| which can also be processed individually.
However, by construction this mechanism
requires manual handling of the content to be output.

%%%%%%%%%%%%%%%%%%%%%%%%%%%%%%%%%%%%%%%%
\DescribeMacro{\ifchilddocmanual}
The main file should be prepared as usual, see \secref{sec:include}.
However, the document body must make a distinction
between processing of an individual part and of the main document, e.g.:
%
\begin{center}
\begin{tabular}{l}
|\ifchilddocmanual|\\
|\input{\childdocname}|\\
|\||else|\\
\textit{document body with }|\input{|\textit{part}|}|\\
|\||fi|
\end{tabular}
\end{center}
%
The conditional |\ifchilddocmanual| is true whenever
a part to be included by |\input| is being compiled,
and the name of the part is stored in |\childdocname|.

%%%%%%%%%%%%%%%%%%%%%%%%%%%%%%%%%%%%%%%%
\DescribeMacro{\childdocby}
Each part to be included by |\input| should start with:
%
\begin{center}
\begin{tabular}{l}
|\input{childdoc.def}|\\
|\childdocby{|\textit{main}|}|\\
\end{tabular}
\end{center}
%
The directive |\childdocby| is similar to |\childdocof|
described in \secref{sec:include},
but the subsequent selection of content must be done manually.
To that end, both |\ifchilddoc| and |\ifchilddocmanual|
will be true upon processing of a part,
and the name of the part is stored in |\childdocname|.
Note that |\jobname| will be set to the filename of the current part
so that each part receives an individual |.aux| file
that does not interfere with the |.aux| file(s) of the main document.
This behaviour can be altered by the alternative form
|\childdocby[*]{|\textit{main}|}| (with a non-empty optional argument)
which uses the |.aux| file of the main document
by setting |\jobname| to \textit{main}.

%%%%%%%%%%%%%%%%%%%%%%%%%%%%%%%%%%%%%%%%%%%%%%%%%%%%%%%%%%%%%%%%%%%%%%%%%%%%%%%%
\subsection{Driver Development}
\label{sec:driver}

The \textsf{childdoc} mechanism can also be use for the development
of definition files such as \LaTeX{} styles or classes.
This case differs from the above setup with multiple parts
included by |\include| in that no |\includeonly| should be invoked.
This can be achieved by starting the include file
(before |\ProvidesPackage|) with:
%
\begin{center}
\begin{tabular}{l}
|\input{childdoc.def}|\\
|\childdocforward{|\textit{main}|}|\\
\end{tabular}
\end{center}
%
or alternatively with:
%
\begin{center}
\begin{tabular}{l}
|\input{childdoc.def}|\\
|\childdocby{|\textit{main}|}|\\
\end{tabular}
\end{center}
%
Both forms have slightly different effects as described above.
The main file is prepared as usual, see \secref{sec:include}.

%%%%%%%%%%%%%%%%%%%%%%%%%%%%%%%%%%%%%%%%%%%%%%%%%%%%%%%%%%%%%%%%%%%%%%%%%%%%%%%%
\subsection{Legacy Detection}
\label{sec:detection}

The directive |\childdocmain| in the main file can detect
whether the complete document or merely a child is to be compiled
even without using the directive |\childdocof|.
This method is deprecated because it is less robust
and there is no compelling reason to use it;
it is merely provided for backward compatibility
and it may be removed in future versions.

If the detection mechanism is to be used,
it is mandatory to correctly specify
the filename of the main file as the argument of |\childdocmain|:
%
\begin{center}
\begin{tabular}{l}
|\input{childdoc.def}|\\
|\childdocmain{|\textit{main}|}|\\
\end{tabular}
\end{center}
%
If |\jobname| does not match the argument \textit{main} of |\childdocmain|,
it is assumed that |\jobname| points to the child file to be compiled.
When using |\childdocmain| with the main file specified as argument,
it suffices to start a child file
with just |\input{|\textit{main}|}|
without loading of the package and using |\childdocof|.
If instead all processing is done
with the appropriate \textsf{childdoc} directives,
the argument of \textit{main} of |\childdocmain| can be empty.

An alternative version of the command line processing described
in \secref{sec:commandline} using the detection mechanism reads:
%
\begin{center}
|... -jobname "|\textit{target}|" "|[\textit{flags}]%
[|\def\jobname{|\textit{dest}|}|]|\input{|\textit{main}|}"|
\end{center}

%%%%%%%%%%%%%%%%%%%%%%%%%%%%%%%%%%%%%%%%%%%%%%%%%%%%%%%%%%%%%%%%%%%%%%%%%%%%%%%%
\subsection{Manual Code}
\label{sec:manual}

In case one cannot be certain whether the definitions file |childdoc.def|
is installed on the target \TeX{} distribution
and one prefers not to ship it,
it is conceivable to paste a few relevant commands into the sources.

To that end, drop all statements |\input{childdoc.def}|
and perform the replacements as outlined below.
Instead of |\childdocmain{|\textit{main}|}| add the following code
to the top of the main file:
%
\begin{center}
\begin{tabular}{l}
|\||ifdefined\childdocname\endinput\||fi\newif\ifchilddoc|\\
|\edef\childdocname{\scantokens\expandafter{\jobname\noexpand}}|\\
|\def\childdocmain{|\textit{main}|}\||ifx\childdocmain\childdocname\||else|\\
|\childdoctrue\includeonly{\childdocname}\let\jobname\childdocmain\||fi|\\
\end{tabular}
\end{center}
%
Instead of |\childdocof{|\textit{main}|}| just include the main file
at the top of each child file:
%
\begin{center}
|\input{|\textit{main}|}|
\end{center}
%
A simple redirection |\childdocforward{|\textit{dest}|}| is achieved by:
%
\begin{center}
|\def\jobname{|\textit{dest}|}\input{\jobname}|
\end{center}
%
The redirection with prefix
|\childdocforwardprefix[|\textit{prefix}|]{|\textit{dest}|}|
is accomplished by:
%
\begin{center}
\begin{tabular}{l}
|{\edef\jobname{\scantokens\expandafter{\jobname\noexpand}}|\\
|\def\redirectjob |\textit{prefix}|#1~~~{\gdef\jobname{|\textit{dest}|#1}}|\\
|\expandafter\redirectjob\jobname~~~}\input{\jobname}|
\end{tabular}
\end{center}

In an alternative approach,
child documents can be compiled by a specific command line
without additional code or specific definitions:
%
\begin{center}
|... -jobname "|\textit{target}|" "|[\textit{flags}]%
|\includeonly{|\textit{dest}|}\input{|\textit{main}|}"|
\end{center}
%

%%%%%%%%%%%%%%%%%%%%%%%%%%%%%%%%%%%%%%%%%%%%%%%%%%%%%%%%%%%%%%%%%%%%%%%%%%%%%%%%
%%%%%%%%%%%%%%%%%%%%%%%%%%%%%%%%%%%%%%%%%%%%%%%%%%%%%%%%%%%%%%%%%%%%%%%%%%%%%%%%
\section{Information}

%%%%%%%%%%%%%%%%%%%%%%%%%%%%%%%%%%%%%%%%%%%%%%%%%%%%%%%%%%%%%%%%%%%%%%%%%%%%%%%%
\subsection{Copyright}

Copyright \copyright{} 2017--2018 Niklas Beisert

This work may be distributed and/or modified under the
conditions of the \LaTeX{} Project Public License, either version 1.3
of this license or (at your option) any later version.
The latest version of this license is in
  \url{http://www.latex-project.org/lppl.txt}
and version 1.3 or later is part of all distributions of \LaTeX{}
version 2005/12/01 or later.

This work has the LPPL maintenance status `maintained'.

The Current Maintainer of this work is Niklas Beisert.

This work consists of the files |README.txt|, |childdoc.ins| and |childdoc.dtx|
as well as the derived files |childdoc.def|, |cdocsamp.tex|
with |cdocsch1.tex|, |cdocsch2.tex|, |cdocspt3.tex|, |cdocspt4.tex|,
|cdocsdrf.tex|, |cdocsfn1.tex|, |cdocsfn2.tex|
as well as |childdoc.pdf|.

%%%%%%%%%%%%%%%%%%%%%%%%%%%%%%%%%%%%%%%%%%%%%%%%%%%%%%%%%%%%%%%%%%%%%%%%%%%%%%%%
\subsection{Files and Installation}

The package consists of the files:
%
\begin{center}
\begin{tabular}{ll}
    |README.txt|   & readme file \\
    |childdoc.ins| & installation file \\
    |childdoc.dtx| & source file \\
    |childdoc.def| & definition file \\
    |cdocsamp.tex| & sample main file \\
    |cdocsch1.tex| & sample include file \\
    |cdocsch2.tex| & sample include file \\
    |cdocspt3.tex| & sample part file \\
    |cdocspt4.tex| & sample part file \\
    |cdocsdrf.tex| & sample redirection file \\
    |cdocsfn1.tex| & sample redirection file \\
    |cdocsfn2.tex| & sample redirection file \\
    |childdoc.pdf| & manual
\end{tabular}
\end{center}
%
The distribution consists of the files
|README.txt|, |childdoc.ins| and |childdoc.dtx|.
%
\begin{itemize}
\item
Run (pdf)\LaTeX{} on |childdoc.dtx|
to compile the manual |childdoc.pdf| (this file).
\item
Run \LaTeX{} on |childdoc.ins| to create the definitions file |childdoc.def|
and the sample |cdocsamp.tex| with include files
|cdocsch1.tex|, |cdocsch2.tex|, |cdocspt3.tex|, |cdocspt4.tex|,
|cdocsdrf.tex|, |cdocsfn1.tex|, |cdocsfn2.tex|.
Then copy the file |childdoc.def| to an appropriate directory of your \LaTeX{}
distribution, e.g.\ \textit{texmf-root}|/tex/latex/childdoc|.
\end{itemize}

%%%%%%%%%%%%%%%%%%%%%%%%%%%%%%%%%%%%%%%%%%%%%%%%%%%%%%%%%%%%%%%%%%%%%%%%%%%%%%%%
\subsection{Related CTAN Packages}

There are several other packages which offer a similar functionality:
%
\begin{itemize}
\item
The packages
\href{http://ctan.org/pkg/docmute}{\textsf{docmute}},
\href{http://ctan.org/pkg/includex}{\textsf{includex}} and
\href{http://ctan.org/pkg/standalone}{\textsf{standalone}}
provide commands to include only the document body of
a child file thus allowing both files to be compiled individually.
\item
The packages \href{http://ctan.org/pkg/subdocs}{\textsf{subdocs}}
and \href{http://ctan.org/pkg/subfiles}{\textsf{subfiles}}
provide structures in which the main and child documents can be
encapsulated and allowing them to be compiled individually.
The inclusion mechanism is different from the conventional |\include|.
\item
The package \href{http://ctan.org/pkg/combine}{\textsf{combine}}
is an elaborate solution to combine several documents into one.
\end{itemize}
%
See also the CTAN topic \href{http://ctan.org/topic/subdocs}{\textsf{subdocs}}
for further related packages.
The present package differs from the above solutions in that
a document structure constructed with the conventional |\include| mechanism
just needs two extra commands at the top of every file
such that all constituent files can be compiled individually.

%%%%%%%%%%%%%%%%%%%%%%%%%%%%%%%%%%%%%%%%%%%%%%%%%%%%%%%%%%%%%%%%%%%%%%%%%%%%%%%%
%\subsection{Feature Suggestions}
%
%The following is a list of features which may be useful for future
%versions of this package:
%%
%\begin{itemize}
%\item
%\ldots
%\end{itemize}

%%%%%%%%%%%%%%%%%%%%%%%%%%%%%%%%%%%%%%%%%%%%%%%%%%%%%%%%%%%%%%%%%%%%%%%%%%%%%%%%
\subsection{Revision History}

%%%%%%%%%%%%%%%%%%%%%%%%%%%%%%%%%%%%%%%%
\paragraph{v2.0:} 2018/12/30

\begin{itemize}
\item
immediate forward processing
\item
added |\childdocby| mechanism
\item
manual restructured
\end{itemize}

%%%%%%%%%%%%%%%%%%%%%%%%%%%%%%%%%%%%%%%%
\paragraph{v1.6:} 2018/01/17

\begin{itemize}
\item
application for development of include files
\item
corrections to manual
\end{itemize}

%%%%%%%%%%%%%%%%%%%%%%%%%%%%%%%%%%%%%%%%
\paragraph{v1.5:} 2017/05/21

\begin{itemize}
\item
more complete structuring introduced
\item
|\childdocof| introduced
\item
|\childdoc| renamed to |\childdocmain|
\item
|\childredirect| renamed to |\childdocforward| and |\childdocforwardprefix|
and functionality expanded
\end{itemize}

%%%%%%%%%%%%%%%%%%%%%%%%%%%%%%%%%%%%%%%%
\paragraph{v1.0:} 2017/04/27

\begin{itemize}
\item
manual and install package
\item
first version published on CTAN
\end{itemize}

%%%%%%%%%%%%%%%%%%%%%%%%%%%%%%%%%%%%%%%%
\paragraph{v0.6:} 2017/04/26

\begin{itemize}
\item
redirection mechanism added
\end{itemize}

%%%%%%%%%%%%%%%%%%%%%%%%%%%%%%%%%%%%%%%%
\paragraph{v0.5:} 2017/04/26

\begin{itemize}
\item
functionality in definition file
\end{itemize}


%%%%%%%%%%%%%%%%%%%%%%%%%%%%%%%%%%%%%%%%%%%%%%%%%%%%%%%%%%%%%%%%%%%%%%%%%%%%%%%%
%%%%%%%%%%%%%%%%%%%%%%%%%%%%%%%%%%%%%%%%%%%%%%%%%%%%%%%%%%%%%%%%%%%%%%%%%%%%%%%%
%%%%%%%%%%%%%%%%%%%%%%%%%%%%%%%%%%%%%%%%%%%%%%%%%%%%%%%%%%%%%%%%%%%%%%%%%%%%%%%%
\appendix

\settowidth\MacroIndent{\rmfamily\scriptsize 000\ }

 \DocInput{childdoc.dtx}

\end{document}
%</driver>
% \fi
%
% %%%%%%%%%%%%%%%%%%%%%%%%%%%%%%%%%%%%%%%%%%%%%%%%%%%%%%%%%%%%%%%%%%%%%%%%%%%%%%
% %%%%%%%%%%%%%%%%%%%%%%%%%%%%%%%%%%%%%%%%%%%%%%%%%%%%%%%%%%%%%%%%%%%%%%%%%%%%%%
% \section{Sample}
%\iffalse
%<*samplemain>
%\fi
%
% The following presents a sample document
% with two chapters, two parts, a title page,
% a compile flag as well as three forwarding files to set the flag.
% It consists of eight |.tex| files:
% \begin{center}
% \begin{tabular}{ll}
% |cdocsamp.tex|&main file\\
% |cdocsch1.tex|&include file for chapter 1\\
% |cdocsch2.tex|&include file for chapter 2\\
% |cdocspt3.tex|&include file for part 3\\
% |cdocspt4.tex|&include file for part 4\\
% |cdocsdrf.tex|&forwarding file for main file in draft mode\\
% |cdocsfi1.tex|&forwarding file for final version of chapter 1\\
% |cdocsfi2.tex|&forwarding file for final version of chapter 2\\
% \end{tabular}
% \end{center}
% Each of the eight files can be compiled directly by the \LaTeX{} compiler.
%
% %%%%%%%%%%%%%%%%%%%%%%%%%%%%%%%%%%%%%%
% \paragraph{Main File.}
%
% The main file is called |cdocsamp.tex|.
%
% Load the \textsf{childdoc} definitions and
% declare the filename for the main document:
%    \begin{macrocode}
\input{childdoc.def}
\childdocmain{}
%    \end{macrocode}

% Optional override for |\version| flag:
%    \begin{macrocode}
%%\ifchilddoc\else\providecommand{\version}{draft}\fi
%    \end{macrocode}

% Define the default values for the |\version| flag
% (|final| for the main file and |draft| for childs):
%    \begin{macrocode}
\ifchilddoc
\providecommand{\version}{draft}
\else
\providecommand{\version}{final}
\fi
%    \end{macrocode}

% Load the standard document class:
%    \begin{macrocode}
\documentclass[12pt]{article}
%    \end{macrocode}

% Start the document body:
%    \begin{macrocode}
\begin{document}
%    \end{macrocode}

% Declare a title page.
% Print title, part of document being processed and version flag:
%    \begin{macrocode}
\addtocounter{page}{-1}
\begin{center}
{\LARGE\bfseries{}childdoc example\par}
\vspace{1cm}
\ifchilddoc
\ifchilddocmanual part\else chapter\fi:
`\childdocname' of `\childdocjob'\par
\else
main document: `\childdocjob'\par
\fi
version: \version\par
\end{center}
\newpage
%    \end{macrocode}

% Manually include selected file,
% otherwise process as usual:
%    \begin{macrocode}
\ifchilddocmanual
\section*{part `\childdocname'}
\input{\childdocname}
\else
%    \end{macrocode}

% Include the two chapters:
%    \begin{macrocode}
\include{cdocsch1}
\include{cdocsch2}
%    \end{macrocode}

% Include the two parts unless only chapters should be displayed:
%    \begin{macrocode}
\ifchilddoc\else
\section{part three}
\input{cdocspt3}
\section{part four}
\input{cdocspt4}
\fi
%    \end{macrocode}

% Process as usual until here:
%    \begin{macrocode}
\fi
%    \end{macrocode}

% End of document body:
%    \begin{macrocode}
\end{document}
%    \end{macrocode}
%\iffalse
%</samplemain>
%\fi
%
% %%%%%%%%%%%%%%%%%%%%%%%%%%%%%%%%%%%%%%
% \paragraph{Chapter Include Files.}
%
% The include files are called |cdocsch1.tex| and |cdocsch2.tex|.
%
%\iffalse
%<*samplechap1|samplechap2>
%\fi

% Optional override for |\version| flag:
%    \begin{macrocode}
%%\providecommand{\version}{final}
%    \end{macrocode}

% Include the main document:
%    \begin{macrocode}
\input{childdoc.def}
\childdocof{cdocsamp}
%    \end{macrocode}

%\iffalse
%</samplechap1|samplechap2>
%\fi
%
%\iffalse
%<*samplechap1>
%\fi
% Some text for chapter 1:
%    \begin{macrocode}
\section{one}
some text in chapter one
%    \end{macrocode}

%\iffalse
%</samplechap1>
%\fi
% Some text for chapter 2:
%\iffalse
%<*samplechap2>
%\fi
%    \begin{macrocode}
\section{two}
more text in chapter two
%    \end{macrocode}

%\iffalse
%</samplechap2>
%\fi
%
% %%%%%%%%%%%%%%%%%%%%%%%%%%%%%%%%%%%%%%
% \paragraph{Part Include Files.}
%
% The include files are called |cdocspt3.tex| and |cdocspt4.tex|.
%
%\iffalse
%<*samplepart3|samplepart4>
%\fi

% Optional override for |\version| flag:
%    \begin{macrocode}
%%\providecommand{\version}{final}
%    \end{macrocode}

% Include the main document:
%    \begin{macrocode}
\input{childdoc.def}
\childdocby{cdocsamp}
%    \end{macrocode}

%\iffalse
%</samplepart3|samplepart4>
%\fi
%
%\iffalse
%<*samplepart3>
%\fi
% Some text for part 3:
%    \begin{macrocode}
some text in part three
%    \end{macrocode}

%\iffalse
%</samplepart3>
%\fi
% Some text for part 4:
%\iffalse
%<*samplepart4>
%\fi
%    \begin{macrocode}
more text in part four
%    \end{macrocode}

%\iffalse
%</samplepart4>
%\fi
%
% %%%%%%%%%%%%%%%%%%%%%%%%%%%%%%%%%%%%%%
% \paragraph{Forwarding for a Complete Draft.}
%
% The following forwarding file |cdocsdrf.tex|
% compiles the main document in draft mode:
%\iffalse
%<*sampledraft>
%\fi
%    \begin{macrocode}
\def\version{draft}
\input{childdoc.def}
\childdocforward{cdocsamp}
%    \end{macrocode}

%\iffalse
%</sampledraft>
%\fi
%
% %%%%%%%%%%%%%%%%%%%%%%%%%%%%%%%%%%%%%%
% \paragraph{Forwarding for Final Version of the Chapters.}
%
% The following forwarding files |cdocsfn1.tex| and |cdocsfn2.tex|
% (with identical content)
% compile the final versions of the child documents
% |cdocsch1.tex| and |cdocsch2.tex|, respectively:
%\iffalse
%<*samplefinal>
%\fi
%    \begin{macrocode}
\def\version{final}
\input{childdoc.def}
\childdocforwardprefix[cdocsamp]{cdocsfn}{cdocsch}
%    \end{macrocode}

%\iffalse
%</samplefinal>
%\fi
%
% %%%%%%%%%%%%%%%%%%%%%%%%%%%%%%%%%%%%%%
% \paragraph{Command Line Processing.}
%
% The following three command lines generate the output files
% |cdocscld|, |cdocscl1| and |cdocscl2|
% which should be identical to
% |cdocsdrf|, |cdocsch1| and |cdocsfn2|, respectively:
% \begin{center}
% \begin{tabular}{l}
% |latex -jobname cdocscld \|\\
% |  "\def\version{draft}\input{childdoc.def}\childdocforward{cdocsamp}"|\\
% |latex -jobname cdocscl1 \|\\
% |  "\input{childdoc.def}\childdocforward[cdocsamp]{cdocsch1}"|\\
% |latex -jobname cdocscl2 \|\\
% |  "\def\version{final}\input{childdoc.def}\childdocforward{cdocsch2}"|
% \end{tabular}
% \end{center}
% Note that the trailing backslash on each first line
% merely continues the input to the second line
% (for convenient cut ant paste).
% Furthermore, the command |latex| can be replaced by any
% of its alternative versions such as |pdflatex|.
%
% %%%%%%%%%%%%%%%%%%%%%%%%%%%%%%%%%%%%%%%%%%%%%%%%%%%%%%%%%%%%%%%%%%%%%%%%%%%%%%
% %%%%%%%%%%%%%%%%%%%%%%%%%%%%%%%%%%%%%%%%%%%%%%%%%%%%%%%%%%%%%%%%%%%%%%%%%%%%%%
% \section{Implementation}
%\iffalse
%<*package>
%\fi
%
% This section describes the definitions file |childdoc.def|.

% The definitions cannot be loaded using |\usepackage| or |\RequirePackage|
% which has a mechanism to prevent loading a style file more than once.
% When loading the definitions by means of |\input|
% multiple instances have to be prevented manually:
%\iffalse
%This code needs to be before the `\ProvidesFile' directive
%which is defined at the beginning of this file.
%Therefore it is also placed there and commented out here.
%</package>
%<*discard>
%\fi
%    \begin{macrocode}
\ifdefined\childdocmain\endinput\fi
%    \end{macrocode}
%\iffalse
%</discard>
%<*package>
%\fi
%
% \macro{\ifchilddoc}
% \macro{\ifchilddocmanual}
% The conditional |\ifchilddoc| tells whether a
% child (true) or main (false) document is being compiled.
% The conditional |\ifchilddocmanual| tells whether
% the |\includeonly| mechanism is used (false) or
% the selection of child files must be performed manually (true).
% The definitions initialise to false:
%    \begin{macrocode}
\newif\ifchilddoc
\newif\ifchilddocmanual
%    \end{macrocode}

% \macro{\childdocname}
% \macro{\childdocjob}
% The macro |\childdocname| stores the name of the main document
% to be compiled. The macro |\childdocjob| stores the name of
% the document on which the \LaTeX{} compiler was originally invoked.
% The content of |\jobname| cannot be compared
% to filenames specified in the source due to different catcodes.
% The following code rescans |\jobname|, stores the result
% in |\childdocname| and saves a copy in |\childdocjob|:
%    \begin{macrocode}
\edef\childdocname{\scantokens\expandafter{\jobname\noexpand}}
\let\childdocjob\childdocname
%    \end{macrocode}

% \macro{\childdocdisable}
% The macro |\childdocdisable| prevents the main file
% from being processed more than once.
% At this stage, the main document command |\childdocmain|
% is assumed to be called once again where it should do nothing.
% Any subsequent call to it should prevent
% a secondary processing of the main document
% It overwrites the forwarding commands
% |\childdocof| and |\childdocforward|
% with empty macros to prevent further inclusions of the main document:
%    \begin{macrocode}
\newcommand{\childdocdisable}
{
  \renewcommand{\childdocmain}[1]{\renewcommand{\childdocmain}[1]{\endinput}}
  \renewcommand{\childdocof}[1]{}
  \renewcommand{\childdocby}[2][]{}
  \renewcommand{\childdocforward}[2][]{}
  \renewcommand{\childdocdisable}{}
}
%    \end{macrocode}

% \macro{\childdocmain}
% The macro |\childdocmain| is to be called at the top of the main file
% with nothing or the main filename (without extension) as argument.
% First, it breaks loops.
% If the argument is not empty and does not match |\childdocname|
% (which is set by the first inclusion of |childdoc.def|),
% |\ifchilddoc| is set to true, |\includeonly| is applied to the child file
% and |\jobname| is set to the main file
% (for proper handling of |.aux| files):
%    \begin{macrocode}
\newcommand{\childdocmain}[1]
{
  \childdocdisable\childdocmain{}
  \if?#1?\else
    \begingroup
      \def\childdoctmp{#1}
      \ifx\childdoctmp\childdocname
        \def\childdoctmp{}
      \else
        \def\childdoctmp
        {
          \childdoctrue
          \includeonly{\childdocname}
          \def\childdocjob{#1}
          \def\jobname{#1}
        }
      \fi
      \expandafter
    \endgroup
    \childdoctmp
  \fi
}
%    \end{macrocode}

% \macro{\childdocof}
% The command |\childdocof| redirects
% compilation to the main file |#1|.
%    \begin{macrocode}
\newcommand{\childdocof}[1]
{
  \childdocdisable
  \childdoctrue
  \includeonly{\childdocname}
  \def\jobname{#1}
  \def\childdocjob{#1}
  \input{#1}
}
%    \end{macrocode}

% \macro{\childdocby}
% The command |\childdocby| ....
%    \begin{macrocode}
\newcommand{\childdocby}[2][]
{
  \childdocdisable
  \childdoctrue
  \childdocmanualtrue
  \if?#1?\else
    \def\jobname{#2}
  \fi
  \def\childdocjob{#2}
  \input{#2}
  \endinput
}
%    \end{macrocode}

% \macro{\childdocforward}
% The command |\childdocforward| redirects
% compilation to the main file or
% (if the optional argument is given) a child file.
% Parameters are set as if the main file
% or a child file starting with |\childdocof| was compiled.
% Then compilation is handed over to the main file:
%    \begin{macrocode}
\newcommand{\childdocforward}[2][]
{
  \begingroup
    \if?#1?
      \def\childdoctmp
      {
        \def\childdocname{#2}
        \def\childdocjob{#2}
        \def\jobname{#2}
        \input{#2}
        \endinput
      }
    \else
      \def\childdoctmp
      {
        \childdocdisable
        \def\childdocname{#2}
        \childdoctrue
        \includeonly{#2}
        \def\childdocjob{#1}
        \def\jobname{#1}
        \input{#1}
        \endinput
      }
    \fi
    \expandafter
  \endgroup
  \childdoctmp
}
%    \end{macrocode}

% \macro{\childdocforwardprefix}
% The command |\childdocforwardprefix| redirects
% compilation to the main or a child file by means of a pattern.
% The prefix |#1| in the current filename is replaced by |#2|
% and the suffix of the current filename is kept
% (it is assumed that the filename does not contain the substring `|~~~|'
% which is used as a delimiter).
% Compilation is handed over to the new file by |\childdocforward|:
%    \begin{macrocode}
\newcommand{\childdocforwardprefix}[3][]
{
  \begingroup
    \def\childdocextract #2##1~~~{\def\childdoctmp{\childdocforward[#1]{#3##1}}}
    \expandafter\childdocextract\childdocname~~~
    \expandafter
  \endgroup
  \childdoctmp
}
%    \end{macrocode}

% \macro{\childdoc}
% The deprecated macro |\childdoc| is a legacy version of |\childdocmain|:
%    \begin{macrocode}
\newcommand{\childdoc}{\childdocmain}
%    \end{macrocode}

% \macro{\childdocredirect}
% The deprecated macro |\childdocredirect| is a legacy version
% of |\childdocforward| and |\childdocforwardprefix|:
%    \begin{macrocode}
\newcommand{\childdocredirect}[2][]
{
  \begingroup
    \if?#1?
      \def\childdoctmp{\childdocforward{#2}}
    \else
      \def\childdoctmp{\childdocforwardprefix{#1}{#2}}
    \fi
    \expandafter
  \endgroup
  \childdoctmp
}
%    \end{macrocode}

%\iffalse
%</package>
%\fi
%
\endinput
\childdocforward{cdocsamp}"|\\
% |latex -jobname cdocscl1 \|\\
% |  "% \iffalse
%
% childdoc.dtx Copyright (C) 2017-2018 Niklas Beisert
%
% This work may be distributed and/or modified under the
% conditions of the LaTeX Project Public License, either version 1.3
% of this license or (at your option) any later version.
% The latest version of this license is in
%   http://www.latex-project.org/lppl.txt
% and version 1.3 or later is part of all distributions of LaTeX
% version 2005/12/01 or later.
%
% This work has the LPPL maintenance status `maintained'.
%
% The Current Maintainer of this work is Niklas Beisert.
%
% This work consists of the files childdoc.dtx and childdoc.ins
% and the derived files childdoc.def and cdocsamp.tex with
% cdocsch1.tex, cdocsch2.tex, cdocsdrf.tex, cdocsfn1.tex, cdocsfn2.tex.
%
%<package>\ifdefined\childdocmain\endinput\fi
%<package>\ProvidesFile{childdoc.def}[2018/12/30 v2.0 child document driver]
%<samplemain>\ProvidesFile{cdocsamp.tex}[2018/12/30 v2.0 sample for childdoc]
%<*driver>
%\ProvidesFile{childdoc.drv}[2018/12/30 v2.0 childdoc reference manual file]
\PassOptionsToClass{10pt,a4paper}{article}
\documentclass{ltxdoc}

\usepackage[margin=35mm]{geometry}
\usepackage{hyperref}
\usepackage{hyperxmp}
\usepackage[usenames]{color}

\hypersetup{colorlinks=true}
\hypersetup{pdfstartview=FitH}
\hypersetup{pdfpagemode=UseNone}
\hypersetup{pdfsource={}}
\hypersetup{pdflang={en-UK}}
\hypersetup{pdfcopyright={Copyright 2017-2018 Niklas Beisert.
  This work may be distributed and/or modified under the
  conditions of the LaTeX Project Public License, either version 1.3
  of this license or (at your option) any later version.}}
\hypersetup{pdflicenseurl={http://www.latex-project.org/lppl.txt}}
\hypersetup{pdfcontactaddress={ETH Zurich, ITP, HIT K,
  Wolfgang-Pauli-Strasse 27}}
\hypersetup{pdfcontactpostcode={8093}}
\hypersetup{pdfcontactcity={Zurich}}
\hypersetup{pdfcontactcountry={Switzerland}}
\hypersetup{pdfcontactemail={nbeisert@itp.phys.ethz.ch}}
\hypersetup{pdfcontacturl={http://people.phys.ethz.ch/\xmptilde nbeisert/}}

\newcommand{\secref}[1]{\hyperref[#1]{section \ref*{#1}}}

\parskip1ex
\parindent0pt
\let\olditemize\itemize
\def\itemize{\olditemize\parskip0pt}

\begin{document}

\title{The \textsf{childdoc} Package}
\hypersetup{pdftitle={The childdoc Package}}
\author{Niklas Beisert\\[2ex]
  Institut f\"ur Theoretische Physik\\
  Eidgen\"ossische Technische Hochschule Z\"urich\\
  Wolfgang-Pauli-Strasse 27, 8093 Z\"urich, Switzerland\\[1ex]
  \href{mailto:nbeisert@itp.phys.ethz.ch}
  {\texttt{nbeisert@itp.phys.ethz.ch}}}
\hypersetup{pdfauthor={Niklas Beisert}}
\hypersetup{pdfsubject={Manual for the LaTeX2e Package childdoc}}
\date{30 December 2018, \textsf{v2.0}}
\maketitle

\begin{abstract}\noindent
\textsf{childdoc} is a \LaTeXe{} package
that enables the direct compilation
of document sections included by |\include|
to individual files.
\end{abstract}

\begingroup
\parskip0ex
\tableofcontents
\endgroup

%%%%%%%%%%%%%%%%%%%%%%%%%%%%%%%%%%%%%%%%%%%%%%%%%%%%%%%%%%%%%%%%%%%%%%%%%%%%%%%%
%%%%%%%%%%%%%%%%%%%%%%%%%%%%%%%%%%%%%%%%%%%%%%%%%%%%%%%%%%%%%%%%%%%%%%%%%%%%%%%%
\section{Introduction}

\LaTeX{} provides a mechanism to structure a large document (such as a book)
into a main file and several child files (containing the chapters)
using the |\include| command.
This mechanism is beneficial for documents
which span hundreds of pages in order to
make the source file(s) more manageable.
Moreover, compilation can be restricted to
selected child files by means of the |\includeonly| command.
The latter feature can be used to reduce the compilation time while editing
(this was significantly more useful in the earlier days of \LaTeX{})
or to generate a smaller document which is easier to navigate.
Another application of |\includeonly| is to generate
documents consisting of selected parts of the complete document.

However, there are a few drawbacks of the plain |\include| mechanism:
\begin{itemize}
\item
The child files cannot be compiled on their own,
they can only be compiled via the main file.
A naive editing environment
(such as a text editor with an option
to have the current file processed by \LaTeX)
may require one to switch to the main file before compiling;
attempting to compile the child file produces errors.
\item
The main file must be modified (each time)
to adjust the |\includeonly| command
to the present needs. This easily leaves the main file in a messy state.
\item
The generated document will always carry the filename
of the main document. This is inconvenient if
several child files are to be compiled and
to be kept for distribution.
\end{itemize}

The present package provides a simple interface
to make child files individually compilable by \LaTeX{}.
Compiling a child file then has the same effect as compiling
the main file with an |\includeonly| command
to select the appropriate child.
Moreover the generated document will carry the name of the child
rather than the main file.
This resolves all three above issues.

This feature is meant to make the editing of books,
thesis documents and lecture notes somewhat more convenient.
However, the package can also be used efficiently for
composing a series of documents (such as exercise sheets)
which are typically distributed individually.
It then assists the author in generating the individual documents
(potentially in different versions)
as well as a document containing the collected series.
Another application is in developing style files
or other kinds of included material
where compilation of the style file could redirect
to a sample or test file.

%%%%%%%%%%%%%%%%%%%%%%%%%%%%%%%%%%%%%%%%%%%%%%%%%%%%%%%%%%%%%%%%%%%%%%%%%%%%%%%%
%%%%%%%%%%%%%%%%%%%%%%%%%%%%%%%%%%%%%%%%%%%%%%%%%%%%%%%%%%%%%%%%%%%%%%%%%%%%%%%%
\section{Usage}

First of all, the package \textsf{childdoc} is \emph{not} a standard
\LaTeXe{} |.sty| style file! Therefore it needs to be invoked in
a non-standard way.

%%%%%%%%%%%%%%%%%%%%%%%%%%%%%%%%%%%%%%%%%%%%%%%%%%%%%%%%%%%%%%%%%%%%%%%%%%%%%%%%
\subsection{Included Files}
\label{sec:include}

%%%%%%%%%%%%%%%%%%%%%%%%%%%%%%%%%%%%%%%%
\DescribeMacro{\childdocmain}
To use the package, add the commands
\begin{center}
\begin{tabular}{l}
|\input{childdoc.def}|\\
|\childdocmain{}|\\
\end{tabular}
\end{center}
at the very top of the main \LaTeX{} file,
in particular \emph{before} the |\documentclass| statement!
The argument of |\childdocmain| should be left empty
(but it must be present).

%%%%%%%%%%%%%%%%%%%%%%%%%%%%%%%%%%%%%%%%
\DescribeMacro{\childdocof}
Furthermore, add the commands
\begin{center}
\begin{tabular}{l}
|\input{childdoc.def}|\\
|\childdocof{|\textit{main}|}|\\
\end{tabular}
\end{center}
at the top of every child file \textit{child}
which is included by |\include{|\textit{child}|}|
from within the main file
(or at least for those files to be compiled individually).
The argument \textit{main} must be the filename of the main file.

There are a couple of
considerations in setting up the main and child documents:

%%%%%%%%%%%%%%%%%%%%%%%%%%%%%%%%%%%%%%%%
\paragraph{Restrictions.}

Please note the following restrictions:
\begin{itemize}
\item
|\childdocmain| must be called with one argument \textit{main}
to ensure compatibility with earlier version of the package.
It must either be empty (|\childdocmain{}|)
or precisely match the filename of the main file in which it is specified.
See \secref{sec:detection} for further information.
\item
The filename \textit{main} must be specified without the |.tex| extension.
\item
The filename \textit{main} is case sensitive
(even in case-insensitive file systems)
due to internal string comparison.
\item
The argument \textit{main} should be fully expanded, it cannot be a macro.
\item
Subdirectories and special characters should be avoided in filenames.
\item
The command |\childdocmain{|\textit{main}|}| must be followed by a whitespace.
It should not be followed immediately by another command
or by a comment mark `|%|'.
This is because the \TeX{} parser reads the token immediately following
the argument of |\childdocmain| and puts it
at the beginning of every child section;
however, a white\-space is ignored.
\end{itemize}

%%%%%%%%%%%%%%%%%%%%%%%%%%%%%%%%%%%%%%%%
\paragraph{Content of Main File.}

It is advisable to place all content in the child files included by |\include|.
Any output contained in the main file will appear in all child documents
unless suppressed manually;
it cannot be suppressed automatically by the |\includeonly| directive
and thus should normally be avoided.
A method to include some content in the main file
by means of conditional processing is described in \secref{sec:conditional}.

%%%%%%%%%%%%%%%%%%%%%%%%%%%%%%%%%%%%%%%%
\paragraph{Page Numbering.}

When only a part of the document is compiled,
the appropriate numbering of pages
(as well as other status parameters)
is determined from the |.aux| files.
The latter contain information from previous passes.
However this information needs to propagate through
all intermediate child documents.
Therefore the page numbering in child documents may well
be inconsistent until the complete document is compiled at least once.

A useful (if unconventional) way to always ensure a consistent
page numbering is to restart the numbering in each child document
and denote the pages by `\textit{child}|.|\textit{page}'
where \textit{child} represents the chapter/section number of the child file.
This can be achieved by the command
|\numberwithin{page}{|\textit{child}|}|
of the \textsf{amsmath} package
where \textit{child} can be |chapter| or |section|
depending on the chosen structuring.
Alternatively, one can modify the macro |\thepage| appropriately
and reset the counter |page| at the start of each child file.

%%%%%%%%%%%%%%%%%%%%%%%%%%%%%%%%%%%%%%%%%%%%%%%%%%%%%%%%%%%%%%%%%%%%%%%%%%%%%%%%
\subsection{Conditional Processing}
\label{sec:conditional}

The package provides a mechanism to compile different versions
of a document. To customise the versions further some conditional processing
can come in handy to distinguish which version is being compiled.
The package provides two macros to describe the compilation context:

%%%%%%%%%%%%%%%%%%%%%%%%%%%%%%%%%%%%%%%%
\DescribeMacro{\ifchilddoc}
The conditional |\ifchilddoc| distinguishes between the compilation of
child documents and the main document:
%
\begin{center}
|\ifchilddoc |\textit{child-code}| |[|\||else |\textit{main-code}]| \||fi|
\end{center}

%%%%%%%%%%%%%%%%%%%%%%%%%%%%%%%%%%%%%%%%
\DescribeMacro{\childdocname}
\DescribeMacro{\childdocjob}
The macro |\childdocname| contains the filename (without extension)
of the main or child file being processed.
Note that |\childdocjob| will always contain the name of the main file.

%%%%%%%%%%%%%%%%%%%%%%%%%%%%%%%%%%%%%%%%
\paragraph{Title Page.}

Conditional processing can be used to include a title or banner page
in the main document when proper precautions are taken.
Importantly, the code in the main file should ensure that the page counter
(as well as other status parameters which are stored in the |.aux| files)
takes the same value after the conditional processing.
Otherwise the page numbers may take divergent values
depending on which part is compiled.

For example, a title page could be declared by:
%
\begin{center}
\begin{tabular}{l}
|\ifchilddoc\||else|\\
|\addtocounter{page}{-1}|\\
\textit{code for title page}\\
|\newpage|\\
|\||fi|
\end{tabular}
\end{center}
%
A banner page for the child documents can be generated by:
%
\begin{center}
\begin{tabular}{l}
|\ifchilddoc|\\
|\addtocounter{page}{-1}|\\
\textit{code for banner page}\\
|\newpage|\\
|\||fi|
\end{tabular}
\end{center}
%
Here one could write a message such as:
\begin{center}
|This is the part \childdocname{} of \childdocjob{}.|
\end{center}

%%%%%%%%%%%%%%%%%%%%%%%%%%%%%%%%%%%%%%%%%%%%%%%%%%%%%%%%%%%%%%%%%%%%%%%%%%%%%%%%
\subsection{Flags}
\label{sec:flags}

The package makes it easy to generate different versions
of the main or child documents.
To this end compilation flags can be defined
and assigned different default values.
They will be particularly useful in conjunction
with the forwarding mechanism described in \secref{sec:forward}.

For example, it may be useful to have a flag |\version|
which can be set to |draft| or |final|.
The document source will contain some conditional code
depending on the value of |\version|.
Suppose further, the flag should default to |final| for the main file
and to |draft| for child files
which is a natural assignment for editing the document.
This is achieved by placing the following code
in the preamble of the main document
(below the |\childdocmain| directive):
%
\begin{center}
\begin{tabular}{l}
|\ifchilddoc|\\
|\providecommand{\version}{draft}|\\
|\||else|\\
|\providecommand{\version}{final}|\\
|\||fi|
\end{tabular}
\end{center}
%
The definition by |\providecommand| makes sure
that previous definitions are not overwritten.
Further statements |\providecommand{\version}{...}|
can thus be added before the above code to override it.

For the main file, one might add a line
(between |\childdocmain| and the above block)
%
\begin{center}
|%\ifchilddoc\||else\providecommand{\version}{draft}\||fi|
\end{center}
%
which can be uncommented to produce a draft version.
Likewise one can add a line to the very top of a child file
(above the |\childdocof{|\textit{main}|}| directive)
%
\begin{center}
|%\providecommand{\version}{final}|
\end{center}
%
which can be uncommented to produce the final version of this child document.

%%%%%%%%%%%%%%%%%%%%%%%%%%%%%%%%%%%%%%%%%%%%%%%%%%%%%%%%%%%%%%%%%%%%%%%%%%%%%%%%
\subsection{Forwarding}
\label{sec:forward}

Different versions of the main or child documents
using compilation flags as described in \secref{sec:flags}
can be (permanently) stored in different files
for convenient compilation, viewing and distribution.
To this end, the package defines a command
to pass on compilation to a different file:

%%%%%%%%%%%%%%%%%%%%%%%%%%%%%%%%%%%%%%%%
\DescribeMacro{\childdocforward}
The command |\childdocforward| redirects processing to
another source file:
%
\begin{center}
\begin{tabular}{l}
|\input{childdoc.def}|\\
|\childdocforward[|\textit{main}|]{|\textit{dest}|}|\\
\end{tabular}
\end{center}
%
The argument \textit{dest} is the destination file
(without extension).
It should be the main file or one of the child files.
Note that further \textsf{childdoc} directives
such as |\childdocof| and |\childdocforward|
in the indicated file will be processed in this form.
The optional argument \textit{main}
passes on directly to the main file \textit{main}
while pretending to compile the child \textit{dest}.
This form behaves as if \textit{dest}
issues |\childdocof{|\textit{main}|}| right away,
and no further \textsf{childdoc} directives will be processed.

%%%%%%%%%%%%%%%%%%%%%%%%%%%%%%%%%%%%%%%%
\DescribeMacro{\...prefix}
In the alternative form |\childdocforwardprefix|,
%
\begin{center}
\begin{tabular}{l}
|\input{childdoc.def}|\\
|\childdocforwardprefix[|\textit{main}|]{|\textit{prefix}|}{|\textit{dest}|}|
\end{tabular}
\end{center}
%
the destination file is determined by a pattern
depending on the current file:
To make this work, the current file must be called
`{\textit{prefix}\hspace{0.2em}\textit{suffix}}'
with \textit{prefix} matching precisely the argument.
Processing is then passed on to the file
`{\textit{dest}\hspace{0.2em}\textit{suffix}}'.
Surely, the same effect is achieved by
directly specifying the
argument `{\textit{dest}\hspace{0.2em}\textit{suffix}}'
in the first form.
However, that requires to set up a different file
for each child. With the alternative form of the command
all these files can have exactly the same content
which simplifies setting them up and maintaining them.

For example, the following file |draft.tex|
with a compilation flag |\version| as described in \secref{sec:flags}
compiles the main document as a draft:
%
\begin{center}
\begin{tabular}{l}
|\def\version{draft}|\\
|\input{childdoc.def}|\\
|\childdocforward{|\textit{main}|}|
\end{tabular}
\end{center}
%
Likewise, the following files |final|\textit{nn}|.tex|
compile the final version of the child document
|child|\textit{nn}|.tex|:
%
\begin{center}
\begin{tabular}{l}
|\def\version{final}|\\
|\input{childdoc.def}|\\
|\childdocforwardprefix{final}{child}|
\end{tabular}
\end{center}
%

Note that when several versions of a main file and/or of each child file
are to be generated, it may be convenient to set up a |Makefile| or
shell script to automatise the process.

%%%%%%%%%%%%%%%%%%%%%%%%%%%%%%%%%%%%%%%%%%%%%%%%%%%%%%%%%%%%%%%%%%%%%%%%%%%%%%%%
\subsection{Command Line Processing}
\label{sec:commandline}

The effect of redirection files can also be achieved by invoking
the \LaTeX{} compiler with a more elaborate command line.
Most conveniently this should be done as part
of a shell script or a |Makefile|.

When using \textsf{childdoc} in the main file, the following
command lines effectively perform a redirection
(note that depending on the shell being used,
backslashes may have to be doubled: `|\|' $\to$ `|\\|'):
%
\begin{center}
|... -jobname "|\textit{target}|" |\\|"|[\textit{flags}]%
|\input{childdoc.def}\childdocforward[|\textit{main}|]{|\textit{dest}|}"|
\end{center}
%
Here \textit{target} is the name of the output file,
\textit{main} is the name of the main file
and \textit{dest} is the name of the main or child file to be processed
(all filenames without extensions).
The optional argument \textit{main} can be omitted
if \textit{main} matches \textit{dest}.
Optionally, compilation \textit{flags} can be defined via |\def| commands.
This command line makes the \TeX{} engine believe
it is compiling the file \textit{target}
whose content is specified as the latter parameter.
The provided code then forwards the processing to
\textit{main} or \textit{dest} as described in \secref{sec:forward}.

%%%%%%%%%%%%%%%%%%%%%%%%%%%%%%%%%%%%%%%%%%%%%%%%%%%%%%%%%%%%%%%%%%%%%%%%%%%%%%%%
\subsection{Include by Input}
\label{sec:input}

Including child documents by |\include| has some restrictions by design.
Most notably, the content of a child document always occupies
its own set of pages; pages cannot be shared between child documents.
Usually, this behaviour makes perfect sense
because each child document contain an essential part of the document.
However, in some situations it may be desirable to compose
a document from a collection of parts
without having mandatory page breaks between then.
For this case, the package
provides a mechanism to include parts
by |\input| which can also be processed individually.
However, by construction this mechanism
requires manual handling of the content to be output.

%%%%%%%%%%%%%%%%%%%%%%%%%%%%%%%%%%%%%%%%
\DescribeMacro{\ifchilddocmanual}
The main file should be prepared as usual, see \secref{sec:include}.
However, the document body must make a distinction
between processing of an individual part and of the main document, e.g.:
%
\begin{center}
\begin{tabular}{l}
|\ifchilddocmanual|\\
|\input{\childdocname}|\\
|\||else|\\
\textit{document body with }|\input{|\textit{part}|}|\\
|\||fi|
\end{tabular}
\end{center}
%
The conditional |\ifchilddocmanual| is true whenever
a part to be included by |\input| is being compiled,
and the name of the part is stored in |\childdocname|.

%%%%%%%%%%%%%%%%%%%%%%%%%%%%%%%%%%%%%%%%
\DescribeMacro{\childdocby}
Each part to be included by |\input| should start with:
%
\begin{center}
\begin{tabular}{l}
|\input{childdoc.def}|\\
|\childdocby{|\textit{main}|}|\\
\end{tabular}
\end{center}
%
The directive |\childdocby| is similar to |\childdocof|
described in \secref{sec:include},
but the subsequent selection of content must be done manually.
To that end, both |\ifchilddoc| and |\ifchilddocmanual|
will be true upon processing of a part,
and the name of the part is stored in |\childdocname|.
Note that |\jobname| will be set to the filename of the current part
so that each part receives an individual |.aux| file
that does not interfere with the |.aux| file(s) of the main document.
This behaviour can be altered by the alternative form
|\childdocby[*]{|\textit{main}|}| (with a non-empty optional argument)
which uses the |.aux| file of the main document
by setting |\jobname| to \textit{main}.

%%%%%%%%%%%%%%%%%%%%%%%%%%%%%%%%%%%%%%%%%%%%%%%%%%%%%%%%%%%%%%%%%%%%%%%%%%%%%%%%
\subsection{Driver Development}
\label{sec:driver}

The \textsf{childdoc} mechanism can also be use for the development
of definition files such as \LaTeX{} styles or classes.
This case differs from the above setup with multiple parts
included by |\include| in that no |\includeonly| should be invoked.
This can be achieved by starting the include file
(before |\ProvidesPackage|) with:
%
\begin{center}
\begin{tabular}{l}
|\input{childdoc.def}|\\
|\childdocforward{|\textit{main}|}|\\
\end{tabular}
\end{center}
%
or alternatively with:
%
\begin{center}
\begin{tabular}{l}
|\input{childdoc.def}|\\
|\childdocby{|\textit{main}|}|\\
\end{tabular}
\end{center}
%
Both forms have slightly different effects as described above.
The main file is prepared as usual, see \secref{sec:include}.

%%%%%%%%%%%%%%%%%%%%%%%%%%%%%%%%%%%%%%%%%%%%%%%%%%%%%%%%%%%%%%%%%%%%%%%%%%%%%%%%
\subsection{Legacy Detection}
\label{sec:detection}

The directive |\childdocmain| in the main file can detect
whether the complete document or merely a child is to be compiled
even without using the directive |\childdocof|.
This method is deprecated because it is less robust
and there is no compelling reason to use it;
it is merely provided for backward compatibility
and it may be removed in future versions.

If the detection mechanism is to be used,
it is mandatory to correctly specify
the filename of the main file as the argument of |\childdocmain|:
%
\begin{center}
\begin{tabular}{l}
|\input{childdoc.def}|\\
|\childdocmain{|\textit{main}|}|\\
\end{tabular}
\end{center}
%
If |\jobname| does not match the argument \textit{main} of |\childdocmain|,
it is assumed that |\jobname| points to the child file to be compiled.
When using |\childdocmain| with the main file specified as argument,
it suffices to start a child file
with just |\input{|\textit{main}|}|
without loading of the package and using |\childdocof|.
If instead all processing is done
with the appropriate \textsf{childdoc} directives,
the argument of \textit{main} of |\childdocmain| can be empty.

An alternative version of the command line processing described
in \secref{sec:commandline} using the detection mechanism reads:
%
\begin{center}
|... -jobname "|\textit{target}|" "|[\textit{flags}]%
[|\def\jobname{|\textit{dest}|}|]|\input{|\textit{main}|}"|
\end{center}

%%%%%%%%%%%%%%%%%%%%%%%%%%%%%%%%%%%%%%%%%%%%%%%%%%%%%%%%%%%%%%%%%%%%%%%%%%%%%%%%
\subsection{Manual Code}
\label{sec:manual}

In case one cannot be certain whether the definitions file |childdoc.def|
is installed on the target \TeX{} distribution
and one prefers not to ship it,
it is conceivable to paste a few relevant commands into the sources.

To that end, drop all statements |\input{childdoc.def}|
and perform the replacements as outlined below.
Instead of |\childdocmain{|\textit{main}|}| add the following code
to the top of the main file:
%
\begin{center}
\begin{tabular}{l}
|\||ifdefined\childdocname\endinput\||fi\newif\ifchilddoc|\\
|\edef\childdocname{\scantokens\expandafter{\jobname\noexpand}}|\\
|\def\childdocmain{|\textit{main}|}\||ifx\childdocmain\childdocname\||else|\\
|\childdoctrue\includeonly{\childdocname}\let\jobname\childdocmain\||fi|\\
\end{tabular}
\end{center}
%
Instead of |\childdocof{|\textit{main}|}| just include the main file
at the top of each child file:
%
\begin{center}
|\input{|\textit{main}|}|
\end{center}
%
A simple redirection |\childdocforward{|\textit{dest}|}| is achieved by:
%
\begin{center}
|\def\jobname{|\textit{dest}|}\input{\jobname}|
\end{center}
%
The redirection with prefix
|\childdocforwardprefix[|\textit{prefix}|]{|\textit{dest}|}|
is accomplished by:
%
\begin{center}
\begin{tabular}{l}
|{\edef\jobname{\scantokens\expandafter{\jobname\noexpand}}|\\
|\def\redirectjob |\textit{prefix}|#1~~~{\gdef\jobname{|\textit{dest}|#1}}|\\
|\expandafter\redirectjob\jobname~~~}\input{\jobname}|
\end{tabular}
\end{center}

In an alternative approach,
child documents can be compiled by a specific command line
without additional code or specific definitions:
%
\begin{center}
|... -jobname "|\textit{target}|" "|[\textit{flags}]%
|\includeonly{|\textit{dest}|}\input{|\textit{main}|}"|
\end{center}
%

%%%%%%%%%%%%%%%%%%%%%%%%%%%%%%%%%%%%%%%%%%%%%%%%%%%%%%%%%%%%%%%%%%%%%%%%%%%%%%%%
%%%%%%%%%%%%%%%%%%%%%%%%%%%%%%%%%%%%%%%%%%%%%%%%%%%%%%%%%%%%%%%%%%%%%%%%%%%%%%%%
\section{Information}

%%%%%%%%%%%%%%%%%%%%%%%%%%%%%%%%%%%%%%%%%%%%%%%%%%%%%%%%%%%%%%%%%%%%%%%%%%%%%%%%
\subsection{Copyright}

Copyright \copyright{} 2017--2018 Niklas Beisert

This work may be distributed and/or modified under the
conditions of the \LaTeX{} Project Public License, either version 1.3
of this license or (at your option) any later version.
The latest version of this license is in
  \url{http://www.latex-project.org/lppl.txt}
and version 1.3 or later is part of all distributions of \LaTeX{}
version 2005/12/01 or later.

This work has the LPPL maintenance status `maintained'.

The Current Maintainer of this work is Niklas Beisert.

This work consists of the files |README.txt|, |childdoc.ins| and |childdoc.dtx|
as well as the derived files |childdoc.def|, |cdocsamp.tex|
with |cdocsch1.tex|, |cdocsch2.tex|, |cdocspt3.tex|, |cdocspt4.tex|,
|cdocsdrf.tex|, |cdocsfn1.tex|, |cdocsfn2.tex|
as well as |childdoc.pdf|.

%%%%%%%%%%%%%%%%%%%%%%%%%%%%%%%%%%%%%%%%%%%%%%%%%%%%%%%%%%%%%%%%%%%%%%%%%%%%%%%%
\subsection{Files and Installation}

The package consists of the files:
%
\begin{center}
\begin{tabular}{ll}
    |README.txt|   & readme file \\
    |childdoc.ins| & installation file \\
    |childdoc.dtx| & source file \\
    |childdoc.def| & definition file \\
    |cdocsamp.tex| & sample main file \\
    |cdocsch1.tex| & sample include file \\
    |cdocsch2.tex| & sample include file \\
    |cdocspt3.tex| & sample part file \\
    |cdocspt4.tex| & sample part file \\
    |cdocsdrf.tex| & sample redirection file \\
    |cdocsfn1.tex| & sample redirection file \\
    |cdocsfn2.tex| & sample redirection file \\
    |childdoc.pdf| & manual
\end{tabular}
\end{center}
%
The distribution consists of the files
|README.txt|, |childdoc.ins| and |childdoc.dtx|.
%
\begin{itemize}
\item
Run (pdf)\LaTeX{} on |childdoc.dtx|
to compile the manual |childdoc.pdf| (this file).
\item
Run \LaTeX{} on |childdoc.ins| to create the definitions file |childdoc.def|
and the sample |cdocsamp.tex| with include files
|cdocsch1.tex|, |cdocsch2.tex|, |cdocspt3.tex|, |cdocspt4.tex|,
|cdocsdrf.tex|, |cdocsfn1.tex|, |cdocsfn2.tex|.
Then copy the file |childdoc.def| to an appropriate directory of your \LaTeX{}
distribution, e.g.\ \textit{texmf-root}|/tex/latex/childdoc|.
\end{itemize}

%%%%%%%%%%%%%%%%%%%%%%%%%%%%%%%%%%%%%%%%%%%%%%%%%%%%%%%%%%%%%%%%%%%%%%%%%%%%%%%%
\subsection{Related CTAN Packages}

There are several other packages which offer a similar functionality:
%
\begin{itemize}
\item
The packages
\href{http://ctan.org/pkg/docmute}{\textsf{docmute}},
\href{http://ctan.org/pkg/includex}{\textsf{includex}} and
\href{http://ctan.org/pkg/standalone}{\textsf{standalone}}
provide commands to include only the document body of
a child file thus allowing both files to be compiled individually.
\item
The packages \href{http://ctan.org/pkg/subdocs}{\textsf{subdocs}}
and \href{http://ctan.org/pkg/subfiles}{\textsf{subfiles}}
provide structures in which the main and child documents can be
encapsulated and allowing them to be compiled individually.
The inclusion mechanism is different from the conventional |\include|.
\item
The package \href{http://ctan.org/pkg/combine}{\textsf{combine}}
is an elaborate solution to combine several documents into one.
\end{itemize}
%
See also the CTAN topic \href{http://ctan.org/topic/subdocs}{\textsf{subdocs}}
for further related packages.
The present package differs from the above solutions in that
a document structure constructed with the conventional |\include| mechanism
just needs two extra commands at the top of every file
such that all constituent files can be compiled individually.

%%%%%%%%%%%%%%%%%%%%%%%%%%%%%%%%%%%%%%%%%%%%%%%%%%%%%%%%%%%%%%%%%%%%%%%%%%%%%%%%
%\subsection{Feature Suggestions}
%
%The following is a list of features which may be useful for future
%versions of this package:
%%
%\begin{itemize}
%\item
%\ldots
%\end{itemize}

%%%%%%%%%%%%%%%%%%%%%%%%%%%%%%%%%%%%%%%%%%%%%%%%%%%%%%%%%%%%%%%%%%%%%%%%%%%%%%%%
\subsection{Revision History}

%%%%%%%%%%%%%%%%%%%%%%%%%%%%%%%%%%%%%%%%
\paragraph{v2.0:} 2018/12/30

\begin{itemize}
\item
immediate forward processing
\item
added |\childdocby| mechanism
\item
manual restructured
\end{itemize}

%%%%%%%%%%%%%%%%%%%%%%%%%%%%%%%%%%%%%%%%
\paragraph{v1.6:} 2018/01/17

\begin{itemize}
\item
application for development of include files
\item
corrections to manual
\end{itemize}

%%%%%%%%%%%%%%%%%%%%%%%%%%%%%%%%%%%%%%%%
\paragraph{v1.5:} 2017/05/21

\begin{itemize}
\item
more complete structuring introduced
\item
|\childdocof| introduced
\item
|\childdoc| renamed to |\childdocmain|
\item
|\childredirect| renamed to |\childdocforward| and |\childdocforwardprefix|
and functionality expanded
\end{itemize}

%%%%%%%%%%%%%%%%%%%%%%%%%%%%%%%%%%%%%%%%
\paragraph{v1.0:} 2017/04/27

\begin{itemize}
\item
manual and install package
\item
first version published on CTAN
\end{itemize}

%%%%%%%%%%%%%%%%%%%%%%%%%%%%%%%%%%%%%%%%
\paragraph{v0.6:} 2017/04/26

\begin{itemize}
\item
redirection mechanism added
\end{itemize}

%%%%%%%%%%%%%%%%%%%%%%%%%%%%%%%%%%%%%%%%
\paragraph{v0.5:} 2017/04/26

\begin{itemize}
\item
functionality in definition file
\end{itemize}


%%%%%%%%%%%%%%%%%%%%%%%%%%%%%%%%%%%%%%%%%%%%%%%%%%%%%%%%%%%%%%%%%%%%%%%%%%%%%%%%
%%%%%%%%%%%%%%%%%%%%%%%%%%%%%%%%%%%%%%%%%%%%%%%%%%%%%%%%%%%%%%%%%%%%%%%%%%%%%%%%
%%%%%%%%%%%%%%%%%%%%%%%%%%%%%%%%%%%%%%%%%%%%%%%%%%%%%%%%%%%%%%%%%%%%%%%%%%%%%%%%
\appendix

\settowidth\MacroIndent{\rmfamily\scriptsize 000\ }

 \DocInput{childdoc.dtx}

\end{document}
%</driver>
% \fi
%
% %%%%%%%%%%%%%%%%%%%%%%%%%%%%%%%%%%%%%%%%%%%%%%%%%%%%%%%%%%%%%%%%%%%%%%%%%%%%%%
% %%%%%%%%%%%%%%%%%%%%%%%%%%%%%%%%%%%%%%%%%%%%%%%%%%%%%%%%%%%%%%%%%%%%%%%%%%%%%%
% \section{Sample}
%\iffalse
%<*samplemain>
%\fi
%
% The following presents a sample document
% with two chapters, two parts, a title page,
% a compile flag as well as three forwarding files to set the flag.
% It consists of eight |.tex| files:
% \begin{center}
% \begin{tabular}{ll}
% |cdocsamp.tex|&main file\\
% |cdocsch1.tex|&include file for chapter 1\\
% |cdocsch2.tex|&include file for chapter 2\\
% |cdocspt3.tex|&include file for part 3\\
% |cdocspt4.tex|&include file for part 4\\
% |cdocsdrf.tex|&forwarding file for main file in draft mode\\
% |cdocsfi1.tex|&forwarding file for final version of chapter 1\\
% |cdocsfi2.tex|&forwarding file for final version of chapter 2\\
% \end{tabular}
% \end{center}
% Each of the eight files can be compiled directly by the \LaTeX{} compiler.
%
% %%%%%%%%%%%%%%%%%%%%%%%%%%%%%%%%%%%%%%
% \paragraph{Main File.}
%
% The main file is called |cdocsamp.tex|.
%
% Load the \textsf{childdoc} definitions and
% declare the filename for the main document:
%    \begin{macrocode}
\input{childdoc.def}
\childdocmain{}
%    \end{macrocode}

% Optional override for |\version| flag:
%    \begin{macrocode}
%%\ifchilddoc\else\providecommand{\version}{draft}\fi
%    \end{macrocode}

% Define the default values for the |\version| flag
% (|final| for the main file and |draft| for childs):
%    \begin{macrocode}
\ifchilddoc
\providecommand{\version}{draft}
\else
\providecommand{\version}{final}
\fi
%    \end{macrocode}

% Load the standard document class:
%    \begin{macrocode}
\documentclass[12pt]{article}
%    \end{macrocode}

% Start the document body:
%    \begin{macrocode}
\begin{document}
%    \end{macrocode}

% Declare a title page.
% Print title, part of document being processed and version flag:
%    \begin{macrocode}
\addtocounter{page}{-1}
\begin{center}
{\LARGE\bfseries{}childdoc example\par}
\vspace{1cm}
\ifchilddoc
\ifchilddocmanual part\else chapter\fi:
`\childdocname' of `\childdocjob'\par
\else
main document: `\childdocjob'\par
\fi
version: \version\par
\end{center}
\newpage
%    \end{macrocode}

% Manually include selected file,
% otherwise process as usual:
%    \begin{macrocode}
\ifchilddocmanual
\section*{part `\childdocname'}
\input{\childdocname}
\else
%    \end{macrocode}

% Include the two chapters:
%    \begin{macrocode}
\include{cdocsch1}
\include{cdocsch2}
%    \end{macrocode}

% Include the two parts unless only chapters should be displayed:
%    \begin{macrocode}
\ifchilddoc\else
\section{part three}
\input{cdocspt3}
\section{part four}
\input{cdocspt4}
\fi
%    \end{macrocode}

% Process as usual until here:
%    \begin{macrocode}
\fi
%    \end{macrocode}

% End of document body:
%    \begin{macrocode}
\end{document}
%    \end{macrocode}
%\iffalse
%</samplemain>
%\fi
%
% %%%%%%%%%%%%%%%%%%%%%%%%%%%%%%%%%%%%%%
% \paragraph{Chapter Include Files.}
%
% The include files are called |cdocsch1.tex| and |cdocsch2.tex|.
%
%\iffalse
%<*samplechap1|samplechap2>
%\fi

% Optional override for |\version| flag:
%    \begin{macrocode}
%%\providecommand{\version}{final}
%    \end{macrocode}

% Include the main document:
%    \begin{macrocode}
\input{childdoc.def}
\childdocof{cdocsamp}
%    \end{macrocode}

%\iffalse
%</samplechap1|samplechap2>
%\fi
%
%\iffalse
%<*samplechap1>
%\fi
% Some text for chapter 1:
%    \begin{macrocode}
\section{one}
some text in chapter one
%    \end{macrocode}

%\iffalse
%</samplechap1>
%\fi
% Some text for chapter 2:
%\iffalse
%<*samplechap2>
%\fi
%    \begin{macrocode}
\section{two}
more text in chapter two
%    \end{macrocode}

%\iffalse
%</samplechap2>
%\fi
%
% %%%%%%%%%%%%%%%%%%%%%%%%%%%%%%%%%%%%%%
% \paragraph{Part Include Files.}
%
% The include files are called |cdocspt3.tex| and |cdocspt4.tex|.
%
%\iffalse
%<*samplepart3|samplepart4>
%\fi

% Optional override for |\version| flag:
%    \begin{macrocode}
%%\providecommand{\version}{final}
%    \end{macrocode}

% Include the main document:
%    \begin{macrocode}
\input{childdoc.def}
\childdocby{cdocsamp}
%    \end{macrocode}

%\iffalse
%</samplepart3|samplepart4>
%\fi
%
%\iffalse
%<*samplepart3>
%\fi
% Some text for part 3:
%    \begin{macrocode}
some text in part three
%    \end{macrocode}

%\iffalse
%</samplepart3>
%\fi
% Some text for part 4:
%\iffalse
%<*samplepart4>
%\fi
%    \begin{macrocode}
more text in part four
%    \end{macrocode}

%\iffalse
%</samplepart4>
%\fi
%
% %%%%%%%%%%%%%%%%%%%%%%%%%%%%%%%%%%%%%%
% \paragraph{Forwarding for a Complete Draft.}
%
% The following forwarding file |cdocsdrf.tex|
% compiles the main document in draft mode:
%\iffalse
%<*sampledraft>
%\fi
%    \begin{macrocode}
\def\version{draft}
\input{childdoc.def}
\childdocforward{cdocsamp}
%    \end{macrocode}

%\iffalse
%</sampledraft>
%\fi
%
% %%%%%%%%%%%%%%%%%%%%%%%%%%%%%%%%%%%%%%
% \paragraph{Forwarding for Final Version of the Chapters.}
%
% The following forwarding files |cdocsfn1.tex| and |cdocsfn2.tex|
% (with identical content)
% compile the final versions of the child documents
% |cdocsch1.tex| and |cdocsch2.tex|, respectively:
%\iffalse
%<*samplefinal>
%\fi
%    \begin{macrocode}
\def\version{final}
\input{childdoc.def}
\childdocforwardprefix[cdocsamp]{cdocsfn}{cdocsch}
%    \end{macrocode}

%\iffalse
%</samplefinal>
%\fi
%
% %%%%%%%%%%%%%%%%%%%%%%%%%%%%%%%%%%%%%%
% \paragraph{Command Line Processing.}
%
% The following three command lines generate the output files
% |cdocscld|, |cdocscl1| and |cdocscl2|
% which should be identical to
% |cdocsdrf|, |cdocsch1| and |cdocsfn2|, respectively:
% \begin{center}
% \begin{tabular}{l}
% |latex -jobname cdocscld \|\\
% |  "\def\version{draft}\input{childdoc.def}\childdocforward{cdocsamp}"|\\
% |latex -jobname cdocscl1 \|\\
% |  "\input{childdoc.def}\childdocforward[cdocsamp]{cdocsch1}"|\\
% |latex -jobname cdocscl2 \|\\
% |  "\def\version{final}\input{childdoc.def}\childdocforward{cdocsch2}"|
% \end{tabular}
% \end{center}
% Note that the trailing backslash on each first line
% merely continues the input to the second line
% (for convenient cut ant paste).
% Furthermore, the command |latex| can be replaced by any
% of its alternative versions such as |pdflatex|.
%
% %%%%%%%%%%%%%%%%%%%%%%%%%%%%%%%%%%%%%%%%%%%%%%%%%%%%%%%%%%%%%%%%%%%%%%%%%%%%%%
% %%%%%%%%%%%%%%%%%%%%%%%%%%%%%%%%%%%%%%%%%%%%%%%%%%%%%%%%%%%%%%%%%%%%%%%%%%%%%%
% \section{Implementation}
%\iffalse
%<*package>
%\fi
%
% This section describes the definitions file |childdoc.def|.

% The definitions cannot be loaded using |\usepackage| or |\RequirePackage|
% which has a mechanism to prevent loading a style file more than once.
% When loading the definitions by means of |\input|
% multiple instances have to be prevented manually:
%\iffalse
%This code needs to be before the `\ProvidesFile' directive
%which is defined at the beginning of this file.
%Therefore it is also placed there and commented out here.
%</package>
%<*discard>
%\fi
%    \begin{macrocode}
\ifdefined\childdocmain\endinput\fi
%    \end{macrocode}
%\iffalse
%</discard>
%<*package>
%\fi
%
% \macro{\ifchilddoc}
% \macro{\ifchilddocmanual}
% The conditional |\ifchilddoc| tells whether a
% child (true) or main (false) document is being compiled.
% The conditional |\ifchilddocmanual| tells whether
% the |\includeonly| mechanism is used (false) or
% the selection of child files must be performed manually (true).
% The definitions initialise to false:
%    \begin{macrocode}
\newif\ifchilddoc
\newif\ifchilddocmanual
%    \end{macrocode}

% \macro{\childdocname}
% \macro{\childdocjob}
% The macro |\childdocname| stores the name of the main document
% to be compiled. The macro |\childdocjob| stores the name of
% the document on which the \LaTeX{} compiler was originally invoked.
% The content of |\jobname| cannot be compared
% to filenames specified in the source due to different catcodes.
% The following code rescans |\jobname|, stores the result
% in |\childdocname| and saves a copy in |\childdocjob|:
%    \begin{macrocode}
\edef\childdocname{\scantokens\expandafter{\jobname\noexpand}}
\let\childdocjob\childdocname
%    \end{macrocode}

% \macro{\childdocdisable}
% The macro |\childdocdisable| prevents the main file
% from being processed more than once.
% At this stage, the main document command |\childdocmain|
% is assumed to be called once again where it should do nothing.
% Any subsequent call to it should prevent
% a secondary processing of the main document
% It overwrites the forwarding commands
% |\childdocof| and |\childdocforward|
% with empty macros to prevent further inclusions of the main document:
%    \begin{macrocode}
\newcommand{\childdocdisable}
{
  \renewcommand{\childdocmain}[1]{\renewcommand{\childdocmain}[1]{\endinput}}
  \renewcommand{\childdocof}[1]{}
  \renewcommand{\childdocby}[2][]{}
  \renewcommand{\childdocforward}[2][]{}
  \renewcommand{\childdocdisable}{}
}
%    \end{macrocode}

% \macro{\childdocmain}
% The macro |\childdocmain| is to be called at the top of the main file
% with nothing or the main filename (without extension) as argument.
% First, it breaks loops.
% If the argument is not empty and does not match |\childdocname|
% (which is set by the first inclusion of |childdoc.def|),
% |\ifchilddoc| is set to true, |\includeonly| is applied to the child file
% and |\jobname| is set to the main file
% (for proper handling of |.aux| files):
%    \begin{macrocode}
\newcommand{\childdocmain}[1]
{
  \childdocdisable\childdocmain{}
  \if?#1?\else
    \begingroup
      \def\childdoctmp{#1}
      \ifx\childdoctmp\childdocname
        \def\childdoctmp{}
      \else
        \def\childdoctmp
        {
          \childdoctrue
          \includeonly{\childdocname}
          \def\childdocjob{#1}
          \def\jobname{#1}
        }
      \fi
      \expandafter
    \endgroup
    \childdoctmp
  \fi
}
%    \end{macrocode}

% \macro{\childdocof}
% The command |\childdocof| redirects
% compilation to the main file |#1|.
%    \begin{macrocode}
\newcommand{\childdocof}[1]
{
  \childdocdisable
  \childdoctrue
  \includeonly{\childdocname}
  \def\jobname{#1}
  \def\childdocjob{#1}
  \input{#1}
}
%    \end{macrocode}

% \macro{\childdocby}
% The command |\childdocby| ....
%    \begin{macrocode}
\newcommand{\childdocby}[2][]
{
  \childdocdisable
  \childdoctrue
  \childdocmanualtrue
  \if?#1?\else
    \def\jobname{#2}
  \fi
  \def\childdocjob{#2}
  \input{#2}
  \endinput
}
%    \end{macrocode}

% \macro{\childdocforward}
% The command |\childdocforward| redirects
% compilation to the main file or
% (if the optional argument is given) a child file.
% Parameters are set as if the main file
% or a child file starting with |\childdocof| was compiled.
% Then compilation is handed over to the main file:
%    \begin{macrocode}
\newcommand{\childdocforward}[2][]
{
  \begingroup
    \if?#1?
      \def\childdoctmp
      {
        \def\childdocname{#2}
        \def\childdocjob{#2}
        \def\jobname{#2}
        \input{#2}
        \endinput
      }
    \else
      \def\childdoctmp
      {
        \childdocdisable
        \def\childdocname{#2}
        \childdoctrue
        \includeonly{#2}
        \def\childdocjob{#1}
        \def\jobname{#1}
        \input{#1}
        \endinput
      }
    \fi
    \expandafter
  \endgroup
  \childdoctmp
}
%    \end{macrocode}

% \macro{\childdocforwardprefix}
% The command |\childdocforwardprefix| redirects
% compilation to the main or a child file by means of a pattern.
% The prefix |#1| in the current filename is replaced by |#2|
% and the suffix of the current filename is kept
% (it is assumed that the filename does not contain the substring `|~~~|'
% which is used as a delimiter).
% Compilation is handed over to the new file by |\childdocforward|:
%    \begin{macrocode}
\newcommand{\childdocforwardprefix}[3][]
{
  \begingroup
    \def\childdocextract #2##1~~~{\def\childdoctmp{\childdocforward[#1]{#3##1}}}
    \expandafter\childdocextract\childdocname~~~
    \expandafter
  \endgroup
  \childdoctmp
}
%    \end{macrocode}

% \macro{\childdoc}
% The deprecated macro |\childdoc| is a legacy version of |\childdocmain|:
%    \begin{macrocode}
\newcommand{\childdoc}{\childdocmain}
%    \end{macrocode}

% \macro{\childdocredirect}
% The deprecated macro |\childdocredirect| is a legacy version
% of |\childdocforward| and |\childdocforwardprefix|:
%    \begin{macrocode}
\newcommand{\childdocredirect}[2][]
{
  \begingroup
    \if?#1?
      \def\childdoctmp{\childdocforward{#2}}
    \else
      \def\childdoctmp{\childdocforwardprefix{#1}{#2}}
    \fi
    \expandafter
  \endgroup
  \childdoctmp
}
%    \end{macrocode}

%\iffalse
%</package>
%\fi
%
\endinput
\childdocforward[cdocsamp]{cdocsch1}"|\\
% |latex -jobname cdocscl2 \|\\
% |  "\def\version{final}% \iffalse
%
% childdoc.dtx Copyright (C) 2017-2018 Niklas Beisert
%
% This work may be distributed and/or modified under the
% conditions of the LaTeX Project Public License, either version 1.3
% of this license or (at your option) any later version.
% The latest version of this license is in
%   http://www.latex-project.org/lppl.txt
% and version 1.3 or later is part of all distributions of LaTeX
% version 2005/12/01 or later.
%
% This work has the LPPL maintenance status `maintained'.
%
% The Current Maintainer of this work is Niklas Beisert.
%
% This work consists of the files childdoc.dtx and childdoc.ins
% and the derived files childdoc.def and cdocsamp.tex with
% cdocsch1.tex, cdocsch2.tex, cdocsdrf.tex, cdocsfn1.tex, cdocsfn2.tex.
%
%<package>\ifdefined\childdocmain\endinput\fi
%<package>\ProvidesFile{childdoc.def}[2018/12/30 v2.0 child document driver]
%<samplemain>\ProvidesFile{cdocsamp.tex}[2018/12/30 v2.0 sample for childdoc]
%<*driver>
%\ProvidesFile{childdoc.drv}[2018/12/30 v2.0 childdoc reference manual file]
\PassOptionsToClass{10pt,a4paper}{article}
\documentclass{ltxdoc}

\usepackage[margin=35mm]{geometry}
\usepackage{hyperref}
\usepackage{hyperxmp}
\usepackage[usenames]{color}

\hypersetup{colorlinks=true}
\hypersetup{pdfstartview=FitH}
\hypersetup{pdfpagemode=UseNone}
\hypersetup{pdfsource={}}
\hypersetup{pdflang={en-UK}}
\hypersetup{pdfcopyright={Copyright 2017-2018 Niklas Beisert.
  This work may be distributed and/or modified under the
  conditions of the LaTeX Project Public License, either version 1.3
  of this license or (at your option) any later version.}}
\hypersetup{pdflicenseurl={http://www.latex-project.org/lppl.txt}}
\hypersetup{pdfcontactaddress={ETH Zurich, ITP, HIT K,
  Wolfgang-Pauli-Strasse 27}}
\hypersetup{pdfcontactpostcode={8093}}
\hypersetup{pdfcontactcity={Zurich}}
\hypersetup{pdfcontactcountry={Switzerland}}
\hypersetup{pdfcontactemail={nbeisert@itp.phys.ethz.ch}}
\hypersetup{pdfcontacturl={http://people.phys.ethz.ch/\xmptilde nbeisert/}}

\newcommand{\secref}[1]{\hyperref[#1]{section \ref*{#1}}}

\parskip1ex
\parindent0pt
\let\olditemize\itemize
\def\itemize{\olditemize\parskip0pt}

\begin{document}

\title{The \textsf{childdoc} Package}
\hypersetup{pdftitle={The childdoc Package}}
\author{Niklas Beisert\\[2ex]
  Institut f\"ur Theoretische Physik\\
  Eidgen\"ossische Technische Hochschule Z\"urich\\
  Wolfgang-Pauli-Strasse 27, 8093 Z\"urich, Switzerland\\[1ex]
  \href{mailto:nbeisert@itp.phys.ethz.ch}
  {\texttt{nbeisert@itp.phys.ethz.ch}}}
\hypersetup{pdfauthor={Niklas Beisert}}
\hypersetup{pdfsubject={Manual for the LaTeX2e Package childdoc}}
\date{30 December 2018, \textsf{v2.0}}
\maketitle

\begin{abstract}\noindent
\textsf{childdoc} is a \LaTeXe{} package
that enables the direct compilation
of document sections included by |\include|
to individual files.
\end{abstract}

\begingroup
\parskip0ex
\tableofcontents
\endgroup

%%%%%%%%%%%%%%%%%%%%%%%%%%%%%%%%%%%%%%%%%%%%%%%%%%%%%%%%%%%%%%%%%%%%%%%%%%%%%%%%
%%%%%%%%%%%%%%%%%%%%%%%%%%%%%%%%%%%%%%%%%%%%%%%%%%%%%%%%%%%%%%%%%%%%%%%%%%%%%%%%
\section{Introduction}

\LaTeX{} provides a mechanism to structure a large document (such as a book)
into a main file and several child files (containing the chapters)
using the |\include| command.
This mechanism is beneficial for documents
which span hundreds of pages in order to
make the source file(s) more manageable.
Moreover, compilation can be restricted to
selected child files by means of the |\includeonly| command.
The latter feature can be used to reduce the compilation time while editing
(this was significantly more useful in the earlier days of \LaTeX{})
or to generate a smaller document which is easier to navigate.
Another application of |\includeonly| is to generate
documents consisting of selected parts of the complete document.

However, there are a few drawbacks of the plain |\include| mechanism:
\begin{itemize}
\item
The child files cannot be compiled on their own,
they can only be compiled via the main file.
A naive editing environment
(such as a text editor with an option
to have the current file processed by \LaTeX)
may require one to switch to the main file before compiling;
attempting to compile the child file produces errors.
\item
The main file must be modified (each time)
to adjust the |\includeonly| command
to the present needs. This easily leaves the main file in a messy state.
\item
The generated document will always carry the filename
of the main document. This is inconvenient if
several child files are to be compiled and
to be kept for distribution.
\end{itemize}

The present package provides a simple interface
to make child files individually compilable by \LaTeX{}.
Compiling a child file then has the same effect as compiling
the main file with an |\includeonly| command
to select the appropriate child.
Moreover the generated document will carry the name of the child
rather than the main file.
This resolves all three above issues.

This feature is meant to make the editing of books,
thesis documents and lecture notes somewhat more convenient.
However, the package can also be used efficiently for
composing a series of documents (such as exercise sheets)
which are typically distributed individually.
It then assists the author in generating the individual documents
(potentially in different versions)
as well as a document containing the collected series.
Another application is in developing style files
or other kinds of included material
where compilation of the style file could redirect
to a sample or test file.

%%%%%%%%%%%%%%%%%%%%%%%%%%%%%%%%%%%%%%%%%%%%%%%%%%%%%%%%%%%%%%%%%%%%%%%%%%%%%%%%
%%%%%%%%%%%%%%%%%%%%%%%%%%%%%%%%%%%%%%%%%%%%%%%%%%%%%%%%%%%%%%%%%%%%%%%%%%%%%%%%
\section{Usage}

First of all, the package \textsf{childdoc} is \emph{not} a standard
\LaTeXe{} |.sty| style file! Therefore it needs to be invoked in
a non-standard way.

%%%%%%%%%%%%%%%%%%%%%%%%%%%%%%%%%%%%%%%%%%%%%%%%%%%%%%%%%%%%%%%%%%%%%%%%%%%%%%%%
\subsection{Included Files}
\label{sec:include}

%%%%%%%%%%%%%%%%%%%%%%%%%%%%%%%%%%%%%%%%
\DescribeMacro{\childdocmain}
To use the package, add the commands
\begin{center}
\begin{tabular}{l}
|\input{childdoc.def}|\\
|\childdocmain{}|\\
\end{tabular}
\end{center}
at the very top of the main \LaTeX{} file,
in particular \emph{before} the |\documentclass| statement!
The argument of |\childdocmain| should be left empty
(but it must be present).

%%%%%%%%%%%%%%%%%%%%%%%%%%%%%%%%%%%%%%%%
\DescribeMacro{\childdocof}
Furthermore, add the commands
\begin{center}
\begin{tabular}{l}
|\input{childdoc.def}|\\
|\childdocof{|\textit{main}|}|\\
\end{tabular}
\end{center}
at the top of every child file \textit{child}
which is included by |\include{|\textit{child}|}|
from within the main file
(or at least for those files to be compiled individually).
The argument \textit{main} must be the filename of the main file.

There are a couple of
considerations in setting up the main and child documents:

%%%%%%%%%%%%%%%%%%%%%%%%%%%%%%%%%%%%%%%%
\paragraph{Restrictions.}

Please note the following restrictions:
\begin{itemize}
\item
|\childdocmain| must be called with one argument \textit{main}
to ensure compatibility with earlier version of the package.
It must either be empty (|\childdocmain{}|)
or precisely match the filename of the main file in which it is specified.
See \secref{sec:detection} for further information.
\item
The filename \textit{main} must be specified without the |.tex| extension.
\item
The filename \textit{main} is case sensitive
(even in case-insensitive file systems)
due to internal string comparison.
\item
The argument \textit{main} should be fully expanded, it cannot be a macro.
\item
Subdirectories and special characters should be avoided in filenames.
\item
The command |\childdocmain{|\textit{main}|}| must be followed by a whitespace.
It should not be followed immediately by another command
or by a comment mark `|%|'.
This is because the \TeX{} parser reads the token immediately following
the argument of |\childdocmain| and puts it
at the beginning of every child section;
however, a white\-space is ignored.
\end{itemize}

%%%%%%%%%%%%%%%%%%%%%%%%%%%%%%%%%%%%%%%%
\paragraph{Content of Main File.}

It is advisable to place all content in the child files included by |\include|.
Any output contained in the main file will appear in all child documents
unless suppressed manually;
it cannot be suppressed automatically by the |\includeonly| directive
and thus should normally be avoided.
A method to include some content in the main file
by means of conditional processing is described in \secref{sec:conditional}.

%%%%%%%%%%%%%%%%%%%%%%%%%%%%%%%%%%%%%%%%
\paragraph{Page Numbering.}

When only a part of the document is compiled,
the appropriate numbering of pages
(as well as other status parameters)
is determined from the |.aux| files.
The latter contain information from previous passes.
However this information needs to propagate through
all intermediate child documents.
Therefore the page numbering in child documents may well
be inconsistent until the complete document is compiled at least once.

A useful (if unconventional) way to always ensure a consistent
page numbering is to restart the numbering in each child document
and denote the pages by `\textit{child}|.|\textit{page}'
where \textit{child} represents the chapter/section number of the child file.
This can be achieved by the command
|\numberwithin{page}{|\textit{child}|}|
of the \textsf{amsmath} package
where \textit{child} can be |chapter| or |section|
depending on the chosen structuring.
Alternatively, one can modify the macro |\thepage| appropriately
and reset the counter |page| at the start of each child file.

%%%%%%%%%%%%%%%%%%%%%%%%%%%%%%%%%%%%%%%%%%%%%%%%%%%%%%%%%%%%%%%%%%%%%%%%%%%%%%%%
\subsection{Conditional Processing}
\label{sec:conditional}

The package provides a mechanism to compile different versions
of a document. To customise the versions further some conditional processing
can come in handy to distinguish which version is being compiled.
The package provides two macros to describe the compilation context:

%%%%%%%%%%%%%%%%%%%%%%%%%%%%%%%%%%%%%%%%
\DescribeMacro{\ifchilddoc}
The conditional |\ifchilddoc| distinguishes between the compilation of
child documents and the main document:
%
\begin{center}
|\ifchilddoc |\textit{child-code}| |[|\||else |\textit{main-code}]| \||fi|
\end{center}

%%%%%%%%%%%%%%%%%%%%%%%%%%%%%%%%%%%%%%%%
\DescribeMacro{\childdocname}
\DescribeMacro{\childdocjob}
The macro |\childdocname| contains the filename (without extension)
of the main or child file being processed.
Note that |\childdocjob| will always contain the name of the main file.

%%%%%%%%%%%%%%%%%%%%%%%%%%%%%%%%%%%%%%%%
\paragraph{Title Page.}

Conditional processing can be used to include a title or banner page
in the main document when proper precautions are taken.
Importantly, the code in the main file should ensure that the page counter
(as well as other status parameters which are stored in the |.aux| files)
takes the same value after the conditional processing.
Otherwise the page numbers may take divergent values
depending on which part is compiled.

For example, a title page could be declared by:
%
\begin{center}
\begin{tabular}{l}
|\ifchilddoc\||else|\\
|\addtocounter{page}{-1}|\\
\textit{code for title page}\\
|\newpage|\\
|\||fi|
\end{tabular}
\end{center}
%
A banner page for the child documents can be generated by:
%
\begin{center}
\begin{tabular}{l}
|\ifchilddoc|\\
|\addtocounter{page}{-1}|\\
\textit{code for banner page}\\
|\newpage|\\
|\||fi|
\end{tabular}
\end{center}
%
Here one could write a message such as:
\begin{center}
|This is the part \childdocname{} of \childdocjob{}.|
\end{center}

%%%%%%%%%%%%%%%%%%%%%%%%%%%%%%%%%%%%%%%%%%%%%%%%%%%%%%%%%%%%%%%%%%%%%%%%%%%%%%%%
\subsection{Flags}
\label{sec:flags}

The package makes it easy to generate different versions
of the main or child documents.
To this end compilation flags can be defined
and assigned different default values.
They will be particularly useful in conjunction
with the forwarding mechanism described in \secref{sec:forward}.

For example, it may be useful to have a flag |\version|
which can be set to |draft| or |final|.
The document source will contain some conditional code
depending on the value of |\version|.
Suppose further, the flag should default to |final| for the main file
and to |draft| for child files
which is a natural assignment for editing the document.
This is achieved by placing the following code
in the preamble of the main document
(below the |\childdocmain| directive):
%
\begin{center}
\begin{tabular}{l}
|\ifchilddoc|\\
|\providecommand{\version}{draft}|\\
|\||else|\\
|\providecommand{\version}{final}|\\
|\||fi|
\end{tabular}
\end{center}
%
The definition by |\providecommand| makes sure
that previous definitions are not overwritten.
Further statements |\providecommand{\version}{...}|
can thus be added before the above code to override it.

For the main file, one might add a line
(between |\childdocmain| and the above block)
%
\begin{center}
|%\ifchilddoc\||else\providecommand{\version}{draft}\||fi|
\end{center}
%
which can be uncommented to produce a draft version.
Likewise one can add a line to the very top of a child file
(above the |\childdocof{|\textit{main}|}| directive)
%
\begin{center}
|%\providecommand{\version}{final}|
\end{center}
%
which can be uncommented to produce the final version of this child document.

%%%%%%%%%%%%%%%%%%%%%%%%%%%%%%%%%%%%%%%%%%%%%%%%%%%%%%%%%%%%%%%%%%%%%%%%%%%%%%%%
\subsection{Forwarding}
\label{sec:forward}

Different versions of the main or child documents
using compilation flags as described in \secref{sec:flags}
can be (permanently) stored in different files
for convenient compilation, viewing and distribution.
To this end, the package defines a command
to pass on compilation to a different file:

%%%%%%%%%%%%%%%%%%%%%%%%%%%%%%%%%%%%%%%%
\DescribeMacro{\childdocforward}
The command |\childdocforward| redirects processing to
another source file:
%
\begin{center}
\begin{tabular}{l}
|\input{childdoc.def}|\\
|\childdocforward[|\textit{main}|]{|\textit{dest}|}|\\
\end{tabular}
\end{center}
%
The argument \textit{dest} is the destination file
(without extension).
It should be the main file or one of the child files.
Note that further \textsf{childdoc} directives
such as |\childdocof| and |\childdocforward|
in the indicated file will be processed in this form.
The optional argument \textit{main}
passes on directly to the main file \textit{main}
while pretending to compile the child \textit{dest}.
This form behaves as if \textit{dest}
issues |\childdocof{|\textit{main}|}| right away,
and no further \textsf{childdoc} directives will be processed.

%%%%%%%%%%%%%%%%%%%%%%%%%%%%%%%%%%%%%%%%
\DescribeMacro{\...prefix}
In the alternative form |\childdocforwardprefix|,
%
\begin{center}
\begin{tabular}{l}
|\input{childdoc.def}|\\
|\childdocforwardprefix[|\textit{main}|]{|\textit{prefix}|}{|\textit{dest}|}|
\end{tabular}
\end{center}
%
the destination file is determined by a pattern
depending on the current file:
To make this work, the current file must be called
`{\textit{prefix}\hspace{0.2em}\textit{suffix}}'
with \textit{prefix} matching precisely the argument.
Processing is then passed on to the file
`{\textit{dest}\hspace{0.2em}\textit{suffix}}'.
Surely, the same effect is achieved by
directly specifying the
argument `{\textit{dest}\hspace{0.2em}\textit{suffix}}'
in the first form.
However, that requires to set up a different file
for each child. With the alternative form of the command
all these files can have exactly the same content
which simplifies setting them up and maintaining them.

For example, the following file |draft.tex|
with a compilation flag |\version| as described in \secref{sec:flags}
compiles the main document as a draft:
%
\begin{center}
\begin{tabular}{l}
|\def\version{draft}|\\
|\input{childdoc.def}|\\
|\childdocforward{|\textit{main}|}|
\end{tabular}
\end{center}
%
Likewise, the following files |final|\textit{nn}|.tex|
compile the final version of the child document
|child|\textit{nn}|.tex|:
%
\begin{center}
\begin{tabular}{l}
|\def\version{final}|\\
|\input{childdoc.def}|\\
|\childdocforwardprefix{final}{child}|
\end{tabular}
\end{center}
%

Note that when several versions of a main file and/or of each child file
are to be generated, it may be convenient to set up a |Makefile| or
shell script to automatise the process.

%%%%%%%%%%%%%%%%%%%%%%%%%%%%%%%%%%%%%%%%%%%%%%%%%%%%%%%%%%%%%%%%%%%%%%%%%%%%%%%%
\subsection{Command Line Processing}
\label{sec:commandline}

The effect of redirection files can also be achieved by invoking
the \LaTeX{} compiler with a more elaborate command line.
Most conveniently this should be done as part
of a shell script or a |Makefile|.

When using \textsf{childdoc} in the main file, the following
command lines effectively perform a redirection
(note that depending on the shell being used,
backslashes may have to be doubled: `|\|' $\to$ `|\\|'):
%
\begin{center}
|... -jobname "|\textit{target}|" |\\|"|[\textit{flags}]%
|\input{childdoc.def}\childdocforward[|\textit{main}|]{|\textit{dest}|}"|
\end{center}
%
Here \textit{target} is the name of the output file,
\textit{main} is the name of the main file
and \textit{dest} is the name of the main or child file to be processed
(all filenames without extensions).
The optional argument \textit{main} can be omitted
if \textit{main} matches \textit{dest}.
Optionally, compilation \textit{flags} can be defined via |\def| commands.
This command line makes the \TeX{} engine believe
it is compiling the file \textit{target}
whose content is specified as the latter parameter.
The provided code then forwards the processing to
\textit{main} or \textit{dest} as described in \secref{sec:forward}.

%%%%%%%%%%%%%%%%%%%%%%%%%%%%%%%%%%%%%%%%%%%%%%%%%%%%%%%%%%%%%%%%%%%%%%%%%%%%%%%%
\subsection{Include by Input}
\label{sec:input}

Including child documents by |\include| has some restrictions by design.
Most notably, the content of a child document always occupies
its own set of pages; pages cannot be shared between child documents.
Usually, this behaviour makes perfect sense
because each child document contain an essential part of the document.
However, in some situations it may be desirable to compose
a document from a collection of parts
without having mandatory page breaks between then.
For this case, the package
provides a mechanism to include parts
by |\input| which can also be processed individually.
However, by construction this mechanism
requires manual handling of the content to be output.

%%%%%%%%%%%%%%%%%%%%%%%%%%%%%%%%%%%%%%%%
\DescribeMacro{\ifchilddocmanual}
The main file should be prepared as usual, see \secref{sec:include}.
However, the document body must make a distinction
between processing of an individual part and of the main document, e.g.:
%
\begin{center}
\begin{tabular}{l}
|\ifchilddocmanual|\\
|\input{\childdocname}|\\
|\||else|\\
\textit{document body with }|\input{|\textit{part}|}|\\
|\||fi|
\end{tabular}
\end{center}
%
The conditional |\ifchilddocmanual| is true whenever
a part to be included by |\input| is being compiled,
and the name of the part is stored in |\childdocname|.

%%%%%%%%%%%%%%%%%%%%%%%%%%%%%%%%%%%%%%%%
\DescribeMacro{\childdocby}
Each part to be included by |\input| should start with:
%
\begin{center}
\begin{tabular}{l}
|\input{childdoc.def}|\\
|\childdocby{|\textit{main}|}|\\
\end{tabular}
\end{center}
%
The directive |\childdocby| is similar to |\childdocof|
described in \secref{sec:include},
but the subsequent selection of content must be done manually.
To that end, both |\ifchilddoc| and |\ifchilddocmanual|
will be true upon processing of a part,
and the name of the part is stored in |\childdocname|.
Note that |\jobname| will be set to the filename of the current part
so that each part receives an individual |.aux| file
that does not interfere with the |.aux| file(s) of the main document.
This behaviour can be altered by the alternative form
|\childdocby[*]{|\textit{main}|}| (with a non-empty optional argument)
which uses the |.aux| file of the main document
by setting |\jobname| to \textit{main}.

%%%%%%%%%%%%%%%%%%%%%%%%%%%%%%%%%%%%%%%%%%%%%%%%%%%%%%%%%%%%%%%%%%%%%%%%%%%%%%%%
\subsection{Driver Development}
\label{sec:driver}

The \textsf{childdoc} mechanism can also be use for the development
of definition files such as \LaTeX{} styles or classes.
This case differs from the above setup with multiple parts
included by |\include| in that no |\includeonly| should be invoked.
This can be achieved by starting the include file
(before |\ProvidesPackage|) with:
%
\begin{center}
\begin{tabular}{l}
|\input{childdoc.def}|\\
|\childdocforward{|\textit{main}|}|\\
\end{tabular}
\end{center}
%
or alternatively with:
%
\begin{center}
\begin{tabular}{l}
|\input{childdoc.def}|\\
|\childdocby{|\textit{main}|}|\\
\end{tabular}
\end{center}
%
Both forms have slightly different effects as described above.
The main file is prepared as usual, see \secref{sec:include}.

%%%%%%%%%%%%%%%%%%%%%%%%%%%%%%%%%%%%%%%%%%%%%%%%%%%%%%%%%%%%%%%%%%%%%%%%%%%%%%%%
\subsection{Legacy Detection}
\label{sec:detection}

The directive |\childdocmain| in the main file can detect
whether the complete document or merely a child is to be compiled
even without using the directive |\childdocof|.
This method is deprecated because it is less robust
and there is no compelling reason to use it;
it is merely provided for backward compatibility
and it may be removed in future versions.

If the detection mechanism is to be used,
it is mandatory to correctly specify
the filename of the main file as the argument of |\childdocmain|:
%
\begin{center}
\begin{tabular}{l}
|\input{childdoc.def}|\\
|\childdocmain{|\textit{main}|}|\\
\end{tabular}
\end{center}
%
If |\jobname| does not match the argument \textit{main} of |\childdocmain|,
it is assumed that |\jobname| points to the child file to be compiled.
When using |\childdocmain| with the main file specified as argument,
it suffices to start a child file
with just |\input{|\textit{main}|}|
without loading of the package and using |\childdocof|.
If instead all processing is done
with the appropriate \textsf{childdoc} directives,
the argument of \textit{main} of |\childdocmain| can be empty.

An alternative version of the command line processing described
in \secref{sec:commandline} using the detection mechanism reads:
%
\begin{center}
|... -jobname "|\textit{target}|" "|[\textit{flags}]%
[|\def\jobname{|\textit{dest}|}|]|\input{|\textit{main}|}"|
\end{center}

%%%%%%%%%%%%%%%%%%%%%%%%%%%%%%%%%%%%%%%%%%%%%%%%%%%%%%%%%%%%%%%%%%%%%%%%%%%%%%%%
\subsection{Manual Code}
\label{sec:manual}

In case one cannot be certain whether the definitions file |childdoc.def|
is installed on the target \TeX{} distribution
and one prefers not to ship it,
it is conceivable to paste a few relevant commands into the sources.

To that end, drop all statements |\input{childdoc.def}|
and perform the replacements as outlined below.
Instead of |\childdocmain{|\textit{main}|}| add the following code
to the top of the main file:
%
\begin{center}
\begin{tabular}{l}
|\||ifdefined\childdocname\endinput\||fi\newif\ifchilddoc|\\
|\edef\childdocname{\scantokens\expandafter{\jobname\noexpand}}|\\
|\def\childdocmain{|\textit{main}|}\||ifx\childdocmain\childdocname\||else|\\
|\childdoctrue\includeonly{\childdocname}\let\jobname\childdocmain\||fi|\\
\end{tabular}
\end{center}
%
Instead of |\childdocof{|\textit{main}|}| just include the main file
at the top of each child file:
%
\begin{center}
|\input{|\textit{main}|}|
\end{center}
%
A simple redirection |\childdocforward{|\textit{dest}|}| is achieved by:
%
\begin{center}
|\def\jobname{|\textit{dest}|}\input{\jobname}|
\end{center}
%
The redirection with prefix
|\childdocforwardprefix[|\textit{prefix}|]{|\textit{dest}|}|
is accomplished by:
%
\begin{center}
\begin{tabular}{l}
|{\edef\jobname{\scantokens\expandafter{\jobname\noexpand}}|\\
|\def\redirectjob |\textit{prefix}|#1~~~{\gdef\jobname{|\textit{dest}|#1}}|\\
|\expandafter\redirectjob\jobname~~~}\input{\jobname}|
\end{tabular}
\end{center}

In an alternative approach,
child documents can be compiled by a specific command line
without additional code or specific definitions:
%
\begin{center}
|... -jobname "|\textit{target}|" "|[\textit{flags}]%
|\includeonly{|\textit{dest}|}\input{|\textit{main}|}"|
\end{center}
%

%%%%%%%%%%%%%%%%%%%%%%%%%%%%%%%%%%%%%%%%%%%%%%%%%%%%%%%%%%%%%%%%%%%%%%%%%%%%%%%%
%%%%%%%%%%%%%%%%%%%%%%%%%%%%%%%%%%%%%%%%%%%%%%%%%%%%%%%%%%%%%%%%%%%%%%%%%%%%%%%%
\section{Information}

%%%%%%%%%%%%%%%%%%%%%%%%%%%%%%%%%%%%%%%%%%%%%%%%%%%%%%%%%%%%%%%%%%%%%%%%%%%%%%%%
\subsection{Copyright}

Copyright \copyright{} 2017--2018 Niklas Beisert

This work may be distributed and/or modified under the
conditions of the \LaTeX{} Project Public License, either version 1.3
of this license or (at your option) any later version.
The latest version of this license is in
  \url{http://www.latex-project.org/lppl.txt}
and version 1.3 or later is part of all distributions of \LaTeX{}
version 2005/12/01 or later.

This work has the LPPL maintenance status `maintained'.

The Current Maintainer of this work is Niklas Beisert.

This work consists of the files |README.txt|, |childdoc.ins| and |childdoc.dtx|
as well as the derived files |childdoc.def|, |cdocsamp.tex|
with |cdocsch1.tex|, |cdocsch2.tex|, |cdocspt3.tex|, |cdocspt4.tex|,
|cdocsdrf.tex|, |cdocsfn1.tex|, |cdocsfn2.tex|
as well as |childdoc.pdf|.

%%%%%%%%%%%%%%%%%%%%%%%%%%%%%%%%%%%%%%%%%%%%%%%%%%%%%%%%%%%%%%%%%%%%%%%%%%%%%%%%
\subsection{Files and Installation}

The package consists of the files:
%
\begin{center}
\begin{tabular}{ll}
    |README.txt|   & readme file \\
    |childdoc.ins| & installation file \\
    |childdoc.dtx| & source file \\
    |childdoc.def| & definition file \\
    |cdocsamp.tex| & sample main file \\
    |cdocsch1.tex| & sample include file \\
    |cdocsch2.tex| & sample include file \\
    |cdocspt3.tex| & sample part file \\
    |cdocspt4.tex| & sample part file \\
    |cdocsdrf.tex| & sample redirection file \\
    |cdocsfn1.tex| & sample redirection file \\
    |cdocsfn2.tex| & sample redirection file \\
    |childdoc.pdf| & manual
\end{tabular}
\end{center}
%
The distribution consists of the files
|README.txt|, |childdoc.ins| and |childdoc.dtx|.
%
\begin{itemize}
\item
Run (pdf)\LaTeX{} on |childdoc.dtx|
to compile the manual |childdoc.pdf| (this file).
\item
Run \LaTeX{} on |childdoc.ins| to create the definitions file |childdoc.def|
and the sample |cdocsamp.tex| with include files
|cdocsch1.tex|, |cdocsch2.tex|, |cdocspt3.tex|, |cdocspt4.tex|,
|cdocsdrf.tex|, |cdocsfn1.tex|, |cdocsfn2.tex|.
Then copy the file |childdoc.def| to an appropriate directory of your \LaTeX{}
distribution, e.g.\ \textit{texmf-root}|/tex/latex/childdoc|.
\end{itemize}

%%%%%%%%%%%%%%%%%%%%%%%%%%%%%%%%%%%%%%%%%%%%%%%%%%%%%%%%%%%%%%%%%%%%%%%%%%%%%%%%
\subsection{Related CTAN Packages}

There are several other packages which offer a similar functionality:
%
\begin{itemize}
\item
The packages
\href{http://ctan.org/pkg/docmute}{\textsf{docmute}},
\href{http://ctan.org/pkg/includex}{\textsf{includex}} and
\href{http://ctan.org/pkg/standalone}{\textsf{standalone}}
provide commands to include only the document body of
a child file thus allowing both files to be compiled individually.
\item
The packages \href{http://ctan.org/pkg/subdocs}{\textsf{subdocs}}
and \href{http://ctan.org/pkg/subfiles}{\textsf{subfiles}}
provide structures in which the main and child documents can be
encapsulated and allowing them to be compiled individually.
The inclusion mechanism is different from the conventional |\include|.
\item
The package \href{http://ctan.org/pkg/combine}{\textsf{combine}}
is an elaborate solution to combine several documents into one.
\end{itemize}
%
See also the CTAN topic \href{http://ctan.org/topic/subdocs}{\textsf{subdocs}}
for further related packages.
The present package differs from the above solutions in that
a document structure constructed with the conventional |\include| mechanism
just needs two extra commands at the top of every file
such that all constituent files can be compiled individually.

%%%%%%%%%%%%%%%%%%%%%%%%%%%%%%%%%%%%%%%%%%%%%%%%%%%%%%%%%%%%%%%%%%%%%%%%%%%%%%%%
%\subsection{Feature Suggestions}
%
%The following is a list of features which may be useful for future
%versions of this package:
%%
%\begin{itemize}
%\item
%\ldots
%\end{itemize}

%%%%%%%%%%%%%%%%%%%%%%%%%%%%%%%%%%%%%%%%%%%%%%%%%%%%%%%%%%%%%%%%%%%%%%%%%%%%%%%%
\subsection{Revision History}

%%%%%%%%%%%%%%%%%%%%%%%%%%%%%%%%%%%%%%%%
\paragraph{v2.0:} 2018/12/30

\begin{itemize}
\item
immediate forward processing
\item
added |\childdocby| mechanism
\item
manual restructured
\end{itemize}

%%%%%%%%%%%%%%%%%%%%%%%%%%%%%%%%%%%%%%%%
\paragraph{v1.6:} 2018/01/17

\begin{itemize}
\item
application for development of include files
\item
corrections to manual
\end{itemize}

%%%%%%%%%%%%%%%%%%%%%%%%%%%%%%%%%%%%%%%%
\paragraph{v1.5:} 2017/05/21

\begin{itemize}
\item
more complete structuring introduced
\item
|\childdocof| introduced
\item
|\childdoc| renamed to |\childdocmain|
\item
|\childredirect| renamed to |\childdocforward| and |\childdocforwardprefix|
and functionality expanded
\end{itemize}

%%%%%%%%%%%%%%%%%%%%%%%%%%%%%%%%%%%%%%%%
\paragraph{v1.0:} 2017/04/27

\begin{itemize}
\item
manual and install package
\item
first version published on CTAN
\end{itemize}

%%%%%%%%%%%%%%%%%%%%%%%%%%%%%%%%%%%%%%%%
\paragraph{v0.6:} 2017/04/26

\begin{itemize}
\item
redirection mechanism added
\end{itemize}

%%%%%%%%%%%%%%%%%%%%%%%%%%%%%%%%%%%%%%%%
\paragraph{v0.5:} 2017/04/26

\begin{itemize}
\item
functionality in definition file
\end{itemize}


%%%%%%%%%%%%%%%%%%%%%%%%%%%%%%%%%%%%%%%%%%%%%%%%%%%%%%%%%%%%%%%%%%%%%%%%%%%%%%%%
%%%%%%%%%%%%%%%%%%%%%%%%%%%%%%%%%%%%%%%%%%%%%%%%%%%%%%%%%%%%%%%%%%%%%%%%%%%%%%%%
%%%%%%%%%%%%%%%%%%%%%%%%%%%%%%%%%%%%%%%%%%%%%%%%%%%%%%%%%%%%%%%%%%%%%%%%%%%%%%%%
\appendix

\settowidth\MacroIndent{\rmfamily\scriptsize 000\ }

 \DocInput{childdoc.dtx}

\end{document}
%</driver>
% \fi
%
% %%%%%%%%%%%%%%%%%%%%%%%%%%%%%%%%%%%%%%%%%%%%%%%%%%%%%%%%%%%%%%%%%%%%%%%%%%%%%%
% %%%%%%%%%%%%%%%%%%%%%%%%%%%%%%%%%%%%%%%%%%%%%%%%%%%%%%%%%%%%%%%%%%%%%%%%%%%%%%
% \section{Sample}
%\iffalse
%<*samplemain>
%\fi
%
% The following presents a sample document
% with two chapters, two parts, a title page,
% a compile flag as well as three forwarding files to set the flag.
% It consists of eight |.tex| files:
% \begin{center}
% \begin{tabular}{ll}
% |cdocsamp.tex|&main file\\
% |cdocsch1.tex|&include file for chapter 1\\
% |cdocsch2.tex|&include file for chapter 2\\
% |cdocspt3.tex|&include file for part 3\\
% |cdocspt4.tex|&include file for part 4\\
% |cdocsdrf.tex|&forwarding file for main file in draft mode\\
% |cdocsfi1.tex|&forwarding file for final version of chapter 1\\
% |cdocsfi2.tex|&forwarding file for final version of chapter 2\\
% \end{tabular}
% \end{center}
% Each of the eight files can be compiled directly by the \LaTeX{} compiler.
%
% %%%%%%%%%%%%%%%%%%%%%%%%%%%%%%%%%%%%%%
% \paragraph{Main File.}
%
% The main file is called |cdocsamp.tex|.
%
% Load the \textsf{childdoc} definitions and
% declare the filename for the main document:
%    \begin{macrocode}
\input{childdoc.def}
\childdocmain{}
%    \end{macrocode}

% Optional override for |\version| flag:
%    \begin{macrocode}
%%\ifchilddoc\else\providecommand{\version}{draft}\fi
%    \end{macrocode}

% Define the default values for the |\version| flag
% (|final| for the main file and |draft| for childs):
%    \begin{macrocode}
\ifchilddoc
\providecommand{\version}{draft}
\else
\providecommand{\version}{final}
\fi
%    \end{macrocode}

% Load the standard document class:
%    \begin{macrocode}
\documentclass[12pt]{article}
%    \end{macrocode}

% Start the document body:
%    \begin{macrocode}
\begin{document}
%    \end{macrocode}

% Declare a title page.
% Print title, part of document being processed and version flag:
%    \begin{macrocode}
\addtocounter{page}{-1}
\begin{center}
{\LARGE\bfseries{}childdoc example\par}
\vspace{1cm}
\ifchilddoc
\ifchilddocmanual part\else chapter\fi:
`\childdocname' of `\childdocjob'\par
\else
main document: `\childdocjob'\par
\fi
version: \version\par
\end{center}
\newpage
%    \end{macrocode}

% Manually include selected file,
% otherwise process as usual:
%    \begin{macrocode}
\ifchilddocmanual
\section*{part `\childdocname'}
\input{\childdocname}
\else
%    \end{macrocode}

% Include the two chapters:
%    \begin{macrocode}
\include{cdocsch1}
\include{cdocsch2}
%    \end{macrocode}

% Include the two parts unless only chapters should be displayed:
%    \begin{macrocode}
\ifchilddoc\else
\section{part three}
\input{cdocspt3}
\section{part four}
\input{cdocspt4}
\fi
%    \end{macrocode}

% Process as usual until here:
%    \begin{macrocode}
\fi
%    \end{macrocode}

% End of document body:
%    \begin{macrocode}
\end{document}
%    \end{macrocode}
%\iffalse
%</samplemain>
%\fi
%
% %%%%%%%%%%%%%%%%%%%%%%%%%%%%%%%%%%%%%%
% \paragraph{Chapter Include Files.}
%
% The include files are called |cdocsch1.tex| and |cdocsch2.tex|.
%
%\iffalse
%<*samplechap1|samplechap2>
%\fi

% Optional override for |\version| flag:
%    \begin{macrocode}
%%\providecommand{\version}{final}
%    \end{macrocode}

% Include the main document:
%    \begin{macrocode}
\input{childdoc.def}
\childdocof{cdocsamp}
%    \end{macrocode}

%\iffalse
%</samplechap1|samplechap2>
%\fi
%
%\iffalse
%<*samplechap1>
%\fi
% Some text for chapter 1:
%    \begin{macrocode}
\section{one}
some text in chapter one
%    \end{macrocode}

%\iffalse
%</samplechap1>
%\fi
% Some text for chapter 2:
%\iffalse
%<*samplechap2>
%\fi
%    \begin{macrocode}
\section{two}
more text in chapter two
%    \end{macrocode}

%\iffalse
%</samplechap2>
%\fi
%
% %%%%%%%%%%%%%%%%%%%%%%%%%%%%%%%%%%%%%%
% \paragraph{Part Include Files.}
%
% The include files are called |cdocspt3.tex| and |cdocspt4.tex|.
%
%\iffalse
%<*samplepart3|samplepart4>
%\fi

% Optional override for |\version| flag:
%    \begin{macrocode}
%%\providecommand{\version}{final}
%    \end{macrocode}

% Include the main document:
%    \begin{macrocode}
\input{childdoc.def}
\childdocby{cdocsamp}
%    \end{macrocode}

%\iffalse
%</samplepart3|samplepart4>
%\fi
%
%\iffalse
%<*samplepart3>
%\fi
% Some text for part 3:
%    \begin{macrocode}
some text in part three
%    \end{macrocode}

%\iffalse
%</samplepart3>
%\fi
% Some text for part 4:
%\iffalse
%<*samplepart4>
%\fi
%    \begin{macrocode}
more text in part four
%    \end{macrocode}

%\iffalse
%</samplepart4>
%\fi
%
% %%%%%%%%%%%%%%%%%%%%%%%%%%%%%%%%%%%%%%
% \paragraph{Forwarding for a Complete Draft.}
%
% The following forwarding file |cdocsdrf.tex|
% compiles the main document in draft mode:
%\iffalse
%<*sampledraft>
%\fi
%    \begin{macrocode}
\def\version{draft}
\input{childdoc.def}
\childdocforward{cdocsamp}
%    \end{macrocode}

%\iffalse
%</sampledraft>
%\fi
%
% %%%%%%%%%%%%%%%%%%%%%%%%%%%%%%%%%%%%%%
% \paragraph{Forwarding for Final Version of the Chapters.}
%
% The following forwarding files |cdocsfn1.tex| and |cdocsfn2.tex|
% (with identical content)
% compile the final versions of the child documents
% |cdocsch1.tex| and |cdocsch2.tex|, respectively:
%\iffalse
%<*samplefinal>
%\fi
%    \begin{macrocode}
\def\version{final}
\input{childdoc.def}
\childdocforwardprefix[cdocsamp]{cdocsfn}{cdocsch}
%    \end{macrocode}

%\iffalse
%</samplefinal>
%\fi
%
% %%%%%%%%%%%%%%%%%%%%%%%%%%%%%%%%%%%%%%
% \paragraph{Command Line Processing.}
%
% The following three command lines generate the output files
% |cdocscld|, |cdocscl1| and |cdocscl2|
% which should be identical to
% |cdocsdrf|, |cdocsch1| and |cdocsfn2|, respectively:
% \begin{center}
% \begin{tabular}{l}
% |latex -jobname cdocscld \|\\
% |  "\def\version{draft}\input{childdoc.def}\childdocforward{cdocsamp}"|\\
% |latex -jobname cdocscl1 \|\\
% |  "\input{childdoc.def}\childdocforward[cdocsamp]{cdocsch1}"|\\
% |latex -jobname cdocscl2 \|\\
% |  "\def\version{final}\input{childdoc.def}\childdocforward{cdocsch2}"|
% \end{tabular}
% \end{center}
% Note that the trailing backslash on each first line
% merely continues the input to the second line
% (for convenient cut ant paste).
% Furthermore, the command |latex| can be replaced by any
% of its alternative versions such as |pdflatex|.
%
% %%%%%%%%%%%%%%%%%%%%%%%%%%%%%%%%%%%%%%%%%%%%%%%%%%%%%%%%%%%%%%%%%%%%%%%%%%%%%%
% %%%%%%%%%%%%%%%%%%%%%%%%%%%%%%%%%%%%%%%%%%%%%%%%%%%%%%%%%%%%%%%%%%%%%%%%%%%%%%
% \section{Implementation}
%\iffalse
%<*package>
%\fi
%
% This section describes the definitions file |childdoc.def|.

% The definitions cannot be loaded using |\usepackage| or |\RequirePackage|
% which has a mechanism to prevent loading a style file more than once.
% When loading the definitions by means of |\input|
% multiple instances have to be prevented manually:
%\iffalse
%This code needs to be before the `\ProvidesFile' directive
%which is defined at the beginning of this file.
%Therefore it is also placed there and commented out here.
%</package>
%<*discard>
%\fi
%    \begin{macrocode}
\ifdefined\childdocmain\endinput\fi
%    \end{macrocode}
%\iffalse
%</discard>
%<*package>
%\fi
%
% \macro{\ifchilddoc}
% \macro{\ifchilddocmanual}
% The conditional |\ifchilddoc| tells whether a
% child (true) or main (false) document is being compiled.
% The conditional |\ifchilddocmanual| tells whether
% the |\includeonly| mechanism is used (false) or
% the selection of child files must be performed manually (true).
% The definitions initialise to false:
%    \begin{macrocode}
\newif\ifchilddoc
\newif\ifchilddocmanual
%    \end{macrocode}

% \macro{\childdocname}
% \macro{\childdocjob}
% The macro |\childdocname| stores the name of the main document
% to be compiled. The macro |\childdocjob| stores the name of
% the document on which the \LaTeX{} compiler was originally invoked.
% The content of |\jobname| cannot be compared
% to filenames specified in the source due to different catcodes.
% The following code rescans |\jobname|, stores the result
% in |\childdocname| and saves a copy in |\childdocjob|:
%    \begin{macrocode}
\edef\childdocname{\scantokens\expandafter{\jobname\noexpand}}
\let\childdocjob\childdocname
%    \end{macrocode}

% \macro{\childdocdisable}
% The macro |\childdocdisable| prevents the main file
% from being processed more than once.
% At this stage, the main document command |\childdocmain|
% is assumed to be called once again where it should do nothing.
% Any subsequent call to it should prevent
% a secondary processing of the main document
% It overwrites the forwarding commands
% |\childdocof| and |\childdocforward|
% with empty macros to prevent further inclusions of the main document:
%    \begin{macrocode}
\newcommand{\childdocdisable}
{
  \renewcommand{\childdocmain}[1]{\renewcommand{\childdocmain}[1]{\endinput}}
  \renewcommand{\childdocof}[1]{}
  \renewcommand{\childdocby}[2][]{}
  \renewcommand{\childdocforward}[2][]{}
  \renewcommand{\childdocdisable}{}
}
%    \end{macrocode}

% \macro{\childdocmain}
% The macro |\childdocmain| is to be called at the top of the main file
% with nothing or the main filename (without extension) as argument.
% First, it breaks loops.
% If the argument is not empty and does not match |\childdocname|
% (which is set by the first inclusion of |childdoc.def|),
% |\ifchilddoc| is set to true, |\includeonly| is applied to the child file
% and |\jobname| is set to the main file
% (for proper handling of |.aux| files):
%    \begin{macrocode}
\newcommand{\childdocmain}[1]
{
  \childdocdisable\childdocmain{}
  \if?#1?\else
    \begingroup
      \def\childdoctmp{#1}
      \ifx\childdoctmp\childdocname
        \def\childdoctmp{}
      \else
        \def\childdoctmp
        {
          \childdoctrue
          \includeonly{\childdocname}
          \def\childdocjob{#1}
          \def\jobname{#1}
        }
      \fi
      \expandafter
    \endgroup
    \childdoctmp
  \fi
}
%    \end{macrocode}

% \macro{\childdocof}
% The command |\childdocof| redirects
% compilation to the main file |#1|.
%    \begin{macrocode}
\newcommand{\childdocof}[1]
{
  \childdocdisable
  \childdoctrue
  \includeonly{\childdocname}
  \def\jobname{#1}
  \def\childdocjob{#1}
  \input{#1}
}
%    \end{macrocode}

% \macro{\childdocby}
% The command |\childdocby| ....
%    \begin{macrocode}
\newcommand{\childdocby}[2][]
{
  \childdocdisable
  \childdoctrue
  \childdocmanualtrue
  \if?#1?\else
    \def\jobname{#2}
  \fi
  \def\childdocjob{#2}
  \input{#2}
  \endinput
}
%    \end{macrocode}

% \macro{\childdocforward}
% The command |\childdocforward| redirects
% compilation to the main file or
% (if the optional argument is given) a child file.
% Parameters are set as if the main file
% or a child file starting with |\childdocof| was compiled.
% Then compilation is handed over to the main file:
%    \begin{macrocode}
\newcommand{\childdocforward}[2][]
{
  \begingroup
    \if?#1?
      \def\childdoctmp
      {
        \def\childdocname{#2}
        \def\childdocjob{#2}
        \def\jobname{#2}
        \input{#2}
        \endinput
      }
    \else
      \def\childdoctmp
      {
        \childdocdisable
        \def\childdocname{#2}
        \childdoctrue
        \includeonly{#2}
        \def\childdocjob{#1}
        \def\jobname{#1}
        \input{#1}
        \endinput
      }
    \fi
    \expandafter
  \endgroup
  \childdoctmp
}
%    \end{macrocode}

% \macro{\childdocforwardprefix}
% The command |\childdocforwardprefix| redirects
% compilation to the main or a child file by means of a pattern.
% The prefix |#1| in the current filename is replaced by |#2|
% and the suffix of the current filename is kept
% (it is assumed that the filename does not contain the substring `|~~~|'
% which is used as a delimiter).
% Compilation is handed over to the new file by |\childdocforward|:
%    \begin{macrocode}
\newcommand{\childdocforwardprefix}[3][]
{
  \begingroup
    \def\childdocextract #2##1~~~{\def\childdoctmp{\childdocforward[#1]{#3##1}}}
    \expandafter\childdocextract\childdocname~~~
    \expandafter
  \endgroup
  \childdoctmp
}
%    \end{macrocode}

% \macro{\childdoc}
% The deprecated macro |\childdoc| is a legacy version of |\childdocmain|:
%    \begin{macrocode}
\newcommand{\childdoc}{\childdocmain}
%    \end{macrocode}

% \macro{\childdocredirect}
% The deprecated macro |\childdocredirect| is a legacy version
% of |\childdocforward| and |\childdocforwardprefix|:
%    \begin{macrocode}
\newcommand{\childdocredirect}[2][]
{
  \begingroup
    \if?#1?
      \def\childdoctmp{\childdocforward{#2}}
    \else
      \def\childdoctmp{\childdocforwardprefix{#1}{#2}}
    \fi
    \expandafter
  \endgroup
  \childdoctmp
}
%    \end{macrocode}

%\iffalse
%</package>
%\fi
%
\endinput
\childdocforward{cdocsch2}"|
% \end{tabular}
% \end{center}
% Note that the trailing backslash on each first line
% merely continues the input to the second line
% (for convenient cut ant paste).
% Furthermore, the command |latex| can be replaced by any
% of its alternative versions such as |pdflatex|.
%
% %%%%%%%%%%%%%%%%%%%%%%%%%%%%%%%%%%%%%%%%%%%%%%%%%%%%%%%%%%%%%%%%%%%%%%%%%%%%%%
% %%%%%%%%%%%%%%%%%%%%%%%%%%%%%%%%%%%%%%%%%%%%%%%%%%%%%%%%%%%%%%%%%%%%%%%%%%%%%%
% \section{Implementation}
%\iffalse
%<*package>
%\fi
%
% This section describes the definitions file |childdoc.def|.

% The definitions cannot be loaded using |\usepackage| or |\RequirePackage|
% which has a mechanism to prevent loading a style file more than once.
% When loading the definitions by means of |\input|
% multiple instances have to be prevented manually:
%\iffalse
%This code needs to be before the `\ProvidesFile' directive
%which is defined at the beginning of this file.
%Therefore it is also placed there and commented out here.
%</package>
%<*discard>
%\fi
%    \begin{macrocode}
\ifdefined\childdocmain\endinput\fi
%    \end{macrocode}
%\iffalse
%</discard>
%<*package>
%\fi
%
% \macro{\ifchilddoc}
% \macro{\ifchilddocmanual}
% The conditional |\ifchilddoc| tells whether a
% child (true) or main (false) document is being compiled.
% The conditional |\ifchilddocmanual| tells whether
% the |\includeonly| mechanism is used (false) or
% the selection of child files must be performed manually (true).
% The definitions initialise to false:
%    \begin{macrocode}
\newif\ifchilddoc
\newif\ifchilddocmanual
%    \end{macrocode}

% \macro{\childdocname}
% \macro{\childdocjob}
% The macro |\childdocname| stores the name of the main document
% to be compiled. The macro |\childdocjob| stores the name of
% the document on which the \LaTeX{} compiler was originally invoked.
% The content of |\jobname| cannot be compared
% to filenames specified in the source due to different catcodes.
% The following code rescans |\jobname|, stores the result
% in |\childdocname| and saves a copy in |\childdocjob|:
%    \begin{macrocode}
\edef\childdocname{\scantokens\expandafter{\jobname\noexpand}}
\let\childdocjob\childdocname
%    \end{macrocode}

% \macro{\childdocdisable}
% The macro |\childdocdisable| prevents the main file
% from being processed more than once.
% At this stage, the main document command |\childdocmain|
% is assumed to be called once again where it should do nothing.
% Any subsequent call to it should prevent
% a secondary processing of the main document
% It overwrites the forwarding commands
% |\childdocof| and |\childdocforward|
% with empty macros to prevent further inclusions of the main document:
%    \begin{macrocode}
\newcommand{\childdocdisable}
{
  \renewcommand{\childdocmain}[1]{\renewcommand{\childdocmain}[1]{\endinput}}
  \renewcommand{\childdocof}[1]{}
  \renewcommand{\childdocby}[2][]{}
  \renewcommand{\childdocforward}[2][]{}
  \renewcommand{\childdocdisable}{}
}
%    \end{macrocode}

% \macro{\childdocmain}
% The macro |\childdocmain| is to be called at the top of the main file
% with nothing or the main filename (without extension) as argument.
% First, it breaks loops.
% If the argument is not empty and does not match |\childdocname|
% (which is set by the first inclusion of |childdoc.def|),
% |\ifchilddoc| is set to true, |\includeonly| is applied to the child file
% and |\jobname| is set to the main file
% (for proper handling of |.aux| files):
%    \begin{macrocode}
\newcommand{\childdocmain}[1]
{
  \childdocdisable\childdocmain{}
  \if?#1?\else
    \begingroup
      \def\childdoctmp{#1}
      \ifx\childdoctmp\childdocname
        \def\childdoctmp{}
      \else
        \def\childdoctmp
        {
          \childdoctrue
          \includeonly{\childdocname}
          \def\childdocjob{#1}
          \def\jobname{#1}
        }
      \fi
      \expandafter
    \endgroup
    \childdoctmp
  \fi
}
%    \end{macrocode}

% \macro{\childdocof}
% The command |\childdocof| redirects
% compilation to the main file |#1|.
%    \begin{macrocode}
\newcommand{\childdocof}[1]
{
  \childdocdisable
  \childdoctrue
  \includeonly{\childdocname}
  \def\jobname{#1}
  \def\childdocjob{#1}
  \input{#1}
}
%    \end{macrocode}

% \macro{\childdocby}
% The command |\childdocby| ....
%    \begin{macrocode}
\newcommand{\childdocby}[2][]
{
  \childdocdisable
  \childdoctrue
  \childdocmanualtrue
  \if?#1?\else
    \def\jobname{#2}
  \fi
  \def\childdocjob{#2}
  \input{#2}
  \endinput
}
%    \end{macrocode}

% \macro{\childdocforward}
% The command |\childdocforward| redirects
% compilation to the main file or
% (if the optional argument is given) a child file.
% Parameters are set as if the main file
% or a child file starting with |\childdocof| was compiled.
% Then compilation is handed over to the main file:
%    \begin{macrocode}
\newcommand{\childdocforward}[2][]
{
  \begingroup
    \if?#1?
      \def\childdoctmp
      {
        \def\childdocname{#2}
        \def\childdocjob{#2}
        \def\jobname{#2}
        \input{#2}
        \endinput
      }
    \else
      \def\childdoctmp
      {
        \childdocdisable
        \def\childdocname{#2}
        \childdoctrue
        \includeonly{#2}
        \def\childdocjob{#1}
        \def\jobname{#1}
        \input{#1}
        \endinput
      }
    \fi
    \expandafter
  \endgroup
  \childdoctmp
}
%    \end{macrocode}

% \macro{\childdocforwardprefix}
% The command |\childdocforwardprefix| redirects
% compilation to the main or a child file by means of a pattern.
% The prefix |#1| in the current filename is replaced by |#2|
% and the suffix of the current filename is kept
% (it is assumed that the filename does not contain the substring `|~~~|'
% which is used as a delimiter).
% Compilation is handed over to the new file by |\childdocforward|:
%    \begin{macrocode}
\newcommand{\childdocforwardprefix}[3][]
{
  \begingroup
    \def\childdocextract #2##1~~~{\def\childdoctmp{\childdocforward[#1]{#3##1}}}
    \expandafter\childdocextract\childdocname~~~
    \expandafter
  \endgroup
  \childdoctmp
}
%    \end{macrocode}

% \macro{\childdoc}
% The deprecated macro |\childdoc| is a legacy version of |\childdocmain|:
%    \begin{macrocode}
\newcommand{\childdoc}{\childdocmain}
%    \end{macrocode}

% \macro{\childdocredirect}
% The deprecated macro |\childdocredirect| is a legacy version
% of |\childdocforward| and |\childdocforwardprefix|:
%    \begin{macrocode}
\newcommand{\childdocredirect}[2][]
{
  \begingroup
    \if?#1?
      \def\childdoctmp{\childdocforward{#2}}
    \else
      \def\childdoctmp{\childdocforwardprefix{#1}{#2}}
    \fi
    \expandafter
  \endgroup
  \childdoctmp
}
%    \end{macrocode}

%\iffalse
%</package>
%\fi
%
\endinput
|\\
|\childdocof{|\textit{main}|}|\\
\end{tabular}
\end{center}
at the top of every child file \textit{child}
which is included by |\include{|\textit{child}|}|
from within the main file
(or at least for those files to be compiled individually).
The argument \textit{main} must be the filename of the main file.

There are a couple of
considerations in setting up the main and child documents:

%%%%%%%%%%%%%%%%%%%%%%%%%%%%%%%%%%%%%%%%
\paragraph{Restrictions.}

Please note the following restrictions:
\begin{itemize}
\item
|\childdocmain| must be called with one argument \textit{main}
to ensure compatibility with earlier version of the package.
It must either be empty (|\childdocmain{}|)
or precisely match the filename of the main file in which it is specified.
See \secref{sec:detection} for further information.
\item
The filename \textit{main} must be specified without the |.tex| extension.
\item
The filename \textit{main} is case sensitive
(even in case-insensitive file systems)
due to internal string comparison.
\item
The argument \textit{main} should be fully expanded, it cannot be a macro.
\item
Subdirectories and special characters should be avoided in filenames.
\item
The command |\childdocmain{|\textit{main}|}| must be followed by a whitespace.
It should not be followed immediately by another command
or by a comment mark `|%|'.
This is because the \TeX{} parser reads the token immediately following
the argument of |\childdocmain| and puts it
at the beginning of every child section;
however, a white\-space is ignored.
\end{itemize}

%%%%%%%%%%%%%%%%%%%%%%%%%%%%%%%%%%%%%%%%
\paragraph{Content of Main File.}

It is advisable to place all content in the child files included by |\include|.
Any output contained in the main file will appear in all child documents
unless suppressed manually;
it cannot be suppressed automatically by the |\includeonly| directive
and thus should normally be avoided.
A method to include some content in the main file
by means of conditional processing is described in \secref{sec:conditional}.

%%%%%%%%%%%%%%%%%%%%%%%%%%%%%%%%%%%%%%%%
\paragraph{Page Numbering.}

When only a part of the document is compiled,
the appropriate numbering of pages
(as well as other status parameters)
is determined from the |.aux| files.
The latter contain information from previous passes.
However this information needs to propagate through
all intermediate child documents.
Therefore the page numbering in child documents may well
be inconsistent until the complete document is compiled at least once.

A useful (if unconventional) way to always ensure a consistent
page numbering is to restart the numbering in each child document
and denote the pages by `\textit{child}|.|\textit{page}'
where \textit{child} represents the chapter/section number of the child file.
This can be achieved by the command
|\numberwithin{page}{|\textit{child}|}|
of the \textsf{amsmath} package
where \textit{child} can be |chapter| or |section|
depending on the chosen structuring.
Alternatively, one can modify the macro |\thepage| appropriately
and reset the counter |page| at the start of each child file.

%%%%%%%%%%%%%%%%%%%%%%%%%%%%%%%%%%%%%%%%%%%%%%%%%%%%%%%%%%%%%%%%%%%%%%%%%%%%%%%%
\subsection{Conditional Processing}
\label{sec:conditional}

The package provides a mechanism to compile different versions
of a document. To customise the versions further some conditional processing
can come in handy to distinguish which version is being compiled.
The package provides two macros to describe the compilation context:

%%%%%%%%%%%%%%%%%%%%%%%%%%%%%%%%%%%%%%%%
\DescribeMacro{\ifchilddoc}
The conditional |\ifchilddoc| distinguishes between the compilation of
child documents and the main document:
%
\begin{center}
|\ifchilddoc |\textit{child-code}| |[|\||else |\textit{main-code}]| \||fi|
\end{center}

%%%%%%%%%%%%%%%%%%%%%%%%%%%%%%%%%%%%%%%%
\DescribeMacro{\childdocname}
\DescribeMacro{\childdocjob}
The macro |\childdocname| contains the filename (without extension)
of the main or child file being processed.
Note that |\childdocjob| will always contain the name of the main file.

%%%%%%%%%%%%%%%%%%%%%%%%%%%%%%%%%%%%%%%%
\paragraph{Title Page.}

Conditional processing can be used to include a title or banner page
in the main document when proper precautions are taken.
Importantly, the code in the main file should ensure that the page counter
(as well as other status parameters which are stored in the |.aux| files)
takes the same value after the conditional processing.
Otherwise the page numbers may take divergent values
depending on which part is compiled.

For example, a title page could be declared by:
%
\begin{center}
\begin{tabular}{l}
|\ifchilddoc\||else|\\
|\addtocounter{page}{-1}|\\
\textit{code for title page}\\
|\newpage|\\
|\||fi|
\end{tabular}
\end{center}
%
A banner page for the child documents can be generated by:
%
\begin{center}
\begin{tabular}{l}
|\ifchilddoc|\\
|\addtocounter{page}{-1}|\\
\textit{code for banner page}\\
|\newpage|\\
|\||fi|
\end{tabular}
\end{center}
%
Here one could write a message such as:
\begin{center}
|This is the part \childdocname{} of \childdocjob{}.|
\end{center}

%%%%%%%%%%%%%%%%%%%%%%%%%%%%%%%%%%%%%%%%%%%%%%%%%%%%%%%%%%%%%%%%%%%%%%%%%%%%%%%%
\subsection{Flags}
\label{sec:flags}

The package makes it easy to generate different versions
of the main or child documents.
To this end compilation flags can be defined
and assigned different default values.
They will be particularly useful in conjunction
with the forwarding mechanism described in \secref{sec:forward}.

For example, it may be useful to have a flag |\version|
which can be set to |draft| or |final|.
The document source will contain some conditional code
depending on the value of |\version|.
Suppose further, the flag should default to |final| for the main file
and to |draft| for child files
which is a natural assignment for editing the document.
This is achieved by placing the following code
in the preamble of the main document
(below the |\childdocmain| directive):
%
\begin{center}
\begin{tabular}{l}
|\ifchilddoc|\\
|\providecommand{\version}{draft}|\\
|\||else|\\
|\providecommand{\version}{final}|\\
|\||fi|
\end{tabular}
\end{center}
%
The definition by |\providecommand| makes sure
that previous definitions are not overwritten.
Further statements |\providecommand{\version}{...}|
can thus be added before the above code to override it.

For the main file, one might add a line
(between |\childdocmain| and the above block)
%
\begin{center}
|%\ifchilddoc\||else\providecommand{\version}{draft}\||fi|
\end{center}
%
which can be uncommented to produce a draft version.
Likewise one can add a line to the very top of a child file
(above the |\childdocof{|\textit{main}|}| directive)
%
\begin{center}
|%\providecommand{\version}{final}|
\end{center}
%
which can be uncommented to produce the final version of this child document.

%%%%%%%%%%%%%%%%%%%%%%%%%%%%%%%%%%%%%%%%%%%%%%%%%%%%%%%%%%%%%%%%%%%%%%%%%%%%%%%%
\subsection{Forwarding}
\label{sec:forward}

Different versions of the main or child documents
using compilation flags as described in \secref{sec:flags}
can be (permanently) stored in different files
for convenient compilation, viewing and distribution.
To this end, the package defines a command
to pass on compilation to a different file:

%%%%%%%%%%%%%%%%%%%%%%%%%%%%%%%%%%%%%%%%
\DescribeMacro{\childdocforward}
The command |\childdocforward| redirects processing to
another source file:
%
\begin{center}
\begin{tabular}{l}
|% \iffalse
%
% childdoc.dtx Copyright (C) 2017-2018 Niklas Beisert
%
% This work may be distributed and/or modified under the
% conditions of the LaTeX Project Public License, either version 1.3
% of this license or (at your option) any later version.
% The latest version of this license is in
%   http://www.latex-project.org/lppl.txt
% and version 1.3 or later is part of all distributions of LaTeX
% version 2005/12/01 or later.
%
% This work has the LPPL maintenance status `maintained'.
%
% The Current Maintainer of this work is Niklas Beisert.
%
% This work consists of the files childdoc.dtx and childdoc.ins
% and the derived files childdoc.def and cdocsamp.tex with
% cdocsch1.tex, cdocsch2.tex, cdocsdrf.tex, cdocsfn1.tex, cdocsfn2.tex.
%
%<package>\ifdefined\childdocmain\endinput\fi
%<package>\ProvidesFile{childdoc.def}[2018/12/30 v2.0 child document driver]
%<samplemain>\ProvidesFile{cdocsamp.tex}[2018/12/30 v2.0 sample for childdoc]
%<*driver>
%\ProvidesFile{childdoc.drv}[2018/12/30 v2.0 childdoc reference manual file]
\PassOptionsToClass{10pt,a4paper}{article}
\documentclass{ltxdoc}

\usepackage[margin=35mm]{geometry}
\usepackage{hyperref}
\usepackage{hyperxmp}
\usepackage[usenames]{color}

\hypersetup{colorlinks=true}
\hypersetup{pdfstartview=FitH}
\hypersetup{pdfpagemode=UseNone}
\hypersetup{pdfsource={}}
\hypersetup{pdflang={en-UK}}
\hypersetup{pdfcopyright={Copyright 2017-2018 Niklas Beisert.
  This work may be distributed and/or modified under the
  conditions of the LaTeX Project Public License, either version 1.3
  of this license or (at your option) any later version.}}
\hypersetup{pdflicenseurl={http://www.latex-project.org/lppl.txt}}
\hypersetup{pdfcontactaddress={ETH Zurich, ITP, HIT K,
  Wolfgang-Pauli-Strasse 27}}
\hypersetup{pdfcontactpostcode={8093}}
\hypersetup{pdfcontactcity={Zurich}}
\hypersetup{pdfcontactcountry={Switzerland}}
\hypersetup{pdfcontactemail={nbeisert@itp.phys.ethz.ch}}
\hypersetup{pdfcontacturl={http://people.phys.ethz.ch/\xmptilde nbeisert/}}

\newcommand{\secref}[1]{\hyperref[#1]{section \ref*{#1}}}

\parskip1ex
\parindent0pt
\let\olditemize\itemize
\def\itemize{\olditemize\parskip0pt}

\begin{document}

\title{The \textsf{childdoc} Package}
\hypersetup{pdftitle={The childdoc Package}}
\author{Niklas Beisert\\[2ex]
  Institut f\"ur Theoretische Physik\\
  Eidgen\"ossische Technische Hochschule Z\"urich\\
  Wolfgang-Pauli-Strasse 27, 8093 Z\"urich, Switzerland\\[1ex]
  \href{mailto:nbeisert@itp.phys.ethz.ch}
  {\texttt{nbeisert@itp.phys.ethz.ch}}}
\hypersetup{pdfauthor={Niklas Beisert}}
\hypersetup{pdfsubject={Manual for the LaTeX2e Package childdoc}}
\date{30 December 2018, \textsf{v2.0}}
\maketitle

\begin{abstract}\noindent
\textsf{childdoc} is a \LaTeXe{} package
that enables the direct compilation
of document sections included by |\include|
to individual files.
\end{abstract}

\begingroup
\parskip0ex
\tableofcontents
\endgroup

%%%%%%%%%%%%%%%%%%%%%%%%%%%%%%%%%%%%%%%%%%%%%%%%%%%%%%%%%%%%%%%%%%%%%%%%%%%%%%%%
%%%%%%%%%%%%%%%%%%%%%%%%%%%%%%%%%%%%%%%%%%%%%%%%%%%%%%%%%%%%%%%%%%%%%%%%%%%%%%%%
\section{Introduction}

\LaTeX{} provides a mechanism to structure a large document (such as a book)
into a main file and several child files (containing the chapters)
using the |\include| command.
This mechanism is beneficial for documents
which span hundreds of pages in order to
make the source file(s) more manageable.
Moreover, compilation can be restricted to
selected child files by means of the |\includeonly| command.
The latter feature can be used to reduce the compilation time while editing
(this was significantly more useful in the earlier days of \LaTeX{})
or to generate a smaller document which is easier to navigate.
Another application of |\includeonly| is to generate
documents consisting of selected parts of the complete document.

However, there are a few drawbacks of the plain |\include| mechanism:
\begin{itemize}
\item
The child files cannot be compiled on their own,
they can only be compiled via the main file.
A naive editing environment
(such as a text editor with an option
to have the current file processed by \LaTeX)
may require one to switch to the main file before compiling;
attempting to compile the child file produces errors.
\item
The main file must be modified (each time)
to adjust the |\includeonly| command
to the present needs. This easily leaves the main file in a messy state.
\item
The generated document will always carry the filename
of the main document. This is inconvenient if
several child files are to be compiled and
to be kept for distribution.
\end{itemize}

The present package provides a simple interface
to make child files individually compilable by \LaTeX{}.
Compiling a child file then has the same effect as compiling
the main file with an |\includeonly| command
to select the appropriate child.
Moreover the generated document will carry the name of the child
rather than the main file.
This resolves all three above issues.

This feature is meant to make the editing of books,
thesis documents and lecture notes somewhat more convenient.
However, the package can also be used efficiently for
composing a series of documents (such as exercise sheets)
which are typically distributed individually.
It then assists the author in generating the individual documents
(potentially in different versions)
as well as a document containing the collected series.
Another application is in developing style files
or other kinds of included material
where compilation of the style file could redirect
to a sample or test file.

%%%%%%%%%%%%%%%%%%%%%%%%%%%%%%%%%%%%%%%%%%%%%%%%%%%%%%%%%%%%%%%%%%%%%%%%%%%%%%%%
%%%%%%%%%%%%%%%%%%%%%%%%%%%%%%%%%%%%%%%%%%%%%%%%%%%%%%%%%%%%%%%%%%%%%%%%%%%%%%%%
\section{Usage}

First of all, the package \textsf{childdoc} is \emph{not} a standard
\LaTeXe{} |.sty| style file! Therefore it needs to be invoked in
a non-standard way.

%%%%%%%%%%%%%%%%%%%%%%%%%%%%%%%%%%%%%%%%%%%%%%%%%%%%%%%%%%%%%%%%%%%%%%%%%%%%%%%%
\subsection{Included Files}
\label{sec:include}

%%%%%%%%%%%%%%%%%%%%%%%%%%%%%%%%%%%%%%%%
\DescribeMacro{\childdocmain}
To use the package, add the commands
\begin{center}
\begin{tabular}{l}
|% \iffalse
%
% childdoc.dtx Copyright (C) 2017-2018 Niklas Beisert
%
% This work may be distributed and/or modified under the
% conditions of the LaTeX Project Public License, either version 1.3
% of this license or (at your option) any later version.
% The latest version of this license is in
%   http://www.latex-project.org/lppl.txt
% and version 1.3 or later is part of all distributions of LaTeX
% version 2005/12/01 or later.
%
% This work has the LPPL maintenance status `maintained'.
%
% The Current Maintainer of this work is Niklas Beisert.
%
% This work consists of the files childdoc.dtx and childdoc.ins
% and the derived files childdoc.def and cdocsamp.tex with
% cdocsch1.tex, cdocsch2.tex, cdocsdrf.tex, cdocsfn1.tex, cdocsfn2.tex.
%
%<package>\ifdefined\childdocmain\endinput\fi
%<package>\ProvidesFile{childdoc.def}[2018/12/30 v2.0 child document driver]
%<samplemain>\ProvidesFile{cdocsamp.tex}[2018/12/30 v2.0 sample for childdoc]
%<*driver>
%\ProvidesFile{childdoc.drv}[2018/12/30 v2.0 childdoc reference manual file]
\PassOptionsToClass{10pt,a4paper}{article}
\documentclass{ltxdoc}

\usepackage[margin=35mm]{geometry}
\usepackage{hyperref}
\usepackage{hyperxmp}
\usepackage[usenames]{color}

\hypersetup{colorlinks=true}
\hypersetup{pdfstartview=FitH}
\hypersetup{pdfpagemode=UseNone}
\hypersetup{pdfsource={}}
\hypersetup{pdflang={en-UK}}
\hypersetup{pdfcopyright={Copyright 2017-2018 Niklas Beisert.
  This work may be distributed and/or modified under the
  conditions of the LaTeX Project Public License, either version 1.3
  of this license or (at your option) any later version.}}
\hypersetup{pdflicenseurl={http://www.latex-project.org/lppl.txt}}
\hypersetup{pdfcontactaddress={ETH Zurich, ITP, HIT K,
  Wolfgang-Pauli-Strasse 27}}
\hypersetup{pdfcontactpostcode={8093}}
\hypersetup{pdfcontactcity={Zurich}}
\hypersetup{pdfcontactcountry={Switzerland}}
\hypersetup{pdfcontactemail={nbeisert@itp.phys.ethz.ch}}
\hypersetup{pdfcontacturl={http://people.phys.ethz.ch/\xmptilde nbeisert/}}

\newcommand{\secref}[1]{\hyperref[#1]{section \ref*{#1}}}

\parskip1ex
\parindent0pt
\let\olditemize\itemize
\def\itemize{\olditemize\parskip0pt}

\begin{document}

\title{The \textsf{childdoc} Package}
\hypersetup{pdftitle={The childdoc Package}}
\author{Niklas Beisert\\[2ex]
  Institut f\"ur Theoretische Physik\\
  Eidgen\"ossische Technische Hochschule Z\"urich\\
  Wolfgang-Pauli-Strasse 27, 8093 Z\"urich, Switzerland\\[1ex]
  \href{mailto:nbeisert@itp.phys.ethz.ch}
  {\texttt{nbeisert@itp.phys.ethz.ch}}}
\hypersetup{pdfauthor={Niklas Beisert}}
\hypersetup{pdfsubject={Manual for the LaTeX2e Package childdoc}}
\date{30 December 2018, \textsf{v2.0}}
\maketitle

\begin{abstract}\noindent
\textsf{childdoc} is a \LaTeXe{} package
that enables the direct compilation
of document sections included by |\include|
to individual files.
\end{abstract}

\begingroup
\parskip0ex
\tableofcontents
\endgroup

%%%%%%%%%%%%%%%%%%%%%%%%%%%%%%%%%%%%%%%%%%%%%%%%%%%%%%%%%%%%%%%%%%%%%%%%%%%%%%%%
%%%%%%%%%%%%%%%%%%%%%%%%%%%%%%%%%%%%%%%%%%%%%%%%%%%%%%%%%%%%%%%%%%%%%%%%%%%%%%%%
\section{Introduction}

\LaTeX{} provides a mechanism to structure a large document (such as a book)
into a main file and several child files (containing the chapters)
using the |\include| command.
This mechanism is beneficial for documents
which span hundreds of pages in order to
make the source file(s) more manageable.
Moreover, compilation can be restricted to
selected child files by means of the |\includeonly| command.
The latter feature can be used to reduce the compilation time while editing
(this was significantly more useful in the earlier days of \LaTeX{})
or to generate a smaller document which is easier to navigate.
Another application of |\includeonly| is to generate
documents consisting of selected parts of the complete document.

However, there are a few drawbacks of the plain |\include| mechanism:
\begin{itemize}
\item
The child files cannot be compiled on their own,
they can only be compiled via the main file.
A naive editing environment
(such as a text editor with an option
to have the current file processed by \LaTeX)
may require one to switch to the main file before compiling;
attempting to compile the child file produces errors.
\item
The main file must be modified (each time)
to adjust the |\includeonly| command
to the present needs. This easily leaves the main file in a messy state.
\item
The generated document will always carry the filename
of the main document. This is inconvenient if
several child files are to be compiled and
to be kept for distribution.
\end{itemize}

The present package provides a simple interface
to make child files individually compilable by \LaTeX{}.
Compiling a child file then has the same effect as compiling
the main file with an |\includeonly| command
to select the appropriate child.
Moreover the generated document will carry the name of the child
rather than the main file.
This resolves all three above issues.

This feature is meant to make the editing of books,
thesis documents and lecture notes somewhat more convenient.
However, the package can also be used efficiently for
composing a series of documents (such as exercise sheets)
which are typically distributed individually.
It then assists the author in generating the individual documents
(potentially in different versions)
as well as a document containing the collected series.
Another application is in developing style files
or other kinds of included material
where compilation of the style file could redirect
to a sample or test file.

%%%%%%%%%%%%%%%%%%%%%%%%%%%%%%%%%%%%%%%%%%%%%%%%%%%%%%%%%%%%%%%%%%%%%%%%%%%%%%%%
%%%%%%%%%%%%%%%%%%%%%%%%%%%%%%%%%%%%%%%%%%%%%%%%%%%%%%%%%%%%%%%%%%%%%%%%%%%%%%%%
\section{Usage}

First of all, the package \textsf{childdoc} is \emph{not} a standard
\LaTeXe{} |.sty| style file! Therefore it needs to be invoked in
a non-standard way.

%%%%%%%%%%%%%%%%%%%%%%%%%%%%%%%%%%%%%%%%%%%%%%%%%%%%%%%%%%%%%%%%%%%%%%%%%%%%%%%%
\subsection{Included Files}
\label{sec:include}

%%%%%%%%%%%%%%%%%%%%%%%%%%%%%%%%%%%%%%%%
\DescribeMacro{\childdocmain}
To use the package, add the commands
\begin{center}
\begin{tabular}{l}
|\input{childdoc.def}|\\
|\childdocmain{}|\\
\end{tabular}
\end{center}
at the very top of the main \LaTeX{} file,
in particular \emph{before} the |\documentclass| statement!
The argument of |\childdocmain| should be left empty
(but it must be present).

%%%%%%%%%%%%%%%%%%%%%%%%%%%%%%%%%%%%%%%%
\DescribeMacro{\childdocof}
Furthermore, add the commands
\begin{center}
\begin{tabular}{l}
|\input{childdoc.def}|\\
|\childdocof{|\textit{main}|}|\\
\end{tabular}
\end{center}
at the top of every child file \textit{child}
which is included by |\include{|\textit{child}|}|
from within the main file
(or at least for those files to be compiled individually).
The argument \textit{main} must be the filename of the main file.

There are a couple of
considerations in setting up the main and child documents:

%%%%%%%%%%%%%%%%%%%%%%%%%%%%%%%%%%%%%%%%
\paragraph{Restrictions.}

Please note the following restrictions:
\begin{itemize}
\item
|\childdocmain| must be called with one argument \textit{main}
to ensure compatibility with earlier version of the package.
It must either be empty (|\childdocmain{}|)
or precisely match the filename of the main file in which it is specified.
See \secref{sec:detection} for further information.
\item
The filename \textit{main} must be specified without the |.tex| extension.
\item
The filename \textit{main} is case sensitive
(even in case-insensitive file systems)
due to internal string comparison.
\item
The argument \textit{main} should be fully expanded, it cannot be a macro.
\item
Subdirectories and special characters should be avoided in filenames.
\item
The command |\childdocmain{|\textit{main}|}| must be followed by a whitespace.
It should not be followed immediately by another command
or by a comment mark `|%|'.
This is because the \TeX{} parser reads the token immediately following
the argument of |\childdocmain| and puts it
at the beginning of every child section;
however, a white\-space is ignored.
\end{itemize}

%%%%%%%%%%%%%%%%%%%%%%%%%%%%%%%%%%%%%%%%
\paragraph{Content of Main File.}

It is advisable to place all content in the child files included by |\include|.
Any output contained in the main file will appear in all child documents
unless suppressed manually;
it cannot be suppressed automatically by the |\includeonly| directive
and thus should normally be avoided.
A method to include some content in the main file
by means of conditional processing is described in \secref{sec:conditional}.

%%%%%%%%%%%%%%%%%%%%%%%%%%%%%%%%%%%%%%%%
\paragraph{Page Numbering.}

When only a part of the document is compiled,
the appropriate numbering of pages
(as well as other status parameters)
is determined from the |.aux| files.
The latter contain information from previous passes.
However this information needs to propagate through
all intermediate child documents.
Therefore the page numbering in child documents may well
be inconsistent until the complete document is compiled at least once.

A useful (if unconventional) way to always ensure a consistent
page numbering is to restart the numbering in each child document
and denote the pages by `\textit{child}|.|\textit{page}'
where \textit{child} represents the chapter/section number of the child file.
This can be achieved by the command
|\numberwithin{page}{|\textit{child}|}|
of the \textsf{amsmath} package
where \textit{child} can be |chapter| or |section|
depending on the chosen structuring.
Alternatively, one can modify the macro |\thepage| appropriately
and reset the counter |page| at the start of each child file.

%%%%%%%%%%%%%%%%%%%%%%%%%%%%%%%%%%%%%%%%%%%%%%%%%%%%%%%%%%%%%%%%%%%%%%%%%%%%%%%%
\subsection{Conditional Processing}
\label{sec:conditional}

The package provides a mechanism to compile different versions
of a document. To customise the versions further some conditional processing
can come in handy to distinguish which version is being compiled.
The package provides two macros to describe the compilation context:

%%%%%%%%%%%%%%%%%%%%%%%%%%%%%%%%%%%%%%%%
\DescribeMacro{\ifchilddoc}
The conditional |\ifchilddoc| distinguishes between the compilation of
child documents and the main document:
%
\begin{center}
|\ifchilddoc |\textit{child-code}| |[|\||else |\textit{main-code}]| \||fi|
\end{center}

%%%%%%%%%%%%%%%%%%%%%%%%%%%%%%%%%%%%%%%%
\DescribeMacro{\childdocname}
\DescribeMacro{\childdocjob}
The macro |\childdocname| contains the filename (without extension)
of the main or child file being processed.
Note that |\childdocjob| will always contain the name of the main file.

%%%%%%%%%%%%%%%%%%%%%%%%%%%%%%%%%%%%%%%%
\paragraph{Title Page.}

Conditional processing can be used to include a title or banner page
in the main document when proper precautions are taken.
Importantly, the code in the main file should ensure that the page counter
(as well as other status parameters which are stored in the |.aux| files)
takes the same value after the conditional processing.
Otherwise the page numbers may take divergent values
depending on which part is compiled.

For example, a title page could be declared by:
%
\begin{center}
\begin{tabular}{l}
|\ifchilddoc\||else|\\
|\addtocounter{page}{-1}|\\
\textit{code for title page}\\
|\newpage|\\
|\||fi|
\end{tabular}
\end{center}
%
A banner page for the child documents can be generated by:
%
\begin{center}
\begin{tabular}{l}
|\ifchilddoc|\\
|\addtocounter{page}{-1}|\\
\textit{code for banner page}\\
|\newpage|\\
|\||fi|
\end{tabular}
\end{center}
%
Here one could write a message such as:
\begin{center}
|This is the part \childdocname{} of \childdocjob{}.|
\end{center}

%%%%%%%%%%%%%%%%%%%%%%%%%%%%%%%%%%%%%%%%%%%%%%%%%%%%%%%%%%%%%%%%%%%%%%%%%%%%%%%%
\subsection{Flags}
\label{sec:flags}

The package makes it easy to generate different versions
of the main or child documents.
To this end compilation flags can be defined
and assigned different default values.
They will be particularly useful in conjunction
with the forwarding mechanism described in \secref{sec:forward}.

For example, it may be useful to have a flag |\version|
which can be set to |draft| or |final|.
The document source will contain some conditional code
depending on the value of |\version|.
Suppose further, the flag should default to |final| for the main file
and to |draft| for child files
which is a natural assignment for editing the document.
This is achieved by placing the following code
in the preamble of the main document
(below the |\childdocmain| directive):
%
\begin{center}
\begin{tabular}{l}
|\ifchilddoc|\\
|\providecommand{\version}{draft}|\\
|\||else|\\
|\providecommand{\version}{final}|\\
|\||fi|
\end{tabular}
\end{center}
%
The definition by |\providecommand| makes sure
that previous definitions are not overwritten.
Further statements |\providecommand{\version}{...}|
can thus be added before the above code to override it.

For the main file, one might add a line
(between |\childdocmain| and the above block)
%
\begin{center}
|%\ifchilddoc\||else\providecommand{\version}{draft}\||fi|
\end{center}
%
which can be uncommented to produce a draft version.
Likewise one can add a line to the very top of a child file
(above the |\childdocof{|\textit{main}|}| directive)
%
\begin{center}
|%\providecommand{\version}{final}|
\end{center}
%
which can be uncommented to produce the final version of this child document.

%%%%%%%%%%%%%%%%%%%%%%%%%%%%%%%%%%%%%%%%%%%%%%%%%%%%%%%%%%%%%%%%%%%%%%%%%%%%%%%%
\subsection{Forwarding}
\label{sec:forward}

Different versions of the main or child documents
using compilation flags as described in \secref{sec:flags}
can be (permanently) stored in different files
for convenient compilation, viewing and distribution.
To this end, the package defines a command
to pass on compilation to a different file:

%%%%%%%%%%%%%%%%%%%%%%%%%%%%%%%%%%%%%%%%
\DescribeMacro{\childdocforward}
The command |\childdocforward| redirects processing to
another source file:
%
\begin{center}
\begin{tabular}{l}
|\input{childdoc.def}|\\
|\childdocforward[|\textit{main}|]{|\textit{dest}|}|\\
\end{tabular}
\end{center}
%
The argument \textit{dest} is the destination file
(without extension).
It should be the main file or one of the child files.
Note that further \textsf{childdoc} directives
such as |\childdocof| and |\childdocforward|
in the indicated file will be processed in this form.
The optional argument \textit{main}
passes on directly to the main file \textit{main}
while pretending to compile the child \textit{dest}.
This form behaves as if \textit{dest}
issues |\childdocof{|\textit{main}|}| right away,
and no further \textsf{childdoc} directives will be processed.

%%%%%%%%%%%%%%%%%%%%%%%%%%%%%%%%%%%%%%%%
\DescribeMacro{\...prefix}
In the alternative form |\childdocforwardprefix|,
%
\begin{center}
\begin{tabular}{l}
|\input{childdoc.def}|\\
|\childdocforwardprefix[|\textit{main}|]{|\textit{prefix}|}{|\textit{dest}|}|
\end{tabular}
\end{center}
%
the destination file is determined by a pattern
depending on the current file:
To make this work, the current file must be called
`{\textit{prefix}\hspace{0.2em}\textit{suffix}}'
with \textit{prefix} matching precisely the argument.
Processing is then passed on to the file
`{\textit{dest}\hspace{0.2em}\textit{suffix}}'.
Surely, the same effect is achieved by
directly specifying the
argument `{\textit{dest}\hspace{0.2em}\textit{suffix}}'
in the first form.
However, that requires to set up a different file
for each child. With the alternative form of the command
all these files can have exactly the same content
which simplifies setting them up and maintaining them.

For example, the following file |draft.tex|
with a compilation flag |\version| as described in \secref{sec:flags}
compiles the main document as a draft:
%
\begin{center}
\begin{tabular}{l}
|\def\version{draft}|\\
|\input{childdoc.def}|\\
|\childdocforward{|\textit{main}|}|
\end{tabular}
\end{center}
%
Likewise, the following files |final|\textit{nn}|.tex|
compile the final version of the child document
|child|\textit{nn}|.tex|:
%
\begin{center}
\begin{tabular}{l}
|\def\version{final}|\\
|\input{childdoc.def}|\\
|\childdocforwardprefix{final}{child}|
\end{tabular}
\end{center}
%

Note that when several versions of a main file and/or of each child file
are to be generated, it may be convenient to set up a |Makefile| or
shell script to automatise the process.

%%%%%%%%%%%%%%%%%%%%%%%%%%%%%%%%%%%%%%%%%%%%%%%%%%%%%%%%%%%%%%%%%%%%%%%%%%%%%%%%
\subsection{Command Line Processing}
\label{sec:commandline}

The effect of redirection files can also be achieved by invoking
the \LaTeX{} compiler with a more elaborate command line.
Most conveniently this should be done as part
of a shell script or a |Makefile|.

When using \textsf{childdoc} in the main file, the following
command lines effectively perform a redirection
(note that depending on the shell being used,
backslashes may have to be doubled: `|\|' $\to$ `|\\|'):
%
\begin{center}
|... -jobname "|\textit{target}|" |\\|"|[\textit{flags}]%
|\input{childdoc.def}\childdocforward[|\textit{main}|]{|\textit{dest}|}"|
\end{center}
%
Here \textit{target} is the name of the output file,
\textit{main} is the name of the main file
and \textit{dest} is the name of the main or child file to be processed
(all filenames without extensions).
The optional argument \textit{main} can be omitted
if \textit{main} matches \textit{dest}.
Optionally, compilation \textit{flags} can be defined via |\def| commands.
This command line makes the \TeX{} engine believe
it is compiling the file \textit{target}
whose content is specified as the latter parameter.
The provided code then forwards the processing to
\textit{main} or \textit{dest} as described in \secref{sec:forward}.

%%%%%%%%%%%%%%%%%%%%%%%%%%%%%%%%%%%%%%%%%%%%%%%%%%%%%%%%%%%%%%%%%%%%%%%%%%%%%%%%
\subsection{Include by Input}
\label{sec:input}

Including child documents by |\include| has some restrictions by design.
Most notably, the content of a child document always occupies
its own set of pages; pages cannot be shared between child documents.
Usually, this behaviour makes perfect sense
because each child document contain an essential part of the document.
However, in some situations it may be desirable to compose
a document from a collection of parts
without having mandatory page breaks between then.
For this case, the package
provides a mechanism to include parts
by |\input| which can also be processed individually.
However, by construction this mechanism
requires manual handling of the content to be output.

%%%%%%%%%%%%%%%%%%%%%%%%%%%%%%%%%%%%%%%%
\DescribeMacro{\ifchilddocmanual}
The main file should be prepared as usual, see \secref{sec:include}.
However, the document body must make a distinction
between processing of an individual part and of the main document, e.g.:
%
\begin{center}
\begin{tabular}{l}
|\ifchilddocmanual|\\
|\input{\childdocname}|\\
|\||else|\\
\textit{document body with }|\input{|\textit{part}|}|\\
|\||fi|
\end{tabular}
\end{center}
%
The conditional |\ifchilddocmanual| is true whenever
a part to be included by |\input| is being compiled,
and the name of the part is stored in |\childdocname|.

%%%%%%%%%%%%%%%%%%%%%%%%%%%%%%%%%%%%%%%%
\DescribeMacro{\childdocby}
Each part to be included by |\input| should start with:
%
\begin{center}
\begin{tabular}{l}
|\input{childdoc.def}|\\
|\childdocby{|\textit{main}|}|\\
\end{tabular}
\end{center}
%
The directive |\childdocby| is similar to |\childdocof|
described in \secref{sec:include},
but the subsequent selection of content must be done manually.
To that end, both |\ifchilddoc| and |\ifchilddocmanual|
will be true upon processing of a part,
and the name of the part is stored in |\childdocname|.
Note that |\jobname| will be set to the filename of the current part
so that each part receives an individual |.aux| file
that does not interfere with the |.aux| file(s) of the main document.
This behaviour can be altered by the alternative form
|\childdocby[*]{|\textit{main}|}| (with a non-empty optional argument)
which uses the |.aux| file of the main document
by setting |\jobname| to \textit{main}.

%%%%%%%%%%%%%%%%%%%%%%%%%%%%%%%%%%%%%%%%%%%%%%%%%%%%%%%%%%%%%%%%%%%%%%%%%%%%%%%%
\subsection{Driver Development}
\label{sec:driver}

The \textsf{childdoc} mechanism can also be use for the development
of definition files such as \LaTeX{} styles or classes.
This case differs from the above setup with multiple parts
included by |\include| in that no |\includeonly| should be invoked.
This can be achieved by starting the include file
(before |\ProvidesPackage|) with:
%
\begin{center}
\begin{tabular}{l}
|\input{childdoc.def}|\\
|\childdocforward{|\textit{main}|}|\\
\end{tabular}
\end{center}
%
or alternatively with:
%
\begin{center}
\begin{tabular}{l}
|\input{childdoc.def}|\\
|\childdocby{|\textit{main}|}|\\
\end{tabular}
\end{center}
%
Both forms have slightly different effects as described above.
The main file is prepared as usual, see \secref{sec:include}.

%%%%%%%%%%%%%%%%%%%%%%%%%%%%%%%%%%%%%%%%%%%%%%%%%%%%%%%%%%%%%%%%%%%%%%%%%%%%%%%%
\subsection{Legacy Detection}
\label{sec:detection}

The directive |\childdocmain| in the main file can detect
whether the complete document or merely a child is to be compiled
even without using the directive |\childdocof|.
This method is deprecated because it is less robust
and there is no compelling reason to use it;
it is merely provided for backward compatibility
and it may be removed in future versions.

If the detection mechanism is to be used,
it is mandatory to correctly specify
the filename of the main file as the argument of |\childdocmain|:
%
\begin{center}
\begin{tabular}{l}
|\input{childdoc.def}|\\
|\childdocmain{|\textit{main}|}|\\
\end{tabular}
\end{center}
%
If |\jobname| does not match the argument \textit{main} of |\childdocmain|,
it is assumed that |\jobname| points to the child file to be compiled.
When using |\childdocmain| with the main file specified as argument,
it suffices to start a child file
with just |\input{|\textit{main}|}|
without loading of the package and using |\childdocof|.
If instead all processing is done
with the appropriate \textsf{childdoc} directives,
the argument of \textit{main} of |\childdocmain| can be empty.

An alternative version of the command line processing described
in \secref{sec:commandline} using the detection mechanism reads:
%
\begin{center}
|... -jobname "|\textit{target}|" "|[\textit{flags}]%
[|\def\jobname{|\textit{dest}|}|]|\input{|\textit{main}|}"|
\end{center}

%%%%%%%%%%%%%%%%%%%%%%%%%%%%%%%%%%%%%%%%%%%%%%%%%%%%%%%%%%%%%%%%%%%%%%%%%%%%%%%%
\subsection{Manual Code}
\label{sec:manual}

In case one cannot be certain whether the definitions file |childdoc.def|
is installed on the target \TeX{} distribution
and one prefers not to ship it,
it is conceivable to paste a few relevant commands into the sources.

To that end, drop all statements |\input{childdoc.def}|
and perform the replacements as outlined below.
Instead of |\childdocmain{|\textit{main}|}| add the following code
to the top of the main file:
%
\begin{center}
\begin{tabular}{l}
|\||ifdefined\childdocname\endinput\||fi\newif\ifchilddoc|\\
|\edef\childdocname{\scantokens\expandafter{\jobname\noexpand}}|\\
|\def\childdocmain{|\textit{main}|}\||ifx\childdocmain\childdocname\||else|\\
|\childdoctrue\includeonly{\childdocname}\let\jobname\childdocmain\||fi|\\
\end{tabular}
\end{center}
%
Instead of |\childdocof{|\textit{main}|}| just include the main file
at the top of each child file:
%
\begin{center}
|\input{|\textit{main}|}|
\end{center}
%
A simple redirection |\childdocforward{|\textit{dest}|}| is achieved by:
%
\begin{center}
|\def\jobname{|\textit{dest}|}\input{\jobname}|
\end{center}
%
The redirection with prefix
|\childdocforwardprefix[|\textit{prefix}|]{|\textit{dest}|}|
is accomplished by:
%
\begin{center}
\begin{tabular}{l}
|{\edef\jobname{\scantokens\expandafter{\jobname\noexpand}}|\\
|\def\redirectjob |\textit{prefix}|#1~~~{\gdef\jobname{|\textit{dest}|#1}}|\\
|\expandafter\redirectjob\jobname~~~}\input{\jobname}|
\end{tabular}
\end{center}

In an alternative approach,
child documents can be compiled by a specific command line
without additional code or specific definitions:
%
\begin{center}
|... -jobname "|\textit{target}|" "|[\textit{flags}]%
|\includeonly{|\textit{dest}|}\input{|\textit{main}|}"|
\end{center}
%

%%%%%%%%%%%%%%%%%%%%%%%%%%%%%%%%%%%%%%%%%%%%%%%%%%%%%%%%%%%%%%%%%%%%%%%%%%%%%%%%
%%%%%%%%%%%%%%%%%%%%%%%%%%%%%%%%%%%%%%%%%%%%%%%%%%%%%%%%%%%%%%%%%%%%%%%%%%%%%%%%
\section{Information}

%%%%%%%%%%%%%%%%%%%%%%%%%%%%%%%%%%%%%%%%%%%%%%%%%%%%%%%%%%%%%%%%%%%%%%%%%%%%%%%%
\subsection{Copyright}

Copyright \copyright{} 2017--2018 Niklas Beisert

This work may be distributed and/or modified under the
conditions of the \LaTeX{} Project Public License, either version 1.3
of this license or (at your option) any later version.
The latest version of this license is in
  \url{http://www.latex-project.org/lppl.txt}
and version 1.3 or later is part of all distributions of \LaTeX{}
version 2005/12/01 or later.

This work has the LPPL maintenance status `maintained'.

The Current Maintainer of this work is Niklas Beisert.

This work consists of the files |README.txt|, |childdoc.ins| and |childdoc.dtx|
as well as the derived files |childdoc.def|, |cdocsamp.tex|
with |cdocsch1.tex|, |cdocsch2.tex|, |cdocspt3.tex|, |cdocspt4.tex|,
|cdocsdrf.tex|, |cdocsfn1.tex|, |cdocsfn2.tex|
as well as |childdoc.pdf|.

%%%%%%%%%%%%%%%%%%%%%%%%%%%%%%%%%%%%%%%%%%%%%%%%%%%%%%%%%%%%%%%%%%%%%%%%%%%%%%%%
\subsection{Files and Installation}

The package consists of the files:
%
\begin{center}
\begin{tabular}{ll}
    |README.txt|   & readme file \\
    |childdoc.ins| & installation file \\
    |childdoc.dtx| & source file \\
    |childdoc.def| & definition file \\
    |cdocsamp.tex| & sample main file \\
    |cdocsch1.tex| & sample include file \\
    |cdocsch2.tex| & sample include file \\
    |cdocspt3.tex| & sample part file \\
    |cdocspt4.tex| & sample part file \\
    |cdocsdrf.tex| & sample redirection file \\
    |cdocsfn1.tex| & sample redirection file \\
    |cdocsfn2.tex| & sample redirection file \\
    |childdoc.pdf| & manual
\end{tabular}
\end{center}
%
The distribution consists of the files
|README.txt|, |childdoc.ins| and |childdoc.dtx|.
%
\begin{itemize}
\item
Run (pdf)\LaTeX{} on |childdoc.dtx|
to compile the manual |childdoc.pdf| (this file).
\item
Run \LaTeX{} on |childdoc.ins| to create the definitions file |childdoc.def|
and the sample |cdocsamp.tex| with include files
|cdocsch1.tex|, |cdocsch2.tex|, |cdocspt3.tex|, |cdocspt4.tex|,
|cdocsdrf.tex|, |cdocsfn1.tex|, |cdocsfn2.tex|.
Then copy the file |childdoc.def| to an appropriate directory of your \LaTeX{}
distribution, e.g.\ \textit{texmf-root}|/tex/latex/childdoc|.
\end{itemize}

%%%%%%%%%%%%%%%%%%%%%%%%%%%%%%%%%%%%%%%%%%%%%%%%%%%%%%%%%%%%%%%%%%%%%%%%%%%%%%%%
\subsection{Related CTAN Packages}

There are several other packages which offer a similar functionality:
%
\begin{itemize}
\item
The packages
\href{http://ctan.org/pkg/docmute}{\textsf{docmute}},
\href{http://ctan.org/pkg/includex}{\textsf{includex}} and
\href{http://ctan.org/pkg/standalone}{\textsf{standalone}}
provide commands to include only the document body of
a child file thus allowing both files to be compiled individually.
\item
The packages \href{http://ctan.org/pkg/subdocs}{\textsf{subdocs}}
and \href{http://ctan.org/pkg/subfiles}{\textsf{subfiles}}
provide structures in which the main and child documents can be
encapsulated and allowing them to be compiled individually.
The inclusion mechanism is different from the conventional |\include|.
\item
The package \href{http://ctan.org/pkg/combine}{\textsf{combine}}
is an elaborate solution to combine several documents into one.
\end{itemize}
%
See also the CTAN topic \href{http://ctan.org/topic/subdocs}{\textsf{subdocs}}
for further related packages.
The present package differs from the above solutions in that
a document structure constructed with the conventional |\include| mechanism
just needs two extra commands at the top of every file
such that all constituent files can be compiled individually.

%%%%%%%%%%%%%%%%%%%%%%%%%%%%%%%%%%%%%%%%%%%%%%%%%%%%%%%%%%%%%%%%%%%%%%%%%%%%%%%%
%\subsection{Feature Suggestions}
%
%The following is a list of features which may be useful for future
%versions of this package:
%%
%\begin{itemize}
%\item
%\ldots
%\end{itemize}

%%%%%%%%%%%%%%%%%%%%%%%%%%%%%%%%%%%%%%%%%%%%%%%%%%%%%%%%%%%%%%%%%%%%%%%%%%%%%%%%
\subsection{Revision History}

%%%%%%%%%%%%%%%%%%%%%%%%%%%%%%%%%%%%%%%%
\paragraph{v2.0:} 2018/12/30

\begin{itemize}
\item
immediate forward processing
\item
added |\childdocby| mechanism
\item
manual restructured
\end{itemize}

%%%%%%%%%%%%%%%%%%%%%%%%%%%%%%%%%%%%%%%%
\paragraph{v1.6:} 2018/01/17

\begin{itemize}
\item
application for development of include files
\item
corrections to manual
\end{itemize}

%%%%%%%%%%%%%%%%%%%%%%%%%%%%%%%%%%%%%%%%
\paragraph{v1.5:} 2017/05/21

\begin{itemize}
\item
more complete structuring introduced
\item
|\childdocof| introduced
\item
|\childdoc| renamed to |\childdocmain|
\item
|\childredirect| renamed to |\childdocforward| and |\childdocforwardprefix|
and functionality expanded
\end{itemize}

%%%%%%%%%%%%%%%%%%%%%%%%%%%%%%%%%%%%%%%%
\paragraph{v1.0:} 2017/04/27

\begin{itemize}
\item
manual and install package
\item
first version published on CTAN
\end{itemize}

%%%%%%%%%%%%%%%%%%%%%%%%%%%%%%%%%%%%%%%%
\paragraph{v0.6:} 2017/04/26

\begin{itemize}
\item
redirection mechanism added
\end{itemize}

%%%%%%%%%%%%%%%%%%%%%%%%%%%%%%%%%%%%%%%%
\paragraph{v0.5:} 2017/04/26

\begin{itemize}
\item
functionality in definition file
\end{itemize}


%%%%%%%%%%%%%%%%%%%%%%%%%%%%%%%%%%%%%%%%%%%%%%%%%%%%%%%%%%%%%%%%%%%%%%%%%%%%%%%%
%%%%%%%%%%%%%%%%%%%%%%%%%%%%%%%%%%%%%%%%%%%%%%%%%%%%%%%%%%%%%%%%%%%%%%%%%%%%%%%%
%%%%%%%%%%%%%%%%%%%%%%%%%%%%%%%%%%%%%%%%%%%%%%%%%%%%%%%%%%%%%%%%%%%%%%%%%%%%%%%%
\appendix

\settowidth\MacroIndent{\rmfamily\scriptsize 000\ }

 \DocInput{childdoc.dtx}

\end{document}
%</driver>
% \fi
%
% %%%%%%%%%%%%%%%%%%%%%%%%%%%%%%%%%%%%%%%%%%%%%%%%%%%%%%%%%%%%%%%%%%%%%%%%%%%%%%
% %%%%%%%%%%%%%%%%%%%%%%%%%%%%%%%%%%%%%%%%%%%%%%%%%%%%%%%%%%%%%%%%%%%%%%%%%%%%%%
% \section{Sample}
%\iffalse
%<*samplemain>
%\fi
%
% The following presents a sample document
% with two chapters, two parts, a title page,
% a compile flag as well as three forwarding files to set the flag.
% It consists of eight |.tex| files:
% \begin{center}
% \begin{tabular}{ll}
% |cdocsamp.tex|&main file\\
% |cdocsch1.tex|&include file for chapter 1\\
% |cdocsch2.tex|&include file for chapter 2\\
% |cdocspt3.tex|&include file for part 3\\
% |cdocspt4.tex|&include file for part 4\\
% |cdocsdrf.tex|&forwarding file for main file in draft mode\\
% |cdocsfi1.tex|&forwarding file for final version of chapter 1\\
% |cdocsfi2.tex|&forwarding file for final version of chapter 2\\
% \end{tabular}
% \end{center}
% Each of the eight files can be compiled directly by the \LaTeX{} compiler.
%
% %%%%%%%%%%%%%%%%%%%%%%%%%%%%%%%%%%%%%%
% \paragraph{Main File.}
%
% The main file is called |cdocsamp.tex|.
%
% Load the \textsf{childdoc} definitions and
% declare the filename for the main document:
%    \begin{macrocode}
\input{childdoc.def}
\childdocmain{}
%    \end{macrocode}

% Optional override for |\version| flag:
%    \begin{macrocode}
%%\ifchilddoc\else\providecommand{\version}{draft}\fi
%    \end{macrocode}

% Define the default values for the |\version| flag
% (|final| for the main file and |draft| for childs):
%    \begin{macrocode}
\ifchilddoc
\providecommand{\version}{draft}
\else
\providecommand{\version}{final}
\fi
%    \end{macrocode}

% Load the standard document class:
%    \begin{macrocode}
\documentclass[12pt]{article}
%    \end{macrocode}

% Start the document body:
%    \begin{macrocode}
\begin{document}
%    \end{macrocode}

% Declare a title page.
% Print title, part of document being processed and version flag:
%    \begin{macrocode}
\addtocounter{page}{-1}
\begin{center}
{\LARGE\bfseries{}childdoc example\par}
\vspace{1cm}
\ifchilddoc
\ifchilddocmanual part\else chapter\fi:
`\childdocname' of `\childdocjob'\par
\else
main document: `\childdocjob'\par
\fi
version: \version\par
\end{center}
\newpage
%    \end{macrocode}

% Manually include selected file,
% otherwise process as usual:
%    \begin{macrocode}
\ifchilddocmanual
\section*{part `\childdocname'}
\input{\childdocname}
\else
%    \end{macrocode}

% Include the two chapters:
%    \begin{macrocode}
\include{cdocsch1}
\include{cdocsch2}
%    \end{macrocode}

% Include the two parts unless only chapters should be displayed:
%    \begin{macrocode}
\ifchilddoc\else
\section{part three}
\input{cdocspt3}
\section{part four}
\input{cdocspt4}
\fi
%    \end{macrocode}

% Process as usual until here:
%    \begin{macrocode}
\fi
%    \end{macrocode}

% End of document body:
%    \begin{macrocode}
\end{document}
%    \end{macrocode}
%\iffalse
%</samplemain>
%\fi
%
% %%%%%%%%%%%%%%%%%%%%%%%%%%%%%%%%%%%%%%
% \paragraph{Chapter Include Files.}
%
% The include files are called |cdocsch1.tex| and |cdocsch2.tex|.
%
%\iffalse
%<*samplechap1|samplechap2>
%\fi

% Optional override for |\version| flag:
%    \begin{macrocode}
%%\providecommand{\version}{final}
%    \end{macrocode}

% Include the main document:
%    \begin{macrocode}
\input{childdoc.def}
\childdocof{cdocsamp}
%    \end{macrocode}

%\iffalse
%</samplechap1|samplechap2>
%\fi
%
%\iffalse
%<*samplechap1>
%\fi
% Some text for chapter 1:
%    \begin{macrocode}
\section{one}
some text in chapter one
%    \end{macrocode}

%\iffalse
%</samplechap1>
%\fi
% Some text for chapter 2:
%\iffalse
%<*samplechap2>
%\fi
%    \begin{macrocode}
\section{two}
more text in chapter two
%    \end{macrocode}

%\iffalse
%</samplechap2>
%\fi
%
% %%%%%%%%%%%%%%%%%%%%%%%%%%%%%%%%%%%%%%
% \paragraph{Part Include Files.}
%
% The include files are called |cdocspt3.tex| and |cdocspt4.tex|.
%
%\iffalse
%<*samplepart3|samplepart4>
%\fi

% Optional override for |\version| flag:
%    \begin{macrocode}
%%\providecommand{\version}{final}
%    \end{macrocode}

% Include the main document:
%    \begin{macrocode}
\input{childdoc.def}
\childdocby{cdocsamp}
%    \end{macrocode}

%\iffalse
%</samplepart3|samplepart4>
%\fi
%
%\iffalse
%<*samplepart3>
%\fi
% Some text for part 3:
%    \begin{macrocode}
some text in part three
%    \end{macrocode}

%\iffalse
%</samplepart3>
%\fi
% Some text for part 4:
%\iffalse
%<*samplepart4>
%\fi
%    \begin{macrocode}
more text in part four
%    \end{macrocode}

%\iffalse
%</samplepart4>
%\fi
%
% %%%%%%%%%%%%%%%%%%%%%%%%%%%%%%%%%%%%%%
% \paragraph{Forwarding for a Complete Draft.}
%
% The following forwarding file |cdocsdrf.tex|
% compiles the main document in draft mode:
%\iffalse
%<*sampledraft>
%\fi
%    \begin{macrocode}
\def\version{draft}
\input{childdoc.def}
\childdocforward{cdocsamp}
%    \end{macrocode}

%\iffalse
%</sampledraft>
%\fi
%
% %%%%%%%%%%%%%%%%%%%%%%%%%%%%%%%%%%%%%%
% \paragraph{Forwarding for Final Version of the Chapters.}
%
% The following forwarding files |cdocsfn1.tex| and |cdocsfn2.tex|
% (with identical content)
% compile the final versions of the child documents
% |cdocsch1.tex| and |cdocsch2.tex|, respectively:
%\iffalse
%<*samplefinal>
%\fi
%    \begin{macrocode}
\def\version{final}
\input{childdoc.def}
\childdocforwardprefix[cdocsamp]{cdocsfn}{cdocsch}
%    \end{macrocode}

%\iffalse
%</samplefinal>
%\fi
%
% %%%%%%%%%%%%%%%%%%%%%%%%%%%%%%%%%%%%%%
% \paragraph{Command Line Processing.}
%
% The following three command lines generate the output files
% |cdocscld|, |cdocscl1| and |cdocscl2|
% which should be identical to
% |cdocsdrf|, |cdocsch1| and |cdocsfn2|, respectively:
% \begin{center}
% \begin{tabular}{l}
% |latex -jobname cdocscld \|\\
% |  "\def\version{draft}\input{childdoc.def}\childdocforward{cdocsamp}"|\\
% |latex -jobname cdocscl1 \|\\
% |  "\input{childdoc.def}\childdocforward[cdocsamp]{cdocsch1}"|\\
% |latex -jobname cdocscl2 \|\\
% |  "\def\version{final}\input{childdoc.def}\childdocforward{cdocsch2}"|
% \end{tabular}
% \end{center}
% Note that the trailing backslash on each first line
% merely continues the input to the second line
% (for convenient cut ant paste).
% Furthermore, the command |latex| can be replaced by any
% of its alternative versions such as |pdflatex|.
%
% %%%%%%%%%%%%%%%%%%%%%%%%%%%%%%%%%%%%%%%%%%%%%%%%%%%%%%%%%%%%%%%%%%%%%%%%%%%%%%
% %%%%%%%%%%%%%%%%%%%%%%%%%%%%%%%%%%%%%%%%%%%%%%%%%%%%%%%%%%%%%%%%%%%%%%%%%%%%%%
% \section{Implementation}
%\iffalse
%<*package>
%\fi
%
% This section describes the definitions file |childdoc.def|.

% The definitions cannot be loaded using |\usepackage| or |\RequirePackage|
% which has a mechanism to prevent loading a style file more than once.
% When loading the definitions by means of |\input|
% multiple instances have to be prevented manually:
%\iffalse
%This code needs to be before the `\ProvidesFile' directive
%which is defined at the beginning of this file.
%Therefore it is also placed there and commented out here.
%</package>
%<*discard>
%\fi
%    \begin{macrocode}
\ifdefined\childdocmain\endinput\fi
%    \end{macrocode}
%\iffalse
%</discard>
%<*package>
%\fi
%
% \macro{\ifchilddoc}
% \macro{\ifchilddocmanual}
% The conditional |\ifchilddoc| tells whether a
% child (true) or main (false) document is being compiled.
% The conditional |\ifchilddocmanual| tells whether
% the |\includeonly| mechanism is used (false) or
% the selection of child files must be performed manually (true).
% The definitions initialise to false:
%    \begin{macrocode}
\newif\ifchilddoc
\newif\ifchilddocmanual
%    \end{macrocode}

% \macro{\childdocname}
% \macro{\childdocjob}
% The macro |\childdocname| stores the name of the main document
% to be compiled. The macro |\childdocjob| stores the name of
% the document on which the \LaTeX{} compiler was originally invoked.
% The content of |\jobname| cannot be compared
% to filenames specified in the source due to different catcodes.
% The following code rescans |\jobname|, stores the result
% in |\childdocname| and saves a copy in |\childdocjob|:
%    \begin{macrocode}
\edef\childdocname{\scantokens\expandafter{\jobname\noexpand}}
\let\childdocjob\childdocname
%    \end{macrocode}

% \macro{\childdocdisable}
% The macro |\childdocdisable| prevents the main file
% from being processed more than once.
% At this stage, the main document command |\childdocmain|
% is assumed to be called once again where it should do nothing.
% Any subsequent call to it should prevent
% a secondary processing of the main document
% It overwrites the forwarding commands
% |\childdocof| and |\childdocforward|
% with empty macros to prevent further inclusions of the main document:
%    \begin{macrocode}
\newcommand{\childdocdisable}
{
  \renewcommand{\childdocmain}[1]{\renewcommand{\childdocmain}[1]{\endinput}}
  \renewcommand{\childdocof}[1]{}
  \renewcommand{\childdocby}[2][]{}
  \renewcommand{\childdocforward}[2][]{}
  \renewcommand{\childdocdisable}{}
}
%    \end{macrocode}

% \macro{\childdocmain}
% The macro |\childdocmain| is to be called at the top of the main file
% with nothing or the main filename (without extension) as argument.
% First, it breaks loops.
% If the argument is not empty and does not match |\childdocname|
% (which is set by the first inclusion of |childdoc.def|),
% |\ifchilddoc| is set to true, |\includeonly| is applied to the child file
% and |\jobname| is set to the main file
% (for proper handling of |.aux| files):
%    \begin{macrocode}
\newcommand{\childdocmain}[1]
{
  \childdocdisable\childdocmain{}
  \if?#1?\else
    \begingroup
      \def\childdoctmp{#1}
      \ifx\childdoctmp\childdocname
        \def\childdoctmp{}
      \else
        \def\childdoctmp
        {
          \childdoctrue
          \includeonly{\childdocname}
          \def\childdocjob{#1}
          \def\jobname{#1}
        }
      \fi
      \expandafter
    \endgroup
    \childdoctmp
  \fi
}
%    \end{macrocode}

% \macro{\childdocof}
% The command |\childdocof| redirects
% compilation to the main file |#1|.
%    \begin{macrocode}
\newcommand{\childdocof}[1]
{
  \childdocdisable
  \childdoctrue
  \includeonly{\childdocname}
  \def\jobname{#1}
  \def\childdocjob{#1}
  \input{#1}
}
%    \end{macrocode}

% \macro{\childdocby}
% The command |\childdocby| ....
%    \begin{macrocode}
\newcommand{\childdocby}[2][]
{
  \childdocdisable
  \childdoctrue
  \childdocmanualtrue
  \if?#1?\else
    \def\jobname{#2}
  \fi
  \def\childdocjob{#2}
  \input{#2}
  \endinput
}
%    \end{macrocode}

% \macro{\childdocforward}
% The command |\childdocforward| redirects
% compilation to the main file or
% (if the optional argument is given) a child file.
% Parameters are set as if the main file
% or a child file starting with |\childdocof| was compiled.
% Then compilation is handed over to the main file:
%    \begin{macrocode}
\newcommand{\childdocforward}[2][]
{
  \begingroup
    \if?#1?
      \def\childdoctmp
      {
        \def\childdocname{#2}
        \def\childdocjob{#2}
        \def\jobname{#2}
        \input{#2}
        \endinput
      }
    \else
      \def\childdoctmp
      {
        \childdocdisable
        \def\childdocname{#2}
        \childdoctrue
        \includeonly{#2}
        \def\childdocjob{#1}
        \def\jobname{#1}
        \input{#1}
        \endinput
      }
    \fi
    \expandafter
  \endgroup
  \childdoctmp
}
%    \end{macrocode}

% \macro{\childdocforwardprefix}
% The command |\childdocforwardprefix| redirects
% compilation to the main or a child file by means of a pattern.
% The prefix |#1| in the current filename is replaced by |#2|
% and the suffix of the current filename is kept
% (it is assumed that the filename does not contain the substring `|~~~|'
% which is used as a delimiter).
% Compilation is handed over to the new file by |\childdocforward|:
%    \begin{macrocode}
\newcommand{\childdocforwardprefix}[3][]
{
  \begingroup
    \def\childdocextract #2##1~~~{\def\childdoctmp{\childdocforward[#1]{#3##1}}}
    \expandafter\childdocextract\childdocname~~~
    \expandafter
  \endgroup
  \childdoctmp
}
%    \end{macrocode}

% \macro{\childdoc}
% The deprecated macro |\childdoc| is a legacy version of |\childdocmain|:
%    \begin{macrocode}
\newcommand{\childdoc}{\childdocmain}
%    \end{macrocode}

% \macro{\childdocredirect}
% The deprecated macro |\childdocredirect| is a legacy version
% of |\childdocforward| and |\childdocforwardprefix|:
%    \begin{macrocode}
\newcommand{\childdocredirect}[2][]
{
  \begingroup
    \if?#1?
      \def\childdoctmp{\childdocforward{#2}}
    \else
      \def\childdoctmp{\childdocforwardprefix{#1}{#2}}
    \fi
    \expandafter
  \endgroup
  \childdoctmp
}
%    \end{macrocode}

%\iffalse
%</package>
%\fi
%
\endinput
|\\
|\childdocmain{}|\\
\end{tabular}
\end{center}
at the very top of the main \LaTeX{} file,
in particular \emph{before} the |\documentclass| statement!
The argument of |\childdocmain| should be left empty
(but it must be present).

%%%%%%%%%%%%%%%%%%%%%%%%%%%%%%%%%%%%%%%%
\DescribeMacro{\childdocof}
Furthermore, add the commands
\begin{center}
\begin{tabular}{l}
|% \iffalse
%
% childdoc.dtx Copyright (C) 2017-2018 Niklas Beisert
%
% This work may be distributed and/or modified under the
% conditions of the LaTeX Project Public License, either version 1.3
% of this license or (at your option) any later version.
% The latest version of this license is in
%   http://www.latex-project.org/lppl.txt
% and version 1.3 or later is part of all distributions of LaTeX
% version 2005/12/01 or later.
%
% This work has the LPPL maintenance status `maintained'.
%
% The Current Maintainer of this work is Niklas Beisert.
%
% This work consists of the files childdoc.dtx and childdoc.ins
% and the derived files childdoc.def and cdocsamp.tex with
% cdocsch1.tex, cdocsch2.tex, cdocsdrf.tex, cdocsfn1.tex, cdocsfn2.tex.
%
%<package>\ifdefined\childdocmain\endinput\fi
%<package>\ProvidesFile{childdoc.def}[2018/12/30 v2.0 child document driver]
%<samplemain>\ProvidesFile{cdocsamp.tex}[2018/12/30 v2.0 sample for childdoc]
%<*driver>
%\ProvidesFile{childdoc.drv}[2018/12/30 v2.0 childdoc reference manual file]
\PassOptionsToClass{10pt,a4paper}{article}
\documentclass{ltxdoc}

\usepackage[margin=35mm]{geometry}
\usepackage{hyperref}
\usepackage{hyperxmp}
\usepackage[usenames]{color}

\hypersetup{colorlinks=true}
\hypersetup{pdfstartview=FitH}
\hypersetup{pdfpagemode=UseNone}
\hypersetup{pdfsource={}}
\hypersetup{pdflang={en-UK}}
\hypersetup{pdfcopyright={Copyright 2017-2018 Niklas Beisert.
  This work may be distributed and/or modified under the
  conditions of the LaTeX Project Public License, either version 1.3
  of this license or (at your option) any later version.}}
\hypersetup{pdflicenseurl={http://www.latex-project.org/lppl.txt}}
\hypersetup{pdfcontactaddress={ETH Zurich, ITP, HIT K,
  Wolfgang-Pauli-Strasse 27}}
\hypersetup{pdfcontactpostcode={8093}}
\hypersetup{pdfcontactcity={Zurich}}
\hypersetup{pdfcontactcountry={Switzerland}}
\hypersetup{pdfcontactemail={nbeisert@itp.phys.ethz.ch}}
\hypersetup{pdfcontacturl={http://people.phys.ethz.ch/\xmptilde nbeisert/}}

\newcommand{\secref}[1]{\hyperref[#1]{section \ref*{#1}}}

\parskip1ex
\parindent0pt
\let\olditemize\itemize
\def\itemize{\olditemize\parskip0pt}

\begin{document}

\title{The \textsf{childdoc} Package}
\hypersetup{pdftitle={The childdoc Package}}
\author{Niklas Beisert\\[2ex]
  Institut f\"ur Theoretische Physik\\
  Eidgen\"ossische Technische Hochschule Z\"urich\\
  Wolfgang-Pauli-Strasse 27, 8093 Z\"urich, Switzerland\\[1ex]
  \href{mailto:nbeisert@itp.phys.ethz.ch}
  {\texttt{nbeisert@itp.phys.ethz.ch}}}
\hypersetup{pdfauthor={Niklas Beisert}}
\hypersetup{pdfsubject={Manual for the LaTeX2e Package childdoc}}
\date{30 December 2018, \textsf{v2.0}}
\maketitle

\begin{abstract}\noindent
\textsf{childdoc} is a \LaTeXe{} package
that enables the direct compilation
of document sections included by |\include|
to individual files.
\end{abstract}

\begingroup
\parskip0ex
\tableofcontents
\endgroup

%%%%%%%%%%%%%%%%%%%%%%%%%%%%%%%%%%%%%%%%%%%%%%%%%%%%%%%%%%%%%%%%%%%%%%%%%%%%%%%%
%%%%%%%%%%%%%%%%%%%%%%%%%%%%%%%%%%%%%%%%%%%%%%%%%%%%%%%%%%%%%%%%%%%%%%%%%%%%%%%%
\section{Introduction}

\LaTeX{} provides a mechanism to structure a large document (such as a book)
into a main file and several child files (containing the chapters)
using the |\include| command.
This mechanism is beneficial for documents
which span hundreds of pages in order to
make the source file(s) more manageable.
Moreover, compilation can be restricted to
selected child files by means of the |\includeonly| command.
The latter feature can be used to reduce the compilation time while editing
(this was significantly more useful in the earlier days of \LaTeX{})
or to generate a smaller document which is easier to navigate.
Another application of |\includeonly| is to generate
documents consisting of selected parts of the complete document.

However, there are a few drawbacks of the plain |\include| mechanism:
\begin{itemize}
\item
The child files cannot be compiled on their own,
they can only be compiled via the main file.
A naive editing environment
(such as a text editor with an option
to have the current file processed by \LaTeX)
may require one to switch to the main file before compiling;
attempting to compile the child file produces errors.
\item
The main file must be modified (each time)
to adjust the |\includeonly| command
to the present needs. This easily leaves the main file in a messy state.
\item
The generated document will always carry the filename
of the main document. This is inconvenient if
several child files are to be compiled and
to be kept for distribution.
\end{itemize}

The present package provides a simple interface
to make child files individually compilable by \LaTeX{}.
Compiling a child file then has the same effect as compiling
the main file with an |\includeonly| command
to select the appropriate child.
Moreover the generated document will carry the name of the child
rather than the main file.
This resolves all three above issues.

This feature is meant to make the editing of books,
thesis documents and lecture notes somewhat more convenient.
However, the package can also be used efficiently for
composing a series of documents (such as exercise sheets)
which are typically distributed individually.
It then assists the author in generating the individual documents
(potentially in different versions)
as well as a document containing the collected series.
Another application is in developing style files
or other kinds of included material
where compilation of the style file could redirect
to a sample or test file.

%%%%%%%%%%%%%%%%%%%%%%%%%%%%%%%%%%%%%%%%%%%%%%%%%%%%%%%%%%%%%%%%%%%%%%%%%%%%%%%%
%%%%%%%%%%%%%%%%%%%%%%%%%%%%%%%%%%%%%%%%%%%%%%%%%%%%%%%%%%%%%%%%%%%%%%%%%%%%%%%%
\section{Usage}

First of all, the package \textsf{childdoc} is \emph{not} a standard
\LaTeXe{} |.sty| style file! Therefore it needs to be invoked in
a non-standard way.

%%%%%%%%%%%%%%%%%%%%%%%%%%%%%%%%%%%%%%%%%%%%%%%%%%%%%%%%%%%%%%%%%%%%%%%%%%%%%%%%
\subsection{Included Files}
\label{sec:include}

%%%%%%%%%%%%%%%%%%%%%%%%%%%%%%%%%%%%%%%%
\DescribeMacro{\childdocmain}
To use the package, add the commands
\begin{center}
\begin{tabular}{l}
|\input{childdoc.def}|\\
|\childdocmain{}|\\
\end{tabular}
\end{center}
at the very top of the main \LaTeX{} file,
in particular \emph{before} the |\documentclass| statement!
The argument of |\childdocmain| should be left empty
(but it must be present).

%%%%%%%%%%%%%%%%%%%%%%%%%%%%%%%%%%%%%%%%
\DescribeMacro{\childdocof}
Furthermore, add the commands
\begin{center}
\begin{tabular}{l}
|\input{childdoc.def}|\\
|\childdocof{|\textit{main}|}|\\
\end{tabular}
\end{center}
at the top of every child file \textit{child}
which is included by |\include{|\textit{child}|}|
from within the main file
(or at least for those files to be compiled individually).
The argument \textit{main} must be the filename of the main file.

There are a couple of
considerations in setting up the main and child documents:

%%%%%%%%%%%%%%%%%%%%%%%%%%%%%%%%%%%%%%%%
\paragraph{Restrictions.}

Please note the following restrictions:
\begin{itemize}
\item
|\childdocmain| must be called with one argument \textit{main}
to ensure compatibility with earlier version of the package.
It must either be empty (|\childdocmain{}|)
or precisely match the filename of the main file in which it is specified.
See \secref{sec:detection} for further information.
\item
The filename \textit{main} must be specified without the |.tex| extension.
\item
The filename \textit{main} is case sensitive
(even in case-insensitive file systems)
due to internal string comparison.
\item
The argument \textit{main} should be fully expanded, it cannot be a macro.
\item
Subdirectories and special characters should be avoided in filenames.
\item
The command |\childdocmain{|\textit{main}|}| must be followed by a whitespace.
It should not be followed immediately by another command
or by a comment mark `|%|'.
This is because the \TeX{} parser reads the token immediately following
the argument of |\childdocmain| and puts it
at the beginning of every child section;
however, a white\-space is ignored.
\end{itemize}

%%%%%%%%%%%%%%%%%%%%%%%%%%%%%%%%%%%%%%%%
\paragraph{Content of Main File.}

It is advisable to place all content in the child files included by |\include|.
Any output contained in the main file will appear in all child documents
unless suppressed manually;
it cannot be suppressed automatically by the |\includeonly| directive
and thus should normally be avoided.
A method to include some content in the main file
by means of conditional processing is described in \secref{sec:conditional}.

%%%%%%%%%%%%%%%%%%%%%%%%%%%%%%%%%%%%%%%%
\paragraph{Page Numbering.}

When only a part of the document is compiled,
the appropriate numbering of pages
(as well as other status parameters)
is determined from the |.aux| files.
The latter contain information from previous passes.
However this information needs to propagate through
all intermediate child documents.
Therefore the page numbering in child documents may well
be inconsistent until the complete document is compiled at least once.

A useful (if unconventional) way to always ensure a consistent
page numbering is to restart the numbering in each child document
and denote the pages by `\textit{child}|.|\textit{page}'
where \textit{child} represents the chapter/section number of the child file.
This can be achieved by the command
|\numberwithin{page}{|\textit{child}|}|
of the \textsf{amsmath} package
where \textit{child} can be |chapter| or |section|
depending on the chosen structuring.
Alternatively, one can modify the macro |\thepage| appropriately
and reset the counter |page| at the start of each child file.

%%%%%%%%%%%%%%%%%%%%%%%%%%%%%%%%%%%%%%%%%%%%%%%%%%%%%%%%%%%%%%%%%%%%%%%%%%%%%%%%
\subsection{Conditional Processing}
\label{sec:conditional}

The package provides a mechanism to compile different versions
of a document. To customise the versions further some conditional processing
can come in handy to distinguish which version is being compiled.
The package provides two macros to describe the compilation context:

%%%%%%%%%%%%%%%%%%%%%%%%%%%%%%%%%%%%%%%%
\DescribeMacro{\ifchilddoc}
The conditional |\ifchilddoc| distinguishes between the compilation of
child documents and the main document:
%
\begin{center}
|\ifchilddoc |\textit{child-code}| |[|\||else |\textit{main-code}]| \||fi|
\end{center}

%%%%%%%%%%%%%%%%%%%%%%%%%%%%%%%%%%%%%%%%
\DescribeMacro{\childdocname}
\DescribeMacro{\childdocjob}
The macro |\childdocname| contains the filename (without extension)
of the main or child file being processed.
Note that |\childdocjob| will always contain the name of the main file.

%%%%%%%%%%%%%%%%%%%%%%%%%%%%%%%%%%%%%%%%
\paragraph{Title Page.}

Conditional processing can be used to include a title or banner page
in the main document when proper precautions are taken.
Importantly, the code in the main file should ensure that the page counter
(as well as other status parameters which are stored in the |.aux| files)
takes the same value after the conditional processing.
Otherwise the page numbers may take divergent values
depending on which part is compiled.

For example, a title page could be declared by:
%
\begin{center}
\begin{tabular}{l}
|\ifchilddoc\||else|\\
|\addtocounter{page}{-1}|\\
\textit{code for title page}\\
|\newpage|\\
|\||fi|
\end{tabular}
\end{center}
%
A banner page for the child documents can be generated by:
%
\begin{center}
\begin{tabular}{l}
|\ifchilddoc|\\
|\addtocounter{page}{-1}|\\
\textit{code for banner page}\\
|\newpage|\\
|\||fi|
\end{tabular}
\end{center}
%
Here one could write a message such as:
\begin{center}
|This is the part \childdocname{} of \childdocjob{}.|
\end{center}

%%%%%%%%%%%%%%%%%%%%%%%%%%%%%%%%%%%%%%%%%%%%%%%%%%%%%%%%%%%%%%%%%%%%%%%%%%%%%%%%
\subsection{Flags}
\label{sec:flags}

The package makes it easy to generate different versions
of the main or child documents.
To this end compilation flags can be defined
and assigned different default values.
They will be particularly useful in conjunction
with the forwarding mechanism described in \secref{sec:forward}.

For example, it may be useful to have a flag |\version|
which can be set to |draft| or |final|.
The document source will contain some conditional code
depending on the value of |\version|.
Suppose further, the flag should default to |final| for the main file
and to |draft| for child files
which is a natural assignment for editing the document.
This is achieved by placing the following code
in the preamble of the main document
(below the |\childdocmain| directive):
%
\begin{center}
\begin{tabular}{l}
|\ifchilddoc|\\
|\providecommand{\version}{draft}|\\
|\||else|\\
|\providecommand{\version}{final}|\\
|\||fi|
\end{tabular}
\end{center}
%
The definition by |\providecommand| makes sure
that previous definitions are not overwritten.
Further statements |\providecommand{\version}{...}|
can thus be added before the above code to override it.

For the main file, one might add a line
(between |\childdocmain| and the above block)
%
\begin{center}
|%\ifchilddoc\||else\providecommand{\version}{draft}\||fi|
\end{center}
%
which can be uncommented to produce a draft version.
Likewise one can add a line to the very top of a child file
(above the |\childdocof{|\textit{main}|}| directive)
%
\begin{center}
|%\providecommand{\version}{final}|
\end{center}
%
which can be uncommented to produce the final version of this child document.

%%%%%%%%%%%%%%%%%%%%%%%%%%%%%%%%%%%%%%%%%%%%%%%%%%%%%%%%%%%%%%%%%%%%%%%%%%%%%%%%
\subsection{Forwarding}
\label{sec:forward}

Different versions of the main or child documents
using compilation flags as described in \secref{sec:flags}
can be (permanently) stored in different files
for convenient compilation, viewing and distribution.
To this end, the package defines a command
to pass on compilation to a different file:

%%%%%%%%%%%%%%%%%%%%%%%%%%%%%%%%%%%%%%%%
\DescribeMacro{\childdocforward}
The command |\childdocforward| redirects processing to
another source file:
%
\begin{center}
\begin{tabular}{l}
|\input{childdoc.def}|\\
|\childdocforward[|\textit{main}|]{|\textit{dest}|}|\\
\end{tabular}
\end{center}
%
The argument \textit{dest} is the destination file
(without extension).
It should be the main file or one of the child files.
Note that further \textsf{childdoc} directives
such as |\childdocof| and |\childdocforward|
in the indicated file will be processed in this form.
The optional argument \textit{main}
passes on directly to the main file \textit{main}
while pretending to compile the child \textit{dest}.
This form behaves as if \textit{dest}
issues |\childdocof{|\textit{main}|}| right away,
and no further \textsf{childdoc} directives will be processed.

%%%%%%%%%%%%%%%%%%%%%%%%%%%%%%%%%%%%%%%%
\DescribeMacro{\...prefix}
In the alternative form |\childdocforwardprefix|,
%
\begin{center}
\begin{tabular}{l}
|\input{childdoc.def}|\\
|\childdocforwardprefix[|\textit{main}|]{|\textit{prefix}|}{|\textit{dest}|}|
\end{tabular}
\end{center}
%
the destination file is determined by a pattern
depending on the current file:
To make this work, the current file must be called
`{\textit{prefix}\hspace{0.2em}\textit{suffix}}'
with \textit{prefix} matching precisely the argument.
Processing is then passed on to the file
`{\textit{dest}\hspace{0.2em}\textit{suffix}}'.
Surely, the same effect is achieved by
directly specifying the
argument `{\textit{dest}\hspace{0.2em}\textit{suffix}}'
in the first form.
However, that requires to set up a different file
for each child. With the alternative form of the command
all these files can have exactly the same content
which simplifies setting them up and maintaining them.

For example, the following file |draft.tex|
with a compilation flag |\version| as described in \secref{sec:flags}
compiles the main document as a draft:
%
\begin{center}
\begin{tabular}{l}
|\def\version{draft}|\\
|\input{childdoc.def}|\\
|\childdocforward{|\textit{main}|}|
\end{tabular}
\end{center}
%
Likewise, the following files |final|\textit{nn}|.tex|
compile the final version of the child document
|child|\textit{nn}|.tex|:
%
\begin{center}
\begin{tabular}{l}
|\def\version{final}|\\
|\input{childdoc.def}|\\
|\childdocforwardprefix{final}{child}|
\end{tabular}
\end{center}
%

Note that when several versions of a main file and/or of each child file
are to be generated, it may be convenient to set up a |Makefile| or
shell script to automatise the process.

%%%%%%%%%%%%%%%%%%%%%%%%%%%%%%%%%%%%%%%%%%%%%%%%%%%%%%%%%%%%%%%%%%%%%%%%%%%%%%%%
\subsection{Command Line Processing}
\label{sec:commandline}

The effect of redirection files can also be achieved by invoking
the \LaTeX{} compiler with a more elaborate command line.
Most conveniently this should be done as part
of a shell script or a |Makefile|.

When using \textsf{childdoc} in the main file, the following
command lines effectively perform a redirection
(note that depending on the shell being used,
backslashes may have to be doubled: `|\|' $\to$ `|\\|'):
%
\begin{center}
|... -jobname "|\textit{target}|" |\\|"|[\textit{flags}]%
|\input{childdoc.def}\childdocforward[|\textit{main}|]{|\textit{dest}|}"|
\end{center}
%
Here \textit{target} is the name of the output file,
\textit{main} is the name of the main file
and \textit{dest} is the name of the main or child file to be processed
(all filenames without extensions).
The optional argument \textit{main} can be omitted
if \textit{main} matches \textit{dest}.
Optionally, compilation \textit{flags} can be defined via |\def| commands.
This command line makes the \TeX{} engine believe
it is compiling the file \textit{target}
whose content is specified as the latter parameter.
The provided code then forwards the processing to
\textit{main} or \textit{dest} as described in \secref{sec:forward}.

%%%%%%%%%%%%%%%%%%%%%%%%%%%%%%%%%%%%%%%%%%%%%%%%%%%%%%%%%%%%%%%%%%%%%%%%%%%%%%%%
\subsection{Include by Input}
\label{sec:input}

Including child documents by |\include| has some restrictions by design.
Most notably, the content of a child document always occupies
its own set of pages; pages cannot be shared between child documents.
Usually, this behaviour makes perfect sense
because each child document contain an essential part of the document.
However, in some situations it may be desirable to compose
a document from a collection of parts
without having mandatory page breaks between then.
For this case, the package
provides a mechanism to include parts
by |\input| which can also be processed individually.
However, by construction this mechanism
requires manual handling of the content to be output.

%%%%%%%%%%%%%%%%%%%%%%%%%%%%%%%%%%%%%%%%
\DescribeMacro{\ifchilddocmanual}
The main file should be prepared as usual, see \secref{sec:include}.
However, the document body must make a distinction
between processing of an individual part and of the main document, e.g.:
%
\begin{center}
\begin{tabular}{l}
|\ifchilddocmanual|\\
|\input{\childdocname}|\\
|\||else|\\
\textit{document body with }|\input{|\textit{part}|}|\\
|\||fi|
\end{tabular}
\end{center}
%
The conditional |\ifchilddocmanual| is true whenever
a part to be included by |\input| is being compiled,
and the name of the part is stored in |\childdocname|.

%%%%%%%%%%%%%%%%%%%%%%%%%%%%%%%%%%%%%%%%
\DescribeMacro{\childdocby}
Each part to be included by |\input| should start with:
%
\begin{center}
\begin{tabular}{l}
|\input{childdoc.def}|\\
|\childdocby{|\textit{main}|}|\\
\end{tabular}
\end{center}
%
The directive |\childdocby| is similar to |\childdocof|
described in \secref{sec:include},
but the subsequent selection of content must be done manually.
To that end, both |\ifchilddoc| and |\ifchilddocmanual|
will be true upon processing of a part,
and the name of the part is stored in |\childdocname|.
Note that |\jobname| will be set to the filename of the current part
so that each part receives an individual |.aux| file
that does not interfere with the |.aux| file(s) of the main document.
This behaviour can be altered by the alternative form
|\childdocby[*]{|\textit{main}|}| (with a non-empty optional argument)
which uses the |.aux| file of the main document
by setting |\jobname| to \textit{main}.

%%%%%%%%%%%%%%%%%%%%%%%%%%%%%%%%%%%%%%%%%%%%%%%%%%%%%%%%%%%%%%%%%%%%%%%%%%%%%%%%
\subsection{Driver Development}
\label{sec:driver}

The \textsf{childdoc} mechanism can also be use for the development
of definition files such as \LaTeX{} styles or classes.
This case differs from the above setup with multiple parts
included by |\include| in that no |\includeonly| should be invoked.
This can be achieved by starting the include file
(before |\ProvidesPackage|) with:
%
\begin{center}
\begin{tabular}{l}
|\input{childdoc.def}|\\
|\childdocforward{|\textit{main}|}|\\
\end{tabular}
\end{center}
%
or alternatively with:
%
\begin{center}
\begin{tabular}{l}
|\input{childdoc.def}|\\
|\childdocby{|\textit{main}|}|\\
\end{tabular}
\end{center}
%
Both forms have slightly different effects as described above.
The main file is prepared as usual, see \secref{sec:include}.

%%%%%%%%%%%%%%%%%%%%%%%%%%%%%%%%%%%%%%%%%%%%%%%%%%%%%%%%%%%%%%%%%%%%%%%%%%%%%%%%
\subsection{Legacy Detection}
\label{sec:detection}

The directive |\childdocmain| in the main file can detect
whether the complete document or merely a child is to be compiled
even without using the directive |\childdocof|.
This method is deprecated because it is less robust
and there is no compelling reason to use it;
it is merely provided for backward compatibility
and it may be removed in future versions.

If the detection mechanism is to be used,
it is mandatory to correctly specify
the filename of the main file as the argument of |\childdocmain|:
%
\begin{center}
\begin{tabular}{l}
|\input{childdoc.def}|\\
|\childdocmain{|\textit{main}|}|\\
\end{tabular}
\end{center}
%
If |\jobname| does not match the argument \textit{main} of |\childdocmain|,
it is assumed that |\jobname| points to the child file to be compiled.
When using |\childdocmain| with the main file specified as argument,
it suffices to start a child file
with just |\input{|\textit{main}|}|
without loading of the package and using |\childdocof|.
If instead all processing is done
with the appropriate \textsf{childdoc} directives,
the argument of \textit{main} of |\childdocmain| can be empty.

An alternative version of the command line processing described
in \secref{sec:commandline} using the detection mechanism reads:
%
\begin{center}
|... -jobname "|\textit{target}|" "|[\textit{flags}]%
[|\def\jobname{|\textit{dest}|}|]|\input{|\textit{main}|}"|
\end{center}

%%%%%%%%%%%%%%%%%%%%%%%%%%%%%%%%%%%%%%%%%%%%%%%%%%%%%%%%%%%%%%%%%%%%%%%%%%%%%%%%
\subsection{Manual Code}
\label{sec:manual}

In case one cannot be certain whether the definitions file |childdoc.def|
is installed on the target \TeX{} distribution
and one prefers not to ship it,
it is conceivable to paste a few relevant commands into the sources.

To that end, drop all statements |\input{childdoc.def}|
and perform the replacements as outlined below.
Instead of |\childdocmain{|\textit{main}|}| add the following code
to the top of the main file:
%
\begin{center}
\begin{tabular}{l}
|\||ifdefined\childdocname\endinput\||fi\newif\ifchilddoc|\\
|\edef\childdocname{\scantokens\expandafter{\jobname\noexpand}}|\\
|\def\childdocmain{|\textit{main}|}\||ifx\childdocmain\childdocname\||else|\\
|\childdoctrue\includeonly{\childdocname}\let\jobname\childdocmain\||fi|\\
\end{tabular}
\end{center}
%
Instead of |\childdocof{|\textit{main}|}| just include the main file
at the top of each child file:
%
\begin{center}
|\input{|\textit{main}|}|
\end{center}
%
A simple redirection |\childdocforward{|\textit{dest}|}| is achieved by:
%
\begin{center}
|\def\jobname{|\textit{dest}|}\input{\jobname}|
\end{center}
%
The redirection with prefix
|\childdocforwardprefix[|\textit{prefix}|]{|\textit{dest}|}|
is accomplished by:
%
\begin{center}
\begin{tabular}{l}
|{\edef\jobname{\scantokens\expandafter{\jobname\noexpand}}|\\
|\def\redirectjob |\textit{prefix}|#1~~~{\gdef\jobname{|\textit{dest}|#1}}|\\
|\expandafter\redirectjob\jobname~~~}\input{\jobname}|
\end{tabular}
\end{center}

In an alternative approach,
child documents can be compiled by a specific command line
without additional code or specific definitions:
%
\begin{center}
|... -jobname "|\textit{target}|" "|[\textit{flags}]%
|\includeonly{|\textit{dest}|}\input{|\textit{main}|}"|
\end{center}
%

%%%%%%%%%%%%%%%%%%%%%%%%%%%%%%%%%%%%%%%%%%%%%%%%%%%%%%%%%%%%%%%%%%%%%%%%%%%%%%%%
%%%%%%%%%%%%%%%%%%%%%%%%%%%%%%%%%%%%%%%%%%%%%%%%%%%%%%%%%%%%%%%%%%%%%%%%%%%%%%%%
\section{Information}

%%%%%%%%%%%%%%%%%%%%%%%%%%%%%%%%%%%%%%%%%%%%%%%%%%%%%%%%%%%%%%%%%%%%%%%%%%%%%%%%
\subsection{Copyright}

Copyright \copyright{} 2017--2018 Niklas Beisert

This work may be distributed and/or modified under the
conditions of the \LaTeX{} Project Public License, either version 1.3
of this license or (at your option) any later version.
The latest version of this license is in
  \url{http://www.latex-project.org/lppl.txt}
and version 1.3 or later is part of all distributions of \LaTeX{}
version 2005/12/01 or later.

This work has the LPPL maintenance status `maintained'.

The Current Maintainer of this work is Niklas Beisert.

This work consists of the files |README.txt|, |childdoc.ins| and |childdoc.dtx|
as well as the derived files |childdoc.def|, |cdocsamp.tex|
with |cdocsch1.tex|, |cdocsch2.tex|, |cdocspt3.tex|, |cdocspt4.tex|,
|cdocsdrf.tex|, |cdocsfn1.tex|, |cdocsfn2.tex|
as well as |childdoc.pdf|.

%%%%%%%%%%%%%%%%%%%%%%%%%%%%%%%%%%%%%%%%%%%%%%%%%%%%%%%%%%%%%%%%%%%%%%%%%%%%%%%%
\subsection{Files and Installation}

The package consists of the files:
%
\begin{center}
\begin{tabular}{ll}
    |README.txt|   & readme file \\
    |childdoc.ins| & installation file \\
    |childdoc.dtx| & source file \\
    |childdoc.def| & definition file \\
    |cdocsamp.tex| & sample main file \\
    |cdocsch1.tex| & sample include file \\
    |cdocsch2.tex| & sample include file \\
    |cdocspt3.tex| & sample part file \\
    |cdocspt4.tex| & sample part file \\
    |cdocsdrf.tex| & sample redirection file \\
    |cdocsfn1.tex| & sample redirection file \\
    |cdocsfn2.tex| & sample redirection file \\
    |childdoc.pdf| & manual
\end{tabular}
\end{center}
%
The distribution consists of the files
|README.txt|, |childdoc.ins| and |childdoc.dtx|.
%
\begin{itemize}
\item
Run (pdf)\LaTeX{} on |childdoc.dtx|
to compile the manual |childdoc.pdf| (this file).
\item
Run \LaTeX{} on |childdoc.ins| to create the definitions file |childdoc.def|
and the sample |cdocsamp.tex| with include files
|cdocsch1.tex|, |cdocsch2.tex|, |cdocspt3.tex|, |cdocspt4.tex|,
|cdocsdrf.tex|, |cdocsfn1.tex|, |cdocsfn2.tex|.
Then copy the file |childdoc.def| to an appropriate directory of your \LaTeX{}
distribution, e.g.\ \textit{texmf-root}|/tex/latex/childdoc|.
\end{itemize}

%%%%%%%%%%%%%%%%%%%%%%%%%%%%%%%%%%%%%%%%%%%%%%%%%%%%%%%%%%%%%%%%%%%%%%%%%%%%%%%%
\subsection{Related CTAN Packages}

There are several other packages which offer a similar functionality:
%
\begin{itemize}
\item
The packages
\href{http://ctan.org/pkg/docmute}{\textsf{docmute}},
\href{http://ctan.org/pkg/includex}{\textsf{includex}} and
\href{http://ctan.org/pkg/standalone}{\textsf{standalone}}
provide commands to include only the document body of
a child file thus allowing both files to be compiled individually.
\item
The packages \href{http://ctan.org/pkg/subdocs}{\textsf{subdocs}}
and \href{http://ctan.org/pkg/subfiles}{\textsf{subfiles}}
provide structures in which the main and child documents can be
encapsulated and allowing them to be compiled individually.
The inclusion mechanism is different from the conventional |\include|.
\item
The package \href{http://ctan.org/pkg/combine}{\textsf{combine}}
is an elaborate solution to combine several documents into one.
\end{itemize}
%
See also the CTAN topic \href{http://ctan.org/topic/subdocs}{\textsf{subdocs}}
for further related packages.
The present package differs from the above solutions in that
a document structure constructed with the conventional |\include| mechanism
just needs two extra commands at the top of every file
such that all constituent files can be compiled individually.

%%%%%%%%%%%%%%%%%%%%%%%%%%%%%%%%%%%%%%%%%%%%%%%%%%%%%%%%%%%%%%%%%%%%%%%%%%%%%%%%
%\subsection{Feature Suggestions}
%
%The following is a list of features which may be useful for future
%versions of this package:
%%
%\begin{itemize}
%\item
%\ldots
%\end{itemize}

%%%%%%%%%%%%%%%%%%%%%%%%%%%%%%%%%%%%%%%%%%%%%%%%%%%%%%%%%%%%%%%%%%%%%%%%%%%%%%%%
\subsection{Revision History}

%%%%%%%%%%%%%%%%%%%%%%%%%%%%%%%%%%%%%%%%
\paragraph{v2.0:} 2018/12/30

\begin{itemize}
\item
immediate forward processing
\item
added |\childdocby| mechanism
\item
manual restructured
\end{itemize}

%%%%%%%%%%%%%%%%%%%%%%%%%%%%%%%%%%%%%%%%
\paragraph{v1.6:} 2018/01/17

\begin{itemize}
\item
application for development of include files
\item
corrections to manual
\end{itemize}

%%%%%%%%%%%%%%%%%%%%%%%%%%%%%%%%%%%%%%%%
\paragraph{v1.5:} 2017/05/21

\begin{itemize}
\item
more complete structuring introduced
\item
|\childdocof| introduced
\item
|\childdoc| renamed to |\childdocmain|
\item
|\childredirect| renamed to |\childdocforward| and |\childdocforwardprefix|
and functionality expanded
\end{itemize}

%%%%%%%%%%%%%%%%%%%%%%%%%%%%%%%%%%%%%%%%
\paragraph{v1.0:} 2017/04/27

\begin{itemize}
\item
manual and install package
\item
first version published on CTAN
\end{itemize}

%%%%%%%%%%%%%%%%%%%%%%%%%%%%%%%%%%%%%%%%
\paragraph{v0.6:} 2017/04/26

\begin{itemize}
\item
redirection mechanism added
\end{itemize}

%%%%%%%%%%%%%%%%%%%%%%%%%%%%%%%%%%%%%%%%
\paragraph{v0.5:} 2017/04/26

\begin{itemize}
\item
functionality in definition file
\end{itemize}


%%%%%%%%%%%%%%%%%%%%%%%%%%%%%%%%%%%%%%%%%%%%%%%%%%%%%%%%%%%%%%%%%%%%%%%%%%%%%%%%
%%%%%%%%%%%%%%%%%%%%%%%%%%%%%%%%%%%%%%%%%%%%%%%%%%%%%%%%%%%%%%%%%%%%%%%%%%%%%%%%
%%%%%%%%%%%%%%%%%%%%%%%%%%%%%%%%%%%%%%%%%%%%%%%%%%%%%%%%%%%%%%%%%%%%%%%%%%%%%%%%
\appendix

\settowidth\MacroIndent{\rmfamily\scriptsize 000\ }

 \DocInput{childdoc.dtx}

\end{document}
%</driver>
% \fi
%
% %%%%%%%%%%%%%%%%%%%%%%%%%%%%%%%%%%%%%%%%%%%%%%%%%%%%%%%%%%%%%%%%%%%%%%%%%%%%%%
% %%%%%%%%%%%%%%%%%%%%%%%%%%%%%%%%%%%%%%%%%%%%%%%%%%%%%%%%%%%%%%%%%%%%%%%%%%%%%%
% \section{Sample}
%\iffalse
%<*samplemain>
%\fi
%
% The following presents a sample document
% with two chapters, two parts, a title page,
% a compile flag as well as three forwarding files to set the flag.
% It consists of eight |.tex| files:
% \begin{center}
% \begin{tabular}{ll}
% |cdocsamp.tex|&main file\\
% |cdocsch1.tex|&include file for chapter 1\\
% |cdocsch2.tex|&include file for chapter 2\\
% |cdocspt3.tex|&include file for part 3\\
% |cdocspt4.tex|&include file for part 4\\
% |cdocsdrf.tex|&forwarding file for main file in draft mode\\
% |cdocsfi1.tex|&forwarding file for final version of chapter 1\\
% |cdocsfi2.tex|&forwarding file for final version of chapter 2\\
% \end{tabular}
% \end{center}
% Each of the eight files can be compiled directly by the \LaTeX{} compiler.
%
% %%%%%%%%%%%%%%%%%%%%%%%%%%%%%%%%%%%%%%
% \paragraph{Main File.}
%
% The main file is called |cdocsamp.tex|.
%
% Load the \textsf{childdoc} definitions and
% declare the filename for the main document:
%    \begin{macrocode}
\input{childdoc.def}
\childdocmain{}
%    \end{macrocode}

% Optional override for |\version| flag:
%    \begin{macrocode}
%%\ifchilddoc\else\providecommand{\version}{draft}\fi
%    \end{macrocode}

% Define the default values for the |\version| flag
% (|final| for the main file and |draft| for childs):
%    \begin{macrocode}
\ifchilddoc
\providecommand{\version}{draft}
\else
\providecommand{\version}{final}
\fi
%    \end{macrocode}

% Load the standard document class:
%    \begin{macrocode}
\documentclass[12pt]{article}
%    \end{macrocode}

% Start the document body:
%    \begin{macrocode}
\begin{document}
%    \end{macrocode}

% Declare a title page.
% Print title, part of document being processed and version flag:
%    \begin{macrocode}
\addtocounter{page}{-1}
\begin{center}
{\LARGE\bfseries{}childdoc example\par}
\vspace{1cm}
\ifchilddoc
\ifchilddocmanual part\else chapter\fi:
`\childdocname' of `\childdocjob'\par
\else
main document: `\childdocjob'\par
\fi
version: \version\par
\end{center}
\newpage
%    \end{macrocode}

% Manually include selected file,
% otherwise process as usual:
%    \begin{macrocode}
\ifchilddocmanual
\section*{part `\childdocname'}
\input{\childdocname}
\else
%    \end{macrocode}

% Include the two chapters:
%    \begin{macrocode}
\include{cdocsch1}
\include{cdocsch2}
%    \end{macrocode}

% Include the two parts unless only chapters should be displayed:
%    \begin{macrocode}
\ifchilddoc\else
\section{part three}
\input{cdocspt3}
\section{part four}
\input{cdocspt4}
\fi
%    \end{macrocode}

% Process as usual until here:
%    \begin{macrocode}
\fi
%    \end{macrocode}

% End of document body:
%    \begin{macrocode}
\end{document}
%    \end{macrocode}
%\iffalse
%</samplemain>
%\fi
%
% %%%%%%%%%%%%%%%%%%%%%%%%%%%%%%%%%%%%%%
% \paragraph{Chapter Include Files.}
%
% The include files are called |cdocsch1.tex| and |cdocsch2.tex|.
%
%\iffalse
%<*samplechap1|samplechap2>
%\fi

% Optional override for |\version| flag:
%    \begin{macrocode}
%%\providecommand{\version}{final}
%    \end{macrocode}

% Include the main document:
%    \begin{macrocode}
\input{childdoc.def}
\childdocof{cdocsamp}
%    \end{macrocode}

%\iffalse
%</samplechap1|samplechap2>
%\fi
%
%\iffalse
%<*samplechap1>
%\fi
% Some text for chapter 1:
%    \begin{macrocode}
\section{one}
some text in chapter one
%    \end{macrocode}

%\iffalse
%</samplechap1>
%\fi
% Some text for chapter 2:
%\iffalse
%<*samplechap2>
%\fi
%    \begin{macrocode}
\section{two}
more text in chapter two
%    \end{macrocode}

%\iffalse
%</samplechap2>
%\fi
%
% %%%%%%%%%%%%%%%%%%%%%%%%%%%%%%%%%%%%%%
% \paragraph{Part Include Files.}
%
% The include files are called |cdocspt3.tex| and |cdocspt4.tex|.
%
%\iffalse
%<*samplepart3|samplepart4>
%\fi

% Optional override for |\version| flag:
%    \begin{macrocode}
%%\providecommand{\version}{final}
%    \end{macrocode}

% Include the main document:
%    \begin{macrocode}
\input{childdoc.def}
\childdocby{cdocsamp}
%    \end{macrocode}

%\iffalse
%</samplepart3|samplepart4>
%\fi
%
%\iffalse
%<*samplepart3>
%\fi
% Some text for part 3:
%    \begin{macrocode}
some text in part three
%    \end{macrocode}

%\iffalse
%</samplepart3>
%\fi
% Some text for part 4:
%\iffalse
%<*samplepart4>
%\fi
%    \begin{macrocode}
more text in part four
%    \end{macrocode}

%\iffalse
%</samplepart4>
%\fi
%
% %%%%%%%%%%%%%%%%%%%%%%%%%%%%%%%%%%%%%%
% \paragraph{Forwarding for a Complete Draft.}
%
% The following forwarding file |cdocsdrf.tex|
% compiles the main document in draft mode:
%\iffalse
%<*sampledraft>
%\fi
%    \begin{macrocode}
\def\version{draft}
\input{childdoc.def}
\childdocforward{cdocsamp}
%    \end{macrocode}

%\iffalse
%</sampledraft>
%\fi
%
% %%%%%%%%%%%%%%%%%%%%%%%%%%%%%%%%%%%%%%
% \paragraph{Forwarding for Final Version of the Chapters.}
%
% The following forwarding files |cdocsfn1.tex| and |cdocsfn2.tex|
% (with identical content)
% compile the final versions of the child documents
% |cdocsch1.tex| and |cdocsch2.tex|, respectively:
%\iffalse
%<*samplefinal>
%\fi
%    \begin{macrocode}
\def\version{final}
\input{childdoc.def}
\childdocforwardprefix[cdocsamp]{cdocsfn}{cdocsch}
%    \end{macrocode}

%\iffalse
%</samplefinal>
%\fi
%
% %%%%%%%%%%%%%%%%%%%%%%%%%%%%%%%%%%%%%%
% \paragraph{Command Line Processing.}
%
% The following three command lines generate the output files
% |cdocscld|, |cdocscl1| and |cdocscl2|
% which should be identical to
% |cdocsdrf|, |cdocsch1| and |cdocsfn2|, respectively:
% \begin{center}
% \begin{tabular}{l}
% |latex -jobname cdocscld \|\\
% |  "\def\version{draft}\input{childdoc.def}\childdocforward{cdocsamp}"|\\
% |latex -jobname cdocscl1 \|\\
% |  "\input{childdoc.def}\childdocforward[cdocsamp]{cdocsch1}"|\\
% |latex -jobname cdocscl2 \|\\
% |  "\def\version{final}\input{childdoc.def}\childdocforward{cdocsch2}"|
% \end{tabular}
% \end{center}
% Note that the trailing backslash on each first line
% merely continues the input to the second line
% (for convenient cut ant paste).
% Furthermore, the command |latex| can be replaced by any
% of its alternative versions such as |pdflatex|.
%
% %%%%%%%%%%%%%%%%%%%%%%%%%%%%%%%%%%%%%%%%%%%%%%%%%%%%%%%%%%%%%%%%%%%%%%%%%%%%%%
% %%%%%%%%%%%%%%%%%%%%%%%%%%%%%%%%%%%%%%%%%%%%%%%%%%%%%%%%%%%%%%%%%%%%%%%%%%%%%%
% \section{Implementation}
%\iffalse
%<*package>
%\fi
%
% This section describes the definitions file |childdoc.def|.

% The definitions cannot be loaded using |\usepackage| or |\RequirePackage|
% which has a mechanism to prevent loading a style file more than once.
% When loading the definitions by means of |\input|
% multiple instances have to be prevented manually:
%\iffalse
%This code needs to be before the `\ProvidesFile' directive
%which is defined at the beginning of this file.
%Therefore it is also placed there and commented out here.
%</package>
%<*discard>
%\fi
%    \begin{macrocode}
\ifdefined\childdocmain\endinput\fi
%    \end{macrocode}
%\iffalse
%</discard>
%<*package>
%\fi
%
% \macro{\ifchilddoc}
% \macro{\ifchilddocmanual}
% The conditional |\ifchilddoc| tells whether a
% child (true) or main (false) document is being compiled.
% The conditional |\ifchilddocmanual| tells whether
% the |\includeonly| mechanism is used (false) or
% the selection of child files must be performed manually (true).
% The definitions initialise to false:
%    \begin{macrocode}
\newif\ifchilddoc
\newif\ifchilddocmanual
%    \end{macrocode}

% \macro{\childdocname}
% \macro{\childdocjob}
% The macro |\childdocname| stores the name of the main document
% to be compiled. The macro |\childdocjob| stores the name of
% the document on which the \LaTeX{} compiler was originally invoked.
% The content of |\jobname| cannot be compared
% to filenames specified in the source due to different catcodes.
% The following code rescans |\jobname|, stores the result
% in |\childdocname| and saves a copy in |\childdocjob|:
%    \begin{macrocode}
\edef\childdocname{\scantokens\expandafter{\jobname\noexpand}}
\let\childdocjob\childdocname
%    \end{macrocode}

% \macro{\childdocdisable}
% The macro |\childdocdisable| prevents the main file
% from being processed more than once.
% At this stage, the main document command |\childdocmain|
% is assumed to be called once again where it should do nothing.
% Any subsequent call to it should prevent
% a secondary processing of the main document
% It overwrites the forwarding commands
% |\childdocof| and |\childdocforward|
% with empty macros to prevent further inclusions of the main document:
%    \begin{macrocode}
\newcommand{\childdocdisable}
{
  \renewcommand{\childdocmain}[1]{\renewcommand{\childdocmain}[1]{\endinput}}
  \renewcommand{\childdocof}[1]{}
  \renewcommand{\childdocby}[2][]{}
  \renewcommand{\childdocforward}[2][]{}
  \renewcommand{\childdocdisable}{}
}
%    \end{macrocode}

% \macro{\childdocmain}
% The macro |\childdocmain| is to be called at the top of the main file
% with nothing or the main filename (without extension) as argument.
% First, it breaks loops.
% If the argument is not empty and does not match |\childdocname|
% (which is set by the first inclusion of |childdoc.def|),
% |\ifchilddoc| is set to true, |\includeonly| is applied to the child file
% and |\jobname| is set to the main file
% (for proper handling of |.aux| files):
%    \begin{macrocode}
\newcommand{\childdocmain}[1]
{
  \childdocdisable\childdocmain{}
  \if?#1?\else
    \begingroup
      \def\childdoctmp{#1}
      \ifx\childdoctmp\childdocname
        \def\childdoctmp{}
      \else
        \def\childdoctmp
        {
          \childdoctrue
          \includeonly{\childdocname}
          \def\childdocjob{#1}
          \def\jobname{#1}
        }
      \fi
      \expandafter
    \endgroup
    \childdoctmp
  \fi
}
%    \end{macrocode}

% \macro{\childdocof}
% The command |\childdocof| redirects
% compilation to the main file |#1|.
%    \begin{macrocode}
\newcommand{\childdocof}[1]
{
  \childdocdisable
  \childdoctrue
  \includeonly{\childdocname}
  \def\jobname{#1}
  \def\childdocjob{#1}
  \input{#1}
}
%    \end{macrocode}

% \macro{\childdocby}
% The command |\childdocby| ....
%    \begin{macrocode}
\newcommand{\childdocby}[2][]
{
  \childdocdisable
  \childdoctrue
  \childdocmanualtrue
  \if?#1?\else
    \def\jobname{#2}
  \fi
  \def\childdocjob{#2}
  \input{#2}
  \endinput
}
%    \end{macrocode}

% \macro{\childdocforward}
% The command |\childdocforward| redirects
% compilation to the main file or
% (if the optional argument is given) a child file.
% Parameters are set as if the main file
% or a child file starting with |\childdocof| was compiled.
% Then compilation is handed over to the main file:
%    \begin{macrocode}
\newcommand{\childdocforward}[2][]
{
  \begingroup
    \if?#1?
      \def\childdoctmp
      {
        \def\childdocname{#2}
        \def\childdocjob{#2}
        \def\jobname{#2}
        \input{#2}
        \endinput
      }
    \else
      \def\childdoctmp
      {
        \childdocdisable
        \def\childdocname{#2}
        \childdoctrue
        \includeonly{#2}
        \def\childdocjob{#1}
        \def\jobname{#1}
        \input{#1}
        \endinput
      }
    \fi
    \expandafter
  \endgroup
  \childdoctmp
}
%    \end{macrocode}

% \macro{\childdocforwardprefix}
% The command |\childdocforwardprefix| redirects
% compilation to the main or a child file by means of a pattern.
% The prefix |#1| in the current filename is replaced by |#2|
% and the suffix of the current filename is kept
% (it is assumed that the filename does not contain the substring `|~~~|'
% which is used as a delimiter).
% Compilation is handed over to the new file by |\childdocforward|:
%    \begin{macrocode}
\newcommand{\childdocforwardprefix}[3][]
{
  \begingroup
    \def\childdocextract #2##1~~~{\def\childdoctmp{\childdocforward[#1]{#3##1}}}
    \expandafter\childdocextract\childdocname~~~
    \expandafter
  \endgroup
  \childdoctmp
}
%    \end{macrocode}

% \macro{\childdoc}
% The deprecated macro |\childdoc| is a legacy version of |\childdocmain|:
%    \begin{macrocode}
\newcommand{\childdoc}{\childdocmain}
%    \end{macrocode}

% \macro{\childdocredirect}
% The deprecated macro |\childdocredirect| is a legacy version
% of |\childdocforward| and |\childdocforwardprefix|:
%    \begin{macrocode}
\newcommand{\childdocredirect}[2][]
{
  \begingroup
    \if?#1?
      \def\childdoctmp{\childdocforward{#2}}
    \else
      \def\childdoctmp{\childdocforwardprefix{#1}{#2}}
    \fi
    \expandafter
  \endgroup
  \childdoctmp
}
%    \end{macrocode}

%\iffalse
%</package>
%\fi
%
\endinput
|\\
|\childdocof{|\textit{main}|}|\\
\end{tabular}
\end{center}
at the top of every child file \textit{child}
which is included by |\include{|\textit{child}|}|
from within the main file
(or at least for those files to be compiled individually).
The argument \textit{main} must be the filename of the main file.

There are a couple of
considerations in setting up the main and child documents:

%%%%%%%%%%%%%%%%%%%%%%%%%%%%%%%%%%%%%%%%
\paragraph{Restrictions.}

Please note the following restrictions:
\begin{itemize}
\item
|\childdocmain| must be called with one argument \textit{main}
to ensure compatibility with earlier version of the package.
It must either be empty (|\childdocmain{}|)
or precisely match the filename of the main file in which it is specified.
See \secref{sec:detection} for further information.
\item
The filename \textit{main} must be specified without the |.tex| extension.
\item
The filename \textit{main} is case sensitive
(even in case-insensitive file systems)
due to internal string comparison.
\item
The argument \textit{main} should be fully expanded, it cannot be a macro.
\item
Subdirectories and special characters should be avoided in filenames.
\item
The command |\childdocmain{|\textit{main}|}| must be followed by a whitespace.
It should not be followed immediately by another command
or by a comment mark `|%|'.
This is because the \TeX{} parser reads the token immediately following
the argument of |\childdocmain| and puts it
at the beginning of every child section;
however, a white\-space is ignored.
\end{itemize}

%%%%%%%%%%%%%%%%%%%%%%%%%%%%%%%%%%%%%%%%
\paragraph{Content of Main File.}

It is advisable to place all content in the child files included by |\include|.
Any output contained in the main file will appear in all child documents
unless suppressed manually;
it cannot be suppressed automatically by the |\includeonly| directive
and thus should normally be avoided.
A method to include some content in the main file
by means of conditional processing is described in \secref{sec:conditional}.

%%%%%%%%%%%%%%%%%%%%%%%%%%%%%%%%%%%%%%%%
\paragraph{Page Numbering.}

When only a part of the document is compiled,
the appropriate numbering of pages
(as well as other status parameters)
is determined from the |.aux| files.
The latter contain information from previous passes.
However this information needs to propagate through
all intermediate child documents.
Therefore the page numbering in child documents may well
be inconsistent until the complete document is compiled at least once.

A useful (if unconventional) way to always ensure a consistent
page numbering is to restart the numbering in each child document
and denote the pages by `\textit{child}|.|\textit{page}'
where \textit{child} represents the chapter/section number of the child file.
This can be achieved by the command
|\numberwithin{page}{|\textit{child}|}|
of the \textsf{amsmath} package
where \textit{child} can be |chapter| or |section|
depending on the chosen structuring.
Alternatively, one can modify the macro |\thepage| appropriately
and reset the counter |page| at the start of each child file.

%%%%%%%%%%%%%%%%%%%%%%%%%%%%%%%%%%%%%%%%%%%%%%%%%%%%%%%%%%%%%%%%%%%%%%%%%%%%%%%%
\subsection{Conditional Processing}
\label{sec:conditional}

The package provides a mechanism to compile different versions
of a document. To customise the versions further some conditional processing
can come in handy to distinguish which version is being compiled.
The package provides two macros to describe the compilation context:

%%%%%%%%%%%%%%%%%%%%%%%%%%%%%%%%%%%%%%%%
\DescribeMacro{\ifchilddoc}
The conditional |\ifchilddoc| distinguishes between the compilation of
child documents and the main document:
%
\begin{center}
|\ifchilddoc |\textit{child-code}| |[|\||else |\textit{main-code}]| \||fi|
\end{center}

%%%%%%%%%%%%%%%%%%%%%%%%%%%%%%%%%%%%%%%%
\DescribeMacro{\childdocname}
\DescribeMacro{\childdocjob}
The macro |\childdocname| contains the filename (without extension)
of the main or child file being processed.
Note that |\childdocjob| will always contain the name of the main file.

%%%%%%%%%%%%%%%%%%%%%%%%%%%%%%%%%%%%%%%%
\paragraph{Title Page.}

Conditional processing can be used to include a title or banner page
in the main document when proper precautions are taken.
Importantly, the code in the main file should ensure that the page counter
(as well as other status parameters which are stored in the |.aux| files)
takes the same value after the conditional processing.
Otherwise the page numbers may take divergent values
depending on which part is compiled.

For example, a title page could be declared by:
%
\begin{center}
\begin{tabular}{l}
|\ifchilddoc\||else|\\
|\addtocounter{page}{-1}|\\
\textit{code for title page}\\
|\newpage|\\
|\||fi|
\end{tabular}
\end{center}
%
A banner page for the child documents can be generated by:
%
\begin{center}
\begin{tabular}{l}
|\ifchilddoc|\\
|\addtocounter{page}{-1}|\\
\textit{code for banner page}\\
|\newpage|\\
|\||fi|
\end{tabular}
\end{center}
%
Here one could write a message such as:
\begin{center}
|This is the part \childdocname{} of \childdocjob{}.|
\end{center}

%%%%%%%%%%%%%%%%%%%%%%%%%%%%%%%%%%%%%%%%%%%%%%%%%%%%%%%%%%%%%%%%%%%%%%%%%%%%%%%%
\subsection{Flags}
\label{sec:flags}

The package makes it easy to generate different versions
of the main or child documents.
To this end compilation flags can be defined
and assigned different default values.
They will be particularly useful in conjunction
with the forwarding mechanism described in \secref{sec:forward}.

For example, it may be useful to have a flag |\version|
which can be set to |draft| or |final|.
The document source will contain some conditional code
depending on the value of |\version|.
Suppose further, the flag should default to |final| for the main file
and to |draft| for child files
which is a natural assignment for editing the document.
This is achieved by placing the following code
in the preamble of the main document
(below the |\childdocmain| directive):
%
\begin{center}
\begin{tabular}{l}
|\ifchilddoc|\\
|\providecommand{\version}{draft}|\\
|\||else|\\
|\providecommand{\version}{final}|\\
|\||fi|
\end{tabular}
\end{center}
%
The definition by |\providecommand| makes sure
that previous definitions are not overwritten.
Further statements |\providecommand{\version}{...}|
can thus be added before the above code to override it.

For the main file, one might add a line
(between |\childdocmain| and the above block)
%
\begin{center}
|%\ifchilddoc\||else\providecommand{\version}{draft}\||fi|
\end{center}
%
which can be uncommented to produce a draft version.
Likewise one can add a line to the very top of a child file
(above the |\childdocof{|\textit{main}|}| directive)
%
\begin{center}
|%\providecommand{\version}{final}|
\end{center}
%
which can be uncommented to produce the final version of this child document.

%%%%%%%%%%%%%%%%%%%%%%%%%%%%%%%%%%%%%%%%%%%%%%%%%%%%%%%%%%%%%%%%%%%%%%%%%%%%%%%%
\subsection{Forwarding}
\label{sec:forward}

Different versions of the main or child documents
using compilation flags as described in \secref{sec:flags}
can be (permanently) stored in different files
for convenient compilation, viewing and distribution.
To this end, the package defines a command
to pass on compilation to a different file:

%%%%%%%%%%%%%%%%%%%%%%%%%%%%%%%%%%%%%%%%
\DescribeMacro{\childdocforward}
The command |\childdocforward| redirects processing to
another source file:
%
\begin{center}
\begin{tabular}{l}
|% \iffalse
%
% childdoc.dtx Copyright (C) 2017-2018 Niklas Beisert
%
% This work may be distributed and/or modified under the
% conditions of the LaTeX Project Public License, either version 1.3
% of this license or (at your option) any later version.
% The latest version of this license is in
%   http://www.latex-project.org/lppl.txt
% and version 1.3 or later is part of all distributions of LaTeX
% version 2005/12/01 or later.
%
% This work has the LPPL maintenance status `maintained'.
%
% The Current Maintainer of this work is Niklas Beisert.
%
% This work consists of the files childdoc.dtx and childdoc.ins
% and the derived files childdoc.def and cdocsamp.tex with
% cdocsch1.tex, cdocsch2.tex, cdocsdrf.tex, cdocsfn1.tex, cdocsfn2.tex.
%
%<package>\ifdefined\childdocmain\endinput\fi
%<package>\ProvidesFile{childdoc.def}[2018/12/30 v2.0 child document driver]
%<samplemain>\ProvidesFile{cdocsamp.tex}[2018/12/30 v2.0 sample for childdoc]
%<*driver>
%\ProvidesFile{childdoc.drv}[2018/12/30 v2.0 childdoc reference manual file]
\PassOptionsToClass{10pt,a4paper}{article}
\documentclass{ltxdoc}

\usepackage[margin=35mm]{geometry}
\usepackage{hyperref}
\usepackage{hyperxmp}
\usepackage[usenames]{color}

\hypersetup{colorlinks=true}
\hypersetup{pdfstartview=FitH}
\hypersetup{pdfpagemode=UseNone}
\hypersetup{pdfsource={}}
\hypersetup{pdflang={en-UK}}
\hypersetup{pdfcopyright={Copyright 2017-2018 Niklas Beisert.
  This work may be distributed and/or modified under the
  conditions of the LaTeX Project Public License, either version 1.3
  of this license or (at your option) any later version.}}
\hypersetup{pdflicenseurl={http://www.latex-project.org/lppl.txt}}
\hypersetup{pdfcontactaddress={ETH Zurich, ITP, HIT K,
  Wolfgang-Pauli-Strasse 27}}
\hypersetup{pdfcontactpostcode={8093}}
\hypersetup{pdfcontactcity={Zurich}}
\hypersetup{pdfcontactcountry={Switzerland}}
\hypersetup{pdfcontactemail={nbeisert@itp.phys.ethz.ch}}
\hypersetup{pdfcontacturl={http://people.phys.ethz.ch/\xmptilde nbeisert/}}

\newcommand{\secref}[1]{\hyperref[#1]{section \ref*{#1}}}

\parskip1ex
\parindent0pt
\let\olditemize\itemize
\def\itemize{\olditemize\parskip0pt}

\begin{document}

\title{The \textsf{childdoc} Package}
\hypersetup{pdftitle={The childdoc Package}}
\author{Niklas Beisert\\[2ex]
  Institut f\"ur Theoretische Physik\\
  Eidgen\"ossische Technische Hochschule Z\"urich\\
  Wolfgang-Pauli-Strasse 27, 8093 Z\"urich, Switzerland\\[1ex]
  \href{mailto:nbeisert@itp.phys.ethz.ch}
  {\texttt{nbeisert@itp.phys.ethz.ch}}}
\hypersetup{pdfauthor={Niklas Beisert}}
\hypersetup{pdfsubject={Manual for the LaTeX2e Package childdoc}}
\date{30 December 2018, \textsf{v2.0}}
\maketitle

\begin{abstract}\noindent
\textsf{childdoc} is a \LaTeXe{} package
that enables the direct compilation
of document sections included by |\include|
to individual files.
\end{abstract}

\begingroup
\parskip0ex
\tableofcontents
\endgroup

%%%%%%%%%%%%%%%%%%%%%%%%%%%%%%%%%%%%%%%%%%%%%%%%%%%%%%%%%%%%%%%%%%%%%%%%%%%%%%%%
%%%%%%%%%%%%%%%%%%%%%%%%%%%%%%%%%%%%%%%%%%%%%%%%%%%%%%%%%%%%%%%%%%%%%%%%%%%%%%%%
\section{Introduction}

\LaTeX{} provides a mechanism to structure a large document (such as a book)
into a main file and several child files (containing the chapters)
using the |\include| command.
This mechanism is beneficial for documents
which span hundreds of pages in order to
make the source file(s) more manageable.
Moreover, compilation can be restricted to
selected child files by means of the |\includeonly| command.
The latter feature can be used to reduce the compilation time while editing
(this was significantly more useful in the earlier days of \LaTeX{})
or to generate a smaller document which is easier to navigate.
Another application of |\includeonly| is to generate
documents consisting of selected parts of the complete document.

However, there are a few drawbacks of the plain |\include| mechanism:
\begin{itemize}
\item
The child files cannot be compiled on their own,
they can only be compiled via the main file.
A naive editing environment
(such as a text editor with an option
to have the current file processed by \LaTeX)
may require one to switch to the main file before compiling;
attempting to compile the child file produces errors.
\item
The main file must be modified (each time)
to adjust the |\includeonly| command
to the present needs. This easily leaves the main file in a messy state.
\item
The generated document will always carry the filename
of the main document. This is inconvenient if
several child files are to be compiled and
to be kept for distribution.
\end{itemize}

The present package provides a simple interface
to make child files individually compilable by \LaTeX{}.
Compiling a child file then has the same effect as compiling
the main file with an |\includeonly| command
to select the appropriate child.
Moreover the generated document will carry the name of the child
rather than the main file.
This resolves all three above issues.

This feature is meant to make the editing of books,
thesis documents and lecture notes somewhat more convenient.
However, the package can also be used efficiently for
composing a series of documents (such as exercise sheets)
which are typically distributed individually.
It then assists the author in generating the individual documents
(potentially in different versions)
as well as a document containing the collected series.
Another application is in developing style files
or other kinds of included material
where compilation of the style file could redirect
to a sample or test file.

%%%%%%%%%%%%%%%%%%%%%%%%%%%%%%%%%%%%%%%%%%%%%%%%%%%%%%%%%%%%%%%%%%%%%%%%%%%%%%%%
%%%%%%%%%%%%%%%%%%%%%%%%%%%%%%%%%%%%%%%%%%%%%%%%%%%%%%%%%%%%%%%%%%%%%%%%%%%%%%%%
\section{Usage}

First of all, the package \textsf{childdoc} is \emph{not} a standard
\LaTeXe{} |.sty| style file! Therefore it needs to be invoked in
a non-standard way.

%%%%%%%%%%%%%%%%%%%%%%%%%%%%%%%%%%%%%%%%%%%%%%%%%%%%%%%%%%%%%%%%%%%%%%%%%%%%%%%%
\subsection{Included Files}
\label{sec:include}

%%%%%%%%%%%%%%%%%%%%%%%%%%%%%%%%%%%%%%%%
\DescribeMacro{\childdocmain}
To use the package, add the commands
\begin{center}
\begin{tabular}{l}
|\input{childdoc.def}|\\
|\childdocmain{}|\\
\end{tabular}
\end{center}
at the very top of the main \LaTeX{} file,
in particular \emph{before} the |\documentclass| statement!
The argument of |\childdocmain| should be left empty
(but it must be present).

%%%%%%%%%%%%%%%%%%%%%%%%%%%%%%%%%%%%%%%%
\DescribeMacro{\childdocof}
Furthermore, add the commands
\begin{center}
\begin{tabular}{l}
|\input{childdoc.def}|\\
|\childdocof{|\textit{main}|}|\\
\end{tabular}
\end{center}
at the top of every child file \textit{child}
which is included by |\include{|\textit{child}|}|
from within the main file
(or at least for those files to be compiled individually).
The argument \textit{main} must be the filename of the main file.

There are a couple of
considerations in setting up the main and child documents:

%%%%%%%%%%%%%%%%%%%%%%%%%%%%%%%%%%%%%%%%
\paragraph{Restrictions.}

Please note the following restrictions:
\begin{itemize}
\item
|\childdocmain| must be called with one argument \textit{main}
to ensure compatibility with earlier version of the package.
It must either be empty (|\childdocmain{}|)
or precisely match the filename of the main file in which it is specified.
See \secref{sec:detection} for further information.
\item
The filename \textit{main} must be specified without the |.tex| extension.
\item
The filename \textit{main} is case sensitive
(even in case-insensitive file systems)
due to internal string comparison.
\item
The argument \textit{main} should be fully expanded, it cannot be a macro.
\item
Subdirectories and special characters should be avoided in filenames.
\item
The command |\childdocmain{|\textit{main}|}| must be followed by a whitespace.
It should not be followed immediately by another command
or by a comment mark `|%|'.
This is because the \TeX{} parser reads the token immediately following
the argument of |\childdocmain| and puts it
at the beginning of every child section;
however, a white\-space is ignored.
\end{itemize}

%%%%%%%%%%%%%%%%%%%%%%%%%%%%%%%%%%%%%%%%
\paragraph{Content of Main File.}

It is advisable to place all content in the child files included by |\include|.
Any output contained in the main file will appear in all child documents
unless suppressed manually;
it cannot be suppressed automatically by the |\includeonly| directive
and thus should normally be avoided.
A method to include some content in the main file
by means of conditional processing is described in \secref{sec:conditional}.

%%%%%%%%%%%%%%%%%%%%%%%%%%%%%%%%%%%%%%%%
\paragraph{Page Numbering.}

When only a part of the document is compiled,
the appropriate numbering of pages
(as well as other status parameters)
is determined from the |.aux| files.
The latter contain information from previous passes.
However this information needs to propagate through
all intermediate child documents.
Therefore the page numbering in child documents may well
be inconsistent until the complete document is compiled at least once.

A useful (if unconventional) way to always ensure a consistent
page numbering is to restart the numbering in each child document
and denote the pages by `\textit{child}|.|\textit{page}'
where \textit{child} represents the chapter/section number of the child file.
This can be achieved by the command
|\numberwithin{page}{|\textit{child}|}|
of the \textsf{amsmath} package
where \textit{child} can be |chapter| or |section|
depending on the chosen structuring.
Alternatively, one can modify the macro |\thepage| appropriately
and reset the counter |page| at the start of each child file.

%%%%%%%%%%%%%%%%%%%%%%%%%%%%%%%%%%%%%%%%%%%%%%%%%%%%%%%%%%%%%%%%%%%%%%%%%%%%%%%%
\subsection{Conditional Processing}
\label{sec:conditional}

The package provides a mechanism to compile different versions
of a document. To customise the versions further some conditional processing
can come in handy to distinguish which version is being compiled.
The package provides two macros to describe the compilation context:

%%%%%%%%%%%%%%%%%%%%%%%%%%%%%%%%%%%%%%%%
\DescribeMacro{\ifchilddoc}
The conditional |\ifchilddoc| distinguishes between the compilation of
child documents and the main document:
%
\begin{center}
|\ifchilddoc |\textit{child-code}| |[|\||else |\textit{main-code}]| \||fi|
\end{center}

%%%%%%%%%%%%%%%%%%%%%%%%%%%%%%%%%%%%%%%%
\DescribeMacro{\childdocname}
\DescribeMacro{\childdocjob}
The macro |\childdocname| contains the filename (without extension)
of the main or child file being processed.
Note that |\childdocjob| will always contain the name of the main file.

%%%%%%%%%%%%%%%%%%%%%%%%%%%%%%%%%%%%%%%%
\paragraph{Title Page.}

Conditional processing can be used to include a title or banner page
in the main document when proper precautions are taken.
Importantly, the code in the main file should ensure that the page counter
(as well as other status parameters which are stored in the |.aux| files)
takes the same value after the conditional processing.
Otherwise the page numbers may take divergent values
depending on which part is compiled.

For example, a title page could be declared by:
%
\begin{center}
\begin{tabular}{l}
|\ifchilddoc\||else|\\
|\addtocounter{page}{-1}|\\
\textit{code for title page}\\
|\newpage|\\
|\||fi|
\end{tabular}
\end{center}
%
A banner page for the child documents can be generated by:
%
\begin{center}
\begin{tabular}{l}
|\ifchilddoc|\\
|\addtocounter{page}{-1}|\\
\textit{code for banner page}\\
|\newpage|\\
|\||fi|
\end{tabular}
\end{center}
%
Here one could write a message such as:
\begin{center}
|This is the part \childdocname{} of \childdocjob{}.|
\end{center}

%%%%%%%%%%%%%%%%%%%%%%%%%%%%%%%%%%%%%%%%%%%%%%%%%%%%%%%%%%%%%%%%%%%%%%%%%%%%%%%%
\subsection{Flags}
\label{sec:flags}

The package makes it easy to generate different versions
of the main or child documents.
To this end compilation flags can be defined
and assigned different default values.
They will be particularly useful in conjunction
with the forwarding mechanism described in \secref{sec:forward}.

For example, it may be useful to have a flag |\version|
which can be set to |draft| or |final|.
The document source will contain some conditional code
depending on the value of |\version|.
Suppose further, the flag should default to |final| for the main file
and to |draft| for child files
which is a natural assignment for editing the document.
This is achieved by placing the following code
in the preamble of the main document
(below the |\childdocmain| directive):
%
\begin{center}
\begin{tabular}{l}
|\ifchilddoc|\\
|\providecommand{\version}{draft}|\\
|\||else|\\
|\providecommand{\version}{final}|\\
|\||fi|
\end{tabular}
\end{center}
%
The definition by |\providecommand| makes sure
that previous definitions are not overwritten.
Further statements |\providecommand{\version}{...}|
can thus be added before the above code to override it.

For the main file, one might add a line
(between |\childdocmain| and the above block)
%
\begin{center}
|%\ifchilddoc\||else\providecommand{\version}{draft}\||fi|
\end{center}
%
which can be uncommented to produce a draft version.
Likewise one can add a line to the very top of a child file
(above the |\childdocof{|\textit{main}|}| directive)
%
\begin{center}
|%\providecommand{\version}{final}|
\end{center}
%
which can be uncommented to produce the final version of this child document.

%%%%%%%%%%%%%%%%%%%%%%%%%%%%%%%%%%%%%%%%%%%%%%%%%%%%%%%%%%%%%%%%%%%%%%%%%%%%%%%%
\subsection{Forwarding}
\label{sec:forward}

Different versions of the main or child documents
using compilation flags as described in \secref{sec:flags}
can be (permanently) stored in different files
for convenient compilation, viewing and distribution.
To this end, the package defines a command
to pass on compilation to a different file:

%%%%%%%%%%%%%%%%%%%%%%%%%%%%%%%%%%%%%%%%
\DescribeMacro{\childdocforward}
The command |\childdocforward| redirects processing to
another source file:
%
\begin{center}
\begin{tabular}{l}
|\input{childdoc.def}|\\
|\childdocforward[|\textit{main}|]{|\textit{dest}|}|\\
\end{tabular}
\end{center}
%
The argument \textit{dest} is the destination file
(without extension).
It should be the main file or one of the child files.
Note that further \textsf{childdoc} directives
such as |\childdocof| and |\childdocforward|
in the indicated file will be processed in this form.
The optional argument \textit{main}
passes on directly to the main file \textit{main}
while pretending to compile the child \textit{dest}.
This form behaves as if \textit{dest}
issues |\childdocof{|\textit{main}|}| right away,
and no further \textsf{childdoc} directives will be processed.

%%%%%%%%%%%%%%%%%%%%%%%%%%%%%%%%%%%%%%%%
\DescribeMacro{\...prefix}
In the alternative form |\childdocforwardprefix|,
%
\begin{center}
\begin{tabular}{l}
|\input{childdoc.def}|\\
|\childdocforwardprefix[|\textit{main}|]{|\textit{prefix}|}{|\textit{dest}|}|
\end{tabular}
\end{center}
%
the destination file is determined by a pattern
depending on the current file:
To make this work, the current file must be called
`{\textit{prefix}\hspace{0.2em}\textit{suffix}}'
with \textit{prefix} matching precisely the argument.
Processing is then passed on to the file
`{\textit{dest}\hspace{0.2em}\textit{suffix}}'.
Surely, the same effect is achieved by
directly specifying the
argument `{\textit{dest}\hspace{0.2em}\textit{suffix}}'
in the first form.
However, that requires to set up a different file
for each child. With the alternative form of the command
all these files can have exactly the same content
which simplifies setting them up and maintaining them.

For example, the following file |draft.tex|
with a compilation flag |\version| as described in \secref{sec:flags}
compiles the main document as a draft:
%
\begin{center}
\begin{tabular}{l}
|\def\version{draft}|\\
|\input{childdoc.def}|\\
|\childdocforward{|\textit{main}|}|
\end{tabular}
\end{center}
%
Likewise, the following files |final|\textit{nn}|.tex|
compile the final version of the child document
|child|\textit{nn}|.tex|:
%
\begin{center}
\begin{tabular}{l}
|\def\version{final}|\\
|\input{childdoc.def}|\\
|\childdocforwardprefix{final}{child}|
\end{tabular}
\end{center}
%

Note that when several versions of a main file and/or of each child file
are to be generated, it may be convenient to set up a |Makefile| or
shell script to automatise the process.

%%%%%%%%%%%%%%%%%%%%%%%%%%%%%%%%%%%%%%%%%%%%%%%%%%%%%%%%%%%%%%%%%%%%%%%%%%%%%%%%
\subsection{Command Line Processing}
\label{sec:commandline}

The effect of redirection files can also be achieved by invoking
the \LaTeX{} compiler with a more elaborate command line.
Most conveniently this should be done as part
of a shell script or a |Makefile|.

When using \textsf{childdoc} in the main file, the following
command lines effectively perform a redirection
(note that depending on the shell being used,
backslashes may have to be doubled: `|\|' $\to$ `|\\|'):
%
\begin{center}
|... -jobname "|\textit{target}|" |\\|"|[\textit{flags}]%
|\input{childdoc.def}\childdocforward[|\textit{main}|]{|\textit{dest}|}"|
\end{center}
%
Here \textit{target} is the name of the output file,
\textit{main} is the name of the main file
and \textit{dest} is the name of the main or child file to be processed
(all filenames without extensions).
The optional argument \textit{main} can be omitted
if \textit{main} matches \textit{dest}.
Optionally, compilation \textit{flags} can be defined via |\def| commands.
This command line makes the \TeX{} engine believe
it is compiling the file \textit{target}
whose content is specified as the latter parameter.
The provided code then forwards the processing to
\textit{main} or \textit{dest} as described in \secref{sec:forward}.

%%%%%%%%%%%%%%%%%%%%%%%%%%%%%%%%%%%%%%%%%%%%%%%%%%%%%%%%%%%%%%%%%%%%%%%%%%%%%%%%
\subsection{Include by Input}
\label{sec:input}

Including child documents by |\include| has some restrictions by design.
Most notably, the content of a child document always occupies
its own set of pages; pages cannot be shared between child documents.
Usually, this behaviour makes perfect sense
because each child document contain an essential part of the document.
However, in some situations it may be desirable to compose
a document from a collection of parts
without having mandatory page breaks between then.
For this case, the package
provides a mechanism to include parts
by |\input| which can also be processed individually.
However, by construction this mechanism
requires manual handling of the content to be output.

%%%%%%%%%%%%%%%%%%%%%%%%%%%%%%%%%%%%%%%%
\DescribeMacro{\ifchilddocmanual}
The main file should be prepared as usual, see \secref{sec:include}.
However, the document body must make a distinction
between processing of an individual part and of the main document, e.g.:
%
\begin{center}
\begin{tabular}{l}
|\ifchilddocmanual|\\
|\input{\childdocname}|\\
|\||else|\\
\textit{document body with }|\input{|\textit{part}|}|\\
|\||fi|
\end{tabular}
\end{center}
%
The conditional |\ifchilddocmanual| is true whenever
a part to be included by |\input| is being compiled,
and the name of the part is stored in |\childdocname|.

%%%%%%%%%%%%%%%%%%%%%%%%%%%%%%%%%%%%%%%%
\DescribeMacro{\childdocby}
Each part to be included by |\input| should start with:
%
\begin{center}
\begin{tabular}{l}
|\input{childdoc.def}|\\
|\childdocby{|\textit{main}|}|\\
\end{tabular}
\end{center}
%
The directive |\childdocby| is similar to |\childdocof|
described in \secref{sec:include},
but the subsequent selection of content must be done manually.
To that end, both |\ifchilddoc| and |\ifchilddocmanual|
will be true upon processing of a part,
and the name of the part is stored in |\childdocname|.
Note that |\jobname| will be set to the filename of the current part
so that each part receives an individual |.aux| file
that does not interfere with the |.aux| file(s) of the main document.
This behaviour can be altered by the alternative form
|\childdocby[*]{|\textit{main}|}| (with a non-empty optional argument)
which uses the |.aux| file of the main document
by setting |\jobname| to \textit{main}.

%%%%%%%%%%%%%%%%%%%%%%%%%%%%%%%%%%%%%%%%%%%%%%%%%%%%%%%%%%%%%%%%%%%%%%%%%%%%%%%%
\subsection{Driver Development}
\label{sec:driver}

The \textsf{childdoc} mechanism can also be use for the development
of definition files such as \LaTeX{} styles or classes.
This case differs from the above setup with multiple parts
included by |\include| in that no |\includeonly| should be invoked.
This can be achieved by starting the include file
(before |\ProvidesPackage|) with:
%
\begin{center}
\begin{tabular}{l}
|\input{childdoc.def}|\\
|\childdocforward{|\textit{main}|}|\\
\end{tabular}
\end{center}
%
or alternatively with:
%
\begin{center}
\begin{tabular}{l}
|\input{childdoc.def}|\\
|\childdocby{|\textit{main}|}|\\
\end{tabular}
\end{center}
%
Both forms have slightly different effects as described above.
The main file is prepared as usual, see \secref{sec:include}.

%%%%%%%%%%%%%%%%%%%%%%%%%%%%%%%%%%%%%%%%%%%%%%%%%%%%%%%%%%%%%%%%%%%%%%%%%%%%%%%%
\subsection{Legacy Detection}
\label{sec:detection}

The directive |\childdocmain| in the main file can detect
whether the complete document or merely a child is to be compiled
even without using the directive |\childdocof|.
This method is deprecated because it is less robust
and there is no compelling reason to use it;
it is merely provided for backward compatibility
and it may be removed in future versions.

If the detection mechanism is to be used,
it is mandatory to correctly specify
the filename of the main file as the argument of |\childdocmain|:
%
\begin{center}
\begin{tabular}{l}
|\input{childdoc.def}|\\
|\childdocmain{|\textit{main}|}|\\
\end{tabular}
\end{center}
%
If |\jobname| does not match the argument \textit{main} of |\childdocmain|,
it is assumed that |\jobname| points to the child file to be compiled.
When using |\childdocmain| with the main file specified as argument,
it suffices to start a child file
with just |\input{|\textit{main}|}|
without loading of the package and using |\childdocof|.
If instead all processing is done
with the appropriate \textsf{childdoc} directives,
the argument of \textit{main} of |\childdocmain| can be empty.

An alternative version of the command line processing described
in \secref{sec:commandline} using the detection mechanism reads:
%
\begin{center}
|... -jobname "|\textit{target}|" "|[\textit{flags}]%
[|\def\jobname{|\textit{dest}|}|]|\input{|\textit{main}|}"|
\end{center}

%%%%%%%%%%%%%%%%%%%%%%%%%%%%%%%%%%%%%%%%%%%%%%%%%%%%%%%%%%%%%%%%%%%%%%%%%%%%%%%%
\subsection{Manual Code}
\label{sec:manual}

In case one cannot be certain whether the definitions file |childdoc.def|
is installed on the target \TeX{} distribution
and one prefers not to ship it,
it is conceivable to paste a few relevant commands into the sources.

To that end, drop all statements |\input{childdoc.def}|
and perform the replacements as outlined below.
Instead of |\childdocmain{|\textit{main}|}| add the following code
to the top of the main file:
%
\begin{center}
\begin{tabular}{l}
|\||ifdefined\childdocname\endinput\||fi\newif\ifchilddoc|\\
|\edef\childdocname{\scantokens\expandafter{\jobname\noexpand}}|\\
|\def\childdocmain{|\textit{main}|}\||ifx\childdocmain\childdocname\||else|\\
|\childdoctrue\includeonly{\childdocname}\let\jobname\childdocmain\||fi|\\
\end{tabular}
\end{center}
%
Instead of |\childdocof{|\textit{main}|}| just include the main file
at the top of each child file:
%
\begin{center}
|\input{|\textit{main}|}|
\end{center}
%
A simple redirection |\childdocforward{|\textit{dest}|}| is achieved by:
%
\begin{center}
|\def\jobname{|\textit{dest}|}\input{\jobname}|
\end{center}
%
The redirection with prefix
|\childdocforwardprefix[|\textit{prefix}|]{|\textit{dest}|}|
is accomplished by:
%
\begin{center}
\begin{tabular}{l}
|{\edef\jobname{\scantokens\expandafter{\jobname\noexpand}}|\\
|\def\redirectjob |\textit{prefix}|#1~~~{\gdef\jobname{|\textit{dest}|#1}}|\\
|\expandafter\redirectjob\jobname~~~}\input{\jobname}|
\end{tabular}
\end{center}

In an alternative approach,
child documents can be compiled by a specific command line
without additional code or specific definitions:
%
\begin{center}
|... -jobname "|\textit{target}|" "|[\textit{flags}]%
|\includeonly{|\textit{dest}|}\input{|\textit{main}|}"|
\end{center}
%

%%%%%%%%%%%%%%%%%%%%%%%%%%%%%%%%%%%%%%%%%%%%%%%%%%%%%%%%%%%%%%%%%%%%%%%%%%%%%%%%
%%%%%%%%%%%%%%%%%%%%%%%%%%%%%%%%%%%%%%%%%%%%%%%%%%%%%%%%%%%%%%%%%%%%%%%%%%%%%%%%
\section{Information}

%%%%%%%%%%%%%%%%%%%%%%%%%%%%%%%%%%%%%%%%%%%%%%%%%%%%%%%%%%%%%%%%%%%%%%%%%%%%%%%%
\subsection{Copyright}

Copyright \copyright{} 2017--2018 Niklas Beisert

This work may be distributed and/or modified under the
conditions of the \LaTeX{} Project Public License, either version 1.3
of this license or (at your option) any later version.
The latest version of this license is in
  \url{http://www.latex-project.org/lppl.txt}
and version 1.3 or later is part of all distributions of \LaTeX{}
version 2005/12/01 or later.

This work has the LPPL maintenance status `maintained'.

The Current Maintainer of this work is Niklas Beisert.

This work consists of the files |README.txt|, |childdoc.ins| and |childdoc.dtx|
as well as the derived files |childdoc.def|, |cdocsamp.tex|
with |cdocsch1.tex|, |cdocsch2.tex|, |cdocspt3.tex|, |cdocspt4.tex|,
|cdocsdrf.tex|, |cdocsfn1.tex|, |cdocsfn2.tex|
as well as |childdoc.pdf|.

%%%%%%%%%%%%%%%%%%%%%%%%%%%%%%%%%%%%%%%%%%%%%%%%%%%%%%%%%%%%%%%%%%%%%%%%%%%%%%%%
\subsection{Files and Installation}

The package consists of the files:
%
\begin{center}
\begin{tabular}{ll}
    |README.txt|   & readme file \\
    |childdoc.ins| & installation file \\
    |childdoc.dtx| & source file \\
    |childdoc.def| & definition file \\
    |cdocsamp.tex| & sample main file \\
    |cdocsch1.tex| & sample include file \\
    |cdocsch2.tex| & sample include file \\
    |cdocspt3.tex| & sample part file \\
    |cdocspt4.tex| & sample part file \\
    |cdocsdrf.tex| & sample redirection file \\
    |cdocsfn1.tex| & sample redirection file \\
    |cdocsfn2.tex| & sample redirection file \\
    |childdoc.pdf| & manual
\end{tabular}
\end{center}
%
The distribution consists of the files
|README.txt|, |childdoc.ins| and |childdoc.dtx|.
%
\begin{itemize}
\item
Run (pdf)\LaTeX{} on |childdoc.dtx|
to compile the manual |childdoc.pdf| (this file).
\item
Run \LaTeX{} on |childdoc.ins| to create the definitions file |childdoc.def|
and the sample |cdocsamp.tex| with include files
|cdocsch1.tex|, |cdocsch2.tex|, |cdocspt3.tex|, |cdocspt4.tex|,
|cdocsdrf.tex|, |cdocsfn1.tex|, |cdocsfn2.tex|.
Then copy the file |childdoc.def| to an appropriate directory of your \LaTeX{}
distribution, e.g.\ \textit{texmf-root}|/tex/latex/childdoc|.
\end{itemize}

%%%%%%%%%%%%%%%%%%%%%%%%%%%%%%%%%%%%%%%%%%%%%%%%%%%%%%%%%%%%%%%%%%%%%%%%%%%%%%%%
\subsection{Related CTAN Packages}

There are several other packages which offer a similar functionality:
%
\begin{itemize}
\item
The packages
\href{http://ctan.org/pkg/docmute}{\textsf{docmute}},
\href{http://ctan.org/pkg/includex}{\textsf{includex}} and
\href{http://ctan.org/pkg/standalone}{\textsf{standalone}}
provide commands to include only the document body of
a child file thus allowing both files to be compiled individually.
\item
The packages \href{http://ctan.org/pkg/subdocs}{\textsf{subdocs}}
and \href{http://ctan.org/pkg/subfiles}{\textsf{subfiles}}
provide structures in which the main and child documents can be
encapsulated and allowing them to be compiled individually.
The inclusion mechanism is different from the conventional |\include|.
\item
The package \href{http://ctan.org/pkg/combine}{\textsf{combine}}
is an elaborate solution to combine several documents into one.
\end{itemize}
%
See also the CTAN topic \href{http://ctan.org/topic/subdocs}{\textsf{subdocs}}
for further related packages.
The present package differs from the above solutions in that
a document structure constructed with the conventional |\include| mechanism
just needs two extra commands at the top of every file
such that all constituent files can be compiled individually.

%%%%%%%%%%%%%%%%%%%%%%%%%%%%%%%%%%%%%%%%%%%%%%%%%%%%%%%%%%%%%%%%%%%%%%%%%%%%%%%%
%\subsection{Feature Suggestions}
%
%The following is a list of features which may be useful for future
%versions of this package:
%%
%\begin{itemize}
%\item
%\ldots
%\end{itemize}

%%%%%%%%%%%%%%%%%%%%%%%%%%%%%%%%%%%%%%%%%%%%%%%%%%%%%%%%%%%%%%%%%%%%%%%%%%%%%%%%
\subsection{Revision History}

%%%%%%%%%%%%%%%%%%%%%%%%%%%%%%%%%%%%%%%%
\paragraph{v2.0:} 2018/12/30

\begin{itemize}
\item
immediate forward processing
\item
added |\childdocby| mechanism
\item
manual restructured
\end{itemize}

%%%%%%%%%%%%%%%%%%%%%%%%%%%%%%%%%%%%%%%%
\paragraph{v1.6:} 2018/01/17

\begin{itemize}
\item
application for development of include files
\item
corrections to manual
\end{itemize}

%%%%%%%%%%%%%%%%%%%%%%%%%%%%%%%%%%%%%%%%
\paragraph{v1.5:} 2017/05/21

\begin{itemize}
\item
more complete structuring introduced
\item
|\childdocof| introduced
\item
|\childdoc| renamed to |\childdocmain|
\item
|\childredirect| renamed to |\childdocforward| and |\childdocforwardprefix|
and functionality expanded
\end{itemize}

%%%%%%%%%%%%%%%%%%%%%%%%%%%%%%%%%%%%%%%%
\paragraph{v1.0:} 2017/04/27

\begin{itemize}
\item
manual and install package
\item
first version published on CTAN
\end{itemize}

%%%%%%%%%%%%%%%%%%%%%%%%%%%%%%%%%%%%%%%%
\paragraph{v0.6:} 2017/04/26

\begin{itemize}
\item
redirection mechanism added
\end{itemize}

%%%%%%%%%%%%%%%%%%%%%%%%%%%%%%%%%%%%%%%%
\paragraph{v0.5:} 2017/04/26

\begin{itemize}
\item
functionality in definition file
\end{itemize}


%%%%%%%%%%%%%%%%%%%%%%%%%%%%%%%%%%%%%%%%%%%%%%%%%%%%%%%%%%%%%%%%%%%%%%%%%%%%%%%%
%%%%%%%%%%%%%%%%%%%%%%%%%%%%%%%%%%%%%%%%%%%%%%%%%%%%%%%%%%%%%%%%%%%%%%%%%%%%%%%%
%%%%%%%%%%%%%%%%%%%%%%%%%%%%%%%%%%%%%%%%%%%%%%%%%%%%%%%%%%%%%%%%%%%%%%%%%%%%%%%%
\appendix

\settowidth\MacroIndent{\rmfamily\scriptsize 000\ }

 \DocInput{childdoc.dtx}

\end{document}
%</driver>
% \fi
%
% %%%%%%%%%%%%%%%%%%%%%%%%%%%%%%%%%%%%%%%%%%%%%%%%%%%%%%%%%%%%%%%%%%%%%%%%%%%%%%
% %%%%%%%%%%%%%%%%%%%%%%%%%%%%%%%%%%%%%%%%%%%%%%%%%%%%%%%%%%%%%%%%%%%%%%%%%%%%%%
% \section{Sample}
%\iffalse
%<*samplemain>
%\fi
%
% The following presents a sample document
% with two chapters, two parts, a title page,
% a compile flag as well as three forwarding files to set the flag.
% It consists of eight |.tex| files:
% \begin{center}
% \begin{tabular}{ll}
% |cdocsamp.tex|&main file\\
% |cdocsch1.tex|&include file for chapter 1\\
% |cdocsch2.tex|&include file for chapter 2\\
% |cdocspt3.tex|&include file for part 3\\
% |cdocspt4.tex|&include file for part 4\\
% |cdocsdrf.tex|&forwarding file for main file in draft mode\\
% |cdocsfi1.tex|&forwarding file for final version of chapter 1\\
% |cdocsfi2.tex|&forwarding file for final version of chapter 2\\
% \end{tabular}
% \end{center}
% Each of the eight files can be compiled directly by the \LaTeX{} compiler.
%
% %%%%%%%%%%%%%%%%%%%%%%%%%%%%%%%%%%%%%%
% \paragraph{Main File.}
%
% The main file is called |cdocsamp.tex|.
%
% Load the \textsf{childdoc} definitions and
% declare the filename for the main document:
%    \begin{macrocode}
\input{childdoc.def}
\childdocmain{}
%    \end{macrocode}

% Optional override for |\version| flag:
%    \begin{macrocode}
%%\ifchilddoc\else\providecommand{\version}{draft}\fi
%    \end{macrocode}

% Define the default values for the |\version| flag
% (|final| for the main file and |draft| for childs):
%    \begin{macrocode}
\ifchilddoc
\providecommand{\version}{draft}
\else
\providecommand{\version}{final}
\fi
%    \end{macrocode}

% Load the standard document class:
%    \begin{macrocode}
\documentclass[12pt]{article}
%    \end{macrocode}

% Start the document body:
%    \begin{macrocode}
\begin{document}
%    \end{macrocode}

% Declare a title page.
% Print title, part of document being processed and version flag:
%    \begin{macrocode}
\addtocounter{page}{-1}
\begin{center}
{\LARGE\bfseries{}childdoc example\par}
\vspace{1cm}
\ifchilddoc
\ifchilddocmanual part\else chapter\fi:
`\childdocname' of `\childdocjob'\par
\else
main document: `\childdocjob'\par
\fi
version: \version\par
\end{center}
\newpage
%    \end{macrocode}

% Manually include selected file,
% otherwise process as usual:
%    \begin{macrocode}
\ifchilddocmanual
\section*{part `\childdocname'}
\input{\childdocname}
\else
%    \end{macrocode}

% Include the two chapters:
%    \begin{macrocode}
\include{cdocsch1}
\include{cdocsch2}
%    \end{macrocode}

% Include the two parts unless only chapters should be displayed:
%    \begin{macrocode}
\ifchilddoc\else
\section{part three}
\input{cdocspt3}
\section{part four}
\input{cdocspt4}
\fi
%    \end{macrocode}

% Process as usual until here:
%    \begin{macrocode}
\fi
%    \end{macrocode}

% End of document body:
%    \begin{macrocode}
\end{document}
%    \end{macrocode}
%\iffalse
%</samplemain>
%\fi
%
% %%%%%%%%%%%%%%%%%%%%%%%%%%%%%%%%%%%%%%
% \paragraph{Chapter Include Files.}
%
% The include files are called |cdocsch1.tex| and |cdocsch2.tex|.
%
%\iffalse
%<*samplechap1|samplechap2>
%\fi

% Optional override for |\version| flag:
%    \begin{macrocode}
%%\providecommand{\version}{final}
%    \end{macrocode}

% Include the main document:
%    \begin{macrocode}
\input{childdoc.def}
\childdocof{cdocsamp}
%    \end{macrocode}

%\iffalse
%</samplechap1|samplechap2>
%\fi
%
%\iffalse
%<*samplechap1>
%\fi
% Some text for chapter 1:
%    \begin{macrocode}
\section{one}
some text in chapter one
%    \end{macrocode}

%\iffalse
%</samplechap1>
%\fi
% Some text for chapter 2:
%\iffalse
%<*samplechap2>
%\fi
%    \begin{macrocode}
\section{two}
more text in chapter two
%    \end{macrocode}

%\iffalse
%</samplechap2>
%\fi
%
% %%%%%%%%%%%%%%%%%%%%%%%%%%%%%%%%%%%%%%
% \paragraph{Part Include Files.}
%
% The include files are called |cdocspt3.tex| and |cdocspt4.tex|.
%
%\iffalse
%<*samplepart3|samplepart4>
%\fi

% Optional override for |\version| flag:
%    \begin{macrocode}
%%\providecommand{\version}{final}
%    \end{macrocode}

% Include the main document:
%    \begin{macrocode}
\input{childdoc.def}
\childdocby{cdocsamp}
%    \end{macrocode}

%\iffalse
%</samplepart3|samplepart4>
%\fi
%
%\iffalse
%<*samplepart3>
%\fi
% Some text for part 3:
%    \begin{macrocode}
some text in part three
%    \end{macrocode}

%\iffalse
%</samplepart3>
%\fi
% Some text for part 4:
%\iffalse
%<*samplepart4>
%\fi
%    \begin{macrocode}
more text in part four
%    \end{macrocode}

%\iffalse
%</samplepart4>
%\fi
%
% %%%%%%%%%%%%%%%%%%%%%%%%%%%%%%%%%%%%%%
% \paragraph{Forwarding for a Complete Draft.}
%
% The following forwarding file |cdocsdrf.tex|
% compiles the main document in draft mode:
%\iffalse
%<*sampledraft>
%\fi
%    \begin{macrocode}
\def\version{draft}
\input{childdoc.def}
\childdocforward{cdocsamp}
%    \end{macrocode}

%\iffalse
%</sampledraft>
%\fi
%
% %%%%%%%%%%%%%%%%%%%%%%%%%%%%%%%%%%%%%%
% \paragraph{Forwarding for Final Version of the Chapters.}
%
% The following forwarding files |cdocsfn1.tex| and |cdocsfn2.tex|
% (with identical content)
% compile the final versions of the child documents
% |cdocsch1.tex| and |cdocsch2.tex|, respectively:
%\iffalse
%<*samplefinal>
%\fi
%    \begin{macrocode}
\def\version{final}
\input{childdoc.def}
\childdocforwardprefix[cdocsamp]{cdocsfn}{cdocsch}
%    \end{macrocode}

%\iffalse
%</samplefinal>
%\fi
%
% %%%%%%%%%%%%%%%%%%%%%%%%%%%%%%%%%%%%%%
% \paragraph{Command Line Processing.}
%
% The following three command lines generate the output files
% |cdocscld|, |cdocscl1| and |cdocscl2|
% which should be identical to
% |cdocsdrf|, |cdocsch1| and |cdocsfn2|, respectively:
% \begin{center}
% \begin{tabular}{l}
% |latex -jobname cdocscld \|\\
% |  "\def\version{draft}\input{childdoc.def}\childdocforward{cdocsamp}"|\\
% |latex -jobname cdocscl1 \|\\
% |  "\input{childdoc.def}\childdocforward[cdocsamp]{cdocsch1}"|\\
% |latex -jobname cdocscl2 \|\\
% |  "\def\version{final}\input{childdoc.def}\childdocforward{cdocsch2}"|
% \end{tabular}
% \end{center}
% Note that the trailing backslash on each first line
% merely continues the input to the second line
% (for convenient cut ant paste).
% Furthermore, the command |latex| can be replaced by any
% of its alternative versions such as |pdflatex|.
%
% %%%%%%%%%%%%%%%%%%%%%%%%%%%%%%%%%%%%%%%%%%%%%%%%%%%%%%%%%%%%%%%%%%%%%%%%%%%%%%
% %%%%%%%%%%%%%%%%%%%%%%%%%%%%%%%%%%%%%%%%%%%%%%%%%%%%%%%%%%%%%%%%%%%%%%%%%%%%%%
% \section{Implementation}
%\iffalse
%<*package>
%\fi
%
% This section describes the definitions file |childdoc.def|.

% The definitions cannot be loaded using |\usepackage| or |\RequirePackage|
% which has a mechanism to prevent loading a style file more than once.
% When loading the definitions by means of |\input|
% multiple instances have to be prevented manually:
%\iffalse
%This code needs to be before the `\ProvidesFile' directive
%which is defined at the beginning of this file.
%Therefore it is also placed there and commented out here.
%</package>
%<*discard>
%\fi
%    \begin{macrocode}
\ifdefined\childdocmain\endinput\fi
%    \end{macrocode}
%\iffalse
%</discard>
%<*package>
%\fi
%
% \macro{\ifchilddoc}
% \macro{\ifchilddocmanual}
% The conditional |\ifchilddoc| tells whether a
% child (true) or main (false) document is being compiled.
% The conditional |\ifchilddocmanual| tells whether
% the |\includeonly| mechanism is used (false) or
% the selection of child files must be performed manually (true).
% The definitions initialise to false:
%    \begin{macrocode}
\newif\ifchilddoc
\newif\ifchilddocmanual
%    \end{macrocode}

% \macro{\childdocname}
% \macro{\childdocjob}
% The macro |\childdocname| stores the name of the main document
% to be compiled. The macro |\childdocjob| stores the name of
% the document on which the \LaTeX{} compiler was originally invoked.
% The content of |\jobname| cannot be compared
% to filenames specified in the source due to different catcodes.
% The following code rescans |\jobname|, stores the result
% in |\childdocname| and saves a copy in |\childdocjob|:
%    \begin{macrocode}
\edef\childdocname{\scantokens\expandafter{\jobname\noexpand}}
\let\childdocjob\childdocname
%    \end{macrocode}

% \macro{\childdocdisable}
% The macro |\childdocdisable| prevents the main file
% from being processed more than once.
% At this stage, the main document command |\childdocmain|
% is assumed to be called once again where it should do nothing.
% Any subsequent call to it should prevent
% a secondary processing of the main document
% It overwrites the forwarding commands
% |\childdocof| and |\childdocforward|
% with empty macros to prevent further inclusions of the main document:
%    \begin{macrocode}
\newcommand{\childdocdisable}
{
  \renewcommand{\childdocmain}[1]{\renewcommand{\childdocmain}[1]{\endinput}}
  \renewcommand{\childdocof}[1]{}
  \renewcommand{\childdocby}[2][]{}
  \renewcommand{\childdocforward}[2][]{}
  \renewcommand{\childdocdisable}{}
}
%    \end{macrocode}

% \macro{\childdocmain}
% The macro |\childdocmain| is to be called at the top of the main file
% with nothing or the main filename (without extension) as argument.
% First, it breaks loops.
% If the argument is not empty and does not match |\childdocname|
% (which is set by the first inclusion of |childdoc.def|),
% |\ifchilddoc| is set to true, |\includeonly| is applied to the child file
% and |\jobname| is set to the main file
% (for proper handling of |.aux| files):
%    \begin{macrocode}
\newcommand{\childdocmain}[1]
{
  \childdocdisable\childdocmain{}
  \if?#1?\else
    \begingroup
      \def\childdoctmp{#1}
      \ifx\childdoctmp\childdocname
        \def\childdoctmp{}
      \else
        \def\childdoctmp
        {
          \childdoctrue
          \includeonly{\childdocname}
          \def\childdocjob{#1}
          \def\jobname{#1}
        }
      \fi
      \expandafter
    \endgroup
    \childdoctmp
  \fi
}
%    \end{macrocode}

% \macro{\childdocof}
% The command |\childdocof| redirects
% compilation to the main file |#1|.
%    \begin{macrocode}
\newcommand{\childdocof}[1]
{
  \childdocdisable
  \childdoctrue
  \includeonly{\childdocname}
  \def\jobname{#1}
  \def\childdocjob{#1}
  \input{#1}
}
%    \end{macrocode}

% \macro{\childdocby}
% The command |\childdocby| ....
%    \begin{macrocode}
\newcommand{\childdocby}[2][]
{
  \childdocdisable
  \childdoctrue
  \childdocmanualtrue
  \if?#1?\else
    \def\jobname{#2}
  \fi
  \def\childdocjob{#2}
  \input{#2}
  \endinput
}
%    \end{macrocode}

% \macro{\childdocforward}
% The command |\childdocforward| redirects
% compilation to the main file or
% (if the optional argument is given) a child file.
% Parameters are set as if the main file
% or a child file starting with |\childdocof| was compiled.
% Then compilation is handed over to the main file:
%    \begin{macrocode}
\newcommand{\childdocforward}[2][]
{
  \begingroup
    \if?#1?
      \def\childdoctmp
      {
        \def\childdocname{#2}
        \def\childdocjob{#2}
        \def\jobname{#2}
        \input{#2}
        \endinput
      }
    \else
      \def\childdoctmp
      {
        \childdocdisable
        \def\childdocname{#2}
        \childdoctrue
        \includeonly{#2}
        \def\childdocjob{#1}
        \def\jobname{#1}
        \input{#1}
        \endinput
      }
    \fi
    \expandafter
  \endgroup
  \childdoctmp
}
%    \end{macrocode}

% \macro{\childdocforwardprefix}
% The command |\childdocforwardprefix| redirects
% compilation to the main or a child file by means of a pattern.
% The prefix |#1| in the current filename is replaced by |#2|
% and the suffix of the current filename is kept
% (it is assumed that the filename does not contain the substring `|~~~|'
% which is used as a delimiter).
% Compilation is handed over to the new file by |\childdocforward|:
%    \begin{macrocode}
\newcommand{\childdocforwardprefix}[3][]
{
  \begingroup
    \def\childdocextract #2##1~~~{\def\childdoctmp{\childdocforward[#1]{#3##1}}}
    \expandafter\childdocextract\childdocname~~~
    \expandafter
  \endgroup
  \childdoctmp
}
%    \end{macrocode}

% \macro{\childdoc}
% The deprecated macro |\childdoc| is a legacy version of |\childdocmain|:
%    \begin{macrocode}
\newcommand{\childdoc}{\childdocmain}
%    \end{macrocode}

% \macro{\childdocredirect}
% The deprecated macro |\childdocredirect| is a legacy version
% of |\childdocforward| and |\childdocforwardprefix|:
%    \begin{macrocode}
\newcommand{\childdocredirect}[2][]
{
  \begingroup
    \if?#1?
      \def\childdoctmp{\childdocforward{#2}}
    \else
      \def\childdoctmp{\childdocforwardprefix{#1}{#2}}
    \fi
    \expandafter
  \endgroup
  \childdoctmp
}
%    \end{macrocode}

%\iffalse
%</package>
%\fi
%
\endinput
|\\
|\childdocforward[|\textit{main}|]{|\textit{dest}|}|\\
\end{tabular}
\end{center}
%
The argument \textit{dest} is the destination file
(without extension).
It should be the main file or one of the child files.
Note that further \textsf{childdoc} directives
such as |\childdocof| and |\childdocforward|
in the indicated file will be processed in this form.
The optional argument \textit{main}
passes on directly to the main file \textit{main}
while pretending to compile the child \textit{dest}.
This form behaves as if \textit{dest}
issues |\childdocof{|\textit{main}|}| right away,
and no further \textsf{childdoc} directives will be processed.

%%%%%%%%%%%%%%%%%%%%%%%%%%%%%%%%%%%%%%%%
\DescribeMacro{\...prefix}
In the alternative form |\childdocforwardprefix|,
%
\begin{center}
\begin{tabular}{l}
|% \iffalse
%
% childdoc.dtx Copyright (C) 2017-2018 Niklas Beisert
%
% This work may be distributed and/or modified under the
% conditions of the LaTeX Project Public License, either version 1.3
% of this license or (at your option) any later version.
% The latest version of this license is in
%   http://www.latex-project.org/lppl.txt
% and version 1.3 or later is part of all distributions of LaTeX
% version 2005/12/01 or later.
%
% This work has the LPPL maintenance status `maintained'.
%
% The Current Maintainer of this work is Niklas Beisert.
%
% This work consists of the files childdoc.dtx and childdoc.ins
% and the derived files childdoc.def and cdocsamp.tex with
% cdocsch1.tex, cdocsch2.tex, cdocsdrf.tex, cdocsfn1.tex, cdocsfn2.tex.
%
%<package>\ifdefined\childdocmain\endinput\fi
%<package>\ProvidesFile{childdoc.def}[2018/12/30 v2.0 child document driver]
%<samplemain>\ProvidesFile{cdocsamp.tex}[2018/12/30 v2.0 sample for childdoc]
%<*driver>
%\ProvidesFile{childdoc.drv}[2018/12/30 v2.0 childdoc reference manual file]
\PassOptionsToClass{10pt,a4paper}{article}
\documentclass{ltxdoc}

\usepackage[margin=35mm]{geometry}
\usepackage{hyperref}
\usepackage{hyperxmp}
\usepackage[usenames]{color}

\hypersetup{colorlinks=true}
\hypersetup{pdfstartview=FitH}
\hypersetup{pdfpagemode=UseNone}
\hypersetup{pdfsource={}}
\hypersetup{pdflang={en-UK}}
\hypersetup{pdfcopyright={Copyright 2017-2018 Niklas Beisert.
  This work may be distributed and/or modified under the
  conditions of the LaTeX Project Public License, either version 1.3
  of this license or (at your option) any later version.}}
\hypersetup{pdflicenseurl={http://www.latex-project.org/lppl.txt}}
\hypersetup{pdfcontactaddress={ETH Zurich, ITP, HIT K,
  Wolfgang-Pauli-Strasse 27}}
\hypersetup{pdfcontactpostcode={8093}}
\hypersetup{pdfcontactcity={Zurich}}
\hypersetup{pdfcontactcountry={Switzerland}}
\hypersetup{pdfcontactemail={nbeisert@itp.phys.ethz.ch}}
\hypersetup{pdfcontacturl={http://people.phys.ethz.ch/\xmptilde nbeisert/}}

\newcommand{\secref}[1]{\hyperref[#1]{section \ref*{#1}}}

\parskip1ex
\parindent0pt
\let\olditemize\itemize
\def\itemize{\olditemize\parskip0pt}

\begin{document}

\title{The \textsf{childdoc} Package}
\hypersetup{pdftitle={The childdoc Package}}
\author{Niklas Beisert\\[2ex]
  Institut f\"ur Theoretische Physik\\
  Eidgen\"ossische Technische Hochschule Z\"urich\\
  Wolfgang-Pauli-Strasse 27, 8093 Z\"urich, Switzerland\\[1ex]
  \href{mailto:nbeisert@itp.phys.ethz.ch}
  {\texttt{nbeisert@itp.phys.ethz.ch}}}
\hypersetup{pdfauthor={Niklas Beisert}}
\hypersetup{pdfsubject={Manual for the LaTeX2e Package childdoc}}
\date{30 December 2018, \textsf{v2.0}}
\maketitle

\begin{abstract}\noindent
\textsf{childdoc} is a \LaTeXe{} package
that enables the direct compilation
of document sections included by |\include|
to individual files.
\end{abstract}

\begingroup
\parskip0ex
\tableofcontents
\endgroup

%%%%%%%%%%%%%%%%%%%%%%%%%%%%%%%%%%%%%%%%%%%%%%%%%%%%%%%%%%%%%%%%%%%%%%%%%%%%%%%%
%%%%%%%%%%%%%%%%%%%%%%%%%%%%%%%%%%%%%%%%%%%%%%%%%%%%%%%%%%%%%%%%%%%%%%%%%%%%%%%%
\section{Introduction}

\LaTeX{} provides a mechanism to structure a large document (such as a book)
into a main file and several child files (containing the chapters)
using the |\include| command.
This mechanism is beneficial for documents
which span hundreds of pages in order to
make the source file(s) more manageable.
Moreover, compilation can be restricted to
selected child files by means of the |\includeonly| command.
The latter feature can be used to reduce the compilation time while editing
(this was significantly more useful in the earlier days of \LaTeX{})
or to generate a smaller document which is easier to navigate.
Another application of |\includeonly| is to generate
documents consisting of selected parts of the complete document.

However, there are a few drawbacks of the plain |\include| mechanism:
\begin{itemize}
\item
The child files cannot be compiled on their own,
they can only be compiled via the main file.
A naive editing environment
(such as a text editor with an option
to have the current file processed by \LaTeX)
may require one to switch to the main file before compiling;
attempting to compile the child file produces errors.
\item
The main file must be modified (each time)
to adjust the |\includeonly| command
to the present needs. This easily leaves the main file in a messy state.
\item
The generated document will always carry the filename
of the main document. This is inconvenient if
several child files are to be compiled and
to be kept for distribution.
\end{itemize}

The present package provides a simple interface
to make child files individually compilable by \LaTeX{}.
Compiling a child file then has the same effect as compiling
the main file with an |\includeonly| command
to select the appropriate child.
Moreover the generated document will carry the name of the child
rather than the main file.
This resolves all three above issues.

This feature is meant to make the editing of books,
thesis documents and lecture notes somewhat more convenient.
However, the package can also be used efficiently for
composing a series of documents (such as exercise sheets)
which are typically distributed individually.
It then assists the author in generating the individual documents
(potentially in different versions)
as well as a document containing the collected series.
Another application is in developing style files
or other kinds of included material
where compilation of the style file could redirect
to a sample or test file.

%%%%%%%%%%%%%%%%%%%%%%%%%%%%%%%%%%%%%%%%%%%%%%%%%%%%%%%%%%%%%%%%%%%%%%%%%%%%%%%%
%%%%%%%%%%%%%%%%%%%%%%%%%%%%%%%%%%%%%%%%%%%%%%%%%%%%%%%%%%%%%%%%%%%%%%%%%%%%%%%%
\section{Usage}

First of all, the package \textsf{childdoc} is \emph{not} a standard
\LaTeXe{} |.sty| style file! Therefore it needs to be invoked in
a non-standard way.

%%%%%%%%%%%%%%%%%%%%%%%%%%%%%%%%%%%%%%%%%%%%%%%%%%%%%%%%%%%%%%%%%%%%%%%%%%%%%%%%
\subsection{Included Files}
\label{sec:include}

%%%%%%%%%%%%%%%%%%%%%%%%%%%%%%%%%%%%%%%%
\DescribeMacro{\childdocmain}
To use the package, add the commands
\begin{center}
\begin{tabular}{l}
|\input{childdoc.def}|\\
|\childdocmain{}|\\
\end{tabular}
\end{center}
at the very top of the main \LaTeX{} file,
in particular \emph{before} the |\documentclass| statement!
The argument of |\childdocmain| should be left empty
(but it must be present).

%%%%%%%%%%%%%%%%%%%%%%%%%%%%%%%%%%%%%%%%
\DescribeMacro{\childdocof}
Furthermore, add the commands
\begin{center}
\begin{tabular}{l}
|\input{childdoc.def}|\\
|\childdocof{|\textit{main}|}|\\
\end{tabular}
\end{center}
at the top of every child file \textit{child}
which is included by |\include{|\textit{child}|}|
from within the main file
(or at least for those files to be compiled individually).
The argument \textit{main} must be the filename of the main file.

There are a couple of
considerations in setting up the main and child documents:

%%%%%%%%%%%%%%%%%%%%%%%%%%%%%%%%%%%%%%%%
\paragraph{Restrictions.}

Please note the following restrictions:
\begin{itemize}
\item
|\childdocmain| must be called with one argument \textit{main}
to ensure compatibility with earlier version of the package.
It must either be empty (|\childdocmain{}|)
or precisely match the filename of the main file in which it is specified.
See \secref{sec:detection} for further information.
\item
The filename \textit{main} must be specified without the |.tex| extension.
\item
The filename \textit{main} is case sensitive
(even in case-insensitive file systems)
due to internal string comparison.
\item
The argument \textit{main} should be fully expanded, it cannot be a macro.
\item
Subdirectories and special characters should be avoided in filenames.
\item
The command |\childdocmain{|\textit{main}|}| must be followed by a whitespace.
It should not be followed immediately by another command
or by a comment mark `|%|'.
This is because the \TeX{} parser reads the token immediately following
the argument of |\childdocmain| and puts it
at the beginning of every child section;
however, a white\-space is ignored.
\end{itemize}

%%%%%%%%%%%%%%%%%%%%%%%%%%%%%%%%%%%%%%%%
\paragraph{Content of Main File.}

It is advisable to place all content in the child files included by |\include|.
Any output contained in the main file will appear in all child documents
unless suppressed manually;
it cannot be suppressed automatically by the |\includeonly| directive
and thus should normally be avoided.
A method to include some content in the main file
by means of conditional processing is described in \secref{sec:conditional}.

%%%%%%%%%%%%%%%%%%%%%%%%%%%%%%%%%%%%%%%%
\paragraph{Page Numbering.}

When only a part of the document is compiled,
the appropriate numbering of pages
(as well as other status parameters)
is determined from the |.aux| files.
The latter contain information from previous passes.
However this information needs to propagate through
all intermediate child documents.
Therefore the page numbering in child documents may well
be inconsistent until the complete document is compiled at least once.

A useful (if unconventional) way to always ensure a consistent
page numbering is to restart the numbering in each child document
and denote the pages by `\textit{child}|.|\textit{page}'
where \textit{child} represents the chapter/section number of the child file.
This can be achieved by the command
|\numberwithin{page}{|\textit{child}|}|
of the \textsf{amsmath} package
where \textit{child} can be |chapter| or |section|
depending on the chosen structuring.
Alternatively, one can modify the macro |\thepage| appropriately
and reset the counter |page| at the start of each child file.

%%%%%%%%%%%%%%%%%%%%%%%%%%%%%%%%%%%%%%%%%%%%%%%%%%%%%%%%%%%%%%%%%%%%%%%%%%%%%%%%
\subsection{Conditional Processing}
\label{sec:conditional}

The package provides a mechanism to compile different versions
of a document. To customise the versions further some conditional processing
can come in handy to distinguish which version is being compiled.
The package provides two macros to describe the compilation context:

%%%%%%%%%%%%%%%%%%%%%%%%%%%%%%%%%%%%%%%%
\DescribeMacro{\ifchilddoc}
The conditional |\ifchilddoc| distinguishes between the compilation of
child documents and the main document:
%
\begin{center}
|\ifchilddoc |\textit{child-code}| |[|\||else |\textit{main-code}]| \||fi|
\end{center}

%%%%%%%%%%%%%%%%%%%%%%%%%%%%%%%%%%%%%%%%
\DescribeMacro{\childdocname}
\DescribeMacro{\childdocjob}
The macro |\childdocname| contains the filename (without extension)
of the main or child file being processed.
Note that |\childdocjob| will always contain the name of the main file.

%%%%%%%%%%%%%%%%%%%%%%%%%%%%%%%%%%%%%%%%
\paragraph{Title Page.}

Conditional processing can be used to include a title or banner page
in the main document when proper precautions are taken.
Importantly, the code in the main file should ensure that the page counter
(as well as other status parameters which are stored in the |.aux| files)
takes the same value after the conditional processing.
Otherwise the page numbers may take divergent values
depending on which part is compiled.

For example, a title page could be declared by:
%
\begin{center}
\begin{tabular}{l}
|\ifchilddoc\||else|\\
|\addtocounter{page}{-1}|\\
\textit{code for title page}\\
|\newpage|\\
|\||fi|
\end{tabular}
\end{center}
%
A banner page for the child documents can be generated by:
%
\begin{center}
\begin{tabular}{l}
|\ifchilddoc|\\
|\addtocounter{page}{-1}|\\
\textit{code for banner page}\\
|\newpage|\\
|\||fi|
\end{tabular}
\end{center}
%
Here one could write a message such as:
\begin{center}
|This is the part \childdocname{} of \childdocjob{}.|
\end{center}

%%%%%%%%%%%%%%%%%%%%%%%%%%%%%%%%%%%%%%%%%%%%%%%%%%%%%%%%%%%%%%%%%%%%%%%%%%%%%%%%
\subsection{Flags}
\label{sec:flags}

The package makes it easy to generate different versions
of the main or child documents.
To this end compilation flags can be defined
and assigned different default values.
They will be particularly useful in conjunction
with the forwarding mechanism described in \secref{sec:forward}.

For example, it may be useful to have a flag |\version|
which can be set to |draft| or |final|.
The document source will contain some conditional code
depending on the value of |\version|.
Suppose further, the flag should default to |final| for the main file
and to |draft| for child files
which is a natural assignment for editing the document.
This is achieved by placing the following code
in the preamble of the main document
(below the |\childdocmain| directive):
%
\begin{center}
\begin{tabular}{l}
|\ifchilddoc|\\
|\providecommand{\version}{draft}|\\
|\||else|\\
|\providecommand{\version}{final}|\\
|\||fi|
\end{tabular}
\end{center}
%
The definition by |\providecommand| makes sure
that previous definitions are not overwritten.
Further statements |\providecommand{\version}{...}|
can thus be added before the above code to override it.

For the main file, one might add a line
(between |\childdocmain| and the above block)
%
\begin{center}
|%\ifchilddoc\||else\providecommand{\version}{draft}\||fi|
\end{center}
%
which can be uncommented to produce a draft version.
Likewise one can add a line to the very top of a child file
(above the |\childdocof{|\textit{main}|}| directive)
%
\begin{center}
|%\providecommand{\version}{final}|
\end{center}
%
which can be uncommented to produce the final version of this child document.

%%%%%%%%%%%%%%%%%%%%%%%%%%%%%%%%%%%%%%%%%%%%%%%%%%%%%%%%%%%%%%%%%%%%%%%%%%%%%%%%
\subsection{Forwarding}
\label{sec:forward}

Different versions of the main or child documents
using compilation flags as described in \secref{sec:flags}
can be (permanently) stored in different files
for convenient compilation, viewing and distribution.
To this end, the package defines a command
to pass on compilation to a different file:

%%%%%%%%%%%%%%%%%%%%%%%%%%%%%%%%%%%%%%%%
\DescribeMacro{\childdocforward}
The command |\childdocforward| redirects processing to
another source file:
%
\begin{center}
\begin{tabular}{l}
|\input{childdoc.def}|\\
|\childdocforward[|\textit{main}|]{|\textit{dest}|}|\\
\end{tabular}
\end{center}
%
The argument \textit{dest} is the destination file
(without extension).
It should be the main file or one of the child files.
Note that further \textsf{childdoc} directives
such as |\childdocof| and |\childdocforward|
in the indicated file will be processed in this form.
The optional argument \textit{main}
passes on directly to the main file \textit{main}
while pretending to compile the child \textit{dest}.
This form behaves as if \textit{dest}
issues |\childdocof{|\textit{main}|}| right away,
and no further \textsf{childdoc} directives will be processed.

%%%%%%%%%%%%%%%%%%%%%%%%%%%%%%%%%%%%%%%%
\DescribeMacro{\...prefix}
In the alternative form |\childdocforwardprefix|,
%
\begin{center}
\begin{tabular}{l}
|\input{childdoc.def}|\\
|\childdocforwardprefix[|\textit{main}|]{|\textit{prefix}|}{|\textit{dest}|}|
\end{tabular}
\end{center}
%
the destination file is determined by a pattern
depending on the current file:
To make this work, the current file must be called
`{\textit{prefix}\hspace{0.2em}\textit{suffix}}'
with \textit{prefix} matching precisely the argument.
Processing is then passed on to the file
`{\textit{dest}\hspace{0.2em}\textit{suffix}}'.
Surely, the same effect is achieved by
directly specifying the
argument `{\textit{dest}\hspace{0.2em}\textit{suffix}}'
in the first form.
However, that requires to set up a different file
for each child. With the alternative form of the command
all these files can have exactly the same content
which simplifies setting them up and maintaining them.

For example, the following file |draft.tex|
with a compilation flag |\version| as described in \secref{sec:flags}
compiles the main document as a draft:
%
\begin{center}
\begin{tabular}{l}
|\def\version{draft}|\\
|\input{childdoc.def}|\\
|\childdocforward{|\textit{main}|}|
\end{tabular}
\end{center}
%
Likewise, the following files |final|\textit{nn}|.tex|
compile the final version of the child document
|child|\textit{nn}|.tex|:
%
\begin{center}
\begin{tabular}{l}
|\def\version{final}|\\
|\input{childdoc.def}|\\
|\childdocforwardprefix{final}{child}|
\end{tabular}
\end{center}
%

Note that when several versions of a main file and/or of each child file
are to be generated, it may be convenient to set up a |Makefile| or
shell script to automatise the process.

%%%%%%%%%%%%%%%%%%%%%%%%%%%%%%%%%%%%%%%%%%%%%%%%%%%%%%%%%%%%%%%%%%%%%%%%%%%%%%%%
\subsection{Command Line Processing}
\label{sec:commandline}

The effect of redirection files can also be achieved by invoking
the \LaTeX{} compiler with a more elaborate command line.
Most conveniently this should be done as part
of a shell script or a |Makefile|.

When using \textsf{childdoc} in the main file, the following
command lines effectively perform a redirection
(note that depending on the shell being used,
backslashes may have to be doubled: `|\|' $\to$ `|\\|'):
%
\begin{center}
|... -jobname "|\textit{target}|" |\\|"|[\textit{flags}]%
|\input{childdoc.def}\childdocforward[|\textit{main}|]{|\textit{dest}|}"|
\end{center}
%
Here \textit{target} is the name of the output file,
\textit{main} is the name of the main file
and \textit{dest} is the name of the main or child file to be processed
(all filenames without extensions).
The optional argument \textit{main} can be omitted
if \textit{main} matches \textit{dest}.
Optionally, compilation \textit{flags} can be defined via |\def| commands.
This command line makes the \TeX{} engine believe
it is compiling the file \textit{target}
whose content is specified as the latter parameter.
The provided code then forwards the processing to
\textit{main} or \textit{dest} as described in \secref{sec:forward}.

%%%%%%%%%%%%%%%%%%%%%%%%%%%%%%%%%%%%%%%%%%%%%%%%%%%%%%%%%%%%%%%%%%%%%%%%%%%%%%%%
\subsection{Include by Input}
\label{sec:input}

Including child documents by |\include| has some restrictions by design.
Most notably, the content of a child document always occupies
its own set of pages; pages cannot be shared between child documents.
Usually, this behaviour makes perfect sense
because each child document contain an essential part of the document.
However, in some situations it may be desirable to compose
a document from a collection of parts
without having mandatory page breaks between then.
For this case, the package
provides a mechanism to include parts
by |\input| which can also be processed individually.
However, by construction this mechanism
requires manual handling of the content to be output.

%%%%%%%%%%%%%%%%%%%%%%%%%%%%%%%%%%%%%%%%
\DescribeMacro{\ifchilddocmanual}
The main file should be prepared as usual, see \secref{sec:include}.
However, the document body must make a distinction
between processing of an individual part and of the main document, e.g.:
%
\begin{center}
\begin{tabular}{l}
|\ifchilddocmanual|\\
|\input{\childdocname}|\\
|\||else|\\
\textit{document body with }|\input{|\textit{part}|}|\\
|\||fi|
\end{tabular}
\end{center}
%
The conditional |\ifchilddocmanual| is true whenever
a part to be included by |\input| is being compiled,
and the name of the part is stored in |\childdocname|.

%%%%%%%%%%%%%%%%%%%%%%%%%%%%%%%%%%%%%%%%
\DescribeMacro{\childdocby}
Each part to be included by |\input| should start with:
%
\begin{center}
\begin{tabular}{l}
|\input{childdoc.def}|\\
|\childdocby{|\textit{main}|}|\\
\end{tabular}
\end{center}
%
The directive |\childdocby| is similar to |\childdocof|
described in \secref{sec:include},
but the subsequent selection of content must be done manually.
To that end, both |\ifchilddoc| and |\ifchilddocmanual|
will be true upon processing of a part,
and the name of the part is stored in |\childdocname|.
Note that |\jobname| will be set to the filename of the current part
so that each part receives an individual |.aux| file
that does not interfere with the |.aux| file(s) of the main document.
This behaviour can be altered by the alternative form
|\childdocby[*]{|\textit{main}|}| (with a non-empty optional argument)
which uses the |.aux| file of the main document
by setting |\jobname| to \textit{main}.

%%%%%%%%%%%%%%%%%%%%%%%%%%%%%%%%%%%%%%%%%%%%%%%%%%%%%%%%%%%%%%%%%%%%%%%%%%%%%%%%
\subsection{Driver Development}
\label{sec:driver}

The \textsf{childdoc} mechanism can also be use for the development
of definition files such as \LaTeX{} styles or classes.
This case differs from the above setup with multiple parts
included by |\include| in that no |\includeonly| should be invoked.
This can be achieved by starting the include file
(before |\ProvidesPackage|) with:
%
\begin{center}
\begin{tabular}{l}
|\input{childdoc.def}|\\
|\childdocforward{|\textit{main}|}|\\
\end{tabular}
\end{center}
%
or alternatively with:
%
\begin{center}
\begin{tabular}{l}
|\input{childdoc.def}|\\
|\childdocby{|\textit{main}|}|\\
\end{tabular}
\end{center}
%
Both forms have slightly different effects as described above.
The main file is prepared as usual, see \secref{sec:include}.

%%%%%%%%%%%%%%%%%%%%%%%%%%%%%%%%%%%%%%%%%%%%%%%%%%%%%%%%%%%%%%%%%%%%%%%%%%%%%%%%
\subsection{Legacy Detection}
\label{sec:detection}

The directive |\childdocmain| in the main file can detect
whether the complete document or merely a child is to be compiled
even without using the directive |\childdocof|.
This method is deprecated because it is less robust
and there is no compelling reason to use it;
it is merely provided for backward compatibility
and it may be removed in future versions.

If the detection mechanism is to be used,
it is mandatory to correctly specify
the filename of the main file as the argument of |\childdocmain|:
%
\begin{center}
\begin{tabular}{l}
|\input{childdoc.def}|\\
|\childdocmain{|\textit{main}|}|\\
\end{tabular}
\end{center}
%
If |\jobname| does not match the argument \textit{main} of |\childdocmain|,
it is assumed that |\jobname| points to the child file to be compiled.
When using |\childdocmain| with the main file specified as argument,
it suffices to start a child file
with just |\input{|\textit{main}|}|
without loading of the package and using |\childdocof|.
If instead all processing is done
with the appropriate \textsf{childdoc} directives,
the argument of \textit{main} of |\childdocmain| can be empty.

An alternative version of the command line processing described
in \secref{sec:commandline} using the detection mechanism reads:
%
\begin{center}
|... -jobname "|\textit{target}|" "|[\textit{flags}]%
[|\def\jobname{|\textit{dest}|}|]|\input{|\textit{main}|}"|
\end{center}

%%%%%%%%%%%%%%%%%%%%%%%%%%%%%%%%%%%%%%%%%%%%%%%%%%%%%%%%%%%%%%%%%%%%%%%%%%%%%%%%
\subsection{Manual Code}
\label{sec:manual}

In case one cannot be certain whether the definitions file |childdoc.def|
is installed on the target \TeX{} distribution
and one prefers not to ship it,
it is conceivable to paste a few relevant commands into the sources.

To that end, drop all statements |\input{childdoc.def}|
and perform the replacements as outlined below.
Instead of |\childdocmain{|\textit{main}|}| add the following code
to the top of the main file:
%
\begin{center}
\begin{tabular}{l}
|\||ifdefined\childdocname\endinput\||fi\newif\ifchilddoc|\\
|\edef\childdocname{\scantokens\expandafter{\jobname\noexpand}}|\\
|\def\childdocmain{|\textit{main}|}\||ifx\childdocmain\childdocname\||else|\\
|\childdoctrue\includeonly{\childdocname}\let\jobname\childdocmain\||fi|\\
\end{tabular}
\end{center}
%
Instead of |\childdocof{|\textit{main}|}| just include the main file
at the top of each child file:
%
\begin{center}
|\input{|\textit{main}|}|
\end{center}
%
A simple redirection |\childdocforward{|\textit{dest}|}| is achieved by:
%
\begin{center}
|\def\jobname{|\textit{dest}|}\input{\jobname}|
\end{center}
%
The redirection with prefix
|\childdocforwardprefix[|\textit{prefix}|]{|\textit{dest}|}|
is accomplished by:
%
\begin{center}
\begin{tabular}{l}
|{\edef\jobname{\scantokens\expandafter{\jobname\noexpand}}|\\
|\def\redirectjob |\textit{prefix}|#1~~~{\gdef\jobname{|\textit{dest}|#1}}|\\
|\expandafter\redirectjob\jobname~~~}\input{\jobname}|
\end{tabular}
\end{center}

In an alternative approach,
child documents can be compiled by a specific command line
without additional code or specific definitions:
%
\begin{center}
|... -jobname "|\textit{target}|" "|[\textit{flags}]%
|\includeonly{|\textit{dest}|}\input{|\textit{main}|}"|
\end{center}
%

%%%%%%%%%%%%%%%%%%%%%%%%%%%%%%%%%%%%%%%%%%%%%%%%%%%%%%%%%%%%%%%%%%%%%%%%%%%%%%%%
%%%%%%%%%%%%%%%%%%%%%%%%%%%%%%%%%%%%%%%%%%%%%%%%%%%%%%%%%%%%%%%%%%%%%%%%%%%%%%%%
\section{Information}

%%%%%%%%%%%%%%%%%%%%%%%%%%%%%%%%%%%%%%%%%%%%%%%%%%%%%%%%%%%%%%%%%%%%%%%%%%%%%%%%
\subsection{Copyright}

Copyright \copyright{} 2017--2018 Niklas Beisert

This work may be distributed and/or modified under the
conditions of the \LaTeX{} Project Public License, either version 1.3
of this license or (at your option) any later version.
The latest version of this license is in
  \url{http://www.latex-project.org/lppl.txt}
and version 1.3 or later is part of all distributions of \LaTeX{}
version 2005/12/01 or later.

This work has the LPPL maintenance status `maintained'.

The Current Maintainer of this work is Niklas Beisert.

This work consists of the files |README.txt|, |childdoc.ins| and |childdoc.dtx|
as well as the derived files |childdoc.def|, |cdocsamp.tex|
with |cdocsch1.tex|, |cdocsch2.tex|, |cdocspt3.tex|, |cdocspt4.tex|,
|cdocsdrf.tex|, |cdocsfn1.tex|, |cdocsfn2.tex|
as well as |childdoc.pdf|.

%%%%%%%%%%%%%%%%%%%%%%%%%%%%%%%%%%%%%%%%%%%%%%%%%%%%%%%%%%%%%%%%%%%%%%%%%%%%%%%%
\subsection{Files and Installation}

The package consists of the files:
%
\begin{center}
\begin{tabular}{ll}
    |README.txt|   & readme file \\
    |childdoc.ins| & installation file \\
    |childdoc.dtx| & source file \\
    |childdoc.def| & definition file \\
    |cdocsamp.tex| & sample main file \\
    |cdocsch1.tex| & sample include file \\
    |cdocsch2.tex| & sample include file \\
    |cdocspt3.tex| & sample part file \\
    |cdocspt4.tex| & sample part file \\
    |cdocsdrf.tex| & sample redirection file \\
    |cdocsfn1.tex| & sample redirection file \\
    |cdocsfn2.tex| & sample redirection file \\
    |childdoc.pdf| & manual
\end{tabular}
\end{center}
%
The distribution consists of the files
|README.txt|, |childdoc.ins| and |childdoc.dtx|.
%
\begin{itemize}
\item
Run (pdf)\LaTeX{} on |childdoc.dtx|
to compile the manual |childdoc.pdf| (this file).
\item
Run \LaTeX{} on |childdoc.ins| to create the definitions file |childdoc.def|
and the sample |cdocsamp.tex| with include files
|cdocsch1.tex|, |cdocsch2.tex|, |cdocspt3.tex|, |cdocspt4.tex|,
|cdocsdrf.tex|, |cdocsfn1.tex|, |cdocsfn2.tex|.
Then copy the file |childdoc.def| to an appropriate directory of your \LaTeX{}
distribution, e.g.\ \textit{texmf-root}|/tex/latex/childdoc|.
\end{itemize}

%%%%%%%%%%%%%%%%%%%%%%%%%%%%%%%%%%%%%%%%%%%%%%%%%%%%%%%%%%%%%%%%%%%%%%%%%%%%%%%%
\subsection{Related CTAN Packages}

There are several other packages which offer a similar functionality:
%
\begin{itemize}
\item
The packages
\href{http://ctan.org/pkg/docmute}{\textsf{docmute}},
\href{http://ctan.org/pkg/includex}{\textsf{includex}} and
\href{http://ctan.org/pkg/standalone}{\textsf{standalone}}
provide commands to include only the document body of
a child file thus allowing both files to be compiled individually.
\item
The packages \href{http://ctan.org/pkg/subdocs}{\textsf{subdocs}}
and \href{http://ctan.org/pkg/subfiles}{\textsf{subfiles}}
provide structures in which the main and child documents can be
encapsulated and allowing them to be compiled individually.
The inclusion mechanism is different from the conventional |\include|.
\item
The package \href{http://ctan.org/pkg/combine}{\textsf{combine}}
is an elaborate solution to combine several documents into one.
\end{itemize}
%
See also the CTAN topic \href{http://ctan.org/topic/subdocs}{\textsf{subdocs}}
for further related packages.
The present package differs from the above solutions in that
a document structure constructed with the conventional |\include| mechanism
just needs two extra commands at the top of every file
such that all constituent files can be compiled individually.

%%%%%%%%%%%%%%%%%%%%%%%%%%%%%%%%%%%%%%%%%%%%%%%%%%%%%%%%%%%%%%%%%%%%%%%%%%%%%%%%
%\subsection{Feature Suggestions}
%
%The following is a list of features which may be useful for future
%versions of this package:
%%
%\begin{itemize}
%\item
%\ldots
%\end{itemize}

%%%%%%%%%%%%%%%%%%%%%%%%%%%%%%%%%%%%%%%%%%%%%%%%%%%%%%%%%%%%%%%%%%%%%%%%%%%%%%%%
\subsection{Revision History}

%%%%%%%%%%%%%%%%%%%%%%%%%%%%%%%%%%%%%%%%
\paragraph{v2.0:} 2018/12/30

\begin{itemize}
\item
immediate forward processing
\item
added |\childdocby| mechanism
\item
manual restructured
\end{itemize}

%%%%%%%%%%%%%%%%%%%%%%%%%%%%%%%%%%%%%%%%
\paragraph{v1.6:} 2018/01/17

\begin{itemize}
\item
application for development of include files
\item
corrections to manual
\end{itemize}

%%%%%%%%%%%%%%%%%%%%%%%%%%%%%%%%%%%%%%%%
\paragraph{v1.5:} 2017/05/21

\begin{itemize}
\item
more complete structuring introduced
\item
|\childdocof| introduced
\item
|\childdoc| renamed to |\childdocmain|
\item
|\childredirect| renamed to |\childdocforward| and |\childdocforwardprefix|
and functionality expanded
\end{itemize}

%%%%%%%%%%%%%%%%%%%%%%%%%%%%%%%%%%%%%%%%
\paragraph{v1.0:} 2017/04/27

\begin{itemize}
\item
manual and install package
\item
first version published on CTAN
\end{itemize}

%%%%%%%%%%%%%%%%%%%%%%%%%%%%%%%%%%%%%%%%
\paragraph{v0.6:} 2017/04/26

\begin{itemize}
\item
redirection mechanism added
\end{itemize}

%%%%%%%%%%%%%%%%%%%%%%%%%%%%%%%%%%%%%%%%
\paragraph{v0.5:} 2017/04/26

\begin{itemize}
\item
functionality in definition file
\end{itemize}


%%%%%%%%%%%%%%%%%%%%%%%%%%%%%%%%%%%%%%%%%%%%%%%%%%%%%%%%%%%%%%%%%%%%%%%%%%%%%%%%
%%%%%%%%%%%%%%%%%%%%%%%%%%%%%%%%%%%%%%%%%%%%%%%%%%%%%%%%%%%%%%%%%%%%%%%%%%%%%%%%
%%%%%%%%%%%%%%%%%%%%%%%%%%%%%%%%%%%%%%%%%%%%%%%%%%%%%%%%%%%%%%%%%%%%%%%%%%%%%%%%
\appendix

\settowidth\MacroIndent{\rmfamily\scriptsize 000\ }

 \DocInput{childdoc.dtx}

\end{document}
%</driver>
% \fi
%
% %%%%%%%%%%%%%%%%%%%%%%%%%%%%%%%%%%%%%%%%%%%%%%%%%%%%%%%%%%%%%%%%%%%%%%%%%%%%%%
% %%%%%%%%%%%%%%%%%%%%%%%%%%%%%%%%%%%%%%%%%%%%%%%%%%%%%%%%%%%%%%%%%%%%%%%%%%%%%%
% \section{Sample}
%\iffalse
%<*samplemain>
%\fi
%
% The following presents a sample document
% with two chapters, two parts, a title page,
% a compile flag as well as three forwarding files to set the flag.
% It consists of eight |.tex| files:
% \begin{center}
% \begin{tabular}{ll}
% |cdocsamp.tex|&main file\\
% |cdocsch1.tex|&include file for chapter 1\\
% |cdocsch2.tex|&include file for chapter 2\\
% |cdocspt3.tex|&include file for part 3\\
% |cdocspt4.tex|&include file for part 4\\
% |cdocsdrf.tex|&forwarding file for main file in draft mode\\
% |cdocsfi1.tex|&forwarding file for final version of chapter 1\\
% |cdocsfi2.tex|&forwarding file for final version of chapter 2\\
% \end{tabular}
% \end{center}
% Each of the eight files can be compiled directly by the \LaTeX{} compiler.
%
% %%%%%%%%%%%%%%%%%%%%%%%%%%%%%%%%%%%%%%
% \paragraph{Main File.}
%
% The main file is called |cdocsamp.tex|.
%
% Load the \textsf{childdoc} definitions and
% declare the filename for the main document:
%    \begin{macrocode}
\input{childdoc.def}
\childdocmain{}
%    \end{macrocode}

% Optional override for |\version| flag:
%    \begin{macrocode}
%%\ifchilddoc\else\providecommand{\version}{draft}\fi
%    \end{macrocode}

% Define the default values for the |\version| flag
% (|final| for the main file and |draft| for childs):
%    \begin{macrocode}
\ifchilddoc
\providecommand{\version}{draft}
\else
\providecommand{\version}{final}
\fi
%    \end{macrocode}

% Load the standard document class:
%    \begin{macrocode}
\documentclass[12pt]{article}
%    \end{macrocode}

% Start the document body:
%    \begin{macrocode}
\begin{document}
%    \end{macrocode}

% Declare a title page.
% Print title, part of document being processed and version flag:
%    \begin{macrocode}
\addtocounter{page}{-1}
\begin{center}
{\LARGE\bfseries{}childdoc example\par}
\vspace{1cm}
\ifchilddoc
\ifchilddocmanual part\else chapter\fi:
`\childdocname' of `\childdocjob'\par
\else
main document: `\childdocjob'\par
\fi
version: \version\par
\end{center}
\newpage
%    \end{macrocode}

% Manually include selected file,
% otherwise process as usual:
%    \begin{macrocode}
\ifchilddocmanual
\section*{part `\childdocname'}
\input{\childdocname}
\else
%    \end{macrocode}

% Include the two chapters:
%    \begin{macrocode}
\include{cdocsch1}
\include{cdocsch2}
%    \end{macrocode}

% Include the two parts unless only chapters should be displayed:
%    \begin{macrocode}
\ifchilddoc\else
\section{part three}
\input{cdocspt3}
\section{part four}
\input{cdocspt4}
\fi
%    \end{macrocode}

% Process as usual until here:
%    \begin{macrocode}
\fi
%    \end{macrocode}

% End of document body:
%    \begin{macrocode}
\end{document}
%    \end{macrocode}
%\iffalse
%</samplemain>
%\fi
%
% %%%%%%%%%%%%%%%%%%%%%%%%%%%%%%%%%%%%%%
% \paragraph{Chapter Include Files.}
%
% The include files are called |cdocsch1.tex| and |cdocsch2.tex|.
%
%\iffalse
%<*samplechap1|samplechap2>
%\fi

% Optional override for |\version| flag:
%    \begin{macrocode}
%%\providecommand{\version}{final}
%    \end{macrocode}

% Include the main document:
%    \begin{macrocode}
\input{childdoc.def}
\childdocof{cdocsamp}
%    \end{macrocode}

%\iffalse
%</samplechap1|samplechap2>
%\fi
%
%\iffalse
%<*samplechap1>
%\fi
% Some text for chapter 1:
%    \begin{macrocode}
\section{one}
some text in chapter one
%    \end{macrocode}

%\iffalse
%</samplechap1>
%\fi
% Some text for chapter 2:
%\iffalse
%<*samplechap2>
%\fi
%    \begin{macrocode}
\section{two}
more text in chapter two
%    \end{macrocode}

%\iffalse
%</samplechap2>
%\fi
%
% %%%%%%%%%%%%%%%%%%%%%%%%%%%%%%%%%%%%%%
% \paragraph{Part Include Files.}
%
% The include files are called |cdocspt3.tex| and |cdocspt4.tex|.
%
%\iffalse
%<*samplepart3|samplepart4>
%\fi

% Optional override for |\version| flag:
%    \begin{macrocode}
%%\providecommand{\version}{final}
%    \end{macrocode}

% Include the main document:
%    \begin{macrocode}
\input{childdoc.def}
\childdocby{cdocsamp}
%    \end{macrocode}

%\iffalse
%</samplepart3|samplepart4>
%\fi
%
%\iffalse
%<*samplepart3>
%\fi
% Some text for part 3:
%    \begin{macrocode}
some text in part three
%    \end{macrocode}

%\iffalse
%</samplepart3>
%\fi
% Some text for part 4:
%\iffalse
%<*samplepart4>
%\fi
%    \begin{macrocode}
more text in part four
%    \end{macrocode}

%\iffalse
%</samplepart4>
%\fi
%
% %%%%%%%%%%%%%%%%%%%%%%%%%%%%%%%%%%%%%%
% \paragraph{Forwarding for a Complete Draft.}
%
% The following forwarding file |cdocsdrf.tex|
% compiles the main document in draft mode:
%\iffalse
%<*sampledraft>
%\fi
%    \begin{macrocode}
\def\version{draft}
\input{childdoc.def}
\childdocforward{cdocsamp}
%    \end{macrocode}

%\iffalse
%</sampledraft>
%\fi
%
% %%%%%%%%%%%%%%%%%%%%%%%%%%%%%%%%%%%%%%
% \paragraph{Forwarding for Final Version of the Chapters.}
%
% The following forwarding files |cdocsfn1.tex| and |cdocsfn2.tex|
% (with identical content)
% compile the final versions of the child documents
% |cdocsch1.tex| and |cdocsch2.tex|, respectively:
%\iffalse
%<*samplefinal>
%\fi
%    \begin{macrocode}
\def\version{final}
\input{childdoc.def}
\childdocforwardprefix[cdocsamp]{cdocsfn}{cdocsch}
%    \end{macrocode}

%\iffalse
%</samplefinal>
%\fi
%
% %%%%%%%%%%%%%%%%%%%%%%%%%%%%%%%%%%%%%%
% \paragraph{Command Line Processing.}
%
% The following three command lines generate the output files
% |cdocscld|, |cdocscl1| and |cdocscl2|
% which should be identical to
% |cdocsdrf|, |cdocsch1| and |cdocsfn2|, respectively:
% \begin{center}
% \begin{tabular}{l}
% |latex -jobname cdocscld \|\\
% |  "\def\version{draft}\input{childdoc.def}\childdocforward{cdocsamp}"|\\
% |latex -jobname cdocscl1 \|\\
% |  "\input{childdoc.def}\childdocforward[cdocsamp]{cdocsch1}"|\\
% |latex -jobname cdocscl2 \|\\
% |  "\def\version{final}\input{childdoc.def}\childdocforward{cdocsch2}"|
% \end{tabular}
% \end{center}
% Note that the trailing backslash on each first line
% merely continues the input to the second line
% (for convenient cut ant paste).
% Furthermore, the command |latex| can be replaced by any
% of its alternative versions such as |pdflatex|.
%
% %%%%%%%%%%%%%%%%%%%%%%%%%%%%%%%%%%%%%%%%%%%%%%%%%%%%%%%%%%%%%%%%%%%%%%%%%%%%%%
% %%%%%%%%%%%%%%%%%%%%%%%%%%%%%%%%%%%%%%%%%%%%%%%%%%%%%%%%%%%%%%%%%%%%%%%%%%%%%%
% \section{Implementation}
%\iffalse
%<*package>
%\fi
%
% This section describes the definitions file |childdoc.def|.

% The definitions cannot be loaded using |\usepackage| or |\RequirePackage|
% which has a mechanism to prevent loading a style file more than once.
% When loading the definitions by means of |\input|
% multiple instances have to be prevented manually:
%\iffalse
%This code needs to be before the `\ProvidesFile' directive
%which is defined at the beginning of this file.
%Therefore it is also placed there and commented out here.
%</package>
%<*discard>
%\fi
%    \begin{macrocode}
\ifdefined\childdocmain\endinput\fi
%    \end{macrocode}
%\iffalse
%</discard>
%<*package>
%\fi
%
% \macro{\ifchilddoc}
% \macro{\ifchilddocmanual}
% The conditional |\ifchilddoc| tells whether a
% child (true) or main (false) document is being compiled.
% The conditional |\ifchilddocmanual| tells whether
% the |\includeonly| mechanism is used (false) or
% the selection of child files must be performed manually (true).
% The definitions initialise to false:
%    \begin{macrocode}
\newif\ifchilddoc
\newif\ifchilddocmanual
%    \end{macrocode}

% \macro{\childdocname}
% \macro{\childdocjob}
% The macro |\childdocname| stores the name of the main document
% to be compiled. The macro |\childdocjob| stores the name of
% the document on which the \LaTeX{} compiler was originally invoked.
% The content of |\jobname| cannot be compared
% to filenames specified in the source due to different catcodes.
% The following code rescans |\jobname|, stores the result
% in |\childdocname| and saves a copy in |\childdocjob|:
%    \begin{macrocode}
\edef\childdocname{\scantokens\expandafter{\jobname\noexpand}}
\let\childdocjob\childdocname
%    \end{macrocode}

% \macro{\childdocdisable}
% The macro |\childdocdisable| prevents the main file
% from being processed more than once.
% At this stage, the main document command |\childdocmain|
% is assumed to be called once again where it should do nothing.
% Any subsequent call to it should prevent
% a secondary processing of the main document
% It overwrites the forwarding commands
% |\childdocof| and |\childdocforward|
% with empty macros to prevent further inclusions of the main document:
%    \begin{macrocode}
\newcommand{\childdocdisable}
{
  \renewcommand{\childdocmain}[1]{\renewcommand{\childdocmain}[1]{\endinput}}
  \renewcommand{\childdocof}[1]{}
  \renewcommand{\childdocby}[2][]{}
  \renewcommand{\childdocforward}[2][]{}
  \renewcommand{\childdocdisable}{}
}
%    \end{macrocode}

% \macro{\childdocmain}
% The macro |\childdocmain| is to be called at the top of the main file
% with nothing or the main filename (without extension) as argument.
% First, it breaks loops.
% If the argument is not empty and does not match |\childdocname|
% (which is set by the first inclusion of |childdoc.def|),
% |\ifchilddoc| is set to true, |\includeonly| is applied to the child file
% and |\jobname| is set to the main file
% (for proper handling of |.aux| files):
%    \begin{macrocode}
\newcommand{\childdocmain}[1]
{
  \childdocdisable\childdocmain{}
  \if?#1?\else
    \begingroup
      \def\childdoctmp{#1}
      \ifx\childdoctmp\childdocname
        \def\childdoctmp{}
      \else
        \def\childdoctmp
        {
          \childdoctrue
          \includeonly{\childdocname}
          \def\childdocjob{#1}
          \def\jobname{#1}
        }
      \fi
      \expandafter
    \endgroup
    \childdoctmp
  \fi
}
%    \end{macrocode}

% \macro{\childdocof}
% The command |\childdocof| redirects
% compilation to the main file |#1|.
%    \begin{macrocode}
\newcommand{\childdocof}[1]
{
  \childdocdisable
  \childdoctrue
  \includeonly{\childdocname}
  \def\jobname{#1}
  \def\childdocjob{#1}
  \input{#1}
}
%    \end{macrocode}

% \macro{\childdocby}
% The command |\childdocby| ....
%    \begin{macrocode}
\newcommand{\childdocby}[2][]
{
  \childdocdisable
  \childdoctrue
  \childdocmanualtrue
  \if?#1?\else
    \def\jobname{#2}
  \fi
  \def\childdocjob{#2}
  \input{#2}
  \endinput
}
%    \end{macrocode}

% \macro{\childdocforward}
% The command |\childdocforward| redirects
% compilation to the main file or
% (if the optional argument is given) a child file.
% Parameters are set as if the main file
% or a child file starting with |\childdocof| was compiled.
% Then compilation is handed over to the main file:
%    \begin{macrocode}
\newcommand{\childdocforward}[2][]
{
  \begingroup
    \if?#1?
      \def\childdoctmp
      {
        \def\childdocname{#2}
        \def\childdocjob{#2}
        \def\jobname{#2}
        \input{#2}
        \endinput
      }
    \else
      \def\childdoctmp
      {
        \childdocdisable
        \def\childdocname{#2}
        \childdoctrue
        \includeonly{#2}
        \def\childdocjob{#1}
        \def\jobname{#1}
        \input{#1}
        \endinput
      }
    \fi
    \expandafter
  \endgroup
  \childdoctmp
}
%    \end{macrocode}

% \macro{\childdocforwardprefix}
% The command |\childdocforwardprefix| redirects
% compilation to the main or a child file by means of a pattern.
% The prefix |#1| in the current filename is replaced by |#2|
% and the suffix of the current filename is kept
% (it is assumed that the filename does not contain the substring `|~~~|'
% which is used as a delimiter).
% Compilation is handed over to the new file by |\childdocforward|:
%    \begin{macrocode}
\newcommand{\childdocforwardprefix}[3][]
{
  \begingroup
    \def\childdocextract #2##1~~~{\def\childdoctmp{\childdocforward[#1]{#3##1}}}
    \expandafter\childdocextract\childdocname~~~
    \expandafter
  \endgroup
  \childdoctmp
}
%    \end{macrocode}

% \macro{\childdoc}
% The deprecated macro |\childdoc| is a legacy version of |\childdocmain|:
%    \begin{macrocode}
\newcommand{\childdoc}{\childdocmain}
%    \end{macrocode}

% \macro{\childdocredirect}
% The deprecated macro |\childdocredirect| is a legacy version
% of |\childdocforward| and |\childdocforwardprefix|:
%    \begin{macrocode}
\newcommand{\childdocredirect}[2][]
{
  \begingroup
    \if?#1?
      \def\childdoctmp{\childdocforward{#2}}
    \else
      \def\childdoctmp{\childdocforwardprefix{#1}{#2}}
    \fi
    \expandafter
  \endgroup
  \childdoctmp
}
%    \end{macrocode}

%\iffalse
%</package>
%\fi
%
\endinput
|\\
|\childdocforwardprefix[|\textit{main}|]{|\textit{prefix}|}{|\textit{dest}|}|
\end{tabular}
\end{center}
%
the destination file is determined by a pattern
depending on the current file:
To make this work, the current file must be called
`{\textit{prefix}\hspace{0.2em}\textit{suffix}}'
with \textit{prefix} matching precisely the argument.
Processing is then passed on to the file
`{\textit{dest}\hspace{0.2em}\textit{suffix}}'.
Surely, the same effect is achieved by
directly specifying the
argument `{\textit{dest}\hspace{0.2em}\textit{suffix}}'
in the first form.
However, that requires to set up a different file
for each child. With the alternative form of the command
all these files can have exactly the same content
which simplifies setting them up and maintaining them.

For example, the following file |draft.tex|
with a compilation flag |\version| as described in \secref{sec:flags}
compiles the main document as a draft:
%
\begin{center}
\begin{tabular}{l}
|\def\version{draft}|\\
|% \iffalse
%
% childdoc.dtx Copyright (C) 2017-2018 Niklas Beisert
%
% This work may be distributed and/or modified under the
% conditions of the LaTeX Project Public License, either version 1.3
% of this license or (at your option) any later version.
% The latest version of this license is in
%   http://www.latex-project.org/lppl.txt
% and version 1.3 or later is part of all distributions of LaTeX
% version 2005/12/01 or later.
%
% This work has the LPPL maintenance status `maintained'.
%
% The Current Maintainer of this work is Niklas Beisert.
%
% This work consists of the files childdoc.dtx and childdoc.ins
% and the derived files childdoc.def and cdocsamp.tex with
% cdocsch1.tex, cdocsch2.tex, cdocsdrf.tex, cdocsfn1.tex, cdocsfn2.tex.
%
%<package>\ifdefined\childdocmain\endinput\fi
%<package>\ProvidesFile{childdoc.def}[2018/12/30 v2.0 child document driver]
%<samplemain>\ProvidesFile{cdocsamp.tex}[2018/12/30 v2.0 sample for childdoc]
%<*driver>
%\ProvidesFile{childdoc.drv}[2018/12/30 v2.0 childdoc reference manual file]
\PassOptionsToClass{10pt,a4paper}{article}
\documentclass{ltxdoc}

\usepackage[margin=35mm]{geometry}
\usepackage{hyperref}
\usepackage{hyperxmp}
\usepackage[usenames]{color}

\hypersetup{colorlinks=true}
\hypersetup{pdfstartview=FitH}
\hypersetup{pdfpagemode=UseNone}
\hypersetup{pdfsource={}}
\hypersetup{pdflang={en-UK}}
\hypersetup{pdfcopyright={Copyright 2017-2018 Niklas Beisert.
  This work may be distributed and/or modified under the
  conditions of the LaTeX Project Public License, either version 1.3
  of this license or (at your option) any later version.}}
\hypersetup{pdflicenseurl={http://www.latex-project.org/lppl.txt}}
\hypersetup{pdfcontactaddress={ETH Zurich, ITP, HIT K,
  Wolfgang-Pauli-Strasse 27}}
\hypersetup{pdfcontactpostcode={8093}}
\hypersetup{pdfcontactcity={Zurich}}
\hypersetup{pdfcontactcountry={Switzerland}}
\hypersetup{pdfcontactemail={nbeisert@itp.phys.ethz.ch}}
\hypersetup{pdfcontacturl={http://people.phys.ethz.ch/\xmptilde nbeisert/}}

\newcommand{\secref}[1]{\hyperref[#1]{section \ref*{#1}}}

\parskip1ex
\parindent0pt
\let\olditemize\itemize
\def\itemize{\olditemize\parskip0pt}

\begin{document}

\title{The \textsf{childdoc} Package}
\hypersetup{pdftitle={The childdoc Package}}
\author{Niklas Beisert\\[2ex]
  Institut f\"ur Theoretische Physik\\
  Eidgen\"ossische Technische Hochschule Z\"urich\\
  Wolfgang-Pauli-Strasse 27, 8093 Z\"urich, Switzerland\\[1ex]
  \href{mailto:nbeisert@itp.phys.ethz.ch}
  {\texttt{nbeisert@itp.phys.ethz.ch}}}
\hypersetup{pdfauthor={Niklas Beisert}}
\hypersetup{pdfsubject={Manual for the LaTeX2e Package childdoc}}
\date{30 December 2018, \textsf{v2.0}}
\maketitle

\begin{abstract}\noindent
\textsf{childdoc} is a \LaTeXe{} package
that enables the direct compilation
of document sections included by |\include|
to individual files.
\end{abstract}

\begingroup
\parskip0ex
\tableofcontents
\endgroup

%%%%%%%%%%%%%%%%%%%%%%%%%%%%%%%%%%%%%%%%%%%%%%%%%%%%%%%%%%%%%%%%%%%%%%%%%%%%%%%%
%%%%%%%%%%%%%%%%%%%%%%%%%%%%%%%%%%%%%%%%%%%%%%%%%%%%%%%%%%%%%%%%%%%%%%%%%%%%%%%%
\section{Introduction}

\LaTeX{} provides a mechanism to structure a large document (such as a book)
into a main file and several child files (containing the chapters)
using the |\include| command.
This mechanism is beneficial for documents
which span hundreds of pages in order to
make the source file(s) more manageable.
Moreover, compilation can be restricted to
selected child files by means of the |\includeonly| command.
The latter feature can be used to reduce the compilation time while editing
(this was significantly more useful in the earlier days of \LaTeX{})
or to generate a smaller document which is easier to navigate.
Another application of |\includeonly| is to generate
documents consisting of selected parts of the complete document.

However, there are a few drawbacks of the plain |\include| mechanism:
\begin{itemize}
\item
The child files cannot be compiled on their own,
they can only be compiled via the main file.
A naive editing environment
(such as a text editor with an option
to have the current file processed by \LaTeX)
may require one to switch to the main file before compiling;
attempting to compile the child file produces errors.
\item
The main file must be modified (each time)
to adjust the |\includeonly| command
to the present needs. This easily leaves the main file in a messy state.
\item
The generated document will always carry the filename
of the main document. This is inconvenient if
several child files are to be compiled and
to be kept for distribution.
\end{itemize}

The present package provides a simple interface
to make child files individually compilable by \LaTeX{}.
Compiling a child file then has the same effect as compiling
the main file with an |\includeonly| command
to select the appropriate child.
Moreover the generated document will carry the name of the child
rather than the main file.
This resolves all three above issues.

This feature is meant to make the editing of books,
thesis documents and lecture notes somewhat more convenient.
However, the package can also be used efficiently for
composing a series of documents (such as exercise sheets)
which are typically distributed individually.
It then assists the author in generating the individual documents
(potentially in different versions)
as well as a document containing the collected series.
Another application is in developing style files
or other kinds of included material
where compilation of the style file could redirect
to a sample or test file.

%%%%%%%%%%%%%%%%%%%%%%%%%%%%%%%%%%%%%%%%%%%%%%%%%%%%%%%%%%%%%%%%%%%%%%%%%%%%%%%%
%%%%%%%%%%%%%%%%%%%%%%%%%%%%%%%%%%%%%%%%%%%%%%%%%%%%%%%%%%%%%%%%%%%%%%%%%%%%%%%%
\section{Usage}

First of all, the package \textsf{childdoc} is \emph{not} a standard
\LaTeXe{} |.sty| style file! Therefore it needs to be invoked in
a non-standard way.

%%%%%%%%%%%%%%%%%%%%%%%%%%%%%%%%%%%%%%%%%%%%%%%%%%%%%%%%%%%%%%%%%%%%%%%%%%%%%%%%
\subsection{Included Files}
\label{sec:include}

%%%%%%%%%%%%%%%%%%%%%%%%%%%%%%%%%%%%%%%%
\DescribeMacro{\childdocmain}
To use the package, add the commands
\begin{center}
\begin{tabular}{l}
|\input{childdoc.def}|\\
|\childdocmain{}|\\
\end{tabular}
\end{center}
at the very top of the main \LaTeX{} file,
in particular \emph{before} the |\documentclass| statement!
The argument of |\childdocmain| should be left empty
(but it must be present).

%%%%%%%%%%%%%%%%%%%%%%%%%%%%%%%%%%%%%%%%
\DescribeMacro{\childdocof}
Furthermore, add the commands
\begin{center}
\begin{tabular}{l}
|\input{childdoc.def}|\\
|\childdocof{|\textit{main}|}|\\
\end{tabular}
\end{center}
at the top of every child file \textit{child}
which is included by |\include{|\textit{child}|}|
from within the main file
(or at least for those files to be compiled individually).
The argument \textit{main} must be the filename of the main file.

There are a couple of
considerations in setting up the main and child documents:

%%%%%%%%%%%%%%%%%%%%%%%%%%%%%%%%%%%%%%%%
\paragraph{Restrictions.}

Please note the following restrictions:
\begin{itemize}
\item
|\childdocmain| must be called with one argument \textit{main}
to ensure compatibility with earlier version of the package.
It must either be empty (|\childdocmain{}|)
or precisely match the filename of the main file in which it is specified.
See \secref{sec:detection} for further information.
\item
The filename \textit{main} must be specified without the |.tex| extension.
\item
The filename \textit{main} is case sensitive
(even in case-insensitive file systems)
due to internal string comparison.
\item
The argument \textit{main} should be fully expanded, it cannot be a macro.
\item
Subdirectories and special characters should be avoided in filenames.
\item
The command |\childdocmain{|\textit{main}|}| must be followed by a whitespace.
It should not be followed immediately by another command
or by a comment mark `|%|'.
This is because the \TeX{} parser reads the token immediately following
the argument of |\childdocmain| and puts it
at the beginning of every child section;
however, a white\-space is ignored.
\end{itemize}

%%%%%%%%%%%%%%%%%%%%%%%%%%%%%%%%%%%%%%%%
\paragraph{Content of Main File.}

It is advisable to place all content in the child files included by |\include|.
Any output contained in the main file will appear in all child documents
unless suppressed manually;
it cannot be suppressed automatically by the |\includeonly| directive
and thus should normally be avoided.
A method to include some content in the main file
by means of conditional processing is described in \secref{sec:conditional}.

%%%%%%%%%%%%%%%%%%%%%%%%%%%%%%%%%%%%%%%%
\paragraph{Page Numbering.}

When only a part of the document is compiled,
the appropriate numbering of pages
(as well as other status parameters)
is determined from the |.aux| files.
The latter contain information from previous passes.
However this information needs to propagate through
all intermediate child documents.
Therefore the page numbering in child documents may well
be inconsistent until the complete document is compiled at least once.

A useful (if unconventional) way to always ensure a consistent
page numbering is to restart the numbering in each child document
and denote the pages by `\textit{child}|.|\textit{page}'
where \textit{child} represents the chapter/section number of the child file.
This can be achieved by the command
|\numberwithin{page}{|\textit{child}|}|
of the \textsf{amsmath} package
where \textit{child} can be |chapter| or |section|
depending on the chosen structuring.
Alternatively, one can modify the macro |\thepage| appropriately
and reset the counter |page| at the start of each child file.

%%%%%%%%%%%%%%%%%%%%%%%%%%%%%%%%%%%%%%%%%%%%%%%%%%%%%%%%%%%%%%%%%%%%%%%%%%%%%%%%
\subsection{Conditional Processing}
\label{sec:conditional}

The package provides a mechanism to compile different versions
of a document. To customise the versions further some conditional processing
can come in handy to distinguish which version is being compiled.
The package provides two macros to describe the compilation context:

%%%%%%%%%%%%%%%%%%%%%%%%%%%%%%%%%%%%%%%%
\DescribeMacro{\ifchilddoc}
The conditional |\ifchilddoc| distinguishes between the compilation of
child documents and the main document:
%
\begin{center}
|\ifchilddoc |\textit{child-code}| |[|\||else |\textit{main-code}]| \||fi|
\end{center}

%%%%%%%%%%%%%%%%%%%%%%%%%%%%%%%%%%%%%%%%
\DescribeMacro{\childdocname}
\DescribeMacro{\childdocjob}
The macro |\childdocname| contains the filename (without extension)
of the main or child file being processed.
Note that |\childdocjob| will always contain the name of the main file.

%%%%%%%%%%%%%%%%%%%%%%%%%%%%%%%%%%%%%%%%
\paragraph{Title Page.}

Conditional processing can be used to include a title or banner page
in the main document when proper precautions are taken.
Importantly, the code in the main file should ensure that the page counter
(as well as other status parameters which are stored in the |.aux| files)
takes the same value after the conditional processing.
Otherwise the page numbers may take divergent values
depending on which part is compiled.

For example, a title page could be declared by:
%
\begin{center}
\begin{tabular}{l}
|\ifchilddoc\||else|\\
|\addtocounter{page}{-1}|\\
\textit{code for title page}\\
|\newpage|\\
|\||fi|
\end{tabular}
\end{center}
%
A banner page for the child documents can be generated by:
%
\begin{center}
\begin{tabular}{l}
|\ifchilddoc|\\
|\addtocounter{page}{-1}|\\
\textit{code for banner page}\\
|\newpage|\\
|\||fi|
\end{tabular}
\end{center}
%
Here one could write a message such as:
\begin{center}
|This is the part \childdocname{} of \childdocjob{}.|
\end{center}

%%%%%%%%%%%%%%%%%%%%%%%%%%%%%%%%%%%%%%%%%%%%%%%%%%%%%%%%%%%%%%%%%%%%%%%%%%%%%%%%
\subsection{Flags}
\label{sec:flags}

The package makes it easy to generate different versions
of the main or child documents.
To this end compilation flags can be defined
and assigned different default values.
They will be particularly useful in conjunction
with the forwarding mechanism described in \secref{sec:forward}.

For example, it may be useful to have a flag |\version|
which can be set to |draft| or |final|.
The document source will contain some conditional code
depending on the value of |\version|.
Suppose further, the flag should default to |final| for the main file
and to |draft| for child files
which is a natural assignment for editing the document.
This is achieved by placing the following code
in the preamble of the main document
(below the |\childdocmain| directive):
%
\begin{center}
\begin{tabular}{l}
|\ifchilddoc|\\
|\providecommand{\version}{draft}|\\
|\||else|\\
|\providecommand{\version}{final}|\\
|\||fi|
\end{tabular}
\end{center}
%
The definition by |\providecommand| makes sure
that previous definitions are not overwritten.
Further statements |\providecommand{\version}{...}|
can thus be added before the above code to override it.

For the main file, one might add a line
(between |\childdocmain| and the above block)
%
\begin{center}
|%\ifchilddoc\||else\providecommand{\version}{draft}\||fi|
\end{center}
%
which can be uncommented to produce a draft version.
Likewise one can add a line to the very top of a child file
(above the |\childdocof{|\textit{main}|}| directive)
%
\begin{center}
|%\providecommand{\version}{final}|
\end{center}
%
which can be uncommented to produce the final version of this child document.

%%%%%%%%%%%%%%%%%%%%%%%%%%%%%%%%%%%%%%%%%%%%%%%%%%%%%%%%%%%%%%%%%%%%%%%%%%%%%%%%
\subsection{Forwarding}
\label{sec:forward}

Different versions of the main or child documents
using compilation flags as described in \secref{sec:flags}
can be (permanently) stored in different files
for convenient compilation, viewing and distribution.
To this end, the package defines a command
to pass on compilation to a different file:

%%%%%%%%%%%%%%%%%%%%%%%%%%%%%%%%%%%%%%%%
\DescribeMacro{\childdocforward}
The command |\childdocforward| redirects processing to
another source file:
%
\begin{center}
\begin{tabular}{l}
|\input{childdoc.def}|\\
|\childdocforward[|\textit{main}|]{|\textit{dest}|}|\\
\end{tabular}
\end{center}
%
The argument \textit{dest} is the destination file
(without extension).
It should be the main file or one of the child files.
Note that further \textsf{childdoc} directives
such as |\childdocof| and |\childdocforward|
in the indicated file will be processed in this form.
The optional argument \textit{main}
passes on directly to the main file \textit{main}
while pretending to compile the child \textit{dest}.
This form behaves as if \textit{dest}
issues |\childdocof{|\textit{main}|}| right away,
and no further \textsf{childdoc} directives will be processed.

%%%%%%%%%%%%%%%%%%%%%%%%%%%%%%%%%%%%%%%%
\DescribeMacro{\...prefix}
In the alternative form |\childdocforwardprefix|,
%
\begin{center}
\begin{tabular}{l}
|\input{childdoc.def}|\\
|\childdocforwardprefix[|\textit{main}|]{|\textit{prefix}|}{|\textit{dest}|}|
\end{tabular}
\end{center}
%
the destination file is determined by a pattern
depending on the current file:
To make this work, the current file must be called
`{\textit{prefix}\hspace{0.2em}\textit{suffix}}'
with \textit{prefix} matching precisely the argument.
Processing is then passed on to the file
`{\textit{dest}\hspace{0.2em}\textit{suffix}}'.
Surely, the same effect is achieved by
directly specifying the
argument `{\textit{dest}\hspace{0.2em}\textit{suffix}}'
in the first form.
However, that requires to set up a different file
for each child. With the alternative form of the command
all these files can have exactly the same content
which simplifies setting them up and maintaining them.

For example, the following file |draft.tex|
with a compilation flag |\version| as described in \secref{sec:flags}
compiles the main document as a draft:
%
\begin{center}
\begin{tabular}{l}
|\def\version{draft}|\\
|\input{childdoc.def}|\\
|\childdocforward{|\textit{main}|}|
\end{tabular}
\end{center}
%
Likewise, the following files |final|\textit{nn}|.tex|
compile the final version of the child document
|child|\textit{nn}|.tex|:
%
\begin{center}
\begin{tabular}{l}
|\def\version{final}|\\
|\input{childdoc.def}|\\
|\childdocforwardprefix{final}{child}|
\end{tabular}
\end{center}
%

Note that when several versions of a main file and/or of each child file
are to be generated, it may be convenient to set up a |Makefile| or
shell script to automatise the process.

%%%%%%%%%%%%%%%%%%%%%%%%%%%%%%%%%%%%%%%%%%%%%%%%%%%%%%%%%%%%%%%%%%%%%%%%%%%%%%%%
\subsection{Command Line Processing}
\label{sec:commandline}

The effect of redirection files can also be achieved by invoking
the \LaTeX{} compiler with a more elaborate command line.
Most conveniently this should be done as part
of a shell script or a |Makefile|.

When using \textsf{childdoc} in the main file, the following
command lines effectively perform a redirection
(note that depending on the shell being used,
backslashes may have to be doubled: `|\|' $\to$ `|\\|'):
%
\begin{center}
|... -jobname "|\textit{target}|" |\\|"|[\textit{flags}]%
|\input{childdoc.def}\childdocforward[|\textit{main}|]{|\textit{dest}|}"|
\end{center}
%
Here \textit{target} is the name of the output file,
\textit{main} is the name of the main file
and \textit{dest} is the name of the main or child file to be processed
(all filenames without extensions).
The optional argument \textit{main} can be omitted
if \textit{main} matches \textit{dest}.
Optionally, compilation \textit{flags} can be defined via |\def| commands.
This command line makes the \TeX{} engine believe
it is compiling the file \textit{target}
whose content is specified as the latter parameter.
The provided code then forwards the processing to
\textit{main} or \textit{dest} as described in \secref{sec:forward}.

%%%%%%%%%%%%%%%%%%%%%%%%%%%%%%%%%%%%%%%%%%%%%%%%%%%%%%%%%%%%%%%%%%%%%%%%%%%%%%%%
\subsection{Include by Input}
\label{sec:input}

Including child documents by |\include| has some restrictions by design.
Most notably, the content of a child document always occupies
its own set of pages; pages cannot be shared between child documents.
Usually, this behaviour makes perfect sense
because each child document contain an essential part of the document.
However, in some situations it may be desirable to compose
a document from a collection of parts
without having mandatory page breaks between then.
For this case, the package
provides a mechanism to include parts
by |\input| which can also be processed individually.
However, by construction this mechanism
requires manual handling of the content to be output.

%%%%%%%%%%%%%%%%%%%%%%%%%%%%%%%%%%%%%%%%
\DescribeMacro{\ifchilddocmanual}
The main file should be prepared as usual, see \secref{sec:include}.
However, the document body must make a distinction
between processing of an individual part and of the main document, e.g.:
%
\begin{center}
\begin{tabular}{l}
|\ifchilddocmanual|\\
|\input{\childdocname}|\\
|\||else|\\
\textit{document body with }|\input{|\textit{part}|}|\\
|\||fi|
\end{tabular}
\end{center}
%
The conditional |\ifchilddocmanual| is true whenever
a part to be included by |\input| is being compiled,
and the name of the part is stored in |\childdocname|.

%%%%%%%%%%%%%%%%%%%%%%%%%%%%%%%%%%%%%%%%
\DescribeMacro{\childdocby}
Each part to be included by |\input| should start with:
%
\begin{center}
\begin{tabular}{l}
|\input{childdoc.def}|\\
|\childdocby{|\textit{main}|}|\\
\end{tabular}
\end{center}
%
The directive |\childdocby| is similar to |\childdocof|
described in \secref{sec:include},
but the subsequent selection of content must be done manually.
To that end, both |\ifchilddoc| and |\ifchilddocmanual|
will be true upon processing of a part,
and the name of the part is stored in |\childdocname|.
Note that |\jobname| will be set to the filename of the current part
so that each part receives an individual |.aux| file
that does not interfere with the |.aux| file(s) of the main document.
This behaviour can be altered by the alternative form
|\childdocby[*]{|\textit{main}|}| (with a non-empty optional argument)
which uses the |.aux| file of the main document
by setting |\jobname| to \textit{main}.

%%%%%%%%%%%%%%%%%%%%%%%%%%%%%%%%%%%%%%%%%%%%%%%%%%%%%%%%%%%%%%%%%%%%%%%%%%%%%%%%
\subsection{Driver Development}
\label{sec:driver}

The \textsf{childdoc} mechanism can also be use for the development
of definition files such as \LaTeX{} styles or classes.
This case differs from the above setup with multiple parts
included by |\include| in that no |\includeonly| should be invoked.
This can be achieved by starting the include file
(before |\ProvidesPackage|) with:
%
\begin{center}
\begin{tabular}{l}
|\input{childdoc.def}|\\
|\childdocforward{|\textit{main}|}|\\
\end{tabular}
\end{center}
%
or alternatively with:
%
\begin{center}
\begin{tabular}{l}
|\input{childdoc.def}|\\
|\childdocby{|\textit{main}|}|\\
\end{tabular}
\end{center}
%
Both forms have slightly different effects as described above.
The main file is prepared as usual, see \secref{sec:include}.

%%%%%%%%%%%%%%%%%%%%%%%%%%%%%%%%%%%%%%%%%%%%%%%%%%%%%%%%%%%%%%%%%%%%%%%%%%%%%%%%
\subsection{Legacy Detection}
\label{sec:detection}

The directive |\childdocmain| in the main file can detect
whether the complete document or merely a child is to be compiled
even without using the directive |\childdocof|.
This method is deprecated because it is less robust
and there is no compelling reason to use it;
it is merely provided for backward compatibility
and it may be removed in future versions.

If the detection mechanism is to be used,
it is mandatory to correctly specify
the filename of the main file as the argument of |\childdocmain|:
%
\begin{center}
\begin{tabular}{l}
|\input{childdoc.def}|\\
|\childdocmain{|\textit{main}|}|\\
\end{tabular}
\end{center}
%
If |\jobname| does not match the argument \textit{main} of |\childdocmain|,
it is assumed that |\jobname| points to the child file to be compiled.
When using |\childdocmain| with the main file specified as argument,
it suffices to start a child file
with just |\input{|\textit{main}|}|
without loading of the package and using |\childdocof|.
If instead all processing is done
with the appropriate \textsf{childdoc} directives,
the argument of \textit{main} of |\childdocmain| can be empty.

An alternative version of the command line processing described
in \secref{sec:commandline} using the detection mechanism reads:
%
\begin{center}
|... -jobname "|\textit{target}|" "|[\textit{flags}]%
[|\def\jobname{|\textit{dest}|}|]|\input{|\textit{main}|}"|
\end{center}

%%%%%%%%%%%%%%%%%%%%%%%%%%%%%%%%%%%%%%%%%%%%%%%%%%%%%%%%%%%%%%%%%%%%%%%%%%%%%%%%
\subsection{Manual Code}
\label{sec:manual}

In case one cannot be certain whether the definitions file |childdoc.def|
is installed on the target \TeX{} distribution
and one prefers not to ship it,
it is conceivable to paste a few relevant commands into the sources.

To that end, drop all statements |\input{childdoc.def}|
and perform the replacements as outlined below.
Instead of |\childdocmain{|\textit{main}|}| add the following code
to the top of the main file:
%
\begin{center}
\begin{tabular}{l}
|\||ifdefined\childdocname\endinput\||fi\newif\ifchilddoc|\\
|\edef\childdocname{\scantokens\expandafter{\jobname\noexpand}}|\\
|\def\childdocmain{|\textit{main}|}\||ifx\childdocmain\childdocname\||else|\\
|\childdoctrue\includeonly{\childdocname}\let\jobname\childdocmain\||fi|\\
\end{tabular}
\end{center}
%
Instead of |\childdocof{|\textit{main}|}| just include the main file
at the top of each child file:
%
\begin{center}
|\input{|\textit{main}|}|
\end{center}
%
A simple redirection |\childdocforward{|\textit{dest}|}| is achieved by:
%
\begin{center}
|\def\jobname{|\textit{dest}|}\input{\jobname}|
\end{center}
%
The redirection with prefix
|\childdocforwardprefix[|\textit{prefix}|]{|\textit{dest}|}|
is accomplished by:
%
\begin{center}
\begin{tabular}{l}
|{\edef\jobname{\scantokens\expandafter{\jobname\noexpand}}|\\
|\def\redirectjob |\textit{prefix}|#1~~~{\gdef\jobname{|\textit{dest}|#1}}|\\
|\expandafter\redirectjob\jobname~~~}\input{\jobname}|
\end{tabular}
\end{center}

In an alternative approach,
child documents can be compiled by a specific command line
without additional code or specific definitions:
%
\begin{center}
|... -jobname "|\textit{target}|" "|[\textit{flags}]%
|\includeonly{|\textit{dest}|}\input{|\textit{main}|}"|
\end{center}
%

%%%%%%%%%%%%%%%%%%%%%%%%%%%%%%%%%%%%%%%%%%%%%%%%%%%%%%%%%%%%%%%%%%%%%%%%%%%%%%%%
%%%%%%%%%%%%%%%%%%%%%%%%%%%%%%%%%%%%%%%%%%%%%%%%%%%%%%%%%%%%%%%%%%%%%%%%%%%%%%%%
\section{Information}

%%%%%%%%%%%%%%%%%%%%%%%%%%%%%%%%%%%%%%%%%%%%%%%%%%%%%%%%%%%%%%%%%%%%%%%%%%%%%%%%
\subsection{Copyright}

Copyright \copyright{} 2017--2018 Niklas Beisert

This work may be distributed and/or modified under the
conditions of the \LaTeX{} Project Public License, either version 1.3
of this license or (at your option) any later version.
The latest version of this license is in
  \url{http://www.latex-project.org/lppl.txt}
and version 1.3 or later is part of all distributions of \LaTeX{}
version 2005/12/01 or later.

This work has the LPPL maintenance status `maintained'.

The Current Maintainer of this work is Niklas Beisert.

This work consists of the files |README.txt|, |childdoc.ins| and |childdoc.dtx|
as well as the derived files |childdoc.def|, |cdocsamp.tex|
with |cdocsch1.tex|, |cdocsch2.tex|, |cdocspt3.tex|, |cdocspt4.tex|,
|cdocsdrf.tex|, |cdocsfn1.tex|, |cdocsfn2.tex|
as well as |childdoc.pdf|.

%%%%%%%%%%%%%%%%%%%%%%%%%%%%%%%%%%%%%%%%%%%%%%%%%%%%%%%%%%%%%%%%%%%%%%%%%%%%%%%%
\subsection{Files and Installation}

The package consists of the files:
%
\begin{center}
\begin{tabular}{ll}
    |README.txt|   & readme file \\
    |childdoc.ins| & installation file \\
    |childdoc.dtx| & source file \\
    |childdoc.def| & definition file \\
    |cdocsamp.tex| & sample main file \\
    |cdocsch1.tex| & sample include file \\
    |cdocsch2.tex| & sample include file \\
    |cdocspt3.tex| & sample part file \\
    |cdocspt4.tex| & sample part file \\
    |cdocsdrf.tex| & sample redirection file \\
    |cdocsfn1.tex| & sample redirection file \\
    |cdocsfn2.tex| & sample redirection file \\
    |childdoc.pdf| & manual
\end{tabular}
\end{center}
%
The distribution consists of the files
|README.txt|, |childdoc.ins| and |childdoc.dtx|.
%
\begin{itemize}
\item
Run (pdf)\LaTeX{} on |childdoc.dtx|
to compile the manual |childdoc.pdf| (this file).
\item
Run \LaTeX{} on |childdoc.ins| to create the definitions file |childdoc.def|
and the sample |cdocsamp.tex| with include files
|cdocsch1.tex|, |cdocsch2.tex|, |cdocspt3.tex|, |cdocspt4.tex|,
|cdocsdrf.tex|, |cdocsfn1.tex|, |cdocsfn2.tex|.
Then copy the file |childdoc.def| to an appropriate directory of your \LaTeX{}
distribution, e.g.\ \textit{texmf-root}|/tex/latex/childdoc|.
\end{itemize}

%%%%%%%%%%%%%%%%%%%%%%%%%%%%%%%%%%%%%%%%%%%%%%%%%%%%%%%%%%%%%%%%%%%%%%%%%%%%%%%%
\subsection{Related CTAN Packages}

There are several other packages which offer a similar functionality:
%
\begin{itemize}
\item
The packages
\href{http://ctan.org/pkg/docmute}{\textsf{docmute}},
\href{http://ctan.org/pkg/includex}{\textsf{includex}} and
\href{http://ctan.org/pkg/standalone}{\textsf{standalone}}
provide commands to include only the document body of
a child file thus allowing both files to be compiled individually.
\item
The packages \href{http://ctan.org/pkg/subdocs}{\textsf{subdocs}}
and \href{http://ctan.org/pkg/subfiles}{\textsf{subfiles}}
provide structures in which the main and child documents can be
encapsulated and allowing them to be compiled individually.
The inclusion mechanism is different from the conventional |\include|.
\item
The package \href{http://ctan.org/pkg/combine}{\textsf{combine}}
is an elaborate solution to combine several documents into one.
\end{itemize}
%
See also the CTAN topic \href{http://ctan.org/topic/subdocs}{\textsf{subdocs}}
for further related packages.
The present package differs from the above solutions in that
a document structure constructed with the conventional |\include| mechanism
just needs two extra commands at the top of every file
such that all constituent files can be compiled individually.

%%%%%%%%%%%%%%%%%%%%%%%%%%%%%%%%%%%%%%%%%%%%%%%%%%%%%%%%%%%%%%%%%%%%%%%%%%%%%%%%
%\subsection{Feature Suggestions}
%
%The following is a list of features which may be useful for future
%versions of this package:
%%
%\begin{itemize}
%\item
%\ldots
%\end{itemize}

%%%%%%%%%%%%%%%%%%%%%%%%%%%%%%%%%%%%%%%%%%%%%%%%%%%%%%%%%%%%%%%%%%%%%%%%%%%%%%%%
\subsection{Revision History}

%%%%%%%%%%%%%%%%%%%%%%%%%%%%%%%%%%%%%%%%
\paragraph{v2.0:} 2018/12/30

\begin{itemize}
\item
immediate forward processing
\item
added |\childdocby| mechanism
\item
manual restructured
\end{itemize}

%%%%%%%%%%%%%%%%%%%%%%%%%%%%%%%%%%%%%%%%
\paragraph{v1.6:} 2018/01/17

\begin{itemize}
\item
application for development of include files
\item
corrections to manual
\end{itemize}

%%%%%%%%%%%%%%%%%%%%%%%%%%%%%%%%%%%%%%%%
\paragraph{v1.5:} 2017/05/21

\begin{itemize}
\item
more complete structuring introduced
\item
|\childdocof| introduced
\item
|\childdoc| renamed to |\childdocmain|
\item
|\childredirect| renamed to |\childdocforward| and |\childdocforwardprefix|
and functionality expanded
\end{itemize}

%%%%%%%%%%%%%%%%%%%%%%%%%%%%%%%%%%%%%%%%
\paragraph{v1.0:} 2017/04/27

\begin{itemize}
\item
manual and install package
\item
first version published on CTAN
\end{itemize}

%%%%%%%%%%%%%%%%%%%%%%%%%%%%%%%%%%%%%%%%
\paragraph{v0.6:} 2017/04/26

\begin{itemize}
\item
redirection mechanism added
\end{itemize}

%%%%%%%%%%%%%%%%%%%%%%%%%%%%%%%%%%%%%%%%
\paragraph{v0.5:} 2017/04/26

\begin{itemize}
\item
functionality in definition file
\end{itemize}


%%%%%%%%%%%%%%%%%%%%%%%%%%%%%%%%%%%%%%%%%%%%%%%%%%%%%%%%%%%%%%%%%%%%%%%%%%%%%%%%
%%%%%%%%%%%%%%%%%%%%%%%%%%%%%%%%%%%%%%%%%%%%%%%%%%%%%%%%%%%%%%%%%%%%%%%%%%%%%%%%
%%%%%%%%%%%%%%%%%%%%%%%%%%%%%%%%%%%%%%%%%%%%%%%%%%%%%%%%%%%%%%%%%%%%%%%%%%%%%%%%
\appendix

\settowidth\MacroIndent{\rmfamily\scriptsize 000\ }

 \DocInput{childdoc.dtx}

\end{document}
%</driver>
% \fi
%
% %%%%%%%%%%%%%%%%%%%%%%%%%%%%%%%%%%%%%%%%%%%%%%%%%%%%%%%%%%%%%%%%%%%%%%%%%%%%%%
% %%%%%%%%%%%%%%%%%%%%%%%%%%%%%%%%%%%%%%%%%%%%%%%%%%%%%%%%%%%%%%%%%%%%%%%%%%%%%%
% \section{Sample}
%\iffalse
%<*samplemain>
%\fi
%
% The following presents a sample document
% with two chapters, two parts, a title page,
% a compile flag as well as three forwarding files to set the flag.
% It consists of eight |.tex| files:
% \begin{center}
% \begin{tabular}{ll}
% |cdocsamp.tex|&main file\\
% |cdocsch1.tex|&include file for chapter 1\\
% |cdocsch2.tex|&include file for chapter 2\\
% |cdocspt3.tex|&include file for part 3\\
% |cdocspt4.tex|&include file for part 4\\
% |cdocsdrf.tex|&forwarding file for main file in draft mode\\
% |cdocsfi1.tex|&forwarding file for final version of chapter 1\\
% |cdocsfi2.tex|&forwarding file for final version of chapter 2\\
% \end{tabular}
% \end{center}
% Each of the eight files can be compiled directly by the \LaTeX{} compiler.
%
% %%%%%%%%%%%%%%%%%%%%%%%%%%%%%%%%%%%%%%
% \paragraph{Main File.}
%
% The main file is called |cdocsamp.tex|.
%
% Load the \textsf{childdoc} definitions and
% declare the filename for the main document:
%    \begin{macrocode}
\input{childdoc.def}
\childdocmain{}
%    \end{macrocode}

% Optional override for |\version| flag:
%    \begin{macrocode}
%%\ifchilddoc\else\providecommand{\version}{draft}\fi
%    \end{macrocode}

% Define the default values for the |\version| flag
% (|final| for the main file and |draft| for childs):
%    \begin{macrocode}
\ifchilddoc
\providecommand{\version}{draft}
\else
\providecommand{\version}{final}
\fi
%    \end{macrocode}

% Load the standard document class:
%    \begin{macrocode}
\documentclass[12pt]{article}
%    \end{macrocode}

% Start the document body:
%    \begin{macrocode}
\begin{document}
%    \end{macrocode}

% Declare a title page.
% Print title, part of document being processed and version flag:
%    \begin{macrocode}
\addtocounter{page}{-1}
\begin{center}
{\LARGE\bfseries{}childdoc example\par}
\vspace{1cm}
\ifchilddoc
\ifchilddocmanual part\else chapter\fi:
`\childdocname' of `\childdocjob'\par
\else
main document: `\childdocjob'\par
\fi
version: \version\par
\end{center}
\newpage
%    \end{macrocode}

% Manually include selected file,
% otherwise process as usual:
%    \begin{macrocode}
\ifchilddocmanual
\section*{part `\childdocname'}
\input{\childdocname}
\else
%    \end{macrocode}

% Include the two chapters:
%    \begin{macrocode}
\include{cdocsch1}
\include{cdocsch2}
%    \end{macrocode}

% Include the two parts unless only chapters should be displayed:
%    \begin{macrocode}
\ifchilddoc\else
\section{part three}
\input{cdocspt3}
\section{part four}
\input{cdocspt4}
\fi
%    \end{macrocode}

% Process as usual until here:
%    \begin{macrocode}
\fi
%    \end{macrocode}

% End of document body:
%    \begin{macrocode}
\end{document}
%    \end{macrocode}
%\iffalse
%</samplemain>
%\fi
%
% %%%%%%%%%%%%%%%%%%%%%%%%%%%%%%%%%%%%%%
% \paragraph{Chapter Include Files.}
%
% The include files are called |cdocsch1.tex| and |cdocsch2.tex|.
%
%\iffalse
%<*samplechap1|samplechap2>
%\fi

% Optional override for |\version| flag:
%    \begin{macrocode}
%%\providecommand{\version}{final}
%    \end{macrocode}

% Include the main document:
%    \begin{macrocode}
\input{childdoc.def}
\childdocof{cdocsamp}
%    \end{macrocode}

%\iffalse
%</samplechap1|samplechap2>
%\fi
%
%\iffalse
%<*samplechap1>
%\fi
% Some text for chapter 1:
%    \begin{macrocode}
\section{one}
some text in chapter one
%    \end{macrocode}

%\iffalse
%</samplechap1>
%\fi
% Some text for chapter 2:
%\iffalse
%<*samplechap2>
%\fi
%    \begin{macrocode}
\section{two}
more text in chapter two
%    \end{macrocode}

%\iffalse
%</samplechap2>
%\fi
%
% %%%%%%%%%%%%%%%%%%%%%%%%%%%%%%%%%%%%%%
% \paragraph{Part Include Files.}
%
% The include files are called |cdocspt3.tex| and |cdocspt4.tex|.
%
%\iffalse
%<*samplepart3|samplepart4>
%\fi

% Optional override for |\version| flag:
%    \begin{macrocode}
%%\providecommand{\version}{final}
%    \end{macrocode}

% Include the main document:
%    \begin{macrocode}
\input{childdoc.def}
\childdocby{cdocsamp}
%    \end{macrocode}

%\iffalse
%</samplepart3|samplepart4>
%\fi
%
%\iffalse
%<*samplepart3>
%\fi
% Some text for part 3:
%    \begin{macrocode}
some text in part three
%    \end{macrocode}

%\iffalse
%</samplepart3>
%\fi
% Some text for part 4:
%\iffalse
%<*samplepart4>
%\fi
%    \begin{macrocode}
more text in part four
%    \end{macrocode}

%\iffalse
%</samplepart4>
%\fi
%
% %%%%%%%%%%%%%%%%%%%%%%%%%%%%%%%%%%%%%%
% \paragraph{Forwarding for a Complete Draft.}
%
% The following forwarding file |cdocsdrf.tex|
% compiles the main document in draft mode:
%\iffalse
%<*sampledraft>
%\fi
%    \begin{macrocode}
\def\version{draft}
\input{childdoc.def}
\childdocforward{cdocsamp}
%    \end{macrocode}

%\iffalse
%</sampledraft>
%\fi
%
% %%%%%%%%%%%%%%%%%%%%%%%%%%%%%%%%%%%%%%
% \paragraph{Forwarding for Final Version of the Chapters.}
%
% The following forwarding files |cdocsfn1.tex| and |cdocsfn2.tex|
% (with identical content)
% compile the final versions of the child documents
% |cdocsch1.tex| and |cdocsch2.tex|, respectively:
%\iffalse
%<*samplefinal>
%\fi
%    \begin{macrocode}
\def\version{final}
\input{childdoc.def}
\childdocforwardprefix[cdocsamp]{cdocsfn}{cdocsch}
%    \end{macrocode}

%\iffalse
%</samplefinal>
%\fi
%
% %%%%%%%%%%%%%%%%%%%%%%%%%%%%%%%%%%%%%%
% \paragraph{Command Line Processing.}
%
% The following three command lines generate the output files
% |cdocscld|, |cdocscl1| and |cdocscl2|
% which should be identical to
% |cdocsdrf|, |cdocsch1| and |cdocsfn2|, respectively:
% \begin{center}
% \begin{tabular}{l}
% |latex -jobname cdocscld \|\\
% |  "\def\version{draft}\input{childdoc.def}\childdocforward{cdocsamp}"|\\
% |latex -jobname cdocscl1 \|\\
% |  "\input{childdoc.def}\childdocforward[cdocsamp]{cdocsch1}"|\\
% |latex -jobname cdocscl2 \|\\
% |  "\def\version{final}\input{childdoc.def}\childdocforward{cdocsch2}"|
% \end{tabular}
% \end{center}
% Note that the trailing backslash on each first line
% merely continues the input to the second line
% (for convenient cut ant paste).
% Furthermore, the command |latex| can be replaced by any
% of its alternative versions such as |pdflatex|.
%
% %%%%%%%%%%%%%%%%%%%%%%%%%%%%%%%%%%%%%%%%%%%%%%%%%%%%%%%%%%%%%%%%%%%%%%%%%%%%%%
% %%%%%%%%%%%%%%%%%%%%%%%%%%%%%%%%%%%%%%%%%%%%%%%%%%%%%%%%%%%%%%%%%%%%%%%%%%%%%%
% \section{Implementation}
%\iffalse
%<*package>
%\fi
%
% This section describes the definitions file |childdoc.def|.

% The definitions cannot be loaded using |\usepackage| or |\RequirePackage|
% which has a mechanism to prevent loading a style file more than once.
% When loading the definitions by means of |\input|
% multiple instances have to be prevented manually:
%\iffalse
%This code needs to be before the `\ProvidesFile' directive
%which is defined at the beginning of this file.
%Therefore it is also placed there and commented out here.
%</package>
%<*discard>
%\fi
%    \begin{macrocode}
\ifdefined\childdocmain\endinput\fi
%    \end{macrocode}
%\iffalse
%</discard>
%<*package>
%\fi
%
% \macro{\ifchilddoc}
% \macro{\ifchilddocmanual}
% The conditional |\ifchilddoc| tells whether a
% child (true) or main (false) document is being compiled.
% The conditional |\ifchilddocmanual| tells whether
% the |\includeonly| mechanism is used (false) or
% the selection of child files must be performed manually (true).
% The definitions initialise to false:
%    \begin{macrocode}
\newif\ifchilddoc
\newif\ifchilddocmanual
%    \end{macrocode}

% \macro{\childdocname}
% \macro{\childdocjob}
% The macro |\childdocname| stores the name of the main document
% to be compiled. The macro |\childdocjob| stores the name of
% the document on which the \LaTeX{} compiler was originally invoked.
% The content of |\jobname| cannot be compared
% to filenames specified in the source due to different catcodes.
% The following code rescans |\jobname|, stores the result
% in |\childdocname| and saves a copy in |\childdocjob|:
%    \begin{macrocode}
\edef\childdocname{\scantokens\expandafter{\jobname\noexpand}}
\let\childdocjob\childdocname
%    \end{macrocode}

% \macro{\childdocdisable}
% The macro |\childdocdisable| prevents the main file
% from being processed more than once.
% At this stage, the main document command |\childdocmain|
% is assumed to be called once again where it should do nothing.
% Any subsequent call to it should prevent
% a secondary processing of the main document
% It overwrites the forwarding commands
% |\childdocof| and |\childdocforward|
% with empty macros to prevent further inclusions of the main document:
%    \begin{macrocode}
\newcommand{\childdocdisable}
{
  \renewcommand{\childdocmain}[1]{\renewcommand{\childdocmain}[1]{\endinput}}
  \renewcommand{\childdocof}[1]{}
  \renewcommand{\childdocby}[2][]{}
  \renewcommand{\childdocforward}[2][]{}
  \renewcommand{\childdocdisable}{}
}
%    \end{macrocode}

% \macro{\childdocmain}
% The macro |\childdocmain| is to be called at the top of the main file
% with nothing or the main filename (without extension) as argument.
% First, it breaks loops.
% If the argument is not empty and does not match |\childdocname|
% (which is set by the first inclusion of |childdoc.def|),
% |\ifchilddoc| is set to true, |\includeonly| is applied to the child file
% and |\jobname| is set to the main file
% (for proper handling of |.aux| files):
%    \begin{macrocode}
\newcommand{\childdocmain}[1]
{
  \childdocdisable\childdocmain{}
  \if?#1?\else
    \begingroup
      \def\childdoctmp{#1}
      \ifx\childdoctmp\childdocname
        \def\childdoctmp{}
      \else
        \def\childdoctmp
        {
          \childdoctrue
          \includeonly{\childdocname}
          \def\childdocjob{#1}
          \def\jobname{#1}
        }
      \fi
      \expandafter
    \endgroup
    \childdoctmp
  \fi
}
%    \end{macrocode}

% \macro{\childdocof}
% The command |\childdocof| redirects
% compilation to the main file |#1|.
%    \begin{macrocode}
\newcommand{\childdocof}[1]
{
  \childdocdisable
  \childdoctrue
  \includeonly{\childdocname}
  \def\jobname{#1}
  \def\childdocjob{#1}
  \input{#1}
}
%    \end{macrocode}

% \macro{\childdocby}
% The command |\childdocby| ....
%    \begin{macrocode}
\newcommand{\childdocby}[2][]
{
  \childdocdisable
  \childdoctrue
  \childdocmanualtrue
  \if?#1?\else
    \def\jobname{#2}
  \fi
  \def\childdocjob{#2}
  \input{#2}
  \endinput
}
%    \end{macrocode}

% \macro{\childdocforward}
% The command |\childdocforward| redirects
% compilation to the main file or
% (if the optional argument is given) a child file.
% Parameters are set as if the main file
% or a child file starting with |\childdocof| was compiled.
% Then compilation is handed over to the main file:
%    \begin{macrocode}
\newcommand{\childdocforward}[2][]
{
  \begingroup
    \if?#1?
      \def\childdoctmp
      {
        \def\childdocname{#2}
        \def\childdocjob{#2}
        \def\jobname{#2}
        \input{#2}
        \endinput
      }
    \else
      \def\childdoctmp
      {
        \childdocdisable
        \def\childdocname{#2}
        \childdoctrue
        \includeonly{#2}
        \def\childdocjob{#1}
        \def\jobname{#1}
        \input{#1}
        \endinput
      }
    \fi
    \expandafter
  \endgroup
  \childdoctmp
}
%    \end{macrocode}

% \macro{\childdocforwardprefix}
% The command |\childdocforwardprefix| redirects
% compilation to the main or a child file by means of a pattern.
% The prefix |#1| in the current filename is replaced by |#2|
% and the suffix of the current filename is kept
% (it is assumed that the filename does not contain the substring `|~~~|'
% which is used as a delimiter).
% Compilation is handed over to the new file by |\childdocforward|:
%    \begin{macrocode}
\newcommand{\childdocforwardprefix}[3][]
{
  \begingroup
    \def\childdocextract #2##1~~~{\def\childdoctmp{\childdocforward[#1]{#3##1}}}
    \expandafter\childdocextract\childdocname~~~
    \expandafter
  \endgroup
  \childdoctmp
}
%    \end{macrocode}

% \macro{\childdoc}
% The deprecated macro |\childdoc| is a legacy version of |\childdocmain|:
%    \begin{macrocode}
\newcommand{\childdoc}{\childdocmain}
%    \end{macrocode}

% \macro{\childdocredirect}
% The deprecated macro |\childdocredirect| is a legacy version
% of |\childdocforward| and |\childdocforwardprefix|:
%    \begin{macrocode}
\newcommand{\childdocredirect}[2][]
{
  \begingroup
    \if?#1?
      \def\childdoctmp{\childdocforward{#2}}
    \else
      \def\childdoctmp{\childdocforwardprefix{#1}{#2}}
    \fi
    \expandafter
  \endgroup
  \childdoctmp
}
%    \end{macrocode}

%\iffalse
%</package>
%\fi
%
\endinput
|\\
|\childdocforward{|\textit{main}|}|
\end{tabular}
\end{center}
%
Likewise, the following files |final|\textit{nn}|.tex|
compile the final version of the child document
|child|\textit{nn}|.tex|:
%
\begin{center}
\begin{tabular}{l}
|\def\version{final}|\\
|% \iffalse
%
% childdoc.dtx Copyright (C) 2017-2018 Niklas Beisert
%
% This work may be distributed and/or modified under the
% conditions of the LaTeX Project Public License, either version 1.3
% of this license or (at your option) any later version.
% The latest version of this license is in
%   http://www.latex-project.org/lppl.txt
% and version 1.3 or later is part of all distributions of LaTeX
% version 2005/12/01 or later.
%
% This work has the LPPL maintenance status `maintained'.
%
% The Current Maintainer of this work is Niklas Beisert.
%
% This work consists of the files childdoc.dtx and childdoc.ins
% and the derived files childdoc.def and cdocsamp.tex with
% cdocsch1.tex, cdocsch2.tex, cdocsdrf.tex, cdocsfn1.tex, cdocsfn2.tex.
%
%<package>\ifdefined\childdocmain\endinput\fi
%<package>\ProvidesFile{childdoc.def}[2018/12/30 v2.0 child document driver]
%<samplemain>\ProvidesFile{cdocsamp.tex}[2018/12/30 v2.0 sample for childdoc]
%<*driver>
%\ProvidesFile{childdoc.drv}[2018/12/30 v2.0 childdoc reference manual file]
\PassOptionsToClass{10pt,a4paper}{article}
\documentclass{ltxdoc}

\usepackage[margin=35mm]{geometry}
\usepackage{hyperref}
\usepackage{hyperxmp}
\usepackage[usenames]{color}

\hypersetup{colorlinks=true}
\hypersetup{pdfstartview=FitH}
\hypersetup{pdfpagemode=UseNone}
\hypersetup{pdfsource={}}
\hypersetup{pdflang={en-UK}}
\hypersetup{pdfcopyright={Copyright 2017-2018 Niklas Beisert.
  This work may be distributed and/or modified under the
  conditions of the LaTeX Project Public License, either version 1.3
  of this license or (at your option) any later version.}}
\hypersetup{pdflicenseurl={http://www.latex-project.org/lppl.txt}}
\hypersetup{pdfcontactaddress={ETH Zurich, ITP, HIT K,
  Wolfgang-Pauli-Strasse 27}}
\hypersetup{pdfcontactpostcode={8093}}
\hypersetup{pdfcontactcity={Zurich}}
\hypersetup{pdfcontactcountry={Switzerland}}
\hypersetup{pdfcontactemail={nbeisert@itp.phys.ethz.ch}}
\hypersetup{pdfcontacturl={http://people.phys.ethz.ch/\xmptilde nbeisert/}}

\newcommand{\secref}[1]{\hyperref[#1]{section \ref*{#1}}}

\parskip1ex
\parindent0pt
\let\olditemize\itemize
\def\itemize{\olditemize\parskip0pt}

\begin{document}

\title{The \textsf{childdoc} Package}
\hypersetup{pdftitle={The childdoc Package}}
\author{Niklas Beisert\\[2ex]
  Institut f\"ur Theoretische Physik\\
  Eidgen\"ossische Technische Hochschule Z\"urich\\
  Wolfgang-Pauli-Strasse 27, 8093 Z\"urich, Switzerland\\[1ex]
  \href{mailto:nbeisert@itp.phys.ethz.ch}
  {\texttt{nbeisert@itp.phys.ethz.ch}}}
\hypersetup{pdfauthor={Niklas Beisert}}
\hypersetup{pdfsubject={Manual for the LaTeX2e Package childdoc}}
\date{30 December 2018, \textsf{v2.0}}
\maketitle

\begin{abstract}\noindent
\textsf{childdoc} is a \LaTeXe{} package
that enables the direct compilation
of document sections included by |\include|
to individual files.
\end{abstract}

\begingroup
\parskip0ex
\tableofcontents
\endgroup

%%%%%%%%%%%%%%%%%%%%%%%%%%%%%%%%%%%%%%%%%%%%%%%%%%%%%%%%%%%%%%%%%%%%%%%%%%%%%%%%
%%%%%%%%%%%%%%%%%%%%%%%%%%%%%%%%%%%%%%%%%%%%%%%%%%%%%%%%%%%%%%%%%%%%%%%%%%%%%%%%
\section{Introduction}

\LaTeX{} provides a mechanism to structure a large document (such as a book)
into a main file and several child files (containing the chapters)
using the |\include| command.
This mechanism is beneficial for documents
which span hundreds of pages in order to
make the source file(s) more manageable.
Moreover, compilation can be restricted to
selected child files by means of the |\includeonly| command.
The latter feature can be used to reduce the compilation time while editing
(this was significantly more useful in the earlier days of \LaTeX{})
or to generate a smaller document which is easier to navigate.
Another application of |\includeonly| is to generate
documents consisting of selected parts of the complete document.

However, there are a few drawbacks of the plain |\include| mechanism:
\begin{itemize}
\item
The child files cannot be compiled on their own,
they can only be compiled via the main file.
A naive editing environment
(such as a text editor with an option
to have the current file processed by \LaTeX)
may require one to switch to the main file before compiling;
attempting to compile the child file produces errors.
\item
The main file must be modified (each time)
to adjust the |\includeonly| command
to the present needs. This easily leaves the main file in a messy state.
\item
The generated document will always carry the filename
of the main document. This is inconvenient if
several child files are to be compiled and
to be kept for distribution.
\end{itemize}

The present package provides a simple interface
to make child files individually compilable by \LaTeX{}.
Compiling a child file then has the same effect as compiling
the main file with an |\includeonly| command
to select the appropriate child.
Moreover the generated document will carry the name of the child
rather than the main file.
This resolves all three above issues.

This feature is meant to make the editing of books,
thesis documents and lecture notes somewhat more convenient.
However, the package can also be used efficiently for
composing a series of documents (such as exercise sheets)
which are typically distributed individually.
It then assists the author in generating the individual documents
(potentially in different versions)
as well as a document containing the collected series.
Another application is in developing style files
or other kinds of included material
where compilation of the style file could redirect
to a sample or test file.

%%%%%%%%%%%%%%%%%%%%%%%%%%%%%%%%%%%%%%%%%%%%%%%%%%%%%%%%%%%%%%%%%%%%%%%%%%%%%%%%
%%%%%%%%%%%%%%%%%%%%%%%%%%%%%%%%%%%%%%%%%%%%%%%%%%%%%%%%%%%%%%%%%%%%%%%%%%%%%%%%
\section{Usage}

First of all, the package \textsf{childdoc} is \emph{not} a standard
\LaTeXe{} |.sty| style file! Therefore it needs to be invoked in
a non-standard way.

%%%%%%%%%%%%%%%%%%%%%%%%%%%%%%%%%%%%%%%%%%%%%%%%%%%%%%%%%%%%%%%%%%%%%%%%%%%%%%%%
\subsection{Included Files}
\label{sec:include}

%%%%%%%%%%%%%%%%%%%%%%%%%%%%%%%%%%%%%%%%
\DescribeMacro{\childdocmain}
To use the package, add the commands
\begin{center}
\begin{tabular}{l}
|\input{childdoc.def}|\\
|\childdocmain{}|\\
\end{tabular}
\end{center}
at the very top of the main \LaTeX{} file,
in particular \emph{before} the |\documentclass| statement!
The argument of |\childdocmain| should be left empty
(but it must be present).

%%%%%%%%%%%%%%%%%%%%%%%%%%%%%%%%%%%%%%%%
\DescribeMacro{\childdocof}
Furthermore, add the commands
\begin{center}
\begin{tabular}{l}
|\input{childdoc.def}|\\
|\childdocof{|\textit{main}|}|\\
\end{tabular}
\end{center}
at the top of every child file \textit{child}
which is included by |\include{|\textit{child}|}|
from within the main file
(or at least for those files to be compiled individually).
The argument \textit{main} must be the filename of the main file.

There are a couple of
considerations in setting up the main and child documents:

%%%%%%%%%%%%%%%%%%%%%%%%%%%%%%%%%%%%%%%%
\paragraph{Restrictions.}

Please note the following restrictions:
\begin{itemize}
\item
|\childdocmain| must be called with one argument \textit{main}
to ensure compatibility with earlier version of the package.
It must either be empty (|\childdocmain{}|)
or precisely match the filename of the main file in which it is specified.
See \secref{sec:detection} for further information.
\item
The filename \textit{main} must be specified without the |.tex| extension.
\item
The filename \textit{main} is case sensitive
(even in case-insensitive file systems)
due to internal string comparison.
\item
The argument \textit{main} should be fully expanded, it cannot be a macro.
\item
Subdirectories and special characters should be avoided in filenames.
\item
The command |\childdocmain{|\textit{main}|}| must be followed by a whitespace.
It should not be followed immediately by another command
or by a comment mark `|%|'.
This is because the \TeX{} parser reads the token immediately following
the argument of |\childdocmain| and puts it
at the beginning of every child section;
however, a white\-space is ignored.
\end{itemize}

%%%%%%%%%%%%%%%%%%%%%%%%%%%%%%%%%%%%%%%%
\paragraph{Content of Main File.}

It is advisable to place all content in the child files included by |\include|.
Any output contained in the main file will appear in all child documents
unless suppressed manually;
it cannot be suppressed automatically by the |\includeonly| directive
and thus should normally be avoided.
A method to include some content in the main file
by means of conditional processing is described in \secref{sec:conditional}.

%%%%%%%%%%%%%%%%%%%%%%%%%%%%%%%%%%%%%%%%
\paragraph{Page Numbering.}

When only a part of the document is compiled,
the appropriate numbering of pages
(as well as other status parameters)
is determined from the |.aux| files.
The latter contain information from previous passes.
However this information needs to propagate through
all intermediate child documents.
Therefore the page numbering in child documents may well
be inconsistent until the complete document is compiled at least once.

A useful (if unconventional) way to always ensure a consistent
page numbering is to restart the numbering in each child document
and denote the pages by `\textit{child}|.|\textit{page}'
where \textit{child} represents the chapter/section number of the child file.
This can be achieved by the command
|\numberwithin{page}{|\textit{child}|}|
of the \textsf{amsmath} package
where \textit{child} can be |chapter| or |section|
depending on the chosen structuring.
Alternatively, one can modify the macro |\thepage| appropriately
and reset the counter |page| at the start of each child file.

%%%%%%%%%%%%%%%%%%%%%%%%%%%%%%%%%%%%%%%%%%%%%%%%%%%%%%%%%%%%%%%%%%%%%%%%%%%%%%%%
\subsection{Conditional Processing}
\label{sec:conditional}

The package provides a mechanism to compile different versions
of a document. To customise the versions further some conditional processing
can come in handy to distinguish which version is being compiled.
The package provides two macros to describe the compilation context:

%%%%%%%%%%%%%%%%%%%%%%%%%%%%%%%%%%%%%%%%
\DescribeMacro{\ifchilddoc}
The conditional |\ifchilddoc| distinguishes between the compilation of
child documents and the main document:
%
\begin{center}
|\ifchilddoc |\textit{child-code}| |[|\||else |\textit{main-code}]| \||fi|
\end{center}

%%%%%%%%%%%%%%%%%%%%%%%%%%%%%%%%%%%%%%%%
\DescribeMacro{\childdocname}
\DescribeMacro{\childdocjob}
The macro |\childdocname| contains the filename (without extension)
of the main or child file being processed.
Note that |\childdocjob| will always contain the name of the main file.

%%%%%%%%%%%%%%%%%%%%%%%%%%%%%%%%%%%%%%%%
\paragraph{Title Page.}

Conditional processing can be used to include a title or banner page
in the main document when proper precautions are taken.
Importantly, the code in the main file should ensure that the page counter
(as well as other status parameters which are stored in the |.aux| files)
takes the same value after the conditional processing.
Otherwise the page numbers may take divergent values
depending on which part is compiled.

For example, a title page could be declared by:
%
\begin{center}
\begin{tabular}{l}
|\ifchilddoc\||else|\\
|\addtocounter{page}{-1}|\\
\textit{code for title page}\\
|\newpage|\\
|\||fi|
\end{tabular}
\end{center}
%
A banner page for the child documents can be generated by:
%
\begin{center}
\begin{tabular}{l}
|\ifchilddoc|\\
|\addtocounter{page}{-1}|\\
\textit{code for banner page}\\
|\newpage|\\
|\||fi|
\end{tabular}
\end{center}
%
Here one could write a message such as:
\begin{center}
|This is the part \childdocname{} of \childdocjob{}.|
\end{center}

%%%%%%%%%%%%%%%%%%%%%%%%%%%%%%%%%%%%%%%%%%%%%%%%%%%%%%%%%%%%%%%%%%%%%%%%%%%%%%%%
\subsection{Flags}
\label{sec:flags}

The package makes it easy to generate different versions
of the main or child documents.
To this end compilation flags can be defined
and assigned different default values.
They will be particularly useful in conjunction
with the forwarding mechanism described in \secref{sec:forward}.

For example, it may be useful to have a flag |\version|
which can be set to |draft| or |final|.
The document source will contain some conditional code
depending on the value of |\version|.
Suppose further, the flag should default to |final| for the main file
and to |draft| for child files
which is a natural assignment for editing the document.
This is achieved by placing the following code
in the preamble of the main document
(below the |\childdocmain| directive):
%
\begin{center}
\begin{tabular}{l}
|\ifchilddoc|\\
|\providecommand{\version}{draft}|\\
|\||else|\\
|\providecommand{\version}{final}|\\
|\||fi|
\end{tabular}
\end{center}
%
The definition by |\providecommand| makes sure
that previous definitions are not overwritten.
Further statements |\providecommand{\version}{...}|
can thus be added before the above code to override it.

For the main file, one might add a line
(between |\childdocmain| and the above block)
%
\begin{center}
|%\ifchilddoc\||else\providecommand{\version}{draft}\||fi|
\end{center}
%
which can be uncommented to produce a draft version.
Likewise one can add a line to the very top of a child file
(above the |\childdocof{|\textit{main}|}| directive)
%
\begin{center}
|%\providecommand{\version}{final}|
\end{center}
%
which can be uncommented to produce the final version of this child document.

%%%%%%%%%%%%%%%%%%%%%%%%%%%%%%%%%%%%%%%%%%%%%%%%%%%%%%%%%%%%%%%%%%%%%%%%%%%%%%%%
\subsection{Forwarding}
\label{sec:forward}

Different versions of the main or child documents
using compilation flags as described in \secref{sec:flags}
can be (permanently) stored in different files
for convenient compilation, viewing and distribution.
To this end, the package defines a command
to pass on compilation to a different file:

%%%%%%%%%%%%%%%%%%%%%%%%%%%%%%%%%%%%%%%%
\DescribeMacro{\childdocforward}
The command |\childdocforward| redirects processing to
another source file:
%
\begin{center}
\begin{tabular}{l}
|\input{childdoc.def}|\\
|\childdocforward[|\textit{main}|]{|\textit{dest}|}|\\
\end{tabular}
\end{center}
%
The argument \textit{dest} is the destination file
(without extension).
It should be the main file or one of the child files.
Note that further \textsf{childdoc} directives
such as |\childdocof| and |\childdocforward|
in the indicated file will be processed in this form.
The optional argument \textit{main}
passes on directly to the main file \textit{main}
while pretending to compile the child \textit{dest}.
This form behaves as if \textit{dest}
issues |\childdocof{|\textit{main}|}| right away,
and no further \textsf{childdoc} directives will be processed.

%%%%%%%%%%%%%%%%%%%%%%%%%%%%%%%%%%%%%%%%
\DescribeMacro{\...prefix}
In the alternative form |\childdocforwardprefix|,
%
\begin{center}
\begin{tabular}{l}
|\input{childdoc.def}|\\
|\childdocforwardprefix[|\textit{main}|]{|\textit{prefix}|}{|\textit{dest}|}|
\end{tabular}
\end{center}
%
the destination file is determined by a pattern
depending on the current file:
To make this work, the current file must be called
`{\textit{prefix}\hspace{0.2em}\textit{suffix}}'
with \textit{prefix} matching precisely the argument.
Processing is then passed on to the file
`{\textit{dest}\hspace{0.2em}\textit{suffix}}'.
Surely, the same effect is achieved by
directly specifying the
argument `{\textit{dest}\hspace{0.2em}\textit{suffix}}'
in the first form.
However, that requires to set up a different file
for each child. With the alternative form of the command
all these files can have exactly the same content
which simplifies setting them up and maintaining them.

For example, the following file |draft.tex|
with a compilation flag |\version| as described in \secref{sec:flags}
compiles the main document as a draft:
%
\begin{center}
\begin{tabular}{l}
|\def\version{draft}|\\
|\input{childdoc.def}|\\
|\childdocforward{|\textit{main}|}|
\end{tabular}
\end{center}
%
Likewise, the following files |final|\textit{nn}|.tex|
compile the final version of the child document
|child|\textit{nn}|.tex|:
%
\begin{center}
\begin{tabular}{l}
|\def\version{final}|\\
|\input{childdoc.def}|\\
|\childdocforwardprefix{final}{child}|
\end{tabular}
\end{center}
%

Note that when several versions of a main file and/or of each child file
are to be generated, it may be convenient to set up a |Makefile| or
shell script to automatise the process.

%%%%%%%%%%%%%%%%%%%%%%%%%%%%%%%%%%%%%%%%%%%%%%%%%%%%%%%%%%%%%%%%%%%%%%%%%%%%%%%%
\subsection{Command Line Processing}
\label{sec:commandline}

The effect of redirection files can also be achieved by invoking
the \LaTeX{} compiler with a more elaborate command line.
Most conveniently this should be done as part
of a shell script or a |Makefile|.

When using \textsf{childdoc} in the main file, the following
command lines effectively perform a redirection
(note that depending on the shell being used,
backslashes may have to be doubled: `|\|' $\to$ `|\\|'):
%
\begin{center}
|... -jobname "|\textit{target}|" |\\|"|[\textit{flags}]%
|\input{childdoc.def}\childdocforward[|\textit{main}|]{|\textit{dest}|}"|
\end{center}
%
Here \textit{target} is the name of the output file,
\textit{main} is the name of the main file
and \textit{dest} is the name of the main or child file to be processed
(all filenames without extensions).
The optional argument \textit{main} can be omitted
if \textit{main} matches \textit{dest}.
Optionally, compilation \textit{flags} can be defined via |\def| commands.
This command line makes the \TeX{} engine believe
it is compiling the file \textit{target}
whose content is specified as the latter parameter.
The provided code then forwards the processing to
\textit{main} or \textit{dest} as described in \secref{sec:forward}.

%%%%%%%%%%%%%%%%%%%%%%%%%%%%%%%%%%%%%%%%%%%%%%%%%%%%%%%%%%%%%%%%%%%%%%%%%%%%%%%%
\subsection{Include by Input}
\label{sec:input}

Including child documents by |\include| has some restrictions by design.
Most notably, the content of a child document always occupies
its own set of pages; pages cannot be shared between child documents.
Usually, this behaviour makes perfect sense
because each child document contain an essential part of the document.
However, in some situations it may be desirable to compose
a document from a collection of parts
without having mandatory page breaks between then.
For this case, the package
provides a mechanism to include parts
by |\input| which can also be processed individually.
However, by construction this mechanism
requires manual handling of the content to be output.

%%%%%%%%%%%%%%%%%%%%%%%%%%%%%%%%%%%%%%%%
\DescribeMacro{\ifchilddocmanual}
The main file should be prepared as usual, see \secref{sec:include}.
However, the document body must make a distinction
between processing of an individual part and of the main document, e.g.:
%
\begin{center}
\begin{tabular}{l}
|\ifchilddocmanual|\\
|\input{\childdocname}|\\
|\||else|\\
\textit{document body with }|\input{|\textit{part}|}|\\
|\||fi|
\end{tabular}
\end{center}
%
The conditional |\ifchilddocmanual| is true whenever
a part to be included by |\input| is being compiled,
and the name of the part is stored in |\childdocname|.

%%%%%%%%%%%%%%%%%%%%%%%%%%%%%%%%%%%%%%%%
\DescribeMacro{\childdocby}
Each part to be included by |\input| should start with:
%
\begin{center}
\begin{tabular}{l}
|\input{childdoc.def}|\\
|\childdocby{|\textit{main}|}|\\
\end{tabular}
\end{center}
%
The directive |\childdocby| is similar to |\childdocof|
described in \secref{sec:include},
but the subsequent selection of content must be done manually.
To that end, both |\ifchilddoc| and |\ifchilddocmanual|
will be true upon processing of a part,
and the name of the part is stored in |\childdocname|.
Note that |\jobname| will be set to the filename of the current part
so that each part receives an individual |.aux| file
that does not interfere with the |.aux| file(s) of the main document.
This behaviour can be altered by the alternative form
|\childdocby[*]{|\textit{main}|}| (with a non-empty optional argument)
which uses the |.aux| file of the main document
by setting |\jobname| to \textit{main}.

%%%%%%%%%%%%%%%%%%%%%%%%%%%%%%%%%%%%%%%%%%%%%%%%%%%%%%%%%%%%%%%%%%%%%%%%%%%%%%%%
\subsection{Driver Development}
\label{sec:driver}

The \textsf{childdoc} mechanism can also be use for the development
of definition files such as \LaTeX{} styles or classes.
This case differs from the above setup with multiple parts
included by |\include| in that no |\includeonly| should be invoked.
This can be achieved by starting the include file
(before |\ProvidesPackage|) with:
%
\begin{center}
\begin{tabular}{l}
|\input{childdoc.def}|\\
|\childdocforward{|\textit{main}|}|\\
\end{tabular}
\end{center}
%
or alternatively with:
%
\begin{center}
\begin{tabular}{l}
|\input{childdoc.def}|\\
|\childdocby{|\textit{main}|}|\\
\end{tabular}
\end{center}
%
Both forms have slightly different effects as described above.
The main file is prepared as usual, see \secref{sec:include}.

%%%%%%%%%%%%%%%%%%%%%%%%%%%%%%%%%%%%%%%%%%%%%%%%%%%%%%%%%%%%%%%%%%%%%%%%%%%%%%%%
\subsection{Legacy Detection}
\label{sec:detection}

The directive |\childdocmain| in the main file can detect
whether the complete document or merely a child is to be compiled
even without using the directive |\childdocof|.
This method is deprecated because it is less robust
and there is no compelling reason to use it;
it is merely provided for backward compatibility
and it may be removed in future versions.

If the detection mechanism is to be used,
it is mandatory to correctly specify
the filename of the main file as the argument of |\childdocmain|:
%
\begin{center}
\begin{tabular}{l}
|\input{childdoc.def}|\\
|\childdocmain{|\textit{main}|}|\\
\end{tabular}
\end{center}
%
If |\jobname| does not match the argument \textit{main} of |\childdocmain|,
it is assumed that |\jobname| points to the child file to be compiled.
When using |\childdocmain| with the main file specified as argument,
it suffices to start a child file
with just |\input{|\textit{main}|}|
without loading of the package and using |\childdocof|.
If instead all processing is done
with the appropriate \textsf{childdoc} directives,
the argument of \textit{main} of |\childdocmain| can be empty.

An alternative version of the command line processing described
in \secref{sec:commandline} using the detection mechanism reads:
%
\begin{center}
|... -jobname "|\textit{target}|" "|[\textit{flags}]%
[|\def\jobname{|\textit{dest}|}|]|\input{|\textit{main}|}"|
\end{center}

%%%%%%%%%%%%%%%%%%%%%%%%%%%%%%%%%%%%%%%%%%%%%%%%%%%%%%%%%%%%%%%%%%%%%%%%%%%%%%%%
\subsection{Manual Code}
\label{sec:manual}

In case one cannot be certain whether the definitions file |childdoc.def|
is installed on the target \TeX{} distribution
and one prefers not to ship it,
it is conceivable to paste a few relevant commands into the sources.

To that end, drop all statements |\input{childdoc.def}|
and perform the replacements as outlined below.
Instead of |\childdocmain{|\textit{main}|}| add the following code
to the top of the main file:
%
\begin{center}
\begin{tabular}{l}
|\||ifdefined\childdocname\endinput\||fi\newif\ifchilddoc|\\
|\edef\childdocname{\scantokens\expandafter{\jobname\noexpand}}|\\
|\def\childdocmain{|\textit{main}|}\||ifx\childdocmain\childdocname\||else|\\
|\childdoctrue\includeonly{\childdocname}\let\jobname\childdocmain\||fi|\\
\end{tabular}
\end{center}
%
Instead of |\childdocof{|\textit{main}|}| just include the main file
at the top of each child file:
%
\begin{center}
|\input{|\textit{main}|}|
\end{center}
%
A simple redirection |\childdocforward{|\textit{dest}|}| is achieved by:
%
\begin{center}
|\def\jobname{|\textit{dest}|}\input{\jobname}|
\end{center}
%
The redirection with prefix
|\childdocforwardprefix[|\textit{prefix}|]{|\textit{dest}|}|
is accomplished by:
%
\begin{center}
\begin{tabular}{l}
|{\edef\jobname{\scantokens\expandafter{\jobname\noexpand}}|\\
|\def\redirectjob |\textit{prefix}|#1~~~{\gdef\jobname{|\textit{dest}|#1}}|\\
|\expandafter\redirectjob\jobname~~~}\input{\jobname}|
\end{tabular}
\end{center}

In an alternative approach,
child documents can be compiled by a specific command line
without additional code or specific definitions:
%
\begin{center}
|... -jobname "|\textit{target}|" "|[\textit{flags}]%
|\includeonly{|\textit{dest}|}\input{|\textit{main}|}"|
\end{center}
%

%%%%%%%%%%%%%%%%%%%%%%%%%%%%%%%%%%%%%%%%%%%%%%%%%%%%%%%%%%%%%%%%%%%%%%%%%%%%%%%%
%%%%%%%%%%%%%%%%%%%%%%%%%%%%%%%%%%%%%%%%%%%%%%%%%%%%%%%%%%%%%%%%%%%%%%%%%%%%%%%%
\section{Information}

%%%%%%%%%%%%%%%%%%%%%%%%%%%%%%%%%%%%%%%%%%%%%%%%%%%%%%%%%%%%%%%%%%%%%%%%%%%%%%%%
\subsection{Copyright}

Copyright \copyright{} 2017--2018 Niklas Beisert

This work may be distributed and/or modified under the
conditions of the \LaTeX{} Project Public License, either version 1.3
of this license or (at your option) any later version.
The latest version of this license is in
  \url{http://www.latex-project.org/lppl.txt}
and version 1.3 or later is part of all distributions of \LaTeX{}
version 2005/12/01 or later.

This work has the LPPL maintenance status `maintained'.

The Current Maintainer of this work is Niklas Beisert.

This work consists of the files |README.txt|, |childdoc.ins| and |childdoc.dtx|
as well as the derived files |childdoc.def|, |cdocsamp.tex|
with |cdocsch1.tex|, |cdocsch2.tex|, |cdocspt3.tex|, |cdocspt4.tex|,
|cdocsdrf.tex|, |cdocsfn1.tex|, |cdocsfn2.tex|
as well as |childdoc.pdf|.

%%%%%%%%%%%%%%%%%%%%%%%%%%%%%%%%%%%%%%%%%%%%%%%%%%%%%%%%%%%%%%%%%%%%%%%%%%%%%%%%
\subsection{Files and Installation}

The package consists of the files:
%
\begin{center}
\begin{tabular}{ll}
    |README.txt|   & readme file \\
    |childdoc.ins| & installation file \\
    |childdoc.dtx| & source file \\
    |childdoc.def| & definition file \\
    |cdocsamp.tex| & sample main file \\
    |cdocsch1.tex| & sample include file \\
    |cdocsch2.tex| & sample include file \\
    |cdocspt3.tex| & sample part file \\
    |cdocspt4.tex| & sample part file \\
    |cdocsdrf.tex| & sample redirection file \\
    |cdocsfn1.tex| & sample redirection file \\
    |cdocsfn2.tex| & sample redirection file \\
    |childdoc.pdf| & manual
\end{tabular}
\end{center}
%
The distribution consists of the files
|README.txt|, |childdoc.ins| and |childdoc.dtx|.
%
\begin{itemize}
\item
Run (pdf)\LaTeX{} on |childdoc.dtx|
to compile the manual |childdoc.pdf| (this file).
\item
Run \LaTeX{} on |childdoc.ins| to create the definitions file |childdoc.def|
and the sample |cdocsamp.tex| with include files
|cdocsch1.tex|, |cdocsch2.tex|, |cdocspt3.tex|, |cdocspt4.tex|,
|cdocsdrf.tex|, |cdocsfn1.tex|, |cdocsfn2.tex|.
Then copy the file |childdoc.def| to an appropriate directory of your \LaTeX{}
distribution, e.g.\ \textit{texmf-root}|/tex/latex/childdoc|.
\end{itemize}

%%%%%%%%%%%%%%%%%%%%%%%%%%%%%%%%%%%%%%%%%%%%%%%%%%%%%%%%%%%%%%%%%%%%%%%%%%%%%%%%
\subsection{Related CTAN Packages}

There are several other packages which offer a similar functionality:
%
\begin{itemize}
\item
The packages
\href{http://ctan.org/pkg/docmute}{\textsf{docmute}},
\href{http://ctan.org/pkg/includex}{\textsf{includex}} and
\href{http://ctan.org/pkg/standalone}{\textsf{standalone}}
provide commands to include only the document body of
a child file thus allowing both files to be compiled individually.
\item
The packages \href{http://ctan.org/pkg/subdocs}{\textsf{subdocs}}
and \href{http://ctan.org/pkg/subfiles}{\textsf{subfiles}}
provide structures in which the main and child documents can be
encapsulated and allowing them to be compiled individually.
The inclusion mechanism is different from the conventional |\include|.
\item
The package \href{http://ctan.org/pkg/combine}{\textsf{combine}}
is an elaborate solution to combine several documents into one.
\end{itemize}
%
See also the CTAN topic \href{http://ctan.org/topic/subdocs}{\textsf{subdocs}}
for further related packages.
The present package differs from the above solutions in that
a document structure constructed with the conventional |\include| mechanism
just needs two extra commands at the top of every file
such that all constituent files can be compiled individually.

%%%%%%%%%%%%%%%%%%%%%%%%%%%%%%%%%%%%%%%%%%%%%%%%%%%%%%%%%%%%%%%%%%%%%%%%%%%%%%%%
%\subsection{Feature Suggestions}
%
%The following is a list of features which may be useful for future
%versions of this package:
%%
%\begin{itemize}
%\item
%\ldots
%\end{itemize}

%%%%%%%%%%%%%%%%%%%%%%%%%%%%%%%%%%%%%%%%%%%%%%%%%%%%%%%%%%%%%%%%%%%%%%%%%%%%%%%%
\subsection{Revision History}

%%%%%%%%%%%%%%%%%%%%%%%%%%%%%%%%%%%%%%%%
\paragraph{v2.0:} 2018/12/30

\begin{itemize}
\item
immediate forward processing
\item
added |\childdocby| mechanism
\item
manual restructured
\end{itemize}

%%%%%%%%%%%%%%%%%%%%%%%%%%%%%%%%%%%%%%%%
\paragraph{v1.6:} 2018/01/17

\begin{itemize}
\item
application for development of include files
\item
corrections to manual
\end{itemize}

%%%%%%%%%%%%%%%%%%%%%%%%%%%%%%%%%%%%%%%%
\paragraph{v1.5:} 2017/05/21

\begin{itemize}
\item
more complete structuring introduced
\item
|\childdocof| introduced
\item
|\childdoc| renamed to |\childdocmain|
\item
|\childredirect| renamed to |\childdocforward| and |\childdocforwardprefix|
and functionality expanded
\end{itemize}

%%%%%%%%%%%%%%%%%%%%%%%%%%%%%%%%%%%%%%%%
\paragraph{v1.0:} 2017/04/27

\begin{itemize}
\item
manual and install package
\item
first version published on CTAN
\end{itemize}

%%%%%%%%%%%%%%%%%%%%%%%%%%%%%%%%%%%%%%%%
\paragraph{v0.6:} 2017/04/26

\begin{itemize}
\item
redirection mechanism added
\end{itemize}

%%%%%%%%%%%%%%%%%%%%%%%%%%%%%%%%%%%%%%%%
\paragraph{v0.5:} 2017/04/26

\begin{itemize}
\item
functionality in definition file
\end{itemize}


%%%%%%%%%%%%%%%%%%%%%%%%%%%%%%%%%%%%%%%%%%%%%%%%%%%%%%%%%%%%%%%%%%%%%%%%%%%%%%%%
%%%%%%%%%%%%%%%%%%%%%%%%%%%%%%%%%%%%%%%%%%%%%%%%%%%%%%%%%%%%%%%%%%%%%%%%%%%%%%%%
%%%%%%%%%%%%%%%%%%%%%%%%%%%%%%%%%%%%%%%%%%%%%%%%%%%%%%%%%%%%%%%%%%%%%%%%%%%%%%%%
\appendix

\settowidth\MacroIndent{\rmfamily\scriptsize 000\ }

 \DocInput{childdoc.dtx}

\end{document}
%</driver>
% \fi
%
% %%%%%%%%%%%%%%%%%%%%%%%%%%%%%%%%%%%%%%%%%%%%%%%%%%%%%%%%%%%%%%%%%%%%%%%%%%%%%%
% %%%%%%%%%%%%%%%%%%%%%%%%%%%%%%%%%%%%%%%%%%%%%%%%%%%%%%%%%%%%%%%%%%%%%%%%%%%%%%
% \section{Sample}
%\iffalse
%<*samplemain>
%\fi
%
% The following presents a sample document
% with two chapters, two parts, a title page,
% a compile flag as well as three forwarding files to set the flag.
% It consists of eight |.tex| files:
% \begin{center}
% \begin{tabular}{ll}
% |cdocsamp.tex|&main file\\
% |cdocsch1.tex|&include file for chapter 1\\
% |cdocsch2.tex|&include file for chapter 2\\
% |cdocspt3.tex|&include file for part 3\\
% |cdocspt4.tex|&include file for part 4\\
% |cdocsdrf.tex|&forwarding file for main file in draft mode\\
% |cdocsfi1.tex|&forwarding file for final version of chapter 1\\
% |cdocsfi2.tex|&forwarding file for final version of chapter 2\\
% \end{tabular}
% \end{center}
% Each of the eight files can be compiled directly by the \LaTeX{} compiler.
%
% %%%%%%%%%%%%%%%%%%%%%%%%%%%%%%%%%%%%%%
% \paragraph{Main File.}
%
% The main file is called |cdocsamp.tex|.
%
% Load the \textsf{childdoc} definitions and
% declare the filename for the main document:
%    \begin{macrocode}
\input{childdoc.def}
\childdocmain{}
%    \end{macrocode}

% Optional override for |\version| flag:
%    \begin{macrocode}
%%\ifchilddoc\else\providecommand{\version}{draft}\fi
%    \end{macrocode}

% Define the default values for the |\version| flag
% (|final| for the main file and |draft| for childs):
%    \begin{macrocode}
\ifchilddoc
\providecommand{\version}{draft}
\else
\providecommand{\version}{final}
\fi
%    \end{macrocode}

% Load the standard document class:
%    \begin{macrocode}
\documentclass[12pt]{article}
%    \end{macrocode}

% Start the document body:
%    \begin{macrocode}
\begin{document}
%    \end{macrocode}

% Declare a title page.
% Print title, part of document being processed and version flag:
%    \begin{macrocode}
\addtocounter{page}{-1}
\begin{center}
{\LARGE\bfseries{}childdoc example\par}
\vspace{1cm}
\ifchilddoc
\ifchilddocmanual part\else chapter\fi:
`\childdocname' of `\childdocjob'\par
\else
main document: `\childdocjob'\par
\fi
version: \version\par
\end{center}
\newpage
%    \end{macrocode}

% Manually include selected file,
% otherwise process as usual:
%    \begin{macrocode}
\ifchilddocmanual
\section*{part `\childdocname'}
\input{\childdocname}
\else
%    \end{macrocode}

% Include the two chapters:
%    \begin{macrocode}
\include{cdocsch1}
\include{cdocsch2}
%    \end{macrocode}

% Include the two parts unless only chapters should be displayed:
%    \begin{macrocode}
\ifchilddoc\else
\section{part three}
\input{cdocspt3}
\section{part four}
\input{cdocspt4}
\fi
%    \end{macrocode}

% Process as usual until here:
%    \begin{macrocode}
\fi
%    \end{macrocode}

% End of document body:
%    \begin{macrocode}
\end{document}
%    \end{macrocode}
%\iffalse
%</samplemain>
%\fi
%
% %%%%%%%%%%%%%%%%%%%%%%%%%%%%%%%%%%%%%%
% \paragraph{Chapter Include Files.}
%
% The include files are called |cdocsch1.tex| and |cdocsch2.tex|.
%
%\iffalse
%<*samplechap1|samplechap2>
%\fi

% Optional override for |\version| flag:
%    \begin{macrocode}
%%\providecommand{\version}{final}
%    \end{macrocode}

% Include the main document:
%    \begin{macrocode}
\input{childdoc.def}
\childdocof{cdocsamp}
%    \end{macrocode}

%\iffalse
%</samplechap1|samplechap2>
%\fi
%
%\iffalse
%<*samplechap1>
%\fi
% Some text for chapter 1:
%    \begin{macrocode}
\section{one}
some text in chapter one
%    \end{macrocode}

%\iffalse
%</samplechap1>
%\fi
% Some text for chapter 2:
%\iffalse
%<*samplechap2>
%\fi
%    \begin{macrocode}
\section{two}
more text in chapter two
%    \end{macrocode}

%\iffalse
%</samplechap2>
%\fi
%
% %%%%%%%%%%%%%%%%%%%%%%%%%%%%%%%%%%%%%%
% \paragraph{Part Include Files.}
%
% The include files are called |cdocspt3.tex| and |cdocspt4.tex|.
%
%\iffalse
%<*samplepart3|samplepart4>
%\fi

% Optional override for |\version| flag:
%    \begin{macrocode}
%%\providecommand{\version}{final}
%    \end{macrocode}

% Include the main document:
%    \begin{macrocode}
\input{childdoc.def}
\childdocby{cdocsamp}
%    \end{macrocode}

%\iffalse
%</samplepart3|samplepart4>
%\fi
%
%\iffalse
%<*samplepart3>
%\fi
% Some text for part 3:
%    \begin{macrocode}
some text in part three
%    \end{macrocode}

%\iffalse
%</samplepart3>
%\fi
% Some text for part 4:
%\iffalse
%<*samplepart4>
%\fi
%    \begin{macrocode}
more text in part four
%    \end{macrocode}

%\iffalse
%</samplepart4>
%\fi
%
% %%%%%%%%%%%%%%%%%%%%%%%%%%%%%%%%%%%%%%
% \paragraph{Forwarding for a Complete Draft.}
%
% The following forwarding file |cdocsdrf.tex|
% compiles the main document in draft mode:
%\iffalse
%<*sampledraft>
%\fi
%    \begin{macrocode}
\def\version{draft}
\input{childdoc.def}
\childdocforward{cdocsamp}
%    \end{macrocode}

%\iffalse
%</sampledraft>
%\fi
%
% %%%%%%%%%%%%%%%%%%%%%%%%%%%%%%%%%%%%%%
% \paragraph{Forwarding for Final Version of the Chapters.}
%
% The following forwarding files |cdocsfn1.tex| and |cdocsfn2.tex|
% (with identical content)
% compile the final versions of the child documents
% |cdocsch1.tex| and |cdocsch2.tex|, respectively:
%\iffalse
%<*samplefinal>
%\fi
%    \begin{macrocode}
\def\version{final}
\input{childdoc.def}
\childdocforwardprefix[cdocsamp]{cdocsfn}{cdocsch}
%    \end{macrocode}

%\iffalse
%</samplefinal>
%\fi
%
% %%%%%%%%%%%%%%%%%%%%%%%%%%%%%%%%%%%%%%
% \paragraph{Command Line Processing.}
%
% The following three command lines generate the output files
% |cdocscld|, |cdocscl1| and |cdocscl2|
% which should be identical to
% |cdocsdrf|, |cdocsch1| and |cdocsfn2|, respectively:
% \begin{center}
% \begin{tabular}{l}
% |latex -jobname cdocscld \|\\
% |  "\def\version{draft}\input{childdoc.def}\childdocforward{cdocsamp}"|\\
% |latex -jobname cdocscl1 \|\\
% |  "\input{childdoc.def}\childdocforward[cdocsamp]{cdocsch1}"|\\
% |latex -jobname cdocscl2 \|\\
% |  "\def\version{final}\input{childdoc.def}\childdocforward{cdocsch2}"|
% \end{tabular}
% \end{center}
% Note that the trailing backslash on each first line
% merely continues the input to the second line
% (for convenient cut ant paste).
% Furthermore, the command |latex| can be replaced by any
% of its alternative versions such as |pdflatex|.
%
% %%%%%%%%%%%%%%%%%%%%%%%%%%%%%%%%%%%%%%%%%%%%%%%%%%%%%%%%%%%%%%%%%%%%%%%%%%%%%%
% %%%%%%%%%%%%%%%%%%%%%%%%%%%%%%%%%%%%%%%%%%%%%%%%%%%%%%%%%%%%%%%%%%%%%%%%%%%%%%
% \section{Implementation}
%\iffalse
%<*package>
%\fi
%
% This section describes the definitions file |childdoc.def|.

% The definitions cannot be loaded using |\usepackage| or |\RequirePackage|
% which has a mechanism to prevent loading a style file more than once.
% When loading the definitions by means of |\input|
% multiple instances have to be prevented manually:
%\iffalse
%This code needs to be before the `\ProvidesFile' directive
%which is defined at the beginning of this file.
%Therefore it is also placed there and commented out here.
%</package>
%<*discard>
%\fi
%    \begin{macrocode}
\ifdefined\childdocmain\endinput\fi
%    \end{macrocode}
%\iffalse
%</discard>
%<*package>
%\fi
%
% \macro{\ifchilddoc}
% \macro{\ifchilddocmanual}
% The conditional |\ifchilddoc| tells whether a
% child (true) or main (false) document is being compiled.
% The conditional |\ifchilddocmanual| tells whether
% the |\includeonly| mechanism is used (false) or
% the selection of child files must be performed manually (true).
% The definitions initialise to false:
%    \begin{macrocode}
\newif\ifchilddoc
\newif\ifchilddocmanual
%    \end{macrocode}

% \macro{\childdocname}
% \macro{\childdocjob}
% The macro |\childdocname| stores the name of the main document
% to be compiled. The macro |\childdocjob| stores the name of
% the document on which the \LaTeX{} compiler was originally invoked.
% The content of |\jobname| cannot be compared
% to filenames specified in the source due to different catcodes.
% The following code rescans |\jobname|, stores the result
% in |\childdocname| and saves a copy in |\childdocjob|:
%    \begin{macrocode}
\edef\childdocname{\scantokens\expandafter{\jobname\noexpand}}
\let\childdocjob\childdocname
%    \end{macrocode}

% \macro{\childdocdisable}
% The macro |\childdocdisable| prevents the main file
% from being processed more than once.
% At this stage, the main document command |\childdocmain|
% is assumed to be called once again where it should do nothing.
% Any subsequent call to it should prevent
% a secondary processing of the main document
% It overwrites the forwarding commands
% |\childdocof| and |\childdocforward|
% with empty macros to prevent further inclusions of the main document:
%    \begin{macrocode}
\newcommand{\childdocdisable}
{
  \renewcommand{\childdocmain}[1]{\renewcommand{\childdocmain}[1]{\endinput}}
  \renewcommand{\childdocof}[1]{}
  \renewcommand{\childdocby}[2][]{}
  \renewcommand{\childdocforward}[2][]{}
  \renewcommand{\childdocdisable}{}
}
%    \end{macrocode}

% \macro{\childdocmain}
% The macro |\childdocmain| is to be called at the top of the main file
% with nothing or the main filename (without extension) as argument.
% First, it breaks loops.
% If the argument is not empty and does not match |\childdocname|
% (which is set by the first inclusion of |childdoc.def|),
% |\ifchilddoc| is set to true, |\includeonly| is applied to the child file
% and |\jobname| is set to the main file
% (for proper handling of |.aux| files):
%    \begin{macrocode}
\newcommand{\childdocmain}[1]
{
  \childdocdisable\childdocmain{}
  \if?#1?\else
    \begingroup
      \def\childdoctmp{#1}
      \ifx\childdoctmp\childdocname
        \def\childdoctmp{}
      \else
        \def\childdoctmp
        {
          \childdoctrue
          \includeonly{\childdocname}
          \def\childdocjob{#1}
          \def\jobname{#1}
        }
      \fi
      \expandafter
    \endgroup
    \childdoctmp
  \fi
}
%    \end{macrocode}

% \macro{\childdocof}
% The command |\childdocof| redirects
% compilation to the main file |#1|.
%    \begin{macrocode}
\newcommand{\childdocof}[1]
{
  \childdocdisable
  \childdoctrue
  \includeonly{\childdocname}
  \def\jobname{#1}
  \def\childdocjob{#1}
  \input{#1}
}
%    \end{macrocode}

% \macro{\childdocby}
% The command |\childdocby| ....
%    \begin{macrocode}
\newcommand{\childdocby}[2][]
{
  \childdocdisable
  \childdoctrue
  \childdocmanualtrue
  \if?#1?\else
    \def\jobname{#2}
  \fi
  \def\childdocjob{#2}
  \input{#2}
  \endinput
}
%    \end{macrocode}

% \macro{\childdocforward}
% The command |\childdocforward| redirects
% compilation to the main file or
% (if the optional argument is given) a child file.
% Parameters are set as if the main file
% or a child file starting with |\childdocof| was compiled.
% Then compilation is handed over to the main file:
%    \begin{macrocode}
\newcommand{\childdocforward}[2][]
{
  \begingroup
    \if?#1?
      \def\childdoctmp
      {
        \def\childdocname{#2}
        \def\childdocjob{#2}
        \def\jobname{#2}
        \input{#2}
        \endinput
      }
    \else
      \def\childdoctmp
      {
        \childdocdisable
        \def\childdocname{#2}
        \childdoctrue
        \includeonly{#2}
        \def\childdocjob{#1}
        \def\jobname{#1}
        \input{#1}
        \endinput
      }
    \fi
    \expandafter
  \endgroup
  \childdoctmp
}
%    \end{macrocode}

% \macro{\childdocforwardprefix}
% The command |\childdocforwardprefix| redirects
% compilation to the main or a child file by means of a pattern.
% The prefix |#1| in the current filename is replaced by |#2|
% and the suffix of the current filename is kept
% (it is assumed that the filename does not contain the substring `|~~~|'
% which is used as a delimiter).
% Compilation is handed over to the new file by |\childdocforward|:
%    \begin{macrocode}
\newcommand{\childdocforwardprefix}[3][]
{
  \begingroup
    \def\childdocextract #2##1~~~{\def\childdoctmp{\childdocforward[#1]{#3##1}}}
    \expandafter\childdocextract\childdocname~~~
    \expandafter
  \endgroup
  \childdoctmp
}
%    \end{macrocode}

% \macro{\childdoc}
% The deprecated macro |\childdoc| is a legacy version of |\childdocmain|:
%    \begin{macrocode}
\newcommand{\childdoc}{\childdocmain}
%    \end{macrocode}

% \macro{\childdocredirect}
% The deprecated macro |\childdocredirect| is a legacy version
% of |\childdocforward| and |\childdocforwardprefix|:
%    \begin{macrocode}
\newcommand{\childdocredirect}[2][]
{
  \begingroup
    \if?#1?
      \def\childdoctmp{\childdocforward{#2}}
    \else
      \def\childdoctmp{\childdocforwardprefix{#1}{#2}}
    \fi
    \expandafter
  \endgroup
  \childdoctmp
}
%    \end{macrocode}

%\iffalse
%</package>
%\fi
%
\endinput
|\\
|\childdocforwardprefix{final}{child}|
\end{tabular}
\end{center}
%

Note that when several versions of a main file and/or of each child file
are to be generated, it may be convenient to set up a |Makefile| or
shell script to automatise the process.

%%%%%%%%%%%%%%%%%%%%%%%%%%%%%%%%%%%%%%%%%%%%%%%%%%%%%%%%%%%%%%%%%%%%%%%%%%%%%%%%
\subsection{Command Line Processing}
\label{sec:commandline}

The effect of redirection files can also be achieved by invoking
the \LaTeX{} compiler with a more elaborate command line.
Most conveniently this should be done as part
of a shell script or a |Makefile|.

When using \textsf{childdoc} in the main file, the following
command lines effectively perform a redirection
(note that depending on the shell being used,
backslashes may have to be doubled: `|\|' $\to$ `|\\|'):
%
\begin{center}
|... -jobname "|\textit{target}|" |\\|"|[\textit{flags}]%
|% \iffalse
%
% childdoc.dtx Copyright (C) 2017-2018 Niklas Beisert
%
% This work may be distributed and/or modified under the
% conditions of the LaTeX Project Public License, either version 1.3
% of this license or (at your option) any later version.
% The latest version of this license is in
%   http://www.latex-project.org/lppl.txt
% and version 1.3 or later is part of all distributions of LaTeX
% version 2005/12/01 or later.
%
% This work has the LPPL maintenance status `maintained'.
%
% The Current Maintainer of this work is Niklas Beisert.
%
% This work consists of the files childdoc.dtx and childdoc.ins
% and the derived files childdoc.def and cdocsamp.tex with
% cdocsch1.tex, cdocsch2.tex, cdocsdrf.tex, cdocsfn1.tex, cdocsfn2.tex.
%
%<package>\ifdefined\childdocmain\endinput\fi
%<package>\ProvidesFile{childdoc.def}[2018/12/30 v2.0 child document driver]
%<samplemain>\ProvidesFile{cdocsamp.tex}[2018/12/30 v2.0 sample for childdoc]
%<*driver>
%\ProvidesFile{childdoc.drv}[2018/12/30 v2.0 childdoc reference manual file]
\PassOptionsToClass{10pt,a4paper}{article}
\documentclass{ltxdoc}

\usepackage[margin=35mm]{geometry}
\usepackage{hyperref}
\usepackage{hyperxmp}
\usepackage[usenames]{color}

\hypersetup{colorlinks=true}
\hypersetup{pdfstartview=FitH}
\hypersetup{pdfpagemode=UseNone}
\hypersetup{pdfsource={}}
\hypersetup{pdflang={en-UK}}
\hypersetup{pdfcopyright={Copyright 2017-2018 Niklas Beisert.
  This work may be distributed and/or modified under the
  conditions of the LaTeX Project Public License, either version 1.3
  of this license or (at your option) any later version.}}
\hypersetup{pdflicenseurl={http://www.latex-project.org/lppl.txt}}
\hypersetup{pdfcontactaddress={ETH Zurich, ITP, HIT K,
  Wolfgang-Pauli-Strasse 27}}
\hypersetup{pdfcontactpostcode={8093}}
\hypersetup{pdfcontactcity={Zurich}}
\hypersetup{pdfcontactcountry={Switzerland}}
\hypersetup{pdfcontactemail={nbeisert@itp.phys.ethz.ch}}
\hypersetup{pdfcontacturl={http://people.phys.ethz.ch/\xmptilde nbeisert/}}

\newcommand{\secref}[1]{\hyperref[#1]{section \ref*{#1}}}

\parskip1ex
\parindent0pt
\let\olditemize\itemize
\def\itemize{\olditemize\parskip0pt}

\begin{document}

\title{The \textsf{childdoc} Package}
\hypersetup{pdftitle={The childdoc Package}}
\author{Niklas Beisert\\[2ex]
  Institut f\"ur Theoretische Physik\\
  Eidgen\"ossische Technische Hochschule Z\"urich\\
  Wolfgang-Pauli-Strasse 27, 8093 Z\"urich, Switzerland\\[1ex]
  \href{mailto:nbeisert@itp.phys.ethz.ch}
  {\texttt{nbeisert@itp.phys.ethz.ch}}}
\hypersetup{pdfauthor={Niklas Beisert}}
\hypersetup{pdfsubject={Manual for the LaTeX2e Package childdoc}}
\date{30 December 2018, \textsf{v2.0}}
\maketitle

\begin{abstract}\noindent
\textsf{childdoc} is a \LaTeXe{} package
that enables the direct compilation
of document sections included by |\include|
to individual files.
\end{abstract}

\begingroup
\parskip0ex
\tableofcontents
\endgroup

%%%%%%%%%%%%%%%%%%%%%%%%%%%%%%%%%%%%%%%%%%%%%%%%%%%%%%%%%%%%%%%%%%%%%%%%%%%%%%%%
%%%%%%%%%%%%%%%%%%%%%%%%%%%%%%%%%%%%%%%%%%%%%%%%%%%%%%%%%%%%%%%%%%%%%%%%%%%%%%%%
\section{Introduction}

\LaTeX{} provides a mechanism to structure a large document (such as a book)
into a main file and several child files (containing the chapters)
using the |\include| command.
This mechanism is beneficial for documents
which span hundreds of pages in order to
make the source file(s) more manageable.
Moreover, compilation can be restricted to
selected child files by means of the |\includeonly| command.
The latter feature can be used to reduce the compilation time while editing
(this was significantly more useful in the earlier days of \LaTeX{})
or to generate a smaller document which is easier to navigate.
Another application of |\includeonly| is to generate
documents consisting of selected parts of the complete document.

However, there are a few drawbacks of the plain |\include| mechanism:
\begin{itemize}
\item
The child files cannot be compiled on their own,
they can only be compiled via the main file.
A naive editing environment
(such as a text editor with an option
to have the current file processed by \LaTeX)
may require one to switch to the main file before compiling;
attempting to compile the child file produces errors.
\item
The main file must be modified (each time)
to adjust the |\includeonly| command
to the present needs. This easily leaves the main file in a messy state.
\item
The generated document will always carry the filename
of the main document. This is inconvenient if
several child files are to be compiled and
to be kept for distribution.
\end{itemize}

The present package provides a simple interface
to make child files individually compilable by \LaTeX{}.
Compiling a child file then has the same effect as compiling
the main file with an |\includeonly| command
to select the appropriate child.
Moreover the generated document will carry the name of the child
rather than the main file.
This resolves all three above issues.

This feature is meant to make the editing of books,
thesis documents and lecture notes somewhat more convenient.
However, the package can also be used efficiently for
composing a series of documents (such as exercise sheets)
which are typically distributed individually.
It then assists the author in generating the individual documents
(potentially in different versions)
as well as a document containing the collected series.
Another application is in developing style files
or other kinds of included material
where compilation of the style file could redirect
to a sample or test file.

%%%%%%%%%%%%%%%%%%%%%%%%%%%%%%%%%%%%%%%%%%%%%%%%%%%%%%%%%%%%%%%%%%%%%%%%%%%%%%%%
%%%%%%%%%%%%%%%%%%%%%%%%%%%%%%%%%%%%%%%%%%%%%%%%%%%%%%%%%%%%%%%%%%%%%%%%%%%%%%%%
\section{Usage}

First of all, the package \textsf{childdoc} is \emph{not} a standard
\LaTeXe{} |.sty| style file! Therefore it needs to be invoked in
a non-standard way.

%%%%%%%%%%%%%%%%%%%%%%%%%%%%%%%%%%%%%%%%%%%%%%%%%%%%%%%%%%%%%%%%%%%%%%%%%%%%%%%%
\subsection{Included Files}
\label{sec:include}

%%%%%%%%%%%%%%%%%%%%%%%%%%%%%%%%%%%%%%%%
\DescribeMacro{\childdocmain}
To use the package, add the commands
\begin{center}
\begin{tabular}{l}
|\input{childdoc.def}|\\
|\childdocmain{}|\\
\end{tabular}
\end{center}
at the very top of the main \LaTeX{} file,
in particular \emph{before} the |\documentclass| statement!
The argument of |\childdocmain| should be left empty
(but it must be present).

%%%%%%%%%%%%%%%%%%%%%%%%%%%%%%%%%%%%%%%%
\DescribeMacro{\childdocof}
Furthermore, add the commands
\begin{center}
\begin{tabular}{l}
|\input{childdoc.def}|\\
|\childdocof{|\textit{main}|}|\\
\end{tabular}
\end{center}
at the top of every child file \textit{child}
which is included by |\include{|\textit{child}|}|
from within the main file
(or at least for those files to be compiled individually).
The argument \textit{main} must be the filename of the main file.

There are a couple of
considerations in setting up the main and child documents:

%%%%%%%%%%%%%%%%%%%%%%%%%%%%%%%%%%%%%%%%
\paragraph{Restrictions.}

Please note the following restrictions:
\begin{itemize}
\item
|\childdocmain| must be called with one argument \textit{main}
to ensure compatibility with earlier version of the package.
It must either be empty (|\childdocmain{}|)
or precisely match the filename of the main file in which it is specified.
See \secref{sec:detection} for further information.
\item
The filename \textit{main} must be specified without the |.tex| extension.
\item
The filename \textit{main} is case sensitive
(even in case-insensitive file systems)
due to internal string comparison.
\item
The argument \textit{main} should be fully expanded, it cannot be a macro.
\item
Subdirectories and special characters should be avoided in filenames.
\item
The command |\childdocmain{|\textit{main}|}| must be followed by a whitespace.
It should not be followed immediately by another command
or by a comment mark `|%|'.
This is because the \TeX{} parser reads the token immediately following
the argument of |\childdocmain| and puts it
at the beginning of every child section;
however, a white\-space is ignored.
\end{itemize}

%%%%%%%%%%%%%%%%%%%%%%%%%%%%%%%%%%%%%%%%
\paragraph{Content of Main File.}

It is advisable to place all content in the child files included by |\include|.
Any output contained in the main file will appear in all child documents
unless suppressed manually;
it cannot be suppressed automatically by the |\includeonly| directive
and thus should normally be avoided.
A method to include some content in the main file
by means of conditional processing is described in \secref{sec:conditional}.

%%%%%%%%%%%%%%%%%%%%%%%%%%%%%%%%%%%%%%%%
\paragraph{Page Numbering.}

When only a part of the document is compiled,
the appropriate numbering of pages
(as well as other status parameters)
is determined from the |.aux| files.
The latter contain information from previous passes.
However this information needs to propagate through
all intermediate child documents.
Therefore the page numbering in child documents may well
be inconsistent until the complete document is compiled at least once.

A useful (if unconventional) way to always ensure a consistent
page numbering is to restart the numbering in each child document
and denote the pages by `\textit{child}|.|\textit{page}'
where \textit{child} represents the chapter/section number of the child file.
This can be achieved by the command
|\numberwithin{page}{|\textit{child}|}|
of the \textsf{amsmath} package
where \textit{child} can be |chapter| or |section|
depending on the chosen structuring.
Alternatively, one can modify the macro |\thepage| appropriately
and reset the counter |page| at the start of each child file.

%%%%%%%%%%%%%%%%%%%%%%%%%%%%%%%%%%%%%%%%%%%%%%%%%%%%%%%%%%%%%%%%%%%%%%%%%%%%%%%%
\subsection{Conditional Processing}
\label{sec:conditional}

The package provides a mechanism to compile different versions
of a document. To customise the versions further some conditional processing
can come in handy to distinguish which version is being compiled.
The package provides two macros to describe the compilation context:

%%%%%%%%%%%%%%%%%%%%%%%%%%%%%%%%%%%%%%%%
\DescribeMacro{\ifchilddoc}
The conditional |\ifchilddoc| distinguishes between the compilation of
child documents and the main document:
%
\begin{center}
|\ifchilddoc |\textit{child-code}| |[|\||else |\textit{main-code}]| \||fi|
\end{center}

%%%%%%%%%%%%%%%%%%%%%%%%%%%%%%%%%%%%%%%%
\DescribeMacro{\childdocname}
\DescribeMacro{\childdocjob}
The macro |\childdocname| contains the filename (without extension)
of the main or child file being processed.
Note that |\childdocjob| will always contain the name of the main file.

%%%%%%%%%%%%%%%%%%%%%%%%%%%%%%%%%%%%%%%%
\paragraph{Title Page.}

Conditional processing can be used to include a title or banner page
in the main document when proper precautions are taken.
Importantly, the code in the main file should ensure that the page counter
(as well as other status parameters which are stored in the |.aux| files)
takes the same value after the conditional processing.
Otherwise the page numbers may take divergent values
depending on which part is compiled.

For example, a title page could be declared by:
%
\begin{center}
\begin{tabular}{l}
|\ifchilddoc\||else|\\
|\addtocounter{page}{-1}|\\
\textit{code for title page}\\
|\newpage|\\
|\||fi|
\end{tabular}
\end{center}
%
A banner page for the child documents can be generated by:
%
\begin{center}
\begin{tabular}{l}
|\ifchilddoc|\\
|\addtocounter{page}{-1}|\\
\textit{code for banner page}\\
|\newpage|\\
|\||fi|
\end{tabular}
\end{center}
%
Here one could write a message such as:
\begin{center}
|This is the part \childdocname{} of \childdocjob{}.|
\end{center}

%%%%%%%%%%%%%%%%%%%%%%%%%%%%%%%%%%%%%%%%%%%%%%%%%%%%%%%%%%%%%%%%%%%%%%%%%%%%%%%%
\subsection{Flags}
\label{sec:flags}

The package makes it easy to generate different versions
of the main or child documents.
To this end compilation flags can be defined
and assigned different default values.
They will be particularly useful in conjunction
with the forwarding mechanism described in \secref{sec:forward}.

For example, it may be useful to have a flag |\version|
which can be set to |draft| or |final|.
The document source will contain some conditional code
depending on the value of |\version|.
Suppose further, the flag should default to |final| for the main file
and to |draft| for child files
which is a natural assignment for editing the document.
This is achieved by placing the following code
in the preamble of the main document
(below the |\childdocmain| directive):
%
\begin{center}
\begin{tabular}{l}
|\ifchilddoc|\\
|\providecommand{\version}{draft}|\\
|\||else|\\
|\providecommand{\version}{final}|\\
|\||fi|
\end{tabular}
\end{center}
%
The definition by |\providecommand| makes sure
that previous definitions are not overwritten.
Further statements |\providecommand{\version}{...}|
can thus be added before the above code to override it.

For the main file, one might add a line
(between |\childdocmain| and the above block)
%
\begin{center}
|%\ifchilddoc\||else\providecommand{\version}{draft}\||fi|
\end{center}
%
which can be uncommented to produce a draft version.
Likewise one can add a line to the very top of a child file
(above the |\childdocof{|\textit{main}|}| directive)
%
\begin{center}
|%\providecommand{\version}{final}|
\end{center}
%
which can be uncommented to produce the final version of this child document.

%%%%%%%%%%%%%%%%%%%%%%%%%%%%%%%%%%%%%%%%%%%%%%%%%%%%%%%%%%%%%%%%%%%%%%%%%%%%%%%%
\subsection{Forwarding}
\label{sec:forward}

Different versions of the main or child documents
using compilation flags as described in \secref{sec:flags}
can be (permanently) stored in different files
for convenient compilation, viewing and distribution.
To this end, the package defines a command
to pass on compilation to a different file:

%%%%%%%%%%%%%%%%%%%%%%%%%%%%%%%%%%%%%%%%
\DescribeMacro{\childdocforward}
The command |\childdocforward| redirects processing to
another source file:
%
\begin{center}
\begin{tabular}{l}
|\input{childdoc.def}|\\
|\childdocforward[|\textit{main}|]{|\textit{dest}|}|\\
\end{tabular}
\end{center}
%
The argument \textit{dest} is the destination file
(without extension).
It should be the main file or one of the child files.
Note that further \textsf{childdoc} directives
such as |\childdocof| and |\childdocforward|
in the indicated file will be processed in this form.
The optional argument \textit{main}
passes on directly to the main file \textit{main}
while pretending to compile the child \textit{dest}.
This form behaves as if \textit{dest}
issues |\childdocof{|\textit{main}|}| right away,
and no further \textsf{childdoc} directives will be processed.

%%%%%%%%%%%%%%%%%%%%%%%%%%%%%%%%%%%%%%%%
\DescribeMacro{\...prefix}
In the alternative form |\childdocforwardprefix|,
%
\begin{center}
\begin{tabular}{l}
|\input{childdoc.def}|\\
|\childdocforwardprefix[|\textit{main}|]{|\textit{prefix}|}{|\textit{dest}|}|
\end{tabular}
\end{center}
%
the destination file is determined by a pattern
depending on the current file:
To make this work, the current file must be called
`{\textit{prefix}\hspace{0.2em}\textit{suffix}}'
with \textit{prefix} matching precisely the argument.
Processing is then passed on to the file
`{\textit{dest}\hspace{0.2em}\textit{suffix}}'.
Surely, the same effect is achieved by
directly specifying the
argument `{\textit{dest}\hspace{0.2em}\textit{suffix}}'
in the first form.
However, that requires to set up a different file
for each child. With the alternative form of the command
all these files can have exactly the same content
which simplifies setting them up and maintaining them.

For example, the following file |draft.tex|
with a compilation flag |\version| as described in \secref{sec:flags}
compiles the main document as a draft:
%
\begin{center}
\begin{tabular}{l}
|\def\version{draft}|\\
|\input{childdoc.def}|\\
|\childdocforward{|\textit{main}|}|
\end{tabular}
\end{center}
%
Likewise, the following files |final|\textit{nn}|.tex|
compile the final version of the child document
|child|\textit{nn}|.tex|:
%
\begin{center}
\begin{tabular}{l}
|\def\version{final}|\\
|\input{childdoc.def}|\\
|\childdocforwardprefix{final}{child}|
\end{tabular}
\end{center}
%

Note that when several versions of a main file and/or of each child file
are to be generated, it may be convenient to set up a |Makefile| or
shell script to automatise the process.

%%%%%%%%%%%%%%%%%%%%%%%%%%%%%%%%%%%%%%%%%%%%%%%%%%%%%%%%%%%%%%%%%%%%%%%%%%%%%%%%
\subsection{Command Line Processing}
\label{sec:commandline}

The effect of redirection files can also be achieved by invoking
the \LaTeX{} compiler with a more elaborate command line.
Most conveniently this should be done as part
of a shell script or a |Makefile|.

When using \textsf{childdoc} in the main file, the following
command lines effectively perform a redirection
(note that depending on the shell being used,
backslashes may have to be doubled: `|\|' $\to$ `|\\|'):
%
\begin{center}
|... -jobname "|\textit{target}|" |\\|"|[\textit{flags}]%
|\input{childdoc.def}\childdocforward[|\textit{main}|]{|\textit{dest}|}"|
\end{center}
%
Here \textit{target} is the name of the output file,
\textit{main} is the name of the main file
and \textit{dest} is the name of the main or child file to be processed
(all filenames without extensions).
The optional argument \textit{main} can be omitted
if \textit{main} matches \textit{dest}.
Optionally, compilation \textit{flags} can be defined via |\def| commands.
This command line makes the \TeX{} engine believe
it is compiling the file \textit{target}
whose content is specified as the latter parameter.
The provided code then forwards the processing to
\textit{main} or \textit{dest} as described in \secref{sec:forward}.

%%%%%%%%%%%%%%%%%%%%%%%%%%%%%%%%%%%%%%%%%%%%%%%%%%%%%%%%%%%%%%%%%%%%%%%%%%%%%%%%
\subsection{Include by Input}
\label{sec:input}

Including child documents by |\include| has some restrictions by design.
Most notably, the content of a child document always occupies
its own set of pages; pages cannot be shared between child documents.
Usually, this behaviour makes perfect sense
because each child document contain an essential part of the document.
However, in some situations it may be desirable to compose
a document from a collection of parts
without having mandatory page breaks between then.
For this case, the package
provides a mechanism to include parts
by |\input| which can also be processed individually.
However, by construction this mechanism
requires manual handling of the content to be output.

%%%%%%%%%%%%%%%%%%%%%%%%%%%%%%%%%%%%%%%%
\DescribeMacro{\ifchilddocmanual}
The main file should be prepared as usual, see \secref{sec:include}.
However, the document body must make a distinction
between processing of an individual part and of the main document, e.g.:
%
\begin{center}
\begin{tabular}{l}
|\ifchilddocmanual|\\
|\input{\childdocname}|\\
|\||else|\\
\textit{document body with }|\input{|\textit{part}|}|\\
|\||fi|
\end{tabular}
\end{center}
%
The conditional |\ifchilddocmanual| is true whenever
a part to be included by |\input| is being compiled,
and the name of the part is stored in |\childdocname|.

%%%%%%%%%%%%%%%%%%%%%%%%%%%%%%%%%%%%%%%%
\DescribeMacro{\childdocby}
Each part to be included by |\input| should start with:
%
\begin{center}
\begin{tabular}{l}
|\input{childdoc.def}|\\
|\childdocby{|\textit{main}|}|\\
\end{tabular}
\end{center}
%
The directive |\childdocby| is similar to |\childdocof|
described in \secref{sec:include},
but the subsequent selection of content must be done manually.
To that end, both |\ifchilddoc| and |\ifchilddocmanual|
will be true upon processing of a part,
and the name of the part is stored in |\childdocname|.
Note that |\jobname| will be set to the filename of the current part
so that each part receives an individual |.aux| file
that does not interfere with the |.aux| file(s) of the main document.
This behaviour can be altered by the alternative form
|\childdocby[*]{|\textit{main}|}| (with a non-empty optional argument)
which uses the |.aux| file of the main document
by setting |\jobname| to \textit{main}.

%%%%%%%%%%%%%%%%%%%%%%%%%%%%%%%%%%%%%%%%%%%%%%%%%%%%%%%%%%%%%%%%%%%%%%%%%%%%%%%%
\subsection{Driver Development}
\label{sec:driver}

The \textsf{childdoc} mechanism can also be use for the development
of definition files such as \LaTeX{} styles or classes.
This case differs from the above setup with multiple parts
included by |\include| in that no |\includeonly| should be invoked.
This can be achieved by starting the include file
(before |\ProvidesPackage|) with:
%
\begin{center}
\begin{tabular}{l}
|\input{childdoc.def}|\\
|\childdocforward{|\textit{main}|}|\\
\end{tabular}
\end{center}
%
or alternatively with:
%
\begin{center}
\begin{tabular}{l}
|\input{childdoc.def}|\\
|\childdocby{|\textit{main}|}|\\
\end{tabular}
\end{center}
%
Both forms have slightly different effects as described above.
The main file is prepared as usual, see \secref{sec:include}.

%%%%%%%%%%%%%%%%%%%%%%%%%%%%%%%%%%%%%%%%%%%%%%%%%%%%%%%%%%%%%%%%%%%%%%%%%%%%%%%%
\subsection{Legacy Detection}
\label{sec:detection}

The directive |\childdocmain| in the main file can detect
whether the complete document or merely a child is to be compiled
even without using the directive |\childdocof|.
This method is deprecated because it is less robust
and there is no compelling reason to use it;
it is merely provided for backward compatibility
and it may be removed in future versions.

If the detection mechanism is to be used,
it is mandatory to correctly specify
the filename of the main file as the argument of |\childdocmain|:
%
\begin{center}
\begin{tabular}{l}
|\input{childdoc.def}|\\
|\childdocmain{|\textit{main}|}|\\
\end{tabular}
\end{center}
%
If |\jobname| does not match the argument \textit{main} of |\childdocmain|,
it is assumed that |\jobname| points to the child file to be compiled.
When using |\childdocmain| with the main file specified as argument,
it suffices to start a child file
with just |\input{|\textit{main}|}|
without loading of the package and using |\childdocof|.
If instead all processing is done
with the appropriate \textsf{childdoc} directives,
the argument of \textit{main} of |\childdocmain| can be empty.

An alternative version of the command line processing described
in \secref{sec:commandline} using the detection mechanism reads:
%
\begin{center}
|... -jobname "|\textit{target}|" "|[\textit{flags}]%
[|\def\jobname{|\textit{dest}|}|]|\input{|\textit{main}|}"|
\end{center}

%%%%%%%%%%%%%%%%%%%%%%%%%%%%%%%%%%%%%%%%%%%%%%%%%%%%%%%%%%%%%%%%%%%%%%%%%%%%%%%%
\subsection{Manual Code}
\label{sec:manual}

In case one cannot be certain whether the definitions file |childdoc.def|
is installed on the target \TeX{} distribution
and one prefers not to ship it,
it is conceivable to paste a few relevant commands into the sources.

To that end, drop all statements |\input{childdoc.def}|
and perform the replacements as outlined below.
Instead of |\childdocmain{|\textit{main}|}| add the following code
to the top of the main file:
%
\begin{center}
\begin{tabular}{l}
|\||ifdefined\childdocname\endinput\||fi\newif\ifchilddoc|\\
|\edef\childdocname{\scantokens\expandafter{\jobname\noexpand}}|\\
|\def\childdocmain{|\textit{main}|}\||ifx\childdocmain\childdocname\||else|\\
|\childdoctrue\includeonly{\childdocname}\let\jobname\childdocmain\||fi|\\
\end{tabular}
\end{center}
%
Instead of |\childdocof{|\textit{main}|}| just include the main file
at the top of each child file:
%
\begin{center}
|\input{|\textit{main}|}|
\end{center}
%
A simple redirection |\childdocforward{|\textit{dest}|}| is achieved by:
%
\begin{center}
|\def\jobname{|\textit{dest}|}\input{\jobname}|
\end{center}
%
The redirection with prefix
|\childdocforwardprefix[|\textit{prefix}|]{|\textit{dest}|}|
is accomplished by:
%
\begin{center}
\begin{tabular}{l}
|{\edef\jobname{\scantokens\expandafter{\jobname\noexpand}}|\\
|\def\redirectjob |\textit{prefix}|#1~~~{\gdef\jobname{|\textit{dest}|#1}}|\\
|\expandafter\redirectjob\jobname~~~}\input{\jobname}|
\end{tabular}
\end{center}

In an alternative approach,
child documents can be compiled by a specific command line
without additional code or specific definitions:
%
\begin{center}
|... -jobname "|\textit{target}|" "|[\textit{flags}]%
|\includeonly{|\textit{dest}|}\input{|\textit{main}|}"|
\end{center}
%

%%%%%%%%%%%%%%%%%%%%%%%%%%%%%%%%%%%%%%%%%%%%%%%%%%%%%%%%%%%%%%%%%%%%%%%%%%%%%%%%
%%%%%%%%%%%%%%%%%%%%%%%%%%%%%%%%%%%%%%%%%%%%%%%%%%%%%%%%%%%%%%%%%%%%%%%%%%%%%%%%
\section{Information}

%%%%%%%%%%%%%%%%%%%%%%%%%%%%%%%%%%%%%%%%%%%%%%%%%%%%%%%%%%%%%%%%%%%%%%%%%%%%%%%%
\subsection{Copyright}

Copyright \copyright{} 2017--2018 Niklas Beisert

This work may be distributed and/or modified under the
conditions of the \LaTeX{} Project Public License, either version 1.3
of this license or (at your option) any later version.
The latest version of this license is in
  \url{http://www.latex-project.org/lppl.txt}
and version 1.3 or later is part of all distributions of \LaTeX{}
version 2005/12/01 or later.

This work has the LPPL maintenance status `maintained'.

The Current Maintainer of this work is Niklas Beisert.

This work consists of the files |README.txt|, |childdoc.ins| and |childdoc.dtx|
as well as the derived files |childdoc.def|, |cdocsamp.tex|
with |cdocsch1.tex|, |cdocsch2.tex|, |cdocspt3.tex|, |cdocspt4.tex|,
|cdocsdrf.tex|, |cdocsfn1.tex|, |cdocsfn2.tex|
as well as |childdoc.pdf|.

%%%%%%%%%%%%%%%%%%%%%%%%%%%%%%%%%%%%%%%%%%%%%%%%%%%%%%%%%%%%%%%%%%%%%%%%%%%%%%%%
\subsection{Files and Installation}

The package consists of the files:
%
\begin{center}
\begin{tabular}{ll}
    |README.txt|   & readme file \\
    |childdoc.ins| & installation file \\
    |childdoc.dtx| & source file \\
    |childdoc.def| & definition file \\
    |cdocsamp.tex| & sample main file \\
    |cdocsch1.tex| & sample include file \\
    |cdocsch2.tex| & sample include file \\
    |cdocspt3.tex| & sample part file \\
    |cdocspt4.tex| & sample part file \\
    |cdocsdrf.tex| & sample redirection file \\
    |cdocsfn1.tex| & sample redirection file \\
    |cdocsfn2.tex| & sample redirection file \\
    |childdoc.pdf| & manual
\end{tabular}
\end{center}
%
The distribution consists of the files
|README.txt|, |childdoc.ins| and |childdoc.dtx|.
%
\begin{itemize}
\item
Run (pdf)\LaTeX{} on |childdoc.dtx|
to compile the manual |childdoc.pdf| (this file).
\item
Run \LaTeX{} on |childdoc.ins| to create the definitions file |childdoc.def|
and the sample |cdocsamp.tex| with include files
|cdocsch1.tex|, |cdocsch2.tex|, |cdocspt3.tex|, |cdocspt4.tex|,
|cdocsdrf.tex|, |cdocsfn1.tex|, |cdocsfn2.tex|.
Then copy the file |childdoc.def| to an appropriate directory of your \LaTeX{}
distribution, e.g.\ \textit{texmf-root}|/tex/latex/childdoc|.
\end{itemize}

%%%%%%%%%%%%%%%%%%%%%%%%%%%%%%%%%%%%%%%%%%%%%%%%%%%%%%%%%%%%%%%%%%%%%%%%%%%%%%%%
\subsection{Related CTAN Packages}

There are several other packages which offer a similar functionality:
%
\begin{itemize}
\item
The packages
\href{http://ctan.org/pkg/docmute}{\textsf{docmute}},
\href{http://ctan.org/pkg/includex}{\textsf{includex}} and
\href{http://ctan.org/pkg/standalone}{\textsf{standalone}}
provide commands to include only the document body of
a child file thus allowing both files to be compiled individually.
\item
The packages \href{http://ctan.org/pkg/subdocs}{\textsf{subdocs}}
and \href{http://ctan.org/pkg/subfiles}{\textsf{subfiles}}
provide structures in which the main and child documents can be
encapsulated and allowing them to be compiled individually.
The inclusion mechanism is different from the conventional |\include|.
\item
The package \href{http://ctan.org/pkg/combine}{\textsf{combine}}
is an elaborate solution to combine several documents into one.
\end{itemize}
%
See also the CTAN topic \href{http://ctan.org/topic/subdocs}{\textsf{subdocs}}
for further related packages.
The present package differs from the above solutions in that
a document structure constructed with the conventional |\include| mechanism
just needs two extra commands at the top of every file
such that all constituent files can be compiled individually.

%%%%%%%%%%%%%%%%%%%%%%%%%%%%%%%%%%%%%%%%%%%%%%%%%%%%%%%%%%%%%%%%%%%%%%%%%%%%%%%%
%\subsection{Feature Suggestions}
%
%The following is a list of features which may be useful for future
%versions of this package:
%%
%\begin{itemize}
%\item
%\ldots
%\end{itemize}

%%%%%%%%%%%%%%%%%%%%%%%%%%%%%%%%%%%%%%%%%%%%%%%%%%%%%%%%%%%%%%%%%%%%%%%%%%%%%%%%
\subsection{Revision History}

%%%%%%%%%%%%%%%%%%%%%%%%%%%%%%%%%%%%%%%%
\paragraph{v2.0:} 2018/12/30

\begin{itemize}
\item
immediate forward processing
\item
added |\childdocby| mechanism
\item
manual restructured
\end{itemize}

%%%%%%%%%%%%%%%%%%%%%%%%%%%%%%%%%%%%%%%%
\paragraph{v1.6:} 2018/01/17

\begin{itemize}
\item
application for development of include files
\item
corrections to manual
\end{itemize}

%%%%%%%%%%%%%%%%%%%%%%%%%%%%%%%%%%%%%%%%
\paragraph{v1.5:} 2017/05/21

\begin{itemize}
\item
more complete structuring introduced
\item
|\childdocof| introduced
\item
|\childdoc| renamed to |\childdocmain|
\item
|\childredirect| renamed to |\childdocforward| and |\childdocforwardprefix|
and functionality expanded
\end{itemize}

%%%%%%%%%%%%%%%%%%%%%%%%%%%%%%%%%%%%%%%%
\paragraph{v1.0:} 2017/04/27

\begin{itemize}
\item
manual and install package
\item
first version published on CTAN
\end{itemize}

%%%%%%%%%%%%%%%%%%%%%%%%%%%%%%%%%%%%%%%%
\paragraph{v0.6:} 2017/04/26

\begin{itemize}
\item
redirection mechanism added
\end{itemize}

%%%%%%%%%%%%%%%%%%%%%%%%%%%%%%%%%%%%%%%%
\paragraph{v0.5:} 2017/04/26

\begin{itemize}
\item
functionality in definition file
\end{itemize}


%%%%%%%%%%%%%%%%%%%%%%%%%%%%%%%%%%%%%%%%%%%%%%%%%%%%%%%%%%%%%%%%%%%%%%%%%%%%%%%%
%%%%%%%%%%%%%%%%%%%%%%%%%%%%%%%%%%%%%%%%%%%%%%%%%%%%%%%%%%%%%%%%%%%%%%%%%%%%%%%%
%%%%%%%%%%%%%%%%%%%%%%%%%%%%%%%%%%%%%%%%%%%%%%%%%%%%%%%%%%%%%%%%%%%%%%%%%%%%%%%%
\appendix

\settowidth\MacroIndent{\rmfamily\scriptsize 000\ }

 \DocInput{childdoc.dtx}

\end{document}
%</driver>
% \fi
%
% %%%%%%%%%%%%%%%%%%%%%%%%%%%%%%%%%%%%%%%%%%%%%%%%%%%%%%%%%%%%%%%%%%%%%%%%%%%%%%
% %%%%%%%%%%%%%%%%%%%%%%%%%%%%%%%%%%%%%%%%%%%%%%%%%%%%%%%%%%%%%%%%%%%%%%%%%%%%%%
% \section{Sample}
%\iffalse
%<*samplemain>
%\fi
%
% The following presents a sample document
% with two chapters, two parts, a title page,
% a compile flag as well as three forwarding files to set the flag.
% It consists of eight |.tex| files:
% \begin{center}
% \begin{tabular}{ll}
% |cdocsamp.tex|&main file\\
% |cdocsch1.tex|&include file for chapter 1\\
% |cdocsch2.tex|&include file for chapter 2\\
% |cdocspt3.tex|&include file for part 3\\
% |cdocspt4.tex|&include file for part 4\\
% |cdocsdrf.tex|&forwarding file for main file in draft mode\\
% |cdocsfi1.tex|&forwarding file for final version of chapter 1\\
% |cdocsfi2.tex|&forwarding file for final version of chapter 2\\
% \end{tabular}
% \end{center}
% Each of the eight files can be compiled directly by the \LaTeX{} compiler.
%
% %%%%%%%%%%%%%%%%%%%%%%%%%%%%%%%%%%%%%%
% \paragraph{Main File.}
%
% The main file is called |cdocsamp.tex|.
%
% Load the \textsf{childdoc} definitions and
% declare the filename for the main document:
%    \begin{macrocode}
\input{childdoc.def}
\childdocmain{}
%    \end{macrocode}

% Optional override for |\version| flag:
%    \begin{macrocode}
%%\ifchilddoc\else\providecommand{\version}{draft}\fi
%    \end{macrocode}

% Define the default values for the |\version| flag
% (|final| for the main file and |draft| for childs):
%    \begin{macrocode}
\ifchilddoc
\providecommand{\version}{draft}
\else
\providecommand{\version}{final}
\fi
%    \end{macrocode}

% Load the standard document class:
%    \begin{macrocode}
\documentclass[12pt]{article}
%    \end{macrocode}

% Start the document body:
%    \begin{macrocode}
\begin{document}
%    \end{macrocode}

% Declare a title page.
% Print title, part of document being processed and version flag:
%    \begin{macrocode}
\addtocounter{page}{-1}
\begin{center}
{\LARGE\bfseries{}childdoc example\par}
\vspace{1cm}
\ifchilddoc
\ifchilddocmanual part\else chapter\fi:
`\childdocname' of `\childdocjob'\par
\else
main document: `\childdocjob'\par
\fi
version: \version\par
\end{center}
\newpage
%    \end{macrocode}

% Manually include selected file,
% otherwise process as usual:
%    \begin{macrocode}
\ifchilddocmanual
\section*{part `\childdocname'}
\input{\childdocname}
\else
%    \end{macrocode}

% Include the two chapters:
%    \begin{macrocode}
\include{cdocsch1}
\include{cdocsch2}
%    \end{macrocode}

% Include the two parts unless only chapters should be displayed:
%    \begin{macrocode}
\ifchilddoc\else
\section{part three}
\input{cdocspt3}
\section{part four}
\input{cdocspt4}
\fi
%    \end{macrocode}

% Process as usual until here:
%    \begin{macrocode}
\fi
%    \end{macrocode}

% End of document body:
%    \begin{macrocode}
\end{document}
%    \end{macrocode}
%\iffalse
%</samplemain>
%\fi
%
% %%%%%%%%%%%%%%%%%%%%%%%%%%%%%%%%%%%%%%
% \paragraph{Chapter Include Files.}
%
% The include files are called |cdocsch1.tex| and |cdocsch2.tex|.
%
%\iffalse
%<*samplechap1|samplechap2>
%\fi

% Optional override for |\version| flag:
%    \begin{macrocode}
%%\providecommand{\version}{final}
%    \end{macrocode}

% Include the main document:
%    \begin{macrocode}
\input{childdoc.def}
\childdocof{cdocsamp}
%    \end{macrocode}

%\iffalse
%</samplechap1|samplechap2>
%\fi
%
%\iffalse
%<*samplechap1>
%\fi
% Some text for chapter 1:
%    \begin{macrocode}
\section{one}
some text in chapter one
%    \end{macrocode}

%\iffalse
%</samplechap1>
%\fi
% Some text for chapter 2:
%\iffalse
%<*samplechap2>
%\fi
%    \begin{macrocode}
\section{two}
more text in chapter two
%    \end{macrocode}

%\iffalse
%</samplechap2>
%\fi
%
% %%%%%%%%%%%%%%%%%%%%%%%%%%%%%%%%%%%%%%
% \paragraph{Part Include Files.}
%
% The include files are called |cdocspt3.tex| and |cdocspt4.tex|.
%
%\iffalse
%<*samplepart3|samplepart4>
%\fi

% Optional override for |\version| flag:
%    \begin{macrocode}
%%\providecommand{\version}{final}
%    \end{macrocode}

% Include the main document:
%    \begin{macrocode}
\input{childdoc.def}
\childdocby{cdocsamp}
%    \end{macrocode}

%\iffalse
%</samplepart3|samplepart4>
%\fi
%
%\iffalse
%<*samplepart3>
%\fi
% Some text for part 3:
%    \begin{macrocode}
some text in part three
%    \end{macrocode}

%\iffalse
%</samplepart3>
%\fi
% Some text for part 4:
%\iffalse
%<*samplepart4>
%\fi
%    \begin{macrocode}
more text in part four
%    \end{macrocode}

%\iffalse
%</samplepart4>
%\fi
%
% %%%%%%%%%%%%%%%%%%%%%%%%%%%%%%%%%%%%%%
% \paragraph{Forwarding for a Complete Draft.}
%
% The following forwarding file |cdocsdrf.tex|
% compiles the main document in draft mode:
%\iffalse
%<*sampledraft>
%\fi
%    \begin{macrocode}
\def\version{draft}
\input{childdoc.def}
\childdocforward{cdocsamp}
%    \end{macrocode}

%\iffalse
%</sampledraft>
%\fi
%
% %%%%%%%%%%%%%%%%%%%%%%%%%%%%%%%%%%%%%%
% \paragraph{Forwarding for Final Version of the Chapters.}
%
% The following forwarding files |cdocsfn1.tex| and |cdocsfn2.tex|
% (with identical content)
% compile the final versions of the child documents
% |cdocsch1.tex| and |cdocsch2.tex|, respectively:
%\iffalse
%<*samplefinal>
%\fi
%    \begin{macrocode}
\def\version{final}
\input{childdoc.def}
\childdocforwardprefix[cdocsamp]{cdocsfn}{cdocsch}
%    \end{macrocode}

%\iffalse
%</samplefinal>
%\fi
%
% %%%%%%%%%%%%%%%%%%%%%%%%%%%%%%%%%%%%%%
% \paragraph{Command Line Processing.}
%
% The following three command lines generate the output files
% |cdocscld|, |cdocscl1| and |cdocscl2|
% which should be identical to
% |cdocsdrf|, |cdocsch1| and |cdocsfn2|, respectively:
% \begin{center}
% \begin{tabular}{l}
% |latex -jobname cdocscld \|\\
% |  "\def\version{draft}\input{childdoc.def}\childdocforward{cdocsamp}"|\\
% |latex -jobname cdocscl1 \|\\
% |  "\input{childdoc.def}\childdocforward[cdocsamp]{cdocsch1}"|\\
% |latex -jobname cdocscl2 \|\\
% |  "\def\version{final}\input{childdoc.def}\childdocforward{cdocsch2}"|
% \end{tabular}
% \end{center}
% Note that the trailing backslash on each first line
% merely continues the input to the second line
% (for convenient cut ant paste).
% Furthermore, the command |latex| can be replaced by any
% of its alternative versions such as |pdflatex|.
%
% %%%%%%%%%%%%%%%%%%%%%%%%%%%%%%%%%%%%%%%%%%%%%%%%%%%%%%%%%%%%%%%%%%%%%%%%%%%%%%
% %%%%%%%%%%%%%%%%%%%%%%%%%%%%%%%%%%%%%%%%%%%%%%%%%%%%%%%%%%%%%%%%%%%%%%%%%%%%%%
% \section{Implementation}
%\iffalse
%<*package>
%\fi
%
% This section describes the definitions file |childdoc.def|.

% The definitions cannot be loaded using |\usepackage| or |\RequirePackage|
% which has a mechanism to prevent loading a style file more than once.
% When loading the definitions by means of |\input|
% multiple instances have to be prevented manually:
%\iffalse
%This code needs to be before the `\ProvidesFile' directive
%which is defined at the beginning of this file.
%Therefore it is also placed there and commented out here.
%</package>
%<*discard>
%\fi
%    \begin{macrocode}
\ifdefined\childdocmain\endinput\fi
%    \end{macrocode}
%\iffalse
%</discard>
%<*package>
%\fi
%
% \macro{\ifchilddoc}
% \macro{\ifchilddocmanual}
% The conditional |\ifchilddoc| tells whether a
% child (true) or main (false) document is being compiled.
% The conditional |\ifchilddocmanual| tells whether
% the |\includeonly| mechanism is used (false) or
% the selection of child files must be performed manually (true).
% The definitions initialise to false:
%    \begin{macrocode}
\newif\ifchilddoc
\newif\ifchilddocmanual
%    \end{macrocode}

% \macro{\childdocname}
% \macro{\childdocjob}
% The macro |\childdocname| stores the name of the main document
% to be compiled. The macro |\childdocjob| stores the name of
% the document on which the \LaTeX{} compiler was originally invoked.
% The content of |\jobname| cannot be compared
% to filenames specified in the source due to different catcodes.
% The following code rescans |\jobname|, stores the result
% in |\childdocname| and saves a copy in |\childdocjob|:
%    \begin{macrocode}
\edef\childdocname{\scantokens\expandafter{\jobname\noexpand}}
\let\childdocjob\childdocname
%    \end{macrocode}

% \macro{\childdocdisable}
% The macro |\childdocdisable| prevents the main file
% from being processed more than once.
% At this stage, the main document command |\childdocmain|
% is assumed to be called once again where it should do nothing.
% Any subsequent call to it should prevent
% a secondary processing of the main document
% It overwrites the forwarding commands
% |\childdocof| and |\childdocforward|
% with empty macros to prevent further inclusions of the main document:
%    \begin{macrocode}
\newcommand{\childdocdisable}
{
  \renewcommand{\childdocmain}[1]{\renewcommand{\childdocmain}[1]{\endinput}}
  \renewcommand{\childdocof}[1]{}
  \renewcommand{\childdocby}[2][]{}
  \renewcommand{\childdocforward}[2][]{}
  \renewcommand{\childdocdisable}{}
}
%    \end{macrocode}

% \macro{\childdocmain}
% The macro |\childdocmain| is to be called at the top of the main file
% with nothing or the main filename (without extension) as argument.
% First, it breaks loops.
% If the argument is not empty and does not match |\childdocname|
% (which is set by the first inclusion of |childdoc.def|),
% |\ifchilddoc| is set to true, |\includeonly| is applied to the child file
% and |\jobname| is set to the main file
% (for proper handling of |.aux| files):
%    \begin{macrocode}
\newcommand{\childdocmain}[1]
{
  \childdocdisable\childdocmain{}
  \if?#1?\else
    \begingroup
      \def\childdoctmp{#1}
      \ifx\childdoctmp\childdocname
        \def\childdoctmp{}
      \else
        \def\childdoctmp
        {
          \childdoctrue
          \includeonly{\childdocname}
          \def\childdocjob{#1}
          \def\jobname{#1}
        }
      \fi
      \expandafter
    \endgroup
    \childdoctmp
  \fi
}
%    \end{macrocode}

% \macro{\childdocof}
% The command |\childdocof| redirects
% compilation to the main file |#1|.
%    \begin{macrocode}
\newcommand{\childdocof}[1]
{
  \childdocdisable
  \childdoctrue
  \includeonly{\childdocname}
  \def\jobname{#1}
  \def\childdocjob{#1}
  \input{#1}
}
%    \end{macrocode}

% \macro{\childdocby}
% The command |\childdocby| ....
%    \begin{macrocode}
\newcommand{\childdocby}[2][]
{
  \childdocdisable
  \childdoctrue
  \childdocmanualtrue
  \if?#1?\else
    \def\jobname{#2}
  \fi
  \def\childdocjob{#2}
  \input{#2}
  \endinput
}
%    \end{macrocode}

% \macro{\childdocforward}
% The command |\childdocforward| redirects
% compilation to the main file or
% (if the optional argument is given) a child file.
% Parameters are set as if the main file
% or a child file starting with |\childdocof| was compiled.
% Then compilation is handed over to the main file:
%    \begin{macrocode}
\newcommand{\childdocforward}[2][]
{
  \begingroup
    \if?#1?
      \def\childdoctmp
      {
        \def\childdocname{#2}
        \def\childdocjob{#2}
        \def\jobname{#2}
        \input{#2}
        \endinput
      }
    \else
      \def\childdoctmp
      {
        \childdocdisable
        \def\childdocname{#2}
        \childdoctrue
        \includeonly{#2}
        \def\childdocjob{#1}
        \def\jobname{#1}
        \input{#1}
        \endinput
      }
    \fi
    \expandafter
  \endgroup
  \childdoctmp
}
%    \end{macrocode}

% \macro{\childdocforwardprefix}
% The command |\childdocforwardprefix| redirects
% compilation to the main or a child file by means of a pattern.
% The prefix |#1| in the current filename is replaced by |#2|
% and the suffix of the current filename is kept
% (it is assumed that the filename does not contain the substring `|~~~|'
% which is used as a delimiter).
% Compilation is handed over to the new file by |\childdocforward|:
%    \begin{macrocode}
\newcommand{\childdocforwardprefix}[3][]
{
  \begingroup
    \def\childdocextract #2##1~~~{\def\childdoctmp{\childdocforward[#1]{#3##1}}}
    \expandafter\childdocextract\childdocname~~~
    \expandafter
  \endgroup
  \childdoctmp
}
%    \end{macrocode}

% \macro{\childdoc}
% The deprecated macro |\childdoc| is a legacy version of |\childdocmain|:
%    \begin{macrocode}
\newcommand{\childdoc}{\childdocmain}
%    \end{macrocode}

% \macro{\childdocredirect}
% The deprecated macro |\childdocredirect| is a legacy version
% of |\childdocforward| and |\childdocforwardprefix|:
%    \begin{macrocode}
\newcommand{\childdocredirect}[2][]
{
  \begingroup
    \if?#1?
      \def\childdoctmp{\childdocforward{#2}}
    \else
      \def\childdoctmp{\childdocforwardprefix{#1}{#2}}
    \fi
    \expandafter
  \endgroup
  \childdoctmp
}
%    \end{macrocode}

%\iffalse
%</package>
%\fi
%
\endinput
\childdocforward[|\textit{main}|]{|\textit{dest}|}"|
\end{center}
%
Here \textit{target} is the name of the output file,
\textit{main} is the name of the main file
and \textit{dest} is the name of the main or child file to be processed
(all filenames without extensions).
The optional argument \textit{main} can be omitted
if \textit{main} matches \textit{dest}.
Optionally, compilation \textit{flags} can be defined via |\def| commands.
This command line makes the \TeX{} engine believe
it is compiling the file \textit{target}
whose content is specified as the latter parameter.
The provided code then forwards the processing to
\textit{main} or \textit{dest} as described in \secref{sec:forward}.

%%%%%%%%%%%%%%%%%%%%%%%%%%%%%%%%%%%%%%%%%%%%%%%%%%%%%%%%%%%%%%%%%%%%%%%%%%%%%%%%
\subsection{Include by Input}
\label{sec:input}

Including child documents by |\include| has some restrictions by design.
Most notably, the content of a child document always occupies
its own set of pages; pages cannot be shared between child documents.
Usually, this behaviour makes perfect sense
because each child document contain an essential part of the document.
However, in some situations it may be desirable to compose
a document from a collection of parts
without having mandatory page breaks between then.
For this case, the package
provides a mechanism to include parts
by |\input| which can also be processed individually.
However, by construction this mechanism
requires manual handling of the content to be output.

%%%%%%%%%%%%%%%%%%%%%%%%%%%%%%%%%%%%%%%%
\DescribeMacro{\ifchilddocmanual}
The main file should be prepared as usual, see \secref{sec:include}.
However, the document body must make a distinction
between processing of an individual part and of the main document, e.g.:
%
\begin{center}
\begin{tabular}{l}
|\ifchilddocmanual|\\
|\input{\childdocname}|\\
|\||else|\\
\textit{document body with }|\input{|\textit{part}|}|\\
|\||fi|
\end{tabular}
\end{center}
%
The conditional |\ifchilddocmanual| is true whenever
a part to be included by |\input| is being compiled,
and the name of the part is stored in |\childdocname|.

%%%%%%%%%%%%%%%%%%%%%%%%%%%%%%%%%%%%%%%%
\DescribeMacro{\childdocby}
Each part to be included by |\input| should start with:
%
\begin{center}
\begin{tabular}{l}
|% \iffalse
%
% childdoc.dtx Copyright (C) 2017-2018 Niklas Beisert
%
% This work may be distributed and/or modified under the
% conditions of the LaTeX Project Public License, either version 1.3
% of this license or (at your option) any later version.
% The latest version of this license is in
%   http://www.latex-project.org/lppl.txt
% and version 1.3 or later is part of all distributions of LaTeX
% version 2005/12/01 or later.
%
% This work has the LPPL maintenance status `maintained'.
%
% The Current Maintainer of this work is Niklas Beisert.
%
% This work consists of the files childdoc.dtx and childdoc.ins
% and the derived files childdoc.def and cdocsamp.tex with
% cdocsch1.tex, cdocsch2.tex, cdocsdrf.tex, cdocsfn1.tex, cdocsfn2.tex.
%
%<package>\ifdefined\childdocmain\endinput\fi
%<package>\ProvidesFile{childdoc.def}[2018/12/30 v2.0 child document driver]
%<samplemain>\ProvidesFile{cdocsamp.tex}[2018/12/30 v2.0 sample for childdoc]
%<*driver>
%\ProvidesFile{childdoc.drv}[2018/12/30 v2.0 childdoc reference manual file]
\PassOptionsToClass{10pt,a4paper}{article}
\documentclass{ltxdoc}

\usepackage[margin=35mm]{geometry}
\usepackage{hyperref}
\usepackage{hyperxmp}
\usepackage[usenames]{color}

\hypersetup{colorlinks=true}
\hypersetup{pdfstartview=FitH}
\hypersetup{pdfpagemode=UseNone}
\hypersetup{pdfsource={}}
\hypersetup{pdflang={en-UK}}
\hypersetup{pdfcopyright={Copyright 2017-2018 Niklas Beisert.
  This work may be distributed and/or modified under the
  conditions of the LaTeX Project Public License, either version 1.3
  of this license or (at your option) any later version.}}
\hypersetup{pdflicenseurl={http://www.latex-project.org/lppl.txt}}
\hypersetup{pdfcontactaddress={ETH Zurich, ITP, HIT K,
  Wolfgang-Pauli-Strasse 27}}
\hypersetup{pdfcontactpostcode={8093}}
\hypersetup{pdfcontactcity={Zurich}}
\hypersetup{pdfcontactcountry={Switzerland}}
\hypersetup{pdfcontactemail={nbeisert@itp.phys.ethz.ch}}
\hypersetup{pdfcontacturl={http://people.phys.ethz.ch/\xmptilde nbeisert/}}

\newcommand{\secref}[1]{\hyperref[#1]{section \ref*{#1}}}

\parskip1ex
\parindent0pt
\let\olditemize\itemize
\def\itemize{\olditemize\parskip0pt}

\begin{document}

\title{The \textsf{childdoc} Package}
\hypersetup{pdftitle={The childdoc Package}}
\author{Niklas Beisert\\[2ex]
  Institut f\"ur Theoretische Physik\\
  Eidgen\"ossische Technische Hochschule Z\"urich\\
  Wolfgang-Pauli-Strasse 27, 8093 Z\"urich, Switzerland\\[1ex]
  \href{mailto:nbeisert@itp.phys.ethz.ch}
  {\texttt{nbeisert@itp.phys.ethz.ch}}}
\hypersetup{pdfauthor={Niklas Beisert}}
\hypersetup{pdfsubject={Manual for the LaTeX2e Package childdoc}}
\date{30 December 2018, \textsf{v2.0}}
\maketitle

\begin{abstract}\noindent
\textsf{childdoc} is a \LaTeXe{} package
that enables the direct compilation
of document sections included by |\include|
to individual files.
\end{abstract}

\begingroup
\parskip0ex
\tableofcontents
\endgroup

%%%%%%%%%%%%%%%%%%%%%%%%%%%%%%%%%%%%%%%%%%%%%%%%%%%%%%%%%%%%%%%%%%%%%%%%%%%%%%%%
%%%%%%%%%%%%%%%%%%%%%%%%%%%%%%%%%%%%%%%%%%%%%%%%%%%%%%%%%%%%%%%%%%%%%%%%%%%%%%%%
\section{Introduction}

\LaTeX{} provides a mechanism to structure a large document (such as a book)
into a main file and several child files (containing the chapters)
using the |\include| command.
This mechanism is beneficial for documents
which span hundreds of pages in order to
make the source file(s) more manageable.
Moreover, compilation can be restricted to
selected child files by means of the |\includeonly| command.
The latter feature can be used to reduce the compilation time while editing
(this was significantly more useful in the earlier days of \LaTeX{})
or to generate a smaller document which is easier to navigate.
Another application of |\includeonly| is to generate
documents consisting of selected parts of the complete document.

However, there are a few drawbacks of the plain |\include| mechanism:
\begin{itemize}
\item
The child files cannot be compiled on their own,
they can only be compiled via the main file.
A naive editing environment
(such as a text editor with an option
to have the current file processed by \LaTeX)
may require one to switch to the main file before compiling;
attempting to compile the child file produces errors.
\item
The main file must be modified (each time)
to adjust the |\includeonly| command
to the present needs. This easily leaves the main file in a messy state.
\item
The generated document will always carry the filename
of the main document. This is inconvenient if
several child files are to be compiled and
to be kept for distribution.
\end{itemize}

The present package provides a simple interface
to make child files individually compilable by \LaTeX{}.
Compiling a child file then has the same effect as compiling
the main file with an |\includeonly| command
to select the appropriate child.
Moreover the generated document will carry the name of the child
rather than the main file.
This resolves all three above issues.

This feature is meant to make the editing of books,
thesis documents and lecture notes somewhat more convenient.
However, the package can also be used efficiently for
composing a series of documents (such as exercise sheets)
which are typically distributed individually.
It then assists the author in generating the individual documents
(potentially in different versions)
as well as a document containing the collected series.
Another application is in developing style files
or other kinds of included material
where compilation of the style file could redirect
to a sample or test file.

%%%%%%%%%%%%%%%%%%%%%%%%%%%%%%%%%%%%%%%%%%%%%%%%%%%%%%%%%%%%%%%%%%%%%%%%%%%%%%%%
%%%%%%%%%%%%%%%%%%%%%%%%%%%%%%%%%%%%%%%%%%%%%%%%%%%%%%%%%%%%%%%%%%%%%%%%%%%%%%%%
\section{Usage}

First of all, the package \textsf{childdoc} is \emph{not} a standard
\LaTeXe{} |.sty| style file! Therefore it needs to be invoked in
a non-standard way.

%%%%%%%%%%%%%%%%%%%%%%%%%%%%%%%%%%%%%%%%%%%%%%%%%%%%%%%%%%%%%%%%%%%%%%%%%%%%%%%%
\subsection{Included Files}
\label{sec:include}

%%%%%%%%%%%%%%%%%%%%%%%%%%%%%%%%%%%%%%%%
\DescribeMacro{\childdocmain}
To use the package, add the commands
\begin{center}
\begin{tabular}{l}
|\input{childdoc.def}|\\
|\childdocmain{}|\\
\end{tabular}
\end{center}
at the very top of the main \LaTeX{} file,
in particular \emph{before} the |\documentclass| statement!
The argument of |\childdocmain| should be left empty
(but it must be present).

%%%%%%%%%%%%%%%%%%%%%%%%%%%%%%%%%%%%%%%%
\DescribeMacro{\childdocof}
Furthermore, add the commands
\begin{center}
\begin{tabular}{l}
|\input{childdoc.def}|\\
|\childdocof{|\textit{main}|}|\\
\end{tabular}
\end{center}
at the top of every child file \textit{child}
which is included by |\include{|\textit{child}|}|
from within the main file
(or at least for those files to be compiled individually).
The argument \textit{main} must be the filename of the main file.

There are a couple of
considerations in setting up the main and child documents:

%%%%%%%%%%%%%%%%%%%%%%%%%%%%%%%%%%%%%%%%
\paragraph{Restrictions.}

Please note the following restrictions:
\begin{itemize}
\item
|\childdocmain| must be called with one argument \textit{main}
to ensure compatibility with earlier version of the package.
It must either be empty (|\childdocmain{}|)
or precisely match the filename of the main file in which it is specified.
See \secref{sec:detection} for further information.
\item
The filename \textit{main} must be specified without the |.tex| extension.
\item
The filename \textit{main} is case sensitive
(even in case-insensitive file systems)
due to internal string comparison.
\item
The argument \textit{main} should be fully expanded, it cannot be a macro.
\item
Subdirectories and special characters should be avoided in filenames.
\item
The command |\childdocmain{|\textit{main}|}| must be followed by a whitespace.
It should not be followed immediately by another command
or by a comment mark `|%|'.
This is because the \TeX{} parser reads the token immediately following
the argument of |\childdocmain| and puts it
at the beginning of every child section;
however, a white\-space is ignored.
\end{itemize}

%%%%%%%%%%%%%%%%%%%%%%%%%%%%%%%%%%%%%%%%
\paragraph{Content of Main File.}

It is advisable to place all content in the child files included by |\include|.
Any output contained in the main file will appear in all child documents
unless suppressed manually;
it cannot be suppressed automatically by the |\includeonly| directive
and thus should normally be avoided.
A method to include some content in the main file
by means of conditional processing is described in \secref{sec:conditional}.

%%%%%%%%%%%%%%%%%%%%%%%%%%%%%%%%%%%%%%%%
\paragraph{Page Numbering.}

When only a part of the document is compiled,
the appropriate numbering of pages
(as well as other status parameters)
is determined from the |.aux| files.
The latter contain information from previous passes.
However this information needs to propagate through
all intermediate child documents.
Therefore the page numbering in child documents may well
be inconsistent until the complete document is compiled at least once.

A useful (if unconventional) way to always ensure a consistent
page numbering is to restart the numbering in each child document
and denote the pages by `\textit{child}|.|\textit{page}'
where \textit{child} represents the chapter/section number of the child file.
This can be achieved by the command
|\numberwithin{page}{|\textit{child}|}|
of the \textsf{amsmath} package
where \textit{child} can be |chapter| or |section|
depending on the chosen structuring.
Alternatively, one can modify the macro |\thepage| appropriately
and reset the counter |page| at the start of each child file.

%%%%%%%%%%%%%%%%%%%%%%%%%%%%%%%%%%%%%%%%%%%%%%%%%%%%%%%%%%%%%%%%%%%%%%%%%%%%%%%%
\subsection{Conditional Processing}
\label{sec:conditional}

The package provides a mechanism to compile different versions
of a document. To customise the versions further some conditional processing
can come in handy to distinguish which version is being compiled.
The package provides two macros to describe the compilation context:

%%%%%%%%%%%%%%%%%%%%%%%%%%%%%%%%%%%%%%%%
\DescribeMacro{\ifchilddoc}
The conditional |\ifchilddoc| distinguishes between the compilation of
child documents and the main document:
%
\begin{center}
|\ifchilddoc |\textit{child-code}| |[|\||else |\textit{main-code}]| \||fi|
\end{center}

%%%%%%%%%%%%%%%%%%%%%%%%%%%%%%%%%%%%%%%%
\DescribeMacro{\childdocname}
\DescribeMacro{\childdocjob}
The macro |\childdocname| contains the filename (without extension)
of the main or child file being processed.
Note that |\childdocjob| will always contain the name of the main file.

%%%%%%%%%%%%%%%%%%%%%%%%%%%%%%%%%%%%%%%%
\paragraph{Title Page.}

Conditional processing can be used to include a title or banner page
in the main document when proper precautions are taken.
Importantly, the code in the main file should ensure that the page counter
(as well as other status parameters which are stored in the |.aux| files)
takes the same value after the conditional processing.
Otherwise the page numbers may take divergent values
depending on which part is compiled.

For example, a title page could be declared by:
%
\begin{center}
\begin{tabular}{l}
|\ifchilddoc\||else|\\
|\addtocounter{page}{-1}|\\
\textit{code for title page}\\
|\newpage|\\
|\||fi|
\end{tabular}
\end{center}
%
A banner page for the child documents can be generated by:
%
\begin{center}
\begin{tabular}{l}
|\ifchilddoc|\\
|\addtocounter{page}{-1}|\\
\textit{code for banner page}\\
|\newpage|\\
|\||fi|
\end{tabular}
\end{center}
%
Here one could write a message such as:
\begin{center}
|This is the part \childdocname{} of \childdocjob{}.|
\end{center}

%%%%%%%%%%%%%%%%%%%%%%%%%%%%%%%%%%%%%%%%%%%%%%%%%%%%%%%%%%%%%%%%%%%%%%%%%%%%%%%%
\subsection{Flags}
\label{sec:flags}

The package makes it easy to generate different versions
of the main or child documents.
To this end compilation flags can be defined
and assigned different default values.
They will be particularly useful in conjunction
with the forwarding mechanism described in \secref{sec:forward}.

For example, it may be useful to have a flag |\version|
which can be set to |draft| or |final|.
The document source will contain some conditional code
depending on the value of |\version|.
Suppose further, the flag should default to |final| for the main file
and to |draft| for child files
which is a natural assignment for editing the document.
This is achieved by placing the following code
in the preamble of the main document
(below the |\childdocmain| directive):
%
\begin{center}
\begin{tabular}{l}
|\ifchilddoc|\\
|\providecommand{\version}{draft}|\\
|\||else|\\
|\providecommand{\version}{final}|\\
|\||fi|
\end{tabular}
\end{center}
%
The definition by |\providecommand| makes sure
that previous definitions are not overwritten.
Further statements |\providecommand{\version}{...}|
can thus be added before the above code to override it.

For the main file, one might add a line
(between |\childdocmain| and the above block)
%
\begin{center}
|%\ifchilddoc\||else\providecommand{\version}{draft}\||fi|
\end{center}
%
which can be uncommented to produce a draft version.
Likewise one can add a line to the very top of a child file
(above the |\childdocof{|\textit{main}|}| directive)
%
\begin{center}
|%\providecommand{\version}{final}|
\end{center}
%
which can be uncommented to produce the final version of this child document.

%%%%%%%%%%%%%%%%%%%%%%%%%%%%%%%%%%%%%%%%%%%%%%%%%%%%%%%%%%%%%%%%%%%%%%%%%%%%%%%%
\subsection{Forwarding}
\label{sec:forward}

Different versions of the main or child documents
using compilation flags as described in \secref{sec:flags}
can be (permanently) stored in different files
for convenient compilation, viewing and distribution.
To this end, the package defines a command
to pass on compilation to a different file:

%%%%%%%%%%%%%%%%%%%%%%%%%%%%%%%%%%%%%%%%
\DescribeMacro{\childdocforward}
The command |\childdocforward| redirects processing to
another source file:
%
\begin{center}
\begin{tabular}{l}
|\input{childdoc.def}|\\
|\childdocforward[|\textit{main}|]{|\textit{dest}|}|\\
\end{tabular}
\end{center}
%
The argument \textit{dest} is the destination file
(without extension).
It should be the main file or one of the child files.
Note that further \textsf{childdoc} directives
such as |\childdocof| and |\childdocforward|
in the indicated file will be processed in this form.
The optional argument \textit{main}
passes on directly to the main file \textit{main}
while pretending to compile the child \textit{dest}.
This form behaves as if \textit{dest}
issues |\childdocof{|\textit{main}|}| right away,
and no further \textsf{childdoc} directives will be processed.

%%%%%%%%%%%%%%%%%%%%%%%%%%%%%%%%%%%%%%%%
\DescribeMacro{\...prefix}
In the alternative form |\childdocforwardprefix|,
%
\begin{center}
\begin{tabular}{l}
|\input{childdoc.def}|\\
|\childdocforwardprefix[|\textit{main}|]{|\textit{prefix}|}{|\textit{dest}|}|
\end{tabular}
\end{center}
%
the destination file is determined by a pattern
depending on the current file:
To make this work, the current file must be called
`{\textit{prefix}\hspace{0.2em}\textit{suffix}}'
with \textit{prefix} matching precisely the argument.
Processing is then passed on to the file
`{\textit{dest}\hspace{0.2em}\textit{suffix}}'.
Surely, the same effect is achieved by
directly specifying the
argument `{\textit{dest}\hspace{0.2em}\textit{suffix}}'
in the first form.
However, that requires to set up a different file
for each child. With the alternative form of the command
all these files can have exactly the same content
which simplifies setting them up and maintaining them.

For example, the following file |draft.tex|
with a compilation flag |\version| as described in \secref{sec:flags}
compiles the main document as a draft:
%
\begin{center}
\begin{tabular}{l}
|\def\version{draft}|\\
|\input{childdoc.def}|\\
|\childdocforward{|\textit{main}|}|
\end{tabular}
\end{center}
%
Likewise, the following files |final|\textit{nn}|.tex|
compile the final version of the child document
|child|\textit{nn}|.tex|:
%
\begin{center}
\begin{tabular}{l}
|\def\version{final}|\\
|\input{childdoc.def}|\\
|\childdocforwardprefix{final}{child}|
\end{tabular}
\end{center}
%

Note that when several versions of a main file and/or of each child file
are to be generated, it may be convenient to set up a |Makefile| or
shell script to automatise the process.

%%%%%%%%%%%%%%%%%%%%%%%%%%%%%%%%%%%%%%%%%%%%%%%%%%%%%%%%%%%%%%%%%%%%%%%%%%%%%%%%
\subsection{Command Line Processing}
\label{sec:commandline}

The effect of redirection files can also be achieved by invoking
the \LaTeX{} compiler with a more elaborate command line.
Most conveniently this should be done as part
of a shell script or a |Makefile|.

When using \textsf{childdoc} in the main file, the following
command lines effectively perform a redirection
(note that depending on the shell being used,
backslashes may have to be doubled: `|\|' $\to$ `|\\|'):
%
\begin{center}
|... -jobname "|\textit{target}|" |\\|"|[\textit{flags}]%
|\input{childdoc.def}\childdocforward[|\textit{main}|]{|\textit{dest}|}"|
\end{center}
%
Here \textit{target} is the name of the output file,
\textit{main} is the name of the main file
and \textit{dest} is the name of the main or child file to be processed
(all filenames without extensions).
The optional argument \textit{main} can be omitted
if \textit{main} matches \textit{dest}.
Optionally, compilation \textit{flags} can be defined via |\def| commands.
This command line makes the \TeX{} engine believe
it is compiling the file \textit{target}
whose content is specified as the latter parameter.
The provided code then forwards the processing to
\textit{main} or \textit{dest} as described in \secref{sec:forward}.

%%%%%%%%%%%%%%%%%%%%%%%%%%%%%%%%%%%%%%%%%%%%%%%%%%%%%%%%%%%%%%%%%%%%%%%%%%%%%%%%
\subsection{Include by Input}
\label{sec:input}

Including child documents by |\include| has some restrictions by design.
Most notably, the content of a child document always occupies
its own set of pages; pages cannot be shared between child documents.
Usually, this behaviour makes perfect sense
because each child document contain an essential part of the document.
However, in some situations it may be desirable to compose
a document from a collection of parts
without having mandatory page breaks between then.
For this case, the package
provides a mechanism to include parts
by |\input| which can also be processed individually.
However, by construction this mechanism
requires manual handling of the content to be output.

%%%%%%%%%%%%%%%%%%%%%%%%%%%%%%%%%%%%%%%%
\DescribeMacro{\ifchilddocmanual}
The main file should be prepared as usual, see \secref{sec:include}.
However, the document body must make a distinction
between processing of an individual part and of the main document, e.g.:
%
\begin{center}
\begin{tabular}{l}
|\ifchilddocmanual|\\
|\input{\childdocname}|\\
|\||else|\\
\textit{document body with }|\input{|\textit{part}|}|\\
|\||fi|
\end{tabular}
\end{center}
%
The conditional |\ifchilddocmanual| is true whenever
a part to be included by |\input| is being compiled,
and the name of the part is stored in |\childdocname|.

%%%%%%%%%%%%%%%%%%%%%%%%%%%%%%%%%%%%%%%%
\DescribeMacro{\childdocby}
Each part to be included by |\input| should start with:
%
\begin{center}
\begin{tabular}{l}
|\input{childdoc.def}|\\
|\childdocby{|\textit{main}|}|\\
\end{tabular}
\end{center}
%
The directive |\childdocby| is similar to |\childdocof|
described in \secref{sec:include},
but the subsequent selection of content must be done manually.
To that end, both |\ifchilddoc| and |\ifchilddocmanual|
will be true upon processing of a part,
and the name of the part is stored in |\childdocname|.
Note that |\jobname| will be set to the filename of the current part
so that each part receives an individual |.aux| file
that does not interfere with the |.aux| file(s) of the main document.
This behaviour can be altered by the alternative form
|\childdocby[*]{|\textit{main}|}| (with a non-empty optional argument)
which uses the |.aux| file of the main document
by setting |\jobname| to \textit{main}.

%%%%%%%%%%%%%%%%%%%%%%%%%%%%%%%%%%%%%%%%%%%%%%%%%%%%%%%%%%%%%%%%%%%%%%%%%%%%%%%%
\subsection{Driver Development}
\label{sec:driver}

The \textsf{childdoc} mechanism can also be use for the development
of definition files such as \LaTeX{} styles or classes.
This case differs from the above setup with multiple parts
included by |\include| in that no |\includeonly| should be invoked.
This can be achieved by starting the include file
(before |\ProvidesPackage|) with:
%
\begin{center}
\begin{tabular}{l}
|\input{childdoc.def}|\\
|\childdocforward{|\textit{main}|}|\\
\end{tabular}
\end{center}
%
or alternatively with:
%
\begin{center}
\begin{tabular}{l}
|\input{childdoc.def}|\\
|\childdocby{|\textit{main}|}|\\
\end{tabular}
\end{center}
%
Both forms have slightly different effects as described above.
The main file is prepared as usual, see \secref{sec:include}.

%%%%%%%%%%%%%%%%%%%%%%%%%%%%%%%%%%%%%%%%%%%%%%%%%%%%%%%%%%%%%%%%%%%%%%%%%%%%%%%%
\subsection{Legacy Detection}
\label{sec:detection}

The directive |\childdocmain| in the main file can detect
whether the complete document or merely a child is to be compiled
even without using the directive |\childdocof|.
This method is deprecated because it is less robust
and there is no compelling reason to use it;
it is merely provided for backward compatibility
and it may be removed in future versions.

If the detection mechanism is to be used,
it is mandatory to correctly specify
the filename of the main file as the argument of |\childdocmain|:
%
\begin{center}
\begin{tabular}{l}
|\input{childdoc.def}|\\
|\childdocmain{|\textit{main}|}|\\
\end{tabular}
\end{center}
%
If |\jobname| does not match the argument \textit{main} of |\childdocmain|,
it is assumed that |\jobname| points to the child file to be compiled.
When using |\childdocmain| with the main file specified as argument,
it suffices to start a child file
with just |\input{|\textit{main}|}|
without loading of the package and using |\childdocof|.
If instead all processing is done
with the appropriate \textsf{childdoc} directives,
the argument of \textit{main} of |\childdocmain| can be empty.

An alternative version of the command line processing described
in \secref{sec:commandline} using the detection mechanism reads:
%
\begin{center}
|... -jobname "|\textit{target}|" "|[\textit{flags}]%
[|\def\jobname{|\textit{dest}|}|]|\input{|\textit{main}|}"|
\end{center}

%%%%%%%%%%%%%%%%%%%%%%%%%%%%%%%%%%%%%%%%%%%%%%%%%%%%%%%%%%%%%%%%%%%%%%%%%%%%%%%%
\subsection{Manual Code}
\label{sec:manual}

In case one cannot be certain whether the definitions file |childdoc.def|
is installed on the target \TeX{} distribution
and one prefers not to ship it,
it is conceivable to paste a few relevant commands into the sources.

To that end, drop all statements |\input{childdoc.def}|
and perform the replacements as outlined below.
Instead of |\childdocmain{|\textit{main}|}| add the following code
to the top of the main file:
%
\begin{center}
\begin{tabular}{l}
|\||ifdefined\childdocname\endinput\||fi\newif\ifchilddoc|\\
|\edef\childdocname{\scantokens\expandafter{\jobname\noexpand}}|\\
|\def\childdocmain{|\textit{main}|}\||ifx\childdocmain\childdocname\||else|\\
|\childdoctrue\includeonly{\childdocname}\let\jobname\childdocmain\||fi|\\
\end{tabular}
\end{center}
%
Instead of |\childdocof{|\textit{main}|}| just include the main file
at the top of each child file:
%
\begin{center}
|\input{|\textit{main}|}|
\end{center}
%
A simple redirection |\childdocforward{|\textit{dest}|}| is achieved by:
%
\begin{center}
|\def\jobname{|\textit{dest}|}\input{\jobname}|
\end{center}
%
The redirection with prefix
|\childdocforwardprefix[|\textit{prefix}|]{|\textit{dest}|}|
is accomplished by:
%
\begin{center}
\begin{tabular}{l}
|{\edef\jobname{\scantokens\expandafter{\jobname\noexpand}}|\\
|\def\redirectjob |\textit{prefix}|#1~~~{\gdef\jobname{|\textit{dest}|#1}}|\\
|\expandafter\redirectjob\jobname~~~}\input{\jobname}|
\end{tabular}
\end{center}

In an alternative approach,
child documents can be compiled by a specific command line
without additional code or specific definitions:
%
\begin{center}
|... -jobname "|\textit{target}|" "|[\textit{flags}]%
|\includeonly{|\textit{dest}|}\input{|\textit{main}|}"|
\end{center}
%

%%%%%%%%%%%%%%%%%%%%%%%%%%%%%%%%%%%%%%%%%%%%%%%%%%%%%%%%%%%%%%%%%%%%%%%%%%%%%%%%
%%%%%%%%%%%%%%%%%%%%%%%%%%%%%%%%%%%%%%%%%%%%%%%%%%%%%%%%%%%%%%%%%%%%%%%%%%%%%%%%
\section{Information}

%%%%%%%%%%%%%%%%%%%%%%%%%%%%%%%%%%%%%%%%%%%%%%%%%%%%%%%%%%%%%%%%%%%%%%%%%%%%%%%%
\subsection{Copyright}

Copyright \copyright{} 2017--2018 Niklas Beisert

This work may be distributed and/or modified under the
conditions of the \LaTeX{} Project Public License, either version 1.3
of this license or (at your option) any later version.
The latest version of this license is in
  \url{http://www.latex-project.org/lppl.txt}
and version 1.3 or later is part of all distributions of \LaTeX{}
version 2005/12/01 or later.

This work has the LPPL maintenance status `maintained'.

The Current Maintainer of this work is Niklas Beisert.

This work consists of the files |README.txt|, |childdoc.ins| and |childdoc.dtx|
as well as the derived files |childdoc.def|, |cdocsamp.tex|
with |cdocsch1.tex|, |cdocsch2.tex|, |cdocspt3.tex|, |cdocspt4.tex|,
|cdocsdrf.tex|, |cdocsfn1.tex|, |cdocsfn2.tex|
as well as |childdoc.pdf|.

%%%%%%%%%%%%%%%%%%%%%%%%%%%%%%%%%%%%%%%%%%%%%%%%%%%%%%%%%%%%%%%%%%%%%%%%%%%%%%%%
\subsection{Files and Installation}

The package consists of the files:
%
\begin{center}
\begin{tabular}{ll}
    |README.txt|   & readme file \\
    |childdoc.ins| & installation file \\
    |childdoc.dtx| & source file \\
    |childdoc.def| & definition file \\
    |cdocsamp.tex| & sample main file \\
    |cdocsch1.tex| & sample include file \\
    |cdocsch2.tex| & sample include file \\
    |cdocspt3.tex| & sample part file \\
    |cdocspt4.tex| & sample part file \\
    |cdocsdrf.tex| & sample redirection file \\
    |cdocsfn1.tex| & sample redirection file \\
    |cdocsfn2.tex| & sample redirection file \\
    |childdoc.pdf| & manual
\end{tabular}
\end{center}
%
The distribution consists of the files
|README.txt|, |childdoc.ins| and |childdoc.dtx|.
%
\begin{itemize}
\item
Run (pdf)\LaTeX{} on |childdoc.dtx|
to compile the manual |childdoc.pdf| (this file).
\item
Run \LaTeX{} on |childdoc.ins| to create the definitions file |childdoc.def|
and the sample |cdocsamp.tex| with include files
|cdocsch1.tex|, |cdocsch2.tex|, |cdocspt3.tex|, |cdocspt4.tex|,
|cdocsdrf.tex|, |cdocsfn1.tex|, |cdocsfn2.tex|.
Then copy the file |childdoc.def| to an appropriate directory of your \LaTeX{}
distribution, e.g.\ \textit{texmf-root}|/tex/latex/childdoc|.
\end{itemize}

%%%%%%%%%%%%%%%%%%%%%%%%%%%%%%%%%%%%%%%%%%%%%%%%%%%%%%%%%%%%%%%%%%%%%%%%%%%%%%%%
\subsection{Related CTAN Packages}

There are several other packages which offer a similar functionality:
%
\begin{itemize}
\item
The packages
\href{http://ctan.org/pkg/docmute}{\textsf{docmute}},
\href{http://ctan.org/pkg/includex}{\textsf{includex}} and
\href{http://ctan.org/pkg/standalone}{\textsf{standalone}}
provide commands to include only the document body of
a child file thus allowing both files to be compiled individually.
\item
The packages \href{http://ctan.org/pkg/subdocs}{\textsf{subdocs}}
and \href{http://ctan.org/pkg/subfiles}{\textsf{subfiles}}
provide structures in which the main and child documents can be
encapsulated and allowing them to be compiled individually.
The inclusion mechanism is different from the conventional |\include|.
\item
The package \href{http://ctan.org/pkg/combine}{\textsf{combine}}
is an elaborate solution to combine several documents into one.
\end{itemize}
%
See also the CTAN topic \href{http://ctan.org/topic/subdocs}{\textsf{subdocs}}
for further related packages.
The present package differs from the above solutions in that
a document structure constructed with the conventional |\include| mechanism
just needs two extra commands at the top of every file
such that all constituent files can be compiled individually.

%%%%%%%%%%%%%%%%%%%%%%%%%%%%%%%%%%%%%%%%%%%%%%%%%%%%%%%%%%%%%%%%%%%%%%%%%%%%%%%%
%\subsection{Feature Suggestions}
%
%The following is a list of features which may be useful for future
%versions of this package:
%%
%\begin{itemize}
%\item
%\ldots
%\end{itemize}

%%%%%%%%%%%%%%%%%%%%%%%%%%%%%%%%%%%%%%%%%%%%%%%%%%%%%%%%%%%%%%%%%%%%%%%%%%%%%%%%
\subsection{Revision History}

%%%%%%%%%%%%%%%%%%%%%%%%%%%%%%%%%%%%%%%%
\paragraph{v2.0:} 2018/12/30

\begin{itemize}
\item
immediate forward processing
\item
added |\childdocby| mechanism
\item
manual restructured
\end{itemize}

%%%%%%%%%%%%%%%%%%%%%%%%%%%%%%%%%%%%%%%%
\paragraph{v1.6:} 2018/01/17

\begin{itemize}
\item
application for development of include files
\item
corrections to manual
\end{itemize}

%%%%%%%%%%%%%%%%%%%%%%%%%%%%%%%%%%%%%%%%
\paragraph{v1.5:} 2017/05/21

\begin{itemize}
\item
more complete structuring introduced
\item
|\childdocof| introduced
\item
|\childdoc| renamed to |\childdocmain|
\item
|\childredirect| renamed to |\childdocforward| and |\childdocforwardprefix|
and functionality expanded
\end{itemize}

%%%%%%%%%%%%%%%%%%%%%%%%%%%%%%%%%%%%%%%%
\paragraph{v1.0:} 2017/04/27

\begin{itemize}
\item
manual and install package
\item
first version published on CTAN
\end{itemize}

%%%%%%%%%%%%%%%%%%%%%%%%%%%%%%%%%%%%%%%%
\paragraph{v0.6:} 2017/04/26

\begin{itemize}
\item
redirection mechanism added
\end{itemize}

%%%%%%%%%%%%%%%%%%%%%%%%%%%%%%%%%%%%%%%%
\paragraph{v0.5:} 2017/04/26

\begin{itemize}
\item
functionality in definition file
\end{itemize}


%%%%%%%%%%%%%%%%%%%%%%%%%%%%%%%%%%%%%%%%%%%%%%%%%%%%%%%%%%%%%%%%%%%%%%%%%%%%%%%%
%%%%%%%%%%%%%%%%%%%%%%%%%%%%%%%%%%%%%%%%%%%%%%%%%%%%%%%%%%%%%%%%%%%%%%%%%%%%%%%%
%%%%%%%%%%%%%%%%%%%%%%%%%%%%%%%%%%%%%%%%%%%%%%%%%%%%%%%%%%%%%%%%%%%%%%%%%%%%%%%%
\appendix

\settowidth\MacroIndent{\rmfamily\scriptsize 000\ }

 \DocInput{childdoc.dtx}

\end{document}
%</driver>
% \fi
%
% %%%%%%%%%%%%%%%%%%%%%%%%%%%%%%%%%%%%%%%%%%%%%%%%%%%%%%%%%%%%%%%%%%%%%%%%%%%%%%
% %%%%%%%%%%%%%%%%%%%%%%%%%%%%%%%%%%%%%%%%%%%%%%%%%%%%%%%%%%%%%%%%%%%%%%%%%%%%%%
% \section{Sample}
%\iffalse
%<*samplemain>
%\fi
%
% The following presents a sample document
% with two chapters, two parts, a title page,
% a compile flag as well as three forwarding files to set the flag.
% It consists of eight |.tex| files:
% \begin{center}
% \begin{tabular}{ll}
% |cdocsamp.tex|&main file\\
% |cdocsch1.tex|&include file for chapter 1\\
% |cdocsch2.tex|&include file for chapter 2\\
% |cdocspt3.tex|&include file for part 3\\
% |cdocspt4.tex|&include file for part 4\\
% |cdocsdrf.tex|&forwarding file for main file in draft mode\\
% |cdocsfi1.tex|&forwarding file for final version of chapter 1\\
% |cdocsfi2.tex|&forwarding file for final version of chapter 2\\
% \end{tabular}
% \end{center}
% Each of the eight files can be compiled directly by the \LaTeX{} compiler.
%
% %%%%%%%%%%%%%%%%%%%%%%%%%%%%%%%%%%%%%%
% \paragraph{Main File.}
%
% The main file is called |cdocsamp.tex|.
%
% Load the \textsf{childdoc} definitions and
% declare the filename for the main document:
%    \begin{macrocode}
\input{childdoc.def}
\childdocmain{}
%    \end{macrocode}

% Optional override for |\version| flag:
%    \begin{macrocode}
%%\ifchilddoc\else\providecommand{\version}{draft}\fi
%    \end{macrocode}

% Define the default values for the |\version| flag
% (|final| for the main file and |draft| for childs):
%    \begin{macrocode}
\ifchilddoc
\providecommand{\version}{draft}
\else
\providecommand{\version}{final}
\fi
%    \end{macrocode}

% Load the standard document class:
%    \begin{macrocode}
\documentclass[12pt]{article}
%    \end{macrocode}

% Start the document body:
%    \begin{macrocode}
\begin{document}
%    \end{macrocode}

% Declare a title page.
% Print title, part of document being processed and version flag:
%    \begin{macrocode}
\addtocounter{page}{-1}
\begin{center}
{\LARGE\bfseries{}childdoc example\par}
\vspace{1cm}
\ifchilddoc
\ifchilddocmanual part\else chapter\fi:
`\childdocname' of `\childdocjob'\par
\else
main document: `\childdocjob'\par
\fi
version: \version\par
\end{center}
\newpage
%    \end{macrocode}

% Manually include selected file,
% otherwise process as usual:
%    \begin{macrocode}
\ifchilddocmanual
\section*{part `\childdocname'}
\input{\childdocname}
\else
%    \end{macrocode}

% Include the two chapters:
%    \begin{macrocode}
\include{cdocsch1}
\include{cdocsch2}
%    \end{macrocode}

% Include the two parts unless only chapters should be displayed:
%    \begin{macrocode}
\ifchilddoc\else
\section{part three}
\input{cdocspt3}
\section{part four}
\input{cdocspt4}
\fi
%    \end{macrocode}

% Process as usual until here:
%    \begin{macrocode}
\fi
%    \end{macrocode}

% End of document body:
%    \begin{macrocode}
\end{document}
%    \end{macrocode}
%\iffalse
%</samplemain>
%\fi
%
% %%%%%%%%%%%%%%%%%%%%%%%%%%%%%%%%%%%%%%
% \paragraph{Chapter Include Files.}
%
% The include files are called |cdocsch1.tex| and |cdocsch2.tex|.
%
%\iffalse
%<*samplechap1|samplechap2>
%\fi

% Optional override for |\version| flag:
%    \begin{macrocode}
%%\providecommand{\version}{final}
%    \end{macrocode}

% Include the main document:
%    \begin{macrocode}
\input{childdoc.def}
\childdocof{cdocsamp}
%    \end{macrocode}

%\iffalse
%</samplechap1|samplechap2>
%\fi
%
%\iffalse
%<*samplechap1>
%\fi
% Some text for chapter 1:
%    \begin{macrocode}
\section{one}
some text in chapter one
%    \end{macrocode}

%\iffalse
%</samplechap1>
%\fi
% Some text for chapter 2:
%\iffalse
%<*samplechap2>
%\fi
%    \begin{macrocode}
\section{two}
more text in chapter two
%    \end{macrocode}

%\iffalse
%</samplechap2>
%\fi
%
% %%%%%%%%%%%%%%%%%%%%%%%%%%%%%%%%%%%%%%
% \paragraph{Part Include Files.}
%
% The include files are called |cdocspt3.tex| and |cdocspt4.tex|.
%
%\iffalse
%<*samplepart3|samplepart4>
%\fi

% Optional override for |\version| flag:
%    \begin{macrocode}
%%\providecommand{\version}{final}
%    \end{macrocode}

% Include the main document:
%    \begin{macrocode}
\input{childdoc.def}
\childdocby{cdocsamp}
%    \end{macrocode}

%\iffalse
%</samplepart3|samplepart4>
%\fi
%
%\iffalse
%<*samplepart3>
%\fi
% Some text for part 3:
%    \begin{macrocode}
some text in part three
%    \end{macrocode}

%\iffalse
%</samplepart3>
%\fi
% Some text for part 4:
%\iffalse
%<*samplepart4>
%\fi
%    \begin{macrocode}
more text in part four
%    \end{macrocode}

%\iffalse
%</samplepart4>
%\fi
%
% %%%%%%%%%%%%%%%%%%%%%%%%%%%%%%%%%%%%%%
% \paragraph{Forwarding for a Complete Draft.}
%
% The following forwarding file |cdocsdrf.tex|
% compiles the main document in draft mode:
%\iffalse
%<*sampledraft>
%\fi
%    \begin{macrocode}
\def\version{draft}
\input{childdoc.def}
\childdocforward{cdocsamp}
%    \end{macrocode}

%\iffalse
%</sampledraft>
%\fi
%
% %%%%%%%%%%%%%%%%%%%%%%%%%%%%%%%%%%%%%%
% \paragraph{Forwarding for Final Version of the Chapters.}
%
% The following forwarding files |cdocsfn1.tex| and |cdocsfn2.tex|
% (with identical content)
% compile the final versions of the child documents
% |cdocsch1.tex| and |cdocsch2.tex|, respectively:
%\iffalse
%<*samplefinal>
%\fi
%    \begin{macrocode}
\def\version{final}
\input{childdoc.def}
\childdocforwardprefix[cdocsamp]{cdocsfn}{cdocsch}
%    \end{macrocode}

%\iffalse
%</samplefinal>
%\fi
%
% %%%%%%%%%%%%%%%%%%%%%%%%%%%%%%%%%%%%%%
% \paragraph{Command Line Processing.}
%
% The following three command lines generate the output files
% |cdocscld|, |cdocscl1| and |cdocscl2|
% which should be identical to
% |cdocsdrf|, |cdocsch1| and |cdocsfn2|, respectively:
% \begin{center}
% \begin{tabular}{l}
% |latex -jobname cdocscld \|\\
% |  "\def\version{draft}\input{childdoc.def}\childdocforward{cdocsamp}"|\\
% |latex -jobname cdocscl1 \|\\
% |  "\input{childdoc.def}\childdocforward[cdocsamp]{cdocsch1}"|\\
% |latex -jobname cdocscl2 \|\\
% |  "\def\version{final}\input{childdoc.def}\childdocforward{cdocsch2}"|
% \end{tabular}
% \end{center}
% Note that the trailing backslash on each first line
% merely continues the input to the second line
% (for convenient cut ant paste).
% Furthermore, the command |latex| can be replaced by any
% of its alternative versions such as |pdflatex|.
%
% %%%%%%%%%%%%%%%%%%%%%%%%%%%%%%%%%%%%%%%%%%%%%%%%%%%%%%%%%%%%%%%%%%%%%%%%%%%%%%
% %%%%%%%%%%%%%%%%%%%%%%%%%%%%%%%%%%%%%%%%%%%%%%%%%%%%%%%%%%%%%%%%%%%%%%%%%%%%%%
% \section{Implementation}
%\iffalse
%<*package>
%\fi
%
% This section describes the definitions file |childdoc.def|.

% The definitions cannot be loaded using |\usepackage| or |\RequirePackage|
% which has a mechanism to prevent loading a style file more than once.
% When loading the definitions by means of |\input|
% multiple instances have to be prevented manually:
%\iffalse
%This code needs to be before the `\ProvidesFile' directive
%which is defined at the beginning of this file.
%Therefore it is also placed there and commented out here.
%</package>
%<*discard>
%\fi
%    \begin{macrocode}
\ifdefined\childdocmain\endinput\fi
%    \end{macrocode}
%\iffalse
%</discard>
%<*package>
%\fi
%
% \macro{\ifchilddoc}
% \macro{\ifchilddocmanual}
% The conditional |\ifchilddoc| tells whether a
% child (true) or main (false) document is being compiled.
% The conditional |\ifchilddocmanual| tells whether
% the |\includeonly| mechanism is used (false) or
% the selection of child files must be performed manually (true).
% The definitions initialise to false:
%    \begin{macrocode}
\newif\ifchilddoc
\newif\ifchilddocmanual
%    \end{macrocode}

% \macro{\childdocname}
% \macro{\childdocjob}
% The macro |\childdocname| stores the name of the main document
% to be compiled. The macro |\childdocjob| stores the name of
% the document on which the \LaTeX{} compiler was originally invoked.
% The content of |\jobname| cannot be compared
% to filenames specified in the source due to different catcodes.
% The following code rescans |\jobname|, stores the result
% in |\childdocname| and saves a copy in |\childdocjob|:
%    \begin{macrocode}
\edef\childdocname{\scantokens\expandafter{\jobname\noexpand}}
\let\childdocjob\childdocname
%    \end{macrocode}

% \macro{\childdocdisable}
% The macro |\childdocdisable| prevents the main file
% from being processed more than once.
% At this stage, the main document command |\childdocmain|
% is assumed to be called once again where it should do nothing.
% Any subsequent call to it should prevent
% a secondary processing of the main document
% It overwrites the forwarding commands
% |\childdocof| and |\childdocforward|
% with empty macros to prevent further inclusions of the main document:
%    \begin{macrocode}
\newcommand{\childdocdisable}
{
  \renewcommand{\childdocmain}[1]{\renewcommand{\childdocmain}[1]{\endinput}}
  \renewcommand{\childdocof}[1]{}
  \renewcommand{\childdocby}[2][]{}
  \renewcommand{\childdocforward}[2][]{}
  \renewcommand{\childdocdisable}{}
}
%    \end{macrocode}

% \macro{\childdocmain}
% The macro |\childdocmain| is to be called at the top of the main file
% with nothing or the main filename (without extension) as argument.
% First, it breaks loops.
% If the argument is not empty and does not match |\childdocname|
% (which is set by the first inclusion of |childdoc.def|),
% |\ifchilddoc| is set to true, |\includeonly| is applied to the child file
% and |\jobname| is set to the main file
% (for proper handling of |.aux| files):
%    \begin{macrocode}
\newcommand{\childdocmain}[1]
{
  \childdocdisable\childdocmain{}
  \if?#1?\else
    \begingroup
      \def\childdoctmp{#1}
      \ifx\childdoctmp\childdocname
        \def\childdoctmp{}
      \else
        \def\childdoctmp
        {
          \childdoctrue
          \includeonly{\childdocname}
          \def\childdocjob{#1}
          \def\jobname{#1}
        }
      \fi
      \expandafter
    \endgroup
    \childdoctmp
  \fi
}
%    \end{macrocode}

% \macro{\childdocof}
% The command |\childdocof| redirects
% compilation to the main file |#1|.
%    \begin{macrocode}
\newcommand{\childdocof}[1]
{
  \childdocdisable
  \childdoctrue
  \includeonly{\childdocname}
  \def\jobname{#1}
  \def\childdocjob{#1}
  \input{#1}
}
%    \end{macrocode}

% \macro{\childdocby}
% The command |\childdocby| ....
%    \begin{macrocode}
\newcommand{\childdocby}[2][]
{
  \childdocdisable
  \childdoctrue
  \childdocmanualtrue
  \if?#1?\else
    \def\jobname{#2}
  \fi
  \def\childdocjob{#2}
  \input{#2}
  \endinput
}
%    \end{macrocode}

% \macro{\childdocforward}
% The command |\childdocforward| redirects
% compilation to the main file or
% (if the optional argument is given) a child file.
% Parameters are set as if the main file
% or a child file starting with |\childdocof| was compiled.
% Then compilation is handed over to the main file:
%    \begin{macrocode}
\newcommand{\childdocforward}[2][]
{
  \begingroup
    \if?#1?
      \def\childdoctmp
      {
        \def\childdocname{#2}
        \def\childdocjob{#2}
        \def\jobname{#2}
        \input{#2}
        \endinput
      }
    \else
      \def\childdoctmp
      {
        \childdocdisable
        \def\childdocname{#2}
        \childdoctrue
        \includeonly{#2}
        \def\childdocjob{#1}
        \def\jobname{#1}
        \input{#1}
        \endinput
      }
    \fi
    \expandafter
  \endgroup
  \childdoctmp
}
%    \end{macrocode}

% \macro{\childdocforwardprefix}
% The command |\childdocforwardprefix| redirects
% compilation to the main or a child file by means of a pattern.
% The prefix |#1| in the current filename is replaced by |#2|
% and the suffix of the current filename is kept
% (it is assumed that the filename does not contain the substring `|~~~|'
% which is used as a delimiter).
% Compilation is handed over to the new file by |\childdocforward|:
%    \begin{macrocode}
\newcommand{\childdocforwardprefix}[3][]
{
  \begingroup
    \def\childdocextract #2##1~~~{\def\childdoctmp{\childdocforward[#1]{#3##1}}}
    \expandafter\childdocextract\childdocname~~~
    \expandafter
  \endgroup
  \childdoctmp
}
%    \end{macrocode}

% \macro{\childdoc}
% The deprecated macro |\childdoc| is a legacy version of |\childdocmain|:
%    \begin{macrocode}
\newcommand{\childdoc}{\childdocmain}
%    \end{macrocode}

% \macro{\childdocredirect}
% The deprecated macro |\childdocredirect| is a legacy version
% of |\childdocforward| and |\childdocforwardprefix|:
%    \begin{macrocode}
\newcommand{\childdocredirect}[2][]
{
  \begingroup
    \if?#1?
      \def\childdoctmp{\childdocforward{#2}}
    \else
      \def\childdoctmp{\childdocforwardprefix{#1}{#2}}
    \fi
    \expandafter
  \endgroup
  \childdoctmp
}
%    \end{macrocode}

%\iffalse
%</package>
%\fi
%
\endinput
|\\
|\childdocby{|\textit{main}|}|\\
\end{tabular}
\end{center}
%
The directive |\childdocby| is similar to |\childdocof|
described in \secref{sec:include},
but the subsequent selection of content must be done manually.
To that end, both |\ifchilddoc| and |\ifchilddocmanual|
will be true upon processing of a part,
and the name of the part is stored in |\childdocname|.
Note that |\jobname| will be set to the filename of the current part
so that each part receives an individual |.aux| file
that does not interfere with the |.aux| file(s) of the main document.
This behaviour can be altered by the alternative form
|\childdocby[*]{|\textit{main}|}| (with a non-empty optional argument)
which uses the |.aux| file of the main document
by setting |\jobname| to \textit{main}.

%%%%%%%%%%%%%%%%%%%%%%%%%%%%%%%%%%%%%%%%%%%%%%%%%%%%%%%%%%%%%%%%%%%%%%%%%%%%%%%%
\subsection{Driver Development}
\label{sec:driver}

The \textsf{childdoc} mechanism can also be use for the development
of definition files such as \LaTeX{} styles or classes.
This case differs from the above setup with multiple parts
included by |\include| in that no |\includeonly| should be invoked.
This can be achieved by starting the include file
(before |\ProvidesPackage|) with:
%
\begin{center}
\begin{tabular}{l}
|% \iffalse
%
% childdoc.dtx Copyright (C) 2017-2018 Niklas Beisert
%
% This work may be distributed and/or modified under the
% conditions of the LaTeX Project Public License, either version 1.3
% of this license or (at your option) any later version.
% The latest version of this license is in
%   http://www.latex-project.org/lppl.txt
% and version 1.3 or later is part of all distributions of LaTeX
% version 2005/12/01 or later.
%
% This work has the LPPL maintenance status `maintained'.
%
% The Current Maintainer of this work is Niklas Beisert.
%
% This work consists of the files childdoc.dtx and childdoc.ins
% and the derived files childdoc.def and cdocsamp.tex with
% cdocsch1.tex, cdocsch2.tex, cdocsdrf.tex, cdocsfn1.tex, cdocsfn2.tex.
%
%<package>\ifdefined\childdocmain\endinput\fi
%<package>\ProvidesFile{childdoc.def}[2018/12/30 v2.0 child document driver]
%<samplemain>\ProvidesFile{cdocsamp.tex}[2018/12/30 v2.0 sample for childdoc]
%<*driver>
%\ProvidesFile{childdoc.drv}[2018/12/30 v2.0 childdoc reference manual file]
\PassOptionsToClass{10pt,a4paper}{article}
\documentclass{ltxdoc}

\usepackage[margin=35mm]{geometry}
\usepackage{hyperref}
\usepackage{hyperxmp}
\usepackage[usenames]{color}

\hypersetup{colorlinks=true}
\hypersetup{pdfstartview=FitH}
\hypersetup{pdfpagemode=UseNone}
\hypersetup{pdfsource={}}
\hypersetup{pdflang={en-UK}}
\hypersetup{pdfcopyright={Copyright 2017-2018 Niklas Beisert.
  This work may be distributed and/or modified under the
  conditions of the LaTeX Project Public License, either version 1.3
  of this license or (at your option) any later version.}}
\hypersetup{pdflicenseurl={http://www.latex-project.org/lppl.txt}}
\hypersetup{pdfcontactaddress={ETH Zurich, ITP, HIT K,
  Wolfgang-Pauli-Strasse 27}}
\hypersetup{pdfcontactpostcode={8093}}
\hypersetup{pdfcontactcity={Zurich}}
\hypersetup{pdfcontactcountry={Switzerland}}
\hypersetup{pdfcontactemail={nbeisert@itp.phys.ethz.ch}}
\hypersetup{pdfcontacturl={http://people.phys.ethz.ch/\xmptilde nbeisert/}}

\newcommand{\secref}[1]{\hyperref[#1]{section \ref*{#1}}}

\parskip1ex
\parindent0pt
\let\olditemize\itemize
\def\itemize{\olditemize\parskip0pt}

\begin{document}

\title{The \textsf{childdoc} Package}
\hypersetup{pdftitle={The childdoc Package}}
\author{Niklas Beisert\\[2ex]
  Institut f\"ur Theoretische Physik\\
  Eidgen\"ossische Technische Hochschule Z\"urich\\
  Wolfgang-Pauli-Strasse 27, 8093 Z\"urich, Switzerland\\[1ex]
  \href{mailto:nbeisert@itp.phys.ethz.ch}
  {\texttt{nbeisert@itp.phys.ethz.ch}}}
\hypersetup{pdfauthor={Niklas Beisert}}
\hypersetup{pdfsubject={Manual for the LaTeX2e Package childdoc}}
\date{30 December 2018, \textsf{v2.0}}
\maketitle

\begin{abstract}\noindent
\textsf{childdoc} is a \LaTeXe{} package
that enables the direct compilation
of document sections included by |\include|
to individual files.
\end{abstract}

\begingroup
\parskip0ex
\tableofcontents
\endgroup

%%%%%%%%%%%%%%%%%%%%%%%%%%%%%%%%%%%%%%%%%%%%%%%%%%%%%%%%%%%%%%%%%%%%%%%%%%%%%%%%
%%%%%%%%%%%%%%%%%%%%%%%%%%%%%%%%%%%%%%%%%%%%%%%%%%%%%%%%%%%%%%%%%%%%%%%%%%%%%%%%
\section{Introduction}

\LaTeX{} provides a mechanism to structure a large document (such as a book)
into a main file and several child files (containing the chapters)
using the |\include| command.
This mechanism is beneficial for documents
which span hundreds of pages in order to
make the source file(s) more manageable.
Moreover, compilation can be restricted to
selected child files by means of the |\includeonly| command.
The latter feature can be used to reduce the compilation time while editing
(this was significantly more useful in the earlier days of \LaTeX{})
or to generate a smaller document which is easier to navigate.
Another application of |\includeonly| is to generate
documents consisting of selected parts of the complete document.

However, there are a few drawbacks of the plain |\include| mechanism:
\begin{itemize}
\item
The child files cannot be compiled on their own,
they can only be compiled via the main file.
A naive editing environment
(such as a text editor with an option
to have the current file processed by \LaTeX)
may require one to switch to the main file before compiling;
attempting to compile the child file produces errors.
\item
The main file must be modified (each time)
to adjust the |\includeonly| command
to the present needs. This easily leaves the main file in a messy state.
\item
The generated document will always carry the filename
of the main document. This is inconvenient if
several child files are to be compiled and
to be kept for distribution.
\end{itemize}

The present package provides a simple interface
to make child files individually compilable by \LaTeX{}.
Compiling a child file then has the same effect as compiling
the main file with an |\includeonly| command
to select the appropriate child.
Moreover the generated document will carry the name of the child
rather than the main file.
This resolves all three above issues.

This feature is meant to make the editing of books,
thesis documents and lecture notes somewhat more convenient.
However, the package can also be used efficiently for
composing a series of documents (such as exercise sheets)
which are typically distributed individually.
It then assists the author in generating the individual documents
(potentially in different versions)
as well as a document containing the collected series.
Another application is in developing style files
or other kinds of included material
where compilation of the style file could redirect
to a sample or test file.

%%%%%%%%%%%%%%%%%%%%%%%%%%%%%%%%%%%%%%%%%%%%%%%%%%%%%%%%%%%%%%%%%%%%%%%%%%%%%%%%
%%%%%%%%%%%%%%%%%%%%%%%%%%%%%%%%%%%%%%%%%%%%%%%%%%%%%%%%%%%%%%%%%%%%%%%%%%%%%%%%
\section{Usage}

First of all, the package \textsf{childdoc} is \emph{not} a standard
\LaTeXe{} |.sty| style file! Therefore it needs to be invoked in
a non-standard way.

%%%%%%%%%%%%%%%%%%%%%%%%%%%%%%%%%%%%%%%%%%%%%%%%%%%%%%%%%%%%%%%%%%%%%%%%%%%%%%%%
\subsection{Included Files}
\label{sec:include}

%%%%%%%%%%%%%%%%%%%%%%%%%%%%%%%%%%%%%%%%
\DescribeMacro{\childdocmain}
To use the package, add the commands
\begin{center}
\begin{tabular}{l}
|\input{childdoc.def}|\\
|\childdocmain{}|\\
\end{tabular}
\end{center}
at the very top of the main \LaTeX{} file,
in particular \emph{before} the |\documentclass| statement!
The argument of |\childdocmain| should be left empty
(but it must be present).

%%%%%%%%%%%%%%%%%%%%%%%%%%%%%%%%%%%%%%%%
\DescribeMacro{\childdocof}
Furthermore, add the commands
\begin{center}
\begin{tabular}{l}
|\input{childdoc.def}|\\
|\childdocof{|\textit{main}|}|\\
\end{tabular}
\end{center}
at the top of every child file \textit{child}
which is included by |\include{|\textit{child}|}|
from within the main file
(or at least for those files to be compiled individually).
The argument \textit{main} must be the filename of the main file.

There are a couple of
considerations in setting up the main and child documents:

%%%%%%%%%%%%%%%%%%%%%%%%%%%%%%%%%%%%%%%%
\paragraph{Restrictions.}

Please note the following restrictions:
\begin{itemize}
\item
|\childdocmain| must be called with one argument \textit{main}
to ensure compatibility with earlier version of the package.
It must either be empty (|\childdocmain{}|)
or precisely match the filename of the main file in which it is specified.
See \secref{sec:detection} for further information.
\item
The filename \textit{main} must be specified without the |.tex| extension.
\item
The filename \textit{main} is case sensitive
(even in case-insensitive file systems)
due to internal string comparison.
\item
The argument \textit{main} should be fully expanded, it cannot be a macro.
\item
Subdirectories and special characters should be avoided in filenames.
\item
The command |\childdocmain{|\textit{main}|}| must be followed by a whitespace.
It should not be followed immediately by another command
or by a comment mark `|%|'.
This is because the \TeX{} parser reads the token immediately following
the argument of |\childdocmain| and puts it
at the beginning of every child section;
however, a white\-space is ignored.
\end{itemize}

%%%%%%%%%%%%%%%%%%%%%%%%%%%%%%%%%%%%%%%%
\paragraph{Content of Main File.}

It is advisable to place all content in the child files included by |\include|.
Any output contained in the main file will appear in all child documents
unless suppressed manually;
it cannot be suppressed automatically by the |\includeonly| directive
and thus should normally be avoided.
A method to include some content in the main file
by means of conditional processing is described in \secref{sec:conditional}.

%%%%%%%%%%%%%%%%%%%%%%%%%%%%%%%%%%%%%%%%
\paragraph{Page Numbering.}

When only a part of the document is compiled,
the appropriate numbering of pages
(as well as other status parameters)
is determined from the |.aux| files.
The latter contain information from previous passes.
However this information needs to propagate through
all intermediate child documents.
Therefore the page numbering in child documents may well
be inconsistent until the complete document is compiled at least once.

A useful (if unconventional) way to always ensure a consistent
page numbering is to restart the numbering in each child document
and denote the pages by `\textit{child}|.|\textit{page}'
where \textit{child} represents the chapter/section number of the child file.
This can be achieved by the command
|\numberwithin{page}{|\textit{child}|}|
of the \textsf{amsmath} package
where \textit{child} can be |chapter| or |section|
depending on the chosen structuring.
Alternatively, one can modify the macro |\thepage| appropriately
and reset the counter |page| at the start of each child file.

%%%%%%%%%%%%%%%%%%%%%%%%%%%%%%%%%%%%%%%%%%%%%%%%%%%%%%%%%%%%%%%%%%%%%%%%%%%%%%%%
\subsection{Conditional Processing}
\label{sec:conditional}

The package provides a mechanism to compile different versions
of a document. To customise the versions further some conditional processing
can come in handy to distinguish which version is being compiled.
The package provides two macros to describe the compilation context:

%%%%%%%%%%%%%%%%%%%%%%%%%%%%%%%%%%%%%%%%
\DescribeMacro{\ifchilddoc}
The conditional |\ifchilddoc| distinguishes between the compilation of
child documents and the main document:
%
\begin{center}
|\ifchilddoc |\textit{child-code}| |[|\||else |\textit{main-code}]| \||fi|
\end{center}

%%%%%%%%%%%%%%%%%%%%%%%%%%%%%%%%%%%%%%%%
\DescribeMacro{\childdocname}
\DescribeMacro{\childdocjob}
The macro |\childdocname| contains the filename (without extension)
of the main or child file being processed.
Note that |\childdocjob| will always contain the name of the main file.

%%%%%%%%%%%%%%%%%%%%%%%%%%%%%%%%%%%%%%%%
\paragraph{Title Page.}

Conditional processing can be used to include a title or banner page
in the main document when proper precautions are taken.
Importantly, the code in the main file should ensure that the page counter
(as well as other status parameters which are stored in the |.aux| files)
takes the same value after the conditional processing.
Otherwise the page numbers may take divergent values
depending on which part is compiled.

For example, a title page could be declared by:
%
\begin{center}
\begin{tabular}{l}
|\ifchilddoc\||else|\\
|\addtocounter{page}{-1}|\\
\textit{code for title page}\\
|\newpage|\\
|\||fi|
\end{tabular}
\end{center}
%
A banner page for the child documents can be generated by:
%
\begin{center}
\begin{tabular}{l}
|\ifchilddoc|\\
|\addtocounter{page}{-1}|\\
\textit{code for banner page}\\
|\newpage|\\
|\||fi|
\end{tabular}
\end{center}
%
Here one could write a message such as:
\begin{center}
|This is the part \childdocname{} of \childdocjob{}.|
\end{center}

%%%%%%%%%%%%%%%%%%%%%%%%%%%%%%%%%%%%%%%%%%%%%%%%%%%%%%%%%%%%%%%%%%%%%%%%%%%%%%%%
\subsection{Flags}
\label{sec:flags}

The package makes it easy to generate different versions
of the main or child documents.
To this end compilation flags can be defined
and assigned different default values.
They will be particularly useful in conjunction
with the forwarding mechanism described in \secref{sec:forward}.

For example, it may be useful to have a flag |\version|
which can be set to |draft| or |final|.
The document source will contain some conditional code
depending on the value of |\version|.
Suppose further, the flag should default to |final| for the main file
and to |draft| for child files
which is a natural assignment for editing the document.
This is achieved by placing the following code
in the preamble of the main document
(below the |\childdocmain| directive):
%
\begin{center}
\begin{tabular}{l}
|\ifchilddoc|\\
|\providecommand{\version}{draft}|\\
|\||else|\\
|\providecommand{\version}{final}|\\
|\||fi|
\end{tabular}
\end{center}
%
The definition by |\providecommand| makes sure
that previous definitions are not overwritten.
Further statements |\providecommand{\version}{...}|
can thus be added before the above code to override it.

For the main file, one might add a line
(between |\childdocmain| and the above block)
%
\begin{center}
|%\ifchilddoc\||else\providecommand{\version}{draft}\||fi|
\end{center}
%
which can be uncommented to produce a draft version.
Likewise one can add a line to the very top of a child file
(above the |\childdocof{|\textit{main}|}| directive)
%
\begin{center}
|%\providecommand{\version}{final}|
\end{center}
%
which can be uncommented to produce the final version of this child document.

%%%%%%%%%%%%%%%%%%%%%%%%%%%%%%%%%%%%%%%%%%%%%%%%%%%%%%%%%%%%%%%%%%%%%%%%%%%%%%%%
\subsection{Forwarding}
\label{sec:forward}

Different versions of the main or child documents
using compilation flags as described in \secref{sec:flags}
can be (permanently) stored in different files
for convenient compilation, viewing and distribution.
To this end, the package defines a command
to pass on compilation to a different file:

%%%%%%%%%%%%%%%%%%%%%%%%%%%%%%%%%%%%%%%%
\DescribeMacro{\childdocforward}
The command |\childdocforward| redirects processing to
another source file:
%
\begin{center}
\begin{tabular}{l}
|\input{childdoc.def}|\\
|\childdocforward[|\textit{main}|]{|\textit{dest}|}|\\
\end{tabular}
\end{center}
%
The argument \textit{dest} is the destination file
(without extension).
It should be the main file or one of the child files.
Note that further \textsf{childdoc} directives
such as |\childdocof| and |\childdocforward|
in the indicated file will be processed in this form.
The optional argument \textit{main}
passes on directly to the main file \textit{main}
while pretending to compile the child \textit{dest}.
This form behaves as if \textit{dest}
issues |\childdocof{|\textit{main}|}| right away,
and no further \textsf{childdoc} directives will be processed.

%%%%%%%%%%%%%%%%%%%%%%%%%%%%%%%%%%%%%%%%
\DescribeMacro{\...prefix}
In the alternative form |\childdocforwardprefix|,
%
\begin{center}
\begin{tabular}{l}
|\input{childdoc.def}|\\
|\childdocforwardprefix[|\textit{main}|]{|\textit{prefix}|}{|\textit{dest}|}|
\end{tabular}
\end{center}
%
the destination file is determined by a pattern
depending on the current file:
To make this work, the current file must be called
`{\textit{prefix}\hspace{0.2em}\textit{suffix}}'
with \textit{prefix} matching precisely the argument.
Processing is then passed on to the file
`{\textit{dest}\hspace{0.2em}\textit{suffix}}'.
Surely, the same effect is achieved by
directly specifying the
argument `{\textit{dest}\hspace{0.2em}\textit{suffix}}'
in the first form.
However, that requires to set up a different file
for each child. With the alternative form of the command
all these files can have exactly the same content
which simplifies setting them up and maintaining them.

For example, the following file |draft.tex|
with a compilation flag |\version| as described in \secref{sec:flags}
compiles the main document as a draft:
%
\begin{center}
\begin{tabular}{l}
|\def\version{draft}|\\
|\input{childdoc.def}|\\
|\childdocforward{|\textit{main}|}|
\end{tabular}
\end{center}
%
Likewise, the following files |final|\textit{nn}|.tex|
compile the final version of the child document
|child|\textit{nn}|.tex|:
%
\begin{center}
\begin{tabular}{l}
|\def\version{final}|\\
|\input{childdoc.def}|\\
|\childdocforwardprefix{final}{child}|
\end{tabular}
\end{center}
%

Note that when several versions of a main file and/or of each child file
are to be generated, it may be convenient to set up a |Makefile| or
shell script to automatise the process.

%%%%%%%%%%%%%%%%%%%%%%%%%%%%%%%%%%%%%%%%%%%%%%%%%%%%%%%%%%%%%%%%%%%%%%%%%%%%%%%%
\subsection{Command Line Processing}
\label{sec:commandline}

The effect of redirection files can also be achieved by invoking
the \LaTeX{} compiler with a more elaborate command line.
Most conveniently this should be done as part
of a shell script or a |Makefile|.

When using \textsf{childdoc} in the main file, the following
command lines effectively perform a redirection
(note that depending on the shell being used,
backslashes may have to be doubled: `|\|' $\to$ `|\\|'):
%
\begin{center}
|... -jobname "|\textit{target}|" |\\|"|[\textit{flags}]%
|\input{childdoc.def}\childdocforward[|\textit{main}|]{|\textit{dest}|}"|
\end{center}
%
Here \textit{target} is the name of the output file,
\textit{main} is the name of the main file
and \textit{dest} is the name of the main or child file to be processed
(all filenames without extensions).
The optional argument \textit{main} can be omitted
if \textit{main} matches \textit{dest}.
Optionally, compilation \textit{flags} can be defined via |\def| commands.
This command line makes the \TeX{} engine believe
it is compiling the file \textit{target}
whose content is specified as the latter parameter.
The provided code then forwards the processing to
\textit{main} or \textit{dest} as described in \secref{sec:forward}.

%%%%%%%%%%%%%%%%%%%%%%%%%%%%%%%%%%%%%%%%%%%%%%%%%%%%%%%%%%%%%%%%%%%%%%%%%%%%%%%%
\subsection{Include by Input}
\label{sec:input}

Including child documents by |\include| has some restrictions by design.
Most notably, the content of a child document always occupies
its own set of pages; pages cannot be shared between child documents.
Usually, this behaviour makes perfect sense
because each child document contain an essential part of the document.
However, in some situations it may be desirable to compose
a document from a collection of parts
without having mandatory page breaks between then.
For this case, the package
provides a mechanism to include parts
by |\input| which can also be processed individually.
However, by construction this mechanism
requires manual handling of the content to be output.

%%%%%%%%%%%%%%%%%%%%%%%%%%%%%%%%%%%%%%%%
\DescribeMacro{\ifchilddocmanual}
The main file should be prepared as usual, see \secref{sec:include}.
However, the document body must make a distinction
between processing of an individual part and of the main document, e.g.:
%
\begin{center}
\begin{tabular}{l}
|\ifchilddocmanual|\\
|\input{\childdocname}|\\
|\||else|\\
\textit{document body with }|\input{|\textit{part}|}|\\
|\||fi|
\end{tabular}
\end{center}
%
The conditional |\ifchilddocmanual| is true whenever
a part to be included by |\input| is being compiled,
and the name of the part is stored in |\childdocname|.

%%%%%%%%%%%%%%%%%%%%%%%%%%%%%%%%%%%%%%%%
\DescribeMacro{\childdocby}
Each part to be included by |\input| should start with:
%
\begin{center}
\begin{tabular}{l}
|\input{childdoc.def}|\\
|\childdocby{|\textit{main}|}|\\
\end{tabular}
\end{center}
%
The directive |\childdocby| is similar to |\childdocof|
described in \secref{sec:include},
but the subsequent selection of content must be done manually.
To that end, both |\ifchilddoc| and |\ifchilddocmanual|
will be true upon processing of a part,
and the name of the part is stored in |\childdocname|.
Note that |\jobname| will be set to the filename of the current part
so that each part receives an individual |.aux| file
that does not interfere with the |.aux| file(s) of the main document.
This behaviour can be altered by the alternative form
|\childdocby[*]{|\textit{main}|}| (with a non-empty optional argument)
which uses the |.aux| file of the main document
by setting |\jobname| to \textit{main}.

%%%%%%%%%%%%%%%%%%%%%%%%%%%%%%%%%%%%%%%%%%%%%%%%%%%%%%%%%%%%%%%%%%%%%%%%%%%%%%%%
\subsection{Driver Development}
\label{sec:driver}

The \textsf{childdoc} mechanism can also be use for the development
of definition files such as \LaTeX{} styles or classes.
This case differs from the above setup with multiple parts
included by |\include| in that no |\includeonly| should be invoked.
This can be achieved by starting the include file
(before |\ProvidesPackage|) with:
%
\begin{center}
\begin{tabular}{l}
|\input{childdoc.def}|\\
|\childdocforward{|\textit{main}|}|\\
\end{tabular}
\end{center}
%
or alternatively with:
%
\begin{center}
\begin{tabular}{l}
|\input{childdoc.def}|\\
|\childdocby{|\textit{main}|}|\\
\end{tabular}
\end{center}
%
Both forms have slightly different effects as described above.
The main file is prepared as usual, see \secref{sec:include}.

%%%%%%%%%%%%%%%%%%%%%%%%%%%%%%%%%%%%%%%%%%%%%%%%%%%%%%%%%%%%%%%%%%%%%%%%%%%%%%%%
\subsection{Legacy Detection}
\label{sec:detection}

The directive |\childdocmain| in the main file can detect
whether the complete document or merely a child is to be compiled
even without using the directive |\childdocof|.
This method is deprecated because it is less robust
and there is no compelling reason to use it;
it is merely provided for backward compatibility
and it may be removed in future versions.

If the detection mechanism is to be used,
it is mandatory to correctly specify
the filename of the main file as the argument of |\childdocmain|:
%
\begin{center}
\begin{tabular}{l}
|\input{childdoc.def}|\\
|\childdocmain{|\textit{main}|}|\\
\end{tabular}
\end{center}
%
If |\jobname| does not match the argument \textit{main} of |\childdocmain|,
it is assumed that |\jobname| points to the child file to be compiled.
When using |\childdocmain| with the main file specified as argument,
it suffices to start a child file
with just |\input{|\textit{main}|}|
without loading of the package and using |\childdocof|.
If instead all processing is done
with the appropriate \textsf{childdoc} directives,
the argument of \textit{main} of |\childdocmain| can be empty.

An alternative version of the command line processing described
in \secref{sec:commandline} using the detection mechanism reads:
%
\begin{center}
|... -jobname "|\textit{target}|" "|[\textit{flags}]%
[|\def\jobname{|\textit{dest}|}|]|\input{|\textit{main}|}"|
\end{center}

%%%%%%%%%%%%%%%%%%%%%%%%%%%%%%%%%%%%%%%%%%%%%%%%%%%%%%%%%%%%%%%%%%%%%%%%%%%%%%%%
\subsection{Manual Code}
\label{sec:manual}

In case one cannot be certain whether the definitions file |childdoc.def|
is installed on the target \TeX{} distribution
and one prefers not to ship it,
it is conceivable to paste a few relevant commands into the sources.

To that end, drop all statements |\input{childdoc.def}|
and perform the replacements as outlined below.
Instead of |\childdocmain{|\textit{main}|}| add the following code
to the top of the main file:
%
\begin{center}
\begin{tabular}{l}
|\||ifdefined\childdocname\endinput\||fi\newif\ifchilddoc|\\
|\edef\childdocname{\scantokens\expandafter{\jobname\noexpand}}|\\
|\def\childdocmain{|\textit{main}|}\||ifx\childdocmain\childdocname\||else|\\
|\childdoctrue\includeonly{\childdocname}\let\jobname\childdocmain\||fi|\\
\end{tabular}
\end{center}
%
Instead of |\childdocof{|\textit{main}|}| just include the main file
at the top of each child file:
%
\begin{center}
|\input{|\textit{main}|}|
\end{center}
%
A simple redirection |\childdocforward{|\textit{dest}|}| is achieved by:
%
\begin{center}
|\def\jobname{|\textit{dest}|}\input{\jobname}|
\end{center}
%
The redirection with prefix
|\childdocforwardprefix[|\textit{prefix}|]{|\textit{dest}|}|
is accomplished by:
%
\begin{center}
\begin{tabular}{l}
|{\edef\jobname{\scantokens\expandafter{\jobname\noexpand}}|\\
|\def\redirectjob |\textit{prefix}|#1~~~{\gdef\jobname{|\textit{dest}|#1}}|\\
|\expandafter\redirectjob\jobname~~~}\input{\jobname}|
\end{tabular}
\end{center}

In an alternative approach,
child documents can be compiled by a specific command line
without additional code or specific definitions:
%
\begin{center}
|... -jobname "|\textit{target}|" "|[\textit{flags}]%
|\includeonly{|\textit{dest}|}\input{|\textit{main}|}"|
\end{center}
%

%%%%%%%%%%%%%%%%%%%%%%%%%%%%%%%%%%%%%%%%%%%%%%%%%%%%%%%%%%%%%%%%%%%%%%%%%%%%%%%%
%%%%%%%%%%%%%%%%%%%%%%%%%%%%%%%%%%%%%%%%%%%%%%%%%%%%%%%%%%%%%%%%%%%%%%%%%%%%%%%%
\section{Information}

%%%%%%%%%%%%%%%%%%%%%%%%%%%%%%%%%%%%%%%%%%%%%%%%%%%%%%%%%%%%%%%%%%%%%%%%%%%%%%%%
\subsection{Copyright}

Copyright \copyright{} 2017--2018 Niklas Beisert

This work may be distributed and/or modified under the
conditions of the \LaTeX{} Project Public License, either version 1.3
of this license or (at your option) any later version.
The latest version of this license is in
  \url{http://www.latex-project.org/lppl.txt}
and version 1.3 or later is part of all distributions of \LaTeX{}
version 2005/12/01 or later.

This work has the LPPL maintenance status `maintained'.

The Current Maintainer of this work is Niklas Beisert.

This work consists of the files |README.txt|, |childdoc.ins| and |childdoc.dtx|
as well as the derived files |childdoc.def|, |cdocsamp.tex|
with |cdocsch1.tex|, |cdocsch2.tex|, |cdocspt3.tex|, |cdocspt4.tex|,
|cdocsdrf.tex|, |cdocsfn1.tex|, |cdocsfn2.tex|
as well as |childdoc.pdf|.

%%%%%%%%%%%%%%%%%%%%%%%%%%%%%%%%%%%%%%%%%%%%%%%%%%%%%%%%%%%%%%%%%%%%%%%%%%%%%%%%
\subsection{Files and Installation}

The package consists of the files:
%
\begin{center}
\begin{tabular}{ll}
    |README.txt|   & readme file \\
    |childdoc.ins| & installation file \\
    |childdoc.dtx| & source file \\
    |childdoc.def| & definition file \\
    |cdocsamp.tex| & sample main file \\
    |cdocsch1.tex| & sample include file \\
    |cdocsch2.tex| & sample include file \\
    |cdocspt3.tex| & sample part file \\
    |cdocspt4.tex| & sample part file \\
    |cdocsdrf.tex| & sample redirection file \\
    |cdocsfn1.tex| & sample redirection file \\
    |cdocsfn2.tex| & sample redirection file \\
    |childdoc.pdf| & manual
\end{tabular}
\end{center}
%
The distribution consists of the files
|README.txt|, |childdoc.ins| and |childdoc.dtx|.
%
\begin{itemize}
\item
Run (pdf)\LaTeX{} on |childdoc.dtx|
to compile the manual |childdoc.pdf| (this file).
\item
Run \LaTeX{} on |childdoc.ins| to create the definitions file |childdoc.def|
and the sample |cdocsamp.tex| with include files
|cdocsch1.tex|, |cdocsch2.tex|, |cdocspt3.tex|, |cdocspt4.tex|,
|cdocsdrf.tex|, |cdocsfn1.tex|, |cdocsfn2.tex|.
Then copy the file |childdoc.def| to an appropriate directory of your \LaTeX{}
distribution, e.g.\ \textit{texmf-root}|/tex/latex/childdoc|.
\end{itemize}

%%%%%%%%%%%%%%%%%%%%%%%%%%%%%%%%%%%%%%%%%%%%%%%%%%%%%%%%%%%%%%%%%%%%%%%%%%%%%%%%
\subsection{Related CTAN Packages}

There are several other packages which offer a similar functionality:
%
\begin{itemize}
\item
The packages
\href{http://ctan.org/pkg/docmute}{\textsf{docmute}},
\href{http://ctan.org/pkg/includex}{\textsf{includex}} and
\href{http://ctan.org/pkg/standalone}{\textsf{standalone}}
provide commands to include only the document body of
a child file thus allowing both files to be compiled individually.
\item
The packages \href{http://ctan.org/pkg/subdocs}{\textsf{subdocs}}
and \href{http://ctan.org/pkg/subfiles}{\textsf{subfiles}}
provide structures in which the main and child documents can be
encapsulated and allowing them to be compiled individually.
The inclusion mechanism is different from the conventional |\include|.
\item
The package \href{http://ctan.org/pkg/combine}{\textsf{combine}}
is an elaborate solution to combine several documents into one.
\end{itemize}
%
See also the CTAN topic \href{http://ctan.org/topic/subdocs}{\textsf{subdocs}}
for further related packages.
The present package differs from the above solutions in that
a document structure constructed with the conventional |\include| mechanism
just needs two extra commands at the top of every file
such that all constituent files can be compiled individually.

%%%%%%%%%%%%%%%%%%%%%%%%%%%%%%%%%%%%%%%%%%%%%%%%%%%%%%%%%%%%%%%%%%%%%%%%%%%%%%%%
%\subsection{Feature Suggestions}
%
%The following is a list of features which may be useful for future
%versions of this package:
%%
%\begin{itemize}
%\item
%\ldots
%\end{itemize}

%%%%%%%%%%%%%%%%%%%%%%%%%%%%%%%%%%%%%%%%%%%%%%%%%%%%%%%%%%%%%%%%%%%%%%%%%%%%%%%%
\subsection{Revision History}

%%%%%%%%%%%%%%%%%%%%%%%%%%%%%%%%%%%%%%%%
\paragraph{v2.0:} 2018/12/30

\begin{itemize}
\item
immediate forward processing
\item
added |\childdocby| mechanism
\item
manual restructured
\end{itemize}

%%%%%%%%%%%%%%%%%%%%%%%%%%%%%%%%%%%%%%%%
\paragraph{v1.6:} 2018/01/17

\begin{itemize}
\item
application for development of include files
\item
corrections to manual
\end{itemize}

%%%%%%%%%%%%%%%%%%%%%%%%%%%%%%%%%%%%%%%%
\paragraph{v1.5:} 2017/05/21

\begin{itemize}
\item
more complete structuring introduced
\item
|\childdocof| introduced
\item
|\childdoc| renamed to |\childdocmain|
\item
|\childredirect| renamed to |\childdocforward| and |\childdocforwardprefix|
and functionality expanded
\end{itemize}

%%%%%%%%%%%%%%%%%%%%%%%%%%%%%%%%%%%%%%%%
\paragraph{v1.0:} 2017/04/27

\begin{itemize}
\item
manual and install package
\item
first version published on CTAN
\end{itemize}

%%%%%%%%%%%%%%%%%%%%%%%%%%%%%%%%%%%%%%%%
\paragraph{v0.6:} 2017/04/26

\begin{itemize}
\item
redirection mechanism added
\end{itemize}

%%%%%%%%%%%%%%%%%%%%%%%%%%%%%%%%%%%%%%%%
\paragraph{v0.5:} 2017/04/26

\begin{itemize}
\item
functionality in definition file
\end{itemize}


%%%%%%%%%%%%%%%%%%%%%%%%%%%%%%%%%%%%%%%%%%%%%%%%%%%%%%%%%%%%%%%%%%%%%%%%%%%%%%%%
%%%%%%%%%%%%%%%%%%%%%%%%%%%%%%%%%%%%%%%%%%%%%%%%%%%%%%%%%%%%%%%%%%%%%%%%%%%%%%%%
%%%%%%%%%%%%%%%%%%%%%%%%%%%%%%%%%%%%%%%%%%%%%%%%%%%%%%%%%%%%%%%%%%%%%%%%%%%%%%%%
\appendix

\settowidth\MacroIndent{\rmfamily\scriptsize 000\ }

 \DocInput{childdoc.dtx}

\end{document}
%</driver>
% \fi
%
% %%%%%%%%%%%%%%%%%%%%%%%%%%%%%%%%%%%%%%%%%%%%%%%%%%%%%%%%%%%%%%%%%%%%%%%%%%%%%%
% %%%%%%%%%%%%%%%%%%%%%%%%%%%%%%%%%%%%%%%%%%%%%%%%%%%%%%%%%%%%%%%%%%%%%%%%%%%%%%
% \section{Sample}
%\iffalse
%<*samplemain>
%\fi
%
% The following presents a sample document
% with two chapters, two parts, a title page,
% a compile flag as well as three forwarding files to set the flag.
% It consists of eight |.tex| files:
% \begin{center}
% \begin{tabular}{ll}
% |cdocsamp.tex|&main file\\
% |cdocsch1.tex|&include file for chapter 1\\
% |cdocsch2.tex|&include file for chapter 2\\
% |cdocspt3.tex|&include file for part 3\\
% |cdocspt4.tex|&include file for part 4\\
% |cdocsdrf.tex|&forwarding file for main file in draft mode\\
% |cdocsfi1.tex|&forwarding file for final version of chapter 1\\
% |cdocsfi2.tex|&forwarding file for final version of chapter 2\\
% \end{tabular}
% \end{center}
% Each of the eight files can be compiled directly by the \LaTeX{} compiler.
%
% %%%%%%%%%%%%%%%%%%%%%%%%%%%%%%%%%%%%%%
% \paragraph{Main File.}
%
% The main file is called |cdocsamp.tex|.
%
% Load the \textsf{childdoc} definitions and
% declare the filename for the main document:
%    \begin{macrocode}
\input{childdoc.def}
\childdocmain{}
%    \end{macrocode}

% Optional override for |\version| flag:
%    \begin{macrocode}
%%\ifchilddoc\else\providecommand{\version}{draft}\fi
%    \end{macrocode}

% Define the default values for the |\version| flag
% (|final| for the main file and |draft| for childs):
%    \begin{macrocode}
\ifchilddoc
\providecommand{\version}{draft}
\else
\providecommand{\version}{final}
\fi
%    \end{macrocode}

% Load the standard document class:
%    \begin{macrocode}
\documentclass[12pt]{article}
%    \end{macrocode}

% Start the document body:
%    \begin{macrocode}
\begin{document}
%    \end{macrocode}

% Declare a title page.
% Print title, part of document being processed and version flag:
%    \begin{macrocode}
\addtocounter{page}{-1}
\begin{center}
{\LARGE\bfseries{}childdoc example\par}
\vspace{1cm}
\ifchilddoc
\ifchilddocmanual part\else chapter\fi:
`\childdocname' of `\childdocjob'\par
\else
main document: `\childdocjob'\par
\fi
version: \version\par
\end{center}
\newpage
%    \end{macrocode}

% Manually include selected file,
% otherwise process as usual:
%    \begin{macrocode}
\ifchilddocmanual
\section*{part `\childdocname'}
\input{\childdocname}
\else
%    \end{macrocode}

% Include the two chapters:
%    \begin{macrocode}
\include{cdocsch1}
\include{cdocsch2}
%    \end{macrocode}

% Include the two parts unless only chapters should be displayed:
%    \begin{macrocode}
\ifchilddoc\else
\section{part three}
\input{cdocspt3}
\section{part four}
\input{cdocspt4}
\fi
%    \end{macrocode}

% Process as usual until here:
%    \begin{macrocode}
\fi
%    \end{macrocode}

% End of document body:
%    \begin{macrocode}
\end{document}
%    \end{macrocode}
%\iffalse
%</samplemain>
%\fi
%
% %%%%%%%%%%%%%%%%%%%%%%%%%%%%%%%%%%%%%%
% \paragraph{Chapter Include Files.}
%
% The include files are called |cdocsch1.tex| and |cdocsch2.tex|.
%
%\iffalse
%<*samplechap1|samplechap2>
%\fi

% Optional override for |\version| flag:
%    \begin{macrocode}
%%\providecommand{\version}{final}
%    \end{macrocode}

% Include the main document:
%    \begin{macrocode}
\input{childdoc.def}
\childdocof{cdocsamp}
%    \end{macrocode}

%\iffalse
%</samplechap1|samplechap2>
%\fi
%
%\iffalse
%<*samplechap1>
%\fi
% Some text for chapter 1:
%    \begin{macrocode}
\section{one}
some text in chapter one
%    \end{macrocode}

%\iffalse
%</samplechap1>
%\fi
% Some text for chapter 2:
%\iffalse
%<*samplechap2>
%\fi
%    \begin{macrocode}
\section{two}
more text in chapter two
%    \end{macrocode}

%\iffalse
%</samplechap2>
%\fi
%
% %%%%%%%%%%%%%%%%%%%%%%%%%%%%%%%%%%%%%%
% \paragraph{Part Include Files.}
%
% The include files are called |cdocspt3.tex| and |cdocspt4.tex|.
%
%\iffalse
%<*samplepart3|samplepart4>
%\fi

% Optional override for |\version| flag:
%    \begin{macrocode}
%%\providecommand{\version}{final}
%    \end{macrocode}

% Include the main document:
%    \begin{macrocode}
\input{childdoc.def}
\childdocby{cdocsamp}
%    \end{macrocode}

%\iffalse
%</samplepart3|samplepart4>
%\fi
%
%\iffalse
%<*samplepart3>
%\fi
% Some text for part 3:
%    \begin{macrocode}
some text in part three
%    \end{macrocode}

%\iffalse
%</samplepart3>
%\fi
% Some text for part 4:
%\iffalse
%<*samplepart4>
%\fi
%    \begin{macrocode}
more text in part four
%    \end{macrocode}

%\iffalse
%</samplepart4>
%\fi
%
% %%%%%%%%%%%%%%%%%%%%%%%%%%%%%%%%%%%%%%
% \paragraph{Forwarding for a Complete Draft.}
%
% The following forwarding file |cdocsdrf.tex|
% compiles the main document in draft mode:
%\iffalse
%<*sampledraft>
%\fi
%    \begin{macrocode}
\def\version{draft}
\input{childdoc.def}
\childdocforward{cdocsamp}
%    \end{macrocode}

%\iffalse
%</sampledraft>
%\fi
%
% %%%%%%%%%%%%%%%%%%%%%%%%%%%%%%%%%%%%%%
% \paragraph{Forwarding for Final Version of the Chapters.}
%
% The following forwarding files |cdocsfn1.tex| and |cdocsfn2.tex|
% (with identical content)
% compile the final versions of the child documents
% |cdocsch1.tex| and |cdocsch2.tex|, respectively:
%\iffalse
%<*samplefinal>
%\fi
%    \begin{macrocode}
\def\version{final}
\input{childdoc.def}
\childdocforwardprefix[cdocsamp]{cdocsfn}{cdocsch}
%    \end{macrocode}

%\iffalse
%</samplefinal>
%\fi
%
% %%%%%%%%%%%%%%%%%%%%%%%%%%%%%%%%%%%%%%
% \paragraph{Command Line Processing.}
%
% The following three command lines generate the output files
% |cdocscld|, |cdocscl1| and |cdocscl2|
% which should be identical to
% |cdocsdrf|, |cdocsch1| and |cdocsfn2|, respectively:
% \begin{center}
% \begin{tabular}{l}
% |latex -jobname cdocscld \|\\
% |  "\def\version{draft}\input{childdoc.def}\childdocforward{cdocsamp}"|\\
% |latex -jobname cdocscl1 \|\\
% |  "\input{childdoc.def}\childdocforward[cdocsamp]{cdocsch1}"|\\
% |latex -jobname cdocscl2 \|\\
% |  "\def\version{final}\input{childdoc.def}\childdocforward{cdocsch2}"|
% \end{tabular}
% \end{center}
% Note that the trailing backslash on each first line
% merely continues the input to the second line
% (for convenient cut ant paste).
% Furthermore, the command |latex| can be replaced by any
% of its alternative versions such as |pdflatex|.
%
% %%%%%%%%%%%%%%%%%%%%%%%%%%%%%%%%%%%%%%%%%%%%%%%%%%%%%%%%%%%%%%%%%%%%%%%%%%%%%%
% %%%%%%%%%%%%%%%%%%%%%%%%%%%%%%%%%%%%%%%%%%%%%%%%%%%%%%%%%%%%%%%%%%%%%%%%%%%%%%
% \section{Implementation}
%\iffalse
%<*package>
%\fi
%
% This section describes the definitions file |childdoc.def|.

% The definitions cannot be loaded using |\usepackage| or |\RequirePackage|
% which has a mechanism to prevent loading a style file more than once.
% When loading the definitions by means of |\input|
% multiple instances have to be prevented manually:
%\iffalse
%This code needs to be before the `\ProvidesFile' directive
%which is defined at the beginning of this file.
%Therefore it is also placed there and commented out here.
%</package>
%<*discard>
%\fi
%    \begin{macrocode}
\ifdefined\childdocmain\endinput\fi
%    \end{macrocode}
%\iffalse
%</discard>
%<*package>
%\fi
%
% \macro{\ifchilddoc}
% \macro{\ifchilddocmanual}
% The conditional |\ifchilddoc| tells whether a
% child (true) or main (false) document is being compiled.
% The conditional |\ifchilddocmanual| tells whether
% the |\includeonly| mechanism is used (false) or
% the selection of child files must be performed manually (true).
% The definitions initialise to false:
%    \begin{macrocode}
\newif\ifchilddoc
\newif\ifchilddocmanual
%    \end{macrocode}

% \macro{\childdocname}
% \macro{\childdocjob}
% The macro |\childdocname| stores the name of the main document
% to be compiled. The macro |\childdocjob| stores the name of
% the document on which the \LaTeX{} compiler was originally invoked.
% The content of |\jobname| cannot be compared
% to filenames specified in the source due to different catcodes.
% The following code rescans |\jobname|, stores the result
% in |\childdocname| and saves a copy in |\childdocjob|:
%    \begin{macrocode}
\edef\childdocname{\scantokens\expandafter{\jobname\noexpand}}
\let\childdocjob\childdocname
%    \end{macrocode}

% \macro{\childdocdisable}
% The macro |\childdocdisable| prevents the main file
% from being processed more than once.
% At this stage, the main document command |\childdocmain|
% is assumed to be called once again where it should do nothing.
% Any subsequent call to it should prevent
% a secondary processing of the main document
% It overwrites the forwarding commands
% |\childdocof| and |\childdocforward|
% with empty macros to prevent further inclusions of the main document:
%    \begin{macrocode}
\newcommand{\childdocdisable}
{
  \renewcommand{\childdocmain}[1]{\renewcommand{\childdocmain}[1]{\endinput}}
  \renewcommand{\childdocof}[1]{}
  \renewcommand{\childdocby}[2][]{}
  \renewcommand{\childdocforward}[2][]{}
  \renewcommand{\childdocdisable}{}
}
%    \end{macrocode}

% \macro{\childdocmain}
% The macro |\childdocmain| is to be called at the top of the main file
% with nothing or the main filename (without extension) as argument.
% First, it breaks loops.
% If the argument is not empty and does not match |\childdocname|
% (which is set by the first inclusion of |childdoc.def|),
% |\ifchilddoc| is set to true, |\includeonly| is applied to the child file
% and |\jobname| is set to the main file
% (for proper handling of |.aux| files):
%    \begin{macrocode}
\newcommand{\childdocmain}[1]
{
  \childdocdisable\childdocmain{}
  \if?#1?\else
    \begingroup
      \def\childdoctmp{#1}
      \ifx\childdoctmp\childdocname
        \def\childdoctmp{}
      \else
        \def\childdoctmp
        {
          \childdoctrue
          \includeonly{\childdocname}
          \def\childdocjob{#1}
          \def\jobname{#1}
        }
      \fi
      \expandafter
    \endgroup
    \childdoctmp
  \fi
}
%    \end{macrocode}

% \macro{\childdocof}
% The command |\childdocof| redirects
% compilation to the main file |#1|.
%    \begin{macrocode}
\newcommand{\childdocof}[1]
{
  \childdocdisable
  \childdoctrue
  \includeonly{\childdocname}
  \def\jobname{#1}
  \def\childdocjob{#1}
  \input{#1}
}
%    \end{macrocode}

% \macro{\childdocby}
% The command |\childdocby| ....
%    \begin{macrocode}
\newcommand{\childdocby}[2][]
{
  \childdocdisable
  \childdoctrue
  \childdocmanualtrue
  \if?#1?\else
    \def\jobname{#2}
  \fi
  \def\childdocjob{#2}
  \input{#2}
  \endinput
}
%    \end{macrocode}

% \macro{\childdocforward}
% The command |\childdocforward| redirects
% compilation to the main file or
% (if the optional argument is given) a child file.
% Parameters are set as if the main file
% or a child file starting with |\childdocof| was compiled.
% Then compilation is handed over to the main file:
%    \begin{macrocode}
\newcommand{\childdocforward}[2][]
{
  \begingroup
    \if?#1?
      \def\childdoctmp
      {
        \def\childdocname{#2}
        \def\childdocjob{#2}
        \def\jobname{#2}
        \input{#2}
        \endinput
      }
    \else
      \def\childdoctmp
      {
        \childdocdisable
        \def\childdocname{#2}
        \childdoctrue
        \includeonly{#2}
        \def\childdocjob{#1}
        \def\jobname{#1}
        \input{#1}
        \endinput
      }
    \fi
    \expandafter
  \endgroup
  \childdoctmp
}
%    \end{macrocode}

% \macro{\childdocforwardprefix}
% The command |\childdocforwardprefix| redirects
% compilation to the main or a child file by means of a pattern.
% The prefix |#1| in the current filename is replaced by |#2|
% and the suffix of the current filename is kept
% (it is assumed that the filename does not contain the substring `|~~~|'
% which is used as a delimiter).
% Compilation is handed over to the new file by |\childdocforward|:
%    \begin{macrocode}
\newcommand{\childdocforwardprefix}[3][]
{
  \begingroup
    \def\childdocextract #2##1~~~{\def\childdoctmp{\childdocforward[#1]{#3##1}}}
    \expandafter\childdocextract\childdocname~~~
    \expandafter
  \endgroup
  \childdoctmp
}
%    \end{macrocode}

% \macro{\childdoc}
% The deprecated macro |\childdoc| is a legacy version of |\childdocmain|:
%    \begin{macrocode}
\newcommand{\childdoc}{\childdocmain}
%    \end{macrocode}

% \macro{\childdocredirect}
% The deprecated macro |\childdocredirect| is a legacy version
% of |\childdocforward| and |\childdocforwardprefix|:
%    \begin{macrocode}
\newcommand{\childdocredirect}[2][]
{
  \begingroup
    \if?#1?
      \def\childdoctmp{\childdocforward{#2}}
    \else
      \def\childdoctmp{\childdocforwardprefix{#1}{#2}}
    \fi
    \expandafter
  \endgroup
  \childdoctmp
}
%    \end{macrocode}

%\iffalse
%</package>
%\fi
%
\endinput
|\\
|\childdocforward{|\textit{main}|}|\\
\end{tabular}
\end{center}
%
or alternatively with:
%
\begin{center}
\begin{tabular}{l}
|% \iffalse
%
% childdoc.dtx Copyright (C) 2017-2018 Niklas Beisert
%
% This work may be distributed and/or modified under the
% conditions of the LaTeX Project Public License, either version 1.3
% of this license or (at your option) any later version.
% The latest version of this license is in
%   http://www.latex-project.org/lppl.txt
% and version 1.3 or later is part of all distributions of LaTeX
% version 2005/12/01 or later.
%
% This work has the LPPL maintenance status `maintained'.
%
% The Current Maintainer of this work is Niklas Beisert.
%
% This work consists of the files childdoc.dtx and childdoc.ins
% and the derived files childdoc.def and cdocsamp.tex with
% cdocsch1.tex, cdocsch2.tex, cdocsdrf.tex, cdocsfn1.tex, cdocsfn2.tex.
%
%<package>\ifdefined\childdocmain\endinput\fi
%<package>\ProvidesFile{childdoc.def}[2018/12/30 v2.0 child document driver]
%<samplemain>\ProvidesFile{cdocsamp.tex}[2018/12/30 v2.0 sample for childdoc]
%<*driver>
%\ProvidesFile{childdoc.drv}[2018/12/30 v2.0 childdoc reference manual file]
\PassOptionsToClass{10pt,a4paper}{article}
\documentclass{ltxdoc}

\usepackage[margin=35mm]{geometry}
\usepackage{hyperref}
\usepackage{hyperxmp}
\usepackage[usenames]{color}

\hypersetup{colorlinks=true}
\hypersetup{pdfstartview=FitH}
\hypersetup{pdfpagemode=UseNone}
\hypersetup{pdfsource={}}
\hypersetup{pdflang={en-UK}}
\hypersetup{pdfcopyright={Copyright 2017-2018 Niklas Beisert.
  This work may be distributed and/or modified under the
  conditions of the LaTeX Project Public License, either version 1.3
  of this license or (at your option) any later version.}}
\hypersetup{pdflicenseurl={http://www.latex-project.org/lppl.txt}}
\hypersetup{pdfcontactaddress={ETH Zurich, ITP, HIT K,
  Wolfgang-Pauli-Strasse 27}}
\hypersetup{pdfcontactpostcode={8093}}
\hypersetup{pdfcontactcity={Zurich}}
\hypersetup{pdfcontactcountry={Switzerland}}
\hypersetup{pdfcontactemail={nbeisert@itp.phys.ethz.ch}}
\hypersetup{pdfcontacturl={http://people.phys.ethz.ch/\xmptilde nbeisert/}}

\newcommand{\secref}[1]{\hyperref[#1]{section \ref*{#1}}}

\parskip1ex
\parindent0pt
\let\olditemize\itemize
\def\itemize{\olditemize\parskip0pt}

\begin{document}

\title{The \textsf{childdoc} Package}
\hypersetup{pdftitle={The childdoc Package}}
\author{Niklas Beisert\\[2ex]
  Institut f\"ur Theoretische Physik\\
  Eidgen\"ossische Technische Hochschule Z\"urich\\
  Wolfgang-Pauli-Strasse 27, 8093 Z\"urich, Switzerland\\[1ex]
  \href{mailto:nbeisert@itp.phys.ethz.ch}
  {\texttt{nbeisert@itp.phys.ethz.ch}}}
\hypersetup{pdfauthor={Niklas Beisert}}
\hypersetup{pdfsubject={Manual for the LaTeX2e Package childdoc}}
\date{30 December 2018, \textsf{v2.0}}
\maketitle

\begin{abstract}\noindent
\textsf{childdoc} is a \LaTeXe{} package
that enables the direct compilation
of document sections included by |\include|
to individual files.
\end{abstract}

\begingroup
\parskip0ex
\tableofcontents
\endgroup

%%%%%%%%%%%%%%%%%%%%%%%%%%%%%%%%%%%%%%%%%%%%%%%%%%%%%%%%%%%%%%%%%%%%%%%%%%%%%%%%
%%%%%%%%%%%%%%%%%%%%%%%%%%%%%%%%%%%%%%%%%%%%%%%%%%%%%%%%%%%%%%%%%%%%%%%%%%%%%%%%
\section{Introduction}

\LaTeX{} provides a mechanism to structure a large document (such as a book)
into a main file and several child files (containing the chapters)
using the |\include| command.
This mechanism is beneficial for documents
which span hundreds of pages in order to
make the source file(s) more manageable.
Moreover, compilation can be restricted to
selected child files by means of the |\includeonly| command.
The latter feature can be used to reduce the compilation time while editing
(this was significantly more useful in the earlier days of \LaTeX{})
or to generate a smaller document which is easier to navigate.
Another application of |\includeonly| is to generate
documents consisting of selected parts of the complete document.

However, there are a few drawbacks of the plain |\include| mechanism:
\begin{itemize}
\item
The child files cannot be compiled on their own,
they can only be compiled via the main file.
A naive editing environment
(such as a text editor with an option
to have the current file processed by \LaTeX)
may require one to switch to the main file before compiling;
attempting to compile the child file produces errors.
\item
The main file must be modified (each time)
to adjust the |\includeonly| command
to the present needs. This easily leaves the main file in a messy state.
\item
The generated document will always carry the filename
of the main document. This is inconvenient if
several child files are to be compiled and
to be kept for distribution.
\end{itemize}

The present package provides a simple interface
to make child files individually compilable by \LaTeX{}.
Compiling a child file then has the same effect as compiling
the main file with an |\includeonly| command
to select the appropriate child.
Moreover the generated document will carry the name of the child
rather than the main file.
This resolves all three above issues.

This feature is meant to make the editing of books,
thesis documents and lecture notes somewhat more convenient.
However, the package can also be used efficiently for
composing a series of documents (such as exercise sheets)
which are typically distributed individually.
It then assists the author in generating the individual documents
(potentially in different versions)
as well as a document containing the collected series.
Another application is in developing style files
or other kinds of included material
where compilation of the style file could redirect
to a sample or test file.

%%%%%%%%%%%%%%%%%%%%%%%%%%%%%%%%%%%%%%%%%%%%%%%%%%%%%%%%%%%%%%%%%%%%%%%%%%%%%%%%
%%%%%%%%%%%%%%%%%%%%%%%%%%%%%%%%%%%%%%%%%%%%%%%%%%%%%%%%%%%%%%%%%%%%%%%%%%%%%%%%
\section{Usage}

First of all, the package \textsf{childdoc} is \emph{not} a standard
\LaTeXe{} |.sty| style file! Therefore it needs to be invoked in
a non-standard way.

%%%%%%%%%%%%%%%%%%%%%%%%%%%%%%%%%%%%%%%%%%%%%%%%%%%%%%%%%%%%%%%%%%%%%%%%%%%%%%%%
\subsection{Included Files}
\label{sec:include}

%%%%%%%%%%%%%%%%%%%%%%%%%%%%%%%%%%%%%%%%
\DescribeMacro{\childdocmain}
To use the package, add the commands
\begin{center}
\begin{tabular}{l}
|\input{childdoc.def}|\\
|\childdocmain{}|\\
\end{tabular}
\end{center}
at the very top of the main \LaTeX{} file,
in particular \emph{before} the |\documentclass| statement!
The argument of |\childdocmain| should be left empty
(but it must be present).

%%%%%%%%%%%%%%%%%%%%%%%%%%%%%%%%%%%%%%%%
\DescribeMacro{\childdocof}
Furthermore, add the commands
\begin{center}
\begin{tabular}{l}
|\input{childdoc.def}|\\
|\childdocof{|\textit{main}|}|\\
\end{tabular}
\end{center}
at the top of every child file \textit{child}
which is included by |\include{|\textit{child}|}|
from within the main file
(or at least for those files to be compiled individually).
The argument \textit{main} must be the filename of the main file.

There are a couple of
considerations in setting up the main and child documents:

%%%%%%%%%%%%%%%%%%%%%%%%%%%%%%%%%%%%%%%%
\paragraph{Restrictions.}

Please note the following restrictions:
\begin{itemize}
\item
|\childdocmain| must be called with one argument \textit{main}
to ensure compatibility with earlier version of the package.
It must either be empty (|\childdocmain{}|)
or precisely match the filename of the main file in which it is specified.
See \secref{sec:detection} for further information.
\item
The filename \textit{main} must be specified without the |.tex| extension.
\item
The filename \textit{main} is case sensitive
(even in case-insensitive file systems)
due to internal string comparison.
\item
The argument \textit{main} should be fully expanded, it cannot be a macro.
\item
Subdirectories and special characters should be avoided in filenames.
\item
The command |\childdocmain{|\textit{main}|}| must be followed by a whitespace.
It should not be followed immediately by another command
or by a comment mark `|%|'.
This is because the \TeX{} parser reads the token immediately following
the argument of |\childdocmain| and puts it
at the beginning of every child section;
however, a white\-space is ignored.
\end{itemize}

%%%%%%%%%%%%%%%%%%%%%%%%%%%%%%%%%%%%%%%%
\paragraph{Content of Main File.}

It is advisable to place all content in the child files included by |\include|.
Any output contained in the main file will appear in all child documents
unless suppressed manually;
it cannot be suppressed automatically by the |\includeonly| directive
and thus should normally be avoided.
A method to include some content in the main file
by means of conditional processing is described in \secref{sec:conditional}.

%%%%%%%%%%%%%%%%%%%%%%%%%%%%%%%%%%%%%%%%
\paragraph{Page Numbering.}

When only a part of the document is compiled,
the appropriate numbering of pages
(as well as other status parameters)
is determined from the |.aux| files.
The latter contain information from previous passes.
However this information needs to propagate through
all intermediate child documents.
Therefore the page numbering in child documents may well
be inconsistent until the complete document is compiled at least once.

A useful (if unconventional) way to always ensure a consistent
page numbering is to restart the numbering in each child document
and denote the pages by `\textit{child}|.|\textit{page}'
where \textit{child} represents the chapter/section number of the child file.
This can be achieved by the command
|\numberwithin{page}{|\textit{child}|}|
of the \textsf{amsmath} package
where \textit{child} can be |chapter| or |section|
depending on the chosen structuring.
Alternatively, one can modify the macro |\thepage| appropriately
and reset the counter |page| at the start of each child file.

%%%%%%%%%%%%%%%%%%%%%%%%%%%%%%%%%%%%%%%%%%%%%%%%%%%%%%%%%%%%%%%%%%%%%%%%%%%%%%%%
\subsection{Conditional Processing}
\label{sec:conditional}

The package provides a mechanism to compile different versions
of a document. To customise the versions further some conditional processing
can come in handy to distinguish which version is being compiled.
The package provides two macros to describe the compilation context:

%%%%%%%%%%%%%%%%%%%%%%%%%%%%%%%%%%%%%%%%
\DescribeMacro{\ifchilddoc}
The conditional |\ifchilddoc| distinguishes between the compilation of
child documents and the main document:
%
\begin{center}
|\ifchilddoc |\textit{child-code}| |[|\||else |\textit{main-code}]| \||fi|
\end{center}

%%%%%%%%%%%%%%%%%%%%%%%%%%%%%%%%%%%%%%%%
\DescribeMacro{\childdocname}
\DescribeMacro{\childdocjob}
The macro |\childdocname| contains the filename (without extension)
of the main or child file being processed.
Note that |\childdocjob| will always contain the name of the main file.

%%%%%%%%%%%%%%%%%%%%%%%%%%%%%%%%%%%%%%%%
\paragraph{Title Page.}

Conditional processing can be used to include a title or banner page
in the main document when proper precautions are taken.
Importantly, the code in the main file should ensure that the page counter
(as well as other status parameters which are stored in the |.aux| files)
takes the same value after the conditional processing.
Otherwise the page numbers may take divergent values
depending on which part is compiled.

For example, a title page could be declared by:
%
\begin{center}
\begin{tabular}{l}
|\ifchilddoc\||else|\\
|\addtocounter{page}{-1}|\\
\textit{code for title page}\\
|\newpage|\\
|\||fi|
\end{tabular}
\end{center}
%
A banner page for the child documents can be generated by:
%
\begin{center}
\begin{tabular}{l}
|\ifchilddoc|\\
|\addtocounter{page}{-1}|\\
\textit{code for banner page}\\
|\newpage|\\
|\||fi|
\end{tabular}
\end{center}
%
Here one could write a message such as:
\begin{center}
|This is the part \childdocname{} of \childdocjob{}.|
\end{center}

%%%%%%%%%%%%%%%%%%%%%%%%%%%%%%%%%%%%%%%%%%%%%%%%%%%%%%%%%%%%%%%%%%%%%%%%%%%%%%%%
\subsection{Flags}
\label{sec:flags}

The package makes it easy to generate different versions
of the main or child documents.
To this end compilation flags can be defined
and assigned different default values.
They will be particularly useful in conjunction
with the forwarding mechanism described in \secref{sec:forward}.

For example, it may be useful to have a flag |\version|
which can be set to |draft| or |final|.
The document source will contain some conditional code
depending on the value of |\version|.
Suppose further, the flag should default to |final| for the main file
and to |draft| for child files
which is a natural assignment for editing the document.
This is achieved by placing the following code
in the preamble of the main document
(below the |\childdocmain| directive):
%
\begin{center}
\begin{tabular}{l}
|\ifchilddoc|\\
|\providecommand{\version}{draft}|\\
|\||else|\\
|\providecommand{\version}{final}|\\
|\||fi|
\end{tabular}
\end{center}
%
The definition by |\providecommand| makes sure
that previous definitions are not overwritten.
Further statements |\providecommand{\version}{...}|
can thus be added before the above code to override it.

For the main file, one might add a line
(between |\childdocmain| and the above block)
%
\begin{center}
|%\ifchilddoc\||else\providecommand{\version}{draft}\||fi|
\end{center}
%
which can be uncommented to produce a draft version.
Likewise one can add a line to the very top of a child file
(above the |\childdocof{|\textit{main}|}| directive)
%
\begin{center}
|%\providecommand{\version}{final}|
\end{center}
%
which can be uncommented to produce the final version of this child document.

%%%%%%%%%%%%%%%%%%%%%%%%%%%%%%%%%%%%%%%%%%%%%%%%%%%%%%%%%%%%%%%%%%%%%%%%%%%%%%%%
\subsection{Forwarding}
\label{sec:forward}

Different versions of the main or child documents
using compilation flags as described in \secref{sec:flags}
can be (permanently) stored in different files
for convenient compilation, viewing and distribution.
To this end, the package defines a command
to pass on compilation to a different file:

%%%%%%%%%%%%%%%%%%%%%%%%%%%%%%%%%%%%%%%%
\DescribeMacro{\childdocforward}
The command |\childdocforward| redirects processing to
another source file:
%
\begin{center}
\begin{tabular}{l}
|\input{childdoc.def}|\\
|\childdocforward[|\textit{main}|]{|\textit{dest}|}|\\
\end{tabular}
\end{center}
%
The argument \textit{dest} is the destination file
(without extension).
It should be the main file or one of the child files.
Note that further \textsf{childdoc} directives
such as |\childdocof| and |\childdocforward|
in the indicated file will be processed in this form.
The optional argument \textit{main}
passes on directly to the main file \textit{main}
while pretending to compile the child \textit{dest}.
This form behaves as if \textit{dest}
issues |\childdocof{|\textit{main}|}| right away,
and no further \textsf{childdoc} directives will be processed.

%%%%%%%%%%%%%%%%%%%%%%%%%%%%%%%%%%%%%%%%
\DescribeMacro{\...prefix}
In the alternative form |\childdocforwardprefix|,
%
\begin{center}
\begin{tabular}{l}
|\input{childdoc.def}|\\
|\childdocforwardprefix[|\textit{main}|]{|\textit{prefix}|}{|\textit{dest}|}|
\end{tabular}
\end{center}
%
the destination file is determined by a pattern
depending on the current file:
To make this work, the current file must be called
`{\textit{prefix}\hspace{0.2em}\textit{suffix}}'
with \textit{prefix} matching precisely the argument.
Processing is then passed on to the file
`{\textit{dest}\hspace{0.2em}\textit{suffix}}'.
Surely, the same effect is achieved by
directly specifying the
argument `{\textit{dest}\hspace{0.2em}\textit{suffix}}'
in the first form.
However, that requires to set up a different file
for each child. With the alternative form of the command
all these files can have exactly the same content
which simplifies setting them up and maintaining them.

For example, the following file |draft.tex|
with a compilation flag |\version| as described in \secref{sec:flags}
compiles the main document as a draft:
%
\begin{center}
\begin{tabular}{l}
|\def\version{draft}|\\
|\input{childdoc.def}|\\
|\childdocforward{|\textit{main}|}|
\end{tabular}
\end{center}
%
Likewise, the following files |final|\textit{nn}|.tex|
compile the final version of the child document
|child|\textit{nn}|.tex|:
%
\begin{center}
\begin{tabular}{l}
|\def\version{final}|\\
|\input{childdoc.def}|\\
|\childdocforwardprefix{final}{child}|
\end{tabular}
\end{center}
%

Note that when several versions of a main file and/or of each child file
are to be generated, it may be convenient to set up a |Makefile| or
shell script to automatise the process.

%%%%%%%%%%%%%%%%%%%%%%%%%%%%%%%%%%%%%%%%%%%%%%%%%%%%%%%%%%%%%%%%%%%%%%%%%%%%%%%%
\subsection{Command Line Processing}
\label{sec:commandline}

The effect of redirection files can also be achieved by invoking
the \LaTeX{} compiler with a more elaborate command line.
Most conveniently this should be done as part
of a shell script or a |Makefile|.

When using \textsf{childdoc} in the main file, the following
command lines effectively perform a redirection
(note that depending on the shell being used,
backslashes may have to be doubled: `|\|' $\to$ `|\\|'):
%
\begin{center}
|... -jobname "|\textit{target}|" |\\|"|[\textit{flags}]%
|\input{childdoc.def}\childdocforward[|\textit{main}|]{|\textit{dest}|}"|
\end{center}
%
Here \textit{target} is the name of the output file,
\textit{main} is the name of the main file
and \textit{dest} is the name of the main or child file to be processed
(all filenames without extensions).
The optional argument \textit{main} can be omitted
if \textit{main} matches \textit{dest}.
Optionally, compilation \textit{flags} can be defined via |\def| commands.
This command line makes the \TeX{} engine believe
it is compiling the file \textit{target}
whose content is specified as the latter parameter.
The provided code then forwards the processing to
\textit{main} or \textit{dest} as described in \secref{sec:forward}.

%%%%%%%%%%%%%%%%%%%%%%%%%%%%%%%%%%%%%%%%%%%%%%%%%%%%%%%%%%%%%%%%%%%%%%%%%%%%%%%%
\subsection{Include by Input}
\label{sec:input}

Including child documents by |\include| has some restrictions by design.
Most notably, the content of a child document always occupies
its own set of pages; pages cannot be shared between child documents.
Usually, this behaviour makes perfect sense
because each child document contain an essential part of the document.
However, in some situations it may be desirable to compose
a document from a collection of parts
without having mandatory page breaks between then.
For this case, the package
provides a mechanism to include parts
by |\input| which can also be processed individually.
However, by construction this mechanism
requires manual handling of the content to be output.

%%%%%%%%%%%%%%%%%%%%%%%%%%%%%%%%%%%%%%%%
\DescribeMacro{\ifchilddocmanual}
The main file should be prepared as usual, see \secref{sec:include}.
However, the document body must make a distinction
between processing of an individual part and of the main document, e.g.:
%
\begin{center}
\begin{tabular}{l}
|\ifchilddocmanual|\\
|\input{\childdocname}|\\
|\||else|\\
\textit{document body with }|\input{|\textit{part}|}|\\
|\||fi|
\end{tabular}
\end{center}
%
The conditional |\ifchilddocmanual| is true whenever
a part to be included by |\input| is being compiled,
and the name of the part is stored in |\childdocname|.

%%%%%%%%%%%%%%%%%%%%%%%%%%%%%%%%%%%%%%%%
\DescribeMacro{\childdocby}
Each part to be included by |\input| should start with:
%
\begin{center}
\begin{tabular}{l}
|\input{childdoc.def}|\\
|\childdocby{|\textit{main}|}|\\
\end{tabular}
\end{center}
%
The directive |\childdocby| is similar to |\childdocof|
described in \secref{sec:include},
but the subsequent selection of content must be done manually.
To that end, both |\ifchilddoc| and |\ifchilddocmanual|
will be true upon processing of a part,
and the name of the part is stored in |\childdocname|.
Note that |\jobname| will be set to the filename of the current part
so that each part receives an individual |.aux| file
that does not interfere with the |.aux| file(s) of the main document.
This behaviour can be altered by the alternative form
|\childdocby[*]{|\textit{main}|}| (with a non-empty optional argument)
which uses the |.aux| file of the main document
by setting |\jobname| to \textit{main}.

%%%%%%%%%%%%%%%%%%%%%%%%%%%%%%%%%%%%%%%%%%%%%%%%%%%%%%%%%%%%%%%%%%%%%%%%%%%%%%%%
\subsection{Driver Development}
\label{sec:driver}

The \textsf{childdoc} mechanism can also be use for the development
of definition files such as \LaTeX{} styles or classes.
This case differs from the above setup with multiple parts
included by |\include| in that no |\includeonly| should be invoked.
This can be achieved by starting the include file
(before |\ProvidesPackage|) with:
%
\begin{center}
\begin{tabular}{l}
|\input{childdoc.def}|\\
|\childdocforward{|\textit{main}|}|\\
\end{tabular}
\end{center}
%
or alternatively with:
%
\begin{center}
\begin{tabular}{l}
|\input{childdoc.def}|\\
|\childdocby{|\textit{main}|}|\\
\end{tabular}
\end{center}
%
Both forms have slightly different effects as described above.
The main file is prepared as usual, see \secref{sec:include}.

%%%%%%%%%%%%%%%%%%%%%%%%%%%%%%%%%%%%%%%%%%%%%%%%%%%%%%%%%%%%%%%%%%%%%%%%%%%%%%%%
\subsection{Legacy Detection}
\label{sec:detection}

The directive |\childdocmain| in the main file can detect
whether the complete document or merely a child is to be compiled
even without using the directive |\childdocof|.
This method is deprecated because it is less robust
and there is no compelling reason to use it;
it is merely provided for backward compatibility
and it may be removed in future versions.

If the detection mechanism is to be used,
it is mandatory to correctly specify
the filename of the main file as the argument of |\childdocmain|:
%
\begin{center}
\begin{tabular}{l}
|\input{childdoc.def}|\\
|\childdocmain{|\textit{main}|}|\\
\end{tabular}
\end{center}
%
If |\jobname| does not match the argument \textit{main} of |\childdocmain|,
it is assumed that |\jobname| points to the child file to be compiled.
When using |\childdocmain| with the main file specified as argument,
it suffices to start a child file
with just |\input{|\textit{main}|}|
without loading of the package and using |\childdocof|.
If instead all processing is done
with the appropriate \textsf{childdoc} directives,
the argument of \textit{main} of |\childdocmain| can be empty.

An alternative version of the command line processing described
in \secref{sec:commandline} using the detection mechanism reads:
%
\begin{center}
|... -jobname "|\textit{target}|" "|[\textit{flags}]%
[|\def\jobname{|\textit{dest}|}|]|\input{|\textit{main}|}"|
\end{center}

%%%%%%%%%%%%%%%%%%%%%%%%%%%%%%%%%%%%%%%%%%%%%%%%%%%%%%%%%%%%%%%%%%%%%%%%%%%%%%%%
\subsection{Manual Code}
\label{sec:manual}

In case one cannot be certain whether the definitions file |childdoc.def|
is installed on the target \TeX{} distribution
and one prefers not to ship it,
it is conceivable to paste a few relevant commands into the sources.

To that end, drop all statements |\input{childdoc.def}|
and perform the replacements as outlined below.
Instead of |\childdocmain{|\textit{main}|}| add the following code
to the top of the main file:
%
\begin{center}
\begin{tabular}{l}
|\||ifdefined\childdocname\endinput\||fi\newif\ifchilddoc|\\
|\edef\childdocname{\scantokens\expandafter{\jobname\noexpand}}|\\
|\def\childdocmain{|\textit{main}|}\||ifx\childdocmain\childdocname\||else|\\
|\childdoctrue\includeonly{\childdocname}\let\jobname\childdocmain\||fi|\\
\end{tabular}
\end{center}
%
Instead of |\childdocof{|\textit{main}|}| just include the main file
at the top of each child file:
%
\begin{center}
|\input{|\textit{main}|}|
\end{center}
%
A simple redirection |\childdocforward{|\textit{dest}|}| is achieved by:
%
\begin{center}
|\def\jobname{|\textit{dest}|}\input{\jobname}|
\end{center}
%
The redirection with prefix
|\childdocforwardprefix[|\textit{prefix}|]{|\textit{dest}|}|
is accomplished by:
%
\begin{center}
\begin{tabular}{l}
|{\edef\jobname{\scantokens\expandafter{\jobname\noexpand}}|\\
|\def\redirectjob |\textit{prefix}|#1~~~{\gdef\jobname{|\textit{dest}|#1}}|\\
|\expandafter\redirectjob\jobname~~~}\input{\jobname}|
\end{tabular}
\end{center}

In an alternative approach,
child documents can be compiled by a specific command line
without additional code or specific definitions:
%
\begin{center}
|... -jobname "|\textit{target}|" "|[\textit{flags}]%
|\includeonly{|\textit{dest}|}\input{|\textit{main}|}"|
\end{center}
%

%%%%%%%%%%%%%%%%%%%%%%%%%%%%%%%%%%%%%%%%%%%%%%%%%%%%%%%%%%%%%%%%%%%%%%%%%%%%%%%%
%%%%%%%%%%%%%%%%%%%%%%%%%%%%%%%%%%%%%%%%%%%%%%%%%%%%%%%%%%%%%%%%%%%%%%%%%%%%%%%%
\section{Information}

%%%%%%%%%%%%%%%%%%%%%%%%%%%%%%%%%%%%%%%%%%%%%%%%%%%%%%%%%%%%%%%%%%%%%%%%%%%%%%%%
\subsection{Copyright}

Copyright \copyright{} 2017--2018 Niklas Beisert

This work may be distributed and/or modified under the
conditions of the \LaTeX{} Project Public License, either version 1.3
of this license or (at your option) any later version.
The latest version of this license is in
  \url{http://www.latex-project.org/lppl.txt}
and version 1.3 or later is part of all distributions of \LaTeX{}
version 2005/12/01 or later.

This work has the LPPL maintenance status `maintained'.

The Current Maintainer of this work is Niklas Beisert.

This work consists of the files |README.txt|, |childdoc.ins| and |childdoc.dtx|
as well as the derived files |childdoc.def|, |cdocsamp.tex|
with |cdocsch1.tex|, |cdocsch2.tex|, |cdocspt3.tex|, |cdocspt4.tex|,
|cdocsdrf.tex|, |cdocsfn1.tex|, |cdocsfn2.tex|
as well as |childdoc.pdf|.

%%%%%%%%%%%%%%%%%%%%%%%%%%%%%%%%%%%%%%%%%%%%%%%%%%%%%%%%%%%%%%%%%%%%%%%%%%%%%%%%
\subsection{Files and Installation}

The package consists of the files:
%
\begin{center}
\begin{tabular}{ll}
    |README.txt|   & readme file \\
    |childdoc.ins| & installation file \\
    |childdoc.dtx| & source file \\
    |childdoc.def| & definition file \\
    |cdocsamp.tex| & sample main file \\
    |cdocsch1.tex| & sample include file \\
    |cdocsch2.tex| & sample include file \\
    |cdocspt3.tex| & sample part file \\
    |cdocspt4.tex| & sample part file \\
    |cdocsdrf.tex| & sample redirection file \\
    |cdocsfn1.tex| & sample redirection file \\
    |cdocsfn2.tex| & sample redirection file \\
    |childdoc.pdf| & manual
\end{tabular}
\end{center}
%
The distribution consists of the files
|README.txt|, |childdoc.ins| and |childdoc.dtx|.
%
\begin{itemize}
\item
Run (pdf)\LaTeX{} on |childdoc.dtx|
to compile the manual |childdoc.pdf| (this file).
\item
Run \LaTeX{} on |childdoc.ins| to create the definitions file |childdoc.def|
and the sample |cdocsamp.tex| with include files
|cdocsch1.tex|, |cdocsch2.tex|, |cdocspt3.tex|, |cdocspt4.tex|,
|cdocsdrf.tex|, |cdocsfn1.tex|, |cdocsfn2.tex|.
Then copy the file |childdoc.def| to an appropriate directory of your \LaTeX{}
distribution, e.g.\ \textit{texmf-root}|/tex/latex/childdoc|.
\end{itemize}

%%%%%%%%%%%%%%%%%%%%%%%%%%%%%%%%%%%%%%%%%%%%%%%%%%%%%%%%%%%%%%%%%%%%%%%%%%%%%%%%
\subsection{Related CTAN Packages}

There are several other packages which offer a similar functionality:
%
\begin{itemize}
\item
The packages
\href{http://ctan.org/pkg/docmute}{\textsf{docmute}},
\href{http://ctan.org/pkg/includex}{\textsf{includex}} and
\href{http://ctan.org/pkg/standalone}{\textsf{standalone}}
provide commands to include only the document body of
a child file thus allowing both files to be compiled individually.
\item
The packages \href{http://ctan.org/pkg/subdocs}{\textsf{subdocs}}
and \href{http://ctan.org/pkg/subfiles}{\textsf{subfiles}}
provide structures in which the main and child documents can be
encapsulated and allowing them to be compiled individually.
The inclusion mechanism is different from the conventional |\include|.
\item
The package \href{http://ctan.org/pkg/combine}{\textsf{combine}}
is an elaborate solution to combine several documents into one.
\end{itemize}
%
See also the CTAN topic \href{http://ctan.org/topic/subdocs}{\textsf{subdocs}}
for further related packages.
The present package differs from the above solutions in that
a document structure constructed with the conventional |\include| mechanism
just needs two extra commands at the top of every file
such that all constituent files can be compiled individually.

%%%%%%%%%%%%%%%%%%%%%%%%%%%%%%%%%%%%%%%%%%%%%%%%%%%%%%%%%%%%%%%%%%%%%%%%%%%%%%%%
%\subsection{Feature Suggestions}
%
%The following is a list of features which may be useful for future
%versions of this package:
%%
%\begin{itemize}
%\item
%\ldots
%\end{itemize}

%%%%%%%%%%%%%%%%%%%%%%%%%%%%%%%%%%%%%%%%%%%%%%%%%%%%%%%%%%%%%%%%%%%%%%%%%%%%%%%%
\subsection{Revision History}

%%%%%%%%%%%%%%%%%%%%%%%%%%%%%%%%%%%%%%%%
\paragraph{v2.0:} 2018/12/30

\begin{itemize}
\item
immediate forward processing
\item
added |\childdocby| mechanism
\item
manual restructured
\end{itemize}

%%%%%%%%%%%%%%%%%%%%%%%%%%%%%%%%%%%%%%%%
\paragraph{v1.6:} 2018/01/17

\begin{itemize}
\item
application for development of include files
\item
corrections to manual
\end{itemize}

%%%%%%%%%%%%%%%%%%%%%%%%%%%%%%%%%%%%%%%%
\paragraph{v1.5:} 2017/05/21

\begin{itemize}
\item
more complete structuring introduced
\item
|\childdocof| introduced
\item
|\childdoc| renamed to |\childdocmain|
\item
|\childredirect| renamed to |\childdocforward| and |\childdocforwardprefix|
and functionality expanded
\end{itemize}

%%%%%%%%%%%%%%%%%%%%%%%%%%%%%%%%%%%%%%%%
\paragraph{v1.0:} 2017/04/27

\begin{itemize}
\item
manual and install package
\item
first version published on CTAN
\end{itemize}

%%%%%%%%%%%%%%%%%%%%%%%%%%%%%%%%%%%%%%%%
\paragraph{v0.6:} 2017/04/26

\begin{itemize}
\item
redirection mechanism added
\end{itemize}

%%%%%%%%%%%%%%%%%%%%%%%%%%%%%%%%%%%%%%%%
\paragraph{v0.5:} 2017/04/26

\begin{itemize}
\item
functionality in definition file
\end{itemize}


%%%%%%%%%%%%%%%%%%%%%%%%%%%%%%%%%%%%%%%%%%%%%%%%%%%%%%%%%%%%%%%%%%%%%%%%%%%%%%%%
%%%%%%%%%%%%%%%%%%%%%%%%%%%%%%%%%%%%%%%%%%%%%%%%%%%%%%%%%%%%%%%%%%%%%%%%%%%%%%%%
%%%%%%%%%%%%%%%%%%%%%%%%%%%%%%%%%%%%%%%%%%%%%%%%%%%%%%%%%%%%%%%%%%%%%%%%%%%%%%%%
\appendix

\settowidth\MacroIndent{\rmfamily\scriptsize 000\ }

 \DocInput{childdoc.dtx}

\end{document}
%</driver>
% \fi
%
% %%%%%%%%%%%%%%%%%%%%%%%%%%%%%%%%%%%%%%%%%%%%%%%%%%%%%%%%%%%%%%%%%%%%%%%%%%%%%%
% %%%%%%%%%%%%%%%%%%%%%%%%%%%%%%%%%%%%%%%%%%%%%%%%%%%%%%%%%%%%%%%%%%%%%%%%%%%%%%
% \section{Sample}
%\iffalse
%<*samplemain>
%\fi
%
% The following presents a sample document
% with two chapters, two parts, a title page,
% a compile flag as well as three forwarding files to set the flag.
% It consists of eight |.tex| files:
% \begin{center}
% \begin{tabular}{ll}
% |cdocsamp.tex|&main file\\
% |cdocsch1.tex|&include file for chapter 1\\
% |cdocsch2.tex|&include file for chapter 2\\
% |cdocspt3.tex|&include file for part 3\\
% |cdocspt4.tex|&include file for part 4\\
% |cdocsdrf.tex|&forwarding file for main file in draft mode\\
% |cdocsfi1.tex|&forwarding file for final version of chapter 1\\
% |cdocsfi2.tex|&forwarding file for final version of chapter 2\\
% \end{tabular}
% \end{center}
% Each of the eight files can be compiled directly by the \LaTeX{} compiler.
%
% %%%%%%%%%%%%%%%%%%%%%%%%%%%%%%%%%%%%%%
% \paragraph{Main File.}
%
% The main file is called |cdocsamp.tex|.
%
% Load the \textsf{childdoc} definitions and
% declare the filename for the main document:
%    \begin{macrocode}
\input{childdoc.def}
\childdocmain{}
%    \end{macrocode}

% Optional override for |\version| flag:
%    \begin{macrocode}
%%\ifchilddoc\else\providecommand{\version}{draft}\fi
%    \end{macrocode}

% Define the default values for the |\version| flag
% (|final| for the main file and |draft| for childs):
%    \begin{macrocode}
\ifchilddoc
\providecommand{\version}{draft}
\else
\providecommand{\version}{final}
\fi
%    \end{macrocode}

% Load the standard document class:
%    \begin{macrocode}
\documentclass[12pt]{article}
%    \end{macrocode}

% Start the document body:
%    \begin{macrocode}
\begin{document}
%    \end{macrocode}

% Declare a title page.
% Print title, part of document being processed and version flag:
%    \begin{macrocode}
\addtocounter{page}{-1}
\begin{center}
{\LARGE\bfseries{}childdoc example\par}
\vspace{1cm}
\ifchilddoc
\ifchilddocmanual part\else chapter\fi:
`\childdocname' of `\childdocjob'\par
\else
main document: `\childdocjob'\par
\fi
version: \version\par
\end{center}
\newpage
%    \end{macrocode}

% Manually include selected file,
% otherwise process as usual:
%    \begin{macrocode}
\ifchilddocmanual
\section*{part `\childdocname'}
\input{\childdocname}
\else
%    \end{macrocode}

% Include the two chapters:
%    \begin{macrocode}
\include{cdocsch1}
\include{cdocsch2}
%    \end{macrocode}

% Include the two parts unless only chapters should be displayed:
%    \begin{macrocode}
\ifchilddoc\else
\section{part three}
\input{cdocspt3}
\section{part four}
\input{cdocspt4}
\fi
%    \end{macrocode}

% Process as usual until here:
%    \begin{macrocode}
\fi
%    \end{macrocode}

% End of document body:
%    \begin{macrocode}
\end{document}
%    \end{macrocode}
%\iffalse
%</samplemain>
%\fi
%
% %%%%%%%%%%%%%%%%%%%%%%%%%%%%%%%%%%%%%%
% \paragraph{Chapter Include Files.}
%
% The include files are called |cdocsch1.tex| and |cdocsch2.tex|.
%
%\iffalse
%<*samplechap1|samplechap2>
%\fi

% Optional override for |\version| flag:
%    \begin{macrocode}
%%\providecommand{\version}{final}
%    \end{macrocode}

% Include the main document:
%    \begin{macrocode}
\input{childdoc.def}
\childdocof{cdocsamp}
%    \end{macrocode}

%\iffalse
%</samplechap1|samplechap2>
%\fi
%
%\iffalse
%<*samplechap1>
%\fi
% Some text for chapter 1:
%    \begin{macrocode}
\section{one}
some text in chapter one
%    \end{macrocode}

%\iffalse
%</samplechap1>
%\fi
% Some text for chapter 2:
%\iffalse
%<*samplechap2>
%\fi
%    \begin{macrocode}
\section{two}
more text in chapter two
%    \end{macrocode}

%\iffalse
%</samplechap2>
%\fi
%
% %%%%%%%%%%%%%%%%%%%%%%%%%%%%%%%%%%%%%%
% \paragraph{Part Include Files.}
%
% The include files are called |cdocspt3.tex| and |cdocspt4.tex|.
%
%\iffalse
%<*samplepart3|samplepart4>
%\fi

% Optional override for |\version| flag:
%    \begin{macrocode}
%%\providecommand{\version}{final}
%    \end{macrocode}

% Include the main document:
%    \begin{macrocode}
\input{childdoc.def}
\childdocby{cdocsamp}
%    \end{macrocode}

%\iffalse
%</samplepart3|samplepart4>
%\fi
%
%\iffalse
%<*samplepart3>
%\fi
% Some text for part 3:
%    \begin{macrocode}
some text in part three
%    \end{macrocode}

%\iffalse
%</samplepart3>
%\fi
% Some text for part 4:
%\iffalse
%<*samplepart4>
%\fi
%    \begin{macrocode}
more text in part four
%    \end{macrocode}

%\iffalse
%</samplepart4>
%\fi
%
% %%%%%%%%%%%%%%%%%%%%%%%%%%%%%%%%%%%%%%
% \paragraph{Forwarding for a Complete Draft.}
%
% The following forwarding file |cdocsdrf.tex|
% compiles the main document in draft mode:
%\iffalse
%<*sampledraft>
%\fi
%    \begin{macrocode}
\def\version{draft}
\input{childdoc.def}
\childdocforward{cdocsamp}
%    \end{macrocode}

%\iffalse
%</sampledraft>
%\fi
%
% %%%%%%%%%%%%%%%%%%%%%%%%%%%%%%%%%%%%%%
% \paragraph{Forwarding for Final Version of the Chapters.}
%
% The following forwarding files |cdocsfn1.tex| and |cdocsfn2.tex|
% (with identical content)
% compile the final versions of the child documents
% |cdocsch1.tex| and |cdocsch2.tex|, respectively:
%\iffalse
%<*samplefinal>
%\fi
%    \begin{macrocode}
\def\version{final}
\input{childdoc.def}
\childdocforwardprefix[cdocsamp]{cdocsfn}{cdocsch}
%    \end{macrocode}

%\iffalse
%</samplefinal>
%\fi
%
% %%%%%%%%%%%%%%%%%%%%%%%%%%%%%%%%%%%%%%
% \paragraph{Command Line Processing.}
%
% The following three command lines generate the output files
% |cdocscld|, |cdocscl1| and |cdocscl2|
% which should be identical to
% |cdocsdrf|, |cdocsch1| and |cdocsfn2|, respectively:
% \begin{center}
% \begin{tabular}{l}
% |latex -jobname cdocscld \|\\
% |  "\def\version{draft}\input{childdoc.def}\childdocforward{cdocsamp}"|\\
% |latex -jobname cdocscl1 \|\\
% |  "\input{childdoc.def}\childdocforward[cdocsamp]{cdocsch1}"|\\
% |latex -jobname cdocscl2 \|\\
% |  "\def\version{final}\input{childdoc.def}\childdocforward{cdocsch2}"|
% \end{tabular}
% \end{center}
% Note that the trailing backslash on each first line
% merely continues the input to the second line
% (for convenient cut ant paste).
% Furthermore, the command |latex| can be replaced by any
% of its alternative versions such as |pdflatex|.
%
% %%%%%%%%%%%%%%%%%%%%%%%%%%%%%%%%%%%%%%%%%%%%%%%%%%%%%%%%%%%%%%%%%%%%%%%%%%%%%%
% %%%%%%%%%%%%%%%%%%%%%%%%%%%%%%%%%%%%%%%%%%%%%%%%%%%%%%%%%%%%%%%%%%%%%%%%%%%%%%
% \section{Implementation}
%\iffalse
%<*package>
%\fi
%
% This section describes the definitions file |childdoc.def|.

% The definitions cannot be loaded using |\usepackage| or |\RequirePackage|
% which has a mechanism to prevent loading a style file more than once.
% When loading the definitions by means of |\input|
% multiple instances have to be prevented manually:
%\iffalse
%This code needs to be before the `\ProvidesFile' directive
%which is defined at the beginning of this file.
%Therefore it is also placed there and commented out here.
%</package>
%<*discard>
%\fi
%    \begin{macrocode}
\ifdefined\childdocmain\endinput\fi
%    \end{macrocode}
%\iffalse
%</discard>
%<*package>
%\fi
%
% \macro{\ifchilddoc}
% \macro{\ifchilddocmanual}
% The conditional |\ifchilddoc| tells whether a
% child (true) or main (false) document is being compiled.
% The conditional |\ifchilddocmanual| tells whether
% the |\includeonly| mechanism is used (false) or
% the selection of child files must be performed manually (true).
% The definitions initialise to false:
%    \begin{macrocode}
\newif\ifchilddoc
\newif\ifchilddocmanual
%    \end{macrocode}

% \macro{\childdocname}
% \macro{\childdocjob}
% The macro |\childdocname| stores the name of the main document
% to be compiled. The macro |\childdocjob| stores the name of
% the document on which the \LaTeX{} compiler was originally invoked.
% The content of |\jobname| cannot be compared
% to filenames specified in the source due to different catcodes.
% The following code rescans |\jobname|, stores the result
% in |\childdocname| and saves a copy in |\childdocjob|:
%    \begin{macrocode}
\edef\childdocname{\scantokens\expandafter{\jobname\noexpand}}
\let\childdocjob\childdocname
%    \end{macrocode}

% \macro{\childdocdisable}
% The macro |\childdocdisable| prevents the main file
% from being processed more than once.
% At this stage, the main document command |\childdocmain|
% is assumed to be called once again where it should do nothing.
% Any subsequent call to it should prevent
% a secondary processing of the main document
% It overwrites the forwarding commands
% |\childdocof| and |\childdocforward|
% with empty macros to prevent further inclusions of the main document:
%    \begin{macrocode}
\newcommand{\childdocdisable}
{
  \renewcommand{\childdocmain}[1]{\renewcommand{\childdocmain}[1]{\endinput}}
  \renewcommand{\childdocof}[1]{}
  \renewcommand{\childdocby}[2][]{}
  \renewcommand{\childdocforward}[2][]{}
  \renewcommand{\childdocdisable}{}
}
%    \end{macrocode}

% \macro{\childdocmain}
% The macro |\childdocmain| is to be called at the top of the main file
% with nothing or the main filename (without extension) as argument.
% First, it breaks loops.
% If the argument is not empty and does not match |\childdocname|
% (which is set by the first inclusion of |childdoc.def|),
% |\ifchilddoc| is set to true, |\includeonly| is applied to the child file
% and |\jobname| is set to the main file
% (for proper handling of |.aux| files):
%    \begin{macrocode}
\newcommand{\childdocmain}[1]
{
  \childdocdisable\childdocmain{}
  \if?#1?\else
    \begingroup
      \def\childdoctmp{#1}
      \ifx\childdoctmp\childdocname
        \def\childdoctmp{}
      \else
        \def\childdoctmp
        {
          \childdoctrue
          \includeonly{\childdocname}
          \def\childdocjob{#1}
          \def\jobname{#1}
        }
      \fi
      \expandafter
    \endgroup
    \childdoctmp
  \fi
}
%    \end{macrocode}

% \macro{\childdocof}
% The command |\childdocof| redirects
% compilation to the main file |#1|.
%    \begin{macrocode}
\newcommand{\childdocof}[1]
{
  \childdocdisable
  \childdoctrue
  \includeonly{\childdocname}
  \def\jobname{#1}
  \def\childdocjob{#1}
  \input{#1}
}
%    \end{macrocode}

% \macro{\childdocby}
% The command |\childdocby| ....
%    \begin{macrocode}
\newcommand{\childdocby}[2][]
{
  \childdocdisable
  \childdoctrue
  \childdocmanualtrue
  \if?#1?\else
    \def\jobname{#2}
  \fi
  \def\childdocjob{#2}
  \input{#2}
  \endinput
}
%    \end{macrocode}

% \macro{\childdocforward}
% The command |\childdocforward| redirects
% compilation to the main file or
% (if the optional argument is given) a child file.
% Parameters are set as if the main file
% or a child file starting with |\childdocof| was compiled.
% Then compilation is handed over to the main file:
%    \begin{macrocode}
\newcommand{\childdocforward}[2][]
{
  \begingroup
    \if?#1?
      \def\childdoctmp
      {
        \def\childdocname{#2}
        \def\childdocjob{#2}
        \def\jobname{#2}
        \input{#2}
        \endinput
      }
    \else
      \def\childdoctmp
      {
        \childdocdisable
        \def\childdocname{#2}
        \childdoctrue
        \includeonly{#2}
        \def\childdocjob{#1}
        \def\jobname{#1}
        \input{#1}
        \endinput
      }
    \fi
    \expandafter
  \endgroup
  \childdoctmp
}
%    \end{macrocode}

% \macro{\childdocforwardprefix}
% The command |\childdocforwardprefix| redirects
% compilation to the main or a child file by means of a pattern.
% The prefix |#1| in the current filename is replaced by |#2|
% and the suffix of the current filename is kept
% (it is assumed that the filename does not contain the substring `|~~~|'
% which is used as a delimiter).
% Compilation is handed over to the new file by |\childdocforward|:
%    \begin{macrocode}
\newcommand{\childdocforwardprefix}[3][]
{
  \begingroup
    \def\childdocextract #2##1~~~{\def\childdoctmp{\childdocforward[#1]{#3##1}}}
    \expandafter\childdocextract\childdocname~~~
    \expandafter
  \endgroup
  \childdoctmp
}
%    \end{macrocode}

% \macro{\childdoc}
% The deprecated macro |\childdoc| is a legacy version of |\childdocmain|:
%    \begin{macrocode}
\newcommand{\childdoc}{\childdocmain}
%    \end{macrocode}

% \macro{\childdocredirect}
% The deprecated macro |\childdocredirect| is a legacy version
% of |\childdocforward| and |\childdocforwardprefix|:
%    \begin{macrocode}
\newcommand{\childdocredirect}[2][]
{
  \begingroup
    \if?#1?
      \def\childdoctmp{\childdocforward{#2}}
    \else
      \def\childdoctmp{\childdocforwardprefix{#1}{#2}}
    \fi
    \expandafter
  \endgroup
  \childdoctmp
}
%    \end{macrocode}

%\iffalse
%</package>
%\fi
%
\endinput
|\\
|\childdocby{|\textit{main}|}|\\
\end{tabular}
\end{center}
%
Both forms have slightly different effects as described above.
The main file is prepared as usual, see \secref{sec:include}.

%%%%%%%%%%%%%%%%%%%%%%%%%%%%%%%%%%%%%%%%%%%%%%%%%%%%%%%%%%%%%%%%%%%%%%%%%%%%%%%%
\subsection{Legacy Detection}
\label{sec:detection}

The directive |\childdocmain| in the main file can detect
whether the complete document or merely a child is to be compiled
even without using the directive |\childdocof|.
This method is deprecated because it is less robust
and there is no compelling reason to use it;
it is merely provided for backward compatibility
and it may be removed in future versions.

If the detection mechanism is to be used,
it is mandatory to correctly specify
the filename of the main file as the argument of |\childdocmain|:
%
\begin{center}
\begin{tabular}{l}
|% \iffalse
%
% childdoc.dtx Copyright (C) 2017-2018 Niklas Beisert
%
% This work may be distributed and/or modified under the
% conditions of the LaTeX Project Public License, either version 1.3
% of this license or (at your option) any later version.
% The latest version of this license is in
%   http://www.latex-project.org/lppl.txt
% and version 1.3 or later is part of all distributions of LaTeX
% version 2005/12/01 or later.
%
% This work has the LPPL maintenance status `maintained'.
%
% The Current Maintainer of this work is Niklas Beisert.
%
% This work consists of the files childdoc.dtx and childdoc.ins
% and the derived files childdoc.def and cdocsamp.tex with
% cdocsch1.tex, cdocsch2.tex, cdocsdrf.tex, cdocsfn1.tex, cdocsfn2.tex.
%
%<package>\ifdefined\childdocmain\endinput\fi
%<package>\ProvidesFile{childdoc.def}[2018/12/30 v2.0 child document driver]
%<samplemain>\ProvidesFile{cdocsamp.tex}[2018/12/30 v2.0 sample for childdoc]
%<*driver>
%\ProvidesFile{childdoc.drv}[2018/12/30 v2.0 childdoc reference manual file]
\PassOptionsToClass{10pt,a4paper}{article}
\documentclass{ltxdoc}

\usepackage[margin=35mm]{geometry}
\usepackage{hyperref}
\usepackage{hyperxmp}
\usepackage[usenames]{color}

\hypersetup{colorlinks=true}
\hypersetup{pdfstartview=FitH}
\hypersetup{pdfpagemode=UseNone}
\hypersetup{pdfsource={}}
\hypersetup{pdflang={en-UK}}
\hypersetup{pdfcopyright={Copyright 2017-2018 Niklas Beisert.
  This work may be distributed and/or modified under the
  conditions of the LaTeX Project Public License, either version 1.3
  of this license or (at your option) any later version.}}
\hypersetup{pdflicenseurl={http://www.latex-project.org/lppl.txt}}
\hypersetup{pdfcontactaddress={ETH Zurich, ITP, HIT K,
  Wolfgang-Pauli-Strasse 27}}
\hypersetup{pdfcontactpostcode={8093}}
\hypersetup{pdfcontactcity={Zurich}}
\hypersetup{pdfcontactcountry={Switzerland}}
\hypersetup{pdfcontactemail={nbeisert@itp.phys.ethz.ch}}
\hypersetup{pdfcontacturl={http://people.phys.ethz.ch/\xmptilde nbeisert/}}

\newcommand{\secref}[1]{\hyperref[#1]{section \ref*{#1}}}

\parskip1ex
\parindent0pt
\let\olditemize\itemize
\def\itemize{\olditemize\parskip0pt}

\begin{document}

\title{The \textsf{childdoc} Package}
\hypersetup{pdftitle={The childdoc Package}}
\author{Niklas Beisert\\[2ex]
  Institut f\"ur Theoretische Physik\\
  Eidgen\"ossische Technische Hochschule Z\"urich\\
  Wolfgang-Pauli-Strasse 27, 8093 Z\"urich, Switzerland\\[1ex]
  \href{mailto:nbeisert@itp.phys.ethz.ch}
  {\texttt{nbeisert@itp.phys.ethz.ch}}}
\hypersetup{pdfauthor={Niklas Beisert}}
\hypersetup{pdfsubject={Manual for the LaTeX2e Package childdoc}}
\date{30 December 2018, \textsf{v2.0}}
\maketitle

\begin{abstract}\noindent
\textsf{childdoc} is a \LaTeXe{} package
that enables the direct compilation
of document sections included by |\include|
to individual files.
\end{abstract}

\begingroup
\parskip0ex
\tableofcontents
\endgroup

%%%%%%%%%%%%%%%%%%%%%%%%%%%%%%%%%%%%%%%%%%%%%%%%%%%%%%%%%%%%%%%%%%%%%%%%%%%%%%%%
%%%%%%%%%%%%%%%%%%%%%%%%%%%%%%%%%%%%%%%%%%%%%%%%%%%%%%%%%%%%%%%%%%%%%%%%%%%%%%%%
\section{Introduction}

\LaTeX{} provides a mechanism to structure a large document (such as a book)
into a main file and several child files (containing the chapters)
using the |\include| command.
This mechanism is beneficial for documents
which span hundreds of pages in order to
make the source file(s) more manageable.
Moreover, compilation can be restricted to
selected child files by means of the |\includeonly| command.
The latter feature can be used to reduce the compilation time while editing
(this was significantly more useful in the earlier days of \LaTeX{})
or to generate a smaller document which is easier to navigate.
Another application of |\includeonly| is to generate
documents consisting of selected parts of the complete document.

However, there are a few drawbacks of the plain |\include| mechanism:
\begin{itemize}
\item
The child files cannot be compiled on their own,
they can only be compiled via the main file.
A naive editing environment
(such as a text editor with an option
to have the current file processed by \LaTeX)
may require one to switch to the main file before compiling;
attempting to compile the child file produces errors.
\item
The main file must be modified (each time)
to adjust the |\includeonly| command
to the present needs. This easily leaves the main file in a messy state.
\item
The generated document will always carry the filename
of the main document. This is inconvenient if
several child files are to be compiled and
to be kept for distribution.
\end{itemize}

The present package provides a simple interface
to make child files individually compilable by \LaTeX{}.
Compiling a child file then has the same effect as compiling
the main file with an |\includeonly| command
to select the appropriate child.
Moreover the generated document will carry the name of the child
rather than the main file.
This resolves all three above issues.

This feature is meant to make the editing of books,
thesis documents and lecture notes somewhat more convenient.
However, the package can also be used efficiently for
composing a series of documents (such as exercise sheets)
which are typically distributed individually.
It then assists the author in generating the individual documents
(potentially in different versions)
as well as a document containing the collected series.
Another application is in developing style files
or other kinds of included material
where compilation of the style file could redirect
to a sample or test file.

%%%%%%%%%%%%%%%%%%%%%%%%%%%%%%%%%%%%%%%%%%%%%%%%%%%%%%%%%%%%%%%%%%%%%%%%%%%%%%%%
%%%%%%%%%%%%%%%%%%%%%%%%%%%%%%%%%%%%%%%%%%%%%%%%%%%%%%%%%%%%%%%%%%%%%%%%%%%%%%%%
\section{Usage}

First of all, the package \textsf{childdoc} is \emph{not} a standard
\LaTeXe{} |.sty| style file! Therefore it needs to be invoked in
a non-standard way.

%%%%%%%%%%%%%%%%%%%%%%%%%%%%%%%%%%%%%%%%%%%%%%%%%%%%%%%%%%%%%%%%%%%%%%%%%%%%%%%%
\subsection{Included Files}
\label{sec:include}

%%%%%%%%%%%%%%%%%%%%%%%%%%%%%%%%%%%%%%%%
\DescribeMacro{\childdocmain}
To use the package, add the commands
\begin{center}
\begin{tabular}{l}
|\input{childdoc.def}|\\
|\childdocmain{}|\\
\end{tabular}
\end{center}
at the very top of the main \LaTeX{} file,
in particular \emph{before} the |\documentclass| statement!
The argument of |\childdocmain| should be left empty
(but it must be present).

%%%%%%%%%%%%%%%%%%%%%%%%%%%%%%%%%%%%%%%%
\DescribeMacro{\childdocof}
Furthermore, add the commands
\begin{center}
\begin{tabular}{l}
|\input{childdoc.def}|\\
|\childdocof{|\textit{main}|}|\\
\end{tabular}
\end{center}
at the top of every child file \textit{child}
which is included by |\include{|\textit{child}|}|
from within the main file
(or at least for those files to be compiled individually).
The argument \textit{main} must be the filename of the main file.

There are a couple of
considerations in setting up the main and child documents:

%%%%%%%%%%%%%%%%%%%%%%%%%%%%%%%%%%%%%%%%
\paragraph{Restrictions.}

Please note the following restrictions:
\begin{itemize}
\item
|\childdocmain| must be called with one argument \textit{main}
to ensure compatibility with earlier version of the package.
It must either be empty (|\childdocmain{}|)
or precisely match the filename of the main file in which it is specified.
See \secref{sec:detection} for further information.
\item
The filename \textit{main} must be specified without the |.tex| extension.
\item
The filename \textit{main} is case sensitive
(even in case-insensitive file systems)
due to internal string comparison.
\item
The argument \textit{main} should be fully expanded, it cannot be a macro.
\item
Subdirectories and special characters should be avoided in filenames.
\item
The command |\childdocmain{|\textit{main}|}| must be followed by a whitespace.
It should not be followed immediately by another command
or by a comment mark `|%|'.
This is because the \TeX{} parser reads the token immediately following
the argument of |\childdocmain| and puts it
at the beginning of every child section;
however, a white\-space is ignored.
\end{itemize}

%%%%%%%%%%%%%%%%%%%%%%%%%%%%%%%%%%%%%%%%
\paragraph{Content of Main File.}

It is advisable to place all content in the child files included by |\include|.
Any output contained in the main file will appear in all child documents
unless suppressed manually;
it cannot be suppressed automatically by the |\includeonly| directive
and thus should normally be avoided.
A method to include some content in the main file
by means of conditional processing is described in \secref{sec:conditional}.

%%%%%%%%%%%%%%%%%%%%%%%%%%%%%%%%%%%%%%%%
\paragraph{Page Numbering.}

When only a part of the document is compiled,
the appropriate numbering of pages
(as well as other status parameters)
is determined from the |.aux| files.
The latter contain information from previous passes.
However this information needs to propagate through
all intermediate child documents.
Therefore the page numbering in child documents may well
be inconsistent until the complete document is compiled at least once.

A useful (if unconventional) way to always ensure a consistent
page numbering is to restart the numbering in each child document
and denote the pages by `\textit{child}|.|\textit{page}'
where \textit{child} represents the chapter/section number of the child file.
This can be achieved by the command
|\numberwithin{page}{|\textit{child}|}|
of the \textsf{amsmath} package
where \textit{child} can be |chapter| or |section|
depending on the chosen structuring.
Alternatively, one can modify the macro |\thepage| appropriately
and reset the counter |page| at the start of each child file.

%%%%%%%%%%%%%%%%%%%%%%%%%%%%%%%%%%%%%%%%%%%%%%%%%%%%%%%%%%%%%%%%%%%%%%%%%%%%%%%%
\subsection{Conditional Processing}
\label{sec:conditional}

The package provides a mechanism to compile different versions
of a document. To customise the versions further some conditional processing
can come in handy to distinguish which version is being compiled.
The package provides two macros to describe the compilation context:

%%%%%%%%%%%%%%%%%%%%%%%%%%%%%%%%%%%%%%%%
\DescribeMacro{\ifchilddoc}
The conditional |\ifchilddoc| distinguishes between the compilation of
child documents and the main document:
%
\begin{center}
|\ifchilddoc |\textit{child-code}| |[|\||else |\textit{main-code}]| \||fi|
\end{center}

%%%%%%%%%%%%%%%%%%%%%%%%%%%%%%%%%%%%%%%%
\DescribeMacro{\childdocname}
\DescribeMacro{\childdocjob}
The macro |\childdocname| contains the filename (without extension)
of the main or child file being processed.
Note that |\childdocjob| will always contain the name of the main file.

%%%%%%%%%%%%%%%%%%%%%%%%%%%%%%%%%%%%%%%%
\paragraph{Title Page.}

Conditional processing can be used to include a title or banner page
in the main document when proper precautions are taken.
Importantly, the code in the main file should ensure that the page counter
(as well as other status parameters which are stored in the |.aux| files)
takes the same value after the conditional processing.
Otherwise the page numbers may take divergent values
depending on which part is compiled.

For example, a title page could be declared by:
%
\begin{center}
\begin{tabular}{l}
|\ifchilddoc\||else|\\
|\addtocounter{page}{-1}|\\
\textit{code for title page}\\
|\newpage|\\
|\||fi|
\end{tabular}
\end{center}
%
A banner page for the child documents can be generated by:
%
\begin{center}
\begin{tabular}{l}
|\ifchilddoc|\\
|\addtocounter{page}{-1}|\\
\textit{code for banner page}\\
|\newpage|\\
|\||fi|
\end{tabular}
\end{center}
%
Here one could write a message such as:
\begin{center}
|This is the part \childdocname{} of \childdocjob{}.|
\end{center}

%%%%%%%%%%%%%%%%%%%%%%%%%%%%%%%%%%%%%%%%%%%%%%%%%%%%%%%%%%%%%%%%%%%%%%%%%%%%%%%%
\subsection{Flags}
\label{sec:flags}

The package makes it easy to generate different versions
of the main or child documents.
To this end compilation flags can be defined
and assigned different default values.
They will be particularly useful in conjunction
with the forwarding mechanism described in \secref{sec:forward}.

For example, it may be useful to have a flag |\version|
which can be set to |draft| or |final|.
The document source will contain some conditional code
depending on the value of |\version|.
Suppose further, the flag should default to |final| for the main file
and to |draft| for child files
which is a natural assignment for editing the document.
This is achieved by placing the following code
in the preamble of the main document
(below the |\childdocmain| directive):
%
\begin{center}
\begin{tabular}{l}
|\ifchilddoc|\\
|\providecommand{\version}{draft}|\\
|\||else|\\
|\providecommand{\version}{final}|\\
|\||fi|
\end{tabular}
\end{center}
%
The definition by |\providecommand| makes sure
that previous definitions are not overwritten.
Further statements |\providecommand{\version}{...}|
can thus be added before the above code to override it.

For the main file, one might add a line
(between |\childdocmain| and the above block)
%
\begin{center}
|%\ifchilddoc\||else\providecommand{\version}{draft}\||fi|
\end{center}
%
which can be uncommented to produce a draft version.
Likewise one can add a line to the very top of a child file
(above the |\childdocof{|\textit{main}|}| directive)
%
\begin{center}
|%\providecommand{\version}{final}|
\end{center}
%
which can be uncommented to produce the final version of this child document.

%%%%%%%%%%%%%%%%%%%%%%%%%%%%%%%%%%%%%%%%%%%%%%%%%%%%%%%%%%%%%%%%%%%%%%%%%%%%%%%%
\subsection{Forwarding}
\label{sec:forward}

Different versions of the main or child documents
using compilation flags as described in \secref{sec:flags}
can be (permanently) stored in different files
for convenient compilation, viewing and distribution.
To this end, the package defines a command
to pass on compilation to a different file:

%%%%%%%%%%%%%%%%%%%%%%%%%%%%%%%%%%%%%%%%
\DescribeMacro{\childdocforward}
The command |\childdocforward| redirects processing to
another source file:
%
\begin{center}
\begin{tabular}{l}
|\input{childdoc.def}|\\
|\childdocforward[|\textit{main}|]{|\textit{dest}|}|\\
\end{tabular}
\end{center}
%
The argument \textit{dest} is the destination file
(without extension).
It should be the main file or one of the child files.
Note that further \textsf{childdoc} directives
such as |\childdocof| and |\childdocforward|
in the indicated file will be processed in this form.
The optional argument \textit{main}
passes on directly to the main file \textit{main}
while pretending to compile the child \textit{dest}.
This form behaves as if \textit{dest}
issues |\childdocof{|\textit{main}|}| right away,
and no further \textsf{childdoc} directives will be processed.

%%%%%%%%%%%%%%%%%%%%%%%%%%%%%%%%%%%%%%%%
\DescribeMacro{\...prefix}
In the alternative form |\childdocforwardprefix|,
%
\begin{center}
\begin{tabular}{l}
|\input{childdoc.def}|\\
|\childdocforwardprefix[|\textit{main}|]{|\textit{prefix}|}{|\textit{dest}|}|
\end{tabular}
\end{center}
%
the destination file is determined by a pattern
depending on the current file:
To make this work, the current file must be called
`{\textit{prefix}\hspace{0.2em}\textit{suffix}}'
with \textit{prefix} matching precisely the argument.
Processing is then passed on to the file
`{\textit{dest}\hspace{0.2em}\textit{suffix}}'.
Surely, the same effect is achieved by
directly specifying the
argument `{\textit{dest}\hspace{0.2em}\textit{suffix}}'
in the first form.
However, that requires to set up a different file
for each child. With the alternative form of the command
all these files can have exactly the same content
which simplifies setting them up and maintaining them.

For example, the following file |draft.tex|
with a compilation flag |\version| as described in \secref{sec:flags}
compiles the main document as a draft:
%
\begin{center}
\begin{tabular}{l}
|\def\version{draft}|\\
|\input{childdoc.def}|\\
|\childdocforward{|\textit{main}|}|
\end{tabular}
\end{center}
%
Likewise, the following files |final|\textit{nn}|.tex|
compile the final version of the child document
|child|\textit{nn}|.tex|:
%
\begin{center}
\begin{tabular}{l}
|\def\version{final}|\\
|\input{childdoc.def}|\\
|\childdocforwardprefix{final}{child}|
\end{tabular}
\end{center}
%

Note that when several versions of a main file and/or of each child file
are to be generated, it may be convenient to set up a |Makefile| or
shell script to automatise the process.

%%%%%%%%%%%%%%%%%%%%%%%%%%%%%%%%%%%%%%%%%%%%%%%%%%%%%%%%%%%%%%%%%%%%%%%%%%%%%%%%
\subsection{Command Line Processing}
\label{sec:commandline}

The effect of redirection files can also be achieved by invoking
the \LaTeX{} compiler with a more elaborate command line.
Most conveniently this should be done as part
of a shell script or a |Makefile|.

When using \textsf{childdoc} in the main file, the following
command lines effectively perform a redirection
(note that depending on the shell being used,
backslashes may have to be doubled: `|\|' $\to$ `|\\|'):
%
\begin{center}
|... -jobname "|\textit{target}|" |\\|"|[\textit{flags}]%
|\input{childdoc.def}\childdocforward[|\textit{main}|]{|\textit{dest}|}"|
\end{center}
%
Here \textit{target} is the name of the output file,
\textit{main} is the name of the main file
and \textit{dest} is the name of the main or child file to be processed
(all filenames without extensions).
The optional argument \textit{main} can be omitted
if \textit{main} matches \textit{dest}.
Optionally, compilation \textit{flags} can be defined via |\def| commands.
This command line makes the \TeX{} engine believe
it is compiling the file \textit{target}
whose content is specified as the latter parameter.
The provided code then forwards the processing to
\textit{main} or \textit{dest} as described in \secref{sec:forward}.

%%%%%%%%%%%%%%%%%%%%%%%%%%%%%%%%%%%%%%%%%%%%%%%%%%%%%%%%%%%%%%%%%%%%%%%%%%%%%%%%
\subsection{Include by Input}
\label{sec:input}

Including child documents by |\include| has some restrictions by design.
Most notably, the content of a child document always occupies
its own set of pages; pages cannot be shared between child documents.
Usually, this behaviour makes perfect sense
because each child document contain an essential part of the document.
However, in some situations it may be desirable to compose
a document from a collection of parts
without having mandatory page breaks between then.
For this case, the package
provides a mechanism to include parts
by |\input| which can also be processed individually.
However, by construction this mechanism
requires manual handling of the content to be output.

%%%%%%%%%%%%%%%%%%%%%%%%%%%%%%%%%%%%%%%%
\DescribeMacro{\ifchilddocmanual}
The main file should be prepared as usual, see \secref{sec:include}.
However, the document body must make a distinction
between processing of an individual part and of the main document, e.g.:
%
\begin{center}
\begin{tabular}{l}
|\ifchilddocmanual|\\
|\input{\childdocname}|\\
|\||else|\\
\textit{document body with }|\input{|\textit{part}|}|\\
|\||fi|
\end{tabular}
\end{center}
%
The conditional |\ifchilddocmanual| is true whenever
a part to be included by |\input| is being compiled,
and the name of the part is stored in |\childdocname|.

%%%%%%%%%%%%%%%%%%%%%%%%%%%%%%%%%%%%%%%%
\DescribeMacro{\childdocby}
Each part to be included by |\input| should start with:
%
\begin{center}
\begin{tabular}{l}
|\input{childdoc.def}|\\
|\childdocby{|\textit{main}|}|\\
\end{tabular}
\end{center}
%
The directive |\childdocby| is similar to |\childdocof|
described in \secref{sec:include},
but the subsequent selection of content must be done manually.
To that end, both |\ifchilddoc| and |\ifchilddocmanual|
will be true upon processing of a part,
and the name of the part is stored in |\childdocname|.
Note that |\jobname| will be set to the filename of the current part
so that each part receives an individual |.aux| file
that does not interfere with the |.aux| file(s) of the main document.
This behaviour can be altered by the alternative form
|\childdocby[*]{|\textit{main}|}| (with a non-empty optional argument)
which uses the |.aux| file of the main document
by setting |\jobname| to \textit{main}.

%%%%%%%%%%%%%%%%%%%%%%%%%%%%%%%%%%%%%%%%%%%%%%%%%%%%%%%%%%%%%%%%%%%%%%%%%%%%%%%%
\subsection{Driver Development}
\label{sec:driver}

The \textsf{childdoc} mechanism can also be use for the development
of definition files such as \LaTeX{} styles or classes.
This case differs from the above setup with multiple parts
included by |\include| in that no |\includeonly| should be invoked.
This can be achieved by starting the include file
(before |\ProvidesPackage|) with:
%
\begin{center}
\begin{tabular}{l}
|\input{childdoc.def}|\\
|\childdocforward{|\textit{main}|}|\\
\end{tabular}
\end{center}
%
or alternatively with:
%
\begin{center}
\begin{tabular}{l}
|\input{childdoc.def}|\\
|\childdocby{|\textit{main}|}|\\
\end{tabular}
\end{center}
%
Both forms have slightly different effects as described above.
The main file is prepared as usual, see \secref{sec:include}.

%%%%%%%%%%%%%%%%%%%%%%%%%%%%%%%%%%%%%%%%%%%%%%%%%%%%%%%%%%%%%%%%%%%%%%%%%%%%%%%%
\subsection{Legacy Detection}
\label{sec:detection}

The directive |\childdocmain| in the main file can detect
whether the complete document or merely a child is to be compiled
even without using the directive |\childdocof|.
This method is deprecated because it is less robust
and there is no compelling reason to use it;
it is merely provided for backward compatibility
and it may be removed in future versions.

If the detection mechanism is to be used,
it is mandatory to correctly specify
the filename of the main file as the argument of |\childdocmain|:
%
\begin{center}
\begin{tabular}{l}
|\input{childdoc.def}|\\
|\childdocmain{|\textit{main}|}|\\
\end{tabular}
\end{center}
%
If |\jobname| does not match the argument \textit{main} of |\childdocmain|,
it is assumed that |\jobname| points to the child file to be compiled.
When using |\childdocmain| with the main file specified as argument,
it suffices to start a child file
with just |\input{|\textit{main}|}|
without loading of the package and using |\childdocof|.
If instead all processing is done
with the appropriate \textsf{childdoc} directives,
the argument of \textit{main} of |\childdocmain| can be empty.

An alternative version of the command line processing described
in \secref{sec:commandline} using the detection mechanism reads:
%
\begin{center}
|... -jobname "|\textit{target}|" "|[\textit{flags}]%
[|\def\jobname{|\textit{dest}|}|]|\input{|\textit{main}|}"|
\end{center}

%%%%%%%%%%%%%%%%%%%%%%%%%%%%%%%%%%%%%%%%%%%%%%%%%%%%%%%%%%%%%%%%%%%%%%%%%%%%%%%%
\subsection{Manual Code}
\label{sec:manual}

In case one cannot be certain whether the definitions file |childdoc.def|
is installed on the target \TeX{} distribution
and one prefers not to ship it,
it is conceivable to paste a few relevant commands into the sources.

To that end, drop all statements |\input{childdoc.def}|
and perform the replacements as outlined below.
Instead of |\childdocmain{|\textit{main}|}| add the following code
to the top of the main file:
%
\begin{center}
\begin{tabular}{l}
|\||ifdefined\childdocname\endinput\||fi\newif\ifchilddoc|\\
|\edef\childdocname{\scantokens\expandafter{\jobname\noexpand}}|\\
|\def\childdocmain{|\textit{main}|}\||ifx\childdocmain\childdocname\||else|\\
|\childdoctrue\includeonly{\childdocname}\let\jobname\childdocmain\||fi|\\
\end{tabular}
\end{center}
%
Instead of |\childdocof{|\textit{main}|}| just include the main file
at the top of each child file:
%
\begin{center}
|\input{|\textit{main}|}|
\end{center}
%
A simple redirection |\childdocforward{|\textit{dest}|}| is achieved by:
%
\begin{center}
|\def\jobname{|\textit{dest}|}\input{\jobname}|
\end{center}
%
The redirection with prefix
|\childdocforwardprefix[|\textit{prefix}|]{|\textit{dest}|}|
is accomplished by:
%
\begin{center}
\begin{tabular}{l}
|{\edef\jobname{\scantokens\expandafter{\jobname\noexpand}}|\\
|\def\redirectjob |\textit{prefix}|#1~~~{\gdef\jobname{|\textit{dest}|#1}}|\\
|\expandafter\redirectjob\jobname~~~}\input{\jobname}|
\end{tabular}
\end{center}

In an alternative approach,
child documents can be compiled by a specific command line
without additional code or specific definitions:
%
\begin{center}
|... -jobname "|\textit{target}|" "|[\textit{flags}]%
|\includeonly{|\textit{dest}|}\input{|\textit{main}|}"|
\end{center}
%

%%%%%%%%%%%%%%%%%%%%%%%%%%%%%%%%%%%%%%%%%%%%%%%%%%%%%%%%%%%%%%%%%%%%%%%%%%%%%%%%
%%%%%%%%%%%%%%%%%%%%%%%%%%%%%%%%%%%%%%%%%%%%%%%%%%%%%%%%%%%%%%%%%%%%%%%%%%%%%%%%
\section{Information}

%%%%%%%%%%%%%%%%%%%%%%%%%%%%%%%%%%%%%%%%%%%%%%%%%%%%%%%%%%%%%%%%%%%%%%%%%%%%%%%%
\subsection{Copyright}

Copyright \copyright{} 2017--2018 Niklas Beisert

This work may be distributed and/or modified under the
conditions of the \LaTeX{} Project Public License, either version 1.3
of this license or (at your option) any later version.
The latest version of this license is in
  \url{http://www.latex-project.org/lppl.txt}
and version 1.3 or later is part of all distributions of \LaTeX{}
version 2005/12/01 or later.

This work has the LPPL maintenance status `maintained'.

The Current Maintainer of this work is Niklas Beisert.

This work consists of the files |README.txt|, |childdoc.ins| and |childdoc.dtx|
as well as the derived files |childdoc.def|, |cdocsamp.tex|
with |cdocsch1.tex|, |cdocsch2.tex|, |cdocspt3.tex|, |cdocspt4.tex|,
|cdocsdrf.tex|, |cdocsfn1.tex|, |cdocsfn2.tex|
as well as |childdoc.pdf|.

%%%%%%%%%%%%%%%%%%%%%%%%%%%%%%%%%%%%%%%%%%%%%%%%%%%%%%%%%%%%%%%%%%%%%%%%%%%%%%%%
\subsection{Files and Installation}

The package consists of the files:
%
\begin{center}
\begin{tabular}{ll}
    |README.txt|   & readme file \\
    |childdoc.ins| & installation file \\
    |childdoc.dtx| & source file \\
    |childdoc.def| & definition file \\
    |cdocsamp.tex| & sample main file \\
    |cdocsch1.tex| & sample include file \\
    |cdocsch2.tex| & sample include file \\
    |cdocspt3.tex| & sample part file \\
    |cdocspt4.tex| & sample part file \\
    |cdocsdrf.tex| & sample redirection file \\
    |cdocsfn1.tex| & sample redirection file \\
    |cdocsfn2.tex| & sample redirection file \\
    |childdoc.pdf| & manual
\end{tabular}
\end{center}
%
The distribution consists of the files
|README.txt|, |childdoc.ins| and |childdoc.dtx|.
%
\begin{itemize}
\item
Run (pdf)\LaTeX{} on |childdoc.dtx|
to compile the manual |childdoc.pdf| (this file).
\item
Run \LaTeX{} on |childdoc.ins| to create the definitions file |childdoc.def|
and the sample |cdocsamp.tex| with include files
|cdocsch1.tex|, |cdocsch2.tex|, |cdocspt3.tex|, |cdocspt4.tex|,
|cdocsdrf.tex|, |cdocsfn1.tex|, |cdocsfn2.tex|.
Then copy the file |childdoc.def| to an appropriate directory of your \LaTeX{}
distribution, e.g.\ \textit{texmf-root}|/tex/latex/childdoc|.
\end{itemize}

%%%%%%%%%%%%%%%%%%%%%%%%%%%%%%%%%%%%%%%%%%%%%%%%%%%%%%%%%%%%%%%%%%%%%%%%%%%%%%%%
\subsection{Related CTAN Packages}

There are several other packages which offer a similar functionality:
%
\begin{itemize}
\item
The packages
\href{http://ctan.org/pkg/docmute}{\textsf{docmute}},
\href{http://ctan.org/pkg/includex}{\textsf{includex}} and
\href{http://ctan.org/pkg/standalone}{\textsf{standalone}}
provide commands to include only the document body of
a child file thus allowing both files to be compiled individually.
\item
The packages \href{http://ctan.org/pkg/subdocs}{\textsf{subdocs}}
and \href{http://ctan.org/pkg/subfiles}{\textsf{subfiles}}
provide structures in which the main and child documents can be
encapsulated and allowing them to be compiled individually.
The inclusion mechanism is different from the conventional |\include|.
\item
The package \href{http://ctan.org/pkg/combine}{\textsf{combine}}
is an elaborate solution to combine several documents into one.
\end{itemize}
%
See also the CTAN topic \href{http://ctan.org/topic/subdocs}{\textsf{subdocs}}
for further related packages.
The present package differs from the above solutions in that
a document structure constructed with the conventional |\include| mechanism
just needs two extra commands at the top of every file
such that all constituent files can be compiled individually.

%%%%%%%%%%%%%%%%%%%%%%%%%%%%%%%%%%%%%%%%%%%%%%%%%%%%%%%%%%%%%%%%%%%%%%%%%%%%%%%%
%\subsection{Feature Suggestions}
%
%The following is a list of features which may be useful for future
%versions of this package:
%%
%\begin{itemize}
%\item
%\ldots
%\end{itemize}

%%%%%%%%%%%%%%%%%%%%%%%%%%%%%%%%%%%%%%%%%%%%%%%%%%%%%%%%%%%%%%%%%%%%%%%%%%%%%%%%
\subsection{Revision History}

%%%%%%%%%%%%%%%%%%%%%%%%%%%%%%%%%%%%%%%%
\paragraph{v2.0:} 2018/12/30

\begin{itemize}
\item
immediate forward processing
\item
added |\childdocby| mechanism
\item
manual restructured
\end{itemize}

%%%%%%%%%%%%%%%%%%%%%%%%%%%%%%%%%%%%%%%%
\paragraph{v1.6:} 2018/01/17

\begin{itemize}
\item
application for development of include files
\item
corrections to manual
\end{itemize}

%%%%%%%%%%%%%%%%%%%%%%%%%%%%%%%%%%%%%%%%
\paragraph{v1.5:} 2017/05/21

\begin{itemize}
\item
more complete structuring introduced
\item
|\childdocof| introduced
\item
|\childdoc| renamed to |\childdocmain|
\item
|\childredirect| renamed to |\childdocforward| and |\childdocforwardprefix|
and functionality expanded
\end{itemize}

%%%%%%%%%%%%%%%%%%%%%%%%%%%%%%%%%%%%%%%%
\paragraph{v1.0:} 2017/04/27

\begin{itemize}
\item
manual and install package
\item
first version published on CTAN
\end{itemize}

%%%%%%%%%%%%%%%%%%%%%%%%%%%%%%%%%%%%%%%%
\paragraph{v0.6:} 2017/04/26

\begin{itemize}
\item
redirection mechanism added
\end{itemize}

%%%%%%%%%%%%%%%%%%%%%%%%%%%%%%%%%%%%%%%%
\paragraph{v0.5:} 2017/04/26

\begin{itemize}
\item
functionality in definition file
\end{itemize}


%%%%%%%%%%%%%%%%%%%%%%%%%%%%%%%%%%%%%%%%%%%%%%%%%%%%%%%%%%%%%%%%%%%%%%%%%%%%%%%%
%%%%%%%%%%%%%%%%%%%%%%%%%%%%%%%%%%%%%%%%%%%%%%%%%%%%%%%%%%%%%%%%%%%%%%%%%%%%%%%%
%%%%%%%%%%%%%%%%%%%%%%%%%%%%%%%%%%%%%%%%%%%%%%%%%%%%%%%%%%%%%%%%%%%%%%%%%%%%%%%%
\appendix

\settowidth\MacroIndent{\rmfamily\scriptsize 000\ }

 \DocInput{childdoc.dtx}

\end{document}
%</driver>
% \fi
%
% %%%%%%%%%%%%%%%%%%%%%%%%%%%%%%%%%%%%%%%%%%%%%%%%%%%%%%%%%%%%%%%%%%%%%%%%%%%%%%
% %%%%%%%%%%%%%%%%%%%%%%%%%%%%%%%%%%%%%%%%%%%%%%%%%%%%%%%%%%%%%%%%%%%%%%%%%%%%%%
% \section{Sample}
%\iffalse
%<*samplemain>
%\fi
%
% The following presents a sample document
% with two chapters, two parts, a title page,
% a compile flag as well as three forwarding files to set the flag.
% It consists of eight |.tex| files:
% \begin{center}
% \begin{tabular}{ll}
% |cdocsamp.tex|&main file\\
% |cdocsch1.tex|&include file for chapter 1\\
% |cdocsch2.tex|&include file for chapter 2\\
% |cdocspt3.tex|&include file for part 3\\
% |cdocspt4.tex|&include file for part 4\\
% |cdocsdrf.tex|&forwarding file for main file in draft mode\\
% |cdocsfi1.tex|&forwarding file for final version of chapter 1\\
% |cdocsfi2.tex|&forwarding file for final version of chapter 2\\
% \end{tabular}
% \end{center}
% Each of the eight files can be compiled directly by the \LaTeX{} compiler.
%
% %%%%%%%%%%%%%%%%%%%%%%%%%%%%%%%%%%%%%%
% \paragraph{Main File.}
%
% The main file is called |cdocsamp.tex|.
%
% Load the \textsf{childdoc} definitions and
% declare the filename for the main document:
%    \begin{macrocode}
\input{childdoc.def}
\childdocmain{}
%    \end{macrocode}

% Optional override for |\version| flag:
%    \begin{macrocode}
%%\ifchilddoc\else\providecommand{\version}{draft}\fi
%    \end{macrocode}

% Define the default values for the |\version| flag
% (|final| for the main file and |draft| for childs):
%    \begin{macrocode}
\ifchilddoc
\providecommand{\version}{draft}
\else
\providecommand{\version}{final}
\fi
%    \end{macrocode}

% Load the standard document class:
%    \begin{macrocode}
\documentclass[12pt]{article}
%    \end{macrocode}

% Start the document body:
%    \begin{macrocode}
\begin{document}
%    \end{macrocode}

% Declare a title page.
% Print title, part of document being processed and version flag:
%    \begin{macrocode}
\addtocounter{page}{-1}
\begin{center}
{\LARGE\bfseries{}childdoc example\par}
\vspace{1cm}
\ifchilddoc
\ifchilddocmanual part\else chapter\fi:
`\childdocname' of `\childdocjob'\par
\else
main document: `\childdocjob'\par
\fi
version: \version\par
\end{center}
\newpage
%    \end{macrocode}

% Manually include selected file,
% otherwise process as usual:
%    \begin{macrocode}
\ifchilddocmanual
\section*{part `\childdocname'}
\input{\childdocname}
\else
%    \end{macrocode}

% Include the two chapters:
%    \begin{macrocode}
\include{cdocsch1}
\include{cdocsch2}
%    \end{macrocode}

% Include the two parts unless only chapters should be displayed:
%    \begin{macrocode}
\ifchilddoc\else
\section{part three}
\input{cdocspt3}
\section{part four}
\input{cdocspt4}
\fi
%    \end{macrocode}

% Process as usual until here:
%    \begin{macrocode}
\fi
%    \end{macrocode}

% End of document body:
%    \begin{macrocode}
\end{document}
%    \end{macrocode}
%\iffalse
%</samplemain>
%\fi
%
% %%%%%%%%%%%%%%%%%%%%%%%%%%%%%%%%%%%%%%
% \paragraph{Chapter Include Files.}
%
% The include files are called |cdocsch1.tex| and |cdocsch2.tex|.
%
%\iffalse
%<*samplechap1|samplechap2>
%\fi

% Optional override for |\version| flag:
%    \begin{macrocode}
%%\providecommand{\version}{final}
%    \end{macrocode}

% Include the main document:
%    \begin{macrocode}
\input{childdoc.def}
\childdocof{cdocsamp}
%    \end{macrocode}

%\iffalse
%</samplechap1|samplechap2>
%\fi
%
%\iffalse
%<*samplechap1>
%\fi
% Some text for chapter 1:
%    \begin{macrocode}
\section{one}
some text in chapter one
%    \end{macrocode}

%\iffalse
%</samplechap1>
%\fi
% Some text for chapter 2:
%\iffalse
%<*samplechap2>
%\fi
%    \begin{macrocode}
\section{two}
more text in chapter two
%    \end{macrocode}

%\iffalse
%</samplechap2>
%\fi
%
% %%%%%%%%%%%%%%%%%%%%%%%%%%%%%%%%%%%%%%
% \paragraph{Part Include Files.}
%
% The include files are called |cdocspt3.tex| and |cdocspt4.tex|.
%
%\iffalse
%<*samplepart3|samplepart4>
%\fi

% Optional override for |\version| flag:
%    \begin{macrocode}
%%\providecommand{\version}{final}
%    \end{macrocode}

% Include the main document:
%    \begin{macrocode}
\input{childdoc.def}
\childdocby{cdocsamp}
%    \end{macrocode}

%\iffalse
%</samplepart3|samplepart4>
%\fi
%
%\iffalse
%<*samplepart3>
%\fi
% Some text for part 3:
%    \begin{macrocode}
some text in part three
%    \end{macrocode}

%\iffalse
%</samplepart3>
%\fi
% Some text for part 4:
%\iffalse
%<*samplepart4>
%\fi
%    \begin{macrocode}
more text in part four
%    \end{macrocode}

%\iffalse
%</samplepart4>
%\fi
%
% %%%%%%%%%%%%%%%%%%%%%%%%%%%%%%%%%%%%%%
% \paragraph{Forwarding for a Complete Draft.}
%
% The following forwarding file |cdocsdrf.tex|
% compiles the main document in draft mode:
%\iffalse
%<*sampledraft>
%\fi
%    \begin{macrocode}
\def\version{draft}
\input{childdoc.def}
\childdocforward{cdocsamp}
%    \end{macrocode}

%\iffalse
%</sampledraft>
%\fi
%
% %%%%%%%%%%%%%%%%%%%%%%%%%%%%%%%%%%%%%%
% \paragraph{Forwarding for Final Version of the Chapters.}
%
% The following forwarding files |cdocsfn1.tex| and |cdocsfn2.tex|
% (with identical content)
% compile the final versions of the child documents
% |cdocsch1.tex| and |cdocsch2.tex|, respectively:
%\iffalse
%<*samplefinal>
%\fi
%    \begin{macrocode}
\def\version{final}
\input{childdoc.def}
\childdocforwardprefix[cdocsamp]{cdocsfn}{cdocsch}
%    \end{macrocode}

%\iffalse
%</samplefinal>
%\fi
%
% %%%%%%%%%%%%%%%%%%%%%%%%%%%%%%%%%%%%%%
% \paragraph{Command Line Processing.}
%
% The following three command lines generate the output files
% |cdocscld|, |cdocscl1| and |cdocscl2|
% which should be identical to
% |cdocsdrf|, |cdocsch1| and |cdocsfn2|, respectively:
% \begin{center}
% \begin{tabular}{l}
% |latex -jobname cdocscld \|\\
% |  "\def\version{draft}\input{childdoc.def}\childdocforward{cdocsamp}"|\\
% |latex -jobname cdocscl1 \|\\
% |  "\input{childdoc.def}\childdocforward[cdocsamp]{cdocsch1}"|\\
% |latex -jobname cdocscl2 \|\\
% |  "\def\version{final}\input{childdoc.def}\childdocforward{cdocsch2}"|
% \end{tabular}
% \end{center}
% Note that the trailing backslash on each first line
% merely continues the input to the second line
% (for convenient cut ant paste).
% Furthermore, the command |latex| can be replaced by any
% of its alternative versions such as |pdflatex|.
%
% %%%%%%%%%%%%%%%%%%%%%%%%%%%%%%%%%%%%%%%%%%%%%%%%%%%%%%%%%%%%%%%%%%%%%%%%%%%%%%
% %%%%%%%%%%%%%%%%%%%%%%%%%%%%%%%%%%%%%%%%%%%%%%%%%%%%%%%%%%%%%%%%%%%%%%%%%%%%%%
% \section{Implementation}
%\iffalse
%<*package>
%\fi
%
% This section describes the definitions file |childdoc.def|.

% The definitions cannot be loaded using |\usepackage| or |\RequirePackage|
% which has a mechanism to prevent loading a style file more than once.
% When loading the definitions by means of |\input|
% multiple instances have to be prevented manually:
%\iffalse
%This code needs to be before the `\ProvidesFile' directive
%which is defined at the beginning of this file.
%Therefore it is also placed there and commented out here.
%</package>
%<*discard>
%\fi
%    \begin{macrocode}
\ifdefined\childdocmain\endinput\fi
%    \end{macrocode}
%\iffalse
%</discard>
%<*package>
%\fi
%
% \macro{\ifchilddoc}
% \macro{\ifchilddocmanual}
% The conditional |\ifchilddoc| tells whether a
% child (true) or main (false) document is being compiled.
% The conditional |\ifchilddocmanual| tells whether
% the |\includeonly| mechanism is used (false) or
% the selection of child files must be performed manually (true).
% The definitions initialise to false:
%    \begin{macrocode}
\newif\ifchilddoc
\newif\ifchilddocmanual
%    \end{macrocode}

% \macro{\childdocname}
% \macro{\childdocjob}
% The macro |\childdocname| stores the name of the main document
% to be compiled. The macro |\childdocjob| stores the name of
% the document on which the \LaTeX{} compiler was originally invoked.
% The content of |\jobname| cannot be compared
% to filenames specified in the source due to different catcodes.
% The following code rescans |\jobname|, stores the result
% in |\childdocname| and saves a copy in |\childdocjob|:
%    \begin{macrocode}
\edef\childdocname{\scantokens\expandafter{\jobname\noexpand}}
\let\childdocjob\childdocname
%    \end{macrocode}

% \macro{\childdocdisable}
% The macro |\childdocdisable| prevents the main file
% from being processed more than once.
% At this stage, the main document command |\childdocmain|
% is assumed to be called once again where it should do nothing.
% Any subsequent call to it should prevent
% a secondary processing of the main document
% It overwrites the forwarding commands
% |\childdocof| and |\childdocforward|
% with empty macros to prevent further inclusions of the main document:
%    \begin{macrocode}
\newcommand{\childdocdisable}
{
  \renewcommand{\childdocmain}[1]{\renewcommand{\childdocmain}[1]{\endinput}}
  \renewcommand{\childdocof}[1]{}
  \renewcommand{\childdocby}[2][]{}
  \renewcommand{\childdocforward}[2][]{}
  \renewcommand{\childdocdisable}{}
}
%    \end{macrocode}

% \macro{\childdocmain}
% The macro |\childdocmain| is to be called at the top of the main file
% with nothing or the main filename (without extension) as argument.
% First, it breaks loops.
% If the argument is not empty and does not match |\childdocname|
% (which is set by the first inclusion of |childdoc.def|),
% |\ifchilddoc| is set to true, |\includeonly| is applied to the child file
% and |\jobname| is set to the main file
% (for proper handling of |.aux| files):
%    \begin{macrocode}
\newcommand{\childdocmain}[1]
{
  \childdocdisable\childdocmain{}
  \if?#1?\else
    \begingroup
      \def\childdoctmp{#1}
      \ifx\childdoctmp\childdocname
        \def\childdoctmp{}
      \else
        \def\childdoctmp
        {
          \childdoctrue
          \includeonly{\childdocname}
          \def\childdocjob{#1}
          \def\jobname{#1}
        }
      \fi
      \expandafter
    \endgroup
    \childdoctmp
  \fi
}
%    \end{macrocode}

% \macro{\childdocof}
% The command |\childdocof| redirects
% compilation to the main file |#1|.
%    \begin{macrocode}
\newcommand{\childdocof}[1]
{
  \childdocdisable
  \childdoctrue
  \includeonly{\childdocname}
  \def\jobname{#1}
  \def\childdocjob{#1}
  \input{#1}
}
%    \end{macrocode}

% \macro{\childdocby}
% The command |\childdocby| ....
%    \begin{macrocode}
\newcommand{\childdocby}[2][]
{
  \childdocdisable
  \childdoctrue
  \childdocmanualtrue
  \if?#1?\else
    \def\jobname{#2}
  \fi
  \def\childdocjob{#2}
  \input{#2}
  \endinput
}
%    \end{macrocode}

% \macro{\childdocforward}
% The command |\childdocforward| redirects
% compilation to the main file or
% (if the optional argument is given) a child file.
% Parameters are set as if the main file
% or a child file starting with |\childdocof| was compiled.
% Then compilation is handed over to the main file:
%    \begin{macrocode}
\newcommand{\childdocforward}[2][]
{
  \begingroup
    \if?#1?
      \def\childdoctmp
      {
        \def\childdocname{#2}
        \def\childdocjob{#2}
        \def\jobname{#2}
        \input{#2}
        \endinput
      }
    \else
      \def\childdoctmp
      {
        \childdocdisable
        \def\childdocname{#2}
        \childdoctrue
        \includeonly{#2}
        \def\childdocjob{#1}
        \def\jobname{#1}
        \input{#1}
        \endinput
      }
    \fi
    \expandafter
  \endgroup
  \childdoctmp
}
%    \end{macrocode}

% \macro{\childdocforwardprefix}
% The command |\childdocforwardprefix| redirects
% compilation to the main or a child file by means of a pattern.
% The prefix |#1| in the current filename is replaced by |#2|
% and the suffix of the current filename is kept
% (it is assumed that the filename does not contain the substring `|~~~|'
% which is used as a delimiter).
% Compilation is handed over to the new file by |\childdocforward|:
%    \begin{macrocode}
\newcommand{\childdocforwardprefix}[3][]
{
  \begingroup
    \def\childdocextract #2##1~~~{\def\childdoctmp{\childdocforward[#1]{#3##1}}}
    \expandafter\childdocextract\childdocname~~~
    \expandafter
  \endgroup
  \childdoctmp
}
%    \end{macrocode}

% \macro{\childdoc}
% The deprecated macro |\childdoc| is a legacy version of |\childdocmain|:
%    \begin{macrocode}
\newcommand{\childdoc}{\childdocmain}
%    \end{macrocode}

% \macro{\childdocredirect}
% The deprecated macro |\childdocredirect| is a legacy version
% of |\childdocforward| and |\childdocforwardprefix|:
%    \begin{macrocode}
\newcommand{\childdocredirect}[2][]
{
  \begingroup
    \if?#1?
      \def\childdoctmp{\childdocforward{#2}}
    \else
      \def\childdoctmp{\childdocforwardprefix{#1}{#2}}
    \fi
    \expandafter
  \endgroup
  \childdoctmp
}
%    \end{macrocode}

%\iffalse
%</package>
%\fi
%
\endinput
|\\
|\childdocmain{|\textit{main}|}|\\
\end{tabular}
\end{center}
%
If |\jobname| does not match the argument \textit{main} of |\childdocmain|,
it is assumed that |\jobname| points to the child file to be compiled.
When using |\childdocmain| with the main file specified as argument,
it suffices to start a child file
with just |\input{|\textit{main}|}|
without loading of the package and using |\childdocof|.
If instead all processing is done
with the appropriate \textsf{childdoc} directives,
the argument of \textit{main} of |\childdocmain| can be empty.

An alternative version of the command line processing described
in \secref{sec:commandline} using the detection mechanism reads:
%
\begin{center}
|... -jobname "|\textit{target}|" "|[\textit{flags}]%
[|\def\jobname{|\textit{dest}|}|]|\input{|\textit{main}|}"|
\end{center}

%%%%%%%%%%%%%%%%%%%%%%%%%%%%%%%%%%%%%%%%%%%%%%%%%%%%%%%%%%%%%%%%%%%%%%%%%%%%%%%%
\subsection{Manual Code}
\label{sec:manual}

In case one cannot be certain whether the definitions file |childdoc.def|
is installed on the target \TeX{} distribution
and one prefers not to ship it,
it is conceivable to paste a few relevant commands into the sources.

To that end, drop all statements |% \iffalse
%
% childdoc.dtx Copyright (C) 2017-2018 Niklas Beisert
%
% This work may be distributed and/or modified under the
% conditions of the LaTeX Project Public License, either version 1.3
% of this license or (at your option) any later version.
% The latest version of this license is in
%   http://www.latex-project.org/lppl.txt
% and version 1.3 or later is part of all distributions of LaTeX
% version 2005/12/01 or later.
%
% This work has the LPPL maintenance status `maintained'.
%
% The Current Maintainer of this work is Niklas Beisert.
%
% This work consists of the files childdoc.dtx and childdoc.ins
% and the derived files childdoc.def and cdocsamp.tex with
% cdocsch1.tex, cdocsch2.tex, cdocsdrf.tex, cdocsfn1.tex, cdocsfn2.tex.
%
%<package>\ifdefined\childdocmain\endinput\fi
%<package>\ProvidesFile{childdoc.def}[2018/12/30 v2.0 child document driver]
%<samplemain>\ProvidesFile{cdocsamp.tex}[2018/12/30 v2.0 sample for childdoc]
%<*driver>
%\ProvidesFile{childdoc.drv}[2018/12/30 v2.0 childdoc reference manual file]
\PassOptionsToClass{10pt,a4paper}{article}
\documentclass{ltxdoc}

\usepackage[margin=35mm]{geometry}
\usepackage{hyperref}
\usepackage{hyperxmp}
\usepackage[usenames]{color}

\hypersetup{colorlinks=true}
\hypersetup{pdfstartview=FitH}
\hypersetup{pdfpagemode=UseNone}
\hypersetup{pdfsource={}}
\hypersetup{pdflang={en-UK}}
\hypersetup{pdfcopyright={Copyright 2017-2018 Niklas Beisert.
  This work may be distributed and/or modified under the
  conditions of the LaTeX Project Public License, either version 1.3
  of this license or (at your option) any later version.}}
\hypersetup{pdflicenseurl={http://www.latex-project.org/lppl.txt}}
\hypersetup{pdfcontactaddress={ETH Zurich, ITP, HIT K,
  Wolfgang-Pauli-Strasse 27}}
\hypersetup{pdfcontactpostcode={8093}}
\hypersetup{pdfcontactcity={Zurich}}
\hypersetup{pdfcontactcountry={Switzerland}}
\hypersetup{pdfcontactemail={nbeisert@itp.phys.ethz.ch}}
\hypersetup{pdfcontacturl={http://people.phys.ethz.ch/\xmptilde nbeisert/}}

\newcommand{\secref}[1]{\hyperref[#1]{section \ref*{#1}}}

\parskip1ex
\parindent0pt
\let\olditemize\itemize
\def\itemize{\olditemize\parskip0pt}

\begin{document}

\title{The \textsf{childdoc} Package}
\hypersetup{pdftitle={The childdoc Package}}
\author{Niklas Beisert\\[2ex]
  Institut f\"ur Theoretische Physik\\
  Eidgen\"ossische Technische Hochschule Z\"urich\\
  Wolfgang-Pauli-Strasse 27, 8093 Z\"urich, Switzerland\\[1ex]
  \href{mailto:nbeisert@itp.phys.ethz.ch}
  {\texttt{nbeisert@itp.phys.ethz.ch}}}
\hypersetup{pdfauthor={Niklas Beisert}}
\hypersetup{pdfsubject={Manual for the LaTeX2e Package childdoc}}
\date{30 December 2018, \textsf{v2.0}}
\maketitle

\begin{abstract}\noindent
\textsf{childdoc} is a \LaTeXe{} package
that enables the direct compilation
of document sections included by |\include|
to individual files.
\end{abstract}

\begingroup
\parskip0ex
\tableofcontents
\endgroup

%%%%%%%%%%%%%%%%%%%%%%%%%%%%%%%%%%%%%%%%%%%%%%%%%%%%%%%%%%%%%%%%%%%%%%%%%%%%%%%%
%%%%%%%%%%%%%%%%%%%%%%%%%%%%%%%%%%%%%%%%%%%%%%%%%%%%%%%%%%%%%%%%%%%%%%%%%%%%%%%%
\section{Introduction}

\LaTeX{} provides a mechanism to structure a large document (such as a book)
into a main file and several child files (containing the chapters)
using the |\include| command.
This mechanism is beneficial for documents
which span hundreds of pages in order to
make the source file(s) more manageable.
Moreover, compilation can be restricted to
selected child files by means of the |\includeonly| command.
The latter feature can be used to reduce the compilation time while editing
(this was significantly more useful in the earlier days of \LaTeX{})
or to generate a smaller document which is easier to navigate.
Another application of |\includeonly| is to generate
documents consisting of selected parts of the complete document.

However, there are a few drawbacks of the plain |\include| mechanism:
\begin{itemize}
\item
The child files cannot be compiled on their own,
they can only be compiled via the main file.
A naive editing environment
(such as a text editor with an option
to have the current file processed by \LaTeX)
may require one to switch to the main file before compiling;
attempting to compile the child file produces errors.
\item
The main file must be modified (each time)
to adjust the |\includeonly| command
to the present needs. This easily leaves the main file in a messy state.
\item
The generated document will always carry the filename
of the main document. This is inconvenient if
several child files are to be compiled and
to be kept for distribution.
\end{itemize}

The present package provides a simple interface
to make child files individually compilable by \LaTeX{}.
Compiling a child file then has the same effect as compiling
the main file with an |\includeonly| command
to select the appropriate child.
Moreover the generated document will carry the name of the child
rather than the main file.
This resolves all three above issues.

This feature is meant to make the editing of books,
thesis documents and lecture notes somewhat more convenient.
However, the package can also be used efficiently for
composing a series of documents (such as exercise sheets)
which are typically distributed individually.
It then assists the author in generating the individual documents
(potentially in different versions)
as well as a document containing the collected series.
Another application is in developing style files
or other kinds of included material
where compilation of the style file could redirect
to a sample or test file.

%%%%%%%%%%%%%%%%%%%%%%%%%%%%%%%%%%%%%%%%%%%%%%%%%%%%%%%%%%%%%%%%%%%%%%%%%%%%%%%%
%%%%%%%%%%%%%%%%%%%%%%%%%%%%%%%%%%%%%%%%%%%%%%%%%%%%%%%%%%%%%%%%%%%%%%%%%%%%%%%%
\section{Usage}

First of all, the package \textsf{childdoc} is \emph{not} a standard
\LaTeXe{} |.sty| style file! Therefore it needs to be invoked in
a non-standard way.

%%%%%%%%%%%%%%%%%%%%%%%%%%%%%%%%%%%%%%%%%%%%%%%%%%%%%%%%%%%%%%%%%%%%%%%%%%%%%%%%
\subsection{Included Files}
\label{sec:include}

%%%%%%%%%%%%%%%%%%%%%%%%%%%%%%%%%%%%%%%%
\DescribeMacro{\childdocmain}
To use the package, add the commands
\begin{center}
\begin{tabular}{l}
|\input{childdoc.def}|\\
|\childdocmain{}|\\
\end{tabular}
\end{center}
at the very top of the main \LaTeX{} file,
in particular \emph{before} the |\documentclass| statement!
The argument of |\childdocmain| should be left empty
(but it must be present).

%%%%%%%%%%%%%%%%%%%%%%%%%%%%%%%%%%%%%%%%
\DescribeMacro{\childdocof}
Furthermore, add the commands
\begin{center}
\begin{tabular}{l}
|\input{childdoc.def}|\\
|\childdocof{|\textit{main}|}|\\
\end{tabular}
\end{center}
at the top of every child file \textit{child}
which is included by |\include{|\textit{child}|}|
from within the main file
(or at least for those files to be compiled individually).
The argument \textit{main} must be the filename of the main file.

There are a couple of
considerations in setting up the main and child documents:

%%%%%%%%%%%%%%%%%%%%%%%%%%%%%%%%%%%%%%%%
\paragraph{Restrictions.}

Please note the following restrictions:
\begin{itemize}
\item
|\childdocmain| must be called with one argument \textit{main}
to ensure compatibility with earlier version of the package.
It must either be empty (|\childdocmain{}|)
or precisely match the filename of the main file in which it is specified.
See \secref{sec:detection} for further information.
\item
The filename \textit{main} must be specified without the |.tex| extension.
\item
The filename \textit{main} is case sensitive
(even in case-insensitive file systems)
due to internal string comparison.
\item
The argument \textit{main} should be fully expanded, it cannot be a macro.
\item
Subdirectories and special characters should be avoided in filenames.
\item
The command |\childdocmain{|\textit{main}|}| must be followed by a whitespace.
It should not be followed immediately by another command
or by a comment mark `|%|'.
This is because the \TeX{} parser reads the token immediately following
the argument of |\childdocmain| and puts it
at the beginning of every child section;
however, a white\-space is ignored.
\end{itemize}

%%%%%%%%%%%%%%%%%%%%%%%%%%%%%%%%%%%%%%%%
\paragraph{Content of Main File.}

It is advisable to place all content in the child files included by |\include|.
Any output contained in the main file will appear in all child documents
unless suppressed manually;
it cannot be suppressed automatically by the |\includeonly| directive
and thus should normally be avoided.
A method to include some content in the main file
by means of conditional processing is described in \secref{sec:conditional}.

%%%%%%%%%%%%%%%%%%%%%%%%%%%%%%%%%%%%%%%%
\paragraph{Page Numbering.}

When only a part of the document is compiled,
the appropriate numbering of pages
(as well as other status parameters)
is determined from the |.aux| files.
The latter contain information from previous passes.
However this information needs to propagate through
all intermediate child documents.
Therefore the page numbering in child documents may well
be inconsistent until the complete document is compiled at least once.

A useful (if unconventional) way to always ensure a consistent
page numbering is to restart the numbering in each child document
and denote the pages by `\textit{child}|.|\textit{page}'
where \textit{child} represents the chapter/section number of the child file.
This can be achieved by the command
|\numberwithin{page}{|\textit{child}|}|
of the \textsf{amsmath} package
where \textit{child} can be |chapter| or |section|
depending on the chosen structuring.
Alternatively, one can modify the macro |\thepage| appropriately
and reset the counter |page| at the start of each child file.

%%%%%%%%%%%%%%%%%%%%%%%%%%%%%%%%%%%%%%%%%%%%%%%%%%%%%%%%%%%%%%%%%%%%%%%%%%%%%%%%
\subsection{Conditional Processing}
\label{sec:conditional}

The package provides a mechanism to compile different versions
of a document. To customise the versions further some conditional processing
can come in handy to distinguish which version is being compiled.
The package provides two macros to describe the compilation context:

%%%%%%%%%%%%%%%%%%%%%%%%%%%%%%%%%%%%%%%%
\DescribeMacro{\ifchilddoc}
The conditional |\ifchilddoc| distinguishes between the compilation of
child documents and the main document:
%
\begin{center}
|\ifchilddoc |\textit{child-code}| |[|\||else |\textit{main-code}]| \||fi|
\end{center}

%%%%%%%%%%%%%%%%%%%%%%%%%%%%%%%%%%%%%%%%
\DescribeMacro{\childdocname}
\DescribeMacro{\childdocjob}
The macro |\childdocname| contains the filename (without extension)
of the main or child file being processed.
Note that |\childdocjob| will always contain the name of the main file.

%%%%%%%%%%%%%%%%%%%%%%%%%%%%%%%%%%%%%%%%
\paragraph{Title Page.}

Conditional processing can be used to include a title or banner page
in the main document when proper precautions are taken.
Importantly, the code in the main file should ensure that the page counter
(as well as other status parameters which are stored in the |.aux| files)
takes the same value after the conditional processing.
Otherwise the page numbers may take divergent values
depending on which part is compiled.

For example, a title page could be declared by:
%
\begin{center}
\begin{tabular}{l}
|\ifchilddoc\||else|\\
|\addtocounter{page}{-1}|\\
\textit{code for title page}\\
|\newpage|\\
|\||fi|
\end{tabular}
\end{center}
%
A banner page for the child documents can be generated by:
%
\begin{center}
\begin{tabular}{l}
|\ifchilddoc|\\
|\addtocounter{page}{-1}|\\
\textit{code for banner page}\\
|\newpage|\\
|\||fi|
\end{tabular}
\end{center}
%
Here one could write a message such as:
\begin{center}
|This is the part \childdocname{} of \childdocjob{}.|
\end{center}

%%%%%%%%%%%%%%%%%%%%%%%%%%%%%%%%%%%%%%%%%%%%%%%%%%%%%%%%%%%%%%%%%%%%%%%%%%%%%%%%
\subsection{Flags}
\label{sec:flags}

The package makes it easy to generate different versions
of the main or child documents.
To this end compilation flags can be defined
and assigned different default values.
They will be particularly useful in conjunction
with the forwarding mechanism described in \secref{sec:forward}.

For example, it may be useful to have a flag |\version|
which can be set to |draft| or |final|.
The document source will contain some conditional code
depending on the value of |\version|.
Suppose further, the flag should default to |final| for the main file
and to |draft| for child files
which is a natural assignment for editing the document.
This is achieved by placing the following code
in the preamble of the main document
(below the |\childdocmain| directive):
%
\begin{center}
\begin{tabular}{l}
|\ifchilddoc|\\
|\providecommand{\version}{draft}|\\
|\||else|\\
|\providecommand{\version}{final}|\\
|\||fi|
\end{tabular}
\end{center}
%
The definition by |\providecommand| makes sure
that previous definitions are not overwritten.
Further statements |\providecommand{\version}{...}|
can thus be added before the above code to override it.

For the main file, one might add a line
(between |\childdocmain| and the above block)
%
\begin{center}
|%\ifchilddoc\||else\providecommand{\version}{draft}\||fi|
\end{center}
%
which can be uncommented to produce a draft version.
Likewise one can add a line to the very top of a child file
(above the |\childdocof{|\textit{main}|}| directive)
%
\begin{center}
|%\providecommand{\version}{final}|
\end{center}
%
which can be uncommented to produce the final version of this child document.

%%%%%%%%%%%%%%%%%%%%%%%%%%%%%%%%%%%%%%%%%%%%%%%%%%%%%%%%%%%%%%%%%%%%%%%%%%%%%%%%
\subsection{Forwarding}
\label{sec:forward}

Different versions of the main or child documents
using compilation flags as described in \secref{sec:flags}
can be (permanently) stored in different files
for convenient compilation, viewing and distribution.
To this end, the package defines a command
to pass on compilation to a different file:

%%%%%%%%%%%%%%%%%%%%%%%%%%%%%%%%%%%%%%%%
\DescribeMacro{\childdocforward}
The command |\childdocforward| redirects processing to
another source file:
%
\begin{center}
\begin{tabular}{l}
|\input{childdoc.def}|\\
|\childdocforward[|\textit{main}|]{|\textit{dest}|}|\\
\end{tabular}
\end{center}
%
The argument \textit{dest} is the destination file
(without extension).
It should be the main file or one of the child files.
Note that further \textsf{childdoc} directives
such as |\childdocof| and |\childdocforward|
in the indicated file will be processed in this form.
The optional argument \textit{main}
passes on directly to the main file \textit{main}
while pretending to compile the child \textit{dest}.
This form behaves as if \textit{dest}
issues |\childdocof{|\textit{main}|}| right away,
and no further \textsf{childdoc} directives will be processed.

%%%%%%%%%%%%%%%%%%%%%%%%%%%%%%%%%%%%%%%%
\DescribeMacro{\...prefix}
In the alternative form |\childdocforwardprefix|,
%
\begin{center}
\begin{tabular}{l}
|\input{childdoc.def}|\\
|\childdocforwardprefix[|\textit{main}|]{|\textit{prefix}|}{|\textit{dest}|}|
\end{tabular}
\end{center}
%
the destination file is determined by a pattern
depending on the current file:
To make this work, the current file must be called
`{\textit{prefix}\hspace{0.2em}\textit{suffix}}'
with \textit{prefix} matching precisely the argument.
Processing is then passed on to the file
`{\textit{dest}\hspace{0.2em}\textit{suffix}}'.
Surely, the same effect is achieved by
directly specifying the
argument `{\textit{dest}\hspace{0.2em}\textit{suffix}}'
in the first form.
However, that requires to set up a different file
for each child. With the alternative form of the command
all these files can have exactly the same content
which simplifies setting them up and maintaining them.

For example, the following file |draft.tex|
with a compilation flag |\version| as described in \secref{sec:flags}
compiles the main document as a draft:
%
\begin{center}
\begin{tabular}{l}
|\def\version{draft}|\\
|\input{childdoc.def}|\\
|\childdocforward{|\textit{main}|}|
\end{tabular}
\end{center}
%
Likewise, the following files |final|\textit{nn}|.tex|
compile the final version of the child document
|child|\textit{nn}|.tex|:
%
\begin{center}
\begin{tabular}{l}
|\def\version{final}|\\
|\input{childdoc.def}|\\
|\childdocforwardprefix{final}{child}|
\end{tabular}
\end{center}
%

Note that when several versions of a main file and/or of each child file
are to be generated, it may be convenient to set up a |Makefile| or
shell script to automatise the process.

%%%%%%%%%%%%%%%%%%%%%%%%%%%%%%%%%%%%%%%%%%%%%%%%%%%%%%%%%%%%%%%%%%%%%%%%%%%%%%%%
\subsection{Command Line Processing}
\label{sec:commandline}

The effect of redirection files can also be achieved by invoking
the \LaTeX{} compiler with a more elaborate command line.
Most conveniently this should be done as part
of a shell script or a |Makefile|.

When using \textsf{childdoc} in the main file, the following
command lines effectively perform a redirection
(note that depending on the shell being used,
backslashes may have to be doubled: `|\|' $\to$ `|\\|'):
%
\begin{center}
|... -jobname "|\textit{target}|" |\\|"|[\textit{flags}]%
|\input{childdoc.def}\childdocforward[|\textit{main}|]{|\textit{dest}|}"|
\end{center}
%
Here \textit{target} is the name of the output file,
\textit{main} is the name of the main file
and \textit{dest} is the name of the main or child file to be processed
(all filenames without extensions).
The optional argument \textit{main} can be omitted
if \textit{main} matches \textit{dest}.
Optionally, compilation \textit{flags} can be defined via |\def| commands.
This command line makes the \TeX{} engine believe
it is compiling the file \textit{target}
whose content is specified as the latter parameter.
The provided code then forwards the processing to
\textit{main} or \textit{dest} as described in \secref{sec:forward}.

%%%%%%%%%%%%%%%%%%%%%%%%%%%%%%%%%%%%%%%%%%%%%%%%%%%%%%%%%%%%%%%%%%%%%%%%%%%%%%%%
\subsection{Include by Input}
\label{sec:input}

Including child documents by |\include| has some restrictions by design.
Most notably, the content of a child document always occupies
its own set of pages; pages cannot be shared between child documents.
Usually, this behaviour makes perfect sense
because each child document contain an essential part of the document.
However, in some situations it may be desirable to compose
a document from a collection of parts
without having mandatory page breaks between then.
For this case, the package
provides a mechanism to include parts
by |\input| which can also be processed individually.
However, by construction this mechanism
requires manual handling of the content to be output.

%%%%%%%%%%%%%%%%%%%%%%%%%%%%%%%%%%%%%%%%
\DescribeMacro{\ifchilddocmanual}
The main file should be prepared as usual, see \secref{sec:include}.
However, the document body must make a distinction
between processing of an individual part and of the main document, e.g.:
%
\begin{center}
\begin{tabular}{l}
|\ifchilddocmanual|\\
|\input{\childdocname}|\\
|\||else|\\
\textit{document body with }|\input{|\textit{part}|}|\\
|\||fi|
\end{tabular}
\end{center}
%
The conditional |\ifchilddocmanual| is true whenever
a part to be included by |\input| is being compiled,
and the name of the part is stored in |\childdocname|.

%%%%%%%%%%%%%%%%%%%%%%%%%%%%%%%%%%%%%%%%
\DescribeMacro{\childdocby}
Each part to be included by |\input| should start with:
%
\begin{center}
\begin{tabular}{l}
|\input{childdoc.def}|\\
|\childdocby{|\textit{main}|}|\\
\end{tabular}
\end{center}
%
The directive |\childdocby| is similar to |\childdocof|
described in \secref{sec:include},
but the subsequent selection of content must be done manually.
To that end, both |\ifchilddoc| and |\ifchilddocmanual|
will be true upon processing of a part,
and the name of the part is stored in |\childdocname|.
Note that |\jobname| will be set to the filename of the current part
so that each part receives an individual |.aux| file
that does not interfere with the |.aux| file(s) of the main document.
This behaviour can be altered by the alternative form
|\childdocby[*]{|\textit{main}|}| (with a non-empty optional argument)
which uses the |.aux| file of the main document
by setting |\jobname| to \textit{main}.

%%%%%%%%%%%%%%%%%%%%%%%%%%%%%%%%%%%%%%%%%%%%%%%%%%%%%%%%%%%%%%%%%%%%%%%%%%%%%%%%
\subsection{Driver Development}
\label{sec:driver}

The \textsf{childdoc} mechanism can also be use for the development
of definition files such as \LaTeX{} styles or classes.
This case differs from the above setup with multiple parts
included by |\include| in that no |\includeonly| should be invoked.
This can be achieved by starting the include file
(before |\ProvidesPackage|) with:
%
\begin{center}
\begin{tabular}{l}
|\input{childdoc.def}|\\
|\childdocforward{|\textit{main}|}|\\
\end{tabular}
\end{center}
%
or alternatively with:
%
\begin{center}
\begin{tabular}{l}
|\input{childdoc.def}|\\
|\childdocby{|\textit{main}|}|\\
\end{tabular}
\end{center}
%
Both forms have slightly different effects as described above.
The main file is prepared as usual, see \secref{sec:include}.

%%%%%%%%%%%%%%%%%%%%%%%%%%%%%%%%%%%%%%%%%%%%%%%%%%%%%%%%%%%%%%%%%%%%%%%%%%%%%%%%
\subsection{Legacy Detection}
\label{sec:detection}

The directive |\childdocmain| in the main file can detect
whether the complete document or merely a child is to be compiled
even without using the directive |\childdocof|.
This method is deprecated because it is less robust
and there is no compelling reason to use it;
it is merely provided for backward compatibility
and it may be removed in future versions.

If the detection mechanism is to be used,
it is mandatory to correctly specify
the filename of the main file as the argument of |\childdocmain|:
%
\begin{center}
\begin{tabular}{l}
|\input{childdoc.def}|\\
|\childdocmain{|\textit{main}|}|\\
\end{tabular}
\end{center}
%
If |\jobname| does not match the argument \textit{main} of |\childdocmain|,
it is assumed that |\jobname| points to the child file to be compiled.
When using |\childdocmain| with the main file specified as argument,
it suffices to start a child file
with just |\input{|\textit{main}|}|
without loading of the package and using |\childdocof|.
If instead all processing is done
with the appropriate \textsf{childdoc} directives,
the argument of \textit{main} of |\childdocmain| can be empty.

An alternative version of the command line processing described
in \secref{sec:commandline} using the detection mechanism reads:
%
\begin{center}
|... -jobname "|\textit{target}|" "|[\textit{flags}]%
[|\def\jobname{|\textit{dest}|}|]|\input{|\textit{main}|}"|
\end{center}

%%%%%%%%%%%%%%%%%%%%%%%%%%%%%%%%%%%%%%%%%%%%%%%%%%%%%%%%%%%%%%%%%%%%%%%%%%%%%%%%
\subsection{Manual Code}
\label{sec:manual}

In case one cannot be certain whether the definitions file |childdoc.def|
is installed on the target \TeX{} distribution
and one prefers not to ship it,
it is conceivable to paste a few relevant commands into the sources.

To that end, drop all statements |\input{childdoc.def}|
and perform the replacements as outlined below.
Instead of |\childdocmain{|\textit{main}|}| add the following code
to the top of the main file:
%
\begin{center}
\begin{tabular}{l}
|\||ifdefined\childdocname\endinput\||fi\newif\ifchilddoc|\\
|\edef\childdocname{\scantokens\expandafter{\jobname\noexpand}}|\\
|\def\childdocmain{|\textit{main}|}\||ifx\childdocmain\childdocname\||else|\\
|\childdoctrue\includeonly{\childdocname}\let\jobname\childdocmain\||fi|\\
\end{tabular}
\end{center}
%
Instead of |\childdocof{|\textit{main}|}| just include the main file
at the top of each child file:
%
\begin{center}
|\input{|\textit{main}|}|
\end{center}
%
A simple redirection |\childdocforward{|\textit{dest}|}| is achieved by:
%
\begin{center}
|\def\jobname{|\textit{dest}|}\input{\jobname}|
\end{center}
%
The redirection with prefix
|\childdocforwardprefix[|\textit{prefix}|]{|\textit{dest}|}|
is accomplished by:
%
\begin{center}
\begin{tabular}{l}
|{\edef\jobname{\scantokens\expandafter{\jobname\noexpand}}|\\
|\def\redirectjob |\textit{prefix}|#1~~~{\gdef\jobname{|\textit{dest}|#1}}|\\
|\expandafter\redirectjob\jobname~~~}\input{\jobname}|
\end{tabular}
\end{center}

In an alternative approach,
child documents can be compiled by a specific command line
without additional code or specific definitions:
%
\begin{center}
|... -jobname "|\textit{target}|" "|[\textit{flags}]%
|\includeonly{|\textit{dest}|}\input{|\textit{main}|}"|
\end{center}
%

%%%%%%%%%%%%%%%%%%%%%%%%%%%%%%%%%%%%%%%%%%%%%%%%%%%%%%%%%%%%%%%%%%%%%%%%%%%%%%%%
%%%%%%%%%%%%%%%%%%%%%%%%%%%%%%%%%%%%%%%%%%%%%%%%%%%%%%%%%%%%%%%%%%%%%%%%%%%%%%%%
\section{Information}

%%%%%%%%%%%%%%%%%%%%%%%%%%%%%%%%%%%%%%%%%%%%%%%%%%%%%%%%%%%%%%%%%%%%%%%%%%%%%%%%
\subsection{Copyright}

Copyright \copyright{} 2017--2018 Niklas Beisert

This work may be distributed and/or modified under the
conditions of the \LaTeX{} Project Public License, either version 1.3
of this license or (at your option) any later version.
The latest version of this license is in
  \url{http://www.latex-project.org/lppl.txt}
and version 1.3 or later is part of all distributions of \LaTeX{}
version 2005/12/01 or later.

This work has the LPPL maintenance status `maintained'.

The Current Maintainer of this work is Niklas Beisert.

This work consists of the files |README.txt|, |childdoc.ins| and |childdoc.dtx|
as well as the derived files |childdoc.def|, |cdocsamp.tex|
with |cdocsch1.tex|, |cdocsch2.tex|, |cdocspt3.tex|, |cdocspt4.tex|,
|cdocsdrf.tex|, |cdocsfn1.tex|, |cdocsfn2.tex|
as well as |childdoc.pdf|.

%%%%%%%%%%%%%%%%%%%%%%%%%%%%%%%%%%%%%%%%%%%%%%%%%%%%%%%%%%%%%%%%%%%%%%%%%%%%%%%%
\subsection{Files and Installation}

The package consists of the files:
%
\begin{center}
\begin{tabular}{ll}
    |README.txt|   & readme file \\
    |childdoc.ins| & installation file \\
    |childdoc.dtx| & source file \\
    |childdoc.def| & definition file \\
    |cdocsamp.tex| & sample main file \\
    |cdocsch1.tex| & sample include file \\
    |cdocsch2.tex| & sample include file \\
    |cdocspt3.tex| & sample part file \\
    |cdocspt4.tex| & sample part file \\
    |cdocsdrf.tex| & sample redirection file \\
    |cdocsfn1.tex| & sample redirection file \\
    |cdocsfn2.tex| & sample redirection file \\
    |childdoc.pdf| & manual
\end{tabular}
\end{center}
%
The distribution consists of the files
|README.txt|, |childdoc.ins| and |childdoc.dtx|.
%
\begin{itemize}
\item
Run (pdf)\LaTeX{} on |childdoc.dtx|
to compile the manual |childdoc.pdf| (this file).
\item
Run \LaTeX{} on |childdoc.ins| to create the definitions file |childdoc.def|
and the sample |cdocsamp.tex| with include files
|cdocsch1.tex|, |cdocsch2.tex|, |cdocspt3.tex|, |cdocspt4.tex|,
|cdocsdrf.tex|, |cdocsfn1.tex|, |cdocsfn2.tex|.
Then copy the file |childdoc.def| to an appropriate directory of your \LaTeX{}
distribution, e.g.\ \textit{texmf-root}|/tex/latex/childdoc|.
\end{itemize}

%%%%%%%%%%%%%%%%%%%%%%%%%%%%%%%%%%%%%%%%%%%%%%%%%%%%%%%%%%%%%%%%%%%%%%%%%%%%%%%%
\subsection{Related CTAN Packages}

There are several other packages which offer a similar functionality:
%
\begin{itemize}
\item
The packages
\href{http://ctan.org/pkg/docmute}{\textsf{docmute}},
\href{http://ctan.org/pkg/includex}{\textsf{includex}} and
\href{http://ctan.org/pkg/standalone}{\textsf{standalone}}
provide commands to include only the document body of
a child file thus allowing both files to be compiled individually.
\item
The packages \href{http://ctan.org/pkg/subdocs}{\textsf{subdocs}}
and \href{http://ctan.org/pkg/subfiles}{\textsf{subfiles}}
provide structures in which the main and child documents can be
encapsulated and allowing them to be compiled individually.
The inclusion mechanism is different from the conventional |\include|.
\item
The package \href{http://ctan.org/pkg/combine}{\textsf{combine}}
is an elaborate solution to combine several documents into one.
\end{itemize}
%
See also the CTAN topic \href{http://ctan.org/topic/subdocs}{\textsf{subdocs}}
for further related packages.
The present package differs from the above solutions in that
a document structure constructed with the conventional |\include| mechanism
just needs two extra commands at the top of every file
such that all constituent files can be compiled individually.

%%%%%%%%%%%%%%%%%%%%%%%%%%%%%%%%%%%%%%%%%%%%%%%%%%%%%%%%%%%%%%%%%%%%%%%%%%%%%%%%
%\subsection{Feature Suggestions}
%
%The following is a list of features which may be useful for future
%versions of this package:
%%
%\begin{itemize}
%\item
%\ldots
%\end{itemize}

%%%%%%%%%%%%%%%%%%%%%%%%%%%%%%%%%%%%%%%%%%%%%%%%%%%%%%%%%%%%%%%%%%%%%%%%%%%%%%%%
\subsection{Revision History}

%%%%%%%%%%%%%%%%%%%%%%%%%%%%%%%%%%%%%%%%
\paragraph{v2.0:} 2018/12/30

\begin{itemize}
\item
immediate forward processing
\item
added |\childdocby| mechanism
\item
manual restructured
\end{itemize}

%%%%%%%%%%%%%%%%%%%%%%%%%%%%%%%%%%%%%%%%
\paragraph{v1.6:} 2018/01/17

\begin{itemize}
\item
application for development of include files
\item
corrections to manual
\end{itemize}

%%%%%%%%%%%%%%%%%%%%%%%%%%%%%%%%%%%%%%%%
\paragraph{v1.5:} 2017/05/21

\begin{itemize}
\item
more complete structuring introduced
\item
|\childdocof| introduced
\item
|\childdoc| renamed to |\childdocmain|
\item
|\childredirect| renamed to |\childdocforward| and |\childdocforwardprefix|
and functionality expanded
\end{itemize}

%%%%%%%%%%%%%%%%%%%%%%%%%%%%%%%%%%%%%%%%
\paragraph{v1.0:} 2017/04/27

\begin{itemize}
\item
manual and install package
\item
first version published on CTAN
\end{itemize}

%%%%%%%%%%%%%%%%%%%%%%%%%%%%%%%%%%%%%%%%
\paragraph{v0.6:} 2017/04/26

\begin{itemize}
\item
redirection mechanism added
\end{itemize}

%%%%%%%%%%%%%%%%%%%%%%%%%%%%%%%%%%%%%%%%
\paragraph{v0.5:} 2017/04/26

\begin{itemize}
\item
functionality in definition file
\end{itemize}


%%%%%%%%%%%%%%%%%%%%%%%%%%%%%%%%%%%%%%%%%%%%%%%%%%%%%%%%%%%%%%%%%%%%%%%%%%%%%%%%
%%%%%%%%%%%%%%%%%%%%%%%%%%%%%%%%%%%%%%%%%%%%%%%%%%%%%%%%%%%%%%%%%%%%%%%%%%%%%%%%
%%%%%%%%%%%%%%%%%%%%%%%%%%%%%%%%%%%%%%%%%%%%%%%%%%%%%%%%%%%%%%%%%%%%%%%%%%%%%%%%
\appendix

\settowidth\MacroIndent{\rmfamily\scriptsize 000\ }

 \DocInput{childdoc.dtx}

\end{document}
%</driver>
% \fi
%
% %%%%%%%%%%%%%%%%%%%%%%%%%%%%%%%%%%%%%%%%%%%%%%%%%%%%%%%%%%%%%%%%%%%%%%%%%%%%%%
% %%%%%%%%%%%%%%%%%%%%%%%%%%%%%%%%%%%%%%%%%%%%%%%%%%%%%%%%%%%%%%%%%%%%%%%%%%%%%%
% \section{Sample}
%\iffalse
%<*samplemain>
%\fi
%
% The following presents a sample document
% with two chapters, two parts, a title page,
% a compile flag as well as three forwarding files to set the flag.
% It consists of eight |.tex| files:
% \begin{center}
% \begin{tabular}{ll}
% |cdocsamp.tex|&main file\\
% |cdocsch1.tex|&include file for chapter 1\\
% |cdocsch2.tex|&include file for chapter 2\\
% |cdocspt3.tex|&include file for part 3\\
% |cdocspt4.tex|&include file for part 4\\
% |cdocsdrf.tex|&forwarding file for main file in draft mode\\
% |cdocsfi1.tex|&forwarding file for final version of chapter 1\\
% |cdocsfi2.tex|&forwarding file for final version of chapter 2\\
% \end{tabular}
% \end{center}
% Each of the eight files can be compiled directly by the \LaTeX{} compiler.
%
% %%%%%%%%%%%%%%%%%%%%%%%%%%%%%%%%%%%%%%
% \paragraph{Main File.}
%
% The main file is called |cdocsamp.tex|.
%
% Load the \textsf{childdoc} definitions and
% declare the filename for the main document:
%    \begin{macrocode}
\input{childdoc.def}
\childdocmain{}
%    \end{macrocode}

% Optional override for |\version| flag:
%    \begin{macrocode}
%%\ifchilddoc\else\providecommand{\version}{draft}\fi
%    \end{macrocode}

% Define the default values for the |\version| flag
% (|final| for the main file and |draft| for childs):
%    \begin{macrocode}
\ifchilddoc
\providecommand{\version}{draft}
\else
\providecommand{\version}{final}
\fi
%    \end{macrocode}

% Load the standard document class:
%    \begin{macrocode}
\documentclass[12pt]{article}
%    \end{macrocode}

% Start the document body:
%    \begin{macrocode}
\begin{document}
%    \end{macrocode}

% Declare a title page.
% Print title, part of document being processed and version flag:
%    \begin{macrocode}
\addtocounter{page}{-1}
\begin{center}
{\LARGE\bfseries{}childdoc example\par}
\vspace{1cm}
\ifchilddoc
\ifchilddocmanual part\else chapter\fi:
`\childdocname' of `\childdocjob'\par
\else
main document: `\childdocjob'\par
\fi
version: \version\par
\end{center}
\newpage
%    \end{macrocode}

% Manually include selected file,
% otherwise process as usual:
%    \begin{macrocode}
\ifchilddocmanual
\section*{part `\childdocname'}
\input{\childdocname}
\else
%    \end{macrocode}

% Include the two chapters:
%    \begin{macrocode}
\include{cdocsch1}
\include{cdocsch2}
%    \end{macrocode}

% Include the two parts unless only chapters should be displayed:
%    \begin{macrocode}
\ifchilddoc\else
\section{part three}
\input{cdocspt3}
\section{part four}
\input{cdocspt4}
\fi
%    \end{macrocode}

% Process as usual until here:
%    \begin{macrocode}
\fi
%    \end{macrocode}

% End of document body:
%    \begin{macrocode}
\end{document}
%    \end{macrocode}
%\iffalse
%</samplemain>
%\fi
%
% %%%%%%%%%%%%%%%%%%%%%%%%%%%%%%%%%%%%%%
% \paragraph{Chapter Include Files.}
%
% The include files are called |cdocsch1.tex| and |cdocsch2.tex|.
%
%\iffalse
%<*samplechap1|samplechap2>
%\fi

% Optional override for |\version| flag:
%    \begin{macrocode}
%%\providecommand{\version}{final}
%    \end{macrocode}

% Include the main document:
%    \begin{macrocode}
\input{childdoc.def}
\childdocof{cdocsamp}
%    \end{macrocode}

%\iffalse
%</samplechap1|samplechap2>
%\fi
%
%\iffalse
%<*samplechap1>
%\fi
% Some text for chapter 1:
%    \begin{macrocode}
\section{one}
some text in chapter one
%    \end{macrocode}

%\iffalse
%</samplechap1>
%\fi
% Some text for chapter 2:
%\iffalse
%<*samplechap2>
%\fi
%    \begin{macrocode}
\section{two}
more text in chapter two
%    \end{macrocode}

%\iffalse
%</samplechap2>
%\fi
%
% %%%%%%%%%%%%%%%%%%%%%%%%%%%%%%%%%%%%%%
% \paragraph{Part Include Files.}
%
% The include files are called |cdocspt3.tex| and |cdocspt4.tex|.
%
%\iffalse
%<*samplepart3|samplepart4>
%\fi

% Optional override for |\version| flag:
%    \begin{macrocode}
%%\providecommand{\version}{final}
%    \end{macrocode}

% Include the main document:
%    \begin{macrocode}
\input{childdoc.def}
\childdocby{cdocsamp}
%    \end{macrocode}

%\iffalse
%</samplepart3|samplepart4>
%\fi
%
%\iffalse
%<*samplepart3>
%\fi
% Some text for part 3:
%    \begin{macrocode}
some text in part three
%    \end{macrocode}

%\iffalse
%</samplepart3>
%\fi
% Some text for part 4:
%\iffalse
%<*samplepart4>
%\fi
%    \begin{macrocode}
more text in part four
%    \end{macrocode}

%\iffalse
%</samplepart4>
%\fi
%
% %%%%%%%%%%%%%%%%%%%%%%%%%%%%%%%%%%%%%%
% \paragraph{Forwarding for a Complete Draft.}
%
% The following forwarding file |cdocsdrf.tex|
% compiles the main document in draft mode:
%\iffalse
%<*sampledraft>
%\fi
%    \begin{macrocode}
\def\version{draft}
\input{childdoc.def}
\childdocforward{cdocsamp}
%    \end{macrocode}

%\iffalse
%</sampledraft>
%\fi
%
% %%%%%%%%%%%%%%%%%%%%%%%%%%%%%%%%%%%%%%
% \paragraph{Forwarding for Final Version of the Chapters.}
%
% The following forwarding files |cdocsfn1.tex| and |cdocsfn2.tex|
% (with identical content)
% compile the final versions of the child documents
% |cdocsch1.tex| and |cdocsch2.tex|, respectively:
%\iffalse
%<*samplefinal>
%\fi
%    \begin{macrocode}
\def\version{final}
\input{childdoc.def}
\childdocforwardprefix[cdocsamp]{cdocsfn}{cdocsch}
%    \end{macrocode}

%\iffalse
%</samplefinal>
%\fi
%
% %%%%%%%%%%%%%%%%%%%%%%%%%%%%%%%%%%%%%%
% \paragraph{Command Line Processing.}
%
% The following three command lines generate the output files
% |cdocscld|, |cdocscl1| and |cdocscl2|
% which should be identical to
% |cdocsdrf|, |cdocsch1| and |cdocsfn2|, respectively:
% \begin{center}
% \begin{tabular}{l}
% |latex -jobname cdocscld \|\\
% |  "\def\version{draft}\input{childdoc.def}\childdocforward{cdocsamp}"|\\
% |latex -jobname cdocscl1 \|\\
% |  "\input{childdoc.def}\childdocforward[cdocsamp]{cdocsch1}"|\\
% |latex -jobname cdocscl2 \|\\
% |  "\def\version{final}\input{childdoc.def}\childdocforward{cdocsch2}"|
% \end{tabular}
% \end{center}
% Note that the trailing backslash on each first line
% merely continues the input to the second line
% (for convenient cut ant paste).
% Furthermore, the command |latex| can be replaced by any
% of its alternative versions such as |pdflatex|.
%
% %%%%%%%%%%%%%%%%%%%%%%%%%%%%%%%%%%%%%%%%%%%%%%%%%%%%%%%%%%%%%%%%%%%%%%%%%%%%%%
% %%%%%%%%%%%%%%%%%%%%%%%%%%%%%%%%%%%%%%%%%%%%%%%%%%%%%%%%%%%%%%%%%%%%%%%%%%%%%%
% \section{Implementation}
%\iffalse
%<*package>
%\fi
%
% This section describes the definitions file |childdoc.def|.

% The definitions cannot be loaded using |\usepackage| or |\RequirePackage|
% which has a mechanism to prevent loading a style file more than once.
% When loading the definitions by means of |\input|
% multiple instances have to be prevented manually:
%\iffalse
%This code needs to be before the `\ProvidesFile' directive
%which is defined at the beginning of this file.
%Therefore it is also placed there and commented out here.
%</package>
%<*discard>
%\fi
%    \begin{macrocode}
\ifdefined\childdocmain\endinput\fi
%    \end{macrocode}
%\iffalse
%</discard>
%<*package>
%\fi
%
% \macro{\ifchilddoc}
% \macro{\ifchilddocmanual}
% The conditional |\ifchilddoc| tells whether a
% child (true) or main (false) document is being compiled.
% The conditional |\ifchilddocmanual| tells whether
% the |\includeonly| mechanism is used (false) or
% the selection of child files must be performed manually (true).
% The definitions initialise to false:
%    \begin{macrocode}
\newif\ifchilddoc
\newif\ifchilddocmanual
%    \end{macrocode}

% \macro{\childdocname}
% \macro{\childdocjob}
% The macro |\childdocname| stores the name of the main document
% to be compiled. The macro |\childdocjob| stores the name of
% the document on which the \LaTeX{} compiler was originally invoked.
% The content of |\jobname| cannot be compared
% to filenames specified in the source due to different catcodes.
% The following code rescans |\jobname|, stores the result
% in |\childdocname| and saves a copy in |\childdocjob|:
%    \begin{macrocode}
\edef\childdocname{\scantokens\expandafter{\jobname\noexpand}}
\let\childdocjob\childdocname
%    \end{macrocode}

% \macro{\childdocdisable}
% The macro |\childdocdisable| prevents the main file
% from being processed more than once.
% At this stage, the main document command |\childdocmain|
% is assumed to be called once again where it should do nothing.
% Any subsequent call to it should prevent
% a secondary processing of the main document
% It overwrites the forwarding commands
% |\childdocof| and |\childdocforward|
% with empty macros to prevent further inclusions of the main document:
%    \begin{macrocode}
\newcommand{\childdocdisable}
{
  \renewcommand{\childdocmain}[1]{\renewcommand{\childdocmain}[1]{\endinput}}
  \renewcommand{\childdocof}[1]{}
  \renewcommand{\childdocby}[2][]{}
  \renewcommand{\childdocforward}[2][]{}
  \renewcommand{\childdocdisable}{}
}
%    \end{macrocode}

% \macro{\childdocmain}
% The macro |\childdocmain| is to be called at the top of the main file
% with nothing or the main filename (without extension) as argument.
% First, it breaks loops.
% If the argument is not empty and does not match |\childdocname|
% (which is set by the first inclusion of |childdoc.def|),
% |\ifchilddoc| is set to true, |\includeonly| is applied to the child file
% and |\jobname| is set to the main file
% (for proper handling of |.aux| files):
%    \begin{macrocode}
\newcommand{\childdocmain}[1]
{
  \childdocdisable\childdocmain{}
  \if?#1?\else
    \begingroup
      \def\childdoctmp{#1}
      \ifx\childdoctmp\childdocname
        \def\childdoctmp{}
      \else
        \def\childdoctmp
        {
          \childdoctrue
          \includeonly{\childdocname}
          \def\childdocjob{#1}
          \def\jobname{#1}
        }
      \fi
      \expandafter
    \endgroup
    \childdoctmp
  \fi
}
%    \end{macrocode}

% \macro{\childdocof}
% The command |\childdocof| redirects
% compilation to the main file |#1|.
%    \begin{macrocode}
\newcommand{\childdocof}[1]
{
  \childdocdisable
  \childdoctrue
  \includeonly{\childdocname}
  \def\jobname{#1}
  \def\childdocjob{#1}
  \input{#1}
}
%    \end{macrocode}

% \macro{\childdocby}
% The command |\childdocby| ....
%    \begin{macrocode}
\newcommand{\childdocby}[2][]
{
  \childdocdisable
  \childdoctrue
  \childdocmanualtrue
  \if?#1?\else
    \def\jobname{#2}
  \fi
  \def\childdocjob{#2}
  \input{#2}
  \endinput
}
%    \end{macrocode}

% \macro{\childdocforward}
% The command |\childdocforward| redirects
% compilation to the main file or
% (if the optional argument is given) a child file.
% Parameters are set as if the main file
% or a child file starting with |\childdocof| was compiled.
% Then compilation is handed over to the main file:
%    \begin{macrocode}
\newcommand{\childdocforward}[2][]
{
  \begingroup
    \if?#1?
      \def\childdoctmp
      {
        \def\childdocname{#2}
        \def\childdocjob{#2}
        \def\jobname{#2}
        \input{#2}
        \endinput
      }
    \else
      \def\childdoctmp
      {
        \childdocdisable
        \def\childdocname{#2}
        \childdoctrue
        \includeonly{#2}
        \def\childdocjob{#1}
        \def\jobname{#1}
        \input{#1}
        \endinput
      }
    \fi
    \expandafter
  \endgroup
  \childdoctmp
}
%    \end{macrocode}

% \macro{\childdocforwardprefix}
% The command |\childdocforwardprefix| redirects
% compilation to the main or a child file by means of a pattern.
% The prefix |#1| in the current filename is replaced by |#2|
% and the suffix of the current filename is kept
% (it is assumed that the filename does not contain the substring `|~~~|'
% which is used as a delimiter).
% Compilation is handed over to the new file by |\childdocforward|:
%    \begin{macrocode}
\newcommand{\childdocforwardprefix}[3][]
{
  \begingroup
    \def\childdocextract #2##1~~~{\def\childdoctmp{\childdocforward[#1]{#3##1}}}
    \expandafter\childdocextract\childdocname~~~
    \expandafter
  \endgroup
  \childdoctmp
}
%    \end{macrocode}

% \macro{\childdoc}
% The deprecated macro |\childdoc| is a legacy version of |\childdocmain|:
%    \begin{macrocode}
\newcommand{\childdoc}{\childdocmain}
%    \end{macrocode}

% \macro{\childdocredirect}
% The deprecated macro |\childdocredirect| is a legacy version
% of |\childdocforward| and |\childdocforwardprefix|:
%    \begin{macrocode}
\newcommand{\childdocredirect}[2][]
{
  \begingroup
    \if?#1?
      \def\childdoctmp{\childdocforward{#2}}
    \else
      \def\childdoctmp{\childdocforwardprefix{#1}{#2}}
    \fi
    \expandafter
  \endgroup
  \childdoctmp
}
%    \end{macrocode}

%\iffalse
%</package>
%\fi
%
\endinput
|
and perform the replacements as outlined below.
Instead of |\childdocmain{|\textit{main}|}| add the following code
to the top of the main file:
%
\begin{center}
\begin{tabular}{l}
|\||ifdefined\childdocname\endinput\||fi\newif\ifchilddoc|\\
|\edef\childdocname{\scantokens\expandafter{\jobname\noexpand}}|\\
|\def\childdocmain{|\textit{main}|}\||ifx\childdocmain\childdocname\||else|\\
|\childdoctrue\includeonly{\childdocname}\let\jobname\childdocmain\||fi|\\
\end{tabular}
\end{center}
%
Instead of |\childdocof{|\textit{main}|}| just include the main file
at the top of each child file:
%
\begin{center}
|\input{|\textit{main}|}|
\end{center}
%
A simple redirection |\childdocforward{|\textit{dest}|}| is achieved by:
%
\begin{center}
|\def\jobname{|\textit{dest}|}\input{\jobname}|
\end{center}
%
The redirection with prefix
|\childdocforwardprefix[|\textit{prefix}|]{|\textit{dest}|}|
is accomplished by:
%
\begin{center}
\begin{tabular}{l}
|{\edef\jobname{\scantokens\expandafter{\jobname\noexpand}}|\\
|\def\redirectjob |\textit{prefix}|#1~~~{\gdef\jobname{|\textit{dest}|#1}}|\\
|\expandafter\redirectjob\jobname~~~}\input{\jobname}|
\end{tabular}
\end{center}

In an alternative approach,
child documents can be compiled by a specific command line
without additional code or specific definitions:
%
\begin{center}
|... -jobname "|\textit{target}|" "|[\textit{flags}]%
|\includeonly{|\textit{dest}|}\input{|\textit{main}|}"|
\end{center}
%

%%%%%%%%%%%%%%%%%%%%%%%%%%%%%%%%%%%%%%%%%%%%%%%%%%%%%%%%%%%%%%%%%%%%%%%%%%%%%%%%
%%%%%%%%%%%%%%%%%%%%%%%%%%%%%%%%%%%%%%%%%%%%%%%%%%%%%%%%%%%%%%%%%%%%%%%%%%%%%%%%
\section{Information}

%%%%%%%%%%%%%%%%%%%%%%%%%%%%%%%%%%%%%%%%%%%%%%%%%%%%%%%%%%%%%%%%%%%%%%%%%%%%%%%%
\subsection{Copyright}

Copyright \copyright{} 2017--2018 Niklas Beisert

This work may be distributed and/or modified under the
conditions of the \LaTeX{} Project Public License, either version 1.3
of this license or (at your option) any later version.
The latest version of this license is in
  \url{http://www.latex-project.org/lppl.txt}
and version 1.3 or later is part of all distributions of \LaTeX{}
version 2005/12/01 or later.

This work has the LPPL maintenance status `maintained'.

The Current Maintainer of this work is Niklas Beisert.

This work consists of the files |README.txt|, |childdoc.ins| and |childdoc.dtx|
as well as the derived files |childdoc.def|, |cdocsamp.tex|
with |cdocsch1.tex|, |cdocsch2.tex|, |cdocspt3.tex|, |cdocspt4.tex|,
|cdocsdrf.tex|, |cdocsfn1.tex|, |cdocsfn2.tex|
as well as |childdoc.pdf|.

%%%%%%%%%%%%%%%%%%%%%%%%%%%%%%%%%%%%%%%%%%%%%%%%%%%%%%%%%%%%%%%%%%%%%%%%%%%%%%%%
\subsection{Files and Installation}

The package consists of the files:
%
\begin{center}
\begin{tabular}{ll}
    |README.txt|   & readme file \\
    |childdoc.ins| & installation file \\
    |childdoc.dtx| & source file \\
    |childdoc.def| & definition file \\
    |cdocsamp.tex| & sample main file \\
    |cdocsch1.tex| & sample include file \\
    |cdocsch2.tex| & sample include file \\
    |cdocspt3.tex| & sample part file \\
    |cdocspt4.tex| & sample part file \\
    |cdocsdrf.tex| & sample redirection file \\
    |cdocsfn1.tex| & sample redirection file \\
    |cdocsfn2.tex| & sample redirection file \\
    |childdoc.pdf| & manual
\end{tabular}
\end{center}
%
The distribution consists of the files
|README.txt|, |childdoc.ins| and |childdoc.dtx|.
%
\begin{itemize}
\item
Run (pdf)\LaTeX{} on |childdoc.dtx|
to compile the manual |childdoc.pdf| (this file).
\item
Run \LaTeX{} on |childdoc.ins| to create the definitions file |childdoc.def|
and the sample |cdocsamp.tex| with include files
|cdocsch1.tex|, |cdocsch2.tex|, |cdocspt3.tex|, |cdocspt4.tex|,
|cdocsdrf.tex|, |cdocsfn1.tex|, |cdocsfn2.tex|.
Then copy the file |childdoc.def| to an appropriate directory of your \LaTeX{}
distribution, e.g.\ \textit{texmf-root}|/tex/latex/childdoc|.
\end{itemize}

%%%%%%%%%%%%%%%%%%%%%%%%%%%%%%%%%%%%%%%%%%%%%%%%%%%%%%%%%%%%%%%%%%%%%%%%%%%%%%%%
\subsection{Related CTAN Packages}

There are several other packages which offer a similar functionality:
%
\begin{itemize}
\item
The packages
\href{http://ctan.org/pkg/docmute}{\textsf{docmute}},
\href{http://ctan.org/pkg/includex}{\textsf{includex}} and
\href{http://ctan.org/pkg/standalone}{\textsf{standalone}}
provide commands to include only the document body of
a child file thus allowing both files to be compiled individually.
\item
The packages \href{http://ctan.org/pkg/subdocs}{\textsf{subdocs}}
and \href{http://ctan.org/pkg/subfiles}{\textsf{subfiles}}
provide structures in which the main and child documents can be
encapsulated and allowing them to be compiled individually.
The inclusion mechanism is different from the conventional |\include|.
\item
The package \href{http://ctan.org/pkg/combine}{\textsf{combine}}
is an elaborate solution to combine several documents into one.
\end{itemize}
%
See also the CTAN topic \href{http://ctan.org/topic/subdocs}{\textsf{subdocs}}
for further related packages.
The present package differs from the above solutions in that
a document structure constructed with the conventional |\include| mechanism
just needs two extra commands at the top of every file
such that all constituent files can be compiled individually.

%%%%%%%%%%%%%%%%%%%%%%%%%%%%%%%%%%%%%%%%%%%%%%%%%%%%%%%%%%%%%%%%%%%%%%%%%%%%%%%%
%\subsection{Feature Suggestions}
%
%The following is a list of features which may be useful for future
%versions of this package:
%%
%\begin{itemize}
%\item
%\ldots
%\end{itemize}

%%%%%%%%%%%%%%%%%%%%%%%%%%%%%%%%%%%%%%%%%%%%%%%%%%%%%%%%%%%%%%%%%%%%%%%%%%%%%%%%
\subsection{Revision History}

%%%%%%%%%%%%%%%%%%%%%%%%%%%%%%%%%%%%%%%%
\paragraph{v2.0:} 2018/12/30

\begin{itemize}
\item
immediate forward processing
\item
added |\childdocby| mechanism
\item
manual restructured
\end{itemize}

%%%%%%%%%%%%%%%%%%%%%%%%%%%%%%%%%%%%%%%%
\paragraph{v1.6:} 2018/01/17

\begin{itemize}
\item
application for development of include files
\item
corrections to manual
\end{itemize}

%%%%%%%%%%%%%%%%%%%%%%%%%%%%%%%%%%%%%%%%
\paragraph{v1.5:} 2017/05/21

\begin{itemize}
\item
more complete structuring introduced
\item
|\childdocof| introduced
\item
|\childdoc| renamed to |\childdocmain|
\item
|\childredirect| renamed to |\childdocforward| and |\childdocforwardprefix|
and functionality expanded
\end{itemize}

%%%%%%%%%%%%%%%%%%%%%%%%%%%%%%%%%%%%%%%%
\paragraph{v1.0:} 2017/04/27

\begin{itemize}
\item
manual and install package
\item
first version published on CTAN
\end{itemize}

%%%%%%%%%%%%%%%%%%%%%%%%%%%%%%%%%%%%%%%%
\paragraph{v0.6:} 2017/04/26

\begin{itemize}
\item
redirection mechanism added
\end{itemize}

%%%%%%%%%%%%%%%%%%%%%%%%%%%%%%%%%%%%%%%%
\paragraph{v0.5:} 2017/04/26

\begin{itemize}
\item
functionality in definition file
\end{itemize}


%%%%%%%%%%%%%%%%%%%%%%%%%%%%%%%%%%%%%%%%%%%%%%%%%%%%%%%%%%%%%%%%%%%%%%%%%%%%%%%%
%%%%%%%%%%%%%%%%%%%%%%%%%%%%%%%%%%%%%%%%%%%%%%%%%%%%%%%%%%%%%%%%%%%%%%%%%%%%%%%%
%%%%%%%%%%%%%%%%%%%%%%%%%%%%%%%%%%%%%%%%%%%%%%%%%%%%%%%%%%%%%%%%%%%%%%%%%%%%%%%%
\appendix

\settowidth\MacroIndent{\rmfamily\scriptsize 000\ }

 \DocInput{childdoc.dtx}

\end{document}
%</driver>
% \fi
%
% %%%%%%%%%%%%%%%%%%%%%%%%%%%%%%%%%%%%%%%%%%%%%%%%%%%%%%%%%%%%%%%%%%%%%%%%%%%%%%
% %%%%%%%%%%%%%%%%%%%%%%%%%%%%%%%%%%%%%%%%%%%%%%%%%%%%%%%%%%%%%%%%%%%%%%%%%%%%%%
% \section{Sample}
%\iffalse
%<*samplemain>
%\fi
%
% The following presents a sample document
% with two chapters, two parts, a title page,
% a compile flag as well as three forwarding files to set the flag.
% It consists of eight |.tex| files:
% \begin{center}
% \begin{tabular}{ll}
% |cdocsamp.tex|&main file\\
% |cdocsch1.tex|&include file for chapter 1\\
% |cdocsch2.tex|&include file for chapter 2\\
% |cdocspt3.tex|&include file for part 3\\
% |cdocspt4.tex|&include file for part 4\\
% |cdocsdrf.tex|&forwarding file for main file in draft mode\\
% |cdocsfi1.tex|&forwarding file for final version of chapter 1\\
% |cdocsfi2.tex|&forwarding file for final version of chapter 2\\
% \end{tabular}
% \end{center}
% Each of the eight files can be compiled directly by the \LaTeX{} compiler.
%
% %%%%%%%%%%%%%%%%%%%%%%%%%%%%%%%%%%%%%%
% \paragraph{Main File.}
%
% The main file is called |cdocsamp.tex|.
%
% Load the \textsf{childdoc} definitions and
% declare the filename for the main document:
%    \begin{macrocode}
% \iffalse
%
% childdoc.dtx Copyright (C) 2017-2018 Niklas Beisert
%
% This work may be distributed and/or modified under the
% conditions of the LaTeX Project Public License, either version 1.3
% of this license or (at your option) any later version.
% The latest version of this license is in
%   http://www.latex-project.org/lppl.txt
% and version 1.3 or later is part of all distributions of LaTeX
% version 2005/12/01 or later.
%
% This work has the LPPL maintenance status `maintained'.
%
% The Current Maintainer of this work is Niklas Beisert.
%
% This work consists of the files childdoc.dtx and childdoc.ins
% and the derived files childdoc.def and cdocsamp.tex with
% cdocsch1.tex, cdocsch2.tex, cdocsdrf.tex, cdocsfn1.tex, cdocsfn2.tex.
%
%<package>\ifdefined\childdocmain\endinput\fi
%<package>\ProvidesFile{childdoc.def}[2018/12/30 v2.0 child document driver]
%<samplemain>\ProvidesFile{cdocsamp.tex}[2018/12/30 v2.0 sample for childdoc]
%<*driver>
%\ProvidesFile{childdoc.drv}[2018/12/30 v2.0 childdoc reference manual file]
\PassOptionsToClass{10pt,a4paper}{article}
\documentclass{ltxdoc}

\usepackage[margin=35mm]{geometry}
\usepackage{hyperref}
\usepackage{hyperxmp}
\usepackage[usenames]{color}

\hypersetup{colorlinks=true}
\hypersetup{pdfstartview=FitH}
\hypersetup{pdfpagemode=UseNone}
\hypersetup{pdfsource={}}
\hypersetup{pdflang={en-UK}}
\hypersetup{pdfcopyright={Copyright 2017-2018 Niklas Beisert.
  This work may be distributed and/or modified under the
  conditions of the LaTeX Project Public License, either version 1.3
  of this license or (at your option) any later version.}}
\hypersetup{pdflicenseurl={http://www.latex-project.org/lppl.txt}}
\hypersetup{pdfcontactaddress={ETH Zurich, ITP, HIT K,
  Wolfgang-Pauli-Strasse 27}}
\hypersetup{pdfcontactpostcode={8093}}
\hypersetup{pdfcontactcity={Zurich}}
\hypersetup{pdfcontactcountry={Switzerland}}
\hypersetup{pdfcontactemail={nbeisert@itp.phys.ethz.ch}}
\hypersetup{pdfcontacturl={http://people.phys.ethz.ch/\xmptilde nbeisert/}}

\newcommand{\secref}[1]{\hyperref[#1]{section \ref*{#1}}}

\parskip1ex
\parindent0pt
\let\olditemize\itemize
\def\itemize{\olditemize\parskip0pt}

\begin{document}

\title{The \textsf{childdoc} Package}
\hypersetup{pdftitle={The childdoc Package}}
\author{Niklas Beisert\\[2ex]
  Institut f\"ur Theoretische Physik\\
  Eidgen\"ossische Technische Hochschule Z\"urich\\
  Wolfgang-Pauli-Strasse 27, 8093 Z\"urich, Switzerland\\[1ex]
  \href{mailto:nbeisert@itp.phys.ethz.ch}
  {\texttt{nbeisert@itp.phys.ethz.ch}}}
\hypersetup{pdfauthor={Niklas Beisert}}
\hypersetup{pdfsubject={Manual for the LaTeX2e Package childdoc}}
\date{30 December 2018, \textsf{v2.0}}
\maketitle

\begin{abstract}\noindent
\textsf{childdoc} is a \LaTeXe{} package
that enables the direct compilation
of document sections included by |\include|
to individual files.
\end{abstract}

\begingroup
\parskip0ex
\tableofcontents
\endgroup

%%%%%%%%%%%%%%%%%%%%%%%%%%%%%%%%%%%%%%%%%%%%%%%%%%%%%%%%%%%%%%%%%%%%%%%%%%%%%%%%
%%%%%%%%%%%%%%%%%%%%%%%%%%%%%%%%%%%%%%%%%%%%%%%%%%%%%%%%%%%%%%%%%%%%%%%%%%%%%%%%
\section{Introduction}

\LaTeX{} provides a mechanism to structure a large document (such as a book)
into a main file and several child files (containing the chapters)
using the |\include| command.
This mechanism is beneficial for documents
which span hundreds of pages in order to
make the source file(s) more manageable.
Moreover, compilation can be restricted to
selected child files by means of the |\includeonly| command.
The latter feature can be used to reduce the compilation time while editing
(this was significantly more useful in the earlier days of \LaTeX{})
or to generate a smaller document which is easier to navigate.
Another application of |\includeonly| is to generate
documents consisting of selected parts of the complete document.

However, there are a few drawbacks of the plain |\include| mechanism:
\begin{itemize}
\item
The child files cannot be compiled on their own,
they can only be compiled via the main file.
A naive editing environment
(such as a text editor with an option
to have the current file processed by \LaTeX)
may require one to switch to the main file before compiling;
attempting to compile the child file produces errors.
\item
The main file must be modified (each time)
to adjust the |\includeonly| command
to the present needs. This easily leaves the main file in a messy state.
\item
The generated document will always carry the filename
of the main document. This is inconvenient if
several child files are to be compiled and
to be kept for distribution.
\end{itemize}

The present package provides a simple interface
to make child files individually compilable by \LaTeX{}.
Compiling a child file then has the same effect as compiling
the main file with an |\includeonly| command
to select the appropriate child.
Moreover the generated document will carry the name of the child
rather than the main file.
This resolves all three above issues.

This feature is meant to make the editing of books,
thesis documents and lecture notes somewhat more convenient.
However, the package can also be used efficiently for
composing a series of documents (such as exercise sheets)
which are typically distributed individually.
It then assists the author in generating the individual documents
(potentially in different versions)
as well as a document containing the collected series.
Another application is in developing style files
or other kinds of included material
where compilation of the style file could redirect
to a sample or test file.

%%%%%%%%%%%%%%%%%%%%%%%%%%%%%%%%%%%%%%%%%%%%%%%%%%%%%%%%%%%%%%%%%%%%%%%%%%%%%%%%
%%%%%%%%%%%%%%%%%%%%%%%%%%%%%%%%%%%%%%%%%%%%%%%%%%%%%%%%%%%%%%%%%%%%%%%%%%%%%%%%
\section{Usage}

First of all, the package \textsf{childdoc} is \emph{not} a standard
\LaTeXe{} |.sty| style file! Therefore it needs to be invoked in
a non-standard way.

%%%%%%%%%%%%%%%%%%%%%%%%%%%%%%%%%%%%%%%%%%%%%%%%%%%%%%%%%%%%%%%%%%%%%%%%%%%%%%%%
\subsection{Included Files}
\label{sec:include}

%%%%%%%%%%%%%%%%%%%%%%%%%%%%%%%%%%%%%%%%
\DescribeMacro{\childdocmain}
To use the package, add the commands
\begin{center}
\begin{tabular}{l}
|\input{childdoc.def}|\\
|\childdocmain{}|\\
\end{tabular}
\end{center}
at the very top of the main \LaTeX{} file,
in particular \emph{before} the |\documentclass| statement!
The argument of |\childdocmain| should be left empty
(but it must be present).

%%%%%%%%%%%%%%%%%%%%%%%%%%%%%%%%%%%%%%%%
\DescribeMacro{\childdocof}
Furthermore, add the commands
\begin{center}
\begin{tabular}{l}
|\input{childdoc.def}|\\
|\childdocof{|\textit{main}|}|\\
\end{tabular}
\end{center}
at the top of every child file \textit{child}
which is included by |\include{|\textit{child}|}|
from within the main file
(or at least for those files to be compiled individually).
The argument \textit{main} must be the filename of the main file.

There are a couple of
considerations in setting up the main and child documents:

%%%%%%%%%%%%%%%%%%%%%%%%%%%%%%%%%%%%%%%%
\paragraph{Restrictions.}

Please note the following restrictions:
\begin{itemize}
\item
|\childdocmain| must be called with one argument \textit{main}
to ensure compatibility with earlier version of the package.
It must either be empty (|\childdocmain{}|)
or precisely match the filename of the main file in which it is specified.
See \secref{sec:detection} for further information.
\item
The filename \textit{main} must be specified without the |.tex| extension.
\item
The filename \textit{main} is case sensitive
(even in case-insensitive file systems)
due to internal string comparison.
\item
The argument \textit{main} should be fully expanded, it cannot be a macro.
\item
Subdirectories and special characters should be avoided in filenames.
\item
The command |\childdocmain{|\textit{main}|}| must be followed by a whitespace.
It should not be followed immediately by another command
or by a comment mark `|%|'.
This is because the \TeX{} parser reads the token immediately following
the argument of |\childdocmain| and puts it
at the beginning of every child section;
however, a white\-space is ignored.
\end{itemize}

%%%%%%%%%%%%%%%%%%%%%%%%%%%%%%%%%%%%%%%%
\paragraph{Content of Main File.}

It is advisable to place all content in the child files included by |\include|.
Any output contained in the main file will appear in all child documents
unless suppressed manually;
it cannot be suppressed automatically by the |\includeonly| directive
and thus should normally be avoided.
A method to include some content in the main file
by means of conditional processing is described in \secref{sec:conditional}.

%%%%%%%%%%%%%%%%%%%%%%%%%%%%%%%%%%%%%%%%
\paragraph{Page Numbering.}

When only a part of the document is compiled,
the appropriate numbering of pages
(as well as other status parameters)
is determined from the |.aux| files.
The latter contain information from previous passes.
However this information needs to propagate through
all intermediate child documents.
Therefore the page numbering in child documents may well
be inconsistent until the complete document is compiled at least once.

A useful (if unconventional) way to always ensure a consistent
page numbering is to restart the numbering in each child document
and denote the pages by `\textit{child}|.|\textit{page}'
where \textit{child} represents the chapter/section number of the child file.
This can be achieved by the command
|\numberwithin{page}{|\textit{child}|}|
of the \textsf{amsmath} package
where \textit{child} can be |chapter| or |section|
depending on the chosen structuring.
Alternatively, one can modify the macro |\thepage| appropriately
and reset the counter |page| at the start of each child file.

%%%%%%%%%%%%%%%%%%%%%%%%%%%%%%%%%%%%%%%%%%%%%%%%%%%%%%%%%%%%%%%%%%%%%%%%%%%%%%%%
\subsection{Conditional Processing}
\label{sec:conditional}

The package provides a mechanism to compile different versions
of a document. To customise the versions further some conditional processing
can come in handy to distinguish which version is being compiled.
The package provides two macros to describe the compilation context:

%%%%%%%%%%%%%%%%%%%%%%%%%%%%%%%%%%%%%%%%
\DescribeMacro{\ifchilddoc}
The conditional |\ifchilddoc| distinguishes between the compilation of
child documents and the main document:
%
\begin{center}
|\ifchilddoc |\textit{child-code}| |[|\||else |\textit{main-code}]| \||fi|
\end{center}

%%%%%%%%%%%%%%%%%%%%%%%%%%%%%%%%%%%%%%%%
\DescribeMacro{\childdocname}
\DescribeMacro{\childdocjob}
The macro |\childdocname| contains the filename (without extension)
of the main or child file being processed.
Note that |\childdocjob| will always contain the name of the main file.

%%%%%%%%%%%%%%%%%%%%%%%%%%%%%%%%%%%%%%%%
\paragraph{Title Page.}

Conditional processing can be used to include a title or banner page
in the main document when proper precautions are taken.
Importantly, the code in the main file should ensure that the page counter
(as well as other status parameters which are stored in the |.aux| files)
takes the same value after the conditional processing.
Otherwise the page numbers may take divergent values
depending on which part is compiled.

For example, a title page could be declared by:
%
\begin{center}
\begin{tabular}{l}
|\ifchilddoc\||else|\\
|\addtocounter{page}{-1}|\\
\textit{code for title page}\\
|\newpage|\\
|\||fi|
\end{tabular}
\end{center}
%
A banner page for the child documents can be generated by:
%
\begin{center}
\begin{tabular}{l}
|\ifchilddoc|\\
|\addtocounter{page}{-1}|\\
\textit{code for banner page}\\
|\newpage|\\
|\||fi|
\end{tabular}
\end{center}
%
Here one could write a message such as:
\begin{center}
|This is the part \childdocname{} of \childdocjob{}.|
\end{center}

%%%%%%%%%%%%%%%%%%%%%%%%%%%%%%%%%%%%%%%%%%%%%%%%%%%%%%%%%%%%%%%%%%%%%%%%%%%%%%%%
\subsection{Flags}
\label{sec:flags}

The package makes it easy to generate different versions
of the main or child documents.
To this end compilation flags can be defined
and assigned different default values.
They will be particularly useful in conjunction
with the forwarding mechanism described in \secref{sec:forward}.

For example, it may be useful to have a flag |\version|
which can be set to |draft| or |final|.
The document source will contain some conditional code
depending on the value of |\version|.
Suppose further, the flag should default to |final| for the main file
and to |draft| for child files
which is a natural assignment for editing the document.
This is achieved by placing the following code
in the preamble of the main document
(below the |\childdocmain| directive):
%
\begin{center}
\begin{tabular}{l}
|\ifchilddoc|\\
|\providecommand{\version}{draft}|\\
|\||else|\\
|\providecommand{\version}{final}|\\
|\||fi|
\end{tabular}
\end{center}
%
The definition by |\providecommand| makes sure
that previous definitions are not overwritten.
Further statements |\providecommand{\version}{...}|
can thus be added before the above code to override it.

For the main file, one might add a line
(between |\childdocmain| and the above block)
%
\begin{center}
|%\ifchilddoc\||else\providecommand{\version}{draft}\||fi|
\end{center}
%
which can be uncommented to produce a draft version.
Likewise one can add a line to the very top of a child file
(above the |\childdocof{|\textit{main}|}| directive)
%
\begin{center}
|%\providecommand{\version}{final}|
\end{center}
%
which can be uncommented to produce the final version of this child document.

%%%%%%%%%%%%%%%%%%%%%%%%%%%%%%%%%%%%%%%%%%%%%%%%%%%%%%%%%%%%%%%%%%%%%%%%%%%%%%%%
\subsection{Forwarding}
\label{sec:forward}

Different versions of the main or child documents
using compilation flags as described in \secref{sec:flags}
can be (permanently) stored in different files
for convenient compilation, viewing and distribution.
To this end, the package defines a command
to pass on compilation to a different file:

%%%%%%%%%%%%%%%%%%%%%%%%%%%%%%%%%%%%%%%%
\DescribeMacro{\childdocforward}
The command |\childdocforward| redirects processing to
another source file:
%
\begin{center}
\begin{tabular}{l}
|\input{childdoc.def}|\\
|\childdocforward[|\textit{main}|]{|\textit{dest}|}|\\
\end{tabular}
\end{center}
%
The argument \textit{dest} is the destination file
(without extension).
It should be the main file or one of the child files.
Note that further \textsf{childdoc} directives
such as |\childdocof| and |\childdocforward|
in the indicated file will be processed in this form.
The optional argument \textit{main}
passes on directly to the main file \textit{main}
while pretending to compile the child \textit{dest}.
This form behaves as if \textit{dest}
issues |\childdocof{|\textit{main}|}| right away,
and no further \textsf{childdoc} directives will be processed.

%%%%%%%%%%%%%%%%%%%%%%%%%%%%%%%%%%%%%%%%
\DescribeMacro{\...prefix}
In the alternative form |\childdocforwardprefix|,
%
\begin{center}
\begin{tabular}{l}
|\input{childdoc.def}|\\
|\childdocforwardprefix[|\textit{main}|]{|\textit{prefix}|}{|\textit{dest}|}|
\end{tabular}
\end{center}
%
the destination file is determined by a pattern
depending on the current file:
To make this work, the current file must be called
`{\textit{prefix}\hspace{0.2em}\textit{suffix}}'
with \textit{prefix} matching precisely the argument.
Processing is then passed on to the file
`{\textit{dest}\hspace{0.2em}\textit{suffix}}'.
Surely, the same effect is achieved by
directly specifying the
argument `{\textit{dest}\hspace{0.2em}\textit{suffix}}'
in the first form.
However, that requires to set up a different file
for each child. With the alternative form of the command
all these files can have exactly the same content
which simplifies setting them up and maintaining them.

For example, the following file |draft.tex|
with a compilation flag |\version| as described in \secref{sec:flags}
compiles the main document as a draft:
%
\begin{center}
\begin{tabular}{l}
|\def\version{draft}|\\
|\input{childdoc.def}|\\
|\childdocforward{|\textit{main}|}|
\end{tabular}
\end{center}
%
Likewise, the following files |final|\textit{nn}|.tex|
compile the final version of the child document
|child|\textit{nn}|.tex|:
%
\begin{center}
\begin{tabular}{l}
|\def\version{final}|\\
|\input{childdoc.def}|\\
|\childdocforwardprefix{final}{child}|
\end{tabular}
\end{center}
%

Note that when several versions of a main file and/or of each child file
are to be generated, it may be convenient to set up a |Makefile| or
shell script to automatise the process.

%%%%%%%%%%%%%%%%%%%%%%%%%%%%%%%%%%%%%%%%%%%%%%%%%%%%%%%%%%%%%%%%%%%%%%%%%%%%%%%%
\subsection{Command Line Processing}
\label{sec:commandline}

The effect of redirection files can also be achieved by invoking
the \LaTeX{} compiler with a more elaborate command line.
Most conveniently this should be done as part
of a shell script or a |Makefile|.

When using \textsf{childdoc} in the main file, the following
command lines effectively perform a redirection
(note that depending on the shell being used,
backslashes may have to be doubled: `|\|' $\to$ `|\\|'):
%
\begin{center}
|... -jobname "|\textit{target}|" |\\|"|[\textit{flags}]%
|\input{childdoc.def}\childdocforward[|\textit{main}|]{|\textit{dest}|}"|
\end{center}
%
Here \textit{target} is the name of the output file,
\textit{main} is the name of the main file
and \textit{dest} is the name of the main or child file to be processed
(all filenames without extensions).
The optional argument \textit{main} can be omitted
if \textit{main} matches \textit{dest}.
Optionally, compilation \textit{flags} can be defined via |\def| commands.
This command line makes the \TeX{} engine believe
it is compiling the file \textit{target}
whose content is specified as the latter parameter.
The provided code then forwards the processing to
\textit{main} or \textit{dest} as described in \secref{sec:forward}.

%%%%%%%%%%%%%%%%%%%%%%%%%%%%%%%%%%%%%%%%%%%%%%%%%%%%%%%%%%%%%%%%%%%%%%%%%%%%%%%%
\subsection{Include by Input}
\label{sec:input}

Including child documents by |\include| has some restrictions by design.
Most notably, the content of a child document always occupies
its own set of pages; pages cannot be shared between child documents.
Usually, this behaviour makes perfect sense
because each child document contain an essential part of the document.
However, in some situations it may be desirable to compose
a document from a collection of parts
without having mandatory page breaks between then.
For this case, the package
provides a mechanism to include parts
by |\input| which can also be processed individually.
However, by construction this mechanism
requires manual handling of the content to be output.

%%%%%%%%%%%%%%%%%%%%%%%%%%%%%%%%%%%%%%%%
\DescribeMacro{\ifchilddocmanual}
The main file should be prepared as usual, see \secref{sec:include}.
However, the document body must make a distinction
between processing of an individual part and of the main document, e.g.:
%
\begin{center}
\begin{tabular}{l}
|\ifchilddocmanual|\\
|\input{\childdocname}|\\
|\||else|\\
\textit{document body with }|\input{|\textit{part}|}|\\
|\||fi|
\end{tabular}
\end{center}
%
The conditional |\ifchilddocmanual| is true whenever
a part to be included by |\input| is being compiled,
and the name of the part is stored in |\childdocname|.

%%%%%%%%%%%%%%%%%%%%%%%%%%%%%%%%%%%%%%%%
\DescribeMacro{\childdocby}
Each part to be included by |\input| should start with:
%
\begin{center}
\begin{tabular}{l}
|\input{childdoc.def}|\\
|\childdocby{|\textit{main}|}|\\
\end{tabular}
\end{center}
%
The directive |\childdocby| is similar to |\childdocof|
described in \secref{sec:include},
but the subsequent selection of content must be done manually.
To that end, both |\ifchilddoc| and |\ifchilddocmanual|
will be true upon processing of a part,
and the name of the part is stored in |\childdocname|.
Note that |\jobname| will be set to the filename of the current part
so that each part receives an individual |.aux| file
that does not interfere with the |.aux| file(s) of the main document.
This behaviour can be altered by the alternative form
|\childdocby[*]{|\textit{main}|}| (with a non-empty optional argument)
which uses the |.aux| file of the main document
by setting |\jobname| to \textit{main}.

%%%%%%%%%%%%%%%%%%%%%%%%%%%%%%%%%%%%%%%%%%%%%%%%%%%%%%%%%%%%%%%%%%%%%%%%%%%%%%%%
\subsection{Driver Development}
\label{sec:driver}

The \textsf{childdoc} mechanism can also be use for the development
of definition files such as \LaTeX{} styles or classes.
This case differs from the above setup with multiple parts
included by |\include| in that no |\includeonly| should be invoked.
This can be achieved by starting the include file
(before |\ProvidesPackage|) with:
%
\begin{center}
\begin{tabular}{l}
|\input{childdoc.def}|\\
|\childdocforward{|\textit{main}|}|\\
\end{tabular}
\end{center}
%
or alternatively with:
%
\begin{center}
\begin{tabular}{l}
|\input{childdoc.def}|\\
|\childdocby{|\textit{main}|}|\\
\end{tabular}
\end{center}
%
Both forms have slightly different effects as described above.
The main file is prepared as usual, see \secref{sec:include}.

%%%%%%%%%%%%%%%%%%%%%%%%%%%%%%%%%%%%%%%%%%%%%%%%%%%%%%%%%%%%%%%%%%%%%%%%%%%%%%%%
\subsection{Legacy Detection}
\label{sec:detection}

The directive |\childdocmain| in the main file can detect
whether the complete document or merely a child is to be compiled
even without using the directive |\childdocof|.
This method is deprecated because it is less robust
and there is no compelling reason to use it;
it is merely provided for backward compatibility
and it may be removed in future versions.

If the detection mechanism is to be used,
it is mandatory to correctly specify
the filename of the main file as the argument of |\childdocmain|:
%
\begin{center}
\begin{tabular}{l}
|\input{childdoc.def}|\\
|\childdocmain{|\textit{main}|}|\\
\end{tabular}
\end{center}
%
If |\jobname| does not match the argument \textit{main} of |\childdocmain|,
it is assumed that |\jobname| points to the child file to be compiled.
When using |\childdocmain| with the main file specified as argument,
it suffices to start a child file
with just |\input{|\textit{main}|}|
without loading of the package and using |\childdocof|.
If instead all processing is done
with the appropriate \textsf{childdoc} directives,
the argument of \textit{main} of |\childdocmain| can be empty.

An alternative version of the command line processing described
in \secref{sec:commandline} using the detection mechanism reads:
%
\begin{center}
|... -jobname "|\textit{target}|" "|[\textit{flags}]%
[|\def\jobname{|\textit{dest}|}|]|\input{|\textit{main}|}"|
\end{center}

%%%%%%%%%%%%%%%%%%%%%%%%%%%%%%%%%%%%%%%%%%%%%%%%%%%%%%%%%%%%%%%%%%%%%%%%%%%%%%%%
\subsection{Manual Code}
\label{sec:manual}

In case one cannot be certain whether the definitions file |childdoc.def|
is installed on the target \TeX{} distribution
and one prefers not to ship it,
it is conceivable to paste a few relevant commands into the sources.

To that end, drop all statements |\input{childdoc.def}|
and perform the replacements as outlined below.
Instead of |\childdocmain{|\textit{main}|}| add the following code
to the top of the main file:
%
\begin{center}
\begin{tabular}{l}
|\||ifdefined\childdocname\endinput\||fi\newif\ifchilddoc|\\
|\edef\childdocname{\scantokens\expandafter{\jobname\noexpand}}|\\
|\def\childdocmain{|\textit{main}|}\||ifx\childdocmain\childdocname\||else|\\
|\childdoctrue\includeonly{\childdocname}\let\jobname\childdocmain\||fi|\\
\end{tabular}
\end{center}
%
Instead of |\childdocof{|\textit{main}|}| just include the main file
at the top of each child file:
%
\begin{center}
|\input{|\textit{main}|}|
\end{center}
%
A simple redirection |\childdocforward{|\textit{dest}|}| is achieved by:
%
\begin{center}
|\def\jobname{|\textit{dest}|}\input{\jobname}|
\end{center}
%
The redirection with prefix
|\childdocforwardprefix[|\textit{prefix}|]{|\textit{dest}|}|
is accomplished by:
%
\begin{center}
\begin{tabular}{l}
|{\edef\jobname{\scantokens\expandafter{\jobname\noexpand}}|\\
|\def\redirectjob |\textit{prefix}|#1~~~{\gdef\jobname{|\textit{dest}|#1}}|\\
|\expandafter\redirectjob\jobname~~~}\input{\jobname}|
\end{tabular}
\end{center}

In an alternative approach,
child documents can be compiled by a specific command line
without additional code or specific definitions:
%
\begin{center}
|... -jobname "|\textit{target}|" "|[\textit{flags}]%
|\includeonly{|\textit{dest}|}\input{|\textit{main}|}"|
\end{center}
%

%%%%%%%%%%%%%%%%%%%%%%%%%%%%%%%%%%%%%%%%%%%%%%%%%%%%%%%%%%%%%%%%%%%%%%%%%%%%%%%%
%%%%%%%%%%%%%%%%%%%%%%%%%%%%%%%%%%%%%%%%%%%%%%%%%%%%%%%%%%%%%%%%%%%%%%%%%%%%%%%%
\section{Information}

%%%%%%%%%%%%%%%%%%%%%%%%%%%%%%%%%%%%%%%%%%%%%%%%%%%%%%%%%%%%%%%%%%%%%%%%%%%%%%%%
\subsection{Copyright}

Copyright \copyright{} 2017--2018 Niklas Beisert

This work may be distributed and/or modified under the
conditions of the \LaTeX{} Project Public License, either version 1.3
of this license or (at your option) any later version.
The latest version of this license is in
  \url{http://www.latex-project.org/lppl.txt}
and version 1.3 or later is part of all distributions of \LaTeX{}
version 2005/12/01 or later.

This work has the LPPL maintenance status `maintained'.

The Current Maintainer of this work is Niklas Beisert.

This work consists of the files |README.txt|, |childdoc.ins| and |childdoc.dtx|
as well as the derived files |childdoc.def|, |cdocsamp.tex|
with |cdocsch1.tex|, |cdocsch2.tex|, |cdocspt3.tex|, |cdocspt4.tex|,
|cdocsdrf.tex|, |cdocsfn1.tex|, |cdocsfn2.tex|
as well as |childdoc.pdf|.

%%%%%%%%%%%%%%%%%%%%%%%%%%%%%%%%%%%%%%%%%%%%%%%%%%%%%%%%%%%%%%%%%%%%%%%%%%%%%%%%
\subsection{Files and Installation}

The package consists of the files:
%
\begin{center}
\begin{tabular}{ll}
    |README.txt|   & readme file \\
    |childdoc.ins| & installation file \\
    |childdoc.dtx| & source file \\
    |childdoc.def| & definition file \\
    |cdocsamp.tex| & sample main file \\
    |cdocsch1.tex| & sample include file \\
    |cdocsch2.tex| & sample include file \\
    |cdocspt3.tex| & sample part file \\
    |cdocspt4.tex| & sample part file \\
    |cdocsdrf.tex| & sample redirection file \\
    |cdocsfn1.tex| & sample redirection file \\
    |cdocsfn2.tex| & sample redirection file \\
    |childdoc.pdf| & manual
\end{tabular}
\end{center}
%
The distribution consists of the files
|README.txt|, |childdoc.ins| and |childdoc.dtx|.
%
\begin{itemize}
\item
Run (pdf)\LaTeX{} on |childdoc.dtx|
to compile the manual |childdoc.pdf| (this file).
\item
Run \LaTeX{} on |childdoc.ins| to create the definitions file |childdoc.def|
and the sample |cdocsamp.tex| with include files
|cdocsch1.tex|, |cdocsch2.tex|, |cdocspt3.tex|, |cdocspt4.tex|,
|cdocsdrf.tex|, |cdocsfn1.tex|, |cdocsfn2.tex|.
Then copy the file |childdoc.def| to an appropriate directory of your \LaTeX{}
distribution, e.g.\ \textit{texmf-root}|/tex/latex/childdoc|.
\end{itemize}

%%%%%%%%%%%%%%%%%%%%%%%%%%%%%%%%%%%%%%%%%%%%%%%%%%%%%%%%%%%%%%%%%%%%%%%%%%%%%%%%
\subsection{Related CTAN Packages}

There are several other packages which offer a similar functionality:
%
\begin{itemize}
\item
The packages
\href{http://ctan.org/pkg/docmute}{\textsf{docmute}},
\href{http://ctan.org/pkg/includex}{\textsf{includex}} and
\href{http://ctan.org/pkg/standalone}{\textsf{standalone}}
provide commands to include only the document body of
a child file thus allowing both files to be compiled individually.
\item
The packages \href{http://ctan.org/pkg/subdocs}{\textsf{subdocs}}
and \href{http://ctan.org/pkg/subfiles}{\textsf{subfiles}}
provide structures in which the main and child documents can be
encapsulated and allowing them to be compiled individually.
The inclusion mechanism is different from the conventional |\include|.
\item
The package \href{http://ctan.org/pkg/combine}{\textsf{combine}}
is an elaborate solution to combine several documents into one.
\end{itemize}
%
See also the CTAN topic \href{http://ctan.org/topic/subdocs}{\textsf{subdocs}}
for further related packages.
The present package differs from the above solutions in that
a document structure constructed with the conventional |\include| mechanism
just needs two extra commands at the top of every file
such that all constituent files can be compiled individually.

%%%%%%%%%%%%%%%%%%%%%%%%%%%%%%%%%%%%%%%%%%%%%%%%%%%%%%%%%%%%%%%%%%%%%%%%%%%%%%%%
%\subsection{Feature Suggestions}
%
%The following is a list of features which may be useful for future
%versions of this package:
%%
%\begin{itemize}
%\item
%\ldots
%\end{itemize}

%%%%%%%%%%%%%%%%%%%%%%%%%%%%%%%%%%%%%%%%%%%%%%%%%%%%%%%%%%%%%%%%%%%%%%%%%%%%%%%%
\subsection{Revision History}

%%%%%%%%%%%%%%%%%%%%%%%%%%%%%%%%%%%%%%%%
\paragraph{v2.0:} 2018/12/30

\begin{itemize}
\item
immediate forward processing
\item
added |\childdocby| mechanism
\item
manual restructured
\end{itemize}

%%%%%%%%%%%%%%%%%%%%%%%%%%%%%%%%%%%%%%%%
\paragraph{v1.6:} 2018/01/17

\begin{itemize}
\item
application for development of include files
\item
corrections to manual
\end{itemize}

%%%%%%%%%%%%%%%%%%%%%%%%%%%%%%%%%%%%%%%%
\paragraph{v1.5:} 2017/05/21

\begin{itemize}
\item
more complete structuring introduced
\item
|\childdocof| introduced
\item
|\childdoc| renamed to |\childdocmain|
\item
|\childredirect| renamed to |\childdocforward| and |\childdocforwardprefix|
and functionality expanded
\end{itemize}

%%%%%%%%%%%%%%%%%%%%%%%%%%%%%%%%%%%%%%%%
\paragraph{v1.0:} 2017/04/27

\begin{itemize}
\item
manual and install package
\item
first version published on CTAN
\end{itemize}

%%%%%%%%%%%%%%%%%%%%%%%%%%%%%%%%%%%%%%%%
\paragraph{v0.6:} 2017/04/26

\begin{itemize}
\item
redirection mechanism added
\end{itemize}

%%%%%%%%%%%%%%%%%%%%%%%%%%%%%%%%%%%%%%%%
\paragraph{v0.5:} 2017/04/26

\begin{itemize}
\item
functionality in definition file
\end{itemize}


%%%%%%%%%%%%%%%%%%%%%%%%%%%%%%%%%%%%%%%%%%%%%%%%%%%%%%%%%%%%%%%%%%%%%%%%%%%%%%%%
%%%%%%%%%%%%%%%%%%%%%%%%%%%%%%%%%%%%%%%%%%%%%%%%%%%%%%%%%%%%%%%%%%%%%%%%%%%%%%%%
%%%%%%%%%%%%%%%%%%%%%%%%%%%%%%%%%%%%%%%%%%%%%%%%%%%%%%%%%%%%%%%%%%%%%%%%%%%%%%%%
\appendix

\settowidth\MacroIndent{\rmfamily\scriptsize 000\ }

 \DocInput{childdoc.dtx}

\end{document}
%</driver>
% \fi
%
% %%%%%%%%%%%%%%%%%%%%%%%%%%%%%%%%%%%%%%%%%%%%%%%%%%%%%%%%%%%%%%%%%%%%%%%%%%%%%%
% %%%%%%%%%%%%%%%%%%%%%%%%%%%%%%%%%%%%%%%%%%%%%%%%%%%%%%%%%%%%%%%%%%%%%%%%%%%%%%
% \section{Sample}
%\iffalse
%<*samplemain>
%\fi
%
% The following presents a sample document
% with two chapters, two parts, a title page,
% a compile flag as well as three forwarding files to set the flag.
% It consists of eight |.tex| files:
% \begin{center}
% \begin{tabular}{ll}
% |cdocsamp.tex|&main file\\
% |cdocsch1.tex|&include file for chapter 1\\
% |cdocsch2.tex|&include file for chapter 2\\
% |cdocspt3.tex|&include file for part 3\\
% |cdocspt4.tex|&include file for part 4\\
% |cdocsdrf.tex|&forwarding file for main file in draft mode\\
% |cdocsfi1.tex|&forwarding file for final version of chapter 1\\
% |cdocsfi2.tex|&forwarding file for final version of chapter 2\\
% \end{tabular}
% \end{center}
% Each of the eight files can be compiled directly by the \LaTeX{} compiler.
%
% %%%%%%%%%%%%%%%%%%%%%%%%%%%%%%%%%%%%%%
% \paragraph{Main File.}
%
% The main file is called |cdocsamp.tex|.
%
% Load the \textsf{childdoc} definitions and
% declare the filename for the main document:
%    \begin{macrocode}
\input{childdoc.def}
\childdocmain{}
%    \end{macrocode}

% Optional override for |\version| flag:
%    \begin{macrocode}
%%\ifchilddoc\else\providecommand{\version}{draft}\fi
%    \end{macrocode}

% Define the default values for the |\version| flag
% (|final| for the main file and |draft| for childs):
%    \begin{macrocode}
\ifchilddoc
\providecommand{\version}{draft}
\else
\providecommand{\version}{final}
\fi
%    \end{macrocode}

% Load the standard document class:
%    \begin{macrocode}
\documentclass[12pt]{article}
%    \end{macrocode}

% Start the document body:
%    \begin{macrocode}
\begin{document}
%    \end{macrocode}

% Declare a title page.
% Print title, part of document being processed and version flag:
%    \begin{macrocode}
\addtocounter{page}{-1}
\begin{center}
{\LARGE\bfseries{}childdoc example\par}
\vspace{1cm}
\ifchilddoc
\ifchilddocmanual part\else chapter\fi:
`\childdocname' of `\childdocjob'\par
\else
main document: `\childdocjob'\par
\fi
version: \version\par
\end{center}
\newpage
%    \end{macrocode}

% Manually include selected file,
% otherwise process as usual:
%    \begin{macrocode}
\ifchilddocmanual
\section*{part `\childdocname'}
\input{\childdocname}
\else
%    \end{macrocode}

% Include the two chapters:
%    \begin{macrocode}
\include{cdocsch1}
\include{cdocsch2}
%    \end{macrocode}

% Include the two parts unless only chapters should be displayed:
%    \begin{macrocode}
\ifchilddoc\else
\section{part three}
\input{cdocspt3}
\section{part four}
\input{cdocspt4}
\fi
%    \end{macrocode}

% Process as usual until here:
%    \begin{macrocode}
\fi
%    \end{macrocode}

% End of document body:
%    \begin{macrocode}
\end{document}
%    \end{macrocode}
%\iffalse
%</samplemain>
%\fi
%
% %%%%%%%%%%%%%%%%%%%%%%%%%%%%%%%%%%%%%%
% \paragraph{Chapter Include Files.}
%
% The include files are called |cdocsch1.tex| and |cdocsch2.tex|.
%
%\iffalse
%<*samplechap1|samplechap2>
%\fi

% Optional override for |\version| flag:
%    \begin{macrocode}
%%\providecommand{\version}{final}
%    \end{macrocode}

% Include the main document:
%    \begin{macrocode}
\input{childdoc.def}
\childdocof{cdocsamp}
%    \end{macrocode}

%\iffalse
%</samplechap1|samplechap2>
%\fi
%
%\iffalse
%<*samplechap1>
%\fi
% Some text for chapter 1:
%    \begin{macrocode}
\section{one}
some text in chapter one
%    \end{macrocode}

%\iffalse
%</samplechap1>
%\fi
% Some text for chapter 2:
%\iffalse
%<*samplechap2>
%\fi
%    \begin{macrocode}
\section{two}
more text in chapter two
%    \end{macrocode}

%\iffalse
%</samplechap2>
%\fi
%
% %%%%%%%%%%%%%%%%%%%%%%%%%%%%%%%%%%%%%%
% \paragraph{Part Include Files.}
%
% The include files are called |cdocspt3.tex| and |cdocspt4.tex|.
%
%\iffalse
%<*samplepart3|samplepart4>
%\fi

% Optional override for |\version| flag:
%    \begin{macrocode}
%%\providecommand{\version}{final}
%    \end{macrocode}

% Include the main document:
%    \begin{macrocode}
\input{childdoc.def}
\childdocby{cdocsamp}
%    \end{macrocode}

%\iffalse
%</samplepart3|samplepart4>
%\fi
%
%\iffalse
%<*samplepart3>
%\fi
% Some text for part 3:
%    \begin{macrocode}
some text in part three
%    \end{macrocode}

%\iffalse
%</samplepart3>
%\fi
% Some text for part 4:
%\iffalse
%<*samplepart4>
%\fi
%    \begin{macrocode}
more text in part four
%    \end{macrocode}

%\iffalse
%</samplepart4>
%\fi
%
% %%%%%%%%%%%%%%%%%%%%%%%%%%%%%%%%%%%%%%
% \paragraph{Forwarding for a Complete Draft.}
%
% The following forwarding file |cdocsdrf.tex|
% compiles the main document in draft mode:
%\iffalse
%<*sampledraft>
%\fi
%    \begin{macrocode}
\def\version{draft}
\input{childdoc.def}
\childdocforward{cdocsamp}
%    \end{macrocode}

%\iffalse
%</sampledraft>
%\fi
%
% %%%%%%%%%%%%%%%%%%%%%%%%%%%%%%%%%%%%%%
% \paragraph{Forwarding for Final Version of the Chapters.}
%
% The following forwarding files |cdocsfn1.tex| and |cdocsfn2.tex|
% (with identical content)
% compile the final versions of the child documents
% |cdocsch1.tex| and |cdocsch2.tex|, respectively:
%\iffalse
%<*samplefinal>
%\fi
%    \begin{macrocode}
\def\version{final}
\input{childdoc.def}
\childdocforwardprefix[cdocsamp]{cdocsfn}{cdocsch}
%    \end{macrocode}

%\iffalse
%</samplefinal>
%\fi
%
% %%%%%%%%%%%%%%%%%%%%%%%%%%%%%%%%%%%%%%
% \paragraph{Command Line Processing.}
%
% The following three command lines generate the output files
% |cdocscld|, |cdocscl1| and |cdocscl2|
% which should be identical to
% |cdocsdrf|, |cdocsch1| and |cdocsfn2|, respectively:
% \begin{center}
% \begin{tabular}{l}
% |latex -jobname cdocscld \|\\
% |  "\def\version{draft}\input{childdoc.def}\childdocforward{cdocsamp}"|\\
% |latex -jobname cdocscl1 \|\\
% |  "\input{childdoc.def}\childdocforward[cdocsamp]{cdocsch1}"|\\
% |latex -jobname cdocscl2 \|\\
% |  "\def\version{final}\input{childdoc.def}\childdocforward{cdocsch2}"|
% \end{tabular}
% \end{center}
% Note that the trailing backslash on each first line
% merely continues the input to the second line
% (for convenient cut ant paste).
% Furthermore, the command |latex| can be replaced by any
% of its alternative versions such as |pdflatex|.
%
% %%%%%%%%%%%%%%%%%%%%%%%%%%%%%%%%%%%%%%%%%%%%%%%%%%%%%%%%%%%%%%%%%%%%%%%%%%%%%%
% %%%%%%%%%%%%%%%%%%%%%%%%%%%%%%%%%%%%%%%%%%%%%%%%%%%%%%%%%%%%%%%%%%%%%%%%%%%%%%
% \section{Implementation}
%\iffalse
%<*package>
%\fi
%
% This section describes the definitions file |childdoc.def|.

% The definitions cannot be loaded using |\usepackage| or |\RequirePackage|
% which has a mechanism to prevent loading a style file more than once.
% When loading the definitions by means of |\input|
% multiple instances have to be prevented manually:
%\iffalse
%This code needs to be before the `\ProvidesFile' directive
%which is defined at the beginning of this file.
%Therefore it is also placed there and commented out here.
%</package>
%<*discard>
%\fi
%    \begin{macrocode}
\ifdefined\childdocmain\endinput\fi
%    \end{macrocode}
%\iffalse
%</discard>
%<*package>
%\fi
%
% \macro{\ifchilddoc}
% \macro{\ifchilddocmanual}
% The conditional |\ifchilddoc| tells whether a
% child (true) or main (false) document is being compiled.
% The conditional |\ifchilddocmanual| tells whether
% the |\includeonly| mechanism is used (false) or
% the selection of child files must be performed manually (true).
% The definitions initialise to false:
%    \begin{macrocode}
\newif\ifchilddoc
\newif\ifchilddocmanual
%    \end{macrocode}

% \macro{\childdocname}
% \macro{\childdocjob}
% The macro |\childdocname| stores the name of the main document
% to be compiled. The macro |\childdocjob| stores the name of
% the document on which the \LaTeX{} compiler was originally invoked.
% The content of |\jobname| cannot be compared
% to filenames specified in the source due to different catcodes.
% The following code rescans |\jobname|, stores the result
% in |\childdocname| and saves a copy in |\childdocjob|:
%    \begin{macrocode}
\edef\childdocname{\scantokens\expandafter{\jobname\noexpand}}
\let\childdocjob\childdocname
%    \end{macrocode}

% \macro{\childdocdisable}
% The macro |\childdocdisable| prevents the main file
% from being processed more than once.
% At this stage, the main document command |\childdocmain|
% is assumed to be called once again where it should do nothing.
% Any subsequent call to it should prevent
% a secondary processing of the main document
% It overwrites the forwarding commands
% |\childdocof| and |\childdocforward|
% with empty macros to prevent further inclusions of the main document:
%    \begin{macrocode}
\newcommand{\childdocdisable}
{
  \renewcommand{\childdocmain}[1]{\renewcommand{\childdocmain}[1]{\endinput}}
  \renewcommand{\childdocof}[1]{}
  \renewcommand{\childdocby}[2][]{}
  \renewcommand{\childdocforward}[2][]{}
  \renewcommand{\childdocdisable}{}
}
%    \end{macrocode}

% \macro{\childdocmain}
% The macro |\childdocmain| is to be called at the top of the main file
% with nothing or the main filename (without extension) as argument.
% First, it breaks loops.
% If the argument is not empty and does not match |\childdocname|
% (which is set by the first inclusion of |childdoc.def|),
% |\ifchilddoc| is set to true, |\includeonly| is applied to the child file
% and |\jobname| is set to the main file
% (for proper handling of |.aux| files):
%    \begin{macrocode}
\newcommand{\childdocmain}[1]
{
  \childdocdisable\childdocmain{}
  \if?#1?\else
    \begingroup
      \def\childdoctmp{#1}
      \ifx\childdoctmp\childdocname
        \def\childdoctmp{}
      \else
        \def\childdoctmp
        {
          \childdoctrue
          \includeonly{\childdocname}
          \def\childdocjob{#1}
          \def\jobname{#1}
        }
      \fi
      \expandafter
    \endgroup
    \childdoctmp
  \fi
}
%    \end{macrocode}

% \macro{\childdocof}
% The command |\childdocof| redirects
% compilation to the main file |#1|.
%    \begin{macrocode}
\newcommand{\childdocof}[1]
{
  \childdocdisable
  \childdoctrue
  \includeonly{\childdocname}
  \def\jobname{#1}
  \def\childdocjob{#1}
  \input{#1}
}
%    \end{macrocode}

% \macro{\childdocby}
% The command |\childdocby| ....
%    \begin{macrocode}
\newcommand{\childdocby}[2][]
{
  \childdocdisable
  \childdoctrue
  \childdocmanualtrue
  \if?#1?\else
    \def\jobname{#2}
  \fi
  \def\childdocjob{#2}
  \input{#2}
  \endinput
}
%    \end{macrocode}

% \macro{\childdocforward}
% The command |\childdocforward| redirects
% compilation to the main file or
% (if the optional argument is given) a child file.
% Parameters are set as if the main file
% or a child file starting with |\childdocof| was compiled.
% Then compilation is handed over to the main file:
%    \begin{macrocode}
\newcommand{\childdocforward}[2][]
{
  \begingroup
    \if?#1?
      \def\childdoctmp
      {
        \def\childdocname{#2}
        \def\childdocjob{#2}
        \def\jobname{#2}
        \input{#2}
        \endinput
      }
    \else
      \def\childdoctmp
      {
        \childdocdisable
        \def\childdocname{#2}
        \childdoctrue
        \includeonly{#2}
        \def\childdocjob{#1}
        \def\jobname{#1}
        \input{#1}
        \endinput
      }
    \fi
    \expandafter
  \endgroup
  \childdoctmp
}
%    \end{macrocode}

% \macro{\childdocforwardprefix}
% The command |\childdocforwardprefix| redirects
% compilation to the main or a child file by means of a pattern.
% The prefix |#1| in the current filename is replaced by |#2|
% and the suffix of the current filename is kept
% (it is assumed that the filename does not contain the substring `|~~~|'
% which is used as a delimiter).
% Compilation is handed over to the new file by |\childdocforward|:
%    \begin{macrocode}
\newcommand{\childdocforwardprefix}[3][]
{
  \begingroup
    \def\childdocextract #2##1~~~{\def\childdoctmp{\childdocforward[#1]{#3##1}}}
    \expandafter\childdocextract\childdocname~~~
    \expandafter
  \endgroup
  \childdoctmp
}
%    \end{macrocode}

% \macro{\childdoc}
% The deprecated macro |\childdoc| is a legacy version of |\childdocmain|:
%    \begin{macrocode}
\newcommand{\childdoc}{\childdocmain}
%    \end{macrocode}

% \macro{\childdocredirect}
% The deprecated macro |\childdocredirect| is a legacy version
% of |\childdocforward| and |\childdocforwardprefix|:
%    \begin{macrocode}
\newcommand{\childdocredirect}[2][]
{
  \begingroup
    \if?#1?
      \def\childdoctmp{\childdocforward{#2}}
    \else
      \def\childdoctmp{\childdocforwardprefix{#1}{#2}}
    \fi
    \expandafter
  \endgroup
  \childdoctmp
}
%    \end{macrocode}

%\iffalse
%</package>
%\fi
%
\endinput

\childdocmain{}
%    \end{macrocode}

% Optional override for |\version| flag:
%    \begin{macrocode}
%%\ifchilddoc\else\providecommand{\version}{draft}\fi
%    \end{macrocode}

% Define the default values for the |\version| flag
% (|final| for the main file and |draft| for childs):
%    \begin{macrocode}
\ifchilddoc
\providecommand{\version}{draft}
\else
\providecommand{\version}{final}
\fi
%    \end{macrocode}

% Load the standard document class:
%    \begin{macrocode}
\documentclass[12pt]{article}
%    \end{macrocode}

% Start the document body:
%    \begin{macrocode}
\begin{document}
%    \end{macrocode}

% Declare a title page.
% Print title, part of document being processed and version flag:
%    \begin{macrocode}
\addtocounter{page}{-1}
\begin{center}
{\LARGE\bfseries{}childdoc example\par}
\vspace{1cm}
\ifchilddoc
\ifchilddocmanual part\else chapter\fi:
`\childdocname' of `\childdocjob'\par
\else
main document: `\childdocjob'\par
\fi
version: \version\par
\end{center}
\newpage
%    \end{macrocode}

% Manually include selected file,
% otherwise process as usual:
%    \begin{macrocode}
\ifchilddocmanual
\section*{part `\childdocname'}
\input{\childdocname}
\else
%    \end{macrocode}

% Include the two chapters:
%    \begin{macrocode}
\include{cdocsch1}
\include{cdocsch2}
%    \end{macrocode}

% Include the two parts unless only chapters should be displayed:
%    \begin{macrocode}
\ifchilddoc\else
\section{part three}
\input{cdocspt3}
\section{part four}
\input{cdocspt4}
\fi
%    \end{macrocode}

% Process as usual until here:
%    \begin{macrocode}
\fi
%    \end{macrocode}

% End of document body:
%    \begin{macrocode}
\end{document}
%    \end{macrocode}
%\iffalse
%</samplemain>
%\fi
%
% %%%%%%%%%%%%%%%%%%%%%%%%%%%%%%%%%%%%%%
% \paragraph{Chapter Include Files.}
%
% The include files are called |cdocsch1.tex| and |cdocsch2.tex|.
%
%\iffalse
%<*samplechap1|samplechap2>
%\fi

% Optional override for |\version| flag:
%    \begin{macrocode}
%%\providecommand{\version}{final}
%    \end{macrocode}

% Include the main document:
%    \begin{macrocode}
% \iffalse
%
% childdoc.dtx Copyright (C) 2017-2018 Niklas Beisert
%
% This work may be distributed and/or modified under the
% conditions of the LaTeX Project Public License, either version 1.3
% of this license or (at your option) any later version.
% The latest version of this license is in
%   http://www.latex-project.org/lppl.txt
% and version 1.3 or later is part of all distributions of LaTeX
% version 2005/12/01 or later.
%
% This work has the LPPL maintenance status `maintained'.
%
% The Current Maintainer of this work is Niklas Beisert.
%
% This work consists of the files childdoc.dtx and childdoc.ins
% and the derived files childdoc.def and cdocsamp.tex with
% cdocsch1.tex, cdocsch2.tex, cdocsdrf.tex, cdocsfn1.tex, cdocsfn2.tex.
%
%<package>\ifdefined\childdocmain\endinput\fi
%<package>\ProvidesFile{childdoc.def}[2018/12/30 v2.0 child document driver]
%<samplemain>\ProvidesFile{cdocsamp.tex}[2018/12/30 v2.0 sample for childdoc]
%<*driver>
%\ProvidesFile{childdoc.drv}[2018/12/30 v2.0 childdoc reference manual file]
\PassOptionsToClass{10pt,a4paper}{article}
\documentclass{ltxdoc}

\usepackage[margin=35mm]{geometry}
\usepackage{hyperref}
\usepackage{hyperxmp}
\usepackage[usenames]{color}

\hypersetup{colorlinks=true}
\hypersetup{pdfstartview=FitH}
\hypersetup{pdfpagemode=UseNone}
\hypersetup{pdfsource={}}
\hypersetup{pdflang={en-UK}}
\hypersetup{pdfcopyright={Copyright 2017-2018 Niklas Beisert.
  This work may be distributed and/or modified under the
  conditions of the LaTeX Project Public License, either version 1.3
  of this license or (at your option) any later version.}}
\hypersetup{pdflicenseurl={http://www.latex-project.org/lppl.txt}}
\hypersetup{pdfcontactaddress={ETH Zurich, ITP, HIT K,
  Wolfgang-Pauli-Strasse 27}}
\hypersetup{pdfcontactpostcode={8093}}
\hypersetup{pdfcontactcity={Zurich}}
\hypersetup{pdfcontactcountry={Switzerland}}
\hypersetup{pdfcontactemail={nbeisert@itp.phys.ethz.ch}}
\hypersetup{pdfcontacturl={http://people.phys.ethz.ch/\xmptilde nbeisert/}}

\newcommand{\secref}[1]{\hyperref[#1]{section \ref*{#1}}}

\parskip1ex
\parindent0pt
\let\olditemize\itemize
\def\itemize{\olditemize\parskip0pt}

\begin{document}

\title{The \textsf{childdoc} Package}
\hypersetup{pdftitle={The childdoc Package}}
\author{Niklas Beisert\\[2ex]
  Institut f\"ur Theoretische Physik\\
  Eidgen\"ossische Technische Hochschule Z\"urich\\
  Wolfgang-Pauli-Strasse 27, 8093 Z\"urich, Switzerland\\[1ex]
  \href{mailto:nbeisert@itp.phys.ethz.ch}
  {\texttt{nbeisert@itp.phys.ethz.ch}}}
\hypersetup{pdfauthor={Niklas Beisert}}
\hypersetup{pdfsubject={Manual for the LaTeX2e Package childdoc}}
\date{30 December 2018, \textsf{v2.0}}
\maketitle

\begin{abstract}\noindent
\textsf{childdoc} is a \LaTeXe{} package
that enables the direct compilation
of document sections included by |\include|
to individual files.
\end{abstract}

\begingroup
\parskip0ex
\tableofcontents
\endgroup

%%%%%%%%%%%%%%%%%%%%%%%%%%%%%%%%%%%%%%%%%%%%%%%%%%%%%%%%%%%%%%%%%%%%%%%%%%%%%%%%
%%%%%%%%%%%%%%%%%%%%%%%%%%%%%%%%%%%%%%%%%%%%%%%%%%%%%%%%%%%%%%%%%%%%%%%%%%%%%%%%
\section{Introduction}

\LaTeX{} provides a mechanism to structure a large document (such as a book)
into a main file and several child files (containing the chapters)
using the |\include| command.
This mechanism is beneficial for documents
which span hundreds of pages in order to
make the source file(s) more manageable.
Moreover, compilation can be restricted to
selected child files by means of the |\includeonly| command.
The latter feature can be used to reduce the compilation time while editing
(this was significantly more useful in the earlier days of \LaTeX{})
or to generate a smaller document which is easier to navigate.
Another application of |\includeonly| is to generate
documents consisting of selected parts of the complete document.

However, there are a few drawbacks of the plain |\include| mechanism:
\begin{itemize}
\item
The child files cannot be compiled on their own,
they can only be compiled via the main file.
A naive editing environment
(such as a text editor with an option
to have the current file processed by \LaTeX)
may require one to switch to the main file before compiling;
attempting to compile the child file produces errors.
\item
The main file must be modified (each time)
to adjust the |\includeonly| command
to the present needs. This easily leaves the main file in a messy state.
\item
The generated document will always carry the filename
of the main document. This is inconvenient if
several child files are to be compiled and
to be kept for distribution.
\end{itemize}

The present package provides a simple interface
to make child files individually compilable by \LaTeX{}.
Compiling a child file then has the same effect as compiling
the main file with an |\includeonly| command
to select the appropriate child.
Moreover the generated document will carry the name of the child
rather than the main file.
This resolves all three above issues.

This feature is meant to make the editing of books,
thesis documents and lecture notes somewhat more convenient.
However, the package can also be used efficiently for
composing a series of documents (such as exercise sheets)
which are typically distributed individually.
It then assists the author in generating the individual documents
(potentially in different versions)
as well as a document containing the collected series.
Another application is in developing style files
or other kinds of included material
where compilation of the style file could redirect
to a sample or test file.

%%%%%%%%%%%%%%%%%%%%%%%%%%%%%%%%%%%%%%%%%%%%%%%%%%%%%%%%%%%%%%%%%%%%%%%%%%%%%%%%
%%%%%%%%%%%%%%%%%%%%%%%%%%%%%%%%%%%%%%%%%%%%%%%%%%%%%%%%%%%%%%%%%%%%%%%%%%%%%%%%
\section{Usage}

First of all, the package \textsf{childdoc} is \emph{not} a standard
\LaTeXe{} |.sty| style file! Therefore it needs to be invoked in
a non-standard way.

%%%%%%%%%%%%%%%%%%%%%%%%%%%%%%%%%%%%%%%%%%%%%%%%%%%%%%%%%%%%%%%%%%%%%%%%%%%%%%%%
\subsection{Included Files}
\label{sec:include}

%%%%%%%%%%%%%%%%%%%%%%%%%%%%%%%%%%%%%%%%
\DescribeMacro{\childdocmain}
To use the package, add the commands
\begin{center}
\begin{tabular}{l}
|\input{childdoc.def}|\\
|\childdocmain{}|\\
\end{tabular}
\end{center}
at the very top of the main \LaTeX{} file,
in particular \emph{before} the |\documentclass| statement!
The argument of |\childdocmain| should be left empty
(but it must be present).

%%%%%%%%%%%%%%%%%%%%%%%%%%%%%%%%%%%%%%%%
\DescribeMacro{\childdocof}
Furthermore, add the commands
\begin{center}
\begin{tabular}{l}
|\input{childdoc.def}|\\
|\childdocof{|\textit{main}|}|\\
\end{tabular}
\end{center}
at the top of every child file \textit{child}
which is included by |\include{|\textit{child}|}|
from within the main file
(or at least for those files to be compiled individually).
The argument \textit{main} must be the filename of the main file.

There are a couple of
considerations in setting up the main and child documents:

%%%%%%%%%%%%%%%%%%%%%%%%%%%%%%%%%%%%%%%%
\paragraph{Restrictions.}

Please note the following restrictions:
\begin{itemize}
\item
|\childdocmain| must be called with one argument \textit{main}
to ensure compatibility with earlier version of the package.
It must either be empty (|\childdocmain{}|)
or precisely match the filename of the main file in which it is specified.
See \secref{sec:detection} for further information.
\item
The filename \textit{main} must be specified without the |.tex| extension.
\item
The filename \textit{main} is case sensitive
(even in case-insensitive file systems)
due to internal string comparison.
\item
The argument \textit{main} should be fully expanded, it cannot be a macro.
\item
Subdirectories and special characters should be avoided in filenames.
\item
The command |\childdocmain{|\textit{main}|}| must be followed by a whitespace.
It should not be followed immediately by another command
or by a comment mark `|%|'.
This is because the \TeX{} parser reads the token immediately following
the argument of |\childdocmain| and puts it
at the beginning of every child section;
however, a white\-space is ignored.
\end{itemize}

%%%%%%%%%%%%%%%%%%%%%%%%%%%%%%%%%%%%%%%%
\paragraph{Content of Main File.}

It is advisable to place all content in the child files included by |\include|.
Any output contained in the main file will appear in all child documents
unless suppressed manually;
it cannot be suppressed automatically by the |\includeonly| directive
and thus should normally be avoided.
A method to include some content in the main file
by means of conditional processing is described in \secref{sec:conditional}.

%%%%%%%%%%%%%%%%%%%%%%%%%%%%%%%%%%%%%%%%
\paragraph{Page Numbering.}

When only a part of the document is compiled,
the appropriate numbering of pages
(as well as other status parameters)
is determined from the |.aux| files.
The latter contain information from previous passes.
However this information needs to propagate through
all intermediate child documents.
Therefore the page numbering in child documents may well
be inconsistent until the complete document is compiled at least once.

A useful (if unconventional) way to always ensure a consistent
page numbering is to restart the numbering in each child document
and denote the pages by `\textit{child}|.|\textit{page}'
where \textit{child} represents the chapter/section number of the child file.
This can be achieved by the command
|\numberwithin{page}{|\textit{child}|}|
of the \textsf{amsmath} package
where \textit{child} can be |chapter| or |section|
depending on the chosen structuring.
Alternatively, one can modify the macro |\thepage| appropriately
and reset the counter |page| at the start of each child file.

%%%%%%%%%%%%%%%%%%%%%%%%%%%%%%%%%%%%%%%%%%%%%%%%%%%%%%%%%%%%%%%%%%%%%%%%%%%%%%%%
\subsection{Conditional Processing}
\label{sec:conditional}

The package provides a mechanism to compile different versions
of a document. To customise the versions further some conditional processing
can come in handy to distinguish which version is being compiled.
The package provides two macros to describe the compilation context:

%%%%%%%%%%%%%%%%%%%%%%%%%%%%%%%%%%%%%%%%
\DescribeMacro{\ifchilddoc}
The conditional |\ifchilddoc| distinguishes between the compilation of
child documents and the main document:
%
\begin{center}
|\ifchilddoc |\textit{child-code}| |[|\||else |\textit{main-code}]| \||fi|
\end{center}

%%%%%%%%%%%%%%%%%%%%%%%%%%%%%%%%%%%%%%%%
\DescribeMacro{\childdocname}
\DescribeMacro{\childdocjob}
The macro |\childdocname| contains the filename (without extension)
of the main or child file being processed.
Note that |\childdocjob| will always contain the name of the main file.

%%%%%%%%%%%%%%%%%%%%%%%%%%%%%%%%%%%%%%%%
\paragraph{Title Page.}

Conditional processing can be used to include a title or banner page
in the main document when proper precautions are taken.
Importantly, the code in the main file should ensure that the page counter
(as well as other status parameters which are stored in the |.aux| files)
takes the same value after the conditional processing.
Otherwise the page numbers may take divergent values
depending on which part is compiled.

For example, a title page could be declared by:
%
\begin{center}
\begin{tabular}{l}
|\ifchilddoc\||else|\\
|\addtocounter{page}{-1}|\\
\textit{code for title page}\\
|\newpage|\\
|\||fi|
\end{tabular}
\end{center}
%
A banner page for the child documents can be generated by:
%
\begin{center}
\begin{tabular}{l}
|\ifchilddoc|\\
|\addtocounter{page}{-1}|\\
\textit{code for banner page}\\
|\newpage|\\
|\||fi|
\end{tabular}
\end{center}
%
Here one could write a message such as:
\begin{center}
|This is the part \childdocname{} of \childdocjob{}.|
\end{center}

%%%%%%%%%%%%%%%%%%%%%%%%%%%%%%%%%%%%%%%%%%%%%%%%%%%%%%%%%%%%%%%%%%%%%%%%%%%%%%%%
\subsection{Flags}
\label{sec:flags}

The package makes it easy to generate different versions
of the main or child documents.
To this end compilation flags can be defined
and assigned different default values.
They will be particularly useful in conjunction
with the forwarding mechanism described in \secref{sec:forward}.

For example, it may be useful to have a flag |\version|
which can be set to |draft| or |final|.
The document source will contain some conditional code
depending on the value of |\version|.
Suppose further, the flag should default to |final| for the main file
and to |draft| for child files
which is a natural assignment for editing the document.
This is achieved by placing the following code
in the preamble of the main document
(below the |\childdocmain| directive):
%
\begin{center}
\begin{tabular}{l}
|\ifchilddoc|\\
|\providecommand{\version}{draft}|\\
|\||else|\\
|\providecommand{\version}{final}|\\
|\||fi|
\end{tabular}
\end{center}
%
The definition by |\providecommand| makes sure
that previous definitions are not overwritten.
Further statements |\providecommand{\version}{...}|
can thus be added before the above code to override it.

For the main file, one might add a line
(between |\childdocmain| and the above block)
%
\begin{center}
|%\ifchilddoc\||else\providecommand{\version}{draft}\||fi|
\end{center}
%
which can be uncommented to produce a draft version.
Likewise one can add a line to the very top of a child file
(above the |\childdocof{|\textit{main}|}| directive)
%
\begin{center}
|%\providecommand{\version}{final}|
\end{center}
%
which can be uncommented to produce the final version of this child document.

%%%%%%%%%%%%%%%%%%%%%%%%%%%%%%%%%%%%%%%%%%%%%%%%%%%%%%%%%%%%%%%%%%%%%%%%%%%%%%%%
\subsection{Forwarding}
\label{sec:forward}

Different versions of the main or child documents
using compilation flags as described in \secref{sec:flags}
can be (permanently) stored in different files
for convenient compilation, viewing and distribution.
To this end, the package defines a command
to pass on compilation to a different file:

%%%%%%%%%%%%%%%%%%%%%%%%%%%%%%%%%%%%%%%%
\DescribeMacro{\childdocforward}
The command |\childdocforward| redirects processing to
another source file:
%
\begin{center}
\begin{tabular}{l}
|\input{childdoc.def}|\\
|\childdocforward[|\textit{main}|]{|\textit{dest}|}|\\
\end{tabular}
\end{center}
%
The argument \textit{dest} is the destination file
(without extension).
It should be the main file or one of the child files.
Note that further \textsf{childdoc} directives
such as |\childdocof| and |\childdocforward|
in the indicated file will be processed in this form.
The optional argument \textit{main}
passes on directly to the main file \textit{main}
while pretending to compile the child \textit{dest}.
This form behaves as if \textit{dest}
issues |\childdocof{|\textit{main}|}| right away,
and no further \textsf{childdoc} directives will be processed.

%%%%%%%%%%%%%%%%%%%%%%%%%%%%%%%%%%%%%%%%
\DescribeMacro{\...prefix}
In the alternative form |\childdocforwardprefix|,
%
\begin{center}
\begin{tabular}{l}
|\input{childdoc.def}|\\
|\childdocforwardprefix[|\textit{main}|]{|\textit{prefix}|}{|\textit{dest}|}|
\end{tabular}
\end{center}
%
the destination file is determined by a pattern
depending on the current file:
To make this work, the current file must be called
`{\textit{prefix}\hspace{0.2em}\textit{suffix}}'
with \textit{prefix} matching precisely the argument.
Processing is then passed on to the file
`{\textit{dest}\hspace{0.2em}\textit{suffix}}'.
Surely, the same effect is achieved by
directly specifying the
argument `{\textit{dest}\hspace{0.2em}\textit{suffix}}'
in the first form.
However, that requires to set up a different file
for each child. With the alternative form of the command
all these files can have exactly the same content
which simplifies setting them up and maintaining them.

For example, the following file |draft.tex|
with a compilation flag |\version| as described in \secref{sec:flags}
compiles the main document as a draft:
%
\begin{center}
\begin{tabular}{l}
|\def\version{draft}|\\
|\input{childdoc.def}|\\
|\childdocforward{|\textit{main}|}|
\end{tabular}
\end{center}
%
Likewise, the following files |final|\textit{nn}|.tex|
compile the final version of the child document
|child|\textit{nn}|.tex|:
%
\begin{center}
\begin{tabular}{l}
|\def\version{final}|\\
|\input{childdoc.def}|\\
|\childdocforwardprefix{final}{child}|
\end{tabular}
\end{center}
%

Note that when several versions of a main file and/or of each child file
are to be generated, it may be convenient to set up a |Makefile| or
shell script to automatise the process.

%%%%%%%%%%%%%%%%%%%%%%%%%%%%%%%%%%%%%%%%%%%%%%%%%%%%%%%%%%%%%%%%%%%%%%%%%%%%%%%%
\subsection{Command Line Processing}
\label{sec:commandline}

The effect of redirection files can also be achieved by invoking
the \LaTeX{} compiler with a more elaborate command line.
Most conveniently this should be done as part
of a shell script or a |Makefile|.

When using \textsf{childdoc} in the main file, the following
command lines effectively perform a redirection
(note that depending on the shell being used,
backslashes may have to be doubled: `|\|' $\to$ `|\\|'):
%
\begin{center}
|... -jobname "|\textit{target}|" |\\|"|[\textit{flags}]%
|\input{childdoc.def}\childdocforward[|\textit{main}|]{|\textit{dest}|}"|
\end{center}
%
Here \textit{target} is the name of the output file,
\textit{main} is the name of the main file
and \textit{dest} is the name of the main or child file to be processed
(all filenames without extensions).
The optional argument \textit{main} can be omitted
if \textit{main} matches \textit{dest}.
Optionally, compilation \textit{flags} can be defined via |\def| commands.
This command line makes the \TeX{} engine believe
it is compiling the file \textit{target}
whose content is specified as the latter parameter.
The provided code then forwards the processing to
\textit{main} or \textit{dest} as described in \secref{sec:forward}.

%%%%%%%%%%%%%%%%%%%%%%%%%%%%%%%%%%%%%%%%%%%%%%%%%%%%%%%%%%%%%%%%%%%%%%%%%%%%%%%%
\subsection{Include by Input}
\label{sec:input}

Including child documents by |\include| has some restrictions by design.
Most notably, the content of a child document always occupies
its own set of pages; pages cannot be shared between child documents.
Usually, this behaviour makes perfect sense
because each child document contain an essential part of the document.
However, in some situations it may be desirable to compose
a document from a collection of parts
without having mandatory page breaks between then.
For this case, the package
provides a mechanism to include parts
by |\input| which can also be processed individually.
However, by construction this mechanism
requires manual handling of the content to be output.

%%%%%%%%%%%%%%%%%%%%%%%%%%%%%%%%%%%%%%%%
\DescribeMacro{\ifchilddocmanual}
The main file should be prepared as usual, see \secref{sec:include}.
However, the document body must make a distinction
between processing of an individual part and of the main document, e.g.:
%
\begin{center}
\begin{tabular}{l}
|\ifchilddocmanual|\\
|\input{\childdocname}|\\
|\||else|\\
\textit{document body with }|\input{|\textit{part}|}|\\
|\||fi|
\end{tabular}
\end{center}
%
The conditional |\ifchilddocmanual| is true whenever
a part to be included by |\input| is being compiled,
and the name of the part is stored in |\childdocname|.

%%%%%%%%%%%%%%%%%%%%%%%%%%%%%%%%%%%%%%%%
\DescribeMacro{\childdocby}
Each part to be included by |\input| should start with:
%
\begin{center}
\begin{tabular}{l}
|\input{childdoc.def}|\\
|\childdocby{|\textit{main}|}|\\
\end{tabular}
\end{center}
%
The directive |\childdocby| is similar to |\childdocof|
described in \secref{sec:include},
but the subsequent selection of content must be done manually.
To that end, both |\ifchilddoc| and |\ifchilddocmanual|
will be true upon processing of a part,
and the name of the part is stored in |\childdocname|.
Note that |\jobname| will be set to the filename of the current part
so that each part receives an individual |.aux| file
that does not interfere with the |.aux| file(s) of the main document.
This behaviour can be altered by the alternative form
|\childdocby[*]{|\textit{main}|}| (with a non-empty optional argument)
which uses the |.aux| file of the main document
by setting |\jobname| to \textit{main}.

%%%%%%%%%%%%%%%%%%%%%%%%%%%%%%%%%%%%%%%%%%%%%%%%%%%%%%%%%%%%%%%%%%%%%%%%%%%%%%%%
\subsection{Driver Development}
\label{sec:driver}

The \textsf{childdoc} mechanism can also be use for the development
of definition files such as \LaTeX{} styles or classes.
This case differs from the above setup with multiple parts
included by |\include| in that no |\includeonly| should be invoked.
This can be achieved by starting the include file
(before |\ProvidesPackage|) with:
%
\begin{center}
\begin{tabular}{l}
|\input{childdoc.def}|\\
|\childdocforward{|\textit{main}|}|\\
\end{tabular}
\end{center}
%
or alternatively with:
%
\begin{center}
\begin{tabular}{l}
|\input{childdoc.def}|\\
|\childdocby{|\textit{main}|}|\\
\end{tabular}
\end{center}
%
Both forms have slightly different effects as described above.
The main file is prepared as usual, see \secref{sec:include}.

%%%%%%%%%%%%%%%%%%%%%%%%%%%%%%%%%%%%%%%%%%%%%%%%%%%%%%%%%%%%%%%%%%%%%%%%%%%%%%%%
\subsection{Legacy Detection}
\label{sec:detection}

The directive |\childdocmain| in the main file can detect
whether the complete document or merely a child is to be compiled
even without using the directive |\childdocof|.
This method is deprecated because it is less robust
and there is no compelling reason to use it;
it is merely provided for backward compatibility
and it may be removed in future versions.

If the detection mechanism is to be used,
it is mandatory to correctly specify
the filename of the main file as the argument of |\childdocmain|:
%
\begin{center}
\begin{tabular}{l}
|\input{childdoc.def}|\\
|\childdocmain{|\textit{main}|}|\\
\end{tabular}
\end{center}
%
If |\jobname| does not match the argument \textit{main} of |\childdocmain|,
it is assumed that |\jobname| points to the child file to be compiled.
When using |\childdocmain| with the main file specified as argument,
it suffices to start a child file
with just |\input{|\textit{main}|}|
without loading of the package and using |\childdocof|.
If instead all processing is done
with the appropriate \textsf{childdoc} directives,
the argument of \textit{main} of |\childdocmain| can be empty.

An alternative version of the command line processing described
in \secref{sec:commandline} using the detection mechanism reads:
%
\begin{center}
|... -jobname "|\textit{target}|" "|[\textit{flags}]%
[|\def\jobname{|\textit{dest}|}|]|\input{|\textit{main}|}"|
\end{center}

%%%%%%%%%%%%%%%%%%%%%%%%%%%%%%%%%%%%%%%%%%%%%%%%%%%%%%%%%%%%%%%%%%%%%%%%%%%%%%%%
\subsection{Manual Code}
\label{sec:manual}

In case one cannot be certain whether the definitions file |childdoc.def|
is installed on the target \TeX{} distribution
and one prefers not to ship it,
it is conceivable to paste a few relevant commands into the sources.

To that end, drop all statements |\input{childdoc.def}|
and perform the replacements as outlined below.
Instead of |\childdocmain{|\textit{main}|}| add the following code
to the top of the main file:
%
\begin{center}
\begin{tabular}{l}
|\||ifdefined\childdocname\endinput\||fi\newif\ifchilddoc|\\
|\edef\childdocname{\scantokens\expandafter{\jobname\noexpand}}|\\
|\def\childdocmain{|\textit{main}|}\||ifx\childdocmain\childdocname\||else|\\
|\childdoctrue\includeonly{\childdocname}\let\jobname\childdocmain\||fi|\\
\end{tabular}
\end{center}
%
Instead of |\childdocof{|\textit{main}|}| just include the main file
at the top of each child file:
%
\begin{center}
|\input{|\textit{main}|}|
\end{center}
%
A simple redirection |\childdocforward{|\textit{dest}|}| is achieved by:
%
\begin{center}
|\def\jobname{|\textit{dest}|}\input{\jobname}|
\end{center}
%
The redirection with prefix
|\childdocforwardprefix[|\textit{prefix}|]{|\textit{dest}|}|
is accomplished by:
%
\begin{center}
\begin{tabular}{l}
|{\edef\jobname{\scantokens\expandafter{\jobname\noexpand}}|\\
|\def\redirectjob |\textit{prefix}|#1~~~{\gdef\jobname{|\textit{dest}|#1}}|\\
|\expandafter\redirectjob\jobname~~~}\input{\jobname}|
\end{tabular}
\end{center}

In an alternative approach,
child documents can be compiled by a specific command line
without additional code or specific definitions:
%
\begin{center}
|... -jobname "|\textit{target}|" "|[\textit{flags}]%
|\includeonly{|\textit{dest}|}\input{|\textit{main}|}"|
\end{center}
%

%%%%%%%%%%%%%%%%%%%%%%%%%%%%%%%%%%%%%%%%%%%%%%%%%%%%%%%%%%%%%%%%%%%%%%%%%%%%%%%%
%%%%%%%%%%%%%%%%%%%%%%%%%%%%%%%%%%%%%%%%%%%%%%%%%%%%%%%%%%%%%%%%%%%%%%%%%%%%%%%%
\section{Information}

%%%%%%%%%%%%%%%%%%%%%%%%%%%%%%%%%%%%%%%%%%%%%%%%%%%%%%%%%%%%%%%%%%%%%%%%%%%%%%%%
\subsection{Copyright}

Copyright \copyright{} 2017--2018 Niklas Beisert

This work may be distributed and/or modified under the
conditions of the \LaTeX{} Project Public License, either version 1.3
of this license or (at your option) any later version.
The latest version of this license is in
  \url{http://www.latex-project.org/lppl.txt}
and version 1.3 or later is part of all distributions of \LaTeX{}
version 2005/12/01 or later.

This work has the LPPL maintenance status `maintained'.

The Current Maintainer of this work is Niklas Beisert.

This work consists of the files |README.txt|, |childdoc.ins| and |childdoc.dtx|
as well as the derived files |childdoc.def|, |cdocsamp.tex|
with |cdocsch1.tex|, |cdocsch2.tex|, |cdocspt3.tex|, |cdocspt4.tex|,
|cdocsdrf.tex|, |cdocsfn1.tex|, |cdocsfn2.tex|
as well as |childdoc.pdf|.

%%%%%%%%%%%%%%%%%%%%%%%%%%%%%%%%%%%%%%%%%%%%%%%%%%%%%%%%%%%%%%%%%%%%%%%%%%%%%%%%
\subsection{Files and Installation}

The package consists of the files:
%
\begin{center}
\begin{tabular}{ll}
    |README.txt|   & readme file \\
    |childdoc.ins| & installation file \\
    |childdoc.dtx| & source file \\
    |childdoc.def| & definition file \\
    |cdocsamp.tex| & sample main file \\
    |cdocsch1.tex| & sample include file \\
    |cdocsch2.tex| & sample include file \\
    |cdocspt3.tex| & sample part file \\
    |cdocspt4.tex| & sample part file \\
    |cdocsdrf.tex| & sample redirection file \\
    |cdocsfn1.tex| & sample redirection file \\
    |cdocsfn2.tex| & sample redirection file \\
    |childdoc.pdf| & manual
\end{tabular}
\end{center}
%
The distribution consists of the files
|README.txt|, |childdoc.ins| and |childdoc.dtx|.
%
\begin{itemize}
\item
Run (pdf)\LaTeX{} on |childdoc.dtx|
to compile the manual |childdoc.pdf| (this file).
\item
Run \LaTeX{} on |childdoc.ins| to create the definitions file |childdoc.def|
and the sample |cdocsamp.tex| with include files
|cdocsch1.tex|, |cdocsch2.tex|, |cdocspt3.tex|, |cdocspt4.tex|,
|cdocsdrf.tex|, |cdocsfn1.tex|, |cdocsfn2.tex|.
Then copy the file |childdoc.def| to an appropriate directory of your \LaTeX{}
distribution, e.g.\ \textit{texmf-root}|/tex/latex/childdoc|.
\end{itemize}

%%%%%%%%%%%%%%%%%%%%%%%%%%%%%%%%%%%%%%%%%%%%%%%%%%%%%%%%%%%%%%%%%%%%%%%%%%%%%%%%
\subsection{Related CTAN Packages}

There are several other packages which offer a similar functionality:
%
\begin{itemize}
\item
The packages
\href{http://ctan.org/pkg/docmute}{\textsf{docmute}},
\href{http://ctan.org/pkg/includex}{\textsf{includex}} and
\href{http://ctan.org/pkg/standalone}{\textsf{standalone}}
provide commands to include only the document body of
a child file thus allowing both files to be compiled individually.
\item
The packages \href{http://ctan.org/pkg/subdocs}{\textsf{subdocs}}
and \href{http://ctan.org/pkg/subfiles}{\textsf{subfiles}}
provide structures in which the main and child documents can be
encapsulated and allowing them to be compiled individually.
The inclusion mechanism is different from the conventional |\include|.
\item
The package \href{http://ctan.org/pkg/combine}{\textsf{combine}}
is an elaborate solution to combine several documents into one.
\end{itemize}
%
See also the CTAN topic \href{http://ctan.org/topic/subdocs}{\textsf{subdocs}}
for further related packages.
The present package differs from the above solutions in that
a document structure constructed with the conventional |\include| mechanism
just needs two extra commands at the top of every file
such that all constituent files can be compiled individually.

%%%%%%%%%%%%%%%%%%%%%%%%%%%%%%%%%%%%%%%%%%%%%%%%%%%%%%%%%%%%%%%%%%%%%%%%%%%%%%%%
%\subsection{Feature Suggestions}
%
%The following is a list of features which may be useful for future
%versions of this package:
%%
%\begin{itemize}
%\item
%\ldots
%\end{itemize}

%%%%%%%%%%%%%%%%%%%%%%%%%%%%%%%%%%%%%%%%%%%%%%%%%%%%%%%%%%%%%%%%%%%%%%%%%%%%%%%%
\subsection{Revision History}

%%%%%%%%%%%%%%%%%%%%%%%%%%%%%%%%%%%%%%%%
\paragraph{v2.0:} 2018/12/30

\begin{itemize}
\item
immediate forward processing
\item
added |\childdocby| mechanism
\item
manual restructured
\end{itemize}

%%%%%%%%%%%%%%%%%%%%%%%%%%%%%%%%%%%%%%%%
\paragraph{v1.6:} 2018/01/17

\begin{itemize}
\item
application for development of include files
\item
corrections to manual
\end{itemize}

%%%%%%%%%%%%%%%%%%%%%%%%%%%%%%%%%%%%%%%%
\paragraph{v1.5:} 2017/05/21

\begin{itemize}
\item
more complete structuring introduced
\item
|\childdocof| introduced
\item
|\childdoc| renamed to |\childdocmain|
\item
|\childredirect| renamed to |\childdocforward| and |\childdocforwardprefix|
and functionality expanded
\end{itemize}

%%%%%%%%%%%%%%%%%%%%%%%%%%%%%%%%%%%%%%%%
\paragraph{v1.0:} 2017/04/27

\begin{itemize}
\item
manual and install package
\item
first version published on CTAN
\end{itemize}

%%%%%%%%%%%%%%%%%%%%%%%%%%%%%%%%%%%%%%%%
\paragraph{v0.6:} 2017/04/26

\begin{itemize}
\item
redirection mechanism added
\end{itemize}

%%%%%%%%%%%%%%%%%%%%%%%%%%%%%%%%%%%%%%%%
\paragraph{v0.5:} 2017/04/26

\begin{itemize}
\item
functionality in definition file
\end{itemize}


%%%%%%%%%%%%%%%%%%%%%%%%%%%%%%%%%%%%%%%%%%%%%%%%%%%%%%%%%%%%%%%%%%%%%%%%%%%%%%%%
%%%%%%%%%%%%%%%%%%%%%%%%%%%%%%%%%%%%%%%%%%%%%%%%%%%%%%%%%%%%%%%%%%%%%%%%%%%%%%%%
%%%%%%%%%%%%%%%%%%%%%%%%%%%%%%%%%%%%%%%%%%%%%%%%%%%%%%%%%%%%%%%%%%%%%%%%%%%%%%%%
\appendix

\settowidth\MacroIndent{\rmfamily\scriptsize 000\ }

 \DocInput{childdoc.dtx}

\end{document}
%</driver>
% \fi
%
% %%%%%%%%%%%%%%%%%%%%%%%%%%%%%%%%%%%%%%%%%%%%%%%%%%%%%%%%%%%%%%%%%%%%%%%%%%%%%%
% %%%%%%%%%%%%%%%%%%%%%%%%%%%%%%%%%%%%%%%%%%%%%%%%%%%%%%%%%%%%%%%%%%%%%%%%%%%%%%
% \section{Sample}
%\iffalse
%<*samplemain>
%\fi
%
% The following presents a sample document
% with two chapters, two parts, a title page,
% a compile flag as well as three forwarding files to set the flag.
% It consists of eight |.tex| files:
% \begin{center}
% \begin{tabular}{ll}
% |cdocsamp.tex|&main file\\
% |cdocsch1.tex|&include file for chapter 1\\
% |cdocsch2.tex|&include file for chapter 2\\
% |cdocspt3.tex|&include file for part 3\\
% |cdocspt4.tex|&include file for part 4\\
% |cdocsdrf.tex|&forwarding file for main file in draft mode\\
% |cdocsfi1.tex|&forwarding file for final version of chapter 1\\
% |cdocsfi2.tex|&forwarding file for final version of chapter 2\\
% \end{tabular}
% \end{center}
% Each of the eight files can be compiled directly by the \LaTeX{} compiler.
%
% %%%%%%%%%%%%%%%%%%%%%%%%%%%%%%%%%%%%%%
% \paragraph{Main File.}
%
% The main file is called |cdocsamp.tex|.
%
% Load the \textsf{childdoc} definitions and
% declare the filename for the main document:
%    \begin{macrocode}
\input{childdoc.def}
\childdocmain{}
%    \end{macrocode}

% Optional override for |\version| flag:
%    \begin{macrocode}
%%\ifchilddoc\else\providecommand{\version}{draft}\fi
%    \end{macrocode}

% Define the default values for the |\version| flag
% (|final| for the main file and |draft| for childs):
%    \begin{macrocode}
\ifchilddoc
\providecommand{\version}{draft}
\else
\providecommand{\version}{final}
\fi
%    \end{macrocode}

% Load the standard document class:
%    \begin{macrocode}
\documentclass[12pt]{article}
%    \end{macrocode}

% Start the document body:
%    \begin{macrocode}
\begin{document}
%    \end{macrocode}

% Declare a title page.
% Print title, part of document being processed and version flag:
%    \begin{macrocode}
\addtocounter{page}{-1}
\begin{center}
{\LARGE\bfseries{}childdoc example\par}
\vspace{1cm}
\ifchilddoc
\ifchilddocmanual part\else chapter\fi:
`\childdocname' of `\childdocjob'\par
\else
main document: `\childdocjob'\par
\fi
version: \version\par
\end{center}
\newpage
%    \end{macrocode}

% Manually include selected file,
% otherwise process as usual:
%    \begin{macrocode}
\ifchilddocmanual
\section*{part `\childdocname'}
\input{\childdocname}
\else
%    \end{macrocode}

% Include the two chapters:
%    \begin{macrocode}
\include{cdocsch1}
\include{cdocsch2}
%    \end{macrocode}

% Include the two parts unless only chapters should be displayed:
%    \begin{macrocode}
\ifchilddoc\else
\section{part three}
\input{cdocspt3}
\section{part four}
\input{cdocspt4}
\fi
%    \end{macrocode}

% Process as usual until here:
%    \begin{macrocode}
\fi
%    \end{macrocode}

% End of document body:
%    \begin{macrocode}
\end{document}
%    \end{macrocode}
%\iffalse
%</samplemain>
%\fi
%
% %%%%%%%%%%%%%%%%%%%%%%%%%%%%%%%%%%%%%%
% \paragraph{Chapter Include Files.}
%
% The include files are called |cdocsch1.tex| and |cdocsch2.tex|.
%
%\iffalse
%<*samplechap1|samplechap2>
%\fi

% Optional override for |\version| flag:
%    \begin{macrocode}
%%\providecommand{\version}{final}
%    \end{macrocode}

% Include the main document:
%    \begin{macrocode}
\input{childdoc.def}
\childdocof{cdocsamp}
%    \end{macrocode}

%\iffalse
%</samplechap1|samplechap2>
%\fi
%
%\iffalse
%<*samplechap1>
%\fi
% Some text for chapter 1:
%    \begin{macrocode}
\section{one}
some text in chapter one
%    \end{macrocode}

%\iffalse
%</samplechap1>
%\fi
% Some text for chapter 2:
%\iffalse
%<*samplechap2>
%\fi
%    \begin{macrocode}
\section{two}
more text in chapter two
%    \end{macrocode}

%\iffalse
%</samplechap2>
%\fi
%
% %%%%%%%%%%%%%%%%%%%%%%%%%%%%%%%%%%%%%%
% \paragraph{Part Include Files.}
%
% The include files are called |cdocspt3.tex| and |cdocspt4.tex|.
%
%\iffalse
%<*samplepart3|samplepart4>
%\fi

% Optional override for |\version| flag:
%    \begin{macrocode}
%%\providecommand{\version}{final}
%    \end{macrocode}

% Include the main document:
%    \begin{macrocode}
\input{childdoc.def}
\childdocby{cdocsamp}
%    \end{macrocode}

%\iffalse
%</samplepart3|samplepart4>
%\fi
%
%\iffalse
%<*samplepart3>
%\fi
% Some text for part 3:
%    \begin{macrocode}
some text in part three
%    \end{macrocode}

%\iffalse
%</samplepart3>
%\fi
% Some text for part 4:
%\iffalse
%<*samplepart4>
%\fi
%    \begin{macrocode}
more text in part four
%    \end{macrocode}

%\iffalse
%</samplepart4>
%\fi
%
% %%%%%%%%%%%%%%%%%%%%%%%%%%%%%%%%%%%%%%
% \paragraph{Forwarding for a Complete Draft.}
%
% The following forwarding file |cdocsdrf.tex|
% compiles the main document in draft mode:
%\iffalse
%<*sampledraft>
%\fi
%    \begin{macrocode}
\def\version{draft}
\input{childdoc.def}
\childdocforward{cdocsamp}
%    \end{macrocode}

%\iffalse
%</sampledraft>
%\fi
%
% %%%%%%%%%%%%%%%%%%%%%%%%%%%%%%%%%%%%%%
% \paragraph{Forwarding for Final Version of the Chapters.}
%
% The following forwarding files |cdocsfn1.tex| and |cdocsfn2.tex|
% (with identical content)
% compile the final versions of the child documents
% |cdocsch1.tex| and |cdocsch2.tex|, respectively:
%\iffalse
%<*samplefinal>
%\fi
%    \begin{macrocode}
\def\version{final}
\input{childdoc.def}
\childdocforwardprefix[cdocsamp]{cdocsfn}{cdocsch}
%    \end{macrocode}

%\iffalse
%</samplefinal>
%\fi
%
% %%%%%%%%%%%%%%%%%%%%%%%%%%%%%%%%%%%%%%
% \paragraph{Command Line Processing.}
%
% The following three command lines generate the output files
% |cdocscld|, |cdocscl1| and |cdocscl2|
% which should be identical to
% |cdocsdrf|, |cdocsch1| and |cdocsfn2|, respectively:
% \begin{center}
% \begin{tabular}{l}
% |latex -jobname cdocscld \|\\
% |  "\def\version{draft}\input{childdoc.def}\childdocforward{cdocsamp}"|\\
% |latex -jobname cdocscl1 \|\\
% |  "\input{childdoc.def}\childdocforward[cdocsamp]{cdocsch1}"|\\
% |latex -jobname cdocscl2 \|\\
% |  "\def\version{final}\input{childdoc.def}\childdocforward{cdocsch2}"|
% \end{tabular}
% \end{center}
% Note that the trailing backslash on each first line
% merely continues the input to the second line
% (for convenient cut ant paste).
% Furthermore, the command |latex| can be replaced by any
% of its alternative versions such as |pdflatex|.
%
% %%%%%%%%%%%%%%%%%%%%%%%%%%%%%%%%%%%%%%%%%%%%%%%%%%%%%%%%%%%%%%%%%%%%%%%%%%%%%%
% %%%%%%%%%%%%%%%%%%%%%%%%%%%%%%%%%%%%%%%%%%%%%%%%%%%%%%%%%%%%%%%%%%%%%%%%%%%%%%
% \section{Implementation}
%\iffalse
%<*package>
%\fi
%
% This section describes the definitions file |childdoc.def|.

% The definitions cannot be loaded using |\usepackage| or |\RequirePackage|
% which has a mechanism to prevent loading a style file more than once.
% When loading the definitions by means of |\input|
% multiple instances have to be prevented manually:
%\iffalse
%This code needs to be before the `\ProvidesFile' directive
%which is defined at the beginning of this file.
%Therefore it is also placed there and commented out here.
%</package>
%<*discard>
%\fi
%    \begin{macrocode}
\ifdefined\childdocmain\endinput\fi
%    \end{macrocode}
%\iffalse
%</discard>
%<*package>
%\fi
%
% \macro{\ifchilddoc}
% \macro{\ifchilddocmanual}
% The conditional |\ifchilddoc| tells whether a
% child (true) or main (false) document is being compiled.
% The conditional |\ifchilddocmanual| tells whether
% the |\includeonly| mechanism is used (false) or
% the selection of child files must be performed manually (true).
% The definitions initialise to false:
%    \begin{macrocode}
\newif\ifchilddoc
\newif\ifchilddocmanual
%    \end{macrocode}

% \macro{\childdocname}
% \macro{\childdocjob}
% The macro |\childdocname| stores the name of the main document
% to be compiled. The macro |\childdocjob| stores the name of
% the document on which the \LaTeX{} compiler was originally invoked.
% The content of |\jobname| cannot be compared
% to filenames specified in the source due to different catcodes.
% The following code rescans |\jobname|, stores the result
% in |\childdocname| and saves a copy in |\childdocjob|:
%    \begin{macrocode}
\edef\childdocname{\scantokens\expandafter{\jobname\noexpand}}
\let\childdocjob\childdocname
%    \end{macrocode}

% \macro{\childdocdisable}
% The macro |\childdocdisable| prevents the main file
% from being processed more than once.
% At this stage, the main document command |\childdocmain|
% is assumed to be called once again where it should do nothing.
% Any subsequent call to it should prevent
% a secondary processing of the main document
% It overwrites the forwarding commands
% |\childdocof| and |\childdocforward|
% with empty macros to prevent further inclusions of the main document:
%    \begin{macrocode}
\newcommand{\childdocdisable}
{
  \renewcommand{\childdocmain}[1]{\renewcommand{\childdocmain}[1]{\endinput}}
  \renewcommand{\childdocof}[1]{}
  \renewcommand{\childdocby}[2][]{}
  \renewcommand{\childdocforward}[2][]{}
  \renewcommand{\childdocdisable}{}
}
%    \end{macrocode}

% \macro{\childdocmain}
% The macro |\childdocmain| is to be called at the top of the main file
% with nothing or the main filename (without extension) as argument.
% First, it breaks loops.
% If the argument is not empty and does not match |\childdocname|
% (which is set by the first inclusion of |childdoc.def|),
% |\ifchilddoc| is set to true, |\includeonly| is applied to the child file
% and |\jobname| is set to the main file
% (for proper handling of |.aux| files):
%    \begin{macrocode}
\newcommand{\childdocmain}[1]
{
  \childdocdisable\childdocmain{}
  \if?#1?\else
    \begingroup
      \def\childdoctmp{#1}
      \ifx\childdoctmp\childdocname
        \def\childdoctmp{}
      \else
        \def\childdoctmp
        {
          \childdoctrue
          \includeonly{\childdocname}
          \def\childdocjob{#1}
          \def\jobname{#1}
        }
      \fi
      \expandafter
    \endgroup
    \childdoctmp
  \fi
}
%    \end{macrocode}

% \macro{\childdocof}
% The command |\childdocof| redirects
% compilation to the main file |#1|.
%    \begin{macrocode}
\newcommand{\childdocof}[1]
{
  \childdocdisable
  \childdoctrue
  \includeonly{\childdocname}
  \def\jobname{#1}
  \def\childdocjob{#1}
  \input{#1}
}
%    \end{macrocode}

% \macro{\childdocby}
% The command |\childdocby| ....
%    \begin{macrocode}
\newcommand{\childdocby}[2][]
{
  \childdocdisable
  \childdoctrue
  \childdocmanualtrue
  \if?#1?\else
    \def\jobname{#2}
  \fi
  \def\childdocjob{#2}
  \input{#2}
  \endinput
}
%    \end{macrocode}

% \macro{\childdocforward}
% The command |\childdocforward| redirects
% compilation to the main file or
% (if the optional argument is given) a child file.
% Parameters are set as if the main file
% or a child file starting with |\childdocof| was compiled.
% Then compilation is handed over to the main file:
%    \begin{macrocode}
\newcommand{\childdocforward}[2][]
{
  \begingroup
    \if?#1?
      \def\childdoctmp
      {
        \def\childdocname{#2}
        \def\childdocjob{#2}
        \def\jobname{#2}
        \input{#2}
        \endinput
      }
    \else
      \def\childdoctmp
      {
        \childdocdisable
        \def\childdocname{#2}
        \childdoctrue
        \includeonly{#2}
        \def\childdocjob{#1}
        \def\jobname{#1}
        \input{#1}
        \endinput
      }
    \fi
    \expandafter
  \endgroup
  \childdoctmp
}
%    \end{macrocode}

% \macro{\childdocforwardprefix}
% The command |\childdocforwardprefix| redirects
% compilation to the main or a child file by means of a pattern.
% The prefix |#1| in the current filename is replaced by |#2|
% and the suffix of the current filename is kept
% (it is assumed that the filename does not contain the substring `|~~~|'
% which is used as a delimiter).
% Compilation is handed over to the new file by |\childdocforward|:
%    \begin{macrocode}
\newcommand{\childdocforwardprefix}[3][]
{
  \begingroup
    \def\childdocextract #2##1~~~{\def\childdoctmp{\childdocforward[#1]{#3##1}}}
    \expandafter\childdocextract\childdocname~~~
    \expandafter
  \endgroup
  \childdoctmp
}
%    \end{macrocode}

% \macro{\childdoc}
% The deprecated macro |\childdoc| is a legacy version of |\childdocmain|:
%    \begin{macrocode}
\newcommand{\childdoc}{\childdocmain}
%    \end{macrocode}

% \macro{\childdocredirect}
% The deprecated macro |\childdocredirect| is a legacy version
% of |\childdocforward| and |\childdocforwardprefix|:
%    \begin{macrocode}
\newcommand{\childdocredirect}[2][]
{
  \begingroup
    \if?#1?
      \def\childdoctmp{\childdocforward{#2}}
    \else
      \def\childdoctmp{\childdocforwardprefix{#1}{#2}}
    \fi
    \expandafter
  \endgroup
  \childdoctmp
}
%    \end{macrocode}

%\iffalse
%</package>
%\fi
%
\endinput

\childdocof{cdocsamp}
%    \end{macrocode}

%\iffalse
%</samplechap1|samplechap2>
%\fi
%
%\iffalse
%<*samplechap1>
%\fi
% Some text for chapter 1:
%    \begin{macrocode}
\section{one}
some text in chapter one
%    \end{macrocode}

%\iffalse
%</samplechap1>
%\fi
% Some text for chapter 2:
%\iffalse
%<*samplechap2>
%\fi
%    \begin{macrocode}
\section{two}
more text in chapter two
%    \end{macrocode}

%\iffalse
%</samplechap2>
%\fi
%
% %%%%%%%%%%%%%%%%%%%%%%%%%%%%%%%%%%%%%%
% \paragraph{Part Include Files.}
%
% The include files are called |cdocspt3.tex| and |cdocspt4.tex|.
%
%\iffalse
%<*samplepart3|samplepart4>
%\fi

% Optional override for |\version| flag:
%    \begin{macrocode}
%%\providecommand{\version}{final}
%    \end{macrocode}

% Include the main document:
%    \begin{macrocode}
% \iffalse
%
% childdoc.dtx Copyright (C) 2017-2018 Niklas Beisert
%
% This work may be distributed and/or modified under the
% conditions of the LaTeX Project Public License, either version 1.3
% of this license or (at your option) any later version.
% The latest version of this license is in
%   http://www.latex-project.org/lppl.txt
% and version 1.3 or later is part of all distributions of LaTeX
% version 2005/12/01 or later.
%
% This work has the LPPL maintenance status `maintained'.
%
% The Current Maintainer of this work is Niklas Beisert.
%
% This work consists of the files childdoc.dtx and childdoc.ins
% and the derived files childdoc.def and cdocsamp.tex with
% cdocsch1.tex, cdocsch2.tex, cdocsdrf.tex, cdocsfn1.tex, cdocsfn2.tex.
%
%<package>\ifdefined\childdocmain\endinput\fi
%<package>\ProvidesFile{childdoc.def}[2018/12/30 v2.0 child document driver]
%<samplemain>\ProvidesFile{cdocsamp.tex}[2018/12/30 v2.0 sample for childdoc]
%<*driver>
%\ProvidesFile{childdoc.drv}[2018/12/30 v2.0 childdoc reference manual file]
\PassOptionsToClass{10pt,a4paper}{article}
\documentclass{ltxdoc}

\usepackage[margin=35mm]{geometry}
\usepackage{hyperref}
\usepackage{hyperxmp}
\usepackage[usenames]{color}

\hypersetup{colorlinks=true}
\hypersetup{pdfstartview=FitH}
\hypersetup{pdfpagemode=UseNone}
\hypersetup{pdfsource={}}
\hypersetup{pdflang={en-UK}}
\hypersetup{pdfcopyright={Copyright 2017-2018 Niklas Beisert.
  This work may be distributed and/or modified under the
  conditions of the LaTeX Project Public License, either version 1.3
  of this license or (at your option) any later version.}}
\hypersetup{pdflicenseurl={http://www.latex-project.org/lppl.txt}}
\hypersetup{pdfcontactaddress={ETH Zurich, ITP, HIT K,
  Wolfgang-Pauli-Strasse 27}}
\hypersetup{pdfcontactpostcode={8093}}
\hypersetup{pdfcontactcity={Zurich}}
\hypersetup{pdfcontactcountry={Switzerland}}
\hypersetup{pdfcontactemail={nbeisert@itp.phys.ethz.ch}}
\hypersetup{pdfcontacturl={http://people.phys.ethz.ch/\xmptilde nbeisert/}}

\newcommand{\secref}[1]{\hyperref[#1]{section \ref*{#1}}}

\parskip1ex
\parindent0pt
\let\olditemize\itemize
\def\itemize{\olditemize\parskip0pt}

\begin{document}

\title{The \textsf{childdoc} Package}
\hypersetup{pdftitle={The childdoc Package}}
\author{Niklas Beisert\\[2ex]
  Institut f\"ur Theoretische Physik\\
  Eidgen\"ossische Technische Hochschule Z\"urich\\
  Wolfgang-Pauli-Strasse 27, 8093 Z\"urich, Switzerland\\[1ex]
  \href{mailto:nbeisert@itp.phys.ethz.ch}
  {\texttt{nbeisert@itp.phys.ethz.ch}}}
\hypersetup{pdfauthor={Niklas Beisert}}
\hypersetup{pdfsubject={Manual for the LaTeX2e Package childdoc}}
\date{30 December 2018, \textsf{v2.0}}
\maketitle

\begin{abstract}\noindent
\textsf{childdoc} is a \LaTeXe{} package
that enables the direct compilation
of document sections included by |\include|
to individual files.
\end{abstract}

\begingroup
\parskip0ex
\tableofcontents
\endgroup

%%%%%%%%%%%%%%%%%%%%%%%%%%%%%%%%%%%%%%%%%%%%%%%%%%%%%%%%%%%%%%%%%%%%%%%%%%%%%%%%
%%%%%%%%%%%%%%%%%%%%%%%%%%%%%%%%%%%%%%%%%%%%%%%%%%%%%%%%%%%%%%%%%%%%%%%%%%%%%%%%
\section{Introduction}

\LaTeX{} provides a mechanism to structure a large document (such as a book)
into a main file and several child files (containing the chapters)
using the |\include| command.
This mechanism is beneficial for documents
which span hundreds of pages in order to
make the source file(s) more manageable.
Moreover, compilation can be restricted to
selected child files by means of the |\includeonly| command.
The latter feature can be used to reduce the compilation time while editing
(this was significantly more useful in the earlier days of \LaTeX{})
or to generate a smaller document which is easier to navigate.
Another application of |\includeonly| is to generate
documents consisting of selected parts of the complete document.

However, there are a few drawbacks of the plain |\include| mechanism:
\begin{itemize}
\item
The child files cannot be compiled on their own,
they can only be compiled via the main file.
A naive editing environment
(such as a text editor with an option
to have the current file processed by \LaTeX)
may require one to switch to the main file before compiling;
attempting to compile the child file produces errors.
\item
The main file must be modified (each time)
to adjust the |\includeonly| command
to the present needs. This easily leaves the main file in a messy state.
\item
The generated document will always carry the filename
of the main document. This is inconvenient if
several child files are to be compiled and
to be kept for distribution.
\end{itemize}

The present package provides a simple interface
to make child files individually compilable by \LaTeX{}.
Compiling a child file then has the same effect as compiling
the main file with an |\includeonly| command
to select the appropriate child.
Moreover the generated document will carry the name of the child
rather than the main file.
This resolves all three above issues.

This feature is meant to make the editing of books,
thesis documents and lecture notes somewhat more convenient.
However, the package can also be used efficiently for
composing a series of documents (such as exercise sheets)
which are typically distributed individually.
It then assists the author in generating the individual documents
(potentially in different versions)
as well as a document containing the collected series.
Another application is in developing style files
or other kinds of included material
where compilation of the style file could redirect
to a sample or test file.

%%%%%%%%%%%%%%%%%%%%%%%%%%%%%%%%%%%%%%%%%%%%%%%%%%%%%%%%%%%%%%%%%%%%%%%%%%%%%%%%
%%%%%%%%%%%%%%%%%%%%%%%%%%%%%%%%%%%%%%%%%%%%%%%%%%%%%%%%%%%%%%%%%%%%%%%%%%%%%%%%
\section{Usage}

First of all, the package \textsf{childdoc} is \emph{not} a standard
\LaTeXe{} |.sty| style file! Therefore it needs to be invoked in
a non-standard way.

%%%%%%%%%%%%%%%%%%%%%%%%%%%%%%%%%%%%%%%%%%%%%%%%%%%%%%%%%%%%%%%%%%%%%%%%%%%%%%%%
\subsection{Included Files}
\label{sec:include}

%%%%%%%%%%%%%%%%%%%%%%%%%%%%%%%%%%%%%%%%
\DescribeMacro{\childdocmain}
To use the package, add the commands
\begin{center}
\begin{tabular}{l}
|\input{childdoc.def}|\\
|\childdocmain{}|\\
\end{tabular}
\end{center}
at the very top of the main \LaTeX{} file,
in particular \emph{before} the |\documentclass| statement!
The argument of |\childdocmain| should be left empty
(but it must be present).

%%%%%%%%%%%%%%%%%%%%%%%%%%%%%%%%%%%%%%%%
\DescribeMacro{\childdocof}
Furthermore, add the commands
\begin{center}
\begin{tabular}{l}
|\input{childdoc.def}|\\
|\childdocof{|\textit{main}|}|\\
\end{tabular}
\end{center}
at the top of every child file \textit{child}
which is included by |\include{|\textit{child}|}|
from within the main file
(or at least for those files to be compiled individually).
The argument \textit{main} must be the filename of the main file.

There are a couple of
considerations in setting up the main and child documents:

%%%%%%%%%%%%%%%%%%%%%%%%%%%%%%%%%%%%%%%%
\paragraph{Restrictions.}

Please note the following restrictions:
\begin{itemize}
\item
|\childdocmain| must be called with one argument \textit{main}
to ensure compatibility with earlier version of the package.
It must either be empty (|\childdocmain{}|)
or precisely match the filename of the main file in which it is specified.
See \secref{sec:detection} for further information.
\item
The filename \textit{main} must be specified without the |.tex| extension.
\item
The filename \textit{main} is case sensitive
(even in case-insensitive file systems)
due to internal string comparison.
\item
The argument \textit{main} should be fully expanded, it cannot be a macro.
\item
Subdirectories and special characters should be avoided in filenames.
\item
The command |\childdocmain{|\textit{main}|}| must be followed by a whitespace.
It should not be followed immediately by another command
or by a comment mark `|%|'.
This is because the \TeX{} parser reads the token immediately following
the argument of |\childdocmain| and puts it
at the beginning of every child section;
however, a white\-space is ignored.
\end{itemize}

%%%%%%%%%%%%%%%%%%%%%%%%%%%%%%%%%%%%%%%%
\paragraph{Content of Main File.}

It is advisable to place all content in the child files included by |\include|.
Any output contained in the main file will appear in all child documents
unless suppressed manually;
it cannot be suppressed automatically by the |\includeonly| directive
and thus should normally be avoided.
A method to include some content in the main file
by means of conditional processing is described in \secref{sec:conditional}.

%%%%%%%%%%%%%%%%%%%%%%%%%%%%%%%%%%%%%%%%
\paragraph{Page Numbering.}

When only a part of the document is compiled,
the appropriate numbering of pages
(as well as other status parameters)
is determined from the |.aux| files.
The latter contain information from previous passes.
However this information needs to propagate through
all intermediate child documents.
Therefore the page numbering in child documents may well
be inconsistent until the complete document is compiled at least once.

A useful (if unconventional) way to always ensure a consistent
page numbering is to restart the numbering in each child document
and denote the pages by `\textit{child}|.|\textit{page}'
where \textit{child} represents the chapter/section number of the child file.
This can be achieved by the command
|\numberwithin{page}{|\textit{child}|}|
of the \textsf{amsmath} package
where \textit{child} can be |chapter| or |section|
depending on the chosen structuring.
Alternatively, one can modify the macro |\thepage| appropriately
and reset the counter |page| at the start of each child file.

%%%%%%%%%%%%%%%%%%%%%%%%%%%%%%%%%%%%%%%%%%%%%%%%%%%%%%%%%%%%%%%%%%%%%%%%%%%%%%%%
\subsection{Conditional Processing}
\label{sec:conditional}

The package provides a mechanism to compile different versions
of a document. To customise the versions further some conditional processing
can come in handy to distinguish which version is being compiled.
The package provides two macros to describe the compilation context:

%%%%%%%%%%%%%%%%%%%%%%%%%%%%%%%%%%%%%%%%
\DescribeMacro{\ifchilddoc}
The conditional |\ifchilddoc| distinguishes between the compilation of
child documents and the main document:
%
\begin{center}
|\ifchilddoc |\textit{child-code}| |[|\||else |\textit{main-code}]| \||fi|
\end{center}

%%%%%%%%%%%%%%%%%%%%%%%%%%%%%%%%%%%%%%%%
\DescribeMacro{\childdocname}
\DescribeMacro{\childdocjob}
The macro |\childdocname| contains the filename (without extension)
of the main or child file being processed.
Note that |\childdocjob| will always contain the name of the main file.

%%%%%%%%%%%%%%%%%%%%%%%%%%%%%%%%%%%%%%%%
\paragraph{Title Page.}

Conditional processing can be used to include a title or banner page
in the main document when proper precautions are taken.
Importantly, the code in the main file should ensure that the page counter
(as well as other status parameters which are stored in the |.aux| files)
takes the same value after the conditional processing.
Otherwise the page numbers may take divergent values
depending on which part is compiled.

For example, a title page could be declared by:
%
\begin{center}
\begin{tabular}{l}
|\ifchilddoc\||else|\\
|\addtocounter{page}{-1}|\\
\textit{code for title page}\\
|\newpage|\\
|\||fi|
\end{tabular}
\end{center}
%
A banner page for the child documents can be generated by:
%
\begin{center}
\begin{tabular}{l}
|\ifchilddoc|\\
|\addtocounter{page}{-1}|\\
\textit{code for banner page}\\
|\newpage|\\
|\||fi|
\end{tabular}
\end{center}
%
Here one could write a message such as:
\begin{center}
|This is the part \childdocname{} of \childdocjob{}.|
\end{center}

%%%%%%%%%%%%%%%%%%%%%%%%%%%%%%%%%%%%%%%%%%%%%%%%%%%%%%%%%%%%%%%%%%%%%%%%%%%%%%%%
\subsection{Flags}
\label{sec:flags}

The package makes it easy to generate different versions
of the main or child documents.
To this end compilation flags can be defined
and assigned different default values.
They will be particularly useful in conjunction
with the forwarding mechanism described in \secref{sec:forward}.

For example, it may be useful to have a flag |\version|
which can be set to |draft| or |final|.
The document source will contain some conditional code
depending on the value of |\version|.
Suppose further, the flag should default to |final| for the main file
and to |draft| for child files
which is a natural assignment for editing the document.
This is achieved by placing the following code
in the preamble of the main document
(below the |\childdocmain| directive):
%
\begin{center}
\begin{tabular}{l}
|\ifchilddoc|\\
|\providecommand{\version}{draft}|\\
|\||else|\\
|\providecommand{\version}{final}|\\
|\||fi|
\end{tabular}
\end{center}
%
The definition by |\providecommand| makes sure
that previous definitions are not overwritten.
Further statements |\providecommand{\version}{...}|
can thus be added before the above code to override it.

For the main file, one might add a line
(between |\childdocmain| and the above block)
%
\begin{center}
|%\ifchilddoc\||else\providecommand{\version}{draft}\||fi|
\end{center}
%
which can be uncommented to produce a draft version.
Likewise one can add a line to the very top of a child file
(above the |\childdocof{|\textit{main}|}| directive)
%
\begin{center}
|%\providecommand{\version}{final}|
\end{center}
%
which can be uncommented to produce the final version of this child document.

%%%%%%%%%%%%%%%%%%%%%%%%%%%%%%%%%%%%%%%%%%%%%%%%%%%%%%%%%%%%%%%%%%%%%%%%%%%%%%%%
\subsection{Forwarding}
\label{sec:forward}

Different versions of the main or child documents
using compilation flags as described in \secref{sec:flags}
can be (permanently) stored in different files
for convenient compilation, viewing and distribution.
To this end, the package defines a command
to pass on compilation to a different file:

%%%%%%%%%%%%%%%%%%%%%%%%%%%%%%%%%%%%%%%%
\DescribeMacro{\childdocforward}
The command |\childdocforward| redirects processing to
another source file:
%
\begin{center}
\begin{tabular}{l}
|\input{childdoc.def}|\\
|\childdocforward[|\textit{main}|]{|\textit{dest}|}|\\
\end{tabular}
\end{center}
%
The argument \textit{dest} is the destination file
(without extension).
It should be the main file or one of the child files.
Note that further \textsf{childdoc} directives
such as |\childdocof| and |\childdocforward|
in the indicated file will be processed in this form.
The optional argument \textit{main}
passes on directly to the main file \textit{main}
while pretending to compile the child \textit{dest}.
This form behaves as if \textit{dest}
issues |\childdocof{|\textit{main}|}| right away,
and no further \textsf{childdoc} directives will be processed.

%%%%%%%%%%%%%%%%%%%%%%%%%%%%%%%%%%%%%%%%
\DescribeMacro{\...prefix}
In the alternative form |\childdocforwardprefix|,
%
\begin{center}
\begin{tabular}{l}
|\input{childdoc.def}|\\
|\childdocforwardprefix[|\textit{main}|]{|\textit{prefix}|}{|\textit{dest}|}|
\end{tabular}
\end{center}
%
the destination file is determined by a pattern
depending on the current file:
To make this work, the current file must be called
`{\textit{prefix}\hspace{0.2em}\textit{suffix}}'
with \textit{prefix} matching precisely the argument.
Processing is then passed on to the file
`{\textit{dest}\hspace{0.2em}\textit{suffix}}'.
Surely, the same effect is achieved by
directly specifying the
argument `{\textit{dest}\hspace{0.2em}\textit{suffix}}'
in the first form.
However, that requires to set up a different file
for each child. With the alternative form of the command
all these files can have exactly the same content
which simplifies setting them up and maintaining them.

For example, the following file |draft.tex|
with a compilation flag |\version| as described in \secref{sec:flags}
compiles the main document as a draft:
%
\begin{center}
\begin{tabular}{l}
|\def\version{draft}|\\
|\input{childdoc.def}|\\
|\childdocforward{|\textit{main}|}|
\end{tabular}
\end{center}
%
Likewise, the following files |final|\textit{nn}|.tex|
compile the final version of the child document
|child|\textit{nn}|.tex|:
%
\begin{center}
\begin{tabular}{l}
|\def\version{final}|\\
|\input{childdoc.def}|\\
|\childdocforwardprefix{final}{child}|
\end{tabular}
\end{center}
%

Note that when several versions of a main file and/or of each child file
are to be generated, it may be convenient to set up a |Makefile| or
shell script to automatise the process.

%%%%%%%%%%%%%%%%%%%%%%%%%%%%%%%%%%%%%%%%%%%%%%%%%%%%%%%%%%%%%%%%%%%%%%%%%%%%%%%%
\subsection{Command Line Processing}
\label{sec:commandline}

The effect of redirection files can also be achieved by invoking
the \LaTeX{} compiler with a more elaborate command line.
Most conveniently this should be done as part
of a shell script or a |Makefile|.

When using \textsf{childdoc} in the main file, the following
command lines effectively perform a redirection
(note that depending on the shell being used,
backslashes may have to be doubled: `|\|' $\to$ `|\\|'):
%
\begin{center}
|... -jobname "|\textit{target}|" |\\|"|[\textit{flags}]%
|\input{childdoc.def}\childdocforward[|\textit{main}|]{|\textit{dest}|}"|
\end{center}
%
Here \textit{target} is the name of the output file,
\textit{main} is the name of the main file
and \textit{dest} is the name of the main or child file to be processed
(all filenames without extensions).
The optional argument \textit{main} can be omitted
if \textit{main} matches \textit{dest}.
Optionally, compilation \textit{flags} can be defined via |\def| commands.
This command line makes the \TeX{} engine believe
it is compiling the file \textit{target}
whose content is specified as the latter parameter.
The provided code then forwards the processing to
\textit{main} or \textit{dest} as described in \secref{sec:forward}.

%%%%%%%%%%%%%%%%%%%%%%%%%%%%%%%%%%%%%%%%%%%%%%%%%%%%%%%%%%%%%%%%%%%%%%%%%%%%%%%%
\subsection{Include by Input}
\label{sec:input}

Including child documents by |\include| has some restrictions by design.
Most notably, the content of a child document always occupies
its own set of pages; pages cannot be shared between child documents.
Usually, this behaviour makes perfect sense
because each child document contain an essential part of the document.
However, in some situations it may be desirable to compose
a document from a collection of parts
without having mandatory page breaks between then.
For this case, the package
provides a mechanism to include parts
by |\input| which can also be processed individually.
However, by construction this mechanism
requires manual handling of the content to be output.

%%%%%%%%%%%%%%%%%%%%%%%%%%%%%%%%%%%%%%%%
\DescribeMacro{\ifchilddocmanual}
The main file should be prepared as usual, see \secref{sec:include}.
However, the document body must make a distinction
between processing of an individual part and of the main document, e.g.:
%
\begin{center}
\begin{tabular}{l}
|\ifchilddocmanual|\\
|\input{\childdocname}|\\
|\||else|\\
\textit{document body with }|\input{|\textit{part}|}|\\
|\||fi|
\end{tabular}
\end{center}
%
The conditional |\ifchilddocmanual| is true whenever
a part to be included by |\input| is being compiled,
and the name of the part is stored in |\childdocname|.

%%%%%%%%%%%%%%%%%%%%%%%%%%%%%%%%%%%%%%%%
\DescribeMacro{\childdocby}
Each part to be included by |\input| should start with:
%
\begin{center}
\begin{tabular}{l}
|\input{childdoc.def}|\\
|\childdocby{|\textit{main}|}|\\
\end{tabular}
\end{center}
%
The directive |\childdocby| is similar to |\childdocof|
described in \secref{sec:include},
but the subsequent selection of content must be done manually.
To that end, both |\ifchilddoc| and |\ifchilddocmanual|
will be true upon processing of a part,
and the name of the part is stored in |\childdocname|.
Note that |\jobname| will be set to the filename of the current part
so that each part receives an individual |.aux| file
that does not interfere with the |.aux| file(s) of the main document.
This behaviour can be altered by the alternative form
|\childdocby[*]{|\textit{main}|}| (with a non-empty optional argument)
which uses the |.aux| file of the main document
by setting |\jobname| to \textit{main}.

%%%%%%%%%%%%%%%%%%%%%%%%%%%%%%%%%%%%%%%%%%%%%%%%%%%%%%%%%%%%%%%%%%%%%%%%%%%%%%%%
\subsection{Driver Development}
\label{sec:driver}

The \textsf{childdoc} mechanism can also be use for the development
of definition files such as \LaTeX{} styles or classes.
This case differs from the above setup with multiple parts
included by |\include| in that no |\includeonly| should be invoked.
This can be achieved by starting the include file
(before |\ProvidesPackage|) with:
%
\begin{center}
\begin{tabular}{l}
|\input{childdoc.def}|\\
|\childdocforward{|\textit{main}|}|\\
\end{tabular}
\end{center}
%
or alternatively with:
%
\begin{center}
\begin{tabular}{l}
|\input{childdoc.def}|\\
|\childdocby{|\textit{main}|}|\\
\end{tabular}
\end{center}
%
Both forms have slightly different effects as described above.
The main file is prepared as usual, see \secref{sec:include}.

%%%%%%%%%%%%%%%%%%%%%%%%%%%%%%%%%%%%%%%%%%%%%%%%%%%%%%%%%%%%%%%%%%%%%%%%%%%%%%%%
\subsection{Legacy Detection}
\label{sec:detection}

The directive |\childdocmain| in the main file can detect
whether the complete document or merely a child is to be compiled
even without using the directive |\childdocof|.
This method is deprecated because it is less robust
and there is no compelling reason to use it;
it is merely provided for backward compatibility
and it may be removed in future versions.

If the detection mechanism is to be used,
it is mandatory to correctly specify
the filename of the main file as the argument of |\childdocmain|:
%
\begin{center}
\begin{tabular}{l}
|\input{childdoc.def}|\\
|\childdocmain{|\textit{main}|}|\\
\end{tabular}
\end{center}
%
If |\jobname| does not match the argument \textit{main} of |\childdocmain|,
it is assumed that |\jobname| points to the child file to be compiled.
When using |\childdocmain| with the main file specified as argument,
it suffices to start a child file
with just |\input{|\textit{main}|}|
without loading of the package and using |\childdocof|.
If instead all processing is done
with the appropriate \textsf{childdoc} directives,
the argument of \textit{main} of |\childdocmain| can be empty.

An alternative version of the command line processing described
in \secref{sec:commandline} using the detection mechanism reads:
%
\begin{center}
|... -jobname "|\textit{target}|" "|[\textit{flags}]%
[|\def\jobname{|\textit{dest}|}|]|\input{|\textit{main}|}"|
\end{center}

%%%%%%%%%%%%%%%%%%%%%%%%%%%%%%%%%%%%%%%%%%%%%%%%%%%%%%%%%%%%%%%%%%%%%%%%%%%%%%%%
\subsection{Manual Code}
\label{sec:manual}

In case one cannot be certain whether the definitions file |childdoc.def|
is installed on the target \TeX{} distribution
and one prefers not to ship it,
it is conceivable to paste a few relevant commands into the sources.

To that end, drop all statements |\input{childdoc.def}|
and perform the replacements as outlined below.
Instead of |\childdocmain{|\textit{main}|}| add the following code
to the top of the main file:
%
\begin{center}
\begin{tabular}{l}
|\||ifdefined\childdocname\endinput\||fi\newif\ifchilddoc|\\
|\edef\childdocname{\scantokens\expandafter{\jobname\noexpand}}|\\
|\def\childdocmain{|\textit{main}|}\||ifx\childdocmain\childdocname\||else|\\
|\childdoctrue\includeonly{\childdocname}\let\jobname\childdocmain\||fi|\\
\end{tabular}
\end{center}
%
Instead of |\childdocof{|\textit{main}|}| just include the main file
at the top of each child file:
%
\begin{center}
|\input{|\textit{main}|}|
\end{center}
%
A simple redirection |\childdocforward{|\textit{dest}|}| is achieved by:
%
\begin{center}
|\def\jobname{|\textit{dest}|}\input{\jobname}|
\end{center}
%
The redirection with prefix
|\childdocforwardprefix[|\textit{prefix}|]{|\textit{dest}|}|
is accomplished by:
%
\begin{center}
\begin{tabular}{l}
|{\edef\jobname{\scantokens\expandafter{\jobname\noexpand}}|\\
|\def\redirectjob |\textit{prefix}|#1~~~{\gdef\jobname{|\textit{dest}|#1}}|\\
|\expandafter\redirectjob\jobname~~~}\input{\jobname}|
\end{tabular}
\end{center}

In an alternative approach,
child documents can be compiled by a specific command line
without additional code or specific definitions:
%
\begin{center}
|... -jobname "|\textit{target}|" "|[\textit{flags}]%
|\includeonly{|\textit{dest}|}\input{|\textit{main}|}"|
\end{center}
%

%%%%%%%%%%%%%%%%%%%%%%%%%%%%%%%%%%%%%%%%%%%%%%%%%%%%%%%%%%%%%%%%%%%%%%%%%%%%%%%%
%%%%%%%%%%%%%%%%%%%%%%%%%%%%%%%%%%%%%%%%%%%%%%%%%%%%%%%%%%%%%%%%%%%%%%%%%%%%%%%%
\section{Information}

%%%%%%%%%%%%%%%%%%%%%%%%%%%%%%%%%%%%%%%%%%%%%%%%%%%%%%%%%%%%%%%%%%%%%%%%%%%%%%%%
\subsection{Copyright}

Copyright \copyright{} 2017--2018 Niklas Beisert

This work may be distributed and/or modified under the
conditions of the \LaTeX{} Project Public License, either version 1.3
of this license or (at your option) any later version.
The latest version of this license is in
  \url{http://www.latex-project.org/lppl.txt}
and version 1.3 or later is part of all distributions of \LaTeX{}
version 2005/12/01 or later.

This work has the LPPL maintenance status `maintained'.

The Current Maintainer of this work is Niklas Beisert.

This work consists of the files |README.txt|, |childdoc.ins| and |childdoc.dtx|
as well as the derived files |childdoc.def|, |cdocsamp.tex|
with |cdocsch1.tex|, |cdocsch2.tex|, |cdocspt3.tex|, |cdocspt4.tex|,
|cdocsdrf.tex|, |cdocsfn1.tex|, |cdocsfn2.tex|
as well as |childdoc.pdf|.

%%%%%%%%%%%%%%%%%%%%%%%%%%%%%%%%%%%%%%%%%%%%%%%%%%%%%%%%%%%%%%%%%%%%%%%%%%%%%%%%
\subsection{Files and Installation}

The package consists of the files:
%
\begin{center}
\begin{tabular}{ll}
    |README.txt|   & readme file \\
    |childdoc.ins| & installation file \\
    |childdoc.dtx| & source file \\
    |childdoc.def| & definition file \\
    |cdocsamp.tex| & sample main file \\
    |cdocsch1.tex| & sample include file \\
    |cdocsch2.tex| & sample include file \\
    |cdocspt3.tex| & sample part file \\
    |cdocspt4.tex| & sample part file \\
    |cdocsdrf.tex| & sample redirection file \\
    |cdocsfn1.tex| & sample redirection file \\
    |cdocsfn2.tex| & sample redirection file \\
    |childdoc.pdf| & manual
\end{tabular}
\end{center}
%
The distribution consists of the files
|README.txt|, |childdoc.ins| and |childdoc.dtx|.
%
\begin{itemize}
\item
Run (pdf)\LaTeX{} on |childdoc.dtx|
to compile the manual |childdoc.pdf| (this file).
\item
Run \LaTeX{} on |childdoc.ins| to create the definitions file |childdoc.def|
and the sample |cdocsamp.tex| with include files
|cdocsch1.tex|, |cdocsch2.tex|, |cdocspt3.tex|, |cdocspt4.tex|,
|cdocsdrf.tex|, |cdocsfn1.tex|, |cdocsfn2.tex|.
Then copy the file |childdoc.def| to an appropriate directory of your \LaTeX{}
distribution, e.g.\ \textit{texmf-root}|/tex/latex/childdoc|.
\end{itemize}

%%%%%%%%%%%%%%%%%%%%%%%%%%%%%%%%%%%%%%%%%%%%%%%%%%%%%%%%%%%%%%%%%%%%%%%%%%%%%%%%
\subsection{Related CTAN Packages}

There are several other packages which offer a similar functionality:
%
\begin{itemize}
\item
The packages
\href{http://ctan.org/pkg/docmute}{\textsf{docmute}},
\href{http://ctan.org/pkg/includex}{\textsf{includex}} and
\href{http://ctan.org/pkg/standalone}{\textsf{standalone}}
provide commands to include only the document body of
a child file thus allowing both files to be compiled individually.
\item
The packages \href{http://ctan.org/pkg/subdocs}{\textsf{subdocs}}
and \href{http://ctan.org/pkg/subfiles}{\textsf{subfiles}}
provide structures in which the main and child documents can be
encapsulated and allowing them to be compiled individually.
The inclusion mechanism is different from the conventional |\include|.
\item
The package \href{http://ctan.org/pkg/combine}{\textsf{combine}}
is an elaborate solution to combine several documents into one.
\end{itemize}
%
See also the CTAN topic \href{http://ctan.org/topic/subdocs}{\textsf{subdocs}}
for further related packages.
The present package differs from the above solutions in that
a document structure constructed with the conventional |\include| mechanism
just needs two extra commands at the top of every file
such that all constituent files can be compiled individually.

%%%%%%%%%%%%%%%%%%%%%%%%%%%%%%%%%%%%%%%%%%%%%%%%%%%%%%%%%%%%%%%%%%%%%%%%%%%%%%%%
%\subsection{Feature Suggestions}
%
%The following is a list of features which may be useful for future
%versions of this package:
%%
%\begin{itemize}
%\item
%\ldots
%\end{itemize}

%%%%%%%%%%%%%%%%%%%%%%%%%%%%%%%%%%%%%%%%%%%%%%%%%%%%%%%%%%%%%%%%%%%%%%%%%%%%%%%%
\subsection{Revision History}

%%%%%%%%%%%%%%%%%%%%%%%%%%%%%%%%%%%%%%%%
\paragraph{v2.0:} 2018/12/30

\begin{itemize}
\item
immediate forward processing
\item
added |\childdocby| mechanism
\item
manual restructured
\end{itemize}

%%%%%%%%%%%%%%%%%%%%%%%%%%%%%%%%%%%%%%%%
\paragraph{v1.6:} 2018/01/17

\begin{itemize}
\item
application for development of include files
\item
corrections to manual
\end{itemize}

%%%%%%%%%%%%%%%%%%%%%%%%%%%%%%%%%%%%%%%%
\paragraph{v1.5:} 2017/05/21

\begin{itemize}
\item
more complete structuring introduced
\item
|\childdocof| introduced
\item
|\childdoc| renamed to |\childdocmain|
\item
|\childredirect| renamed to |\childdocforward| and |\childdocforwardprefix|
and functionality expanded
\end{itemize}

%%%%%%%%%%%%%%%%%%%%%%%%%%%%%%%%%%%%%%%%
\paragraph{v1.0:} 2017/04/27

\begin{itemize}
\item
manual and install package
\item
first version published on CTAN
\end{itemize}

%%%%%%%%%%%%%%%%%%%%%%%%%%%%%%%%%%%%%%%%
\paragraph{v0.6:} 2017/04/26

\begin{itemize}
\item
redirection mechanism added
\end{itemize}

%%%%%%%%%%%%%%%%%%%%%%%%%%%%%%%%%%%%%%%%
\paragraph{v0.5:} 2017/04/26

\begin{itemize}
\item
functionality in definition file
\end{itemize}


%%%%%%%%%%%%%%%%%%%%%%%%%%%%%%%%%%%%%%%%%%%%%%%%%%%%%%%%%%%%%%%%%%%%%%%%%%%%%%%%
%%%%%%%%%%%%%%%%%%%%%%%%%%%%%%%%%%%%%%%%%%%%%%%%%%%%%%%%%%%%%%%%%%%%%%%%%%%%%%%%
%%%%%%%%%%%%%%%%%%%%%%%%%%%%%%%%%%%%%%%%%%%%%%%%%%%%%%%%%%%%%%%%%%%%%%%%%%%%%%%%
\appendix

\settowidth\MacroIndent{\rmfamily\scriptsize 000\ }

 \DocInput{childdoc.dtx}

\end{document}
%</driver>
% \fi
%
% %%%%%%%%%%%%%%%%%%%%%%%%%%%%%%%%%%%%%%%%%%%%%%%%%%%%%%%%%%%%%%%%%%%%%%%%%%%%%%
% %%%%%%%%%%%%%%%%%%%%%%%%%%%%%%%%%%%%%%%%%%%%%%%%%%%%%%%%%%%%%%%%%%%%%%%%%%%%%%
% \section{Sample}
%\iffalse
%<*samplemain>
%\fi
%
% The following presents a sample document
% with two chapters, two parts, a title page,
% a compile flag as well as three forwarding files to set the flag.
% It consists of eight |.tex| files:
% \begin{center}
% \begin{tabular}{ll}
% |cdocsamp.tex|&main file\\
% |cdocsch1.tex|&include file for chapter 1\\
% |cdocsch2.tex|&include file for chapter 2\\
% |cdocspt3.tex|&include file for part 3\\
% |cdocspt4.tex|&include file for part 4\\
% |cdocsdrf.tex|&forwarding file for main file in draft mode\\
% |cdocsfi1.tex|&forwarding file for final version of chapter 1\\
% |cdocsfi2.tex|&forwarding file for final version of chapter 2\\
% \end{tabular}
% \end{center}
% Each of the eight files can be compiled directly by the \LaTeX{} compiler.
%
% %%%%%%%%%%%%%%%%%%%%%%%%%%%%%%%%%%%%%%
% \paragraph{Main File.}
%
% The main file is called |cdocsamp.tex|.
%
% Load the \textsf{childdoc} definitions and
% declare the filename for the main document:
%    \begin{macrocode}
\input{childdoc.def}
\childdocmain{}
%    \end{macrocode}

% Optional override for |\version| flag:
%    \begin{macrocode}
%%\ifchilddoc\else\providecommand{\version}{draft}\fi
%    \end{macrocode}

% Define the default values for the |\version| flag
% (|final| for the main file and |draft| for childs):
%    \begin{macrocode}
\ifchilddoc
\providecommand{\version}{draft}
\else
\providecommand{\version}{final}
\fi
%    \end{macrocode}

% Load the standard document class:
%    \begin{macrocode}
\documentclass[12pt]{article}
%    \end{macrocode}

% Start the document body:
%    \begin{macrocode}
\begin{document}
%    \end{macrocode}

% Declare a title page.
% Print title, part of document being processed and version flag:
%    \begin{macrocode}
\addtocounter{page}{-1}
\begin{center}
{\LARGE\bfseries{}childdoc example\par}
\vspace{1cm}
\ifchilddoc
\ifchilddocmanual part\else chapter\fi:
`\childdocname' of `\childdocjob'\par
\else
main document: `\childdocjob'\par
\fi
version: \version\par
\end{center}
\newpage
%    \end{macrocode}

% Manually include selected file,
% otherwise process as usual:
%    \begin{macrocode}
\ifchilddocmanual
\section*{part `\childdocname'}
\input{\childdocname}
\else
%    \end{macrocode}

% Include the two chapters:
%    \begin{macrocode}
\include{cdocsch1}
\include{cdocsch2}
%    \end{macrocode}

% Include the two parts unless only chapters should be displayed:
%    \begin{macrocode}
\ifchilddoc\else
\section{part three}
\input{cdocspt3}
\section{part four}
\input{cdocspt4}
\fi
%    \end{macrocode}

% Process as usual until here:
%    \begin{macrocode}
\fi
%    \end{macrocode}

% End of document body:
%    \begin{macrocode}
\end{document}
%    \end{macrocode}
%\iffalse
%</samplemain>
%\fi
%
% %%%%%%%%%%%%%%%%%%%%%%%%%%%%%%%%%%%%%%
% \paragraph{Chapter Include Files.}
%
% The include files are called |cdocsch1.tex| and |cdocsch2.tex|.
%
%\iffalse
%<*samplechap1|samplechap2>
%\fi

% Optional override for |\version| flag:
%    \begin{macrocode}
%%\providecommand{\version}{final}
%    \end{macrocode}

% Include the main document:
%    \begin{macrocode}
\input{childdoc.def}
\childdocof{cdocsamp}
%    \end{macrocode}

%\iffalse
%</samplechap1|samplechap2>
%\fi
%
%\iffalse
%<*samplechap1>
%\fi
% Some text for chapter 1:
%    \begin{macrocode}
\section{one}
some text in chapter one
%    \end{macrocode}

%\iffalse
%</samplechap1>
%\fi
% Some text for chapter 2:
%\iffalse
%<*samplechap2>
%\fi
%    \begin{macrocode}
\section{two}
more text in chapter two
%    \end{macrocode}

%\iffalse
%</samplechap2>
%\fi
%
% %%%%%%%%%%%%%%%%%%%%%%%%%%%%%%%%%%%%%%
% \paragraph{Part Include Files.}
%
% The include files are called |cdocspt3.tex| and |cdocspt4.tex|.
%
%\iffalse
%<*samplepart3|samplepart4>
%\fi

% Optional override for |\version| flag:
%    \begin{macrocode}
%%\providecommand{\version}{final}
%    \end{macrocode}

% Include the main document:
%    \begin{macrocode}
\input{childdoc.def}
\childdocby{cdocsamp}
%    \end{macrocode}

%\iffalse
%</samplepart3|samplepart4>
%\fi
%
%\iffalse
%<*samplepart3>
%\fi
% Some text for part 3:
%    \begin{macrocode}
some text in part three
%    \end{macrocode}

%\iffalse
%</samplepart3>
%\fi
% Some text for part 4:
%\iffalse
%<*samplepart4>
%\fi
%    \begin{macrocode}
more text in part four
%    \end{macrocode}

%\iffalse
%</samplepart4>
%\fi
%
% %%%%%%%%%%%%%%%%%%%%%%%%%%%%%%%%%%%%%%
% \paragraph{Forwarding for a Complete Draft.}
%
% The following forwarding file |cdocsdrf.tex|
% compiles the main document in draft mode:
%\iffalse
%<*sampledraft>
%\fi
%    \begin{macrocode}
\def\version{draft}
\input{childdoc.def}
\childdocforward{cdocsamp}
%    \end{macrocode}

%\iffalse
%</sampledraft>
%\fi
%
% %%%%%%%%%%%%%%%%%%%%%%%%%%%%%%%%%%%%%%
% \paragraph{Forwarding for Final Version of the Chapters.}
%
% The following forwarding files |cdocsfn1.tex| and |cdocsfn2.tex|
% (with identical content)
% compile the final versions of the child documents
% |cdocsch1.tex| and |cdocsch2.tex|, respectively:
%\iffalse
%<*samplefinal>
%\fi
%    \begin{macrocode}
\def\version{final}
\input{childdoc.def}
\childdocforwardprefix[cdocsamp]{cdocsfn}{cdocsch}
%    \end{macrocode}

%\iffalse
%</samplefinal>
%\fi
%
% %%%%%%%%%%%%%%%%%%%%%%%%%%%%%%%%%%%%%%
% \paragraph{Command Line Processing.}
%
% The following three command lines generate the output files
% |cdocscld|, |cdocscl1| and |cdocscl2|
% which should be identical to
% |cdocsdrf|, |cdocsch1| and |cdocsfn2|, respectively:
% \begin{center}
% \begin{tabular}{l}
% |latex -jobname cdocscld \|\\
% |  "\def\version{draft}\input{childdoc.def}\childdocforward{cdocsamp}"|\\
% |latex -jobname cdocscl1 \|\\
% |  "\input{childdoc.def}\childdocforward[cdocsamp]{cdocsch1}"|\\
% |latex -jobname cdocscl2 \|\\
% |  "\def\version{final}\input{childdoc.def}\childdocforward{cdocsch2}"|
% \end{tabular}
% \end{center}
% Note that the trailing backslash on each first line
% merely continues the input to the second line
% (for convenient cut ant paste).
% Furthermore, the command |latex| can be replaced by any
% of its alternative versions such as |pdflatex|.
%
% %%%%%%%%%%%%%%%%%%%%%%%%%%%%%%%%%%%%%%%%%%%%%%%%%%%%%%%%%%%%%%%%%%%%%%%%%%%%%%
% %%%%%%%%%%%%%%%%%%%%%%%%%%%%%%%%%%%%%%%%%%%%%%%%%%%%%%%%%%%%%%%%%%%%%%%%%%%%%%
% \section{Implementation}
%\iffalse
%<*package>
%\fi
%
% This section describes the definitions file |childdoc.def|.

% The definitions cannot be loaded using |\usepackage| or |\RequirePackage|
% which has a mechanism to prevent loading a style file more than once.
% When loading the definitions by means of |\input|
% multiple instances have to be prevented manually:
%\iffalse
%This code needs to be before the `\ProvidesFile' directive
%which is defined at the beginning of this file.
%Therefore it is also placed there and commented out here.
%</package>
%<*discard>
%\fi
%    \begin{macrocode}
\ifdefined\childdocmain\endinput\fi
%    \end{macrocode}
%\iffalse
%</discard>
%<*package>
%\fi
%
% \macro{\ifchilddoc}
% \macro{\ifchilddocmanual}
% The conditional |\ifchilddoc| tells whether a
% child (true) or main (false) document is being compiled.
% The conditional |\ifchilddocmanual| tells whether
% the |\includeonly| mechanism is used (false) or
% the selection of child files must be performed manually (true).
% The definitions initialise to false:
%    \begin{macrocode}
\newif\ifchilddoc
\newif\ifchilddocmanual
%    \end{macrocode}

% \macro{\childdocname}
% \macro{\childdocjob}
% The macro |\childdocname| stores the name of the main document
% to be compiled. The macro |\childdocjob| stores the name of
% the document on which the \LaTeX{} compiler was originally invoked.
% The content of |\jobname| cannot be compared
% to filenames specified in the source due to different catcodes.
% The following code rescans |\jobname|, stores the result
% in |\childdocname| and saves a copy in |\childdocjob|:
%    \begin{macrocode}
\edef\childdocname{\scantokens\expandafter{\jobname\noexpand}}
\let\childdocjob\childdocname
%    \end{macrocode}

% \macro{\childdocdisable}
% The macro |\childdocdisable| prevents the main file
% from being processed more than once.
% At this stage, the main document command |\childdocmain|
% is assumed to be called once again where it should do nothing.
% Any subsequent call to it should prevent
% a secondary processing of the main document
% It overwrites the forwarding commands
% |\childdocof| and |\childdocforward|
% with empty macros to prevent further inclusions of the main document:
%    \begin{macrocode}
\newcommand{\childdocdisable}
{
  \renewcommand{\childdocmain}[1]{\renewcommand{\childdocmain}[1]{\endinput}}
  \renewcommand{\childdocof}[1]{}
  \renewcommand{\childdocby}[2][]{}
  \renewcommand{\childdocforward}[2][]{}
  \renewcommand{\childdocdisable}{}
}
%    \end{macrocode}

% \macro{\childdocmain}
% The macro |\childdocmain| is to be called at the top of the main file
% with nothing or the main filename (without extension) as argument.
% First, it breaks loops.
% If the argument is not empty and does not match |\childdocname|
% (which is set by the first inclusion of |childdoc.def|),
% |\ifchilddoc| is set to true, |\includeonly| is applied to the child file
% and |\jobname| is set to the main file
% (for proper handling of |.aux| files):
%    \begin{macrocode}
\newcommand{\childdocmain}[1]
{
  \childdocdisable\childdocmain{}
  \if?#1?\else
    \begingroup
      \def\childdoctmp{#1}
      \ifx\childdoctmp\childdocname
        \def\childdoctmp{}
      \else
        \def\childdoctmp
        {
          \childdoctrue
          \includeonly{\childdocname}
          \def\childdocjob{#1}
          \def\jobname{#1}
        }
      \fi
      \expandafter
    \endgroup
    \childdoctmp
  \fi
}
%    \end{macrocode}

% \macro{\childdocof}
% The command |\childdocof| redirects
% compilation to the main file |#1|.
%    \begin{macrocode}
\newcommand{\childdocof}[1]
{
  \childdocdisable
  \childdoctrue
  \includeonly{\childdocname}
  \def\jobname{#1}
  \def\childdocjob{#1}
  \input{#1}
}
%    \end{macrocode}

% \macro{\childdocby}
% The command |\childdocby| ....
%    \begin{macrocode}
\newcommand{\childdocby}[2][]
{
  \childdocdisable
  \childdoctrue
  \childdocmanualtrue
  \if?#1?\else
    \def\jobname{#2}
  \fi
  \def\childdocjob{#2}
  \input{#2}
  \endinput
}
%    \end{macrocode}

% \macro{\childdocforward}
% The command |\childdocforward| redirects
% compilation to the main file or
% (if the optional argument is given) a child file.
% Parameters are set as if the main file
% or a child file starting with |\childdocof| was compiled.
% Then compilation is handed over to the main file:
%    \begin{macrocode}
\newcommand{\childdocforward}[2][]
{
  \begingroup
    \if?#1?
      \def\childdoctmp
      {
        \def\childdocname{#2}
        \def\childdocjob{#2}
        \def\jobname{#2}
        \input{#2}
        \endinput
      }
    \else
      \def\childdoctmp
      {
        \childdocdisable
        \def\childdocname{#2}
        \childdoctrue
        \includeonly{#2}
        \def\childdocjob{#1}
        \def\jobname{#1}
        \input{#1}
        \endinput
      }
    \fi
    \expandafter
  \endgroup
  \childdoctmp
}
%    \end{macrocode}

% \macro{\childdocforwardprefix}
% The command |\childdocforwardprefix| redirects
% compilation to the main or a child file by means of a pattern.
% The prefix |#1| in the current filename is replaced by |#2|
% and the suffix of the current filename is kept
% (it is assumed that the filename does not contain the substring `|~~~|'
% which is used as a delimiter).
% Compilation is handed over to the new file by |\childdocforward|:
%    \begin{macrocode}
\newcommand{\childdocforwardprefix}[3][]
{
  \begingroup
    \def\childdocextract #2##1~~~{\def\childdoctmp{\childdocforward[#1]{#3##1}}}
    \expandafter\childdocextract\childdocname~~~
    \expandafter
  \endgroup
  \childdoctmp
}
%    \end{macrocode}

% \macro{\childdoc}
% The deprecated macro |\childdoc| is a legacy version of |\childdocmain|:
%    \begin{macrocode}
\newcommand{\childdoc}{\childdocmain}
%    \end{macrocode}

% \macro{\childdocredirect}
% The deprecated macro |\childdocredirect| is a legacy version
% of |\childdocforward| and |\childdocforwardprefix|:
%    \begin{macrocode}
\newcommand{\childdocredirect}[2][]
{
  \begingroup
    \if?#1?
      \def\childdoctmp{\childdocforward{#2}}
    \else
      \def\childdoctmp{\childdocforwardprefix{#1}{#2}}
    \fi
    \expandafter
  \endgroup
  \childdoctmp
}
%    \end{macrocode}

%\iffalse
%</package>
%\fi
%
\endinput

\childdocby{cdocsamp}
%    \end{macrocode}

%\iffalse
%</samplepart3|samplepart4>
%\fi
%
%\iffalse
%<*samplepart3>
%\fi
% Some text for part 3:
%    \begin{macrocode}
some text in part three
%    \end{macrocode}

%\iffalse
%</samplepart3>
%\fi
% Some text for part 4:
%\iffalse
%<*samplepart4>
%\fi
%    \begin{macrocode}
more text in part four
%    \end{macrocode}

%\iffalse
%</samplepart4>
%\fi
%
% %%%%%%%%%%%%%%%%%%%%%%%%%%%%%%%%%%%%%%
% \paragraph{Forwarding for a Complete Draft.}
%
% The following forwarding file |cdocsdrf.tex|
% compiles the main document in draft mode:
%\iffalse
%<*sampledraft>
%\fi
%    \begin{macrocode}
\def\version{draft}
% \iffalse
%
% childdoc.dtx Copyright (C) 2017-2018 Niklas Beisert
%
% This work may be distributed and/or modified under the
% conditions of the LaTeX Project Public License, either version 1.3
% of this license or (at your option) any later version.
% The latest version of this license is in
%   http://www.latex-project.org/lppl.txt
% and version 1.3 or later is part of all distributions of LaTeX
% version 2005/12/01 or later.
%
% This work has the LPPL maintenance status `maintained'.
%
% The Current Maintainer of this work is Niklas Beisert.
%
% This work consists of the files childdoc.dtx and childdoc.ins
% and the derived files childdoc.def and cdocsamp.tex with
% cdocsch1.tex, cdocsch2.tex, cdocsdrf.tex, cdocsfn1.tex, cdocsfn2.tex.
%
%<package>\ifdefined\childdocmain\endinput\fi
%<package>\ProvidesFile{childdoc.def}[2018/12/30 v2.0 child document driver]
%<samplemain>\ProvidesFile{cdocsamp.tex}[2018/12/30 v2.0 sample for childdoc]
%<*driver>
%\ProvidesFile{childdoc.drv}[2018/12/30 v2.0 childdoc reference manual file]
\PassOptionsToClass{10pt,a4paper}{article}
\documentclass{ltxdoc}

\usepackage[margin=35mm]{geometry}
\usepackage{hyperref}
\usepackage{hyperxmp}
\usepackage[usenames]{color}

\hypersetup{colorlinks=true}
\hypersetup{pdfstartview=FitH}
\hypersetup{pdfpagemode=UseNone}
\hypersetup{pdfsource={}}
\hypersetup{pdflang={en-UK}}
\hypersetup{pdfcopyright={Copyright 2017-2018 Niklas Beisert.
  This work may be distributed and/or modified under the
  conditions of the LaTeX Project Public License, either version 1.3
  of this license or (at your option) any later version.}}
\hypersetup{pdflicenseurl={http://www.latex-project.org/lppl.txt}}
\hypersetup{pdfcontactaddress={ETH Zurich, ITP, HIT K,
  Wolfgang-Pauli-Strasse 27}}
\hypersetup{pdfcontactpostcode={8093}}
\hypersetup{pdfcontactcity={Zurich}}
\hypersetup{pdfcontactcountry={Switzerland}}
\hypersetup{pdfcontactemail={nbeisert@itp.phys.ethz.ch}}
\hypersetup{pdfcontacturl={http://people.phys.ethz.ch/\xmptilde nbeisert/}}

\newcommand{\secref}[1]{\hyperref[#1]{section \ref*{#1}}}

\parskip1ex
\parindent0pt
\let\olditemize\itemize
\def\itemize{\olditemize\parskip0pt}

\begin{document}

\title{The \textsf{childdoc} Package}
\hypersetup{pdftitle={The childdoc Package}}
\author{Niklas Beisert\\[2ex]
  Institut f\"ur Theoretische Physik\\
  Eidgen\"ossische Technische Hochschule Z\"urich\\
  Wolfgang-Pauli-Strasse 27, 8093 Z\"urich, Switzerland\\[1ex]
  \href{mailto:nbeisert@itp.phys.ethz.ch}
  {\texttt{nbeisert@itp.phys.ethz.ch}}}
\hypersetup{pdfauthor={Niklas Beisert}}
\hypersetup{pdfsubject={Manual for the LaTeX2e Package childdoc}}
\date{30 December 2018, \textsf{v2.0}}
\maketitle

\begin{abstract}\noindent
\textsf{childdoc} is a \LaTeXe{} package
that enables the direct compilation
of document sections included by |\include|
to individual files.
\end{abstract}

\begingroup
\parskip0ex
\tableofcontents
\endgroup

%%%%%%%%%%%%%%%%%%%%%%%%%%%%%%%%%%%%%%%%%%%%%%%%%%%%%%%%%%%%%%%%%%%%%%%%%%%%%%%%
%%%%%%%%%%%%%%%%%%%%%%%%%%%%%%%%%%%%%%%%%%%%%%%%%%%%%%%%%%%%%%%%%%%%%%%%%%%%%%%%
\section{Introduction}

\LaTeX{} provides a mechanism to structure a large document (such as a book)
into a main file and several child files (containing the chapters)
using the |\include| command.
This mechanism is beneficial for documents
which span hundreds of pages in order to
make the source file(s) more manageable.
Moreover, compilation can be restricted to
selected child files by means of the |\includeonly| command.
The latter feature can be used to reduce the compilation time while editing
(this was significantly more useful in the earlier days of \LaTeX{})
or to generate a smaller document which is easier to navigate.
Another application of |\includeonly| is to generate
documents consisting of selected parts of the complete document.

However, there are a few drawbacks of the plain |\include| mechanism:
\begin{itemize}
\item
The child files cannot be compiled on their own,
they can only be compiled via the main file.
A naive editing environment
(such as a text editor with an option
to have the current file processed by \LaTeX)
may require one to switch to the main file before compiling;
attempting to compile the child file produces errors.
\item
The main file must be modified (each time)
to adjust the |\includeonly| command
to the present needs. This easily leaves the main file in a messy state.
\item
The generated document will always carry the filename
of the main document. This is inconvenient if
several child files are to be compiled and
to be kept for distribution.
\end{itemize}

The present package provides a simple interface
to make child files individually compilable by \LaTeX{}.
Compiling a child file then has the same effect as compiling
the main file with an |\includeonly| command
to select the appropriate child.
Moreover the generated document will carry the name of the child
rather than the main file.
This resolves all three above issues.

This feature is meant to make the editing of books,
thesis documents and lecture notes somewhat more convenient.
However, the package can also be used efficiently for
composing a series of documents (such as exercise sheets)
which are typically distributed individually.
It then assists the author in generating the individual documents
(potentially in different versions)
as well as a document containing the collected series.
Another application is in developing style files
or other kinds of included material
where compilation of the style file could redirect
to a sample or test file.

%%%%%%%%%%%%%%%%%%%%%%%%%%%%%%%%%%%%%%%%%%%%%%%%%%%%%%%%%%%%%%%%%%%%%%%%%%%%%%%%
%%%%%%%%%%%%%%%%%%%%%%%%%%%%%%%%%%%%%%%%%%%%%%%%%%%%%%%%%%%%%%%%%%%%%%%%%%%%%%%%
\section{Usage}

First of all, the package \textsf{childdoc} is \emph{not} a standard
\LaTeXe{} |.sty| style file! Therefore it needs to be invoked in
a non-standard way.

%%%%%%%%%%%%%%%%%%%%%%%%%%%%%%%%%%%%%%%%%%%%%%%%%%%%%%%%%%%%%%%%%%%%%%%%%%%%%%%%
\subsection{Included Files}
\label{sec:include}

%%%%%%%%%%%%%%%%%%%%%%%%%%%%%%%%%%%%%%%%
\DescribeMacro{\childdocmain}
To use the package, add the commands
\begin{center}
\begin{tabular}{l}
|\input{childdoc.def}|\\
|\childdocmain{}|\\
\end{tabular}
\end{center}
at the very top of the main \LaTeX{} file,
in particular \emph{before} the |\documentclass| statement!
The argument of |\childdocmain| should be left empty
(but it must be present).

%%%%%%%%%%%%%%%%%%%%%%%%%%%%%%%%%%%%%%%%
\DescribeMacro{\childdocof}
Furthermore, add the commands
\begin{center}
\begin{tabular}{l}
|\input{childdoc.def}|\\
|\childdocof{|\textit{main}|}|\\
\end{tabular}
\end{center}
at the top of every child file \textit{child}
which is included by |\include{|\textit{child}|}|
from within the main file
(or at least for those files to be compiled individually).
The argument \textit{main} must be the filename of the main file.

There are a couple of
considerations in setting up the main and child documents:

%%%%%%%%%%%%%%%%%%%%%%%%%%%%%%%%%%%%%%%%
\paragraph{Restrictions.}

Please note the following restrictions:
\begin{itemize}
\item
|\childdocmain| must be called with one argument \textit{main}
to ensure compatibility with earlier version of the package.
It must either be empty (|\childdocmain{}|)
or precisely match the filename of the main file in which it is specified.
See \secref{sec:detection} for further information.
\item
The filename \textit{main} must be specified without the |.tex| extension.
\item
The filename \textit{main} is case sensitive
(even in case-insensitive file systems)
due to internal string comparison.
\item
The argument \textit{main} should be fully expanded, it cannot be a macro.
\item
Subdirectories and special characters should be avoided in filenames.
\item
The command |\childdocmain{|\textit{main}|}| must be followed by a whitespace.
It should not be followed immediately by another command
or by a comment mark `|%|'.
This is because the \TeX{} parser reads the token immediately following
the argument of |\childdocmain| and puts it
at the beginning of every child section;
however, a white\-space is ignored.
\end{itemize}

%%%%%%%%%%%%%%%%%%%%%%%%%%%%%%%%%%%%%%%%
\paragraph{Content of Main File.}

It is advisable to place all content in the child files included by |\include|.
Any output contained in the main file will appear in all child documents
unless suppressed manually;
it cannot be suppressed automatically by the |\includeonly| directive
and thus should normally be avoided.
A method to include some content in the main file
by means of conditional processing is described in \secref{sec:conditional}.

%%%%%%%%%%%%%%%%%%%%%%%%%%%%%%%%%%%%%%%%
\paragraph{Page Numbering.}

When only a part of the document is compiled,
the appropriate numbering of pages
(as well as other status parameters)
is determined from the |.aux| files.
The latter contain information from previous passes.
However this information needs to propagate through
all intermediate child documents.
Therefore the page numbering in child documents may well
be inconsistent until the complete document is compiled at least once.

A useful (if unconventional) way to always ensure a consistent
page numbering is to restart the numbering in each child document
and denote the pages by `\textit{child}|.|\textit{page}'
where \textit{child} represents the chapter/section number of the child file.
This can be achieved by the command
|\numberwithin{page}{|\textit{child}|}|
of the \textsf{amsmath} package
where \textit{child} can be |chapter| or |section|
depending on the chosen structuring.
Alternatively, one can modify the macro |\thepage| appropriately
and reset the counter |page| at the start of each child file.

%%%%%%%%%%%%%%%%%%%%%%%%%%%%%%%%%%%%%%%%%%%%%%%%%%%%%%%%%%%%%%%%%%%%%%%%%%%%%%%%
\subsection{Conditional Processing}
\label{sec:conditional}

The package provides a mechanism to compile different versions
of a document. To customise the versions further some conditional processing
can come in handy to distinguish which version is being compiled.
The package provides two macros to describe the compilation context:

%%%%%%%%%%%%%%%%%%%%%%%%%%%%%%%%%%%%%%%%
\DescribeMacro{\ifchilddoc}
The conditional |\ifchilddoc| distinguishes between the compilation of
child documents and the main document:
%
\begin{center}
|\ifchilddoc |\textit{child-code}| |[|\||else |\textit{main-code}]| \||fi|
\end{center}

%%%%%%%%%%%%%%%%%%%%%%%%%%%%%%%%%%%%%%%%
\DescribeMacro{\childdocname}
\DescribeMacro{\childdocjob}
The macro |\childdocname| contains the filename (without extension)
of the main or child file being processed.
Note that |\childdocjob| will always contain the name of the main file.

%%%%%%%%%%%%%%%%%%%%%%%%%%%%%%%%%%%%%%%%
\paragraph{Title Page.}

Conditional processing can be used to include a title or banner page
in the main document when proper precautions are taken.
Importantly, the code in the main file should ensure that the page counter
(as well as other status parameters which are stored in the |.aux| files)
takes the same value after the conditional processing.
Otherwise the page numbers may take divergent values
depending on which part is compiled.

For example, a title page could be declared by:
%
\begin{center}
\begin{tabular}{l}
|\ifchilddoc\||else|\\
|\addtocounter{page}{-1}|\\
\textit{code for title page}\\
|\newpage|\\
|\||fi|
\end{tabular}
\end{center}
%
A banner page for the child documents can be generated by:
%
\begin{center}
\begin{tabular}{l}
|\ifchilddoc|\\
|\addtocounter{page}{-1}|\\
\textit{code for banner page}\\
|\newpage|\\
|\||fi|
\end{tabular}
\end{center}
%
Here one could write a message such as:
\begin{center}
|This is the part \childdocname{} of \childdocjob{}.|
\end{center}

%%%%%%%%%%%%%%%%%%%%%%%%%%%%%%%%%%%%%%%%%%%%%%%%%%%%%%%%%%%%%%%%%%%%%%%%%%%%%%%%
\subsection{Flags}
\label{sec:flags}

The package makes it easy to generate different versions
of the main or child documents.
To this end compilation flags can be defined
and assigned different default values.
They will be particularly useful in conjunction
with the forwarding mechanism described in \secref{sec:forward}.

For example, it may be useful to have a flag |\version|
which can be set to |draft| or |final|.
The document source will contain some conditional code
depending on the value of |\version|.
Suppose further, the flag should default to |final| for the main file
and to |draft| for child files
which is a natural assignment for editing the document.
This is achieved by placing the following code
in the preamble of the main document
(below the |\childdocmain| directive):
%
\begin{center}
\begin{tabular}{l}
|\ifchilddoc|\\
|\providecommand{\version}{draft}|\\
|\||else|\\
|\providecommand{\version}{final}|\\
|\||fi|
\end{tabular}
\end{center}
%
The definition by |\providecommand| makes sure
that previous definitions are not overwritten.
Further statements |\providecommand{\version}{...}|
can thus be added before the above code to override it.

For the main file, one might add a line
(between |\childdocmain| and the above block)
%
\begin{center}
|%\ifchilddoc\||else\providecommand{\version}{draft}\||fi|
\end{center}
%
which can be uncommented to produce a draft version.
Likewise one can add a line to the very top of a child file
(above the |\childdocof{|\textit{main}|}| directive)
%
\begin{center}
|%\providecommand{\version}{final}|
\end{center}
%
which can be uncommented to produce the final version of this child document.

%%%%%%%%%%%%%%%%%%%%%%%%%%%%%%%%%%%%%%%%%%%%%%%%%%%%%%%%%%%%%%%%%%%%%%%%%%%%%%%%
\subsection{Forwarding}
\label{sec:forward}

Different versions of the main or child documents
using compilation flags as described in \secref{sec:flags}
can be (permanently) stored in different files
for convenient compilation, viewing and distribution.
To this end, the package defines a command
to pass on compilation to a different file:

%%%%%%%%%%%%%%%%%%%%%%%%%%%%%%%%%%%%%%%%
\DescribeMacro{\childdocforward}
The command |\childdocforward| redirects processing to
another source file:
%
\begin{center}
\begin{tabular}{l}
|\input{childdoc.def}|\\
|\childdocforward[|\textit{main}|]{|\textit{dest}|}|\\
\end{tabular}
\end{center}
%
The argument \textit{dest} is the destination file
(without extension).
It should be the main file or one of the child files.
Note that further \textsf{childdoc} directives
such as |\childdocof| and |\childdocforward|
in the indicated file will be processed in this form.
The optional argument \textit{main}
passes on directly to the main file \textit{main}
while pretending to compile the child \textit{dest}.
This form behaves as if \textit{dest}
issues |\childdocof{|\textit{main}|}| right away,
and no further \textsf{childdoc} directives will be processed.

%%%%%%%%%%%%%%%%%%%%%%%%%%%%%%%%%%%%%%%%
\DescribeMacro{\...prefix}
In the alternative form |\childdocforwardprefix|,
%
\begin{center}
\begin{tabular}{l}
|\input{childdoc.def}|\\
|\childdocforwardprefix[|\textit{main}|]{|\textit{prefix}|}{|\textit{dest}|}|
\end{tabular}
\end{center}
%
the destination file is determined by a pattern
depending on the current file:
To make this work, the current file must be called
`{\textit{prefix}\hspace{0.2em}\textit{suffix}}'
with \textit{prefix} matching precisely the argument.
Processing is then passed on to the file
`{\textit{dest}\hspace{0.2em}\textit{suffix}}'.
Surely, the same effect is achieved by
directly specifying the
argument `{\textit{dest}\hspace{0.2em}\textit{suffix}}'
in the first form.
However, that requires to set up a different file
for each child. With the alternative form of the command
all these files can have exactly the same content
which simplifies setting them up and maintaining them.

For example, the following file |draft.tex|
with a compilation flag |\version| as described in \secref{sec:flags}
compiles the main document as a draft:
%
\begin{center}
\begin{tabular}{l}
|\def\version{draft}|\\
|\input{childdoc.def}|\\
|\childdocforward{|\textit{main}|}|
\end{tabular}
\end{center}
%
Likewise, the following files |final|\textit{nn}|.tex|
compile the final version of the child document
|child|\textit{nn}|.tex|:
%
\begin{center}
\begin{tabular}{l}
|\def\version{final}|\\
|\input{childdoc.def}|\\
|\childdocforwardprefix{final}{child}|
\end{tabular}
\end{center}
%

Note that when several versions of a main file and/or of each child file
are to be generated, it may be convenient to set up a |Makefile| or
shell script to automatise the process.

%%%%%%%%%%%%%%%%%%%%%%%%%%%%%%%%%%%%%%%%%%%%%%%%%%%%%%%%%%%%%%%%%%%%%%%%%%%%%%%%
\subsection{Command Line Processing}
\label{sec:commandline}

The effect of redirection files can also be achieved by invoking
the \LaTeX{} compiler with a more elaborate command line.
Most conveniently this should be done as part
of a shell script or a |Makefile|.

When using \textsf{childdoc} in the main file, the following
command lines effectively perform a redirection
(note that depending on the shell being used,
backslashes may have to be doubled: `|\|' $\to$ `|\\|'):
%
\begin{center}
|... -jobname "|\textit{target}|" |\\|"|[\textit{flags}]%
|\input{childdoc.def}\childdocforward[|\textit{main}|]{|\textit{dest}|}"|
\end{center}
%
Here \textit{target} is the name of the output file,
\textit{main} is the name of the main file
and \textit{dest} is the name of the main or child file to be processed
(all filenames without extensions).
The optional argument \textit{main} can be omitted
if \textit{main} matches \textit{dest}.
Optionally, compilation \textit{flags} can be defined via |\def| commands.
This command line makes the \TeX{} engine believe
it is compiling the file \textit{target}
whose content is specified as the latter parameter.
The provided code then forwards the processing to
\textit{main} or \textit{dest} as described in \secref{sec:forward}.

%%%%%%%%%%%%%%%%%%%%%%%%%%%%%%%%%%%%%%%%%%%%%%%%%%%%%%%%%%%%%%%%%%%%%%%%%%%%%%%%
\subsection{Include by Input}
\label{sec:input}

Including child documents by |\include| has some restrictions by design.
Most notably, the content of a child document always occupies
its own set of pages; pages cannot be shared between child documents.
Usually, this behaviour makes perfect sense
because each child document contain an essential part of the document.
However, in some situations it may be desirable to compose
a document from a collection of parts
without having mandatory page breaks between then.
For this case, the package
provides a mechanism to include parts
by |\input| which can also be processed individually.
However, by construction this mechanism
requires manual handling of the content to be output.

%%%%%%%%%%%%%%%%%%%%%%%%%%%%%%%%%%%%%%%%
\DescribeMacro{\ifchilddocmanual}
The main file should be prepared as usual, see \secref{sec:include}.
However, the document body must make a distinction
between processing of an individual part and of the main document, e.g.:
%
\begin{center}
\begin{tabular}{l}
|\ifchilddocmanual|\\
|\input{\childdocname}|\\
|\||else|\\
\textit{document body with }|\input{|\textit{part}|}|\\
|\||fi|
\end{tabular}
\end{center}
%
The conditional |\ifchilddocmanual| is true whenever
a part to be included by |\input| is being compiled,
and the name of the part is stored in |\childdocname|.

%%%%%%%%%%%%%%%%%%%%%%%%%%%%%%%%%%%%%%%%
\DescribeMacro{\childdocby}
Each part to be included by |\input| should start with:
%
\begin{center}
\begin{tabular}{l}
|\input{childdoc.def}|\\
|\childdocby{|\textit{main}|}|\\
\end{tabular}
\end{center}
%
The directive |\childdocby| is similar to |\childdocof|
described in \secref{sec:include},
but the subsequent selection of content must be done manually.
To that end, both |\ifchilddoc| and |\ifchilddocmanual|
will be true upon processing of a part,
and the name of the part is stored in |\childdocname|.
Note that |\jobname| will be set to the filename of the current part
so that each part receives an individual |.aux| file
that does not interfere with the |.aux| file(s) of the main document.
This behaviour can be altered by the alternative form
|\childdocby[*]{|\textit{main}|}| (with a non-empty optional argument)
which uses the |.aux| file of the main document
by setting |\jobname| to \textit{main}.

%%%%%%%%%%%%%%%%%%%%%%%%%%%%%%%%%%%%%%%%%%%%%%%%%%%%%%%%%%%%%%%%%%%%%%%%%%%%%%%%
\subsection{Driver Development}
\label{sec:driver}

The \textsf{childdoc} mechanism can also be use for the development
of definition files such as \LaTeX{} styles or classes.
This case differs from the above setup with multiple parts
included by |\include| in that no |\includeonly| should be invoked.
This can be achieved by starting the include file
(before |\ProvidesPackage|) with:
%
\begin{center}
\begin{tabular}{l}
|\input{childdoc.def}|\\
|\childdocforward{|\textit{main}|}|\\
\end{tabular}
\end{center}
%
or alternatively with:
%
\begin{center}
\begin{tabular}{l}
|\input{childdoc.def}|\\
|\childdocby{|\textit{main}|}|\\
\end{tabular}
\end{center}
%
Both forms have slightly different effects as described above.
The main file is prepared as usual, see \secref{sec:include}.

%%%%%%%%%%%%%%%%%%%%%%%%%%%%%%%%%%%%%%%%%%%%%%%%%%%%%%%%%%%%%%%%%%%%%%%%%%%%%%%%
\subsection{Legacy Detection}
\label{sec:detection}

The directive |\childdocmain| in the main file can detect
whether the complete document or merely a child is to be compiled
even without using the directive |\childdocof|.
This method is deprecated because it is less robust
and there is no compelling reason to use it;
it is merely provided for backward compatibility
and it may be removed in future versions.

If the detection mechanism is to be used,
it is mandatory to correctly specify
the filename of the main file as the argument of |\childdocmain|:
%
\begin{center}
\begin{tabular}{l}
|\input{childdoc.def}|\\
|\childdocmain{|\textit{main}|}|\\
\end{tabular}
\end{center}
%
If |\jobname| does not match the argument \textit{main} of |\childdocmain|,
it is assumed that |\jobname| points to the child file to be compiled.
When using |\childdocmain| with the main file specified as argument,
it suffices to start a child file
with just |\input{|\textit{main}|}|
without loading of the package and using |\childdocof|.
If instead all processing is done
with the appropriate \textsf{childdoc} directives,
the argument of \textit{main} of |\childdocmain| can be empty.

An alternative version of the command line processing described
in \secref{sec:commandline} using the detection mechanism reads:
%
\begin{center}
|... -jobname "|\textit{target}|" "|[\textit{flags}]%
[|\def\jobname{|\textit{dest}|}|]|\input{|\textit{main}|}"|
\end{center}

%%%%%%%%%%%%%%%%%%%%%%%%%%%%%%%%%%%%%%%%%%%%%%%%%%%%%%%%%%%%%%%%%%%%%%%%%%%%%%%%
\subsection{Manual Code}
\label{sec:manual}

In case one cannot be certain whether the definitions file |childdoc.def|
is installed on the target \TeX{} distribution
and one prefers not to ship it,
it is conceivable to paste a few relevant commands into the sources.

To that end, drop all statements |\input{childdoc.def}|
and perform the replacements as outlined below.
Instead of |\childdocmain{|\textit{main}|}| add the following code
to the top of the main file:
%
\begin{center}
\begin{tabular}{l}
|\||ifdefined\childdocname\endinput\||fi\newif\ifchilddoc|\\
|\edef\childdocname{\scantokens\expandafter{\jobname\noexpand}}|\\
|\def\childdocmain{|\textit{main}|}\||ifx\childdocmain\childdocname\||else|\\
|\childdoctrue\includeonly{\childdocname}\let\jobname\childdocmain\||fi|\\
\end{tabular}
\end{center}
%
Instead of |\childdocof{|\textit{main}|}| just include the main file
at the top of each child file:
%
\begin{center}
|\input{|\textit{main}|}|
\end{center}
%
A simple redirection |\childdocforward{|\textit{dest}|}| is achieved by:
%
\begin{center}
|\def\jobname{|\textit{dest}|}\input{\jobname}|
\end{center}
%
The redirection with prefix
|\childdocforwardprefix[|\textit{prefix}|]{|\textit{dest}|}|
is accomplished by:
%
\begin{center}
\begin{tabular}{l}
|{\edef\jobname{\scantokens\expandafter{\jobname\noexpand}}|\\
|\def\redirectjob |\textit{prefix}|#1~~~{\gdef\jobname{|\textit{dest}|#1}}|\\
|\expandafter\redirectjob\jobname~~~}\input{\jobname}|
\end{tabular}
\end{center}

In an alternative approach,
child documents can be compiled by a specific command line
without additional code or specific definitions:
%
\begin{center}
|... -jobname "|\textit{target}|" "|[\textit{flags}]%
|\includeonly{|\textit{dest}|}\input{|\textit{main}|}"|
\end{center}
%

%%%%%%%%%%%%%%%%%%%%%%%%%%%%%%%%%%%%%%%%%%%%%%%%%%%%%%%%%%%%%%%%%%%%%%%%%%%%%%%%
%%%%%%%%%%%%%%%%%%%%%%%%%%%%%%%%%%%%%%%%%%%%%%%%%%%%%%%%%%%%%%%%%%%%%%%%%%%%%%%%
\section{Information}

%%%%%%%%%%%%%%%%%%%%%%%%%%%%%%%%%%%%%%%%%%%%%%%%%%%%%%%%%%%%%%%%%%%%%%%%%%%%%%%%
\subsection{Copyright}

Copyright \copyright{} 2017--2018 Niklas Beisert

This work may be distributed and/or modified under the
conditions of the \LaTeX{} Project Public License, either version 1.3
of this license or (at your option) any later version.
The latest version of this license is in
  \url{http://www.latex-project.org/lppl.txt}
and version 1.3 or later is part of all distributions of \LaTeX{}
version 2005/12/01 or later.

This work has the LPPL maintenance status `maintained'.

The Current Maintainer of this work is Niklas Beisert.

This work consists of the files |README.txt|, |childdoc.ins| and |childdoc.dtx|
as well as the derived files |childdoc.def|, |cdocsamp.tex|
with |cdocsch1.tex|, |cdocsch2.tex|, |cdocspt3.tex|, |cdocspt4.tex|,
|cdocsdrf.tex|, |cdocsfn1.tex|, |cdocsfn2.tex|
as well as |childdoc.pdf|.

%%%%%%%%%%%%%%%%%%%%%%%%%%%%%%%%%%%%%%%%%%%%%%%%%%%%%%%%%%%%%%%%%%%%%%%%%%%%%%%%
\subsection{Files and Installation}

The package consists of the files:
%
\begin{center}
\begin{tabular}{ll}
    |README.txt|   & readme file \\
    |childdoc.ins| & installation file \\
    |childdoc.dtx| & source file \\
    |childdoc.def| & definition file \\
    |cdocsamp.tex| & sample main file \\
    |cdocsch1.tex| & sample include file \\
    |cdocsch2.tex| & sample include file \\
    |cdocspt3.tex| & sample part file \\
    |cdocspt4.tex| & sample part file \\
    |cdocsdrf.tex| & sample redirection file \\
    |cdocsfn1.tex| & sample redirection file \\
    |cdocsfn2.tex| & sample redirection file \\
    |childdoc.pdf| & manual
\end{tabular}
\end{center}
%
The distribution consists of the files
|README.txt|, |childdoc.ins| and |childdoc.dtx|.
%
\begin{itemize}
\item
Run (pdf)\LaTeX{} on |childdoc.dtx|
to compile the manual |childdoc.pdf| (this file).
\item
Run \LaTeX{} on |childdoc.ins| to create the definitions file |childdoc.def|
and the sample |cdocsamp.tex| with include files
|cdocsch1.tex|, |cdocsch2.tex|, |cdocspt3.tex|, |cdocspt4.tex|,
|cdocsdrf.tex|, |cdocsfn1.tex|, |cdocsfn2.tex|.
Then copy the file |childdoc.def| to an appropriate directory of your \LaTeX{}
distribution, e.g.\ \textit{texmf-root}|/tex/latex/childdoc|.
\end{itemize}

%%%%%%%%%%%%%%%%%%%%%%%%%%%%%%%%%%%%%%%%%%%%%%%%%%%%%%%%%%%%%%%%%%%%%%%%%%%%%%%%
\subsection{Related CTAN Packages}

There are several other packages which offer a similar functionality:
%
\begin{itemize}
\item
The packages
\href{http://ctan.org/pkg/docmute}{\textsf{docmute}},
\href{http://ctan.org/pkg/includex}{\textsf{includex}} and
\href{http://ctan.org/pkg/standalone}{\textsf{standalone}}
provide commands to include only the document body of
a child file thus allowing both files to be compiled individually.
\item
The packages \href{http://ctan.org/pkg/subdocs}{\textsf{subdocs}}
and \href{http://ctan.org/pkg/subfiles}{\textsf{subfiles}}
provide structures in which the main and child documents can be
encapsulated and allowing them to be compiled individually.
The inclusion mechanism is different from the conventional |\include|.
\item
The package \href{http://ctan.org/pkg/combine}{\textsf{combine}}
is an elaborate solution to combine several documents into one.
\end{itemize}
%
See also the CTAN topic \href{http://ctan.org/topic/subdocs}{\textsf{subdocs}}
for further related packages.
The present package differs from the above solutions in that
a document structure constructed with the conventional |\include| mechanism
just needs two extra commands at the top of every file
such that all constituent files can be compiled individually.

%%%%%%%%%%%%%%%%%%%%%%%%%%%%%%%%%%%%%%%%%%%%%%%%%%%%%%%%%%%%%%%%%%%%%%%%%%%%%%%%
%\subsection{Feature Suggestions}
%
%The following is a list of features which may be useful for future
%versions of this package:
%%
%\begin{itemize}
%\item
%\ldots
%\end{itemize}

%%%%%%%%%%%%%%%%%%%%%%%%%%%%%%%%%%%%%%%%%%%%%%%%%%%%%%%%%%%%%%%%%%%%%%%%%%%%%%%%
\subsection{Revision History}

%%%%%%%%%%%%%%%%%%%%%%%%%%%%%%%%%%%%%%%%
\paragraph{v2.0:} 2018/12/30

\begin{itemize}
\item
immediate forward processing
\item
added |\childdocby| mechanism
\item
manual restructured
\end{itemize}

%%%%%%%%%%%%%%%%%%%%%%%%%%%%%%%%%%%%%%%%
\paragraph{v1.6:} 2018/01/17

\begin{itemize}
\item
application for development of include files
\item
corrections to manual
\end{itemize}

%%%%%%%%%%%%%%%%%%%%%%%%%%%%%%%%%%%%%%%%
\paragraph{v1.5:} 2017/05/21

\begin{itemize}
\item
more complete structuring introduced
\item
|\childdocof| introduced
\item
|\childdoc| renamed to |\childdocmain|
\item
|\childredirect| renamed to |\childdocforward| and |\childdocforwardprefix|
and functionality expanded
\end{itemize}

%%%%%%%%%%%%%%%%%%%%%%%%%%%%%%%%%%%%%%%%
\paragraph{v1.0:} 2017/04/27

\begin{itemize}
\item
manual and install package
\item
first version published on CTAN
\end{itemize}

%%%%%%%%%%%%%%%%%%%%%%%%%%%%%%%%%%%%%%%%
\paragraph{v0.6:} 2017/04/26

\begin{itemize}
\item
redirection mechanism added
\end{itemize}

%%%%%%%%%%%%%%%%%%%%%%%%%%%%%%%%%%%%%%%%
\paragraph{v0.5:} 2017/04/26

\begin{itemize}
\item
functionality in definition file
\end{itemize}


%%%%%%%%%%%%%%%%%%%%%%%%%%%%%%%%%%%%%%%%%%%%%%%%%%%%%%%%%%%%%%%%%%%%%%%%%%%%%%%%
%%%%%%%%%%%%%%%%%%%%%%%%%%%%%%%%%%%%%%%%%%%%%%%%%%%%%%%%%%%%%%%%%%%%%%%%%%%%%%%%
%%%%%%%%%%%%%%%%%%%%%%%%%%%%%%%%%%%%%%%%%%%%%%%%%%%%%%%%%%%%%%%%%%%%%%%%%%%%%%%%
\appendix

\settowidth\MacroIndent{\rmfamily\scriptsize 000\ }

 \DocInput{childdoc.dtx}

\end{document}
%</driver>
% \fi
%
% %%%%%%%%%%%%%%%%%%%%%%%%%%%%%%%%%%%%%%%%%%%%%%%%%%%%%%%%%%%%%%%%%%%%%%%%%%%%%%
% %%%%%%%%%%%%%%%%%%%%%%%%%%%%%%%%%%%%%%%%%%%%%%%%%%%%%%%%%%%%%%%%%%%%%%%%%%%%%%
% \section{Sample}
%\iffalse
%<*samplemain>
%\fi
%
% The following presents a sample document
% with two chapters, two parts, a title page,
% a compile flag as well as three forwarding files to set the flag.
% It consists of eight |.tex| files:
% \begin{center}
% \begin{tabular}{ll}
% |cdocsamp.tex|&main file\\
% |cdocsch1.tex|&include file for chapter 1\\
% |cdocsch2.tex|&include file for chapter 2\\
% |cdocspt3.tex|&include file for part 3\\
% |cdocspt4.tex|&include file for part 4\\
% |cdocsdrf.tex|&forwarding file for main file in draft mode\\
% |cdocsfi1.tex|&forwarding file for final version of chapter 1\\
% |cdocsfi2.tex|&forwarding file for final version of chapter 2\\
% \end{tabular}
% \end{center}
% Each of the eight files can be compiled directly by the \LaTeX{} compiler.
%
% %%%%%%%%%%%%%%%%%%%%%%%%%%%%%%%%%%%%%%
% \paragraph{Main File.}
%
% The main file is called |cdocsamp.tex|.
%
% Load the \textsf{childdoc} definitions and
% declare the filename for the main document:
%    \begin{macrocode}
\input{childdoc.def}
\childdocmain{}
%    \end{macrocode}

% Optional override for |\version| flag:
%    \begin{macrocode}
%%\ifchilddoc\else\providecommand{\version}{draft}\fi
%    \end{macrocode}

% Define the default values for the |\version| flag
% (|final| for the main file and |draft| for childs):
%    \begin{macrocode}
\ifchilddoc
\providecommand{\version}{draft}
\else
\providecommand{\version}{final}
\fi
%    \end{macrocode}

% Load the standard document class:
%    \begin{macrocode}
\documentclass[12pt]{article}
%    \end{macrocode}

% Start the document body:
%    \begin{macrocode}
\begin{document}
%    \end{macrocode}

% Declare a title page.
% Print title, part of document being processed and version flag:
%    \begin{macrocode}
\addtocounter{page}{-1}
\begin{center}
{\LARGE\bfseries{}childdoc example\par}
\vspace{1cm}
\ifchilddoc
\ifchilddocmanual part\else chapter\fi:
`\childdocname' of `\childdocjob'\par
\else
main document: `\childdocjob'\par
\fi
version: \version\par
\end{center}
\newpage
%    \end{macrocode}

% Manually include selected file,
% otherwise process as usual:
%    \begin{macrocode}
\ifchilddocmanual
\section*{part `\childdocname'}
\input{\childdocname}
\else
%    \end{macrocode}

% Include the two chapters:
%    \begin{macrocode}
\include{cdocsch1}
\include{cdocsch2}
%    \end{macrocode}

% Include the two parts unless only chapters should be displayed:
%    \begin{macrocode}
\ifchilddoc\else
\section{part three}
\input{cdocspt3}
\section{part four}
\input{cdocspt4}
\fi
%    \end{macrocode}

% Process as usual until here:
%    \begin{macrocode}
\fi
%    \end{macrocode}

% End of document body:
%    \begin{macrocode}
\end{document}
%    \end{macrocode}
%\iffalse
%</samplemain>
%\fi
%
% %%%%%%%%%%%%%%%%%%%%%%%%%%%%%%%%%%%%%%
% \paragraph{Chapter Include Files.}
%
% The include files are called |cdocsch1.tex| and |cdocsch2.tex|.
%
%\iffalse
%<*samplechap1|samplechap2>
%\fi

% Optional override for |\version| flag:
%    \begin{macrocode}
%%\providecommand{\version}{final}
%    \end{macrocode}

% Include the main document:
%    \begin{macrocode}
\input{childdoc.def}
\childdocof{cdocsamp}
%    \end{macrocode}

%\iffalse
%</samplechap1|samplechap2>
%\fi
%
%\iffalse
%<*samplechap1>
%\fi
% Some text for chapter 1:
%    \begin{macrocode}
\section{one}
some text in chapter one
%    \end{macrocode}

%\iffalse
%</samplechap1>
%\fi
% Some text for chapter 2:
%\iffalse
%<*samplechap2>
%\fi
%    \begin{macrocode}
\section{two}
more text in chapter two
%    \end{macrocode}

%\iffalse
%</samplechap2>
%\fi
%
% %%%%%%%%%%%%%%%%%%%%%%%%%%%%%%%%%%%%%%
% \paragraph{Part Include Files.}
%
% The include files are called |cdocspt3.tex| and |cdocspt4.tex|.
%
%\iffalse
%<*samplepart3|samplepart4>
%\fi

% Optional override for |\version| flag:
%    \begin{macrocode}
%%\providecommand{\version}{final}
%    \end{macrocode}

% Include the main document:
%    \begin{macrocode}
\input{childdoc.def}
\childdocby{cdocsamp}
%    \end{macrocode}

%\iffalse
%</samplepart3|samplepart4>
%\fi
%
%\iffalse
%<*samplepart3>
%\fi
% Some text for part 3:
%    \begin{macrocode}
some text in part three
%    \end{macrocode}

%\iffalse
%</samplepart3>
%\fi
% Some text for part 4:
%\iffalse
%<*samplepart4>
%\fi
%    \begin{macrocode}
more text in part four
%    \end{macrocode}

%\iffalse
%</samplepart4>
%\fi
%
% %%%%%%%%%%%%%%%%%%%%%%%%%%%%%%%%%%%%%%
% \paragraph{Forwarding for a Complete Draft.}
%
% The following forwarding file |cdocsdrf.tex|
% compiles the main document in draft mode:
%\iffalse
%<*sampledraft>
%\fi
%    \begin{macrocode}
\def\version{draft}
\input{childdoc.def}
\childdocforward{cdocsamp}
%    \end{macrocode}

%\iffalse
%</sampledraft>
%\fi
%
% %%%%%%%%%%%%%%%%%%%%%%%%%%%%%%%%%%%%%%
% \paragraph{Forwarding for Final Version of the Chapters.}
%
% The following forwarding files |cdocsfn1.tex| and |cdocsfn2.tex|
% (with identical content)
% compile the final versions of the child documents
% |cdocsch1.tex| and |cdocsch2.tex|, respectively:
%\iffalse
%<*samplefinal>
%\fi
%    \begin{macrocode}
\def\version{final}
\input{childdoc.def}
\childdocforwardprefix[cdocsamp]{cdocsfn}{cdocsch}
%    \end{macrocode}

%\iffalse
%</samplefinal>
%\fi
%
% %%%%%%%%%%%%%%%%%%%%%%%%%%%%%%%%%%%%%%
% \paragraph{Command Line Processing.}
%
% The following three command lines generate the output files
% |cdocscld|, |cdocscl1| and |cdocscl2|
% which should be identical to
% |cdocsdrf|, |cdocsch1| and |cdocsfn2|, respectively:
% \begin{center}
% \begin{tabular}{l}
% |latex -jobname cdocscld \|\\
% |  "\def\version{draft}\input{childdoc.def}\childdocforward{cdocsamp}"|\\
% |latex -jobname cdocscl1 \|\\
% |  "\input{childdoc.def}\childdocforward[cdocsamp]{cdocsch1}"|\\
% |latex -jobname cdocscl2 \|\\
% |  "\def\version{final}\input{childdoc.def}\childdocforward{cdocsch2}"|
% \end{tabular}
% \end{center}
% Note that the trailing backslash on each first line
% merely continues the input to the second line
% (for convenient cut ant paste).
% Furthermore, the command |latex| can be replaced by any
% of its alternative versions such as |pdflatex|.
%
% %%%%%%%%%%%%%%%%%%%%%%%%%%%%%%%%%%%%%%%%%%%%%%%%%%%%%%%%%%%%%%%%%%%%%%%%%%%%%%
% %%%%%%%%%%%%%%%%%%%%%%%%%%%%%%%%%%%%%%%%%%%%%%%%%%%%%%%%%%%%%%%%%%%%%%%%%%%%%%
% \section{Implementation}
%\iffalse
%<*package>
%\fi
%
% This section describes the definitions file |childdoc.def|.

% The definitions cannot be loaded using |\usepackage| or |\RequirePackage|
% which has a mechanism to prevent loading a style file more than once.
% When loading the definitions by means of |\input|
% multiple instances have to be prevented manually:
%\iffalse
%This code needs to be before the `\ProvidesFile' directive
%which is defined at the beginning of this file.
%Therefore it is also placed there and commented out here.
%</package>
%<*discard>
%\fi
%    \begin{macrocode}
\ifdefined\childdocmain\endinput\fi
%    \end{macrocode}
%\iffalse
%</discard>
%<*package>
%\fi
%
% \macro{\ifchilddoc}
% \macro{\ifchilddocmanual}
% The conditional |\ifchilddoc| tells whether a
% child (true) or main (false) document is being compiled.
% The conditional |\ifchilddocmanual| tells whether
% the |\includeonly| mechanism is used (false) or
% the selection of child files must be performed manually (true).
% The definitions initialise to false:
%    \begin{macrocode}
\newif\ifchilddoc
\newif\ifchilddocmanual
%    \end{macrocode}

% \macro{\childdocname}
% \macro{\childdocjob}
% The macro |\childdocname| stores the name of the main document
% to be compiled. The macro |\childdocjob| stores the name of
% the document on which the \LaTeX{} compiler was originally invoked.
% The content of |\jobname| cannot be compared
% to filenames specified in the source due to different catcodes.
% The following code rescans |\jobname|, stores the result
% in |\childdocname| and saves a copy in |\childdocjob|:
%    \begin{macrocode}
\edef\childdocname{\scantokens\expandafter{\jobname\noexpand}}
\let\childdocjob\childdocname
%    \end{macrocode}

% \macro{\childdocdisable}
% The macro |\childdocdisable| prevents the main file
% from being processed more than once.
% At this stage, the main document command |\childdocmain|
% is assumed to be called once again where it should do nothing.
% Any subsequent call to it should prevent
% a secondary processing of the main document
% It overwrites the forwarding commands
% |\childdocof| and |\childdocforward|
% with empty macros to prevent further inclusions of the main document:
%    \begin{macrocode}
\newcommand{\childdocdisable}
{
  \renewcommand{\childdocmain}[1]{\renewcommand{\childdocmain}[1]{\endinput}}
  \renewcommand{\childdocof}[1]{}
  \renewcommand{\childdocby}[2][]{}
  \renewcommand{\childdocforward}[2][]{}
  \renewcommand{\childdocdisable}{}
}
%    \end{macrocode}

% \macro{\childdocmain}
% The macro |\childdocmain| is to be called at the top of the main file
% with nothing or the main filename (without extension) as argument.
% First, it breaks loops.
% If the argument is not empty and does not match |\childdocname|
% (which is set by the first inclusion of |childdoc.def|),
% |\ifchilddoc| is set to true, |\includeonly| is applied to the child file
% and |\jobname| is set to the main file
% (for proper handling of |.aux| files):
%    \begin{macrocode}
\newcommand{\childdocmain}[1]
{
  \childdocdisable\childdocmain{}
  \if?#1?\else
    \begingroup
      \def\childdoctmp{#1}
      \ifx\childdoctmp\childdocname
        \def\childdoctmp{}
      \else
        \def\childdoctmp
        {
          \childdoctrue
          \includeonly{\childdocname}
          \def\childdocjob{#1}
          \def\jobname{#1}
        }
      \fi
      \expandafter
    \endgroup
    \childdoctmp
  \fi
}
%    \end{macrocode}

% \macro{\childdocof}
% The command |\childdocof| redirects
% compilation to the main file |#1|.
%    \begin{macrocode}
\newcommand{\childdocof}[1]
{
  \childdocdisable
  \childdoctrue
  \includeonly{\childdocname}
  \def\jobname{#1}
  \def\childdocjob{#1}
  \input{#1}
}
%    \end{macrocode}

% \macro{\childdocby}
% The command |\childdocby| ....
%    \begin{macrocode}
\newcommand{\childdocby}[2][]
{
  \childdocdisable
  \childdoctrue
  \childdocmanualtrue
  \if?#1?\else
    \def\jobname{#2}
  \fi
  \def\childdocjob{#2}
  \input{#2}
  \endinput
}
%    \end{macrocode}

% \macro{\childdocforward}
% The command |\childdocforward| redirects
% compilation to the main file or
% (if the optional argument is given) a child file.
% Parameters are set as if the main file
% or a child file starting with |\childdocof| was compiled.
% Then compilation is handed over to the main file:
%    \begin{macrocode}
\newcommand{\childdocforward}[2][]
{
  \begingroup
    \if?#1?
      \def\childdoctmp
      {
        \def\childdocname{#2}
        \def\childdocjob{#2}
        \def\jobname{#2}
        \input{#2}
        \endinput
      }
    \else
      \def\childdoctmp
      {
        \childdocdisable
        \def\childdocname{#2}
        \childdoctrue
        \includeonly{#2}
        \def\childdocjob{#1}
        \def\jobname{#1}
        \input{#1}
        \endinput
      }
    \fi
    \expandafter
  \endgroup
  \childdoctmp
}
%    \end{macrocode}

% \macro{\childdocforwardprefix}
% The command |\childdocforwardprefix| redirects
% compilation to the main or a child file by means of a pattern.
% The prefix |#1| in the current filename is replaced by |#2|
% and the suffix of the current filename is kept
% (it is assumed that the filename does not contain the substring `|~~~|'
% which is used as a delimiter).
% Compilation is handed over to the new file by |\childdocforward|:
%    \begin{macrocode}
\newcommand{\childdocforwardprefix}[3][]
{
  \begingroup
    \def\childdocextract #2##1~~~{\def\childdoctmp{\childdocforward[#1]{#3##1}}}
    \expandafter\childdocextract\childdocname~~~
    \expandafter
  \endgroup
  \childdoctmp
}
%    \end{macrocode}

% \macro{\childdoc}
% The deprecated macro |\childdoc| is a legacy version of |\childdocmain|:
%    \begin{macrocode}
\newcommand{\childdoc}{\childdocmain}
%    \end{macrocode}

% \macro{\childdocredirect}
% The deprecated macro |\childdocredirect| is a legacy version
% of |\childdocforward| and |\childdocforwardprefix|:
%    \begin{macrocode}
\newcommand{\childdocredirect}[2][]
{
  \begingroup
    \if?#1?
      \def\childdoctmp{\childdocforward{#2}}
    \else
      \def\childdoctmp{\childdocforwardprefix{#1}{#2}}
    \fi
    \expandafter
  \endgroup
  \childdoctmp
}
%    \end{macrocode}

%\iffalse
%</package>
%\fi
%
\endinput

\childdocforward{cdocsamp}
%    \end{macrocode}

%\iffalse
%</sampledraft>
%\fi
%
% %%%%%%%%%%%%%%%%%%%%%%%%%%%%%%%%%%%%%%
% \paragraph{Forwarding for Final Version of the Chapters.}
%
% The following forwarding files |cdocsfn1.tex| and |cdocsfn2.tex|
% (with identical content)
% compile the final versions of the child documents
% |cdocsch1.tex| and |cdocsch2.tex|, respectively:
%\iffalse
%<*samplefinal>
%\fi
%    \begin{macrocode}
\def\version{final}
% \iffalse
%
% childdoc.dtx Copyright (C) 2017-2018 Niklas Beisert
%
% This work may be distributed and/or modified under the
% conditions of the LaTeX Project Public License, either version 1.3
% of this license or (at your option) any later version.
% The latest version of this license is in
%   http://www.latex-project.org/lppl.txt
% and version 1.3 or later is part of all distributions of LaTeX
% version 2005/12/01 or later.
%
% This work has the LPPL maintenance status `maintained'.
%
% The Current Maintainer of this work is Niklas Beisert.
%
% This work consists of the files childdoc.dtx and childdoc.ins
% and the derived files childdoc.def and cdocsamp.tex with
% cdocsch1.tex, cdocsch2.tex, cdocsdrf.tex, cdocsfn1.tex, cdocsfn2.tex.
%
%<package>\ifdefined\childdocmain\endinput\fi
%<package>\ProvidesFile{childdoc.def}[2018/12/30 v2.0 child document driver]
%<samplemain>\ProvidesFile{cdocsamp.tex}[2018/12/30 v2.0 sample for childdoc]
%<*driver>
%\ProvidesFile{childdoc.drv}[2018/12/30 v2.0 childdoc reference manual file]
\PassOptionsToClass{10pt,a4paper}{article}
\documentclass{ltxdoc}

\usepackage[margin=35mm]{geometry}
\usepackage{hyperref}
\usepackage{hyperxmp}
\usepackage[usenames]{color}

\hypersetup{colorlinks=true}
\hypersetup{pdfstartview=FitH}
\hypersetup{pdfpagemode=UseNone}
\hypersetup{pdfsource={}}
\hypersetup{pdflang={en-UK}}
\hypersetup{pdfcopyright={Copyright 2017-2018 Niklas Beisert.
  This work may be distributed and/or modified under the
  conditions of the LaTeX Project Public License, either version 1.3
  of this license or (at your option) any later version.}}
\hypersetup{pdflicenseurl={http://www.latex-project.org/lppl.txt}}
\hypersetup{pdfcontactaddress={ETH Zurich, ITP, HIT K,
  Wolfgang-Pauli-Strasse 27}}
\hypersetup{pdfcontactpostcode={8093}}
\hypersetup{pdfcontactcity={Zurich}}
\hypersetup{pdfcontactcountry={Switzerland}}
\hypersetup{pdfcontactemail={nbeisert@itp.phys.ethz.ch}}
\hypersetup{pdfcontacturl={http://people.phys.ethz.ch/\xmptilde nbeisert/}}

\newcommand{\secref}[1]{\hyperref[#1]{section \ref*{#1}}}

\parskip1ex
\parindent0pt
\let\olditemize\itemize
\def\itemize{\olditemize\parskip0pt}

\begin{document}

\title{The \textsf{childdoc} Package}
\hypersetup{pdftitle={The childdoc Package}}
\author{Niklas Beisert\\[2ex]
  Institut f\"ur Theoretische Physik\\
  Eidgen\"ossische Technische Hochschule Z\"urich\\
  Wolfgang-Pauli-Strasse 27, 8093 Z\"urich, Switzerland\\[1ex]
  \href{mailto:nbeisert@itp.phys.ethz.ch}
  {\texttt{nbeisert@itp.phys.ethz.ch}}}
\hypersetup{pdfauthor={Niklas Beisert}}
\hypersetup{pdfsubject={Manual for the LaTeX2e Package childdoc}}
\date{30 December 2018, \textsf{v2.0}}
\maketitle

\begin{abstract}\noindent
\textsf{childdoc} is a \LaTeXe{} package
that enables the direct compilation
of document sections included by |\include|
to individual files.
\end{abstract}

\begingroup
\parskip0ex
\tableofcontents
\endgroup

%%%%%%%%%%%%%%%%%%%%%%%%%%%%%%%%%%%%%%%%%%%%%%%%%%%%%%%%%%%%%%%%%%%%%%%%%%%%%%%%
%%%%%%%%%%%%%%%%%%%%%%%%%%%%%%%%%%%%%%%%%%%%%%%%%%%%%%%%%%%%%%%%%%%%%%%%%%%%%%%%
\section{Introduction}

\LaTeX{} provides a mechanism to structure a large document (such as a book)
into a main file and several child files (containing the chapters)
using the |\include| command.
This mechanism is beneficial for documents
which span hundreds of pages in order to
make the source file(s) more manageable.
Moreover, compilation can be restricted to
selected child files by means of the |\includeonly| command.
The latter feature can be used to reduce the compilation time while editing
(this was significantly more useful in the earlier days of \LaTeX{})
or to generate a smaller document which is easier to navigate.
Another application of |\includeonly| is to generate
documents consisting of selected parts of the complete document.

However, there are a few drawbacks of the plain |\include| mechanism:
\begin{itemize}
\item
The child files cannot be compiled on their own,
they can only be compiled via the main file.
A naive editing environment
(such as a text editor with an option
to have the current file processed by \LaTeX)
may require one to switch to the main file before compiling;
attempting to compile the child file produces errors.
\item
The main file must be modified (each time)
to adjust the |\includeonly| command
to the present needs. This easily leaves the main file in a messy state.
\item
The generated document will always carry the filename
of the main document. This is inconvenient if
several child files are to be compiled and
to be kept for distribution.
\end{itemize}

The present package provides a simple interface
to make child files individually compilable by \LaTeX{}.
Compiling a child file then has the same effect as compiling
the main file with an |\includeonly| command
to select the appropriate child.
Moreover the generated document will carry the name of the child
rather than the main file.
This resolves all three above issues.

This feature is meant to make the editing of books,
thesis documents and lecture notes somewhat more convenient.
However, the package can also be used efficiently for
composing a series of documents (such as exercise sheets)
which are typically distributed individually.
It then assists the author in generating the individual documents
(potentially in different versions)
as well as a document containing the collected series.
Another application is in developing style files
or other kinds of included material
where compilation of the style file could redirect
to a sample or test file.

%%%%%%%%%%%%%%%%%%%%%%%%%%%%%%%%%%%%%%%%%%%%%%%%%%%%%%%%%%%%%%%%%%%%%%%%%%%%%%%%
%%%%%%%%%%%%%%%%%%%%%%%%%%%%%%%%%%%%%%%%%%%%%%%%%%%%%%%%%%%%%%%%%%%%%%%%%%%%%%%%
\section{Usage}

First of all, the package \textsf{childdoc} is \emph{not} a standard
\LaTeXe{} |.sty| style file! Therefore it needs to be invoked in
a non-standard way.

%%%%%%%%%%%%%%%%%%%%%%%%%%%%%%%%%%%%%%%%%%%%%%%%%%%%%%%%%%%%%%%%%%%%%%%%%%%%%%%%
\subsection{Included Files}
\label{sec:include}

%%%%%%%%%%%%%%%%%%%%%%%%%%%%%%%%%%%%%%%%
\DescribeMacro{\childdocmain}
To use the package, add the commands
\begin{center}
\begin{tabular}{l}
|\input{childdoc.def}|\\
|\childdocmain{}|\\
\end{tabular}
\end{center}
at the very top of the main \LaTeX{} file,
in particular \emph{before} the |\documentclass| statement!
The argument of |\childdocmain| should be left empty
(but it must be present).

%%%%%%%%%%%%%%%%%%%%%%%%%%%%%%%%%%%%%%%%
\DescribeMacro{\childdocof}
Furthermore, add the commands
\begin{center}
\begin{tabular}{l}
|\input{childdoc.def}|\\
|\childdocof{|\textit{main}|}|\\
\end{tabular}
\end{center}
at the top of every child file \textit{child}
which is included by |\include{|\textit{child}|}|
from within the main file
(or at least for those files to be compiled individually).
The argument \textit{main} must be the filename of the main file.

There are a couple of
considerations in setting up the main and child documents:

%%%%%%%%%%%%%%%%%%%%%%%%%%%%%%%%%%%%%%%%
\paragraph{Restrictions.}

Please note the following restrictions:
\begin{itemize}
\item
|\childdocmain| must be called with one argument \textit{main}
to ensure compatibility with earlier version of the package.
It must either be empty (|\childdocmain{}|)
or precisely match the filename of the main file in which it is specified.
See \secref{sec:detection} for further information.
\item
The filename \textit{main} must be specified without the |.tex| extension.
\item
The filename \textit{main} is case sensitive
(even in case-insensitive file systems)
due to internal string comparison.
\item
The argument \textit{main} should be fully expanded, it cannot be a macro.
\item
Subdirectories and special characters should be avoided in filenames.
\item
The command |\childdocmain{|\textit{main}|}| must be followed by a whitespace.
It should not be followed immediately by another command
or by a comment mark `|%|'.
This is because the \TeX{} parser reads the token immediately following
the argument of |\childdocmain| and puts it
at the beginning of every child section;
however, a white\-space is ignored.
\end{itemize}

%%%%%%%%%%%%%%%%%%%%%%%%%%%%%%%%%%%%%%%%
\paragraph{Content of Main File.}

It is advisable to place all content in the child files included by |\include|.
Any output contained in the main file will appear in all child documents
unless suppressed manually;
it cannot be suppressed automatically by the |\includeonly| directive
and thus should normally be avoided.
A method to include some content in the main file
by means of conditional processing is described in \secref{sec:conditional}.

%%%%%%%%%%%%%%%%%%%%%%%%%%%%%%%%%%%%%%%%
\paragraph{Page Numbering.}

When only a part of the document is compiled,
the appropriate numbering of pages
(as well as other status parameters)
is determined from the |.aux| files.
The latter contain information from previous passes.
However this information needs to propagate through
all intermediate child documents.
Therefore the page numbering in child documents may well
be inconsistent until the complete document is compiled at least once.

A useful (if unconventional) way to always ensure a consistent
page numbering is to restart the numbering in each child document
and denote the pages by `\textit{child}|.|\textit{page}'
where \textit{child} represents the chapter/section number of the child file.
This can be achieved by the command
|\numberwithin{page}{|\textit{child}|}|
of the \textsf{amsmath} package
where \textit{child} can be |chapter| or |section|
depending on the chosen structuring.
Alternatively, one can modify the macro |\thepage| appropriately
and reset the counter |page| at the start of each child file.

%%%%%%%%%%%%%%%%%%%%%%%%%%%%%%%%%%%%%%%%%%%%%%%%%%%%%%%%%%%%%%%%%%%%%%%%%%%%%%%%
\subsection{Conditional Processing}
\label{sec:conditional}

The package provides a mechanism to compile different versions
of a document. To customise the versions further some conditional processing
can come in handy to distinguish which version is being compiled.
The package provides two macros to describe the compilation context:

%%%%%%%%%%%%%%%%%%%%%%%%%%%%%%%%%%%%%%%%
\DescribeMacro{\ifchilddoc}
The conditional |\ifchilddoc| distinguishes between the compilation of
child documents and the main document:
%
\begin{center}
|\ifchilddoc |\textit{child-code}| |[|\||else |\textit{main-code}]| \||fi|
\end{center}

%%%%%%%%%%%%%%%%%%%%%%%%%%%%%%%%%%%%%%%%
\DescribeMacro{\childdocname}
\DescribeMacro{\childdocjob}
The macro |\childdocname| contains the filename (without extension)
of the main or child file being processed.
Note that |\childdocjob| will always contain the name of the main file.

%%%%%%%%%%%%%%%%%%%%%%%%%%%%%%%%%%%%%%%%
\paragraph{Title Page.}

Conditional processing can be used to include a title or banner page
in the main document when proper precautions are taken.
Importantly, the code in the main file should ensure that the page counter
(as well as other status parameters which are stored in the |.aux| files)
takes the same value after the conditional processing.
Otherwise the page numbers may take divergent values
depending on which part is compiled.

For example, a title page could be declared by:
%
\begin{center}
\begin{tabular}{l}
|\ifchilddoc\||else|\\
|\addtocounter{page}{-1}|\\
\textit{code for title page}\\
|\newpage|\\
|\||fi|
\end{tabular}
\end{center}
%
A banner page for the child documents can be generated by:
%
\begin{center}
\begin{tabular}{l}
|\ifchilddoc|\\
|\addtocounter{page}{-1}|\\
\textit{code for banner page}\\
|\newpage|\\
|\||fi|
\end{tabular}
\end{center}
%
Here one could write a message such as:
\begin{center}
|This is the part \childdocname{} of \childdocjob{}.|
\end{center}

%%%%%%%%%%%%%%%%%%%%%%%%%%%%%%%%%%%%%%%%%%%%%%%%%%%%%%%%%%%%%%%%%%%%%%%%%%%%%%%%
\subsection{Flags}
\label{sec:flags}

The package makes it easy to generate different versions
of the main or child documents.
To this end compilation flags can be defined
and assigned different default values.
They will be particularly useful in conjunction
with the forwarding mechanism described in \secref{sec:forward}.

For example, it may be useful to have a flag |\version|
which can be set to |draft| or |final|.
The document source will contain some conditional code
depending on the value of |\version|.
Suppose further, the flag should default to |final| for the main file
and to |draft| for child files
which is a natural assignment for editing the document.
This is achieved by placing the following code
in the preamble of the main document
(below the |\childdocmain| directive):
%
\begin{center}
\begin{tabular}{l}
|\ifchilddoc|\\
|\providecommand{\version}{draft}|\\
|\||else|\\
|\providecommand{\version}{final}|\\
|\||fi|
\end{tabular}
\end{center}
%
The definition by |\providecommand| makes sure
that previous definitions are not overwritten.
Further statements |\providecommand{\version}{...}|
can thus be added before the above code to override it.

For the main file, one might add a line
(between |\childdocmain| and the above block)
%
\begin{center}
|%\ifchilddoc\||else\providecommand{\version}{draft}\||fi|
\end{center}
%
which can be uncommented to produce a draft version.
Likewise one can add a line to the very top of a child file
(above the |\childdocof{|\textit{main}|}| directive)
%
\begin{center}
|%\providecommand{\version}{final}|
\end{center}
%
which can be uncommented to produce the final version of this child document.

%%%%%%%%%%%%%%%%%%%%%%%%%%%%%%%%%%%%%%%%%%%%%%%%%%%%%%%%%%%%%%%%%%%%%%%%%%%%%%%%
\subsection{Forwarding}
\label{sec:forward}

Different versions of the main or child documents
using compilation flags as described in \secref{sec:flags}
can be (permanently) stored in different files
for convenient compilation, viewing and distribution.
To this end, the package defines a command
to pass on compilation to a different file:

%%%%%%%%%%%%%%%%%%%%%%%%%%%%%%%%%%%%%%%%
\DescribeMacro{\childdocforward}
The command |\childdocforward| redirects processing to
another source file:
%
\begin{center}
\begin{tabular}{l}
|\input{childdoc.def}|\\
|\childdocforward[|\textit{main}|]{|\textit{dest}|}|\\
\end{tabular}
\end{center}
%
The argument \textit{dest} is the destination file
(without extension).
It should be the main file or one of the child files.
Note that further \textsf{childdoc} directives
such as |\childdocof| and |\childdocforward|
in the indicated file will be processed in this form.
The optional argument \textit{main}
passes on directly to the main file \textit{main}
while pretending to compile the child \textit{dest}.
This form behaves as if \textit{dest}
issues |\childdocof{|\textit{main}|}| right away,
and no further \textsf{childdoc} directives will be processed.

%%%%%%%%%%%%%%%%%%%%%%%%%%%%%%%%%%%%%%%%
\DescribeMacro{\...prefix}
In the alternative form |\childdocforwardprefix|,
%
\begin{center}
\begin{tabular}{l}
|\input{childdoc.def}|\\
|\childdocforwardprefix[|\textit{main}|]{|\textit{prefix}|}{|\textit{dest}|}|
\end{tabular}
\end{center}
%
the destination file is determined by a pattern
depending on the current file:
To make this work, the current file must be called
`{\textit{prefix}\hspace{0.2em}\textit{suffix}}'
with \textit{prefix} matching precisely the argument.
Processing is then passed on to the file
`{\textit{dest}\hspace{0.2em}\textit{suffix}}'.
Surely, the same effect is achieved by
directly specifying the
argument `{\textit{dest}\hspace{0.2em}\textit{suffix}}'
in the first form.
However, that requires to set up a different file
for each child. With the alternative form of the command
all these files can have exactly the same content
which simplifies setting them up and maintaining them.

For example, the following file |draft.tex|
with a compilation flag |\version| as described in \secref{sec:flags}
compiles the main document as a draft:
%
\begin{center}
\begin{tabular}{l}
|\def\version{draft}|\\
|\input{childdoc.def}|\\
|\childdocforward{|\textit{main}|}|
\end{tabular}
\end{center}
%
Likewise, the following files |final|\textit{nn}|.tex|
compile the final version of the child document
|child|\textit{nn}|.tex|:
%
\begin{center}
\begin{tabular}{l}
|\def\version{final}|\\
|\input{childdoc.def}|\\
|\childdocforwardprefix{final}{child}|
\end{tabular}
\end{center}
%

Note that when several versions of a main file and/or of each child file
are to be generated, it may be convenient to set up a |Makefile| or
shell script to automatise the process.

%%%%%%%%%%%%%%%%%%%%%%%%%%%%%%%%%%%%%%%%%%%%%%%%%%%%%%%%%%%%%%%%%%%%%%%%%%%%%%%%
\subsection{Command Line Processing}
\label{sec:commandline}

The effect of redirection files can also be achieved by invoking
the \LaTeX{} compiler with a more elaborate command line.
Most conveniently this should be done as part
of a shell script or a |Makefile|.

When using \textsf{childdoc} in the main file, the following
command lines effectively perform a redirection
(note that depending on the shell being used,
backslashes may have to be doubled: `|\|' $\to$ `|\\|'):
%
\begin{center}
|... -jobname "|\textit{target}|" |\\|"|[\textit{flags}]%
|\input{childdoc.def}\childdocforward[|\textit{main}|]{|\textit{dest}|}"|
\end{center}
%
Here \textit{target} is the name of the output file,
\textit{main} is the name of the main file
and \textit{dest} is the name of the main or child file to be processed
(all filenames without extensions).
The optional argument \textit{main} can be omitted
if \textit{main} matches \textit{dest}.
Optionally, compilation \textit{flags} can be defined via |\def| commands.
This command line makes the \TeX{} engine believe
it is compiling the file \textit{target}
whose content is specified as the latter parameter.
The provided code then forwards the processing to
\textit{main} or \textit{dest} as described in \secref{sec:forward}.

%%%%%%%%%%%%%%%%%%%%%%%%%%%%%%%%%%%%%%%%%%%%%%%%%%%%%%%%%%%%%%%%%%%%%%%%%%%%%%%%
\subsection{Include by Input}
\label{sec:input}

Including child documents by |\include| has some restrictions by design.
Most notably, the content of a child document always occupies
its own set of pages; pages cannot be shared between child documents.
Usually, this behaviour makes perfect sense
because each child document contain an essential part of the document.
However, in some situations it may be desirable to compose
a document from a collection of parts
without having mandatory page breaks between then.
For this case, the package
provides a mechanism to include parts
by |\input| which can also be processed individually.
However, by construction this mechanism
requires manual handling of the content to be output.

%%%%%%%%%%%%%%%%%%%%%%%%%%%%%%%%%%%%%%%%
\DescribeMacro{\ifchilddocmanual}
The main file should be prepared as usual, see \secref{sec:include}.
However, the document body must make a distinction
between processing of an individual part and of the main document, e.g.:
%
\begin{center}
\begin{tabular}{l}
|\ifchilddocmanual|\\
|\input{\childdocname}|\\
|\||else|\\
\textit{document body with }|\input{|\textit{part}|}|\\
|\||fi|
\end{tabular}
\end{center}
%
The conditional |\ifchilddocmanual| is true whenever
a part to be included by |\input| is being compiled,
and the name of the part is stored in |\childdocname|.

%%%%%%%%%%%%%%%%%%%%%%%%%%%%%%%%%%%%%%%%
\DescribeMacro{\childdocby}
Each part to be included by |\input| should start with:
%
\begin{center}
\begin{tabular}{l}
|\input{childdoc.def}|\\
|\childdocby{|\textit{main}|}|\\
\end{tabular}
\end{center}
%
The directive |\childdocby| is similar to |\childdocof|
described in \secref{sec:include},
but the subsequent selection of content must be done manually.
To that end, both |\ifchilddoc| and |\ifchilddocmanual|
will be true upon processing of a part,
and the name of the part is stored in |\childdocname|.
Note that |\jobname| will be set to the filename of the current part
so that each part receives an individual |.aux| file
that does not interfere with the |.aux| file(s) of the main document.
This behaviour can be altered by the alternative form
|\childdocby[*]{|\textit{main}|}| (with a non-empty optional argument)
which uses the |.aux| file of the main document
by setting |\jobname| to \textit{main}.

%%%%%%%%%%%%%%%%%%%%%%%%%%%%%%%%%%%%%%%%%%%%%%%%%%%%%%%%%%%%%%%%%%%%%%%%%%%%%%%%
\subsection{Driver Development}
\label{sec:driver}

The \textsf{childdoc} mechanism can also be use for the development
of definition files such as \LaTeX{} styles or classes.
This case differs from the above setup with multiple parts
included by |\include| in that no |\includeonly| should be invoked.
This can be achieved by starting the include file
(before |\ProvidesPackage|) with:
%
\begin{center}
\begin{tabular}{l}
|\input{childdoc.def}|\\
|\childdocforward{|\textit{main}|}|\\
\end{tabular}
\end{center}
%
or alternatively with:
%
\begin{center}
\begin{tabular}{l}
|\input{childdoc.def}|\\
|\childdocby{|\textit{main}|}|\\
\end{tabular}
\end{center}
%
Both forms have slightly different effects as described above.
The main file is prepared as usual, see \secref{sec:include}.

%%%%%%%%%%%%%%%%%%%%%%%%%%%%%%%%%%%%%%%%%%%%%%%%%%%%%%%%%%%%%%%%%%%%%%%%%%%%%%%%
\subsection{Legacy Detection}
\label{sec:detection}

The directive |\childdocmain| in the main file can detect
whether the complete document or merely a child is to be compiled
even without using the directive |\childdocof|.
This method is deprecated because it is less robust
and there is no compelling reason to use it;
it is merely provided for backward compatibility
and it may be removed in future versions.

If the detection mechanism is to be used,
it is mandatory to correctly specify
the filename of the main file as the argument of |\childdocmain|:
%
\begin{center}
\begin{tabular}{l}
|\input{childdoc.def}|\\
|\childdocmain{|\textit{main}|}|\\
\end{tabular}
\end{center}
%
If |\jobname| does not match the argument \textit{main} of |\childdocmain|,
it is assumed that |\jobname| points to the child file to be compiled.
When using |\childdocmain| with the main file specified as argument,
it suffices to start a child file
with just |\input{|\textit{main}|}|
without loading of the package and using |\childdocof|.
If instead all processing is done
with the appropriate \textsf{childdoc} directives,
the argument of \textit{main} of |\childdocmain| can be empty.

An alternative version of the command line processing described
in \secref{sec:commandline} using the detection mechanism reads:
%
\begin{center}
|... -jobname "|\textit{target}|" "|[\textit{flags}]%
[|\def\jobname{|\textit{dest}|}|]|\input{|\textit{main}|}"|
\end{center}

%%%%%%%%%%%%%%%%%%%%%%%%%%%%%%%%%%%%%%%%%%%%%%%%%%%%%%%%%%%%%%%%%%%%%%%%%%%%%%%%
\subsection{Manual Code}
\label{sec:manual}

In case one cannot be certain whether the definitions file |childdoc.def|
is installed on the target \TeX{} distribution
and one prefers not to ship it,
it is conceivable to paste a few relevant commands into the sources.

To that end, drop all statements |\input{childdoc.def}|
and perform the replacements as outlined below.
Instead of |\childdocmain{|\textit{main}|}| add the following code
to the top of the main file:
%
\begin{center}
\begin{tabular}{l}
|\||ifdefined\childdocname\endinput\||fi\newif\ifchilddoc|\\
|\edef\childdocname{\scantokens\expandafter{\jobname\noexpand}}|\\
|\def\childdocmain{|\textit{main}|}\||ifx\childdocmain\childdocname\||else|\\
|\childdoctrue\includeonly{\childdocname}\let\jobname\childdocmain\||fi|\\
\end{tabular}
\end{center}
%
Instead of |\childdocof{|\textit{main}|}| just include the main file
at the top of each child file:
%
\begin{center}
|\input{|\textit{main}|}|
\end{center}
%
A simple redirection |\childdocforward{|\textit{dest}|}| is achieved by:
%
\begin{center}
|\def\jobname{|\textit{dest}|}\input{\jobname}|
\end{center}
%
The redirection with prefix
|\childdocforwardprefix[|\textit{prefix}|]{|\textit{dest}|}|
is accomplished by:
%
\begin{center}
\begin{tabular}{l}
|{\edef\jobname{\scantokens\expandafter{\jobname\noexpand}}|\\
|\def\redirectjob |\textit{prefix}|#1~~~{\gdef\jobname{|\textit{dest}|#1}}|\\
|\expandafter\redirectjob\jobname~~~}\input{\jobname}|
\end{tabular}
\end{center}

In an alternative approach,
child documents can be compiled by a specific command line
without additional code or specific definitions:
%
\begin{center}
|... -jobname "|\textit{target}|" "|[\textit{flags}]%
|\includeonly{|\textit{dest}|}\input{|\textit{main}|}"|
\end{center}
%

%%%%%%%%%%%%%%%%%%%%%%%%%%%%%%%%%%%%%%%%%%%%%%%%%%%%%%%%%%%%%%%%%%%%%%%%%%%%%%%%
%%%%%%%%%%%%%%%%%%%%%%%%%%%%%%%%%%%%%%%%%%%%%%%%%%%%%%%%%%%%%%%%%%%%%%%%%%%%%%%%
\section{Information}

%%%%%%%%%%%%%%%%%%%%%%%%%%%%%%%%%%%%%%%%%%%%%%%%%%%%%%%%%%%%%%%%%%%%%%%%%%%%%%%%
\subsection{Copyright}

Copyright \copyright{} 2017--2018 Niklas Beisert

This work may be distributed and/or modified under the
conditions of the \LaTeX{} Project Public License, either version 1.3
of this license or (at your option) any later version.
The latest version of this license is in
  \url{http://www.latex-project.org/lppl.txt}
and version 1.3 or later is part of all distributions of \LaTeX{}
version 2005/12/01 or later.

This work has the LPPL maintenance status `maintained'.

The Current Maintainer of this work is Niklas Beisert.

This work consists of the files |README.txt|, |childdoc.ins| and |childdoc.dtx|
as well as the derived files |childdoc.def|, |cdocsamp.tex|
with |cdocsch1.tex|, |cdocsch2.tex|, |cdocspt3.tex|, |cdocspt4.tex|,
|cdocsdrf.tex|, |cdocsfn1.tex|, |cdocsfn2.tex|
as well as |childdoc.pdf|.

%%%%%%%%%%%%%%%%%%%%%%%%%%%%%%%%%%%%%%%%%%%%%%%%%%%%%%%%%%%%%%%%%%%%%%%%%%%%%%%%
\subsection{Files and Installation}

The package consists of the files:
%
\begin{center}
\begin{tabular}{ll}
    |README.txt|   & readme file \\
    |childdoc.ins| & installation file \\
    |childdoc.dtx| & source file \\
    |childdoc.def| & definition file \\
    |cdocsamp.tex| & sample main file \\
    |cdocsch1.tex| & sample include file \\
    |cdocsch2.tex| & sample include file \\
    |cdocspt3.tex| & sample part file \\
    |cdocspt4.tex| & sample part file \\
    |cdocsdrf.tex| & sample redirection file \\
    |cdocsfn1.tex| & sample redirection file \\
    |cdocsfn2.tex| & sample redirection file \\
    |childdoc.pdf| & manual
\end{tabular}
\end{center}
%
The distribution consists of the files
|README.txt|, |childdoc.ins| and |childdoc.dtx|.
%
\begin{itemize}
\item
Run (pdf)\LaTeX{} on |childdoc.dtx|
to compile the manual |childdoc.pdf| (this file).
\item
Run \LaTeX{} on |childdoc.ins| to create the definitions file |childdoc.def|
and the sample |cdocsamp.tex| with include files
|cdocsch1.tex|, |cdocsch2.tex|, |cdocspt3.tex|, |cdocspt4.tex|,
|cdocsdrf.tex|, |cdocsfn1.tex|, |cdocsfn2.tex|.
Then copy the file |childdoc.def| to an appropriate directory of your \LaTeX{}
distribution, e.g.\ \textit{texmf-root}|/tex/latex/childdoc|.
\end{itemize}

%%%%%%%%%%%%%%%%%%%%%%%%%%%%%%%%%%%%%%%%%%%%%%%%%%%%%%%%%%%%%%%%%%%%%%%%%%%%%%%%
\subsection{Related CTAN Packages}

There are several other packages which offer a similar functionality:
%
\begin{itemize}
\item
The packages
\href{http://ctan.org/pkg/docmute}{\textsf{docmute}},
\href{http://ctan.org/pkg/includex}{\textsf{includex}} and
\href{http://ctan.org/pkg/standalone}{\textsf{standalone}}
provide commands to include only the document body of
a child file thus allowing both files to be compiled individually.
\item
The packages \href{http://ctan.org/pkg/subdocs}{\textsf{subdocs}}
and \href{http://ctan.org/pkg/subfiles}{\textsf{subfiles}}
provide structures in which the main and child documents can be
encapsulated and allowing them to be compiled individually.
The inclusion mechanism is different from the conventional |\include|.
\item
The package \href{http://ctan.org/pkg/combine}{\textsf{combine}}
is an elaborate solution to combine several documents into one.
\end{itemize}
%
See also the CTAN topic \href{http://ctan.org/topic/subdocs}{\textsf{subdocs}}
for further related packages.
The present package differs from the above solutions in that
a document structure constructed with the conventional |\include| mechanism
just needs two extra commands at the top of every file
such that all constituent files can be compiled individually.

%%%%%%%%%%%%%%%%%%%%%%%%%%%%%%%%%%%%%%%%%%%%%%%%%%%%%%%%%%%%%%%%%%%%%%%%%%%%%%%%
%\subsection{Feature Suggestions}
%
%The following is a list of features which may be useful for future
%versions of this package:
%%
%\begin{itemize}
%\item
%\ldots
%\end{itemize}

%%%%%%%%%%%%%%%%%%%%%%%%%%%%%%%%%%%%%%%%%%%%%%%%%%%%%%%%%%%%%%%%%%%%%%%%%%%%%%%%
\subsection{Revision History}

%%%%%%%%%%%%%%%%%%%%%%%%%%%%%%%%%%%%%%%%
\paragraph{v2.0:} 2018/12/30

\begin{itemize}
\item
immediate forward processing
\item
added |\childdocby| mechanism
\item
manual restructured
\end{itemize}

%%%%%%%%%%%%%%%%%%%%%%%%%%%%%%%%%%%%%%%%
\paragraph{v1.6:} 2018/01/17

\begin{itemize}
\item
application for development of include files
\item
corrections to manual
\end{itemize}

%%%%%%%%%%%%%%%%%%%%%%%%%%%%%%%%%%%%%%%%
\paragraph{v1.5:} 2017/05/21

\begin{itemize}
\item
more complete structuring introduced
\item
|\childdocof| introduced
\item
|\childdoc| renamed to |\childdocmain|
\item
|\childredirect| renamed to |\childdocforward| and |\childdocforwardprefix|
and functionality expanded
\end{itemize}

%%%%%%%%%%%%%%%%%%%%%%%%%%%%%%%%%%%%%%%%
\paragraph{v1.0:} 2017/04/27

\begin{itemize}
\item
manual and install package
\item
first version published on CTAN
\end{itemize}

%%%%%%%%%%%%%%%%%%%%%%%%%%%%%%%%%%%%%%%%
\paragraph{v0.6:} 2017/04/26

\begin{itemize}
\item
redirection mechanism added
\end{itemize}

%%%%%%%%%%%%%%%%%%%%%%%%%%%%%%%%%%%%%%%%
\paragraph{v0.5:} 2017/04/26

\begin{itemize}
\item
functionality in definition file
\end{itemize}


%%%%%%%%%%%%%%%%%%%%%%%%%%%%%%%%%%%%%%%%%%%%%%%%%%%%%%%%%%%%%%%%%%%%%%%%%%%%%%%%
%%%%%%%%%%%%%%%%%%%%%%%%%%%%%%%%%%%%%%%%%%%%%%%%%%%%%%%%%%%%%%%%%%%%%%%%%%%%%%%%
%%%%%%%%%%%%%%%%%%%%%%%%%%%%%%%%%%%%%%%%%%%%%%%%%%%%%%%%%%%%%%%%%%%%%%%%%%%%%%%%
\appendix

\settowidth\MacroIndent{\rmfamily\scriptsize 000\ }

 \DocInput{childdoc.dtx}

\end{document}
%</driver>
% \fi
%
% %%%%%%%%%%%%%%%%%%%%%%%%%%%%%%%%%%%%%%%%%%%%%%%%%%%%%%%%%%%%%%%%%%%%%%%%%%%%%%
% %%%%%%%%%%%%%%%%%%%%%%%%%%%%%%%%%%%%%%%%%%%%%%%%%%%%%%%%%%%%%%%%%%%%%%%%%%%%%%
% \section{Sample}
%\iffalse
%<*samplemain>
%\fi
%
% The following presents a sample document
% with two chapters, two parts, a title page,
% a compile flag as well as three forwarding files to set the flag.
% It consists of eight |.tex| files:
% \begin{center}
% \begin{tabular}{ll}
% |cdocsamp.tex|&main file\\
% |cdocsch1.tex|&include file for chapter 1\\
% |cdocsch2.tex|&include file for chapter 2\\
% |cdocspt3.tex|&include file for part 3\\
% |cdocspt4.tex|&include file for part 4\\
% |cdocsdrf.tex|&forwarding file for main file in draft mode\\
% |cdocsfi1.tex|&forwarding file for final version of chapter 1\\
% |cdocsfi2.tex|&forwarding file for final version of chapter 2\\
% \end{tabular}
% \end{center}
% Each of the eight files can be compiled directly by the \LaTeX{} compiler.
%
% %%%%%%%%%%%%%%%%%%%%%%%%%%%%%%%%%%%%%%
% \paragraph{Main File.}
%
% The main file is called |cdocsamp.tex|.
%
% Load the \textsf{childdoc} definitions and
% declare the filename for the main document:
%    \begin{macrocode}
\input{childdoc.def}
\childdocmain{}
%    \end{macrocode}

% Optional override for |\version| flag:
%    \begin{macrocode}
%%\ifchilddoc\else\providecommand{\version}{draft}\fi
%    \end{macrocode}

% Define the default values for the |\version| flag
% (|final| for the main file and |draft| for childs):
%    \begin{macrocode}
\ifchilddoc
\providecommand{\version}{draft}
\else
\providecommand{\version}{final}
\fi
%    \end{macrocode}

% Load the standard document class:
%    \begin{macrocode}
\documentclass[12pt]{article}
%    \end{macrocode}

% Start the document body:
%    \begin{macrocode}
\begin{document}
%    \end{macrocode}

% Declare a title page.
% Print title, part of document being processed and version flag:
%    \begin{macrocode}
\addtocounter{page}{-1}
\begin{center}
{\LARGE\bfseries{}childdoc example\par}
\vspace{1cm}
\ifchilddoc
\ifchilddocmanual part\else chapter\fi:
`\childdocname' of `\childdocjob'\par
\else
main document: `\childdocjob'\par
\fi
version: \version\par
\end{center}
\newpage
%    \end{macrocode}

% Manually include selected file,
% otherwise process as usual:
%    \begin{macrocode}
\ifchilddocmanual
\section*{part `\childdocname'}
\input{\childdocname}
\else
%    \end{macrocode}

% Include the two chapters:
%    \begin{macrocode}
\include{cdocsch1}
\include{cdocsch2}
%    \end{macrocode}

% Include the two parts unless only chapters should be displayed:
%    \begin{macrocode}
\ifchilddoc\else
\section{part three}
\input{cdocspt3}
\section{part four}
\input{cdocspt4}
\fi
%    \end{macrocode}

% Process as usual until here:
%    \begin{macrocode}
\fi
%    \end{macrocode}

% End of document body:
%    \begin{macrocode}
\end{document}
%    \end{macrocode}
%\iffalse
%</samplemain>
%\fi
%
% %%%%%%%%%%%%%%%%%%%%%%%%%%%%%%%%%%%%%%
% \paragraph{Chapter Include Files.}
%
% The include files are called |cdocsch1.tex| and |cdocsch2.tex|.
%
%\iffalse
%<*samplechap1|samplechap2>
%\fi

% Optional override for |\version| flag:
%    \begin{macrocode}
%%\providecommand{\version}{final}
%    \end{macrocode}

% Include the main document:
%    \begin{macrocode}
\input{childdoc.def}
\childdocof{cdocsamp}
%    \end{macrocode}

%\iffalse
%</samplechap1|samplechap2>
%\fi
%
%\iffalse
%<*samplechap1>
%\fi
% Some text for chapter 1:
%    \begin{macrocode}
\section{one}
some text in chapter one
%    \end{macrocode}

%\iffalse
%</samplechap1>
%\fi
% Some text for chapter 2:
%\iffalse
%<*samplechap2>
%\fi
%    \begin{macrocode}
\section{two}
more text in chapter two
%    \end{macrocode}

%\iffalse
%</samplechap2>
%\fi
%
% %%%%%%%%%%%%%%%%%%%%%%%%%%%%%%%%%%%%%%
% \paragraph{Part Include Files.}
%
% The include files are called |cdocspt3.tex| and |cdocspt4.tex|.
%
%\iffalse
%<*samplepart3|samplepart4>
%\fi

% Optional override for |\version| flag:
%    \begin{macrocode}
%%\providecommand{\version}{final}
%    \end{macrocode}

% Include the main document:
%    \begin{macrocode}
\input{childdoc.def}
\childdocby{cdocsamp}
%    \end{macrocode}

%\iffalse
%</samplepart3|samplepart4>
%\fi
%
%\iffalse
%<*samplepart3>
%\fi
% Some text for part 3:
%    \begin{macrocode}
some text in part three
%    \end{macrocode}

%\iffalse
%</samplepart3>
%\fi
% Some text for part 4:
%\iffalse
%<*samplepart4>
%\fi
%    \begin{macrocode}
more text in part four
%    \end{macrocode}

%\iffalse
%</samplepart4>
%\fi
%
% %%%%%%%%%%%%%%%%%%%%%%%%%%%%%%%%%%%%%%
% \paragraph{Forwarding for a Complete Draft.}
%
% The following forwarding file |cdocsdrf.tex|
% compiles the main document in draft mode:
%\iffalse
%<*sampledraft>
%\fi
%    \begin{macrocode}
\def\version{draft}
\input{childdoc.def}
\childdocforward{cdocsamp}
%    \end{macrocode}

%\iffalse
%</sampledraft>
%\fi
%
% %%%%%%%%%%%%%%%%%%%%%%%%%%%%%%%%%%%%%%
% \paragraph{Forwarding for Final Version of the Chapters.}
%
% The following forwarding files |cdocsfn1.tex| and |cdocsfn2.tex|
% (with identical content)
% compile the final versions of the child documents
% |cdocsch1.tex| and |cdocsch2.tex|, respectively:
%\iffalse
%<*samplefinal>
%\fi
%    \begin{macrocode}
\def\version{final}
\input{childdoc.def}
\childdocforwardprefix[cdocsamp]{cdocsfn}{cdocsch}
%    \end{macrocode}

%\iffalse
%</samplefinal>
%\fi
%
% %%%%%%%%%%%%%%%%%%%%%%%%%%%%%%%%%%%%%%
% \paragraph{Command Line Processing.}
%
% The following three command lines generate the output files
% |cdocscld|, |cdocscl1| and |cdocscl2|
% which should be identical to
% |cdocsdrf|, |cdocsch1| and |cdocsfn2|, respectively:
% \begin{center}
% \begin{tabular}{l}
% |latex -jobname cdocscld \|\\
% |  "\def\version{draft}\input{childdoc.def}\childdocforward{cdocsamp}"|\\
% |latex -jobname cdocscl1 \|\\
% |  "\input{childdoc.def}\childdocforward[cdocsamp]{cdocsch1}"|\\
% |latex -jobname cdocscl2 \|\\
% |  "\def\version{final}\input{childdoc.def}\childdocforward{cdocsch2}"|
% \end{tabular}
% \end{center}
% Note that the trailing backslash on each first line
% merely continues the input to the second line
% (for convenient cut ant paste).
% Furthermore, the command |latex| can be replaced by any
% of its alternative versions such as |pdflatex|.
%
% %%%%%%%%%%%%%%%%%%%%%%%%%%%%%%%%%%%%%%%%%%%%%%%%%%%%%%%%%%%%%%%%%%%%%%%%%%%%%%
% %%%%%%%%%%%%%%%%%%%%%%%%%%%%%%%%%%%%%%%%%%%%%%%%%%%%%%%%%%%%%%%%%%%%%%%%%%%%%%
% \section{Implementation}
%\iffalse
%<*package>
%\fi
%
% This section describes the definitions file |childdoc.def|.

% The definitions cannot be loaded using |\usepackage| or |\RequirePackage|
% which has a mechanism to prevent loading a style file more than once.
% When loading the definitions by means of |\input|
% multiple instances have to be prevented manually:
%\iffalse
%This code needs to be before the `\ProvidesFile' directive
%which is defined at the beginning of this file.
%Therefore it is also placed there and commented out here.
%</package>
%<*discard>
%\fi
%    \begin{macrocode}
\ifdefined\childdocmain\endinput\fi
%    \end{macrocode}
%\iffalse
%</discard>
%<*package>
%\fi
%
% \macro{\ifchilddoc}
% \macro{\ifchilddocmanual}
% The conditional |\ifchilddoc| tells whether a
% child (true) or main (false) document is being compiled.
% The conditional |\ifchilddocmanual| tells whether
% the |\includeonly| mechanism is used (false) or
% the selection of child files must be performed manually (true).
% The definitions initialise to false:
%    \begin{macrocode}
\newif\ifchilddoc
\newif\ifchilddocmanual
%    \end{macrocode}

% \macro{\childdocname}
% \macro{\childdocjob}
% The macro |\childdocname| stores the name of the main document
% to be compiled. The macro |\childdocjob| stores the name of
% the document on which the \LaTeX{} compiler was originally invoked.
% The content of |\jobname| cannot be compared
% to filenames specified in the source due to different catcodes.
% The following code rescans |\jobname|, stores the result
% in |\childdocname| and saves a copy in |\childdocjob|:
%    \begin{macrocode}
\edef\childdocname{\scantokens\expandafter{\jobname\noexpand}}
\let\childdocjob\childdocname
%    \end{macrocode}

% \macro{\childdocdisable}
% The macro |\childdocdisable| prevents the main file
% from being processed more than once.
% At this stage, the main document command |\childdocmain|
% is assumed to be called once again where it should do nothing.
% Any subsequent call to it should prevent
% a secondary processing of the main document
% It overwrites the forwarding commands
% |\childdocof| and |\childdocforward|
% with empty macros to prevent further inclusions of the main document:
%    \begin{macrocode}
\newcommand{\childdocdisable}
{
  \renewcommand{\childdocmain}[1]{\renewcommand{\childdocmain}[1]{\endinput}}
  \renewcommand{\childdocof}[1]{}
  \renewcommand{\childdocby}[2][]{}
  \renewcommand{\childdocforward}[2][]{}
  \renewcommand{\childdocdisable}{}
}
%    \end{macrocode}

% \macro{\childdocmain}
% The macro |\childdocmain| is to be called at the top of the main file
% with nothing or the main filename (without extension) as argument.
% First, it breaks loops.
% If the argument is not empty and does not match |\childdocname|
% (which is set by the first inclusion of |childdoc.def|),
% |\ifchilddoc| is set to true, |\includeonly| is applied to the child file
% and |\jobname| is set to the main file
% (for proper handling of |.aux| files):
%    \begin{macrocode}
\newcommand{\childdocmain}[1]
{
  \childdocdisable\childdocmain{}
  \if?#1?\else
    \begingroup
      \def\childdoctmp{#1}
      \ifx\childdoctmp\childdocname
        \def\childdoctmp{}
      \else
        \def\childdoctmp
        {
          \childdoctrue
          \includeonly{\childdocname}
          \def\childdocjob{#1}
          \def\jobname{#1}
        }
      \fi
      \expandafter
    \endgroup
    \childdoctmp
  \fi
}
%    \end{macrocode}

% \macro{\childdocof}
% The command |\childdocof| redirects
% compilation to the main file |#1|.
%    \begin{macrocode}
\newcommand{\childdocof}[1]
{
  \childdocdisable
  \childdoctrue
  \includeonly{\childdocname}
  \def\jobname{#1}
  \def\childdocjob{#1}
  \input{#1}
}
%    \end{macrocode}

% \macro{\childdocby}
% The command |\childdocby| ....
%    \begin{macrocode}
\newcommand{\childdocby}[2][]
{
  \childdocdisable
  \childdoctrue
  \childdocmanualtrue
  \if?#1?\else
    \def\jobname{#2}
  \fi
  \def\childdocjob{#2}
  \input{#2}
  \endinput
}
%    \end{macrocode}

% \macro{\childdocforward}
% The command |\childdocforward| redirects
% compilation to the main file or
% (if the optional argument is given) a child file.
% Parameters are set as if the main file
% or a child file starting with |\childdocof| was compiled.
% Then compilation is handed over to the main file:
%    \begin{macrocode}
\newcommand{\childdocforward}[2][]
{
  \begingroup
    \if?#1?
      \def\childdoctmp
      {
        \def\childdocname{#2}
        \def\childdocjob{#2}
        \def\jobname{#2}
        \input{#2}
        \endinput
      }
    \else
      \def\childdoctmp
      {
        \childdocdisable
        \def\childdocname{#2}
        \childdoctrue
        \includeonly{#2}
        \def\childdocjob{#1}
        \def\jobname{#1}
        \input{#1}
        \endinput
      }
    \fi
    \expandafter
  \endgroup
  \childdoctmp
}
%    \end{macrocode}

% \macro{\childdocforwardprefix}
% The command |\childdocforwardprefix| redirects
% compilation to the main or a child file by means of a pattern.
% The prefix |#1| in the current filename is replaced by |#2|
% and the suffix of the current filename is kept
% (it is assumed that the filename does not contain the substring `|~~~|'
% which is used as a delimiter).
% Compilation is handed over to the new file by |\childdocforward|:
%    \begin{macrocode}
\newcommand{\childdocforwardprefix}[3][]
{
  \begingroup
    \def\childdocextract #2##1~~~{\def\childdoctmp{\childdocforward[#1]{#3##1}}}
    \expandafter\childdocextract\childdocname~~~
    \expandafter
  \endgroup
  \childdoctmp
}
%    \end{macrocode}

% \macro{\childdoc}
% The deprecated macro |\childdoc| is a legacy version of |\childdocmain|:
%    \begin{macrocode}
\newcommand{\childdoc}{\childdocmain}
%    \end{macrocode}

% \macro{\childdocredirect}
% The deprecated macro |\childdocredirect| is a legacy version
% of |\childdocforward| and |\childdocforwardprefix|:
%    \begin{macrocode}
\newcommand{\childdocredirect}[2][]
{
  \begingroup
    \if?#1?
      \def\childdoctmp{\childdocforward{#2}}
    \else
      \def\childdoctmp{\childdocforwardprefix{#1}{#2}}
    \fi
    \expandafter
  \endgroup
  \childdoctmp
}
%    \end{macrocode}

%\iffalse
%</package>
%\fi
%
\endinput

\childdocforwardprefix[cdocsamp]{cdocsfn}{cdocsch}
%    \end{macrocode}

%\iffalse
%</samplefinal>
%\fi
%
% %%%%%%%%%%%%%%%%%%%%%%%%%%%%%%%%%%%%%%
% \paragraph{Command Line Processing.}
%
% The following three command lines generate the output files
% |cdocscld|, |cdocscl1| and |cdocscl2|
% which should be identical to
% |cdocsdrf|, |cdocsch1| and |cdocsfn2|, respectively:
% \begin{center}
% \begin{tabular}{l}
% |latex -jobname cdocscld \|\\
% |  "\def\version{draft}% \iffalse
%
% childdoc.dtx Copyright (C) 2017-2018 Niklas Beisert
%
% This work may be distributed and/or modified under the
% conditions of the LaTeX Project Public License, either version 1.3
% of this license or (at your option) any later version.
% The latest version of this license is in
%   http://www.latex-project.org/lppl.txt
% and version 1.3 or later is part of all distributions of LaTeX
% version 2005/12/01 or later.
%
% This work has the LPPL maintenance status `maintained'.
%
% The Current Maintainer of this work is Niklas Beisert.
%
% This work consists of the files childdoc.dtx and childdoc.ins
% and the derived files childdoc.def and cdocsamp.tex with
% cdocsch1.tex, cdocsch2.tex, cdocsdrf.tex, cdocsfn1.tex, cdocsfn2.tex.
%
%<package>\ifdefined\childdocmain\endinput\fi
%<package>\ProvidesFile{childdoc.def}[2018/12/30 v2.0 child document driver]
%<samplemain>\ProvidesFile{cdocsamp.tex}[2018/12/30 v2.0 sample for childdoc]
%<*driver>
%\ProvidesFile{childdoc.drv}[2018/12/30 v2.0 childdoc reference manual file]
\PassOptionsToClass{10pt,a4paper}{article}
\documentclass{ltxdoc}

\usepackage[margin=35mm]{geometry}
\usepackage{hyperref}
\usepackage{hyperxmp}
\usepackage[usenames]{color}

\hypersetup{colorlinks=true}
\hypersetup{pdfstartview=FitH}
\hypersetup{pdfpagemode=UseNone}
\hypersetup{pdfsource={}}
\hypersetup{pdflang={en-UK}}
\hypersetup{pdfcopyright={Copyright 2017-2018 Niklas Beisert.
  This work may be distributed and/or modified under the
  conditions of the LaTeX Project Public License, either version 1.3
  of this license or (at your option) any later version.}}
\hypersetup{pdflicenseurl={http://www.latex-project.org/lppl.txt}}
\hypersetup{pdfcontactaddress={ETH Zurich, ITP, HIT K,
  Wolfgang-Pauli-Strasse 27}}
\hypersetup{pdfcontactpostcode={8093}}
\hypersetup{pdfcontactcity={Zurich}}
\hypersetup{pdfcontactcountry={Switzerland}}
\hypersetup{pdfcontactemail={nbeisert@itp.phys.ethz.ch}}
\hypersetup{pdfcontacturl={http://people.phys.ethz.ch/\xmptilde nbeisert/}}

\newcommand{\secref}[1]{\hyperref[#1]{section \ref*{#1}}}

\parskip1ex
\parindent0pt
\let\olditemize\itemize
\def\itemize{\olditemize\parskip0pt}

\begin{document}

\title{The \textsf{childdoc} Package}
\hypersetup{pdftitle={The childdoc Package}}
\author{Niklas Beisert\\[2ex]
  Institut f\"ur Theoretische Physik\\
  Eidgen\"ossische Technische Hochschule Z\"urich\\
  Wolfgang-Pauli-Strasse 27, 8093 Z\"urich, Switzerland\\[1ex]
  \href{mailto:nbeisert@itp.phys.ethz.ch}
  {\texttt{nbeisert@itp.phys.ethz.ch}}}
\hypersetup{pdfauthor={Niklas Beisert}}
\hypersetup{pdfsubject={Manual for the LaTeX2e Package childdoc}}
\date{30 December 2018, \textsf{v2.0}}
\maketitle

\begin{abstract}\noindent
\textsf{childdoc} is a \LaTeXe{} package
that enables the direct compilation
of document sections included by |\include|
to individual files.
\end{abstract}

\begingroup
\parskip0ex
\tableofcontents
\endgroup

%%%%%%%%%%%%%%%%%%%%%%%%%%%%%%%%%%%%%%%%%%%%%%%%%%%%%%%%%%%%%%%%%%%%%%%%%%%%%%%%
%%%%%%%%%%%%%%%%%%%%%%%%%%%%%%%%%%%%%%%%%%%%%%%%%%%%%%%%%%%%%%%%%%%%%%%%%%%%%%%%
\section{Introduction}

\LaTeX{} provides a mechanism to structure a large document (such as a book)
into a main file and several child files (containing the chapters)
using the |\include| command.
This mechanism is beneficial for documents
which span hundreds of pages in order to
make the source file(s) more manageable.
Moreover, compilation can be restricted to
selected child files by means of the |\includeonly| command.
The latter feature can be used to reduce the compilation time while editing
(this was significantly more useful in the earlier days of \LaTeX{})
or to generate a smaller document which is easier to navigate.
Another application of |\includeonly| is to generate
documents consisting of selected parts of the complete document.

However, there are a few drawbacks of the plain |\include| mechanism:
\begin{itemize}
\item
The child files cannot be compiled on their own,
they can only be compiled via the main file.
A naive editing environment
(such as a text editor with an option
to have the current file processed by \LaTeX)
may require one to switch to the main file before compiling;
attempting to compile the child file produces errors.
\item
The main file must be modified (each time)
to adjust the |\includeonly| command
to the present needs. This easily leaves the main file in a messy state.
\item
The generated document will always carry the filename
of the main document. This is inconvenient if
several child files are to be compiled and
to be kept for distribution.
\end{itemize}

The present package provides a simple interface
to make child files individually compilable by \LaTeX{}.
Compiling a child file then has the same effect as compiling
the main file with an |\includeonly| command
to select the appropriate child.
Moreover the generated document will carry the name of the child
rather than the main file.
This resolves all three above issues.

This feature is meant to make the editing of books,
thesis documents and lecture notes somewhat more convenient.
However, the package can also be used efficiently for
composing a series of documents (such as exercise sheets)
which are typically distributed individually.
It then assists the author in generating the individual documents
(potentially in different versions)
as well as a document containing the collected series.
Another application is in developing style files
or other kinds of included material
where compilation of the style file could redirect
to a sample or test file.

%%%%%%%%%%%%%%%%%%%%%%%%%%%%%%%%%%%%%%%%%%%%%%%%%%%%%%%%%%%%%%%%%%%%%%%%%%%%%%%%
%%%%%%%%%%%%%%%%%%%%%%%%%%%%%%%%%%%%%%%%%%%%%%%%%%%%%%%%%%%%%%%%%%%%%%%%%%%%%%%%
\section{Usage}

First of all, the package \textsf{childdoc} is \emph{not} a standard
\LaTeXe{} |.sty| style file! Therefore it needs to be invoked in
a non-standard way.

%%%%%%%%%%%%%%%%%%%%%%%%%%%%%%%%%%%%%%%%%%%%%%%%%%%%%%%%%%%%%%%%%%%%%%%%%%%%%%%%
\subsection{Included Files}
\label{sec:include}

%%%%%%%%%%%%%%%%%%%%%%%%%%%%%%%%%%%%%%%%
\DescribeMacro{\childdocmain}
To use the package, add the commands
\begin{center}
\begin{tabular}{l}
|\input{childdoc.def}|\\
|\childdocmain{}|\\
\end{tabular}
\end{center}
at the very top of the main \LaTeX{} file,
in particular \emph{before} the |\documentclass| statement!
The argument of |\childdocmain| should be left empty
(but it must be present).

%%%%%%%%%%%%%%%%%%%%%%%%%%%%%%%%%%%%%%%%
\DescribeMacro{\childdocof}
Furthermore, add the commands
\begin{center}
\begin{tabular}{l}
|\input{childdoc.def}|\\
|\childdocof{|\textit{main}|}|\\
\end{tabular}
\end{center}
at the top of every child file \textit{child}
which is included by |\include{|\textit{child}|}|
from within the main file
(or at least for those files to be compiled individually).
The argument \textit{main} must be the filename of the main file.

There are a couple of
considerations in setting up the main and child documents:

%%%%%%%%%%%%%%%%%%%%%%%%%%%%%%%%%%%%%%%%
\paragraph{Restrictions.}

Please note the following restrictions:
\begin{itemize}
\item
|\childdocmain| must be called with one argument \textit{main}
to ensure compatibility with earlier version of the package.
It must either be empty (|\childdocmain{}|)
or precisely match the filename of the main file in which it is specified.
See \secref{sec:detection} for further information.
\item
The filename \textit{main} must be specified without the |.tex| extension.
\item
The filename \textit{main} is case sensitive
(even in case-insensitive file systems)
due to internal string comparison.
\item
The argument \textit{main} should be fully expanded, it cannot be a macro.
\item
Subdirectories and special characters should be avoided in filenames.
\item
The command |\childdocmain{|\textit{main}|}| must be followed by a whitespace.
It should not be followed immediately by another command
or by a comment mark `|%|'.
This is because the \TeX{} parser reads the token immediately following
the argument of |\childdocmain| and puts it
at the beginning of every child section;
however, a white\-space is ignored.
\end{itemize}

%%%%%%%%%%%%%%%%%%%%%%%%%%%%%%%%%%%%%%%%
\paragraph{Content of Main File.}

It is advisable to place all content in the child files included by |\include|.
Any output contained in the main file will appear in all child documents
unless suppressed manually;
it cannot be suppressed automatically by the |\includeonly| directive
and thus should normally be avoided.
A method to include some content in the main file
by means of conditional processing is described in \secref{sec:conditional}.

%%%%%%%%%%%%%%%%%%%%%%%%%%%%%%%%%%%%%%%%
\paragraph{Page Numbering.}

When only a part of the document is compiled,
the appropriate numbering of pages
(as well as other status parameters)
is determined from the |.aux| files.
The latter contain information from previous passes.
However this information needs to propagate through
all intermediate child documents.
Therefore the page numbering in child documents may well
be inconsistent until the complete document is compiled at least once.

A useful (if unconventional) way to always ensure a consistent
page numbering is to restart the numbering in each child document
and denote the pages by `\textit{child}|.|\textit{page}'
where \textit{child} represents the chapter/section number of the child file.
This can be achieved by the command
|\numberwithin{page}{|\textit{child}|}|
of the \textsf{amsmath} package
where \textit{child} can be |chapter| or |section|
depending on the chosen structuring.
Alternatively, one can modify the macro |\thepage| appropriately
and reset the counter |page| at the start of each child file.

%%%%%%%%%%%%%%%%%%%%%%%%%%%%%%%%%%%%%%%%%%%%%%%%%%%%%%%%%%%%%%%%%%%%%%%%%%%%%%%%
\subsection{Conditional Processing}
\label{sec:conditional}

The package provides a mechanism to compile different versions
of a document. To customise the versions further some conditional processing
can come in handy to distinguish which version is being compiled.
The package provides two macros to describe the compilation context:

%%%%%%%%%%%%%%%%%%%%%%%%%%%%%%%%%%%%%%%%
\DescribeMacro{\ifchilddoc}
The conditional |\ifchilddoc| distinguishes between the compilation of
child documents and the main document:
%
\begin{center}
|\ifchilddoc |\textit{child-code}| |[|\||else |\textit{main-code}]| \||fi|
\end{center}

%%%%%%%%%%%%%%%%%%%%%%%%%%%%%%%%%%%%%%%%
\DescribeMacro{\childdocname}
\DescribeMacro{\childdocjob}
The macro |\childdocname| contains the filename (without extension)
of the main or child file being processed.
Note that |\childdocjob| will always contain the name of the main file.

%%%%%%%%%%%%%%%%%%%%%%%%%%%%%%%%%%%%%%%%
\paragraph{Title Page.}

Conditional processing can be used to include a title or banner page
in the main document when proper precautions are taken.
Importantly, the code in the main file should ensure that the page counter
(as well as other status parameters which are stored in the |.aux| files)
takes the same value after the conditional processing.
Otherwise the page numbers may take divergent values
depending on which part is compiled.

For example, a title page could be declared by:
%
\begin{center}
\begin{tabular}{l}
|\ifchilddoc\||else|\\
|\addtocounter{page}{-1}|\\
\textit{code for title page}\\
|\newpage|\\
|\||fi|
\end{tabular}
\end{center}
%
A banner page for the child documents can be generated by:
%
\begin{center}
\begin{tabular}{l}
|\ifchilddoc|\\
|\addtocounter{page}{-1}|\\
\textit{code for banner page}\\
|\newpage|\\
|\||fi|
\end{tabular}
\end{center}
%
Here one could write a message such as:
\begin{center}
|This is the part \childdocname{} of \childdocjob{}.|
\end{center}

%%%%%%%%%%%%%%%%%%%%%%%%%%%%%%%%%%%%%%%%%%%%%%%%%%%%%%%%%%%%%%%%%%%%%%%%%%%%%%%%
\subsection{Flags}
\label{sec:flags}

The package makes it easy to generate different versions
of the main or child documents.
To this end compilation flags can be defined
and assigned different default values.
They will be particularly useful in conjunction
with the forwarding mechanism described in \secref{sec:forward}.

For example, it may be useful to have a flag |\version|
which can be set to |draft| or |final|.
The document source will contain some conditional code
depending on the value of |\version|.
Suppose further, the flag should default to |final| for the main file
and to |draft| for child files
which is a natural assignment for editing the document.
This is achieved by placing the following code
in the preamble of the main document
(below the |\childdocmain| directive):
%
\begin{center}
\begin{tabular}{l}
|\ifchilddoc|\\
|\providecommand{\version}{draft}|\\
|\||else|\\
|\providecommand{\version}{final}|\\
|\||fi|
\end{tabular}
\end{center}
%
The definition by |\providecommand| makes sure
that previous definitions are not overwritten.
Further statements |\providecommand{\version}{...}|
can thus be added before the above code to override it.

For the main file, one might add a line
(between |\childdocmain| and the above block)
%
\begin{center}
|%\ifchilddoc\||else\providecommand{\version}{draft}\||fi|
\end{center}
%
which can be uncommented to produce a draft version.
Likewise one can add a line to the very top of a child file
(above the |\childdocof{|\textit{main}|}| directive)
%
\begin{center}
|%\providecommand{\version}{final}|
\end{center}
%
which can be uncommented to produce the final version of this child document.

%%%%%%%%%%%%%%%%%%%%%%%%%%%%%%%%%%%%%%%%%%%%%%%%%%%%%%%%%%%%%%%%%%%%%%%%%%%%%%%%
\subsection{Forwarding}
\label{sec:forward}

Different versions of the main or child documents
using compilation flags as described in \secref{sec:flags}
can be (permanently) stored in different files
for convenient compilation, viewing and distribution.
To this end, the package defines a command
to pass on compilation to a different file:

%%%%%%%%%%%%%%%%%%%%%%%%%%%%%%%%%%%%%%%%
\DescribeMacro{\childdocforward}
The command |\childdocforward| redirects processing to
another source file:
%
\begin{center}
\begin{tabular}{l}
|\input{childdoc.def}|\\
|\childdocforward[|\textit{main}|]{|\textit{dest}|}|\\
\end{tabular}
\end{center}
%
The argument \textit{dest} is the destination file
(without extension).
It should be the main file or one of the child files.
Note that further \textsf{childdoc} directives
such as |\childdocof| and |\childdocforward|
in the indicated file will be processed in this form.
The optional argument \textit{main}
passes on directly to the main file \textit{main}
while pretending to compile the child \textit{dest}.
This form behaves as if \textit{dest}
issues |\childdocof{|\textit{main}|}| right away,
and no further \textsf{childdoc} directives will be processed.

%%%%%%%%%%%%%%%%%%%%%%%%%%%%%%%%%%%%%%%%
\DescribeMacro{\...prefix}
In the alternative form |\childdocforwardprefix|,
%
\begin{center}
\begin{tabular}{l}
|\input{childdoc.def}|\\
|\childdocforwardprefix[|\textit{main}|]{|\textit{prefix}|}{|\textit{dest}|}|
\end{tabular}
\end{center}
%
the destination file is determined by a pattern
depending on the current file:
To make this work, the current file must be called
`{\textit{prefix}\hspace{0.2em}\textit{suffix}}'
with \textit{prefix} matching precisely the argument.
Processing is then passed on to the file
`{\textit{dest}\hspace{0.2em}\textit{suffix}}'.
Surely, the same effect is achieved by
directly specifying the
argument `{\textit{dest}\hspace{0.2em}\textit{suffix}}'
in the first form.
However, that requires to set up a different file
for each child. With the alternative form of the command
all these files can have exactly the same content
which simplifies setting them up and maintaining them.

For example, the following file |draft.tex|
with a compilation flag |\version| as described in \secref{sec:flags}
compiles the main document as a draft:
%
\begin{center}
\begin{tabular}{l}
|\def\version{draft}|\\
|\input{childdoc.def}|\\
|\childdocforward{|\textit{main}|}|
\end{tabular}
\end{center}
%
Likewise, the following files |final|\textit{nn}|.tex|
compile the final version of the child document
|child|\textit{nn}|.tex|:
%
\begin{center}
\begin{tabular}{l}
|\def\version{final}|\\
|\input{childdoc.def}|\\
|\childdocforwardprefix{final}{child}|
\end{tabular}
\end{center}
%

Note that when several versions of a main file and/or of each child file
are to be generated, it may be convenient to set up a |Makefile| or
shell script to automatise the process.

%%%%%%%%%%%%%%%%%%%%%%%%%%%%%%%%%%%%%%%%%%%%%%%%%%%%%%%%%%%%%%%%%%%%%%%%%%%%%%%%
\subsection{Command Line Processing}
\label{sec:commandline}

The effect of redirection files can also be achieved by invoking
the \LaTeX{} compiler with a more elaborate command line.
Most conveniently this should be done as part
of a shell script or a |Makefile|.

When using \textsf{childdoc} in the main file, the following
command lines effectively perform a redirection
(note that depending on the shell being used,
backslashes may have to be doubled: `|\|' $\to$ `|\\|'):
%
\begin{center}
|... -jobname "|\textit{target}|" |\\|"|[\textit{flags}]%
|\input{childdoc.def}\childdocforward[|\textit{main}|]{|\textit{dest}|}"|
\end{center}
%
Here \textit{target} is the name of the output file,
\textit{main} is the name of the main file
and \textit{dest} is the name of the main or child file to be processed
(all filenames without extensions).
The optional argument \textit{main} can be omitted
if \textit{main} matches \textit{dest}.
Optionally, compilation \textit{flags} can be defined via |\def| commands.
This command line makes the \TeX{} engine believe
it is compiling the file \textit{target}
whose content is specified as the latter parameter.
The provided code then forwards the processing to
\textit{main} or \textit{dest} as described in \secref{sec:forward}.

%%%%%%%%%%%%%%%%%%%%%%%%%%%%%%%%%%%%%%%%%%%%%%%%%%%%%%%%%%%%%%%%%%%%%%%%%%%%%%%%
\subsection{Include by Input}
\label{sec:input}

Including child documents by |\include| has some restrictions by design.
Most notably, the content of a child document always occupies
its own set of pages; pages cannot be shared between child documents.
Usually, this behaviour makes perfect sense
because each child document contain an essential part of the document.
However, in some situations it may be desirable to compose
a document from a collection of parts
without having mandatory page breaks between then.
For this case, the package
provides a mechanism to include parts
by |\input| which can also be processed individually.
However, by construction this mechanism
requires manual handling of the content to be output.

%%%%%%%%%%%%%%%%%%%%%%%%%%%%%%%%%%%%%%%%
\DescribeMacro{\ifchilddocmanual}
The main file should be prepared as usual, see \secref{sec:include}.
However, the document body must make a distinction
between processing of an individual part and of the main document, e.g.:
%
\begin{center}
\begin{tabular}{l}
|\ifchilddocmanual|\\
|\input{\childdocname}|\\
|\||else|\\
\textit{document body with }|\input{|\textit{part}|}|\\
|\||fi|
\end{tabular}
\end{center}
%
The conditional |\ifchilddocmanual| is true whenever
a part to be included by |\input| is being compiled,
and the name of the part is stored in |\childdocname|.

%%%%%%%%%%%%%%%%%%%%%%%%%%%%%%%%%%%%%%%%
\DescribeMacro{\childdocby}
Each part to be included by |\input| should start with:
%
\begin{center}
\begin{tabular}{l}
|\input{childdoc.def}|\\
|\childdocby{|\textit{main}|}|\\
\end{tabular}
\end{center}
%
The directive |\childdocby| is similar to |\childdocof|
described in \secref{sec:include},
but the subsequent selection of content must be done manually.
To that end, both |\ifchilddoc| and |\ifchilddocmanual|
will be true upon processing of a part,
and the name of the part is stored in |\childdocname|.
Note that |\jobname| will be set to the filename of the current part
so that each part receives an individual |.aux| file
that does not interfere with the |.aux| file(s) of the main document.
This behaviour can be altered by the alternative form
|\childdocby[*]{|\textit{main}|}| (with a non-empty optional argument)
which uses the |.aux| file of the main document
by setting |\jobname| to \textit{main}.

%%%%%%%%%%%%%%%%%%%%%%%%%%%%%%%%%%%%%%%%%%%%%%%%%%%%%%%%%%%%%%%%%%%%%%%%%%%%%%%%
\subsection{Driver Development}
\label{sec:driver}

The \textsf{childdoc} mechanism can also be use for the development
of definition files such as \LaTeX{} styles or classes.
This case differs from the above setup with multiple parts
included by |\include| in that no |\includeonly| should be invoked.
This can be achieved by starting the include file
(before |\ProvidesPackage|) with:
%
\begin{center}
\begin{tabular}{l}
|\input{childdoc.def}|\\
|\childdocforward{|\textit{main}|}|\\
\end{tabular}
\end{center}
%
or alternatively with:
%
\begin{center}
\begin{tabular}{l}
|\input{childdoc.def}|\\
|\childdocby{|\textit{main}|}|\\
\end{tabular}
\end{center}
%
Both forms have slightly different effects as described above.
The main file is prepared as usual, see \secref{sec:include}.

%%%%%%%%%%%%%%%%%%%%%%%%%%%%%%%%%%%%%%%%%%%%%%%%%%%%%%%%%%%%%%%%%%%%%%%%%%%%%%%%
\subsection{Legacy Detection}
\label{sec:detection}

The directive |\childdocmain| in the main file can detect
whether the complete document or merely a child is to be compiled
even without using the directive |\childdocof|.
This method is deprecated because it is less robust
and there is no compelling reason to use it;
it is merely provided for backward compatibility
and it may be removed in future versions.

If the detection mechanism is to be used,
it is mandatory to correctly specify
the filename of the main file as the argument of |\childdocmain|:
%
\begin{center}
\begin{tabular}{l}
|\input{childdoc.def}|\\
|\childdocmain{|\textit{main}|}|\\
\end{tabular}
\end{center}
%
If |\jobname| does not match the argument \textit{main} of |\childdocmain|,
it is assumed that |\jobname| points to the child file to be compiled.
When using |\childdocmain| with the main file specified as argument,
it suffices to start a child file
with just |\input{|\textit{main}|}|
without loading of the package and using |\childdocof|.
If instead all processing is done
with the appropriate \textsf{childdoc} directives,
the argument of \textit{main} of |\childdocmain| can be empty.

An alternative version of the command line processing described
in \secref{sec:commandline} using the detection mechanism reads:
%
\begin{center}
|... -jobname "|\textit{target}|" "|[\textit{flags}]%
[|\def\jobname{|\textit{dest}|}|]|\input{|\textit{main}|}"|
\end{center}

%%%%%%%%%%%%%%%%%%%%%%%%%%%%%%%%%%%%%%%%%%%%%%%%%%%%%%%%%%%%%%%%%%%%%%%%%%%%%%%%
\subsection{Manual Code}
\label{sec:manual}

In case one cannot be certain whether the definitions file |childdoc.def|
is installed on the target \TeX{} distribution
and one prefers not to ship it,
it is conceivable to paste a few relevant commands into the sources.

To that end, drop all statements |\input{childdoc.def}|
and perform the replacements as outlined below.
Instead of |\childdocmain{|\textit{main}|}| add the following code
to the top of the main file:
%
\begin{center}
\begin{tabular}{l}
|\||ifdefined\childdocname\endinput\||fi\newif\ifchilddoc|\\
|\edef\childdocname{\scantokens\expandafter{\jobname\noexpand}}|\\
|\def\childdocmain{|\textit{main}|}\||ifx\childdocmain\childdocname\||else|\\
|\childdoctrue\includeonly{\childdocname}\let\jobname\childdocmain\||fi|\\
\end{tabular}
\end{center}
%
Instead of |\childdocof{|\textit{main}|}| just include the main file
at the top of each child file:
%
\begin{center}
|\input{|\textit{main}|}|
\end{center}
%
A simple redirection |\childdocforward{|\textit{dest}|}| is achieved by:
%
\begin{center}
|\def\jobname{|\textit{dest}|}\input{\jobname}|
\end{center}
%
The redirection with prefix
|\childdocforwardprefix[|\textit{prefix}|]{|\textit{dest}|}|
is accomplished by:
%
\begin{center}
\begin{tabular}{l}
|{\edef\jobname{\scantokens\expandafter{\jobname\noexpand}}|\\
|\def\redirectjob |\textit{prefix}|#1~~~{\gdef\jobname{|\textit{dest}|#1}}|\\
|\expandafter\redirectjob\jobname~~~}\input{\jobname}|
\end{tabular}
\end{center}

In an alternative approach,
child documents can be compiled by a specific command line
without additional code or specific definitions:
%
\begin{center}
|... -jobname "|\textit{target}|" "|[\textit{flags}]%
|\includeonly{|\textit{dest}|}\input{|\textit{main}|}"|
\end{center}
%

%%%%%%%%%%%%%%%%%%%%%%%%%%%%%%%%%%%%%%%%%%%%%%%%%%%%%%%%%%%%%%%%%%%%%%%%%%%%%%%%
%%%%%%%%%%%%%%%%%%%%%%%%%%%%%%%%%%%%%%%%%%%%%%%%%%%%%%%%%%%%%%%%%%%%%%%%%%%%%%%%
\section{Information}

%%%%%%%%%%%%%%%%%%%%%%%%%%%%%%%%%%%%%%%%%%%%%%%%%%%%%%%%%%%%%%%%%%%%%%%%%%%%%%%%
\subsection{Copyright}

Copyright \copyright{} 2017--2018 Niklas Beisert

This work may be distributed and/or modified under the
conditions of the \LaTeX{} Project Public License, either version 1.3
of this license or (at your option) any later version.
The latest version of this license is in
  \url{http://www.latex-project.org/lppl.txt}
and version 1.3 or later is part of all distributions of \LaTeX{}
version 2005/12/01 or later.

This work has the LPPL maintenance status `maintained'.

The Current Maintainer of this work is Niklas Beisert.

This work consists of the files |README.txt|, |childdoc.ins| and |childdoc.dtx|
as well as the derived files |childdoc.def|, |cdocsamp.tex|
with |cdocsch1.tex|, |cdocsch2.tex|, |cdocspt3.tex|, |cdocspt4.tex|,
|cdocsdrf.tex|, |cdocsfn1.tex|, |cdocsfn2.tex|
as well as |childdoc.pdf|.

%%%%%%%%%%%%%%%%%%%%%%%%%%%%%%%%%%%%%%%%%%%%%%%%%%%%%%%%%%%%%%%%%%%%%%%%%%%%%%%%
\subsection{Files and Installation}

The package consists of the files:
%
\begin{center}
\begin{tabular}{ll}
    |README.txt|   & readme file \\
    |childdoc.ins| & installation file \\
    |childdoc.dtx| & source file \\
    |childdoc.def| & definition file \\
    |cdocsamp.tex| & sample main file \\
    |cdocsch1.tex| & sample include file \\
    |cdocsch2.tex| & sample include file \\
    |cdocspt3.tex| & sample part file \\
    |cdocspt4.tex| & sample part file \\
    |cdocsdrf.tex| & sample redirection file \\
    |cdocsfn1.tex| & sample redirection file \\
    |cdocsfn2.tex| & sample redirection file \\
    |childdoc.pdf| & manual
\end{tabular}
\end{center}
%
The distribution consists of the files
|README.txt|, |childdoc.ins| and |childdoc.dtx|.
%
\begin{itemize}
\item
Run (pdf)\LaTeX{} on |childdoc.dtx|
to compile the manual |childdoc.pdf| (this file).
\item
Run \LaTeX{} on |childdoc.ins| to create the definitions file |childdoc.def|
and the sample |cdocsamp.tex| with include files
|cdocsch1.tex|, |cdocsch2.tex|, |cdocspt3.tex|, |cdocspt4.tex|,
|cdocsdrf.tex|, |cdocsfn1.tex|, |cdocsfn2.tex|.
Then copy the file |childdoc.def| to an appropriate directory of your \LaTeX{}
distribution, e.g.\ \textit{texmf-root}|/tex/latex/childdoc|.
\end{itemize}

%%%%%%%%%%%%%%%%%%%%%%%%%%%%%%%%%%%%%%%%%%%%%%%%%%%%%%%%%%%%%%%%%%%%%%%%%%%%%%%%
\subsection{Related CTAN Packages}

There are several other packages which offer a similar functionality:
%
\begin{itemize}
\item
The packages
\href{http://ctan.org/pkg/docmute}{\textsf{docmute}},
\href{http://ctan.org/pkg/includex}{\textsf{includex}} and
\href{http://ctan.org/pkg/standalone}{\textsf{standalone}}
provide commands to include only the document body of
a child file thus allowing both files to be compiled individually.
\item
The packages \href{http://ctan.org/pkg/subdocs}{\textsf{subdocs}}
and \href{http://ctan.org/pkg/subfiles}{\textsf{subfiles}}
provide structures in which the main and child documents can be
encapsulated and allowing them to be compiled individually.
The inclusion mechanism is different from the conventional |\include|.
\item
The package \href{http://ctan.org/pkg/combine}{\textsf{combine}}
is an elaborate solution to combine several documents into one.
\end{itemize}
%
See also the CTAN topic \href{http://ctan.org/topic/subdocs}{\textsf{subdocs}}
for further related packages.
The present package differs from the above solutions in that
a document structure constructed with the conventional |\include| mechanism
just needs two extra commands at the top of every file
such that all constituent files can be compiled individually.

%%%%%%%%%%%%%%%%%%%%%%%%%%%%%%%%%%%%%%%%%%%%%%%%%%%%%%%%%%%%%%%%%%%%%%%%%%%%%%%%
%\subsection{Feature Suggestions}
%
%The following is a list of features which may be useful for future
%versions of this package:
%%
%\begin{itemize}
%\item
%\ldots
%\end{itemize}

%%%%%%%%%%%%%%%%%%%%%%%%%%%%%%%%%%%%%%%%%%%%%%%%%%%%%%%%%%%%%%%%%%%%%%%%%%%%%%%%
\subsection{Revision History}

%%%%%%%%%%%%%%%%%%%%%%%%%%%%%%%%%%%%%%%%
\paragraph{v2.0:} 2018/12/30

\begin{itemize}
\item
immediate forward processing
\item
added |\childdocby| mechanism
\item
manual restructured
\end{itemize}

%%%%%%%%%%%%%%%%%%%%%%%%%%%%%%%%%%%%%%%%
\paragraph{v1.6:} 2018/01/17

\begin{itemize}
\item
application for development of include files
\item
corrections to manual
\end{itemize}

%%%%%%%%%%%%%%%%%%%%%%%%%%%%%%%%%%%%%%%%
\paragraph{v1.5:} 2017/05/21

\begin{itemize}
\item
more complete structuring introduced
\item
|\childdocof| introduced
\item
|\childdoc| renamed to |\childdocmain|
\item
|\childredirect| renamed to |\childdocforward| and |\childdocforwardprefix|
and functionality expanded
\end{itemize}

%%%%%%%%%%%%%%%%%%%%%%%%%%%%%%%%%%%%%%%%
\paragraph{v1.0:} 2017/04/27

\begin{itemize}
\item
manual and install package
\item
first version published on CTAN
\end{itemize}

%%%%%%%%%%%%%%%%%%%%%%%%%%%%%%%%%%%%%%%%
\paragraph{v0.6:} 2017/04/26

\begin{itemize}
\item
redirection mechanism added
\end{itemize}

%%%%%%%%%%%%%%%%%%%%%%%%%%%%%%%%%%%%%%%%
\paragraph{v0.5:} 2017/04/26

\begin{itemize}
\item
functionality in definition file
\end{itemize}


%%%%%%%%%%%%%%%%%%%%%%%%%%%%%%%%%%%%%%%%%%%%%%%%%%%%%%%%%%%%%%%%%%%%%%%%%%%%%%%%
%%%%%%%%%%%%%%%%%%%%%%%%%%%%%%%%%%%%%%%%%%%%%%%%%%%%%%%%%%%%%%%%%%%%%%%%%%%%%%%%
%%%%%%%%%%%%%%%%%%%%%%%%%%%%%%%%%%%%%%%%%%%%%%%%%%%%%%%%%%%%%%%%%%%%%%%%%%%%%%%%
\appendix

\settowidth\MacroIndent{\rmfamily\scriptsize 000\ }

 \DocInput{childdoc.dtx}

\end{document}
%</driver>
% \fi
%
% %%%%%%%%%%%%%%%%%%%%%%%%%%%%%%%%%%%%%%%%%%%%%%%%%%%%%%%%%%%%%%%%%%%%%%%%%%%%%%
% %%%%%%%%%%%%%%%%%%%%%%%%%%%%%%%%%%%%%%%%%%%%%%%%%%%%%%%%%%%%%%%%%%%%%%%%%%%%%%
% \section{Sample}
%\iffalse
%<*samplemain>
%\fi
%
% The following presents a sample document
% with two chapters, two parts, a title page,
% a compile flag as well as three forwarding files to set the flag.
% It consists of eight |.tex| files:
% \begin{center}
% \begin{tabular}{ll}
% |cdocsamp.tex|&main file\\
% |cdocsch1.tex|&include file for chapter 1\\
% |cdocsch2.tex|&include file for chapter 2\\
% |cdocspt3.tex|&include file for part 3\\
% |cdocspt4.tex|&include file for part 4\\
% |cdocsdrf.tex|&forwarding file for main file in draft mode\\
% |cdocsfi1.tex|&forwarding file for final version of chapter 1\\
% |cdocsfi2.tex|&forwarding file for final version of chapter 2\\
% \end{tabular}
% \end{center}
% Each of the eight files can be compiled directly by the \LaTeX{} compiler.
%
% %%%%%%%%%%%%%%%%%%%%%%%%%%%%%%%%%%%%%%
% \paragraph{Main File.}
%
% The main file is called |cdocsamp.tex|.
%
% Load the \textsf{childdoc} definitions and
% declare the filename for the main document:
%    \begin{macrocode}
\input{childdoc.def}
\childdocmain{}
%    \end{macrocode}

% Optional override for |\version| flag:
%    \begin{macrocode}
%%\ifchilddoc\else\providecommand{\version}{draft}\fi
%    \end{macrocode}

% Define the default values for the |\version| flag
% (|final| for the main file and |draft| for childs):
%    \begin{macrocode}
\ifchilddoc
\providecommand{\version}{draft}
\else
\providecommand{\version}{final}
\fi
%    \end{macrocode}

% Load the standard document class:
%    \begin{macrocode}
\documentclass[12pt]{article}
%    \end{macrocode}

% Start the document body:
%    \begin{macrocode}
\begin{document}
%    \end{macrocode}

% Declare a title page.
% Print title, part of document being processed and version flag:
%    \begin{macrocode}
\addtocounter{page}{-1}
\begin{center}
{\LARGE\bfseries{}childdoc example\par}
\vspace{1cm}
\ifchilddoc
\ifchilddocmanual part\else chapter\fi:
`\childdocname' of `\childdocjob'\par
\else
main document: `\childdocjob'\par
\fi
version: \version\par
\end{center}
\newpage
%    \end{macrocode}

% Manually include selected file,
% otherwise process as usual:
%    \begin{macrocode}
\ifchilddocmanual
\section*{part `\childdocname'}
\input{\childdocname}
\else
%    \end{macrocode}

% Include the two chapters:
%    \begin{macrocode}
\include{cdocsch1}
\include{cdocsch2}
%    \end{macrocode}

% Include the two parts unless only chapters should be displayed:
%    \begin{macrocode}
\ifchilddoc\else
\section{part three}
\input{cdocspt3}
\section{part four}
\input{cdocspt4}
\fi
%    \end{macrocode}

% Process as usual until here:
%    \begin{macrocode}
\fi
%    \end{macrocode}

% End of document body:
%    \begin{macrocode}
\end{document}
%    \end{macrocode}
%\iffalse
%</samplemain>
%\fi
%
% %%%%%%%%%%%%%%%%%%%%%%%%%%%%%%%%%%%%%%
% \paragraph{Chapter Include Files.}
%
% The include files are called |cdocsch1.tex| and |cdocsch2.tex|.
%
%\iffalse
%<*samplechap1|samplechap2>
%\fi

% Optional override for |\version| flag:
%    \begin{macrocode}
%%\providecommand{\version}{final}
%    \end{macrocode}

% Include the main document:
%    \begin{macrocode}
\input{childdoc.def}
\childdocof{cdocsamp}
%    \end{macrocode}

%\iffalse
%</samplechap1|samplechap2>
%\fi
%
%\iffalse
%<*samplechap1>
%\fi
% Some text for chapter 1:
%    \begin{macrocode}
\section{one}
some text in chapter one
%    \end{macrocode}

%\iffalse
%</samplechap1>
%\fi
% Some text for chapter 2:
%\iffalse
%<*samplechap2>
%\fi
%    \begin{macrocode}
\section{two}
more text in chapter two
%    \end{macrocode}

%\iffalse
%</samplechap2>
%\fi
%
% %%%%%%%%%%%%%%%%%%%%%%%%%%%%%%%%%%%%%%
% \paragraph{Part Include Files.}
%
% The include files are called |cdocspt3.tex| and |cdocspt4.tex|.
%
%\iffalse
%<*samplepart3|samplepart4>
%\fi

% Optional override for |\version| flag:
%    \begin{macrocode}
%%\providecommand{\version}{final}
%    \end{macrocode}

% Include the main document:
%    \begin{macrocode}
\input{childdoc.def}
\childdocby{cdocsamp}
%    \end{macrocode}

%\iffalse
%</samplepart3|samplepart4>
%\fi
%
%\iffalse
%<*samplepart3>
%\fi
% Some text for part 3:
%    \begin{macrocode}
some text in part three
%    \end{macrocode}

%\iffalse
%</samplepart3>
%\fi
% Some text for part 4:
%\iffalse
%<*samplepart4>
%\fi
%    \begin{macrocode}
more text in part four
%    \end{macrocode}

%\iffalse
%</samplepart4>
%\fi
%
% %%%%%%%%%%%%%%%%%%%%%%%%%%%%%%%%%%%%%%
% \paragraph{Forwarding for a Complete Draft.}
%
% The following forwarding file |cdocsdrf.tex|
% compiles the main document in draft mode:
%\iffalse
%<*sampledraft>
%\fi
%    \begin{macrocode}
\def\version{draft}
\input{childdoc.def}
\childdocforward{cdocsamp}
%    \end{macrocode}

%\iffalse
%</sampledraft>
%\fi
%
% %%%%%%%%%%%%%%%%%%%%%%%%%%%%%%%%%%%%%%
% \paragraph{Forwarding for Final Version of the Chapters.}
%
% The following forwarding files |cdocsfn1.tex| and |cdocsfn2.tex|
% (with identical content)
% compile the final versions of the child documents
% |cdocsch1.tex| and |cdocsch2.tex|, respectively:
%\iffalse
%<*samplefinal>
%\fi
%    \begin{macrocode}
\def\version{final}
\input{childdoc.def}
\childdocforwardprefix[cdocsamp]{cdocsfn}{cdocsch}
%    \end{macrocode}

%\iffalse
%</samplefinal>
%\fi
%
% %%%%%%%%%%%%%%%%%%%%%%%%%%%%%%%%%%%%%%
% \paragraph{Command Line Processing.}
%
% The following three command lines generate the output files
% |cdocscld|, |cdocscl1| and |cdocscl2|
% which should be identical to
% |cdocsdrf|, |cdocsch1| and |cdocsfn2|, respectively:
% \begin{center}
% \begin{tabular}{l}
% |latex -jobname cdocscld \|\\
% |  "\def\version{draft}\input{childdoc.def}\childdocforward{cdocsamp}"|\\
% |latex -jobname cdocscl1 \|\\
% |  "\input{childdoc.def}\childdocforward[cdocsamp]{cdocsch1}"|\\
% |latex -jobname cdocscl2 \|\\
% |  "\def\version{final}\input{childdoc.def}\childdocforward{cdocsch2}"|
% \end{tabular}
% \end{center}
% Note that the trailing backslash on each first line
% merely continues the input to the second line
% (for convenient cut ant paste).
% Furthermore, the command |latex| can be replaced by any
% of its alternative versions such as |pdflatex|.
%
% %%%%%%%%%%%%%%%%%%%%%%%%%%%%%%%%%%%%%%%%%%%%%%%%%%%%%%%%%%%%%%%%%%%%%%%%%%%%%%
% %%%%%%%%%%%%%%%%%%%%%%%%%%%%%%%%%%%%%%%%%%%%%%%%%%%%%%%%%%%%%%%%%%%%%%%%%%%%%%
% \section{Implementation}
%\iffalse
%<*package>
%\fi
%
% This section describes the definitions file |childdoc.def|.

% The definitions cannot be loaded using |\usepackage| or |\RequirePackage|
% which has a mechanism to prevent loading a style file more than once.
% When loading the definitions by means of |\input|
% multiple instances have to be prevented manually:
%\iffalse
%This code needs to be before the `\ProvidesFile' directive
%which is defined at the beginning of this file.
%Therefore it is also placed there and commented out here.
%</package>
%<*discard>
%\fi
%    \begin{macrocode}
\ifdefined\childdocmain\endinput\fi
%    \end{macrocode}
%\iffalse
%</discard>
%<*package>
%\fi
%
% \macro{\ifchilddoc}
% \macro{\ifchilddocmanual}
% The conditional |\ifchilddoc| tells whether a
% child (true) or main (false) document is being compiled.
% The conditional |\ifchilddocmanual| tells whether
% the |\includeonly| mechanism is used (false) or
% the selection of child files must be performed manually (true).
% The definitions initialise to false:
%    \begin{macrocode}
\newif\ifchilddoc
\newif\ifchilddocmanual
%    \end{macrocode}

% \macro{\childdocname}
% \macro{\childdocjob}
% The macro |\childdocname| stores the name of the main document
% to be compiled. The macro |\childdocjob| stores the name of
% the document on which the \LaTeX{} compiler was originally invoked.
% The content of |\jobname| cannot be compared
% to filenames specified in the source due to different catcodes.
% The following code rescans |\jobname|, stores the result
% in |\childdocname| and saves a copy in |\childdocjob|:
%    \begin{macrocode}
\edef\childdocname{\scantokens\expandafter{\jobname\noexpand}}
\let\childdocjob\childdocname
%    \end{macrocode}

% \macro{\childdocdisable}
% The macro |\childdocdisable| prevents the main file
% from being processed more than once.
% At this stage, the main document command |\childdocmain|
% is assumed to be called once again where it should do nothing.
% Any subsequent call to it should prevent
% a secondary processing of the main document
% It overwrites the forwarding commands
% |\childdocof| and |\childdocforward|
% with empty macros to prevent further inclusions of the main document:
%    \begin{macrocode}
\newcommand{\childdocdisable}
{
  \renewcommand{\childdocmain}[1]{\renewcommand{\childdocmain}[1]{\endinput}}
  \renewcommand{\childdocof}[1]{}
  \renewcommand{\childdocby}[2][]{}
  \renewcommand{\childdocforward}[2][]{}
  \renewcommand{\childdocdisable}{}
}
%    \end{macrocode}

% \macro{\childdocmain}
% The macro |\childdocmain| is to be called at the top of the main file
% with nothing or the main filename (without extension) as argument.
% First, it breaks loops.
% If the argument is not empty and does not match |\childdocname|
% (which is set by the first inclusion of |childdoc.def|),
% |\ifchilddoc| is set to true, |\includeonly| is applied to the child file
% and |\jobname| is set to the main file
% (for proper handling of |.aux| files):
%    \begin{macrocode}
\newcommand{\childdocmain}[1]
{
  \childdocdisable\childdocmain{}
  \if?#1?\else
    \begingroup
      \def\childdoctmp{#1}
      \ifx\childdoctmp\childdocname
        \def\childdoctmp{}
      \else
        \def\childdoctmp
        {
          \childdoctrue
          \includeonly{\childdocname}
          \def\childdocjob{#1}
          \def\jobname{#1}
        }
      \fi
      \expandafter
    \endgroup
    \childdoctmp
  \fi
}
%    \end{macrocode}

% \macro{\childdocof}
% The command |\childdocof| redirects
% compilation to the main file |#1|.
%    \begin{macrocode}
\newcommand{\childdocof}[1]
{
  \childdocdisable
  \childdoctrue
  \includeonly{\childdocname}
  \def\jobname{#1}
  \def\childdocjob{#1}
  \input{#1}
}
%    \end{macrocode}

% \macro{\childdocby}
% The command |\childdocby| ....
%    \begin{macrocode}
\newcommand{\childdocby}[2][]
{
  \childdocdisable
  \childdoctrue
  \childdocmanualtrue
  \if?#1?\else
    \def\jobname{#2}
  \fi
  \def\childdocjob{#2}
  \input{#2}
  \endinput
}
%    \end{macrocode}

% \macro{\childdocforward}
% The command |\childdocforward| redirects
% compilation to the main file or
% (if the optional argument is given) a child file.
% Parameters are set as if the main file
% or a child file starting with |\childdocof| was compiled.
% Then compilation is handed over to the main file:
%    \begin{macrocode}
\newcommand{\childdocforward}[2][]
{
  \begingroup
    \if?#1?
      \def\childdoctmp
      {
        \def\childdocname{#2}
        \def\childdocjob{#2}
        \def\jobname{#2}
        \input{#2}
        \endinput
      }
    \else
      \def\childdoctmp
      {
        \childdocdisable
        \def\childdocname{#2}
        \childdoctrue
        \includeonly{#2}
        \def\childdocjob{#1}
        \def\jobname{#1}
        \input{#1}
        \endinput
      }
    \fi
    \expandafter
  \endgroup
  \childdoctmp
}
%    \end{macrocode}

% \macro{\childdocforwardprefix}
% The command |\childdocforwardprefix| redirects
% compilation to the main or a child file by means of a pattern.
% The prefix |#1| in the current filename is replaced by |#2|
% and the suffix of the current filename is kept
% (it is assumed that the filename does not contain the substring `|~~~|'
% which is used as a delimiter).
% Compilation is handed over to the new file by |\childdocforward|:
%    \begin{macrocode}
\newcommand{\childdocforwardprefix}[3][]
{
  \begingroup
    \def\childdocextract #2##1~~~{\def\childdoctmp{\childdocforward[#1]{#3##1}}}
    \expandafter\childdocextract\childdocname~~~
    \expandafter
  \endgroup
  \childdoctmp
}
%    \end{macrocode}

% \macro{\childdoc}
% The deprecated macro |\childdoc| is a legacy version of |\childdocmain|:
%    \begin{macrocode}
\newcommand{\childdoc}{\childdocmain}
%    \end{macrocode}

% \macro{\childdocredirect}
% The deprecated macro |\childdocredirect| is a legacy version
% of |\childdocforward| and |\childdocforwardprefix|:
%    \begin{macrocode}
\newcommand{\childdocredirect}[2][]
{
  \begingroup
    \if?#1?
      \def\childdoctmp{\childdocforward{#2}}
    \else
      \def\childdoctmp{\childdocforwardprefix{#1}{#2}}
    \fi
    \expandafter
  \endgroup
  \childdoctmp
}
%    \end{macrocode}

%\iffalse
%</package>
%\fi
%
\endinput
\childdocforward{cdocsamp}"|\\
% |latex -jobname cdocscl1 \|\\
% |  "% \iffalse
%
% childdoc.dtx Copyright (C) 2017-2018 Niklas Beisert
%
% This work may be distributed and/or modified under the
% conditions of the LaTeX Project Public License, either version 1.3
% of this license or (at your option) any later version.
% The latest version of this license is in
%   http://www.latex-project.org/lppl.txt
% and version 1.3 or later is part of all distributions of LaTeX
% version 2005/12/01 or later.
%
% This work has the LPPL maintenance status `maintained'.
%
% The Current Maintainer of this work is Niklas Beisert.
%
% This work consists of the files childdoc.dtx and childdoc.ins
% and the derived files childdoc.def and cdocsamp.tex with
% cdocsch1.tex, cdocsch2.tex, cdocsdrf.tex, cdocsfn1.tex, cdocsfn2.tex.
%
%<package>\ifdefined\childdocmain\endinput\fi
%<package>\ProvidesFile{childdoc.def}[2018/12/30 v2.0 child document driver]
%<samplemain>\ProvidesFile{cdocsamp.tex}[2018/12/30 v2.0 sample for childdoc]
%<*driver>
%\ProvidesFile{childdoc.drv}[2018/12/30 v2.0 childdoc reference manual file]
\PassOptionsToClass{10pt,a4paper}{article}
\documentclass{ltxdoc}

\usepackage[margin=35mm]{geometry}
\usepackage{hyperref}
\usepackage{hyperxmp}
\usepackage[usenames]{color}

\hypersetup{colorlinks=true}
\hypersetup{pdfstartview=FitH}
\hypersetup{pdfpagemode=UseNone}
\hypersetup{pdfsource={}}
\hypersetup{pdflang={en-UK}}
\hypersetup{pdfcopyright={Copyright 2017-2018 Niklas Beisert.
  This work may be distributed and/or modified under the
  conditions of the LaTeX Project Public License, either version 1.3
  of this license or (at your option) any later version.}}
\hypersetup{pdflicenseurl={http://www.latex-project.org/lppl.txt}}
\hypersetup{pdfcontactaddress={ETH Zurich, ITP, HIT K,
  Wolfgang-Pauli-Strasse 27}}
\hypersetup{pdfcontactpostcode={8093}}
\hypersetup{pdfcontactcity={Zurich}}
\hypersetup{pdfcontactcountry={Switzerland}}
\hypersetup{pdfcontactemail={nbeisert@itp.phys.ethz.ch}}
\hypersetup{pdfcontacturl={http://people.phys.ethz.ch/\xmptilde nbeisert/}}

\newcommand{\secref}[1]{\hyperref[#1]{section \ref*{#1}}}

\parskip1ex
\parindent0pt
\let\olditemize\itemize
\def\itemize{\olditemize\parskip0pt}

\begin{document}

\title{The \textsf{childdoc} Package}
\hypersetup{pdftitle={The childdoc Package}}
\author{Niklas Beisert\\[2ex]
  Institut f\"ur Theoretische Physik\\
  Eidgen\"ossische Technische Hochschule Z\"urich\\
  Wolfgang-Pauli-Strasse 27, 8093 Z\"urich, Switzerland\\[1ex]
  \href{mailto:nbeisert@itp.phys.ethz.ch}
  {\texttt{nbeisert@itp.phys.ethz.ch}}}
\hypersetup{pdfauthor={Niklas Beisert}}
\hypersetup{pdfsubject={Manual for the LaTeX2e Package childdoc}}
\date{30 December 2018, \textsf{v2.0}}
\maketitle

\begin{abstract}\noindent
\textsf{childdoc} is a \LaTeXe{} package
that enables the direct compilation
of document sections included by |\include|
to individual files.
\end{abstract}

\begingroup
\parskip0ex
\tableofcontents
\endgroup

%%%%%%%%%%%%%%%%%%%%%%%%%%%%%%%%%%%%%%%%%%%%%%%%%%%%%%%%%%%%%%%%%%%%%%%%%%%%%%%%
%%%%%%%%%%%%%%%%%%%%%%%%%%%%%%%%%%%%%%%%%%%%%%%%%%%%%%%%%%%%%%%%%%%%%%%%%%%%%%%%
\section{Introduction}

\LaTeX{} provides a mechanism to structure a large document (such as a book)
into a main file and several child files (containing the chapters)
using the |\include| command.
This mechanism is beneficial for documents
which span hundreds of pages in order to
make the source file(s) more manageable.
Moreover, compilation can be restricted to
selected child files by means of the |\includeonly| command.
The latter feature can be used to reduce the compilation time while editing
(this was significantly more useful in the earlier days of \LaTeX{})
or to generate a smaller document which is easier to navigate.
Another application of |\includeonly| is to generate
documents consisting of selected parts of the complete document.

However, there are a few drawbacks of the plain |\include| mechanism:
\begin{itemize}
\item
The child files cannot be compiled on their own,
they can only be compiled via the main file.
A naive editing environment
(such as a text editor with an option
to have the current file processed by \LaTeX)
may require one to switch to the main file before compiling;
attempting to compile the child file produces errors.
\item
The main file must be modified (each time)
to adjust the |\includeonly| command
to the present needs. This easily leaves the main file in a messy state.
\item
The generated document will always carry the filename
of the main document. This is inconvenient if
several child files are to be compiled and
to be kept for distribution.
\end{itemize}

The present package provides a simple interface
to make child files individually compilable by \LaTeX{}.
Compiling a child file then has the same effect as compiling
the main file with an |\includeonly| command
to select the appropriate child.
Moreover the generated document will carry the name of the child
rather than the main file.
This resolves all three above issues.

This feature is meant to make the editing of books,
thesis documents and lecture notes somewhat more convenient.
However, the package can also be used efficiently for
composing a series of documents (such as exercise sheets)
which are typically distributed individually.
It then assists the author in generating the individual documents
(potentially in different versions)
as well as a document containing the collected series.
Another application is in developing style files
or other kinds of included material
where compilation of the style file could redirect
to a sample or test file.

%%%%%%%%%%%%%%%%%%%%%%%%%%%%%%%%%%%%%%%%%%%%%%%%%%%%%%%%%%%%%%%%%%%%%%%%%%%%%%%%
%%%%%%%%%%%%%%%%%%%%%%%%%%%%%%%%%%%%%%%%%%%%%%%%%%%%%%%%%%%%%%%%%%%%%%%%%%%%%%%%
\section{Usage}

First of all, the package \textsf{childdoc} is \emph{not} a standard
\LaTeXe{} |.sty| style file! Therefore it needs to be invoked in
a non-standard way.

%%%%%%%%%%%%%%%%%%%%%%%%%%%%%%%%%%%%%%%%%%%%%%%%%%%%%%%%%%%%%%%%%%%%%%%%%%%%%%%%
\subsection{Included Files}
\label{sec:include}

%%%%%%%%%%%%%%%%%%%%%%%%%%%%%%%%%%%%%%%%
\DescribeMacro{\childdocmain}
To use the package, add the commands
\begin{center}
\begin{tabular}{l}
|\input{childdoc.def}|\\
|\childdocmain{}|\\
\end{tabular}
\end{center}
at the very top of the main \LaTeX{} file,
in particular \emph{before} the |\documentclass| statement!
The argument of |\childdocmain| should be left empty
(but it must be present).

%%%%%%%%%%%%%%%%%%%%%%%%%%%%%%%%%%%%%%%%
\DescribeMacro{\childdocof}
Furthermore, add the commands
\begin{center}
\begin{tabular}{l}
|\input{childdoc.def}|\\
|\childdocof{|\textit{main}|}|\\
\end{tabular}
\end{center}
at the top of every child file \textit{child}
which is included by |\include{|\textit{child}|}|
from within the main file
(or at least for those files to be compiled individually).
The argument \textit{main} must be the filename of the main file.

There are a couple of
considerations in setting up the main and child documents:

%%%%%%%%%%%%%%%%%%%%%%%%%%%%%%%%%%%%%%%%
\paragraph{Restrictions.}

Please note the following restrictions:
\begin{itemize}
\item
|\childdocmain| must be called with one argument \textit{main}
to ensure compatibility with earlier version of the package.
It must either be empty (|\childdocmain{}|)
or precisely match the filename of the main file in which it is specified.
See \secref{sec:detection} for further information.
\item
The filename \textit{main} must be specified without the |.tex| extension.
\item
The filename \textit{main} is case sensitive
(even in case-insensitive file systems)
due to internal string comparison.
\item
The argument \textit{main} should be fully expanded, it cannot be a macro.
\item
Subdirectories and special characters should be avoided in filenames.
\item
The command |\childdocmain{|\textit{main}|}| must be followed by a whitespace.
It should not be followed immediately by another command
or by a comment mark `|%|'.
This is because the \TeX{} parser reads the token immediately following
the argument of |\childdocmain| and puts it
at the beginning of every child section;
however, a white\-space is ignored.
\end{itemize}

%%%%%%%%%%%%%%%%%%%%%%%%%%%%%%%%%%%%%%%%
\paragraph{Content of Main File.}

It is advisable to place all content in the child files included by |\include|.
Any output contained in the main file will appear in all child documents
unless suppressed manually;
it cannot be suppressed automatically by the |\includeonly| directive
and thus should normally be avoided.
A method to include some content in the main file
by means of conditional processing is described in \secref{sec:conditional}.

%%%%%%%%%%%%%%%%%%%%%%%%%%%%%%%%%%%%%%%%
\paragraph{Page Numbering.}

When only a part of the document is compiled,
the appropriate numbering of pages
(as well as other status parameters)
is determined from the |.aux| files.
The latter contain information from previous passes.
However this information needs to propagate through
all intermediate child documents.
Therefore the page numbering in child documents may well
be inconsistent until the complete document is compiled at least once.

A useful (if unconventional) way to always ensure a consistent
page numbering is to restart the numbering in each child document
and denote the pages by `\textit{child}|.|\textit{page}'
where \textit{child} represents the chapter/section number of the child file.
This can be achieved by the command
|\numberwithin{page}{|\textit{child}|}|
of the \textsf{amsmath} package
where \textit{child} can be |chapter| or |section|
depending on the chosen structuring.
Alternatively, one can modify the macro |\thepage| appropriately
and reset the counter |page| at the start of each child file.

%%%%%%%%%%%%%%%%%%%%%%%%%%%%%%%%%%%%%%%%%%%%%%%%%%%%%%%%%%%%%%%%%%%%%%%%%%%%%%%%
\subsection{Conditional Processing}
\label{sec:conditional}

The package provides a mechanism to compile different versions
of a document. To customise the versions further some conditional processing
can come in handy to distinguish which version is being compiled.
The package provides two macros to describe the compilation context:

%%%%%%%%%%%%%%%%%%%%%%%%%%%%%%%%%%%%%%%%
\DescribeMacro{\ifchilddoc}
The conditional |\ifchilddoc| distinguishes between the compilation of
child documents and the main document:
%
\begin{center}
|\ifchilddoc |\textit{child-code}| |[|\||else |\textit{main-code}]| \||fi|
\end{center}

%%%%%%%%%%%%%%%%%%%%%%%%%%%%%%%%%%%%%%%%
\DescribeMacro{\childdocname}
\DescribeMacro{\childdocjob}
The macro |\childdocname| contains the filename (without extension)
of the main or child file being processed.
Note that |\childdocjob| will always contain the name of the main file.

%%%%%%%%%%%%%%%%%%%%%%%%%%%%%%%%%%%%%%%%
\paragraph{Title Page.}

Conditional processing can be used to include a title or banner page
in the main document when proper precautions are taken.
Importantly, the code in the main file should ensure that the page counter
(as well as other status parameters which are stored in the |.aux| files)
takes the same value after the conditional processing.
Otherwise the page numbers may take divergent values
depending on which part is compiled.

For example, a title page could be declared by:
%
\begin{center}
\begin{tabular}{l}
|\ifchilddoc\||else|\\
|\addtocounter{page}{-1}|\\
\textit{code for title page}\\
|\newpage|\\
|\||fi|
\end{tabular}
\end{center}
%
A banner page for the child documents can be generated by:
%
\begin{center}
\begin{tabular}{l}
|\ifchilddoc|\\
|\addtocounter{page}{-1}|\\
\textit{code for banner page}\\
|\newpage|\\
|\||fi|
\end{tabular}
\end{center}
%
Here one could write a message such as:
\begin{center}
|This is the part \childdocname{} of \childdocjob{}.|
\end{center}

%%%%%%%%%%%%%%%%%%%%%%%%%%%%%%%%%%%%%%%%%%%%%%%%%%%%%%%%%%%%%%%%%%%%%%%%%%%%%%%%
\subsection{Flags}
\label{sec:flags}

The package makes it easy to generate different versions
of the main or child documents.
To this end compilation flags can be defined
and assigned different default values.
They will be particularly useful in conjunction
with the forwarding mechanism described in \secref{sec:forward}.

For example, it may be useful to have a flag |\version|
which can be set to |draft| or |final|.
The document source will contain some conditional code
depending on the value of |\version|.
Suppose further, the flag should default to |final| for the main file
and to |draft| for child files
which is a natural assignment for editing the document.
This is achieved by placing the following code
in the preamble of the main document
(below the |\childdocmain| directive):
%
\begin{center}
\begin{tabular}{l}
|\ifchilddoc|\\
|\providecommand{\version}{draft}|\\
|\||else|\\
|\providecommand{\version}{final}|\\
|\||fi|
\end{tabular}
\end{center}
%
The definition by |\providecommand| makes sure
that previous definitions are not overwritten.
Further statements |\providecommand{\version}{...}|
can thus be added before the above code to override it.

For the main file, one might add a line
(between |\childdocmain| and the above block)
%
\begin{center}
|%\ifchilddoc\||else\providecommand{\version}{draft}\||fi|
\end{center}
%
which can be uncommented to produce a draft version.
Likewise one can add a line to the very top of a child file
(above the |\childdocof{|\textit{main}|}| directive)
%
\begin{center}
|%\providecommand{\version}{final}|
\end{center}
%
which can be uncommented to produce the final version of this child document.

%%%%%%%%%%%%%%%%%%%%%%%%%%%%%%%%%%%%%%%%%%%%%%%%%%%%%%%%%%%%%%%%%%%%%%%%%%%%%%%%
\subsection{Forwarding}
\label{sec:forward}

Different versions of the main or child documents
using compilation flags as described in \secref{sec:flags}
can be (permanently) stored in different files
for convenient compilation, viewing and distribution.
To this end, the package defines a command
to pass on compilation to a different file:

%%%%%%%%%%%%%%%%%%%%%%%%%%%%%%%%%%%%%%%%
\DescribeMacro{\childdocforward}
The command |\childdocforward| redirects processing to
another source file:
%
\begin{center}
\begin{tabular}{l}
|\input{childdoc.def}|\\
|\childdocforward[|\textit{main}|]{|\textit{dest}|}|\\
\end{tabular}
\end{center}
%
The argument \textit{dest} is the destination file
(without extension).
It should be the main file or one of the child files.
Note that further \textsf{childdoc} directives
such as |\childdocof| and |\childdocforward|
in the indicated file will be processed in this form.
The optional argument \textit{main}
passes on directly to the main file \textit{main}
while pretending to compile the child \textit{dest}.
This form behaves as if \textit{dest}
issues |\childdocof{|\textit{main}|}| right away,
and no further \textsf{childdoc} directives will be processed.

%%%%%%%%%%%%%%%%%%%%%%%%%%%%%%%%%%%%%%%%
\DescribeMacro{\...prefix}
In the alternative form |\childdocforwardprefix|,
%
\begin{center}
\begin{tabular}{l}
|\input{childdoc.def}|\\
|\childdocforwardprefix[|\textit{main}|]{|\textit{prefix}|}{|\textit{dest}|}|
\end{tabular}
\end{center}
%
the destination file is determined by a pattern
depending on the current file:
To make this work, the current file must be called
`{\textit{prefix}\hspace{0.2em}\textit{suffix}}'
with \textit{prefix} matching precisely the argument.
Processing is then passed on to the file
`{\textit{dest}\hspace{0.2em}\textit{suffix}}'.
Surely, the same effect is achieved by
directly specifying the
argument `{\textit{dest}\hspace{0.2em}\textit{suffix}}'
in the first form.
However, that requires to set up a different file
for each child. With the alternative form of the command
all these files can have exactly the same content
which simplifies setting them up and maintaining them.

For example, the following file |draft.tex|
with a compilation flag |\version| as described in \secref{sec:flags}
compiles the main document as a draft:
%
\begin{center}
\begin{tabular}{l}
|\def\version{draft}|\\
|\input{childdoc.def}|\\
|\childdocforward{|\textit{main}|}|
\end{tabular}
\end{center}
%
Likewise, the following files |final|\textit{nn}|.tex|
compile the final version of the child document
|child|\textit{nn}|.tex|:
%
\begin{center}
\begin{tabular}{l}
|\def\version{final}|\\
|\input{childdoc.def}|\\
|\childdocforwardprefix{final}{child}|
\end{tabular}
\end{center}
%

Note that when several versions of a main file and/or of each child file
are to be generated, it may be convenient to set up a |Makefile| or
shell script to automatise the process.

%%%%%%%%%%%%%%%%%%%%%%%%%%%%%%%%%%%%%%%%%%%%%%%%%%%%%%%%%%%%%%%%%%%%%%%%%%%%%%%%
\subsection{Command Line Processing}
\label{sec:commandline}

The effect of redirection files can also be achieved by invoking
the \LaTeX{} compiler with a more elaborate command line.
Most conveniently this should be done as part
of a shell script or a |Makefile|.

When using \textsf{childdoc} in the main file, the following
command lines effectively perform a redirection
(note that depending on the shell being used,
backslashes may have to be doubled: `|\|' $\to$ `|\\|'):
%
\begin{center}
|... -jobname "|\textit{target}|" |\\|"|[\textit{flags}]%
|\input{childdoc.def}\childdocforward[|\textit{main}|]{|\textit{dest}|}"|
\end{center}
%
Here \textit{target} is the name of the output file,
\textit{main} is the name of the main file
and \textit{dest} is the name of the main or child file to be processed
(all filenames without extensions).
The optional argument \textit{main} can be omitted
if \textit{main} matches \textit{dest}.
Optionally, compilation \textit{flags} can be defined via |\def| commands.
This command line makes the \TeX{} engine believe
it is compiling the file \textit{target}
whose content is specified as the latter parameter.
The provided code then forwards the processing to
\textit{main} or \textit{dest} as described in \secref{sec:forward}.

%%%%%%%%%%%%%%%%%%%%%%%%%%%%%%%%%%%%%%%%%%%%%%%%%%%%%%%%%%%%%%%%%%%%%%%%%%%%%%%%
\subsection{Include by Input}
\label{sec:input}

Including child documents by |\include| has some restrictions by design.
Most notably, the content of a child document always occupies
its own set of pages; pages cannot be shared between child documents.
Usually, this behaviour makes perfect sense
because each child document contain an essential part of the document.
However, in some situations it may be desirable to compose
a document from a collection of parts
without having mandatory page breaks between then.
For this case, the package
provides a mechanism to include parts
by |\input| which can also be processed individually.
However, by construction this mechanism
requires manual handling of the content to be output.

%%%%%%%%%%%%%%%%%%%%%%%%%%%%%%%%%%%%%%%%
\DescribeMacro{\ifchilddocmanual}
The main file should be prepared as usual, see \secref{sec:include}.
However, the document body must make a distinction
between processing of an individual part and of the main document, e.g.:
%
\begin{center}
\begin{tabular}{l}
|\ifchilddocmanual|\\
|\input{\childdocname}|\\
|\||else|\\
\textit{document body with }|\input{|\textit{part}|}|\\
|\||fi|
\end{tabular}
\end{center}
%
The conditional |\ifchilddocmanual| is true whenever
a part to be included by |\input| is being compiled,
and the name of the part is stored in |\childdocname|.

%%%%%%%%%%%%%%%%%%%%%%%%%%%%%%%%%%%%%%%%
\DescribeMacro{\childdocby}
Each part to be included by |\input| should start with:
%
\begin{center}
\begin{tabular}{l}
|\input{childdoc.def}|\\
|\childdocby{|\textit{main}|}|\\
\end{tabular}
\end{center}
%
The directive |\childdocby| is similar to |\childdocof|
described in \secref{sec:include},
but the subsequent selection of content must be done manually.
To that end, both |\ifchilddoc| and |\ifchilddocmanual|
will be true upon processing of a part,
and the name of the part is stored in |\childdocname|.
Note that |\jobname| will be set to the filename of the current part
so that each part receives an individual |.aux| file
that does not interfere with the |.aux| file(s) of the main document.
This behaviour can be altered by the alternative form
|\childdocby[*]{|\textit{main}|}| (with a non-empty optional argument)
which uses the |.aux| file of the main document
by setting |\jobname| to \textit{main}.

%%%%%%%%%%%%%%%%%%%%%%%%%%%%%%%%%%%%%%%%%%%%%%%%%%%%%%%%%%%%%%%%%%%%%%%%%%%%%%%%
\subsection{Driver Development}
\label{sec:driver}

The \textsf{childdoc} mechanism can also be use for the development
of definition files such as \LaTeX{} styles or classes.
This case differs from the above setup with multiple parts
included by |\include| in that no |\includeonly| should be invoked.
This can be achieved by starting the include file
(before |\ProvidesPackage|) with:
%
\begin{center}
\begin{tabular}{l}
|\input{childdoc.def}|\\
|\childdocforward{|\textit{main}|}|\\
\end{tabular}
\end{center}
%
or alternatively with:
%
\begin{center}
\begin{tabular}{l}
|\input{childdoc.def}|\\
|\childdocby{|\textit{main}|}|\\
\end{tabular}
\end{center}
%
Both forms have slightly different effects as described above.
The main file is prepared as usual, see \secref{sec:include}.

%%%%%%%%%%%%%%%%%%%%%%%%%%%%%%%%%%%%%%%%%%%%%%%%%%%%%%%%%%%%%%%%%%%%%%%%%%%%%%%%
\subsection{Legacy Detection}
\label{sec:detection}

The directive |\childdocmain| in the main file can detect
whether the complete document or merely a child is to be compiled
even without using the directive |\childdocof|.
This method is deprecated because it is less robust
and there is no compelling reason to use it;
it is merely provided for backward compatibility
and it may be removed in future versions.

If the detection mechanism is to be used,
it is mandatory to correctly specify
the filename of the main file as the argument of |\childdocmain|:
%
\begin{center}
\begin{tabular}{l}
|\input{childdoc.def}|\\
|\childdocmain{|\textit{main}|}|\\
\end{tabular}
\end{center}
%
If |\jobname| does not match the argument \textit{main} of |\childdocmain|,
it is assumed that |\jobname| points to the child file to be compiled.
When using |\childdocmain| with the main file specified as argument,
it suffices to start a child file
with just |\input{|\textit{main}|}|
without loading of the package and using |\childdocof|.
If instead all processing is done
with the appropriate \textsf{childdoc} directives,
the argument of \textit{main} of |\childdocmain| can be empty.

An alternative version of the command line processing described
in \secref{sec:commandline} using the detection mechanism reads:
%
\begin{center}
|... -jobname "|\textit{target}|" "|[\textit{flags}]%
[|\def\jobname{|\textit{dest}|}|]|\input{|\textit{main}|}"|
\end{center}

%%%%%%%%%%%%%%%%%%%%%%%%%%%%%%%%%%%%%%%%%%%%%%%%%%%%%%%%%%%%%%%%%%%%%%%%%%%%%%%%
\subsection{Manual Code}
\label{sec:manual}

In case one cannot be certain whether the definitions file |childdoc.def|
is installed on the target \TeX{} distribution
and one prefers not to ship it,
it is conceivable to paste a few relevant commands into the sources.

To that end, drop all statements |\input{childdoc.def}|
and perform the replacements as outlined below.
Instead of |\childdocmain{|\textit{main}|}| add the following code
to the top of the main file:
%
\begin{center}
\begin{tabular}{l}
|\||ifdefined\childdocname\endinput\||fi\newif\ifchilddoc|\\
|\edef\childdocname{\scantokens\expandafter{\jobname\noexpand}}|\\
|\def\childdocmain{|\textit{main}|}\||ifx\childdocmain\childdocname\||else|\\
|\childdoctrue\includeonly{\childdocname}\let\jobname\childdocmain\||fi|\\
\end{tabular}
\end{center}
%
Instead of |\childdocof{|\textit{main}|}| just include the main file
at the top of each child file:
%
\begin{center}
|\input{|\textit{main}|}|
\end{center}
%
A simple redirection |\childdocforward{|\textit{dest}|}| is achieved by:
%
\begin{center}
|\def\jobname{|\textit{dest}|}\input{\jobname}|
\end{center}
%
The redirection with prefix
|\childdocforwardprefix[|\textit{prefix}|]{|\textit{dest}|}|
is accomplished by:
%
\begin{center}
\begin{tabular}{l}
|{\edef\jobname{\scantokens\expandafter{\jobname\noexpand}}|\\
|\def\redirectjob |\textit{prefix}|#1~~~{\gdef\jobname{|\textit{dest}|#1}}|\\
|\expandafter\redirectjob\jobname~~~}\input{\jobname}|
\end{tabular}
\end{center}

In an alternative approach,
child documents can be compiled by a specific command line
without additional code or specific definitions:
%
\begin{center}
|... -jobname "|\textit{target}|" "|[\textit{flags}]%
|\includeonly{|\textit{dest}|}\input{|\textit{main}|}"|
\end{center}
%

%%%%%%%%%%%%%%%%%%%%%%%%%%%%%%%%%%%%%%%%%%%%%%%%%%%%%%%%%%%%%%%%%%%%%%%%%%%%%%%%
%%%%%%%%%%%%%%%%%%%%%%%%%%%%%%%%%%%%%%%%%%%%%%%%%%%%%%%%%%%%%%%%%%%%%%%%%%%%%%%%
\section{Information}

%%%%%%%%%%%%%%%%%%%%%%%%%%%%%%%%%%%%%%%%%%%%%%%%%%%%%%%%%%%%%%%%%%%%%%%%%%%%%%%%
\subsection{Copyright}

Copyright \copyright{} 2017--2018 Niklas Beisert

This work may be distributed and/or modified under the
conditions of the \LaTeX{} Project Public License, either version 1.3
of this license or (at your option) any later version.
The latest version of this license is in
  \url{http://www.latex-project.org/lppl.txt}
and version 1.3 or later is part of all distributions of \LaTeX{}
version 2005/12/01 or later.

This work has the LPPL maintenance status `maintained'.

The Current Maintainer of this work is Niklas Beisert.

This work consists of the files |README.txt|, |childdoc.ins| and |childdoc.dtx|
as well as the derived files |childdoc.def|, |cdocsamp.tex|
with |cdocsch1.tex|, |cdocsch2.tex|, |cdocspt3.tex|, |cdocspt4.tex|,
|cdocsdrf.tex|, |cdocsfn1.tex|, |cdocsfn2.tex|
as well as |childdoc.pdf|.

%%%%%%%%%%%%%%%%%%%%%%%%%%%%%%%%%%%%%%%%%%%%%%%%%%%%%%%%%%%%%%%%%%%%%%%%%%%%%%%%
\subsection{Files and Installation}

The package consists of the files:
%
\begin{center}
\begin{tabular}{ll}
    |README.txt|   & readme file \\
    |childdoc.ins| & installation file \\
    |childdoc.dtx| & source file \\
    |childdoc.def| & definition file \\
    |cdocsamp.tex| & sample main file \\
    |cdocsch1.tex| & sample include file \\
    |cdocsch2.tex| & sample include file \\
    |cdocspt3.tex| & sample part file \\
    |cdocspt4.tex| & sample part file \\
    |cdocsdrf.tex| & sample redirection file \\
    |cdocsfn1.tex| & sample redirection file \\
    |cdocsfn2.tex| & sample redirection file \\
    |childdoc.pdf| & manual
\end{tabular}
\end{center}
%
The distribution consists of the files
|README.txt|, |childdoc.ins| and |childdoc.dtx|.
%
\begin{itemize}
\item
Run (pdf)\LaTeX{} on |childdoc.dtx|
to compile the manual |childdoc.pdf| (this file).
\item
Run \LaTeX{} on |childdoc.ins| to create the definitions file |childdoc.def|
and the sample |cdocsamp.tex| with include files
|cdocsch1.tex|, |cdocsch2.tex|, |cdocspt3.tex|, |cdocspt4.tex|,
|cdocsdrf.tex|, |cdocsfn1.tex|, |cdocsfn2.tex|.
Then copy the file |childdoc.def| to an appropriate directory of your \LaTeX{}
distribution, e.g.\ \textit{texmf-root}|/tex/latex/childdoc|.
\end{itemize}

%%%%%%%%%%%%%%%%%%%%%%%%%%%%%%%%%%%%%%%%%%%%%%%%%%%%%%%%%%%%%%%%%%%%%%%%%%%%%%%%
\subsection{Related CTAN Packages}

There are several other packages which offer a similar functionality:
%
\begin{itemize}
\item
The packages
\href{http://ctan.org/pkg/docmute}{\textsf{docmute}},
\href{http://ctan.org/pkg/includex}{\textsf{includex}} and
\href{http://ctan.org/pkg/standalone}{\textsf{standalone}}
provide commands to include only the document body of
a child file thus allowing both files to be compiled individually.
\item
The packages \href{http://ctan.org/pkg/subdocs}{\textsf{subdocs}}
and \href{http://ctan.org/pkg/subfiles}{\textsf{subfiles}}
provide structures in which the main and child documents can be
encapsulated and allowing them to be compiled individually.
The inclusion mechanism is different from the conventional |\include|.
\item
The package \href{http://ctan.org/pkg/combine}{\textsf{combine}}
is an elaborate solution to combine several documents into one.
\end{itemize}
%
See also the CTAN topic \href{http://ctan.org/topic/subdocs}{\textsf{subdocs}}
for further related packages.
The present package differs from the above solutions in that
a document structure constructed with the conventional |\include| mechanism
just needs two extra commands at the top of every file
such that all constituent files can be compiled individually.

%%%%%%%%%%%%%%%%%%%%%%%%%%%%%%%%%%%%%%%%%%%%%%%%%%%%%%%%%%%%%%%%%%%%%%%%%%%%%%%%
%\subsection{Feature Suggestions}
%
%The following is a list of features which may be useful for future
%versions of this package:
%%
%\begin{itemize}
%\item
%\ldots
%\end{itemize}

%%%%%%%%%%%%%%%%%%%%%%%%%%%%%%%%%%%%%%%%%%%%%%%%%%%%%%%%%%%%%%%%%%%%%%%%%%%%%%%%
\subsection{Revision History}

%%%%%%%%%%%%%%%%%%%%%%%%%%%%%%%%%%%%%%%%
\paragraph{v2.0:} 2018/12/30

\begin{itemize}
\item
immediate forward processing
\item
added |\childdocby| mechanism
\item
manual restructured
\end{itemize}

%%%%%%%%%%%%%%%%%%%%%%%%%%%%%%%%%%%%%%%%
\paragraph{v1.6:} 2018/01/17

\begin{itemize}
\item
application for development of include files
\item
corrections to manual
\end{itemize}

%%%%%%%%%%%%%%%%%%%%%%%%%%%%%%%%%%%%%%%%
\paragraph{v1.5:} 2017/05/21

\begin{itemize}
\item
more complete structuring introduced
\item
|\childdocof| introduced
\item
|\childdoc| renamed to |\childdocmain|
\item
|\childredirect| renamed to |\childdocforward| and |\childdocforwardprefix|
and functionality expanded
\end{itemize}

%%%%%%%%%%%%%%%%%%%%%%%%%%%%%%%%%%%%%%%%
\paragraph{v1.0:} 2017/04/27

\begin{itemize}
\item
manual and install package
\item
first version published on CTAN
\end{itemize}

%%%%%%%%%%%%%%%%%%%%%%%%%%%%%%%%%%%%%%%%
\paragraph{v0.6:} 2017/04/26

\begin{itemize}
\item
redirection mechanism added
\end{itemize}

%%%%%%%%%%%%%%%%%%%%%%%%%%%%%%%%%%%%%%%%
\paragraph{v0.5:} 2017/04/26

\begin{itemize}
\item
functionality in definition file
\end{itemize}


%%%%%%%%%%%%%%%%%%%%%%%%%%%%%%%%%%%%%%%%%%%%%%%%%%%%%%%%%%%%%%%%%%%%%%%%%%%%%%%%
%%%%%%%%%%%%%%%%%%%%%%%%%%%%%%%%%%%%%%%%%%%%%%%%%%%%%%%%%%%%%%%%%%%%%%%%%%%%%%%%
%%%%%%%%%%%%%%%%%%%%%%%%%%%%%%%%%%%%%%%%%%%%%%%%%%%%%%%%%%%%%%%%%%%%%%%%%%%%%%%%
\appendix

\settowidth\MacroIndent{\rmfamily\scriptsize 000\ }

 \DocInput{childdoc.dtx}

\end{document}
%</driver>
% \fi
%
% %%%%%%%%%%%%%%%%%%%%%%%%%%%%%%%%%%%%%%%%%%%%%%%%%%%%%%%%%%%%%%%%%%%%%%%%%%%%%%
% %%%%%%%%%%%%%%%%%%%%%%%%%%%%%%%%%%%%%%%%%%%%%%%%%%%%%%%%%%%%%%%%%%%%%%%%%%%%%%
% \section{Sample}
%\iffalse
%<*samplemain>
%\fi
%
% The following presents a sample document
% with two chapters, two parts, a title page,
% a compile flag as well as three forwarding files to set the flag.
% It consists of eight |.tex| files:
% \begin{center}
% \begin{tabular}{ll}
% |cdocsamp.tex|&main file\\
% |cdocsch1.tex|&include file for chapter 1\\
% |cdocsch2.tex|&include file for chapter 2\\
% |cdocspt3.tex|&include file for part 3\\
% |cdocspt4.tex|&include file for part 4\\
% |cdocsdrf.tex|&forwarding file for main file in draft mode\\
% |cdocsfi1.tex|&forwarding file for final version of chapter 1\\
% |cdocsfi2.tex|&forwarding file for final version of chapter 2\\
% \end{tabular}
% \end{center}
% Each of the eight files can be compiled directly by the \LaTeX{} compiler.
%
% %%%%%%%%%%%%%%%%%%%%%%%%%%%%%%%%%%%%%%
% \paragraph{Main File.}
%
% The main file is called |cdocsamp.tex|.
%
% Load the \textsf{childdoc} definitions and
% declare the filename for the main document:
%    \begin{macrocode}
\input{childdoc.def}
\childdocmain{}
%    \end{macrocode}

% Optional override for |\version| flag:
%    \begin{macrocode}
%%\ifchilddoc\else\providecommand{\version}{draft}\fi
%    \end{macrocode}

% Define the default values for the |\version| flag
% (|final| for the main file and |draft| for childs):
%    \begin{macrocode}
\ifchilddoc
\providecommand{\version}{draft}
\else
\providecommand{\version}{final}
\fi
%    \end{macrocode}

% Load the standard document class:
%    \begin{macrocode}
\documentclass[12pt]{article}
%    \end{macrocode}

% Start the document body:
%    \begin{macrocode}
\begin{document}
%    \end{macrocode}

% Declare a title page.
% Print title, part of document being processed and version flag:
%    \begin{macrocode}
\addtocounter{page}{-1}
\begin{center}
{\LARGE\bfseries{}childdoc example\par}
\vspace{1cm}
\ifchilddoc
\ifchilddocmanual part\else chapter\fi:
`\childdocname' of `\childdocjob'\par
\else
main document: `\childdocjob'\par
\fi
version: \version\par
\end{center}
\newpage
%    \end{macrocode}

% Manually include selected file,
% otherwise process as usual:
%    \begin{macrocode}
\ifchilddocmanual
\section*{part `\childdocname'}
\input{\childdocname}
\else
%    \end{macrocode}

% Include the two chapters:
%    \begin{macrocode}
\include{cdocsch1}
\include{cdocsch2}
%    \end{macrocode}

% Include the two parts unless only chapters should be displayed:
%    \begin{macrocode}
\ifchilddoc\else
\section{part three}
\input{cdocspt3}
\section{part four}
\input{cdocspt4}
\fi
%    \end{macrocode}

% Process as usual until here:
%    \begin{macrocode}
\fi
%    \end{macrocode}

% End of document body:
%    \begin{macrocode}
\end{document}
%    \end{macrocode}
%\iffalse
%</samplemain>
%\fi
%
% %%%%%%%%%%%%%%%%%%%%%%%%%%%%%%%%%%%%%%
% \paragraph{Chapter Include Files.}
%
% The include files are called |cdocsch1.tex| and |cdocsch2.tex|.
%
%\iffalse
%<*samplechap1|samplechap2>
%\fi

% Optional override for |\version| flag:
%    \begin{macrocode}
%%\providecommand{\version}{final}
%    \end{macrocode}

% Include the main document:
%    \begin{macrocode}
\input{childdoc.def}
\childdocof{cdocsamp}
%    \end{macrocode}

%\iffalse
%</samplechap1|samplechap2>
%\fi
%
%\iffalse
%<*samplechap1>
%\fi
% Some text for chapter 1:
%    \begin{macrocode}
\section{one}
some text in chapter one
%    \end{macrocode}

%\iffalse
%</samplechap1>
%\fi
% Some text for chapter 2:
%\iffalse
%<*samplechap2>
%\fi
%    \begin{macrocode}
\section{two}
more text in chapter two
%    \end{macrocode}

%\iffalse
%</samplechap2>
%\fi
%
% %%%%%%%%%%%%%%%%%%%%%%%%%%%%%%%%%%%%%%
% \paragraph{Part Include Files.}
%
% The include files are called |cdocspt3.tex| and |cdocspt4.tex|.
%
%\iffalse
%<*samplepart3|samplepart4>
%\fi

% Optional override for |\version| flag:
%    \begin{macrocode}
%%\providecommand{\version}{final}
%    \end{macrocode}

% Include the main document:
%    \begin{macrocode}
\input{childdoc.def}
\childdocby{cdocsamp}
%    \end{macrocode}

%\iffalse
%</samplepart3|samplepart4>
%\fi
%
%\iffalse
%<*samplepart3>
%\fi
% Some text for part 3:
%    \begin{macrocode}
some text in part three
%    \end{macrocode}

%\iffalse
%</samplepart3>
%\fi
% Some text for part 4:
%\iffalse
%<*samplepart4>
%\fi
%    \begin{macrocode}
more text in part four
%    \end{macrocode}

%\iffalse
%</samplepart4>
%\fi
%
% %%%%%%%%%%%%%%%%%%%%%%%%%%%%%%%%%%%%%%
% \paragraph{Forwarding for a Complete Draft.}
%
% The following forwarding file |cdocsdrf.tex|
% compiles the main document in draft mode:
%\iffalse
%<*sampledraft>
%\fi
%    \begin{macrocode}
\def\version{draft}
\input{childdoc.def}
\childdocforward{cdocsamp}
%    \end{macrocode}

%\iffalse
%</sampledraft>
%\fi
%
% %%%%%%%%%%%%%%%%%%%%%%%%%%%%%%%%%%%%%%
% \paragraph{Forwarding for Final Version of the Chapters.}
%
% The following forwarding files |cdocsfn1.tex| and |cdocsfn2.tex|
% (with identical content)
% compile the final versions of the child documents
% |cdocsch1.tex| and |cdocsch2.tex|, respectively:
%\iffalse
%<*samplefinal>
%\fi
%    \begin{macrocode}
\def\version{final}
\input{childdoc.def}
\childdocforwardprefix[cdocsamp]{cdocsfn}{cdocsch}
%    \end{macrocode}

%\iffalse
%</samplefinal>
%\fi
%
% %%%%%%%%%%%%%%%%%%%%%%%%%%%%%%%%%%%%%%
% \paragraph{Command Line Processing.}
%
% The following three command lines generate the output files
% |cdocscld|, |cdocscl1| and |cdocscl2|
% which should be identical to
% |cdocsdrf|, |cdocsch1| and |cdocsfn2|, respectively:
% \begin{center}
% \begin{tabular}{l}
% |latex -jobname cdocscld \|\\
% |  "\def\version{draft}\input{childdoc.def}\childdocforward{cdocsamp}"|\\
% |latex -jobname cdocscl1 \|\\
% |  "\input{childdoc.def}\childdocforward[cdocsamp]{cdocsch1}"|\\
% |latex -jobname cdocscl2 \|\\
% |  "\def\version{final}\input{childdoc.def}\childdocforward{cdocsch2}"|
% \end{tabular}
% \end{center}
% Note that the trailing backslash on each first line
% merely continues the input to the second line
% (for convenient cut ant paste).
% Furthermore, the command |latex| can be replaced by any
% of its alternative versions such as |pdflatex|.
%
% %%%%%%%%%%%%%%%%%%%%%%%%%%%%%%%%%%%%%%%%%%%%%%%%%%%%%%%%%%%%%%%%%%%%%%%%%%%%%%
% %%%%%%%%%%%%%%%%%%%%%%%%%%%%%%%%%%%%%%%%%%%%%%%%%%%%%%%%%%%%%%%%%%%%%%%%%%%%%%
% \section{Implementation}
%\iffalse
%<*package>
%\fi
%
% This section describes the definitions file |childdoc.def|.

% The definitions cannot be loaded using |\usepackage| or |\RequirePackage|
% which has a mechanism to prevent loading a style file more than once.
% When loading the definitions by means of |\input|
% multiple instances have to be prevented manually:
%\iffalse
%This code needs to be before the `\ProvidesFile' directive
%which is defined at the beginning of this file.
%Therefore it is also placed there and commented out here.
%</package>
%<*discard>
%\fi
%    \begin{macrocode}
\ifdefined\childdocmain\endinput\fi
%    \end{macrocode}
%\iffalse
%</discard>
%<*package>
%\fi
%
% \macro{\ifchilddoc}
% \macro{\ifchilddocmanual}
% The conditional |\ifchilddoc| tells whether a
% child (true) or main (false) document is being compiled.
% The conditional |\ifchilddocmanual| tells whether
% the |\includeonly| mechanism is used (false) or
% the selection of child files must be performed manually (true).
% The definitions initialise to false:
%    \begin{macrocode}
\newif\ifchilddoc
\newif\ifchilddocmanual
%    \end{macrocode}

% \macro{\childdocname}
% \macro{\childdocjob}
% The macro |\childdocname| stores the name of the main document
% to be compiled. The macro |\childdocjob| stores the name of
% the document on which the \LaTeX{} compiler was originally invoked.
% The content of |\jobname| cannot be compared
% to filenames specified in the source due to different catcodes.
% The following code rescans |\jobname|, stores the result
% in |\childdocname| and saves a copy in |\childdocjob|:
%    \begin{macrocode}
\edef\childdocname{\scantokens\expandafter{\jobname\noexpand}}
\let\childdocjob\childdocname
%    \end{macrocode}

% \macro{\childdocdisable}
% The macro |\childdocdisable| prevents the main file
% from being processed more than once.
% At this stage, the main document command |\childdocmain|
% is assumed to be called once again where it should do nothing.
% Any subsequent call to it should prevent
% a secondary processing of the main document
% It overwrites the forwarding commands
% |\childdocof| and |\childdocforward|
% with empty macros to prevent further inclusions of the main document:
%    \begin{macrocode}
\newcommand{\childdocdisable}
{
  \renewcommand{\childdocmain}[1]{\renewcommand{\childdocmain}[1]{\endinput}}
  \renewcommand{\childdocof}[1]{}
  \renewcommand{\childdocby}[2][]{}
  \renewcommand{\childdocforward}[2][]{}
  \renewcommand{\childdocdisable}{}
}
%    \end{macrocode}

% \macro{\childdocmain}
% The macro |\childdocmain| is to be called at the top of the main file
% with nothing or the main filename (without extension) as argument.
% First, it breaks loops.
% If the argument is not empty and does not match |\childdocname|
% (which is set by the first inclusion of |childdoc.def|),
% |\ifchilddoc| is set to true, |\includeonly| is applied to the child file
% and |\jobname| is set to the main file
% (for proper handling of |.aux| files):
%    \begin{macrocode}
\newcommand{\childdocmain}[1]
{
  \childdocdisable\childdocmain{}
  \if?#1?\else
    \begingroup
      \def\childdoctmp{#1}
      \ifx\childdoctmp\childdocname
        \def\childdoctmp{}
      \else
        \def\childdoctmp
        {
          \childdoctrue
          \includeonly{\childdocname}
          \def\childdocjob{#1}
          \def\jobname{#1}
        }
      \fi
      \expandafter
    \endgroup
    \childdoctmp
  \fi
}
%    \end{macrocode}

% \macro{\childdocof}
% The command |\childdocof| redirects
% compilation to the main file |#1|.
%    \begin{macrocode}
\newcommand{\childdocof}[1]
{
  \childdocdisable
  \childdoctrue
  \includeonly{\childdocname}
  \def\jobname{#1}
  \def\childdocjob{#1}
  \input{#1}
}
%    \end{macrocode}

% \macro{\childdocby}
% The command |\childdocby| ....
%    \begin{macrocode}
\newcommand{\childdocby}[2][]
{
  \childdocdisable
  \childdoctrue
  \childdocmanualtrue
  \if?#1?\else
    \def\jobname{#2}
  \fi
  \def\childdocjob{#2}
  \input{#2}
  \endinput
}
%    \end{macrocode}

% \macro{\childdocforward}
% The command |\childdocforward| redirects
% compilation to the main file or
% (if the optional argument is given) a child file.
% Parameters are set as if the main file
% or a child file starting with |\childdocof| was compiled.
% Then compilation is handed over to the main file:
%    \begin{macrocode}
\newcommand{\childdocforward}[2][]
{
  \begingroup
    \if?#1?
      \def\childdoctmp
      {
        \def\childdocname{#2}
        \def\childdocjob{#2}
        \def\jobname{#2}
        \input{#2}
        \endinput
      }
    \else
      \def\childdoctmp
      {
        \childdocdisable
        \def\childdocname{#2}
        \childdoctrue
        \includeonly{#2}
        \def\childdocjob{#1}
        \def\jobname{#1}
        \input{#1}
        \endinput
      }
    \fi
    \expandafter
  \endgroup
  \childdoctmp
}
%    \end{macrocode}

% \macro{\childdocforwardprefix}
% The command |\childdocforwardprefix| redirects
% compilation to the main or a child file by means of a pattern.
% The prefix |#1| in the current filename is replaced by |#2|
% and the suffix of the current filename is kept
% (it is assumed that the filename does not contain the substring `|~~~|'
% which is used as a delimiter).
% Compilation is handed over to the new file by |\childdocforward|:
%    \begin{macrocode}
\newcommand{\childdocforwardprefix}[3][]
{
  \begingroup
    \def\childdocextract #2##1~~~{\def\childdoctmp{\childdocforward[#1]{#3##1}}}
    \expandafter\childdocextract\childdocname~~~
    \expandafter
  \endgroup
  \childdoctmp
}
%    \end{macrocode}

% \macro{\childdoc}
% The deprecated macro |\childdoc| is a legacy version of |\childdocmain|:
%    \begin{macrocode}
\newcommand{\childdoc}{\childdocmain}
%    \end{macrocode}

% \macro{\childdocredirect}
% The deprecated macro |\childdocredirect| is a legacy version
% of |\childdocforward| and |\childdocforwardprefix|:
%    \begin{macrocode}
\newcommand{\childdocredirect}[2][]
{
  \begingroup
    \if?#1?
      \def\childdoctmp{\childdocforward{#2}}
    \else
      \def\childdoctmp{\childdocforwardprefix{#1}{#2}}
    \fi
    \expandafter
  \endgroup
  \childdoctmp
}
%    \end{macrocode}

%\iffalse
%</package>
%\fi
%
\endinput
\childdocforward[cdocsamp]{cdocsch1}"|\\
% |latex -jobname cdocscl2 \|\\
% |  "\def\version{final}% \iffalse
%
% childdoc.dtx Copyright (C) 2017-2018 Niklas Beisert
%
% This work may be distributed and/or modified under the
% conditions of the LaTeX Project Public License, either version 1.3
% of this license or (at your option) any later version.
% The latest version of this license is in
%   http://www.latex-project.org/lppl.txt
% and version 1.3 or later is part of all distributions of LaTeX
% version 2005/12/01 or later.
%
% This work has the LPPL maintenance status `maintained'.
%
% The Current Maintainer of this work is Niklas Beisert.
%
% This work consists of the files childdoc.dtx and childdoc.ins
% and the derived files childdoc.def and cdocsamp.tex with
% cdocsch1.tex, cdocsch2.tex, cdocsdrf.tex, cdocsfn1.tex, cdocsfn2.tex.
%
%<package>\ifdefined\childdocmain\endinput\fi
%<package>\ProvidesFile{childdoc.def}[2018/12/30 v2.0 child document driver]
%<samplemain>\ProvidesFile{cdocsamp.tex}[2018/12/30 v2.0 sample for childdoc]
%<*driver>
%\ProvidesFile{childdoc.drv}[2018/12/30 v2.0 childdoc reference manual file]
\PassOptionsToClass{10pt,a4paper}{article}
\documentclass{ltxdoc}

\usepackage[margin=35mm]{geometry}
\usepackage{hyperref}
\usepackage{hyperxmp}
\usepackage[usenames]{color}

\hypersetup{colorlinks=true}
\hypersetup{pdfstartview=FitH}
\hypersetup{pdfpagemode=UseNone}
\hypersetup{pdfsource={}}
\hypersetup{pdflang={en-UK}}
\hypersetup{pdfcopyright={Copyright 2017-2018 Niklas Beisert.
  This work may be distributed and/or modified under the
  conditions of the LaTeX Project Public License, either version 1.3
  of this license or (at your option) any later version.}}
\hypersetup{pdflicenseurl={http://www.latex-project.org/lppl.txt}}
\hypersetup{pdfcontactaddress={ETH Zurich, ITP, HIT K,
  Wolfgang-Pauli-Strasse 27}}
\hypersetup{pdfcontactpostcode={8093}}
\hypersetup{pdfcontactcity={Zurich}}
\hypersetup{pdfcontactcountry={Switzerland}}
\hypersetup{pdfcontactemail={nbeisert@itp.phys.ethz.ch}}
\hypersetup{pdfcontacturl={http://people.phys.ethz.ch/\xmptilde nbeisert/}}

\newcommand{\secref}[1]{\hyperref[#1]{section \ref*{#1}}}

\parskip1ex
\parindent0pt
\let\olditemize\itemize
\def\itemize{\olditemize\parskip0pt}

\begin{document}

\title{The \textsf{childdoc} Package}
\hypersetup{pdftitle={The childdoc Package}}
\author{Niklas Beisert\\[2ex]
  Institut f\"ur Theoretische Physik\\
  Eidgen\"ossische Technische Hochschule Z\"urich\\
  Wolfgang-Pauli-Strasse 27, 8093 Z\"urich, Switzerland\\[1ex]
  \href{mailto:nbeisert@itp.phys.ethz.ch}
  {\texttt{nbeisert@itp.phys.ethz.ch}}}
\hypersetup{pdfauthor={Niklas Beisert}}
\hypersetup{pdfsubject={Manual for the LaTeX2e Package childdoc}}
\date{30 December 2018, \textsf{v2.0}}
\maketitle

\begin{abstract}\noindent
\textsf{childdoc} is a \LaTeXe{} package
that enables the direct compilation
of document sections included by |\include|
to individual files.
\end{abstract}

\begingroup
\parskip0ex
\tableofcontents
\endgroup

%%%%%%%%%%%%%%%%%%%%%%%%%%%%%%%%%%%%%%%%%%%%%%%%%%%%%%%%%%%%%%%%%%%%%%%%%%%%%%%%
%%%%%%%%%%%%%%%%%%%%%%%%%%%%%%%%%%%%%%%%%%%%%%%%%%%%%%%%%%%%%%%%%%%%%%%%%%%%%%%%
\section{Introduction}

\LaTeX{} provides a mechanism to structure a large document (such as a book)
into a main file and several child files (containing the chapters)
using the |\include| command.
This mechanism is beneficial for documents
which span hundreds of pages in order to
make the source file(s) more manageable.
Moreover, compilation can be restricted to
selected child files by means of the |\includeonly| command.
The latter feature can be used to reduce the compilation time while editing
(this was significantly more useful in the earlier days of \LaTeX{})
or to generate a smaller document which is easier to navigate.
Another application of |\includeonly| is to generate
documents consisting of selected parts of the complete document.

However, there are a few drawbacks of the plain |\include| mechanism:
\begin{itemize}
\item
The child files cannot be compiled on their own,
they can only be compiled via the main file.
A naive editing environment
(such as a text editor with an option
to have the current file processed by \LaTeX)
may require one to switch to the main file before compiling;
attempting to compile the child file produces errors.
\item
The main file must be modified (each time)
to adjust the |\includeonly| command
to the present needs. This easily leaves the main file in a messy state.
\item
The generated document will always carry the filename
of the main document. This is inconvenient if
several child files are to be compiled and
to be kept for distribution.
\end{itemize}

The present package provides a simple interface
to make child files individually compilable by \LaTeX{}.
Compiling a child file then has the same effect as compiling
the main file with an |\includeonly| command
to select the appropriate child.
Moreover the generated document will carry the name of the child
rather than the main file.
This resolves all three above issues.

This feature is meant to make the editing of books,
thesis documents and lecture notes somewhat more convenient.
However, the package can also be used efficiently for
composing a series of documents (such as exercise sheets)
which are typically distributed individually.
It then assists the author in generating the individual documents
(potentially in different versions)
as well as a document containing the collected series.
Another application is in developing style files
or other kinds of included material
where compilation of the style file could redirect
to a sample or test file.

%%%%%%%%%%%%%%%%%%%%%%%%%%%%%%%%%%%%%%%%%%%%%%%%%%%%%%%%%%%%%%%%%%%%%%%%%%%%%%%%
%%%%%%%%%%%%%%%%%%%%%%%%%%%%%%%%%%%%%%%%%%%%%%%%%%%%%%%%%%%%%%%%%%%%%%%%%%%%%%%%
\section{Usage}

First of all, the package \textsf{childdoc} is \emph{not} a standard
\LaTeXe{} |.sty| style file! Therefore it needs to be invoked in
a non-standard way.

%%%%%%%%%%%%%%%%%%%%%%%%%%%%%%%%%%%%%%%%%%%%%%%%%%%%%%%%%%%%%%%%%%%%%%%%%%%%%%%%
\subsection{Included Files}
\label{sec:include}

%%%%%%%%%%%%%%%%%%%%%%%%%%%%%%%%%%%%%%%%
\DescribeMacro{\childdocmain}
To use the package, add the commands
\begin{center}
\begin{tabular}{l}
|\input{childdoc.def}|\\
|\childdocmain{}|\\
\end{tabular}
\end{center}
at the very top of the main \LaTeX{} file,
in particular \emph{before} the |\documentclass| statement!
The argument of |\childdocmain| should be left empty
(but it must be present).

%%%%%%%%%%%%%%%%%%%%%%%%%%%%%%%%%%%%%%%%
\DescribeMacro{\childdocof}
Furthermore, add the commands
\begin{center}
\begin{tabular}{l}
|\input{childdoc.def}|\\
|\childdocof{|\textit{main}|}|\\
\end{tabular}
\end{center}
at the top of every child file \textit{child}
which is included by |\include{|\textit{child}|}|
from within the main file
(or at least for those files to be compiled individually).
The argument \textit{main} must be the filename of the main file.

There are a couple of
considerations in setting up the main and child documents:

%%%%%%%%%%%%%%%%%%%%%%%%%%%%%%%%%%%%%%%%
\paragraph{Restrictions.}

Please note the following restrictions:
\begin{itemize}
\item
|\childdocmain| must be called with one argument \textit{main}
to ensure compatibility with earlier version of the package.
It must either be empty (|\childdocmain{}|)
or precisely match the filename of the main file in which it is specified.
See \secref{sec:detection} for further information.
\item
The filename \textit{main} must be specified without the |.tex| extension.
\item
The filename \textit{main} is case sensitive
(even in case-insensitive file systems)
due to internal string comparison.
\item
The argument \textit{main} should be fully expanded, it cannot be a macro.
\item
Subdirectories and special characters should be avoided in filenames.
\item
The command |\childdocmain{|\textit{main}|}| must be followed by a whitespace.
It should not be followed immediately by another command
or by a comment mark `|%|'.
This is because the \TeX{} parser reads the token immediately following
the argument of |\childdocmain| and puts it
at the beginning of every child section;
however, a white\-space is ignored.
\end{itemize}

%%%%%%%%%%%%%%%%%%%%%%%%%%%%%%%%%%%%%%%%
\paragraph{Content of Main File.}

It is advisable to place all content in the child files included by |\include|.
Any output contained in the main file will appear in all child documents
unless suppressed manually;
it cannot be suppressed automatically by the |\includeonly| directive
and thus should normally be avoided.
A method to include some content in the main file
by means of conditional processing is described in \secref{sec:conditional}.

%%%%%%%%%%%%%%%%%%%%%%%%%%%%%%%%%%%%%%%%
\paragraph{Page Numbering.}

When only a part of the document is compiled,
the appropriate numbering of pages
(as well as other status parameters)
is determined from the |.aux| files.
The latter contain information from previous passes.
However this information needs to propagate through
all intermediate child documents.
Therefore the page numbering in child documents may well
be inconsistent until the complete document is compiled at least once.

A useful (if unconventional) way to always ensure a consistent
page numbering is to restart the numbering in each child document
and denote the pages by `\textit{child}|.|\textit{page}'
where \textit{child} represents the chapter/section number of the child file.
This can be achieved by the command
|\numberwithin{page}{|\textit{child}|}|
of the \textsf{amsmath} package
where \textit{child} can be |chapter| or |section|
depending on the chosen structuring.
Alternatively, one can modify the macro |\thepage| appropriately
and reset the counter |page| at the start of each child file.

%%%%%%%%%%%%%%%%%%%%%%%%%%%%%%%%%%%%%%%%%%%%%%%%%%%%%%%%%%%%%%%%%%%%%%%%%%%%%%%%
\subsection{Conditional Processing}
\label{sec:conditional}

The package provides a mechanism to compile different versions
of a document. To customise the versions further some conditional processing
can come in handy to distinguish which version is being compiled.
The package provides two macros to describe the compilation context:

%%%%%%%%%%%%%%%%%%%%%%%%%%%%%%%%%%%%%%%%
\DescribeMacro{\ifchilddoc}
The conditional |\ifchilddoc| distinguishes between the compilation of
child documents and the main document:
%
\begin{center}
|\ifchilddoc |\textit{child-code}| |[|\||else |\textit{main-code}]| \||fi|
\end{center}

%%%%%%%%%%%%%%%%%%%%%%%%%%%%%%%%%%%%%%%%
\DescribeMacro{\childdocname}
\DescribeMacro{\childdocjob}
The macro |\childdocname| contains the filename (without extension)
of the main or child file being processed.
Note that |\childdocjob| will always contain the name of the main file.

%%%%%%%%%%%%%%%%%%%%%%%%%%%%%%%%%%%%%%%%
\paragraph{Title Page.}

Conditional processing can be used to include a title or banner page
in the main document when proper precautions are taken.
Importantly, the code in the main file should ensure that the page counter
(as well as other status parameters which are stored in the |.aux| files)
takes the same value after the conditional processing.
Otherwise the page numbers may take divergent values
depending on which part is compiled.

For example, a title page could be declared by:
%
\begin{center}
\begin{tabular}{l}
|\ifchilddoc\||else|\\
|\addtocounter{page}{-1}|\\
\textit{code for title page}\\
|\newpage|\\
|\||fi|
\end{tabular}
\end{center}
%
A banner page for the child documents can be generated by:
%
\begin{center}
\begin{tabular}{l}
|\ifchilddoc|\\
|\addtocounter{page}{-1}|\\
\textit{code for banner page}\\
|\newpage|\\
|\||fi|
\end{tabular}
\end{center}
%
Here one could write a message such as:
\begin{center}
|This is the part \childdocname{} of \childdocjob{}.|
\end{center}

%%%%%%%%%%%%%%%%%%%%%%%%%%%%%%%%%%%%%%%%%%%%%%%%%%%%%%%%%%%%%%%%%%%%%%%%%%%%%%%%
\subsection{Flags}
\label{sec:flags}

The package makes it easy to generate different versions
of the main or child documents.
To this end compilation flags can be defined
and assigned different default values.
They will be particularly useful in conjunction
with the forwarding mechanism described in \secref{sec:forward}.

For example, it may be useful to have a flag |\version|
which can be set to |draft| or |final|.
The document source will contain some conditional code
depending on the value of |\version|.
Suppose further, the flag should default to |final| for the main file
and to |draft| for child files
which is a natural assignment for editing the document.
This is achieved by placing the following code
in the preamble of the main document
(below the |\childdocmain| directive):
%
\begin{center}
\begin{tabular}{l}
|\ifchilddoc|\\
|\providecommand{\version}{draft}|\\
|\||else|\\
|\providecommand{\version}{final}|\\
|\||fi|
\end{tabular}
\end{center}
%
The definition by |\providecommand| makes sure
that previous definitions are not overwritten.
Further statements |\providecommand{\version}{...}|
can thus be added before the above code to override it.

For the main file, one might add a line
(between |\childdocmain| and the above block)
%
\begin{center}
|%\ifchilddoc\||else\providecommand{\version}{draft}\||fi|
\end{center}
%
which can be uncommented to produce a draft version.
Likewise one can add a line to the very top of a child file
(above the |\childdocof{|\textit{main}|}| directive)
%
\begin{center}
|%\providecommand{\version}{final}|
\end{center}
%
which can be uncommented to produce the final version of this child document.

%%%%%%%%%%%%%%%%%%%%%%%%%%%%%%%%%%%%%%%%%%%%%%%%%%%%%%%%%%%%%%%%%%%%%%%%%%%%%%%%
\subsection{Forwarding}
\label{sec:forward}

Different versions of the main or child documents
using compilation flags as described in \secref{sec:flags}
can be (permanently) stored in different files
for convenient compilation, viewing and distribution.
To this end, the package defines a command
to pass on compilation to a different file:

%%%%%%%%%%%%%%%%%%%%%%%%%%%%%%%%%%%%%%%%
\DescribeMacro{\childdocforward}
The command |\childdocforward| redirects processing to
another source file:
%
\begin{center}
\begin{tabular}{l}
|\input{childdoc.def}|\\
|\childdocforward[|\textit{main}|]{|\textit{dest}|}|\\
\end{tabular}
\end{center}
%
The argument \textit{dest} is the destination file
(without extension).
It should be the main file or one of the child files.
Note that further \textsf{childdoc} directives
such as |\childdocof| and |\childdocforward|
in the indicated file will be processed in this form.
The optional argument \textit{main}
passes on directly to the main file \textit{main}
while pretending to compile the child \textit{dest}.
This form behaves as if \textit{dest}
issues |\childdocof{|\textit{main}|}| right away,
and no further \textsf{childdoc} directives will be processed.

%%%%%%%%%%%%%%%%%%%%%%%%%%%%%%%%%%%%%%%%
\DescribeMacro{\...prefix}
In the alternative form |\childdocforwardprefix|,
%
\begin{center}
\begin{tabular}{l}
|\input{childdoc.def}|\\
|\childdocforwardprefix[|\textit{main}|]{|\textit{prefix}|}{|\textit{dest}|}|
\end{tabular}
\end{center}
%
the destination file is determined by a pattern
depending on the current file:
To make this work, the current file must be called
`{\textit{prefix}\hspace{0.2em}\textit{suffix}}'
with \textit{prefix} matching precisely the argument.
Processing is then passed on to the file
`{\textit{dest}\hspace{0.2em}\textit{suffix}}'.
Surely, the same effect is achieved by
directly specifying the
argument `{\textit{dest}\hspace{0.2em}\textit{suffix}}'
in the first form.
However, that requires to set up a different file
for each child. With the alternative form of the command
all these files can have exactly the same content
which simplifies setting them up and maintaining them.

For example, the following file |draft.tex|
with a compilation flag |\version| as described in \secref{sec:flags}
compiles the main document as a draft:
%
\begin{center}
\begin{tabular}{l}
|\def\version{draft}|\\
|\input{childdoc.def}|\\
|\childdocforward{|\textit{main}|}|
\end{tabular}
\end{center}
%
Likewise, the following files |final|\textit{nn}|.tex|
compile the final version of the child document
|child|\textit{nn}|.tex|:
%
\begin{center}
\begin{tabular}{l}
|\def\version{final}|\\
|\input{childdoc.def}|\\
|\childdocforwardprefix{final}{child}|
\end{tabular}
\end{center}
%

Note that when several versions of a main file and/or of each child file
are to be generated, it may be convenient to set up a |Makefile| or
shell script to automatise the process.

%%%%%%%%%%%%%%%%%%%%%%%%%%%%%%%%%%%%%%%%%%%%%%%%%%%%%%%%%%%%%%%%%%%%%%%%%%%%%%%%
\subsection{Command Line Processing}
\label{sec:commandline}

The effect of redirection files can also be achieved by invoking
the \LaTeX{} compiler with a more elaborate command line.
Most conveniently this should be done as part
of a shell script or a |Makefile|.

When using \textsf{childdoc} in the main file, the following
command lines effectively perform a redirection
(note that depending on the shell being used,
backslashes may have to be doubled: `|\|' $\to$ `|\\|'):
%
\begin{center}
|... -jobname "|\textit{target}|" |\\|"|[\textit{flags}]%
|\input{childdoc.def}\childdocforward[|\textit{main}|]{|\textit{dest}|}"|
\end{center}
%
Here \textit{target} is the name of the output file,
\textit{main} is the name of the main file
and \textit{dest} is the name of the main or child file to be processed
(all filenames without extensions).
The optional argument \textit{main} can be omitted
if \textit{main} matches \textit{dest}.
Optionally, compilation \textit{flags} can be defined via |\def| commands.
This command line makes the \TeX{} engine believe
it is compiling the file \textit{target}
whose content is specified as the latter parameter.
The provided code then forwards the processing to
\textit{main} or \textit{dest} as described in \secref{sec:forward}.

%%%%%%%%%%%%%%%%%%%%%%%%%%%%%%%%%%%%%%%%%%%%%%%%%%%%%%%%%%%%%%%%%%%%%%%%%%%%%%%%
\subsection{Include by Input}
\label{sec:input}

Including child documents by |\include| has some restrictions by design.
Most notably, the content of a child document always occupies
its own set of pages; pages cannot be shared between child documents.
Usually, this behaviour makes perfect sense
because each child document contain an essential part of the document.
However, in some situations it may be desirable to compose
a document from a collection of parts
without having mandatory page breaks between then.
For this case, the package
provides a mechanism to include parts
by |\input| which can also be processed individually.
However, by construction this mechanism
requires manual handling of the content to be output.

%%%%%%%%%%%%%%%%%%%%%%%%%%%%%%%%%%%%%%%%
\DescribeMacro{\ifchilddocmanual}
The main file should be prepared as usual, see \secref{sec:include}.
However, the document body must make a distinction
between processing of an individual part and of the main document, e.g.:
%
\begin{center}
\begin{tabular}{l}
|\ifchilddocmanual|\\
|\input{\childdocname}|\\
|\||else|\\
\textit{document body with }|\input{|\textit{part}|}|\\
|\||fi|
\end{tabular}
\end{center}
%
The conditional |\ifchilddocmanual| is true whenever
a part to be included by |\input| is being compiled,
and the name of the part is stored in |\childdocname|.

%%%%%%%%%%%%%%%%%%%%%%%%%%%%%%%%%%%%%%%%
\DescribeMacro{\childdocby}
Each part to be included by |\input| should start with:
%
\begin{center}
\begin{tabular}{l}
|\input{childdoc.def}|\\
|\childdocby{|\textit{main}|}|\\
\end{tabular}
\end{center}
%
The directive |\childdocby| is similar to |\childdocof|
described in \secref{sec:include},
but the subsequent selection of content must be done manually.
To that end, both |\ifchilddoc| and |\ifchilddocmanual|
will be true upon processing of a part,
and the name of the part is stored in |\childdocname|.
Note that |\jobname| will be set to the filename of the current part
so that each part receives an individual |.aux| file
that does not interfere with the |.aux| file(s) of the main document.
This behaviour can be altered by the alternative form
|\childdocby[*]{|\textit{main}|}| (with a non-empty optional argument)
which uses the |.aux| file of the main document
by setting |\jobname| to \textit{main}.

%%%%%%%%%%%%%%%%%%%%%%%%%%%%%%%%%%%%%%%%%%%%%%%%%%%%%%%%%%%%%%%%%%%%%%%%%%%%%%%%
\subsection{Driver Development}
\label{sec:driver}

The \textsf{childdoc} mechanism can also be use for the development
of definition files such as \LaTeX{} styles or classes.
This case differs from the above setup with multiple parts
included by |\include| in that no |\includeonly| should be invoked.
This can be achieved by starting the include file
(before |\ProvidesPackage|) with:
%
\begin{center}
\begin{tabular}{l}
|\input{childdoc.def}|\\
|\childdocforward{|\textit{main}|}|\\
\end{tabular}
\end{center}
%
or alternatively with:
%
\begin{center}
\begin{tabular}{l}
|\input{childdoc.def}|\\
|\childdocby{|\textit{main}|}|\\
\end{tabular}
\end{center}
%
Both forms have slightly different effects as described above.
The main file is prepared as usual, see \secref{sec:include}.

%%%%%%%%%%%%%%%%%%%%%%%%%%%%%%%%%%%%%%%%%%%%%%%%%%%%%%%%%%%%%%%%%%%%%%%%%%%%%%%%
\subsection{Legacy Detection}
\label{sec:detection}

The directive |\childdocmain| in the main file can detect
whether the complete document or merely a child is to be compiled
even without using the directive |\childdocof|.
This method is deprecated because it is less robust
and there is no compelling reason to use it;
it is merely provided for backward compatibility
and it may be removed in future versions.

If the detection mechanism is to be used,
it is mandatory to correctly specify
the filename of the main file as the argument of |\childdocmain|:
%
\begin{center}
\begin{tabular}{l}
|\input{childdoc.def}|\\
|\childdocmain{|\textit{main}|}|\\
\end{tabular}
\end{center}
%
If |\jobname| does not match the argument \textit{main} of |\childdocmain|,
it is assumed that |\jobname| points to the child file to be compiled.
When using |\childdocmain| with the main file specified as argument,
it suffices to start a child file
with just |\input{|\textit{main}|}|
without loading of the package and using |\childdocof|.
If instead all processing is done
with the appropriate \textsf{childdoc} directives,
the argument of \textit{main} of |\childdocmain| can be empty.

An alternative version of the command line processing described
in \secref{sec:commandline} using the detection mechanism reads:
%
\begin{center}
|... -jobname "|\textit{target}|" "|[\textit{flags}]%
[|\def\jobname{|\textit{dest}|}|]|\input{|\textit{main}|}"|
\end{center}

%%%%%%%%%%%%%%%%%%%%%%%%%%%%%%%%%%%%%%%%%%%%%%%%%%%%%%%%%%%%%%%%%%%%%%%%%%%%%%%%
\subsection{Manual Code}
\label{sec:manual}

In case one cannot be certain whether the definitions file |childdoc.def|
is installed on the target \TeX{} distribution
and one prefers not to ship it,
it is conceivable to paste a few relevant commands into the sources.

To that end, drop all statements |\input{childdoc.def}|
and perform the replacements as outlined below.
Instead of |\childdocmain{|\textit{main}|}| add the following code
to the top of the main file:
%
\begin{center}
\begin{tabular}{l}
|\||ifdefined\childdocname\endinput\||fi\newif\ifchilddoc|\\
|\edef\childdocname{\scantokens\expandafter{\jobname\noexpand}}|\\
|\def\childdocmain{|\textit{main}|}\||ifx\childdocmain\childdocname\||else|\\
|\childdoctrue\includeonly{\childdocname}\let\jobname\childdocmain\||fi|\\
\end{tabular}
\end{center}
%
Instead of |\childdocof{|\textit{main}|}| just include the main file
at the top of each child file:
%
\begin{center}
|\input{|\textit{main}|}|
\end{center}
%
A simple redirection |\childdocforward{|\textit{dest}|}| is achieved by:
%
\begin{center}
|\def\jobname{|\textit{dest}|}\input{\jobname}|
\end{center}
%
The redirection with prefix
|\childdocforwardprefix[|\textit{prefix}|]{|\textit{dest}|}|
is accomplished by:
%
\begin{center}
\begin{tabular}{l}
|{\edef\jobname{\scantokens\expandafter{\jobname\noexpand}}|\\
|\def\redirectjob |\textit{prefix}|#1~~~{\gdef\jobname{|\textit{dest}|#1}}|\\
|\expandafter\redirectjob\jobname~~~}\input{\jobname}|
\end{tabular}
\end{center}

In an alternative approach,
child documents can be compiled by a specific command line
without additional code or specific definitions:
%
\begin{center}
|... -jobname "|\textit{target}|" "|[\textit{flags}]%
|\includeonly{|\textit{dest}|}\input{|\textit{main}|}"|
\end{center}
%

%%%%%%%%%%%%%%%%%%%%%%%%%%%%%%%%%%%%%%%%%%%%%%%%%%%%%%%%%%%%%%%%%%%%%%%%%%%%%%%%
%%%%%%%%%%%%%%%%%%%%%%%%%%%%%%%%%%%%%%%%%%%%%%%%%%%%%%%%%%%%%%%%%%%%%%%%%%%%%%%%
\section{Information}

%%%%%%%%%%%%%%%%%%%%%%%%%%%%%%%%%%%%%%%%%%%%%%%%%%%%%%%%%%%%%%%%%%%%%%%%%%%%%%%%
\subsection{Copyright}

Copyright \copyright{} 2017--2018 Niklas Beisert

This work may be distributed and/or modified under the
conditions of the \LaTeX{} Project Public License, either version 1.3
of this license or (at your option) any later version.
The latest version of this license is in
  \url{http://www.latex-project.org/lppl.txt}
and version 1.3 or later is part of all distributions of \LaTeX{}
version 2005/12/01 or later.

This work has the LPPL maintenance status `maintained'.

The Current Maintainer of this work is Niklas Beisert.

This work consists of the files |README.txt|, |childdoc.ins| and |childdoc.dtx|
as well as the derived files |childdoc.def|, |cdocsamp.tex|
with |cdocsch1.tex|, |cdocsch2.tex|, |cdocspt3.tex|, |cdocspt4.tex|,
|cdocsdrf.tex|, |cdocsfn1.tex|, |cdocsfn2.tex|
as well as |childdoc.pdf|.

%%%%%%%%%%%%%%%%%%%%%%%%%%%%%%%%%%%%%%%%%%%%%%%%%%%%%%%%%%%%%%%%%%%%%%%%%%%%%%%%
\subsection{Files and Installation}

The package consists of the files:
%
\begin{center}
\begin{tabular}{ll}
    |README.txt|   & readme file \\
    |childdoc.ins| & installation file \\
    |childdoc.dtx| & source file \\
    |childdoc.def| & definition file \\
    |cdocsamp.tex| & sample main file \\
    |cdocsch1.tex| & sample include file \\
    |cdocsch2.tex| & sample include file \\
    |cdocspt3.tex| & sample part file \\
    |cdocspt4.tex| & sample part file \\
    |cdocsdrf.tex| & sample redirection file \\
    |cdocsfn1.tex| & sample redirection file \\
    |cdocsfn2.tex| & sample redirection file \\
    |childdoc.pdf| & manual
\end{tabular}
\end{center}
%
The distribution consists of the files
|README.txt|, |childdoc.ins| and |childdoc.dtx|.
%
\begin{itemize}
\item
Run (pdf)\LaTeX{} on |childdoc.dtx|
to compile the manual |childdoc.pdf| (this file).
\item
Run \LaTeX{} on |childdoc.ins| to create the definitions file |childdoc.def|
and the sample |cdocsamp.tex| with include files
|cdocsch1.tex|, |cdocsch2.tex|, |cdocspt3.tex|, |cdocspt4.tex|,
|cdocsdrf.tex|, |cdocsfn1.tex|, |cdocsfn2.tex|.
Then copy the file |childdoc.def| to an appropriate directory of your \LaTeX{}
distribution, e.g.\ \textit{texmf-root}|/tex/latex/childdoc|.
\end{itemize}

%%%%%%%%%%%%%%%%%%%%%%%%%%%%%%%%%%%%%%%%%%%%%%%%%%%%%%%%%%%%%%%%%%%%%%%%%%%%%%%%
\subsection{Related CTAN Packages}

There are several other packages which offer a similar functionality:
%
\begin{itemize}
\item
The packages
\href{http://ctan.org/pkg/docmute}{\textsf{docmute}},
\href{http://ctan.org/pkg/includex}{\textsf{includex}} and
\href{http://ctan.org/pkg/standalone}{\textsf{standalone}}
provide commands to include only the document body of
a child file thus allowing both files to be compiled individually.
\item
The packages \href{http://ctan.org/pkg/subdocs}{\textsf{subdocs}}
and \href{http://ctan.org/pkg/subfiles}{\textsf{subfiles}}
provide structures in which the main and child documents can be
encapsulated and allowing them to be compiled individually.
The inclusion mechanism is different from the conventional |\include|.
\item
The package \href{http://ctan.org/pkg/combine}{\textsf{combine}}
is an elaborate solution to combine several documents into one.
\end{itemize}
%
See also the CTAN topic \href{http://ctan.org/topic/subdocs}{\textsf{subdocs}}
for further related packages.
The present package differs from the above solutions in that
a document structure constructed with the conventional |\include| mechanism
just needs two extra commands at the top of every file
such that all constituent files can be compiled individually.

%%%%%%%%%%%%%%%%%%%%%%%%%%%%%%%%%%%%%%%%%%%%%%%%%%%%%%%%%%%%%%%%%%%%%%%%%%%%%%%%
%\subsection{Feature Suggestions}
%
%The following is a list of features which may be useful for future
%versions of this package:
%%
%\begin{itemize}
%\item
%\ldots
%\end{itemize}

%%%%%%%%%%%%%%%%%%%%%%%%%%%%%%%%%%%%%%%%%%%%%%%%%%%%%%%%%%%%%%%%%%%%%%%%%%%%%%%%
\subsection{Revision History}

%%%%%%%%%%%%%%%%%%%%%%%%%%%%%%%%%%%%%%%%
\paragraph{v2.0:} 2018/12/30

\begin{itemize}
\item
immediate forward processing
\item
added |\childdocby| mechanism
\item
manual restructured
\end{itemize}

%%%%%%%%%%%%%%%%%%%%%%%%%%%%%%%%%%%%%%%%
\paragraph{v1.6:} 2018/01/17

\begin{itemize}
\item
application for development of include files
\item
corrections to manual
\end{itemize}

%%%%%%%%%%%%%%%%%%%%%%%%%%%%%%%%%%%%%%%%
\paragraph{v1.5:} 2017/05/21

\begin{itemize}
\item
more complete structuring introduced
\item
|\childdocof| introduced
\item
|\childdoc| renamed to |\childdocmain|
\item
|\childredirect| renamed to |\childdocforward| and |\childdocforwardprefix|
and functionality expanded
\end{itemize}

%%%%%%%%%%%%%%%%%%%%%%%%%%%%%%%%%%%%%%%%
\paragraph{v1.0:} 2017/04/27

\begin{itemize}
\item
manual and install package
\item
first version published on CTAN
\end{itemize}

%%%%%%%%%%%%%%%%%%%%%%%%%%%%%%%%%%%%%%%%
\paragraph{v0.6:} 2017/04/26

\begin{itemize}
\item
redirection mechanism added
\end{itemize}

%%%%%%%%%%%%%%%%%%%%%%%%%%%%%%%%%%%%%%%%
\paragraph{v0.5:} 2017/04/26

\begin{itemize}
\item
functionality in definition file
\end{itemize}


%%%%%%%%%%%%%%%%%%%%%%%%%%%%%%%%%%%%%%%%%%%%%%%%%%%%%%%%%%%%%%%%%%%%%%%%%%%%%%%%
%%%%%%%%%%%%%%%%%%%%%%%%%%%%%%%%%%%%%%%%%%%%%%%%%%%%%%%%%%%%%%%%%%%%%%%%%%%%%%%%
%%%%%%%%%%%%%%%%%%%%%%%%%%%%%%%%%%%%%%%%%%%%%%%%%%%%%%%%%%%%%%%%%%%%%%%%%%%%%%%%
\appendix

\settowidth\MacroIndent{\rmfamily\scriptsize 000\ }

 \DocInput{childdoc.dtx}

\end{document}
%</driver>
% \fi
%
% %%%%%%%%%%%%%%%%%%%%%%%%%%%%%%%%%%%%%%%%%%%%%%%%%%%%%%%%%%%%%%%%%%%%%%%%%%%%%%
% %%%%%%%%%%%%%%%%%%%%%%%%%%%%%%%%%%%%%%%%%%%%%%%%%%%%%%%%%%%%%%%%%%%%%%%%%%%%%%
% \section{Sample}
%\iffalse
%<*samplemain>
%\fi
%
% The following presents a sample document
% with two chapters, two parts, a title page,
% a compile flag as well as three forwarding files to set the flag.
% It consists of eight |.tex| files:
% \begin{center}
% \begin{tabular}{ll}
% |cdocsamp.tex|&main file\\
% |cdocsch1.tex|&include file for chapter 1\\
% |cdocsch2.tex|&include file for chapter 2\\
% |cdocspt3.tex|&include file for part 3\\
% |cdocspt4.tex|&include file for part 4\\
% |cdocsdrf.tex|&forwarding file for main file in draft mode\\
% |cdocsfi1.tex|&forwarding file for final version of chapter 1\\
% |cdocsfi2.tex|&forwarding file for final version of chapter 2\\
% \end{tabular}
% \end{center}
% Each of the eight files can be compiled directly by the \LaTeX{} compiler.
%
% %%%%%%%%%%%%%%%%%%%%%%%%%%%%%%%%%%%%%%
% \paragraph{Main File.}
%
% The main file is called |cdocsamp.tex|.
%
% Load the \textsf{childdoc} definitions and
% declare the filename for the main document:
%    \begin{macrocode}
\input{childdoc.def}
\childdocmain{}
%    \end{macrocode}

% Optional override for |\version| flag:
%    \begin{macrocode}
%%\ifchilddoc\else\providecommand{\version}{draft}\fi
%    \end{macrocode}

% Define the default values for the |\version| flag
% (|final| for the main file and |draft| for childs):
%    \begin{macrocode}
\ifchilddoc
\providecommand{\version}{draft}
\else
\providecommand{\version}{final}
\fi
%    \end{macrocode}

% Load the standard document class:
%    \begin{macrocode}
\documentclass[12pt]{article}
%    \end{macrocode}

% Start the document body:
%    \begin{macrocode}
\begin{document}
%    \end{macrocode}

% Declare a title page.
% Print title, part of document being processed and version flag:
%    \begin{macrocode}
\addtocounter{page}{-1}
\begin{center}
{\LARGE\bfseries{}childdoc example\par}
\vspace{1cm}
\ifchilddoc
\ifchilddocmanual part\else chapter\fi:
`\childdocname' of `\childdocjob'\par
\else
main document: `\childdocjob'\par
\fi
version: \version\par
\end{center}
\newpage
%    \end{macrocode}

% Manually include selected file,
% otherwise process as usual:
%    \begin{macrocode}
\ifchilddocmanual
\section*{part `\childdocname'}
\input{\childdocname}
\else
%    \end{macrocode}

% Include the two chapters:
%    \begin{macrocode}
\include{cdocsch1}
\include{cdocsch2}
%    \end{macrocode}

% Include the two parts unless only chapters should be displayed:
%    \begin{macrocode}
\ifchilddoc\else
\section{part three}
\input{cdocspt3}
\section{part four}
\input{cdocspt4}
\fi
%    \end{macrocode}

% Process as usual until here:
%    \begin{macrocode}
\fi
%    \end{macrocode}

% End of document body:
%    \begin{macrocode}
\end{document}
%    \end{macrocode}
%\iffalse
%</samplemain>
%\fi
%
% %%%%%%%%%%%%%%%%%%%%%%%%%%%%%%%%%%%%%%
% \paragraph{Chapter Include Files.}
%
% The include files are called |cdocsch1.tex| and |cdocsch2.tex|.
%
%\iffalse
%<*samplechap1|samplechap2>
%\fi

% Optional override for |\version| flag:
%    \begin{macrocode}
%%\providecommand{\version}{final}
%    \end{macrocode}

% Include the main document:
%    \begin{macrocode}
\input{childdoc.def}
\childdocof{cdocsamp}
%    \end{macrocode}

%\iffalse
%</samplechap1|samplechap2>
%\fi
%
%\iffalse
%<*samplechap1>
%\fi
% Some text for chapter 1:
%    \begin{macrocode}
\section{one}
some text in chapter one
%    \end{macrocode}

%\iffalse
%</samplechap1>
%\fi
% Some text for chapter 2:
%\iffalse
%<*samplechap2>
%\fi
%    \begin{macrocode}
\section{two}
more text in chapter two
%    \end{macrocode}

%\iffalse
%</samplechap2>
%\fi
%
% %%%%%%%%%%%%%%%%%%%%%%%%%%%%%%%%%%%%%%
% \paragraph{Part Include Files.}
%
% The include files are called |cdocspt3.tex| and |cdocspt4.tex|.
%
%\iffalse
%<*samplepart3|samplepart4>
%\fi

% Optional override for |\version| flag:
%    \begin{macrocode}
%%\providecommand{\version}{final}
%    \end{macrocode}

% Include the main document:
%    \begin{macrocode}
\input{childdoc.def}
\childdocby{cdocsamp}
%    \end{macrocode}

%\iffalse
%</samplepart3|samplepart4>
%\fi
%
%\iffalse
%<*samplepart3>
%\fi
% Some text for part 3:
%    \begin{macrocode}
some text in part three
%    \end{macrocode}

%\iffalse
%</samplepart3>
%\fi
% Some text for part 4:
%\iffalse
%<*samplepart4>
%\fi
%    \begin{macrocode}
more text in part four
%    \end{macrocode}

%\iffalse
%</samplepart4>
%\fi
%
% %%%%%%%%%%%%%%%%%%%%%%%%%%%%%%%%%%%%%%
% \paragraph{Forwarding for a Complete Draft.}
%
% The following forwarding file |cdocsdrf.tex|
% compiles the main document in draft mode:
%\iffalse
%<*sampledraft>
%\fi
%    \begin{macrocode}
\def\version{draft}
\input{childdoc.def}
\childdocforward{cdocsamp}
%    \end{macrocode}

%\iffalse
%</sampledraft>
%\fi
%
% %%%%%%%%%%%%%%%%%%%%%%%%%%%%%%%%%%%%%%
% \paragraph{Forwarding for Final Version of the Chapters.}
%
% The following forwarding files |cdocsfn1.tex| and |cdocsfn2.tex|
% (with identical content)
% compile the final versions of the child documents
% |cdocsch1.tex| and |cdocsch2.tex|, respectively:
%\iffalse
%<*samplefinal>
%\fi
%    \begin{macrocode}
\def\version{final}
\input{childdoc.def}
\childdocforwardprefix[cdocsamp]{cdocsfn}{cdocsch}
%    \end{macrocode}

%\iffalse
%</samplefinal>
%\fi
%
% %%%%%%%%%%%%%%%%%%%%%%%%%%%%%%%%%%%%%%
% \paragraph{Command Line Processing.}
%
% The following three command lines generate the output files
% |cdocscld|, |cdocscl1| and |cdocscl2|
% which should be identical to
% |cdocsdrf|, |cdocsch1| and |cdocsfn2|, respectively:
% \begin{center}
% \begin{tabular}{l}
% |latex -jobname cdocscld \|\\
% |  "\def\version{draft}\input{childdoc.def}\childdocforward{cdocsamp}"|\\
% |latex -jobname cdocscl1 \|\\
% |  "\input{childdoc.def}\childdocforward[cdocsamp]{cdocsch1}"|\\
% |latex -jobname cdocscl2 \|\\
% |  "\def\version{final}\input{childdoc.def}\childdocforward{cdocsch2}"|
% \end{tabular}
% \end{center}
% Note that the trailing backslash on each first line
% merely continues the input to the second line
% (for convenient cut ant paste).
% Furthermore, the command |latex| can be replaced by any
% of its alternative versions such as |pdflatex|.
%
% %%%%%%%%%%%%%%%%%%%%%%%%%%%%%%%%%%%%%%%%%%%%%%%%%%%%%%%%%%%%%%%%%%%%%%%%%%%%%%
% %%%%%%%%%%%%%%%%%%%%%%%%%%%%%%%%%%%%%%%%%%%%%%%%%%%%%%%%%%%%%%%%%%%%%%%%%%%%%%
% \section{Implementation}
%\iffalse
%<*package>
%\fi
%
% This section describes the definitions file |childdoc.def|.

% The definitions cannot be loaded using |\usepackage| or |\RequirePackage|
% which has a mechanism to prevent loading a style file more than once.
% When loading the definitions by means of |\input|
% multiple instances have to be prevented manually:
%\iffalse
%This code needs to be before the `\ProvidesFile' directive
%which is defined at the beginning of this file.
%Therefore it is also placed there and commented out here.
%</package>
%<*discard>
%\fi
%    \begin{macrocode}
\ifdefined\childdocmain\endinput\fi
%    \end{macrocode}
%\iffalse
%</discard>
%<*package>
%\fi
%
% \macro{\ifchilddoc}
% \macro{\ifchilddocmanual}
% The conditional |\ifchilddoc| tells whether a
% child (true) or main (false) document is being compiled.
% The conditional |\ifchilddocmanual| tells whether
% the |\includeonly| mechanism is used (false) or
% the selection of child files must be performed manually (true).
% The definitions initialise to false:
%    \begin{macrocode}
\newif\ifchilddoc
\newif\ifchilddocmanual
%    \end{macrocode}

% \macro{\childdocname}
% \macro{\childdocjob}
% The macro |\childdocname| stores the name of the main document
% to be compiled. The macro |\childdocjob| stores the name of
% the document on which the \LaTeX{} compiler was originally invoked.
% The content of |\jobname| cannot be compared
% to filenames specified in the source due to different catcodes.
% The following code rescans |\jobname|, stores the result
% in |\childdocname| and saves a copy in |\childdocjob|:
%    \begin{macrocode}
\edef\childdocname{\scantokens\expandafter{\jobname\noexpand}}
\let\childdocjob\childdocname
%    \end{macrocode}

% \macro{\childdocdisable}
% The macro |\childdocdisable| prevents the main file
% from being processed more than once.
% At this stage, the main document command |\childdocmain|
% is assumed to be called once again where it should do nothing.
% Any subsequent call to it should prevent
% a secondary processing of the main document
% It overwrites the forwarding commands
% |\childdocof| and |\childdocforward|
% with empty macros to prevent further inclusions of the main document:
%    \begin{macrocode}
\newcommand{\childdocdisable}
{
  \renewcommand{\childdocmain}[1]{\renewcommand{\childdocmain}[1]{\endinput}}
  \renewcommand{\childdocof}[1]{}
  \renewcommand{\childdocby}[2][]{}
  \renewcommand{\childdocforward}[2][]{}
  \renewcommand{\childdocdisable}{}
}
%    \end{macrocode}

% \macro{\childdocmain}
% The macro |\childdocmain| is to be called at the top of the main file
% with nothing or the main filename (without extension) as argument.
% First, it breaks loops.
% If the argument is not empty and does not match |\childdocname|
% (which is set by the first inclusion of |childdoc.def|),
% |\ifchilddoc| is set to true, |\includeonly| is applied to the child file
% and |\jobname| is set to the main file
% (for proper handling of |.aux| files):
%    \begin{macrocode}
\newcommand{\childdocmain}[1]
{
  \childdocdisable\childdocmain{}
  \if?#1?\else
    \begingroup
      \def\childdoctmp{#1}
      \ifx\childdoctmp\childdocname
        \def\childdoctmp{}
      \else
        \def\childdoctmp
        {
          \childdoctrue
          \includeonly{\childdocname}
          \def\childdocjob{#1}
          \def\jobname{#1}
        }
      \fi
      \expandafter
    \endgroup
    \childdoctmp
  \fi
}
%    \end{macrocode}

% \macro{\childdocof}
% The command |\childdocof| redirects
% compilation to the main file |#1|.
%    \begin{macrocode}
\newcommand{\childdocof}[1]
{
  \childdocdisable
  \childdoctrue
  \includeonly{\childdocname}
  \def\jobname{#1}
  \def\childdocjob{#1}
  \input{#1}
}
%    \end{macrocode}

% \macro{\childdocby}
% The command |\childdocby| ....
%    \begin{macrocode}
\newcommand{\childdocby}[2][]
{
  \childdocdisable
  \childdoctrue
  \childdocmanualtrue
  \if?#1?\else
    \def\jobname{#2}
  \fi
  \def\childdocjob{#2}
  \input{#2}
  \endinput
}
%    \end{macrocode}

% \macro{\childdocforward}
% The command |\childdocforward| redirects
% compilation to the main file or
% (if the optional argument is given) a child file.
% Parameters are set as if the main file
% or a child file starting with |\childdocof| was compiled.
% Then compilation is handed over to the main file:
%    \begin{macrocode}
\newcommand{\childdocforward}[2][]
{
  \begingroup
    \if?#1?
      \def\childdoctmp
      {
        \def\childdocname{#2}
        \def\childdocjob{#2}
        \def\jobname{#2}
        \input{#2}
        \endinput
      }
    \else
      \def\childdoctmp
      {
        \childdocdisable
        \def\childdocname{#2}
        \childdoctrue
        \includeonly{#2}
        \def\childdocjob{#1}
        \def\jobname{#1}
        \input{#1}
        \endinput
      }
    \fi
    \expandafter
  \endgroup
  \childdoctmp
}
%    \end{macrocode}

% \macro{\childdocforwardprefix}
% The command |\childdocforwardprefix| redirects
% compilation to the main or a child file by means of a pattern.
% The prefix |#1| in the current filename is replaced by |#2|
% and the suffix of the current filename is kept
% (it is assumed that the filename does not contain the substring `|~~~|'
% which is used as a delimiter).
% Compilation is handed over to the new file by |\childdocforward|:
%    \begin{macrocode}
\newcommand{\childdocforwardprefix}[3][]
{
  \begingroup
    \def\childdocextract #2##1~~~{\def\childdoctmp{\childdocforward[#1]{#3##1}}}
    \expandafter\childdocextract\childdocname~~~
    \expandafter
  \endgroup
  \childdoctmp
}
%    \end{macrocode}

% \macro{\childdoc}
% The deprecated macro |\childdoc| is a legacy version of |\childdocmain|:
%    \begin{macrocode}
\newcommand{\childdoc}{\childdocmain}
%    \end{macrocode}

% \macro{\childdocredirect}
% The deprecated macro |\childdocredirect| is a legacy version
% of |\childdocforward| and |\childdocforwardprefix|:
%    \begin{macrocode}
\newcommand{\childdocredirect}[2][]
{
  \begingroup
    \if?#1?
      \def\childdoctmp{\childdocforward{#2}}
    \else
      \def\childdoctmp{\childdocforwardprefix{#1}{#2}}
    \fi
    \expandafter
  \endgroup
  \childdoctmp
}
%    \end{macrocode}

%\iffalse
%</package>
%\fi
%
\endinput
\childdocforward{cdocsch2}"|
% \end{tabular}
% \end{center}
% Note that the trailing backslash on each first line
% merely continues the input to the second line
% (for convenient cut ant paste).
% Furthermore, the command |latex| can be replaced by any
% of its alternative versions such as |pdflatex|.
%
% %%%%%%%%%%%%%%%%%%%%%%%%%%%%%%%%%%%%%%%%%%%%%%%%%%%%%%%%%%%%%%%%%%%%%%%%%%%%%%
% %%%%%%%%%%%%%%%%%%%%%%%%%%%%%%%%%%%%%%%%%%%%%%%%%%%%%%%%%%%%%%%%%%%%%%%%%%%%%%
% \section{Implementation}
%\iffalse
%<*package>
%\fi
%
% This section describes the definitions file |childdoc.def|.

% The definitions cannot be loaded using |\usepackage| or |\RequirePackage|
% which has a mechanism to prevent loading a style file more than once.
% When loading the definitions by means of |\input|
% multiple instances have to be prevented manually:
%\iffalse
%This code needs to be before the `\ProvidesFile' directive
%which is defined at the beginning of this file.
%Therefore it is also placed there and commented out here.
%</package>
%<*discard>
%\fi
%    \begin{macrocode}
\ifdefined\childdocmain\endinput\fi
%    \end{macrocode}
%\iffalse
%</discard>
%<*package>
%\fi
%
% \macro{\ifchilddoc}
% \macro{\ifchilddocmanual}
% The conditional |\ifchilddoc| tells whether a
% child (true) or main (false) document is being compiled.
% The conditional |\ifchilddocmanual| tells whether
% the |\includeonly| mechanism is used (false) or
% the selection of child files must be performed manually (true).
% The definitions initialise to false:
%    \begin{macrocode}
\newif\ifchilddoc
\newif\ifchilddocmanual
%    \end{macrocode}

% \macro{\childdocname}
% \macro{\childdocjob}
% The macro |\childdocname| stores the name of the main document
% to be compiled. The macro |\childdocjob| stores the name of
% the document on which the \LaTeX{} compiler was originally invoked.
% The content of |\jobname| cannot be compared
% to filenames specified in the source due to different catcodes.
% The following code rescans |\jobname|, stores the result
% in |\childdocname| and saves a copy in |\childdocjob|:
%    \begin{macrocode}
\edef\childdocname{\scantokens\expandafter{\jobname\noexpand}}
\let\childdocjob\childdocname
%    \end{macrocode}

% \macro{\childdocdisable}
% The macro |\childdocdisable| prevents the main file
% from being processed more than once.
% At this stage, the main document command |\childdocmain|
% is assumed to be called once again where it should do nothing.
% Any subsequent call to it should prevent
% a secondary processing of the main document
% It overwrites the forwarding commands
% |\childdocof| and |\childdocforward|
% with empty macros to prevent further inclusions of the main document:
%    \begin{macrocode}
\newcommand{\childdocdisable}
{
  \renewcommand{\childdocmain}[1]{\renewcommand{\childdocmain}[1]{\endinput}}
  \renewcommand{\childdocof}[1]{}
  \renewcommand{\childdocby}[2][]{}
  \renewcommand{\childdocforward}[2][]{}
  \renewcommand{\childdocdisable}{}
}
%    \end{macrocode}

% \macro{\childdocmain}
% The macro |\childdocmain| is to be called at the top of the main file
% with nothing or the main filename (without extension) as argument.
% First, it breaks loops.
% If the argument is not empty and does not match |\childdocname|
% (which is set by the first inclusion of |childdoc.def|),
% |\ifchilddoc| is set to true, |\includeonly| is applied to the child file
% and |\jobname| is set to the main file
% (for proper handling of |.aux| files):
%    \begin{macrocode}
\newcommand{\childdocmain}[1]
{
  \childdocdisable\childdocmain{}
  \if?#1?\else
    \begingroup
      \def\childdoctmp{#1}
      \ifx\childdoctmp\childdocname
        \def\childdoctmp{}
      \else
        \def\childdoctmp
        {
          \childdoctrue
          \includeonly{\childdocname}
          \def\childdocjob{#1}
          \def\jobname{#1}
        }
      \fi
      \expandafter
    \endgroup
    \childdoctmp
  \fi
}
%    \end{macrocode}

% \macro{\childdocof}
% The command |\childdocof| redirects
% compilation to the main file |#1|.
%    \begin{macrocode}
\newcommand{\childdocof}[1]
{
  \childdocdisable
  \childdoctrue
  \includeonly{\childdocname}
  \def\jobname{#1}
  \def\childdocjob{#1}
  \input{#1}
}
%    \end{macrocode}

% \macro{\childdocby}
% The command |\childdocby| ....
%    \begin{macrocode}
\newcommand{\childdocby}[2][]
{
  \childdocdisable
  \childdoctrue
  \childdocmanualtrue
  \if?#1?\else
    \def\jobname{#2}
  \fi
  \def\childdocjob{#2}
  \input{#2}
  \endinput
}
%    \end{macrocode}

% \macro{\childdocforward}
% The command |\childdocforward| redirects
% compilation to the main file or
% (if the optional argument is given) a child file.
% Parameters are set as if the main file
% or a child file starting with |\childdocof| was compiled.
% Then compilation is handed over to the main file:
%    \begin{macrocode}
\newcommand{\childdocforward}[2][]
{
  \begingroup
    \if?#1?
      \def\childdoctmp
      {
        \def\childdocname{#2}
        \def\childdocjob{#2}
        \def\jobname{#2}
        \input{#2}
        \endinput
      }
    \else
      \def\childdoctmp
      {
        \childdocdisable
        \def\childdocname{#2}
        \childdoctrue
        \includeonly{#2}
        \def\childdocjob{#1}
        \def\jobname{#1}
        \input{#1}
        \endinput
      }
    \fi
    \expandafter
  \endgroup
  \childdoctmp
}
%    \end{macrocode}

% \macro{\childdocforwardprefix}
% The command |\childdocforwardprefix| redirects
% compilation to the main or a child file by means of a pattern.
% The prefix |#1| in the current filename is replaced by |#2|
% and the suffix of the current filename is kept
% (it is assumed that the filename does not contain the substring `|~~~|'
% which is used as a delimiter).
% Compilation is handed over to the new file by |\childdocforward|:
%    \begin{macrocode}
\newcommand{\childdocforwardprefix}[3][]
{
  \begingroup
    \def\childdocextract #2##1~~~{\def\childdoctmp{\childdocforward[#1]{#3##1}}}
    \expandafter\childdocextract\childdocname~~~
    \expandafter
  \endgroup
  \childdoctmp
}
%    \end{macrocode}

% \macro{\childdoc}
% The deprecated macro |\childdoc| is a legacy version of |\childdocmain|:
%    \begin{macrocode}
\newcommand{\childdoc}{\childdocmain}
%    \end{macrocode}

% \macro{\childdocredirect}
% The deprecated macro |\childdocredirect| is a legacy version
% of |\childdocforward| and |\childdocforwardprefix|:
%    \begin{macrocode}
\newcommand{\childdocredirect}[2][]
{
  \begingroup
    \if?#1?
      \def\childdoctmp{\childdocforward{#2}}
    \else
      \def\childdoctmp{\childdocforwardprefix{#1}{#2}}
    \fi
    \expandafter
  \endgroup
  \childdoctmp
}
%    \end{macrocode}

%\iffalse
%</package>
%\fi
%
\endinput
|\\
|\childdocforward[|\textit{main}|]{|\textit{dest}|}|\\
\end{tabular}
\end{center}
%
The argument \textit{dest} is the destination file
(without extension).
It should be the main file or one of the child files.
Note that further \textsf{childdoc} directives
such as |\childdocof| and |\childdocforward|
in the indicated file will be processed in this form.
The optional argument \textit{main}
passes on directly to the main file \textit{main}
while pretending to compile the child \textit{dest}.
This form behaves as if \textit{dest}
issues |\childdocof{|\textit{main}|}| right away,
and no further \textsf{childdoc} directives will be processed.

%%%%%%%%%%%%%%%%%%%%%%%%%%%%%%%%%%%%%%%%
\DescribeMacro{\...prefix}
In the alternative form |\childdocforwardprefix|,
%
\begin{center}
\begin{tabular}{l}
|% \iffalse
%
% childdoc.dtx Copyright (C) 2017-2018 Niklas Beisert
%
% This work may be distributed and/or modified under the
% conditions of the LaTeX Project Public License, either version 1.3
% of this license or (at your option) any later version.
% The latest version of this license is in
%   http://www.latex-project.org/lppl.txt
% and version 1.3 or later is part of all distributions of LaTeX
% version 2005/12/01 or later.
%
% This work has the LPPL maintenance status `maintained'.
%
% The Current Maintainer of this work is Niklas Beisert.
%
% This work consists of the files childdoc.dtx and childdoc.ins
% and the derived files childdoc.def and cdocsamp.tex with
% cdocsch1.tex, cdocsch2.tex, cdocsdrf.tex, cdocsfn1.tex, cdocsfn2.tex.
%
%<package>\ifdefined\childdocmain\endinput\fi
%<package>\ProvidesFile{childdoc.def}[2018/12/30 v2.0 child document driver]
%<samplemain>\ProvidesFile{cdocsamp.tex}[2018/12/30 v2.0 sample for childdoc]
%<*driver>
%\ProvidesFile{childdoc.drv}[2018/12/30 v2.0 childdoc reference manual file]
\PassOptionsToClass{10pt,a4paper}{article}
\documentclass{ltxdoc}

\usepackage[margin=35mm]{geometry}
\usepackage{hyperref}
\usepackage{hyperxmp}
\usepackage[usenames]{color}

\hypersetup{colorlinks=true}
\hypersetup{pdfstartview=FitH}
\hypersetup{pdfpagemode=UseNone}
\hypersetup{pdfsource={}}
\hypersetup{pdflang={en-UK}}
\hypersetup{pdfcopyright={Copyright 2017-2018 Niklas Beisert.
  This work may be distributed and/or modified under the
  conditions of the LaTeX Project Public License, either version 1.3
  of this license or (at your option) any later version.}}
\hypersetup{pdflicenseurl={http://www.latex-project.org/lppl.txt}}
\hypersetup{pdfcontactaddress={ETH Zurich, ITP, HIT K,
  Wolfgang-Pauli-Strasse 27}}
\hypersetup{pdfcontactpostcode={8093}}
\hypersetup{pdfcontactcity={Zurich}}
\hypersetup{pdfcontactcountry={Switzerland}}
\hypersetup{pdfcontactemail={nbeisert@itp.phys.ethz.ch}}
\hypersetup{pdfcontacturl={http://people.phys.ethz.ch/\xmptilde nbeisert/}}

\newcommand{\secref}[1]{\hyperref[#1]{section \ref*{#1}}}

\parskip1ex
\parindent0pt
\let\olditemize\itemize
\def\itemize{\olditemize\parskip0pt}

\begin{document}

\title{The \textsf{childdoc} Package}
\hypersetup{pdftitle={The childdoc Package}}
\author{Niklas Beisert\\[2ex]
  Institut f\"ur Theoretische Physik\\
  Eidgen\"ossische Technische Hochschule Z\"urich\\
  Wolfgang-Pauli-Strasse 27, 8093 Z\"urich, Switzerland\\[1ex]
  \href{mailto:nbeisert@itp.phys.ethz.ch}
  {\texttt{nbeisert@itp.phys.ethz.ch}}}
\hypersetup{pdfauthor={Niklas Beisert}}
\hypersetup{pdfsubject={Manual for the LaTeX2e Package childdoc}}
\date{30 December 2018, \textsf{v2.0}}
\maketitle

\begin{abstract}\noindent
\textsf{childdoc} is a \LaTeXe{} package
that enables the direct compilation
of document sections included by |\include|
to individual files.
\end{abstract}

\begingroup
\parskip0ex
\tableofcontents
\endgroup

%%%%%%%%%%%%%%%%%%%%%%%%%%%%%%%%%%%%%%%%%%%%%%%%%%%%%%%%%%%%%%%%%%%%%%%%%%%%%%%%
%%%%%%%%%%%%%%%%%%%%%%%%%%%%%%%%%%%%%%%%%%%%%%%%%%%%%%%%%%%%%%%%%%%%%%%%%%%%%%%%
\section{Introduction}

\LaTeX{} provides a mechanism to structure a large document (such as a book)
into a main file and several child files (containing the chapters)
using the |\include| command.
This mechanism is beneficial for documents
which span hundreds of pages in order to
make the source file(s) more manageable.
Moreover, compilation can be restricted to
selected child files by means of the |\includeonly| command.
The latter feature can be used to reduce the compilation time while editing
(this was significantly more useful in the earlier days of \LaTeX{})
or to generate a smaller document which is easier to navigate.
Another application of |\includeonly| is to generate
documents consisting of selected parts of the complete document.

However, there are a few drawbacks of the plain |\include| mechanism:
\begin{itemize}
\item
The child files cannot be compiled on their own,
they can only be compiled via the main file.
A naive editing environment
(such as a text editor with an option
to have the current file processed by \LaTeX)
may require one to switch to the main file before compiling;
attempting to compile the child file produces errors.
\item
The main file must be modified (each time)
to adjust the |\includeonly| command
to the present needs. This easily leaves the main file in a messy state.
\item
The generated document will always carry the filename
of the main document. This is inconvenient if
several child files are to be compiled and
to be kept for distribution.
\end{itemize}

The present package provides a simple interface
to make child files individually compilable by \LaTeX{}.
Compiling a child file then has the same effect as compiling
the main file with an |\includeonly| command
to select the appropriate child.
Moreover the generated document will carry the name of the child
rather than the main file.
This resolves all three above issues.

This feature is meant to make the editing of books,
thesis documents and lecture notes somewhat more convenient.
However, the package can also be used efficiently for
composing a series of documents (such as exercise sheets)
which are typically distributed individually.
It then assists the author in generating the individual documents
(potentially in different versions)
as well as a document containing the collected series.
Another application is in developing style files
or other kinds of included material
where compilation of the style file could redirect
to a sample or test file.

%%%%%%%%%%%%%%%%%%%%%%%%%%%%%%%%%%%%%%%%%%%%%%%%%%%%%%%%%%%%%%%%%%%%%%%%%%%%%%%%
%%%%%%%%%%%%%%%%%%%%%%%%%%%%%%%%%%%%%%%%%%%%%%%%%%%%%%%%%%%%%%%%%%%%%%%%%%%%%%%%
\section{Usage}

First of all, the package \textsf{childdoc} is \emph{not} a standard
\LaTeXe{} |.sty| style file! Therefore it needs to be invoked in
a non-standard way.

%%%%%%%%%%%%%%%%%%%%%%%%%%%%%%%%%%%%%%%%%%%%%%%%%%%%%%%%%%%%%%%%%%%%%%%%%%%%%%%%
\subsection{Included Files}
\label{sec:include}

%%%%%%%%%%%%%%%%%%%%%%%%%%%%%%%%%%%%%%%%
\DescribeMacro{\childdocmain}
To use the package, add the commands
\begin{center}
\begin{tabular}{l}
|% \iffalse
%
% childdoc.dtx Copyright (C) 2017-2018 Niklas Beisert
%
% This work may be distributed and/or modified under the
% conditions of the LaTeX Project Public License, either version 1.3
% of this license or (at your option) any later version.
% The latest version of this license is in
%   http://www.latex-project.org/lppl.txt
% and version 1.3 or later is part of all distributions of LaTeX
% version 2005/12/01 or later.
%
% This work has the LPPL maintenance status `maintained'.
%
% The Current Maintainer of this work is Niklas Beisert.
%
% This work consists of the files childdoc.dtx and childdoc.ins
% and the derived files childdoc.def and cdocsamp.tex with
% cdocsch1.tex, cdocsch2.tex, cdocsdrf.tex, cdocsfn1.tex, cdocsfn2.tex.
%
%<package>\ifdefined\childdocmain\endinput\fi
%<package>\ProvidesFile{childdoc.def}[2018/12/30 v2.0 child document driver]
%<samplemain>\ProvidesFile{cdocsamp.tex}[2018/12/30 v2.0 sample for childdoc]
%<*driver>
%\ProvidesFile{childdoc.drv}[2018/12/30 v2.0 childdoc reference manual file]
\PassOptionsToClass{10pt,a4paper}{article}
\documentclass{ltxdoc}

\usepackage[margin=35mm]{geometry}
\usepackage{hyperref}
\usepackage{hyperxmp}
\usepackage[usenames]{color}

\hypersetup{colorlinks=true}
\hypersetup{pdfstartview=FitH}
\hypersetup{pdfpagemode=UseNone}
\hypersetup{pdfsource={}}
\hypersetup{pdflang={en-UK}}
\hypersetup{pdfcopyright={Copyright 2017-2018 Niklas Beisert.
  This work may be distributed and/or modified under the
  conditions of the LaTeX Project Public License, either version 1.3
  of this license or (at your option) any later version.}}
\hypersetup{pdflicenseurl={http://www.latex-project.org/lppl.txt}}
\hypersetup{pdfcontactaddress={ETH Zurich, ITP, HIT K,
  Wolfgang-Pauli-Strasse 27}}
\hypersetup{pdfcontactpostcode={8093}}
\hypersetup{pdfcontactcity={Zurich}}
\hypersetup{pdfcontactcountry={Switzerland}}
\hypersetup{pdfcontactemail={nbeisert@itp.phys.ethz.ch}}
\hypersetup{pdfcontacturl={http://people.phys.ethz.ch/\xmptilde nbeisert/}}

\newcommand{\secref}[1]{\hyperref[#1]{section \ref*{#1}}}

\parskip1ex
\parindent0pt
\let\olditemize\itemize
\def\itemize{\olditemize\parskip0pt}

\begin{document}

\title{The \textsf{childdoc} Package}
\hypersetup{pdftitle={The childdoc Package}}
\author{Niklas Beisert\\[2ex]
  Institut f\"ur Theoretische Physik\\
  Eidgen\"ossische Technische Hochschule Z\"urich\\
  Wolfgang-Pauli-Strasse 27, 8093 Z\"urich, Switzerland\\[1ex]
  \href{mailto:nbeisert@itp.phys.ethz.ch}
  {\texttt{nbeisert@itp.phys.ethz.ch}}}
\hypersetup{pdfauthor={Niklas Beisert}}
\hypersetup{pdfsubject={Manual for the LaTeX2e Package childdoc}}
\date{30 December 2018, \textsf{v2.0}}
\maketitle

\begin{abstract}\noindent
\textsf{childdoc} is a \LaTeXe{} package
that enables the direct compilation
of document sections included by |\include|
to individual files.
\end{abstract}

\begingroup
\parskip0ex
\tableofcontents
\endgroup

%%%%%%%%%%%%%%%%%%%%%%%%%%%%%%%%%%%%%%%%%%%%%%%%%%%%%%%%%%%%%%%%%%%%%%%%%%%%%%%%
%%%%%%%%%%%%%%%%%%%%%%%%%%%%%%%%%%%%%%%%%%%%%%%%%%%%%%%%%%%%%%%%%%%%%%%%%%%%%%%%
\section{Introduction}

\LaTeX{} provides a mechanism to structure a large document (such as a book)
into a main file and several child files (containing the chapters)
using the |\include| command.
This mechanism is beneficial for documents
which span hundreds of pages in order to
make the source file(s) more manageable.
Moreover, compilation can be restricted to
selected child files by means of the |\includeonly| command.
The latter feature can be used to reduce the compilation time while editing
(this was significantly more useful in the earlier days of \LaTeX{})
or to generate a smaller document which is easier to navigate.
Another application of |\includeonly| is to generate
documents consisting of selected parts of the complete document.

However, there are a few drawbacks of the plain |\include| mechanism:
\begin{itemize}
\item
The child files cannot be compiled on their own,
they can only be compiled via the main file.
A naive editing environment
(such as a text editor with an option
to have the current file processed by \LaTeX)
may require one to switch to the main file before compiling;
attempting to compile the child file produces errors.
\item
The main file must be modified (each time)
to adjust the |\includeonly| command
to the present needs. This easily leaves the main file in a messy state.
\item
The generated document will always carry the filename
of the main document. This is inconvenient if
several child files are to be compiled and
to be kept for distribution.
\end{itemize}

The present package provides a simple interface
to make child files individually compilable by \LaTeX{}.
Compiling a child file then has the same effect as compiling
the main file with an |\includeonly| command
to select the appropriate child.
Moreover the generated document will carry the name of the child
rather than the main file.
This resolves all three above issues.

This feature is meant to make the editing of books,
thesis documents and lecture notes somewhat more convenient.
However, the package can also be used efficiently for
composing a series of documents (such as exercise sheets)
which are typically distributed individually.
It then assists the author in generating the individual documents
(potentially in different versions)
as well as a document containing the collected series.
Another application is in developing style files
or other kinds of included material
where compilation of the style file could redirect
to a sample or test file.

%%%%%%%%%%%%%%%%%%%%%%%%%%%%%%%%%%%%%%%%%%%%%%%%%%%%%%%%%%%%%%%%%%%%%%%%%%%%%%%%
%%%%%%%%%%%%%%%%%%%%%%%%%%%%%%%%%%%%%%%%%%%%%%%%%%%%%%%%%%%%%%%%%%%%%%%%%%%%%%%%
\section{Usage}

First of all, the package \textsf{childdoc} is \emph{not} a standard
\LaTeXe{} |.sty| style file! Therefore it needs to be invoked in
a non-standard way.

%%%%%%%%%%%%%%%%%%%%%%%%%%%%%%%%%%%%%%%%%%%%%%%%%%%%%%%%%%%%%%%%%%%%%%%%%%%%%%%%
\subsection{Included Files}
\label{sec:include}

%%%%%%%%%%%%%%%%%%%%%%%%%%%%%%%%%%%%%%%%
\DescribeMacro{\childdocmain}
To use the package, add the commands
\begin{center}
\begin{tabular}{l}
|\input{childdoc.def}|\\
|\childdocmain{}|\\
\end{tabular}
\end{center}
at the very top of the main \LaTeX{} file,
in particular \emph{before} the |\documentclass| statement!
The argument of |\childdocmain| should be left empty
(but it must be present).

%%%%%%%%%%%%%%%%%%%%%%%%%%%%%%%%%%%%%%%%
\DescribeMacro{\childdocof}
Furthermore, add the commands
\begin{center}
\begin{tabular}{l}
|\input{childdoc.def}|\\
|\childdocof{|\textit{main}|}|\\
\end{tabular}
\end{center}
at the top of every child file \textit{child}
which is included by |\include{|\textit{child}|}|
from within the main file
(or at least for those files to be compiled individually).
The argument \textit{main} must be the filename of the main file.

There are a couple of
considerations in setting up the main and child documents:

%%%%%%%%%%%%%%%%%%%%%%%%%%%%%%%%%%%%%%%%
\paragraph{Restrictions.}

Please note the following restrictions:
\begin{itemize}
\item
|\childdocmain| must be called with one argument \textit{main}
to ensure compatibility with earlier version of the package.
It must either be empty (|\childdocmain{}|)
or precisely match the filename of the main file in which it is specified.
See \secref{sec:detection} for further information.
\item
The filename \textit{main} must be specified without the |.tex| extension.
\item
The filename \textit{main} is case sensitive
(even in case-insensitive file systems)
due to internal string comparison.
\item
The argument \textit{main} should be fully expanded, it cannot be a macro.
\item
Subdirectories and special characters should be avoided in filenames.
\item
The command |\childdocmain{|\textit{main}|}| must be followed by a whitespace.
It should not be followed immediately by another command
or by a comment mark `|%|'.
This is because the \TeX{} parser reads the token immediately following
the argument of |\childdocmain| and puts it
at the beginning of every child section;
however, a white\-space is ignored.
\end{itemize}

%%%%%%%%%%%%%%%%%%%%%%%%%%%%%%%%%%%%%%%%
\paragraph{Content of Main File.}

It is advisable to place all content in the child files included by |\include|.
Any output contained in the main file will appear in all child documents
unless suppressed manually;
it cannot be suppressed automatically by the |\includeonly| directive
and thus should normally be avoided.
A method to include some content in the main file
by means of conditional processing is described in \secref{sec:conditional}.

%%%%%%%%%%%%%%%%%%%%%%%%%%%%%%%%%%%%%%%%
\paragraph{Page Numbering.}

When only a part of the document is compiled,
the appropriate numbering of pages
(as well as other status parameters)
is determined from the |.aux| files.
The latter contain information from previous passes.
However this information needs to propagate through
all intermediate child documents.
Therefore the page numbering in child documents may well
be inconsistent until the complete document is compiled at least once.

A useful (if unconventional) way to always ensure a consistent
page numbering is to restart the numbering in each child document
and denote the pages by `\textit{child}|.|\textit{page}'
where \textit{child} represents the chapter/section number of the child file.
This can be achieved by the command
|\numberwithin{page}{|\textit{child}|}|
of the \textsf{amsmath} package
where \textit{child} can be |chapter| or |section|
depending on the chosen structuring.
Alternatively, one can modify the macro |\thepage| appropriately
and reset the counter |page| at the start of each child file.

%%%%%%%%%%%%%%%%%%%%%%%%%%%%%%%%%%%%%%%%%%%%%%%%%%%%%%%%%%%%%%%%%%%%%%%%%%%%%%%%
\subsection{Conditional Processing}
\label{sec:conditional}

The package provides a mechanism to compile different versions
of a document. To customise the versions further some conditional processing
can come in handy to distinguish which version is being compiled.
The package provides two macros to describe the compilation context:

%%%%%%%%%%%%%%%%%%%%%%%%%%%%%%%%%%%%%%%%
\DescribeMacro{\ifchilddoc}
The conditional |\ifchilddoc| distinguishes between the compilation of
child documents and the main document:
%
\begin{center}
|\ifchilddoc |\textit{child-code}| |[|\||else |\textit{main-code}]| \||fi|
\end{center}

%%%%%%%%%%%%%%%%%%%%%%%%%%%%%%%%%%%%%%%%
\DescribeMacro{\childdocname}
\DescribeMacro{\childdocjob}
The macro |\childdocname| contains the filename (without extension)
of the main or child file being processed.
Note that |\childdocjob| will always contain the name of the main file.

%%%%%%%%%%%%%%%%%%%%%%%%%%%%%%%%%%%%%%%%
\paragraph{Title Page.}

Conditional processing can be used to include a title or banner page
in the main document when proper precautions are taken.
Importantly, the code in the main file should ensure that the page counter
(as well as other status parameters which are stored in the |.aux| files)
takes the same value after the conditional processing.
Otherwise the page numbers may take divergent values
depending on which part is compiled.

For example, a title page could be declared by:
%
\begin{center}
\begin{tabular}{l}
|\ifchilddoc\||else|\\
|\addtocounter{page}{-1}|\\
\textit{code for title page}\\
|\newpage|\\
|\||fi|
\end{tabular}
\end{center}
%
A banner page for the child documents can be generated by:
%
\begin{center}
\begin{tabular}{l}
|\ifchilddoc|\\
|\addtocounter{page}{-1}|\\
\textit{code for banner page}\\
|\newpage|\\
|\||fi|
\end{tabular}
\end{center}
%
Here one could write a message such as:
\begin{center}
|This is the part \childdocname{} of \childdocjob{}.|
\end{center}

%%%%%%%%%%%%%%%%%%%%%%%%%%%%%%%%%%%%%%%%%%%%%%%%%%%%%%%%%%%%%%%%%%%%%%%%%%%%%%%%
\subsection{Flags}
\label{sec:flags}

The package makes it easy to generate different versions
of the main or child documents.
To this end compilation flags can be defined
and assigned different default values.
They will be particularly useful in conjunction
with the forwarding mechanism described in \secref{sec:forward}.

For example, it may be useful to have a flag |\version|
which can be set to |draft| or |final|.
The document source will contain some conditional code
depending on the value of |\version|.
Suppose further, the flag should default to |final| for the main file
and to |draft| for child files
which is a natural assignment for editing the document.
This is achieved by placing the following code
in the preamble of the main document
(below the |\childdocmain| directive):
%
\begin{center}
\begin{tabular}{l}
|\ifchilddoc|\\
|\providecommand{\version}{draft}|\\
|\||else|\\
|\providecommand{\version}{final}|\\
|\||fi|
\end{tabular}
\end{center}
%
The definition by |\providecommand| makes sure
that previous definitions are not overwritten.
Further statements |\providecommand{\version}{...}|
can thus be added before the above code to override it.

For the main file, one might add a line
(between |\childdocmain| and the above block)
%
\begin{center}
|%\ifchilddoc\||else\providecommand{\version}{draft}\||fi|
\end{center}
%
which can be uncommented to produce a draft version.
Likewise one can add a line to the very top of a child file
(above the |\childdocof{|\textit{main}|}| directive)
%
\begin{center}
|%\providecommand{\version}{final}|
\end{center}
%
which can be uncommented to produce the final version of this child document.

%%%%%%%%%%%%%%%%%%%%%%%%%%%%%%%%%%%%%%%%%%%%%%%%%%%%%%%%%%%%%%%%%%%%%%%%%%%%%%%%
\subsection{Forwarding}
\label{sec:forward}

Different versions of the main or child documents
using compilation flags as described in \secref{sec:flags}
can be (permanently) stored in different files
for convenient compilation, viewing and distribution.
To this end, the package defines a command
to pass on compilation to a different file:

%%%%%%%%%%%%%%%%%%%%%%%%%%%%%%%%%%%%%%%%
\DescribeMacro{\childdocforward}
The command |\childdocforward| redirects processing to
another source file:
%
\begin{center}
\begin{tabular}{l}
|\input{childdoc.def}|\\
|\childdocforward[|\textit{main}|]{|\textit{dest}|}|\\
\end{tabular}
\end{center}
%
The argument \textit{dest} is the destination file
(without extension).
It should be the main file or one of the child files.
Note that further \textsf{childdoc} directives
such as |\childdocof| and |\childdocforward|
in the indicated file will be processed in this form.
The optional argument \textit{main}
passes on directly to the main file \textit{main}
while pretending to compile the child \textit{dest}.
This form behaves as if \textit{dest}
issues |\childdocof{|\textit{main}|}| right away,
and no further \textsf{childdoc} directives will be processed.

%%%%%%%%%%%%%%%%%%%%%%%%%%%%%%%%%%%%%%%%
\DescribeMacro{\...prefix}
In the alternative form |\childdocforwardprefix|,
%
\begin{center}
\begin{tabular}{l}
|\input{childdoc.def}|\\
|\childdocforwardprefix[|\textit{main}|]{|\textit{prefix}|}{|\textit{dest}|}|
\end{tabular}
\end{center}
%
the destination file is determined by a pattern
depending on the current file:
To make this work, the current file must be called
`{\textit{prefix}\hspace{0.2em}\textit{suffix}}'
with \textit{prefix} matching precisely the argument.
Processing is then passed on to the file
`{\textit{dest}\hspace{0.2em}\textit{suffix}}'.
Surely, the same effect is achieved by
directly specifying the
argument `{\textit{dest}\hspace{0.2em}\textit{suffix}}'
in the first form.
However, that requires to set up a different file
for each child. With the alternative form of the command
all these files can have exactly the same content
which simplifies setting them up and maintaining them.

For example, the following file |draft.tex|
with a compilation flag |\version| as described in \secref{sec:flags}
compiles the main document as a draft:
%
\begin{center}
\begin{tabular}{l}
|\def\version{draft}|\\
|\input{childdoc.def}|\\
|\childdocforward{|\textit{main}|}|
\end{tabular}
\end{center}
%
Likewise, the following files |final|\textit{nn}|.tex|
compile the final version of the child document
|child|\textit{nn}|.tex|:
%
\begin{center}
\begin{tabular}{l}
|\def\version{final}|\\
|\input{childdoc.def}|\\
|\childdocforwardprefix{final}{child}|
\end{tabular}
\end{center}
%

Note that when several versions of a main file and/or of each child file
are to be generated, it may be convenient to set up a |Makefile| or
shell script to automatise the process.

%%%%%%%%%%%%%%%%%%%%%%%%%%%%%%%%%%%%%%%%%%%%%%%%%%%%%%%%%%%%%%%%%%%%%%%%%%%%%%%%
\subsection{Command Line Processing}
\label{sec:commandline}

The effect of redirection files can also be achieved by invoking
the \LaTeX{} compiler with a more elaborate command line.
Most conveniently this should be done as part
of a shell script or a |Makefile|.

When using \textsf{childdoc} in the main file, the following
command lines effectively perform a redirection
(note that depending on the shell being used,
backslashes may have to be doubled: `|\|' $\to$ `|\\|'):
%
\begin{center}
|... -jobname "|\textit{target}|" |\\|"|[\textit{flags}]%
|\input{childdoc.def}\childdocforward[|\textit{main}|]{|\textit{dest}|}"|
\end{center}
%
Here \textit{target} is the name of the output file,
\textit{main} is the name of the main file
and \textit{dest} is the name of the main or child file to be processed
(all filenames without extensions).
The optional argument \textit{main} can be omitted
if \textit{main} matches \textit{dest}.
Optionally, compilation \textit{flags} can be defined via |\def| commands.
This command line makes the \TeX{} engine believe
it is compiling the file \textit{target}
whose content is specified as the latter parameter.
The provided code then forwards the processing to
\textit{main} or \textit{dest} as described in \secref{sec:forward}.

%%%%%%%%%%%%%%%%%%%%%%%%%%%%%%%%%%%%%%%%%%%%%%%%%%%%%%%%%%%%%%%%%%%%%%%%%%%%%%%%
\subsection{Include by Input}
\label{sec:input}

Including child documents by |\include| has some restrictions by design.
Most notably, the content of a child document always occupies
its own set of pages; pages cannot be shared between child documents.
Usually, this behaviour makes perfect sense
because each child document contain an essential part of the document.
However, in some situations it may be desirable to compose
a document from a collection of parts
without having mandatory page breaks between then.
For this case, the package
provides a mechanism to include parts
by |\input| which can also be processed individually.
However, by construction this mechanism
requires manual handling of the content to be output.

%%%%%%%%%%%%%%%%%%%%%%%%%%%%%%%%%%%%%%%%
\DescribeMacro{\ifchilddocmanual}
The main file should be prepared as usual, see \secref{sec:include}.
However, the document body must make a distinction
between processing of an individual part and of the main document, e.g.:
%
\begin{center}
\begin{tabular}{l}
|\ifchilddocmanual|\\
|\input{\childdocname}|\\
|\||else|\\
\textit{document body with }|\input{|\textit{part}|}|\\
|\||fi|
\end{tabular}
\end{center}
%
The conditional |\ifchilddocmanual| is true whenever
a part to be included by |\input| is being compiled,
and the name of the part is stored in |\childdocname|.

%%%%%%%%%%%%%%%%%%%%%%%%%%%%%%%%%%%%%%%%
\DescribeMacro{\childdocby}
Each part to be included by |\input| should start with:
%
\begin{center}
\begin{tabular}{l}
|\input{childdoc.def}|\\
|\childdocby{|\textit{main}|}|\\
\end{tabular}
\end{center}
%
The directive |\childdocby| is similar to |\childdocof|
described in \secref{sec:include},
but the subsequent selection of content must be done manually.
To that end, both |\ifchilddoc| and |\ifchilddocmanual|
will be true upon processing of a part,
and the name of the part is stored in |\childdocname|.
Note that |\jobname| will be set to the filename of the current part
so that each part receives an individual |.aux| file
that does not interfere with the |.aux| file(s) of the main document.
This behaviour can be altered by the alternative form
|\childdocby[*]{|\textit{main}|}| (with a non-empty optional argument)
which uses the |.aux| file of the main document
by setting |\jobname| to \textit{main}.

%%%%%%%%%%%%%%%%%%%%%%%%%%%%%%%%%%%%%%%%%%%%%%%%%%%%%%%%%%%%%%%%%%%%%%%%%%%%%%%%
\subsection{Driver Development}
\label{sec:driver}

The \textsf{childdoc} mechanism can also be use for the development
of definition files such as \LaTeX{} styles or classes.
This case differs from the above setup with multiple parts
included by |\include| in that no |\includeonly| should be invoked.
This can be achieved by starting the include file
(before |\ProvidesPackage|) with:
%
\begin{center}
\begin{tabular}{l}
|\input{childdoc.def}|\\
|\childdocforward{|\textit{main}|}|\\
\end{tabular}
\end{center}
%
or alternatively with:
%
\begin{center}
\begin{tabular}{l}
|\input{childdoc.def}|\\
|\childdocby{|\textit{main}|}|\\
\end{tabular}
\end{center}
%
Both forms have slightly different effects as described above.
The main file is prepared as usual, see \secref{sec:include}.

%%%%%%%%%%%%%%%%%%%%%%%%%%%%%%%%%%%%%%%%%%%%%%%%%%%%%%%%%%%%%%%%%%%%%%%%%%%%%%%%
\subsection{Legacy Detection}
\label{sec:detection}

The directive |\childdocmain| in the main file can detect
whether the complete document or merely a child is to be compiled
even without using the directive |\childdocof|.
This method is deprecated because it is less robust
and there is no compelling reason to use it;
it is merely provided for backward compatibility
and it may be removed in future versions.

If the detection mechanism is to be used,
it is mandatory to correctly specify
the filename of the main file as the argument of |\childdocmain|:
%
\begin{center}
\begin{tabular}{l}
|\input{childdoc.def}|\\
|\childdocmain{|\textit{main}|}|\\
\end{tabular}
\end{center}
%
If |\jobname| does not match the argument \textit{main} of |\childdocmain|,
it is assumed that |\jobname| points to the child file to be compiled.
When using |\childdocmain| with the main file specified as argument,
it suffices to start a child file
with just |\input{|\textit{main}|}|
without loading of the package and using |\childdocof|.
If instead all processing is done
with the appropriate \textsf{childdoc} directives,
the argument of \textit{main} of |\childdocmain| can be empty.

An alternative version of the command line processing described
in \secref{sec:commandline} using the detection mechanism reads:
%
\begin{center}
|... -jobname "|\textit{target}|" "|[\textit{flags}]%
[|\def\jobname{|\textit{dest}|}|]|\input{|\textit{main}|}"|
\end{center}

%%%%%%%%%%%%%%%%%%%%%%%%%%%%%%%%%%%%%%%%%%%%%%%%%%%%%%%%%%%%%%%%%%%%%%%%%%%%%%%%
\subsection{Manual Code}
\label{sec:manual}

In case one cannot be certain whether the definitions file |childdoc.def|
is installed on the target \TeX{} distribution
and one prefers not to ship it,
it is conceivable to paste a few relevant commands into the sources.

To that end, drop all statements |\input{childdoc.def}|
and perform the replacements as outlined below.
Instead of |\childdocmain{|\textit{main}|}| add the following code
to the top of the main file:
%
\begin{center}
\begin{tabular}{l}
|\||ifdefined\childdocname\endinput\||fi\newif\ifchilddoc|\\
|\edef\childdocname{\scantokens\expandafter{\jobname\noexpand}}|\\
|\def\childdocmain{|\textit{main}|}\||ifx\childdocmain\childdocname\||else|\\
|\childdoctrue\includeonly{\childdocname}\let\jobname\childdocmain\||fi|\\
\end{tabular}
\end{center}
%
Instead of |\childdocof{|\textit{main}|}| just include the main file
at the top of each child file:
%
\begin{center}
|\input{|\textit{main}|}|
\end{center}
%
A simple redirection |\childdocforward{|\textit{dest}|}| is achieved by:
%
\begin{center}
|\def\jobname{|\textit{dest}|}\input{\jobname}|
\end{center}
%
The redirection with prefix
|\childdocforwardprefix[|\textit{prefix}|]{|\textit{dest}|}|
is accomplished by:
%
\begin{center}
\begin{tabular}{l}
|{\edef\jobname{\scantokens\expandafter{\jobname\noexpand}}|\\
|\def\redirectjob |\textit{prefix}|#1~~~{\gdef\jobname{|\textit{dest}|#1}}|\\
|\expandafter\redirectjob\jobname~~~}\input{\jobname}|
\end{tabular}
\end{center}

In an alternative approach,
child documents can be compiled by a specific command line
without additional code or specific definitions:
%
\begin{center}
|... -jobname "|\textit{target}|" "|[\textit{flags}]%
|\includeonly{|\textit{dest}|}\input{|\textit{main}|}"|
\end{center}
%

%%%%%%%%%%%%%%%%%%%%%%%%%%%%%%%%%%%%%%%%%%%%%%%%%%%%%%%%%%%%%%%%%%%%%%%%%%%%%%%%
%%%%%%%%%%%%%%%%%%%%%%%%%%%%%%%%%%%%%%%%%%%%%%%%%%%%%%%%%%%%%%%%%%%%%%%%%%%%%%%%
\section{Information}

%%%%%%%%%%%%%%%%%%%%%%%%%%%%%%%%%%%%%%%%%%%%%%%%%%%%%%%%%%%%%%%%%%%%%%%%%%%%%%%%
\subsection{Copyright}

Copyright \copyright{} 2017--2018 Niklas Beisert

This work may be distributed and/or modified under the
conditions of the \LaTeX{} Project Public License, either version 1.3
of this license or (at your option) any later version.
The latest version of this license is in
  \url{http://www.latex-project.org/lppl.txt}
and version 1.3 or later is part of all distributions of \LaTeX{}
version 2005/12/01 or later.

This work has the LPPL maintenance status `maintained'.

The Current Maintainer of this work is Niklas Beisert.

This work consists of the files |README.txt|, |childdoc.ins| and |childdoc.dtx|
as well as the derived files |childdoc.def|, |cdocsamp.tex|
with |cdocsch1.tex|, |cdocsch2.tex|, |cdocspt3.tex|, |cdocspt4.tex|,
|cdocsdrf.tex|, |cdocsfn1.tex|, |cdocsfn2.tex|
as well as |childdoc.pdf|.

%%%%%%%%%%%%%%%%%%%%%%%%%%%%%%%%%%%%%%%%%%%%%%%%%%%%%%%%%%%%%%%%%%%%%%%%%%%%%%%%
\subsection{Files and Installation}

The package consists of the files:
%
\begin{center}
\begin{tabular}{ll}
    |README.txt|   & readme file \\
    |childdoc.ins| & installation file \\
    |childdoc.dtx| & source file \\
    |childdoc.def| & definition file \\
    |cdocsamp.tex| & sample main file \\
    |cdocsch1.tex| & sample include file \\
    |cdocsch2.tex| & sample include file \\
    |cdocspt3.tex| & sample part file \\
    |cdocspt4.tex| & sample part file \\
    |cdocsdrf.tex| & sample redirection file \\
    |cdocsfn1.tex| & sample redirection file \\
    |cdocsfn2.tex| & sample redirection file \\
    |childdoc.pdf| & manual
\end{tabular}
\end{center}
%
The distribution consists of the files
|README.txt|, |childdoc.ins| and |childdoc.dtx|.
%
\begin{itemize}
\item
Run (pdf)\LaTeX{} on |childdoc.dtx|
to compile the manual |childdoc.pdf| (this file).
\item
Run \LaTeX{} on |childdoc.ins| to create the definitions file |childdoc.def|
and the sample |cdocsamp.tex| with include files
|cdocsch1.tex|, |cdocsch2.tex|, |cdocspt3.tex|, |cdocspt4.tex|,
|cdocsdrf.tex|, |cdocsfn1.tex|, |cdocsfn2.tex|.
Then copy the file |childdoc.def| to an appropriate directory of your \LaTeX{}
distribution, e.g.\ \textit{texmf-root}|/tex/latex/childdoc|.
\end{itemize}

%%%%%%%%%%%%%%%%%%%%%%%%%%%%%%%%%%%%%%%%%%%%%%%%%%%%%%%%%%%%%%%%%%%%%%%%%%%%%%%%
\subsection{Related CTAN Packages}

There are several other packages which offer a similar functionality:
%
\begin{itemize}
\item
The packages
\href{http://ctan.org/pkg/docmute}{\textsf{docmute}},
\href{http://ctan.org/pkg/includex}{\textsf{includex}} and
\href{http://ctan.org/pkg/standalone}{\textsf{standalone}}
provide commands to include only the document body of
a child file thus allowing both files to be compiled individually.
\item
The packages \href{http://ctan.org/pkg/subdocs}{\textsf{subdocs}}
and \href{http://ctan.org/pkg/subfiles}{\textsf{subfiles}}
provide structures in which the main and child documents can be
encapsulated and allowing them to be compiled individually.
The inclusion mechanism is different from the conventional |\include|.
\item
The package \href{http://ctan.org/pkg/combine}{\textsf{combine}}
is an elaborate solution to combine several documents into one.
\end{itemize}
%
See also the CTAN topic \href{http://ctan.org/topic/subdocs}{\textsf{subdocs}}
for further related packages.
The present package differs from the above solutions in that
a document structure constructed with the conventional |\include| mechanism
just needs two extra commands at the top of every file
such that all constituent files can be compiled individually.

%%%%%%%%%%%%%%%%%%%%%%%%%%%%%%%%%%%%%%%%%%%%%%%%%%%%%%%%%%%%%%%%%%%%%%%%%%%%%%%%
%\subsection{Feature Suggestions}
%
%The following is a list of features which may be useful for future
%versions of this package:
%%
%\begin{itemize}
%\item
%\ldots
%\end{itemize}

%%%%%%%%%%%%%%%%%%%%%%%%%%%%%%%%%%%%%%%%%%%%%%%%%%%%%%%%%%%%%%%%%%%%%%%%%%%%%%%%
\subsection{Revision History}

%%%%%%%%%%%%%%%%%%%%%%%%%%%%%%%%%%%%%%%%
\paragraph{v2.0:} 2018/12/30

\begin{itemize}
\item
immediate forward processing
\item
added |\childdocby| mechanism
\item
manual restructured
\end{itemize}

%%%%%%%%%%%%%%%%%%%%%%%%%%%%%%%%%%%%%%%%
\paragraph{v1.6:} 2018/01/17

\begin{itemize}
\item
application for development of include files
\item
corrections to manual
\end{itemize}

%%%%%%%%%%%%%%%%%%%%%%%%%%%%%%%%%%%%%%%%
\paragraph{v1.5:} 2017/05/21

\begin{itemize}
\item
more complete structuring introduced
\item
|\childdocof| introduced
\item
|\childdoc| renamed to |\childdocmain|
\item
|\childredirect| renamed to |\childdocforward| and |\childdocforwardprefix|
and functionality expanded
\end{itemize}

%%%%%%%%%%%%%%%%%%%%%%%%%%%%%%%%%%%%%%%%
\paragraph{v1.0:} 2017/04/27

\begin{itemize}
\item
manual and install package
\item
first version published on CTAN
\end{itemize}

%%%%%%%%%%%%%%%%%%%%%%%%%%%%%%%%%%%%%%%%
\paragraph{v0.6:} 2017/04/26

\begin{itemize}
\item
redirection mechanism added
\end{itemize}

%%%%%%%%%%%%%%%%%%%%%%%%%%%%%%%%%%%%%%%%
\paragraph{v0.5:} 2017/04/26

\begin{itemize}
\item
functionality in definition file
\end{itemize}


%%%%%%%%%%%%%%%%%%%%%%%%%%%%%%%%%%%%%%%%%%%%%%%%%%%%%%%%%%%%%%%%%%%%%%%%%%%%%%%%
%%%%%%%%%%%%%%%%%%%%%%%%%%%%%%%%%%%%%%%%%%%%%%%%%%%%%%%%%%%%%%%%%%%%%%%%%%%%%%%%
%%%%%%%%%%%%%%%%%%%%%%%%%%%%%%%%%%%%%%%%%%%%%%%%%%%%%%%%%%%%%%%%%%%%%%%%%%%%%%%%
\appendix

\settowidth\MacroIndent{\rmfamily\scriptsize 000\ }

 \DocInput{childdoc.dtx}

\end{document}
%</driver>
% \fi
%
% %%%%%%%%%%%%%%%%%%%%%%%%%%%%%%%%%%%%%%%%%%%%%%%%%%%%%%%%%%%%%%%%%%%%%%%%%%%%%%
% %%%%%%%%%%%%%%%%%%%%%%%%%%%%%%%%%%%%%%%%%%%%%%%%%%%%%%%%%%%%%%%%%%%%%%%%%%%%%%
% \section{Sample}
%\iffalse
%<*samplemain>
%\fi
%
% The following presents a sample document
% with two chapters, two parts, a title page,
% a compile flag as well as three forwarding files to set the flag.
% It consists of eight |.tex| files:
% \begin{center}
% \begin{tabular}{ll}
% |cdocsamp.tex|&main file\\
% |cdocsch1.tex|&include file for chapter 1\\
% |cdocsch2.tex|&include file for chapter 2\\
% |cdocspt3.tex|&include file for part 3\\
% |cdocspt4.tex|&include file for part 4\\
% |cdocsdrf.tex|&forwarding file for main file in draft mode\\
% |cdocsfi1.tex|&forwarding file for final version of chapter 1\\
% |cdocsfi2.tex|&forwarding file for final version of chapter 2\\
% \end{tabular}
% \end{center}
% Each of the eight files can be compiled directly by the \LaTeX{} compiler.
%
% %%%%%%%%%%%%%%%%%%%%%%%%%%%%%%%%%%%%%%
% \paragraph{Main File.}
%
% The main file is called |cdocsamp.tex|.
%
% Load the \textsf{childdoc} definitions and
% declare the filename for the main document:
%    \begin{macrocode}
\input{childdoc.def}
\childdocmain{}
%    \end{macrocode}

% Optional override for |\version| flag:
%    \begin{macrocode}
%%\ifchilddoc\else\providecommand{\version}{draft}\fi
%    \end{macrocode}

% Define the default values for the |\version| flag
% (|final| for the main file and |draft| for childs):
%    \begin{macrocode}
\ifchilddoc
\providecommand{\version}{draft}
\else
\providecommand{\version}{final}
\fi
%    \end{macrocode}

% Load the standard document class:
%    \begin{macrocode}
\documentclass[12pt]{article}
%    \end{macrocode}

% Start the document body:
%    \begin{macrocode}
\begin{document}
%    \end{macrocode}

% Declare a title page.
% Print title, part of document being processed and version flag:
%    \begin{macrocode}
\addtocounter{page}{-1}
\begin{center}
{\LARGE\bfseries{}childdoc example\par}
\vspace{1cm}
\ifchilddoc
\ifchilddocmanual part\else chapter\fi:
`\childdocname' of `\childdocjob'\par
\else
main document: `\childdocjob'\par
\fi
version: \version\par
\end{center}
\newpage
%    \end{macrocode}

% Manually include selected file,
% otherwise process as usual:
%    \begin{macrocode}
\ifchilddocmanual
\section*{part `\childdocname'}
\input{\childdocname}
\else
%    \end{macrocode}

% Include the two chapters:
%    \begin{macrocode}
\include{cdocsch1}
\include{cdocsch2}
%    \end{macrocode}

% Include the two parts unless only chapters should be displayed:
%    \begin{macrocode}
\ifchilddoc\else
\section{part three}
\input{cdocspt3}
\section{part four}
\input{cdocspt4}
\fi
%    \end{macrocode}

% Process as usual until here:
%    \begin{macrocode}
\fi
%    \end{macrocode}

% End of document body:
%    \begin{macrocode}
\end{document}
%    \end{macrocode}
%\iffalse
%</samplemain>
%\fi
%
% %%%%%%%%%%%%%%%%%%%%%%%%%%%%%%%%%%%%%%
% \paragraph{Chapter Include Files.}
%
% The include files are called |cdocsch1.tex| and |cdocsch2.tex|.
%
%\iffalse
%<*samplechap1|samplechap2>
%\fi

% Optional override for |\version| flag:
%    \begin{macrocode}
%%\providecommand{\version}{final}
%    \end{macrocode}

% Include the main document:
%    \begin{macrocode}
\input{childdoc.def}
\childdocof{cdocsamp}
%    \end{macrocode}

%\iffalse
%</samplechap1|samplechap2>
%\fi
%
%\iffalse
%<*samplechap1>
%\fi
% Some text for chapter 1:
%    \begin{macrocode}
\section{one}
some text in chapter one
%    \end{macrocode}

%\iffalse
%</samplechap1>
%\fi
% Some text for chapter 2:
%\iffalse
%<*samplechap2>
%\fi
%    \begin{macrocode}
\section{two}
more text in chapter two
%    \end{macrocode}

%\iffalse
%</samplechap2>
%\fi
%
% %%%%%%%%%%%%%%%%%%%%%%%%%%%%%%%%%%%%%%
% \paragraph{Part Include Files.}
%
% The include files are called |cdocspt3.tex| and |cdocspt4.tex|.
%
%\iffalse
%<*samplepart3|samplepart4>
%\fi

% Optional override for |\version| flag:
%    \begin{macrocode}
%%\providecommand{\version}{final}
%    \end{macrocode}

% Include the main document:
%    \begin{macrocode}
\input{childdoc.def}
\childdocby{cdocsamp}
%    \end{macrocode}

%\iffalse
%</samplepart3|samplepart4>
%\fi
%
%\iffalse
%<*samplepart3>
%\fi
% Some text for part 3:
%    \begin{macrocode}
some text in part three
%    \end{macrocode}

%\iffalse
%</samplepart3>
%\fi
% Some text for part 4:
%\iffalse
%<*samplepart4>
%\fi
%    \begin{macrocode}
more text in part four
%    \end{macrocode}

%\iffalse
%</samplepart4>
%\fi
%
% %%%%%%%%%%%%%%%%%%%%%%%%%%%%%%%%%%%%%%
% \paragraph{Forwarding for a Complete Draft.}
%
% The following forwarding file |cdocsdrf.tex|
% compiles the main document in draft mode:
%\iffalse
%<*sampledraft>
%\fi
%    \begin{macrocode}
\def\version{draft}
\input{childdoc.def}
\childdocforward{cdocsamp}
%    \end{macrocode}

%\iffalse
%</sampledraft>
%\fi
%
% %%%%%%%%%%%%%%%%%%%%%%%%%%%%%%%%%%%%%%
% \paragraph{Forwarding for Final Version of the Chapters.}
%
% The following forwarding files |cdocsfn1.tex| and |cdocsfn2.tex|
% (with identical content)
% compile the final versions of the child documents
% |cdocsch1.tex| and |cdocsch2.tex|, respectively:
%\iffalse
%<*samplefinal>
%\fi
%    \begin{macrocode}
\def\version{final}
\input{childdoc.def}
\childdocforwardprefix[cdocsamp]{cdocsfn}{cdocsch}
%    \end{macrocode}

%\iffalse
%</samplefinal>
%\fi
%
% %%%%%%%%%%%%%%%%%%%%%%%%%%%%%%%%%%%%%%
% \paragraph{Command Line Processing.}
%
% The following three command lines generate the output files
% |cdocscld|, |cdocscl1| and |cdocscl2|
% which should be identical to
% |cdocsdrf|, |cdocsch1| and |cdocsfn2|, respectively:
% \begin{center}
% \begin{tabular}{l}
% |latex -jobname cdocscld \|\\
% |  "\def\version{draft}\input{childdoc.def}\childdocforward{cdocsamp}"|\\
% |latex -jobname cdocscl1 \|\\
% |  "\input{childdoc.def}\childdocforward[cdocsamp]{cdocsch1}"|\\
% |latex -jobname cdocscl2 \|\\
% |  "\def\version{final}\input{childdoc.def}\childdocforward{cdocsch2}"|
% \end{tabular}
% \end{center}
% Note that the trailing backslash on each first line
% merely continues the input to the second line
% (for convenient cut ant paste).
% Furthermore, the command |latex| can be replaced by any
% of its alternative versions such as |pdflatex|.
%
% %%%%%%%%%%%%%%%%%%%%%%%%%%%%%%%%%%%%%%%%%%%%%%%%%%%%%%%%%%%%%%%%%%%%%%%%%%%%%%
% %%%%%%%%%%%%%%%%%%%%%%%%%%%%%%%%%%%%%%%%%%%%%%%%%%%%%%%%%%%%%%%%%%%%%%%%%%%%%%
% \section{Implementation}
%\iffalse
%<*package>
%\fi
%
% This section describes the definitions file |childdoc.def|.

% The definitions cannot be loaded using |\usepackage| or |\RequirePackage|
% which has a mechanism to prevent loading a style file more than once.
% When loading the definitions by means of |\input|
% multiple instances have to be prevented manually:
%\iffalse
%This code needs to be before the `\ProvidesFile' directive
%which is defined at the beginning of this file.
%Therefore it is also placed there and commented out here.
%</package>
%<*discard>
%\fi
%    \begin{macrocode}
\ifdefined\childdocmain\endinput\fi
%    \end{macrocode}
%\iffalse
%</discard>
%<*package>
%\fi
%
% \macro{\ifchilddoc}
% \macro{\ifchilddocmanual}
% The conditional |\ifchilddoc| tells whether a
% child (true) or main (false) document is being compiled.
% The conditional |\ifchilddocmanual| tells whether
% the |\includeonly| mechanism is used (false) or
% the selection of child files must be performed manually (true).
% The definitions initialise to false:
%    \begin{macrocode}
\newif\ifchilddoc
\newif\ifchilddocmanual
%    \end{macrocode}

% \macro{\childdocname}
% \macro{\childdocjob}
% The macro |\childdocname| stores the name of the main document
% to be compiled. The macro |\childdocjob| stores the name of
% the document on which the \LaTeX{} compiler was originally invoked.
% The content of |\jobname| cannot be compared
% to filenames specified in the source due to different catcodes.
% The following code rescans |\jobname|, stores the result
% in |\childdocname| and saves a copy in |\childdocjob|:
%    \begin{macrocode}
\edef\childdocname{\scantokens\expandafter{\jobname\noexpand}}
\let\childdocjob\childdocname
%    \end{macrocode}

% \macro{\childdocdisable}
% The macro |\childdocdisable| prevents the main file
% from being processed more than once.
% At this stage, the main document command |\childdocmain|
% is assumed to be called once again where it should do nothing.
% Any subsequent call to it should prevent
% a secondary processing of the main document
% It overwrites the forwarding commands
% |\childdocof| and |\childdocforward|
% with empty macros to prevent further inclusions of the main document:
%    \begin{macrocode}
\newcommand{\childdocdisable}
{
  \renewcommand{\childdocmain}[1]{\renewcommand{\childdocmain}[1]{\endinput}}
  \renewcommand{\childdocof}[1]{}
  \renewcommand{\childdocby}[2][]{}
  \renewcommand{\childdocforward}[2][]{}
  \renewcommand{\childdocdisable}{}
}
%    \end{macrocode}

% \macro{\childdocmain}
% The macro |\childdocmain| is to be called at the top of the main file
% with nothing or the main filename (without extension) as argument.
% First, it breaks loops.
% If the argument is not empty and does not match |\childdocname|
% (which is set by the first inclusion of |childdoc.def|),
% |\ifchilddoc| is set to true, |\includeonly| is applied to the child file
% and |\jobname| is set to the main file
% (for proper handling of |.aux| files):
%    \begin{macrocode}
\newcommand{\childdocmain}[1]
{
  \childdocdisable\childdocmain{}
  \if?#1?\else
    \begingroup
      \def\childdoctmp{#1}
      \ifx\childdoctmp\childdocname
        \def\childdoctmp{}
      \else
        \def\childdoctmp
        {
          \childdoctrue
          \includeonly{\childdocname}
          \def\childdocjob{#1}
          \def\jobname{#1}
        }
      \fi
      \expandafter
    \endgroup
    \childdoctmp
  \fi
}
%    \end{macrocode}

% \macro{\childdocof}
% The command |\childdocof| redirects
% compilation to the main file |#1|.
%    \begin{macrocode}
\newcommand{\childdocof}[1]
{
  \childdocdisable
  \childdoctrue
  \includeonly{\childdocname}
  \def\jobname{#1}
  \def\childdocjob{#1}
  \input{#1}
}
%    \end{macrocode}

% \macro{\childdocby}
% The command |\childdocby| ....
%    \begin{macrocode}
\newcommand{\childdocby}[2][]
{
  \childdocdisable
  \childdoctrue
  \childdocmanualtrue
  \if?#1?\else
    \def\jobname{#2}
  \fi
  \def\childdocjob{#2}
  \input{#2}
  \endinput
}
%    \end{macrocode}

% \macro{\childdocforward}
% The command |\childdocforward| redirects
% compilation to the main file or
% (if the optional argument is given) a child file.
% Parameters are set as if the main file
% or a child file starting with |\childdocof| was compiled.
% Then compilation is handed over to the main file:
%    \begin{macrocode}
\newcommand{\childdocforward}[2][]
{
  \begingroup
    \if?#1?
      \def\childdoctmp
      {
        \def\childdocname{#2}
        \def\childdocjob{#2}
        \def\jobname{#2}
        \input{#2}
        \endinput
      }
    \else
      \def\childdoctmp
      {
        \childdocdisable
        \def\childdocname{#2}
        \childdoctrue
        \includeonly{#2}
        \def\childdocjob{#1}
        \def\jobname{#1}
        \input{#1}
        \endinput
      }
    \fi
    \expandafter
  \endgroup
  \childdoctmp
}
%    \end{macrocode}

% \macro{\childdocforwardprefix}
% The command |\childdocforwardprefix| redirects
% compilation to the main or a child file by means of a pattern.
% The prefix |#1| in the current filename is replaced by |#2|
% and the suffix of the current filename is kept
% (it is assumed that the filename does not contain the substring `|~~~|'
% which is used as a delimiter).
% Compilation is handed over to the new file by |\childdocforward|:
%    \begin{macrocode}
\newcommand{\childdocforwardprefix}[3][]
{
  \begingroup
    \def\childdocextract #2##1~~~{\def\childdoctmp{\childdocforward[#1]{#3##1}}}
    \expandafter\childdocextract\childdocname~~~
    \expandafter
  \endgroup
  \childdoctmp
}
%    \end{macrocode}

% \macro{\childdoc}
% The deprecated macro |\childdoc| is a legacy version of |\childdocmain|:
%    \begin{macrocode}
\newcommand{\childdoc}{\childdocmain}
%    \end{macrocode}

% \macro{\childdocredirect}
% The deprecated macro |\childdocredirect| is a legacy version
% of |\childdocforward| and |\childdocforwardprefix|:
%    \begin{macrocode}
\newcommand{\childdocredirect}[2][]
{
  \begingroup
    \if?#1?
      \def\childdoctmp{\childdocforward{#2}}
    \else
      \def\childdoctmp{\childdocforwardprefix{#1}{#2}}
    \fi
    \expandafter
  \endgroup
  \childdoctmp
}
%    \end{macrocode}

%\iffalse
%</package>
%\fi
%
\endinput
|\\
|\childdocmain{}|\\
\end{tabular}
\end{center}
at the very top of the main \LaTeX{} file,
in particular \emph{before} the |\documentclass| statement!
The argument of |\childdocmain| should be left empty
(but it must be present).

%%%%%%%%%%%%%%%%%%%%%%%%%%%%%%%%%%%%%%%%
\DescribeMacro{\childdocof}
Furthermore, add the commands
\begin{center}
\begin{tabular}{l}
|% \iffalse
%
% childdoc.dtx Copyright (C) 2017-2018 Niklas Beisert
%
% This work may be distributed and/or modified under the
% conditions of the LaTeX Project Public License, either version 1.3
% of this license or (at your option) any later version.
% The latest version of this license is in
%   http://www.latex-project.org/lppl.txt
% and version 1.3 or later is part of all distributions of LaTeX
% version 2005/12/01 or later.
%
% This work has the LPPL maintenance status `maintained'.
%
% The Current Maintainer of this work is Niklas Beisert.
%
% This work consists of the files childdoc.dtx and childdoc.ins
% and the derived files childdoc.def and cdocsamp.tex with
% cdocsch1.tex, cdocsch2.tex, cdocsdrf.tex, cdocsfn1.tex, cdocsfn2.tex.
%
%<package>\ifdefined\childdocmain\endinput\fi
%<package>\ProvidesFile{childdoc.def}[2018/12/30 v2.0 child document driver]
%<samplemain>\ProvidesFile{cdocsamp.tex}[2018/12/30 v2.0 sample for childdoc]
%<*driver>
%\ProvidesFile{childdoc.drv}[2018/12/30 v2.0 childdoc reference manual file]
\PassOptionsToClass{10pt,a4paper}{article}
\documentclass{ltxdoc}

\usepackage[margin=35mm]{geometry}
\usepackage{hyperref}
\usepackage{hyperxmp}
\usepackage[usenames]{color}

\hypersetup{colorlinks=true}
\hypersetup{pdfstartview=FitH}
\hypersetup{pdfpagemode=UseNone}
\hypersetup{pdfsource={}}
\hypersetup{pdflang={en-UK}}
\hypersetup{pdfcopyright={Copyright 2017-2018 Niklas Beisert.
  This work may be distributed and/or modified under the
  conditions of the LaTeX Project Public License, either version 1.3
  of this license or (at your option) any later version.}}
\hypersetup{pdflicenseurl={http://www.latex-project.org/lppl.txt}}
\hypersetup{pdfcontactaddress={ETH Zurich, ITP, HIT K,
  Wolfgang-Pauli-Strasse 27}}
\hypersetup{pdfcontactpostcode={8093}}
\hypersetup{pdfcontactcity={Zurich}}
\hypersetup{pdfcontactcountry={Switzerland}}
\hypersetup{pdfcontactemail={nbeisert@itp.phys.ethz.ch}}
\hypersetup{pdfcontacturl={http://people.phys.ethz.ch/\xmptilde nbeisert/}}

\newcommand{\secref}[1]{\hyperref[#1]{section \ref*{#1}}}

\parskip1ex
\parindent0pt
\let\olditemize\itemize
\def\itemize{\olditemize\parskip0pt}

\begin{document}

\title{The \textsf{childdoc} Package}
\hypersetup{pdftitle={The childdoc Package}}
\author{Niklas Beisert\\[2ex]
  Institut f\"ur Theoretische Physik\\
  Eidgen\"ossische Technische Hochschule Z\"urich\\
  Wolfgang-Pauli-Strasse 27, 8093 Z\"urich, Switzerland\\[1ex]
  \href{mailto:nbeisert@itp.phys.ethz.ch}
  {\texttt{nbeisert@itp.phys.ethz.ch}}}
\hypersetup{pdfauthor={Niklas Beisert}}
\hypersetup{pdfsubject={Manual for the LaTeX2e Package childdoc}}
\date{30 December 2018, \textsf{v2.0}}
\maketitle

\begin{abstract}\noindent
\textsf{childdoc} is a \LaTeXe{} package
that enables the direct compilation
of document sections included by |\include|
to individual files.
\end{abstract}

\begingroup
\parskip0ex
\tableofcontents
\endgroup

%%%%%%%%%%%%%%%%%%%%%%%%%%%%%%%%%%%%%%%%%%%%%%%%%%%%%%%%%%%%%%%%%%%%%%%%%%%%%%%%
%%%%%%%%%%%%%%%%%%%%%%%%%%%%%%%%%%%%%%%%%%%%%%%%%%%%%%%%%%%%%%%%%%%%%%%%%%%%%%%%
\section{Introduction}

\LaTeX{} provides a mechanism to structure a large document (such as a book)
into a main file and several child files (containing the chapters)
using the |\include| command.
This mechanism is beneficial for documents
which span hundreds of pages in order to
make the source file(s) more manageable.
Moreover, compilation can be restricted to
selected child files by means of the |\includeonly| command.
The latter feature can be used to reduce the compilation time while editing
(this was significantly more useful in the earlier days of \LaTeX{})
or to generate a smaller document which is easier to navigate.
Another application of |\includeonly| is to generate
documents consisting of selected parts of the complete document.

However, there are a few drawbacks of the plain |\include| mechanism:
\begin{itemize}
\item
The child files cannot be compiled on their own,
they can only be compiled via the main file.
A naive editing environment
(such as a text editor with an option
to have the current file processed by \LaTeX)
may require one to switch to the main file before compiling;
attempting to compile the child file produces errors.
\item
The main file must be modified (each time)
to adjust the |\includeonly| command
to the present needs. This easily leaves the main file in a messy state.
\item
The generated document will always carry the filename
of the main document. This is inconvenient if
several child files are to be compiled and
to be kept for distribution.
\end{itemize}

The present package provides a simple interface
to make child files individually compilable by \LaTeX{}.
Compiling a child file then has the same effect as compiling
the main file with an |\includeonly| command
to select the appropriate child.
Moreover the generated document will carry the name of the child
rather than the main file.
This resolves all three above issues.

This feature is meant to make the editing of books,
thesis documents and lecture notes somewhat more convenient.
However, the package can also be used efficiently for
composing a series of documents (such as exercise sheets)
which are typically distributed individually.
It then assists the author in generating the individual documents
(potentially in different versions)
as well as a document containing the collected series.
Another application is in developing style files
or other kinds of included material
where compilation of the style file could redirect
to a sample or test file.

%%%%%%%%%%%%%%%%%%%%%%%%%%%%%%%%%%%%%%%%%%%%%%%%%%%%%%%%%%%%%%%%%%%%%%%%%%%%%%%%
%%%%%%%%%%%%%%%%%%%%%%%%%%%%%%%%%%%%%%%%%%%%%%%%%%%%%%%%%%%%%%%%%%%%%%%%%%%%%%%%
\section{Usage}

First of all, the package \textsf{childdoc} is \emph{not} a standard
\LaTeXe{} |.sty| style file! Therefore it needs to be invoked in
a non-standard way.

%%%%%%%%%%%%%%%%%%%%%%%%%%%%%%%%%%%%%%%%%%%%%%%%%%%%%%%%%%%%%%%%%%%%%%%%%%%%%%%%
\subsection{Included Files}
\label{sec:include}

%%%%%%%%%%%%%%%%%%%%%%%%%%%%%%%%%%%%%%%%
\DescribeMacro{\childdocmain}
To use the package, add the commands
\begin{center}
\begin{tabular}{l}
|\input{childdoc.def}|\\
|\childdocmain{}|\\
\end{tabular}
\end{center}
at the very top of the main \LaTeX{} file,
in particular \emph{before} the |\documentclass| statement!
The argument of |\childdocmain| should be left empty
(but it must be present).

%%%%%%%%%%%%%%%%%%%%%%%%%%%%%%%%%%%%%%%%
\DescribeMacro{\childdocof}
Furthermore, add the commands
\begin{center}
\begin{tabular}{l}
|\input{childdoc.def}|\\
|\childdocof{|\textit{main}|}|\\
\end{tabular}
\end{center}
at the top of every child file \textit{child}
which is included by |\include{|\textit{child}|}|
from within the main file
(or at least for those files to be compiled individually).
The argument \textit{main} must be the filename of the main file.

There are a couple of
considerations in setting up the main and child documents:

%%%%%%%%%%%%%%%%%%%%%%%%%%%%%%%%%%%%%%%%
\paragraph{Restrictions.}

Please note the following restrictions:
\begin{itemize}
\item
|\childdocmain| must be called with one argument \textit{main}
to ensure compatibility with earlier version of the package.
It must either be empty (|\childdocmain{}|)
or precisely match the filename of the main file in which it is specified.
See \secref{sec:detection} for further information.
\item
The filename \textit{main} must be specified without the |.tex| extension.
\item
The filename \textit{main} is case sensitive
(even in case-insensitive file systems)
due to internal string comparison.
\item
The argument \textit{main} should be fully expanded, it cannot be a macro.
\item
Subdirectories and special characters should be avoided in filenames.
\item
The command |\childdocmain{|\textit{main}|}| must be followed by a whitespace.
It should not be followed immediately by another command
or by a comment mark `|%|'.
This is because the \TeX{} parser reads the token immediately following
the argument of |\childdocmain| and puts it
at the beginning of every child section;
however, a white\-space is ignored.
\end{itemize}

%%%%%%%%%%%%%%%%%%%%%%%%%%%%%%%%%%%%%%%%
\paragraph{Content of Main File.}

It is advisable to place all content in the child files included by |\include|.
Any output contained in the main file will appear in all child documents
unless suppressed manually;
it cannot be suppressed automatically by the |\includeonly| directive
and thus should normally be avoided.
A method to include some content in the main file
by means of conditional processing is described in \secref{sec:conditional}.

%%%%%%%%%%%%%%%%%%%%%%%%%%%%%%%%%%%%%%%%
\paragraph{Page Numbering.}

When only a part of the document is compiled,
the appropriate numbering of pages
(as well as other status parameters)
is determined from the |.aux| files.
The latter contain information from previous passes.
However this information needs to propagate through
all intermediate child documents.
Therefore the page numbering in child documents may well
be inconsistent until the complete document is compiled at least once.

A useful (if unconventional) way to always ensure a consistent
page numbering is to restart the numbering in each child document
and denote the pages by `\textit{child}|.|\textit{page}'
where \textit{child} represents the chapter/section number of the child file.
This can be achieved by the command
|\numberwithin{page}{|\textit{child}|}|
of the \textsf{amsmath} package
where \textit{child} can be |chapter| or |section|
depending on the chosen structuring.
Alternatively, one can modify the macro |\thepage| appropriately
and reset the counter |page| at the start of each child file.

%%%%%%%%%%%%%%%%%%%%%%%%%%%%%%%%%%%%%%%%%%%%%%%%%%%%%%%%%%%%%%%%%%%%%%%%%%%%%%%%
\subsection{Conditional Processing}
\label{sec:conditional}

The package provides a mechanism to compile different versions
of a document. To customise the versions further some conditional processing
can come in handy to distinguish which version is being compiled.
The package provides two macros to describe the compilation context:

%%%%%%%%%%%%%%%%%%%%%%%%%%%%%%%%%%%%%%%%
\DescribeMacro{\ifchilddoc}
The conditional |\ifchilddoc| distinguishes between the compilation of
child documents and the main document:
%
\begin{center}
|\ifchilddoc |\textit{child-code}| |[|\||else |\textit{main-code}]| \||fi|
\end{center}

%%%%%%%%%%%%%%%%%%%%%%%%%%%%%%%%%%%%%%%%
\DescribeMacro{\childdocname}
\DescribeMacro{\childdocjob}
The macro |\childdocname| contains the filename (without extension)
of the main or child file being processed.
Note that |\childdocjob| will always contain the name of the main file.

%%%%%%%%%%%%%%%%%%%%%%%%%%%%%%%%%%%%%%%%
\paragraph{Title Page.}

Conditional processing can be used to include a title or banner page
in the main document when proper precautions are taken.
Importantly, the code in the main file should ensure that the page counter
(as well as other status parameters which are stored in the |.aux| files)
takes the same value after the conditional processing.
Otherwise the page numbers may take divergent values
depending on which part is compiled.

For example, a title page could be declared by:
%
\begin{center}
\begin{tabular}{l}
|\ifchilddoc\||else|\\
|\addtocounter{page}{-1}|\\
\textit{code for title page}\\
|\newpage|\\
|\||fi|
\end{tabular}
\end{center}
%
A banner page for the child documents can be generated by:
%
\begin{center}
\begin{tabular}{l}
|\ifchilddoc|\\
|\addtocounter{page}{-1}|\\
\textit{code for banner page}\\
|\newpage|\\
|\||fi|
\end{tabular}
\end{center}
%
Here one could write a message such as:
\begin{center}
|This is the part \childdocname{} of \childdocjob{}.|
\end{center}

%%%%%%%%%%%%%%%%%%%%%%%%%%%%%%%%%%%%%%%%%%%%%%%%%%%%%%%%%%%%%%%%%%%%%%%%%%%%%%%%
\subsection{Flags}
\label{sec:flags}

The package makes it easy to generate different versions
of the main or child documents.
To this end compilation flags can be defined
and assigned different default values.
They will be particularly useful in conjunction
with the forwarding mechanism described in \secref{sec:forward}.

For example, it may be useful to have a flag |\version|
which can be set to |draft| or |final|.
The document source will contain some conditional code
depending on the value of |\version|.
Suppose further, the flag should default to |final| for the main file
and to |draft| for child files
which is a natural assignment for editing the document.
This is achieved by placing the following code
in the preamble of the main document
(below the |\childdocmain| directive):
%
\begin{center}
\begin{tabular}{l}
|\ifchilddoc|\\
|\providecommand{\version}{draft}|\\
|\||else|\\
|\providecommand{\version}{final}|\\
|\||fi|
\end{tabular}
\end{center}
%
The definition by |\providecommand| makes sure
that previous definitions are not overwritten.
Further statements |\providecommand{\version}{...}|
can thus be added before the above code to override it.

For the main file, one might add a line
(between |\childdocmain| and the above block)
%
\begin{center}
|%\ifchilddoc\||else\providecommand{\version}{draft}\||fi|
\end{center}
%
which can be uncommented to produce a draft version.
Likewise one can add a line to the very top of a child file
(above the |\childdocof{|\textit{main}|}| directive)
%
\begin{center}
|%\providecommand{\version}{final}|
\end{center}
%
which can be uncommented to produce the final version of this child document.

%%%%%%%%%%%%%%%%%%%%%%%%%%%%%%%%%%%%%%%%%%%%%%%%%%%%%%%%%%%%%%%%%%%%%%%%%%%%%%%%
\subsection{Forwarding}
\label{sec:forward}

Different versions of the main or child documents
using compilation flags as described in \secref{sec:flags}
can be (permanently) stored in different files
for convenient compilation, viewing and distribution.
To this end, the package defines a command
to pass on compilation to a different file:

%%%%%%%%%%%%%%%%%%%%%%%%%%%%%%%%%%%%%%%%
\DescribeMacro{\childdocforward}
The command |\childdocforward| redirects processing to
another source file:
%
\begin{center}
\begin{tabular}{l}
|\input{childdoc.def}|\\
|\childdocforward[|\textit{main}|]{|\textit{dest}|}|\\
\end{tabular}
\end{center}
%
The argument \textit{dest} is the destination file
(without extension).
It should be the main file or one of the child files.
Note that further \textsf{childdoc} directives
such as |\childdocof| and |\childdocforward|
in the indicated file will be processed in this form.
The optional argument \textit{main}
passes on directly to the main file \textit{main}
while pretending to compile the child \textit{dest}.
This form behaves as if \textit{dest}
issues |\childdocof{|\textit{main}|}| right away,
and no further \textsf{childdoc} directives will be processed.

%%%%%%%%%%%%%%%%%%%%%%%%%%%%%%%%%%%%%%%%
\DescribeMacro{\...prefix}
In the alternative form |\childdocforwardprefix|,
%
\begin{center}
\begin{tabular}{l}
|\input{childdoc.def}|\\
|\childdocforwardprefix[|\textit{main}|]{|\textit{prefix}|}{|\textit{dest}|}|
\end{tabular}
\end{center}
%
the destination file is determined by a pattern
depending on the current file:
To make this work, the current file must be called
`{\textit{prefix}\hspace{0.2em}\textit{suffix}}'
with \textit{prefix} matching precisely the argument.
Processing is then passed on to the file
`{\textit{dest}\hspace{0.2em}\textit{suffix}}'.
Surely, the same effect is achieved by
directly specifying the
argument `{\textit{dest}\hspace{0.2em}\textit{suffix}}'
in the first form.
However, that requires to set up a different file
for each child. With the alternative form of the command
all these files can have exactly the same content
which simplifies setting them up and maintaining them.

For example, the following file |draft.tex|
with a compilation flag |\version| as described in \secref{sec:flags}
compiles the main document as a draft:
%
\begin{center}
\begin{tabular}{l}
|\def\version{draft}|\\
|\input{childdoc.def}|\\
|\childdocforward{|\textit{main}|}|
\end{tabular}
\end{center}
%
Likewise, the following files |final|\textit{nn}|.tex|
compile the final version of the child document
|child|\textit{nn}|.tex|:
%
\begin{center}
\begin{tabular}{l}
|\def\version{final}|\\
|\input{childdoc.def}|\\
|\childdocforwardprefix{final}{child}|
\end{tabular}
\end{center}
%

Note that when several versions of a main file and/or of each child file
are to be generated, it may be convenient to set up a |Makefile| or
shell script to automatise the process.

%%%%%%%%%%%%%%%%%%%%%%%%%%%%%%%%%%%%%%%%%%%%%%%%%%%%%%%%%%%%%%%%%%%%%%%%%%%%%%%%
\subsection{Command Line Processing}
\label{sec:commandline}

The effect of redirection files can also be achieved by invoking
the \LaTeX{} compiler with a more elaborate command line.
Most conveniently this should be done as part
of a shell script or a |Makefile|.

When using \textsf{childdoc} in the main file, the following
command lines effectively perform a redirection
(note that depending on the shell being used,
backslashes may have to be doubled: `|\|' $\to$ `|\\|'):
%
\begin{center}
|... -jobname "|\textit{target}|" |\\|"|[\textit{flags}]%
|\input{childdoc.def}\childdocforward[|\textit{main}|]{|\textit{dest}|}"|
\end{center}
%
Here \textit{target} is the name of the output file,
\textit{main} is the name of the main file
and \textit{dest} is the name of the main or child file to be processed
(all filenames without extensions).
The optional argument \textit{main} can be omitted
if \textit{main} matches \textit{dest}.
Optionally, compilation \textit{flags} can be defined via |\def| commands.
This command line makes the \TeX{} engine believe
it is compiling the file \textit{target}
whose content is specified as the latter parameter.
The provided code then forwards the processing to
\textit{main} or \textit{dest} as described in \secref{sec:forward}.

%%%%%%%%%%%%%%%%%%%%%%%%%%%%%%%%%%%%%%%%%%%%%%%%%%%%%%%%%%%%%%%%%%%%%%%%%%%%%%%%
\subsection{Include by Input}
\label{sec:input}

Including child documents by |\include| has some restrictions by design.
Most notably, the content of a child document always occupies
its own set of pages; pages cannot be shared between child documents.
Usually, this behaviour makes perfect sense
because each child document contain an essential part of the document.
However, in some situations it may be desirable to compose
a document from a collection of parts
without having mandatory page breaks between then.
For this case, the package
provides a mechanism to include parts
by |\input| which can also be processed individually.
However, by construction this mechanism
requires manual handling of the content to be output.

%%%%%%%%%%%%%%%%%%%%%%%%%%%%%%%%%%%%%%%%
\DescribeMacro{\ifchilddocmanual}
The main file should be prepared as usual, see \secref{sec:include}.
However, the document body must make a distinction
between processing of an individual part and of the main document, e.g.:
%
\begin{center}
\begin{tabular}{l}
|\ifchilddocmanual|\\
|\input{\childdocname}|\\
|\||else|\\
\textit{document body with }|\input{|\textit{part}|}|\\
|\||fi|
\end{tabular}
\end{center}
%
The conditional |\ifchilddocmanual| is true whenever
a part to be included by |\input| is being compiled,
and the name of the part is stored in |\childdocname|.

%%%%%%%%%%%%%%%%%%%%%%%%%%%%%%%%%%%%%%%%
\DescribeMacro{\childdocby}
Each part to be included by |\input| should start with:
%
\begin{center}
\begin{tabular}{l}
|\input{childdoc.def}|\\
|\childdocby{|\textit{main}|}|\\
\end{tabular}
\end{center}
%
The directive |\childdocby| is similar to |\childdocof|
described in \secref{sec:include},
but the subsequent selection of content must be done manually.
To that end, both |\ifchilddoc| and |\ifchilddocmanual|
will be true upon processing of a part,
and the name of the part is stored in |\childdocname|.
Note that |\jobname| will be set to the filename of the current part
so that each part receives an individual |.aux| file
that does not interfere with the |.aux| file(s) of the main document.
This behaviour can be altered by the alternative form
|\childdocby[*]{|\textit{main}|}| (with a non-empty optional argument)
which uses the |.aux| file of the main document
by setting |\jobname| to \textit{main}.

%%%%%%%%%%%%%%%%%%%%%%%%%%%%%%%%%%%%%%%%%%%%%%%%%%%%%%%%%%%%%%%%%%%%%%%%%%%%%%%%
\subsection{Driver Development}
\label{sec:driver}

The \textsf{childdoc} mechanism can also be use for the development
of definition files such as \LaTeX{} styles or classes.
This case differs from the above setup with multiple parts
included by |\include| in that no |\includeonly| should be invoked.
This can be achieved by starting the include file
(before |\ProvidesPackage|) with:
%
\begin{center}
\begin{tabular}{l}
|\input{childdoc.def}|\\
|\childdocforward{|\textit{main}|}|\\
\end{tabular}
\end{center}
%
or alternatively with:
%
\begin{center}
\begin{tabular}{l}
|\input{childdoc.def}|\\
|\childdocby{|\textit{main}|}|\\
\end{tabular}
\end{center}
%
Both forms have slightly different effects as described above.
The main file is prepared as usual, see \secref{sec:include}.

%%%%%%%%%%%%%%%%%%%%%%%%%%%%%%%%%%%%%%%%%%%%%%%%%%%%%%%%%%%%%%%%%%%%%%%%%%%%%%%%
\subsection{Legacy Detection}
\label{sec:detection}

The directive |\childdocmain| in the main file can detect
whether the complete document or merely a child is to be compiled
even without using the directive |\childdocof|.
This method is deprecated because it is less robust
and there is no compelling reason to use it;
it is merely provided for backward compatibility
and it may be removed in future versions.

If the detection mechanism is to be used,
it is mandatory to correctly specify
the filename of the main file as the argument of |\childdocmain|:
%
\begin{center}
\begin{tabular}{l}
|\input{childdoc.def}|\\
|\childdocmain{|\textit{main}|}|\\
\end{tabular}
\end{center}
%
If |\jobname| does not match the argument \textit{main} of |\childdocmain|,
it is assumed that |\jobname| points to the child file to be compiled.
When using |\childdocmain| with the main file specified as argument,
it suffices to start a child file
with just |\input{|\textit{main}|}|
without loading of the package and using |\childdocof|.
If instead all processing is done
with the appropriate \textsf{childdoc} directives,
the argument of \textit{main} of |\childdocmain| can be empty.

An alternative version of the command line processing described
in \secref{sec:commandline} using the detection mechanism reads:
%
\begin{center}
|... -jobname "|\textit{target}|" "|[\textit{flags}]%
[|\def\jobname{|\textit{dest}|}|]|\input{|\textit{main}|}"|
\end{center}

%%%%%%%%%%%%%%%%%%%%%%%%%%%%%%%%%%%%%%%%%%%%%%%%%%%%%%%%%%%%%%%%%%%%%%%%%%%%%%%%
\subsection{Manual Code}
\label{sec:manual}

In case one cannot be certain whether the definitions file |childdoc.def|
is installed on the target \TeX{} distribution
and one prefers not to ship it,
it is conceivable to paste a few relevant commands into the sources.

To that end, drop all statements |\input{childdoc.def}|
and perform the replacements as outlined below.
Instead of |\childdocmain{|\textit{main}|}| add the following code
to the top of the main file:
%
\begin{center}
\begin{tabular}{l}
|\||ifdefined\childdocname\endinput\||fi\newif\ifchilddoc|\\
|\edef\childdocname{\scantokens\expandafter{\jobname\noexpand}}|\\
|\def\childdocmain{|\textit{main}|}\||ifx\childdocmain\childdocname\||else|\\
|\childdoctrue\includeonly{\childdocname}\let\jobname\childdocmain\||fi|\\
\end{tabular}
\end{center}
%
Instead of |\childdocof{|\textit{main}|}| just include the main file
at the top of each child file:
%
\begin{center}
|\input{|\textit{main}|}|
\end{center}
%
A simple redirection |\childdocforward{|\textit{dest}|}| is achieved by:
%
\begin{center}
|\def\jobname{|\textit{dest}|}\input{\jobname}|
\end{center}
%
The redirection with prefix
|\childdocforwardprefix[|\textit{prefix}|]{|\textit{dest}|}|
is accomplished by:
%
\begin{center}
\begin{tabular}{l}
|{\edef\jobname{\scantokens\expandafter{\jobname\noexpand}}|\\
|\def\redirectjob |\textit{prefix}|#1~~~{\gdef\jobname{|\textit{dest}|#1}}|\\
|\expandafter\redirectjob\jobname~~~}\input{\jobname}|
\end{tabular}
\end{center}

In an alternative approach,
child documents can be compiled by a specific command line
without additional code or specific definitions:
%
\begin{center}
|... -jobname "|\textit{target}|" "|[\textit{flags}]%
|\includeonly{|\textit{dest}|}\input{|\textit{main}|}"|
\end{center}
%

%%%%%%%%%%%%%%%%%%%%%%%%%%%%%%%%%%%%%%%%%%%%%%%%%%%%%%%%%%%%%%%%%%%%%%%%%%%%%%%%
%%%%%%%%%%%%%%%%%%%%%%%%%%%%%%%%%%%%%%%%%%%%%%%%%%%%%%%%%%%%%%%%%%%%%%%%%%%%%%%%
\section{Information}

%%%%%%%%%%%%%%%%%%%%%%%%%%%%%%%%%%%%%%%%%%%%%%%%%%%%%%%%%%%%%%%%%%%%%%%%%%%%%%%%
\subsection{Copyright}

Copyright \copyright{} 2017--2018 Niklas Beisert

This work may be distributed and/or modified under the
conditions of the \LaTeX{} Project Public License, either version 1.3
of this license or (at your option) any later version.
The latest version of this license is in
  \url{http://www.latex-project.org/lppl.txt}
and version 1.3 or later is part of all distributions of \LaTeX{}
version 2005/12/01 or later.

This work has the LPPL maintenance status `maintained'.

The Current Maintainer of this work is Niklas Beisert.

This work consists of the files |README.txt|, |childdoc.ins| and |childdoc.dtx|
as well as the derived files |childdoc.def|, |cdocsamp.tex|
with |cdocsch1.tex|, |cdocsch2.tex|, |cdocspt3.tex|, |cdocspt4.tex|,
|cdocsdrf.tex|, |cdocsfn1.tex|, |cdocsfn2.tex|
as well as |childdoc.pdf|.

%%%%%%%%%%%%%%%%%%%%%%%%%%%%%%%%%%%%%%%%%%%%%%%%%%%%%%%%%%%%%%%%%%%%%%%%%%%%%%%%
\subsection{Files and Installation}

The package consists of the files:
%
\begin{center}
\begin{tabular}{ll}
    |README.txt|   & readme file \\
    |childdoc.ins| & installation file \\
    |childdoc.dtx| & source file \\
    |childdoc.def| & definition file \\
    |cdocsamp.tex| & sample main file \\
    |cdocsch1.tex| & sample include file \\
    |cdocsch2.tex| & sample include file \\
    |cdocspt3.tex| & sample part file \\
    |cdocspt4.tex| & sample part file \\
    |cdocsdrf.tex| & sample redirection file \\
    |cdocsfn1.tex| & sample redirection file \\
    |cdocsfn2.tex| & sample redirection file \\
    |childdoc.pdf| & manual
\end{tabular}
\end{center}
%
The distribution consists of the files
|README.txt|, |childdoc.ins| and |childdoc.dtx|.
%
\begin{itemize}
\item
Run (pdf)\LaTeX{} on |childdoc.dtx|
to compile the manual |childdoc.pdf| (this file).
\item
Run \LaTeX{} on |childdoc.ins| to create the definitions file |childdoc.def|
and the sample |cdocsamp.tex| with include files
|cdocsch1.tex|, |cdocsch2.tex|, |cdocspt3.tex|, |cdocspt4.tex|,
|cdocsdrf.tex|, |cdocsfn1.tex|, |cdocsfn2.tex|.
Then copy the file |childdoc.def| to an appropriate directory of your \LaTeX{}
distribution, e.g.\ \textit{texmf-root}|/tex/latex/childdoc|.
\end{itemize}

%%%%%%%%%%%%%%%%%%%%%%%%%%%%%%%%%%%%%%%%%%%%%%%%%%%%%%%%%%%%%%%%%%%%%%%%%%%%%%%%
\subsection{Related CTAN Packages}

There are several other packages which offer a similar functionality:
%
\begin{itemize}
\item
The packages
\href{http://ctan.org/pkg/docmute}{\textsf{docmute}},
\href{http://ctan.org/pkg/includex}{\textsf{includex}} and
\href{http://ctan.org/pkg/standalone}{\textsf{standalone}}
provide commands to include only the document body of
a child file thus allowing both files to be compiled individually.
\item
The packages \href{http://ctan.org/pkg/subdocs}{\textsf{subdocs}}
and \href{http://ctan.org/pkg/subfiles}{\textsf{subfiles}}
provide structures in which the main and child documents can be
encapsulated and allowing them to be compiled individually.
The inclusion mechanism is different from the conventional |\include|.
\item
The package \href{http://ctan.org/pkg/combine}{\textsf{combine}}
is an elaborate solution to combine several documents into one.
\end{itemize}
%
See also the CTAN topic \href{http://ctan.org/topic/subdocs}{\textsf{subdocs}}
for further related packages.
The present package differs from the above solutions in that
a document structure constructed with the conventional |\include| mechanism
just needs two extra commands at the top of every file
such that all constituent files can be compiled individually.

%%%%%%%%%%%%%%%%%%%%%%%%%%%%%%%%%%%%%%%%%%%%%%%%%%%%%%%%%%%%%%%%%%%%%%%%%%%%%%%%
%\subsection{Feature Suggestions}
%
%The following is a list of features which may be useful for future
%versions of this package:
%%
%\begin{itemize}
%\item
%\ldots
%\end{itemize}

%%%%%%%%%%%%%%%%%%%%%%%%%%%%%%%%%%%%%%%%%%%%%%%%%%%%%%%%%%%%%%%%%%%%%%%%%%%%%%%%
\subsection{Revision History}

%%%%%%%%%%%%%%%%%%%%%%%%%%%%%%%%%%%%%%%%
\paragraph{v2.0:} 2018/12/30

\begin{itemize}
\item
immediate forward processing
\item
added |\childdocby| mechanism
\item
manual restructured
\end{itemize}

%%%%%%%%%%%%%%%%%%%%%%%%%%%%%%%%%%%%%%%%
\paragraph{v1.6:} 2018/01/17

\begin{itemize}
\item
application for development of include files
\item
corrections to manual
\end{itemize}

%%%%%%%%%%%%%%%%%%%%%%%%%%%%%%%%%%%%%%%%
\paragraph{v1.5:} 2017/05/21

\begin{itemize}
\item
more complete structuring introduced
\item
|\childdocof| introduced
\item
|\childdoc| renamed to |\childdocmain|
\item
|\childredirect| renamed to |\childdocforward| and |\childdocforwardprefix|
and functionality expanded
\end{itemize}

%%%%%%%%%%%%%%%%%%%%%%%%%%%%%%%%%%%%%%%%
\paragraph{v1.0:} 2017/04/27

\begin{itemize}
\item
manual and install package
\item
first version published on CTAN
\end{itemize}

%%%%%%%%%%%%%%%%%%%%%%%%%%%%%%%%%%%%%%%%
\paragraph{v0.6:} 2017/04/26

\begin{itemize}
\item
redirection mechanism added
\end{itemize}

%%%%%%%%%%%%%%%%%%%%%%%%%%%%%%%%%%%%%%%%
\paragraph{v0.5:} 2017/04/26

\begin{itemize}
\item
functionality in definition file
\end{itemize}


%%%%%%%%%%%%%%%%%%%%%%%%%%%%%%%%%%%%%%%%%%%%%%%%%%%%%%%%%%%%%%%%%%%%%%%%%%%%%%%%
%%%%%%%%%%%%%%%%%%%%%%%%%%%%%%%%%%%%%%%%%%%%%%%%%%%%%%%%%%%%%%%%%%%%%%%%%%%%%%%%
%%%%%%%%%%%%%%%%%%%%%%%%%%%%%%%%%%%%%%%%%%%%%%%%%%%%%%%%%%%%%%%%%%%%%%%%%%%%%%%%
\appendix

\settowidth\MacroIndent{\rmfamily\scriptsize 000\ }

 \DocInput{childdoc.dtx}

\end{document}
%</driver>
% \fi
%
% %%%%%%%%%%%%%%%%%%%%%%%%%%%%%%%%%%%%%%%%%%%%%%%%%%%%%%%%%%%%%%%%%%%%%%%%%%%%%%
% %%%%%%%%%%%%%%%%%%%%%%%%%%%%%%%%%%%%%%%%%%%%%%%%%%%%%%%%%%%%%%%%%%%%%%%%%%%%%%
% \section{Sample}
%\iffalse
%<*samplemain>
%\fi
%
% The following presents a sample document
% with two chapters, two parts, a title page,
% a compile flag as well as three forwarding files to set the flag.
% It consists of eight |.tex| files:
% \begin{center}
% \begin{tabular}{ll}
% |cdocsamp.tex|&main file\\
% |cdocsch1.tex|&include file for chapter 1\\
% |cdocsch2.tex|&include file for chapter 2\\
% |cdocspt3.tex|&include file for part 3\\
% |cdocspt4.tex|&include file for part 4\\
% |cdocsdrf.tex|&forwarding file for main file in draft mode\\
% |cdocsfi1.tex|&forwarding file for final version of chapter 1\\
% |cdocsfi2.tex|&forwarding file for final version of chapter 2\\
% \end{tabular}
% \end{center}
% Each of the eight files can be compiled directly by the \LaTeX{} compiler.
%
% %%%%%%%%%%%%%%%%%%%%%%%%%%%%%%%%%%%%%%
% \paragraph{Main File.}
%
% The main file is called |cdocsamp.tex|.
%
% Load the \textsf{childdoc} definitions and
% declare the filename for the main document:
%    \begin{macrocode}
\input{childdoc.def}
\childdocmain{}
%    \end{macrocode}

% Optional override for |\version| flag:
%    \begin{macrocode}
%%\ifchilddoc\else\providecommand{\version}{draft}\fi
%    \end{macrocode}

% Define the default values for the |\version| flag
% (|final| for the main file and |draft| for childs):
%    \begin{macrocode}
\ifchilddoc
\providecommand{\version}{draft}
\else
\providecommand{\version}{final}
\fi
%    \end{macrocode}

% Load the standard document class:
%    \begin{macrocode}
\documentclass[12pt]{article}
%    \end{macrocode}

% Start the document body:
%    \begin{macrocode}
\begin{document}
%    \end{macrocode}

% Declare a title page.
% Print title, part of document being processed and version flag:
%    \begin{macrocode}
\addtocounter{page}{-1}
\begin{center}
{\LARGE\bfseries{}childdoc example\par}
\vspace{1cm}
\ifchilddoc
\ifchilddocmanual part\else chapter\fi:
`\childdocname' of `\childdocjob'\par
\else
main document: `\childdocjob'\par
\fi
version: \version\par
\end{center}
\newpage
%    \end{macrocode}

% Manually include selected file,
% otherwise process as usual:
%    \begin{macrocode}
\ifchilddocmanual
\section*{part `\childdocname'}
\input{\childdocname}
\else
%    \end{macrocode}

% Include the two chapters:
%    \begin{macrocode}
\include{cdocsch1}
\include{cdocsch2}
%    \end{macrocode}

% Include the two parts unless only chapters should be displayed:
%    \begin{macrocode}
\ifchilddoc\else
\section{part three}
\input{cdocspt3}
\section{part four}
\input{cdocspt4}
\fi
%    \end{macrocode}

% Process as usual until here:
%    \begin{macrocode}
\fi
%    \end{macrocode}

% End of document body:
%    \begin{macrocode}
\end{document}
%    \end{macrocode}
%\iffalse
%</samplemain>
%\fi
%
% %%%%%%%%%%%%%%%%%%%%%%%%%%%%%%%%%%%%%%
% \paragraph{Chapter Include Files.}
%
% The include files are called |cdocsch1.tex| and |cdocsch2.tex|.
%
%\iffalse
%<*samplechap1|samplechap2>
%\fi

% Optional override for |\version| flag:
%    \begin{macrocode}
%%\providecommand{\version}{final}
%    \end{macrocode}

% Include the main document:
%    \begin{macrocode}
\input{childdoc.def}
\childdocof{cdocsamp}
%    \end{macrocode}

%\iffalse
%</samplechap1|samplechap2>
%\fi
%
%\iffalse
%<*samplechap1>
%\fi
% Some text for chapter 1:
%    \begin{macrocode}
\section{one}
some text in chapter one
%    \end{macrocode}

%\iffalse
%</samplechap1>
%\fi
% Some text for chapter 2:
%\iffalse
%<*samplechap2>
%\fi
%    \begin{macrocode}
\section{two}
more text in chapter two
%    \end{macrocode}

%\iffalse
%</samplechap2>
%\fi
%
% %%%%%%%%%%%%%%%%%%%%%%%%%%%%%%%%%%%%%%
% \paragraph{Part Include Files.}
%
% The include files are called |cdocspt3.tex| and |cdocspt4.tex|.
%
%\iffalse
%<*samplepart3|samplepart4>
%\fi

% Optional override for |\version| flag:
%    \begin{macrocode}
%%\providecommand{\version}{final}
%    \end{macrocode}

% Include the main document:
%    \begin{macrocode}
\input{childdoc.def}
\childdocby{cdocsamp}
%    \end{macrocode}

%\iffalse
%</samplepart3|samplepart4>
%\fi
%
%\iffalse
%<*samplepart3>
%\fi
% Some text for part 3:
%    \begin{macrocode}
some text in part three
%    \end{macrocode}

%\iffalse
%</samplepart3>
%\fi
% Some text for part 4:
%\iffalse
%<*samplepart4>
%\fi
%    \begin{macrocode}
more text in part four
%    \end{macrocode}

%\iffalse
%</samplepart4>
%\fi
%
% %%%%%%%%%%%%%%%%%%%%%%%%%%%%%%%%%%%%%%
% \paragraph{Forwarding for a Complete Draft.}
%
% The following forwarding file |cdocsdrf.tex|
% compiles the main document in draft mode:
%\iffalse
%<*sampledraft>
%\fi
%    \begin{macrocode}
\def\version{draft}
\input{childdoc.def}
\childdocforward{cdocsamp}
%    \end{macrocode}

%\iffalse
%</sampledraft>
%\fi
%
% %%%%%%%%%%%%%%%%%%%%%%%%%%%%%%%%%%%%%%
% \paragraph{Forwarding for Final Version of the Chapters.}
%
% The following forwarding files |cdocsfn1.tex| and |cdocsfn2.tex|
% (with identical content)
% compile the final versions of the child documents
% |cdocsch1.tex| and |cdocsch2.tex|, respectively:
%\iffalse
%<*samplefinal>
%\fi
%    \begin{macrocode}
\def\version{final}
\input{childdoc.def}
\childdocforwardprefix[cdocsamp]{cdocsfn}{cdocsch}
%    \end{macrocode}

%\iffalse
%</samplefinal>
%\fi
%
% %%%%%%%%%%%%%%%%%%%%%%%%%%%%%%%%%%%%%%
% \paragraph{Command Line Processing.}
%
% The following three command lines generate the output files
% |cdocscld|, |cdocscl1| and |cdocscl2|
% which should be identical to
% |cdocsdrf|, |cdocsch1| and |cdocsfn2|, respectively:
% \begin{center}
% \begin{tabular}{l}
% |latex -jobname cdocscld \|\\
% |  "\def\version{draft}\input{childdoc.def}\childdocforward{cdocsamp}"|\\
% |latex -jobname cdocscl1 \|\\
% |  "\input{childdoc.def}\childdocforward[cdocsamp]{cdocsch1}"|\\
% |latex -jobname cdocscl2 \|\\
% |  "\def\version{final}\input{childdoc.def}\childdocforward{cdocsch2}"|
% \end{tabular}
% \end{center}
% Note that the trailing backslash on each first line
% merely continues the input to the second line
% (for convenient cut ant paste).
% Furthermore, the command |latex| can be replaced by any
% of its alternative versions such as |pdflatex|.
%
% %%%%%%%%%%%%%%%%%%%%%%%%%%%%%%%%%%%%%%%%%%%%%%%%%%%%%%%%%%%%%%%%%%%%%%%%%%%%%%
% %%%%%%%%%%%%%%%%%%%%%%%%%%%%%%%%%%%%%%%%%%%%%%%%%%%%%%%%%%%%%%%%%%%%%%%%%%%%%%
% \section{Implementation}
%\iffalse
%<*package>
%\fi
%
% This section describes the definitions file |childdoc.def|.

% The definitions cannot be loaded using |\usepackage| or |\RequirePackage|
% which has a mechanism to prevent loading a style file more than once.
% When loading the definitions by means of |\input|
% multiple instances have to be prevented manually:
%\iffalse
%This code needs to be before the `\ProvidesFile' directive
%which is defined at the beginning of this file.
%Therefore it is also placed there and commented out here.
%</package>
%<*discard>
%\fi
%    \begin{macrocode}
\ifdefined\childdocmain\endinput\fi
%    \end{macrocode}
%\iffalse
%</discard>
%<*package>
%\fi
%
% \macro{\ifchilddoc}
% \macro{\ifchilddocmanual}
% The conditional |\ifchilddoc| tells whether a
% child (true) or main (false) document is being compiled.
% The conditional |\ifchilddocmanual| tells whether
% the |\includeonly| mechanism is used (false) or
% the selection of child files must be performed manually (true).
% The definitions initialise to false:
%    \begin{macrocode}
\newif\ifchilddoc
\newif\ifchilddocmanual
%    \end{macrocode}

% \macro{\childdocname}
% \macro{\childdocjob}
% The macro |\childdocname| stores the name of the main document
% to be compiled. The macro |\childdocjob| stores the name of
% the document on which the \LaTeX{} compiler was originally invoked.
% The content of |\jobname| cannot be compared
% to filenames specified in the source due to different catcodes.
% The following code rescans |\jobname|, stores the result
% in |\childdocname| and saves a copy in |\childdocjob|:
%    \begin{macrocode}
\edef\childdocname{\scantokens\expandafter{\jobname\noexpand}}
\let\childdocjob\childdocname
%    \end{macrocode}

% \macro{\childdocdisable}
% The macro |\childdocdisable| prevents the main file
% from being processed more than once.
% At this stage, the main document command |\childdocmain|
% is assumed to be called once again where it should do nothing.
% Any subsequent call to it should prevent
% a secondary processing of the main document
% It overwrites the forwarding commands
% |\childdocof| and |\childdocforward|
% with empty macros to prevent further inclusions of the main document:
%    \begin{macrocode}
\newcommand{\childdocdisable}
{
  \renewcommand{\childdocmain}[1]{\renewcommand{\childdocmain}[1]{\endinput}}
  \renewcommand{\childdocof}[1]{}
  \renewcommand{\childdocby}[2][]{}
  \renewcommand{\childdocforward}[2][]{}
  \renewcommand{\childdocdisable}{}
}
%    \end{macrocode}

% \macro{\childdocmain}
% The macro |\childdocmain| is to be called at the top of the main file
% with nothing or the main filename (without extension) as argument.
% First, it breaks loops.
% If the argument is not empty and does not match |\childdocname|
% (which is set by the first inclusion of |childdoc.def|),
% |\ifchilddoc| is set to true, |\includeonly| is applied to the child file
% and |\jobname| is set to the main file
% (for proper handling of |.aux| files):
%    \begin{macrocode}
\newcommand{\childdocmain}[1]
{
  \childdocdisable\childdocmain{}
  \if?#1?\else
    \begingroup
      \def\childdoctmp{#1}
      \ifx\childdoctmp\childdocname
        \def\childdoctmp{}
      \else
        \def\childdoctmp
        {
          \childdoctrue
          \includeonly{\childdocname}
          \def\childdocjob{#1}
          \def\jobname{#1}
        }
      \fi
      \expandafter
    \endgroup
    \childdoctmp
  \fi
}
%    \end{macrocode}

% \macro{\childdocof}
% The command |\childdocof| redirects
% compilation to the main file |#1|.
%    \begin{macrocode}
\newcommand{\childdocof}[1]
{
  \childdocdisable
  \childdoctrue
  \includeonly{\childdocname}
  \def\jobname{#1}
  \def\childdocjob{#1}
  \input{#1}
}
%    \end{macrocode}

% \macro{\childdocby}
% The command |\childdocby| ....
%    \begin{macrocode}
\newcommand{\childdocby}[2][]
{
  \childdocdisable
  \childdoctrue
  \childdocmanualtrue
  \if?#1?\else
    \def\jobname{#2}
  \fi
  \def\childdocjob{#2}
  \input{#2}
  \endinput
}
%    \end{macrocode}

% \macro{\childdocforward}
% The command |\childdocforward| redirects
% compilation to the main file or
% (if the optional argument is given) a child file.
% Parameters are set as if the main file
% or a child file starting with |\childdocof| was compiled.
% Then compilation is handed over to the main file:
%    \begin{macrocode}
\newcommand{\childdocforward}[2][]
{
  \begingroup
    \if?#1?
      \def\childdoctmp
      {
        \def\childdocname{#2}
        \def\childdocjob{#2}
        \def\jobname{#2}
        \input{#2}
        \endinput
      }
    \else
      \def\childdoctmp
      {
        \childdocdisable
        \def\childdocname{#2}
        \childdoctrue
        \includeonly{#2}
        \def\childdocjob{#1}
        \def\jobname{#1}
        \input{#1}
        \endinput
      }
    \fi
    \expandafter
  \endgroup
  \childdoctmp
}
%    \end{macrocode}

% \macro{\childdocforwardprefix}
% The command |\childdocforwardprefix| redirects
% compilation to the main or a child file by means of a pattern.
% The prefix |#1| in the current filename is replaced by |#2|
% and the suffix of the current filename is kept
% (it is assumed that the filename does not contain the substring `|~~~|'
% which is used as a delimiter).
% Compilation is handed over to the new file by |\childdocforward|:
%    \begin{macrocode}
\newcommand{\childdocforwardprefix}[3][]
{
  \begingroup
    \def\childdocextract #2##1~~~{\def\childdoctmp{\childdocforward[#1]{#3##1}}}
    \expandafter\childdocextract\childdocname~~~
    \expandafter
  \endgroup
  \childdoctmp
}
%    \end{macrocode}

% \macro{\childdoc}
% The deprecated macro |\childdoc| is a legacy version of |\childdocmain|:
%    \begin{macrocode}
\newcommand{\childdoc}{\childdocmain}
%    \end{macrocode}

% \macro{\childdocredirect}
% The deprecated macro |\childdocredirect| is a legacy version
% of |\childdocforward| and |\childdocforwardprefix|:
%    \begin{macrocode}
\newcommand{\childdocredirect}[2][]
{
  \begingroup
    \if?#1?
      \def\childdoctmp{\childdocforward{#2}}
    \else
      \def\childdoctmp{\childdocforwardprefix{#1}{#2}}
    \fi
    \expandafter
  \endgroup
  \childdoctmp
}
%    \end{macrocode}

%\iffalse
%</package>
%\fi
%
\endinput
|\\
|\childdocof{|\textit{main}|}|\\
\end{tabular}
\end{center}
at the top of every child file \textit{child}
which is included by |\include{|\textit{child}|}|
from within the main file
(or at least for those files to be compiled individually).
The argument \textit{main} must be the filename of the main file.

There are a couple of
considerations in setting up the main and child documents:

%%%%%%%%%%%%%%%%%%%%%%%%%%%%%%%%%%%%%%%%
\paragraph{Restrictions.}

Please note the following restrictions:
\begin{itemize}
\item
|\childdocmain| must be called with one argument \textit{main}
to ensure compatibility with earlier version of the package.
It must either be empty (|\childdocmain{}|)
or precisely match the filename of the main file in which it is specified.
See \secref{sec:detection} for further information.
\item
The filename \textit{main} must be specified without the |.tex| extension.
\item
The filename \textit{main} is case sensitive
(even in case-insensitive file systems)
due to internal string comparison.
\item
The argument \textit{main} should be fully expanded, it cannot be a macro.
\item
Subdirectories and special characters should be avoided in filenames.
\item
The command |\childdocmain{|\textit{main}|}| must be followed by a whitespace.
It should not be followed immediately by another command
or by a comment mark `|%|'.
This is because the \TeX{} parser reads the token immediately following
the argument of |\childdocmain| and puts it
at the beginning of every child section;
however, a white\-space is ignored.
\end{itemize}

%%%%%%%%%%%%%%%%%%%%%%%%%%%%%%%%%%%%%%%%
\paragraph{Content of Main File.}

It is advisable to place all content in the child files included by |\include|.
Any output contained in the main file will appear in all child documents
unless suppressed manually;
it cannot be suppressed automatically by the |\includeonly| directive
and thus should normally be avoided.
A method to include some content in the main file
by means of conditional processing is described in \secref{sec:conditional}.

%%%%%%%%%%%%%%%%%%%%%%%%%%%%%%%%%%%%%%%%
\paragraph{Page Numbering.}

When only a part of the document is compiled,
the appropriate numbering of pages
(as well as other status parameters)
is determined from the |.aux| files.
The latter contain information from previous passes.
However this information needs to propagate through
all intermediate child documents.
Therefore the page numbering in child documents may well
be inconsistent until the complete document is compiled at least once.

A useful (if unconventional) way to always ensure a consistent
page numbering is to restart the numbering in each child document
and denote the pages by `\textit{child}|.|\textit{page}'
where \textit{child} represents the chapter/section number of the child file.
This can be achieved by the command
|\numberwithin{page}{|\textit{child}|}|
of the \textsf{amsmath} package
where \textit{child} can be |chapter| or |section|
depending on the chosen structuring.
Alternatively, one can modify the macro |\thepage| appropriately
and reset the counter |page| at the start of each child file.

%%%%%%%%%%%%%%%%%%%%%%%%%%%%%%%%%%%%%%%%%%%%%%%%%%%%%%%%%%%%%%%%%%%%%%%%%%%%%%%%
\subsection{Conditional Processing}
\label{sec:conditional}

The package provides a mechanism to compile different versions
of a document. To customise the versions further some conditional processing
can come in handy to distinguish which version is being compiled.
The package provides two macros to describe the compilation context:

%%%%%%%%%%%%%%%%%%%%%%%%%%%%%%%%%%%%%%%%
\DescribeMacro{\ifchilddoc}
The conditional |\ifchilddoc| distinguishes between the compilation of
child documents and the main document:
%
\begin{center}
|\ifchilddoc |\textit{child-code}| |[|\||else |\textit{main-code}]| \||fi|
\end{center}

%%%%%%%%%%%%%%%%%%%%%%%%%%%%%%%%%%%%%%%%
\DescribeMacro{\childdocname}
\DescribeMacro{\childdocjob}
The macro |\childdocname| contains the filename (without extension)
of the main or child file being processed.
Note that |\childdocjob| will always contain the name of the main file.

%%%%%%%%%%%%%%%%%%%%%%%%%%%%%%%%%%%%%%%%
\paragraph{Title Page.}

Conditional processing can be used to include a title or banner page
in the main document when proper precautions are taken.
Importantly, the code in the main file should ensure that the page counter
(as well as other status parameters which are stored in the |.aux| files)
takes the same value after the conditional processing.
Otherwise the page numbers may take divergent values
depending on which part is compiled.

For example, a title page could be declared by:
%
\begin{center}
\begin{tabular}{l}
|\ifchilddoc\||else|\\
|\addtocounter{page}{-1}|\\
\textit{code for title page}\\
|\newpage|\\
|\||fi|
\end{tabular}
\end{center}
%
A banner page for the child documents can be generated by:
%
\begin{center}
\begin{tabular}{l}
|\ifchilddoc|\\
|\addtocounter{page}{-1}|\\
\textit{code for banner page}\\
|\newpage|\\
|\||fi|
\end{tabular}
\end{center}
%
Here one could write a message such as:
\begin{center}
|This is the part \childdocname{} of \childdocjob{}.|
\end{center}

%%%%%%%%%%%%%%%%%%%%%%%%%%%%%%%%%%%%%%%%%%%%%%%%%%%%%%%%%%%%%%%%%%%%%%%%%%%%%%%%
\subsection{Flags}
\label{sec:flags}

The package makes it easy to generate different versions
of the main or child documents.
To this end compilation flags can be defined
and assigned different default values.
They will be particularly useful in conjunction
with the forwarding mechanism described in \secref{sec:forward}.

For example, it may be useful to have a flag |\version|
which can be set to |draft| or |final|.
The document source will contain some conditional code
depending on the value of |\version|.
Suppose further, the flag should default to |final| for the main file
and to |draft| for child files
which is a natural assignment for editing the document.
This is achieved by placing the following code
in the preamble of the main document
(below the |\childdocmain| directive):
%
\begin{center}
\begin{tabular}{l}
|\ifchilddoc|\\
|\providecommand{\version}{draft}|\\
|\||else|\\
|\providecommand{\version}{final}|\\
|\||fi|
\end{tabular}
\end{center}
%
The definition by |\providecommand| makes sure
that previous definitions are not overwritten.
Further statements |\providecommand{\version}{...}|
can thus be added before the above code to override it.

For the main file, one might add a line
(between |\childdocmain| and the above block)
%
\begin{center}
|%\ifchilddoc\||else\providecommand{\version}{draft}\||fi|
\end{center}
%
which can be uncommented to produce a draft version.
Likewise one can add a line to the very top of a child file
(above the |\childdocof{|\textit{main}|}| directive)
%
\begin{center}
|%\providecommand{\version}{final}|
\end{center}
%
which can be uncommented to produce the final version of this child document.

%%%%%%%%%%%%%%%%%%%%%%%%%%%%%%%%%%%%%%%%%%%%%%%%%%%%%%%%%%%%%%%%%%%%%%%%%%%%%%%%
\subsection{Forwarding}
\label{sec:forward}

Different versions of the main or child documents
using compilation flags as described in \secref{sec:flags}
can be (permanently) stored in different files
for convenient compilation, viewing and distribution.
To this end, the package defines a command
to pass on compilation to a different file:

%%%%%%%%%%%%%%%%%%%%%%%%%%%%%%%%%%%%%%%%
\DescribeMacro{\childdocforward}
The command |\childdocforward| redirects processing to
another source file:
%
\begin{center}
\begin{tabular}{l}
|% \iffalse
%
% childdoc.dtx Copyright (C) 2017-2018 Niklas Beisert
%
% This work may be distributed and/or modified under the
% conditions of the LaTeX Project Public License, either version 1.3
% of this license or (at your option) any later version.
% The latest version of this license is in
%   http://www.latex-project.org/lppl.txt
% and version 1.3 or later is part of all distributions of LaTeX
% version 2005/12/01 or later.
%
% This work has the LPPL maintenance status `maintained'.
%
% The Current Maintainer of this work is Niklas Beisert.
%
% This work consists of the files childdoc.dtx and childdoc.ins
% and the derived files childdoc.def and cdocsamp.tex with
% cdocsch1.tex, cdocsch2.tex, cdocsdrf.tex, cdocsfn1.tex, cdocsfn2.tex.
%
%<package>\ifdefined\childdocmain\endinput\fi
%<package>\ProvidesFile{childdoc.def}[2018/12/30 v2.0 child document driver]
%<samplemain>\ProvidesFile{cdocsamp.tex}[2018/12/30 v2.0 sample for childdoc]
%<*driver>
%\ProvidesFile{childdoc.drv}[2018/12/30 v2.0 childdoc reference manual file]
\PassOptionsToClass{10pt,a4paper}{article}
\documentclass{ltxdoc}

\usepackage[margin=35mm]{geometry}
\usepackage{hyperref}
\usepackage{hyperxmp}
\usepackage[usenames]{color}

\hypersetup{colorlinks=true}
\hypersetup{pdfstartview=FitH}
\hypersetup{pdfpagemode=UseNone}
\hypersetup{pdfsource={}}
\hypersetup{pdflang={en-UK}}
\hypersetup{pdfcopyright={Copyright 2017-2018 Niklas Beisert.
  This work may be distributed and/or modified under the
  conditions of the LaTeX Project Public License, either version 1.3
  of this license or (at your option) any later version.}}
\hypersetup{pdflicenseurl={http://www.latex-project.org/lppl.txt}}
\hypersetup{pdfcontactaddress={ETH Zurich, ITP, HIT K,
  Wolfgang-Pauli-Strasse 27}}
\hypersetup{pdfcontactpostcode={8093}}
\hypersetup{pdfcontactcity={Zurich}}
\hypersetup{pdfcontactcountry={Switzerland}}
\hypersetup{pdfcontactemail={nbeisert@itp.phys.ethz.ch}}
\hypersetup{pdfcontacturl={http://people.phys.ethz.ch/\xmptilde nbeisert/}}

\newcommand{\secref}[1]{\hyperref[#1]{section \ref*{#1}}}

\parskip1ex
\parindent0pt
\let\olditemize\itemize
\def\itemize{\olditemize\parskip0pt}

\begin{document}

\title{The \textsf{childdoc} Package}
\hypersetup{pdftitle={The childdoc Package}}
\author{Niklas Beisert\\[2ex]
  Institut f\"ur Theoretische Physik\\
  Eidgen\"ossische Technische Hochschule Z\"urich\\
  Wolfgang-Pauli-Strasse 27, 8093 Z\"urich, Switzerland\\[1ex]
  \href{mailto:nbeisert@itp.phys.ethz.ch}
  {\texttt{nbeisert@itp.phys.ethz.ch}}}
\hypersetup{pdfauthor={Niklas Beisert}}
\hypersetup{pdfsubject={Manual for the LaTeX2e Package childdoc}}
\date{30 December 2018, \textsf{v2.0}}
\maketitle

\begin{abstract}\noindent
\textsf{childdoc} is a \LaTeXe{} package
that enables the direct compilation
of document sections included by |\include|
to individual files.
\end{abstract}

\begingroup
\parskip0ex
\tableofcontents
\endgroup

%%%%%%%%%%%%%%%%%%%%%%%%%%%%%%%%%%%%%%%%%%%%%%%%%%%%%%%%%%%%%%%%%%%%%%%%%%%%%%%%
%%%%%%%%%%%%%%%%%%%%%%%%%%%%%%%%%%%%%%%%%%%%%%%%%%%%%%%%%%%%%%%%%%%%%%%%%%%%%%%%
\section{Introduction}

\LaTeX{} provides a mechanism to structure a large document (such as a book)
into a main file and several child files (containing the chapters)
using the |\include| command.
This mechanism is beneficial for documents
which span hundreds of pages in order to
make the source file(s) more manageable.
Moreover, compilation can be restricted to
selected child files by means of the |\includeonly| command.
The latter feature can be used to reduce the compilation time while editing
(this was significantly more useful in the earlier days of \LaTeX{})
or to generate a smaller document which is easier to navigate.
Another application of |\includeonly| is to generate
documents consisting of selected parts of the complete document.

However, there are a few drawbacks of the plain |\include| mechanism:
\begin{itemize}
\item
The child files cannot be compiled on their own,
they can only be compiled via the main file.
A naive editing environment
(such as a text editor with an option
to have the current file processed by \LaTeX)
may require one to switch to the main file before compiling;
attempting to compile the child file produces errors.
\item
The main file must be modified (each time)
to adjust the |\includeonly| command
to the present needs. This easily leaves the main file in a messy state.
\item
The generated document will always carry the filename
of the main document. This is inconvenient if
several child files are to be compiled and
to be kept for distribution.
\end{itemize}

The present package provides a simple interface
to make child files individually compilable by \LaTeX{}.
Compiling a child file then has the same effect as compiling
the main file with an |\includeonly| command
to select the appropriate child.
Moreover the generated document will carry the name of the child
rather than the main file.
This resolves all three above issues.

This feature is meant to make the editing of books,
thesis documents and lecture notes somewhat more convenient.
However, the package can also be used efficiently for
composing a series of documents (such as exercise sheets)
which are typically distributed individually.
It then assists the author in generating the individual documents
(potentially in different versions)
as well as a document containing the collected series.
Another application is in developing style files
or other kinds of included material
where compilation of the style file could redirect
to a sample or test file.

%%%%%%%%%%%%%%%%%%%%%%%%%%%%%%%%%%%%%%%%%%%%%%%%%%%%%%%%%%%%%%%%%%%%%%%%%%%%%%%%
%%%%%%%%%%%%%%%%%%%%%%%%%%%%%%%%%%%%%%%%%%%%%%%%%%%%%%%%%%%%%%%%%%%%%%%%%%%%%%%%
\section{Usage}

First of all, the package \textsf{childdoc} is \emph{not} a standard
\LaTeXe{} |.sty| style file! Therefore it needs to be invoked in
a non-standard way.

%%%%%%%%%%%%%%%%%%%%%%%%%%%%%%%%%%%%%%%%%%%%%%%%%%%%%%%%%%%%%%%%%%%%%%%%%%%%%%%%
\subsection{Included Files}
\label{sec:include}

%%%%%%%%%%%%%%%%%%%%%%%%%%%%%%%%%%%%%%%%
\DescribeMacro{\childdocmain}
To use the package, add the commands
\begin{center}
\begin{tabular}{l}
|\input{childdoc.def}|\\
|\childdocmain{}|\\
\end{tabular}
\end{center}
at the very top of the main \LaTeX{} file,
in particular \emph{before} the |\documentclass| statement!
The argument of |\childdocmain| should be left empty
(but it must be present).

%%%%%%%%%%%%%%%%%%%%%%%%%%%%%%%%%%%%%%%%
\DescribeMacro{\childdocof}
Furthermore, add the commands
\begin{center}
\begin{tabular}{l}
|\input{childdoc.def}|\\
|\childdocof{|\textit{main}|}|\\
\end{tabular}
\end{center}
at the top of every child file \textit{child}
which is included by |\include{|\textit{child}|}|
from within the main file
(or at least for those files to be compiled individually).
The argument \textit{main} must be the filename of the main file.

There are a couple of
considerations in setting up the main and child documents:

%%%%%%%%%%%%%%%%%%%%%%%%%%%%%%%%%%%%%%%%
\paragraph{Restrictions.}

Please note the following restrictions:
\begin{itemize}
\item
|\childdocmain| must be called with one argument \textit{main}
to ensure compatibility with earlier version of the package.
It must either be empty (|\childdocmain{}|)
or precisely match the filename of the main file in which it is specified.
See \secref{sec:detection} for further information.
\item
The filename \textit{main} must be specified without the |.tex| extension.
\item
The filename \textit{main} is case sensitive
(even in case-insensitive file systems)
due to internal string comparison.
\item
The argument \textit{main} should be fully expanded, it cannot be a macro.
\item
Subdirectories and special characters should be avoided in filenames.
\item
The command |\childdocmain{|\textit{main}|}| must be followed by a whitespace.
It should not be followed immediately by another command
or by a comment mark `|%|'.
This is because the \TeX{} parser reads the token immediately following
the argument of |\childdocmain| and puts it
at the beginning of every child section;
however, a white\-space is ignored.
\end{itemize}

%%%%%%%%%%%%%%%%%%%%%%%%%%%%%%%%%%%%%%%%
\paragraph{Content of Main File.}

It is advisable to place all content in the child files included by |\include|.
Any output contained in the main file will appear in all child documents
unless suppressed manually;
it cannot be suppressed automatically by the |\includeonly| directive
and thus should normally be avoided.
A method to include some content in the main file
by means of conditional processing is described in \secref{sec:conditional}.

%%%%%%%%%%%%%%%%%%%%%%%%%%%%%%%%%%%%%%%%
\paragraph{Page Numbering.}

When only a part of the document is compiled,
the appropriate numbering of pages
(as well as other status parameters)
is determined from the |.aux| files.
The latter contain information from previous passes.
However this information needs to propagate through
all intermediate child documents.
Therefore the page numbering in child documents may well
be inconsistent until the complete document is compiled at least once.

A useful (if unconventional) way to always ensure a consistent
page numbering is to restart the numbering in each child document
and denote the pages by `\textit{child}|.|\textit{page}'
where \textit{child} represents the chapter/section number of the child file.
This can be achieved by the command
|\numberwithin{page}{|\textit{child}|}|
of the \textsf{amsmath} package
where \textit{child} can be |chapter| or |section|
depending on the chosen structuring.
Alternatively, one can modify the macro |\thepage| appropriately
and reset the counter |page| at the start of each child file.

%%%%%%%%%%%%%%%%%%%%%%%%%%%%%%%%%%%%%%%%%%%%%%%%%%%%%%%%%%%%%%%%%%%%%%%%%%%%%%%%
\subsection{Conditional Processing}
\label{sec:conditional}

The package provides a mechanism to compile different versions
of a document. To customise the versions further some conditional processing
can come in handy to distinguish which version is being compiled.
The package provides two macros to describe the compilation context:

%%%%%%%%%%%%%%%%%%%%%%%%%%%%%%%%%%%%%%%%
\DescribeMacro{\ifchilddoc}
The conditional |\ifchilddoc| distinguishes between the compilation of
child documents and the main document:
%
\begin{center}
|\ifchilddoc |\textit{child-code}| |[|\||else |\textit{main-code}]| \||fi|
\end{center}

%%%%%%%%%%%%%%%%%%%%%%%%%%%%%%%%%%%%%%%%
\DescribeMacro{\childdocname}
\DescribeMacro{\childdocjob}
The macro |\childdocname| contains the filename (without extension)
of the main or child file being processed.
Note that |\childdocjob| will always contain the name of the main file.

%%%%%%%%%%%%%%%%%%%%%%%%%%%%%%%%%%%%%%%%
\paragraph{Title Page.}

Conditional processing can be used to include a title or banner page
in the main document when proper precautions are taken.
Importantly, the code in the main file should ensure that the page counter
(as well as other status parameters which are stored in the |.aux| files)
takes the same value after the conditional processing.
Otherwise the page numbers may take divergent values
depending on which part is compiled.

For example, a title page could be declared by:
%
\begin{center}
\begin{tabular}{l}
|\ifchilddoc\||else|\\
|\addtocounter{page}{-1}|\\
\textit{code for title page}\\
|\newpage|\\
|\||fi|
\end{tabular}
\end{center}
%
A banner page for the child documents can be generated by:
%
\begin{center}
\begin{tabular}{l}
|\ifchilddoc|\\
|\addtocounter{page}{-1}|\\
\textit{code for banner page}\\
|\newpage|\\
|\||fi|
\end{tabular}
\end{center}
%
Here one could write a message such as:
\begin{center}
|This is the part \childdocname{} of \childdocjob{}.|
\end{center}

%%%%%%%%%%%%%%%%%%%%%%%%%%%%%%%%%%%%%%%%%%%%%%%%%%%%%%%%%%%%%%%%%%%%%%%%%%%%%%%%
\subsection{Flags}
\label{sec:flags}

The package makes it easy to generate different versions
of the main or child documents.
To this end compilation flags can be defined
and assigned different default values.
They will be particularly useful in conjunction
with the forwarding mechanism described in \secref{sec:forward}.

For example, it may be useful to have a flag |\version|
which can be set to |draft| or |final|.
The document source will contain some conditional code
depending on the value of |\version|.
Suppose further, the flag should default to |final| for the main file
and to |draft| for child files
which is a natural assignment for editing the document.
This is achieved by placing the following code
in the preamble of the main document
(below the |\childdocmain| directive):
%
\begin{center}
\begin{tabular}{l}
|\ifchilddoc|\\
|\providecommand{\version}{draft}|\\
|\||else|\\
|\providecommand{\version}{final}|\\
|\||fi|
\end{tabular}
\end{center}
%
The definition by |\providecommand| makes sure
that previous definitions are not overwritten.
Further statements |\providecommand{\version}{...}|
can thus be added before the above code to override it.

For the main file, one might add a line
(between |\childdocmain| and the above block)
%
\begin{center}
|%\ifchilddoc\||else\providecommand{\version}{draft}\||fi|
\end{center}
%
which can be uncommented to produce a draft version.
Likewise one can add a line to the very top of a child file
(above the |\childdocof{|\textit{main}|}| directive)
%
\begin{center}
|%\providecommand{\version}{final}|
\end{center}
%
which can be uncommented to produce the final version of this child document.

%%%%%%%%%%%%%%%%%%%%%%%%%%%%%%%%%%%%%%%%%%%%%%%%%%%%%%%%%%%%%%%%%%%%%%%%%%%%%%%%
\subsection{Forwarding}
\label{sec:forward}

Different versions of the main or child documents
using compilation flags as described in \secref{sec:flags}
can be (permanently) stored in different files
for convenient compilation, viewing and distribution.
To this end, the package defines a command
to pass on compilation to a different file:

%%%%%%%%%%%%%%%%%%%%%%%%%%%%%%%%%%%%%%%%
\DescribeMacro{\childdocforward}
The command |\childdocforward| redirects processing to
another source file:
%
\begin{center}
\begin{tabular}{l}
|\input{childdoc.def}|\\
|\childdocforward[|\textit{main}|]{|\textit{dest}|}|\\
\end{tabular}
\end{center}
%
The argument \textit{dest} is the destination file
(without extension).
It should be the main file or one of the child files.
Note that further \textsf{childdoc} directives
such as |\childdocof| and |\childdocforward|
in the indicated file will be processed in this form.
The optional argument \textit{main}
passes on directly to the main file \textit{main}
while pretending to compile the child \textit{dest}.
This form behaves as if \textit{dest}
issues |\childdocof{|\textit{main}|}| right away,
and no further \textsf{childdoc} directives will be processed.

%%%%%%%%%%%%%%%%%%%%%%%%%%%%%%%%%%%%%%%%
\DescribeMacro{\...prefix}
In the alternative form |\childdocforwardprefix|,
%
\begin{center}
\begin{tabular}{l}
|\input{childdoc.def}|\\
|\childdocforwardprefix[|\textit{main}|]{|\textit{prefix}|}{|\textit{dest}|}|
\end{tabular}
\end{center}
%
the destination file is determined by a pattern
depending on the current file:
To make this work, the current file must be called
`{\textit{prefix}\hspace{0.2em}\textit{suffix}}'
with \textit{prefix} matching precisely the argument.
Processing is then passed on to the file
`{\textit{dest}\hspace{0.2em}\textit{suffix}}'.
Surely, the same effect is achieved by
directly specifying the
argument `{\textit{dest}\hspace{0.2em}\textit{suffix}}'
in the first form.
However, that requires to set up a different file
for each child. With the alternative form of the command
all these files can have exactly the same content
which simplifies setting them up and maintaining them.

For example, the following file |draft.tex|
with a compilation flag |\version| as described in \secref{sec:flags}
compiles the main document as a draft:
%
\begin{center}
\begin{tabular}{l}
|\def\version{draft}|\\
|\input{childdoc.def}|\\
|\childdocforward{|\textit{main}|}|
\end{tabular}
\end{center}
%
Likewise, the following files |final|\textit{nn}|.tex|
compile the final version of the child document
|child|\textit{nn}|.tex|:
%
\begin{center}
\begin{tabular}{l}
|\def\version{final}|\\
|\input{childdoc.def}|\\
|\childdocforwardprefix{final}{child}|
\end{tabular}
\end{center}
%

Note that when several versions of a main file and/or of each child file
are to be generated, it may be convenient to set up a |Makefile| or
shell script to automatise the process.

%%%%%%%%%%%%%%%%%%%%%%%%%%%%%%%%%%%%%%%%%%%%%%%%%%%%%%%%%%%%%%%%%%%%%%%%%%%%%%%%
\subsection{Command Line Processing}
\label{sec:commandline}

The effect of redirection files can also be achieved by invoking
the \LaTeX{} compiler with a more elaborate command line.
Most conveniently this should be done as part
of a shell script or a |Makefile|.

When using \textsf{childdoc} in the main file, the following
command lines effectively perform a redirection
(note that depending on the shell being used,
backslashes may have to be doubled: `|\|' $\to$ `|\\|'):
%
\begin{center}
|... -jobname "|\textit{target}|" |\\|"|[\textit{flags}]%
|\input{childdoc.def}\childdocforward[|\textit{main}|]{|\textit{dest}|}"|
\end{center}
%
Here \textit{target} is the name of the output file,
\textit{main} is the name of the main file
and \textit{dest} is the name of the main or child file to be processed
(all filenames without extensions).
The optional argument \textit{main} can be omitted
if \textit{main} matches \textit{dest}.
Optionally, compilation \textit{flags} can be defined via |\def| commands.
This command line makes the \TeX{} engine believe
it is compiling the file \textit{target}
whose content is specified as the latter parameter.
The provided code then forwards the processing to
\textit{main} or \textit{dest} as described in \secref{sec:forward}.

%%%%%%%%%%%%%%%%%%%%%%%%%%%%%%%%%%%%%%%%%%%%%%%%%%%%%%%%%%%%%%%%%%%%%%%%%%%%%%%%
\subsection{Include by Input}
\label{sec:input}

Including child documents by |\include| has some restrictions by design.
Most notably, the content of a child document always occupies
its own set of pages; pages cannot be shared between child documents.
Usually, this behaviour makes perfect sense
because each child document contain an essential part of the document.
However, in some situations it may be desirable to compose
a document from a collection of parts
without having mandatory page breaks between then.
For this case, the package
provides a mechanism to include parts
by |\input| which can also be processed individually.
However, by construction this mechanism
requires manual handling of the content to be output.

%%%%%%%%%%%%%%%%%%%%%%%%%%%%%%%%%%%%%%%%
\DescribeMacro{\ifchilddocmanual}
The main file should be prepared as usual, see \secref{sec:include}.
However, the document body must make a distinction
between processing of an individual part and of the main document, e.g.:
%
\begin{center}
\begin{tabular}{l}
|\ifchilddocmanual|\\
|\input{\childdocname}|\\
|\||else|\\
\textit{document body with }|\input{|\textit{part}|}|\\
|\||fi|
\end{tabular}
\end{center}
%
The conditional |\ifchilddocmanual| is true whenever
a part to be included by |\input| is being compiled,
and the name of the part is stored in |\childdocname|.

%%%%%%%%%%%%%%%%%%%%%%%%%%%%%%%%%%%%%%%%
\DescribeMacro{\childdocby}
Each part to be included by |\input| should start with:
%
\begin{center}
\begin{tabular}{l}
|\input{childdoc.def}|\\
|\childdocby{|\textit{main}|}|\\
\end{tabular}
\end{center}
%
The directive |\childdocby| is similar to |\childdocof|
described in \secref{sec:include},
but the subsequent selection of content must be done manually.
To that end, both |\ifchilddoc| and |\ifchilddocmanual|
will be true upon processing of a part,
and the name of the part is stored in |\childdocname|.
Note that |\jobname| will be set to the filename of the current part
so that each part receives an individual |.aux| file
that does not interfere with the |.aux| file(s) of the main document.
This behaviour can be altered by the alternative form
|\childdocby[*]{|\textit{main}|}| (with a non-empty optional argument)
which uses the |.aux| file of the main document
by setting |\jobname| to \textit{main}.

%%%%%%%%%%%%%%%%%%%%%%%%%%%%%%%%%%%%%%%%%%%%%%%%%%%%%%%%%%%%%%%%%%%%%%%%%%%%%%%%
\subsection{Driver Development}
\label{sec:driver}

The \textsf{childdoc} mechanism can also be use for the development
of definition files such as \LaTeX{} styles or classes.
This case differs from the above setup with multiple parts
included by |\include| in that no |\includeonly| should be invoked.
This can be achieved by starting the include file
(before |\ProvidesPackage|) with:
%
\begin{center}
\begin{tabular}{l}
|\input{childdoc.def}|\\
|\childdocforward{|\textit{main}|}|\\
\end{tabular}
\end{center}
%
or alternatively with:
%
\begin{center}
\begin{tabular}{l}
|\input{childdoc.def}|\\
|\childdocby{|\textit{main}|}|\\
\end{tabular}
\end{center}
%
Both forms have slightly different effects as described above.
The main file is prepared as usual, see \secref{sec:include}.

%%%%%%%%%%%%%%%%%%%%%%%%%%%%%%%%%%%%%%%%%%%%%%%%%%%%%%%%%%%%%%%%%%%%%%%%%%%%%%%%
\subsection{Legacy Detection}
\label{sec:detection}

The directive |\childdocmain| in the main file can detect
whether the complete document or merely a child is to be compiled
even without using the directive |\childdocof|.
This method is deprecated because it is less robust
and there is no compelling reason to use it;
it is merely provided for backward compatibility
and it may be removed in future versions.

If the detection mechanism is to be used,
it is mandatory to correctly specify
the filename of the main file as the argument of |\childdocmain|:
%
\begin{center}
\begin{tabular}{l}
|\input{childdoc.def}|\\
|\childdocmain{|\textit{main}|}|\\
\end{tabular}
\end{center}
%
If |\jobname| does not match the argument \textit{main} of |\childdocmain|,
it is assumed that |\jobname| points to the child file to be compiled.
When using |\childdocmain| with the main file specified as argument,
it suffices to start a child file
with just |\input{|\textit{main}|}|
without loading of the package and using |\childdocof|.
If instead all processing is done
with the appropriate \textsf{childdoc} directives,
the argument of \textit{main} of |\childdocmain| can be empty.

An alternative version of the command line processing described
in \secref{sec:commandline} using the detection mechanism reads:
%
\begin{center}
|... -jobname "|\textit{target}|" "|[\textit{flags}]%
[|\def\jobname{|\textit{dest}|}|]|\input{|\textit{main}|}"|
\end{center}

%%%%%%%%%%%%%%%%%%%%%%%%%%%%%%%%%%%%%%%%%%%%%%%%%%%%%%%%%%%%%%%%%%%%%%%%%%%%%%%%
\subsection{Manual Code}
\label{sec:manual}

In case one cannot be certain whether the definitions file |childdoc.def|
is installed on the target \TeX{} distribution
and one prefers not to ship it,
it is conceivable to paste a few relevant commands into the sources.

To that end, drop all statements |\input{childdoc.def}|
and perform the replacements as outlined below.
Instead of |\childdocmain{|\textit{main}|}| add the following code
to the top of the main file:
%
\begin{center}
\begin{tabular}{l}
|\||ifdefined\childdocname\endinput\||fi\newif\ifchilddoc|\\
|\edef\childdocname{\scantokens\expandafter{\jobname\noexpand}}|\\
|\def\childdocmain{|\textit{main}|}\||ifx\childdocmain\childdocname\||else|\\
|\childdoctrue\includeonly{\childdocname}\let\jobname\childdocmain\||fi|\\
\end{tabular}
\end{center}
%
Instead of |\childdocof{|\textit{main}|}| just include the main file
at the top of each child file:
%
\begin{center}
|\input{|\textit{main}|}|
\end{center}
%
A simple redirection |\childdocforward{|\textit{dest}|}| is achieved by:
%
\begin{center}
|\def\jobname{|\textit{dest}|}\input{\jobname}|
\end{center}
%
The redirection with prefix
|\childdocforwardprefix[|\textit{prefix}|]{|\textit{dest}|}|
is accomplished by:
%
\begin{center}
\begin{tabular}{l}
|{\edef\jobname{\scantokens\expandafter{\jobname\noexpand}}|\\
|\def\redirectjob |\textit{prefix}|#1~~~{\gdef\jobname{|\textit{dest}|#1}}|\\
|\expandafter\redirectjob\jobname~~~}\input{\jobname}|
\end{tabular}
\end{center}

In an alternative approach,
child documents can be compiled by a specific command line
without additional code or specific definitions:
%
\begin{center}
|... -jobname "|\textit{target}|" "|[\textit{flags}]%
|\includeonly{|\textit{dest}|}\input{|\textit{main}|}"|
\end{center}
%

%%%%%%%%%%%%%%%%%%%%%%%%%%%%%%%%%%%%%%%%%%%%%%%%%%%%%%%%%%%%%%%%%%%%%%%%%%%%%%%%
%%%%%%%%%%%%%%%%%%%%%%%%%%%%%%%%%%%%%%%%%%%%%%%%%%%%%%%%%%%%%%%%%%%%%%%%%%%%%%%%
\section{Information}

%%%%%%%%%%%%%%%%%%%%%%%%%%%%%%%%%%%%%%%%%%%%%%%%%%%%%%%%%%%%%%%%%%%%%%%%%%%%%%%%
\subsection{Copyright}

Copyright \copyright{} 2017--2018 Niklas Beisert

This work may be distributed and/or modified under the
conditions of the \LaTeX{} Project Public License, either version 1.3
of this license or (at your option) any later version.
The latest version of this license is in
  \url{http://www.latex-project.org/lppl.txt}
and version 1.3 or later is part of all distributions of \LaTeX{}
version 2005/12/01 or later.

This work has the LPPL maintenance status `maintained'.

The Current Maintainer of this work is Niklas Beisert.

This work consists of the files |README.txt|, |childdoc.ins| and |childdoc.dtx|
as well as the derived files |childdoc.def|, |cdocsamp.tex|
with |cdocsch1.tex|, |cdocsch2.tex|, |cdocspt3.tex|, |cdocspt4.tex|,
|cdocsdrf.tex|, |cdocsfn1.tex|, |cdocsfn2.tex|
as well as |childdoc.pdf|.

%%%%%%%%%%%%%%%%%%%%%%%%%%%%%%%%%%%%%%%%%%%%%%%%%%%%%%%%%%%%%%%%%%%%%%%%%%%%%%%%
\subsection{Files and Installation}

The package consists of the files:
%
\begin{center}
\begin{tabular}{ll}
    |README.txt|   & readme file \\
    |childdoc.ins| & installation file \\
    |childdoc.dtx| & source file \\
    |childdoc.def| & definition file \\
    |cdocsamp.tex| & sample main file \\
    |cdocsch1.tex| & sample include file \\
    |cdocsch2.tex| & sample include file \\
    |cdocspt3.tex| & sample part file \\
    |cdocspt4.tex| & sample part file \\
    |cdocsdrf.tex| & sample redirection file \\
    |cdocsfn1.tex| & sample redirection file \\
    |cdocsfn2.tex| & sample redirection file \\
    |childdoc.pdf| & manual
\end{tabular}
\end{center}
%
The distribution consists of the files
|README.txt|, |childdoc.ins| and |childdoc.dtx|.
%
\begin{itemize}
\item
Run (pdf)\LaTeX{} on |childdoc.dtx|
to compile the manual |childdoc.pdf| (this file).
\item
Run \LaTeX{} on |childdoc.ins| to create the definitions file |childdoc.def|
and the sample |cdocsamp.tex| with include files
|cdocsch1.tex|, |cdocsch2.tex|, |cdocspt3.tex|, |cdocspt4.tex|,
|cdocsdrf.tex|, |cdocsfn1.tex|, |cdocsfn2.tex|.
Then copy the file |childdoc.def| to an appropriate directory of your \LaTeX{}
distribution, e.g.\ \textit{texmf-root}|/tex/latex/childdoc|.
\end{itemize}

%%%%%%%%%%%%%%%%%%%%%%%%%%%%%%%%%%%%%%%%%%%%%%%%%%%%%%%%%%%%%%%%%%%%%%%%%%%%%%%%
\subsection{Related CTAN Packages}

There are several other packages which offer a similar functionality:
%
\begin{itemize}
\item
The packages
\href{http://ctan.org/pkg/docmute}{\textsf{docmute}},
\href{http://ctan.org/pkg/includex}{\textsf{includex}} and
\href{http://ctan.org/pkg/standalone}{\textsf{standalone}}
provide commands to include only the document body of
a child file thus allowing both files to be compiled individually.
\item
The packages \href{http://ctan.org/pkg/subdocs}{\textsf{subdocs}}
and \href{http://ctan.org/pkg/subfiles}{\textsf{subfiles}}
provide structures in which the main and child documents can be
encapsulated and allowing them to be compiled individually.
The inclusion mechanism is different from the conventional |\include|.
\item
The package \href{http://ctan.org/pkg/combine}{\textsf{combine}}
is an elaborate solution to combine several documents into one.
\end{itemize}
%
See also the CTAN topic \href{http://ctan.org/topic/subdocs}{\textsf{subdocs}}
for further related packages.
The present package differs from the above solutions in that
a document structure constructed with the conventional |\include| mechanism
just needs two extra commands at the top of every file
such that all constituent files can be compiled individually.

%%%%%%%%%%%%%%%%%%%%%%%%%%%%%%%%%%%%%%%%%%%%%%%%%%%%%%%%%%%%%%%%%%%%%%%%%%%%%%%%
%\subsection{Feature Suggestions}
%
%The following is a list of features which may be useful for future
%versions of this package:
%%
%\begin{itemize}
%\item
%\ldots
%\end{itemize}

%%%%%%%%%%%%%%%%%%%%%%%%%%%%%%%%%%%%%%%%%%%%%%%%%%%%%%%%%%%%%%%%%%%%%%%%%%%%%%%%
\subsection{Revision History}

%%%%%%%%%%%%%%%%%%%%%%%%%%%%%%%%%%%%%%%%
\paragraph{v2.0:} 2018/12/30

\begin{itemize}
\item
immediate forward processing
\item
added |\childdocby| mechanism
\item
manual restructured
\end{itemize}

%%%%%%%%%%%%%%%%%%%%%%%%%%%%%%%%%%%%%%%%
\paragraph{v1.6:} 2018/01/17

\begin{itemize}
\item
application for development of include files
\item
corrections to manual
\end{itemize}

%%%%%%%%%%%%%%%%%%%%%%%%%%%%%%%%%%%%%%%%
\paragraph{v1.5:} 2017/05/21

\begin{itemize}
\item
more complete structuring introduced
\item
|\childdocof| introduced
\item
|\childdoc| renamed to |\childdocmain|
\item
|\childredirect| renamed to |\childdocforward| and |\childdocforwardprefix|
and functionality expanded
\end{itemize}

%%%%%%%%%%%%%%%%%%%%%%%%%%%%%%%%%%%%%%%%
\paragraph{v1.0:} 2017/04/27

\begin{itemize}
\item
manual and install package
\item
first version published on CTAN
\end{itemize}

%%%%%%%%%%%%%%%%%%%%%%%%%%%%%%%%%%%%%%%%
\paragraph{v0.6:} 2017/04/26

\begin{itemize}
\item
redirection mechanism added
\end{itemize}

%%%%%%%%%%%%%%%%%%%%%%%%%%%%%%%%%%%%%%%%
\paragraph{v0.5:} 2017/04/26

\begin{itemize}
\item
functionality in definition file
\end{itemize}


%%%%%%%%%%%%%%%%%%%%%%%%%%%%%%%%%%%%%%%%%%%%%%%%%%%%%%%%%%%%%%%%%%%%%%%%%%%%%%%%
%%%%%%%%%%%%%%%%%%%%%%%%%%%%%%%%%%%%%%%%%%%%%%%%%%%%%%%%%%%%%%%%%%%%%%%%%%%%%%%%
%%%%%%%%%%%%%%%%%%%%%%%%%%%%%%%%%%%%%%%%%%%%%%%%%%%%%%%%%%%%%%%%%%%%%%%%%%%%%%%%
\appendix

\settowidth\MacroIndent{\rmfamily\scriptsize 000\ }

 \DocInput{childdoc.dtx}

\end{document}
%</driver>
% \fi
%
% %%%%%%%%%%%%%%%%%%%%%%%%%%%%%%%%%%%%%%%%%%%%%%%%%%%%%%%%%%%%%%%%%%%%%%%%%%%%%%
% %%%%%%%%%%%%%%%%%%%%%%%%%%%%%%%%%%%%%%%%%%%%%%%%%%%%%%%%%%%%%%%%%%%%%%%%%%%%%%
% \section{Sample}
%\iffalse
%<*samplemain>
%\fi
%
% The following presents a sample document
% with two chapters, two parts, a title page,
% a compile flag as well as three forwarding files to set the flag.
% It consists of eight |.tex| files:
% \begin{center}
% \begin{tabular}{ll}
% |cdocsamp.tex|&main file\\
% |cdocsch1.tex|&include file for chapter 1\\
% |cdocsch2.tex|&include file for chapter 2\\
% |cdocspt3.tex|&include file for part 3\\
% |cdocspt4.tex|&include file for part 4\\
% |cdocsdrf.tex|&forwarding file for main file in draft mode\\
% |cdocsfi1.tex|&forwarding file for final version of chapter 1\\
% |cdocsfi2.tex|&forwarding file for final version of chapter 2\\
% \end{tabular}
% \end{center}
% Each of the eight files can be compiled directly by the \LaTeX{} compiler.
%
% %%%%%%%%%%%%%%%%%%%%%%%%%%%%%%%%%%%%%%
% \paragraph{Main File.}
%
% The main file is called |cdocsamp.tex|.
%
% Load the \textsf{childdoc} definitions and
% declare the filename for the main document:
%    \begin{macrocode}
\input{childdoc.def}
\childdocmain{}
%    \end{macrocode}

% Optional override for |\version| flag:
%    \begin{macrocode}
%%\ifchilddoc\else\providecommand{\version}{draft}\fi
%    \end{macrocode}

% Define the default values for the |\version| flag
% (|final| for the main file and |draft| for childs):
%    \begin{macrocode}
\ifchilddoc
\providecommand{\version}{draft}
\else
\providecommand{\version}{final}
\fi
%    \end{macrocode}

% Load the standard document class:
%    \begin{macrocode}
\documentclass[12pt]{article}
%    \end{macrocode}

% Start the document body:
%    \begin{macrocode}
\begin{document}
%    \end{macrocode}

% Declare a title page.
% Print title, part of document being processed and version flag:
%    \begin{macrocode}
\addtocounter{page}{-1}
\begin{center}
{\LARGE\bfseries{}childdoc example\par}
\vspace{1cm}
\ifchilddoc
\ifchilddocmanual part\else chapter\fi:
`\childdocname' of `\childdocjob'\par
\else
main document: `\childdocjob'\par
\fi
version: \version\par
\end{center}
\newpage
%    \end{macrocode}

% Manually include selected file,
% otherwise process as usual:
%    \begin{macrocode}
\ifchilddocmanual
\section*{part `\childdocname'}
\input{\childdocname}
\else
%    \end{macrocode}

% Include the two chapters:
%    \begin{macrocode}
\include{cdocsch1}
\include{cdocsch2}
%    \end{macrocode}

% Include the two parts unless only chapters should be displayed:
%    \begin{macrocode}
\ifchilddoc\else
\section{part three}
\input{cdocspt3}
\section{part four}
\input{cdocspt4}
\fi
%    \end{macrocode}

% Process as usual until here:
%    \begin{macrocode}
\fi
%    \end{macrocode}

% End of document body:
%    \begin{macrocode}
\end{document}
%    \end{macrocode}
%\iffalse
%</samplemain>
%\fi
%
% %%%%%%%%%%%%%%%%%%%%%%%%%%%%%%%%%%%%%%
% \paragraph{Chapter Include Files.}
%
% The include files are called |cdocsch1.tex| and |cdocsch2.tex|.
%
%\iffalse
%<*samplechap1|samplechap2>
%\fi

% Optional override for |\version| flag:
%    \begin{macrocode}
%%\providecommand{\version}{final}
%    \end{macrocode}

% Include the main document:
%    \begin{macrocode}
\input{childdoc.def}
\childdocof{cdocsamp}
%    \end{macrocode}

%\iffalse
%</samplechap1|samplechap2>
%\fi
%
%\iffalse
%<*samplechap1>
%\fi
% Some text for chapter 1:
%    \begin{macrocode}
\section{one}
some text in chapter one
%    \end{macrocode}

%\iffalse
%</samplechap1>
%\fi
% Some text for chapter 2:
%\iffalse
%<*samplechap2>
%\fi
%    \begin{macrocode}
\section{two}
more text in chapter two
%    \end{macrocode}

%\iffalse
%</samplechap2>
%\fi
%
% %%%%%%%%%%%%%%%%%%%%%%%%%%%%%%%%%%%%%%
% \paragraph{Part Include Files.}
%
% The include files are called |cdocspt3.tex| and |cdocspt4.tex|.
%
%\iffalse
%<*samplepart3|samplepart4>
%\fi

% Optional override for |\version| flag:
%    \begin{macrocode}
%%\providecommand{\version}{final}
%    \end{macrocode}

% Include the main document:
%    \begin{macrocode}
\input{childdoc.def}
\childdocby{cdocsamp}
%    \end{macrocode}

%\iffalse
%</samplepart3|samplepart4>
%\fi
%
%\iffalse
%<*samplepart3>
%\fi
% Some text for part 3:
%    \begin{macrocode}
some text in part three
%    \end{macrocode}

%\iffalse
%</samplepart3>
%\fi
% Some text for part 4:
%\iffalse
%<*samplepart4>
%\fi
%    \begin{macrocode}
more text in part four
%    \end{macrocode}

%\iffalse
%</samplepart4>
%\fi
%
% %%%%%%%%%%%%%%%%%%%%%%%%%%%%%%%%%%%%%%
% \paragraph{Forwarding for a Complete Draft.}
%
% The following forwarding file |cdocsdrf.tex|
% compiles the main document in draft mode:
%\iffalse
%<*sampledraft>
%\fi
%    \begin{macrocode}
\def\version{draft}
\input{childdoc.def}
\childdocforward{cdocsamp}
%    \end{macrocode}

%\iffalse
%</sampledraft>
%\fi
%
% %%%%%%%%%%%%%%%%%%%%%%%%%%%%%%%%%%%%%%
% \paragraph{Forwarding for Final Version of the Chapters.}
%
% The following forwarding files |cdocsfn1.tex| and |cdocsfn2.tex|
% (with identical content)
% compile the final versions of the child documents
% |cdocsch1.tex| and |cdocsch2.tex|, respectively:
%\iffalse
%<*samplefinal>
%\fi
%    \begin{macrocode}
\def\version{final}
\input{childdoc.def}
\childdocforwardprefix[cdocsamp]{cdocsfn}{cdocsch}
%    \end{macrocode}

%\iffalse
%</samplefinal>
%\fi
%
% %%%%%%%%%%%%%%%%%%%%%%%%%%%%%%%%%%%%%%
% \paragraph{Command Line Processing.}
%
% The following three command lines generate the output files
% |cdocscld|, |cdocscl1| and |cdocscl2|
% which should be identical to
% |cdocsdrf|, |cdocsch1| and |cdocsfn2|, respectively:
% \begin{center}
% \begin{tabular}{l}
% |latex -jobname cdocscld \|\\
% |  "\def\version{draft}\input{childdoc.def}\childdocforward{cdocsamp}"|\\
% |latex -jobname cdocscl1 \|\\
% |  "\input{childdoc.def}\childdocforward[cdocsamp]{cdocsch1}"|\\
% |latex -jobname cdocscl2 \|\\
% |  "\def\version{final}\input{childdoc.def}\childdocforward{cdocsch2}"|
% \end{tabular}
% \end{center}
% Note that the trailing backslash on each first line
% merely continues the input to the second line
% (for convenient cut ant paste).
% Furthermore, the command |latex| can be replaced by any
% of its alternative versions such as |pdflatex|.
%
% %%%%%%%%%%%%%%%%%%%%%%%%%%%%%%%%%%%%%%%%%%%%%%%%%%%%%%%%%%%%%%%%%%%%%%%%%%%%%%
% %%%%%%%%%%%%%%%%%%%%%%%%%%%%%%%%%%%%%%%%%%%%%%%%%%%%%%%%%%%%%%%%%%%%%%%%%%%%%%
% \section{Implementation}
%\iffalse
%<*package>
%\fi
%
% This section describes the definitions file |childdoc.def|.

% The definitions cannot be loaded using |\usepackage| or |\RequirePackage|
% which has a mechanism to prevent loading a style file more than once.
% When loading the definitions by means of |\input|
% multiple instances have to be prevented manually:
%\iffalse
%This code needs to be before the `\ProvidesFile' directive
%which is defined at the beginning of this file.
%Therefore it is also placed there and commented out here.
%</package>
%<*discard>
%\fi
%    \begin{macrocode}
\ifdefined\childdocmain\endinput\fi
%    \end{macrocode}
%\iffalse
%</discard>
%<*package>
%\fi
%
% \macro{\ifchilddoc}
% \macro{\ifchilddocmanual}
% The conditional |\ifchilddoc| tells whether a
% child (true) or main (false) document is being compiled.
% The conditional |\ifchilddocmanual| tells whether
% the |\includeonly| mechanism is used (false) or
% the selection of child files must be performed manually (true).
% The definitions initialise to false:
%    \begin{macrocode}
\newif\ifchilddoc
\newif\ifchilddocmanual
%    \end{macrocode}

% \macro{\childdocname}
% \macro{\childdocjob}
% The macro |\childdocname| stores the name of the main document
% to be compiled. The macro |\childdocjob| stores the name of
% the document on which the \LaTeX{} compiler was originally invoked.
% The content of |\jobname| cannot be compared
% to filenames specified in the source due to different catcodes.
% The following code rescans |\jobname|, stores the result
% in |\childdocname| and saves a copy in |\childdocjob|:
%    \begin{macrocode}
\edef\childdocname{\scantokens\expandafter{\jobname\noexpand}}
\let\childdocjob\childdocname
%    \end{macrocode}

% \macro{\childdocdisable}
% The macro |\childdocdisable| prevents the main file
% from being processed more than once.
% At this stage, the main document command |\childdocmain|
% is assumed to be called once again where it should do nothing.
% Any subsequent call to it should prevent
% a secondary processing of the main document
% It overwrites the forwarding commands
% |\childdocof| and |\childdocforward|
% with empty macros to prevent further inclusions of the main document:
%    \begin{macrocode}
\newcommand{\childdocdisable}
{
  \renewcommand{\childdocmain}[1]{\renewcommand{\childdocmain}[1]{\endinput}}
  \renewcommand{\childdocof}[1]{}
  \renewcommand{\childdocby}[2][]{}
  \renewcommand{\childdocforward}[2][]{}
  \renewcommand{\childdocdisable}{}
}
%    \end{macrocode}

% \macro{\childdocmain}
% The macro |\childdocmain| is to be called at the top of the main file
% with nothing or the main filename (without extension) as argument.
% First, it breaks loops.
% If the argument is not empty and does not match |\childdocname|
% (which is set by the first inclusion of |childdoc.def|),
% |\ifchilddoc| is set to true, |\includeonly| is applied to the child file
% and |\jobname| is set to the main file
% (for proper handling of |.aux| files):
%    \begin{macrocode}
\newcommand{\childdocmain}[1]
{
  \childdocdisable\childdocmain{}
  \if?#1?\else
    \begingroup
      \def\childdoctmp{#1}
      \ifx\childdoctmp\childdocname
        \def\childdoctmp{}
      \else
        \def\childdoctmp
        {
          \childdoctrue
          \includeonly{\childdocname}
          \def\childdocjob{#1}
          \def\jobname{#1}
        }
      \fi
      \expandafter
    \endgroup
    \childdoctmp
  \fi
}
%    \end{macrocode}

% \macro{\childdocof}
% The command |\childdocof| redirects
% compilation to the main file |#1|.
%    \begin{macrocode}
\newcommand{\childdocof}[1]
{
  \childdocdisable
  \childdoctrue
  \includeonly{\childdocname}
  \def\jobname{#1}
  \def\childdocjob{#1}
  \input{#1}
}
%    \end{macrocode}

% \macro{\childdocby}
% The command |\childdocby| ....
%    \begin{macrocode}
\newcommand{\childdocby}[2][]
{
  \childdocdisable
  \childdoctrue
  \childdocmanualtrue
  \if?#1?\else
    \def\jobname{#2}
  \fi
  \def\childdocjob{#2}
  \input{#2}
  \endinput
}
%    \end{macrocode}

% \macro{\childdocforward}
% The command |\childdocforward| redirects
% compilation to the main file or
% (if the optional argument is given) a child file.
% Parameters are set as if the main file
% or a child file starting with |\childdocof| was compiled.
% Then compilation is handed over to the main file:
%    \begin{macrocode}
\newcommand{\childdocforward}[2][]
{
  \begingroup
    \if?#1?
      \def\childdoctmp
      {
        \def\childdocname{#2}
        \def\childdocjob{#2}
        \def\jobname{#2}
        \input{#2}
        \endinput
      }
    \else
      \def\childdoctmp
      {
        \childdocdisable
        \def\childdocname{#2}
        \childdoctrue
        \includeonly{#2}
        \def\childdocjob{#1}
        \def\jobname{#1}
        \input{#1}
        \endinput
      }
    \fi
    \expandafter
  \endgroup
  \childdoctmp
}
%    \end{macrocode}

% \macro{\childdocforwardprefix}
% The command |\childdocforwardprefix| redirects
% compilation to the main or a child file by means of a pattern.
% The prefix |#1| in the current filename is replaced by |#2|
% and the suffix of the current filename is kept
% (it is assumed that the filename does not contain the substring `|~~~|'
% which is used as a delimiter).
% Compilation is handed over to the new file by |\childdocforward|:
%    \begin{macrocode}
\newcommand{\childdocforwardprefix}[3][]
{
  \begingroup
    \def\childdocextract #2##1~~~{\def\childdoctmp{\childdocforward[#1]{#3##1}}}
    \expandafter\childdocextract\childdocname~~~
    \expandafter
  \endgroup
  \childdoctmp
}
%    \end{macrocode}

% \macro{\childdoc}
% The deprecated macro |\childdoc| is a legacy version of |\childdocmain|:
%    \begin{macrocode}
\newcommand{\childdoc}{\childdocmain}
%    \end{macrocode}

% \macro{\childdocredirect}
% The deprecated macro |\childdocredirect| is a legacy version
% of |\childdocforward| and |\childdocforwardprefix|:
%    \begin{macrocode}
\newcommand{\childdocredirect}[2][]
{
  \begingroup
    \if?#1?
      \def\childdoctmp{\childdocforward{#2}}
    \else
      \def\childdoctmp{\childdocforwardprefix{#1}{#2}}
    \fi
    \expandafter
  \endgroup
  \childdoctmp
}
%    \end{macrocode}

%\iffalse
%</package>
%\fi
%
\endinput
|\\
|\childdocforward[|\textit{main}|]{|\textit{dest}|}|\\
\end{tabular}
\end{center}
%
The argument \textit{dest} is the destination file
(without extension).
It should be the main file or one of the child files.
Note that further \textsf{childdoc} directives
such as |\childdocof| and |\childdocforward|
in the indicated file will be processed in this form.
The optional argument \textit{main}
passes on directly to the main file \textit{main}
while pretending to compile the child \textit{dest}.
This form behaves as if \textit{dest}
issues |\childdocof{|\textit{main}|}| right away,
and no further \textsf{childdoc} directives will be processed.

%%%%%%%%%%%%%%%%%%%%%%%%%%%%%%%%%%%%%%%%
\DescribeMacro{\...prefix}
In the alternative form |\childdocforwardprefix|,
%
\begin{center}
\begin{tabular}{l}
|% \iffalse
%
% childdoc.dtx Copyright (C) 2017-2018 Niklas Beisert
%
% This work may be distributed and/or modified under the
% conditions of the LaTeX Project Public License, either version 1.3
% of this license or (at your option) any later version.
% The latest version of this license is in
%   http://www.latex-project.org/lppl.txt
% and version 1.3 or later is part of all distributions of LaTeX
% version 2005/12/01 or later.
%
% This work has the LPPL maintenance status `maintained'.
%
% The Current Maintainer of this work is Niklas Beisert.
%
% This work consists of the files childdoc.dtx and childdoc.ins
% and the derived files childdoc.def and cdocsamp.tex with
% cdocsch1.tex, cdocsch2.tex, cdocsdrf.tex, cdocsfn1.tex, cdocsfn2.tex.
%
%<package>\ifdefined\childdocmain\endinput\fi
%<package>\ProvidesFile{childdoc.def}[2018/12/30 v2.0 child document driver]
%<samplemain>\ProvidesFile{cdocsamp.tex}[2018/12/30 v2.0 sample for childdoc]
%<*driver>
%\ProvidesFile{childdoc.drv}[2018/12/30 v2.0 childdoc reference manual file]
\PassOptionsToClass{10pt,a4paper}{article}
\documentclass{ltxdoc}

\usepackage[margin=35mm]{geometry}
\usepackage{hyperref}
\usepackage{hyperxmp}
\usepackage[usenames]{color}

\hypersetup{colorlinks=true}
\hypersetup{pdfstartview=FitH}
\hypersetup{pdfpagemode=UseNone}
\hypersetup{pdfsource={}}
\hypersetup{pdflang={en-UK}}
\hypersetup{pdfcopyright={Copyright 2017-2018 Niklas Beisert.
  This work may be distributed and/or modified under the
  conditions of the LaTeX Project Public License, either version 1.3
  of this license or (at your option) any later version.}}
\hypersetup{pdflicenseurl={http://www.latex-project.org/lppl.txt}}
\hypersetup{pdfcontactaddress={ETH Zurich, ITP, HIT K,
  Wolfgang-Pauli-Strasse 27}}
\hypersetup{pdfcontactpostcode={8093}}
\hypersetup{pdfcontactcity={Zurich}}
\hypersetup{pdfcontactcountry={Switzerland}}
\hypersetup{pdfcontactemail={nbeisert@itp.phys.ethz.ch}}
\hypersetup{pdfcontacturl={http://people.phys.ethz.ch/\xmptilde nbeisert/}}

\newcommand{\secref}[1]{\hyperref[#1]{section \ref*{#1}}}

\parskip1ex
\parindent0pt
\let\olditemize\itemize
\def\itemize{\olditemize\parskip0pt}

\begin{document}

\title{The \textsf{childdoc} Package}
\hypersetup{pdftitle={The childdoc Package}}
\author{Niklas Beisert\\[2ex]
  Institut f\"ur Theoretische Physik\\
  Eidgen\"ossische Technische Hochschule Z\"urich\\
  Wolfgang-Pauli-Strasse 27, 8093 Z\"urich, Switzerland\\[1ex]
  \href{mailto:nbeisert@itp.phys.ethz.ch}
  {\texttt{nbeisert@itp.phys.ethz.ch}}}
\hypersetup{pdfauthor={Niklas Beisert}}
\hypersetup{pdfsubject={Manual for the LaTeX2e Package childdoc}}
\date{30 December 2018, \textsf{v2.0}}
\maketitle

\begin{abstract}\noindent
\textsf{childdoc} is a \LaTeXe{} package
that enables the direct compilation
of document sections included by |\include|
to individual files.
\end{abstract}

\begingroup
\parskip0ex
\tableofcontents
\endgroup

%%%%%%%%%%%%%%%%%%%%%%%%%%%%%%%%%%%%%%%%%%%%%%%%%%%%%%%%%%%%%%%%%%%%%%%%%%%%%%%%
%%%%%%%%%%%%%%%%%%%%%%%%%%%%%%%%%%%%%%%%%%%%%%%%%%%%%%%%%%%%%%%%%%%%%%%%%%%%%%%%
\section{Introduction}

\LaTeX{} provides a mechanism to structure a large document (such as a book)
into a main file and several child files (containing the chapters)
using the |\include| command.
This mechanism is beneficial for documents
which span hundreds of pages in order to
make the source file(s) more manageable.
Moreover, compilation can be restricted to
selected child files by means of the |\includeonly| command.
The latter feature can be used to reduce the compilation time while editing
(this was significantly more useful in the earlier days of \LaTeX{})
or to generate a smaller document which is easier to navigate.
Another application of |\includeonly| is to generate
documents consisting of selected parts of the complete document.

However, there are a few drawbacks of the plain |\include| mechanism:
\begin{itemize}
\item
The child files cannot be compiled on their own,
they can only be compiled via the main file.
A naive editing environment
(such as a text editor with an option
to have the current file processed by \LaTeX)
may require one to switch to the main file before compiling;
attempting to compile the child file produces errors.
\item
The main file must be modified (each time)
to adjust the |\includeonly| command
to the present needs. This easily leaves the main file in a messy state.
\item
The generated document will always carry the filename
of the main document. This is inconvenient if
several child files are to be compiled and
to be kept for distribution.
\end{itemize}

The present package provides a simple interface
to make child files individually compilable by \LaTeX{}.
Compiling a child file then has the same effect as compiling
the main file with an |\includeonly| command
to select the appropriate child.
Moreover the generated document will carry the name of the child
rather than the main file.
This resolves all three above issues.

This feature is meant to make the editing of books,
thesis documents and lecture notes somewhat more convenient.
However, the package can also be used efficiently for
composing a series of documents (such as exercise sheets)
which are typically distributed individually.
It then assists the author in generating the individual documents
(potentially in different versions)
as well as a document containing the collected series.
Another application is in developing style files
or other kinds of included material
where compilation of the style file could redirect
to a sample or test file.

%%%%%%%%%%%%%%%%%%%%%%%%%%%%%%%%%%%%%%%%%%%%%%%%%%%%%%%%%%%%%%%%%%%%%%%%%%%%%%%%
%%%%%%%%%%%%%%%%%%%%%%%%%%%%%%%%%%%%%%%%%%%%%%%%%%%%%%%%%%%%%%%%%%%%%%%%%%%%%%%%
\section{Usage}

First of all, the package \textsf{childdoc} is \emph{not} a standard
\LaTeXe{} |.sty| style file! Therefore it needs to be invoked in
a non-standard way.

%%%%%%%%%%%%%%%%%%%%%%%%%%%%%%%%%%%%%%%%%%%%%%%%%%%%%%%%%%%%%%%%%%%%%%%%%%%%%%%%
\subsection{Included Files}
\label{sec:include}

%%%%%%%%%%%%%%%%%%%%%%%%%%%%%%%%%%%%%%%%
\DescribeMacro{\childdocmain}
To use the package, add the commands
\begin{center}
\begin{tabular}{l}
|\input{childdoc.def}|\\
|\childdocmain{}|\\
\end{tabular}
\end{center}
at the very top of the main \LaTeX{} file,
in particular \emph{before} the |\documentclass| statement!
The argument of |\childdocmain| should be left empty
(but it must be present).

%%%%%%%%%%%%%%%%%%%%%%%%%%%%%%%%%%%%%%%%
\DescribeMacro{\childdocof}
Furthermore, add the commands
\begin{center}
\begin{tabular}{l}
|\input{childdoc.def}|\\
|\childdocof{|\textit{main}|}|\\
\end{tabular}
\end{center}
at the top of every child file \textit{child}
which is included by |\include{|\textit{child}|}|
from within the main file
(or at least for those files to be compiled individually).
The argument \textit{main} must be the filename of the main file.

There are a couple of
considerations in setting up the main and child documents:

%%%%%%%%%%%%%%%%%%%%%%%%%%%%%%%%%%%%%%%%
\paragraph{Restrictions.}

Please note the following restrictions:
\begin{itemize}
\item
|\childdocmain| must be called with one argument \textit{main}
to ensure compatibility with earlier version of the package.
It must either be empty (|\childdocmain{}|)
or precisely match the filename of the main file in which it is specified.
See \secref{sec:detection} for further information.
\item
The filename \textit{main} must be specified without the |.tex| extension.
\item
The filename \textit{main} is case sensitive
(even in case-insensitive file systems)
due to internal string comparison.
\item
The argument \textit{main} should be fully expanded, it cannot be a macro.
\item
Subdirectories and special characters should be avoided in filenames.
\item
The command |\childdocmain{|\textit{main}|}| must be followed by a whitespace.
It should not be followed immediately by another command
or by a comment mark `|%|'.
This is because the \TeX{} parser reads the token immediately following
the argument of |\childdocmain| and puts it
at the beginning of every child section;
however, a white\-space is ignored.
\end{itemize}

%%%%%%%%%%%%%%%%%%%%%%%%%%%%%%%%%%%%%%%%
\paragraph{Content of Main File.}

It is advisable to place all content in the child files included by |\include|.
Any output contained in the main file will appear in all child documents
unless suppressed manually;
it cannot be suppressed automatically by the |\includeonly| directive
and thus should normally be avoided.
A method to include some content in the main file
by means of conditional processing is described in \secref{sec:conditional}.

%%%%%%%%%%%%%%%%%%%%%%%%%%%%%%%%%%%%%%%%
\paragraph{Page Numbering.}

When only a part of the document is compiled,
the appropriate numbering of pages
(as well as other status parameters)
is determined from the |.aux| files.
The latter contain information from previous passes.
However this information needs to propagate through
all intermediate child documents.
Therefore the page numbering in child documents may well
be inconsistent until the complete document is compiled at least once.

A useful (if unconventional) way to always ensure a consistent
page numbering is to restart the numbering in each child document
and denote the pages by `\textit{child}|.|\textit{page}'
where \textit{child} represents the chapter/section number of the child file.
This can be achieved by the command
|\numberwithin{page}{|\textit{child}|}|
of the \textsf{amsmath} package
where \textit{child} can be |chapter| or |section|
depending on the chosen structuring.
Alternatively, one can modify the macro |\thepage| appropriately
and reset the counter |page| at the start of each child file.

%%%%%%%%%%%%%%%%%%%%%%%%%%%%%%%%%%%%%%%%%%%%%%%%%%%%%%%%%%%%%%%%%%%%%%%%%%%%%%%%
\subsection{Conditional Processing}
\label{sec:conditional}

The package provides a mechanism to compile different versions
of a document. To customise the versions further some conditional processing
can come in handy to distinguish which version is being compiled.
The package provides two macros to describe the compilation context:

%%%%%%%%%%%%%%%%%%%%%%%%%%%%%%%%%%%%%%%%
\DescribeMacro{\ifchilddoc}
The conditional |\ifchilddoc| distinguishes between the compilation of
child documents and the main document:
%
\begin{center}
|\ifchilddoc |\textit{child-code}| |[|\||else |\textit{main-code}]| \||fi|
\end{center}

%%%%%%%%%%%%%%%%%%%%%%%%%%%%%%%%%%%%%%%%
\DescribeMacro{\childdocname}
\DescribeMacro{\childdocjob}
The macro |\childdocname| contains the filename (without extension)
of the main or child file being processed.
Note that |\childdocjob| will always contain the name of the main file.

%%%%%%%%%%%%%%%%%%%%%%%%%%%%%%%%%%%%%%%%
\paragraph{Title Page.}

Conditional processing can be used to include a title or banner page
in the main document when proper precautions are taken.
Importantly, the code in the main file should ensure that the page counter
(as well as other status parameters which are stored in the |.aux| files)
takes the same value after the conditional processing.
Otherwise the page numbers may take divergent values
depending on which part is compiled.

For example, a title page could be declared by:
%
\begin{center}
\begin{tabular}{l}
|\ifchilddoc\||else|\\
|\addtocounter{page}{-1}|\\
\textit{code for title page}\\
|\newpage|\\
|\||fi|
\end{tabular}
\end{center}
%
A banner page for the child documents can be generated by:
%
\begin{center}
\begin{tabular}{l}
|\ifchilddoc|\\
|\addtocounter{page}{-1}|\\
\textit{code for banner page}\\
|\newpage|\\
|\||fi|
\end{tabular}
\end{center}
%
Here one could write a message such as:
\begin{center}
|This is the part \childdocname{} of \childdocjob{}.|
\end{center}

%%%%%%%%%%%%%%%%%%%%%%%%%%%%%%%%%%%%%%%%%%%%%%%%%%%%%%%%%%%%%%%%%%%%%%%%%%%%%%%%
\subsection{Flags}
\label{sec:flags}

The package makes it easy to generate different versions
of the main or child documents.
To this end compilation flags can be defined
and assigned different default values.
They will be particularly useful in conjunction
with the forwarding mechanism described in \secref{sec:forward}.

For example, it may be useful to have a flag |\version|
which can be set to |draft| or |final|.
The document source will contain some conditional code
depending on the value of |\version|.
Suppose further, the flag should default to |final| for the main file
and to |draft| for child files
which is a natural assignment for editing the document.
This is achieved by placing the following code
in the preamble of the main document
(below the |\childdocmain| directive):
%
\begin{center}
\begin{tabular}{l}
|\ifchilddoc|\\
|\providecommand{\version}{draft}|\\
|\||else|\\
|\providecommand{\version}{final}|\\
|\||fi|
\end{tabular}
\end{center}
%
The definition by |\providecommand| makes sure
that previous definitions are not overwritten.
Further statements |\providecommand{\version}{...}|
can thus be added before the above code to override it.

For the main file, one might add a line
(between |\childdocmain| and the above block)
%
\begin{center}
|%\ifchilddoc\||else\providecommand{\version}{draft}\||fi|
\end{center}
%
which can be uncommented to produce a draft version.
Likewise one can add a line to the very top of a child file
(above the |\childdocof{|\textit{main}|}| directive)
%
\begin{center}
|%\providecommand{\version}{final}|
\end{center}
%
which can be uncommented to produce the final version of this child document.

%%%%%%%%%%%%%%%%%%%%%%%%%%%%%%%%%%%%%%%%%%%%%%%%%%%%%%%%%%%%%%%%%%%%%%%%%%%%%%%%
\subsection{Forwarding}
\label{sec:forward}

Different versions of the main or child documents
using compilation flags as described in \secref{sec:flags}
can be (permanently) stored in different files
for convenient compilation, viewing and distribution.
To this end, the package defines a command
to pass on compilation to a different file:

%%%%%%%%%%%%%%%%%%%%%%%%%%%%%%%%%%%%%%%%
\DescribeMacro{\childdocforward}
The command |\childdocforward| redirects processing to
another source file:
%
\begin{center}
\begin{tabular}{l}
|\input{childdoc.def}|\\
|\childdocforward[|\textit{main}|]{|\textit{dest}|}|\\
\end{tabular}
\end{center}
%
The argument \textit{dest} is the destination file
(without extension).
It should be the main file or one of the child files.
Note that further \textsf{childdoc} directives
such as |\childdocof| and |\childdocforward|
in the indicated file will be processed in this form.
The optional argument \textit{main}
passes on directly to the main file \textit{main}
while pretending to compile the child \textit{dest}.
This form behaves as if \textit{dest}
issues |\childdocof{|\textit{main}|}| right away,
and no further \textsf{childdoc} directives will be processed.

%%%%%%%%%%%%%%%%%%%%%%%%%%%%%%%%%%%%%%%%
\DescribeMacro{\...prefix}
In the alternative form |\childdocforwardprefix|,
%
\begin{center}
\begin{tabular}{l}
|\input{childdoc.def}|\\
|\childdocforwardprefix[|\textit{main}|]{|\textit{prefix}|}{|\textit{dest}|}|
\end{tabular}
\end{center}
%
the destination file is determined by a pattern
depending on the current file:
To make this work, the current file must be called
`{\textit{prefix}\hspace{0.2em}\textit{suffix}}'
with \textit{prefix} matching precisely the argument.
Processing is then passed on to the file
`{\textit{dest}\hspace{0.2em}\textit{suffix}}'.
Surely, the same effect is achieved by
directly specifying the
argument `{\textit{dest}\hspace{0.2em}\textit{suffix}}'
in the first form.
However, that requires to set up a different file
for each child. With the alternative form of the command
all these files can have exactly the same content
which simplifies setting them up and maintaining them.

For example, the following file |draft.tex|
with a compilation flag |\version| as described in \secref{sec:flags}
compiles the main document as a draft:
%
\begin{center}
\begin{tabular}{l}
|\def\version{draft}|\\
|\input{childdoc.def}|\\
|\childdocforward{|\textit{main}|}|
\end{tabular}
\end{center}
%
Likewise, the following files |final|\textit{nn}|.tex|
compile the final version of the child document
|child|\textit{nn}|.tex|:
%
\begin{center}
\begin{tabular}{l}
|\def\version{final}|\\
|\input{childdoc.def}|\\
|\childdocforwardprefix{final}{child}|
\end{tabular}
\end{center}
%

Note that when several versions of a main file and/or of each child file
are to be generated, it may be convenient to set up a |Makefile| or
shell script to automatise the process.

%%%%%%%%%%%%%%%%%%%%%%%%%%%%%%%%%%%%%%%%%%%%%%%%%%%%%%%%%%%%%%%%%%%%%%%%%%%%%%%%
\subsection{Command Line Processing}
\label{sec:commandline}

The effect of redirection files can also be achieved by invoking
the \LaTeX{} compiler with a more elaborate command line.
Most conveniently this should be done as part
of a shell script or a |Makefile|.

When using \textsf{childdoc} in the main file, the following
command lines effectively perform a redirection
(note that depending on the shell being used,
backslashes may have to be doubled: `|\|' $\to$ `|\\|'):
%
\begin{center}
|... -jobname "|\textit{target}|" |\\|"|[\textit{flags}]%
|\input{childdoc.def}\childdocforward[|\textit{main}|]{|\textit{dest}|}"|
\end{center}
%
Here \textit{target} is the name of the output file,
\textit{main} is the name of the main file
and \textit{dest} is the name of the main or child file to be processed
(all filenames without extensions).
The optional argument \textit{main} can be omitted
if \textit{main} matches \textit{dest}.
Optionally, compilation \textit{flags} can be defined via |\def| commands.
This command line makes the \TeX{} engine believe
it is compiling the file \textit{target}
whose content is specified as the latter parameter.
The provided code then forwards the processing to
\textit{main} or \textit{dest} as described in \secref{sec:forward}.

%%%%%%%%%%%%%%%%%%%%%%%%%%%%%%%%%%%%%%%%%%%%%%%%%%%%%%%%%%%%%%%%%%%%%%%%%%%%%%%%
\subsection{Include by Input}
\label{sec:input}

Including child documents by |\include| has some restrictions by design.
Most notably, the content of a child document always occupies
its own set of pages; pages cannot be shared between child documents.
Usually, this behaviour makes perfect sense
because each child document contain an essential part of the document.
However, in some situations it may be desirable to compose
a document from a collection of parts
without having mandatory page breaks between then.
For this case, the package
provides a mechanism to include parts
by |\input| which can also be processed individually.
However, by construction this mechanism
requires manual handling of the content to be output.

%%%%%%%%%%%%%%%%%%%%%%%%%%%%%%%%%%%%%%%%
\DescribeMacro{\ifchilddocmanual}
The main file should be prepared as usual, see \secref{sec:include}.
However, the document body must make a distinction
between processing of an individual part and of the main document, e.g.:
%
\begin{center}
\begin{tabular}{l}
|\ifchilddocmanual|\\
|\input{\childdocname}|\\
|\||else|\\
\textit{document body with }|\input{|\textit{part}|}|\\
|\||fi|
\end{tabular}
\end{center}
%
The conditional |\ifchilddocmanual| is true whenever
a part to be included by |\input| is being compiled,
and the name of the part is stored in |\childdocname|.

%%%%%%%%%%%%%%%%%%%%%%%%%%%%%%%%%%%%%%%%
\DescribeMacro{\childdocby}
Each part to be included by |\input| should start with:
%
\begin{center}
\begin{tabular}{l}
|\input{childdoc.def}|\\
|\childdocby{|\textit{main}|}|\\
\end{tabular}
\end{center}
%
The directive |\childdocby| is similar to |\childdocof|
described in \secref{sec:include},
but the subsequent selection of content must be done manually.
To that end, both |\ifchilddoc| and |\ifchilddocmanual|
will be true upon processing of a part,
and the name of the part is stored in |\childdocname|.
Note that |\jobname| will be set to the filename of the current part
so that each part receives an individual |.aux| file
that does not interfere with the |.aux| file(s) of the main document.
This behaviour can be altered by the alternative form
|\childdocby[*]{|\textit{main}|}| (with a non-empty optional argument)
which uses the |.aux| file of the main document
by setting |\jobname| to \textit{main}.

%%%%%%%%%%%%%%%%%%%%%%%%%%%%%%%%%%%%%%%%%%%%%%%%%%%%%%%%%%%%%%%%%%%%%%%%%%%%%%%%
\subsection{Driver Development}
\label{sec:driver}

The \textsf{childdoc} mechanism can also be use for the development
of definition files such as \LaTeX{} styles or classes.
This case differs from the above setup with multiple parts
included by |\include| in that no |\includeonly| should be invoked.
This can be achieved by starting the include file
(before |\ProvidesPackage|) with:
%
\begin{center}
\begin{tabular}{l}
|\input{childdoc.def}|\\
|\childdocforward{|\textit{main}|}|\\
\end{tabular}
\end{center}
%
or alternatively with:
%
\begin{center}
\begin{tabular}{l}
|\input{childdoc.def}|\\
|\childdocby{|\textit{main}|}|\\
\end{tabular}
\end{center}
%
Both forms have slightly different effects as described above.
The main file is prepared as usual, see \secref{sec:include}.

%%%%%%%%%%%%%%%%%%%%%%%%%%%%%%%%%%%%%%%%%%%%%%%%%%%%%%%%%%%%%%%%%%%%%%%%%%%%%%%%
\subsection{Legacy Detection}
\label{sec:detection}

The directive |\childdocmain| in the main file can detect
whether the complete document or merely a child is to be compiled
even without using the directive |\childdocof|.
This method is deprecated because it is less robust
and there is no compelling reason to use it;
it is merely provided for backward compatibility
and it may be removed in future versions.

If the detection mechanism is to be used,
it is mandatory to correctly specify
the filename of the main file as the argument of |\childdocmain|:
%
\begin{center}
\begin{tabular}{l}
|\input{childdoc.def}|\\
|\childdocmain{|\textit{main}|}|\\
\end{tabular}
\end{center}
%
If |\jobname| does not match the argument \textit{main} of |\childdocmain|,
it is assumed that |\jobname| points to the child file to be compiled.
When using |\childdocmain| with the main file specified as argument,
it suffices to start a child file
with just |\input{|\textit{main}|}|
without loading of the package and using |\childdocof|.
If instead all processing is done
with the appropriate \textsf{childdoc} directives,
the argument of \textit{main} of |\childdocmain| can be empty.

An alternative version of the command line processing described
in \secref{sec:commandline} using the detection mechanism reads:
%
\begin{center}
|... -jobname "|\textit{target}|" "|[\textit{flags}]%
[|\def\jobname{|\textit{dest}|}|]|\input{|\textit{main}|}"|
\end{center}

%%%%%%%%%%%%%%%%%%%%%%%%%%%%%%%%%%%%%%%%%%%%%%%%%%%%%%%%%%%%%%%%%%%%%%%%%%%%%%%%
\subsection{Manual Code}
\label{sec:manual}

In case one cannot be certain whether the definitions file |childdoc.def|
is installed on the target \TeX{} distribution
and one prefers not to ship it,
it is conceivable to paste a few relevant commands into the sources.

To that end, drop all statements |\input{childdoc.def}|
and perform the replacements as outlined below.
Instead of |\childdocmain{|\textit{main}|}| add the following code
to the top of the main file:
%
\begin{center}
\begin{tabular}{l}
|\||ifdefined\childdocname\endinput\||fi\newif\ifchilddoc|\\
|\edef\childdocname{\scantokens\expandafter{\jobname\noexpand}}|\\
|\def\childdocmain{|\textit{main}|}\||ifx\childdocmain\childdocname\||else|\\
|\childdoctrue\includeonly{\childdocname}\let\jobname\childdocmain\||fi|\\
\end{tabular}
\end{center}
%
Instead of |\childdocof{|\textit{main}|}| just include the main file
at the top of each child file:
%
\begin{center}
|\input{|\textit{main}|}|
\end{center}
%
A simple redirection |\childdocforward{|\textit{dest}|}| is achieved by:
%
\begin{center}
|\def\jobname{|\textit{dest}|}\input{\jobname}|
\end{center}
%
The redirection with prefix
|\childdocforwardprefix[|\textit{prefix}|]{|\textit{dest}|}|
is accomplished by:
%
\begin{center}
\begin{tabular}{l}
|{\edef\jobname{\scantokens\expandafter{\jobname\noexpand}}|\\
|\def\redirectjob |\textit{prefix}|#1~~~{\gdef\jobname{|\textit{dest}|#1}}|\\
|\expandafter\redirectjob\jobname~~~}\input{\jobname}|
\end{tabular}
\end{center}

In an alternative approach,
child documents can be compiled by a specific command line
without additional code or specific definitions:
%
\begin{center}
|... -jobname "|\textit{target}|" "|[\textit{flags}]%
|\includeonly{|\textit{dest}|}\input{|\textit{main}|}"|
\end{center}
%

%%%%%%%%%%%%%%%%%%%%%%%%%%%%%%%%%%%%%%%%%%%%%%%%%%%%%%%%%%%%%%%%%%%%%%%%%%%%%%%%
%%%%%%%%%%%%%%%%%%%%%%%%%%%%%%%%%%%%%%%%%%%%%%%%%%%%%%%%%%%%%%%%%%%%%%%%%%%%%%%%
\section{Information}

%%%%%%%%%%%%%%%%%%%%%%%%%%%%%%%%%%%%%%%%%%%%%%%%%%%%%%%%%%%%%%%%%%%%%%%%%%%%%%%%
\subsection{Copyright}

Copyright \copyright{} 2017--2018 Niklas Beisert

This work may be distributed and/or modified under the
conditions of the \LaTeX{} Project Public License, either version 1.3
of this license or (at your option) any later version.
The latest version of this license is in
  \url{http://www.latex-project.org/lppl.txt}
and version 1.3 or later is part of all distributions of \LaTeX{}
version 2005/12/01 or later.

This work has the LPPL maintenance status `maintained'.

The Current Maintainer of this work is Niklas Beisert.

This work consists of the files |README.txt|, |childdoc.ins| and |childdoc.dtx|
as well as the derived files |childdoc.def|, |cdocsamp.tex|
with |cdocsch1.tex|, |cdocsch2.tex|, |cdocspt3.tex|, |cdocspt4.tex|,
|cdocsdrf.tex|, |cdocsfn1.tex|, |cdocsfn2.tex|
as well as |childdoc.pdf|.

%%%%%%%%%%%%%%%%%%%%%%%%%%%%%%%%%%%%%%%%%%%%%%%%%%%%%%%%%%%%%%%%%%%%%%%%%%%%%%%%
\subsection{Files and Installation}

The package consists of the files:
%
\begin{center}
\begin{tabular}{ll}
    |README.txt|   & readme file \\
    |childdoc.ins| & installation file \\
    |childdoc.dtx| & source file \\
    |childdoc.def| & definition file \\
    |cdocsamp.tex| & sample main file \\
    |cdocsch1.tex| & sample include file \\
    |cdocsch2.tex| & sample include file \\
    |cdocspt3.tex| & sample part file \\
    |cdocspt4.tex| & sample part file \\
    |cdocsdrf.tex| & sample redirection file \\
    |cdocsfn1.tex| & sample redirection file \\
    |cdocsfn2.tex| & sample redirection file \\
    |childdoc.pdf| & manual
\end{tabular}
\end{center}
%
The distribution consists of the files
|README.txt|, |childdoc.ins| and |childdoc.dtx|.
%
\begin{itemize}
\item
Run (pdf)\LaTeX{} on |childdoc.dtx|
to compile the manual |childdoc.pdf| (this file).
\item
Run \LaTeX{} on |childdoc.ins| to create the definitions file |childdoc.def|
and the sample |cdocsamp.tex| with include files
|cdocsch1.tex|, |cdocsch2.tex|, |cdocspt3.tex|, |cdocspt4.tex|,
|cdocsdrf.tex|, |cdocsfn1.tex|, |cdocsfn2.tex|.
Then copy the file |childdoc.def| to an appropriate directory of your \LaTeX{}
distribution, e.g.\ \textit{texmf-root}|/tex/latex/childdoc|.
\end{itemize}

%%%%%%%%%%%%%%%%%%%%%%%%%%%%%%%%%%%%%%%%%%%%%%%%%%%%%%%%%%%%%%%%%%%%%%%%%%%%%%%%
\subsection{Related CTAN Packages}

There are several other packages which offer a similar functionality:
%
\begin{itemize}
\item
The packages
\href{http://ctan.org/pkg/docmute}{\textsf{docmute}},
\href{http://ctan.org/pkg/includex}{\textsf{includex}} and
\href{http://ctan.org/pkg/standalone}{\textsf{standalone}}
provide commands to include only the document body of
a child file thus allowing both files to be compiled individually.
\item
The packages \href{http://ctan.org/pkg/subdocs}{\textsf{subdocs}}
and \href{http://ctan.org/pkg/subfiles}{\textsf{subfiles}}
provide structures in which the main and child documents can be
encapsulated and allowing them to be compiled individually.
The inclusion mechanism is different from the conventional |\include|.
\item
The package \href{http://ctan.org/pkg/combine}{\textsf{combine}}
is an elaborate solution to combine several documents into one.
\end{itemize}
%
See also the CTAN topic \href{http://ctan.org/topic/subdocs}{\textsf{subdocs}}
for further related packages.
The present package differs from the above solutions in that
a document structure constructed with the conventional |\include| mechanism
just needs two extra commands at the top of every file
such that all constituent files can be compiled individually.

%%%%%%%%%%%%%%%%%%%%%%%%%%%%%%%%%%%%%%%%%%%%%%%%%%%%%%%%%%%%%%%%%%%%%%%%%%%%%%%%
%\subsection{Feature Suggestions}
%
%The following is a list of features which may be useful for future
%versions of this package:
%%
%\begin{itemize}
%\item
%\ldots
%\end{itemize}

%%%%%%%%%%%%%%%%%%%%%%%%%%%%%%%%%%%%%%%%%%%%%%%%%%%%%%%%%%%%%%%%%%%%%%%%%%%%%%%%
\subsection{Revision History}

%%%%%%%%%%%%%%%%%%%%%%%%%%%%%%%%%%%%%%%%
\paragraph{v2.0:} 2018/12/30

\begin{itemize}
\item
immediate forward processing
\item
added |\childdocby| mechanism
\item
manual restructured
\end{itemize}

%%%%%%%%%%%%%%%%%%%%%%%%%%%%%%%%%%%%%%%%
\paragraph{v1.6:} 2018/01/17

\begin{itemize}
\item
application for development of include files
\item
corrections to manual
\end{itemize}

%%%%%%%%%%%%%%%%%%%%%%%%%%%%%%%%%%%%%%%%
\paragraph{v1.5:} 2017/05/21

\begin{itemize}
\item
more complete structuring introduced
\item
|\childdocof| introduced
\item
|\childdoc| renamed to |\childdocmain|
\item
|\childredirect| renamed to |\childdocforward| and |\childdocforwardprefix|
and functionality expanded
\end{itemize}

%%%%%%%%%%%%%%%%%%%%%%%%%%%%%%%%%%%%%%%%
\paragraph{v1.0:} 2017/04/27

\begin{itemize}
\item
manual and install package
\item
first version published on CTAN
\end{itemize}

%%%%%%%%%%%%%%%%%%%%%%%%%%%%%%%%%%%%%%%%
\paragraph{v0.6:} 2017/04/26

\begin{itemize}
\item
redirection mechanism added
\end{itemize}

%%%%%%%%%%%%%%%%%%%%%%%%%%%%%%%%%%%%%%%%
\paragraph{v0.5:} 2017/04/26

\begin{itemize}
\item
functionality in definition file
\end{itemize}


%%%%%%%%%%%%%%%%%%%%%%%%%%%%%%%%%%%%%%%%%%%%%%%%%%%%%%%%%%%%%%%%%%%%%%%%%%%%%%%%
%%%%%%%%%%%%%%%%%%%%%%%%%%%%%%%%%%%%%%%%%%%%%%%%%%%%%%%%%%%%%%%%%%%%%%%%%%%%%%%%
%%%%%%%%%%%%%%%%%%%%%%%%%%%%%%%%%%%%%%%%%%%%%%%%%%%%%%%%%%%%%%%%%%%%%%%%%%%%%%%%
\appendix

\settowidth\MacroIndent{\rmfamily\scriptsize 000\ }

 \DocInput{childdoc.dtx}

\end{document}
%</driver>
% \fi
%
% %%%%%%%%%%%%%%%%%%%%%%%%%%%%%%%%%%%%%%%%%%%%%%%%%%%%%%%%%%%%%%%%%%%%%%%%%%%%%%
% %%%%%%%%%%%%%%%%%%%%%%%%%%%%%%%%%%%%%%%%%%%%%%%%%%%%%%%%%%%%%%%%%%%%%%%%%%%%%%
% \section{Sample}
%\iffalse
%<*samplemain>
%\fi
%
% The following presents a sample document
% with two chapters, two parts, a title page,
% a compile flag as well as three forwarding files to set the flag.
% It consists of eight |.tex| files:
% \begin{center}
% \begin{tabular}{ll}
% |cdocsamp.tex|&main file\\
% |cdocsch1.tex|&include file for chapter 1\\
% |cdocsch2.tex|&include file for chapter 2\\
% |cdocspt3.tex|&include file for part 3\\
% |cdocspt4.tex|&include file for part 4\\
% |cdocsdrf.tex|&forwarding file for main file in draft mode\\
% |cdocsfi1.tex|&forwarding file for final version of chapter 1\\
% |cdocsfi2.tex|&forwarding file for final version of chapter 2\\
% \end{tabular}
% \end{center}
% Each of the eight files can be compiled directly by the \LaTeX{} compiler.
%
% %%%%%%%%%%%%%%%%%%%%%%%%%%%%%%%%%%%%%%
% \paragraph{Main File.}
%
% The main file is called |cdocsamp.tex|.
%
% Load the \textsf{childdoc} definitions and
% declare the filename for the main document:
%    \begin{macrocode}
\input{childdoc.def}
\childdocmain{}
%    \end{macrocode}

% Optional override for |\version| flag:
%    \begin{macrocode}
%%\ifchilddoc\else\providecommand{\version}{draft}\fi
%    \end{macrocode}

% Define the default values for the |\version| flag
% (|final| for the main file and |draft| for childs):
%    \begin{macrocode}
\ifchilddoc
\providecommand{\version}{draft}
\else
\providecommand{\version}{final}
\fi
%    \end{macrocode}

% Load the standard document class:
%    \begin{macrocode}
\documentclass[12pt]{article}
%    \end{macrocode}

% Start the document body:
%    \begin{macrocode}
\begin{document}
%    \end{macrocode}

% Declare a title page.
% Print title, part of document being processed and version flag:
%    \begin{macrocode}
\addtocounter{page}{-1}
\begin{center}
{\LARGE\bfseries{}childdoc example\par}
\vspace{1cm}
\ifchilddoc
\ifchilddocmanual part\else chapter\fi:
`\childdocname' of `\childdocjob'\par
\else
main document: `\childdocjob'\par
\fi
version: \version\par
\end{center}
\newpage
%    \end{macrocode}

% Manually include selected file,
% otherwise process as usual:
%    \begin{macrocode}
\ifchilddocmanual
\section*{part `\childdocname'}
\input{\childdocname}
\else
%    \end{macrocode}

% Include the two chapters:
%    \begin{macrocode}
\include{cdocsch1}
\include{cdocsch2}
%    \end{macrocode}

% Include the two parts unless only chapters should be displayed:
%    \begin{macrocode}
\ifchilddoc\else
\section{part three}
\input{cdocspt3}
\section{part four}
\input{cdocspt4}
\fi
%    \end{macrocode}

% Process as usual until here:
%    \begin{macrocode}
\fi
%    \end{macrocode}

% End of document body:
%    \begin{macrocode}
\end{document}
%    \end{macrocode}
%\iffalse
%</samplemain>
%\fi
%
% %%%%%%%%%%%%%%%%%%%%%%%%%%%%%%%%%%%%%%
% \paragraph{Chapter Include Files.}
%
% The include files are called |cdocsch1.tex| and |cdocsch2.tex|.
%
%\iffalse
%<*samplechap1|samplechap2>
%\fi

% Optional override for |\version| flag:
%    \begin{macrocode}
%%\providecommand{\version}{final}
%    \end{macrocode}

% Include the main document:
%    \begin{macrocode}
\input{childdoc.def}
\childdocof{cdocsamp}
%    \end{macrocode}

%\iffalse
%</samplechap1|samplechap2>
%\fi
%
%\iffalse
%<*samplechap1>
%\fi
% Some text for chapter 1:
%    \begin{macrocode}
\section{one}
some text in chapter one
%    \end{macrocode}

%\iffalse
%</samplechap1>
%\fi
% Some text for chapter 2:
%\iffalse
%<*samplechap2>
%\fi
%    \begin{macrocode}
\section{two}
more text in chapter two
%    \end{macrocode}

%\iffalse
%</samplechap2>
%\fi
%
% %%%%%%%%%%%%%%%%%%%%%%%%%%%%%%%%%%%%%%
% \paragraph{Part Include Files.}
%
% The include files are called |cdocspt3.tex| and |cdocspt4.tex|.
%
%\iffalse
%<*samplepart3|samplepart4>
%\fi

% Optional override for |\version| flag:
%    \begin{macrocode}
%%\providecommand{\version}{final}
%    \end{macrocode}

% Include the main document:
%    \begin{macrocode}
\input{childdoc.def}
\childdocby{cdocsamp}
%    \end{macrocode}

%\iffalse
%</samplepart3|samplepart4>
%\fi
%
%\iffalse
%<*samplepart3>
%\fi
% Some text for part 3:
%    \begin{macrocode}
some text in part three
%    \end{macrocode}

%\iffalse
%</samplepart3>
%\fi
% Some text for part 4:
%\iffalse
%<*samplepart4>
%\fi
%    \begin{macrocode}
more text in part four
%    \end{macrocode}

%\iffalse
%</samplepart4>
%\fi
%
% %%%%%%%%%%%%%%%%%%%%%%%%%%%%%%%%%%%%%%
% \paragraph{Forwarding for a Complete Draft.}
%
% The following forwarding file |cdocsdrf.tex|
% compiles the main document in draft mode:
%\iffalse
%<*sampledraft>
%\fi
%    \begin{macrocode}
\def\version{draft}
\input{childdoc.def}
\childdocforward{cdocsamp}
%    \end{macrocode}

%\iffalse
%</sampledraft>
%\fi
%
% %%%%%%%%%%%%%%%%%%%%%%%%%%%%%%%%%%%%%%
% \paragraph{Forwarding for Final Version of the Chapters.}
%
% The following forwarding files |cdocsfn1.tex| and |cdocsfn2.tex|
% (with identical content)
% compile the final versions of the child documents
% |cdocsch1.tex| and |cdocsch2.tex|, respectively:
%\iffalse
%<*samplefinal>
%\fi
%    \begin{macrocode}
\def\version{final}
\input{childdoc.def}
\childdocforwardprefix[cdocsamp]{cdocsfn}{cdocsch}
%    \end{macrocode}

%\iffalse
%</samplefinal>
%\fi
%
% %%%%%%%%%%%%%%%%%%%%%%%%%%%%%%%%%%%%%%
% \paragraph{Command Line Processing.}
%
% The following three command lines generate the output files
% |cdocscld|, |cdocscl1| and |cdocscl2|
% which should be identical to
% |cdocsdrf|, |cdocsch1| and |cdocsfn2|, respectively:
% \begin{center}
% \begin{tabular}{l}
% |latex -jobname cdocscld \|\\
% |  "\def\version{draft}\input{childdoc.def}\childdocforward{cdocsamp}"|\\
% |latex -jobname cdocscl1 \|\\
% |  "\input{childdoc.def}\childdocforward[cdocsamp]{cdocsch1}"|\\
% |latex -jobname cdocscl2 \|\\
% |  "\def\version{final}\input{childdoc.def}\childdocforward{cdocsch2}"|
% \end{tabular}
% \end{center}
% Note that the trailing backslash on each first line
% merely continues the input to the second line
% (for convenient cut ant paste).
% Furthermore, the command |latex| can be replaced by any
% of its alternative versions such as |pdflatex|.
%
% %%%%%%%%%%%%%%%%%%%%%%%%%%%%%%%%%%%%%%%%%%%%%%%%%%%%%%%%%%%%%%%%%%%%%%%%%%%%%%
% %%%%%%%%%%%%%%%%%%%%%%%%%%%%%%%%%%%%%%%%%%%%%%%%%%%%%%%%%%%%%%%%%%%%%%%%%%%%%%
% \section{Implementation}
%\iffalse
%<*package>
%\fi
%
% This section describes the definitions file |childdoc.def|.

% The definitions cannot be loaded using |\usepackage| or |\RequirePackage|
% which has a mechanism to prevent loading a style file more than once.
% When loading the definitions by means of |\input|
% multiple instances have to be prevented manually:
%\iffalse
%This code needs to be before the `\ProvidesFile' directive
%which is defined at the beginning of this file.
%Therefore it is also placed there and commented out here.
%</package>
%<*discard>
%\fi
%    \begin{macrocode}
\ifdefined\childdocmain\endinput\fi
%    \end{macrocode}
%\iffalse
%</discard>
%<*package>
%\fi
%
% \macro{\ifchilddoc}
% \macro{\ifchilddocmanual}
% The conditional |\ifchilddoc| tells whether a
% child (true) or main (false) document is being compiled.
% The conditional |\ifchilddocmanual| tells whether
% the |\includeonly| mechanism is used (false) or
% the selection of child files must be performed manually (true).
% The definitions initialise to false:
%    \begin{macrocode}
\newif\ifchilddoc
\newif\ifchilddocmanual
%    \end{macrocode}

% \macro{\childdocname}
% \macro{\childdocjob}
% The macro |\childdocname| stores the name of the main document
% to be compiled. The macro |\childdocjob| stores the name of
% the document on which the \LaTeX{} compiler was originally invoked.
% The content of |\jobname| cannot be compared
% to filenames specified in the source due to different catcodes.
% The following code rescans |\jobname|, stores the result
% in |\childdocname| and saves a copy in |\childdocjob|:
%    \begin{macrocode}
\edef\childdocname{\scantokens\expandafter{\jobname\noexpand}}
\let\childdocjob\childdocname
%    \end{macrocode}

% \macro{\childdocdisable}
% The macro |\childdocdisable| prevents the main file
% from being processed more than once.
% At this stage, the main document command |\childdocmain|
% is assumed to be called once again where it should do nothing.
% Any subsequent call to it should prevent
% a secondary processing of the main document
% It overwrites the forwarding commands
% |\childdocof| and |\childdocforward|
% with empty macros to prevent further inclusions of the main document:
%    \begin{macrocode}
\newcommand{\childdocdisable}
{
  \renewcommand{\childdocmain}[1]{\renewcommand{\childdocmain}[1]{\endinput}}
  \renewcommand{\childdocof}[1]{}
  \renewcommand{\childdocby}[2][]{}
  \renewcommand{\childdocforward}[2][]{}
  \renewcommand{\childdocdisable}{}
}
%    \end{macrocode}

% \macro{\childdocmain}
% The macro |\childdocmain| is to be called at the top of the main file
% with nothing or the main filename (without extension) as argument.
% First, it breaks loops.
% If the argument is not empty and does not match |\childdocname|
% (which is set by the first inclusion of |childdoc.def|),
% |\ifchilddoc| is set to true, |\includeonly| is applied to the child file
% and |\jobname| is set to the main file
% (for proper handling of |.aux| files):
%    \begin{macrocode}
\newcommand{\childdocmain}[1]
{
  \childdocdisable\childdocmain{}
  \if?#1?\else
    \begingroup
      \def\childdoctmp{#1}
      \ifx\childdoctmp\childdocname
        \def\childdoctmp{}
      \else
        \def\childdoctmp
        {
          \childdoctrue
          \includeonly{\childdocname}
          \def\childdocjob{#1}
          \def\jobname{#1}
        }
      \fi
      \expandafter
    \endgroup
    \childdoctmp
  \fi
}
%    \end{macrocode}

% \macro{\childdocof}
% The command |\childdocof| redirects
% compilation to the main file |#1|.
%    \begin{macrocode}
\newcommand{\childdocof}[1]
{
  \childdocdisable
  \childdoctrue
  \includeonly{\childdocname}
  \def\jobname{#1}
  \def\childdocjob{#1}
  \input{#1}
}
%    \end{macrocode}

% \macro{\childdocby}
% The command |\childdocby| ....
%    \begin{macrocode}
\newcommand{\childdocby}[2][]
{
  \childdocdisable
  \childdoctrue
  \childdocmanualtrue
  \if?#1?\else
    \def\jobname{#2}
  \fi
  \def\childdocjob{#2}
  \input{#2}
  \endinput
}
%    \end{macrocode}

% \macro{\childdocforward}
% The command |\childdocforward| redirects
% compilation to the main file or
% (if the optional argument is given) a child file.
% Parameters are set as if the main file
% or a child file starting with |\childdocof| was compiled.
% Then compilation is handed over to the main file:
%    \begin{macrocode}
\newcommand{\childdocforward}[2][]
{
  \begingroup
    \if?#1?
      \def\childdoctmp
      {
        \def\childdocname{#2}
        \def\childdocjob{#2}
        \def\jobname{#2}
        \input{#2}
        \endinput
      }
    \else
      \def\childdoctmp
      {
        \childdocdisable
        \def\childdocname{#2}
        \childdoctrue
        \includeonly{#2}
        \def\childdocjob{#1}
        \def\jobname{#1}
        \input{#1}
        \endinput
      }
    \fi
    \expandafter
  \endgroup
  \childdoctmp
}
%    \end{macrocode}

% \macro{\childdocforwardprefix}
% The command |\childdocforwardprefix| redirects
% compilation to the main or a child file by means of a pattern.
% The prefix |#1| in the current filename is replaced by |#2|
% and the suffix of the current filename is kept
% (it is assumed that the filename does not contain the substring `|~~~|'
% which is used as a delimiter).
% Compilation is handed over to the new file by |\childdocforward|:
%    \begin{macrocode}
\newcommand{\childdocforwardprefix}[3][]
{
  \begingroup
    \def\childdocextract #2##1~~~{\def\childdoctmp{\childdocforward[#1]{#3##1}}}
    \expandafter\childdocextract\childdocname~~~
    \expandafter
  \endgroup
  \childdoctmp
}
%    \end{macrocode}

% \macro{\childdoc}
% The deprecated macro |\childdoc| is a legacy version of |\childdocmain|:
%    \begin{macrocode}
\newcommand{\childdoc}{\childdocmain}
%    \end{macrocode}

% \macro{\childdocredirect}
% The deprecated macro |\childdocredirect| is a legacy version
% of |\childdocforward| and |\childdocforwardprefix|:
%    \begin{macrocode}
\newcommand{\childdocredirect}[2][]
{
  \begingroup
    \if?#1?
      \def\childdoctmp{\childdocforward{#2}}
    \else
      \def\childdoctmp{\childdocforwardprefix{#1}{#2}}
    \fi
    \expandafter
  \endgroup
  \childdoctmp
}
%    \end{macrocode}

%\iffalse
%</package>
%\fi
%
\endinput
|\\
|\childdocforwardprefix[|\textit{main}|]{|\textit{prefix}|}{|\textit{dest}|}|
\end{tabular}
\end{center}
%
the destination file is determined by a pattern
depending on the current file:
To make this work, the current file must be called
`{\textit{prefix}\hspace{0.2em}\textit{suffix}}'
with \textit{prefix} matching precisely the argument.
Processing is then passed on to the file
`{\textit{dest}\hspace{0.2em}\textit{suffix}}'.
Surely, the same effect is achieved by
directly specifying the
argument `{\textit{dest}\hspace{0.2em}\textit{suffix}}'
in the first form.
However, that requires to set up a different file
for each child. With the alternative form of the command
all these files can have exactly the same content
which simplifies setting them up and maintaining them.

For example, the following file |draft.tex|
with a compilation flag |\version| as described in \secref{sec:flags}
compiles the main document as a draft:
%
\begin{center}
\begin{tabular}{l}
|\def\version{draft}|\\
|% \iffalse
%
% childdoc.dtx Copyright (C) 2017-2018 Niklas Beisert
%
% This work may be distributed and/or modified under the
% conditions of the LaTeX Project Public License, either version 1.3
% of this license or (at your option) any later version.
% The latest version of this license is in
%   http://www.latex-project.org/lppl.txt
% and version 1.3 or later is part of all distributions of LaTeX
% version 2005/12/01 or later.
%
% This work has the LPPL maintenance status `maintained'.
%
% The Current Maintainer of this work is Niklas Beisert.
%
% This work consists of the files childdoc.dtx and childdoc.ins
% and the derived files childdoc.def and cdocsamp.tex with
% cdocsch1.tex, cdocsch2.tex, cdocsdrf.tex, cdocsfn1.tex, cdocsfn2.tex.
%
%<package>\ifdefined\childdocmain\endinput\fi
%<package>\ProvidesFile{childdoc.def}[2018/12/30 v2.0 child document driver]
%<samplemain>\ProvidesFile{cdocsamp.tex}[2018/12/30 v2.0 sample for childdoc]
%<*driver>
%\ProvidesFile{childdoc.drv}[2018/12/30 v2.0 childdoc reference manual file]
\PassOptionsToClass{10pt,a4paper}{article}
\documentclass{ltxdoc}

\usepackage[margin=35mm]{geometry}
\usepackage{hyperref}
\usepackage{hyperxmp}
\usepackage[usenames]{color}

\hypersetup{colorlinks=true}
\hypersetup{pdfstartview=FitH}
\hypersetup{pdfpagemode=UseNone}
\hypersetup{pdfsource={}}
\hypersetup{pdflang={en-UK}}
\hypersetup{pdfcopyright={Copyright 2017-2018 Niklas Beisert.
  This work may be distributed and/or modified under the
  conditions of the LaTeX Project Public License, either version 1.3
  of this license or (at your option) any later version.}}
\hypersetup{pdflicenseurl={http://www.latex-project.org/lppl.txt}}
\hypersetup{pdfcontactaddress={ETH Zurich, ITP, HIT K,
  Wolfgang-Pauli-Strasse 27}}
\hypersetup{pdfcontactpostcode={8093}}
\hypersetup{pdfcontactcity={Zurich}}
\hypersetup{pdfcontactcountry={Switzerland}}
\hypersetup{pdfcontactemail={nbeisert@itp.phys.ethz.ch}}
\hypersetup{pdfcontacturl={http://people.phys.ethz.ch/\xmptilde nbeisert/}}

\newcommand{\secref}[1]{\hyperref[#1]{section \ref*{#1}}}

\parskip1ex
\parindent0pt
\let\olditemize\itemize
\def\itemize{\olditemize\parskip0pt}

\begin{document}

\title{The \textsf{childdoc} Package}
\hypersetup{pdftitle={The childdoc Package}}
\author{Niklas Beisert\\[2ex]
  Institut f\"ur Theoretische Physik\\
  Eidgen\"ossische Technische Hochschule Z\"urich\\
  Wolfgang-Pauli-Strasse 27, 8093 Z\"urich, Switzerland\\[1ex]
  \href{mailto:nbeisert@itp.phys.ethz.ch}
  {\texttt{nbeisert@itp.phys.ethz.ch}}}
\hypersetup{pdfauthor={Niklas Beisert}}
\hypersetup{pdfsubject={Manual for the LaTeX2e Package childdoc}}
\date{30 December 2018, \textsf{v2.0}}
\maketitle

\begin{abstract}\noindent
\textsf{childdoc} is a \LaTeXe{} package
that enables the direct compilation
of document sections included by |\include|
to individual files.
\end{abstract}

\begingroup
\parskip0ex
\tableofcontents
\endgroup

%%%%%%%%%%%%%%%%%%%%%%%%%%%%%%%%%%%%%%%%%%%%%%%%%%%%%%%%%%%%%%%%%%%%%%%%%%%%%%%%
%%%%%%%%%%%%%%%%%%%%%%%%%%%%%%%%%%%%%%%%%%%%%%%%%%%%%%%%%%%%%%%%%%%%%%%%%%%%%%%%
\section{Introduction}

\LaTeX{} provides a mechanism to structure a large document (such as a book)
into a main file and several child files (containing the chapters)
using the |\include| command.
This mechanism is beneficial for documents
which span hundreds of pages in order to
make the source file(s) more manageable.
Moreover, compilation can be restricted to
selected child files by means of the |\includeonly| command.
The latter feature can be used to reduce the compilation time while editing
(this was significantly more useful in the earlier days of \LaTeX{})
or to generate a smaller document which is easier to navigate.
Another application of |\includeonly| is to generate
documents consisting of selected parts of the complete document.

However, there are a few drawbacks of the plain |\include| mechanism:
\begin{itemize}
\item
The child files cannot be compiled on their own,
they can only be compiled via the main file.
A naive editing environment
(such as a text editor with an option
to have the current file processed by \LaTeX)
may require one to switch to the main file before compiling;
attempting to compile the child file produces errors.
\item
The main file must be modified (each time)
to adjust the |\includeonly| command
to the present needs. This easily leaves the main file in a messy state.
\item
The generated document will always carry the filename
of the main document. This is inconvenient if
several child files are to be compiled and
to be kept for distribution.
\end{itemize}

The present package provides a simple interface
to make child files individually compilable by \LaTeX{}.
Compiling a child file then has the same effect as compiling
the main file with an |\includeonly| command
to select the appropriate child.
Moreover the generated document will carry the name of the child
rather than the main file.
This resolves all three above issues.

This feature is meant to make the editing of books,
thesis documents and lecture notes somewhat more convenient.
However, the package can also be used efficiently for
composing a series of documents (such as exercise sheets)
which are typically distributed individually.
It then assists the author in generating the individual documents
(potentially in different versions)
as well as a document containing the collected series.
Another application is in developing style files
or other kinds of included material
where compilation of the style file could redirect
to a sample or test file.

%%%%%%%%%%%%%%%%%%%%%%%%%%%%%%%%%%%%%%%%%%%%%%%%%%%%%%%%%%%%%%%%%%%%%%%%%%%%%%%%
%%%%%%%%%%%%%%%%%%%%%%%%%%%%%%%%%%%%%%%%%%%%%%%%%%%%%%%%%%%%%%%%%%%%%%%%%%%%%%%%
\section{Usage}

First of all, the package \textsf{childdoc} is \emph{not} a standard
\LaTeXe{} |.sty| style file! Therefore it needs to be invoked in
a non-standard way.

%%%%%%%%%%%%%%%%%%%%%%%%%%%%%%%%%%%%%%%%%%%%%%%%%%%%%%%%%%%%%%%%%%%%%%%%%%%%%%%%
\subsection{Included Files}
\label{sec:include}

%%%%%%%%%%%%%%%%%%%%%%%%%%%%%%%%%%%%%%%%
\DescribeMacro{\childdocmain}
To use the package, add the commands
\begin{center}
\begin{tabular}{l}
|\input{childdoc.def}|\\
|\childdocmain{}|\\
\end{tabular}
\end{center}
at the very top of the main \LaTeX{} file,
in particular \emph{before} the |\documentclass| statement!
The argument of |\childdocmain| should be left empty
(but it must be present).

%%%%%%%%%%%%%%%%%%%%%%%%%%%%%%%%%%%%%%%%
\DescribeMacro{\childdocof}
Furthermore, add the commands
\begin{center}
\begin{tabular}{l}
|\input{childdoc.def}|\\
|\childdocof{|\textit{main}|}|\\
\end{tabular}
\end{center}
at the top of every child file \textit{child}
which is included by |\include{|\textit{child}|}|
from within the main file
(or at least for those files to be compiled individually).
The argument \textit{main} must be the filename of the main file.

There are a couple of
considerations in setting up the main and child documents:

%%%%%%%%%%%%%%%%%%%%%%%%%%%%%%%%%%%%%%%%
\paragraph{Restrictions.}

Please note the following restrictions:
\begin{itemize}
\item
|\childdocmain| must be called with one argument \textit{main}
to ensure compatibility with earlier version of the package.
It must either be empty (|\childdocmain{}|)
or precisely match the filename of the main file in which it is specified.
See \secref{sec:detection} for further information.
\item
The filename \textit{main} must be specified without the |.tex| extension.
\item
The filename \textit{main} is case sensitive
(even in case-insensitive file systems)
due to internal string comparison.
\item
The argument \textit{main} should be fully expanded, it cannot be a macro.
\item
Subdirectories and special characters should be avoided in filenames.
\item
The command |\childdocmain{|\textit{main}|}| must be followed by a whitespace.
It should not be followed immediately by another command
or by a comment mark `|%|'.
This is because the \TeX{} parser reads the token immediately following
the argument of |\childdocmain| and puts it
at the beginning of every child section;
however, a white\-space is ignored.
\end{itemize}

%%%%%%%%%%%%%%%%%%%%%%%%%%%%%%%%%%%%%%%%
\paragraph{Content of Main File.}

It is advisable to place all content in the child files included by |\include|.
Any output contained in the main file will appear in all child documents
unless suppressed manually;
it cannot be suppressed automatically by the |\includeonly| directive
and thus should normally be avoided.
A method to include some content in the main file
by means of conditional processing is described in \secref{sec:conditional}.

%%%%%%%%%%%%%%%%%%%%%%%%%%%%%%%%%%%%%%%%
\paragraph{Page Numbering.}

When only a part of the document is compiled,
the appropriate numbering of pages
(as well as other status parameters)
is determined from the |.aux| files.
The latter contain information from previous passes.
However this information needs to propagate through
all intermediate child documents.
Therefore the page numbering in child documents may well
be inconsistent until the complete document is compiled at least once.

A useful (if unconventional) way to always ensure a consistent
page numbering is to restart the numbering in each child document
and denote the pages by `\textit{child}|.|\textit{page}'
where \textit{child} represents the chapter/section number of the child file.
This can be achieved by the command
|\numberwithin{page}{|\textit{child}|}|
of the \textsf{amsmath} package
where \textit{child} can be |chapter| or |section|
depending on the chosen structuring.
Alternatively, one can modify the macro |\thepage| appropriately
and reset the counter |page| at the start of each child file.

%%%%%%%%%%%%%%%%%%%%%%%%%%%%%%%%%%%%%%%%%%%%%%%%%%%%%%%%%%%%%%%%%%%%%%%%%%%%%%%%
\subsection{Conditional Processing}
\label{sec:conditional}

The package provides a mechanism to compile different versions
of a document. To customise the versions further some conditional processing
can come in handy to distinguish which version is being compiled.
The package provides two macros to describe the compilation context:

%%%%%%%%%%%%%%%%%%%%%%%%%%%%%%%%%%%%%%%%
\DescribeMacro{\ifchilddoc}
The conditional |\ifchilddoc| distinguishes between the compilation of
child documents and the main document:
%
\begin{center}
|\ifchilddoc |\textit{child-code}| |[|\||else |\textit{main-code}]| \||fi|
\end{center}

%%%%%%%%%%%%%%%%%%%%%%%%%%%%%%%%%%%%%%%%
\DescribeMacro{\childdocname}
\DescribeMacro{\childdocjob}
The macro |\childdocname| contains the filename (without extension)
of the main or child file being processed.
Note that |\childdocjob| will always contain the name of the main file.

%%%%%%%%%%%%%%%%%%%%%%%%%%%%%%%%%%%%%%%%
\paragraph{Title Page.}

Conditional processing can be used to include a title or banner page
in the main document when proper precautions are taken.
Importantly, the code in the main file should ensure that the page counter
(as well as other status parameters which are stored in the |.aux| files)
takes the same value after the conditional processing.
Otherwise the page numbers may take divergent values
depending on which part is compiled.

For example, a title page could be declared by:
%
\begin{center}
\begin{tabular}{l}
|\ifchilddoc\||else|\\
|\addtocounter{page}{-1}|\\
\textit{code for title page}\\
|\newpage|\\
|\||fi|
\end{tabular}
\end{center}
%
A banner page for the child documents can be generated by:
%
\begin{center}
\begin{tabular}{l}
|\ifchilddoc|\\
|\addtocounter{page}{-1}|\\
\textit{code for banner page}\\
|\newpage|\\
|\||fi|
\end{tabular}
\end{center}
%
Here one could write a message such as:
\begin{center}
|This is the part \childdocname{} of \childdocjob{}.|
\end{center}

%%%%%%%%%%%%%%%%%%%%%%%%%%%%%%%%%%%%%%%%%%%%%%%%%%%%%%%%%%%%%%%%%%%%%%%%%%%%%%%%
\subsection{Flags}
\label{sec:flags}

The package makes it easy to generate different versions
of the main or child documents.
To this end compilation flags can be defined
and assigned different default values.
They will be particularly useful in conjunction
with the forwarding mechanism described in \secref{sec:forward}.

For example, it may be useful to have a flag |\version|
which can be set to |draft| or |final|.
The document source will contain some conditional code
depending on the value of |\version|.
Suppose further, the flag should default to |final| for the main file
and to |draft| for child files
which is a natural assignment for editing the document.
This is achieved by placing the following code
in the preamble of the main document
(below the |\childdocmain| directive):
%
\begin{center}
\begin{tabular}{l}
|\ifchilddoc|\\
|\providecommand{\version}{draft}|\\
|\||else|\\
|\providecommand{\version}{final}|\\
|\||fi|
\end{tabular}
\end{center}
%
The definition by |\providecommand| makes sure
that previous definitions are not overwritten.
Further statements |\providecommand{\version}{...}|
can thus be added before the above code to override it.

For the main file, one might add a line
(between |\childdocmain| and the above block)
%
\begin{center}
|%\ifchilddoc\||else\providecommand{\version}{draft}\||fi|
\end{center}
%
which can be uncommented to produce a draft version.
Likewise one can add a line to the very top of a child file
(above the |\childdocof{|\textit{main}|}| directive)
%
\begin{center}
|%\providecommand{\version}{final}|
\end{center}
%
which can be uncommented to produce the final version of this child document.

%%%%%%%%%%%%%%%%%%%%%%%%%%%%%%%%%%%%%%%%%%%%%%%%%%%%%%%%%%%%%%%%%%%%%%%%%%%%%%%%
\subsection{Forwarding}
\label{sec:forward}

Different versions of the main or child documents
using compilation flags as described in \secref{sec:flags}
can be (permanently) stored in different files
for convenient compilation, viewing and distribution.
To this end, the package defines a command
to pass on compilation to a different file:

%%%%%%%%%%%%%%%%%%%%%%%%%%%%%%%%%%%%%%%%
\DescribeMacro{\childdocforward}
The command |\childdocforward| redirects processing to
another source file:
%
\begin{center}
\begin{tabular}{l}
|\input{childdoc.def}|\\
|\childdocforward[|\textit{main}|]{|\textit{dest}|}|\\
\end{tabular}
\end{center}
%
The argument \textit{dest} is the destination file
(without extension).
It should be the main file or one of the child files.
Note that further \textsf{childdoc} directives
such as |\childdocof| and |\childdocforward|
in the indicated file will be processed in this form.
The optional argument \textit{main}
passes on directly to the main file \textit{main}
while pretending to compile the child \textit{dest}.
This form behaves as if \textit{dest}
issues |\childdocof{|\textit{main}|}| right away,
and no further \textsf{childdoc} directives will be processed.

%%%%%%%%%%%%%%%%%%%%%%%%%%%%%%%%%%%%%%%%
\DescribeMacro{\...prefix}
In the alternative form |\childdocforwardprefix|,
%
\begin{center}
\begin{tabular}{l}
|\input{childdoc.def}|\\
|\childdocforwardprefix[|\textit{main}|]{|\textit{prefix}|}{|\textit{dest}|}|
\end{tabular}
\end{center}
%
the destination file is determined by a pattern
depending on the current file:
To make this work, the current file must be called
`{\textit{prefix}\hspace{0.2em}\textit{suffix}}'
with \textit{prefix} matching precisely the argument.
Processing is then passed on to the file
`{\textit{dest}\hspace{0.2em}\textit{suffix}}'.
Surely, the same effect is achieved by
directly specifying the
argument `{\textit{dest}\hspace{0.2em}\textit{suffix}}'
in the first form.
However, that requires to set up a different file
for each child. With the alternative form of the command
all these files can have exactly the same content
which simplifies setting them up and maintaining them.

For example, the following file |draft.tex|
with a compilation flag |\version| as described in \secref{sec:flags}
compiles the main document as a draft:
%
\begin{center}
\begin{tabular}{l}
|\def\version{draft}|\\
|\input{childdoc.def}|\\
|\childdocforward{|\textit{main}|}|
\end{tabular}
\end{center}
%
Likewise, the following files |final|\textit{nn}|.tex|
compile the final version of the child document
|child|\textit{nn}|.tex|:
%
\begin{center}
\begin{tabular}{l}
|\def\version{final}|\\
|\input{childdoc.def}|\\
|\childdocforwardprefix{final}{child}|
\end{tabular}
\end{center}
%

Note that when several versions of a main file and/or of each child file
are to be generated, it may be convenient to set up a |Makefile| or
shell script to automatise the process.

%%%%%%%%%%%%%%%%%%%%%%%%%%%%%%%%%%%%%%%%%%%%%%%%%%%%%%%%%%%%%%%%%%%%%%%%%%%%%%%%
\subsection{Command Line Processing}
\label{sec:commandline}

The effect of redirection files can also be achieved by invoking
the \LaTeX{} compiler with a more elaborate command line.
Most conveniently this should be done as part
of a shell script or a |Makefile|.

When using \textsf{childdoc} in the main file, the following
command lines effectively perform a redirection
(note that depending on the shell being used,
backslashes may have to be doubled: `|\|' $\to$ `|\\|'):
%
\begin{center}
|... -jobname "|\textit{target}|" |\\|"|[\textit{flags}]%
|\input{childdoc.def}\childdocforward[|\textit{main}|]{|\textit{dest}|}"|
\end{center}
%
Here \textit{target} is the name of the output file,
\textit{main} is the name of the main file
and \textit{dest} is the name of the main or child file to be processed
(all filenames without extensions).
The optional argument \textit{main} can be omitted
if \textit{main} matches \textit{dest}.
Optionally, compilation \textit{flags} can be defined via |\def| commands.
This command line makes the \TeX{} engine believe
it is compiling the file \textit{target}
whose content is specified as the latter parameter.
The provided code then forwards the processing to
\textit{main} or \textit{dest} as described in \secref{sec:forward}.

%%%%%%%%%%%%%%%%%%%%%%%%%%%%%%%%%%%%%%%%%%%%%%%%%%%%%%%%%%%%%%%%%%%%%%%%%%%%%%%%
\subsection{Include by Input}
\label{sec:input}

Including child documents by |\include| has some restrictions by design.
Most notably, the content of a child document always occupies
its own set of pages; pages cannot be shared between child documents.
Usually, this behaviour makes perfect sense
because each child document contain an essential part of the document.
However, in some situations it may be desirable to compose
a document from a collection of parts
without having mandatory page breaks between then.
For this case, the package
provides a mechanism to include parts
by |\input| which can also be processed individually.
However, by construction this mechanism
requires manual handling of the content to be output.

%%%%%%%%%%%%%%%%%%%%%%%%%%%%%%%%%%%%%%%%
\DescribeMacro{\ifchilddocmanual}
The main file should be prepared as usual, see \secref{sec:include}.
However, the document body must make a distinction
between processing of an individual part and of the main document, e.g.:
%
\begin{center}
\begin{tabular}{l}
|\ifchilddocmanual|\\
|\input{\childdocname}|\\
|\||else|\\
\textit{document body with }|\input{|\textit{part}|}|\\
|\||fi|
\end{tabular}
\end{center}
%
The conditional |\ifchilddocmanual| is true whenever
a part to be included by |\input| is being compiled,
and the name of the part is stored in |\childdocname|.

%%%%%%%%%%%%%%%%%%%%%%%%%%%%%%%%%%%%%%%%
\DescribeMacro{\childdocby}
Each part to be included by |\input| should start with:
%
\begin{center}
\begin{tabular}{l}
|\input{childdoc.def}|\\
|\childdocby{|\textit{main}|}|\\
\end{tabular}
\end{center}
%
The directive |\childdocby| is similar to |\childdocof|
described in \secref{sec:include},
but the subsequent selection of content must be done manually.
To that end, both |\ifchilddoc| and |\ifchilddocmanual|
will be true upon processing of a part,
and the name of the part is stored in |\childdocname|.
Note that |\jobname| will be set to the filename of the current part
so that each part receives an individual |.aux| file
that does not interfere with the |.aux| file(s) of the main document.
This behaviour can be altered by the alternative form
|\childdocby[*]{|\textit{main}|}| (with a non-empty optional argument)
which uses the |.aux| file of the main document
by setting |\jobname| to \textit{main}.

%%%%%%%%%%%%%%%%%%%%%%%%%%%%%%%%%%%%%%%%%%%%%%%%%%%%%%%%%%%%%%%%%%%%%%%%%%%%%%%%
\subsection{Driver Development}
\label{sec:driver}

The \textsf{childdoc} mechanism can also be use for the development
of definition files such as \LaTeX{} styles or classes.
This case differs from the above setup with multiple parts
included by |\include| in that no |\includeonly| should be invoked.
This can be achieved by starting the include file
(before |\ProvidesPackage|) with:
%
\begin{center}
\begin{tabular}{l}
|\input{childdoc.def}|\\
|\childdocforward{|\textit{main}|}|\\
\end{tabular}
\end{center}
%
or alternatively with:
%
\begin{center}
\begin{tabular}{l}
|\input{childdoc.def}|\\
|\childdocby{|\textit{main}|}|\\
\end{tabular}
\end{center}
%
Both forms have slightly different effects as described above.
The main file is prepared as usual, see \secref{sec:include}.

%%%%%%%%%%%%%%%%%%%%%%%%%%%%%%%%%%%%%%%%%%%%%%%%%%%%%%%%%%%%%%%%%%%%%%%%%%%%%%%%
\subsection{Legacy Detection}
\label{sec:detection}

The directive |\childdocmain| in the main file can detect
whether the complete document or merely a child is to be compiled
even without using the directive |\childdocof|.
This method is deprecated because it is less robust
and there is no compelling reason to use it;
it is merely provided for backward compatibility
and it may be removed in future versions.

If the detection mechanism is to be used,
it is mandatory to correctly specify
the filename of the main file as the argument of |\childdocmain|:
%
\begin{center}
\begin{tabular}{l}
|\input{childdoc.def}|\\
|\childdocmain{|\textit{main}|}|\\
\end{tabular}
\end{center}
%
If |\jobname| does not match the argument \textit{main} of |\childdocmain|,
it is assumed that |\jobname| points to the child file to be compiled.
When using |\childdocmain| with the main file specified as argument,
it suffices to start a child file
with just |\input{|\textit{main}|}|
without loading of the package and using |\childdocof|.
If instead all processing is done
with the appropriate \textsf{childdoc} directives,
the argument of \textit{main} of |\childdocmain| can be empty.

An alternative version of the command line processing described
in \secref{sec:commandline} using the detection mechanism reads:
%
\begin{center}
|... -jobname "|\textit{target}|" "|[\textit{flags}]%
[|\def\jobname{|\textit{dest}|}|]|\input{|\textit{main}|}"|
\end{center}

%%%%%%%%%%%%%%%%%%%%%%%%%%%%%%%%%%%%%%%%%%%%%%%%%%%%%%%%%%%%%%%%%%%%%%%%%%%%%%%%
\subsection{Manual Code}
\label{sec:manual}

In case one cannot be certain whether the definitions file |childdoc.def|
is installed on the target \TeX{} distribution
and one prefers not to ship it,
it is conceivable to paste a few relevant commands into the sources.

To that end, drop all statements |\input{childdoc.def}|
and perform the replacements as outlined below.
Instead of |\childdocmain{|\textit{main}|}| add the following code
to the top of the main file:
%
\begin{center}
\begin{tabular}{l}
|\||ifdefined\childdocname\endinput\||fi\newif\ifchilddoc|\\
|\edef\childdocname{\scantokens\expandafter{\jobname\noexpand}}|\\
|\def\childdocmain{|\textit{main}|}\||ifx\childdocmain\childdocname\||else|\\
|\childdoctrue\includeonly{\childdocname}\let\jobname\childdocmain\||fi|\\
\end{tabular}
\end{center}
%
Instead of |\childdocof{|\textit{main}|}| just include the main file
at the top of each child file:
%
\begin{center}
|\input{|\textit{main}|}|
\end{center}
%
A simple redirection |\childdocforward{|\textit{dest}|}| is achieved by:
%
\begin{center}
|\def\jobname{|\textit{dest}|}\input{\jobname}|
\end{center}
%
The redirection with prefix
|\childdocforwardprefix[|\textit{prefix}|]{|\textit{dest}|}|
is accomplished by:
%
\begin{center}
\begin{tabular}{l}
|{\edef\jobname{\scantokens\expandafter{\jobname\noexpand}}|\\
|\def\redirectjob |\textit{prefix}|#1~~~{\gdef\jobname{|\textit{dest}|#1}}|\\
|\expandafter\redirectjob\jobname~~~}\input{\jobname}|
\end{tabular}
\end{center}

In an alternative approach,
child documents can be compiled by a specific command line
without additional code or specific definitions:
%
\begin{center}
|... -jobname "|\textit{target}|" "|[\textit{flags}]%
|\includeonly{|\textit{dest}|}\input{|\textit{main}|}"|
\end{center}
%

%%%%%%%%%%%%%%%%%%%%%%%%%%%%%%%%%%%%%%%%%%%%%%%%%%%%%%%%%%%%%%%%%%%%%%%%%%%%%%%%
%%%%%%%%%%%%%%%%%%%%%%%%%%%%%%%%%%%%%%%%%%%%%%%%%%%%%%%%%%%%%%%%%%%%%%%%%%%%%%%%
\section{Information}

%%%%%%%%%%%%%%%%%%%%%%%%%%%%%%%%%%%%%%%%%%%%%%%%%%%%%%%%%%%%%%%%%%%%%%%%%%%%%%%%
\subsection{Copyright}

Copyright \copyright{} 2017--2018 Niklas Beisert

This work may be distributed and/or modified under the
conditions of the \LaTeX{} Project Public License, either version 1.3
of this license or (at your option) any later version.
The latest version of this license is in
  \url{http://www.latex-project.org/lppl.txt}
and version 1.3 or later is part of all distributions of \LaTeX{}
version 2005/12/01 or later.

This work has the LPPL maintenance status `maintained'.

The Current Maintainer of this work is Niklas Beisert.

This work consists of the files |README.txt|, |childdoc.ins| and |childdoc.dtx|
as well as the derived files |childdoc.def|, |cdocsamp.tex|
with |cdocsch1.tex|, |cdocsch2.tex|, |cdocspt3.tex|, |cdocspt4.tex|,
|cdocsdrf.tex|, |cdocsfn1.tex|, |cdocsfn2.tex|
as well as |childdoc.pdf|.

%%%%%%%%%%%%%%%%%%%%%%%%%%%%%%%%%%%%%%%%%%%%%%%%%%%%%%%%%%%%%%%%%%%%%%%%%%%%%%%%
\subsection{Files and Installation}

The package consists of the files:
%
\begin{center}
\begin{tabular}{ll}
    |README.txt|   & readme file \\
    |childdoc.ins| & installation file \\
    |childdoc.dtx| & source file \\
    |childdoc.def| & definition file \\
    |cdocsamp.tex| & sample main file \\
    |cdocsch1.tex| & sample include file \\
    |cdocsch2.tex| & sample include file \\
    |cdocspt3.tex| & sample part file \\
    |cdocspt4.tex| & sample part file \\
    |cdocsdrf.tex| & sample redirection file \\
    |cdocsfn1.tex| & sample redirection file \\
    |cdocsfn2.tex| & sample redirection file \\
    |childdoc.pdf| & manual
\end{tabular}
\end{center}
%
The distribution consists of the files
|README.txt|, |childdoc.ins| and |childdoc.dtx|.
%
\begin{itemize}
\item
Run (pdf)\LaTeX{} on |childdoc.dtx|
to compile the manual |childdoc.pdf| (this file).
\item
Run \LaTeX{} on |childdoc.ins| to create the definitions file |childdoc.def|
and the sample |cdocsamp.tex| with include files
|cdocsch1.tex|, |cdocsch2.tex|, |cdocspt3.tex|, |cdocspt4.tex|,
|cdocsdrf.tex|, |cdocsfn1.tex|, |cdocsfn2.tex|.
Then copy the file |childdoc.def| to an appropriate directory of your \LaTeX{}
distribution, e.g.\ \textit{texmf-root}|/tex/latex/childdoc|.
\end{itemize}

%%%%%%%%%%%%%%%%%%%%%%%%%%%%%%%%%%%%%%%%%%%%%%%%%%%%%%%%%%%%%%%%%%%%%%%%%%%%%%%%
\subsection{Related CTAN Packages}

There are several other packages which offer a similar functionality:
%
\begin{itemize}
\item
The packages
\href{http://ctan.org/pkg/docmute}{\textsf{docmute}},
\href{http://ctan.org/pkg/includex}{\textsf{includex}} and
\href{http://ctan.org/pkg/standalone}{\textsf{standalone}}
provide commands to include only the document body of
a child file thus allowing both files to be compiled individually.
\item
The packages \href{http://ctan.org/pkg/subdocs}{\textsf{subdocs}}
and \href{http://ctan.org/pkg/subfiles}{\textsf{subfiles}}
provide structures in which the main and child documents can be
encapsulated and allowing them to be compiled individually.
The inclusion mechanism is different from the conventional |\include|.
\item
The package \href{http://ctan.org/pkg/combine}{\textsf{combine}}
is an elaborate solution to combine several documents into one.
\end{itemize}
%
See also the CTAN topic \href{http://ctan.org/topic/subdocs}{\textsf{subdocs}}
for further related packages.
The present package differs from the above solutions in that
a document structure constructed with the conventional |\include| mechanism
just needs two extra commands at the top of every file
such that all constituent files can be compiled individually.

%%%%%%%%%%%%%%%%%%%%%%%%%%%%%%%%%%%%%%%%%%%%%%%%%%%%%%%%%%%%%%%%%%%%%%%%%%%%%%%%
%\subsection{Feature Suggestions}
%
%The following is a list of features which may be useful for future
%versions of this package:
%%
%\begin{itemize}
%\item
%\ldots
%\end{itemize}

%%%%%%%%%%%%%%%%%%%%%%%%%%%%%%%%%%%%%%%%%%%%%%%%%%%%%%%%%%%%%%%%%%%%%%%%%%%%%%%%
\subsection{Revision History}

%%%%%%%%%%%%%%%%%%%%%%%%%%%%%%%%%%%%%%%%
\paragraph{v2.0:} 2018/12/30

\begin{itemize}
\item
immediate forward processing
\item
added |\childdocby| mechanism
\item
manual restructured
\end{itemize}

%%%%%%%%%%%%%%%%%%%%%%%%%%%%%%%%%%%%%%%%
\paragraph{v1.6:} 2018/01/17

\begin{itemize}
\item
application for development of include files
\item
corrections to manual
\end{itemize}

%%%%%%%%%%%%%%%%%%%%%%%%%%%%%%%%%%%%%%%%
\paragraph{v1.5:} 2017/05/21

\begin{itemize}
\item
more complete structuring introduced
\item
|\childdocof| introduced
\item
|\childdoc| renamed to |\childdocmain|
\item
|\childredirect| renamed to |\childdocforward| and |\childdocforwardprefix|
and functionality expanded
\end{itemize}

%%%%%%%%%%%%%%%%%%%%%%%%%%%%%%%%%%%%%%%%
\paragraph{v1.0:} 2017/04/27

\begin{itemize}
\item
manual and install package
\item
first version published on CTAN
\end{itemize}

%%%%%%%%%%%%%%%%%%%%%%%%%%%%%%%%%%%%%%%%
\paragraph{v0.6:} 2017/04/26

\begin{itemize}
\item
redirection mechanism added
\end{itemize}

%%%%%%%%%%%%%%%%%%%%%%%%%%%%%%%%%%%%%%%%
\paragraph{v0.5:} 2017/04/26

\begin{itemize}
\item
functionality in definition file
\end{itemize}


%%%%%%%%%%%%%%%%%%%%%%%%%%%%%%%%%%%%%%%%%%%%%%%%%%%%%%%%%%%%%%%%%%%%%%%%%%%%%%%%
%%%%%%%%%%%%%%%%%%%%%%%%%%%%%%%%%%%%%%%%%%%%%%%%%%%%%%%%%%%%%%%%%%%%%%%%%%%%%%%%
%%%%%%%%%%%%%%%%%%%%%%%%%%%%%%%%%%%%%%%%%%%%%%%%%%%%%%%%%%%%%%%%%%%%%%%%%%%%%%%%
\appendix

\settowidth\MacroIndent{\rmfamily\scriptsize 000\ }

 \DocInput{childdoc.dtx}

\end{document}
%</driver>
% \fi
%
% %%%%%%%%%%%%%%%%%%%%%%%%%%%%%%%%%%%%%%%%%%%%%%%%%%%%%%%%%%%%%%%%%%%%%%%%%%%%%%
% %%%%%%%%%%%%%%%%%%%%%%%%%%%%%%%%%%%%%%%%%%%%%%%%%%%%%%%%%%%%%%%%%%%%%%%%%%%%%%
% \section{Sample}
%\iffalse
%<*samplemain>
%\fi
%
% The following presents a sample document
% with two chapters, two parts, a title page,
% a compile flag as well as three forwarding files to set the flag.
% It consists of eight |.tex| files:
% \begin{center}
% \begin{tabular}{ll}
% |cdocsamp.tex|&main file\\
% |cdocsch1.tex|&include file for chapter 1\\
% |cdocsch2.tex|&include file for chapter 2\\
% |cdocspt3.tex|&include file for part 3\\
% |cdocspt4.tex|&include file for part 4\\
% |cdocsdrf.tex|&forwarding file for main file in draft mode\\
% |cdocsfi1.tex|&forwarding file for final version of chapter 1\\
% |cdocsfi2.tex|&forwarding file for final version of chapter 2\\
% \end{tabular}
% \end{center}
% Each of the eight files can be compiled directly by the \LaTeX{} compiler.
%
% %%%%%%%%%%%%%%%%%%%%%%%%%%%%%%%%%%%%%%
% \paragraph{Main File.}
%
% The main file is called |cdocsamp.tex|.
%
% Load the \textsf{childdoc} definitions and
% declare the filename for the main document:
%    \begin{macrocode}
\input{childdoc.def}
\childdocmain{}
%    \end{macrocode}

% Optional override for |\version| flag:
%    \begin{macrocode}
%%\ifchilddoc\else\providecommand{\version}{draft}\fi
%    \end{macrocode}

% Define the default values for the |\version| flag
% (|final| for the main file and |draft| for childs):
%    \begin{macrocode}
\ifchilddoc
\providecommand{\version}{draft}
\else
\providecommand{\version}{final}
\fi
%    \end{macrocode}

% Load the standard document class:
%    \begin{macrocode}
\documentclass[12pt]{article}
%    \end{macrocode}

% Start the document body:
%    \begin{macrocode}
\begin{document}
%    \end{macrocode}

% Declare a title page.
% Print title, part of document being processed and version flag:
%    \begin{macrocode}
\addtocounter{page}{-1}
\begin{center}
{\LARGE\bfseries{}childdoc example\par}
\vspace{1cm}
\ifchilddoc
\ifchilddocmanual part\else chapter\fi:
`\childdocname' of `\childdocjob'\par
\else
main document: `\childdocjob'\par
\fi
version: \version\par
\end{center}
\newpage
%    \end{macrocode}

% Manually include selected file,
% otherwise process as usual:
%    \begin{macrocode}
\ifchilddocmanual
\section*{part `\childdocname'}
\input{\childdocname}
\else
%    \end{macrocode}

% Include the two chapters:
%    \begin{macrocode}
\include{cdocsch1}
\include{cdocsch2}
%    \end{macrocode}

% Include the two parts unless only chapters should be displayed:
%    \begin{macrocode}
\ifchilddoc\else
\section{part three}
\input{cdocspt3}
\section{part four}
\input{cdocspt4}
\fi
%    \end{macrocode}

% Process as usual until here:
%    \begin{macrocode}
\fi
%    \end{macrocode}

% End of document body:
%    \begin{macrocode}
\end{document}
%    \end{macrocode}
%\iffalse
%</samplemain>
%\fi
%
% %%%%%%%%%%%%%%%%%%%%%%%%%%%%%%%%%%%%%%
% \paragraph{Chapter Include Files.}
%
% The include files are called |cdocsch1.tex| and |cdocsch2.tex|.
%
%\iffalse
%<*samplechap1|samplechap2>
%\fi

% Optional override for |\version| flag:
%    \begin{macrocode}
%%\providecommand{\version}{final}
%    \end{macrocode}

% Include the main document:
%    \begin{macrocode}
\input{childdoc.def}
\childdocof{cdocsamp}
%    \end{macrocode}

%\iffalse
%</samplechap1|samplechap2>
%\fi
%
%\iffalse
%<*samplechap1>
%\fi
% Some text for chapter 1:
%    \begin{macrocode}
\section{one}
some text in chapter one
%    \end{macrocode}

%\iffalse
%</samplechap1>
%\fi
% Some text for chapter 2:
%\iffalse
%<*samplechap2>
%\fi
%    \begin{macrocode}
\section{two}
more text in chapter two
%    \end{macrocode}

%\iffalse
%</samplechap2>
%\fi
%
% %%%%%%%%%%%%%%%%%%%%%%%%%%%%%%%%%%%%%%
% \paragraph{Part Include Files.}
%
% The include files are called |cdocspt3.tex| and |cdocspt4.tex|.
%
%\iffalse
%<*samplepart3|samplepart4>
%\fi

% Optional override for |\version| flag:
%    \begin{macrocode}
%%\providecommand{\version}{final}
%    \end{macrocode}

% Include the main document:
%    \begin{macrocode}
\input{childdoc.def}
\childdocby{cdocsamp}
%    \end{macrocode}

%\iffalse
%</samplepart3|samplepart4>
%\fi
%
%\iffalse
%<*samplepart3>
%\fi
% Some text for part 3:
%    \begin{macrocode}
some text in part three
%    \end{macrocode}

%\iffalse
%</samplepart3>
%\fi
% Some text for part 4:
%\iffalse
%<*samplepart4>
%\fi
%    \begin{macrocode}
more text in part four
%    \end{macrocode}

%\iffalse
%</samplepart4>
%\fi
%
% %%%%%%%%%%%%%%%%%%%%%%%%%%%%%%%%%%%%%%
% \paragraph{Forwarding for a Complete Draft.}
%
% The following forwarding file |cdocsdrf.tex|
% compiles the main document in draft mode:
%\iffalse
%<*sampledraft>
%\fi
%    \begin{macrocode}
\def\version{draft}
\input{childdoc.def}
\childdocforward{cdocsamp}
%    \end{macrocode}

%\iffalse
%</sampledraft>
%\fi
%
% %%%%%%%%%%%%%%%%%%%%%%%%%%%%%%%%%%%%%%
% \paragraph{Forwarding for Final Version of the Chapters.}
%
% The following forwarding files |cdocsfn1.tex| and |cdocsfn2.tex|
% (with identical content)
% compile the final versions of the child documents
% |cdocsch1.tex| and |cdocsch2.tex|, respectively:
%\iffalse
%<*samplefinal>
%\fi
%    \begin{macrocode}
\def\version{final}
\input{childdoc.def}
\childdocforwardprefix[cdocsamp]{cdocsfn}{cdocsch}
%    \end{macrocode}

%\iffalse
%</samplefinal>
%\fi
%
% %%%%%%%%%%%%%%%%%%%%%%%%%%%%%%%%%%%%%%
% \paragraph{Command Line Processing.}
%
% The following three command lines generate the output files
% |cdocscld|, |cdocscl1| and |cdocscl2|
% which should be identical to
% |cdocsdrf|, |cdocsch1| and |cdocsfn2|, respectively:
% \begin{center}
% \begin{tabular}{l}
% |latex -jobname cdocscld \|\\
% |  "\def\version{draft}\input{childdoc.def}\childdocforward{cdocsamp}"|\\
% |latex -jobname cdocscl1 \|\\
% |  "\input{childdoc.def}\childdocforward[cdocsamp]{cdocsch1}"|\\
% |latex -jobname cdocscl2 \|\\
% |  "\def\version{final}\input{childdoc.def}\childdocforward{cdocsch2}"|
% \end{tabular}
% \end{center}
% Note that the trailing backslash on each first line
% merely continues the input to the second line
% (for convenient cut ant paste).
% Furthermore, the command |latex| can be replaced by any
% of its alternative versions such as |pdflatex|.
%
% %%%%%%%%%%%%%%%%%%%%%%%%%%%%%%%%%%%%%%%%%%%%%%%%%%%%%%%%%%%%%%%%%%%%%%%%%%%%%%
% %%%%%%%%%%%%%%%%%%%%%%%%%%%%%%%%%%%%%%%%%%%%%%%%%%%%%%%%%%%%%%%%%%%%%%%%%%%%%%
% \section{Implementation}
%\iffalse
%<*package>
%\fi
%
% This section describes the definitions file |childdoc.def|.

% The definitions cannot be loaded using |\usepackage| or |\RequirePackage|
% which has a mechanism to prevent loading a style file more than once.
% When loading the definitions by means of |\input|
% multiple instances have to be prevented manually:
%\iffalse
%This code needs to be before the `\ProvidesFile' directive
%which is defined at the beginning of this file.
%Therefore it is also placed there and commented out here.
%</package>
%<*discard>
%\fi
%    \begin{macrocode}
\ifdefined\childdocmain\endinput\fi
%    \end{macrocode}
%\iffalse
%</discard>
%<*package>
%\fi
%
% \macro{\ifchilddoc}
% \macro{\ifchilddocmanual}
% The conditional |\ifchilddoc| tells whether a
% child (true) or main (false) document is being compiled.
% The conditional |\ifchilddocmanual| tells whether
% the |\includeonly| mechanism is used (false) or
% the selection of child files must be performed manually (true).
% The definitions initialise to false:
%    \begin{macrocode}
\newif\ifchilddoc
\newif\ifchilddocmanual
%    \end{macrocode}

% \macro{\childdocname}
% \macro{\childdocjob}
% The macro |\childdocname| stores the name of the main document
% to be compiled. The macro |\childdocjob| stores the name of
% the document on which the \LaTeX{} compiler was originally invoked.
% The content of |\jobname| cannot be compared
% to filenames specified in the source due to different catcodes.
% The following code rescans |\jobname|, stores the result
% in |\childdocname| and saves a copy in |\childdocjob|:
%    \begin{macrocode}
\edef\childdocname{\scantokens\expandafter{\jobname\noexpand}}
\let\childdocjob\childdocname
%    \end{macrocode}

% \macro{\childdocdisable}
% The macro |\childdocdisable| prevents the main file
% from being processed more than once.
% At this stage, the main document command |\childdocmain|
% is assumed to be called once again where it should do nothing.
% Any subsequent call to it should prevent
% a secondary processing of the main document
% It overwrites the forwarding commands
% |\childdocof| and |\childdocforward|
% with empty macros to prevent further inclusions of the main document:
%    \begin{macrocode}
\newcommand{\childdocdisable}
{
  \renewcommand{\childdocmain}[1]{\renewcommand{\childdocmain}[1]{\endinput}}
  \renewcommand{\childdocof}[1]{}
  \renewcommand{\childdocby}[2][]{}
  \renewcommand{\childdocforward}[2][]{}
  \renewcommand{\childdocdisable}{}
}
%    \end{macrocode}

% \macro{\childdocmain}
% The macro |\childdocmain| is to be called at the top of the main file
% with nothing or the main filename (without extension) as argument.
% First, it breaks loops.
% If the argument is not empty and does not match |\childdocname|
% (which is set by the first inclusion of |childdoc.def|),
% |\ifchilddoc| is set to true, |\includeonly| is applied to the child file
% and |\jobname| is set to the main file
% (for proper handling of |.aux| files):
%    \begin{macrocode}
\newcommand{\childdocmain}[1]
{
  \childdocdisable\childdocmain{}
  \if?#1?\else
    \begingroup
      \def\childdoctmp{#1}
      \ifx\childdoctmp\childdocname
        \def\childdoctmp{}
      \else
        \def\childdoctmp
        {
          \childdoctrue
          \includeonly{\childdocname}
          \def\childdocjob{#1}
          \def\jobname{#1}
        }
      \fi
      \expandafter
    \endgroup
    \childdoctmp
  \fi
}
%    \end{macrocode}

% \macro{\childdocof}
% The command |\childdocof| redirects
% compilation to the main file |#1|.
%    \begin{macrocode}
\newcommand{\childdocof}[1]
{
  \childdocdisable
  \childdoctrue
  \includeonly{\childdocname}
  \def\jobname{#1}
  \def\childdocjob{#1}
  \input{#1}
}
%    \end{macrocode}

% \macro{\childdocby}
% The command |\childdocby| ....
%    \begin{macrocode}
\newcommand{\childdocby}[2][]
{
  \childdocdisable
  \childdoctrue
  \childdocmanualtrue
  \if?#1?\else
    \def\jobname{#2}
  \fi
  \def\childdocjob{#2}
  \input{#2}
  \endinput
}
%    \end{macrocode}

% \macro{\childdocforward}
% The command |\childdocforward| redirects
% compilation to the main file or
% (if the optional argument is given) a child file.
% Parameters are set as if the main file
% or a child file starting with |\childdocof| was compiled.
% Then compilation is handed over to the main file:
%    \begin{macrocode}
\newcommand{\childdocforward}[2][]
{
  \begingroup
    \if?#1?
      \def\childdoctmp
      {
        \def\childdocname{#2}
        \def\childdocjob{#2}
        \def\jobname{#2}
        \input{#2}
        \endinput
      }
    \else
      \def\childdoctmp
      {
        \childdocdisable
        \def\childdocname{#2}
        \childdoctrue
        \includeonly{#2}
        \def\childdocjob{#1}
        \def\jobname{#1}
        \input{#1}
        \endinput
      }
    \fi
    \expandafter
  \endgroup
  \childdoctmp
}
%    \end{macrocode}

% \macro{\childdocforwardprefix}
% The command |\childdocforwardprefix| redirects
% compilation to the main or a child file by means of a pattern.
% The prefix |#1| in the current filename is replaced by |#2|
% and the suffix of the current filename is kept
% (it is assumed that the filename does not contain the substring `|~~~|'
% which is used as a delimiter).
% Compilation is handed over to the new file by |\childdocforward|:
%    \begin{macrocode}
\newcommand{\childdocforwardprefix}[3][]
{
  \begingroup
    \def\childdocextract #2##1~~~{\def\childdoctmp{\childdocforward[#1]{#3##1}}}
    \expandafter\childdocextract\childdocname~~~
    \expandafter
  \endgroup
  \childdoctmp
}
%    \end{macrocode}

% \macro{\childdoc}
% The deprecated macro |\childdoc| is a legacy version of |\childdocmain|:
%    \begin{macrocode}
\newcommand{\childdoc}{\childdocmain}
%    \end{macrocode}

% \macro{\childdocredirect}
% The deprecated macro |\childdocredirect| is a legacy version
% of |\childdocforward| and |\childdocforwardprefix|:
%    \begin{macrocode}
\newcommand{\childdocredirect}[2][]
{
  \begingroup
    \if?#1?
      \def\childdoctmp{\childdocforward{#2}}
    \else
      \def\childdoctmp{\childdocforwardprefix{#1}{#2}}
    \fi
    \expandafter
  \endgroup
  \childdoctmp
}
%    \end{macrocode}

%\iffalse
%</package>
%\fi
%
\endinput
|\\
|\childdocforward{|\textit{main}|}|
\end{tabular}
\end{center}
%
Likewise, the following files |final|\textit{nn}|.tex|
compile the final version of the child document
|child|\textit{nn}|.tex|:
%
\begin{center}
\begin{tabular}{l}
|\def\version{final}|\\
|% \iffalse
%
% childdoc.dtx Copyright (C) 2017-2018 Niklas Beisert
%
% This work may be distributed and/or modified under the
% conditions of the LaTeX Project Public License, either version 1.3
% of this license or (at your option) any later version.
% The latest version of this license is in
%   http://www.latex-project.org/lppl.txt
% and version 1.3 or later is part of all distributions of LaTeX
% version 2005/12/01 or later.
%
% This work has the LPPL maintenance status `maintained'.
%
% The Current Maintainer of this work is Niklas Beisert.
%
% This work consists of the files childdoc.dtx and childdoc.ins
% and the derived files childdoc.def and cdocsamp.tex with
% cdocsch1.tex, cdocsch2.tex, cdocsdrf.tex, cdocsfn1.tex, cdocsfn2.tex.
%
%<package>\ifdefined\childdocmain\endinput\fi
%<package>\ProvidesFile{childdoc.def}[2018/12/30 v2.0 child document driver]
%<samplemain>\ProvidesFile{cdocsamp.tex}[2018/12/30 v2.0 sample for childdoc]
%<*driver>
%\ProvidesFile{childdoc.drv}[2018/12/30 v2.0 childdoc reference manual file]
\PassOptionsToClass{10pt,a4paper}{article}
\documentclass{ltxdoc}

\usepackage[margin=35mm]{geometry}
\usepackage{hyperref}
\usepackage{hyperxmp}
\usepackage[usenames]{color}

\hypersetup{colorlinks=true}
\hypersetup{pdfstartview=FitH}
\hypersetup{pdfpagemode=UseNone}
\hypersetup{pdfsource={}}
\hypersetup{pdflang={en-UK}}
\hypersetup{pdfcopyright={Copyright 2017-2018 Niklas Beisert.
  This work may be distributed and/or modified under the
  conditions of the LaTeX Project Public License, either version 1.3
  of this license or (at your option) any later version.}}
\hypersetup{pdflicenseurl={http://www.latex-project.org/lppl.txt}}
\hypersetup{pdfcontactaddress={ETH Zurich, ITP, HIT K,
  Wolfgang-Pauli-Strasse 27}}
\hypersetup{pdfcontactpostcode={8093}}
\hypersetup{pdfcontactcity={Zurich}}
\hypersetup{pdfcontactcountry={Switzerland}}
\hypersetup{pdfcontactemail={nbeisert@itp.phys.ethz.ch}}
\hypersetup{pdfcontacturl={http://people.phys.ethz.ch/\xmptilde nbeisert/}}

\newcommand{\secref}[1]{\hyperref[#1]{section \ref*{#1}}}

\parskip1ex
\parindent0pt
\let\olditemize\itemize
\def\itemize{\olditemize\parskip0pt}

\begin{document}

\title{The \textsf{childdoc} Package}
\hypersetup{pdftitle={The childdoc Package}}
\author{Niklas Beisert\\[2ex]
  Institut f\"ur Theoretische Physik\\
  Eidgen\"ossische Technische Hochschule Z\"urich\\
  Wolfgang-Pauli-Strasse 27, 8093 Z\"urich, Switzerland\\[1ex]
  \href{mailto:nbeisert@itp.phys.ethz.ch}
  {\texttt{nbeisert@itp.phys.ethz.ch}}}
\hypersetup{pdfauthor={Niklas Beisert}}
\hypersetup{pdfsubject={Manual for the LaTeX2e Package childdoc}}
\date{30 December 2018, \textsf{v2.0}}
\maketitle

\begin{abstract}\noindent
\textsf{childdoc} is a \LaTeXe{} package
that enables the direct compilation
of document sections included by |\include|
to individual files.
\end{abstract}

\begingroup
\parskip0ex
\tableofcontents
\endgroup

%%%%%%%%%%%%%%%%%%%%%%%%%%%%%%%%%%%%%%%%%%%%%%%%%%%%%%%%%%%%%%%%%%%%%%%%%%%%%%%%
%%%%%%%%%%%%%%%%%%%%%%%%%%%%%%%%%%%%%%%%%%%%%%%%%%%%%%%%%%%%%%%%%%%%%%%%%%%%%%%%
\section{Introduction}

\LaTeX{} provides a mechanism to structure a large document (such as a book)
into a main file and several child files (containing the chapters)
using the |\include| command.
This mechanism is beneficial for documents
which span hundreds of pages in order to
make the source file(s) more manageable.
Moreover, compilation can be restricted to
selected child files by means of the |\includeonly| command.
The latter feature can be used to reduce the compilation time while editing
(this was significantly more useful in the earlier days of \LaTeX{})
or to generate a smaller document which is easier to navigate.
Another application of |\includeonly| is to generate
documents consisting of selected parts of the complete document.

However, there are a few drawbacks of the plain |\include| mechanism:
\begin{itemize}
\item
The child files cannot be compiled on their own,
they can only be compiled via the main file.
A naive editing environment
(such as a text editor with an option
to have the current file processed by \LaTeX)
may require one to switch to the main file before compiling;
attempting to compile the child file produces errors.
\item
The main file must be modified (each time)
to adjust the |\includeonly| command
to the present needs. This easily leaves the main file in a messy state.
\item
The generated document will always carry the filename
of the main document. This is inconvenient if
several child files are to be compiled and
to be kept for distribution.
\end{itemize}

The present package provides a simple interface
to make child files individually compilable by \LaTeX{}.
Compiling a child file then has the same effect as compiling
the main file with an |\includeonly| command
to select the appropriate child.
Moreover the generated document will carry the name of the child
rather than the main file.
This resolves all three above issues.

This feature is meant to make the editing of books,
thesis documents and lecture notes somewhat more convenient.
However, the package can also be used efficiently for
composing a series of documents (such as exercise sheets)
which are typically distributed individually.
It then assists the author in generating the individual documents
(potentially in different versions)
as well as a document containing the collected series.
Another application is in developing style files
or other kinds of included material
where compilation of the style file could redirect
to a sample or test file.

%%%%%%%%%%%%%%%%%%%%%%%%%%%%%%%%%%%%%%%%%%%%%%%%%%%%%%%%%%%%%%%%%%%%%%%%%%%%%%%%
%%%%%%%%%%%%%%%%%%%%%%%%%%%%%%%%%%%%%%%%%%%%%%%%%%%%%%%%%%%%%%%%%%%%%%%%%%%%%%%%
\section{Usage}

First of all, the package \textsf{childdoc} is \emph{not} a standard
\LaTeXe{} |.sty| style file! Therefore it needs to be invoked in
a non-standard way.

%%%%%%%%%%%%%%%%%%%%%%%%%%%%%%%%%%%%%%%%%%%%%%%%%%%%%%%%%%%%%%%%%%%%%%%%%%%%%%%%
\subsection{Included Files}
\label{sec:include}

%%%%%%%%%%%%%%%%%%%%%%%%%%%%%%%%%%%%%%%%
\DescribeMacro{\childdocmain}
To use the package, add the commands
\begin{center}
\begin{tabular}{l}
|\input{childdoc.def}|\\
|\childdocmain{}|\\
\end{tabular}
\end{center}
at the very top of the main \LaTeX{} file,
in particular \emph{before} the |\documentclass| statement!
The argument of |\childdocmain| should be left empty
(but it must be present).

%%%%%%%%%%%%%%%%%%%%%%%%%%%%%%%%%%%%%%%%
\DescribeMacro{\childdocof}
Furthermore, add the commands
\begin{center}
\begin{tabular}{l}
|\input{childdoc.def}|\\
|\childdocof{|\textit{main}|}|\\
\end{tabular}
\end{center}
at the top of every child file \textit{child}
which is included by |\include{|\textit{child}|}|
from within the main file
(or at least for those files to be compiled individually).
The argument \textit{main} must be the filename of the main file.

There are a couple of
considerations in setting up the main and child documents:

%%%%%%%%%%%%%%%%%%%%%%%%%%%%%%%%%%%%%%%%
\paragraph{Restrictions.}

Please note the following restrictions:
\begin{itemize}
\item
|\childdocmain| must be called with one argument \textit{main}
to ensure compatibility with earlier version of the package.
It must either be empty (|\childdocmain{}|)
or precisely match the filename of the main file in which it is specified.
See \secref{sec:detection} for further information.
\item
The filename \textit{main} must be specified without the |.tex| extension.
\item
The filename \textit{main} is case sensitive
(even in case-insensitive file systems)
due to internal string comparison.
\item
The argument \textit{main} should be fully expanded, it cannot be a macro.
\item
Subdirectories and special characters should be avoided in filenames.
\item
The command |\childdocmain{|\textit{main}|}| must be followed by a whitespace.
It should not be followed immediately by another command
or by a comment mark `|%|'.
This is because the \TeX{} parser reads the token immediately following
the argument of |\childdocmain| and puts it
at the beginning of every child section;
however, a white\-space is ignored.
\end{itemize}

%%%%%%%%%%%%%%%%%%%%%%%%%%%%%%%%%%%%%%%%
\paragraph{Content of Main File.}

It is advisable to place all content in the child files included by |\include|.
Any output contained in the main file will appear in all child documents
unless suppressed manually;
it cannot be suppressed automatically by the |\includeonly| directive
and thus should normally be avoided.
A method to include some content in the main file
by means of conditional processing is described in \secref{sec:conditional}.

%%%%%%%%%%%%%%%%%%%%%%%%%%%%%%%%%%%%%%%%
\paragraph{Page Numbering.}

When only a part of the document is compiled,
the appropriate numbering of pages
(as well as other status parameters)
is determined from the |.aux| files.
The latter contain information from previous passes.
However this information needs to propagate through
all intermediate child documents.
Therefore the page numbering in child documents may well
be inconsistent until the complete document is compiled at least once.

A useful (if unconventional) way to always ensure a consistent
page numbering is to restart the numbering in each child document
and denote the pages by `\textit{child}|.|\textit{page}'
where \textit{child} represents the chapter/section number of the child file.
This can be achieved by the command
|\numberwithin{page}{|\textit{child}|}|
of the \textsf{amsmath} package
where \textit{child} can be |chapter| or |section|
depending on the chosen structuring.
Alternatively, one can modify the macro |\thepage| appropriately
and reset the counter |page| at the start of each child file.

%%%%%%%%%%%%%%%%%%%%%%%%%%%%%%%%%%%%%%%%%%%%%%%%%%%%%%%%%%%%%%%%%%%%%%%%%%%%%%%%
\subsection{Conditional Processing}
\label{sec:conditional}

The package provides a mechanism to compile different versions
of a document. To customise the versions further some conditional processing
can come in handy to distinguish which version is being compiled.
The package provides two macros to describe the compilation context:

%%%%%%%%%%%%%%%%%%%%%%%%%%%%%%%%%%%%%%%%
\DescribeMacro{\ifchilddoc}
The conditional |\ifchilddoc| distinguishes between the compilation of
child documents and the main document:
%
\begin{center}
|\ifchilddoc |\textit{child-code}| |[|\||else |\textit{main-code}]| \||fi|
\end{center}

%%%%%%%%%%%%%%%%%%%%%%%%%%%%%%%%%%%%%%%%
\DescribeMacro{\childdocname}
\DescribeMacro{\childdocjob}
The macro |\childdocname| contains the filename (without extension)
of the main or child file being processed.
Note that |\childdocjob| will always contain the name of the main file.

%%%%%%%%%%%%%%%%%%%%%%%%%%%%%%%%%%%%%%%%
\paragraph{Title Page.}

Conditional processing can be used to include a title or banner page
in the main document when proper precautions are taken.
Importantly, the code in the main file should ensure that the page counter
(as well as other status parameters which are stored in the |.aux| files)
takes the same value after the conditional processing.
Otherwise the page numbers may take divergent values
depending on which part is compiled.

For example, a title page could be declared by:
%
\begin{center}
\begin{tabular}{l}
|\ifchilddoc\||else|\\
|\addtocounter{page}{-1}|\\
\textit{code for title page}\\
|\newpage|\\
|\||fi|
\end{tabular}
\end{center}
%
A banner page for the child documents can be generated by:
%
\begin{center}
\begin{tabular}{l}
|\ifchilddoc|\\
|\addtocounter{page}{-1}|\\
\textit{code for banner page}\\
|\newpage|\\
|\||fi|
\end{tabular}
\end{center}
%
Here one could write a message such as:
\begin{center}
|This is the part \childdocname{} of \childdocjob{}.|
\end{center}

%%%%%%%%%%%%%%%%%%%%%%%%%%%%%%%%%%%%%%%%%%%%%%%%%%%%%%%%%%%%%%%%%%%%%%%%%%%%%%%%
\subsection{Flags}
\label{sec:flags}

The package makes it easy to generate different versions
of the main or child documents.
To this end compilation flags can be defined
and assigned different default values.
They will be particularly useful in conjunction
with the forwarding mechanism described in \secref{sec:forward}.

For example, it may be useful to have a flag |\version|
which can be set to |draft| or |final|.
The document source will contain some conditional code
depending on the value of |\version|.
Suppose further, the flag should default to |final| for the main file
and to |draft| for child files
which is a natural assignment for editing the document.
This is achieved by placing the following code
in the preamble of the main document
(below the |\childdocmain| directive):
%
\begin{center}
\begin{tabular}{l}
|\ifchilddoc|\\
|\providecommand{\version}{draft}|\\
|\||else|\\
|\providecommand{\version}{final}|\\
|\||fi|
\end{tabular}
\end{center}
%
The definition by |\providecommand| makes sure
that previous definitions are not overwritten.
Further statements |\providecommand{\version}{...}|
can thus be added before the above code to override it.

For the main file, one might add a line
(between |\childdocmain| and the above block)
%
\begin{center}
|%\ifchilddoc\||else\providecommand{\version}{draft}\||fi|
\end{center}
%
which can be uncommented to produce a draft version.
Likewise one can add a line to the very top of a child file
(above the |\childdocof{|\textit{main}|}| directive)
%
\begin{center}
|%\providecommand{\version}{final}|
\end{center}
%
which can be uncommented to produce the final version of this child document.

%%%%%%%%%%%%%%%%%%%%%%%%%%%%%%%%%%%%%%%%%%%%%%%%%%%%%%%%%%%%%%%%%%%%%%%%%%%%%%%%
\subsection{Forwarding}
\label{sec:forward}

Different versions of the main or child documents
using compilation flags as described in \secref{sec:flags}
can be (permanently) stored in different files
for convenient compilation, viewing and distribution.
To this end, the package defines a command
to pass on compilation to a different file:

%%%%%%%%%%%%%%%%%%%%%%%%%%%%%%%%%%%%%%%%
\DescribeMacro{\childdocforward}
The command |\childdocforward| redirects processing to
another source file:
%
\begin{center}
\begin{tabular}{l}
|\input{childdoc.def}|\\
|\childdocforward[|\textit{main}|]{|\textit{dest}|}|\\
\end{tabular}
\end{center}
%
The argument \textit{dest} is the destination file
(without extension).
It should be the main file or one of the child files.
Note that further \textsf{childdoc} directives
such as |\childdocof| and |\childdocforward|
in the indicated file will be processed in this form.
The optional argument \textit{main}
passes on directly to the main file \textit{main}
while pretending to compile the child \textit{dest}.
This form behaves as if \textit{dest}
issues |\childdocof{|\textit{main}|}| right away,
and no further \textsf{childdoc} directives will be processed.

%%%%%%%%%%%%%%%%%%%%%%%%%%%%%%%%%%%%%%%%
\DescribeMacro{\...prefix}
In the alternative form |\childdocforwardprefix|,
%
\begin{center}
\begin{tabular}{l}
|\input{childdoc.def}|\\
|\childdocforwardprefix[|\textit{main}|]{|\textit{prefix}|}{|\textit{dest}|}|
\end{tabular}
\end{center}
%
the destination file is determined by a pattern
depending on the current file:
To make this work, the current file must be called
`{\textit{prefix}\hspace{0.2em}\textit{suffix}}'
with \textit{prefix} matching precisely the argument.
Processing is then passed on to the file
`{\textit{dest}\hspace{0.2em}\textit{suffix}}'.
Surely, the same effect is achieved by
directly specifying the
argument `{\textit{dest}\hspace{0.2em}\textit{suffix}}'
in the first form.
However, that requires to set up a different file
for each child. With the alternative form of the command
all these files can have exactly the same content
which simplifies setting them up and maintaining them.

For example, the following file |draft.tex|
with a compilation flag |\version| as described in \secref{sec:flags}
compiles the main document as a draft:
%
\begin{center}
\begin{tabular}{l}
|\def\version{draft}|\\
|\input{childdoc.def}|\\
|\childdocforward{|\textit{main}|}|
\end{tabular}
\end{center}
%
Likewise, the following files |final|\textit{nn}|.tex|
compile the final version of the child document
|child|\textit{nn}|.tex|:
%
\begin{center}
\begin{tabular}{l}
|\def\version{final}|\\
|\input{childdoc.def}|\\
|\childdocforwardprefix{final}{child}|
\end{tabular}
\end{center}
%

Note that when several versions of a main file and/or of each child file
are to be generated, it may be convenient to set up a |Makefile| or
shell script to automatise the process.

%%%%%%%%%%%%%%%%%%%%%%%%%%%%%%%%%%%%%%%%%%%%%%%%%%%%%%%%%%%%%%%%%%%%%%%%%%%%%%%%
\subsection{Command Line Processing}
\label{sec:commandline}

The effect of redirection files can also be achieved by invoking
the \LaTeX{} compiler with a more elaborate command line.
Most conveniently this should be done as part
of a shell script or a |Makefile|.

When using \textsf{childdoc} in the main file, the following
command lines effectively perform a redirection
(note that depending on the shell being used,
backslashes may have to be doubled: `|\|' $\to$ `|\\|'):
%
\begin{center}
|... -jobname "|\textit{target}|" |\\|"|[\textit{flags}]%
|\input{childdoc.def}\childdocforward[|\textit{main}|]{|\textit{dest}|}"|
\end{center}
%
Here \textit{target} is the name of the output file,
\textit{main} is the name of the main file
and \textit{dest} is the name of the main or child file to be processed
(all filenames without extensions).
The optional argument \textit{main} can be omitted
if \textit{main} matches \textit{dest}.
Optionally, compilation \textit{flags} can be defined via |\def| commands.
This command line makes the \TeX{} engine believe
it is compiling the file \textit{target}
whose content is specified as the latter parameter.
The provided code then forwards the processing to
\textit{main} or \textit{dest} as described in \secref{sec:forward}.

%%%%%%%%%%%%%%%%%%%%%%%%%%%%%%%%%%%%%%%%%%%%%%%%%%%%%%%%%%%%%%%%%%%%%%%%%%%%%%%%
\subsection{Include by Input}
\label{sec:input}

Including child documents by |\include| has some restrictions by design.
Most notably, the content of a child document always occupies
its own set of pages; pages cannot be shared between child documents.
Usually, this behaviour makes perfect sense
because each child document contain an essential part of the document.
However, in some situations it may be desirable to compose
a document from a collection of parts
without having mandatory page breaks between then.
For this case, the package
provides a mechanism to include parts
by |\input| which can also be processed individually.
However, by construction this mechanism
requires manual handling of the content to be output.

%%%%%%%%%%%%%%%%%%%%%%%%%%%%%%%%%%%%%%%%
\DescribeMacro{\ifchilddocmanual}
The main file should be prepared as usual, see \secref{sec:include}.
However, the document body must make a distinction
between processing of an individual part and of the main document, e.g.:
%
\begin{center}
\begin{tabular}{l}
|\ifchilddocmanual|\\
|\input{\childdocname}|\\
|\||else|\\
\textit{document body with }|\input{|\textit{part}|}|\\
|\||fi|
\end{tabular}
\end{center}
%
The conditional |\ifchilddocmanual| is true whenever
a part to be included by |\input| is being compiled,
and the name of the part is stored in |\childdocname|.

%%%%%%%%%%%%%%%%%%%%%%%%%%%%%%%%%%%%%%%%
\DescribeMacro{\childdocby}
Each part to be included by |\input| should start with:
%
\begin{center}
\begin{tabular}{l}
|\input{childdoc.def}|\\
|\childdocby{|\textit{main}|}|\\
\end{tabular}
\end{center}
%
The directive |\childdocby| is similar to |\childdocof|
described in \secref{sec:include},
but the subsequent selection of content must be done manually.
To that end, both |\ifchilddoc| and |\ifchilddocmanual|
will be true upon processing of a part,
and the name of the part is stored in |\childdocname|.
Note that |\jobname| will be set to the filename of the current part
so that each part receives an individual |.aux| file
that does not interfere with the |.aux| file(s) of the main document.
This behaviour can be altered by the alternative form
|\childdocby[*]{|\textit{main}|}| (with a non-empty optional argument)
which uses the |.aux| file of the main document
by setting |\jobname| to \textit{main}.

%%%%%%%%%%%%%%%%%%%%%%%%%%%%%%%%%%%%%%%%%%%%%%%%%%%%%%%%%%%%%%%%%%%%%%%%%%%%%%%%
\subsection{Driver Development}
\label{sec:driver}

The \textsf{childdoc} mechanism can also be use for the development
of definition files such as \LaTeX{} styles or classes.
This case differs from the above setup with multiple parts
included by |\include| in that no |\includeonly| should be invoked.
This can be achieved by starting the include file
(before |\ProvidesPackage|) with:
%
\begin{center}
\begin{tabular}{l}
|\input{childdoc.def}|\\
|\childdocforward{|\textit{main}|}|\\
\end{tabular}
\end{center}
%
or alternatively with:
%
\begin{center}
\begin{tabular}{l}
|\input{childdoc.def}|\\
|\childdocby{|\textit{main}|}|\\
\end{tabular}
\end{center}
%
Both forms have slightly different effects as described above.
The main file is prepared as usual, see \secref{sec:include}.

%%%%%%%%%%%%%%%%%%%%%%%%%%%%%%%%%%%%%%%%%%%%%%%%%%%%%%%%%%%%%%%%%%%%%%%%%%%%%%%%
\subsection{Legacy Detection}
\label{sec:detection}

The directive |\childdocmain| in the main file can detect
whether the complete document or merely a child is to be compiled
even without using the directive |\childdocof|.
This method is deprecated because it is less robust
and there is no compelling reason to use it;
it is merely provided for backward compatibility
and it may be removed in future versions.

If the detection mechanism is to be used,
it is mandatory to correctly specify
the filename of the main file as the argument of |\childdocmain|:
%
\begin{center}
\begin{tabular}{l}
|\input{childdoc.def}|\\
|\childdocmain{|\textit{main}|}|\\
\end{tabular}
\end{center}
%
If |\jobname| does not match the argument \textit{main} of |\childdocmain|,
it is assumed that |\jobname| points to the child file to be compiled.
When using |\childdocmain| with the main file specified as argument,
it suffices to start a child file
with just |\input{|\textit{main}|}|
without loading of the package and using |\childdocof|.
If instead all processing is done
with the appropriate \textsf{childdoc} directives,
the argument of \textit{main} of |\childdocmain| can be empty.

An alternative version of the command line processing described
in \secref{sec:commandline} using the detection mechanism reads:
%
\begin{center}
|... -jobname "|\textit{target}|" "|[\textit{flags}]%
[|\def\jobname{|\textit{dest}|}|]|\input{|\textit{main}|}"|
\end{center}

%%%%%%%%%%%%%%%%%%%%%%%%%%%%%%%%%%%%%%%%%%%%%%%%%%%%%%%%%%%%%%%%%%%%%%%%%%%%%%%%
\subsection{Manual Code}
\label{sec:manual}

In case one cannot be certain whether the definitions file |childdoc.def|
is installed on the target \TeX{} distribution
and one prefers not to ship it,
it is conceivable to paste a few relevant commands into the sources.

To that end, drop all statements |\input{childdoc.def}|
and perform the replacements as outlined below.
Instead of |\childdocmain{|\textit{main}|}| add the following code
to the top of the main file:
%
\begin{center}
\begin{tabular}{l}
|\||ifdefined\childdocname\endinput\||fi\newif\ifchilddoc|\\
|\edef\childdocname{\scantokens\expandafter{\jobname\noexpand}}|\\
|\def\childdocmain{|\textit{main}|}\||ifx\childdocmain\childdocname\||else|\\
|\childdoctrue\includeonly{\childdocname}\let\jobname\childdocmain\||fi|\\
\end{tabular}
\end{center}
%
Instead of |\childdocof{|\textit{main}|}| just include the main file
at the top of each child file:
%
\begin{center}
|\input{|\textit{main}|}|
\end{center}
%
A simple redirection |\childdocforward{|\textit{dest}|}| is achieved by:
%
\begin{center}
|\def\jobname{|\textit{dest}|}\input{\jobname}|
\end{center}
%
The redirection with prefix
|\childdocforwardprefix[|\textit{prefix}|]{|\textit{dest}|}|
is accomplished by:
%
\begin{center}
\begin{tabular}{l}
|{\edef\jobname{\scantokens\expandafter{\jobname\noexpand}}|\\
|\def\redirectjob |\textit{prefix}|#1~~~{\gdef\jobname{|\textit{dest}|#1}}|\\
|\expandafter\redirectjob\jobname~~~}\input{\jobname}|
\end{tabular}
\end{center}

In an alternative approach,
child documents can be compiled by a specific command line
without additional code or specific definitions:
%
\begin{center}
|... -jobname "|\textit{target}|" "|[\textit{flags}]%
|\includeonly{|\textit{dest}|}\input{|\textit{main}|}"|
\end{center}
%

%%%%%%%%%%%%%%%%%%%%%%%%%%%%%%%%%%%%%%%%%%%%%%%%%%%%%%%%%%%%%%%%%%%%%%%%%%%%%%%%
%%%%%%%%%%%%%%%%%%%%%%%%%%%%%%%%%%%%%%%%%%%%%%%%%%%%%%%%%%%%%%%%%%%%%%%%%%%%%%%%
\section{Information}

%%%%%%%%%%%%%%%%%%%%%%%%%%%%%%%%%%%%%%%%%%%%%%%%%%%%%%%%%%%%%%%%%%%%%%%%%%%%%%%%
\subsection{Copyright}

Copyright \copyright{} 2017--2018 Niklas Beisert

This work may be distributed and/or modified under the
conditions of the \LaTeX{} Project Public License, either version 1.3
of this license or (at your option) any later version.
The latest version of this license is in
  \url{http://www.latex-project.org/lppl.txt}
and version 1.3 or later is part of all distributions of \LaTeX{}
version 2005/12/01 or later.

This work has the LPPL maintenance status `maintained'.

The Current Maintainer of this work is Niklas Beisert.

This work consists of the files |README.txt|, |childdoc.ins| and |childdoc.dtx|
as well as the derived files |childdoc.def|, |cdocsamp.tex|
with |cdocsch1.tex|, |cdocsch2.tex|, |cdocspt3.tex|, |cdocspt4.tex|,
|cdocsdrf.tex|, |cdocsfn1.tex|, |cdocsfn2.tex|
as well as |childdoc.pdf|.

%%%%%%%%%%%%%%%%%%%%%%%%%%%%%%%%%%%%%%%%%%%%%%%%%%%%%%%%%%%%%%%%%%%%%%%%%%%%%%%%
\subsection{Files and Installation}

The package consists of the files:
%
\begin{center}
\begin{tabular}{ll}
    |README.txt|   & readme file \\
    |childdoc.ins| & installation file \\
    |childdoc.dtx| & source file \\
    |childdoc.def| & definition file \\
    |cdocsamp.tex| & sample main file \\
    |cdocsch1.tex| & sample include file \\
    |cdocsch2.tex| & sample include file \\
    |cdocspt3.tex| & sample part file \\
    |cdocspt4.tex| & sample part file \\
    |cdocsdrf.tex| & sample redirection file \\
    |cdocsfn1.tex| & sample redirection file \\
    |cdocsfn2.tex| & sample redirection file \\
    |childdoc.pdf| & manual
\end{tabular}
\end{center}
%
The distribution consists of the files
|README.txt|, |childdoc.ins| and |childdoc.dtx|.
%
\begin{itemize}
\item
Run (pdf)\LaTeX{} on |childdoc.dtx|
to compile the manual |childdoc.pdf| (this file).
\item
Run \LaTeX{} on |childdoc.ins| to create the definitions file |childdoc.def|
and the sample |cdocsamp.tex| with include files
|cdocsch1.tex|, |cdocsch2.tex|, |cdocspt3.tex|, |cdocspt4.tex|,
|cdocsdrf.tex|, |cdocsfn1.tex|, |cdocsfn2.tex|.
Then copy the file |childdoc.def| to an appropriate directory of your \LaTeX{}
distribution, e.g.\ \textit{texmf-root}|/tex/latex/childdoc|.
\end{itemize}

%%%%%%%%%%%%%%%%%%%%%%%%%%%%%%%%%%%%%%%%%%%%%%%%%%%%%%%%%%%%%%%%%%%%%%%%%%%%%%%%
\subsection{Related CTAN Packages}

There are several other packages which offer a similar functionality:
%
\begin{itemize}
\item
The packages
\href{http://ctan.org/pkg/docmute}{\textsf{docmute}},
\href{http://ctan.org/pkg/includex}{\textsf{includex}} and
\href{http://ctan.org/pkg/standalone}{\textsf{standalone}}
provide commands to include only the document body of
a child file thus allowing both files to be compiled individually.
\item
The packages \href{http://ctan.org/pkg/subdocs}{\textsf{subdocs}}
and \href{http://ctan.org/pkg/subfiles}{\textsf{subfiles}}
provide structures in which the main and child documents can be
encapsulated and allowing them to be compiled individually.
The inclusion mechanism is different from the conventional |\include|.
\item
The package \href{http://ctan.org/pkg/combine}{\textsf{combine}}
is an elaborate solution to combine several documents into one.
\end{itemize}
%
See also the CTAN topic \href{http://ctan.org/topic/subdocs}{\textsf{subdocs}}
for further related packages.
The present package differs from the above solutions in that
a document structure constructed with the conventional |\include| mechanism
just needs two extra commands at the top of every file
such that all constituent files can be compiled individually.

%%%%%%%%%%%%%%%%%%%%%%%%%%%%%%%%%%%%%%%%%%%%%%%%%%%%%%%%%%%%%%%%%%%%%%%%%%%%%%%%
%\subsection{Feature Suggestions}
%
%The following is a list of features which may be useful for future
%versions of this package:
%%
%\begin{itemize}
%\item
%\ldots
%\end{itemize}

%%%%%%%%%%%%%%%%%%%%%%%%%%%%%%%%%%%%%%%%%%%%%%%%%%%%%%%%%%%%%%%%%%%%%%%%%%%%%%%%
\subsection{Revision History}

%%%%%%%%%%%%%%%%%%%%%%%%%%%%%%%%%%%%%%%%
\paragraph{v2.0:} 2018/12/30

\begin{itemize}
\item
immediate forward processing
\item
added |\childdocby| mechanism
\item
manual restructured
\end{itemize}

%%%%%%%%%%%%%%%%%%%%%%%%%%%%%%%%%%%%%%%%
\paragraph{v1.6:} 2018/01/17

\begin{itemize}
\item
application for development of include files
\item
corrections to manual
\end{itemize}

%%%%%%%%%%%%%%%%%%%%%%%%%%%%%%%%%%%%%%%%
\paragraph{v1.5:} 2017/05/21

\begin{itemize}
\item
more complete structuring introduced
\item
|\childdocof| introduced
\item
|\childdoc| renamed to |\childdocmain|
\item
|\childredirect| renamed to |\childdocforward| and |\childdocforwardprefix|
and functionality expanded
\end{itemize}

%%%%%%%%%%%%%%%%%%%%%%%%%%%%%%%%%%%%%%%%
\paragraph{v1.0:} 2017/04/27

\begin{itemize}
\item
manual and install package
\item
first version published on CTAN
\end{itemize}

%%%%%%%%%%%%%%%%%%%%%%%%%%%%%%%%%%%%%%%%
\paragraph{v0.6:} 2017/04/26

\begin{itemize}
\item
redirection mechanism added
\end{itemize}

%%%%%%%%%%%%%%%%%%%%%%%%%%%%%%%%%%%%%%%%
\paragraph{v0.5:} 2017/04/26

\begin{itemize}
\item
functionality in definition file
\end{itemize}


%%%%%%%%%%%%%%%%%%%%%%%%%%%%%%%%%%%%%%%%%%%%%%%%%%%%%%%%%%%%%%%%%%%%%%%%%%%%%%%%
%%%%%%%%%%%%%%%%%%%%%%%%%%%%%%%%%%%%%%%%%%%%%%%%%%%%%%%%%%%%%%%%%%%%%%%%%%%%%%%%
%%%%%%%%%%%%%%%%%%%%%%%%%%%%%%%%%%%%%%%%%%%%%%%%%%%%%%%%%%%%%%%%%%%%%%%%%%%%%%%%
\appendix

\settowidth\MacroIndent{\rmfamily\scriptsize 000\ }

 \DocInput{childdoc.dtx}

\end{document}
%</driver>
% \fi
%
% %%%%%%%%%%%%%%%%%%%%%%%%%%%%%%%%%%%%%%%%%%%%%%%%%%%%%%%%%%%%%%%%%%%%%%%%%%%%%%
% %%%%%%%%%%%%%%%%%%%%%%%%%%%%%%%%%%%%%%%%%%%%%%%%%%%%%%%%%%%%%%%%%%%%%%%%%%%%%%
% \section{Sample}
%\iffalse
%<*samplemain>
%\fi
%
% The following presents a sample document
% with two chapters, two parts, a title page,
% a compile flag as well as three forwarding files to set the flag.
% It consists of eight |.tex| files:
% \begin{center}
% \begin{tabular}{ll}
% |cdocsamp.tex|&main file\\
% |cdocsch1.tex|&include file for chapter 1\\
% |cdocsch2.tex|&include file for chapter 2\\
% |cdocspt3.tex|&include file for part 3\\
% |cdocspt4.tex|&include file for part 4\\
% |cdocsdrf.tex|&forwarding file for main file in draft mode\\
% |cdocsfi1.tex|&forwarding file for final version of chapter 1\\
% |cdocsfi2.tex|&forwarding file for final version of chapter 2\\
% \end{tabular}
% \end{center}
% Each of the eight files can be compiled directly by the \LaTeX{} compiler.
%
% %%%%%%%%%%%%%%%%%%%%%%%%%%%%%%%%%%%%%%
% \paragraph{Main File.}
%
% The main file is called |cdocsamp.tex|.
%
% Load the \textsf{childdoc} definitions and
% declare the filename for the main document:
%    \begin{macrocode}
\input{childdoc.def}
\childdocmain{}
%    \end{macrocode}

% Optional override for |\version| flag:
%    \begin{macrocode}
%%\ifchilddoc\else\providecommand{\version}{draft}\fi
%    \end{macrocode}

% Define the default values for the |\version| flag
% (|final| for the main file and |draft| for childs):
%    \begin{macrocode}
\ifchilddoc
\providecommand{\version}{draft}
\else
\providecommand{\version}{final}
\fi
%    \end{macrocode}

% Load the standard document class:
%    \begin{macrocode}
\documentclass[12pt]{article}
%    \end{macrocode}

% Start the document body:
%    \begin{macrocode}
\begin{document}
%    \end{macrocode}

% Declare a title page.
% Print title, part of document being processed and version flag:
%    \begin{macrocode}
\addtocounter{page}{-1}
\begin{center}
{\LARGE\bfseries{}childdoc example\par}
\vspace{1cm}
\ifchilddoc
\ifchilddocmanual part\else chapter\fi:
`\childdocname' of `\childdocjob'\par
\else
main document: `\childdocjob'\par
\fi
version: \version\par
\end{center}
\newpage
%    \end{macrocode}

% Manually include selected file,
% otherwise process as usual:
%    \begin{macrocode}
\ifchilddocmanual
\section*{part `\childdocname'}
\input{\childdocname}
\else
%    \end{macrocode}

% Include the two chapters:
%    \begin{macrocode}
\include{cdocsch1}
\include{cdocsch2}
%    \end{macrocode}

% Include the two parts unless only chapters should be displayed:
%    \begin{macrocode}
\ifchilddoc\else
\section{part three}
\input{cdocspt3}
\section{part four}
\input{cdocspt4}
\fi
%    \end{macrocode}

% Process as usual until here:
%    \begin{macrocode}
\fi
%    \end{macrocode}

% End of document body:
%    \begin{macrocode}
\end{document}
%    \end{macrocode}
%\iffalse
%</samplemain>
%\fi
%
% %%%%%%%%%%%%%%%%%%%%%%%%%%%%%%%%%%%%%%
% \paragraph{Chapter Include Files.}
%
% The include files are called |cdocsch1.tex| and |cdocsch2.tex|.
%
%\iffalse
%<*samplechap1|samplechap2>
%\fi

% Optional override for |\version| flag:
%    \begin{macrocode}
%%\providecommand{\version}{final}
%    \end{macrocode}

% Include the main document:
%    \begin{macrocode}
\input{childdoc.def}
\childdocof{cdocsamp}
%    \end{macrocode}

%\iffalse
%</samplechap1|samplechap2>
%\fi
%
%\iffalse
%<*samplechap1>
%\fi
% Some text for chapter 1:
%    \begin{macrocode}
\section{one}
some text in chapter one
%    \end{macrocode}

%\iffalse
%</samplechap1>
%\fi
% Some text for chapter 2:
%\iffalse
%<*samplechap2>
%\fi
%    \begin{macrocode}
\section{two}
more text in chapter two
%    \end{macrocode}

%\iffalse
%</samplechap2>
%\fi
%
% %%%%%%%%%%%%%%%%%%%%%%%%%%%%%%%%%%%%%%
% \paragraph{Part Include Files.}
%
% The include files are called |cdocspt3.tex| and |cdocspt4.tex|.
%
%\iffalse
%<*samplepart3|samplepart4>
%\fi

% Optional override for |\version| flag:
%    \begin{macrocode}
%%\providecommand{\version}{final}
%    \end{macrocode}

% Include the main document:
%    \begin{macrocode}
\input{childdoc.def}
\childdocby{cdocsamp}
%    \end{macrocode}

%\iffalse
%</samplepart3|samplepart4>
%\fi
%
%\iffalse
%<*samplepart3>
%\fi
% Some text for part 3:
%    \begin{macrocode}
some text in part three
%    \end{macrocode}

%\iffalse
%</samplepart3>
%\fi
% Some text for part 4:
%\iffalse
%<*samplepart4>
%\fi
%    \begin{macrocode}
more text in part four
%    \end{macrocode}

%\iffalse
%</samplepart4>
%\fi
%
% %%%%%%%%%%%%%%%%%%%%%%%%%%%%%%%%%%%%%%
% \paragraph{Forwarding for a Complete Draft.}
%
% The following forwarding file |cdocsdrf.tex|
% compiles the main document in draft mode:
%\iffalse
%<*sampledraft>
%\fi
%    \begin{macrocode}
\def\version{draft}
\input{childdoc.def}
\childdocforward{cdocsamp}
%    \end{macrocode}

%\iffalse
%</sampledraft>
%\fi
%
% %%%%%%%%%%%%%%%%%%%%%%%%%%%%%%%%%%%%%%
% \paragraph{Forwarding for Final Version of the Chapters.}
%
% The following forwarding files |cdocsfn1.tex| and |cdocsfn2.tex|
% (with identical content)
% compile the final versions of the child documents
% |cdocsch1.tex| and |cdocsch2.tex|, respectively:
%\iffalse
%<*samplefinal>
%\fi
%    \begin{macrocode}
\def\version{final}
\input{childdoc.def}
\childdocforwardprefix[cdocsamp]{cdocsfn}{cdocsch}
%    \end{macrocode}

%\iffalse
%</samplefinal>
%\fi
%
% %%%%%%%%%%%%%%%%%%%%%%%%%%%%%%%%%%%%%%
% \paragraph{Command Line Processing.}
%
% The following three command lines generate the output files
% |cdocscld|, |cdocscl1| and |cdocscl2|
% which should be identical to
% |cdocsdrf|, |cdocsch1| and |cdocsfn2|, respectively:
% \begin{center}
% \begin{tabular}{l}
% |latex -jobname cdocscld \|\\
% |  "\def\version{draft}\input{childdoc.def}\childdocforward{cdocsamp}"|\\
% |latex -jobname cdocscl1 \|\\
% |  "\input{childdoc.def}\childdocforward[cdocsamp]{cdocsch1}"|\\
% |latex -jobname cdocscl2 \|\\
% |  "\def\version{final}\input{childdoc.def}\childdocforward{cdocsch2}"|
% \end{tabular}
% \end{center}
% Note that the trailing backslash on each first line
% merely continues the input to the second line
% (for convenient cut ant paste).
% Furthermore, the command |latex| can be replaced by any
% of its alternative versions such as |pdflatex|.
%
% %%%%%%%%%%%%%%%%%%%%%%%%%%%%%%%%%%%%%%%%%%%%%%%%%%%%%%%%%%%%%%%%%%%%%%%%%%%%%%
% %%%%%%%%%%%%%%%%%%%%%%%%%%%%%%%%%%%%%%%%%%%%%%%%%%%%%%%%%%%%%%%%%%%%%%%%%%%%%%
% \section{Implementation}
%\iffalse
%<*package>
%\fi
%
% This section describes the definitions file |childdoc.def|.

% The definitions cannot be loaded using |\usepackage| or |\RequirePackage|
% which has a mechanism to prevent loading a style file more than once.
% When loading the definitions by means of |\input|
% multiple instances have to be prevented manually:
%\iffalse
%This code needs to be before the `\ProvidesFile' directive
%which is defined at the beginning of this file.
%Therefore it is also placed there and commented out here.
%</package>
%<*discard>
%\fi
%    \begin{macrocode}
\ifdefined\childdocmain\endinput\fi
%    \end{macrocode}
%\iffalse
%</discard>
%<*package>
%\fi
%
% \macro{\ifchilddoc}
% \macro{\ifchilddocmanual}
% The conditional |\ifchilddoc| tells whether a
% child (true) or main (false) document is being compiled.
% The conditional |\ifchilddocmanual| tells whether
% the |\includeonly| mechanism is used (false) or
% the selection of child files must be performed manually (true).
% The definitions initialise to false:
%    \begin{macrocode}
\newif\ifchilddoc
\newif\ifchilddocmanual
%    \end{macrocode}

% \macro{\childdocname}
% \macro{\childdocjob}
% The macro |\childdocname| stores the name of the main document
% to be compiled. The macro |\childdocjob| stores the name of
% the document on which the \LaTeX{} compiler was originally invoked.
% The content of |\jobname| cannot be compared
% to filenames specified in the source due to different catcodes.
% The following code rescans |\jobname|, stores the result
% in |\childdocname| and saves a copy in |\childdocjob|:
%    \begin{macrocode}
\edef\childdocname{\scantokens\expandafter{\jobname\noexpand}}
\let\childdocjob\childdocname
%    \end{macrocode}

% \macro{\childdocdisable}
% The macro |\childdocdisable| prevents the main file
% from being processed more than once.
% At this stage, the main document command |\childdocmain|
% is assumed to be called once again where it should do nothing.
% Any subsequent call to it should prevent
% a secondary processing of the main document
% It overwrites the forwarding commands
% |\childdocof| and |\childdocforward|
% with empty macros to prevent further inclusions of the main document:
%    \begin{macrocode}
\newcommand{\childdocdisable}
{
  \renewcommand{\childdocmain}[1]{\renewcommand{\childdocmain}[1]{\endinput}}
  \renewcommand{\childdocof}[1]{}
  \renewcommand{\childdocby}[2][]{}
  \renewcommand{\childdocforward}[2][]{}
  \renewcommand{\childdocdisable}{}
}
%    \end{macrocode}

% \macro{\childdocmain}
% The macro |\childdocmain| is to be called at the top of the main file
% with nothing or the main filename (without extension) as argument.
% First, it breaks loops.
% If the argument is not empty and does not match |\childdocname|
% (which is set by the first inclusion of |childdoc.def|),
% |\ifchilddoc| is set to true, |\includeonly| is applied to the child file
% and |\jobname| is set to the main file
% (for proper handling of |.aux| files):
%    \begin{macrocode}
\newcommand{\childdocmain}[1]
{
  \childdocdisable\childdocmain{}
  \if?#1?\else
    \begingroup
      \def\childdoctmp{#1}
      \ifx\childdoctmp\childdocname
        \def\childdoctmp{}
      \else
        \def\childdoctmp
        {
          \childdoctrue
          \includeonly{\childdocname}
          \def\childdocjob{#1}
          \def\jobname{#1}
        }
      \fi
      \expandafter
    \endgroup
    \childdoctmp
  \fi
}
%    \end{macrocode}

% \macro{\childdocof}
% The command |\childdocof| redirects
% compilation to the main file |#1|.
%    \begin{macrocode}
\newcommand{\childdocof}[1]
{
  \childdocdisable
  \childdoctrue
  \includeonly{\childdocname}
  \def\jobname{#1}
  \def\childdocjob{#1}
  \input{#1}
}
%    \end{macrocode}

% \macro{\childdocby}
% The command |\childdocby| ....
%    \begin{macrocode}
\newcommand{\childdocby}[2][]
{
  \childdocdisable
  \childdoctrue
  \childdocmanualtrue
  \if?#1?\else
    \def\jobname{#2}
  \fi
  \def\childdocjob{#2}
  \input{#2}
  \endinput
}
%    \end{macrocode}

% \macro{\childdocforward}
% The command |\childdocforward| redirects
% compilation to the main file or
% (if the optional argument is given) a child file.
% Parameters are set as if the main file
% or a child file starting with |\childdocof| was compiled.
% Then compilation is handed over to the main file:
%    \begin{macrocode}
\newcommand{\childdocforward}[2][]
{
  \begingroup
    \if?#1?
      \def\childdoctmp
      {
        \def\childdocname{#2}
        \def\childdocjob{#2}
        \def\jobname{#2}
        \input{#2}
        \endinput
      }
    \else
      \def\childdoctmp
      {
        \childdocdisable
        \def\childdocname{#2}
        \childdoctrue
        \includeonly{#2}
        \def\childdocjob{#1}
        \def\jobname{#1}
        \input{#1}
        \endinput
      }
    \fi
    \expandafter
  \endgroup
  \childdoctmp
}
%    \end{macrocode}

% \macro{\childdocforwardprefix}
% The command |\childdocforwardprefix| redirects
% compilation to the main or a child file by means of a pattern.
% The prefix |#1| in the current filename is replaced by |#2|
% and the suffix of the current filename is kept
% (it is assumed that the filename does not contain the substring `|~~~|'
% which is used as a delimiter).
% Compilation is handed over to the new file by |\childdocforward|:
%    \begin{macrocode}
\newcommand{\childdocforwardprefix}[3][]
{
  \begingroup
    \def\childdocextract #2##1~~~{\def\childdoctmp{\childdocforward[#1]{#3##1}}}
    \expandafter\childdocextract\childdocname~~~
    \expandafter
  \endgroup
  \childdoctmp
}
%    \end{macrocode}

% \macro{\childdoc}
% The deprecated macro |\childdoc| is a legacy version of |\childdocmain|:
%    \begin{macrocode}
\newcommand{\childdoc}{\childdocmain}
%    \end{macrocode}

% \macro{\childdocredirect}
% The deprecated macro |\childdocredirect| is a legacy version
% of |\childdocforward| and |\childdocforwardprefix|:
%    \begin{macrocode}
\newcommand{\childdocredirect}[2][]
{
  \begingroup
    \if?#1?
      \def\childdoctmp{\childdocforward{#2}}
    \else
      \def\childdoctmp{\childdocforwardprefix{#1}{#2}}
    \fi
    \expandafter
  \endgroup
  \childdoctmp
}
%    \end{macrocode}

%\iffalse
%</package>
%\fi
%
\endinput
|\\
|\childdocforwardprefix{final}{child}|
\end{tabular}
\end{center}
%

Note that when several versions of a main file and/or of each child file
are to be generated, it may be convenient to set up a |Makefile| or
shell script to automatise the process.

%%%%%%%%%%%%%%%%%%%%%%%%%%%%%%%%%%%%%%%%%%%%%%%%%%%%%%%%%%%%%%%%%%%%%%%%%%%%%%%%
\subsection{Command Line Processing}
\label{sec:commandline}

The effect of redirection files can also be achieved by invoking
the \LaTeX{} compiler with a more elaborate command line.
Most conveniently this should be done as part
of a shell script or a |Makefile|.

When using \textsf{childdoc} in the main file, the following
command lines effectively perform a redirection
(note that depending on the shell being used,
backslashes may have to be doubled: `|\|' $\to$ `|\\|'):
%
\begin{center}
|... -jobname "|\textit{target}|" |\\|"|[\textit{flags}]%
|% \iffalse
%
% childdoc.dtx Copyright (C) 2017-2018 Niklas Beisert
%
% This work may be distributed and/or modified under the
% conditions of the LaTeX Project Public License, either version 1.3
% of this license or (at your option) any later version.
% The latest version of this license is in
%   http://www.latex-project.org/lppl.txt
% and version 1.3 or later is part of all distributions of LaTeX
% version 2005/12/01 or later.
%
% This work has the LPPL maintenance status `maintained'.
%
% The Current Maintainer of this work is Niklas Beisert.
%
% This work consists of the files childdoc.dtx and childdoc.ins
% and the derived files childdoc.def and cdocsamp.tex with
% cdocsch1.tex, cdocsch2.tex, cdocsdrf.tex, cdocsfn1.tex, cdocsfn2.tex.
%
%<package>\ifdefined\childdocmain\endinput\fi
%<package>\ProvidesFile{childdoc.def}[2018/12/30 v2.0 child document driver]
%<samplemain>\ProvidesFile{cdocsamp.tex}[2018/12/30 v2.0 sample for childdoc]
%<*driver>
%\ProvidesFile{childdoc.drv}[2018/12/30 v2.0 childdoc reference manual file]
\PassOptionsToClass{10pt,a4paper}{article}
\documentclass{ltxdoc}

\usepackage[margin=35mm]{geometry}
\usepackage{hyperref}
\usepackage{hyperxmp}
\usepackage[usenames]{color}

\hypersetup{colorlinks=true}
\hypersetup{pdfstartview=FitH}
\hypersetup{pdfpagemode=UseNone}
\hypersetup{pdfsource={}}
\hypersetup{pdflang={en-UK}}
\hypersetup{pdfcopyright={Copyright 2017-2018 Niklas Beisert.
  This work may be distributed and/or modified under the
  conditions of the LaTeX Project Public License, either version 1.3
  of this license or (at your option) any later version.}}
\hypersetup{pdflicenseurl={http://www.latex-project.org/lppl.txt}}
\hypersetup{pdfcontactaddress={ETH Zurich, ITP, HIT K,
  Wolfgang-Pauli-Strasse 27}}
\hypersetup{pdfcontactpostcode={8093}}
\hypersetup{pdfcontactcity={Zurich}}
\hypersetup{pdfcontactcountry={Switzerland}}
\hypersetup{pdfcontactemail={nbeisert@itp.phys.ethz.ch}}
\hypersetup{pdfcontacturl={http://people.phys.ethz.ch/\xmptilde nbeisert/}}

\newcommand{\secref}[1]{\hyperref[#1]{section \ref*{#1}}}

\parskip1ex
\parindent0pt
\let\olditemize\itemize
\def\itemize{\olditemize\parskip0pt}

\begin{document}

\title{The \textsf{childdoc} Package}
\hypersetup{pdftitle={The childdoc Package}}
\author{Niklas Beisert\\[2ex]
  Institut f\"ur Theoretische Physik\\
  Eidgen\"ossische Technische Hochschule Z\"urich\\
  Wolfgang-Pauli-Strasse 27, 8093 Z\"urich, Switzerland\\[1ex]
  \href{mailto:nbeisert@itp.phys.ethz.ch}
  {\texttt{nbeisert@itp.phys.ethz.ch}}}
\hypersetup{pdfauthor={Niklas Beisert}}
\hypersetup{pdfsubject={Manual for the LaTeX2e Package childdoc}}
\date{30 December 2018, \textsf{v2.0}}
\maketitle

\begin{abstract}\noindent
\textsf{childdoc} is a \LaTeXe{} package
that enables the direct compilation
of document sections included by |\include|
to individual files.
\end{abstract}

\begingroup
\parskip0ex
\tableofcontents
\endgroup

%%%%%%%%%%%%%%%%%%%%%%%%%%%%%%%%%%%%%%%%%%%%%%%%%%%%%%%%%%%%%%%%%%%%%%%%%%%%%%%%
%%%%%%%%%%%%%%%%%%%%%%%%%%%%%%%%%%%%%%%%%%%%%%%%%%%%%%%%%%%%%%%%%%%%%%%%%%%%%%%%
\section{Introduction}

\LaTeX{} provides a mechanism to structure a large document (such as a book)
into a main file and several child files (containing the chapters)
using the |\include| command.
This mechanism is beneficial for documents
which span hundreds of pages in order to
make the source file(s) more manageable.
Moreover, compilation can be restricted to
selected child files by means of the |\includeonly| command.
The latter feature can be used to reduce the compilation time while editing
(this was significantly more useful in the earlier days of \LaTeX{})
or to generate a smaller document which is easier to navigate.
Another application of |\includeonly| is to generate
documents consisting of selected parts of the complete document.

However, there are a few drawbacks of the plain |\include| mechanism:
\begin{itemize}
\item
The child files cannot be compiled on their own,
they can only be compiled via the main file.
A naive editing environment
(such as a text editor with an option
to have the current file processed by \LaTeX)
may require one to switch to the main file before compiling;
attempting to compile the child file produces errors.
\item
The main file must be modified (each time)
to adjust the |\includeonly| command
to the present needs. This easily leaves the main file in a messy state.
\item
The generated document will always carry the filename
of the main document. This is inconvenient if
several child files are to be compiled and
to be kept for distribution.
\end{itemize}

The present package provides a simple interface
to make child files individually compilable by \LaTeX{}.
Compiling a child file then has the same effect as compiling
the main file with an |\includeonly| command
to select the appropriate child.
Moreover the generated document will carry the name of the child
rather than the main file.
This resolves all three above issues.

This feature is meant to make the editing of books,
thesis documents and lecture notes somewhat more convenient.
However, the package can also be used efficiently for
composing a series of documents (such as exercise sheets)
which are typically distributed individually.
It then assists the author in generating the individual documents
(potentially in different versions)
as well as a document containing the collected series.
Another application is in developing style files
or other kinds of included material
where compilation of the style file could redirect
to a sample or test file.

%%%%%%%%%%%%%%%%%%%%%%%%%%%%%%%%%%%%%%%%%%%%%%%%%%%%%%%%%%%%%%%%%%%%%%%%%%%%%%%%
%%%%%%%%%%%%%%%%%%%%%%%%%%%%%%%%%%%%%%%%%%%%%%%%%%%%%%%%%%%%%%%%%%%%%%%%%%%%%%%%
\section{Usage}

First of all, the package \textsf{childdoc} is \emph{not} a standard
\LaTeXe{} |.sty| style file! Therefore it needs to be invoked in
a non-standard way.

%%%%%%%%%%%%%%%%%%%%%%%%%%%%%%%%%%%%%%%%%%%%%%%%%%%%%%%%%%%%%%%%%%%%%%%%%%%%%%%%
\subsection{Included Files}
\label{sec:include}

%%%%%%%%%%%%%%%%%%%%%%%%%%%%%%%%%%%%%%%%
\DescribeMacro{\childdocmain}
To use the package, add the commands
\begin{center}
\begin{tabular}{l}
|\input{childdoc.def}|\\
|\childdocmain{}|\\
\end{tabular}
\end{center}
at the very top of the main \LaTeX{} file,
in particular \emph{before} the |\documentclass| statement!
The argument of |\childdocmain| should be left empty
(but it must be present).

%%%%%%%%%%%%%%%%%%%%%%%%%%%%%%%%%%%%%%%%
\DescribeMacro{\childdocof}
Furthermore, add the commands
\begin{center}
\begin{tabular}{l}
|\input{childdoc.def}|\\
|\childdocof{|\textit{main}|}|\\
\end{tabular}
\end{center}
at the top of every child file \textit{child}
which is included by |\include{|\textit{child}|}|
from within the main file
(or at least for those files to be compiled individually).
The argument \textit{main} must be the filename of the main file.

There are a couple of
considerations in setting up the main and child documents:

%%%%%%%%%%%%%%%%%%%%%%%%%%%%%%%%%%%%%%%%
\paragraph{Restrictions.}

Please note the following restrictions:
\begin{itemize}
\item
|\childdocmain| must be called with one argument \textit{main}
to ensure compatibility with earlier version of the package.
It must either be empty (|\childdocmain{}|)
or precisely match the filename of the main file in which it is specified.
See \secref{sec:detection} for further information.
\item
The filename \textit{main} must be specified without the |.tex| extension.
\item
The filename \textit{main} is case sensitive
(even in case-insensitive file systems)
due to internal string comparison.
\item
The argument \textit{main} should be fully expanded, it cannot be a macro.
\item
Subdirectories and special characters should be avoided in filenames.
\item
The command |\childdocmain{|\textit{main}|}| must be followed by a whitespace.
It should not be followed immediately by another command
or by a comment mark `|%|'.
This is because the \TeX{} parser reads the token immediately following
the argument of |\childdocmain| and puts it
at the beginning of every child section;
however, a white\-space is ignored.
\end{itemize}

%%%%%%%%%%%%%%%%%%%%%%%%%%%%%%%%%%%%%%%%
\paragraph{Content of Main File.}

It is advisable to place all content in the child files included by |\include|.
Any output contained in the main file will appear in all child documents
unless suppressed manually;
it cannot be suppressed automatically by the |\includeonly| directive
and thus should normally be avoided.
A method to include some content in the main file
by means of conditional processing is described in \secref{sec:conditional}.

%%%%%%%%%%%%%%%%%%%%%%%%%%%%%%%%%%%%%%%%
\paragraph{Page Numbering.}

When only a part of the document is compiled,
the appropriate numbering of pages
(as well as other status parameters)
is determined from the |.aux| files.
The latter contain information from previous passes.
However this information needs to propagate through
all intermediate child documents.
Therefore the page numbering in child documents may well
be inconsistent until the complete document is compiled at least once.

A useful (if unconventional) way to always ensure a consistent
page numbering is to restart the numbering in each child document
and denote the pages by `\textit{child}|.|\textit{page}'
where \textit{child} represents the chapter/section number of the child file.
This can be achieved by the command
|\numberwithin{page}{|\textit{child}|}|
of the \textsf{amsmath} package
where \textit{child} can be |chapter| or |section|
depending on the chosen structuring.
Alternatively, one can modify the macro |\thepage| appropriately
and reset the counter |page| at the start of each child file.

%%%%%%%%%%%%%%%%%%%%%%%%%%%%%%%%%%%%%%%%%%%%%%%%%%%%%%%%%%%%%%%%%%%%%%%%%%%%%%%%
\subsection{Conditional Processing}
\label{sec:conditional}

The package provides a mechanism to compile different versions
of a document. To customise the versions further some conditional processing
can come in handy to distinguish which version is being compiled.
The package provides two macros to describe the compilation context:

%%%%%%%%%%%%%%%%%%%%%%%%%%%%%%%%%%%%%%%%
\DescribeMacro{\ifchilddoc}
The conditional |\ifchilddoc| distinguishes between the compilation of
child documents and the main document:
%
\begin{center}
|\ifchilddoc |\textit{child-code}| |[|\||else |\textit{main-code}]| \||fi|
\end{center}

%%%%%%%%%%%%%%%%%%%%%%%%%%%%%%%%%%%%%%%%
\DescribeMacro{\childdocname}
\DescribeMacro{\childdocjob}
The macro |\childdocname| contains the filename (without extension)
of the main or child file being processed.
Note that |\childdocjob| will always contain the name of the main file.

%%%%%%%%%%%%%%%%%%%%%%%%%%%%%%%%%%%%%%%%
\paragraph{Title Page.}

Conditional processing can be used to include a title or banner page
in the main document when proper precautions are taken.
Importantly, the code in the main file should ensure that the page counter
(as well as other status parameters which are stored in the |.aux| files)
takes the same value after the conditional processing.
Otherwise the page numbers may take divergent values
depending on which part is compiled.

For example, a title page could be declared by:
%
\begin{center}
\begin{tabular}{l}
|\ifchilddoc\||else|\\
|\addtocounter{page}{-1}|\\
\textit{code for title page}\\
|\newpage|\\
|\||fi|
\end{tabular}
\end{center}
%
A banner page for the child documents can be generated by:
%
\begin{center}
\begin{tabular}{l}
|\ifchilddoc|\\
|\addtocounter{page}{-1}|\\
\textit{code for banner page}\\
|\newpage|\\
|\||fi|
\end{tabular}
\end{center}
%
Here one could write a message such as:
\begin{center}
|This is the part \childdocname{} of \childdocjob{}.|
\end{center}

%%%%%%%%%%%%%%%%%%%%%%%%%%%%%%%%%%%%%%%%%%%%%%%%%%%%%%%%%%%%%%%%%%%%%%%%%%%%%%%%
\subsection{Flags}
\label{sec:flags}

The package makes it easy to generate different versions
of the main or child documents.
To this end compilation flags can be defined
and assigned different default values.
They will be particularly useful in conjunction
with the forwarding mechanism described in \secref{sec:forward}.

For example, it may be useful to have a flag |\version|
which can be set to |draft| or |final|.
The document source will contain some conditional code
depending on the value of |\version|.
Suppose further, the flag should default to |final| for the main file
and to |draft| for child files
which is a natural assignment for editing the document.
This is achieved by placing the following code
in the preamble of the main document
(below the |\childdocmain| directive):
%
\begin{center}
\begin{tabular}{l}
|\ifchilddoc|\\
|\providecommand{\version}{draft}|\\
|\||else|\\
|\providecommand{\version}{final}|\\
|\||fi|
\end{tabular}
\end{center}
%
The definition by |\providecommand| makes sure
that previous definitions are not overwritten.
Further statements |\providecommand{\version}{...}|
can thus be added before the above code to override it.

For the main file, one might add a line
(between |\childdocmain| and the above block)
%
\begin{center}
|%\ifchilddoc\||else\providecommand{\version}{draft}\||fi|
\end{center}
%
which can be uncommented to produce a draft version.
Likewise one can add a line to the very top of a child file
(above the |\childdocof{|\textit{main}|}| directive)
%
\begin{center}
|%\providecommand{\version}{final}|
\end{center}
%
which can be uncommented to produce the final version of this child document.

%%%%%%%%%%%%%%%%%%%%%%%%%%%%%%%%%%%%%%%%%%%%%%%%%%%%%%%%%%%%%%%%%%%%%%%%%%%%%%%%
\subsection{Forwarding}
\label{sec:forward}

Different versions of the main or child documents
using compilation flags as described in \secref{sec:flags}
can be (permanently) stored in different files
for convenient compilation, viewing and distribution.
To this end, the package defines a command
to pass on compilation to a different file:

%%%%%%%%%%%%%%%%%%%%%%%%%%%%%%%%%%%%%%%%
\DescribeMacro{\childdocforward}
The command |\childdocforward| redirects processing to
another source file:
%
\begin{center}
\begin{tabular}{l}
|\input{childdoc.def}|\\
|\childdocforward[|\textit{main}|]{|\textit{dest}|}|\\
\end{tabular}
\end{center}
%
The argument \textit{dest} is the destination file
(without extension).
It should be the main file or one of the child files.
Note that further \textsf{childdoc} directives
such as |\childdocof| and |\childdocforward|
in the indicated file will be processed in this form.
The optional argument \textit{main}
passes on directly to the main file \textit{main}
while pretending to compile the child \textit{dest}.
This form behaves as if \textit{dest}
issues |\childdocof{|\textit{main}|}| right away,
and no further \textsf{childdoc} directives will be processed.

%%%%%%%%%%%%%%%%%%%%%%%%%%%%%%%%%%%%%%%%
\DescribeMacro{\...prefix}
In the alternative form |\childdocforwardprefix|,
%
\begin{center}
\begin{tabular}{l}
|\input{childdoc.def}|\\
|\childdocforwardprefix[|\textit{main}|]{|\textit{prefix}|}{|\textit{dest}|}|
\end{tabular}
\end{center}
%
the destination file is determined by a pattern
depending on the current file:
To make this work, the current file must be called
`{\textit{prefix}\hspace{0.2em}\textit{suffix}}'
with \textit{prefix} matching precisely the argument.
Processing is then passed on to the file
`{\textit{dest}\hspace{0.2em}\textit{suffix}}'.
Surely, the same effect is achieved by
directly specifying the
argument `{\textit{dest}\hspace{0.2em}\textit{suffix}}'
in the first form.
However, that requires to set up a different file
for each child. With the alternative form of the command
all these files can have exactly the same content
which simplifies setting them up and maintaining them.

For example, the following file |draft.tex|
with a compilation flag |\version| as described in \secref{sec:flags}
compiles the main document as a draft:
%
\begin{center}
\begin{tabular}{l}
|\def\version{draft}|\\
|\input{childdoc.def}|\\
|\childdocforward{|\textit{main}|}|
\end{tabular}
\end{center}
%
Likewise, the following files |final|\textit{nn}|.tex|
compile the final version of the child document
|child|\textit{nn}|.tex|:
%
\begin{center}
\begin{tabular}{l}
|\def\version{final}|\\
|\input{childdoc.def}|\\
|\childdocforwardprefix{final}{child}|
\end{tabular}
\end{center}
%

Note that when several versions of a main file and/or of each child file
are to be generated, it may be convenient to set up a |Makefile| or
shell script to automatise the process.

%%%%%%%%%%%%%%%%%%%%%%%%%%%%%%%%%%%%%%%%%%%%%%%%%%%%%%%%%%%%%%%%%%%%%%%%%%%%%%%%
\subsection{Command Line Processing}
\label{sec:commandline}

The effect of redirection files can also be achieved by invoking
the \LaTeX{} compiler with a more elaborate command line.
Most conveniently this should be done as part
of a shell script or a |Makefile|.

When using \textsf{childdoc} in the main file, the following
command lines effectively perform a redirection
(note that depending on the shell being used,
backslashes may have to be doubled: `|\|' $\to$ `|\\|'):
%
\begin{center}
|... -jobname "|\textit{target}|" |\\|"|[\textit{flags}]%
|\input{childdoc.def}\childdocforward[|\textit{main}|]{|\textit{dest}|}"|
\end{center}
%
Here \textit{target} is the name of the output file,
\textit{main} is the name of the main file
and \textit{dest} is the name of the main or child file to be processed
(all filenames without extensions).
The optional argument \textit{main} can be omitted
if \textit{main} matches \textit{dest}.
Optionally, compilation \textit{flags} can be defined via |\def| commands.
This command line makes the \TeX{} engine believe
it is compiling the file \textit{target}
whose content is specified as the latter parameter.
The provided code then forwards the processing to
\textit{main} or \textit{dest} as described in \secref{sec:forward}.

%%%%%%%%%%%%%%%%%%%%%%%%%%%%%%%%%%%%%%%%%%%%%%%%%%%%%%%%%%%%%%%%%%%%%%%%%%%%%%%%
\subsection{Include by Input}
\label{sec:input}

Including child documents by |\include| has some restrictions by design.
Most notably, the content of a child document always occupies
its own set of pages; pages cannot be shared between child documents.
Usually, this behaviour makes perfect sense
because each child document contain an essential part of the document.
However, in some situations it may be desirable to compose
a document from a collection of parts
without having mandatory page breaks between then.
For this case, the package
provides a mechanism to include parts
by |\input| which can also be processed individually.
However, by construction this mechanism
requires manual handling of the content to be output.

%%%%%%%%%%%%%%%%%%%%%%%%%%%%%%%%%%%%%%%%
\DescribeMacro{\ifchilddocmanual}
The main file should be prepared as usual, see \secref{sec:include}.
However, the document body must make a distinction
between processing of an individual part and of the main document, e.g.:
%
\begin{center}
\begin{tabular}{l}
|\ifchilddocmanual|\\
|\input{\childdocname}|\\
|\||else|\\
\textit{document body with }|\input{|\textit{part}|}|\\
|\||fi|
\end{tabular}
\end{center}
%
The conditional |\ifchilddocmanual| is true whenever
a part to be included by |\input| is being compiled,
and the name of the part is stored in |\childdocname|.

%%%%%%%%%%%%%%%%%%%%%%%%%%%%%%%%%%%%%%%%
\DescribeMacro{\childdocby}
Each part to be included by |\input| should start with:
%
\begin{center}
\begin{tabular}{l}
|\input{childdoc.def}|\\
|\childdocby{|\textit{main}|}|\\
\end{tabular}
\end{center}
%
The directive |\childdocby| is similar to |\childdocof|
described in \secref{sec:include},
but the subsequent selection of content must be done manually.
To that end, both |\ifchilddoc| and |\ifchilddocmanual|
will be true upon processing of a part,
and the name of the part is stored in |\childdocname|.
Note that |\jobname| will be set to the filename of the current part
so that each part receives an individual |.aux| file
that does not interfere with the |.aux| file(s) of the main document.
This behaviour can be altered by the alternative form
|\childdocby[*]{|\textit{main}|}| (with a non-empty optional argument)
which uses the |.aux| file of the main document
by setting |\jobname| to \textit{main}.

%%%%%%%%%%%%%%%%%%%%%%%%%%%%%%%%%%%%%%%%%%%%%%%%%%%%%%%%%%%%%%%%%%%%%%%%%%%%%%%%
\subsection{Driver Development}
\label{sec:driver}

The \textsf{childdoc} mechanism can also be use for the development
of definition files such as \LaTeX{} styles or classes.
This case differs from the above setup with multiple parts
included by |\include| in that no |\includeonly| should be invoked.
This can be achieved by starting the include file
(before |\ProvidesPackage|) with:
%
\begin{center}
\begin{tabular}{l}
|\input{childdoc.def}|\\
|\childdocforward{|\textit{main}|}|\\
\end{tabular}
\end{center}
%
or alternatively with:
%
\begin{center}
\begin{tabular}{l}
|\input{childdoc.def}|\\
|\childdocby{|\textit{main}|}|\\
\end{tabular}
\end{center}
%
Both forms have slightly different effects as described above.
The main file is prepared as usual, see \secref{sec:include}.

%%%%%%%%%%%%%%%%%%%%%%%%%%%%%%%%%%%%%%%%%%%%%%%%%%%%%%%%%%%%%%%%%%%%%%%%%%%%%%%%
\subsection{Legacy Detection}
\label{sec:detection}

The directive |\childdocmain| in the main file can detect
whether the complete document or merely a child is to be compiled
even without using the directive |\childdocof|.
This method is deprecated because it is less robust
and there is no compelling reason to use it;
it is merely provided for backward compatibility
and it may be removed in future versions.

If the detection mechanism is to be used,
it is mandatory to correctly specify
the filename of the main file as the argument of |\childdocmain|:
%
\begin{center}
\begin{tabular}{l}
|\input{childdoc.def}|\\
|\childdocmain{|\textit{main}|}|\\
\end{tabular}
\end{center}
%
If |\jobname| does not match the argument \textit{main} of |\childdocmain|,
it is assumed that |\jobname| points to the child file to be compiled.
When using |\childdocmain| with the main file specified as argument,
it suffices to start a child file
with just |\input{|\textit{main}|}|
without loading of the package and using |\childdocof|.
If instead all processing is done
with the appropriate \textsf{childdoc} directives,
the argument of \textit{main} of |\childdocmain| can be empty.

An alternative version of the command line processing described
in \secref{sec:commandline} using the detection mechanism reads:
%
\begin{center}
|... -jobname "|\textit{target}|" "|[\textit{flags}]%
[|\def\jobname{|\textit{dest}|}|]|\input{|\textit{main}|}"|
\end{center}

%%%%%%%%%%%%%%%%%%%%%%%%%%%%%%%%%%%%%%%%%%%%%%%%%%%%%%%%%%%%%%%%%%%%%%%%%%%%%%%%
\subsection{Manual Code}
\label{sec:manual}

In case one cannot be certain whether the definitions file |childdoc.def|
is installed on the target \TeX{} distribution
and one prefers not to ship it,
it is conceivable to paste a few relevant commands into the sources.

To that end, drop all statements |\input{childdoc.def}|
and perform the replacements as outlined below.
Instead of |\childdocmain{|\textit{main}|}| add the following code
to the top of the main file:
%
\begin{center}
\begin{tabular}{l}
|\||ifdefined\childdocname\endinput\||fi\newif\ifchilddoc|\\
|\edef\childdocname{\scantokens\expandafter{\jobname\noexpand}}|\\
|\def\childdocmain{|\textit{main}|}\||ifx\childdocmain\childdocname\||else|\\
|\childdoctrue\includeonly{\childdocname}\let\jobname\childdocmain\||fi|\\
\end{tabular}
\end{center}
%
Instead of |\childdocof{|\textit{main}|}| just include the main file
at the top of each child file:
%
\begin{center}
|\input{|\textit{main}|}|
\end{center}
%
A simple redirection |\childdocforward{|\textit{dest}|}| is achieved by:
%
\begin{center}
|\def\jobname{|\textit{dest}|}\input{\jobname}|
\end{center}
%
The redirection with prefix
|\childdocforwardprefix[|\textit{prefix}|]{|\textit{dest}|}|
is accomplished by:
%
\begin{center}
\begin{tabular}{l}
|{\edef\jobname{\scantokens\expandafter{\jobname\noexpand}}|\\
|\def\redirectjob |\textit{prefix}|#1~~~{\gdef\jobname{|\textit{dest}|#1}}|\\
|\expandafter\redirectjob\jobname~~~}\input{\jobname}|
\end{tabular}
\end{center}

In an alternative approach,
child documents can be compiled by a specific command line
without additional code or specific definitions:
%
\begin{center}
|... -jobname "|\textit{target}|" "|[\textit{flags}]%
|\includeonly{|\textit{dest}|}\input{|\textit{main}|}"|
\end{center}
%

%%%%%%%%%%%%%%%%%%%%%%%%%%%%%%%%%%%%%%%%%%%%%%%%%%%%%%%%%%%%%%%%%%%%%%%%%%%%%%%%
%%%%%%%%%%%%%%%%%%%%%%%%%%%%%%%%%%%%%%%%%%%%%%%%%%%%%%%%%%%%%%%%%%%%%%%%%%%%%%%%
\section{Information}

%%%%%%%%%%%%%%%%%%%%%%%%%%%%%%%%%%%%%%%%%%%%%%%%%%%%%%%%%%%%%%%%%%%%%%%%%%%%%%%%
\subsection{Copyright}

Copyright \copyright{} 2017--2018 Niklas Beisert

This work may be distributed and/or modified under the
conditions of the \LaTeX{} Project Public License, either version 1.3
of this license or (at your option) any later version.
The latest version of this license is in
  \url{http://www.latex-project.org/lppl.txt}
and version 1.3 or later is part of all distributions of \LaTeX{}
version 2005/12/01 or later.

This work has the LPPL maintenance status `maintained'.

The Current Maintainer of this work is Niklas Beisert.

This work consists of the files |README.txt|, |childdoc.ins| and |childdoc.dtx|
as well as the derived files |childdoc.def|, |cdocsamp.tex|
with |cdocsch1.tex|, |cdocsch2.tex|, |cdocspt3.tex|, |cdocspt4.tex|,
|cdocsdrf.tex|, |cdocsfn1.tex|, |cdocsfn2.tex|
as well as |childdoc.pdf|.

%%%%%%%%%%%%%%%%%%%%%%%%%%%%%%%%%%%%%%%%%%%%%%%%%%%%%%%%%%%%%%%%%%%%%%%%%%%%%%%%
\subsection{Files and Installation}

The package consists of the files:
%
\begin{center}
\begin{tabular}{ll}
    |README.txt|   & readme file \\
    |childdoc.ins| & installation file \\
    |childdoc.dtx| & source file \\
    |childdoc.def| & definition file \\
    |cdocsamp.tex| & sample main file \\
    |cdocsch1.tex| & sample include file \\
    |cdocsch2.tex| & sample include file \\
    |cdocspt3.tex| & sample part file \\
    |cdocspt4.tex| & sample part file \\
    |cdocsdrf.tex| & sample redirection file \\
    |cdocsfn1.tex| & sample redirection file \\
    |cdocsfn2.tex| & sample redirection file \\
    |childdoc.pdf| & manual
\end{tabular}
\end{center}
%
The distribution consists of the files
|README.txt|, |childdoc.ins| and |childdoc.dtx|.
%
\begin{itemize}
\item
Run (pdf)\LaTeX{} on |childdoc.dtx|
to compile the manual |childdoc.pdf| (this file).
\item
Run \LaTeX{} on |childdoc.ins| to create the definitions file |childdoc.def|
and the sample |cdocsamp.tex| with include files
|cdocsch1.tex|, |cdocsch2.tex|, |cdocspt3.tex|, |cdocspt4.tex|,
|cdocsdrf.tex|, |cdocsfn1.tex|, |cdocsfn2.tex|.
Then copy the file |childdoc.def| to an appropriate directory of your \LaTeX{}
distribution, e.g.\ \textit{texmf-root}|/tex/latex/childdoc|.
\end{itemize}

%%%%%%%%%%%%%%%%%%%%%%%%%%%%%%%%%%%%%%%%%%%%%%%%%%%%%%%%%%%%%%%%%%%%%%%%%%%%%%%%
\subsection{Related CTAN Packages}

There are several other packages which offer a similar functionality:
%
\begin{itemize}
\item
The packages
\href{http://ctan.org/pkg/docmute}{\textsf{docmute}},
\href{http://ctan.org/pkg/includex}{\textsf{includex}} and
\href{http://ctan.org/pkg/standalone}{\textsf{standalone}}
provide commands to include only the document body of
a child file thus allowing both files to be compiled individually.
\item
The packages \href{http://ctan.org/pkg/subdocs}{\textsf{subdocs}}
and \href{http://ctan.org/pkg/subfiles}{\textsf{subfiles}}
provide structures in which the main and child documents can be
encapsulated and allowing them to be compiled individually.
The inclusion mechanism is different from the conventional |\include|.
\item
The package \href{http://ctan.org/pkg/combine}{\textsf{combine}}
is an elaborate solution to combine several documents into one.
\end{itemize}
%
See also the CTAN topic \href{http://ctan.org/topic/subdocs}{\textsf{subdocs}}
for further related packages.
The present package differs from the above solutions in that
a document structure constructed with the conventional |\include| mechanism
just needs two extra commands at the top of every file
such that all constituent files can be compiled individually.

%%%%%%%%%%%%%%%%%%%%%%%%%%%%%%%%%%%%%%%%%%%%%%%%%%%%%%%%%%%%%%%%%%%%%%%%%%%%%%%%
%\subsection{Feature Suggestions}
%
%The following is a list of features which may be useful for future
%versions of this package:
%%
%\begin{itemize}
%\item
%\ldots
%\end{itemize}

%%%%%%%%%%%%%%%%%%%%%%%%%%%%%%%%%%%%%%%%%%%%%%%%%%%%%%%%%%%%%%%%%%%%%%%%%%%%%%%%
\subsection{Revision History}

%%%%%%%%%%%%%%%%%%%%%%%%%%%%%%%%%%%%%%%%
\paragraph{v2.0:} 2018/12/30

\begin{itemize}
\item
immediate forward processing
\item
added |\childdocby| mechanism
\item
manual restructured
\end{itemize}

%%%%%%%%%%%%%%%%%%%%%%%%%%%%%%%%%%%%%%%%
\paragraph{v1.6:} 2018/01/17

\begin{itemize}
\item
application for development of include files
\item
corrections to manual
\end{itemize}

%%%%%%%%%%%%%%%%%%%%%%%%%%%%%%%%%%%%%%%%
\paragraph{v1.5:} 2017/05/21

\begin{itemize}
\item
more complete structuring introduced
\item
|\childdocof| introduced
\item
|\childdoc| renamed to |\childdocmain|
\item
|\childredirect| renamed to |\childdocforward| and |\childdocforwardprefix|
and functionality expanded
\end{itemize}

%%%%%%%%%%%%%%%%%%%%%%%%%%%%%%%%%%%%%%%%
\paragraph{v1.0:} 2017/04/27

\begin{itemize}
\item
manual and install package
\item
first version published on CTAN
\end{itemize}

%%%%%%%%%%%%%%%%%%%%%%%%%%%%%%%%%%%%%%%%
\paragraph{v0.6:} 2017/04/26

\begin{itemize}
\item
redirection mechanism added
\end{itemize}

%%%%%%%%%%%%%%%%%%%%%%%%%%%%%%%%%%%%%%%%
\paragraph{v0.5:} 2017/04/26

\begin{itemize}
\item
functionality in definition file
\end{itemize}


%%%%%%%%%%%%%%%%%%%%%%%%%%%%%%%%%%%%%%%%%%%%%%%%%%%%%%%%%%%%%%%%%%%%%%%%%%%%%%%%
%%%%%%%%%%%%%%%%%%%%%%%%%%%%%%%%%%%%%%%%%%%%%%%%%%%%%%%%%%%%%%%%%%%%%%%%%%%%%%%%
%%%%%%%%%%%%%%%%%%%%%%%%%%%%%%%%%%%%%%%%%%%%%%%%%%%%%%%%%%%%%%%%%%%%%%%%%%%%%%%%
\appendix

\settowidth\MacroIndent{\rmfamily\scriptsize 000\ }

 \DocInput{childdoc.dtx}

\end{document}
%</driver>
% \fi
%
% %%%%%%%%%%%%%%%%%%%%%%%%%%%%%%%%%%%%%%%%%%%%%%%%%%%%%%%%%%%%%%%%%%%%%%%%%%%%%%
% %%%%%%%%%%%%%%%%%%%%%%%%%%%%%%%%%%%%%%%%%%%%%%%%%%%%%%%%%%%%%%%%%%%%%%%%%%%%%%
% \section{Sample}
%\iffalse
%<*samplemain>
%\fi
%
% The following presents a sample document
% with two chapters, two parts, a title page,
% a compile flag as well as three forwarding files to set the flag.
% It consists of eight |.tex| files:
% \begin{center}
% \begin{tabular}{ll}
% |cdocsamp.tex|&main file\\
% |cdocsch1.tex|&include file for chapter 1\\
% |cdocsch2.tex|&include file for chapter 2\\
% |cdocspt3.tex|&include file for part 3\\
% |cdocspt4.tex|&include file for part 4\\
% |cdocsdrf.tex|&forwarding file for main file in draft mode\\
% |cdocsfi1.tex|&forwarding file for final version of chapter 1\\
% |cdocsfi2.tex|&forwarding file for final version of chapter 2\\
% \end{tabular}
% \end{center}
% Each of the eight files can be compiled directly by the \LaTeX{} compiler.
%
% %%%%%%%%%%%%%%%%%%%%%%%%%%%%%%%%%%%%%%
% \paragraph{Main File.}
%
% The main file is called |cdocsamp.tex|.
%
% Load the \textsf{childdoc} definitions and
% declare the filename for the main document:
%    \begin{macrocode}
\input{childdoc.def}
\childdocmain{}
%    \end{macrocode}

% Optional override for |\version| flag:
%    \begin{macrocode}
%%\ifchilddoc\else\providecommand{\version}{draft}\fi
%    \end{macrocode}

% Define the default values for the |\version| flag
% (|final| for the main file and |draft| for childs):
%    \begin{macrocode}
\ifchilddoc
\providecommand{\version}{draft}
\else
\providecommand{\version}{final}
\fi
%    \end{macrocode}

% Load the standard document class:
%    \begin{macrocode}
\documentclass[12pt]{article}
%    \end{macrocode}

% Start the document body:
%    \begin{macrocode}
\begin{document}
%    \end{macrocode}

% Declare a title page.
% Print title, part of document being processed and version flag:
%    \begin{macrocode}
\addtocounter{page}{-1}
\begin{center}
{\LARGE\bfseries{}childdoc example\par}
\vspace{1cm}
\ifchilddoc
\ifchilddocmanual part\else chapter\fi:
`\childdocname' of `\childdocjob'\par
\else
main document: `\childdocjob'\par
\fi
version: \version\par
\end{center}
\newpage
%    \end{macrocode}

% Manually include selected file,
% otherwise process as usual:
%    \begin{macrocode}
\ifchilddocmanual
\section*{part `\childdocname'}
\input{\childdocname}
\else
%    \end{macrocode}

% Include the two chapters:
%    \begin{macrocode}
\include{cdocsch1}
\include{cdocsch2}
%    \end{macrocode}

% Include the two parts unless only chapters should be displayed:
%    \begin{macrocode}
\ifchilddoc\else
\section{part three}
\input{cdocspt3}
\section{part four}
\input{cdocspt4}
\fi
%    \end{macrocode}

% Process as usual until here:
%    \begin{macrocode}
\fi
%    \end{macrocode}

% End of document body:
%    \begin{macrocode}
\end{document}
%    \end{macrocode}
%\iffalse
%</samplemain>
%\fi
%
% %%%%%%%%%%%%%%%%%%%%%%%%%%%%%%%%%%%%%%
% \paragraph{Chapter Include Files.}
%
% The include files are called |cdocsch1.tex| and |cdocsch2.tex|.
%
%\iffalse
%<*samplechap1|samplechap2>
%\fi

% Optional override for |\version| flag:
%    \begin{macrocode}
%%\providecommand{\version}{final}
%    \end{macrocode}

% Include the main document:
%    \begin{macrocode}
\input{childdoc.def}
\childdocof{cdocsamp}
%    \end{macrocode}

%\iffalse
%</samplechap1|samplechap2>
%\fi
%
%\iffalse
%<*samplechap1>
%\fi
% Some text for chapter 1:
%    \begin{macrocode}
\section{one}
some text in chapter one
%    \end{macrocode}

%\iffalse
%</samplechap1>
%\fi
% Some text for chapter 2:
%\iffalse
%<*samplechap2>
%\fi
%    \begin{macrocode}
\section{two}
more text in chapter two
%    \end{macrocode}

%\iffalse
%</samplechap2>
%\fi
%
% %%%%%%%%%%%%%%%%%%%%%%%%%%%%%%%%%%%%%%
% \paragraph{Part Include Files.}
%
% The include files are called |cdocspt3.tex| and |cdocspt4.tex|.
%
%\iffalse
%<*samplepart3|samplepart4>
%\fi

% Optional override for |\version| flag:
%    \begin{macrocode}
%%\providecommand{\version}{final}
%    \end{macrocode}

% Include the main document:
%    \begin{macrocode}
\input{childdoc.def}
\childdocby{cdocsamp}
%    \end{macrocode}

%\iffalse
%</samplepart3|samplepart4>
%\fi
%
%\iffalse
%<*samplepart3>
%\fi
% Some text for part 3:
%    \begin{macrocode}
some text in part three
%    \end{macrocode}

%\iffalse
%</samplepart3>
%\fi
% Some text for part 4:
%\iffalse
%<*samplepart4>
%\fi
%    \begin{macrocode}
more text in part four
%    \end{macrocode}

%\iffalse
%</samplepart4>
%\fi
%
% %%%%%%%%%%%%%%%%%%%%%%%%%%%%%%%%%%%%%%
% \paragraph{Forwarding for a Complete Draft.}
%
% The following forwarding file |cdocsdrf.tex|
% compiles the main document in draft mode:
%\iffalse
%<*sampledraft>
%\fi
%    \begin{macrocode}
\def\version{draft}
\input{childdoc.def}
\childdocforward{cdocsamp}
%    \end{macrocode}

%\iffalse
%</sampledraft>
%\fi
%
% %%%%%%%%%%%%%%%%%%%%%%%%%%%%%%%%%%%%%%
% \paragraph{Forwarding for Final Version of the Chapters.}
%
% The following forwarding files |cdocsfn1.tex| and |cdocsfn2.tex|
% (with identical content)
% compile the final versions of the child documents
% |cdocsch1.tex| and |cdocsch2.tex|, respectively:
%\iffalse
%<*samplefinal>
%\fi
%    \begin{macrocode}
\def\version{final}
\input{childdoc.def}
\childdocforwardprefix[cdocsamp]{cdocsfn}{cdocsch}
%    \end{macrocode}

%\iffalse
%</samplefinal>
%\fi
%
% %%%%%%%%%%%%%%%%%%%%%%%%%%%%%%%%%%%%%%
% \paragraph{Command Line Processing.}
%
% The following three command lines generate the output files
% |cdocscld|, |cdocscl1| and |cdocscl2|
% which should be identical to
% |cdocsdrf|, |cdocsch1| and |cdocsfn2|, respectively:
% \begin{center}
% \begin{tabular}{l}
% |latex -jobname cdocscld \|\\
% |  "\def\version{draft}\input{childdoc.def}\childdocforward{cdocsamp}"|\\
% |latex -jobname cdocscl1 \|\\
% |  "\input{childdoc.def}\childdocforward[cdocsamp]{cdocsch1}"|\\
% |latex -jobname cdocscl2 \|\\
% |  "\def\version{final}\input{childdoc.def}\childdocforward{cdocsch2}"|
% \end{tabular}
% \end{center}
% Note that the trailing backslash on each first line
% merely continues the input to the second line
% (for convenient cut ant paste).
% Furthermore, the command |latex| can be replaced by any
% of its alternative versions such as |pdflatex|.
%
% %%%%%%%%%%%%%%%%%%%%%%%%%%%%%%%%%%%%%%%%%%%%%%%%%%%%%%%%%%%%%%%%%%%%%%%%%%%%%%
% %%%%%%%%%%%%%%%%%%%%%%%%%%%%%%%%%%%%%%%%%%%%%%%%%%%%%%%%%%%%%%%%%%%%%%%%%%%%%%
% \section{Implementation}
%\iffalse
%<*package>
%\fi
%
% This section describes the definitions file |childdoc.def|.

% The definitions cannot be loaded using |\usepackage| or |\RequirePackage|
% which has a mechanism to prevent loading a style file more than once.
% When loading the definitions by means of |\input|
% multiple instances have to be prevented manually:
%\iffalse
%This code needs to be before the `\ProvidesFile' directive
%which is defined at the beginning of this file.
%Therefore it is also placed there and commented out here.
%</package>
%<*discard>
%\fi
%    \begin{macrocode}
\ifdefined\childdocmain\endinput\fi
%    \end{macrocode}
%\iffalse
%</discard>
%<*package>
%\fi
%
% \macro{\ifchilddoc}
% \macro{\ifchilddocmanual}
% The conditional |\ifchilddoc| tells whether a
% child (true) or main (false) document is being compiled.
% The conditional |\ifchilddocmanual| tells whether
% the |\includeonly| mechanism is used (false) or
% the selection of child files must be performed manually (true).
% The definitions initialise to false:
%    \begin{macrocode}
\newif\ifchilddoc
\newif\ifchilddocmanual
%    \end{macrocode}

% \macro{\childdocname}
% \macro{\childdocjob}
% The macro |\childdocname| stores the name of the main document
% to be compiled. The macro |\childdocjob| stores the name of
% the document on which the \LaTeX{} compiler was originally invoked.
% The content of |\jobname| cannot be compared
% to filenames specified in the source due to different catcodes.
% The following code rescans |\jobname|, stores the result
% in |\childdocname| and saves a copy in |\childdocjob|:
%    \begin{macrocode}
\edef\childdocname{\scantokens\expandafter{\jobname\noexpand}}
\let\childdocjob\childdocname
%    \end{macrocode}

% \macro{\childdocdisable}
% The macro |\childdocdisable| prevents the main file
% from being processed more than once.
% At this stage, the main document command |\childdocmain|
% is assumed to be called once again where it should do nothing.
% Any subsequent call to it should prevent
% a secondary processing of the main document
% It overwrites the forwarding commands
% |\childdocof| and |\childdocforward|
% with empty macros to prevent further inclusions of the main document:
%    \begin{macrocode}
\newcommand{\childdocdisable}
{
  \renewcommand{\childdocmain}[1]{\renewcommand{\childdocmain}[1]{\endinput}}
  \renewcommand{\childdocof}[1]{}
  \renewcommand{\childdocby}[2][]{}
  \renewcommand{\childdocforward}[2][]{}
  \renewcommand{\childdocdisable}{}
}
%    \end{macrocode}

% \macro{\childdocmain}
% The macro |\childdocmain| is to be called at the top of the main file
% with nothing or the main filename (without extension) as argument.
% First, it breaks loops.
% If the argument is not empty and does not match |\childdocname|
% (which is set by the first inclusion of |childdoc.def|),
% |\ifchilddoc| is set to true, |\includeonly| is applied to the child file
% and |\jobname| is set to the main file
% (for proper handling of |.aux| files):
%    \begin{macrocode}
\newcommand{\childdocmain}[1]
{
  \childdocdisable\childdocmain{}
  \if?#1?\else
    \begingroup
      \def\childdoctmp{#1}
      \ifx\childdoctmp\childdocname
        \def\childdoctmp{}
      \else
        \def\childdoctmp
        {
          \childdoctrue
          \includeonly{\childdocname}
          \def\childdocjob{#1}
          \def\jobname{#1}
        }
      \fi
      \expandafter
    \endgroup
    \childdoctmp
  \fi
}
%    \end{macrocode}

% \macro{\childdocof}
% The command |\childdocof| redirects
% compilation to the main file |#1|.
%    \begin{macrocode}
\newcommand{\childdocof}[1]
{
  \childdocdisable
  \childdoctrue
  \includeonly{\childdocname}
  \def\jobname{#1}
  \def\childdocjob{#1}
  \input{#1}
}
%    \end{macrocode}

% \macro{\childdocby}
% The command |\childdocby| ....
%    \begin{macrocode}
\newcommand{\childdocby}[2][]
{
  \childdocdisable
  \childdoctrue
  \childdocmanualtrue
  \if?#1?\else
    \def\jobname{#2}
  \fi
  \def\childdocjob{#2}
  \input{#2}
  \endinput
}
%    \end{macrocode}

% \macro{\childdocforward}
% The command |\childdocforward| redirects
% compilation to the main file or
% (if the optional argument is given) a child file.
% Parameters are set as if the main file
% or a child file starting with |\childdocof| was compiled.
% Then compilation is handed over to the main file:
%    \begin{macrocode}
\newcommand{\childdocforward}[2][]
{
  \begingroup
    \if?#1?
      \def\childdoctmp
      {
        \def\childdocname{#2}
        \def\childdocjob{#2}
        \def\jobname{#2}
        \input{#2}
        \endinput
      }
    \else
      \def\childdoctmp
      {
        \childdocdisable
        \def\childdocname{#2}
        \childdoctrue
        \includeonly{#2}
        \def\childdocjob{#1}
        \def\jobname{#1}
        \input{#1}
        \endinput
      }
    \fi
    \expandafter
  \endgroup
  \childdoctmp
}
%    \end{macrocode}

% \macro{\childdocforwardprefix}
% The command |\childdocforwardprefix| redirects
% compilation to the main or a child file by means of a pattern.
% The prefix |#1| in the current filename is replaced by |#2|
% and the suffix of the current filename is kept
% (it is assumed that the filename does not contain the substring `|~~~|'
% which is used as a delimiter).
% Compilation is handed over to the new file by |\childdocforward|:
%    \begin{macrocode}
\newcommand{\childdocforwardprefix}[3][]
{
  \begingroup
    \def\childdocextract #2##1~~~{\def\childdoctmp{\childdocforward[#1]{#3##1}}}
    \expandafter\childdocextract\childdocname~~~
    \expandafter
  \endgroup
  \childdoctmp
}
%    \end{macrocode}

% \macro{\childdoc}
% The deprecated macro |\childdoc| is a legacy version of |\childdocmain|:
%    \begin{macrocode}
\newcommand{\childdoc}{\childdocmain}
%    \end{macrocode}

% \macro{\childdocredirect}
% The deprecated macro |\childdocredirect| is a legacy version
% of |\childdocforward| and |\childdocforwardprefix|:
%    \begin{macrocode}
\newcommand{\childdocredirect}[2][]
{
  \begingroup
    \if?#1?
      \def\childdoctmp{\childdocforward{#2}}
    \else
      \def\childdoctmp{\childdocforwardprefix{#1}{#2}}
    \fi
    \expandafter
  \endgroup
  \childdoctmp
}
%    \end{macrocode}

%\iffalse
%</package>
%\fi
%
\endinput
\childdocforward[|\textit{main}|]{|\textit{dest}|}"|
\end{center}
%
Here \textit{target} is the name of the output file,
\textit{main} is the name of the main file
and \textit{dest} is the name of the main or child file to be processed
(all filenames without extensions).
The optional argument \textit{main} can be omitted
if \textit{main} matches \textit{dest}.
Optionally, compilation \textit{flags} can be defined via |\def| commands.
This command line makes the \TeX{} engine believe
it is compiling the file \textit{target}
whose content is specified as the latter parameter.
The provided code then forwards the processing to
\textit{main} or \textit{dest} as described in \secref{sec:forward}.

%%%%%%%%%%%%%%%%%%%%%%%%%%%%%%%%%%%%%%%%%%%%%%%%%%%%%%%%%%%%%%%%%%%%%%%%%%%%%%%%
\subsection{Include by Input}
\label{sec:input}

Including child documents by |\include| has some restrictions by design.
Most notably, the content of a child document always occupies
its own set of pages; pages cannot be shared between child documents.
Usually, this behaviour makes perfect sense
because each child document contain an essential part of the document.
However, in some situations it may be desirable to compose
a document from a collection of parts
without having mandatory page breaks between then.
For this case, the package
provides a mechanism to include parts
by |\input| which can also be processed individually.
However, by construction this mechanism
requires manual handling of the content to be output.

%%%%%%%%%%%%%%%%%%%%%%%%%%%%%%%%%%%%%%%%
\DescribeMacro{\ifchilddocmanual}
The main file should be prepared as usual, see \secref{sec:include}.
However, the document body must make a distinction
between processing of an individual part and of the main document, e.g.:
%
\begin{center}
\begin{tabular}{l}
|\ifchilddocmanual|\\
|\input{\childdocname}|\\
|\||else|\\
\textit{document body with }|\input{|\textit{part}|}|\\
|\||fi|
\end{tabular}
\end{center}
%
The conditional |\ifchilddocmanual| is true whenever
a part to be included by |\input| is being compiled,
and the name of the part is stored in |\childdocname|.

%%%%%%%%%%%%%%%%%%%%%%%%%%%%%%%%%%%%%%%%
\DescribeMacro{\childdocby}
Each part to be included by |\input| should start with:
%
\begin{center}
\begin{tabular}{l}
|% \iffalse
%
% childdoc.dtx Copyright (C) 2017-2018 Niklas Beisert
%
% This work may be distributed and/or modified under the
% conditions of the LaTeX Project Public License, either version 1.3
% of this license or (at your option) any later version.
% The latest version of this license is in
%   http://www.latex-project.org/lppl.txt
% and version 1.3 or later is part of all distributions of LaTeX
% version 2005/12/01 or later.
%
% This work has the LPPL maintenance status `maintained'.
%
% The Current Maintainer of this work is Niklas Beisert.
%
% This work consists of the files childdoc.dtx and childdoc.ins
% and the derived files childdoc.def and cdocsamp.tex with
% cdocsch1.tex, cdocsch2.tex, cdocsdrf.tex, cdocsfn1.tex, cdocsfn2.tex.
%
%<package>\ifdefined\childdocmain\endinput\fi
%<package>\ProvidesFile{childdoc.def}[2018/12/30 v2.0 child document driver]
%<samplemain>\ProvidesFile{cdocsamp.tex}[2018/12/30 v2.0 sample for childdoc]
%<*driver>
%\ProvidesFile{childdoc.drv}[2018/12/30 v2.0 childdoc reference manual file]
\PassOptionsToClass{10pt,a4paper}{article}
\documentclass{ltxdoc}

\usepackage[margin=35mm]{geometry}
\usepackage{hyperref}
\usepackage{hyperxmp}
\usepackage[usenames]{color}

\hypersetup{colorlinks=true}
\hypersetup{pdfstartview=FitH}
\hypersetup{pdfpagemode=UseNone}
\hypersetup{pdfsource={}}
\hypersetup{pdflang={en-UK}}
\hypersetup{pdfcopyright={Copyright 2017-2018 Niklas Beisert.
  This work may be distributed and/or modified under the
  conditions of the LaTeX Project Public License, either version 1.3
  of this license or (at your option) any later version.}}
\hypersetup{pdflicenseurl={http://www.latex-project.org/lppl.txt}}
\hypersetup{pdfcontactaddress={ETH Zurich, ITP, HIT K,
  Wolfgang-Pauli-Strasse 27}}
\hypersetup{pdfcontactpostcode={8093}}
\hypersetup{pdfcontactcity={Zurich}}
\hypersetup{pdfcontactcountry={Switzerland}}
\hypersetup{pdfcontactemail={nbeisert@itp.phys.ethz.ch}}
\hypersetup{pdfcontacturl={http://people.phys.ethz.ch/\xmptilde nbeisert/}}

\newcommand{\secref}[1]{\hyperref[#1]{section \ref*{#1}}}

\parskip1ex
\parindent0pt
\let\olditemize\itemize
\def\itemize{\olditemize\parskip0pt}

\begin{document}

\title{The \textsf{childdoc} Package}
\hypersetup{pdftitle={The childdoc Package}}
\author{Niklas Beisert\\[2ex]
  Institut f\"ur Theoretische Physik\\
  Eidgen\"ossische Technische Hochschule Z\"urich\\
  Wolfgang-Pauli-Strasse 27, 8093 Z\"urich, Switzerland\\[1ex]
  \href{mailto:nbeisert@itp.phys.ethz.ch}
  {\texttt{nbeisert@itp.phys.ethz.ch}}}
\hypersetup{pdfauthor={Niklas Beisert}}
\hypersetup{pdfsubject={Manual for the LaTeX2e Package childdoc}}
\date{30 December 2018, \textsf{v2.0}}
\maketitle

\begin{abstract}\noindent
\textsf{childdoc} is a \LaTeXe{} package
that enables the direct compilation
of document sections included by |\include|
to individual files.
\end{abstract}

\begingroup
\parskip0ex
\tableofcontents
\endgroup

%%%%%%%%%%%%%%%%%%%%%%%%%%%%%%%%%%%%%%%%%%%%%%%%%%%%%%%%%%%%%%%%%%%%%%%%%%%%%%%%
%%%%%%%%%%%%%%%%%%%%%%%%%%%%%%%%%%%%%%%%%%%%%%%%%%%%%%%%%%%%%%%%%%%%%%%%%%%%%%%%
\section{Introduction}

\LaTeX{} provides a mechanism to structure a large document (such as a book)
into a main file and several child files (containing the chapters)
using the |\include| command.
This mechanism is beneficial for documents
which span hundreds of pages in order to
make the source file(s) more manageable.
Moreover, compilation can be restricted to
selected child files by means of the |\includeonly| command.
The latter feature can be used to reduce the compilation time while editing
(this was significantly more useful in the earlier days of \LaTeX{})
or to generate a smaller document which is easier to navigate.
Another application of |\includeonly| is to generate
documents consisting of selected parts of the complete document.

However, there are a few drawbacks of the plain |\include| mechanism:
\begin{itemize}
\item
The child files cannot be compiled on their own,
they can only be compiled via the main file.
A naive editing environment
(such as a text editor with an option
to have the current file processed by \LaTeX)
may require one to switch to the main file before compiling;
attempting to compile the child file produces errors.
\item
The main file must be modified (each time)
to adjust the |\includeonly| command
to the present needs. This easily leaves the main file in a messy state.
\item
The generated document will always carry the filename
of the main document. This is inconvenient if
several child files are to be compiled and
to be kept for distribution.
\end{itemize}

The present package provides a simple interface
to make child files individually compilable by \LaTeX{}.
Compiling a child file then has the same effect as compiling
the main file with an |\includeonly| command
to select the appropriate child.
Moreover the generated document will carry the name of the child
rather than the main file.
This resolves all three above issues.

This feature is meant to make the editing of books,
thesis documents and lecture notes somewhat more convenient.
However, the package can also be used efficiently for
composing a series of documents (such as exercise sheets)
which are typically distributed individually.
It then assists the author in generating the individual documents
(potentially in different versions)
as well as a document containing the collected series.
Another application is in developing style files
or other kinds of included material
where compilation of the style file could redirect
to a sample or test file.

%%%%%%%%%%%%%%%%%%%%%%%%%%%%%%%%%%%%%%%%%%%%%%%%%%%%%%%%%%%%%%%%%%%%%%%%%%%%%%%%
%%%%%%%%%%%%%%%%%%%%%%%%%%%%%%%%%%%%%%%%%%%%%%%%%%%%%%%%%%%%%%%%%%%%%%%%%%%%%%%%
\section{Usage}

First of all, the package \textsf{childdoc} is \emph{not} a standard
\LaTeXe{} |.sty| style file! Therefore it needs to be invoked in
a non-standard way.

%%%%%%%%%%%%%%%%%%%%%%%%%%%%%%%%%%%%%%%%%%%%%%%%%%%%%%%%%%%%%%%%%%%%%%%%%%%%%%%%
\subsection{Included Files}
\label{sec:include}

%%%%%%%%%%%%%%%%%%%%%%%%%%%%%%%%%%%%%%%%
\DescribeMacro{\childdocmain}
To use the package, add the commands
\begin{center}
\begin{tabular}{l}
|\input{childdoc.def}|\\
|\childdocmain{}|\\
\end{tabular}
\end{center}
at the very top of the main \LaTeX{} file,
in particular \emph{before} the |\documentclass| statement!
The argument of |\childdocmain| should be left empty
(but it must be present).

%%%%%%%%%%%%%%%%%%%%%%%%%%%%%%%%%%%%%%%%
\DescribeMacro{\childdocof}
Furthermore, add the commands
\begin{center}
\begin{tabular}{l}
|\input{childdoc.def}|\\
|\childdocof{|\textit{main}|}|\\
\end{tabular}
\end{center}
at the top of every child file \textit{child}
which is included by |\include{|\textit{child}|}|
from within the main file
(or at least for those files to be compiled individually).
The argument \textit{main} must be the filename of the main file.

There are a couple of
considerations in setting up the main and child documents:

%%%%%%%%%%%%%%%%%%%%%%%%%%%%%%%%%%%%%%%%
\paragraph{Restrictions.}

Please note the following restrictions:
\begin{itemize}
\item
|\childdocmain| must be called with one argument \textit{main}
to ensure compatibility with earlier version of the package.
It must either be empty (|\childdocmain{}|)
or precisely match the filename of the main file in which it is specified.
See \secref{sec:detection} for further information.
\item
The filename \textit{main} must be specified without the |.tex| extension.
\item
The filename \textit{main} is case sensitive
(even in case-insensitive file systems)
due to internal string comparison.
\item
The argument \textit{main} should be fully expanded, it cannot be a macro.
\item
Subdirectories and special characters should be avoided in filenames.
\item
The command |\childdocmain{|\textit{main}|}| must be followed by a whitespace.
It should not be followed immediately by another command
or by a comment mark `|%|'.
This is because the \TeX{} parser reads the token immediately following
the argument of |\childdocmain| and puts it
at the beginning of every child section;
however, a white\-space is ignored.
\end{itemize}

%%%%%%%%%%%%%%%%%%%%%%%%%%%%%%%%%%%%%%%%
\paragraph{Content of Main File.}

It is advisable to place all content in the child files included by |\include|.
Any output contained in the main file will appear in all child documents
unless suppressed manually;
it cannot be suppressed automatically by the |\includeonly| directive
and thus should normally be avoided.
A method to include some content in the main file
by means of conditional processing is described in \secref{sec:conditional}.

%%%%%%%%%%%%%%%%%%%%%%%%%%%%%%%%%%%%%%%%
\paragraph{Page Numbering.}

When only a part of the document is compiled,
the appropriate numbering of pages
(as well as other status parameters)
is determined from the |.aux| files.
The latter contain information from previous passes.
However this information needs to propagate through
all intermediate child documents.
Therefore the page numbering in child documents may well
be inconsistent until the complete document is compiled at least once.

A useful (if unconventional) way to always ensure a consistent
page numbering is to restart the numbering in each child document
and denote the pages by `\textit{child}|.|\textit{page}'
where \textit{child} represents the chapter/section number of the child file.
This can be achieved by the command
|\numberwithin{page}{|\textit{child}|}|
of the \textsf{amsmath} package
where \textit{child} can be |chapter| or |section|
depending on the chosen structuring.
Alternatively, one can modify the macro |\thepage| appropriately
and reset the counter |page| at the start of each child file.

%%%%%%%%%%%%%%%%%%%%%%%%%%%%%%%%%%%%%%%%%%%%%%%%%%%%%%%%%%%%%%%%%%%%%%%%%%%%%%%%
\subsection{Conditional Processing}
\label{sec:conditional}

The package provides a mechanism to compile different versions
of a document. To customise the versions further some conditional processing
can come in handy to distinguish which version is being compiled.
The package provides two macros to describe the compilation context:

%%%%%%%%%%%%%%%%%%%%%%%%%%%%%%%%%%%%%%%%
\DescribeMacro{\ifchilddoc}
The conditional |\ifchilddoc| distinguishes between the compilation of
child documents and the main document:
%
\begin{center}
|\ifchilddoc |\textit{child-code}| |[|\||else |\textit{main-code}]| \||fi|
\end{center}

%%%%%%%%%%%%%%%%%%%%%%%%%%%%%%%%%%%%%%%%
\DescribeMacro{\childdocname}
\DescribeMacro{\childdocjob}
The macro |\childdocname| contains the filename (without extension)
of the main or child file being processed.
Note that |\childdocjob| will always contain the name of the main file.

%%%%%%%%%%%%%%%%%%%%%%%%%%%%%%%%%%%%%%%%
\paragraph{Title Page.}

Conditional processing can be used to include a title or banner page
in the main document when proper precautions are taken.
Importantly, the code in the main file should ensure that the page counter
(as well as other status parameters which are stored in the |.aux| files)
takes the same value after the conditional processing.
Otherwise the page numbers may take divergent values
depending on which part is compiled.

For example, a title page could be declared by:
%
\begin{center}
\begin{tabular}{l}
|\ifchilddoc\||else|\\
|\addtocounter{page}{-1}|\\
\textit{code for title page}\\
|\newpage|\\
|\||fi|
\end{tabular}
\end{center}
%
A banner page for the child documents can be generated by:
%
\begin{center}
\begin{tabular}{l}
|\ifchilddoc|\\
|\addtocounter{page}{-1}|\\
\textit{code for banner page}\\
|\newpage|\\
|\||fi|
\end{tabular}
\end{center}
%
Here one could write a message such as:
\begin{center}
|This is the part \childdocname{} of \childdocjob{}.|
\end{center}

%%%%%%%%%%%%%%%%%%%%%%%%%%%%%%%%%%%%%%%%%%%%%%%%%%%%%%%%%%%%%%%%%%%%%%%%%%%%%%%%
\subsection{Flags}
\label{sec:flags}

The package makes it easy to generate different versions
of the main or child documents.
To this end compilation flags can be defined
and assigned different default values.
They will be particularly useful in conjunction
with the forwarding mechanism described in \secref{sec:forward}.

For example, it may be useful to have a flag |\version|
which can be set to |draft| or |final|.
The document source will contain some conditional code
depending on the value of |\version|.
Suppose further, the flag should default to |final| for the main file
and to |draft| for child files
which is a natural assignment for editing the document.
This is achieved by placing the following code
in the preamble of the main document
(below the |\childdocmain| directive):
%
\begin{center}
\begin{tabular}{l}
|\ifchilddoc|\\
|\providecommand{\version}{draft}|\\
|\||else|\\
|\providecommand{\version}{final}|\\
|\||fi|
\end{tabular}
\end{center}
%
The definition by |\providecommand| makes sure
that previous definitions are not overwritten.
Further statements |\providecommand{\version}{...}|
can thus be added before the above code to override it.

For the main file, one might add a line
(between |\childdocmain| and the above block)
%
\begin{center}
|%\ifchilddoc\||else\providecommand{\version}{draft}\||fi|
\end{center}
%
which can be uncommented to produce a draft version.
Likewise one can add a line to the very top of a child file
(above the |\childdocof{|\textit{main}|}| directive)
%
\begin{center}
|%\providecommand{\version}{final}|
\end{center}
%
which can be uncommented to produce the final version of this child document.

%%%%%%%%%%%%%%%%%%%%%%%%%%%%%%%%%%%%%%%%%%%%%%%%%%%%%%%%%%%%%%%%%%%%%%%%%%%%%%%%
\subsection{Forwarding}
\label{sec:forward}

Different versions of the main or child documents
using compilation flags as described in \secref{sec:flags}
can be (permanently) stored in different files
for convenient compilation, viewing and distribution.
To this end, the package defines a command
to pass on compilation to a different file:

%%%%%%%%%%%%%%%%%%%%%%%%%%%%%%%%%%%%%%%%
\DescribeMacro{\childdocforward}
The command |\childdocforward| redirects processing to
another source file:
%
\begin{center}
\begin{tabular}{l}
|\input{childdoc.def}|\\
|\childdocforward[|\textit{main}|]{|\textit{dest}|}|\\
\end{tabular}
\end{center}
%
The argument \textit{dest} is the destination file
(without extension).
It should be the main file or one of the child files.
Note that further \textsf{childdoc} directives
such as |\childdocof| and |\childdocforward|
in the indicated file will be processed in this form.
The optional argument \textit{main}
passes on directly to the main file \textit{main}
while pretending to compile the child \textit{dest}.
This form behaves as if \textit{dest}
issues |\childdocof{|\textit{main}|}| right away,
and no further \textsf{childdoc} directives will be processed.

%%%%%%%%%%%%%%%%%%%%%%%%%%%%%%%%%%%%%%%%
\DescribeMacro{\...prefix}
In the alternative form |\childdocforwardprefix|,
%
\begin{center}
\begin{tabular}{l}
|\input{childdoc.def}|\\
|\childdocforwardprefix[|\textit{main}|]{|\textit{prefix}|}{|\textit{dest}|}|
\end{tabular}
\end{center}
%
the destination file is determined by a pattern
depending on the current file:
To make this work, the current file must be called
`{\textit{prefix}\hspace{0.2em}\textit{suffix}}'
with \textit{prefix} matching precisely the argument.
Processing is then passed on to the file
`{\textit{dest}\hspace{0.2em}\textit{suffix}}'.
Surely, the same effect is achieved by
directly specifying the
argument `{\textit{dest}\hspace{0.2em}\textit{suffix}}'
in the first form.
However, that requires to set up a different file
for each child. With the alternative form of the command
all these files can have exactly the same content
which simplifies setting them up and maintaining them.

For example, the following file |draft.tex|
with a compilation flag |\version| as described in \secref{sec:flags}
compiles the main document as a draft:
%
\begin{center}
\begin{tabular}{l}
|\def\version{draft}|\\
|\input{childdoc.def}|\\
|\childdocforward{|\textit{main}|}|
\end{tabular}
\end{center}
%
Likewise, the following files |final|\textit{nn}|.tex|
compile the final version of the child document
|child|\textit{nn}|.tex|:
%
\begin{center}
\begin{tabular}{l}
|\def\version{final}|\\
|\input{childdoc.def}|\\
|\childdocforwardprefix{final}{child}|
\end{tabular}
\end{center}
%

Note that when several versions of a main file and/or of each child file
are to be generated, it may be convenient to set up a |Makefile| or
shell script to automatise the process.

%%%%%%%%%%%%%%%%%%%%%%%%%%%%%%%%%%%%%%%%%%%%%%%%%%%%%%%%%%%%%%%%%%%%%%%%%%%%%%%%
\subsection{Command Line Processing}
\label{sec:commandline}

The effect of redirection files can also be achieved by invoking
the \LaTeX{} compiler with a more elaborate command line.
Most conveniently this should be done as part
of a shell script or a |Makefile|.

When using \textsf{childdoc} in the main file, the following
command lines effectively perform a redirection
(note that depending on the shell being used,
backslashes may have to be doubled: `|\|' $\to$ `|\\|'):
%
\begin{center}
|... -jobname "|\textit{target}|" |\\|"|[\textit{flags}]%
|\input{childdoc.def}\childdocforward[|\textit{main}|]{|\textit{dest}|}"|
\end{center}
%
Here \textit{target} is the name of the output file,
\textit{main} is the name of the main file
and \textit{dest} is the name of the main or child file to be processed
(all filenames without extensions).
The optional argument \textit{main} can be omitted
if \textit{main} matches \textit{dest}.
Optionally, compilation \textit{flags} can be defined via |\def| commands.
This command line makes the \TeX{} engine believe
it is compiling the file \textit{target}
whose content is specified as the latter parameter.
The provided code then forwards the processing to
\textit{main} or \textit{dest} as described in \secref{sec:forward}.

%%%%%%%%%%%%%%%%%%%%%%%%%%%%%%%%%%%%%%%%%%%%%%%%%%%%%%%%%%%%%%%%%%%%%%%%%%%%%%%%
\subsection{Include by Input}
\label{sec:input}

Including child documents by |\include| has some restrictions by design.
Most notably, the content of a child document always occupies
its own set of pages; pages cannot be shared between child documents.
Usually, this behaviour makes perfect sense
because each child document contain an essential part of the document.
However, in some situations it may be desirable to compose
a document from a collection of parts
without having mandatory page breaks between then.
For this case, the package
provides a mechanism to include parts
by |\input| which can also be processed individually.
However, by construction this mechanism
requires manual handling of the content to be output.

%%%%%%%%%%%%%%%%%%%%%%%%%%%%%%%%%%%%%%%%
\DescribeMacro{\ifchilddocmanual}
The main file should be prepared as usual, see \secref{sec:include}.
However, the document body must make a distinction
between processing of an individual part and of the main document, e.g.:
%
\begin{center}
\begin{tabular}{l}
|\ifchilddocmanual|\\
|\input{\childdocname}|\\
|\||else|\\
\textit{document body with }|\input{|\textit{part}|}|\\
|\||fi|
\end{tabular}
\end{center}
%
The conditional |\ifchilddocmanual| is true whenever
a part to be included by |\input| is being compiled,
and the name of the part is stored in |\childdocname|.

%%%%%%%%%%%%%%%%%%%%%%%%%%%%%%%%%%%%%%%%
\DescribeMacro{\childdocby}
Each part to be included by |\input| should start with:
%
\begin{center}
\begin{tabular}{l}
|\input{childdoc.def}|\\
|\childdocby{|\textit{main}|}|\\
\end{tabular}
\end{center}
%
The directive |\childdocby| is similar to |\childdocof|
described in \secref{sec:include},
but the subsequent selection of content must be done manually.
To that end, both |\ifchilddoc| and |\ifchilddocmanual|
will be true upon processing of a part,
and the name of the part is stored in |\childdocname|.
Note that |\jobname| will be set to the filename of the current part
so that each part receives an individual |.aux| file
that does not interfere with the |.aux| file(s) of the main document.
This behaviour can be altered by the alternative form
|\childdocby[*]{|\textit{main}|}| (with a non-empty optional argument)
which uses the |.aux| file of the main document
by setting |\jobname| to \textit{main}.

%%%%%%%%%%%%%%%%%%%%%%%%%%%%%%%%%%%%%%%%%%%%%%%%%%%%%%%%%%%%%%%%%%%%%%%%%%%%%%%%
\subsection{Driver Development}
\label{sec:driver}

The \textsf{childdoc} mechanism can also be use for the development
of definition files such as \LaTeX{} styles or classes.
This case differs from the above setup with multiple parts
included by |\include| in that no |\includeonly| should be invoked.
This can be achieved by starting the include file
(before |\ProvidesPackage|) with:
%
\begin{center}
\begin{tabular}{l}
|\input{childdoc.def}|\\
|\childdocforward{|\textit{main}|}|\\
\end{tabular}
\end{center}
%
or alternatively with:
%
\begin{center}
\begin{tabular}{l}
|\input{childdoc.def}|\\
|\childdocby{|\textit{main}|}|\\
\end{tabular}
\end{center}
%
Both forms have slightly different effects as described above.
The main file is prepared as usual, see \secref{sec:include}.

%%%%%%%%%%%%%%%%%%%%%%%%%%%%%%%%%%%%%%%%%%%%%%%%%%%%%%%%%%%%%%%%%%%%%%%%%%%%%%%%
\subsection{Legacy Detection}
\label{sec:detection}

The directive |\childdocmain| in the main file can detect
whether the complete document or merely a child is to be compiled
even without using the directive |\childdocof|.
This method is deprecated because it is less robust
and there is no compelling reason to use it;
it is merely provided for backward compatibility
and it may be removed in future versions.

If the detection mechanism is to be used,
it is mandatory to correctly specify
the filename of the main file as the argument of |\childdocmain|:
%
\begin{center}
\begin{tabular}{l}
|\input{childdoc.def}|\\
|\childdocmain{|\textit{main}|}|\\
\end{tabular}
\end{center}
%
If |\jobname| does not match the argument \textit{main} of |\childdocmain|,
it is assumed that |\jobname| points to the child file to be compiled.
When using |\childdocmain| with the main file specified as argument,
it suffices to start a child file
with just |\input{|\textit{main}|}|
without loading of the package and using |\childdocof|.
If instead all processing is done
with the appropriate \textsf{childdoc} directives,
the argument of \textit{main} of |\childdocmain| can be empty.

An alternative version of the command line processing described
in \secref{sec:commandline} using the detection mechanism reads:
%
\begin{center}
|... -jobname "|\textit{target}|" "|[\textit{flags}]%
[|\def\jobname{|\textit{dest}|}|]|\input{|\textit{main}|}"|
\end{center}

%%%%%%%%%%%%%%%%%%%%%%%%%%%%%%%%%%%%%%%%%%%%%%%%%%%%%%%%%%%%%%%%%%%%%%%%%%%%%%%%
\subsection{Manual Code}
\label{sec:manual}

In case one cannot be certain whether the definitions file |childdoc.def|
is installed on the target \TeX{} distribution
and one prefers not to ship it,
it is conceivable to paste a few relevant commands into the sources.

To that end, drop all statements |\input{childdoc.def}|
and perform the replacements as outlined below.
Instead of |\childdocmain{|\textit{main}|}| add the following code
to the top of the main file:
%
\begin{center}
\begin{tabular}{l}
|\||ifdefined\childdocname\endinput\||fi\newif\ifchilddoc|\\
|\edef\childdocname{\scantokens\expandafter{\jobname\noexpand}}|\\
|\def\childdocmain{|\textit{main}|}\||ifx\childdocmain\childdocname\||else|\\
|\childdoctrue\includeonly{\childdocname}\let\jobname\childdocmain\||fi|\\
\end{tabular}
\end{center}
%
Instead of |\childdocof{|\textit{main}|}| just include the main file
at the top of each child file:
%
\begin{center}
|\input{|\textit{main}|}|
\end{center}
%
A simple redirection |\childdocforward{|\textit{dest}|}| is achieved by:
%
\begin{center}
|\def\jobname{|\textit{dest}|}\input{\jobname}|
\end{center}
%
The redirection with prefix
|\childdocforwardprefix[|\textit{prefix}|]{|\textit{dest}|}|
is accomplished by:
%
\begin{center}
\begin{tabular}{l}
|{\edef\jobname{\scantokens\expandafter{\jobname\noexpand}}|\\
|\def\redirectjob |\textit{prefix}|#1~~~{\gdef\jobname{|\textit{dest}|#1}}|\\
|\expandafter\redirectjob\jobname~~~}\input{\jobname}|
\end{tabular}
\end{center}

In an alternative approach,
child documents can be compiled by a specific command line
without additional code or specific definitions:
%
\begin{center}
|... -jobname "|\textit{target}|" "|[\textit{flags}]%
|\includeonly{|\textit{dest}|}\input{|\textit{main}|}"|
\end{center}
%

%%%%%%%%%%%%%%%%%%%%%%%%%%%%%%%%%%%%%%%%%%%%%%%%%%%%%%%%%%%%%%%%%%%%%%%%%%%%%%%%
%%%%%%%%%%%%%%%%%%%%%%%%%%%%%%%%%%%%%%%%%%%%%%%%%%%%%%%%%%%%%%%%%%%%%%%%%%%%%%%%
\section{Information}

%%%%%%%%%%%%%%%%%%%%%%%%%%%%%%%%%%%%%%%%%%%%%%%%%%%%%%%%%%%%%%%%%%%%%%%%%%%%%%%%
\subsection{Copyright}

Copyright \copyright{} 2017--2018 Niklas Beisert

This work may be distributed and/or modified under the
conditions of the \LaTeX{} Project Public License, either version 1.3
of this license or (at your option) any later version.
The latest version of this license is in
  \url{http://www.latex-project.org/lppl.txt}
and version 1.3 or later is part of all distributions of \LaTeX{}
version 2005/12/01 or later.

This work has the LPPL maintenance status `maintained'.

The Current Maintainer of this work is Niklas Beisert.

This work consists of the files |README.txt|, |childdoc.ins| and |childdoc.dtx|
as well as the derived files |childdoc.def|, |cdocsamp.tex|
with |cdocsch1.tex|, |cdocsch2.tex|, |cdocspt3.tex|, |cdocspt4.tex|,
|cdocsdrf.tex|, |cdocsfn1.tex|, |cdocsfn2.tex|
as well as |childdoc.pdf|.

%%%%%%%%%%%%%%%%%%%%%%%%%%%%%%%%%%%%%%%%%%%%%%%%%%%%%%%%%%%%%%%%%%%%%%%%%%%%%%%%
\subsection{Files and Installation}

The package consists of the files:
%
\begin{center}
\begin{tabular}{ll}
    |README.txt|   & readme file \\
    |childdoc.ins| & installation file \\
    |childdoc.dtx| & source file \\
    |childdoc.def| & definition file \\
    |cdocsamp.tex| & sample main file \\
    |cdocsch1.tex| & sample include file \\
    |cdocsch2.tex| & sample include file \\
    |cdocspt3.tex| & sample part file \\
    |cdocspt4.tex| & sample part file \\
    |cdocsdrf.tex| & sample redirection file \\
    |cdocsfn1.tex| & sample redirection file \\
    |cdocsfn2.tex| & sample redirection file \\
    |childdoc.pdf| & manual
\end{tabular}
\end{center}
%
The distribution consists of the files
|README.txt|, |childdoc.ins| and |childdoc.dtx|.
%
\begin{itemize}
\item
Run (pdf)\LaTeX{} on |childdoc.dtx|
to compile the manual |childdoc.pdf| (this file).
\item
Run \LaTeX{} on |childdoc.ins| to create the definitions file |childdoc.def|
and the sample |cdocsamp.tex| with include files
|cdocsch1.tex|, |cdocsch2.tex|, |cdocspt3.tex|, |cdocspt4.tex|,
|cdocsdrf.tex|, |cdocsfn1.tex|, |cdocsfn2.tex|.
Then copy the file |childdoc.def| to an appropriate directory of your \LaTeX{}
distribution, e.g.\ \textit{texmf-root}|/tex/latex/childdoc|.
\end{itemize}

%%%%%%%%%%%%%%%%%%%%%%%%%%%%%%%%%%%%%%%%%%%%%%%%%%%%%%%%%%%%%%%%%%%%%%%%%%%%%%%%
\subsection{Related CTAN Packages}

There are several other packages which offer a similar functionality:
%
\begin{itemize}
\item
The packages
\href{http://ctan.org/pkg/docmute}{\textsf{docmute}},
\href{http://ctan.org/pkg/includex}{\textsf{includex}} and
\href{http://ctan.org/pkg/standalone}{\textsf{standalone}}
provide commands to include only the document body of
a child file thus allowing both files to be compiled individually.
\item
The packages \href{http://ctan.org/pkg/subdocs}{\textsf{subdocs}}
and \href{http://ctan.org/pkg/subfiles}{\textsf{subfiles}}
provide structures in which the main and child documents can be
encapsulated and allowing them to be compiled individually.
The inclusion mechanism is different from the conventional |\include|.
\item
The package \href{http://ctan.org/pkg/combine}{\textsf{combine}}
is an elaborate solution to combine several documents into one.
\end{itemize}
%
See also the CTAN topic \href{http://ctan.org/topic/subdocs}{\textsf{subdocs}}
for further related packages.
The present package differs from the above solutions in that
a document structure constructed with the conventional |\include| mechanism
just needs two extra commands at the top of every file
such that all constituent files can be compiled individually.

%%%%%%%%%%%%%%%%%%%%%%%%%%%%%%%%%%%%%%%%%%%%%%%%%%%%%%%%%%%%%%%%%%%%%%%%%%%%%%%%
%\subsection{Feature Suggestions}
%
%The following is a list of features which may be useful for future
%versions of this package:
%%
%\begin{itemize}
%\item
%\ldots
%\end{itemize}

%%%%%%%%%%%%%%%%%%%%%%%%%%%%%%%%%%%%%%%%%%%%%%%%%%%%%%%%%%%%%%%%%%%%%%%%%%%%%%%%
\subsection{Revision History}

%%%%%%%%%%%%%%%%%%%%%%%%%%%%%%%%%%%%%%%%
\paragraph{v2.0:} 2018/12/30

\begin{itemize}
\item
immediate forward processing
\item
added |\childdocby| mechanism
\item
manual restructured
\end{itemize}

%%%%%%%%%%%%%%%%%%%%%%%%%%%%%%%%%%%%%%%%
\paragraph{v1.6:} 2018/01/17

\begin{itemize}
\item
application for development of include files
\item
corrections to manual
\end{itemize}

%%%%%%%%%%%%%%%%%%%%%%%%%%%%%%%%%%%%%%%%
\paragraph{v1.5:} 2017/05/21

\begin{itemize}
\item
more complete structuring introduced
\item
|\childdocof| introduced
\item
|\childdoc| renamed to |\childdocmain|
\item
|\childredirect| renamed to |\childdocforward| and |\childdocforwardprefix|
and functionality expanded
\end{itemize}

%%%%%%%%%%%%%%%%%%%%%%%%%%%%%%%%%%%%%%%%
\paragraph{v1.0:} 2017/04/27

\begin{itemize}
\item
manual and install package
\item
first version published on CTAN
\end{itemize}

%%%%%%%%%%%%%%%%%%%%%%%%%%%%%%%%%%%%%%%%
\paragraph{v0.6:} 2017/04/26

\begin{itemize}
\item
redirection mechanism added
\end{itemize}

%%%%%%%%%%%%%%%%%%%%%%%%%%%%%%%%%%%%%%%%
\paragraph{v0.5:} 2017/04/26

\begin{itemize}
\item
functionality in definition file
\end{itemize}


%%%%%%%%%%%%%%%%%%%%%%%%%%%%%%%%%%%%%%%%%%%%%%%%%%%%%%%%%%%%%%%%%%%%%%%%%%%%%%%%
%%%%%%%%%%%%%%%%%%%%%%%%%%%%%%%%%%%%%%%%%%%%%%%%%%%%%%%%%%%%%%%%%%%%%%%%%%%%%%%%
%%%%%%%%%%%%%%%%%%%%%%%%%%%%%%%%%%%%%%%%%%%%%%%%%%%%%%%%%%%%%%%%%%%%%%%%%%%%%%%%
\appendix

\settowidth\MacroIndent{\rmfamily\scriptsize 000\ }

 \DocInput{childdoc.dtx}

\end{document}
%</driver>
% \fi
%
% %%%%%%%%%%%%%%%%%%%%%%%%%%%%%%%%%%%%%%%%%%%%%%%%%%%%%%%%%%%%%%%%%%%%%%%%%%%%%%
% %%%%%%%%%%%%%%%%%%%%%%%%%%%%%%%%%%%%%%%%%%%%%%%%%%%%%%%%%%%%%%%%%%%%%%%%%%%%%%
% \section{Sample}
%\iffalse
%<*samplemain>
%\fi
%
% The following presents a sample document
% with two chapters, two parts, a title page,
% a compile flag as well as three forwarding files to set the flag.
% It consists of eight |.tex| files:
% \begin{center}
% \begin{tabular}{ll}
% |cdocsamp.tex|&main file\\
% |cdocsch1.tex|&include file for chapter 1\\
% |cdocsch2.tex|&include file for chapter 2\\
% |cdocspt3.tex|&include file for part 3\\
% |cdocspt4.tex|&include file for part 4\\
% |cdocsdrf.tex|&forwarding file for main file in draft mode\\
% |cdocsfi1.tex|&forwarding file for final version of chapter 1\\
% |cdocsfi2.tex|&forwarding file for final version of chapter 2\\
% \end{tabular}
% \end{center}
% Each of the eight files can be compiled directly by the \LaTeX{} compiler.
%
% %%%%%%%%%%%%%%%%%%%%%%%%%%%%%%%%%%%%%%
% \paragraph{Main File.}
%
% The main file is called |cdocsamp.tex|.
%
% Load the \textsf{childdoc} definitions and
% declare the filename for the main document:
%    \begin{macrocode}
\input{childdoc.def}
\childdocmain{}
%    \end{macrocode}

% Optional override for |\version| flag:
%    \begin{macrocode}
%%\ifchilddoc\else\providecommand{\version}{draft}\fi
%    \end{macrocode}

% Define the default values for the |\version| flag
% (|final| for the main file and |draft| for childs):
%    \begin{macrocode}
\ifchilddoc
\providecommand{\version}{draft}
\else
\providecommand{\version}{final}
\fi
%    \end{macrocode}

% Load the standard document class:
%    \begin{macrocode}
\documentclass[12pt]{article}
%    \end{macrocode}

% Start the document body:
%    \begin{macrocode}
\begin{document}
%    \end{macrocode}

% Declare a title page.
% Print title, part of document being processed and version flag:
%    \begin{macrocode}
\addtocounter{page}{-1}
\begin{center}
{\LARGE\bfseries{}childdoc example\par}
\vspace{1cm}
\ifchilddoc
\ifchilddocmanual part\else chapter\fi:
`\childdocname' of `\childdocjob'\par
\else
main document: `\childdocjob'\par
\fi
version: \version\par
\end{center}
\newpage
%    \end{macrocode}

% Manually include selected file,
% otherwise process as usual:
%    \begin{macrocode}
\ifchilddocmanual
\section*{part `\childdocname'}
\input{\childdocname}
\else
%    \end{macrocode}

% Include the two chapters:
%    \begin{macrocode}
\include{cdocsch1}
\include{cdocsch2}
%    \end{macrocode}

% Include the two parts unless only chapters should be displayed:
%    \begin{macrocode}
\ifchilddoc\else
\section{part three}
\input{cdocspt3}
\section{part four}
\input{cdocspt4}
\fi
%    \end{macrocode}

% Process as usual until here:
%    \begin{macrocode}
\fi
%    \end{macrocode}

% End of document body:
%    \begin{macrocode}
\end{document}
%    \end{macrocode}
%\iffalse
%</samplemain>
%\fi
%
% %%%%%%%%%%%%%%%%%%%%%%%%%%%%%%%%%%%%%%
% \paragraph{Chapter Include Files.}
%
% The include files are called |cdocsch1.tex| and |cdocsch2.tex|.
%
%\iffalse
%<*samplechap1|samplechap2>
%\fi

% Optional override for |\version| flag:
%    \begin{macrocode}
%%\providecommand{\version}{final}
%    \end{macrocode}

% Include the main document:
%    \begin{macrocode}
\input{childdoc.def}
\childdocof{cdocsamp}
%    \end{macrocode}

%\iffalse
%</samplechap1|samplechap2>
%\fi
%
%\iffalse
%<*samplechap1>
%\fi
% Some text for chapter 1:
%    \begin{macrocode}
\section{one}
some text in chapter one
%    \end{macrocode}

%\iffalse
%</samplechap1>
%\fi
% Some text for chapter 2:
%\iffalse
%<*samplechap2>
%\fi
%    \begin{macrocode}
\section{two}
more text in chapter two
%    \end{macrocode}

%\iffalse
%</samplechap2>
%\fi
%
% %%%%%%%%%%%%%%%%%%%%%%%%%%%%%%%%%%%%%%
% \paragraph{Part Include Files.}
%
% The include files are called |cdocspt3.tex| and |cdocspt4.tex|.
%
%\iffalse
%<*samplepart3|samplepart4>
%\fi

% Optional override for |\version| flag:
%    \begin{macrocode}
%%\providecommand{\version}{final}
%    \end{macrocode}

% Include the main document:
%    \begin{macrocode}
\input{childdoc.def}
\childdocby{cdocsamp}
%    \end{macrocode}

%\iffalse
%</samplepart3|samplepart4>
%\fi
%
%\iffalse
%<*samplepart3>
%\fi
% Some text for part 3:
%    \begin{macrocode}
some text in part three
%    \end{macrocode}

%\iffalse
%</samplepart3>
%\fi
% Some text for part 4:
%\iffalse
%<*samplepart4>
%\fi
%    \begin{macrocode}
more text in part four
%    \end{macrocode}

%\iffalse
%</samplepart4>
%\fi
%
% %%%%%%%%%%%%%%%%%%%%%%%%%%%%%%%%%%%%%%
% \paragraph{Forwarding for a Complete Draft.}
%
% The following forwarding file |cdocsdrf.tex|
% compiles the main document in draft mode:
%\iffalse
%<*sampledraft>
%\fi
%    \begin{macrocode}
\def\version{draft}
\input{childdoc.def}
\childdocforward{cdocsamp}
%    \end{macrocode}

%\iffalse
%</sampledraft>
%\fi
%
% %%%%%%%%%%%%%%%%%%%%%%%%%%%%%%%%%%%%%%
% \paragraph{Forwarding for Final Version of the Chapters.}
%
% The following forwarding files |cdocsfn1.tex| and |cdocsfn2.tex|
% (with identical content)
% compile the final versions of the child documents
% |cdocsch1.tex| and |cdocsch2.tex|, respectively:
%\iffalse
%<*samplefinal>
%\fi
%    \begin{macrocode}
\def\version{final}
\input{childdoc.def}
\childdocforwardprefix[cdocsamp]{cdocsfn}{cdocsch}
%    \end{macrocode}

%\iffalse
%</samplefinal>
%\fi
%
% %%%%%%%%%%%%%%%%%%%%%%%%%%%%%%%%%%%%%%
% \paragraph{Command Line Processing.}
%
% The following three command lines generate the output files
% |cdocscld|, |cdocscl1| and |cdocscl2|
% which should be identical to
% |cdocsdrf|, |cdocsch1| and |cdocsfn2|, respectively:
% \begin{center}
% \begin{tabular}{l}
% |latex -jobname cdocscld \|\\
% |  "\def\version{draft}\input{childdoc.def}\childdocforward{cdocsamp}"|\\
% |latex -jobname cdocscl1 \|\\
% |  "\input{childdoc.def}\childdocforward[cdocsamp]{cdocsch1}"|\\
% |latex -jobname cdocscl2 \|\\
% |  "\def\version{final}\input{childdoc.def}\childdocforward{cdocsch2}"|
% \end{tabular}
% \end{center}
% Note that the trailing backslash on each first line
% merely continues the input to the second line
% (for convenient cut ant paste).
% Furthermore, the command |latex| can be replaced by any
% of its alternative versions such as |pdflatex|.
%
% %%%%%%%%%%%%%%%%%%%%%%%%%%%%%%%%%%%%%%%%%%%%%%%%%%%%%%%%%%%%%%%%%%%%%%%%%%%%%%
% %%%%%%%%%%%%%%%%%%%%%%%%%%%%%%%%%%%%%%%%%%%%%%%%%%%%%%%%%%%%%%%%%%%%%%%%%%%%%%
% \section{Implementation}
%\iffalse
%<*package>
%\fi
%
% This section describes the definitions file |childdoc.def|.

% The definitions cannot be loaded using |\usepackage| or |\RequirePackage|
% which has a mechanism to prevent loading a style file more than once.
% When loading the definitions by means of |\input|
% multiple instances have to be prevented manually:
%\iffalse
%This code needs to be before the `\ProvidesFile' directive
%which is defined at the beginning of this file.
%Therefore it is also placed there and commented out here.
%</package>
%<*discard>
%\fi
%    \begin{macrocode}
\ifdefined\childdocmain\endinput\fi
%    \end{macrocode}
%\iffalse
%</discard>
%<*package>
%\fi
%
% \macro{\ifchilddoc}
% \macro{\ifchilddocmanual}
% The conditional |\ifchilddoc| tells whether a
% child (true) or main (false) document is being compiled.
% The conditional |\ifchilddocmanual| tells whether
% the |\includeonly| mechanism is used (false) or
% the selection of child files must be performed manually (true).
% The definitions initialise to false:
%    \begin{macrocode}
\newif\ifchilddoc
\newif\ifchilddocmanual
%    \end{macrocode}

% \macro{\childdocname}
% \macro{\childdocjob}
% The macro |\childdocname| stores the name of the main document
% to be compiled. The macro |\childdocjob| stores the name of
% the document on which the \LaTeX{} compiler was originally invoked.
% The content of |\jobname| cannot be compared
% to filenames specified in the source due to different catcodes.
% The following code rescans |\jobname|, stores the result
% in |\childdocname| and saves a copy in |\childdocjob|:
%    \begin{macrocode}
\edef\childdocname{\scantokens\expandafter{\jobname\noexpand}}
\let\childdocjob\childdocname
%    \end{macrocode}

% \macro{\childdocdisable}
% The macro |\childdocdisable| prevents the main file
% from being processed more than once.
% At this stage, the main document command |\childdocmain|
% is assumed to be called once again where it should do nothing.
% Any subsequent call to it should prevent
% a secondary processing of the main document
% It overwrites the forwarding commands
% |\childdocof| and |\childdocforward|
% with empty macros to prevent further inclusions of the main document:
%    \begin{macrocode}
\newcommand{\childdocdisable}
{
  \renewcommand{\childdocmain}[1]{\renewcommand{\childdocmain}[1]{\endinput}}
  \renewcommand{\childdocof}[1]{}
  \renewcommand{\childdocby}[2][]{}
  \renewcommand{\childdocforward}[2][]{}
  \renewcommand{\childdocdisable}{}
}
%    \end{macrocode}

% \macro{\childdocmain}
% The macro |\childdocmain| is to be called at the top of the main file
% with nothing or the main filename (without extension) as argument.
% First, it breaks loops.
% If the argument is not empty and does not match |\childdocname|
% (which is set by the first inclusion of |childdoc.def|),
% |\ifchilddoc| is set to true, |\includeonly| is applied to the child file
% and |\jobname| is set to the main file
% (for proper handling of |.aux| files):
%    \begin{macrocode}
\newcommand{\childdocmain}[1]
{
  \childdocdisable\childdocmain{}
  \if?#1?\else
    \begingroup
      \def\childdoctmp{#1}
      \ifx\childdoctmp\childdocname
        \def\childdoctmp{}
      \else
        \def\childdoctmp
        {
          \childdoctrue
          \includeonly{\childdocname}
          \def\childdocjob{#1}
          \def\jobname{#1}
        }
      \fi
      \expandafter
    \endgroup
    \childdoctmp
  \fi
}
%    \end{macrocode}

% \macro{\childdocof}
% The command |\childdocof| redirects
% compilation to the main file |#1|.
%    \begin{macrocode}
\newcommand{\childdocof}[1]
{
  \childdocdisable
  \childdoctrue
  \includeonly{\childdocname}
  \def\jobname{#1}
  \def\childdocjob{#1}
  \input{#1}
}
%    \end{macrocode}

% \macro{\childdocby}
% The command |\childdocby| ....
%    \begin{macrocode}
\newcommand{\childdocby}[2][]
{
  \childdocdisable
  \childdoctrue
  \childdocmanualtrue
  \if?#1?\else
    \def\jobname{#2}
  \fi
  \def\childdocjob{#2}
  \input{#2}
  \endinput
}
%    \end{macrocode}

% \macro{\childdocforward}
% The command |\childdocforward| redirects
% compilation to the main file or
% (if the optional argument is given) a child file.
% Parameters are set as if the main file
% or a child file starting with |\childdocof| was compiled.
% Then compilation is handed over to the main file:
%    \begin{macrocode}
\newcommand{\childdocforward}[2][]
{
  \begingroup
    \if?#1?
      \def\childdoctmp
      {
        \def\childdocname{#2}
        \def\childdocjob{#2}
        \def\jobname{#2}
        \input{#2}
        \endinput
      }
    \else
      \def\childdoctmp
      {
        \childdocdisable
        \def\childdocname{#2}
        \childdoctrue
        \includeonly{#2}
        \def\childdocjob{#1}
        \def\jobname{#1}
        \input{#1}
        \endinput
      }
    \fi
    \expandafter
  \endgroup
  \childdoctmp
}
%    \end{macrocode}

% \macro{\childdocforwardprefix}
% The command |\childdocforwardprefix| redirects
% compilation to the main or a child file by means of a pattern.
% The prefix |#1| in the current filename is replaced by |#2|
% and the suffix of the current filename is kept
% (it is assumed that the filename does not contain the substring `|~~~|'
% which is used as a delimiter).
% Compilation is handed over to the new file by |\childdocforward|:
%    \begin{macrocode}
\newcommand{\childdocforwardprefix}[3][]
{
  \begingroup
    \def\childdocextract #2##1~~~{\def\childdoctmp{\childdocforward[#1]{#3##1}}}
    \expandafter\childdocextract\childdocname~~~
    \expandafter
  \endgroup
  \childdoctmp
}
%    \end{macrocode}

% \macro{\childdoc}
% The deprecated macro |\childdoc| is a legacy version of |\childdocmain|:
%    \begin{macrocode}
\newcommand{\childdoc}{\childdocmain}
%    \end{macrocode}

% \macro{\childdocredirect}
% The deprecated macro |\childdocredirect| is a legacy version
% of |\childdocforward| and |\childdocforwardprefix|:
%    \begin{macrocode}
\newcommand{\childdocredirect}[2][]
{
  \begingroup
    \if?#1?
      \def\childdoctmp{\childdocforward{#2}}
    \else
      \def\childdoctmp{\childdocforwardprefix{#1}{#2}}
    \fi
    \expandafter
  \endgroup
  \childdoctmp
}
%    \end{macrocode}

%\iffalse
%</package>
%\fi
%
\endinput
|\\
|\childdocby{|\textit{main}|}|\\
\end{tabular}
\end{center}
%
The directive |\childdocby| is similar to |\childdocof|
described in \secref{sec:include},
but the subsequent selection of content must be done manually.
To that end, both |\ifchilddoc| and |\ifchilddocmanual|
will be true upon processing of a part,
and the name of the part is stored in |\childdocname|.
Note that |\jobname| will be set to the filename of the current part
so that each part receives an individual |.aux| file
that does not interfere with the |.aux| file(s) of the main document.
This behaviour can be altered by the alternative form
|\childdocby[*]{|\textit{main}|}| (with a non-empty optional argument)
which uses the |.aux| file of the main document
by setting |\jobname| to \textit{main}.

%%%%%%%%%%%%%%%%%%%%%%%%%%%%%%%%%%%%%%%%%%%%%%%%%%%%%%%%%%%%%%%%%%%%%%%%%%%%%%%%
\subsection{Driver Development}
\label{sec:driver}

The \textsf{childdoc} mechanism can also be use for the development
of definition files such as \LaTeX{} styles or classes.
This case differs from the above setup with multiple parts
included by |\include| in that no |\includeonly| should be invoked.
This can be achieved by starting the include file
(before |\ProvidesPackage|) with:
%
\begin{center}
\begin{tabular}{l}
|% \iffalse
%
% childdoc.dtx Copyright (C) 2017-2018 Niklas Beisert
%
% This work may be distributed and/or modified under the
% conditions of the LaTeX Project Public License, either version 1.3
% of this license or (at your option) any later version.
% The latest version of this license is in
%   http://www.latex-project.org/lppl.txt
% and version 1.3 or later is part of all distributions of LaTeX
% version 2005/12/01 or later.
%
% This work has the LPPL maintenance status `maintained'.
%
% The Current Maintainer of this work is Niklas Beisert.
%
% This work consists of the files childdoc.dtx and childdoc.ins
% and the derived files childdoc.def and cdocsamp.tex with
% cdocsch1.tex, cdocsch2.tex, cdocsdrf.tex, cdocsfn1.tex, cdocsfn2.tex.
%
%<package>\ifdefined\childdocmain\endinput\fi
%<package>\ProvidesFile{childdoc.def}[2018/12/30 v2.0 child document driver]
%<samplemain>\ProvidesFile{cdocsamp.tex}[2018/12/30 v2.0 sample for childdoc]
%<*driver>
%\ProvidesFile{childdoc.drv}[2018/12/30 v2.0 childdoc reference manual file]
\PassOptionsToClass{10pt,a4paper}{article}
\documentclass{ltxdoc}

\usepackage[margin=35mm]{geometry}
\usepackage{hyperref}
\usepackage{hyperxmp}
\usepackage[usenames]{color}

\hypersetup{colorlinks=true}
\hypersetup{pdfstartview=FitH}
\hypersetup{pdfpagemode=UseNone}
\hypersetup{pdfsource={}}
\hypersetup{pdflang={en-UK}}
\hypersetup{pdfcopyright={Copyright 2017-2018 Niklas Beisert.
  This work may be distributed and/or modified under the
  conditions of the LaTeX Project Public License, either version 1.3
  of this license or (at your option) any later version.}}
\hypersetup{pdflicenseurl={http://www.latex-project.org/lppl.txt}}
\hypersetup{pdfcontactaddress={ETH Zurich, ITP, HIT K,
  Wolfgang-Pauli-Strasse 27}}
\hypersetup{pdfcontactpostcode={8093}}
\hypersetup{pdfcontactcity={Zurich}}
\hypersetup{pdfcontactcountry={Switzerland}}
\hypersetup{pdfcontactemail={nbeisert@itp.phys.ethz.ch}}
\hypersetup{pdfcontacturl={http://people.phys.ethz.ch/\xmptilde nbeisert/}}

\newcommand{\secref}[1]{\hyperref[#1]{section \ref*{#1}}}

\parskip1ex
\parindent0pt
\let\olditemize\itemize
\def\itemize{\olditemize\parskip0pt}

\begin{document}

\title{The \textsf{childdoc} Package}
\hypersetup{pdftitle={The childdoc Package}}
\author{Niklas Beisert\\[2ex]
  Institut f\"ur Theoretische Physik\\
  Eidgen\"ossische Technische Hochschule Z\"urich\\
  Wolfgang-Pauli-Strasse 27, 8093 Z\"urich, Switzerland\\[1ex]
  \href{mailto:nbeisert@itp.phys.ethz.ch}
  {\texttt{nbeisert@itp.phys.ethz.ch}}}
\hypersetup{pdfauthor={Niklas Beisert}}
\hypersetup{pdfsubject={Manual for the LaTeX2e Package childdoc}}
\date{30 December 2018, \textsf{v2.0}}
\maketitle

\begin{abstract}\noindent
\textsf{childdoc} is a \LaTeXe{} package
that enables the direct compilation
of document sections included by |\include|
to individual files.
\end{abstract}

\begingroup
\parskip0ex
\tableofcontents
\endgroup

%%%%%%%%%%%%%%%%%%%%%%%%%%%%%%%%%%%%%%%%%%%%%%%%%%%%%%%%%%%%%%%%%%%%%%%%%%%%%%%%
%%%%%%%%%%%%%%%%%%%%%%%%%%%%%%%%%%%%%%%%%%%%%%%%%%%%%%%%%%%%%%%%%%%%%%%%%%%%%%%%
\section{Introduction}

\LaTeX{} provides a mechanism to structure a large document (such as a book)
into a main file and several child files (containing the chapters)
using the |\include| command.
This mechanism is beneficial for documents
which span hundreds of pages in order to
make the source file(s) more manageable.
Moreover, compilation can be restricted to
selected child files by means of the |\includeonly| command.
The latter feature can be used to reduce the compilation time while editing
(this was significantly more useful in the earlier days of \LaTeX{})
or to generate a smaller document which is easier to navigate.
Another application of |\includeonly| is to generate
documents consisting of selected parts of the complete document.

However, there are a few drawbacks of the plain |\include| mechanism:
\begin{itemize}
\item
The child files cannot be compiled on their own,
they can only be compiled via the main file.
A naive editing environment
(such as a text editor with an option
to have the current file processed by \LaTeX)
may require one to switch to the main file before compiling;
attempting to compile the child file produces errors.
\item
The main file must be modified (each time)
to adjust the |\includeonly| command
to the present needs. This easily leaves the main file in a messy state.
\item
The generated document will always carry the filename
of the main document. This is inconvenient if
several child files are to be compiled and
to be kept for distribution.
\end{itemize}

The present package provides a simple interface
to make child files individually compilable by \LaTeX{}.
Compiling a child file then has the same effect as compiling
the main file with an |\includeonly| command
to select the appropriate child.
Moreover the generated document will carry the name of the child
rather than the main file.
This resolves all three above issues.

This feature is meant to make the editing of books,
thesis documents and lecture notes somewhat more convenient.
However, the package can also be used efficiently for
composing a series of documents (such as exercise sheets)
which are typically distributed individually.
It then assists the author in generating the individual documents
(potentially in different versions)
as well as a document containing the collected series.
Another application is in developing style files
or other kinds of included material
where compilation of the style file could redirect
to a sample or test file.

%%%%%%%%%%%%%%%%%%%%%%%%%%%%%%%%%%%%%%%%%%%%%%%%%%%%%%%%%%%%%%%%%%%%%%%%%%%%%%%%
%%%%%%%%%%%%%%%%%%%%%%%%%%%%%%%%%%%%%%%%%%%%%%%%%%%%%%%%%%%%%%%%%%%%%%%%%%%%%%%%
\section{Usage}

First of all, the package \textsf{childdoc} is \emph{not} a standard
\LaTeXe{} |.sty| style file! Therefore it needs to be invoked in
a non-standard way.

%%%%%%%%%%%%%%%%%%%%%%%%%%%%%%%%%%%%%%%%%%%%%%%%%%%%%%%%%%%%%%%%%%%%%%%%%%%%%%%%
\subsection{Included Files}
\label{sec:include}

%%%%%%%%%%%%%%%%%%%%%%%%%%%%%%%%%%%%%%%%
\DescribeMacro{\childdocmain}
To use the package, add the commands
\begin{center}
\begin{tabular}{l}
|\input{childdoc.def}|\\
|\childdocmain{}|\\
\end{tabular}
\end{center}
at the very top of the main \LaTeX{} file,
in particular \emph{before} the |\documentclass| statement!
The argument of |\childdocmain| should be left empty
(but it must be present).

%%%%%%%%%%%%%%%%%%%%%%%%%%%%%%%%%%%%%%%%
\DescribeMacro{\childdocof}
Furthermore, add the commands
\begin{center}
\begin{tabular}{l}
|\input{childdoc.def}|\\
|\childdocof{|\textit{main}|}|\\
\end{tabular}
\end{center}
at the top of every child file \textit{child}
which is included by |\include{|\textit{child}|}|
from within the main file
(or at least for those files to be compiled individually).
The argument \textit{main} must be the filename of the main file.

There are a couple of
considerations in setting up the main and child documents:

%%%%%%%%%%%%%%%%%%%%%%%%%%%%%%%%%%%%%%%%
\paragraph{Restrictions.}

Please note the following restrictions:
\begin{itemize}
\item
|\childdocmain| must be called with one argument \textit{main}
to ensure compatibility with earlier version of the package.
It must either be empty (|\childdocmain{}|)
or precisely match the filename of the main file in which it is specified.
See \secref{sec:detection} for further information.
\item
The filename \textit{main} must be specified without the |.tex| extension.
\item
The filename \textit{main} is case sensitive
(even in case-insensitive file systems)
due to internal string comparison.
\item
The argument \textit{main} should be fully expanded, it cannot be a macro.
\item
Subdirectories and special characters should be avoided in filenames.
\item
The command |\childdocmain{|\textit{main}|}| must be followed by a whitespace.
It should not be followed immediately by another command
or by a comment mark `|%|'.
This is because the \TeX{} parser reads the token immediately following
the argument of |\childdocmain| and puts it
at the beginning of every child section;
however, a white\-space is ignored.
\end{itemize}

%%%%%%%%%%%%%%%%%%%%%%%%%%%%%%%%%%%%%%%%
\paragraph{Content of Main File.}

It is advisable to place all content in the child files included by |\include|.
Any output contained in the main file will appear in all child documents
unless suppressed manually;
it cannot be suppressed automatically by the |\includeonly| directive
and thus should normally be avoided.
A method to include some content in the main file
by means of conditional processing is described in \secref{sec:conditional}.

%%%%%%%%%%%%%%%%%%%%%%%%%%%%%%%%%%%%%%%%
\paragraph{Page Numbering.}

When only a part of the document is compiled,
the appropriate numbering of pages
(as well as other status parameters)
is determined from the |.aux| files.
The latter contain information from previous passes.
However this information needs to propagate through
all intermediate child documents.
Therefore the page numbering in child documents may well
be inconsistent until the complete document is compiled at least once.

A useful (if unconventional) way to always ensure a consistent
page numbering is to restart the numbering in each child document
and denote the pages by `\textit{child}|.|\textit{page}'
where \textit{child} represents the chapter/section number of the child file.
This can be achieved by the command
|\numberwithin{page}{|\textit{child}|}|
of the \textsf{amsmath} package
where \textit{child} can be |chapter| or |section|
depending on the chosen structuring.
Alternatively, one can modify the macro |\thepage| appropriately
and reset the counter |page| at the start of each child file.

%%%%%%%%%%%%%%%%%%%%%%%%%%%%%%%%%%%%%%%%%%%%%%%%%%%%%%%%%%%%%%%%%%%%%%%%%%%%%%%%
\subsection{Conditional Processing}
\label{sec:conditional}

The package provides a mechanism to compile different versions
of a document. To customise the versions further some conditional processing
can come in handy to distinguish which version is being compiled.
The package provides two macros to describe the compilation context:

%%%%%%%%%%%%%%%%%%%%%%%%%%%%%%%%%%%%%%%%
\DescribeMacro{\ifchilddoc}
The conditional |\ifchilddoc| distinguishes between the compilation of
child documents and the main document:
%
\begin{center}
|\ifchilddoc |\textit{child-code}| |[|\||else |\textit{main-code}]| \||fi|
\end{center}

%%%%%%%%%%%%%%%%%%%%%%%%%%%%%%%%%%%%%%%%
\DescribeMacro{\childdocname}
\DescribeMacro{\childdocjob}
The macro |\childdocname| contains the filename (without extension)
of the main or child file being processed.
Note that |\childdocjob| will always contain the name of the main file.

%%%%%%%%%%%%%%%%%%%%%%%%%%%%%%%%%%%%%%%%
\paragraph{Title Page.}

Conditional processing can be used to include a title or banner page
in the main document when proper precautions are taken.
Importantly, the code in the main file should ensure that the page counter
(as well as other status parameters which are stored in the |.aux| files)
takes the same value after the conditional processing.
Otherwise the page numbers may take divergent values
depending on which part is compiled.

For example, a title page could be declared by:
%
\begin{center}
\begin{tabular}{l}
|\ifchilddoc\||else|\\
|\addtocounter{page}{-1}|\\
\textit{code for title page}\\
|\newpage|\\
|\||fi|
\end{tabular}
\end{center}
%
A banner page for the child documents can be generated by:
%
\begin{center}
\begin{tabular}{l}
|\ifchilddoc|\\
|\addtocounter{page}{-1}|\\
\textit{code for banner page}\\
|\newpage|\\
|\||fi|
\end{tabular}
\end{center}
%
Here one could write a message such as:
\begin{center}
|This is the part \childdocname{} of \childdocjob{}.|
\end{center}

%%%%%%%%%%%%%%%%%%%%%%%%%%%%%%%%%%%%%%%%%%%%%%%%%%%%%%%%%%%%%%%%%%%%%%%%%%%%%%%%
\subsection{Flags}
\label{sec:flags}

The package makes it easy to generate different versions
of the main or child documents.
To this end compilation flags can be defined
and assigned different default values.
They will be particularly useful in conjunction
with the forwarding mechanism described in \secref{sec:forward}.

For example, it may be useful to have a flag |\version|
which can be set to |draft| or |final|.
The document source will contain some conditional code
depending on the value of |\version|.
Suppose further, the flag should default to |final| for the main file
and to |draft| for child files
which is a natural assignment for editing the document.
This is achieved by placing the following code
in the preamble of the main document
(below the |\childdocmain| directive):
%
\begin{center}
\begin{tabular}{l}
|\ifchilddoc|\\
|\providecommand{\version}{draft}|\\
|\||else|\\
|\providecommand{\version}{final}|\\
|\||fi|
\end{tabular}
\end{center}
%
The definition by |\providecommand| makes sure
that previous definitions are not overwritten.
Further statements |\providecommand{\version}{...}|
can thus be added before the above code to override it.

For the main file, one might add a line
(between |\childdocmain| and the above block)
%
\begin{center}
|%\ifchilddoc\||else\providecommand{\version}{draft}\||fi|
\end{center}
%
which can be uncommented to produce a draft version.
Likewise one can add a line to the very top of a child file
(above the |\childdocof{|\textit{main}|}| directive)
%
\begin{center}
|%\providecommand{\version}{final}|
\end{center}
%
which can be uncommented to produce the final version of this child document.

%%%%%%%%%%%%%%%%%%%%%%%%%%%%%%%%%%%%%%%%%%%%%%%%%%%%%%%%%%%%%%%%%%%%%%%%%%%%%%%%
\subsection{Forwarding}
\label{sec:forward}

Different versions of the main or child documents
using compilation flags as described in \secref{sec:flags}
can be (permanently) stored in different files
for convenient compilation, viewing and distribution.
To this end, the package defines a command
to pass on compilation to a different file:

%%%%%%%%%%%%%%%%%%%%%%%%%%%%%%%%%%%%%%%%
\DescribeMacro{\childdocforward}
The command |\childdocforward| redirects processing to
another source file:
%
\begin{center}
\begin{tabular}{l}
|\input{childdoc.def}|\\
|\childdocforward[|\textit{main}|]{|\textit{dest}|}|\\
\end{tabular}
\end{center}
%
The argument \textit{dest} is the destination file
(without extension).
It should be the main file or one of the child files.
Note that further \textsf{childdoc} directives
such as |\childdocof| and |\childdocforward|
in the indicated file will be processed in this form.
The optional argument \textit{main}
passes on directly to the main file \textit{main}
while pretending to compile the child \textit{dest}.
This form behaves as if \textit{dest}
issues |\childdocof{|\textit{main}|}| right away,
and no further \textsf{childdoc} directives will be processed.

%%%%%%%%%%%%%%%%%%%%%%%%%%%%%%%%%%%%%%%%
\DescribeMacro{\...prefix}
In the alternative form |\childdocforwardprefix|,
%
\begin{center}
\begin{tabular}{l}
|\input{childdoc.def}|\\
|\childdocforwardprefix[|\textit{main}|]{|\textit{prefix}|}{|\textit{dest}|}|
\end{tabular}
\end{center}
%
the destination file is determined by a pattern
depending on the current file:
To make this work, the current file must be called
`{\textit{prefix}\hspace{0.2em}\textit{suffix}}'
with \textit{prefix} matching precisely the argument.
Processing is then passed on to the file
`{\textit{dest}\hspace{0.2em}\textit{suffix}}'.
Surely, the same effect is achieved by
directly specifying the
argument `{\textit{dest}\hspace{0.2em}\textit{suffix}}'
in the first form.
However, that requires to set up a different file
for each child. With the alternative form of the command
all these files can have exactly the same content
which simplifies setting them up and maintaining them.

For example, the following file |draft.tex|
with a compilation flag |\version| as described in \secref{sec:flags}
compiles the main document as a draft:
%
\begin{center}
\begin{tabular}{l}
|\def\version{draft}|\\
|\input{childdoc.def}|\\
|\childdocforward{|\textit{main}|}|
\end{tabular}
\end{center}
%
Likewise, the following files |final|\textit{nn}|.tex|
compile the final version of the child document
|child|\textit{nn}|.tex|:
%
\begin{center}
\begin{tabular}{l}
|\def\version{final}|\\
|\input{childdoc.def}|\\
|\childdocforwardprefix{final}{child}|
\end{tabular}
\end{center}
%

Note that when several versions of a main file and/or of each child file
are to be generated, it may be convenient to set up a |Makefile| or
shell script to automatise the process.

%%%%%%%%%%%%%%%%%%%%%%%%%%%%%%%%%%%%%%%%%%%%%%%%%%%%%%%%%%%%%%%%%%%%%%%%%%%%%%%%
\subsection{Command Line Processing}
\label{sec:commandline}

The effect of redirection files can also be achieved by invoking
the \LaTeX{} compiler with a more elaborate command line.
Most conveniently this should be done as part
of a shell script or a |Makefile|.

When using \textsf{childdoc} in the main file, the following
command lines effectively perform a redirection
(note that depending on the shell being used,
backslashes may have to be doubled: `|\|' $\to$ `|\\|'):
%
\begin{center}
|... -jobname "|\textit{target}|" |\\|"|[\textit{flags}]%
|\input{childdoc.def}\childdocforward[|\textit{main}|]{|\textit{dest}|}"|
\end{center}
%
Here \textit{target} is the name of the output file,
\textit{main} is the name of the main file
and \textit{dest} is the name of the main or child file to be processed
(all filenames without extensions).
The optional argument \textit{main} can be omitted
if \textit{main} matches \textit{dest}.
Optionally, compilation \textit{flags} can be defined via |\def| commands.
This command line makes the \TeX{} engine believe
it is compiling the file \textit{target}
whose content is specified as the latter parameter.
The provided code then forwards the processing to
\textit{main} or \textit{dest} as described in \secref{sec:forward}.

%%%%%%%%%%%%%%%%%%%%%%%%%%%%%%%%%%%%%%%%%%%%%%%%%%%%%%%%%%%%%%%%%%%%%%%%%%%%%%%%
\subsection{Include by Input}
\label{sec:input}

Including child documents by |\include| has some restrictions by design.
Most notably, the content of a child document always occupies
its own set of pages; pages cannot be shared between child documents.
Usually, this behaviour makes perfect sense
because each child document contain an essential part of the document.
However, in some situations it may be desirable to compose
a document from a collection of parts
without having mandatory page breaks between then.
For this case, the package
provides a mechanism to include parts
by |\input| which can also be processed individually.
However, by construction this mechanism
requires manual handling of the content to be output.

%%%%%%%%%%%%%%%%%%%%%%%%%%%%%%%%%%%%%%%%
\DescribeMacro{\ifchilddocmanual}
The main file should be prepared as usual, see \secref{sec:include}.
However, the document body must make a distinction
between processing of an individual part and of the main document, e.g.:
%
\begin{center}
\begin{tabular}{l}
|\ifchilddocmanual|\\
|\input{\childdocname}|\\
|\||else|\\
\textit{document body with }|\input{|\textit{part}|}|\\
|\||fi|
\end{tabular}
\end{center}
%
The conditional |\ifchilddocmanual| is true whenever
a part to be included by |\input| is being compiled,
and the name of the part is stored in |\childdocname|.

%%%%%%%%%%%%%%%%%%%%%%%%%%%%%%%%%%%%%%%%
\DescribeMacro{\childdocby}
Each part to be included by |\input| should start with:
%
\begin{center}
\begin{tabular}{l}
|\input{childdoc.def}|\\
|\childdocby{|\textit{main}|}|\\
\end{tabular}
\end{center}
%
The directive |\childdocby| is similar to |\childdocof|
described in \secref{sec:include},
but the subsequent selection of content must be done manually.
To that end, both |\ifchilddoc| and |\ifchilddocmanual|
will be true upon processing of a part,
and the name of the part is stored in |\childdocname|.
Note that |\jobname| will be set to the filename of the current part
so that each part receives an individual |.aux| file
that does not interfere with the |.aux| file(s) of the main document.
This behaviour can be altered by the alternative form
|\childdocby[*]{|\textit{main}|}| (with a non-empty optional argument)
which uses the |.aux| file of the main document
by setting |\jobname| to \textit{main}.

%%%%%%%%%%%%%%%%%%%%%%%%%%%%%%%%%%%%%%%%%%%%%%%%%%%%%%%%%%%%%%%%%%%%%%%%%%%%%%%%
\subsection{Driver Development}
\label{sec:driver}

The \textsf{childdoc} mechanism can also be use for the development
of definition files such as \LaTeX{} styles or classes.
This case differs from the above setup with multiple parts
included by |\include| in that no |\includeonly| should be invoked.
This can be achieved by starting the include file
(before |\ProvidesPackage|) with:
%
\begin{center}
\begin{tabular}{l}
|\input{childdoc.def}|\\
|\childdocforward{|\textit{main}|}|\\
\end{tabular}
\end{center}
%
or alternatively with:
%
\begin{center}
\begin{tabular}{l}
|\input{childdoc.def}|\\
|\childdocby{|\textit{main}|}|\\
\end{tabular}
\end{center}
%
Both forms have slightly different effects as described above.
The main file is prepared as usual, see \secref{sec:include}.

%%%%%%%%%%%%%%%%%%%%%%%%%%%%%%%%%%%%%%%%%%%%%%%%%%%%%%%%%%%%%%%%%%%%%%%%%%%%%%%%
\subsection{Legacy Detection}
\label{sec:detection}

The directive |\childdocmain| in the main file can detect
whether the complete document or merely a child is to be compiled
even without using the directive |\childdocof|.
This method is deprecated because it is less robust
and there is no compelling reason to use it;
it is merely provided for backward compatibility
and it may be removed in future versions.

If the detection mechanism is to be used,
it is mandatory to correctly specify
the filename of the main file as the argument of |\childdocmain|:
%
\begin{center}
\begin{tabular}{l}
|\input{childdoc.def}|\\
|\childdocmain{|\textit{main}|}|\\
\end{tabular}
\end{center}
%
If |\jobname| does not match the argument \textit{main} of |\childdocmain|,
it is assumed that |\jobname| points to the child file to be compiled.
When using |\childdocmain| with the main file specified as argument,
it suffices to start a child file
with just |\input{|\textit{main}|}|
without loading of the package and using |\childdocof|.
If instead all processing is done
with the appropriate \textsf{childdoc} directives,
the argument of \textit{main} of |\childdocmain| can be empty.

An alternative version of the command line processing described
in \secref{sec:commandline} using the detection mechanism reads:
%
\begin{center}
|... -jobname "|\textit{target}|" "|[\textit{flags}]%
[|\def\jobname{|\textit{dest}|}|]|\input{|\textit{main}|}"|
\end{center}

%%%%%%%%%%%%%%%%%%%%%%%%%%%%%%%%%%%%%%%%%%%%%%%%%%%%%%%%%%%%%%%%%%%%%%%%%%%%%%%%
\subsection{Manual Code}
\label{sec:manual}

In case one cannot be certain whether the definitions file |childdoc.def|
is installed on the target \TeX{} distribution
and one prefers not to ship it,
it is conceivable to paste a few relevant commands into the sources.

To that end, drop all statements |\input{childdoc.def}|
and perform the replacements as outlined below.
Instead of |\childdocmain{|\textit{main}|}| add the following code
to the top of the main file:
%
\begin{center}
\begin{tabular}{l}
|\||ifdefined\childdocname\endinput\||fi\newif\ifchilddoc|\\
|\edef\childdocname{\scantokens\expandafter{\jobname\noexpand}}|\\
|\def\childdocmain{|\textit{main}|}\||ifx\childdocmain\childdocname\||else|\\
|\childdoctrue\includeonly{\childdocname}\let\jobname\childdocmain\||fi|\\
\end{tabular}
\end{center}
%
Instead of |\childdocof{|\textit{main}|}| just include the main file
at the top of each child file:
%
\begin{center}
|\input{|\textit{main}|}|
\end{center}
%
A simple redirection |\childdocforward{|\textit{dest}|}| is achieved by:
%
\begin{center}
|\def\jobname{|\textit{dest}|}\input{\jobname}|
\end{center}
%
The redirection with prefix
|\childdocforwardprefix[|\textit{prefix}|]{|\textit{dest}|}|
is accomplished by:
%
\begin{center}
\begin{tabular}{l}
|{\edef\jobname{\scantokens\expandafter{\jobname\noexpand}}|\\
|\def\redirectjob |\textit{prefix}|#1~~~{\gdef\jobname{|\textit{dest}|#1}}|\\
|\expandafter\redirectjob\jobname~~~}\input{\jobname}|
\end{tabular}
\end{center}

In an alternative approach,
child documents can be compiled by a specific command line
without additional code or specific definitions:
%
\begin{center}
|... -jobname "|\textit{target}|" "|[\textit{flags}]%
|\includeonly{|\textit{dest}|}\input{|\textit{main}|}"|
\end{center}
%

%%%%%%%%%%%%%%%%%%%%%%%%%%%%%%%%%%%%%%%%%%%%%%%%%%%%%%%%%%%%%%%%%%%%%%%%%%%%%%%%
%%%%%%%%%%%%%%%%%%%%%%%%%%%%%%%%%%%%%%%%%%%%%%%%%%%%%%%%%%%%%%%%%%%%%%%%%%%%%%%%
\section{Information}

%%%%%%%%%%%%%%%%%%%%%%%%%%%%%%%%%%%%%%%%%%%%%%%%%%%%%%%%%%%%%%%%%%%%%%%%%%%%%%%%
\subsection{Copyright}

Copyright \copyright{} 2017--2018 Niklas Beisert

This work may be distributed and/or modified under the
conditions of the \LaTeX{} Project Public License, either version 1.3
of this license or (at your option) any later version.
The latest version of this license is in
  \url{http://www.latex-project.org/lppl.txt}
and version 1.3 or later is part of all distributions of \LaTeX{}
version 2005/12/01 or later.

This work has the LPPL maintenance status `maintained'.

The Current Maintainer of this work is Niklas Beisert.

This work consists of the files |README.txt|, |childdoc.ins| and |childdoc.dtx|
as well as the derived files |childdoc.def|, |cdocsamp.tex|
with |cdocsch1.tex|, |cdocsch2.tex|, |cdocspt3.tex|, |cdocspt4.tex|,
|cdocsdrf.tex|, |cdocsfn1.tex|, |cdocsfn2.tex|
as well as |childdoc.pdf|.

%%%%%%%%%%%%%%%%%%%%%%%%%%%%%%%%%%%%%%%%%%%%%%%%%%%%%%%%%%%%%%%%%%%%%%%%%%%%%%%%
\subsection{Files and Installation}

The package consists of the files:
%
\begin{center}
\begin{tabular}{ll}
    |README.txt|   & readme file \\
    |childdoc.ins| & installation file \\
    |childdoc.dtx| & source file \\
    |childdoc.def| & definition file \\
    |cdocsamp.tex| & sample main file \\
    |cdocsch1.tex| & sample include file \\
    |cdocsch2.tex| & sample include file \\
    |cdocspt3.tex| & sample part file \\
    |cdocspt4.tex| & sample part file \\
    |cdocsdrf.tex| & sample redirection file \\
    |cdocsfn1.tex| & sample redirection file \\
    |cdocsfn2.tex| & sample redirection file \\
    |childdoc.pdf| & manual
\end{tabular}
\end{center}
%
The distribution consists of the files
|README.txt|, |childdoc.ins| and |childdoc.dtx|.
%
\begin{itemize}
\item
Run (pdf)\LaTeX{} on |childdoc.dtx|
to compile the manual |childdoc.pdf| (this file).
\item
Run \LaTeX{} on |childdoc.ins| to create the definitions file |childdoc.def|
and the sample |cdocsamp.tex| with include files
|cdocsch1.tex|, |cdocsch2.tex|, |cdocspt3.tex|, |cdocspt4.tex|,
|cdocsdrf.tex|, |cdocsfn1.tex|, |cdocsfn2.tex|.
Then copy the file |childdoc.def| to an appropriate directory of your \LaTeX{}
distribution, e.g.\ \textit{texmf-root}|/tex/latex/childdoc|.
\end{itemize}

%%%%%%%%%%%%%%%%%%%%%%%%%%%%%%%%%%%%%%%%%%%%%%%%%%%%%%%%%%%%%%%%%%%%%%%%%%%%%%%%
\subsection{Related CTAN Packages}

There are several other packages which offer a similar functionality:
%
\begin{itemize}
\item
The packages
\href{http://ctan.org/pkg/docmute}{\textsf{docmute}},
\href{http://ctan.org/pkg/includex}{\textsf{includex}} and
\href{http://ctan.org/pkg/standalone}{\textsf{standalone}}
provide commands to include only the document body of
a child file thus allowing both files to be compiled individually.
\item
The packages \href{http://ctan.org/pkg/subdocs}{\textsf{subdocs}}
and \href{http://ctan.org/pkg/subfiles}{\textsf{subfiles}}
provide structures in which the main and child documents can be
encapsulated and allowing them to be compiled individually.
The inclusion mechanism is different from the conventional |\include|.
\item
The package \href{http://ctan.org/pkg/combine}{\textsf{combine}}
is an elaborate solution to combine several documents into one.
\end{itemize}
%
See also the CTAN topic \href{http://ctan.org/topic/subdocs}{\textsf{subdocs}}
for further related packages.
The present package differs from the above solutions in that
a document structure constructed with the conventional |\include| mechanism
just needs two extra commands at the top of every file
such that all constituent files can be compiled individually.

%%%%%%%%%%%%%%%%%%%%%%%%%%%%%%%%%%%%%%%%%%%%%%%%%%%%%%%%%%%%%%%%%%%%%%%%%%%%%%%%
%\subsection{Feature Suggestions}
%
%The following is a list of features which may be useful for future
%versions of this package:
%%
%\begin{itemize}
%\item
%\ldots
%\end{itemize}

%%%%%%%%%%%%%%%%%%%%%%%%%%%%%%%%%%%%%%%%%%%%%%%%%%%%%%%%%%%%%%%%%%%%%%%%%%%%%%%%
\subsection{Revision History}

%%%%%%%%%%%%%%%%%%%%%%%%%%%%%%%%%%%%%%%%
\paragraph{v2.0:} 2018/12/30

\begin{itemize}
\item
immediate forward processing
\item
added |\childdocby| mechanism
\item
manual restructured
\end{itemize}

%%%%%%%%%%%%%%%%%%%%%%%%%%%%%%%%%%%%%%%%
\paragraph{v1.6:} 2018/01/17

\begin{itemize}
\item
application for development of include files
\item
corrections to manual
\end{itemize}

%%%%%%%%%%%%%%%%%%%%%%%%%%%%%%%%%%%%%%%%
\paragraph{v1.5:} 2017/05/21

\begin{itemize}
\item
more complete structuring introduced
\item
|\childdocof| introduced
\item
|\childdoc| renamed to |\childdocmain|
\item
|\childredirect| renamed to |\childdocforward| and |\childdocforwardprefix|
and functionality expanded
\end{itemize}

%%%%%%%%%%%%%%%%%%%%%%%%%%%%%%%%%%%%%%%%
\paragraph{v1.0:} 2017/04/27

\begin{itemize}
\item
manual and install package
\item
first version published on CTAN
\end{itemize}

%%%%%%%%%%%%%%%%%%%%%%%%%%%%%%%%%%%%%%%%
\paragraph{v0.6:} 2017/04/26

\begin{itemize}
\item
redirection mechanism added
\end{itemize}

%%%%%%%%%%%%%%%%%%%%%%%%%%%%%%%%%%%%%%%%
\paragraph{v0.5:} 2017/04/26

\begin{itemize}
\item
functionality in definition file
\end{itemize}


%%%%%%%%%%%%%%%%%%%%%%%%%%%%%%%%%%%%%%%%%%%%%%%%%%%%%%%%%%%%%%%%%%%%%%%%%%%%%%%%
%%%%%%%%%%%%%%%%%%%%%%%%%%%%%%%%%%%%%%%%%%%%%%%%%%%%%%%%%%%%%%%%%%%%%%%%%%%%%%%%
%%%%%%%%%%%%%%%%%%%%%%%%%%%%%%%%%%%%%%%%%%%%%%%%%%%%%%%%%%%%%%%%%%%%%%%%%%%%%%%%
\appendix

\settowidth\MacroIndent{\rmfamily\scriptsize 000\ }

 \DocInput{childdoc.dtx}

\end{document}
%</driver>
% \fi
%
% %%%%%%%%%%%%%%%%%%%%%%%%%%%%%%%%%%%%%%%%%%%%%%%%%%%%%%%%%%%%%%%%%%%%%%%%%%%%%%
% %%%%%%%%%%%%%%%%%%%%%%%%%%%%%%%%%%%%%%%%%%%%%%%%%%%%%%%%%%%%%%%%%%%%%%%%%%%%%%
% \section{Sample}
%\iffalse
%<*samplemain>
%\fi
%
% The following presents a sample document
% with two chapters, two parts, a title page,
% a compile flag as well as three forwarding files to set the flag.
% It consists of eight |.tex| files:
% \begin{center}
% \begin{tabular}{ll}
% |cdocsamp.tex|&main file\\
% |cdocsch1.tex|&include file for chapter 1\\
% |cdocsch2.tex|&include file for chapter 2\\
% |cdocspt3.tex|&include file for part 3\\
% |cdocspt4.tex|&include file for part 4\\
% |cdocsdrf.tex|&forwarding file for main file in draft mode\\
% |cdocsfi1.tex|&forwarding file for final version of chapter 1\\
% |cdocsfi2.tex|&forwarding file for final version of chapter 2\\
% \end{tabular}
% \end{center}
% Each of the eight files can be compiled directly by the \LaTeX{} compiler.
%
% %%%%%%%%%%%%%%%%%%%%%%%%%%%%%%%%%%%%%%
% \paragraph{Main File.}
%
% The main file is called |cdocsamp.tex|.
%
% Load the \textsf{childdoc} definitions and
% declare the filename for the main document:
%    \begin{macrocode}
\input{childdoc.def}
\childdocmain{}
%    \end{macrocode}

% Optional override for |\version| flag:
%    \begin{macrocode}
%%\ifchilddoc\else\providecommand{\version}{draft}\fi
%    \end{macrocode}

% Define the default values for the |\version| flag
% (|final| for the main file and |draft| for childs):
%    \begin{macrocode}
\ifchilddoc
\providecommand{\version}{draft}
\else
\providecommand{\version}{final}
\fi
%    \end{macrocode}

% Load the standard document class:
%    \begin{macrocode}
\documentclass[12pt]{article}
%    \end{macrocode}

% Start the document body:
%    \begin{macrocode}
\begin{document}
%    \end{macrocode}

% Declare a title page.
% Print title, part of document being processed and version flag:
%    \begin{macrocode}
\addtocounter{page}{-1}
\begin{center}
{\LARGE\bfseries{}childdoc example\par}
\vspace{1cm}
\ifchilddoc
\ifchilddocmanual part\else chapter\fi:
`\childdocname' of `\childdocjob'\par
\else
main document: `\childdocjob'\par
\fi
version: \version\par
\end{center}
\newpage
%    \end{macrocode}

% Manually include selected file,
% otherwise process as usual:
%    \begin{macrocode}
\ifchilddocmanual
\section*{part `\childdocname'}
\input{\childdocname}
\else
%    \end{macrocode}

% Include the two chapters:
%    \begin{macrocode}
\include{cdocsch1}
\include{cdocsch2}
%    \end{macrocode}

% Include the two parts unless only chapters should be displayed:
%    \begin{macrocode}
\ifchilddoc\else
\section{part three}
\input{cdocspt3}
\section{part four}
\input{cdocspt4}
\fi
%    \end{macrocode}

% Process as usual until here:
%    \begin{macrocode}
\fi
%    \end{macrocode}

% End of document body:
%    \begin{macrocode}
\end{document}
%    \end{macrocode}
%\iffalse
%</samplemain>
%\fi
%
% %%%%%%%%%%%%%%%%%%%%%%%%%%%%%%%%%%%%%%
% \paragraph{Chapter Include Files.}
%
% The include files are called |cdocsch1.tex| and |cdocsch2.tex|.
%
%\iffalse
%<*samplechap1|samplechap2>
%\fi

% Optional override for |\version| flag:
%    \begin{macrocode}
%%\providecommand{\version}{final}
%    \end{macrocode}

% Include the main document:
%    \begin{macrocode}
\input{childdoc.def}
\childdocof{cdocsamp}
%    \end{macrocode}

%\iffalse
%</samplechap1|samplechap2>
%\fi
%
%\iffalse
%<*samplechap1>
%\fi
% Some text for chapter 1:
%    \begin{macrocode}
\section{one}
some text in chapter one
%    \end{macrocode}

%\iffalse
%</samplechap1>
%\fi
% Some text for chapter 2:
%\iffalse
%<*samplechap2>
%\fi
%    \begin{macrocode}
\section{two}
more text in chapter two
%    \end{macrocode}

%\iffalse
%</samplechap2>
%\fi
%
% %%%%%%%%%%%%%%%%%%%%%%%%%%%%%%%%%%%%%%
% \paragraph{Part Include Files.}
%
% The include files are called |cdocspt3.tex| and |cdocspt4.tex|.
%
%\iffalse
%<*samplepart3|samplepart4>
%\fi

% Optional override for |\version| flag:
%    \begin{macrocode}
%%\providecommand{\version}{final}
%    \end{macrocode}

% Include the main document:
%    \begin{macrocode}
\input{childdoc.def}
\childdocby{cdocsamp}
%    \end{macrocode}

%\iffalse
%</samplepart3|samplepart4>
%\fi
%
%\iffalse
%<*samplepart3>
%\fi
% Some text for part 3:
%    \begin{macrocode}
some text in part three
%    \end{macrocode}

%\iffalse
%</samplepart3>
%\fi
% Some text for part 4:
%\iffalse
%<*samplepart4>
%\fi
%    \begin{macrocode}
more text in part four
%    \end{macrocode}

%\iffalse
%</samplepart4>
%\fi
%
% %%%%%%%%%%%%%%%%%%%%%%%%%%%%%%%%%%%%%%
% \paragraph{Forwarding for a Complete Draft.}
%
% The following forwarding file |cdocsdrf.tex|
% compiles the main document in draft mode:
%\iffalse
%<*sampledraft>
%\fi
%    \begin{macrocode}
\def\version{draft}
\input{childdoc.def}
\childdocforward{cdocsamp}
%    \end{macrocode}

%\iffalse
%</sampledraft>
%\fi
%
% %%%%%%%%%%%%%%%%%%%%%%%%%%%%%%%%%%%%%%
% \paragraph{Forwarding for Final Version of the Chapters.}
%
% The following forwarding files |cdocsfn1.tex| and |cdocsfn2.tex|
% (with identical content)
% compile the final versions of the child documents
% |cdocsch1.tex| and |cdocsch2.tex|, respectively:
%\iffalse
%<*samplefinal>
%\fi
%    \begin{macrocode}
\def\version{final}
\input{childdoc.def}
\childdocforwardprefix[cdocsamp]{cdocsfn}{cdocsch}
%    \end{macrocode}

%\iffalse
%</samplefinal>
%\fi
%
% %%%%%%%%%%%%%%%%%%%%%%%%%%%%%%%%%%%%%%
% \paragraph{Command Line Processing.}
%
% The following three command lines generate the output files
% |cdocscld|, |cdocscl1| and |cdocscl2|
% which should be identical to
% |cdocsdrf|, |cdocsch1| and |cdocsfn2|, respectively:
% \begin{center}
% \begin{tabular}{l}
% |latex -jobname cdocscld \|\\
% |  "\def\version{draft}\input{childdoc.def}\childdocforward{cdocsamp}"|\\
% |latex -jobname cdocscl1 \|\\
% |  "\input{childdoc.def}\childdocforward[cdocsamp]{cdocsch1}"|\\
% |latex -jobname cdocscl2 \|\\
% |  "\def\version{final}\input{childdoc.def}\childdocforward{cdocsch2}"|
% \end{tabular}
% \end{center}
% Note that the trailing backslash on each first line
% merely continues the input to the second line
% (for convenient cut ant paste).
% Furthermore, the command |latex| can be replaced by any
% of its alternative versions such as |pdflatex|.
%
% %%%%%%%%%%%%%%%%%%%%%%%%%%%%%%%%%%%%%%%%%%%%%%%%%%%%%%%%%%%%%%%%%%%%%%%%%%%%%%
% %%%%%%%%%%%%%%%%%%%%%%%%%%%%%%%%%%%%%%%%%%%%%%%%%%%%%%%%%%%%%%%%%%%%%%%%%%%%%%
% \section{Implementation}
%\iffalse
%<*package>
%\fi
%
% This section describes the definitions file |childdoc.def|.

% The definitions cannot be loaded using |\usepackage| or |\RequirePackage|
% which has a mechanism to prevent loading a style file more than once.
% When loading the definitions by means of |\input|
% multiple instances have to be prevented manually:
%\iffalse
%This code needs to be before the `\ProvidesFile' directive
%which is defined at the beginning of this file.
%Therefore it is also placed there and commented out here.
%</package>
%<*discard>
%\fi
%    \begin{macrocode}
\ifdefined\childdocmain\endinput\fi
%    \end{macrocode}
%\iffalse
%</discard>
%<*package>
%\fi
%
% \macro{\ifchilddoc}
% \macro{\ifchilddocmanual}
% The conditional |\ifchilddoc| tells whether a
% child (true) or main (false) document is being compiled.
% The conditional |\ifchilddocmanual| tells whether
% the |\includeonly| mechanism is used (false) or
% the selection of child files must be performed manually (true).
% The definitions initialise to false:
%    \begin{macrocode}
\newif\ifchilddoc
\newif\ifchilddocmanual
%    \end{macrocode}

% \macro{\childdocname}
% \macro{\childdocjob}
% The macro |\childdocname| stores the name of the main document
% to be compiled. The macro |\childdocjob| stores the name of
% the document on which the \LaTeX{} compiler was originally invoked.
% The content of |\jobname| cannot be compared
% to filenames specified in the source due to different catcodes.
% The following code rescans |\jobname|, stores the result
% in |\childdocname| and saves a copy in |\childdocjob|:
%    \begin{macrocode}
\edef\childdocname{\scantokens\expandafter{\jobname\noexpand}}
\let\childdocjob\childdocname
%    \end{macrocode}

% \macro{\childdocdisable}
% The macro |\childdocdisable| prevents the main file
% from being processed more than once.
% At this stage, the main document command |\childdocmain|
% is assumed to be called once again where it should do nothing.
% Any subsequent call to it should prevent
% a secondary processing of the main document
% It overwrites the forwarding commands
% |\childdocof| and |\childdocforward|
% with empty macros to prevent further inclusions of the main document:
%    \begin{macrocode}
\newcommand{\childdocdisable}
{
  \renewcommand{\childdocmain}[1]{\renewcommand{\childdocmain}[1]{\endinput}}
  \renewcommand{\childdocof}[1]{}
  \renewcommand{\childdocby}[2][]{}
  \renewcommand{\childdocforward}[2][]{}
  \renewcommand{\childdocdisable}{}
}
%    \end{macrocode}

% \macro{\childdocmain}
% The macro |\childdocmain| is to be called at the top of the main file
% with nothing or the main filename (without extension) as argument.
% First, it breaks loops.
% If the argument is not empty and does not match |\childdocname|
% (which is set by the first inclusion of |childdoc.def|),
% |\ifchilddoc| is set to true, |\includeonly| is applied to the child file
% and |\jobname| is set to the main file
% (for proper handling of |.aux| files):
%    \begin{macrocode}
\newcommand{\childdocmain}[1]
{
  \childdocdisable\childdocmain{}
  \if?#1?\else
    \begingroup
      \def\childdoctmp{#1}
      \ifx\childdoctmp\childdocname
        \def\childdoctmp{}
      \else
        \def\childdoctmp
        {
          \childdoctrue
          \includeonly{\childdocname}
          \def\childdocjob{#1}
          \def\jobname{#1}
        }
      \fi
      \expandafter
    \endgroup
    \childdoctmp
  \fi
}
%    \end{macrocode}

% \macro{\childdocof}
% The command |\childdocof| redirects
% compilation to the main file |#1|.
%    \begin{macrocode}
\newcommand{\childdocof}[1]
{
  \childdocdisable
  \childdoctrue
  \includeonly{\childdocname}
  \def\jobname{#1}
  \def\childdocjob{#1}
  \input{#1}
}
%    \end{macrocode}

% \macro{\childdocby}
% The command |\childdocby| ....
%    \begin{macrocode}
\newcommand{\childdocby}[2][]
{
  \childdocdisable
  \childdoctrue
  \childdocmanualtrue
  \if?#1?\else
    \def\jobname{#2}
  \fi
  \def\childdocjob{#2}
  \input{#2}
  \endinput
}
%    \end{macrocode}

% \macro{\childdocforward}
% The command |\childdocforward| redirects
% compilation to the main file or
% (if the optional argument is given) a child file.
% Parameters are set as if the main file
% or a child file starting with |\childdocof| was compiled.
% Then compilation is handed over to the main file:
%    \begin{macrocode}
\newcommand{\childdocforward}[2][]
{
  \begingroup
    \if?#1?
      \def\childdoctmp
      {
        \def\childdocname{#2}
        \def\childdocjob{#2}
        \def\jobname{#2}
        \input{#2}
        \endinput
      }
    \else
      \def\childdoctmp
      {
        \childdocdisable
        \def\childdocname{#2}
        \childdoctrue
        \includeonly{#2}
        \def\childdocjob{#1}
        \def\jobname{#1}
        \input{#1}
        \endinput
      }
    \fi
    \expandafter
  \endgroup
  \childdoctmp
}
%    \end{macrocode}

% \macro{\childdocforwardprefix}
% The command |\childdocforwardprefix| redirects
% compilation to the main or a child file by means of a pattern.
% The prefix |#1| in the current filename is replaced by |#2|
% and the suffix of the current filename is kept
% (it is assumed that the filename does not contain the substring `|~~~|'
% which is used as a delimiter).
% Compilation is handed over to the new file by |\childdocforward|:
%    \begin{macrocode}
\newcommand{\childdocforwardprefix}[3][]
{
  \begingroup
    \def\childdocextract #2##1~~~{\def\childdoctmp{\childdocforward[#1]{#3##1}}}
    \expandafter\childdocextract\childdocname~~~
    \expandafter
  \endgroup
  \childdoctmp
}
%    \end{macrocode}

% \macro{\childdoc}
% The deprecated macro |\childdoc| is a legacy version of |\childdocmain|:
%    \begin{macrocode}
\newcommand{\childdoc}{\childdocmain}
%    \end{macrocode}

% \macro{\childdocredirect}
% The deprecated macro |\childdocredirect| is a legacy version
% of |\childdocforward| and |\childdocforwardprefix|:
%    \begin{macrocode}
\newcommand{\childdocredirect}[2][]
{
  \begingroup
    \if?#1?
      \def\childdoctmp{\childdocforward{#2}}
    \else
      \def\childdoctmp{\childdocforwardprefix{#1}{#2}}
    \fi
    \expandafter
  \endgroup
  \childdoctmp
}
%    \end{macrocode}

%\iffalse
%</package>
%\fi
%
\endinput
|\\
|\childdocforward{|\textit{main}|}|\\
\end{tabular}
\end{center}
%
or alternatively with:
%
\begin{center}
\begin{tabular}{l}
|% \iffalse
%
% childdoc.dtx Copyright (C) 2017-2018 Niklas Beisert
%
% This work may be distributed and/or modified under the
% conditions of the LaTeX Project Public License, either version 1.3
% of this license or (at your option) any later version.
% The latest version of this license is in
%   http://www.latex-project.org/lppl.txt
% and version 1.3 or later is part of all distributions of LaTeX
% version 2005/12/01 or later.
%
% This work has the LPPL maintenance status `maintained'.
%
% The Current Maintainer of this work is Niklas Beisert.
%
% This work consists of the files childdoc.dtx and childdoc.ins
% and the derived files childdoc.def and cdocsamp.tex with
% cdocsch1.tex, cdocsch2.tex, cdocsdrf.tex, cdocsfn1.tex, cdocsfn2.tex.
%
%<package>\ifdefined\childdocmain\endinput\fi
%<package>\ProvidesFile{childdoc.def}[2018/12/30 v2.0 child document driver]
%<samplemain>\ProvidesFile{cdocsamp.tex}[2018/12/30 v2.0 sample for childdoc]
%<*driver>
%\ProvidesFile{childdoc.drv}[2018/12/30 v2.0 childdoc reference manual file]
\PassOptionsToClass{10pt,a4paper}{article}
\documentclass{ltxdoc}

\usepackage[margin=35mm]{geometry}
\usepackage{hyperref}
\usepackage{hyperxmp}
\usepackage[usenames]{color}

\hypersetup{colorlinks=true}
\hypersetup{pdfstartview=FitH}
\hypersetup{pdfpagemode=UseNone}
\hypersetup{pdfsource={}}
\hypersetup{pdflang={en-UK}}
\hypersetup{pdfcopyright={Copyright 2017-2018 Niklas Beisert.
  This work may be distributed and/or modified under the
  conditions of the LaTeX Project Public License, either version 1.3
  of this license or (at your option) any later version.}}
\hypersetup{pdflicenseurl={http://www.latex-project.org/lppl.txt}}
\hypersetup{pdfcontactaddress={ETH Zurich, ITP, HIT K,
  Wolfgang-Pauli-Strasse 27}}
\hypersetup{pdfcontactpostcode={8093}}
\hypersetup{pdfcontactcity={Zurich}}
\hypersetup{pdfcontactcountry={Switzerland}}
\hypersetup{pdfcontactemail={nbeisert@itp.phys.ethz.ch}}
\hypersetup{pdfcontacturl={http://people.phys.ethz.ch/\xmptilde nbeisert/}}

\newcommand{\secref}[1]{\hyperref[#1]{section \ref*{#1}}}

\parskip1ex
\parindent0pt
\let\olditemize\itemize
\def\itemize{\olditemize\parskip0pt}

\begin{document}

\title{The \textsf{childdoc} Package}
\hypersetup{pdftitle={The childdoc Package}}
\author{Niklas Beisert\\[2ex]
  Institut f\"ur Theoretische Physik\\
  Eidgen\"ossische Technische Hochschule Z\"urich\\
  Wolfgang-Pauli-Strasse 27, 8093 Z\"urich, Switzerland\\[1ex]
  \href{mailto:nbeisert@itp.phys.ethz.ch}
  {\texttt{nbeisert@itp.phys.ethz.ch}}}
\hypersetup{pdfauthor={Niklas Beisert}}
\hypersetup{pdfsubject={Manual for the LaTeX2e Package childdoc}}
\date{30 December 2018, \textsf{v2.0}}
\maketitle

\begin{abstract}\noindent
\textsf{childdoc} is a \LaTeXe{} package
that enables the direct compilation
of document sections included by |\include|
to individual files.
\end{abstract}

\begingroup
\parskip0ex
\tableofcontents
\endgroup

%%%%%%%%%%%%%%%%%%%%%%%%%%%%%%%%%%%%%%%%%%%%%%%%%%%%%%%%%%%%%%%%%%%%%%%%%%%%%%%%
%%%%%%%%%%%%%%%%%%%%%%%%%%%%%%%%%%%%%%%%%%%%%%%%%%%%%%%%%%%%%%%%%%%%%%%%%%%%%%%%
\section{Introduction}

\LaTeX{} provides a mechanism to structure a large document (such as a book)
into a main file and several child files (containing the chapters)
using the |\include| command.
This mechanism is beneficial for documents
which span hundreds of pages in order to
make the source file(s) more manageable.
Moreover, compilation can be restricted to
selected child files by means of the |\includeonly| command.
The latter feature can be used to reduce the compilation time while editing
(this was significantly more useful in the earlier days of \LaTeX{})
or to generate a smaller document which is easier to navigate.
Another application of |\includeonly| is to generate
documents consisting of selected parts of the complete document.

However, there are a few drawbacks of the plain |\include| mechanism:
\begin{itemize}
\item
The child files cannot be compiled on their own,
they can only be compiled via the main file.
A naive editing environment
(such as a text editor with an option
to have the current file processed by \LaTeX)
may require one to switch to the main file before compiling;
attempting to compile the child file produces errors.
\item
The main file must be modified (each time)
to adjust the |\includeonly| command
to the present needs. This easily leaves the main file in a messy state.
\item
The generated document will always carry the filename
of the main document. This is inconvenient if
several child files are to be compiled and
to be kept for distribution.
\end{itemize}

The present package provides a simple interface
to make child files individually compilable by \LaTeX{}.
Compiling a child file then has the same effect as compiling
the main file with an |\includeonly| command
to select the appropriate child.
Moreover the generated document will carry the name of the child
rather than the main file.
This resolves all three above issues.

This feature is meant to make the editing of books,
thesis documents and lecture notes somewhat more convenient.
However, the package can also be used efficiently for
composing a series of documents (such as exercise sheets)
which are typically distributed individually.
It then assists the author in generating the individual documents
(potentially in different versions)
as well as a document containing the collected series.
Another application is in developing style files
or other kinds of included material
where compilation of the style file could redirect
to a sample or test file.

%%%%%%%%%%%%%%%%%%%%%%%%%%%%%%%%%%%%%%%%%%%%%%%%%%%%%%%%%%%%%%%%%%%%%%%%%%%%%%%%
%%%%%%%%%%%%%%%%%%%%%%%%%%%%%%%%%%%%%%%%%%%%%%%%%%%%%%%%%%%%%%%%%%%%%%%%%%%%%%%%
\section{Usage}

First of all, the package \textsf{childdoc} is \emph{not} a standard
\LaTeXe{} |.sty| style file! Therefore it needs to be invoked in
a non-standard way.

%%%%%%%%%%%%%%%%%%%%%%%%%%%%%%%%%%%%%%%%%%%%%%%%%%%%%%%%%%%%%%%%%%%%%%%%%%%%%%%%
\subsection{Included Files}
\label{sec:include}

%%%%%%%%%%%%%%%%%%%%%%%%%%%%%%%%%%%%%%%%
\DescribeMacro{\childdocmain}
To use the package, add the commands
\begin{center}
\begin{tabular}{l}
|\input{childdoc.def}|\\
|\childdocmain{}|\\
\end{tabular}
\end{center}
at the very top of the main \LaTeX{} file,
in particular \emph{before} the |\documentclass| statement!
The argument of |\childdocmain| should be left empty
(but it must be present).

%%%%%%%%%%%%%%%%%%%%%%%%%%%%%%%%%%%%%%%%
\DescribeMacro{\childdocof}
Furthermore, add the commands
\begin{center}
\begin{tabular}{l}
|\input{childdoc.def}|\\
|\childdocof{|\textit{main}|}|\\
\end{tabular}
\end{center}
at the top of every child file \textit{child}
which is included by |\include{|\textit{child}|}|
from within the main file
(or at least for those files to be compiled individually).
The argument \textit{main} must be the filename of the main file.

There are a couple of
considerations in setting up the main and child documents:

%%%%%%%%%%%%%%%%%%%%%%%%%%%%%%%%%%%%%%%%
\paragraph{Restrictions.}

Please note the following restrictions:
\begin{itemize}
\item
|\childdocmain| must be called with one argument \textit{main}
to ensure compatibility with earlier version of the package.
It must either be empty (|\childdocmain{}|)
or precisely match the filename of the main file in which it is specified.
See \secref{sec:detection} for further information.
\item
The filename \textit{main} must be specified without the |.tex| extension.
\item
The filename \textit{main} is case sensitive
(even in case-insensitive file systems)
due to internal string comparison.
\item
The argument \textit{main} should be fully expanded, it cannot be a macro.
\item
Subdirectories and special characters should be avoided in filenames.
\item
The command |\childdocmain{|\textit{main}|}| must be followed by a whitespace.
It should not be followed immediately by another command
or by a comment mark `|%|'.
This is because the \TeX{} parser reads the token immediately following
the argument of |\childdocmain| and puts it
at the beginning of every child section;
however, a white\-space is ignored.
\end{itemize}

%%%%%%%%%%%%%%%%%%%%%%%%%%%%%%%%%%%%%%%%
\paragraph{Content of Main File.}

It is advisable to place all content in the child files included by |\include|.
Any output contained in the main file will appear in all child documents
unless suppressed manually;
it cannot be suppressed automatically by the |\includeonly| directive
and thus should normally be avoided.
A method to include some content in the main file
by means of conditional processing is described in \secref{sec:conditional}.

%%%%%%%%%%%%%%%%%%%%%%%%%%%%%%%%%%%%%%%%
\paragraph{Page Numbering.}

When only a part of the document is compiled,
the appropriate numbering of pages
(as well as other status parameters)
is determined from the |.aux| files.
The latter contain information from previous passes.
However this information needs to propagate through
all intermediate child documents.
Therefore the page numbering in child documents may well
be inconsistent until the complete document is compiled at least once.

A useful (if unconventional) way to always ensure a consistent
page numbering is to restart the numbering in each child document
and denote the pages by `\textit{child}|.|\textit{page}'
where \textit{child} represents the chapter/section number of the child file.
This can be achieved by the command
|\numberwithin{page}{|\textit{child}|}|
of the \textsf{amsmath} package
where \textit{child} can be |chapter| or |section|
depending on the chosen structuring.
Alternatively, one can modify the macro |\thepage| appropriately
and reset the counter |page| at the start of each child file.

%%%%%%%%%%%%%%%%%%%%%%%%%%%%%%%%%%%%%%%%%%%%%%%%%%%%%%%%%%%%%%%%%%%%%%%%%%%%%%%%
\subsection{Conditional Processing}
\label{sec:conditional}

The package provides a mechanism to compile different versions
of a document. To customise the versions further some conditional processing
can come in handy to distinguish which version is being compiled.
The package provides two macros to describe the compilation context:

%%%%%%%%%%%%%%%%%%%%%%%%%%%%%%%%%%%%%%%%
\DescribeMacro{\ifchilddoc}
The conditional |\ifchilddoc| distinguishes between the compilation of
child documents and the main document:
%
\begin{center}
|\ifchilddoc |\textit{child-code}| |[|\||else |\textit{main-code}]| \||fi|
\end{center}

%%%%%%%%%%%%%%%%%%%%%%%%%%%%%%%%%%%%%%%%
\DescribeMacro{\childdocname}
\DescribeMacro{\childdocjob}
The macro |\childdocname| contains the filename (without extension)
of the main or child file being processed.
Note that |\childdocjob| will always contain the name of the main file.

%%%%%%%%%%%%%%%%%%%%%%%%%%%%%%%%%%%%%%%%
\paragraph{Title Page.}

Conditional processing can be used to include a title or banner page
in the main document when proper precautions are taken.
Importantly, the code in the main file should ensure that the page counter
(as well as other status parameters which are stored in the |.aux| files)
takes the same value after the conditional processing.
Otherwise the page numbers may take divergent values
depending on which part is compiled.

For example, a title page could be declared by:
%
\begin{center}
\begin{tabular}{l}
|\ifchilddoc\||else|\\
|\addtocounter{page}{-1}|\\
\textit{code for title page}\\
|\newpage|\\
|\||fi|
\end{tabular}
\end{center}
%
A banner page for the child documents can be generated by:
%
\begin{center}
\begin{tabular}{l}
|\ifchilddoc|\\
|\addtocounter{page}{-1}|\\
\textit{code for banner page}\\
|\newpage|\\
|\||fi|
\end{tabular}
\end{center}
%
Here one could write a message such as:
\begin{center}
|This is the part \childdocname{} of \childdocjob{}.|
\end{center}

%%%%%%%%%%%%%%%%%%%%%%%%%%%%%%%%%%%%%%%%%%%%%%%%%%%%%%%%%%%%%%%%%%%%%%%%%%%%%%%%
\subsection{Flags}
\label{sec:flags}

The package makes it easy to generate different versions
of the main or child documents.
To this end compilation flags can be defined
and assigned different default values.
They will be particularly useful in conjunction
with the forwarding mechanism described in \secref{sec:forward}.

For example, it may be useful to have a flag |\version|
which can be set to |draft| or |final|.
The document source will contain some conditional code
depending on the value of |\version|.
Suppose further, the flag should default to |final| for the main file
and to |draft| for child files
which is a natural assignment for editing the document.
This is achieved by placing the following code
in the preamble of the main document
(below the |\childdocmain| directive):
%
\begin{center}
\begin{tabular}{l}
|\ifchilddoc|\\
|\providecommand{\version}{draft}|\\
|\||else|\\
|\providecommand{\version}{final}|\\
|\||fi|
\end{tabular}
\end{center}
%
The definition by |\providecommand| makes sure
that previous definitions are not overwritten.
Further statements |\providecommand{\version}{...}|
can thus be added before the above code to override it.

For the main file, one might add a line
(between |\childdocmain| and the above block)
%
\begin{center}
|%\ifchilddoc\||else\providecommand{\version}{draft}\||fi|
\end{center}
%
which can be uncommented to produce a draft version.
Likewise one can add a line to the very top of a child file
(above the |\childdocof{|\textit{main}|}| directive)
%
\begin{center}
|%\providecommand{\version}{final}|
\end{center}
%
which can be uncommented to produce the final version of this child document.

%%%%%%%%%%%%%%%%%%%%%%%%%%%%%%%%%%%%%%%%%%%%%%%%%%%%%%%%%%%%%%%%%%%%%%%%%%%%%%%%
\subsection{Forwarding}
\label{sec:forward}

Different versions of the main or child documents
using compilation flags as described in \secref{sec:flags}
can be (permanently) stored in different files
for convenient compilation, viewing and distribution.
To this end, the package defines a command
to pass on compilation to a different file:

%%%%%%%%%%%%%%%%%%%%%%%%%%%%%%%%%%%%%%%%
\DescribeMacro{\childdocforward}
The command |\childdocforward| redirects processing to
another source file:
%
\begin{center}
\begin{tabular}{l}
|\input{childdoc.def}|\\
|\childdocforward[|\textit{main}|]{|\textit{dest}|}|\\
\end{tabular}
\end{center}
%
The argument \textit{dest} is the destination file
(without extension).
It should be the main file or one of the child files.
Note that further \textsf{childdoc} directives
such as |\childdocof| and |\childdocforward|
in the indicated file will be processed in this form.
The optional argument \textit{main}
passes on directly to the main file \textit{main}
while pretending to compile the child \textit{dest}.
This form behaves as if \textit{dest}
issues |\childdocof{|\textit{main}|}| right away,
and no further \textsf{childdoc} directives will be processed.

%%%%%%%%%%%%%%%%%%%%%%%%%%%%%%%%%%%%%%%%
\DescribeMacro{\...prefix}
In the alternative form |\childdocforwardprefix|,
%
\begin{center}
\begin{tabular}{l}
|\input{childdoc.def}|\\
|\childdocforwardprefix[|\textit{main}|]{|\textit{prefix}|}{|\textit{dest}|}|
\end{tabular}
\end{center}
%
the destination file is determined by a pattern
depending on the current file:
To make this work, the current file must be called
`{\textit{prefix}\hspace{0.2em}\textit{suffix}}'
with \textit{prefix} matching precisely the argument.
Processing is then passed on to the file
`{\textit{dest}\hspace{0.2em}\textit{suffix}}'.
Surely, the same effect is achieved by
directly specifying the
argument `{\textit{dest}\hspace{0.2em}\textit{suffix}}'
in the first form.
However, that requires to set up a different file
for each child. With the alternative form of the command
all these files can have exactly the same content
which simplifies setting them up and maintaining them.

For example, the following file |draft.tex|
with a compilation flag |\version| as described in \secref{sec:flags}
compiles the main document as a draft:
%
\begin{center}
\begin{tabular}{l}
|\def\version{draft}|\\
|\input{childdoc.def}|\\
|\childdocforward{|\textit{main}|}|
\end{tabular}
\end{center}
%
Likewise, the following files |final|\textit{nn}|.tex|
compile the final version of the child document
|child|\textit{nn}|.tex|:
%
\begin{center}
\begin{tabular}{l}
|\def\version{final}|\\
|\input{childdoc.def}|\\
|\childdocforwardprefix{final}{child}|
\end{tabular}
\end{center}
%

Note that when several versions of a main file and/or of each child file
are to be generated, it may be convenient to set up a |Makefile| or
shell script to automatise the process.

%%%%%%%%%%%%%%%%%%%%%%%%%%%%%%%%%%%%%%%%%%%%%%%%%%%%%%%%%%%%%%%%%%%%%%%%%%%%%%%%
\subsection{Command Line Processing}
\label{sec:commandline}

The effect of redirection files can also be achieved by invoking
the \LaTeX{} compiler with a more elaborate command line.
Most conveniently this should be done as part
of a shell script or a |Makefile|.

When using \textsf{childdoc} in the main file, the following
command lines effectively perform a redirection
(note that depending on the shell being used,
backslashes may have to be doubled: `|\|' $\to$ `|\\|'):
%
\begin{center}
|... -jobname "|\textit{target}|" |\\|"|[\textit{flags}]%
|\input{childdoc.def}\childdocforward[|\textit{main}|]{|\textit{dest}|}"|
\end{center}
%
Here \textit{target} is the name of the output file,
\textit{main} is the name of the main file
and \textit{dest} is the name of the main or child file to be processed
(all filenames without extensions).
The optional argument \textit{main} can be omitted
if \textit{main} matches \textit{dest}.
Optionally, compilation \textit{flags} can be defined via |\def| commands.
This command line makes the \TeX{} engine believe
it is compiling the file \textit{target}
whose content is specified as the latter parameter.
The provided code then forwards the processing to
\textit{main} or \textit{dest} as described in \secref{sec:forward}.

%%%%%%%%%%%%%%%%%%%%%%%%%%%%%%%%%%%%%%%%%%%%%%%%%%%%%%%%%%%%%%%%%%%%%%%%%%%%%%%%
\subsection{Include by Input}
\label{sec:input}

Including child documents by |\include| has some restrictions by design.
Most notably, the content of a child document always occupies
its own set of pages; pages cannot be shared between child documents.
Usually, this behaviour makes perfect sense
because each child document contain an essential part of the document.
However, in some situations it may be desirable to compose
a document from a collection of parts
without having mandatory page breaks between then.
For this case, the package
provides a mechanism to include parts
by |\input| which can also be processed individually.
However, by construction this mechanism
requires manual handling of the content to be output.

%%%%%%%%%%%%%%%%%%%%%%%%%%%%%%%%%%%%%%%%
\DescribeMacro{\ifchilddocmanual}
The main file should be prepared as usual, see \secref{sec:include}.
However, the document body must make a distinction
between processing of an individual part and of the main document, e.g.:
%
\begin{center}
\begin{tabular}{l}
|\ifchilddocmanual|\\
|\input{\childdocname}|\\
|\||else|\\
\textit{document body with }|\input{|\textit{part}|}|\\
|\||fi|
\end{tabular}
\end{center}
%
The conditional |\ifchilddocmanual| is true whenever
a part to be included by |\input| is being compiled,
and the name of the part is stored in |\childdocname|.

%%%%%%%%%%%%%%%%%%%%%%%%%%%%%%%%%%%%%%%%
\DescribeMacro{\childdocby}
Each part to be included by |\input| should start with:
%
\begin{center}
\begin{tabular}{l}
|\input{childdoc.def}|\\
|\childdocby{|\textit{main}|}|\\
\end{tabular}
\end{center}
%
The directive |\childdocby| is similar to |\childdocof|
described in \secref{sec:include},
but the subsequent selection of content must be done manually.
To that end, both |\ifchilddoc| and |\ifchilddocmanual|
will be true upon processing of a part,
and the name of the part is stored in |\childdocname|.
Note that |\jobname| will be set to the filename of the current part
so that each part receives an individual |.aux| file
that does not interfere with the |.aux| file(s) of the main document.
This behaviour can be altered by the alternative form
|\childdocby[*]{|\textit{main}|}| (with a non-empty optional argument)
which uses the |.aux| file of the main document
by setting |\jobname| to \textit{main}.

%%%%%%%%%%%%%%%%%%%%%%%%%%%%%%%%%%%%%%%%%%%%%%%%%%%%%%%%%%%%%%%%%%%%%%%%%%%%%%%%
\subsection{Driver Development}
\label{sec:driver}

The \textsf{childdoc} mechanism can also be use for the development
of definition files such as \LaTeX{} styles or classes.
This case differs from the above setup with multiple parts
included by |\include| in that no |\includeonly| should be invoked.
This can be achieved by starting the include file
(before |\ProvidesPackage|) with:
%
\begin{center}
\begin{tabular}{l}
|\input{childdoc.def}|\\
|\childdocforward{|\textit{main}|}|\\
\end{tabular}
\end{center}
%
or alternatively with:
%
\begin{center}
\begin{tabular}{l}
|\input{childdoc.def}|\\
|\childdocby{|\textit{main}|}|\\
\end{tabular}
\end{center}
%
Both forms have slightly different effects as described above.
The main file is prepared as usual, see \secref{sec:include}.

%%%%%%%%%%%%%%%%%%%%%%%%%%%%%%%%%%%%%%%%%%%%%%%%%%%%%%%%%%%%%%%%%%%%%%%%%%%%%%%%
\subsection{Legacy Detection}
\label{sec:detection}

The directive |\childdocmain| in the main file can detect
whether the complete document or merely a child is to be compiled
even without using the directive |\childdocof|.
This method is deprecated because it is less robust
and there is no compelling reason to use it;
it is merely provided for backward compatibility
and it may be removed in future versions.

If the detection mechanism is to be used,
it is mandatory to correctly specify
the filename of the main file as the argument of |\childdocmain|:
%
\begin{center}
\begin{tabular}{l}
|\input{childdoc.def}|\\
|\childdocmain{|\textit{main}|}|\\
\end{tabular}
\end{center}
%
If |\jobname| does not match the argument \textit{main} of |\childdocmain|,
it is assumed that |\jobname| points to the child file to be compiled.
When using |\childdocmain| with the main file specified as argument,
it suffices to start a child file
with just |\input{|\textit{main}|}|
without loading of the package and using |\childdocof|.
If instead all processing is done
with the appropriate \textsf{childdoc} directives,
the argument of \textit{main} of |\childdocmain| can be empty.

An alternative version of the command line processing described
in \secref{sec:commandline} using the detection mechanism reads:
%
\begin{center}
|... -jobname "|\textit{target}|" "|[\textit{flags}]%
[|\def\jobname{|\textit{dest}|}|]|\input{|\textit{main}|}"|
\end{center}

%%%%%%%%%%%%%%%%%%%%%%%%%%%%%%%%%%%%%%%%%%%%%%%%%%%%%%%%%%%%%%%%%%%%%%%%%%%%%%%%
\subsection{Manual Code}
\label{sec:manual}

In case one cannot be certain whether the definitions file |childdoc.def|
is installed on the target \TeX{} distribution
and one prefers not to ship it,
it is conceivable to paste a few relevant commands into the sources.

To that end, drop all statements |\input{childdoc.def}|
and perform the replacements as outlined below.
Instead of |\childdocmain{|\textit{main}|}| add the following code
to the top of the main file:
%
\begin{center}
\begin{tabular}{l}
|\||ifdefined\childdocname\endinput\||fi\newif\ifchilddoc|\\
|\edef\childdocname{\scantokens\expandafter{\jobname\noexpand}}|\\
|\def\childdocmain{|\textit{main}|}\||ifx\childdocmain\childdocname\||else|\\
|\childdoctrue\includeonly{\childdocname}\let\jobname\childdocmain\||fi|\\
\end{tabular}
\end{center}
%
Instead of |\childdocof{|\textit{main}|}| just include the main file
at the top of each child file:
%
\begin{center}
|\input{|\textit{main}|}|
\end{center}
%
A simple redirection |\childdocforward{|\textit{dest}|}| is achieved by:
%
\begin{center}
|\def\jobname{|\textit{dest}|}\input{\jobname}|
\end{center}
%
The redirection with prefix
|\childdocforwardprefix[|\textit{prefix}|]{|\textit{dest}|}|
is accomplished by:
%
\begin{center}
\begin{tabular}{l}
|{\edef\jobname{\scantokens\expandafter{\jobname\noexpand}}|\\
|\def\redirectjob |\textit{prefix}|#1~~~{\gdef\jobname{|\textit{dest}|#1}}|\\
|\expandafter\redirectjob\jobname~~~}\input{\jobname}|
\end{tabular}
\end{center}

In an alternative approach,
child documents can be compiled by a specific command line
without additional code or specific definitions:
%
\begin{center}
|... -jobname "|\textit{target}|" "|[\textit{flags}]%
|\includeonly{|\textit{dest}|}\input{|\textit{main}|}"|
\end{center}
%

%%%%%%%%%%%%%%%%%%%%%%%%%%%%%%%%%%%%%%%%%%%%%%%%%%%%%%%%%%%%%%%%%%%%%%%%%%%%%%%%
%%%%%%%%%%%%%%%%%%%%%%%%%%%%%%%%%%%%%%%%%%%%%%%%%%%%%%%%%%%%%%%%%%%%%%%%%%%%%%%%
\section{Information}

%%%%%%%%%%%%%%%%%%%%%%%%%%%%%%%%%%%%%%%%%%%%%%%%%%%%%%%%%%%%%%%%%%%%%%%%%%%%%%%%
\subsection{Copyright}

Copyright \copyright{} 2017--2018 Niklas Beisert

This work may be distributed and/or modified under the
conditions of the \LaTeX{} Project Public License, either version 1.3
of this license or (at your option) any later version.
The latest version of this license is in
  \url{http://www.latex-project.org/lppl.txt}
and version 1.3 or later is part of all distributions of \LaTeX{}
version 2005/12/01 or later.

This work has the LPPL maintenance status `maintained'.

The Current Maintainer of this work is Niklas Beisert.

This work consists of the files |README.txt|, |childdoc.ins| and |childdoc.dtx|
as well as the derived files |childdoc.def|, |cdocsamp.tex|
with |cdocsch1.tex|, |cdocsch2.tex|, |cdocspt3.tex|, |cdocspt4.tex|,
|cdocsdrf.tex|, |cdocsfn1.tex|, |cdocsfn2.tex|
as well as |childdoc.pdf|.

%%%%%%%%%%%%%%%%%%%%%%%%%%%%%%%%%%%%%%%%%%%%%%%%%%%%%%%%%%%%%%%%%%%%%%%%%%%%%%%%
\subsection{Files and Installation}

The package consists of the files:
%
\begin{center}
\begin{tabular}{ll}
    |README.txt|   & readme file \\
    |childdoc.ins| & installation file \\
    |childdoc.dtx| & source file \\
    |childdoc.def| & definition file \\
    |cdocsamp.tex| & sample main file \\
    |cdocsch1.tex| & sample include file \\
    |cdocsch2.tex| & sample include file \\
    |cdocspt3.tex| & sample part file \\
    |cdocspt4.tex| & sample part file \\
    |cdocsdrf.tex| & sample redirection file \\
    |cdocsfn1.tex| & sample redirection file \\
    |cdocsfn2.tex| & sample redirection file \\
    |childdoc.pdf| & manual
\end{tabular}
\end{center}
%
The distribution consists of the files
|README.txt|, |childdoc.ins| and |childdoc.dtx|.
%
\begin{itemize}
\item
Run (pdf)\LaTeX{} on |childdoc.dtx|
to compile the manual |childdoc.pdf| (this file).
\item
Run \LaTeX{} on |childdoc.ins| to create the definitions file |childdoc.def|
and the sample |cdocsamp.tex| with include files
|cdocsch1.tex|, |cdocsch2.tex|, |cdocspt3.tex|, |cdocspt4.tex|,
|cdocsdrf.tex|, |cdocsfn1.tex|, |cdocsfn2.tex|.
Then copy the file |childdoc.def| to an appropriate directory of your \LaTeX{}
distribution, e.g.\ \textit{texmf-root}|/tex/latex/childdoc|.
\end{itemize}

%%%%%%%%%%%%%%%%%%%%%%%%%%%%%%%%%%%%%%%%%%%%%%%%%%%%%%%%%%%%%%%%%%%%%%%%%%%%%%%%
\subsection{Related CTAN Packages}

There are several other packages which offer a similar functionality:
%
\begin{itemize}
\item
The packages
\href{http://ctan.org/pkg/docmute}{\textsf{docmute}},
\href{http://ctan.org/pkg/includex}{\textsf{includex}} and
\href{http://ctan.org/pkg/standalone}{\textsf{standalone}}
provide commands to include only the document body of
a child file thus allowing both files to be compiled individually.
\item
The packages \href{http://ctan.org/pkg/subdocs}{\textsf{subdocs}}
and \href{http://ctan.org/pkg/subfiles}{\textsf{subfiles}}
provide structures in which the main and child documents can be
encapsulated and allowing them to be compiled individually.
The inclusion mechanism is different from the conventional |\include|.
\item
The package \href{http://ctan.org/pkg/combine}{\textsf{combine}}
is an elaborate solution to combine several documents into one.
\end{itemize}
%
See also the CTAN topic \href{http://ctan.org/topic/subdocs}{\textsf{subdocs}}
for further related packages.
The present package differs from the above solutions in that
a document structure constructed with the conventional |\include| mechanism
just needs two extra commands at the top of every file
such that all constituent files can be compiled individually.

%%%%%%%%%%%%%%%%%%%%%%%%%%%%%%%%%%%%%%%%%%%%%%%%%%%%%%%%%%%%%%%%%%%%%%%%%%%%%%%%
%\subsection{Feature Suggestions}
%
%The following is a list of features which may be useful for future
%versions of this package:
%%
%\begin{itemize}
%\item
%\ldots
%\end{itemize}

%%%%%%%%%%%%%%%%%%%%%%%%%%%%%%%%%%%%%%%%%%%%%%%%%%%%%%%%%%%%%%%%%%%%%%%%%%%%%%%%
\subsection{Revision History}

%%%%%%%%%%%%%%%%%%%%%%%%%%%%%%%%%%%%%%%%
\paragraph{v2.0:} 2018/12/30

\begin{itemize}
\item
immediate forward processing
\item
added |\childdocby| mechanism
\item
manual restructured
\end{itemize}

%%%%%%%%%%%%%%%%%%%%%%%%%%%%%%%%%%%%%%%%
\paragraph{v1.6:} 2018/01/17

\begin{itemize}
\item
application for development of include files
\item
corrections to manual
\end{itemize}

%%%%%%%%%%%%%%%%%%%%%%%%%%%%%%%%%%%%%%%%
\paragraph{v1.5:} 2017/05/21

\begin{itemize}
\item
more complete structuring introduced
\item
|\childdocof| introduced
\item
|\childdoc| renamed to |\childdocmain|
\item
|\childredirect| renamed to |\childdocforward| and |\childdocforwardprefix|
and functionality expanded
\end{itemize}

%%%%%%%%%%%%%%%%%%%%%%%%%%%%%%%%%%%%%%%%
\paragraph{v1.0:} 2017/04/27

\begin{itemize}
\item
manual and install package
\item
first version published on CTAN
\end{itemize}

%%%%%%%%%%%%%%%%%%%%%%%%%%%%%%%%%%%%%%%%
\paragraph{v0.6:} 2017/04/26

\begin{itemize}
\item
redirection mechanism added
\end{itemize}

%%%%%%%%%%%%%%%%%%%%%%%%%%%%%%%%%%%%%%%%
\paragraph{v0.5:} 2017/04/26

\begin{itemize}
\item
functionality in definition file
\end{itemize}


%%%%%%%%%%%%%%%%%%%%%%%%%%%%%%%%%%%%%%%%%%%%%%%%%%%%%%%%%%%%%%%%%%%%%%%%%%%%%%%%
%%%%%%%%%%%%%%%%%%%%%%%%%%%%%%%%%%%%%%%%%%%%%%%%%%%%%%%%%%%%%%%%%%%%%%%%%%%%%%%%
%%%%%%%%%%%%%%%%%%%%%%%%%%%%%%%%%%%%%%%%%%%%%%%%%%%%%%%%%%%%%%%%%%%%%%%%%%%%%%%%
\appendix

\settowidth\MacroIndent{\rmfamily\scriptsize 000\ }

 \DocInput{childdoc.dtx}

\end{document}
%</driver>
% \fi
%
% %%%%%%%%%%%%%%%%%%%%%%%%%%%%%%%%%%%%%%%%%%%%%%%%%%%%%%%%%%%%%%%%%%%%%%%%%%%%%%
% %%%%%%%%%%%%%%%%%%%%%%%%%%%%%%%%%%%%%%%%%%%%%%%%%%%%%%%%%%%%%%%%%%%%%%%%%%%%%%
% \section{Sample}
%\iffalse
%<*samplemain>
%\fi
%
% The following presents a sample document
% with two chapters, two parts, a title page,
% a compile flag as well as three forwarding files to set the flag.
% It consists of eight |.tex| files:
% \begin{center}
% \begin{tabular}{ll}
% |cdocsamp.tex|&main file\\
% |cdocsch1.tex|&include file for chapter 1\\
% |cdocsch2.tex|&include file for chapter 2\\
% |cdocspt3.tex|&include file for part 3\\
% |cdocspt4.tex|&include file for part 4\\
% |cdocsdrf.tex|&forwarding file for main file in draft mode\\
% |cdocsfi1.tex|&forwarding file for final version of chapter 1\\
% |cdocsfi2.tex|&forwarding file for final version of chapter 2\\
% \end{tabular}
% \end{center}
% Each of the eight files can be compiled directly by the \LaTeX{} compiler.
%
% %%%%%%%%%%%%%%%%%%%%%%%%%%%%%%%%%%%%%%
% \paragraph{Main File.}
%
% The main file is called |cdocsamp.tex|.
%
% Load the \textsf{childdoc} definitions and
% declare the filename for the main document:
%    \begin{macrocode}
\input{childdoc.def}
\childdocmain{}
%    \end{macrocode}

% Optional override for |\version| flag:
%    \begin{macrocode}
%%\ifchilddoc\else\providecommand{\version}{draft}\fi
%    \end{macrocode}

% Define the default values for the |\version| flag
% (|final| for the main file and |draft| for childs):
%    \begin{macrocode}
\ifchilddoc
\providecommand{\version}{draft}
\else
\providecommand{\version}{final}
\fi
%    \end{macrocode}

% Load the standard document class:
%    \begin{macrocode}
\documentclass[12pt]{article}
%    \end{macrocode}

% Start the document body:
%    \begin{macrocode}
\begin{document}
%    \end{macrocode}

% Declare a title page.
% Print title, part of document being processed and version flag:
%    \begin{macrocode}
\addtocounter{page}{-1}
\begin{center}
{\LARGE\bfseries{}childdoc example\par}
\vspace{1cm}
\ifchilddoc
\ifchilddocmanual part\else chapter\fi:
`\childdocname' of `\childdocjob'\par
\else
main document: `\childdocjob'\par
\fi
version: \version\par
\end{center}
\newpage
%    \end{macrocode}

% Manually include selected file,
% otherwise process as usual:
%    \begin{macrocode}
\ifchilddocmanual
\section*{part `\childdocname'}
\input{\childdocname}
\else
%    \end{macrocode}

% Include the two chapters:
%    \begin{macrocode}
\include{cdocsch1}
\include{cdocsch2}
%    \end{macrocode}

% Include the two parts unless only chapters should be displayed:
%    \begin{macrocode}
\ifchilddoc\else
\section{part three}
\input{cdocspt3}
\section{part four}
\input{cdocspt4}
\fi
%    \end{macrocode}

% Process as usual until here:
%    \begin{macrocode}
\fi
%    \end{macrocode}

% End of document body:
%    \begin{macrocode}
\end{document}
%    \end{macrocode}
%\iffalse
%</samplemain>
%\fi
%
% %%%%%%%%%%%%%%%%%%%%%%%%%%%%%%%%%%%%%%
% \paragraph{Chapter Include Files.}
%
% The include files are called |cdocsch1.tex| and |cdocsch2.tex|.
%
%\iffalse
%<*samplechap1|samplechap2>
%\fi

% Optional override for |\version| flag:
%    \begin{macrocode}
%%\providecommand{\version}{final}
%    \end{macrocode}

% Include the main document:
%    \begin{macrocode}
\input{childdoc.def}
\childdocof{cdocsamp}
%    \end{macrocode}

%\iffalse
%</samplechap1|samplechap2>
%\fi
%
%\iffalse
%<*samplechap1>
%\fi
% Some text for chapter 1:
%    \begin{macrocode}
\section{one}
some text in chapter one
%    \end{macrocode}

%\iffalse
%</samplechap1>
%\fi
% Some text for chapter 2:
%\iffalse
%<*samplechap2>
%\fi
%    \begin{macrocode}
\section{two}
more text in chapter two
%    \end{macrocode}

%\iffalse
%</samplechap2>
%\fi
%
% %%%%%%%%%%%%%%%%%%%%%%%%%%%%%%%%%%%%%%
% \paragraph{Part Include Files.}
%
% The include files are called |cdocspt3.tex| and |cdocspt4.tex|.
%
%\iffalse
%<*samplepart3|samplepart4>
%\fi

% Optional override for |\version| flag:
%    \begin{macrocode}
%%\providecommand{\version}{final}
%    \end{macrocode}

% Include the main document:
%    \begin{macrocode}
\input{childdoc.def}
\childdocby{cdocsamp}
%    \end{macrocode}

%\iffalse
%</samplepart3|samplepart4>
%\fi
%
%\iffalse
%<*samplepart3>
%\fi
% Some text for part 3:
%    \begin{macrocode}
some text in part three
%    \end{macrocode}

%\iffalse
%</samplepart3>
%\fi
% Some text for part 4:
%\iffalse
%<*samplepart4>
%\fi
%    \begin{macrocode}
more text in part four
%    \end{macrocode}

%\iffalse
%</samplepart4>
%\fi
%
% %%%%%%%%%%%%%%%%%%%%%%%%%%%%%%%%%%%%%%
% \paragraph{Forwarding for a Complete Draft.}
%
% The following forwarding file |cdocsdrf.tex|
% compiles the main document in draft mode:
%\iffalse
%<*sampledraft>
%\fi
%    \begin{macrocode}
\def\version{draft}
\input{childdoc.def}
\childdocforward{cdocsamp}
%    \end{macrocode}

%\iffalse
%</sampledraft>
%\fi
%
% %%%%%%%%%%%%%%%%%%%%%%%%%%%%%%%%%%%%%%
% \paragraph{Forwarding for Final Version of the Chapters.}
%
% The following forwarding files |cdocsfn1.tex| and |cdocsfn2.tex|
% (with identical content)
% compile the final versions of the child documents
% |cdocsch1.tex| and |cdocsch2.tex|, respectively:
%\iffalse
%<*samplefinal>
%\fi
%    \begin{macrocode}
\def\version{final}
\input{childdoc.def}
\childdocforwardprefix[cdocsamp]{cdocsfn}{cdocsch}
%    \end{macrocode}

%\iffalse
%</samplefinal>
%\fi
%
% %%%%%%%%%%%%%%%%%%%%%%%%%%%%%%%%%%%%%%
% \paragraph{Command Line Processing.}
%
% The following three command lines generate the output files
% |cdocscld|, |cdocscl1| and |cdocscl2|
% which should be identical to
% |cdocsdrf|, |cdocsch1| and |cdocsfn2|, respectively:
% \begin{center}
% \begin{tabular}{l}
% |latex -jobname cdocscld \|\\
% |  "\def\version{draft}\input{childdoc.def}\childdocforward{cdocsamp}"|\\
% |latex -jobname cdocscl1 \|\\
% |  "\input{childdoc.def}\childdocforward[cdocsamp]{cdocsch1}"|\\
% |latex -jobname cdocscl2 \|\\
% |  "\def\version{final}\input{childdoc.def}\childdocforward{cdocsch2}"|
% \end{tabular}
% \end{center}
% Note that the trailing backslash on each first line
% merely continues the input to the second line
% (for convenient cut ant paste).
% Furthermore, the command |latex| can be replaced by any
% of its alternative versions such as |pdflatex|.
%
% %%%%%%%%%%%%%%%%%%%%%%%%%%%%%%%%%%%%%%%%%%%%%%%%%%%%%%%%%%%%%%%%%%%%%%%%%%%%%%
% %%%%%%%%%%%%%%%%%%%%%%%%%%%%%%%%%%%%%%%%%%%%%%%%%%%%%%%%%%%%%%%%%%%%%%%%%%%%%%
% \section{Implementation}
%\iffalse
%<*package>
%\fi
%
% This section describes the definitions file |childdoc.def|.

% The definitions cannot be loaded using |\usepackage| or |\RequirePackage|
% which has a mechanism to prevent loading a style file more than once.
% When loading the definitions by means of |\input|
% multiple instances have to be prevented manually:
%\iffalse
%This code needs to be before the `\ProvidesFile' directive
%which is defined at the beginning of this file.
%Therefore it is also placed there and commented out here.
%</package>
%<*discard>
%\fi
%    \begin{macrocode}
\ifdefined\childdocmain\endinput\fi
%    \end{macrocode}
%\iffalse
%</discard>
%<*package>
%\fi
%
% \macro{\ifchilddoc}
% \macro{\ifchilddocmanual}
% The conditional |\ifchilddoc| tells whether a
% child (true) or main (false) document is being compiled.
% The conditional |\ifchilddocmanual| tells whether
% the |\includeonly| mechanism is used (false) or
% the selection of child files must be performed manually (true).
% The definitions initialise to false:
%    \begin{macrocode}
\newif\ifchilddoc
\newif\ifchilddocmanual
%    \end{macrocode}

% \macro{\childdocname}
% \macro{\childdocjob}
% The macro |\childdocname| stores the name of the main document
% to be compiled. The macro |\childdocjob| stores the name of
% the document on which the \LaTeX{} compiler was originally invoked.
% The content of |\jobname| cannot be compared
% to filenames specified in the source due to different catcodes.
% The following code rescans |\jobname|, stores the result
% in |\childdocname| and saves a copy in |\childdocjob|:
%    \begin{macrocode}
\edef\childdocname{\scantokens\expandafter{\jobname\noexpand}}
\let\childdocjob\childdocname
%    \end{macrocode}

% \macro{\childdocdisable}
% The macro |\childdocdisable| prevents the main file
% from being processed more than once.
% At this stage, the main document command |\childdocmain|
% is assumed to be called once again where it should do nothing.
% Any subsequent call to it should prevent
% a secondary processing of the main document
% It overwrites the forwarding commands
% |\childdocof| and |\childdocforward|
% with empty macros to prevent further inclusions of the main document:
%    \begin{macrocode}
\newcommand{\childdocdisable}
{
  \renewcommand{\childdocmain}[1]{\renewcommand{\childdocmain}[1]{\endinput}}
  \renewcommand{\childdocof}[1]{}
  \renewcommand{\childdocby}[2][]{}
  \renewcommand{\childdocforward}[2][]{}
  \renewcommand{\childdocdisable}{}
}
%    \end{macrocode}

% \macro{\childdocmain}
% The macro |\childdocmain| is to be called at the top of the main file
% with nothing or the main filename (without extension) as argument.
% First, it breaks loops.
% If the argument is not empty and does not match |\childdocname|
% (which is set by the first inclusion of |childdoc.def|),
% |\ifchilddoc| is set to true, |\includeonly| is applied to the child file
% and |\jobname| is set to the main file
% (for proper handling of |.aux| files):
%    \begin{macrocode}
\newcommand{\childdocmain}[1]
{
  \childdocdisable\childdocmain{}
  \if?#1?\else
    \begingroup
      \def\childdoctmp{#1}
      \ifx\childdoctmp\childdocname
        \def\childdoctmp{}
      \else
        \def\childdoctmp
        {
          \childdoctrue
          \includeonly{\childdocname}
          \def\childdocjob{#1}
          \def\jobname{#1}
        }
      \fi
      \expandafter
    \endgroup
    \childdoctmp
  \fi
}
%    \end{macrocode}

% \macro{\childdocof}
% The command |\childdocof| redirects
% compilation to the main file |#1|.
%    \begin{macrocode}
\newcommand{\childdocof}[1]
{
  \childdocdisable
  \childdoctrue
  \includeonly{\childdocname}
  \def\jobname{#1}
  \def\childdocjob{#1}
  \input{#1}
}
%    \end{macrocode}

% \macro{\childdocby}
% The command |\childdocby| ....
%    \begin{macrocode}
\newcommand{\childdocby}[2][]
{
  \childdocdisable
  \childdoctrue
  \childdocmanualtrue
  \if?#1?\else
    \def\jobname{#2}
  \fi
  \def\childdocjob{#2}
  \input{#2}
  \endinput
}
%    \end{macrocode}

% \macro{\childdocforward}
% The command |\childdocforward| redirects
% compilation to the main file or
% (if the optional argument is given) a child file.
% Parameters are set as if the main file
% or a child file starting with |\childdocof| was compiled.
% Then compilation is handed over to the main file:
%    \begin{macrocode}
\newcommand{\childdocforward}[2][]
{
  \begingroup
    \if?#1?
      \def\childdoctmp
      {
        \def\childdocname{#2}
        \def\childdocjob{#2}
        \def\jobname{#2}
        \input{#2}
        \endinput
      }
    \else
      \def\childdoctmp
      {
        \childdocdisable
        \def\childdocname{#2}
        \childdoctrue
        \includeonly{#2}
        \def\childdocjob{#1}
        \def\jobname{#1}
        \input{#1}
        \endinput
      }
    \fi
    \expandafter
  \endgroup
  \childdoctmp
}
%    \end{macrocode}

% \macro{\childdocforwardprefix}
% The command |\childdocforwardprefix| redirects
% compilation to the main or a child file by means of a pattern.
% The prefix |#1| in the current filename is replaced by |#2|
% and the suffix of the current filename is kept
% (it is assumed that the filename does not contain the substring `|~~~|'
% which is used as a delimiter).
% Compilation is handed over to the new file by |\childdocforward|:
%    \begin{macrocode}
\newcommand{\childdocforwardprefix}[3][]
{
  \begingroup
    \def\childdocextract #2##1~~~{\def\childdoctmp{\childdocforward[#1]{#3##1}}}
    \expandafter\childdocextract\childdocname~~~
    \expandafter
  \endgroup
  \childdoctmp
}
%    \end{macrocode}

% \macro{\childdoc}
% The deprecated macro |\childdoc| is a legacy version of |\childdocmain|:
%    \begin{macrocode}
\newcommand{\childdoc}{\childdocmain}
%    \end{macrocode}

% \macro{\childdocredirect}
% The deprecated macro |\childdocredirect| is a legacy version
% of |\childdocforward| and |\childdocforwardprefix|:
%    \begin{macrocode}
\newcommand{\childdocredirect}[2][]
{
  \begingroup
    \if?#1?
      \def\childdoctmp{\childdocforward{#2}}
    \else
      \def\childdoctmp{\childdocforwardprefix{#1}{#2}}
    \fi
    \expandafter
  \endgroup
  \childdoctmp
}
%    \end{macrocode}

%\iffalse
%</package>
%\fi
%
\endinput
|\\
|\childdocby{|\textit{main}|}|\\
\end{tabular}
\end{center}
%
Both forms have slightly different effects as described above.
The main file is prepared as usual, see \secref{sec:include}.

%%%%%%%%%%%%%%%%%%%%%%%%%%%%%%%%%%%%%%%%%%%%%%%%%%%%%%%%%%%%%%%%%%%%%%%%%%%%%%%%
\subsection{Legacy Detection}
\label{sec:detection}

The directive |\childdocmain| in the main file can detect
whether the complete document or merely a child is to be compiled
even without using the directive |\childdocof|.
This method is deprecated because it is less robust
and there is no compelling reason to use it;
it is merely provided for backward compatibility
and it may be removed in future versions.

If the detection mechanism is to be used,
it is mandatory to correctly specify
the filename of the main file as the argument of |\childdocmain|:
%
\begin{center}
\begin{tabular}{l}
|% \iffalse
%
% childdoc.dtx Copyright (C) 2017-2018 Niklas Beisert
%
% This work may be distributed and/or modified under the
% conditions of the LaTeX Project Public License, either version 1.3
% of this license or (at your option) any later version.
% The latest version of this license is in
%   http://www.latex-project.org/lppl.txt
% and version 1.3 or later is part of all distributions of LaTeX
% version 2005/12/01 or later.
%
% This work has the LPPL maintenance status `maintained'.
%
% The Current Maintainer of this work is Niklas Beisert.
%
% This work consists of the files childdoc.dtx and childdoc.ins
% and the derived files childdoc.def and cdocsamp.tex with
% cdocsch1.tex, cdocsch2.tex, cdocsdrf.tex, cdocsfn1.tex, cdocsfn2.tex.
%
%<package>\ifdefined\childdocmain\endinput\fi
%<package>\ProvidesFile{childdoc.def}[2018/12/30 v2.0 child document driver]
%<samplemain>\ProvidesFile{cdocsamp.tex}[2018/12/30 v2.0 sample for childdoc]
%<*driver>
%\ProvidesFile{childdoc.drv}[2018/12/30 v2.0 childdoc reference manual file]
\PassOptionsToClass{10pt,a4paper}{article}
\documentclass{ltxdoc}

\usepackage[margin=35mm]{geometry}
\usepackage{hyperref}
\usepackage{hyperxmp}
\usepackage[usenames]{color}

\hypersetup{colorlinks=true}
\hypersetup{pdfstartview=FitH}
\hypersetup{pdfpagemode=UseNone}
\hypersetup{pdfsource={}}
\hypersetup{pdflang={en-UK}}
\hypersetup{pdfcopyright={Copyright 2017-2018 Niklas Beisert.
  This work may be distributed and/or modified under the
  conditions of the LaTeX Project Public License, either version 1.3
  of this license or (at your option) any later version.}}
\hypersetup{pdflicenseurl={http://www.latex-project.org/lppl.txt}}
\hypersetup{pdfcontactaddress={ETH Zurich, ITP, HIT K,
  Wolfgang-Pauli-Strasse 27}}
\hypersetup{pdfcontactpostcode={8093}}
\hypersetup{pdfcontactcity={Zurich}}
\hypersetup{pdfcontactcountry={Switzerland}}
\hypersetup{pdfcontactemail={nbeisert@itp.phys.ethz.ch}}
\hypersetup{pdfcontacturl={http://people.phys.ethz.ch/\xmptilde nbeisert/}}

\newcommand{\secref}[1]{\hyperref[#1]{section \ref*{#1}}}

\parskip1ex
\parindent0pt
\let\olditemize\itemize
\def\itemize{\olditemize\parskip0pt}

\begin{document}

\title{The \textsf{childdoc} Package}
\hypersetup{pdftitle={The childdoc Package}}
\author{Niklas Beisert\\[2ex]
  Institut f\"ur Theoretische Physik\\
  Eidgen\"ossische Technische Hochschule Z\"urich\\
  Wolfgang-Pauli-Strasse 27, 8093 Z\"urich, Switzerland\\[1ex]
  \href{mailto:nbeisert@itp.phys.ethz.ch}
  {\texttt{nbeisert@itp.phys.ethz.ch}}}
\hypersetup{pdfauthor={Niklas Beisert}}
\hypersetup{pdfsubject={Manual for the LaTeX2e Package childdoc}}
\date{30 December 2018, \textsf{v2.0}}
\maketitle

\begin{abstract}\noindent
\textsf{childdoc} is a \LaTeXe{} package
that enables the direct compilation
of document sections included by |\include|
to individual files.
\end{abstract}

\begingroup
\parskip0ex
\tableofcontents
\endgroup

%%%%%%%%%%%%%%%%%%%%%%%%%%%%%%%%%%%%%%%%%%%%%%%%%%%%%%%%%%%%%%%%%%%%%%%%%%%%%%%%
%%%%%%%%%%%%%%%%%%%%%%%%%%%%%%%%%%%%%%%%%%%%%%%%%%%%%%%%%%%%%%%%%%%%%%%%%%%%%%%%
\section{Introduction}

\LaTeX{} provides a mechanism to structure a large document (such as a book)
into a main file and several child files (containing the chapters)
using the |\include| command.
This mechanism is beneficial for documents
which span hundreds of pages in order to
make the source file(s) more manageable.
Moreover, compilation can be restricted to
selected child files by means of the |\includeonly| command.
The latter feature can be used to reduce the compilation time while editing
(this was significantly more useful in the earlier days of \LaTeX{})
or to generate a smaller document which is easier to navigate.
Another application of |\includeonly| is to generate
documents consisting of selected parts of the complete document.

However, there are a few drawbacks of the plain |\include| mechanism:
\begin{itemize}
\item
The child files cannot be compiled on their own,
they can only be compiled via the main file.
A naive editing environment
(such as a text editor with an option
to have the current file processed by \LaTeX)
may require one to switch to the main file before compiling;
attempting to compile the child file produces errors.
\item
The main file must be modified (each time)
to adjust the |\includeonly| command
to the present needs. This easily leaves the main file in a messy state.
\item
The generated document will always carry the filename
of the main document. This is inconvenient if
several child files are to be compiled and
to be kept for distribution.
\end{itemize}

The present package provides a simple interface
to make child files individually compilable by \LaTeX{}.
Compiling a child file then has the same effect as compiling
the main file with an |\includeonly| command
to select the appropriate child.
Moreover the generated document will carry the name of the child
rather than the main file.
This resolves all three above issues.

This feature is meant to make the editing of books,
thesis documents and lecture notes somewhat more convenient.
However, the package can also be used efficiently for
composing a series of documents (such as exercise sheets)
which are typically distributed individually.
It then assists the author in generating the individual documents
(potentially in different versions)
as well as a document containing the collected series.
Another application is in developing style files
or other kinds of included material
where compilation of the style file could redirect
to a sample or test file.

%%%%%%%%%%%%%%%%%%%%%%%%%%%%%%%%%%%%%%%%%%%%%%%%%%%%%%%%%%%%%%%%%%%%%%%%%%%%%%%%
%%%%%%%%%%%%%%%%%%%%%%%%%%%%%%%%%%%%%%%%%%%%%%%%%%%%%%%%%%%%%%%%%%%%%%%%%%%%%%%%
\section{Usage}

First of all, the package \textsf{childdoc} is \emph{not} a standard
\LaTeXe{} |.sty| style file! Therefore it needs to be invoked in
a non-standard way.

%%%%%%%%%%%%%%%%%%%%%%%%%%%%%%%%%%%%%%%%%%%%%%%%%%%%%%%%%%%%%%%%%%%%%%%%%%%%%%%%
\subsection{Included Files}
\label{sec:include}

%%%%%%%%%%%%%%%%%%%%%%%%%%%%%%%%%%%%%%%%
\DescribeMacro{\childdocmain}
To use the package, add the commands
\begin{center}
\begin{tabular}{l}
|\input{childdoc.def}|\\
|\childdocmain{}|\\
\end{tabular}
\end{center}
at the very top of the main \LaTeX{} file,
in particular \emph{before} the |\documentclass| statement!
The argument of |\childdocmain| should be left empty
(but it must be present).

%%%%%%%%%%%%%%%%%%%%%%%%%%%%%%%%%%%%%%%%
\DescribeMacro{\childdocof}
Furthermore, add the commands
\begin{center}
\begin{tabular}{l}
|\input{childdoc.def}|\\
|\childdocof{|\textit{main}|}|\\
\end{tabular}
\end{center}
at the top of every child file \textit{child}
which is included by |\include{|\textit{child}|}|
from within the main file
(or at least for those files to be compiled individually).
The argument \textit{main} must be the filename of the main file.

There are a couple of
considerations in setting up the main and child documents:

%%%%%%%%%%%%%%%%%%%%%%%%%%%%%%%%%%%%%%%%
\paragraph{Restrictions.}

Please note the following restrictions:
\begin{itemize}
\item
|\childdocmain| must be called with one argument \textit{main}
to ensure compatibility with earlier version of the package.
It must either be empty (|\childdocmain{}|)
or precisely match the filename of the main file in which it is specified.
See \secref{sec:detection} for further information.
\item
The filename \textit{main} must be specified without the |.tex| extension.
\item
The filename \textit{main} is case sensitive
(even in case-insensitive file systems)
due to internal string comparison.
\item
The argument \textit{main} should be fully expanded, it cannot be a macro.
\item
Subdirectories and special characters should be avoided in filenames.
\item
The command |\childdocmain{|\textit{main}|}| must be followed by a whitespace.
It should not be followed immediately by another command
or by a comment mark `|%|'.
This is because the \TeX{} parser reads the token immediately following
the argument of |\childdocmain| and puts it
at the beginning of every child section;
however, a white\-space is ignored.
\end{itemize}

%%%%%%%%%%%%%%%%%%%%%%%%%%%%%%%%%%%%%%%%
\paragraph{Content of Main File.}

It is advisable to place all content in the child files included by |\include|.
Any output contained in the main file will appear in all child documents
unless suppressed manually;
it cannot be suppressed automatically by the |\includeonly| directive
and thus should normally be avoided.
A method to include some content in the main file
by means of conditional processing is described in \secref{sec:conditional}.

%%%%%%%%%%%%%%%%%%%%%%%%%%%%%%%%%%%%%%%%
\paragraph{Page Numbering.}

When only a part of the document is compiled,
the appropriate numbering of pages
(as well as other status parameters)
is determined from the |.aux| files.
The latter contain information from previous passes.
However this information needs to propagate through
all intermediate child documents.
Therefore the page numbering in child documents may well
be inconsistent until the complete document is compiled at least once.

A useful (if unconventional) way to always ensure a consistent
page numbering is to restart the numbering in each child document
and denote the pages by `\textit{child}|.|\textit{page}'
where \textit{child} represents the chapter/section number of the child file.
This can be achieved by the command
|\numberwithin{page}{|\textit{child}|}|
of the \textsf{amsmath} package
where \textit{child} can be |chapter| or |section|
depending on the chosen structuring.
Alternatively, one can modify the macro |\thepage| appropriately
and reset the counter |page| at the start of each child file.

%%%%%%%%%%%%%%%%%%%%%%%%%%%%%%%%%%%%%%%%%%%%%%%%%%%%%%%%%%%%%%%%%%%%%%%%%%%%%%%%
\subsection{Conditional Processing}
\label{sec:conditional}

The package provides a mechanism to compile different versions
of a document. To customise the versions further some conditional processing
can come in handy to distinguish which version is being compiled.
The package provides two macros to describe the compilation context:

%%%%%%%%%%%%%%%%%%%%%%%%%%%%%%%%%%%%%%%%
\DescribeMacro{\ifchilddoc}
The conditional |\ifchilddoc| distinguishes between the compilation of
child documents and the main document:
%
\begin{center}
|\ifchilddoc |\textit{child-code}| |[|\||else |\textit{main-code}]| \||fi|
\end{center}

%%%%%%%%%%%%%%%%%%%%%%%%%%%%%%%%%%%%%%%%
\DescribeMacro{\childdocname}
\DescribeMacro{\childdocjob}
The macro |\childdocname| contains the filename (without extension)
of the main or child file being processed.
Note that |\childdocjob| will always contain the name of the main file.

%%%%%%%%%%%%%%%%%%%%%%%%%%%%%%%%%%%%%%%%
\paragraph{Title Page.}

Conditional processing can be used to include a title or banner page
in the main document when proper precautions are taken.
Importantly, the code in the main file should ensure that the page counter
(as well as other status parameters which are stored in the |.aux| files)
takes the same value after the conditional processing.
Otherwise the page numbers may take divergent values
depending on which part is compiled.

For example, a title page could be declared by:
%
\begin{center}
\begin{tabular}{l}
|\ifchilddoc\||else|\\
|\addtocounter{page}{-1}|\\
\textit{code for title page}\\
|\newpage|\\
|\||fi|
\end{tabular}
\end{center}
%
A banner page for the child documents can be generated by:
%
\begin{center}
\begin{tabular}{l}
|\ifchilddoc|\\
|\addtocounter{page}{-1}|\\
\textit{code for banner page}\\
|\newpage|\\
|\||fi|
\end{tabular}
\end{center}
%
Here one could write a message such as:
\begin{center}
|This is the part \childdocname{} of \childdocjob{}.|
\end{center}

%%%%%%%%%%%%%%%%%%%%%%%%%%%%%%%%%%%%%%%%%%%%%%%%%%%%%%%%%%%%%%%%%%%%%%%%%%%%%%%%
\subsection{Flags}
\label{sec:flags}

The package makes it easy to generate different versions
of the main or child documents.
To this end compilation flags can be defined
and assigned different default values.
They will be particularly useful in conjunction
with the forwarding mechanism described in \secref{sec:forward}.

For example, it may be useful to have a flag |\version|
which can be set to |draft| or |final|.
The document source will contain some conditional code
depending on the value of |\version|.
Suppose further, the flag should default to |final| for the main file
and to |draft| for child files
which is a natural assignment for editing the document.
This is achieved by placing the following code
in the preamble of the main document
(below the |\childdocmain| directive):
%
\begin{center}
\begin{tabular}{l}
|\ifchilddoc|\\
|\providecommand{\version}{draft}|\\
|\||else|\\
|\providecommand{\version}{final}|\\
|\||fi|
\end{tabular}
\end{center}
%
The definition by |\providecommand| makes sure
that previous definitions are not overwritten.
Further statements |\providecommand{\version}{...}|
can thus be added before the above code to override it.

For the main file, one might add a line
(between |\childdocmain| and the above block)
%
\begin{center}
|%\ifchilddoc\||else\providecommand{\version}{draft}\||fi|
\end{center}
%
which can be uncommented to produce a draft version.
Likewise one can add a line to the very top of a child file
(above the |\childdocof{|\textit{main}|}| directive)
%
\begin{center}
|%\providecommand{\version}{final}|
\end{center}
%
which can be uncommented to produce the final version of this child document.

%%%%%%%%%%%%%%%%%%%%%%%%%%%%%%%%%%%%%%%%%%%%%%%%%%%%%%%%%%%%%%%%%%%%%%%%%%%%%%%%
\subsection{Forwarding}
\label{sec:forward}

Different versions of the main or child documents
using compilation flags as described in \secref{sec:flags}
can be (permanently) stored in different files
for convenient compilation, viewing and distribution.
To this end, the package defines a command
to pass on compilation to a different file:

%%%%%%%%%%%%%%%%%%%%%%%%%%%%%%%%%%%%%%%%
\DescribeMacro{\childdocforward}
The command |\childdocforward| redirects processing to
another source file:
%
\begin{center}
\begin{tabular}{l}
|\input{childdoc.def}|\\
|\childdocforward[|\textit{main}|]{|\textit{dest}|}|\\
\end{tabular}
\end{center}
%
The argument \textit{dest} is the destination file
(without extension).
It should be the main file or one of the child files.
Note that further \textsf{childdoc} directives
such as |\childdocof| and |\childdocforward|
in the indicated file will be processed in this form.
The optional argument \textit{main}
passes on directly to the main file \textit{main}
while pretending to compile the child \textit{dest}.
This form behaves as if \textit{dest}
issues |\childdocof{|\textit{main}|}| right away,
and no further \textsf{childdoc} directives will be processed.

%%%%%%%%%%%%%%%%%%%%%%%%%%%%%%%%%%%%%%%%
\DescribeMacro{\...prefix}
In the alternative form |\childdocforwardprefix|,
%
\begin{center}
\begin{tabular}{l}
|\input{childdoc.def}|\\
|\childdocforwardprefix[|\textit{main}|]{|\textit{prefix}|}{|\textit{dest}|}|
\end{tabular}
\end{center}
%
the destination file is determined by a pattern
depending on the current file:
To make this work, the current file must be called
`{\textit{prefix}\hspace{0.2em}\textit{suffix}}'
with \textit{prefix} matching precisely the argument.
Processing is then passed on to the file
`{\textit{dest}\hspace{0.2em}\textit{suffix}}'.
Surely, the same effect is achieved by
directly specifying the
argument `{\textit{dest}\hspace{0.2em}\textit{suffix}}'
in the first form.
However, that requires to set up a different file
for each child. With the alternative form of the command
all these files can have exactly the same content
which simplifies setting them up and maintaining them.

For example, the following file |draft.tex|
with a compilation flag |\version| as described in \secref{sec:flags}
compiles the main document as a draft:
%
\begin{center}
\begin{tabular}{l}
|\def\version{draft}|\\
|\input{childdoc.def}|\\
|\childdocforward{|\textit{main}|}|
\end{tabular}
\end{center}
%
Likewise, the following files |final|\textit{nn}|.tex|
compile the final version of the child document
|child|\textit{nn}|.tex|:
%
\begin{center}
\begin{tabular}{l}
|\def\version{final}|\\
|\input{childdoc.def}|\\
|\childdocforwardprefix{final}{child}|
\end{tabular}
\end{center}
%

Note that when several versions of a main file and/or of each child file
are to be generated, it may be convenient to set up a |Makefile| or
shell script to automatise the process.

%%%%%%%%%%%%%%%%%%%%%%%%%%%%%%%%%%%%%%%%%%%%%%%%%%%%%%%%%%%%%%%%%%%%%%%%%%%%%%%%
\subsection{Command Line Processing}
\label{sec:commandline}

The effect of redirection files can also be achieved by invoking
the \LaTeX{} compiler with a more elaborate command line.
Most conveniently this should be done as part
of a shell script or a |Makefile|.

When using \textsf{childdoc} in the main file, the following
command lines effectively perform a redirection
(note that depending on the shell being used,
backslashes may have to be doubled: `|\|' $\to$ `|\\|'):
%
\begin{center}
|... -jobname "|\textit{target}|" |\\|"|[\textit{flags}]%
|\input{childdoc.def}\childdocforward[|\textit{main}|]{|\textit{dest}|}"|
\end{center}
%
Here \textit{target} is the name of the output file,
\textit{main} is the name of the main file
and \textit{dest} is the name of the main or child file to be processed
(all filenames without extensions).
The optional argument \textit{main} can be omitted
if \textit{main} matches \textit{dest}.
Optionally, compilation \textit{flags} can be defined via |\def| commands.
This command line makes the \TeX{} engine believe
it is compiling the file \textit{target}
whose content is specified as the latter parameter.
The provided code then forwards the processing to
\textit{main} or \textit{dest} as described in \secref{sec:forward}.

%%%%%%%%%%%%%%%%%%%%%%%%%%%%%%%%%%%%%%%%%%%%%%%%%%%%%%%%%%%%%%%%%%%%%%%%%%%%%%%%
\subsection{Include by Input}
\label{sec:input}

Including child documents by |\include| has some restrictions by design.
Most notably, the content of a child document always occupies
its own set of pages; pages cannot be shared between child documents.
Usually, this behaviour makes perfect sense
because each child document contain an essential part of the document.
However, in some situations it may be desirable to compose
a document from a collection of parts
without having mandatory page breaks between then.
For this case, the package
provides a mechanism to include parts
by |\input| which can also be processed individually.
However, by construction this mechanism
requires manual handling of the content to be output.

%%%%%%%%%%%%%%%%%%%%%%%%%%%%%%%%%%%%%%%%
\DescribeMacro{\ifchilddocmanual}
The main file should be prepared as usual, see \secref{sec:include}.
However, the document body must make a distinction
between processing of an individual part and of the main document, e.g.:
%
\begin{center}
\begin{tabular}{l}
|\ifchilddocmanual|\\
|\input{\childdocname}|\\
|\||else|\\
\textit{document body with }|\input{|\textit{part}|}|\\
|\||fi|
\end{tabular}
\end{center}
%
The conditional |\ifchilddocmanual| is true whenever
a part to be included by |\input| is being compiled,
and the name of the part is stored in |\childdocname|.

%%%%%%%%%%%%%%%%%%%%%%%%%%%%%%%%%%%%%%%%
\DescribeMacro{\childdocby}
Each part to be included by |\input| should start with:
%
\begin{center}
\begin{tabular}{l}
|\input{childdoc.def}|\\
|\childdocby{|\textit{main}|}|\\
\end{tabular}
\end{center}
%
The directive |\childdocby| is similar to |\childdocof|
described in \secref{sec:include},
but the subsequent selection of content must be done manually.
To that end, both |\ifchilddoc| and |\ifchilddocmanual|
will be true upon processing of a part,
and the name of the part is stored in |\childdocname|.
Note that |\jobname| will be set to the filename of the current part
so that each part receives an individual |.aux| file
that does not interfere with the |.aux| file(s) of the main document.
This behaviour can be altered by the alternative form
|\childdocby[*]{|\textit{main}|}| (with a non-empty optional argument)
which uses the |.aux| file of the main document
by setting |\jobname| to \textit{main}.

%%%%%%%%%%%%%%%%%%%%%%%%%%%%%%%%%%%%%%%%%%%%%%%%%%%%%%%%%%%%%%%%%%%%%%%%%%%%%%%%
\subsection{Driver Development}
\label{sec:driver}

The \textsf{childdoc} mechanism can also be use for the development
of definition files such as \LaTeX{} styles or classes.
This case differs from the above setup with multiple parts
included by |\include| in that no |\includeonly| should be invoked.
This can be achieved by starting the include file
(before |\ProvidesPackage|) with:
%
\begin{center}
\begin{tabular}{l}
|\input{childdoc.def}|\\
|\childdocforward{|\textit{main}|}|\\
\end{tabular}
\end{center}
%
or alternatively with:
%
\begin{center}
\begin{tabular}{l}
|\input{childdoc.def}|\\
|\childdocby{|\textit{main}|}|\\
\end{tabular}
\end{center}
%
Both forms have slightly different effects as described above.
The main file is prepared as usual, see \secref{sec:include}.

%%%%%%%%%%%%%%%%%%%%%%%%%%%%%%%%%%%%%%%%%%%%%%%%%%%%%%%%%%%%%%%%%%%%%%%%%%%%%%%%
\subsection{Legacy Detection}
\label{sec:detection}

The directive |\childdocmain| in the main file can detect
whether the complete document or merely a child is to be compiled
even without using the directive |\childdocof|.
This method is deprecated because it is less robust
and there is no compelling reason to use it;
it is merely provided for backward compatibility
and it may be removed in future versions.

If the detection mechanism is to be used,
it is mandatory to correctly specify
the filename of the main file as the argument of |\childdocmain|:
%
\begin{center}
\begin{tabular}{l}
|\input{childdoc.def}|\\
|\childdocmain{|\textit{main}|}|\\
\end{tabular}
\end{center}
%
If |\jobname| does not match the argument \textit{main} of |\childdocmain|,
it is assumed that |\jobname| points to the child file to be compiled.
When using |\childdocmain| with the main file specified as argument,
it suffices to start a child file
with just |\input{|\textit{main}|}|
without loading of the package and using |\childdocof|.
If instead all processing is done
with the appropriate \textsf{childdoc} directives,
the argument of \textit{main} of |\childdocmain| can be empty.

An alternative version of the command line processing described
in \secref{sec:commandline} using the detection mechanism reads:
%
\begin{center}
|... -jobname "|\textit{target}|" "|[\textit{flags}]%
[|\def\jobname{|\textit{dest}|}|]|\input{|\textit{main}|}"|
\end{center}

%%%%%%%%%%%%%%%%%%%%%%%%%%%%%%%%%%%%%%%%%%%%%%%%%%%%%%%%%%%%%%%%%%%%%%%%%%%%%%%%
\subsection{Manual Code}
\label{sec:manual}

In case one cannot be certain whether the definitions file |childdoc.def|
is installed on the target \TeX{} distribution
and one prefers not to ship it,
it is conceivable to paste a few relevant commands into the sources.

To that end, drop all statements |\input{childdoc.def}|
and perform the replacements as outlined below.
Instead of |\childdocmain{|\textit{main}|}| add the following code
to the top of the main file:
%
\begin{center}
\begin{tabular}{l}
|\||ifdefined\childdocname\endinput\||fi\newif\ifchilddoc|\\
|\edef\childdocname{\scantokens\expandafter{\jobname\noexpand}}|\\
|\def\childdocmain{|\textit{main}|}\||ifx\childdocmain\childdocname\||else|\\
|\childdoctrue\includeonly{\childdocname}\let\jobname\childdocmain\||fi|\\
\end{tabular}
\end{center}
%
Instead of |\childdocof{|\textit{main}|}| just include the main file
at the top of each child file:
%
\begin{center}
|\input{|\textit{main}|}|
\end{center}
%
A simple redirection |\childdocforward{|\textit{dest}|}| is achieved by:
%
\begin{center}
|\def\jobname{|\textit{dest}|}\input{\jobname}|
\end{center}
%
The redirection with prefix
|\childdocforwardprefix[|\textit{prefix}|]{|\textit{dest}|}|
is accomplished by:
%
\begin{center}
\begin{tabular}{l}
|{\edef\jobname{\scantokens\expandafter{\jobname\noexpand}}|\\
|\def\redirectjob |\textit{prefix}|#1~~~{\gdef\jobname{|\textit{dest}|#1}}|\\
|\expandafter\redirectjob\jobname~~~}\input{\jobname}|
\end{tabular}
\end{center}

In an alternative approach,
child documents can be compiled by a specific command line
without additional code or specific definitions:
%
\begin{center}
|... -jobname "|\textit{target}|" "|[\textit{flags}]%
|\includeonly{|\textit{dest}|}\input{|\textit{main}|}"|
\end{center}
%

%%%%%%%%%%%%%%%%%%%%%%%%%%%%%%%%%%%%%%%%%%%%%%%%%%%%%%%%%%%%%%%%%%%%%%%%%%%%%%%%
%%%%%%%%%%%%%%%%%%%%%%%%%%%%%%%%%%%%%%%%%%%%%%%%%%%%%%%%%%%%%%%%%%%%%%%%%%%%%%%%
\section{Information}

%%%%%%%%%%%%%%%%%%%%%%%%%%%%%%%%%%%%%%%%%%%%%%%%%%%%%%%%%%%%%%%%%%%%%%%%%%%%%%%%
\subsection{Copyright}

Copyright \copyright{} 2017--2018 Niklas Beisert

This work may be distributed and/or modified under the
conditions of the \LaTeX{} Project Public License, either version 1.3
of this license or (at your option) any later version.
The latest version of this license is in
  \url{http://www.latex-project.org/lppl.txt}
and version 1.3 or later is part of all distributions of \LaTeX{}
version 2005/12/01 or later.

This work has the LPPL maintenance status `maintained'.

The Current Maintainer of this work is Niklas Beisert.

This work consists of the files |README.txt|, |childdoc.ins| and |childdoc.dtx|
as well as the derived files |childdoc.def|, |cdocsamp.tex|
with |cdocsch1.tex|, |cdocsch2.tex|, |cdocspt3.tex|, |cdocspt4.tex|,
|cdocsdrf.tex|, |cdocsfn1.tex|, |cdocsfn2.tex|
as well as |childdoc.pdf|.

%%%%%%%%%%%%%%%%%%%%%%%%%%%%%%%%%%%%%%%%%%%%%%%%%%%%%%%%%%%%%%%%%%%%%%%%%%%%%%%%
\subsection{Files and Installation}

The package consists of the files:
%
\begin{center}
\begin{tabular}{ll}
    |README.txt|   & readme file \\
    |childdoc.ins| & installation file \\
    |childdoc.dtx| & source file \\
    |childdoc.def| & definition file \\
    |cdocsamp.tex| & sample main file \\
    |cdocsch1.tex| & sample include file \\
    |cdocsch2.tex| & sample include file \\
    |cdocspt3.tex| & sample part file \\
    |cdocspt4.tex| & sample part file \\
    |cdocsdrf.tex| & sample redirection file \\
    |cdocsfn1.tex| & sample redirection file \\
    |cdocsfn2.tex| & sample redirection file \\
    |childdoc.pdf| & manual
\end{tabular}
\end{center}
%
The distribution consists of the files
|README.txt|, |childdoc.ins| and |childdoc.dtx|.
%
\begin{itemize}
\item
Run (pdf)\LaTeX{} on |childdoc.dtx|
to compile the manual |childdoc.pdf| (this file).
\item
Run \LaTeX{} on |childdoc.ins| to create the definitions file |childdoc.def|
and the sample |cdocsamp.tex| with include files
|cdocsch1.tex|, |cdocsch2.tex|, |cdocspt3.tex|, |cdocspt4.tex|,
|cdocsdrf.tex|, |cdocsfn1.tex|, |cdocsfn2.tex|.
Then copy the file |childdoc.def| to an appropriate directory of your \LaTeX{}
distribution, e.g.\ \textit{texmf-root}|/tex/latex/childdoc|.
\end{itemize}

%%%%%%%%%%%%%%%%%%%%%%%%%%%%%%%%%%%%%%%%%%%%%%%%%%%%%%%%%%%%%%%%%%%%%%%%%%%%%%%%
\subsection{Related CTAN Packages}

There are several other packages which offer a similar functionality:
%
\begin{itemize}
\item
The packages
\href{http://ctan.org/pkg/docmute}{\textsf{docmute}},
\href{http://ctan.org/pkg/includex}{\textsf{includex}} and
\href{http://ctan.org/pkg/standalone}{\textsf{standalone}}
provide commands to include only the document body of
a child file thus allowing both files to be compiled individually.
\item
The packages \href{http://ctan.org/pkg/subdocs}{\textsf{subdocs}}
and \href{http://ctan.org/pkg/subfiles}{\textsf{subfiles}}
provide structures in which the main and child documents can be
encapsulated and allowing them to be compiled individually.
The inclusion mechanism is different from the conventional |\include|.
\item
The package \href{http://ctan.org/pkg/combine}{\textsf{combine}}
is an elaborate solution to combine several documents into one.
\end{itemize}
%
See also the CTAN topic \href{http://ctan.org/topic/subdocs}{\textsf{subdocs}}
for further related packages.
The present package differs from the above solutions in that
a document structure constructed with the conventional |\include| mechanism
just needs two extra commands at the top of every file
such that all constituent files can be compiled individually.

%%%%%%%%%%%%%%%%%%%%%%%%%%%%%%%%%%%%%%%%%%%%%%%%%%%%%%%%%%%%%%%%%%%%%%%%%%%%%%%%
%\subsection{Feature Suggestions}
%
%The following is a list of features which may be useful for future
%versions of this package:
%%
%\begin{itemize}
%\item
%\ldots
%\end{itemize}

%%%%%%%%%%%%%%%%%%%%%%%%%%%%%%%%%%%%%%%%%%%%%%%%%%%%%%%%%%%%%%%%%%%%%%%%%%%%%%%%
\subsection{Revision History}

%%%%%%%%%%%%%%%%%%%%%%%%%%%%%%%%%%%%%%%%
\paragraph{v2.0:} 2018/12/30

\begin{itemize}
\item
immediate forward processing
\item
added |\childdocby| mechanism
\item
manual restructured
\end{itemize}

%%%%%%%%%%%%%%%%%%%%%%%%%%%%%%%%%%%%%%%%
\paragraph{v1.6:} 2018/01/17

\begin{itemize}
\item
application for development of include files
\item
corrections to manual
\end{itemize}

%%%%%%%%%%%%%%%%%%%%%%%%%%%%%%%%%%%%%%%%
\paragraph{v1.5:} 2017/05/21

\begin{itemize}
\item
more complete structuring introduced
\item
|\childdocof| introduced
\item
|\childdoc| renamed to |\childdocmain|
\item
|\childredirect| renamed to |\childdocforward| and |\childdocforwardprefix|
and functionality expanded
\end{itemize}

%%%%%%%%%%%%%%%%%%%%%%%%%%%%%%%%%%%%%%%%
\paragraph{v1.0:} 2017/04/27

\begin{itemize}
\item
manual and install package
\item
first version published on CTAN
\end{itemize}

%%%%%%%%%%%%%%%%%%%%%%%%%%%%%%%%%%%%%%%%
\paragraph{v0.6:} 2017/04/26

\begin{itemize}
\item
redirection mechanism added
\end{itemize}

%%%%%%%%%%%%%%%%%%%%%%%%%%%%%%%%%%%%%%%%
\paragraph{v0.5:} 2017/04/26

\begin{itemize}
\item
functionality in definition file
\end{itemize}


%%%%%%%%%%%%%%%%%%%%%%%%%%%%%%%%%%%%%%%%%%%%%%%%%%%%%%%%%%%%%%%%%%%%%%%%%%%%%%%%
%%%%%%%%%%%%%%%%%%%%%%%%%%%%%%%%%%%%%%%%%%%%%%%%%%%%%%%%%%%%%%%%%%%%%%%%%%%%%%%%
%%%%%%%%%%%%%%%%%%%%%%%%%%%%%%%%%%%%%%%%%%%%%%%%%%%%%%%%%%%%%%%%%%%%%%%%%%%%%%%%
\appendix

\settowidth\MacroIndent{\rmfamily\scriptsize 000\ }

 \DocInput{childdoc.dtx}

\end{document}
%</driver>
% \fi
%
% %%%%%%%%%%%%%%%%%%%%%%%%%%%%%%%%%%%%%%%%%%%%%%%%%%%%%%%%%%%%%%%%%%%%%%%%%%%%%%
% %%%%%%%%%%%%%%%%%%%%%%%%%%%%%%%%%%%%%%%%%%%%%%%%%%%%%%%%%%%%%%%%%%%%%%%%%%%%%%
% \section{Sample}
%\iffalse
%<*samplemain>
%\fi
%
% The following presents a sample document
% with two chapters, two parts, a title page,
% a compile flag as well as three forwarding files to set the flag.
% It consists of eight |.tex| files:
% \begin{center}
% \begin{tabular}{ll}
% |cdocsamp.tex|&main file\\
% |cdocsch1.tex|&include file for chapter 1\\
% |cdocsch2.tex|&include file for chapter 2\\
% |cdocspt3.tex|&include file for part 3\\
% |cdocspt4.tex|&include file for part 4\\
% |cdocsdrf.tex|&forwarding file for main file in draft mode\\
% |cdocsfi1.tex|&forwarding file for final version of chapter 1\\
% |cdocsfi2.tex|&forwarding file for final version of chapter 2\\
% \end{tabular}
% \end{center}
% Each of the eight files can be compiled directly by the \LaTeX{} compiler.
%
% %%%%%%%%%%%%%%%%%%%%%%%%%%%%%%%%%%%%%%
% \paragraph{Main File.}
%
% The main file is called |cdocsamp.tex|.
%
% Load the \textsf{childdoc} definitions and
% declare the filename for the main document:
%    \begin{macrocode}
\input{childdoc.def}
\childdocmain{}
%    \end{macrocode}

% Optional override for |\version| flag:
%    \begin{macrocode}
%%\ifchilddoc\else\providecommand{\version}{draft}\fi
%    \end{macrocode}

% Define the default values for the |\version| flag
% (|final| for the main file and |draft| for childs):
%    \begin{macrocode}
\ifchilddoc
\providecommand{\version}{draft}
\else
\providecommand{\version}{final}
\fi
%    \end{macrocode}

% Load the standard document class:
%    \begin{macrocode}
\documentclass[12pt]{article}
%    \end{macrocode}

% Start the document body:
%    \begin{macrocode}
\begin{document}
%    \end{macrocode}

% Declare a title page.
% Print title, part of document being processed and version flag:
%    \begin{macrocode}
\addtocounter{page}{-1}
\begin{center}
{\LARGE\bfseries{}childdoc example\par}
\vspace{1cm}
\ifchilddoc
\ifchilddocmanual part\else chapter\fi:
`\childdocname' of `\childdocjob'\par
\else
main document: `\childdocjob'\par
\fi
version: \version\par
\end{center}
\newpage
%    \end{macrocode}

% Manually include selected file,
% otherwise process as usual:
%    \begin{macrocode}
\ifchilddocmanual
\section*{part `\childdocname'}
\input{\childdocname}
\else
%    \end{macrocode}

% Include the two chapters:
%    \begin{macrocode}
\include{cdocsch1}
\include{cdocsch2}
%    \end{macrocode}

% Include the two parts unless only chapters should be displayed:
%    \begin{macrocode}
\ifchilddoc\else
\section{part three}
\input{cdocspt3}
\section{part four}
\input{cdocspt4}
\fi
%    \end{macrocode}

% Process as usual until here:
%    \begin{macrocode}
\fi
%    \end{macrocode}

% End of document body:
%    \begin{macrocode}
\end{document}
%    \end{macrocode}
%\iffalse
%</samplemain>
%\fi
%
% %%%%%%%%%%%%%%%%%%%%%%%%%%%%%%%%%%%%%%
% \paragraph{Chapter Include Files.}
%
% The include files are called |cdocsch1.tex| and |cdocsch2.tex|.
%
%\iffalse
%<*samplechap1|samplechap2>
%\fi

% Optional override for |\version| flag:
%    \begin{macrocode}
%%\providecommand{\version}{final}
%    \end{macrocode}

% Include the main document:
%    \begin{macrocode}
\input{childdoc.def}
\childdocof{cdocsamp}
%    \end{macrocode}

%\iffalse
%</samplechap1|samplechap2>
%\fi
%
%\iffalse
%<*samplechap1>
%\fi
% Some text for chapter 1:
%    \begin{macrocode}
\section{one}
some text in chapter one
%    \end{macrocode}

%\iffalse
%</samplechap1>
%\fi
% Some text for chapter 2:
%\iffalse
%<*samplechap2>
%\fi
%    \begin{macrocode}
\section{two}
more text in chapter two
%    \end{macrocode}

%\iffalse
%</samplechap2>
%\fi
%
% %%%%%%%%%%%%%%%%%%%%%%%%%%%%%%%%%%%%%%
% \paragraph{Part Include Files.}
%
% The include files are called |cdocspt3.tex| and |cdocspt4.tex|.
%
%\iffalse
%<*samplepart3|samplepart4>
%\fi

% Optional override for |\version| flag:
%    \begin{macrocode}
%%\providecommand{\version}{final}
%    \end{macrocode}

% Include the main document:
%    \begin{macrocode}
\input{childdoc.def}
\childdocby{cdocsamp}
%    \end{macrocode}

%\iffalse
%</samplepart3|samplepart4>
%\fi
%
%\iffalse
%<*samplepart3>
%\fi
% Some text for part 3:
%    \begin{macrocode}
some text in part three
%    \end{macrocode}

%\iffalse
%</samplepart3>
%\fi
% Some text for part 4:
%\iffalse
%<*samplepart4>
%\fi
%    \begin{macrocode}
more text in part four
%    \end{macrocode}

%\iffalse
%</samplepart4>
%\fi
%
% %%%%%%%%%%%%%%%%%%%%%%%%%%%%%%%%%%%%%%
% \paragraph{Forwarding for a Complete Draft.}
%
% The following forwarding file |cdocsdrf.tex|
% compiles the main document in draft mode:
%\iffalse
%<*sampledraft>
%\fi
%    \begin{macrocode}
\def\version{draft}
\input{childdoc.def}
\childdocforward{cdocsamp}
%    \end{macrocode}

%\iffalse
%</sampledraft>
%\fi
%
% %%%%%%%%%%%%%%%%%%%%%%%%%%%%%%%%%%%%%%
% \paragraph{Forwarding for Final Version of the Chapters.}
%
% The following forwarding files |cdocsfn1.tex| and |cdocsfn2.tex|
% (with identical content)
% compile the final versions of the child documents
% |cdocsch1.tex| and |cdocsch2.tex|, respectively:
%\iffalse
%<*samplefinal>
%\fi
%    \begin{macrocode}
\def\version{final}
\input{childdoc.def}
\childdocforwardprefix[cdocsamp]{cdocsfn}{cdocsch}
%    \end{macrocode}

%\iffalse
%</samplefinal>
%\fi
%
% %%%%%%%%%%%%%%%%%%%%%%%%%%%%%%%%%%%%%%
% \paragraph{Command Line Processing.}
%
% The following three command lines generate the output files
% |cdocscld|, |cdocscl1| and |cdocscl2|
% which should be identical to
% |cdocsdrf|, |cdocsch1| and |cdocsfn2|, respectively:
% \begin{center}
% \begin{tabular}{l}
% |latex -jobname cdocscld \|\\
% |  "\def\version{draft}\input{childdoc.def}\childdocforward{cdocsamp}"|\\
% |latex -jobname cdocscl1 \|\\
% |  "\input{childdoc.def}\childdocforward[cdocsamp]{cdocsch1}"|\\
% |latex -jobname cdocscl2 \|\\
% |  "\def\version{final}\input{childdoc.def}\childdocforward{cdocsch2}"|
% \end{tabular}
% \end{center}
% Note that the trailing backslash on each first line
% merely continues the input to the second line
% (for convenient cut ant paste).
% Furthermore, the command |latex| can be replaced by any
% of its alternative versions such as |pdflatex|.
%
% %%%%%%%%%%%%%%%%%%%%%%%%%%%%%%%%%%%%%%%%%%%%%%%%%%%%%%%%%%%%%%%%%%%%%%%%%%%%%%
% %%%%%%%%%%%%%%%%%%%%%%%%%%%%%%%%%%%%%%%%%%%%%%%%%%%%%%%%%%%%%%%%%%%%%%%%%%%%%%
% \section{Implementation}
%\iffalse
%<*package>
%\fi
%
% This section describes the definitions file |childdoc.def|.

% The definitions cannot be loaded using |\usepackage| or |\RequirePackage|
% which has a mechanism to prevent loading a style file more than once.
% When loading the definitions by means of |\input|
% multiple instances have to be prevented manually:
%\iffalse
%This code needs to be before the `\ProvidesFile' directive
%which is defined at the beginning of this file.
%Therefore it is also placed there and commented out here.
%</package>
%<*discard>
%\fi
%    \begin{macrocode}
\ifdefined\childdocmain\endinput\fi
%    \end{macrocode}
%\iffalse
%</discard>
%<*package>
%\fi
%
% \macro{\ifchilddoc}
% \macro{\ifchilddocmanual}
% The conditional |\ifchilddoc| tells whether a
% child (true) or main (false) document is being compiled.
% The conditional |\ifchilddocmanual| tells whether
% the |\includeonly| mechanism is used (false) or
% the selection of child files must be performed manually (true).
% The definitions initialise to false:
%    \begin{macrocode}
\newif\ifchilddoc
\newif\ifchilddocmanual
%    \end{macrocode}

% \macro{\childdocname}
% \macro{\childdocjob}
% The macro |\childdocname| stores the name of the main document
% to be compiled. The macro |\childdocjob| stores the name of
% the document on which the \LaTeX{} compiler was originally invoked.
% The content of |\jobname| cannot be compared
% to filenames specified in the source due to different catcodes.
% The following code rescans |\jobname|, stores the result
% in |\childdocname| and saves a copy in |\childdocjob|:
%    \begin{macrocode}
\edef\childdocname{\scantokens\expandafter{\jobname\noexpand}}
\let\childdocjob\childdocname
%    \end{macrocode}

% \macro{\childdocdisable}
% The macro |\childdocdisable| prevents the main file
% from being processed more than once.
% At this stage, the main document command |\childdocmain|
% is assumed to be called once again where it should do nothing.
% Any subsequent call to it should prevent
% a secondary processing of the main document
% It overwrites the forwarding commands
% |\childdocof| and |\childdocforward|
% with empty macros to prevent further inclusions of the main document:
%    \begin{macrocode}
\newcommand{\childdocdisable}
{
  \renewcommand{\childdocmain}[1]{\renewcommand{\childdocmain}[1]{\endinput}}
  \renewcommand{\childdocof}[1]{}
  \renewcommand{\childdocby}[2][]{}
  \renewcommand{\childdocforward}[2][]{}
  \renewcommand{\childdocdisable}{}
}
%    \end{macrocode}

% \macro{\childdocmain}
% The macro |\childdocmain| is to be called at the top of the main file
% with nothing or the main filename (without extension) as argument.
% First, it breaks loops.
% If the argument is not empty and does not match |\childdocname|
% (which is set by the first inclusion of |childdoc.def|),
% |\ifchilddoc| is set to true, |\includeonly| is applied to the child file
% and |\jobname| is set to the main file
% (for proper handling of |.aux| files):
%    \begin{macrocode}
\newcommand{\childdocmain}[1]
{
  \childdocdisable\childdocmain{}
  \if?#1?\else
    \begingroup
      \def\childdoctmp{#1}
      \ifx\childdoctmp\childdocname
        \def\childdoctmp{}
      \else
        \def\childdoctmp
        {
          \childdoctrue
          \includeonly{\childdocname}
          \def\childdocjob{#1}
          \def\jobname{#1}
        }
      \fi
      \expandafter
    \endgroup
    \childdoctmp
  \fi
}
%    \end{macrocode}

% \macro{\childdocof}
% The command |\childdocof| redirects
% compilation to the main file |#1|.
%    \begin{macrocode}
\newcommand{\childdocof}[1]
{
  \childdocdisable
  \childdoctrue
  \includeonly{\childdocname}
  \def\jobname{#1}
  \def\childdocjob{#1}
  \input{#1}
}
%    \end{macrocode}

% \macro{\childdocby}
% The command |\childdocby| ....
%    \begin{macrocode}
\newcommand{\childdocby}[2][]
{
  \childdocdisable
  \childdoctrue
  \childdocmanualtrue
  \if?#1?\else
    \def\jobname{#2}
  \fi
  \def\childdocjob{#2}
  \input{#2}
  \endinput
}
%    \end{macrocode}

% \macro{\childdocforward}
% The command |\childdocforward| redirects
% compilation to the main file or
% (if the optional argument is given) a child file.
% Parameters are set as if the main file
% or a child file starting with |\childdocof| was compiled.
% Then compilation is handed over to the main file:
%    \begin{macrocode}
\newcommand{\childdocforward}[2][]
{
  \begingroup
    \if?#1?
      \def\childdoctmp
      {
        \def\childdocname{#2}
        \def\childdocjob{#2}
        \def\jobname{#2}
        \input{#2}
        \endinput
      }
    \else
      \def\childdoctmp
      {
        \childdocdisable
        \def\childdocname{#2}
        \childdoctrue
        \includeonly{#2}
        \def\childdocjob{#1}
        \def\jobname{#1}
        \input{#1}
        \endinput
      }
    \fi
    \expandafter
  \endgroup
  \childdoctmp
}
%    \end{macrocode}

% \macro{\childdocforwardprefix}
% The command |\childdocforwardprefix| redirects
% compilation to the main or a child file by means of a pattern.
% The prefix |#1| in the current filename is replaced by |#2|
% and the suffix of the current filename is kept
% (it is assumed that the filename does not contain the substring `|~~~|'
% which is used as a delimiter).
% Compilation is handed over to the new file by |\childdocforward|:
%    \begin{macrocode}
\newcommand{\childdocforwardprefix}[3][]
{
  \begingroup
    \def\childdocextract #2##1~~~{\def\childdoctmp{\childdocforward[#1]{#3##1}}}
    \expandafter\childdocextract\childdocname~~~
    \expandafter
  \endgroup
  \childdoctmp
}
%    \end{macrocode}

% \macro{\childdoc}
% The deprecated macro |\childdoc| is a legacy version of |\childdocmain|:
%    \begin{macrocode}
\newcommand{\childdoc}{\childdocmain}
%    \end{macrocode}

% \macro{\childdocredirect}
% The deprecated macro |\childdocredirect| is a legacy version
% of |\childdocforward| and |\childdocforwardprefix|:
%    \begin{macrocode}
\newcommand{\childdocredirect}[2][]
{
  \begingroup
    \if?#1?
      \def\childdoctmp{\childdocforward{#2}}
    \else
      \def\childdoctmp{\childdocforwardprefix{#1}{#2}}
    \fi
    \expandafter
  \endgroup
  \childdoctmp
}
%    \end{macrocode}

%\iffalse
%</package>
%\fi
%
\endinput
|\\
|\childdocmain{|\textit{main}|}|\\
\end{tabular}
\end{center}
%
If |\jobname| does not match the argument \textit{main} of |\childdocmain|,
it is assumed that |\jobname| points to the child file to be compiled.
When using |\childdocmain| with the main file specified as argument,
it suffices to start a child file
with just |\input{|\textit{main}|}|
without loading of the package and using |\childdocof|.
If instead all processing is done
with the appropriate \textsf{childdoc} directives,
the argument of \textit{main} of |\childdocmain| can be empty.

An alternative version of the command line processing described
in \secref{sec:commandline} using the detection mechanism reads:
%
\begin{center}
|... -jobname "|\textit{target}|" "|[\textit{flags}]%
[|\def\jobname{|\textit{dest}|}|]|\input{|\textit{main}|}"|
\end{center}

%%%%%%%%%%%%%%%%%%%%%%%%%%%%%%%%%%%%%%%%%%%%%%%%%%%%%%%%%%%%%%%%%%%%%%%%%%%%%%%%
\subsection{Manual Code}
\label{sec:manual}

In case one cannot be certain whether the definitions file |childdoc.def|
is installed on the target \TeX{} distribution
and one prefers not to ship it,
it is conceivable to paste a few relevant commands into the sources.

To that end, drop all statements |% \iffalse
%
% childdoc.dtx Copyright (C) 2017-2018 Niklas Beisert
%
% This work may be distributed and/or modified under the
% conditions of the LaTeX Project Public License, either version 1.3
% of this license or (at your option) any later version.
% The latest version of this license is in
%   http://www.latex-project.org/lppl.txt
% and version 1.3 or later is part of all distributions of LaTeX
% version 2005/12/01 or later.
%
% This work has the LPPL maintenance status `maintained'.
%
% The Current Maintainer of this work is Niklas Beisert.
%
% This work consists of the files childdoc.dtx and childdoc.ins
% and the derived files childdoc.def and cdocsamp.tex with
% cdocsch1.tex, cdocsch2.tex, cdocsdrf.tex, cdocsfn1.tex, cdocsfn2.tex.
%
%<package>\ifdefined\childdocmain\endinput\fi
%<package>\ProvidesFile{childdoc.def}[2018/12/30 v2.0 child document driver]
%<samplemain>\ProvidesFile{cdocsamp.tex}[2018/12/30 v2.0 sample for childdoc]
%<*driver>
%\ProvidesFile{childdoc.drv}[2018/12/30 v2.0 childdoc reference manual file]
\PassOptionsToClass{10pt,a4paper}{article}
\documentclass{ltxdoc}

\usepackage[margin=35mm]{geometry}
\usepackage{hyperref}
\usepackage{hyperxmp}
\usepackage[usenames]{color}

\hypersetup{colorlinks=true}
\hypersetup{pdfstartview=FitH}
\hypersetup{pdfpagemode=UseNone}
\hypersetup{pdfsource={}}
\hypersetup{pdflang={en-UK}}
\hypersetup{pdfcopyright={Copyright 2017-2018 Niklas Beisert.
  This work may be distributed and/or modified under the
  conditions of the LaTeX Project Public License, either version 1.3
  of this license or (at your option) any later version.}}
\hypersetup{pdflicenseurl={http://www.latex-project.org/lppl.txt}}
\hypersetup{pdfcontactaddress={ETH Zurich, ITP, HIT K,
  Wolfgang-Pauli-Strasse 27}}
\hypersetup{pdfcontactpostcode={8093}}
\hypersetup{pdfcontactcity={Zurich}}
\hypersetup{pdfcontactcountry={Switzerland}}
\hypersetup{pdfcontactemail={nbeisert@itp.phys.ethz.ch}}
\hypersetup{pdfcontacturl={http://people.phys.ethz.ch/\xmptilde nbeisert/}}

\newcommand{\secref}[1]{\hyperref[#1]{section \ref*{#1}}}

\parskip1ex
\parindent0pt
\let\olditemize\itemize
\def\itemize{\olditemize\parskip0pt}

\begin{document}

\title{The \textsf{childdoc} Package}
\hypersetup{pdftitle={The childdoc Package}}
\author{Niklas Beisert\\[2ex]
  Institut f\"ur Theoretische Physik\\
  Eidgen\"ossische Technische Hochschule Z\"urich\\
  Wolfgang-Pauli-Strasse 27, 8093 Z\"urich, Switzerland\\[1ex]
  \href{mailto:nbeisert@itp.phys.ethz.ch}
  {\texttt{nbeisert@itp.phys.ethz.ch}}}
\hypersetup{pdfauthor={Niklas Beisert}}
\hypersetup{pdfsubject={Manual for the LaTeX2e Package childdoc}}
\date{30 December 2018, \textsf{v2.0}}
\maketitle

\begin{abstract}\noindent
\textsf{childdoc} is a \LaTeXe{} package
that enables the direct compilation
of document sections included by |\include|
to individual files.
\end{abstract}

\begingroup
\parskip0ex
\tableofcontents
\endgroup

%%%%%%%%%%%%%%%%%%%%%%%%%%%%%%%%%%%%%%%%%%%%%%%%%%%%%%%%%%%%%%%%%%%%%%%%%%%%%%%%
%%%%%%%%%%%%%%%%%%%%%%%%%%%%%%%%%%%%%%%%%%%%%%%%%%%%%%%%%%%%%%%%%%%%%%%%%%%%%%%%
\section{Introduction}

\LaTeX{} provides a mechanism to structure a large document (such as a book)
into a main file and several child files (containing the chapters)
using the |\include| command.
This mechanism is beneficial for documents
which span hundreds of pages in order to
make the source file(s) more manageable.
Moreover, compilation can be restricted to
selected child files by means of the |\includeonly| command.
The latter feature can be used to reduce the compilation time while editing
(this was significantly more useful in the earlier days of \LaTeX{})
or to generate a smaller document which is easier to navigate.
Another application of |\includeonly| is to generate
documents consisting of selected parts of the complete document.

However, there are a few drawbacks of the plain |\include| mechanism:
\begin{itemize}
\item
The child files cannot be compiled on their own,
they can only be compiled via the main file.
A naive editing environment
(such as a text editor with an option
to have the current file processed by \LaTeX)
may require one to switch to the main file before compiling;
attempting to compile the child file produces errors.
\item
The main file must be modified (each time)
to adjust the |\includeonly| command
to the present needs. This easily leaves the main file in a messy state.
\item
The generated document will always carry the filename
of the main document. This is inconvenient if
several child files are to be compiled and
to be kept for distribution.
\end{itemize}

The present package provides a simple interface
to make child files individually compilable by \LaTeX{}.
Compiling a child file then has the same effect as compiling
the main file with an |\includeonly| command
to select the appropriate child.
Moreover the generated document will carry the name of the child
rather than the main file.
This resolves all three above issues.

This feature is meant to make the editing of books,
thesis documents and lecture notes somewhat more convenient.
However, the package can also be used efficiently for
composing a series of documents (such as exercise sheets)
which are typically distributed individually.
It then assists the author in generating the individual documents
(potentially in different versions)
as well as a document containing the collected series.
Another application is in developing style files
or other kinds of included material
where compilation of the style file could redirect
to a sample or test file.

%%%%%%%%%%%%%%%%%%%%%%%%%%%%%%%%%%%%%%%%%%%%%%%%%%%%%%%%%%%%%%%%%%%%%%%%%%%%%%%%
%%%%%%%%%%%%%%%%%%%%%%%%%%%%%%%%%%%%%%%%%%%%%%%%%%%%%%%%%%%%%%%%%%%%%%%%%%%%%%%%
\section{Usage}

First of all, the package \textsf{childdoc} is \emph{not} a standard
\LaTeXe{} |.sty| style file! Therefore it needs to be invoked in
a non-standard way.

%%%%%%%%%%%%%%%%%%%%%%%%%%%%%%%%%%%%%%%%%%%%%%%%%%%%%%%%%%%%%%%%%%%%%%%%%%%%%%%%
\subsection{Included Files}
\label{sec:include}

%%%%%%%%%%%%%%%%%%%%%%%%%%%%%%%%%%%%%%%%
\DescribeMacro{\childdocmain}
To use the package, add the commands
\begin{center}
\begin{tabular}{l}
|\input{childdoc.def}|\\
|\childdocmain{}|\\
\end{tabular}
\end{center}
at the very top of the main \LaTeX{} file,
in particular \emph{before} the |\documentclass| statement!
The argument of |\childdocmain| should be left empty
(but it must be present).

%%%%%%%%%%%%%%%%%%%%%%%%%%%%%%%%%%%%%%%%
\DescribeMacro{\childdocof}
Furthermore, add the commands
\begin{center}
\begin{tabular}{l}
|\input{childdoc.def}|\\
|\childdocof{|\textit{main}|}|\\
\end{tabular}
\end{center}
at the top of every child file \textit{child}
which is included by |\include{|\textit{child}|}|
from within the main file
(or at least for those files to be compiled individually).
The argument \textit{main} must be the filename of the main file.

There are a couple of
considerations in setting up the main and child documents:

%%%%%%%%%%%%%%%%%%%%%%%%%%%%%%%%%%%%%%%%
\paragraph{Restrictions.}

Please note the following restrictions:
\begin{itemize}
\item
|\childdocmain| must be called with one argument \textit{main}
to ensure compatibility with earlier version of the package.
It must either be empty (|\childdocmain{}|)
or precisely match the filename of the main file in which it is specified.
See \secref{sec:detection} for further information.
\item
The filename \textit{main} must be specified without the |.tex| extension.
\item
The filename \textit{main} is case sensitive
(even in case-insensitive file systems)
due to internal string comparison.
\item
The argument \textit{main} should be fully expanded, it cannot be a macro.
\item
Subdirectories and special characters should be avoided in filenames.
\item
The command |\childdocmain{|\textit{main}|}| must be followed by a whitespace.
It should not be followed immediately by another command
or by a comment mark `|%|'.
This is because the \TeX{} parser reads the token immediately following
the argument of |\childdocmain| and puts it
at the beginning of every child section;
however, a white\-space is ignored.
\end{itemize}

%%%%%%%%%%%%%%%%%%%%%%%%%%%%%%%%%%%%%%%%
\paragraph{Content of Main File.}

It is advisable to place all content in the child files included by |\include|.
Any output contained in the main file will appear in all child documents
unless suppressed manually;
it cannot be suppressed automatically by the |\includeonly| directive
and thus should normally be avoided.
A method to include some content in the main file
by means of conditional processing is described in \secref{sec:conditional}.

%%%%%%%%%%%%%%%%%%%%%%%%%%%%%%%%%%%%%%%%
\paragraph{Page Numbering.}

When only a part of the document is compiled,
the appropriate numbering of pages
(as well as other status parameters)
is determined from the |.aux| files.
The latter contain information from previous passes.
However this information needs to propagate through
all intermediate child documents.
Therefore the page numbering in child documents may well
be inconsistent until the complete document is compiled at least once.

A useful (if unconventional) way to always ensure a consistent
page numbering is to restart the numbering in each child document
and denote the pages by `\textit{child}|.|\textit{page}'
where \textit{child} represents the chapter/section number of the child file.
This can be achieved by the command
|\numberwithin{page}{|\textit{child}|}|
of the \textsf{amsmath} package
where \textit{child} can be |chapter| or |section|
depending on the chosen structuring.
Alternatively, one can modify the macro |\thepage| appropriately
and reset the counter |page| at the start of each child file.

%%%%%%%%%%%%%%%%%%%%%%%%%%%%%%%%%%%%%%%%%%%%%%%%%%%%%%%%%%%%%%%%%%%%%%%%%%%%%%%%
\subsection{Conditional Processing}
\label{sec:conditional}

The package provides a mechanism to compile different versions
of a document. To customise the versions further some conditional processing
can come in handy to distinguish which version is being compiled.
The package provides two macros to describe the compilation context:

%%%%%%%%%%%%%%%%%%%%%%%%%%%%%%%%%%%%%%%%
\DescribeMacro{\ifchilddoc}
The conditional |\ifchilddoc| distinguishes between the compilation of
child documents and the main document:
%
\begin{center}
|\ifchilddoc |\textit{child-code}| |[|\||else |\textit{main-code}]| \||fi|
\end{center}

%%%%%%%%%%%%%%%%%%%%%%%%%%%%%%%%%%%%%%%%
\DescribeMacro{\childdocname}
\DescribeMacro{\childdocjob}
The macro |\childdocname| contains the filename (without extension)
of the main or child file being processed.
Note that |\childdocjob| will always contain the name of the main file.

%%%%%%%%%%%%%%%%%%%%%%%%%%%%%%%%%%%%%%%%
\paragraph{Title Page.}

Conditional processing can be used to include a title or banner page
in the main document when proper precautions are taken.
Importantly, the code in the main file should ensure that the page counter
(as well as other status parameters which are stored in the |.aux| files)
takes the same value after the conditional processing.
Otherwise the page numbers may take divergent values
depending on which part is compiled.

For example, a title page could be declared by:
%
\begin{center}
\begin{tabular}{l}
|\ifchilddoc\||else|\\
|\addtocounter{page}{-1}|\\
\textit{code for title page}\\
|\newpage|\\
|\||fi|
\end{tabular}
\end{center}
%
A banner page for the child documents can be generated by:
%
\begin{center}
\begin{tabular}{l}
|\ifchilddoc|\\
|\addtocounter{page}{-1}|\\
\textit{code for banner page}\\
|\newpage|\\
|\||fi|
\end{tabular}
\end{center}
%
Here one could write a message such as:
\begin{center}
|This is the part \childdocname{} of \childdocjob{}.|
\end{center}

%%%%%%%%%%%%%%%%%%%%%%%%%%%%%%%%%%%%%%%%%%%%%%%%%%%%%%%%%%%%%%%%%%%%%%%%%%%%%%%%
\subsection{Flags}
\label{sec:flags}

The package makes it easy to generate different versions
of the main or child documents.
To this end compilation flags can be defined
and assigned different default values.
They will be particularly useful in conjunction
with the forwarding mechanism described in \secref{sec:forward}.

For example, it may be useful to have a flag |\version|
which can be set to |draft| or |final|.
The document source will contain some conditional code
depending on the value of |\version|.
Suppose further, the flag should default to |final| for the main file
and to |draft| for child files
which is a natural assignment for editing the document.
This is achieved by placing the following code
in the preamble of the main document
(below the |\childdocmain| directive):
%
\begin{center}
\begin{tabular}{l}
|\ifchilddoc|\\
|\providecommand{\version}{draft}|\\
|\||else|\\
|\providecommand{\version}{final}|\\
|\||fi|
\end{tabular}
\end{center}
%
The definition by |\providecommand| makes sure
that previous definitions are not overwritten.
Further statements |\providecommand{\version}{...}|
can thus be added before the above code to override it.

For the main file, one might add a line
(between |\childdocmain| and the above block)
%
\begin{center}
|%\ifchilddoc\||else\providecommand{\version}{draft}\||fi|
\end{center}
%
which can be uncommented to produce a draft version.
Likewise one can add a line to the very top of a child file
(above the |\childdocof{|\textit{main}|}| directive)
%
\begin{center}
|%\providecommand{\version}{final}|
\end{center}
%
which can be uncommented to produce the final version of this child document.

%%%%%%%%%%%%%%%%%%%%%%%%%%%%%%%%%%%%%%%%%%%%%%%%%%%%%%%%%%%%%%%%%%%%%%%%%%%%%%%%
\subsection{Forwarding}
\label{sec:forward}

Different versions of the main or child documents
using compilation flags as described in \secref{sec:flags}
can be (permanently) stored in different files
for convenient compilation, viewing and distribution.
To this end, the package defines a command
to pass on compilation to a different file:

%%%%%%%%%%%%%%%%%%%%%%%%%%%%%%%%%%%%%%%%
\DescribeMacro{\childdocforward}
The command |\childdocforward| redirects processing to
another source file:
%
\begin{center}
\begin{tabular}{l}
|\input{childdoc.def}|\\
|\childdocforward[|\textit{main}|]{|\textit{dest}|}|\\
\end{tabular}
\end{center}
%
The argument \textit{dest} is the destination file
(without extension).
It should be the main file or one of the child files.
Note that further \textsf{childdoc} directives
such as |\childdocof| and |\childdocforward|
in the indicated file will be processed in this form.
The optional argument \textit{main}
passes on directly to the main file \textit{main}
while pretending to compile the child \textit{dest}.
This form behaves as if \textit{dest}
issues |\childdocof{|\textit{main}|}| right away,
and no further \textsf{childdoc} directives will be processed.

%%%%%%%%%%%%%%%%%%%%%%%%%%%%%%%%%%%%%%%%
\DescribeMacro{\...prefix}
In the alternative form |\childdocforwardprefix|,
%
\begin{center}
\begin{tabular}{l}
|\input{childdoc.def}|\\
|\childdocforwardprefix[|\textit{main}|]{|\textit{prefix}|}{|\textit{dest}|}|
\end{tabular}
\end{center}
%
the destination file is determined by a pattern
depending on the current file:
To make this work, the current file must be called
`{\textit{prefix}\hspace{0.2em}\textit{suffix}}'
with \textit{prefix} matching precisely the argument.
Processing is then passed on to the file
`{\textit{dest}\hspace{0.2em}\textit{suffix}}'.
Surely, the same effect is achieved by
directly specifying the
argument `{\textit{dest}\hspace{0.2em}\textit{suffix}}'
in the first form.
However, that requires to set up a different file
for each child. With the alternative form of the command
all these files can have exactly the same content
which simplifies setting them up and maintaining them.

For example, the following file |draft.tex|
with a compilation flag |\version| as described in \secref{sec:flags}
compiles the main document as a draft:
%
\begin{center}
\begin{tabular}{l}
|\def\version{draft}|\\
|\input{childdoc.def}|\\
|\childdocforward{|\textit{main}|}|
\end{tabular}
\end{center}
%
Likewise, the following files |final|\textit{nn}|.tex|
compile the final version of the child document
|child|\textit{nn}|.tex|:
%
\begin{center}
\begin{tabular}{l}
|\def\version{final}|\\
|\input{childdoc.def}|\\
|\childdocforwardprefix{final}{child}|
\end{tabular}
\end{center}
%

Note that when several versions of a main file and/or of each child file
are to be generated, it may be convenient to set up a |Makefile| or
shell script to automatise the process.

%%%%%%%%%%%%%%%%%%%%%%%%%%%%%%%%%%%%%%%%%%%%%%%%%%%%%%%%%%%%%%%%%%%%%%%%%%%%%%%%
\subsection{Command Line Processing}
\label{sec:commandline}

The effect of redirection files can also be achieved by invoking
the \LaTeX{} compiler with a more elaborate command line.
Most conveniently this should be done as part
of a shell script or a |Makefile|.

When using \textsf{childdoc} in the main file, the following
command lines effectively perform a redirection
(note that depending on the shell being used,
backslashes may have to be doubled: `|\|' $\to$ `|\\|'):
%
\begin{center}
|... -jobname "|\textit{target}|" |\\|"|[\textit{flags}]%
|\input{childdoc.def}\childdocforward[|\textit{main}|]{|\textit{dest}|}"|
\end{center}
%
Here \textit{target} is the name of the output file,
\textit{main} is the name of the main file
and \textit{dest} is the name of the main or child file to be processed
(all filenames without extensions).
The optional argument \textit{main} can be omitted
if \textit{main} matches \textit{dest}.
Optionally, compilation \textit{flags} can be defined via |\def| commands.
This command line makes the \TeX{} engine believe
it is compiling the file \textit{target}
whose content is specified as the latter parameter.
The provided code then forwards the processing to
\textit{main} or \textit{dest} as described in \secref{sec:forward}.

%%%%%%%%%%%%%%%%%%%%%%%%%%%%%%%%%%%%%%%%%%%%%%%%%%%%%%%%%%%%%%%%%%%%%%%%%%%%%%%%
\subsection{Include by Input}
\label{sec:input}

Including child documents by |\include| has some restrictions by design.
Most notably, the content of a child document always occupies
its own set of pages; pages cannot be shared between child documents.
Usually, this behaviour makes perfect sense
because each child document contain an essential part of the document.
However, in some situations it may be desirable to compose
a document from a collection of parts
without having mandatory page breaks between then.
For this case, the package
provides a mechanism to include parts
by |\input| which can also be processed individually.
However, by construction this mechanism
requires manual handling of the content to be output.

%%%%%%%%%%%%%%%%%%%%%%%%%%%%%%%%%%%%%%%%
\DescribeMacro{\ifchilddocmanual}
The main file should be prepared as usual, see \secref{sec:include}.
However, the document body must make a distinction
between processing of an individual part and of the main document, e.g.:
%
\begin{center}
\begin{tabular}{l}
|\ifchilddocmanual|\\
|\input{\childdocname}|\\
|\||else|\\
\textit{document body with }|\input{|\textit{part}|}|\\
|\||fi|
\end{tabular}
\end{center}
%
The conditional |\ifchilddocmanual| is true whenever
a part to be included by |\input| is being compiled,
and the name of the part is stored in |\childdocname|.

%%%%%%%%%%%%%%%%%%%%%%%%%%%%%%%%%%%%%%%%
\DescribeMacro{\childdocby}
Each part to be included by |\input| should start with:
%
\begin{center}
\begin{tabular}{l}
|\input{childdoc.def}|\\
|\childdocby{|\textit{main}|}|\\
\end{tabular}
\end{center}
%
The directive |\childdocby| is similar to |\childdocof|
described in \secref{sec:include},
but the subsequent selection of content must be done manually.
To that end, both |\ifchilddoc| and |\ifchilddocmanual|
will be true upon processing of a part,
and the name of the part is stored in |\childdocname|.
Note that |\jobname| will be set to the filename of the current part
so that each part receives an individual |.aux| file
that does not interfere with the |.aux| file(s) of the main document.
This behaviour can be altered by the alternative form
|\childdocby[*]{|\textit{main}|}| (with a non-empty optional argument)
which uses the |.aux| file of the main document
by setting |\jobname| to \textit{main}.

%%%%%%%%%%%%%%%%%%%%%%%%%%%%%%%%%%%%%%%%%%%%%%%%%%%%%%%%%%%%%%%%%%%%%%%%%%%%%%%%
\subsection{Driver Development}
\label{sec:driver}

The \textsf{childdoc} mechanism can also be use for the development
of definition files such as \LaTeX{} styles or classes.
This case differs from the above setup with multiple parts
included by |\include| in that no |\includeonly| should be invoked.
This can be achieved by starting the include file
(before |\ProvidesPackage|) with:
%
\begin{center}
\begin{tabular}{l}
|\input{childdoc.def}|\\
|\childdocforward{|\textit{main}|}|\\
\end{tabular}
\end{center}
%
or alternatively with:
%
\begin{center}
\begin{tabular}{l}
|\input{childdoc.def}|\\
|\childdocby{|\textit{main}|}|\\
\end{tabular}
\end{center}
%
Both forms have slightly different effects as described above.
The main file is prepared as usual, see \secref{sec:include}.

%%%%%%%%%%%%%%%%%%%%%%%%%%%%%%%%%%%%%%%%%%%%%%%%%%%%%%%%%%%%%%%%%%%%%%%%%%%%%%%%
\subsection{Legacy Detection}
\label{sec:detection}

The directive |\childdocmain| in the main file can detect
whether the complete document or merely a child is to be compiled
even without using the directive |\childdocof|.
This method is deprecated because it is less robust
and there is no compelling reason to use it;
it is merely provided for backward compatibility
and it may be removed in future versions.

If the detection mechanism is to be used,
it is mandatory to correctly specify
the filename of the main file as the argument of |\childdocmain|:
%
\begin{center}
\begin{tabular}{l}
|\input{childdoc.def}|\\
|\childdocmain{|\textit{main}|}|\\
\end{tabular}
\end{center}
%
If |\jobname| does not match the argument \textit{main} of |\childdocmain|,
it is assumed that |\jobname| points to the child file to be compiled.
When using |\childdocmain| with the main file specified as argument,
it suffices to start a child file
with just |\input{|\textit{main}|}|
without loading of the package and using |\childdocof|.
If instead all processing is done
with the appropriate \textsf{childdoc} directives,
the argument of \textit{main} of |\childdocmain| can be empty.

An alternative version of the command line processing described
in \secref{sec:commandline} using the detection mechanism reads:
%
\begin{center}
|... -jobname "|\textit{target}|" "|[\textit{flags}]%
[|\def\jobname{|\textit{dest}|}|]|\input{|\textit{main}|}"|
\end{center}

%%%%%%%%%%%%%%%%%%%%%%%%%%%%%%%%%%%%%%%%%%%%%%%%%%%%%%%%%%%%%%%%%%%%%%%%%%%%%%%%
\subsection{Manual Code}
\label{sec:manual}

In case one cannot be certain whether the definitions file |childdoc.def|
is installed on the target \TeX{} distribution
and one prefers not to ship it,
it is conceivable to paste a few relevant commands into the sources.

To that end, drop all statements |\input{childdoc.def}|
and perform the replacements as outlined below.
Instead of |\childdocmain{|\textit{main}|}| add the following code
to the top of the main file:
%
\begin{center}
\begin{tabular}{l}
|\||ifdefined\childdocname\endinput\||fi\newif\ifchilddoc|\\
|\edef\childdocname{\scantokens\expandafter{\jobname\noexpand}}|\\
|\def\childdocmain{|\textit{main}|}\||ifx\childdocmain\childdocname\||else|\\
|\childdoctrue\includeonly{\childdocname}\let\jobname\childdocmain\||fi|\\
\end{tabular}
\end{center}
%
Instead of |\childdocof{|\textit{main}|}| just include the main file
at the top of each child file:
%
\begin{center}
|\input{|\textit{main}|}|
\end{center}
%
A simple redirection |\childdocforward{|\textit{dest}|}| is achieved by:
%
\begin{center}
|\def\jobname{|\textit{dest}|}\input{\jobname}|
\end{center}
%
The redirection with prefix
|\childdocforwardprefix[|\textit{prefix}|]{|\textit{dest}|}|
is accomplished by:
%
\begin{center}
\begin{tabular}{l}
|{\edef\jobname{\scantokens\expandafter{\jobname\noexpand}}|\\
|\def\redirectjob |\textit{prefix}|#1~~~{\gdef\jobname{|\textit{dest}|#1}}|\\
|\expandafter\redirectjob\jobname~~~}\input{\jobname}|
\end{tabular}
\end{center}

In an alternative approach,
child documents can be compiled by a specific command line
without additional code or specific definitions:
%
\begin{center}
|... -jobname "|\textit{target}|" "|[\textit{flags}]%
|\includeonly{|\textit{dest}|}\input{|\textit{main}|}"|
\end{center}
%

%%%%%%%%%%%%%%%%%%%%%%%%%%%%%%%%%%%%%%%%%%%%%%%%%%%%%%%%%%%%%%%%%%%%%%%%%%%%%%%%
%%%%%%%%%%%%%%%%%%%%%%%%%%%%%%%%%%%%%%%%%%%%%%%%%%%%%%%%%%%%%%%%%%%%%%%%%%%%%%%%
\section{Information}

%%%%%%%%%%%%%%%%%%%%%%%%%%%%%%%%%%%%%%%%%%%%%%%%%%%%%%%%%%%%%%%%%%%%%%%%%%%%%%%%
\subsection{Copyright}

Copyright \copyright{} 2017--2018 Niklas Beisert

This work may be distributed and/or modified under the
conditions of the \LaTeX{} Project Public License, either version 1.3
of this license or (at your option) any later version.
The latest version of this license is in
  \url{http://www.latex-project.org/lppl.txt}
and version 1.3 or later is part of all distributions of \LaTeX{}
version 2005/12/01 or later.

This work has the LPPL maintenance status `maintained'.

The Current Maintainer of this work is Niklas Beisert.

This work consists of the files |README.txt|, |childdoc.ins| and |childdoc.dtx|
as well as the derived files |childdoc.def|, |cdocsamp.tex|
with |cdocsch1.tex|, |cdocsch2.tex|, |cdocspt3.tex|, |cdocspt4.tex|,
|cdocsdrf.tex|, |cdocsfn1.tex|, |cdocsfn2.tex|
as well as |childdoc.pdf|.

%%%%%%%%%%%%%%%%%%%%%%%%%%%%%%%%%%%%%%%%%%%%%%%%%%%%%%%%%%%%%%%%%%%%%%%%%%%%%%%%
\subsection{Files and Installation}

The package consists of the files:
%
\begin{center}
\begin{tabular}{ll}
    |README.txt|   & readme file \\
    |childdoc.ins| & installation file \\
    |childdoc.dtx| & source file \\
    |childdoc.def| & definition file \\
    |cdocsamp.tex| & sample main file \\
    |cdocsch1.tex| & sample include file \\
    |cdocsch2.tex| & sample include file \\
    |cdocspt3.tex| & sample part file \\
    |cdocspt4.tex| & sample part file \\
    |cdocsdrf.tex| & sample redirection file \\
    |cdocsfn1.tex| & sample redirection file \\
    |cdocsfn2.tex| & sample redirection file \\
    |childdoc.pdf| & manual
\end{tabular}
\end{center}
%
The distribution consists of the files
|README.txt|, |childdoc.ins| and |childdoc.dtx|.
%
\begin{itemize}
\item
Run (pdf)\LaTeX{} on |childdoc.dtx|
to compile the manual |childdoc.pdf| (this file).
\item
Run \LaTeX{} on |childdoc.ins| to create the definitions file |childdoc.def|
and the sample |cdocsamp.tex| with include files
|cdocsch1.tex|, |cdocsch2.tex|, |cdocspt3.tex|, |cdocspt4.tex|,
|cdocsdrf.tex|, |cdocsfn1.tex|, |cdocsfn2.tex|.
Then copy the file |childdoc.def| to an appropriate directory of your \LaTeX{}
distribution, e.g.\ \textit{texmf-root}|/tex/latex/childdoc|.
\end{itemize}

%%%%%%%%%%%%%%%%%%%%%%%%%%%%%%%%%%%%%%%%%%%%%%%%%%%%%%%%%%%%%%%%%%%%%%%%%%%%%%%%
\subsection{Related CTAN Packages}

There are several other packages which offer a similar functionality:
%
\begin{itemize}
\item
The packages
\href{http://ctan.org/pkg/docmute}{\textsf{docmute}},
\href{http://ctan.org/pkg/includex}{\textsf{includex}} and
\href{http://ctan.org/pkg/standalone}{\textsf{standalone}}
provide commands to include only the document body of
a child file thus allowing both files to be compiled individually.
\item
The packages \href{http://ctan.org/pkg/subdocs}{\textsf{subdocs}}
and \href{http://ctan.org/pkg/subfiles}{\textsf{subfiles}}
provide structures in which the main and child documents can be
encapsulated and allowing them to be compiled individually.
The inclusion mechanism is different from the conventional |\include|.
\item
The package \href{http://ctan.org/pkg/combine}{\textsf{combine}}
is an elaborate solution to combine several documents into one.
\end{itemize}
%
See also the CTAN topic \href{http://ctan.org/topic/subdocs}{\textsf{subdocs}}
for further related packages.
The present package differs from the above solutions in that
a document structure constructed with the conventional |\include| mechanism
just needs two extra commands at the top of every file
such that all constituent files can be compiled individually.

%%%%%%%%%%%%%%%%%%%%%%%%%%%%%%%%%%%%%%%%%%%%%%%%%%%%%%%%%%%%%%%%%%%%%%%%%%%%%%%%
%\subsection{Feature Suggestions}
%
%The following is a list of features which may be useful for future
%versions of this package:
%%
%\begin{itemize}
%\item
%\ldots
%\end{itemize}

%%%%%%%%%%%%%%%%%%%%%%%%%%%%%%%%%%%%%%%%%%%%%%%%%%%%%%%%%%%%%%%%%%%%%%%%%%%%%%%%
\subsection{Revision History}

%%%%%%%%%%%%%%%%%%%%%%%%%%%%%%%%%%%%%%%%
\paragraph{v2.0:} 2018/12/30

\begin{itemize}
\item
immediate forward processing
\item
added |\childdocby| mechanism
\item
manual restructured
\end{itemize}

%%%%%%%%%%%%%%%%%%%%%%%%%%%%%%%%%%%%%%%%
\paragraph{v1.6:} 2018/01/17

\begin{itemize}
\item
application for development of include files
\item
corrections to manual
\end{itemize}

%%%%%%%%%%%%%%%%%%%%%%%%%%%%%%%%%%%%%%%%
\paragraph{v1.5:} 2017/05/21

\begin{itemize}
\item
more complete structuring introduced
\item
|\childdocof| introduced
\item
|\childdoc| renamed to |\childdocmain|
\item
|\childredirect| renamed to |\childdocforward| and |\childdocforwardprefix|
and functionality expanded
\end{itemize}

%%%%%%%%%%%%%%%%%%%%%%%%%%%%%%%%%%%%%%%%
\paragraph{v1.0:} 2017/04/27

\begin{itemize}
\item
manual and install package
\item
first version published on CTAN
\end{itemize}

%%%%%%%%%%%%%%%%%%%%%%%%%%%%%%%%%%%%%%%%
\paragraph{v0.6:} 2017/04/26

\begin{itemize}
\item
redirection mechanism added
\end{itemize}

%%%%%%%%%%%%%%%%%%%%%%%%%%%%%%%%%%%%%%%%
\paragraph{v0.5:} 2017/04/26

\begin{itemize}
\item
functionality in definition file
\end{itemize}


%%%%%%%%%%%%%%%%%%%%%%%%%%%%%%%%%%%%%%%%%%%%%%%%%%%%%%%%%%%%%%%%%%%%%%%%%%%%%%%%
%%%%%%%%%%%%%%%%%%%%%%%%%%%%%%%%%%%%%%%%%%%%%%%%%%%%%%%%%%%%%%%%%%%%%%%%%%%%%%%%
%%%%%%%%%%%%%%%%%%%%%%%%%%%%%%%%%%%%%%%%%%%%%%%%%%%%%%%%%%%%%%%%%%%%%%%%%%%%%%%%
\appendix

\settowidth\MacroIndent{\rmfamily\scriptsize 000\ }

 \DocInput{childdoc.dtx}

\end{document}
%</driver>
% \fi
%
% %%%%%%%%%%%%%%%%%%%%%%%%%%%%%%%%%%%%%%%%%%%%%%%%%%%%%%%%%%%%%%%%%%%%%%%%%%%%%%
% %%%%%%%%%%%%%%%%%%%%%%%%%%%%%%%%%%%%%%%%%%%%%%%%%%%%%%%%%%%%%%%%%%%%%%%%%%%%%%
% \section{Sample}
%\iffalse
%<*samplemain>
%\fi
%
% The following presents a sample document
% with two chapters, two parts, a title page,
% a compile flag as well as three forwarding files to set the flag.
% It consists of eight |.tex| files:
% \begin{center}
% \begin{tabular}{ll}
% |cdocsamp.tex|&main file\\
% |cdocsch1.tex|&include file for chapter 1\\
% |cdocsch2.tex|&include file for chapter 2\\
% |cdocspt3.tex|&include file for part 3\\
% |cdocspt4.tex|&include file for part 4\\
% |cdocsdrf.tex|&forwarding file for main file in draft mode\\
% |cdocsfi1.tex|&forwarding file for final version of chapter 1\\
% |cdocsfi2.tex|&forwarding file for final version of chapter 2\\
% \end{tabular}
% \end{center}
% Each of the eight files can be compiled directly by the \LaTeX{} compiler.
%
% %%%%%%%%%%%%%%%%%%%%%%%%%%%%%%%%%%%%%%
% \paragraph{Main File.}
%
% The main file is called |cdocsamp.tex|.
%
% Load the \textsf{childdoc} definitions and
% declare the filename for the main document:
%    \begin{macrocode}
\input{childdoc.def}
\childdocmain{}
%    \end{macrocode}

% Optional override for |\version| flag:
%    \begin{macrocode}
%%\ifchilddoc\else\providecommand{\version}{draft}\fi
%    \end{macrocode}

% Define the default values for the |\version| flag
% (|final| for the main file and |draft| for childs):
%    \begin{macrocode}
\ifchilddoc
\providecommand{\version}{draft}
\else
\providecommand{\version}{final}
\fi
%    \end{macrocode}

% Load the standard document class:
%    \begin{macrocode}
\documentclass[12pt]{article}
%    \end{macrocode}

% Start the document body:
%    \begin{macrocode}
\begin{document}
%    \end{macrocode}

% Declare a title page.
% Print title, part of document being processed and version flag:
%    \begin{macrocode}
\addtocounter{page}{-1}
\begin{center}
{\LARGE\bfseries{}childdoc example\par}
\vspace{1cm}
\ifchilddoc
\ifchilddocmanual part\else chapter\fi:
`\childdocname' of `\childdocjob'\par
\else
main document: `\childdocjob'\par
\fi
version: \version\par
\end{center}
\newpage
%    \end{macrocode}

% Manually include selected file,
% otherwise process as usual:
%    \begin{macrocode}
\ifchilddocmanual
\section*{part `\childdocname'}
\input{\childdocname}
\else
%    \end{macrocode}

% Include the two chapters:
%    \begin{macrocode}
\include{cdocsch1}
\include{cdocsch2}
%    \end{macrocode}

% Include the two parts unless only chapters should be displayed:
%    \begin{macrocode}
\ifchilddoc\else
\section{part three}
\input{cdocspt3}
\section{part four}
\input{cdocspt4}
\fi
%    \end{macrocode}

% Process as usual until here:
%    \begin{macrocode}
\fi
%    \end{macrocode}

% End of document body:
%    \begin{macrocode}
\end{document}
%    \end{macrocode}
%\iffalse
%</samplemain>
%\fi
%
% %%%%%%%%%%%%%%%%%%%%%%%%%%%%%%%%%%%%%%
% \paragraph{Chapter Include Files.}
%
% The include files are called |cdocsch1.tex| and |cdocsch2.tex|.
%
%\iffalse
%<*samplechap1|samplechap2>
%\fi

% Optional override for |\version| flag:
%    \begin{macrocode}
%%\providecommand{\version}{final}
%    \end{macrocode}

% Include the main document:
%    \begin{macrocode}
\input{childdoc.def}
\childdocof{cdocsamp}
%    \end{macrocode}

%\iffalse
%</samplechap1|samplechap2>
%\fi
%
%\iffalse
%<*samplechap1>
%\fi
% Some text for chapter 1:
%    \begin{macrocode}
\section{one}
some text in chapter one
%    \end{macrocode}

%\iffalse
%</samplechap1>
%\fi
% Some text for chapter 2:
%\iffalse
%<*samplechap2>
%\fi
%    \begin{macrocode}
\section{two}
more text in chapter two
%    \end{macrocode}

%\iffalse
%</samplechap2>
%\fi
%
% %%%%%%%%%%%%%%%%%%%%%%%%%%%%%%%%%%%%%%
% \paragraph{Part Include Files.}
%
% The include files are called |cdocspt3.tex| and |cdocspt4.tex|.
%
%\iffalse
%<*samplepart3|samplepart4>
%\fi

% Optional override for |\version| flag:
%    \begin{macrocode}
%%\providecommand{\version}{final}
%    \end{macrocode}

% Include the main document:
%    \begin{macrocode}
\input{childdoc.def}
\childdocby{cdocsamp}
%    \end{macrocode}

%\iffalse
%</samplepart3|samplepart4>
%\fi
%
%\iffalse
%<*samplepart3>
%\fi
% Some text for part 3:
%    \begin{macrocode}
some text in part three
%    \end{macrocode}

%\iffalse
%</samplepart3>
%\fi
% Some text for part 4:
%\iffalse
%<*samplepart4>
%\fi
%    \begin{macrocode}
more text in part four
%    \end{macrocode}

%\iffalse
%</samplepart4>
%\fi
%
% %%%%%%%%%%%%%%%%%%%%%%%%%%%%%%%%%%%%%%
% \paragraph{Forwarding for a Complete Draft.}
%
% The following forwarding file |cdocsdrf.tex|
% compiles the main document in draft mode:
%\iffalse
%<*sampledraft>
%\fi
%    \begin{macrocode}
\def\version{draft}
\input{childdoc.def}
\childdocforward{cdocsamp}
%    \end{macrocode}

%\iffalse
%</sampledraft>
%\fi
%
% %%%%%%%%%%%%%%%%%%%%%%%%%%%%%%%%%%%%%%
% \paragraph{Forwarding for Final Version of the Chapters.}
%
% The following forwarding files |cdocsfn1.tex| and |cdocsfn2.tex|
% (with identical content)
% compile the final versions of the child documents
% |cdocsch1.tex| and |cdocsch2.tex|, respectively:
%\iffalse
%<*samplefinal>
%\fi
%    \begin{macrocode}
\def\version{final}
\input{childdoc.def}
\childdocforwardprefix[cdocsamp]{cdocsfn}{cdocsch}
%    \end{macrocode}

%\iffalse
%</samplefinal>
%\fi
%
% %%%%%%%%%%%%%%%%%%%%%%%%%%%%%%%%%%%%%%
% \paragraph{Command Line Processing.}
%
% The following three command lines generate the output files
% |cdocscld|, |cdocscl1| and |cdocscl2|
% which should be identical to
% |cdocsdrf|, |cdocsch1| and |cdocsfn2|, respectively:
% \begin{center}
% \begin{tabular}{l}
% |latex -jobname cdocscld \|\\
% |  "\def\version{draft}\input{childdoc.def}\childdocforward{cdocsamp}"|\\
% |latex -jobname cdocscl1 \|\\
% |  "\input{childdoc.def}\childdocforward[cdocsamp]{cdocsch1}"|\\
% |latex -jobname cdocscl2 \|\\
% |  "\def\version{final}\input{childdoc.def}\childdocforward{cdocsch2}"|
% \end{tabular}
% \end{center}
% Note that the trailing backslash on each first line
% merely continues the input to the second line
% (for convenient cut ant paste).
% Furthermore, the command |latex| can be replaced by any
% of its alternative versions such as |pdflatex|.
%
% %%%%%%%%%%%%%%%%%%%%%%%%%%%%%%%%%%%%%%%%%%%%%%%%%%%%%%%%%%%%%%%%%%%%%%%%%%%%%%
% %%%%%%%%%%%%%%%%%%%%%%%%%%%%%%%%%%%%%%%%%%%%%%%%%%%%%%%%%%%%%%%%%%%%%%%%%%%%%%
% \section{Implementation}
%\iffalse
%<*package>
%\fi
%
% This section describes the definitions file |childdoc.def|.

% The definitions cannot be loaded using |\usepackage| or |\RequirePackage|
% which has a mechanism to prevent loading a style file more than once.
% When loading the definitions by means of |\input|
% multiple instances have to be prevented manually:
%\iffalse
%This code needs to be before the `\ProvidesFile' directive
%which is defined at the beginning of this file.
%Therefore it is also placed there and commented out here.
%</package>
%<*discard>
%\fi
%    \begin{macrocode}
\ifdefined\childdocmain\endinput\fi
%    \end{macrocode}
%\iffalse
%</discard>
%<*package>
%\fi
%
% \macro{\ifchilddoc}
% \macro{\ifchilddocmanual}
% The conditional |\ifchilddoc| tells whether a
% child (true) or main (false) document is being compiled.
% The conditional |\ifchilddocmanual| tells whether
% the |\includeonly| mechanism is used (false) or
% the selection of child files must be performed manually (true).
% The definitions initialise to false:
%    \begin{macrocode}
\newif\ifchilddoc
\newif\ifchilddocmanual
%    \end{macrocode}

% \macro{\childdocname}
% \macro{\childdocjob}
% The macro |\childdocname| stores the name of the main document
% to be compiled. The macro |\childdocjob| stores the name of
% the document on which the \LaTeX{} compiler was originally invoked.
% The content of |\jobname| cannot be compared
% to filenames specified in the source due to different catcodes.
% The following code rescans |\jobname|, stores the result
% in |\childdocname| and saves a copy in |\childdocjob|:
%    \begin{macrocode}
\edef\childdocname{\scantokens\expandafter{\jobname\noexpand}}
\let\childdocjob\childdocname
%    \end{macrocode}

% \macro{\childdocdisable}
% The macro |\childdocdisable| prevents the main file
% from being processed more than once.
% At this stage, the main document command |\childdocmain|
% is assumed to be called once again where it should do nothing.
% Any subsequent call to it should prevent
% a secondary processing of the main document
% It overwrites the forwarding commands
% |\childdocof| and |\childdocforward|
% with empty macros to prevent further inclusions of the main document:
%    \begin{macrocode}
\newcommand{\childdocdisable}
{
  \renewcommand{\childdocmain}[1]{\renewcommand{\childdocmain}[1]{\endinput}}
  \renewcommand{\childdocof}[1]{}
  \renewcommand{\childdocby}[2][]{}
  \renewcommand{\childdocforward}[2][]{}
  \renewcommand{\childdocdisable}{}
}
%    \end{macrocode}

% \macro{\childdocmain}
% The macro |\childdocmain| is to be called at the top of the main file
% with nothing or the main filename (without extension) as argument.
% First, it breaks loops.
% If the argument is not empty and does not match |\childdocname|
% (which is set by the first inclusion of |childdoc.def|),
% |\ifchilddoc| is set to true, |\includeonly| is applied to the child file
% and |\jobname| is set to the main file
% (for proper handling of |.aux| files):
%    \begin{macrocode}
\newcommand{\childdocmain}[1]
{
  \childdocdisable\childdocmain{}
  \if?#1?\else
    \begingroup
      \def\childdoctmp{#1}
      \ifx\childdoctmp\childdocname
        \def\childdoctmp{}
      \else
        \def\childdoctmp
        {
          \childdoctrue
          \includeonly{\childdocname}
          \def\childdocjob{#1}
          \def\jobname{#1}
        }
      \fi
      \expandafter
    \endgroup
    \childdoctmp
  \fi
}
%    \end{macrocode}

% \macro{\childdocof}
% The command |\childdocof| redirects
% compilation to the main file |#1|.
%    \begin{macrocode}
\newcommand{\childdocof}[1]
{
  \childdocdisable
  \childdoctrue
  \includeonly{\childdocname}
  \def\jobname{#1}
  \def\childdocjob{#1}
  \input{#1}
}
%    \end{macrocode}

% \macro{\childdocby}
% The command |\childdocby| ....
%    \begin{macrocode}
\newcommand{\childdocby}[2][]
{
  \childdocdisable
  \childdoctrue
  \childdocmanualtrue
  \if?#1?\else
    \def\jobname{#2}
  \fi
  \def\childdocjob{#2}
  \input{#2}
  \endinput
}
%    \end{macrocode}

% \macro{\childdocforward}
% The command |\childdocforward| redirects
% compilation to the main file or
% (if the optional argument is given) a child file.
% Parameters are set as if the main file
% or a child file starting with |\childdocof| was compiled.
% Then compilation is handed over to the main file:
%    \begin{macrocode}
\newcommand{\childdocforward}[2][]
{
  \begingroup
    \if?#1?
      \def\childdoctmp
      {
        \def\childdocname{#2}
        \def\childdocjob{#2}
        \def\jobname{#2}
        \input{#2}
        \endinput
      }
    \else
      \def\childdoctmp
      {
        \childdocdisable
        \def\childdocname{#2}
        \childdoctrue
        \includeonly{#2}
        \def\childdocjob{#1}
        \def\jobname{#1}
        \input{#1}
        \endinput
      }
    \fi
    \expandafter
  \endgroup
  \childdoctmp
}
%    \end{macrocode}

% \macro{\childdocforwardprefix}
% The command |\childdocforwardprefix| redirects
% compilation to the main or a child file by means of a pattern.
% The prefix |#1| in the current filename is replaced by |#2|
% and the suffix of the current filename is kept
% (it is assumed that the filename does not contain the substring `|~~~|'
% which is used as a delimiter).
% Compilation is handed over to the new file by |\childdocforward|:
%    \begin{macrocode}
\newcommand{\childdocforwardprefix}[3][]
{
  \begingroup
    \def\childdocextract #2##1~~~{\def\childdoctmp{\childdocforward[#1]{#3##1}}}
    \expandafter\childdocextract\childdocname~~~
    \expandafter
  \endgroup
  \childdoctmp
}
%    \end{macrocode}

% \macro{\childdoc}
% The deprecated macro |\childdoc| is a legacy version of |\childdocmain|:
%    \begin{macrocode}
\newcommand{\childdoc}{\childdocmain}
%    \end{macrocode}

% \macro{\childdocredirect}
% The deprecated macro |\childdocredirect| is a legacy version
% of |\childdocforward| and |\childdocforwardprefix|:
%    \begin{macrocode}
\newcommand{\childdocredirect}[2][]
{
  \begingroup
    \if?#1?
      \def\childdoctmp{\childdocforward{#2}}
    \else
      \def\childdoctmp{\childdocforwardprefix{#1}{#2}}
    \fi
    \expandafter
  \endgroup
  \childdoctmp
}
%    \end{macrocode}

%\iffalse
%</package>
%\fi
%
\endinput
|
and perform the replacements as outlined below.
Instead of |\childdocmain{|\textit{main}|}| add the following code
to the top of the main file:
%
\begin{center}
\begin{tabular}{l}
|\||ifdefined\childdocname\endinput\||fi\newif\ifchilddoc|\\
|\edef\childdocname{\scantokens\expandafter{\jobname\noexpand}}|\\
|\def\childdocmain{|\textit{main}|}\||ifx\childdocmain\childdocname\||else|\\
|\childdoctrue\includeonly{\childdocname}\let\jobname\childdocmain\||fi|\\
\end{tabular}
\end{center}
%
Instead of |\childdocof{|\textit{main}|}| just include the main file
at the top of each child file:
%
\begin{center}
|\input{|\textit{main}|}|
\end{center}
%
A simple redirection |\childdocforward{|\textit{dest}|}| is achieved by:
%
\begin{center}
|\def\jobname{|\textit{dest}|}\input{\jobname}|
\end{center}
%
The redirection with prefix
|\childdocforwardprefix[|\textit{prefix}|]{|\textit{dest}|}|
is accomplished by:
%
\begin{center}
\begin{tabular}{l}
|{\edef\jobname{\scantokens\expandafter{\jobname\noexpand}}|\\
|\def\redirectjob |\textit{prefix}|#1~~~{\gdef\jobname{|\textit{dest}|#1}}|\\
|\expandafter\redirectjob\jobname~~~}\input{\jobname}|
\end{tabular}
\end{center}

In an alternative approach,
child documents can be compiled by a specific command line
without additional code or specific definitions:
%
\begin{center}
|... -jobname "|\textit{target}|" "|[\textit{flags}]%
|\includeonly{|\textit{dest}|}\input{|\textit{main}|}"|
\end{center}
%

%%%%%%%%%%%%%%%%%%%%%%%%%%%%%%%%%%%%%%%%%%%%%%%%%%%%%%%%%%%%%%%%%%%%%%%%%%%%%%%%
%%%%%%%%%%%%%%%%%%%%%%%%%%%%%%%%%%%%%%%%%%%%%%%%%%%%%%%%%%%%%%%%%%%%%%%%%%%%%%%%
\section{Information}

%%%%%%%%%%%%%%%%%%%%%%%%%%%%%%%%%%%%%%%%%%%%%%%%%%%%%%%%%%%%%%%%%%%%%%%%%%%%%%%%
\subsection{Copyright}

Copyright \copyright{} 2017--2018 Niklas Beisert

This work may be distributed and/or modified under the
conditions of the \LaTeX{} Project Public License, either version 1.3
of this license or (at your option) any later version.
The latest version of this license is in
  \url{http://www.latex-project.org/lppl.txt}
and version 1.3 or later is part of all distributions of \LaTeX{}
version 2005/12/01 or later.

This work has the LPPL maintenance status `maintained'.

The Current Maintainer of this work is Niklas Beisert.

This work consists of the files |README.txt|, |childdoc.ins| and |childdoc.dtx|
as well as the derived files |childdoc.def|, |cdocsamp.tex|
with |cdocsch1.tex|, |cdocsch2.tex|, |cdocspt3.tex|, |cdocspt4.tex|,
|cdocsdrf.tex|, |cdocsfn1.tex|, |cdocsfn2.tex|
as well as |childdoc.pdf|.

%%%%%%%%%%%%%%%%%%%%%%%%%%%%%%%%%%%%%%%%%%%%%%%%%%%%%%%%%%%%%%%%%%%%%%%%%%%%%%%%
\subsection{Files and Installation}

The package consists of the files:
%
\begin{center}
\begin{tabular}{ll}
    |README.txt|   & readme file \\
    |childdoc.ins| & installation file \\
    |childdoc.dtx| & source file \\
    |childdoc.def| & definition file \\
    |cdocsamp.tex| & sample main file \\
    |cdocsch1.tex| & sample include file \\
    |cdocsch2.tex| & sample include file \\
    |cdocspt3.tex| & sample part file \\
    |cdocspt4.tex| & sample part file \\
    |cdocsdrf.tex| & sample redirection file \\
    |cdocsfn1.tex| & sample redirection file \\
    |cdocsfn2.tex| & sample redirection file \\
    |childdoc.pdf| & manual
\end{tabular}
\end{center}
%
The distribution consists of the files
|README.txt|, |childdoc.ins| and |childdoc.dtx|.
%
\begin{itemize}
\item
Run (pdf)\LaTeX{} on |childdoc.dtx|
to compile the manual |childdoc.pdf| (this file).
\item
Run \LaTeX{} on |childdoc.ins| to create the definitions file |childdoc.def|
and the sample |cdocsamp.tex| with include files
|cdocsch1.tex|, |cdocsch2.tex|, |cdocspt3.tex|, |cdocspt4.tex|,
|cdocsdrf.tex|, |cdocsfn1.tex|, |cdocsfn2.tex|.
Then copy the file |childdoc.def| to an appropriate directory of your \LaTeX{}
distribution, e.g.\ \textit{texmf-root}|/tex/latex/childdoc|.
\end{itemize}

%%%%%%%%%%%%%%%%%%%%%%%%%%%%%%%%%%%%%%%%%%%%%%%%%%%%%%%%%%%%%%%%%%%%%%%%%%%%%%%%
\subsection{Related CTAN Packages}

There are several other packages which offer a similar functionality:
%
\begin{itemize}
\item
The packages
\href{http://ctan.org/pkg/docmute}{\textsf{docmute}},
\href{http://ctan.org/pkg/includex}{\textsf{includex}} and
\href{http://ctan.org/pkg/standalone}{\textsf{standalone}}
provide commands to include only the document body of
a child file thus allowing both files to be compiled individually.
\item
The packages \href{http://ctan.org/pkg/subdocs}{\textsf{subdocs}}
and \href{http://ctan.org/pkg/subfiles}{\textsf{subfiles}}
provide structures in which the main and child documents can be
encapsulated and allowing them to be compiled individually.
The inclusion mechanism is different from the conventional |\include|.
\item
The package \href{http://ctan.org/pkg/combine}{\textsf{combine}}
is an elaborate solution to combine several documents into one.
\end{itemize}
%
See also the CTAN topic \href{http://ctan.org/topic/subdocs}{\textsf{subdocs}}
for further related packages.
The present package differs from the above solutions in that
a document structure constructed with the conventional |\include| mechanism
just needs two extra commands at the top of every file
such that all constituent files can be compiled individually.

%%%%%%%%%%%%%%%%%%%%%%%%%%%%%%%%%%%%%%%%%%%%%%%%%%%%%%%%%%%%%%%%%%%%%%%%%%%%%%%%
%\subsection{Feature Suggestions}
%
%The following is a list of features which may be useful for future
%versions of this package:
%%
%\begin{itemize}
%\item
%\ldots
%\end{itemize}

%%%%%%%%%%%%%%%%%%%%%%%%%%%%%%%%%%%%%%%%%%%%%%%%%%%%%%%%%%%%%%%%%%%%%%%%%%%%%%%%
\subsection{Revision History}

%%%%%%%%%%%%%%%%%%%%%%%%%%%%%%%%%%%%%%%%
\paragraph{v2.0:} 2018/12/30

\begin{itemize}
\item
immediate forward processing
\item
added |\childdocby| mechanism
\item
manual restructured
\end{itemize}

%%%%%%%%%%%%%%%%%%%%%%%%%%%%%%%%%%%%%%%%
\paragraph{v1.6:} 2018/01/17

\begin{itemize}
\item
application for development of include files
\item
corrections to manual
\end{itemize}

%%%%%%%%%%%%%%%%%%%%%%%%%%%%%%%%%%%%%%%%
\paragraph{v1.5:} 2017/05/21

\begin{itemize}
\item
more complete structuring introduced
\item
|\childdocof| introduced
\item
|\childdoc| renamed to |\childdocmain|
\item
|\childredirect| renamed to |\childdocforward| and |\childdocforwardprefix|
and functionality expanded
\end{itemize}

%%%%%%%%%%%%%%%%%%%%%%%%%%%%%%%%%%%%%%%%
\paragraph{v1.0:} 2017/04/27

\begin{itemize}
\item
manual and install package
\item
first version published on CTAN
\end{itemize}

%%%%%%%%%%%%%%%%%%%%%%%%%%%%%%%%%%%%%%%%
\paragraph{v0.6:} 2017/04/26

\begin{itemize}
\item
redirection mechanism added
\end{itemize}

%%%%%%%%%%%%%%%%%%%%%%%%%%%%%%%%%%%%%%%%
\paragraph{v0.5:} 2017/04/26

\begin{itemize}
\item
functionality in definition file
\end{itemize}


%%%%%%%%%%%%%%%%%%%%%%%%%%%%%%%%%%%%%%%%%%%%%%%%%%%%%%%%%%%%%%%%%%%%%%%%%%%%%%%%
%%%%%%%%%%%%%%%%%%%%%%%%%%%%%%%%%%%%%%%%%%%%%%%%%%%%%%%%%%%%%%%%%%%%%%%%%%%%%%%%
%%%%%%%%%%%%%%%%%%%%%%%%%%%%%%%%%%%%%%%%%%%%%%%%%%%%%%%%%%%%%%%%%%%%%%%%%%%%%%%%
\appendix

\settowidth\MacroIndent{\rmfamily\scriptsize 000\ }

 \DocInput{childdoc.dtx}

\end{document}
%</driver>
% \fi
%
% %%%%%%%%%%%%%%%%%%%%%%%%%%%%%%%%%%%%%%%%%%%%%%%%%%%%%%%%%%%%%%%%%%%%%%%%%%%%%%
% %%%%%%%%%%%%%%%%%%%%%%%%%%%%%%%%%%%%%%%%%%%%%%%%%%%%%%%%%%%%%%%%%%%%%%%%%%%%%%
% \section{Sample}
%\iffalse
%<*samplemain>
%\fi
%
% The following presents a sample document
% with two chapters, two parts, a title page,
% a compile flag as well as three forwarding files to set the flag.
% It consists of eight |.tex| files:
% \begin{center}
% \begin{tabular}{ll}
% |cdocsamp.tex|&main file\\
% |cdocsch1.tex|&include file for chapter 1\\
% |cdocsch2.tex|&include file for chapter 2\\
% |cdocspt3.tex|&include file for part 3\\
% |cdocspt4.tex|&include file for part 4\\
% |cdocsdrf.tex|&forwarding file for main file in draft mode\\
% |cdocsfi1.tex|&forwarding file for final version of chapter 1\\
% |cdocsfi2.tex|&forwarding file for final version of chapter 2\\
% \end{tabular}
% \end{center}
% Each of the eight files can be compiled directly by the \LaTeX{} compiler.
%
% %%%%%%%%%%%%%%%%%%%%%%%%%%%%%%%%%%%%%%
% \paragraph{Main File.}
%
% The main file is called |cdocsamp.tex|.
%
% Load the \textsf{childdoc} definitions and
% declare the filename for the main document:
%    \begin{macrocode}
% \iffalse
%
% childdoc.dtx Copyright (C) 2017-2018 Niklas Beisert
%
% This work may be distributed and/or modified under the
% conditions of the LaTeX Project Public License, either version 1.3
% of this license or (at your option) any later version.
% The latest version of this license is in
%   http://www.latex-project.org/lppl.txt
% and version 1.3 or later is part of all distributions of LaTeX
% version 2005/12/01 or later.
%
% This work has the LPPL maintenance status `maintained'.
%
% The Current Maintainer of this work is Niklas Beisert.
%
% This work consists of the files childdoc.dtx and childdoc.ins
% and the derived files childdoc.def and cdocsamp.tex with
% cdocsch1.tex, cdocsch2.tex, cdocsdrf.tex, cdocsfn1.tex, cdocsfn2.tex.
%
%<package>\ifdefined\childdocmain\endinput\fi
%<package>\ProvidesFile{childdoc.def}[2018/12/30 v2.0 child document driver]
%<samplemain>\ProvidesFile{cdocsamp.tex}[2018/12/30 v2.0 sample for childdoc]
%<*driver>
%\ProvidesFile{childdoc.drv}[2018/12/30 v2.0 childdoc reference manual file]
\PassOptionsToClass{10pt,a4paper}{article}
\documentclass{ltxdoc}

\usepackage[margin=35mm]{geometry}
\usepackage{hyperref}
\usepackage{hyperxmp}
\usepackage[usenames]{color}

\hypersetup{colorlinks=true}
\hypersetup{pdfstartview=FitH}
\hypersetup{pdfpagemode=UseNone}
\hypersetup{pdfsource={}}
\hypersetup{pdflang={en-UK}}
\hypersetup{pdfcopyright={Copyright 2017-2018 Niklas Beisert.
  This work may be distributed and/or modified under the
  conditions of the LaTeX Project Public License, either version 1.3
  of this license or (at your option) any later version.}}
\hypersetup{pdflicenseurl={http://www.latex-project.org/lppl.txt}}
\hypersetup{pdfcontactaddress={ETH Zurich, ITP, HIT K,
  Wolfgang-Pauli-Strasse 27}}
\hypersetup{pdfcontactpostcode={8093}}
\hypersetup{pdfcontactcity={Zurich}}
\hypersetup{pdfcontactcountry={Switzerland}}
\hypersetup{pdfcontactemail={nbeisert@itp.phys.ethz.ch}}
\hypersetup{pdfcontacturl={http://people.phys.ethz.ch/\xmptilde nbeisert/}}

\newcommand{\secref}[1]{\hyperref[#1]{section \ref*{#1}}}

\parskip1ex
\parindent0pt
\let\olditemize\itemize
\def\itemize{\olditemize\parskip0pt}

\begin{document}

\title{The \textsf{childdoc} Package}
\hypersetup{pdftitle={The childdoc Package}}
\author{Niklas Beisert\\[2ex]
  Institut f\"ur Theoretische Physik\\
  Eidgen\"ossische Technische Hochschule Z\"urich\\
  Wolfgang-Pauli-Strasse 27, 8093 Z\"urich, Switzerland\\[1ex]
  \href{mailto:nbeisert@itp.phys.ethz.ch}
  {\texttt{nbeisert@itp.phys.ethz.ch}}}
\hypersetup{pdfauthor={Niklas Beisert}}
\hypersetup{pdfsubject={Manual for the LaTeX2e Package childdoc}}
\date{30 December 2018, \textsf{v2.0}}
\maketitle

\begin{abstract}\noindent
\textsf{childdoc} is a \LaTeXe{} package
that enables the direct compilation
of document sections included by |\include|
to individual files.
\end{abstract}

\begingroup
\parskip0ex
\tableofcontents
\endgroup

%%%%%%%%%%%%%%%%%%%%%%%%%%%%%%%%%%%%%%%%%%%%%%%%%%%%%%%%%%%%%%%%%%%%%%%%%%%%%%%%
%%%%%%%%%%%%%%%%%%%%%%%%%%%%%%%%%%%%%%%%%%%%%%%%%%%%%%%%%%%%%%%%%%%%%%%%%%%%%%%%
\section{Introduction}

\LaTeX{} provides a mechanism to structure a large document (such as a book)
into a main file and several child files (containing the chapters)
using the |\include| command.
This mechanism is beneficial for documents
which span hundreds of pages in order to
make the source file(s) more manageable.
Moreover, compilation can be restricted to
selected child files by means of the |\includeonly| command.
The latter feature can be used to reduce the compilation time while editing
(this was significantly more useful in the earlier days of \LaTeX{})
or to generate a smaller document which is easier to navigate.
Another application of |\includeonly| is to generate
documents consisting of selected parts of the complete document.

However, there are a few drawbacks of the plain |\include| mechanism:
\begin{itemize}
\item
The child files cannot be compiled on their own,
they can only be compiled via the main file.
A naive editing environment
(such as a text editor with an option
to have the current file processed by \LaTeX)
may require one to switch to the main file before compiling;
attempting to compile the child file produces errors.
\item
The main file must be modified (each time)
to adjust the |\includeonly| command
to the present needs. This easily leaves the main file in a messy state.
\item
The generated document will always carry the filename
of the main document. This is inconvenient if
several child files are to be compiled and
to be kept for distribution.
\end{itemize}

The present package provides a simple interface
to make child files individually compilable by \LaTeX{}.
Compiling a child file then has the same effect as compiling
the main file with an |\includeonly| command
to select the appropriate child.
Moreover the generated document will carry the name of the child
rather than the main file.
This resolves all three above issues.

This feature is meant to make the editing of books,
thesis documents and lecture notes somewhat more convenient.
However, the package can also be used efficiently for
composing a series of documents (such as exercise sheets)
which are typically distributed individually.
It then assists the author in generating the individual documents
(potentially in different versions)
as well as a document containing the collected series.
Another application is in developing style files
or other kinds of included material
where compilation of the style file could redirect
to a sample or test file.

%%%%%%%%%%%%%%%%%%%%%%%%%%%%%%%%%%%%%%%%%%%%%%%%%%%%%%%%%%%%%%%%%%%%%%%%%%%%%%%%
%%%%%%%%%%%%%%%%%%%%%%%%%%%%%%%%%%%%%%%%%%%%%%%%%%%%%%%%%%%%%%%%%%%%%%%%%%%%%%%%
\section{Usage}

First of all, the package \textsf{childdoc} is \emph{not} a standard
\LaTeXe{} |.sty| style file! Therefore it needs to be invoked in
a non-standard way.

%%%%%%%%%%%%%%%%%%%%%%%%%%%%%%%%%%%%%%%%%%%%%%%%%%%%%%%%%%%%%%%%%%%%%%%%%%%%%%%%
\subsection{Included Files}
\label{sec:include}

%%%%%%%%%%%%%%%%%%%%%%%%%%%%%%%%%%%%%%%%
\DescribeMacro{\childdocmain}
To use the package, add the commands
\begin{center}
\begin{tabular}{l}
|\input{childdoc.def}|\\
|\childdocmain{}|\\
\end{tabular}
\end{center}
at the very top of the main \LaTeX{} file,
in particular \emph{before} the |\documentclass| statement!
The argument of |\childdocmain| should be left empty
(but it must be present).

%%%%%%%%%%%%%%%%%%%%%%%%%%%%%%%%%%%%%%%%
\DescribeMacro{\childdocof}
Furthermore, add the commands
\begin{center}
\begin{tabular}{l}
|\input{childdoc.def}|\\
|\childdocof{|\textit{main}|}|\\
\end{tabular}
\end{center}
at the top of every child file \textit{child}
which is included by |\include{|\textit{child}|}|
from within the main file
(or at least for those files to be compiled individually).
The argument \textit{main} must be the filename of the main file.

There are a couple of
considerations in setting up the main and child documents:

%%%%%%%%%%%%%%%%%%%%%%%%%%%%%%%%%%%%%%%%
\paragraph{Restrictions.}

Please note the following restrictions:
\begin{itemize}
\item
|\childdocmain| must be called with one argument \textit{main}
to ensure compatibility with earlier version of the package.
It must either be empty (|\childdocmain{}|)
or precisely match the filename of the main file in which it is specified.
See \secref{sec:detection} for further information.
\item
The filename \textit{main} must be specified without the |.tex| extension.
\item
The filename \textit{main} is case sensitive
(even in case-insensitive file systems)
due to internal string comparison.
\item
The argument \textit{main} should be fully expanded, it cannot be a macro.
\item
Subdirectories and special characters should be avoided in filenames.
\item
The command |\childdocmain{|\textit{main}|}| must be followed by a whitespace.
It should not be followed immediately by another command
or by a comment mark `|%|'.
This is because the \TeX{} parser reads the token immediately following
the argument of |\childdocmain| and puts it
at the beginning of every child section;
however, a white\-space is ignored.
\end{itemize}

%%%%%%%%%%%%%%%%%%%%%%%%%%%%%%%%%%%%%%%%
\paragraph{Content of Main File.}

It is advisable to place all content in the child files included by |\include|.
Any output contained in the main file will appear in all child documents
unless suppressed manually;
it cannot be suppressed automatically by the |\includeonly| directive
and thus should normally be avoided.
A method to include some content in the main file
by means of conditional processing is described in \secref{sec:conditional}.

%%%%%%%%%%%%%%%%%%%%%%%%%%%%%%%%%%%%%%%%
\paragraph{Page Numbering.}

When only a part of the document is compiled,
the appropriate numbering of pages
(as well as other status parameters)
is determined from the |.aux| files.
The latter contain information from previous passes.
However this information needs to propagate through
all intermediate child documents.
Therefore the page numbering in child documents may well
be inconsistent until the complete document is compiled at least once.

A useful (if unconventional) way to always ensure a consistent
page numbering is to restart the numbering in each child document
and denote the pages by `\textit{child}|.|\textit{page}'
where \textit{child} represents the chapter/section number of the child file.
This can be achieved by the command
|\numberwithin{page}{|\textit{child}|}|
of the \textsf{amsmath} package
where \textit{child} can be |chapter| or |section|
depending on the chosen structuring.
Alternatively, one can modify the macro |\thepage| appropriately
and reset the counter |page| at the start of each child file.

%%%%%%%%%%%%%%%%%%%%%%%%%%%%%%%%%%%%%%%%%%%%%%%%%%%%%%%%%%%%%%%%%%%%%%%%%%%%%%%%
\subsection{Conditional Processing}
\label{sec:conditional}

The package provides a mechanism to compile different versions
of a document. To customise the versions further some conditional processing
can come in handy to distinguish which version is being compiled.
The package provides two macros to describe the compilation context:

%%%%%%%%%%%%%%%%%%%%%%%%%%%%%%%%%%%%%%%%
\DescribeMacro{\ifchilddoc}
The conditional |\ifchilddoc| distinguishes between the compilation of
child documents and the main document:
%
\begin{center}
|\ifchilddoc |\textit{child-code}| |[|\||else |\textit{main-code}]| \||fi|
\end{center}

%%%%%%%%%%%%%%%%%%%%%%%%%%%%%%%%%%%%%%%%
\DescribeMacro{\childdocname}
\DescribeMacro{\childdocjob}
The macro |\childdocname| contains the filename (without extension)
of the main or child file being processed.
Note that |\childdocjob| will always contain the name of the main file.

%%%%%%%%%%%%%%%%%%%%%%%%%%%%%%%%%%%%%%%%
\paragraph{Title Page.}

Conditional processing can be used to include a title or banner page
in the main document when proper precautions are taken.
Importantly, the code in the main file should ensure that the page counter
(as well as other status parameters which are stored in the |.aux| files)
takes the same value after the conditional processing.
Otherwise the page numbers may take divergent values
depending on which part is compiled.

For example, a title page could be declared by:
%
\begin{center}
\begin{tabular}{l}
|\ifchilddoc\||else|\\
|\addtocounter{page}{-1}|\\
\textit{code for title page}\\
|\newpage|\\
|\||fi|
\end{tabular}
\end{center}
%
A banner page for the child documents can be generated by:
%
\begin{center}
\begin{tabular}{l}
|\ifchilddoc|\\
|\addtocounter{page}{-1}|\\
\textit{code for banner page}\\
|\newpage|\\
|\||fi|
\end{tabular}
\end{center}
%
Here one could write a message such as:
\begin{center}
|This is the part \childdocname{} of \childdocjob{}.|
\end{center}

%%%%%%%%%%%%%%%%%%%%%%%%%%%%%%%%%%%%%%%%%%%%%%%%%%%%%%%%%%%%%%%%%%%%%%%%%%%%%%%%
\subsection{Flags}
\label{sec:flags}

The package makes it easy to generate different versions
of the main or child documents.
To this end compilation flags can be defined
and assigned different default values.
They will be particularly useful in conjunction
with the forwarding mechanism described in \secref{sec:forward}.

For example, it may be useful to have a flag |\version|
which can be set to |draft| or |final|.
The document source will contain some conditional code
depending on the value of |\version|.
Suppose further, the flag should default to |final| for the main file
and to |draft| for child files
which is a natural assignment for editing the document.
This is achieved by placing the following code
in the preamble of the main document
(below the |\childdocmain| directive):
%
\begin{center}
\begin{tabular}{l}
|\ifchilddoc|\\
|\providecommand{\version}{draft}|\\
|\||else|\\
|\providecommand{\version}{final}|\\
|\||fi|
\end{tabular}
\end{center}
%
The definition by |\providecommand| makes sure
that previous definitions are not overwritten.
Further statements |\providecommand{\version}{...}|
can thus be added before the above code to override it.

For the main file, one might add a line
(between |\childdocmain| and the above block)
%
\begin{center}
|%\ifchilddoc\||else\providecommand{\version}{draft}\||fi|
\end{center}
%
which can be uncommented to produce a draft version.
Likewise one can add a line to the very top of a child file
(above the |\childdocof{|\textit{main}|}| directive)
%
\begin{center}
|%\providecommand{\version}{final}|
\end{center}
%
which can be uncommented to produce the final version of this child document.

%%%%%%%%%%%%%%%%%%%%%%%%%%%%%%%%%%%%%%%%%%%%%%%%%%%%%%%%%%%%%%%%%%%%%%%%%%%%%%%%
\subsection{Forwarding}
\label{sec:forward}

Different versions of the main or child documents
using compilation flags as described in \secref{sec:flags}
can be (permanently) stored in different files
for convenient compilation, viewing and distribution.
To this end, the package defines a command
to pass on compilation to a different file:

%%%%%%%%%%%%%%%%%%%%%%%%%%%%%%%%%%%%%%%%
\DescribeMacro{\childdocforward}
The command |\childdocforward| redirects processing to
another source file:
%
\begin{center}
\begin{tabular}{l}
|\input{childdoc.def}|\\
|\childdocforward[|\textit{main}|]{|\textit{dest}|}|\\
\end{tabular}
\end{center}
%
The argument \textit{dest} is the destination file
(without extension).
It should be the main file or one of the child files.
Note that further \textsf{childdoc} directives
such as |\childdocof| and |\childdocforward|
in the indicated file will be processed in this form.
The optional argument \textit{main}
passes on directly to the main file \textit{main}
while pretending to compile the child \textit{dest}.
This form behaves as if \textit{dest}
issues |\childdocof{|\textit{main}|}| right away,
and no further \textsf{childdoc} directives will be processed.

%%%%%%%%%%%%%%%%%%%%%%%%%%%%%%%%%%%%%%%%
\DescribeMacro{\...prefix}
In the alternative form |\childdocforwardprefix|,
%
\begin{center}
\begin{tabular}{l}
|\input{childdoc.def}|\\
|\childdocforwardprefix[|\textit{main}|]{|\textit{prefix}|}{|\textit{dest}|}|
\end{tabular}
\end{center}
%
the destination file is determined by a pattern
depending on the current file:
To make this work, the current file must be called
`{\textit{prefix}\hspace{0.2em}\textit{suffix}}'
with \textit{prefix} matching precisely the argument.
Processing is then passed on to the file
`{\textit{dest}\hspace{0.2em}\textit{suffix}}'.
Surely, the same effect is achieved by
directly specifying the
argument `{\textit{dest}\hspace{0.2em}\textit{suffix}}'
in the first form.
However, that requires to set up a different file
for each child. With the alternative form of the command
all these files can have exactly the same content
which simplifies setting them up and maintaining them.

For example, the following file |draft.tex|
with a compilation flag |\version| as described in \secref{sec:flags}
compiles the main document as a draft:
%
\begin{center}
\begin{tabular}{l}
|\def\version{draft}|\\
|\input{childdoc.def}|\\
|\childdocforward{|\textit{main}|}|
\end{tabular}
\end{center}
%
Likewise, the following files |final|\textit{nn}|.tex|
compile the final version of the child document
|child|\textit{nn}|.tex|:
%
\begin{center}
\begin{tabular}{l}
|\def\version{final}|\\
|\input{childdoc.def}|\\
|\childdocforwardprefix{final}{child}|
\end{tabular}
\end{center}
%

Note that when several versions of a main file and/or of each child file
are to be generated, it may be convenient to set up a |Makefile| or
shell script to automatise the process.

%%%%%%%%%%%%%%%%%%%%%%%%%%%%%%%%%%%%%%%%%%%%%%%%%%%%%%%%%%%%%%%%%%%%%%%%%%%%%%%%
\subsection{Command Line Processing}
\label{sec:commandline}

The effect of redirection files can also be achieved by invoking
the \LaTeX{} compiler with a more elaborate command line.
Most conveniently this should be done as part
of a shell script or a |Makefile|.

When using \textsf{childdoc} in the main file, the following
command lines effectively perform a redirection
(note that depending on the shell being used,
backslashes may have to be doubled: `|\|' $\to$ `|\\|'):
%
\begin{center}
|... -jobname "|\textit{target}|" |\\|"|[\textit{flags}]%
|\input{childdoc.def}\childdocforward[|\textit{main}|]{|\textit{dest}|}"|
\end{center}
%
Here \textit{target} is the name of the output file,
\textit{main} is the name of the main file
and \textit{dest} is the name of the main or child file to be processed
(all filenames without extensions).
The optional argument \textit{main} can be omitted
if \textit{main} matches \textit{dest}.
Optionally, compilation \textit{flags} can be defined via |\def| commands.
This command line makes the \TeX{} engine believe
it is compiling the file \textit{target}
whose content is specified as the latter parameter.
The provided code then forwards the processing to
\textit{main} or \textit{dest} as described in \secref{sec:forward}.

%%%%%%%%%%%%%%%%%%%%%%%%%%%%%%%%%%%%%%%%%%%%%%%%%%%%%%%%%%%%%%%%%%%%%%%%%%%%%%%%
\subsection{Include by Input}
\label{sec:input}

Including child documents by |\include| has some restrictions by design.
Most notably, the content of a child document always occupies
its own set of pages; pages cannot be shared between child documents.
Usually, this behaviour makes perfect sense
because each child document contain an essential part of the document.
However, in some situations it may be desirable to compose
a document from a collection of parts
without having mandatory page breaks between then.
For this case, the package
provides a mechanism to include parts
by |\input| which can also be processed individually.
However, by construction this mechanism
requires manual handling of the content to be output.

%%%%%%%%%%%%%%%%%%%%%%%%%%%%%%%%%%%%%%%%
\DescribeMacro{\ifchilddocmanual}
The main file should be prepared as usual, see \secref{sec:include}.
However, the document body must make a distinction
between processing of an individual part and of the main document, e.g.:
%
\begin{center}
\begin{tabular}{l}
|\ifchilddocmanual|\\
|\input{\childdocname}|\\
|\||else|\\
\textit{document body with }|\input{|\textit{part}|}|\\
|\||fi|
\end{tabular}
\end{center}
%
The conditional |\ifchilddocmanual| is true whenever
a part to be included by |\input| is being compiled,
and the name of the part is stored in |\childdocname|.

%%%%%%%%%%%%%%%%%%%%%%%%%%%%%%%%%%%%%%%%
\DescribeMacro{\childdocby}
Each part to be included by |\input| should start with:
%
\begin{center}
\begin{tabular}{l}
|\input{childdoc.def}|\\
|\childdocby{|\textit{main}|}|\\
\end{tabular}
\end{center}
%
The directive |\childdocby| is similar to |\childdocof|
described in \secref{sec:include},
but the subsequent selection of content must be done manually.
To that end, both |\ifchilddoc| and |\ifchilddocmanual|
will be true upon processing of a part,
and the name of the part is stored in |\childdocname|.
Note that |\jobname| will be set to the filename of the current part
so that each part receives an individual |.aux| file
that does not interfere with the |.aux| file(s) of the main document.
This behaviour can be altered by the alternative form
|\childdocby[*]{|\textit{main}|}| (with a non-empty optional argument)
which uses the |.aux| file of the main document
by setting |\jobname| to \textit{main}.

%%%%%%%%%%%%%%%%%%%%%%%%%%%%%%%%%%%%%%%%%%%%%%%%%%%%%%%%%%%%%%%%%%%%%%%%%%%%%%%%
\subsection{Driver Development}
\label{sec:driver}

The \textsf{childdoc} mechanism can also be use for the development
of definition files such as \LaTeX{} styles or classes.
This case differs from the above setup with multiple parts
included by |\include| in that no |\includeonly| should be invoked.
This can be achieved by starting the include file
(before |\ProvidesPackage|) with:
%
\begin{center}
\begin{tabular}{l}
|\input{childdoc.def}|\\
|\childdocforward{|\textit{main}|}|\\
\end{tabular}
\end{center}
%
or alternatively with:
%
\begin{center}
\begin{tabular}{l}
|\input{childdoc.def}|\\
|\childdocby{|\textit{main}|}|\\
\end{tabular}
\end{center}
%
Both forms have slightly different effects as described above.
The main file is prepared as usual, see \secref{sec:include}.

%%%%%%%%%%%%%%%%%%%%%%%%%%%%%%%%%%%%%%%%%%%%%%%%%%%%%%%%%%%%%%%%%%%%%%%%%%%%%%%%
\subsection{Legacy Detection}
\label{sec:detection}

The directive |\childdocmain| in the main file can detect
whether the complete document or merely a child is to be compiled
even without using the directive |\childdocof|.
This method is deprecated because it is less robust
and there is no compelling reason to use it;
it is merely provided for backward compatibility
and it may be removed in future versions.

If the detection mechanism is to be used,
it is mandatory to correctly specify
the filename of the main file as the argument of |\childdocmain|:
%
\begin{center}
\begin{tabular}{l}
|\input{childdoc.def}|\\
|\childdocmain{|\textit{main}|}|\\
\end{tabular}
\end{center}
%
If |\jobname| does not match the argument \textit{main} of |\childdocmain|,
it is assumed that |\jobname| points to the child file to be compiled.
When using |\childdocmain| with the main file specified as argument,
it suffices to start a child file
with just |\input{|\textit{main}|}|
without loading of the package and using |\childdocof|.
If instead all processing is done
with the appropriate \textsf{childdoc} directives,
the argument of \textit{main} of |\childdocmain| can be empty.

An alternative version of the command line processing described
in \secref{sec:commandline} using the detection mechanism reads:
%
\begin{center}
|... -jobname "|\textit{target}|" "|[\textit{flags}]%
[|\def\jobname{|\textit{dest}|}|]|\input{|\textit{main}|}"|
\end{center}

%%%%%%%%%%%%%%%%%%%%%%%%%%%%%%%%%%%%%%%%%%%%%%%%%%%%%%%%%%%%%%%%%%%%%%%%%%%%%%%%
\subsection{Manual Code}
\label{sec:manual}

In case one cannot be certain whether the definitions file |childdoc.def|
is installed on the target \TeX{} distribution
and one prefers not to ship it,
it is conceivable to paste a few relevant commands into the sources.

To that end, drop all statements |\input{childdoc.def}|
and perform the replacements as outlined below.
Instead of |\childdocmain{|\textit{main}|}| add the following code
to the top of the main file:
%
\begin{center}
\begin{tabular}{l}
|\||ifdefined\childdocname\endinput\||fi\newif\ifchilddoc|\\
|\edef\childdocname{\scantokens\expandafter{\jobname\noexpand}}|\\
|\def\childdocmain{|\textit{main}|}\||ifx\childdocmain\childdocname\||else|\\
|\childdoctrue\includeonly{\childdocname}\let\jobname\childdocmain\||fi|\\
\end{tabular}
\end{center}
%
Instead of |\childdocof{|\textit{main}|}| just include the main file
at the top of each child file:
%
\begin{center}
|\input{|\textit{main}|}|
\end{center}
%
A simple redirection |\childdocforward{|\textit{dest}|}| is achieved by:
%
\begin{center}
|\def\jobname{|\textit{dest}|}\input{\jobname}|
\end{center}
%
The redirection with prefix
|\childdocforwardprefix[|\textit{prefix}|]{|\textit{dest}|}|
is accomplished by:
%
\begin{center}
\begin{tabular}{l}
|{\edef\jobname{\scantokens\expandafter{\jobname\noexpand}}|\\
|\def\redirectjob |\textit{prefix}|#1~~~{\gdef\jobname{|\textit{dest}|#1}}|\\
|\expandafter\redirectjob\jobname~~~}\input{\jobname}|
\end{tabular}
\end{center}

In an alternative approach,
child documents can be compiled by a specific command line
without additional code or specific definitions:
%
\begin{center}
|... -jobname "|\textit{target}|" "|[\textit{flags}]%
|\includeonly{|\textit{dest}|}\input{|\textit{main}|}"|
\end{center}
%

%%%%%%%%%%%%%%%%%%%%%%%%%%%%%%%%%%%%%%%%%%%%%%%%%%%%%%%%%%%%%%%%%%%%%%%%%%%%%%%%
%%%%%%%%%%%%%%%%%%%%%%%%%%%%%%%%%%%%%%%%%%%%%%%%%%%%%%%%%%%%%%%%%%%%%%%%%%%%%%%%
\section{Information}

%%%%%%%%%%%%%%%%%%%%%%%%%%%%%%%%%%%%%%%%%%%%%%%%%%%%%%%%%%%%%%%%%%%%%%%%%%%%%%%%
\subsection{Copyright}

Copyright \copyright{} 2017--2018 Niklas Beisert

This work may be distributed and/or modified under the
conditions of the \LaTeX{} Project Public License, either version 1.3
of this license or (at your option) any later version.
The latest version of this license is in
  \url{http://www.latex-project.org/lppl.txt}
and version 1.3 or later is part of all distributions of \LaTeX{}
version 2005/12/01 or later.

This work has the LPPL maintenance status `maintained'.

The Current Maintainer of this work is Niklas Beisert.

This work consists of the files |README.txt|, |childdoc.ins| and |childdoc.dtx|
as well as the derived files |childdoc.def|, |cdocsamp.tex|
with |cdocsch1.tex|, |cdocsch2.tex|, |cdocspt3.tex|, |cdocspt4.tex|,
|cdocsdrf.tex|, |cdocsfn1.tex|, |cdocsfn2.tex|
as well as |childdoc.pdf|.

%%%%%%%%%%%%%%%%%%%%%%%%%%%%%%%%%%%%%%%%%%%%%%%%%%%%%%%%%%%%%%%%%%%%%%%%%%%%%%%%
\subsection{Files and Installation}

The package consists of the files:
%
\begin{center}
\begin{tabular}{ll}
    |README.txt|   & readme file \\
    |childdoc.ins| & installation file \\
    |childdoc.dtx| & source file \\
    |childdoc.def| & definition file \\
    |cdocsamp.tex| & sample main file \\
    |cdocsch1.tex| & sample include file \\
    |cdocsch2.tex| & sample include file \\
    |cdocspt3.tex| & sample part file \\
    |cdocspt4.tex| & sample part file \\
    |cdocsdrf.tex| & sample redirection file \\
    |cdocsfn1.tex| & sample redirection file \\
    |cdocsfn2.tex| & sample redirection file \\
    |childdoc.pdf| & manual
\end{tabular}
\end{center}
%
The distribution consists of the files
|README.txt|, |childdoc.ins| and |childdoc.dtx|.
%
\begin{itemize}
\item
Run (pdf)\LaTeX{} on |childdoc.dtx|
to compile the manual |childdoc.pdf| (this file).
\item
Run \LaTeX{} on |childdoc.ins| to create the definitions file |childdoc.def|
and the sample |cdocsamp.tex| with include files
|cdocsch1.tex|, |cdocsch2.tex|, |cdocspt3.tex|, |cdocspt4.tex|,
|cdocsdrf.tex|, |cdocsfn1.tex|, |cdocsfn2.tex|.
Then copy the file |childdoc.def| to an appropriate directory of your \LaTeX{}
distribution, e.g.\ \textit{texmf-root}|/tex/latex/childdoc|.
\end{itemize}

%%%%%%%%%%%%%%%%%%%%%%%%%%%%%%%%%%%%%%%%%%%%%%%%%%%%%%%%%%%%%%%%%%%%%%%%%%%%%%%%
\subsection{Related CTAN Packages}

There are several other packages which offer a similar functionality:
%
\begin{itemize}
\item
The packages
\href{http://ctan.org/pkg/docmute}{\textsf{docmute}},
\href{http://ctan.org/pkg/includex}{\textsf{includex}} and
\href{http://ctan.org/pkg/standalone}{\textsf{standalone}}
provide commands to include only the document body of
a child file thus allowing both files to be compiled individually.
\item
The packages \href{http://ctan.org/pkg/subdocs}{\textsf{subdocs}}
and \href{http://ctan.org/pkg/subfiles}{\textsf{subfiles}}
provide structures in which the main and child documents can be
encapsulated and allowing them to be compiled individually.
The inclusion mechanism is different from the conventional |\include|.
\item
The package \href{http://ctan.org/pkg/combine}{\textsf{combine}}
is an elaborate solution to combine several documents into one.
\end{itemize}
%
See also the CTAN topic \href{http://ctan.org/topic/subdocs}{\textsf{subdocs}}
for further related packages.
The present package differs from the above solutions in that
a document structure constructed with the conventional |\include| mechanism
just needs two extra commands at the top of every file
such that all constituent files can be compiled individually.

%%%%%%%%%%%%%%%%%%%%%%%%%%%%%%%%%%%%%%%%%%%%%%%%%%%%%%%%%%%%%%%%%%%%%%%%%%%%%%%%
%\subsection{Feature Suggestions}
%
%The following is a list of features which may be useful for future
%versions of this package:
%%
%\begin{itemize}
%\item
%\ldots
%\end{itemize}

%%%%%%%%%%%%%%%%%%%%%%%%%%%%%%%%%%%%%%%%%%%%%%%%%%%%%%%%%%%%%%%%%%%%%%%%%%%%%%%%
\subsection{Revision History}

%%%%%%%%%%%%%%%%%%%%%%%%%%%%%%%%%%%%%%%%
\paragraph{v2.0:} 2018/12/30

\begin{itemize}
\item
immediate forward processing
\item
added |\childdocby| mechanism
\item
manual restructured
\end{itemize}

%%%%%%%%%%%%%%%%%%%%%%%%%%%%%%%%%%%%%%%%
\paragraph{v1.6:} 2018/01/17

\begin{itemize}
\item
application for development of include files
\item
corrections to manual
\end{itemize}

%%%%%%%%%%%%%%%%%%%%%%%%%%%%%%%%%%%%%%%%
\paragraph{v1.5:} 2017/05/21

\begin{itemize}
\item
more complete structuring introduced
\item
|\childdocof| introduced
\item
|\childdoc| renamed to |\childdocmain|
\item
|\childredirect| renamed to |\childdocforward| and |\childdocforwardprefix|
and functionality expanded
\end{itemize}

%%%%%%%%%%%%%%%%%%%%%%%%%%%%%%%%%%%%%%%%
\paragraph{v1.0:} 2017/04/27

\begin{itemize}
\item
manual and install package
\item
first version published on CTAN
\end{itemize}

%%%%%%%%%%%%%%%%%%%%%%%%%%%%%%%%%%%%%%%%
\paragraph{v0.6:} 2017/04/26

\begin{itemize}
\item
redirection mechanism added
\end{itemize}

%%%%%%%%%%%%%%%%%%%%%%%%%%%%%%%%%%%%%%%%
\paragraph{v0.5:} 2017/04/26

\begin{itemize}
\item
functionality in definition file
\end{itemize}


%%%%%%%%%%%%%%%%%%%%%%%%%%%%%%%%%%%%%%%%%%%%%%%%%%%%%%%%%%%%%%%%%%%%%%%%%%%%%%%%
%%%%%%%%%%%%%%%%%%%%%%%%%%%%%%%%%%%%%%%%%%%%%%%%%%%%%%%%%%%%%%%%%%%%%%%%%%%%%%%%
%%%%%%%%%%%%%%%%%%%%%%%%%%%%%%%%%%%%%%%%%%%%%%%%%%%%%%%%%%%%%%%%%%%%%%%%%%%%%%%%
\appendix

\settowidth\MacroIndent{\rmfamily\scriptsize 000\ }

 \DocInput{childdoc.dtx}

\end{document}
%</driver>
% \fi
%
% %%%%%%%%%%%%%%%%%%%%%%%%%%%%%%%%%%%%%%%%%%%%%%%%%%%%%%%%%%%%%%%%%%%%%%%%%%%%%%
% %%%%%%%%%%%%%%%%%%%%%%%%%%%%%%%%%%%%%%%%%%%%%%%%%%%%%%%%%%%%%%%%%%%%%%%%%%%%%%
% \section{Sample}
%\iffalse
%<*samplemain>
%\fi
%
% The following presents a sample document
% with two chapters, two parts, a title page,
% a compile flag as well as three forwarding files to set the flag.
% It consists of eight |.tex| files:
% \begin{center}
% \begin{tabular}{ll}
% |cdocsamp.tex|&main file\\
% |cdocsch1.tex|&include file for chapter 1\\
% |cdocsch2.tex|&include file for chapter 2\\
% |cdocspt3.tex|&include file for part 3\\
% |cdocspt4.tex|&include file for part 4\\
% |cdocsdrf.tex|&forwarding file for main file in draft mode\\
% |cdocsfi1.tex|&forwarding file for final version of chapter 1\\
% |cdocsfi2.tex|&forwarding file for final version of chapter 2\\
% \end{tabular}
% \end{center}
% Each of the eight files can be compiled directly by the \LaTeX{} compiler.
%
% %%%%%%%%%%%%%%%%%%%%%%%%%%%%%%%%%%%%%%
% \paragraph{Main File.}
%
% The main file is called |cdocsamp.tex|.
%
% Load the \textsf{childdoc} definitions and
% declare the filename for the main document:
%    \begin{macrocode}
\input{childdoc.def}
\childdocmain{}
%    \end{macrocode}

% Optional override for |\version| flag:
%    \begin{macrocode}
%%\ifchilddoc\else\providecommand{\version}{draft}\fi
%    \end{macrocode}

% Define the default values for the |\version| flag
% (|final| for the main file and |draft| for childs):
%    \begin{macrocode}
\ifchilddoc
\providecommand{\version}{draft}
\else
\providecommand{\version}{final}
\fi
%    \end{macrocode}

% Load the standard document class:
%    \begin{macrocode}
\documentclass[12pt]{article}
%    \end{macrocode}

% Start the document body:
%    \begin{macrocode}
\begin{document}
%    \end{macrocode}

% Declare a title page.
% Print title, part of document being processed and version flag:
%    \begin{macrocode}
\addtocounter{page}{-1}
\begin{center}
{\LARGE\bfseries{}childdoc example\par}
\vspace{1cm}
\ifchilddoc
\ifchilddocmanual part\else chapter\fi:
`\childdocname' of `\childdocjob'\par
\else
main document: `\childdocjob'\par
\fi
version: \version\par
\end{center}
\newpage
%    \end{macrocode}

% Manually include selected file,
% otherwise process as usual:
%    \begin{macrocode}
\ifchilddocmanual
\section*{part `\childdocname'}
\input{\childdocname}
\else
%    \end{macrocode}

% Include the two chapters:
%    \begin{macrocode}
\include{cdocsch1}
\include{cdocsch2}
%    \end{macrocode}

% Include the two parts unless only chapters should be displayed:
%    \begin{macrocode}
\ifchilddoc\else
\section{part three}
\input{cdocspt3}
\section{part four}
\input{cdocspt4}
\fi
%    \end{macrocode}

% Process as usual until here:
%    \begin{macrocode}
\fi
%    \end{macrocode}

% End of document body:
%    \begin{macrocode}
\end{document}
%    \end{macrocode}
%\iffalse
%</samplemain>
%\fi
%
% %%%%%%%%%%%%%%%%%%%%%%%%%%%%%%%%%%%%%%
% \paragraph{Chapter Include Files.}
%
% The include files are called |cdocsch1.tex| and |cdocsch2.tex|.
%
%\iffalse
%<*samplechap1|samplechap2>
%\fi

% Optional override for |\version| flag:
%    \begin{macrocode}
%%\providecommand{\version}{final}
%    \end{macrocode}

% Include the main document:
%    \begin{macrocode}
\input{childdoc.def}
\childdocof{cdocsamp}
%    \end{macrocode}

%\iffalse
%</samplechap1|samplechap2>
%\fi
%
%\iffalse
%<*samplechap1>
%\fi
% Some text for chapter 1:
%    \begin{macrocode}
\section{one}
some text in chapter one
%    \end{macrocode}

%\iffalse
%</samplechap1>
%\fi
% Some text for chapter 2:
%\iffalse
%<*samplechap2>
%\fi
%    \begin{macrocode}
\section{two}
more text in chapter two
%    \end{macrocode}

%\iffalse
%</samplechap2>
%\fi
%
% %%%%%%%%%%%%%%%%%%%%%%%%%%%%%%%%%%%%%%
% \paragraph{Part Include Files.}
%
% The include files are called |cdocspt3.tex| and |cdocspt4.tex|.
%
%\iffalse
%<*samplepart3|samplepart4>
%\fi

% Optional override for |\version| flag:
%    \begin{macrocode}
%%\providecommand{\version}{final}
%    \end{macrocode}

% Include the main document:
%    \begin{macrocode}
\input{childdoc.def}
\childdocby{cdocsamp}
%    \end{macrocode}

%\iffalse
%</samplepart3|samplepart4>
%\fi
%
%\iffalse
%<*samplepart3>
%\fi
% Some text for part 3:
%    \begin{macrocode}
some text in part three
%    \end{macrocode}

%\iffalse
%</samplepart3>
%\fi
% Some text for part 4:
%\iffalse
%<*samplepart4>
%\fi
%    \begin{macrocode}
more text in part four
%    \end{macrocode}

%\iffalse
%</samplepart4>
%\fi
%
% %%%%%%%%%%%%%%%%%%%%%%%%%%%%%%%%%%%%%%
% \paragraph{Forwarding for a Complete Draft.}
%
% The following forwarding file |cdocsdrf.tex|
% compiles the main document in draft mode:
%\iffalse
%<*sampledraft>
%\fi
%    \begin{macrocode}
\def\version{draft}
\input{childdoc.def}
\childdocforward{cdocsamp}
%    \end{macrocode}

%\iffalse
%</sampledraft>
%\fi
%
% %%%%%%%%%%%%%%%%%%%%%%%%%%%%%%%%%%%%%%
% \paragraph{Forwarding for Final Version of the Chapters.}
%
% The following forwarding files |cdocsfn1.tex| and |cdocsfn2.tex|
% (with identical content)
% compile the final versions of the child documents
% |cdocsch1.tex| and |cdocsch2.tex|, respectively:
%\iffalse
%<*samplefinal>
%\fi
%    \begin{macrocode}
\def\version{final}
\input{childdoc.def}
\childdocforwardprefix[cdocsamp]{cdocsfn}{cdocsch}
%    \end{macrocode}

%\iffalse
%</samplefinal>
%\fi
%
% %%%%%%%%%%%%%%%%%%%%%%%%%%%%%%%%%%%%%%
% \paragraph{Command Line Processing.}
%
% The following three command lines generate the output files
% |cdocscld|, |cdocscl1| and |cdocscl2|
% which should be identical to
% |cdocsdrf|, |cdocsch1| and |cdocsfn2|, respectively:
% \begin{center}
% \begin{tabular}{l}
% |latex -jobname cdocscld \|\\
% |  "\def\version{draft}\input{childdoc.def}\childdocforward{cdocsamp}"|\\
% |latex -jobname cdocscl1 \|\\
% |  "\input{childdoc.def}\childdocforward[cdocsamp]{cdocsch1}"|\\
% |latex -jobname cdocscl2 \|\\
% |  "\def\version{final}\input{childdoc.def}\childdocforward{cdocsch2}"|
% \end{tabular}
% \end{center}
% Note that the trailing backslash on each first line
% merely continues the input to the second line
% (for convenient cut ant paste).
% Furthermore, the command |latex| can be replaced by any
% of its alternative versions such as |pdflatex|.
%
% %%%%%%%%%%%%%%%%%%%%%%%%%%%%%%%%%%%%%%%%%%%%%%%%%%%%%%%%%%%%%%%%%%%%%%%%%%%%%%
% %%%%%%%%%%%%%%%%%%%%%%%%%%%%%%%%%%%%%%%%%%%%%%%%%%%%%%%%%%%%%%%%%%%%%%%%%%%%%%
% \section{Implementation}
%\iffalse
%<*package>
%\fi
%
% This section describes the definitions file |childdoc.def|.

% The definitions cannot be loaded using |\usepackage| or |\RequirePackage|
% which has a mechanism to prevent loading a style file more than once.
% When loading the definitions by means of |\input|
% multiple instances have to be prevented manually:
%\iffalse
%This code needs to be before the `\ProvidesFile' directive
%which is defined at the beginning of this file.
%Therefore it is also placed there and commented out here.
%</package>
%<*discard>
%\fi
%    \begin{macrocode}
\ifdefined\childdocmain\endinput\fi
%    \end{macrocode}
%\iffalse
%</discard>
%<*package>
%\fi
%
% \macro{\ifchilddoc}
% \macro{\ifchilddocmanual}
% The conditional |\ifchilddoc| tells whether a
% child (true) or main (false) document is being compiled.
% The conditional |\ifchilddocmanual| tells whether
% the |\includeonly| mechanism is used (false) or
% the selection of child files must be performed manually (true).
% The definitions initialise to false:
%    \begin{macrocode}
\newif\ifchilddoc
\newif\ifchilddocmanual
%    \end{macrocode}

% \macro{\childdocname}
% \macro{\childdocjob}
% The macro |\childdocname| stores the name of the main document
% to be compiled. The macro |\childdocjob| stores the name of
% the document on which the \LaTeX{} compiler was originally invoked.
% The content of |\jobname| cannot be compared
% to filenames specified in the source due to different catcodes.
% The following code rescans |\jobname|, stores the result
% in |\childdocname| and saves a copy in |\childdocjob|:
%    \begin{macrocode}
\edef\childdocname{\scantokens\expandafter{\jobname\noexpand}}
\let\childdocjob\childdocname
%    \end{macrocode}

% \macro{\childdocdisable}
% The macro |\childdocdisable| prevents the main file
% from being processed more than once.
% At this stage, the main document command |\childdocmain|
% is assumed to be called once again where it should do nothing.
% Any subsequent call to it should prevent
% a secondary processing of the main document
% It overwrites the forwarding commands
% |\childdocof| and |\childdocforward|
% with empty macros to prevent further inclusions of the main document:
%    \begin{macrocode}
\newcommand{\childdocdisable}
{
  \renewcommand{\childdocmain}[1]{\renewcommand{\childdocmain}[1]{\endinput}}
  \renewcommand{\childdocof}[1]{}
  \renewcommand{\childdocby}[2][]{}
  \renewcommand{\childdocforward}[2][]{}
  \renewcommand{\childdocdisable}{}
}
%    \end{macrocode}

% \macro{\childdocmain}
% The macro |\childdocmain| is to be called at the top of the main file
% with nothing or the main filename (without extension) as argument.
% First, it breaks loops.
% If the argument is not empty and does not match |\childdocname|
% (which is set by the first inclusion of |childdoc.def|),
% |\ifchilddoc| is set to true, |\includeonly| is applied to the child file
% and |\jobname| is set to the main file
% (for proper handling of |.aux| files):
%    \begin{macrocode}
\newcommand{\childdocmain}[1]
{
  \childdocdisable\childdocmain{}
  \if?#1?\else
    \begingroup
      \def\childdoctmp{#1}
      \ifx\childdoctmp\childdocname
        \def\childdoctmp{}
      \else
        \def\childdoctmp
        {
          \childdoctrue
          \includeonly{\childdocname}
          \def\childdocjob{#1}
          \def\jobname{#1}
        }
      \fi
      \expandafter
    \endgroup
    \childdoctmp
  \fi
}
%    \end{macrocode}

% \macro{\childdocof}
% The command |\childdocof| redirects
% compilation to the main file |#1|.
%    \begin{macrocode}
\newcommand{\childdocof}[1]
{
  \childdocdisable
  \childdoctrue
  \includeonly{\childdocname}
  \def\jobname{#1}
  \def\childdocjob{#1}
  \input{#1}
}
%    \end{macrocode}

% \macro{\childdocby}
% The command |\childdocby| ....
%    \begin{macrocode}
\newcommand{\childdocby}[2][]
{
  \childdocdisable
  \childdoctrue
  \childdocmanualtrue
  \if?#1?\else
    \def\jobname{#2}
  \fi
  \def\childdocjob{#2}
  \input{#2}
  \endinput
}
%    \end{macrocode}

% \macro{\childdocforward}
% The command |\childdocforward| redirects
% compilation to the main file or
% (if the optional argument is given) a child file.
% Parameters are set as if the main file
% or a child file starting with |\childdocof| was compiled.
% Then compilation is handed over to the main file:
%    \begin{macrocode}
\newcommand{\childdocforward}[2][]
{
  \begingroup
    \if?#1?
      \def\childdoctmp
      {
        \def\childdocname{#2}
        \def\childdocjob{#2}
        \def\jobname{#2}
        \input{#2}
        \endinput
      }
    \else
      \def\childdoctmp
      {
        \childdocdisable
        \def\childdocname{#2}
        \childdoctrue
        \includeonly{#2}
        \def\childdocjob{#1}
        \def\jobname{#1}
        \input{#1}
        \endinput
      }
    \fi
    \expandafter
  \endgroup
  \childdoctmp
}
%    \end{macrocode}

% \macro{\childdocforwardprefix}
% The command |\childdocforwardprefix| redirects
% compilation to the main or a child file by means of a pattern.
% The prefix |#1| in the current filename is replaced by |#2|
% and the suffix of the current filename is kept
% (it is assumed that the filename does not contain the substring `|~~~|'
% which is used as a delimiter).
% Compilation is handed over to the new file by |\childdocforward|:
%    \begin{macrocode}
\newcommand{\childdocforwardprefix}[3][]
{
  \begingroup
    \def\childdocextract #2##1~~~{\def\childdoctmp{\childdocforward[#1]{#3##1}}}
    \expandafter\childdocextract\childdocname~~~
    \expandafter
  \endgroup
  \childdoctmp
}
%    \end{macrocode}

% \macro{\childdoc}
% The deprecated macro |\childdoc| is a legacy version of |\childdocmain|:
%    \begin{macrocode}
\newcommand{\childdoc}{\childdocmain}
%    \end{macrocode}

% \macro{\childdocredirect}
% The deprecated macro |\childdocredirect| is a legacy version
% of |\childdocforward| and |\childdocforwardprefix|:
%    \begin{macrocode}
\newcommand{\childdocredirect}[2][]
{
  \begingroup
    \if?#1?
      \def\childdoctmp{\childdocforward{#2}}
    \else
      \def\childdoctmp{\childdocforwardprefix{#1}{#2}}
    \fi
    \expandafter
  \endgroup
  \childdoctmp
}
%    \end{macrocode}

%\iffalse
%</package>
%\fi
%
\endinput

\childdocmain{}
%    \end{macrocode}

% Optional override for |\version| flag:
%    \begin{macrocode}
%%\ifchilddoc\else\providecommand{\version}{draft}\fi
%    \end{macrocode}

% Define the default values for the |\version| flag
% (|final| for the main file and |draft| for childs):
%    \begin{macrocode}
\ifchilddoc
\providecommand{\version}{draft}
\else
\providecommand{\version}{final}
\fi
%    \end{macrocode}

% Load the standard document class:
%    \begin{macrocode}
\documentclass[12pt]{article}
%    \end{macrocode}

% Start the document body:
%    \begin{macrocode}
\begin{document}
%    \end{macrocode}

% Declare a title page.
% Print title, part of document being processed and version flag:
%    \begin{macrocode}
\addtocounter{page}{-1}
\begin{center}
{\LARGE\bfseries{}childdoc example\par}
\vspace{1cm}
\ifchilddoc
\ifchilddocmanual part\else chapter\fi:
`\childdocname' of `\childdocjob'\par
\else
main document: `\childdocjob'\par
\fi
version: \version\par
\end{center}
\newpage
%    \end{macrocode}

% Manually include selected file,
% otherwise process as usual:
%    \begin{macrocode}
\ifchilddocmanual
\section*{part `\childdocname'}
\input{\childdocname}
\else
%    \end{macrocode}

% Include the two chapters:
%    \begin{macrocode}
\include{cdocsch1}
\include{cdocsch2}
%    \end{macrocode}

% Include the two parts unless only chapters should be displayed:
%    \begin{macrocode}
\ifchilddoc\else
\section{part three}
\input{cdocspt3}
\section{part four}
\input{cdocspt4}
\fi
%    \end{macrocode}

% Process as usual until here:
%    \begin{macrocode}
\fi
%    \end{macrocode}

% End of document body:
%    \begin{macrocode}
\end{document}
%    \end{macrocode}
%\iffalse
%</samplemain>
%\fi
%
% %%%%%%%%%%%%%%%%%%%%%%%%%%%%%%%%%%%%%%
% \paragraph{Chapter Include Files.}
%
% The include files are called |cdocsch1.tex| and |cdocsch2.tex|.
%
%\iffalse
%<*samplechap1|samplechap2>
%\fi

% Optional override for |\version| flag:
%    \begin{macrocode}
%%\providecommand{\version}{final}
%    \end{macrocode}

% Include the main document:
%    \begin{macrocode}
% \iffalse
%
% childdoc.dtx Copyright (C) 2017-2018 Niklas Beisert
%
% This work may be distributed and/or modified under the
% conditions of the LaTeX Project Public License, either version 1.3
% of this license or (at your option) any later version.
% The latest version of this license is in
%   http://www.latex-project.org/lppl.txt
% and version 1.3 or later is part of all distributions of LaTeX
% version 2005/12/01 or later.
%
% This work has the LPPL maintenance status `maintained'.
%
% The Current Maintainer of this work is Niklas Beisert.
%
% This work consists of the files childdoc.dtx and childdoc.ins
% and the derived files childdoc.def and cdocsamp.tex with
% cdocsch1.tex, cdocsch2.tex, cdocsdrf.tex, cdocsfn1.tex, cdocsfn2.tex.
%
%<package>\ifdefined\childdocmain\endinput\fi
%<package>\ProvidesFile{childdoc.def}[2018/12/30 v2.0 child document driver]
%<samplemain>\ProvidesFile{cdocsamp.tex}[2018/12/30 v2.0 sample for childdoc]
%<*driver>
%\ProvidesFile{childdoc.drv}[2018/12/30 v2.0 childdoc reference manual file]
\PassOptionsToClass{10pt,a4paper}{article}
\documentclass{ltxdoc}

\usepackage[margin=35mm]{geometry}
\usepackage{hyperref}
\usepackage{hyperxmp}
\usepackage[usenames]{color}

\hypersetup{colorlinks=true}
\hypersetup{pdfstartview=FitH}
\hypersetup{pdfpagemode=UseNone}
\hypersetup{pdfsource={}}
\hypersetup{pdflang={en-UK}}
\hypersetup{pdfcopyright={Copyright 2017-2018 Niklas Beisert.
  This work may be distributed and/or modified under the
  conditions of the LaTeX Project Public License, either version 1.3
  of this license or (at your option) any later version.}}
\hypersetup{pdflicenseurl={http://www.latex-project.org/lppl.txt}}
\hypersetup{pdfcontactaddress={ETH Zurich, ITP, HIT K,
  Wolfgang-Pauli-Strasse 27}}
\hypersetup{pdfcontactpostcode={8093}}
\hypersetup{pdfcontactcity={Zurich}}
\hypersetup{pdfcontactcountry={Switzerland}}
\hypersetup{pdfcontactemail={nbeisert@itp.phys.ethz.ch}}
\hypersetup{pdfcontacturl={http://people.phys.ethz.ch/\xmptilde nbeisert/}}

\newcommand{\secref}[1]{\hyperref[#1]{section \ref*{#1}}}

\parskip1ex
\parindent0pt
\let\olditemize\itemize
\def\itemize{\olditemize\parskip0pt}

\begin{document}

\title{The \textsf{childdoc} Package}
\hypersetup{pdftitle={The childdoc Package}}
\author{Niklas Beisert\\[2ex]
  Institut f\"ur Theoretische Physik\\
  Eidgen\"ossische Technische Hochschule Z\"urich\\
  Wolfgang-Pauli-Strasse 27, 8093 Z\"urich, Switzerland\\[1ex]
  \href{mailto:nbeisert@itp.phys.ethz.ch}
  {\texttt{nbeisert@itp.phys.ethz.ch}}}
\hypersetup{pdfauthor={Niklas Beisert}}
\hypersetup{pdfsubject={Manual for the LaTeX2e Package childdoc}}
\date{30 December 2018, \textsf{v2.0}}
\maketitle

\begin{abstract}\noindent
\textsf{childdoc} is a \LaTeXe{} package
that enables the direct compilation
of document sections included by |\include|
to individual files.
\end{abstract}

\begingroup
\parskip0ex
\tableofcontents
\endgroup

%%%%%%%%%%%%%%%%%%%%%%%%%%%%%%%%%%%%%%%%%%%%%%%%%%%%%%%%%%%%%%%%%%%%%%%%%%%%%%%%
%%%%%%%%%%%%%%%%%%%%%%%%%%%%%%%%%%%%%%%%%%%%%%%%%%%%%%%%%%%%%%%%%%%%%%%%%%%%%%%%
\section{Introduction}

\LaTeX{} provides a mechanism to structure a large document (such as a book)
into a main file and several child files (containing the chapters)
using the |\include| command.
This mechanism is beneficial for documents
which span hundreds of pages in order to
make the source file(s) more manageable.
Moreover, compilation can be restricted to
selected child files by means of the |\includeonly| command.
The latter feature can be used to reduce the compilation time while editing
(this was significantly more useful in the earlier days of \LaTeX{})
or to generate a smaller document which is easier to navigate.
Another application of |\includeonly| is to generate
documents consisting of selected parts of the complete document.

However, there are a few drawbacks of the plain |\include| mechanism:
\begin{itemize}
\item
The child files cannot be compiled on their own,
they can only be compiled via the main file.
A naive editing environment
(such as a text editor with an option
to have the current file processed by \LaTeX)
may require one to switch to the main file before compiling;
attempting to compile the child file produces errors.
\item
The main file must be modified (each time)
to adjust the |\includeonly| command
to the present needs. This easily leaves the main file in a messy state.
\item
The generated document will always carry the filename
of the main document. This is inconvenient if
several child files are to be compiled and
to be kept for distribution.
\end{itemize}

The present package provides a simple interface
to make child files individually compilable by \LaTeX{}.
Compiling a child file then has the same effect as compiling
the main file with an |\includeonly| command
to select the appropriate child.
Moreover the generated document will carry the name of the child
rather than the main file.
This resolves all three above issues.

This feature is meant to make the editing of books,
thesis documents and lecture notes somewhat more convenient.
However, the package can also be used efficiently for
composing a series of documents (such as exercise sheets)
which are typically distributed individually.
It then assists the author in generating the individual documents
(potentially in different versions)
as well as a document containing the collected series.
Another application is in developing style files
or other kinds of included material
where compilation of the style file could redirect
to a sample or test file.

%%%%%%%%%%%%%%%%%%%%%%%%%%%%%%%%%%%%%%%%%%%%%%%%%%%%%%%%%%%%%%%%%%%%%%%%%%%%%%%%
%%%%%%%%%%%%%%%%%%%%%%%%%%%%%%%%%%%%%%%%%%%%%%%%%%%%%%%%%%%%%%%%%%%%%%%%%%%%%%%%
\section{Usage}

First of all, the package \textsf{childdoc} is \emph{not} a standard
\LaTeXe{} |.sty| style file! Therefore it needs to be invoked in
a non-standard way.

%%%%%%%%%%%%%%%%%%%%%%%%%%%%%%%%%%%%%%%%%%%%%%%%%%%%%%%%%%%%%%%%%%%%%%%%%%%%%%%%
\subsection{Included Files}
\label{sec:include}

%%%%%%%%%%%%%%%%%%%%%%%%%%%%%%%%%%%%%%%%
\DescribeMacro{\childdocmain}
To use the package, add the commands
\begin{center}
\begin{tabular}{l}
|\input{childdoc.def}|\\
|\childdocmain{}|\\
\end{tabular}
\end{center}
at the very top of the main \LaTeX{} file,
in particular \emph{before} the |\documentclass| statement!
The argument of |\childdocmain| should be left empty
(but it must be present).

%%%%%%%%%%%%%%%%%%%%%%%%%%%%%%%%%%%%%%%%
\DescribeMacro{\childdocof}
Furthermore, add the commands
\begin{center}
\begin{tabular}{l}
|\input{childdoc.def}|\\
|\childdocof{|\textit{main}|}|\\
\end{tabular}
\end{center}
at the top of every child file \textit{child}
which is included by |\include{|\textit{child}|}|
from within the main file
(or at least for those files to be compiled individually).
The argument \textit{main} must be the filename of the main file.

There are a couple of
considerations in setting up the main and child documents:

%%%%%%%%%%%%%%%%%%%%%%%%%%%%%%%%%%%%%%%%
\paragraph{Restrictions.}

Please note the following restrictions:
\begin{itemize}
\item
|\childdocmain| must be called with one argument \textit{main}
to ensure compatibility with earlier version of the package.
It must either be empty (|\childdocmain{}|)
or precisely match the filename of the main file in which it is specified.
See \secref{sec:detection} for further information.
\item
The filename \textit{main} must be specified without the |.tex| extension.
\item
The filename \textit{main} is case sensitive
(even in case-insensitive file systems)
due to internal string comparison.
\item
The argument \textit{main} should be fully expanded, it cannot be a macro.
\item
Subdirectories and special characters should be avoided in filenames.
\item
The command |\childdocmain{|\textit{main}|}| must be followed by a whitespace.
It should not be followed immediately by another command
or by a comment mark `|%|'.
This is because the \TeX{} parser reads the token immediately following
the argument of |\childdocmain| and puts it
at the beginning of every child section;
however, a white\-space is ignored.
\end{itemize}

%%%%%%%%%%%%%%%%%%%%%%%%%%%%%%%%%%%%%%%%
\paragraph{Content of Main File.}

It is advisable to place all content in the child files included by |\include|.
Any output contained in the main file will appear in all child documents
unless suppressed manually;
it cannot be suppressed automatically by the |\includeonly| directive
and thus should normally be avoided.
A method to include some content in the main file
by means of conditional processing is described in \secref{sec:conditional}.

%%%%%%%%%%%%%%%%%%%%%%%%%%%%%%%%%%%%%%%%
\paragraph{Page Numbering.}

When only a part of the document is compiled,
the appropriate numbering of pages
(as well as other status parameters)
is determined from the |.aux| files.
The latter contain information from previous passes.
However this information needs to propagate through
all intermediate child documents.
Therefore the page numbering in child documents may well
be inconsistent until the complete document is compiled at least once.

A useful (if unconventional) way to always ensure a consistent
page numbering is to restart the numbering in each child document
and denote the pages by `\textit{child}|.|\textit{page}'
where \textit{child} represents the chapter/section number of the child file.
This can be achieved by the command
|\numberwithin{page}{|\textit{child}|}|
of the \textsf{amsmath} package
where \textit{child} can be |chapter| or |section|
depending on the chosen structuring.
Alternatively, one can modify the macro |\thepage| appropriately
and reset the counter |page| at the start of each child file.

%%%%%%%%%%%%%%%%%%%%%%%%%%%%%%%%%%%%%%%%%%%%%%%%%%%%%%%%%%%%%%%%%%%%%%%%%%%%%%%%
\subsection{Conditional Processing}
\label{sec:conditional}

The package provides a mechanism to compile different versions
of a document. To customise the versions further some conditional processing
can come in handy to distinguish which version is being compiled.
The package provides two macros to describe the compilation context:

%%%%%%%%%%%%%%%%%%%%%%%%%%%%%%%%%%%%%%%%
\DescribeMacro{\ifchilddoc}
The conditional |\ifchilddoc| distinguishes between the compilation of
child documents and the main document:
%
\begin{center}
|\ifchilddoc |\textit{child-code}| |[|\||else |\textit{main-code}]| \||fi|
\end{center}

%%%%%%%%%%%%%%%%%%%%%%%%%%%%%%%%%%%%%%%%
\DescribeMacro{\childdocname}
\DescribeMacro{\childdocjob}
The macro |\childdocname| contains the filename (without extension)
of the main or child file being processed.
Note that |\childdocjob| will always contain the name of the main file.

%%%%%%%%%%%%%%%%%%%%%%%%%%%%%%%%%%%%%%%%
\paragraph{Title Page.}

Conditional processing can be used to include a title or banner page
in the main document when proper precautions are taken.
Importantly, the code in the main file should ensure that the page counter
(as well as other status parameters which are stored in the |.aux| files)
takes the same value after the conditional processing.
Otherwise the page numbers may take divergent values
depending on which part is compiled.

For example, a title page could be declared by:
%
\begin{center}
\begin{tabular}{l}
|\ifchilddoc\||else|\\
|\addtocounter{page}{-1}|\\
\textit{code for title page}\\
|\newpage|\\
|\||fi|
\end{tabular}
\end{center}
%
A banner page for the child documents can be generated by:
%
\begin{center}
\begin{tabular}{l}
|\ifchilddoc|\\
|\addtocounter{page}{-1}|\\
\textit{code for banner page}\\
|\newpage|\\
|\||fi|
\end{tabular}
\end{center}
%
Here one could write a message such as:
\begin{center}
|This is the part \childdocname{} of \childdocjob{}.|
\end{center}

%%%%%%%%%%%%%%%%%%%%%%%%%%%%%%%%%%%%%%%%%%%%%%%%%%%%%%%%%%%%%%%%%%%%%%%%%%%%%%%%
\subsection{Flags}
\label{sec:flags}

The package makes it easy to generate different versions
of the main or child documents.
To this end compilation flags can be defined
and assigned different default values.
They will be particularly useful in conjunction
with the forwarding mechanism described in \secref{sec:forward}.

For example, it may be useful to have a flag |\version|
which can be set to |draft| or |final|.
The document source will contain some conditional code
depending on the value of |\version|.
Suppose further, the flag should default to |final| for the main file
and to |draft| for child files
which is a natural assignment for editing the document.
This is achieved by placing the following code
in the preamble of the main document
(below the |\childdocmain| directive):
%
\begin{center}
\begin{tabular}{l}
|\ifchilddoc|\\
|\providecommand{\version}{draft}|\\
|\||else|\\
|\providecommand{\version}{final}|\\
|\||fi|
\end{tabular}
\end{center}
%
The definition by |\providecommand| makes sure
that previous definitions are not overwritten.
Further statements |\providecommand{\version}{...}|
can thus be added before the above code to override it.

For the main file, one might add a line
(between |\childdocmain| and the above block)
%
\begin{center}
|%\ifchilddoc\||else\providecommand{\version}{draft}\||fi|
\end{center}
%
which can be uncommented to produce a draft version.
Likewise one can add a line to the very top of a child file
(above the |\childdocof{|\textit{main}|}| directive)
%
\begin{center}
|%\providecommand{\version}{final}|
\end{center}
%
which can be uncommented to produce the final version of this child document.

%%%%%%%%%%%%%%%%%%%%%%%%%%%%%%%%%%%%%%%%%%%%%%%%%%%%%%%%%%%%%%%%%%%%%%%%%%%%%%%%
\subsection{Forwarding}
\label{sec:forward}

Different versions of the main or child documents
using compilation flags as described in \secref{sec:flags}
can be (permanently) stored in different files
for convenient compilation, viewing and distribution.
To this end, the package defines a command
to pass on compilation to a different file:

%%%%%%%%%%%%%%%%%%%%%%%%%%%%%%%%%%%%%%%%
\DescribeMacro{\childdocforward}
The command |\childdocforward| redirects processing to
another source file:
%
\begin{center}
\begin{tabular}{l}
|\input{childdoc.def}|\\
|\childdocforward[|\textit{main}|]{|\textit{dest}|}|\\
\end{tabular}
\end{center}
%
The argument \textit{dest} is the destination file
(without extension).
It should be the main file or one of the child files.
Note that further \textsf{childdoc} directives
such as |\childdocof| and |\childdocforward|
in the indicated file will be processed in this form.
The optional argument \textit{main}
passes on directly to the main file \textit{main}
while pretending to compile the child \textit{dest}.
This form behaves as if \textit{dest}
issues |\childdocof{|\textit{main}|}| right away,
and no further \textsf{childdoc} directives will be processed.

%%%%%%%%%%%%%%%%%%%%%%%%%%%%%%%%%%%%%%%%
\DescribeMacro{\...prefix}
In the alternative form |\childdocforwardprefix|,
%
\begin{center}
\begin{tabular}{l}
|\input{childdoc.def}|\\
|\childdocforwardprefix[|\textit{main}|]{|\textit{prefix}|}{|\textit{dest}|}|
\end{tabular}
\end{center}
%
the destination file is determined by a pattern
depending on the current file:
To make this work, the current file must be called
`{\textit{prefix}\hspace{0.2em}\textit{suffix}}'
with \textit{prefix} matching precisely the argument.
Processing is then passed on to the file
`{\textit{dest}\hspace{0.2em}\textit{suffix}}'.
Surely, the same effect is achieved by
directly specifying the
argument `{\textit{dest}\hspace{0.2em}\textit{suffix}}'
in the first form.
However, that requires to set up a different file
for each child. With the alternative form of the command
all these files can have exactly the same content
which simplifies setting them up and maintaining them.

For example, the following file |draft.tex|
with a compilation flag |\version| as described in \secref{sec:flags}
compiles the main document as a draft:
%
\begin{center}
\begin{tabular}{l}
|\def\version{draft}|\\
|\input{childdoc.def}|\\
|\childdocforward{|\textit{main}|}|
\end{tabular}
\end{center}
%
Likewise, the following files |final|\textit{nn}|.tex|
compile the final version of the child document
|child|\textit{nn}|.tex|:
%
\begin{center}
\begin{tabular}{l}
|\def\version{final}|\\
|\input{childdoc.def}|\\
|\childdocforwardprefix{final}{child}|
\end{tabular}
\end{center}
%

Note that when several versions of a main file and/or of each child file
are to be generated, it may be convenient to set up a |Makefile| or
shell script to automatise the process.

%%%%%%%%%%%%%%%%%%%%%%%%%%%%%%%%%%%%%%%%%%%%%%%%%%%%%%%%%%%%%%%%%%%%%%%%%%%%%%%%
\subsection{Command Line Processing}
\label{sec:commandline}

The effect of redirection files can also be achieved by invoking
the \LaTeX{} compiler with a more elaborate command line.
Most conveniently this should be done as part
of a shell script or a |Makefile|.

When using \textsf{childdoc} in the main file, the following
command lines effectively perform a redirection
(note that depending on the shell being used,
backslashes may have to be doubled: `|\|' $\to$ `|\\|'):
%
\begin{center}
|... -jobname "|\textit{target}|" |\\|"|[\textit{flags}]%
|\input{childdoc.def}\childdocforward[|\textit{main}|]{|\textit{dest}|}"|
\end{center}
%
Here \textit{target} is the name of the output file,
\textit{main} is the name of the main file
and \textit{dest} is the name of the main or child file to be processed
(all filenames without extensions).
The optional argument \textit{main} can be omitted
if \textit{main} matches \textit{dest}.
Optionally, compilation \textit{flags} can be defined via |\def| commands.
This command line makes the \TeX{} engine believe
it is compiling the file \textit{target}
whose content is specified as the latter parameter.
The provided code then forwards the processing to
\textit{main} or \textit{dest} as described in \secref{sec:forward}.

%%%%%%%%%%%%%%%%%%%%%%%%%%%%%%%%%%%%%%%%%%%%%%%%%%%%%%%%%%%%%%%%%%%%%%%%%%%%%%%%
\subsection{Include by Input}
\label{sec:input}

Including child documents by |\include| has some restrictions by design.
Most notably, the content of a child document always occupies
its own set of pages; pages cannot be shared between child documents.
Usually, this behaviour makes perfect sense
because each child document contain an essential part of the document.
However, in some situations it may be desirable to compose
a document from a collection of parts
without having mandatory page breaks between then.
For this case, the package
provides a mechanism to include parts
by |\input| which can also be processed individually.
However, by construction this mechanism
requires manual handling of the content to be output.

%%%%%%%%%%%%%%%%%%%%%%%%%%%%%%%%%%%%%%%%
\DescribeMacro{\ifchilddocmanual}
The main file should be prepared as usual, see \secref{sec:include}.
However, the document body must make a distinction
between processing of an individual part and of the main document, e.g.:
%
\begin{center}
\begin{tabular}{l}
|\ifchilddocmanual|\\
|\input{\childdocname}|\\
|\||else|\\
\textit{document body with }|\input{|\textit{part}|}|\\
|\||fi|
\end{tabular}
\end{center}
%
The conditional |\ifchilddocmanual| is true whenever
a part to be included by |\input| is being compiled,
and the name of the part is stored in |\childdocname|.

%%%%%%%%%%%%%%%%%%%%%%%%%%%%%%%%%%%%%%%%
\DescribeMacro{\childdocby}
Each part to be included by |\input| should start with:
%
\begin{center}
\begin{tabular}{l}
|\input{childdoc.def}|\\
|\childdocby{|\textit{main}|}|\\
\end{tabular}
\end{center}
%
The directive |\childdocby| is similar to |\childdocof|
described in \secref{sec:include},
but the subsequent selection of content must be done manually.
To that end, both |\ifchilddoc| and |\ifchilddocmanual|
will be true upon processing of a part,
and the name of the part is stored in |\childdocname|.
Note that |\jobname| will be set to the filename of the current part
so that each part receives an individual |.aux| file
that does not interfere with the |.aux| file(s) of the main document.
This behaviour can be altered by the alternative form
|\childdocby[*]{|\textit{main}|}| (with a non-empty optional argument)
which uses the |.aux| file of the main document
by setting |\jobname| to \textit{main}.

%%%%%%%%%%%%%%%%%%%%%%%%%%%%%%%%%%%%%%%%%%%%%%%%%%%%%%%%%%%%%%%%%%%%%%%%%%%%%%%%
\subsection{Driver Development}
\label{sec:driver}

The \textsf{childdoc} mechanism can also be use for the development
of definition files such as \LaTeX{} styles or classes.
This case differs from the above setup with multiple parts
included by |\include| in that no |\includeonly| should be invoked.
This can be achieved by starting the include file
(before |\ProvidesPackage|) with:
%
\begin{center}
\begin{tabular}{l}
|\input{childdoc.def}|\\
|\childdocforward{|\textit{main}|}|\\
\end{tabular}
\end{center}
%
or alternatively with:
%
\begin{center}
\begin{tabular}{l}
|\input{childdoc.def}|\\
|\childdocby{|\textit{main}|}|\\
\end{tabular}
\end{center}
%
Both forms have slightly different effects as described above.
The main file is prepared as usual, see \secref{sec:include}.

%%%%%%%%%%%%%%%%%%%%%%%%%%%%%%%%%%%%%%%%%%%%%%%%%%%%%%%%%%%%%%%%%%%%%%%%%%%%%%%%
\subsection{Legacy Detection}
\label{sec:detection}

The directive |\childdocmain| in the main file can detect
whether the complete document or merely a child is to be compiled
even without using the directive |\childdocof|.
This method is deprecated because it is less robust
and there is no compelling reason to use it;
it is merely provided for backward compatibility
and it may be removed in future versions.

If the detection mechanism is to be used,
it is mandatory to correctly specify
the filename of the main file as the argument of |\childdocmain|:
%
\begin{center}
\begin{tabular}{l}
|\input{childdoc.def}|\\
|\childdocmain{|\textit{main}|}|\\
\end{tabular}
\end{center}
%
If |\jobname| does not match the argument \textit{main} of |\childdocmain|,
it is assumed that |\jobname| points to the child file to be compiled.
When using |\childdocmain| with the main file specified as argument,
it suffices to start a child file
with just |\input{|\textit{main}|}|
without loading of the package and using |\childdocof|.
If instead all processing is done
with the appropriate \textsf{childdoc} directives,
the argument of \textit{main} of |\childdocmain| can be empty.

An alternative version of the command line processing described
in \secref{sec:commandline} using the detection mechanism reads:
%
\begin{center}
|... -jobname "|\textit{target}|" "|[\textit{flags}]%
[|\def\jobname{|\textit{dest}|}|]|\input{|\textit{main}|}"|
\end{center}

%%%%%%%%%%%%%%%%%%%%%%%%%%%%%%%%%%%%%%%%%%%%%%%%%%%%%%%%%%%%%%%%%%%%%%%%%%%%%%%%
\subsection{Manual Code}
\label{sec:manual}

In case one cannot be certain whether the definitions file |childdoc.def|
is installed on the target \TeX{} distribution
and one prefers not to ship it,
it is conceivable to paste a few relevant commands into the sources.

To that end, drop all statements |\input{childdoc.def}|
and perform the replacements as outlined below.
Instead of |\childdocmain{|\textit{main}|}| add the following code
to the top of the main file:
%
\begin{center}
\begin{tabular}{l}
|\||ifdefined\childdocname\endinput\||fi\newif\ifchilddoc|\\
|\edef\childdocname{\scantokens\expandafter{\jobname\noexpand}}|\\
|\def\childdocmain{|\textit{main}|}\||ifx\childdocmain\childdocname\||else|\\
|\childdoctrue\includeonly{\childdocname}\let\jobname\childdocmain\||fi|\\
\end{tabular}
\end{center}
%
Instead of |\childdocof{|\textit{main}|}| just include the main file
at the top of each child file:
%
\begin{center}
|\input{|\textit{main}|}|
\end{center}
%
A simple redirection |\childdocforward{|\textit{dest}|}| is achieved by:
%
\begin{center}
|\def\jobname{|\textit{dest}|}\input{\jobname}|
\end{center}
%
The redirection with prefix
|\childdocforwardprefix[|\textit{prefix}|]{|\textit{dest}|}|
is accomplished by:
%
\begin{center}
\begin{tabular}{l}
|{\edef\jobname{\scantokens\expandafter{\jobname\noexpand}}|\\
|\def\redirectjob |\textit{prefix}|#1~~~{\gdef\jobname{|\textit{dest}|#1}}|\\
|\expandafter\redirectjob\jobname~~~}\input{\jobname}|
\end{tabular}
\end{center}

In an alternative approach,
child documents can be compiled by a specific command line
without additional code or specific definitions:
%
\begin{center}
|... -jobname "|\textit{target}|" "|[\textit{flags}]%
|\includeonly{|\textit{dest}|}\input{|\textit{main}|}"|
\end{center}
%

%%%%%%%%%%%%%%%%%%%%%%%%%%%%%%%%%%%%%%%%%%%%%%%%%%%%%%%%%%%%%%%%%%%%%%%%%%%%%%%%
%%%%%%%%%%%%%%%%%%%%%%%%%%%%%%%%%%%%%%%%%%%%%%%%%%%%%%%%%%%%%%%%%%%%%%%%%%%%%%%%
\section{Information}

%%%%%%%%%%%%%%%%%%%%%%%%%%%%%%%%%%%%%%%%%%%%%%%%%%%%%%%%%%%%%%%%%%%%%%%%%%%%%%%%
\subsection{Copyright}

Copyright \copyright{} 2017--2018 Niklas Beisert

This work may be distributed and/or modified under the
conditions of the \LaTeX{} Project Public License, either version 1.3
of this license or (at your option) any later version.
The latest version of this license is in
  \url{http://www.latex-project.org/lppl.txt}
and version 1.3 or later is part of all distributions of \LaTeX{}
version 2005/12/01 or later.

This work has the LPPL maintenance status `maintained'.

The Current Maintainer of this work is Niklas Beisert.

This work consists of the files |README.txt|, |childdoc.ins| and |childdoc.dtx|
as well as the derived files |childdoc.def|, |cdocsamp.tex|
with |cdocsch1.tex|, |cdocsch2.tex|, |cdocspt3.tex|, |cdocspt4.tex|,
|cdocsdrf.tex|, |cdocsfn1.tex|, |cdocsfn2.tex|
as well as |childdoc.pdf|.

%%%%%%%%%%%%%%%%%%%%%%%%%%%%%%%%%%%%%%%%%%%%%%%%%%%%%%%%%%%%%%%%%%%%%%%%%%%%%%%%
\subsection{Files and Installation}

The package consists of the files:
%
\begin{center}
\begin{tabular}{ll}
    |README.txt|   & readme file \\
    |childdoc.ins| & installation file \\
    |childdoc.dtx| & source file \\
    |childdoc.def| & definition file \\
    |cdocsamp.tex| & sample main file \\
    |cdocsch1.tex| & sample include file \\
    |cdocsch2.tex| & sample include file \\
    |cdocspt3.tex| & sample part file \\
    |cdocspt4.tex| & sample part file \\
    |cdocsdrf.tex| & sample redirection file \\
    |cdocsfn1.tex| & sample redirection file \\
    |cdocsfn2.tex| & sample redirection file \\
    |childdoc.pdf| & manual
\end{tabular}
\end{center}
%
The distribution consists of the files
|README.txt|, |childdoc.ins| and |childdoc.dtx|.
%
\begin{itemize}
\item
Run (pdf)\LaTeX{} on |childdoc.dtx|
to compile the manual |childdoc.pdf| (this file).
\item
Run \LaTeX{} on |childdoc.ins| to create the definitions file |childdoc.def|
and the sample |cdocsamp.tex| with include files
|cdocsch1.tex|, |cdocsch2.tex|, |cdocspt3.tex|, |cdocspt4.tex|,
|cdocsdrf.tex|, |cdocsfn1.tex|, |cdocsfn2.tex|.
Then copy the file |childdoc.def| to an appropriate directory of your \LaTeX{}
distribution, e.g.\ \textit{texmf-root}|/tex/latex/childdoc|.
\end{itemize}

%%%%%%%%%%%%%%%%%%%%%%%%%%%%%%%%%%%%%%%%%%%%%%%%%%%%%%%%%%%%%%%%%%%%%%%%%%%%%%%%
\subsection{Related CTAN Packages}

There are several other packages which offer a similar functionality:
%
\begin{itemize}
\item
The packages
\href{http://ctan.org/pkg/docmute}{\textsf{docmute}},
\href{http://ctan.org/pkg/includex}{\textsf{includex}} and
\href{http://ctan.org/pkg/standalone}{\textsf{standalone}}
provide commands to include only the document body of
a child file thus allowing both files to be compiled individually.
\item
The packages \href{http://ctan.org/pkg/subdocs}{\textsf{subdocs}}
and \href{http://ctan.org/pkg/subfiles}{\textsf{subfiles}}
provide structures in which the main and child documents can be
encapsulated and allowing them to be compiled individually.
The inclusion mechanism is different from the conventional |\include|.
\item
The package \href{http://ctan.org/pkg/combine}{\textsf{combine}}
is an elaborate solution to combine several documents into one.
\end{itemize}
%
See also the CTAN topic \href{http://ctan.org/topic/subdocs}{\textsf{subdocs}}
for further related packages.
The present package differs from the above solutions in that
a document structure constructed with the conventional |\include| mechanism
just needs two extra commands at the top of every file
such that all constituent files can be compiled individually.

%%%%%%%%%%%%%%%%%%%%%%%%%%%%%%%%%%%%%%%%%%%%%%%%%%%%%%%%%%%%%%%%%%%%%%%%%%%%%%%%
%\subsection{Feature Suggestions}
%
%The following is a list of features which may be useful for future
%versions of this package:
%%
%\begin{itemize}
%\item
%\ldots
%\end{itemize}

%%%%%%%%%%%%%%%%%%%%%%%%%%%%%%%%%%%%%%%%%%%%%%%%%%%%%%%%%%%%%%%%%%%%%%%%%%%%%%%%
\subsection{Revision History}

%%%%%%%%%%%%%%%%%%%%%%%%%%%%%%%%%%%%%%%%
\paragraph{v2.0:} 2018/12/30

\begin{itemize}
\item
immediate forward processing
\item
added |\childdocby| mechanism
\item
manual restructured
\end{itemize}

%%%%%%%%%%%%%%%%%%%%%%%%%%%%%%%%%%%%%%%%
\paragraph{v1.6:} 2018/01/17

\begin{itemize}
\item
application for development of include files
\item
corrections to manual
\end{itemize}

%%%%%%%%%%%%%%%%%%%%%%%%%%%%%%%%%%%%%%%%
\paragraph{v1.5:} 2017/05/21

\begin{itemize}
\item
more complete structuring introduced
\item
|\childdocof| introduced
\item
|\childdoc| renamed to |\childdocmain|
\item
|\childredirect| renamed to |\childdocforward| and |\childdocforwardprefix|
and functionality expanded
\end{itemize}

%%%%%%%%%%%%%%%%%%%%%%%%%%%%%%%%%%%%%%%%
\paragraph{v1.0:} 2017/04/27

\begin{itemize}
\item
manual and install package
\item
first version published on CTAN
\end{itemize}

%%%%%%%%%%%%%%%%%%%%%%%%%%%%%%%%%%%%%%%%
\paragraph{v0.6:} 2017/04/26

\begin{itemize}
\item
redirection mechanism added
\end{itemize}

%%%%%%%%%%%%%%%%%%%%%%%%%%%%%%%%%%%%%%%%
\paragraph{v0.5:} 2017/04/26

\begin{itemize}
\item
functionality in definition file
\end{itemize}


%%%%%%%%%%%%%%%%%%%%%%%%%%%%%%%%%%%%%%%%%%%%%%%%%%%%%%%%%%%%%%%%%%%%%%%%%%%%%%%%
%%%%%%%%%%%%%%%%%%%%%%%%%%%%%%%%%%%%%%%%%%%%%%%%%%%%%%%%%%%%%%%%%%%%%%%%%%%%%%%%
%%%%%%%%%%%%%%%%%%%%%%%%%%%%%%%%%%%%%%%%%%%%%%%%%%%%%%%%%%%%%%%%%%%%%%%%%%%%%%%%
\appendix

\settowidth\MacroIndent{\rmfamily\scriptsize 000\ }

 \DocInput{childdoc.dtx}

\end{document}
%</driver>
% \fi
%
% %%%%%%%%%%%%%%%%%%%%%%%%%%%%%%%%%%%%%%%%%%%%%%%%%%%%%%%%%%%%%%%%%%%%%%%%%%%%%%
% %%%%%%%%%%%%%%%%%%%%%%%%%%%%%%%%%%%%%%%%%%%%%%%%%%%%%%%%%%%%%%%%%%%%%%%%%%%%%%
% \section{Sample}
%\iffalse
%<*samplemain>
%\fi
%
% The following presents a sample document
% with two chapters, two parts, a title page,
% a compile flag as well as three forwarding files to set the flag.
% It consists of eight |.tex| files:
% \begin{center}
% \begin{tabular}{ll}
% |cdocsamp.tex|&main file\\
% |cdocsch1.tex|&include file for chapter 1\\
% |cdocsch2.tex|&include file for chapter 2\\
% |cdocspt3.tex|&include file for part 3\\
% |cdocspt4.tex|&include file for part 4\\
% |cdocsdrf.tex|&forwarding file for main file in draft mode\\
% |cdocsfi1.tex|&forwarding file for final version of chapter 1\\
% |cdocsfi2.tex|&forwarding file for final version of chapter 2\\
% \end{tabular}
% \end{center}
% Each of the eight files can be compiled directly by the \LaTeX{} compiler.
%
% %%%%%%%%%%%%%%%%%%%%%%%%%%%%%%%%%%%%%%
% \paragraph{Main File.}
%
% The main file is called |cdocsamp.tex|.
%
% Load the \textsf{childdoc} definitions and
% declare the filename for the main document:
%    \begin{macrocode}
\input{childdoc.def}
\childdocmain{}
%    \end{macrocode}

% Optional override for |\version| flag:
%    \begin{macrocode}
%%\ifchilddoc\else\providecommand{\version}{draft}\fi
%    \end{macrocode}

% Define the default values for the |\version| flag
% (|final| for the main file and |draft| for childs):
%    \begin{macrocode}
\ifchilddoc
\providecommand{\version}{draft}
\else
\providecommand{\version}{final}
\fi
%    \end{macrocode}

% Load the standard document class:
%    \begin{macrocode}
\documentclass[12pt]{article}
%    \end{macrocode}

% Start the document body:
%    \begin{macrocode}
\begin{document}
%    \end{macrocode}

% Declare a title page.
% Print title, part of document being processed and version flag:
%    \begin{macrocode}
\addtocounter{page}{-1}
\begin{center}
{\LARGE\bfseries{}childdoc example\par}
\vspace{1cm}
\ifchilddoc
\ifchilddocmanual part\else chapter\fi:
`\childdocname' of `\childdocjob'\par
\else
main document: `\childdocjob'\par
\fi
version: \version\par
\end{center}
\newpage
%    \end{macrocode}

% Manually include selected file,
% otherwise process as usual:
%    \begin{macrocode}
\ifchilddocmanual
\section*{part `\childdocname'}
\input{\childdocname}
\else
%    \end{macrocode}

% Include the two chapters:
%    \begin{macrocode}
\include{cdocsch1}
\include{cdocsch2}
%    \end{macrocode}

% Include the two parts unless only chapters should be displayed:
%    \begin{macrocode}
\ifchilddoc\else
\section{part three}
\input{cdocspt3}
\section{part four}
\input{cdocspt4}
\fi
%    \end{macrocode}

% Process as usual until here:
%    \begin{macrocode}
\fi
%    \end{macrocode}

% End of document body:
%    \begin{macrocode}
\end{document}
%    \end{macrocode}
%\iffalse
%</samplemain>
%\fi
%
% %%%%%%%%%%%%%%%%%%%%%%%%%%%%%%%%%%%%%%
% \paragraph{Chapter Include Files.}
%
% The include files are called |cdocsch1.tex| and |cdocsch2.tex|.
%
%\iffalse
%<*samplechap1|samplechap2>
%\fi

% Optional override for |\version| flag:
%    \begin{macrocode}
%%\providecommand{\version}{final}
%    \end{macrocode}

% Include the main document:
%    \begin{macrocode}
\input{childdoc.def}
\childdocof{cdocsamp}
%    \end{macrocode}

%\iffalse
%</samplechap1|samplechap2>
%\fi
%
%\iffalse
%<*samplechap1>
%\fi
% Some text for chapter 1:
%    \begin{macrocode}
\section{one}
some text in chapter one
%    \end{macrocode}

%\iffalse
%</samplechap1>
%\fi
% Some text for chapter 2:
%\iffalse
%<*samplechap2>
%\fi
%    \begin{macrocode}
\section{two}
more text in chapter two
%    \end{macrocode}

%\iffalse
%</samplechap2>
%\fi
%
% %%%%%%%%%%%%%%%%%%%%%%%%%%%%%%%%%%%%%%
% \paragraph{Part Include Files.}
%
% The include files are called |cdocspt3.tex| and |cdocspt4.tex|.
%
%\iffalse
%<*samplepart3|samplepart4>
%\fi

% Optional override for |\version| flag:
%    \begin{macrocode}
%%\providecommand{\version}{final}
%    \end{macrocode}

% Include the main document:
%    \begin{macrocode}
\input{childdoc.def}
\childdocby{cdocsamp}
%    \end{macrocode}

%\iffalse
%</samplepart3|samplepart4>
%\fi
%
%\iffalse
%<*samplepart3>
%\fi
% Some text for part 3:
%    \begin{macrocode}
some text in part three
%    \end{macrocode}

%\iffalse
%</samplepart3>
%\fi
% Some text for part 4:
%\iffalse
%<*samplepart4>
%\fi
%    \begin{macrocode}
more text in part four
%    \end{macrocode}

%\iffalse
%</samplepart4>
%\fi
%
% %%%%%%%%%%%%%%%%%%%%%%%%%%%%%%%%%%%%%%
% \paragraph{Forwarding for a Complete Draft.}
%
% The following forwarding file |cdocsdrf.tex|
% compiles the main document in draft mode:
%\iffalse
%<*sampledraft>
%\fi
%    \begin{macrocode}
\def\version{draft}
\input{childdoc.def}
\childdocforward{cdocsamp}
%    \end{macrocode}

%\iffalse
%</sampledraft>
%\fi
%
% %%%%%%%%%%%%%%%%%%%%%%%%%%%%%%%%%%%%%%
% \paragraph{Forwarding for Final Version of the Chapters.}
%
% The following forwarding files |cdocsfn1.tex| and |cdocsfn2.tex|
% (with identical content)
% compile the final versions of the child documents
% |cdocsch1.tex| and |cdocsch2.tex|, respectively:
%\iffalse
%<*samplefinal>
%\fi
%    \begin{macrocode}
\def\version{final}
\input{childdoc.def}
\childdocforwardprefix[cdocsamp]{cdocsfn}{cdocsch}
%    \end{macrocode}

%\iffalse
%</samplefinal>
%\fi
%
% %%%%%%%%%%%%%%%%%%%%%%%%%%%%%%%%%%%%%%
% \paragraph{Command Line Processing.}
%
% The following three command lines generate the output files
% |cdocscld|, |cdocscl1| and |cdocscl2|
% which should be identical to
% |cdocsdrf|, |cdocsch1| and |cdocsfn2|, respectively:
% \begin{center}
% \begin{tabular}{l}
% |latex -jobname cdocscld \|\\
% |  "\def\version{draft}\input{childdoc.def}\childdocforward{cdocsamp}"|\\
% |latex -jobname cdocscl1 \|\\
% |  "\input{childdoc.def}\childdocforward[cdocsamp]{cdocsch1}"|\\
% |latex -jobname cdocscl2 \|\\
% |  "\def\version{final}\input{childdoc.def}\childdocforward{cdocsch2}"|
% \end{tabular}
% \end{center}
% Note that the trailing backslash on each first line
% merely continues the input to the second line
% (for convenient cut ant paste).
% Furthermore, the command |latex| can be replaced by any
% of its alternative versions such as |pdflatex|.
%
% %%%%%%%%%%%%%%%%%%%%%%%%%%%%%%%%%%%%%%%%%%%%%%%%%%%%%%%%%%%%%%%%%%%%%%%%%%%%%%
% %%%%%%%%%%%%%%%%%%%%%%%%%%%%%%%%%%%%%%%%%%%%%%%%%%%%%%%%%%%%%%%%%%%%%%%%%%%%%%
% \section{Implementation}
%\iffalse
%<*package>
%\fi
%
% This section describes the definitions file |childdoc.def|.

% The definitions cannot be loaded using |\usepackage| or |\RequirePackage|
% which has a mechanism to prevent loading a style file more than once.
% When loading the definitions by means of |\input|
% multiple instances have to be prevented manually:
%\iffalse
%This code needs to be before the `\ProvidesFile' directive
%which is defined at the beginning of this file.
%Therefore it is also placed there and commented out here.
%</package>
%<*discard>
%\fi
%    \begin{macrocode}
\ifdefined\childdocmain\endinput\fi
%    \end{macrocode}
%\iffalse
%</discard>
%<*package>
%\fi
%
% \macro{\ifchilddoc}
% \macro{\ifchilddocmanual}
% The conditional |\ifchilddoc| tells whether a
% child (true) or main (false) document is being compiled.
% The conditional |\ifchilddocmanual| tells whether
% the |\includeonly| mechanism is used (false) or
% the selection of child files must be performed manually (true).
% The definitions initialise to false:
%    \begin{macrocode}
\newif\ifchilddoc
\newif\ifchilddocmanual
%    \end{macrocode}

% \macro{\childdocname}
% \macro{\childdocjob}
% The macro |\childdocname| stores the name of the main document
% to be compiled. The macro |\childdocjob| stores the name of
% the document on which the \LaTeX{} compiler was originally invoked.
% The content of |\jobname| cannot be compared
% to filenames specified in the source due to different catcodes.
% The following code rescans |\jobname|, stores the result
% in |\childdocname| and saves a copy in |\childdocjob|:
%    \begin{macrocode}
\edef\childdocname{\scantokens\expandafter{\jobname\noexpand}}
\let\childdocjob\childdocname
%    \end{macrocode}

% \macro{\childdocdisable}
% The macro |\childdocdisable| prevents the main file
% from being processed more than once.
% At this stage, the main document command |\childdocmain|
% is assumed to be called once again where it should do nothing.
% Any subsequent call to it should prevent
% a secondary processing of the main document
% It overwrites the forwarding commands
% |\childdocof| and |\childdocforward|
% with empty macros to prevent further inclusions of the main document:
%    \begin{macrocode}
\newcommand{\childdocdisable}
{
  \renewcommand{\childdocmain}[1]{\renewcommand{\childdocmain}[1]{\endinput}}
  \renewcommand{\childdocof}[1]{}
  \renewcommand{\childdocby}[2][]{}
  \renewcommand{\childdocforward}[2][]{}
  \renewcommand{\childdocdisable}{}
}
%    \end{macrocode}

% \macro{\childdocmain}
% The macro |\childdocmain| is to be called at the top of the main file
% with nothing or the main filename (without extension) as argument.
% First, it breaks loops.
% If the argument is not empty and does not match |\childdocname|
% (which is set by the first inclusion of |childdoc.def|),
% |\ifchilddoc| is set to true, |\includeonly| is applied to the child file
% and |\jobname| is set to the main file
% (for proper handling of |.aux| files):
%    \begin{macrocode}
\newcommand{\childdocmain}[1]
{
  \childdocdisable\childdocmain{}
  \if?#1?\else
    \begingroup
      \def\childdoctmp{#1}
      \ifx\childdoctmp\childdocname
        \def\childdoctmp{}
      \else
        \def\childdoctmp
        {
          \childdoctrue
          \includeonly{\childdocname}
          \def\childdocjob{#1}
          \def\jobname{#1}
        }
      \fi
      \expandafter
    \endgroup
    \childdoctmp
  \fi
}
%    \end{macrocode}

% \macro{\childdocof}
% The command |\childdocof| redirects
% compilation to the main file |#1|.
%    \begin{macrocode}
\newcommand{\childdocof}[1]
{
  \childdocdisable
  \childdoctrue
  \includeonly{\childdocname}
  \def\jobname{#1}
  \def\childdocjob{#1}
  \input{#1}
}
%    \end{macrocode}

% \macro{\childdocby}
% The command |\childdocby| ....
%    \begin{macrocode}
\newcommand{\childdocby}[2][]
{
  \childdocdisable
  \childdoctrue
  \childdocmanualtrue
  \if?#1?\else
    \def\jobname{#2}
  \fi
  \def\childdocjob{#2}
  \input{#2}
  \endinput
}
%    \end{macrocode}

% \macro{\childdocforward}
% The command |\childdocforward| redirects
% compilation to the main file or
% (if the optional argument is given) a child file.
% Parameters are set as if the main file
% or a child file starting with |\childdocof| was compiled.
% Then compilation is handed over to the main file:
%    \begin{macrocode}
\newcommand{\childdocforward}[2][]
{
  \begingroup
    \if?#1?
      \def\childdoctmp
      {
        \def\childdocname{#2}
        \def\childdocjob{#2}
        \def\jobname{#2}
        \input{#2}
        \endinput
      }
    \else
      \def\childdoctmp
      {
        \childdocdisable
        \def\childdocname{#2}
        \childdoctrue
        \includeonly{#2}
        \def\childdocjob{#1}
        \def\jobname{#1}
        \input{#1}
        \endinput
      }
    \fi
    \expandafter
  \endgroup
  \childdoctmp
}
%    \end{macrocode}

% \macro{\childdocforwardprefix}
% The command |\childdocforwardprefix| redirects
% compilation to the main or a child file by means of a pattern.
% The prefix |#1| in the current filename is replaced by |#2|
% and the suffix of the current filename is kept
% (it is assumed that the filename does not contain the substring `|~~~|'
% which is used as a delimiter).
% Compilation is handed over to the new file by |\childdocforward|:
%    \begin{macrocode}
\newcommand{\childdocforwardprefix}[3][]
{
  \begingroup
    \def\childdocextract #2##1~~~{\def\childdoctmp{\childdocforward[#1]{#3##1}}}
    \expandafter\childdocextract\childdocname~~~
    \expandafter
  \endgroup
  \childdoctmp
}
%    \end{macrocode}

% \macro{\childdoc}
% The deprecated macro |\childdoc| is a legacy version of |\childdocmain|:
%    \begin{macrocode}
\newcommand{\childdoc}{\childdocmain}
%    \end{macrocode}

% \macro{\childdocredirect}
% The deprecated macro |\childdocredirect| is a legacy version
% of |\childdocforward| and |\childdocforwardprefix|:
%    \begin{macrocode}
\newcommand{\childdocredirect}[2][]
{
  \begingroup
    \if?#1?
      \def\childdoctmp{\childdocforward{#2}}
    \else
      \def\childdoctmp{\childdocforwardprefix{#1}{#2}}
    \fi
    \expandafter
  \endgroup
  \childdoctmp
}
%    \end{macrocode}

%\iffalse
%</package>
%\fi
%
\endinput

\childdocof{cdocsamp}
%    \end{macrocode}

%\iffalse
%</samplechap1|samplechap2>
%\fi
%
%\iffalse
%<*samplechap1>
%\fi
% Some text for chapter 1:
%    \begin{macrocode}
\section{one}
some text in chapter one
%    \end{macrocode}

%\iffalse
%</samplechap1>
%\fi
% Some text for chapter 2:
%\iffalse
%<*samplechap2>
%\fi
%    \begin{macrocode}
\section{two}
more text in chapter two
%    \end{macrocode}

%\iffalse
%</samplechap2>
%\fi
%
% %%%%%%%%%%%%%%%%%%%%%%%%%%%%%%%%%%%%%%
% \paragraph{Part Include Files.}
%
% The include files are called |cdocspt3.tex| and |cdocspt4.tex|.
%
%\iffalse
%<*samplepart3|samplepart4>
%\fi

% Optional override for |\version| flag:
%    \begin{macrocode}
%%\providecommand{\version}{final}
%    \end{macrocode}

% Include the main document:
%    \begin{macrocode}
% \iffalse
%
% childdoc.dtx Copyright (C) 2017-2018 Niklas Beisert
%
% This work may be distributed and/or modified under the
% conditions of the LaTeX Project Public License, either version 1.3
% of this license or (at your option) any later version.
% The latest version of this license is in
%   http://www.latex-project.org/lppl.txt
% and version 1.3 or later is part of all distributions of LaTeX
% version 2005/12/01 or later.
%
% This work has the LPPL maintenance status `maintained'.
%
% The Current Maintainer of this work is Niklas Beisert.
%
% This work consists of the files childdoc.dtx and childdoc.ins
% and the derived files childdoc.def and cdocsamp.tex with
% cdocsch1.tex, cdocsch2.tex, cdocsdrf.tex, cdocsfn1.tex, cdocsfn2.tex.
%
%<package>\ifdefined\childdocmain\endinput\fi
%<package>\ProvidesFile{childdoc.def}[2018/12/30 v2.0 child document driver]
%<samplemain>\ProvidesFile{cdocsamp.tex}[2018/12/30 v2.0 sample for childdoc]
%<*driver>
%\ProvidesFile{childdoc.drv}[2018/12/30 v2.0 childdoc reference manual file]
\PassOptionsToClass{10pt,a4paper}{article}
\documentclass{ltxdoc}

\usepackage[margin=35mm]{geometry}
\usepackage{hyperref}
\usepackage{hyperxmp}
\usepackage[usenames]{color}

\hypersetup{colorlinks=true}
\hypersetup{pdfstartview=FitH}
\hypersetup{pdfpagemode=UseNone}
\hypersetup{pdfsource={}}
\hypersetup{pdflang={en-UK}}
\hypersetup{pdfcopyright={Copyright 2017-2018 Niklas Beisert.
  This work may be distributed and/or modified under the
  conditions of the LaTeX Project Public License, either version 1.3
  of this license or (at your option) any later version.}}
\hypersetup{pdflicenseurl={http://www.latex-project.org/lppl.txt}}
\hypersetup{pdfcontactaddress={ETH Zurich, ITP, HIT K,
  Wolfgang-Pauli-Strasse 27}}
\hypersetup{pdfcontactpostcode={8093}}
\hypersetup{pdfcontactcity={Zurich}}
\hypersetup{pdfcontactcountry={Switzerland}}
\hypersetup{pdfcontactemail={nbeisert@itp.phys.ethz.ch}}
\hypersetup{pdfcontacturl={http://people.phys.ethz.ch/\xmptilde nbeisert/}}

\newcommand{\secref}[1]{\hyperref[#1]{section \ref*{#1}}}

\parskip1ex
\parindent0pt
\let\olditemize\itemize
\def\itemize{\olditemize\parskip0pt}

\begin{document}

\title{The \textsf{childdoc} Package}
\hypersetup{pdftitle={The childdoc Package}}
\author{Niklas Beisert\\[2ex]
  Institut f\"ur Theoretische Physik\\
  Eidgen\"ossische Technische Hochschule Z\"urich\\
  Wolfgang-Pauli-Strasse 27, 8093 Z\"urich, Switzerland\\[1ex]
  \href{mailto:nbeisert@itp.phys.ethz.ch}
  {\texttt{nbeisert@itp.phys.ethz.ch}}}
\hypersetup{pdfauthor={Niklas Beisert}}
\hypersetup{pdfsubject={Manual for the LaTeX2e Package childdoc}}
\date{30 December 2018, \textsf{v2.0}}
\maketitle

\begin{abstract}\noindent
\textsf{childdoc} is a \LaTeXe{} package
that enables the direct compilation
of document sections included by |\include|
to individual files.
\end{abstract}

\begingroup
\parskip0ex
\tableofcontents
\endgroup

%%%%%%%%%%%%%%%%%%%%%%%%%%%%%%%%%%%%%%%%%%%%%%%%%%%%%%%%%%%%%%%%%%%%%%%%%%%%%%%%
%%%%%%%%%%%%%%%%%%%%%%%%%%%%%%%%%%%%%%%%%%%%%%%%%%%%%%%%%%%%%%%%%%%%%%%%%%%%%%%%
\section{Introduction}

\LaTeX{} provides a mechanism to structure a large document (such as a book)
into a main file and several child files (containing the chapters)
using the |\include| command.
This mechanism is beneficial for documents
which span hundreds of pages in order to
make the source file(s) more manageable.
Moreover, compilation can be restricted to
selected child files by means of the |\includeonly| command.
The latter feature can be used to reduce the compilation time while editing
(this was significantly more useful in the earlier days of \LaTeX{})
or to generate a smaller document which is easier to navigate.
Another application of |\includeonly| is to generate
documents consisting of selected parts of the complete document.

However, there are a few drawbacks of the plain |\include| mechanism:
\begin{itemize}
\item
The child files cannot be compiled on their own,
they can only be compiled via the main file.
A naive editing environment
(such as a text editor with an option
to have the current file processed by \LaTeX)
may require one to switch to the main file before compiling;
attempting to compile the child file produces errors.
\item
The main file must be modified (each time)
to adjust the |\includeonly| command
to the present needs. This easily leaves the main file in a messy state.
\item
The generated document will always carry the filename
of the main document. This is inconvenient if
several child files are to be compiled and
to be kept for distribution.
\end{itemize}

The present package provides a simple interface
to make child files individually compilable by \LaTeX{}.
Compiling a child file then has the same effect as compiling
the main file with an |\includeonly| command
to select the appropriate child.
Moreover the generated document will carry the name of the child
rather than the main file.
This resolves all three above issues.

This feature is meant to make the editing of books,
thesis documents and lecture notes somewhat more convenient.
However, the package can also be used efficiently for
composing a series of documents (such as exercise sheets)
which are typically distributed individually.
It then assists the author in generating the individual documents
(potentially in different versions)
as well as a document containing the collected series.
Another application is in developing style files
or other kinds of included material
where compilation of the style file could redirect
to a sample or test file.

%%%%%%%%%%%%%%%%%%%%%%%%%%%%%%%%%%%%%%%%%%%%%%%%%%%%%%%%%%%%%%%%%%%%%%%%%%%%%%%%
%%%%%%%%%%%%%%%%%%%%%%%%%%%%%%%%%%%%%%%%%%%%%%%%%%%%%%%%%%%%%%%%%%%%%%%%%%%%%%%%
\section{Usage}

First of all, the package \textsf{childdoc} is \emph{not} a standard
\LaTeXe{} |.sty| style file! Therefore it needs to be invoked in
a non-standard way.

%%%%%%%%%%%%%%%%%%%%%%%%%%%%%%%%%%%%%%%%%%%%%%%%%%%%%%%%%%%%%%%%%%%%%%%%%%%%%%%%
\subsection{Included Files}
\label{sec:include}

%%%%%%%%%%%%%%%%%%%%%%%%%%%%%%%%%%%%%%%%
\DescribeMacro{\childdocmain}
To use the package, add the commands
\begin{center}
\begin{tabular}{l}
|\input{childdoc.def}|\\
|\childdocmain{}|\\
\end{tabular}
\end{center}
at the very top of the main \LaTeX{} file,
in particular \emph{before} the |\documentclass| statement!
The argument of |\childdocmain| should be left empty
(but it must be present).

%%%%%%%%%%%%%%%%%%%%%%%%%%%%%%%%%%%%%%%%
\DescribeMacro{\childdocof}
Furthermore, add the commands
\begin{center}
\begin{tabular}{l}
|\input{childdoc.def}|\\
|\childdocof{|\textit{main}|}|\\
\end{tabular}
\end{center}
at the top of every child file \textit{child}
which is included by |\include{|\textit{child}|}|
from within the main file
(or at least for those files to be compiled individually).
The argument \textit{main} must be the filename of the main file.

There are a couple of
considerations in setting up the main and child documents:

%%%%%%%%%%%%%%%%%%%%%%%%%%%%%%%%%%%%%%%%
\paragraph{Restrictions.}

Please note the following restrictions:
\begin{itemize}
\item
|\childdocmain| must be called with one argument \textit{main}
to ensure compatibility with earlier version of the package.
It must either be empty (|\childdocmain{}|)
or precisely match the filename of the main file in which it is specified.
See \secref{sec:detection} for further information.
\item
The filename \textit{main} must be specified without the |.tex| extension.
\item
The filename \textit{main} is case sensitive
(even in case-insensitive file systems)
due to internal string comparison.
\item
The argument \textit{main} should be fully expanded, it cannot be a macro.
\item
Subdirectories and special characters should be avoided in filenames.
\item
The command |\childdocmain{|\textit{main}|}| must be followed by a whitespace.
It should not be followed immediately by another command
or by a comment mark `|%|'.
This is because the \TeX{} parser reads the token immediately following
the argument of |\childdocmain| and puts it
at the beginning of every child section;
however, a white\-space is ignored.
\end{itemize}

%%%%%%%%%%%%%%%%%%%%%%%%%%%%%%%%%%%%%%%%
\paragraph{Content of Main File.}

It is advisable to place all content in the child files included by |\include|.
Any output contained in the main file will appear in all child documents
unless suppressed manually;
it cannot be suppressed automatically by the |\includeonly| directive
and thus should normally be avoided.
A method to include some content in the main file
by means of conditional processing is described in \secref{sec:conditional}.

%%%%%%%%%%%%%%%%%%%%%%%%%%%%%%%%%%%%%%%%
\paragraph{Page Numbering.}

When only a part of the document is compiled,
the appropriate numbering of pages
(as well as other status parameters)
is determined from the |.aux| files.
The latter contain information from previous passes.
However this information needs to propagate through
all intermediate child documents.
Therefore the page numbering in child documents may well
be inconsistent until the complete document is compiled at least once.

A useful (if unconventional) way to always ensure a consistent
page numbering is to restart the numbering in each child document
and denote the pages by `\textit{child}|.|\textit{page}'
where \textit{child} represents the chapter/section number of the child file.
This can be achieved by the command
|\numberwithin{page}{|\textit{child}|}|
of the \textsf{amsmath} package
where \textit{child} can be |chapter| or |section|
depending on the chosen structuring.
Alternatively, one can modify the macro |\thepage| appropriately
and reset the counter |page| at the start of each child file.

%%%%%%%%%%%%%%%%%%%%%%%%%%%%%%%%%%%%%%%%%%%%%%%%%%%%%%%%%%%%%%%%%%%%%%%%%%%%%%%%
\subsection{Conditional Processing}
\label{sec:conditional}

The package provides a mechanism to compile different versions
of a document. To customise the versions further some conditional processing
can come in handy to distinguish which version is being compiled.
The package provides two macros to describe the compilation context:

%%%%%%%%%%%%%%%%%%%%%%%%%%%%%%%%%%%%%%%%
\DescribeMacro{\ifchilddoc}
The conditional |\ifchilddoc| distinguishes between the compilation of
child documents and the main document:
%
\begin{center}
|\ifchilddoc |\textit{child-code}| |[|\||else |\textit{main-code}]| \||fi|
\end{center}

%%%%%%%%%%%%%%%%%%%%%%%%%%%%%%%%%%%%%%%%
\DescribeMacro{\childdocname}
\DescribeMacro{\childdocjob}
The macro |\childdocname| contains the filename (without extension)
of the main or child file being processed.
Note that |\childdocjob| will always contain the name of the main file.

%%%%%%%%%%%%%%%%%%%%%%%%%%%%%%%%%%%%%%%%
\paragraph{Title Page.}

Conditional processing can be used to include a title or banner page
in the main document when proper precautions are taken.
Importantly, the code in the main file should ensure that the page counter
(as well as other status parameters which are stored in the |.aux| files)
takes the same value after the conditional processing.
Otherwise the page numbers may take divergent values
depending on which part is compiled.

For example, a title page could be declared by:
%
\begin{center}
\begin{tabular}{l}
|\ifchilddoc\||else|\\
|\addtocounter{page}{-1}|\\
\textit{code for title page}\\
|\newpage|\\
|\||fi|
\end{tabular}
\end{center}
%
A banner page for the child documents can be generated by:
%
\begin{center}
\begin{tabular}{l}
|\ifchilddoc|\\
|\addtocounter{page}{-1}|\\
\textit{code for banner page}\\
|\newpage|\\
|\||fi|
\end{tabular}
\end{center}
%
Here one could write a message such as:
\begin{center}
|This is the part \childdocname{} of \childdocjob{}.|
\end{center}

%%%%%%%%%%%%%%%%%%%%%%%%%%%%%%%%%%%%%%%%%%%%%%%%%%%%%%%%%%%%%%%%%%%%%%%%%%%%%%%%
\subsection{Flags}
\label{sec:flags}

The package makes it easy to generate different versions
of the main or child documents.
To this end compilation flags can be defined
and assigned different default values.
They will be particularly useful in conjunction
with the forwarding mechanism described in \secref{sec:forward}.

For example, it may be useful to have a flag |\version|
which can be set to |draft| or |final|.
The document source will contain some conditional code
depending on the value of |\version|.
Suppose further, the flag should default to |final| for the main file
and to |draft| for child files
which is a natural assignment for editing the document.
This is achieved by placing the following code
in the preamble of the main document
(below the |\childdocmain| directive):
%
\begin{center}
\begin{tabular}{l}
|\ifchilddoc|\\
|\providecommand{\version}{draft}|\\
|\||else|\\
|\providecommand{\version}{final}|\\
|\||fi|
\end{tabular}
\end{center}
%
The definition by |\providecommand| makes sure
that previous definitions are not overwritten.
Further statements |\providecommand{\version}{...}|
can thus be added before the above code to override it.

For the main file, one might add a line
(between |\childdocmain| and the above block)
%
\begin{center}
|%\ifchilddoc\||else\providecommand{\version}{draft}\||fi|
\end{center}
%
which can be uncommented to produce a draft version.
Likewise one can add a line to the very top of a child file
(above the |\childdocof{|\textit{main}|}| directive)
%
\begin{center}
|%\providecommand{\version}{final}|
\end{center}
%
which can be uncommented to produce the final version of this child document.

%%%%%%%%%%%%%%%%%%%%%%%%%%%%%%%%%%%%%%%%%%%%%%%%%%%%%%%%%%%%%%%%%%%%%%%%%%%%%%%%
\subsection{Forwarding}
\label{sec:forward}

Different versions of the main or child documents
using compilation flags as described in \secref{sec:flags}
can be (permanently) stored in different files
for convenient compilation, viewing and distribution.
To this end, the package defines a command
to pass on compilation to a different file:

%%%%%%%%%%%%%%%%%%%%%%%%%%%%%%%%%%%%%%%%
\DescribeMacro{\childdocforward}
The command |\childdocforward| redirects processing to
another source file:
%
\begin{center}
\begin{tabular}{l}
|\input{childdoc.def}|\\
|\childdocforward[|\textit{main}|]{|\textit{dest}|}|\\
\end{tabular}
\end{center}
%
The argument \textit{dest} is the destination file
(without extension).
It should be the main file or one of the child files.
Note that further \textsf{childdoc} directives
such as |\childdocof| and |\childdocforward|
in the indicated file will be processed in this form.
The optional argument \textit{main}
passes on directly to the main file \textit{main}
while pretending to compile the child \textit{dest}.
This form behaves as if \textit{dest}
issues |\childdocof{|\textit{main}|}| right away,
and no further \textsf{childdoc} directives will be processed.

%%%%%%%%%%%%%%%%%%%%%%%%%%%%%%%%%%%%%%%%
\DescribeMacro{\...prefix}
In the alternative form |\childdocforwardprefix|,
%
\begin{center}
\begin{tabular}{l}
|\input{childdoc.def}|\\
|\childdocforwardprefix[|\textit{main}|]{|\textit{prefix}|}{|\textit{dest}|}|
\end{tabular}
\end{center}
%
the destination file is determined by a pattern
depending on the current file:
To make this work, the current file must be called
`{\textit{prefix}\hspace{0.2em}\textit{suffix}}'
with \textit{prefix} matching precisely the argument.
Processing is then passed on to the file
`{\textit{dest}\hspace{0.2em}\textit{suffix}}'.
Surely, the same effect is achieved by
directly specifying the
argument `{\textit{dest}\hspace{0.2em}\textit{suffix}}'
in the first form.
However, that requires to set up a different file
for each child. With the alternative form of the command
all these files can have exactly the same content
which simplifies setting them up and maintaining them.

For example, the following file |draft.tex|
with a compilation flag |\version| as described in \secref{sec:flags}
compiles the main document as a draft:
%
\begin{center}
\begin{tabular}{l}
|\def\version{draft}|\\
|\input{childdoc.def}|\\
|\childdocforward{|\textit{main}|}|
\end{tabular}
\end{center}
%
Likewise, the following files |final|\textit{nn}|.tex|
compile the final version of the child document
|child|\textit{nn}|.tex|:
%
\begin{center}
\begin{tabular}{l}
|\def\version{final}|\\
|\input{childdoc.def}|\\
|\childdocforwardprefix{final}{child}|
\end{tabular}
\end{center}
%

Note that when several versions of a main file and/or of each child file
are to be generated, it may be convenient to set up a |Makefile| or
shell script to automatise the process.

%%%%%%%%%%%%%%%%%%%%%%%%%%%%%%%%%%%%%%%%%%%%%%%%%%%%%%%%%%%%%%%%%%%%%%%%%%%%%%%%
\subsection{Command Line Processing}
\label{sec:commandline}

The effect of redirection files can also be achieved by invoking
the \LaTeX{} compiler with a more elaborate command line.
Most conveniently this should be done as part
of a shell script or a |Makefile|.

When using \textsf{childdoc} in the main file, the following
command lines effectively perform a redirection
(note that depending on the shell being used,
backslashes may have to be doubled: `|\|' $\to$ `|\\|'):
%
\begin{center}
|... -jobname "|\textit{target}|" |\\|"|[\textit{flags}]%
|\input{childdoc.def}\childdocforward[|\textit{main}|]{|\textit{dest}|}"|
\end{center}
%
Here \textit{target} is the name of the output file,
\textit{main} is the name of the main file
and \textit{dest} is the name of the main or child file to be processed
(all filenames without extensions).
The optional argument \textit{main} can be omitted
if \textit{main} matches \textit{dest}.
Optionally, compilation \textit{flags} can be defined via |\def| commands.
This command line makes the \TeX{} engine believe
it is compiling the file \textit{target}
whose content is specified as the latter parameter.
The provided code then forwards the processing to
\textit{main} or \textit{dest} as described in \secref{sec:forward}.

%%%%%%%%%%%%%%%%%%%%%%%%%%%%%%%%%%%%%%%%%%%%%%%%%%%%%%%%%%%%%%%%%%%%%%%%%%%%%%%%
\subsection{Include by Input}
\label{sec:input}

Including child documents by |\include| has some restrictions by design.
Most notably, the content of a child document always occupies
its own set of pages; pages cannot be shared between child documents.
Usually, this behaviour makes perfect sense
because each child document contain an essential part of the document.
However, in some situations it may be desirable to compose
a document from a collection of parts
without having mandatory page breaks between then.
For this case, the package
provides a mechanism to include parts
by |\input| which can also be processed individually.
However, by construction this mechanism
requires manual handling of the content to be output.

%%%%%%%%%%%%%%%%%%%%%%%%%%%%%%%%%%%%%%%%
\DescribeMacro{\ifchilddocmanual}
The main file should be prepared as usual, see \secref{sec:include}.
However, the document body must make a distinction
between processing of an individual part and of the main document, e.g.:
%
\begin{center}
\begin{tabular}{l}
|\ifchilddocmanual|\\
|\input{\childdocname}|\\
|\||else|\\
\textit{document body with }|\input{|\textit{part}|}|\\
|\||fi|
\end{tabular}
\end{center}
%
The conditional |\ifchilddocmanual| is true whenever
a part to be included by |\input| is being compiled,
and the name of the part is stored in |\childdocname|.

%%%%%%%%%%%%%%%%%%%%%%%%%%%%%%%%%%%%%%%%
\DescribeMacro{\childdocby}
Each part to be included by |\input| should start with:
%
\begin{center}
\begin{tabular}{l}
|\input{childdoc.def}|\\
|\childdocby{|\textit{main}|}|\\
\end{tabular}
\end{center}
%
The directive |\childdocby| is similar to |\childdocof|
described in \secref{sec:include},
but the subsequent selection of content must be done manually.
To that end, both |\ifchilddoc| and |\ifchilddocmanual|
will be true upon processing of a part,
and the name of the part is stored in |\childdocname|.
Note that |\jobname| will be set to the filename of the current part
so that each part receives an individual |.aux| file
that does not interfere with the |.aux| file(s) of the main document.
This behaviour can be altered by the alternative form
|\childdocby[*]{|\textit{main}|}| (with a non-empty optional argument)
which uses the |.aux| file of the main document
by setting |\jobname| to \textit{main}.

%%%%%%%%%%%%%%%%%%%%%%%%%%%%%%%%%%%%%%%%%%%%%%%%%%%%%%%%%%%%%%%%%%%%%%%%%%%%%%%%
\subsection{Driver Development}
\label{sec:driver}

The \textsf{childdoc} mechanism can also be use for the development
of definition files such as \LaTeX{} styles or classes.
This case differs from the above setup with multiple parts
included by |\include| in that no |\includeonly| should be invoked.
This can be achieved by starting the include file
(before |\ProvidesPackage|) with:
%
\begin{center}
\begin{tabular}{l}
|\input{childdoc.def}|\\
|\childdocforward{|\textit{main}|}|\\
\end{tabular}
\end{center}
%
or alternatively with:
%
\begin{center}
\begin{tabular}{l}
|\input{childdoc.def}|\\
|\childdocby{|\textit{main}|}|\\
\end{tabular}
\end{center}
%
Both forms have slightly different effects as described above.
The main file is prepared as usual, see \secref{sec:include}.

%%%%%%%%%%%%%%%%%%%%%%%%%%%%%%%%%%%%%%%%%%%%%%%%%%%%%%%%%%%%%%%%%%%%%%%%%%%%%%%%
\subsection{Legacy Detection}
\label{sec:detection}

The directive |\childdocmain| in the main file can detect
whether the complete document or merely a child is to be compiled
even without using the directive |\childdocof|.
This method is deprecated because it is less robust
and there is no compelling reason to use it;
it is merely provided for backward compatibility
and it may be removed in future versions.

If the detection mechanism is to be used,
it is mandatory to correctly specify
the filename of the main file as the argument of |\childdocmain|:
%
\begin{center}
\begin{tabular}{l}
|\input{childdoc.def}|\\
|\childdocmain{|\textit{main}|}|\\
\end{tabular}
\end{center}
%
If |\jobname| does not match the argument \textit{main} of |\childdocmain|,
it is assumed that |\jobname| points to the child file to be compiled.
When using |\childdocmain| with the main file specified as argument,
it suffices to start a child file
with just |\input{|\textit{main}|}|
without loading of the package and using |\childdocof|.
If instead all processing is done
with the appropriate \textsf{childdoc} directives,
the argument of \textit{main} of |\childdocmain| can be empty.

An alternative version of the command line processing described
in \secref{sec:commandline} using the detection mechanism reads:
%
\begin{center}
|... -jobname "|\textit{target}|" "|[\textit{flags}]%
[|\def\jobname{|\textit{dest}|}|]|\input{|\textit{main}|}"|
\end{center}

%%%%%%%%%%%%%%%%%%%%%%%%%%%%%%%%%%%%%%%%%%%%%%%%%%%%%%%%%%%%%%%%%%%%%%%%%%%%%%%%
\subsection{Manual Code}
\label{sec:manual}

In case one cannot be certain whether the definitions file |childdoc.def|
is installed on the target \TeX{} distribution
and one prefers not to ship it,
it is conceivable to paste a few relevant commands into the sources.

To that end, drop all statements |\input{childdoc.def}|
and perform the replacements as outlined below.
Instead of |\childdocmain{|\textit{main}|}| add the following code
to the top of the main file:
%
\begin{center}
\begin{tabular}{l}
|\||ifdefined\childdocname\endinput\||fi\newif\ifchilddoc|\\
|\edef\childdocname{\scantokens\expandafter{\jobname\noexpand}}|\\
|\def\childdocmain{|\textit{main}|}\||ifx\childdocmain\childdocname\||else|\\
|\childdoctrue\includeonly{\childdocname}\let\jobname\childdocmain\||fi|\\
\end{tabular}
\end{center}
%
Instead of |\childdocof{|\textit{main}|}| just include the main file
at the top of each child file:
%
\begin{center}
|\input{|\textit{main}|}|
\end{center}
%
A simple redirection |\childdocforward{|\textit{dest}|}| is achieved by:
%
\begin{center}
|\def\jobname{|\textit{dest}|}\input{\jobname}|
\end{center}
%
The redirection with prefix
|\childdocforwardprefix[|\textit{prefix}|]{|\textit{dest}|}|
is accomplished by:
%
\begin{center}
\begin{tabular}{l}
|{\edef\jobname{\scantokens\expandafter{\jobname\noexpand}}|\\
|\def\redirectjob |\textit{prefix}|#1~~~{\gdef\jobname{|\textit{dest}|#1}}|\\
|\expandafter\redirectjob\jobname~~~}\input{\jobname}|
\end{tabular}
\end{center}

In an alternative approach,
child documents can be compiled by a specific command line
without additional code or specific definitions:
%
\begin{center}
|... -jobname "|\textit{target}|" "|[\textit{flags}]%
|\includeonly{|\textit{dest}|}\input{|\textit{main}|}"|
\end{center}
%

%%%%%%%%%%%%%%%%%%%%%%%%%%%%%%%%%%%%%%%%%%%%%%%%%%%%%%%%%%%%%%%%%%%%%%%%%%%%%%%%
%%%%%%%%%%%%%%%%%%%%%%%%%%%%%%%%%%%%%%%%%%%%%%%%%%%%%%%%%%%%%%%%%%%%%%%%%%%%%%%%
\section{Information}

%%%%%%%%%%%%%%%%%%%%%%%%%%%%%%%%%%%%%%%%%%%%%%%%%%%%%%%%%%%%%%%%%%%%%%%%%%%%%%%%
\subsection{Copyright}

Copyright \copyright{} 2017--2018 Niklas Beisert

This work may be distributed and/or modified under the
conditions of the \LaTeX{} Project Public License, either version 1.3
of this license or (at your option) any later version.
The latest version of this license is in
  \url{http://www.latex-project.org/lppl.txt}
and version 1.3 or later is part of all distributions of \LaTeX{}
version 2005/12/01 or later.

This work has the LPPL maintenance status `maintained'.

The Current Maintainer of this work is Niklas Beisert.

This work consists of the files |README.txt|, |childdoc.ins| and |childdoc.dtx|
as well as the derived files |childdoc.def|, |cdocsamp.tex|
with |cdocsch1.tex|, |cdocsch2.tex|, |cdocspt3.tex|, |cdocspt4.tex|,
|cdocsdrf.tex|, |cdocsfn1.tex|, |cdocsfn2.tex|
as well as |childdoc.pdf|.

%%%%%%%%%%%%%%%%%%%%%%%%%%%%%%%%%%%%%%%%%%%%%%%%%%%%%%%%%%%%%%%%%%%%%%%%%%%%%%%%
\subsection{Files and Installation}

The package consists of the files:
%
\begin{center}
\begin{tabular}{ll}
    |README.txt|   & readme file \\
    |childdoc.ins| & installation file \\
    |childdoc.dtx| & source file \\
    |childdoc.def| & definition file \\
    |cdocsamp.tex| & sample main file \\
    |cdocsch1.tex| & sample include file \\
    |cdocsch2.tex| & sample include file \\
    |cdocspt3.tex| & sample part file \\
    |cdocspt4.tex| & sample part file \\
    |cdocsdrf.tex| & sample redirection file \\
    |cdocsfn1.tex| & sample redirection file \\
    |cdocsfn2.tex| & sample redirection file \\
    |childdoc.pdf| & manual
\end{tabular}
\end{center}
%
The distribution consists of the files
|README.txt|, |childdoc.ins| and |childdoc.dtx|.
%
\begin{itemize}
\item
Run (pdf)\LaTeX{} on |childdoc.dtx|
to compile the manual |childdoc.pdf| (this file).
\item
Run \LaTeX{} on |childdoc.ins| to create the definitions file |childdoc.def|
and the sample |cdocsamp.tex| with include files
|cdocsch1.tex|, |cdocsch2.tex|, |cdocspt3.tex|, |cdocspt4.tex|,
|cdocsdrf.tex|, |cdocsfn1.tex|, |cdocsfn2.tex|.
Then copy the file |childdoc.def| to an appropriate directory of your \LaTeX{}
distribution, e.g.\ \textit{texmf-root}|/tex/latex/childdoc|.
\end{itemize}

%%%%%%%%%%%%%%%%%%%%%%%%%%%%%%%%%%%%%%%%%%%%%%%%%%%%%%%%%%%%%%%%%%%%%%%%%%%%%%%%
\subsection{Related CTAN Packages}

There are several other packages which offer a similar functionality:
%
\begin{itemize}
\item
The packages
\href{http://ctan.org/pkg/docmute}{\textsf{docmute}},
\href{http://ctan.org/pkg/includex}{\textsf{includex}} and
\href{http://ctan.org/pkg/standalone}{\textsf{standalone}}
provide commands to include only the document body of
a child file thus allowing both files to be compiled individually.
\item
The packages \href{http://ctan.org/pkg/subdocs}{\textsf{subdocs}}
and \href{http://ctan.org/pkg/subfiles}{\textsf{subfiles}}
provide structures in which the main and child documents can be
encapsulated and allowing them to be compiled individually.
The inclusion mechanism is different from the conventional |\include|.
\item
The package \href{http://ctan.org/pkg/combine}{\textsf{combine}}
is an elaborate solution to combine several documents into one.
\end{itemize}
%
See also the CTAN topic \href{http://ctan.org/topic/subdocs}{\textsf{subdocs}}
for further related packages.
The present package differs from the above solutions in that
a document structure constructed with the conventional |\include| mechanism
just needs two extra commands at the top of every file
such that all constituent files can be compiled individually.

%%%%%%%%%%%%%%%%%%%%%%%%%%%%%%%%%%%%%%%%%%%%%%%%%%%%%%%%%%%%%%%%%%%%%%%%%%%%%%%%
%\subsection{Feature Suggestions}
%
%The following is a list of features which may be useful for future
%versions of this package:
%%
%\begin{itemize}
%\item
%\ldots
%\end{itemize}

%%%%%%%%%%%%%%%%%%%%%%%%%%%%%%%%%%%%%%%%%%%%%%%%%%%%%%%%%%%%%%%%%%%%%%%%%%%%%%%%
\subsection{Revision History}

%%%%%%%%%%%%%%%%%%%%%%%%%%%%%%%%%%%%%%%%
\paragraph{v2.0:} 2018/12/30

\begin{itemize}
\item
immediate forward processing
\item
added |\childdocby| mechanism
\item
manual restructured
\end{itemize}

%%%%%%%%%%%%%%%%%%%%%%%%%%%%%%%%%%%%%%%%
\paragraph{v1.6:} 2018/01/17

\begin{itemize}
\item
application for development of include files
\item
corrections to manual
\end{itemize}

%%%%%%%%%%%%%%%%%%%%%%%%%%%%%%%%%%%%%%%%
\paragraph{v1.5:} 2017/05/21

\begin{itemize}
\item
more complete structuring introduced
\item
|\childdocof| introduced
\item
|\childdoc| renamed to |\childdocmain|
\item
|\childredirect| renamed to |\childdocforward| and |\childdocforwardprefix|
and functionality expanded
\end{itemize}

%%%%%%%%%%%%%%%%%%%%%%%%%%%%%%%%%%%%%%%%
\paragraph{v1.0:} 2017/04/27

\begin{itemize}
\item
manual and install package
\item
first version published on CTAN
\end{itemize}

%%%%%%%%%%%%%%%%%%%%%%%%%%%%%%%%%%%%%%%%
\paragraph{v0.6:} 2017/04/26

\begin{itemize}
\item
redirection mechanism added
\end{itemize}

%%%%%%%%%%%%%%%%%%%%%%%%%%%%%%%%%%%%%%%%
\paragraph{v0.5:} 2017/04/26

\begin{itemize}
\item
functionality in definition file
\end{itemize}


%%%%%%%%%%%%%%%%%%%%%%%%%%%%%%%%%%%%%%%%%%%%%%%%%%%%%%%%%%%%%%%%%%%%%%%%%%%%%%%%
%%%%%%%%%%%%%%%%%%%%%%%%%%%%%%%%%%%%%%%%%%%%%%%%%%%%%%%%%%%%%%%%%%%%%%%%%%%%%%%%
%%%%%%%%%%%%%%%%%%%%%%%%%%%%%%%%%%%%%%%%%%%%%%%%%%%%%%%%%%%%%%%%%%%%%%%%%%%%%%%%
\appendix

\settowidth\MacroIndent{\rmfamily\scriptsize 000\ }

 \DocInput{childdoc.dtx}

\end{document}
%</driver>
% \fi
%
% %%%%%%%%%%%%%%%%%%%%%%%%%%%%%%%%%%%%%%%%%%%%%%%%%%%%%%%%%%%%%%%%%%%%%%%%%%%%%%
% %%%%%%%%%%%%%%%%%%%%%%%%%%%%%%%%%%%%%%%%%%%%%%%%%%%%%%%%%%%%%%%%%%%%%%%%%%%%%%
% \section{Sample}
%\iffalse
%<*samplemain>
%\fi
%
% The following presents a sample document
% with two chapters, two parts, a title page,
% a compile flag as well as three forwarding files to set the flag.
% It consists of eight |.tex| files:
% \begin{center}
% \begin{tabular}{ll}
% |cdocsamp.tex|&main file\\
% |cdocsch1.tex|&include file for chapter 1\\
% |cdocsch2.tex|&include file for chapter 2\\
% |cdocspt3.tex|&include file for part 3\\
% |cdocspt4.tex|&include file for part 4\\
% |cdocsdrf.tex|&forwarding file for main file in draft mode\\
% |cdocsfi1.tex|&forwarding file for final version of chapter 1\\
% |cdocsfi2.tex|&forwarding file for final version of chapter 2\\
% \end{tabular}
% \end{center}
% Each of the eight files can be compiled directly by the \LaTeX{} compiler.
%
% %%%%%%%%%%%%%%%%%%%%%%%%%%%%%%%%%%%%%%
% \paragraph{Main File.}
%
% The main file is called |cdocsamp.tex|.
%
% Load the \textsf{childdoc} definitions and
% declare the filename for the main document:
%    \begin{macrocode}
\input{childdoc.def}
\childdocmain{}
%    \end{macrocode}

% Optional override for |\version| flag:
%    \begin{macrocode}
%%\ifchilddoc\else\providecommand{\version}{draft}\fi
%    \end{macrocode}

% Define the default values for the |\version| flag
% (|final| for the main file and |draft| for childs):
%    \begin{macrocode}
\ifchilddoc
\providecommand{\version}{draft}
\else
\providecommand{\version}{final}
\fi
%    \end{macrocode}

% Load the standard document class:
%    \begin{macrocode}
\documentclass[12pt]{article}
%    \end{macrocode}

% Start the document body:
%    \begin{macrocode}
\begin{document}
%    \end{macrocode}

% Declare a title page.
% Print title, part of document being processed and version flag:
%    \begin{macrocode}
\addtocounter{page}{-1}
\begin{center}
{\LARGE\bfseries{}childdoc example\par}
\vspace{1cm}
\ifchilddoc
\ifchilddocmanual part\else chapter\fi:
`\childdocname' of `\childdocjob'\par
\else
main document: `\childdocjob'\par
\fi
version: \version\par
\end{center}
\newpage
%    \end{macrocode}

% Manually include selected file,
% otherwise process as usual:
%    \begin{macrocode}
\ifchilddocmanual
\section*{part `\childdocname'}
\input{\childdocname}
\else
%    \end{macrocode}

% Include the two chapters:
%    \begin{macrocode}
\include{cdocsch1}
\include{cdocsch2}
%    \end{macrocode}

% Include the two parts unless only chapters should be displayed:
%    \begin{macrocode}
\ifchilddoc\else
\section{part three}
\input{cdocspt3}
\section{part four}
\input{cdocspt4}
\fi
%    \end{macrocode}

% Process as usual until here:
%    \begin{macrocode}
\fi
%    \end{macrocode}

% End of document body:
%    \begin{macrocode}
\end{document}
%    \end{macrocode}
%\iffalse
%</samplemain>
%\fi
%
% %%%%%%%%%%%%%%%%%%%%%%%%%%%%%%%%%%%%%%
% \paragraph{Chapter Include Files.}
%
% The include files are called |cdocsch1.tex| and |cdocsch2.tex|.
%
%\iffalse
%<*samplechap1|samplechap2>
%\fi

% Optional override for |\version| flag:
%    \begin{macrocode}
%%\providecommand{\version}{final}
%    \end{macrocode}

% Include the main document:
%    \begin{macrocode}
\input{childdoc.def}
\childdocof{cdocsamp}
%    \end{macrocode}

%\iffalse
%</samplechap1|samplechap2>
%\fi
%
%\iffalse
%<*samplechap1>
%\fi
% Some text for chapter 1:
%    \begin{macrocode}
\section{one}
some text in chapter one
%    \end{macrocode}

%\iffalse
%</samplechap1>
%\fi
% Some text for chapter 2:
%\iffalse
%<*samplechap2>
%\fi
%    \begin{macrocode}
\section{two}
more text in chapter two
%    \end{macrocode}

%\iffalse
%</samplechap2>
%\fi
%
% %%%%%%%%%%%%%%%%%%%%%%%%%%%%%%%%%%%%%%
% \paragraph{Part Include Files.}
%
% The include files are called |cdocspt3.tex| and |cdocspt4.tex|.
%
%\iffalse
%<*samplepart3|samplepart4>
%\fi

% Optional override for |\version| flag:
%    \begin{macrocode}
%%\providecommand{\version}{final}
%    \end{macrocode}

% Include the main document:
%    \begin{macrocode}
\input{childdoc.def}
\childdocby{cdocsamp}
%    \end{macrocode}

%\iffalse
%</samplepart3|samplepart4>
%\fi
%
%\iffalse
%<*samplepart3>
%\fi
% Some text for part 3:
%    \begin{macrocode}
some text in part three
%    \end{macrocode}

%\iffalse
%</samplepart3>
%\fi
% Some text for part 4:
%\iffalse
%<*samplepart4>
%\fi
%    \begin{macrocode}
more text in part four
%    \end{macrocode}

%\iffalse
%</samplepart4>
%\fi
%
% %%%%%%%%%%%%%%%%%%%%%%%%%%%%%%%%%%%%%%
% \paragraph{Forwarding for a Complete Draft.}
%
% The following forwarding file |cdocsdrf.tex|
% compiles the main document in draft mode:
%\iffalse
%<*sampledraft>
%\fi
%    \begin{macrocode}
\def\version{draft}
\input{childdoc.def}
\childdocforward{cdocsamp}
%    \end{macrocode}

%\iffalse
%</sampledraft>
%\fi
%
% %%%%%%%%%%%%%%%%%%%%%%%%%%%%%%%%%%%%%%
% \paragraph{Forwarding for Final Version of the Chapters.}
%
% The following forwarding files |cdocsfn1.tex| and |cdocsfn2.tex|
% (with identical content)
% compile the final versions of the child documents
% |cdocsch1.tex| and |cdocsch2.tex|, respectively:
%\iffalse
%<*samplefinal>
%\fi
%    \begin{macrocode}
\def\version{final}
\input{childdoc.def}
\childdocforwardprefix[cdocsamp]{cdocsfn}{cdocsch}
%    \end{macrocode}

%\iffalse
%</samplefinal>
%\fi
%
% %%%%%%%%%%%%%%%%%%%%%%%%%%%%%%%%%%%%%%
% \paragraph{Command Line Processing.}
%
% The following three command lines generate the output files
% |cdocscld|, |cdocscl1| and |cdocscl2|
% which should be identical to
% |cdocsdrf|, |cdocsch1| and |cdocsfn2|, respectively:
% \begin{center}
% \begin{tabular}{l}
% |latex -jobname cdocscld \|\\
% |  "\def\version{draft}\input{childdoc.def}\childdocforward{cdocsamp}"|\\
% |latex -jobname cdocscl1 \|\\
% |  "\input{childdoc.def}\childdocforward[cdocsamp]{cdocsch1}"|\\
% |latex -jobname cdocscl2 \|\\
% |  "\def\version{final}\input{childdoc.def}\childdocforward{cdocsch2}"|
% \end{tabular}
% \end{center}
% Note that the trailing backslash on each first line
% merely continues the input to the second line
% (for convenient cut ant paste).
% Furthermore, the command |latex| can be replaced by any
% of its alternative versions such as |pdflatex|.
%
% %%%%%%%%%%%%%%%%%%%%%%%%%%%%%%%%%%%%%%%%%%%%%%%%%%%%%%%%%%%%%%%%%%%%%%%%%%%%%%
% %%%%%%%%%%%%%%%%%%%%%%%%%%%%%%%%%%%%%%%%%%%%%%%%%%%%%%%%%%%%%%%%%%%%%%%%%%%%%%
% \section{Implementation}
%\iffalse
%<*package>
%\fi
%
% This section describes the definitions file |childdoc.def|.

% The definitions cannot be loaded using |\usepackage| or |\RequirePackage|
% which has a mechanism to prevent loading a style file more than once.
% When loading the definitions by means of |\input|
% multiple instances have to be prevented manually:
%\iffalse
%This code needs to be before the `\ProvidesFile' directive
%which is defined at the beginning of this file.
%Therefore it is also placed there and commented out here.
%</package>
%<*discard>
%\fi
%    \begin{macrocode}
\ifdefined\childdocmain\endinput\fi
%    \end{macrocode}
%\iffalse
%</discard>
%<*package>
%\fi
%
% \macro{\ifchilddoc}
% \macro{\ifchilddocmanual}
% The conditional |\ifchilddoc| tells whether a
% child (true) or main (false) document is being compiled.
% The conditional |\ifchilddocmanual| tells whether
% the |\includeonly| mechanism is used (false) or
% the selection of child files must be performed manually (true).
% The definitions initialise to false:
%    \begin{macrocode}
\newif\ifchilddoc
\newif\ifchilddocmanual
%    \end{macrocode}

% \macro{\childdocname}
% \macro{\childdocjob}
% The macro |\childdocname| stores the name of the main document
% to be compiled. The macro |\childdocjob| stores the name of
% the document on which the \LaTeX{} compiler was originally invoked.
% The content of |\jobname| cannot be compared
% to filenames specified in the source due to different catcodes.
% The following code rescans |\jobname|, stores the result
% in |\childdocname| and saves a copy in |\childdocjob|:
%    \begin{macrocode}
\edef\childdocname{\scantokens\expandafter{\jobname\noexpand}}
\let\childdocjob\childdocname
%    \end{macrocode}

% \macro{\childdocdisable}
% The macro |\childdocdisable| prevents the main file
% from being processed more than once.
% At this stage, the main document command |\childdocmain|
% is assumed to be called once again where it should do nothing.
% Any subsequent call to it should prevent
% a secondary processing of the main document
% It overwrites the forwarding commands
% |\childdocof| and |\childdocforward|
% with empty macros to prevent further inclusions of the main document:
%    \begin{macrocode}
\newcommand{\childdocdisable}
{
  \renewcommand{\childdocmain}[1]{\renewcommand{\childdocmain}[1]{\endinput}}
  \renewcommand{\childdocof}[1]{}
  \renewcommand{\childdocby}[2][]{}
  \renewcommand{\childdocforward}[2][]{}
  \renewcommand{\childdocdisable}{}
}
%    \end{macrocode}

% \macro{\childdocmain}
% The macro |\childdocmain| is to be called at the top of the main file
% with nothing or the main filename (without extension) as argument.
% First, it breaks loops.
% If the argument is not empty and does not match |\childdocname|
% (which is set by the first inclusion of |childdoc.def|),
% |\ifchilddoc| is set to true, |\includeonly| is applied to the child file
% and |\jobname| is set to the main file
% (for proper handling of |.aux| files):
%    \begin{macrocode}
\newcommand{\childdocmain}[1]
{
  \childdocdisable\childdocmain{}
  \if?#1?\else
    \begingroup
      \def\childdoctmp{#1}
      \ifx\childdoctmp\childdocname
        \def\childdoctmp{}
      \else
        \def\childdoctmp
        {
          \childdoctrue
          \includeonly{\childdocname}
          \def\childdocjob{#1}
          \def\jobname{#1}
        }
      \fi
      \expandafter
    \endgroup
    \childdoctmp
  \fi
}
%    \end{macrocode}

% \macro{\childdocof}
% The command |\childdocof| redirects
% compilation to the main file |#1|.
%    \begin{macrocode}
\newcommand{\childdocof}[1]
{
  \childdocdisable
  \childdoctrue
  \includeonly{\childdocname}
  \def\jobname{#1}
  \def\childdocjob{#1}
  \input{#1}
}
%    \end{macrocode}

% \macro{\childdocby}
% The command |\childdocby| ....
%    \begin{macrocode}
\newcommand{\childdocby}[2][]
{
  \childdocdisable
  \childdoctrue
  \childdocmanualtrue
  \if?#1?\else
    \def\jobname{#2}
  \fi
  \def\childdocjob{#2}
  \input{#2}
  \endinput
}
%    \end{macrocode}

% \macro{\childdocforward}
% The command |\childdocforward| redirects
% compilation to the main file or
% (if the optional argument is given) a child file.
% Parameters are set as if the main file
% or a child file starting with |\childdocof| was compiled.
% Then compilation is handed over to the main file:
%    \begin{macrocode}
\newcommand{\childdocforward}[2][]
{
  \begingroup
    \if?#1?
      \def\childdoctmp
      {
        \def\childdocname{#2}
        \def\childdocjob{#2}
        \def\jobname{#2}
        \input{#2}
        \endinput
      }
    \else
      \def\childdoctmp
      {
        \childdocdisable
        \def\childdocname{#2}
        \childdoctrue
        \includeonly{#2}
        \def\childdocjob{#1}
        \def\jobname{#1}
        \input{#1}
        \endinput
      }
    \fi
    \expandafter
  \endgroup
  \childdoctmp
}
%    \end{macrocode}

% \macro{\childdocforwardprefix}
% The command |\childdocforwardprefix| redirects
% compilation to the main or a child file by means of a pattern.
% The prefix |#1| in the current filename is replaced by |#2|
% and the suffix of the current filename is kept
% (it is assumed that the filename does not contain the substring `|~~~|'
% which is used as a delimiter).
% Compilation is handed over to the new file by |\childdocforward|:
%    \begin{macrocode}
\newcommand{\childdocforwardprefix}[3][]
{
  \begingroup
    \def\childdocextract #2##1~~~{\def\childdoctmp{\childdocforward[#1]{#3##1}}}
    \expandafter\childdocextract\childdocname~~~
    \expandafter
  \endgroup
  \childdoctmp
}
%    \end{macrocode}

% \macro{\childdoc}
% The deprecated macro |\childdoc| is a legacy version of |\childdocmain|:
%    \begin{macrocode}
\newcommand{\childdoc}{\childdocmain}
%    \end{macrocode}

% \macro{\childdocredirect}
% The deprecated macro |\childdocredirect| is a legacy version
% of |\childdocforward| and |\childdocforwardprefix|:
%    \begin{macrocode}
\newcommand{\childdocredirect}[2][]
{
  \begingroup
    \if?#1?
      \def\childdoctmp{\childdocforward{#2}}
    \else
      \def\childdoctmp{\childdocforwardprefix{#1}{#2}}
    \fi
    \expandafter
  \endgroup
  \childdoctmp
}
%    \end{macrocode}

%\iffalse
%</package>
%\fi
%
\endinput

\childdocby{cdocsamp}
%    \end{macrocode}

%\iffalse
%</samplepart3|samplepart4>
%\fi
%
%\iffalse
%<*samplepart3>
%\fi
% Some text for part 3:
%    \begin{macrocode}
some text in part three
%    \end{macrocode}

%\iffalse
%</samplepart3>
%\fi
% Some text for part 4:
%\iffalse
%<*samplepart4>
%\fi
%    \begin{macrocode}
more text in part four
%    \end{macrocode}

%\iffalse
%</samplepart4>
%\fi
%
% %%%%%%%%%%%%%%%%%%%%%%%%%%%%%%%%%%%%%%
% \paragraph{Forwarding for a Complete Draft.}
%
% The following forwarding file |cdocsdrf.tex|
% compiles the main document in draft mode:
%\iffalse
%<*sampledraft>
%\fi
%    \begin{macrocode}
\def\version{draft}
% \iffalse
%
% childdoc.dtx Copyright (C) 2017-2018 Niklas Beisert
%
% This work may be distributed and/or modified under the
% conditions of the LaTeX Project Public License, either version 1.3
% of this license or (at your option) any later version.
% The latest version of this license is in
%   http://www.latex-project.org/lppl.txt
% and version 1.3 or later is part of all distributions of LaTeX
% version 2005/12/01 or later.
%
% This work has the LPPL maintenance status `maintained'.
%
% The Current Maintainer of this work is Niklas Beisert.
%
% This work consists of the files childdoc.dtx and childdoc.ins
% and the derived files childdoc.def and cdocsamp.tex with
% cdocsch1.tex, cdocsch2.tex, cdocsdrf.tex, cdocsfn1.tex, cdocsfn2.tex.
%
%<package>\ifdefined\childdocmain\endinput\fi
%<package>\ProvidesFile{childdoc.def}[2018/12/30 v2.0 child document driver]
%<samplemain>\ProvidesFile{cdocsamp.tex}[2018/12/30 v2.0 sample for childdoc]
%<*driver>
%\ProvidesFile{childdoc.drv}[2018/12/30 v2.0 childdoc reference manual file]
\PassOptionsToClass{10pt,a4paper}{article}
\documentclass{ltxdoc}

\usepackage[margin=35mm]{geometry}
\usepackage{hyperref}
\usepackage{hyperxmp}
\usepackage[usenames]{color}

\hypersetup{colorlinks=true}
\hypersetup{pdfstartview=FitH}
\hypersetup{pdfpagemode=UseNone}
\hypersetup{pdfsource={}}
\hypersetup{pdflang={en-UK}}
\hypersetup{pdfcopyright={Copyright 2017-2018 Niklas Beisert.
  This work may be distributed and/or modified under the
  conditions of the LaTeX Project Public License, either version 1.3
  of this license or (at your option) any later version.}}
\hypersetup{pdflicenseurl={http://www.latex-project.org/lppl.txt}}
\hypersetup{pdfcontactaddress={ETH Zurich, ITP, HIT K,
  Wolfgang-Pauli-Strasse 27}}
\hypersetup{pdfcontactpostcode={8093}}
\hypersetup{pdfcontactcity={Zurich}}
\hypersetup{pdfcontactcountry={Switzerland}}
\hypersetup{pdfcontactemail={nbeisert@itp.phys.ethz.ch}}
\hypersetup{pdfcontacturl={http://people.phys.ethz.ch/\xmptilde nbeisert/}}

\newcommand{\secref}[1]{\hyperref[#1]{section \ref*{#1}}}

\parskip1ex
\parindent0pt
\let\olditemize\itemize
\def\itemize{\olditemize\parskip0pt}

\begin{document}

\title{The \textsf{childdoc} Package}
\hypersetup{pdftitle={The childdoc Package}}
\author{Niklas Beisert\\[2ex]
  Institut f\"ur Theoretische Physik\\
  Eidgen\"ossische Technische Hochschule Z\"urich\\
  Wolfgang-Pauli-Strasse 27, 8093 Z\"urich, Switzerland\\[1ex]
  \href{mailto:nbeisert@itp.phys.ethz.ch}
  {\texttt{nbeisert@itp.phys.ethz.ch}}}
\hypersetup{pdfauthor={Niklas Beisert}}
\hypersetup{pdfsubject={Manual for the LaTeX2e Package childdoc}}
\date{30 December 2018, \textsf{v2.0}}
\maketitle

\begin{abstract}\noindent
\textsf{childdoc} is a \LaTeXe{} package
that enables the direct compilation
of document sections included by |\include|
to individual files.
\end{abstract}

\begingroup
\parskip0ex
\tableofcontents
\endgroup

%%%%%%%%%%%%%%%%%%%%%%%%%%%%%%%%%%%%%%%%%%%%%%%%%%%%%%%%%%%%%%%%%%%%%%%%%%%%%%%%
%%%%%%%%%%%%%%%%%%%%%%%%%%%%%%%%%%%%%%%%%%%%%%%%%%%%%%%%%%%%%%%%%%%%%%%%%%%%%%%%
\section{Introduction}

\LaTeX{} provides a mechanism to structure a large document (such as a book)
into a main file and several child files (containing the chapters)
using the |\include| command.
This mechanism is beneficial for documents
which span hundreds of pages in order to
make the source file(s) more manageable.
Moreover, compilation can be restricted to
selected child files by means of the |\includeonly| command.
The latter feature can be used to reduce the compilation time while editing
(this was significantly more useful in the earlier days of \LaTeX{})
or to generate a smaller document which is easier to navigate.
Another application of |\includeonly| is to generate
documents consisting of selected parts of the complete document.

However, there are a few drawbacks of the plain |\include| mechanism:
\begin{itemize}
\item
The child files cannot be compiled on their own,
they can only be compiled via the main file.
A naive editing environment
(such as a text editor with an option
to have the current file processed by \LaTeX)
may require one to switch to the main file before compiling;
attempting to compile the child file produces errors.
\item
The main file must be modified (each time)
to adjust the |\includeonly| command
to the present needs. This easily leaves the main file in a messy state.
\item
The generated document will always carry the filename
of the main document. This is inconvenient if
several child files are to be compiled and
to be kept for distribution.
\end{itemize}

The present package provides a simple interface
to make child files individually compilable by \LaTeX{}.
Compiling a child file then has the same effect as compiling
the main file with an |\includeonly| command
to select the appropriate child.
Moreover the generated document will carry the name of the child
rather than the main file.
This resolves all three above issues.

This feature is meant to make the editing of books,
thesis documents and lecture notes somewhat more convenient.
However, the package can also be used efficiently for
composing a series of documents (such as exercise sheets)
which are typically distributed individually.
It then assists the author in generating the individual documents
(potentially in different versions)
as well as a document containing the collected series.
Another application is in developing style files
or other kinds of included material
where compilation of the style file could redirect
to a sample or test file.

%%%%%%%%%%%%%%%%%%%%%%%%%%%%%%%%%%%%%%%%%%%%%%%%%%%%%%%%%%%%%%%%%%%%%%%%%%%%%%%%
%%%%%%%%%%%%%%%%%%%%%%%%%%%%%%%%%%%%%%%%%%%%%%%%%%%%%%%%%%%%%%%%%%%%%%%%%%%%%%%%
\section{Usage}

First of all, the package \textsf{childdoc} is \emph{not} a standard
\LaTeXe{} |.sty| style file! Therefore it needs to be invoked in
a non-standard way.

%%%%%%%%%%%%%%%%%%%%%%%%%%%%%%%%%%%%%%%%%%%%%%%%%%%%%%%%%%%%%%%%%%%%%%%%%%%%%%%%
\subsection{Included Files}
\label{sec:include}

%%%%%%%%%%%%%%%%%%%%%%%%%%%%%%%%%%%%%%%%
\DescribeMacro{\childdocmain}
To use the package, add the commands
\begin{center}
\begin{tabular}{l}
|\input{childdoc.def}|\\
|\childdocmain{}|\\
\end{tabular}
\end{center}
at the very top of the main \LaTeX{} file,
in particular \emph{before} the |\documentclass| statement!
The argument of |\childdocmain| should be left empty
(but it must be present).

%%%%%%%%%%%%%%%%%%%%%%%%%%%%%%%%%%%%%%%%
\DescribeMacro{\childdocof}
Furthermore, add the commands
\begin{center}
\begin{tabular}{l}
|\input{childdoc.def}|\\
|\childdocof{|\textit{main}|}|\\
\end{tabular}
\end{center}
at the top of every child file \textit{child}
which is included by |\include{|\textit{child}|}|
from within the main file
(or at least for those files to be compiled individually).
The argument \textit{main} must be the filename of the main file.

There are a couple of
considerations in setting up the main and child documents:

%%%%%%%%%%%%%%%%%%%%%%%%%%%%%%%%%%%%%%%%
\paragraph{Restrictions.}

Please note the following restrictions:
\begin{itemize}
\item
|\childdocmain| must be called with one argument \textit{main}
to ensure compatibility with earlier version of the package.
It must either be empty (|\childdocmain{}|)
or precisely match the filename of the main file in which it is specified.
See \secref{sec:detection} for further information.
\item
The filename \textit{main} must be specified without the |.tex| extension.
\item
The filename \textit{main} is case sensitive
(even in case-insensitive file systems)
due to internal string comparison.
\item
The argument \textit{main} should be fully expanded, it cannot be a macro.
\item
Subdirectories and special characters should be avoided in filenames.
\item
The command |\childdocmain{|\textit{main}|}| must be followed by a whitespace.
It should not be followed immediately by another command
or by a comment mark `|%|'.
This is because the \TeX{} parser reads the token immediately following
the argument of |\childdocmain| and puts it
at the beginning of every child section;
however, a white\-space is ignored.
\end{itemize}

%%%%%%%%%%%%%%%%%%%%%%%%%%%%%%%%%%%%%%%%
\paragraph{Content of Main File.}

It is advisable to place all content in the child files included by |\include|.
Any output contained in the main file will appear in all child documents
unless suppressed manually;
it cannot be suppressed automatically by the |\includeonly| directive
and thus should normally be avoided.
A method to include some content in the main file
by means of conditional processing is described in \secref{sec:conditional}.

%%%%%%%%%%%%%%%%%%%%%%%%%%%%%%%%%%%%%%%%
\paragraph{Page Numbering.}

When only a part of the document is compiled,
the appropriate numbering of pages
(as well as other status parameters)
is determined from the |.aux| files.
The latter contain information from previous passes.
However this information needs to propagate through
all intermediate child documents.
Therefore the page numbering in child documents may well
be inconsistent until the complete document is compiled at least once.

A useful (if unconventional) way to always ensure a consistent
page numbering is to restart the numbering in each child document
and denote the pages by `\textit{child}|.|\textit{page}'
where \textit{child} represents the chapter/section number of the child file.
This can be achieved by the command
|\numberwithin{page}{|\textit{child}|}|
of the \textsf{amsmath} package
where \textit{child} can be |chapter| or |section|
depending on the chosen structuring.
Alternatively, one can modify the macro |\thepage| appropriately
and reset the counter |page| at the start of each child file.

%%%%%%%%%%%%%%%%%%%%%%%%%%%%%%%%%%%%%%%%%%%%%%%%%%%%%%%%%%%%%%%%%%%%%%%%%%%%%%%%
\subsection{Conditional Processing}
\label{sec:conditional}

The package provides a mechanism to compile different versions
of a document. To customise the versions further some conditional processing
can come in handy to distinguish which version is being compiled.
The package provides two macros to describe the compilation context:

%%%%%%%%%%%%%%%%%%%%%%%%%%%%%%%%%%%%%%%%
\DescribeMacro{\ifchilddoc}
The conditional |\ifchilddoc| distinguishes between the compilation of
child documents and the main document:
%
\begin{center}
|\ifchilddoc |\textit{child-code}| |[|\||else |\textit{main-code}]| \||fi|
\end{center}

%%%%%%%%%%%%%%%%%%%%%%%%%%%%%%%%%%%%%%%%
\DescribeMacro{\childdocname}
\DescribeMacro{\childdocjob}
The macro |\childdocname| contains the filename (without extension)
of the main or child file being processed.
Note that |\childdocjob| will always contain the name of the main file.

%%%%%%%%%%%%%%%%%%%%%%%%%%%%%%%%%%%%%%%%
\paragraph{Title Page.}

Conditional processing can be used to include a title or banner page
in the main document when proper precautions are taken.
Importantly, the code in the main file should ensure that the page counter
(as well as other status parameters which are stored in the |.aux| files)
takes the same value after the conditional processing.
Otherwise the page numbers may take divergent values
depending on which part is compiled.

For example, a title page could be declared by:
%
\begin{center}
\begin{tabular}{l}
|\ifchilddoc\||else|\\
|\addtocounter{page}{-1}|\\
\textit{code for title page}\\
|\newpage|\\
|\||fi|
\end{tabular}
\end{center}
%
A banner page for the child documents can be generated by:
%
\begin{center}
\begin{tabular}{l}
|\ifchilddoc|\\
|\addtocounter{page}{-1}|\\
\textit{code for banner page}\\
|\newpage|\\
|\||fi|
\end{tabular}
\end{center}
%
Here one could write a message such as:
\begin{center}
|This is the part \childdocname{} of \childdocjob{}.|
\end{center}

%%%%%%%%%%%%%%%%%%%%%%%%%%%%%%%%%%%%%%%%%%%%%%%%%%%%%%%%%%%%%%%%%%%%%%%%%%%%%%%%
\subsection{Flags}
\label{sec:flags}

The package makes it easy to generate different versions
of the main or child documents.
To this end compilation flags can be defined
and assigned different default values.
They will be particularly useful in conjunction
with the forwarding mechanism described in \secref{sec:forward}.

For example, it may be useful to have a flag |\version|
which can be set to |draft| or |final|.
The document source will contain some conditional code
depending on the value of |\version|.
Suppose further, the flag should default to |final| for the main file
and to |draft| for child files
which is a natural assignment for editing the document.
This is achieved by placing the following code
in the preamble of the main document
(below the |\childdocmain| directive):
%
\begin{center}
\begin{tabular}{l}
|\ifchilddoc|\\
|\providecommand{\version}{draft}|\\
|\||else|\\
|\providecommand{\version}{final}|\\
|\||fi|
\end{tabular}
\end{center}
%
The definition by |\providecommand| makes sure
that previous definitions are not overwritten.
Further statements |\providecommand{\version}{...}|
can thus be added before the above code to override it.

For the main file, one might add a line
(between |\childdocmain| and the above block)
%
\begin{center}
|%\ifchilddoc\||else\providecommand{\version}{draft}\||fi|
\end{center}
%
which can be uncommented to produce a draft version.
Likewise one can add a line to the very top of a child file
(above the |\childdocof{|\textit{main}|}| directive)
%
\begin{center}
|%\providecommand{\version}{final}|
\end{center}
%
which can be uncommented to produce the final version of this child document.

%%%%%%%%%%%%%%%%%%%%%%%%%%%%%%%%%%%%%%%%%%%%%%%%%%%%%%%%%%%%%%%%%%%%%%%%%%%%%%%%
\subsection{Forwarding}
\label{sec:forward}

Different versions of the main or child documents
using compilation flags as described in \secref{sec:flags}
can be (permanently) stored in different files
for convenient compilation, viewing and distribution.
To this end, the package defines a command
to pass on compilation to a different file:

%%%%%%%%%%%%%%%%%%%%%%%%%%%%%%%%%%%%%%%%
\DescribeMacro{\childdocforward}
The command |\childdocforward| redirects processing to
another source file:
%
\begin{center}
\begin{tabular}{l}
|\input{childdoc.def}|\\
|\childdocforward[|\textit{main}|]{|\textit{dest}|}|\\
\end{tabular}
\end{center}
%
The argument \textit{dest} is the destination file
(without extension).
It should be the main file or one of the child files.
Note that further \textsf{childdoc} directives
such as |\childdocof| and |\childdocforward|
in the indicated file will be processed in this form.
The optional argument \textit{main}
passes on directly to the main file \textit{main}
while pretending to compile the child \textit{dest}.
This form behaves as if \textit{dest}
issues |\childdocof{|\textit{main}|}| right away,
and no further \textsf{childdoc} directives will be processed.

%%%%%%%%%%%%%%%%%%%%%%%%%%%%%%%%%%%%%%%%
\DescribeMacro{\...prefix}
In the alternative form |\childdocforwardprefix|,
%
\begin{center}
\begin{tabular}{l}
|\input{childdoc.def}|\\
|\childdocforwardprefix[|\textit{main}|]{|\textit{prefix}|}{|\textit{dest}|}|
\end{tabular}
\end{center}
%
the destination file is determined by a pattern
depending on the current file:
To make this work, the current file must be called
`{\textit{prefix}\hspace{0.2em}\textit{suffix}}'
with \textit{prefix} matching precisely the argument.
Processing is then passed on to the file
`{\textit{dest}\hspace{0.2em}\textit{suffix}}'.
Surely, the same effect is achieved by
directly specifying the
argument `{\textit{dest}\hspace{0.2em}\textit{suffix}}'
in the first form.
However, that requires to set up a different file
for each child. With the alternative form of the command
all these files can have exactly the same content
which simplifies setting them up and maintaining them.

For example, the following file |draft.tex|
with a compilation flag |\version| as described in \secref{sec:flags}
compiles the main document as a draft:
%
\begin{center}
\begin{tabular}{l}
|\def\version{draft}|\\
|\input{childdoc.def}|\\
|\childdocforward{|\textit{main}|}|
\end{tabular}
\end{center}
%
Likewise, the following files |final|\textit{nn}|.tex|
compile the final version of the child document
|child|\textit{nn}|.tex|:
%
\begin{center}
\begin{tabular}{l}
|\def\version{final}|\\
|\input{childdoc.def}|\\
|\childdocforwardprefix{final}{child}|
\end{tabular}
\end{center}
%

Note that when several versions of a main file and/or of each child file
are to be generated, it may be convenient to set up a |Makefile| or
shell script to automatise the process.

%%%%%%%%%%%%%%%%%%%%%%%%%%%%%%%%%%%%%%%%%%%%%%%%%%%%%%%%%%%%%%%%%%%%%%%%%%%%%%%%
\subsection{Command Line Processing}
\label{sec:commandline}

The effect of redirection files can also be achieved by invoking
the \LaTeX{} compiler with a more elaborate command line.
Most conveniently this should be done as part
of a shell script or a |Makefile|.

When using \textsf{childdoc} in the main file, the following
command lines effectively perform a redirection
(note that depending on the shell being used,
backslashes may have to be doubled: `|\|' $\to$ `|\\|'):
%
\begin{center}
|... -jobname "|\textit{target}|" |\\|"|[\textit{flags}]%
|\input{childdoc.def}\childdocforward[|\textit{main}|]{|\textit{dest}|}"|
\end{center}
%
Here \textit{target} is the name of the output file,
\textit{main} is the name of the main file
and \textit{dest} is the name of the main or child file to be processed
(all filenames without extensions).
The optional argument \textit{main} can be omitted
if \textit{main} matches \textit{dest}.
Optionally, compilation \textit{flags} can be defined via |\def| commands.
This command line makes the \TeX{} engine believe
it is compiling the file \textit{target}
whose content is specified as the latter parameter.
The provided code then forwards the processing to
\textit{main} or \textit{dest} as described in \secref{sec:forward}.

%%%%%%%%%%%%%%%%%%%%%%%%%%%%%%%%%%%%%%%%%%%%%%%%%%%%%%%%%%%%%%%%%%%%%%%%%%%%%%%%
\subsection{Include by Input}
\label{sec:input}

Including child documents by |\include| has some restrictions by design.
Most notably, the content of a child document always occupies
its own set of pages; pages cannot be shared between child documents.
Usually, this behaviour makes perfect sense
because each child document contain an essential part of the document.
However, in some situations it may be desirable to compose
a document from a collection of parts
without having mandatory page breaks between then.
For this case, the package
provides a mechanism to include parts
by |\input| which can also be processed individually.
However, by construction this mechanism
requires manual handling of the content to be output.

%%%%%%%%%%%%%%%%%%%%%%%%%%%%%%%%%%%%%%%%
\DescribeMacro{\ifchilddocmanual}
The main file should be prepared as usual, see \secref{sec:include}.
However, the document body must make a distinction
between processing of an individual part and of the main document, e.g.:
%
\begin{center}
\begin{tabular}{l}
|\ifchilddocmanual|\\
|\input{\childdocname}|\\
|\||else|\\
\textit{document body with }|\input{|\textit{part}|}|\\
|\||fi|
\end{tabular}
\end{center}
%
The conditional |\ifchilddocmanual| is true whenever
a part to be included by |\input| is being compiled,
and the name of the part is stored in |\childdocname|.

%%%%%%%%%%%%%%%%%%%%%%%%%%%%%%%%%%%%%%%%
\DescribeMacro{\childdocby}
Each part to be included by |\input| should start with:
%
\begin{center}
\begin{tabular}{l}
|\input{childdoc.def}|\\
|\childdocby{|\textit{main}|}|\\
\end{tabular}
\end{center}
%
The directive |\childdocby| is similar to |\childdocof|
described in \secref{sec:include},
but the subsequent selection of content must be done manually.
To that end, both |\ifchilddoc| and |\ifchilddocmanual|
will be true upon processing of a part,
and the name of the part is stored in |\childdocname|.
Note that |\jobname| will be set to the filename of the current part
so that each part receives an individual |.aux| file
that does not interfere with the |.aux| file(s) of the main document.
This behaviour can be altered by the alternative form
|\childdocby[*]{|\textit{main}|}| (with a non-empty optional argument)
which uses the |.aux| file of the main document
by setting |\jobname| to \textit{main}.

%%%%%%%%%%%%%%%%%%%%%%%%%%%%%%%%%%%%%%%%%%%%%%%%%%%%%%%%%%%%%%%%%%%%%%%%%%%%%%%%
\subsection{Driver Development}
\label{sec:driver}

The \textsf{childdoc} mechanism can also be use for the development
of definition files such as \LaTeX{} styles or classes.
This case differs from the above setup with multiple parts
included by |\include| in that no |\includeonly| should be invoked.
This can be achieved by starting the include file
(before |\ProvidesPackage|) with:
%
\begin{center}
\begin{tabular}{l}
|\input{childdoc.def}|\\
|\childdocforward{|\textit{main}|}|\\
\end{tabular}
\end{center}
%
or alternatively with:
%
\begin{center}
\begin{tabular}{l}
|\input{childdoc.def}|\\
|\childdocby{|\textit{main}|}|\\
\end{tabular}
\end{center}
%
Both forms have slightly different effects as described above.
The main file is prepared as usual, see \secref{sec:include}.

%%%%%%%%%%%%%%%%%%%%%%%%%%%%%%%%%%%%%%%%%%%%%%%%%%%%%%%%%%%%%%%%%%%%%%%%%%%%%%%%
\subsection{Legacy Detection}
\label{sec:detection}

The directive |\childdocmain| in the main file can detect
whether the complete document or merely a child is to be compiled
even without using the directive |\childdocof|.
This method is deprecated because it is less robust
and there is no compelling reason to use it;
it is merely provided for backward compatibility
and it may be removed in future versions.

If the detection mechanism is to be used,
it is mandatory to correctly specify
the filename of the main file as the argument of |\childdocmain|:
%
\begin{center}
\begin{tabular}{l}
|\input{childdoc.def}|\\
|\childdocmain{|\textit{main}|}|\\
\end{tabular}
\end{center}
%
If |\jobname| does not match the argument \textit{main} of |\childdocmain|,
it is assumed that |\jobname| points to the child file to be compiled.
When using |\childdocmain| with the main file specified as argument,
it suffices to start a child file
with just |\input{|\textit{main}|}|
without loading of the package and using |\childdocof|.
If instead all processing is done
with the appropriate \textsf{childdoc} directives,
the argument of \textit{main} of |\childdocmain| can be empty.

An alternative version of the command line processing described
in \secref{sec:commandline} using the detection mechanism reads:
%
\begin{center}
|... -jobname "|\textit{target}|" "|[\textit{flags}]%
[|\def\jobname{|\textit{dest}|}|]|\input{|\textit{main}|}"|
\end{center}

%%%%%%%%%%%%%%%%%%%%%%%%%%%%%%%%%%%%%%%%%%%%%%%%%%%%%%%%%%%%%%%%%%%%%%%%%%%%%%%%
\subsection{Manual Code}
\label{sec:manual}

In case one cannot be certain whether the definitions file |childdoc.def|
is installed on the target \TeX{} distribution
and one prefers not to ship it,
it is conceivable to paste a few relevant commands into the sources.

To that end, drop all statements |\input{childdoc.def}|
and perform the replacements as outlined below.
Instead of |\childdocmain{|\textit{main}|}| add the following code
to the top of the main file:
%
\begin{center}
\begin{tabular}{l}
|\||ifdefined\childdocname\endinput\||fi\newif\ifchilddoc|\\
|\edef\childdocname{\scantokens\expandafter{\jobname\noexpand}}|\\
|\def\childdocmain{|\textit{main}|}\||ifx\childdocmain\childdocname\||else|\\
|\childdoctrue\includeonly{\childdocname}\let\jobname\childdocmain\||fi|\\
\end{tabular}
\end{center}
%
Instead of |\childdocof{|\textit{main}|}| just include the main file
at the top of each child file:
%
\begin{center}
|\input{|\textit{main}|}|
\end{center}
%
A simple redirection |\childdocforward{|\textit{dest}|}| is achieved by:
%
\begin{center}
|\def\jobname{|\textit{dest}|}\input{\jobname}|
\end{center}
%
The redirection with prefix
|\childdocforwardprefix[|\textit{prefix}|]{|\textit{dest}|}|
is accomplished by:
%
\begin{center}
\begin{tabular}{l}
|{\edef\jobname{\scantokens\expandafter{\jobname\noexpand}}|\\
|\def\redirectjob |\textit{prefix}|#1~~~{\gdef\jobname{|\textit{dest}|#1}}|\\
|\expandafter\redirectjob\jobname~~~}\input{\jobname}|
\end{tabular}
\end{center}

In an alternative approach,
child documents can be compiled by a specific command line
without additional code or specific definitions:
%
\begin{center}
|... -jobname "|\textit{target}|" "|[\textit{flags}]%
|\includeonly{|\textit{dest}|}\input{|\textit{main}|}"|
\end{center}
%

%%%%%%%%%%%%%%%%%%%%%%%%%%%%%%%%%%%%%%%%%%%%%%%%%%%%%%%%%%%%%%%%%%%%%%%%%%%%%%%%
%%%%%%%%%%%%%%%%%%%%%%%%%%%%%%%%%%%%%%%%%%%%%%%%%%%%%%%%%%%%%%%%%%%%%%%%%%%%%%%%
\section{Information}

%%%%%%%%%%%%%%%%%%%%%%%%%%%%%%%%%%%%%%%%%%%%%%%%%%%%%%%%%%%%%%%%%%%%%%%%%%%%%%%%
\subsection{Copyright}

Copyright \copyright{} 2017--2018 Niklas Beisert

This work may be distributed and/or modified under the
conditions of the \LaTeX{} Project Public License, either version 1.3
of this license or (at your option) any later version.
The latest version of this license is in
  \url{http://www.latex-project.org/lppl.txt}
and version 1.3 or later is part of all distributions of \LaTeX{}
version 2005/12/01 or later.

This work has the LPPL maintenance status `maintained'.

The Current Maintainer of this work is Niklas Beisert.

This work consists of the files |README.txt|, |childdoc.ins| and |childdoc.dtx|
as well as the derived files |childdoc.def|, |cdocsamp.tex|
with |cdocsch1.tex|, |cdocsch2.tex|, |cdocspt3.tex|, |cdocspt4.tex|,
|cdocsdrf.tex|, |cdocsfn1.tex|, |cdocsfn2.tex|
as well as |childdoc.pdf|.

%%%%%%%%%%%%%%%%%%%%%%%%%%%%%%%%%%%%%%%%%%%%%%%%%%%%%%%%%%%%%%%%%%%%%%%%%%%%%%%%
\subsection{Files and Installation}

The package consists of the files:
%
\begin{center}
\begin{tabular}{ll}
    |README.txt|   & readme file \\
    |childdoc.ins| & installation file \\
    |childdoc.dtx| & source file \\
    |childdoc.def| & definition file \\
    |cdocsamp.tex| & sample main file \\
    |cdocsch1.tex| & sample include file \\
    |cdocsch2.tex| & sample include file \\
    |cdocspt3.tex| & sample part file \\
    |cdocspt4.tex| & sample part file \\
    |cdocsdrf.tex| & sample redirection file \\
    |cdocsfn1.tex| & sample redirection file \\
    |cdocsfn2.tex| & sample redirection file \\
    |childdoc.pdf| & manual
\end{tabular}
\end{center}
%
The distribution consists of the files
|README.txt|, |childdoc.ins| and |childdoc.dtx|.
%
\begin{itemize}
\item
Run (pdf)\LaTeX{} on |childdoc.dtx|
to compile the manual |childdoc.pdf| (this file).
\item
Run \LaTeX{} on |childdoc.ins| to create the definitions file |childdoc.def|
and the sample |cdocsamp.tex| with include files
|cdocsch1.tex|, |cdocsch2.tex|, |cdocspt3.tex|, |cdocspt4.tex|,
|cdocsdrf.tex|, |cdocsfn1.tex|, |cdocsfn2.tex|.
Then copy the file |childdoc.def| to an appropriate directory of your \LaTeX{}
distribution, e.g.\ \textit{texmf-root}|/tex/latex/childdoc|.
\end{itemize}

%%%%%%%%%%%%%%%%%%%%%%%%%%%%%%%%%%%%%%%%%%%%%%%%%%%%%%%%%%%%%%%%%%%%%%%%%%%%%%%%
\subsection{Related CTAN Packages}

There are several other packages which offer a similar functionality:
%
\begin{itemize}
\item
The packages
\href{http://ctan.org/pkg/docmute}{\textsf{docmute}},
\href{http://ctan.org/pkg/includex}{\textsf{includex}} and
\href{http://ctan.org/pkg/standalone}{\textsf{standalone}}
provide commands to include only the document body of
a child file thus allowing both files to be compiled individually.
\item
The packages \href{http://ctan.org/pkg/subdocs}{\textsf{subdocs}}
and \href{http://ctan.org/pkg/subfiles}{\textsf{subfiles}}
provide structures in which the main and child documents can be
encapsulated and allowing them to be compiled individually.
The inclusion mechanism is different from the conventional |\include|.
\item
The package \href{http://ctan.org/pkg/combine}{\textsf{combine}}
is an elaborate solution to combine several documents into one.
\end{itemize}
%
See also the CTAN topic \href{http://ctan.org/topic/subdocs}{\textsf{subdocs}}
for further related packages.
The present package differs from the above solutions in that
a document structure constructed with the conventional |\include| mechanism
just needs two extra commands at the top of every file
such that all constituent files can be compiled individually.

%%%%%%%%%%%%%%%%%%%%%%%%%%%%%%%%%%%%%%%%%%%%%%%%%%%%%%%%%%%%%%%%%%%%%%%%%%%%%%%%
%\subsection{Feature Suggestions}
%
%The following is a list of features which may be useful for future
%versions of this package:
%%
%\begin{itemize}
%\item
%\ldots
%\end{itemize}

%%%%%%%%%%%%%%%%%%%%%%%%%%%%%%%%%%%%%%%%%%%%%%%%%%%%%%%%%%%%%%%%%%%%%%%%%%%%%%%%
\subsection{Revision History}

%%%%%%%%%%%%%%%%%%%%%%%%%%%%%%%%%%%%%%%%
\paragraph{v2.0:} 2018/12/30

\begin{itemize}
\item
immediate forward processing
\item
added |\childdocby| mechanism
\item
manual restructured
\end{itemize}

%%%%%%%%%%%%%%%%%%%%%%%%%%%%%%%%%%%%%%%%
\paragraph{v1.6:} 2018/01/17

\begin{itemize}
\item
application for development of include files
\item
corrections to manual
\end{itemize}

%%%%%%%%%%%%%%%%%%%%%%%%%%%%%%%%%%%%%%%%
\paragraph{v1.5:} 2017/05/21

\begin{itemize}
\item
more complete structuring introduced
\item
|\childdocof| introduced
\item
|\childdoc| renamed to |\childdocmain|
\item
|\childredirect| renamed to |\childdocforward| and |\childdocforwardprefix|
and functionality expanded
\end{itemize}

%%%%%%%%%%%%%%%%%%%%%%%%%%%%%%%%%%%%%%%%
\paragraph{v1.0:} 2017/04/27

\begin{itemize}
\item
manual and install package
\item
first version published on CTAN
\end{itemize}

%%%%%%%%%%%%%%%%%%%%%%%%%%%%%%%%%%%%%%%%
\paragraph{v0.6:} 2017/04/26

\begin{itemize}
\item
redirection mechanism added
\end{itemize}

%%%%%%%%%%%%%%%%%%%%%%%%%%%%%%%%%%%%%%%%
\paragraph{v0.5:} 2017/04/26

\begin{itemize}
\item
functionality in definition file
\end{itemize}


%%%%%%%%%%%%%%%%%%%%%%%%%%%%%%%%%%%%%%%%%%%%%%%%%%%%%%%%%%%%%%%%%%%%%%%%%%%%%%%%
%%%%%%%%%%%%%%%%%%%%%%%%%%%%%%%%%%%%%%%%%%%%%%%%%%%%%%%%%%%%%%%%%%%%%%%%%%%%%%%%
%%%%%%%%%%%%%%%%%%%%%%%%%%%%%%%%%%%%%%%%%%%%%%%%%%%%%%%%%%%%%%%%%%%%%%%%%%%%%%%%
\appendix

\settowidth\MacroIndent{\rmfamily\scriptsize 000\ }

 \DocInput{childdoc.dtx}

\end{document}
%</driver>
% \fi
%
% %%%%%%%%%%%%%%%%%%%%%%%%%%%%%%%%%%%%%%%%%%%%%%%%%%%%%%%%%%%%%%%%%%%%%%%%%%%%%%
% %%%%%%%%%%%%%%%%%%%%%%%%%%%%%%%%%%%%%%%%%%%%%%%%%%%%%%%%%%%%%%%%%%%%%%%%%%%%%%
% \section{Sample}
%\iffalse
%<*samplemain>
%\fi
%
% The following presents a sample document
% with two chapters, two parts, a title page,
% a compile flag as well as three forwarding files to set the flag.
% It consists of eight |.tex| files:
% \begin{center}
% \begin{tabular}{ll}
% |cdocsamp.tex|&main file\\
% |cdocsch1.tex|&include file for chapter 1\\
% |cdocsch2.tex|&include file for chapter 2\\
% |cdocspt3.tex|&include file for part 3\\
% |cdocspt4.tex|&include file for part 4\\
% |cdocsdrf.tex|&forwarding file for main file in draft mode\\
% |cdocsfi1.tex|&forwarding file for final version of chapter 1\\
% |cdocsfi2.tex|&forwarding file for final version of chapter 2\\
% \end{tabular}
% \end{center}
% Each of the eight files can be compiled directly by the \LaTeX{} compiler.
%
% %%%%%%%%%%%%%%%%%%%%%%%%%%%%%%%%%%%%%%
% \paragraph{Main File.}
%
% The main file is called |cdocsamp.tex|.
%
% Load the \textsf{childdoc} definitions and
% declare the filename for the main document:
%    \begin{macrocode}
\input{childdoc.def}
\childdocmain{}
%    \end{macrocode}

% Optional override for |\version| flag:
%    \begin{macrocode}
%%\ifchilddoc\else\providecommand{\version}{draft}\fi
%    \end{macrocode}

% Define the default values for the |\version| flag
% (|final| for the main file and |draft| for childs):
%    \begin{macrocode}
\ifchilddoc
\providecommand{\version}{draft}
\else
\providecommand{\version}{final}
\fi
%    \end{macrocode}

% Load the standard document class:
%    \begin{macrocode}
\documentclass[12pt]{article}
%    \end{macrocode}

% Start the document body:
%    \begin{macrocode}
\begin{document}
%    \end{macrocode}

% Declare a title page.
% Print title, part of document being processed and version flag:
%    \begin{macrocode}
\addtocounter{page}{-1}
\begin{center}
{\LARGE\bfseries{}childdoc example\par}
\vspace{1cm}
\ifchilddoc
\ifchilddocmanual part\else chapter\fi:
`\childdocname' of `\childdocjob'\par
\else
main document: `\childdocjob'\par
\fi
version: \version\par
\end{center}
\newpage
%    \end{macrocode}

% Manually include selected file,
% otherwise process as usual:
%    \begin{macrocode}
\ifchilddocmanual
\section*{part `\childdocname'}
\input{\childdocname}
\else
%    \end{macrocode}

% Include the two chapters:
%    \begin{macrocode}
\include{cdocsch1}
\include{cdocsch2}
%    \end{macrocode}

% Include the two parts unless only chapters should be displayed:
%    \begin{macrocode}
\ifchilddoc\else
\section{part three}
\input{cdocspt3}
\section{part four}
\input{cdocspt4}
\fi
%    \end{macrocode}

% Process as usual until here:
%    \begin{macrocode}
\fi
%    \end{macrocode}

% End of document body:
%    \begin{macrocode}
\end{document}
%    \end{macrocode}
%\iffalse
%</samplemain>
%\fi
%
% %%%%%%%%%%%%%%%%%%%%%%%%%%%%%%%%%%%%%%
% \paragraph{Chapter Include Files.}
%
% The include files are called |cdocsch1.tex| and |cdocsch2.tex|.
%
%\iffalse
%<*samplechap1|samplechap2>
%\fi

% Optional override for |\version| flag:
%    \begin{macrocode}
%%\providecommand{\version}{final}
%    \end{macrocode}

% Include the main document:
%    \begin{macrocode}
\input{childdoc.def}
\childdocof{cdocsamp}
%    \end{macrocode}

%\iffalse
%</samplechap1|samplechap2>
%\fi
%
%\iffalse
%<*samplechap1>
%\fi
% Some text for chapter 1:
%    \begin{macrocode}
\section{one}
some text in chapter one
%    \end{macrocode}

%\iffalse
%</samplechap1>
%\fi
% Some text for chapter 2:
%\iffalse
%<*samplechap2>
%\fi
%    \begin{macrocode}
\section{two}
more text in chapter two
%    \end{macrocode}

%\iffalse
%</samplechap2>
%\fi
%
% %%%%%%%%%%%%%%%%%%%%%%%%%%%%%%%%%%%%%%
% \paragraph{Part Include Files.}
%
% The include files are called |cdocspt3.tex| and |cdocspt4.tex|.
%
%\iffalse
%<*samplepart3|samplepart4>
%\fi

% Optional override for |\version| flag:
%    \begin{macrocode}
%%\providecommand{\version}{final}
%    \end{macrocode}

% Include the main document:
%    \begin{macrocode}
\input{childdoc.def}
\childdocby{cdocsamp}
%    \end{macrocode}

%\iffalse
%</samplepart3|samplepart4>
%\fi
%
%\iffalse
%<*samplepart3>
%\fi
% Some text for part 3:
%    \begin{macrocode}
some text in part three
%    \end{macrocode}

%\iffalse
%</samplepart3>
%\fi
% Some text for part 4:
%\iffalse
%<*samplepart4>
%\fi
%    \begin{macrocode}
more text in part four
%    \end{macrocode}

%\iffalse
%</samplepart4>
%\fi
%
% %%%%%%%%%%%%%%%%%%%%%%%%%%%%%%%%%%%%%%
% \paragraph{Forwarding for a Complete Draft.}
%
% The following forwarding file |cdocsdrf.tex|
% compiles the main document in draft mode:
%\iffalse
%<*sampledraft>
%\fi
%    \begin{macrocode}
\def\version{draft}
\input{childdoc.def}
\childdocforward{cdocsamp}
%    \end{macrocode}

%\iffalse
%</sampledraft>
%\fi
%
% %%%%%%%%%%%%%%%%%%%%%%%%%%%%%%%%%%%%%%
% \paragraph{Forwarding for Final Version of the Chapters.}
%
% The following forwarding files |cdocsfn1.tex| and |cdocsfn2.tex|
% (with identical content)
% compile the final versions of the child documents
% |cdocsch1.tex| and |cdocsch2.tex|, respectively:
%\iffalse
%<*samplefinal>
%\fi
%    \begin{macrocode}
\def\version{final}
\input{childdoc.def}
\childdocforwardprefix[cdocsamp]{cdocsfn}{cdocsch}
%    \end{macrocode}

%\iffalse
%</samplefinal>
%\fi
%
% %%%%%%%%%%%%%%%%%%%%%%%%%%%%%%%%%%%%%%
% \paragraph{Command Line Processing.}
%
% The following three command lines generate the output files
% |cdocscld|, |cdocscl1| and |cdocscl2|
% which should be identical to
% |cdocsdrf|, |cdocsch1| and |cdocsfn2|, respectively:
% \begin{center}
% \begin{tabular}{l}
% |latex -jobname cdocscld \|\\
% |  "\def\version{draft}\input{childdoc.def}\childdocforward{cdocsamp}"|\\
% |latex -jobname cdocscl1 \|\\
% |  "\input{childdoc.def}\childdocforward[cdocsamp]{cdocsch1}"|\\
% |latex -jobname cdocscl2 \|\\
% |  "\def\version{final}\input{childdoc.def}\childdocforward{cdocsch2}"|
% \end{tabular}
% \end{center}
% Note that the trailing backslash on each first line
% merely continues the input to the second line
% (for convenient cut ant paste).
% Furthermore, the command |latex| can be replaced by any
% of its alternative versions such as |pdflatex|.
%
% %%%%%%%%%%%%%%%%%%%%%%%%%%%%%%%%%%%%%%%%%%%%%%%%%%%%%%%%%%%%%%%%%%%%%%%%%%%%%%
% %%%%%%%%%%%%%%%%%%%%%%%%%%%%%%%%%%%%%%%%%%%%%%%%%%%%%%%%%%%%%%%%%%%%%%%%%%%%%%
% \section{Implementation}
%\iffalse
%<*package>
%\fi
%
% This section describes the definitions file |childdoc.def|.

% The definitions cannot be loaded using |\usepackage| or |\RequirePackage|
% which has a mechanism to prevent loading a style file more than once.
% When loading the definitions by means of |\input|
% multiple instances have to be prevented manually:
%\iffalse
%This code needs to be before the `\ProvidesFile' directive
%which is defined at the beginning of this file.
%Therefore it is also placed there and commented out here.
%</package>
%<*discard>
%\fi
%    \begin{macrocode}
\ifdefined\childdocmain\endinput\fi
%    \end{macrocode}
%\iffalse
%</discard>
%<*package>
%\fi
%
% \macro{\ifchilddoc}
% \macro{\ifchilddocmanual}
% The conditional |\ifchilddoc| tells whether a
% child (true) or main (false) document is being compiled.
% The conditional |\ifchilddocmanual| tells whether
% the |\includeonly| mechanism is used (false) or
% the selection of child files must be performed manually (true).
% The definitions initialise to false:
%    \begin{macrocode}
\newif\ifchilddoc
\newif\ifchilddocmanual
%    \end{macrocode}

% \macro{\childdocname}
% \macro{\childdocjob}
% The macro |\childdocname| stores the name of the main document
% to be compiled. The macro |\childdocjob| stores the name of
% the document on which the \LaTeX{} compiler was originally invoked.
% The content of |\jobname| cannot be compared
% to filenames specified in the source due to different catcodes.
% The following code rescans |\jobname|, stores the result
% in |\childdocname| and saves a copy in |\childdocjob|:
%    \begin{macrocode}
\edef\childdocname{\scantokens\expandafter{\jobname\noexpand}}
\let\childdocjob\childdocname
%    \end{macrocode}

% \macro{\childdocdisable}
% The macro |\childdocdisable| prevents the main file
% from being processed more than once.
% At this stage, the main document command |\childdocmain|
% is assumed to be called once again where it should do nothing.
% Any subsequent call to it should prevent
% a secondary processing of the main document
% It overwrites the forwarding commands
% |\childdocof| and |\childdocforward|
% with empty macros to prevent further inclusions of the main document:
%    \begin{macrocode}
\newcommand{\childdocdisable}
{
  \renewcommand{\childdocmain}[1]{\renewcommand{\childdocmain}[1]{\endinput}}
  \renewcommand{\childdocof}[1]{}
  \renewcommand{\childdocby}[2][]{}
  \renewcommand{\childdocforward}[2][]{}
  \renewcommand{\childdocdisable}{}
}
%    \end{macrocode}

% \macro{\childdocmain}
% The macro |\childdocmain| is to be called at the top of the main file
% with nothing or the main filename (without extension) as argument.
% First, it breaks loops.
% If the argument is not empty and does not match |\childdocname|
% (which is set by the first inclusion of |childdoc.def|),
% |\ifchilddoc| is set to true, |\includeonly| is applied to the child file
% and |\jobname| is set to the main file
% (for proper handling of |.aux| files):
%    \begin{macrocode}
\newcommand{\childdocmain}[1]
{
  \childdocdisable\childdocmain{}
  \if?#1?\else
    \begingroup
      \def\childdoctmp{#1}
      \ifx\childdoctmp\childdocname
        \def\childdoctmp{}
      \else
        \def\childdoctmp
        {
          \childdoctrue
          \includeonly{\childdocname}
          \def\childdocjob{#1}
          \def\jobname{#1}
        }
      \fi
      \expandafter
    \endgroup
    \childdoctmp
  \fi
}
%    \end{macrocode}

% \macro{\childdocof}
% The command |\childdocof| redirects
% compilation to the main file |#1|.
%    \begin{macrocode}
\newcommand{\childdocof}[1]
{
  \childdocdisable
  \childdoctrue
  \includeonly{\childdocname}
  \def\jobname{#1}
  \def\childdocjob{#1}
  \input{#1}
}
%    \end{macrocode}

% \macro{\childdocby}
% The command |\childdocby| ....
%    \begin{macrocode}
\newcommand{\childdocby}[2][]
{
  \childdocdisable
  \childdoctrue
  \childdocmanualtrue
  \if?#1?\else
    \def\jobname{#2}
  \fi
  \def\childdocjob{#2}
  \input{#2}
  \endinput
}
%    \end{macrocode}

% \macro{\childdocforward}
% The command |\childdocforward| redirects
% compilation to the main file or
% (if the optional argument is given) a child file.
% Parameters are set as if the main file
% or a child file starting with |\childdocof| was compiled.
% Then compilation is handed over to the main file:
%    \begin{macrocode}
\newcommand{\childdocforward}[2][]
{
  \begingroup
    \if?#1?
      \def\childdoctmp
      {
        \def\childdocname{#2}
        \def\childdocjob{#2}
        \def\jobname{#2}
        \input{#2}
        \endinput
      }
    \else
      \def\childdoctmp
      {
        \childdocdisable
        \def\childdocname{#2}
        \childdoctrue
        \includeonly{#2}
        \def\childdocjob{#1}
        \def\jobname{#1}
        \input{#1}
        \endinput
      }
    \fi
    \expandafter
  \endgroup
  \childdoctmp
}
%    \end{macrocode}

% \macro{\childdocforwardprefix}
% The command |\childdocforwardprefix| redirects
% compilation to the main or a child file by means of a pattern.
% The prefix |#1| in the current filename is replaced by |#2|
% and the suffix of the current filename is kept
% (it is assumed that the filename does not contain the substring `|~~~|'
% which is used as a delimiter).
% Compilation is handed over to the new file by |\childdocforward|:
%    \begin{macrocode}
\newcommand{\childdocforwardprefix}[3][]
{
  \begingroup
    \def\childdocextract #2##1~~~{\def\childdoctmp{\childdocforward[#1]{#3##1}}}
    \expandafter\childdocextract\childdocname~~~
    \expandafter
  \endgroup
  \childdoctmp
}
%    \end{macrocode}

% \macro{\childdoc}
% The deprecated macro |\childdoc| is a legacy version of |\childdocmain|:
%    \begin{macrocode}
\newcommand{\childdoc}{\childdocmain}
%    \end{macrocode}

% \macro{\childdocredirect}
% The deprecated macro |\childdocredirect| is a legacy version
% of |\childdocforward| and |\childdocforwardprefix|:
%    \begin{macrocode}
\newcommand{\childdocredirect}[2][]
{
  \begingroup
    \if?#1?
      \def\childdoctmp{\childdocforward{#2}}
    \else
      \def\childdoctmp{\childdocforwardprefix{#1}{#2}}
    \fi
    \expandafter
  \endgroup
  \childdoctmp
}
%    \end{macrocode}

%\iffalse
%</package>
%\fi
%
\endinput

\childdocforward{cdocsamp}
%    \end{macrocode}

%\iffalse
%</sampledraft>
%\fi
%
% %%%%%%%%%%%%%%%%%%%%%%%%%%%%%%%%%%%%%%
% \paragraph{Forwarding for Final Version of the Chapters.}
%
% The following forwarding files |cdocsfn1.tex| and |cdocsfn2.tex|
% (with identical content)
% compile the final versions of the child documents
% |cdocsch1.tex| and |cdocsch2.tex|, respectively:
%\iffalse
%<*samplefinal>
%\fi
%    \begin{macrocode}
\def\version{final}
% \iffalse
%
% childdoc.dtx Copyright (C) 2017-2018 Niklas Beisert
%
% This work may be distributed and/or modified under the
% conditions of the LaTeX Project Public License, either version 1.3
% of this license or (at your option) any later version.
% The latest version of this license is in
%   http://www.latex-project.org/lppl.txt
% and version 1.3 or later is part of all distributions of LaTeX
% version 2005/12/01 or later.
%
% This work has the LPPL maintenance status `maintained'.
%
% The Current Maintainer of this work is Niklas Beisert.
%
% This work consists of the files childdoc.dtx and childdoc.ins
% and the derived files childdoc.def and cdocsamp.tex with
% cdocsch1.tex, cdocsch2.tex, cdocsdrf.tex, cdocsfn1.tex, cdocsfn2.tex.
%
%<package>\ifdefined\childdocmain\endinput\fi
%<package>\ProvidesFile{childdoc.def}[2018/12/30 v2.0 child document driver]
%<samplemain>\ProvidesFile{cdocsamp.tex}[2018/12/30 v2.0 sample for childdoc]
%<*driver>
%\ProvidesFile{childdoc.drv}[2018/12/30 v2.0 childdoc reference manual file]
\PassOptionsToClass{10pt,a4paper}{article}
\documentclass{ltxdoc}

\usepackage[margin=35mm]{geometry}
\usepackage{hyperref}
\usepackage{hyperxmp}
\usepackage[usenames]{color}

\hypersetup{colorlinks=true}
\hypersetup{pdfstartview=FitH}
\hypersetup{pdfpagemode=UseNone}
\hypersetup{pdfsource={}}
\hypersetup{pdflang={en-UK}}
\hypersetup{pdfcopyright={Copyright 2017-2018 Niklas Beisert.
  This work may be distributed and/or modified under the
  conditions of the LaTeX Project Public License, either version 1.3
  of this license or (at your option) any later version.}}
\hypersetup{pdflicenseurl={http://www.latex-project.org/lppl.txt}}
\hypersetup{pdfcontactaddress={ETH Zurich, ITP, HIT K,
  Wolfgang-Pauli-Strasse 27}}
\hypersetup{pdfcontactpostcode={8093}}
\hypersetup{pdfcontactcity={Zurich}}
\hypersetup{pdfcontactcountry={Switzerland}}
\hypersetup{pdfcontactemail={nbeisert@itp.phys.ethz.ch}}
\hypersetup{pdfcontacturl={http://people.phys.ethz.ch/\xmptilde nbeisert/}}

\newcommand{\secref}[1]{\hyperref[#1]{section \ref*{#1}}}

\parskip1ex
\parindent0pt
\let\olditemize\itemize
\def\itemize{\olditemize\parskip0pt}

\begin{document}

\title{The \textsf{childdoc} Package}
\hypersetup{pdftitle={The childdoc Package}}
\author{Niklas Beisert\\[2ex]
  Institut f\"ur Theoretische Physik\\
  Eidgen\"ossische Technische Hochschule Z\"urich\\
  Wolfgang-Pauli-Strasse 27, 8093 Z\"urich, Switzerland\\[1ex]
  \href{mailto:nbeisert@itp.phys.ethz.ch}
  {\texttt{nbeisert@itp.phys.ethz.ch}}}
\hypersetup{pdfauthor={Niklas Beisert}}
\hypersetup{pdfsubject={Manual for the LaTeX2e Package childdoc}}
\date{30 December 2018, \textsf{v2.0}}
\maketitle

\begin{abstract}\noindent
\textsf{childdoc} is a \LaTeXe{} package
that enables the direct compilation
of document sections included by |\include|
to individual files.
\end{abstract}

\begingroup
\parskip0ex
\tableofcontents
\endgroup

%%%%%%%%%%%%%%%%%%%%%%%%%%%%%%%%%%%%%%%%%%%%%%%%%%%%%%%%%%%%%%%%%%%%%%%%%%%%%%%%
%%%%%%%%%%%%%%%%%%%%%%%%%%%%%%%%%%%%%%%%%%%%%%%%%%%%%%%%%%%%%%%%%%%%%%%%%%%%%%%%
\section{Introduction}

\LaTeX{} provides a mechanism to structure a large document (such as a book)
into a main file and several child files (containing the chapters)
using the |\include| command.
This mechanism is beneficial for documents
which span hundreds of pages in order to
make the source file(s) more manageable.
Moreover, compilation can be restricted to
selected child files by means of the |\includeonly| command.
The latter feature can be used to reduce the compilation time while editing
(this was significantly more useful in the earlier days of \LaTeX{})
or to generate a smaller document which is easier to navigate.
Another application of |\includeonly| is to generate
documents consisting of selected parts of the complete document.

However, there are a few drawbacks of the plain |\include| mechanism:
\begin{itemize}
\item
The child files cannot be compiled on their own,
they can only be compiled via the main file.
A naive editing environment
(such as a text editor with an option
to have the current file processed by \LaTeX)
may require one to switch to the main file before compiling;
attempting to compile the child file produces errors.
\item
The main file must be modified (each time)
to adjust the |\includeonly| command
to the present needs. This easily leaves the main file in a messy state.
\item
The generated document will always carry the filename
of the main document. This is inconvenient if
several child files are to be compiled and
to be kept for distribution.
\end{itemize}

The present package provides a simple interface
to make child files individually compilable by \LaTeX{}.
Compiling a child file then has the same effect as compiling
the main file with an |\includeonly| command
to select the appropriate child.
Moreover the generated document will carry the name of the child
rather than the main file.
This resolves all three above issues.

This feature is meant to make the editing of books,
thesis documents and lecture notes somewhat more convenient.
However, the package can also be used efficiently for
composing a series of documents (such as exercise sheets)
which are typically distributed individually.
It then assists the author in generating the individual documents
(potentially in different versions)
as well as a document containing the collected series.
Another application is in developing style files
or other kinds of included material
where compilation of the style file could redirect
to a sample or test file.

%%%%%%%%%%%%%%%%%%%%%%%%%%%%%%%%%%%%%%%%%%%%%%%%%%%%%%%%%%%%%%%%%%%%%%%%%%%%%%%%
%%%%%%%%%%%%%%%%%%%%%%%%%%%%%%%%%%%%%%%%%%%%%%%%%%%%%%%%%%%%%%%%%%%%%%%%%%%%%%%%
\section{Usage}

First of all, the package \textsf{childdoc} is \emph{not} a standard
\LaTeXe{} |.sty| style file! Therefore it needs to be invoked in
a non-standard way.

%%%%%%%%%%%%%%%%%%%%%%%%%%%%%%%%%%%%%%%%%%%%%%%%%%%%%%%%%%%%%%%%%%%%%%%%%%%%%%%%
\subsection{Included Files}
\label{sec:include}

%%%%%%%%%%%%%%%%%%%%%%%%%%%%%%%%%%%%%%%%
\DescribeMacro{\childdocmain}
To use the package, add the commands
\begin{center}
\begin{tabular}{l}
|\input{childdoc.def}|\\
|\childdocmain{}|\\
\end{tabular}
\end{center}
at the very top of the main \LaTeX{} file,
in particular \emph{before} the |\documentclass| statement!
The argument of |\childdocmain| should be left empty
(but it must be present).

%%%%%%%%%%%%%%%%%%%%%%%%%%%%%%%%%%%%%%%%
\DescribeMacro{\childdocof}
Furthermore, add the commands
\begin{center}
\begin{tabular}{l}
|\input{childdoc.def}|\\
|\childdocof{|\textit{main}|}|\\
\end{tabular}
\end{center}
at the top of every child file \textit{child}
which is included by |\include{|\textit{child}|}|
from within the main file
(or at least for those files to be compiled individually).
The argument \textit{main} must be the filename of the main file.

There are a couple of
considerations in setting up the main and child documents:

%%%%%%%%%%%%%%%%%%%%%%%%%%%%%%%%%%%%%%%%
\paragraph{Restrictions.}

Please note the following restrictions:
\begin{itemize}
\item
|\childdocmain| must be called with one argument \textit{main}
to ensure compatibility with earlier version of the package.
It must either be empty (|\childdocmain{}|)
or precisely match the filename of the main file in which it is specified.
See \secref{sec:detection} for further information.
\item
The filename \textit{main} must be specified without the |.tex| extension.
\item
The filename \textit{main} is case sensitive
(even in case-insensitive file systems)
due to internal string comparison.
\item
The argument \textit{main} should be fully expanded, it cannot be a macro.
\item
Subdirectories and special characters should be avoided in filenames.
\item
The command |\childdocmain{|\textit{main}|}| must be followed by a whitespace.
It should not be followed immediately by another command
or by a comment mark `|%|'.
This is because the \TeX{} parser reads the token immediately following
the argument of |\childdocmain| and puts it
at the beginning of every child section;
however, a white\-space is ignored.
\end{itemize}

%%%%%%%%%%%%%%%%%%%%%%%%%%%%%%%%%%%%%%%%
\paragraph{Content of Main File.}

It is advisable to place all content in the child files included by |\include|.
Any output contained in the main file will appear in all child documents
unless suppressed manually;
it cannot be suppressed automatically by the |\includeonly| directive
and thus should normally be avoided.
A method to include some content in the main file
by means of conditional processing is described in \secref{sec:conditional}.

%%%%%%%%%%%%%%%%%%%%%%%%%%%%%%%%%%%%%%%%
\paragraph{Page Numbering.}

When only a part of the document is compiled,
the appropriate numbering of pages
(as well as other status parameters)
is determined from the |.aux| files.
The latter contain information from previous passes.
However this information needs to propagate through
all intermediate child documents.
Therefore the page numbering in child documents may well
be inconsistent until the complete document is compiled at least once.

A useful (if unconventional) way to always ensure a consistent
page numbering is to restart the numbering in each child document
and denote the pages by `\textit{child}|.|\textit{page}'
where \textit{child} represents the chapter/section number of the child file.
This can be achieved by the command
|\numberwithin{page}{|\textit{child}|}|
of the \textsf{amsmath} package
where \textit{child} can be |chapter| or |section|
depending on the chosen structuring.
Alternatively, one can modify the macro |\thepage| appropriately
and reset the counter |page| at the start of each child file.

%%%%%%%%%%%%%%%%%%%%%%%%%%%%%%%%%%%%%%%%%%%%%%%%%%%%%%%%%%%%%%%%%%%%%%%%%%%%%%%%
\subsection{Conditional Processing}
\label{sec:conditional}

The package provides a mechanism to compile different versions
of a document. To customise the versions further some conditional processing
can come in handy to distinguish which version is being compiled.
The package provides two macros to describe the compilation context:

%%%%%%%%%%%%%%%%%%%%%%%%%%%%%%%%%%%%%%%%
\DescribeMacro{\ifchilddoc}
The conditional |\ifchilddoc| distinguishes between the compilation of
child documents and the main document:
%
\begin{center}
|\ifchilddoc |\textit{child-code}| |[|\||else |\textit{main-code}]| \||fi|
\end{center}

%%%%%%%%%%%%%%%%%%%%%%%%%%%%%%%%%%%%%%%%
\DescribeMacro{\childdocname}
\DescribeMacro{\childdocjob}
The macro |\childdocname| contains the filename (without extension)
of the main or child file being processed.
Note that |\childdocjob| will always contain the name of the main file.

%%%%%%%%%%%%%%%%%%%%%%%%%%%%%%%%%%%%%%%%
\paragraph{Title Page.}

Conditional processing can be used to include a title or banner page
in the main document when proper precautions are taken.
Importantly, the code in the main file should ensure that the page counter
(as well as other status parameters which are stored in the |.aux| files)
takes the same value after the conditional processing.
Otherwise the page numbers may take divergent values
depending on which part is compiled.

For example, a title page could be declared by:
%
\begin{center}
\begin{tabular}{l}
|\ifchilddoc\||else|\\
|\addtocounter{page}{-1}|\\
\textit{code for title page}\\
|\newpage|\\
|\||fi|
\end{tabular}
\end{center}
%
A banner page for the child documents can be generated by:
%
\begin{center}
\begin{tabular}{l}
|\ifchilddoc|\\
|\addtocounter{page}{-1}|\\
\textit{code for banner page}\\
|\newpage|\\
|\||fi|
\end{tabular}
\end{center}
%
Here one could write a message such as:
\begin{center}
|This is the part \childdocname{} of \childdocjob{}.|
\end{center}

%%%%%%%%%%%%%%%%%%%%%%%%%%%%%%%%%%%%%%%%%%%%%%%%%%%%%%%%%%%%%%%%%%%%%%%%%%%%%%%%
\subsection{Flags}
\label{sec:flags}

The package makes it easy to generate different versions
of the main or child documents.
To this end compilation flags can be defined
and assigned different default values.
They will be particularly useful in conjunction
with the forwarding mechanism described in \secref{sec:forward}.

For example, it may be useful to have a flag |\version|
which can be set to |draft| or |final|.
The document source will contain some conditional code
depending on the value of |\version|.
Suppose further, the flag should default to |final| for the main file
and to |draft| for child files
which is a natural assignment for editing the document.
This is achieved by placing the following code
in the preamble of the main document
(below the |\childdocmain| directive):
%
\begin{center}
\begin{tabular}{l}
|\ifchilddoc|\\
|\providecommand{\version}{draft}|\\
|\||else|\\
|\providecommand{\version}{final}|\\
|\||fi|
\end{tabular}
\end{center}
%
The definition by |\providecommand| makes sure
that previous definitions are not overwritten.
Further statements |\providecommand{\version}{...}|
can thus be added before the above code to override it.

For the main file, one might add a line
(between |\childdocmain| and the above block)
%
\begin{center}
|%\ifchilddoc\||else\providecommand{\version}{draft}\||fi|
\end{center}
%
which can be uncommented to produce a draft version.
Likewise one can add a line to the very top of a child file
(above the |\childdocof{|\textit{main}|}| directive)
%
\begin{center}
|%\providecommand{\version}{final}|
\end{center}
%
which can be uncommented to produce the final version of this child document.

%%%%%%%%%%%%%%%%%%%%%%%%%%%%%%%%%%%%%%%%%%%%%%%%%%%%%%%%%%%%%%%%%%%%%%%%%%%%%%%%
\subsection{Forwarding}
\label{sec:forward}

Different versions of the main or child documents
using compilation flags as described in \secref{sec:flags}
can be (permanently) stored in different files
for convenient compilation, viewing and distribution.
To this end, the package defines a command
to pass on compilation to a different file:

%%%%%%%%%%%%%%%%%%%%%%%%%%%%%%%%%%%%%%%%
\DescribeMacro{\childdocforward}
The command |\childdocforward| redirects processing to
another source file:
%
\begin{center}
\begin{tabular}{l}
|\input{childdoc.def}|\\
|\childdocforward[|\textit{main}|]{|\textit{dest}|}|\\
\end{tabular}
\end{center}
%
The argument \textit{dest} is the destination file
(without extension).
It should be the main file or one of the child files.
Note that further \textsf{childdoc} directives
such as |\childdocof| and |\childdocforward|
in the indicated file will be processed in this form.
The optional argument \textit{main}
passes on directly to the main file \textit{main}
while pretending to compile the child \textit{dest}.
This form behaves as if \textit{dest}
issues |\childdocof{|\textit{main}|}| right away,
and no further \textsf{childdoc} directives will be processed.

%%%%%%%%%%%%%%%%%%%%%%%%%%%%%%%%%%%%%%%%
\DescribeMacro{\...prefix}
In the alternative form |\childdocforwardprefix|,
%
\begin{center}
\begin{tabular}{l}
|\input{childdoc.def}|\\
|\childdocforwardprefix[|\textit{main}|]{|\textit{prefix}|}{|\textit{dest}|}|
\end{tabular}
\end{center}
%
the destination file is determined by a pattern
depending on the current file:
To make this work, the current file must be called
`{\textit{prefix}\hspace{0.2em}\textit{suffix}}'
with \textit{prefix} matching precisely the argument.
Processing is then passed on to the file
`{\textit{dest}\hspace{0.2em}\textit{suffix}}'.
Surely, the same effect is achieved by
directly specifying the
argument `{\textit{dest}\hspace{0.2em}\textit{suffix}}'
in the first form.
However, that requires to set up a different file
for each child. With the alternative form of the command
all these files can have exactly the same content
which simplifies setting them up and maintaining them.

For example, the following file |draft.tex|
with a compilation flag |\version| as described in \secref{sec:flags}
compiles the main document as a draft:
%
\begin{center}
\begin{tabular}{l}
|\def\version{draft}|\\
|\input{childdoc.def}|\\
|\childdocforward{|\textit{main}|}|
\end{tabular}
\end{center}
%
Likewise, the following files |final|\textit{nn}|.tex|
compile the final version of the child document
|child|\textit{nn}|.tex|:
%
\begin{center}
\begin{tabular}{l}
|\def\version{final}|\\
|\input{childdoc.def}|\\
|\childdocforwardprefix{final}{child}|
\end{tabular}
\end{center}
%

Note that when several versions of a main file and/or of each child file
are to be generated, it may be convenient to set up a |Makefile| or
shell script to automatise the process.

%%%%%%%%%%%%%%%%%%%%%%%%%%%%%%%%%%%%%%%%%%%%%%%%%%%%%%%%%%%%%%%%%%%%%%%%%%%%%%%%
\subsection{Command Line Processing}
\label{sec:commandline}

The effect of redirection files can also be achieved by invoking
the \LaTeX{} compiler with a more elaborate command line.
Most conveniently this should be done as part
of a shell script or a |Makefile|.

When using \textsf{childdoc} in the main file, the following
command lines effectively perform a redirection
(note that depending on the shell being used,
backslashes may have to be doubled: `|\|' $\to$ `|\\|'):
%
\begin{center}
|... -jobname "|\textit{target}|" |\\|"|[\textit{flags}]%
|\input{childdoc.def}\childdocforward[|\textit{main}|]{|\textit{dest}|}"|
\end{center}
%
Here \textit{target} is the name of the output file,
\textit{main} is the name of the main file
and \textit{dest} is the name of the main or child file to be processed
(all filenames without extensions).
The optional argument \textit{main} can be omitted
if \textit{main} matches \textit{dest}.
Optionally, compilation \textit{flags} can be defined via |\def| commands.
This command line makes the \TeX{} engine believe
it is compiling the file \textit{target}
whose content is specified as the latter parameter.
The provided code then forwards the processing to
\textit{main} or \textit{dest} as described in \secref{sec:forward}.

%%%%%%%%%%%%%%%%%%%%%%%%%%%%%%%%%%%%%%%%%%%%%%%%%%%%%%%%%%%%%%%%%%%%%%%%%%%%%%%%
\subsection{Include by Input}
\label{sec:input}

Including child documents by |\include| has some restrictions by design.
Most notably, the content of a child document always occupies
its own set of pages; pages cannot be shared between child documents.
Usually, this behaviour makes perfect sense
because each child document contain an essential part of the document.
However, in some situations it may be desirable to compose
a document from a collection of parts
without having mandatory page breaks between then.
For this case, the package
provides a mechanism to include parts
by |\input| which can also be processed individually.
However, by construction this mechanism
requires manual handling of the content to be output.

%%%%%%%%%%%%%%%%%%%%%%%%%%%%%%%%%%%%%%%%
\DescribeMacro{\ifchilddocmanual}
The main file should be prepared as usual, see \secref{sec:include}.
However, the document body must make a distinction
between processing of an individual part and of the main document, e.g.:
%
\begin{center}
\begin{tabular}{l}
|\ifchilddocmanual|\\
|\input{\childdocname}|\\
|\||else|\\
\textit{document body with }|\input{|\textit{part}|}|\\
|\||fi|
\end{tabular}
\end{center}
%
The conditional |\ifchilddocmanual| is true whenever
a part to be included by |\input| is being compiled,
and the name of the part is stored in |\childdocname|.

%%%%%%%%%%%%%%%%%%%%%%%%%%%%%%%%%%%%%%%%
\DescribeMacro{\childdocby}
Each part to be included by |\input| should start with:
%
\begin{center}
\begin{tabular}{l}
|\input{childdoc.def}|\\
|\childdocby{|\textit{main}|}|\\
\end{tabular}
\end{center}
%
The directive |\childdocby| is similar to |\childdocof|
described in \secref{sec:include},
but the subsequent selection of content must be done manually.
To that end, both |\ifchilddoc| and |\ifchilddocmanual|
will be true upon processing of a part,
and the name of the part is stored in |\childdocname|.
Note that |\jobname| will be set to the filename of the current part
so that each part receives an individual |.aux| file
that does not interfere with the |.aux| file(s) of the main document.
This behaviour can be altered by the alternative form
|\childdocby[*]{|\textit{main}|}| (with a non-empty optional argument)
which uses the |.aux| file of the main document
by setting |\jobname| to \textit{main}.

%%%%%%%%%%%%%%%%%%%%%%%%%%%%%%%%%%%%%%%%%%%%%%%%%%%%%%%%%%%%%%%%%%%%%%%%%%%%%%%%
\subsection{Driver Development}
\label{sec:driver}

The \textsf{childdoc} mechanism can also be use for the development
of definition files such as \LaTeX{} styles or classes.
This case differs from the above setup with multiple parts
included by |\include| in that no |\includeonly| should be invoked.
This can be achieved by starting the include file
(before |\ProvidesPackage|) with:
%
\begin{center}
\begin{tabular}{l}
|\input{childdoc.def}|\\
|\childdocforward{|\textit{main}|}|\\
\end{tabular}
\end{center}
%
or alternatively with:
%
\begin{center}
\begin{tabular}{l}
|\input{childdoc.def}|\\
|\childdocby{|\textit{main}|}|\\
\end{tabular}
\end{center}
%
Both forms have slightly different effects as described above.
The main file is prepared as usual, see \secref{sec:include}.

%%%%%%%%%%%%%%%%%%%%%%%%%%%%%%%%%%%%%%%%%%%%%%%%%%%%%%%%%%%%%%%%%%%%%%%%%%%%%%%%
\subsection{Legacy Detection}
\label{sec:detection}

The directive |\childdocmain| in the main file can detect
whether the complete document or merely a child is to be compiled
even without using the directive |\childdocof|.
This method is deprecated because it is less robust
and there is no compelling reason to use it;
it is merely provided for backward compatibility
and it may be removed in future versions.

If the detection mechanism is to be used,
it is mandatory to correctly specify
the filename of the main file as the argument of |\childdocmain|:
%
\begin{center}
\begin{tabular}{l}
|\input{childdoc.def}|\\
|\childdocmain{|\textit{main}|}|\\
\end{tabular}
\end{center}
%
If |\jobname| does not match the argument \textit{main} of |\childdocmain|,
it is assumed that |\jobname| points to the child file to be compiled.
When using |\childdocmain| with the main file specified as argument,
it suffices to start a child file
with just |\input{|\textit{main}|}|
without loading of the package and using |\childdocof|.
If instead all processing is done
with the appropriate \textsf{childdoc} directives,
the argument of \textit{main} of |\childdocmain| can be empty.

An alternative version of the command line processing described
in \secref{sec:commandline} using the detection mechanism reads:
%
\begin{center}
|... -jobname "|\textit{target}|" "|[\textit{flags}]%
[|\def\jobname{|\textit{dest}|}|]|\input{|\textit{main}|}"|
\end{center}

%%%%%%%%%%%%%%%%%%%%%%%%%%%%%%%%%%%%%%%%%%%%%%%%%%%%%%%%%%%%%%%%%%%%%%%%%%%%%%%%
\subsection{Manual Code}
\label{sec:manual}

In case one cannot be certain whether the definitions file |childdoc.def|
is installed on the target \TeX{} distribution
and one prefers not to ship it,
it is conceivable to paste a few relevant commands into the sources.

To that end, drop all statements |\input{childdoc.def}|
and perform the replacements as outlined below.
Instead of |\childdocmain{|\textit{main}|}| add the following code
to the top of the main file:
%
\begin{center}
\begin{tabular}{l}
|\||ifdefined\childdocname\endinput\||fi\newif\ifchilddoc|\\
|\edef\childdocname{\scantokens\expandafter{\jobname\noexpand}}|\\
|\def\childdocmain{|\textit{main}|}\||ifx\childdocmain\childdocname\||else|\\
|\childdoctrue\includeonly{\childdocname}\let\jobname\childdocmain\||fi|\\
\end{tabular}
\end{center}
%
Instead of |\childdocof{|\textit{main}|}| just include the main file
at the top of each child file:
%
\begin{center}
|\input{|\textit{main}|}|
\end{center}
%
A simple redirection |\childdocforward{|\textit{dest}|}| is achieved by:
%
\begin{center}
|\def\jobname{|\textit{dest}|}\input{\jobname}|
\end{center}
%
The redirection with prefix
|\childdocforwardprefix[|\textit{prefix}|]{|\textit{dest}|}|
is accomplished by:
%
\begin{center}
\begin{tabular}{l}
|{\edef\jobname{\scantokens\expandafter{\jobname\noexpand}}|\\
|\def\redirectjob |\textit{prefix}|#1~~~{\gdef\jobname{|\textit{dest}|#1}}|\\
|\expandafter\redirectjob\jobname~~~}\input{\jobname}|
\end{tabular}
\end{center}

In an alternative approach,
child documents can be compiled by a specific command line
without additional code or specific definitions:
%
\begin{center}
|... -jobname "|\textit{target}|" "|[\textit{flags}]%
|\includeonly{|\textit{dest}|}\input{|\textit{main}|}"|
\end{center}
%

%%%%%%%%%%%%%%%%%%%%%%%%%%%%%%%%%%%%%%%%%%%%%%%%%%%%%%%%%%%%%%%%%%%%%%%%%%%%%%%%
%%%%%%%%%%%%%%%%%%%%%%%%%%%%%%%%%%%%%%%%%%%%%%%%%%%%%%%%%%%%%%%%%%%%%%%%%%%%%%%%
\section{Information}

%%%%%%%%%%%%%%%%%%%%%%%%%%%%%%%%%%%%%%%%%%%%%%%%%%%%%%%%%%%%%%%%%%%%%%%%%%%%%%%%
\subsection{Copyright}

Copyright \copyright{} 2017--2018 Niklas Beisert

This work may be distributed and/or modified under the
conditions of the \LaTeX{} Project Public License, either version 1.3
of this license or (at your option) any later version.
The latest version of this license is in
  \url{http://www.latex-project.org/lppl.txt}
and version 1.3 or later is part of all distributions of \LaTeX{}
version 2005/12/01 or later.

This work has the LPPL maintenance status `maintained'.

The Current Maintainer of this work is Niklas Beisert.

This work consists of the files |README.txt|, |childdoc.ins| and |childdoc.dtx|
as well as the derived files |childdoc.def|, |cdocsamp.tex|
with |cdocsch1.tex|, |cdocsch2.tex|, |cdocspt3.tex|, |cdocspt4.tex|,
|cdocsdrf.tex|, |cdocsfn1.tex|, |cdocsfn2.tex|
as well as |childdoc.pdf|.

%%%%%%%%%%%%%%%%%%%%%%%%%%%%%%%%%%%%%%%%%%%%%%%%%%%%%%%%%%%%%%%%%%%%%%%%%%%%%%%%
\subsection{Files and Installation}

The package consists of the files:
%
\begin{center}
\begin{tabular}{ll}
    |README.txt|   & readme file \\
    |childdoc.ins| & installation file \\
    |childdoc.dtx| & source file \\
    |childdoc.def| & definition file \\
    |cdocsamp.tex| & sample main file \\
    |cdocsch1.tex| & sample include file \\
    |cdocsch2.tex| & sample include file \\
    |cdocspt3.tex| & sample part file \\
    |cdocspt4.tex| & sample part file \\
    |cdocsdrf.tex| & sample redirection file \\
    |cdocsfn1.tex| & sample redirection file \\
    |cdocsfn2.tex| & sample redirection file \\
    |childdoc.pdf| & manual
\end{tabular}
\end{center}
%
The distribution consists of the files
|README.txt|, |childdoc.ins| and |childdoc.dtx|.
%
\begin{itemize}
\item
Run (pdf)\LaTeX{} on |childdoc.dtx|
to compile the manual |childdoc.pdf| (this file).
\item
Run \LaTeX{} on |childdoc.ins| to create the definitions file |childdoc.def|
and the sample |cdocsamp.tex| with include files
|cdocsch1.tex|, |cdocsch2.tex|, |cdocspt3.tex|, |cdocspt4.tex|,
|cdocsdrf.tex|, |cdocsfn1.tex|, |cdocsfn2.tex|.
Then copy the file |childdoc.def| to an appropriate directory of your \LaTeX{}
distribution, e.g.\ \textit{texmf-root}|/tex/latex/childdoc|.
\end{itemize}

%%%%%%%%%%%%%%%%%%%%%%%%%%%%%%%%%%%%%%%%%%%%%%%%%%%%%%%%%%%%%%%%%%%%%%%%%%%%%%%%
\subsection{Related CTAN Packages}

There are several other packages which offer a similar functionality:
%
\begin{itemize}
\item
The packages
\href{http://ctan.org/pkg/docmute}{\textsf{docmute}},
\href{http://ctan.org/pkg/includex}{\textsf{includex}} and
\href{http://ctan.org/pkg/standalone}{\textsf{standalone}}
provide commands to include only the document body of
a child file thus allowing both files to be compiled individually.
\item
The packages \href{http://ctan.org/pkg/subdocs}{\textsf{subdocs}}
and \href{http://ctan.org/pkg/subfiles}{\textsf{subfiles}}
provide structures in which the main and child documents can be
encapsulated and allowing them to be compiled individually.
The inclusion mechanism is different from the conventional |\include|.
\item
The package \href{http://ctan.org/pkg/combine}{\textsf{combine}}
is an elaborate solution to combine several documents into one.
\end{itemize}
%
See also the CTAN topic \href{http://ctan.org/topic/subdocs}{\textsf{subdocs}}
for further related packages.
The present package differs from the above solutions in that
a document structure constructed with the conventional |\include| mechanism
just needs two extra commands at the top of every file
such that all constituent files can be compiled individually.

%%%%%%%%%%%%%%%%%%%%%%%%%%%%%%%%%%%%%%%%%%%%%%%%%%%%%%%%%%%%%%%%%%%%%%%%%%%%%%%%
%\subsection{Feature Suggestions}
%
%The following is a list of features which may be useful for future
%versions of this package:
%%
%\begin{itemize}
%\item
%\ldots
%\end{itemize}

%%%%%%%%%%%%%%%%%%%%%%%%%%%%%%%%%%%%%%%%%%%%%%%%%%%%%%%%%%%%%%%%%%%%%%%%%%%%%%%%
\subsection{Revision History}

%%%%%%%%%%%%%%%%%%%%%%%%%%%%%%%%%%%%%%%%
\paragraph{v2.0:} 2018/12/30

\begin{itemize}
\item
immediate forward processing
\item
added |\childdocby| mechanism
\item
manual restructured
\end{itemize}

%%%%%%%%%%%%%%%%%%%%%%%%%%%%%%%%%%%%%%%%
\paragraph{v1.6:} 2018/01/17

\begin{itemize}
\item
application for development of include files
\item
corrections to manual
\end{itemize}

%%%%%%%%%%%%%%%%%%%%%%%%%%%%%%%%%%%%%%%%
\paragraph{v1.5:} 2017/05/21

\begin{itemize}
\item
more complete structuring introduced
\item
|\childdocof| introduced
\item
|\childdoc| renamed to |\childdocmain|
\item
|\childredirect| renamed to |\childdocforward| and |\childdocforwardprefix|
and functionality expanded
\end{itemize}

%%%%%%%%%%%%%%%%%%%%%%%%%%%%%%%%%%%%%%%%
\paragraph{v1.0:} 2017/04/27

\begin{itemize}
\item
manual and install package
\item
first version published on CTAN
\end{itemize}

%%%%%%%%%%%%%%%%%%%%%%%%%%%%%%%%%%%%%%%%
\paragraph{v0.6:} 2017/04/26

\begin{itemize}
\item
redirection mechanism added
\end{itemize}

%%%%%%%%%%%%%%%%%%%%%%%%%%%%%%%%%%%%%%%%
\paragraph{v0.5:} 2017/04/26

\begin{itemize}
\item
functionality in definition file
\end{itemize}


%%%%%%%%%%%%%%%%%%%%%%%%%%%%%%%%%%%%%%%%%%%%%%%%%%%%%%%%%%%%%%%%%%%%%%%%%%%%%%%%
%%%%%%%%%%%%%%%%%%%%%%%%%%%%%%%%%%%%%%%%%%%%%%%%%%%%%%%%%%%%%%%%%%%%%%%%%%%%%%%%
%%%%%%%%%%%%%%%%%%%%%%%%%%%%%%%%%%%%%%%%%%%%%%%%%%%%%%%%%%%%%%%%%%%%%%%%%%%%%%%%
\appendix

\settowidth\MacroIndent{\rmfamily\scriptsize 000\ }

 \DocInput{childdoc.dtx}

\end{document}
%</driver>
% \fi
%
% %%%%%%%%%%%%%%%%%%%%%%%%%%%%%%%%%%%%%%%%%%%%%%%%%%%%%%%%%%%%%%%%%%%%%%%%%%%%%%
% %%%%%%%%%%%%%%%%%%%%%%%%%%%%%%%%%%%%%%%%%%%%%%%%%%%%%%%%%%%%%%%%%%%%%%%%%%%%%%
% \section{Sample}
%\iffalse
%<*samplemain>
%\fi
%
% The following presents a sample document
% with two chapters, two parts, a title page,
% a compile flag as well as three forwarding files to set the flag.
% It consists of eight |.tex| files:
% \begin{center}
% \begin{tabular}{ll}
% |cdocsamp.tex|&main file\\
% |cdocsch1.tex|&include file for chapter 1\\
% |cdocsch2.tex|&include file for chapter 2\\
% |cdocspt3.tex|&include file for part 3\\
% |cdocspt4.tex|&include file for part 4\\
% |cdocsdrf.tex|&forwarding file for main file in draft mode\\
% |cdocsfi1.tex|&forwarding file for final version of chapter 1\\
% |cdocsfi2.tex|&forwarding file for final version of chapter 2\\
% \end{tabular}
% \end{center}
% Each of the eight files can be compiled directly by the \LaTeX{} compiler.
%
% %%%%%%%%%%%%%%%%%%%%%%%%%%%%%%%%%%%%%%
% \paragraph{Main File.}
%
% The main file is called |cdocsamp.tex|.
%
% Load the \textsf{childdoc} definitions and
% declare the filename for the main document:
%    \begin{macrocode}
\input{childdoc.def}
\childdocmain{}
%    \end{macrocode}

% Optional override for |\version| flag:
%    \begin{macrocode}
%%\ifchilddoc\else\providecommand{\version}{draft}\fi
%    \end{macrocode}

% Define the default values for the |\version| flag
% (|final| for the main file and |draft| for childs):
%    \begin{macrocode}
\ifchilddoc
\providecommand{\version}{draft}
\else
\providecommand{\version}{final}
\fi
%    \end{macrocode}

% Load the standard document class:
%    \begin{macrocode}
\documentclass[12pt]{article}
%    \end{macrocode}

% Start the document body:
%    \begin{macrocode}
\begin{document}
%    \end{macrocode}

% Declare a title page.
% Print title, part of document being processed and version flag:
%    \begin{macrocode}
\addtocounter{page}{-1}
\begin{center}
{\LARGE\bfseries{}childdoc example\par}
\vspace{1cm}
\ifchilddoc
\ifchilddocmanual part\else chapter\fi:
`\childdocname' of `\childdocjob'\par
\else
main document: `\childdocjob'\par
\fi
version: \version\par
\end{center}
\newpage
%    \end{macrocode}

% Manually include selected file,
% otherwise process as usual:
%    \begin{macrocode}
\ifchilddocmanual
\section*{part `\childdocname'}
\input{\childdocname}
\else
%    \end{macrocode}

% Include the two chapters:
%    \begin{macrocode}
\include{cdocsch1}
\include{cdocsch2}
%    \end{macrocode}

% Include the two parts unless only chapters should be displayed:
%    \begin{macrocode}
\ifchilddoc\else
\section{part three}
\input{cdocspt3}
\section{part four}
\input{cdocspt4}
\fi
%    \end{macrocode}

% Process as usual until here:
%    \begin{macrocode}
\fi
%    \end{macrocode}

% End of document body:
%    \begin{macrocode}
\end{document}
%    \end{macrocode}
%\iffalse
%</samplemain>
%\fi
%
% %%%%%%%%%%%%%%%%%%%%%%%%%%%%%%%%%%%%%%
% \paragraph{Chapter Include Files.}
%
% The include files are called |cdocsch1.tex| and |cdocsch2.tex|.
%
%\iffalse
%<*samplechap1|samplechap2>
%\fi

% Optional override for |\version| flag:
%    \begin{macrocode}
%%\providecommand{\version}{final}
%    \end{macrocode}

% Include the main document:
%    \begin{macrocode}
\input{childdoc.def}
\childdocof{cdocsamp}
%    \end{macrocode}

%\iffalse
%</samplechap1|samplechap2>
%\fi
%
%\iffalse
%<*samplechap1>
%\fi
% Some text for chapter 1:
%    \begin{macrocode}
\section{one}
some text in chapter one
%    \end{macrocode}

%\iffalse
%</samplechap1>
%\fi
% Some text for chapter 2:
%\iffalse
%<*samplechap2>
%\fi
%    \begin{macrocode}
\section{two}
more text in chapter two
%    \end{macrocode}

%\iffalse
%</samplechap2>
%\fi
%
% %%%%%%%%%%%%%%%%%%%%%%%%%%%%%%%%%%%%%%
% \paragraph{Part Include Files.}
%
% The include files are called |cdocspt3.tex| and |cdocspt4.tex|.
%
%\iffalse
%<*samplepart3|samplepart4>
%\fi

% Optional override for |\version| flag:
%    \begin{macrocode}
%%\providecommand{\version}{final}
%    \end{macrocode}

% Include the main document:
%    \begin{macrocode}
\input{childdoc.def}
\childdocby{cdocsamp}
%    \end{macrocode}

%\iffalse
%</samplepart3|samplepart4>
%\fi
%
%\iffalse
%<*samplepart3>
%\fi
% Some text for part 3:
%    \begin{macrocode}
some text in part three
%    \end{macrocode}

%\iffalse
%</samplepart3>
%\fi
% Some text for part 4:
%\iffalse
%<*samplepart4>
%\fi
%    \begin{macrocode}
more text in part four
%    \end{macrocode}

%\iffalse
%</samplepart4>
%\fi
%
% %%%%%%%%%%%%%%%%%%%%%%%%%%%%%%%%%%%%%%
% \paragraph{Forwarding for a Complete Draft.}
%
% The following forwarding file |cdocsdrf.tex|
% compiles the main document in draft mode:
%\iffalse
%<*sampledraft>
%\fi
%    \begin{macrocode}
\def\version{draft}
\input{childdoc.def}
\childdocforward{cdocsamp}
%    \end{macrocode}

%\iffalse
%</sampledraft>
%\fi
%
% %%%%%%%%%%%%%%%%%%%%%%%%%%%%%%%%%%%%%%
% \paragraph{Forwarding for Final Version of the Chapters.}
%
% The following forwarding files |cdocsfn1.tex| and |cdocsfn2.tex|
% (with identical content)
% compile the final versions of the child documents
% |cdocsch1.tex| and |cdocsch2.tex|, respectively:
%\iffalse
%<*samplefinal>
%\fi
%    \begin{macrocode}
\def\version{final}
\input{childdoc.def}
\childdocforwardprefix[cdocsamp]{cdocsfn}{cdocsch}
%    \end{macrocode}

%\iffalse
%</samplefinal>
%\fi
%
% %%%%%%%%%%%%%%%%%%%%%%%%%%%%%%%%%%%%%%
% \paragraph{Command Line Processing.}
%
% The following three command lines generate the output files
% |cdocscld|, |cdocscl1| and |cdocscl2|
% which should be identical to
% |cdocsdrf|, |cdocsch1| and |cdocsfn2|, respectively:
% \begin{center}
% \begin{tabular}{l}
% |latex -jobname cdocscld \|\\
% |  "\def\version{draft}\input{childdoc.def}\childdocforward{cdocsamp}"|\\
% |latex -jobname cdocscl1 \|\\
% |  "\input{childdoc.def}\childdocforward[cdocsamp]{cdocsch1}"|\\
% |latex -jobname cdocscl2 \|\\
% |  "\def\version{final}\input{childdoc.def}\childdocforward{cdocsch2}"|
% \end{tabular}
% \end{center}
% Note that the trailing backslash on each first line
% merely continues the input to the second line
% (for convenient cut ant paste).
% Furthermore, the command |latex| can be replaced by any
% of its alternative versions such as |pdflatex|.
%
% %%%%%%%%%%%%%%%%%%%%%%%%%%%%%%%%%%%%%%%%%%%%%%%%%%%%%%%%%%%%%%%%%%%%%%%%%%%%%%
% %%%%%%%%%%%%%%%%%%%%%%%%%%%%%%%%%%%%%%%%%%%%%%%%%%%%%%%%%%%%%%%%%%%%%%%%%%%%%%
% \section{Implementation}
%\iffalse
%<*package>
%\fi
%
% This section describes the definitions file |childdoc.def|.

% The definitions cannot be loaded using |\usepackage| or |\RequirePackage|
% which has a mechanism to prevent loading a style file more than once.
% When loading the definitions by means of |\input|
% multiple instances have to be prevented manually:
%\iffalse
%This code needs to be before the `\ProvidesFile' directive
%which is defined at the beginning of this file.
%Therefore it is also placed there and commented out here.
%</package>
%<*discard>
%\fi
%    \begin{macrocode}
\ifdefined\childdocmain\endinput\fi
%    \end{macrocode}
%\iffalse
%</discard>
%<*package>
%\fi
%
% \macro{\ifchilddoc}
% \macro{\ifchilddocmanual}
% The conditional |\ifchilddoc| tells whether a
% child (true) or main (false) document is being compiled.
% The conditional |\ifchilddocmanual| tells whether
% the |\includeonly| mechanism is used (false) or
% the selection of child files must be performed manually (true).
% The definitions initialise to false:
%    \begin{macrocode}
\newif\ifchilddoc
\newif\ifchilddocmanual
%    \end{macrocode}

% \macro{\childdocname}
% \macro{\childdocjob}
% The macro |\childdocname| stores the name of the main document
% to be compiled. The macro |\childdocjob| stores the name of
% the document on which the \LaTeX{} compiler was originally invoked.
% The content of |\jobname| cannot be compared
% to filenames specified in the source due to different catcodes.
% The following code rescans |\jobname|, stores the result
% in |\childdocname| and saves a copy in |\childdocjob|:
%    \begin{macrocode}
\edef\childdocname{\scantokens\expandafter{\jobname\noexpand}}
\let\childdocjob\childdocname
%    \end{macrocode}

% \macro{\childdocdisable}
% The macro |\childdocdisable| prevents the main file
% from being processed more than once.
% At this stage, the main document command |\childdocmain|
% is assumed to be called once again where it should do nothing.
% Any subsequent call to it should prevent
% a secondary processing of the main document
% It overwrites the forwarding commands
% |\childdocof| and |\childdocforward|
% with empty macros to prevent further inclusions of the main document:
%    \begin{macrocode}
\newcommand{\childdocdisable}
{
  \renewcommand{\childdocmain}[1]{\renewcommand{\childdocmain}[1]{\endinput}}
  \renewcommand{\childdocof}[1]{}
  \renewcommand{\childdocby}[2][]{}
  \renewcommand{\childdocforward}[2][]{}
  \renewcommand{\childdocdisable}{}
}
%    \end{macrocode}

% \macro{\childdocmain}
% The macro |\childdocmain| is to be called at the top of the main file
% with nothing or the main filename (without extension) as argument.
% First, it breaks loops.
% If the argument is not empty and does not match |\childdocname|
% (which is set by the first inclusion of |childdoc.def|),
% |\ifchilddoc| is set to true, |\includeonly| is applied to the child file
% and |\jobname| is set to the main file
% (for proper handling of |.aux| files):
%    \begin{macrocode}
\newcommand{\childdocmain}[1]
{
  \childdocdisable\childdocmain{}
  \if?#1?\else
    \begingroup
      \def\childdoctmp{#1}
      \ifx\childdoctmp\childdocname
        \def\childdoctmp{}
      \else
        \def\childdoctmp
        {
          \childdoctrue
          \includeonly{\childdocname}
          \def\childdocjob{#1}
          \def\jobname{#1}
        }
      \fi
      \expandafter
    \endgroup
    \childdoctmp
  \fi
}
%    \end{macrocode}

% \macro{\childdocof}
% The command |\childdocof| redirects
% compilation to the main file |#1|.
%    \begin{macrocode}
\newcommand{\childdocof}[1]
{
  \childdocdisable
  \childdoctrue
  \includeonly{\childdocname}
  \def\jobname{#1}
  \def\childdocjob{#1}
  \input{#1}
}
%    \end{macrocode}

% \macro{\childdocby}
% The command |\childdocby| ....
%    \begin{macrocode}
\newcommand{\childdocby}[2][]
{
  \childdocdisable
  \childdoctrue
  \childdocmanualtrue
  \if?#1?\else
    \def\jobname{#2}
  \fi
  \def\childdocjob{#2}
  \input{#2}
  \endinput
}
%    \end{macrocode}

% \macro{\childdocforward}
% The command |\childdocforward| redirects
% compilation to the main file or
% (if the optional argument is given) a child file.
% Parameters are set as if the main file
% or a child file starting with |\childdocof| was compiled.
% Then compilation is handed over to the main file:
%    \begin{macrocode}
\newcommand{\childdocforward}[2][]
{
  \begingroup
    \if?#1?
      \def\childdoctmp
      {
        \def\childdocname{#2}
        \def\childdocjob{#2}
        \def\jobname{#2}
        \input{#2}
        \endinput
      }
    \else
      \def\childdoctmp
      {
        \childdocdisable
        \def\childdocname{#2}
        \childdoctrue
        \includeonly{#2}
        \def\childdocjob{#1}
        \def\jobname{#1}
        \input{#1}
        \endinput
      }
    \fi
    \expandafter
  \endgroup
  \childdoctmp
}
%    \end{macrocode}

% \macro{\childdocforwardprefix}
% The command |\childdocforwardprefix| redirects
% compilation to the main or a child file by means of a pattern.
% The prefix |#1| in the current filename is replaced by |#2|
% and the suffix of the current filename is kept
% (it is assumed that the filename does not contain the substring `|~~~|'
% which is used as a delimiter).
% Compilation is handed over to the new file by |\childdocforward|:
%    \begin{macrocode}
\newcommand{\childdocforwardprefix}[3][]
{
  \begingroup
    \def\childdocextract #2##1~~~{\def\childdoctmp{\childdocforward[#1]{#3##1}}}
    \expandafter\childdocextract\childdocname~~~
    \expandafter
  \endgroup
  \childdoctmp
}
%    \end{macrocode}

% \macro{\childdoc}
% The deprecated macro |\childdoc| is a legacy version of |\childdocmain|:
%    \begin{macrocode}
\newcommand{\childdoc}{\childdocmain}
%    \end{macrocode}

% \macro{\childdocredirect}
% The deprecated macro |\childdocredirect| is a legacy version
% of |\childdocforward| and |\childdocforwardprefix|:
%    \begin{macrocode}
\newcommand{\childdocredirect}[2][]
{
  \begingroup
    \if?#1?
      \def\childdoctmp{\childdocforward{#2}}
    \else
      \def\childdoctmp{\childdocforwardprefix{#1}{#2}}
    \fi
    \expandafter
  \endgroup
  \childdoctmp
}
%    \end{macrocode}

%\iffalse
%</package>
%\fi
%
\endinput

\childdocforwardprefix[cdocsamp]{cdocsfn}{cdocsch}
%    \end{macrocode}

%\iffalse
%</samplefinal>
%\fi
%
% %%%%%%%%%%%%%%%%%%%%%%%%%%%%%%%%%%%%%%
% \paragraph{Command Line Processing.}
%
% The following three command lines generate the output files
% |cdocscld|, |cdocscl1| and |cdocscl2|
% which should be identical to
% |cdocsdrf|, |cdocsch1| and |cdocsfn2|, respectively:
% \begin{center}
% \begin{tabular}{l}
% |latex -jobname cdocscld \|\\
% |  "\def\version{draft}% \iffalse
%
% childdoc.dtx Copyright (C) 2017-2018 Niklas Beisert
%
% This work may be distributed and/or modified under the
% conditions of the LaTeX Project Public License, either version 1.3
% of this license or (at your option) any later version.
% The latest version of this license is in
%   http://www.latex-project.org/lppl.txt
% and version 1.3 or later is part of all distributions of LaTeX
% version 2005/12/01 or later.
%
% This work has the LPPL maintenance status `maintained'.
%
% The Current Maintainer of this work is Niklas Beisert.
%
% This work consists of the files childdoc.dtx and childdoc.ins
% and the derived files childdoc.def and cdocsamp.tex with
% cdocsch1.tex, cdocsch2.tex, cdocsdrf.tex, cdocsfn1.tex, cdocsfn2.tex.
%
%<package>\ifdefined\childdocmain\endinput\fi
%<package>\ProvidesFile{childdoc.def}[2018/12/30 v2.0 child document driver]
%<samplemain>\ProvidesFile{cdocsamp.tex}[2018/12/30 v2.0 sample for childdoc]
%<*driver>
%\ProvidesFile{childdoc.drv}[2018/12/30 v2.0 childdoc reference manual file]
\PassOptionsToClass{10pt,a4paper}{article}
\documentclass{ltxdoc}

\usepackage[margin=35mm]{geometry}
\usepackage{hyperref}
\usepackage{hyperxmp}
\usepackage[usenames]{color}

\hypersetup{colorlinks=true}
\hypersetup{pdfstartview=FitH}
\hypersetup{pdfpagemode=UseNone}
\hypersetup{pdfsource={}}
\hypersetup{pdflang={en-UK}}
\hypersetup{pdfcopyright={Copyright 2017-2018 Niklas Beisert.
  This work may be distributed and/or modified under the
  conditions of the LaTeX Project Public License, either version 1.3
  of this license or (at your option) any later version.}}
\hypersetup{pdflicenseurl={http://www.latex-project.org/lppl.txt}}
\hypersetup{pdfcontactaddress={ETH Zurich, ITP, HIT K,
  Wolfgang-Pauli-Strasse 27}}
\hypersetup{pdfcontactpostcode={8093}}
\hypersetup{pdfcontactcity={Zurich}}
\hypersetup{pdfcontactcountry={Switzerland}}
\hypersetup{pdfcontactemail={nbeisert@itp.phys.ethz.ch}}
\hypersetup{pdfcontacturl={http://people.phys.ethz.ch/\xmptilde nbeisert/}}

\newcommand{\secref}[1]{\hyperref[#1]{section \ref*{#1}}}

\parskip1ex
\parindent0pt
\let\olditemize\itemize
\def\itemize{\olditemize\parskip0pt}

\begin{document}

\title{The \textsf{childdoc} Package}
\hypersetup{pdftitle={The childdoc Package}}
\author{Niklas Beisert\\[2ex]
  Institut f\"ur Theoretische Physik\\
  Eidgen\"ossische Technische Hochschule Z\"urich\\
  Wolfgang-Pauli-Strasse 27, 8093 Z\"urich, Switzerland\\[1ex]
  \href{mailto:nbeisert@itp.phys.ethz.ch}
  {\texttt{nbeisert@itp.phys.ethz.ch}}}
\hypersetup{pdfauthor={Niklas Beisert}}
\hypersetup{pdfsubject={Manual for the LaTeX2e Package childdoc}}
\date{30 December 2018, \textsf{v2.0}}
\maketitle

\begin{abstract}\noindent
\textsf{childdoc} is a \LaTeXe{} package
that enables the direct compilation
of document sections included by |\include|
to individual files.
\end{abstract}

\begingroup
\parskip0ex
\tableofcontents
\endgroup

%%%%%%%%%%%%%%%%%%%%%%%%%%%%%%%%%%%%%%%%%%%%%%%%%%%%%%%%%%%%%%%%%%%%%%%%%%%%%%%%
%%%%%%%%%%%%%%%%%%%%%%%%%%%%%%%%%%%%%%%%%%%%%%%%%%%%%%%%%%%%%%%%%%%%%%%%%%%%%%%%
\section{Introduction}

\LaTeX{} provides a mechanism to structure a large document (such as a book)
into a main file and several child files (containing the chapters)
using the |\include| command.
This mechanism is beneficial for documents
which span hundreds of pages in order to
make the source file(s) more manageable.
Moreover, compilation can be restricted to
selected child files by means of the |\includeonly| command.
The latter feature can be used to reduce the compilation time while editing
(this was significantly more useful in the earlier days of \LaTeX{})
or to generate a smaller document which is easier to navigate.
Another application of |\includeonly| is to generate
documents consisting of selected parts of the complete document.

However, there are a few drawbacks of the plain |\include| mechanism:
\begin{itemize}
\item
The child files cannot be compiled on their own,
they can only be compiled via the main file.
A naive editing environment
(such as a text editor with an option
to have the current file processed by \LaTeX)
may require one to switch to the main file before compiling;
attempting to compile the child file produces errors.
\item
The main file must be modified (each time)
to adjust the |\includeonly| command
to the present needs. This easily leaves the main file in a messy state.
\item
The generated document will always carry the filename
of the main document. This is inconvenient if
several child files are to be compiled and
to be kept for distribution.
\end{itemize}

The present package provides a simple interface
to make child files individually compilable by \LaTeX{}.
Compiling a child file then has the same effect as compiling
the main file with an |\includeonly| command
to select the appropriate child.
Moreover the generated document will carry the name of the child
rather than the main file.
This resolves all three above issues.

This feature is meant to make the editing of books,
thesis documents and lecture notes somewhat more convenient.
However, the package can also be used efficiently for
composing a series of documents (such as exercise sheets)
which are typically distributed individually.
It then assists the author in generating the individual documents
(potentially in different versions)
as well as a document containing the collected series.
Another application is in developing style files
or other kinds of included material
where compilation of the style file could redirect
to a sample or test file.

%%%%%%%%%%%%%%%%%%%%%%%%%%%%%%%%%%%%%%%%%%%%%%%%%%%%%%%%%%%%%%%%%%%%%%%%%%%%%%%%
%%%%%%%%%%%%%%%%%%%%%%%%%%%%%%%%%%%%%%%%%%%%%%%%%%%%%%%%%%%%%%%%%%%%%%%%%%%%%%%%
\section{Usage}

First of all, the package \textsf{childdoc} is \emph{not} a standard
\LaTeXe{} |.sty| style file! Therefore it needs to be invoked in
a non-standard way.

%%%%%%%%%%%%%%%%%%%%%%%%%%%%%%%%%%%%%%%%%%%%%%%%%%%%%%%%%%%%%%%%%%%%%%%%%%%%%%%%
\subsection{Included Files}
\label{sec:include}

%%%%%%%%%%%%%%%%%%%%%%%%%%%%%%%%%%%%%%%%
\DescribeMacro{\childdocmain}
To use the package, add the commands
\begin{center}
\begin{tabular}{l}
|\input{childdoc.def}|\\
|\childdocmain{}|\\
\end{tabular}
\end{center}
at the very top of the main \LaTeX{} file,
in particular \emph{before} the |\documentclass| statement!
The argument of |\childdocmain| should be left empty
(but it must be present).

%%%%%%%%%%%%%%%%%%%%%%%%%%%%%%%%%%%%%%%%
\DescribeMacro{\childdocof}
Furthermore, add the commands
\begin{center}
\begin{tabular}{l}
|\input{childdoc.def}|\\
|\childdocof{|\textit{main}|}|\\
\end{tabular}
\end{center}
at the top of every child file \textit{child}
which is included by |\include{|\textit{child}|}|
from within the main file
(or at least for those files to be compiled individually).
The argument \textit{main} must be the filename of the main file.

There are a couple of
considerations in setting up the main and child documents:

%%%%%%%%%%%%%%%%%%%%%%%%%%%%%%%%%%%%%%%%
\paragraph{Restrictions.}

Please note the following restrictions:
\begin{itemize}
\item
|\childdocmain| must be called with one argument \textit{main}
to ensure compatibility with earlier version of the package.
It must either be empty (|\childdocmain{}|)
or precisely match the filename of the main file in which it is specified.
See \secref{sec:detection} for further information.
\item
The filename \textit{main} must be specified without the |.tex| extension.
\item
The filename \textit{main} is case sensitive
(even in case-insensitive file systems)
due to internal string comparison.
\item
The argument \textit{main} should be fully expanded, it cannot be a macro.
\item
Subdirectories and special characters should be avoided in filenames.
\item
The command |\childdocmain{|\textit{main}|}| must be followed by a whitespace.
It should not be followed immediately by another command
or by a comment mark `|%|'.
This is because the \TeX{} parser reads the token immediately following
the argument of |\childdocmain| and puts it
at the beginning of every child section;
however, a white\-space is ignored.
\end{itemize}

%%%%%%%%%%%%%%%%%%%%%%%%%%%%%%%%%%%%%%%%
\paragraph{Content of Main File.}

It is advisable to place all content in the child files included by |\include|.
Any output contained in the main file will appear in all child documents
unless suppressed manually;
it cannot be suppressed automatically by the |\includeonly| directive
and thus should normally be avoided.
A method to include some content in the main file
by means of conditional processing is described in \secref{sec:conditional}.

%%%%%%%%%%%%%%%%%%%%%%%%%%%%%%%%%%%%%%%%
\paragraph{Page Numbering.}

When only a part of the document is compiled,
the appropriate numbering of pages
(as well as other status parameters)
is determined from the |.aux| files.
The latter contain information from previous passes.
However this information needs to propagate through
all intermediate child documents.
Therefore the page numbering in child documents may well
be inconsistent until the complete document is compiled at least once.

A useful (if unconventional) way to always ensure a consistent
page numbering is to restart the numbering in each child document
and denote the pages by `\textit{child}|.|\textit{page}'
where \textit{child} represents the chapter/section number of the child file.
This can be achieved by the command
|\numberwithin{page}{|\textit{child}|}|
of the \textsf{amsmath} package
where \textit{child} can be |chapter| or |section|
depending on the chosen structuring.
Alternatively, one can modify the macro |\thepage| appropriately
and reset the counter |page| at the start of each child file.

%%%%%%%%%%%%%%%%%%%%%%%%%%%%%%%%%%%%%%%%%%%%%%%%%%%%%%%%%%%%%%%%%%%%%%%%%%%%%%%%
\subsection{Conditional Processing}
\label{sec:conditional}

The package provides a mechanism to compile different versions
of a document. To customise the versions further some conditional processing
can come in handy to distinguish which version is being compiled.
The package provides two macros to describe the compilation context:

%%%%%%%%%%%%%%%%%%%%%%%%%%%%%%%%%%%%%%%%
\DescribeMacro{\ifchilddoc}
The conditional |\ifchilddoc| distinguishes between the compilation of
child documents and the main document:
%
\begin{center}
|\ifchilddoc |\textit{child-code}| |[|\||else |\textit{main-code}]| \||fi|
\end{center}

%%%%%%%%%%%%%%%%%%%%%%%%%%%%%%%%%%%%%%%%
\DescribeMacro{\childdocname}
\DescribeMacro{\childdocjob}
The macro |\childdocname| contains the filename (without extension)
of the main or child file being processed.
Note that |\childdocjob| will always contain the name of the main file.

%%%%%%%%%%%%%%%%%%%%%%%%%%%%%%%%%%%%%%%%
\paragraph{Title Page.}

Conditional processing can be used to include a title or banner page
in the main document when proper precautions are taken.
Importantly, the code in the main file should ensure that the page counter
(as well as other status parameters which are stored in the |.aux| files)
takes the same value after the conditional processing.
Otherwise the page numbers may take divergent values
depending on which part is compiled.

For example, a title page could be declared by:
%
\begin{center}
\begin{tabular}{l}
|\ifchilddoc\||else|\\
|\addtocounter{page}{-1}|\\
\textit{code for title page}\\
|\newpage|\\
|\||fi|
\end{tabular}
\end{center}
%
A banner page for the child documents can be generated by:
%
\begin{center}
\begin{tabular}{l}
|\ifchilddoc|\\
|\addtocounter{page}{-1}|\\
\textit{code for banner page}\\
|\newpage|\\
|\||fi|
\end{tabular}
\end{center}
%
Here one could write a message such as:
\begin{center}
|This is the part \childdocname{} of \childdocjob{}.|
\end{center}

%%%%%%%%%%%%%%%%%%%%%%%%%%%%%%%%%%%%%%%%%%%%%%%%%%%%%%%%%%%%%%%%%%%%%%%%%%%%%%%%
\subsection{Flags}
\label{sec:flags}

The package makes it easy to generate different versions
of the main or child documents.
To this end compilation flags can be defined
and assigned different default values.
They will be particularly useful in conjunction
with the forwarding mechanism described in \secref{sec:forward}.

For example, it may be useful to have a flag |\version|
which can be set to |draft| or |final|.
The document source will contain some conditional code
depending on the value of |\version|.
Suppose further, the flag should default to |final| for the main file
and to |draft| for child files
which is a natural assignment for editing the document.
This is achieved by placing the following code
in the preamble of the main document
(below the |\childdocmain| directive):
%
\begin{center}
\begin{tabular}{l}
|\ifchilddoc|\\
|\providecommand{\version}{draft}|\\
|\||else|\\
|\providecommand{\version}{final}|\\
|\||fi|
\end{tabular}
\end{center}
%
The definition by |\providecommand| makes sure
that previous definitions are not overwritten.
Further statements |\providecommand{\version}{...}|
can thus be added before the above code to override it.

For the main file, one might add a line
(between |\childdocmain| and the above block)
%
\begin{center}
|%\ifchilddoc\||else\providecommand{\version}{draft}\||fi|
\end{center}
%
which can be uncommented to produce a draft version.
Likewise one can add a line to the very top of a child file
(above the |\childdocof{|\textit{main}|}| directive)
%
\begin{center}
|%\providecommand{\version}{final}|
\end{center}
%
which can be uncommented to produce the final version of this child document.

%%%%%%%%%%%%%%%%%%%%%%%%%%%%%%%%%%%%%%%%%%%%%%%%%%%%%%%%%%%%%%%%%%%%%%%%%%%%%%%%
\subsection{Forwarding}
\label{sec:forward}

Different versions of the main or child documents
using compilation flags as described in \secref{sec:flags}
can be (permanently) stored in different files
for convenient compilation, viewing and distribution.
To this end, the package defines a command
to pass on compilation to a different file:

%%%%%%%%%%%%%%%%%%%%%%%%%%%%%%%%%%%%%%%%
\DescribeMacro{\childdocforward}
The command |\childdocforward| redirects processing to
another source file:
%
\begin{center}
\begin{tabular}{l}
|\input{childdoc.def}|\\
|\childdocforward[|\textit{main}|]{|\textit{dest}|}|\\
\end{tabular}
\end{center}
%
The argument \textit{dest} is the destination file
(without extension).
It should be the main file or one of the child files.
Note that further \textsf{childdoc} directives
such as |\childdocof| and |\childdocforward|
in the indicated file will be processed in this form.
The optional argument \textit{main}
passes on directly to the main file \textit{main}
while pretending to compile the child \textit{dest}.
This form behaves as if \textit{dest}
issues |\childdocof{|\textit{main}|}| right away,
and no further \textsf{childdoc} directives will be processed.

%%%%%%%%%%%%%%%%%%%%%%%%%%%%%%%%%%%%%%%%
\DescribeMacro{\...prefix}
In the alternative form |\childdocforwardprefix|,
%
\begin{center}
\begin{tabular}{l}
|\input{childdoc.def}|\\
|\childdocforwardprefix[|\textit{main}|]{|\textit{prefix}|}{|\textit{dest}|}|
\end{tabular}
\end{center}
%
the destination file is determined by a pattern
depending on the current file:
To make this work, the current file must be called
`{\textit{prefix}\hspace{0.2em}\textit{suffix}}'
with \textit{prefix} matching precisely the argument.
Processing is then passed on to the file
`{\textit{dest}\hspace{0.2em}\textit{suffix}}'.
Surely, the same effect is achieved by
directly specifying the
argument `{\textit{dest}\hspace{0.2em}\textit{suffix}}'
in the first form.
However, that requires to set up a different file
for each child. With the alternative form of the command
all these files can have exactly the same content
which simplifies setting them up and maintaining them.

For example, the following file |draft.tex|
with a compilation flag |\version| as described in \secref{sec:flags}
compiles the main document as a draft:
%
\begin{center}
\begin{tabular}{l}
|\def\version{draft}|\\
|\input{childdoc.def}|\\
|\childdocforward{|\textit{main}|}|
\end{tabular}
\end{center}
%
Likewise, the following files |final|\textit{nn}|.tex|
compile the final version of the child document
|child|\textit{nn}|.tex|:
%
\begin{center}
\begin{tabular}{l}
|\def\version{final}|\\
|\input{childdoc.def}|\\
|\childdocforwardprefix{final}{child}|
\end{tabular}
\end{center}
%

Note that when several versions of a main file and/or of each child file
are to be generated, it may be convenient to set up a |Makefile| or
shell script to automatise the process.

%%%%%%%%%%%%%%%%%%%%%%%%%%%%%%%%%%%%%%%%%%%%%%%%%%%%%%%%%%%%%%%%%%%%%%%%%%%%%%%%
\subsection{Command Line Processing}
\label{sec:commandline}

The effect of redirection files can also be achieved by invoking
the \LaTeX{} compiler with a more elaborate command line.
Most conveniently this should be done as part
of a shell script or a |Makefile|.

When using \textsf{childdoc} in the main file, the following
command lines effectively perform a redirection
(note that depending on the shell being used,
backslashes may have to be doubled: `|\|' $\to$ `|\\|'):
%
\begin{center}
|... -jobname "|\textit{target}|" |\\|"|[\textit{flags}]%
|\input{childdoc.def}\childdocforward[|\textit{main}|]{|\textit{dest}|}"|
\end{center}
%
Here \textit{target} is the name of the output file,
\textit{main} is the name of the main file
and \textit{dest} is the name of the main or child file to be processed
(all filenames without extensions).
The optional argument \textit{main} can be omitted
if \textit{main} matches \textit{dest}.
Optionally, compilation \textit{flags} can be defined via |\def| commands.
This command line makes the \TeX{} engine believe
it is compiling the file \textit{target}
whose content is specified as the latter parameter.
The provided code then forwards the processing to
\textit{main} or \textit{dest} as described in \secref{sec:forward}.

%%%%%%%%%%%%%%%%%%%%%%%%%%%%%%%%%%%%%%%%%%%%%%%%%%%%%%%%%%%%%%%%%%%%%%%%%%%%%%%%
\subsection{Include by Input}
\label{sec:input}

Including child documents by |\include| has some restrictions by design.
Most notably, the content of a child document always occupies
its own set of pages; pages cannot be shared between child documents.
Usually, this behaviour makes perfect sense
because each child document contain an essential part of the document.
However, in some situations it may be desirable to compose
a document from a collection of parts
without having mandatory page breaks between then.
For this case, the package
provides a mechanism to include parts
by |\input| which can also be processed individually.
However, by construction this mechanism
requires manual handling of the content to be output.

%%%%%%%%%%%%%%%%%%%%%%%%%%%%%%%%%%%%%%%%
\DescribeMacro{\ifchilddocmanual}
The main file should be prepared as usual, see \secref{sec:include}.
However, the document body must make a distinction
between processing of an individual part and of the main document, e.g.:
%
\begin{center}
\begin{tabular}{l}
|\ifchilddocmanual|\\
|\input{\childdocname}|\\
|\||else|\\
\textit{document body with }|\input{|\textit{part}|}|\\
|\||fi|
\end{tabular}
\end{center}
%
The conditional |\ifchilddocmanual| is true whenever
a part to be included by |\input| is being compiled,
and the name of the part is stored in |\childdocname|.

%%%%%%%%%%%%%%%%%%%%%%%%%%%%%%%%%%%%%%%%
\DescribeMacro{\childdocby}
Each part to be included by |\input| should start with:
%
\begin{center}
\begin{tabular}{l}
|\input{childdoc.def}|\\
|\childdocby{|\textit{main}|}|\\
\end{tabular}
\end{center}
%
The directive |\childdocby| is similar to |\childdocof|
described in \secref{sec:include},
but the subsequent selection of content must be done manually.
To that end, both |\ifchilddoc| and |\ifchilddocmanual|
will be true upon processing of a part,
and the name of the part is stored in |\childdocname|.
Note that |\jobname| will be set to the filename of the current part
so that each part receives an individual |.aux| file
that does not interfere with the |.aux| file(s) of the main document.
This behaviour can be altered by the alternative form
|\childdocby[*]{|\textit{main}|}| (with a non-empty optional argument)
which uses the |.aux| file of the main document
by setting |\jobname| to \textit{main}.

%%%%%%%%%%%%%%%%%%%%%%%%%%%%%%%%%%%%%%%%%%%%%%%%%%%%%%%%%%%%%%%%%%%%%%%%%%%%%%%%
\subsection{Driver Development}
\label{sec:driver}

The \textsf{childdoc} mechanism can also be use for the development
of definition files such as \LaTeX{} styles or classes.
This case differs from the above setup with multiple parts
included by |\include| in that no |\includeonly| should be invoked.
This can be achieved by starting the include file
(before |\ProvidesPackage|) with:
%
\begin{center}
\begin{tabular}{l}
|\input{childdoc.def}|\\
|\childdocforward{|\textit{main}|}|\\
\end{tabular}
\end{center}
%
or alternatively with:
%
\begin{center}
\begin{tabular}{l}
|\input{childdoc.def}|\\
|\childdocby{|\textit{main}|}|\\
\end{tabular}
\end{center}
%
Both forms have slightly different effects as described above.
The main file is prepared as usual, see \secref{sec:include}.

%%%%%%%%%%%%%%%%%%%%%%%%%%%%%%%%%%%%%%%%%%%%%%%%%%%%%%%%%%%%%%%%%%%%%%%%%%%%%%%%
\subsection{Legacy Detection}
\label{sec:detection}

The directive |\childdocmain| in the main file can detect
whether the complete document or merely a child is to be compiled
even without using the directive |\childdocof|.
This method is deprecated because it is less robust
and there is no compelling reason to use it;
it is merely provided for backward compatibility
and it may be removed in future versions.

If the detection mechanism is to be used,
it is mandatory to correctly specify
the filename of the main file as the argument of |\childdocmain|:
%
\begin{center}
\begin{tabular}{l}
|\input{childdoc.def}|\\
|\childdocmain{|\textit{main}|}|\\
\end{tabular}
\end{center}
%
If |\jobname| does not match the argument \textit{main} of |\childdocmain|,
it is assumed that |\jobname| points to the child file to be compiled.
When using |\childdocmain| with the main file specified as argument,
it suffices to start a child file
with just |\input{|\textit{main}|}|
without loading of the package and using |\childdocof|.
If instead all processing is done
with the appropriate \textsf{childdoc} directives,
the argument of \textit{main} of |\childdocmain| can be empty.

An alternative version of the command line processing described
in \secref{sec:commandline} using the detection mechanism reads:
%
\begin{center}
|... -jobname "|\textit{target}|" "|[\textit{flags}]%
[|\def\jobname{|\textit{dest}|}|]|\input{|\textit{main}|}"|
\end{center}

%%%%%%%%%%%%%%%%%%%%%%%%%%%%%%%%%%%%%%%%%%%%%%%%%%%%%%%%%%%%%%%%%%%%%%%%%%%%%%%%
\subsection{Manual Code}
\label{sec:manual}

In case one cannot be certain whether the definitions file |childdoc.def|
is installed on the target \TeX{} distribution
and one prefers not to ship it,
it is conceivable to paste a few relevant commands into the sources.

To that end, drop all statements |\input{childdoc.def}|
and perform the replacements as outlined below.
Instead of |\childdocmain{|\textit{main}|}| add the following code
to the top of the main file:
%
\begin{center}
\begin{tabular}{l}
|\||ifdefined\childdocname\endinput\||fi\newif\ifchilddoc|\\
|\edef\childdocname{\scantokens\expandafter{\jobname\noexpand}}|\\
|\def\childdocmain{|\textit{main}|}\||ifx\childdocmain\childdocname\||else|\\
|\childdoctrue\includeonly{\childdocname}\let\jobname\childdocmain\||fi|\\
\end{tabular}
\end{center}
%
Instead of |\childdocof{|\textit{main}|}| just include the main file
at the top of each child file:
%
\begin{center}
|\input{|\textit{main}|}|
\end{center}
%
A simple redirection |\childdocforward{|\textit{dest}|}| is achieved by:
%
\begin{center}
|\def\jobname{|\textit{dest}|}\input{\jobname}|
\end{center}
%
The redirection with prefix
|\childdocforwardprefix[|\textit{prefix}|]{|\textit{dest}|}|
is accomplished by:
%
\begin{center}
\begin{tabular}{l}
|{\edef\jobname{\scantokens\expandafter{\jobname\noexpand}}|\\
|\def\redirectjob |\textit{prefix}|#1~~~{\gdef\jobname{|\textit{dest}|#1}}|\\
|\expandafter\redirectjob\jobname~~~}\input{\jobname}|
\end{tabular}
\end{center}

In an alternative approach,
child documents can be compiled by a specific command line
without additional code or specific definitions:
%
\begin{center}
|... -jobname "|\textit{target}|" "|[\textit{flags}]%
|\includeonly{|\textit{dest}|}\input{|\textit{main}|}"|
\end{center}
%

%%%%%%%%%%%%%%%%%%%%%%%%%%%%%%%%%%%%%%%%%%%%%%%%%%%%%%%%%%%%%%%%%%%%%%%%%%%%%%%%
%%%%%%%%%%%%%%%%%%%%%%%%%%%%%%%%%%%%%%%%%%%%%%%%%%%%%%%%%%%%%%%%%%%%%%%%%%%%%%%%
\section{Information}

%%%%%%%%%%%%%%%%%%%%%%%%%%%%%%%%%%%%%%%%%%%%%%%%%%%%%%%%%%%%%%%%%%%%%%%%%%%%%%%%
\subsection{Copyright}

Copyright \copyright{} 2017--2018 Niklas Beisert

This work may be distributed and/or modified under the
conditions of the \LaTeX{} Project Public License, either version 1.3
of this license or (at your option) any later version.
The latest version of this license is in
  \url{http://www.latex-project.org/lppl.txt}
and version 1.3 or later is part of all distributions of \LaTeX{}
version 2005/12/01 or later.

This work has the LPPL maintenance status `maintained'.

The Current Maintainer of this work is Niklas Beisert.

This work consists of the files |README.txt|, |childdoc.ins| and |childdoc.dtx|
as well as the derived files |childdoc.def|, |cdocsamp.tex|
with |cdocsch1.tex|, |cdocsch2.tex|, |cdocspt3.tex|, |cdocspt4.tex|,
|cdocsdrf.tex|, |cdocsfn1.tex|, |cdocsfn2.tex|
as well as |childdoc.pdf|.

%%%%%%%%%%%%%%%%%%%%%%%%%%%%%%%%%%%%%%%%%%%%%%%%%%%%%%%%%%%%%%%%%%%%%%%%%%%%%%%%
\subsection{Files and Installation}

The package consists of the files:
%
\begin{center}
\begin{tabular}{ll}
    |README.txt|   & readme file \\
    |childdoc.ins| & installation file \\
    |childdoc.dtx| & source file \\
    |childdoc.def| & definition file \\
    |cdocsamp.tex| & sample main file \\
    |cdocsch1.tex| & sample include file \\
    |cdocsch2.tex| & sample include file \\
    |cdocspt3.tex| & sample part file \\
    |cdocspt4.tex| & sample part file \\
    |cdocsdrf.tex| & sample redirection file \\
    |cdocsfn1.tex| & sample redirection file \\
    |cdocsfn2.tex| & sample redirection file \\
    |childdoc.pdf| & manual
\end{tabular}
\end{center}
%
The distribution consists of the files
|README.txt|, |childdoc.ins| and |childdoc.dtx|.
%
\begin{itemize}
\item
Run (pdf)\LaTeX{} on |childdoc.dtx|
to compile the manual |childdoc.pdf| (this file).
\item
Run \LaTeX{} on |childdoc.ins| to create the definitions file |childdoc.def|
and the sample |cdocsamp.tex| with include files
|cdocsch1.tex|, |cdocsch2.tex|, |cdocspt3.tex|, |cdocspt4.tex|,
|cdocsdrf.tex|, |cdocsfn1.tex|, |cdocsfn2.tex|.
Then copy the file |childdoc.def| to an appropriate directory of your \LaTeX{}
distribution, e.g.\ \textit{texmf-root}|/tex/latex/childdoc|.
\end{itemize}

%%%%%%%%%%%%%%%%%%%%%%%%%%%%%%%%%%%%%%%%%%%%%%%%%%%%%%%%%%%%%%%%%%%%%%%%%%%%%%%%
\subsection{Related CTAN Packages}

There are several other packages which offer a similar functionality:
%
\begin{itemize}
\item
The packages
\href{http://ctan.org/pkg/docmute}{\textsf{docmute}},
\href{http://ctan.org/pkg/includex}{\textsf{includex}} and
\href{http://ctan.org/pkg/standalone}{\textsf{standalone}}
provide commands to include only the document body of
a child file thus allowing both files to be compiled individually.
\item
The packages \href{http://ctan.org/pkg/subdocs}{\textsf{subdocs}}
and \href{http://ctan.org/pkg/subfiles}{\textsf{subfiles}}
provide structures in which the main and child documents can be
encapsulated and allowing them to be compiled individually.
The inclusion mechanism is different from the conventional |\include|.
\item
The package \href{http://ctan.org/pkg/combine}{\textsf{combine}}
is an elaborate solution to combine several documents into one.
\end{itemize}
%
See also the CTAN topic \href{http://ctan.org/topic/subdocs}{\textsf{subdocs}}
for further related packages.
The present package differs from the above solutions in that
a document structure constructed with the conventional |\include| mechanism
just needs two extra commands at the top of every file
such that all constituent files can be compiled individually.

%%%%%%%%%%%%%%%%%%%%%%%%%%%%%%%%%%%%%%%%%%%%%%%%%%%%%%%%%%%%%%%%%%%%%%%%%%%%%%%%
%\subsection{Feature Suggestions}
%
%The following is a list of features which may be useful for future
%versions of this package:
%%
%\begin{itemize}
%\item
%\ldots
%\end{itemize}

%%%%%%%%%%%%%%%%%%%%%%%%%%%%%%%%%%%%%%%%%%%%%%%%%%%%%%%%%%%%%%%%%%%%%%%%%%%%%%%%
\subsection{Revision History}

%%%%%%%%%%%%%%%%%%%%%%%%%%%%%%%%%%%%%%%%
\paragraph{v2.0:} 2018/12/30

\begin{itemize}
\item
immediate forward processing
\item
added |\childdocby| mechanism
\item
manual restructured
\end{itemize}

%%%%%%%%%%%%%%%%%%%%%%%%%%%%%%%%%%%%%%%%
\paragraph{v1.6:} 2018/01/17

\begin{itemize}
\item
application for development of include files
\item
corrections to manual
\end{itemize}

%%%%%%%%%%%%%%%%%%%%%%%%%%%%%%%%%%%%%%%%
\paragraph{v1.5:} 2017/05/21

\begin{itemize}
\item
more complete structuring introduced
\item
|\childdocof| introduced
\item
|\childdoc| renamed to |\childdocmain|
\item
|\childredirect| renamed to |\childdocforward| and |\childdocforwardprefix|
and functionality expanded
\end{itemize}

%%%%%%%%%%%%%%%%%%%%%%%%%%%%%%%%%%%%%%%%
\paragraph{v1.0:} 2017/04/27

\begin{itemize}
\item
manual and install package
\item
first version published on CTAN
\end{itemize}

%%%%%%%%%%%%%%%%%%%%%%%%%%%%%%%%%%%%%%%%
\paragraph{v0.6:} 2017/04/26

\begin{itemize}
\item
redirection mechanism added
\end{itemize}

%%%%%%%%%%%%%%%%%%%%%%%%%%%%%%%%%%%%%%%%
\paragraph{v0.5:} 2017/04/26

\begin{itemize}
\item
functionality in definition file
\end{itemize}


%%%%%%%%%%%%%%%%%%%%%%%%%%%%%%%%%%%%%%%%%%%%%%%%%%%%%%%%%%%%%%%%%%%%%%%%%%%%%%%%
%%%%%%%%%%%%%%%%%%%%%%%%%%%%%%%%%%%%%%%%%%%%%%%%%%%%%%%%%%%%%%%%%%%%%%%%%%%%%%%%
%%%%%%%%%%%%%%%%%%%%%%%%%%%%%%%%%%%%%%%%%%%%%%%%%%%%%%%%%%%%%%%%%%%%%%%%%%%%%%%%
\appendix

\settowidth\MacroIndent{\rmfamily\scriptsize 000\ }

 \DocInput{childdoc.dtx}

\end{document}
%</driver>
% \fi
%
% %%%%%%%%%%%%%%%%%%%%%%%%%%%%%%%%%%%%%%%%%%%%%%%%%%%%%%%%%%%%%%%%%%%%%%%%%%%%%%
% %%%%%%%%%%%%%%%%%%%%%%%%%%%%%%%%%%%%%%%%%%%%%%%%%%%%%%%%%%%%%%%%%%%%%%%%%%%%%%
% \section{Sample}
%\iffalse
%<*samplemain>
%\fi
%
% The following presents a sample document
% with two chapters, two parts, a title page,
% a compile flag as well as three forwarding files to set the flag.
% It consists of eight |.tex| files:
% \begin{center}
% \begin{tabular}{ll}
% |cdocsamp.tex|&main file\\
% |cdocsch1.tex|&include file for chapter 1\\
% |cdocsch2.tex|&include file for chapter 2\\
% |cdocspt3.tex|&include file for part 3\\
% |cdocspt4.tex|&include file for part 4\\
% |cdocsdrf.tex|&forwarding file for main file in draft mode\\
% |cdocsfi1.tex|&forwarding file for final version of chapter 1\\
% |cdocsfi2.tex|&forwarding file for final version of chapter 2\\
% \end{tabular}
% \end{center}
% Each of the eight files can be compiled directly by the \LaTeX{} compiler.
%
% %%%%%%%%%%%%%%%%%%%%%%%%%%%%%%%%%%%%%%
% \paragraph{Main File.}
%
% The main file is called |cdocsamp.tex|.
%
% Load the \textsf{childdoc} definitions and
% declare the filename for the main document:
%    \begin{macrocode}
\input{childdoc.def}
\childdocmain{}
%    \end{macrocode}

% Optional override for |\version| flag:
%    \begin{macrocode}
%%\ifchilddoc\else\providecommand{\version}{draft}\fi
%    \end{macrocode}

% Define the default values for the |\version| flag
% (|final| for the main file and |draft| for childs):
%    \begin{macrocode}
\ifchilddoc
\providecommand{\version}{draft}
\else
\providecommand{\version}{final}
\fi
%    \end{macrocode}

% Load the standard document class:
%    \begin{macrocode}
\documentclass[12pt]{article}
%    \end{macrocode}

% Start the document body:
%    \begin{macrocode}
\begin{document}
%    \end{macrocode}

% Declare a title page.
% Print title, part of document being processed and version flag:
%    \begin{macrocode}
\addtocounter{page}{-1}
\begin{center}
{\LARGE\bfseries{}childdoc example\par}
\vspace{1cm}
\ifchilddoc
\ifchilddocmanual part\else chapter\fi:
`\childdocname' of `\childdocjob'\par
\else
main document: `\childdocjob'\par
\fi
version: \version\par
\end{center}
\newpage
%    \end{macrocode}

% Manually include selected file,
% otherwise process as usual:
%    \begin{macrocode}
\ifchilddocmanual
\section*{part `\childdocname'}
\input{\childdocname}
\else
%    \end{macrocode}

% Include the two chapters:
%    \begin{macrocode}
\include{cdocsch1}
\include{cdocsch2}
%    \end{macrocode}

% Include the two parts unless only chapters should be displayed:
%    \begin{macrocode}
\ifchilddoc\else
\section{part three}
\input{cdocspt3}
\section{part four}
\input{cdocspt4}
\fi
%    \end{macrocode}

% Process as usual until here:
%    \begin{macrocode}
\fi
%    \end{macrocode}

% End of document body:
%    \begin{macrocode}
\end{document}
%    \end{macrocode}
%\iffalse
%</samplemain>
%\fi
%
% %%%%%%%%%%%%%%%%%%%%%%%%%%%%%%%%%%%%%%
% \paragraph{Chapter Include Files.}
%
% The include files are called |cdocsch1.tex| and |cdocsch2.tex|.
%
%\iffalse
%<*samplechap1|samplechap2>
%\fi

% Optional override for |\version| flag:
%    \begin{macrocode}
%%\providecommand{\version}{final}
%    \end{macrocode}

% Include the main document:
%    \begin{macrocode}
\input{childdoc.def}
\childdocof{cdocsamp}
%    \end{macrocode}

%\iffalse
%</samplechap1|samplechap2>
%\fi
%
%\iffalse
%<*samplechap1>
%\fi
% Some text for chapter 1:
%    \begin{macrocode}
\section{one}
some text in chapter one
%    \end{macrocode}

%\iffalse
%</samplechap1>
%\fi
% Some text for chapter 2:
%\iffalse
%<*samplechap2>
%\fi
%    \begin{macrocode}
\section{two}
more text in chapter two
%    \end{macrocode}

%\iffalse
%</samplechap2>
%\fi
%
% %%%%%%%%%%%%%%%%%%%%%%%%%%%%%%%%%%%%%%
% \paragraph{Part Include Files.}
%
% The include files are called |cdocspt3.tex| and |cdocspt4.tex|.
%
%\iffalse
%<*samplepart3|samplepart4>
%\fi

% Optional override for |\version| flag:
%    \begin{macrocode}
%%\providecommand{\version}{final}
%    \end{macrocode}

% Include the main document:
%    \begin{macrocode}
\input{childdoc.def}
\childdocby{cdocsamp}
%    \end{macrocode}

%\iffalse
%</samplepart3|samplepart4>
%\fi
%
%\iffalse
%<*samplepart3>
%\fi
% Some text for part 3:
%    \begin{macrocode}
some text in part three
%    \end{macrocode}

%\iffalse
%</samplepart3>
%\fi
% Some text for part 4:
%\iffalse
%<*samplepart4>
%\fi
%    \begin{macrocode}
more text in part four
%    \end{macrocode}

%\iffalse
%</samplepart4>
%\fi
%
% %%%%%%%%%%%%%%%%%%%%%%%%%%%%%%%%%%%%%%
% \paragraph{Forwarding for a Complete Draft.}
%
% The following forwarding file |cdocsdrf.tex|
% compiles the main document in draft mode:
%\iffalse
%<*sampledraft>
%\fi
%    \begin{macrocode}
\def\version{draft}
\input{childdoc.def}
\childdocforward{cdocsamp}
%    \end{macrocode}

%\iffalse
%</sampledraft>
%\fi
%
% %%%%%%%%%%%%%%%%%%%%%%%%%%%%%%%%%%%%%%
% \paragraph{Forwarding for Final Version of the Chapters.}
%
% The following forwarding files |cdocsfn1.tex| and |cdocsfn2.tex|
% (with identical content)
% compile the final versions of the child documents
% |cdocsch1.tex| and |cdocsch2.tex|, respectively:
%\iffalse
%<*samplefinal>
%\fi
%    \begin{macrocode}
\def\version{final}
\input{childdoc.def}
\childdocforwardprefix[cdocsamp]{cdocsfn}{cdocsch}
%    \end{macrocode}

%\iffalse
%</samplefinal>
%\fi
%
% %%%%%%%%%%%%%%%%%%%%%%%%%%%%%%%%%%%%%%
% \paragraph{Command Line Processing.}
%
% The following three command lines generate the output files
% |cdocscld|, |cdocscl1| and |cdocscl2|
% which should be identical to
% |cdocsdrf|, |cdocsch1| and |cdocsfn2|, respectively:
% \begin{center}
% \begin{tabular}{l}
% |latex -jobname cdocscld \|\\
% |  "\def\version{draft}\input{childdoc.def}\childdocforward{cdocsamp}"|\\
% |latex -jobname cdocscl1 \|\\
% |  "\input{childdoc.def}\childdocforward[cdocsamp]{cdocsch1}"|\\
% |latex -jobname cdocscl2 \|\\
% |  "\def\version{final}\input{childdoc.def}\childdocforward{cdocsch2}"|
% \end{tabular}
% \end{center}
% Note that the trailing backslash on each first line
% merely continues the input to the second line
% (for convenient cut ant paste).
% Furthermore, the command |latex| can be replaced by any
% of its alternative versions such as |pdflatex|.
%
% %%%%%%%%%%%%%%%%%%%%%%%%%%%%%%%%%%%%%%%%%%%%%%%%%%%%%%%%%%%%%%%%%%%%%%%%%%%%%%
% %%%%%%%%%%%%%%%%%%%%%%%%%%%%%%%%%%%%%%%%%%%%%%%%%%%%%%%%%%%%%%%%%%%%%%%%%%%%%%
% \section{Implementation}
%\iffalse
%<*package>
%\fi
%
% This section describes the definitions file |childdoc.def|.

% The definitions cannot be loaded using |\usepackage| or |\RequirePackage|
% which has a mechanism to prevent loading a style file more than once.
% When loading the definitions by means of |\input|
% multiple instances have to be prevented manually:
%\iffalse
%This code needs to be before the `\ProvidesFile' directive
%which is defined at the beginning of this file.
%Therefore it is also placed there and commented out here.
%</package>
%<*discard>
%\fi
%    \begin{macrocode}
\ifdefined\childdocmain\endinput\fi
%    \end{macrocode}
%\iffalse
%</discard>
%<*package>
%\fi
%
% \macro{\ifchilddoc}
% \macro{\ifchilddocmanual}
% The conditional |\ifchilddoc| tells whether a
% child (true) or main (false) document is being compiled.
% The conditional |\ifchilddocmanual| tells whether
% the |\includeonly| mechanism is used (false) or
% the selection of child files must be performed manually (true).
% The definitions initialise to false:
%    \begin{macrocode}
\newif\ifchilddoc
\newif\ifchilddocmanual
%    \end{macrocode}

% \macro{\childdocname}
% \macro{\childdocjob}
% The macro |\childdocname| stores the name of the main document
% to be compiled. The macro |\childdocjob| stores the name of
% the document on which the \LaTeX{} compiler was originally invoked.
% The content of |\jobname| cannot be compared
% to filenames specified in the source due to different catcodes.
% The following code rescans |\jobname|, stores the result
% in |\childdocname| and saves a copy in |\childdocjob|:
%    \begin{macrocode}
\edef\childdocname{\scantokens\expandafter{\jobname\noexpand}}
\let\childdocjob\childdocname
%    \end{macrocode}

% \macro{\childdocdisable}
% The macro |\childdocdisable| prevents the main file
% from being processed more than once.
% At this stage, the main document command |\childdocmain|
% is assumed to be called once again where it should do nothing.
% Any subsequent call to it should prevent
% a secondary processing of the main document
% It overwrites the forwarding commands
% |\childdocof| and |\childdocforward|
% with empty macros to prevent further inclusions of the main document:
%    \begin{macrocode}
\newcommand{\childdocdisable}
{
  \renewcommand{\childdocmain}[1]{\renewcommand{\childdocmain}[1]{\endinput}}
  \renewcommand{\childdocof}[1]{}
  \renewcommand{\childdocby}[2][]{}
  \renewcommand{\childdocforward}[2][]{}
  \renewcommand{\childdocdisable}{}
}
%    \end{macrocode}

% \macro{\childdocmain}
% The macro |\childdocmain| is to be called at the top of the main file
% with nothing or the main filename (without extension) as argument.
% First, it breaks loops.
% If the argument is not empty and does not match |\childdocname|
% (which is set by the first inclusion of |childdoc.def|),
% |\ifchilddoc| is set to true, |\includeonly| is applied to the child file
% and |\jobname| is set to the main file
% (for proper handling of |.aux| files):
%    \begin{macrocode}
\newcommand{\childdocmain}[1]
{
  \childdocdisable\childdocmain{}
  \if?#1?\else
    \begingroup
      \def\childdoctmp{#1}
      \ifx\childdoctmp\childdocname
        \def\childdoctmp{}
      \else
        \def\childdoctmp
        {
          \childdoctrue
          \includeonly{\childdocname}
          \def\childdocjob{#1}
          \def\jobname{#1}
        }
      \fi
      \expandafter
    \endgroup
    \childdoctmp
  \fi
}
%    \end{macrocode}

% \macro{\childdocof}
% The command |\childdocof| redirects
% compilation to the main file |#1|.
%    \begin{macrocode}
\newcommand{\childdocof}[1]
{
  \childdocdisable
  \childdoctrue
  \includeonly{\childdocname}
  \def\jobname{#1}
  \def\childdocjob{#1}
  \input{#1}
}
%    \end{macrocode}

% \macro{\childdocby}
% The command |\childdocby| ....
%    \begin{macrocode}
\newcommand{\childdocby}[2][]
{
  \childdocdisable
  \childdoctrue
  \childdocmanualtrue
  \if?#1?\else
    \def\jobname{#2}
  \fi
  \def\childdocjob{#2}
  \input{#2}
  \endinput
}
%    \end{macrocode}

% \macro{\childdocforward}
% The command |\childdocforward| redirects
% compilation to the main file or
% (if the optional argument is given) a child file.
% Parameters are set as if the main file
% or a child file starting with |\childdocof| was compiled.
% Then compilation is handed over to the main file:
%    \begin{macrocode}
\newcommand{\childdocforward}[2][]
{
  \begingroup
    \if?#1?
      \def\childdoctmp
      {
        \def\childdocname{#2}
        \def\childdocjob{#2}
        \def\jobname{#2}
        \input{#2}
        \endinput
      }
    \else
      \def\childdoctmp
      {
        \childdocdisable
        \def\childdocname{#2}
        \childdoctrue
        \includeonly{#2}
        \def\childdocjob{#1}
        \def\jobname{#1}
        \input{#1}
        \endinput
      }
    \fi
    \expandafter
  \endgroup
  \childdoctmp
}
%    \end{macrocode}

% \macro{\childdocforwardprefix}
% The command |\childdocforwardprefix| redirects
% compilation to the main or a child file by means of a pattern.
% The prefix |#1| in the current filename is replaced by |#2|
% and the suffix of the current filename is kept
% (it is assumed that the filename does not contain the substring `|~~~|'
% which is used as a delimiter).
% Compilation is handed over to the new file by |\childdocforward|:
%    \begin{macrocode}
\newcommand{\childdocforwardprefix}[3][]
{
  \begingroup
    \def\childdocextract #2##1~~~{\def\childdoctmp{\childdocforward[#1]{#3##1}}}
    \expandafter\childdocextract\childdocname~~~
    \expandafter
  \endgroup
  \childdoctmp
}
%    \end{macrocode}

% \macro{\childdoc}
% The deprecated macro |\childdoc| is a legacy version of |\childdocmain|:
%    \begin{macrocode}
\newcommand{\childdoc}{\childdocmain}
%    \end{macrocode}

% \macro{\childdocredirect}
% The deprecated macro |\childdocredirect| is a legacy version
% of |\childdocforward| and |\childdocforwardprefix|:
%    \begin{macrocode}
\newcommand{\childdocredirect}[2][]
{
  \begingroup
    \if?#1?
      \def\childdoctmp{\childdocforward{#2}}
    \else
      \def\childdoctmp{\childdocforwardprefix{#1}{#2}}
    \fi
    \expandafter
  \endgroup
  \childdoctmp
}
%    \end{macrocode}

%\iffalse
%</package>
%\fi
%
\endinput
\childdocforward{cdocsamp}"|\\
% |latex -jobname cdocscl1 \|\\
% |  "% \iffalse
%
% childdoc.dtx Copyright (C) 2017-2018 Niklas Beisert
%
% This work may be distributed and/or modified under the
% conditions of the LaTeX Project Public License, either version 1.3
% of this license or (at your option) any later version.
% The latest version of this license is in
%   http://www.latex-project.org/lppl.txt
% and version 1.3 or later is part of all distributions of LaTeX
% version 2005/12/01 or later.
%
% This work has the LPPL maintenance status `maintained'.
%
% The Current Maintainer of this work is Niklas Beisert.
%
% This work consists of the files childdoc.dtx and childdoc.ins
% and the derived files childdoc.def and cdocsamp.tex with
% cdocsch1.tex, cdocsch2.tex, cdocsdrf.tex, cdocsfn1.tex, cdocsfn2.tex.
%
%<package>\ifdefined\childdocmain\endinput\fi
%<package>\ProvidesFile{childdoc.def}[2018/12/30 v2.0 child document driver]
%<samplemain>\ProvidesFile{cdocsamp.tex}[2018/12/30 v2.0 sample for childdoc]
%<*driver>
%\ProvidesFile{childdoc.drv}[2018/12/30 v2.0 childdoc reference manual file]
\PassOptionsToClass{10pt,a4paper}{article}
\documentclass{ltxdoc}

\usepackage[margin=35mm]{geometry}
\usepackage{hyperref}
\usepackage{hyperxmp}
\usepackage[usenames]{color}

\hypersetup{colorlinks=true}
\hypersetup{pdfstartview=FitH}
\hypersetup{pdfpagemode=UseNone}
\hypersetup{pdfsource={}}
\hypersetup{pdflang={en-UK}}
\hypersetup{pdfcopyright={Copyright 2017-2018 Niklas Beisert.
  This work may be distributed and/or modified under the
  conditions of the LaTeX Project Public License, either version 1.3
  of this license or (at your option) any later version.}}
\hypersetup{pdflicenseurl={http://www.latex-project.org/lppl.txt}}
\hypersetup{pdfcontactaddress={ETH Zurich, ITP, HIT K,
  Wolfgang-Pauli-Strasse 27}}
\hypersetup{pdfcontactpostcode={8093}}
\hypersetup{pdfcontactcity={Zurich}}
\hypersetup{pdfcontactcountry={Switzerland}}
\hypersetup{pdfcontactemail={nbeisert@itp.phys.ethz.ch}}
\hypersetup{pdfcontacturl={http://people.phys.ethz.ch/\xmptilde nbeisert/}}

\newcommand{\secref}[1]{\hyperref[#1]{section \ref*{#1}}}

\parskip1ex
\parindent0pt
\let\olditemize\itemize
\def\itemize{\olditemize\parskip0pt}

\begin{document}

\title{The \textsf{childdoc} Package}
\hypersetup{pdftitle={The childdoc Package}}
\author{Niklas Beisert\\[2ex]
  Institut f\"ur Theoretische Physik\\
  Eidgen\"ossische Technische Hochschule Z\"urich\\
  Wolfgang-Pauli-Strasse 27, 8093 Z\"urich, Switzerland\\[1ex]
  \href{mailto:nbeisert@itp.phys.ethz.ch}
  {\texttt{nbeisert@itp.phys.ethz.ch}}}
\hypersetup{pdfauthor={Niklas Beisert}}
\hypersetup{pdfsubject={Manual for the LaTeX2e Package childdoc}}
\date{30 December 2018, \textsf{v2.0}}
\maketitle

\begin{abstract}\noindent
\textsf{childdoc} is a \LaTeXe{} package
that enables the direct compilation
of document sections included by |\include|
to individual files.
\end{abstract}

\begingroup
\parskip0ex
\tableofcontents
\endgroup

%%%%%%%%%%%%%%%%%%%%%%%%%%%%%%%%%%%%%%%%%%%%%%%%%%%%%%%%%%%%%%%%%%%%%%%%%%%%%%%%
%%%%%%%%%%%%%%%%%%%%%%%%%%%%%%%%%%%%%%%%%%%%%%%%%%%%%%%%%%%%%%%%%%%%%%%%%%%%%%%%
\section{Introduction}

\LaTeX{} provides a mechanism to structure a large document (such as a book)
into a main file and several child files (containing the chapters)
using the |\include| command.
This mechanism is beneficial for documents
which span hundreds of pages in order to
make the source file(s) more manageable.
Moreover, compilation can be restricted to
selected child files by means of the |\includeonly| command.
The latter feature can be used to reduce the compilation time while editing
(this was significantly more useful in the earlier days of \LaTeX{})
or to generate a smaller document which is easier to navigate.
Another application of |\includeonly| is to generate
documents consisting of selected parts of the complete document.

However, there are a few drawbacks of the plain |\include| mechanism:
\begin{itemize}
\item
The child files cannot be compiled on their own,
they can only be compiled via the main file.
A naive editing environment
(such as a text editor with an option
to have the current file processed by \LaTeX)
may require one to switch to the main file before compiling;
attempting to compile the child file produces errors.
\item
The main file must be modified (each time)
to adjust the |\includeonly| command
to the present needs. This easily leaves the main file in a messy state.
\item
The generated document will always carry the filename
of the main document. This is inconvenient if
several child files are to be compiled and
to be kept for distribution.
\end{itemize}

The present package provides a simple interface
to make child files individually compilable by \LaTeX{}.
Compiling a child file then has the same effect as compiling
the main file with an |\includeonly| command
to select the appropriate child.
Moreover the generated document will carry the name of the child
rather than the main file.
This resolves all three above issues.

This feature is meant to make the editing of books,
thesis documents and lecture notes somewhat more convenient.
However, the package can also be used efficiently for
composing a series of documents (such as exercise sheets)
which are typically distributed individually.
It then assists the author in generating the individual documents
(potentially in different versions)
as well as a document containing the collected series.
Another application is in developing style files
or other kinds of included material
where compilation of the style file could redirect
to a sample or test file.

%%%%%%%%%%%%%%%%%%%%%%%%%%%%%%%%%%%%%%%%%%%%%%%%%%%%%%%%%%%%%%%%%%%%%%%%%%%%%%%%
%%%%%%%%%%%%%%%%%%%%%%%%%%%%%%%%%%%%%%%%%%%%%%%%%%%%%%%%%%%%%%%%%%%%%%%%%%%%%%%%
\section{Usage}

First of all, the package \textsf{childdoc} is \emph{not} a standard
\LaTeXe{} |.sty| style file! Therefore it needs to be invoked in
a non-standard way.

%%%%%%%%%%%%%%%%%%%%%%%%%%%%%%%%%%%%%%%%%%%%%%%%%%%%%%%%%%%%%%%%%%%%%%%%%%%%%%%%
\subsection{Included Files}
\label{sec:include}

%%%%%%%%%%%%%%%%%%%%%%%%%%%%%%%%%%%%%%%%
\DescribeMacro{\childdocmain}
To use the package, add the commands
\begin{center}
\begin{tabular}{l}
|\input{childdoc.def}|\\
|\childdocmain{}|\\
\end{tabular}
\end{center}
at the very top of the main \LaTeX{} file,
in particular \emph{before} the |\documentclass| statement!
The argument of |\childdocmain| should be left empty
(but it must be present).

%%%%%%%%%%%%%%%%%%%%%%%%%%%%%%%%%%%%%%%%
\DescribeMacro{\childdocof}
Furthermore, add the commands
\begin{center}
\begin{tabular}{l}
|\input{childdoc.def}|\\
|\childdocof{|\textit{main}|}|\\
\end{tabular}
\end{center}
at the top of every child file \textit{child}
which is included by |\include{|\textit{child}|}|
from within the main file
(or at least for those files to be compiled individually).
The argument \textit{main} must be the filename of the main file.

There are a couple of
considerations in setting up the main and child documents:

%%%%%%%%%%%%%%%%%%%%%%%%%%%%%%%%%%%%%%%%
\paragraph{Restrictions.}

Please note the following restrictions:
\begin{itemize}
\item
|\childdocmain| must be called with one argument \textit{main}
to ensure compatibility with earlier version of the package.
It must either be empty (|\childdocmain{}|)
or precisely match the filename of the main file in which it is specified.
See \secref{sec:detection} for further information.
\item
The filename \textit{main} must be specified without the |.tex| extension.
\item
The filename \textit{main} is case sensitive
(even in case-insensitive file systems)
due to internal string comparison.
\item
The argument \textit{main} should be fully expanded, it cannot be a macro.
\item
Subdirectories and special characters should be avoided in filenames.
\item
The command |\childdocmain{|\textit{main}|}| must be followed by a whitespace.
It should not be followed immediately by another command
or by a comment mark `|%|'.
This is because the \TeX{} parser reads the token immediately following
the argument of |\childdocmain| and puts it
at the beginning of every child section;
however, a white\-space is ignored.
\end{itemize}

%%%%%%%%%%%%%%%%%%%%%%%%%%%%%%%%%%%%%%%%
\paragraph{Content of Main File.}

It is advisable to place all content in the child files included by |\include|.
Any output contained in the main file will appear in all child documents
unless suppressed manually;
it cannot be suppressed automatically by the |\includeonly| directive
and thus should normally be avoided.
A method to include some content in the main file
by means of conditional processing is described in \secref{sec:conditional}.

%%%%%%%%%%%%%%%%%%%%%%%%%%%%%%%%%%%%%%%%
\paragraph{Page Numbering.}

When only a part of the document is compiled,
the appropriate numbering of pages
(as well as other status parameters)
is determined from the |.aux| files.
The latter contain information from previous passes.
However this information needs to propagate through
all intermediate child documents.
Therefore the page numbering in child documents may well
be inconsistent until the complete document is compiled at least once.

A useful (if unconventional) way to always ensure a consistent
page numbering is to restart the numbering in each child document
and denote the pages by `\textit{child}|.|\textit{page}'
where \textit{child} represents the chapter/section number of the child file.
This can be achieved by the command
|\numberwithin{page}{|\textit{child}|}|
of the \textsf{amsmath} package
where \textit{child} can be |chapter| or |section|
depending on the chosen structuring.
Alternatively, one can modify the macro |\thepage| appropriately
and reset the counter |page| at the start of each child file.

%%%%%%%%%%%%%%%%%%%%%%%%%%%%%%%%%%%%%%%%%%%%%%%%%%%%%%%%%%%%%%%%%%%%%%%%%%%%%%%%
\subsection{Conditional Processing}
\label{sec:conditional}

The package provides a mechanism to compile different versions
of a document. To customise the versions further some conditional processing
can come in handy to distinguish which version is being compiled.
The package provides two macros to describe the compilation context:

%%%%%%%%%%%%%%%%%%%%%%%%%%%%%%%%%%%%%%%%
\DescribeMacro{\ifchilddoc}
The conditional |\ifchilddoc| distinguishes between the compilation of
child documents and the main document:
%
\begin{center}
|\ifchilddoc |\textit{child-code}| |[|\||else |\textit{main-code}]| \||fi|
\end{center}

%%%%%%%%%%%%%%%%%%%%%%%%%%%%%%%%%%%%%%%%
\DescribeMacro{\childdocname}
\DescribeMacro{\childdocjob}
The macro |\childdocname| contains the filename (without extension)
of the main or child file being processed.
Note that |\childdocjob| will always contain the name of the main file.

%%%%%%%%%%%%%%%%%%%%%%%%%%%%%%%%%%%%%%%%
\paragraph{Title Page.}

Conditional processing can be used to include a title or banner page
in the main document when proper precautions are taken.
Importantly, the code in the main file should ensure that the page counter
(as well as other status parameters which are stored in the |.aux| files)
takes the same value after the conditional processing.
Otherwise the page numbers may take divergent values
depending on which part is compiled.

For example, a title page could be declared by:
%
\begin{center}
\begin{tabular}{l}
|\ifchilddoc\||else|\\
|\addtocounter{page}{-1}|\\
\textit{code for title page}\\
|\newpage|\\
|\||fi|
\end{tabular}
\end{center}
%
A banner page for the child documents can be generated by:
%
\begin{center}
\begin{tabular}{l}
|\ifchilddoc|\\
|\addtocounter{page}{-1}|\\
\textit{code for banner page}\\
|\newpage|\\
|\||fi|
\end{tabular}
\end{center}
%
Here one could write a message such as:
\begin{center}
|This is the part \childdocname{} of \childdocjob{}.|
\end{center}

%%%%%%%%%%%%%%%%%%%%%%%%%%%%%%%%%%%%%%%%%%%%%%%%%%%%%%%%%%%%%%%%%%%%%%%%%%%%%%%%
\subsection{Flags}
\label{sec:flags}

The package makes it easy to generate different versions
of the main or child documents.
To this end compilation flags can be defined
and assigned different default values.
They will be particularly useful in conjunction
with the forwarding mechanism described in \secref{sec:forward}.

For example, it may be useful to have a flag |\version|
which can be set to |draft| or |final|.
The document source will contain some conditional code
depending on the value of |\version|.
Suppose further, the flag should default to |final| for the main file
and to |draft| for child files
which is a natural assignment for editing the document.
This is achieved by placing the following code
in the preamble of the main document
(below the |\childdocmain| directive):
%
\begin{center}
\begin{tabular}{l}
|\ifchilddoc|\\
|\providecommand{\version}{draft}|\\
|\||else|\\
|\providecommand{\version}{final}|\\
|\||fi|
\end{tabular}
\end{center}
%
The definition by |\providecommand| makes sure
that previous definitions are not overwritten.
Further statements |\providecommand{\version}{...}|
can thus be added before the above code to override it.

For the main file, one might add a line
(between |\childdocmain| and the above block)
%
\begin{center}
|%\ifchilddoc\||else\providecommand{\version}{draft}\||fi|
\end{center}
%
which can be uncommented to produce a draft version.
Likewise one can add a line to the very top of a child file
(above the |\childdocof{|\textit{main}|}| directive)
%
\begin{center}
|%\providecommand{\version}{final}|
\end{center}
%
which can be uncommented to produce the final version of this child document.

%%%%%%%%%%%%%%%%%%%%%%%%%%%%%%%%%%%%%%%%%%%%%%%%%%%%%%%%%%%%%%%%%%%%%%%%%%%%%%%%
\subsection{Forwarding}
\label{sec:forward}

Different versions of the main or child documents
using compilation flags as described in \secref{sec:flags}
can be (permanently) stored in different files
for convenient compilation, viewing and distribution.
To this end, the package defines a command
to pass on compilation to a different file:

%%%%%%%%%%%%%%%%%%%%%%%%%%%%%%%%%%%%%%%%
\DescribeMacro{\childdocforward}
The command |\childdocforward| redirects processing to
another source file:
%
\begin{center}
\begin{tabular}{l}
|\input{childdoc.def}|\\
|\childdocforward[|\textit{main}|]{|\textit{dest}|}|\\
\end{tabular}
\end{center}
%
The argument \textit{dest} is the destination file
(without extension).
It should be the main file or one of the child files.
Note that further \textsf{childdoc} directives
such as |\childdocof| and |\childdocforward|
in the indicated file will be processed in this form.
The optional argument \textit{main}
passes on directly to the main file \textit{main}
while pretending to compile the child \textit{dest}.
This form behaves as if \textit{dest}
issues |\childdocof{|\textit{main}|}| right away,
and no further \textsf{childdoc} directives will be processed.

%%%%%%%%%%%%%%%%%%%%%%%%%%%%%%%%%%%%%%%%
\DescribeMacro{\...prefix}
In the alternative form |\childdocforwardprefix|,
%
\begin{center}
\begin{tabular}{l}
|\input{childdoc.def}|\\
|\childdocforwardprefix[|\textit{main}|]{|\textit{prefix}|}{|\textit{dest}|}|
\end{tabular}
\end{center}
%
the destination file is determined by a pattern
depending on the current file:
To make this work, the current file must be called
`{\textit{prefix}\hspace{0.2em}\textit{suffix}}'
with \textit{prefix} matching precisely the argument.
Processing is then passed on to the file
`{\textit{dest}\hspace{0.2em}\textit{suffix}}'.
Surely, the same effect is achieved by
directly specifying the
argument `{\textit{dest}\hspace{0.2em}\textit{suffix}}'
in the first form.
However, that requires to set up a different file
for each child. With the alternative form of the command
all these files can have exactly the same content
which simplifies setting them up and maintaining them.

For example, the following file |draft.tex|
with a compilation flag |\version| as described in \secref{sec:flags}
compiles the main document as a draft:
%
\begin{center}
\begin{tabular}{l}
|\def\version{draft}|\\
|\input{childdoc.def}|\\
|\childdocforward{|\textit{main}|}|
\end{tabular}
\end{center}
%
Likewise, the following files |final|\textit{nn}|.tex|
compile the final version of the child document
|child|\textit{nn}|.tex|:
%
\begin{center}
\begin{tabular}{l}
|\def\version{final}|\\
|\input{childdoc.def}|\\
|\childdocforwardprefix{final}{child}|
\end{tabular}
\end{center}
%

Note that when several versions of a main file and/or of each child file
are to be generated, it may be convenient to set up a |Makefile| or
shell script to automatise the process.

%%%%%%%%%%%%%%%%%%%%%%%%%%%%%%%%%%%%%%%%%%%%%%%%%%%%%%%%%%%%%%%%%%%%%%%%%%%%%%%%
\subsection{Command Line Processing}
\label{sec:commandline}

The effect of redirection files can also be achieved by invoking
the \LaTeX{} compiler with a more elaborate command line.
Most conveniently this should be done as part
of a shell script or a |Makefile|.

When using \textsf{childdoc} in the main file, the following
command lines effectively perform a redirection
(note that depending on the shell being used,
backslashes may have to be doubled: `|\|' $\to$ `|\\|'):
%
\begin{center}
|... -jobname "|\textit{target}|" |\\|"|[\textit{flags}]%
|\input{childdoc.def}\childdocforward[|\textit{main}|]{|\textit{dest}|}"|
\end{center}
%
Here \textit{target} is the name of the output file,
\textit{main} is the name of the main file
and \textit{dest} is the name of the main or child file to be processed
(all filenames without extensions).
The optional argument \textit{main} can be omitted
if \textit{main} matches \textit{dest}.
Optionally, compilation \textit{flags} can be defined via |\def| commands.
This command line makes the \TeX{} engine believe
it is compiling the file \textit{target}
whose content is specified as the latter parameter.
The provided code then forwards the processing to
\textit{main} or \textit{dest} as described in \secref{sec:forward}.

%%%%%%%%%%%%%%%%%%%%%%%%%%%%%%%%%%%%%%%%%%%%%%%%%%%%%%%%%%%%%%%%%%%%%%%%%%%%%%%%
\subsection{Include by Input}
\label{sec:input}

Including child documents by |\include| has some restrictions by design.
Most notably, the content of a child document always occupies
its own set of pages; pages cannot be shared between child documents.
Usually, this behaviour makes perfect sense
because each child document contain an essential part of the document.
However, in some situations it may be desirable to compose
a document from a collection of parts
without having mandatory page breaks between then.
For this case, the package
provides a mechanism to include parts
by |\input| which can also be processed individually.
However, by construction this mechanism
requires manual handling of the content to be output.

%%%%%%%%%%%%%%%%%%%%%%%%%%%%%%%%%%%%%%%%
\DescribeMacro{\ifchilddocmanual}
The main file should be prepared as usual, see \secref{sec:include}.
However, the document body must make a distinction
between processing of an individual part and of the main document, e.g.:
%
\begin{center}
\begin{tabular}{l}
|\ifchilddocmanual|\\
|\input{\childdocname}|\\
|\||else|\\
\textit{document body with }|\input{|\textit{part}|}|\\
|\||fi|
\end{tabular}
\end{center}
%
The conditional |\ifchilddocmanual| is true whenever
a part to be included by |\input| is being compiled,
and the name of the part is stored in |\childdocname|.

%%%%%%%%%%%%%%%%%%%%%%%%%%%%%%%%%%%%%%%%
\DescribeMacro{\childdocby}
Each part to be included by |\input| should start with:
%
\begin{center}
\begin{tabular}{l}
|\input{childdoc.def}|\\
|\childdocby{|\textit{main}|}|\\
\end{tabular}
\end{center}
%
The directive |\childdocby| is similar to |\childdocof|
described in \secref{sec:include},
but the subsequent selection of content must be done manually.
To that end, both |\ifchilddoc| and |\ifchilddocmanual|
will be true upon processing of a part,
and the name of the part is stored in |\childdocname|.
Note that |\jobname| will be set to the filename of the current part
so that each part receives an individual |.aux| file
that does not interfere with the |.aux| file(s) of the main document.
This behaviour can be altered by the alternative form
|\childdocby[*]{|\textit{main}|}| (with a non-empty optional argument)
which uses the |.aux| file of the main document
by setting |\jobname| to \textit{main}.

%%%%%%%%%%%%%%%%%%%%%%%%%%%%%%%%%%%%%%%%%%%%%%%%%%%%%%%%%%%%%%%%%%%%%%%%%%%%%%%%
\subsection{Driver Development}
\label{sec:driver}

The \textsf{childdoc} mechanism can also be use for the development
of definition files such as \LaTeX{} styles or classes.
This case differs from the above setup with multiple parts
included by |\include| in that no |\includeonly| should be invoked.
This can be achieved by starting the include file
(before |\ProvidesPackage|) with:
%
\begin{center}
\begin{tabular}{l}
|\input{childdoc.def}|\\
|\childdocforward{|\textit{main}|}|\\
\end{tabular}
\end{center}
%
or alternatively with:
%
\begin{center}
\begin{tabular}{l}
|\input{childdoc.def}|\\
|\childdocby{|\textit{main}|}|\\
\end{tabular}
\end{center}
%
Both forms have slightly different effects as described above.
The main file is prepared as usual, see \secref{sec:include}.

%%%%%%%%%%%%%%%%%%%%%%%%%%%%%%%%%%%%%%%%%%%%%%%%%%%%%%%%%%%%%%%%%%%%%%%%%%%%%%%%
\subsection{Legacy Detection}
\label{sec:detection}

The directive |\childdocmain| in the main file can detect
whether the complete document or merely a child is to be compiled
even without using the directive |\childdocof|.
This method is deprecated because it is less robust
and there is no compelling reason to use it;
it is merely provided for backward compatibility
and it may be removed in future versions.

If the detection mechanism is to be used,
it is mandatory to correctly specify
the filename of the main file as the argument of |\childdocmain|:
%
\begin{center}
\begin{tabular}{l}
|\input{childdoc.def}|\\
|\childdocmain{|\textit{main}|}|\\
\end{tabular}
\end{center}
%
If |\jobname| does not match the argument \textit{main} of |\childdocmain|,
it is assumed that |\jobname| points to the child file to be compiled.
When using |\childdocmain| with the main file specified as argument,
it suffices to start a child file
with just |\input{|\textit{main}|}|
without loading of the package and using |\childdocof|.
If instead all processing is done
with the appropriate \textsf{childdoc} directives,
the argument of \textit{main} of |\childdocmain| can be empty.

An alternative version of the command line processing described
in \secref{sec:commandline} using the detection mechanism reads:
%
\begin{center}
|... -jobname "|\textit{target}|" "|[\textit{flags}]%
[|\def\jobname{|\textit{dest}|}|]|\input{|\textit{main}|}"|
\end{center}

%%%%%%%%%%%%%%%%%%%%%%%%%%%%%%%%%%%%%%%%%%%%%%%%%%%%%%%%%%%%%%%%%%%%%%%%%%%%%%%%
\subsection{Manual Code}
\label{sec:manual}

In case one cannot be certain whether the definitions file |childdoc.def|
is installed on the target \TeX{} distribution
and one prefers not to ship it,
it is conceivable to paste a few relevant commands into the sources.

To that end, drop all statements |\input{childdoc.def}|
and perform the replacements as outlined below.
Instead of |\childdocmain{|\textit{main}|}| add the following code
to the top of the main file:
%
\begin{center}
\begin{tabular}{l}
|\||ifdefined\childdocname\endinput\||fi\newif\ifchilddoc|\\
|\edef\childdocname{\scantokens\expandafter{\jobname\noexpand}}|\\
|\def\childdocmain{|\textit{main}|}\||ifx\childdocmain\childdocname\||else|\\
|\childdoctrue\includeonly{\childdocname}\let\jobname\childdocmain\||fi|\\
\end{tabular}
\end{center}
%
Instead of |\childdocof{|\textit{main}|}| just include the main file
at the top of each child file:
%
\begin{center}
|\input{|\textit{main}|}|
\end{center}
%
A simple redirection |\childdocforward{|\textit{dest}|}| is achieved by:
%
\begin{center}
|\def\jobname{|\textit{dest}|}\input{\jobname}|
\end{center}
%
The redirection with prefix
|\childdocforwardprefix[|\textit{prefix}|]{|\textit{dest}|}|
is accomplished by:
%
\begin{center}
\begin{tabular}{l}
|{\edef\jobname{\scantokens\expandafter{\jobname\noexpand}}|\\
|\def\redirectjob |\textit{prefix}|#1~~~{\gdef\jobname{|\textit{dest}|#1}}|\\
|\expandafter\redirectjob\jobname~~~}\input{\jobname}|
\end{tabular}
\end{center}

In an alternative approach,
child documents can be compiled by a specific command line
without additional code or specific definitions:
%
\begin{center}
|... -jobname "|\textit{target}|" "|[\textit{flags}]%
|\includeonly{|\textit{dest}|}\input{|\textit{main}|}"|
\end{center}
%

%%%%%%%%%%%%%%%%%%%%%%%%%%%%%%%%%%%%%%%%%%%%%%%%%%%%%%%%%%%%%%%%%%%%%%%%%%%%%%%%
%%%%%%%%%%%%%%%%%%%%%%%%%%%%%%%%%%%%%%%%%%%%%%%%%%%%%%%%%%%%%%%%%%%%%%%%%%%%%%%%
\section{Information}

%%%%%%%%%%%%%%%%%%%%%%%%%%%%%%%%%%%%%%%%%%%%%%%%%%%%%%%%%%%%%%%%%%%%%%%%%%%%%%%%
\subsection{Copyright}

Copyright \copyright{} 2017--2018 Niklas Beisert

This work may be distributed and/or modified under the
conditions of the \LaTeX{} Project Public License, either version 1.3
of this license or (at your option) any later version.
The latest version of this license is in
  \url{http://www.latex-project.org/lppl.txt}
and version 1.3 or later is part of all distributions of \LaTeX{}
version 2005/12/01 or later.

This work has the LPPL maintenance status `maintained'.

The Current Maintainer of this work is Niklas Beisert.

This work consists of the files |README.txt|, |childdoc.ins| and |childdoc.dtx|
as well as the derived files |childdoc.def|, |cdocsamp.tex|
with |cdocsch1.tex|, |cdocsch2.tex|, |cdocspt3.tex|, |cdocspt4.tex|,
|cdocsdrf.tex|, |cdocsfn1.tex|, |cdocsfn2.tex|
as well as |childdoc.pdf|.

%%%%%%%%%%%%%%%%%%%%%%%%%%%%%%%%%%%%%%%%%%%%%%%%%%%%%%%%%%%%%%%%%%%%%%%%%%%%%%%%
\subsection{Files and Installation}

The package consists of the files:
%
\begin{center}
\begin{tabular}{ll}
    |README.txt|   & readme file \\
    |childdoc.ins| & installation file \\
    |childdoc.dtx| & source file \\
    |childdoc.def| & definition file \\
    |cdocsamp.tex| & sample main file \\
    |cdocsch1.tex| & sample include file \\
    |cdocsch2.tex| & sample include file \\
    |cdocspt3.tex| & sample part file \\
    |cdocspt4.tex| & sample part file \\
    |cdocsdrf.tex| & sample redirection file \\
    |cdocsfn1.tex| & sample redirection file \\
    |cdocsfn2.tex| & sample redirection file \\
    |childdoc.pdf| & manual
\end{tabular}
\end{center}
%
The distribution consists of the files
|README.txt|, |childdoc.ins| and |childdoc.dtx|.
%
\begin{itemize}
\item
Run (pdf)\LaTeX{} on |childdoc.dtx|
to compile the manual |childdoc.pdf| (this file).
\item
Run \LaTeX{} on |childdoc.ins| to create the definitions file |childdoc.def|
and the sample |cdocsamp.tex| with include files
|cdocsch1.tex|, |cdocsch2.tex|, |cdocspt3.tex|, |cdocspt4.tex|,
|cdocsdrf.tex|, |cdocsfn1.tex|, |cdocsfn2.tex|.
Then copy the file |childdoc.def| to an appropriate directory of your \LaTeX{}
distribution, e.g.\ \textit{texmf-root}|/tex/latex/childdoc|.
\end{itemize}

%%%%%%%%%%%%%%%%%%%%%%%%%%%%%%%%%%%%%%%%%%%%%%%%%%%%%%%%%%%%%%%%%%%%%%%%%%%%%%%%
\subsection{Related CTAN Packages}

There are several other packages which offer a similar functionality:
%
\begin{itemize}
\item
The packages
\href{http://ctan.org/pkg/docmute}{\textsf{docmute}},
\href{http://ctan.org/pkg/includex}{\textsf{includex}} and
\href{http://ctan.org/pkg/standalone}{\textsf{standalone}}
provide commands to include only the document body of
a child file thus allowing both files to be compiled individually.
\item
The packages \href{http://ctan.org/pkg/subdocs}{\textsf{subdocs}}
and \href{http://ctan.org/pkg/subfiles}{\textsf{subfiles}}
provide structures in which the main and child documents can be
encapsulated and allowing them to be compiled individually.
The inclusion mechanism is different from the conventional |\include|.
\item
The package \href{http://ctan.org/pkg/combine}{\textsf{combine}}
is an elaborate solution to combine several documents into one.
\end{itemize}
%
See also the CTAN topic \href{http://ctan.org/topic/subdocs}{\textsf{subdocs}}
for further related packages.
The present package differs from the above solutions in that
a document structure constructed with the conventional |\include| mechanism
just needs two extra commands at the top of every file
such that all constituent files can be compiled individually.

%%%%%%%%%%%%%%%%%%%%%%%%%%%%%%%%%%%%%%%%%%%%%%%%%%%%%%%%%%%%%%%%%%%%%%%%%%%%%%%%
%\subsection{Feature Suggestions}
%
%The following is a list of features which may be useful for future
%versions of this package:
%%
%\begin{itemize}
%\item
%\ldots
%\end{itemize}

%%%%%%%%%%%%%%%%%%%%%%%%%%%%%%%%%%%%%%%%%%%%%%%%%%%%%%%%%%%%%%%%%%%%%%%%%%%%%%%%
\subsection{Revision History}

%%%%%%%%%%%%%%%%%%%%%%%%%%%%%%%%%%%%%%%%
\paragraph{v2.0:} 2018/12/30

\begin{itemize}
\item
immediate forward processing
\item
added |\childdocby| mechanism
\item
manual restructured
\end{itemize}

%%%%%%%%%%%%%%%%%%%%%%%%%%%%%%%%%%%%%%%%
\paragraph{v1.6:} 2018/01/17

\begin{itemize}
\item
application for development of include files
\item
corrections to manual
\end{itemize}

%%%%%%%%%%%%%%%%%%%%%%%%%%%%%%%%%%%%%%%%
\paragraph{v1.5:} 2017/05/21

\begin{itemize}
\item
more complete structuring introduced
\item
|\childdocof| introduced
\item
|\childdoc| renamed to |\childdocmain|
\item
|\childredirect| renamed to |\childdocforward| and |\childdocforwardprefix|
and functionality expanded
\end{itemize}

%%%%%%%%%%%%%%%%%%%%%%%%%%%%%%%%%%%%%%%%
\paragraph{v1.0:} 2017/04/27

\begin{itemize}
\item
manual and install package
\item
first version published on CTAN
\end{itemize}

%%%%%%%%%%%%%%%%%%%%%%%%%%%%%%%%%%%%%%%%
\paragraph{v0.6:} 2017/04/26

\begin{itemize}
\item
redirection mechanism added
\end{itemize}

%%%%%%%%%%%%%%%%%%%%%%%%%%%%%%%%%%%%%%%%
\paragraph{v0.5:} 2017/04/26

\begin{itemize}
\item
functionality in definition file
\end{itemize}


%%%%%%%%%%%%%%%%%%%%%%%%%%%%%%%%%%%%%%%%%%%%%%%%%%%%%%%%%%%%%%%%%%%%%%%%%%%%%%%%
%%%%%%%%%%%%%%%%%%%%%%%%%%%%%%%%%%%%%%%%%%%%%%%%%%%%%%%%%%%%%%%%%%%%%%%%%%%%%%%%
%%%%%%%%%%%%%%%%%%%%%%%%%%%%%%%%%%%%%%%%%%%%%%%%%%%%%%%%%%%%%%%%%%%%%%%%%%%%%%%%
\appendix

\settowidth\MacroIndent{\rmfamily\scriptsize 000\ }

 \DocInput{childdoc.dtx}

\end{document}
%</driver>
% \fi
%
% %%%%%%%%%%%%%%%%%%%%%%%%%%%%%%%%%%%%%%%%%%%%%%%%%%%%%%%%%%%%%%%%%%%%%%%%%%%%%%
% %%%%%%%%%%%%%%%%%%%%%%%%%%%%%%%%%%%%%%%%%%%%%%%%%%%%%%%%%%%%%%%%%%%%%%%%%%%%%%
% \section{Sample}
%\iffalse
%<*samplemain>
%\fi
%
% The following presents a sample document
% with two chapters, two parts, a title page,
% a compile flag as well as three forwarding files to set the flag.
% It consists of eight |.tex| files:
% \begin{center}
% \begin{tabular}{ll}
% |cdocsamp.tex|&main file\\
% |cdocsch1.tex|&include file for chapter 1\\
% |cdocsch2.tex|&include file for chapter 2\\
% |cdocspt3.tex|&include file for part 3\\
% |cdocspt4.tex|&include file for part 4\\
% |cdocsdrf.tex|&forwarding file for main file in draft mode\\
% |cdocsfi1.tex|&forwarding file for final version of chapter 1\\
% |cdocsfi2.tex|&forwarding file for final version of chapter 2\\
% \end{tabular}
% \end{center}
% Each of the eight files can be compiled directly by the \LaTeX{} compiler.
%
% %%%%%%%%%%%%%%%%%%%%%%%%%%%%%%%%%%%%%%
% \paragraph{Main File.}
%
% The main file is called |cdocsamp.tex|.
%
% Load the \textsf{childdoc} definitions and
% declare the filename for the main document:
%    \begin{macrocode}
\input{childdoc.def}
\childdocmain{}
%    \end{macrocode}

% Optional override for |\version| flag:
%    \begin{macrocode}
%%\ifchilddoc\else\providecommand{\version}{draft}\fi
%    \end{macrocode}

% Define the default values for the |\version| flag
% (|final| for the main file and |draft| for childs):
%    \begin{macrocode}
\ifchilddoc
\providecommand{\version}{draft}
\else
\providecommand{\version}{final}
\fi
%    \end{macrocode}

% Load the standard document class:
%    \begin{macrocode}
\documentclass[12pt]{article}
%    \end{macrocode}

% Start the document body:
%    \begin{macrocode}
\begin{document}
%    \end{macrocode}

% Declare a title page.
% Print title, part of document being processed and version flag:
%    \begin{macrocode}
\addtocounter{page}{-1}
\begin{center}
{\LARGE\bfseries{}childdoc example\par}
\vspace{1cm}
\ifchilddoc
\ifchilddocmanual part\else chapter\fi:
`\childdocname' of `\childdocjob'\par
\else
main document: `\childdocjob'\par
\fi
version: \version\par
\end{center}
\newpage
%    \end{macrocode}

% Manually include selected file,
% otherwise process as usual:
%    \begin{macrocode}
\ifchilddocmanual
\section*{part `\childdocname'}
\input{\childdocname}
\else
%    \end{macrocode}

% Include the two chapters:
%    \begin{macrocode}
\include{cdocsch1}
\include{cdocsch2}
%    \end{macrocode}

% Include the two parts unless only chapters should be displayed:
%    \begin{macrocode}
\ifchilddoc\else
\section{part three}
\input{cdocspt3}
\section{part four}
\input{cdocspt4}
\fi
%    \end{macrocode}

% Process as usual until here:
%    \begin{macrocode}
\fi
%    \end{macrocode}

% End of document body:
%    \begin{macrocode}
\end{document}
%    \end{macrocode}
%\iffalse
%</samplemain>
%\fi
%
% %%%%%%%%%%%%%%%%%%%%%%%%%%%%%%%%%%%%%%
% \paragraph{Chapter Include Files.}
%
% The include files are called |cdocsch1.tex| and |cdocsch2.tex|.
%
%\iffalse
%<*samplechap1|samplechap2>
%\fi

% Optional override for |\version| flag:
%    \begin{macrocode}
%%\providecommand{\version}{final}
%    \end{macrocode}

% Include the main document:
%    \begin{macrocode}
\input{childdoc.def}
\childdocof{cdocsamp}
%    \end{macrocode}

%\iffalse
%</samplechap1|samplechap2>
%\fi
%
%\iffalse
%<*samplechap1>
%\fi
% Some text for chapter 1:
%    \begin{macrocode}
\section{one}
some text in chapter one
%    \end{macrocode}

%\iffalse
%</samplechap1>
%\fi
% Some text for chapter 2:
%\iffalse
%<*samplechap2>
%\fi
%    \begin{macrocode}
\section{two}
more text in chapter two
%    \end{macrocode}

%\iffalse
%</samplechap2>
%\fi
%
% %%%%%%%%%%%%%%%%%%%%%%%%%%%%%%%%%%%%%%
% \paragraph{Part Include Files.}
%
% The include files are called |cdocspt3.tex| and |cdocspt4.tex|.
%
%\iffalse
%<*samplepart3|samplepart4>
%\fi

% Optional override for |\version| flag:
%    \begin{macrocode}
%%\providecommand{\version}{final}
%    \end{macrocode}

% Include the main document:
%    \begin{macrocode}
\input{childdoc.def}
\childdocby{cdocsamp}
%    \end{macrocode}

%\iffalse
%</samplepart3|samplepart4>
%\fi
%
%\iffalse
%<*samplepart3>
%\fi
% Some text for part 3:
%    \begin{macrocode}
some text in part three
%    \end{macrocode}

%\iffalse
%</samplepart3>
%\fi
% Some text for part 4:
%\iffalse
%<*samplepart4>
%\fi
%    \begin{macrocode}
more text in part four
%    \end{macrocode}

%\iffalse
%</samplepart4>
%\fi
%
% %%%%%%%%%%%%%%%%%%%%%%%%%%%%%%%%%%%%%%
% \paragraph{Forwarding for a Complete Draft.}
%
% The following forwarding file |cdocsdrf.tex|
% compiles the main document in draft mode:
%\iffalse
%<*sampledraft>
%\fi
%    \begin{macrocode}
\def\version{draft}
\input{childdoc.def}
\childdocforward{cdocsamp}
%    \end{macrocode}

%\iffalse
%</sampledraft>
%\fi
%
% %%%%%%%%%%%%%%%%%%%%%%%%%%%%%%%%%%%%%%
% \paragraph{Forwarding for Final Version of the Chapters.}
%
% The following forwarding files |cdocsfn1.tex| and |cdocsfn2.tex|
% (with identical content)
% compile the final versions of the child documents
% |cdocsch1.tex| and |cdocsch2.tex|, respectively:
%\iffalse
%<*samplefinal>
%\fi
%    \begin{macrocode}
\def\version{final}
\input{childdoc.def}
\childdocforwardprefix[cdocsamp]{cdocsfn}{cdocsch}
%    \end{macrocode}

%\iffalse
%</samplefinal>
%\fi
%
% %%%%%%%%%%%%%%%%%%%%%%%%%%%%%%%%%%%%%%
% \paragraph{Command Line Processing.}
%
% The following three command lines generate the output files
% |cdocscld|, |cdocscl1| and |cdocscl2|
% which should be identical to
% |cdocsdrf|, |cdocsch1| and |cdocsfn2|, respectively:
% \begin{center}
% \begin{tabular}{l}
% |latex -jobname cdocscld \|\\
% |  "\def\version{draft}\input{childdoc.def}\childdocforward{cdocsamp}"|\\
% |latex -jobname cdocscl1 \|\\
% |  "\input{childdoc.def}\childdocforward[cdocsamp]{cdocsch1}"|\\
% |latex -jobname cdocscl2 \|\\
% |  "\def\version{final}\input{childdoc.def}\childdocforward{cdocsch2}"|
% \end{tabular}
% \end{center}
% Note that the trailing backslash on each first line
% merely continues the input to the second line
% (for convenient cut ant paste).
% Furthermore, the command |latex| can be replaced by any
% of its alternative versions such as |pdflatex|.
%
% %%%%%%%%%%%%%%%%%%%%%%%%%%%%%%%%%%%%%%%%%%%%%%%%%%%%%%%%%%%%%%%%%%%%%%%%%%%%%%
% %%%%%%%%%%%%%%%%%%%%%%%%%%%%%%%%%%%%%%%%%%%%%%%%%%%%%%%%%%%%%%%%%%%%%%%%%%%%%%
% \section{Implementation}
%\iffalse
%<*package>
%\fi
%
% This section describes the definitions file |childdoc.def|.

% The definitions cannot be loaded using |\usepackage| or |\RequirePackage|
% which has a mechanism to prevent loading a style file more than once.
% When loading the definitions by means of |\input|
% multiple instances have to be prevented manually:
%\iffalse
%This code needs to be before the `\ProvidesFile' directive
%which is defined at the beginning of this file.
%Therefore it is also placed there and commented out here.
%</package>
%<*discard>
%\fi
%    \begin{macrocode}
\ifdefined\childdocmain\endinput\fi
%    \end{macrocode}
%\iffalse
%</discard>
%<*package>
%\fi
%
% \macro{\ifchilddoc}
% \macro{\ifchilddocmanual}
% The conditional |\ifchilddoc| tells whether a
% child (true) or main (false) document is being compiled.
% The conditional |\ifchilddocmanual| tells whether
% the |\includeonly| mechanism is used (false) or
% the selection of child files must be performed manually (true).
% The definitions initialise to false:
%    \begin{macrocode}
\newif\ifchilddoc
\newif\ifchilddocmanual
%    \end{macrocode}

% \macro{\childdocname}
% \macro{\childdocjob}
% The macro |\childdocname| stores the name of the main document
% to be compiled. The macro |\childdocjob| stores the name of
% the document on which the \LaTeX{} compiler was originally invoked.
% The content of |\jobname| cannot be compared
% to filenames specified in the source due to different catcodes.
% The following code rescans |\jobname|, stores the result
% in |\childdocname| and saves a copy in |\childdocjob|:
%    \begin{macrocode}
\edef\childdocname{\scantokens\expandafter{\jobname\noexpand}}
\let\childdocjob\childdocname
%    \end{macrocode}

% \macro{\childdocdisable}
% The macro |\childdocdisable| prevents the main file
% from being processed more than once.
% At this stage, the main document command |\childdocmain|
% is assumed to be called once again where it should do nothing.
% Any subsequent call to it should prevent
% a secondary processing of the main document
% It overwrites the forwarding commands
% |\childdocof| and |\childdocforward|
% with empty macros to prevent further inclusions of the main document:
%    \begin{macrocode}
\newcommand{\childdocdisable}
{
  \renewcommand{\childdocmain}[1]{\renewcommand{\childdocmain}[1]{\endinput}}
  \renewcommand{\childdocof}[1]{}
  \renewcommand{\childdocby}[2][]{}
  \renewcommand{\childdocforward}[2][]{}
  \renewcommand{\childdocdisable}{}
}
%    \end{macrocode}

% \macro{\childdocmain}
% The macro |\childdocmain| is to be called at the top of the main file
% with nothing or the main filename (without extension) as argument.
% First, it breaks loops.
% If the argument is not empty and does not match |\childdocname|
% (which is set by the first inclusion of |childdoc.def|),
% |\ifchilddoc| is set to true, |\includeonly| is applied to the child file
% and |\jobname| is set to the main file
% (for proper handling of |.aux| files):
%    \begin{macrocode}
\newcommand{\childdocmain}[1]
{
  \childdocdisable\childdocmain{}
  \if?#1?\else
    \begingroup
      \def\childdoctmp{#1}
      \ifx\childdoctmp\childdocname
        \def\childdoctmp{}
      \else
        \def\childdoctmp
        {
          \childdoctrue
          \includeonly{\childdocname}
          \def\childdocjob{#1}
          \def\jobname{#1}
        }
      \fi
      \expandafter
    \endgroup
    \childdoctmp
  \fi
}
%    \end{macrocode}

% \macro{\childdocof}
% The command |\childdocof| redirects
% compilation to the main file |#1|.
%    \begin{macrocode}
\newcommand{\childdocof}[1]
{
  \childdocdisable
  \childdoctrue
  \includeonly{\childdocname}
  \def\jobname{#1}
  \def\childdocjob{#1}
  \input{#1}
}
%    \end{macrocode}

% \macro{\childdocby}
% The command |\childdocby| ....
%    \begin{macrocode}
\newcommand{\childdocby}[2][]
{
  \childdocdisable
  \childdoctrue
  \childdocmanualtrue
  \if?#1?\else
    \def\jobname{#2}
  \fi
  \def\childdocjob{#2}
  \input{#2}
  \endinput
}
%    \end{macrocode}

% \macro{\childdocforward}
% The command |\childdocforward| redirects
% compilation to the main file or
% (if the optional argument is given) a child file.
% Parameters are set as if the main file
% or a child file starting with |\childdocof| was compiled.
% Then compilation is handed over to the main file:
%    \begin{macrocode}
\newcommand{\childdocforward}[2][]
{
  \begingroup
    \if?#1?
      \def\childdoctmp
      {
        \def\childdocname{#2}
        \def\childdocjob{#2}
        \def\jobname{#2}
        \input{#2}
        \endinput
      }
    \else
      \def\childdoctmp
      {
        \childdocdisable
        \def\childdocname{#2}
        \childdoctrue
        \includeonly{#2}
        \def\childdocjob{#1}
        \def\jobname{#1}
        \input{#1}
        \endinput
      }
    \fi
    \expandafter
  \endgroup
  \childdoctmp
}
%    \end{macrocode}

% \macro{\childdocforwardprefix}
% The command |\childdocforwardprefix| redirects
% compilation to the main or a child file by means of a pattern.
% The prefix |#1| in the current filename is replaced by |#2|
% and the suffix of the current filename is kept
% (it is assumed that the filename does not contain the substring `|~~~|'
% which is used as a delimiter).
% Compilation is handed over to the new file by |\childdocforward|:
%    \begin{macrocode}
\newcommand{\childdocforwardprefix}[3][]
{
  \begingroup
    \def\childdocextract #2##1~~~{\def\childdoctmp{\childdocforward[#1]{#3##1}}}
    \expandafter\childdocextract\childdocname~~~
    \expandafter
  \endgroup
  \childdoctmp
}
%    \end{macrocode}

% \macro{\childdoc}
% The deprecated macro |\childdoc| is a legacy version of |\childdocmain|:
%    \begin{macrocode}
\newcommand{\childdoc}{\childdocmain}
%    \end{macrocode}

% \macro{\childdocredirect}
% The deprecated macro |\childdocredirect| is a legacy version
% of |\childdocforward| and |\childdocforwardprefix|:
%    \begin{macrocode}
\newcommand{\childdocredirect}[2][]
{
  \begingroup
    \if?#1?
      \def\childdoctmp{\childdocforward{#2}}
    \else
      \def\childdoctmp{\childdocforwardprefix{#1}{#2}}
    \fi
    \expandafter
  \endgroup
  \childdoctmp
}
%    \end{macrocode}

%\iffalse
%</package>
%\fi
%
\endinput
\childdocforward[cdocsamp]{cdocsch1}"|\\
% |latex -jobname cdocscl2 \|\\
% |  "\def\version{final}% \iffalse
%
% childdoc.dtx Copyright (C) 2017-2018 Niklas Beisert
%
% This work may be distributed and/or modified under the
% conditions of the LaTeX Project Public License, either version 1.3
% of this license or (at your option) any later version.
% The latest version of this license is in
%   http://www.latex-project.org/lppl.txt
% and version 1.3 or later is part of all distributions of LaTeX
% version 2005/12/01 or later.
%
% This work has the LPPL maintenance status `maintained'.
%
% The Current Maintainer of this work is Niklas Beisert.
%
% This work consists of the files childdoc.dtx and childdoc.ins
% and the derived files childdoc.def and cdocsamp.tex with
% cdocsch1.tex, cdocsch2.tex, cdocsdrf.tex, cdocsfn1.tex, cdocsfn2.tex.
%
%<package>\ifdefined\childdocmain\endinput\fi
%<package>\ProvidesFile{childdoc.def}[2018/12/30 v2.0 child document driver]
%<samplemain>\ProvidesFile{cdocsamp.tex}[2018/12/30 v2.0 sample for childdoc]
%<*driver>
%\ProvidesFile{childdoc.drv}[2018/12/30 v2.0 childdoc reference manual file]
\PassOptionsToClass{10pt,a4paper}{article}
\documentclass{ltxdoc}

\usepackage[margin=35mm]{geometry}
\usepackage{hyperref}
\usepackage{hyperxmp}
\usepackage[usenames]{color}

\hypersetup{colorlinks=true}
\hypersetup{pdfstartview=FitH}
\hypersetup{pdfpagemode=UseNone}
\hypersetup{pdfsource={}}
\hypersetup{pdflang={en-UK}}
\hypersetup{pdfcopyright={Copyright 2017-2018 Niklas Beisert.
  This work may be distributed and/or modified under the
  conditions of the LaTeX Project Public License, either version 1.3
  of this license or (at your option) any later version.}}
\hypersetup{pdflicenseurl={http://www.latex-project.org/lppl.txt}}
\hypersetup{pdfcontactaddress={ETH Zurich, ITP, HIT K,
  Wolfgang-Pauli-Strasse 27}}
\hypersetup{pdfcontactpostcode={8093}}
\hypersetup{pdfcontactcity={Zurich}}
\hypersetup{pdfcontactcountry={Switzerland}}
\hypersetup{pdfcontactemail={nbeisert@itp.phys.ethz.ch}}
\hypersetup{pdfcontacturl={http://people.phys.ethz.ch/\xmptilde nbeisert/}}

\newcommand{\secref}[1]{\hyperref[#1]{section \ref*{#1}}}

\parskip1ex
\parindent0pt
\let\olditemize\itemize
\def\itemize{\olditemize\parskip0pt}

\begin{document}

\title{The \textsf{childdoc} Package}
\hypersetup{pdftitle={The childdoc Package}}
\author{Niklas Beisert\\[2ex]
  Institut f\"ur Theoretische Physik\\
  Eidgen\"ossische Technische Hochschule Z\"urich\\
  Wolfgang-Pauli-Strasse 27, 8093 Z\"urich, Switzerland\\[1ex]
  \href{mailto:nbeisert@itp.phys.ethz.ch}
  {\texttt{nbeisert@itp.phys.ethz.ch}}}
\hypersetup{pdfauthor={Niklas Beisert}}
\hypersetup{pdfsubject={Manual for the LaTeX2e Package childdoc}}
\date{30 December 2018, \textsf{v2.0}}
\maketitle

\begin{abstract}\noindent
\textsf{childdoc} is a \LaTeXe{} package
that enables the direct compilation
of document sections included by |\include|
to individual files.
\end{abstract}

\begingroup
\parskip0ex
\tableofcontents
\endgroup

%%%%%%%%%%%%%%%%%%%%%%%%%%%%%%%%%%%%%%%%%%%%%%%%%%%%%%%%%%%%%%%%%%%%%%%%%%%%%%%%
%%%%%%%%%%%%%%%%%%%%%%%%%%%%%%%%%%%%%%%%%%%%%%%%%%%%%%%%%%%%%%%%%%%%%%%%%%%%%%%%
\section{Introduction}

\LaTeX{} provides a mechanism to structure a large document (such as a book)
into a main file and several child files (containing the chapters)
using the |\include| command.
This mechanism is beneficial for documents
which span hundreds of pages in order to
make the source file(s) more manageable.
Moreover, compilation can be restricted to
selected child files by means of the |\includeonly| command.
The latter feature can be used to reduce the compilation time while editing
(this was significantly more useful in the earlier days of \LaTeX{})
or to generate a smaller document which is easier to navigate.
Another application of |\includeonly| is to generate
documents consisting of selected parts of the complete document.

However, there are a few drawbacks of the plain |\include| mechanism:
\begin{itemize}
\item
The child files cannot be compiled on their own,
they can only be compiled via the main file.
A naive editing environment
(such as a text editor with an option
to have the current file processed by \LaTeX)
may require one to switch to the main file before compiling;
attempting to compile the child file produces errors.
\item
The main file must be modified (each time)
to adjust the |\includeonly| command
to the present needs. This easily leaves the main file in a messy state.
\item
The generated document will always carry the filename
of the main document. This is inconvenient if
several child files are to be compiled and
to be kept for distribution.
\end{itemize}

The present package provides a simple interface
to make child files individually compilable by \LaTeX{}.
Compiling a child file then has the same effect as compiling
the main file with an |\includeonly| command
to select the appropriate child.
Moreover the generated document will carry the name of the child
rather than the main file.
This resolves all three above issues.

This feature is meant to make the editing of books,
thesis documents and lecture notes somewhat more convenient.
However, the package can also be used efficiently for
composing a series of documents (such as exercise sheets)
which are typically distributed individually.
It then assists the author in generating the individual documents
(potentially in different versions)
as well as a document containing the collected series.
Another application is in developing style files
or other kinds of included material
where compilation of the style file could redirect
to a sample or test file.

%%%%%%%%%%%%%%%%%%%%%%%%%%%%%%%%%%%%%%%%%%%%%%%%%%%%%%%%%%%%%%%%%%%%%%%%%%%%%%%%
%%%%%%%%%%%%%%%%%%%%%%%%%%%%%%%%%%%%%%%%%%%%%%%%%%%%%%%%%%%%%%%%%%%%%%%%%%%%%%%%
\section{Usage}

First of all, the package \textsf{childdoc} is \emph{not} a standard
\LaTeXe{} |.sty| style file! Therefore it needs to be invoked in
a non-standard way.

%%%%%%%%%%%%%%%%%%%%%%%%%%%%%%%%%%%%%%%%%%%%%%%%%%%%%%%%%%%%%%%%%%%%%%%%%%%%%%%%
\subsection{Included Files}
\label{sec:include}

%%%%%%%%%%%%%%%%%%%%%%%%%%%%%%%%%%%%%%%%
\DescribeMacro{\childdocmain}
To use the package, add the commands
\begin{center}
\begin{tabular}{l}
|\input{childdoc.def}|\\
|\childdocmain{}|\\
\end{tabular}
\end{center}
at the very top of the main \LaTeX{} file,
in particular \emph{before} the |\documentclass| statement!
The argument of |\childdocmain| should be left empty
(but it must be present).

%%%%%%%%%%%%%%%%%%%%%%%%%%%%%%%%%%%%%%%%
\DescribeMacro{\childdocof}
Furthermore, add the commands
\begin{center}
\begin{tabular}{l}
|\input{childdoc.def}|\\
|\childdocof{|\textit{main}|}|\\
\end{tabular}
\end{center}
at the top of every child file \textit{child}
which is included by |\include{|\textit{child}|}|
from within the main file
(or at least for those files to be compiled individually).
The argument \textit{main} must be the filename of the main file.

There are a couple of
considerations in setting up the main and child documents:

%%%%%%%%%%%%%%%%%%%%%%%%%%%%%%%%%%%%%%%%
\paragraph{Restrictions.}

Please note the following restrictions:
\begin{itemize}
\item
|\childdocmain| must be called with one argument \textit{main}
to ensure compatibility with earlier version of the package.
It must either be empty (|\childdocmain{}|)
or precisely match the filename of the main file in which it is specified.
See \secref{sec:detection} for further information.
\item
The filename \textit{main} must be specified without the |.tex| extension.
\item
The filename \textit{main} is case sensitive
(even in case-insensitive file systems)
due to internal string comparison.
\item
The argument \textit{main} should be fully expanded, it cannot be a macro.
\item
Subdirectories and special characters should be avoided in filenames.
\item
The command |\childdocmain{|\textit{main}|}| must be followed by a whitespace.
It should not be followed immediately by another command
or by a comment mark `|%|'.
This is because the \TeX{} parser reads the token immediately following
the argument of |\childdocmain| and puts it
at the beginning of every child section;
however, a white\-space is ignored.
\end{itemize}

%%%%%%%%%%%%%%%%%%%%%%%%%%%%%%%%%%%%%%%%
\paragraph{Content of Main File.}

It is advisable to place all content in the child files included by |\include|.
Any output contained in the main file will appear in all child documents
unless suppressed manually;
it cannot be suppressed automatically by the |\includeonly| directive
and thus should normally be avoided.
A method to include some content in the main file
by means of conditional processing is described in \secref{sec:conditional}.

%%%%%%%%%%%%%%%%%%%%%%%%%%%%%%%%%%%%%%%%
\paragraph{Page Numbering.}

When only a part of the document is compiled,
the appropriate numbering of pages
(as well as other status parameters)
is determined from the |.aux| files.
The latter contain information from previous passes.
However this information needs to propagate through
all intermediate child documents.
Therefore the page numbering in child documents may well
be inconsistent until the complete document is compiled at least once.

A useful (if unconventional) way to always ensure a consistent
page numbering is to restart the numbering in each child document
and denote the pages by `\textit{child}|.|\textit{page}'
where \textit{child} represents the chapter/section number of the child file.
This can be achieved by the command
|\numberwithin{page}{|\textit{child}|}|
of the \textsf{amsmath} package
where \textit{child} can be |chapter| or |section|
depending on the chosen structuring.
Alternatively, one can modify the macro |\thepage| appropriately
and reset the counter |page| at the start of each child file.

%%%%%%%%%%%%%%%%%%%%%%%%%%%%%%%%%%%%%%%%%%%%%%%%%%%%%%%%%%%%%%%%%%%%%%%%%%%%%%%%
\subsection{Conditional Processing}
\label{sec:conditional}

The package provides a mechanism to compile different versions
of a document. To customise the versions further some conditional processing
can come in handy to distinguish which version is being compiled.
The package provides two macros to describe the compilation context:

%%%%%%%%%%%%%%%%%%%%%%%%%%%%%%%%%%%%%%%%
\DescribeMacro{\ifchilddoc}
The conditional |\ifchilddoc| distinguishes between the compilation of
child documents and the main document:
%
\begin{center}
|\ifchilddoc |\textit{child-code}| |[|\||else |\textit{main-code}]| \||fi|
\end{center}

%%%%%%%%%%%%%%%%%%%%%%%%%%%%%%%%%%%%%%%%
\DescribeMacro{\childdocname}
\DescribeMacro{\childdocjob}
The macro |\childdocname| contains the filename (without extension)
of the main or child file being processed.
Note that |\childdocjob| will always contain the name of the main file.

%%%%%%%%%%%%%%%%%%%%%%%%%%%%%%%%%%%%%%%%
\paragraph{Title Page.}

Conditional processing can be used to include a title or banner page
in the main document when proper precautions are taken.
Importantly, the code in the main file should ensure that the page counter
(as well as other status parameters which are stored in the |.aux| files)
takes the same value after the conditional processing.
Otherwise the page numbers may take divergent values
depending on which part is compiled.

For example, a title page could be declared by:
%
\begin{center}
\begin{tabular}{l}
|\ifchilddoc\||else|\\
|\addtocounter{page}{-1}|\\
\textit{code for title page}\\
|\newpage|\\
|\||fi|
\end{tabular}
\end{center}
%
A banner page for the child documents can be generated by:
%
\begin{center}
\begin{tabular}{l}
|\ifchilddoc|\\
|\addtocounter{page}{-1}|\\
\textit{code for banner page}\\
|\newpage|\\
|\||fi|
\end{tabular}
\end{center}
%
Here one could write a message such as:
\begin{center}
|This is the part \childdocname{} of \childdocjob{}.|
\end{center}

%%%%%%%%%%%%%%%%%%%%%%%%%%%%%%%%%%%%%%%%%%%%%%%%%%%%%%%%%%%%%%%%%%%%%%%%%%%%%%%%
\subsection{Flags}
\label{sec:flags}

The package makes it easy to generate different versions
of the main or child documents.
To this end compilation flags can be defined
and assigned different default values.
They will be particularly useful in conjunction
with the forwarding mechanism described in \secref{sec:forward}.

For example, it may be useful to have a flag |\version|
which can be set to |draft| or |final|.
The document source will contain some conditional code
depending on the value of |\version|.
Suppose further, the flag should default to |final| for the main file
and to |draft| for child files
which is a natural assignment for editing the document.
This is achieved by placing the following code
in the preamble of the main document
(below the |\childdocmain| directive):
%
\begin{center}
\begin{tabular}{l}
|\ifchilddoc|\\
|\providecommand{\version}{draft}|\\
|\||else|\\
|\providecommand{\version}{final}|\\
|\||fi|
\end{tabular}
\end{center}
%
The definition by |\providecommand| makes sure
that previous definitions are not overwritten.
Further statements |\providecommand{\version}{...}|
can thus be added before the above code to override it.

For the main file, one might add a line
(between |\childdocmain| and the above block)
%
\begin{center}
|%\ifchilddoc\||else\providecommand{\version}{draft}\||fi|
\end{center}
%
which can be uncommented to produce a draft version.
Likewise one can add a line to the very top of a child file
(above the |\childdocof{|\textit{main}|}| directive)
%
\begin{center}
|%\providecommand{\version}{final}|
\end{center}
%
which can be uncommented to produce the final version of this child document.

%%%%%%%%%%%%%%%%%%%%%%%%%%%%%%%%%%%%%%%%%%%%%%%%%%%%%%%%%%%%%%%%%%%%%%%%%%%%%%%%
\subsection{Forwarding}
\label{sec:forward}

Different versions of the main or child documents
using compilation flags as described in \secref{sec:flags}
can be (permanently) stored in different files
for convenient compilation, viewing and distribution.
To this end, the package defines a command
to pass on compilation to a different file:

%%%%%%%%%%%%%%%%%%%%%%%%%%%%%%%%%%%%%%%%
\DescribeMacro{\childdocforward}
The command |\childdocforward| redirects processing to
another source file:
%
\begin{center}
\begin{tabular}{l}
|\input{childdoc.def}|\\
|\childdocforward[|\textit{main}|]{|\textit{dest}|}|\\
\end{tabular}
\end{center}
%
The argument \textit{dest} is the destination file
(without extension).
It should be the main file or one of the child files.
Note that further \textsf{childdoc} directives
such as |\childdocof| and |\childdocforward|
in the indicated file will be processed in this form.
The optional argument \textit{main}
passes on directly to the main file \textit{main}
while pretending to compile the child \textit{dest}.
This form behaves as if \textit{dest}
issues |\childdocof{|\textit{main}|}| right away,
and no further \textsf{childdoc} directives will be processed.

%%%%%%%%%%%%%%%%%%%%%%%%%%%%%%%%%%%%%%%%
\DescribeMacro{\...prefix}
In the alternative form |\childdocforwardprefix|,
%
\begin{center}
\begin{tabular}{l}
|\input{childdoc.def}|\\
|\childdocforwardprefix[|\textit{main}|]{|\textit{prefix}|}{|\textit{dest}|}|
\end{tabular}
\end{center}
%
the destination file is determined by a pattern
depending on the current file:
To make this work, the current file must be called
`{\textit{prefix}\hspace{0.2em}\textit{suffix}}'
with \textit{prefix} matching precisely the argument.
Processing is then passed on to the file
`{\textit{dest}\hspace{0.2em}\textit{suffix}}'.
Surely, the same effect is achieved by
directly specifying the
argument `{\textit{dest}\hspace{0.2em}\textit{suffix}}'
in the first form.
However, that requires to set up a different file
for each child. With the alternative form of the command
all these files can have exactly the same content
which simplifies setting them up and maintaining them.

For example, the following file |draft.tex|
with a compilation flag |\version| as described in \secref{sec:flags}
compiles the main document as a draft:
%
\begin{center}
\begin{tabular}{l}
|\def\version{draft}|\\
|\input{childdoc.def}|\\
|\childdocforward{|\textit{main}|}|
\end{tabular}
\end{center}
%
Likewise, the following files |final|\textit{nn}|.tex|
compile the final version of the child document
|child|\textit{nn}|.tex|:
%
\begin{center}
\begin{tabular}{l}
|\def\version{final}|\\
|\input{childdoc.def}|\\
|\childdocforwardprefix{final}{child}|
\end{tabular}
\end{center}
%

Note that when several versions of a main file and/or of each child file
are to be generated, it may be convenient to set up a |Makefile| or
shell script to automatise the process.

%%%%%%%%%%%%%%%%%%%%%%%%%%%%%%%%%%%%%%%%%%%%%%%%%%%%%%%%%%%%%%%%%%%%%%%%%%%%%%%%
\subsection{Command Line Processing}
\label{sec:commandline}

The effect of redirection files can also be achieved by invoking
the \LaTeX{} compiler with a more elaborate command line.
Most conveniently this should be done as part
of a shell script or a |Makefile|.

When using \textsf{childdoc} in the main file, the following
command lines effectively perform a redirection
(note that depending on the shell being used,
backslashes may have to be doubled: `|\|' $\to$ `|\\|'):
%
\begin{center}
|... -jobname "|\textit{target}|" |\\|"|[\textit{flags}]%
|\input{childdoc.def}\childdocforward[|\textit{main}|]{|\textit{dest}|}"|
\end{center}
%
Here \textit{target} is the name of the output file,
\textit{main} is the name of the main file
and \textit{dest} is the name of the main or child file to be processed
(all filenames without extensions).
The optional argument \textit{main} can be omitted
if \textit{main} matches \textit{dest}.
Optionally, compilation \textit{flags} can be defined via |\def| commands.
This command line makes the \TeX{} engine believe
it is compiling the file \textit{target}
whose content is specified as the latter parameter.
The provided code then forwards the processing to
\textit{main} or \textit{dest} as described in \secref{sec:forward}.

%%%%%%%%%%%%%%%%%%%%%%%%%%%%%%%%%%%%%%%%%%%%%%%%%%%%%%%%%%%%%%%%%%%%%%%%%%%%%%%%
\subsection{Include by Input}
\label{sec:input}

Including child documents by |\include| has some restrictions by design.
Most notably, the content of a child document always occupies
its own set of pages; pages cannot be shared between child documents.
Usually, this behaviour makes perfect sense
because each child document contain an essential part of the document.
However, in some situations it may be desirable to compose
a document from a collection of parts
without having mandatory page breaks between then.
For this case, the package
provides a mechanism to include parts
by |\input| which can also be processed individually.
However, by construction this mechanism
requires manual handling of the content to be output.

%%%%%%%%%%%%%%%%%%%%%%%%%%%%%%%%%%%%%%%%
\DescribeMacro{\ifchilddocmanual}
The main file should be prepared as usual, see \secref{sec:include}.
However, the document body must make a distinction
between processing of an individual part and of the main document, e.g.:
%
\begin{center}
\begin{tabular}{l}
|\ifchilddocmanual|\\
|\input{\childdocname}|\\
|\||else|\\
\textit{document body with }|\input{|\textit{part}|}|\\
|\||fi|
\end{tabular}
\end{center}
%
The conditional |\ifchilddocmanual| is true whenever
a part to be included by |\input| is being compiled,
and the name of the part is stored in |\childdocname|.

%%%%%%%%%%%%%%%%%%%%%%%%%%%%%%%%%%%%%%%%
\DescribeMacro{\childdocby}
Each part to be included by |\input| should start with:
%
\begin{center}
\begin{tabular}{l}
|\input{childdoc.def}|\\
|\childdocby{|\textit{main}|}|\\
\end{tabular}
\end{center}
%
The directive |\childdocby| is similar to |\childdocof|
described in \secref{sec:include},
but the subsequent selection of content must be done manually.
To that end, both |\ifchilddoc| and |\ifchilddocmanual|
will be true upon processing of a part,
and the name of the part is stored in |\childdocname|.
Note that |\jobname| will be set to the filename of the current part
so that each part receives an individual |.aux| file
that does not interfere with the |.aux| file(s) of the main document.
This behaviour can be altered by the alternative form
|\childdocby[*]{|\textit{main}|}| (with a non-empty optional argument)
which uses the |.aux| file of the main document
by setting |\jobname| to \textit{main}.

%%%%%%%%%%%%%%%%%%%%%%%%%%%%%%%%%%%%%%%%%%%%%%%%%%%%%%%%%%%%%%%%%%%%%%%%%%%%%%%%
\subsection{Driver Development}
\label{sec:driver}

The \textsf{childdoc} mechanism can also be use for the development
of definition files such as \LaTeX{} styles or classes.
This case differs from the above setup with multiple parts
included by |\include| in that no |\includeonly| should be invoked.
This can be achieved by starting the include file
(before |\ProvidesPackage|) with:
%
\begin{center}
\begin{tabular}{l}
|\input{childdoc.def}|\\
|\childdocforward{|\textit{main}|}|\\
\end{tabular}
\end{center}
%
or alternatively with:
%
\begin{center}
\begin{tabular}{l}
|\input{childdoc.def}|\\
|\childdocby{|\textit{main}|}|\\
\end{tabular}
\end{center}
%
Both forms have slightly different effects as described above.
The main file is prepared as usual, see \secref{sec:include}.

%%%%%%%%%%%%%%%%%%%%%%%%%%%%%%%%%%%%%%%%%%%%%%%%%%%%%%%%%%%%%%%%%%%%%%%%%%%%%%%%
\subsection{Legacy Detection}
\label{sec:detection}

The directive |\childdocmain| in the main file can detect
whether the complete document or merely a child is to be compiled
even without using the directive |\childdocof|.
This method is deprecated because it is less robust
and there is no compelling reason to use it;
it is merely provided for backward compatibility
and it may be removed in future versions.

If the detection mechanism is to be used,
it is mandatory to correctly specify
the filename of the main file as the argument of |\childdocmain|:
%
\begin{center}
\begin{tabular}{l}
|\input{childdoc.def}|\\
|\childdocmain{|\textit{main}|}|\\
\end{tabular}
\end{center}
%
If |\jobname| does not match the argument \textit{main} of |\childdocmain|,
it is assumed that |\jobname| points to the child file to be compiled.
When using |\childdocmain| with the main file specified as argument,
it suffices to start a child file
with just |\input{|\textit{main}|}|
without loading of the package and using |\childdocof|.
If instead all processing is done
with the appropriate \textsf{childdoc} directives,
the argument of \textit{main} of |\childdocmain| can be empty.

An alternative version of the command line processing described
in \secref{sec:commandline} using the detection mechanism reads:
%
\begin{center}
|... -jobname "|\textit{target}|" "|[\textit{flags}]%
[|\def\jobname{|\textit{dest}|}|]|\input{|\textit{main}|}"|
\end{center}

%%%%%%%%%%%%%%%%%%%%%%%%%%%%%%%%%%%%%%%%%%%%%%%%%%%%%%%%%%%%%%%%%%%%%%%%%%%%%%%%
\subsection{Manual Code}
\label{sec:manual}

In case one cannot be certain whether the definitions file |childdoc.def|
is installed on the target \TeX{} distribution
and one prefers not to ship it,
it is conceivable to paste a few relevant commands into the sources.

To that end, drop all statements |\input{childdoc.def}|
and perform the replacements as outlined below.
Instead of |\childdocmain{|\textit{main}|}| add the following code
to the top of the main file:
%
\begin{center}
\begin{tabular}{l}
|\||ifdefined\childdocname\endinput\||fi\newif\ifchilddoc|\\
|\edef\childdocname{\scantokens\expandafter{\jobname\noexpand}}|\\
|\def\childdocmain{|\textit{main}|}\||ifx\childdocmain\childdocname\||else|\\
|\childdoctrue\includeonly{\childdocname}\let\jobname\childdocmain\||fi|\\
\end{tabular}
\end{center}
%
Instead of |\childdocof{|\textit{main}|}| just include the main file
at the top of each child file:
%
\begin{center}
|\input{|\textit{main}|}|
\end{center}
%
A simple redirection |\childdocforward{|\textit{dest}|}| is achieved by:
%
\begin{center}
|\def\jobname{|\textit{dest}|}\input{\jobname}|
\end{center}
%
The redirection with prefix
|\childdocforwardprefix[|\textit{prefix}|]{|\textit{dest}|}|
is accomplished by:
%
\begin{center}
\begin{tabular}{l}
|{\edef\jobname{\scantokens\expandafter{\jobname\noexpand}}|\\
|\def\redirectjob |\textit{prefix}|#1~~~{\gdef\jobname{|\textit{dest}|#1}}|\\
|\expandafter\redirectjob\jobname~~~}\input{\jobname}|
\end{tabular}
\end{center}

In an alternative approach,
child documents can be compiled by a specific command line
without additional code or specific definitions:
%
\begin{center}
|... -jobname "|\textit{target}|" "|[\textit{flags}]%
|\includeonly{|\textit{dest}|}\input{|\textit{main}|}"|
\end{center}
%

%%%%%%%%%%%%%%%%%%%%%%%%%%%%%%%%%%%%%%%%%%%%%%%%%%%%%%%%%%%%%%%%%%%%%%%%%%%%%%%%
%%%%%%%%%%%%%%%%%%%%%%%%%%%%%%%%%%%%%%%%%%%%%%%%%%%%%%%%%%%%%%%%%%%%%%%%%%%%%%%%
\section{Information}

%%%%%%%%%%%%%%%%%%%%%%%%%%%%%%%%%%%%%%%%%%%%%%%%%%%%%%%%%%%%%%%%%%%%%%%%%%%%%%%%
\subsection{Copyright}

Copyright \copyright{} 2017--2018 Niklas Beisert

This work may be distributed and/or modified under the
conditions of the \LaTeX{} Project Public License, either version 1.3
of this license or (at your option) any later version.
The latest version of this license is in
  \url{http://www.latex-project.org/lppl.txt}
and version 1.3 or later is part of all distributions of \LaTeX{}
version 2005/12/01 or later.

This work has the LPPL maintenance status `maintained'.

The Current Maintainer of this work is Niklas Beisert.

This work consists of the files |README.txt|, |childdoc.ins| and |childdoc.dtx|
as well as the derived files |childdoc.def|, |cdocsamp.tex|
with |cdocsch1.tex|, |cdocsch2.tex|, |cdocspt3.tex|, |cdocspt4.tex|,
|cdocsdrf.tex|, |cdocsfn1.tex|, |cdocsfn2.tex|
as well as |childdoc.pdf|.

%%%%%%%%%%%%%%%%%%%%%%%%%%%%%%%%%%%%%%%%%%%%%%%%%%%%%%%%%%%%%%%%%%%%%%%%%%%%%%%%
\subsection{Files and Installation}

The package consists of the files:
%
\begin{center}
\begin{tabular}{ll}
    |README.txt|   & readme file \\
    |childdoc.ins| & installation file \\
    |childdoc.dtx| & source file \\
    |childdoc.def| & definition file \\
    |cdocsamp.tex| & sample main file \\
    |cdocsch1.tex| & sample include file \\
    |cdocsch2.tex| & sample include file \\
    |cdocspt3.tex| & sample part file \\
    |cdocspt4.tex| & sample part file \\
    |cdocsdrf.tex| & sample redirection file \\
    |cdocsfn1.tex| & sample redirection file \\
    |cdocsfn2.tex| & sample redirection file \\
    |childdoc.pdf| & manual
\end{tabular}
\end{center}
%
The distribution consists of the files
|README.txt|, |childdoc.ins| and |childdoc.dtx|.
%
\begin{itemize}
\item
Run (pdf)\LaTeX{} on |childdoc.dtx|
to compile the manual |childdoc.pdf| (this file).
\item
Run \LaTeX{} on |childdoc.ins| to create the definitions file |childdoc.def|
and the sample |cdocsamp.tex| with include files
|cdocsch1.tex|, |cdocsch2.tex|, |cdocspt3.tex|, |cdocspt4.tex|,
|cdocsdrf.tex|, |cdocsfn1.tex|, |cdocsfn2.tex|.
Then copy the file |childdoc.def| to an appropriate directory of your \LaTeX{}
distribution, e.g.\ \textit{texmf-root}|/tex/latex/childdoc|.
\end{itemize}

%%%%%%%%%%%%%%%%%%%%%%%%%%%%%%%%%%%%%%%%%%%%%%%%%%%%%%%%%%%%%%%%%%%%%%%%%%%%%%%%
\subsection{Related CTAN Packages}

There are several other packages which offer a similar functionality:
%
\begin{itemize}
\item
The packages
\href{http://ctan.org/pkg/docmute}{\textsf{docmute}},
\href{http://ctan.org/pkg/includex}{\textsf{includex}} and
\href{http://ctan.org/pkg/standalone}{\textsf{standalone}}
provide commands to include only the document body of
a child file thus allowing both files to be compiled individually.
\item
The packages \href{http://ctan.org/pkg/subdocs}{\textsf{subdocs}}
and \href{http://ctan.org/pkg/subfiles}{\textsf{subfiles}}
provide structures in which the main and child documents can be
encapsulated and allowing them to be compiled individually.
The inclusion mechanism is different from the conventional |\include|.
\item
The package \href{http://ctan.org/pkg/combine}{\textsf{combine}}
is an elaborate solution to combine several documents into one.
\end{itemize}
%
See also the CTAN topic \href{http://ctan.org/topic/subdocs}{\textsf{subdocs}}
for further related packages.
The present package differs from the above solutions in that
a document structure constructed with the conventional |\include| mechanism
just needs two extra commands at the top of every file
such that all constituent files can be compiled individually.

%%%%%%%%%%%%%%%%%%%%%%%%%%%%%%%%%%%%%%%%%%%%%%%%%%%%%%%%%%%%%%%%%%%%%%%%%%%%%%%%
%\subsection{Feature Suggestions}
%
%The following is a list of features which may be useful for future
%versions of this package:
%%
%\begin{itemize}
%\item
%\ldots
%\end{itemize}

%%%%%%%%%%%%%%%%%%%%%%%%%%%%%%%%%%%%%%%%%%%%%%%%%%%%%%%%%%%%%%%%%%%%%%%%%%%%%%%%
\subsection{Revision History}

%%%%%%%%%%%%%%%%%%%%%%%%%%%%%%%%%%%%%%%%
\paragraph{v2.0:} 2018/12/30

\begin{itemize}
\item
immediate forward processing
\item
added |\childdocby| mechanism
\item
manual restructured
\end{itemize}

%%%%%%%%%%%%%%%%%%%%%%%%%%%%%%%%%%%%%%%%
\paragraph{v1.6:} 2018/01/17

\begin{itemize}
\item
application for development of include files
\item
corrections to manual
\end{itemize}

%%%%%%%%%%%%%%%%%%%%%%%%%%%%%%%%%%%%%%%%
\paragraph{v1.5:} 2017/05/21

\begin{itemize}
\item
more complete structuring introduced
\item
|\childdocof| introduced
\item
|\childdoc| renamed to |\childdocmain|
\item
|\childredirect| renamed to |\childdocforward| and |\childdocforwardprefix|
and functionality expanded
\end{itemize}

%%%%%%%%%%%%%%%%%%%%%%%%%%%%%%%%%%%%%%%%
\paragraph{v1.0:} 2017/04/27

\begin{itemize}
\item
manual and install package
\item
first version published on CTAN
\end{itemize}

%%%%%%%%%%%%%%%%%%%%%%%%%%%%%%%%%%%%%%%%
\paragraph{v0.6:} 2017/04/26

\begin{itemize}
\item
redirection mechanism added
\end{itemize}

%%%%%%%%%%%%%%%%%%%%%%%%%%%%%%%%%%%%%%%%
\paragraph{v0.5:} 2017/04/26

\begin{itemize}
\item
functionality in definition file
\end{itemize}


%%%%%%%%%%%%%%%%%%%%%%%%%%%%%%%%%%%%%%%%%%%%%%%%%%%%%%%%%%%%%%%%%%%%%%%%%%%%%%%%
%%%%%%%%%%%%%%%%%%%%%%%%%%%%%%%%%%%%%%%%%%%%%%%%%%%%%%%%%%%%%%%%%%%%%%%%%%%%%%%%
%%%%%%%%%%%%%%%%%%%%%%%%%%%%%%%%%%%%%%%%%%%%%%%%%%%%%%%%%%%%%%%%%%%%%%%%%%%%%%%%
\appendix

\settowidth\MacroIndent{\rmfamily\scriptsize 000\ }

 \DocInput{childdoc.dtx}

\end{document}
%</driver>
% \fi
%
% %%%%%%%%%%%%%%%%%%%%%%%%%%%%%%%%%%%%%%%%%%%%%%%%%%%%%%%%%%%%%%%%%%%%%%%%%%%%%%
% %%%%%%%%%%%%%%%%%%%%%%%%%%%%%%%%%%%%%%%%%%%%%%%%%%%%%%%%%%%%%%%%%%%%%%%%%%%%%%
% \section{Sample}
%\iffalse
%<*samplemain>
%\fi
%
% The following presents a sample document
% with two chapters, two parts, a title page,
% a compile flag as well as three forwarding files to set the flag.
% It consists of eight |.tex| files:
% \begin{center}
% \begin{tabular}{ll}
% |cdocsamp.tex|&main file\\
% |cdocsch1.tex|&include file for chapter 1\\
% |cdocsch2.tex|&include file for chapter 2\\
% |cdocspt3.tex|&include file for part 3\\
% |cdocspt4.tex|&include file for part 4\\
% |cdocsdrf.tex|&forwarding file for main file in draft mode\\
% |cdocsfi1.tex|&forwarding file for final version of chapter 1\\
% |cdocsfi2.tex|&forwarding file for final version of chapter 2\\
% \end{tabular}
% \end{center}
% Each of the eight files can be compiled directly by the \LaTeX{} compiler.
%
% %%%%%%%%%%%%%%%%%%%%%%%%%%%%%%%%%%%%%%
% \paragraph{Main File.}
%
% The main file is called |cdocsamp.tex|.
%
% Load the \textsf{childdoc} definitions and
% declare the filename for the main document:
%    \begin{macrocode}
\input{childdoc.def}
\childdocmain{}
%    \end{macrocode}

% Optional override for |\version| flag:
%    \begin{macrocode}
%%\ifchilddoc\else\providecommand{\version}{draft}\fi
%    \end{macrocode}

% Define the default values for the |\version| flag
% (|final| for the main file and |draft| for childs):
%    \begin{macrocode}
\ifchilddoc
\providecommand{\version}{draft}
\else
\providecommand{\version}{final}
\fi
%    \end{macrocode}

% Load the standard document class:
%    \begin{macrocode}
\documentclass[12pt]{article}
%    \end{macrocode}

% Start the document body:
%    \begin{macrocode}
\begin{document}
%    \end{macrocode}

% Declare a title page.
% Print title, part of document being processed and version flag:
%    \begin{macrocode}
\addtocounter{page}{-1}
\begin{center}
{\LARGE\bfseries{}childdoc example\par}
\vspace{1cm}
\ifchilddoc
\ifchilddocmanual part\else chapter\fi:
`\childdocname' of `\childdocjob'\par
\else
main document: `\childdocjob'\par
\fi
version: \version\par
\end{center}
\newpage
%    \end{macrocode}

% Manually include selected file,
% otherwise process as usual:
%    \begin{macrocode}
\ifchilddocmanual
\section*{part `\childdocname'}
\input{\childdocname}
\else
%    \end{macrocode}

% Include the two chapters:
%    \begin{macrocode}
\include{cdocsch1}
\include{cdocsch2}
%    \end{macrocode}

% Include the two parts unless only chapters should be displayed:
%    \begin{macrocode}
\ifchilddoc\else
\section{part three}
\input{cdocspt3}
\section{part four}
\input{cdocspt4}
\fi
%    \end{macrocode}

% Process as usual until here:
%    \begin{macrocode}
\fi
%    \end{macrocode}

% End of document body:
%    \begin{macrocode}
\end{document}
%    \end{macrocode}
%\iffalse
%</samplemain>
%\fi
%
% %%%%%%%%%%%%%%%%%%%%%%%%%%%%%%%%%%%%%%
% \paragraph{Chapter Include Files.}
%
% The include files are called |cdocsch1.tex| and |cdocsch2.tex|.
%
%\iffalse
%<*samplechap1|samplechap2>
%\fi

% Optional override for |\version| flag:
%    \begin{macrocode}
%%\providecommand{\version}{final}
%    \end{macrocode}

% Include the main document:
%    \begin{macrocode}
\input{childdoc.def}
\childdocof{cdocsamp}
%    \end{macrocode}

%\iffalse
%</samplechap1|samplechap2>
%\fi
%
%\iffalse
%<*samplechap1>
%\fi
% Some text for chapter 1:
%    \begin{macrocode}
\section{one}
some text in chapter one
%    \end{macrocode}

%\iffalse
%</samplechap1>
%\fi
% Some text for chapter 2:
%\iffalse
%<*samplechap2>
%\fi
%    \begin{macrocode}
\section{two}
more text in chapter two
%    \end{macrocode}

%\iffalse
%</samplechap2>
%\fi
%
% %%%%%%%%%%%%%%%%%%%%%%%%%%%%%%%%%%%%%%
% \paragraph{Part Include Files.}
%
% The include files are called |cdocspt3.tex| and |cdocspt4.tex|.
%
%\iffalse
%<*samplepart3|samplepart4>
%\fi

% Optional override for |\version| flag:
%    \begin{macrocode}
%%\providecommand{\version}{final}
%    \end{macrocode}

% Include the main document:
%    \begin{macrocode}
\input{childdoc.def}
\childdocby{cdocsamp}
%    \end{macrocode}

%\iffalse
%</samplepart3|samplepart4>
%\fi
%
%\iffalse
%<*samplepart3>
%\fi
% Some text for part 3:
%    \begin{macrocode}
some text in part three
%    \end{macrocode}

%\iffalse
%</samplepart3>
%\fi
% Some text for part 4:
%\iffalse
%<*samplepart4>
%\fi
%    \begin{macrocode}
more text in part four
%    \end{macrocode}

%\iffalse
%</samplepart4>
%\fi
%
% %%%%%%%%%%%%%%%%%%%%%%%%%%%%%%%%%%%%%%
% \paragraph{Forwarding for a Complete Draft.}
%
% The following forwarding file |cdocsdrf.tex|
% compiles the main document in draft mode:
%\iffalse
%<*sampledraft>
%\fi
%    \begin{macrocode}
\def\version{draft}
\input{childdoc.def}
\childdocforward{cdocsamp}
%    \end{macrocode}

%\iffalse
%</sampledraft>
%\fi
%
% %%%%%%%%%%%%%%%%%%%%%%%%%%%%%%%%%%%%%%
% \paragraph{Forwarding for Final Version of the Chapters.}
%
% The following forwarding files |cdocsfn1.tex| and |cdocsfn2.tex|
% (with identical content)
% compile the final versions of the child documents
% |cdocsch1.tex| and |cdocsch2.tex|, respectively:
%\iffalse
%<*samplefinal>
%\fi
%    \begin{macrocode}
\def\version{final}
\input{childdoc.def}
\childdocforwardprefix[cdocsamp]{cdocsfn}{cdocsch}
%    \end{macrocode}

%\iffalse
%</samplefinal>
%\fi
%
% %%%%%%%%%%%%%%%%%%%%%%%%%%%%%%%%%%%%%%
% \paragraph{Command Line Processing.}
%
% The following three command lines generate the output files
% |cdocscld|, |cdocscl1| and |cdocscl2|
% which should be identical to
% |cdocsdrf|, |cdocsch1| and |cdocsfn2|, respectively:
% \begin{center}
% \begin{tabular}{l}
% |latex -jobname cdocscld \|\\
% |  "\def\version{draft}\input{childdoc.def}\childdocforward{cdocsamp}"|\\
% |latex -jobname cdocscl1 \|\\
% |  "\input{childdoc.def}\childdocforward[cdocsamp]{cdocsch1}"|\\
% |latex -jobname cdocscl2 \|\\
% |  "\def\version{final}\input{childdoc.def}\childdocforward{cdocsch2}"|
% \end{tabular}
% \end{center}
% Note that the trailing backslash on each first line
% merely continues the input to the second line
% (for convenient cut ant paste).
% Furthermore, the command |latex| can be replaced by any
% of its alternative versions such as |pdflatex|.
%
% %%%%%%%%%%%%%%%%%%%%%%%%%%%%%%%%%%%%%%%%%%%%%%%%%%%%%%%%%%%%%%%%%%%%%%%%%%%%%%
% %%%%%%%%%%%%%%%%%%%%%%%%%%%%%%%%%%%%%%%%%%%%%%%%%%%%%%%%%%%%%%%%%%%%%%%%%%%%%%
% \section{Implementation}
%\iffalse
%<*package>
%\fi
%
% This section describes the definitions file |childdoc.def|.

% The definitions cannot be loaded using |\usepackage| or |\RequirePackage|
% which has a mechanism to prevent loading a style file more than once.
% When loading the definitions by means of |\input|
% multiple instances have to be prevented manually:
%\iffalse
%This code needs to be before the `\ProvidesFile' directive
%which is defined at the beginning of this file.
%Therefore it is also placed there and commented out here.
%</package>
%<*discard>
%\fi
%    \begin{macrocode}
\ifdefined\childdocmain\endinput\fi
%    \end{macrocode}
%\iffalse
%</discard>
%<*package>
%\fi
%
% \macro{\ifchilddoc}
% \macro{\ifchilddocmanual}
% The conditional |\ifchilddoc| tells whether a
% child (true) or main (false) document is being compiled.
% The conditional |\ifchilddocmanual| tells whether
% the |\includeonly| mechanism is used (false) or
% the selection of child files must be performed manually (true).
% The definitions initialise to false:
%    \begin{macrocode}
\newif\ifchilddoc
\newif\ifchilddocmanual
%    \end{macrocode}

% \macro{\childdocname}
% \macro{\childdocjob}
% The macro |\childdocname| stores the name of the main document
% to be compiled. The macro |\childdocjob| stores the name of
% the document on which the \LaTeX{} compiler was originally invoked.
% The content of |\jobname| cannot be compared
% to filenames specified in the source due to different catcodes.
% The following code rescans |\jobname|, stores the result
% in |\childdocname| and saves a copy in |\childdocjob|:
%    \begin{macrocode}
\edef\childdocname{\scantokens\expandafter{\jobname\noexpand}}
\let\childdocjob\childdocname
%    \end{macrocode}

% \macro{\childdocdisable}
% The macro |\childdocdisable| prevents the main file
% from being processed more than once.
% At this stage, the main document command |\childdocmain|
% is assumed to be called once again where it should do nothing.
% Any subsequent call to it should prevent
% a secondary processing of the main document
% It overwrites the forwarding commands
% |\childdocof| and |\childdocforward|
% with empty macros to prevent further inclusions of the main document:
%    \begin{macrocode}
\newcommand{\childdocdisable}
{
  \renewcommand{\childdocmain}[1]{\renewcommand{\childdocmain}[1]{\endinput}}
  \renewcommand{\childdocof}[1]{}
  \renewcommand{\childdocby}[2][]{}
  \renewcommand{\childdocforward}[2][]{}
  \renewcommand{\childdocdisable}{}
}
%    \end{macrocode}

% \macro{\childdocmain}
% The macro |\childdocmain| is to be called at the top of the main file
% with nothing or the main filename (without extension) as argument.
% First, it breaks loops.
% If the argument is not empty and does not match |\childdocname|
% (which is set by the first inclusion of |childdoc.def|),
% |\ifchilddoc| is set to true, |\includeonly| is applied to the child file
% and |\jobname| is set to the main file
% (for proper handling of |.aux| files):
%    \begin{macrocode}
\newcommand{\childdocmain}[1]
{
  \childdocdisable\childdocmain{}
  \if?#1?\else
    \begingroup
      \def\childdoctmp{#1}
      \ifx\childdoctmp\childdocname
        \def\childdoctmp{}
      \else
        \def\childdoctmp
        {
          \childdoctrue
          \includeonly{\childdocname}
          \def\childdocjob{#1}
          \def\jobname{#1}
        }
      \fi
      \expandafter
    \endgroup
    \childdoctmp
  \fi
}
%    \end{macrocode}

% \macro{\childdocof}
% The command |\childdocof| redirects
% compilation to the main file |#1|.
%    \begin{macrocode}
\newcommand{\childdocof}[1]
{
  \childdocdisable
  \childdoctrue
  \includeonly{\childdocname}
  \def\jobname{#1}
  \def\childdocjob{#1}
  \input{#1}
}
%    \end{macrocode}

% \macro{\childdocby}
% The command |\childdocby| ....
%    \begin{macrocode}
\newcommand{\childdocby}[2][]
{
  \childdocdisable
  \childdoctrue
  \childdocmanualtrue
  \if?#1?\else
    \def\jobname{#2}
  \fi
  \def\childdocjob{#2}
  \input{#2}
  \endinput
}
%    \end{macrocode}

% \macro{\childdocforward}
% The command |\childdocforward| redirects
% compilation to the main file or
% (if the optional argument is given) a child file.
% Parameters are set as if the main file
% or a child file starting with |\childdocof| was compiled.
% Then compilation is handed over to the main file:
%    \begin{macrocode}
\newcommand{\childdocforward}[2][]
{
  \begingroup
    \if?#1?
      \def\childdoctmp
      {
        \def\childdocname{#2}
        \def\childdocjob{#2}
        \def\jobname{#2}
        \input{#2}
        \endinput
      }
    \else
      \def\childdoctmp
      {
        \childdocdisable
        \def\childdocname{#2}
        \childdoctrue
        \includeonly{#2}
        \def\childdocjob{#1}
        \def\jobname{#1}
        \input{#1}
        \endinput
      }
    \fi
    \expandafter
  \endgroup
  \childdoctmp
}
%    \end{macrocode}

% \macro{\childdocforwardprefix}
% The command |\childdocforwardprefix| redirects
% compilation to the main or a child file by means of a pattern.
% The prefix |#1| in the current filename is replaced by |#2|
% and the suffix of the current filename is kept
% (it is assumed that the filename does not contain the substring `|~~~|'
% which is used as a delimiter).
% Compilation is handed over to the new file by |\childdocforward|:
%    \begin{macrocode}
\newcommand{\childdocforwardprefix}[3][]
{
  \begingroup
    \def\childdocextract #2##1~~~{\def\childdoctmp{\childdocforward[#1]{#3##1}}}
    \expandafter\childdocextract\childdocname~~~
    \expandafter
  \endgroup
  \childdoctmp
}
%    \end{macrocode}

% \macro{\childdoc}
% The deprecated macro |\childdoc| is a legacy version of |\childdocmain|:
%    \begin{macrocode}
\newcommand{\childdoc}{\childdocmain}
%    \end{macrocode}

% \macro{\childdocredirect}
% The deprecated macro |\childdocredirect| is a legacy version
% of |\childdocforward| and |\childdocforwardprefix|:
%    \begin{macrocode}
\newcommand{\childdocredirect}[2][]
{
  \begingroup
    \if?#1?
      \def\childdoctmp{\childdocforward{#2}}
    \else
      \def\childdoctmp{\childdocforwardprefix{#1}{#2}}
    \fi
    \expandafter
  \endgroup
  \childdoctmp
}
%    \end{macrocode}

%\iffalse
%</package>
%\fi
%
\endinput
\childdocforward{cdocsch2}"|
% \end{tabular}
% \end{center}
% Note that the trailing backslash on each first line
% merely continues the input to the second line
% (for convenient cut ant paste).
% Furthermore, the command |latex| can be replaced by any
% of its alternative versions such as |pdflatex|.
%
% %%%%%%%%%%%%%%%%%%%%%%%%%%%%%%%%%%%%%%%%%%%%%%%%%%%%%%%%%%%%%%%%%%%%%%%%%%%%%%
% %%%%%%%%%%%%%%%%%%%%%%%%%%%%%%%%%%%%%%%%%%%%%%%%%%%%%%%%%%%%%%%%%%%%%%%%%%%%%%
% \section{Implementation}
%\iffalse
%<*package>
%\fi
%
% This section describes the definitions file |childdoc.def|.

% The definitions cannot be loaded using |\usepackage| or |\RequirePackage|
% which has a mechanism to prevent loading a style file more than once.
% When loading the definitions by means of |\input|
% multiple instances have to be prevented manually:
%\iffalse
%This code needs to be before the `\ProvidesFile' directive
%which is defined at the beginning of this file.
%Therefore it is also placed there and commented out here.
%</package>
%<*discard>
%\fi
%    \begin{macrocode}
\ifdefined\childdocmain\endinput\fi
%    \end{macrocode}
%\iffalse
%</discard>
%<*package>
%\fi
%
% \macro{\ifchilddoc}
% \macro{\ifchilddocmanual}
% The conditional |\ifchilddoc| tells whether a
% child (true) or main (false) document is being compiled.
% The conditional |\ifchilddocmanual| tells whether
% the |\includeonly| mechanism is used (false) or
% the selection of child files must be performed manually (true).
% The definitions initialise to false:
%    \begin{macrocode}
\newif\ifchilddoc
\newif\ifchilddocmanual
%    \end{macrocode}

% \macro{\childdocname}
% \macro{\childdocjob}
% The macro |\childdocname| stores the name of the main document
% to be compiled. The macro |\childdocjob| stores the name of
% the document on which the \LaTeX{} compiler was originally invoked.
% The content of |\jobname| cannot be compared
% to filenames specified in the source due to different catcodes.
% The following code rescans |\jobname|, stores the result
% in |\childdocname| and saves a copy in |\childdocjob|:
%    \begin{macrocode}
\edef\childdocname{\scantokens\expandafter{\jobname\noexpand}}
\let\childdocjob\childdocname
%    \end{macrocode}

% \macro{\childdocdisable}
% The macro |\childdocdisable| prevents the main file
% from being processed more than once.
% At this stage, the main document command |\childdocmain|
% is assumed to be called once again where it should do nothing.
% Any subsequent call to it should prevent
% a secondary processing of the main document
% It overwrites the forwarding commands
% |\childdocof| and |\childdocforward|
% with empty macros to prevent further inclusions of the main document:
%    \begin{macrocode}
\newcommand{\childdocdisable}
{
  \renewcommand{\childdocmain}[1]{\renewcommand{\childdocmain}[1]{\endinput}}
  \renewcommand{\childdocof}[1]{}
  \renewcommand{\childdocby}[2][]{}
  \renewcommand{\childdocforward}[2][]{}
  \renewcommand{\childdocdisable}{}
}
%    \end{macrocode}

% \macro{\childdocmain}
% The macro |\childdocmain| is to be called at the top of the main file
% with nothing or the main filename (without extension) as argument.
% First, it breaks loops.
% If the argument is not empty and does not match |\childdocname|
% (which is set by the first inclusion of |childdoc.def|),
% |\ifchilddoc| is set to true, |\includeonly| is applied to the child file
% and |\jobname| is set to the main file
% (for proper handling of |.aux| files):
%    \begin{macrocode}
\newcommand{\childdocmain}[1]
{
  \childdocdisable\childdocmain{}
  \if?#1?\else
    \begingroup
      \def\childdoctmp{#1}
      \ifx\childdoctmp\childdocname
        \def\childdoctmp{}
      \else
        \def\childdoctmp
        {
          \childdoctrue
          \includeonly{\childdocname}
          \def\childdocjob{#1}
          \def\jobname{#1}
        }
      \fi
      \expandafter
    \endgroup
    \childdoctmp
  \fi
}
%    \end{macrocode}

% \macro{\childdocof}
% The command |\childdocof| redirects
% compilation to the main file |#1|.
%    \begin{macrocode}
\newcommand{\childdocof}[1]
{
  \childdocdisable
  \childdoctrue
  \includeonly{\childdocname}
  \def\jobname{#1}
  \def\childdocjob{#1}
  \input{#1}
}
%    \end{macrocode}

% \macro{\childdocby}
% The command |\childdocby| ....
%    \begin{macrocode}
\newcommand{\childdocby}[2][]
{
  \childdocdisable
  \childdoctrue
  \childdocmanualtrue
  \if?#1?\else
    \def\jobname{#2}
  \fi
  \def\childdocjob{#2}
  \input{#2}
  \endinput
}
%    \end{macrocode}

% \macro{\childdocforward}
% The command |\childdocforward| redirects
% compilation to the main file or
% (if the optional argument is given) a child file.
% Parameters are set as if the main file
% or a child file starting with |\childdocof| was compiled.
% Then compilation is handed over to the main file:
%    \begin{macrocode}
\newcommand{\childdocforward}[2][]
{
  \begingroup
    \if?#1?
      \def\childdoctmp
      {
        \def\childdocname{#2}
        \def\childdocjob{#2}
        \def\jobname{#2}
        \input{#2}
        \endinput
      }
    \else
      \def\childdoctmp
      {
        \childdocdisable
        \def\childdocname{#2}
        \childdoctrue
        \includeonly{#2}
        \def\childdocjob{#1}
        \def\jobname{#1}
        \input{#1}
        \endinput
      }
    \fi
    \expandafter
  \endgroup
  \childdoctmp
}
%    \end{macrocode}

% \macro{\childdocforwardprefix}
% The command |\childdocforwardprefix| redirects
% compilation to the main or a child file by means of a pattern.
% The prefix |#1| in the current filename is replaced by |#2|
% and the suffix of the current filename is kept
% (it is assumed that the filename does not contain the substring `|~~~|'
% which is used as a delimiter).
% Compilation is handed over to the new file by |\childdocforward|:
%    \begin{macrocode}
\newcommand{\childdocforwardprefix}[3][]
{
  \begingroup
    \def\childdocextract #2##1~~~{\def\childdoctmp{\childdocforward[#1]{#3##1}}}
    \expandafter\childdocextract\childdocname~~~
    \expandafter
  \endgroup
  \childdoctmp
}
%    \end{macrocode}

% \macro{\childdoc}
% The deprecated macro |\childdoc| is a legacy version of |\childdocmain|:
%    \begin{macrocode}
\newcommand{\childdoc}{\childdocmain}
%    \end{macrocode}

% \macro{\childdocredirect}
% The deprecated macro |\childdocredirect| is a legacy version
% of |\childdocforward| and |\childdocforwardprefix|:
%    \begin{macrocode}
\newcommand{\childdocredirect}[2][]
{
  \begingroup
    \if?#1?
      \def\childdoctmp{\childdocforward{#2}}
    \else
      \def\childdoctmp{\childdocforwardprefix{#1}{#2}}
    \fi
    \expandafter
  \endgroup
  \childdoctmp
}
%    \end{macrocode}

%\iffalse
%</package>
%\fi
%
\endinput
|\\
|\childdocforwardprefix[|\textit{main}|]{|\textit{prefix}|}{|\textit{dest}|}|
\end{tabular}
\end{center}
%
the destination file is determined by a pattern
depending on the current file:
To make this work, the current file must be called
`{\textit{prefix}\hspace{0.2em}\textit{suffix}}'
with \textit{prefix} matching precisely the argument.
Processing is then passed on to the file
`{\textit{dest}\hspace{0.2em}\textit{suffix}}'.
Surely, the same effect is achieved by
directly specifying the
argument `{\textit{dest}\hspace{0.2em}\textit{suffix}}'
in the first form.
However, that requires to set up a different file
for each child. With the alternative form of the command
all these files can have exactly the same content
which simplifies setting them up and maintaining them.

For example, the following file |draft.tex|
with a compilation flag |\version| as described in \secref{sec:flags}
compiles the main document as a draft:
%
\begin{center}
\begin{tabular}{l}
|\def\version{draft}|\\
|% \iffalse
%
% childdoc.dtx Copyright (C) 2017-2018 Niklas Beisert
%
% This work may be distributed and/or modified under the
% conditions of the LaTeX Project Public License, either version 1.3
% of this license or (at your option) any later version.
% The latest version of this license is in
%   http://www.latex-project.org/lppl.txt
% and version 1.3 or later is part of all distributions of LaTeX
% version 2005/12/01 or later.
%
% This work has the LPPL maintenance status `maintained'.
%
% The Current Maintainer of this work is Niklas Beisert.
%
% This work consists of the files childdoc.dtx and childdoc.ins
% and the derived files childdoc.def and cdocsamp.tex with
% cdocsch1.tex, cdocsch2.tex, cdocsdrf.tex, cdocsfn1.tex, cdocsfn2.tex.
%
%<package>\ifdefined\childdocmain\endinput\fi
%<package>\ProvidesFile{childdoc.def}[2018/12/30 v2.0 child document driver]
%<samplemain>\ProvidesFile{cdocsamp.tex}[2018/12/30 v2.0 sample for childdoc]
%<*driver>
%\ProvidesFile{childdoc.drv}[2018/12/30 v2.0 childdoc reference manual file]
\PassOptionsToClass{10pt,a4paper}{article}
\documentclass{ltxdoc}

\usepackage[margin=35mm]{geometry}
\usepackage{hyperref}
\usepackage{hyperxmp}
\usepackage[usenames]{color}

\hypersetup{colorlinks=true}
\hypersetup{pdfstartview=FitH}
\hypersetup{pdfpagemode=UseNone}
\hypersetup{pdfsource={}}
\hypersetup{pdflang={en-UK}}
\hypersetup{pdfcopyright={Copyright 2017-2018 Niklas Beisert.
  This work may be distributed and/or modified under the
  conditions of the LaTeX Project Public License, either version 1.3
  of this license or (at your option) any later version.}}
\hypersetup{pdflicenseurl={http://www.latex-project.org/lppl.txt}}
\hypersetup{pdfcontactaddress={ETH Zurich, ITP, HIT K,
  Wolfgang-Pauli-Strasse 27}}
\hypersetup{pdfcontactpostcode={8093}}
\hypersetup{pdfcontactcity={Zurich}}
\hypersetup{pdfcontactcountry={Switzerland}}
\hypersetup{pdfcontactemail={nbeisert@itp.phys.ethz.ch}}
\hypersetup{pdfcontacturl={http://people.phys.ethz.ch/\xmptilde nbeisert/}}

\newcommand{\secref}[1]{\hyperref[#1]{section \ref*{#1}}}

\parskip1ex
\parindent0pt
\let\olditemize\itemize
\def\itemize{\olditemize\parskip0pt}

\begin{document}

\title{The \textsf{childdoc} Package}
\hypersetup{pdftitle={The childdoc Package}}
\author{Niklas Beisert\\[2ex]
  Institut f\"ur Theoretische Physik\\
  Eidgen\"ossische Technische Hochschule Z\"urich\\
  Wolfgang-Pauli-Strasse 27, 8093 Z\"urich, Switzerland\\[1ex]
  \href{mailto:nbeisert@itp.phys.ethz.ch}
  {\texttt{nbeisert@itp.phys.ethz.ch}}}
\hypersetup{pdfauthor={Niklas Beisert}}
\hypersetup{pdfsubject={Manual for the LaTeX2e Package childdoc}}
\date{30 December 2018, \textsf{v2.0}}
\maketitle

\begin{abstract}\noindent
\textsf{childdoc} is a \LaTeXe{} package
that enables the direct compilation
of document sections included by |\include|
to individual files.
\end{abstract}

\begingroup
\parskip0ex
\tableofcontents
\endgroup

%%%%%%%%%%%%%%%%%%%%%%%%%%%%%%%%%%%%%%%%%%%%%%%%%%%%%%%%%%%%%%%%%%%%%%%%%%%%%%%%
%%%%%%%%%%%%%%%%%%%%%%%%%%%%%%%%%%%%%%%%%%%%%%%%%%%%%%%%%%%%%%%%%%%%%%%%%%%%%%%%
\section{Introduction}

\LaTeX{} provides a mechanism to structure a large document (such as a book)
into a main file and several child files (containing the chapters)
using the |\include| command.
This mechanism is beneficial for documents
which span hundreds of pages in order to
make the source file(s) more manageable.
Moreover, compilation can be restricted to
selected child files by means of the |\includeonly| command.
The latter feature can be used to reduce the compilation time while editing
(this was significantly more useful in the earlier days of \LaTeX{})
or to generate a smaller document which is easier to navigate.
Another application of |\includeonly| is to generate
documents consisting of selected parts of the complete document.

However, there are a few drawbacks of the plain |\include| mechanism:
\begin{itemize}
\item
The child files cannot be compiled on their own,
they can only be compiled via the main file.
A naive editing environment
(such as a text editor with an option
to have the current file processed by \LaTeX)
may require one to switch to the main file before compiling;
attempting to compile the child file produces errors.
\item
The main file must be modified (each time)
to adjust the |\includeonly| command
to the present needs. This easily leaves the main file in a messy state.
\item
The generated document will always carry the filename
of the main document. This is inconvenient if
several child files are to be compiled and
to be kept for distribution.
\end{itemize}

The present package provides a simple interface
to make child files individually compilable by \LaTeX{}.
Compiling a child file then has the same effect as compiling
the main file with an |\includeonly| command
to select the appropriate child.
Moreover the generated document will carry the name of the child
rather than the main file.
This resolves all three above issues.

This feature is meant to make the editing of books,
thesis documents and lecture notes somewhat more convenient.
However, the package can also be used efficiently for
composing a series of documents (such as exercise sheets)
which are typically distributed individually.
It then assists the author in generating the individual documents
(potentially in different versions)
as well as a document containing the collected series.
Another application is in developing style files
or other kinds of included material
where compilation of the style file could redirect
to a sample or test file.

%%%%%%%%%%%%%%%%%%%%%%%%%%%%%%%%%%%%%%%%%%%%%%%%%%%%%%%%%%%%%%%%%%%%%%%%%%%%%%%%
%%%%%%%%%%%%%%%%%%%%%%%%%%%%%%%%%%%%%%%%%%%%%%%%%%%%%%%%%%%%%%%%%%%%%%%%%%%%%%%%
\section{Usage}

First of all, the package \textsf{childdoc} is \emph{not} a standard
\LaTeXe{} |.sty| style file! Therefore it needs to be invoked in
a non-standard way.

%%%%%%%%%%%%%%%%%%%%%%%%%%%%%%%%%%%%%%%%%%%%%%%%%%%%%%%%%%%%%%%%%%%%%%%%%%%%%%%%
\subsection{Included Files}
\label{sec:include}

%%%%%%%%%%%%%%%%%%%%%%%%%%%%%%%%%%%%%%%%
\DescribeMacro{\childdocmain}
To use the package, add the commands
\begin{center}
\begin{tabular}{l}
|% \iffalse
%
% childdoc.dtx Copyright (C) 2017-2018 Niklas Beisert
%
% This work may be distributed and/or modified under the
% conditions of the LaTeX Project Public License, either version 1.3
% of this license or (at your option) any later version.
% The latest version of this license is in
%   http://www.latex-project.org/lppl.txt
% and version 1.3 or later is part of all distributions of LaTeX
% version 2005/12/01 or later.
%
% This work has the LPPL maintenance status `maintained'.
%
% The Current Maintainer of this work is Niklas Beisert.
%
% This work consists of the files childdoc.dtx and childdoc.ins
% and the derived files childdoc.def and cdocsamp.tex with
% cdocsch1.tex, cdocsch2.tex, cdocsdrf.tex, cdocsfn1.tex, cdocsfn2.tex.
%
%<package>\ifdefined\childdocmain\endinput\fi
%<package>\ProvidesFile{childdoc.def}[2018/12/30 v2.0 child document driver]
%<samplemain>\ProvidesFile{cdocsamp.tex}[2018/12/30 v2.0 sample for childdoc]
%<*driver>
%\ProvidesFile{childdoc.drv}[2018/12/30 v2.0 childdoc reference manual file]
\PassOptionsToClass{10pt,a4paper}{article}
\documentclass{ltxdoc}

\usepackage[margin=35mm]{geometry}
\usepackage{hyperref}
\usepackage{hyperxmp}
\usepackage[usenames]{color}

\hypersetup{colorlinks=true}
\hypersetup{pdfstartview=FitH}
\hypersetup{pdfpagemode=UseNone}
\hypersetup{pdfsource={}}
\hypersetup{pdflang={en-UK}}
\hypersetup{pdfcopyright={Copyright 2017-2018 Niklas Beisert.
  This work may be distributed and/or modified under the
  conditions of the LaTeX Project Public License, either version 1.3
  of this license or (at your option) any later version.}}
\hypersetup{pdflicenseurl={http://www.latex-project.org/lppl.txt}}
\hypersetup{pdfcontactaddress={ETH Zurich, ITP, HIT K,
  Wolfgang-Pauli-Strasse 27}}
\hypersetup{pdfcontactpostcode={8093}}
\hypersetup{pdfcontactcity={Zurich}}
\hypersetup{pdfcontactcountry={Switzerland}}
\hypersetup{pdfcontactemail={nbeisert@itp.phys.ethz.ch}}
\hypersetup{pdfcontacturl={http://people.phys.ethz.ch/\xmptilde nbeisert/}}

\newcommand{\secref}[1]{\hyperref[#1]{section \ref*{#1}}}

\parskip1ex
\parindent0pt
\let\olditemize\itemize
\def\itemize{\olditemize\parskip0pt}

\begin{document}

\title{The \textsf{childdoc} Package}
\hypersetup{pdftitle={The childdoc Package}}
\author{Niklas Beisert\\[2ex]
  Institut f\"ur Theoretische Physik\\
  Eidgen\"ossische Technische Hochschule Z\"urich\\
  Wolfgang-Pauli-Strasse 27, 8093 Z\"urich, Switzerland\\[1ex]
  \href{mailto:nbeisert@itp.phys.ethz.ch}
  {\texttt{nbeisert@itp.phys.ethz.ch}}}
\hypersetup{pdfauthor={Niklas Beisert}}
\hypersetup{pdfsubject={Manual for the LaTeX2e Package childdoc}}
\date{30 December 2018, \textsf{v2.0}}
\maketitle

\begin{abstract}\noindent
\textsf{childdoc} is a \LaTeXe{} package
that enables the direct compilation
of document sections included by |\include|
to individual files.
\end{abstract}

\begingroup
\parskip0ex
\tableofcontents
\endgroup

%%%%%%%%%%%%%%%%%%%%%%%%%%%%%%%%%%%%%%%%%%%%%%%%%%%%%%%%%%%%%%%%%%%%%%%%%%%%%%%%
%%%%%%%%%%%%%%%%%%%%%%%%%%%%%%%%%%%%%%%%%%%%%%%%%%%%%%%%%%%%%%%%%%%%%%%%%%%%%%%%
\section{Introduction}

\LaTeX{} provides a mechanism to structure a large document (such as a book)
into a main file and several child files (containing the chapters)
using the |\include| command.
This mechanism is beneficial for documents
which span hundreds of pages in order to
make the source file(s) more manageable.
Moreover, compilation can be restricted to
selected child files by means of the |\includeonly| command.
The latter feature can be used to reduce the compilation time while editing
(this was significantly more useful in the earlier days of \LaTeX{})
or to generate a smaller document which is easier to navigate.
Another application of |\includeonly| is to generate
documents consisting of selected parts of the complete document.

However, there are a few drawbacks of the plain |\include| mechanism:
\begin{itemize}
\item
The child files cannot be compiled on their own,
they can only be compiled via the main file.
A naive editing environment
(such as a text editor with an option
to have the current file processed by \LaTeX)
may require one to switch to the main file before compiling;
attempting to compile the child file produces errors.
\item
The main file must be modified (each time)
to adjust the |\includeonly| command
to the present needs. This easily leaves the main file in a messy state.
\item
The generated document will always carry the filename
of the main document. This is inconvenient if
several child files are to be compiled and
to be kept for distribution.
\end{itemize}

The present package provides a simple interface
to make child files individually compilable by \LaTeX{}.
Compiling a child file then has the same effect as compiling
the main file with an |\includeonly| command
to select the appropriate child.
Moreover the generated document will carry the name of the child
rather than the main file.
This resolves all three above issues.

This feature is meant to make the editing of books,
thesis documents and lecture notes somewhat more convenient.
However, the package can also be used efficiently for
composing a series of documents (such as exercise sheets)
which are typically distributed individually.
It then assists the author in generating the individual documents
(potentially in different versions)
as well as a document containing the collected series.
Another application is in developing style files
or other kinds of included material
where compilation of the style file could redirect
to a sample or test file.

%%%%%%%%%%%%%%%%%%%%%%%%%%%%%%%%%%%%%%%%%%%%%%%%%%%%%%%%%%%%%%%%%%%%%%%%%%%%%%%%
%%%%%%%%%%%%%%%%%%%%%%%%%%%%%%%%%%%%%%%%%%%%%%%%%%%%%%%%%%%%%%%%%%%%%%%%%%%%%%%%
\section{Usage}

First of all, the package \textsf{childdoc} is \emph{not} a standard
\LaTeXe{} |.sty| style file! Therefore it needs to be invoked in
a non-standard way.

%%%%%%%%%%%%%%%%%%%%%%%%%%%%%%%%%%%%%%%%%%%%%%%%%%%%%%%%%%%%%%%%%%%%%%%%%%%%%%%%
\subsection{Included Files}
\label{sec:include}

%%%%%%%%%%%%%%%%%%%%%%%%%%%%%%%%%%%%%%%%
\DescribeMacro{\childdocmain}
To use the package, add the commands
\begin{center}
\begin{tabular}{l}
|\input{childdoc.def}|\\
|\childdocmain{}|\\
\end{tabular}
\end{center}
at the very top of the main \LaTeX{} file,
in particular \emph{before} the |\documentclass| statement!
The argument of |\childdocmain| should be left empty
(but it must be present).

%%%%%%%%%%%%%%%%%%%%%%%%%%%%%%%%%%%%%%%%
\DescribeMacro{\childdocof}
Furthermore, add the commands
\begin{center}
\begin{tabular}{l}
|\input{childdoc.def}|\\
|\childdocof{|\textit{main}|}|\\
\end{tabular}
\end{center}
at the top of every child file \textit{child}
which is included by |\include{|\textit{child}|}|
from within the main file
(or at least for those files to be compiled individually).
The argument \textit{main} must be the filename of the main file.

There are a couple of
considerations in setting up the main and child documents:

%%%%%%%%%%%%%%%%%%%%%%%%%%%%%%%%%%%%%%%%
\paragraph{Restrictions.}

Please note the following restrictions:
\begin{itemize}
\item
|\childdocmain| must be called with one argument \textit{main}
to ensure compatibility with earlier version of the package.
It must either be empty (|\childdocmain{}|)
or precisely match the filename of the main file in which it is specified.
See \secref{sec:detection} for further information.
\item
The filename \textit{main} must be specified without the |.tex| extension.
\item
The filename \textit{main} is case sensitive
(even in case-insensitive file systems)
due to internal string comparison.
\item
The argument \textit{main} should be fully expanded, it cannot be a macro.
\item
Subdirectories and special characters should be avoided in filenames.
\item
The command |\childdocmain{|\textit{main}|}| must be followed by a whitespace.
It should not be followed immediately by another command
or by a comment mark `|%|'.
This is because the \TeX{} parser reads the token immediately following
the argument of |\childdocmain| and puts it
at the beginning of every child section;
however, a white\-space is ignored.
\end{itemize}

%%%%%%%%%%%%%%%%%%%%%%%%%%%%%%%%%%%%%%%%
\paragraph{Content of Main File.}

It is advisable to place all content in the child files included by |\include|.
Any output contained in the main file will appear in all child documents
unless suppressed manually;
it cannot be suppressed automatically by the |\includeonly| directive
and thus should normally be avoided.
A method to include some content in the main file
by means of conditional processing is described in \secref{sec:conditional}.

%%%%%%%%%%%%%%%%%%%%%%%%%%%%%%%%%%%%%%%%
\paragraph{Page Numbering.}

When only a part of the document is compiled,
the appropriate numbering of pages
(as well as other status parameters)
is determined from the |.aux| files.
The latter contain information from previous passes.
However this information needs to propagate through
all intermediate child documents.
Therefore the page numbering in child documents may well
be inconsistent until the complete document is compiled at least once.

A useful (if unconventional) way to always ensure a consistent
page numbering is to restart the numbering in each child document
and denote the pages by `\textit{child}|.|\textit{page}'
where \textit{child} represents the chapter/section number of the child file.
This can be achieved by the command
|\numberwithin{page}{|\textit{child}|}|
of the \textsf{amsmath} package
where \textit{child} can be |chapter| or |section|
depending on the chosen structuring.
Alternatively, one can modify the macro |\thepage| appropriately
and reset the counter |page| at the start of each child file.

%%%%%%%%%%%%%%%%%%%%%%%%%%%%%%%%%%%%%%%%%%%%%%%%%%%%%%%%%%%%%%%%%%%%%%%%%%%%%%%%
\subsection{Conditional Processing}
\label{sec:conditional}

The package provides a mechanism to compile different versions
of a document. To customise the versions further some conditional processing
can come in handy to distinguish which version is being compiled.
The package provides two macros to describe the compilation context:

%%%%%%%%%%%%%%%%%%%%%%%%%%%%%%%%%%%%%%%%
\DescribeMacro{\ifchilddoc}
The conditional |\ifchilddoc| distinguishes between the compilation of
child documents and the main document:
%
\begin{center}
|\ifchilddoc |\textit{child-code}| |[|\||else |\textit{main-code}]| \||fi|
\end{center}

%%%%%%%%%%%%%%%%%%%%%%%%%%%%%%%%%%%%%%%%
\DescribeMacro{\childdocname}
\DescribeMacro{\childdocjob}
The macro |\childdocname| contains the filename (without extension)
of the main or child file being processed.
Note that |\childdocjob| will always contain the name of the main file.

%%%%%%%%%%%%%%%%%%%%%%%%%%%%%%%%%%%%%%%%
\paragraph{Title Page.}

Conditional processing can be used to include a title or banner page
in the main document when proper precautions are taken.
Importantly, the code in the main file should ensure that the page counter
(as well as other status parameters which are stored in the |.aux| files)
takes the same value after the conditional processing.
Otherwise the page numbers may take divergent values
depending on which part is compiled.

For example, a title page could be declared by:
%
\begin{center}
\begin{tabular}{l}
|\ifchilddoc\||else|\\
|\addtocounter{page}{-1}|\\
\textit{code for title page}\\
|\newpage|\\
|\||fi|
\end{tabular}
\end{center}
%
A banner page for the child documents can be generated by:
%
\begin{center}
\begin{tabular}{l}
|\ifchilddoc|\\
|\addtocounter{page}{-1}|\\
\textit{code for banner page}\\
|\newpage|\\
|\||fi|
\end{tabular}
\end{center}
%
Here one could write a message such as:
\begin{center}
|This is the part \childdocname{} of \childdocjob{}.|
\end{center}

%%%%%%%%%%%%%%%%%%%%%%%%%%%%%%%%%%%%%%%%%%%%%%%%%%%%%%%%%%%%%%%%%%%%%%%%%%%%%%%%
\subsection{Flags}
\label{sec:flags}

The package makes it easy to generate different versions
of the main or child documents.
To this end compilation flags can be defined
and assigned different default values.
They will be particularly useful in conjunction
with the forwarding mechanism described in \secref{sec:forward}.

For example, it may be useful to have a flag |\version|
which can be set to |draft| or |final|.
The document source will contain some conditional code
depending on the value of |\version|.
Suppose further, the flag should default to |final| for the main file
and to |draft| for child files
which is a natural assignment for editing the document.
This is achieved by placing the following code
in the preamble of the main document
(below the |\childdocmain| directive):
%
\begin{center}
\begin{tabular}{l}
|\ifchilddoc|\\
|\providecommand{\version}{draft}|\\
|\||else|\\
|\providecommand{\version}{final}|\\
|\||fi|
\end{tabular}
\end{center}
%
The definition by |\providecommand| makes sure
that previous definitions are not overwritten.
Further statements |\providecommand{\version}{...}|
can thus be added before the above code to override it.

For the main file, one might add a line
(between |\childdocmain| and the above block)
%
\begin{center}
|%\ifchilddoc\||else\providecommand{\version}{draft}\||fi|
\end{center}
%
which can be uncommented to produce a draft version.
Likewise one can add a line to the very top of a child file
(above the |\childdocof{|\textit{main}|}| directive)
%
\begin{center}
|%\providecommand{\version}{final}|
\end{center}
%
which can be uncommented to produce the final version of this child document.

%%%%%%%%%%%%%%%%%%%%%%%%%%%%%%%%%%%%%%%%%%%%%%%%%%%%%%%%%%%%%%%%%%%%%%%%%%%%%%%%
\subsection{Forwarding}
\label{sec:forward}

Different versions of the main or child documents
using compilation flags as described in \secref{sec:flags}
can be (permanently) stored in different files
for convenient compilation, viewing and distribution.
To this end, the package defines a command
to pass on compilation to a different file:

%%%%%%%%%%%%%%%%%%%%%%%%%%%%%%%%%%%%%%%%
\DescribeMacro{\childdocforward}
The command |\childdocforward| redirects processing to
another source file:
%
\begin{center}
\begin{tabular}{l}
|\input{childdoc.def}|\\
|\childdocforward[|\textit{main}|]{|\textit{dest}|}|\\
\end{tabular}
\end{center}
%
The argument \textit{dest} is the destination file
(without extension).
It should be the main file or one of the child files.
Note that further \textsf{childdoc} directives
such as |\childdocof| and |\childdocforward|
in the indicated file will be processed in this form.
The optional argument \textit{main}
passes on directly to the main file \textit{main}
while pretending to compile the child \textit{dest}.
This form behaves as if \textit{dest}
issues |\childdocof{|\textit{main}|}| right away,
and no further \textsf{childdoc} directives will be processed.

%%%%%%%%%%%%%%%%%%%%%%%%%%%%%%%%%%%%%%%%
\DescribeMacro{\...prefix}
In the alternative form |\childdocforwardprefix|,
%
\begin{center}
\begin{tabular}{l}
|\input{childdoc.def}|\\
|\childdocforwardprefix[|\textit{main}|]{|\textit{prefix}|}{|\textit{dest}|}|
\end{tabular}
\end{center}
%
the destination file is determined by a pattern
depending on the current file:
To make this work, the current file must be called
`{\textit{prefix}\hspace{0.2em}\textit{suffix}}'
with \textit{prefix} matching precisely the argument.
Processing is then passed on to the file
`{\textit{dest}\hspace{0.2em}\textit{suffix}}'.
Surely, the same effect is achieved by
directly specifying the
argument `{\textit{dest}\hspace{0.2em}\textit{suffix}}'
in the first form.
However, that requires to set up a different file
for each child. With the alternative form of the command
all these files can have exactly the same content
which simplifies setting them up and maintaining them.

For example, the following file |draft.tex|
with a compilation flag |\version| as described in \secref{sec:flags}
compiles the main document as a draft:
%
\begin{center}
\begin{tabular}{l}
|\def\version{draft}|\\
|\input{childdoc.def}|\\
|\childdocforward{|\textit{main}|}|
\end{tabular}
\end{center}
%
Likewise, the following files |final|\textit{nn}|.tex|
compile the final version of the child document
|child|\textit{nn}|.tex|:
%
\begin{center}
\begin{tabular}{l}
|\def\version{final}|\\
|\input{childdoc.def}|\\
|\childdocforwardprefix{final}{child}|
\end{tabular}
\end{center}
%

Note that when several versions of a main file and/or of each child file
are to be generated, it may be convenient to set up a |Makefile| or
shell script to automatise the process.

%%%%%%%%%%%%%%%%%%%%%%%%%%%%%%%%%%%%%%%%%%%%%%%%%%%%%%%%%%%%%%%%%%%%%%%%%%%%%%%%
\subsection{Command Line Processing}
\label{sec:commandline}

The effect of redirection files can also be achieved by invoking
the \LaTeX{} compiler with a more elaborate command line.
Most conveniently this should be done as part
of a shell script or a |Makefile|.

When using \textsf{childdoc} in the main file, the following
command lines effectively perform a redirection
(note that depending on the shell being used,
backslashes may have to be doubled: `|\|' $\to$ `|\\|'):
%
\begin{center}
|... -jobname "|\textit{target}|" |\\|"|[\textit{flags}]%
|\input{childdoc.def}\childdocforward[|\textit{main}|]{|\textit{dest}|}"|
\end{center}
%
Here \textit{target} is the name of the output file,
\textit{main} is the name of the main file
and \textit{dest} is the name of the main or child file to be processed
(all filenames without extensions).
The optional argument \textit{main} can be omitted
if \textit{main} matches \textit{dest}.
Optionally, compilation \textit{flags} can be defined via |\def| commands.
This command line makes the \TeX{} engine believe
it is compiling the file \textit{target}
whose content is specified as the latter parameter.
The provided code then forwards the processing to
\textit{main} or \textit{dest} as described in \secref{sec:forward}.

%%%%%%%%%%%%%%%%%%%%%%%%%%%%%%%%%%%%%%%%%%%%%%%%%%%%%%%%%%%%%%%%%%%%%%%%%%%%%%%%
\subsection{Include by Input}
\label{sec:input}

Including child documents by |\include| has some restrictions by design.
Most notably, the content of a child document always occupies
its own set of pages; pages cannot be shared between child documents.
Usually, this behaviour makes perfect sense
because each child document contain an essential part of the document.
However, in some situations it may be desirable to compose
a document from a collection of parts
without having mandatory page breaks between then.
For this case, the package
provides a mechanism to include parts
by |\input| which can also be processed individually.
However, by construction this mechanism
requires manual handling of the content to be output.

%%%%%%%%%%%%%%%%%%%%%%%%%%%%%%%%%%%%%%%%
\DescribeMacro{\ifchilddocmanual}
The main file should be prepared as usual, see \secref{sec:include}.
However, the document body must make a distinction
between processing of an individual part and of the main document, e.g.:
%
\begin{center}
\begin{tabular}{l}
|\ifchilddocmanual|\\
|\input{\childdocname}|\\
|\||else|\\
\textit{document body with }|\input{|\textit{part}|}|\\
|\||fi|
\end{tabular}
\end{center}
%
The conditional |\ifchilddocmanual| is true whenever
a part to be included by |\input| is being compiled,
and the name of the part is stored in |\childdocname|.

%%%%%%%%%%%%%%%%%%%%%%%%%%%%%%%%%%%%%%%%
\DescribeMacro{\childdocby}
Each part to be included by |\input| should start with:
%
\begin{center}
\begin{tabular}{l}
|\input{childdoc.def}|\\
|\childdocby{|\textit{main}|}|\\
\end{tabular}
\end{center}
%
The directive |\childdocby| is similar to |\childdocof|
described in \secref{sec:include},
but the subsequent selection of content must be done manually.
To that end, both |\ifchilddoc| and |\ifchilddocmanual|
will be true upon processing of a part,
and the name of the part is stored in |\childdocname|.
Note that |\jobname| will be set to the filename of the current part
so that each part receives an individual |.aux| file
that does not interfere with the |.aux| file(s) of the main document.
This behaviour can be altered by the alternative form
|\childdocby[*]{|\textit{main}|}| (with a non-empty optional argument)
which uses the |.aux| file of the main document
by setting |\jobname| to \textit{main}.

%%%%%%%%%%%%%%%%%%%%%%%%%%%%%%%%%%%%%%%%%%%%%%%%%%%%%%%%%%%%%%%%%%%%%%%%%%%%%%%%
\subsection{Driver Development}
\label{sec:driver}

The \textsf{childdoc} mechanism can also be use for the development
of definition files such as \LaTeX{} styles or classes.
This case differs from the above setup with multiple parts
included by |\include| in that no |\includeonly| should be invoked.
This can be achieved by starting the include file
(before |\ProvidesPackage|) with:
%
\begin{center}
\begin{tabular}{l}
|\input{childdoc.def}|\\
|\childdocforward{|\textit{main}|}|\\
\end{tabular}
\end{center}
%
or alternatively with:
%
\begin{center}
\begin{tabular}{l}
|\input{childdoc.def}|\\
|\childdocby{|\textit{main}|}|\\
\end{tabular}
\end{center}
%
Both forms have slightly different effects as described above.
The main file is prepared as usual, see \secref{sec:include}.

%%%%%%%%%%%%%%%%%%%%%%%%%%%%%%%%%%%%%%%%%%%%%%%%%%%%%%%%%%%%%%%%%%%%%%%%%%%%%%%%
\subsection{Legacy Detection}
\label{sec:detection}

The directive |\childdocmain| in the main file can detect
whether the complete document or merely a child is to be compiled
even without using the directive |\childdocof|.
This method is deprecated because it is less robust
and there is no compelling reason to use it;
it is merely provided for backward compatibility
and it may be removed in future versions.

If the detection mechanism is to be used,
it is mandatory to correctly specify
the filename of the main file as the argument of |\childdocmain|:
%
\begin{center}
\begin{tabular}{l}
|\input{childdoc.def}|\\
|\childdocmain{|\textit{main}|}|\\
\end{tabular}
\end{center}
%
If |\jobname| does not match the argument \textit{main} of |\childdocmain|,
it is assumed that |\jobname| points to the child file to be compiled.
When using |\childdocmain| with the main file specified as argument,
it suffices to start a child file
with just |\input{|\textit{main}|}|
without loading of the package and using |\childdocof|.
If instead all processing is done
with the appropriate \textsf{childdoc} directives,
the argument of \textit{main} of |\childdocmain| can be empty.

An alternative version of the command line processing described
in \secref{sec:commandline} using the detection mechanism reads:
%
\begin{center}
|... -jobname "|\textit{target}|" "|[\textit{flags}]%
[|\def\jobname{|\textit{dest}|}|]|\input{|\textit{main}|}"|
\end{center}

%%%%%%%%%%%%%%%%%%%%%%%%%%%%%%%%%%%%%%%%%%%%%%%%%%%%%%%%%%%%%%%%%%%%%%%%%%%%%%%%
\subsection{Manual Code}
\label{sec:manual}

In case one cannot be certain whether the definitions file |childdoc.def|
is installed on the target \TeX{} distribution
and one prefers not to ship it,
it is conceivable to paste a few relevant commands into the sources.

To that end, drop all statements |\input{childdoc.def}|
and perform the replacements as outlined below.
Instead of |\childdocmain{|\textit{main}|}| add the following code
to the top of the main file:
%
\begin{center}
\begin{tabular}{l}
|\||ifdefined\childdocname\endinput\||fi\newif\ifchilddoc|\\
|\edef\childdocname{\scantokens\expandafter{\jobname\noexpand}}|\\
|\def\childdocmain{|\textit{main}|}\||ifx\childdocmain\childdocname\||else|\\
|\childdoctrue\includeonly{\childdocname}\let\jobname\childdocmain\||fi|\\
\end{tabular}
\end{center}
%
Instead of |\childdocof{|\textit{main}|}| just include the main file
at the top of each child file:
%
\begin{center}
|\input{|\textit{main}|}|
\end{center}
%
A simple redirection |\childdocforward{|\textit{dest}|}| is achieved by:
%
\begin{center}
|\def\jobname{|\textit{dest}|}\input{\jobname}|
\end{center}
%
The redirection with prefix
|\childdocforwardprefix[|\textit{prefix}|]{|\textit{dest}|}|
is accomplished by:
%
\begin{center}
\begin{tabular}{l}
|{\edef\jobname{\scantokens\expandafter{\jobname\noexpand}}|\\
|\def\redirectjob |\textit{prefix}|#1~~~{\gdef\jobname{|\textit{dest}|#1}}|\\
|\expandafter\redirectjob\jobname~~~}\input{\jobname}|
\end{tabular}
\end{center}

In an alternative approach,
child documents can be compiled by a specific command line
without additional code or specific definitions:
%
\begin{center}
|... -jobname "|\textit{target}|" "|[\textit{flags}]%
|\includeonly{|\textit{dest}|}\input{|\textit{main}|}"|
\end{center}
%

%%%%%%%%%%%%%%%%%%%%%%%%%%%%%%%%%%%%%%%%%%%%%%%%%%%%%%%%%%%%%%%%%%%%%%%%%%%%%%%%
%%%%%%%%%%%%%%%%%%%%%%%%%%%%%%%%%%%%%%%%%%%%%%%%%%%%%%%%%%%%%%%%%%%%%%%%%%%%%%%%
\section{Information}

%%%%%%%%%%%%%%%%%%%%%%%%%%%%%%%%%%%%%%%%%%%%%%%%%%%%%%%%%%%%%%%%%%%%%%%%%%%%%%%%
\subsection{Copyright}

Copyright \copyright{} 2017--2018 Niklas Beisert

This work may be distributed and/or modified under the
conditions of the \LaTeX{} Project Public License, either version 1.3
of this license or (at your option) any later version.
The latest version of this license is in
  \url{http://www.latex-project.org/lppl.txt}
and version 1.3 or later is part of all distributions of \LaTeX{}
version 2005/12/01 or later.

This work has the LPPL maintenance status `maintained'.

The Current Maintainer of this work is Niklas Beisert.

This work consists of the files |README.txt|, |childdoc.ins| and |childdoc.dtx|
as well as the derived files |childdoc.def|, |cdocsamp.tex|
with |cdocsch1.tex|, |cdocsch2.tex|, |cdocspt3.tex|, |cdocspt4.tex|,
|cdocsdrf.tex|, |cdocsfn1.tex|, |cdocsfn2.tex|
as well as |childdoc.pdf|.

%%%%%%%%%%%%%%%%%%%%%%%%%%%%%%%%%%%%%%%%%%%%%%%%%%%%%%%%%%%%%%%%%%%%%%%%%%%%%%%%
\subsection{Files and Installation}

The package consists of the files:
%
\begin{center}
\begin{tabular}{ll}
    |README.txt|   & readme file \\
    |childdoc.ins| & installation file \\
    |childdoc.dtx| & source file \\
    |childdoc.def| & definition file \\
    |cdocsamp.tex| & sample main file \\
    |cdocsch1.tex| & sample include file \\
    |cdocsch2.tex| & sample include file \\
    |cdocspt3.tex| & sample part file \\
    |cdocspt4.tex| & sample part file \\
    |cdocsdrf.tex| & sample redirection file \\
    |cdocsfn1.tex| & sample redirection file \\
    |cdocsfn2.tex| & sample redirection file \\
    |childdoc.pdf| & manual
\end{tabular}
\end{center}
%
The distribution consists of the files
|README.txt|, |childdoc.ins| and |childdoc.dtx|.
%
\begin{itemize}
\item
Run (pdf)\LaTeX{} on |childdoc.dtx|
to compile the manual |childdoc.pdf| (this file).
\item
Run \LaTeX{} on |childdoc.ins| to create the definitions file |childdoc.def|
and the sample |cdocsamp.tex| with include files
|cdocsch1.tex|, |cdocsch2.tex|, |cdocspt3.tex|, |cdocspt4.tex|,
|cdocsdrf.tex|, |cdocsfn1.tex|, |cdocsfn2.tex|.
Then copy the file |childdoc.def| to an appropriate directory of your \LaTeX{}
distribution, e.g.\ \textit{texmf-root}|/tex/latex/childdoc|.
\end{itemize}

%%%%%%%%%%%%%%%%%%%%%%%%%%%%%%%%%%%%%%%%%%%%%%%%%%%%%%%%%%%%%%%%%%%%%%%%%%%%%%%%
\subsection{Related CTAN Packages}

There are several other packages which offer a similar functionality:
%
\begin{itemize}
\item
The packages
\href{http://ctan.org/pkg/docmute}{\textsf{docmute}},
\href{http://ctan.org/pkg/includex}{\textsf{includex}} and
\href{http://ctan.org/pkg/standalone}{\textsf{standalone}}
provide commands to include only the document body of
a child file thus allowing both files to be compiled individually.
\item
The packages \href{http://ctan.org/pkg/subdocs}{\textsf{subdocs}}
and \href{http://ctan.org/pkg/subfiles}{\textsf{subfiles}}
provide structures in which the main and child documents can be
encapsulated and allowing them to be compiled individually.
The inclusion mechanism is different from the conventional |\include|.
\item
The package \href{http://ctan.org/pkg/combine}{\textsf{combine}}
is an elaborate solution to combine several documents into one.
\end{itemize}
%
See also the CTAN topic \href{http://ctan.org/topic/subdocs}{\textsf{subdocs}}
for further related packages.
The present package differs from the above solutions in that
a document structure constructed with the conventional |\include| mechanism
just needs two extra commands at the top of every file
such that all constituent files can be compiled individually.

%%%%%%%%%%%%%%%%%%%%%%%%%%%%%%%%%%%%%%%%%%%%%%%%%%%%%%%%%%%%%%%%%%%%%%%%%%%%%%%%
%\subsection{Feature Suggestions}
%
%The following is a list of features which may be useful for future
%versions of this package:
%%
%\begin{itemize}
%\item
%\ldots
%\end{itemize}

%%%%%%%%%%%%%%%%%%%%%%%%%%%%%%%%%%%%%%%%%%%%%%%%%%%%%%%%%%%%%%%%%%%%%%%%%%%%%%%%
\subsection{Revision History}

%%%%%%%%%%%%%%%%%%%%%%%%%%%%%%%%%%%%%%%%
\paragraph{v2.0:} 2018/12/30

\begin{itemize}
\item
immediate forward processing
\item
added |\childdocby| mechanism
\item
manual restructured
\end{itemize}

%%%%%%%%%%%%%%%%%%%%%%%%%%%%%%%%%%%%%%%%
\paragraph{v1.6:} 2018/01/17

\begin{itemize}
\item
application for development of include files
\item
corrections to manual
\end{itemize}

%%%%%%%%%%%%%%%%%%%%%%%%%%%%%%%%%%%%%%%%
\paragraph{v1.5:} 2017/05/21

\begin{itemize}
\item
more complete structuring introduced
\item
|\childdocof| introduced
\item
|\childdoc| renamed to |\childdocmain|
\item
|\childredirect| renamed to |\childdocforward| and |\childdocforwardprefix|
and functionality expanded
\end{itemize}

%%%%%%%%%%%%%%%%%%%%%%%%%%%%%%%%%%%%%%%%
\paragraph{v1.0:} 2017/04/27

\begin{itemize}
\item
manual and install package
\item
first version published on CTAN
\end{itemize}

%%%%%%%%%%%%%%%%%%%%%%%%%%%%%%%%%%%%%%%%
\paragraph{v0.6:} 2017/04/26

\begin{itemize}
\item
redirection mechanism added
\end{itemize}

%%%%%%%%%%%%%%%%%%%%%%%%%%%%%%%%%%%%%%%%
\paragraph{v0.5:} 2017/04/26

\begin{itemize}
\item
functionality in definition file
\end{itemize}


%%%%%%%%%%%%%%%%%%%%%%%%%%%%%%%%%%%%%%%%%%%%%%%%%%%%%%%%%%%%%%%%%%%%%%%%%%%%%%%%
%%%%%%%%%%%%%%%%%%%%%%%%%%%%%%%%%%%%%%%%%%%%%%%%%%%%%%%%%%%%%%%%%%%%%%%%%%%%%%%%
%%%%%%%%%%%%%%%%%%%%%%%%%%%%%%%%%%%%%%%%%%%%%%%%%%%%%%%%%%%%%%%%%%%%%%%%%%%%%%%%
\appendix

\settowidth\MacroIndent{\rmfamily\scriptsize 000\ }

 \DocInput{childdoc.dtx}

\end{document}
%</driver>
% \fi
%
% %%%%%%%%%%%%%%%%%%%%%%%%%%%%%%%%%%%%%%%%%%%%%%%%%%%%%%%%%%%%%%%%%%%%%%%%%%%%%%
% %%%%%%%%%%%%%%%%%%%%%%%%%%%%%%%%%%%%%%%%%%%%%%%%%%%%%%%%%%%%%%%%%%%%%%%%%%%%%%
% \section{Sample}
%\iffalse
%<*samplemain>
%\fi
%
% The following presents a sample document
% with two chapters, two parts, a title page,
% a compile flag as well as three forwarding files to set the flag.
% It consists of eight |.tex| files:
% \begin{center}
% \begin{tabular}{ll}
% |cdocsamp.tex|&main file\\
% |cdocsch1.tex|&include file for chapter 1\\
% |cdocsch2.tex|&include file for chapter 2\\
% |cdocspt3.tex|&include file for part 3\\
% |cdocspt4.tex|&include file for part 4\\
% |cdocsdrf.tex|&forwarding file for main file in draft mode\\
% |cdocsfi1.tex|&forwarding file for final version of chapter 1\\
% |cdocsfi2.tex|&forwarding file for final version of chapter 2\\
% \end{tabular}
% \end{center}
% Each of the eight files can be compiled directly by the \LaTeX{} compiler.
%
% %%%%%%%%%%%%%%%%%%%%%%%%%%%%%%%%%%%%%%
% \paragraph{Main File.}
%
% The main file is called |cdocsamp.tex|.
%
% Load the \textsf{childdoc} definitions and
% declare the filename for the main document:
%    \begin{macrocode}
\input{childdoc.def}
\childdocmain{}
%    \end{macrocode}

% Optional override for |\version| flag:
%    \begin{macrocode}
%%\ifchilddoc\else\providecommand{\version}{draft}\fi
%    \end{macrocode}

% Define the default values for the |\version| flag
% (|final| for the main file and |draft| for childs):
%    \begin{macrocode}
\ifchilddoc
\providecommand{\version}{draft}
\else
\providecommand{\version}{final}
\fi
%    \end{macrocode}

% Load the standard document class:
%    \begin{macrocode}
\documentclass[12pt]{article}
%    \end{macrocode}

% Start the document body:
%    \begin{macrocode}
\begin{document}
%    \end{macrocode}

% Declare a title page.
% Print title, part of document being processed and version flag:
%    \begin{macrocode}
\addtocounter{page}{-1}
\begin{center}
{\LARGE\bfseries{}childdoc example\par}
\vspace{1cm}
\ifchilddoc
\ifchilddocmanual part\else chapter\fi:
`\childdocname' of `\childdocjob'\par
\else
main document: `\childdocjob'\par
\fi
version: \version\par
\end{center}
\newpage
%    \end{macrocode}

% Manually include selected file,
% otherwise process as usual:
%    \begin{macrocode}
\ifchilddocmanual
\section*{part `\childdocname'}
\input{\childdocname}
\else
%    \end{macrocode}

% Include the two chapters:
%    \begin{macrocode}
\include{cdocsch1}
\include{cdocsch2}
%    \end{macrocode}

% Include the two parts unless only chapters should be displayed:
%    \begin{macrocode}
\ifchilddoc\else
\section{part three}
\input{cdocspt3}
\section{part four}
\input{cdocspt4}
\fi
%    \end{macrocode}

% Process as usual until here:
%    \begin{macrocode}
\fi
%    \end{macrocode}

% End of document body:
%    \begin{macrocode}
\end{document}
%    \end{macrocode}
%\iffalse
%</samplemain>
%\fi
%
% %%%%%%%%%%%%%%%%%%%%%%%%%%%%%%%%%%%%%%
% \paragraph{Chapter Include Files.}
%
% The include files are called |cdocsch1.tex| and |cdocsch2.tex|.
%
%\iffalse
%<*samplechap1|samplechap2>
%\fi

% Optional override for |\version| flag:
%    \begin{macrocode}
%%\providecommand{\version}{final}
%    \end{macrocode}

% Include the main document:
%    \begin{macrocode}
\input{childdoc.def}
\childdocof{cdocsamp}
%    \end{macrocode}

%\iffalse
%</samplechap1|samplechap2>
%\fi
%
%\iffalse
%<*samplechap1>
%\fi
% Some text for chapter 1:
%    \begin{macrocode}
\section{one}
some text in chapter one
%    \end{macrocode}

%\iffalse
%</samplechap1>
%\fi
% Some text for chapter 2:
%\iffalse
%<*samplechap2>
%\fi
%    \begin{macrocode}
\section{two}
more text in chapter two
%    \end{macrocode}

%\iffalse
%</samplechap2>
%\fi
%
% %%%%%%%%%%%%%%%%%%%%%%%%%%%%%%%%%%%%%%
% \paragraph{Part Include Files.}
%
% The include files are called |cdocspt3.tex| and |cdocspt4.tex|.
%
%\iffalse
%<*samplepart3|samplepart4>
%\fi

% Optional override for |\version| flag:
%    \begin{macrocode}
%%\providecommand{\version}{final}
%    \end{macrocode}

% Include the main document:
%    \begin{macrocode}
\input{childdoc.def}
\childdocby{cdocsamp}
%    \end{macrocode}

%\iffalse
%</samplepart3|samplepart4>
%\fi
%
%\iffalse
%<*samplepart3>
%\fi
% Some text for part 3:
%    \begin{macrocode}
some text in part three
%    \end{macrocode}

%\iffalse
%</samplepart3>
%\fi
% Some text for part 4:
%\iffalse
%<*samplepart4>
%\fi
%    \begin{macrocode}
more text in part four
%    \end{macrocode}

%\iffalse
%</samplepart4>
%\fi
%
% %%%%%%%%%%%%%%%%%%%%%%%%%%%%%%%%%%%%%%
% \paragraph{Forwarding for a Complete Draft.}
%
% The following forwarding file |cdocsdrf.tex|
% compiles the main document in draft mode:
%\iffalse
%<*sampledraft>
%\fi
%    \begin{macrocode}
\def\version{draft}
\input{childdoc.def}
\childdocforward{cdocsamp}
%    \end{macrocode}

%\iffalse
%</sampledraft>
%\fi
%
% %%%%%%%%%%%%%%%%%%%%%%%%%%%%%%%%%%%%%%
% \paragraph{Forwarding for Final Version of the Chapters.}
%
% The following forwarding files |cdocsfn1.tex| and |cdocsfn2.tex|
% (with identical content)
% compile the final versions of the child documents
% |cdocsch1.tex| and |cdocsch2.tex|, respectively:
%\iffalse
%<*samplefinal>
%\fi
%    \begin{macrocode}
\def\version{final}
\input{childdoc.def}
\childdocforwardprefix[cdocsamp]{cdocsfn}{cdocsch}
%    \end{macrocode}

%\iffalse
%</samplefinal>
%\fi
%
% %%%%%%%%%%%%%%%%%%%%%%%%%%%%%%%%%%%%%%
% \paragraph{Command Line Processing.}
%
% The following three command lines generate the output files
% |cdocscld|, |cdocscl1| and |cdocscl2|
% which should be identical to
% |cdocsdrf|, |cdocsch1| and |cdocsfn2|, respectively:
% \begin{center}
% \begin{tabular}{l}
% |latex -jobname cdocscld \|\\
% |  "\def\version{draft}\input{childdoc.def}\childdocforward{cdocsamp}"|\\
% |latex -jobname cdocscl1 \|\\
% |  "\input{childdoc.def}\childdocforward[cdocsamp]{cdocsch1}"|\\
% |latex -jobname cdocscl2 \|\\
% |  "\def\version{final}\input{childdoc.def}\childdocforward{cdocsch2}"|
% \end{tabular}
% \end{center}
% Note that the trailing backslash on each first line
% merely continues the input to the second line
% (for convenient cut ant paste).
% Furthermore, the command |latex| can be replaced by any
% of its alternative versions such as |pdflatex|.
%
% %%%%%%%%%%%%%%%%%%%%%%%%%%%%%%%%%%%%%%%%%%%%%%%%%%%%%%%%%%%%%%%%%%%%%%%%%%%%%%
% %%%%%%%%%%%%%%%%%%%%%%%%%%%%%%%%%%%%%%%%%%%%%%%%%%%%%%%%%%%%%%%%%%%%%%%%%%%%%%
% \section{Implementation}
%\iffalse
%<*package>
%\fi
%
% This section describes the definitions file |childdoc.def|.

% The definitions cannot be loaded using |\usepackage| or |\RequirePackage|
% which has a mechanism to prevent loading a style file more than once.
% When loading the definitions by means of |\input|
% multiple instances have to be prevented manually:
%\iffalse
%This code needs to be before the `\ProvidesFile' directive
%which is defined at the beginning of this file.
%Therefore it is also placed there and commented out here.
%</package>
%<*discard>
%\fi
%    \begin{macrocode}
\ifdefined\childdocmain\endinput\fi
%    \end{macrocode}
%\iffalse
%</discard>
%<*package>
%\fi
%
% \macro{\ifchilddoc}
% \macro{\ifchilddocmanual}
% The conditional |\ifchilddoc| tells whether a
% child (true) or main (false) document is being compiled.
% The conditional |\ifchilddocmanual| tells whether
% the |\includeonly| mechanism is used (false) or
% the selection of child files must be performed manually (true).
% The definitions initialise to false:
%    \begin{macrocode}
\newif\ifchilddoc
\newif\ifchilddocmanual
%    \end{macrocode}

% \macro{\childdocname}
% \macro{\childdocjob}
% The macro |\childdocname| stores the name of the main document
% to be compiled. The macro |\childdocjob| stores the name of
% the document on which the \LaTeX{} compiler was originally invoked.
% The content of |\jobname| cannot be compared
% to filenames specified in the source due to different catcodes.
% The following code rescans |\jobname|, stores the result
% in |\childdocname| and saves a copy in |\childdocjob|:
%    \begin{macrocode}
\edef\childdocname{\scantokens\expandafter{\jobname\noexpand}}
\let\childdocjob\childdocname
%    \end{macrocode}

% \macro{\childdocdisable}
% The macro |\childdocdisable| prevents the main file
% from being processed more than once.
% At this stage, the main document command |\childdocmain|
% is assumed to be called once again where it should do nothing.
% Any subsequent call to it should prevent
% a secondary processing of the main document
% It overwrites the forwarding commands
% |\childdocof| and |\childdocforward|
% with empty macros to prevent further inclusions of the main document:
%    \begin{macrocode}
\newcommand{\childdocdisable}
{
  \renewcommand{\childdocmain}[1]{\renewcommand{\childdocmain}[1]{\endinput}}
  \renewcommand{\childdocof}[1]{}
  \renewcommand{\childdocby}[2][]{}
  \renewcommand{\childdocforward}[2][]{}
  \renewcommand{\childdocdisable}{}
}
%    \end{macrocode}

% \macro{\childdocmain}
% The macro |\childdocmain| is to be called at the top of the main file
% with nothing or the main filename (without extension) as argument.
% First, it breaks loops.
% If the argument is not empty and does not match |\childdocname|
% (which is set by the first inclusion of |childdoc.def|),
% |\ifchilddoc| is set to true, |\includeonly| is applied to the child file
% and |\jobname| is set to the main file
% (for proper handling of |.aux| files):
%    \begin{macrocode}
\newcommand{\childdocmain}[1]
{
  \childdocdisable\childdocmain{}
  \if?#1?\else
    \begingroup
      \def\childdoctmp{#1}
      \ifx\childdoctmp\childdocname
        \def\childdoctmp{}
      \else
        \def\childdoctmp
        {
          \childdoctrue
          \includeonly{\childdocname}
          \def\childdocjob{#1}
          \def\jobname{#1}
        }
      \fi
      \expandafter
    \endgroup
    \childdoctmp
  \fi
}
%    \end{macrocode}

% \macro{\childdocof}
% The command |\childdocof| redirects
% compilation to the main file |#1|.
%    \begin{macrocode}
\newcommand{\childdocof}[1]
{
  \childdocdisable
  \childdoctrue
  \includeonly{\childdocname}
  \def\jobname{#1}
  \def\childdocjob{#1}
  \input{#1}
}
%    \end{macrocode}

% \macro{\childdocby}
% The command |\childdocby| ....
%    \begin{macrocode}
\newcommand{\childdocby}[2][]
{
  \childdocdisable
  \childdoctrue
  \childdocmanualtrue
  \if?#1?\else
    \def\jobname{#2}
  \fi
  \def\childdocjob{#2}
  \input{#2}
  \endinput
}
%    \end{macrocode}

% \macro{\childdocforward}
% The command |\childdocforward| redirects
% compilation to the main file or
% (if the optional argument is given) a child file.
% Parameters are set as if the main file
% or a child file starting with |\childdocof| was compiled.
% Then compilation is handed over to the main file:
%    \begin{macrocode}
\newcommand{\childdocforward}[2][]
{
  \begingroup
    \if?#1?
      \def\childdoctmp
      {
        \def\childdocname{#2}
        \def\childdocjob{#2}
        \def\jobname{#2}
        \input{#2}
        \endinput
      }
    \else
      \def\childdoctmp
      {
        \childdocdisable
        \def\childdocname{#2}
        \childdoctrue
        \includeonly{#2}
        \def\childdocjob{#1}
        \def\jobname{#1}
        \input{#1}
        \endinput
      }
    \fi
    \expandafter
  \endgroup
  \childdoctmp
}
%    \end{macrocode}

% \macro{\childdocforwardprefix}
% The command |\childdocforwardprefix| redirects
% compilation to the main or a child file by means of a pattern.
% The prefix |#1| in the current filename is replaced by |#2|
% and the suffix of the current filename is kept
% (it is assumed that the filename does not contain the substring `|~~~|'
% which is used as a delimiter).
% Compilation is handed over to the new file by |\childdocforward|:
%    \begin{macrocode}
\newcommand{\childdocforwardprefix}[3][]
{
  \begingroup
    \def\childdocextract #2##1~~~{\def\childdoctmp{\childdocforward[#1]{#3##1}}}
    \expandafter\childdocextract\childdocname~~~
    \expandafter
  \endgroup
  \childdoctmp
}
%    \end{macrocode}

% \macro{\childdoc}
% The deprecated macro |\childdoc| is a legacy version of |\childdocmain|:
%    \begin{macrocode}
\newcommand{\childdoc}{\childdocmain}
%    \end{macrocode}

% \macro{\childdocredirect}
% The deprecated macro |\childdocredirect| is a legacy version
% of |\childdocforward| and |\childdocforwardprefix|:
%    \begin{macrocode}
\newcommand{\childdocredirect}[2][]
{
  \begingroup
    \if?#1?
      \def\childdoctmp{\childdocforward{#2}}
    \else
      \def\childdoctmp{\childdocforwardprefix{#1}{#2}}
    \fi
    \expandafter
  \endgroup
  \childdoctmp
}
%    \end{macrocode}

%\iffalse
%</package>
%\fi
%
\endinput
|\\
|\childdocmain{}|\\
\end{tabular}
\end{center}
at the very top of the main \LaTeX{} file,
in particular \emph{before} the |\documentclass| statement!
The argument of |\childdocmain| should be left empty
(but it must be present).

%%%%%%%%%%%%%%%%%%%%%%%%%%%%%%%%%%%%%%%%
\DescribeMacro{\childdocof}
Furthermore, add the commands
\begin{center}
\begin{tabular}{l}
|% \iffalse
%
% childdoc.dtx Copyright (C) 2017-2018 Niklas Beisert
%
% This work may be distributed and/or modified under the
% conditions of the LaTeX Project Public License, either version 1.3
% of this license or (at your option) any later version.
% The latest version of this license is in
%   http://www.latex-project.org/lppl.txt
% and version 1.3 or later is part of all distributions of LaTeX
% version 2005/12/01 or later.
%
% This work has the LPPL maintenance status `maintained'.
%
% The Current Maintainer of this work is Niklas Beisert.
%
% This work consists of the files childdoc.dtx and childdoc.ins
% and the derived files childdoc.def and cdocsamp.tex with
% cdocsch1.tex, cdocsch2.tex, cdocsdrf.tex, cdocsfn1.tex, cdocsfn2.tex.
%
%<package>\ifdefined\childdocmain\endinput\fi
%<package>\ProvidesFile{childdoc.def}[2018/12/30 v2.0 child document driver]
%<samplemain>\ProvidesFile{cdocsamp.tex}[2018/12/30 v2.0 sample for childdoc]
%<*driver>
%\ProvidesFile{childdoc.drv}[2018/12/30 v2.0 childdoc reference manual file]
\PassOptionsToClass{10pt,a4paper}{article}
\documentclass{ltxdoc}

\usepackage[margin=35mm]{geometry}
\usepackage{hyperref}
\usepackage{hyperxmp}
\usepackage[usenames]{color}

\hypersetup{colorlinks=true}
\hypersetup{pdfstartview=FitH}
\hypersetup{pdfpagemode=UseNone}
\hypersetup{pdfsource={}}
\hypersetup{pdflang={en-UK}}
\hypersetup{pdfcopyright={Copyright 2017-2018 Niklas Beisert.
  This work may be distributed and/or modified under the
  conditions of the LaTeX Project Public License, either version 1.3
  of this license or (at your option) any later version.}}
\hypersetup{pdflicenseurl={http://www.latex-project.org/lppl.txt}}
\hypersetup{pdfcontactaddress={ETH Zurich, ITP, HIT K,
  Wolfgang-Pauli-Strasse 27}}
\hypersetup{pdfcontactpostcode={8093}}
\hypersetup{pdfcontactcity={Zurich}}
\hypersetup{pdfcontactcountry={Switzerland}}
\hypersetup{pdfcontactemail={nbeisert@itp.phys.ethz.ch}}
\hypersetup{pdfcontacturl={http://people.phys.ethz.ch/\xmptilde nbeisert/}}

\newcommand{\secref}[1]{\hyperref[#1]{section \ref*{#1}}}

\parskip1ex
\parindent0pt
\let\olditemize\itemize
\def\itemize{\olditemize\parskip0pt}

\begin{document}

\title{The \textsf{childdoc} Package}
\hypersetup{pdftitle={The childdoc Package}}
\author{Niklas Beisert\\[2ex]
  Institut f\"ur Theoretische Physik\\
  Eidgen\"ossische Technische Hochschule Z\"urich\\
  Wolfgang-Pauli-Strasse 27, 8093 Z\"urich, Switzerland\\[1ex]
  \href{mailto:nbeisert@itp.phys.ethz.ch}
  {\texttt{nbeisert@itp.phys.ethz.ch}}}
\hypersetup{pdfauthor={Niklas Beisert}}
\hypersetup{pdfsubject={Manual for the LaTeX2e Package childdoc}}
\date{30 December 2018, \textsf{v2.0}}
\maketitle

\begin{abstract}\noindent
\textsf{childdoc} is a \LaTeXe{} package
that enables the direct compilation
of document sections included by |\include|
to individual files.
\end{abstract}

\begingroup
\parskip0ex
\tableofcontents
\endgroup

%%%%%%%%%%%%%%%%%%%%%%%%%%%%%%%%%%%%%%%%%%%%%%%%%%%%%%%%%%%%%%%%%%%%%%%%%%%%%%%%
%%%%%%%%%%%%%%%%%%%%%%%%%%%%%%%%%%%%%%%%%%%%%%%%%%%%%%%%%%%%%%%%%%%%%%%%%%%%%%%%
\section{Introduction}

\LaTeX{} provides a mechanism to structure a large document (such as a book)
into a main file and several child files (containing the chapters)
using the |\include| command.
This mechanism is beneficial for documents
which span hundreds of pages in order to
make the source file(s) more manageable.
Moreover, compilation can be restricted to
selected child files by means of the |\includeonly| command.
The latter feature can be used to reduce the compilation time while editing
(this was significantly more useful in the earlier days of \LaTeX{})
or to generate a smaller document which is easier to navigate.
Another application of |\includeonly| is to generate
documents consisting of selected parts of the complete document.

However, there are a few drawbacks of the plain |\include| mechanism:
\begin{itemize}
\item
The child files cannot be compiled on their own,
they can only be compiled via the main file.
A naive editing environment
(such as a text editor with an option
to have the current file processed by \LaTeX)
may require one to switch to the main file before compiling;
attempting to compile the child file produces errors.
\item
The main file must be modified (each time)
to adjust the |\includeonly| command
to the present needs. This easily leaves the main file in a messy state.
\item
The generated document will always carry the filename
of the main document. This is inconvenient if
several child files are to be compiled and
to be kept for distribution.
\end{itemize}

The present package provides a simple interface
to make child files individually compilable by \LaTeX{}.
Compiling a child file then has the same effect as compiling
the main file with an |\includeonly| command
to select the appropriate child.
Moreover the generated document will carry the name of the child
rather than the main file.
This resolves all three above issues.

This feature is meant to make the editing of books,
thesis documents and lecture notes somewhat more convenient.
However, the package can also be used efficiently for
composing a series of documents (such as exercise sheets)
which are typically distributed individually.
It then assists the author in generating the individual documents
(potentially in different versions)
as well as a document containing the collected series.
Another application is in developing style files
or other kinds of included material
where compilation of the style file could redirect
to a sample or test file.

%%%%%%%%%%%%%%%%%%%%%%%%%%%%%%%%%%%%%%%%%%%%%%%%%%%%%%%%%%%%%%%%%%%%%%%%%%%%%%%%
%%%%%%%%%%%%%%%%%%%%%%%%%%%%%%%%%%%%%%%%%%%%%%%%%%%%%%%%%%%%%%%%%%%%%%%%%%%%%%%%
\section{Usage}

First of all, the package \textsf{childdoc} is \emph{not} a standard
\LaTeXe{} |.sty| style file! Therefore it needs to be invoked in
a non-standard way.

%%%%%%%%%%%%%%%%%%%%%%%%%%%%%%%%%%%%%%%%%%%%%%%%%%%%%%%%%%%%%%%%%%%%%%%%%%%%%%%%
\subsection{Included Files}
\label{sec:include}

%%%%%%%%%%%%%%%%%%%%%%%%%%%%%%%%%%%%%%%%
\DescribeMacro{\childdocmain}
To use the package, add the commands
\begin{center}
\begin{tabular}{l}
|\input{childdoc.def}|\\
|\childdocmain{}|\\
\end{tabular}
\end{center}
at the very top of the main \LaTeX{} file,
in particular \emph{before} the |\documentclass| statement!
The argument of |\childdocmain| should be left empty
(but it must be present).

%%%%%%%%%%%%%%%%%%%%%%%%%%%%%%%%%%%%%%%%
\DescribeMacro{\childdocof}
Furthermore, add the commands
\begin{center}
\begin{tabular}{l}
|\input{childdoc.def}|\\
|\childdocof{|\textit{main}|}|\\
\end{tabular}
\end{center}
at the top of every child file \textit{child}
which is included by |\include{|\textit{child}|}|
from within the main file
(or at least for those files to be compiled individually).
The argument \textit{main} must be the filename of the main file.

There are a couple of
considerations in setting up the main and child documents:

%%%%%%%%%%%%%%%%%%%%%%%%%%%%%%%%%%%%%%%%
\paragraph{Restrictions.}

Please note the following restrictions:
\begin{itemize}
\item
|\childdocmain| must be called with one argument \textit{main}
to ensure compatibility with earlier version of the package.
It must either be empty (|\childdocmain{}|)
or precisely match the filename of the main file in which it is specified.
See \secref{sec:detection} for further information.
\item
The filename \textit{main} must be specified without the |.tex| extension.
\item
The filename \textit{main} is case sensitive
(even in case-insensitive file systems)
due to internal string comparison.
\item
The argument \textit{main} should be fully expanded, it cannot be a macro.
\item
Subdirectories and special characters should be avoided in filenames.
\item
The command |\childdocmain{|\textit{main}|}| must be followed by a whitespace.
It should not be followed immediately by another command
or by a comment mark `|%|'.
This is because the \TeX{} parser reads the token immediately following
the argument of |\childdocmain| and puts it
at the beginning of every child section;
however, a white\-space is ignored.
\end{itemize}

%%%%%%%%%%%%%%%%%%%%%%%%%%%%%%%%%%%%%%%%
\paragraph{Content of Main File.}

It is advisable to place all content in the child files included by |\include|.
Any output contained in the main file will appear in all child documents
unless suppressed manually;
it cannot be suppressed automatically by the |\includeonly| directive
and thus should normally be avoided.
A method to include some content in the main file
by means of conditional processing is described in \secref{sec:conditional}.

%%%%%%%%%%%%%%%%%%%%%%%%%%%%%%%%%%%%%%%%
\paragraph{Page Numbering.}

When only a part of the document is compiled,
the appropriate numbering of pages
(as well as other status parameters)
is determined from the |.aux| files.
The latter contain information from previous passes.
However this information needs to propagate through
all intermediate child documents.
Therefore the page numbering in child documents may well
be inconsistent until the complete document is compiled at least once.

A useful (if unconventional) way to always ensure a consistent
page numbering is to restart the numbering in each child document
and denote the pages by `\textit{child}|.|\textit{page}'
where \textit{child} represents the chapter/section number of the child file.
This can be achieved by the command
|\numberwithin{page}{|\textit{child}|}|
of the \textsf{amsmath} package
where \textit{child} can be |chapter| or |section|
depending on the chosen structuring.
Alternatively, one can modify the macro |\thepage| appropriately
and reset the counter |page| at the start of each child file.

%%%%%%%%%%%%%%%%%%%%%%%%%%%%%%%%%%%%%%%%%%%%%%%%%%%%%%%%%%%%%%%%%%%%%%%%%%%%%%%%
\subsection{Conditional Processing}
\label{sec:conditional}

The package provides a mechanism to compile different versions
of a document. To customise the versions further some conditional processing
can come in handy to distinguish which version is being compiled.
The package provides two macros to describe the compilation context:

%%%%%%%%%%%%%%%%%%%%%%%%%%%%%%%%%%%%%%%%
\DescribeMacro{\ifchilddoc}
The conditional |\ifchilddoc| distinguishes between the compilation of
child documents and the main document:
%
\begin{center}
|\ifchilddoc |\textit{child-code}| |[|\||else |\textit{main-code}]| \||fi|
\end{center}

%%%%%%%%%%%%%%%%%%%%%%%%%%%%%%%%%%%%%%%%
\DescribeMacro{\childdocname}
\DescribeMacro{\childdocjob}
The macro |\childdocname| contains the filename (without extension)
of the main or child file being processed.
Note that |\childdocjob| will always contain the name of the main file.

%%%%%%%%%%%%%%%%%%%%%%%%%%%%%%%%%%%%%%%%
\paragraph{Title Page.}

Conditional processing can be used to include a title or banner page
in the main document when proper precautions are taken.
Importantly, the code in the main file should ensure that the page counter
(as well as other status parameters which are stored in the |.aux| files)
takes the same value after the conditional processing.
Otherwise the page numbers may take divergent values
depending on which part is compiled.

For example, a title page could be declared by:
%
\begin{center}
\begin{tabular}{l}
|\ifchilddoc\||else|\\
|\addtocounter{page}{-1}|\\
\textit{code for title page}\\
|\newpage|\\
|\||fi|
\end{tabular}
\end{center}
%
A banner page for the child documents can be generated by:
%
\begin{center}
\begin{tabular}{l}
|\ifchilddoc|\\
|\addtocounter{page}{-1}|\\
\textit{code for banner page}\\
|\newpage|\\
|\||fi|
\end{tabular}
\end{center}
%
Here one could write a message such as:
\begin{center}
|This is the part \childdocname{} of \childdocjob{}.|
\end{center}

%%%%%%%%%%%%%%%%%%%%%%%%%%%%%%%%%%%%%%%%%%%%%%%%%%%%%%%%%%%%%%%%%%%%%%%%%%%%%%%%
\subsection{Flags}
\label{sec:flags}

The package makes it easy to generate different versions
of the main or child documents.
To this end compilation flags can be defined
and assigned different default values.
They will be particularly useful in conjunction
with the forwarding mechanism described in \secref{sec:forward}.

For example, it may be useful to have a flag |\version|
which can be set to |draft| or |final|.
The document source will contain some conditional code
depending on the value of |\version|.
Suppose further, the flag should default to |final| for the main file
and to |draft| for child files
which is a natural assignment for editing the document.
This is achieved by placing the following code
in the preamble of the main document
(below the |\childdocmain| directive):
%
\begin{center}
\begin{tabular}{l}
|\ifchilddoc|\\
|\providecommand{\version}{draft}|\\
|\||else|\\
|\providecommand{\version}{final}|\\
|\||fi|
\end{tabular}
\end{center}
%
The definition by |\providecommand| makes sure
that previous definitions are not overwritten.
Further statements |\providecommand{\version}{...}|
can thus be added before the above code to override it.

For the main file, one might add a line
(between |\childdocmain| and the above block)
%
\begin{center}
|%\ifchilddoc\||else\providecommand{\version}{draft}\||fi|
\end{center}
%
which can be uncommented to produce a draft version.
Likewise one can add a line to the very top of a child file
(above the |\childdocof{|\textit{main}|}| directive)
%
\begin{center}
|%\providecommand{\version}{final}|
\end{center}
%
which can be uncommented to produce the final version of this child document.

%%%%%%%%%%%%%%%%%%%%%%%%%%%%%%%%%%%%%%%%%%%%%%%%%%%%%%%%%%%%%%%%%%%%%%%%%%%%%%%%
\subsection{Forwarding}
\label{sec:forward}

Different versions of the main or child documents
using compilation flags as described in \secref{sec:flags}
can be (permanently) stored in different files
for convenient compilation, viewing and distribution.
To this end, the package defines a command
to pass on compilation to a different file:

%%%%%%%%%%%%%%%%%%%%%%%%%%%%%%%%%%%%%%%%
\DescribeMacro{\childdocforward}
The command |\childdocforward| redirects processing to
another source file:
%
\begin{center}
\begin{tabular}{l}
|\input{childdoc.def}|\\
|\childdocforward[|\textit{main}|]{|\textit{dest}|}|\\
\end{tabular}
\end{center}
%
The argument \textit{dest} is the destination file
(without extension).
It should be the main file or one of the child files.
Note that further \textsf{childdoc} directives
such as |\childdocof| and |\childdocforward|
in the indicated file will be processed in this form.
The optional argument \textit{main}
passes on directly to the main file \textit{main}
while pretending to compile the child \textit{dest}.
This form behaves as if \textit{dest}
issues |\childdocof{|\textit{main}|}| right away,
and no further \textsf{childdoc} directives will be processed.

%%%%%%%%%%%%%%%%%%%%%%%%%%%%%%%%%%%%%%%%
\DescribeMacro{\...prefix}
In the alternative form |\childdocforwardprefix|,
%
\begin{center}
\begin{tabular}{l}
|\input{childdoc.def}|\\
|\childdocforwardprefix[|\textit{main}|]{|\textit{prefix}|}{|\textit{dest}|}|
\end{tabular}
\end{center}
%
the destination file is determined by a pattern
depending on the current file:
To make this work, the current file must be called
`{\textit{prefix}\hspace{0.2em}\textit{suffix}}'
with \textit{prefix} matching precisely the argument.
Processing is then passed on to the file
`{\textit{dest}\hspace{0.2em}\textit{suffix}}'.
Surely, the same effect is achieved by
directly specifying the
argument `{\textit{dest}\hspace{0.2em}\textit{suffix}}'
in the first form.
However, that requires to set up a different file
for each child. With the alternative form of the command
all these files can have exactly the same content
which simplifies setting them up and maintaining them.

For example, the following file |draft.tex|
with a compilation flag |\version| as described in \secref{sec:flags}
compiles the main document as a draft:
%
\begin{center}
\begin{tabular}{l}
|\def\version{draft}|\\
|\input{childdoc.def}|\\
|\childdocforward{|\textit{main}|}|
\end{tabular}
\end{center}
%
Likewise, the following files |final|\textit{nn}|.tex|
compile the final version of the child document
|child|\textit{nn}|.tex|:
%
\begin{center}
\begin{tabular}{l}
|\def\version{final}|\\
|\input{childdoc.def}|\\
|\childdocforwardprefix{final}{child}|
\end{tabular}
\end{center}
%

Note that when several versions of a main file and/or of each child file
are to be generated, it may be convenient to set up a |Makefile| or
shell script to automatise the process.

%%%%%%%%%%%%%%%%%%%%%%%%%%%%%%%%%%%%%%%%%%%%%%%%%%%%%%%%%%%%%%%%%%%%%%%%%%%%%%%%
\subsection{Command Line Processing}
\label{sec:commandline}

The effect of redirection files can also be achieved by invoking
the \LaTeX{} compiler with a more elaborate command line.
Most conveniently this should be done as part
of a shell script or a |Makefile|.

When using \textsf{childdoc} in the main file, the following
command lines effectively perform a redirection
(note that depending on the shell being used,
backslashes may have to be doubled: `|\|' $\to$ `|\\|'):
%
\begin{center}
|... -jobname "|\textit{target}|" |\\|"|[\textit{flags}]%
|\input{childdoc.def}\childdocforward[|\textit{main}|]{|\textit{dest}|}"|
\end{center}
%
Here \textit{target} is the name of the output file,
\textit{main} is the name of the main file
and \textit{dest} is the name of the main or child file to be processed
(all filenames without extensions).
The optional argument \textit{main} can be omitted
if \textit{main} matches \textit{dest}.
Optionally, compilation \textit{flags} can be defined via |\def| commands.
This command line makes the \TeX{} engine believe
it is compiling the file \textit{target}
whose content is specified as the latter parameter.
The provided code then forwards the processing to
\textit{main} or \textit{dest} as described in \secref{sec:forward}.

%%%%%%%%%%%%%%%%%%%%%%%%%%%%%%%%%%%%%%%%%%%%%%%%%%%%%%%%%%%%%%%%%%%%%%%%%%%%%%%%
\subsection{Include by Input}
\label{sec:input}

Including child documents by |\include| has some restrictions by design.
Most notably, the content of a child document always occupies
its own set of pages; pages cannot be shared between child documents.
Usually, this behaviour makes perfect sense
because each child document contain an essential part of the document.
However, in some situations it may be desirable to compose
a document from a collection of parts
without having mandatory page breaks between then.
For this case, the package
provides a mechanism to include parts
by |\input| which can also be processed individually.
However, by construction this mechanism
requires manual handling of the content to be output.

%%%%%%%%%%%%%%%%%%%%%%%%%%%%%%%%%%%%%%%%
\DescribeMacro{\ifchilddocmanual}
The main file should be prepared as usual, see \secref{sec:include}.
However, the document body must make a distinction
between processing of an individual part and of the main document, e.g.:
%
\begin{center}
\begin{tabular}{l}
|\ifchilddocmanual|\\
|\input{\childdocname}|\\
|\||else|\\
\textit{document body with }|\input{|\textit{part}|}|\\
|\||fi|
\end{tabular}
\end{center}
%
The conditional |\ifchilddocmanual| is true whenever
a part to be included by |\input| is being compiled,
and the name of the part is stored in |\childdocname|.

%%%%%%%%%%%%%%%%%%%%%%%%%%%%%%%%%%%%%%%%
\DescribeMacro{\childdocby}
Each part to be included by |\input| should start with:
%
\begin{center}
\begin{tabular}{l}
|\input{childdoc.def}|\\
|\childdocby{|\textit{main}|}|\\
\end{tabular}
\end{center}
%
The directive |\childdocby| is similar to |\childdocof|
described in \secref{sec:include},
but the subsequent selection of content must be done manually.
To that end, both |\ifchilddoc| and |\ifchilddocmanual|
will be true upon processing of a part,
and the name of the part is stored in |\childdocname|.
Note that |\jobname| will be set to the filename of the current part
so that each part receives an individual |.aux| file
that does not interfere with the |.aux| file(s) of the main document.
This behaviour can be altered by the alternative form
|\childdocby[*]{|\textit{main}|}| (with a non-empty optional argument)
which uses the |.aux| file of the main document
by setting |\jobname| to \textit{main}.

%%%%%%%%%%%%%%%%%%%%%%%%%%%%%%%%%%%%%%%%%%%%%%%%%%%%%%%%%%%%%%%%%%%%%%%%%%%%%%%%
\subsection{Driver Development}
\label{sec:driver}

The \textsf{childdoc} mechanism can also be use for the development
of definition files such as \LaTeX{} styles or classes.
This case differs from the above setup with multiple parts
included by |\include| in that no |\includeonly| should be invoked.
This can be achieved by starting the include file
(before |\ProvidesPackage|) with:
%
\begin{center}
\begin{tabular}{l}
|\input{childdoc.def}|\\
|\childdocforward{|\textit{main}|}|\\
\end{tabular}
\end{center}
%
or alternatively with:
%
\begin{center}
\begin{tabular}{l}
|\input{childdoc.def}|\\
|\childdocby{|\textit{main}|}|\\
\end{tabular}
\end{center}
%
Both forms have slightly different effects as described above.
The main file is prepared as usual, see \secref{sec:include}.

%%%%%%%%%%%%%%%%%%%%%%%%%%%%%%%%%%%%%%%%%%%%%%%%%%%%%%%%%%%%%%%%%%%%%%%%%%%%%%%%
\subsection{Legacy Detection}
\label{sec:detection}

The directive |\childdocmain| in the main file can detect
whether the complete document or merely a child is to be compiled
even without using the directive |\childdocof|.
This method is deprecated because it is less robust
and there is no compelling reason to use it;
it is merely provided for backward compatibility
and it may be removed in future versions.

If the detection mechanism is to be used,
it is mandatory to correctly specify
the filename of the main file as the argument of |\childdocmain|:
%
\begin{center}
\begin{tabular}{l}
|\input{childdoc.def}|\\
|\childdocmain{|\textit{main}|}|\\
\end{tabular}
\end{center}
%
If |\jobname| does not match the argument \textit{main} of |\childdocmain|,
it is assumed that |\jobname| points to the child file to be compiled.
When using |\childdocmain| with the main file specified as argument,
it suffices to start a child file
with just |\input{|\textit{main}|}|
without loading of the package and using |\childdocof|.
If instead all processing is done
with the appropriate \textsf{childdoc} directives,
the argument of \textit{main} of |\childdocmain| can be empty.

An alternative version of the command line processing described
in \secref{sec:commandline} using the detection mechanism reads:
%
\begin{center}
|... -jobname "|\textit{target}|" "|[\textit{flags}]%
[|\def\jobname{|\textit{dest}|}|]|\input{|\textit{main}|}"|
\end{center}

%%%%%%%%%%%%%%%%%%%%%%%%%%%%%%%%%%%%%%%%%%%%%%%%%%%%%%%%%%%%%%%%%%%%%%%%%%%%%%%%
\subsection{Manual Code}
\label{sec:manual}

In case one cannot be certain whether the definitions file |childdoc.def|
is installed on the target \TeX{} distribution
and one prefers not to ship it,
it is conceivable to paste a few relevant commands into the sources.

To that end, drop all statements |\input{childdoc.def}|
and perform the replacements as outlined below.
Instead of |\childdocmain{|\textit{main}|}| add the following code
to the top of the main file:
%
\begin{center}
\begin{tabular}{l}
|\||ifdefined\childdocname\endinput\||fi\newif\ifchilddoc|\\
|\edef\childdocname{\scantokens\expandafter{\jobname\noexpand}}|\\
|\def\childdocmain{|\textit{main}|}\||ifx\childdocmain\childdocname\||else|\\
|\childdoctrue\includeonly{\childdocname}\let\jobname\childdocmain\||fi|\\
\end{tabular}
\end{center}
%
Instead of |\childdocof{|\textit{main}|}| just include the main file
at the top of each child file:
%
\begin{center}
|\input{|\textit{main}|}|
\end{center}
%
A simple redirection |\childdocforward{|\textit{dest}|}| is achieved by:
%
\begin{center}
|\def\jobname{|\textit{dest}|}\input{\jobname}|
\end{center}
%
The redirection with prefix
|\childdocforwardprefix[|\textit{prefix}|]{|\textit{dest}|}|
is accomplished by:
%
\begin{center}
\begin{tabular}{l}
|{\edef\jobname{\scantokens\expandafter{\jobname\noexpand}}|\\
|\def\redirectjob |\textit{prefix}|#1~~~{\gdef\jobname{|\textit{dest}|#1}}|\\
|\expandafter\redirectjob\jobname~~~}\input{\jobname}|
\end{tabular}
\end{center}

In an alternative approach,
child documents can be compiled by a specific command line
without additional code or specific definitions:
%
\begin{center}
|... -jobname "|\textit{target}|" "|[\textit{flags}]%
|\includeonly{|\textit{dest}|}\input{|\textit{main}|}"|
\end{center}
%

%%%%%%%%%%%%%%%%%%%%%%%%%%%%%%%%%%%%%%%%%%%%%%%%%%%%%%%%%%%%%%%%%%%%%%%%%%%%%%%%
%%%%%%%%%%%%%%%%%%%%%%%%%%%%%%%%%%%%%%%%%%%%%%%%%%%%%%%%%%%%%%%%%%%%%%%%%%%%%%%%
\section{Information}

%%%%%%%%%%%%%%%%%%%%%%%%%%%%%%%%%%%%%%%%%%%%%%%%%%%%%%%%%%%%%%%%%%%%%%%%%%%%%%%%
\subsection{Copyright}

Copyright \copyright{} 2017--2018 Niklas Beisert

This work may be distributed and/or modified under the
conditions of the \LaTeX{} Project Public License, either version 1.3
of this license or (at your option) any later version.
The latest version of this license is in
  \url{http://www.latex-project.org/lppl.txt}
and version 1.3 or later is part of all distributions of \LaTeX{}
version 2005/12/01 or later.

This work has the LPPL maintenance status `maintained'.

The Current Maintainer of this work is Niklas Beisert.

This work consists of the files |README.txt|, |childdoc.ins| and |childdoc.dtx|
as well as the derived files |childdoc.def|, |cdocsamp.tex|
with |cdocsch1.tex|, |cdocsch2.tex|, |cdocspt3.tex|, |cdocspt4.tex|,
|cdocsdrf.tex|, |cdocsfn1.tex|, |cdocsfn2.tex|
as well as |childdoc.pdf|.

%%%%%%%%%%%%%%%%%%%%%%%%%%%%%%%%%%%%%%%%%%%%%%%%%%%%%%%%%%%%%%%%%%%%%%%%%%%%%%%%
\subsection{Files and Installation}

The package consists of the files:
%
\begin{center}
\begin{tabular}{ll}
    |README.txt|   & readme file \\
    |childdoc.ins| & installation file \\
    |childdoc.dtx| & source file \\
    |childdoc.def| & definition file \\
    |cdocsamp.tex| & sample main file \\
    |cdocsch1.tex| & sample include file \\
    |cdocsch2.tex| & sample include file \\
    |cdocspt3.tex| & sample part file \\
    |cdocspt4.tex| & sample part file \\
    |cdocsdrf.tex| & sample redirection file \\
    |cdocsfn1.tex| & sample redirection file \\
    |cdocsfn2.tex| & sample redirection file \\
    |childdoc.pdf| & manual
\end{tabular}
\end{center}
%
The distribution consists of the files
|README.txt|, |childdoc.ins| and |childdoc.dtx|.
%
\begin{itemize}
\item
Run (pdf)\LaTeX{} on |childdoc.dtx|
to compile the manual |childdoc.pdf| (this file).
\item
Run \LaTeX{} on |childdoc.ins| to create the definitions file |childdoc.def|
and the sample |cdocsamp.tex| with include files
|cdocsch1.tex|, |cdocsch2.tex|, |cdocspt3.tex|, |cdocspt4.tex|,
|cdocsdrf.tex|, |cdocsfn1.tex|, |cdocsfn2.tex|.
Then copy the file |childdoc.def| to an appropriate directory of your \LaTeX{}
distribution, e.g.\ \textit{texmf-root}|/tex/latex/childdoc|.
\end{itemize}

%%%%%%%%%%%%%%%%%%%%%%%%%%%%%%%%%%%%%%%%%%%%%%%%%%%%%%%%%%%%%%%%%%%%%%%%%%%%%%%%
\subsection{Related CTAN Packages}

There are several other packages which offer a similar functionality:
%
\begin{itemize}
\item
The packages
\href{http://ctan.org/pkg/docmute}{\textsf{docmute}},
\href{http://ctan.org/pkg/includex}{\textsf{includex}} and
\href{http://ctan.org/pkg/standalone}{\textsf{standalone}}
provide commands to include only the document body of
a child file thus allowing both files to be compiled individually.
\item
The packages \href{http://ctan.org/pkg/subdocs}{\textsf{subdocs}}
and \href{http://ctan.org/pkg/subfiles}{\textsf{subfiles}}
provide structures in which the main and child documents can be
encapsulated and allowing them to be compiled individually.
The inclusion mechanism is different from the conventional |\include|.
\item
The package \href{http://ctan.org/pkg/combine}{\textsf{combine}}
is an elaborate solution to combine several documents into one.
\end{itemize}
%
See also the CTAN topic \href{http://ctan.org/topic/subdocs}{\textsf{subdocs}}
for further related packages.
The present package differs from the above solutions in that
a document structure constructed with the conventional |\include| mechanism
just needs two extra commands at the top of every file
such that all constituent files can be compiled individually.

%%%%%%%%%%%%%%%%%%%%%%%%%%%%%%%%%%%%%%%%%%%%%%%%%%%%%%%%%%%%%%%%%%%%%%%%%%%%%%%%
%\subsection{Feature Suggestions}
%
%The following is a list of features which may be useful for future
%versions of this package:
%%
%\begin{itemize}
%\item
%\ldots
%\end{itemize}

%%%%%%%%%%%%%%%%%%%%%%%%%%%%%%%%%%%%%%%%%%%%%%%%%%%%%%%%%%%%%%%%%%%%%%%%%%%%%%%%
\subsection{Revision History}

%%%%%%%%%%%%%%%%%%%%%%%%%%%%%%%%%%%%%%%%
\paragraph{v2.0:} 2018/12/30

\begin{itemize}
\item
immediate forward processing
\item
added |\childdocby| mechanism
\item
manual restructured
\end{itemize}

%%%%%%%%%%%%%%%%%%%%%%%%%%%%%%%%%%%%%%%%
\paragraph{v1.6:} 2018/01/17

\begin{itemize}
\item
application for development of include files
\item
corrections to manual
\end{itemize}

%%%%%%%%%%%%%%%%%%%%%%%%%%%%%%%%%%%%%%%%
\paragraph{v1.5:} 2017/05/21

\begin{itemize}
\item
more complete structuring introduced
\item
|\childdocof| introduced
\item
|\childdoc| renamed to |\childdocmain|
\item
|\childredirect| renamed to |\childdocforward| and |\childdocforwardprefix|
and functionality expanded
\end{itemize}

%%%%%%%%%%%%%%%%%%%%%%%%%%%%%%%%%%%%%%%%
\paragraph{v1.0:} 2017/04/27

\begin{itemize}
\item
manual and install package
\item
first version published on CTAN
\end{itemize}

%%%%%%%%%%%%%%%%%%%%%%%%%%%%%%%%%%%%%%%%
\paragraph{v0.6:} 2017/04/26

\begin{itemize}
\item
redirection mechanism added
\end{itemize}

%%%%%%%%%%%%%%%%%%%%%%%%%%%%%%%%%%%%%%%%
\paragraph{v0.5:} 2017/04/26

\begin{itemize}
\item
functionality in definition file
\end{itemize}


%%%%%%%%%%%%%%%%%%%%%%%%%%%%%%%%%%%%%%%%%%%%%%%%%%%%%%%%%%%%%%%%%%%%%%%%%%%%%%%%
%%%%%%%%%%%%%%%%%%%%%%%%%%%%%%%%%%%%%%%%%%%%%%%%%%%%%%%%%%%%%%%%%%%%%%%%%%%%%%%%
%%%%%%%%%%%%%%%%%%%%%%%%%%%%%%%%%%%%%%%%%%%%%%%%%%%%%%%%%%%%%%%%%%%%%%%%%%%%%%%%
\appendix

\settowidth\MacroIndent{\rmfamily\scriptsize 000\ }

 \DocInput{childdoc.dtx}

\end{document}
%</driver>
% \fi
%
% %%%%%%%%%%%%%%%%%%%%%%%%%%%%%%%%%%%%%%%%%%%%%%%%%%%%%%%%%%%%%%%%%%%%%%%%%%%%%%
% %%%%%%%%%%%%%%%%%%%%%%%%%%%%%%%%%%%%%%%%%%%%%%%%%%%%%%%%%%%%%%%%%%%%%%%%%%%%%%
% \section{Sample}
%\iffalse
%<*samplemain>
%\fi
%
% The following presents a sample document
% with two chapters, two parts, a title page,
% a compile flag as well as three forwarding files to set the flag.
% It consists of eight |.tex| files:
% \begin{center}
% \begin{tabular}{ll}
% |cdocsamp.tex|&main file\\
% |cdocsch1.tex|&include file for chapter 1\\
% |cdocsch2.tex|&include file for chapter 2\\
% |cdocspt3.tex|&include file for part 3\\
% |cdocspt4.tex|&include file for part 4\\
% |cdocsdrf.tex|&forwarding file for main file in draft mode\\
% |cdocsfi1.tex|&forwarding file for final version of chapter 1\\
% |cdocsfi2.tex|&forwarding file for final version of chapter 2\\
% \end{tabular}
% \end{center}
% Each of the eight files can be compiled directly by the \LaTeX{} compiler.
%
% %%%%%%%%%%%%%%%%%%%%%%%%%%%%%%%%%%%%%%
% \paragraph{Main File.}
%
% The main file is called |cdocsamp.tex|.
%
% Load the \textsf{childdoc} definitions and
% declare the filename for the main document:
%    \begin{macrocode}
\input{childdoc.def}
\childdocmain{}
%    \end{macrocode}

% Optional override for |\version| flag:
%    \begin{macrocode}
%%\ifchilddoc\else\providecommand{\version}{draft}\fi
%    \end{macrocode}

% Define the default values for the |\version| flag
% (|final| for the main file and |draft| for childs):
%    \begin{macrocode}
\ifchilddoc
\providecommand{\version}{draft}
\else
\providecommand{\version}{final}
\fi
%    \end{macrocode}

% Load the standard document class:
%    \begin{macrocode}
\documentclass[12pt]{article}
%    \end{macrocode}

% Start the document body:
%    \begin{macrocode}
\begin{document}
%    \end{macrocode}

% Declare a title page.
% Print title, part of document being processed and version flag:
%    \begin{macrocode}
\addtocounter{page}{-1}
\begin{center}
{\LARGE\bfseries{}childdoc example\par}
\vspace{1cm}
\ifchilddoc
\ifchilddocmanual part\else chapter\fi:
`\childdocname' of `\childdocjob'\par
\else
main document: `\childdocjob'\par
\fi
version: \version\par
\end{center}
\newpage
%    \end{macrocode}

% Manually include selected file,
% otherwise process as usual:
%    \begin{macrocode}
\ifchilddocmanual
\section*{part `\childdocname'}
\input{\childdocname}
\else
%    \end{macrocode}

% Include the two chapters:
%    \begin{macrocode}
\include{cdocsch1}
\include{cdocsch2}
%    \end{macrocode}

% Include the two parts unless only chapters should be displayed:
%    \begin{macrocode}
\ifchilddoc\else
\section{part three}
\input{cdocspt3}
\section{part four}
\input{cdocspt4}
\fi
%    \end{macrocode}

% Process as usual until here:
%    \begin{macrocode}
\fi
%    \end{macrocode}

% End of document body:
%    \begin{macrocode}
\end{document}
%    \end{macrocode}
%\iffalse
%</samplemain>
%\fi
%
% %%%%%%%%%%%%%%%%%%%%%%%%%%%%%%%%%%%%%%
% \paragraph{Chapter Include Files.}
%
% The include files are called |cdocsch1.tex| and |cdocsch2.tex|.
%
%\iffalse
%<*samplechap1|samplechap2>
%\fi

% Optional override for |\version| flag:
%    \begin{macrocode}
%%\providecommand{\version}{final}
%    \end{macrocode}

% Include the main document:
%    \begin{macrocode}
\input{childdoc.def}
\childdocof{cdocsamp}
%    \end{macrocode}

%\iffalse
%</samplechap1|samplechap2>
%\fi
%
%\iffalse
%<*samplechap1>
%\fi
% Some text for chapter 1:
%    \begin{macrocode}
\section{one}
some text in chapter one
%    \end{macrocode}

%\iffalse
%</samplechap1>
%\fi
% Some text for chapter 2:
%\iffalse
%<*samplechap2>
%\fi
%    \begin{macrocode}
\section{two}
more text in chapter two
%    \end{macrocode}

%\iffalse
%</samplechap2>
%\fi
%
% %%%%%%%%%%%%%%%%%%%%%%%%%%%%%%%%%%%%%%
% \paragraph{Part Include Files.}
%
% The include files are called |cdocspt3.tex| and |cdocspt4.tex|.
%
%\iffalse
%<*samplepart3|samplepart4>
%\fi

% Optional override for |\version| flag:
%    \begin{macrocode}
%%\providecommand{\version}{final}
%    \end{macrocode}

% Include the main document:
%    \begin{macrocode}
\input{childdoc.def}
\childdocby{cdocsamp}
%    \end{macrocode}

%\iffalse
%</samplepart3|samplepart4>
%\fi
%
%\iffalse
%<*samplepart3>
%\fi
% Some text for part 3:
%    \begin{macrocode}
some text in part three
%    \end{macrocode}

%\iffalse
%</samplepart3>
%\fi
% Some text for part 4:
%\iffalse
%<*samplepart4>
%\fi
%    \begin{macrocode}
more text in part four
%    \end{macrocode}

%\iffalse
%</samplepart4>
%\fi
%
% %%%%%%%%%%%%%%%%%%%%%%%%%%%%%%%%%%%%%%
% \paragraph{Forwarding for a Complete Draft.}
%
% The following forwarding file |cdocsdrf.tex|
% compiles the main document in draft mode:
%\iffalse
%<*sampledraft>
%\fi
%    \begin{macrocode}
\def\version{draft}
\input{childdoc.def}
\childdocforward{cdocsamp}
%    \end{macrocode}

%\iffalse
%</sampledraft>
%\fi
%
% %%%%%%%%%%%%%%%%%%%%%%%%%%%%%%%%%%%%%%
% \paragraph{Forwarding for Final Version of the Chapters.}
%
% The following forwarding files |cdocsfn1.tex| and |cdocsfn2.tex|
% (with identical content)
% compile the final versions of the child documents
% |cdocsch1.tex| and |cdocsch2.tex|, respectively:
%\iffalse
%<*samplefinal>
%\fi
%    \begin{macrocode}
\def\version{final}
\input{childdoc.def}
\childdocforwardprefix[cdocsamp]{cdocsfn}{cdocsch}
%    \end{macrocode}

%\iffalse
%</samplefinal>
%\fi
%
% %%%%%%%%%%%%%%%%%%%%%%%%%%%%%%%%%%%%%%
% \paragraph{Command Line Processing.}
%
% The following three command lines generate the output files
% |cdocscld|, |cdocscl1| and |cdocscl2|
% which should be identical to
% |cdocsdrf|, |cdocsch1| and |cdocsfn2|, respectively:
% \begin{center}
% \begin{tabular}{l}
% |latex -jobname cdocscld \|\\
% |  "\def\version{draft}\input{childdoc.def}\childdocforward{cdocsamp}"|\\
% |latex -jobname cdocscl1 \|\\
% |  "\input{childdoc.def}\childdocforward[cdocsamp]{cdocsch1}"|\\
% |latex -jobname cdocscl2 \|\\
% |  "\def\version{final}\input{childdoc.def}\childdocforward{cdocsch2}"|
% \end{tabular}
% \end{center}
% Note that the trailing backslash on each first line
% merely continues the input to the second line
% (for convenient cut ant paste).
% Furthermore, the command |latex| can be replaced by any
% of its alternative versions such as |pdflatex|.
%
% %%%%%%%%%%%%%%%%%%%%%%%%%%%%%%%%%%%%%%%%%%%%%%%%%%%%%%%%%%%%%%%%%%%%%%%%%%%%%%
% %%%%%%%%%%%%%%%%%%%%%%%%%%%%%%%%%%%%%%%%%%%%%%%%%%%%%%%%%%%%%%%%%%%%%%%%%%%%%%
% \section{Implementation}
%\iffalse
%<*package>
%\fi
%
% This section describes the definitions file |childdoc.def|.

% The definitions cannot be loaded using |\usepackage| or |\RequirePackage|
% which has a mechanism to prevent loading a style file more than once.
% When loading the definitions by means of |\input|
% multiple instances have to be prevented manually:
%\iffalse
%This code needs to be before the `\ProvidesFile' directive
%which is defined at the beginning of this file.
%Therefore it is also placed there and commented out here.
%</package>
%<*discard>
%\fi
%    \begin{macrocode}
\ifdefined\childdocmain\endinput\fi
%    \end{macrocode}
%\iffalse
%</discard>
%<*package>
%\fi
%
% \macro{\ifchilddoc}
% \macro{\ifchilddocmanual}
% The conditional |\ifchilddoc| tells whether a
% child (true) or main (false) document is being compiled.
% The conditional |\ifchilddocmanual| tells whether
% the |\includeonly| mechanism is used (false) or
% the selection of child files must be performed manually (true).
% The definitions initialise to false:
%    \begin{macrocode}
\newif\ifchilddoc
\newif\ifchilddocmanual
%    \end{macrocode}

% \macro{\childdocname}
% \macro{\childdocjob}
% The macro |\childdocname| stores the name of the main document
% to be compiled. The macro |\childdocjob| stores the name of
% the document on which the \LaTeX{} compiler was originally invoked.
% The content of |\jobname| cannot be compared
% to filenames specified in the source due to different catcodes.
% The following code rescans |\jobname|, stores the result
% in |\childdocname| and saves a copy in |\childdocjob|:
%    \begin{macrocode}
\edef\childdocname{\scantokens\expandafter{\jobname\noexpand}}
\let\childdocjob\childdocname
%    \end{macrocode}

% \macro{\childdocdisable}
% The macro |\childdocdisable| prevents the main file
% from being processed more than once.
% At this stage, the main document command |\childdocmain|
% is assumed to be called once again where it should do nothing.
% Any subsequent call to it should prevent
% a secondary processing of the main document
% It overwrites the forwarding commands
% |\childdocof| and |\childdocforward|
% with empty macros to prevent further inclusions of the main document:
%    \begin{macrocode}
\newcommand{\childdocdisable}
{
  \renewcommand{\childdocmain}[1]{\renewcommand{\childdocmain}[1]{\endinput}}
  \renewcommand{\childdocof}[1]{}
  \renewcommand{\childdocby}[2][]{}
  \renewcommand{\childdocforward}[2][]{}
  \renewcommand{\childdocdisable}{}
}
%    \end{macrocode}

% \macro{\childdocmain}
% The macro |\childdocmain| is to be called at the top of the main file
% with nothing or the main filename (without extension) as argument.
% First, it breaks loops.
% If the argument is not empty and does not match |\childdocname|
% (which is set by the first inclusion of |childdoc.def|),
% |\ifchilddoc| is set to true, |\includeonly| is applied to the child file
% and |\jobname| is set to the main file
% (for proper handling of |.aux| files):
%    \begin{macrocode}
\newcommand{\childdocmain}[1]
{
  \childdocdisable\childdocmain{}
  \if?#1?\else
    \begingroup
      \def\childdoctmp{#1}
      \ifx\childdoctmp\childdocname
        \def\childdoctmp{}
      \else
        \def\childdoctmp
        {
          \childdoctrue
          \includeonly{\childdocname}
          \def\childdocjob{#1}
          \def\jobname{#1}
        }
      \fi
      \expandafter
    \endgroup
    \childdoctmp
  \fi
}
%    \end{macrocode}

% \macro{\childdocof}
% The command |\childdocof| redirects
% compilation to the main file |#1|.
%    \begin{macrocode}
\newcommand{\childdocof}[1]
{
  \childdocdisable
  \childdoctrue
  \includeonly{\childdocname}
  \def\jobname{#1}
  \def\childdocjob{#1}
  \input{#1}
}
%    \end{macrocode}

% \macro{\childdocby}
% The command |\childdocby| ....
%    \begin{macrocode}
\newcommand{\childdocby}[2][]
{
  \childdocdisable
  \childdoctrue
  \childdocmanualtrue
  \if?#1?\else
    \def\jobname{#2}
  \fi
  \def\childdocjob{#2}
  \input{#2}
  \endinput
}
%    \end{macrocode}

% \macro{\childdocforward}
% The command |\childdocforward| redirects
% compilation to the main file or
% (if the optional argument is given) a child file.
% Parameters are set as if the main file
% or a child file starting with |\childdocof| was compiled.
% Then compilation is handed over to the main file:
%    \begin{macrocode}
\newcommand{\childdocforward}[2][]
{
  \begingroup
    \if?#1?
      \def\childdoctmp
      {
        \def\childdocname{#2}
        \def\childdocjob{#2}
        \def\jobname{#2}
        \input{#2}
        \endinput
      }
    \else
      \def\childdoctmp
      {
        \childdocdisable
        \def\childdocname{#2}
        \childdoctrue
        \includeonly{#2}
        \def\childdocjob{#1}
        \def\jobname{#1}
        \input{#1}
        \endinput
      }
    \fi
    \expandafter
  \endgroup
  \childdoctmp
}
%    \end{macrocode}

% \macro{\childdocforwardprefix}
% The command |\childdocforwardprefix| redirects
% compilation to the main or a child file by means of a pattern.
% The prefix |#1| in the current filename is replaced by |#2|
% and the suffix of the current filename is kept
% (it is assumed that the filename does not contain the substring `|~~~|'
% which is used as a delimiter).
% Compilation is handed over to the new file by |\childdocforward|:
%    \begin{macrocode}
\newcommand{\childdocforwardprefix}[3][]
{
  \begingroup
    \def\childdocextract #2##1~~~{\def\childdoctmp{\childdocforward[#1]{#3##1}}}
    \expandafter\childdocextract\childdocname~~~
    \expandafter
  \endgroup
  \childdoctmp
}
%    \end{macrocode}

% \macro{\childdoc}
% The deprecated macro |\childdoc| is a legacy version of |\childdocmain|:
%    \begin{macrocode}
\newcommand{\childdoc}{\childdocmain}
%    \end{macrocode}

% \macro{\childdocredirect}
% The deprecated macro |\childdocredirect| is a legacy version
% of |\childdocforward| and |\childdocforwardprefix|:
%    \begin{macrocode}
\newcommand{\childdocredirect}[2][]
{
  \begingroup
    \if?#1?
      \def\childdoctmp{\childdocforward{#2}}
    \else
      \def\childdoctmp{\childdocforwardprefix{#1}{#2}}
    \fi
    \expandafter
  \endgroup
  \childdoctmp
}
%    \end{macrocode}

%\iffalse
%</package>
%\fi
%
\endinput
|\\
|\childdocof{|\textit{main}|}|\\
\end{tabular}
\end{center}
at the top of every child file \textit{child}
which is included by |\include{|\textit{child}|}|
from within the main file
(or at least for those files to be compiled individually).
The argument \textit{main} must be the filename of the main file.

There are a couple of
considerations in setting up the main and child documents:

%%%%%%%%%%%%%%%%%%%%%%%%%%%%%%%%%%%%%%%%
\paragraph{Restrictions.}

Please note the following restrictions:
\begin{itemize}
\item
|\childdocmain| must be called with one argument \textit{main}
to ensure compatibility with earlier version of the package.
It must either be empty (|\childdocmain{}|)
or precisely match the filename of the main file in which it is specified.
See \secref{sec:detection} for further information.
\item
The filename \textit{main} must be specified without the |.tex| extension.
\item
The filename \textit{main} is case sensitive
(even in case-insensitive file systems)
due to internal string comparison.
\item
The argument \textit{main} should be fully expanded, it cannot be a macro.
\item
Subdirectories and special characters should be avoided in filenames.
\item
The command |\childdocmain{|\textit{main}|}| must be followed by a whitespace.
It should not be followed immediately by another command
or by a comment mark `|%|'.
This is because the \TeX{} parser reads the token immediately following
the argument of |\childdocmain| and puts it
at the beginning of every child section;
however, a white\-space is ignored.
\end{itemize}

%%%%%%%%%%%%%%%%%%%%%%%%%%%%%%%%%%%%%%%%
\paragraph{Content of Main File.}

It is advisable to place all content in the child files included by |\include|.
Any output contained in the main file will appear in all child documents
unless suppressed manually;
it cannot be suppressed automatically by the |\includeonly| directive
and thus should normally be avoided.
A method to include some content in the main file
by means of conditional processing is described in \secref{sec:conditional}.

%%%%%%%%%%%%%%%%%%%%%%%%%%%%%%%%%%%%%%%%
\paragraph{Page Numbering.}

When only a part of the document is compiled,
the appropriate numbering of pages
(as well as other status parameters)
is determined from the |.aux| files.
The latter contain information from previous passes.
However this information needs to propagate through
all intermediate child documents.
Therefore the page numbering in child documents may well
be inconsistent until the complete document is compiled at least once.

A useful (if unconventional) way to always ensure a consistent
page numbering is to restart the numbering in each child document
and denote the pages by `\textit{child}|.|\textit{page}'
where \textit{child} represents the chapter/section number of the child file.
This can be achieved by the command
|\numberwithin{page}{|\textit{child}|}|
of the \textsf{amsmath} package
where \textit{child} can be |chapter| or |section|
depending on the chosen structuring.
Alternatively, one can modify the macro |\thepage| appropriately
and reset the counter |page| at the start of each child file.

%%%%%%%%%%%%%%%%%%%%%%%%%%%%%%%%%%%%%%%%%%%%%%%%%%%%%%%%%%%%%%%%%%%%%%%%%%%%%%%%
\subsection{Conditional Processing}
\label{sec:conditional}

The package provides a mechanism to compile different versions
of a document. To customise the versions further some conditional processing
can come in handy to distinguish which version is being compiled.
The package provides two macros to describe the compilation context:

%%%%%%%%%%%%%%%%%%%%%%%%%%%%%%%%%%%%%%%%
\DescribeMacro{\ifchilddoc}
The conditional |\ifchilddoc| distinguishes between the compilation of
child documents and the main document:
%
\begin{center}
|\ifchilddoc |\textit{child-code}| |[|\||else |\textit{main-code}]| \||fi|
\end{center}

%%%%%%%%%%%%%%%%%%%%%%%%%%%%%%%%%%%%%%%%
\DescribeMacro{\childdocname}
\DescribeMacro{\childdocjob}
The macro |\childdocname| contains the filename (without extension)
of the main or child file being processed.
Note that |\childdocjob| will always contain the name of the main file.

%%%%%%%%%%%%%%%%%%%%%%%%%%%%%%%%%%%%%%%%
\paragraph{Title Page.}

Conditional processing can be used to include a title or banner page
in the main document when proper precautions are taken.
Importantly, the code in the main file should ensure that the page counter
(as well as other status parameters which are stored in the |.aux| files)
takes the same value after the conditional processing.
Otherwise the page numbers may take divergent values
depending on which part is compiled.

For example, a title page could be declared by:
%
\begin{center}
\begin{tabular}{l}
|\ifchilddoc\||else|\\
|\addtocounter{page}{-1}|\\
\textit{code for title page}\\
|\newpage|\\
|\||fi|
\end{tabular}
\end{center}
%
A banner page for the child documents can be generated by:
%
\begin{center}
\begin{tabular}{l}
|\ifchilddoc|\\
|\addtocounter{page}{-1}|\\
\textit{code for banner page}\\
|\newpage|\\
|\||fi|
\end{tabular}
\end{center}
%
Here one could write a message such as:
\begin{center}
|This is the part \childdocname{} of \childdocjob{}.|
\end{center}

%%%%%%%%%%%%%%%%%%%%%%%%%%%%%%%%%%%%%%%%%%%%%%%%%%%%%%%%%%%%%%%%%%%%%%%%%%%%%%%%
\subsection{Flags}
\label{sec:flags}

The package makes it easy to generate different versions
of the main or child documents.
To this end compilation flags can be defined
and assigned different default values.
They will be particularly useful in conjunction
with the forwarding mechanism described in \secref{sec:forward}.

For example, it may be useful to have a flag |\version|
which can be set to |draft| or |final|.
The document source will contain some conditional code
depending on the value of |\version|.
Suppose further, the flag should default to |final| for the main file
and to |draft| for child files
which is a natural assignment for editing the document.
This is achieved by placing the following code
in the preamble of the main document
(below the |\childdocmain| directive):
%
\begin{center}
\begin{tabular}{l}
|\ifchilddoc|\\
|\providecommand{\version}{draft}|\\
|\||else|\\
|\providecommand{\version}{final}|\\
|\||fi|
\end{tabular}
\end{center}
%
The definition by |\providecommand| makes sure
that previous definitions are not overwritten.
Further statements |\providecommand{\version}{...}|
can thus be added before the above code to override it.

For the main file, one might add a line
(between |\childdocmain| and the above block)
%
\begin{center}
|%\ifchilddoc\||else\providecommand{\version}{draft}\||fi|
\end{center}
%
which can be uncommented to produce a draft version.
Likewise one can add a line to the very top of a child file
(above the |\childdocof{|\textit{main}|}| directive)
%
\begin{center}
|%\providecommand{\version}{final}|
\end{center}
%
which can be uncommented to produce the final version of this child document.

%%%%%%%%%%%%%%%%%%%%%%%%%%%%%%%%%%%%%%%%%%%%%%%%%%%%%%%%%%%%%%%%%%%%%%%%%%%%%%%%
\subsection{Forwarding}
\label{sec:forward}

Different versions of the main or child documents
using compilation flags as described in \secref{sec:flags}
can be (permanently) stored in different files
for convenient compilation, viewing and distribution.
To this end, the package defines a command
to pass on compilation to a different file:

%%%%%%%%%%%%%%%%%%%%%%%%%%%%%%%%%%%%%%%%
\DescribeMacro{\childdocforward}
The command |\childdocforward| redirects processing to
another source file:
%
\begin{center}
\begin{tabular}{l}
|% \iffalse
%
% childdoc.dtx Copyright (C) 2017-2018 Niklas Beisert
%
% This work may be distributed and/or modified under the
% conditions of the LaTeX Project Public License, either version 1.3
% of this license or (at your option) any later version.
% The latest version of this license is in
%   http://www.latex-project.org/lppl.txt
% and version 1.3 or later is part of all distributions of LaTeX
% version 2005/12/01 or later.
%
% This work has the LPPL maintenance status `maintained'.
%
% The Current Maintainer of this work is Niklas Beisert.
%
% This work consists of the files childdoc.dtx and childdoc.ins
% and the derived files childdoc.def and cdocsamp.tex with
% cdocsch1.tex, cdocsch2.tex, cdocsdrf.tex, cdocsfn1.tex, cdocsfn2.tex.
%
%<package>\ifdefined\childdocmain\endinput\fi
%<package>\ProvidesFile{childdoc.def}[2018/12/30 v2.0 child document driver]
%<samplemain>\ProvidesFile{cdocsamp.tex}[2018/12/30 v2.0 sample for childdoc]
%<*driver>
%\ProvidesFile{childdoc.drv}[2018/12/30 v2.0 childdoc reference manual file]
\PassOptionsToClass{10pt,a4paper}{article}
\documentclass{ltxdoc}

\usepackage[margin=35mm]{geometry}
\usepackage{hyperref}
\usepackage{hyperxmp}
\usepackage[usenames]{color}

\hypersetup{colorlinks=true}
\hypersetup{pdfstartview=FitH}
\hypersetup{pdfpagemode=UseNone}
\hypersetup{pdfsource={}}
\hypersetup{pdflang={en-UK}}
\hypersetup{pdfcopyright={Copyright 2017-2018 Niklas Beisert.
  This work may be distributed and/or modified under the
  conditions of the LaTeX Project Public License, either version 1.3
  of this license or (at your option) any later version.}}
\hypersetup{pdflicenseurl={http://www.latex-project.org/lppl.txt}}
\hypersetup{pdfcontactaddress={ETH Zurich, ITP, HIT K,
  Wolfgang-Pauli-Strasse 27}}
\hypersetup{pdfcontactpostcode={8093}}
\hypersetup{pdfcontactcity={Zurich}}
\hypersetup{pdfcontactcountry={Switzerland}}
\hypersetup{pdfcontactemail={nbeisert@itp.phys.ethz.ch}}
\hypersetup{pdfcontacturl={http://people.phys.ethz.ch/\xmptilde nbeisert/}}

\newcommand{\secref}[1]{\hyperref[#1]{section \ref*{#1}}}

\parskip1ex
\parindent0pt
\let\olditemize\itemize
\def\itemize{\olditemize\parskip0pt}

\begin{document}

\title{The \textsf{childdoc} Package}
\hypersetup{pdftitle={The childdoc Package}}
\author{Niklas Beisert\\[2ex]
  Institut f\"ur Theoretische Physik\\
  Eidgen\"ossische Technische Hochschule Z\"urich\\
  Wolfgang-Pauli-Strasse 27, 8093 Z\"urich, Switzerland\\[1ex]
  \href{mailto:nbeisert@itp.phys.ethz.ch}
  {\texttt{nbeisert@itp.phys.ethz.ch}}}
\hypersetup{pdfauthor={Niklas Beisert}}
\hypersetup{pdfsubject={Manual for the LaTeX2e Package childdoc}}
\date{30 December 2018, \textsf{v2.0}}
\maketitle

\begin{abstract}\noindent
\textsf{childdoc} is a \LaTeXe{} package
that enables the direct compilation
of document sections included by |\include|
to individual files.
\end{abstract}

\begingroup
\parskip0ex
\tableofcontents
\endgroup

%%%%%%%%%%%%%%%%%%%%%%%%%%%%%%%%%%%%%%%%%%%%%%%%%%%%%%%%%%%%%%%%%%%%%%%%%%%%%%%%
%%%%%%%%%%%%%%%%%%%%%%%%%%%%%%%%%%%%%%%%%%%%%%%%%%%%%%%%%%%%%%%%%%%%%%%%%%%%%%%%
\section{Introduction}

\LaTeX{} provides a mechanism to structure a large document (such as a book)
into a main file and several child files (containing the chapters)
using the |\include| command.
This mechanism is beneficial for documents
which span hundreds of pages in order to
make the source file(s) more manageable.
Moreover, compilation can be restricted to
selected child files by means of the |\includeonly| command.
The latter feature can be used to reduce the compilation time while editing
(this was significantly more useful in the earlier days of \LaTeX{})
or to generate a smaller document which is easier to navigate.
Another application of |\includeonly| is to generate
documents consisting of selected parts of the complete document.

However, there are a few drawbacks of the plain |\include| mechanism:
\begin{itemize}
\item
The child files cannot be compiled on their own,
they can only be compiled via the main file.
A naive editing environment
(such as a text editor with an option
to have the current file processed by \LaTeX)
may require one to switch to the main file before compiling;
attempting to compile the child file produces errors.
\item
The main file must be modified (each time)
to adjust the |\includeonly| command
to the present needs. This easily leaves the main file in a messy state.
\item
The generated document will always carry the filename
of the main document. This is inconvenient if
several child files are to be compiled and
to be kept for distribution.
\end{itemize}

The present package provides a simple interface
to make child files individually compilable by \LaTeX{}.
Compiling a child file then has the same effect as compiling
the main file with an |\includeonly| command
to select the appropriate child.
Moreover the generated document will carry the name of the child
rather than the main file.
This resolves all three above issues.

This feature is meant to make the editing of books,
thesis documents and lecture notes somewhat more convenient.
However, the package can also be used efficiently for
composing a series of documents (such as exercise sheets)
which are typically distributed individually.
It then assists the author in generating the individual documents
(potentially in different versions)
as well as a document containing the collected series.
Another application is in developing style files
or other kinds of included material
where compilation of the style file could redirect
to a sample or test file.

%%%%%%%%%%%%%%%%%%%%%%%%%%%%%%%%%%%%%%%%%%%%%%%%%%%%%%%%%%%%%%%%%%%%%%%%%%%%%%%%
%%%%%%%%%%%%%%%%%%%%%%%%%%%%%%%%%%%%%%%%%%%%%%%%%%%%%%%%%%%%%%%%%%%%%%%%%%%%%%%%
\section{Usage}

First of all, the package \textsf{childdoc} is \emph{not} a standard
\LaTeXe{} |.sty| style file! Therefore it needs to be invoked in
a non-standard way.

%%%%%%%%%%%%%%%%%%%%%%%%%%%%%%%%%%%%%%%%%%%%%%%%%%%%%%%%%%%%%%%%%%%%%%%%%%%%%%%%
\subsection{Included Files}
\label{sec:include}

%%%%%%%%%%%%%%%%%%%%%%%%%%%%%%%%%%%%%%%%
\DescribeMacro{\childdocmain}
To use the package, add the commands
\begin{center}
\begin{tabular}{l}
|\input{childdoc.def}|\\
|\childdocmain{}|\\
\end{tabular}
\end{center}
at the very top of the main \LaTeX{} file,
in particular \emph{before} the |\documentclass| statement!
The argument of |\childdocmain| should be left empty
(but it must be present).

%%%%%%%%%%%%%%%%%%%%%%%%%%%%%%%%%%%%%%%%
\DescribeMacro{\childdocof}
Furthermore, add the commands
\begin{center}
\begin{tabular}{l}
|\input{childdoc.def}|\\
|\childdocof{|\textit{main}|}|\\
\end{tabular}
\end{center}
at the top of every child file \textit{child}
which is included by |\include{|\textit{child}|}|
from within the main file
(or at least for those files to be compiled individually).
The argument \textit{main} must be the filename of the main file.

There are a couple of
considerations in setting up the main and child documents:

%%%%%%%%%%%%%%%%%%%%%%%%%%%%%%%%%%%%%%%%
\paragraph{Restrictions.}

Please note the following restrictions:
\begin{itemize}
\item
|\childdocmain| must be called with one argument \textit{main}
to ensure compatibility with earlier version of the package.
It must either be empty (|\childdocmain{}|)
or precisely match the filename of the main file in which it is specified.
See \secref{sec:detection} for further information.
\item
The filename \textit{main} must be specified without the |.tex| extension.
\item
The filename \textit{main} is case sensitive
(even in case-insensitive file systems)
due to internal string comparison.
\item
The argument \textit{main} should be fully expanded, it cannot be a macro.
\item
Subdirectories and special characters should be avoided in filenames.
\item
The command |\childdocmain{|\textit{main}|}| must be followed by a whitespace.
It should not be followed immediately by another command
or by a comment mark `|%|'.
This is because the \TeX{} parser reads the token immediately following
the argument of |\childdocmain| and puts it
at the beginning of every child section;
however, a white\-space is ignored.
\end{itemize}

%%%%%%%%%%%%%%%%%%%%%%%%%%%%%%%%%%%%%%%%
\paragraph{Content of Main File.}

It is advisable to place all content in the child files included by |\include|.
Any output contained in the main file will appear in all child documents
unless suppressed manually;
it cannot be suppressed automatically by the |\includeonly| directive
and thus should normally be avoided.
A method to include some content in the main file
by means of conditional processing is described in \secref{sec:conditional}.

%%%%%%%%%%%%%%%%%%%%%%%%%%%%%%%%%%%%%%%%
\paragraph{Page Numbering.}

When only a part of the document is compiled,
the appropriate numbering of pages
(as well as other status parameters)
is determined from the |.aux| files.
The latter contain information from previous passes.
However this information needs to propagate through
all intermediate child documents.
Therefore the page numbering in child documents may well
be inconsistent until the complete document is compiled at least once.

A useful (if unconventional) way to always ensure a consistent
page numbering is to restart the numbering in each child document
and denote the pages by `\textit{child}|.|\textit{page}'
where \textit{child} represents the chapter/section number of the child file.
This can be achieved by the command
|\numberwithin{page}{|\textit{child}|}|
of the \textsf{amsmath} package
where \textit{child} can be |chapter| or |section|
depending on the chosen structuring.
Alternatively, one can modify the macro |\thepage| appropriately
and reset the counter |page| at the start of each child file.

%%%%%%%%%%%%%%%%%%%%%%%%%%%%%%%%%%%%%%%%%%%%%%%%%%%%%%%%%%%%%%%%%%%%%%%%%%%%%%%%
\subsection{Conditional Processing}
\label{sec:conditional}

The package provides a mechanism to compile different versions
of a document. To customise the versions further some conditional processing
can come in handy to distinguish which version is being compiled.
The package provides two macros to describe the compilation context:

%%%%%%%%%%%%%%%%%%%%%%%%%%%%%%%%%%%%%%%%
\DescribeMacro{\ifchilddoc}
The conditional |\ifchilddoc| distinguishes between the compilation of
child documents and the main document:
%
\begin{center}
|\ifchilddoc |\textit{child-code}| |[|\||else |\textit{main-code}]| \||fi|
\end{center}

%%%%%%%%%%%%%%%%%%%%%%%%%%%%%%%%%%%%%%%%
\DescribeMacro{\childdocname}
\DescribeMacro{\childdocjob}
The macro |\childdocname| contains the filename (without extension)
of the main or child file being processed.
Note that |\childdocjob| will always contain the name of the main file.

%%%%%%%%%%%%%%%%%%%%%%%%%%%%%%%%%%%%%%%%
\paragraph{Title Page.}

Conditional processing can be used to include a title or banner page
in the main document when proper precautions are taken.
Importantly, the code in the main file should ensure that the page counter
(as well as other status parameters which are stored in the |.aux| files)
takes the same value after the conditional processing.
Otherwise the page numbers may take divergent values
depending on which part is compiled.

For example, a title page could be declared by:
%
\begin{center}
\begin{tabular}{l}
|\ifchilddoc\||else|\\
|\addtocounter{page}{-1}|\\
\textit{code for title page}\\
|\newpage|\\
|\||fi|
\end{tabular}
\end{center}
%
A banner page for the child documents can be generated by:
%
\begin{center}
\begin{tabular}{l}
|\ifchilddoc|\\
|\addtocounter{page}{-1}|\\
\textit{code for banner page}\\
|\newpage|\\
|\||fi|
\end{tabular}
\end{center}
%
Here one could write a message such as:
\begin{center}
|This is the part \childdocname{} of \childdocjob{}.|
\end{center}

%%%%%%%%%%%%%%%%%%%%%%%%%%%%%%%%%%%%%%%%%%%%%%%%%%%%%%%%%%%%%%%%%%%%%%%%%%%%%%%%
\subsection{Flags}
\label{sec:flags}

The package makes it easy to generate different versions
of the main or child documents.
To this end compilation flags can be defined
and assigned different default values.
They will be particularly useful in conjunction
with the forwarding mechanism described in \secref{sec:forward}.

For example, it may be useful to have a flag |\version|
which can be set to |draft| or |final|.
The document source will contain some conditional code
depending on the value of |\version|.
Suppose further, the flag should default to |final| for the main file
and to |draft| for child files
which is a natural assignment for editing the document.
This is achieved by placing the following code
in the preamble of the main document
(below the |\childdocmain| directive):
%
\begin{center}
\begin{tabular}{l}
|\ifchilddoc|\\
|\providecommand{\version}{draft}|\\
|\||else|\\
|\providecommand{\version}{final}|\\
|\||fi|
\end{tabular}
\end{center}
%
The definition by |\providecommand| makes sure
that previous definitions are not overwritten.
Further statements |\providecommand{\version}{...}|
can thus be added before the above code to override it.

For the main file, one might add a line
(between |\childdocmain| and the above block)
%
\begin{center}
|%\ifchilddoc\||else\providecommand{\version}{draft}\||fi|
\end{center}
%
which can be uncommented to produce a draft version.
Likewise one can add a line to the very top of a child file
(above the |\childdocof{|\textit{main}|}| directive)
%
\begin{center}
|%\providecommand{\version}{final}|
\end{center}
%
which can be uncommented to produce the final version of this child document.

%%%%%%%%%%%%%%%%%%%%%%%%%%%%%%%%%%%%%%%%%%%%%%%%%%%%%%%%%%%%%%%%%%%%%%%%%%%%%%%%
\subsection{Forwarding}
\label{sec:forward}

Different versions of the main or child documents
using compilation flags as described in \secref{sec:flags}
can be (permanently) stored in different files
for convenient compilation, viewing and distribution.
To this end, the package defines a command
to pass on compilation to a different file:

%%%%%%%%%%%%%%%%%%%%%%%%%%%%%%%%%%%%%%%%
\DescribeMacro{\childdocforward}
The command |\childdocforward| redirects processing to
another source file:
%
\begin{center}
\begin{tabular}{l}
|\input{childdoc.def}|\\
|\childdocforward[|\textit{main}|]{|\textit{dest}|}|\\
\end{tabular}
\end{center}
%
The argument \textit{dest} is the destination file
(without extension).
It should be the main file or one of the child files.
Note that further \textsf{childdoc} directives
such as |\childdocof| and |\childdocforward|
in the indicated file will be processed in this form.
The optional argument \textit{main}
passes on directly to the main file \textit{main}
while pretending to compile the child \textit{dest}.
This form behaves as if \textit{dest}
issues |\childdocof{|\textit{main}|}| right away,
and no further \textsf{childdoc} directives will be processed.

%%%%%%%%%%%%%%%%%%%%%%%%%%%%%%%%%%%%%%%%
\DescribeMacro{\...prefix}
In the alternative form |\childdocforwardprefix|,
%
\begin{center}
\begin{tabular}{l}
|\input{childdoc.def}|\\
|\childdocforwardprefix[|\textit{main}|]{|\textit{prefix}|}{|\textit{dest}|}|
\end{tabular}
\end{center}
%
the destination file is determined by a pattern
depending on the current file:
To make this work, the current file must be called
`{\textit{prefix}\hspace{0.2em}\textit{suffix}}'
with \textit{prefix} matching precisely the argument.
Processing is then passed on to the file
`{\textit{dest}\hspace{0.2em}\textit{suffix}}'.
Surely, the same effect is achieved by
directly specifying the
argument `{\textit{dest}\hspace{0.2em}\textit{suffix}}'
in the first form.
However, that requires to set up a different file
for each child. With the alternative form of the command
all these files can have exactly the same content
which simplifies setting them up and maintaining them.

For example, the following file |draft.tex|
with a compilation flag |\version| as described in \secref{sec:flags}
compiles the main document as a draft:
%
\begin{center}
\begin{tabular}{l}
|\def\version{draft}|\\
|\input{childdoc.def}|\\
|\childdocforward{|\textit{main}|}|
\end{tabular}
\end{center}
%
Likewise, the following files |final|\textit{nn}|.tex|
compile the final version of the child document
|child|\textit{nn}|.tex|:
%
\begin{center}
\begin{tabular}{l}
|\def\version{final}|\\
|\input{childdoc.def}|\\
|\childdocforwardprefix{final}{child}|
\end{tabular}
\end{center}
%

Note that when several versions of a main file and/or of each child file
are to be generated, it may be convenient to set up a |Makefile| or
shell script to automatise the process.

%%%%%%%%%%%%%%%%%%%%%%%%%%%%%%%%%%%%%%%%%%%%%%%%%%%%%%%%%%%%%%%%%%%%%%%%%%%%%%%%
\subsection{Command Line Processing}
\label{sec:commandline}

The effect of redirection files can also be achieved by invoking
the \LaTeX{} compiler with a more elaborate command line.
Most conveniently this should be done as part
of a shell script or a |Makefile|.

When using \textsf{childdoc} in the main file, the following
command lines effectively perform a redirection
(note that depending on the shell being used,
backslashes may have to be doubled: `|\|' $\to$ `|\\|'):
%
\begin{center}
|... -jobname "|\textit{target}|" |\\|"|[\textit{flags}]%
|\input{childdoc.def}\childdocforward[|\textit{main}|]{|\textit{dest}|}"|
\end{center}
%
Here \textit{target} is the name of the output file,
\textit{main} is the name of the main file
and \textit{dest} is the name of the main or child file to be processed
(all filenames without extensions).
The optional argument \textit{main} can be omitted
if \textit{main} matches \textit{dest}.
Optionally, compilation \textit{flags} can be defined via |\def| commands.
This command line makes the \TeX{} engine believe
it is compiling the file \textit{target}
whose content is specified as the latter parameter.
The provided code then forwards the processing to
\textit{main} or \textit{dest} as described in \secref{sec:forward}.

%%%%%%%%%%%%%%%%%%%%%%%%%%%%%%%%%%%%%%%%%%%%%%%%%%%%%%%%%%%%%%%%%%%%%%%%%%%%%%%%
\subsection{Include by Input}
\label{sec:input}

Including child documents by |\include| has some restrictions by design.
Most notably, the content of a child document always occupies
its own set of pages; pages cannot be shared between child documents.
Usually, this behaviour makes perfect sense
because each child document contain an essential part of the document.
However, in some situations it may be desirable to compose
a document from a collection of parts
without having mandatory page breaks between then.
For this case, the package
provides a mechanism to include parts
by |\input| which can also be processed individually.
However, by construction this mechanism
requires manual handling of the content to be output.

%%%%%%%%%%%%%%%%%%%%%%%%%%%%%%%%%%%%%%%%
\DescribeMacro{\ifchilddocmanual}
The main file should be prepared as usual, see \secref{sec:include}.
However, the document body must make a distinction
between processing of an individual part and of the main document, e.g.:
%
\begin{center}
\begin{tabular}{l}
|\ifchilddocmanual|\\
|\input{\childdocname}|\\
|\||else|\\
\textit{document body with }|\input{|\textit{part}|}|\\
|\||fi|
\end{tabular}
\end{center}
%
The conditional |\ifchilddocmanual| is true whenever
a part to be included by |\input| is being compiled,
and the name of the part is stored in |\childdocname|.

%%%%%%%%%%%%%%%%%%%%%%%%%%%%%%%%%%%%%%%%
\DescribeMacro{\childdocby}
Each part to be included by |\input| should start with:
%
\begin{center}
\begin{tabular}{l}
|\input{childdoc.def}|\\
|\childdocby{|\textit{main}|}|\\
\end{tabular}
\end{center}
%
The directive |\childdocby| is similar to |\childdocof|
described in \secref{sec:include},
but the subsequent selection of content must be done manually.
To that end, both |\ifchilddoc| and |\ifchilddocmanual|
will be true upon processing of a part,
and the name of the part is stored in |\childdocname|.
Note that |\jobname| will be set to the filename of the current part
so that each part receives an individual |.aux| file
that does not interfere with the |.aux| file(s) of the main document.
This behaviour can be altered by the alternative form
|\childdocby[*]{|\textit{main}|}| (with a non-empty optional argument)
which uses the |.aux| file of the main document
by setting |\jobname| to \textit{main}.

%%%%%%%%%%%%%%%%%%%%%%%%%%%%%%%%%%%%%%%%%%%%%%%%%%%%%%%%%%%%%%%%%%%%%%%%%%%%%%%%
\subsection{Driver Development}
\label{sec:driver}

The \textsf{childdoc} mechanism can also be use for the development
of definition files such as \LaTeX{} styles or classes.
This case differs from the above setup with multiple parts
included by |\include| in that no |\includeonly| should be invoked.
This can be achieved by starting the include file
(before |\ProvidesPackage|) with:
%
\begin{center}
\begin{tabular}{l}
|\input{childdoc.def}|\\
|\childdocforward{|\textit{main}|}|\\
\end{tabular}
\end{center}
%
or alternatively with:
%
\begin{center}
\begin{tabular}{l}
|\input{childdoc.def}|\\
|\childdocby{|\textit{main}|}|\\
\end{tabular}
\end{center}
%
Both forms have slightly different effects as described above.
The main file is prepared as usual, see \secref{sec:include}.

%%%%%%%%%%%%%%%%%%%%%%%%%%%%%%%%%%%%%%%%%%%%%%%%%%%%%%%%%%%%%%%%%%%%%%%%%%%%%%%%
\subsection{Legacy Detection}
\label{sec:detection}

The directive |\childdocmain| in the main file can detect
whether the complete document or merely a child is to be compiled
even without using the directive |\childdocof|.
This method is deprecated because it is less robust
and there is no compelling reason to use it;
it is merely provided for backward compatibility
and it may be removed in future versions.

If the detection mechanism is to be used,
it is mandatory to correctly specify
the filename of the main file as the argument of |\childdocmain|:
%
\begin{center}
\begin{tabular}{l}
|\input{childdoc.def}|\\
|\childdocmain{|\textit{main}|}|\\
\end{tabular}
\end{center}
%
If |\jobname| does not match the argument \textit{main} of |\childdocmain|,
it is assumed that |\jobname| points to the child file to be compiled.
When using |\childdocmain| with the main file specified as argument,
it suffices to start a child file
with just |\input{|\textit{main}|}|
without loading of the package and using |\childdocof|.
If instead all processing is done
with the appropriate \textsf{childdoc} directives,
the argument of \textit{main} of |\childdocmain| can be empty.

An alternative version of the command line processing described
in \secref{sec:commandline} using the detection mechanism reads:
%
\begin{center}
|... -jobname "|\textit{target}|" "|[\textit{flags}]%
[|\def\jobname{|\textit{dest}|}|]|\input{|\textit{main}|}"|
\end{center}

%%%%%%%%%%%%%%%%%%%%%%%%%%%%%%%%%%%%%%%%%%%%%%%%%%%%%%%%%%%%%%%%%%%%%%%%%%%%%%%%
\subsection{Manual Code}
\label{sec:manual}

In case one cannot be certain whether the definitions file |childdoc.def|
is installed on the target \TeX{} distribution
and one prefers not to ship it,
it is conceivable to paste a few relevant commands into the sources.

To that end, drop all statements |\input{childdoc.def}|
and perform the replacements as outlined below.
Instead of |\childdocmain{|\textit{main}|}| add the following code
to the top of the main file:
%
\begin{center}
\begin{tabular}{l}
|\||ifdefined\childdocname\endinput\||fi\newif\ifchilddoc|\\
|\edef\childdocname{\scantokens\expandafter{\jobname\noexpand}}|\\
|\def\childdocmain{|\textit{main}|}\||ifx\childdocmain\childdocname\||else|\\
|\childdoctrue\includeonly{\childdocname}\let\jobname\childdocmain\||fi|\\
\end{tabular}
\end{center}
%
Instead of |\childdocof{|\textit{main}|}| just include the main file
at the top of each child file:
%
\begin{center}
|\input{|\textit{main}|}|
\end{center}
%
A simple redirection |\childdocforward{|\textit{dest}|}| is achieved by:
%
\begin{center}
|\def\jobname{|\textit{dest}|}\input{\jobname}|
\end{center}
%
The redirection with prefix
|\childdocforwardprefix[|\textit{prefix}|]{|\textit{dest}|}|
is accomplished by:
%
\begin{center}
\begin{tabular}{l}
|{\edef\jobname{\scantokens\expandafter{\jobname\noexpand}}|\\
|\def\redirectjob |\textit{prefix}|#1~~~{\gdef\jobname{|\textit{dest}|#1}}|\\
|\expandafter\redirectjob\jobname~~~}\input{\jobname}|
\end{tabular}
\end{center}

In an alternative approach,
child documents can be compiled by a specific command line
without additional code or specific definitions:
%
\begin{center}
|... -jobname "|\textit{target}|" "|[\textit{flags}]%
|\includeonly{|\textit{dest}|}\input{|\textit{main}|}"|
\end{center}
%

%%%%%%%%%%%%%%%%%%%%%%%%%%%%%%%%%%%%%%%%%%%%%%%%%%%%%%%%%%%%%%%%%%%%%%%%%%%%%%%%
%%%%%%%%%%%%%%%%%%%%%%%%%%%%%%%%%%%%%%%%%%%%%%%%%%%%%%%%%%%%%%%%%%%%%%%%%%%%%%%%
\section{Information}

%%%%%%%%%%%%%%%%%%%%%%%%%%%%%%%%%%%%%%%%%%%%%%%%%%%%%%%%%%%%%%%%%%%%%%%%%%%%%%%%
\subsection{Copyright}

Copyright \copyright{} 2017--2018 Niklas Beisert

This work may be distributed and/or modified under the
conditions of the \LaTeX{} Project Public License, either version 1.3
of this license or (at your option) any later version.
The latest version of this license is in
  \url{http://www.latex-project.org/lppl.txt}
and version 1.3 or later is part of all distributions of \LaTeX{}
version 2005/12/01 or later.

This work has the LPPL maintenance status `maintained'.

The Current Maintainer of this work is Niklas Beisert.

This work consists of the files |README.txt|, |childdoc.ins| and |childdoc.dtx|
as well as the derived files |childdoc.def|, |cdocsamp.tex|
with |cdocsch1.tex|, |cdocsch2.tex|, |cdocspt3.tex|, |cdocspt4.tex|,
|cdocsdrf.tex|, |cdocsfn1.tex|, |cdocsfn2.tex|
as well as |childdoc.pdf|.

%%%%%%%%%%%%%%%%%%%%%%%%%%%%%%%%%%%%%%%%%%%%%%%%%%%%%%%%%%%%%%%%%%%%%%%%%%%%%%%%
\subsection{Files and Installation}

The package consists of the files:
%
\begin{center}
\begin{tabular}{ll}
    |README.txt|   & readme file \\
    |childdoc.ins| & installation file \\
    |childdoc.dtx| & source file \\
    |childdoc.def| & definition file \\
    |cdocsamp.tex| & sample main file \\
    |cdocsch1.tex| & sample include file \\
    |cdocsch2.tex| & sample include file \\
    |cdocspt3.tex| & sample part file \\
    |cdocspt4.tex| & sample part file \\
    |cdocsdrf.tex| & sample redirection file \\
    |cdocsfn1.tex| & sample redirection file \\
    |cdocsfn2.tex| & sample redirection file \\
    |childdoc.pdf| & manual
\end{tabular}
\end{center}
%
The distribution consists of the files
|README.txt|, |childdoc.ins| and |childdoc.dtx|.
%
\begin{itemize}
\item
Run (pdf)\LaTeX{} on |childdoc.dtx|
to compile the manual |childdoc.pdf| (this file).
\item
Run \LaTeX{} on |childdoc.ins| to create the definitions file |childdoc.def|
and the sample |cdocsamp.tex| with include files
|cdocsch1.tex|, |cdocsch2.tex|, |cdocspt3.tex|, |cdocspt4.tex|,
|cdocsdrf.tex|, |cdocsfn1.tex|, |cdocsfn2.tex|.
Then copy the file |childdoc.def| to an appropriate directory of your \LaTeX{}
distribution, e.g.\ \textit{texmf-root}|/tex/latex/childdoc|.
\end{itemize}

%%%%%%%%%%%%%%%%%%%%%%%%%%%%%%%%%%%%%%%%%%%%%%%%%%%%%%%%%%%%%%%%%%%%%%%%%%%%%%%%
\subsection{Related CTAN Packages}

There are several other packages which offer a similar functionality:
%
\begin{itemize}
\item
The packages
\href{http://ctan.org/pkg/docmute}{\textsf{docmute}},
\href{http://ctan.org/pkg/includex}{\textsf{includex}} and
\href{http://ctan.org/pkg/standalone}{\textsf{standalone}}
provide commands to include only the document body of
a child file thus allowing both files to be compiled individually.
\item
The packages \href{http://ctan.org/pkg/subdocs}{\textsf{subdocs}}
and \href{http://ctan.org/pkg/subfiles}{\textsf{subfiles}}
provide structures in which the main and child documents can be
encapsulated and allowing them to be compiled individually.
The inclusion mechanism is different from the conventional |\include|.
\item
The package \href{http://ctan.org/pkg/combine}{\textsf{combine}}
is an elaborate solution to combine several documents into one.
\end{itemize}
%
See also the CTAN topic \href{http://ctan.org/topic/subdocs}{\textsf{subdocs}}
for further related packages.
The present package differs from the above solutions in that
a document structure constructed with the conventional |\include| mechanism
just needs two extra commands at the top of every file
such that all constituent files can be compiled individually.

%%%%%%%%%%%%%%%%%%%%%%%%%%%%%%%%%%%%%%%%%%%%%%%%%%%%%%%%%%%%%%%%%%%%%%%%%%%%%%%%
%\subsection{Feature Suggestions}
%
%The following is a list of features which may be useful for future
%versions of this package:
%%
%\begin{itemize}
%\item
%\ldots
%\end{itemize}

%%%%%%%%%%%%%%%%%%%%%%%%%%%%%%%%%%%%%%%%%%%%%%%%%%%%%%%%%%%%%%%%%%%%%%%%%%%%%%%%
\subsection{Revision History}

%%%%%%%%%%%%%%%%%%%%%%%%%%%%%%%%%%%%%%%%
\paragraph{v2.0:} 2018/12/30

\begin{itemize}
\item
immediate forward processing
\item
added |\childdocby| mechanism
\item
manual restructured
\end{itemize}

%%%%%%%%%%%%%%%%%%%%%%%%%%%%%%%%%%%%%%%%
\paragraph{v1.6:} 2018/01/17

\begin{itemize}
\item
application for development of include files
\item
corrections to manual
\end{itemize}

%%%%%%%%%%%%%%%%%%%%%%%%%%%%%%%%%%%%%%%%
\paragraph{v1.5:} 2017/05/21

\begin{itemize}
\item
more complete structuring introduced
\item
|\childdocof| introduced
\item
|\childdoc| renamed to |\childdocmain|
\item
|\childredirect| renamed to |\childdocforward| and |\childdocforwardprefix|
and functionality expanded
\end{itemize}

%%%%%%%%%%%%%%%%%%%%%%%%%%%%%%%%%%%%%%%%
\paragraph{v1.0:} 2017/04/27

\begin{itemize}
\item
manual and install package
\item
first version published on CTAN
\end{itemize}

%%%%%%%%%%%%%%%%%%%%%%%%%%%%%%%%%%%%%%%%
\paragraph{v0.6:} 2017/04/26

\begin{itemize}
\item
redirection mechanism added
\end{itemize}

%%%%%%%%%%%%%%%%%%%%%%%%%%%%%%%%%%%%%%%%
\paragraph{v0.5:} 2017/04/26

\begin{itemize}
\item
functionality in definition file
\end{itemize}


%%%%%%%%%%%%%%%%%%%%%%%%%%%%%%%%%%%%%%%%%%%%%%%%%%%%%%%%%%%%%%%%%%%%%%%%%%%%%%%%
%%%%%%%%%%%%%%%%%%%%%%%%%%%%%%%%%%%%%%%%%%%%%%%%%%%%%%%%%%%%%%%%%%%%%%%%%%%%%%%%
%%%%%%%%%%%%%%%%%%%%%%%%%%%%%%%%%%%%%%%%%%%%%%%%%%%%%%%%%%%%%%%%%%%%%%%%%%%%%%%%
\appendix

\settowidth\MacroIndent{\rmfamily\scriptsize 000\ }

 \DocInput{childdoc.dtx}

\end{document}
%</driver>
% \fi
%
% %%%%%%%%%%%%%%%%%%%%%%%%%%%%%%%%%%%%%%%%%%%%%%%%%%%%%%%%%%%%%%%%%%%%%%%%%%%%%%
% %%%%%%%%%%%%%%%%%%%%%%%%%%%%%%%%%%%%%%%%%%%%%%%%%%%%%%%%%%%%%%%%%%%%%%%%%%%%%%
% \section{Sample}
%\iffalse
%<*samplemain>
%\fi
%
% The following presents a sample document
% with two chapters, two parts, a title page,
% a compile flag as well as three forwarding files to set the flag.
% It consists of eight |.tex| files:
% \begin{center}
% \begin{tabular}{ll}
% |cdocsamp.tex|&main file\\
% |cdocsch1.tex|&include file for chapter 1\\
% |cdocsch2.tex|&include file for chapter 2\\
% |cdocspt3.tex|&include file for part 3\\
% |cdocspt4.tex|&include file for part 4\\
% |cdocsdrf.tex|&forwarding file for main file in draft mode\\
% |cdocsfi1.tex|&forwarding file for final version of chapter 1\\
% |cdocsfi2.tex|&forwarding file for final version of chapter 2\\
% \end{tabular}
% \end{center}
% Each of the eight files can be compiled directly by the \LaTeX{} compiler.
%
% %%%%%%%%%%%%%%%%%%%%%%%%%%%%%%%%%%%%%%
% \paragraph{Main File.}
%
% The main file is called |cdocsamp.tex|.
%
% Load the \textsf{childdoc} definitions and
% declare the filename for the main document:
%    \begin{macrocode}
\input{childdoc.def}
\childdocmain{}
%    \end{macrocode}

% Optional override for |\version| flag:
%    \begin{macrocode}
%%\ifchilddoc\else\providecommand{\version}{draft}\fi
%    \end{macrocode}

% Define the default values for the |\version| flag
% (|final| for the main file and |draft| for childs):
%    \begin{macrocode}
\ifchilddoc
\providecommand{\version}{draft}
\else
\providecommand{\version}{final}
\fi
%    \end{macrocode}

% Load the standard document class:
%    \begin{macrocode}
\documentclass[12pt]{article}
%    \end{macrocode}

% Start the document body:
%    \begin{macrocode}
\begin{document}
%    \end{macrocode}

% Declare a title page.
% Print title, part of document being processed and version flag:
%    \begin{macrocode}
\addtocounter{page}{-1}
\begin{center}
{\LARGE\bfseries{}childdoc example\par}
\vspace{1cm}
\ifchilddoc
\ifchilddocmanual part\else chapter\fi:
`\childdocname' of `\childdocjob'\par
\else
main document: `\childdocjob'\par
\fi
version: \version\par
\end{center}
\newpage
%    \end{macrocode}

% Manually include selected file,
% otherwise process as usual:
%    \begin{macrocode}
\ifchilddocmanual
\section*{part `\childdocname'}
\input{\childdocname}
\else
%    \end{macrocode}

% Include the two chapters:
%    \begin{macrocode}
\include{cdocsch1}
\include{cdocsch2}
%    \end{macrocode}

% Include the two parts unless only chapters should be displayed:
%    \begin{macrocode}
\ifchilddoc\else
\section{part three}
\input{cdocspt3}
\section{part four}
\input{cdocspt4}
\fi
%    \end{macrocode}

% Process as usual until here:
%    \begin{macrocode}
\fi
%    \end{macrocode}

% End of document body:
%    \begin{macrocode}
\end{document}
%    \end{macrocode}
%\iffalse
%</samplemain>
%\fi
%
% %%%%%%%%%%%%%%%%%%%%%%%%%%%%%%%%%%%%%%
% \paragraph{Chapter Include Files.}
%
% The include files are called |cdocsch1.tex| and |cdocsch2.tex|.
%
%\iffalse
%<*samplechap1|samplechap2>
%\fi

% Optional override for |\version| flag:
%    \begin{macrocode}
%%\providecommand{\version}{final}
%    \end{macrocode}

% Include the main document:
%    \begin{macrocode}
\input{childdoc.def}
\childdocof{cdocsamp}
%    \end{macrocode}

%\iffalse
%</samplechap1|samplechap2>
%\fi
%
%\iffalse
%<*samplechap1>
%\fi
% Some text for chapter 1:
%    \begin{macrocode}
\section{one}
some text in chapter one
%    \end{macrocode}

%\iffalse
%</samplechap1>
%\fi
% Some text for chapter 2:
%\iffalse
%<*samplechap2>
%\fi
%    \begin{macrocode}
\section{two}
more text in chapter two
%    \end{macrocode}

%\iffalse
%</samplechap2>
%\fi
%
% %%%%%%%%%%%%%%%%%%%%%%%%%%%%%%%%%%%%%%
% \paragraph{Part Include Files.}
%
% The include files are called |cdocspt3.tex| and |cdocspt4.tex|.
%
%\iffalse
%<*samplepart3|samplepart4>
%\fi

% Optional override for |\version| flag:
%    \begin{macrocode}
%%\providecommand{\version}{final}
%    \end{macrocode}

% Include the main document:
%    \begin{macrocode}
\input{childdoc.def}
\childdocby{cdocsamp}
%    \end{macrocode}

%\iffalse
%</samplepart3|samplepart4>
%\fi
%
%\iffalse
%<*samplepart3>
%\fi
% Some text for part 3:
%    \begin{macrocode}
some text in part three
%    \end{macrocode}

%\iffalse
%</samplepart3>
%\fi
% Some text for part 4:
%\iffalse
%<*samplepart4>
%\fi
%    \begin{macrocode}
more text in part four
%    \end{macrocode}

%\iffalse
%</samplepart4>
%\fi
%
% %%%%%%%%%%%%%%%%%%%%%%%%%%%%%%%%%%%%%%
% \paragraph{Forwarding for a Complete Draft.}
%
% The following forwarding file |cdocsdrf.tex|
% compiles the main document in draft mode:
%\iffalse
%<*sampledraft>
%\fi
%    \begin{macrocode}
\def\version{draft}
\input{childdoc.def}
\childdocforward{cdocsamp}
%    \end{macrocode}

%\iffalse
%</sampledraft>
%\fi
%
% %%%%%%%%%%%%%%%%%%%%%%%%%%%%%%%%%%%%%%
% \paragraph{Forwarding for Final Version of the Chapters.}
%
% The following forwarding files |cdocsfn1.tex| and |cdocsfn2.tex|
% (with identical content)
% compile the final versions of the child documents
% |cdocsch1.tex| and |cdocsch2.tex|, respectively:
%\iffalse
%<*samplefinal>
%\fi
%    \begin{macrocode}
\def\version{final}
\input{childdoc.def}
\childdocforwardprefix[cdocsamp]{cdocsfn}{cdocsch}
%    \end{macrocode}

%\iffalse
%</samplefinal>
%\fi
%
% %%%%%%%%%%%%%%%%%%%%%%%%%%%%%%%%%%%%%%
% \paragraph{Command Line Processing.}
%
% The following three command lines generate the output files
% |cdocscld|, |cdocscl1| and |cdocscl2|
% which should be identical to
% |cdocsdrf|, |cdocsch1| and |cdocsfn2|, respectively:
% \begin{center}
% \begin{tabular}{l}
% |latex -jobname cdocscld \|\\
% |  "\def\version{draft}\input{childdoc.def}\childdocforward{cdocsamp}"|\\
% |latex -jobname cdocscl1 \|\\
% |  "\input{childdoc.def}\childdocforward[cdocsamp]{cdocsch1}"|\\
% |latex -jobname cdocscl2 \|\\
% |  "\def\version{final}\input{childdoc.def}\childdocforward{cdocsch2}"|
% \end{tabular}
% \end{center}
% Note that the trailing backslash on each first line
% merely continues the input to the second line
% (for convenient cut ant paste).
% Furthermore, the command |latex| can be replaced by any
% of its alternative versions such as |pdflatex|.
%
% %%%%%%%%%%%%%%%%%%%%%%%%%%%%%%%%%%%%%%%%%%%%%%%%%%%%%%%%%%%%%%%%%%%%%%%%%%%%%%
% %%%%%%%%%%%%%%%%%%%%%%%%%%%%%%%%%%%%%%%%%%%%%%%%%%%%%%%%%%%%%%%%%%%%%%%%%%%%%%
% \section{Implementation}
%\iffalse
%<*package>
%\fi
%
% This section describes the definitions file |childdoc.def|.

% The definitions cannot be loaded using |\usepackage| or |\RequirePackage|
% which has a mechanism to prevent loading a style file more than once.
% When loading the definitions by means of |\input|
% multiple instances have to be prevented manually:
%\iffalse
%This code needs to be before the `\ProvidesFile' directive
%which is defined at the beginning of this file.
%Therefore it is also placed there and commented out here.
%</package>
%<*discard>
%\fi
%    \begin{macrocode}
\ifdefined\childdocmain\endinput\fi
%    \end{macrocode}
%\iffalse
%</discard>
%<*package>
%\fi
%
% \macro{\ifchilddoc}
% \macro{\ifchilddocmanual}
% The conditional |\ifchilddoc| tells whether a
% child (true) or main (false) document is being compiled.
% The conditional |\ifchilddocmanual| tells whether
% the |\includeonly| mechanism is used (false) or
% the selection of child files must be performed manually (true).
% The definitions initialise to false:
%    \begin{macrocode}
\newif\ifchilddoc
\newif\ifchilddocmanual
%    \end{macrocode}

% \macro{\childdocname}
% \macro{\childdocjob}
% The macro |\childdocname| stores the name of the main document
% to be compiled. The macro |\childdocjob| stores the name of
% the document on which the \LaTeX{} compiler was originally invoked.
% The content of |\jobname| cannot be compared
% to filenames specified in the source due to different catcodes.
% The following code rescans |\jobname|, stores the result
% in |\childdocname| and saves a copy in |\childdocjob|:
%    \begin{macrocode}
\edef\childdocname{\scantokens\expandafter{\jobname\noexpand}}
\let\childdocjob\childdocname
%    \end{macrocode}

% \macro{\childdocdisable}
% The macro |\childdocdisable| prevents the main file
% from being processed more than once.
% At this stage, the main document command |\childdocmain|
% is assumed to be called once again where it should do nothing.
% Any subsequent call to it should prevent
% a secondary processing of the main document
% It overwrites the forwarding commands
% |\childdocof| and |\childdocforward|
% with empty macros to prevent further inclusions of the main document:
%    \begin{macrocode}
\newcommand{\childdocdisable}
{
  \renewcommand{\childdocmain}[1]{\renewcommand{\childdocmain}[1]{\endinput}}
  \renewcommand{\childdocof}[1]{}
  \renewcommand{\childdocby}[2][]{}
  \renewcommand{\childdocforward}[2][]{}
  \renewcommand{\childdocdisable}{}
}
%    \end{macrocode}

% \macro{\childdocmain}
% The macro |\childdocmain| is to be called at the top of the main file
% with nothing or the main filename (without extension) as argument.
% First, it breaks loops.
% If the argument is not empty and does not match |\childdocname|
% (which is set by the first inclusion of |childdoc.def|),
% |\ifchilddoc| is set to true, |\includeonly| is applied to the child file
% and |\jobname| is set to the main file
% (for proper handling of |.aux| files):
%    \begin{macrocode}
\newcommand{\childdocmain}[1]
{
  \childdocdisable\childdocmain{}
  \if?#1?\else
    \begingroup
      \def\childdoctmp{#1}
      \ifx\childdoctmp\childdocname
        \def\childdoctmp{}
      \else
        \def\childdoctmp
        {
          \childdoctrue
          \includeonly{\childdocname}
          \def\childdocjob{#1}
          \def\jobname{#1}
        }
      \fi
      \expandafter
    \endgroup
    \childdoctmp
  \fi
}
%    \end{macrocode}

% \macro{\childdocof}
% The command |\childdocof| redirects
% compilation to the main file |#1|.
%    \begin{macrocode}
\newcommand{\childdocof}[1]
{
  \childdocdisable
  \childdoctrue
  \includeonly{\childdocname}
  \def\jobname{#1}
  \def\childdocjob{#1}
  \input{#1}
}
%    \end{macrocode}

% \macro{\childdocby}
% The command |\childdocby| ....
%    \begin{macrocode}
\newcommand{\childdocby}[2][]
{
  \childdocdisable
  \childdoctrue
  \childdocmanualtrue
  \if?#1?\else
    \def\jobname{#2}
  \fi
  \def\childdocjob{#2}
  \input{#2}
  \endinput
}
%    \end{macrocode}

% \macro{\childdocforward}
% The command |\childdocforward| redirects
% compilation to the main file or
% (if the optional argument is given) a child file.
% Parameters are set as if the main file
% or a child file starting with |\childdocof| was compiled.
% Then compilation is handed over to the main file:
%    \begin{macrocode}
\newcommand{\childdocforward}[2][]
{
  \begingroup
    \if?#1?
      \def\childdoctmp
      {
        \def\childdocname{#2}
        \def\childdocjob{#2}
        \def\jobname{#2}
        \input{#2}
        \endinput
      }
    \else
      \def\childdoctmp
      {
        \childdocdisable
        \def\childdocname{#2}
        \childdoctrue
        \includeonly{#2}
        \def\childdocjob{#1}
        \def\jobname{#1}
        \input{#1}
        \endinput
      }
    \fi
    \expandafter
  \endgroup
  \childdoctmp
}
%    \end{macrocode}

% \macro{\childdocforwardprefix}
% The command |\childdocforwardprefix| redirects
% compilation to the main or a child file by means of a pattern.
% The prefix |#1| in the current filename is replaced by |#2|
% and the suffix of the current filename is kept
% (it is assumed that the filename does not contain the substring `|~~~|'
% which is used as a delimiter).
% Compilation is handed over to the new file by |\childdocforward|:
%    \begin{macrocode}
\newcommand{\childdocforwardprefix}[3][]
{
  \begingroup
    \def\childdocextract #2##1~~~{\def\childdoctmp{\childdocforward[#1]{#3##1}}}
    \expandafter\childdocextract\childdocname~~~
    \expandafter
  \endgroup
  \childdoctmp
}
%    \end{macrocode}

% \macro{\childdoc}
% The deprecated macro |\childdoc| is a legacy version of |\childdocmain|:
%    \begin{macrocode}
\newcommand{\childdoc}{\childdocmain}
%    \end{macrocode}

% \macro{\childdocredirect}
% The deprecated macro |\childdocredirect| is a legacy version
% of |\childdocforward| and |\childdocforwardprefix|:
%    \begin{macrocode}
\newcommand{\childdocredirect}[2][]
{
  \begingroup
    \if?#1?
      \def\childdoctmp{\childdocforward{#2}}
    \else
      \def\childdoctmp{\childdocforwardprefix{#1}{#2}}
    \fi
    \expandafter
  \endgroup
  \childdoctmp
}
%    \end{macrocode}

%\iffalse
%</package>
%\fi
%
\endinput
|\\
|\childdocforward[|\textit{main}|]{|\textit{dest}|}|\\
\end{tabular}
\end{center}
%
The argument \textit{dest} is the destination file
(without extension).
It should be the main file or one of the child files.
Note that further \textsf{childdoc} directives
such as |\childdocof| and |\childdocforward|
in the indicated file will be processed in this form.
The optional argument \textit{main}
passes on directly to the main file \textit{main}
while pretending to compile the child \textit{dest}.
This form behaves as if \textit{dest}
issues |\childdocof{|\textit{main}|}| right away,
and no further \textsf{childdoc} directives will be processed.

%%%%%%%%%%%%%%%%%%%%%%%%%%%%%%%%%%%%%%%%
\DescribeMacro{\...prefix}
In the alternative form |\childdocforwardprefix|,
%
\begin{center}
\begin{tabular}{l}
|% \iffalse
%
% childdoc.dtx Copyright (C) 2017-2018 Niklas Beisert
%
% This work may be distributed and/or modified under the
% conditions of the LaTeX Project Public License, either version 1.3
% of this license or (at your option) any later version.
% The latest version of this license is in
%   http://www.latex-project.org/lppl.txt
% and version 1.3 or later is part of all distributions of LaTeX
% version 2005/12/01 or later.
%
% This work has the LPPL maintenance status `maintained'.
%
% The Current Maintainer of this work is Niklas Beisert.
%
% This work consists of the files childdoc.dtx and childdoc.ins
% and the derived files childdoc.def and cdocsamp.tex with
% cdocsch1.tex, cdocsch2.tex, cdocsdrf.tex, cdocsfn1.tex, cdocsfn2.tex.
%
%<package>\ifdefined\childdocmain\endinput\fi
%<package>\ProvidesFile{childdoc.def}[2018/12/30 v2.0 child document driver]
%<samplemain>\ProvidesFile{cdocsamp.tex}[2018/12/30 v2.0 sample for childdoc]
%<*driver>
%\ProvidesFile{childdoc.drv}[2018/12/30 v2.0 childdoc reference manual file]
\PassOptionsToClass{10pt,a4paper}{article}
\documentclass{ltxdoc}

\usepackage[margin=35mm]{geometry}
\usepackage{hyperref}
\usepackage{hyperxmp}
\usepackage[usenames]{color}

\hypersetup{colorlinks=true}
\hypersetup{pdfstartview=FitH}
\hypersetup{pdfpagemode=UseNone}
\hypersetup{pdfsource={}}
\hypersetup{pdflang={en-UK}}
\hypersetup{pdfcopyright={Copyright 2017-2018 Niklas Beisert.
  This work may be distributed and/or modified under the
  conditions of the LaTeX Project Public License, either version 1.3
  of this license or (at your option) any later version.}}
\hypersetup{pdflicenseurl={http://www.latex-project.org/lppl.txt}}
\hypersetup{pdfcontactaddress={ETH Zurich, ITP, HIT K,
  Wolfgang-Pauli-Strasse 27}}
\hypersetup{pdfcontactpostcode={8093}}
\hypersetup{pdfcontactcity={Zurich}}
\hypersetup{pdfcontactcountry={Switzerland}}
\hypersetup{pdfcontactemail={nbeisert@itp.phys.ethz.ch}}
\hypersetup{pdfcontacturl={http://people.phys.ethz.ch/\xmptilde nbeisert/}}

\newcommand{\secref}[1]{\hyperref[#1]{section \ref*{#1}}}

\parskip1ex
\parindent0pt
\let\olditemize\itemize
\def\itemize{\olditemize\parskip0pt}

\begin{document}

\title{The \textsf{childdoc} Package}
\hypersetup{pdftitle={The childdoc Package}}
\author{Niklas Beisert\\[2ex]
  Institut f\"ur Theoretische Physik\\
  Eidgen\"ossische Technische Hochschule Z\"urich\\
  Wolfgang-Pauli-Strasse 27, 8093 Z\"urich, Switzerland\\[1ex]
  \href{mailto:nbeisert@itp.phys.ethz.ch}
  {\texttt{nbeisert@itp.phys.ethz.ch}}}
\hypersetup{pdfauthor={Niklas Beisert}}
\hypersetup{pdfsubject={Manual for the LaTeX2e Package childdoc}}
\date{30 December 2018, \textsf{v2.0}}
\maketitle

\begin{abstract}\noindent
\textsf{childdoc} is a \LaTeXe{} package
that enables the direct compilation
of document sections included by |\include|
to individual files.
\end{abstract}

\begingroup
\parskip0ex
\tableofcontents
\endgroup

%%%%%%%%%%%%%%%%%%%%%%%%%%%%%%%%%%%%%%%%%%%%%%%%%%%%%%%%%%%%%%%%%%%%%%%%%%%%%%%%
%%%%%%%%%%%%%%%%%%%%%%%%%%%%%%%%%%%%%%%%%%%%%%%%%%%%%%%%%%%%%%%%%%%%%%%%%%%%%%%%
\section{Introduction}

\LaTeX{} provides a mechanism to structure a large document (such as a book)
into a main file and several child files (containing the chapters)
using the |\include| command.
This mechanism is beneficial for documents
which span hundreds of pages in order to
make the source file(s) more manageable.
Moreover, compilation can be restricted to
selected child files by means of the |\includeonly| command.
The latter feature can be used to reduce the compilation time while editing
(this was significantly more useful in the earlier days of \LaTeX{})
or to generate a smaller document which is easier to navigate.
Another application of |\includeonly| is to generate
documents consisting of selected parts of the complete document.

However, there are a few drawbacks of the plain |\include| mechanism:
\begin{itemize}
\item
The child files cannot be compiled on their own,
they can only be compiled via the main file.
A naive editing environment
(such as a text editor with an option
to have the current file processed by \LaTeX)
may require one to switch to the main file before compiling;
attempting to compile the child file produces errors.
\item
The main file must be modified (each time)
to adjust the |\includeonly| command
to the present needs. This easily leaves the main file in a messy state.
\item
The generated document will always carry the filename
of the main document. This is inconvenient if
several child files are to be compiled and
to be kept for distribution.
\end{itemize}

The present package provides a simple interface
to make child files individually compilable by \LaTeX{}.
Compiling a child file then has the same effect as compiling
the main file with an |\includeonly| command
to select the appropriate child.
Moreover the generated document will carry the name of the child
rather than the main file.
This resolves all three above issues.

This feature is meant to make the editing of books,
thesis documents and lecture notes somewhat more convenient.
However, the package can also be used efficiently for
composing a series of documents (such as exercise sheets)
which are typically distributed individually.
It then assists the author in generating the individual documents
(potentially in different versions)
as well as a document containing the collected series.
Another application is in developing style files
or other kinds of included material
where compilation of the style file could redirect
to a sample or test file.

%%%%%%%%%%%%%%%%%%%%%%%%%%%%%%%%%%%%%%%%%%%%%%%%%%%%%%%%%%%%%%%%%%%%%%%%%%%%%%%%
%%%%%%%%%%%%%%%%%%%%%%%%%%%%%%%%%%%%%%%%%%%%%%%%%%%%%%%%%%%%%%%%%%%%%%%%%%%%%%%%
\section{Usage}

First of all, the package \textsf{childdoc} is \emph{not} a standard
\LaTeXe{} |.sty| style file! Therefore it needs to be invoked in
a non-standard way.

%%%%%%%%%%%%%%%%%%%%%%%%%%%%%%%%%%%%%%%%%%%%%%%%%%%%%%%%%%%%%%%%%%%%%%%%%%%%%%%%
\subsection{Included Files}
\label{sec:include}

%%%%%%%%%%%%%%%%%%%%%%%%%%%%%%%%%%%%%%%%
\DescribeMacro{\childdocmain}
To use the package, add the commands
\begin{center}
\begin{tabular}{l}
|\input{childdoc.def}|\\
|\childdocmain{}|\\
\end{tabular}
\end{center}
at the very top of the main \LaTeX{} file,
in particular \emph{before} the |\documentclass| statement!
The argument of |\childdocmain| should be left empty
(but it must be present).

%%%%%%%%%%%%%%%%%%%%%%%%%%%%%%%%%%%%%%%%
\DescribeMacro{\childdocof}
Furthermore, add the commands
\begin{center}
\begin{tabular}{l}
|\input{childdoc.def}|\\
|\childdocof{|\textit{main}|}|\\
\end{tabular}
\end{center}
at the top of every child file \textit{child}
which is included by |\include{|\textit{child}|}|
from within the main file
(or at least for those files to be compiled individually).
The argument \textit{main} must be the filename of the main file.

There are a couple of
considerations in setting up the main and child documents:

%%%%%%%%%%%%%%%%%%%%%%%%%%%%%%%%%%%%%%%%
\paragraph{Restrictions.}

Please note the following restrictions:
\begin{itemize}
\item
|\childdocmain| must be called with one argument \textit{main}
to ensure compatibility with earlier version of the package.
It must either be empty (|\childdocmain{}|)
or precisely match the filename of the main file in which it is specified.
See \secref{sec:detection} for further information.
\item
The filename \textit{main} must be specified without the |.tex| extension.
\item
The filename \textit{main} is case sensitive
(even in case-insensitive file systems)
due to internal string comparison.
\item
The argument \textit{main} should be fully expanded, it cannot be a macro.
\item
Subdirectories and special characters should be avoided in filenames.
\item
The command |\childdocmain{|\textit{main}|}| must be followed by a whitespace.
It should not be followed immediately by another command
or by a comment mark `|%|'.
This is because the \TeX{} parser reads the token immediately following
the argument of |\childdocmain| and puts it
at the beginning of every child section;
however, a white\-space is ignored.
\end{itemize}

%%%%%%%%%%%%%%%%%%%%%%%%%%%%%%%%%%%%%%%%
\paragraph{Content of Main File.}

It is advisable to place all content in the child files included by |\include|.
Any output contained in the main file will appear in all child documents
unless suppressed manually;
it cannot be suppressed automatically by the |\includeonly| directive
and thus should normally be avoided.
A method to include some content in the main file
by means of conditional processing is described in \secref{sec:conditional}.

%%%%%%%%%%%%%%%%%%%%%%%%%%%%%%%%%%%%%%%%
\paragraph{Page Numbering.}

When only a part of the document is compiled,
the appropriate numbering of pages
(as well as other status parameters)
is determined from the |.aux| files.
The latter contain information from previous passes.
However this information needs to propagate through
all intermediate child documents.
Therefore the page numbering in child documents may well
be inconsistent until the complete document is compiled at least once.

A useful (if unconventional) way to always ensure a consistent
page numbering is to restart the numbering in each child document
and denote the pages by `\textit{child}|.|\textit{page}'
where \textit{child} represents the chapter/section number of the child file.
This can be achieved by the command
|\numberwithin{page}{|\textit{child}|}|
of the \textsf{amsmath} package
where \textit{child} can be |chapter| or |section|
depending on the chosen structuring.
Alternatively, one can modify the macro |\thepage| appropriately
and reset the counter |page| at the start of each child file.

%%%%%%%%%%%%%%%%%%%%%%%%%%%%%%%%%%%%%%%%%%%%%%%%%%%%%%%%%%%%%%%%%%%%%%%%%%%%%%%%
\subsection{Conditional Processing}
\label{sec:conditional}

The package provides a mechanism to compile different versions
of a document. To customise the versions further some conditional processing
can come in handy to distinguish which version is being compiled.
The package provides two macros to describe the compilation context:

%%%%%%%%%%%%%%%%%%%%%%%%%%%%%%%%%%%%%%%%
\DescribeMacro{\ifchilddoc}
The conditional |\ifchilddoc| distinguishes between the compilation of
child documents and the main document:
%
\begin{center}
|\ifchilddoc |\textit{child-code}| |[|\||else |\textit{main-code}]| \||fi|
\end{center}

%%%%%%%%%%%%%%%%%%%%%%%%%%%%%%%%%%%%%%%%
\DescribeMacro{\childdocname}
\DescribeMacro{\childdocjob}
The macro |\childdocname| contains the filename (without extension)
of the main or child file being processed.
Note that |\childdocjob| will always contain the name of the main file.

%%%%%%%%%%%%%%%%%%%%%%%%%%%%%%%%%%%%%%%%
\paragraph{Title Page.}

Conditional processing can be used to include a title or banner page
in the main document when proper precautions are taken.
Importantly, the code in the main file should ensure that the page counter
(as well as other status parameters which are stored in the |.aux| files)
takes the same value after the conditional processing.
Otherwise the page numbers may take divergent values
depending on which part is compiled.

For example, a title page could be declared by:
%
\begin{center}
\begin{tabular}{l}
|\ifchilddoc\||else|\\
|\addtocounter{page}{-1}|\\
\textit{code for title page}\\
|\newpage|\\
|\||fi|
\end{tabular}
\end{center}
%
A banner page for the child documents can be generated by:
%
\begin{center}
\begin{tabular}{l}
|\ifchilddoc|\\
|\addtocounter{page}{-1}|\\
\textit{code for banner page}\\
|\newpage|\\
|\||fi|
\end{tabular}
\end{center}
%
Here one could write a message such as:
\begin{center}
|This is the part \childdocname{} of \childdocjob{}.|
\end{center}

%%%%%%%%%%%%%%%%%%%%%%%%%%%%%%%%%%%%%%%%%%%%%%%%%%%%%%%%%%%%%%%%%%%%%%%%%%%%%%%%
\subsection{Flags}
\label{sec:flags}

The package makes it easy to generate different versions
of the main or child documents.
To this end compilation flags can be defined
and assigned different default values.
They will be particularly useful in conjunction
with the forwarding mechanism described in \secref{sec:forward}.

For example, it may be useful to have a flag |\version|
which can be set to |draft| or |final|.
The document source will contain some conditional code
depending on the value of |\version|.
Suppose further, the flag should default to |final| for the main file
and to |draft| for child files
which is a natural assignment for editing the document.
This is achieved by placing the following code
in the preamble of the main document
(below the |\childdocmain| directive):
%
\begin{center}
\begin{tabular}{l}
|\ifchilddoc|\\
|\providecommand{\version}{draft}|\\
|\||else|\\
|\providecommand{\version}{final}|\\
|\||fi|
\end{tabular}
\end{center}
%
The definition by |\providecommand| makes sure
that previous definitions are not overwritten.
Further statements |\providecommand{\version}{...}|
can thus be added before the above code to override it.

For the main file, one might add a line
(between |\childdocmain| and the above block)
%
\begin{center}
|%\ifchilddoc\||else\providecommand{\version}{draft}\||fi|
\end{center}
%
which can be uncommented to produce a draft version.
Likewise one can add a line to the very top of a child file
(above the |\childdocof{|\textit{main}|}| directive)
%
\begin{center}
|%\providecommand{\version}{final}|
\end{center}
%
which can be uncommented to produce the final version of this child document.

%%%%%%%%%%%%%%%%%%%%%%%%%%%%%%%%%%%%%%%%%%%%%%%%%%%%%%%%%%%%%%%%%%%%%%%%%%%%%%%%
\subsection{Forwarding}
\label{sec:forward}

Different versions of the main or child documents
using compilation flags as described in \secref{sec:flags}
can be (permanently) stored in different files
for convenient compilation, viewing and distribution.
To this end, the package defines a command
to pass on compilation to a different file:

%%%%%%%%%%%%%%%%%%%%%%%%%%%%%%%%%%%%%%%%
\DescribeMacro{\childdocforward}
The command |\childdocforward| redirects processing to
another source file:
%
\begin{center}
\begin{tabular}{l}
|\input{childdoc.def}|\\
|\childdocforward[|\textit{main}|]{|\textit{dest}|}|\\
\end{tabular}
\end{center}
%
The argument \textit{dest} is the destination file
(without extension).
It should be the main file or one of the child files.
Note that further \textsf{childdoc} directives
such as |\childdocof| and |\childdocforward|
in the indicated file will be processed in this form.
The optional argument \textit{main}
passes on directly to the main file \textit{main}
while pretending to compile the child \textit{dest}.
This form behaves as if \textit{dest}
issues |\childdocof{|\textit{main}|}| right away,
and no further \textsf{childdoc} directives will be processed.

%%%%%%%%%%%%%%%%%%%%%%%%%%%%%%%%%%%%%%%%
\DescribeMacro{\...prefix}
In the alternative form |\childdocforwardprefix|,
%
\begin{center}
\begin{tabular}{l}
|\input{childdoc.def}|\\
|\childdocforwardprefix[|\textit{main}|]{|\textit{prefix}|}{|\textit{dest}|}|
\end{tabular}
\end{center}
%
the destination file is determined by a pattern
depending on the current file:
To make this work, the current file must be called
`{\textit{prefix}\hspace{0.2em}\textit{suffix}}'
with \textit{prefix} matching precisely the argument.
Processing is then passed on to the file
`{\textit{dest}\hspace{0.2em}\textit{suffix}}'.
Surely, the same effect is achieved by
directly specifying the
argument `{\textit{dest}\hspace{0.2em}\textit{suffix}}'
in the first form.
However, that requires to set up a different file
for each child. With the alternative form of the command
all these files can have exactly the same content
which simplifies setting them up and maintaining them.

For example, the following file |draft.tex|
with a compilation flag |\version| as described in \secref{sec:flags}
compiles the main document as a draft:
%
\begin{center}
\begin{tabular}{l}
|\def\version{draft}|\\
|\input{childdoc.def}|\\
|\childdocforward{|\textit{main}|}|
\end{tabular}
\end{center}
%
Likewise, the following files |final|\textit{nn}|.tex|
compile the final version of the child document
|child|\textit{nn}|.tex|:
%
\begin{center}
\begin{tabular}{l}
|\def\version{final}|\\
|\input{childdoc.def}|\\
|\childdocforwardprefix{final}{child}|
\end{tabular}
\end{center}
%

Note that when several versions of a main file and/or of each child file
are to be generated, it may be convenient to set up a |Makefile| or
shell script to automatise the process.

%%%%%%%%%%%%%%%%%%%%%%%%%%%%%%%%%%%%%%%%%%%%%%%%%%%%%%%%%%%%%%%%%%%%%%%%%%%%%%%%
\subsection{Command Line Processing}
\label{sec:commandline}

The effect of redirection files can also be achieved by invoking
the \LaTeX{} compiler with a more elaborate command line.
Most conveniently this should be done as part
of a shell script or a |Makefile|.

When using \textsf{childdoc} in the main file, the following
command lines effectively perform a redirection
(note that depending on the shell being used,
backslashes may have to be doubled: `|\|' $\to$ `|\\|'):
%
\begin{center}
|... -jobname "|\textit{target}|" |\\|"|[\textit{flags}]%
|\input{childdoc.def}\childdocforward[|\textit{main}|]{|\textit{dest}|}"|
\end{center}
%
Here \textit{target} is the name of the output file,
\textit{main} is the name of the main file
and \textit{dest} is the name of the main or child file to be processed
(all filenames without extensions).
The optional argument \textit{main} can be omitted
if \textit{main} matches \textit{dest}.
Optionally, compilation \textit{flags} can be defined via |\def| commands.
This command line makes the \TeX{} engine believe
it is compiling the file \textit{target}
whose content is specified as the latter parameter.
The provided code then forwards the processing to
\textit{main} or \textit{dest} as described in \secref{sec:forward}.

%%%%%%%%%%%%%%%%%%%%%%%%%%%%%%%%%%%%%%%%%%%%%%%%%%%%%%%%%%%%%%%%%%%%%%%%%%%%%%%%
\subsection{Include by Input}
\label{sec:input}

Including child documents by |\include| has some restrictions by design.
Most notably, the content of a child document always occupies
its own set of pages; pages cannot be shared between child documents.
Usually, this behaviour makes perfect sense
because each child document contain an essential part of the document.
However, in some situations it may be desirable to compose
a document from a collection of parts
without having mandatory page breaks between then.
For this case, the package
provides a mechanism to include parts
by |\input| which can also be processed individually.
However, by construction this mechanism
requires manual handling of the content to be output.

%%%%%%%%%%%%%%%%%%%%%%%%%%%%%%%%%%%%%%%%
\DescribeMacro{\ifchilddocmanual}
The main file should be prepared as usual, see \secref{sec:include}.
However, the document body must make a distinction
between processing of an individual part and of the main document, e.g.:
%
\begin{center}
\begin{tabular}{l}
|\ifchilddocmanual|\\
|\input{\childdocname}|\\
|\||else|\\
\textit{document body with }|\input{|\textit{part}|}|\\
|\||fi|
\end{tabular}
\end{center}
%
The conditional |\ifchilddocmanual| is true whenever
a part to be included by |\input| is being compiled,
and the name of the part is stored in |\childdocname|.

%%%%%%%%%%%%%%%%%%%%%%%%%%%%%%%%%%%%%%%%
\DescribeMacro{\childdocby}
Each part to be included by |\input| should start with:
%
\begin{center}
\begin{tabular}{l}
|\input{childdoc.def}|\\
|\childdocby{|\textit{main}|}|\\
\end{tabular}
\end{center}
%
The directive |\childdocby| is similar to |\childdocof|
described in \secref{sec:include},
but the subsequent selection of content must be done manually.
To that end, both |\ifchilddoc| and |\ifchilddocmanual|
will be true upon processing of a part,
and the name of the part is stored in |\childdocname|.
Note that |\jobname| will be set to the filename of the current part
so that each part receives an individual |.aux| file
that does not interfere with the |.aux| file(s) of the main document.
This behaviour can be altered by the alternative form
|\childdocby[*]{|\textit{main}|}| (with a non-empty optional argument)
which uses the |.aux| file of the main document
by setting |\jobname| to \textit{main}.

%%%%%%%%%%%%%%%%%%%%%%%%%%%%%%%%%%%%%%%%%%%%%%%%%%%%%%%%%%%%%%%%%%%%%%%%%%%%%%%%
\subsection{Driver Development}
\label{sec:driver}

The \textsf{childdoc} mechanism can also be use for the development
of definition files such as \LaTeX{} styles or classes.
This case differs from the above setup with multiple parts
included by |\include| in that no |\includeonly| should be invoked.
This can be achieved by starting the include file
(before |\ProvidesPackage|) with:
%
\begin{center}
\begin{tabular}{l}
|\input{childdoc.def}|\\
|\childdocforward{|\textit{main}|}|\\
\end{tabular}
\end{center}
%
or alternatively with:
%
\begin{center}
\begin{tabular}{l}
|\input{childdoc.def}|\\
|\childdocby{|\textit{main}|}|\\
\end{tabular}
\end{center}
%
Both forms have slightly different effects as described above.
The main file is prepared as usual, see \secref{sec:include}.

%%%%%%%%%%%%%%%%%%%%%%%%%%%%%%%%%%%%%%%%%%%%%%%%%%%%%%%%%%%%%%%%%%%%%%%%%%%%%%%%
\subsection{Legacy Detection}
\label{sec:detection}

The directive |\childdocmain| in the main file can detect
whether the complete document or merely a child is to be compiled
even without using the directive |\childdocof|.
This method is deprecated because it is less robust
and there is no compelling reason to use it;
it is merely provided for backward compatibility
and it may be removed in future versions.

If the detection mechanism is to be used,
it is mandatory to correctly specify
the filename of the main file as the argument of |\childdocmain|:
%
\begin{center}
\begin{tabular}{l}
|\input{childdoc.def}|\\
|\childdocmain{|\textit{main}|}|\\
\end{tabular}
\end{center}
%
If |\jobname| does not match the argument \textit{main} of |\childdocmain|,
it is assumed that |\jobname| points to the child file to be compiled.
When using |\childdocmain| with the main file specified as argument,
it suffices to start a child file
with just |\input{|\textit{main}|}|
without loading of the package and using |\childdocof|.
If instead all processing is done
with the appropriate \textsf{childdoc} directives,
the argument of \textit{main} of |\childdocmain| can be empty.

An alternative version of the command line processing described
in \secref{sec:commandline} using the detection mechanism reads:
%
\begin{center}
|... -jobname "|\textit{target}|" "|[\textit{flags}]%
[|\def\jobname{|\textit{dest}|}|]|\input{|\textit{main}|}"|
\end{center}

%%%%%%%%%%%%%%%%%%%%%%%%%%%%%%%%%%%%%%%%%%%%%%%%%%%%%%%%%%%%%%%%%%%%%%%%%%%%%%%%
\subsection{Manual Code}
\label{sec:manual}

In case one cannot be certain whether the definitions file |childdoc.def|
is installed on the target \TeX{} distribution
and one prefers not to ship it,
it is conceivable to paste a few relevant commands into the sources.

To that end, drop all statements |\input{childdoc.def}|
and perform the replacements as outlined below.
Instead of |\childdocmain{|\textit{main}|}| add the following code
to the top of the main file:
%
\begin{center}
\begin{tabular}{l}
|\||ifdefined\childdocname\endinput\||fi\newif\ifchilddoc|\\
|\edef\childdocname{\scantokens\expandafter{\jobname\noexpand}}|\\
|\def\childdocmain{|\textit{main}|}\||ifx\childdocmain\childdocname\||else|\\
|\childdoctrue\includeonly{\childdocname}\let\jobname\childdocmain\||fi|\\
\end{tabular}
\end{center}
%
Instead of |\childdocof{|\textit{main}|}| just include the main file
at the top of each child file:
%
\begin{center}
|\input{|\textit{main}|}|
\end{center}
%
A simple redirection |\childdocforward{|\textit{dest}|}| is achieved by:
%
\begin{center}
|\def\jobname{|\textit{dest}|}\input{\jobname}|
\end{center}
%
The redirection with prefix
|\childdocforwardprefix[|\textit{prefix}|]{|\textit{dest}|}|
is accomplished by:
%
\begin{center}
\begin{tabular}{l}
|{\edef\jobname{\scantokens\expandafter{\jobname\noexpand}}|\\
|\def\redirectjob |\textit{prefix}|#1~~~{\gdef\jobname{|\textit{dest}|#1}}|\\
|\expandafter\redirectjob\jobname~~~}\input{\jobname}|
\end{tabular}
\end{center}

In an alternative approach,
child documents can be compiled by a specific command line
without additional code or specific definitions:
%
\begin{center}
|... -jobname "|\textit{target}|" "|[\textit{flags}]%
|\includeonly{|\textit{dest}|}\input{|\textit{main}|}"|
\end{center}
%

%%%%%%%%%%%%%%%%%%%%%%%%%%%%%%%%%%%%%%%%%%%%%%%%%%%%%%%%%%%%%%%%%%%%%%%%%%%%%%%%
%%%%%%%%%%%%%%%%%%%%%%%%%%%%%%%%%%%%%%%%%%%%%%%%%%%%%%%%%%%%%%%%%%%%%%%%%%%%%%%%
\section{Information}

%%%%%%%%%%%%%%%%%%%%%%%%%%%%%%%%%%%%%%%%%%%%%%%%%%%%%%%%%%%%%%%%%%%%%%%%%%%%%%%%
\subsection{Copyright}

Copyright \copyright{} 2017--2018 Niklas Beisert

This work may be distributed and/or modified under the
conditions of the \LaTeX{} Project Public License, either version 1.3
of this license or (at your option) any later version.
The latest version of this license is in
  \url{http://www.latex-project.org/lppl.txt}
and version 1.3 or later is part of all distributions of \LaTeX{}
version 2005/12/01 or later.

This work has the LPPL maintenance status `maintained'.

The Current Maintainer of this work is Niklas Beisert.

This work consists of the files |README.txt|, |childdoc.ins| and |childdoc.dtx|
as well as the derived files |childdoc.def|, |cdocsamp.tex|
with |cdocsch1.tex|, |cdocsch2.tex|, |cdocspt3.tex|, |cdocspt4.tex|,
|cdocsdrf.tex|, |cdocsfn1.tex|, |cdocsfn2.tex|
as well as |childdoc.pdf|.

%%%%%%%%%%%%%%%%%%%%%%%%%%%%%%%%%%%%%%%%%%%%%%%%%%%%%%%%%%%%%%%%%%%%%%%%%%%%%%%%
\subsection{Files and Installation}

The package consists of the files:
%
\begin{center}
\begin{tabular}{ll}
    |README.txt|   & readme file \\
    |childdoc.ins| & installation file \\
    |childdoc.dtx| & source file \\
    |childdoc.def| & definition file \\
    |cdocsamp.tex| & sample main file \\
    |cdocsch1.tex| & sample include file \\
    |cdocsch2.tex| & sample include file \\
    |cdocspt3.tex| & sample part file \\
    |cdocspt4.tex| & sample part file \\
    |cdocsdrf.tex| & sample redirection file \\
    |cdocsfn1.tex| & sample redirection file \\
    |cdocsfn2.tex| & sample redirection file \\
    |childdoc.pdf| & manual
\end{tabular}
\end{center}
%
The distribution consists of the files
|README.txt|, |childdoc.ins| and |childdoc.dtx|.
%
\begin{itemize}
\item
Run (pdf)\LaTeX{} on |childdoc.dtx|
to compile the manual |childdoc.pdf| (this file).
\item
Run \LaTeX{} on |childdoc.ins| to create the definitions file |childdoc.def|
and the sample |cdocsamp.tex| with include files
|cdocsch1.tex|, |cdocsch2.tex|, |cdocspt3.tex|, |cdocspt4.tex|,
|cdocsdrf.tex|, |cdocsfn1.tex|, |cdocsfn2.tex|.
Then copy the file |childdoc.def| to an appropriate directory of your \LaTeX{}
distribution, e.g.\ \textit{texmf-root}|/tex/latex/childdoc|.
\end{itemize}

%%%%%%%%%%%%%%%%%%%%%%%%%%%%%%%%%%%%%%%%%%%%%%%%%%%%%%%%%%%%%%%%%%%%%%%%%%%%%%%%
\subsection{Related CTAN Packages}

There are several other packages which offer a similar functionality:
%
\begin{itemize}
\item
The packages
\href{http://ctan.org/pkg/docmute}{\textsf{docmute}},
\href{http://ctan.org/pkg/includex}{\textsf{includex}} and
\href{http://ctan.org/pkg/standalone}{\textsf{standalone}}
provide commands to include only the document body of
a child file thus allowing both files to be compiled individually.
\item
The packages \href{http://ctan.org/pkg/subdocs}{\textsf{subdocs}}
and \href{http://ctan.org/pkg/subfiles}{\textsf{subfiles}}
provide structures in which the main and child documents can be
encapsulated and allowing them to be compiled individually.
The inclusion mechanism is different from the conventional |\include|.
\item
The package \href{http://ctan.org/pkg/combine}{\textsf{combine}}
is an elaborate solution to combine several documents into one.
\end{itemize}
%
See also the CTAN topic \href{http://ctan.org/topic/subdocs}{\textsf{subdocs}}
for further related packages.
The present package differs from the above solutions in that
a document structure constructed with the conventional |\include| mechanism
just needs two extra commands at the top of every file
such that all constituent files can be compiled individually.

%%%%%%%%%%%%%%%%%%%%%%%%%%%%%%%%%%%%%%%%%%%%%%%%%%%%%%%%%%%%%%%%%%%%%%%%%%%%%%%%
%\subsection{Feature Suggestions}
%
%The following is a list of features which may be useful for future
%versions of this package:
%%
%\begin{itemize}
%\item
%\ldots
%\end{itemize}

%%%%%%%%%%%%%%%%%%%%%%%%%%%%%%%%%%%%%%%%%%%%%%%%%%%%%%%%%%%%%%%%%%%%%%%%%%%%%%%%
\subsection{Revision History}

%%%%%%%%%%%%%%%%%%%%%%%%%%%%%%%%%%%%%%%%
\paragraph{v2.0:} 2018/12/30

\begin{itemize}
\item
immediate forward processing
\item
added |\childdocby| mechanism
\item
manual restructured
\end{itemize}

%%%%%%%%%%%%%%%%%%%%%%%%%%%%%%%%%%%%%%%%
\paragraph{v1.6:} 2018/01/17

\begin{itemize}
\item
application for development of include files
\item
corrections to manual
\end{itemize}

%%%%%%%%%%%%%%%%%%%%%%%%%%%%%%%%%%%%%%%%
\paragraph{v1.5:} 2017/05/21

\begin{itemize}
\item
more complete structuring introduced
\item
|\childdocof| introduced
\item
|\childdoc| renamed to |\childdocmain|
\item
|\childredirect| renamed to |\childdocforward| and |\childdocforwardprefix|
and functionality expanded
\end{itemize}

%%%%%%%%%%%%%%%%%%%%%%%%%%%%%%%%%%%%%%%%
\paragraph{v1.0:} 2017/04/27

\begin{itemize}
\item
manual and install package
\item
first version published on CTAN
\end{itemize}

%%%%%%%%%%%%%%%%%%%%%%%%%%%%%%%%%%%%%%%%
\paragraph{v0.6:} 2017/04/26

\begin{itemize}
\item
redirection mechanism added
\end{itemize}

%%%%%%%%%%%%%%%%%%%%%%%%%%%%%%%%%%%%%%%%
\paragraph{v0.5:} 2017/04/26

\begin{itemize}
\item
functionality in definition file
\end{itemize}


%%%%%%%%%%%%%%%%%%%%%%%%%%%%%%%%%%%%%%%%%%%%%%%%%%%%%%%%%%%%%%%%%%%%%%%%%%%%%%%%
%%%%%%%%%%%%%%%%%%%%%%%%%%%%%%%%%%%%%%%%%%%%%%%%%%%%%%%%%%%%%%%%%%%%%%%%%%%%%%%%
%%%%%%%%%%%%%%%%%%%%%%%%%%%%%%%%%%%%%%%%%%%%%%%%%%%%%%%%%%%%%%%%%%%%%%%%%%%%%%%%
\appendix

\settowidth\MacroIndent{\rmfamily\scriptsize 000\ }

 \DocInput{childdoc.dtx}

\end{document}
%</driver>
% \fi
%
% %%%%%%%%%%%%%%%%%%%%%%%%%%%%%%%%%%%%%%%%%%%%%%%%%%%%%%%%%%%%%%%%%%%%%%%%%%%%%%
% %%%%%%%%%%%%%%%%%%%%%%%%%%%%%%%%%%%%%%%%%%%%%%%%%%%%%%%%%%%%%%%%%%%%%%%%%%%%%%
% \section{Sample}
%\iffalse
%<*samplemain>
%\fi
%
% The following presents a sample document
% with two chapters, two parts, a title page,
% a compile flag as well as three forwarding files to set the flag.
% It consists of eight |.tex| files:
% \begin{center}
% \begin{tabular}{ll}
% |cdocsamp.tex|&main file\\
% |cdocsch1.tex|&include file for chapter 1\\
% |cdocsch2.tex|&include file for chapter 2\\
% |cdocspt3.tex|&include file for part 3\\
% |cdocspt4.tex|&include file for part 4\\
% |cdocsdrf.tex|&forwarding file for main file in draft mode\\
% |cdocsfi1.tex|&forwarding file for final version of chapter 1\\
% |cdocsfi2.tex|&forwarding file for final version of chapter 2\\
% \end{tabular}
% \end{center}
% Each of the eight files can be compiled directly by the \LaTeX{} compiler.
%
% %%%%%%%%%%%%%%%%%%%%%%%%%%%%%%%%%%%%%%
% \paragraph{Main File.}
%
% The main file is called |cdocsamp.tex|.
%
% Load the \textsf{childdoc} definitions and
% declare the filename for the main document:
%    \begin{macrocode}
\input{childdoc.def}
\childdocmain{}
%    \end{macrocode}

% Optional override for |\version| flag:
%    \begin{macrocode}
%%\ifchilddoc\else\providecommand{\version}{draft}\fi
%    \end{macrocode}

% Define the default values for the |\version| flag
% (|final| for the main file and |draft| for childs):
%    \begin{macrocode}
\ifchilddoc
\providecommand{\version}{draft}
\else
\providecommand{\version}{final}
\fi
%    \end{macrocode}

% Load the standard document class:
%    \begin{macrocode}
\documentclass[12pt]{article}
%    \end{macrocode}

% Start the document body:
%    \begin{macrocode}
\begin{document}
%    \end{macrocode}

% Declare a title page.
% Print title, part of document being processed and version flag:
%    \begin{macrocode}
\addtocounter{page}{-1}
\begin{center}
{\LARGE\bfseries{}childdoc example\par}
\vspace{1cm}
\ifchilddoc
\ifchilddocmanual part\else chapter\fi:
`\childdocname' of `\childdocjob'\par
\else
main document: `\childdocjob'\par
\fi
version: \version\par
\end{center}
\newpage
%    \end{macrocode}

% Manually include selected file,
% otherwise process as usual:
%    \begin{macrocode}
\ifchilddocmanual
\section*{part `\childdocname'}
\input{\childdocname}
\else
%    \end{macrocode}

% Include the two chapters:
%    \begin{macrocode}
\include{cdocsch1}
\include{cdocsch2}
%    \end{macrocode}

% Include the two parts unless only chapters should be displayed:
%    \begin{macrocode}
\ifchilddoc\else
\section{part three}
\input{cdocspt3}
\section{part four}
\input{cdocspt4}
\fi
%    \end{macrocode}

% Process as usual until here:
%    \begin{macrocode}
\fi
%    \end{macrocode}

% End of document body:
%    \begin{macrocode}
\end{document}
%    \end{macrocode}
%\iffalse
%</samplemain>
%\fi
%
% %%%%%%%%%%%%%%%%%%%%%%%%%%%%%%%%%%%%%%
% \paragraph{Chapter Include Files.}
%
% The include files are called |cdocsch1.tex| and |cdocsch2.tex|.
%
%\iffalse
%<*samplechap1|samplechap2>
%\fi

% Optional override for |\version| flag:
%    \begin{macrocode}
%%\providecommand{\version}{final}
%    \end{macrocode}

% Include the main document:
%    \begin{macrocode}
\input{childdoc.def}
\childdocof{cdocsamp}
%    \end{macrocode}

%\iffalse
%</samplechap1|samplechap2>
%\fi
%
%\iffalse
%<*samplechap1>
%\fi
% Some text for chapter 1:
%    \begin{macrocode}
\section{one}
some text in chapter one
%    \end{macrocode}

%\iffalse
%</samplechap1>
%\fi
% Some text for chapter 2:
%\iffalse
%<*samplechap2>
%\fi
%    \begin{macrocode}
\section{two}
more text in chapter two
%    \end{macrocode}

%\iffalse
%</samplechap2>
%\fi
%
% %%%%%%%%%%%%%%%%%%%%%%%%%%%%%%%%%%%%%%
% \paragraph{Part Include Files.}
%
% The include files are called |cdocspt3.tex| and |cdocspt4.tex|.
%
%\iffalse
%<*samplepart3|samplepart4>
%\fi

% Optional override for |\version| flag:
%    \begin{macrocode}
%%\providecommand{\version}{final}
%    \end{macrocode}

% Include the main document:
%    \begin{macrocode}
\input{childdoc.def}
\childdocby{cdocsamp}
%    \end{macrocode}

%\iffalse
%</samplepart3|samplepart4>
%\fi
%
%\iffalse
%<*samplepart3>
%\fi
% Some text for part 3:
%    \begin{macrocode}
some text in part three
%    \end{macrocode}

%\iffalse
%</samplepart3>
%\fi
% Some text for part 4:
%\iffalse
%<*samplepart4>
%\fi
%    \begin{macrocode}
more text in part four
%    \end{macrocode}

%\iffalse
%</samplepart4>
%\fi
%
% %%%%%%%%%%%%%%%%%%%%%%%%%%%%%%%%%%%%%%
% \paragraph{Forwarding for a Complete Draft.}
%
% The following forwarding file |cdocsdrf.tex|
% compiles the main document in draft mode:
%\iffalse
%<*sampledraft>
%\fi
%    \begin{macrocode}
\def\version{draft}
\input{childdoc.def}
\childdocforward{cdocsamp}
%    \end{macrocode}

%\iffalse
%</sampledraft>
%\fi
%
% %%%%%%%%%%%%%%%%%%%%%%%%%%%%%%%%%%%%%%
% \paragraph{Forwarding for Final Version of the Chapters.}
%
% The following forwarding files |cdocsfn1.tex| and |cdocsfn2.tex|
% (with identical content)
% compile the final versions of the child documents
% |cdocsch1.tex| and |cdocsch2.tex|, respectively:
%\iffalse
%<*samplefinal>
%\fi
%    \begin{macrocode}
\def\version{final}
\input{childdoc.def}
\childdocforwardprefix[cdocsamp]{cdocsfn}{cdocsch}
%    \end{macrocode}

%\iffalse
%</samplefinal>
%\fi
%
% %%%%%%%%%%%%%%%%%%%%%%%%%%%%%%%%%%%%%%
% \paragraph{Command Line Processing.}
%
% The following three command lines generate the output files
% |cdocscld|, |cdocscl1| and |cdocscl2|
% which should be identical to
% |cdocsdrf|, |cdocsch1| and |cdocsfn2|, respectively:
% \begin{center}
% \begin{tabular}{l}
% |latex -jobname cdocscld \|\\
% |  "\def\version{draft}\input{childdoc.def}\childdocforward{cdocsamp}"|\\
% |latex -jobname cdocscl1 \|\\
% |  "\input{childdoc.def}\childdocforward[cdocsamp]{cdocsch1}"|\\
% |latex -jobname cdocscl2 \|\\
% |  "\def\version{final}\input{childdoc.def}\childdocforward{cdocsch2}"|
% \end{tabular}
% \end{center}
% Note that the trailing backslash on each first line
% merely continues the input to the second line
% (for convenient cut ant paste).
% Furthermore, the command |latex| can be replaced by any
% of its alternative versions such as |pdflatex|.
%
% %%%%%%%%%%%%%%%%%%%%%%%%%%%%%%%%%%%%%%%%%%%%%%%%%%%%%%%%%%%%%%%%%%%%%%%%%%%%%%
% %%%%%%%%%%%%%%%%%%%%%%%%%%%%%%%%%%%%%%%%%%%%%%%%%%%%%%%%%%%%%%%%%%%%%%%%%%%%%%
% \section{Implementation}
%\iffalse
%<*package>
%\fi
%
% This section describes the definitions file |childdoc.def|.

% The definitions cannot be loaded using |\usepackage| or |\RequirePackage|
% which has a mechanism to prevent loading a style file more than once.
% When loading the definitions by means of |\input|
% multiple instances have to be prevented manually:
%\iffalse
%This code needs to be before the `\ProvidesFile' directive
%which is defined at the beginning of this file.
%Therefore it is also placed there and commented out here.
%</package>
%<*discard>
%\fi
%    \begin{macrocode}
\ifdefined\childdocmain\endinput\fi
%    \end{macrocode}
%\iffalse
%</discard>
%<*package>
%\fi
%
% \macro{\ifchilddoc}
% \macro{\ifchilddocmanual}
% The conditional |\ifchilddoc| tells whether a
% child (true) or main (false) document is being compiled.
% The conditional |\ifchilddocmanual| tells whether
% the |\includeonly| mechanism is used (false) or
% the selection of child files must be performed manually (true).
% The definitions initialise to false:
%    \begin{macrocode}
\newif\ifchilddoc
\newif\ifchilddocmanual
%    \end{macrocode}

% \macro{\childdocname}
% \macro{\childdocjob}
% The macro |\childdocname| stores the name of the main document
% to be compiled. The macro |\childdocjob| stores the name of
% the document on which the \LaTeX{} compiler was originally invoked.
% The content of |\jobname| cannot be compared
% to filenames specified in the source due to different catcodes.
% The following code rescans |\jobname|, stores the result
% in |\childdocname| and saves a copy in |\childdocjob|:
%    \begin{macrocode}
\edef\childdocname{\scantokens\expandafter{\jobname\noexpand}}
\let\childdocjob\childdocname
%    \end{macrocode}

% \macro{\childdocdisable}
% The macro |\childdocdisable| prevents the main file
% from being processed more than once.
% At this stage, the main document command |\childdocmain|
% is assumed to be called once again where it should do nothing.
% Any subsequent call to it should prevent
% a secondary processing of the main document
% It overwrites the forwarding commands
% |\childdocof| and |\childdocforward|
% with empty macros to prevent further inclusions of the main document:
%    \begin{macrocode}
\newcommand{\childdocdisable}
{
  \renewcommand{\childdocmain}[1]{\renewcommand{\childdocmain}[1]{\endinput}}
  \renewcommand{\childdocof}[1]{}
  \renewcommand{\childdocby}[2][]{}
  \renewcommand{\childdocforward}[2][]{}
  \renewcommand{\childdocdisable}{}
}
%    \end{macrocode}

% \macro{\childdocmain}
% The macro |\childdocmain| is to be called at the top of the main file
% with nothing or the main filename (without extension) as argument.
% First, it breaks loops.
% If the argument is not empty and does not match |\childdocname|
% (which is set by the first inclusion of |childdoc.def|),
% |\ifchilddoc| is set to true, |\includeonly| is applied to the child file
% and |\jobname| is set to the main file
% (for proper handling of |.aux| files):
%    \begin{macrocode}
\newcommand{\childdocmain}[1]
{
  \childdocdisable\childdocmain{}
  \if?#1?\else
    \begingroup
      \def\childdoctmp{#1}
      \ifx\childdoctmp\childdocname
        \def\childdoctmp{}
      \else
        \def\childdoctmp
        {
          \childdoctrue
          \includeonly{\childdocname}
          \def\childdocjob{#1}
          \def\jobname{#1}
        }
      \fi
      \expandafter
    \endgroup
    \childdoctmp
  \fi
}
%    \end{macrocode}

% \macro{\childdocof}
% The command |\childdocof| redirects
% compilation to the main file |#1|.
%    \begin{macrocode}
\newcommand{\childdocof}[1]
{
  \childdocdisable
  \childdoctrue
  \includeonly{\childdocname}
  \def\jobname{#1}
  \def\childdocjob{#1}
  \input{#1}
}
%    \end{macrocode}

% \macro{\childdocby}
% The command |\childdocby| ....
%    \begin{macrocode}
\newcommand{\childdocby}[2][]
{
  \childdocdisable
  \childdoctrue
  \childdocmanualtrue
  \if?#1?\else
    \def\jobname{#2}
  \fi
  \def\childdocjob{#2}
  \input{#2}
  \endinput
}
%    \end{macrocode}

% \macro{\childdocforward}
% The command |\childdocforward| redirects
% compilation to the main file or
% (if the optional argument is given) a child file.
% Parameters are set as if the main file
% or a child file starting with |\childdocof| was compiled.
% Then compilation is handed over to the main file:
%    \begin{macrocode}
\newcommand{\childdocforward}[2][]
{
  \begingroup
    \if?#1?
      \def\childdoctmp
      {
        \def\childdocname{#2}
        \def\childdocjob{#2}
        \def\jobname{#2}
        \input{#2}
        \endinput
      }
    \else
      \def\childdoctmp
      {
        \childdocdisable
        \def\childdocname{#2}
        \childdoctrue
        \includeonly{#2}
        \def\childdocjob{#1}
        \def\jobname{#1}
        \input{#1}
        \endinput
      }
    \fi
    \expandafter
  \endgroup
  \childdoctmp
}
%    \end{macrocode}

% \macro{\childdocforwardprefix}
% The command |\childdocforwardprefix| redirects
% compilation to the main or a child file by means of a pattern.
% The prefix |#1| in the current filename is replaced by |#2|
% and the suffix of the current filename is kept
% (it is assumed that the filename does not contain the substring `|~~~|'
% which is used as a delimiter).
% Compilation is handed over to the new file by |\childdocforward|:
%    \begin{macrocode}
\newcommand{\childdocforwardprefix}[3][]
{
  \begingroup
    \def\childdocextract #2##1~~~{\def\childdoctmp{\childdocforward[#1]{#3##1}}}
    \expandafter\childdocextract\childdocname~~~
    \expandafter
  \endgroup
  \childdoctmp
}
%    \end{macrocode}

% \macro{\childdoc}
% The deprecated macro |\childdoc| is a legacy version of |\childdocmain|:
%    \begin{macrocode}
\newcommand{\childdoc}{\childdocmain}
%    \end{macrocode}

% \macro{\childdocredirect}
% The deprecated macro |\childdocredirect| is a legacy version
% of |\childdocforward| and |\childdocforwardprefix|:
%    \begin{macrocode}
\newcommand{\childdocredirect}[2][]
{
  \begingroup
    \if?#1?
      \def\childdoctmp{\childdocforward{#2}}
    \else
      \def\childdoctmp{\childdocforwardprefix{#1}{#2}}
    \fi
    \expandafter
  \endgroup
  \childdoctmp
}
%    \end{macrocode}

%\iffalse
%</package>
%\fi
%
\endinput
|\\
|\childdocforwardprefix[|\textit{main}|]{|\textit{prefix}|}{|\textit{dest}|}|
\end{tabular}
\end{center}
%
the destination file is determined by a pattern
depending on the current file:
To make this work, the current file must be called
`{\textit{prefix}\hspace{0.2em}\textit{suffix}}'
with \textit{prefix} matching precisely the argument.
Processing is then passed on to the file
`{\textit{dest}\hspace{0.2em}\textit{suffix}}'.
Surely, the same effect is achieved by
directly specifying the
argument `{\textit{dest}\hspace{0.2em}\textit{suffix}}'
in the first form.
However, that requires to set up a different file
for each child. With the alternative form of the command
all these files can have exactly the same content
which simplifies setting them up and maintaining them.

For example, the following file |draft.tex|
with a compilation flag |\version| as described in \secref{sec:flags}
compiles the main document as a draft:
%
\begin{center}
\begin{tabular}{l}
|\def\version{draft}|\\
|% \iffalse
%
% childdoc.dtx Copyright (C) 2017-2018 Niklas Beisert
%
% This work may be distributed and/or modified under the
% conditions of the LaTeX Project Public License, either version 1.3
% of this license or (at your option) any later version.
% The latest version of this license is in
%   http://www.latex-project.org/lppl.txt
% and version 1.3 or later is part of all distributions of LaTeX
% version 2005/12/01 or later.
%
% This work has the LPPL maintenance status `maintained'.
%
% The Current Maintainer of this work is Niklas Beisert.
%
% This work consists of the files childdoc.dtx and childdoc.ins
% and the derived files childdoc.def and cdocsamp.tex with
% cdocsch1.tex, cdocsch2.tex, cdocsdrf.tex, cdocsfn1.tex, cdocsfn2.tex.
%
%<package>\ifdefined\childdocmain\endinput\fi
%<package>\ProvidesFile{childdoc.def}[2018/12/30 v2.0 child document driver]
%<samplemain>\ProvidesFile{cdocsamp.tex}[2018/12/30 v2.0 sample for childdoc]
%<*driver>
%\ProvidesFile{childdoc.drv}[2018/12/30 v2.0 childdoc reference manual file]
\PassOptionsToClass{10pt,a4paper}{article}
\documentclass{ltxdoc}

\usepackage[margin=35mm]{geometry}
\usepackage{hyperref}
\usepackage{hyperxmp}
\usepackage[usenames]{color}

\hypersetup{colorlinks=true}
\hypersetup{pdfstartview=FitH}
\hypersetup{pdfpagemode=UseNone}
\hypersetup{pdfsource={}}
\hypersetup{pdflang={en-UK}}
\hypersetup{pdfcopyright={Copyright 2017-2018 Niklas Beisert.
  This work may be distributed and/or modified under the
  conditions of the LaTeX Project Public License, either version 1.3
  of this license or (at your option) any later version.}}
\hypersetup{pdflicenseurl={http://www.latex-project.org/lppl.txt}}
\hypersetup{pdfcontactaddress={ETH Zurich, ITP, HIT K,
  Wolfgang-Pauli-Strasse 27}}
\hypersetup{pdfcontactpostcode={8093}}
\hypersetup{pdfcontactcity={Zurich}}
\hypersetup{pdfcontactcountry={Switzerland}}
\hypersetup{pdfcontactemail={nbeisert@itp.phys.ethz.ch}}
\hypersetup{pdfcontacturl={http://people.phys.ethz.ch/\xmptilde nbeisert/}}

\newcommand{\secref}[1]{\hyperref[#1]{section \ref*{#1}}}

\parskip1ex
\parindent0pt
\let\olditemize\itemize
\def\itemize{\olditemize\parskip0pt}

\begin{document}

\title{The \textsf{childdoc} Package}
\hypersetup{pdftitle={The childdoc Package}}
\author{Niklas Beisert\\[2ex]
  Institut f\"ur Theoretische Physik\\
  Eidgen\"ossische Technische Hochschule Z\"urich\\
  Wolfgang-Pauli-Strasse 27, 8093 Z\"urich, Switzerland\\[1ex]
  \href{mailto:nbeisert@itp.phys.ethz.ch}
  {\texttt{nbeisert@itp.phys.ethz.ch}}}
\hypersetup{pdfauthor={Niklas Beisert}}
\hypersetup{pdfsubject={Manual for the LaTeX2e Package childdoc}}
\date{30 December 2018, \textsf{v2.0}}
\maketitle

\begin{abstract}\noindent
\textsf{childdoc} is a \LaTeXe{} package
that enables the direct compilation
of document sections included by |\include|
to individual files.
\end{abstract}

\begingroup
\parskip0ex
\tableofcontents
\endgroup

%%%%%%%%%%%%%%%%%%%%%%%%%%%%%%%%%%%%%%%%%%%%%%%%%%%%%%%%%%%%%%%%%%%%%%%%%%%%%%%%
%%%%%%%%%%%%%%%%%%%%%%%%%%%%%%%%%%%%%%%%%%%%%%%%%%%%%%%%%%%%%%%%%%%%%%%%%%%%%%%%
\section{Introduction}

\LaTeX{} provides a mechanism to structure a large document (such as a book)
into a main file and several child files (containing the chapters)
using the |\include| command.
This mechanism is beneficial for documents
which span hundreds of pages in order to
make the source file(s) more manageable.
Moreover, compilation can be restricted to
selected child files by means of the |\includeonly| command.
The latter feature can be used to reduce the compilation time while editing
(this was significantly more useful in the earlier days of \LaTeX{})
or to generate a smaller document which is easier to navigate.
Another application of |\includeonly| is to generate
documents consisting of selected parts of the complete document.

However, there are a few drawbacks of the plain |\include| mechanism:
\begin{itemize}
\item
The child files cannot be compiled on their own,
they can only be compiled via the main file.
A naive editing environment
(such as a text editor with an option
to have the current file processed by \LaTeX)
may require one to switch to the main file before compiling;
attempting to compile the child file produces errors.
\item
The main file must be modified (each time)
to adjust the |\includeonly| command
to the present needs. This easily leaves the main file in a messy state.
\item
The generated document will always carry the filename
of the main document. This is inconvenient if
several child files are to be compiled and
to be kept for distribution.
\end{itemize}

The present package provides a simple interface
to make child files individually compilable by \LaTeX{}.
Compiling a child file then has the same effect as compiling
the main file with an |\includeonly| command
to select the appropriate child.
Moreover the generated document will carry the name of the child
rather than the main file.
This resolves all three above issues.

This feature is meant to make the editing of books,
thesis documents and lecture notes somewhat more convenient.
However, the package can also be used efficiently for
composing a series of documents (such as exercise sheets)
which are typically distributed individually.
It then assists the author in generating the individual documents
(potentially in different versions)
as well as a document containing the collected series.
Another application is in developing style files
or other kinds of included material
where compilation of the style file could redirect
to a sample or test file.

%%%%%%%%%%%%%%%%%%%%%%%%%%%%%%%%%%%%%%%%%%%%%%%%%%%%%%%%%%%%%%%%%%%%%%%%%%%%%%%%
%%%%%%%%%%%%%%%%%%%%%%%%%%%%%%%%%%%%%%%%%%%%%%%%%%%%%%%%%%%%%%%%%%%%%%%%%%%%%%%%
\section{Usage}

First of all, the package \textsf{childdoc} is \emph{not} a standard
\LaTeXe{} |.sty| style file! Therefore it needs to be invoked in
a non-standard way.

%%%%%%%%%%%%%%%%%%%%%%%%%%%%%%%%%%%%%%%%%%%%%%%%%%%%%%%%%%%%%%%%%%%%%%%%%%%%%%%%
\subsection{Included Files}
\label{sec:include}

%%%%%%%%%%%%%%%%%%%%%%%%%%%%%%%%%%%%%%%%
\DescribeMacro{\childdocmain}
To use the package, add the commands
\begin{center}
\begin{tabular}{l}
|\input{childdoc.def}|\\
|\childdocmain{}|\\
\end{tabular}
\end{center}
at the very top of the main \LaTeX{} file,
in particular \emph{before} the |\documentclass| statement!
The argument of |\childdocmain| should be left empty
(but it must be present).

%%%%%%%%%%%%%%%%%%%%%%%%%%%%%%%%%%%%%%%%
\DescribeMacro{\childdocof}
Furthermore, add the commands
\begin{center}
\begin{tabular}{l}
|\input{childdoc.def}|\\
|\childdocof{|\textit{main}|}|\\
\end{tabular}
\end{center}
at the top of every child file \textit{child}
which is included by |\include{|\textit{child}|}|
from within the main file
(or at least for those files to be compiled individually).
The argument \textit{main} must be the filename of the main file.

There are a couple of
considerations in setting up the main and child documents:

%%%%%%%%%%%%%%%%%%%%%%%%%%%%%%%%%%%%%%%%
\paragraph{Restrictions.}

Please note the following restrictions:
\begin{itemize}
\item
|\childdocmain| must be called with one argument \textit{main}
to ensure compatibility with earlier version of the package.
It must either be empty (|\childdocmain{}|)
or precisely match the filename of the main file in which it is specified.
See \secref{sec:detection} for further information.
\item
The filename \textit{main} must be specified without the |.tex| extension.
\item
The filename \textit{main} is case sensitive
(even in case-insensitive file systems)
due to internal string comparison.
\item
The argument \textit{main} should be fully expanded, it cannot be a macro.
\item
Subdirectories and special characters should be avoided in filenames.
\item
The command |\childdocmain{|\textit{main}|}| must be followed by a whitespace.
It should not be followed immediately by another command
or by a comment mark `|%|'.
This is because the \TeX{} parser reads the token immediately following
the argument of |\childdocmain| and puts it
at the beginning of every child section;
however, a white\-space is ignored.
\end{itemize}

%%%%%%%%%%%%%%%%%%%%%%%%%%%%%%%%%%%%%%%%
\paragraph{Content of Main File.}

It is advisable to place all content in the child files included by |\include|.
Any output contained in the main file will appear in all child documents
unless suppressed manually;
it cannot be suppressed automatically by the |\includeonly| directive
and thus should normally be avoided.
A method to include some content in the main file
by means of conditional processing is described in \secref{sec:conditional}.

%%%%%%%%%%%%%%%%%%%%%%%%%%%%%%%%%%%%%%%%
\paragraph{Page Numbering.}

When only a part of the document is compiled,
the appropriate numbering of pages
(as well as other status parameters)
is determined from the |.aux| files.
The latter contain information from previous passes.
However this information needs to propagate through
all intermediate child documents.
Therefore the page numbering in child documents may well
be inconsistent until the complete document is compiled at least once.

A useful (if unconventional) way to always ensure a consistent
page numbering is to restart the numbering in each child document
and denote the pages by `\textit{child}|.|\textit{page}'
where \textit{child} represents the chapter/section number of the child file.
This can be achieved by the command
|\numberwithin{page}{|\textit{child}|}|
of the \textsf{amsmath} package
where \textit{child} can be |chapter| or |section|
depending on the chosen structuring.
Alternatively, one can modify the macro |\thepage| appropriately
and reset the counter |page| at the start of each child file.

%%%%%%%%%%%%%%%%%%%%%%%%%%%%%%%%%%%%%%%%%%%%%%%%%%%%%%%%%%%%%%%%%%%%%%%%%%%%%%%%
\subsection{Conditional Processing}
\label{sec:conditional}

The package provides a mechanism to compile different versions
of a document. To customise the versions further some conditional processing
can come in handy to distinguish which version is being compiled.
The package provides two macros to describe the compilation context:

%%%%%%%%%%%%%%%%%%%%%%%%%%%%%%%%%%%%%%%%
\DescribeMacro{\ifchilddoc}
The conditional |\ifchilddoc| distinguishes between the compilation of
child documents and the main document:
%
\begin{center}
|\ifchilddoc |\textit{child-code}| |[|\||else |\textit{main-code}]| \||fi|
\end{center}

%%%%%%%%%%%%%%%%%%%%%%%%%%%%%%%%%%%%%%%%
\DescribeMacro{\childdocname}
\DescribeMacro{\childdocjob}
The macro |\childdocname| contains the filename (without extension)
of the main or child file being processed.
Note that |\childdocjob| will always contain the name of the main file.

%%%%%%%%%%%%%%%%%%%%%%%%%%%%%%%%%%%%%%%%
\paragraph{Title Page.}

Conditional processing can be used to include a title or banner page
in the main document when proper precautions are taken.
Importantly, the code in the main file should ensure that the page counter
(as well as other status parameters which are stored in the |.aux| files)
takes the same value after the conditional processing.
Otherwise the page numbers may take divergent values
depending on which part is compiled.

For example, a title page could be declared by:
%
\begin{center}
\begin{tabular}{l}
|\ifchilddoc\||else|\\
|\addtocounter{page}{-1}|\\
\textit{code for title page}\\
|\newpage|\\
|\||fi|
\end{tabular}
\end{center}
%
A banner page for the child documents can be generated by:
%
\begin{center}
\begin{tabular}{l}
|\ifchilddoc|\\
|\addtocounter{page}{-1}|\\
\textit{code for banner page}\\
|\newpage|\\
|\||fi|
\end{tabular}
\end{center}
%
Here one could write a message such as:
\begin{center}
|This is the part \childdocname{} of \childdocjob{}.|
\end{center}

%%%%%%%%%%%%%%%%%%%%%%%%%%%%%%%%%%%%%%%%%%%%%%%%%%%%%%%%%%%%%%%%%%%%%%%%%%%%%%%%
\subsection{Flags}
\label{sec:flags}

The package makes it easy to generate different versions
of the main or child documents.
To this end compilation flags can be defined
and assigned different default values.
They will be particularly useful in conjunction
with the forwarding mechanism described in \secref{sec:forward}.

For example, it may be useful to have a flag |\version|
which can be set to |draft| or |final|.
The document source will contain some conditional code
depending on the value of |\version|.
Suppose further, the flag should default to |final| for the main file
and to |draft| for child files
which is a natural assignment for editing the document.
This is achieved by placing the following code
in the preamble of the main document
(below the |\childdocmain| directive):
%
\begin{center}
\begin{tabular}{l}
|\ifchilddoc|\\
|\providecommand{\version}{draft}|\\
|\||else|\\
|\providecommand{\version}{final}|\\
|\||fi|
\end{tabular}
\end{center}
%
The definition by |\providecommand| makes sure
that previous definitions are not overwritten.
Further statements |\providecommand{\version}{...}|
can thus be added before the above code to override it.

For the main file, one might add a line
(between |\childdocmain| and the above block)
%
\begin{center}
|%\ifchilddoc\||else\providecommand{\version}{draft}\||fi|
\end{center}
%
which can be uncommented to produce a draft version.
Likewise one can add a line to the very top of a child file
(above the |\childdocof{|\textit{main}|}| directive)
%
\begin{center}
|%\providecommand{\version}{final}|
\end{center}
%
which can be uncommented to produce the final version of this child document.

%%%%%%%%%%%%%%%%%%%%%%%%%%%%%%%%%%%%%%%%%%%%%%%%%%%%%%%%%%%%%%%%%%%%%%%%%%%%%%%%
\subsection{Forwarding}
\label{sec:forward}

Different versions of the main or child documents
using compilation flags as described in \secref{sec:flags}
can be (permanently) stored in different files
for convenient compilation, viewing and distribution.
To this end, the package defines a command
to pass on compilation to a different file:

%%%%%%%%%%%%%%%%%%%%%%%%%%%%%%%%%%%%%%%%
\DescribeMacro{\childdocforward}
The command |\childdocforward| redirects processing to
another source file:
%
\begin{center}
\begin{tabular}{l}
|\input{childdoc.def}|\\
|\childdocforward[|\textit{main}|]{|\textit{dest}|}|\\
\end{tabular}
\end{center}
%
The argument \textit{dest} is the destination file
(without extension).
It should be the main file or one of the child files.
Note that further \textsf{childdoc} directives
such as |\childdocof| and |\childdocforward|
in the indicated file will be processed in this form.
The optional argument \textit{main}
passes on directly to the main file \textit{main}
while pretending to compile the child \textit{dest}.
This form behaves as if \textit{dest}
issues |\childdocof{|\textit{main}|}| right away,
and no further \textsf{childdoc} directives will be processed.

%%%%%%%%%%%%%%%%%%%%%%%%%%%%%%%%%%%%%%%%
\DescribeMacro{\...prefix}
In the alternative form |\childdocforwardprefix|,
%
\begin{center}
\begin{tabular}{l}
|\input{childdoc.def}|\\
|\childdocforwardprefix[|\textit{main}|]{|\textit{prefix}|}{|\textit{dest}|}|
\end{tabular}
\end{center}
%
the destination file is determined by a pattern
depending on the current file:
To make this work, the current file must be called
`{\textit{prefix}\hspace{0.2em}\textit{suffix}}'
with \textit{prefix} matching precisely the argument.
Processing is then passed on to the file
`{\textit{dest}\hspace{0.2em}\textit{suffix}}'.
Surely, the same effect is achieved by
directly specifying the
argument `{\textit{dest}\hspace{0.2em}\textit{suffix}}'
in the first form.
However, that requires to set up a different file
for each child. With the alternative form of the command
all these files can have exactly the same content
which simplifies setting them up and maintaining them.

For example, the following file |draft.tex|
with a compilation flag |\version| as described in \secref{sec:flags}
compiles the main document as a draft:
%
\begin{center}
\begin{tabular}{l}
|\def\version{draft}|\\
|\input{childdoc.def}|\\
|\childdocforward{|\textit{main}|}|
\end{tabular}
\end{center}
%
Likewise, the following files |final|\textit{nn}|.tex|
compile the final version of the child document
|child|\textit{nn}|.tex|:
%
\begin{center}
\begin{tabular}{l}
|\def\version{final}|\\
|\input{childdoc.def}|\\
|\childdocforwardprefix{final}{child}|
\end{tabular}
\end{center}
%

Note that when several versions of a main file and/or of each child file
are to be generated, it may be convenient to set up a |Makefile| or
shell script to automatise the process.

%%%%%%%%%%%%%%%%%%%%%%%%%%%%%%%%%%%%%%%%%%%%%%%%%%%%%%%%%%%%%%%%%%%%%%%%%%%%%%%%
\subsection{Command Line Processing}
\label{sec:commandline}

The effect of redirection files can also be achieved by invoking
the \LaTeX{} compiler with a more elaborate command line.
Most conveniently this should be done as part
of a shell script or a |Makefile|.

When using \textsf{childdoc} in the main file, the following
command lines effectively perform a redirection
(note that depending on the shell being used,
backslashes may have to be doubled: `|\|' $\to$ `|\\|'):
%
\begin{center}
|... -jobname "|\textit{target}|" |\\|"|[\textit{flags}]%
|\input{childdoc.def}\childdocforward[|\textit{main}|]{|\textit{dest}|}"|
\end{center}
%
Here \textit{target} is the name of the output file,
\textit{main} is the name of the main file
and \textit{dest} is the name of the main or child file to be processed
(all filenames without extensions).
The optional argument \textit{main} can be omitted
if \textit{main} matches \textit{dest}.
Optionally, compilation \textit{flags} can be defined via |\def| commands.
This command line makes the \TeX{} engine believe
it is compiling the file \textit{target}
whose content is specified as the latter parameter.
The provided code then forwards the processing to
\textit{main} or \textit{dest} as described in \secref{sec:forward}.

%%%%%%%%%%%%%%%%%%%%%%%%%%%%%%%%%%%%%%%%%%%%%%%%%%%%%%%%%%%%%%%%%%%%%%%%%%%%%%%%
\subsection{Include by Input}
\label{sec:input}

Including child documents by |\include| has some restrictions by design.
Most notably, the content of a child document always occupies
its own set of pages; pages cannot be shared between child documents.
Usually, this behaviour makes perfect sense
because each child document contain an essential part of the document.
However, in some situations it may be desirable to compose
a document from a collection of parts
without having mandatory page breaks between then.
For this case, the package
provides a mechanism to include parts
by |\input| which can also be processed individually.
However, by construction this mechanism
requires manual handling of the content to be output.

%%%%%%%%%%%%%%%%%%%%%%%%%%%%%%%%%%%%%%%%
\DescribeMacro{\ifchilddocmanual}
The main file should be prepared as usual, see \secref{sec:include}.
However, the document body must make a distinction
between processing of an individual part and of the main document, e.g.:
%
\begin{center}
\begin{tabular}{l}
|\ifchilddocmanual|\\
|\input{\childdocname}|\\
|\||else|\\
\textit{document body with }|\input{|\textit{part}|}|\\
|\||fi|
\end{tabular}
\end{center}
%
The conditional |\ifchilddocmanual| is true whenever
a part to be included by |\input| is being compiled,
and the name of the part is stored in |\childdocname|.

%%%%%%%%%%%%%%%%%%%%%%%%%%%%%%%%%%%%%%%%
\DescribeMacro{\childdocby}
Each part to be included by |\input| should start with:
%
\begin{center}
\begin{tabular}{l}
|\input{childdoc.def}|\\
|\childdocby{|\textit{main}|}|\\
\end{tabular}
\end{center}
%
The directive |\childdocby| is similar to |\childdocof|
described in \secref{sec:include},
but the subsequent selection of content must be done manually.
To that end, both |\ifchilddoc| and |\ifchilddocmanual|
will be true upon processing of a part,
and the name of the part is stored in |\childdocname|.
Note that |\jobname| will be set to the filename of the current part
so that each part receives an individual |.aux| file
that does not interfere with the |.aux| file(s) of the main document.
This behaviour can be altered by the alternative form
|\childdocby[*]{|\textit{main}|}| (with a non-empty optional argument)
which uses the |.aux| file of the main document
by setting |\jobname| to \textit{main}.

%%%%%%%%%%%%%%%%%%%%%%%%%%%%%%%%%%%%%%%%%%%%%%%%%%%%%%%%%%%%%%%%%%%%%%%%%%%%%%%%
\subsection{Driver Development}
\label{sec:driver}

The \textsf{childdoc} mechanism can also be use for the development
of definition files such as \LaTeX{} styles or classes.
This case differs from the above setup with multiple parts
included by |\include| in that no |\includeonly| should be invoked.
This can be achieved by starting the include file
(before |\ProvidesPackage|) with:
%
\begin{center}
\begin{tabular}{l}
|\input{childdoc.def}|\\
|\childdocforward{|\textit{main}|}|\\
\end{tabular}
\end{center}
%
or alternatively with:
%
\begin{center}
\begin{tabular}{l}
|\input{childdoc.def}|\\
|\childdocby{|\textit{main}|}|\\
\end{tabular}
\end{center}
%
Both forms have slightly different effects as described above.
The main file is prepared as usual, see \secref{sec:include}.

%%%%%%%%%%%%%%%%%%%%%%%%%%%%%%%%%%%%%%%%%%%%%%%%%%%%%%%%%%%%%%%%%%%%%%%%%%%%%%%%
\subsection{Legacy Detection}
\label{sec:detection}

The directive |\childdocmain| in the main file can detect
whether the complete document or merely a child is to be compiled
even without using the directive |\childdocof|.
This method is deprecated because it is less robust
and there is no compelling reason to use it;
it is merely provided for backward compatibility
and it may be removed in future versions.

If the detection mechanism is to be used,
it is mandatory to correctly specify
the filename of the main file as the argument of |\childdocmain|:
%
\begin{center}
\begin{tabular}{l}
|\input{childdoc.def}|\\
|\childdocmain{|\textit{main}|}|\\
\end{tabular}
\end{center}
%
If |\jobname| does not match the argument \textit{main} of |\childdocmain|,
it is assumed that |\jobname| points to the child file to be compiled.
When using |\childdocmain| with the main file specified as argument,
it suffices to start a child file
with just |\input{|\textit{main}|}|
without loading of the package and using |\childdocof|.
If instead all processing is done
with the appropriate \textsf{childdoc} directives,
the argument of \textit{main} of |\childdocmain| can be empty.

An alternative version of the command line processing described
in \secref{sec:commandline} using the detection mechanism reads:
%
\begin{center}
|... -jobname "|\textit{target}|" "|[\textit{flags}]%
[|\def\jobname{|\textit{dest}|}|]|\input{|\textit{main}|}"|
\end{center}

%%%%%%%%%%%%%%%%%%%%%%%%%%%%%%%%%%%%%%%%%%%%%%%%%%%%%%%%%%%%%%%%%%%%%%%%%%%%%%%%
\subsection{Manual Code}
\label{sec:manual}

In case one cannot be certain whether the definitions file |childdoc.def|
is installed on the target \TeX{} distribution
and one prefers not to ship it,
it is conceivable to paste a few relevant commands into the sources.

To that end, drop all statements |\input{childdoc.def}|
and perform the replacements as outlined below.
Instead of |\childdocmain{|\textit{main}|}| add the following code
to the top of the main file:
%
\begin{center}
\begin{tabular}{l}
|\||ifdefined\childdocname\endinput\||fi\newif\ifchilddoc|\\
|\edef\childdocname{\scantokens\expandafter{\jobname\noexpand}}|\\
|\def\childdocmain{|\textit{main}|}\||ifx\childdocmain\childdocname\||else|\\
|\childdoctrue\includeonly{\childdocname}\let\jobname\childdocmain\||fi|\\
\end{tabular}
\end{center}
%
Instead of |\childdocof{|\textit{main}|}| just include the main file
at the top of each child file:
%
\begin{center}
|\input{|\textit{main}|}|
\end{center}
%
A simple redirection |\childdocforward{|\textit{dest}|}| is achieved by:
%
\begin{center}
|\def\jobname{|\textit{dest}|}\input{\jobname}|
\end{center}
%
The redirection with prefix
|\childdocforwardprefix[|\textit{prefix}|]{|\textit{dest}|}|
is accomplished by:
%
\begin{center}
\begin{tabular}{l}
|{\edef\jobname{\scantokens\expandafter{\jobname\noexpand}}|\\
|\def\redirectjob |\textit{prefix}|#1~~~{\gdef\jobname{|\textit{dest}|#1}}|\\
|\expandafter\redirectjob\jobname~~~}\input{\jobname}|
\end{tabular}
\end{center}

In an alternative approach,
child documents can be compiled by a specific command line
without additional code or specific definitions:
%
\begin{center}
|... -jobname "|\textit{target}|" "|[\textit{flags}]%
|\includeonly{|\textit{dest}|}\input{|\textit{main}|}"|
\end{center}
%

%%%%%%%%%%%%%%%%%%%%%%%%%%%%%%%%%%%%%%%%%%%%%%%%%%%%%%%%%%%%%%%%%%%%%%%%%%%%%%%%
%%%%%%%%%%%%%%%%%%%%%%%%%%%%%%%%%%%%%%%%%%%%%%%%%%%%%%%%%%%%%%%%%%%%%%%%%%%%%%%%
\section{Information}

%%%%%%%%%%%%%%%%%%%%%%%%%%%%%%%%%%%%%%%%%%%%%%%%%%%%%%%%%%%%%%%%%%%%%%%%%%%%%%%%
\subsection{Copyright}

Copyright \copyright{} 2017--2018 Niklas Beisert

This work may be distributed and/or modified under the
conditions of the \LaTeX{} Project Public License, either version 1.3
of this license or (at your option) any later version.
The latest version of this license is in
  \url{http://www.latex-project.org/lppl.txt}
and version 1.3 or later is part of all distributions of \LaTeX{}
version 2005/12/01 or later.

This work has the LPPL maintenance status `maintained'.

The Current Maintainer of this work is Niklas Beisert.

This work consists of the files |README.txt|, |childdoc.ins| and |childdoc.dtx|
as well as the derived files |childdoc.def|, |cdocsamp.tex|
with |cdocsch1.tex|, |cdocsch2.tex|, |cdocspt3.tex|, |cdocspt4.tex|,
|cdocsdrf.tex|, |cdocsfn1.tex|, |cdocsfn2.tex|
as well as |childdoc.pdf|.

%%%%%%%%%%%%%%%%%%%%%%%%%%%%%%%%%%%%%%%%%%%%%%%%%%%%%%%%%%%%%%%%%%%%%%%%%%%%%%%%
\subsection{Files and Installation}

The package consists of the files:
%
\begin{center}
\begin{tabular}{ll}
    |README.txt|   & readme file \\
    |childdoc.ins| & installation file \\
    |childdoc.dtx| & source file \\
    |childdoc.def| & definition file \\
    |cdocsamp.tex| & sample main file \\
    |cdocsch1.tex| & sample include file \\
    |cdocsch2.tex| & sample include file \\
    |cdocspt3.tex| & sample part file \\
    |cdocspt4.tex| & sample part file \\
    |cdocsdrf.tex| & sample redirection file \\
    |cdocsfn1.tex| & sample redirection file \\
    |cdocsfn2.tex| & sample redirection file \\
    |childdoc.pdf| & manual
\end{tabular}
\end{center}
%
The distribution consists of the files
|README.txt|, |childdoc.ins| and |childdoc.dtx|.
%
\begin{itemize}
\item
Run (pdf)\LaTeX{} on |childdoc.dtx|
to compile the manual |childdoc.pdf| (this file).
\item
Run \LaTeX{} on |childdoc.ins| to create the definitions file |childdoc.def|
and the sample |cdocsamp.tex| with include files
|cdocsch1.tex|, |cdocsch2.tex|, |cdocspt3.tex|, |cdocspt4.tex|,
|cdocsdrf.tex|, |cdocsfn1.tex|, |cdocsfn2.tex|.
Then copy the file |childdoc.def| to an appropriate directory of your \LaTeX{}
distribution, e.g.\ \textit{texmf-root}|/tex/latex/childdoc|.
\end{itemize}

%%%%%%%%%%%%%%%%%%%%%%%%%%%%%%%%%%%%%%%%%%%%%%%%%%%%%%%%%%%%%%%%%%%%%%%%%%%%%%%%
\subsection{Related CTAN Packages}

There are several other packages which offer a similar functionality:
%
\begin{itemize}
\item
The packages
\href{http://ctan.org/pkg/docmute}{\textsf{docmute}},
\href{http://ctan.org/pkg/includex}{\textsf{includex}} and
\href{http://ctan.org/pkg/standalone}{\textsf{standalone}}
provide commands to include only the document body of
a child file thus allowing both files to be compiled individually.
\item
The packages \href{http://ctan.org/pkg/subdocs}{\textsf{subdocs}}
and \href{http://ctan.org/pkg/subfiles}{\textsf{subfiles}}
provide structures in which the main and child documents can be
encapsulated and allowing them to be compiled individually.
The inclusion mechanism is different from the conventional |\include|.
\item
The package \href{http://ctan.org/pkg/combine}{\textsf{combine}}
is an elaborate solution to combine several documents into one.
\end{itemize}
%
See also the CTAN topic \href{http://ctan.org/topic/subdocs}{\textsf{subdocs}}
for further related packages.
The present package differs from the above solutions in that
a document structure constructed with the conventional |\include| mechanism
just needs two extra commands at the top of every file
such that all constituent files can be compiled individually.

%%%%%%%%%%%%%%%%%%%%%%%%%%%%%%%%%%%%%%%%%%%%%%%%%%%%%%%%%%%%%%%%%%%%%%%%%%%%%%%%
%\subsection{Feature Suggestions}
%
%The following is a list of features which may be useful for future
%versions of this package:
%%
%\begin{itemize}
%\item
%\ldots
%\end{itemize}

%%%%%%%%%%%%%%%%%%%%%%%%%%%%%%%%%%%%%%%%%%%%%%%%%%%%%%%%%%%%%%%%%%%%%%%%%%%%%%%%
\subsection{Revision History}

%%%%%%%%%%%%%%%%%%%%%%%%%%%%%%%%%%%%%%%%
\paragraph{v2.0:} 2018/12/30

\begin{itemize}
\item
immediate forward processing
\item
added |\childdocby| mechanism
\item
manual restructured
\end{itemize}

%%%%%%%%%%%%%%%%%%%%%%%%%%%%%%%%%%%%%%%%
\paragraph{v1.6:} 2018/01/17

\begin{itemize}
\item
application for development of include files
\item
corrections to manual
\end{itemize}

%%%%%%%%%%%%%%%%%%%%%%%%%%%%%%%%%%%%%%%%
\paragraph{v1.5:} 2017/05/21

\begin{itemize}
\item
more complete structuring introduced
\item
|\childdocof| introduced
\item
|\childdoc| renamed to |\childdocmain|
\item
|\childredirect| renamed to |\childdocforward| and |\childdocforwardprefix|
and functionality expanded
\end{itemize}

%%%%%%%%%%%%%%%%%%%%%%%%%%%%%%%%%%%%%%%%
\paragraph{v1.0:} 2017/04/27

\begin{itemize}
\item
manual and install package
\item
first version published on CTAN
\end{itemize}

%%%%%%%%%%%%%%%%%%%%%%%%%%%%%%%%%%%%%%%%
\paragraph{v0.6:} 2017/04/26

\begin{itemize}
\item
redirection mechanism added
\end{itemize}

%%%%%%%%%%%%%%%%%%%%%%%%%%%%%%%%%%%%%%%%
\paragraph{v0.5:} 2017/04/26

\begin{itemize}
\item
functionality in definition file
\end{itemize}


%%%%%%%%%%%%%%%%%%%%%%%%%%%%%%%%%%%%%%%%%%%%%%%%%%%%%%%%%%%%%%%%%%%%%%%%%%%%%%%%
%%%%%%%%%%%%%%%%%%%%%%%%%%%%%%%%%%%%%%%%%%%%%%%%%%%%%%%%%%%%%%%%%%%%%%%%%%%%%%%%
%%%%%%%%%%%%%%%%%%%%%%%%%%%%%%%%%%%%%%%%%%%%%%%%%%%%%%%%%%%%%%%%%%%%%%%%%%%%%%%%
\appendix

\settowidth\MacroIndent{\rmfamily\scriptsize 000\ }

 \DocInput{childdoc.dtx}

\end{document}
%</driver>
% \fi
%
% %%%%%%%%%%%%%%%%%%%%%%%%%%%%%%%%%%%%%%%%%%%%%%%%%%%%%%%%%%%%%%%%%%%%%%%%%%%%%%
% %%%%%%%%%%%%%%%%%%%%%%%%%%%%%%%%%%%%%%%%%%%%%%%%%%%%%%%%%%%%%%%%%%%%%%%%%%%%%%
% \section{Sample}
%\iffalse
%<*samplemain>
%\fi
%
% The following presents a sample document
% with two chapters, two parts, a title page,
% a compile flag as well as three forwarding files to set the flag.
% It consists of eight |.tex| files:
% \begin{center}
% \begin{tabular}{ll}
% |cdocsamp.tex|&main file\\
% |cdocsch1.tex|&include file for chapter 1\\
% |cdocsch2.tex|&include file for chapter 2\\
% |cdocspt3.tex|&include file for part 3\\
% |cdocspt4.tex|&include file for part 4\\
% |cdocsdrf.tex|&forwarding file for main file in draft mode\\
% |cdocsfi1.tex|&forwarding file for final version of chapter 1\\
% |cdocsfi2.tex|&forwarding file for final version of chapter 2\\
% \end{tabular}
% \end{center}
% Each of the eight files can be compiled directly by the \LaTeX{} compiler.
%
% %%%%%%%%%%%%%%%%%%%%%%%%%%%%%%%%%%%%%%
% \paragraph{Main File.}
%
% The main file is called |cdocsamp.tex|.
%
% Load the \textsf{childdoc} definitions and
% declare the filename for the main document:
%    \begin{macrocode}
\input{childdoc.def}
\childdocmain{}
%    \end{macrocode}

% Optional override for |\version| flag:
%    \begin{macrocode}
%%\ifchilddoc\else\providecommand{\version}{draft}\fi
%    \end{macrocode}

% Define the default values for the |\version| flag
% (|final| for the main file and |draft| for childs):
%    \begin{macrocode}
\ifchilddoc
\providecommand{\version}{draft}
\else
\providecommand{\version}{final}
\fi
%    \end{macrocode}

% Load the standard document class:
%    \begin{macrocode}
\documentclass[12pt]{article}
%    \end{macrocode}

% Start the document body:
%    \begin{macrocode}
\begin{document}
%    \end{macrocode}

% Declare a title page.
% Print title, part of document being processed and version flag:
%    \begin{macrocode}
\addtocounter{page}{-1}
\begin{center}
{\LARGE\bfseries{}childdoc example\par}
\vspace{1cm}
\ifchilddoc
\ifchilddocmanual part\else chapter\fi:
`\childdocname' of `\childdocjob'\par
\else
main document: `\childdocjob'\par
\fi
version: \version\par
\end{center}
\newpage
%    \end{macrocode}

% Manually include selected file,
% otherwise process as usual:
%    \begin{macrocode}
\ifchilddocmanual
\section*{part `\childdocname'}
\input{\childdocname}
\else
%    \end{macrocode}

% Include the two chapters:
%    \begin{macrocode}
\include{cdocsch1}
\include{cdocsch2}
%    \end{macrocode}

% Include the two parts unless only chapters should be displayed:
%    \begin{macrocode}
\ifchilddoc\else
\section{part three}
\input{cdocspt3}
\section{part four}
\input{cdocspt4}
\fi
%    \end{macrocode}

% Process as usual until here:
%    \begin{macrocode}
\fi
%    \end{macrocode}

% End of document body:
%    \begin{macrocode}
\end{document}
%    \end{macrocode}
%\iffalse
%</samplemain>
%\fi
%
% %%%%%%%%%%%%%%%%%%%%%%%%%%%%%%%%%%%%%%
% \paragraph{Chapter Include Files.}
%
% The include files are called |cdocsch1.tex| and |cdocsch2.tex|.
%
%\iffalse
%<*samplechap1|samplechap2>
%\fi

% Optional override for |\version| flag:
%    \begin{macrocode}
%%\providecommand{\version}{final}
%    \end{macrocode}

% Include the main document:
%    \begin{macrocode}
\input{childdoc.def}
\childdocof{cdocsamp}
%    \end{macrocode}

%\iffalse
%</samplechap1|samplechap2>
%\fi
%
%\iffalse
%<*samplechap1>
%\fi
% Some text for chapter 1:
%    \begin{macrocode}
\section{one}
some text in chapter one
%    \end{macrocode}

%\iffalse
%</samplechap1>
%\fi
% Some text for chapter 2:
%\iffalse
%<*samplechap2>
%\fi
%    \begin{macrocode}
\section{two}
more text in chapter two
%    \end{macrocode}

%\iffalse
%</samplechap2>
%\fi
%
% %%%%%%%%%%%%%%%%%%%%%%%%%%%%%%%%%%%%%%
% \paragraph{Part Include Files.}
%
% The include files are called |cdocspt3.tex| and |cdocspt4.tex|.
%
%\iffalse
%<*samplepart3|samplepart4>
%\fi

% Optional override for |\version| flag:
%    \begin{macrocode}
%%\providecommand{\version}{final}
%    \end{macrocode}

% Include the main document:
%    \begin{macrocode}
\input{childdoc.def}
\childdocby{cdocsamp}
%    \end{macrocode}

%\iffalse
%</samplepart3|samplepart4>
%\fi
%
%\iffalse
%<*samplepart3>
%\fi
% Some text for part 3:
%    \begin{macrocode}
some text in part three
%    \end{macrocode}

%\iffalse
%</samplepart3>
%\fi
% Some text for part 4:
%\iffalse
%<*samplepart4>
%\fi
%    \begin{macrocode}
more text in part four
%    \end{macrocode}

%\iffalse
%</samplepart4>
%\fi
%
% %%%%%%%%%%%%%%%%%%%%%%%%%%%%%%%%%%%%%%
% \paragraph{Forwarding for a Complete Draft.}
%
% The following forwarding file |cdocsdrf.tex|
% compiles the main document in draft mode:
%\iffalse
%<*sampledraft>
%\fi
%    \begin{macrocode}
\def\version{draft}
\input{childdoc.def}
\childdocforward{cdocsamp}
%    \end{macrocode}

%\iffalse
%</sampledraft>
%\fi
%
% %%%%%%%%%%%%%%%%%%%%%%%%%%%%%%%%%%%%%%
% \paragraph{Forwarding for Final Version of the Chapters.}
%
% The following forwarding files |cdocsfn1.tex| and |cdocsfn2.tex|
% (with identical content)
% compile the final versions of the child documents
% |cdocsch1.tex| and |cdocsch2.tex|, respectively:
%\iffalse
%<*samplefinal>
%\fi
%    \begin{macrocode}
\def\version{final}
\input{childdoc.def}
\childdocforwardprefix[cdocsamp]{cdocsfn}{cdocsch}
%    \end{macrocode}

%\iffalse
%</samplefinal>
%\fi
%
% %%%%%%%%%%%%%%%%%%%%%%%%%%%%%%%%%%%%%%
% \paragraph{Command Line Processing.}
%
% The following three command lines generate the output files
% |cdocscld|, |cdocscl1| and |cdocscl2|
% which should be identical to
% |cdocsdrf|, |cdocsch1| and |cdocsfn2|, respectively:
% \begin{center}
% \begin{tabular}{l}
% |latex -jobname cdocscld \|\\
% |  "\def\version{draft}\input{childdoc.def}\childdocforward{cdocsamp}"|\\
% |latex -jobname cdocscl1 \|\\
% |  "\input{childdoc.def}\childdocforward[cdocsamp]{cdocsch1}"|\\
% |latex -jobname cdocscl2 \|\\
% |  "\def\version{final}\input{childdoc.def}\childdocforward{cdocsch2}"|
% \end{tabular}
% \end{center}
% Note that the trailing backslash on each first line
% merely continues the input to the second line
% (for convenient cut ant paste).
% Furthermore, the command |latex| can be replaced by any
% of its alternative versions such as |pdflatex|.
%
% %%%%%%%%%%%%%%%%%%%%%%%%%%%%%%%%%%%%%%%%%%%%%%%%%%%%%%%%%%%%%%%%%%%%%%%%%%%%%%
% %%%%%%%%%%%%%%%%%%%%%%%%%%%%%%%%%%%%%%%%%%%%%%%%%%%%%%%%%%%%%%%%%%%%%%%%%%%%%%
% \section{Implementation}
%\iffalse
%<*package>
%\fi
%
% This section describes the definitions file |childdoc.def|.

% The definitions cannot be loaded using |\usepackage| or |\RequirePackage|
% which has a mechanism to prevent loading a style file more than once.
% When loading the definitions by means of |\input|
% multiple instances have to be prevented manually:
%\iffalse
%This code needs to be before the `\ProvidesFile' directive
%which is defined at the beginning of this file.
%Therefore it is also placed there and commented out here.
%</package>
%<*discard>
%\fi
%    \begin{macrocode}
\ifdefined\childdocmain\endinput\fi
%    \end{macrocode}
%\iffalse
%</discard>
%<*package>
%\fi
%
% \macro{\ifchilddoc}
% \macro{\ifchilddocmanual}
% The conditional |\ifchilddoc| tells whether a
% child (true) or main (false) document is being compiled.
% The conditional |\ifchilddocmanual| tells whether
% the |\includeonly| mechanism is used (false) or
% the selection of child files must be performed manually (true).
% The definitions initialise to false:
%    \begin{macrocode}
\newif\ifchilddoc
\newif\ifchilddocmanual
%    \end{macrocode}

% \macro{\childdocname}
% \macro{\childdocjob}
% The macro |\childdocname| stores the name of the main document
% to be compiled. The macro |\childdocjob| stores the name of
% the document on which the \LaTeX{} compiler was originally invoked.
% The content of |\jobname| cannot be compared
% to filenames specified in the source due to different catcodes.
% The following code rescans |\jobname|, stores the result
% in |\childdocname| and saves a copy in |\childdocjob|:
%    \begin{macrocode}
\edef\childdocname{\scantokens\expandafter{\jobname\noexpand}}
\let\childdocjob\childdocname
%    \end{macrocode}

% \macro{\childdocdisable}
% The macro |\childdocdisable| prevents the main file
% from being processed more than once.
% At this stage, the main document command |\childdocmain|
% is assumed to be called once again where it should do nothing.
% Any subsequent call to it should prevent
% a secondary processing of the main document
% It overwrites the forwarding commands
% |\childdocof| and |\childdocforward|
% with empty macros to prevent further inclusions of the main document:
%    \begin{macrocode}
\newcommand{\childdocdisable}
{
  \renewcommand{\childdocmain}[1]{\renewcommand{\childdocmain}[1]{\endinput}}
  \renewcommand{\childdocof}[1]{}
  \renewcommand{\childdocby}[2][]{}
  \renewcommand{\childdocforward}[2][]{}
  \renewcommand{\childdocdisable}{}
}
%    \end{macrocode}

% \macro{\childdocmain}
% The macro |\childdocmain| is to be called at the top of the main file
% with nothing or the main filename (without extension) as argument.
% First, it breaks loops.
% If the argument is not empty and does not match |\childdocname|
% (which is set by the first inclusion of |childdoc.def|),
% |\ifchilddoc| is set to true, |\includeonly| is applied to the child file
% and |\jobname| is set to the main file
% (for proper handling of |.aux| files):
%    \begin{macrocode}
\newcommand{\childdocmain}[1]
{
  \childdocdisable\childdocmain{}
  \if?#1?\else
    \begingroup
      \def\childdoctmp{#1}
      \ifx\childdoctmp\childdocname
        \def\childdoctmp{}
      \else
        \def\childdoctmp
        {
          \childdoctrue
          \includeonly{\childdocname}
          \def\childdocjob{#1}
          \def\jobname{#1}
        }
      \fi
      \expandafter
    \endgroup
    \childdoctmp
  \fi
}
%    \end{macrocode}

% \macro{\childdocof}
% The command |\childdocof| redirects
% compilation to the main file |#1|.
%    \begin{macrocode}
\newcommand{\childdocof}[1]
{
  \childdocdisable
  \childdoctrue
  \includeonly{\childdocname}
  \def\jobname{#1}
  \def\childdocjob{#1}
  \input{#1}
}
%    \end{macrocode}

% \macro{\childdocby}
% The command |\childdocby| ....
%    \begin{macrocode}
\newcommand{\childdocby}[2][]
{
  \childdocdisable
  \childdoctrue
  \childdocmanualtrue
  \if?#1?\else
    \def\jobname{#2}
  \fi
  \def\childdocjob{#2}
  \input{#2}
  \endinput
}
%    \end{macrocode}

% \macro{\childdocforward}
% The command |\childdocforward| redirects
% compilation to the main file or
% (if the optional argument is given) a child file.
% Parameters are set as if the main file
% or a child file starting with |\childdocof| was compiled.
% Then compilation is handed over to the main file:
%    \begin{macrocode}
\newcommand{\childdocforward}[2][]
{
  \begingroup
    \if?#1?
      \def\childdoctmp
      {
        \def\childdocname{#2}
        \def\childdocjob{#2}
        \def\jobname{#2}
        \input{#2}
        \endinput
      }
    \else
      \def\childdoctmp
      {
        \childdocdisable
        \def\childdocname{#2}
        \childdoctrue
        \includeonly{#2}
        \def\childdocjob{#1}
        \def\jobname{#1}
        \input{#1}
        \endinput
      }
    \fi
    \expandafter
  \endgroup
  \childdoctmp
}
%    \end{macrocode}

% \macro{\childdocforwardprefix}
% The command |\childdocforwardprefix| redirects
% compilation to the main or a child file by means of a pattern.
% The prefix |#1| in the current filename is replaced by |#2|
% and the suffix of the current filename is kept
% (it is assumed that the filename does not contain the substring `|~~~|'
% which is used as a delimiter).
% Compilation is handed over to the new file by |\childdocforward|:
%    \begin{macrocode}
\newcommand{\childdocforwardprefix}[3][]
{
  \begingroup
    \def\childdocextract #2##1~~~{\def\childdoctmp{\childdocforward[#1]{#3##1}}}
    \expandafter\childdocextract\childdocname~~~
    \expandafter
  \endgroup
  \childdoctmp
}
%    \end{macrocode}

% \macro{\childdoc}
% The deprecated macro |\childdoc| is a legacy version of |\childdocmain|:
%    \begin{macrocode}
\newcommand{\childdoc}{\childdocmain}
%    \end{macrocode}

% \macro{\childdocredirect}
% The deprecated macro |\childdocredirect| is a legacy version
% of |\childdocforward| and |\childdocforwardprefix|:
%    \begin{macrocode}
\newcommand{\childdocredirect}[2][]
{
  \begingroup
    \if?#1?
      \def\childdoctmp{\childdocforward{#2}}
    \else
      \def\childdoctmp{\childdocforwardprefix{#1}{#2}}
    \fi
    \expandafter
  \endgroup
  \childdoctmp
}
%    \end{macrocode}

%\iffalse
%</package>
%\fi
%
\endinput
|\\
|\childdocforward{|\textit{main}|}|
\end{tabular}
\end{center}
%
Likewise, the following files |final|\textit{nn}|.tex|
compile the final version of the child document
|child|\textit{nn}|.tex|:
%
\begin{center}
\begin{tabular}{l}
|\def\version{final}|\\
|% \iffalse
%
% childdoc.dtx Copyright (C) 2017-2018 Niklas Beisert
%
% This work may be distributed and/or modified under the
% conditions of the LaTeX Project Public License, either version 1.3
% of this license or (at your option) any later version.
% The latest version of this license is in
%   http://www.latex-project.org/lppl.txt
% and version 1.3 or later is part of all distributions of LaTeX
% version 2005/12/01 or later.
%
% This work has the LPPL maintenance status `maintained'.
%
% The Current Maintainer of this work is Niklas Beisert.
%
% This work consists of the files childdoc.dtx and childdoc.ins
% and the derived files childdoc.def and cdocsamp.tex with
% cdocsch1.tex, cdocsch2.tex, cdocsdrf.tex, cdocsfn1.tex, cdocsfn2.tex.
%
%<package>\ifdefined\childdocmain\endinput\fi
%<package>\ProvidesFile{childdoc.def}[2018/12/30 v2.0 child document driver]
%<samplemain>\ProvidesFile{cdocsamp.tex}[2018/12/30 v2.0 sample for childdoc]
%<*driver>
%\ProvidesFile{childdoc.drv}[2018/12/30 v2.0 childdoc reference manual file]
\PassOptionsToClass{10pt,a4paper}{article}
\documentclass{ltxdoc}

\usepackage[margin=35mm]{geometry}
\usepackage{hyperref}
\usepackage{hyperxmp}
\usepackage[usenames]{color}

\hypersetup{colorlinks=true}
\hypersetup{pdfstartview=FitH}
\hypersetup{pdfpagemode=UseNone}
\hypersetup{pdfsource={}}
\hypersetup{pdflang={en-UK}}
\hypersetup{pdfcopyright={Copyright 2017-2018 Niklas Beisert.
  This work may be distributed and/or modified under the
  conditions of the LaTeX Project Public License, either version 1.3
  of this license or (at your option) any later version.}}
\hypersetup{pdflicenseurl={http://www.latex-project.org/lppl.txt}}
\hypersetup{pdfcontactaddress={ETH Zurich, ITP, HIT K,
  Wolfgang-Pauli-Strasse 27}}
\hypersetup{pdfcontactpostcode={8093}}
\hypersetup{pdfcontactcity={Zurich}}
\hypersetup{pdfcontactcountry={Switzerland}}
\hypersetup{pdfcontactemail={nbeisert@itp.phys.ethz.ch}}
\hypersetup{pdfcontacturl={http://people.phys.ethz.ch/\xmptilde nbeisert/}}

\newcommand{\secref}[1]{\hyperref[#1]{section \ref*{#1}}}

\parskip1ex
\parindent0pt
\let\olditemize\itemize
\def\itemize{\olditemize\parskip0pt}

\begin{document}

\title{The \textsf{childdoc} Package}
\hypersetup{pdftitle={The childdoc Package}}
\author{Niklas Beisert\\[2ex]
  Institut f\"ur Theoretische Physik\\
  Eidgen\"ossische Technische Hochschule Z\"urich\\
  Wolfgang-Pauli-Strasse 27, 8093 Z\"urich, Switzerland\\[1ex]
  \href{mailto:nbeisert@itp.phys.ethz.ch}
  {\texttt{nbeisert@itp.phys.ethz.ch}}}
\hypersetup{pdfauthor={Niklas Beisert}}
\hypersetup{pdfsubject={Manual for the LaTeX2e Package childdoc}}
\date{30 December 2018, \textsf{v2.0}}
\maketitle

\begin{abstract}\noindent
\textsf{childdoc} is a \LaTeXe{} package
that enables the direct compilation
of document sections included by |\include|
to individual files.
\end{abstract}

\begingroup
\parskip0ex
\tableofcontents
\endgroup

%%%%%%%%%%%%%%%%%%%%%%%%%%%%%%%%%%%%%%%%%%%%%%%%%%%%%%%%%%%%%%%%%%%%%%%%%%%%%%%%
%%%%%%%%%%%%%%%%%%%%%%%%%%%%%%%%%%%%%%%%%%%%%%%%%%%%%%%%%%%%%%%%%%%%%%%%%%%%%%%%
\section{Introduction}

\LaTeX{} provides a mechanism to structure a large document (such as a book)
into a main file and several child files (containing the chapters)
using the |\include| command.
This mechanism is beneficial for documents
which span hundreds of pages in order to
make the source file(s) more manageable.
Moreover, compilation can be restricted to
selected child files by means of the |\includeonly| command.
The latter feature can be used to reduce the compilation time while editing
(this was significantly more useful in the earlier days of \LaTeX{})
or to generate a smaller document which is easier to navigate.
Another application of |\includeonly| is to generate
documents consisting of selected parts of the complete document.

However, there are a few drawbacks of the plain |\include| mechanism:
\begin{itemize}
\item
The child files cannot be compiled on their own,
they can only be compiled via the main file.
A naive editing environment
(such as a text editor with an option
to have the current file processed by \LaTeX)
may require one to switch to the main file before compiling;
attempting to compile the child file produces errors.
\item
The main file must be modified (each time)
to adjust the |\includeonly| command
to the present needs. This easily leaves the main file in a messy state.
\item
The generated document will always carry the filename
of the main document. This is inconvenient if
several child files are to be compiled and
to be kept for distribution.
\end{itemize}

The present package provides a simple interface
to make child files individually compilable by \LaTeX{}.
Compiling a child file then has the same effect as compiling
the main file with an |\includeonly| command
to select the appropriate child.
Moreover the generated document will carry the name of the child
rather than the main file.
This resolves all three above issues.

This feature is meant to make the editing of books,
thesis documents and lecture notes somewhat more convenient.
However, the package can also be used efficiently for
composing a series of documents (such as exercise sheets)
which are typically distributed individually.
It then assists the author in generating the individual documents
(potentially in different versions)
as well as a document containing the collected series.
Another application is in developing style files
or other kinds of included material
where compilation of the style file could redirect
to a sample or test file.

%%%%%%%%%%%%%%%%%%%%%%%%%%%%%%%%%%%%%%%%%%%%%%%%%%%%%%%%%%%%%%%%%%%%%%%%%%%%%%%%
%%%%%%%%%%%%%%%%%%%%%%%%%%%%%%%%%%%%%%%%%%%%%%%%%%%%%%%%%%%%%%%%%%%%%%%%%%%%%%%%
\section{Usage}

First of all, the package \textsf{childdoc} is \emph{not} a standard
\LaTeXe{} |.sty| style file! Therefore it needs to be invoked in
a non-standard way.

%%%%%%%%%%%%%%%%%%%%%%%%%%%%%%%%%%%%%%%%%%%%%%%%%%%%%%%%%%%%%%%%%%%%%%%%%%%%%%%%
\subsection{Included Files}
\label{sec:include}

%%%%%%%%%%%%%%%%%%%%%%%%%%%%%%%%%%%%%%%%
\DescribeMacro{\childdocmain}
To use the package, add the commands
\begin{center}
\begin{tabular}{l}
|\input{childdoc.def}|\\
|\childdocmain{}|\\
\end{tabular}
\end{center}
at the very top of the main \LaTeX{} file,
in particular \emph{before} the |\documentclass| statement!
The argument of |\childdocmain| should be left empty
(but it must be present).

%%%%%%%%%%%%%%%%%%%%%%%%%%%%%%%%%%%%%%%%
\DescribeMacro{\childdocof}
Furthermore, add the commands
\begin{center}
\begin{tabular}{l}
|\input{childdoc.def}|\\
|\childdocof{|\textit{main}|}|\\
\end{tabular}
\end{center}
at the top of every child file \textit{child}
which is included by |\include{|\textit{child}|}|
from within the main file
(or at least for those files to be compiled individually).
The argument \textit{main} must be the filename of the main file.

There are a couple of
considerations in setting up the main and child documents:

%%%%%%%%%%%%%%%%%%%%%%%%%%%%%%%%%%%%%%%%
\paragraph{Restrictions.}

Please note the following restrictions:
\begin{itemize}
\item
|\childdocmain| must be called with one argument \textit{main}
to ensure compatibility with earlier version of the package.
It must either be empty (|\childdocmain{}|)
or precisely match the filename of the main file in which it is specified.
See \secref{sec:detection} for further information.
\item
The filename \textit{main} must be specified without the |.tex| extension.
\item
The filename \textit{main} is case sensitive
(even in case-insensitive file systems)
due to internal string comparison.
\item
The argument \textit{main} should be fully expanded, it cannot be a macro.
\item
Subdirectories and special characters should be avoided in filenames.
\item
The command |\childdocmain{|\textit{main}|}| must be followed by a whitespace.
It should not be followed immediately by another command
or by a comment mark `|%|'.
This is because the \TeX{} parser reads the token immediately following
the argument of |\childdocmain| and puts it
at the beginning of every child section;
however, a white\-space is ignored.
\end{itemize}

%%%%%%%%%%%%%%%%%%%%%%%%%%%%%%%%%%%%%%%%
\paragraph{Content of Main File.}

It is advisable to place all content in the child files included by |\include|.
Any output contained in the main file will appear in all child documents
unless suppressed manually;
it cannot be suppressed automatically by the |\includeonly| directive
and thus should normally be avoided.
A method to include some content in the main file
by means of conditional processing is described in \secref{sec:conditional}.

%%%%%%%%%%%%%%%%%%%%%%%%%%%%%%%%%%%%%%%%
\paragraph{Page Numbering.}

When only a part of the document is compiled,
the appropriate numbering of pages
(as well as other status parameters)
is determined from the |.aux| files.
The latter contain information from previous passes.
However this information needs to propagate through
all intermediate child documents.
Therefore the page numbering in child documents may well
be inconsistent until the complete document is compiled at least once.

A useful (if unconventional) way to always ensure a consistent
page numbering is to restart the numbering in each child document
and denote the pages by `\textit{child}|.|\textit{page}'
where \textit{child} represents the chapter/section number of the child file.
This can be achieved by the command
|\numberwithin{page}{|\textit{child}|}|
of the \textsf{amsmath} package
where \textit{child} can be |chapter| or |section|
depending on the chosen structuring.
Alternatively, one can modify the macro |\thepage| appropriately
and reset the counter |page| at the start of each child file.

%%%%%%%%%%%%%%%%%%%%%%%%%%%%%%%%%%%%%%%%%%%%%%%%%%%%%%%%%%%%%%%%%%%%%%%%%%%%%%%%
\subsection{Conditional Processing}
\label{sec:conditional}

The package provides a mechanism to compile different versions
of a document. To customise the versions further some conditional processing
can come in handy to distinguish which version is being compiled.
The package provides two macros to describe the compilation context:

%%%%%%%%%%%%%%%%%%%%%%%%%%%%%%%%%%%%%%%%
\DescribeMacro{\ifchilddoc}
The conditional |\ifchilddoc| distinguishes between the compilation of
child documents and the main document:
%
\begin{center}
|\ifchilddoc |\textit{child-code}| |[|\||else |\textit{main-code}]| \||fi|
\end{center}

%%%%%%%%%%%%%%%%%%%%%%%%%%%%%%%%%%%%%%%%
\DescribeMacro{\childdocname}
\DescribeMacro{\childdocjob}
The macro |\childdocname| contains the filename (without extension)
of the main or child file being processed.
Note that |\childdocjob| will always contain the name of the main file.

%%%%%%%%%%%%%%%%%%%%%%%%%%%%%%%%%%%%%%%%
\paragraph{Title Page.}

Conditional processing can be used to include a title or banner page
in the main document when proper precautions are taken.
Importantly, the code in the main file should ensure that the page counter
(as well as other status parameters which are stored in the |.aux| files)
takes the same value after the conditional processing.
Otherwise the page numbers may take divergent values
depending on which part is compiled.

For example, a title page could be declared by:
%
\begin{center}
\begin{tabular}{l}
|\ifchilddoc\||else|\\
|\addtocounter{page}{-1}|\\
\textit{code for title page}\\
|\newpage|\\
|\||fi|
\end{tabular}
\end{center}
%
A banner page for the child documents can be generated by:
%
\begin{center}
\begin{tabular}{l}
|\ifchilddoc|\\
|\addtocounter{page}{-1}|\\
\textit{code for banner page}\\
|\newpage|\\
|\||fi|
\end{tabular}
\end{center}
%
Here one could write a message such as:
\begin{center}
|This is the part \childdocname{} of \childdocjob{}.|
\end{center}

%%%%%%%%%%%%%%%%%%%%%%%%%%%%%%%%%%%%%%%%%%%%%%%%%%%%%%%%%%%%%%%%%%%%%%%%%%%%%%%%
\subsection{Flags}
\label{sec:flags}

The package makes it easy to generate different versions
of the main or child documents.
To this end compilation flags can be defined
and assigned different default values.
They will be particularly useful in conjunction
with the forwarding mechanism described in \secref{sec:forward}.

For example, it may be useful to have a flag |\version|
which can be set to |draft| or |final|.
The document source will contain some conditional code
depending on the value of |\version|.
Suppose further, the flag should default to |final| for the main file
and to |draft| for child files
which is a natural assignment for editing the document.
This is achieved by placing the following code
in the preamble of the main document
(below the |\childdocmain| directive):
%
\begin{center}
\begin{tabular}{l}
|\ifchilddoc|\\
|\providecommand{\version}{draft}|\\
|\||else|\\
|\providecommand{\version}{final}|\\
|\||fi|
\end{tabular}
\end{center}
%
The definition by |\providecommand| makes sure
that previous definitions are not overwritten.
Further statements |\providecommand{\version}{...}|
can thus be added before the above code to override it.

For the main file, one might add a line
(between |\childdocmain| and the above block)
%
\begin{center}
|%\ifchilddoc\||else\providecommand{\version}{draft}\||fi|
\end{center}
%
which can be uncommented to produce a draft version.
Likewise one can add a line to the very top of a child file
(above the |\childdocof{|\textit{main}|}| directive)
%
\begin{center}
|%\providecommand{\version}{final}|
\end{center}
%
which can be uncommented to produce the final version of this child document.

%%%%%%%%%%%%%%%%%%%%%%%%%%%%%%%%%%%%%%%%%%%%%%%%%%%%%%%%%%%%%%%%%%%%%%%%%%%%%%%%
\subsection{Forwarding}
\label{sec:forward}

Different versions of the main or child documents
using compilation flags as described in \secref{sec:flags}
can be (permanently) stored in different files
for convenient compilation, viewing and distribution.
To this end, the package defines a command
to pass on compilation to a different file:

%%%%%%%%%%%%%%%%%%%%%%%%%%%%%%%%%%%%%%%%
\DescribeMacro{\childdocforward}
The command |\childdocforward| redirects processing to
another source file:
%
\begin{center}
\begin{tabular}{l}
|\input{childdoc.def}|\\
|\childdocforward[|\textit{main}|]{|\textit{dest}|}|\\
\end{tabular}
\end{center}
%
The argument \textit{dest} is the destination file
(without extension).
It should be the main file or one of the child files.
Note that further \textsf{childdoc} directives
such as |\childdocof| and |\childdocforward|
in the indicated file will be processed in this form.
The optional argument \textit{main}
passes on directly to the main file \textit{main}
while pretending to compile the child \textit{dest}.
This form behaves as if \textit{dest}
issues |\childdocof{|\textit{main}|}| right away,
and no further \textsf{childdoc} directives will be processed.

%%%%%%%%%%%%%%%%%%%%%%%%%%%%%%%%%%%%%%%%
\DescribeMacro{\...prefix}
In the alternative form |\childdocforwardprefix|,
%
\begin{center}
\begin{tabular}{l}
|\input{childdoc.def}|\\
|\childdocforwardprefix[|\textit{main}|]{|\textit{prefix}|}{|\textit{dest}|}|
\end{tabular}
\end{center}
%
the destination file is determined by a pattern
depending on the current file:
To make this work, the current file must be called
`{\textit{prefix}\hspace{0.2em}\textit{suffix}}'
with \textit{prefix} matching precisely the argument.
Processing is then passed on to the file
`{\textit{dest}\hspace{0.2em}\textit{suffix}}'.
Surely, the same effect is achieved by
directly specifying the
argument `{\textit{dest}\hspace{0.2em}\textit{suffix}}'
in the first form.
However, that requires to set up a different file
for each child. With the alternative form of the command
all these files can have exactly the same content
which simplifies setting them up and maintaining them.

For example, the following file |draft.tex|
with a compilation flag |\version| as described in \secref{sec:flags}
compiles the main document as a draft:
%
\begin{center}
\begin{tabular}{l}
|\def\version{draft}|\\
|\input{childdoc.def}|\\
|\childdocforward{|\textit{main}|}|
\end{tabular}
\end{center}
%
Likewise, the following files |final|\textit{nn}|.tex|
compile the final version of the child document
|child|\textit{nn}|.tex|:
%
\begin{center}
\begin{tabular}{l}
|\def\version{final}|\\
|\input{childdoc.def}|\\
|\childdocforwardprefix{final}{child}|
\end{tabular}
\end{center}
%

Note that when several versions of a main file and/or of each child file
are to be generated, it may be convenient to set up a |Makefile| or
shell script to automatise the process.

%%%%%%%%%%%%%%%%%%%%%%%%%%%%%%%%%%%%%%%%%%%%%%%%%%%%%%%%%%%%%%%%%%%%%%%%%%%%%%%%
\subsection{Command Line Processing}
\label{sec:commandline}

The effect of redirection files can also be achieved by invoking
the \LaTeX{} compiler with a more elaborate command line.
Most conveniently this should be done as part
of a shell script or a |Makefile|.

When using \textsf{childdoc} in the main file, the following
command lines effectively perform a redirection
(note that depending on the shell being used,
backslashes may have to be doubled: `|\|' $\to$ `|\\|'):
%
\begin{center}
|... -jobname "|\textit{target}|" |\\|"|[\textit{flags}]%
|\input{childdoc.def}\childdocforward[|\textit{main}|]{|\textit{dest}|}"|
\end{center}
%
Here \textit{target} is the name of the output file,
\textit{main} is the name of the main file
and \textit{dest} is the name of the main or child file to be processed
(all filenames without extensions).
The optional argument \textit{main} can be omitted
if \textit{main} matches \textit{dest}.
Optionally, compilation \textit{flags} can be defined via |\def| commands.
This command line makes the \TeX{} engine believe
it is compiling the file \textit{target}
whose content is specified as the latter parameter.
The provided code then forwards the processing to
\textit{main} or \textit{dest} as described in \secref{sec:forward}.

%%%%%%%%%%%%%%%%%%%%%%%%%%%%%%%%%%%%%%%%%%%%%%%%%%%%%%%%%%%%%%%%%%%%%%%%%%%%%%%%
\subsection{Include by Input}
\label{sec:input}

Including child documents by |\include| has some restrictions by design.
Most notably, the content of a child document always occupies
its own set of pages; pages cannot be shared between child documents.
Usually, this behaviour makes perfect sense
because each child document contain an essential part of the document.
However, in some situations it may be desirable to compose
a document from a collection of parts
without having mandatory page breaks between then.
For this case, the package
provides a mechanism to include parts
by |\input| which can also be processed individually.
However, by construction this mechanism
requires manual handling of the content to be output.

%%%%%%%%%%%%%%%%%%%%%%%%%%%%%%%%%%%%%%%%
\DescribeMacro{\ifchilddocmanual}
The main file should be prepared as usual, see \secref{sec:include}.
However, the document body must make a distinction
between processing of an individual part and of the main document, e.g.:
%
\begin{center}
\begin{tabular}{l}
|\ifchilddocmanual|\\
|\input{\childdocname}|\\
|\||else|\\
\textit{document body with }|\input{|\textit{part}|}|\\
|\||fi|
\end{tabular}
\end{center}
%
The conditional |\ifchilddocmanual| is true whenever
a part to be included by |\input| is being compiled,
and the name of the part is stored in |\childdocname|.

%%%%%%%%%%%%%%%%%%%%%%%%%%%%%%%%%%%%%%%%
\DescribeMacro{\childdocby}
Each part to be included by |\input| should start with:
%
\begin{center}
\begin{tabular}{l}
|\input{childdoc.def}|\\
|\childdocby{|\textit{main}|}|\\
\end{tabular}
\end{center}
%
The directive |\childdocby| is similar to |\childdocof|
described in \secref{sec:include},
but the subsequent selection of content must be done manually.
To that end, both |\ifchilddoc| and |\ifchilddocmanual|
will be true upon processing of a part,
and the name of the part is stored in |\childdocname|.
Note that |\jobname| will be set to the filename of the current part
so that each part receives an individual |.aux| file
that does not interfere with the |.aux| file(s) of the main document.
This behaviour can be altered by the alternative form
|\childdocby[*]{|\textit{main}|}| (with a non-empty optional argument)
which uses the |.aux| file of the main document
by setting |\jobname| to \textit{main}.

%%%%%%%%%%%%%%%%%%%%%%%%%%%%%%%%%%%%%%%%%%%%%%%%%%%%%%%%%%%%%%%%%%%%%%%%%%%%%%%%
\subsection{Driver Development}
\label{sec:driver}

The \textsf{childdoc} mechanism can also be use for the development
of definition files such as \LaTeX{} styles or classes.
This case differs from the above setup with multiple parts
included by |\include| in that no |\includeonly| should be invoked.
This can be achieved by starting the include file
(before |\ProvidesPackage|) with:
%
\begin{center}
\begin{tabular}{l}
|\input{childdoc.def}|\\
|\childdocforward{|\textit{main}|}|\\
\end{tabular}
\end{center}
%
or alternatively with:
%
\begin{center}
\begin{tabular}{l}
|\input{childdoc.def}|\\
|\childdocby{|\textit{main}|}|\\
\end{tabular}
\end{center}
%
Both forms have slightly different effects as described above.
The main file is prepared as usual, see \secref{sec:include}.

%%%%%%%%%%%%%%%%%%%%%%%%%%%%%%%%%%%%%%%%%%%%%%%%%%%%%%%%%%%%%%%%%%%%%%%%%%%%%%%%
\subsection{Legacy Detection}
\label{sec:detection}

The directive |\childdocmain| in the main file can detect
whether the complete document or merely a child is to be compiled
even without using the directive |\childdocof|.
This method is deprecated because it is less robust
and there is no compelling reason to use it;
it is merely provided for backward compatibility
and it may be removed in future versions.

If the detection mechanism is to be used,
it is mandatory to correctly specify
the filename of the main file as the argument of |\childdocmain|:
%
\begin{center}
\begin{tabular}{l}
|\input{childdoc.def}|\\
|\childdocmain{|\textit{main}|}|\\
\end{tabular}
\end{center}
%
If |\jobname| does not match the argument \textit{main} of |\childdocmain|,
it is assumed that |\jobname| points to the child file to be compiled.
When using |\childdocmain| with the main file specified as argument,
it suffices to start a child file
with just |\input{|\textit{main}|}|
without loading of the package and using |\childdocof|.
If instead all processing is done
with the appropriate \textsf{childdoc} directives,
the argument of \textit{main} of |\childdocmain| can be empty.

An alternative version of the command line processing described
in \secref{sec:commandline} using the detection mechanism reads:
%
\begin{center}
|... -jobname "|\textit{target}|" "|[\textit{flags}]%
[|\def\jobname{|\textit{dest}|}|]|\input{|\textit{main}|}"|
\end{center}

%%%%%%%%%%%%%%%%%%%%%%%%%%%%%%%%%%%%%%%%%%%%%%%%%%%%%%%%%%%%%%%%%%%%%%%%%%%%%%%%
\subsection{Manual Code}
\label{sec:manual}

In case one cannot be certain whether the definitions file |childdoc.def|
is installed on the target \TeX{} distribution
and one prefers not to ship it,
it is conceivable to paste a few relevant commands into the sources.

To that end, drop all statements |\input{childdoc.def}|
and perform the replacements as outlined below.
Instead of |\childdocmain{|\textit{main}|}| add the following code
to the top of the main file:
%
\begin{center}
\begin{tabular}{l}
|\||ifdefined\childdocname\endinput\||fi\newif\ifchilddoc|\\
|\edef\childdocname{\scantokens\expandafter{\jobname\noexpand}}|\\
|\def\childdocmain{|\textit{main}|}\||ifx\childdocmain\childdocname\||else|\\
|\childdoctrue\includeonly{\childdocname}\let\jobname\childdocmain\||fi|\\
\end{tabular}
\end{center}
%
Instead of |\childdocof{|\textit{main}|}| just include the main file
at the top of each child file:
%
\begin{center}
|\input{|\textit{main}|}|
\end{center}
%
A simple redirection |\childdocforward{|\textit{dest}|}| is achieved by:
%
\begin{center}
|\def\jobname{|\textit{dest}|}\input{\jobname}|
\end{center}
%
The redirection with prefix
|\childdocforwardprefix[|\textit{prefix}|]{|\textit{dest}|}|
is accomplished by:
%
\begin{center}
\begin{tabular}{l}
|{\edef\jobname{\scantokens\expandafter{\jobname\noexpand}}|\\
|\def\redirectjob |\textit{prefix}|#1~~~{\gdef\jobname{|\textit{dest}|#1}}|\\
|\expandafter\redirectjob\jobname~~~}\input{\jobname}|
\end{tabular}
\end{center}

In an alternative approach,
child documents can be compiled by a specific command line
without additional code or specific definitions:
%
\begin{center}
|... -jobname "|\textit{target}|" "|[\textit{flags}]%
|\includeonly{|\textit{dest}|}\input{|\textit{main}|}"|
\end{center}
%

%%%%%%%%%%%%%%%%%%%%%%%%%%%%%%%%%%%%%%%%%%%%%%%%%%%%%%%%%%%%%%%%%%%%%%%%%%%%%%%%
%%%%%%%%%%%%%%%%%%%%%%%%%%%%%%%%%%%%%%%%%%%%%%%%%%%%%%%%%%%%%%%%%%%%%%%%%%%%%%%%
\section{Information}

%%%%%%%%%%%%%%%%%%%%%%%%%%%%%%%%%%%%%%%%%%%%%%%%%%%%%%%%%%%%%%%%%%%%%%%%%%%%%%%%
\subsection{Copyright}

Copyright \copyright{} 2017--2018 Niklas Beisert

This work may be distributed and/or modified under the
conditions of the \LaTeX{} Project Public License, either version 1.3
of this license or (at your option) any later version.
The latest version of this license is in
  \url{http://www.latex-project.org/lppl.txt}
and version 1.3 or later is part of all distributions of \LaTeX{}
version 2005/12/01 or later.

This work has the LPPL maintenance status `maintained'.

The Current Maintainer of this work is Niklas Beisert.

This work consists of the files |README.txt|, |childdoc.ins| and |childdoc.dtx|
as well as the derived files |childdoc.def|, |cdocsamp.tex|
with |cdocsch1.tex|, |cdocsch2.tex|, |cdocspt3.tex|, |cdocspt4.tex|,
|cdocsdrf.tex|, |cdocsfn1.tex|, |cdocsfn2.tex|
as well as |childdoc.pdf|.

%%%%%%%%%%%%%%%%%%%%%%%%%%%%%%%%%%%%%%%%%%%%%%%%%%%%%%%%%%%%%%%%%%%%%%%%%%%%%%%%
\subsection{Files and Installation}

The package consists of the files:
%
\begin{center}
\begin{tabular}{ll}
    |README.txt|   & readme file \\
    |childdoc.ins| & installation file \\
    |childdoc.dtx| & source file \\
    |childdoc.def| & definition file \\
    |cdocsamp.tex| & sample main file \\
    |cdocsch1.tex| & sample include file \\
    |cdocsch2.tex| & sample include file \\
    |cdocspt3.tex| & sample part file \\
    |cdocspt4.tex| & sample part file \\
    |cdocsdrf.tex| & sample redirection file \\
    |cdocsfn1.tex| & sample redirection file \\
    |cdocsfn2.tex| & sample redirection file \\
    |childdoc.pdf| & manual
\end{tabular}
\end{center}
%
The distribution consists of the files
|README.txt|, |childdoc.ins| and |childdoc.dtx|.
%
\begin{itemize}
\item
Run (pdf)\LaTeX{} on |childdoc.dtx|
to compile the manual |childdoc.pdf| (this file).
\item
Run \LaTeX{} on |childdoc.ins| to create the definitions file |childdoc.def|
and the sample |cdocsamp.tex| with include files
|cdocsch1.tex|, |cdocsch2.tex|, |cdocspt3.tex|, |cdocspt4.tex|,
|cdocsdrf.tex|, |cdocsfn1.tex|, |cdocsfn2.tex|.
Then copy the file |childdoc.def| to an appropriate directory of your \LaTeX{}
distribution, e.g.\ \textit{texmf-root}|/tex/latex/childdoc|.
\end{itemize}

%%%%%%%%%%%%%%%%%%%%%%%%%%%%%%%%%%%%%%%%%%%%%%%%%%%%%%%%%%%%%%%%%%%%%%%%%%%%%%%%
\subsection{Related CTAN Packages}

There are several other packages which offer a similar functionality:
%
\begin{itemize}
\item
The packages
\href{http://ctan.org/pkg/docmute}{\textsf{docmute}},
\href{http://ctan.org/pkg/includex}{\textsf{includex}} and
\href{http://ctan.org/pkg/standalone}{\textsf{standalone}}
provide commands to include only the document body of
a child file thus allowing both files to be compiled individually.
\item
The packages \href{http://ctan.org/pkg/subdocs}{\textsf{subdocs}}
and \href{http://ctan.org/pkg/subfiles}{\textsf{subfiles}}
provide structures in which the main and child documents can be
encapsulated and allowing them to be compiled individually.
The inclusion mechanism is different from the conventional |\include|.
\item
The package \href{http://ctan.org/pkg/combine}{\textsf{combine}}
is an elaborate solution to combine several documents into one.
\end{itemize}
%
See also the CTAN topic \href{http://ctan.org/topic/subdocs}{\textsf{subdocs}}
for further related packages.
The present package differs from the above solutions in that
a document structure constructed with the conventional |\include| mechanism
just needs two extra commands at the top of every file
such that all constituent files can be compiled individually.

%%%%%%%%%%%%%%%%%%%%%%%%%%%%%%%%%%%%%%%%%%%%%%%%%%%%%%%%%%%%%%%%%%%%%%%%%%%%%%%%
%\subsection{Feature Suggestions}
%
%The following is a list of features which may be useful for future
%versions of this package:
%%
%\begin{itemize}
%\item
%\ldots
%\end{itemize}

%%%%%%%%%%%%%%%%%%%%%%%%%%%%%%%%%%%%%%%%%%%%%%%%%%%%%%%%%%%%%%%%%%%%%%%%%%%%%%%%
\subsection{Revision History}

%%%%%%%%%%%%%%%%%%%%%%%%%%%%%%%%%%%%%%%%
\paragraph{v2.0:} 2018/12/30

\begin{itemize}
\item
immediate forward processing
\item
added |\childdocby| mechanism
\item
manual restructured
\end{itemize}

%%%%%%%%%%%%%%%%%%%%%%%%%%%%%%%%%%%%%%%%
\paragraph{v1.6:} 2018/01/17

\begin{itemize}
\item
application for development of include files
\item
corrections to manual
\end{itemize}

%%%%%%%%%%%%%%%%%%%%%%%%%%%%%%%%%%%%%%%%
\paragraph{v1.5:} 2017/05/21

\begin{itemize}
\item
more complete structuring introduced
\item
|\childdocof| introduced
\item
|\childdoc| renamed to |\childdocmain|
\item
|\childredirect| renamed to |\childdocforward| and |\childdocforwardprefix|
and functionality expanded
\end{itemize}

%%%%%%%%%%%%%%%%%%%%%%%%%%%%%%%%%%%%%%%%
\paragraph{v1.0:} 2017/04/27

\begin{itemize}
\item
manual and install package
\item
first version published on CTAN
\end{itemize}

%%%%%%%%%%%%%%%%%%%%%%%%%%%%%%%%%%%%%%%%
\paragraph{v0.6:} 2017/04/26

\begin{itemize}
\item
redirection mechanism added
\end{itemize}

%%%%%%%%%%%%%%%%%%%%%%%%%%%%%%%%%%%%%%%%
\paragraph{v0.5:} 2017/04/26

\begin{itemize}
\item
functionality in definition file
\end{itemize}


%%%%%%%%%%%%%%%%%%%%%%%%%%%%%%%%%%%%%%%%%%%%%%%%%%%%%%%%%%%%%%%%%%%%%%%%%%%%%%%%
%%%%%%%%%%%%%%%%%%%%%%%%%%%%%%%%%%%%%%%%%%%%%%%%%%%%%%%%%%%%%%%%%%%%%%%%%%%%%%%%
%%%%%%%%%%%%%%%%%%%%%%%%%%%%%%%%%%%%%%%%%%%%%%%%%%%%%%%%%%%%%%%%%%%%%%%%%%%%%%%%
\appendix

\settowidth\MacroIndent{\rmfamily\scriptsize 000\ }

 \DocInput{childdoc.dtx}

\end{document}
%</driver>
% \fi
%
% %%%%%%%%%%%%%%%%%%%%%%%%%%%%%%%%%%%%%%%%%%%%%%%%%%%%%%%%%%%%%%%%%%%%%%%%%%%%%%
% %%%%%%%%%%%%%%%%%%%%%%%%%%%%%%%%%%%%%%%%%%%%%%%%%%%%%%%%%%%%%%%%%%%%%%%%%%%%%%
% \section{Sample}
%\iffalse
%<*samplemain>
%\fi
%
% The following presents a sample document
% with two chapters, two parts, a title page,
% a compile flag as well as three forwarding files to set the flag.
% It consists of eight |.tex| files:
% \begin{center}
% \begin{tabular}{ll}
% |cdocsamp.tex|&main file\\
% |cdocsch1.tex|&include file for chapter 1\\
% |cdocsch2.tex|&include file for chapter 2\\
% |cdocspt3.tex|&include file for part 3\\
% |cdocspt4.tex|&include file for part 4\\
% |cdocsdrf.tex|&forwarding file for main file in draft mode\\
% |cdocsfi1.tex|&forwarding file for final version of chapter 1\\
% |cdocsfi2.tex|&forwarding file for final version of chapter 2\\
% \end{tabular}
% \end{center}
% Each of the eight files can be compiled directly by the \LaTeX{} compiler.
%
% %%%%%%%%%%%%%%%%%%%%%%%%%%%%%%%%%%%%%%
% \paragraph{Main File.}
%
% The main file is called |cdocsamp.tex|.
%
% Load the \textsf{childdoc} definitions and
% declare the filename for the main document:
%    \begin{macrocode}
\input{childdoc.def}
\childdocmain{}
%    \end{macrocode}

% Optional override for |\version| flag:
%    \begin{macrocode}
%%\ifchilddoc\else\providecommand{\version}{draft}\fi
%    \end{macrocode}

% Define the default values for the |\version| flag
% (|final| for the main file and |draft| for childs):
%    \begin{macrocode}
\ifchilddoc
\providecommand{\version}{draft}
\else
\providecommand{\version}{final}
\fi
%    \end{macrocode}

% Load the standard document class:
%    \begin{macrocode}
\documentclass[12pt]{article}
%    \end{macrocode}

% Start the document body:
%    \begin{macrocode}
\begin{document}
%    \end{macrocode}

% Declare a title page.
% Print title, part of document being processed and version flag:
%    \begin{macrocode}
\addtocounter{page}{-1}
\begin{center}
{\LARGE\bfseries{}childdoc example\par}
\vspace{1cm}
\ifchilddoc
\ifchilddocmanual part\else chapter\fi:
`\childdocname' of `\childdocjob'\par
\else
main document: `\childdocjob'\par
\fi
version: \version\par
\end{center}
\newpage
%    \end{macrocode}

% Manually include selected file,
% otherwise process as usual:
%    \begin{macrocode}
\ifchilddocmanual
\section*{part `\childdocname'}
\input{\childdocname}
\else
%    \end{macrocode}

% Include the two chapters:
%    \begin{macrocode}
\include{cdocsch1}
\include{cdocsch2}
%    \end{macrocode}

% Include the two parts unless only chapters should be displayed:
%    \begin{macrocode}
\ifchilddoc\else
\section{part three}
\input{cdocspt3}
\section{part four}
\input{cdocspt4}
\fi
%    \end{macrocode}

% Process as usual until here:
%    \begin{macrocode}
\fi
%    \end{macrocode}

% End of document body:
%    \begin{macrocode}
\end{document}
%    \end{macrocode}
%\iffalse
%</samplemain>
%\fi
%
% %%%%%%%%%%%%%%%%%%%%%%%%%%%%%%%%%%%%%%
% \paragraph{Chapter Include Files.}
%
% The include files are called |cdocsch1.tex| and |cdocsch2.tex|.
%
%\iffalse
%<*samplechap1|samplechap2>
%\fi

% Optional override for |\version| flag:
%    \begin{macrocode}
%%\providecommand{\version}{final}
%    \end{macrocode}

% Include the main document:
%    \begin{macrocode}
\input{childdoc.def}
\childdocof{cdocsamp}
%    \end{macrocode}

%\iffalse
%</samplechap1|samplechap2>
%\fi
%
%\iffalse
%<*samplechap1>
%\fi
% Some text for chapter 1:
%    \begin{macrocode}
\section{one}
some text in chapter one
%    \end{macrocode}

%\iffalse
%</samplechap1>
%\fi
% Some text for chapter 2:
%\iffalse
%<*samplechap2>
%\fi
%    \begin{macrocode}
\section{two}
more text in chapter two
%    \end{macrocode}

%\iffalse
%</samplechap2>
%\fi
%
% %%%%%%%%%%%%%%%%%%%%%%%%%%%%%%%%%%%%%%
% \paragraph{Part Include Files.}
%
% The include files are called |cdocspt3.tex| and |cdocspt4.tex|.
%
%\iffalse
%<*samplepart3|samplepart4>
%\fi

% Optional override for |\version| flag:
%    \begin{macrocode}
%%\providecommand{\version}{final}
%    \end{macrocode}

% Include the main document:
%    \begin{macrocode}
\input{childdoc.def}
\childdocby{cdocsamp}
%    \end{macrocode}

%\iffalse
%</samplepart3|samplepart4>
%\fi
%
%\iffalse
%<*samplepart3>
%\fi
% Some text for part 3:
%    \begin{macrocode}
some text in part three
%    \end{macrocode}

%\iffalse
%</samplepart3>
%\fi
% Some text for part 4:
%\iffalse
%<*samplepart4>
%\fi
%    \begin{macrocode}
more text in part four
%    \end{macrocode}

%\iffalse
%</samplepart4>
%\fi
%
% %%%%%%%%%%%%%%%%%%%%%%%%%%%%%%%%%%%%%%
% \paragraph{Forwarding for a Complete Draft.}
%
% The following forwarding file |cdocsdrf.tex|
% compiles the main document in draft mode:
%\iffalse
%<*sampledraft>
%\fi
%    \begin{macrocode}
\def\version{draft}
\input{childdoc.def}
\childdocforward{cdocsamp}
%    \end{macrocode}

%\iffalse
%</sampledraft>
%\fi
%
% %%%%%%%%%%%%%%%%%%%%%%%%%%%%%%%%%%%%%%
% \paragraph{Forwarding for Final Version of the Chapters.}
%
% The following forwarding files |cdocsfn1.tex| and |cdocsfn2.tex|
% (with identical content)
% compile the final versions of the child documents
% |cdocsch1.tex| and |cdocsch2.tex|, respectively:
%\iffalse
%<*samplefinal>
%\fi
%    \begin{macrocode}
\def\version{final}
\input{childdoc.def}
\childdocforwardprefix[cdocsamp]{cdocsfn}{cdocsch}
%    \end{macrocode}

%\iffalse
%</samplefinal>
%\fi
%
% %%%%%%%%%%%%%%%%%%%%%%%%%%%%%%%%%%%%%%
% \paragraph{Command Line Processing.}
%
% The following three command lines generate the output files
% |cdocscld|, |cdocscl1| and |cdocscl2|
% which should be identical to
% |cdocsdrf|, |cdocsch1| and |cdocsfn2|, respectively:
% \begin{center}
% \begin{tabular}{l}
% |latex -jobname cdocscld \|\\
% |  "\def\version{draft}\input{childdoc.def}\childdocforward{cdocsamp}"|\\
% |latex -jobname cdocscl1 \|\\
% |  "\input{childdoc.def}\childdocforward[cdocsamp]{cdocsch1}"|\\
% |latex -jobname cdocscl2 \|\\
% |  "\def\version{final}\input{childdoc.def}\childdocforward{cdocsch2}"|
% \end{tabular}
% \end{center}
% Note that the trailing backslash on each first line
% merely continues the input to the second line
% (for convenient cut ant paste).
% Furthermore, the command |latex| can be replaced by any
% of its alternative versions such as |pdflatex|.
%
% %%%%%%%%%%%%%%%%%%%%%%%%%%%%%%%%%%%%%%%%%%%%%%%%%%%%%%%%%%%%%%%%%%%%%%%%%%%%%%
% %%%%%%%%%%%%%%%%%%%%%%%%%%%%%%%%%%%%%%%%%%%%%%%%%%%%%%%%%%%%%%%%%%%%%%%%%%%%%%
% \section{Implementation}
%\iffalse
%<*package>
%\fi
%
% This section describes the definitions file |childdoc.def|.

% The definitions cannot be loaded using |\usepackage| or |\RequirePackage|
% which has a mechanism to prevent loading a style file more than once.
% When loading the definitions by means of |\input|
% multiple instances have to be prevented manually:
%\iffalse
%This code needs to be before the `\ProvidesFile' directive
%which is defined at the beginning of this file.
%Therefore it is also placed there and commented out here.
%</package>
%<*discard>
%\fi
%    \begin{macrocode}
\ifdefined\childdocmain\endinput\fi
%    \end{macrocode}
%\iffalse
%</discard>
%<*package>
%\fi
%
% \macro{\ifchilddoc}
% \macro{\ifchilddocmanual}
% The conditional |\ifchilddoc| tells whether a
% child (true) or main (false) document is being compiled.
% The conditional |\ifchilddocmanual| tells whether
% the |\includeonly| mechanism is used (false) or
% the selection of child files must be performed manually (true).
% The definitions initialise to false:
%    \begin{macrocode}
\newif\ifchilddoc
\newif\ifchilddocmanual
%    \end{macrocode}

% \macro{\childdocname}
% \macro{\childdocjob}
% The macro |\childdocname| stores the name of the main document
% to be compiled. The macro |\childdocjob| stores the name of
% the document on which the \LaTeX{} compiler was originally invoked.
% The content of |\jobname| cannot be compared
% to filenames specified in the source due to different catcodes.
% The following code rescans |\jobname|, stores the result
% in |\childdocname| and saves a copy in |\childdocjob|:
%    \begin{macrocode}
\edef\childdocname{\scantokens\expandafter{\jobname\noexpand}}
\let\childdocjob\childdocname
%    \end{macrocode}

% \macro{\childdocdisable}
% The macro |\childdocdisable| prevents the main file
% from being processed more than once.
% At this stage, the main document command |\childdocmain|
% is assumed to be called once again where it should do nothing.
% Any subsequent call to it should prevent
% a secondary processing of the main document
% It overwrites the forwarding commands
% |\childdocof| and |\childdocforward|
% with empty macros to prevent further inclusions of the main document:
%    \begin{macrocode}
\newcommand{\childdocdisable}
{
  \renewcommand{\childdocmain}[1]{\renewcommand{\childdocmain}[1]{\endinput}}
  \renewcommand{\childdocof}[1]{}
  \renewcommand{\childdocby}[2][]{}
  \renewcommand{\childdocforward}[2][]{}
  \renewcommand{\childdocdisable}{}
}
%    \end{macrocode}

% \macro{\childdocmain}
% The macro |\childdocmain| is to be called at the top of the main file
% with nothing or the main filename (without extension) as argument.
% First, it breaks loops.
% If the argument is not empty and does not match |\childdocname|
% (which is set by the first inclusion of |childdoc.def|),
% |\ifchilddoc| is set to true, |\includeonly| is applied to the child file
% and |\jobname| is set to the main file
% (for proper handling of |.aux| files):
%    \begin{macrocode}
\newcommand{\childdocmain}[1]
{
  \childdocdisable\childdocmain{}
  \if?#1?\else
    \begingroup
      \def\childdoctmp{#1}
      \ifx\childdoctmp\childdocname
        \def\childdoctmp{}
      \else
        \def\childdoctmp
        {
          \childdoctrue
          \includeonly{\childdocname}
          \def\childdocjob{#1}
          \def\jobname{#1}
        }
      \fi
      \expandafter
    \endgroup
    \childdoctmp
  \fi
}
%    \end{macrocode}

% \macro{\childdocof}
% The command |\childdocof| redirects
% compilation to the main file |#1|.
%    \begin{macrocode}
\newcommand{\childdocof}[1]
{
  \childdocdisable
  \childdoctrue
  \includeonly{\childdocname}
  \def\jobname{#1}
  \def\childdocjob{#1}
  \input{#1}
}
%    \end{macrocode}

% \macro{\childdocby}
% The command |\childdocby| ....
%    \begin{macrocode}
\newcommand{\childdocby}[2][]
{
  \childdocdisable
  \childdoctrue
  \childdocmanualtrue
  \if?#1?\else
    \def\jobname{#2}
  \fi
  \def\childdocjob{#2}
  \input{#2}
  \endinput
}
%    \end{macrocode}

% \macro{\childdocforward}
% The command |\childdocforward| redirects
% compilation to the main file or
% (if the optional argument is given) a child file.
% Parameters are set as if the main file
% or a child file starting with |\childdocof| was compiled.
% Then compilation is handed over to the main file:
%    \begin{macrocode}
\newcommand{\childdocforward}[2][]
{
  \begingroup
    \if?#1?
      \def\childdoctmp
      {
        \def\childdocname{#2}
        \def\childdocjob{#2}
        \def\jobname{#2}
        \input{#2}
        \endinput
      }
    \else
      \def\childdoctmp
      {
        \childdocdisable
        \def\childdocname{#2}
        \childdoctrue
        \includeonly{#2}
        \def\childdocjob{#1}
        \def\jobname{#1}
        \input{#1}
        \endinput
      }
    \fi
    \expandafter
  \endgroup
  \childdoctmp
}
%    \end{macrocode}

% \macro{\childdocforwardprefix}
% The command |\childdocforwardprefix| redirects
% compilation to the main or a child file by means of a pattern.
% The prefix |#1| in the current filename is replaced by |#2|
% and the suffix of the current filename is kept
% (it is assumed that the filename does not contain the substring `|~~~|'
% which is used as a delimiter).
% Compilation is handed over to the new file by |\childdocforward|:
%    \begin{macrocode}
\newcommand{\childdocforwardprefix}[3][]
{
  \begingroup
    \def\childdocextract #2##1~~~{\def\childdoctmp{\childdocforward[#1]{#3##1}}}
    \expandafter\childdocextract\childdocname~~~
    \expandafter
  \endgroup
  \childdoctmp
}
%    \end{macrocode}

% \macro{\childdoc}
% The deprecated macro |\childdoc| is a legacy version of |\childdocmain|:
%    \begin{macrocode}
\newcommand{\childdoc}{\childdocmain}
%    \end{macrocode}

% \macro{\childdocredirect}
% The deprecated macro |\childdocredirect| is a legacy version
% of |\childdocforward| and |\childdocforwardprefix|:
%    \begin{macrocode}
\newcommand{\childdocredirect}[2][]
{
  \begingroup
    \if?#1?
      \def\childdoctmp{\childdocforward{#2}}
    \else
      \def\childdoctmp{\childdocforwardprefix{#1}{#2}}
    \fi
    \expandafter
  \endgroup
  \childdoctmp
}
%    \end{macrocode}

%\iffalse
%</package>
%\fi
%
\endinput
|\\
|\childdocforwardprefix{final}{child}|
\end{tabular}
\end{center}
%

Note that when several versions of a main file and/or of each child file
are to be generated, it may be convenient to set up a |Makefile| or
shell script to automatise the process.

%%%%%%%%%%%%%%%%%%%%%%%%%%%%%%%%%%%%%%%%%%%%%%%%%%%%%%%%%%%%%%%%%%%%%%%%%%%%%%%%
\subsection{Command Line Processing}
\label{sec:commandline}

The effect of redirection files can also be achieved by invoking
the \LaTeX{} compiler with a more elaborate command line.
Most conveniently this should be done as part
of a shell script or a |Makefile|.

When using \textsf{childdoc} in the main file, the following
command lines effectively perform a redirection
(note that depending on the shell being used,
backslashes may have to be doubled: `|\|' $\to$ `|\\|'):
%
\begin{center}
|... -jobname "|\textit{target}|" |\\|"|[\textit{flags}]%
|% \iffalse
%
% childdoc.dtx Copyright (C) 2017-2018 Niklas Beisert
%
% This work may be distributed and/or modified under the
% conditions of the LaTeX Project Public License, either version 1.3
% of this license or (at your option) any later version.
% The latest version of this license is in
%   http://www.latex-project.org/lppl.txt
% and version 1.3 or later is part of all distributions of LaTeX
% version 2005/12/01 or later.
%
% This work has the LPPL maintenance status `maintained'.
%
% The Current Maintainer of this work is Niklas Beisert.
%
% This work consists of the files childdoc.dtx and childdoc.ins
% and the derived files childdoc.def and cdocsamp.tex with
% cdocsch1.tex, cdocsch2.tex, cdocsdrf.tex, cdocsfn1.tex, cdocsfn2.tex.
%
%<package>\ifdefined\childdocmain\endinput\fi
%<package>\ProvidesFile{childdoc.def}[2018/12/30 v2.0 child document driver]
%<samplemain>\ProvidesFile{cdocsamp.tex}[2018/12/30 v2.0 sample for childdoc]
%<*driver>
%\ProvidesFile{childdoc.drv}[2018/12/30 v2.0 childdoc reference manual file]
\PassOptionsToClass{10pt,a4paper}{article}
\documentclass{ltxdoc}

\usepackage[margin=35mm]{geometry}
\usepackage{hyperref}
\usepackage{hyperxmp}
\usepackage[usenames]{color}

\hypersetup{colorlinks=true}
\hypersetup{pdfstartview=FitH}
\hypersetup{pdfpagemode=UseNone}
\hypersetup{pdfsource={}}
\hypersetup{pdflang={en-UK}}
\hypersetup{pdfcopyright={Copyright 2017-2018 Niklas Beisert.
  This work may be distributed and/or modified under the
  conditions of the LaTeX Project Public License, either version 1.3
  of this license or (at your option) any later version.}}
\hypersetup{pdflicenseurl={http://www.latex-project.org/lppl.txt}}
\hypersetup{pdfcontactaddress={ETH Zurich, ITP, HIT K,
  Wolfgang-Pauli-Strasse 27}}
\hypersetup{pdfcontactpostcode={8093}}
\hypersetup{pdfcontactcity={Zurich}}
\hypersetup{pdfcontactcountry={Switzerland}}
\hypersetup{pdfcontactemail={nbeisert@itp.phys.ethz.ch}}
\hypersetup{pdfcontacturl={http://people.phys.ethz.ch/\xmptilde nbeisert/}}

\newcommand{\secref}[1]{\hyperref[#1]{section \ref*{#1}}}

\parskip1ex
\parindent0pt
\let\olditemize\itemize
\def\itemize{\olditemize\parskip0pt}

\begin{document}

\title{The \textsf{childdoc} Package}
\hypersetup{pdftitle={The childdoc Package}}
\author{Niklas Beisert\\[2ex]
  Institut f\"ur Theoretische Physik\\
  Eidgen\"ossische Technische Hochschule Z\"urich\\
  Wolfgang-Pauli-Strasse 27, 8093 Z\"urich, Switzerland\\[1ex]
  \href{mailto:nbeisert@itp.phys.ethz.ch}
  {\texttt{nbeisert@itp.phys.ethz.ch}}}
\hypersetup{pdfauthor={Niklas Beisert}}
\hypersetup{pdfsubject={Manual for the LaTeX2e Package childdoc}}
\date{30 December 2018, \textsf{v2.0}}
\maketitle

\begin{abstract}\noindent
\textsf{childdoc} is a \LaTeXe{} package
that enables the direct compilation
of document sections included by |\include|
to individual files.
\end{abstract}

\begingroup
\parskip0ex
\tableofcontents
\endgroup

%%%%%%%%%%%%%%%%%%%%%%%%%%%%%%%%%%%%%%%%%%%%%%%%%%%%%%%%%%%%%%%%%%%%%%%%%%%%%%%%
%%%%%%%%%%%%%%%%%%%%%%%%%%%%%%%%%%%%%%%%%%%%%%%%%%%%%%%%%%%%%%%%%%%%%%%%%%%%%%%%
\section{Introduction}

\LaTeX{} provides a mechanism to structure a large document (such as a book)
into a main file and several child files (containing the chapters)
using the |\include| command.
This mechanism is beneficial for documents
which span hundreds of pages in order to
make the source file(s) more manageable.
Moreover, compilation can be restricted to
selected child files by means of the |\includeonly| command.
The latter feature can be used to reduce the compilation time while editing
(this was significantly more useful in the earlier days of \LaTeX{})
or to generate a smaller document which is easier to navigate.
Another application of |\includeonly| is to generate
documents consisting of selected parts of the complete document.

However, there are a few drawbacks of the plain |\include| mechanism:
\begin{itemize}
\item
The child files cannot be compiled on their own,
they can only be compiled via the main file.
A naive editing environment
(such as a text editor with an option
to have the current file processed by \LaTeX)
may require one to switch to the main file before compiling;
attempting to compile the child file produces errors.
\item
The main file must be modified (each time)
to adjust the |\includeonly| command
to the present needs. This easily leaves the main file in a messy state.
\item
The generated document will always carry the filename
of the main document. This is inconvenient if
several child files are to be compiled and
to be kept for distribution.
\end{itemize}

The present package provides a simple interface
to make child files individually compilable by \LaTeX{}.
Compiling a child file then has the same effect as compiling
the main file with an |\includeonly| command
to select the appropriate child.
Moreover the generated document will carry the name of the child
rather than the main file.
This resolves all three above issues.

This feature is meant to make the editing of books,
thesis documents and lecture notes somewhat more convenient.
However, the package can also be used efficiently for
composing a series of documents (such as exercise sheets)
which are typically distributed individually.
It then assists the author in generating the individual documents
(potentially in different versions)
as well as a document containing the collected series.
Another application is in developing style files
or other kinds of included material
where compilation of the style file could redirect
to a sample or test file.

%%%%%%%%%%%%%%%%%%%%%%%%%%%%%%%%%%%%%%%%%%%%%%%%%%%%%%%%%%%%%%%%%%%%%%%%%%%%%%%%
%%%%%%%%%%%%%%%%%%%%%%%%%%%%%%%%%%%%%%%%%%%%%%%%%%%%%%%%%%%%%%%%%%%%%%%%%%%%%%%%
\section{Usage}

First of all, the package \textsf{childdoc} is \emph{not} a standard
\LaTeXe{} |.sty| style file! Therefore it needs to be invoked in
a non-standard way.

%%%%%%%%%%%%%%%%%%%%%%%%%%%%%%%%%%%%%%%%%%%%%%%%%%%%%%%%%%%%%%%%%%%%%%%%%%%%%%%%
\subsection{Included Files}
\label{sec:include}

%%%%%%%%%%%%%%%%%%%%%%%%%%%%%%%%%%%%%%%%
\DescribeMacro{\childdocmain}
To use the package, add the commands
\begin{center}
\begin{tabular}{l}
|\input{childdoc.def}|\\
|\childdocmain{}|\\
\end{tabular}
\end{center}
at the very top of the main \LaTeX{} file,
in particular \emph{before} the |\documentclass| statement!
The argument of |\childdocmain| should be left empty
(but it must be present).

%%%%%%%%%%%%%%%%%%%%%%%%%%%%%%%%%%%%%%%%
\DescribeMacro{\childdocof}
Furthermore, add the commands
\begin{center}
\begin{tabular}{l}
|\input{childdoc.def}|\\
|\childdocof{|\textit{main}|}|\\
\end{tabular}
\end{center}
at the top of every child file \textit{child}
which is included by |\include{|\textit{child}|}|
from within the main file
(or at least for those files to be compiled individually).
The argument \textit{main} must be the filename of the main file.

There are a couple of
considerations in setting up the main and child documents:

%%%%%%%%%%%%%%%%%%%%%%%%%%%%%%%%%%%%%%%%
\paragraph{Restrictions.}

Please note the following restrictions:
\begin{itemize}
\item
|\childdocmain| must be called with one argument \textit{main}
to ensure compatibility with earlier version of the package.
It must either be empty (|\childdocmain{}|)
or precisely match the filename of the main file in which it is specified.
See \secref{sec:detection} for further information.
\item
The filename \textit{main} must be specified without the |.tex| extension.
\item
The filename \textit{main} is case sensitive
(even in case-insensitive file systems)
due to internal string comparison.
\item
The argument \textit{main} should be fully expanded, it cannot be a macro.
\item
Subdirectories and special characters should be avoided in filenames.
\item
The command |\childdocmain{|\textit{main}|}| must be followed by a whitespace.
It should not be followed immediately by another command
or by a comment mark `|%|'.
This is because the \TeX{} parser reads the token immediately following
the argument of |\childdocmain| and puts it
at the beginning of every child section;
however, a white\-space is ignored.
\end{itemize}

%%%%%%%%%%%%%%%%%%%%%%%%%%%%%%%%%%%%%%%%
\paragraph{Content of Main File.}

It is advisable to place all content in the child files included by |\include|.
Any output contained in the main file will appear in all child documents
unless suppressed manually;
it cannot be suppressed automatically by the |\includeonly| directive
and thus should normally be avoided.
A method to include some content in the main file
by means of conditional processing is described in \secref{sec:conditional}.

%%%%%%%%%%%%%%%%%%%%%%%%%%%%%%%%%%%%%%%%
\paragraph{Page Numbering.}

When only a part of the document is compiled,
the appropriate numbering of pages
(as well as other status parameters)
is determined from the |.aux| files.
The latter contain information from previous passes.
However this information needs to propagate through
all intermediate child documents.
Therefore the page numbering in child documents may well
be inconsistent until the complete document is compiled at least once.

A useful (if unconventional) way to always ensure a consistent
page numbering is to restart the numbering in each child document
and denote the pages by `\textit{child}|.|\textit{page}'
where \textit{child} represents the chapter/section number of the child file.
This can be achieved by the command
|\numberwithin{page}{|\textit{child}|}|
of the \textsf{amsmath} package
where \textit{child} can be |chapter| or |section|
depending on the chosen structuring.
Alternatively, one can modify the macro |\thepage| appropriately
and reset the counter |page| at the start of each child file.

%%%%%%%%%%%%%%%%%%%%%%%%%%%%%%%%%%%%%%%%%%%%%%%%%%%%%%%%%%%%%%%%%%%%%%%%%%%%%%%%
\subsection{Conditional Processing}
\label{sec:conditional}

The package provides a mechanism to compile different versions
of a document. To customise the versions further some conditional processing
can come in handy to distinguish which version is being compiled.
The package provides two macros to describe the compilation context:

%%%%%%%%%%%%%%%%%%%%%%%%%%%%%%%%%%%%%%%%
\DescribeMacro{\ifchilddoc}
The conditional |\ifchilddoc| distinguishes between the compilation of
child documents and the main document:
%
\begin{center}
|\ifchilddoc |\textit{child-code}| |[|\||else |\textit{main-code}]| \||fi|
\end{center}

%%%%%%%%%%%%%%%%%%%%%%%%%%%%%%%%%%%%%%%%
\DescribeMacro{\childdocname}
\DescribeMacro{\childdocjob}
The macro |\childdocname| contains the filename (without extension)
of the main or child file being processed.
Note that |\childdocjob| will always contain the name of the main file.

%%%%%%%%%%%%%%%%%%%%%%%%%%%%%%%%%%%%%%%%
\paragraph{Title Page.}

Conditional processing can be used to include a title or banner page
in the main document when proper precautions are taken.
Importantly, the code in the main file should ensure that the page counter
(as well as other status parameters which are stored in the |.aux| files)
takes the same value after the conditional processing.
Otherwise the page numbers may take divergent values
depending on which part is compiled.

For example, a title page could be declared by:
%
\begin{center}
\begin{tabular}{l}
|\ifchilddoc\||else|\\
|\addtocounter{page}{-1}|\\
\textit{code for title page}\\
|\newpage|\\
|\||fi|
\end{tabular}
\end{center}
%
A banner page for the child documents can be generated by:
%
\begin{center}
\begin{tabular}{l}
|\ifchilddoc|\\
|\addtocounter{page}{-1}|\\
\textit{code for banner page}\\
|\newpage|\\
|\||fi|
\end{tabular}
\end{center}
%
Here one could write a message such as:
\begin{center}
|This is the part \childdocname{} of \childdocjob{}.|
\end{center}

%%%%%%%%%%%%%%%%%%%%%%%%%%%%%%%%%%%%%%%%%%%%%%%%%%%%%%%%%%%%%%%%%%%%%%%%%%%%%%%%
\subsection{Flags}
\label{sec:flags}

The package makes it easy to generate different versions
of the main or child documents.
To this end compilation flags can be defined
and assigned different default values.
They will be particularly useful in conjunction
with the forwarding mechanism described in \secref{sec:forward}.

For example, it may be useful to have a flag |\version|
which can be set to |draft| or |final|.
The document source will contain some conditional code
depending on the value of |\version|.
Suppose further, the flag should default to |final| for the main file
and to |draft| for child files
which is a natural assignment for editing the document.
This is achieved by placing the following code
in the preamble of the main document
(below the |\childdocmain| directive):
%
\begin{center}
\begin{tabular}{l}
|\ifchilddoc|\\
|\providecommand{\version}{draft}|\\
|\||else|\\
|\providecommand{\version}{final}|\\
|\||fi|
\end{tabular}
\end{center}
%
The definition by |\providecommand| makes sure
that previous definitions are not overwritten.
Further statements |\providecommand{\version}{...}|
can thus be added before the above code to override it.

For the main file, one might add a line
(between |\childdocmain| and the above block)
%
\begin{center}
|%\ifchilddoc\||else\providecommand{\version}{draft}\||fi|
\end{center}
%
which can be uncommented to produce a draft version.
Likewise one can add a line to the very top of a child file
(above the |\childdocof{|\textit{main}|}| directive)
%
\begin{center}
|%\providecommand{\version}{final}|
\end{center}
%
which can be uncommented to produce the final version of this child document.

%%%%%%%%%%%%%%%%%%%%%%%%%%%%%%%%%%%%%%%%%%%%%%%%%%%%%%%%%%%%%%%%%%%%%%%%%%%%%%%%
\subsection{Forwarding}
\label{sec:forward}

Different versions of the main or child documents
using compilation flags as described in \secref{sec:flags}
can be (permanently) stored in different files
for convenient compilation, viewing and distribution.
To this end, the package defines a command
to pass on compilation to a different file:

%%%%%%%%%%%%%%%%%%%%%%%%%%%%%%%%%%%%%%%%
\DescribeMacro{\childdocforward}
The command |\childdocforward| redirects processing to
another source file:
%
\begin{center}
\begin{tabular}{l}
|\input{childdoc.def}|\\
|\childdocforward[|\textit{main}|]{|\textit{dest}|}|\\
\end{tabular}
\end{center}
%
The argument \textit{dest} is the destination file
(without extension).
It should be the main file or one of the child files.
Note that further \textsf{childdoc} directives
such as |\childdocof| and |\childdocforward|
in the indicated file will be processed in this form.
The optional argument \textit{main}
passes on directly to the main file \textit{main}
while pretending to compile the child \textit{dest}.
This form behaves as if \textit{dest}
issues |\childdocof{|\textit{main}|}| right away,
and no further \textsf{childdoc} directives will be processed.

%%%%%%%%%%%%%%%%%%%%%%%%%%%%%%%%%%%%%%%%
\DescribeMacro{\...prefix}
In the alternative form |\childdocforwardprefix|,
%
\begin{center}
\begin{tabular}{l}
|\input{childdoc.def}|\\
|\childdocforwardprefix[|\textit{main}|]{|\textit{prefix}|}{|\textit{dest}|}|
\end{tabular}
\end{center}
%
the destination file is determined by a pattern
depending on the current file:
To make this work, the current file must be called
`{\textit{prefix}\hspace{0.2em}\textit{suffix}}'
with \textit{prefix} matching precisely the argument.
Processing is then passed on to the file
`{\textit{dest}\hspace{0.2em}\textit{suffix}}'.
Surely, the same effect is achieved by
directly specifying the
argument `{\textit{dest}\hspace{0.2em}\textit{suffix}}'
in the first form.
However, that requires to set up a different file
for each child. With the alternative form of the command
all these files can have exactly the same content
which simplifies setting them up and maintaining them.

For example, the following file |draft.tex|
with a compilation flag |\version| as described in \secref{sec:flags}
compiles the main document as a draft:
%
\begin{center}
\begin{tabular}{l}
|\def\version{draft}|\\
|\input{childdoc.def}|\\
|\childdocforward{|\textit{main}|}|
\end{tabular}
\end{center}
%
Likewise, the following files |final|\textit{nn}|.tex|
compile the final version of the child document
|child|\textit{nn}|.tex|:
%
\begin{center}
\begin{tabular}{l}
|\def\version{final}|\\
|\input{childdoc.def}|\\
|\childdocforwardprefix{final}{child}|
\end{tabular}
\end{center}
%

Note that when several versions of a main file and/or of each child file
are to be generated, it may be convenient to set up a |Makefile| or
shell script to automatise the process.

%%%%%%%%%%%%%%%%%%%%%%%%%%%%%%%%%%%%%%%%%%%%%%%%%%%%%%%%%%%%%%%%%%%%%%%%%%%%%%%%
\subsection{Command Line Processing}
\label{sec:commandline}

The effect of redirection files can also be achieved by invoking
the \LaTeX{} compiler with a more elaborate command line.
Most conveniently this should be done as part
of a shell script or a |Makefile|.

When using \textsf{childdoc} in the main file, the following
command lines effectively perform a redirection
(note that depending on the shell being used,
backslashes may have to be doubled: `|\|' $\to$ `|\\|'):
%
\begin{center}
|... -jobname "|\textit{target}|" |\\|"|[\textit{flags}]%
|\input{childdoc.def}\childdocforward[|\textit{main}|]{|\textit{dest}|}"|
\end{center}
%
Here \textit{target} is the name of the output file,
\textit{main} is the name of the main file
and \textit{dest} is the name of the main or child file to be processed
(all filenames without extensions).
The optional argument \textit{main} can be omitted
if \textit{main} matches \textit{dest}.
Optionally, compilation \textit{flags} can be defined via |\def| commands.
This command line makes the \TeX{} engine believe
it is compiling the file \textit{target}
whose content is specified as the latter parameter.
The provided code then forwards the processing to
\textit{main} or \textit{dest} as described in \secref{sec:forward}.

%%%%%%%%%%%%%%%%%%%%%%%%%%%%%%%%%%%%%%%%%%%%%%%%%%%%%%%%%%%%%%%%%%%%%%%%%%%%%%%%
\subsection{Include by Input}
\label{sec:input}

Including child documents by |\include| has some restrictions by design.
Most notably, the content of a child document always occupies
its own set of pages; pages cannot be shared between child documents.
Usually, this behaviour makes perfect sense
because each child document contain an essential part of the document.
However, in some situations it may be desirable to compose
a document from a collection of parts
without having mandatory page breaks between then.
For this case, the package
provides a mechanism to include parts
by |\input| which can also be processed individually.
However, by construction this mechanism
requires manual handling of the content to be output.

%%%%%%%%%%%%%%%%%%%%%%%%%%%%%%%%%%%%%%%%
\DescribeMacro{\ifchilddocmanual}
The main file should be prepared as usual, see \secref{sec:include}.
However, the document body must make a distinction
between processing of an individual part and of the main document, e.g.:
%
\begin{center}
\begin{tabular}{l}
|\ifchilddocmanual|\\
|\input{\childdocname}|\\
|\||else|\\
\textit{document body with }|\input{|\textit{part}|}|\\
|\||fi|
\end{tabular}
\end{center}
%
The conditional |\ifchilddocmanual| is true whenever
a part to be included by |\input| is being compiled,
and the name of the part is stored in |\childdocname|.

%%%%%%%%%%%%%%%%%%%%%%%%%%%%%%%%%%%%%%%%
\DescribeMacro{\childdocby}
Each part to be included by |\input| should start with:
%
\begin{center}
\begin{tabular}{l}
|\input{childdoc.def}|\\
|\childdocby{|\textit{main}|}|\\
\end{tabular}
\end{center}
%
The directive |\childdocby| is similar to |\childdocof|
described in \secref{sec:include},
but the subsequent selection of content must be done manually.
To that end, both |\ifchilddoc| and |\ifchilddocmanual|
will be true upon processing of a part,
and the name of the part is stored in |\childdocname|.
Note that |\jobname| will be set to the filename of the current part
so that each part receives an individual |.aux| file
that does not interfere with the |.aux| file(s) of the main document.
This behaviour can be altered by the alternative form
|\childdocby[*]{|\textit{main}|}| (with a non-empty optional argument)
which uses the |.aux| file of the main document
by setting |\jobname| to \textit{main}.

%%%%%%%%%%%%%%%%%%%%%%%%%%%%%%%%%%%%%%%%%%%%%%%%%%%%%%%%%%%%%%%%%%%%%%%%%%%%%%%%
\subsection{Driver Development}
\label{sec:driver}

The \textsf{childdoc} mechanism can also be use for the development
of definition files such as \LaTeX{} styles or classes.
This case differs from the above setup with multiple parts
included by |\include| in that no |\includeonly| should be invoked.
This can be achieved by starting the include file
(before |\ProvidesPackage|) with:
%
\begin{center}
\begin{tabular}{l}
|\input{childdoc.def}|\\
|\childdocforward{|\textit{main}|}|\\
\end{tabular}
\end{center}
%
or alternatively with:
%
\begin{center}
\begin{tabular}{l}
|\input{childdoc.def}|\\
|\childdocby{|\textit{main}|}|\\
\end{tabular}
\end{center}
%
Both forms have slightly different effects as described above.
The main file is prepared as usual, see \secref{sec:include}.

%%%%%%%%%%%%%%%%%%%%%%%%%%%%%%%%%%%%%%%%%%%%%%%%%%%%%%%%%%%%%%%%%%%%%%%%%%%%%%%%
\subsection{Legacy Detection}
\label{sec:detection}

The directive |\childdocmain| in the main file can detect
whether the complete document or merely a child is to be compiled
even without using the directive |\childdocof|.
This method is deprecated because it is less robust
and there is no compelling reason to use it;
it is merely provided for backward compatibility
and it may be removed in future versions.

If the detection mechanism is to be used,
it is mandatory to correctly specify
the filename of the main file as the argument of |\childdocmain|:
%
\begin{center}
\begin{tabular}{l}
|\input{childdoc.def}|\\
|\childdocmain{|\textit{main}|}|\\
\end{tabular}
\end{center}
%
If |\jobname| does not match the argument \textit{main} of |\childdocmain|,
it is assumed that |\jobname| points to the child file to be compiled.
When using |\childdocmain| with the main file specified as argument,
it suffices to start a child file
with just |\input{|\textit{main}|}|
without loading of the package and using |\childdocof|.
If instead all processing is done
with the appropriate \textsf{childdoc} directives,
the argument of \textit{main} of |\childdocmain| can be empty.

An alternative version of the command line processing described
in \secref{sec:commandline} using the detection mechanism reads:
%
\begin{center}
|... -jobname "|\textit{target}|" "|[\textit{flags}]%
[|\def\jobname{|\textit{dest}|}|]|\input{|\textit{main}|}"|
\end{center}

%%%%%%%%%%%%%%%%%%%%%%%%%%%%%%%%%%%%%%%%%%%%%%%%%%%%%%%%%%%%%%%%%%%%%%%%%%%%%%%%
\subsection{Manual Code}
\label{sec:manual}

In case one cannot be certain whether the definitions file |childdoc.def|
is installed on the target \TeX{} distribution
and one prefers not to ship it,
it is conceivable to paste a few relevant commands into the sources.

To that end, drop all statements |\input{childdoc.def}|
and perform the replacements as outlined below.
Instead of |\childdocmain{|\textit{main}|}| add the following code
to the top of the main file:
%
\begin{center}
\begin{tabular}{l}
|\||ifdefined\childdocname\endinput\||fi\newif\ifchilddoc|\\
|\edef\childdocname{\scantokens\expandafter{\jobname\noexpand}}|\\
|\def\childdocmain{|\textit{main}|}\||ifx\childdocmain\childdocname\||else|\\
|\childdoctrue\includeonly{\childdocname}\let\jobname\childdocmain\||fi|\\
\end{tabular}
\end{center}
%
Instead of |\childdocof{|\textit{main}|}| just include the main file
at the top of each child file:
%
\begin{center}
|\input{|\textit{main}|}|
\end{center}
%
A simple redirection |\childdocforward{|\textit{dest}|}| is achieved by:
%
\begin{center}
|\def\jobname{|\textit{dest}|}\input{\jobname}|
\end{center}
%
The redirection with prefix
|\childdocforwardprefix[|\textit{prefix}|]{|\textit{dest}|}|
is accomplished by:
%
\begin{center}
\begin{tabular}{l}
|{\edef\jobname{\scantokens\expandafter{\jobname\noexpand}}|\\
|\def\redirectjob |\textit{prefix}|#1~~~{\gdef\jobname{|\textit{dest}|#1}}|\\
|\expandafter\redirectjob\jobname~~~}\input{\jobname}|
\end{tabular}
\end{center}

In an alternative approach,
child documents can be compiled by a specific command line
without additional code or specific definitions:
%
\begin{center}
|... -jobname "|\textit{target}|" "|[\textit{flags}]%
|\includeonly{|\textit{dest}|}\input{|\textit{main}|}"|
\end{center}
%

%%%%%%%%%%%%%%%%%%%%%%%%%%%%%%%%%%%%%%%%%%%%%%%%%%%%%%%%%%%%%%%%%%%%%%%%%%%%%%%%
%%%%%%%%%%%%%%%%%%%%%%%%%%%%%%%%%%%%%%%%%%%%%%%%%%%%%%%%%%%%%%%%%%%%%%%%%%%%%%%%
\section{Information}

%%%%%%%%%%%%%%%%%%%%%%%%%%%%%%%%%%%%%%%%%%%%%%%%%%%%%%%%%%%%%%%%%%%%%%%%%%%%%%%%
\subsection{Copyright}

Copyright \copyright{} 2017--2018 Niklas Beisert

This work may be distributed and/or modified under the
conditions of the \LaTeX{} Project Public License, either version 1.3
of this license or (at your option) any later version.
The latest version of this license is in
  \url{http://www.latex-project.org/lppl.txt}
and version 1.3 or later is part of all distributions of \LaTeX{}
version 2005/12/01 or later.

This work has the LPPL maintenance status `maintained'.

The Current Maintainer of this work is Niklas Beisert.

This work consists of the files |README.txt|, |childdoc.ins| and |childdoc.dtx|
as well as the derived files |childdoc.def|, |cdocsamp.tex|
with |cdocsch1.tex|, |cdocsch2.tex|, |cdocspt3.tex|, |cdocspt4.tex|,
|cdocsdrf.tex|, |cdocsfn1.tex|, |cdocsfn2.tex|
as well as |childdoc.pdf|.

%%%%%%%%%%%%%%%%%%%%%%%%%%%%%%%%%%%%%%%%%%%%%%%%%%%%%%%%%%%%%%%%%%%%%%%%%%%%%%%%
\subsection{Files and Installation}

The package consists of the files:
%
\begin{center}
\begin{tabular}{ll}
    |README.txt|   & readme file \\
    |childdoc.ins| & installation file \\
    |childdoc.dtx| & source file \\
    |childdoc.def| & definition file \\
    |cdocsamp.tex| & sample main file \\
    |cdocsch1.tex| & sample include file \\
    |cdocsch2.tex| & sample include file \\
    |cdocspt3.tex| & sample part file \\
    |cdocspt4.tex| & sample part file \\
    |cdocsdrf.tex| & sample redirection file \\
    |cdocsfn1.tex| & sample redirection file \\
    |cdocsfn2.tex| & sample redirection file \\
    |childdoc.pdf| & manual
\end{tabular}
\end{center}
%
The distribution consists of the files
|README.txt|, |childdoc.ins| and |childdoc.dtx|.
%
\begin{itemize}
\item
Run (pdf)\LaTeX{} on |childdoc.dtx|
to compile the manual |childdoc.pdf| (this file).
\item
Run \LaTeX{} on |childdoc.ins| to create the definitions file |childdoc.def|
and the sample |cdocsamp.tex| with include files
|cdocsch1.tex|, |cdocsch2.tex|, |cdocspt3.tex|, |cdocspt4.tex|,
|cdocsdrf.tex|, |cdocsfn1.tex|, |cdocsfn2.tex|.
Then copy the file |childdoc.def| to an appropriate directory of your \LaTeX{}
distribution, e.g.\ \textit{texmf-root}|/tex/latex/childdoc|.
\end{itemize}

%%%%%%%%%%%%%%%%%%%%%%%%%%%%%%%%%%%%%%%%%%%%%%%%%%%%%%%%%%%%%%%%%%%%%%%%%%%%%%%%
\subsection{Related CTAN Packages}

There are several other packages which offer a similar functionality:
%
\begin{itemize}
\item
The packages
\href{http://ctan.org/pkg/docmute}{\textsf{docmute}},
\href{http://ctan.org/pkg/includex}{\textsf{includex}} and
\href{http://ctan.org/pkg/standalone}{\textsf{standalone}}
provide commands to include only the document body of
a child file thus allowing both files to be compiled individually.
\item
The packages \href{http://ctan.org/pkg/subdocs}{\textsf{subdocs}}
and \href{http://ctan.org/pkg/subfiles}{\textsf{subfiles}}
provide structures in which the main and child documents can be
encapsulated and allowing them to be compiled individually.
The inclusion mechanism is different from the conventional |\include|.
\item
The package \href{http://ctan.org/pkg/combine}{\textsf{combine}}
is an elaborate solution to combine several documents into one.
\end{itemize}
%
See also the CTAN topic \href{http://ctan.org/topic/subdocs}{\textsf{subdocs}}
for further related packages.
The present package differs from the above solutions in that
a document structure constructed with the conventional |\include| mechanism
just needs two extra commands at the top of every file
such that all constituent files can be compiled individually.

%%%%%%%%%%%%%%%%%%%%%%%%%%%%%%%%%%%%%%%%%%%%%%%%%%%%%%%%%%%%%%%%%%%%%%%%%%%%%%%%
%\subsection{Feature Suggestions}
%
%The following is a list of features which may be useful for future
%versions of this package:
%%
%\begin{itemize}
%\item
%\ldots
%\end{itemize}

%%%%%%%%%%%%%%%%%%%%%%%%%%%%%%%%%%%%%%%%%%%%%%%%%%%%%%%%%%%%%%%%%%%%%%%%%%%%%%%%
\subsection{Revision History}

%%%%%%%%%%%%%%%%%%%%%%%%%%%%%%%%%%%%%%%%
\paragraph{v2.0:} 2018/12/30

\begin{itemize}
\item
immediate forward processing
\item
added |\childdocby| mechanism
\item
manual restructured
\end{itemize}

%%%%%%%%%%%%%%%%%%%%%%%%%%%%%%%%%%%%%%%%
\paragraph{v1.6:} 2018/01/17

\begin{itemize}
\item
application for development of include files
\item
corrections to manual
\end{itemize}

%%%%%%%%%%%%%%%%%%%%%%%%%%%%%%%%%%%%%%%%
\paragraph{v1.5:} 2017/05/21

\begin{itemize}
\item
more complete structuring introduced
\item
|\childdocof| introduced
\item
|\childdoc| renamed to |\childdocmain|
\item
|\childredirect| renamed to |\childdocforward| and |\childdocforwardprefix|
and functionality expanded
\end{itemize}

%%%%%%%%%%%%%%%%%%%%%%%%%%%%%%%%%%%%%%%%
\paragraph{v1.0:} 2017/04/27

\begin{itemize}
\item
manual and install package
\item
first version published on CTAN
\end{itemize}

%%%%%%%%%%%%%%%%%%%%%%%%%%%%%%%%%%%%%%%%
\paragraph{v0.6:} 2017/04/26

\begin{itemize}
\item
redirection mechanism added
\end{itemize}

%%%%%%%%%%%%%%%%%%%%%%%%%%%%%%%%%%%%%%%%
\paragraph{v0.5:} 2017/04/26

\begin{itemize}
\item
functionality in definition file
\end{itemize}


%%%%%%%%%%%%%%%%%%%%%%%%%%%%%%%%%%%%%%%%%%%%%%%%%%%%%%%%%%%%%%%%%%%%%%%%%%%%%%%%
%%%%%%%%%%%%%%%%%%%%%%%%%%%%%%%%%%%%%%%%%%%%%%%%%%%%%%%%%%%%%%%%%%%%%%%%%%%%%%%%
%%%%%%%%%%%%%%%%%%%%%%%%%%%%%%%%%%%%%%%%%%%%%%%%%%%%%%%%%%%%%%%%%%%%%%%%%%%%%%%%
\appendix

\settowidth\MacroIndent{\rmfamily\scriptsize 000\ }

 \DocInput{childdoc.dtx}

\end{document}
%</driver>
% \fi
%
% %%%%%%%%%%%%%%%%%%%%%%%%%%%%%%%%%%%%%%%%%%%%%%%%%%%%%%%%%%%%%%%%%%%%%%%%%%%%%%
% %%%%%%%%%%%%%%%%%%%%%%%%%%%%%%%%%%%%%%%%%%%%%%%%%%%%%%%%%%%%%%%%%%%%%%%%%%%%%%
% \section{Sample}
%\iffalse
%<*samplemain>
%\fi
%
% The following presents a sample document
% with two chapters, two parts, a title page,
% a compile flag as well as three forwarding files to set the flag.
% It consists of eight |.tex| files:
% \begin{center}
% \begin{tabular}{ll}
% |cdocsamp.tex|&main file\\
% |cdocsch1.tex|&include file for chapter 1\\
% |cdocsch2.tex|&include file for chapter 2\\
% |cdocspt3.tex|&include file for part 3\\
% |cdocspt4.tex|&include file for part 4\\
% |cdocsdrf.tex|&forwarding file for main file in draft mode\\
% |cdocsfi1.tex|&forwarding file for final version of chapter 1\\
% |cdocsfi2.tex|&forwarding file for final version of chapter 2\\
% \end{tabular}
% \end{center}
% Each of the eight files can be compiled directly by the \LaTeX{} compiler.
%
% %%%%%%%%%%%%%%%%%%%%%%%%%%%%%%%%%%%%%%
% \paragraph{Main File.}
%
% The main file is called |cdocsamp.tex|.
%
% Load the \textsf{childdoc} definitions and
% declare the filename for the main document:
%    \begin{macrocode}
\input{childdoc.def}
\childdocmain{}
%    \end{macrocode}

% Optional override for |\version| flag:
%    \begin{macrocode}
%%\ifchilddoc\else\providecommand{\version}{draft}\fi
%    \end{macrocode}

% Define the default values for the |\version| flag
% (|final| for the main file and |draft| for childs):
%    \begin{macrocode}
\ifchilddoc
\providecommand{\version}{draft}
\else
\providecommand{\version}{final}
\fi
%    \end{macrocode}

% Load the standard document class:
%    \begin{macrocode}
\documentclass[12pt]{article}
%    \end{macrocode}

% Start the document body:
%    \begin{macrocode}
\begin{document}
%    \end{macrocode}

% Declare a title page.
% Print title, part of document being processed and version flag:
%    \begin{macrocode}
\addtocounter{page}{-1}
\begin{center}
{\LARGE\bfseries{}childdoc example\par}
\vspace{1cm}
\ifchilddoc
\ifchilddocmanual part\else chapter\fi:
`\childdocname' of `\childdocjob'\par
\else
main document: `\childdocjob'\par
\fi
version: \version\par
\end{center}
\newpage
%    \end{macrocode}

% Manually include selected file,
% otherwise process as usual:
%    \begin{macrocode}
\ifchilddocmanual
\section*{part `\childdocname'}
\input{\childdocname}
\else
%    \end{macrocode}

% Include the two chapters:
%    \begin{macrocode}
\include{cdocsch1}
\include{cdocsch2}
%    \end{macrocode}

% Include the two parts unless only chapters should be displayed:
%    \begin{macrocode}
\ifchilddoc\else
\section{part three}
\input{cdocspt3}
\section{part four}
\input{cdocspt4}
\fi
%    \end{macrocode}

% Process as usual until here:
%    \begin{macrocode}
\fi
%    \end{macrocode}

% End of document body:
%    \begin{macrocode}
\end{document}
%    \end{macrocode}
%\iffalse
%</samplemain>
%\fi
%
% %%%%%%%%%%%%%%%%%%%%%%%%%%%%%%%%%%%%%%
% \paragraph{Chapter Include Files.}
%
% The include files are called |cdocsch1.tex| and |cdocsch2.tex|.
%
%\iffalse
%<*samplechap1|samplechap2>
%\fi

% Optional override for |\version| flag:
%    \begin{macrocode}
%%\providecommand{\version}{final}
%    \end{macrocode}

% Include the main document:
%    \begin{macrocode}
\input{childdoc.def}
\childdocof{cdocsamp}
%    \end{macrocode}

%\iffalse
%</samplechap1|samplechap2>
%\fi
%
%\iffalse
%<*samplechap1>
%\fi
% Some text for chapter 1:
%    \begin{macrocode}
\section{one}
some text in chapter one
%    \end{macrocode}

%\iffalse
%</samplechap1>
%\fi
% Some text for chapter 2:
%\iffalse
%<*samplechap2>
%\fi
%    \begin{macrocode}
\section{two}
more text in chapter two
%    \end{macrocode}

%\iffalse
%</samplechap2>
%\fi
%
% %%%%%%%%%%%%%%%%%%%%%%%%%%%%%%%%%%%%%%
% \paragraph{Part Include Files.}
%
% The include files are called |cdocspt3.tex| and |cdocspt4.tex|.
%
%\iffalse
%<*samplepart3|samplepart4>
%\fi

% Optional override for |\version| flag:
%    \begin{macrocode}
%%\providecommand{\version}{final}
%    \end{macrocode}

% Include the main document:
%    \begin{macrocode}
\input{childdoc.def}
\childdocby{cdocsamp}
%    \end{macrocode}

%\iffalse
%</samplepart3|samplepart4>
%\fi
%
%\iffalse
%<*samplepart3>
%\fi
% Some text for part 3:
%    \begin{macrocode}
some text in part three
%    \end{macrocode}

%\iffalse
%</samplepart3>
%\fi
% Some text for part 4:
%\iffalse
%<*samplepart4>
%\fi
%    \begin{macrocode}
more text in part four
%    \end{macrocode}

%\iffalse
%</samplepart4>
%\fi
%
% %%%%%%%%%%%%%%%%%%%%%%%%%%%%%%%%%%%%%%
% \paragraph{Forwarding for a Complete Draft.}
%
% The following forwarding file |cdocsdrf.tex|
% compiles the main document in draft mode:
%\iffalse
%<*sampledraft>
%\fi
%    \begin{macrocode}
\def\version{draft}
\input{childdoc.def}
\childdocforward{cdocsamp}
%    \end{macrocode}

%\iffalse
%</sampledraft>
%\fi
%
% %%%%%%%%%%%%%%%%%%%%%%%%%%%%%%%%%%%%%%
% \paragraph{Forwarding for Final Version of the Chapters.}
%
% The following forwarding files |cdocsfn1.tex| and |cdocsfn2.tex|
% (with identical content)
% compile the final versions of the child documents
% |cdocsch1.tex| and |cdocsch2.tex|, respectively:
%\iffalse
%<*samplefinal>
%\fi
%    \begin{macrocode}
\def\version{final}
\input{childdoc.def}
\childdocforwardprefix[cdocsamp]{cdocsfn}{cdocsch}
%    \end{macrocode}

%\iffalse
%</samplefinal>
%\fi
%
% %%%%%%%%%%%%%%%%%%%%%%%%%%%%%%%%%%%%%%
% \paragraph{Command Line Processing.}
%
% The following three command lines generate the output files
% |cdocscld|, |cdocscl1| and |cdocscl2|
% which should be identical to
% |cdocsdrf|, |cdocsch1| and |cdocsfn2|, respectively:
% \begin{center}
% \begin{tabular}{l}
% |latex -jobname cdocscld \|\\
% |  "\def\version{draft}\input{childdoc.def}\childdocforward{cdocsamp}"|\\
% |latex -jobname cdocscl1 \|\\
% |  "\input{childdoc.def}\childdocforward[cdocsamp]{cdocsch1}"|\\
% |latex -jobname cdocscl2 \|\\
% |  "\def\version{final}\input{childdoc.def}\childdocforward{cdocsch2}"|
% \end{tabular}
% \end{center}
% Note that the trailing backslash on each first line
% merely continues the input to the second line
% (for convenient cut ant paste).
% Furthermore, the command |latex| can be replaced by any
% of its alternative versions such as |pdflatex|.
%
% %%%%%%%%%%%%%%%%%%%%%%%%%%%%%%%%%%%%%%%%%%%%%%%%%%%%%%%%%%%%%%%%%%%%%%%%%%%%%%
% %%%%%%%%%%%%%%%%%%%%%%%%%%%%%%%%%%%%%%%%%%%%%%%%%%%%%%%%%%%%%%%%%%%%%%%%%%%%%%
% \section{Implementation}
%\iffalse
%<*package>
%\fi
%
% This section describes the definitions file |childdoc.def|.

% The definitions cannot be loaded using |\usepackage| or |\RequirePackage|
% which has a mechanism to prevent loading a style file more than once.
% When loading the definitions by means of |\input|
% multiple instances have to be prevented manually:
%\iffalse
%This code needs to be before the `\ProvidesFile' directive
%which is defined at the beginning of this file.
%Therefore it is also placed there and commented out here.
%</package>
%<*discard>
%\fi
%    \begin{macrocode}
\ifdefined\childdocmain\endinput\fi
%    \end{macrocode}
%\iffalse
%</discard>
%<*package>
%\fi
%
% \macro{\ifchilddoc}
% \macro{\ifchilddocmanual}
% The conditional |\ifchilddoc| tells whether a
% child (true) or main (false) document is being compiled.
% The conditional |\ifchilddocmanual| tells whether
% the |\includeonly| mechanism is used (false) or
% the selection of child files must be performed manually (true).
% The definitions initialise to false:
%    \begin{macrocode}
\newif\ifchilddoc
\newif\ifchilddocmanual
%    \end{macrocode}

% \macro{\childdocname}
% \macro{\childdocjob}
% The macro |\childdocname| stores the name of the main document
% to be compiled. The macro |\childdocjob| stores the name of
% the document on which the \LaTeX{} compiler was originally invoked.
% The content of |\jobname| cannot be compared
% to filenames specified in the source due to different catcodes.
% The following code rescans |\jobname|, stores the result
% in |\childdocname| and saves a copy in |\childdocjob|:
%    \begin{macrocode}
\edef\childdocname{\scantokens\expandafter{\jobname\noexpand}}
\let\childdocjob\childdocname
%    \end{macrocode}

% \macro{\childdocdisable}
% The macro |\childdocdisable| prevents the main file
% from being processed more than once.
% At this stage, the main document command |\childdocmain|
% is assumed to be called once again where it should do nothing.
% Any subsequent call to it should prevent
% a secondary processing of the main document
% It overwrites the forwarding commands
% |\childdocof| and |\childdocforward|
% with empty macros to prevent further inclusions of the main document:
%    \begin{macrocode}
\newcommand{\childdocdisable}
{
  \renewcommand{\childdocmain}[1]{\renewcommand{\childdocmain}[1]{\endinput}}
  \renewcommand{\childdocof}[1]{}
  \renewcommand{\childdocby}[2][]{}
  \renewcommand{\childdocforward}[2][]{}
  \renewcommand{\childdocdisable}{}
}
%    \end{macrocode}

% \macro{\childdocmain}
% The macro |\childdocmain| is to be called at the top of the main file
% with nothing or the main filename (without extension) as argument.
% First, it breaks loops.
% If the argument is not empty and does not match |\childdocname|
% (which is set by the first inclusion of |childdoc.def|),
% |\ifchilddoc| is set to true, |\includeonly| is applied to the child file
% and |\jobname| is set to the main file
% (for proper handling of |.aux| files):
%    \begin{macrocode}
\newcommand{\childdocmain}[1]
{
  \childdocdisable\childdocmain{}
  \if?#1?\else
    \begingroup
      \def\childdoctmp{#1}
      \ifx\childdoctmp\childdocname
        \def\childdoctmp{}
      \else
        \def\childdoctmp
        {
          \childdoctrue
          \includeonly{\childdocname}
          \def\childdocjob{#1}
          \def\jobname{#1}
        }
      \fi
      \expandafter
    \endgroup
    \childdoctmp
  \fi
}
%    \end{macrocode}

% \macro{\childdocof}
% The command |\childdocof| redirects
% compilation to the main file |#1|.
%    \begin{macrocode}
\newcommand{\childdocof}[1]
{
  \childdocdisable
  \childdoctrue
  \includeonly{\childdocname}
  \def\jobname{#1}
  \def\childdocjob{#1}
  \input{#1}
}
%    \end{macrocode}

% \macro{\childdocby}
% The command |\childdocby| ....
%    \begin{macrocode}
\newcommand{\childdocby}[2][]
{
  \childdocdisable
  \childdoctrue
  \childdocmanualtrue
  \if?#1?\else
    \def\jobname{#2}
  \fi
  \def\childdocjob{#2}
  \input{#2}
  \endinput
}
%    \end{macrocode}

% \macro{\childdocforward}
% The command |\childdocforward| redirects
% compilation to the main file or
% (if the optional argument is given) a child file.
% Parameters are set as if the main file
% or a child file starting with |\childdocof| was compiled.
% Then compilation is handed over to the main file:
%    \begin{macrocode}
\newcommand{\childdocforward}[2][]
{
  \begingroup
    \if?#1?
      \def\childdoctmp
      {
        \def\childdocname{#2}
        \def\childdocjob{#2}
        \def\jobname{#2}
        \input{#2}
        \endinput
      }
    \else
      \def\childdoctmp
      {
        \childdocdisable
        \def\childdocname{#2}
        \childdoctrue
        \includeonly{#2}
        \def\childdocjob{#1}
        \def\jobname{#1}
        \input{#1}
        \endinput
      }
    \fi
    \expandafter
  \endgroup
  \childdoctmp
}
%    \end{macrocode}

% \macro{\childdocforwardprefix}
% The command |\childdocforwardprefix| redirects
% compilation to the main or a child file by means of a pattern.
% The prefix |#1| in the current filename is replaced by |#2|
% and the suffix of the current filename is kept
% (it is assumed that the filename does not contain the substring `|~~~|'
% which is used as a delimiter).
% Compilation is handed over to the new file by |\childdocforward|:
%    \begin{macrocode}
\newcommand{\childdocforwardprefix}[3][]
{
  \begingroup
    \def\childdocextract #2##1~~~{\def\childdoctmp{\childdocforward[#1]{#3##1}}}
    \expandafter\childdocextract\childdocname~~~
    \expandafter
  \endgroup
  \childdoctmp
}
%    \end{macrocode}

% \macro{\childdoc}
% The deprecated macro |\childdoc| is a legacy version of |\childdocmain|:
%    \begin{macrocode}
\newcommand{\childdoc}{\childdocmain}
%    \end{macrocode}

% \macro{\childdocredirect}
% The deprecated macro |\childdocredirect| is a legacy version
% of |\childdocforward| and |\childdocforwardprefix|:
%    \begin{macrocode}
\newcommand{\childdocredirect}[2][]
{
  \begingroup
    \if?#1?
      \def\childdoctmp{\childdocforward{#2}}
    \else
      \def\childdoctmp{\childdocforwardprefix{#1}{#2}}
    \fi
    \expandafter
  \endgroup
  \childdoctmp
}
%    \end{macrocode}

%\iffalse
%</package>
%\fi
%
\endinput
\childdocforward[|\textit{main}|]{|\textit{dest}|}"|
\end{center}
%
Here \textit{target} is the name of the output file,
\textit{main} is the name of the main file
and \textit{dest} is the name of the main or child file to be processed
(all filenames without extensions).
The optional argument \textit{main} can be omitted
if \textit{main} matches \textit{dest}.
Optionally, compilation \textit{flags} can be defined via |\def| commands.
This command line makes the \TeX{} engine believe
it is compiling the file \textit{target}
whose content is specified as the latter parameter.
The provided code then forwards the processing to
\textit{main} or \textit{dest} as described in \secref{sec:forward}.

%%%%%%%%%%%%%%%%%%%%%%%%%%%%%%%%%%%%%%%%%%%%%%%%%%%%%%%%%%%%%%%%%%%%%%%%%%%%%%%%
\subsection{Include by Input}
\label{sec:input}

Including child documents by |\include| has some restrictions by design.
Most notably, the content of a child document always occupies
its own set of pages; pages cannot be shared between child documents.
Usually, this behaviour makes perfect sense
because each child document contain an essential part of the document.
However, in some situations it may be desirable to compose
a document from a collection of parts
without having mandatory page breaks between then.
For this case, the package
provides a mechanism to include parts
by |\input| which can also be processed individually.
However, by construction this mechanism
requires manual handling of the content to be output.

%%%%%%%%%%%%%%%%%%%%%%%%%%%%%%%%%%%%%%%%
\DescribeMacro{\ifchilddocmanual}
The main file should be prepared as usual, see \secref{sec:include}.
However, the document body must make a distinction
between processing of an individual part and of the main document, e.g.:
%
\begin{center}
\begin{tabular}{l}
|\ifchilddocmanual|\\
|\input{\childdocname}|\\
|\||else|\\
\textit{document body with }|\input{|\textit{part}|}|\\
|\||fi|
\end{tabular}
\end{center}
%
The conditional |\ifchilddocmanual| is true whenever
a part to be included by |\input| is being compiled,
and the name of the part is stored in |\childdocname|.

%%%%%%%%%%%%%%%%%%%%%%%%%%%%%%%%%%%%%%%%
\DescribeMacro{\childdocby}
Each part to be included by |\input| should start with:
%
\begin{center}
\begin{tabular}{l}
|% \iffalse
%
% childdoc.dtx Copyright (C) 2017-2018 Niklas Beisert
%
% This work may be distributed and/or modified under the
% conditions of the LaTeX Project Public License, either version 1.3
% of this license or (at your option) any later version.
% The latest version of this license is in
%   http://www.latex-project.org/lppl.txt
% and version 1.3 or later is part of all distributions of LaTeX
% version 2005/12/01 or later.
%
% This work has the LPPL maintenance status `maintained'.
%
% The Current Maintainer of this work is Niklas Beisert.
%
% This work consists of the files childdoc.dtx and childdoc.ins
% and the derived files childdoc.def and cdocsamp.tex with
% cdocsch1.tex, cdocsch2.tex, cdocsdrf.tex, cdocsfn1.tex, cdocsfn2.tex.
%
%<package>\ifdefined\childdocmain\endinput\fi
%<package>\ProvidesFile{childdoc.def}[2018/12/30 v2.0 child document driver]
%<samplemain>\ProvidesFile{cdocsamp.tex}[2018/12/30 v2.0 sample for childdoc]
%<*driver>
%\ProvidesFile{childdoc.drv}[2018/12/30 v2.0 childdoc reference manual file]
\PassOptionsToClass{10pt,a4paper}{article}
\documentclass{ltxdoc}

\usepackage[margin=35mm]{geometry}
\usepackage{hyperref}
\usepackage{hyperxmp}
\usepackage[usenames]{color}

\hypersetup{colorlinks=true}
\hypersetup{pdfstartview=FitH}
\hypersetup{pdfpagemode=UseNone}
\hypersetup{pdfsource={}}
\hypersetup{pdflang={en-UK}}
\hypersetup{pdfcopyright={Copyright 2017-2018 Niklas Beisert.
  This work may be distributed and/or modified under the
  conditions of the LaTeX Project Public License, either version 1.3
  of this license or (at your option) any later version.}}
\hypersetup{pdflicenseurl={http://www.latex-project.org/lppl.txt}}
\hypersetup{pdfcontactaddress={ETH Zurich, ITP, HIT K,
  Wolfgang-Pauli-Strasse 27}}
\hypersetup{pdfcontactpostcode={8093}}
\hypersetup{pdfcontactcity={Zurich}}
\hypersetup{pdfcontactcountry={Switzerland}}
\hypersetup{pdfcontactemail={nbeisert@itp.phys.ethz.ch}}
\hypersetup{pdfcontacturl={http://people.phys.ethz.ch/\xmptilde nbeisert/}}

\newcommand{\secref}[1]{\hyperref[#1]{section \ref*{#1}}}

\parskip1ex
\parindent0pt
\let\olditemize\itemize
\def\itemize{\olditemize\parskip0pt}

\begin{document}

\title{The \textsf{childdoc} Package}
\hypersetup{pdftitle={The childdoc Package}}
\author{Niklas Beisert\\[2ex]
  Institut f\"ur Theoretische Physik\\
  Eidgen\"ossische Technische Hochschule Z\"urich\\
  Wolfgang-Pauli-Strasse 27, 8093 Z\"urich, Switzerland\\[1ex]
  \href{mailto:nbeisert@itp.phys.ethz.ch}
  {\texttt{nbeisert@itp.phys.ethz.ch}}}
\hypersetup{pdfauthor={Niklas Beisert}}
\hypersetup{pdfsubject={Manual for the LaTeX2e Package childdoc}}
\date{30 December 2018, \textsf{v2.0}}
\maketitle

\begin{abstract}\noindent
\textsf{childdoc} is a \LaTeXe{} package
that enables the direct compilation
of document sections included by |\include|
to individual files.
\end{abstract}

\begingroup
\parskip0ex
\tableofcontents
\endgroup

%%%%%%%%%%%%%%%%%%%%%%%%%%%%%%%%%%%%%%%%%%%%%%%%%%%%%%%%%%%%%%%%%%%%%%%%%%%%%%%%
%%%%%%%%%%%%%%%%%%%%%%%%%%%%%%%%%%%%%%%%%%%%%%%%%%%%%%%%%%%%%%%%%%%%%%%%%%%%%%%%
\section{Introduction}

\LaTeX{} provides a mechanism to structure a large document (such as a book)
into a main file and several child files (containing the chapters)
using the |\include| command.
This mechanism is beneficial for documents
which span hundreds of pages in order to
make the source file(s) more manageable.
Moreover, compilation can be restricted to
selected child files by means of the |\includeonly| command.
The latter feature can be used to reduce the compilation time while editing
(this was significantly more useful in the earlier days of \LaTeX{})
or to generate a smaller document which is easier to navigate.
Another application of |\includeonly| is to generate
documents consisting of selected parts of the complete document.

However, there are a few drawbacks of the plain |\include| mechanism:
\begin{itemize}
\item
The child files cannot be compiled on their own,
they can only be compiled via the main file.
A naive editing environment
(such as a text editor with an option
to have the current file processed by \LaTeX)
may require one to switch to the main file before compiling;
attempting to compile the child file produces errors.
\item
The main file must be modified (each time)
to adjust the |\includeonly| command
to the present needs. This easily leaves the main file in a messy state.
\item
The generated document will always carry the filename
of the main document. This is inconvenient if
several child files are to be compiled and
to be kept for distribution.
\end{itemize}

The present package provides a simple interface
to make child files individually compilable by \LaTeX{}.
Compiling a child file then has the same effect as compiling
the main file with an |\includeonly| command
to select the appropriate child.
Moreover the generated document will carry the name of the child
rather than the main file.
This resolves all three above issues.

This feature is meant to make the editing of books,
thesis documents and lecture notes somewhat more convenient.
However, the package can also be used efficiently for
composing a series of documents (such as exercise sheets)
which are typically distributed individually.
It then assists the author in generating the individual documents
(potentially in different versions)
as well as a document containing the collected series.
Another application is in developing style files
or other kinds of included material
where compilation of the style file could redirect
to a sample or test file.

%%%%%%%%%%%%%%%%%%%%%%%%%%%%%%%%%%%%%%%%%%%%%%%%%%%%%%%%%%%%%%%%%%%%%%%%%%%%%%%%
%%%%%%%%%%%%%%%%%%%%%%%%%%%%%%%%%%%%%%%%%%%%%%%%%%%%%%%%%%%%%%%%%%%%%%%%%%%%%%%%
\section{Usage}

First of all, the package \textsf{childdoc} is \emph{not} a standard
\LaTeXe{} |.sty| style file! Therefore it needs to be invoked in
a non-standard way.

%%%%%%%%%%%%%%%%%%%%%%%%%%%%%%%%%%%%%%%%%%%%%%%%%%%%%%%%%%%%%%%%%%%%%%%%%%%%%%%%
\subsection{Included Files}
\label{sec:include}

%%%%%%%%%%%%%%%%%%%%%%%%%%%%%%%%%%%%%%%%
\DescribeMacro{\childdocmain}
To use the package, add the commands
\begin{center}
\begin{tabular}{l}
|\input{childdoc.def}|\\
|\childdocmain{}|\\
\end{tabular}
\end{center}
at the very top of the main \LaTeX{} file,
in particular \emph{before} the |\documentclass| statement!
The argument of |\childdocmain| should be left empty
(but it must be present).

%%%%%%%%%%%%%%%%%%%%%%%%%%%%%%%%%%%%%%%%
\DescribeMacro{\childdocof}
Furthermore, add the commands
\begin{center}
\begin{tabular}{l}
|\input{childdoc.def}|\\
|\childdocof{|\textit{main}|}|\\
\end{tabular}
\end{center}
at the top of every child file \textit{child}
which is included by |\include{|\textit{child}|}|
from within the main file
(or at least for those files to be compiled individually).
The argument \textit{main} must be the filename of the main file.

There are a couple of
considerations in setting up the main and child documents:

%%%%%%%%%%%%%%%%%%%%%%%%%%%%%%%%%%%%%%%%
\paragraph{Restrictions.}

Please note the following restrictions:
\begin{itemize}
\item
|\childdocmain| must be called with one argument \textit{main}
to ensure compatibility with earlier version of the package.
It must either be empty (|\childdocmain{}|)
or precisely match the filename of the main file in which it is specified.
See \secref{sec:detection} for further information.
\item
The filename \textit{main} must be specified without the |.tex| extension.
\item
The filename \textit{main} is case sensitive
(even in case-insensitive file systems)
due to internal string comparison.
\item
The argument \textit{main} should be fully expanded, it cannot be a macro.
\item
Subdirectories and special characters should be avoided in filenames.
\item
The command |\childdocmain{|\textit{main}|}| must be followed by a whitespace.
It should not be followed immediately by another command
or by a comment mark `|%|'.
This is because the \TeX{} parser reads the token immediately following
the argument of |\childdocmain| and puts it
at the beginning of every child section;
however, a white\-space is ignored.
\end{itemize}

%%%%%%%%%%%%%%%%%%%%%%%%%%%%%%%%%%%%%%%%
\paragraph{Content of Main File.}

It is advisable to place all content in the child files included by |\include|.
Any output contained in the main file will appear in all child documents
unless suppressed manually;
it cannot be suppressed automatically by the |\includeonly| directive
and thus should normally be avoided.
A method to include some content in the main file
by means of conditional processing is described in \secref{sec:conditional}.

%%%%%%%%%%%%%%%%%%%%%%%%%%%%%%%%%%%%%%%%
\paragraph{Page Numbering.}

When only a part of the document is compiled,
the appropriate numbering of pages
(as well as other status parameters)
is determined from the |.aux| files.
The latter contain information from previous passes.
However this information needs to propagate through
all intermediate child documents.
Therefore the page numbering in child documents may well
be inconsistent until the complete document is compiled at least once.

A useful (if unconventional) way to always ensure a consistent
page numbering is to restart the numbering in each child document
and denote the pages by `\textit{child}|.|\textit{page}'
where \textit{child} represents the chapter/section number of the child file.
This can be achieved by the command
|\numberwithin{page}{|\textit{child}|}|
of the \textsf{amsmath} package
where \textit{child} can be |chapter| or |section|
depending on the chosen structuring.
Alternatively, one can modify the macro |\thepage| appropriately
and reset the counter |page| at the start of each child file.

%%%%%%%%%%%%%%%%%%%%%%%%%%%%%%%%%%%%%%%%%%%%%%%%%%%%%%%%%%%%%%%%%%%%%%%%%%%%%%%%
\subsection{Conditional Processing}
\label{sec:conditional}

The package provides a mechanism to compile different versions
of a document. To customise the versions further some conditional processing
can come in handy to distinguish which version is being compiled.
The package provides two macros to describe the compilation context:

%%%%%%%%%%%%%%%%%%%%%%%%%%%%%%%%%%%%%%%%
\DescribeMacro{\ifchilddoc}
The conditional |\ifchilddoc| distinguishes between the compilation of
child documents and the main document:
%
\begin{center}
|\ifchilddoc |\textit{child-code}| |[|\||else |\textit{main-code}]| \||fi|
\end{center}

%%%%%%%%%%%%%%%%%%%%%%%%%%%%%%%%%%%%%%%%
\DescribeMacro{\childdocname}
\DescribeMacro{\childdocjob}
The macro |\childdocname| contains the filename (without extension)
of the main or child file being processed.
Note that |\childdocjob| will always contain the name of the main file.

%%%%%%%%%%%%%%%%%%%%%%%%%%%%%%%%%%%%%%%%
\paragraph{Title Page.}

Conditional processing can be used to include a title or banner page
in the main document when proper precautions are taken.
Importantly, the code in the main file should ensure that the page counter
(as well as other status parameters which are stored in the |.aux| files)
takes the same value after the conditional processing.
Otherwise the page numbers may take divergent values
depending on which part is compiled.

For example, a title page could be declared by:
%
\begin{center}
\begin{tabular}{l}
|\ifchilddoc\||else|\\
|\addtocounter{page}{-1}|\\
\textit{code for title page}\\
|\newpage|\\
|\||fi|
\end{tabular}
\end{center}
%
A banner page for the child documents can be generated by:
%
\begin{center}
\begin{tabular}{l}
|\ifchilddoc|\\
|\addtocounter{page}{-1}|\\
\textit{code for banner page}\\
|\newpage|\\
|\||fi|
\end{tabular}
\end{center}
%
Here one could write a message such as:
\begin{center}
|This is the part \childdocname{} of \childdocjob{}.|
\end{center}

%%%%%%%%%%%%%%%%%%%%%%%%%%%%%%%%%%%%%%%%%%%%%%%%%%%%%%%%%%%%%%%%%%%%%%%%%%%%%%%%
\subsection{Flags}
\label{sec:flags}

The package makes it easy to generate different versions
of the main or child documents.
To this end compilation flags can be defined
and assigned different default values.
They will be particularly useful in conjunction
with the forwarding mechanism described in \secref{sec:forward}.

For example, it may be useful to have a flag |\version|
which can be set to |draft| or |final|.
The document source will contain some conditional code
depending on the value of |\version|.
Suppose further, the flag should default to |final| for the main file
and to |draft| for child files
which is a natural assignment for editing the document.
This is achieved by placing the following code
in the preamble of the main document
(below the |\childdocmain| directive):
%
\begin{center}
\begin{tabular}{l}
|\ifchilddoc|\\
|\providecommand{\version}{draft}|\\
|\||else|\\
|\providecommand{\version}{final}|\\
|\||fi|
\end{tabular}
\end{center}
%
The definition by |\providecommand| makes sure
that previous definitions are not overwritten.
Further statements |\providecommand{\version}{...}|
can thus be added before the above code to override it.

For the main file, one might add a line
(between |\childdocmain| and the above block)
%
\begin{center}
|%\ifchilddoc\||else\providecommand{\version}{draft}\||fi|
\end{center}
%
which can be uncommented to produce a draft version.
Likewise one can add a line to the very top of a child file
(above the |\childdocof{|\textit{main}|}| directive)
%
\begin{center}
|%\providecommand{\version}{final}|
\end{center}
%
which can be uncommented to produce the final version of this child document.

%%%%%%%%%%%%%%%%%%%%%%%%%%%%%%%%%%%%%%%%%%%%%%%%%%%%%%%%%%%%%%%%%%%%%%%%%%%%%%%%
\subsection{Forwarding}
\label{sec:forward}

Different versions of the main or child documents
using compilation flags as described in \secref{sec:flags}
can be (permanently) stored in different files
for convenient compilation, viewing and distribution.
To this end, the package defines a command
to pass on compilation to a different file:

%%%%%%%%%%%%%%%%%%%%%%%%%%%%%%%%%%%%%%%%
\DescribeMacro{\childdocforward}
The command |\childdocforward| redirects processing to
another source file:
%
\begin{center}
\begin{tabular}{l}
|\input{childdoc.def}|\\
|\childdocforward[|\textit{main}|]{|\textit{dest}|}|\\
\end{tabular}
\end{center}
%
The argument \textit{dest} is the destination file
(without extension).
It should be the main file or one of the child files.
Note that further \textsf{childdoc} directives
such as |\childdocof| and |\childdocforward|
in the indicated file will be processed in this form.
The optional argument \textit{main}
passes on directly to the main file \textit{main}
while pretending to compile the child \textit{dest}.
This form behaves as if \textit{dest}
issues |\childdocof{|\textit{main}|}| right away,
and no further \textsf{childdoc} directives will be processed.

%%%%%%%%%%%%%%%%%%%%%%%%%%%%%%%%%%%%%%%%
\DescribeMacro{\...prefix}
In the alternative form |\childdocforwardprefix|,
%
\begin{center}
\begin{tabular}{l}
|\input{childdoc.def}|\\
|\childdocforwardprefix[|\textit{main}|]{|\textit{prefix}|}{|\textit{dest}|}|
\end{tabular}
\end{center}
%
the destination file is determined by a pattern
depending on the current file:
To make this work, the current file must be called
`{\textit{prefix}\hspace{0.2em}\textit{suffix}}'
with \textit{prefix} matching precisely the argument.
Processing is then passed on to the file
`{\textit{dest}\hspace{0.2em}\textit{suffix}}'.
Surely, the same effect is achieved by
directly specifying the
argument `{\textit{dest}\hspace{0.2em}\textit{suffix}}'
in the first form.
However, that requires to set up a different file
for each child. With the alternative form of the command
all these files can have exactly the same content
which simplifies setting them up and maintaining them.

For example, the following file |draft.tex|
with a compilation flag |\version| as described in \secref{sec:flags}
compiles the main document as a draft:
%
\begin{center}
\begin{tabular}{l}
|\def\version{draft}|\\
|\input{childdoc.def}|\\
|\childdocforward{|\textit{main}|}|
\end{tabular}
\end{center}
%
Likewise, the following files |final|\textit{nn}|.tex|
compile the final version of the child document
|child|\textit{nn}|.tex|:
%
\begin{center}
\begin{tabular}{l}
|\def\version{final}|\\
|\input{childdoc.def}|\\
|\childdocforwardprefix{final}{child}|
\end{tabular}
\end{center}
%

Note that when several versions of a main file and/or of each child file
are to be generated, it may be convenient to set up a |Makefile| or
shell script to automatise the process.

%%%%%%%%%%%%%%%%%%%%%%%%%%%%%%%%%%%%%%%%%%%%%%%%%%%%%%%%%%%%%%%%%%%%%%%%%%%%%%%%
\subsection{Command Line Processing}
\label{sec:commandline}

The effect of redirection files can also be achieved by invoking
the \LaTeX{} compiler with a more elaborate command line.
Most conveniently this should be done as part
of a shell script or a |Makefile|.

When using \textsf{childdoc} in the main file, the following
command lines effectively perform a redirection
(note that depending on the shell being used,
backslashes may have to be doubled: `|\|' $\to$ `|\\|'):
%
\begin{center}
|... -jobname "|\textit{target}|" |\\|"|[\textit{flags}]%
|\input{childdoc.def}\childdocforward[|\textit{main}|]{|\textit{dest}|}"|
\end{center}
%
Here \textit{target} is the name of the output file,
\textit{main} is the name of the main file
and \textit{dest} is the name of the main or child file to be processed
(all filenames without extensions).
The optional argument \textit{main} can be omitted
if \textit{main} matches \textit{dest}.
Optionally, compilation \textit{flags} can be defined via |\def| commands.
This command line makes the \TeX{} engine believe
it is compiling the file \textit{target}
whose content is specified as the latter parameter.
The provided code then forwards the processing to
\textit{main} or \textit{dest} as described in \secref{sec:forward}.

%%%%%%%%%%%%%%%%%%%%%%%%%%%%%%%%%%%%%%%%%%%%%%%%%%%%%%%%%%%%%%%%%%%%%%%%%%%%%%%%
\subsection{Include by Input}
\label{sec:input}

Including child documents by |\include| has some restrictions by design.
Most notably, the content of a child document always occupies
its own set of pages; pages cannot be shared between child documents.
Usually, this behaviour makes perfect sense
because each child document contain an essential part of the document.
However, in some situations it may be desirable to compose
a document from a collection of parts
without having mandatory page breaks between then.
For this case, the package
provides a mechanism to include parts
by |\input| which can also be processed individually.
However, by construction this mechanism
requires manual handling of the content to be output.

%%%%%%%%%%%%%%%%%%%%%%%%%%%%%%%%%%%%%%%%
\DescribeMacro{\ifchilddocmanual}
The main file should be prepared as usual, see \secref{sec:include}.
However, the document body must make a distinction
between processing of an individual part and of the main document, e.g.:
%
\begin{center}
\begin{tabular}{l}
|\ifchilddocmanual|\\
|\input{\childdocname}|\\
|\||else|\\
\textit{document body with }|\input{|\textit{part}|}|\\
|\||fi|
\end{tabular}
\end{center}
%
The conditional |\ifchilddocmanual| is true whenever
a part to be included by |\input| is being compiled,
and the name of the part is stored in |\childdocname|.

%%%%%%%%%%%%%%%%%%%%%%%%%%%%%%%%%%%%%%%%
\DescribeMacro{\childdocby}
Each part to be included by |\input| should start with:
%
\begin{center}
\begin{tabular}{l}
|\input{childdoc.def}|\\
|\childdocby{|\textit{main}|}|\\
\end{tabular}
\end{center}
%
The directive |\childdocby| is similar to |\childdocof|
described in \secref{sec:include},
but the subsequent selection of content must be done manually.
To that end, both |\ifchilddoc| and |\ifchilddocmanual|
will be true upon processing of a part,
and the name of the part is stored in |\childdocname|.
Note that |\jobname| will be set to the filename of the current part
so that each part receives an individual |.aux| file
that does not interfere with the |.aux| file(s) of the main document.
This behaviour can be altered by the alternative form
|\childdocby[*]{|\textit{main}|}| (with a non-empty optional argument)
which uses the |.aux| file of the main document
by setting |\jobname| to \textit{main}.

%%%%%%%%%%%%%%%%%%%%%%%%%%%%%%%%%%%%%%%%%%%%%%%%%%%%%%%%%%%%%%%%%%%%%%%%%%%%%%%%
\subsection{Driver Development}
\label{sec:driver}

The \textsf{childdoc} mechanism can also be use for the development
of definition files such as \LaTeX{} styles or classes.
This case differs from the above setup with multiple parts
included by |\include| in that no |\includeonly| should be invoked.
This can be achieved by starting the include file
(before |\ProvidesPackage|) with:
%
\begin{center}
\begin{tabular}{l}
|\input{childdoc.def}|\\
|\childdocforward{|\textit{main}|}|\\
\end{tabular}
\end{center}
%
or alternatively with:
%
\begin{center}
\begin{tabular}{l}
|\input{childdoc.def}|\\
|\childdocby{|\textit{main}|}|\\
\end{tabular}
\end{center}
%
Both forms have slightly different effects as described above.
The main file is prepared as usual, see \secref{sec:include}.

%%%%%%%%%%%%%%%%%%%%%%%%%%%%%%%%%%%%%%%%%%%%%%%%%%%%%%%%%%%%%%%%%%%%%%%%%%%%%%%%
\subsection{Legacy Detection}
\label{sec:detection}

The directive |\childdocmain| in the main file can detect
whether the complete document or merely a child is to be compiled
even without using the directive |\childdocof|.
This method is deprecated because it is less robust
and there is no compelling reason to use it;
it is merely provided for backward compatibility
and it may be removed in future versions.

If the detection mechanism is to be used,
it is mandatory to correctly specify
the filename of the main file as the argument of |\childdocmain|:
%
\begin{center}
\begin{tabular}{l}
|\input{childdoc.def}|\\
|\childdocmain{|\textit{main}|}|\\
\end{tabular}
\end{center}
%
If |\jobname| does not match the argument \textit{main} of |\childdocmain|,
it is assumed that |\jobname| points to the child file to be compiled.
When using |\childdocmain| with the main file specified as argument,
it suffices to start a child file
with just |\input{|\textit{main}|}|
without loading of the package and using |\childdocof|.
If instead all processing is done
with the appropriate \textsf{childdoc} directives,
the argument of \textit{main} of |\childdocmain| can be empty.

An alternative version of the command line processing described
in \secref{sec:commandline} using the detection mechanism reads:
%
\begin{center}
|... -jobname "|\textit{target}|" "|[\textit{flags}]%
[|\def\jobname{|\textit{dest}|}|]|\input{|\textit{main}|}"|
\end{center}

%%%%%%%%%%%%%%%%%%%%%%%%%%%%%%%%%%%%%%%%%%%%%%%%%%%%%%%%%%%%%%%%%%%%%%%%%%%%%%%%
\subsection{Manual Code}
\label{sec:manual}

In case one cannot be certain whether the definitions file |childdoc.def|
is installed on the target \TeX{} distribution
and one prefers not to ship it,
it is conceivable to paste a few relevant commands into the sources.

To that end, drop all statements |\input{childdoc.def}|
and perform the replacements as outlined below.
Instead of |\childdocmain{|\textit{main}|}| add the following code
to the top of the main file:
%
\begin{center}
\begin{tabular}{l}
|\||ifdefined\childdocname\endinput\||fi\newif\ifchilddoc|\\
|\edef\childdocname{\scantokens\expandafter{\jobname\noexpand}}|\\
|\def\childdocmain{|\textit{main}|}\||ifx\childdocmain\childdocname\||else|\\
|\childdoctrue\includeonly{\childdocname}\let\jobname\childdocmain\||fi|\\
\end{tabular}
\end{center}
%
Instead of |\childdocof{|\textit{main}|}| just include the main file
at the top of each child file:
%
\begin{center}
|\input{|\textit{main}|}|
\end{center}
%
A simple redirection |\childdocforward{|\textit{dest}|}| is achieved by:
%
\begin{center}
|\def\jobname{|\textit{dest}|}\input{\jobname}|
\end{center}
%
The redirection with prefix
|\childdocforwardprefix[|\textit{prefix}|]{|\textit{dest}|}|
is accomplished by:
%
\begin{center}
\begin{tabular}{l}
|{\edef\jobname{\scantokens\expandafter{\jobname\noexpand}}|\\
|\def\redirectjob |\textit{prefix}|#1~~~{\gdef\jobname{|\textit{dest}|#1}}|\\
|\expandafter\redirectjob\jobname~~~}\input{\jobname}|
\end{tabular}
\end{center}

In an alternative approach,
child documents can be compiled by a specific command line
without additional code or specific definitions:
%
\begin{center}
|... -jobname "|\textit{target}|" "|[\textit{flags}]%
|\includeonly{|\textit{dest}|}\input{|\textit{main}|}"|
\end{center}
%

%%%%%%%%%%%%%%%%%%%%%%%%%%%%%%%%%%%%%%%%%%%%%%%%%%%%%%%%%%%%%%%%%%%%%%%%%%%%%%%%
%%%%%%%%%%%%%%%%%%%%%%%%%%%%%%%%%%%%%%%%%%%%%%%%%%%%%%%%%%%%%%%%%%%%%%%%%%%%%%%%
\section{Information}

%%%%%%%%%%%%%%%%%%%%%%%%%%%%%%%%%%%%%%%%%%%%%%%%%%%%%%%%%%%%%%%%%%%%%%%%%%%%%%%%
\subsection{Copyright}

Copyright \copyright{} 2017--2018 Niklas Beisert

This work may be distributed and/or modified under the
conditions of the \LaTeX{} Project Public License, either version 1.3
of this license or (at your option) any later version.
The latest version of this license is in
  \url{http://www.latex-project.org/lppl.txt}
and version 1.3 or later is part of all distributions of \LaTeX{}
version 2005/12/01 or later.

This work has the LPPL maintenance status `maintained'.

The Current Maintainer of this work is Niklas Beisert.

This work consists of the files |README.txt|, |childdoc.ins| and |childdoc.dtx|
as well as the derived files |childdoc.def|, |cdocsamp.tex|
with |cdocsch1.tex|, |cdocsch2.tex|, |cdocspt3.tex|, |cdocspt4.tex|,
|cdocsdrf.tex|, |cdocsfn1.tex|, |cdocsfn2.tex|
as well as |childdoc.pdf|.

%%%%%%%%%%%%%%%%%%%%%%%%%%%%%%%%%%%%%%%%%%%%%%%%%%%%%%%%%%%%%%%%%%%%%%%%%%%%%%%%
\subsection{Files and Installation}

The package consists of the files:
%
\begin{center}
\begin{tabular}{ll}
    |README.txt|   & readme file \\
    |childdoc.ins| & installation file \\
    |childdoc.dtx| & source file \\
    |childdoc.def| & definition file \\
    |cdocsamp.tex| & sample main file \\
    |cdocsch1.tex| & sample include file \\
    |cdocsch2.tex| & sample include file \\
    |cdocspt3.tex| & sample part file \\
    |cdocspt4.tex| & sample part file \\
    |cdocsdrf.tex| & sample redirection file \\
    |cdocsfn1.tex| & sample redirection file \\
    |cdocsfn2.tex| & sample redirection file \\
    |childdoc.pdf| & manual
\end{tabular}
\end{center}
%
The distribution consists of the files
|README.txt|, |childdoc.ins| and |childdoc.dtx|.
%
\begin{itemize}
\item
Run (pdf)\LaTeX{} on |childdoc.dtx|
to compile the manual |childdoc.pdf| (this file).
\item
Run \LaTeX{} on |childdoc.ins| to create the definitions file |childdoc.def|
and the sample |cdocsamp.tex| with include files
|cdocsch1.tex|, |cdocsch2.tex|, |cdocspt3.tex|, |cdocspt4.tex|,
|cdocsdrf.tex|, |cdocsfn1.tex|, |cdocsfn2.tex|.
Then copy the file |childdoc.def| to an appropriate directory of your \LaTeX{}
distribution, e.g.\ \textit{texmf-root}|/tex/latex/childdoc|.
\end{itemize}

%%%%%%%%%%%%%%%%%%%%%%%%%%%%%%%%%%%%%%%%%%%%%%%%%%%%%%%%%%%%%%%%%%%%%%%%%%%%%%%%
\subsection{Related CTAN Packages}

There are several other packages which offer a similar functionality:
%
\begin{itemize}
\item
The packages
\href{http://ctan.org/pkg/docmute}{\textsf{docmute}},
\href{http://ctan.org/pkg/includex}{\textsf{includex}} and
\href{http://ctan.org/pkg/standalone}{\textsf{standalone}}
provide commands to include only the document body of
a child file thus allowing both files to be compiled individually.
\item
The packages \href{http://ctan.org/pkg/subdocs}{\textsf{subdocs}}
and \href{http://ctan.org/pkg/subfiles}{\textsf{subfiles}}
provide structures in which the main and child documents can be
encapsulated and allowing them to be compiled individually.
The inclusion mechanism is different from the conventional |\include|.
\item
The package \href{http://ctan.org/pkg/combine}{\textsf{combine}}
is an elaborate solution to combine several documents into one.
\end{itemize}
%
See also the CTAN topic \href{http://ctan.org/topic/subdocs}{\textsf{subdocs}}
for further related packages.
The present package differs from the above solutions in that
a document structure constructed with the conventional |\include| mechanism
just needs two extra commands at the top of every file
such that all constituent files can be compiled individually.

%%%%%%%%%%%%%%%%%%%%%%%%%%%%%%%%%%%%%%%%%%%%%%%%%%%%%%%%%%%%%%%%%%%%%%%%%%%%%%%%
%\subsection{Feature Suggestions}
%
%The following is a list of features which may be useful for future
%versions of this package:
%%
%\begin{itemize}
%\item
%\ldots
%\end{itemize}

%%%%%%%%%%%%%%%%%%%%%%%%%%%%%%%%%%%%%%%%%%%%%%%%%%%%%%%%%%%%%%%%%%%%%%%%%%%%%%%%
\subsection{Revision History}

%%%%%%%%%%%%%%%%%%%%%%%%%%%%%%%%%%%%%%%%
\paragraph{v2.0:} 2018/12/30

\begin{itemize}
\item
immediate forward processing
\item
added |\childdocby| mechanism
\item
manual restructured
\end{itemize}

%%%%%%%%%%%%%%%%%%%%%%%%%%%%%%%%%%%%%%%%
\paragraph{v1.6:} 2018/01/17

\begin{itemize}
\item
application for development of include files
\item
corrections to manual
\end{itemize}

%%%%%%%%%%%%%%%%%%%%%%%%%%%%%%%%%%%%%%%%
\paragraph{v1.5:} 2017/05/21

\begin{itemize}
\item
more complete structuring introduced
\item
|\childdocof| introduced
\item
|\childdoc| renamed to |\childdocmain|
\item
|\childredirect| renamed to |\childdocforward| and |\childdocforwardprefix|
and functionality expanded
\end{itemize}

%%%%%%%%%%%%%%%%%%%%%%%%%%%%%%%%%%%%%%%%
\paragraph{v1.0:} 2017/04/27

\begin{itemize}
\item
manual and install package
\item
first version published on CTAN
\end{itemize}

%%%%%%%%%%%%%%%%%%%%%%%%%%%%%%%%%%%%%%%%
\paragraph{v0.6:} 2017/04/26

\begin{itemize}
\item
redirection mechanism added
\end{itemize}

%%%%%%%%%%%%%%%%%%%%%%%%%%%%%%%%%%%%%%%%
\paragraph{v0.5:} 2017/04/26

\begin{itemize}
\item
functionality in definition file
\end{itemize}


%%%%%%%%%%%%%%%%%%%%%%%%%%%%%%%%%%%%%%%%%%%%%%%%%%%%%%%%%%%%%%%%%%%%%%%%%%%%%%%%
%%%%%%%%%%%%%%%%%%%%%%%%%%%%%%%%%%%%%%%%%%%%%%%%%%%%%%%%%%%%%%%%%%%%%%%%%%%%%%%%
%%%%%%%%%%%%%%%%%%%%%%%%%%%%%%%%%%%%%%%%%%%%%%%%%%%%%%%%%%%%%%%%%%%%%%%%%%%%%%%%
\appendix

\settowidth\MacroIndent{\rmfamily\scriptsize 000\ }

 \DocInput{childdoc.dtx}

\end{document}
%</driver>
% \fi
%
% %%%%%%%%%%%%%%%%%%%%%%%%%%%%%%%%%%%%%%%%%%%%%%%%%%%%%%%%%%%%%%%%%%%%%%%%%%%%%%
% %%%%%%%%%%%%%%%%%%%%%%%%%%%%%%%%%%%%%%%%%%%%%%%%%%%%%%%%%%%%%%%%%%%%%%%%%%%%%%
% \section{Sample}
%\iffalse
%<*samplemain>
%\fi
%
% The following presents a sample document
% with two chapters, two parts, a title page,
% a compile flag as well as three forwarding files to set the flag.
% It consists of eight |.tex| files:
% \begin{center}
% \begin{tabular}{ll}
% |cdocsamp.tex|&main file\\
% |cdocsch1.tex|&include file for chapter 1\\
% |cdocsch2.tex|&include file for chapter 2\\
% |cdocspt3.tex|&include file for part 3\\
% |cdocspt4.tex|&include file for part 4\\
% |cdocsdrf.tex|&forwarding file for main file in draft mode\\
% |cdocsfi1.tex|&forwarding file for final version of chapter 1\\
% |cdocsfi2.tex|&forwarding file for final version of chapter 2\\
% \end{tabular}
% \end{center}
% Each of the eight files can be compiled directly by the \LaTeX{} compiler.
%
% %%%%%%%%%%%%%%%%%%%%%%%%%%%%%%%%%%%%%%
% \paragraph{Main File.}
%
% The main file is called |cdocsamp.tex|.
%
% Load the \textsf{childdoc} definitions and
% declare the filename for the main document:
%    \begin{macrocode}
\input{childdoc.def}
\childdocmain{}
%    \end{macrocode}

% Optional override for |\version| flag:
%    \begin{macrocode}
%%\ifchilddoc\else\providecommand{\version}{draft}\fi
%    \end{macrocode}

% Define the default values for the |\version| flag
% (|final| for the main file and |draft| for childs):
%    \begin{macrocode}
\ifchilddoc
\providecommand{\version}{draft}
\else
\providecommand{\version}{final}
\fi
%    \end{macrocode}

% Load the standard document class:
%    \begin{macrocode}
\documentclass[12pt]{article}
%    \end{macrocode}

% Start the document body:
%    \begin{macrocode}
\begin{document}
%    \end{macrocode}

% Declare a title page.
% Print title, part of document being processed and version flag:
%    \begin{macrocode}
\addtocounter{page}{-1}
\begin{center}
{\LARGE\bfseries{}childdoc example\par}
\vspace{1cm}
\ifchilddoc
\ifchilddocmanual part\else chapter\fi:
`\childdocname' of `\childdocjob'\par
\else
main document: `\childdocjob'\par
\fi
version: \version\par
\end{center}
\newpage
%    \end{macrocode}

% Manually include selected file,
% otherwise process as usual:
%    \begin{macrocode}
\ifchilddocmanual
\section*{part `\childdocname'}
\input{\childdocname}
\else
%    \end{macrocode}

% Include the two chapters:
%    \begin{macrocode}
\include{cdocsch1}
\include{cdocsch2}
%    \end{macrocode}

% Include the two parts unless only chapters should be displayed:
%    \begin{macrocode}
\ifchilddoc\else
\section{part three}
\input{cdocspt3}
\section{part four}
\input{cdocspt4}
\fi
%    \end{macrocode}

% Process as usual until here:
%    \begin{macrocode}
\fi
%    \end{macrocode}

% End of document body:
%    \begin{macrocode}
\end{document}
%    \end{macrocode}
%\iffalse
%</samplemain>
%\fi
%
% %%%%%%%%%%%%%%%%%%%%%%%%%%%%%%%%%%%%%%
% \paragraph{Chapter Include Files.}
%
% The include files are called |cdocsch1.tex| and |cdocsch2.tex|.
%
%\iffalse
%<*samplechap1|samplechap2>
%\fi

% Optional override for |\version| flag:
%    \begin{macrocode}
%%\providecommand{\version}{final}
%    \end{macrocode}

% Include the main document:
%    \begin{macrocode}
\input{childdoc.def}
\childdocof{cdocsamp}
%    \end{macrocode}

%\iffalse
%</samplechap1|samplechap2>
%\fi
%
%\iffalse
%<*samplechap1>
%\fi
% Some text for chapter 1:
%    \begin{macrocode}
\section{one}
some text in chapter one
%    \end{macrocode}

%\iffalse
%</samplechap1>
%\fi
% Some text for chapter 2:
%\iffalse
%<*samplechap2>
%\fi
%    \begin{macrocode}
\section{two}
more text in chapter two
%    \end{macrocode}

%\iffalse
%</samplechap2>
%\fi
%
% %%%%%%%%%%%%%%%%%%%%%%%%%%%%%%%%%%%%%%
% \paragraph{Part Include Files.}
%
% The include files are called |cdocspt3.tex| and |cdocspt4.tex|.
%
%\iffalse
%<*samplepart3|samplepart4>
%\fi

% Optional override for |\version| flag:
%    \begin{macrocode}
%%\providecommand{\version}{final}
%    \end{macrocode}

% Include the main document:
%    \begin{macrocode}
\input{childdoc.def}
\childdocby{cdocsamp}
%    \end{macrocode}

%\iffalse
%</samplepart3|samplepart4>
%\fi
%
%\iffalse
%<*samplepart3>
%\fi
% Some text for part 3:
%    \begin{macrocode}
some text in part three
%    \end{macrocode}

%\iffalse
%</samplepart3>
%\fi
% Some text for part 4:
%\iffalse
%<*samplepart4>
%\fi
%    \begin{macrocode}
more text in part four
%    \end{macrocode}

%\iffalse
%</samplepart4>
%\fi
%
% %%%%%%%%%%%%%%%%%%%%%%%%%%%%%%%%%%%%%%
% \paragraph{Forwarding for a Complete Draft.}
%
% The following forwarding file |cdocsdrf.tex|
% compiles the main document in draft mode:
%\iffalse
%<*sampledraft>
%\fi
%    \begin{macrocode}
\def\version{draft}
\input{childdoc.def}
\childdocforward{cdocsamp}
%    \end{macrocode}

%\iffalse
%</sampledraft>
%\fi
%
% %%%%%%%%%%%%%%%%%%%%%%%%%%%%%%%%%%%%%%
% \paragraph{Forwarding for Final Version of the Chapters.}
%
% The following forwarding files |cdocsfn1.tex| and |cdocsfn2.tex|
% (with identical content)
% compile the final versions of the child documents
% |cdocsch1.tex| and |cdocsch2.tex|, respectively:
%\iffalse
%<*samplefinal>
%\fi
%    \begin{macrocode}
\def\version{final}
\input{childdoc.def}
\childdocforwardprefix[cdocsamp]{cdocsfn}{cdocsch}
%    \end{macrocode}

%\iffalse
%</samplefinal>
%\fi
%
% %%%%%%%%%%%%%%%%%%%%%%%%%%%%%%%%%%%%%%
% \paragraph{Command Line Processing.}
%
% The following three command lines generate the output files
% |cdocscld|, |cdocscl1| and |cdocscl2|
% which should be identical to
% |cdocsdrf|, |cdocsch1| and |cdocsfn2|, respectively:
% \begin{center}
% \begin{tabular}{l}
% |latex -jobname cdocscld \|\\
% |  "\def\version{draft}\input{childdoc.def}\childdocforward{cdocsamp}"|\\
% |latex -jobname cdocscl1 \|\\
% |  "\input{childdoc.def}\childdocforward[cdocsamp]{cdocsch1}"|\\
% |latex -jobname cdocscl2 \|\\
% |  "\def\version{final}\input{childdoc.def}\childdocforward{cdocsch2}"|
% \end{tabular}
% \end{center}
% Note that the trailing backslash on each first line
% merely continues the input to the second line
% (for convenient cut ant paste).
% Furthermore, the command |latex| can be replaced by any
% of its alternative versions such as |pdflatex|.
%
% %%%%%%%%%%%%%%%%%%%%%%%%%%%%%%%%%%%%%%%%%%%%%%%%%%%%%%%%%%%%%%%%%%%%%%%%%%%%%%
% %%%%%%%%%%%%%%%%%%%%%%%%%%%%%%%%%%%%%%%%%%%%%%%%%%%%%%%%%%%%%%%%%%%%%%%%%%%%%%
% \section{Implementation}
%\iffalse
%<*package>
%\fi
%
% This section describes the definitions file |childdoc.def|.

% The definitions cannot be loaded using |\usepackage| or |\RequirePackage|
% which has a mechanism to prevent loading a style file more than once.
% When loading the definitions by means of |\input|
% multiple instances have to be prevented manually:
%\iffalse
%This code needs to be before the `\ProvidesFile' directive
%which is defined at the beginning of this file.
%Therefore it is also placed there and commented out here.
%</package>
%<*discard>
%\fi
%    \begin{macrocode}
\ifdefined\childdocmain\endinput\fi
%    \end{macrocode}
%\iffalse
%</discard>
%<*package>
%\fi
%
% \macro{\ifchilddoc}
% \macro{\ifchilddocmanual}
% The conditional |\ifchilddoc| tells whether a
% child (true) or main (false) document is being compiled.
% The conditional |\ifchilddocmanual| tells whether
% the |\includeonly| mechanism is used (false) or
% the selection of child files must be performed manually (true).
% The definitions initialise to false:
%    \begin{macrocode}
\newif\ifchilddoc
\newif\ifchilddocmanual
%    \end{macrocode}

% \macro{\childdocname}
% \macro{\childdocjob}
% The macro |\childdocname| stores the name of the main document
% to be compiled. The macro |\childdocjob| stores the name of
% the document on which the \LaTeX{} compiler was originally invoked.
% The content of |\jobname| cannot be compared
% to filenames specified in the source due to different catcodes.
% The following code rescans |\jobname|, stores the result
% in |\childdocname| and saves a copy in |\childdocjob|:
%    \begin{macrocode}
\edef\childdocname{\scantokens\expandafter{\jobname\noexpand}}
\let\childdocjob\childdocname
%    \end{macrocode}

% \macro{\childdocdisable}
% The macro |\childdocdisable| prevents the main file
% from being processed more than once.
% At this stage, the main document command |\childdocmain|
% is assumed to be called once again where it should do nothing.
% Any subsequent call to it should prevent
% a secondary processing of the main document
% It overwrites the forwarding commands
% |\childdocof| and |\childdocforward|
% with empty macros to prevent further inclusions of the main document:
%    \begin{macrocode}
\newcommand{\childdocdisable}
{
  \renewcommand{\childdocmain}[1]{\renewcommand{\childdocmain}[1]{\endinput}}
  \renewcommand{\childdocof}[1]{}
  \renewcommand{\childdocby}[2][]{}
  \renewcommand{\childdocforward}[2][]{}
  \renewcommand{\childdocdisable}{}
}
%    \end{macrocode}

% \macro{\childdocmain}
% The macro |\childdocmain| is to be called at the top of the main file
% with nothing or the main filename (without extension) as argument.
% First, it breaks loops.
% If the argument is not empty and does not match |\childdocname|
% (which is set by the first inclusion of |childdoc.def|),
% |\ifchilddoc| is set to true, |\includeonly| is applied to the child file
% and |\jobname| is set to the main file
% (for proper handling of |.aux| files):
%    \begin{macrocode}
\newcommand{\childdocmain}[1]
{
  \childdocdisable\childdocmain{}
  \if?#1?\else
    \begingroup
      \def\childdoctmp{#1}
      \ifx\childdoctmp\childdocname
        \def\childdoctmp{}
      \else
        \def\childdoctmp
        {
          \childdoctrue
          \includeonly{\childdocname}
          \def\childdocjob{#1}
          \def\jobname{#1}
        }
      \fi
      \expandafter
    \endgroup
    \childdoctmp
  \fi
}
%    \end{macrocode}

% \macro{\childdocof}
% The command |\childdocof| redirects
% compilation to the main file |#1|.
%    \begin{macrocode}
\newcommand{\childdocof}[1]
{
  \childdocdisable
  \childdoctrue
  \includeonly{\childdocname}
  \def\jobname{#1}
  \def\childdocjob{#1}
  \input{#1}
}
%    \end{macrocode}

% \macro{\childdocby}
% The command |\childdocby| ....
%    \begin{macrocode}
\newcommand{\childdocby}[2][]
{
  \childdocdisable
  \childdoctrue
  \childdocmanualtrue
  \if?#1?\else
    \def\jobname{#2}
  \fi
  \def\childdocjob{#2}
  \input{#2}
  \endinput
}
%    \end{macrocode}

% \macro{\childdocforward}
% The command |\childdocforward| redirects
% compilation to the main file or
% (if the optional argument is given) a child file.
% Parameters are set as if the main file
% or a child file starting with |\childdocof| was compiled.
% Then compilation is handed over to the main file:
%    \begin{macrocode}
\newcommand{\childdocforward}[2][]
{
  \begingroup
    \if?#1?
      \def\childdoctmp
      {
        \def\childdocname{#2}
        \def\childdocjob{#2}
        \def\jobname{#2}
        \input{#2}
        \endinput
      }
    \else
      \def\childdoctmp
      {
        \childdocdisable
        \def\childdocname{#2}
        \childdoctrue
        \includeonly{#2}
        \def\childdocjob{#1}
        \def\jobname{#1}
        \input{#1}
        \endinput
      }
    \fi
    \expandafter
  \endgroup
  \childdoctmp
}
%    \end{macrocode}

% \macro{\childdocforwardprefix}
% The command |\childdocforwardprefix| redirects
% compilation to the main or a child file by means of a pattern.
% The prefix |#1| in the current filename is replaced by |#2|
% and the suffix of the current filename is kept
% (it is assumed that the filename does not contain the substring `|~~~|'
% which is used as a delimiter).
% Compilation is handed over to the new file by |\childdocforward|:
%    \begin{macrocode}
\newcommand{\childdocforwardprefix}[3][]
{
  \begingroup
    \def\childdocextract #2##1~~~{\def\childdoctmp{\childdocforward[#1]{#3##1}}}
    \expandafter\childdocextract\childdocname~~~
    \expandafter
  \endgroup
  \childdoctmp
}
%    \end{macrocode}

% \macro{\childdoc}
% The deprecated macro |\childdoc| is a legacy version of |\childdocmain|:
%    \begin{macrocode}
\newcommand{\childdoc}{\childdocmain}
%    \end{macrocode}

% \macro{\childdocredirect}
% The deprecated macro |\childdocredirect| is a legacy version
% of |\childdocforward| and |\childdocforwardprefix|:
%    \begin{macrocode}
\newcommand{\childdocredirect}[2][]
{
  \begingroup
    \if?#1?
      \def\childdoctmp{\childdocforward{#2}}
    \else
      \def\childdoctmp{\childdocforwardprefix{#1}{#2}}
    \fi
    \expandafter
  \endgroup
  \childdoctmp
}
%    \end{macrocode}

%\iffalse
%</package>
%\fi
%
\endinput
|\\
|\childdocby{|\textit{main}|}|\\
\end{tabular}
\end{center}
%
The directive |\childdocby| is similar to |\childdocof|
described in \secref{sec:include},
but the subsequent selection of content must be done manually.
To that end, both |\ifchilddoc| and |\ifchilddocmanual|
will be true upon processing of a part,
and the name of the part is stored in |\childdocname|.
Note that |\jobname| will be set to the filename of the current part
so that each part receives an individual |.aux| file
that does not interfere with the |.aux| file(s) of the main document.
This behaviour can be altered by the alternative form
|\childdocby[*]{|\textit{main}|}| (with a non-empty optional argument)
which uses the |.aux| file of the main document
by setting |\jobname| to \textit{main}.

%%%%%%%%%%%%%%%%%%%%%%%%%%%%%%%%%%%%%%%%%%%%%%%%%%%%%%%%%%%%%%%%%%%%%%%%%%%%%%%%
\subsection{Driver Development}
\label{sec:driver}

The \textsf{childdoc} mechanism can also be use for the development
of definition files such as \LaTeX{} styles or classes.
This case differs from the above setup with multiple parts
included by |\include| in that no |\includeonly| should be invoked.
This can be achieved by starting the include file
(before |\ProvidesPackage|) with:
%
\begin{center}
\begin{tabular}{l}
|% \iffalse
%
% childdoc.dtx Copyright (C) 2017-2018 Niklas Beisert
%
% This work may be distributed and/or modified under the
% conditions of the LaTeX Project Public License, either version 1.3
% of this license or (at your option) any later version.
% The latest version of this license is in
%   http://www.latex-project.org/lppl.txt
% and version 1.3 or later is part of all distributions of LaTeX
% version 2005/12/01 or later.
%
% This work has the LPPL maintenance status `maintained'.
%
% The Current Maintainer of this work is Niklas Beisert.
%
% This work consists of the files childdoc.dtx and childdoc.ins
% and the derived files childdoc.def and cdocsamp.tex with
% cdocsch1.tex, cdocsch2.tex, cdocsdrf.tex, cdocsfn1.tex, cdocsfn2.tex.
%
%<package>\ifdefined\childdocmain\endinput\fi
%<package>\ProvidesFile{childdoc.def}[2018/12/30 v2.0 child document driver]
%<samplemain>\ProvidesFile{cdocsamp.tex}[2018/12/30 v2.0 sample for childdoc]
%<*driver>
%\ProvidesFile{childdoc.drv}[2018/12/30 v2.0 childdoc reference manual file]
\PassOptionsToClass{10pt,a4paper}{article}
\documentclass{ltxdoc}

\usepackage[margin=35mm]{geometry}
\usepackage{hyperref}
\usepackage{hyperxmp}
\usepackage[usenames]{color}

\hypersetup{colorlinks=true}
\hypersetup{pdfstartview=FitH}
\hypersetup{pdfpagemode=UseNone}
\hypersetup{pdfsource={}}
\hypersetup{pdflang={en-UK}}
\hypersetup{pdfcopyright={Copyright 2017-2018 Niklas Beisert.
  This work may be distributed and/or modified under the
  conditions of the LaTeX Project Public License, either version 1.3
  of this license or (at your option) any later version.}}
\hypersetup{pdflicenseurl={http://www.latex-project.org/lppl.txt}}
\hypersetup{pdfcontactaddress={ETH Zurich, ITP, HIT K,
  Wolfgang-Pauli-Strasse 27}}
\hypersetup{pdfcontactpostcode={8093}}
\hypersetup{pdfcontactcity={Zurich}}
\hypersetup{pdfcontactcountry={Switzerland}}
\hypersetup{pdfcontactemail={nbeisert@itp.phys.ethz.ch}}
\hypersetup{pdfcontacturl={http://people.phys.ethz.ch/\xmptilde nbeisert/}}

\newcommand{\secref}[1]{\hyperref[#1]{section \ref*{#1}}}

\parskip1ex
\parindent0pt
\let\olditemize\itemize
\def\itemize{\olditemize\parskip0pt}

\begin{document}

\title{The \textsf{childdoc} Package}
\hypersetup{pdftitle={The childdoc Package}}
\author{Niklas Beisert\\[2ex]
  Institut f\"ur Theoretische Physik\\
  Eidgen\"ossische Technische Hochschule Z\"urich\\
  Wolfgang-Pauli-Strasse 27, 8093 Z\"urich, Switzerland\\[1ex]
  \href{mailto:nbeisert@itp.phys.ethz.ch}
  {\texttt{nbeisert@itp.phys.ethz.ch}}}
\hypersetup{pdfauthor={Niklas Beisert}}
\hypersetup{pdfsubject={Manual for the LaTeX2e Package childdoc}}
\date{30 December 2018, \textsf{v2.0}}
\maketitle

\begin{abstract}\noindent
\textsf{childdoc} is a \LaTeXe{} package
that enables the direct compilation
of document sections included by |\include|
to individual files.
\end{abstract}

\begingroup
\parskip0ex
\tableofcontents
\endgroup

%%%%%%%%%%%%%%%%%%%%%%%%%%%%%%%%%%%%%%%%%%%%%%%%%%%%%%%%%%%%%%%%%%%%%%%%%%%%%%%%
%%%%%%%%%%%%%%%%%%%%%%%%%%%%%%%%%%%%%%%%%%%%%%%%%%%%%%%%%%%%%%%%%%%%%%%%%%%%%%%%
\section{Introduction}

\LaTeX{} provides a mechanism to structure a large document (such as a book)
into a main file and several child files (containing the chapters)
using the |\include| command.
This mechanism is beneficial for documents
which span hundreds of pages in order to
make the source file(s) more manageable.
Moreover, compilation can be restricted to
selected child files by means of the |\includeonly| command.
The latter feature can be used to reduce the compilation time while editing
(this was significantly more useful in the earlier days of \LaTeX{})
or to generate a smaller document which is easier to navigate.
Another application of |\includeonly| is to generate
documents consisting of selected parts of the complete document.

However, there are a few drawbacks of the plain |\include| mechanism:
\begin{itemize}
\item
The child files cannot be compiled on their own,
they can only be compiled via the main file.
A naive editing environment
(such as a text editor with an option
to have the current file processed by \LaTeX)
may require one to switch to the main file before compiling;
attempting to compile the child file produces errors.
\item
The main file must be modified (each time)
to adjust the |\includeonly| command
to the present needs. This easily leaves the main file in a messy state.
\item
The generated document will always carry the filename
of the main document. This is inconvenient if
several child files are to be compiled and
to be kept for distribution.
\end{itemize}

The present package provides a simple interface
to make child files individually compilable by \LaTeX{}.
Compiling a child file then has the same effect as compiling
the main file with an |\includeonly| command
to select the appropriate child.
Moreover the generated document will carry the name of the child
rather than the main file.
This resolves all three above issues.

This feature is meant to make the editing of books,
thesis documents and lecture notes somewhat more convenient.
However, the package can also be used efficiently for
composing a series of documents (such as exercise sheets)
which are typically distributed individually.
It then assists the author in generating the individual documents
(potentially in different versions)
as well as a document containing the collected series.
Another application is in developing style files
or other kinds of included material
where compilation of the style file could redirect
to a sample or test file.

%%%%%%%%%%%%%%%%%%%%%%%%%%%%%%%%%%%%%%%%%%%%%%%%%%%%%%%%%%%%%%%%%%%%%%%%%%%%%%%%
%%%%%%%%%%%%%%%%%%%%%%%%%%%%%%%%%%%%%%%%%%%%%%%%%%%%%%%%%%%%%%%%%%%%%%%%%%%%%%%%
\section{Usage}

First of all, the package \textsf{childdoc} is \emph{not} a standard
\LaTeXe{} |.sty| style file! Therefore it needs to be invoked in
a non-standard way.

%%%%%%%%%%%%%%%%%%%%%%%%%%%%%%%%%%%%%%%%%%%%%%%%%%%%%%%%%%%%%%%%%%%%%%%%%%%%%%%%
\subsection{Included Files}
\label{sec:include}

%%%%%%%%%%%%%%%%%%%%%%%%%%%%%%%%%%%%%%%%
\DescribeMacro{\childdocmain}
To use the package, add the commands
\begin{center}
\begin{tabular}{l}
|\input{childdoc.def}|\\
|\childdocmain{}|\\
\end{tabular}
\end{center}
at the very top of the main \LaTeX{} file,
in particular \emph{before} the |\documentclass| statement!
The argument of |\childdocmain| should be left empty
(but it must be present).

%%%%%%%%%%%%%%%%%%%%%%%%%%%%%%%%%%%%%%%%
\DescribeMacro{\childdocof}
Furthermore, add the commands
\begin{center}
\begin{tabular}{l}
|\input{childdoc.def}|\\
|\childdocof{|\textit{main}|}|\\
\end{tabular}
\end{center}
at the top of every child file \textit{child}
which is included by |\include{|\textit{child}|}|
from within the main file
(or at least for those files to be compiled individually).
The argument \textit{main} must be the filename of the main file.

There are a couple of
considerations in setting up the main and child documents:

%%%%%%%%%%%%%%%%%%%%%%%%%%%%%%%%%%%%%%%%
\paragraph{Restrictions.}

Please note the following restrictions:
\begin{itemize}
\item
|\childdocmain| must be called with one argument \textit{main}
to ensure compatibility with earlier version of the package.
It must either be empty (|\childdocmain{}|)
or precisely match the filename of the main file in which it is specified.
See \secref{sec:detection} for further information.
\item
The filename \textit{main} must be specified without the |.tex| extension.
\item
The filename \textit{main} is case sensitive
(even in case-insensitive file systems)
due to internal string comparison.
\item
The argument \textit{main} should be fully expanded, it cannot be a macro.
\item
Subdirectories and special characters should be avoided in filenames.
\item
The command |\childdocmain{|\textit{main}|}| must be followed by a whitespace.
It should not be followed immediately by another command
or by a comment mark `|%|'.
This is because the \TeX{} parser reads the token immediately following
the argument of |\childdocmain| and puts it
at the beginning of every child section;
however, a white\-space is ignored.
\end{itemize}

%%%%%%%%%%%%%%%%%%%%%%%%%%%%%%%%%%%%%%%%
\paragraph{Content of Main File.}

It is advisable to place all content in the child files included by |\include|.
Any output contained in the main file will appear in all child documents
unless suppressed manually;
it cannot be suppressed automatically by the |\includeonly| directive
and thus should normally be avoided.
A method to include some content in the main file
by means of conditional processing is described in \secref{sec:conditional}.

%%%%%%%%%%%%%%%%%%%%%%%%%%%%%%%%%%%%%%%%
\paragraph{Page Numbering.}

When only a part of the document is compiled,
the appropriate numbering of pages
(as well as other status parameters)
is determined from the |.aux| files.
The latter contain information from previous passes.
However this information needs to propagate through
all intermediate child documents.
Therefore the page numbering in child documents may well
be inconsistent until the complete document is compiled at least once.

A useful (if unconventional) way to always ensure a consistent
page numbering is to restart the numbering in each child document
and denote the pages by `\textit{child}|.|\textit{page}'
where \textit{child} represents the chapter/section number of the child file.
This can be achieved by the command
|\numberwithin{page}{|\textit{child}|}|
of the \textsf{amsmath} package
where \textit{child} can be |chapter| or |section|
depending on the chosen structuring.
Alternatively, one can modify the macro |\thepage| appropriately
and reset the counter |page| at the start of each child file.

%%%%%%%%%%%%%%%%%%%%%%%%%%%%%%%%%%%%%%%%%%%%%%%%%%%%%%%%%%%%%%%%%%%%%%%%%%%%%%%%
\subsection{Conditional Processing}
\label{sec:conditional}

The package provides a mechanism to compile different versions
of a document. To customise the versions further some conditional processing
can come in handy to distinguish which version is being compiled.
The package provides two macros to describe the compilation context:

%%%%%%%%%%%%%%%%%%%%%%%%%%%%%%%%%%%%%%%%
\DescribeMacro{\ifchilddoc}
The conditional |\ifchilddoc| distinguishes between the compilation of
child documents and the main document:
%
\begin{center}
|\ifchilddoc |\textit{child-code}| |[|\||else |\textit{main-code}]| \||fi|
\end{center}

%%%%%%%%%%%%%%%%%%%%%%%%%%%%%%%%%%%%%%%%
\DescribeMacro{\childdocname}
\DescribeMacro{\childdocjob}
The macro |\childdocname| contains the filename (without extension)
of the main or child file being processed.
Note that |\childdocjob| will always contain the name of the main file.

%%%%%%%%%%%%%%%%%%%%%%%%%%%%%%%%%%%%%%%%
\paragraph{Title Page.}

Conditional processing can be used to include a title or banner page
in the main document when proper precautions are taken.
Importantly, the code in the main file should ensure that the page counter
(as well as other status parameters which are stored in the |.aux| files)
takes the same value after the conditional processing.
Otherwise the page numbers may take divergent values
depending on which part is compiled.

For example, a title page could be declared by:
%
\begin{center}
\begin{tabular}{l}
|\ifchilddoc\||else|\\
|\addtocounter{page}{-1}|\\
\textit{code for title page}\\
|\newpage|\\
|\||fi|
\end{tabular}
\end{center}
%
A banner page for the child documents can be generated by:
%
\begin{center}
\begin{tabular}{l}
|\ifchilddoc|\\
|\addtocounter{page}{-1}|\\
\textit{code for banner page}\\
|\newpage|\\
|\||fi|
\end{tabular}
\end{center}
%
Here one could write a message such as:
\begin{center}
|This is the part \childdocname{} of \childdocjob{}.|
\end{center}

%%%%%%%%%%%%%%%%%%%%%%%%%%%%%%%%%%%%%%%%%%%%%%%%%%%%%%%%%%%%%%%%%%%%%%%%%%%%%%%%
\subsection{Flags}
\label{sec:flags}

The package makes it easy to generate different versions
of the main or child documents.
To this end compilation flags can be defined
and assigned different default values.
They will be particularly useful in conjunction
with the forwarding mechanism described in \secref{sec:forward}.

For example, it may be useful to have a flag |\version|
which can be set to |draft| or |final|.
The document source will contain some conditional code
depending on the value of |\version|.
Suppose further, the flag should default to |final| for the main file
and to |draft| for child files
which is a natural assignment for editing the document.
This is achieved by placing the following code
in the preamble of the main document
(below the |\childdocmain| directive):
%
\begin{center}
\begin{tabular}{l}
|\ifchilddoc|\\
|\providecommand{\version}{draft}|\\
|\||else|\\
|\providecommand{\version}{final}|\\
|\||fi|
\end{tabular}
\end{center}
%
The definition by |\providecommand| makes sure
that previous definitions are not overwritten.
Further statements |\providecommand{\version}{...}|
can thus be added before the above code to override it.

For the main file, one might add a line
(between |\childdocmain| and the above block)
%
\begin{center}
|%\ifchilddoc\||else\providecommand{\version}{draft}\||fi|
\end{center}
%
which can be uncommented to produce a draft version.
Likewise one can add a line to the very top of a child file
(above the |\childdocof{|\textit{main}|}| directive)
%
\begin{center}
|%\providecommand{\version}{final}|
\end{center}
%
which can be uncommented to produce the final version of this child document.

%%%%%%%%%%%%%%%%%%%%%%%%%%%%%%%%%%%%%%%%%%%%%%%%%%%%%%%%%%%%%%%%%%%%%%%%%%%%%%%%
\subsection{Forwarding}
\label{sec:forward}

Different versions of the main or child documents
using compilation flags as described in \secref{sec:flags}
can be (permanently) stored in different files
for convenient compilation, viewing and distribution.
To this end, the package defines a command
to pass on compilation to a different file:

%%%%%%%%%%%%%%%%%%%%%%%%%%%%%%%%%%%%%%%%
\DescribeMacro{\childdocforward}
The command |\childdocforward| redirects processing to
another source file:
%
\begin{center}
\begin{tabular}{l}
|\input{childdoc.def}|\\
|\childdocforward[|\textit{main}|]{|\textit{dest}|}|\\
\end{tabular}
\end{center}
%
The argument \textit{dest} is the destination file
(without extension).
It should be the main file or one of the child files.
Note that further \textsf{childdoc} directives
such as |\childdocof| and |\childdocforward|
in the indicated file will be processed in this form.
The optional argument \textit{main}
passes on directly to the main file \textit{main}
while pretending to compile the child \textit{dest}.
This form behaves as if \textit{dest}
issues |\childdocof{|\textit{main}|}| right away,
and no further \textsf{childdoc} directives will be processed.

%%%%%%%%%%%%%%%%%%%%%%%%%%%%%%%%%%%%%%%%
\DescribeMacro{\...prefix}
In the alternative form |\childdocforwardprefix|,
%
\begin{center}
\begin{tabular}{l}
|\input{childdoc.def}|\\
|\childdocforwardprefix[|\textit{main}|]{|\textit{prefix}|}{|\textit{dest}|}|
\end{tabular}
\end{center}
%
the destination file is determined by a pattern
depending on the current file:
To make this work, the current file must be called
`{\textit{prefix}\hspace{0.2em}\textit{suffix}}'
with \textit{prefix} matching precisely the argument.
Processing is then passed on to the file
`{\textit{dest}\hspace{0.2em}\textit{suffix}}'.
Surely, the same effect is achieved by
directly specifying the
argument `{\textit{dest}\hspace{0.2em}\textit{suffix}}'
in the first form.
However, that requires to set up a different file
for each child. With the alternative form of the command
all these files can have exactly the same content
which simplifies setting them up and maintaining them.

For example, the following file |draft.tex|
with a compilation flag |\version| as described in \secref{sec:flags}
compiles the main document as a draft:
%
\begin{center}
\begin{tabular}{l}
|\def\version{draft}|\\
|\input{childdoc.def}|\\
|\childdocforward{|\textit{main}|}|
\end{tabular}
\end{center}
%
Likewise, the following files |final|\textit{nn}|.tex|
compile the final version of the child document
|child|\textit{nn}|.tex|:
%
\begin{center}
\begin{tabular}{l}
|\def\version{final}|\\
|\input{childdoc.def}|\\
|\childdocforwardprefix{final}{child}|
\end{tabular}
\end{center}
%

Note that when several versions of a main file and/or of each child file
are to be generated, it may be convenient to set up a |Makefile| or
shell script to automatise the process.

%%%%%%%%%%%%%%%%%%%%%%%%%%%%%%%%%%%%%%%%%%%%%%%%%%%%%%%%%%%%%%%%%%%%%%%%%%%%%%%%
\subsection{Command Line Processing}
\label{sec:commandline}

The effect of redirection files can also be achieved by invoking
the \LaTeX{} compiler with a more elaborate command line.
Most conveniently this should be done as part
of a shell script or a |Makefile|.

When using \textsf{childdoc} in the main file, the following
command lines effectively perform a redirection
(note that depending on the shell being used,
backslashes may have to be doubled: `|\|' $\to$ `|\\|'):
%
\begin{center}
|... -jobname "|\textit{target}|" |\\|"|[\textit{flags}]%
|\input{childdoc.def}\childdocforward[|\textit{main}|]{|\textit{dest}|}"|
\end{center}
%
Here \textit{target} is the name of the output file,
\textit{main} is the name of the main file
and \textit{dest} is the name of the main or child file to be processed
(all filenames without extensions).
The optional argument \textit{main} can be omitted
if \textit{main} matches \textit{dest}.
Optionally, compilation \textit{flags} can be defined via |\def| commands.
This command line makes the \TeX{} engine believe
it is compiling the file \textit{target}
whose content is specified as the latter parameter.
The provided code then forwards the processing to
\textit{main} or \textit{dest} as described in \secref{sec:forward}.

%%%%%%%%%%%%%%%%%%%%%%%%%%%%%%%%%%%%%%%%%%%%%%%%%%%%%%%%%%%%%%%%%%%%%%%%%%%%%%%%
\subsection{Include by Input}
\label{sec:input}

Including child documents by |\include| has some restrictions by design.
Most notably, the content of a child document always occupies
its own set of pages; pages cannot be shared between child documents.
Usually, this behaviour makes perfect sense
because each child document contain an essential part of the document.
However, in some situations it may be desirable to compose
a document from a collection of parts
without having mandatory page breaks between then.
For this case, the package
provides a mechanism to include parts
by |\input| which can also be processed individually.
However, by construction this mechanism
requires manual handling of the content to be output.

%%%%%%%%%%%%%%%%%%%%%%%%%%%%%%%%%%%%%%%%
\DescribeMacro{\ifchilddocmanual}
The main file should be prepared as usual, see \secref{sec:include}.
However, the document body must make a distinction
between processing of an individual part and of the main document, e.g.:
%
\begin{center}
\begin{tabular}{l}
|\ifchilddocmanual|\\
|\input{\childdocname}|\\
|\||else|\\
\textit{document body with }|\input{|\textit{part}|}|\\
|\||fi|
\end{tabular}
\end{center}
%
The conditional |\ifchilddocmanual| is true whenever
a part to be included by |\input| is being compiled,
and the name of the part is stored in |\childdocname|.

%%%%%%%%%%%%%%%%%%%%%%%%%%%%%%%%%%%%%%%%
\DescribeMacro{\childdocby}
Each part to be included by |\input| should start with:
%
\begin{center}
\begin{tabular}{l}
|\input{childdoc.def}|\\
|\childdocby{|\textit{main}|}|\\
\end{tabular}
\end{center}
%
The directive |\childdocby| is similar to |\childdocof|
described in \secref{sec:include},
but the subsequent selection of content must be done manually.
To that end, both |\ifchilddoc| and |\ifchilddocmanual|
will be true upon processing of a part,
and the name of the part is stored in |\childdocname|.
Note that |\jobname| will be set to the filename of the current part
so that each part receives an individual |.aux| file
that does not interfere with the |.aux| file(s) of the main document.
This behaviour can be altered by the alternative form
|\childdocby[*]{|\textit{main}|}| (with a non-empty optional argument)
which uses the |.aux| file of the main document
by setting |\jobname| to \textit{main}.

%%%%%%%%%%%%%%%%%%%%%%%%%%%%%%%%%%%%%%%%%%%%%%%%%%%%%%%%%%%%%%%%%%%%%%%%%%%%%%%%
\subsection{Driver Development}
\label{sec:driver}

The \textsf{childdoc} mechanism can also be use for the development
of definition files such as \LaTeX{} styles or classes.
This case differs from the above setup with multiple parts
included by |\include| in that no |\includeonly| should be invoked.
This can be achieved by starting the include file
(before |\ProvidesPackage|) with:
%
\begin{center}
\begin{tabular}{l}
|\input{childdoc.def}|\\
|\childdocforward{|\textit{main}|}|\\
\end{tabular}
\end{center}
%
or alternatively with:
%
\begin{center}
\begin{tabular}{l}
|\input{childdoc.def}|\\
|\childdocby{|\textit{main}|}|\\
\end{tabular}
\end{center}
%
Both forms have slightly different effects as described above.
The main file is prepared as usual, see \secref{sec:include}.

%%%%%%%%%%%%%%%%%%%%%%%%%%%%%%%%%%%%%%%%%%%%%%%%%%%%%%%%%%%%%%%%%%%%%%%%%%%%%%%%
\subsection{Legacy Detection}
\label{sec:detection}

The directive |\childdocmain| in the main file can detect
whether the complete document or merely a child is to be compiled
even without using the directive |\childdocof|.
This method is deprecated because it is less robust
and there is no compelling reason to use it;
it is merely provided for backward compatibility
and it may be removed in future versions.

If the detection mechanism is to be used,
it is mandatory to correctly specify
the filename of the main file as the argument of |\childdocmain|:
%
\begin{center}
\begin{tabular}{l}
|\input{childdoc.def}|\\
|\childdocmain{|\textit{main}|}|\\
\end{tabular}
\end{center}
%
If |\jobname| does not match the argument \textit{main} of |\childdocmain|,
it is assumed that |\jobname| points to the child file to be compiled.
When using |\childdocmain| with the main file specified as argument,
it suffices to start a child file
with just |\input{|\textit{main}|}|
without loading of the package and using |\childdocof|.
If instead all processing is done
with the appropriate \textsf{childdoc} directives,
the argument of \textit{main} of |\childdocmain| can be empty.

An alternative version of the command line processing described
in \secref{sec:commandline} using the detection mechanism reads:
%
\begin{center}
|... -jobname "|\textit{target}|" "|[\textit{flags}]%
[|\def\jobname{|\textit{dest}|}|]|\input{|\textit{main}|}"|
\end{center}

%%%%%%%%%%%%%%%%%%%%%%%%%%%%%%%%%%%%%%%%%%%%%%%%%%%%%%%%%%%%%%%%%%%%%%%%%%%%%%%%
\subsection{Manual Code}
\label{sec:manual}

In case one cannot be certain whether the definitions file |childdoc.def|
is installed on the target \TeX{} distribution
and one prefers not to ship it,
it is conceivable to paste a few relevant commands into the sources.

To that end, drop all statements |\input{childdoc.def}|
and perform the replacements as outlined below.
Instead of |\childdocmain{|\textit{main}|}| add the following code
to the top of the main file:
%
\begin{center}
\begin{tabular}{l}
|\||ifdefined\childdocname\endinput\||fi\newif\ifchilddoc|\\
|\edef\childdocname{\scantokens\expandafter{\jobname\noexpand}}|\\
|\def\childdocmain{|\textit{main}|}\||ifx\childdocmain\childdocname\||else|\\
|\childdoctrue\includeonly{\childdocname}\let\jobname\childdocmain\||fi|\\
\end{tabular}
\end{center}
%
Instead of |\childdocof{|\textit{main}|}| just include the main file
at the top of each child file:
%
\begin{center}
|\input{|\textit{main}|}|
\end{center}
%
A simple redirection |\childdocforward{|\textit{dest}|}| is achieved by:
%
\begin{center}
|\def\jobname{|\textit{dest}|}\input{\jobname}|
\end{center}
%
The redirection with prefix
|\childdocforwardprefix[|\textit{prefix}|]{|\textit{dest}|}|
is accomplished by:
%
\begin{center}
\begin{tabular}{l}
|{\edef\jobname{\scantokens\expandafter{\jobname\noexpand}}|\\
|\def\redirectjob |\textit{prefix}|#1~~~{\gdef\jobname{|\textit{dest}|#1}}|\\
|\expandafter\redirectjob\jobname~~~}\input{\jobname}|
\end{tabular}
\end{center}

In an alternative approach,
child documents can be compiled by a specific command line
without additional code or specific definitions:
%
\begin{center}
|... -jobname "|\textit{target}|" "|[\textit{flags}]%
|\includeonly{|\textit{dest}|}\input{|\textit{main}|}"|
\end{center}
%

%%%%%%%%%%%%%%%%%%%%%%%%%%%%%%%%%%%%%%%%%%%%%%%%%%%%%%%%%%%%%%%%%%%%%%%%%%%%%%%%
%%%%%%%%%%%%%%%%%%%%%%%%%%%%%%%%%%%%%%%%%%%%%%%%%%%%%%%%%%%%%%%%%%%%%%%%%%%%%%%%
\section{Information}

%%%%%%%%%%%%%%%%%%%%%%%%%%%%%%%%%%%%%%%%%%%%%%%%%%%%%%%%%%%%%%%%%%%%%%%%%%%%%%%%
\subsection{Copyright}

Copyright \copyright{} 2017--2018 Niklas Beisert

This work may be distributed and/or modified under the
conditions of the \LaTeX{} Project Public License, either version 1.3
of this license or (at your option) any later version.
The latest version of this license is in
  \url{http://www.latex-project.org/lppl.txt}
and version 1.3 or later is part of all distributions of \LaTeX{}
version 2005/12/01 or later.

This work has the LPPL maintenance status `maintained'.

The Current Maintainer of this work is Niklas Beisert.

This work consists of the files |README.txt|, |childdoc.ins| and |childdoc.dtx|
as well as the derived files |childdoc.def|, |cdocsamp.tex|
with |cdocsch1.tex|, |cdocsch2.tex|, |cdocspt3.tex|, |cdocspt4.tex|,
|cdocsdrf.tex|, |cdocsfn1.tex|, |cdocsfn2.tex|
as well as |childdoc.pdf|.

%%%%%%%%%%%%%%%%%%%%%%%%%%%%%%%%%%%%%%%%%%%%%%%%%%%%%%%%%%%%%%%%%%%%%%%%%%%%%%%%
\subsection{Files and Installation}

The package consists of the files:
%
\begin{center}
\begin{tabular}{ll}
    |README.txt|   & readme file \\
    |childdoc.ins| & installation file \\
    |childdoc.dtx| & source file \\
    |childdoc.def| & definition file \\
    |cdocsamp.tex| & sample main file \\
    |cdocsch1.tex| & sample include file \\
    |cdocsch2.tex| & sample include file \\
    |cdocspt3.tex| & sample part file \\
    |cdocspt4.tex| & sample part file \\
    |cdocsdrf.tex| & sample redirection file \\
    |cdocsfn1.tex| & sample redirection file \\
    |cdocsfn2.tex| & sample redirection file \\
    |childdoc.pdf| & manual
\end{tabular}
\end{center}
%
The distribution consists of the files
|README.txt|, |childdoc.ins| and |childdoc.dtx|.
%
\begin{itemize}
\item
Run (pdf)\LaTeX{} on |childdoc.dtx|
to compile the manual |childdoc.pdf| (this file).
\item
Run \LaTeX{} on |childdoc.ins| to create the definitions file |childdoc.def|
and the sample |cdocsamp.tex| with include files
|cdocsch1.tex|, |cdocsch2.tex|, |cdocspt3.tex|, |cdocspt4.tex|,
|cdocsdrf.tex|, |cdocsfn1.tex|, |cdocsfn2.tex|.
Then copy the file |childdoc.def| to an appropriate directory of your \LaTeX{}
distribution, e.g.\ \textit{texmf-root}|/tex/latex/childdoc|.
\end{itemize}

%%%%%%%%%%%%%%%%%%%%%%%%%%%%%%%%%%%%%%%%%%%%%%%%%%%%%%%%%%%%%%%%%%%%%%%%%%%%%%%%
\subsection{Related CTAN Packages}

There are several other packages which offer a similar functionality:
%
\begin{itemize}
\item
The packages
\href{http://ctan.org/pkg/docmute}{\textsf{docmute}},
\href{http://ctan.org/pkg/includex}{\textsf{includex}} and
\href{http://ctan.org/pkg/standalone}{\textsf{standalone}}
provide commands to include only the document body of
a child file thus allowing both files to be compiled individually.
\item
The packages \href{http://ctan.org/pkg/subdocs}{\textsf{subdocs}}
and \href{http://ctan.org/pkg/subfiles}{\textsf{subfiles}}
provide structures in which the main and child documents can be
encapsulated and allowing them to be compiled individually.
The inclusion mechanism is different from the conventional |\include|.
\item
The package \href{http://ctan.org/pkg/combine}{\textsf{combine}}
is an elaborate solution to combine several documents into one.
\end{itemize}
%
See also the CTAN topic \href{http://ctan.org/topic/subdocs}{\textsf{subdocs}}
for further related packages.
The present package differs from the above solutions in that
a document structure constructed with the conventional |\include| mechanism
just needs two extra commands at the top of every file
such that all constituent files can be compiled individually.

%%%%%%%%%%%%%%%%%%%%%%%%%%%%%%%%%%%%%%%%%%%%%%%%%%%%%%%%%%%%%%%%%%%%%%%%%%%%%%%%
%\subsection{Feature Suggestions}
%
%The following is a list of features which may be useful for future
%versions of this package:
%%
%\begin{itemize}
%\item
%\ldots
%\end{itemize}

%%%%%%%%%%%%%%%%%%%%%%%%%%%%%%%%%%%%%%%%%%%%%%%%%%%%%%%%%%%%%%%%%%%%%%%%%%%%%%%%
\subsection{Revision History}

%%%%%%%%%%%%%%%%%%%%%%%%%%%%%%%%%%%%%%%%
\paragraph{v2.0:} 2018/12/30

\begin{itemize}
\item
immediate forward processing
\item
added |\childdocby| mechanism
\item
manual restructured
\end{itemize}

%%%%%%%%%%%%%%%%%%%%%%%%%%%%%%%%%%%%%%%%
\paragraph{v1.6:} 2018/01/17

\begin{itemize}
\item
application for development of include files
\item
corrections to manual
\end{itemize}

%%%%%%%%%%%%%%%%%%%%%%%%%%%%%%%%%%%%%%%%
\paragraph{v1.5:} 2017/05/21

\begin{itemize}
\item
more complete structuring introduced
\item
|\childdocof| introduced
\item
|\childdoc| renamed to |\childdocmain|
\item
|\childredirect| renamed to |\childdocforward| and |\childdocforwardprefix|
and functionality expanded
\end{itemize}

%%%%%%%%%%%%%%%%%%%%%%%%%%%%%%%%%%%%%%%%
\paragraph{v1.0:} 2017/04/27

\begin{itemize}
\item
manual and install package
\item
first version published on CTAN
\end{itemize}

%%%%%%%%%%%%%%%%%%%%%%%%%%%%%%%%%%%%%%%%
\paragraph{v0.6:} 2017/04/26

\begin{itemize}
\item
redirection mechanism added
\end{itemize}

%%%%%%%%%%%%%%%%%%%%%%%%%%%%%%%%%%%%%%%%
\paragraph{v0.5:} 2017/04/26

\begin{itemize}
\item
functionality in definition file
\end{itemize}


%%%%%%%%%%%%%%%%%%%%%%%%%%%%%%%%%%%%%%%%%%%%%%%%%%%%%%%%%%%%%%%%%%%%%%%%%%%%%%%%
%%%%%%%%%%%%%%%%%%%%%%%%%%%%%%%%%%%%%%%%%%%%%%%%%%%%%%%%%%%%%%%%%%%%%%%%%%%%%%%%
%%%%%%%%%%%%%%%%%%%%%%%%%%%%%%%%%%%%%%%%%%%%%%%%%%%%%%%%%%%%%%%%%%%%%%%%%%%%%%%%
\appendix

\settowidth\MacroIndent{\rmfamily\scriptsize 000\ }

 \DocInput{childdoc.dtx}

\end{document}
%</driver>
% \fi
%
% %%%%%%%%%%%%%%%%%%%%%%%%%%%%%%%%%%%%%%%%%%%%%%%%%%%%%%%%%%%%%%%%%%%%%%%%%%%%%%
% %%%%%%%%%%%%%%%%%%%%%%%%%%%%%%%%%%%%%%%%%%%%%%%%%%%%%%%%%%%%%%%%%%%%%%%%%%%%%%
% \section{Sample}
%\iffalse
%<*samplemain>
%\fi
%
% The following presents a sample document
% with two chapters, two parts, a title page,
% a compile flag as well as three forwarding files to set the flag.
% It consists of eight |.tex| files:
% \begin{center}
% \begin{tabular}{ll}
% |cdocsamp.tex|&main file\\
% |cdocsch1.tex|&include file for chapter 1\\
% |cdocsch2.tex|&include file for chapter 2\\
% |cdocspt3.tex|&include file for part 3\\
% |cdocspt4.tex|&include file for part 4\\
% |cdocsdrf.tex|&forwarding file for main file in draft mode\\
% |cdocsfi1.tex|&forwarding file for final version of chapter 1\\
% |cdocsfi2.tex|&forwarding file for final version of chapter 2\\
% \end{tabular}
% \end{center}
% Each of the eight files can be compiled directly by the \LaTeX{} compiler.
%
% %%%%%%%%%%%%%%%%%%%%%%%%%%%%%%%%%%%%%%
% \paragraph{Main File.}
%
% The main file is called |cdocsamp.tex|.
%
% Load the \textsf{childdoc} definitions and
% declare the filename for the main document:
%    \begin{macrocode}
\input{childdoc.def}
\childdocmain{}
%    \end{macrocode}

% Optional override for |\version| flag:
%    \begin{macrocode}
%%\ifchilddoc\else\providecommand{\version}{draft}\fi
%    \end{macrocode}

% Define the default values for the |\version| flag
% (|final| for the main file and |draft| for childs):
%    \begin{macrocode}
\ifchilddoc
\providecommand{\version}{draft}
\else
\providecommand{\version}{final}
\fi
%    \end{macrocode}

% Load the standard document class:
%    \begin{macrocode}
\documentclass[12pt]{article}
%    \end{macrocode}

% Start the document body:
%    \begin{macrocode}
\begin{document}
%    \end{macrocode}

% Declare a title page.
% Print title, part of document being processed and version flag:
%    \begin{macrocode}
\addtocounter{page}{-1}
\begin{center}
{\LARGE\bfseries{}childdoc example\par}
\vspace{1cm}
\ifchilddoc
\ifchilddocmanual part\else chapter\fi:
`\childdocname' of `\childdocjob'\par
\else
main document: `\childdocjob'\par
\fi
version: \version\par
\end{center}
\newpage
%    \end{macrocode}

% Manually include selected file,
% otherwise process as usual:
%    \begin{macrocode}
\ifchilddocmanual
\section*{part `\childdocname'}
\input{\childdocname}
\else
%    \end{macrocode}

% Include the two chapters:
%    \begin{macrocode}
\include{cdocsch1}
\include{cdocsch2}
%    \end{macrocode}

% Include the two parts unless only chapters should be displayed:
%    \begin{macrocode}
\ifchilddoc\else
\section{part three}
\input{cdocspt3}
\section{part four}
\input{cdocspt4}
\fi
%    \end{macrocode}

% Process as usual until here:
%    \begin{macrocode}
\fi
%    \end{macrocode}

% End of document body:
%    \begin{macrocode}
\end{document}
%    \end{macrocode}
%\iffalse
%</samplemain>
%\fi
%
% %%%%%%%%%%%%%%%%%%%%%%%%%%%%%%%%%%%%%%
% \paragraph{Chapter Include Files.}
%
% The include files are called |cdocsch1.tex| and |cdocsch2.tex|.
%
%\iffalse
%<*samplechap1|samplechap2>
%\fi

% Optional override for |\version| flag:
%    \begin{macrocode}
%%\providecommand{\version}{final}
%    \end{macrocode}

% Include the main document:
%    \begin{macrocode}
\input{childdoc.def}
\childdocof{cdocsamp}
%    \end{macrocode}

%\iffalse
%</samplechap1|samplechap2>
%\fi
%
%\iffalse
%<*samplechap1>
%\fi
% Some text for chapter 1:
%    \begin{macrocode}
\section{one}
some text in chapter one
%    \end{macrocode}

%\iffalse
%</samplechap1>
%\fi
% Some text for chapter 2:
%\iffalse
%<*samplechap2>
%\fi
%    \begin{macrocode}
\section{two}
more text in chapter two
%    \end{macrocode}

%\iffalse
%</samplechap2>
%\fi
%
% %%%%%%%%%%%%%%%%%%%%%%%%%%%%%%%%%%%%%%
% \paragraph{Part Include Files.}
%
% The include files are called |cdocspt3.tex| and |cdocspt4.tex|.
%
%\iffalse
%<*samplepart3|samplepart4>
%\fi

% Optional override for |\version| flag:
%    \begin{macrocode}
%%\providecommand{\version}{final}
%    \end{macrocode}

% Include the main document:
%    \begin{macrocode}
\input{childdoc.def}
\childdocby{cdocsamp}
%    \end{macrocode}

%\iffalse
%</samplepart3|samplepart4>
%\fi
%
%\iffalse
%<*samplepart3>
%\fi
% Some text for part 3:
%    \begin{macrocode}
some text in part three
%    \end{macrocode}

%\iffalse
%</samplepart3>
%\fi
% Some text for part 4:
%\iffalse
%<*samplepart4>
%\fi
%    \begin{macrocode}
more text in part four
%    \end{macrocode}

%\iffalse
%</samplepart4>
%\fi
%
% %%%%%%%%%%%%%%%%%%%%%%%%%%%%%%%%%%%%%%
% \paragraph{Forwarding for a Complete Draft.}
%
% The following forwarding file |cdocsdrf.tex|
% compiles the main document in draft mode:
%\iffalse
%<*sampledraft>
%\fi
%    \begin{macrocode}
\def\version{draft}
\input{childdoc.def}
\childdocforward{cdocsamp}
%    \end{macrocode}

%\iffalse
%</sampledraft>
%\fi
%
% %%%%%%%%%%%%%%%%%%%%%%%%%%%%%%%%%%%%%%
% \paragraph{Forwarding for Final Version of the Chapters.}
%
% The following forwarding files |cdocsfn1.tex| and |cdocsfn2.tex|
% (with identical content)
% compile the final versions of the child documents
% |cdocsch1.tex| and |cdocsch2.tex|, respectively:
%\iffalse
%<*samplefinal>
%\fi
%    \begin{macrocode}
\def\version{final}
\input{childdoc.def}
\childdocforwardprefix[cdocsamp]{cdocsfn}{cdocsch}
%    \end{macrocode}

%\iffalse
%</samplefinal>
%\fi
%
% %%%%%%%%%%%%%%%%%%%%%%%%%%%%%%%%%%%%%%
% \paragraph{Command Line Processing.}
%
% The following three command lines generate the output files
% |cdocscld|, |cdocscl1| and |cdocscl2|
% which should be identical to
% |cdocsdrf|, |cdocsch1| and |cdocsfn2|, respectively:
% \begin{center}
% \begin{tabular}{l}
% |latex -jobname cdocscld \|\\
% |  "\def\version{draft}\input{childdoc.def}\childdocforward{cdocsamp}"|\\
% |latex -jobname cdocscl1 \|\\
% |  "\input{childdoc.def}\childdocforward[cdocsamp]{cdocsch1}"|\\
% |latex -jobname cdocscl2 \|\\
% |  "\def\version{final}\input{childdoc.def}\childdocforward{cdocsch2}"|
% \end{tabular}
% \end{center}
% Note that the trailing backslash on each first line
% merely continues the input to the second line
% (for convenient cut ant paste).
% Furthermore, the command |latex| can be replaced by any
% of its alternative versions such as |pdflatex|.
%
% %%%%%%%%%%%%%%%%%%%%%%%%%%%%%%%%%%%%%%%%%%%%%%%%%%%%%%%%%%%%%%%%%%%%%%%%%%%%%%
% %%%%%%%%%%%%%%%%%%%%%%%%%%%%%%%%%%%%%%%%%%%%%%%%%%%%%%%%%%%%%%%%%%%%%%%%%%%%%%
% \section{Implementation}
%\iffalse
%<*package>
%\fi
%
% This section describes the definitions file |childdoc.def|.

% The definitions cannot be loaded using |\usepackage| or |\RequirePackage|
% which has a mechanism to prevent loading a style file more than once.
% When loading the definitions by means of |\input|
% multiple instances have to be prevented manually:
%\iffalse
%This code needs to be before the `\ProvidesFile' directive
%which is defined at the beginning of this file.
%Therefore it is also placed there and commented out here.
%</package>
%<*discard>
%\fi
%    \begin{macrocode}
\ifdefined\childdocmain\endinput\fi
%    \end{macrocode}
%\iffalse
%</discard>
%<*package>
%\fi
%
% \macro{\ifchilddoc}
% \macro{\ifchilddocmanual}
% The conditional |\ifchilddoc| tells whether a
% child (true) or main (false) document is being compiled.
% The conditional |\ifchilddocmanual| tells whether
% the |\includeonly| mechanism is used (false) or
% the selection of child files must be performed manually (true).
% The definitions initialise to false:
%    \begin{macrocode}
\newif\ifchilddoc
\newif\ifchilddocmanual
%    \end{macrocode}

% \macro{\childdocname}
% \macro{\childdocjob}
% The macro |\childdocname| stores the name of the main document
% to be compiled. The macro |\childdocjob| stores the name of
% the document on which the \LaTeX{} compiler was originally invoked.
% The content of |\jobname| cannot be compared
% to filenames specified in the source due to different catcodes.
% The following code rescans |\jobname|, stores the result
% in |\childdocname| and saves a copy in |\childdocjob|:
%    \begin{macrocode}
\edef\childdocname{\scantokens\expandafter{\jobname\noexpand}}
\let\childdocjob\childdocname
%    \end{macrocode}

% \macro{\childdocdisable}
% The macro |\childdocdisable| prevents the main file
% from being processed more than once.
% At this stage, the main document command |\childdocmain|
% is assumed to be called once again where it should do nothing.
% Any subsequent call to it should prevent
% a secondary processing of the main document
% It overwrites the forwarding commands
% |\childdocof| and |\childdocforward|
% with empty macros to prevent further inclusions of the main document:
%    \begin{macrocode}
\newcommand{\childdocdisable}
{
  \renewcommand{\childdocmain}[1]{\renewcommand{\childdocmain}[1]{\endinput}}
  \renewcommand{\childdocof}[1]{}
  \renewcommand{\childdocby}[2][]{}
  \renewcommand{\childdocforward}[2][]{}
  \renewcommand{\childdocdisable}{}
}
%    \end{macrocode}

% \macro{\childdocmain}
% The macro |\childdocmain| is to be called at the top of the main file
% with nothing or the main filename (without extension) as argument.
% First, it breaks loops.
% If the argument is not empty and does not match |\childdocname|
% (which is set by the first inclusion of |childdoc.def|),
% |\ifchilddoc| is set to true, |\includeonly| is applied to the child file
% and |\jobname| is set to the main file
% (for proper handling of |.aux| files):
%    \begin{macrocode}
\newcommand{\childdocmain}[1]
{
  \childdocdisable\childdocmain{}
  \if?#1?\else
    \begingroup
      \def\childdoctmp{#1}
      \ifx\childdoctmp\childdocname
        \def\childdoctmp{}
      \else
        \def\childdoctmp
        {
          \childdoctrue
          \includeonly{\childdocname}
          \def\childdocjob{#1}
          \def\jobname{#1}
        }
      \fi
      \expandafter
    \endgroup
    \childdoctmp
  \fi
}
%    \end{macrocode}

% \macro{\childdocof}
% The command |\childdocof| redirects
% compilation to the main file |#1|.
%    \begin{macrocode}
\newcommand{\childdocof}[1]
{
  \childdocdisable
  \childdoctrue
  \includeonly{\childdocname}
  \def\jobname{#1}
  \def\childdocjob{#1}
  \input{#1}
}
%    \end{macrocode}

% \macro{\childdocby}
% The command |\childdocby| ....
%    \begin{macrocode}
\newcommand{\childdocby}[2][]
{
  \childdocdisable
  \childdoctrue
  \childdocmanualtrue
  \if?#1?\else
    \def\jobname{#2}
  \fi
  \def\childdocjob{#2}
  \input{#2}
  \endinput
}
%    \end{macrocode}

% \macro{\childdocforward}
% The command |\childdocforward| redirects
% compilation to the main file or
% (if the optional argument is given) a child file.
% Parameters are set as if the main file
% or a child file starting with |\childdocof| was compiled.
% Then compilation is handed over to the main file:
%    \begin{macrocode}
\newcommand{\childdocforward}[2][]
{
  \begingroup
    \if?#1?
      \def\childdoctmp
      {
        \def\childdocname{#2}
        \def\childdocjob{#2}
        \def\jobname{#2}
        \input{#2}
        \endinput
      }
    \else
      \def\childdoctmp
      {
        \childdocdisable
        \def\childdocname{#2}
        \childdoctrue
        \includeonly{#2}
        \def\childdocjob{#1}
        \def\jobname{#1}
        \input{#1}
        \endinput
      }
    \fi
    \expandafter
  \endgroup
  \childdoctmp
}
%    \end{macrocode}

% \macro{\childdocforwardprefix}
% The command |\childdocforwardprefix| redirects
% compilation to the main or a child file by means of a pattern.
% The prefix |#1| in the current filename is replaced by |#2|
% and the suffix of the current filename is kept
% (it is assumed that the filename does not contain the substring `|~~~|'
% which is used as a delimiter).
% Compilation is handed over to the new file by |\childdocforward|:
%    \begin{macrocode}
\newcommand{\childdocforwardprefix}[3][]
{
  \begingroup
    \def\childdocextract #2##1~~~{\def\childdoctmp{\childdocforward[#1]{#3##1}}}
    \expandafter\childdocextract\childdocname~~~
    \expandafter
  \endgroup
  \childdoctmp
}
%    \end{macrocode}

% \macro{\childdoc}
% The deprecated macro |\childdoc| is a legacy version of |\childdocmain|:
%    \begin{macrocode}
\newcommand{\childdoc}{\childdocmain}
%    \end{macrocode}

% \macro{\childdocredirect}
% The deprecated macro |\childdocredirect| is a legacy version
% of |\childdocforward| and |\childdocforwardprefix|:
%    \begin{macrocode}
\newcommand{\childdocredirect}[2][]
{
  \begingroup
    \if?#1?
      \def\childdoctmp{\childdocforward{#2}}
    \else
      \def\childdoctmp{\childdocforwardprefix{#1}{#2}}
    \fi
    \expandafter
  \endgroup
  \childdoctmp
}
%    \end{macrocode}

%\iffalse
%</package>
%\fi
%
\endinput
|\\
|\childdocforward{|\textit{main}|}|\\
\end{tabular}
\end{center}
%
or alternatively with:
%
\begin{center}
\begin{tabular}{l}
|% \iffalse
%
% childdoc.dtx Copyright (C) 2017-2018 Niklas Beisert
%
% This work may be distributed and/or modified under the
% conditions of the LaTeX Project Public License, either version 1.3
% of this license or (at your option) any later version.
% The latest version of this license is in
%   http://www.latex-project.org/lppl.txt
% and version 1.3 or later is part of all distributions of LaTeX
% version 2005/12/01 or later.
%
% This work has the LPPL maintenance status `maintained'.
%
% The Current Maintainer of this work is Niklas Beisert.
%
% This work consists of the files childdoc.dtx and childdoc.ins
% and the derived files childdoc.def and cdocsamp.tex with
% cdocsch1.tex, cdocsch2.tex, cdocsdrf.tex, cdocsfn1.tex, cdocsfn2.tex.
%
%<package>\ifdefined\childdocmain\endinput\fi
%<package>\ProvidesFile{childdoc.def}[2018/12/30 v2.0 child document driver]
%<samplemain>\ProvidesFile{cdocsamp.tex}[2018/12/30 v2.0 sample for childdoc]
%<*driver>
%\ProvidesFile{childdoc.drv}[2018/12/30 v2.0 childdoc reference manual file]
\PassOptionsToClass{10pt,a4paper}{article}
\documentclass{ltxdoc}

\usepackage[margin=35mm]{geometry}
\usepackage{hyperref}
\usepackage{hyperxmp}
\usepackage[usenames]{color}

\hypersetup{colorlinks=true}
\hypersetup{pdfstartview=FitH}
\hypersetup{pdfpagemode=UseNone}
\hypersetup{pdfsource={}}
\hypersetup{pdflang={en-UK}}
\hypersetup{pdfcopyright={Copyright 2017-2018 Niklas Beisert.
  This work may be distributed and/or modified under the
  conditions of the LaTeX Project Public License, either version 1.3
  of this license or (at your option) any later version.}}
\hypersetup{pdflicenseurl={http://www.latex-project.org/lppl.txt}}
\hypersetup{pdfcontactaddress={ETH Zurich, ITP, HIT K,
  Wolfgang-Pauli-Strasse 27}}
\hypersetup{pdfcontactpostcode={8093}}
\hypersetup{pdfcontactcity={Zurich}}
\hypersetup{pdfcontactcountry={Switzerland}}
\hypersetup{pdfcontactemail={nbeisert@itp.phys.ethz.ch}}
\hypersetup{pdfcontacturl={http://people.phys.ethz.ch/\xmptilde nbeisert/}}

\newcommand{\secref}[1]{\hyperref[#1]{section \ref*{#1}}}

\parskip1ex
\parindent0pt
\let\olditemize\itemize
\def\itemize{\olditemize\parskip0pt}

\begin{document}

\title{The \textsf{childdoc} Package}
\hypersetup{pdftitle={The childdoc Package}}
\author{Niklas Beisert\\[2ex]
  Institut f\"ur Theoretische Physik\\
  Eidgen\"ossische Technische Hochschule Z\"urich\\
  Wolfgang-Pauli-Strasse 27, 8093 Z\"urich, Switzerland\\[1ex]
  \href{mailto:nbeisert@itp.phys.ethz.ch}
  {\texttt{nbeisert@itp.phys.ethz.ch}}}
\hypersetup{pdfauthor={Niklas Beisert}}
\hypersetup{pdfsubject={Manual for the LaTeX2e Package childdoc}}
\date{30 December 2018, \textsf{v2.0}}
\maketitle

\begin{abstract}\noindent
\textsf{childdoc} is a \LaTeXe{} package
that enables the direct compilation
of document sections included by |\include|
to individual files.
\end{abstract}

\begingroup
\parskip0ex
\tableofcontents
\endgroup

%%%%%%%%%%%%%%%%%%%%%%%%%%%%%%%%%%%%%%%%%%%%%%%%%%%%%%%%%%%%%%%%%%%%%%%%%%%%%%%%
%%%%%%%%%%%%%%%%%%%%%%%%%%%%%%%%%%%%%%%%%%%%%%%%%%%%%%%%%%%%%%%%%%%%%%%%%%%%%%%%
\section{Introduction}

\LaTeX{} provides a mechanism to structure a large document (such as a book)
into a main file and several child files (containing the chapters)
using the |\include| command.
This mechanism is beneficial for documents
which span hundreds of pages in order to
make the source file(s) more manageable.
Moreover, compilation can be restricted to
selected child files by means of the |\includeonly| command.
The latter feature can be used to reduce the compilation time while editing
(this was significantly more useful in the earlier days of \LaTeX{})
or to generate a smaller document which is easier to navigate.
Another application of |\includeonly| is to generate
documents consisting of selected parts of the complete document.

However, there are a few drawbacks of the plain |\include| mechanism:
\begin{itemize}
\item
The child files cannot be compiled on their own,
they can only be compiled via the main file.
A naive editing environment
(such as a text editor with an option
to have the current file processed by \LaTeX)
may require one to switch to the main file before compiling;
attempting to compile the child file produces errors.
\item
The main file must be modified (each time)
to adjust the |\includeonly| command
to the present needs. This easily leaves the main file in a messy state.
\item
The generated document will always carry the filename
of the main document. This is inconvenient if
several child files are to be compiled and
to be kept for distribution.
\end{itemize}

The present package provides a simple interface
to make child files individually compilable by \LaTeX{}.
Compiling a child file then has the same effect as compiling
the main file with an |\includeonly| command
to select the appropriate child.
Moreover the generated document will carry the name of the child
rather than the main file.
This resolves all three above issues.

This feature is meant to make the editing of books,
thesis documents and lecture notes somewhat more convenient.
However, the package can also be used efficiently for
composing a series of documents (such as exercise sheets)
which are typically distributed individually.
It then assists the author in generating the individual documents
(potentially in different versions)
as well as a document containing the collected series.
Another application is in developing style files
or other kinds of included material
where compilation of the style file could redirect
to a sample or test file.

%%%%%%%%%%%%%%%%%%%%%%%%%%%%%%%%%%%%%%%%%%%%%%%%%%%%%%%%%%%%%%%%%%%%%%%%%%%%%%%%
%%%%%%%%%%%%%%%%%%%%%%%%%%%%%%%%%%%%%%%%%%%%%%%%%%%%%%%%%%%%%%%%%%%%%%%%%%%%%%%%
\section{Usage}

First of all, the package \textsf{childdoc} is \emph{not} a standard
\LaTeXe{} |.sty| style file! Therefore it needs to be invoked in
a non-standard way.

%%%%%%%%%%%%%%%%%%%%%%%%%%%%%%%%%%%%%%%%%%%%%%%%%%%%%%%%%%%%%%%%%%%%%%%%%%%%%%%%
\subsection{Included Files}
\label{sec:include}

%%%%%%%%%%%%%%%%%%%%%%%%%%%%%%%%%%%%%%%%
\DescribeMacro{\childdocmain}
To use the package, add the commands
\begin{center}
\begin{tabular}{l}
|\input{childdoc.def}|\\
|\childdocmain{}|\\
\end{tabular}
\end{center}
at the very top of the main \LaTeX{} file,
in particular \emph{before} the |\documentclass| statement!
The argument of |\childdocmain| should be left empty
(but it must be present).

%%%%%%%%%%%%%%%%%%%%%%%%%%%%%%%%%%%%%%%%
\DescribeMacro{\childdocof}
Furthermore, add the commands
\begin{center}
\begin{tabular}{l}
|\input{childdoc.def}|\\
|\childdocof{|\textit{main}|}|\\
\end{tabular}
\end{center}
at the top of every child file \textit{child}
which is included by |\include{|\textit{child}|}|
from within the main file
(or at least for those files to be compiled individually).
The argument \textit{main} must be the filename of the main file.

There are a couple of
considerations in setting up the main and child documents:

%%%%%%%%%%%%%%%%%%%%%%%%%%%%%%%%%%%%%%%%
\paragraph{Restrictions.}

Please note the following restrictions:
\begin{itemize}
\item
|\childdocmain| must be called with one argument \textit{main}
to ensure compatibility with earlier version of the package.
It must either be empty (|\childdocmain{}|)
or precisely match the filename of the main file in which it is specified.
See \secref{sec:detection} for further information.
\item
The filename \textit{main} must be specified without the |.tex| extension.
\item
The filename \textit{main} is case sensitive
(even in case-insensitive file systems)
due to internal string comparison.
\item
The argument \textit{main} should be fully expanded, it cannot be a macro.
\item
Subdirectories and special characters should be avoided in filenames.
\item
The command |\childdocmain{|\textit{main}|}| must be followed by a whitespace.
It should not be followed immediately by another command
or by a comment mark `|%|'.
This is because the \TeX{} parser reads the token immediately following
the argument of |\childdocmain| and puts it
at the beginning of every child section;
however, a white\-space is ignored.
\end{itemize}

%%%%%%%%%%%%%%%%%%%%%%%%%%%%%%%%%%%%%%%%
\paragraph{Content of Main File.}

It is advisable to place all content in the child files included by |\include|.
Any output contained in the main file will appear in all child documents
unless suppressed manually;
it cannot be suppressed automatically by the |\includeonly| directive
and thus should normally be avoided.
A method to include some content in the main file
by means of conditional processing is described in \secref{sec:conditional}.

%%%%%%%%%%%%%%%%%%%%%%%%%%%%%%%%%%%%%%%%
\paragraph{Page Numbering.}

When only a part of the document is compiled,
the appropriate numbering of pages
(as well as other status parameters)
is determined from the |.aux| files.
The latter contain information from previous passes.
However this information needs to propagate through
all intermediate child documents.
Therefore the page numbering in child documents may well
be inconsistent until the complete document is compiled at least once.

A useful (if unconventional) way to always ensure a consistent
page numbering is to restart the numbering in each child document
and denote the pages by `\textit{child}|.|\textit{page}'
where \textit{child} represents the chapter/section number of the child file.
This can be achieved by the command
|\numberwithin{page}{|\textit{child}|}|
of the \textsf{amsmath} package
where \textit{child} can be |chapter| or |section|
depending on the chosen structuring.
Alternatively, one can modify the macro |\thepage| appropriately
and reset the counter |page| at the start of each child file.

%%%%%%%%%%%%%%%%%%%%%%%%%%%%%%%%%%%%%%%%%%%%%%%%%%%%%%%%%%%%%%%%%%%%%%%%%%%%%%%%
\subsection{Conditional Processing}
\label{sec:conditional}

The package provides a mechanism to compile different versions
of a document. To customise the versions further some conditional processing
can come in handy to distinguish which version is being compiled.
The package provides two macros to describe the compilation context:

%%%%%%%%%%%%%%%%%%%%%%%%%%%%%%%%%%%%%%%%
\DescribeMacro{\ifchilddoc}
The conditional |\ifchilddoc| distinguishes between the compilation of
child documents and the main document:
%
\begin{center}
|\ifchilddoc |\textit{child-code}| |[|\||else |\textit{main-code}]| \||fi|
\end{center}

%%%%%%%%%%%%%%%%%%%%%%%%%%%%%%%%%%%%%%%%
\DescribeMacro{\childdocname}
\DescribeMacro{\childdocjob}
The macro |\childdocname| contains the filename (without extension)
of the main or child file being processed.
Note that |\childdocjob| will always contain the name of the main file.

%%%%%%%%%%%%%%%%%%%%%%%%%%%%%%%%%%%%%%%%
\paragraph{Title Page.}

Conditional processing can be used to include a title or banner page
in the main document when proper precautions are taken.
Importantly, the code in the main file should ensure that the page counter
(as well as other status parameters which are stored in the |.aux| files)
takes the same value after the conditional processing.
Otherwise the page numbers may take divergent values
depending on which part is compiled.

For example, a title page could be declared by:
%
\begin{center}
\begin{tabular}{l}
|\ifchilddoc\||else|\\
|\addtocounter{page}{-1}|\\
\textit{code for title page}\\
|\newpage|\\
|\||fi|
\end{tabular}
\end{center}
%
A banner page for the child documents can be generated by:
%
\begin{center}
\begin{tabular}{l}
|\ifchilddoc|\\
|\addtocounter{page}{-1}|\\
\textit{code for banner page}\\
|\newpage|\\
|\||fi|
\end{tabular}
\end{center}
%
Here one could write a message such as:
\begin{center}
|This is the part \childdocname{} of \childdocjob{}.|
\end{center}

%%%%%%%%%%%%%%%%%%%%%%%%%%%%%%%%%%%%%%%%%%%%%%%%%%%%%%%%%%%%%%%%%%%%%%%%%%%%%%%%
\subsection{Flags}
\label{sec:flags}

The package makes it easy to generate different versions
of the main or child documents.
To this end compilation flags can be defined
and assigned different default values.
They will be particularly useful in conjunction
with the forwarding mechanism described in \secref{sec:forward}.

For example, it may be useful to have a flag |\version|
which can be set to |draft| or |final|.
The document source will contain some conditional code
depending on the value of |\version|.
Suppose further, the flag should default to |final| for the main file
and to |draft| for child files
which is a natural assignment for editing the document.
This is achieved by placing the following code
in the preamble of the main document
(below the |\childdocmain| directive):
%
\begin{center}
\begin{tabular}{l}
|\ifchilddoc|\\
|\providecommand{\version}{draft}|\\
|\||else|\\
|\providecommand{\version}{final}|\\
|\||fi|
\end{tabular}
\end{center}
%
The definition by |\providecommand| makes sure
that previous definitions are not overwritten.
Further statements |\providecommand{\version}{...}|
can thus be added before the above code to override it.

For the main file, one might add a line
(between |\childdocmain| and the above block)
%
\begin{center}
|%\ifchilddoc\||else\providecommand{\version}{draft}\||fi|
\end{center}
%
which can be uncommented to produce a draft version.
Likewise one can add a line to the very top of a child file
(above the |\childdocof{|\textit{main}|}| directive)
%
\begin{center}
|%\providecommand{\version}{final}|
\end{center}
%
which can be uncommented to produce the final version of this child document.

%%%%%%%%%%%%%%%%%%%%%%%%%%%%%%%%%%%%%%%%%%%%%%%%%%%%%%%%%%%%%%%%%%%%%%%%%%%%%%%%
\subsection{Forwarding}
\label{sec:forward}

Different versions of the main or child documents
using compilation flags as described in \secref{sec:flags}
can be (permanently) stored in different files
for convenient compilation, viewing and distribution.
To this end, the package defines a command
to pass on compilation to a different file:

%%%%%%%%%%%%%%%%%%%%%%%%%%%%%%%%%%%%%%%%
\DescribeMacro{\childdocforward}
The command |\childdocforward| redirects processing to
another source file:
%
\begin{center}
\begin{tabular}{l}
|\input{childdoc.def}|\\
|\childdocforward[|\textit{main}|]{|\textit{dest}|}|\\
\end{tabular}
\end{center}
%
The argument \textit{dest} is the destination file
(without extension).
It should be the main file or one of the child files.
Note that further \textsf{childdoc} directives
such as |\childdocof| and |\childdocforward|
in the indicated file will be processed in this form.
The optional argument \textit{main}
passes on directly to the main file \textit{main}
while pretending to compile the child \textit{dest}.
This form behaves as if \textit{dest}
issues |\childdocof{|\textit{main}|}| right away,
and no further \textsf{childdoc} directives will be processed.

%%%%%%%%%%%%%%%%%%%%%%%%%%%%%%%%%%%%%%%%
\DescribeMacro{\...prefix}
In the alternative form |\childdocforwardprefix|,
%
\begin{center}
\begin{tabular}{l}
|\input{childdoc.def}|\\
|\childdocforwardprefix[|\textit{main}|]{|\textit{prefix}|}{|\textit{dest}|}|
\end{tabular}
\end{center}
%
the destination file is determined by a pattern
depending on the current file:
To make this work, the current file must be called
`{\textit{prefix}\hspace{0.2em}\textit{suffix}}'
with \textit{prefix} matching precisely the argument.
Processing is then passed on to the file
`{\textit{dest}\hspace{0.2em}\textit{suffix}}'.
Surely, the same effect is achieved by
directly specifying the
argument `{\textit{dest}\hspace{0.2em}\textit{suffix}}'
in the first form.
However, that requires to set up a different file
for each child. With the alternative form of the command
all these files can have exactly the same content
which simplifies setting them up and maintaining them.

For example, the following file |draft.tex|
with a compilation flag |\version| as described in \secref{sec:flags}
compiles the main document as a draft:
%
\begin{center}
\begin{tabular}{l}
|\def\version{draft}|\\
|\input{childdoc.def}|\\
|\childdocforward{|\textit{main}|}|
\end{tabular}
\end{center}
%
Likewise, the following files |final|\textit{nn}|.tex|
compile the final version of the child document
|child|\textit{nn}|.tex|:
%
\begin{center}
\begin{tabular}{l}
|\def\version{final}|\\
|\input{childdoc.def}|\\
|\childdocforwardprefix{final}{child}|
\end{tabular}
\end{center}
%

Note that when several versions of a main file and/or of each child file
are to be generated, it may be convenient to set up a |Makefile| or
shell script to automatise the process.

%%%%%%%%%%%%%%%%%%%%%%%%%%%%%%%%%%%%%%%%%%%%%%%%%%%%%%%%%%%%%%%%%%%%%%%%%%%%%%%%
\subsection{Command Line Processing}
\label{sec:commandline}

The effect of redirection files can also be achieved by invoking
the \LaTeX{} compiler with a more elaborate command line.
Most conveniently this should be done as part
of a shell script or a |Makefile|.

When using \textsf{childdoc} in the main file, the following
command lines effectively perform a redirection
(note that depending on the shell being used,
backslashes may have to be doubled: `|\|' $\to$ `|\\|'):
%
\begin{center}
|... -jobname "|\textit{target}|" |\\|"|[\textit{flags}]%
|\input{childdoc.def}\childdocforward[|\textit{main}|]{|\textit{dest}|}"|
\end{center}
%
Here \textit{target} is the name of the output file,
\textit{main} is the name of the main file
and \textit{dest} is the name of the main or child file to be processed
(all filenames without extensions).
The optional argument \textit{main} can be omitted
if \textit{main} matches \textit{dest}.
Optionally, compilation \textit{flags} can be defined via |\def| commands.
This command line makes the \TeX{} engine believe
it is compiling the file \textit{target}
whose content is specified as the latter parameter.
The provided code then forwards the processing to
\textit{main} or \textit{dest} as described in \secref{sec:forward}.

%%%%%%%%%%%%%%%%%%%%%%%%%%%%%%%%%%%%%%%%%%%%%%%%%%%%%%%%%%%%%%%%%%%%%%%%%%%%%%%%
\subsection{Include by Input}
\label{sec:input}

Including child documents by |\include| has some restrictions by design.
Most notably, the content of a child document always occupies
its own set of pages; pages cannot be shared between child documents.
Usually, this behaviour makes perfect sense
because each child document contain an essential part of the document.
However, in some situations it may be desirable to compose
a document from a collection of parts
without having mandatory page breaks between then.
For this case, the package
provides a mechanism to include parts
by |\input| which can also be processed individually.
However, by construction this mechanism
requires manual handling of the content to be output.

%%%%%%%%%%%%%%%%%%%%%%%%%%%%%%%%%%%%%%%%
\DescribeMacro{\ifchilddocmanual}
The main file should be prepared as usual, see \secref{sec:include}.
However, the document body must make a distinction
between processing of an individual part and of the main document, e.g.:
%
\begin{center}
\begin{tabular}{l}
|\ifchilddocmanual|\\
|\input{\childdocname}|\\
|\||else|\\
\textit{document body with }|\input{|\textit{part}|}|\\
|\||fi|
\end{tabular}
\end{center}
%
The conditional |\ifchilddocmanual| is true whenever
a part to be included by |\input| is being compiled,
and the name of the part is stored in |\childdocname|.

%%%%%%%%%%%%%%%%%%%%%%%%%%%%%%%%%%%%%%%%
\DescribeMacro{\childdocby}
Each part to be included by |\input| should start with:
%
\begin{center}
\begin{tabular}{l}
|\input{childdoc.def}|\\
|\childdocby{|\textit{main}|}|\\
\end{tabular}
\end{center}
%
The directive |\childdocby| is similar to |\childdocof|
described in \secref{sec:include},
but the subsequent selection of content must be done manually.
To that end, both |\ifchilddoc| and |\ifchilddocmanual|
will be true upon processing of a part,
and the name of the part is stored in |\childdocname|.
Note that |\jobname| will be set to the filename of the current part
so that each part receives an individual |.aux| file
that does not interfere with the |.aux| file(s) of the main document.
This behaviour can be altered by the alternative form
|\childdocby[*]{|\textit{main}|}| (with a non-empty optional argument)
which uses the |.aux| file of the main document
by setting |\jobname| to \textit{main}.

%%%%%%%%%%%%%%%%%%%%%%%%%%%%%%%%%%%%%%%%%%%%%%%%%%%%%%%%%%%%%%%%%%%%%%%%%%%%%%%%
\subsection{Driver Development}
\label{sec:driver}

The \textsf{childdoc} mechanism can also be use for the development
of definition files such as \LaTeX{} styles or classes.
This case differs from the above setup with multiple parts
included by |\include| in that no |\includeonly| should be invoked.
This can be achieved by starting the include file
(before |\ProvidesPackage|) with:
%
\begin{center}
\begin{tabular}{l}
|\input{childdoc.def}|\\
|\childdocforward{|\textit{main}|}|\\
\end{tabular}
\end{center}
%
or alternatively with:
%
\begin{center}
\begin{tabular}{l}
|\input{childdoc.def}|\\
|\childdocby{|\textit{main}|}|\\
\end{tabular}
\end{center}
%
Both forms have slightly different effects as described above.
The main file is prepared as usual, see \secref{sec:include}.

%%%%%%%%%%%%%%%%%%%%%%%%%%%%%%%%%%%%%%%%%%%%%%%%%%%%%%%%%%%%%%%%%%%%%%%%%%%%%%%%
\subsection{Legacy Detection}
\label{sec:detection}

The directive |\childdocmain| in the main file can detect
whether the complete document or merely a child is to be compiled
even without using the directive |\childdocof|.
This method is deprecated because it is less robust
and there is no compelling reason to use it;
it is merely provided for backward compatibility
and it may be removed in future versions.

If the detection mechanism is to be used,
it is mandatory to correctly specify
the filename of the main file as the argument of |\childdocmain|:
%
\begin{center}
\begin{tabular}{l}
|\input{childdoc.def}|\\
|\childdocmain{|\textit{main}|}|\\
\end{tabular}
\end{center}
%
If |\jobname| does not match the argument \textit{main} of |\childdocmain|,
it is assumed that |\jobname| points to the child file to be compiled.
When using |\childdocmain| with the main file specified as argument,
it suffices to start a child file
with just |\input{|\textit{main}|}|
without loading of the package and using |\childdocof|.
If instead all processing is done
with the appropriate \textsf{childdoc} directives,
the argument of \textit{main} of |\childdocmain| can be empty.

An alternative version of the command line processing described
in \secref{sec:commandline} using the detection mechanism reads:
%
\begin{center}
|... -jobname "|\textit{target}|" "|[\textit{flags}]%
[|\def\jobname{|\textit{dest}|}|]|\input{|\textit{main}|}"|
\end{center}

%%%%%%%%%%%%%%%%%%%%%%%%%%%%%%%%%%%%%%%%%%%%%%%%%%%%%%%%%%%%%%%%%%%%%%%%%%%%%%%%
\subsection{Manual Code}
\label{sec:manual}

In case one cannot be certain whether the definitions file |childdoc.def|
is installed on the target \TeX{} distribution
and one prefers not to ship it,
it is conceivable to paste a few relevant commands into the sources.

To that end, drop all statements |\input{childdoc.def}|
and perform the replacements as outlined below.
Instead of |\childdocmain{|\textit{main}|}| add the following code
to the top of the main file:
%
\begin{center}
\begin{tabular}{l}
|\||ifdefined\childdocname\endinput\||fi\newif\ifchilddoc|\\
|\edef\childdocname{\scantokens\expandafter{\jobname\noexpand}}|\\
|\def\childdocmain{|\textit{main}|}\||ifx\childdocmain\childdocname\||else|\\
|\childdoctrue\includeonly{\childdocname}\let\jobname\childdocmain\||fi|\\
\end{tabular}
\end{center}
%
Instead of |\childdocof{|\textit{main}|}| just include the main file
at the top of each child file:
%
\begin{center}
|\input{|\textit{main}|}|
\end{center}
%
A simple redirection |\childdocforward{|\textit{dest}|}| is achieved by:
%
\begin{center}
|\def\jobname{|\textit{dest}|}\input{\jobname}|
\end{center}
%
The redirection with prefix
|\childdocforwardprefix[|\textit{prefix}|]{|\textit{dest}|}|
is accomplished by:
%
\begin{center}
\begin{tabular}{l}
|{\edef\jobname{\scantokens\expandafter{\jobname\noexpand}}|\\
|\def\redirectjob |\textit{prefix}|#1~~~{\gdef\jobname{|\textit{dest}|#1}}|\\
|\expandafter\redirectjob\jobname~~~}\input{\jobname}|
\end{tabular}
\end{center}

In an alternative approach,
child documents can be compiled by a specific command line
without additional code or specific definitions:
%
\begin{center}
|... -jobname "|\textit{target}|" "|[\textit{flags}]%
|\includeonly{|\textit{dest}|}\input{|\textit{main}|}"|
\end{center}
%

%%%%%%%%%%%%%%%%%%%%%%%%%%%%%%%%%%%%%%%%%%%%%%%%%%%%%%%%%%%%%%%%%%%%%%%%%%%%%%%%
%%%%%%%%%%%%%%%%%%%%%%%%%%%%%%%%%%%%%%%%%%%%%%%%%%%%%%%%%%%%%%%%%%%%%%%%%%%%%%%%
\section{Information}

%%%%%%%%%%%%%%%%%%%%%%%%%%%%%%%%%%%%%%%%%%%%%%%%%%%%%%%%%%%%%%%%%%%%%%%%%%%%%%%%
\subsection{Copyright}

Copyright \copyright{} 2017--2018 Niklas Beisert

This work may be distributed and/or modified under the
conditions of the \LaTeX{} Project Public License, either version 1.3
of this license or (at your option) any later version.
The latest version of this license is in
  \url{http://www.latex-project.org/lppl.txt}
and version 1.3 or later is part of all distributions of \LaTeX{}
version 2005/12/01 or later.

This work has the LPPL maintenance status `maintained'.

The Current Maintainer of this work is Niklas Beisert.

This work consists of the files |README.txt|, |childdoc.ins| and |childdoc.dtx|
as well as the derived files |childdoc.def|, |cdocsamp.tex|
with |cdocsch1.tex|, |cdocsch2.tex|, |cdocspt3.tex|, |cdocspt4.tex|,
|cdocsdrf.tex|, |cdocsfn1.tex|, |cdocsfn2.tex|
as well as |childdoc.pdf|.

%%%%%%%%%%%%%%%%%%%%%%%%%%%%%%%%%%%%%%%%%%%%%%%%%%%%%%%%%%%%%%%%%%%%%%%%%%%%%%%%
\subsection{Files and Installation}

The package consists of the files:
%
\begin{center}
\begin{tabular}{ll}
    |README.txt|   & readme file \\
    |childdoc.ins| & installation file \\
    |childdoc.dtx| & source file \\
    |childdoc.def| & definition file \\
    |cdocsamp.tex| & sample main file \\
    |cdocsch1.tex| & sample include file \\
    |cdocsch2.tex| & sample include file \\
    |cdocspt3.tex| & sample part file \\
    |cdocspt4.tex| & sample part file \\
    |cdocsdrf.tex| & sample redirection file \\
    |cdocsfn1.tex| & sample redirection file \\
    |cdocsfn2.tex| & sample redirection file \\
    |childdoc.pdf| & manual
\end{tabular}
\end{center}
%
The distribution consists of the files
|README.txt|, |childdoc.ins| and |childdoc.dtx|.
%
\begin{itemize}
\item
Run (pdf)\LaTeX{} on |childdoc.dtx|
to compile the manual |childdoc.pdf| (this file).
\item
Run \LaTeX{} on |childdoc.ins| to create the definitions file |childdoc.def|
and the sample |cdocsamp.tex| with include files
|cdocsch1.tex|, |cdocsch2.tex|, |cdocspt3.tex|, |cdocspt4.tex|,
|cdocsdrf.tex|, |cdocsfn1.tex|, |cdocsfn2.tex|.
Then copy the file |childdoc.def| to an appropriate directory of your \LaTeX{}
distribution, e.g.\ \textit{texmf-root}|/tex/latex/childdoc|.
\end{itemize}

%%%%%%%%%%%%%%%%%%%%%%%%%%%%%%%%%%%%%%%%%%%%%%%%%%%%%%%%%%%%%%%%%%%%%%%%%%%%%%%%
\subsection{Related CTAN Packages}

There are several other packages which offer a similar functionality:
%
\begin{itemize}
\item
The packages
\href{http://ctan.org/pkg/docmute}{\textsf{docmute}},
\href{http://ctan.org/pkg/includex}{\textsf{includex}} and
\href{http://ctan.org/pkg/standalone}{\textsf{standalone}}
provide commands to include only the document body of
a child file thus allowing both files to be compiled individually.
\item
The packages \href{http://ctan.org/pkg/subdocs}{\textsf{subdocs}}
and \href{http://ctan.org/pkg/subfiles}{\textsf{subfiles}}
provide structures in which the main and child documents can be
encapsulated and allowing them to be compiled individually.
The inclusion mechanism is different from the conventional |\include|.
\item
The package \href{http://ctan.org/pkg/combine}{\textsf{combine}}
is an elaborate solution to combine several documents into one.
\end{itemize}
%
See also the CTAN topic \href{http://ctan.org/topic/subdocs}{\textsf{subdocs}}
for further related packages.
The present package differs from the above solutions in that
a document structure constructed with the conventional |\include| mechanism
just needs two extra commands at the top of every file
such that all constituent files can be compiled individually.

%%%%%%%%%%%%%%%%%%%%%%%%%%%%%%%%%%%%%%%%%%%%%%%%%%%%%%%%%%%%%%%%%%%%%%%%%%%%%%%%
%\subsection{Feature Suggestions}
%
%The following is a list of features which may be useful for future
%versions of this package:
%%
%\begin{itemize}
%\item
%\ldots
%\end{itemize}

%%%%%%%%%%%%%%%%%%%%%%%%%%%%%%%%%%%%%%%%%%%%%%%%%%%%%%%%%%%%%%%%%%%%%%%%%%%%%%%%
\subsection{Revision History}

%%%%%%%%%%%%%%%%%%%%%%%%%%%%%%%%%%%%%%%%
\paragraph{v2.0:} 2018/12/30

\begin{itemize}
\item
immediate forward processing
\item
added |\childdocby| mechanism
\item
manual restructured
\end{itemize}

%%%%%%%%%%%%%%%%%%%%%%%%%%%%%%%%%%%%%%%%
\paragraph{v1.6:} 2018/01/17

\begin{itemize}
\item
application for development of include files
\item
corrections to manual
\end{itemize}

%%%%%%%%%%%%%%%%%%%%%%%%%%%%%%%%%%%%%%%%
\paragraph{v1.5:} 2017/05/21

\begin{itemize}
\item
more complete structuring introduced
\item
|\childdocof| introduced
\item
|\childdoc| renamed to |\childdocmain|
\item
|\childredirect| renamed to |\childdocforward| and |\childdocforwardprefix|
and functionality expanded
\end{itemize}

%%%%%%%%%%%%%%%%%%%%%%%%%%%%%%%%%%%%%%%%
\paragraph{v1.0:} 2017/04/27

\begin{itemize}
\item
manual and install package
\item
first version published on CTAN
\end{itemize}

%%%%%%%%%%%%%%%%%%%%%%%%%%%%%%%%%%%%%%%%
\paragraph{v0.6:} 2017/04/26

\begin{itemize}
\item
redirection mechanism added
\end{itemize}

%%%%%%%%%%%%%%%%%%%%%%%%%%%%%%%%%%%%%%%%
\paragraph{v0.5:} 2017/04/26

\begin{itemize}
\item
functionality in definition file
\end{itemize}


%%%%%%%%%%%%%%%%%%%%%%%%%%%%%%%%%%%%%%%%%%%%%%%%%%%%%%%%%%%%%%%%%%%%%%%%%%%%%%%%
%%%%%%%%%%%%%%%%%%%%%%%%%%%%%%%%%%%%%%%%%%%%%%%%%%%%%%%%%%%%%%%%%%%%%%%%%%%%%%%%
%%%%%%%%%%%%%%%%%%%%%%%%%%%%%%%%%%%%%%%%%%%%%%%%%%%%%%%%%%%%%%%%%%%%%%%%%%%%%%%%
\appendix

\settowidth\MacroIndent{\rmfamily\scriptsize 000\ }

 \DocInput{childdoc.dtx}

\end{document}
%</driver>
% \fi
%
% %%%%%%%%%%%%%%%%%%%%%%%%%%%%%%%%%%%%%%%%%%%%%%%%%%%%%%%%%%%%%%%%%%%%%%%%%%%%%%
% %%%%%%%%%%%%%%%%%%%%%%%%%%%%%%%%%%%%%%%%%%%%%%%%%%%%%%%%%%%%%%%%%%%%%%%%%%%%%%
% \section{Sample}
%\iffalse
%<*samplemain>
%\fi
%
% The following presents a sample document
% with two chapters, two parts, a title page,
% a compile flag as well as three forwarding files to set the flag.
% It consists of eight |.tex| files:
% \begin{center}
% \begin{tabular}{ll}
% |cdocsamp.tex|&main file\\
% |cdocsch1.tex|&include file for chapter 1\\
% |cdocsch2.tex|&include file for chapter 2\\
% |cdocspt3.tex|&include file for part 3\\
% |cdocspt4.tex|&include file for part 4\\
% |cdocsdrf.tex|&forwarding file for main file in draft mode\\
% |cdocsfi1.tex|&forwarding file for final version of chapter 1\\
% |cdocsfi2.tex|&forwarding file for final version of chapter 2\\
% \end{tabular}
% \end{center}
% Each of the eight files can be compiled directly by the \LaTeX{} compiler.
%
% %%%%%%%%%%%%%%%%%%%%%%%%%%%%%%%%%%%%%%
% \paragraph{Main File.}
%
% The main file is called |cdocsamp.tex|.
%
% Load the \textsf{childdoc} definitions and
% declare the filename for the main document:
%    \begin{macrocode}
\input{childdoc.def}
\childdocmain{}
%    \end{macrocode}

% Optional override for |\version| flag:
%    \begin{macrocode}
%%\ifchilddoc\else\providecommand{\version}{draft}\fi
%    \end{macrocode}

% Define the default values for the |\version| flag
% (|final| for the main file and |draft| for childs):
%    \begin{macrocode}
\ifchilddoc
\providecommand{\version}{draft}
\else
\providecommand{\version}{final}
\fi
%    \end{macrocode}

% Load the standard document class:
%    \begin{macrocode}
\documentclass[12pt]{article}
%    \end{macrocode}

% Start the document body:
%    \begin{macrocode}
\begin{document}
%    \end{macrocode}

% Declare a title page.
% Print title, part of document being processed and version flag:
%    \begin{macrocode}
\addtocounter{page}{-1}
\begin{center}
{\LARGE\bfseries{}childdoc example\par}
\vspace{1cm}
\ifchilddoc
\ifchilddocmanual part\else chapter\fi:
`\childdocname' of `\childdocjob'\par
\else
main document: `\childdocjob'\par
\fi
version: \version\par
\end{center}
\newpage
%    \end{macrocode}

% Manually include selected file,
% otherwise process as usual:
%    \begin{macrocode}
\ifchilddocmanual
\section*{part `\childdocname'}
\input{\childdocname}
\else
%    \end{macrocode}

% Include the two chapters:
%    \begin{macrocode}
\include{cdocsch1}
\include{cdocsch2}
%    \end{macrocode}

% Include the two parts unless only chapters should be displayed:
%    \begin{macrocode}
\ifchilddoc\else
\section{part three}
\input{cdocspt3}
\section{part four}
\input{cdocspt4}
\fi
%    \end{macrocode}

% Process as usual until here:
%    \begin{macrocode}
\fi
%    \end{macrocode}

% End of document body:
%    \begin{macrocode}
\end{document}
%    \end{macrocode}
%\iffalse
%</samplemain>
%\fi
%
% %%%%%%%%%%%%%%%%%%%%%%%%%%%%%%%%%%%%%%
% \paragraph{Chapter Include Files.}
%
% The include files are called |cdocsch1.tex| and |cdocsch2.tex|.
%
%\iffalse
%<*samplechap1|samplechap2>
%\fi

% Optional override for |\version| flag:
%    \begin{macrocode}
%%\providecommand{\version}{final}
%    \end{macrocode}

% Include the main document:
%    \begin{macrocode}
\input{childdoc.def}
\childdocof{cdocsamp}
%    \end{macrocode}

%\iffalse
%</samplechap1|samplechap2>
%\fi
%
%\iffalse
%<*samplechap1>
%\fi
% Some text for chapter 1:
%    \begin{macrocode}
\section{one}
some text in chapter one
%    \end{macrocode}

%\iffalse
%</samplechap1>
%\fi
% Some text for chapter 2:
%\iffalse
%<*samplechap2>
%\fi
%    \begin{macrocode}
\section{two}
more text in chapter two
%    \end{macrocode}

%\iffalse
%</samplechap2>
%\fi
%
% %%%%%%%%%%%%%%%%%%%%%%%%%%%%%%%%%%%%%%
% \paragraph{Part Include Files.}
%
% The include files are called |cdocspt3.tex| and |cdocspt4.tex|.
%
%\iffalse
%<*samplepart3|samplepart4>
%\fi

% Optional override for |\version| flag:
%    \begin{macrocode}
%%\providecommand{\version}{final}
%    \end{macrocode}

% Include the main document:
%    \begin{macrocode}
\input{childdoc.def}
\childdocby{cdocsamp}
%    \end{macrocode}

%\iffalse
%</samplepart3|samplepart4>
%\fi
%
%\iffalse
%<*samplepart3>
%\fi
% Some text for part 3:
%    \begin{macrocode}
some text in part three
%    \end{macrocode}

%\iffalse
%</samplepart3>
%\fi
% Some text for part 4:
%\iffalse
%<*samplepart4>
%\fi
%    \begin{macrocode}
more text in part four
%    \end{macrocode}

%\iffalse
%</samplepart4>
%\fi
%
% %%%%%%%%%%%%%%%%%%%%%%%%%%%%%%%%%%%%%%
% \paragraph{Forwarding for a Complete Draft.}
%
% The following forwarding file |cdocsdrf.tex|
% compiles the main document in draft mode:
%\iffalse
%<*sampledraft>
%\fi
%    \begin{macrocode}
\def\version{draft}
\input{childdoc.def}
\childdocforward{cdocsamp}
%    \end{macrocode}

%\iffalse
%</sampledraft>
%\fi
%
% %%%%%%%%%%%%%%%%%%%%%%%%%%%%%%%%%%%%%%
% \paragraph{Forwarding for Final Version of the Chapters.}
%
% The following forwarding files |cdocsfn1.tex| and |cdocsfn2.tex|
% (with identical content)
% compile the final versions of the child documents
% |cdocsch1.tex| and |cdocsch2.tex|, respectively:
%\iffalse
%<*samplefinal>
%\fi
%    \begin{macrocode}
\def\version{final}
\input{childdoc.def}
\childdocforwardprefix[cdocsamp]{cdocsfn}{cdocsch}
%    \end{macrocode}

%\iffalse
%</samplefinal>
%\fi
%
% %%%%%%%%%%%%%%%%%%%%%%%%%%%%%%%%%%%%%%
% \paragraph{Command Line Processing.}
%
% The following three command lines generate the output files
% |cdocscld|, |cdocscl1| and |cdocscl2|
% which should be identical to
% |cdocsdrf|, |cdocsch1| and |cdocsfn2|, respectively:
% \begin{center}
% \begin{tabular}{l}
% |latex -jobname cdocscld \|\\
% |  "\def\version{draft}\input{childdoc.def}\childdocforward{cdocsamp}"|\\
% |latex -jobname cdocscl1 \|\\
% |  "\input{childdoc.def}\childdocforward[cdocsamp]{cdocsch1}"|\\
% |latex -jobname cdocscl2 \|\\
% |  "\def\version{final}\input{childdoc.def}\childdocforward{cdocsch2}"|
% \end{tabular}
% \end{center}
% Note that the trailing backslash on each first line
% merely continues the input to the second line
% (for convenient cut ant paste).
% Furthermore, the command |latex| can be replaced by any
% of its alternative versions such as |pdflatex|.
%
% %%%%%%%%%%%%%%%%%%%%%%%%%%%%%%%%%%%%%%%%%%%%%%%%%%%%%%%%%%%%%%%%%%%%%%%%%%%%%%
% %%%%%%%%%%%%%%%%%%%%%%%%%%%%%%%%%%%%%%%%%%%%%%%%%%%%%%%%%%%%%%%%%%%%%%%%%%%%%%
% \section{Implementation}
%\iffalse
%<*package>
%\fi
%
% This section describes the definitions file |childdoc.def|.

% The definitions cannot be loaded using |\usepackage| or |\RequirePackage|
% which has a mechanism to prevent loading a style file more than once.
% When loading the definitions by means of |\input|
% multiple instances have to be prevented manually:
%\iffalse
%This code needs to be before the `\ProvidesFile' directive
%which is defined at the beginning of this file.
%Therefore it is also placed there and commented out here.
%</package>
%<*discard>
%\fi
%    \begin{macrocode}
\ifdefined\childdocmain\endinput\fi
%    \end{macrocode}
%\iffalse
%</discard>
%<*package>
%\fi
%
% \macro{\ifchilddoc}
% \macro{\ifchilddocmanual}
% The conditional |\ifchilddoc| tells whether a
% child (true) or main (false) document is being compiled.
% The conditional |\ifchilddocmanual| tells whether
% the |\includeonly| mechanism is used (false) or
% the selection of child files must be performed manually (true).
% The definitions initialise to false:
%    \begin{macrocode}
\newif\ifchilddoc
\newif\ifchilddocmanual
%    \end{macrocode}

% \macro{\childdocname}
% \macro{\childdocjob}
% The macro |\childdocname| stores the name of the main document
% to be compiled. The macro |\childdocjob| stores the name of
% the document on which the \LaTeX{} compiler was originally invoked.
% The content of |\jobname| cannot be compared
% to filenames specified in the source due to different catcodes.
% The following code rescans |\jobname|, stores the result
% in |\childdocname| and saves a copy in |\childdocjob|:
%    \begin{macrocode}
\edef\childdocname{\scantokens\expandafter{\jobname\noexpand}}
\let\childdocjob\childdocname
%    \end{macrocode}

% \macro{\childdocdisable}
% The macro |\childdocdisable| prevents the main file
% from being processed more than once.
% At this stage, the main document command |\childdocmain|
% is assumed to be called once again where it should do nothing.
% Any subsequent call to it should prevent
% a secondary processing of the main document
% It overwrites the forwarding commands
% |\childdocof| and |\childdocforward|
% with empty macros to prevent further inclusions of the main document:
%    \begin{macrocode}
\newcommand{\childdocdisable}
{
  \renewcommand{\childdocmain}[1]{\renewcommand{\childdocmain}[1]{\endinput}}
  \renewcommand{\childdocof}[1]{}
  \renewcommand{\childdocby}[2][]{}
  \renewcommand{\childdocforward}[2][]{}
  \renewcommand{\childdocdisable}{}
}
%    \end{macrocode}

% \macro{\childdocmain}
% The macro |\childdocmain| is to be called at the top of the main file
% with nothing or the main filename (without extension) as argument.
% First, it breaks loops.
% If the argument is not empty and does not match |\childdocname|
% (which is set by the first inclusion of |childdoc.def|),
% |\ifchilddoc| is set to true, |\includeonly| is applied to the child file
% and |\jobname| is set to the main file
% (for proper handling of |.aux| files):
%    \begin{macrocode}
\newcommand{\childdocmain}[1]
{
  \childdocdisable\childdocmain{}
  \if?#1?\else
    \begingroup
      \def\childdoctmp{#1}
      \ifx\childdoctmp\childdocname
        \def\childdoctmp{}
      \else
        \def\childdoctmp
        {
          \childdoctrue
          \includeonly{\childdocname}
          \def\childdocjob{#1}
          \def\jobname{#1}
        }
      \fi
      \expandafter
    \endgroup
    \childdoctmp
  \fi
}
%    \end{macrocode}

% \macro{\childdocof}
% The command |\childdocof| redirects
% compilation to the main file |#1|.
%    \begin{macrocode}
\newcommand{\childdocof}[1]
{
  \childdocdisable
  \childdoctrue
  \includeonly{\childdocname}
  \def\jobname{#1}
  \def\childdocjob{#1}
  \input{#1}
}
%    \end{macrocode}

% \macro{\childdocby}
% The command |\childdocby| ....
%    \begin{macrocode}
\newcommand{\childdocby}[2][]
{
  \childdocdisable
  \childdoctrue
  \childdocmanualtrue
  \if?#1?\else
    \def\jobname{#2}
  \fi
  \def\childdocjob{#2}
  \input{#2}
  \endinput
}
%    \end{macrocode}

% \macro{\childdocforward}
% The command |\childdocforward| redirects
% compilation to the main file or
% (if the optional argument is given) a child file.
% Parameters are set as if the main file
% or a child file starting with |\childdocof| was compiled.
% Then compilation is handed over to the main file:
%    \begin{macrocode}
\newcommand{\childdocforward}[2][]
{
  \begingroup
    \if?#1?
      \def\childdoctmp
      {
        \def\childdocname{#2}
        \def\childdocjob{#2}
        \def\jobname{#2}
        \input{#2}
        \endinput
      }
    \else
      \def\childdoctmp
      {
        \childdocdisable
        \def\childdocname{#2}
        \childdoctrue
        \includeonly{#2}
        \def\childdocjob{#1}
        \def\jobname{#1}
        \input{#1}
        \endinput
      }
    \fi
    \expandafter
  \endgroup
  \childdoctmp
}
%    \end{macrocode}

% \macro{\childdocforwardprefix}
% The command |\childdocforwardprefix| redirects
% compilation to the main or a child file by means of a pattern.
% The prefix |#1| in the current filename is replaced by |#2|
% and the suffix of the current filename is kept
% (it is assumed that the filename does not contain the substring `|~~~|'
% which is used as a delimiter).
% Compilation is handed over to the new file by |\childdocforward|:
%    \begin{macrocode}
\newcommand{\childdocforwardprefix}[3][]
{
  \begingroup
    \def\childdocextract #2##1~~~{\def\childdoctmp{\childdocforward[#1]{#3##1}}}
    \expandafter\childdocextract\childdocname~~~
    \expandafter
  \endgroup
  \childdoctmp
}
%    \end{macrocode}

% \macro{\childdoc}
% The deprecated macro |\childdoc| is a legacy version of |\childdocmain|:
%    \begin{macrocode}
\newcommand{\childdoc}{\childdocmain}
%    \end{macrocode}

% \macro{\childdocredirect}
% The deprecated macro |\childdocredirect| is a legacy version
% of |\childdocforward| and |\childdocforwardprefix|:
%    \begin{macrocode}
\newcommand{\childdocredirect}[2][]
{
  \begingroup
    \if?#1?
      \def\childdoctmp{\childdocforward{#2}}
    \else
      \def\childdoctmp{\childdocforwardprefix{#1}{#2}}
    \fi
    \expandafter
  \endgroup
  \childdoctmp
}
%    \end{macrocode}

%\iffalse
%</package>
%\fi
%
\endinput
|\\
|\childdocby{|\textit{main}|}|\\
\end{tabular}
\end{center}
%
Both forms have slightly different effects as described above.
The main file is prepared as usual, see \secref{sec:include}.

%%%%%%%%%%%%%%%%%%%%%%%%%%%%%%%%%%%%%%%%%%%%%%%%%%%%%%%%%%%%%%%%%%%%%%%%%%%%%%%%
\subsection{Legacy Detection}
\label{sec:detection}

The directive |\childdocmain| in the main file can detect
whether the complete document or merely a child is to be compiled
even without using the directive |\childdocof|.
This method is deprecated because it is less robust
and there is no compelling reason to use it;
it is merely provided for backward compatibility
and it may be removed in future versions.

If the detection mechanism is to be used,
it is mandatory to correctly specify
the filename of the main file as the argument of |\childdocmain|:
%
\begin{center}
\begin{tabular}{l}
|% \iffalse
%
% childdoc.dtx Copyright (C) 2017-2018 Niklas Beisert
%
% This work may be distributed and/or modified under the
% conditions of the LaTeX Project Public License, either version 1.3
% of this license or (at your option) any later version.
% The latest version of this license is in
%   http://www.latex-project.org/lppl.txt
% and version 1.3 or later is part of all distributions of LaTeX
% version 2005/12/01 or later.
%
% This work has the LPPL maintenance status `maintained'.
%
% The Current Maintainer of this work is Niklas Beisert.
%
% This work consists of the files childdoc.dtx and childdoc.ins
% and the derived files childdoc.def and cdocsamp.tex with
% cdocsch1.tex, cdocsch2.tex, cdocsdrf.tex, cdocsfn1.tex, cdocsfn2.tex.
%
%<package>\ifdefined\childdocmain\endinput\fi
%<package>\ProvidesFile{childdoc.def}[2018/12/30 v2.0 child document driver]
%<samplemain>\ProvidesFile{cdocsamp.tex}[2018/12/30 v2.0 sample for childdoc]
%<*driver>
%\ProvidesFile{childdoc.drv}[2018/12/30 v2.0 childdoc reference manual file]
\PassOptionsToClass{10pt,a4paper}{article}
\documentclass{ltxdoc}

\usepackage[margin=35mm]{geometry}
\usepackage{hyperref}
\usepackage{hyperxmp}
\usepackage[usenames]{color}

\hypersetup{colorlinks=true}
\hypersetup{pdfstartview=FitH}
\hypersetup{pdfpagemode=UseNone}
\hypersetup{pdfsource={}}
\hypersetup{pdflang={en-UK}}
\hypersetup{pdfcopyright={Copyright 2017-2018 Niklas Beisert.
  This work may be distributed and/or modified under the
  conditions of the LaTeX Project Public License, either version 1.3
  of this license or (at your option) any later version.}}
\hypersetup{pdflicenseurl={http://www.latex-project.org/lppl.txt}}
\hypersetup{pdfcontactaddress={ETH Zurich, ITP, HIT K,
  Wolfgang-Pauli-Strasse 27}}
\hypersetup{pdfcontactpostcode={8093}}
\hypersetup{pdfcontactcity={Zurich}}
\hypersetup{pdfcontactcountry={Switzerland}}
\hypersetup{pdfcontactemail={nbeisert@itp.phys.ethz.ch}}
\hypersetup{pdfcontacturl={http://people.phys.ethz.ch/\xmptilde nbeisert/}}

\newcommand{\secref}[1]{\hyperref[#1]{section \ref*{#1}}}

\parskip1ex
\parindent0pt
\let\olditemize\itemize
\def\itemize{\olditemize\parskip0pt}

\begin{document}

\title{The \textsf{childdoc} Package}
\hypersetup{pdftitle={The childdoc Package}}
\author{Niklas Beisert\\[2ex]
  Institut f\"ur Theoretische Physik\\
  Eidgen\"ossische Technische Hochschule Z\"urich\\
  Wolfgang-Pauli-Strasse 27, 8093 Z\"urich, Switzerland\\[1ex]
  \href{mailto:nbeisert@itp.phys.ethz.ch}
  {\texttt{nbeisert@itp.phys.ethz.ch}}}
\hypersetup{pdfauthor={Niklas Beisert}}
\hypersetup{pdfsubject={Manual for the LaTeX2e Package childdoc}}
\date{30 December 2018, \textsf{v2.0}}
\maketitle

\begin{abstract}\noindent
\textsf{childdoc} is a \LaTeXe{} package
that enables the direct compilation
of document sections included by |\include|
to individual files.
\end{abstract}

\begingroup
\parskip0ex
\tableofcontents
\endgroup

%%%%%%%%%%%%%%%%%%%%%%%%%%%%%%%%%%%%%%%%%%%%%%%%%%%%%%%%%%%%%%%%%%%%%%%%%%%%%%%%
%%%%%%%%%%%%%%%%%%%%%%%%%%%%%%%%%%%%%%%%%%%%%%%%%%%%%%%%%%%%%%%%%%%%%%%%%%%%%%%%
\section{Introduction}

\LaTeX{} provides a mechanism to structure a large document (such as a book)
into a main file and several child files (containing the chapters)
using the |\include| command.
This mechanism is beneficial for documents
which span hundreds of pages in order to
make the source file(s) more manageable.
Moreover, compilation can be restricted to
selected child files by means of the |\includeonly| command.
The latter feature can be used to reduce the compilation time while editing
(this was significantly more useful in the earlier days of \LaTeX{})
or to generate a smaller document which is easier to navigate.
Another application of |\includeonly| is to generate
documents consisting of selected parts of the complete document.

However, there are a few drawbacks of the plain |\include| mechanism:
\begin{itemize}
\item
The child files cannot be compiled on their own,
they can only be compiled via the main file.
A naive editing environment
(such as a text editor with an option
to have the current file processed by \LaTeX)
may require one to switch to the main file before compiling;
attempting to compile the child file produces errors.
\item
The main file must be modified (each time)
to adjust the |\includeonly| command
to the present needs. This easily leaves the main file in a messy state.
\item
The generated document will always carry the filename
of the main document. This is inconvenient if
several child files are to be compiled and
to be kept for distribution.
\end{itemize}

The present package provides a simple interface
to make child files individually compilable by \LaTeX{}.
Compiling a child file then has the same effect as compiling
the main file with an |\includeonly| command
to select the appropriate child.
Moreover the generated document will carry the name of the child
rather than the main file.
This resolves all three above issues.

This feature is meant to make the editing of books,
thesis documents and lecture notes somewhat more convenient.
However, the package can also be used efficiently for
composing a series of documents (such as exercise sheets)
which are typically distributed individually.
It then assists the author in generating the individual documents
(potentially in different versions)
as well as a document containing the collected series.
Another application is in developing style files
or other kinds of included material
where compilation of the style file could redirect
to a sample or test file.

%%%%%%%%%%%%%%%%%%%%%%%%%%%%%%%%%%%%%%%%%%%%%%%%%%%%%%%%%%%%%%%%%%%%%%%%%%%%%%%%
%%%%%%%%%%%%%%%%%%%%%%%%%%%%%%%%%%%%%%%%%%%%%%%%%%%%%%%%%%%%%%%%%%%%%%%%%%%%%%%%
\section{Usage}

First of all, the package \textsf{childdoc} is \emph{not} a standard
\LaTeXe{} |.sty| style file! Therefore it needs to be invoked in
a non-standard way.

%%%%%%%%%%%%%%%%%%%%%%%%%%%%%%%%%%%%%%%%%%%%%%%%%%%%%%%%%%%%%%%%%%%%%%%%%%%%%%%%
\subsection{Included Files}
\label{sec:include}

%%%%%%%%%%%%%%%%%%%%%%%%%%%%%%%%%%%%%%%%
\DescribeMacro{\childdocmain}
To use the package, add the commands
\begin{center}
\begin{tabular}{l}
|\input{childdoc.def}|\\
|\childdocmain{}|\\
\end{tabular}
\end{center}
at the very top of the main \LaTeX{} file,
in particular \emph{before} the |\documentclass| statement!
The argument of |\childdocmain| should be left empty
(but it must be present).

%%%%%%%%%%%%%%%%%%%%%%%%%%%%%%%%%%%%%%%%
\DescribeMacro{\childdocof}
Furthermore, add the commands
\begin{center}
\begin{tabular}{l}
|\input{childdoc.def}|\\
|\childdocof{|\textit{main}|}|\\
\end{tabular}
\end{center}
at the top of every child file \textit{child}
which is included by |\include{|\textit{child}|}|
from within the main file
(or at least for those files to be compiled individually).
The argument \textit{main} must be the filename of the main file.

There are a couple of
considerations in setting up the main and child documents:

%%%%%%%%%%%%%%%%%%%%%%%%%%%%%%%%%%%%%%%%
\paragraph{Restrictions.}

Please note the following restrictions:
\begin{itemize}
\item
|\childdocmain| must be called with one argument \textit{main}
to ensure compatibility with earlier version of the package.
It must either be empty (|\childdocmain{}|)
or precisely match the filename of the main file in which it is specified.
See \secref{sec:detection} for further information.
\item
The filename \textit{main} must be specified without the |.tex| extension.
\item
The filename \textit{main} is case sensitive
(even in case-insensitive file systems)
due to internal string comparison.
\item
The argument \textit{main} should be fully expanded, it cannot be a macro.
\item
Subdirectories and special characters should be avoided in filenames.
\item
The command |\childdocmain{|\textit{main}|}| must be followed by a whitespace.
It should not be followed immediately by another command
or by a comment mark `|%|'.
This is because the \TeX{} parser reads the token immediately following
the argument of |\childdocmain| and puts it
at the beginning of every child section;
however, a white\-space is ignored.
\end{itemize}

%%%%%%%%%%%%%%%%%%%%%%%%%%%%%%%%%%%%%%%%
\paragraph{Content of Main File.}

It is advisable to place all content in the child files included by |\include|.
Any output contained in the main file will appear in all child documents
unless suppressed manually;
it cannot be suppressed automatically by the |\includeonly| directive
and thus should normally be avoided.
A method to include some content in the main file
by means of conditional processing is described in \secref{sec:conditional}.

%%%%%%%%%%%%%%%%%%%%%%%%%%%%%%%%%%%%%%%%
\paragraph{Page Numbering.}

When only a part of the document is compiled,
the appropriate numbering of pages
(as well as other status parameters)
is determined from the |.aux| files.
The latter contain information from previous passes.
However this information needs to propagate through
all intermediate child documents.
Therefore the page numbering in child documents may well
be inconsistent until the complete document is compiled at least once.

A useful (if unconventional) way to always ensure a consistent
page numbering is to restart the numbering in each child document
and denote the pages by `\textit{child}|.|\textit{page}'
where \textit{child} represents the chapter/section number of the child file.
This can be achieved by the command
|\numberwithin{page}{|\textit{child}|}|
of the \textsf{amsmath} package
where \textit{child} can be |chapter| or |section|
depending on the chosen structuring.
Alternatively, one can modify the macro |\thepage| appropriately
and reset the counter |page| at the start of each child file.

%%%%%%%%%%%%%%%%%%%%%%%%%%%%%%%%%%%%%%%%%%%%%%%%%%%%%%%%%%%%%%%%%%%%%%%%%%%%%%%%
\subsection{Conditional Processing}
\label{sec:conditional}

The package provides a mechanism to compile different versions
of a document. To customise the versions further some conditional processing
can come in handy to distinguish which version is being compiled.
The package provides two macros to describe the compilation context:

%%%%%%%%%%%%%%%%%%%%%%%%%%%%%%%%%%%%%%%%
\DescribeMacro{\ifchilddoc}
The conditional |\ifchilddoc| distinguishes between the compilation of
child documents and the main document:
%
\begin{center}
|\ifchilddoc |\textit{child-code}| |[|\||else |\textit{main-code}]| \||fi|
\end{center}

%%%%%%%%%%%%%%%%%%%%%%%%%%%%%%%%%%%%%%%%
\DescribeMacro{\childdocname}
\DescribeMacro{\childdocjob}
The macro |\childdocname| contains the filename (without extension)
of the main or child file being processed.
Note that |\childdocjob| will always contain the name of the main file.

%%%%%%%%%%%%%%%%%%%%%%%%%%%%%%%%%%%%%%%%
\paragraph{Title Page.}

Conditional processing can be used to include a title or banner page
in the main document when proper precautions are taken.
Importantly, the code in the main file should ensure that the page counter
(as well as other status parameters which are stored in the |.aux| files)
takes the same value after the conditional processing.
Otherwise the page numbers may take divergent values
depending on which part is compiled.

For example, a title page could be declared by:
%
\begin{center}
\begin{tabular}{l}
|\ifchilddoc\||else|\\
|\addtocounter{page}{-1}|\\
\textit{code for title page}\\
|\newpage|\\
|\||fi|
\end{tabular}
\end{center}
%
A banner page for the child documents can be generated by:
%
\begin{center}
\begin{tabular}{l}
|\ifchilddoc|\\
|\addtocounter{page}{-1}|\\
\textit{code for banner page}\\
|\newpage|\\
|\||fi|
\end{tabular}
\end{center}
%
Here one could write a message such as:
\begin{center}
|This is the part \childdocname{} of \childdocjob{}.|
\end{center}

%%%%%%%%%%%%%%%%%%%%%%%%%%%%%%%%%%%%%%%%%%%%%%%%%%%%%%%%%%%%%%%%%%%%%%%%%%%%%%%%
\subsection{Flags}
\label{sec:flags}

The package makes it easy to generate different versions
of the main or child documents.
To this end compilation flags can be defined
and assigned different default values.
They will be particularly useful in conjunction
with the forwarding mechanism described in \secref{sec:forward}.

For example, it may be useful to have a flag |\version|
which can be set to |draft| or |final|.
The document source will contain some conditional code
depending on the value of |\version|.
Suppose further, the flag should default to |final| for the main file
and to |draft| for child files
which is a natural assignment for editing the document.
This is achieved by placing the following code
in the preamble of the main document
(below the |\childdocmain| directive):
%
\begin{center}
\begin{tabular}{l}
|\ifchilddoc|\\
|\providecommand{\version}{draft}|\\
|\||else|\\
|\providecommand{\version}{final}|\\
|\||fi|
\end{tabular}
\end{center}
%
The definition by |\providecommand| makes sure
that previous definitions are not overwritten.
Further statements |\providecommand{\version}{...}|
can thus be added before the above code to override it.

For the main file, one might add a line
(between |\childdocmain| and the above block)
%
\begin{center}
|%\ifchilddoc\||else\providecommand{\version}{draft}\||fi|
\end{center}
%
which can be uncommented to produce a draft version.
Likewise one can add a line to the very top of a child file
(above the |\childdocof{|\textit{main}|}| directive)
%
\begin{center}
|%\providecommand{\version}{final}|
\end{center}
%
which can be uncommented to produce the final version of this child document.

%%%%%%%%%%%%%%%%%%%%%%%%%%%%%%%%%%%%%%%%%%%%%%%%%%%%%%%%%%%%%%%%%%%%%%%%%%%%%%%%
\subsection{Forwarding}
\label{sec:forward}

Different versions of the main or child documents
using compilation flags as described in \secref{sec:flags}
can be (permanently) stored in different files
for convenient compilation, viewing and distribution.
To this end, the package defines a command
to pass on compilation to a different file:

%%%%%%%%%%%%%%%%%%%%%%%%%%%%%%%%%%%%%%%%
\DescribeMacro{\childdocforward}
The command |\childdocforward| redirects processing to
another source file:
%
\begin{center}
\begin{tabular}{l}
|\input{childdoc.def}|\\
|\childdocforward[|\textit{main}|]{|\textit{dest}|}|\\
\end{tabular}
\end{center}
%
The argument \textit{dest} is the destination file
(without extension).
It should be the main file or one of the child files.
Note that further \textsf{childdoc} directives
such as |\childdocof| and |\childdocforward|
in the indicated file will be processed in this form.
The optional argument \textit{main}
passes on directly to the main file \textit{main}
while pretending to compile the child \textit{dest}.
This form behaves as if \textit{dest}
issues |\childdocof{|\textit{main}|}| right away,
and no further \textsf{childdoc} directives will be processed.

%%%%%%%%%%%%%%%%%%%%%%%%%%%%%%%%%%%%%%%%
\DescribeMacro{\...prefix}
In the alternative form |\childdocforwardprefix|,
%
\begin{center}
\begin{tabular}{l}
|\input{childdoc.def}|\\
|\childdocforwardprefix[|\textit{main}|]{|\textit{prefix}|}{|\textit{dest}|}|
\end{tabular}
\end{center}
%
the destination file is determined by a pattern
depending on the current file:
To make this work, the current file must be called
`{\textit{prefix}\hspace{0.2em}\textit{suffix}}'
with \textit{prefix} matching precisely the argument.
Processing is then passed on to the file
`{\textit{dest}\hspace{0.2em}\textit{suffix}}'.
Surely, the same effect is achieved by
directly specifying the
argument `{\textit{dest}\hspace{0.2em}\textit{suffix}}'
in the first form.
However, that requires to set up a different file
for each child. With the alternative form of the command
all these files can have exactly the same content
which simplifies setting them up and maintaining them.

For example, the following file |draft.tex|
with a compilation flag |\version| as described in \secref{sec:flags}
compiles the main document as a draft:
%
\begin{center}
\begin{tabular}{l}
|\def\version{draft}|\\
|\input{childdoc.def}|\\
|\childdocforward{|\textit{main}|}|
\end{tabular}
\end{center}
%
Likewise, the following files |final|\textit{nn}|.tex|
compile the final version of the child document
|child|\textit{nn}|.tex|:
%
\begin{center}
\begin{tabular}{l}
|\def\version{final}|\\
|\input{childdoc.def}|\\
|\childdocforwardprefix{final}{child}|
\end{tabular}
\end{center}
%

Note that when several versions of a main file and/or of each child file
are to be generated, it may be convenient to set up a |Makefile| or
shell script to automatise the process.

%%%%%%%%%%%%%%%%%%%%%%%%%%%%%%%%%%%%%%%%%%%%%%%%%%%%%%%%%%%%%%%%%%%%%%%%%%%%%%%%
\subsection{Command Line Processing}
\label{sec:commandline}

The effect of redirection files can also be achieved by invoking
the \LaTeX{} compiler with a more elaborate command line.
Most conveniently this should be done as part
of a shell script or a |Makefile|.

When using \textsf{childdoc} in the main file, the following
command lines effectively perform a redirection
(note that depending on the shell being used,
backslashes may have to be doubled: `|\|' $\to$ `|\\|'):
%
\begin{center}
|... -jobname "|\textit{target}|" |\\|"|[\textit{flags}]%
|\input{childdoc.def}\childdocforward[|\textit{main}|]{|\textit{dest}|}"|
\end{center}
%
Here \textit{target} is the name of the output file,
\textit{main} is the name of the main file
and \textit{dest} is the name of the main or child file to be processed
(all filenames without extensions).
The optional argument \textit{main} can be omitted
if \textit{main} matches \textit{dest}.
Optionally, compilation \textit{flags} can be defined via |\def| commands.
This command line makes the \TeX{} engine believe
it is compiling the file \textit{target}
whose content is specified as the latter parameter.
The provided code then forwards the processing to
\textit{main} or \textit{dest} as described in \secref{sec:forward}.

%%%%%%%%%%%%%%%%%%%%%%%%%%%%%%%%%%%%%%%%%%%%%%%%%%%%%%%%%%%%%%%%%%%%%%%%%%%%%%%%
\subsection{Include by Input}
\label{sec:input}

Including child documents by |\include| has some restrictions by design.
Most notably, the content of a child document always occupies
its own set of pages; pages cannot be shared between child documents.
Usually, this behaviour makes perfect sense
because each child document contain an essential part of the document.
However, in some situations it may be desirable to compose
a document from a collection of parts
without having mandatory page breaks between then.
For this case, the package
provides a mechanism to include parts
by |\input| which can also be processed individually.
However, by construction this mechanism
requires manual handling of the content to be output.

%%%%%%%%%%%%%%%%%%%%%%%%%%%%%%%%%%%%%%%%
\DescribeMacro{\ifchilddocmanual}
The main file should be prepared as usual, see \secref{sec:include}.
However, the document body must make a distinction
between processing of an individual part and of the main document, e.g.:
%
\begin{center}
\begin{tabular}{l}
|\ifchilddocmanual|\\
|\input{\childdocname}|\\
|\||else|\\
\textit{document body with }|\input{|\textit{part}|}|\\
|\||fi|
\end{tabular}
\end{center}
%
The conditional |\ifchilddocmanual| is true whenever
a part to be included by |\input| is being compiled,
and the name of the part is stored in |\childdocname|.

%%%%%%%%%%%%%%%%%%%%%%%%%%%%%%%%%%%%%%%%
\DescribeMacro{\childdocby}
Each part to be included by |\input| should start with:
%
\begin{center}
\begin{tabular}{l}
|\input{childdoc.def}|\\
|\childdocby{|\textit{main}|}|\\
\end{tabular}
\end{center}
%
The directive |\childdocby| is similar to |\childdocof|
described in \secref{sec:include},
but the subsequent selection of content must be done manually.
To that end, both |\ifchilddoc| and |\ifchilddocmanual|
will be true upon processing of a part,
and the name of the part is stored in |\childdocname|.
Note that |\jobname| will be set to the filename of the current part
so that each part receives an individual |.aux| file
that does not interfere with the |.aux| file(s) of the main document.
This behaviour can be altered by the alternative form
|\childdocby[*]{|\textit{main}|}| (with a non-empty optional argument)
which uses the |.aux| file of the main document
by setting |\jobname| to \textit{main}.

%%%%%%%%%%%%%%%%%%%%%%%%%%%%%%%%%%%%%%%%%%%%%%%%%%%%%%%%%%%%%%%%%%%%%%%%%%%%%%%%
\subsection{Driver Development}
\label{sec:driver}

The \textsf{childdoc} mechanism can also be use for the development
of definition files such as \LaTeX{} styles or classes.
This case differs from the above setup with multiple parts
included by |\include| in that no |\includeonly| should be invoked.
This can be achieved by starting the include file
(before |\ProvidesPackage|) with:
%
\begin{center}
\begin{tabular}{l}
|\input{childdoc.def}|\\
|\childdocforward{|\textit{main}|}|\\
\end{tabular}
\end{center}
%
or alternatively with:
%
\begin{center}
\begin{tabular}{l}
|\input{childdoc.def}|\\
|\childdocby{|\textit{main}|}|\\
\end{tabular}
\end{center}
%
Both forms have slightly different effects as described above.
The main file is prepared as usual, see \secref{sec:include}.

%%%%%%%%%%%%%%%%%%%%%%%%%%%%%%%%%%%%%%%%%%%%%%%%%%%%%%%%%%%%%%%%%%%%%%%%%%%%%%%%
\subsection{Legacy Detection}
\label{sec:detection}

The directive |\childdocmain| in the main file can detect
whether the complete document or merely a child is to be compiled
even without using the directive |\childdocof|.
This method is deprecated because it is less robust
and there is no compelling reason to use it;
it is merely provided for backward compatibility
and it may be removed in future versions.

If the detection mechanism is to be used,
it is mandatory to correctly specify
the filename of the main file as the argument of |\childdocmain|:
%
\begin{center}
\begin{tabular}{l}
|\input{childdoc.def}|\\
|\childdocmain{|\textit{main}|}|\\
\end{tabular}
\end{center}
%
If |\jobname| does not match the argument \textit{main} of |\childdocmain|,
it is assumed that |\jobname| points to the child file to be compiled.
When using |\childdocmain| with the main file specified as argument,
it suffices to start a child file
with just |\input{|\textit{main}|}|
without loading of the package and using |\childdocof|.
If instead all processing is done
with the appropriate \textsf{childdoc} directives,
the argument of \textit{main} of |\childdocmain| can be empty.

An alternative version of the command line processing described
in \secref{sec:commandline} using the detection mechanism reads:
%
\begin{center}
|... -jobname "|\textit{target}|" "|[\textit{flags}]%
[|\def\jobname{|\textit{dest}|}|]|\input{|\textit{main}|}"|
\end{center}

%%%%%%%%%%%%%%%%%%%%%%%%%%%%%%%%%%%%%%%%%%%%%%%%%%%%%%%%%%%%%%%%%%%%%%%%%%%%%%%%
\subsection{Manual Code}
\label{sec:manual}

In case one cannot be certain whether the definitions file |childdoc.def|
is installed on the target \TeX{} distribution
and one prefers not to ship it,
it is conceivable to paste a few relevant commands into the sources.

To that end, drop all statements |\input{childdoc.def}|
and perform the replacements as outlined below.
Instead of |\childdocmain{|\textit{main}|}| add the following code
to the top of the main file:
%
\begin{center}
\begin{tabular}{l}
|\||ifdefined\childdocname\endinput\||fi\newif\ifchilddoc|\\
|\edef\childdocname{\scantokens\expandafter{\jobname\noexpand}}|\\
|\def\childdocmain{|\textit{main}|}\||ifx\childdocmain\childdocname\||else|\\
|\childdoctrue\includeonly{\childdocname}\let\jobname\childdocmain\||fi|\\
\end{tabular}
\end{center}
%
Instead of |\childdocof{|\textit{main}|}| just include the main file
at the top of each child file:
%
\begin{center}
|\input{|\textit{main}|}|
\end{center}
%
A simple redirection |\childdocforward{|\textit{dest}|}| is achieved by:
%
\begin{center}
|\def\jobname{|\textit{dest}|}\input{\jobname}|
\end{center}
%
The redirection with prefix
|\childdocforwardprefix[|\textit{prefix}|]{|\textit{dest}|}|
is accomplished by:
%
\begin{center}
\begin{tabular}{l}
|{\edef\jobname{\scantokens\expandafter{\jobname\noexpand}}|\\
|\def\redirectjob |\textit{prefix}|#1~~~{\gdef\jobname{|\textit{dest}|#1}}|\\
|\expandafter\redirectjob\jobname~~~}\input{\jobname}|
\end{tabular}
\end{center}

In an alternative approach,
child documents can be compiled by a specific command line
without additional code or specific definitions:
%
\begin{center}
|... -jobname "|\textit{target}|" "|[\textit{flags}]%
|\includeonly{|\textit{dest}|}\input{|\textit{main}|}"|
\end{center}
%

%%%%%%%%%%%%%%%%%%%%%%%%%%%%%%%%%%%%%%%%%%%%%%%%%%%%%%%%%%%%%%%%%%%%%%%%%%%%%%%%
%%%%%%%%%%%%%%%%%%%%%%%%%%%%%%%%%%%%%%%%%%%%%%%%%%%%%%%%%%%%%%%%%%%%%%%%%%%%%%%%
\section{Information}

%%%%%%%%%%%%%%%%%%%%%%%%%%%%%%%%%%%%%%%%%%%%%%%%%%%%%%%%%%%%%%%%%%%%%%%%%%%%%%%%
\subsection{Copyright}

Copyright \copyright{} 2017--2018 Niklas Beisert

This work may be distributed and/or modified under the
conditions of the \LaTeX{} Project Public License, either version 1.3
of this license or (at your option) any later version.
The latest version of this license is in
  \url{http://www.latex-project.org/lppl.txt}
and version 1.3 or later is part of all distributions of \LaTeX{}
version 2005/12/01 or later.

This work has the LPPL maintenance status `maintained'.

The Current Maintainer of this work is Niklas Beisert.

This work consists of the files |README.txt|, |childdoc.ins| and |childdoc.dtx|
as well as the derived files |childdoc.def|, |cdocsamp.tex|
with |cdocsch1.tex|, |cdocsch2.tex|, |cdocspt3.tex|, |cdocspt4.tex|,
|cdocsdrf.tex|, |cdocsfn1.tex|, |cdocsfn2.tex|
as well as |childdoc.pdf|.

%%%%%%%%%%%%%%%%%%%%%%%%%%%%%%%%%%%%%%%%%%%%%%%%%%%%%%%%%%%%%%%%%%%%%%%%%%%%%%%%
\subsection{Files and Installation}

The package consists of the files:
%
\begin{center}
\begin{tabular}{ll}
    |README.txt|   & readme file \\
    |childdoc.ins| & installation file \\
    |childdoc.dtx| & source file \\
    |childdoc.def| & definition file \\
    |cdocsamp.tex| & sample main file \\
    |cdocsch1.tex| & sample include file \\
    |cdocsch2.tex| & sample include file \\
    |cdocspt3.tex| & sample part file \\
    |cdocspt4.tex| & sample part file \\
    |cdocsdrf.tex| & sample redirection file \\
    |cdocsfn1.tex| & sample redirection file \\
    |cdocsfn2.tex| & sample redirection file \\
    |childdoc.pdf| & manual
\end{tabular}
\end{center}
%
The distribution consists of the files
|README.txt|, |childdoc.ins| and |childdoc.dtx|.
%
\begin{itemize}
\item
Run (pdf)\LaTeX{} on |childdoc.dtx|
to compile the manual |childdoc.pdf| (this file).
\item
Run \LaTeX{} on |childdoc.ins| to create the definitions file |childdoc.def|
and the sample |cdocsamp.tex| with include files
|cdocsch1.tex|, |cdocsch2.tex|, |cdocspt3.tex|, |cdocspt4.tex|,
|cdocsdrf.tex|, |cdocsfn1.tex|, |cdocsfn2.tex|.
Then copy the file |childdoc.def| to an appropriate directory of your \LaTeX{}
distribution, e.g.\ \textit{texmf-root}|/tex/latex/childdoc|.
\end{itemize}

%%%%%%%%%%%%%%%%%%%%%%%%%%%%%%%%%%%%%%%%%%%%%%%%%%%%%%%%%%%%%%%%%%%%%%%%%%%%%%%%
\subsection{Related CTAN Packages}

There are several other packages which offer a similar functionality:
%
\begin{itemize}
\item
The packages
\href{http://ctan.org/pkg/docmute}{\textsf{docmute}},
\href{http://ctan.org/pkg/includex}{\textsf{includex}} and
\href{http://ctan.org/pkg/standalone}{\textsf{standalone}}
provide commands to include only the document body of
a child file thus allowing both files to be compiled individually.
\item
The packages \href{http://ctan.org/pkg/subdocs}{\textsf{subdocs}}
and \href{http://ctan.org/pkg/subfiles}{\textsf{subfiles}}
provide structures in which the main and child documents can be
encapsulated and allowing them to be compiled individually.
The inclusion mechanism is different from the conventional |\include|.
\item
The package \href{http://ctan.org/pkg/combine}{\textsf{combine}}
is an elaborate solution to combine several documents into one.
\end{itemize}
%
See also the CTAN topic \href{http://ctan.org/topic/subdocs}{\textsf{subdocs}}
for further related packages.
The present package differs from the above solutions in that
a document structure constructed with the conventional |\include| mechanism
just needs two extra commands at the top of every file
such that all constituent files can be compiled individually.

%%%%%%%%%%%%%%%%%%%%%%%%%%%%%%%%%%%%%%%%%%%%%%%%%%%%%%%%%%%%%%%%%%%%%%%%%%%%%%%%
%\subsection{Feature Suggestions}
%
%The following is a list of features which may be useful for future
%versions of this package:
%%
%\begin{itemize}
%\item
%\ldots
%\end{itemize}

%%%%%%%%%%%%%%%%%%%%%%%%%%%%%%%%%%%%%%%%%%%%%%%%%%%%%%%%%%%%%%%%%%%%%%%%%%%%%%%%
\subsection{Revision History}

%%%%%%%%%%%%%%%%%%%%%%%%%%%%%%%%%%%%%%%%
\paragraph{v2.0:} 2018/12/30

\begin{itemize}
\item
immediate forward processing
\item
added |\childdocby| mechanism
\item
manual restructured
\end{itemize}

%%%%%%%%%%%%%%%%%%%%%%%%%%%%%%%%%%%%%%%%
\paragraph{v1.6:} 2018/01/17

\begin{itemize}
\item
application for development of include files
\item
corrections to manual
\end{itemize}

%%%%%%%%%%%%%%%%%%%%%%%%%%%%%%%%%%%%%%%%
\paragraph{v1.5:} 2017/05/21

\begin{itemize}
\item
more complete structuring introduced
\item
|\childdocof| introduced
\item
|\childdoc| renamed to |\childdocmain|
\item
|\childredirect| renamed to |\childdocforward| and |\childdocforwardprefix|
and functionality expanded
\end{itemize}

%%%%%%%%%%%%%%%%%%%%%%%%%%%%%%%%%%%%%%%%
\paragraph{v1.0:} 2017/04/27

\begin{itemize}
\item
manual and install package
\item
first version published on CTAN
\end{itemize}

%%%%%%%%%%%%%%%%%%%%%%%%%%%%%%%%%%%%%%%%
\paragraph{v0.6:} 2017/04/26

\begin{itemize}
\item
redirection mechanism added
\end{itemize}

%%%%%%%%%%%%%%%%%%%%%%%%%%%%%%%%%%%%%%%%
\paragraph{v0.5:} 2017/04/26

\begin{itemize}
\item
functionality in definition file
\end{itemize}


%%%%%%%%%%%%%%%%%%%%%%%%%%%%%%%%%%%%%%%%%%%%%%%%%%%%%%%%%%%%%%%%%%%%%%%%%%%%%%%%
%%%%%%%%%%%%%%%%%%%%%%%%%%%%%%%%%%%%%%%%%%%%%%%%%%%%%%%%%%%%%%%%%%%%%%%%%%%%%%%%
%%%%%%%%%%%%%%%%%%%%%%%%%%%%%%%%%%%%%%%%%%%%%%%%%%%%%%%%%%%%%%%%%%%%%%%%%%%%%%%%
\appendix

\settowidth\MacroIndent{\rmfamily\scriptsize 000\ }

 \DocInput{childdoc.dtx}

\end{document}
%</driver>
% \fi
%
% %%%%%%%%%%%%%%%%%%%%%%%%%%%%%%%%%%%%%%%%%%%%%%%%%%%%%%%%%%%%%%%%%%%%%%%%%%%%%%
% %%%%%%%%%%%%%%%%%%%%%%%%%%%%%%%%%%%%%%%%%%%%%%%%%%%%%%%%%%%%%%%%%%%%%%%%%%%%%%
% \section{Sample}
%\iffalse
%<*samplemain>
%\fi
%
% The following presents a sample document
% with two chapters, two parts, a title page,
% a compile flag as well as three forwarding files to set the flag.
% It consists of eight |.tex| files:
% \begin{center}
% \begin{tabular}{ll}
% |cdocsamp.tex|&main file\\
% |cdocsch1.tex|&include file for chapter 1\\
% |cdocsch2.tex|&include file for chapter 2\\
% |cdocspt3.tex|&include file for part 3\\
% |cdocspt4.tex|&include file for part 4\\
% |cdocsdrf.tex|&forwarding file for main file in draft mode\\
% |cdocsfi1.tex|&forwarding file for final version of chapter 1\\
% |cdocsfi2.tex|&forwarding file for final version of chapter 2\\
% \end{tabular}
% \end{center}
% Each of the eight files can be compiled directly by the \LaTeX{} compiler.
%
% %%%%%%%%%%%%%%%%%%%%%%%%%%%%%%%%%%%%%%
% \paragraph{Main File.}
%
% The main file is called |cdocsamp.tex|.
%
% Load the \textsf{childdoc} definitions and
% declare the filename for the main document:
%    \begin{macrocode}
\input{childdoc.def}
\childdocmain{}
%    \end{macrocode}

% Optional override for |\version| flag:
%    \begin{macrocode}
%%\ifchilddoc\else\providecommand{\version}{draft}\fi
%    \end{macrocode}

% Define the default values for the |\version| flag
% (|final| for the main file and |draft| for childs):
%    \begin{macrocode}
\ifchilddoc
\providecommand{\version}{draft}
\else
\providecommand{\version}{final}
\fi
%    \end{macrocode}

% Load the standard document class:
%    \begin{macrocode}
\documentclass[12pt]{article}
%    \end{macrocode}

% Start the document body:
%    \begin{macrocode}
\begin{document}
%    \end{macrocode}

% Declare a title page.
% Print title, part of document being processed and version flag:
%    \begin{macrocode}
\addtocounter{page}{-1}
\begin{center}
{\LARGE\bfseries{}childdoc example\par}
\vspace{1cm}
\ifchilddoc
\ifchilddocmanual part\else chapter\fi:
`\childdocname' of `\childdocjob'\par
\else
main document: `\childdocjob'\par
\fi
version: \version\par
\end{center}
\newpage
%    \end{macrocode}

% Manually include selected file,
% otherwise process as usual:
%    \begin{macrocode}
\ifchilddocmanual
\section*{part `\childdocname'}
\input{\childdocname}
\else
%    \end{macrocode}

% Include the two chapters:
%    \begin{macrocode}
\include{cdocsch1}
\include{cdocsch2}
%    \end{macrocode}

% Include the two parts unless only chapters should be displayed:
%    \begin{macrocode}
\ifchilddoc\else
\section{part three}
\input{cdocspt3}
\section{part four}
\input{cdocspt4}
\fi
%    \end{macrocode}

% Process as usual until here:
%    \begin{macrocode}
\fi
%    \end{macrocode}

% End of document body:
%    \begin{macrocode}
\end{document}
%    \end{macrocode}
%\iffalse
%</samplemain>
%\fi
%
% %%%%%%%%%%%%%%%%%%%%%%%%%%%%%%%%%%%%%%
% \paragraph{Chapter Include Files.}
%
% The include files are called |cdocsch1.tex| and |cdocsch2.tex|.
%
%\iffalse
%<*samplechap1|samplechap2>
%\fi

% Optional override for |\version| flag:
%    \begin{macrocode}
%%\providecommand{\version}{final}
%    \end{macrocode}

% Include the main document:
%    \begin{macrocode}
\input{childdoc.def}
\childdocof{cdocsamp}
%    \end{macrocode}

%\iffalse
%</samplechap1|samplechap2>
%\fi
%
%\iffalse
%<*samplechap1>
%\fi
% Some text for chapter 1:
%    \begin{macrocode}
\section{one}
some text in chapter one
%    \end{macrocode}

%\iffalse
%</samplechap1>
%\fi
% Some text for chapter 2:
%\iffalse
%<*samplechap2>
%\fi
%    \begin{macrocode}
\section{two}
more text in chapter two
%    \end{macrocode}

%\iffalse
%</samplechap2>
%\fi
%
% %%%%%%%%%%%%%%%%%%%%%%%%%%%%%%%%%%%%%%
% \paragraph{Part Include Files.}
%
% The include files are called |cdocspt3.tex| and |cdocspt4.tex|.
%
%\iffalse
%<*samplepart3|samplepart4>
%\fi

% Optional override for |\version| flag:
%    \begin{macrocode}
%%\providecommand{\version}{final}
%    \end{macrocode}

% Include the main document:
%    \begin{macrocode}
\input{childdoc.def}
\childdocby{cdocsamp}
%    \end{macrocode}

%\iffalse
%</samplepart3|samplepart4>
%\fi
%
%\iffalse
%<*samplepart3>
%\fi
% Some text for part 3:
%    \begin{macrocode}
some text in part three
%    \end{macrocode}

%\iffalse
%</samplepart3>
%\fi
% Some text for part 4:
%\iffalse
%<*samplepart4>
%\fi
%    \begin{macrocode}
more text in part four
%    \end{macrocode}

%\iffalse
%</samplepart4>
%\fi
%
% %%%%%%%%%%%%%%%%%%%%%%%%%%%%%%%%%%%%%%
% \paragraph{Forwarding for a Complete Draft.}
%
% The following forwarding file |cdocsdrf.tex|
% compiles the main document in draft mode:
%\iffalse
%<*sampledraft>
%\fi
%    \begin{macrocode}
\def\version{draft}
\input{childdoc.def}
\childdocforward{cdocsamp}
%    \end{macrocode}

%\iffalse
%</sampledraft>
%\fi
%
% %%%%%%%%%%%%%%%%%%%%%%%%%%%%%%%%%%%%%%
% \paragraph{Forwarding for Final Version of the Chapters.}
%
% The following forwarding files |cdocsfn1.tex| and |cdocsfn2.tex|
% (with identical content)
% compile the final versions of the child documents
% |cdocsch1.tex| and |cdocsch2.tex|, respectively:
%\iffalse
%<*samplefinal>
%\fi
%    \begin{macrocode}
\def\version{final}
\input{childdoc.def}
\childdocforwardprefix[cdocsamp]{cdocsfn}{cdocsch}
%    \end{macrocode}

%\iffalse
%</samplefinal>
%\fi
%
% %%%%%%%%%%%%%%%%%%%%%%%%%%%%%%%%%%%%%%
% \paragraph{Command Line Processing.}
%
% The following three command lines generate the output files
% |cdocscld|, |cdocscl1| and |cdocscl2|
% which should be identical to
% |cdocsdrf|, |cdocsch1| and |cdocsfn2|, respectively:
% \begin{center}
% \begin{tabular}{l}
% |latex -jobname cdocscld \|\\
% |  "\def\version{draft}\input{childdoc.def}\childdocforward{cdocsamp}"|\\
% |latex -jobname cdocscl1 \|\\
% |  "\input{childdoc.def}\childdocforward[cdocsamp]{cdocsch1}"|\\
% |latex -jobname cdocscl2 \|\\
% |  "\def\version{final}\input{childdoc.def}\childdocforward{cdocsch2}"|
% \end{tabular}
% \end{center}
% Note that the trailing backslash on each first line
% merely continues the input to the second line
% (for convenient cut ant paste).
% Furthermore, the command |latex| can be replaced by any
% of its alternative versions such as |pdflatex|.
%
% %%%%%%%%%%%%%%%%%%%%%%%%%%%%%%%%%%%%%%%%%%%%%%%%%%%%%%%%%%%%%%%%%%%%%%%%%%%%%%
% %%%%%%%%%%%%%%%%%%%%%%%%%%%%%%%%%%%%%%%%%%%%%%%%%%%%%%%%%%%%%%%%%%%%%%%%%%%%%%
% \section{Implementation}
%\iffalse
%<*package>
%\fi
%
% This section describes the definitions file |childdoc.def|.

% The definitions cannot be loaded using |\usepackage| or |\RequirePackage|
% which has a mechanism to prevent loading a style file more than once.
% When loading the definitions by means of |\input|
% multiple instances have to be prevented manually:
%\iffalse
%This code needs to be before the `\ProvidesFile' directive
%which is defined at the beginning of this file.
%Therefore it is also placed there and commented out here.
%</package>
%<*discard>
%\fi
%    \begin{macrocode}
\ifdefined\childdocmain\endinput\fi
%    \end{macrocode}
%\iffalse
%</discard>
%<*package>
%\fi
%
% \macro{\ifchilddoc}
% \macro{\ifchilddocmanual}
% The conditional |\ifchilddoc| tells whether a
% child (true) or main (false) document is being compiled.
% The conditional |\ifchilddocmanual| tells whether
% the |\includeonly| mechanism is used (false) or
% the selection of child files must be performed manually (true).
% The definitions initialise to false:
%    \begin{macrocode}
\newif\ifchilddoc
\newif\ifchilddocmanual
%    \end{macrocode}

% \macro{\childdocname}
% \macro{\childdocjob}
% The macro |\childdocname| stores the name of the main document
% to be compiled. The macro |\childdocjob| stores the name of
% the document on which the \LaTeX{} compiler was originally invoked.
% The content of |\jobname| cannot be compared
% to filenames specified in the source due to different catcodes.
% The following code rescans |\jobname|, stores the result
% in |\childdocname| and saves a copy in |\childdocjob|:
%    \begin{macrocode}
\edef\childdocname{\scantokens\expandafter{\jobname\noexpand}}
\let\childdocjob\childdocname
%    \end{macrocode}

% \macro{\childdocdisable}
% The macro |\childdocdisable| prevents the main file
% from being processed more than once.
% At this stage, the main document command |\childdocmain|
% is assumed to be called once again where it should do nothing.
% Any subsequent call to it should prevent
% a secondary processing of the main document
% It overwrites the forwarding commands
% |\childdocof| and |\childdocforward|
% with empty macros to prevent further inclusions of the main document:
%    \begin{macrocode}
\newcommand{\childdocdisable}
{
  \renewcommand{\childdocmain}[1]{\renewcommand{\childdocmain}[1]{\endinput}}
  \renewcommand{\childdocof}[1]{}
  \renewcommand{\childdocby}[2][]{}
  \renewcommand{\childdocforward}[2][]{}
  \renewcommand{\childdocdisable}{}
}
%    \end{macrocode}

% \macro{\childdocmain}
% The macro |\childdocmain| is to be called at the top of the main file
% with nothing or the main filename (without extension) as argument.
% First, it breaks loops.
% If the argument is not empty and does not match |\childdocname|
% (which is set by the first inclusion of |childdoc.def|),
% |\ifchilddoc| is set to true, |\includeonly| is applied to the child file
% and |\jobname| is set to the main file
% (for proper handling of |.aux| files):
%    \begin{macrocode}
\newcommand{\childdocmain}[1]
{
  \childdocdisable\childdocmain{}
  \if?#1?\else
    \begingroup
      \def\childdoctmp{#1}
      \ifx\childdoctmp\childdocname
        \def\childdoctmp{}
      \else
        \def\childdoctmp
        {
          \childdoctrue
          \includeonly{\childdocname}
          \def\childdocjob{#1}
          \def\jobname{#1}
        }
      \fi
      \expandafter
    \endgroup
    \childdoctmp
  \fi
}
%    \end{macrocode}

% \macro{\childdocof}
% The command |\childdocof| redirects
% compilation to the main file |#1|.
%    \begin{macrocode}
\newcommand{\childdocof}[1]
{
  \childdocdisable
  \childdoctrue
  \includeonly{\childdocname}
  \def\jobname{#1}
  \def\childdocjob{#1}
  \input{#1}
}
%    \end{macrocode}

% \macro{\childdocby}
% The command |\childdocby| ....
%    \begin{macrocode}
\newcommand{\childdocby}[2][]
{
  \childdocdisable
  \childdoctrue
  \childdocmanualtrue
  \if?#1?\else
    \def\jobname{#2}
  \fi
  \def\childdocjob{#2}
  \input{#2}
  \endinput
}
%    \end{macrocode}

% \macro{\childdocforward}
% The command |\childdocforward| redirects
% compilation to the main file or
% (if the optional argument is given) a child file.
% Parameters are set as if the main file
% or a child file starting with |\childdocof| was compiled.
% Then compilation is handed over to the main file:
%    \begin{macrocode}
\newcommand{\childdocforward}[2][]
{
  \begingroup
    \if?#1?
      \def\childdoctmp
      {
        \def\childdocname{#2}
        \def\childdocjob{#2}
        \def\jobname{#2}
        \input{#2}
        \endinput
      }
    \else
      \def\childdoctmp
      {
        \childdocdisable
        \def\childdocname{#2}
        \childdoctrue
        \includeonly{#2}
        \def\childdocjob{#1}
        \def\jobname{#1}
        \input{#1}
        \endinput
      }
    \fi
    \expandafter
  \endgroup
  \childdoctmp
}
%    \end{macrocode}

% \macro{\childdocforwardprefix}
% The command |\childdocforwardprefix| redirects
% compilation to the main or a child file by means of a pattern.
% The prefix |#1| in the current filename is replaced by |#2|
% and the suffix of the current filename is kept
% (it is assumed that the filename does not contain the substring `|~~~|'
% which is used as a delimiter).
% Compilation is handed over to the new file by |\childdocforward|:
%    \begin{macrocode}
\newcommand{\childdocforwardprefix}[3][]
{
  \begingroup
    \def\childdocextract #2##1~~~{\def\childdoctmp{\childdocforward[#1]{#3##1}}}
    \expandafter\childdocextract\childdocname~~~
    \expandafter
  \endgroup
  \childdoctmp
}
%    \end{macrocode}

% \macro{\childdoc}
% The deprecated macro |\childdoc| is a legacy version of |\childdocmain|:
%    \begin{macrocode}
\newcommand{\childdoc}{\childdocmain}
%    \end{macrocode}

% \macro{\childdocredirect}
% The deprecated macro |\childdocredirect| is a legacy version
% of |\childdocforward| and |\childdocforwardprefix|:
%    \begin{macrocode}
\newcommand{\childdocredirect}[2][]
{
  \begingroup
    \if?#1?
      \def\childdoctmp{\childdocforward{#2}}
    \else
      \def\childdoctmp{\childdocforwardprefix{#1}{#2}}
    \fi
    \expandafter
  \endgroup
  \childdoctmp
}
%    \end{macrocode}

%\iffalse
%</package>
%\fi
%
\endinput
|\\
|\childdocmain{|\textit{main}|}|\\
\end{tabular}
\end{center}
%
If |\jobname| does not match the argument \textit{main} of |\childdocmain|,
it is assumed that |\jobname| points to the child file to be compiled.
When using |\childdocmain| with the main file specified as argument,
it suffices to start a child file
with just |\input{|\textit{main}|}|
without loading of the package and using |\childdocof|.
If instead all processing is done
with the appropriate \textsf{childdoc} directives,
the argument of \textit{main} of |\childdocmain| can be empty.

An alternative version of the command line processing described
in \secref{sec:commandline} using the detection mechanism reads:
%
\begin{center}
|... -jobname "|\textit{target}|" "|[\textit{flags}]%
[|\def\jobname{|\textit{dest}|}|]|\input{|\textit{main}|}"|
\end{center}

%%%%%%%%%%%%%%%%%%%%%%%%%%%%%%%%%%%%%%%%%%%%%%%%%%%%%%%%%%%%%%%%%%%%%%%%%%%%%%%%
\subsection{Manual Code}
\label{sec:manual}

In case one cannot be certain whether the definitions file |childdoc.def|
is installed on the target \TeX{} distribution
and one prefers not to ship it,
it is conceivable to paste a few relevant commands into the sources.

To that end, drop all statements |% \iffalse
%
% childdoc.dtx Copyright (C) 2017-2018 Niklas Beisert
%
% This work may be distributed and/or modified under the
% conditions of the LaTeX Project Public License, either version 1.3
% of this license or (at your option) any later version.
% The latest version of this license is in
%   http://www.latex-project.org/lppl.txt
% and version 1.3 or later is part of all distributions of LaTeX
% version 2005/12/01 or later.
%
% This work has the LPPL maintenance status `maintained'.
%
% The Current Maintainer of this work is Niklas Beisert.
%
% This work consists of the files childdoc.dtx and childdoc.ins
% and the derived files childdoc.def and cdocsamp.tex with
% cdocsch1.tex, cdocsch2.tex, cdocsdrf.tex, cdocsfn1.tex, cdocsfn2.tex.
%
%<package>\ifdefined\childdocmain\endinput\fi
%<package>\ProvidesFile{childdoc.def}[2018/12/30 v2.0 child document driver]
%<samplemain>\ProvidesFile{cdocsamp.tex}[2018/12/30 v2.0 sample for childdoc]
%<*driver>
%\ProvidesFile{childdoc.drv}[2018/12/30 v2.0 childdoc reference manual file]
\PassOptionsToClass{10pt,a4paper}{article}
\documentclass{ltxdoc}

\usepackage[margin=35mm]{geometry}
\usepackage{hyperref}
\usepackage{hyperxmp}
\usepackage[usenames]{color}

\hypersetup{colorlinks=true}
\hypersetup{pdfstartview=FitH}
\hypersetup{pdfpagemode=UseNone}
\hypersetup{pdfsource={}}
\hypersetup{pdflang={en-UK}}
\hypersetup{pdfcopyright={Copyright 2017-2018 Niklas Beisert.
  This work may be distributed and/or modified under the
  conditions of the LaTeX Project Public License, either version 1.3
  of this license or (at your option) any later version.}}
\hypersetup{pdflicenseurl={http://www.latex-project.org/lppl.txt}}
\hypersetup{pdfcontactaddress={ETH Zurich, ITP, HIT K,
  Wolfgang-Pauli-Strasse 27}}
\hypersetup{pdfcontactpostcode={8093}}
\hypersetup{pdfcontactcity={Zurich}}
\hypersetup{pdfcontactcountry={Switzerland}}
\hypersetup{pdfcontactemail={nbeisert@itp.phys.ethz.ch}}
\hypersetup{pdfcontacturl={http://people.phys.ethz.ch/\xmptilde nbeisert/}}

\newcommand{\secref}[1]{\hyperref[#1]{section \ref*{#1}}}

\parskip1ex
\parindent0pt
\let\olditemize\itemize
\def\itemize{\olditemize\parskip0pt}

\begin{document}

\title{The \textsf{childdoc} Package}
\hypersetup{pdftitle={The childdoc Package}}
\author{Niklas Beisert\\[2ex]
  Institut f\"ur Theoretische Physik\\
  Eidgen\"ossische Technische Hochschule Z\"urich\\
  Wolfgang-Pauli-Strasse 27, 8093 Z\"urich, Switzerland\\[1ex]
  \href{mailto:nbeisert@itp.phys.ethz.ch}
  {\texttt{nbeisert@itp.phys.ethz.ch}}}
\hypersetup{pdfauthor={Niklas Beisert}}
\hypersetup{pdfsubject={Manual for the LaTeX2e Package childdoc}}
\date{30 December 2018, \textsf{v2.0}}
\maketitle

\begin{abstract}\noindent
\textsf{childdoc} is a \LaTeXe{} package
that enables the direct compilation
of document sections included by |\include|
to individual files.
\end{abstract}

\begingroup
\parskip0ex
\tableofcontents
\endgroup

%%%%%%%%%%%%%%%%%%%%%%%%%%%%%%%%%%%%%%%%%%%%%%%%%%%%%%%%%%%%%%%%%%%%%%%%%%%%%%%%
%%%%%%%%%%%%%%%%%%%%%%%%%%%%%%%%%%%%%%%%%%%%%%%%%%%%%%%%%%%%%%%%%%%%%%%%%%%%%%%%
\section{Introduction}

\LaTeX{} provides a mechanism to structure a large document (such as a book)
into a main file and several child files (containing the chapters)
using the |\include| command.
This mechanism is beneficial for documents
which span hundreds of pages in order to
make the source file(s) more manageable.
Moreover, compilation can be restricted to
selected child files by means of the |\includeonly| command.
The latter feature can be used to reduce the compilation time while editing
(this was significantly more useful in the earlier days of \LaTeX{})
or to generate a smaller document which is easier to navigate.
Another application of |\includeonly| is to generate
documents consisting of selected parts of the complete document.

However, there are a few drawbacks of the plain |\include| mechanism:
\begin{itemize}
\item
The child files cannot be compiled on their own,
they can only be compiled via the main file.
A naive editing environment
(such as a text editor with an option
to have the current file processed by \LaTeX)
may require one to switch to the main file before compiling;
attempting to compile the child file produces errors.
\item
The main file must be modified (each time)
to adjust the |\includeonly| command
to the present needs. This easily leaves the main file in a messy state.
\item
The generated document will always carry the filename
of the main document. This is inconvenient if
several child files are to be compiled and
to be kept for distribution.
\end{itemize}

The present package provides a simple interface
to make child files individually compilable by \LaTeX{}.
Compiling a child file then has the same effect as compiling
the main file with an |\includeonly| command
to select the appropriate child.
Moreover the generated document will carry the name of the child
rather than the main file.
This resolves all three above issues.

This feature is meant to make the editing of books,
thesis documents and lecture notes somewhat more convenient.
However, the package can also be used efficiently for
composing a series of documents (such as exercise sheets)
which are typically distributed individually.
It then assists the author in generating the individual documents
(potentially in different versions)
as well as a document containing the collected series.
Another application is in developing style files
or other kinds of included material
where compilation of the style file could redirect
to a sample or test file.

%%%%%%%%%%%%%%%%%%%%%%%%%%%%%%%%%%%%%%%%%%%%%%%%%%%%%%%%%%%%%%%%%%%%%%%%%%%%%%%%
%%%%%%%%%%%%%%%%%%%%%%%%%%%%%%%%%%%%%%%%%%%%%%%%%%%%%%%%%%%%%%%%%%%%%%%%%%%%%%%%
\section{Usage}

First of all, the package \textsf{childdoc} is \emph{not} a standard
\LaTeXe{} |.sty| style file! Therefore it needs to be invoked in
a non-standard way.

%%%%%%%%%%%%%%%%%%%%%%%%%%%%%%%%%%%%%%%%%%%%%%%%%%%%%%%%%%%%%%%%%%%%%%%%%%%%%%%%
\subsection{Included Files}
\label{sec:include}

%%%%%%%%%%%%%%%%%%%%%%%%%%%%%%%%%%%%%%%%
\DescribeMacro{\childdocmain}
To use the package, add the commands
\begin{center}
\begin{tabular}{l}
|\input{childdoc.def}|\\
|\childdocmain{}|\\
\end{tabular}
\end{center}
at the very top of the main \LaTeX{} file,
in particular \emph{before} the |\documentclass| statement!
The argument of |\childdocmain| should be left empty
(but it must be present).

%%%%%%%%%%%%%%%%%%%%%%%%%%%%%%%%%%%%%%%%
\DescribeMacro{\childdocof}
Furthermore, add the commands
\begin{center}
\begin{tabular}{l}
|\input{childdoc.def}|\\
|\childdocof{|\textit{main}|}|\\
\end{tabular}
\end{center}
at the top of every child file \textit{child}
which is included by |\include{|\textit{child}|}|
from within the main file
(or at least for those files to be compiled individually).
The argument \textit{main} must be the filename of the main file.

There are a couple of
considerations in setting up the main and child documents:

%%%%%%%%%%%%%%%%%%%%%%%%%%%%%%%%%%%%%%%%
\paragraph{Restrictions.}

Please note the following restrictions:
\begin{itemize}
\item
|\childdocmain| must be called with one argument \textit{main}
to ensure compatibility with earlier version of the package.
It must either be empty (|\childdocmain{}|)
or precisely match the filename of the main file in which it is specified.
See \secref{sec:detection} for further information.
\item
The filename \textit{main} must be specified without the |.tex| extension.
\item
The filename \textit{main} is case sensitive
(even in case-insensitive file systems)
due to internal string comparison.
\item
The argument \textit{main} should be fully expanded, it cannot be a macro.
\item
Subdirectories and special characters should be avoided in filenames.
\item
The command |\childdocmain{|\textit{main}|}| must be followed by a whitespace.
It should not be followed immediately by another command
or by a comment mark `|%|'.
This is because the \TeX{} parser reads the token immediately following
the argument of |\childdocmain| and puts it
at the beginning of every child section;
however, a white\-space is ignored.
\end{itemize}

%%%%%%%%%%%%%%%%%%%%%%%%%%%%%%%%%%%%%%%%
\paragraph{Content of Main File.}

It is advisable to place all content in the child files included by |\include|.
Any output contained in the main file will appear in all child documents
unless suppressed manually;
it cannot be suppressed automatically by the |\includeonly| directive
and thus should normally be avoided.
A method to include some content in the main file
by means of conditional processing is described in \secref{sec:conditional}.

%%%%%%%%%%%%%%%%%%%%%%%%%%%%%%%%%%%%%%%%
\paragraph{Page Numbering.}

When only a part of the document is compiled,
the appropriate numbering of pages
(as well as other status parameters)
is determined from the |.aux| files.
The latter contain information from previous passes.
However this information needs to propagate through
all intermediate child documents.
Therefore the page numbering in child documents may well
be inconsistent until the complete document is compiled at least once.

A useful (if unconventional) way to always ensure a consistent
page numbering is to restart the numbering in each child document
and denote the pages by `\textit{child}|.|\textit{page}'
where \textit{child} represents the chapter/section number of the child file.
This can be achieved by the command
|\numberwithin{page}{|\textit{child}|}|
of the \textsf{amsmath} package
where \textit{child} can be |chapter| or |section|
depending on the chosen structuring.
Alternatively, one can modify the macro |\thepage| appropriately
and reset the counter |page| at the start of each child file.

%%%%%%%%%%%%%%%%%%%%%%%%%%%%%%%%%%%%%%%%%%%%%%%%%%%%%%%%%%%%%%%%%%%%%%%%%%%%%%%%
\subsection{Conditional Processing}
\label{sec:conditional}

The package provides a mechanism to compile different versions
of a document. To customise the versions further some conditional processing
can come in handy to distinguish which version is being compiled.
The package provides two macros to describe the compilation context:

%%%%%%%%%%%%%%%%%%%%%%%%%%%%%%%%%%%%%%%%
\DescribeMacro{\ifchilddoc}
The conditional |\ifchilddoc| distinguishes between the compilation of
child documents and the main document:
%
\begin{center}
|\ifchilddoc |\textit{child-code}| |[|\||else |\textit{main-code}]| \||fi|
\end{center}

%%%%%%%%%%%%%%%%%%%%%%%%%%%%%%%%%%%%%%%%
\DescribeMacro{\childdocname}
\DescribeMacro{\childdocjob}
The macro |\childdocname| contains the filename (without extension)
of the main or child file being processed.
Note that |\childdocjob| will always contain the name of the main file.

%%%%%%%%%%%%%%%%%%%%%%%%%%%%%%%%%%%%%%%%
\paragraph{Title Page.}

Conditional processing can be used to include a title or banner page
in the main document when proper precautions are taken.
Importantly, the code in the main file should ensure that the page counter
(as well as other status parameters which are stored in the |.aux| files)
takes the same value after the conditional processing.
Otherwise the page numbers may take divergent values
depending on which part is compiled.

For example, a title page could be declared by:
%
\begin{center}
\begin{tabular}{l}
|\ifchilddoc\||else|\\
|\addtocounter{page}{-1}|\\
\textit{code for title page}\\
|\newpage|\\
|\||fi|
\end{tabular}
\end{center}
%
A banner page for the child documents can be generated by:
%
\begin{center}
\begin{tabular}{l}
|\ifchilddoc|\\
|\addtocounter{page}{-1}|\\
\textit{code for banner page}\\
|\newpage|\\
|\||fi|
\end{tabular}
\end{center}
%
Here one could write a message such as:
\begin{center}
|This is the part \childdocname{} of \childdocjob{}.|
\end{center}

%%%%%%%%%%%%%%%%%%%%%%%%%%%%%%%%%%%%%%%%%%%%%%%%%%%%%%%%%%%%%%%%%%%%%%%%%%%%%%%%
\subsection{Flags}
\label{sec:flags}

The package makes it easy to generate different versions
of the main or child documents.
To this end compilation flags can be defined
and assigned different default values.
They will be particularly useful in conjunction
with the forwarding mechanism described in \secref{sec:forward}.

For example, it may be useful to have a flag |\version|
which can be set to |draft| or |final|.
The document source will contain some conditional code
depending on the value of |\version|.
Suppose further, the flag should default to |final| for the main file
and to |draft| for child files
which is a natural assignment for editing the document.
This is achieved by placing the following code
in the preamble of the main document
(below the |\childdocmain| directive):
%
\begin{center}
\begin{tabular}{l}
|\ifchilddoc|\\
|\providecommand{\version}{draft}|\\
|\||else|\\
|\providecommand{\version}{final}|\\
|\||fi|
\end{tabular}
\end{center}
%
The definition by |\providecommand| makes sure
that previous definitions are not overwritten.
Further statements |\providecommand{\version}{...}|
can thus be added before the above code to override it.

For the main file, one might add a line
(between |\childdocmain| and the above block)
%
\begin{center}
|%\ifchilddoc\||else\providecommand{\version}{draft}\||fi|
\end{center}
%
which can be uncommented to produce a draft version.
Likewise one can add a line to the very top of a child file
(above the |\childdocof{|\textit{main}|}| directive)
%
\begin{center}
|%\providecommand{\version}{final}|
\end{center}
%
which can be uncommented to produce the final version of this child document.

%%%%%%%%%%%%%%%%%%%%%%%%%%%%%%%%%%%%%%%%%%%%%%%%%%%%%%%%%%%%%%%%%%%%%%%%%%%%%%%%
\subsection{Forwarding}
\label{sec:forward}

Different versions of the main or child documents
using compilation flags as described in \secref{sec:flags}
can be (permanently) stored in different files
for convenient compilation, viewing and distribution.
To this end, the package defines a command
to pass on compilation to a different file:

%%%%%%%%%%%%%%%%%%%%%%%%%%%%%%%%%%%%%%%%
\DescribeMacro{\childdocforward}
The command |\childdocforward| redirects processing to
another source file:
%
\begin{center}
\begin{tabular}{l}
|\input{childdoc.def}|\\
|\childdocforward[|\textit{main}|]{|\textit{dest}|}|\\
\end{tabular}
\end{center}
%
The argument \textit{dest} is the destination file
(without extension).
It should be the main file or one of the child files.
Note that further \textsf{childdoc} directives
such as |\childdocof| and |\childdocforward|
in the indicated file will be processed in this form.
The optional argument \textit{main}
passes on directly to the main file \textit{main}
while pretending to compile the child \textit{dest}.
This form behaves as if \textit{dest}
issues |\childdocof{|\textit{main}|}| right away,
and no further \textsf{childdoc} directives will be processed.

%%%%%%%%%%%%%%%%%%%%%%%%%%%%%%%%%%%%%%%%
\DescribeMacro{\...prefix}
In the alternative form |\childdocforwardprefix|,
%
\begin{center}
\begin{tabular}{l}
|\input{childdoc.def}|\\
|\childdocforwardprefix[|\textit{main}|]{|\textit{prefix}|}{|\textit{dest}|}|
\end{tabular}
\end{center}
%
the destination file is determined by a pattern
depending on the current file:
To make this work, the current file must be called
`{\textit{prefix}\hspace{0.2em}\textit{suffix}}'
with \textit{prefix} matching precisely the argument.
Processing is then passed on to the file
`{\textit{dest}\hspace{0.2em}\textit{suffix}}'.
Surely, the same effect is achieved by
directly specifying the
argument `{\textit{dest}\hspace{0.2em}\textit{suffix}}'
in the first form.
However, that requires to set up a different file
for each child. With the alternative form of the command
all these files can have exactly the same content
which simplifies setting them up and maintaining them.

For example, the following file |draft.tex|
with a compilation flag |\version| as described in \secref{sec:flags}
compiles the main document as a draft:
%
\begin{center}
\begin{tabular}{l}
|\def\version{draft}|\\
|\input{childdoc.def}|\\
|\childdocforward{|\textit{main}|}|
\end{tabular}
\end{center}
%
Likewise, the following files |final|\textit{nn}|.tex|
compile the final version of the child document
|child|\textit{nn}|.tex|:
%
\begin{center}
\begin{tabular}{l}
|\def\version{final}|\\
|\input{childdoc.def}|\\
|\childdocforwardprefix{final}{child}|
\end{tabular}
\end{center}
%

Note that when several versions of a main file and/or of each child file
are to be generated, it may be convenient to set up a |Makefile| or
shell script to automatise the process.

%%%%%%%%%%%%%%%%%%%%%%%%%%%%%%%%%%%%%%%%%%%%%%%%%%%%%%%%%%%%%%%%%%%%%%%%%%%%%%%%
\subsection{Command Line Processing}
\label{sec:commandline}

The effect of redirection files can also be achieved by invoking
the \LaTeX{} compiler with a more elaborate command line.
Most conveniently this should be done as part
of a shell script or a |Makefile|.

When using \textsf{childdoc} in the main file, the following
command lines effectively perform a redirection
(note that depending on the shell being used,
backslashes may have to be doubled: `|\|' $\to$ `|\\|'):
%
\begin{center}
|... -jobname "|\textit{target}|" |\\|"|[\textit{flags}]%
|\input{childdoc.def}\childdocforward[|\textit{main}|]{|\textit{dest}|}"|
\end{center}
%
Here \textit{target} is the name of the output file,
\textit{main} is the name of the main file
and \textit{dest} is the name of the main or child file to be processed
(all filenames without extensions).
The optional argument \textit{main} can be omitted
if \textit{main} matches \textit{dest}.
Optionally, compilation \textit{flags} can be defined via |\def| commands.
This command line makes the \TeX{} engine believe
it is compiling the file \textit{target}
whose content is specified as the latter parameter.
The provided code then forwards the processing to
\textit{main} or \textit{dest} as described in \secref{sec:forward}.

%%%%%%%%%%%%%%%%%%%%%%%%%%%%%%%%%%%%%%%%%%%%%%%%%%%%%%%%%%%%%%%%%%%%%%%%%%%%%%%%
\subsection{Include by Input}
\label{sec:input}

Including child documents by |\include| has some restrictions by design.
Most notably, the content of a child document always occupies
its own set of pages; pages cannot be shared between child documents.
Usually, this behaviour makes perfect sense
because each child document contain an essential part of the document.
However, in some situations it may be desirable to compose
a document from a collection of parts
without having mandatory page breaks between then.
For this case, the package
provides a mechanism to include parts
by |\input| which can also be processed individually.
However, by construction this mechanism
requires manual handling of the content to be output.

%%%%%%%%%%%%%%%%%%%%%%%%%%%%%%%%%%%%%%%%
\DescribeMacro{\ifchilddocmanual}
The main file should be prepared as usual, see \secref{sec:include}.
However, the document body must make a distinction
between processing of an individual part and of the main document, e.g.:
%
\begin{center}
\begin{tabular}{l}
|\ifchilddocmanual|\\
|\input{\childdocname}|\\
|\||else|\\
\textit{document body with }|\input{|\textit{part}|}|\\
|\||fi|
\end{tabular}
\end{center}
%
The conditional |\ifchilddocmanual| is true whenever
a part to be included by |\input| is being compiled,
and the name of the part is stored in |\childdocname|.

%%%%%%%%%%%%%%%%%%%%%%%%%%%%%%%%%%%%%%%%
\DescribeMacro{\childdocby}
Each part to be included by |\input| should start with:
%
\begin{center}
\begin{tabular}{l}
|\input{childdoc.def}|\\
|\childdocby{|\textit{main}|}|\\
\end{tabular}
\end{center}
%
The directive |\childdocby| is similar to |\childdocof|
described in \secref{sec:include},
but the subsequent selection of content must be done manually.
To that end, both |\ifchilddoc| and |\ifchilddocmanual|
will be true upon processing of a part,
and the name of the part is stored in |\childdocname|.
Note that |\jobname| will be set to the filename of the current part
so that each part receives an individual |.aux| file
that does not interfere with the |.aux| file(s) of the main document.
This behaviour can be altered by the alternative form
|\childdocby[*]{|\textit{main}|}| (with a non-empty optional argument)
which uses the |.aux| file of the main document
by setting |\jobname| to \textit{main}.

%%%%%%%%%%%%%%%%%%%%%%%%%%%%%%%%%%%%%%%%%%%%%%%%%%%%%%%%%%%%%%%%%%%%%%%%%%%%%%%%
\subsection{Driver Development}
\label{sec:driver}

The \textsf{childdoc} mechanism can also be use for the development
of definition files such as \LaTeX{} styles or classes.
This case differs from the above setup with multiple parts
included by |\include| in that no |\includeonly| should be invoked.
This can be achieved by starting the include file
(before |\ProvidesPackage|) with:
%
\begin{center}
\begin{tabular}{l}
|\input{childdoc.def}|\\
|\childdocforward{|\textit{main}|}|\\
\end{tabular}
\end{center}
%
or alternatively with:
%
\begin{center}
\begin{tabular}{l}
|\input{childdoc.def}|\\
|\childdocby{|\textit{main}|}|\\
\end{tabular}
\end{center}
%
Both forms have slightly different effects as described above.
The main file is prepared as usual, see \secref{sec:include}.

%%%%%%%%%%%%%%%%%%%%%%%%%%%%%%%%%%%%%%%%%%%%%%%%%%%%%%%%%%%%%%%%%%%%%%%%%%%%%%%%
\subsection{Legacy Detection}
\label{sec:detection}

The directive |\childdocmain| in the main file can detect
whether the complete document or merely a child is to be compiled
even without using the directive |\childdocof|.
This method is deprecated because it is less robust
and there is no compelling reason to use it;
it is merely provided for backward compatibility
and it may be removed in future versions.

If the detection mechanism is to be used,
it is mandatory to correctly specify
the filename of the main file as the argument of |\childdocmain|:
%
\begin{center}
\begin{tabular}{l}
|\input{childdoc.def}|\\
|\childdocmain{|\textit{main}|}|\\
\end{tabular}
\end{center}
%
If |\jobname| does not match the argument \textit{main} of |\childdocmain|,
it is assumed that |\jobname| points to the child file to be compiled.
When using |\childdocmain| with the main file specified as argument,
it suffices to start a child file
with just |\input{|\textit{main}|}|
without loading of the package and using |\childdocof|.
If instead all processing is done
with the appropriate \textsf{childdoc} directives,
the argument of \textit{main} of |\childdocmain| can be empty.

An alternative version of the command line processing described
in \secref{sec:commandline} using the detection mechanism reads:
%
\begin{center}
|... -jobname "|\textit{target}|" "|[\textit{flags}]%
[|\def\jobname{|\textit{dest}|}|]|\input{|\textit{main}|}"|
\end{center}

%%%%%%%%%%%%%%%%%%%%%%%%%%%%%%%%%%%%%%%%%%%%%%%%%%%%%%%%%%%%%%%%%%%%%%%%%%%%%%%%
\subsection{Manual Code}
\label{sec:manual}

In case one cannot be certain whether the definitions file |childdoc.def|
is installed on the target \TeX{} distribution
and one prefers not to ship it,
it is conceivable to paste a few relevant commands into the sources.

To that end, drop all statements |\input{childdoc.def}|
and perform the replacements as outlined below.
Instead of |\childdocmain{|\textit{main}|}| add the following code
to the top of the main file:
%
\begin{center}
\begin{tabular}{l}
|\||ifdefined\childdocname\endinput\||fi\newif\ifchilddoc|\\
|\edef\childdocname{\scantokens\expandafter{\jobname\noexpand}}|\\
|\def\childdocmain{|\textit{main}|}\||ifx\childdocmain\childdocname\||else|\\
|\childdoctrue\includeonly{\childdocname}\let\jobname\childdocmain\||fi|\\
\end{tabular}
\end{center}
%
Instead of |\childdocof{|\textit{main}|}| just include the main file
at the top of each child file:
%
\begin{center}
|\input{|\textit{main}|}|
\end{center}
%
A simple redirection |\childdocforward{|\textit{dest}|}| is achieved by:
%
\begin{center}
|\def\jobname{|\textit{dest}|}\input{\jobname}|
\end{center}
%
The redirection with prefix
|\childdocforwardprefix[|\textit{prefix}|]{|\textit{dest}|}|
is accomplished by:
%
\begin{center}
\begin{tabular}{l}
|{\edef\jobname{\scantokens\expandafter{\jobname\noexpand}}|\\
|\def\redirectjob |\textit{prefix}|#1~~~{\gdef\jobname{|\textit{dest}|#1}}|\\
|\expandafter\redirectjob\jobname~~~}\input{\jobname}|
\end{tabular}
\end{center}

In an alternative approach,
child documents can be compiled by a specific command line
without additional code or specific definitions:
%
\begin{center}
|... -jobname "|\textit{target}|" "|[\textit{flags}]%
|\includeonly{|\textit{dest}|}\input{|\textit{main}|}"|
\end{center}
%

%%%%%%%%%%%%%%%%%%%%%%%%%%%%%%%%%%%%%%%%%%%%%%%%%%%%%%%%%%%%%%%%%%%%%%%%%%%%%%%%
%%%%%%%%%%%%%%%%%%%%%%%%%%%%%%%%%%%%%%%%%%%%%%%%%%%%%%%%%%%%%%%%%%%%%%%%%%%%%%%%
\section{Information}

%%%%%%%%%%%%%%%%%%%%%%%%%%%%%%%%%%%%%%%%%%%%%%%%%%%%%%%%%%%%%%%%%%%%%%%%%%%%%%%%
\subsection{Copyright}

Copyright \copyright{} 2017--2018 Niklas Beisert

This work may be distributed and/or modified under the
conditions of the \LaTeX{} Project Public License, either version 1.3
of this license or (at your option) any later version.
The latest version of this license is in
  \url{http://www.latex-project.org/lppl.txt}
and version 1.3 or later is part of all distributions of \LaTeX{}
version 2005/12/01 or later.

This work has the LPPL maintenance status `maintained'.

The Current Maintainer of this work is Niklas Beisert.

This work consists of the files |README.txt|, |childdoc.ins| and |childdoc.dtx|
as well as the derived files |childdoc.def|, |cdocsamp.tex|
with |cdocsch1.tex|, |cdocsch2.tex|, |cdocspt3.tex|, |cdocspt4.tex|,
|cdocsdrf.tex|, |cdocsfn1.tex|, |cdocsfn2.tex|
as well as |childdoc.pdf|.

%%%%%%%%%%%%%%%%%%%%%%%%%%%%%%%%%%%%%%%%%%%%%%%%%%%%%%%%%%%%%%%%%%%%%%%%%%%%%%%%
\subsection{Files and Installation}

The package consists of the files:
%
\begin{center}
\begin{tabular}{ll}
    |README.txt|   & readme file \\
    |childdoc.ins| & installation file \\
    |childdoc.dtx| & source file \\
    |childdoc.def| & definition file \\
    |cdocsamp.tex| & sample main file \\
    |cdocsch1.tex| & sample include file \\
    |cdocsch2.tex| & sample include file \\
    |cdocspt3.tex| & sample part file \\
    |cdocspt4.tex| & sample part file \\
    |cdocsdrf.tex| & sample redirection file \\
    |cdocsfn1.tex| & sample redirection file \\
    |cdocsfn2.tex| & sample redirection file \\
    |childdoc.pdf| & manual
\end{tabular}
\end{center}
%
The distribution consists of the files
|README.txt|, |childdoc.ins| and |childdoc.dtx|.
%
\begin{itemize}
\item
Run (pdf)\LaTeX{} on |childdoc.dtx|
to compile the manual |childdoc.pdf| (this file).
\item
Run \LaTeX{} on |childdoc.ins| to create the definitions file |childdoc.def|
and the sample |cdocsamp.tex| with include files
|cdocsch1.tex|, |cdocsch2.tex|, |cdocspt3.tex|, |cdocspt4.tex|,
|cdocsdrf.tex|, |cdocsfn1.tex|, |cdocsfn2.tex|.
Then copy the file |childdoc.def| to an appropriate directory of your \LaTeX{}
distribution, e.g.\ \textit{texmf-root}|/tex/latex/childdoc|.
\end{itemize}

%%%%%%%%%%%%%%%%%%%%%%%%%%%%%%%%%%%%%%%%%%%%%%%%%%%%%%%%%%%%%%%%%%%%%%%%%%%%%%%%
\subsection{Related CTAN Packages}

There are several other packages which offer a similar functionality:
%
\begin{itemize}
\item
The packages
\href{http://ctan.org/pkg/docmute}{\textsf{docmute}},
\href{http://ctan.org/pkg/includex}{\textsf{includex}} and
\href{http://ctan.org/pkg/standalone}{\textsf{standalone}}
provide commands to include only the document body of
a child file thus allowing both files to be compiled individually.
\item
The packages \href{http://ctan.org/pkg/subdocs}{\textsf{subdocs}}
and \href{http://ctan.org/pkg/subfiles}{\textsf{subfiles}}
provide structures in which the main and child documents can be
encapsulated and allowing them to be compiled individually.
The inclusion mechanism is different from the conventional |\include|.
\item
The package \href{http://ctan.org/pkg/combine}{\textsf{combine}}
is an elaborate solution to combine several documents into one.
\end{itemize}
%
See also the CTAN topic \href{http://ctan.org/topic/subdocs}{\textsf{subdocs}}
for further related packages.
The present package differs from the above solutions in that
a document structure constructed with the conventional |\include| mechanism
just needs two extra commands at the top of every file
such that all constituent files can be compiled individually.

%%%%%%%%%%%%%%%%%%%%%%%%%%%%%%%%%%%%%%%%%%%%%%%%%%%%%%%%%%%%%%%%%%%%%%%%%%%%%%%%
%\subsection{Feature Suggestions}
%
%The following is a list of features which may be useful for future
%versions of this package:
%%
%\begin{itemize}
%\item
%\ldots
%\end{itemize}

%%%%%%%%%%%%%%%%%%%%%%%%%%%%%%%%%%%%%%%%%%%%%%%%%%%%%%%%%%%%%%%%%%%%%%%%%%%%%%%%
\subsection{Revision History}

%%%%%%%%%%%%%%%%%%%%%%%%%%%%%%%%%%%%%%%%
\paragraph{v2.0:} 2018/12/30

\begin{itemize}
\item
immediate forward processing
\item
added |\childdocby| mechanism
\item
manual restructured
\end{itemize}

%%%%%%%%%%%%%%%%%%%%%%%%%%%%%%%%%%%%%%%%
\paragraph{v1.6:} 2018/01/17

\begin{itemize}
\item
application for development of include files
\item
corrections to manual
\end{itemize}

%%%%%%%%%%%%%%%%%%%%%%%%%%%%%%%%%%%%%%%%
\paragraph{v1.5:} 2017/05/21

\begin{itemize}
\item
more complete structuring introduced
\item
|\childdocof| introduced
\item
|\childdoc| renamed to |\childdocmain|
\item
|\childredirect| renamed to |\childdocforward| and |\childdocforwardprefix|
and functionality expanded
\end{itemize}

%%%%%%%%%%%%%%%%%%%%%%%%%%%%%%%%%%%%%%%%
\paragraph{v1.0:} 2017/04/27

\begin{itemize}
\item
manual and install package
\item
first version published on CTAN
\end{itemize}

%%%%%%%%%%%%%%%%%%%%%%%%%%%%%%%%%%%%%%%%
\paragraph{v0.6:} 2017/04/26

\begin{itemize}
\item
redirection mechanism added
\end{itemize}

%%%%%%%%%%%%%%%%%%%%%%%%%%%%%%%%%%%%%%%%
\paragraph{v0.5:} 2017/04/26

\begin{itemize}
\item
functionality in definition file
\end{itemize}


%%%%%%%%%%%%%%%%%%%%%%%%%%%%%%%%%%%%%%%%%%%%%%%%%%%%%%%%%%%%%%%%%%%%%%%%%%%%%%%%
%%%%%%%%%%%%%%%%%%%%%%%%%%%%%%%%%%%%%%%%%%%%%%%%%%%%%%%%%%%%%%%%%%%%%%%%%%%%%%%%
%%%%%%%%%%%%%%%%%%%%%%%%%%%%%%%%%%%%%%%%%%%%%%%%%%%%%%%%%%%%%%%%%%%%%%%%%%%%%%%%
\appendix

\settowidth\MacroIndent{\rmfamily\scriptsize 000\ }

 \DocInput{childdoc.dtx}

\end{document}
%</driver>
% \fi
%
% %%%%%%%%%%%%%%%%%%%%%%%%%%%%%%%%%%%%%%%%%%%%%%%%%%%%%%%%%%%%%%%%%%%%%%%%%%%%%%
% %%%%%%%%%%%%%%%%%%%%%%%%%%%%%%%%%%%%%%%%%%%%%%%%%%%%%%%%%%%%%%%%%%%%%%%%%%%%%%
% \section{Sample}
%\iffalse
%<*samplemain>
%\fi
%
% The following presents a sample document
% with two chapters, two parts, a title page,
% a compile flag as well as three forwarding files to set the flag.
% It consists of eight |.tex| files:
% \begin{center}
% \begin{tabular}{ll}
% |cdocsamp.tex|&main file\\
% |cdocsch1.tex|&include file for chapter 1\\
% |cdocsch2.tex|&include file for chapter 2\\
% |cdocspt3.tex|&include file for part 3\\
% |cdocspt4.tex|&include file for part 4\\
% |cdocsdrf.tex|&forwarding file for main file in draft mode\\
% |cdocsfi1.tex|&forwarding file for final version of chapter 1\\
% |cdocsfi2.tex|&forwarding file for final version of chapter 2\\
% \end{tabular}
% \end{center}
% Each of the eight files can be compiled directly by the \LaTeX{} compiler.
%
% %%%%%%%%%%%%%%%%%%%%%%%%%%%%%%%%%%%%%%
% \paragraph{Main File.}
%
% The main file is called |cdocsamp.tex|.
%
% Load the \textsf{childdoc} definitions and
% declare the filename for the main document:
%    \begin{macrocode}
\input{childdoc.def}
\childdocmain{}
%    \end{macrocode}

% Optional override for |\version| flag:
%    \begin{macrocode}
%%\ifchilddoc\else\providecommand{\version}{draft}\fi
%    \end{macrocode}

% Define the default values for the |\version| flag
% (|final| for the main file and |draft| for childs):
%    \begin{macrocode}
\ifchilddoc
\providecommand{\version}{draft}
\else
\providecommand{\version}{final}
\fi
%    \end{macrocode}

% Load the standard document class:
%    \begin{macrocode}
\documentclass[12pt]{article}
%    \end{macrocode}

% Start the document body:
%    \begin{macrocode}
\begin{document}
%    \end{macrocode}

% Declare a title page.
% Print title, part of document being processed and version flag:
%    \begin{macrocode}
\addtocounter{page}{-1}
\begin{center}
{\LARGE\bfseries{}childdoc example\par}
\vspace{1cm}
\ifchilddoc
\ifchilddocmanual part\else chapter\fi:
`\childdocname' of `\childdocjob'\par
\else
main document: `\childdocjob'\par
\fi
version: \version\par
\end{center}
\newpage
%    \end{macrocode}

% Manually include selected file,
% otherwise process as usual:
%    \begin{macrocode}
\ifchilddocmanual
\section*{part `\childdocname'}
\input{\childdocname}
\else
%    \end{macrocode}

% Include the two chapters:
%    \begin{macrocode}
\include{cdocsch1}
\include{cdocsch2}
%    \end{macrocode}

% Include the two parts unless only chapters should be displayed:
%    \begin{macrocode}
\ifchilddoc\else
\section{part three}
\input{cdocspt3}
\section{part four}
\input{cdocspt4}
\fi
%    \end{macrocode}

% Process as usual until here:
%    \begin{macrocode}
\fi
%    \end{macrocode}

% End of document body:
%    \begin{macrocode}
\end{document}
%    \end{macrocode}
%\iffalse
%</samplemain>
%\fi
%
% %%%%%%%%%%%%%%%%%%%%%%%%%%%%%%%%%%%%%%
% \paragraph{Chapter Include Files.}
%
% The include files are called |cdocsch1.tex| and |cdocsch2.tex|.
%
%\iffalse
%<*samplechap1|samplechap2>
%\fi

% Optional override for |\version| flag:
%    \begin{macrocode}
%%\providecommand{\version}{final}
%    \end{macrocode}

% Include the main document:
%    \begin{macrocode}
\input{childdoc.def}
\childdocof{cdocsamp}
%    \end{macrocode}

%\iffalse
%</samplechap1|samplechap2>
%\fi
%
%\iffalse
%<*samplechap1>
%\fi
% Some text for chapter 1:
%    \begin{macrocode}
\section{one}
some text in chapter one
%    \end{macrocode}

%\iffalse
%</samplechap1>
%\fi
% Some text for chapter 2:
%\iffalse
%<*samplechap2>
%\fi
%    \begin{macrocode}
\section{two}
more text in chapter two
%    \end{macrocode}

%\iffalse
%</samplechap2>
%\fi
%
% %%%%%%%%%%%%%%%%%%%%%%%%%%%%%%%%%%%%%%
% \paragraph{Part Include Files.}
%
% The include files are called |cdocspt3.tex| and |cdocspt4.tex|.
%
%\iffalse
%<*samplepart3|samplepart4>
%\fi

% Optional override for |\version| flag:
%    \begin{macrocode}
%%\providecommand{\version}{final}
%    \end{macrocode}

% Include the main document:
%    \begin{macrocode}
\input{childdoc.def}
\childdocby{cdocsamp}
%    \end{macrocode}

%\iffalse
%</samplepart3|samplepart4>
%\fi
%
%\iffalse
%<*samplepart3>
%\fi
% Some text for part 3:
%    \begin{macrocode}
some text in part three
%    \end{macrocode}

%\iffalse
%</samplepart3>
%\fi
% Some text for part 4:
%\iffalse
%<*samplepart4>
%\fi
%    \begin{macrocode}
more text in part four
%    \end{macrocode}

%\iffalse
%</samplepart4>
%\fi
%
% %%%%%%%%%%%%%%%%%%%%%%%%%%%%%%%%%%%%%%
% \paragraph{Forwarding for a Complete Draft.}
%
% The following forwarding file |cdocsdrf.tex|
% compiles the main document in draft mode:
%\iffalse
%<*sampledraft>
%\fi
%    \begin{macrocode}
\def\version{draft}
\input{childdoc.def}
\childdocforward{cdocsamp}
%    \end{macrocode}

%\iffalse
%</sampledraft>
%\fi
%
% %%%%%%%%%%%%%%%%%%%%%%%%%%%%%%%%%%%%%%
% \paragraph{Forwarding for Final Version of the Chapters.}
%
% The following forwarding files |cdocsfn1.tex| and |cdocsfn2.tex|
% (with identical content)
% compile the final versions of the child documents
% |cdocsch1.tex| and |cdocsch2.tex|, respectively:
%\iffalse
%<*samplefinal>
%\fi
%    \begin{macrocode}
\def\version{final}
\input{childdoc.def}
\childdocforwardprefix[cdocsamp]{cdocsfn}{cdocsch}
%    \end{macrocode}

%\iffalse
%</samplefinal>
%\fi
%
% %%%%%%%%%%%%%%%%%%%%%%%%%%%%%%%%%%%%%%
% \paragraph{Command Line Processing.}
%
% The following three command lines generate the output files
% |cdocscld|, |cdocscl1| and |cdocscl2|
% which should be identical to
% |cdocsdrf|, |cdocsch1| and |cdocsfn2|, respectively:
% \begin{center}
% \begin{tabular}{l}
% |latex -jobname cdocscld \|\\
% |  "\def\version{draft}\input{childdoc.def}\childdocforward{cdocsamp}"|\\
% |latex -jobname cdocscl1 \|\\
% |  "\input{childdoc.def}\childdocforward[cdocsamp]{cdocsch1}"|\\
% |latex -jobname cdocscl2 \|\\
% |  "\def\version{final}\input{childdoc.def}\childdocforward{cdocsch2}"|
% \end{tabular}
% \end{center}
% Note that the trailing backslash on each first line
% merely continues the input to the second line
% (for convenient cut ant paste).
% Furthermore, the command |latex| can be replaced by any
% of its alternative versions such as |pdflatex|.
%
% %%%%%%%%%%%%%%%%%%%%%%%%%%%%%%%%%%%%%%%%%%%%%%%%%%%%%%%%%%%%%%%%%%%%%%%%%%%%%%
% %%%%%%%%%%%%%%%%%%%%%%%%%%%%%%%%%%%%%%%%%%%%%%%%%%%%%%%%%%%%%%%%%%%%%%%%%%%%%%
% \section{Implementation}
%\iffalse
%<*package>
%\fi
%
% This section describes the definitions file |childdoc.def|.

% The definitions cannot be loaded using |\usepackage| or |\RequirePackage|
% which has a mechanism to prevent loading a style file more than once.
% When loading the definitions by means of |\input|
% multiple instances have to be prevented manually:
%\iffalse
%This code needs to be before the `\ProvidesFile' directive
%which is defined at the beginning of this file.
%Therefore it is also placed there and commented out here.
%</package>
%<*discard>
%\fi
%    \begin{macrocode}
\ifdefined\childdocmain\endinput\fi
%    \end{macrocode}
%\iffalse
%</discard>
%<*package>
%\fi
%
% \macro{\ifchilddoc}
% \macro{\ifchilddocmanual}
% The conditional |\ifchilddoc| tells whether a
% child (true) or main (false) document is being compiled.
% The conditional |\ifchilddocmanual| tells whether
% the |\includeonly| mechanism is used (false) or
% the selection of child files must be performed manually (true).
% The definitions initialise to false:
%    \begin{macrocode}
\newif\ifchilddoc
\newif\ifchilddocmanual
%    \end{macrocode}

% \macro{\childdocname}
% \macro{\childdocjob}
% The macro |\childdocname| stores the name of the main document
% to be compiled. The macro |\childdocjob| stores the name of
% the document on which the \LaTeX{} compiler was originally invoked.
% The content of |\jobname| cannot be compared
% to filenames specified in the source due to different catcodes.
% The following code rescans |\jobname|, stores the result
% in |\childdocname| and saves a copy in |\childdocjob|:
%    \begin{macrocode}
\edef\childdocname{\scantokens\expandafter{\jobname\noexpand}}
\let\childdocjob\childdocname
%    \end{macrocode}

% \macro{\childdocdisable}
% The macro |\childdocdisable| prevents the main file
% from being processed more than once.
% At this stage, the main document command |\childdocmain|
% is assumed to be called once again where it should do nothing.
% Any subsequent call to it should prevent
% a secondary processing of the main document
% It overwrites the forwarding commands
% |\childdocof| and |\childdocforward|
% with empty macros to prevent further inclusions of the main document:
%    \begin{macrocode}
\newcommand{\childdocdisable}
{
  \renewcommand{\childdocmain}[1]{\renewcommand{\childdocmain}[1]{\endinput}}
  \renewcommand{\childdocof}[1]{}
  \renewcommand{\childdocby}[2][]{}
  \renewcommand{\childdocforward}[2][]{}
  \renewcommand{\childdocdisable}{}
}
%    \end{macrocode}

% \macro{\childdocmain}
% The macro |\childdocmain| is to be called at the top of the main file
% with nothing or the main filename (without extension) as argument.
% First, it breaks loops.
% If the argument is not empty and does not match |\childdocname|
% (which is set by the first inclusion of |childdoc.def|),
% |\ifchilddoc| is set to true, |\includeonly| is applied to the child file
% and |\jobname| is set to the main file
% (for proper handling of |.aux| files):
%    \begin{macrocode}
\newcommand{\childdocmain}[1]
{
  \childdocdisable\childdocmain{}
  \if?#1?\else
    \begingroup
      \def\childdoctmp{#1}
      \ifx\childdoctmp\childdocname
        \def\childdoctmp{}
      \else
        \def\childdoctmp
        {
          \childdoctrue
          \includeonly{\childdocname}
          \def\childdocjob{#1}
          \def\jobname{#1}
        }
      \fi
      \expandafter
    \endgroup
    \childdoctmp
  \fi
}
%    \end{macrocode}

% \macro{\childdocof}
% The command |\childdocof| redirects
% compilation to the main file |#1|.
%    \begin{macrocode}
\newcommand{\childdocof}[1]
{
  \childdocdisable
  \childdoctrue
  \includeonly{\childdocname}
  \def\jobname{#1}
  \def\childdocjob{#1}
  \input{#1}
}
%    \end{macrocode}

% \macro{\childdocby}
% The command |\childdocby| ....
%    \begin{macrocode}
\newcommand{\childdocby}[2][]
{
  \childdocdisable
  \childdoctrue
  \childdocmanualtrue
  \if?#1?\else
    \def\jobname{#2}
  \fi
  \def\childdocjob{#2}
  \input{#2}
  \endinput
}
%    \end{macrocode}

% \macro{\childdocforward}
% The command |\childdocforward| redirects
% compilation to the main file or
% (if the optional argument is given) a child file.
% Parameters are set as if the main file
% or a child file starting with |\childdocof| was compiled.
% Then compilation is handed over to the main file:
%    \begin{macrocode}
\newcommand{\childdocforward}[2][]
{
  \begingroup
    \if?#1?
      \def\childdoctmp
      {
        \def\childdocname{#2}
        \def\childdocjob{#2}
        \def\jobname{#2}
        \input{#2}
        \endinput
      }
    \else
      \def\childdoctmp
      {
        \childdocdisable
        \def\childdocname{#2}
        \childdoctrue
        \includeonly{#2}
        \def\childdocjob{#1}
        \def\jobname{#1}
        \input{#1}
        \endinput
      }
    \fi
    \expandafter
  \endgroup
  \childdoctmp
}
%    \end{macrocode}

% \macro{\childdocforwardprefix}
% The command |\childdocforwardprefix| redirects
% compilation to the main or a child file by means of a pattern.
% The prefix |#1| in the current filename is replaced by |#2|
% and the suffix of the current filename is kept
% (it is assumed that the filename does not contain the substring `|~~~|'
% which is used as a delimiter).
% Compilation is handed over to the new file by |\childdocforward|:
%    \begin{macrocode}
\newcommand{\childdocforwardprefix}[3][]
{
  \begingroup
    \def\childdocextract #2##1~~~{\def\childdoctmp{\childdocforward[#1]{#3##1}}}
    \expandafter\childdocextract\childdocname~~~
    \expandafter
  \endgroup
  \childdoctmp
}
%    \end{macrocode}

% \macro{\childdoc}
% The deprecated macro |\childdoc| is a legacy version of |\childdocmain|:
%    \begin{macrocode}
\newcommand{\childdoc}{\childdocmain}
%    \end{macrocode}

% \macro{\childdocredirect}
% The deprecated macro |\childdocredirect| is a legacy version
% of |\childdocforward| and |\childdocforwardprefix|:
%    \begin{macrocode}
\newcommand{\childdocredirect}[2][]
{
  \begingroup
    \if?#1?
      \def\childdoctmp{\childdocforward{#2}}
    \else
      \def\childdoctmp{\childdocforwardprefix{#1}{#2}}
    \fi
    \expandafter
  \endgroup
  \childdoctmp
}
%    \end{macrocode}

%\iffalse
%</package>
%\fi
%
\endinput
|
and perform the replacements as outlined below.
Instead of |\childdocmain{|\textit{main}|}| add the following code
to the top of the main file:
%
\begin{center}
\begin{tabular}{l}
|\||ifdefined\childdocname\endinput\||fi\newif\ifchilddoc|\\
|\edef\childdocname{\scantokens\expandafter{\jobname\noexpand}}|\\
|\def\childdocmain{|\textit{main}|}\||ifx\childdocmain\childdocname\||else|\\
|\childdoctrue\includeonly{\childdocname}\let\jobname\childdocmain\||fi|\\
\end{tabular}
\end{center}
%
Instead of |\childdocof{|\textit{main}|}| just include the main file
at the top of each child file:
%
\begin{center}
|\input{|\textit{main}|}|
\end{center}
%
A simple redirection |\childdocforward{|\textit{dest}|}| is achieved by:
%
\begin{center}
|\def\jobname{|\textit{dest}|}\input{\jobname}|
\end{center}
%
The redirection with prefix
|\childdocforwardprefix[|\textit{prefix}|]{|\textit{dest}|}|
is accomplished by:
%
\begin{center}
\begin{tabular}{l}
|{\edef\jobname{\scantokens\expandafter{\jobname\noexpand}}|\\
|\def\redirectjob |\textit{prefix}|#1~~~{\gdef\jobname{|\textit{dest}|#1}}|\\
|\expandafter\redirectjob\jobname~~~}\input{\jobname}|
\end{tabular}
\end{center}

In an alternative approach,
child documents can be compiled by a specific command line
without additional code or specific definitions:
%
\begin{center}
|... -jobname "|\textit{target}|" "|[\textit{flags}]%
|\includeonly{|\textit{dest}|}\input{|\textit{main}|}"|
\end{center}
%

%%%%%%%%%%%%%%%%%%%%%%%%%%%%%%%%%%%%%%%%%%%%%%%%%%%%%%%%%%%%%%%%%%%%%%%%%%%%%%%%
%%%%%%%%%%%%%%%%%%%%%%%%%%%%%%%%%%%%%%%%%%%%%%%%%%%%%%%%%%%%%%%%%%%%%%%%%%%%%%%%
\section{Information}

%%%%%%%%%%%%%%%%%%%%%%%%%%%%%%%%%%%%%%%%%%%%%%%%%%%%%%%%%%%%%%%%%%%%%%%%%%%%%%%%
\subsection{Copyright}

Copyright \copyright{} 2017--2018 Niklas Beisert

This work may be distributed and/or modified under the
conditions of the \LaTeX{} Project Public License, either version 1.3
of this license or (at your option) any later version.
The latest version of this license is in
  \url{http://www.latex-project.org/lppl.txt}
and version 1.3 or later is part of all distributions of \LaTeX{}
version 2005/12/01 or later.

This work has the LPPL maintenance status `maintained'.

The Current Maintainer of this work is Niklas Beisert.

This work consists of the files |README.txt|, |childdoc.ins| and |childdoc.dtx|
as well as the derived files |childdoc.def|, |cdocsamp.tex|
with |cdocsch1.tex|, |cdocsch2.tex|, |cdocspt3.tex|, |cdocspt4.tex|,
|cdocsdrf.tex|, |cdocsfn1.tex|, |cdocsfn2.tex|
as well as |childdoc.pdf|.

%%%%%%%%%%%%%%%%%%%%%%%%%%%%%%%%%%%%%%%%%%%%%%%%%%%%%%%%%%%%%%%%%%%%%%%%%%%%%%%%
\subsection{Files and Installation}

The package consists of the files:
%
\begin{center}
\begin{tabular}{ll}
    |README.txt|   & readme file \\
    |childdoc.ins| & installation file \\
    |childdoc.dtx| & source file \\
    |childdoc.def| & definition file \\
    |cdocsamp.tex| & sample main file \\
    |cdocsch1.tex| & sample include file \\
    |cdocsch2.tex| & sample include file \\
    |cdocspt3.tex| & sample part file \\
    |cdocspt4.tex| & sample part file \\
    |cdocsdrf.tex| & sample redirection file \\
    |cdocsfn1.tex| & sample redirection file \\
    |cdocsfn2.tex| & sample redirection file \\
    |childdoc.pdf| & manual
\end{tabular}
\end{center}
%
The distribution consists of the files
|README.txt|, |childdoc.ins| and |childdoc.dtx|.
%
\begin{itemize}
\item
Run (pdf)\LaTeX{} on |childdoc.dtx|
to compile the manual |childdoc.pdf| (this file).
\item
Run \LaTeX{} on |childdoc.ins| to create the definitions file |childdoc.def|
and the sample |cdocsamp.tex| with include files
|cdocsch1.tex|, |cdocsch2.tex|, |cdocspt3.tex|, |cdocspt4.tex|,
|cdocsdrf.tex|, |cdocsfn1.tex|, |cdocsfn2.tex|.
Then copy the file |childdoc.def| to an appropriate directory of your \LaTeX{}
distribution, e.g.\ \textit{texmf-root}|/tex/latex/childdoc|.
\end{itemize}

%%%%%%%%%%%%%%%%%%%%%%%%%%%%%%%%%%%%%%%%%%%%%%%%%%%%%%%%%%%%%%%%%%%%%%%%%%%%%%%%
\subsection{Related CTAN Packages}

There are several other packages which offer a similar functionality:
%
\begin{itemize}
\item
The packages
\href{http://ctan.org/pkg/docmute}{\textsf{docmute}},
\href{http://ctan.org/pkg/includex}{\textsf{includex}} and
\href{http://ctan.org/pkg/standalone}{\textsf{standalone}}
provide commands to include only the document body of
a child file thus allowing both files to be compiled individually.
\item
The packages \href{http://ctan.org/pkg/subdocs}{\textsf{subdocs}}
and \href{http://ctan.org/pkg/subfiles}{\textsf{subfiles}}
provide structures in which the main and child documents can be
encapsulated and allowing them to be compiled individually.
The inclusion mechanism is different from the conventional |\include|.
\item
The package \href{http://ctan.org/pkg/combine}{\textsf{combine}}
is an elaborate solution to combine several documents into one.
\end{itemize}
%
See also the CTAN topic \href{http://ctan.org/topic/subdocs}{\textsf{subdocs}}
for further related packages.
The present package differs from the above solutions in that
a document structure constructed with the conventional |\include| mechanism
just needs two extra commands at the top of every file
such that all constituent files can be compiled individually.

%%%%%%%%%%%%%%%%%%%%%%%%%%%%%%%%%%%%%%%%%%%%%%%%%%%%%%%%%%%%%%%%%%%%%%%%%%%%%%%%
%\subsection{Feature Suggestions}
%
%The following is a list of features which may be useful for future
%versions of this package:
%%
%\begin{itemize}
%\item
%\ldots
%\end{itemize}

%%%%%%%%%%%%%%%%%%%%%%%%%%%%%%%%%%%%%%%%%%%%%%%%%%%%%%%%%%%%%%%%%%%%%%%%%%%%%%%%
\subsection{Revision History}

%%%%%%%%%%%%%%%%%%%%%%%%%%%%%%%%%%%%%%%%
\paragraph{v2.0:} 2018/12/30

\begin{itemize}
\item
immediate forward processing
\item
added |\childdocby| mechanism
\item
manual restructured
\end{itemize}

%%%%%%%%%%%%%%%%%%%%%%%%%%%%%%%%%%%%%%%%
\paragraph{v1.6:} 2018/01/17

\begin{itemize}
\item
application for development of include files
\item
corrections to manual
\end{itemize}

%%%%%%%%%%%%%%%%%%%%%%%%%%%%%%%%%%%%%%%%
\paragraph{v1.5:} 2017/05/21

\begin{itemize}
\item
more complete structuring introduced
\item
|\childdocof| introduced
\item
|\childdoc| renamed to |\childdocmain|
\item
|\childredirect| renamed to |\childdocforward| and |\childdocforwardprefix|
and functionality expanded
\end{itemize}

%%%%%%%%%%%%%%%%%%%%%%%%%%%%%%%%%%%%%%%%
\paragraph{v1.0:} 2017/04/27

\begin{itemize}
\item
manual and install package
\item
first version published on CTAN
\end{itemize}

%%%%%%%%%%%%%%%%%%%%%%%%%%%%%%%%%%%%%%%%
\paragraph{v0.6:} 2017/04/26

\begin{itemize}
\item
redirection mechanism added
\end{itemize}

%%%%%%%%%%%%%%%%%%%%%%%%%%%%%%%%%%%%%%%%
\paragraph{v0.5:} 2017/04/26

\begin{itemize}
\item
functionality in definition file
\end{itemize}


%%%%%%%%%%%%%%%%%%%%%%%%%%%%%%%%%%%%%%%%%%%%%%%%%%%%%%%%%%%%%%%%%%%%%%%%%%%%%%%%
%%%%%%%%%%%%%%%%%%%%%%%%%%%%%%%%%%%%%%%%%%%%%%%%%%%%%%%%%%%%%%%%%%%%%%%%%%%%%%%%
%%%%%%%%%%%%%%%%%%%%%%%%%%%%%%%%%%%%%%%%%%%%%%%%%%%%%%%%%%%%%%%%%%%%%%%%%%%%%%%%
\appendix

\settowidth\MacroIndent{\rmfamily\scriptsize 000\ }

 \DocInput{childdoc.dtx}

\end{document}
%</driver>
% \fi
%
% %%%%%%%%%%%%%%%%%%%%%%%%%%%%%%%%%%%%%%%%%%%%%%%%%%%%%%%%%%%%%%%%%%%%%%%%%%%%%%
% %%%%%%%%%%%%%%%%%%%%%%%%%%%%%%%%%%%%%%%%%%%%%%%%%%%%%%%%%%%%%%%%%%%%%%%%%%%%%%
% \section{Sample}
%\iffalse
%<*samplemain>
%\fi
%
% The following presents a sample document
% with two chapters, two parts, a title page,
% a compile flag as well as three forwarding files to set the flag.
% It consists of eight |.tex| files:
% \begin{center}
% \begin{tabular}{ll}
% |cdocsamp.tex|&main file\\
% |cdocsch1.tex|&include file for chapter 1\\
% |cdocsch2.tex|&include file for chapter 2\\
% |cdocspt3.tex|&include file for part 3\\
% |cdocspt4.tex|&include file for part 4\\
% |cdocsdrf.tex|&forwarding file for main file in draft mode\\
% |cdocsfi1.tex|&forwarding file for final version of chapter 1\\
% |cdocsfi2.tex|&forwarding file for final version of chapter 2\\
% \end{tabular}
% \end{center}
% Each of the eight files can be compiled directly by the \LaTeX{} compiler.
%
% %%%%%%%%%%%%%%%%%%%%%%%%%%%%%%%%%%%%%%
% \paragraph{Main File.}
%
% The main file is called |cdocsamp.tex|.
%
% Load the \textsf{childdoc} definitions and
% declare the filename for the main document:
%    \begin{macrocode}
% \iffalse
%
% childdoc.dtx Copyright (C) 2017-2018 Niklas Beisert
%
% This work may be distributed and/or modified under the
% conditions of the LaTeX Project Public License, either version 1.3
% of this license or (at your option) any later version.
% The latest version of this license is in
%   http://www.latex-project.org/lppl.txt
% and version 1.3 or later is part of all distributions of LaTeX
% version 2005/12/01 or later.
%
% This work has the LPPL maintenance status `maintained'.
%
% The Current Maintainer of this work is Niklas Beisert.
%
% This work consists of the files childdoc.dtx and childdoc.ins
% and the derived files childdoc.def and cdocsamp.tex with
% cdocsch1.tex, cdocsch2.tex, cdocsdrf.tex, cdocsfn1.tex, cdocsfn2.tex.
%
%<package>\ifdefined\childdocmain\endinput\fi
%<package>\ProvidesFile{childdoc.def}[2018/12/30 v2.0 child document driver]
%<samplemain>\ProvidesFile{cdocsamp.tex}[2018/12/30 v2.0 sample for childdoc]
%<*driver>
%\ProvidesFile{childdoc.drv}[2018/12/30 v2.0 childdoc reference manual file]
\PassOptionsToClass{10pt,a4paper}{article}
\documentclass{ltxdoc}

\usepackage[margin=35mm]{geometry}
\usepackage{hyperref}
\usepackage{hyperxmp}
\usepackage[usenames]{color}

\hypersetup{colorlinks=true}
\hypersetup{pdfstartview=FitH}
\hypersetup{pdfpagemode=UseNone}
\hypersetup{pdfsource={}}
\hypersetup{pdflang={en-UK}}
\hypersetup{pdfcopyright={Copyright 2017-2018 Niklas Beisert.
  This work may be distributed and/or modified under the
  conditions of the LaTeX Project Public License, either version 1.3
  of this license or (at your option) any later version.}}
\hypersetup{pdflicenseurl={http://www.latex-project.org/lppl.txt}}
\hypersetup{pdfcontactaddress={ETH Zurich, ITP, HIT K,
  Wolfgang-Pauli-Strasse 27}}
\hypersetup{pdfcontactpostcode={8093}}
\hypersetup{pdfcontactcity={Zurich}}
\hypersetup{pdfcontactcountry={Switzerland}}
\hypersetup{pdfcontactemail={nbeisert@itp.phys.ethz.ch}}
\hypersetup{pdfcontacturl={http://people.phys.ethz.ch/\xmptilde nbeisert/}}

\newcommand{\secref}[1]{\hyperref[#1]{section \ref*{#1}}}

\parskip1ex
\parindent0pt
\let\olditemize\itemize
\def\itemize{\olditemize\parskip0pt}

\begin{document}

\title{The \textsf{childdoc} Package}
\hypersetup{pdftitle={The childdoc Package}}
\author{Niklas Beisert\\[2ex]
  Institut f\"ur Theoretische Physik\\
  Eidgen\"ossische Technische Hochschule Z\"urich\\
  Wolfgang-Pauli-Strasse 27, 8093 Z\"urich, Switzerland\\[1ex]
  \href{mailto:nbeisert@itp.phys.ethz.ch}
  {\texttt{nbeisert@itp.phys.ethz.ch}}}
\hypersetup{pdfauthor={Niklas Beisert}}
\hypersetup{pdfsubject={Manual for the LaTeX2e Package childdoc}}
\date{30 December 2018, \textsf{v2.0}}
\maketitle

\begin{abstract}\noindent
\textsf{childdoc} is a \LaTeXe{} package
that enables the direct compilation
of document sections included by |\include|
to individual files.
\end{abstract}

\begingroup
\parskip0ex
\tableofcontents
\endgroup

%%%%%%%%%%%%%%%%%%%%%%%%%%%%%%%%%%%%%%%%%%%%%%%%%%%%%%%%%%%%%%%%%%%%%%%%%%%%%%%%
%%%%%%%%%%%%%%%%%%%%%%%%%%%%%%%%%%%%%%%%%%%%%%%%%%%%%%%%%%%%%%%%%%%%%%%%%%%%%%%%
\section{Introduction}

\LaTeX{} provides a mechanism to structure a large document (such as a book)
into a main file and several child files (containing the chapters)
using the |\include| command.
This mechanism is beneficial for documents
which span hundreds of pages in order to
make the source file(s) more manageable.
Moreover, compilation can be restricted to
selected child files by means of the |\includeonly| command.
The latter feature can be used to reduce the compilation time while editing
(this was significantly more useful in the earlier days of \LaTeX{})
or to generate a smaller document which is easier to navigate.
Another application of |\includeonly| is to generate
documents consisting of selected parts of the complete document.

However, there are a few drawbacks of the plain |\include| mechanism:
\begin{itemize}
\item
The child files cannot be compiled on their own,
they can only be compiled via the main file.
A naive editing environment
(such as a text editor with an option
to have the current file processed by \LaTeX)
may require one to switch to the main file before compiling;
attempting to compile the child file produces errors.
\item
The main file must be modified (each time)
to adjust the |\includeonly| command
to the present needs. This easily leaves the main file in a messy state.
\item
The generated document will always carry the filename
of the main document. This is inconvenient if
several child files are to be compiled and
to be kept for distribution.
\end{itemize}

The present package provides a simple interface
to make child files individually compilable by \LaTeX{}.
Compiling a child file then has the same effect as compiling
the main file with an |\includeonly| command
to select the appropriate child.
Moreover the generated document will carry the name of the child
rather than the main file.
This resolves all three above issues.

This feature is meant to make the editing of books,
thesis documents and lecture notes somewhat more convenient.
However, the package can also be used efficiently for
composing a series of documents (such as exercise sheets)
which are typically distributed individually.
It then assists the author in generating the individual documents
(potentially in different versions)
as well as a document containing the collected series.
Another application is in developing style files
or other kinds of included material
where compilation of the style file could redirect
to a sample or test file.

%%%%%%%%%%%%%%%%%%%%%%%%%%%%%%%%%%%%%%%%%%%%%%%%%%%%%%%%%%%%%%%%%%%%%%%%%%%%%%%%
%%%%%%%%%%%%%%%%%%%%%%%%%%%%%%%%%%%%%%%%%%%%%%%%%%%%%%%%%%%%%%%%%%%%%%%%%%%%%%%%
\section{Usage}

First of all, the package \textsf{childdoc} is \emph{not} a standard
\LaTeXe{} |.sty| style file! Therefore it needs to be invoked in
a non-standard way.

%%%%%%%%%%%%%%%%%%%%%%%%%%%%%%%%%%%%%%%%%%%%%%%%%%%%%%%%%%%%%%%%%%%%%%%%%%%%%%%%
\subsection{Included Files}
\label{sec:include}

%%%%%%%%%%%%%%%%%%%%%%%%%%%%%%%%%%%%%%%%
\DescribeMacro{\childdocmain}
To use the package, add the commands
\begin{center}
\begin{tabular}{l}
|\input{childdoc.def}|\\
|\childdocmain{}|\\
\end{tabular}
\end{center}
at the very top of the main \LaTeX{} file,
in particular \emph{before} the |\documentclass| statement!
The argument of |\childdocmain| should be left empty
(but it must be present).

%%%%%%%%%%%%%%%%%%%%%%%%%%%%%%%%%%%%%%%%
\DescribeMacro{\childdocof}
Furthermore, add the commands
\begin{center}
\begin{tabular}{l}
|\input{childdoc.def}|\\
|\childdocof{|\textit{main}|}|\\
\end{tabular}
\end{center}
at the top of every child file \textit{child}
which is included by |\include{|\textit{child}|}|
from within the main file
(or at least for those files to be compiled individually).
The argument \textit{main} must be the filename of the main file.

There are a couple of
considerations in setting up the main and child documents:

%%%%%%%%%%%%%%%%%%%%%%%%%%%%%%%%%%%%%%%%
\paragraph{Restrictions.}

Please note the following restrictions:
\begin{itemize}
\item
|\childdocmain| must be called with one argument \textit{main}
to ensure compatibility with earlier version of the package.
It must either be empty (|\childdocmain{}|)
or precisely match the filename of the main file in which it is specified.
See \secref{sec:detection} for further information.
\item
The filename \textit{main} must be specified without the |.tex| extension.
\item
The filename \textit{main} is case sensitive
(even in case-insensitive file systems)
due to internal string comparison.
\item
The argument \textit{main} should be fully expanded, it cannot be a macro.
\item
Subdirectories and special characters should be avoided in filenames.
\item
The command |\childdocmain{|\textit{main}|}| must be followed by a whitespace.
It should not be followed immediately by another command
or by a comment mark `|%|'.
This is because the \TeX{} parser reads the token immediately following
the argument of |\childdocmain| and puts it
at the beginning of every child section;
however, a white\-space is ignored.
\end{itemize}

%%%%%%%%%%%%%%%%%%%%%%%%%%%%%%%%%%%%%%%%
\paragraph{Content of Main File.}

It is advisable to place all content in the child files included by |\include|.
Any output contained in the main file will appear in all child documents
unless suppressed manually;
it cannot be suppressed automatically by the |\includeonly| directive
and thus should normally be avoided.
A method to include some content in the main file
by means of conditional processing is described in \secref{sec:conditional}.

%%%%%%%%%%%%%%%%%%%%%%%%%%%%%%%%%%%%%%%%
\paragraph{Page Numbering.}

When only a part of the document is compiled,
the appropriate numbering of pages
(as well as other status parameters)
is determined from the |.aux| files.
The latter contain information from previous passes.
However this information needs to propagate through
all intermediate child documents.
Therefore the page numbering in child documents may well
be inconsistent until the complete document is compiled at least once.

A useful (if unconventional) way to always ensure a consistent
page numbering is to restart the numbering in each child document
and denote the pages by `\textit{child}|.|\textit{page}'
where \textit{child} represents the chapter/section number of the child file.
This can be achieved by the command
|\numberwithin{page}{|\textit{child}|}|
of the \textsf{amsmath} package
where \textit{child} can be |chapter| or |section|
depending on the chosen structuring.
Alternatively, one can modify the macro |\thepage| appropriately
and reset the counter |page| at the start of each child file.

%%%%%%%%%%%%%%%%%%%%%%%%%%%%%%%%%%%%%%%%%%%%%%%%%%%%%%%%%%%%%%%%%%%%%%%%%%%%%%%%
\subsection{Conditional Processing}
\label{sec:conditional}

The package provides a mechanism to compile different versions
of a document. To customise the versions further some conditional processing
can come in handy to distinguish which version is being compiled.
The package provides two macros to describe the compilation context:

%%%%%%%%%%%%%%%%%%%%%%%%%%%%%%%%%%%%%%%%
\DescribeMacro{\ifchilddoc}
The conditional |\ifchilddoc| distinguishes between the compilation of
child documents and the main document:
%
\begin{center}
|\ifchilddoc |\textit{child-code}| |[|\||else |\textit{main-code}]| \||fi|
\end{center}

%%%%%%%%%%%%%%%%%%%%%%%%%%%%%%%%%%%%%%%%
\DescribeMacro{\childdocname}
\DescribeMacro{\childdocjob}
The macro |\childdocname| contains the filename (without extension)
of the main or child file being processed.
Note that |\childdocjob| will always contain the name of the main file.

%%%%%%%%%%%%%%%%%%%%%%%%%%%%%%%%%%%%%%%%
\paragraph{Title Page.}

Conditional processing can be used to include a title or banner page
in the main document when proper precautions are taken.
Importantly, the code in the main file should ensure that the page counter
(as well as other status parameters which are stored in the |.aux| files)
takes the same value after the conditional processing.
Otherwise the page numbers may take divergent values
depending on which part is compiled.

For example, a title page could be declared by:
%
\begin{center}
\begin{tabular}{l}
|\ifchilddoc\||else|\\
|\addtocounter{page}{-1}|\\
\textit{code for title page}\\
|\newpage|\\
|\||fi|
\end{tabular}
\end{center}
%
A banner page for the child documents can be generated by:
%
\begin{center}
\begin{tabular}{l}
|\ifchilddoc|\\
|\addtocounter{page}{-1}|\\
\textit{code for banner page}\\
|\newpage|\\
|\||fi|
\end{tabular}
\end{center}
%
Here one could write a message such as:
\begin{center}
|This is the part \childdocname{} of \childdocjob{}.|
\end{center}

%%%%%%%%%%%%%%%%%%%%%%%%%%%%%%%%%%%%%%%%%%%%%%%%%%%%%%%%%%%%%%%%%%%%%%%%%%%%%%%%
\subsection{Flags}
\label{sec:flags}

The package makes it easy to generate different versions
of the main or child documents.
To this end compilation flags can be defined
and assigned different default values.
They will be particularly useful in conjunction
with the forwarding mechanism described in \secref{sec:forward}.

For example, it may be useful to have a flag |\version|
which can be set to |draft| or |final|.
The document source will contain some conditional code
depending on the value of |\version|.
Suppose further, the flag should default to |final| for the main file
and to |draft| for child files
which is a natural assignment for editing the document.
This is achieved by placing the following code
in the preamble of the main document
(below the |\childdocmain| directive):
%
\begin{center}
\begin{tabular}{l}
|\ifchilddoc|\\
|\providecommand{\version}{draft}|\\
|\||else|\\
|\providecommand{\version}{final}|\\
|\||fi|
\end{tabular}
\end{center}
%
The definition by |\providecommand| makes sure
that previous definitions are not overwritten.
Further statements |\providecommand{\version}{...}|
can thus be added before the above code to override it.

For the main file, one might add a line
(between |\childdocmain| and the above block)
%
\begin{center}
|%\ifchilddoc\||else\providecommand{\version}{draft}\||fi|
\end{center}
%
which can be uncommented to produce a draft version.
Likewise one can add a line to the very top of a child file
(above the |\childdocof{|\textit{main}|}| directive)
%
\begin{center}
|%\providecommand{\version}{final}|
\end{center}
%
which can be uncommented to produce the final version of this child document.

%%%%%%%%%%%%%%%%%%%%%%%%%%%%%%%%%%%%%%%%%%%%%%%%%%%%%%%%%%%%%%%%%%%%%%%%%%%%%%%%
\subsection{Forwarding}
\label{sec:forward}

Different versions of the main or child documents
using compilation flags as described in \secref{sec:flags}
can be (permanently) stored in different files
for convenient compilation, viewing and distribution.
To this end, the package defines a command
to pass on compilation to a different file:

%%%%%%%%%%%%%%%%%%%%%%%%%%%%%%%%%%%%%%%%
\DescribeMacro{\childdocforward}
The command |\childdocforward| redirects processing to
another source file:
%
\begin{center}
\begin{tabular}{l}
|\input{childdoc.def}|\\
|\childdocforward[|\textit{main}|]{|\textit{dest}|}|\\
\end{tabular}
\end{center}
%
The argument \textit{dest} is the destination file
(without extension).
It should be the main file or one of the child files.
Note that further \textsf{childdoc} directives
such as |\childdocof| and |\childdocforward|
in the indicated file will be processed in this form.
The optional argument \textit{main}
passes on directly to the main file \textit{main}
while pretending to compile the child \textit{dest}.
This form behaves as if \textit{dest}
issues |\childdocof{|\textit{main}|}| right away,
and no further \textsf{childdoc} directives will be processed.

%%%%%%%%%%%%%%%%%%%%%%%%%%%%%%%%%%%%%%%%
\DescribeMacro{\...prefix}
In the alternative form |\childdocforwardprefix|,
%
\begin{center}
\begin{tabular}{l}
|\input{childdoc.def}|\\
|\childdocforwardprefix[|\textit{main}|]{|\textit{prefix}|}{|\textit{dest}|}|
\end{tabular}
\end{center}
%
the destination file is determined by a pattern
depending on the current file:
To make this work, the current file must be called
`{\textit{prefix}\hspace{0.2em}\textit{suffix}}'
with \textit{prefix} matching precisely the argument.
Processing is then passed on to the file
`{\textit{dest}\hspace{0.2em}\textit{suffix}}'.
Surely, the same effect is achieved by
directly specifying the
argument `{\textit{dest}\hspace{0.2em}\textit{suffix}}'
in the first form.
However, that requires to set up a different file
for each child. With the alternative form of the command
all these files can have exactly the same content
which simplifies setting them up and maintaining them.

For example, the following file |draft.tex|
with a compilation flag |\version| as described in \secref{sec:flags}
compiles the main document as a draft:
%
\begin{center}
\begin{tabular}{l}
|\def\version{draft}|\\
|\input{childdoc.def}|\\
|\childdocforward{|\textit{main}|}|
\end{tabular}
\end{center}
%
Likewise, the following files |final|\textit{nn}|.tex|
compile the final version of the child document
|child|\textit{nn}|.tex|:
%
\begin{center}
\begin{tabular}{l}
|\def\version{final}|\\
|\input{childdoc.def}|\\
|\childdocforwardprefix{final}{child}|
\end{tabular}
\end{center}
%

Note that when several versions of a main file and/or of each child file
are to be generated, it may be convenient to set up a |Makefile| or
shell script to automatise the process.

%%%%%%%%%%%%%%%%%%%%%%%%%%%%%%%%%%%%%%%%%%%%%%%%%%%%%%%%%%%%%%%%%%%%%%%%%%%%%%%%
\subsection{Command Line Processing}
\label{sec:commandline}

The effect of redirection files can also be achieved by invoking
the \LaTeX{} compiler with a more elaborate command line.
Most conveniently this should be done as part
of a shell script or a |Makefile|.

When using \textsf{childdoc} in the main file, the following
command lines effectively perform a redirection
(note that depending on the shell being used,
backslashes may have to be doubled: `|\|' $\to$ `|\\|'):
%
\begin{center}
|... -jobname "|\textit{target}|" |\\|"|[\textit{flags}]%
|\input{childdoc.def}\childdocforward[|\textit{main}|]{|\textit{dest}|}"|
\end{center}
%
Here \textit{target} is the name of the output file,
\textit{main} is the name of the main file
and \textit{dest} is the name of the main or child file to be processed
(all filenames without extensions).
The optional argument \textit{main} can be omitted
if \textit{main} matches \textit{dest}.
Optionally, compilation \textit{flags} can be defined via |\def| commands.
This command line makes the \TeX{} engine believe
it is compiling the file \textit{target}
whose content is specified as the latter parameter.
The provided code then forwards the processing to
\textit{main} or \textit{dest} as described in \secref{sec:forward}.

%%%%%%%%%%%%%%%%%%%%%%%%%%%%%%%%%%%%%%%%%%%%%%%%%%%%%%%%%%%%%%%%%%%%%%%%%%%%%%%%
\subsection{Include by Input}
\label{sec:input}

Including child documents by |\include| has some restrictions by design.
Most notably, the content of a child document always occupies
its own set of pages; pages cannot be shared between child documents.
Usually, this behaviour makes perfect sense
because each child document contain an essential part of the document.
However, in some situations it may be desirable to compose
a document from a collection of parts
without having mandatory page breaks between then.
For this case, the package
provides a mechanism to include parts
by |\input| which can also be processed individually.
However, by construction this mechanism
requires manual handling of the content to be output.

%%%%%%%%%%%%%%%%%%%%%%%%%%%%%%%%%%%%%%%%
\DescribeMacro{\ifchilddocmanual}
The main file should be prepared as usual, see \secref{sec:include}.
However, the document body must make a distinction
between processing of an individual part and of the main document, e.g.:
%
\begin{center}
\begin{tabular}{l}
|\ifchilddocmanual|\\
|\input{\childdocname}|\\
|\||else|\\
\textit{document body with }|\input{|\textit{part}|}|\\
|\||fi|
\end{tabular}
\end{center}
%
The conditional |\ifchilddocmanual| is true whenever
a part to be included by |\input| is being compiled,
and the name of the part is stored in |\childdocname|.

%%%%%%%%%%%%%%%%%%%%%%%%%%%%%%%%%%%%%%%%
\DescribeMacro{\childdocby}
Each part to be included by |\input| should start with:
%
\begin{center}
\begin{tabular}{l}
|\input{childdoc.def}|\\
|\childdocby{|\textit{main}|}|\\
\end{tabular}
\end{center}
%
The directive |\childdocby| is similar to |\childdocof|
described in \secref{sec:include},
but the subsequent selection of content must be done manually.
To that end, both |\ifchilddoc| and |\ifchilddocmanual|
will be true upon processing of a part,
and the name of the part is stored in |\childdocname|.
Note that |\jobname| will be set to the filename of the current part
so that each part receives an individual |.aux| file
that does not interfere with the |.aux| file(s) of the main document.
This behaviour can be altered by the alternative form
|\childdocby[*]{|\textit{main}|}| (with a non-empty optional argument)
which uses the |.aux| file of the main document
by setting |\jobname| to \textit{main}.

%%%%%%%%%%%%%%%%%%%%%%%%%%%%%%%%%%%%%%%%%%%%%%%%%%%%%%%%%%%%%%%%%%%%%%%%%%%%%%%%
\subsection{Driver Development}
\label{sec:driver}

The \textsf{childdoc} mechanism can also be use for the development
of definition files such as \LaTeX{} styles or classes.
This case differs from the above setup with multiple parts
included by |\include| in that no |\includeonly| should be invoked.
This can be achieved by starting the include file
(before |\ProvidesPackage|) with:
%
\begin{center}
\begin{tabular}{l}
|\input{childdoc.def}|\\
|\childdocforward{|\textit{main}|}|\\
\end{tabular}
\end{center}
%
or alternatively with:
%
\begin{center}
\begin{tabular}{l}
|\input{childdoc.def}|\\
|\childdocby{|\textit{main}|}|\\
\end{tabular}
\end{center}
%
Both forms have slightly different effects as described above.
The main file is prepared as usual, see \secref{sec:include}.

%%%%%%%%%%%%%%%%%%%%%%%%%%%%%%%%%%%%%%%%%%%%%%%%%%%%%%%%%%%%%%%%%%%%%%%%%%%%%%%%
\subsection{Legacy Detection}
\label{sec:detection}

The directive |\childdocmain| in the main file can detect
whether the complete document or merely a child is to be compiled
even without using the directive |\childdocof|.
This method is deprecated because it is less robust
and there is no compelling reason to use it;
it is merely provided for backward compatibility
and it may be removed in future versions.

If the detection mechanism is to be used,
it is mandatory to correctly specify
the filename of the main file as the argument of |\childdocmain|:
%
\begin{center}
\begin{tabular}{l}
|\input{childdoc.def}|\\
|\childdocmain{|\textit{main}|}|\\
\end{tabular}
\end{center}
%
If |\jobname| does not match the argument \textit{main} of |\childdocmain|,
it is assumed that |\jobname| points to the child file to be compiled.
When using |\childdocmain| with the main file specified as argument,
it suffices to start a child file
with just |\input{|\textit{main}|}|
without loading of the package and using |\childdocof|.
If instead all processing is done
with the appropriate \textsf{childdoc} directives,
the argument of \textit{main} of |\childdocmain| can be empty.

An alternative version of the command line processing described
in \secref{sec:commandline} using the detection mechanism reads:
%
\begin{center}
|... -jobname "|\textit{target}|" "|[\textit{flags}]%
[|\def\jobname{|\textit{dest}|}|]|\input{|\textit{main}|}"|
\end{center}

%%%%%%%%%%%%%%%%%%%%%%%%%%%%%%%%%%%%%%%%%%%%%%%%%%%%%%%%%%%%%%%%%%%%%%%%%%%%%%%%
\subsection{Manual Code}
\label{sec:manual}

In case one cannot be certain whether the definitions file |childdoc.def|
is installed on the target \TeX{} distribution
and one prefers not to ship it,
it is conceivable to paste a few relevant commands into the sources.

To that end, drop all statements |\input{childdoc.def}|
and perform the replacements as outlined below.
Instead of |\childdocmain{|\textit{main}|}| add the following code
to the top of the main file:
%
\begin{center}
\begin{tabular}{l}
|\||ifdefined\childdocname\endinput\||fi\newif\ifchilddoc|\\
|\edef\childdocname{\scantokens\expandafter{\jobname\noexpand}}|\\
|\def\childdocmain{|\textit{main}|}\||ifx\childdocmain\childdocname\||else|\\
|\childdoctrue\includeonly{\childdocname}\let\jobname\childdocmain\||fi|\\
\end{tabular}
\end{center}
%
Instead of |\childdocof{|\textit{main}|}| just include the main file
at the top of each child file:
%
\begin{center}
|\input{|\textit{main}|}|
\end{center}
%
A simple redirection |\childdocforward{|\textit{dest}|}| is achieved by:
%
\begin{center}
|\def\jobname{|\textit{dest}|}\input{\jobname}|
\end{center}
%
The redirection with prefix
|\childdocforwardprefix[|\textit{prefix}|]{|\textit{dest}|}|
is accomplished by:
%
\begin{center}
\begin{tabular}{l}
|{\edef\jobname{\scantokens\expandafter{\jobname\noexpand}}|\\
|\def\redirectjob |\textit{prefix}|#1~~~{\gdef\jobname{|\textit{dest}|#1}}|\\
|\expandafter\redirectjob\jobname~~~}\input{\jobname}|
\end{tabular}
\end{center}

In an alternative approach,
child documents can be compiled by a specific command line
without additional code or specific definitions:
%
\begin{center}
|... -jobname "|\textit{target}|" "|[\textit{flags}]%
|\includeonly{|\textit{dest}|}\input{|\textit{main}|}"|
\end{center}
%

%%%%%%%%%%%%%%%%%%%%%%%%%%%%%%%%%%%%%%%%%%%%%%%%%%%%%%%%%%%%%%%%%%%%%%%%%%%%%%%%
%%%%%%%%%%%%%%%%%%%%%%%%%%%%%%%%%%%%%%%%%%%%%%%%%%%%%%%%%%%%%%%%%%%%%%%%%%%%%%%%
\section{Information}

%%%%%%%%%%%%%%%%%%%%%%%%%%%%%%%%%%%%%%%%%%%%%%%%%%%%%%%%%%%%%%%%%%%%%%%%%%%%%%%%
\subsection{Copyright}

Copyright \copyright{} 2017--2018 Niklas Beisert

This work may be distributed and/or modified under the
conditions of the \LaTeX{} Project Public License, either version 1.3
of this license or (at your option) any later version.
The latest version of this license is in
  \url{http://www.latex-project.org/lppl.txt}
and version 1.3 or later is part of all distributions of \LaTeX{}
version 2005/12/01 or later.

This work has the LPPL maintenance status `maintained'.

The Current Maintainer of this work is Niklas Beisert.

This work consists of the files |README.txt|, |childdoc.ins| and |childdoc.dtx|
as well as the derived files |childdoc.def|, |cdocsamp.tex|
with |cdocsch1.tex|, |cdocsch2.tex|, |cdocspt3.tex|, |cdocspt4.tex|,
|cdocsdrf.tex|, |cdocsfn1.tex|, |cdocsfn2.tex|
as well as |childdoc.pdf|.

%%%%%%%%%%%%%%%%%%%%%%%%%%%%%%%%%%%%%%%%%%%%%%%%%%%%%%%%%%%%%%%%%%%%%%%%%%%%%%%%
\subsection{Files and Installation}

The package consists of the files:
%
\begin{center}
\begin{tabular}{ll}
    |README.txt|   & readme file \\
    |childdoc.ins| & installation file \\
    |childdoc.dtx| & source file \\
    |childdoc.def| & definition file \\
    |cdocsamp.tex| & sample main file \\
    |cdocsch1.tex| & sample include file \\
    |cdocsch2.tex| & sample include file \\
    |cdocspt3.tex| & sample part file \\
    |cdocspt4.tex| & sample part file \\
    |cdocsdrf.tex| & sample redirection file \\
    |cdocsfn1.tex| & sample redirection file \\
    |cdocsfn2.tex| & sample redirection file \\
    |childdoc.pdf| & manual
\end{tabular}
\end{center}
%
The distribution consists of the files
|README.txt|, |childdoc.ins| and |childdoc.dtx|.
%
\begin{itemize}
\item
Run (pdf)\LaTeX{} on |childdoc.dtx|
to compile the manual |childdoc.pdf| (this file).
\item
Run \LaTeX{} on |childdoc.ins| to create the definitions file |childdoc.def|
and the sample |cdocsamp.tex| with include files
|cdocsch1.tex|, |cdocsch2.tex|, |cdocspt3.tex|, |cdocspt4.tex|,
|cdocsdrf.tex|, |cdocsfn1.tex|, |cdocsfn2.tex|.
Then copy the file |childdoc.def| to an appropriate directory of your \LaTeX{}
distribution, e.g.\ \textit{texmf-root}|/tex/latex/childdoc|.
\end{itemize}

%%%%%%%%%%%%%%%%%%%%%%%%%%%%%%%%%%%%%%%%%%%%%%%%%%%%%%%%%%%%%%%%%%%%%%%%%%%%%%%%
\subsection{Related CTAN Packages}

There are several other packages which offer a similar functionality:
%
\begin{itemize}
\item
The packages
\href{http://ctan.org/pkg/docmute}{\textsf{docmute}},
\href{http://ctan.org/pkg/includex}{\textsf{includex}} and
\href{http://ctan.org/pkg/standalone}{\textsf{standalone}}
provide commands to include only the document body of
a child file thus allowing both files to be compiled individually.
\item
The packages \href{http://ctan.org/pkg/subdocs}{\textsf{subdocs}}
and \href{http://ctan.org/pkg/subfiles}{\textsf{subfiles}}
provide structures in which the main and child documents can be
encapsulated and allowing them to be compiled individually.
The inclusion mechanism is different from the conventional |\include|.
\item
The package \href{http://ctan.org/pkg/combine}{\textsf{combine}}
is an elaborate solution to combine several documents into one.
\end{itemize}
%
See also the CTAN topic \href{http://ctan.org/topic/subdocs}{\textsf{subdocs}}
for further related packages.
The present package differs from the above solutions in that
a document structure constructed with the conventional |\include| mechanism
just needs two extra commands at the top of every file
such that all constituent files can be compiled individually.

%%%%%%%%%%%%%%%%%%%%%%%%%%%%%%%%%%%%%%%%%%%%%%%%%%%%%%%%%%%%%%%%%%%%%%%%%%%%%%%%
%\subsection{Feature Suggestions}
%
%The following is a list of features which may be useful for future
%versions of this package:
%%
%\begin{itemize}
%\item
%\ldots
%\end{itemize}

%%%%%%%%%%%%%%%%%%%%%%%%%%%%%%%%%%%%%%%%%%%%%%%%%%%%%%%%%%%%%%%%%%%%%%%%%%%%%%%%
\subsection{Revision History}

%%%%%%%%%%%%%%%%%%%%%%%%%%%%%%%%%%%%%%%%
\paragraph{v2.0:} 2018/12/30

\begin{itemize}
\item
immediate forward processing
\item
added |\childdocby| mechanism
\item
manual restructured
\end{itemize}

%%%%%%%%%%%%%%%%%%%%%%%%%%%%%%%%%%%%%%%%
\paragraph{v1.6:} 2018/01/17

\begin{itemize}
\item
application for development of include files
\item
corrections to manual
\end{itemize}

%%%%%%%%%%%%%%%%%%%%%%%%%%%%%%%%%%%%%%%%
\paragraph{v1.5:} 2017/05/21

\begin{itemize}
\item
more complete structuring introduced
\item
|\childdocof| introduced
\item
|\childdoc| renamed to |\childdocmain|
\item
|\childredirect| renamed to |\childdocforward| and |\childdocforwardprefix|
and functionality expanded
\end{itemize}

%%%%%%%%%%%%%%%%%%%%%%%%%%%%%%%%%%%%%%%%
\paragraph{v1.0:} 2017/04/27

\begin{itemize}
\item
manual and install package
\item
first version published on CTAN
\end{itemize}

%%%%%%%%%%%%%%%%%%%%%%%%%%%%%%%%%%%%%%%%
\paragraph{v0.6:} 2017/04/26

\begin{itemize}
\item
redirection mechanism added
\end{itemize}

%%%%%%%%%%%%%%%%%%%%%%%%%%%%%%%%%%%%%%%%
\paragraph{v0.5:} 2017/04/26

\begin{itemize}
\item
functionality in definition file
\end{itemize}


%%%%%%%%%%%%%%%%%%%%%%%%%%%%%%%%%%%%%%%%%%%%%%%%%%%%%%%%%%%%%%%%%%%%%%%%%%%%%%%%
%%%%%%%%%%%%%%%%%%%%%%%%%%%%%%%%%%%%%%%%%%%%%%%%%%%%%%%%%%%%%%%%%%%%%%%%%%%%%%%%
%%%%%%%%%%%%%%%%%%%%%%%%%%%%%%%%%%%%%%%%%%%%%%%%%%%%%%%%%%%%%%%%%%%%%%%%%%%%%%%%
\appendix

\settowidth\MacroIndent{\rmfamily\scriptsize 000\ }

 \DocInput{childdoc.dtx}

\end{document}
%</driver>
% \fi
%
% %%%%%%%%%%%%%%%%%%%%%%%%%%%%%%%%%%%%%%%%%%%%%%%%%%%%%%%%%%%%%%%%%%%%%%%%%%%%%%
% %%%%%%%%%%%%%%%%%%%%%%%%%%%%%%%%%%%%%%%%%%%%%%%%%%%%%%%%%%%%%%%%%%%%%%%%%%%%%%
% \section{Sample}
%\iffalse
%<*samplemain>
%\fi
%
% The following presents a sample document
% with two chapters, two parts, a title page,
% a compile flag as well as three forwarding files to set the flag.
% It consists of eight |.tex| files:
% \begin{center}
% \begin{tabular}{ll}
% |cdocsamp.tex|&main file\\
% |cdocsch1.tex|&include file for chapter 1\\
% |cdocsch2.tex|&include file for chapter 2\\
% |cdocspt3.tex|&include file for part 3\\
% |cdocspt4.tex|&include file for part 4\\
% |cdocsdrf.tex|&forwarding file for main file in draft mode\\
% |cdocsfi1.tex|&forwarding file for final version of chapter 1\\
% |cdocsfi2.tex|&forwarding file for final version of chapter 2\\
% \end{tabular}
% \end{center}
% Each of the eight files can be compiled directly by the \LaTeX{} compiler.
%
% %%%%%%%%%%%%%%%%%%%%%%%%%%%%%%%%%%%%%%
% \paragraph{Main File.}
%
% The main file is called |cdocsamp.tex|.
%
% Load the \textsf{childdoc} definitions and
% declare the filename for the main document:
%    \begin{macrocode}
\input{childdoc.def}
\childdocmain{}
%    \end{macrocode}

% Optional override for |\version| flag:
%    \begin{macrocode}
%%\ifchilddoc\else\providecommand{\version}{draft}\fi
%    \end{macrocode}

% Define the default values for the |\version| flag
% (|final| for the main file and |draft| for childs):
%    \begin{macrocode}
\ifchilddoc
\providecommand{\version}{draft}
\else
\providecommand{\version}{final}
\fi
%    \end{macrocode}

% Load the standard document class:
%    \begin{macrocode}
\documentclass[12pt]{article}
%    \end{macrocode}

% Start the document body:
%    \begin{macrocode}
\begin{document}
%    \end{macrocode}

% Declare a title page.
% Print title, part of document being processed and version flag:
%    \begin{macrocode}
\addtocounter{page}{-1}
\begin{center}
{\LARGE\bfseries{}childdoc example\par}
\vspace{1cm}
\ifchilddoc
\ifchilddocmanual part\else chapter\fi:
`\childdocname' of `\childdocjob'\par
\else
main document: `\childdocjob'\par
\fi
version: \version\par
\end{center}
\newpage
%    \end{macrocode}

% Manually include selected file,
% otherwise process as usual:
%    \begin{macrocode}
\ifchilddocmanual
\section*{part `\childdocname'}
\input{\childdocname}
\else
%    \end{macrocode}

% Include the two chapters:
%    \begin{macrocode}
\include{cdocsch1}
\include{cdocsch2}
%    \end{macrocode}

% Include the two parts unless only chapters should be displayed:
%    \begin{macrocode}
\ifchilddoc\else
\section{part three}
\input{cdocspt3}
\section{part four}
\input{cdocspt4}
\fi
%    \end{macrocode}

% Process as usual until here:
%    \begin{macrocode}
\fi
%    \end{macrocode}

% End of document body:
%    \begin{macrocode}
\end{document}
%    \end{macrocode}
%\iffalse
%</samplemain>
%\fi
%
% %%%%%%%%%%%%%%%%%%%%%%%%%%%%%%%%%%%%%%
% \paragraph{Chapter Include Files.}
%
% The include files are called |cdocsch1.tex| and |cdocsch2.tex|.
%
%\iffalse
%<*samplechap1|samplechap2>
%\fi

% Optional override for |\version| flag:
%    \begin{macrocode}
%%\providecommand{\version}{final}
%    \end{macrocode}

% Include the main document:
%    \begin{macrocode}
\input{childdoc.def}
\childdocof{cdocsamp}
%    \end{macrocode}

%\iffalse
%</samplechap1|samplechap2>
%\fi
%
%\iffalse
%<*samplechap1>
%\fi
% Some text for chapter 1:
%    \begin{macrocode}
\section{one}
some text in chapter one
%    \end{macrocode}

%\iffalse
%</samplechap1>
%\fi
% Some text for chapter 2:
%\iffalse
%<*samplechap2>
%\fi
%    \begin{macrocode}
\section{two}
more text in chapter two
%    \end{macrocode}

%\iffalse
%</samplechap2>
%\fi
%
% %%%%%%%%%%%%%%%%%%%%%%%%%%%%%%%%%%%%%%
% \paragraph{Part Include Files.}
%
% The include files are called |cdocspt3.tex| and |cdocspt4.tex|.
%
%\iffalse
%<*samplepart3|samplepart4>
%\fi

% Optional override for |\version| flag:
%    \begin{macrocode}
%%\providecommand{\version}{final}
%    \end{macrocode}

% Include the main document:
%    \begin{macrocode}
\input{childdoc.def}
\childdocby{cdocsamp}
%    \end{macrocode}

%\iffalse
%</samplepart3|samplepart4>
%\fi
%
%\iffalse
%<*samplepart3>
%\fi
% Some text for part 3:
%    \begin{macrocode}
some text in part three
%    \end{macrocode}

%\iffalse
%</samplepart3>
%\fi
% Some text for part 4:
%\iffalse
%<*samplepart4>
%\fi
%    \begin{macrocode}
more text in part four
%    \end{macrocode}

%\iffalse
%</samplepart4>
%\fi
%
% %%%%%%%%%%%%%%%%%%%%%%%%%%%%%%%%%%%%%%
% \paragraph{Forwarding for a Complete Draft.}
%
% The following forwarding file |cdocsdrf.tex|
% compiles the main document in draft mode:
%\iffalse
%<*sampledraft>
%\fi
%    \begin{macrocode}
\def\version{draft}
\input{childdoc.def}
\childdocforward{cdocsamp}
%    \end{macrocode}

%\iffalse
%</sampledraft>
%\fi
%
% %%%%%%%%%%%%%%%%%%%%%%%%%%%%%%%%%%%%%%
% \paragraph{Forwarding for Final Version of the Chapters.}
%
% The following forwarding files |cdocsfn1.tex| and |cdocsfn2.tex|
% (with identical content)
% compile the final versions of the child documents
% |cdocsch1.tex| and |cdocsch2.tex|, respectively:
%\iffalse
%<*samplefinal>
%\fi
%    \begin{macrocode}
\def\version{final}
\input{childdoc.def}
\childdocforwardprefix[cdocsamp]{cdocsfn}{cdocsch}
%    \end{macrocode}

%\iffalse
%</samplefinal>
%\fi
%
% %%%%%%%%%%%%%%%%%%%%%%%%%%%%%%%%%%%%%%
% \paragraph{Command Line Processing.}
%
% The following three command lines generate the output files
% |cdocscld|, |cdocscl1| and |cdocscl2|
% which should be identical to
% |cdocsdrf|, |cdocsch1| and |cdocsfn2|, respectively:
% \begin{center}
% \begin{tabular}{l}
% |latex -jobname cdocscld \|\\
% |  "\def\version{draft}\input{childdoc.def}\childdocforward{cdocsamp}"|\\
% |latex -jobname cdocscl1 \|\\
% |  "\input{childdoc.def}\childdocforward[cdocsamp]{cdocsch1}"|\\
% |latex -jobname cdocscl2 \|\\
% |  "\def\version{final}\input{childdoc.def}\childdocforward{cdocsch2}"|
% \end{tabular}
% \end{center}
% Note that the trailing backslash on each first line
% merely continues the input to the second line
% (for convenient cut ant paste).
% Furthermore, the command |latex| can be replaced by any
% of its alternative versions such as |pdflatex|.
%
% %%%%%%%%%%%%%%%%%%%%%%%%%%%%%%%%%%%%%%%%%%%%%%%%%%%%%%%%%%%%%%%%%%%%%%%%%%%%%%
% %%%%%%%%%%%%%%%%%%%%%%%%%%%%%%%%%%%%%%%%%%%%%%%%%%%%%%%%%%%%%%%%%%%%%%%%%%%%%%
% \section{Implementation}
%\iffalse
%<*package>
%\fi
%
% This section describes the definitions file |childdoc.def|.

% The definitions cannot be loaded using |\usepackage| or |\RequirePackage|
% which has a mechanism to prevent loading a style file more than once.
% When loading the definitions by means of |\input|
% multiple instances have to be prevented manually:
%\iffalse
%This code needs to be before the `\ProvidesFile' directive
%which is defined at the beginning of this file.
%Therefore it is also placed there and commented out here.
%</package>
%<*discard>
%\fi
%    \begin{macrocode}
\ifdefined\childdocmain\endinput\fi
%    \end{macrocode}
%\iffalse
%</discard>
%<*package>
%\fi
%
% \macro{\ifchilddoc}
% \macro{\ifchilddocmanual}
% The conditional |\ifchilddoc| tells whether a
% child (true) or main (false) document is being compiled.
% The conditional |\ifchilddocmanual| tells whether
% the |\includeonly| mechanism is used (false) or
% the selection of child files must be performed manually (true).
% The definitions initialise to false:
%    \begin{macrocode}
\newif\ifchilddoc
\newif\ifchilddocmanual
%    \end{macrocode}

% \macro{\childdocname}
% \macro{\childdocjob}
% The macro |\childdocname| stores the name of the main document
% to be compiled. The macro |\childdocjob| stores the name of
% the document on which the \LaTeX{} compiler was originally invoked.
% The content of |\jobname| cannot be compared
% to filenames specified in the source due to different catcodes.
% The following code rescans |\jobname|, stores the result
% in |\childdocname| and saves a copy in |\childdocjob|:
%    \begin{macrocode}
\edef\childdocname{\scantokens\expandafter{\jobname\noexpand}}
\let\childdocjob\childdocname
%    \end{macrocode}

% \macro{\childdocdisable}
% The macro |\childdocdisable| prevents the main file
% from being processed more than once.
% At this stage, the main document command |\childdocmain|
% is assumed to be called once again where it should do nothing.
% Any subsequent call to it should prevent
% a secondary processing of the main document
% It overwrites the forwarding commands
% |\childdocof| and |\childdocforward|
% with empty macros to prevent further inclusions of the main document:
%    \begin{macrocode}
\newcommand{\childdocdisable}
{
  \renewcommand{\childdocmain}[1]{\renewcommand{\childdocmain}[1]{\endinput}}
  \renewcommand{\childdocof}[1]{}
  \renewcommand{\childdocby}[2][]{}
  \renewcommand{\childdocforward}[2][]{}
  \renewcommand{\childdocdisable}{}
}
%    \end{macrocode}

% \macro{\childdocmain}
% The macro |\childdocmain| is to be called at the top of the main file
% with nothing or the main filename (without extension) as argument.
% First, it breaks loops.
% If the argument is not empty and does not match |\childdocname|
% (which is set by the first inclusion of |childdoc.def|),
% |\ifchilddoc| is set to true, |\includeonly| is applied to the child file
% and |\jobname| is set to the main file
% (for proper handling of |.aux| files):
%    \begin{macrocode}
\newcommand{\childdocmain}[1]
{
  \childdocdisable\childdocmain{}
  \if?#1?\else
    \begingroup
      \def\childdoctmp{#1}
      \ifx\childdoctmp\childdocname
        \def\childdoctmp{}
      \else
        \def\childdoctmp
        {
          \childdoctrue
          \includeonly{\childdocname}
          \def\childdocjob{#1}
          \def\jobname{#1}
        }
      \fi
      \expandafter
    \endgroup
    \childdoctmp
  \fi
}
%    \end{macrocode}

% \macro{\childdocof}
% The command |\childdocof| redirects
% compilation to the main file |#1|.
%    \begin{macrocode}
\newcommand{\childdocof}[1]
{
  \childdocdisable
  \childdoctrue
  \includeonly{\childdocname}
  \def\jobname{#1}
  \def\childdocjob{#1}
  \input{#1}
}
%    \end{macrocode}

% \macro{\childdocby}
% The command |\childdocby| ....
%    \begin{macrocode}
\newcommand{\childdocby}[2][]
{
  \childdocdisable
  \childdoctrue
  \childdocmanualtrue
  \if?#1?\else
    \def\jobname{#2}
  \fi
  \def\childdocjob{#2}
  \input{#2}
  \endinput
}
%    \end{macrocode}

% \macro{\childdocforward}
% The command |\childdocforward| redirects
% compilation to the main file or
% (if the optional argument is given) a child file.
% Parameters are set as if the main file
% or a child file starting with |\childdocof| was compiled.
% Then compilation is handed over to the main file:
%    \begin{macrocode}
\newcommand{\childdocforward}[2][]
{
  \begingroup
    \if?#1?
      \def\childdoctmp
      {
        \def\childdocname{#2}
        \def\childdocjob{#2}
        \def\jobname{#2}
        \input{#2}
        \endinput
      }
    \else
      \def\childdoctmp
      {
        \childdocdisable
        \def\childdocname{#2}
        \childdoctrue
        \includeonly{#2}
        \def\childdocjob{#1}
        \def\jobname{#1}
        \input{#1}
        \endinput
      }
    \fi
    \expandafter
  \endgroup
  \childdoctmp
}
%    \end{macrocode}

% \macro{\childdocforwardprefix}
% The command |\childdocforwardprefix| redirects
% compilation to the main or a child file by means of a pattern.
% The prefix |#1| in the current filename is replaced by |#2|
% and the suffix of the current filename is kept
% (it is assumed that the filename does not contain the substring `|~~~|'
% which is used as a delimiter).
% Compilation is handed over to the new file by |\childdocforward|:
%    \begin{macrocode}
\newcommand{\childdocforwardprefix}[3][]
{
  \begingroup
    \def\childdocextract #2##1~~~{\def\childdoctmp{\childdocforward[#1]{#3##1}}}
    \expandafter\childdocextract\childdocname~~~
    \expandafter
  \endgroup
  \childdoctmp
}
%    \end{macrocode}

% \macro{\childdoc}
% The deprecated macro |\childdoc| is a legacy version of |\childdocmain|:
%    \begin{macrocode}
\newcommand{\childdoc}{\childdocmain}
%    \end{macrocode}

% \macro{\childdocredirect}
% The deprecated macro |\childdocredirect| is a legacy version
% of |\childdocforward| and |\childdocforwardprefix|:
%    \begin{macrocode}
\newcommand{\childdocredirect}[2][]
{
  \begingroup
    \if?#1?
      \def\childdoctmp{\childdocforward{#2}}
    \else
      \def\childdoctmp{\childdocforwardprefix{#1}{#2}}
    \fi
    \expandafter
  \endgroup
  \childdoctmp
}
%    \end{macrocode}

%\iffalse
%</package>
%\fi
%
\endinput

\childdocmain{}
%    \end{macrocode}

% Optional override for |\version| flag:
%    \begin{macrocode}
%%\ifchilddoc\else\providecommand{\version}{draft}\fi
%    \end{macrocode}

% Define the default values for the |\version| flag
% (|final| for the main file and |draft| for childs):
%    \begin{macrocode}
\ifchilddoc
\providecommand{\version}{draft}
\else
\providecommand{\version}{final}
\fi
%    \end{macrocode}

% Load the standard document class:
%    \begin{macrocode}
\documentclass[12pt]{article}
%    \end{macrocode}

% Start the document body:
%    \begin{macrocode}
\begin{document}
%    \end{macrocode}

% Declare a title page.
% Print title, part of document being processed and version flag:
%    \begin{macrocode}
\addtocounter{page}{-1}
\begin{center}
{\LARGE\bfseries{}childdoc example\par}
\vspace{1cm}
\ifchilddoc
\ifchilddocmanual part\else chapter\fi:
`\childdocname' of `\childdocjob'\par
\else
main document: `\childdocjob'\par
\fi
version: \version\par
\end{center}
\newpage
%    \end{macrocode}

% Manually include selected file,
% otherwise process as usual:
%    \begin{macrocode}
\ifchilddocmanual
\section*{part `\childdocname'}
\input{\childdocname}
\else
%    \end{macrocode}

% Include the two chapters:
%    \begin{macrocode}
\include{cdocsch1}
\include{cdocsch2}
%    \end{macrocode}

% Include the two parts unless only chapters should be displayed:
%    \begin{macrocode}
\ifchilddoc\else
\section{part three}
\input{cdocspt3}
\section{part four}
\input{cdocspt4}
\fi
%    \end{macrocode}

% Process as usual until here:
%    \begin{macrocode}
\fi
%    \end{macrocode}

% End of document body:
%    \begin{macrocode}
\end{document}
%    \end{macrocode}
%\iffalse
%</samplemain>
%\fi
%
% %%%%%%%%%%%%%%%%%%%%%%%%%%%%%%%%%%%%%%
% \paragraph{Chapter Include Files.}
%
% The include files are called |cdocsch1.tex| and |cdocsch2.tex|.
%
%\iffalse
%<*samplechap1|samplechap2>
%\fi

% Optional override for |\version| flag:
%    \begin{macrocode}
%%\providecommand{\version}{final}
%    \end{macrocode}

% Include the main document:
%    \begin{macrocode}
% \iffalse
%
% childdoc.dtx Copyright (C) 2017-2018 Niklas Beisert
%
% This work may be distributed and/or modified under the
% conditions of the LaTeX Project Public License, either version 1.3
% of this license or (at your option) any later version.
% The latest version of this license is in
%   http://www.latex-project.org/lppl.txt
% and version 1.3 or later is part of all distributions of LaTeX
% version 2005/12/01 or later.
%
% This work has the LPPL maintenance status `maintained'.
%
% The Current Maintainer of this work is Niklas Beisert.
%
% This work consists of the files childdoc.dtx and childdoc.ins
% and the derived files childdoc.def and cdocsamp.tex with
% cdocsch1.tex, cdocsch2.tex, cdocsdrf.tex, cdocsfn1.tex, cdocsfn2.tex.
%
%<package>\ifdefined\childdocmain\endinput\fi
%<package>\ProvidesFile{childdoc.def}[2018/12/30 v2.0 child document driver]
%<samplemain>\ProvidesFile{cdocsamp.tex}[2018/12/30 v2.0 sample for childdoc]
%<*driver>
%\ProvidesFile{childdoc.drv}[2018/12/30 v2.0 childdoc reference manual file]
\PassOptionsToClass{10pt,a4paper}{article}
\documentclass{ltxdoc}

\usepackage[margin=35mm]{geometry}
\usepackage{hyperref}
\usepackage{hyperxmp}
\usepackage[usenames]{color}

\hypersetup{colorlinks=true}
\hypersetup{pdfstartview=FitH}
\hypersetup{pdfpagemode=UseNone}
\hypersetup{pdfsource={}}
\hypersetup{pdflang={en-UK}}
\hypersetup{pdfcopyright={Copyright 2017-2018 Niklas Beisert.
  This work may be distributed and/or modified under the
  conditions of the LaTeX Project Public License, either version 1.3
  of this license or (at your option) any later version.}}
\hypersetup{pdflicenseurl={http://www.latex-project.org/lppl.txt}}
\hypersetup{pdfcontactaddress={ETH Zurich, ITP, HIT K,
  Wolfgang-Pauli-Strasse 27}}
\hypersetup{pdfcontactpostcode={8093}}
\hypersetup{pdfcontactcity={Zurich}}
\hypersetup{pdfcontactcountry={Switzerland}}
\hypersetup{pdfcontactemail={nbeisert@itp.phys.ethz.ch}}
\hypersetup{pdfcontacturl={http://people.phys.ethz.ch/\xmptilde nbeisert/}}

\newcommand{\secref}[1]{\hyperref[#1]{section \ref*{#1}}}

\parskip1ex
\parindent0pt
\let\olditemize\itemize
\def\itemize{\olditemize\parskip0pt}

\begin{document}

\title{The \textsf{childdoc} Package}
\hypersetup{pdftitle={The childdoc Package}}
\author{Niklas Beisert\\[2ex]
  Institut f\"ur Theoretische Physik\\
  Eidgen\"ossische Technische Hochschule Z\"urich\\
  Wolfgang-Pauli-Strasse 27, 8093 Z\"urich, Switzerland\\[1ex]
  \href{mailto:nbeisert@itp.phys.ethz.ch}
  {\texttt{nbeisert@itp.phys.ethz.ch}}}
\hypersetup{pdfauthor={Niklas Beisert}}
\hypersetup{pdfsubject={Manual for the LaTeX2e Package childdoc}}
\date{30 December 2018, \textsf{v2.0}}
\maketitle

\begin{abstract}\noindent
\textsf{childdoc} is a \LaTeXe{} package
that enables the direct compilation
of document sections included by |\include|
to individual files.
\end{abstract}

\begingroup
\parskip0ex
\tableofcontents
\endgroup

%%%%%%%%%%%%%%%%%%%%%%%%%%%%%%%%%%%%%%%%%%%%%%%%%%%%%%%%%%%%%%%%%%%%%%%%%%%%%%%%
%%%%%%%%%%%%%%%%%%%%%%%%%%%%%%%%%%%%%%%%%%%%%%%%%%%%%%%%%%%%%%%%%%%%%%%%%%%%%%%%
\section{Introduction}

\LaTeX{} provides a mechanism to structure a large document (such as a book)
into a main file and several child files (containing the chapters)
using the |\include| command.
This mechanism is beneficial for documents
which span hundreds of pages in order to
make the source file(s) more manageable.
Moreover, compilation can be restricted to
selected child files by means of the |\includeonly| command.
The latter feature can be used to reduce the compilation time while editing
(this was significantly more useful in the earlier days of \LaTeX{})
or to generate a smaller document which is easier to navigate.
Another application of |\includeonly| is to generate
documents consisting of selected parts of the complete document.

However, there are a few drawbacks of the plain |\include| mechanism:
\begin{itemize}
\item
The child files cannot be compiled on their own,
they can only be compiled via the main file.
A naive editing environment
(such as a text editor with an option
to have the current file processed by \LaTeX)
may require one to switch to the main file before compiling;
attempting to compile the child file produces errors.
\item
The main file must be modified (each time)
to adjust the |\includeonly| command
to the present needs. This easily leaves the main file in a messy state.
\item
The generated document will always carry the filename
of the main document. This is inconvenient if
several child files are to be compiled and
to be kept for distribution.
\end{itemize}

The present package provides a simple interface
to make child files individually compilable by \LaTeX{}.
Compiling a child file then has the same effect as compiling
the main file with an |\includeonly| command
to select the appropriate child.
Moreover the generated document will carry the name of the child
rather than the main file.
This resolves all three above issues.

This feature is meant to make the editing of books,
thesis documents and lecture notes somewhat more convenient.
However, the package can also be used efficiently for
composing a series of documents (such as exercise sheets)
which are typically distributed individually.
It then assists the author in generating the individual documents
(potentially in different versions)
as well as a document containing the collected series.
Another application is in developing style files
or other kinds of included material
where compilation of the style file could redirect
to a sample or test file.

%%%%%%%%%%%%%%%%%%%%%%%%%%%%%%%%%%%%%%%%%%%%%%%%%%%%%%%%%%%%%%%%%%%%%%%%%%%%%%%%
%%%%%%%%%%%%%%%%%%%%%%%%%%%%%%%%%%%%%%%%%%%%%%%%%%%%%%%%%%%%%%%%%%%%%%%%%%%%%%%%
\section{Usage}

First of all, the package \textsf{childdoc} is \emph{not} a standard
\LaTeXe{} |.sty| style file! Therefore it needs to be invoked in
a non-standard way.

%%%%%%%%%%%%%%%%%%%%%%%%%%%%%%%%%%%%%%%%%%%%%%%%%%%%%%%%%%%%%%%%%%%%%%%%%%%%%%%%
\subsection{Included Files}
\label{sec:include}

%%%%%%%%%%%%%%%%%%%%%%%%%%%%%%%%%%%%%%%%
\DescribeMacro{\childdocmain}
To use the package, add the commands
\begin{center}
\begin{tabular}{l}
|\input{childdoc.def}|\\
|\childdocmain{}|\\
\end{tabular}
\end{center}
at the very top of the main \LaTeX{} file,
in particular \emph{before} the |\documentclass| statement!
The argument of |\childdocmain| should be left empty
(but it must be present).

%%%%%%%%%%%%%%%%%%%%%%%%%%%%%%%%%%%%%%%%
\DescribeMacro{\childdocof}
Furthermore, add the commands
\begin{center}
\begin{tabular}{l}
|\input{childdoc.def}|\\
|\childdocof{|\textit{main}|}|\\
\end{tabular}
\end{center}
at the top of every child file \textit{child}
which is included by |\include{|\textit{child}|}|
from within the main file
(or at least for those files to be compiled individually).
The argument \textit{main} must be the filename of the main file.

There are a couple of
considerations in setting up the main and child documents:

%%%%%%%%%%%%%%%%%%%%%%%%%%%%%%%%%%%%%%%%
\paragraph{Restrictions.}

Please note the following restrictions:
\begin{itemize}
\item
|\childdocmain| must be called with one argument \textit{main}
to ensure compatibility with earlier version of the package.
It must either be empty (|\childdocmain{}|)
or precisely match the filename of the main file in which it is specified.
See \secref{sec:detection} for further information.
\item
The filename \textit{main} must be specified without the |.tex| extension.
\item
The filename \textit{main} is case sensitive
(even in case-insensitive file systems)
due to internal string comparison.
\item
The argument \textit{main} should be fully expanded, it cannot be a macro.
\item
Subdirectories and special characters should be avoided in filenames.
\item
The command |\childdocmain{|\textit{main}|}| must be followed by a whitespace.
It should not be followed immediately by another command
or by a comment mark `|%|'.
This is because the \TeX{} parser reads the token immediately following
the argument of |\childdocmain| and puts it
at the beginning of every child section;
however, a white\-space is ignored.
\end{itemize}

%%%%%%%%%%%%%%%%%%%%%%%%%%%%%%%%%%%%%%%%
\paragraph{Content of Main File.}

It is advisable to place all content in the child files included by |\include|.
Any output contained in the main file will appear in all child documents
unless suppressed manually;
it cannot be suppressed automatically by the |\includeonly| directive
and thus should normally be avoided.
A method to include some content in the main file
by means of conditional processing is described in \secref{sec:conditional}.

%%%%%%%%%%%%%%%%%%%%%%%%%%%%%%%%%%%%%%%%
\paragraph{Page Numbering.}

When only a part of the document is compiled,
the appropriate numbering of pages
(as well as other status parameters)
is determined from the |.aux| files.
The latter contain information from previous passes.
However this information needs to propagate through
all intermediate child documents.
Therefore the page numbering in child documents may well
be inconsistent until the complete document is compiled at least once.

A useful (if unconventional) way to always ensure a consistent
page numbering is to restart the numbering in each child document
and denote the pages by `\textit{child}|.|\textit{page}'
where \textit{child} represents the chapter/section number of the child file.
This can be achieved by the command
|\numberwithin{page}{|\textit{child}|}|
of the \textsf{amsmath} package
where \textit{child} can be |chapter| or |section|
depending on the chosen structuring.
Alternatively, one can modify the macro |\thepage| appropriately
and reset the counter |page| at the start of each child file.

%%%%%%%%%%%%%%%%%%%%%%%%%%%%%%%%%%%%%%%%%%%%%%%%%%%%%%%%%%%%%%%%%%%%%%%%%%%%%%%%
\subsection{Conditional Processing}
\label{sec:conditional}

The package provides a mechanism to compile different versions
of a document. To customise the versions further some conditional processing
can come in handy to distinguish which version is being compiled.
The package provides two macros to describe the compilation context:

%%%%%%%%%%%%%%%%%%%%%%%%%%%%%%%%%%%%%%%%
\DescribeMacro{\ifchilddoc}
The conditional |\ifchilddoc| distinguishes between the compilation of
child documents and the main document:
%
\begin{center}
|\ifchilddoc |\textit{child-code}| |[|\||else |\textit{main-code}]| \||fi|
\end{center}

%%%%%%%%%%%%%%%%%%%%%%%%%%%%%%%%%%%%%%%%
\DescribeMacro{\childdocname}
\DescribeMacro{\childdocjob}
The macro |\childdocname| contains the filename (without extension)
of the main or child file being processed.
Note that |\childdocjob| will always contain the name of the main file.

%%%%%%%%%%%%%%%%%%%%%%%%%%%%%%%%%%%%%%%%
\paragraph{Title Page.}

Conditional processing can be used to include a title or banner page
in the main document when proper precautions are taken.
Importantly, the code in the main file should ensure that the page counter
(as well as other status parameters which are stored in the |.aux| files)
takes the same value after the conditional processing.
Otherwise the page numbers may take divergent values
depending on which part is compiled.

For example, a title page could be declared by:
%
\begin{center}
\begin{tabular}{l}
|\ifchilddoc\||else|\\
|\addtocounter{page}{-1}|\\
\textit{code for title page}\\
|\newpage|\\
|\||fi|
\end{tabular}
\end{center}
%
A banner page for the child documents can be generated by:
%
\begin{center}
\begin{tabular}{l}
|\ifchilddoc|\\
|\addtocounter{page}{-1}|\\
\textit{code for banner page}\\
|\newpage|\\
|\||fi|
\end{tabular}
\end{center}
%
Here one could write a message such as:
\begin{center}
|This is the part \childdocname{} of \childdocjob{}.|
\end{center}

%%%%%%%%%%%%%%%%%%%%%%%%%%%%%%%%%%%%%%%%%%%%%%%%%%%%%%%%%%%%%%%%%%%%%%%%%%%%%%%%
\subsection{Flags}
\label{sec:flags}

The package makes it easy to generate different versions
of the main or child documents.
To this end compilation flags can be defined
and assigned different default values.
They will be particularly useful in conjunction
with the forwarding mechanism described in \secref{sec:forward}.

For example, it may be useful to have a flag |\version|
which can be set to |draft| or |final|.
The document source will contain some conditional code
depending on the value of |\version|.
Suppose further, the flag should default to |final| for the main file
and to |draft| for child files
which is a natural assignment for editing the document.
This is achieved by placing the following code
in the preamble of the main document
(below the |\childdocmain| directive):
%
\begin{center}
\begin{tabular}{l}
|\ifchilddoc|\\
|\providecommand{\version}{draft}|\\
|\||else|\\
|\providecommand{\version}{final}|\\
|\||fi|
\end{tabular}
\end{center}
%
The definition by |\providecommand| makes sure
that previous definitions are not overwritten.
Further statements |\providecommand{\version}{...}|
can thus be added before the above code to override it.

For the main file, one might add a line
(between |\childdocmain| and the above block)
%
\begin{center}
|%\ifchilddoc\||else\providecommand{\version}{draft}\||fi|
\end{center}
%
which can be uncommented to produce a draft version.
Likewise one can add a line to the very top of a child file
(above the |\childdocof{|\textit{main}|}| directive)
%
\begin{center}
|%\providecommand{\version}{final}|
\end{center}
%
which can be uncommented to produce the final version of this child document.

%%%%%%%%%%%%%%%%%%%%%%%%%%%%%%%%%%%%%%%%%%%%%%%%%%%%%%%%%%%%%%%%%%%%%%%%%%%%%%%%
\subsection{Forwarding}
\label{sec:forward}

Different versions of the main or child documents
using compilation flags as described in \secref{sec:flags}
can be (permanently) stored in different files
for convenient compilation, viewing and distribution.
To this end, the package defines a command
to pass on compilation to a different file:

%%%%%%%%%%%%%%%%%%%%%%%%%%%%%%%%%%%%%%%%
\DescribeMacro{\childdocforward}
The command |\childdocforward| redirects processing to
another source file:
%
\begin{center}
\begin{tabular}{l}
|\input{childdoc.def}|\\
|\childdocforward[|\textit{main}|]{|\textit{dest}|}|\\
\end{tabular}
\end{center}
%
The argument \textit{dest} is the destination file
(without extension).
It should be the main file or one of the child files.
Note that further \textsf{childdoc} directives
such as |\childdocof| and |\childdocforward|
in the indicated file will be processed in this form.
The optional argument \textit{main}
passes on directly to the main file \textit{main}
while pretending to compile the child \textit{dest}.
This form behaves as if \textit{dest}
issues |\childdocof{|\textit{main}|}| right away,
and no further \textsf{childdoc} directives will be processed.

%%%%%%%%%%%%%%%%%%%%%%%%%%%%%%%%%%%%%%%%
\DescribeMacro{\...prefix}
In the alternative form |\childdocforwardprefix|,
%
\begin{center}
\begin{tabular}{l}
|\input{childdoc.def}|\\
|\childdocforwardprefix[|\textit{main}|]{|\textit{prefix}|}{|\textit{dest}|}|
\end{tabular}
\end{center}
%
the destination file is determined by a pattern
depending on the current file:
To make this work, the current file must be called
`{\textit{prefix}\hspace{0.2em}\textit{suffix}}'
with \textit{prefix} matching precisely the argument.
Processing is then passed on to the file
`{\textit{dest}\hspace{0.2em}\textit{suffix}}'.
Surely, the same effect is achieved by
directly specifying the
argument `{\textit{dest}\hspace{0.2em}\textit{suffix}}'
in the first form.
However, that requires to set up a different file
for each child. With the alternative form of the command
all these files can have exactly the same content
which simplifies setting them up and maintaining them.

For example, the following file |draft.tex|
with a compilation flag |\version| as described in \secref{sec:flags}
compiles the main document as a draft:
%
\begin{center}
\begin{tabular}{l}
|\def\version{draft}|\\
|\input{childdoc.def}|\\
|\childdocforward{|\textit{main}|}|
\end{tabular}
\end{center}
%
Likewise, the following files |final|\textit{nn}|.tex|
compile the final version of the child document
|child|\textit{nn}|.tex|:
%
\begin{center}
\begin{tabular}{l}
|\def\version{final}|\\
|\input{childdoc.def}|\\
|\childdocforwardprefix{final}{child}|
\end{tabular}
\end{center}
%

Note that when several versions of a main file and/or of each child file
are to be generated, it may be convenient to set up a |Makefile| or
shell script to automatise the process.

%%%%%%%%%%%%%%%%%%%%%%%%%%%%%%%%%%%%%%%%%%%%%%%%%%%%%%%%%%%%%%%%%%%%%%%%%%%%%%%%
\subsection{Command Line Processing}
\label{sec:commandline}

The effect of redirection files can also be achieved by invoking
the \LaTeX{} compiler with a more elaborate command line.
Most conveniently this should be done as part
of a shell script or a |Makefile|.

When using \textsf{childdoc} in the main file, the following
command lines effectively perform a redirection
(note that depending on the shell being used,
backslashes may have to be doubled: `|\|' $\to$ `|\\|'):
%
\begin{center}
|... -jobname "|\textit{target}|" |\\|"|[\textit{flags}]%
|\input{childdoc.def}\childdocforward[|\textit{main}|]{|\textit{dest}|}"|
\end{center}
%
Here \textit{target} is the name of the output file,
\textit{main} is the name of the main file
and \textit{dest} is the name of the main or child file to be processed
(all filenames without extensions).
The optional argument \textit{main} can be omitted
if \textit{main} matches \textit{dest}.
Optionally, compilation \textit{flags} can be defined via |\def| commands.
This command line makes the \TeX{} engine believe
it is compiling the file \textit{target}
whose content is specified as the latter parameter.
The provided code then forwards the processing to
\textit{main} or \textit{dest} as described in \secref{sec:forward}.

%%%%%%%%%%%%%%%%%%%%%%%%%%%%%%%%%%%%%%%%%%%%%%%%%%%%%%%%%%%%%%%%%%%%%%%%%%%%%%%%
\subsection{Include by Input}
\label{sec:input}

Including child documents by |\include| has some restrictions by design.
Most notably, the content of a child document always occupies
its own set of pages; pages cannot be shared between child documents.
Usually, this behaviour makes perfect sense
because each child document contain an essential part of the document.
However, in some situations it may be desirable to compose
a document from a collection of parts
without having mandatory page breaks between then.
For this case, the package
provides a mechanism to include parts
by |\input| which can also be processed individually.
However, by construction this mechanism
requires manual handling of the content to be output.

%%%%%%%%%%%%%%%%%%%%%%%%%%%%%%%%%%%%%%%%
\DescribeMacro{\ifchilddocmanual}
The main file should be prepared as usual, see \secref{sec:include}.
However, the document body must make a distinction
between processing of an individual part and of the main document, e.g.:
%
\begin{center}
\begin{tabular}{l}
|\ifchilddocmanual|\\
|\input{\childdocname}|\\
|\||else|\\
\textit{document body with }|\input{|\textit{part}|}|\\
|\||fi|
\end{tabular}
\end{center}
%
The conditional |\ifchilddocmanual| is true whenever
a part to be included by |\input| is being compiled,
and the name of the part is stored in |\childdocname|.

%%%%%%%%%%%%%%%%%%%%%%%%%%%%%%%%%%%%%%%%
\DescribeMacro{\childdocby}
Each part to be included by |\input| should start with:
%
\begin{center}
\begin{tabular}{l}
|\input{childdoc.def}|\\
|\childdocby{|\textit{main}|}|\\
\end{tabular}
\end{center}
%
The directive |\childdocby| is similar to |\childdocof|
described in \secref{sec:include},
but the subsequent selection of content must be done manually.
To that end, both |\ifchilddoc| and |\ifchilddocmanual|
will be true upon processing of a part,
and the name of the part is stored in |\childdocname|.
Note that |\jobname| will be set to the filename of the current part
so that each part receives an individual |.aux| file
that does not interfere with the |.aux| file(s) of the main document.
This behaviour can be altered by the alternative form
|\childdocby[*]{|\textit{main}|}| (with a non-empty optional argument)
which uses the |.aux| file of the main document
by setting |\jobname| to \textit{main}.

%%%%%%%%%%%%%%%%%%%%%%%%%%%%%%%%%%%%%%%%%%%%%%%%%%%%%%%%%%%%%%%%%%%%%%%%%%%%%%%%
\subsection{Driver Development}
\label{sec:driver}

The \textsf{childdoc} mechanism can also be use for the development
of definition files such as \LaTeX{} styles or classes.
This case differs from the above setup with multiple parts
included by |\include| in that no |\includeonly| should be invoked.
This can be achieved by starting the include file
(before |\ProvidesPackage|) with:
%
\begin{center}
\begin{tabular}{l}
|\input{childdoc.def}|\\
|\childdocforward{|\textit{main}|}|\\
\end{tabular}
\end{center}
%
or alternatively with:
%
\begin{center}
\begin{tabular}{l}
|\input{childdoc.def}|\\
|\childdocby{|\textit{main}|}|\\
\end{tabular}
\end{center}
%
Both forms have slightly different effects as described above.
The main file is prepared as usual, see \secref{sec:include}.

%%%%%%%%%%%%%%%%%%%%%%%%%%%%%%%%%%%%%%%%%%%%%%%%%%%%%%%%%%%%%%%%%%%%%%%%%%%%%%%%
\subsection{Legacy Detection}
\label{sec:detection}

The directive |\childdocmain| in the main file can detect
whether the complete document or merely a child is to be compiled
even without using the directive |\childdocof|.
This method is deprecated because it is less robust
and there is no compelling reason to use it;
it is merely provided for backward compatibility
and it may be removed in future versions.

If the detection mechanism is to be used,
it is mandatory to correctly specify
the filename of the main file as the argument of |\childdocmain|:
%
\begin{center}
\begin{tabular}{l}
|\input{childdoc.def}|\\
|\childdocmain{|\textit{main}|}|\\
\end{tabular}
\end{center}
%
If |\jobname| does not match the argument \textit{main} of |\childdocmain|,
it is assumed that |\jobname| points to the child file to be compiled.
When using |\childdocmain| with the main file specified as argument,
it suffices to start a child file
with just |\input{|\textit{main}|}|
without loading of the package and using |\childdocof|.
If instead all processing is done
with the appropriate \textsf{childdoc} directives,
the argument of \textit{main} of |\childdocmain| can be empty.

An alternative version of the command line processing described
in \secref{sec:commandline} using the detection mechanism reads:
%
\begin{center}
|... -jobname "|\textit{target}|" "|[\textit{flags}]%
[|\def\jobname{|\textit{dest}|}|]|\input{|\textit{main}|}"|
\end{center}

%%%%%%%%%%%%%%%%%%%%%%%%%%%%%%%%%%%%%%%%%%%%%%%%%%%%%%%%%%%%%%%%%%%%%%%%%%%%%%%%
\subsection{Manual Code}
\label{sec:manual}

In case one cannot be certain whether the definitions file |childdoc.def|
is installed on the target \TeX{} distribution
and one prefers not to ship it,
it is conceivable to paste a few relevant commands into the sources.

To that end, drop all statements |\input{childdoc.def}|
and perform the replacements as outlined below.
Instead of |\childdocmain{|\textit{main}|}| add the following code
to the top of the main file:
%
\begin{center}
\begin{tabular}{l}
|\||ifdefined\childdocname\endinput\||fi\newif\ifchilddoc|\\
|\edef\childdocname{\scantokens\expandafter{\jobname\noexpand}}|\\
|\def\childdocmain{|\textit{main}|}\||ifx\childdocmain\childdocname\||else|\\
|\childdoctrue\includeonly{\childdocname}\let\jobname\childdocmain\||fi|\\
\end{tabular}
\end{center}
%
Instead of |\childdocof{|\textit{main}|}| just include the main file
at the top of each child file:
%
\begin{center}
|\input{|\textit{main}|}|
\end{center}
%
A simple redirection |\childdocforward{|\textit{dest}|}| is achieved by:
%
\begin{center}
|\def\jobname{|\textit{dest}|}\input{\jobname}|
\end{center}
%
The redirection with prefix
|\childdocforwardprefix[|\textit{prefix}|]{|\textit{dest}|}|
is accomplished by:
%
\begin{center}
\begin{tabular}{l}
|{\edef\jobname{\scantokens\expandafter{\jobname\noexpand}}|\\
|\def\redirectjob |\textit{prefix}|#1~~~{\gdef\jobname{|\textit{dest}|#1}}|\\
|\expandafter\redirectjob\jobname~~~}\input{\jobname}|
\end{tabular}
\end{center}

In an alternative approach,
child documents can be compiled by a specific command line
without additional code or specific definitions:
%
\begin{center}
|... -jobname "|\textit{target}|" "|[\textit{flags}]%
|\includeonly{|\textit{dest}|}\input{|\textit{main}|}"|
\end{center}
%

%%%%%%%%%%%%%%%%%%%%%%%%%%%%%%%%%%%%%%%%%%%%%%%%%%%%%%%%%%%%%%%%%%%%%%%%%%%%%%%%
%%%%%%%%%%%%%%%%%%%%%%%%%%%%%%%%%%%%%%%%%%%%%%%%%%%%%%%%%%%%%%%%%%%%%%%%%%%%%%%%
\section{Information}

%%%%%%%%%%%%%%%%%%%%%%%%%%%%%%%%%%%%%%%%%%%%%%%%%%%%%%%%%%%%%%%%%%%%%%%%%%%%%%%%
\subsection{Copyright}

Copyright \copyright{} 2017--2018 Niklas Beisert

This work may be distributed and/or modified under the
conditions of the \LaTeX{} Project Public License, either version 1.3
of this license or (at your option) any later version.
The latest version of this license is in
  \url{http://www.latex-project.org/lppl.txt}
and version 1.3 or later is part of all distributions of \LaTeX{}
version 2005/12/01 or later.

This work has the LPPL maintenance status `maintained'.

The Current Maintainer of this work is Niklas Beisert.

This work consists of the files |README.txt|, |childdoc.ins| and |childdoc.dtx|
as well as the derived files |childdoc.def|, |cdocsamp.tex|
with |cdocsch1.tex|, |cdocsch2.tex|, |cdocspt3.tex|, |cdocspt4.tex|,
|cdocsdrf.tex|, |cdocsfn1.tex|, |cdocsfn2.tex|
as well as |childdoc.pdf|.

%%%%%%%%%%%%%%%%%%%%%%%%%%%%%%%%%%%%%%%%%%%%%%%%%%%%%%%%%%%%%%%%%%%%%%%%%%%%%%%%
\subsection{Files and Installation}

The package consists of the files:
%
\begin{center}
\begin{tabular}{ll}
    |README.txt|   & readme file \\
    |childdoc.ins| & installation file \\
    |childdoc.dtx| & source file \\
    |childdoc.def| & definition file \\
    |cdocsamp.tex| & sample main file \\
    |cdocsch1.tex| & sample include file \\
    |cdocsch2.tex| & sample include file \\
    |cdocspt3.tex| & sample part file \\
    |cdocspt4.tex| & sample part file \\
    |cdocsdrf.tex| & sample redirection file \\
    |cdocsfn1.tex| & sample redirection file \\
    |cdocsfn2.tex| & sample redirection file \\
    |childdoc.pdf| & manual
\end{tabular}
\end{center}
%
The distribution consists of the files
|README.txt|, |childdoc.ins| and |childdoc.dtx|.
%
\begin{itemize}
\item
Run (pdf)\LaTeX{} on |childdoc.dtx|
to compile the manual |childdoc.pdf| (this file).
\item
Run \LaTeX{} on |childdoc.ins| to create the definitions file |childdoc.def|
and the sample |cdocsamp.tex| with include files
|cdocsch1.tex|, |cdocsch2.tex|, |cdocspt3.tex|, |cdocspt4.tex|,
|cdocsdrf.tex|, |cdocsfn1.tex|, |cdocsfn2.tex|.
Then copy the file |childdoc.def| to an appropriate directory of your \LaTeX{}
distribution, e.g.\ \textit{texmf-root}|/tex/latex/childdoc|.
\end{itemize}

%%%%%%%%%%%%%%%%%%%%%%%%%%%%%%%%%%%%%%%%%%%%%%%%%%%%%%%%%%%%%%%%%%%%%%%%%%%%%%%%
\subsection{Related CTAN Packages}

There are several other packages which offer a similar functionality:
%
\begin{itemize}
\item
The packages
\href{http://ctan.org/pkg/docmute}{\textsf{docmute}},
\href{http://ctan.org/pkg/includex}{\textsf{includex}} and
\href{http://ctan.org/pkg/standalone}{\textsf{standalone}}
provide commands to include only the document body of
a child file thus allowing both files to be compiled individually.
\item
The packages \href{http://ctan.org/pkg/subdocs}{\textsf{subdocs}}
and \href{http://ctan.org/pkg/subfiles}{\textsf{subfiles}}
provide structures in which the main and child documents can be
encapsulated and allowing them to be compiled individually.
The inclusion mechanism is different from the conventional |\include|.
\item
The package \href{http://ctan.org/pkg/combine}{\textsf{combine}}
is an elaborate solution to combine several documents into one.
\end{itemize}
%
See also the CTAN topic \href{http://ctan.org/topic/subdocs}{\textsf{subdocs}}
for further related packages.
The present package differs from the above solutions in that
a document structure constructed with the conventional |\include| mechanism
just needs two extra commands at the top of every file
such that all constituent files can be compiled individually.

%%%%%%%%%%%%%%%%%%%%%%%%%%%%%%%%%%%%%%%%%%%%%%%%%%%%%%%%%%%%%%%%%%%%%%%%%%%%%%%%
%\subsection{Feature Suggestions}
%
%The following is a list of features which may be useful for future
%versions of this package:
%%
%\begin{itemize}
%\item
%\ldots
%\end{itemize}

%%%%%%%%%%%%%%%%%%%%%%%%%%%%%%%%%%%%%%%%%%%%%%%%%%%%%%%%%%%%%%%%%%%%%%%%%%%%%%%%
\subsection{Revision History}

%%%%%%%%%%%%%%%%%%%%%%%%%%%%%%%%%%%%%%%%
\paragraph{v2.0:} 2018/12/30

\begin{itemize}
\item
immediate forward processing
\item
added |\childdocby| mechanism
\item
manual restructured
\end{itemize}

%%%%%%%%%%%%%%%%%%%%%%%%%%%%%%%%%%%%%%%%
\paragraph{v1.6:} 2018/01/17

\begin{itemize}
\item
application for development of include files
\item
corrections to manual
\end{itemize}

%%%%%%%%%%%%%%%%%%%%%%%%%%%%%%%%%%%%%%%%
\paragraph{v1.5:} 2017/05/21

\begin{itemize}
\item
more complete structuring introduced
\item
|\childdocof| introduced
\item
|\childdoc| renamed to |\childdocmain|
\item
|\childredirect| renamed to |\childdocforward| and |\childdocforwardprefix|
and functionality expanded
\end{itemize}

%%%%%%%%%%%%%%%%%%%%%%%%%%%%%%%%%%%%%%%%
\paragraph{v1.0:} 2017/04/27

\begin{itemize}
\item
manual and install package
\item
first version published on CTAN
\end{itemize}

%%%%%%%%%%%%%%%%%%%%%%%%%%%%%%%%%%%%%%%%
\paragraph{v0.6:} 2017/04/26

\begin{itemize}
\item
redirection mechanism added
\end{itemize}

%%%%%%%%%%%%%%%%%%%%%%%%%%%%%%%%%%%%%%%%
\paragraph{v0.5:} 2017/04/26

\begin{itemize}
\item
functionality in definition file
\end{itemize}


%%%%%%%%%%%%%%%%%%%%%%%%%%%%%%%%%%%%%%%%%%%%%%%%%%%%%%%%%%%%%%%%%%%%%%%%%%%%%%%%
%%%%%%%%%%%%%%%%%%%%%%%%%%%%%%%%%%%%%%%%%%%%%%%%%%%%%%%%%%%%%%%%%%%%%%%%%%%%%%%%
%%%%%%%%%%%%%%%%%%%%%%%%%%%%%%%%%%%%%%%%%%%%%%%%%%%%%%%%%%%%%%%%%%%%%%%%%%%%%%%%
\appendix

\settowidth\MacroIndent{\rmfamily\scriptsize 000\ }

 \DocInput{childdoc.dtx}

\end{document}
%</driver>
% \fi
%
% %%%%%%%%%%%%%%%%%%%%%%%%%%%%%%%%%%%%%%%%%%%%%%%%%%%%%%%%%%%%%%%%%%%%%%%%%%%%%%
% %%%%%%%%%%%%%%%%%%%%%%%%%%%%%%%%%%%%%%%%%%%%%%%%%%%%%%%%%%%%%%%%%%%%%%%%%%%%%%
% \section{Sample}
%\iffalse
%<*samplemain>
%\fi
%
% The following presents a sample document
% with two chapters, two parts, a title page,
% a compile flag as well as three forwarding files to set the flag.
% It consists of eight |.tex| files:
% \begin{center}
% \begin{tabular}{ll}
% |cdocsamp.tex|&main file\\
% |cdocsch1.tex|&include file for chapter 1\\
% |cdocsch2.tex|&include file for chapter 2\\
% |cdocspt3.tex|&include file for part 3\\
% |cdocspt4.tex|&include file for part 4\\
% |cdocsdrf.tex|&forwarding file for main file in draft mode\\
% |cdocsfi1.tex|&forwarding file for final version of chapter 1\\
% |cdocsfi2.tex|&forwarding file for final version of chapter 2\\
% \end{tabular}
% \end{center}
% Each of the eight files can be compiled directly by the \LaTeX{} compiler.
%
% %%%%%%%%%%%%%%%%%%%%%%%%%%%%%%%%%%%%%%
% \paragraph{Main File.}
%
% The main file is called |cdocsamp.tex|.
%
% Load the \textsf{childdoc} definitions and
% declare the filename for the main document:
%    \begin{macrocode}
\input{childdoc.def}
\childdocmain{}
%    \end{macrocode}

% Optional override for |\version| flag:
%    \begin{macrocode}
%%\ifchilddoc\else\providecommand{\version}{draft}\fi
%    \end{macrocode}

% Define the default values for the |\version| flag
% (|final| for the main file and |draft| for childs):
%    \begin{macrocode}
\ifchilddoc
\providecommand{\version}{draft}
\else
\providecommand{\version}{final}
\fi
%    \end{macrocode}

% Load the standard document class:
%    \begin{macrocode}
\documentclass[12pt]{article}
%    \end{macrocode}

% Start the document body:
%    \begin{macrocode}
\begin{document}
%    \end{macrocode}

% Declare a title page.
% Print title, part of document being processed and version flag:
%    \begin{macrocode}
\addtocounter{page}{-1}
\begin{center}
{\LARGE\bfseries{}childdoc example\par}
\vspace{1cm}
\ifchilddoc
\ifchilddocmanual part\else chapter\fi:
`\childdocname' of `\childdocjob'\par
\else
main document: `\childdocjob'\par
\fi
version: \version\par
\end{center}
\newpage
%    \end{macrocode}

% Manually include selected file,
% otherwise process as usual:
%    \begin{macrocode}
\ifchilddocmanual
\section*{part `\childdocname'}
\input{\childdocname}
\else
%    \end{macrocode}

% Include the two chapters:
%    \begin{macrocode}
\include{cdocsch1}
\include{cdocsch2}
%    \end{macrocode}

% Include the two parts unless only chapters should be displayed:
%    \begin{macrocode}
\ifchilddoc\else
\section{part three}
\input{cdocspt3}
\section{part four}
\input{cdocspt4}
\fi
%    \end{macrocode}

% Process as usual until here:
%    \begin{macrocode}
\fi
%    \end{macrocode}

% End of document body:
%    \begin{macrocode}
\end{document}
%    \end{macrocode}
%\iffalse
%</samplemain>
%\fi
%
% %%%%%%%%%%%%%%%%%%%%%%%%%%%%%%%%%%%%%%
% \paragraph{Chapter Include Files.}
%
% The include files are called |cdocsch1.tex| and |cdocsch2.tex|.
%
%\iffalse
%<*samplechap1|samplechap2>
%\fi

% Optional override for |\version| flag:
%    \begin{macrocode}
%%\providecommand{\version}{final}
%    \end{macrocode}

% Include the main document:
%    \begin{macrocode}
\input{childdoc.def}
\childdocof{cdocsamp}
%    \end{macrocode}

%\iffalse
%</samplechap1|samplechap2>
%\fi
%
%\iffalse
%<*samplechap1>
%\fi
% Some text for chapter 1:
%    \begin{macrocode}
\section{one}
some text in chapter one
%    \end{macrocode}

%\iffalse
%</samplechap1>
%\fi
% Some text for chapter 2:
%\iffalse
%<*samplechap2>
%\fi
%    \begin{macrocode}
\section{two}
more text in chapter two
%    \end{macrocode}

%\iffalse
%</samplechap2>
%\fi
%
% %%%%%%%%%%%%%%%%%%%%%%%%%%%%%%%%%%%%%%
% \paragraph{Part Include Files.}
%
% The include files are called |cdocspt3.tex| and |cdocspt4.tex|.
%
%\iffalse
%<*samplepart3|samplepart4>
%\fi

% Optional override for |\version| flag:
%    \begin{macrocode}
%%\providecommand{\version}{final}
%    \end{macrocode}

% Include the main document:
%    \begin{macrocode}
\input{childdoc.def}
\childdocby{cdocsamp}
%    \end{macrocode}

%\iffalse
%</samplepart3|samplepart4>
%\fi
%
%\iffalse
%<*samplepart3>
%\fi
% Some text for part 3:
%    \begin{macrocode}
some text in part three
%    \end{macrocode}

%\iffalse
%</samplepart3>
%\fi
% Some text for part 4:
%\iffalse
%<*samplepart4>
%\fi
%    \begin{macrocode}
more text in part four
%    \end{macrocode}

%\iffalse
%</samplepart4>
%\fi
%
% %%%%%%%%%%%%%%%%%%%%%%%%%%%%%%%%%%%%%%
% \paragraph{Forwarding for a Complete Draft.}
%
% The following forwarding file |cdocsdrf.tex|
% compiles the main document in draft mode:
%\iffalse
%<*sampledraft>
%\fi
%    \begin{macrocode}
\def\version{draft}
\input{childdoc.def}
\childdocforward{cdocsamp}
%    \end{macrocode}

%\iffalse
%</sampledraft>
%\fi
%
% %%%%%%%%%%%%%%%%%%%%%%%%%%%%%%%%%%%%%%
% \paragraph{Forwarding for Final Version of the Chapters.}
%
% The following forwarding files |cdocsfn1.tex| and |cdocsfn2.tex|
% (with identical content)
% compile the final versions of the child documents
% |cdocsch1.tex| and |cdocsch2.tex|, respectively:
%\iffalse
%<*samplefinal>
%\fi
%    \begin{macrocode}
\def\version{final}
\input{childdoc.def}
\childdocforwardprefix[cdocsamp]{cdocsfn}{cdocsch}
%    \end{macrocode}

%\iffalse
%</samplefinal>
%\fi
%
% %%%%%%%%%%%%%%%%%%%%%%%%%%%%%%%%%%%%%%
% \paragraph{Command Line Processing.}
%
% The following three command lines generate the output files
% |cdocscld|, |cdocscl1| and |cdocscl2|
% which should be identical to
% |cdocsdrf|, |cdocsch1| and |cdocsfn2|, respectively:
% \begin{center}
% \begin{tabular}{l}
% |latex -jobname cdocscld \|\\
% |  "\def\version{draft}\input{childdoc.def}\childdocforward{cdocsamp}"|\\
% |latex -jobname cdocscl1 \|\\
% |  "\input{childdoc.def}\childdocforward[cdocsamp]{cdocsch1}"|\\
% |latex -jobname cdocscl2 \|\\
% |  "\def\version{final}\input{childdoc.def}\childdocforward{cdocsch2}"|
% \end{tabular}
% \end{center}
% Note that the trailing backslash on each first line
% merely continues the input to the second line
% (for convenient cut ant paste).
% Furthermore, the command |latex| can be replaced by any
% of its alternative versions such as |pdflatex|.
%
% %%%%%%%%%%%%%%%%%%%%%%%%%%%%%%%%%%%%%%%%%%%%%%%%%%%%%%%%%%%%%%%%%%%%%%%%%%%%%%
% %%%%%%%%%%%%%%%%%%%%%%%%%%%%%%%%%%%%%%%%%%%%%%%%%%%%%%%%%%%%%%%%%%%%%%%%%%%%%%
% \section{Implementation}
%\iffalse
%<*package>
%\fi
%
% This section describes the definitions file |childdoc.def|.

% The definitions cannot be loaded using |\usepackage| or |\RequirePackage|
% which has a mechanism to prevent loading a style file more than once.
% When loading the definitions by means of |\input|
% multiple instances have to be prevented manually:
%\iffalse
%This code needs to be before the `\ProvidesFile' directive
%which is defined at the beginning of this file.
%Therefore it is also placed there and commented out here.
%</package>
%<*discard>
%\fi
%    \begin{macrocode}
\ifdefined\childdocmain\endinput\fi
%    \end{macrocode}
%\iffalse
%</discard>
%<*package>
%\fi
%
% \macro{\ifchilddoc}
% \macro{\ifchilddocmanual}
% The conditional |\ifchilddoc| tells whether a
% child (true) or main (false) document is being compiled.
% The conditional |\ifchilddocmanual| tells whether
% the |\includeonly| mechanism is used (false) or
% the selection of child files must be performed manually (true).
% The definitions initialise to false:
%    \begin{macrocode}
\newif\ifchilddoc
\newif\ifchilddocmanual
%    \end{macrocode}

% \macro{\childdocname}
% \macro{\childdocjob}
% The macro |\childdocname| stores the name of the main document
% to be compiled. The macro |\childdocjob| stores the name of
% the document on which the \LaTeX{} compiler was originally invoked.
% The content of |\jobname| cannot be compared
% to filenames specified in the source due to different catcodes.
% The following code rescans |\jobname|, stores the result
% in |\childdocname| and saves a copy in |\childdocjob|:
%    \begin{macrocode}
\edef\childdocname{\scantokens\expandafter{\jobname\noexpand}}
\let\childdocjob\childdocname
%    \end{macrocode}

% \macro{\childdocdisable}
% The macro |\childdocdisable| prevents the main file
% from being processed more than once.
% At this stage, the main document command |\childdocmain|
% is assumed to be called once again where it should do nothing.
% Any subsequent call to it should prevent
% a secondary processing of the main document
% It overwrites the forwarding commands
% |\childdocof| and |\childdocforward|
% with empty macros to prevent further inclusions of the main document:
%    \begin{macrocode}
\newcommand{\childdocdisable}
{
  \renewcommand{\childdocmain}[1]{\renewcommand{\childdocmain}[1]{\endinput}}
  \renewcommand{\childdocof}[1]{}
  \renewcommand{\childdocby}[2][]{}
  \renewcommand{\childdocforward}[2][]{}
  \renewcommand{\childdocdisable}{}
}
%    \end{macrocode}

% \macro{\childdocmain}
% The macro |\childdocmain| is to be called at the top of the main file
% with nothing or the main filename (without extension) as argument.
% First, it breaks loops.
% If the argument is not empty and does not match |\childdocname|
% (which is set by the first inclusion of |childdoc.def|),
% |\ifchilddoc| is set to true, |\includeonly| is applied to the child file
% and |\jobname| is set to the main file
% (for proper handling of |.aux| files):
%    \begin{macrocode}
\newcommand{\childdocmain}[1]
{
  \childdocdisable\childdocmain{}
  \if?#1?\else
    \begingroup
      \def\childdoctmp{#1}
      \ifx\childdoctmp\childdocname
        \def\childdoctmp{}
      \else
        \def\childdoctmp
        {
          \childdoctrue
          \includeonly{\childdocname}
          \def\childdocjob{#1}
          \def\jobname{#1}
        }
      \fi
      \expandafter
    \endgroup
    \childdoctmp
  \fi
}
%    \end{macrocode}

% \macro{\childdocof}
% The command |\childdocof| redirects
% compilation to the main file |#1|.
%    \begin{macrocode}
\newcommand{\childdocof}[1]
{
  \childdocdisable
  \childdoctrue
  \includeonly{\childdocname}
  \def\jobname{#1}
  \def\childdocjob{#1}
  \input{#1}
}
%    \end{macrocode}

% \macro{\childdocby}
% The command |\childdocby| ....
%    \begin{macrocode}
\newcommand{\childdocby}[2][]
{
  \childdocdisable
  \childdoctrue
  \childdocmanualtrue
  \if?#1?\else
    \def\jobname{#2}
  \fi
  \def\childdocjob{#2}
  \input{#2}
  \endinput
}
%    \end{macrocode}

% \macro{\childdocforward}
% The command |\childdocforward| redirects
% compilation to the main file or
% (if the optional argument is given) a child file.
% Parameters are set as if the main file
% or a child file starting with |\childdocof| was compiled.
% Then compilation is handed over to the main file:
%    \begin{macrocode}
\newcommand{\childdocforward}[2][]
{
  \begingroup
    \if?#1?
      \def\childdoctmp
      {
        \def\childdocname{#2}
        \def\childdocjob{#2}
        \def\jobname{#2}
        \input{#2}
        \endinput
      }
    \else
      \def\childdoctmp
      {
        \childdocdisable
        \def\childdocname{#2}
        \childdoctrue
        \includeonly{#2}
        \def\childdocjob{#1}
        \def\jobname{#1}
        \input{#1}
        \endinput
      }
    \fi
    \expandafter
  \endgroup
  \childdoctmp
}
%    \end{macrocode}

% \macro{\childdocforwardprefix}
% The command |\childdocforwardprefix| redirects
% compilation to the main or a child file by means of a pattern.
% The prefix |#1| in the current filename is replaced by |#2|
% and the suffix of the current filename is kept
% (it is assumed that the filename does not contain the substring `|~~~|'
% which is used as a delimiter).
% Compilation is handed over to the new file by |\childdocforward|:
%    \begin{macrocode}
\newcommand{\childdocforwardprefix}[3][]
{
  \begingroup
    \def\childdocextract #2##1~~~{\def\childdoctmp{\childdocforward[#1]{#3##1}}}
    \expandafter\childdocextract\childdocname~~~
    \expandafter
  \endgroup
  \childdoctmp
}
%    \end{macrocode}

% \macro{\childdoc}
% The deprecated macro |\childdoc| is a legacy version of |\childdocmain|:
%    \begin{macrocode}
\newcommand{\childdoc}{\childdocmain}
%    \end{macrocode}

% \macro{\childdocredirect}
% The deprecated macro |\childdocredirect| is a legacy version
% of |\childdocforward| and |\childdocforwardprefix|:
%    \begin{macrocode}
\newcommand{\childdocredirect}[2][]
{
  \begingroup
    \if?#1?
      \def\childdoctmp{\childdocforward{#2}}
    \else
      \def\childdoctmp{\childdocforwardprefix{#1}{#2}}
    \fi
    \expandafter
  \endgroup
  \childdoctmp
}
%    \end{macrocode}

%\iffalse
%</package>
%\fi
%
\endinput

\childdocof{cdocsamp}
%    \end{macrocode}

%\iffalse
%</samplechap1|samplechap2>
%\fi
%
%\iffalse
%<*samplechap1>
%\fi
% Some text for chapter 1:
%    \begin{macrocode}
\section{one}
some text in chapter one
%    \end{macrocode}

%\iffalse
%</samplechap1>
%\fi
% Some text for chapter 2:
%\iffalse
%<*samplechap2>
%\fi
%    \begin{macrocode}
\section{two}
more text in chapter two
%    \end{macrocode}

%\iffalse
%</samplechap2>
%\fi
%
% %%%%%%%%%%%%%%%%%%%%%%%%%%%%%%%%%%%%%%
% \paragraph{Part Include Files.}
%
% The include files are called |cdocspt3.tex| and |cdocspt4.tex|.
%
%\iffalse
%<*samplepart3|samplepart4>
%\fi

% Optional override for |\version| flag:
%    \begin{macrocode}
%%\providecommand{\version}{final}
%    \end{macrocode}

% Include the main document:
%    \begin{macrocode}
% \iffalse
%
% childdoc.dtx Copyright (C) 2017-2018 Niklas Beisert
%
% This work may be distributed and/or modified under the
% conditions of the LaTeX Project Public License, either version 1.3
% of this license or (at your option) any later version.
% The latest version of this license is in
%   http://www.latex-project.org/lppl.txt
% and version 1.3 or later is part of all distributions of LaTeX
% version 2005/12/01 or later.
%
% This work has the LPPL maintenance status `maintained'.
%
% The Current Maintainer of this work is Niklas Beisert.
%
% This work consists of the files childdoc.dtx and childdoc.ins
% and the derived files childdoc.def and cdocsamp.tex with
% cdocsch1.tex, cdocsch2.tex, cdocsdrf.tex, cdocsfn1.tex, cdocsfn2.tex.
%
%<package>\ifdefined\childdocmain\endinput\fi
%<package>\ProvidesFile{childdoc.def}[2018/12/30 v2.0 child document driver]
%<samplemain>\ProvidesFile{cdocsamp.tex}[2018/12/30 v2.0 sample for childdoc]
%<*driver>
%\ProvidesFile{childdoc.drv}[2018/12/30 v2.0 childdoc reference manual file]
\PassOptionsToClass{10pt,a4paper}{article}
\documentclass{ltxdoc}

\usepackage[margin=35mm]{geometry}
\usepackage{hyperref}
\usepackage{hyperxmp}
\usepackage[usenames]{color}

\hypersetup{colorlinks=true}
\hypersetup{pdfstartview=FitH}
\hypersetup{pdfpagemode=UseNone}
\hypersetup{pdfsource={}}
\hypersetup{pdflang={en-UK}}
\hypersetup{pdfcopyright={Copyright 2017-2018 Niklas Beisert.
  This work may be distributed and/or modified under the
  conditions of the LaTeX Project Public License, either version 1.3
  of this license or (at your option) any later version.}}
\hypersetup{pdflicenseurl={http://www.latex-project.org/lppl.txt}}
\hypersetup{pdfcontactaddress={ETH Zurich, ITP, HIT K,
  Wolfgang-Pauli-Strasse 27}}
\hypersetup{pdfcontactpostcode={8093}}
\hypersetup{pdfcontactcity={Zurich}}
\hypersetup{pdfcontactcountry={Switzerland}}
\hypersetup{pdfcontactemail={nbeisert@itp.phys.ethz.ch}}
\hypersetup{pdfcontacturl={http://people.phys.ethz.ch/\xmptilde nbeisert/}}

\newcommand{\secref}[1]{\hyperref[#1]{section \ref*{#1}}}

\parskip1ex
\parindent0pt
\let\olditemize\itemize
\def\itemize{\olditemize\parskip0pt}

\begin{document}

\title{The \textsf{childdoc} Package}
\hypersetup{pdftitle={The childdoc Package}}
\author{Niklas Beisert\\[2ex]
  Institut f\"ur Theoretische Physik\\
  Eidgen\"ossische Technische Hochschule Z\"urich\\
  Wolfgang-Pauli-Strasse 27, 8093 Z\"urich, Switzerland\\[1ex]
  \href{mailto:nbeisert@itp.phys.ethz.ch}
  {\texttt{nbeisert@itp.phys.ethz.ch}}}
\hypersetup{pdfauthor={Niklas Beisert}}
\hypersetup{pdfsubject={Manual for the LaTeX2e Package childdoc}}
\date{30 December 2018, \textsf{v2.0}}
\maketitle

\begin{abstract}\noindent
\textsf{childdoc} is a \LaTeXe{} package
that enables the direct compilation
of document sections included by |\include|
to individual files.
\end{abstract}

\begingroup
\parskip0ex
\tableofcontents
\endgroup

%%%%%%%%%%%%%%%%%%%%%%%%%%%%%%%%%%%%%%%%%%%%%%%%%%%%%%%%%%%%%%%%%%%%%%%%%%%%%%%%
%%%%%%%%%%%%%%%%%%%%%%%%%%%%%%%%%%%%%%%%%%%%%%%%%%%%%%%%%%%%%%%%%%%%%%%%%%%%%%%%
\section{Introduction}

\LaTeX{} provides a mechanism to structure a large document (such as a book)
into a main file and several child files (containing the chapters)
using the |\include| command.
This mechanism is beneficial for documents
which span hundreds of pages in order to
make the source file(s) more manageable.
Moreover, compilation can be restricted to
selected child files by means of the |\includeonly| command.
The latter feature can be used to reduce the compilation time while editing
(this was significantly more useful in the earlier days of \LaTeX{})
or to generate a smaller document which is easier to navigate.
Another application of |\includeonly| is to generate
documents consisting of selected parts of the complete document.

However, there are a few drawbacks of the plain |\include| mechanism:
\begin{itemize}
\item
The child files cannot be compiled on their own,
they can only be compiled via the main file.
A naive editing environment
(such as a text editor with an option
to have the current file processed by \LaTeX)
may require one to switch to the main file before compiling;
attempting to compile the child file produces errors.
\item
The main file must be modified (each time)
to adjust the |\includeonly| command
to the present needs. This easily leaves the main file in a messy state.
\item
The generated document will always carry the filename
of the main document. This is inconvenient if
several child files are to be compiled and
to be kept for distribution.
\end{itemize}

The present package provides a simple interface
to make child files individually compilable by \LaTeX{}.
Compiling a child file then has the same effect as compiling
the main file with an |\includeonly| command
to select the appropriate child.
Moreover the generated document will carry the name of the child
rather than the main file.
This resolves all three above issues.

This feature is meant to make the editing of books,
thesis documents and lecture notes somewhat more convenient.
However, the package can also be used efficiently for
composing a series of documents (such as exercise sheets)
which are typically distributed individually.
It then assists the author in generating the individual documents
(potentially in different versions)
as well as a document containing the collected series.
Another application is in developing style files
or other kinds of included material
where compilation of the style file could redirect
to a sample or test file.

%%%%%%%%%%%%%%%%%%%%%%%%%%%%%%%%%%%%%%%%%%%%%%%%%%%%%%%%%%%%%%%%%%%%%%%%%%%%%%%%
%%%%%%%%%%%%%%%%%%%%%%%%%%%%%%%%%%%%%%%%%%%%%%%%%%%%%%%%%%%%%%%%%%%%%%%%%%%%%%%%
\section{Usage}

First of all, the package \textsf{childdoc} is \emph{not} a standard
\LaTeXe{} |.sty| style file! Therefore it needs to be invoked in
a non-standard way.

%%%%%%%%%%%%%%%%%%%%%%%%%%%%%%%%%%%%%%%%%%%%%%%%%%%%%%%%%%%%%%%%%%%%%%%%%%%%%%%%
\subsection{Included Files}
\label{sec:include}

%%%%%%%%%%%%%%%%%%%%%%%%%%%%%%%%%%%%%%%%
\DescribeMacro{\childdocmain}
To use the package, add the commands
\begin{center}
\begin{tabular}{l}
|\input{childdoc.def}|\\
|\childdocmain{}|\\
\end{tabular}
\end{center}
at the very top of the main \LaTeX{} file,
in particular \emph{before} the |\documentclass| statement!
The argument of |\childdocmain| should be left empty
(but it must be present).

%%%%%%%%%%%%%%%%%%%%%%%%%%%%%%%%%%%%%%%%
\DescribeMacro{\childdocof}
Furthermore, add the commands
\begin{center}
\begin{tabular}{l}
|\input{childdoc.def}|\\
|\childdocof{|\textit{main}|}|\\
\end{tabular}
\end{center}
at the top of every child file \textit{child}
which is included by |\include{|\textit{child}|}|
from within the main file
(or at least for those files to be compiled individually).
The argument \textit{main} must be the filename of the main file.

There are a couple of
considerations in setting up the main and child documents:

%%%%%%%%%%%%%%%%%%%%%%%%%%%%%%%%%%%%%%%%
\paragraph{Restrictions.}

Please note the following restrictions:
\begin{itemize}
\item
|\childdocmain| must be called with one argument \textit{main}
to ensure compatibility with earlier version of the package.
It must either be empty (|\childdocmain{}|)
or precisely match the filename of the main file in which it is specified.
See \secref{sec:detection} for further information.
\item
The filename \textit{main} must be specified without the |.tex| extension.
\item
The filename \textit{main} is case sensitive
(even in case-insensitive file systems)
due to internal string comparison.
\item
The argument \textit{main} should be fully expanded, it cannot be a macro.
\item
Subdirectories and special characters should be avoided in filenames.
\item
The command |\childdocmain{|\textit{main}|}| must be followed by a whitespace.
It should not be followed immediately by another command
or by a comment mark `|%|'.
This is because the \TeX{} parser reads the token immediately following
the argument of |\childdocmain| and puts it
at the beginning of every child section;
however, a white\-space is ignored.
\end{itemize}

%%%%%%%%%%%%%%%%%%%%%%%%%%%%%%%%%%%%%%%%
\paragraph{Content of Main File.}

It is advisable to place all content in the child files included by |\include|.
Any output contained in the main file will appear in all child documents
unless suppressed manually;
it cannot be suppressed automatically by the |\includeonly| directive
and thus should normally be avoided.
A method to include some content in the main file
by means of conditional processing is described in \secref{sec:conditional}.

%%%%%%%%%%%%%%%%%%%%%%%%%%%%%%%%%%%%%%%%
\paragraph{Page Numbering.}

When only a part of the document is compiled,
the appropriate numbering of pages
(as well as other status parameters)
is determined from the |.aux| files.
The latter contain information from previous passes.
However this information needs to propagate through
all intermediate child documents.
Therefore the page numbering in child documents may well
be inconsistent until the complete document is compiled at least once.

A useful (if unconventional) way to always ensure a consistent
page numbering is to restart the numbering in each child document
and denote the pages by `\textit{child}|.|\textit{page}'
where \textit{child} represents the chapter/section number of the child file.
This can be achieved by the command
|\numberwithin{page}{|\textit{child}|}|
of the \textsf{amsmath} package
where \textit{child} can be |chapter| or |section|
depending on the chosen structuring.
Alternatively, one can modify the macro |\thepage| appropriately
and reset the counter |page| at the start of each child file.

%%%%%%%%%%%%%%%%%%%%%%%%%%%%%%%%%%%%%%%%%%%%%%%%%%%%%%%%%%%%%%%%%%%%%%%%%%%%%%%%
\subsection{Conditional Processing}
\label{sec:conditional}

The package provides a mechanism to compile different versions
of a document. To customise the versions further some conditional processing
can come in handy to distinguish which version is being compiled.
The package provides two macros to describe the compilation context:

%%%%%%%%%%%%%%%%%%%%%%%%%%%%%%%%%%%%%%%%
\DescribeMacro{\ifchilddoc}
The conditional |\ifchilddoc| distinguishes between the compilation of
child documents and the main document:
%
\begin{center}
|\ifchilddoc |\textit{child-code}| |[|\||else |\textit{main-code}]| \||fi|
\end{center}

%%%%%%%%%%%%%%%%%%%%%%%%%%%%%%%%%%%%%%%%
\DescribeMacro{\childdocname}
\DescribeMacro{\childdocjob}
The macro |\childdocname| contains the filename (without extension)
of the main or child file being processed.
Note that |\childdocjob| will always contain the name of the main file.

%%%%%%%%%%%%%%%%%%%%%%%%%%%%%%%%%%%%%%%%
\paragraph{Title Page.}

Conditional processing can be used to include a title or banner page
in the main document when proper precautions are taken.
Importantly, the code in the main file should ensure that the page counter
(as well as other status parameters which are stored in the |.aux| files)
takes the same value after the conditional processing.
Otherwise the page numbers may take divergent values
depending on which part is compiled.

For example, a title page could be declared by:
%
\begin{center}
\begin{tabular}{l}
|\ifchilddoc\||else|\\
|\addtocounter{page}{-1}|\\
\textit{code for title page}\\
|\newpage|\\
|\||fi|
\end{tabular}
\end{center}
%
A banner page for the child documents can be generated by:
%
\begin{center}
\begin{tabular}{l}
|\ifchilddoc|\\
|\addtocounter{page}{-1}|\\
\textit{code for banner page}\\
|\newpage|\\
|\||fi|
\end{tabular}
\end{center}
%
Here one could write a message such as:
\begin{center}
|This is the part \childdocname{} of \childdocjob{}.|
\end{center}

%%%%%%%%%%%%%%%%%%%%%%%%%%%%%%%%%%%%%%%%%%%%%%%%%%%%%%%%%%%%%%%%%%%%%%%%%%%%%%%%
\subsection{Flags}
\label{sec:flags}

The package makes it easy to generate different versions
of the main or child documents.
To this end compilation flags can be defined
and assigned different default values.
They will be particularly useful in conjunction
with the forwarding mechanism described in \secref{sec:forward}.

For example, it may be useful to have a flag |\version|
which can be set to |draft| or |final|.
The document source will contain some conditional code
depending on the value of |\version|.
Suppose further, the flag should default to |final| for the main file
and to |draft| for child files
which is a natural assignment for editing the document.
This is achieved by placing the following code
in the preamble of the main document
(below the |\childdocmain| directive):
%
\begin{center}
\begin{tabular}{l}
|\ifchilddoc|\\
|\providecommand{\version}{draft}|\\
|\||else|\\
|\providecommand{\version}{final}|\\
|\||fi|
\end{tabular}
\end{center}
%
The definition by |\providecommand| makes sure
that previous definitions are not overwritten.
Further statements |\providecommand{\version}{...}|
can thus be added before the above code to override it.

For the main file, one might add a line
(between |\childdocmain| and the above block)
%
\begin{center}
|%\ifchilddoc\||else\providecommand{\version}{draft}\||fi|
\end{center}
%
which can be uncommented to produce a draft version.
Likewise one can add a line to the very top of a child file
(above the |\childdocof{|\textit{main}|}| directive)
%
\begin{center}
|%\providecommand{\version}{final}|
\end{center}
%
which can be uncommented to produce the final version of this child document.

%%%%%%%%%%%%%%%%%%%%%%%%%%%%%%%%%%%%%%%%%%%%%%%%%%%%%%%%%%%%%%%%%%%%%%%%%%%%%%%%
\subsection{Forwarding}
\label{sec:forward}

Different versions of the main or child documents
using compilation flags as described in \secref{sec:flags}
can be (permanently) stored in different files
for convenient compilation, viewing and distribution.
To this end, the package defines a command
to pass on compilation to a different file:

%%%%%%%%%%%%%%%%%%%%%%%%%%%%%%%%%%%%%%%%
\DescribeMacro{\childdocforward}
The command |\childdocforward| redirects processing to
another source file:
%
\begin{center}
\begin{tabular}{l}
|\input{childdoc.def}|\\
|\childdocforward[|\textit{main}|]{|\textit{dest}|}|\\
\end{tabular}
\end{center}
%
The argument \textit{dest} is the destination file
(without extension).
It should be the main file or one of the child files.
Note that further \textsf{childdoc} directives
such as |\childdocof| and |\childdocforward|
in the indicated file will be processed in this form.
The optional argument \textit{main}
passes on directly to the main file \textit{main}
while pretending to compile the child \textit{dest}.
This form behaves as if \textit{dest}
issues |\childdocof{|\textit{main}|}| right away,
and no further \textsf{childdoc} directives will be processed.

%%%%%%%%%%%%%%%%%%%%%%%%%%%%%%%%%%%%%%%%
\DescribeMacro{\...prefix}
In the alternative form |\childdocforwardprefix|,
%
\begin{center}
\begin{tabular}{l}
|\input{childdoc.def}|\\
|\childdocforwardprefix[|\textit{main}|]{|\textit{prefix}|}{|\textit{dest}|}|
\end{tabular}
\end{center}
%
the destination file is determined by a pattern
depending on the current file:
To make this work, the current file must be called
`{\textit{prefix}\hspace{0.2em}\textit{suffix}}'
with \textit{prefix} matching precisely the argument.
Processing is then passed on to the file
`{\textit{dest}\hspace{0.2em}\textit{suffix}}'.
Surely, the same effect is achieved by
directly specifying the
argument `{\textit{dest}\hspace{0.2em}\textit{suffix}}'
in the first form.
However, that requires to set up a different file
for each child. With the alternative form of the command
all these files can have exactly the same content
which simplifies setting them up and maintaining them.

For example, the following file |draft.tex|
with a compilation flag |\version| as described in \secref{sec:flags}
compiles the main document as a draft:
%
\begin{center}
\begin{tabular}{l}
|\def\version{draft}|\\
|\input{childdoc.def}|\\
|\childdocforward{|\textit{main}|}|
\end{tabular}
\end{center}
%
Likewise, the following files |final|\textit{nn}|.tex|
compile the final version of the child document
|child|\textit{nn}|.tex|:
%
\begin{center}
\begin{tabular}{l}
|\def\version{final}|\\
|\input{childdoc.def}|\\
|\childdocforwardprefix{final}{child}|
\end{tabular}
\end{center}
%

Note that when several versions of a main file and/or of each child file
are to be generated, it may be convenient to set up a |Makefile| or
shell script to automatise the process.

%%%%%%%%%%%%%%%%%%%%%%%%%%%%%%%%%%%%%%%%%%%%%%%%%%%%%%%%%%%%%%%%%%%%%%%%%%%%%%%%
\subsection{Command Line Processing}
\label{sec:commandline}

The effect of redirection files can also be achieved by invoking
the \LaTeX{} compiler with a more elaborate command line.
Most conveniently this should be done as part
of a shell script or a |Makefile|.

When using \textsf{childdoc} in the main file, the following
command lines effectively perform a redirection
(note that depending on the shell being used,
backslashes may have to be doubled: `|\|' $\to$ `|\\|'):
%
\begin{center}
|... -jobname "|\textit{target}|" |\\|"|[\textit{flags}]%
|\input{childdoc.def}\childdocforward[|\textit{main}|]{|\textit{dest}|}"|
\end{center}
%
Here \textit{target} is the name of the output file,
\textit{main} is the name of the main file
and \textit{dest} is the name of the main or child file to be processed
(all filenames without extensions).
The optional argument \textit{main} can be omitted
if \textit{main} matches \textit{dest}.
Optionally, compilation \textit{flags} can be defined via |\def| commands.
This command line makes the \TeX{} engine believe
it is compiling the file \textit{target}
whose content is specified as the latter parameter.
The provided code then forwards the processing to
\textit{main} or \textit{dest} as described in \secref{sec:forward}.

%%%%%%%%%%%%%%%%%%%%%%%%%%%%%%%%%%%%%%%%%%%%%%%%%%%%%%%%%%%%%%%%%%%%%%%%%%%%%%%%
\subsection{Include by Input}
\label{sec:input}

Including child documents by |\include| has some restrictions by design.
Most notably, the content of a child document always occupies
its own set of pages; pages cannot be shared between child documents.
Usually, this behaviour makes perfect sense
because each child document contain an essential part of the document.
However, in some situations it may be desirable to compose
a document from a collection of parts
without having mandatory page breaks between then.
For this case, the package
provides a mechanism to include parts
by |\input| which can also be processed individually.
However, by construction this mechanism
requires manual handling of the content to be output.

%%%%%%%%%%%%%%%%%%%%%%%%%%%%%%%%%%%%%%%%
\DescribeMacro{\ifchilddocmanual}
The main file should be prepared as usual, see \secref{sec:include}.
However, the document body must make a distinction
between processing of an individual part and of the main document, e.g.:
%
\begin{center}
\begin{tabular}{l}
|\ifchilddocmanual|\\
|\input{\childdocname}|\\
|\||else|\\
\textit{document body with }|\input{|\textit{part}|}|\\
|\||fi|
\end{tabular}
\end{center}
%
The conditional |\ifchilddocmanual| is true whenever
a part to be included by |\input| is being compiled,
and the name of the part is stored in |\childdocname|.

%%%%%%%%%%%%%%%%%%%%%%%%%%%%%%%%%%%%%%%%
\DescribeMacro{\childdocby}
Each part to be included by |\input| should start with:
%
\begin{center}
\begin{tabular}{l}
|\input{childdoc.def}|\\
|\childdocby{|\textit{main}|}|\\
\end{tabular}
\end{center}
%
The directive |\childdocby| is similar to |\childdocof|
described in \secref{sec:include},
but the subsequent selection of content must be done manually.
To that end, both |\ifchilddoc| and |\ifchilddocmanual|
will be true upon processing of a part,
and the name of the part is stored in |\childdocname|.
Note that |\jobname| will be set to the filename of the current part
so that each part receives an individual |.aux| file
that does not interfere with the |.aux| file(s) of the main document.
This behaviour can be altered by the alternative form
|\childdocby[*]{|\textit{main}|}| (with a non-empty optional argument)
which uses the |.aux| file of the main document
by setting |\jobname| to \textit{main}.

%%%%%%%%%%%%%%%%%%%%%%%%%%%%%%%%%%%%%%%%%%%%%%%%%%%%%%%%%%%%%%%%%%%%%%%%%%%%%%%%
\subsection{Driver Development}
\label{sec:driver}

The \textsf{childdoc} mechanism can also be use for the development
of definition files such as \LaTeX{} styles or classes.
This case differs from the above setup with multiple parts
included by |\include| in that no |\includeonly| should be invoked.
This can be achieved by starting the include file
(before |\ProvidesPackage|) with:
%
\begin{center}
\begin{tabular}{l}
|\input{childdoc.def}|\\
|\childdocforward{|\textit{main}|}|\\
\end{tabular}
\end{center}
%
or alternatively with:
%
\begin{center}
\begin{tabular}{l}
|\input{childdoc.def}|\\
|\childdocby{|\textit{main}|}|\\
\end{tabular}
\end{center}
%
Both forms have slightly different effects as described above.
The main file is prepared as usual, see \secref{sec:include}.

%%%%%%%%%%%%%%%%%%%%%%%%%%%%%%%%%%%%%%%%%%%%%%%%%%%%%%%%%%%%%%%%%%%%%%%%%%%%%%%%
\subsection{Legacy Detection}
\label{sec:detection}

The directive |\childdocmain| in the main file can detect
whether the complete document or merely a child is to be compiled
even without using the directive |\childdocof|.
This method is deprecated because it is less robust
and there is no compelling reason to use it;
it is merely provided for backward compatibility
and it may be removed in future versions.

If the detection mechanism is to be used,
it is mandatory to correctly specify
the filename of the main file as the argument of |\childdocmain|:
%
\begin{center}
\begin{tabular}{l}
|\input{childdoc.def}|\\
|\childdocmain{|\textit{main}|}|\\
\end{tabular}
\end{center}
%
If |\jobname| does not match the argument \textit{main} of |\childdocmain|,
it is assumed that |\jobname| points to the child file to be compiled.
When using |\childdocmain| with the main file specified as argument,
it suffices to start a child file
with just |\input{|\textit{main}|}|
without loading of the package and using |\childdocof|.
If instead all processing is done
with the appropriate \textsf{childdoc} directives,
the argument of \textit{main} of |\childdocmain| can be empty.

An alternative version of the command line processing described
in \secref{sec:commandline} using the detection mechanism reads:
%
\begin{center}
|... -jobname "|\textit{target}|" "|[\textit{flags}]%
[|\def\jobname{|\textit{dest}|}|]|\input{|\textit{main}|}"|
\end{center}

%%%%%%%%%%%%%%%%%%%%%%%%%%%%%%%%%%%%%%%%%%%%%%%%%%%%%%%%%%%%%%%%%%%%%%%%%%%%%%%%
\subsection{Manual Code}
\label{sec:manual}

In case one cannot be certain whether the definitions file |childdoc.def|
is installed on the target \TeX{} distribution
and one prefers not to ship it,
it is conceivable to paste a few relevant commands into the sources.

To that end, drop all statements |\input{childdoc.def}|
and perform the replacements as outlined below.
Instead of |\childdocmain{|\textit{main}|}| add the following code
to the top of the main file:
%
\begin{center}
\begin{tabular}{l}
|\||ifdefined\childdocname\endinput\||fi\newif\ifchilddoc|\\
|\edef\childdocname{\scantokens\expandafter{\jobname\noexpand}}|\\
|\def\childdocmain{|\textit{main}|}\||ifx\childdocmain\childdocname\||else|\\
|\childdoctrue\includeonly{\childdocname}\let\jobname\childdocmain\||fi|\\
\end{tabular}
\end{center}
%
Instead of |\childdocof{|\textit{main}|}| just include the main file
at the top of each child file:
%
\begin{center}
|\input{|\textit{main}|}|
\end{center}
%
A simple redirection |\childdocforward{|\textit{dest}|}| is achieved by:
%
\begin{center}
|\def\jobname{|\textit{dest}|}\input{\jobname}|
\end{center}
%
The redirection with prefix
|\childdocforwardprefix[|\textit{prefix}|]{|\textit{dest}|}|
is accomplished by:
%
\begin{center}
\begin{tabular}{l}
|{\edef\jobname{\scantokens\expandafter{\jobname\noexpand}}|\\
|\def\redirectjob |\textit{prefix}|#1~~~{\gdef\jobname{|\textit{dest}|#1}}|\\
|\expandafter\redirectjob\jobname~~~}\input{\jobname}|
\end{tabular}
\end{center}

In an alternative approach,
child documents can be compiled by a specific command line
without additional code or specific definitions:
%
\begin{center}
|... -jobname "|\textit{target}|" "|[\textit{flags}]%
|\includeonly{|\textit{dest}|}\input{|\textit{main}|}"|
\end{center}
%

%%%%%%%%%%%%%%%%%%%%%%%%%%%%%%%%%%%%%%%%%%%%%%%%%%%%%%%%%%%%%%%%%%%%%%%%%%%%%%%%
%%%%%%%%%%%%%%%%%%%%%%%%%%%%%%%%%%%%%%%%%%%%%%%%%%%%%%%%%%%%%%%%%%%%%%%%%%%%%%%%
\section{Information}

%%%%%%%%%%%%%%%%%%%%%%%%%%%%%%%%%%%%%%%%%%%%%%%%%%%%%%%%%%%%%%%%%%%%%%%%%%%%%%%%
\subsection{Copyright}

Copyright \copyright{} 2017--2018 Niklas Beisert

This work may be distributed and/or modified under the
conditions of the \LaTeX{} Project Public License, either version 1.3
of this license or (at your option) any later version.
The latest version of this license is in
  \url{http://www.latex-project.org/lppl.txt}
and version 1.3 or later is part of all distributions of \LaTeX{}
version 2005/12/01 or later.

This work has the LPPL maintenance status `maintained'.

The Current Maintainer of this work is Niklas Beisert.

This work consists of the files |README.txt|, |childdoc.ins| and |childdoc.dtx|
as well as the derived files |childdoc.def|, |cdocsamp.tex|
with |cdocsch1.tex|, |cdocsch2.tex|, |cdocspt3.tex|, |cdocspt4.tex|,
|cdocsdrf.tex|, |cdocsfn1.tex|, |cdocsfn2.tex|
as well as |childdoc.pdf|.

%%%%%%%%%%%%%%%%%%%%%%%%%%%%%%%%%%%%%%%%%%%%%%%%%%%%%%%%%%%%%%%%%%%%%%%%%%%%%%%%
\subsection{Files and Installation}

The package consists of the files:
%
\begin{center}
\begin{tabular}{ll}
    |README.txt|   & readme file \\
    |childdoc.ins| & installation file \\
    |childdoc.dtx| & source file \\
    |childdoc.def| & definition file \\
    |cdocsamp.tex| & sample main file \\
    |cdocsch1.tex| & sample include file \\
    |cdocsch2.tex| & sample include file \\
    |cdocspt3.tex| & sample part file \\
    |cdocspt4.tex| & sample part file \\
    |cdocsdrf.tex| & sample redirection file \\
    |cdocsfn1.tex| & sample redirection file \\
    |cdocsfn2.tex| & sample redirection file \\
    |childdoc.pdf| & manual
\end{tabular}
\end{center}
%
The distribution consists of the files
|README.txt|, |childdoc.ins| and |childdoc.dtx|.
%
\begin{itemize}
\item
Run (pdf)\LaTeX{} on |childdoc.dtx|
to compile the manual |childdoc.pdf| (this file).
\item
Run \LaTeX{} on |childdoc.ins| to create the definitions file |childdoc.def|
and the sample |cdocsamp.tex| with include files
|cdocsch1.tex|, |cdocsch2.tex|, |cdocspt3.tex|, |cdocspt4.tex|,
|cdocsdrf.tex|, |cdocsfn1.tex|, |cdocsfn2.tex|.
Then copy the file |childdoc.def| to an appropriate directory of your \LaTeX{}
distribution, e.g.\ \textit{texmf-root}|/tex/latex/childdoc|.
\end{itemize}

%%%%%%%%%%%%%%%%%%%%%%%%%%%%%%%%%%%%%%%%%%%%%%%%%%%%%%%%%%%%%%%%%%%%%%%%%%%%%%%%
\subsection{Related CTAN Packages}

There are several other packages which offer a similar functionality:
%
\begin{itemize}
\item
The packages
\href{http://ctan.org/pkg/docmute}{\textsf{docmute}},
\href{http://ctan.org/pkg/includex}{\textsf{includex}} and
\href{http://ctan.org/pkg/standalone}{\textsf{standalone}}
provide commands to include only the document body of
a child file thus allowing both files to be compiled individually.
\item
The packages \href{http://ctan.org/pkg/subdocs}{\textsf{subdocs}}
and \href{http://ctan.org/pkg/subfiles}{\textsf{subfiles}}
provide structures in which the main and child documents can be
encapsulated and allowing them to be compiled individually.
The inclusion mechanism is different from the conventional |\include|.
\item
The package \href{http://ctan.org/pkg/combine}{\textsf{combine}}
is an elaborate solution to combine several documents into one.
\end{itemize}
%
See also the CTAN topic \href{http://ctan.org/topic/subdocs}{\textsf{subdocs}}
for further related packages.
The present package differs from the above solutions in that
a document structure constructed with the conventional |\include| mechanism
just needs two extra commands at the top of every file
such that all constituent files can be compiled individually.

%%%%%%%%%%%%%%%%%%%%%%%%%%%%%%%%%%%%%%%%%%%%%%%%%%%%%%%%%%%%%%%%%%%%%%%%%%%%%%%%
%\subsection{Feature Suggestions}
%
%The following is a list of features which may be useful for future
%versions of this package:
%%
%\begin{itemize}
%\item
%\ldots
%\end{itemize}

%%%%%%%%%%%%%%%%%%%%%%%%%%%%%%%%%%%%%%%%%%%%%%%%%%%%%%%%%%%%%%%%%%%%%%%%%%%%%%%%
\subsection{Revision History}

%%%%%%%%%%%%%%%%%%%%%%%%%%%%%%%%%%%%%%%%
\paragraph{v2.0:} 2018/12/30

\begin{itemize}
\item
immediate forward processing
\item
added |\childdocby| mechanism
\item
manual restructured
\end{itemize}

%%%%%%%%%%%%%%%%%%%%%%%%%%%%%%%%%%%%%%%%
\paragraph{v1.6:} 2018/01/17

\begin{itemize}
\item
application for development of include files
\item
corrections to manual
\end{itemize}

%%%%%%%%%%%%%%%%%%%%%%%%%%%%%%%%%%%%%%%%
\paragraph{v1.5:} 2017/05/21

\begin{itemize}
\item
more complete structuring introduced
\item
|\childdocof| introduced
\item
|\childdoc| renamed to |\childdocmain|
\item
|\childredirect| renamed to |\childdocforward| and |\childdocforwardprefix|
and functionality expanded
\end{itemize}

%%%%%%%%%%%%%%%%%%%%%%%%%%%%%%%%%%%%%%%%
\paragraph{v1.0:} 2017/04/27

\begin{itemize}
\item
manual and install package
\item
first version published on CTAN
\end{itemize}

%%%%%%%%%%%%%%%%%%%%%%%%%%%%%%%%%%%%%%%%
\paragraph{v0.6:} 2017/04/26

\begin{itemize}
\item
redirection mechanism added
\end{itemize}

%%%%%%%%%%%%%%%%%%%%%%%%%%%%%%%%%%%%%%%%
\paragraph{v0.5:} 2017/04/26

\begin{itemize}
\item
functionality in definition file
\end{itemize}


%%%%%%%%%%%%%%%%%%%%%%%%%%%%%%%%%%%%%%%%%%%%%%%%%%%%%%%%%%%%%%%%%%%%%%%%%%%%%%%%
%%%%%%%%%%%%%%%%%%%%%%%%%%%%%%%%%%%%%%%%%%%%%%%%%%%%%%%%%%%%%%%%%%%%%%%%%%%%%%%%
%%%%%%%%%%%%%%%%%%%%%%%%%%%%%%%%%%%%%%%%%%%%%%%%%%%%%%%%%%%%%%%%%%%%%%%%%%%%%%%%
\appendix

\settowidth\MacroIndent{\rmfamily\scriptsize 000\ }

 \DocInput{childdoc.dtx}

\end{document}
%</driver>
% \fi
%
% %%%%%%%%%%%%%%%%%%%%%%%%%%%%%%%%%%%%%%%%%%%%%%%%%%%%%%%%%%%%%%%%%%%%%%%%%%%%%%
% %%%%%%%%%%%%%%%%%%%%%%%%%%%%%%%%%%%%%%%%%%%%%%%%%%%%%%%%%%%%%%%%%%%%%%%%%%%%%%
% \section{Sample}
%\iffalse
%<*samplemain>
%\fi
%
% The following presents a sample document
% with two chapters, two parts, a title page,
% a compile flag as well as three forwarding files to set the flag.
% It consists of eight |.tex| files:
% \begin{center}
% \begin{tabular}{ll}
% |cdocsamp.tex|&main file\\
% |cdocsch1.tex|&include file for chapter 1\\
% |cdocsch2.tex|&include file for chapter 2\\
% |cdocspt3.tex|&include file for part 3\\
% |cdocspt4.tex|&include file for part 4\\
% |cdocsdrf.tex|&forwarding file for main file in draft mode\\
% |cdocsfi1.tex|&forwarding file for final version of chapter 1\\
% |cdocsfi2.tex|&forwarding file for final version of chapter 2\\
% \end{tabular}
% \end{center}
% Each of the eight files can be compiled directly by the \LaTeX{} compiler.
%
% %%%%%%%%%%%%%%%%%%%%%%%%%%%%%%%%%%%%%%
% \paragraph{Main File.}
%
% The main file is called |cdocsamp.tex|.
%
% Load the \textsf{childdoc} definitions and
% declare the filename for the main document:
%    \begin{macrocode}
\input{childdoc.def}
\childdocmain{}
%    \end{macrocode}

% Optional override for |\version| flag:
%    \begin{macrocode}
%%\ifchilddoc\else\providecommand{\version}{draft}\fi
%    \end{macrocode}

% Define the default values for the |\version| flag
% (|final| for the main file and |draft| for childs):
%    \begin{macrocode}
\ifchilddoc
\providecommand{\version}{draft}
\else
\providecommand{\version}{final}
\fi
%    \end{macrocode}

% Load the standard document class:
%    \begin{macrocode}
\documentclass[12pt]{article}
%    \end{macrocode}

% Start the document body:
%    \begin{macrocode}
\begin{document}
%    \end{macrocode}

% Declare a title page.
% Print title, part of document being processed and version flag:
%    \begin{macrocode}
\addtocounter{page}{-1}
\begin{center}
{\LARGE\bfseries{}childdoc example\par}
\vspace{1cm}
\ifchilddoc
\ifchilddocmanual part\else chapter\fi:
`\childdocname' of `\childdocjob'\par
\else
main document: `\childdocjob'\par
\fi
version: \version\par
\end{center}
\newpage
%    \end{macrocode}

% Manually include selected file,
% otherwise process as usual:
%    \begin{macrocode}
\ifchilddocmanual
\section*{part `\childdocname'}
\input{\childdocname}
\else
%    \end{macrocode}

% Include the two chapters:
%    \begin{macrocode}
\include{cdocsch1}
\include{cdocsch2}
%    \end{macrocode}

% Include the two parts unless only chapters should be displayed:
%    \begin{macrocode}
\ifchilddoc\else
\section{part three}
\input{cdocspt3}
\section{part four}
\input{cdocspt4}
\fi
%    \end{macrocode}

% Process as usual until here:
%    \begin{macrocode}
\fi
%    \end{macrocode}

% End of document body:
%    \begin{macrocode}
\end{document}
%    \end{macrocode}
%\iffalse
%</samplemain>
%\fi
%
% %%%%%%%%%%%%%%%%%%%%%%%%%%%%%%%%%%%%%%
% \paragraph{Chapter Include Files.}
%
% The include files are called |cdocsch1.tex| and |cdocsch2.tex|.
%
%\iffalse
%<*samplechap1|samplechap2>
%\fi

% Optional override for |\version| flag:
%    \begin{macrocode}
%%\providecommand{\version}{final}
%    \end{macrocode}

% Include the main document:
%    \begin{macrocode}
\input{childdoc.def}
\childdocof{cdocsamp}
%    \end{macrocode}

%\iffalse
%</samplechap1|samplechap2>
%\fi
%
%\iffalse
%<*samplechap1>
%\fi
% Some text for chapter 1:
%    \begin{macrocode}
\section{one}
some text in chapter one
%    \end{macrocode}

%\iffalse
%</samplechap1>
%\fi
% Some text for chapter 2:
%\iffalse
%<*samplechap2>
%\fi
%    \begin{macrocode}
\section{two}
more text in chapter two
%    \end{macrocode}

%\iffalse
%</samplechap2>
%\fi
%
% %%%%%%%%%%%%%%%%%%%%%%%%%%%%%%%%%%%%%%
% \paragraph{Part Include Files.}
%
% The include files are called |cdocspt3.tex| and |cdocspt4.tex|.
%
%\iffalse
%<*samplepart3|samplepart4>
%\fi

% Optional override for |\version| flag:
%    \begin{macrocode}
%%\providecommand{\version}{final}
%    \end{macrocode}

% Include the main document:
%    \begin{macrocode}
\input{childdoc.def}
\childdocby{cdocsamp}
%    \end{macrocode}

%\iffalse
%</samplepart3|samplepart4>
%\fi
%
%\iffalse
%<*samplepart3>
%\fi
% Some text for part 3:
%    \begin{macrocode}
some text in part three
%    \end{macrocode}

%\iffalse
%</samplepart3>
%\fi
% Some text for part 4:
%\iffalse
%<*samplepart4>
%\fi
%    \begin{macrocode}
more text in part four
%    \end{macrocode}

%\iffalse
%</samplepart4>
%\fi
%
% %%%%%%%%%%%%%%%%%%%%%%%%%%%%%%%%%%%%%%
% \paragraph{Forwarding for a Complete Draft.}
%
% The following forwarding file |cdocsdrf.tex|
% compiles the main document in draft mode:
%\iffalse
%<*sampledraft>
%\fi
%    \begin{macrocode}
\def\version{draft}
\input{childdoc.def}
\childdocforward{cdocsamp}
%    \end{macrocode}

%\iffalse
%</sampledraft>
%\fi
%
% %%%%%%%%%%%%%%%%%%%%%%%%%%%%%%%%%%%%%%
% \paragraph{Forwarding for Final Version of the Chapters.}
%
% The following forwarding files |cdocsfn1.tex| and |cdocsfn2.tex|
% (with identical content)
% compile the final versions of the child documents
% |cdocsch1.tex| and |cdocsch2.tex|, respectively:
%\iffalse
%<*samplefinal>
%\fi
%    \begin{macrocode}
\def\version{final}
\input{childdoc.def}
\childdocforwardprefix[cdocsamp]{cdocsfn}{cdocsch}
%    \end{macrocode}

%\iffalse
%</samplefinal>
%\fi
%
% %%%%%%%%%%%%%%%%%%%%%%%%%%%%%%%%%%%%%%
% \paragraph{Command Line Processing.}
%
% The following three command lines generate the output files
% |cdocscld|, |cdocscl1| and |cdocscl2|
% which should be identical to
% |cdocsdrf|, |cdocsch1| and |cdocsfn2|, respectively:
% \begin{center}
% \begin{tabular}{l}
% |latex -jobname cdocscld \|\\
% |  "\def\version{draft}\input{childdoc.def}\childdocforward{cdocsamp}"|\\
% |latex -jobname cdocscl1 \|\\
% |  "\input{childdoc.def}\childdocforward[cdocsamp]{cdocsch1}"|\\
% |latex -jobname cdocscl2 \|\\
% |  "\def\version{final}\input{childdoc.def}\childdocforward{cdocsch2}"|
% \end{tabular}
% \end{center}
% Note that the trailing backslash on each first line
% merely continues the input to the second line
% (for convenient cut ant paste).
% Furthermore, the command |latex| can be replaced by any
% of its alternative versions such as |pdflatex|.
%
% %%%%%%%%%%%%%%%%%%%%%%%%%%%%%%%%%%%%%%%%%%%%%%%%%%%%%%%%%%%%%%%%%%%%%%%%%%%%%%
% %%%%%%%%%%%%%%%%%%%%%%%%%%%%%%%%%%%%%%%%%%%%%%%%%%%%%%%%%%%%%%%%%%%%%%%%%%%%%%
% \section{Implementation}
%\iffalse
%<*package>
%\fi
%
% This section describes the definitions file |childdoc.def|.

% The definitions cannot be loaded using |\usepackage| or |\RequirePackage|
% which has a mechanism to prevent loading a style file more than once.
% When loading the definitions by means of |\input|
% multiple instances have to be prevented manually:
%\iffalse
%This code needs to be before the `\ProvidesFile' directive
%which is defined at the beginning of this file.
%Therefore it is also placed there and commented out here.
%</package>
%<*discard>
%\fi
%    \begin{macrocode}
\ifdefined\childdocmain\endinput\fi
%    \end{macrocode}
%\iffalse
%</discard>
%<*package>
%\fi
%
% \macro{\ifchilddoc}
% \macro{\ifchilddocmanual}
% The conditional |\ifchilddoc| tells whether a
% child (true) or main (false) document is being compiled.
% The conditional |\ifchilddocmanual| tells whether
% the |\includeonly| mechanism is used (false) or
% the selection of child files must be performed manually (true).
% The definitions initialise to false:
%    \begin{macrocode}
\newif\ifchilddoc
\newif\ifchilddocmanual
%    \end{macrocode}

% \macro{\childdocname}
% \macro{\childdocjob}
% The macro |\childdocname| stores the name of the main document
% to be compiled. The macro |\childdocjob| stores the name of
% the document on which the \LaTeX{} compiler was originally invoked.
% The content of |\jobname| cannot be compared
% to filenames specified in the source due to different catcodes.
% The following code rescans |\jobname|, stores the result
% in |\childdocname| and saves a copy in |\childdocjob|:
%    \begin{macrocode}
\edef\childdocname{\scantokens\expandafter{\jobname\noexpand}}
\let\childdocjob\childdocname
%    \end{macrocode}

% \macro{\childdocdisable}
% The macro |\childdocdisable| prevents the main file
% from being processed more than once.
% At this stage, the main document command |\childdocmain|
% is assumed to be called once again where it should do nothing.
% Any subsequent call to it should prevent
% a secondary processing of the main document
% It overwrites the forwarding commands
% |\childdocof| and |\childdocforward|
% with empty macros to prevent further inclusions of the main document:
%    \begin{macrocode}
\newcommand{\childdocdisable}
{
  \renewcommand{\childdocmain}[1]{\renewcommand{\childdocmain}[1]{\endinput}}
  \renewcommand{\childdocof}[1]{}
  \renewcommand{\childdocby}[2][]{}
  \renewcommand{\childdocforward}[2][]{}
  \renewcommand{\childdocdisable}{}
}
%    \end{macrocode}

% \macro{\childdocmain}
% The macro |\childdocmain| is to be called at the top of the main file
% with nothing or the main filename (without extension) as argument.
% First, it breaks loops.
% If the argument is not empty and does not match |\childdocname|
% (which is set by the first inclusion of |childdoc.def|),
% |\ifchilddoc| is set to true, |\includeonly| is applied to the child file
% and |\jobname| is set to the main file
% (for proper handling of |.aux| files):
%    \begin{macrocode}
\newcommand{\childdocmain}[1]
{
  \childdocdisable\childdocmain{}
  \if?#1?\else
    \begingroup
      \def\childdoctmp{#1}
      \ifx\childdoctmp\childdocname
        \def\childdoctmp{}
      \else
        \def\childdoctmp
        {
          \childdoctrue
          \includeonly{\childdocname}
          \def\childdocjob{#1}
          \def\jobname{#1}
        }
      \fi
      \expandafter
    \endgroup
    \childdoctmp
  \fi
}
%    \end{macrocode}

% \macro{\childdocof}
% The command |\childdocof| redirects
% compilation to the main file |#1|.
%    \begin{macrocode}
\newcommand{\childdocof}[1]
{
  \childdocdisable
  \childdoctrue
  \includeonly{\childdocname}
  \def\jobname{#1}
  \def\childdocjob{#1}
  \input{#1}
}
%    \end{macrocode}

% \macro{\childdocby}
% The command |\childdocby| ....
%    \begin{macrocode}
\newcommand{\childdocby}[2][]
{
  \childdocdisable
  \childdoctrue
  \childdocmanualtrue
  \if?#1?\else
    \def\jobname{#2}
  \fi
  \def\childdocjob{#2}
  \input{#2}
  \endinput
}
%    \end{macrocode}

% \macro{\childdocforward}
% The command |\childdocforward| redirects
% compilation to the main file or
% (if the optional argument is given) a child file.
% Parameters are set as if the main file
% or a child file starting with |\childdocof| was compiled.
% Then compilation is handed over to the main file:
%    \begin{macrocode}
\newcommand{\childdocforward}[2][]
{
  \begingroup
    \if?#1?
      \def\childdoctmp
      {
        \def\childdocname{#2}
        \def\childdocjob{#2}
        \def\jobname{#2}
        \input{#2}
        \endinput
      }
    \else
      \def\childdoctmp
      {
        \childdocdisable
        \def\childdocname{#2}
        \childdoctrue
        \includeonly{#2}
        \def\childdocjob{#1}
        \def\jobname{#1}
        \input{#1}
        \endinput
      }
    \fi
    \expandafter
  \endgroup
  \childdoctmp
}
%    \end{macrocode}

% \macro{\childdocforwardprefix}
% The command |\childdocforwardprefix| redirects
% compilation to the main or a child file by means of a pattern.
% The prefix |#1| in the current filename is replaced by |#2|
% and the suffix of the current filename is kept
% (it is assumed that the filename does not contain the substring `|~~~|'
% which is used as a delimiter).
% Compilation is handed over to the new file by |\childdocforward|:
%    \begin{macrocode}
\newcommand{\childdocforwardprefix}[3][]
{
  \begingroup
    \def\childdocextract #2##1~~~{\def\childdoctmp{\childdocforward[#1]{#3##1}}}
    \expandafter\childdocextract\childdocname~~~
    \expandafter
  \endgroup
  \childdoctmp
}
%    \end{macrocode}

% \macro{\childdoc}
% The deprecated macro |\childdoc| is a legacy version of |\childdocmain|:
%    \begin{macrocode}
\newcommand{\childdoc}{\childdocmain}
%    \end{macrocode}

% \macro{\childdocredirect}
% The deprecated macro |\childdocredirect| is a legacy version
% of |\childdocforward| and |\childdocforwardprefix|:
%    \begin{macrocode}
\newcommand{\childdocredirect}[2][]
{
  \begingroup
    \if?#1?
      \def\childdoctmp{\childdocforward{#2}}
    \else
      \def\childdoctmp{\childdocforwardprefix{#1}{#2}}
    \fi
    \expandafter
  \endgroup
  \childdoctmp
}
%    \end{macrocode}

%\iffalse
%</package>
%\fi
%
\endinput

\childdocby{cdocsamp}
%    \end{macrocode}

%\iffalse
%</samplepart3|samplepart4>
%\fi
%
%\iffalse
%<*samplepart3>
%\fi
% Some text for part 3:
%    \begin{macrocode}
some text in part three
%    \end{macrocode}

%\iffalse
%</samplepart3>
%\fi
% Some text for part 4:
%\iffalse
%<*samplepart4>
%\fi
%    \begin{macrocode}
more text in part four
%    \end{macrocode}

%\iffalse
%</samplepart4>
%\fi
%
% %%%%%%%%%%%%%%%%%%%%%%%%%%%%%%%%%%%%%%
% \paragraph{Forwarding for a Complete Draft.}
%
% The following forwarding file |cdocsdrf.tex|
% compiles the main document in draft mode:
%\iffalse
%<*sampledraft>
%\fi
%    \begin{macrocode}
\def\version{draft}
% \iffalse
%
% childdoc.dtx Copyright (C) 2017-2018 Niklas Beisert
%
% This work may be distributed and/or modified under the
% conditions of the LaTeX Project Public License, either version 1.3
% of this license or (at your option) any later version.
% The latest version of this license is in
%   http://www.latex-project.org/lppl.txt
% and version 1.3 or later is part of all distributions of LaTeX
% version 2005/12/01 or later.
%
% This work has the LPPL maintenance status `maintained'.
%
% The Current Maintainer of this work is Niklas Beisert.
%
% This work consists of the files childdoc.dtx and childdoc.ins
% and the derived files childdoc.def and cdocsamp.tex with
% cdocsch1.tex, cdocsch2.tex, cdocsdrf.tex, cdocsfn1.tex, cdocsfn2.tex.
%
%<package>\ifdefined\childdocmain\endinput\fi
%<package>\ProvidesFile{childdoc.def}[2018/12/30 v2.0 child document driver]
%<samplemain>\ProvidesFile{cdocsamp.tex}[2018/12/30 v2.0 sample for childdoc]
%<*driver>
%\ProvidesFile{childdoc.drv}[2018/12/30 v2.0 childdoc reference manual file]
\PassOptionsToClass{10pt,a4paper}{article}
\documentclass{ltxdoc}

\usepackage[margin=35mm]{geometry}
\usepackage{hyperref}
\usepackage{hyperxmp}
\usepackage[usenames]{color}

\hypersetup{colorlinks=true}
\hypersetup{pdfstartview=FitH}
\hypersetup{pdfpagemode=UseNone}
\hypersetup{pdfsource={}}
\hypersetup{pdflang={en-UK}}
\hypersetup{pdfcopyright={Copyright 2017-2018 Niklas Beisert.
  This work may be distributed and/or modified under the
  conditions of the LaTeX Project Public License, either version 1.3
  of this license or (at your option) any later version.}}
\hypersetup{pdflicenseurl={http://www.latex-project.org/lppl.txt}}
\hypersetup{pdfcontactaddress={ETH Zurich, ITP, HIT K,
  Wolfgang-Pauli-Strasse 27}}
\hypersetup{pdfcontactpostcode={8093}}
\hypersetup{pdfcontactcity={Zurich}}
\hypersetup{pdfcontactcountry={Switzerland}}
\hypersetup{pdfcontactemail={nbeisert@itp.phys.ethz.ch}}
\hypersetup{pdfcontacturl={http://people.phys.ethz.ch/\xmptilde nbeisert/}}

\newcommand{\secref}[1]{\hyperref[#1]{section \ref*{#1}}}

\parskip1ex
\parindent0pt
\let\olditemize\itemize
\def\itemize{\olditemize\parskip0pt}

\begin{document}

\title{The \textsf{childdoc} Package}
\hypersetup{pdftitle={The childdoc Package}}
\author{Niklas Beisert\\[2ex]
  Institut f\"ur Theoretische Physik\\
  Eidgen\"ossische Technische Hochschule Z\"urich\\
  Wolfgang-Pauli-Strasse 27, 8093 Z\"urich, Switzerland\\[1ex]
  \href{mailto:nbeisert@itp.phys.ethz.ch}
  {\texttt{nbeisert@itp.phys.ethz.ch}}}
\hypersetup{pdfauthor={Niklas Beisert}}
\hypersetup{pdfsubject={Manual for the LaTeX2e Package childdoc}}
\date{30 December 2018, \textsf{v2.0}}
\maketitle

\begin{abstract}\noindent
\textsf{childdoc} is a \LaTeXe{} package
that enables the direct compilation
of document sections included by |\include|
to individual files.
\end{abstract}

\begingroup
\parskip0ex
\tableofcontents
\endgroup

%%%%%%%%%%%%%%%%%%%%%%%%%%%%%%%%%%%%%%%%%%%%%%%%%%%%%%%%%%%%%%%%%%%%%%%%%%%%%%%%
%%%%%%%%%%%%%%%%%%%%%%%%%%%%%%%%%%%%%%%%%%%%%%%%%%%%%%%%%%%%%%%%%%%%%%%%%%%%%%%%
\section{Introduction}

\LaTeX{} provides a mechanism to structure a large document (such as a book)
into a main file and several child files (containing the chapters)
using the |\include| command.
This mechanism is beneficial for documents
which span hundreds of pages in order to
make the source file(s) more manageable.
Moreover, compilation can be restricted to
selected child files by means of the |\includeonly| command.
The latter feature can be used to reduce the compilation time while editing
(this was significantly more useful in the earlier days of \LaTeX{})
or to generate a smaller document which is easier to navigate.
Another application of |\includeonly| is to generate
documents consisting of selected parts of the complete document.

However, there are a few drawbacks of the plain |\include| mechanism:
\begin{itemize}
\item
The child files cannot be compiled on their own,
they can only be compiled via the main file.
A naive editing environment
(such as a text editor with an option
to have the current file processed by \LaTeX)
may require one to switch to the main file before compiling;
attempting to compile the child file produces errors.
\item
The main file must be modified (each time)
to adjust the |\includeonly| command
to the present needs. This easily leaves the main file in a messy state.
\item
The generated document will always carry the filename
of the main document. This is inconvenient if
several child files are to be compiled and
to be kept for distribution.
\end{itemize}

The present package provides a simple interface
to make child files individually compilable by \LaTeX{}.
Compiling a child file then has the same effect as compiling
the main file with an |\includeonly| command
to select the appropriate child.
Moreover the generated document will carry the name of the child
rather than the main file.
This resolves all three above issues.

This feature is meant to make the editing of books,
thesis documents and lecture notes somewhat more convenient.
However, the package can also be used efficiently for
composing a series of documents (such as exercise sheets)
which are typically distributed individually.
It then assists the author in generating the individual documents
(potentially in different versions)
as well as a document containing the collected series.
Another application is in developing style files
or other kinds of included material
where compilation of the style file could redirect
to a sample or test file.

%%%%%%%%%%%%%%%%%%%%%%%%%%%%%%%%%%%%%%%%%%%%%%%%%%%%%%%%%%%%%%%%%%%%%%%%%%%%%%%%
%%%%%%%%%%%%%%%%%%%%%%%%%%%%%%%%%%%%%%%%%%%%%%%%%%%%%%%%%%%%%%%%%%%%%%%%%%%%%%%%
\section{Usage}

First of all, the package \textsf{childdoc} is \emph{not} a standard
\LaTeXe{} |.sty| style file! Therefore it needs to be invoked in
a non-standard way.

%%%%%%%%%%%%%%%%%%%%%%%%%%%%%%%%%%%%%%%%%%%%%%%%%%%%%%%%%%%%%%%%%%%%%%%%%%%%%%%%
\subsection{Included Files}
\label{sec:include}

%%%%%%%%%%%%%%%%%%%%%%%%%%%%%%%%%%%%%%%%
\DescribeMacro{\childdocmain}
To use the package, add the commands
\begin{center}
\begin{tabular}{l}
|\input{childdoc.def}|\\
|\childdocmain{}|\\
\end{tabular}
\end{center}
at the very top of the main \LaTeX{} file,
in particular \emph{before} the |\documentclass| statement!
The argument of |\childdocmain| should be left empty
(but it must be present).

%%%%%%%%%%%%%%%%%%%%%%%%%%%%%%%%%%%%%%%%
\DescribeMacro{\childdocof}
Furthermore, add the commands
\begin{center}
\begin{tabular}{l}
|\input{childdoc.def}|\\
|\childdocof{|\textit{main}|}|\\
\end{tabular}
\end{center}
at the top of every child file \textit{child}
which is included by |\include{|\textit{child}|}|
from within the main file
(or at least for those files to be compiled individually).
The argument \textit{main} must be the filename of the main file.

There are a couple of
considerations in setting up the main and child documents:

%%%%%%%%%%%%%%%%%%%%%%%%%%%%%%%%%%%%%%%%
\paragraph{Restrictions.}

Please note the following restrictions:
\begin{itemize}
\item
|\childdocmain| must be called with one argument \textit{main}
to ensure compatibility with earlier version of the package.
It must either be empty (|\childdocmain{}|)
or precisely match the filename of the main file in which it is specified.
See \secref{sec:detection} for further information.
\item
The filename \textit{main} must be specified without the |.tex| extension.
\item
The filename \textit{main} is case sensitive
(even in case-insensitive file systems)
due to internal string comparison.
\item
The argument \textit{main} should be fully expanded, it cannot be a macro.
\item
Subdirectories and special characters should be avoided in filenames.
\item
The command |\childdocmain{|\textit{main}|}| must be followed by a whitespace.
It should not be followed immediately by another command
or by a comment mark `|%|'.
This is because the \TeX{} parser reads the token immediately following
the argument of |\childdocmain| and puts it
at the beginning of every child section;
however, a white\-space is ignored.
\end{itemize}

%%%%%%%%%%%%%%%%%%%%%%%%%%%%%%%%%%%%%%%%
\paragraph{Content of Main File.}

It is advisable to place all content in the child files included by |\include|.
Any output contained in the main file will appear in all child documents
unless suppressed manually;
it cannot be suppressed automatically by the |\includeonly| directive
and thus should normally be avoided.
A method to include some content in the main file
by means of conditional processing is described in \secref{sec:conditional}.

%%%%%%%%%%%%%%%%%%%%%%%%%%%%%%%%%%%%%%%%
\paragraph{Page Numbering.}

When only a part of the document is compiled,
the appropriate numbering of pages
(as well as other status parameters)
is determined from the |.aux| files.
The latter contain information from previous passes.
However this information needs to propagate through
all intermediate child documents.
Therefore the page numbering in child documents may well
be inconsistent until the complete document is compiled at least once.

A useful (if unconventional) way to always ensure a consistent
page numbering is to restart the numbering in each child document
and denote the pages by `\textit{child}|.|\textit{page}'
where \textit{child} represents the chapter/section number of the child file.
This can be achieved by the command
|\numberwithin{page}{|\textit{child}|}|
of the \textsf{amsmath} package
where \textit{child} can be |chapter| or |section|
depending on the chosen structuring.
Alternatively, one can modify the macro |\thepage| appropriately
and reset the counter |page| at the start of each child file.

%%%%%%%%%%%%%%%%%%%%%%%%%%%%%%%%%%%%%%%%%%%%%%%%%%%%%%%%%%%%%%%%%%%%%%%%%%%%%%%%
\subsection{Conditional Processing}
\label{sec:conditional}

The package provides a mechanism to compile different versions
of a document. To customise the versions further some conditional processing
can come in handy to distinguish which version is being compiled.
The package provides two macros to describe the compilation context:

%%%%%%%%%%%%%%%%%%%%%%%%%%%%%%%%%%%%%%%%
\DescribeMacro{\ifchilddoc}
The conditional |\ifchilddoc| distinguishes between the compilation of
child documents and the main document:
%
\begin{center}
|\ifchilddoc |\textit{child-code}| |[|\||else |\textit{main-code}]| \||fi|
\end{center}

%%%%%%%%%%%%%%%%%%%%%%%%%%%%%%%%%%%%%%%%
\DescribeMacro{\childdocname}
\DescribeMacro{\childdocjob}
The macro |\childdocname| contains the filename (without extension)
of the main or child file being processed.
Note that |\childdocjob| will always contain the name of the main file.

%%%%%%%%%%%%%%%%%%%%%%%%%%%%%%%%%%%%%%%%
\paragraph{Title Page.}

Conditional processing can be used to include a title or banner page
in the main document when proper precautions are taken.
Importantly, the code in the main file should ensure that the page counter
(as well as other status parameters which are stored in the |.aux| files)
takes the same value after the conditional processing.
Otherwise the page numbers may take divergent values
depending on which part is compiled.

For example, a title page could be declared by:
%
\begin{center}
\begin{tabular}{l}
|\ifchilddoc\||else|\\
|\addtocounter{page}{-1}|\\
\textit{code for title page}\\
|\newpage|\\
|\||fi|
\end{tabular}
\end{center}
%
A banner page for the child documents can be generated by:
%
\begin{center}
\begin{tabular}{l}
|\ifchilddoc|\\
|\addtocounter{page}{-1}|\\
\textit{code for banner page}\\
|\newpage|\\
|\||fi|
\end{tabular}
\end{center}
%
Here one could write a message such as:
\begin{center}
|This is the part \childdocname{} of \childdocjob{}.|
\end{center}

%%%%%%%%%%%%%%%%%%%%%%%%%%%%%%%%%%%%%%%%%%%%%%%%%%%%%%%%%%%%%%%%%%%%%%%%%%%%%%%%
\subsection{Flags}
\label{sec:flags}

The package makes it easy to generate different versions
of the main or child documents.
To this end compilation flags can be defined
and assigned different default values.
They will be particularly useful in conjunction
with the forwarding mechanism described in \secref{sec:forward}.

For example, it may be useful to have a flag |\version|
which can be set to |draft| or |final|.
The document source will contain some conditional code
depending on the value of |\version|.
Suppose further, the flag should default to |final| for the main file
and to |draft| for child files
which is a natural assignment for editing the document.
This is achieved by placing the following code
in the preamble of the main document
(below the |\childdocmain| directive):
%
\begin{center}
\begin{tabular}{l}
|\ifchilddoc|\\
|\providecommand{\version}{draft}|\\
|\||else|\\
|\providecommand{\version}{final}|\\
|\||fi|
\end{tabular}
\end{center}
%
The definition by |\providecommand| makes sure
that previous definitions are not overwritten.
Further statements |\providecommand{\version}{...}|
can thus be added before the above code to override it.

For the main file, one might add a line
(between |\childdocmain| and the above block)
%
\begin{center}
|%\ifchilddoc\||else\providecommand{\version}{draft}\||fi|
\end{center}
%
which can be uncommented to produce a draft version.
Likewise one can add a line to the very top of a child file
(above the |\childdocof{|\textit{main}|}| directive)
%
\begin{center}
|%\providecommand{\version}{final}|
\end{center}
%
which can be uncommented to produce the final version of this child document.

%%%%%%%%%%%%%%%%%%%%%%%%%%%%%%%%%%%%%%%%%%%%%%%%%%%%%%%%%%%%%%%%%%%%%%%%%%%%%%%%
\subsection{Forwarding}
\label{sec:forward}

Different versions of the main or child documents
using compilation flags as described in \secref{sec:flags}
can be (permanently) stored in different files
for convenient compilation, viewing and distribution.
To this end, the package defines a command
to pass on compilation to a different file:

%%%%%%%%%%%%%%%%%%%%%%%%%%%%%%%%%%%%%%%%
\DescribeMacro{\childdocforward}
The command |\childdocforward| redirects processing to
another source file:
%
\begin{center}
\begin{tabular}{l}
|\input{childdoc.def}|\\
|\childdocforward[|\textit{main}|]{|\textit{dest}|}|\\
\end{tabular}
\end{center}
%
The argument \textit{dest} is the destination file
(without extension).
It should be the main file or one of the child files.
Note that further \textsf{childdoc} directives
such as |\childdocof| and |\childdocforward|
in the indicated file will be processed in this form.
The optional argument \textit{main}
passes on directly to the main file \textit{main}
while pretending to compile the child \textit{dest}.
This form behaves as if \textit{dest}
issues |\childdocof{|\textit{main}|}| right away,
and no further \textsf{childdoc} directives will be processed.

%%%%%%%%%%%%%%%%%%%%%%%%%%%%%%%%%%%%%%%%
\DescribeMacro{\...prefix}
In the alternative form |\childdocforwardprefix|,
%
\begin{center}
\begin{tabular}{l}
|\input{childdoc.def}|\\
|\childdocforwardprefix[|\textit{main}|]{|\textit{prefix}|}{|\textit{dest}|}|
\end{tabular}
\end{center}
%
the destination file is determined by a pattern
depending on the current file:
To make this work, the current file must be called
`{\textit{prefix}\hspace{0.2em}\textit{suffix}}'
with \textit{prefix} matching precisely the argument.
Processing is then passed on to the file
`{\textit{dest}\hspace{0.2em}\textit{suffix}}'.
Surely, the same effect is achieved by
directly specifying the
argument `{\textit{dest}\hspace{0.2em}\textit{suffix}}'
in the first form.
However, that requires to set up a different file
for each child. With the alternative form of the command
all these files can have exactly the same content
which simplifies setting them up and maintaining them.

For example, the following file |draft.tex|
with a compilation flag |\version| as described in \secref{sec:flags}
compiles the main document as a draft:
%
\begin{center}
\begin{tabular}{l}
|\def\version{draft}|\\
|\input{childdoc.def}|\\
|\childdocforward{|\textit{main}|}|
\end{tabular}
\end{center}
%
Likewise, the following files |final|\textit{nn}|.tex|
compile the final version of the child document
|child|\textit{nn}|.tex|:
%
\begin{center}
\begin{tabular}{l}
|\def\version{final}|\\
|\input{childdoc.def}|\\
|\childdocforwardprefix{final}{child}|
\end{tabular}
\end{center}
%

Note that when several versions of a main file and/or of each child file
are to be generated, it may be convenient to set up a |Makefile| or
shell script to automatise the process.

%%%%%%%%%%%%%%%%%%%%%%%%%%%%%%%%%%%%%%%%%%%%%%%%%%%%%%%%%%%%%%%%%%%%%%%%%%%%%%%%
\subsection{Command Line Processing}
\label{sec:commandline}

The effect of redirection files can also be achieved by invoking
the \LaTeX{} compiler with a more elaborate command line.
Most conveniently this should be done as part
of a shell script or a |Makefile|.

When using \textsf{childdoc} in the main file, the following
command lines effectively perform a redirection
(note that depending on the shell being used,
backslashes may have to be doubled: `|\|' $\to$ `|\\|'):
%
\begin{center}
|... -jobname "|\textit{target}|" |\\|"|[\textit{flags}]%
|\input{childdoc.def}\childdocforward[|\textit{main}|]{|\textit{dest}|}"|
\end{center}
%
Here \textit{target} is the name of the output file,
\textit{main} is the name of the main file
and \textit{dest} is the name of the main or child file to be processed
(all filenames without extensions).
The optional argument \textit{main} can be omitted
if \textit{main} matches \textit{dest}.
Optionally, compilation \textit{flags} can be defined via |\def| commands.
This command line makes the \TeX{} engine believe
it is compiling the file \textit{target}
whose content is specified as the latter parameter.
The provided code then forwards the processing to
\textit{main} or \textit{dest} as described in \secref{sec:forward}.

%%%%%%%%%%%%%%%%%%%%%%%%%%%%%%%%%%%%%%%%%%%%%%%%%%%%%%%%%%%%%%%%%%%%%%%%%%%%%%%%
\subsection{Include by Input}
\label{sec:input}

Including child documents by |\include| has some restrictions by design.
Most notably, the content of a child document always occupies
its own set of pages; pages cannot be shared between child documents.
Usually, this behaviour makes perfect sense
because each child document contain an essential part of the document.
However, in some situations it may be desirable to compose
a document from a collection of parts
without having mandatory page breaks between then.
For this case, the package
provides a mechanism to include parts
by |\input| which can also be processed individually.
However, by construction this mechanism
requires manual handling of the content to be output.

%%%%%%%%%%%%%%%%%%%%%%%%%%%%%%%%%%%%%%%%
\DescribeMacro{\ifchilddocmanual}
The main file should be prepared as usual, see \secref{sec:include}.
However, the document body must make a distinction
between processing of an individual part and of the main document, e.g.:
%
\begin{center}
\begin{tabular}{l}
|\ifchilddocmanual|\\
|\input{\childdocname}|\\
|\||else|\\
\textit{document body with }|\input{|\textit{part}|}|\\
|\||fi|
\end{tabular}
\end{center}
%
The conditional |\ifchilddocmanual| is true whenever
a part to be included by |\input| is being compiled,
and the name of the part is stored in |\childdocname|.

%%%%%%%%%%%%%%%%%%%%%%%%%%%%%%%%%%%%%%%%
\DescribeMacro{\childdocby}
Each part to be included by |\input| should start with:
%
\begin{center}
\begin{tabular}{l}
|\input{childdoc.def}|\\
|\childdocby{|\textit{main}|}|\\
\end{tabular}
\end{center}
%
The directive |\childdocby| is similar to |\childdocof|
described in \secref{sec:include},
but the subsequent selection of content must be done manually.
To that end, both |\ifchilddoc| and |\ifchilddocmanual|
will be true upon processing of a part,
and the name of the part is stored in |\childdocname|.
Note that |\jobname| will be set to the filename of the current part
so that each part receives an individual |.aux| file
that does not interfere with the |.aux| file(s) of the main document.
This behaviour can be altered by the alternative form
|\childdocby[*]{|\textit{main}|}| (with a non-empty optional argument)
which uses the |.aux| file of the main document
by setting |\jobname| to \textit{main}.

%%%%%%%%%%%%%%%%%%%%%%%%%%%%%%%%%%%%%%%%%%%%%%%%%%%%%%%%%%%%%%%%%%%%%%%%%%%%%%%%
\subsection{Driver Development}
\label{sec:driver}

The \textsf{childdoc} mechanism can also be use for the development
of definition files such as \LaTeX{} styles or classes.
This case differs from the above setup with multiple parts
included by |\include| in that no |\includeonly| should be invoked.
This can be achieved by starting the include file
(before |\ProvidesPackage|) with:
%
\begin{center}
\begin{tabular}{l}
|\input{childdoc.def}|\\
|\childdocforward{|\textit{main}|}|\\
\end{tabular}
\end{center}
%
or alternatively with:
%
\begin{center}
\begin{tabular}{l}
|\input{childdoc.def}|\\
|\childdocby{|\textit{main}|}|\\
\end{tabular}
\end{center}
%
Both forms have slightly different effects as described above.
The main file is prepared as usual, see \secref{sec:include}.

%%%%%%%%%%%%%%%%%%%%%%%%%%%%%%%%%%%%%%%%%%%%%%%%%%%%%%%%%%%%%%%%%%%%%%%%%%%%%%%%
\subsection{Legacy Detection}
\label{sec:detection}

The directive |\childdocmain| in the main file can detect
whether the complete document or merely a child is to be compiled
even without using the directive |\childdocof|.
This method is deprecated because it is less robust
and there is no compelling reason to use it;
it is merely provided for backward compatibility
and it may be removed in future versions.

If the detection mechanism is to be used,
it is mandatory to correctly specify
the filename of the main file as the argument of |\childdocmain|:
%
\begin{center}
\begin{tabular}{l}
|\input{childdoc.def}|\\
|\childdocmain{|\textit{main}|}|\\
\end{tabular}
\end{center}
%
If |\jobname| does not match the argument \textit{main} of |\childdocmain|,
it is assumed that |\jobname| points to the child file to be compiled.
When using |\childdocmain| with the main file specified as argument,
it suffices to start a child file
with just |\input{|\textit{main}|}|
without loading of the package and using |\childdocof|.
If instead all processing is done
with the appropriate \textsf{childdoc} directives,
the argument of \textit{main} of |\childdocmain| can be empty.

An alternative version of the command line processing described
in \secref{sec:commandline} using the detection mechanism reads:
%
\begin{center}
|... -jobname "|\textit{target}|" "|[\textit{flags}]%
[|\def\jobname{|\textit{dest}|}|]|\input{|\textit{main}|}"|
\end{center}

%%%%%%%%%%%%%%%%%%%%%%%%%%%%%%%%%%%%%%%%%%%%%%%%%%%%%%%%%%%%%%%%%%%%%%%%%%%%%%%%
\subsection{Manual Code}
\label{sec:manual}

In case one cannot be certain whether the definitions file |childdoc.def|
is installed on the target \TeX{} distribution
and one prefers not to ship it,
it is conceivable to paste a few relevant commands into the sources.

To that end, drop all statements |\input{childdoc.def}|
and perform the replacements as outlined below.
Instead of |\childdocmain{|\textit{main}|}| add the following code
to the top of the main file:
%
\begin{center}
\begin{tabular}{l}
|\||ifdefined\childdocname\endinput\||fi\newif\ifchilddoc|\\
|\edef\childdocname{\scantokens\expandafter{\jobname\noexpand}}|\\
|\def\childdocmain{|\textit{main}|}\||ifx\childdocmain\childdocname\||else|\\
|\childdoctrue\includeonly{\childdocname}\let\jobname\childdocmain\||fi|\\
\end{tabular}
\end{center}
%
Instead of |\childdocof{|\textit{main}|}| just include the main file
at the top of each child file:
%
\begin{center}
|\input{|\textit{main}|}|
\end{center}
%
A simple redirection |\childdocforward{|\textit{dest}|}| is achieved by:
%
\begin{center}
|\def\jobname{|\textit{dest}|}\input{\jobname}|
\end{center}
%
The redirection with prefix
|\childdocforwardprefix[|\textit{prefix}|]{|\textit{dest}|}|
is accomplished by:
%
\begin{center}
\begin{tabular}{l}
|{\edef\jobname{\scantokens\expandafter{\jobname\noexpand}}|\\
|\def\redirectjob |\textit{prefix}|#1~~~{\gdef\jobname{|\textit{dest}|#1}}|\\
|\expandafter\redirectjob\jobname~~~}\input{\jobname}|
\end{tabular}
\end{center}

In an alternative approach,
child documents can be compiled by a specific command line
without additional code or specific definitions:
%
\begin{center}
|... -jobname "|\textit{target}|" "|[\textit{flags}]%
|\includeonly{|\textit{dest}|}\input{|\textit{main}|}"|
\end{center}
%

%%%%%%%%%%%%%%%%%%%%%%%%%%%%%%%%%%%%%%%%%%%%%%%%%%%%%%%%%%%%%%%%%%%%%%%%%%%%%%%%
%%%%%%%%%%%%%%%%%%%%%%%%%%%%%%%%%%%%%%%%%%%%%%%%%%%%%%%%%%%%%%%%%%%%%%%%%%%%%%%%
\section{Information}

%%%%%%%%%%%%%%%%%%%%%%%%%%%%%%%%%%%%%%%%%%%%%%%%%%%%%%%%%%%%%%%%%%%%%%%%%%%%%%%%
\subsection{Copyright}

Copyright \copyright{} 2017--2018 Niklas Beisert

This work may be distributed and/or modified under the
conditions of the \LaTeX{} Project Public License, either version 1.3
of this license or (at your option) any later version.
The latest version of this license is in
  \url{http://www.latex-project.org/lppl.txt}
and version 1.3 or later is part of all distributions of \LaTeX{}
version 2005/12/01 or later.

This work has the LPPL maintenance status `maintained'.

The Current Maintainer of this work is Niklas Beisert.

This work consists of the files |README.txt|, |childdoc.ins| and |childdoc.dtx|
as well as the derived files |childdoc.def|, |cdocsamp.tex|
with |cdocsch1.tex|, |cdocsch2.tex|, |cdocspt3.tex|, |cdocspt4.tex|,
|cdocsdrf.tex|, |cdocsfn1.tex|, |cdocsfn2.tex|
as well as |childdoc.pdf|.

%%%%%%%%%%%%%%%%%%%%%%%%%%%%%%%%%%%%%%%%%%%%%%%%%%%%%%%%%%%%%%%%%%%%%%%%%%%%%%%%
\subsection{Files and Installation}

The package consists of the files:
%
\begin{center}
\begin{tabular}{ll}
    |README.txt|   & readme file \\
    |childdoc.ins| & installation file \\
    |childdoc.dtx| & source file \\
    |childdoc.def| & definition file \\
    |cdocsamp.tex| & sample main file \\
    |cdocsch1.tex| & sample include file \\
    |cdocsch2.tex| & sample include file \\
    |cdocspt3.tex| & sample part file \\
    |cdocspt4.tex| & sample part file \\
    |cdocsdrf.tex| & sample redirection file \\
    |cdocsfn1.tex| & sample redirection file \\
    |cdocsfn2.tex| & sample redirection file \\
    |childdoc.pdf| & manual
\end{tabular}
\end{center}
%
The distribution consists of the files
|README.txt|, |childdoc.ins| and |childdoc.dtx|.
%
\begin{itemize}
\item
Run (pdf)\LaTeX{} on |childdoc.dtx|
to compile the manual |childdoc.pdf| (this file).
\item
Run \LaTeX{} on |childdoc.ins| to create the definitions file |childdoc.def|
and the sample |cdocsamp.tex| with include files
|cdocsch1.tex|, |cdocsch2.tex|, |cdocspt3.tex|, |cdocspt4.tex|,
|cdocsdrf.tex|, |cdocsfn1.tex|, |cdocsfn2.tex|.
Then copy the file |childdoc.def| to an appropriate directory of your \LaTeX{}
distribution, e.g.\ \textit{texmf-root}|/tex/latex/childdoc|.
\end{itemize}

%%%%%%%%%%%%%%%%%%%%%%%%%%%%%%%%%%%%%%%%%%%%%%%%%%%%%%%%%%%%%%%%%%%%%%%%%%%%%%%%
\subsection{Related CTAN Packages}

There are several other packages which offer a similar functionality:
%
\begin{itemize}
\item
The packages
\href{http://ctan.org/pkg/docmute}{\textsf{docmute}},
\href{http://ctan.org/pkg/includex}{\textsf{includex}} and
\href{http://ctan.org/pkg/standalone}{\textsf{standalone}}
provide commands to include only the document body of
a child file thus allowing both files to be compiled individually.
\item
The packages \href{http://ctan.org/pkg/subdocs}{\textsf{subdocs}}
and \href{http://ctan.org/pkg/subfiles}{\textsf{subfiles}}
provide structures in which the main and child documents can be
encapsulated and allowing them to be compiled individually.
The inclusion mechanism is different from the conventional |\include|.
\item
The package \href{http://ctan.org/pkg/combine}{\textsf{combine}}
is an elaborate solution to combine several documents into one.
\end{itemize}
%
See also the CTAN topic \href{http://ctan.org/topic/subdocs}{\textsf{subdocs}}
for further related packages.
The present package differs from the above solutions in that
a document structure constructed with the conventional |\include| mechanism
just needs two extra commands at the top of every file
such that all constituent files can be compiled individually.

%%%%%%%%%%%%%%%%%%%%%%%%%%%%%%%%%%%%%%%%%%%%%%%%%%%%%%%%%%%%%%%%%%%%%%%%%%%%%%%%
%\subsection{Feature Suggestions}
%
%The following is a list of features which may be useful for future
%versions of this package:
%%
%\begin{itemize}
%\item
%\ldots
%\end{itemize}

%%%%%%%%%%%%%%%%%%%%%%%%%%%%%%%%%%%%%%%%%%%%%%%%%%%%%%%%%%%%%%%%%%%%%%%%%%%%%%%%
\subsection{Revision History}

%%%%%%%%%%%%%%%%%%%%%%%%%%%%%%%%%%%%%%%%
\paragraph{v2.0:} 2018/12/30

\begin{itemize}
\item
immediate forward processing
\item
added |\childdocby| mechanism
\item
manual restructured
\end{itemize}

%%%%%%%%%%%%%%%%%%%%%%%%%%%%%%%%%%%%%%%%
\paragraph{v1.6:} 2018/01/17

\begin{itemize}
\item
application for development of include files
\item
corrections to manual
\end{itemize}

%%%%%%%%%%%%%%%%%%%%%%%%%%%%%%%%%%%%%%%%
\paragraph{v1.5:} 2017/05/21

\begin{itemize}
\item
more complete structuring introduced
\item
|\childdocof| introduced
\item
|\childdoc| renamed to |\childdocmain|
\item
|\childredirect| renamed to |\childdocforward| and |\childdocforwardprefix|
and functionality expanded
\end{itemize}

%%%%%%%%%%%%%%%%%%%%%%%%%%%%%%%%%%%%%%%%
\paragraph{v1.0:} 2017/04/27

\begin{itemize}
\item
manual and install package
\item
first version published on CTAN
\end{itemize}

%%%%%%%%%%%%%%%%%%%%%%%%%%%%%%%%%%%%%%%%
\paragraph{v0.6:} 2017/04/26

\begin{itemize}
\item
redirection mechanism added
\end{itemize}

%%%%%%%%%%%%%%%%%%%%%%%%%%%%%%%%%%%%%%%%
\paragraph{v0.5:} 2017/04/26

\begin{itemize}
\item
functionality in definition file
\end{itemize}


%%%%%%%%%%%%%%%%%%%%%%%%%%%%%%%%%%%%%%%%%%%%%%%%%%%%%%%%%%%%%%%%%%%%%%%%%%%%%%%%
%%%%%%%%%%%%%%%%%%%%%%%%%%%%%%%%%%%%%%%%%%%%%%%%%%%%%%%%%%%%%%%%%%%%%%%%%%%%%%%%
%%%%%%%%%%%%%%%%%%%%%%%%%%%%%%%%%%%%%%%%%%%%%%%%%%%%%%%%%%%%%%%%%%%%%%%%%%%%%%%%
\appendix

\settowidth\MacroIndent{\rmfamily\scriptsize 000\ }

 \DocInput{childdoc.dtx}

\end{document}
%</driver>
% \fi
%
% %%%%%%%%%%%%%%%%%%%%%%%%%%%%%%%%%%%%%%%%%%%%%%%%%%%%%%%%%%%%%%%%%%%%%%%%%%%%%%
% %%%%%%%%%%%%%%%%%%%%%%%%%%%%%%%%%%%%%%%%%%%%%%%%%%%%%%%%%%%%%%%%%%%%%%%%%%%%%%
% \section{Sample}
%\iffalse
%<*samplemain>
%\fi
%
% The following presents a sample document
% with two chapters, two parts, a title page,
% a compile flag as well as three forwarding files to set the flag.
% It consists of eight |.tex| files:
% \begin{center}
% \begin{tabular}{ll}
% |cdocsamp.tex|&main file\\
% |cdocsch1.tex|&include file for chapter 1\\
% |cdocsch2.tex|&include file for chapter 2\\
% |cdocspt3.tex|&include file for part 3\\
% |cdocspt4.tex|&include file for part 4\\
% |cdocsdrf.tex|&forwarding file for main file in draft mode\\
% |cdocsfi1.tex|&forwarding file for final version of chapter 1\\
% |cdocsfi2.tex|&forwarding file for final version of chapter 2\\
% \end{tabular}
% \end{center}
% Each of the eight files can be compiled directly by the \LaTeX{} compiler.
%
% %%%%%%%%%%%%%%%%%%%%%%%%%%%%%%%%%%%%%%
% \paragraph{Main File.}
%
% The main file is called |cdocsamp.tex|.
%
% Load the \textsf{childdoc} definitions and
% declare the filename for the main document:
%    \begin{macrocode}
\input{childdoc.def}
\childdocmain{}
%    \end{macrocode}

% Optional override for |\version| flag:
%    \begin{macrocode}
%%\ifchilddoc\else\providecommand{\version}{draft}\fi
%    \end{macrocode}

% Define the default values for the |\version| flag
% (|final| for the main file and |draft| for childs):
%    \begin{macrocode}
\ifchilddoc
\providecommand{\version}{draft}
\else
\providecommand{\version}{final}
\fi
%    \end{macrocode}

% Load the standard document class:
%    \begin{macrocode}
\documentclass[12pt]{article}
%    \end{macrocode}

% Start the document body:
%    \begin{macrocode}
\begin{document}
%    \end{macrocode}

% Declare a title page.
% Print title, part of document being processed and version flag:
%    \begin{macrocode}
\addtocounter{page}{-1}
\begin{center}
{\LARGE\bfseries{}childdoc example\par}
\vspace{1cm}
\ifchilddoc
\ifchilddocmanual part\else chapter\fi:
`\childdocname' of `\childdocjob'\par
\else
main document: `\childdocjob'\par
\fi
version: \version\par
\end{center}
\newpage
%    \end{macrocode}

% Manually include selected file,
% otherwise process as usual:
%    \begin{macrocode}
\ifchilddocmanual
\section*{part `\childdocname'}
\input{\childdocname}
\else
%    \end{macrocode}

% Include the two chapters:
%    \begin{macrocode}
\include{cdocsch1}
\include{cdocsch2}
%    \end{macrocode}

% Include the two parts unless only chapters should be displayed:
%    \begin{macrocode}
\ifchilddoc\else
\section{part three}
\input{cdocspt3}
\section{part four}
\input{cdocspt4}
\fi
%    \end{macrocode}

% Process as usual until here:
%    \begin{macrocode}
\fi
%    \end{macrocode}

% End of document body:
%    \begin{macrocode}
\end{document}
%    \end{macrocode}
%\iffalse
%</samplemain>
%\fi
%
% %%%%%%%%%%%%%%%%%%%%%%%%%%%%%%%%%%%%%%
% \paragraph{Chapter Include Files.}
%
% The include files are called |cdocsch1.tex| and |cdocsch2.tex|.
%
%\iffalse
%<*samplechap1|samplechap2>
%\fi

% Optional override for |\version| flag:
%    \begin{macrocode}
%%\providecommand{\version}{final}
%    \end{macrocode}

% Include the main document:
%    \begin{macrocode}
\input{childdoc.def}
\childdocof{cdocsamp}
%    \end{macrocode}

%\iffalse
%</samplechap1|samplechap2>
%\fi
%
%\iffalse
%<*samplechap1>
%\fi
% Some text for chapter 1:
%    \begin{macrocode}
\section{one}
some text in chapter one
%    \end{macrocode}

%\iffalse
%</samplechap1>
%\fi
% Some text for chapter 2:
%\iffalse
%<*samplechap2>
%\fi
%    \begin{macrocode}
\section{two}
more text in chapter two
%    \end{macrocode}

%\iffalse
%</samplechap2>
%\fi
%
% %%%%%%%%%%%%%%%%%%%%%%%%%%%%%%%%%%%%%%
% \paragraph{Part Include Files.}
%
% The include files are called |cdocspt3.tex| and |cdocspt4.tex|.
%
%\iffalse
%<*samplepart3|samplepart4>
%\fi

% Optional override for |\version| flag:
%    \begin{macrocode}
%%\providecommand{\version}{final}
%    \end{macrocode}

% Include the main document:
%    \begin{macrocode}
\input{childdoc.def}
\childdocby{cdocsamp}
%    \end{macrocode}

%\iffalse
%</samplepart3|samplepart4>
%\fi
%
%\iffalse
%<*samplepart3>
%\fi
% Some text for part 3:
%    \begin{macrocode}
some text in part three
%    \end{macrocode}

%\iffalse
%</samplepart3>
%\fi
% Some text for part 4:
%\iffalse
%<*samplepart4>
%\fi
%    \begin{macrocode}
more text in part four
%    \end{macrocode}

%\iffalse
%</samplepart4>
%\fi
%
% %%%%%%%%%%%%%%%%%%%%%%%%%%%%%%%%%%%%%%
% \paragraph{Forwarding for a Complete Draft.}
%
% The following forwarding file |cdocsdrf.tex|
% compiles the main document in draft mode:
%\iffalse
%<*sampledraft>
%\fi
%    \begin{macrocode}
\def\version{draft}
\input{childdoc.def}
\childdocforward{cdocsamp}
%    \end{macrocode}

%\iffalse
%</sampledraft>
%\fi
%
% %%%%%%%%%%%%%%%%%%%%%%%%%%%%%%%%%%%%%%
% \paragraph{Forwarding for Final Version of the Chapters.}
%
% The following forwarding files |cdocsfn1.tex| and |cdocsfn2.tex|
% (with identical content)
% compile the final versions of the child documents
% |cdocsch1.tex| and |cdocsch2.tex|, respectively:
%\iffalse
%<*samplefinal>
%\fi
%    \begin{macrocode}
\def\version{final}
\input{childdoc.def}
\childdocforwardprefix[cdocsamp]{cdocsfn}{cdocsch}
%    \end{macrocode}

%\iffalse
%</samplefinal>
%\fi
%
% %%%%%%%%%%%%%%%%%%%%%%%%%%%%%%%%%%%%%%
% \paragraph{Command Line Processing.}
%
% The following three command lines generate the output files
% |cdocscld|, |cdocscl1| and |cdocscl2|
% which should be identical to
% |cdocsdrf|, |cdocsch1| and |cdocsfn2|, respectively:
% \begin{center}
% \begin{tabular}{l}
% |latex -jobname cdocscld \|\\
% |  "\def\version{draft}\input{childdoc.def}\childdocforward{cdocsamp}"|\\
% |latex -jobname cdocscl1 \|\\
% |  "\input{childdoc.def}\childdocforward[cdocsamp]{cdocsch1}"|\\
% |latex -jobname cdocscl2 \|\\
% |  "\def\version{final}\input{childdoc.def}\childdocforward{cdocsch2}"|
% \end{tabular}
% \end{center}
% Note that the trailing backslash on each first line
% merely continues the input to the second line
% (for convenient cut ant paste).
% Furthermore, the command |latex| can be replaced by any
% of its alternative versions such as |pdflatex|.
%
% %%%%%%%%%%%%%%%%%%%%%%%%%%%%%%%%%%%%%%%%%%%%%%%%%%%%%%%%%%%%%%%%%%%%%%%%%%%%%%
% %%%%%%%%%%%%%%%%%%%%%%%%%%%%%%%%%%%%%%%%%%%%%%%%%%%%%%%%%%%%%%%%%%%%%%%%%%%%%%
% \section{Implementation}
%\iffalse
%<*package>
%\fi
%
% This section describes the definitions file |childdoc.def|.

% The definitions cannot be loaded using |\usepackage| or |\RequirePackage|
% which has a mechanism to prevent loading a style file more than once.
% When loading the definitions by means of |\input|
% multiple instances have to be prevented manually:
%\iffalse
%This code needs to be before the `\ProvidesFile' directive
%which is defined at the beginning of this file.
%Therefore it is also placed there and commented out here.
%</package>
%<*discard>
%\fi
%    \begin{macrocode}
\ifdefined\childdocmain\endinput\fi
%    \end{macrocode}
%\iffalse
%</discard>
%<*package>
%\fi
%
% \macro{\ifchilddoc}
% \macro{\ifchilddocmanual}
% The conditional |\ifchilddoc| tells whether a
% child (true) or main (false) document is being compiled.
% The conditional |\ifchilddocmanual| tells whether
% the |\includeonly| mechanism is used (false) or
% the selection of child files must be performed manually (true).
% The definitions initialise to false:
%    \begin{macrocode}
\newif\ifchilddoc
\newif\ifchilddocmanual
%    \end{macrocode}

% \macro{\childdocname}
% \macro{\childdocjob}
% The macro |\childdocname| stores the name of the main document
% to be compiled. The macro |\childdocjob| stores the name of
% the document on which the \LaTeX{} compiler was originally invoked.
% The content of |\jobname| cannot be compared
% to filenames specified in the source due to different catcodes.
% The following code rescans |\jobname|, stores the result
% in |\childdocname| and saves a copy in |\childdocjob|:
%    \begin{macrocode}
\edef\childdocname{\scantokens\expandafter{\jobname\noexpand}}
\let\childdocjob\childdocname
%    \end{macrocode}

% \macro{\childdocdisable}
% The macro |\childdocdisable| prevents the main file
% from being processed more than once.
% At this stage, the main document command |\childdocmain|
% is assumed to be called once again where it should do nothing.
% Any subsequent call to it should prevent
% a secondary processing of the main document
% It overwrites the forwarding commands
% |\childdocof| and |\childdocforward|
% with empty macros to prevent further inclusions of the main document:
%    \begin{macrocode}
\newcommand{\childdocdisable}
{
  \renewcommand{\childdocmain}[1]{\renewcommand{\childdocmain}[1]{\endinput}}
  \renewcommand{\childdocof}[1]{}
  \renewcommand{\childdocby}[2][]{}
  \renewcommand{\childdocforward}[2][]{}
  \renewcommand{\childdocdisable}{}
}
%    \end{macrocode}

% \macro{\childdocmain}
% The macro |\childdocmain| is to be called at the top of the main file
% with nothing or the main filename (without extension) as argument.
% First, it breaks loops.
% If the argument is not empty and does not match |\childdocname|
% (which is set by the first inclusion of |childdoc.def|),
% |\ifchilddoc| is set to true, |\includeonly| is applied to the child file
% and |\jobname| is set to the main file
% (for proper handling of |.aux| files):
%    \begin{macrocode}
\newcommand{\childdocmain}[1]
{
  \childdocdisable\childdocmain{}
  \if?#1?\else
    \begingroup
      \def\childdoctmp{#1}
      \ifx\childdoctmp\childdocname
        \def\childdoctmp{}
      \else
        \def\childdoctmp
        {
          \childdoctrue
          \includeonly{\childdocname}
          \def\childdocjob{#1}
          \def\jobname{#1}
        }
      \fi
      \expandafter
    \endgroup
    \childdoctmp
  \fi
}
%    \end{macrocode}

% \macro{\childdocof}
% The command |\childdocof| redirects
% compilation to the main file |#1|.
%    \begin{macrocode}
\newcommand{\childdocof}[1]
{
  \childdocdisable
  \childdoctrue
  \includeonly{\childdocname}
  \def\jobname{#1}
  \def\childdocjob{#1}
  \input{#1}
}
%    \end{macrocode}

% \macro{\childdocby}
% The command |\childdocby| ....
%    \begin{macrocode}
\newcommand{\childdocby}[2][]
{
  \childdocdisable
  \childdoctrue
  \childdocmanualtrue
  \if?#1?\else
    \def\jobname{#2}
  \fi
  \def\childdocjob{#2}
  \input{#2}
  \endinput
}
%    \end{macrocode}

% \macro{\childdocforward}
% The command |\childdocforward| redirects
% compilation to the main file or
% (if the optional argument is given) a child file.
% Parameters are set as if the main file
% or a child file starting with |\childdocof| was compiled.
% Then compilation is handed over to the main file:
%    \begin{macrocode}
\newcommand{\childdocforward}[2][]
{
  \begingroup
    \if?#1?
      \def\childdoctmp
      {
        \def\childdocname{#2}
        \def\childdocjob{#2}
        \def\jobname{#2}
        \input{#2}
        \endinput
      }
    \else
      \def\childdoctmp
      {
        \childdocdisable
        \def\childdocname{#2}
        \childdoctrue
        \includeonly{#2}
        \def\childdocjob{#1}
        \def\jobname{#1}
        \input{#1}
        \endinput
      }
    \fi
    \expandafter
  \endgroup
  \childdoctmp
}
%    \end{macrocode}

% \macro{\childdocforwardprefix}
% The command |\childdocforwardprefix| redirects
% compilation to the main or a child file by means of a pattern.
% The prefix |#1| in the current filename is replaced by |#2|
% and the suffix of the current filename is kept
% (it is assumed that the filename does not contain the substring `|~~~|'
% which is used as a delimiter).
% Compilation is handed over to the new file by |\childdocforward|:
%    \begin{macrocode}
\newcommand{\childdocforwardprefix}[3][]
{
  \begingroup
    \def\childdocextract #2##1~~~{\def\childdoctmp{\childdocforward[#1]{#3##1}}}
    \expandafter\childdocextract\childdocname~~~
    \expandafter
  \endgroup
  \childdoctmp
}
%    \end{macrocode}

% \macro{\childdoc}
% The deprecated macro |\childdoc| is a legacy version of |\childdocmain|:
%    \begin{macrocode}
\newcommand{\childdoc}{\childdocmain}
%    \end{macrocode}

% \macro{\childdocredirect}
% The deprecated macro |\childdocredirect| is a legacy version
% of |\childdocforward| and |\childdocforwardprefix|:
%    \begin{macrocode}
\newcommand{\childdocredirect}[2][]
{
  \begingroup
    \if?#1?
      \def\childdoctmp{\childdocforward{#2}}
    \else
      \def\childdoctmp{\childdocforwardprefix{#1}{#2}}
    \fi
    \expandafter
  \endgroup
  \childdoctmp
}
%    \end{macrocode}

%\iffalse
%</package>
%\fi
%
\endinput

\childdocforward{cdocsamp}
%    \end{macrocode}

%\iffalse
%</sampledraft>
%\fi
%
% %%%%%%%%%%%%%%%%%%%%%%%%%%%%%%%%%%%%%%
% \paragraph{Forwarding for Final Version of the Chapters.}
%
% The following forwarding files |cdocsfn1.tex| and |cdocsfn2.tex|
% (with identical content)
% compile the final versions of the child documents
% |cdocsch1.tex| and |cdocsch2.tex|, respectively:
%\iffalse
%<*samplefinal>
%\fi
%    \begin{macrocode}
\def\version{final}
% \iffalse
%
% childdoc.dtx Copyright (C) 2017-2018 Niklas Beisert
%
% This work may be distributed and/or modified under the
% conditions of the LaTeX Project Public License, either version 1.3
% of this license or (at your option) any later version.
% The latest version of this license is in
%   http://www.latex-project.org/lppl.txt
% and version 1.3 or later is part of all distributions of LaTeX
% version 2005/12/01 or later.
%
% This work has the LPPL maintenance status `maintained'.
%
% The Current Maintainer of this work is Niklas Beisert.
%
% This work consists of the files childdoc.dtx and childdoc.ins
% and the derived files childdoc.def and cdocsamp.tex with
% cdocsch1.tex, cdocsch2.tex, cdocsdrf.tex, cdocsfn1.tex, cdocsfn2.tex.
%
%<package>\ifdefined\childdocmain\endinput\fi
%<package>\ProvidesFile{childdoc.def}[2018/12/30 v2.0 child document driver]
%<samplemain>\ProvidesFile{cdocsamp.tex}[2018/12/30 v2.0 sample for childdoc]
%<*driver>
%\ProvidesFile{childdoc.drv}[2018/12/30 v2.0 childdoc reference manual file]
\PassOptionsToClass{10pt,a4paper}{article}
\documentclass{ltxdoc}

\usepackage[margin=35mm]{geometry}
\usepackage{hyperref}
\usepackage{hyperxmp}
\usepackage[usenames]{color}

\hypersetup{colorlinks=true}
\hypersetup{pdfstartview=FitH}
\hypersetup{pdfpagemode=UseNone}
\hypersetup{pdfsource={}}
\hypersetup{pdflang={en-UK}}
\hypersetup{pdfcopyright={Copyright 2017-2018 Niklas Beisert.
  This work may be distributed and/or modified under the
  conditions of the LaTeX Project Public License, either version 1.3
  of this license or (at your option) any later version.}}
\hypersetup{pdflicenseurl={http://www.latex-project.org/lppl.txt}}
\hypersetup{pdfcontactaddress={ETH Zurich, ITP, HIT K,
  Wolfgang-Pauli-Strasse 27}}
\hypersetup{pdfcontactpostcode={8093}}
\hypersetup{pdfcontactcity={Zurich}}
\hypersetup{pdfcontactcountry={Switzerland}}
\hypersetup{pdfcontactemail={nbeisert@itp.phys.ethz.ch}}
\hypersetup{pdfcontacturl={http://people.phys.ethz.ch/\xmptilde nbeisert/}}

\newcommand{\secref}[1]{\hyperref[#1]{section \ref*{#1}}}

\parskip1ex
\parindent0pt
\let\olditemize\itemize
\def\itemize{\olditemize\parskip0pt}

\begin{document}

\title{The \textsf{childdoc} Package}
\hypersetup{pdftitle={The childdoc Package}}
\author{Niklas Beisert\\[2ex]
  Institut f\"ur Theoretische Physik\\
  Eidgen\"ossische Technische Hochschule Z\"urich\\
  Wolfgang-Pauli-Strasse 27, 8093 Z\"urich, Switzerland\\[1ex]
  \href{mailto:nbeisert@itp.phys.ethz.ch}
  {\texttt{nbeisert@itp.phys.ethz.ch}}}
\hypersetup{pdfauthor={Niklas Beisert}}
\hypersetup{pdfsubject={Manual for the LaTeX2e Package childdoc}}
\date{30 December 2018, \textsf{v2.0}}
\maketitle

\begin{abstract}\noindent
\textsf{childdoc} is a \LaTeXe{} package
that enables the direct compilation
of document sections included by |\include|
to individual files.
\end{abstract}

\begingroup
\parskip0ex
\tableofcontents
\endgroup

%%%%%%%%%%%%%%%%%%%%%%%%%%%%%%%%%%%%%%%%%%%%%%%%%%%%%%%%%%%%%%%%%%%%%%%%%%%%%%%%
%%%%%%%%%%%%%%%%%%%%%%%%%%%%%%%%%%%%%%%%%%%%%%%%%%%%%%%%%%%%%%%%%%%%%%%%%%%%%%%%
\section{Introduction}

\LaTeX{} provides a mechanism to structure a large document (such as a book)
into a main file and several child files (containing the chapters)
using the |\include| command.
This mechanism is beneficial for documents
which span hundreds of pages in order to
make the source file(s) more manageable.
Moreover, compilation can be restricted to
selected child files by means of the |\includeonly| command.
The latter feature can be used to reduce the compilation time while editing
(this was significantly more useful in the earlier days of \LaTeX{})
or to generate a smaller document which is easier to navigate.
Another application of |\includeonly| is to generate
documents consisting of selected parts of the complete document.

However, there are a few drawbacks of the plain |\include| mechanism:
\begin{itemize}
\item
The child files cannot be compiled on their own,
they can only be compiled via the main file.
A naive editing environment
(such as a text editor with an option
to have the current file processed by \LaTeX)
may require one to switch to the main file before compiling;
attempting to compile the child file produces errors.
\item
The main file must be modified (each time)
to adjust the |\includeonly| command
to the present needs. This easily leaves the main file in a messy state.
\item
The generated document will always carry the filename
of the main document. This is inconvenient if
several child files are to be compiled and
to be kept for distribution.
\end{itemize}

The present package provides a simple interface
to make child files individually compilable by \LaTeX{}.
Compiling a child file then has the same effect as compiling
the main file with an |\includeonly| command
to select the appropriate child.
Moreover the generated document will carry the name of the child
rather than the main file.
This resolves all three above issues.

This feature is meant to make the editing of books,
thesis documents and lecture notes somewhat more convenient.
However, the package can also be used efficiently for
composing a series of documents (such as exercise sheets)
which are typically distributed individually.
It then assists the author in generating the individual documents
(potentially in different versions)
as well as a document containing the collected series.
Another application is in developing style files
or other kinds of included material
where compilation of the style file could redirect
to a sample or test file.

%%%%%%%%%%%%%%%%%%%%%%%%%%%%%%%%%%%%%%%%%%%%%%%%%%%%%%%%%%%%%%%%%%%%%%%%%%%%%%%%
%%%%%%%%%%%%%%%%%%%%%%%%%%%%%%%%%%%%%%%%%%%%%%%%%%%%%%%%%%%%%%%%%%%%%%%%%%%%%%%%
\section{Usage}

First of all, the package \textsf{childdoc} is \emph{not} a standard
\LaTeXe{} |.sty| style file! Therefore it needs to be invoked in
a non-standard way.

%%%%%%%%%%%%%%%%%%%%%%%%%%%%%%%%%%%%%%%%%%%%%%%%%%%%%%%%%%%%%%%%%%%%%%%%%%%%%%%%
\subsection{Included Files}
\label{sec:include}

%%%%%%%%%%%%%%%%%%%%%%%%%%%%%%%%%%%%%%%%
\DescribeMacro{\childdocmain}
To use the package, add the commands
\begin{center}
\begin{tabular}{l}
|\input{childdoc.def}|\\
|\childdocmain{}|\\
\end{tabular}
\end{center}
at the very top of the main \LaTeX{} file,
in particular \emph{before} the |\documentclass| statement!
The argument of |\childdocmain| should be left empty
(but it must be present).

%%%%%%%%%%%%%%%%%%%%%%%%%%%%%%%%%%%%%%%%
\DescribeMacro{\childdocof}
Furthermore, add the commands
\begin{center}
\begin{tabular}{l}
|\input{childdoc.def}|\\
|\childdocof{|\textit{main}|}|\\
\end{tabular}
\end{center}
at the top of every child file \textit{child}
which is included by |\include{|\textit{child}|}|
from within the main file
(or at least for those files to be compiled individually).
The argument \textit{main} must be the filename of the main file.

There are a couple of
considerations in setting up the main and child documents:

%%%%%%%%%%%%%%%%%%%%%%%%%%%%%%%%%%%%%%%%
\paragraph{Restrictions.}

Please note the following restrictions:
\begin{itemize}
\item
|\childdocmain| must be called with one argument \textit{main}
to ensure compatibility with earlier version of the package.
It must either be empty (|\childdocmain{}|)
or precisely match the filename of the main file in which it is specified.
See \secref{sec:detection} for further information.
\item
The filename \textit{main} must be specified without the |.tex| extension.
\item
The filename \textit{main} is case sensitive
(even in case-insensitive file systems)
due to internal string comparison.
\item
The argument \textit{main} should be fully expanded, it cannot be a macro.
\item
Subdirectories and special characters should be avoided in filenames.
\item
The command |\childdocmain{|\textit{main}|}| must be followed by a whitespace.
It should not be followed immediately by another command
or by a comment mark `|%|'.
This is because the \TeX{} parser reads the token immediately following
the argument of |\childdocmain| and puts it
at the beginning of every child section;
however, a white\-space is ignored.
\end{itemize}

%%%%%%%%%%%%%%%%%%%%%%%%%%%%%%%%%%%%%%%%
\paragraph{Content of Main File.}

It is advisable to place all content in the child files included by |\include|.
Any output contained in the main file will appear in all child documents
unless suppressed manually;
it cannot be suppressed automatically by the |\includeonly| directive
and thus should normally be avoided.
A method to include some content in the main file
by means of conditional processing is described in \secref{sec:conditional}.

%%%%%%%%%%%%%%%%%%%%%%%%%%%%%%%%%%%%%%%%
\paragraph{Page Numbering.}

When only a part of the document is compiled,
the appropriate numbering of pages
(as well as other status parameters)
is determined from the |.aux| files.
The latter contain information from previous passes.
However this information needs to propagate through
all intermediate child documents.
Therefore the page numbering in child documents may well
be inconsistent until the complete document is compiled at least once.

A useful (if unconventional) way to always ensure a consistent
page numbering is to restart the numbering in each child document
and denote the pages by `\textit{child}|.|\textit{page}'
where \textit{child} represents the chapter/section number of the child file.
This can be achieved by the command
|\numberwithin{page}{|\textit{child}|}|
of the \textsf{amsmath} package
where \textit{child} can be |chapter| or |section|
depending on the chosen structuring.
Alternatively, one can modify the macro |\thepage| appropriately
and reset the counter |page| at the start of each child file.

%%%%%%%%%%%%%%%%%%%%%%%%%%%%%%%%%%%%%%%%%%%%%%%%%%%%%%%%%%%%%%%%%%%%%%%%%%%%%%%%
\subsection{Conditional Processing}
\label{sec:conditional}

The package provides a mechanism to compile different versions
of a document. To customise the versions further some conditional processing
can come in handy to distinguish which version is being compiled.
The package provides two macros to describe the compilation context:

%%%%%%%%%%%%%%%%%%%%%%%%%%%%%%%%%%%%%%%%
\DescribeMacro{\ifchilddoc}
The conditional |\ifchilddoc| distinguishes between the compilation of
child documents and the main document:
%
\begin{center}
|\ifchilddoc |\textit{child-code}| |[|\||else |\textit{main-code}]| \||fi|
\end{center}

%%%%%%%%%%%%%%%%%%%%%%%%%%%%%%%%%%%%%%%%
\DescribeMacro{\childdocname}
\DescribeMacro{\childdocjob}
The macro |\childdocname| contains the filename (without extension)
of the main or child file being processed.
Note that |\childdocjob| will always contain the name of the main file.

%%%%%%%%%%%%%%%%%%%%%%%%%%%%%%%%%%%%%%%%
\paragraph{Title Page.}

Conditional processing can be used to include a title or banner page
in the main document when proper precautions are taken.
Importantly, the code in the main file should ensure that the page counter
(as well as other status parameters which are stored in the |.aux| files)
takes the same value after the conditional processing.
Otherwise the page numbers may take divergent values
depending on which part is compiled.

For example, a title page could be declared by:
%
\begin{center}
\begin{tabular}{l}
|\ifchilddoc\||else|\\
|\addtocounter{page}{-1}|\\
\textit{code for title page}\\
|\newpage|\\
|\||fi|
\end{tabular}
\end{center}
%
A banner page for the child documents can be generated by:
%
\begin{center}
\begin{tabular}{l}
|\ifchilddoc|\\
|\addtocounter{page}{-1}|\\
\textit{code for banner page}\\
|\newpage|\\
|\||fi|
\end{tabular}
\end{center}
%
Here one could write a message such as:
\begin{center}
|This is the part \childdocname{} of \childdocjob{}.|
\end{center}

%%%%%%%%%%%%%%%%%%%%%%%%%%%%%%%%%%%%%%%%%%%%%%%%%%%%%%%%%%%%%%%%%%%%%%%%%%%%%%%%
\subsection{Flags}
\label{sec:flags}

The package makes it easy to generate different versions
of the main or child documents.
To this end compilation flags can be defined
and assigned different default values.
They will be particularly useful in conjunction
with the forwarding mechanism described in \secref{sec:forward}.

For example, it may be useful to have a flag |\version|
which can be set to |draft| or |final|.
The document source will contain some conditional code
depending on the value of |\version|.
Suppose further, the flag should default to |final| for the main file
and to |draft| for child files
which is a natural assignment for editing the document.
This is achieved by placing the following code
in the preamble of the main document
(below the |\childdocmain| directive):
%
\begin{center}
\begin{tabular}{l}
|\ifchilddoc|\\
|\providecommand{\version}{draft}|\\
|\||else|\\
|\providecommand{\version}{final}|\\
|\||fi|
\end{tabular}
\end{center}
%
The definition by |\providecommand| makes sure
that previous definitions are not overwritten.
Further statements |\providecommand{\version}{...}|
can thus be added before the above code to override it.

For the main file, one might add a line
(between |\childdocmain| and the above block)
%
\begin{center}
|%\ifchilddoc\||else\providecommand{\version}{draft}\||fi|
\end{center}
%
which can be uncommented to produce a draft version.
Likewise one can add a line to the very top of a child file
(above the |\childdocof{|\textit{main}|}| directive)
%
\begin{center}
|%\providecommand{\version}{final}|
\end{center}
%
which can be uncommented to produce the final version of this child document.

%%%%%%%%%%%%%%%%%%%%%%%%%%%%%%%%%%%%%%%%%%%%%%%%%%%%%%%%%%%%%%%%%%%%%%%%%%%%%%%%
\subsection{Forwarding}
\label{sec:forward}

Different versions of the main or child documents
using compilation flags as described in \secref{sec:flags}
can be (permanently) stored in different files
for convenient compilation, viewing and distribution.
To this end, the package defines a command
to pass on compilation to a different file:

%%%%%%%%%%%%%%%%%%%%%%%%%%%%%%%%%%%%%%%%
\DescribeMacro{\childdocforward}
The command |\childdocforward| redirects processing to
another source file:
%
\begin{center}
\begin{tabular}{l}
|\input{childdoc.def}|\\
|\childdocforward[|\textit{main}|]{|\textit{dest}|}|\\
\end{tabular}
\end{center}
%
The argument \textit{dest} is the destination file
(without extension).
It should be the main file or one of the child files.
Note that further \textsf{childdoc} directives
such as |\childdocof| and |\childdocforward|
in the indicated file will be processed in this form.
The optional argument \textit{main}
passes on directly to the main file \textit{main}
while pretending to compile the child \textit{dest}.
This form behaves as if \textit{dest}
issues |\childdocof{|\textit{main}|}| right away,
and no further \textsf{childdoc} directives will be processed.

%%%%%%%%%%%%%%%%%%%%%%%%%%%%%%%%%%%%%%%%
\DescribeMacro{\...prefix}
In the alternative form |\childdocforwardprefix|,
%
\begin{center}
\begin{tabular}{l}
|\input{childdoc.def}|\\
|\childdocforwardprefix[|\textit{main}|]{|\textit{prefix}|}{|\textit{dest}|}|
\end{tabular}
\end{center}
%
the destination file is determined by a pattern
depending on the current file:
To make this work, the current file must be called
`{\textit{prefix}\hspace{0.2em}\textit{suffix}}'
with \textit{prefix} matching precisely the argument.
Processing is then passed on to the file
`{\textit{dest}\hspace{0.2em}\textit{suffix}}'.
Surely, the same effect is achieved by
directly specifying the
argument `{\textit{dest}\hspace{0.2em}\textit{suffix}}'
in the first form.
However, that requires to set up a different file
for each child. With the alternative form of the command
all these files can have exactly the same content
which simplifies setting them up and maintaining them.

For example, the following file |draft.tex|
with a compilation flag |\version| as described in \secref{sec:flags}
compiles the main document as a draft:
%
\begin{center}
\begin{tabular}{l}
|\def\version{draft}|\\
|\input{childdoc.def}|\\
|\childdocforward{|\textit{main}|}|
\end{tabular}
\end{center}
%
Likewise, the following files |final|\textit{nn}|.tex|
compile the final version of the child document
|child|\textit{nn}|.tex|:
%
\begin{center}
\begin{tabular}{l}
|\def\version{final}|\\
|\input{childdoc.def}|\\
|\childdocforwardprefix{final}{child}|
\end{tabular}
\end{center}
%

Note that when several versions of a main file and/or of each child file
are to be generated, it may be convenient to set up a |Makefile| or
shell script to automatise the process.

%%%%%%%%%%%%%%%%%%%%%%%%%%%%%%%%%%%%%%%%%%%%%%%%%%%%%%%%%%%%%%%%%%%%%%%%%%%%%%%%
\subsection{Command Line Processing}
\label{sec:commandline}

The effect of redirection files can also be achieved by invoking
the \LaTeX{} compiler with a more elaborate command line.
Most conveniently this should be done as part
of a shell script or a |Makefile|.

When using \textsf{childdoc} in the main file, the following
command lines effectively perform a redirection
(note that depending on the shell being used,
backslashes may have to be doubled: `|\|' $\to$ `|\\|'):
%
\begin{center}
|... -jobname "|\textit{target}|" |\\|"|[\textit{flags}]%
|\input{childdoc.def}\childdocforward[|\textit{main}|]{|\textit{dest}|}"|
\end{center}
%
Here \textit{target} is the name of the output file,
\textit{main} is the name of the main file
and \textit{dest} is the name of the main or child file to be processed
(all filenames without extensions).
The optional argument \textit{main} can be omitted
if \textit{main} matches \textit{dest}.
Optionally, compilation \textit{flags} can be defined via |\def| commands.
This command line makes the \TeX{} engine believe
it is compiling the file \textit{target}
whose content is specified as the latter parameter.
The provided code then forwards the processing to
\textit{main} or \textit{dest} as described in \secref{sec:forward}.

%%%%%%%%%%%%%%%%%%%%%%%%%%%%%%%%%%%%%%%%%%%%%%%%%%%%%%%%%%%%%%%%%%%%%%%%%%%%%%%%
\subsection{Include by Input}
\label{sec:input}

Including child documents by |\include| has some restrictions by design.
Most notably, the content of a child document always occupies
its own set of pages; pages cannot be shared between child documents.
Usually, this behaviour makes perfect sense
because each child document contain an essential part of the document.
However, in some situations it may be desirable to compose
a document from a collection of parts
without having mandatory page breaks between then.
For this case, the package
provides a mechanism to include parts
by |\input| which can also be processed individually.
However, by construction this mechanism
requires manual handling of the content to be output.

%%%%%%%%%%%%%%%%%%%%%%%%%%%%%%%%%%%%%%%%
\DescribeMacro{\ifchilddocmanual}
The main file should be prepared as usual, see \secref{sec:include}.
However, the document body must make a distinction
between processing of an individual part and of the main document, e.g.:
%
\begin{center}
\begin{tabular}{l}
|\ifchilddocmanual|\\
|\input{\childdocname}|\\
|\||else|\\
\textit{document body with }|\input{|\textit{part}|}|\\
|\||fi|
\end{tabular}
\end{center}
%
The conditional |\ifchilddocmanual| is true whenever
a part to be included by |\input| is being compiled,
and the name of the part is stored in |\childdocname|.

%%%%%%%%%%%%%%%%%%%%%%%%%%%%%%%%%%%%%%%%
\DescribeMacro{\childdocby}
Each part to be included by |\input| should start with:
%
\begin{center}
\begin{tabular}{l}
|\input{childdoc.def}|\\
|\childdocby{|\textit{main}|}|\\
\end{tabular}
\end{center}
%
The directive |\childdocby| is similar to |\childdocof|
described in \secref{sec:include},
but the subsequent selection of content must be done manually.
To that end, both |\ifchilddoc| and |\ifchilddocmanual|
will be true upon processing of a part,
and the name of the part is stored in |\childdocname|.
Note that |\jobname| will be set to the filename of the current part
so that each part receives an individual |.aux| file
that does not interfere with the |.aux| file(s) of the main document.
This behaviour can be altered by the alternative form
|\childdocby[*]{|\textit{main}|}| (with a non-empty optional argument)
which uses the |.aux| file of the main document
by setting |\jobname| to \textit{main}.

%%%%%%%%%%%%%%%%%%%%%%%%%%%%%%%%%%%%%%%%%%%%%%%%%%%%%%%%%%%%%%%%%%%%%%%%%%%%%%%%
\subsection{Driver Development}
\label{sec:driver}

The \textsf{childdoc} mechanism can also be use for the development
of definition files such as \LaTeX{} styles or classes.
This case differs from the above setup with multiple parts
included by |\include| in that no |\includeonly| should be invoked.
This can be achieved by starting the include file
(before |\ProvidesPackage|) with:
%
\begin{center}
\begin{tabular}{l}
|\input{childdoc.def}|\\
|\childdocforward{|\textit{main}|}|\\
\end{tabular}
\end{center}
%
or alternatively with:
%
\begin{center}
\begin{tabular}{l}
|\input{childdoc.def}|\\
|\childdocby{|\textit{main}|}|\\
\end{tabular}
\end{center}
%
Both forms have slightly different effects as described above.
The main file is prepared as usual, see \secref{sec:include}.

%%%%%%%%%%%%%%%%%%%%%%%%%%%%%%%%%%%%%%%%%%%%%%%%%%%%%%%%%%%%%%%%%%%%%%%%%%%%%%%%
\subsection{Legacy Detection}
\label{sec:detection}

The directive |\childdocmain| in the main file can detect
whether the complete document or merely a child is to be compiled
even without using the directive |\childdocof|.
This method is deprecated because it is less robust
and there is no compelling reason to use it;
it is merely provided for backward compatibility
and it may be removed in future versions.

If the detection mechanism is to be used,
it is mandatory to correctly specify
the filename of the main file as the argument of |\childdocmain|:
%
\begin{center}
\begin{tabular}{l}
|\input{childdoc.def}|\\
|\childdocmain{|\textit{main}|}|\\
\end{tabular}
\end{center}
%
If |\jobname| does not match the argument \textit{main} of |\childdocmain|,
it is assumed that |\jobname| points to the child file to be compiled.
When using |\childdocmain| with the main file specified as argument,
it suffices to start a child file
with just |\input{|\textit{main}|}|
without loading of the package and using |\childdocof|.
If instead all processing is done
with the appropriate \textsf{childdoc} directives,
the argument of \textit{main} of |\childdocmain| can be empty.

An alternative version of the command line processing described
in \secref{sec:commandline} using the detection mechanism reads:
%
\begin{center}
|... -jobname "|\textit{target}|" "|[\textit{flags}]%
[|\def\jobname{|\textit{dest}|}|]|\input{|\textit{main}|}"|
\end{center}

%%%%%%%%%%%%%%%%%%%%%%%%%%%%%%%%%%%%%%%%%%%%%%%%%%%%%%%%%%%%%%%%%%%%%%%%%%%%%%%%
\subsection{Manual Code}
\label{sec:manual}

In case one cannot be certain whether the definitions file |childdoc.def|
is installed on the target \TeX{} distribution
and one prefers not to ship it,
it is conceivable to paste a few relevant commands into the sources.

To that end, drop all statements |\input{childdoc.def}|
and perform the replacements as outlined below.
Instead of |\childdocmain{|\textit{main}|}| add the following code
to the top of the main file:
%
\begin{center}
\begin{tabular}{l}
|\||ifdefined\childdocname\endinput\||fi\newif\ifchilddoc|\\
|\edef\childdocname{\scantokens\expandafter{\jobname\noexpand}}|\\
|\def\childdocmain{|\textit{main}|}\||ifx\childdocmain\childdocname\||else|\\
|\childdoctrue\includeonly{\childdocname}\let\jobname\childdocmain\||fi|\\
\end{tabular}
\end{center}
%
Instead of |\childdocof{|\textit{main}|}| just include the main file
at the top of each child file:
%
\begin{center}
|\input{|\textit{main}|}|
\end{center}
%
A simple redirection |\childdocforward{|\textit{dest}|}| is achieved by:
%
\begin{center}
|\def\jobname{|\textit{dest}|}\input{\jobname}|
\end{center}
%
The redirection with prefix
|\childdocforwardprefix[|\textit{prefix}|]{|\textit{dest}|}|
is accomplished by:
%
\begin{center}
\begin{tabular}{l}
|{\edef\jobname{\scantokens\expandafter{\jobname\noexpand}}|\\
|\def\redirectjob |\textit{prefix}|#1~~~{\gdef\jobname{|\textit{dest}|#1}}|\\
|\expandafter\redirectjob\jobname~~~}\input{\jobname}|
\end{tabular}
\end{center}

In an alternative approach,
child documents can be compiled by a specific command line
without additional code or specific definitions:
%
\begin{center}
|... -jobname "|\textit{target}|" "|[\textit{flags}]%
|\includeonly{|\textit{dest}|}\input{|\textit{main}|}"|
\end{center}
%

%%%%%%%%%%%%%%%%%%%%%%%%%%%%%%%%%%%%%%%%%%%%%%%%%%%%%%%%%%%%%%%%%%%%%%%%%%%%%%%%
%%%%%%%%%%%%%%%%%%%%%%%%%%%%%%%%%%%%%%%%%%%%%%%%%%%%%%%%%%%%%%%%%%%%%%%%%%%%%%%%
\section{Information}

%%%%%%%%%%%%%%%%%%%%%%%%%%%%%%%%%%%%%%%%%%%%%%%%%%%%%%%%%%%%%%%%%%%%%%%%%%%%%%%%
\subsection{Copyright}

Copyright \copyright{} 2017--2018 Niklas Beisert

This work may be distributed and/or modified under the
conditions of the \LaTeX{} Project Public License, either version 1.3
of this license or (at your option) any later version.
The latest version of this license is in
  \url{http://www.latex-project.org/lppl.txt}
and version 1.3 or later is part of all distributions of \LaTeX{}
version 2005/12/01 or later.

This work has the LPPL maintenance status `maintained'.

The Current Maintainer of this work is Niklas Beisert.

This work consists of the files |README.txt|, |childdoc.ins| and |childdoc.dtx|
as well as the derived files |childdoc.def|, |cdocsamp.tex|
with |cdocsch1.tex|, |cdocsch2.tex|, |cdocspt3.tex|, |cdocspt4.tex|,
|cdocsdrf.tex|, |cdocsfn1.tex|, |cdocsfn2.tex|
as well as |childdoc.pdf|.

%%%%%%%%%%%%%%%%%%%%%%%%%%%%%%%%%%%%%%%%%%%%%%%%%%%%%%%%%%%%%%%%%%%%%%%%%%%%%%%%
\subsection{Files and Installation}

The package consists of the files:
%
\begin{center}
\begin{tabular}{ll}
    |README.txt|   & readme file \\
    |childdoc.ins| & installation file \\
    |childdoc.dtx| & source file \\
    |childdoc.def| & definition file \\
    |cdocsamp.tex| & sample main file \\
    |cdocsch1.tex| & sample include file \\
    |cdocsch2.tex| & sample include file \\
    |cdocspt3.tex| & sample part file \\
    |cdocspt4.tex| & sample part file \\
    |cdocsdrf.tex| & sample redirection file \\
    |cdocsfn1.tex| & sample redirection file \\
    |cdocsfn2.tex| & sample redirection file \\
    |childdoc.pdf| & manual
\end{tabular}
\end{center}
%
The distribution consists of the files
|README.txt|, |childdoc.ins| and |childdoc.dtx|.
%
\begin{itemize}
\item
Run (pdf)\LaTeX{} on |childdoc.dtx|
to compile the manual |childdoc.pdf| (this file).
\item
Run \LaTeX{} on |childdoc.ins| to create the definitions file |childdoc.def|
and the sample |cdocsamp.tex| with include files
|cdocsch1.tex|, |cdocsch2.tex|, |cdocspt3.tex|, |cdocspt4.tex|,
|cdocsdrf.tex|, |cdocsfn1.tex|, |cdocsfn2.tex|.
Then copy the file |childdoc.def| to an appropriate directory of your \LaTeX{}
distribution, e.g.\ \textit{texmf-root}|/tex/latex/childdoc|.
\end{itemize}

%%%%%%%%%%%%%%%%%%%%%%%%%%%%%%%%%%%%%%%%%%%%%%%%%%%%%%%%%%%%%%%%%%%%%%%%%%%%%%%%
\subsection{Related CTAN Packages}

There are several other packages which offer a similar functionality:
%
\begin{itemize}
\item
The packages
\href{http://ctan.org/pkg/docmute}{\textsf{docmute}},
\href{http://ctan.org/pkg/includex}{\textsf{includex}} and
\href{http://ctan.org/pkg/standalone}{\textsf{standalone}}
provide commands to include only the document body of
a child file thus allowing both files to be compiled individually.
\item
The packages \href{http://ctan.org/pkg/subdocs}{\textsf{subdocs}}
and \href{http://ctan.org/pkg/subfiles}{\textsf{subfiles}}
provide structures in which the main and child documents can be
encapsulated and allowing them to be compiled individually.
The inclusion mechanism is different from the conventional |\include|.
\item
The package \href{http://ctan.org/pkg/combine}{\textsf{combine}}
is an elaborate solution to combine several documents into one.
\end{itemize}
%
See also the CTAN topic \href{http://ctan.org/topic/subdocs}{\textsf{subdocs}}
for further related packages.
The present package differs from the above solutions in that
a document structure constructed with the conventional |\include| mechanism
just needs two extra commands at the top of every file
such that all constituent files can be compiled individually.

%%%%%%%%%%%%%%%%%%%%%%%%%%%%%%%%%%%%%%%%%%%%%%%%%%%%%%%%%%%%%%%%%%%%%%%%%%%%%%%%
%\subsection{Feature Suggestions}
%
%The following is a list of features which may be useful for future
%versions of this package:
%%
%\begin{itemize}
%\item
%\ldots
%\end{itemize}

%%%%%%%%%%%%%%%%%%%%%%%%%%%%%%%%%%%%%%%%%%%%%%%%%%%%%%%%%%%%%%%%%%%%%%%%%%%%%%%%
\subsection{Revision History}

%%%%%%%%%%%%%%%%%%%%%%%%%%%%%%%%%%%%%%%%
\paragraph{v2.0:} 2018/12/30

\begin{itemize}
\item
immediate forward processing
\item
added |\childdocby| mechanism
\item
manual restructured
\end{itemize}

%%%%%%%%%%%%%%%%%%%%%%%%%%%%%%%%%%%%%%%%
\paragraph{v1.6:} 2018/01/17

\begin{itemize}
\item
application for development of include files
\item
corrections to manual
\end{itemize}

%%%%%%%%%%%%%%%%%%%%%%%%%%%%%%%%%%%%%%%%
\paragraph{v1.5:} 2017/05/21

\begin{itemize}
\item
more complete structuring introduced
\item
|\childdocof| introduced
\item
|\childdoc| renamed to |\childdocmain|
\item
|\childredirect| renamed to |\childdocforward| and |\childdocforwardprefix|
and functionality expanded
\end{itemize}

%%%%%%%%%%%%%%%%%%%%%%%%%%%%%%%%%%%%%%%%
\paragraph{v1.0:} 2017/04/27

\begin{itemize}
\item
manual and install package
\item
first version published on CTAN
\end{itemize}

%%%%%%%%%%%%%%%%%%%%%%%%%%%%%%%%%%%%%%%%
\paragraph{v0.6:} 2017/04/26

\begin{itemize}
\item
redirection mechanism added
\end{itemize}

%%%%%%%%%%%%%%%%%%%%%%%%%%%%%%%%%%%%%%%%
\paragraph{v0.5:} 2017/04/26

\begin{itemize}
\item
functionality in definition file
\end{itemize}


%%%%%%%%%%%%%%%%%%%%%%%%%%%%%%%%%%%%%%%%%%%%%%%%%%%%%%%%%%%%%%%%%%%%%%%%%%%%%%%%
%%%%%%%%%%%%%%%%%%%%%%%%%%%%%%%%%%%%%%%%%%%%%%%%%%%%%%%%%%%%%%%%%%%%%%%%%%%%%%%%
%%%%%%%%%%%%%%%%%%%%%%%%%%%%%%%%%%%%%%%%%%%%%%%%%%%%%%%%%%%%%%%%%%%%%%%%%%%%%%%%
\appendix

\settowidth\MacroIndent{\rmfamily\scriptsize 000\ }

 \DocInput{childdoc.dtx}

\end{document}
%</driver>
% \fi
%
% %%%%%%%%%%%%%%%%%%%%%%%%%%%%%%%%%%%%%%%%%%%%%%%%%%%%%%%%%%%%%%%%%%%%%%%%%%%%%%
% %%%%%%%%%%%%%%%%%%%%%%%%%%%%%%%%%%%%%%%%%%%%%%%%%%%%%%%%%%%%%%%%%%%%%%%%%%%%%%
% \section{Sample}
%\iffalse
%<*samplemain>
%\fi
%
% The following presents a sample document
% with two chapters, two parts, a title page,
% a compile flag as well as three forwarding files to set the flag.
% It consists of eight |.tex| files:
% \begin{center}
% \begin{tabular}{ll}
% |cdocsamp.tex|&main file\\
% |cdocsch1.tex|&include file for chapter 1\\
% |cdocsch2.tex|&include file for chapter 2\\
% |cdocspt3.tex|&include file for part 3\\
% |cdocspt4.tex|&include file for part 4\\
% |cdocsdrf.tex|&forwarding file for main file in draft mode\\
% |cdocsfi1.tex|&forwarding file for final version of chapter 1\\
% |cdocsfi2.tex|&forwarding file for final version of chapter 2\\
% \end{tabular}
% \end{center}
% Each of the eight files can be compiled directly by the \LaTeX{} compiler.
%
% %%%%%%%%%%%%%%%%%%%%%%%%%%%%%%%%%%%%%%
% \paragraph{Main File.}
%
% The main file is called |cdocsamp.tex|.
%
% Load the \textsf{childdoc} definitions and
% declare the filename for the main document:
%    \begin{macrocode}
\input{childdoc.def}
\childdocmain{}
%    \end{macrocode}

% Optional override for |\version| flag:
%    \begin{macrocode}
%%\ifchilddoc\else\providecommand{\version}{draft}\fi
%    \end{macrocode}

% Define the default values for the |\version| flag
% (|final| for the main file and |draft| for childs):
%    \begin{macrocode}
\ifchilddoc
\providecommand{\version}{draft}
\else
\providecommand{\version}{final}
\fi
%    \end{macrocode}

% Load the standard document class:
%    \begin{macrocode}
\documentclass[12pt]{article}
%    \end{macrocode}

% Start the document body:
%    \begin{macrocode}
\begin{document}
%    \end{macrocode}

% Declare a title page.
% Print title, part of document being processed and version flag:
%    \begin{macrocode}
\addtocounter{page}{-1}
\begin{center}
{\LARGE\bfseries{}childdoc example\par}
\vspace{1cm}
\ifchilddoc
\ifchilddocmanual part\else chapter\fi:
`\childdocname' of `\childdocjob'\par
\else
main document: `\childdocjob'\par
\fi
version: \version\par
\end{center}
\newpage
%    \end{macrocode}

% Manually include selected file,
% otherwise process as usual:
%    \begin{macrocode}
\ifchilddocmanual
\section*{part `\childdocname'}
\input{\childdocname}
\else
%    \end{macrocode}

% Include the two chapters:
%    \begin{macrocode}
\include{cdocsch1}
\include{cdocsch2}
%    \end{macrocode}

% Include the two parts unless only chapters should be displayed:
%    \begin{macrocode}
\ifchilddoc\else
\section{part three}
\input{cdocspt3}
\section{part four}
\input{cdocspt4}
\fi
%    \end{macrocode}

% Process as usual until here:
%    \begin{macrocode}
\fi
%    \end{macrocode}

% End of document body:
%    \begin{macrocode}
\end{document}
%    \end{macrocode}
%\iffalse
%</samplemain>
%\fi
%
% %%%%%%%%%%%%%%%%%%%%%%%%%%%%%%%%%%%%%%
% \paragraph{Chapter Include Files.}
%
% The include files are called |cdocsch1.tex| and |cdocsch2.tex|.
%
%\iffalse
%<*samplechap1|samplechap2>
%\fi

% Optional override for |\version| flag:
%    \begin{macrocode}
%%\providecommand{\version}{final}
%    \end{macrocode}

% Include the main document:
%    \begin{macrocode}
\input{childdoc.def}
\childdocof{cdocsamp}
%    \end{macrocode}

%\iffalse
%</samplechap1|samplechap2>
%\fi
%
%\iffalse
%<*samplechap1>
%\fi
% Some text for chapter 1:
%    \begin{macrocode}
\section{one}
some text in chapter one
%    \end{macrocode}

%\iffalse
%</samplechap1>
%\fi
% Some text for chapter 2:
%\iffalse
%<*samplechap2>
%\fi
%    \begin{macrocode}
\section{two}
more text in chapter two
%    \end{macrocode}

%\iffalse
%</samplechap2>
%\fi
%
% %%%%%%%%%%%%%%%%%%%%%%%%%%%%%%%%%%%%%%
% \paragraph{Part Include Files.}
%
% The include files are called |cdocspt3.tex| and |cdocspt4.tex|.
%
%\iffalse
%<*samplepart3|samplepart4>
%\fi

% Optional override for |\version| flag:
%    \begin{macrocode}
%%\providecommand{\version}{final}
%    \end{macrocode}

% Include the main document:
%    \begin{macrocode}
\input{childdoc.def}
\childdocby{cdocsamp}
%    \end{macrocode}

%\iffalse
%</samplepart3|samplepart4>
%\fi
%
%\iffalse
%<*samplepart3>
%\fi
% Some text for part 3:
%    \begin{macrocode}
some text in part three
%    \end{macrocode}

%\iffalse
%</samplepart3>
%\fi
% Some text for part 4:
%\iffalse
%<*samplepart4>
%\fi
%    \begin{macrocode}
more text in part four
%    \end{macrocode}

%\iffalse
%</samplepart4>
%\fi
%
% %%%%%%%%%%%%%%%%%%%%%%%%%%%%%%%%%%%%%%
% \paragraph{Forwarding for a Complete Draft.}
%
% The following forwarding file |cdocsdrf.tex|
% compiles the main document in draft mode:
%\iffalse
%<*sampledraft>
%\fi
%    \begin{macrocode}
\def\version{draft}
\input{childdoc.def}
\childdocforward{cdocsamp}
%    \end{macrocode}

%\iffalse
%</sampledraft>
%\fi
%
% %%%%%%%%%%%%%%%%%%%%%%%%%%%%%%%%%%%%%%
% \paragraph{Forwarding for Final Version of the Chapters.}
%
% The following forwarding files |cdocsfn1.tex| and |cdocsfn2.tex|
% (with identical content)
% compile the final versions of the child documents
% |cdocsch1.tex| and |cdocsch2.tex|, respectively:
%\iffalse
%<*samplefinal>
%\fi
%    \begin{macrocode}
\def\version{final}
\input{childdoc.def}
\childdocforwardprefix[cdocsamp]{cdocsfn}{cdocsch}
%    \end{macrocode}

%\iffalse
%</samplefinal>
%\fi
%
% %%%%%%%%%%%%%%%%%%%%%%%%%%%%%%%%%%%%%%
% \paragraph{Command Line Processing.}
%
% The following three command lines generate the output files
% |cdocscld|, |cdocscl1| and |cdocscl2|
% which should be identical to
% |cdocsdrf|, |cdocsch1| and |cdocsfn2|, respectively:
% \begin{center}
% \begin{tabular}{l}
% |latex -jobname cdocscld \|\\
% |  "\def\version{draft}\input{childdoc.def}\childdocforward{cdocsamp}"|\\
% |latex -jobname cdocscl1 \|\\
% |  "\input{childdoc.def}\childdocforward[cdocsamp]{cdocsch1}"|\\
% |latex -jobname cdocscl2 \|\\
% |  "\def\version{final}\input{childdoc.def}\childdocforward{cdocsch2}"|
% \end{tabular}
% \end{center}
% Note that the trailing backslash on each first line
% merely continues the input to the second line
% (for convenient cut ant paste).
% Furthermore, the command |latex| can be replaced by any
% of its alternative versions such as |pdflatex|.
%
% %%%%%%%%%%%%%%%%%%%%%%%%%%%%%%%%%%%%%%%%%%%%%%%%%%%%%%%%%%%%%%%%%%%%%%%%%%%%%%
% %%%%%%%%%%%%%%%%%%%%%%%%%%%%%%%%%%%%%%%%%%%%%%%%%%%%%%%%%%%%%%%%%%%%%%%%%%%%%%
% \section{Implementation}
%\iffalse
%<*package>
%\fi
%
% This section describes the definitions file |childdoc.def|.

% The definitions cannot be loaded using |\usepackage| or |\RequirePackage|
% which has a mechanism to prevent loading a style file more than once.
% When loading the definitions by means of |\input|
% multiple instances have to be prevented manually:
%\iffalse
%This code needs to be before the `\ProvidesFile' directive
%which is defined at the beginning of this file.
%Therefore it is also placed there and commented out here.
%</package>
%<*discard>
%\fi
%    \begin{macrocode}
\ifdefined\childdocmain\endinput\fi
%    \end{macrocode}
%\iffalse
%</discard>
%<*package>
%\fi
%
% \macro{\ifchilddoc}
% \macro{\ifchilddocmanual}
% The conditional |\ifchilddoc| tells whether a
% child (true) or main (false) document is being compiled.
% The conditional |\ifchilddocmanual| tells whether
% the |\includeonly| mechanism is used (false) or
% the selection of child files must be performed manually (true).
% The definitions initialise to false:
%    \begin{macrocode}
\newif\ifchilddoc
\newif\ifchilddocmanual
%    \end{macrocode}

% \macro{\childdocname}
% \macro{\childdocjob}
% The macro |\childdocname| stores the name of the main document
% to be compiled. The macro |\childdocjob| stores the name of
% the document on which the \LaTeX{} compiler was originally invoked.
% The content of |\jobname| cannot be compared
% to filenames specified in the source due to different catcodes.
% The following code rescans |\jobname|, stores the result
% in |\childdocname| and saves a copy in |\childdocjob|:
%    \begin{macrocode}
\edef\childdocname{\scantokens\expandafter{\jobname\noexpand}}
\let\childdocjob\childdocname
%    \end{macrocode}

% \macro{\childdocdisable}
% The macro |\childdocdisable| prevents the main file
% from being processed more than once.
% At this stage, the main document command |\childdocmain|
% is assumed to be called once again where it should do nothing.
% Any subsequent call to it should prevent
% a secondary processing of the main document
% It overwrites the forwarding commands
% |\childdocof| and |\childdocforward|
% with empty macros to prevent further inclusions of the main document:
%    \begin{macrocode}
\newcommand{\childdocdisable}
{
  \renewcommand{\childdocmain}[1]{\renewcommand{\childdocmain}[1]{\endinput}}
  \renewcommand{\childdocof}[1]{}
  \renewcommand{\childdocby}[2][]{}
  \renewcommand{\childdocforward}[2][]{}
  \renewcommand{\childdocdisable}{}
}
%    \end{macrocode}

% \macro{\childdocmain}
% The macro |\childdocmain| is to be called at the top of the main file
% with nothing or the main filename (without extension) as argument.
% First, it breaks loops.
% If the argument is not empty and does not match |\childdocname|
% (which is set by the first inclusion of |childdoc.def|),
% |\ifchilddoc| is set to true, |\includeonly| is applied to the child file
% and |\jobname| is set to the main file
% (for proper handling of |.aux| files):
%    \begin{macrocode}
\newcommand{\childdocmain}[1]
{
  \childdocdisable\childdocmain{}
  \if?#1?\else
    \begingroup
      \def\childdoctmp{#1}
      \ifx\childdoctmp\childdocname
        \def\childdoctmp{}
      \else
        \def\childdoctmp
        {
          \childdoctrue
          \includeonly{\childdocname}
          \def\childdocjob{#1}
          \def\jobname{#1}
        }
      \fi
      \expandafter
    \endgroup
    \childdoctmp
  \fi
}
%    \end{macrocode}

% \macro{\childdocof}
% The command |\childdocof| redirects
% compilation to the main file |#1|.
%    \begin{macrocode}
\newcommand{\childdocof}[1]
{
  \childdocdisable
  \childdoctrue
  \includeonly{\childdocname}
  \def\jobname{#1}
  \def\childdocjob{#1}
  \input{#1}
}
%    \end{macrocode}

% \macro{\childdocby}
% The command |\childdocby| ....
%    \begin{macrocode}
\newcommand{\childdocby}[2][]
{
  \childdocdisable
  \childdoctrue
  \childdocmanualtrue
  \if?#1?\else
    \def\jobname{#2}
  \fi
  \def\childdocjob{#2}
  \input{#2}
  \endinput
}
%    \end{macrocode}

% \macro{\childdocforward}
% The command |\childdocforward| redirects
% compilation to the main file or
% (if the optional argument is given) a child file.
% Parameters are set as if the main file
% or a child file starting with |\childdocof| was compiled.
% Then compilation is handed over to the main file:
%    \begin{macrocode}
\newcommand{\childdocforward}[2][]
{
  \begingroup
    \if?#1?
      \def\childdoctmp
      {
        \def\childdocname{#2}
        \def\childdocjob{#2}
        \def\jobname{#2}
        \input{#2}
        \endinput
      }
    \else
      \def\childdoctmp
      {
        \childdocdisable
        \def\childdocname{#2}
        \childdoctrue
        \includeonly{#2}
        \def\childdocjob{#1}
        \def\jobname{#1}
        \input{#1}
        \endinput
      }
    \fi
    \expandafter
  \endgroup
  \childdoctmp
}
%    \end{macrocode}

% \macro{\childdocforwardprefix}
% The command |\childdocforwardprefix| redirects
% compilation to the main or a child file by means of a pattern.
% The prefix |#1| in the current filename is replaced by |#2|
% and the suffix of the current filename is kept
% (it is assumed that the filename does not contain the substring `|~~~|'
% which is used as a delimiter).
% Compilation is handed over to the new file by |\childdocforward|:
%    \begin{macrocode}
\newcommand{\childdocforwardprefix}[3][]
{
  \begingroup
    \def\childdocextract #2##1~~~{\def\childdoctmp{\childdocforward[#1]{#3##1}}}
    \expandafter\childdocextract\childdocname~~~
    \expandafter
  \endgroup
  \childdoctmp
}
%    \end{macrocode}

% \macro{\childdoc}
% The deprecated macro |\childdoc| is a legacy version of |\childdocmain|:
%    \begin{macrocode}
\newcommand{\childdoc}{\childdocmain}
%    \end{macrocode}

% \macro{\childdocredirect}
% The deprecated macro |\childdocredirect| is a legacy version
% of |\childdocforward| and |\childdocforwardprefix|:
%    \begin{macrocode}
\newcommand{\childdocredirect}[2][]
{
  \begingroup
    \if?#1?
      \def\childdoctmp{\childdocforward{#2}}
    \else
      \def\childdoctmp{\childdocforwardprefix{#1}{#2}}
    \fi
    \expandafter
  \endgroup
  \childdoctmp
}
%    \end{macrocode}

%\iffalse
%</package>
%\fi
%
\endinput

\childdocforwardprefix[cdocsamp]{cdocsfn}{cdocsch}
%    \end{macrocode}

%\iffalse
%</samplefinal>
%\fi
%
% %%%%%%%%%%%%%%%%%%%%%%%%%%%%%%%%%%%%%%
% \paragraph{Command Line Processing.}
%
% The following three command lines generate the output files
% |cdocscld|, |cdocscl1| and |cdocscl2|
% which should be identical to
% |cdocsdrf|, |cdocsch1| and |cdocsfn2|, respectively:
% \begin{center}
% \begin{tabular}{l}
% |latex -jobname cdocscld \|\\
% |  "\def\version{draft}% \iffalse
%
% childdoc.dtx Copyright (C) 2017-2018 Niklas Beisert
%
% This work may be distributed and/or modified under the
% conditions of the LaTeX Project Public License, either version 1.3
% of this license or (at your option) any later version.
% The latest version of this license is in
%   http://www.latex-project.org/lppl.txt
% and version 1.3 or later is part of all distributions of LaTeX
% version 2005/12/01 or later.
%
% This work has the LPPL maintenance status `maintained'.
%
% The Current Maintainer of this work is Niklas Beisert.
%
% This work consists of the files childdoc.dtx and childdoc.ins
% and the derived files childdoc.def and cdocsamp.tex with
% cdocsch1.tex, cdocsch2.tex, cdocsdrf.tex, cdocsfn1.tex, cdocsfn2.tex.
%
%<package>\ifdefined\childdocmain\endinput\fi
%<package>\ProvidesFile{childdoc.def}[2018/12/30 v2.0 child document driver]
%<samplemain>\ProvidesFile{cdocsamp.tex}[2018/12/30 v2.0 sample for childdoc]
%<*driver>
%\ProvidesFile{childdoc.drv}[2018/12/30 v2.0 childdoc reference manual file]
\PassOptionsToClass{10pt,a4paper}{article}
\documentclass{ltxdoc}

\usepackage[margin=35mm]{geometry}
\usepackage{hyperref}
\usepackage{hyperxmp}
\usepackage[usenames]{color}

\hypersetup{colorlinks=true}
\hypersetup{pdfstartview=FitH}
\hypersetup{pdfpagemode=UseNone}
\hypersetup{pdfsource={}}
\hypersetup{pdflang={en-UK}}
\hypersetup{pdfcopyright={Copyright 2017-2018 Niklas Beisert.
  This work may be distributed and/or modified under the
  conditions of the LaTeX Project Public License, either version 1.3
  of this license or (at your option) any later version.}}
\hypersetup{pdflicenseurl={http://www.latex-project.org/lppl.txt}}
\hypersetup{pdfcontactaddress={ETH Zurich, ITP, HIT K,
  Wolfgang-Pauli-Strasse 27}}
\hypersetup{pdfcontactpostcode={8093}}
\hypersetup{pdfcontactcity={Zurich}}
\hypersetup{pdfcontactcountry={Switzerland}}
\hypersetup{pdfcontactemail={nbeisert@itp.phys.ethz.ch}}
\hypersetup{pdfcontacturl={http://people.phys.ethz.ch/\xmptilde nbeisert/}}

\newcommand{\secref}[1]{\hyperref[#1]{section \ref*{#1}}}

\parskip1ex
\parindent0pt
\let\olditemize\itemize
\def\itemize{\olditemize\parskip0pt}

\begin{document}

\title{The \textsf{childdoc} Package}
\hypersetup{pdftitle={The childdoc Package}}
\author{Niklas Beisert\\[2ex]
  Institut f\"ur Theoretische Physik\\
  Eidgen\"ossische Technische Hochschule Z\"urich\\
  Wolfgang-Pauli-Strasse 27, 8093 Z\"urich, Switzerland\\[1ex]
  \href{mailto:nbeisert@itp.phys.ethz.ch}
  {\texttt{nbeisert@itp.phys.ethz.ch}}}
\hypersetup{pdfauthor={Niklas Beisert}}
\hypersetup{pdfsubject={Manual for the LaTeX2e Package childdoc}}
\date{30 December 2018, \textsf{v2.0}}
\maketitle

\begin{abstract}\noindent
\textsf{childdoc} is a \LaTeXe{} package
that enables the direct compilation
of document sections included by |\include|
to individual files.
\end{abstract}

\begingroup
\parskip0ex
\tableofcontents
\endgroup

%%%%%%%%%%%%%%%%%%%%%%%%%%%%%%%%%%%%%%%%%%%%%%%%%%%%%%%%%%%%%%%%%%%%%%%%%%%%%%%%
%%%%%%%%%%%%%%%%%%%%%%%%%%%%%%%%%%%%%%%%%%%%%%%%%%%%%%%%%%%%%%%%%%%%%%%%%%%%%%%%
\section{Introduction}

\LaTeX{} provides a mechanism to structure a large document (such as a book)
into a main file and several child files (containing the chapters)
using the |\include| command.
This mechanism is beneficial for documents
which span hundreds of pages in order to
make the source file(s) more manageable.
Moreover, compilation can be restricted to
selected child files by means of the |\includeonly| command.
The latter feature can be used to reduce the compilation time while editing
(this was significantly more useful in the earlier days of \LaTeX{})
or to generate a smaller document which is easier to navigate.
Another application of |\includeonly| is to generate
documents consisting of selected parts of the complete document.

However, there are a few drawbacks of the plain |\include| mechanism:
\begin{itemize}
\item
The child files cannot be compiled on their own,
they can only be compiled via the main file.
A naive editing environment
(such as a text editor with an option
to have the current file processed by \LaTeX)
may require one to switch to the main file before compiling;
attempting to compile the child file produces errors.
\item
The main file must be modified (each time)
to adjust the |\includeonly| command
to the present needs. This easily leaves the main file in a messy state.
\item
The generated document will always carry the filename
of the main document. This is inconvenient if
several child files are to be compiled and
to be kept for distribution.
\end{itemize}

The present package provides a simple interface
to make child files individually compilable by \LaTeX{}.
Compiling a child file then has the same effect as compiling
the main file with an |\includeonly| command
to select the appropriate child.
Moreover the generated document will carry the name of the child
rather than the main file.
This resolves all three above issues.

This feature is meant to make the editing of books,
thesis documents and lecture notes somewhat more convenient.
However, the package can also be used efficiently for
composing a series of documents (such as exercise sheets)
which are typically distributed individually.
It then assists the author in generating the individual documents
(potentially in different versions)
as well as a document containing the collected series.
Another application is in developing style files
or other kinds of included material
where compilation of the style file could redirect
to a sample or test file.

%%%%%%%%%%%%%%%%%%%%%%%%%%%%%%%%%%%%%%%%%%%%%%%%%%%%%%%%%%%%%%%%%%%%%%%%%%%%%%%%
%%%%%%%%%%%%%%%%%%%%%%%%%%%%%%%%%%%%%%%%%%%%%%%%%%%%%%%%%%%%%%%%%%%%%%%%%%%%%%%%
\section{Usage}

First of all, the package \textsf{childdoc} is \emph{not} a standard
\LaTeXe{} |.sty| style file! Therefore it needs to be invoked in
a non-standard way.

%%%%%%%%%%%%%%%%%%%%%%%%%%%%%%%%%%%%%%%%%%%%%%%%%%%%%%%%%%%%%%%%%%%%%%%%%%%%%%%%
\subsection{Included Files}
\label{sec:include}

%%%%%%%%%%%%%%%%%%%%%%%%%%%%%%%%%%%%%%%%
\DescribeMacro{\childdocmain}
To use the package, add the commands
\begin{center}
\begin{tabular}{l}
|\input{childdoc.def}|\\
|\childdocmain{}|\\
\end{tabular}
\end{center}
at the very top of the main \LaTeX{} file,
in particular \emph{before} the |\documentclass| statement!
The argument of |\childdocmain| should be left empty
(but it must be present).

%%%%%%%%%%%%%%%%%%%%%%%%%%%%%%%%%%%%%%%%
\DescribeMacro{\childdocof}
Furthermore, add the commands
\begin{center}
\begin{tabular}{l}
|\input{childdoc.def}|\\
|\childdocof{|\textit{main}|}|\\
\end{tabular}
\end{center}
at the top of every child file \textit{child}
which is included by |\include{|\textit{child}|}|
from within the main file
(or at least for those files to be compiled individually).
The argument \textit{main} must be the filename of the main file.

There are a couple of
considerations in setting up the main and child documents:

%%%%%%%%%%%%%%%%%%%%%%%%%%%%%%%%%%%%%%%%
\paragraph{Restrictions.}

Please note the following restrictions:
\begin{itemize}
\item
|\childdocmain| must be called with one argument \textit{main}
to ensure compatibility with earlier version of the package.
It must either be empty (|\childdocmain{}|)
or precisely match the filename of the main file in which it is specified.
See \secref{sec:detection} for further information.
\item
The filename \textit{main} must be specified without the |.tex| extension.
\item
The filename \textit{main} is case sensitive
(even in case-insensitive file systems)
due to internal string comparison.
\item
The argument \textit{main} should be fully expanded, it cannot be a macro.
\item
Subdirectories and special characters should be avoided in filenames.
\item
The command |\childdocmain{|\textit{main}|}| must be followed by a whitespace.
It should not be followed immediately by another command
or by a comment mark `|%|'.
This is because the \TeX{} parser reads the token immediately following
the argument of |\childdocmain| and puts it
at the beginning of every child section;
however, a white\-space is ignored.
\end{itemize}

%%%%%%%%%%%%%%%%%%%%%%%%%%%%%%%%%%%%%%%%
\paragraph{Content of Main File.}

It is advisable to place all content in the child files included by |\include|.
Any output contained in the main file will appear in all child documents
unless suppressed manually;
it cannot be suppressed automatically by the |\includeonly| directive
and thus should normally be avoided.
A method to include some content in the main file
by means of conditional processing is described in \secref{sec:conditional}.

%%%%%%%%%%%%%%%%%%%%%%%%%%%%%%%%%%%%%%%%
\paragraph{Page Numbering.}

When only a part of the document is compiled,
the appropriate numbering of pages
(as well as other status parameters)
is determined from the |.aux| files.
The latter contain information from previous passes.
However this information needs to propagate through
all intermediate child documents.
Therefore the page numbering in child documents may well
be inconsistent until the complete document is compiled at least once.

A useful (if unconventional) way to always ensure a consistent
page numbering is to restart the numbering in each child document
and denote the pages by `\textit{child}|.|\textit{page}'
where \textit{child} represents the chapter/section number of the child file.
This can be achieved by the command
|\numberwithin{page}{|\textit{child}|}|
of the \textsf{amsmath} package
where \textit{child} can be |chapter| or |section|
depending on the chosen structuring.
Alternatively, one can modify the macro |\thepage| appropriately
and reset the counter |page| at the start of each child file.

%%%%%%%%%%%%%%%%%%%%%%%%%%%%%%%%%%%%%%%%%%%%%%%%%%%%%%%%%%%%%%%%%%%%%%%%%%%%%%%%
\subsection{Conditional Processing}
\label{sec:conditional}

The package provides a mechanism to compile different versions
of a document. To customise the versions further some conditional processing
can come in handy to distinguish which version is being compiled.
The package provides two macros to describe the compilation context:

%%%%%%%%%%%%%%%%%%%%%%%%%%%%%%%%%%%%%%%%
\DescribeMacro{\ifchilddoc}
The conditional |\ifchilddoc| distinguishes between the compilation of
child documents and the main document:
%
\begin{center}
|\ifchilddoc |\textit{child-code}| |[|\||else |\textit{main-code}]| \||fi|
\end{center}

%%%%%%%%%%%%%%%%%%%%%%%%%%%%%%%%%%%%%%%%
\DescribeMacro{\childdocname}
\DescribeMacro{\childdocjob}
The macro |\childdocname| contains the filename (without extension)
of the main or child file being processed.
Note that |\childdocjob| will always contain the name of the main file.

%%%%%%%%%%%%%%%%%%%%%%%%%%%%%%%%%%%%%%%%
\paragraph{Title Page.}

Conditional processing can be used to include a title or banner page
in the main document when proper precautions are taken.
Importantly, the code in the main file should ensure that the page counter
(as well as other status parameters which are stored in the |.aux| files)
takes the same value after the conditional processing.
Otherwise the page numbers may take divergent values
depending on which part is compiled.

For example, a title page could be declared by:
%
\begin{center}
\begin{tabular}{l}
|\ifchilddoc\||else|\\
|\addtocounter{page}{-1}|\\
\textit{code for title page}\\
|\newpage|\\
|\||fi|
\end{tabular}
\end{center}
%
A banner page for the child documents can be generated by:
%
\begin{center}
\begin{tabular}{l}
|\ifchilddoc|\\
|\addtocounter{page}{-1}|\\
\textit{code for banner page}\\
|\newpage|\\
|\||fi|
\end{tabular}
\end{center}
%
Here one could write a message such as:
\begin{center}
|This is the part \childdocname{} of \childdocjob{}.|
\end{center}

%%%%%%%%%%%%%%%%%%%%%%%%%%%%%%%%%%%%%%%%%%%%%%%%%%%%%%%%%%%%%%%%%%%%%%%%%%%%%%%%
\subsection{Flags}
\label{sec:flags}

The package makes it easy to generate different versions
of the main or child documents.
To this end compilation flags can be defined
and assigned different default values.
They will be particularly useful in conjunction
with the forwarding mechanism described in \secref{sec:forward}.

For example, it may be useful to have a flag |\version|
which can be set to |draft| or |final|.
The document source will contain some conditional code
depending on the value of |\version|.
Suppose further, the flag should default to |final| for the main file
and to |draft| for child files
which is a natural assignment for editing the document.
This is achieved by placing the following code
in the preamble of the main document
(below the |\childdocmain| directive):
%
\begin{center}
\begin{tabular}{l}
|\ifchilddoc|\\
|\providecommand{\version}{draft}|\\
|\||else|\\
|\providecommand{\version}{final}|\\
|\||fi|
\end{tabular}
\end{center}
%
The definition by |\providecommand| makes sure
that previous definitions are not overwritten.
Further statements |\providecommand{\version}{...}|
can thus be added before the above code to override it.

For the main file, one might add a line
(between |\childdocmain| and the above block)
%
\begin{center}
|%\ifchilddoc\||else\providecommand{\version}{draft}\||fi|
\end{center}
%
which can be uncommented to produce a draft version.
Likewise one can add a line to the very top of a child file
(above the |\childdocof{|\textit{main}|}| directive)
%
\begin{center}
|%\providecommand{\version}{final}|
\end{center}
%
which can be uncommented to produce the final version of this child document.

%%%%%%%%%%%%%%%%%%%%%%%%%%%%%%%%%%%%%%%%%%%%%%%%%%%%%%%%%%%%%%%%%%%%%%%%%%%%%%%%
\subsection{Forwarding}
\label{sec:forward}

Different versions of the main or child documents
using compilation flags as described in \secref{sec:flags}
can be (permanently) stored in different files
for convenient compilation, viewing and distribution.
To this end, the package defines a command
to pass on compilation to a different file:

%%%%%%%%%%%%%%%%%%%%%%%%%%%%%%%%%%%%%%%%
\DescribeMacro{\childdocforward}
The command |\childdocforward| redirects processing to
another source file:
%
\begin{center}
\begin{tabular}{l}
|\input{childdoc.def}|\\
|\childdocforward[|\textit{main}|]{|\textit{dest}|}|\\
\end{tabular}
\end{center}
%
The argument \textit{dest} is the destination file
(without extension).
It should be the main file or one of the child files.
Note that further \textsf{childdoc} directives
such as |\childdocof| and |\childdocforward|
in the indicated file will be processed in this form.
The optional argument \textit{main}
passes on directly to the main file \textit{main}
while pretending to compile the child \textit{dest}.
This form behaves as if \textit{dest}
issues |\childdocof{|\textit{main}|}| right away,
and no further \textsf{childdoc} directives will be processed.

%%%%%%%%%%%%%%%%%%%%%%%%%%%%%%%%%%%%%%%%
\DescribeMacro{\...prefix}
In the alternative form |\childdocforwardprefix|,
%
\begin{center}
\begin{tabular}{l}
|\input{childdoc.def}|\\
|\childdocforwardprefix[|\textit{main}|]{|\textit{prefix}|}{|\textit{dest}|}|
\end{tabular}
\end{center}
%
the destination file is determined by a pattern
depending on the current file:
To make this work, the current file must be called
`{\textit{prefix}\hspace{0.2em}\textit{suffix}}'
with \textit{prefix} matching precisely the argument.
Processing is then passed on to the file
`{\textit{dest}\hspace{0.2em}\textit{suffix}}'.
Surely, the same effect is achieved by
directly specifying the
argument `{\textit{dest}\hspace{0.2em}\textit{suffix}}'
in the first form.
However, that requires to set up a different file
for each child. With the alternative form of the command
all these files can have exactly the same content
which simplifies setting them up and maintaining them.

For example, the following file |draft.tex|
with a compilation flag |\version| as described in \secref{sec:flags}
compiles the main document as a draft:
%
\begin{center}
\begin{tabular}{l}
|\def\version{draft}|\\
|\input{childdoc.def}|\\
|\childdocforward{|\textit{main}|}|
\end{tabular}
\end{center}
%
Likewise, the following files |final|\textit{nn}|.tex|
compile the final version of the child document
|child|\textit{nn}|.tex|:
%
\begin{center}
\begin{tabular}{l}
|\def\version{final}|\\
|\input{childdoc.def}|\\
|\childdocforwardprefix{final}{child}|
\end{tabular}
\end{center}
%

Note that when several versions of a main file and/or of each child file
are to be generated, it may be convenient to set up a |Makefile| or
shell script to automatise the process.

%%%%%%%%%%%%%%%%%%%%%%%%%%%%%%%%%%%%%%%%%%%%%%%%%%%%%%%%%%%%%%%%%%%%%%%%%%%%%%%%
\subsection{Command Line Processing}
\label{sec:commandline}

The effect of redirection files can also be achieved by invoking
the \LaTeX{} compiler with a more elaborate command line.
Most conveniently this should be done as part
of a shell script or a |Makefile|.

When using \textsf{childdoc} in the main file, the following
command lines effectively perform a redirection
(note that depending on the shell being used,
backslashes may have to be doubled: `|\|' $\to$ `|\\|'):
%
\begin{center}
|... -jobname "|\textit{target}|" |\\|"|[\textit{flags}]%
|\input{childdoc.def}\childdocforward[|\textit{main}|]{|\textit{dest}|}"|
\end{center}
%
Here \textit{target} is the name of the output file,
\textit{main} is the name of the main file
and \textit{dest} is the name of the main or child file to be processed
(all filenames without extensions).
The optional argument \textit{main} can be omitted
if \textit{main} matches \textit{dest}.
Optionally, compilation \textit{flags} can be defined via |\def| commands.
This command line makes the \TeX{} engine believe
it is compiling the file \textit{target}
whose content is specified as the latter parameter.
The provided code then forwards the processing to
\textit{main} or \textit{dest} as described in \secref{sec:forward}.

%%%%%%%%%%%%%%%%%%%%%%%%%%%%%%%%%%%%%%%%%%%%%%%%%%%%%%%%%%%%%%%%%%%%%%%%%%%%%%%%
\subsection{Include by Input}
\label{sec:input}

Including child documents by |\include| has some restrictions by design.
Most notably, the content of a child document always occupies
its own set of pages; pages cannot be shared between child documents.
Usually, this behaviour makes perfect sense
because each child document contain an essential part of the document.
However, in some situations it may be desirable to compose
a document from a collection of parts
without having mandatory page breaks between then.
For this case, the package
provides a mechanism to include parts
by |\input| which can also be processed individually.
However, by construction this mechanism
requires manual handling of the content to be output.

%%%%%%%%%%%%%%%%%%%%%%%%%%%%%%%%%%%%%%%%
\DescribeMacro{\ifchilddocmanual}
The main file should be prepared as usual, see \secref{sec:include}.
However, the document body must make a distinction
between processing of an individual part and of the main document, e.g.:
%
\begin{center}
\begin{tabular}{l}
|\ifchilddocmanual|\\
|\input{\childdocname}|\\
|\||else|\\
\textit{document body with }|\input{|\textit{part}|}|\\
|\||fi|
\end{tabular}
\end{center}
%
The conditional |\ifchilddocmanual| is true whenever
a part to be included by |\input| is being compiled,
and the name of the part is stored in |\childdocname|.

%%%%%%%%%%%%%%%%%%%%%%%%%%%%%%%%%%%%%%%%
\DescribeMacro{\childdocby}
Each part to be included by |\input| should start with:
%
\begin{center}
\begin{tabular}{l}
|\input{childdoc.def}|\\
|\childdocby{|\textit{main}|}|\\
\end{tabular}
\end{center}
%
The directive |\childdocby| is similar to |\childdocof|
described in \secref{sec:include},
but the subsequent selection of content must be done manually.
To that end, both |\ifchilddoc| and |\ifchilddocmanual|
will be true upon processing of a part,
and the name of the part is stored in |\childdocname|.
Note that |\jobname| will be set to the filename of the current part
so that each part receives an individual |.aux| file
that does not interfere with the |.aux| file(s) of the main document.
This behaviour can be altered by the alternative form
|\childdocby[*]{|\textit{main}|}| (with a non-empty optional argument)
which uses the |.aux| file of the main document
by setting |\jobname| to \textit{main}.

%%%%%%%%%%%%%%%%%%%%%%%%%%%%%%%%%%%%%%%%%%%%%%%%%%%%%%%%%%%%%%%%%%%%%%%%%%%%%%%%
\subsection{Driver Development}
\label{sec:driver}

The \textsf{childdoc} mechanism can also be use for the development
of definition files such as \LaTeX{} styles or classes.
This case differs from the above setup with multiple parts
included by |\include| in that no |\includeonly| should be invoked.
This can be achieved by starting the include file
(before |\ProvidesPackage|) with:
%
\begin{center}
\begin{tabular}{l}
|\input{childdoc.def}|\\
|\childdocforward{|\textit{main}|}|\\
\end{tabular}
\end{center}
%
or alternatively with:
%
\begin{center}
\begin{tabular}{l}
|\input{childdoc.def}|\\
|\childdocby{|\textit{main}|}|\\
\end{tabular}
\end{center}
%
Both forms have slightly different effects as described above.
The main file is prepared as usual, see \secref{sec:include}.

%%%%%%%%%%%%%%%%%%%%%%%%%%%%%%%%%%%%%%%%%%%%%%%%%%%%%%%%%%%%%%%%%%%%%%%%%%%%%%%%
\subsection{Legacy Detection}
\label{sec:detection}

The directive |\childdocmain| in the main file can detect
whether the complete document or merely a child is to be compiled
even without using the directive |\childdocof|.
This method is deprecated because it is less robust
and there is no compelling reason to use it;
it is merely provided for backward compatibility
and it may be removed in future versions.

If the detection mechanism is to be used,
it is mandatory to correctly specify
the filename of the main file as the argument of |\childdocmain|:
%
\begin{center}
\begin{tabular}{l}
|\input{childdoc.def}|\\
|\childdocmain{|\textit{main}|}|\\
\end{tabular}
\end{center}
%
If |\jobname| does not match the argument \textit{main} of |\childdocmain|,
it is assumed that |\jobname| points to the child file to be compiled.
When using |\childdocmain| with the main file specified as argument,
it suffices to start a child file
with just |\input{|\textit{main}|}|
without loading of the package and using |\childdocof|.
If instead all processing is done
with the appropriate \textsf{childdoc} directives,
the argument of \textit{main} of |\childdocmain| can be empty.

An alternative version of the command line processing described
in \secref{sec:commandline} using the detection mechanism reads:
%
\begin{center}
|... -jobname "|\textit{target}|" "|[\textit{flags}]%
[|\def\jobname{|\textit{dest}|}|]|\input{|\textit{main}|}"|
\end{center}

%%%%%%%%%%%%%%%%%%%%%%%%%%%%%%%%%%%%%%%%%%%%%%%%%%%%%%%%%%%%%%%%%%%%%%%%%%%%%%%%
\subsection{Manual Code}
\label{sec:manual}

In case one cannot be certain whether the definitions file |childdoc.def|
is installed on the target \TeX{} distribution
and one prefers not to ship it,
it is conceivable to paste a few relevant commands into the sources.

To that end, drop all statements |\input{childdoc.def}|
and perform the replacements as outlined below.
Instead of |\childdocmain{|\textit{main}|}| add the following code
to the top of the main file:
%
\begin{center}
\begin{tabular}{l}
|\||ifdefined\childdocname\endinput\||fi\newif\ifchilddoc|\\
|\edef\childdocname{\scantokens\expandafter{\jobname\noexpand}}|\\
|\def\childdocmain{|\textit{main}|}\||ifx\childdocmain\childdocname\||else|\\
|\childdoctrue\includeonly{\childdocname}\let\jobname\childdocmain\||fi|\\
\end{tabular}
\end{center}
%
Instead of |\childdocof{|\textit{main}|}| just include the main file
at the top of each child file:
%
\begin{center}
|\input{|\textit{main}|}|
\end{center}
%
A simple redirection |\childdocforward{|\textit{dest}|}| is achieved by:
%
\begin{center}
|\def\jobname{|\textit{dest}|}\input{\jobname}|
\end{center}
%
The redirection with prefix
|\childdocforwardprefix[|\textit{prefix}|]{|\textit{dest}|}|
is accomplished by:
%
\begin{center}
\begin{tabular}{l}
|{\edef\jobname{\scantokens\expandafter{\jobname\noexpand}}|\\
|\def\redirectjob |\textit{prefix}|#1~~~{\gdef\jobname{|\textit{dest}|#1}}|\\
|\expandafter\redirectjob\jobname~~~}\input{\jobname}|
\end{tabular}
\end{center}

In an alternative approach,
child documents can be compiled by a specific command line
without additional code or specific definitions:
%
\begin{center}
|... -jobname "|\textit{target}|" "|[\textit{flags}]%
|\includeonly{|\textit{dest}|}\input{|\textit{main}|}"|
\end{center}
%

%%%%%%%%%%%%%%%%%%%%%%%%%%%%%%%%%%%%%%%%%%%%%%%%%%%%%%%%%%%%%%%%%%%%%%%%%%%%%%%%
%%%%%%%%%%%%%%%%%%%%%%%%%%%%%%%%%%%%%%%%%%%%%%%%%%%%%%%%%%%%%%%%%%%%%%%%%%%%%%%%
\section{Information}

%%%%%%%%%%%%%%%%%%%%%%%%%%%%%%%%%%%%%%%%%%%%%%%%%%%%%%%%%%%%%%%%%%%%%%%%%%%%%%%%
\subsection{Copyright}

Copyright \copyright{} 2017--2018 Niklas Beisert

This work may be distributed and/or modified under the
conditions of the \LaTeX{} Project Public License, either version 1.3
of this license or (at your option) any later version.
The latest version of this license is in
  \url{http://www.latex-project.org/lppl.txt}
and version 1.3 or later is part of all distributions of \LaTeX{}
version 2005/12/01 or later.

This work has the LPPL maintenance status `maintained'.

The Current Maintainer of this work is Niklas Beisert.

This work consists of the files |README.txt|, |childdoc.ins| and |childdoc.dtx|
as well as the derived files |childdoc.def|, |cdocsamp.tex|
with |cdocsch1.tex|, |cdocsch2.tex|, |cdocspt3.tex|, |cdocspt4.tex|,
|cdocsdrf.tex|, |cdocsfn1.tex|, |cdocsfn2.tex|
as well as |childdoc.pdf|.

%%%%%%%%%%%%%%%%%%%%%%%%%%%%%%%%%%%%%%%%%%%%%%%%%%%%%%%%%%%%%%%%%%%%%%%%%%%%%%%%
\subsection{Files and Installation}

The package consists of the files:
%
\begin{center}
\begin{tabular}{ll}
    |README.txt|   & readme file \\
    |childdoc.ins| & installation file \\
    |childdoc.dtx| & source file \\
    |childdoc.def| & definition file \\
    |cdocsamp.tex| & sample main file \\
    |cdocsch1.tex| & sample include file \\
    |cdocsch2.tex| & sample include file \\
    |cdocspt3.tex| & sample part file \\
    |cdocspt4.tex| & sample part file \\
    |cdocsdrf.tex| & sample redirection file \\
    |cdocsfn1.tex| & sample redirection file \\
    |cdocsfn2.tex| & sample redirection file \\
    |childdoc.pdf| & manual
\end{tabular}
\end{center}
%
The distribution consists of the files
|README.txt|, |childdoc.ins| and |childdoc.dtx|.
%
\begin{itemize}
\item
Run (pdf)\LaTeX{} on |childdoc.dtx|
to compile the manual |childdoc.pdf| (this file).
\item
Run \LaTeX{} on |childdoc.ins| to create the definitions file |childdoc.def|
and the sample |cdocsamp.tex| with include files
|cdocsch1.tex|, |cdocsch2.tex|, |cdocspt3.tex|, |cdocspt4.tex|,
|cdocsdrf.tex|, |cdocsfn1.tex|, |cdocsfn2.tex|.
Then copy the file |childdoc.def| to an appropriate directory of your \LaTeX{}
distribution, e.g.\ \textit{texmf-root}|/tex/latex/childdoc|.
\end{itemize}

%%%%%%%%%%%%%%%%%%%%%%%%%%%%%%%%%%%%%%%%%%%%%%%%%%%%%%%%%%%%%%%%%%%%%%%%%%%%%%%%
\subsection{Related CTAN Packages}

There are several other packages which offer a similar functionality:
%
\begin{itemize}
\item
The packages
\href{http://ctan.org/pkg/docmute}{\textsf{docmute}},
\href{http://ctan.org/pkg/includex}{\textsf{includex}} and
\href{http://ctan.org/pkg/standalone}{\textsf{standalone}}
provide commands to include only the document body of
a child file thus allowing both files to be compiled individually.
\item
The packages \href{http://ctan.org/pkg/subdocs}{\textsf{subdocs}}
and \href{http://ctan.org/pkg/subfiles}{\textsf{subfiles}}
provide structures in which the main and child documents can be
encapsulated and allowing them to be compiled individually.
The inclusion mechanism is different from the conventional |\include|.
\item
The package \href{http://ctan.org/pkg/combine}{\textsf{combine}}
is an elaborate solution to combine several documents into one.
\end{itemize}
%
See also the CTAN topic \href{http://ctan.org/topic/subdocs}{\textsf{subdocs}}
for further related packages.
The present package differs from the above solutions in that
a document structure constructed with the conventional |\include| mechanism
just needs two extra commands at the top of every file
such that all constituent files can be compiled individually.

%%%%%%%%%%%%%%%%%%%%%%%%%%%%%%%%%%%%%%%%%%%%%%%%%%%%%%%%%%%%%%%%%%%%%%%%%%%%%%%%
%\subsection{Feature Suggestions}
%
%The following is a list of features which may be useful for future
%versions of this package:
%%
%\begin{itemize}
%\item
%\ldots
%\end{itemize}

%%%%%%%%%%%%%%%%%%%%%%%%%%%%%%%%%%%%%%%%%%%%%%%%%%%%%%%%%%%%%%%%%%%%%%%%%%%%%%%%
\subsection{Revision History}

%%%%%%%%%%%%%%%%%%%%%%%%%%%%%%%%%%%%%%%%
\paragraph{v2.0:} 2018/12/30

\begin{itemize}
\item
immediate forward processing
\item
added |\childdocby| mechanism
\item
manual restructured
\end{itemize}

%%%%%%%%%%%%%%%%%%%%%%%%%%%%%%%%%%%%%%%%
\paragraph{v1.6:} 2018/01/17

\begin{itemize}
\item
application for development of include files
\item
corrections to manual
\end{itemize}

%%%%%%%%%%%%%%%%%%%%%%%%%%%%%%%%%%%%%%%%
\paragraph{v1.5:} 2017/05/21

\begin{itemize}
\item
more complete structuring introduced
\item
|\childdocof| introduced
\item
|\childdoc| renamed to |\childdocmain|
\item
|\childredirect| renamed to |\childdocforward| and |\childdocforwardprefix|
and functionality expanded
\end{itemize}

%%%%%%%%%%%%%%%%%%%%%%%%%%%%%%%%%%%%%%%%
\paragraph{v1.0:} 2017/04/27

\begin{itemize}
\item
manual and install package
\item
first version published on CTAN
\end{itemize}

%%%%%%%%%%%%%%%%%%%%%%%%%%%%%%%%%%%%%%%%
\paragraph{v0.6:} 2017/04/26

\begin{itemize}
\item
redirection mechanism added
\end{itemize}

%%%%%%%%%%%%%%%%%%%%%%%%%%%%%%%%%%%%%%%%
\paragraph{v0.5:} 2017/04/26

\begin{itemize}
\item
functionality in definition file
\end{itemize}


%%%%%%%%%%%%%%%%%%%%%%%%%%%%%%%%%%%%%%%%%%%%%%%%%%%%%%%%%%%%%%%%%%%%%%%%%%%%%%%%
%%%%%%%%%%%%%%%%%%%%%%%%%%%%%%%%%%%%%%%%%%%%%%%%%%%%%%%%%%%%%%%%%%%%%%%%%%%%%%%%
%%%%%%%%%%%%%%%%%%%%%%%%%%%%%%%%%%%%%%%%%%%%%%%%%%%%%%%%%%%%%%%%%%%%%%%%%%%%%%%%
\appendix

\settowidth\MacroIndent{\rmfamily\scriptsize 000\ }

 \DocInput{childdoc.dtx}

\end{document}
%</driver>
% \fi
%
% %%%%%%%%%%%%%%%%%%%%%%%%%%%%%%%%%%%%%%%%%%%%%%%%%%%%%%%%%%%%%%%%%%%%%%%%%%%%%%
% %%%%%%%%%%%%%%%%%%%%%%%%%%%%%%%%%%%%%%%%%%%%%%%%%%%%%%%%%%%%%%%%%%%%%%%%%%%%%%
% \section{Sample}
%\iffalse
%<*samplemain>
%\fi
%
% The following presents a sample document
% with two chapters, two parts, a title page,
% a compile flag as well as three forwarding files to set the flag.
% It consists of eight |.tex| files:
% \begin{center}
% \begin{tabular}{ll}
% |cdocsamp.tex|&main file\\
% |cdocsch1.tex|&include file for chapter 1\\
% |cdocsch2.tex|&include file for chapter 2\\
% |cdocspt3.tex|&include file for part 3\\
% |cdocspt4.tex|&include file for part 4\\
% |cdocsdrf.tex|&forwarding file for main file in draft mode\\
% |cdocsfi1.tex|&forwarding file for final version of chapter 1\\
% |cdocsfi2.tex|&forwarding file for final version of chapter 2\\
% \end{tabular}
% \end{center}
% Each of the eight files can be compiled directly by the \LaTeX{} compiler.
%
% %%%%%%%%%%%%%%%%%%%%%%%%%%%%%%%%%%%%%%
% \paragraph{Main File.}
%
% The main file is called |cdocsamp.tex|.
%
% Load the \textsf{childdoc} definitions and
% declare the filename for the main document:
%    \begin{macrocode}
\input{childdoc.def}
\childdocmain{}
%    \end{macrocode}

% Optional override for |\version| flag:
%    \begin{macrocode}
%%\ifchilddoc\else\providecommand{\version}{draft}\fi
%    \end{macrocode}

% Define the default values for the |\version| flag
% (|final| for the main file and |draft| for childs):
%    \begin{macrocode}
\ifchilddoc
\providecommand{\version}{draft}
\else
\providecommand{\version}{final}
\fi
%    \end{macrocode}

% Load the standard document class:
%    \begin{macrocode}
\documentclass[12pt]{article}
%    \end{macrocode}

% Start the document body:
%    \begin{macrocode}
\begin{document}
%    \end{macrocode}

% Declare a title page.
% Print title, part of document being processed and version flag:
%    \begin{macrocode}
\addtocounter{page}{-1}
\begin{center}
{\LARGE\bfseries{}childdoc example\par}
\vspace{1cm}
\ifchilddoc
\ifchilddocmanual part\else chapter\fi:
`\childdocname' of `\childdocjob'\par
\else
main document: `\childdocjob'\par
\fi
version: \version\par
\end{center}
\newpage
%    \end{macrocode}

% Manually include selected file,
% otherwise process as usual:
%    \begin{macrocode}
\ifchilddocmanual
\section*{part `\childdocname'}
\input{\childdocname}
\else
%    \end{macrocode}

% Include the two chapters:
%    \begin{macrocode}
\include{cdocsch1}
\include{cdocsch2}
%    \end{macrocode}

% Include the two parts unless only chapters should be displayed:
%    \begin{macrocode}
\ifchilddoc\else
\section{part three}
\input{cdocspt3}
\section{part four}
\input{cdocspt4}
\fi
%    \end{macrocode}

% Process as usual until here:
%    \begin{macrocode}
\fi
%    \end{macrocode}

% End of document body:
%    \begin{macrocode}
\end{document}
%    \end{macrocode}
%\iffalse
%</samplemain>
%\fi
%
% %%%%%%%%%%%%%%%%%%%%%%%%%%%%%%%%%%%%%%
% \paragraph{Chapter Include Files.}
%
% The include files are called |cdocsch1.tex| and |cdocsch2.tex|.
%
%\iffalse
%<*samplechap1|samplechap2>
%\fi

% Optional override for |\version| flag:
%    \begin{macrocode}
%%\providecommand{\version}{final}
%    \end{macrocode}

% Include the main document:
%    \begin{macrocode}
\input{childdoc.def}
\childdocof{cdocsamp}
%    \end{macrocode}

%\iffalse
%</samplechap1|samplechap2>
%\fi
%
%\iffalse
%<*samplechap1>
%\fi
% Some text for chapter 1:
%    \begin{macrocode}
\section{one}
some text in chapter one
%    \end{macrocode}

%\iffalse
%</samplechap1>
%\fi
% Some text for chapter 2:
%\iffalse
%<*samplechap2>
%\fi
%    \begin{macrocode}
\section{two}
more text in chapter two
%    \end{macrocode}

%\iffalse
%</samplechap2>
%\fi
%
% %%%%%%%%%%%%%%%%%%%%%%%%%%%%%%%%%%%%%%
% \paragraph{Part Include Files.}
%
% The include files are called |cdocspt3.tex| and |cdocspt4.tex|.
%
%\iffalse
%<*samplepart3|samplepart4>
%\fi

% Optional override for |\version| flag:
%    \begin{macrocode}
%%\providecommand{\version}{final}
%    \end{macrocode}

% Include the main document:
%    \begin{macrocode}
\input{childdoc.def}
\childdocby{cdocsamp}
%    \end{macrocode}

%\iffalse
%</samplepart3|samplepart4>
%\fi
%
%\iffalse
%<*samplepart3>
%\fi
% Some text for part 3:
%    \begin{macrocode}
some text in part three
%    \end{macrocode}

%\iffalse
%</samplepart3>
%\fi
% Some text for part 4:
%\iffalse
%<*samplepart4>
%\fi
%    \begin{macrocode}
more text in part four
%    \end{macrocode}

%\iffalse
%</samplepart4>
%\fi
%
% %%%%%%%%%%%%%%%%%%%%%%%%%%%%%%%%%%%%%%
% \paragraph{Forwarding for a Complete Draft.}
%
% The following forwarding file |cdocsdrf.tex|
% compiles the main document in draft mode:
%\iffalse
%<*sampledraft>
%\fi
%    \begin{macrocode}
\def\version{draft}
\input{childdoc.def}
\childdocforward{cdocsamp}
%    \end{macrocode}

%\iffalse
%</sampledraft>
%\fi
%
% %%%%%%%%%%%%%%%%%%%%%%%%%%%%%%%%%%%%%%
% \paragraph{Forwarding for Final Version of the Chapters.}
%
% The following forwarding files |cdocsfn1.tex| and |cdocsfn2.tex|
% (with identical content)
% compile the final versions of the child documents
% |cdocsch1.tex| and |cdocsch2.tex|, respectively:
%\iffalse
%<*samplefinal>
%\fi
%    \begin{macrocode}
\def\version{final}
\input{childdoc.def}
\childdocforwardprefix[cdocsamp]{cdocsfn}{cdocsch}
%    \end{macrocode}

%\iffalse
%</samplefinal>
%\fi
%
% %%%%%%%%%%%%%%%%%%%%%%%%%%%%%%%%%%%%%%
% \paragraph{Command Line Processing.}
%
% The following three command lines generate the output files
% |cdocscld|, |cdocscl1| and |cdocscl2|
% which should be identical to
% |cdocsdrf|, |cdocsch1| and |cdocsfn2|, respectively:
% \begin{center}
% \begin{tabular}{l}
% |latex -jobname cdocscld \|\\
% |  "\def\version{draft}\input{childdoc.def}\childdocforward{cdocsamp}"|\\
% |latex -jobname cdocscl1 \|\\
% |  "\input{childdoc.def}\childdocforward[cdocsamp]{cdocsch1}"|\\
% |latex -jobname cdocscl2 \|\\
% |  "\def\version{final}\input{childdoc.def}\childdocforward{cdocsch2}"|
% \end{tabular}
% \end{center}
% Note that the trailing backslash on each first line
% merely continues the input to the second line
% (for convenient cut ant paste).
% Furthermore, the command |latex| can be replaced by any
% of its alternative versions such as |pdflatex|.
%
% %%%%%%%%%%%%%%%%%%%%%%%%%%%%%%%%%%%%%%%%%%%%%%%%%%%%%%%%%%%%%%%%%%%%%%%%%%%%%%
% %%%%%%%%%%%%%%%%%%%%%%%%%%%%%%%%%%%%%%%%%%%%%%%%%%%%%%%%%%%%%%%%%%%%%%%%%%%%%%
% \section{Implementation}
%\iffalse
%<*package>
%\fi
%
% This section describes the definitions file |childdoc.def|.

% The definitions cannot be loaded using |\usepackage| or |\RequirePackage|
% which has a mechanism to prevent loading a style file more than once.
% When loading the definitions by means of |\input|
% multiple instances have to be prevented manually:
%\iffalse
%This code needs to be before the `\ProvidesFile' directive
%which is defined at the beginning of this file.
%Therefore it is also placed there and commented out here.
%</package>
%<*discard>
%\fi
%    \begin{macrocode}
\ifdefined\childdocmain\endinput\fi
%    \end{macrocode}
%\iffalse
%</discard>
%<*package>
%\fi
%
% \macro{\ifchilddoc}
% \macro{\ifchilddocmanual}
% The conditional |\ifchilddoc| tells whether a
% child (true) or main (false) document is being compiled.
% The conditional |\ifchilddocmanual| tells whether
% the |\includeonly| mechanism is used (false) or
% the selection of child files must be performed manually (true).
% The definitions initialise to false:
%    \begin{macrocode}
\newif\ifchilddoc
\newif\ifchilddocmanual
%    \end{macrocode}

% \macro{\childdocname}
% \macro{\childdocjob}
% The macro |\childdocname| stores the name of the main document
% to be compiled. The macro |\childdocjob| stores the name of
% the document on which the \LaTeX{} compiler was originally invoked.
% The content of |\jobname| cannot be compared
% to filenames specified in the source due to different catcodes.
% The following code rescans |\jobname|, stores the result
% in |\childdocname| and saves a copy in |\childdocjob|:
%    \begin{macrocode}
\edef\childdocname{\scantokens\expandafter{\jobname\noexpand}}
\let\childdocjob\childdocname
%    \end{macrocode}

% \macro{\childdocdisable}
% The macro |\childdocdisable| prevents the main file
% from being processed more than once.
% At this stage, the main document command |\childdocmain|
% is assumed to be called once again where it should do nothing.
% Any subsequent call to it should prevent
% a secondary processing of the main document
% It overwrites the forwarding commands
% |\childdocof| and |\childdocforward|
% with empty macros to prevent further inclusions of the main document:
%    \begin{macrocode}
\newcommand{\childdocdisable}
{
  \renewcommand{\childdocmain}[1]{\renewcommand{\childdocmain}[1]{\endinput}}
  \renewcommand{\childdocof}[1]{}
  \renewcommand{\childdocby}[2][]{}
  \renewcommand{\childdocforward}[2][]{}
  \renewcommand{\childdocdisable}{}
}
%    \end{macrocode}

% \macro{\childdocmain}
% The macro |\childdocmain| is to be called at the top of the main file
% with nothing or the main filename (without extension) as argument.
% First, it breaks loops.
% If the argument is not empty and does not match |\childdocname|
% (which is set by the first inclusion of |childdoc.def|),
% |\ifchilddoc| is set to true, |\includeonly| is applied to the child file
% and |\jobname| is set to the main file
% (for proper handling of |.aux| files):
%    \begin{macrocode}
\newcommand{\childdocmain}[1]
{
  \childdocdisable\childdocmain{}
  \if?#1?\else
    \begingroup
      \def\childdoctmp{#1}
      \ifx\childdoctmp\childdocname
        \def\childdoctmp{}
      \else
        \def\childdoctmp
        {
          \childdoctrue
          \includeonly{\childdocname}
          \def\childdocjob{#1}
          \def\jobname{#1}
        }
      \fi
      \expandafter
    \endgroup
    \childdoctmp
  \fi
}
%    \end{macrocode}

% \macro{\childdocof}
% The command |\childdocof| redirects
% compilation to the main file |#1|.
%    \begin{macrocode}
\newcommand{\childdocof}[1]
{
  \childdocdisable
  \childdoctrue
  \includeonly{\childdocname}
  \def\jobname{#1}
  \def\childdocjob{#1}
  \input{#1}
}
%    \end{macrocode}

% \macro{\childdocby}
% The command |\childdocby| ....
%    \begin{macrocode}
\newcommand{\childdocby}[2][]
{
  \childdocdisable
  \childdoctrue
  \childdocmanualtrue
  \if?#1?\else
    \def\jobname{#2}
  \fi
  \def\childdocjob{#2}
  \input{#2}
  \endinput
}
%    \end{macrocode}

% \macro{\childdocforward}
% The command |\childdocforward| redirects
% compilation to the main file or
% (if the optional argument is given) a child file.
% Parameters are set as if the main file
% or a child file starting with |\childdocof| was compiled.
% Then compilation is handed over to the main file:
%    \begin{macrocode}
\newcommand{\childdocforward}[2][]
{
  \begingroup
    \if?#1?
      \def\childdoctmp
      {
        \def\childdocname{#2}
        \def\childdocjob{#2}
        \def\jobname{#2}
        \input{#2}
        \endinput
      }
    \else
      \def\childdoctmp
      {
        \childdocdisable
        \def\childdocname{#2}
        \childdoctrue
        \includeonly{#2}
        \def\childdocjob{#1}
        \def\jobname{#1}
        \input{#1}
        \endinput
      }
    \fi
    \expandafter
  \endgroup
  \childdoctmp
}
%    \end{macrocode}

% \macro{\childdocforwardprefix}
% The command |\childdocforwardprefix| redirects
% compilation to the main or a child file by means of a pattern.
% The prefix |#1| in the current filename is replaced by |#2|
% and the suffix of the current filename is kept
% (it is assumed that the filename does not contain the substring `|~~~|'
% which is used as a delimiter).
% Compilation is handed over to the new file by |\childdocforward|:
%    \begin{macrocode}
\newcommand{\childdocforwardprefix}[3][]
{
  \begingroup
    \def\childdocextract #2##1~~~{\def\childdoctmp{\childdocforward[#1]{#3##1}}}
    \expandafter\childdocextract\childdocname~~~
    \expandafter
  \endgroup
  \childdoctmp
}
%    \end{macrocode}

% \macro{\childdoc}
% The deprecated macro |\childdoc| is a legacy version of |\childdocmain|:
%    \begin{macrocode}
\newcommand{\childdoc}{\childdocmain}
%    \end{macrocode}

% \macro{\childdocredirect}
% The deprecated macro |\childdocredirect| is a legacy version
% of |\childdocforward| and |\childdocforwardprefix|:
%    \begin{macrocode}
\newcommand{\childdocredirect}[2][]
{
  \begingroup
    \if?#1?
      \def\childdoctmp{\childdocforward{#2}}
    \else
      \def\childdoctmp{\childdocforwardprefix{#1}{#2}}
    \fi
    \expandafter
  \endgroup
  \childdoctmp
}
%    \end{macrocode}

%\iffalse
%</package>
%\fi
%
\endinput
\childdocforward{cdocsamp}"|\\
% |latex -jobname cdocscl1 \|\\
% |  "% \iffalse
%
% childdoc.dtx Copyright (C) 2017-2018 Niklas Beisert
%
% This work may be distributed and/or modified under the
% conditions of the LaTeX Project Public License, either version 1.3
% of this license or (at your option) any later version.
% The latest version of this license is in
%   http://www.latex-project.org/lppl.txt
% and version 1.3 or later is part of all distributions of LaTeX
% version 2005/12/01 or later.
%
% This work has the LPPL maintenance status `maintained'.
%
% The Current Maintainer of this work is Niklas Beisert.
%
% This work consists of the files childdoc.dtx and childdoc.ins
% and the derived files childdoc.def and cdocsamp.tex with
% cdocsch1.tex, cdocsch2.tex, cdocsdrf.tex, cdocsfn1.tex, cdocsfn2.tex.
%
%<package>\ifdefined\childdocmain\endinput\fi
%<package>\ProvidesFile{childdoc.def}[2018/12/30 v2.0 child document driver]
%<samplemain>\ProvidesFile{cdocsamp.tex}[2018/12/30 v2.0 sample for childdoc]
%<*driver>
%\ProvidesFile{childdoc.drv}[2018/12/30 v2.0 childdoc reference manual file]
\PassOptionsToClass{10pt,a4paper}{article}
\documentclass{ltxdoc}

\usepackage[margin=35mm]{geometry}
\usepackage{hyperref}
\usepackage{hyperxmp}
\usepackage[usenames]{color}

\hypersetup{colorlinks=true}
\hypersetup{pdfstartview=FitH}
\hypersetup{pdfpagemode=UseNone}
\hypersetup{pdfsource={}}
\hypersetup{pdflang={en-UK}}
\hypersetup{pdfcopyright={Copyright 2017-2018 Niklas Beisert.
  This work may be distributed and/or modified under the
  conditions of the LaTeX Project Public License, either version 1.3
  of this license or (at your option) any later version.}}
\hypersetup{pdflicenseurl={http://www.latex-project.org/lppl.txt}}
\hypersetup{pdfcontactaddress={ETH Zurich, ITP, HIT K,
  Wolfgang-Pauli-Strasse 27}}
\hypersetup{pdfcontactpostcode={8093}}
\hypersetup{pdfcontactcity={Zurich}}
\hypersetup{pdfcontactcountry={Switzerland}}
\hypersetup{pdfcontactemail={nbeisert@itp.phys.ethz.ch}}
\hypersetup{pdfcontacturl={http://people.phys.ethz.ch/\xmptilde nbeisert/}}

\newcommand{\secref}[1]{\hyperref[#1]{section \ref*{#1}}}

\parskip1ex
\parindent0pt
\let\olditemize\itemize
\def\itemize{\olditemize\parskip0pt}

\begin{document}

\title{The \textsf{childdoc} Package}
\hypersetup{pdftitle={The childdoc Package}}
\author{Niklas Beisert\\[2ex]
  Institut f\"ur Theoretische Physik\\
  Eidgen\"ossische Technische Hochschule Z\"urich\\
  Wolfgang-Pauli-Strasse 27, 8093 Z\"urich, Switzerland\\[1ex]
  \href{mailto:nbeisert@itp.phys.ethz.ch}
  {\texttt{nbeisert@itp.phys.ethz.ch}}}
\hypersetup{pdfauthor={Niklas Beisert}}
\hypersetup{pdfsubject={Manual for the LaTeX2e Package childdoc}}
\date{30 December 2018, \textsf{v2.0}}
\maketitle

\begin{abstract}\noindent
\textsf{childdoc} is a \LaTeXe{} package
that enables the direct compilation
of document sections included by |\include|
to individual files.
\end{abstract}

\begingroup
\parskip0ex
\tableofcontents
\endgroup

%%%%%%%%%%%%%%%%%%%%%%%%%%%%%%%%%%%%%%%%%%%%%%%%%%%%%%%%%%%%%%%%%%%%%%%%%%%%%%%%
%%%%%%%%%%%%%%%%%%%%%%%%%%%%%%%%%%%%%%%%%%%%%%%%%%%%%%%%%%%%%%%%%%%%%%%%%%%%%%%%
\section{Introduction}

\LaTeX{} provides a mechanism to structure a large document (such as a book)
into a main file and several child files (containing the chapters)
using the |\include| command.
This mechanism is beneficial for documents
which span hundreds of pages in order to
make the source file(s) more manageable.
Moreover, compilation can be restricted to
selected child files by means of the |\includeonly| command.
The latter feature can be used to reduce the compilation time while editing
(this was significantly more useful in the earlier days of \LaTeX{})
or to generate a smaller document which is easier to navigate.
Another application of |\includeonly| is to generate
documents consisting of selected parts of the complete document.

However, there are a few drawbacks of the plain |\include| mechanism:
\begin{itemize}
\item
The child files cannot be compiled on their own,
they can only be compiled via the main file.
A naive editing environment
(such as a text editor with an option
to have the current file processed by \LaTeX)
may require one to switch to the main file before compiling;
attempting to compile the child file produces errors.
\item
The main file must be modified (each time)
to adjust the |\includeonly| command
to the present needs. This easily leaves the main file in a messy state.
\item
The generated document will always carry the filename
of the main document. This is inconvenient if
several child files are to be compiled and
to be kept for distribution.
\end{itemize}

The present package provides a simple interface
to make child files individually compilable by \LaTeX{}.
Compiling a child file then has the same effect as compiling
the main file with an |\includeonly| command
to select the appropriate child.
Moreover the generated document will carry the name of the child
rather than the main file.
This resolves all three above issues.

This feature is meant to make the editing of books,
thesis documents and lecture notes somewhat more convenient.
However, the package can also be used efficiently for
composing a series of documents (such as exercise sheets)
which are typically distributed individually.
It then assists the author in generating the individual documents
(potentially in different versions)
as well as a document containing the collected series.
Another application is in developing style files
or other kinds of included material
where compilation of the style file could redirect
to a sample or test file.

%%%%%%%%%%%%%%%%%%%%%%%%%%%%%%%%%%%%%%%%%%%%%%%%%%%%%%%%%%%%%%%%%%%%%%%%%%%%%%%%
%%%%%%%%%%%%%%%%%%%%%%%%%%%%%%%%%%%%%%%%%%%%%%%%%%%%%%%%%%%%%%%%%%%%%%%%%%%%%%%%
\section{Usage}

First of all, the package \textsf{childdoc} is \emph{not} a standard
\LaTeXe{} |.sty| style file! Therefore it needs to be invoked in
a non-standard way.

%%%%%%%%%%%%%%%%%%%%%%%%%%%%%%%%%%%%%%%%%%%%%%%%%%%%%%%%%%%%%%%%%%%%%%%%%%%%%%%%
\subsection{Included Files}
\label{sec:include}

%%%%%%%%%%%%%%%%%%%%%%%%%%%%%%%%%%%%%%%%
\DescribeMacro{\childdocmain}
To use the package, add the commands
\begin{center}
\begin{tabular}{l}
|\input{childdoc.def}|\\
|\childdocmain{}|\\
\end{tabular}
\end{center}
at the very top of the main \LaTeX{} file,
in particular \emph{before} the |\documentclass| statement!
The argument of |\childdocmain| should be left empty
(but it must be present).

%%%%%%%%%%%%%%%%%%%%%%%%%%%%%%%%%%%%%%%%
\DescribeMacro{\childdocof}
Furthermore, add the commands
\begin{center}
\begin{tabular}{l}
|\input{childdoc.def}|\\
|\childdocof{|\textit{main}|}|\\
\end{tabular}
\end{center}
at the top of every child file \textit{child}
which is included by |\include{|\textit{child}|}|
from within the main file
(or at least for those files to be compiled individually).
The argument \textit{main} must be the filename of the main file.

There are a couple of
considerations in setting up the main and child documents:

%%%%%%%%%%%%%%%%%%%%%%%%%%%%%%%%%%%%%%%%
\paragraph{Restrictions.}

Please note the following restrictions:
\begin{itemize}
\item
|\childdocmain| must be called with one argument \textit{main}
to ensure compatibility with earlier version of the package.
It must either be empty (|\childdocmain{}|)
or precisely match the filename of the main file in which it is specified.
See \secref{sec:detection} for further information.
\item
The filename \textit{main} must be specified without the |.tex| extension.
\item
The filename \textit{main} is case sensitive
(even in case-insensitive file systems)
due to internal string comparison.
\item
The argument \textit{main} should be fully expanded, it cannot be a macro.
\item
Subdirectories and special characters should be avoided in filenames.
\item
The command |\childdocmain{|\textit{main}|}| must be followed by a whitespace.
It should not be followed immediately by another command
or by a comment mark `|%|'.
This is because the \TeX{} parser reads the token immediately following
the argument of |\childdocmain| and puts it
at the beginning of every child section;
however, a white\-space is ignored.
\end{itemize}

%%%%%%%%%%%%%%%%%%%%%%%%%%%%%%%%%%%%%%%%
\paragraph{Content of Main File.}

It is advisable to place all content in the child files included by |\include|.
Any output contained in the main file will appear in all child documents
unless suppressed manually;
it cannot be suppressed automatically by the |\includeonly| directive
and thus should normally be avoided.
A method to include some content in the main file
by means of conditional processing is described in \secref{sec:conditional}.

%%%%%%%%%%%%%%%%%%%%%%%%%%%%%%%%%%%%%%%%
\paragraph{Page Numbering.}

When only a part of the document is compiled,
the appropriate numbering of pages
(as well as other status parameters)
is determined from the |.aux| files.
The latter contain information from previous passes.
However this information needs to propagate through
all intermediate child documents.
Therefore the page numbering in child documents may well
be inconsistent until the complete document is compiled at least once.

A useful (if unconventional) way to always ensure a consistent
page numbering is to restart the numbering in each child document
and denote the pages by `\textit{child}|.|\textit{page}'
where \textit{child} represents the chapter/section number of the child file.
This can be achieved by the command
|\numberwithin{page}{|\textit{child}|}|
of the \textsf{amsmath} package
where \textit{child} can be |chapter| or |section|
depending on the chosen structuring.
Alternatively, one can modify the macro |\thepage| appropriately
and reset the counter |page| at the start of each child file.

%%%%%%%%%%%%%%%%%%%%%%%%%%%%%%%%%%%%%%%%%%%%%%%%%%%%%%%%%%%%%%%%%%%%%%%%%%%%%%%%
\subsection{Conditional Processing}
\label{sec:conditional}

The package provides a mechanism to compile different versions
of a document. To customise the versions further some conditional processing
can come in handy to distinguish which version is being compiled.
The package provides two macros to describe the compilation context:

%%%%%%%%%%%%%%%%%%%%%%%%%%%%%%%%%%%%%%%%
\DescribeMacro{\ifchilddoc}
The conditional |\ifchilddoc| distinguishes between the compilation of
child documents and the main document:
%
\begin{center}
|\ifchilddoc |\textit{child-code}| |[|\||else |\textit{main-code}]| \||fi|
\end{center}

%%%%%%%%%%%%%%%%%%%%%%%%%%%%%%%%%%%%%%%%
\DescribeMacro{\childdocname}
\DescribeMacro{\childdocjob}
The macro |\childdocname| contains the filename (without extension)
of the main or child file being processed.
Note that |\childdocjob| will always contain the name of the main file.

%%%%%%%%%%%%%%%%%%%%%%%%%%%%%%%%%%%%%%%%
\paragraph{Title Page.}

Conditional processing can be used to include a title or banner page
in the main document when proper precautions are taken.
Importantly, the code in the main file should ensure that the page counter
(as well as other status parameters which are stored in the |.aux| files)
takes the same value after the conditional processing.
Otherwise the page numbers may take divergent values
depending on which part is compiled.

For example, a title page could be declared by:
%
\begin{center}
\begin{tabular}{l}
|\ifchilddoc\||else|\\
|\addtocounter{page}{-1}|\\
\textit{code for title page}\\
|\newpage|\\
|\||fi|
\end{tabular}
\end{center}
%
A banner page for the child documents can be generated by:
%
\begin{center}
\begin{tabular}{l}
|\ifchilddoc|\\
|\addtocounter{page}{-1}|\\
\textit{code for banner page}\\
|\newpage|\\
|\||fi|
\end{tabular}
\end{center}
%
Here one could write a message such as:
\begin{center}
|This is the part \childdocname{} of \childdocjob{}.|
\end{center}

%%%%%%%%%%%%%%%%%%%%%%%%%%%%%%%%%%%%%%%%%%%%%%%%%%%%%%%%%%%%%%%%%%%%%%%%%%%%%%%%
\subsection{Flags}
\label{sec:flags}

The package makes it easy to generate different versions
of the main or child documents.
To this end compilation flags can be defined
and assigned different default values.
They will be particularly useful in conjunction
with the forwarding mechanism described in \secref{sec:forward}.

For example, it may be useful to have a flag |\version|
which can be set to |draft| or |final|.
The document source will contain some conditional code
depending on the value of |\version|.
Suppose further, the flag should default to |final| for the main file
and to |draft| for child files
which is a natural assignment for editing the document.
This is achieved by placing the following code
in the preamble of the main document
(below the |\childdocmain| directive):
%
\begin{center}
\begin{tabular}{l}
|\ifchilddoc|\\
|\providecommand{\version}{draft}|\\
|\||else|\\
|\providecommand{\version}{final}|\\
|\||fi|
\end{tabular}
\end{center}
%
The definition by |\providecommand| makes sure
that previous definitions are not overwritten.
Further statements |\providecommand{\version}{...}|
can thus be added before the above code to override it.

For the main file, one might add a line
(between |\childdocmain| and the above block)
%
\begin{center}
|%\ifchilddoc\||else\providecommand{\version}{draft}\||fi|
\end{center}
%
which can be uncommented to produce a draft version.
Likewise one can add a line to the very top of a child file
(above the |\childdocof{|\textit{main}|}| directive)
%
\begin{center}
|%\providecommand{\version}{final}|
\end{center}
%
which can be uncommented to produce the final version of this child document.

%%%%%%%%%%%%%%%%%%%%%%%%%%%%%%%%%%%%%%%%%%%%%%%%%%%%%%%%%%%%%%%%%%%%%%%%%%%%%%%%
\subsection{Forwarding}
\label{sec:forward}

Different versions of the main or child documents
using compilation flags as described in \secref{sec:flags}
can be (permanently) stored in different files
for convenient compilation, viewing and distribution.
To this end, the package defines a command
to pass on compilation to a different file:

%%%%%%%%%%%%%%%%%%%%%%%%%%%%%%%%%%%%%%%%
\DescribeMacro{\childdocforward}
The command |\childdocforward| redirects processing to
another source file:
%
\begin{center}
\begin{tabular}{l}
|\input{childdoc.def}|\\
|\childdocforward[|\textit{main}|]{|\textit{dest}|}|\\
\end{tabular}
\end{center}
%
The argument \textit{dest} is the destination file
(without extension).
It should be the main file or one of the child files.
Note that further \textsf{childdoc} directives
such as |\childdocof| and |\childdocforward|
in the indicated file will be processed in this form.
The optional argument \textit{main}
passes on directly to the main file \textit{main}
while pretending to compile the child \textit{dest}.
This form behaves as if \textit{dest}
issues |\childdocof{|\textit{main}|}| right away,
and no further \textsf{childdoc} directives will be processed.

%%%%%%%%%%%%%%%%%%%%%%%%%%%%%%%%%%%%%%%%
\DescribeMacro{\...prefix}
In the alternative form |\childdocforwardprefix|,
%
\begin{center}
\begin{tabular}{l}
|\input{childdoc.def}|\\
|\childdocforwardprefix[|\textit{main}|]{|\textit{prefix}|}{|\textit{dest}|}|
\end{tabular}
\end{center}
%
the destination file is determined by a pattern
depending on the current file:
To make this work, the current file must be called
`{\textit{prefix}\hspace{0.2em}\textit{suffix}}'
with \textit{prefix} matching precisely the argument.
Processing is then passed on to the file
`{\textit{dest}\hspace{0.2em}\textit{suffix}}'.
Surely, the same effect is achieved by
directly specifying the
argument `{\textit{dest}\hspace{0.2em}\textit{suffix}}'
in the first form.
However, that requires to set up a different file
for each child. With the alternative form of the command
all these files can have exactly the same content
which simplifies setting them up and maintaining them.

For example, the following file |draft.tex|
with a compilation flag |\version| as described in \secref{sec:flags}
compiles the main document as a draft:
%
\begin{center}
\begin{tabular}{l}
|\def\version{draft}|\\
|\input{childdoc.def}|\\
|\childdocforward{|\textit{main}|}|
\end{tabular}
\end{center}
%
Likewise, the following files |final|\textit{nn}|.tex|
compile the final version of the child document
|child|\textit{nn}|.tex|:
%
\begin{center}
\begin{tabular}{l}
|\def\version{final}|\\
|\input{childdoc.def}|\\
|\childdocforwardprefix{final}{child}|
\end{tabular}
\end{center}
%

Note that when several versions of a main file and/or of each child file
are to be generated, it may be convenient to set up a |Makefile| or
shell script to automatise the process.

%%%%%%%%%%%%%%%%%%%%%%%%%%%%%%%%%%%%%%%%%%%%%%%%%%%%%%%%%%%%%%%%%%%%%%%%%%%%%%%%
\subsection{Command Line Processing}
\label{sec:commandline}

The effect of redirection files can also be achieved by invoking
the \LaTeX{} compiler with a more elaborate command line.
Most conveniently this should be done as part
of a shell script or a |Makefile|.

When using \textsf{childdoc} in the main file, the following
command lines effectively perform a redirection
(note that depending on the shell being used,
backslashes may have to be doubled: `|\|' $\to$ `|\\|'):
%
\begin{center}
|... -jobname "|\textit{target}|" |\\|"|[\textit{flags}]%
|\input{childdoc.def}\childdocforward[|\textit{main}|]{|\textit{dest}|}"|
\end{center}
%
Here \textit{target} is the name of the output file,
\textit{main} is the name of the main file
and \textit{dest} is the name of the main or child file to be processed
(all filenames without extensions).
The optional argument \textit{main} can be omitted
if \textit{main} matches \textit{dest}.
Optionally, compilation \textit{flags} can be defined via |\def| commands.
This command line makes the \TeX{} engine believe
it is compiling the file \textit{target}
whose content is specified as the latter parameter.
The provided code then forwards the processing to
\textit{main} or \textit{dest} as described in \secref{sec:forward}.

%%%%%%%%%%%%%%%%%%%%%%%%%%%%%%%%%%%%%%%%%%%%%%%%%%%%%%%%%%%%%%%%%%%%%%%%%%%%%%%%
\subsection{Include by Input}
\label{sec:input}

Including child documents by |\include| has some restrictions by design.
Most notably, the content of a child document always occupies
its own set of pages; pages cannot be shared between child documents.
Usually, this behaviour makes perfect sense
because each child document contain an essential part of the document.
However, in some situations it may be desirable to compose
a document from a collection of parts
without having mandatory page breaks between then.
For this case, the package
provides a mechanism to include parts
by |\input| which can also be processed individually.
However, by construction this mechanism
requires manual handling of the content to be output.

%%%%%%%%%%%%%%%%%%%%%%%%%%%%%%%%%%%%%%%%
\DescribeMacro{\ifchilddocmanual}
The main file should be prepared as usual, see \secref{sec:include}.
However, the document body must make a distinction
between processing of an individual part and of the main document, e.g.:
%
\begin{center}
\begin{tabular}{l}
|\ifchilddocmanual|\\
|\input{\childdocname}|\\
|\||else|\\
\textit{document body with }|\input{|\textit{part}|}|\\
|\||fi|
\end{tabular}
\end{center}
%
The conditional |\ifchilddocmanual| is true whenever
a part to be included by |\input| is being compiled,
and the name of the part is stored in |\childdocname|.

%%%%%%%%%%%%%%%%%%%%%%%%%%%%%%%%%%%%%%%%
\DescribeMacro{\childdocby}
Each part to be included by |\input| should start with:
%
\begin{center}
\begin{tabular}{l}
|\input{childdoc.def}|\\
|\childdocby{|\textit{main}|}|\\
\end{tabular}
\end{center}
%
The directive |\childdocby| is similar to |\childdocof|
described in \secref{sec:include},
but the subsequent selection of content must be done manually.
To that end, both |\ifchilddoc| and |\ifchilddocmanual|
will be true upon processing of a part,
and the name of the part is stored in |\childdocname|.
Note that |\jobname| will be set to the filename of the current part
so that each part receives an individual |.aux| file
that does not interfere with the |.aux| file(s) of the main document.
This behaviour can be altered by the alternative form
|\childdocby[*]{|\textit{main}|}| (with a non-empty optional argument)
which uses the |.aux| file of the main document
by setting |\jobname| to \textit{main}.

%%%%%%%%%%%%%%%%%%%%%%%%%%%%%%%%%%%%%%%%%%%%%%%%%%%%%%%%%%%%%%%%%%%%%%%%%%%%%%%%
\subsection{Driver Development}
\label{sec:driver}

The \textsf{childdoc} mechanism can also be use for the development
of definition files such as \LaTeX{} styles or classes.
This case differs from the above setup with multiple parts
included by |\include| in that no |\includeonly| should be invoked.
This can be achieved by starting the include file
(before |\ProvidesPackage|) with:
%
\begin{center}
\begin{tabular}{l}
|\input{childdoc.def}|\\
|\childdocforward{|\textit{main}|}|\\
\end{tabular}
\end{center}
%
or alternatively with:
%
\begin{center}
\begin{tabular}{l}
|\input{childdoc.def}|\\
|\childdocby{|\textit{main}|}|\\
\end{tabular}
\end{center}
%
Both forms have slightly different effects as described above.
The main file is prepared as usual, see \secref{sec:include}.

%%%%%%%%%%%%%%%%%%%%%%%%%%%%%%%%%%%%%%%%%%%%%%%%%%%%%%%%%%%%%%%%%%%%%%%%%%%%%%%%
\subsection{Legacy Detection}
\label{sec:detection}

The directive |\childdocmain| in the main file can detect
whether the complete document or merely a child is to be compiled
even without using the directive |\childdocof|.
This method is deprecated because it is less robust
and there is no compelling reason to use it;
it is merely provided for backward compatibility
and it may be removed in future versions.

If the detection mechanism is to be used,
it is mandatory to correctly specify
the filename of the main file as the argument of |\childdocmain|:
%
\begin{center}
\begin{tabular}{l}
|\input{childdoc.def}|\\
|\childdocmain{|\textit{main}|}|\\
\end{tabular}
\end{center}
%
If |\jobname| does not match the argument \textit{main} of |\childdocmain|,
it is assumed that |\jobname| points to the child file to be compiled.
When using |\childdocmain| with the main file specified as argument,
it suffices to start a child file
with just |\input{|\textit{main}|}|
without loading of the package and using |\childdocof|.
If instead all processing is done
with the appropriate \textsf{childdoc} directives,
the argument of \textit{main} of |\childdocmain| can be empty.

An alternative version of the command line processing described
in \secref{sec:commandline} using the detection mechanism reads:
%
\begin{center}
|... -jobname "|\textit{target}|" "|[\textit{flags}]%
[|\def\jobname{|\textit{dest}|}|]|\input{|\textit{main}|}"|
\end{center}

%%%%%%%%%%%%%%%%%%%%%%%%%%%%%%%%%%%%%%%%%%%%%%%%%%%%%%%%%%%%%%%%%%%%%%%%%%%%%%%%
\subsection{Manual Code}
\label{sec:manual}

In case one cannot be certain whether the definitions file |childdoc.def|
is installed on the target \TeX{} distribution
and one prefers not to ship it,
it is conceivable to paste a few relevant commands into the sources.

To that end, drop all statements |\input{childdoc.def}|
and perform the replacements as outlined below.
Instead of |\childdocmain{|\textit{main}|}| add the following code
to the top of the main file:
%
\begin{center}
\begin{tabular}{l}
|\||ifdefined\childdocname\endinput\||fi\newif\ifchilddoc|\\
|\edef\childdocname{\scantokens\expandafter{\jobname\noexpand}}|\\
|\def\childdocmain{|\textit{main}|}\||ifx\childdocmain\childdocname\||else|\\
|\childdoctrue\includeonly{\childdocname}\let\jobname\childdocmain\||fi|\\
\end{tabular}
\end{center}
%
Instead of |\childdocof{|\textit{main}|}| just include the main file
at the top of each child file:
%
\begin{center}
|\input{|\textit{main}|}|
\end{center}
%
A simple redirection |\childdocforward{|\textit{dest}|}| is achieved by:
%
\begin{center}
|\def\jobname{|\textit{dest}|}\input{\jobname}|
\end{center}
%
The redirection with prefix
|\childdocforwardprefix[|\textit{prefix}|]{|\textit{dest}|}|
is accomplished by:
%
\begin{center}
\begin{tabular}{l}
|{\edef\jobname{\scantokens\expandafter{\jobname\noexpand}}|\\
|\def\redirectjob |\textit{prefix}|#1~~~{\gdef\jobname{|\textit{dest}|#1}}|\\
|\expandafter\redirectjob\jobname~~~}\input{\jobname}|
\end{tabular}
\end{center}

In an alternative approach,
child documents can be compiled by a specific command line
without additional code or specific definitions:
%
\begin{center}
|... -jobname "|\textit{target}|" "|[\textit{flags}]%
|\includeonly{|\textit{dest}|}\input{|\textit{main}|}"|
\end{center}
%

%%%%%%%%%%%%%%%%%%%%%%%%%%%%%%%%%%%%%%%%%%%%%%%%%%%%%%%%%%%%%%%%%%%%%%%%%%%%%%%%
%%%%%%%%%%%%%%%%%%%%%%%%%%%%%%%%%%%%%%%%%%%%%%%%%%%%%%%%%%%%%%%%%%%%%%%%%%%%%%%%
\section{Information}

%%%%%%%%%%%%%%%%%%%%%%%%%%%%%%%%%%%%%%%%%%%%%%%%%%%%%%%%%%%%%%%%%%%%%%%%%%%%%%%%
\subsection{Copyright}

Copyright \copyright{} 2017--2018 Niklas Beisert

This work may be distributed and/or modified under the
conditions of the \LaTeX{} Project Public License, either version 1.3
of this license or (at your option) any later version.
The latest version of this license is in
  \url{http://www.latex-project.org/lppl.txt}
and version 1.3 or later is part of all distributions of \LaTeX{}
version 2005/12/01 or later.

This work has the LPPL maintenance status `maintained'.

The Current Maintainer of this work is Niklas Beisert.

This work consists of the files |README.txt|, |childdoc.ins| and |childdoc.dtx|
as well as the derived files |childdoc.def|, |cdocsamp.tex|
with |cdocsch1.tex|, |cdocsch2.tex|, |cdocspt3.tex|, |cdocspt4.tex|,
|cdocsdrf.tex|, |cdocsfn1.tex|, |cdocsfn2.tex|
as well as |childdoc.pdf|.

%%%%%%%%%%%%%%%%%%%%%%%%%%%%%%%%%%%%%%%%%%%%%%%%%%%%%%%%%%%%%%%%%%%%%%%%%%%%%%%%
\subsection{Files and Installation}

The package consists of the files:
%
\begin{center}
\begin{tabular}{ll}
    |README.txt|   & readme file \\
    |childdoc.ins| & installation file \\
    |childdoc.dtx| & source file \\
    |childdoc.def| & definition file \\
    |cdocsamp.tex| & sample main file \\
    |cdocsch1.tex| & sample include file \\
    |cdocsch2.tex| & sample include file \\
    |cdocspt3.tex| & sample part file \\
    |cdocspt4.tex| & sample part file \\
    |cdocsdrf.tex| & sample redirection file \\
    |cdocsfn1.tex| & sample redirection file \\
    |cdocsfn2.tex| & sample redirection file \\
    |childdoc.pdf| & manual
\end{tabular}
\end{center}
%
The distribution consists of the files
|README.txt|, |childdoc.ins| and |childdoc.dtx|.
%
\begin{itemize}
\item
Run (pdf)\LaTeX{} on |childdoc.dtx|
to compile the manual |childdoc.pdf| (this file).
\item
Run \LaTeX{} on |childdoc.ins| to create the definitions file |childdoc.def|
and the sample |cdocsamp.tex| with include files
|cdocsch1.tex|, |cdocsch2.tex|, |cdocspt3.tex|, |cdocspt4.tex|,
|cdocsdrf.tex|, |cdocsfn1.tex|, |cdocsfn2.tex|.
Then copy the file |childdoc.def| to an appropriate directory of your \LaTeX{}
distribution, e.g.\ \textit{texmf-root}|/tex/latex/childdoc|.
\end{itemize}

%%%%%%%%%%%%%%%%%%%%%%%%%%%%%%%%%%%%%%%%%%%%%%%%%%%%%%%%%%%%%%%%%%%%%%%%%%%%%%%%
\subsection{Related CTAN Packages}

There are several other packages which offer a similar functionality:
%
\begin{itemize}
\item
The packages
\href{http://ctan.org/pkg/docmute}{\textsf{docmute}},
\href{http://ctan.org/pkg/includex}{\textsf{includex}} and
\href{http://ctan.org/pkg/standalone}{\textsf{standalone}}
provide commands to include only the document body of
a child file thus allowing both files to be compiled individually.
\item
The packages \href{http://ctan.org/pkg/subdocs}{\textsf{subdocs}}
and \href{http://ctan.org/pkg/subfiles}{\textsf{subfiles}}
provide structures in which the main and child documents can be
encapsulated and allowing them to be compiled individually.
The inclusion mechanism is different from the conventional |\include|.
\item
The package \href{http://ctan.org/pkg/combine}{\textsf{combine}}
is an elaborate solution to combine several documents into one.
\end{itemize}
%
See also the CTAN topic \href{http://ctan.org/topic/subdocs}{\textsf{subdocs}}
for further related packages.
The present package differs from the above solutions in that
a document structure constructed with the conventional |\include| mechanism
just needs two extra commands at the top of every file
such that all constituent files can be compiled individually.

%%%%%%%%%%%%%%%%%%%%%%%%%%%%%%%%%%%%%%%%%%%%%%%%%%%%%%%%%%%%%%%%%%%%%%%%%%%%%%%%
%\subsection{Feature Suggestions}
%
%The following is a list of features which may be useful for future
%versions of this package:
%%
%\begin{itemize}
%\item
%\ldots
%\end{itemize}

%%%%%%%%%%%%%%%%%%%%%%%%%%%%%%%%%%%%%%%%%%%%%%%%%%%%%%%%%%%%%%%%%%%%%%%%%%%%%%%%
\subsection{Revision History}

%%%%%%%%%%%%%%%%%%%%%%%%%%%%%%%%%%%%%%%%
\paragraph{v2.0:} 2018/12/30

\begin{itemize}
\item
immediate forward processing
\item
added |\childdocby| mechanism
\item
manual restructured
\end{itemize}

%%%%%%%%%%%%%%%%%%%%%%%%%%%%%%%%%%%%%%%%
\paragraph{v1.6:} 2018/01/17

\begin{itemize}
\item
application for development of include files
\item
corrections to manual
\end{itemize}

%%%%%%%%%%%%%%%%%%%%%%%%%%%%%%%%%%%%%%%%
\paragraph{v1.5:} 2017/05/21

\begin{itemize}
\item
more complete structuring introduced
\item
|\childdocof| introduced
\item
|\childdoc| renamed to |\childdocmain|
\item
|\childredirect| renamed to |\childdocforward| and |\childdocforwardprefix|
and functionality expanded
\end{itemize}

%%%%%%%%%%%%%%%%%%%%%%%%%%%%%%%%%%%%%%%%
\paragraph{v1.0:} 2017/04/27

\begin{itemize}
\item
manual and install package
\item
first version published on CTAN
\end{itemize}

%%%%%%%%%%%%%%%%%%%%%%%%%%%%%%%%%%%%%%%%
\paragraph{v0.6:} 2017/04/26

\begin{itemize}
\item
redirection mechanism added
\end{itemize}

%%%%%%%%%%%%%%%%%%%%%%%%%%%%%%%%%%%%%%%%
\paragraph{v0.5:} 2017/04/26

\begin{itemize}
\item
functionality in definition file
\end{itemize}


%%%%%%%%%%%%%%%%%%%%%%%%%%%%%%%%%%%%%%%%%%%%%%%%%%%%%%%%%%%%%%%%%%%%%%%%%%%%%%%%
%%%%%%%%%%%%%%%%%%%%%%%%%%%%%%%%%%%%%%%%%%%%%%%%%%%%%%%%%%%%%%%%%%%%%%%%%%%%%%%%
%%%%%%%%%%%%%%%%%%%%%%%%%%%%%%%%%%%%%%%%%%%%%%%%%%%%%%%%%%%%%%%%%%%%%%%%%%%%%%%%
\appendix

\settowidth\MacroIndent{\rmfamily\scriptsize 000\ }

 \DocInput{childdoc.dtx}

\end{document}
%</driver>
% \fi
%
% %%%%%%%%%%%%%%%%%%%%%%%%%%%%%%%%%%%%%%%%%%%%%%%%%%%%%%%%%%%%%%%%%%%%%%%%%%%%%%
% %%%%%%%%%%%%%%%%%%%%%%%%%%%%%%%%%%%%%%%%%%%%%%%%%%%%%%%%%%%%%%%%%%%%%%%%%%%%%%
% \section{Sample}
%\iffalse
%<*samplemain>
%\fi
%
% The following presents a sample document
% with two chapters, two parts, a title page,
% a compile flag as well as three forwarding files to set the flag.
% It consists of eight |.tex| files:
% \begin{center}
% \begin{tabular}{ll}
% |cdocsamp.tex|&main file\\
% |cdocsch1.tex|&include file for chapter 1\\
% |cdocsch2.tex|&include file for chapter 2\\
% |cdocspt3.tex|&include file for part 3\\
% |cdocspt4.tex|&include file for part 4\\
% |cdocsdrf.tex|&forwarding file for main file in draft mode\\
% |cdocsfi1.tex|&forwarding file for final version of chapter 1\\
% |cdocsfi2.tex|&forwarding file for final version of chapter 2\\
% \end{tabular}
% \end{center}
% Each of the eight files can be compiled directly by the \LaTeX{} compiler.
%
% %%%%%%%%%%%%%%%%%%%%%%%%%%%%%%%%%%%%%%
% \paragraph{Main File.}
%
% The main file is called |cdocsamp.tex|.
%
% Load the \textsf{childdoc} definitions and
% declare the filename for the main document:
%    \begin{macrocode}
\input{childdoc.def}
\childdocmain{}
%    \end{macrocode}

% Optional override for |\version| flag:
%    \begin{macrocode}
%%\ifchilddoc\else\providecommand{\version}{draft}\fi
%    \end{macrocode}

% Define the default values for the |\version| flag
% (|final| for the main file and |draft| for childs):
%    \begin{macrocode}
\ifchilddoc
\providecommand{\version}{draft}
\else
\providecommand{\version}{final}
\fi
%    \end{macrocode}

% Load the standard document class:
%    \begin{macrocode}
\documentclass[12pt]{article}
%    \end{macrocode}

% Start the document body:
%    \begin{macrocode}
\begin{document}
%    \end{macrocode}

% Declare a title page.
% Print title, part of document being processed and version flag:
%    \begin{macrocode}
\addtocounter{page}{-1}
\begin{center}
{\LARGE\bfseries{}childdoc example\par}
\vspace{1cm}
\ifchilddoc
\ifchilddocmanual part\else chapter\fi:
`\childdocname' of `\childdocjob'\par
\else
main document: `\childdocjob'\par
\fi
version: \version\par
\end{center}
\newpage
%    \end{macrocode}

% Manually include selected file,
% otherwise process as usual:
%    \begin{macrocode}
\ifchilddocmanual
\section*{part `\childdocname'}
\input{\childdocname}
\else
%    \end{macrocode}

% Include the two chapters:
%    \begin{macrocode}
\include{cdocsch1}
\include{cdocsch2}
%    \end{macrocode}

% Include the two parts unless only chapters should be displayed:
%    \begin{macrocode}
\ifchilddoc\else
\section{part three}
\input{cdocspt3}
\section{part four}
\input{cdocspt4}
\fi
%    \end{macrocode}

% Process as usual until here:
%    \begin{macrocode}
\fi
%    \end{macrocode}

% End of document body:
%    \begin{macrocode}
\end{document}
%    \end{macrocode}
%\iffalse
%</samplemain>
%\fi
%
% %%%%%%%%%%%%%%%%%%%%%%%%%%%%%%%%%%%%%%
% \paragraph{Chapter Include Files.}
%
% The include files are called |cdocsch1.tex| and |cdocsch2.tex|.
%
%\iffalse
%<*samplechap1|samplechap2>
%\fi

% Optional override for |\version| flag:
%    \begin{macrocode}
%%\providecommand{\version}{final}
%    \end{macrocode}

% Include the main document:
%    \begin{macrocode}
\input{childdoc.def}
\childdocof{cdocsamp}
%    \end{macrocode}

%\iffalse
%</samplechap1|samplechap2>
%\fi
%
%\iffalse
%<*samplechap1>
%\fi
% Some text for chapter 1:
%    \begin{macrocode}
\section{one}
some text in chapter one
%    \end{macrocode}

%\iffalse
%</samplechap1>
%\fi
% Some text for chapter 2:
%\iffalse
%<*samplechap2>
%\fi
%    \begin{macrocode}
\section{two}
more text in chapter two
%    \end{macrocode}

%\iffalse
%</samplechap2>
%\fi
%
% %%%%%%%%%%%%%%%%%%%%%%%%%%%%%%%%%%%%%%
% \paragraph{Part Include Files.}
%
% The include files are called |cdocspt3.tex| and |cdocspt4.tex|.
%
%\iffalse
%<*samplepart3|samplepart4>
%\fi

% Optional override for |\version| flag:
%    \begin{macrocode}
%%\providecommand{\version}{final}
%    \end{macrocode}

% Include the main document:
%    \begin{macrocode}
\input{childdoc.def}
\childdocby{cdocsamp}
%    \end{macrocode}

%\iffalse
%</samplepart3|samplepart4>
%\fi
%
%\iffalse
%<*samplepart3>
%\fi
% Some text for part 3:
%    \begin{macrocode}
some text in part three
%    \end{macrocode}

%\iffalse
%</samplepart3>
%\fi
% Some text for part 4:
%\iffalse
%<*samplepart4>
%\fi
%    \begin{macrocode}
more text in part four
%    \end{macrocode}

%\iffalse
%</samplepart4>
%\fi
%
% %%%%%%%%%%%%%%%%%%%%%%%%%%%%%%%%%%%%%%
% \paragraph{Forwarding for a Complete Draft.}
%
% The following forwarding file |cdocsdrf.tex|
% compiles the main document in draft mode:
%\iffalse
%<*sampledraft>
%\fi
%    \begin{macrocode}
\def\version{draft}
\input{childdoc.def}
\childdocforward{cdocsamp}
%    \end{macrocode}

%\iffalse
%</sampledraft>
%\fi
%
% %%%%%%%%%%%%%%%%%%%%%%%%%%%%%%%%%%%%%%
% \paragraph{Forwarding for Final Version of the Chapters.}
%
% The following forwarding files |cdocsfn1.tex| and |cdocsfn2.tex|
% (with identical content)
% compile the final versions of the child documents
% |cdocsch1.tex| and |cdocsch2.tex|, respectively:
%\iffalse
%<*samplefinal>
%\fi
%    \begin{macrocode}
\def\version{final}
\input{childdoc.def}
\childdocforwardprefix[cdocsamp]{cdocsfn}{cdocsch}
%    \end{macrocode}

%\iffalse
%</samplefinal>
%\fi
%
% %%%%%%%%%%%%%%%%%%%%%%%%%%%%%%%%%%%%%%
% \paragraph{Command Line Processing.}
%
% The following three command lines generate the output files
% |cdocscld|, |cdocscl1| and |cdocscl2|
% which should be identical to
% |cdocsdrf|, |cdocsch1| and |cdocsfn2|, respectively:
% \begin{center}
% \begin{tabular}{l}
% |latex -jobname cdocscld \|\\
% |  "\def\version{draft}\input{childdoc.def}\childdocforward{cdocsamp}"|\\
% |latex -jobname cdocscl1 \|\\
% |  "\input{childdoc.def}\childdocforward[cdocsamp]{cdocsch1}"|\\
% |latex -jobname cdocscl2 \|\\
% |  "\def\version{final}\input{childdoc.def}\childdocforward{cdocsch2}"|
% \end{tabular}
% \end{center}
% Note that the trailing backslash on each first line
% merely continues the input to the second line
% (for convenient cut ant paste).
% Furthermore, the command |latex| can be replaced by any
% of its alternative versions such as |pdflatex|.
%
% %%%%%%%%%%%%%%%%%%%%%%%%%%%%%%%%%%%%%%%%%%%%%%%%%%%%%%%%%%%%%%%%%%%%%%%%%%%%%%
% %%%%%%%%%%%%%%%%%%%%%%%%%%%%%%%%%%%%%%%%%%%%%%%%%%%%%%%%%%%%%%%%%%%%%%%%%%%%%%
% \section{Implementation}
%\iffalse
%<*package>
%\fi
%
% This section describes the definitions file |childdoc.def|.

% The definitions cannot be loaded using |\usepackage| or |\RequirePackage|
% which has a mechanism to prevent loading a style file more than once.
% When loading the definitions by means of |\input|
% multiple instances have to be prevented manually:
%\iffalse
%This code needs to be before the `\ProvidesFile' directive
%which is defined at the beginning of this file.
%Therefore it is also placed there and commented out here.
%</package>
%<*discard>
%\fi
%    \begin{macrocode}
\ifdefined\childdocmain\endinput\fi
%    \end{macrocode}
%\iffalse
%</discard>
%<*package>
%\fi
%
% \macro{\ifchilddoc}
% \macro{\ifchilddocmanual}
% The conditional |\ifchilddoc| tells whether a
% child (true) or main (false) document is being compiled.
% The conditional |\ifchilddocmanual| tells whether
% the |\includeonly| mechanism is used (false) or
% the selection of child files must be performed manually (true).
% The definitions initialise to false:
%    \begin{macrocode}
\newif\ifchilddoc
\newif\ifchilddocmanual
%    \end{macrocode}

% \macro{\childdocname}
% \macro{\childdocjob}
% The macro |\childdocname| stores the name of the main document
% to be compiled. The macro |\childdocjob| stores the name of
% the document on which the \LaTeX{} compiler was originally invoked.
% The content of |\jobname| cannot be compared
% to filenames specified in the source due to different catcodes.
% The following code rescans |\jobname|, stores the result
% in |\childdocname| and saves a copy in |\childdocjob|:
%    \begin{macrocode}
\edef\childdocname{\scantokens\expandafter{\jobname\noexpand}}
\let\childdocjob\childdocname
%    \end{macrocode}

% \macro{\childdocdisable}
% The macro |\childdocdisable| prevents the main file
% from being processed more than once.
% At this stage, the main document command |\childdocmain|
% is assumed to be called once again where it should do nothing.
% Any subsequent call to it should prevent
% a secondary processing of the main document
% It overwrites the forwarding commands
% |\childdocof| and |\childdocforward|
% with empty macros to prevent further inclusions of the main document:
%    \begin{macrocode}
\newcommand{\childdocdisable}
{
  \renewcommand{\childdocmain}[1]{\renewcommand{\childdocmain}[1]{\endinput}}
  \renewcommand{\childdocof}[1]{}
  \renewcommand{\childdocby}[2][]{}
  \renewcommand{\childdocforward}[2][]{}
  \renewcommand{\childdocdisable}{}
}
%    \end{macrocode}

% \macro{\childdocmain}
% The macro |\childdocmain| is to be called at the top of the main file
% with nothing or the main filename (without extension) as argument.
% First, it breaks loops.
% If the argument is not empty and does not match |\childdocname|
% (which is set by the first inclusion of |childdoc.def|),
% |\ifchilddoc| is set to true, |\includeonly| is applied to the child file
% and |\jobname| is set to the main file
% (for proper handling of |.aux| files):
%    \begin{macrocode}
\newcommand{\childdocmain}[1]
{
  \childdocdisable\childdocmain{}
  \if?#1?\else
    \begingroup
      \def\childdoctmp{#1}
      \ifx\childdoctmp\childdocname
        \def\childdoctmp{}
      \else
        \def\childdoctmp
        {
          \childdoctrue
          \includeonly{\childdocname}
          \def\childdocjob{#1}
          \def\jobname{#1}
        }
      \fi
      \expandafter
    \endgroup
    \childdoctmp
  \fi
}
%    \end{macrocode}

% \macro{\childdocof}
% The command |\childdocof| redirects
% compilation to the main file |#1|.
%    \begin{macrocode}
\newcommand{\childdocof}[1]
{
  \childdocdisable
  \childdoctrue
  \includeonly{\childdocname}
  \def\jobname{#1}
  \def\childdocjob{#1}
  \input{#1}
}
%    \end{macrocode}

% \macro{\childdocby}
% The command |\childdocby| ....
%    \begin{macrocode}
\newcommand{\childdocby}[2][]
{
  \childdocdisable
  \childdoctrue
  \childdocmanualtrue
  \if?#1?\else
    \def\jobname{#2}
  \fi
  \def\childdocjob{#2}
  \input{#2}
  \endinput
}
%    \end{macrocode}

% \macro{\childdocforward}
% The command |\childdocforward| redirects
% compilation to the main file or
% (if the optional argument is given) a child file.
% Parameters are set as if the main file
% or a child file starting with |\childdocof| was compiled.
% Then compilation is handed over to the main file:
%    \begin{macrocode}
\newcommand{\childdocforward}[2][]
{
  \begingroup
    \if?#1?
      \def\childdoctmp
      {
        \def\childdocname{#2}
        \def\childdocjob{#2}
        \def\jobname{#2}
        \input{#2}
        \endinput
      }
    \else
      \def\childdoctmp
      {
        \childdocdisable
        \def\childdocname{#2}
        \childdoctrue
        \includeonly{#2}
        \def\childdocjob{#1}
        \def\jobname{#1}
        \input{#1}
        \endinput
      }
    \fi
    \expandafter
  \endgroup
  \childdoctmp
}
%    \end{macrocode}

% \macro{\childdocforwardprefix}
% The command |\childdocforwardprefix| redirects
% compilation to the main or a child file by means of a pattern.
% The prefix |#1| in the current filename is replaced by |#2|
% and the suffix of the current filename is kept
% (it is assumed that the filename does not contain the substring `|~~~|'
% which is used as a delimiter).
% Compilation is handed over to the new file by |\childdocforward|:
%    \begin{macrocode}
\newcommand{\childdocforwardprefix}[3][]
{
  \begingroup
    \def\childdocextract #2##1~~~{\def\childdoctmp{\childdocforward[#1]{#3##1}}}
    \expandafter\childdocextract\childdocname~~~
    \expandafter
  \endgroup
  \childdoctmp
}
%    \end{macrocode}

% \macro{\childdoc}
% The deprecated macro |\childdoc| is a legacy version of |\childdocmain|:
%    \begin{macrocode}
\newcommand{\childdoc}{\childdocmain}
%    \end{macrocode}

% \macro{\childdocredirect}
% The deprecated macro |\childdocredirect| is a legacy version
% of |\childdocforward| and |\childdocforwardprefix|:
%    \begin{macrocode}
\newcommand{\childdocredirect}[2][]
{
  \begingroup
    \if?#1?
      \def\childdoctmp{\childdocforward{#2}}
    \else
      \def\childdoctmp{\childdocforwardprefix{#1}{#2}}
    \fi
    \expandafter
  \endgroup
  \childdoctmp
}
%    \end{macrocode}

%\iffalse
%</package>
%\fi
%
\endinput
\childdocforward[cdocsamp]{cdocsch1}"|\\
% |latex -jobname cdocscl2 \|\\
% |  "\def\version{final}% \iffalse
%
% childdoc.dtx Copyright (C) 2017-2018 Niklas Beisert
%
% This work may be distributed and/or modified under the
% conditions of the LaTeX Project Public License, either version 1.3
% of this license or (at your option) any later version.
% The latest version of this license is in
%   http://www.latex-project.org/lppl.txt
% and version 1.3 or later is part of all distributions of LaTeX
% version 2005/12/01 or later.
%
% This work has the LPPL maintenance status `maintained'.
%
% The Current Maintainer of this work is Niklas Beisert.
%
% This work consists of the files childdoc.dtx and childdoc.ins
% and the derived files childdoc.def and cdocsamp.tex with
% cdocsch1.tex, cdocsch2.tex, cdocsdrf.tex, cdocsfn1.tex, cdocsfn2.tex.
%
%<package>\ifdefined\childdocmain\endinput\fi
%<package>\ProvidesFile{childdoc.def}[2018/12/30 v2.0 child document driver]
%<samplemain>\ProvidesFile{cdocsamp.tex}[2018/12/30 v2.0 sample for childdoc]
%<*driver>
%\ProvidesFile{childdoc.drv}[2018/12/30 v2.0 childdoc reference manual file]
\PassOptionsToClass{10pt,a4paper}{article}
\documentclass{ltxdoc}

\usepackage[margin=35mm]{geometry}
\usepackage{hyperref}
\usepackage{hyperxmp}
\usepackage[usenames]{color}

\hypersetup{colorlinks=true}
\hypersetup{pdfstartview=FitH}
\hypersetup{pdfpagemode=UseNone}
\hypersetup{pdfsource={}}
\hypersetup{pdflang={en-UK}}
\hypersetup{pdfcopyright={Copyright 2017-2018 Niklas Beisert.
  This work may be distributed and/or modified under the
  conditions of the LaTeX Project Public License, either version 1.3
  of this license or (at your option) any later version.}}
\hypersetup{pdflicenseurl={http://www.latex-project.org/lppl.txt}}
\hypersetup{pdfcontactaddress={ETH Zurich, ITP, HIT K,
  Wolfgang-Pauli-Strasse 27}}
\hypersetup{pdfcontactpostcode={8093}}
\hypersetup{pdfcontactcity={Zurich}}
\hypersetup{pdfcontactcountry={Switzerland}}
\hypersetup{pdfcontactemail={nbeisert@itp.phys.ethz.ch}}
\hypersetup{pdfcontacturl={http://people.phys.ethz.ch/\xmptilde nbeisert/}}

\newcommand{\secref}[1]{\hyperref[#1]{section \ref*{#1}}}

\parskip1ex
\parindent0pt
\let\olditemize\itemize
\def\itemize{\olditemize\parskip0pt}

\begin{document}

\title{The \textsf{childdoc} Package}
\hypersetup{pdftitle={The childdoc Package}}
\author{Niklas Beisert\\[2ex]
  Institut f\"ur Theoretische Physik\\
  Eidgen\"ossische Technische Hochschule Z\"urich\\
  Wolfgang-Pauli-Strasse 27, 8093 Z\"urich, Switzerland\\[1ex]
  \href{mailto:nbeisert@itp.phys.ethz.ch}
  {\texttt{nbeisert@itp.phys.ethz.ch}}}
\hypersetup{pdfauthor={Niklas Beisert}}
\hypersetup{pdfsubject={Manual for the LaTeX2e Package childdoc}}
\date{30 December 2018, \textsf{v2.0}}
\maketitle

\begin{abstract}\noindent
\textsf{childdoc} is a \LaTeXe{} package
that enables the direct compilation
of document sections included by |\include|
to individual files.
\end{abstract}

\begingroup
\parskip0ex
\tableofcontents
\endgroup

%%%%%%%%%%%%%%%%%%%%%%%%%%%%%%%%%%%%%%%%%%%%%%%%%%%%%%%%%%%%%%%%%%%%%%%%%%%%%%%%
%%%%%%%%%%%%%%%%%%%%%%%%%%%%%%%%%%%%%%%%%%%%%%%%%%%%%%%%%%%%%%%%%%%%%%%%%%%%%%%%
\section{Introduction}

\LaTeX{} provides a mechanism to structure a large document (such as a book)
into a main file and several child files (containing the chapters)
using the |\include| command.
This mechanism is beneficial for documents
which span hundreds of pages in order to
make the source file(s) more manageable.
Moreover, compilation can be restricted to
selected child files by means of the |\includeonly| command.
The latter feature can be used to reduce the compilation time while editing
(this was significantly more useful in the earlier days of \LaTeX{})
or to generate a smaller document which is easier to navigate.
Another application of |\includeonly| is to generate
documents consisting of selected parts of the complete document.

However, there are a few drawbacks of the plain |\include| mechanism:
\begin{itemize}
\item
The child files cannot be compiled on their own,
they can only be compiled via the main file.
A naive editing environment
(such as a text editor with an option
to have the current file processed by \LaTeX)
may require one to switch to the main file before compiling;
attempting to compile the child file produces errors.
\item
The main file must be modified (each time)
to adjust the |\includeonly| command
to the present needs. This easily leaves the main file in a messy state.
\item
The generated document will always carry the filename
of the main document. This is inconvenient if
several child files are to be compiled and
to be kept for distribution.
\end{itemize}

The present package provides a simple interface
to make child files individually compilable by \LaTeX{}.
Compiling a child file then has the same effect as compiling
the main file with an |\includeonly| command
to select the appropriate child.
Moreover the generated document will carry the name of the child
rather than the main file.
This resolves all three above issues.

This feature is meant to make the editing of books,
thesis documents and lecture notes somewhat more convenient.
However, the package can also be used efficiently for
composing a series of documents (such as exercise sheets)
which are typically distributed individually.
It then assists the author in generating the individual documents
(potentially in different versions)
as well as a document containing the collected series.
Another application is in developing style files
or other kinds of included material
where compilation of the style file could redirect
to a sample or test file.

%%%%%%%%%%%%%%%%%%%%%%%%%%%%%%%%%%%%%%%%%%%%%%%%%%%%%%%%%%%%%%%%%%%%%%%%%%%%%%%%
%%%%%%%%%%%%%%%%%%%%%%%%%%%%%%%%%%%%%%%%%%%%%%%%%%%%%%%%%%%%%%%%%%%%%%%%%%%%%%%%
\section{Usage}

First of all, the package \textsf{childdoc} is \emph{not} a standard
\LaTeXe{} |.sty| style file! Therefore it needs to be invoked in
a non-standard way.

%%%%%%%%%%%%%%%%%%%%%%%%%%%%%%%%%%%%%%%%%%%%%%%%%%%%%%%%%%%%%%%%%%%%%%%%%%%%%%%%
\subsection{Included Files}
\label{sec:include}

%%%%%%%%%%%%%%%%%%%%%%%%%%%%%%%%%%%%%%%%
\DescribeMacro{\childdocmain}
To use the package, add the commands
\begin{center}
\begin{tabular}{l}
|\input{childdoc.def}|\\
|\childdocmain{}|\\
\end{tabular}
\end{center}
at the very top of the main \LaTeX{} file,
in particular \emph{before} the |\documentclass| statement!
The argument of |\childdocmain| should be left empty
(but it must be present).

%%%%%%%%%%%%%%%%%%%%%%%%%%%%%%%%%%%%%%%%
\DescribeMacro{\childdocof}
Furthermore, add the commands
\begin{center}
\begin{tabular}{l}
|\input{childdoc.def}|\\
|\childdocof{|\textit{main}|}|\\
\end{tabular}
\end{center}
at the top of every child file \textit{child}
which is included by |\include{|\textit{child}|}|
from within the main file
(or at least for those files to be compiled individually).
The argument \textit{main} must be the filename of the main file.

There are a couple of
considerations in setting up the main and child documents:

%%%%%%%%%%%%%%%%%%%%%%%%%%%%%%%%%%%%%%%%
\paragraph{Restrictions.}

Please note the following restrictions:
\begin{itemize}
\item
|\childdocmain| must be called with one argument \textit{main}
to ensure compatibility with earlier version of the package.
It must either be empty (|\childdocmain{}|)
or precisely match the filename of the main file in which it is specified.
See \secref{sec:detection} for further information.
\item
The filename \textit{main} must be specified without the |.tex| extension.
\item
The filename \textit{main} is case sensitive
(even in case-insensitive file systems)
due to internal string comparison.
\item
The argument \textit{main} should be fully expanded, it cannot be a macro.
\item
Subdirectories and special characters should be avoided in filenames.
\item
The command |\childdocmain{|\textit{main}|}| must be followed by a whitespace.
It should not be followed immediately by another command
or by a comment mark `|%|'.
This is because the \TeX{} parser reads the token immediately following
the argument of |\childdocmain| and puts it
at the beginning of every child section;
however, a white\-space is ignored.
\end{itemize}

%%%%%%%%%%%%%%%%%%%%%%%%%%%%%%%%%%%%%%%%
\paragraph{Content of Main File.}

It is advisable to place all content in the child files included by |\include|.
Any output contained in the main file will appear in all child documents
unless suppressed manually;
it cannot be suppressed automatically by the |\includeonly| directive
and thus should normally be avoided.
A method to include some content in the main file
by means of conditional processing is described in \secref{sec:conditional}.

%%%%%%%%%%%%%%%%%%%%%%%%%%%%%%%%%%%%%%%%
\paragraph{Page Numbering.}

When only a part of the document is compiled,
the appropriate numbering of pages
(as well as other status parameters)
is determined from the |.aux| files.
The latter contain information from previous passes.
However this information needs to propagate through
all intermediate child documents.
Therefore the page numbering in child documents may well
be inconsistent until the complete document is compiled at least once.

A useful (if unconventional) way to always ensure a consistent
page numbering is to restart the numbering in each child document
and denote the pages by `\textit{child}|.|\textit{page}'
where \textit{child} represents the chapter/section number of the child file.
This can be achieved by the command
|\numberwithin{page}{|\textit{child}|}|
of the \textsf{amsmath} package
where \textit{child} can be |chapter| or |section|
depending on the chosen structuring.
Alternatively, one can modify the macro |\thepage| appropriately
and reset the counter |page| at the start of each child file.

%%%%%%%%%%%%%%%%%%%%%%%%%%%%%%%%%%%%%%%%%%%%%%%%%%%%%%%%%%%%%%%%%%%%%%%%%%%%%%%%
\subsection{Conditional Processing}
\label{sec:conditional}

The package provides a mechanism to compile different versions
of a document. To customise the versions further some conditional processing
can come in handy to distinguish which version is being compiled.
The package provides two macros to describe the compilation context:

%%%%%%%%%%%%%%%%%%%%%%%%%%%%%%%%%%%%%%%%
\DescribeMacro{\ifchilddoc}
The conditional |\ifchilddoc| distinguishes between the compilation of
child documents and the main document:
%
\begin{center}
|\ifchilddoc |\textit{child-code}| |[|\||else |\textit{main-code}]| \||fi|
\end{center}

%%%%%%%%%%%%%%%%%%%%%%%%%%%%%%%%%%%%%%%%
\DescribeMacro{\childdocname}
\DescribeMacro{\childdocjob}
The macro |\childdocname| contains the filename (without extension)
of the main or child file being processed.
Note that |\childdocjob| will always contain the name of the main file.

%%%%%%%%%%%%%%%%%%%%%%%%%%%%%%%%%%%%%%%%
\paragraph{Title Page.}

Conditional processing can be used to include a title or banner page
in the main document when proper precautions are taken.
Importantly, the code in the main file should ensure that the page counter
(as well as other status parameters which are stored in the |.aux| files)
takes the same value after the conditional processing.
Otherwise the page numbers may take divergent values
depending on which part is compiled.

For example, a title page could be declared by:
%
\begin{center}
\begin{tabular}{l}
|\ifchilddoc\||else|\\
|\addtocounter{page}{-1}|\\
\textit{code for title page}\\
|\newpage|\\
|\||fi|
\end{tabular}
\end{center}
%
A banner page for the child documents can be generated by:
%
\begin{center}
\begin{tabular}{l}
|\ifchilddoc|\\
|\addtocounter{page}{-1}|\\
\textit{code for banner page}\\
|\newpage|\\
|\||fi|
\end{tabular}
\end{center}
%
Here one could write a message such as:
\begin{center}
|This is the part \childdocname{} of \childdocjob{}.|
\end{center}

%%%%%%%%%%%%%%%%%%%%%%%%%%%%%%%%%%%%%%%%%%%%%%%%%%%%%%%%%%%%%%%%%%%%%%%%%%%%%%%%
\subsection{Flags}
\label{sec:flags}

The package makes it easy to generate different versions
of the main or child documents.
To this end compilation flags can be defined
and assigned different default values.
They will be particularly useful in conjunction
with the forwarding mechanism described in \secref{sec:forward}.

For example, it may be useful to have a flag |\version|
which can be set to |draft| or |final|.
The document source will contain some conditional code
depending on the value of |\version|.
Suppose further, the flag should default to |final| for the main file
and to |draft| for child files
which is a natural assignment for editing the document.
This is achieved by placing the following code
in the preamble of the main document
(below the |\childdocmain| directive):
%
\begin{center}
\begin{tabular}{l}
|\ifchilddoc|\\
|\providecommand{\version}{draft}|\\
|\||else|\\
|\providecommand{\version}{final}|\\
|\||fi|
\end{tabular}
\end{center}
%
The definition by |\providecommand| makes sure
that previous definitions are not overwritten.
Further statements |\providecommand{\version}{...}|
can thus be added before the above code to override it.

For the main file, one might add a line
(between |\childdocmain| and the above block)
%
\begin{center}
|%\ifchilddoc\||else\providecommand{\version}{draft}\||fi|
\end{center}
%
which can be uncommented to produce a draft version.
Likewise one can add a line to the very top of a child file
(above the |\childdocof{|\textit{main}|}| directive)
%
\begin{center}
|%\providecommand{\version}{final}|
\end{center}
%
which can be uncommented to produce the final version of this child document.

%%%%%%%%%%%%%%%%%%%%%%%%%%%%%%%%%%%%%%%%%%%%%%%%%%%%%%%%%%%%%%%%%%%%%%%%%%%%%%%%
\subsection{Forwarding}
\label{sec:forward}

Different versions of the main or child documents
using compilation flags as described in \secref{sec:flags}
can be (permanently) stored in different files
for convenient compilation, viewing and distribution.
To this end, the package defines a command
to pass on compilation to a different file:

%%%%%%%%%%%%%%%%%%%%%%%%%%%%%%%%%%%%%%%%
\DescribeMacro{\childdocforward}
The command |\childdocforward| redirects processing to
another source file:
%
\begin{center}
\begin{tabular}{l}
|\input{childdoc.def}|\\
|\childdocforward[|\textit{main}|]{|\textit{dest}|}|\\
\end{tabular}
\end{center}
%
The argument \textit{dest} is the destination file
(without extension).
It should be the main file or one of the child files.
Note that further \textsf{childdoc} directives
such as |\childdocof| and |\childdocforward|
in the indicated file will be processed in this form.
The optional argument \textit{main}
passes on directly to the main file \textit{main}
while pretending to compile the child \textit{dest}.
This form behaves as if \textit{dest}
issues |\childdocof{|\textit{main}|}| right away,
and no further \textsf{childdoc} directives will be processed.

%%%%%%%%%%%%%%%%%%%%%%%%%%%%%%%%%%%%%%%%
\DescribeMacro{\...prefix}
In the alternative form |\childdocforwardprefix|,
%
\begin{center}
\begin{tabular}{l}
|\input{childdoc.def}|\\
|\childdocforwardprefix[|\textit{main}|]{|\textit{prefix}|}{|\textit{dest}|}|
\end{tabular}
\end{center}
%
the destination file is determined by a pattern
depending on the current file:
To make this work, the current file must be called
`{\textit{prefix}\hspace{0.2em}\textit{suffix}}'
with \textit{prefix} matching precisely the argument.
Processing is then passed on to the file
`{\textit{dest}\hspace{0.2em}\textit{suffix}}'.
Surely, the same effect is achieved by
directly specifying the
argument `{\textit{dest}\hspace{0.2em}\textit{suffix}}'
in the first form.
However, that requires to set up a different file
for each child. With the alternative form of the command
all these files can have exactly the same content
which simplifies setting them up and maintaining them.

For example, the following file |draft.tex|
with a compilation flag |\version| as described in \secref{sec:flags}
compiles the main document as a draft:
%
\begin{center}
\begin{tabular}{l}
|\def\version{draft}|\\
|\input{childdoc.def}|\\
|\childdocforward{|\textit{main}|}|
\end{tabular}
\end{center}
%
Likewise, the following files |final|\textit{nn}|.tex|
compile the final version of the child document
|child|\textit{nn}|.tex|:
%
\begin{center}
\begin{tabular}{l}
|\def\version{final}|\\
|\input{childdoc.def}|\\
|\childdocforwardprefix{final}{child}|
\end{tabular}
\end{center}
%

Note that when several versions of a main file and/or of each child file
are to be generated, it may be convenient to set up a |Makefile| or
shell script to automatise the process.

%%%%%%%%%%%%%%%%%%%%%%%%%%%%%%%%%%%%%%%%%%%%%%%%%%%%%%%%%%%%%%%%%%%%%%%%%%%%%%%%
\subsection{Command Line Processing}
\label{sec:commandline}

The effect of redirection files can also be achieved by invoking
the \LaTeX{} compiler with a more elaborate command line.
Most conveniently this should be done as part
of a shell script or a |Makefile|.

When using \textsf{childdoc} in the main file, the following
command lines effectively perform a redirection
(note that depending on the shell being used,
backslashes may have to be doubled: `|\|' $\to$ `|\\|'):
%
\begin{center}
|... -jobname "|\textit{target}|" |\\|"|[\textit{flags}]%
|\input{childdoc.def}\childdocforward[|\textit{main}|]{|\textit{dest}|}"|
\end{center}
%
Here \textit{target} is the name of the output file,
\textit{main} is the name of the main file
and \textit{dest} is the name of the main or child file to be processed
(all filenames without extensions).
The optional argument \textit{main} can be omitted
if \textit{main} matches \textit{dest}.
Optionally, compilation \textit{flags} can be defined via |\def| commands.
This command line makes the \TeX{} engine believe
it is compiling the file \textit{target}
whose content is specified as the latter parameter.
The provided code then forwards the processing to
\textit{main} or \textit{dest} as described in \secref{sec:forward}.

%%%%%%%%%%%%%%%%%%%%%%%%%%%%%%%%%%%%%%%%%%%%%%%%%%%%%%%%%%%%%%%%%%%%%%%%%%%%%%%%
\subsection{Include by Input}
\label{sec:input}

Including child documents by |\include| has some restrictions by design.
Most notably, the content of a child document always occupies
its own set of pages; pages cannot be shared between child documents.
Usually, this behaviour makes perfect sense
because each child document contain an essential part of the document.
However, in some situations it may be desirable to compose
a document from a collection of parts
without having mandatory page breaks between then.
For this case, the package
provides a mechanism to include parts
by |\input| which can also be processed individually.
However, by construction this mechanism
requires manual handling of the content to be output.

%%%%%%%%%%%%%%%%%%%%%%%%%%%%%%%%%%%%%%%%
\DescribeMacro{\ifchilddocmanual}
The main file should be prepared as usual, see \secref{sec:include}.
However, the document body must make a distinction
between processing of an individual part and of the main document, e.g.:
%
\begin{center}
\begin{tabular}{l}
|\ifchilddocmanual|\\
|\input{\childdocname}|\\
|\||else|\\
\textit{document body with }|\input{|\textit{part}|}|\\
|\||fi|
\end{tabular}
\end{center}
%
The conditional |\ifchilddocmanual| is true whenever
a part to be included by |\input| is being compiled,
and the name of the part is stored in |\childdocname|.

%%%%%%%%%%%%%%%%%%%%%%%%%%%%%%%%%%%%%%%%
\DescribeMacro{\childdocby}
Each part to be included by |\input| should start with:
%
\begin{center}
\begin{tabular}{l}
|\input{childdoc.def}|\\
|\childdocby{|\textit{main}|}|\\
\end{tabular}
\end{center}
%
The directive |\childdocby| is similar to |\childdocof|
described in \secref{sec:include},
but the subsequent selection of content must be done manually.
To that end, both |\ifchilddoc| and |\ifchilddocmanual|
will be true upon processing of a part,
and the name of the part is stored in |\childdocname|.
Note that |\jobname| will be set to the filename of the current part
so that each part receives an individual |.aux| file
that does not interfere with the |.aux| file(s) of the main document.
This behaviour can be altered by the alternative form
|\childdocby[*]{|\textit{main}|}| (with a non-empty optional argument)
which uses the |.aux| file of the main document
by setting |\jobname| to \textit{main}.

%%%%%%%%%%%%%%%%%%%%%%%%%%%%%%%%%%%%%%%%%%%%%%%%%%%%%%%%%%%%%%%%%%%%%%%%%%%%%%%%
\subsection{Driver Development}
\label{sec:driver}

The \textsf{childdoc} mechanism can also be use for the development
of definition files such as \LaTeX{} styles or classes.
This case differs from the above setup with multiple parts
included by |\include| in that no |\includeonly| should be invoked.
This can be achieved by starting the include file
(before |\ProvidesPackage|) with:
%
\begin{center}
\begin{tabular}{l}
|\input{childdoc.def}|\\
|\childdocforward{|\textit{main}|}|\\
\end{tabular}
\end{center}
%
or alternatively with:
%
\begin{center}
\begin{tabular}{l}
|\input{childdoc.def}|\\
|\childdocby{|\textit{main}|}|\\
\end{tabular}
\end{center}
%
Both forms have slightly different effects as described above.
The main file is prepared as usual, see \secref{sec:include}.

%%%%%%%%%%%%%%%%%%%%%%%%%%%%%%%%%%%%%%%%%%%%%%%%%%%%%%%%%%%%%%%%%%%%%%%%%%%%%%%%
\subsection{Legacy Detection}
\label{sec:detection}

The directive |\childdocmain| in the main file can detect
whether the complete document or merely a child is to be compiled
even without using the directive |\childdocof|.
This method is deprecated because it is less robust
and there is no compelling reason to use it;
it is merely provided for backward compatibility
and it may be removed in future versions.

If the detection mechanism is to be used,
it is mandatory to correctly specify
the filename of the main file as the argument of |\childdocmain|:
%
\begin{center}
\begin{tabular}{l}
|\input{childdoc.def}|\\
|\childdocmain{|\textit{main}|}|\\
\end{tabular}
\end{center}
%
If |\jobname| does not match the argument \textit{main} of |\childdocmain|,
it is assumed that |\jobname| points to the child file to be compiled.
When using |\childdocmain| with the main file specified as argument,
it suffices to start a child file
with just |\input{|\textit{main}|}|
without loading of the package and using |\childdocof|.
If instead all processing is done
with the appropriate \textsf{childdoc} directives,
the argument of \textit{main} of |\childdocmain| can be empty.

An alternative version of the command line processing described
in \secref{sec:commandline} using the detection mechanism reads:
%
\begin{center}
|... -jobname "|\textit{target}|" "|[\textit{flags}]%
[|\def\jobname{|\textit{dest}|}|]|\input{|\textit{main}|}"|
\end{center}

%%%%%%%%%%%%%%%%%%%%%%%%%%%%%%%%%%%%%%%%%%%%%%%%%%%%%%%%%%%%%%%%%%%%%%%%%%%%%%%%
\subsection{Manual Code}
\label{sec:manual}

In case one cannot be certain whether the definitions file |childdoc.def|
is installed on the target \TeX{} distribution
and one prefers not to ship it,
it is conceivable to paste a few relevant commands into the sources.

To that end, drop all statements |\input{childdoc.def}|
and perform the replacements as outlined below.
Instead of |\childdocmain{|\textit{main}|}| add the following code
to the top of the main file:
%
\begin{center}
\begin{tabular}{l}
|\||ifdefined\childdocname\endinput\||fi\newif\ifchilddoc|\\
|\edef\childdocname{\scantokens\expandafter{\jobname\noexpand}}|\\
|\def\childdocmain{|\textit{main}|}\||ifx\childdocmain\childdocname\||else|\\
|\childdoctrue\includeonly{\childdocname}\let\jobname\childdocmain\||fi|\\
\end{tabular}
\end{center}
%
Instead of |\childdocof{|\textit{main}|}| just include the main file
at the top of each child file:
%
\begin{center}
|\input{|\textit{main}|}|
\end{center}
%
A simple redirection |\childdocforward{|\textit{dest}|}| is achieved by:
%
\begin{center}
|\def\jobname{|\textit{dest}|}\input{\jobname}|
\end{center}
%
The redirection with prefix
|\childdocforwardprefix[|\textit{prefix}|]{|\textit{dest}|}|
is accomplished by:
%
\begin{center}
\begin{tabular}{l}
|{\edef\jobname{\scantokens\expandafter{\jobname\noexpand}}|\\
|\def\redirectjob |\textit{prefix}|#1~~~{\gdef\jobname{|\textit{dest}|#1}}|\\
|\expandafter\redirectjob\jobname~~~}\input{\jobname}|
\end{tabular}
\end{center}

In an alternative approach,
child documents can be compiled by a specific command line
without additional code or specific definitions:
%
\begin{center}
|... -jobname "|\textit{target}|" "|[\textit{flags}]%
|\includeonly{|\textit{dest}|}\input{|\textit{main}|}"|
\end{center}
%

%%%%%%%%%%%%%%%%%%%%%%%%%%%%%%%%%%%%%%%%%%%%%%%%%%%%%%%%%%%%%%%%%%%%%%%%%%%%%%%%
%%%%%%%%%%%%%%%%%%%%%%%%%%%%%%%%%%%%%%%%%%%%%%%%%%%%%%%%%%%%%%%%%%%%%%%%%%%%%%%%
\section{Information}

%%%%%%%%%%%%%%%%%%%%%%%%%%%%%%%%%%%%%%%%%%%%%%%%%%%%%%%%%%%%%%%%%%%%%%%%%%%%%%%%
\subsection{Copyright}

Copyright \copyright{} 2017--2018 Niklas Beisert

This work may be distributed and/or modified under the
conditions of the \LaTeX{} Project Public License, either version 1.3
of this license or (at your option) any later version.
The latest version of this license is in
  \url{http://www.latex-project.org/lppl.txt}
and version 1.3 or later is part of all distributions of \LaTeX{}
version 2005/12/01 or later.

This work has the LPPL maintenance status `maintained'.

The Current Maintainer of this work is Niklas Beisert.

This work consists of the files |README.txt|, |childdoc.ins| and |childdoc.dtx|
as well as the derived files |childdoc.def|, |cdocsamp.tex|
with |cdocsch1.tex|, |cdocsch2.tex|, |cdocspt3.tex|, |cdocspt4.tex|,
|cdocsdrf.tex|, |cdocsfn1.tex|, |cdocsfn2.tex|
as well as |childdoc.pdf|.

%%%%%%%%%%%%%%%%%%%%%%%%%%%%%%%%%%%%%%%%%%%%%%%%%%%%%%%%%%%%%%%%%%%%%%%%%%%%%%%%
\subsection{Files and Installation}

The package consists of the files:
%
\begin{center}
\begin{tabular}{ll}
    |README.txt|   & readme file \\
    |childdoc.ins| & installation file \\
    |childdoc.dtx| & source file \\
    |childdoc.def| & definition file \\
    |cdocsamp.tex| & sample main file \\
    |cdocsch1.tex| & sample include file \\
    |cdocsch2.tex| & sample include file \\
    |cdocspt3.tex| & sample part file \\
    |cdocspt4.tex| & sample part file \\
    |cdocsdrf.tex| & sample redirection file \\
    |cdocsfn1.tex| & sample redirection file \\
    |cdocsfn2.tex| & sample redirection file \\
    |childdoc.pdf| & manual
\end{tabular}
\end{center}
%
The distribution consists of the files
|README.txt|, |childdoc.ins| and |childdoc.dtx|.
%
\begin{itemize}
\item
Run (pdf)\LaTeX{} on |childdoc.dtx|
to compile the manual |childdoc.pdf| (this file).
\item
Run \LaTeX{} on |childdoc.ins| to create the definitions file |childdoc.def|
and the sample |cdocsamp.tex| with include files
|cdocsch1.tex|, |cdocsch2.tex|, |cdocspt3.tex|, |cdocspt4.tex|,
|cdocsdrf.tex|, |cdocsfn1.tex|, |cdocsfn2.tex|.
Then copy the file |childdoc.def| to an appropriate directory of your \LaTeX{}
distribution, e.g.\ \textit{texmf-root}|/tex/latex/childdoc|.
\end{itemize}

%%%%%%%%%%%%%%%%%%%%%%%%%%%%%%%%%%%%%%%%%%%%%%%%%%%%%%%%%%%%%%%%%%%%%%%%%%%%%%%%
\subsection{Related CTAN Packages}

There are several other packages which offer a similar functionality:
%
\begin{itemize}
\item
The packages
\href{http://ctan.org/pkg/docmute}{\textsf{docmute}},
\href{http://ctan.org/pkg/includex}{\textsf{includex}} and
\href{http://ctan.org/pkg/standalone}{\textsf{standalone}}
provide commands to include only the document body of
a child file thus allowing both files to be compiled individually.
\item
The packages \href{http://ctan.org/pkg/subdocs}{\textsf{subdocs}}
and \href{http://ctan.org/pkg/subfiles}{\textsf{subfiles}}
provide structures in which the main and child documents can be
encapsulated and allowing them to be compiled individually.
The inclusion mechanism is different from the conventional |\include|.
\item
The package \href{http://ctan.org/pkg/combine}{\textsf{combine}}
is an elaborate solution to combine several documents into one.
\end{itemize}
%
See also the CTAN topic \href{http://ctan.org/topic/subdocs}{\textsf{subdocs}}
for further related packages.
The present package differs from the above solutions in that
a document structure constructed with the conventional |\include| mechanism
just needs two extra commands at the top of every file
such that all constituent files can be compiled individually.

%%%%%%%%%%%%%%%%%%%%%%%%%%%%%%%%%%%%%%%%%%%%%%%%%%%%%%%%%%%%%%%%%%%%%%%%%%%%%%%%
%\subsection{Feature Suggestions}
%
%The following is a list of features which may be useful for future
%versions of this package:
%%
%\begin{itemize}
%\item
%\ldots
%\end{itemize}

%%%%%%%%%%%%%%%%%%%%%%%%%%%%%%%%%%%%%%%%%%%%%%%%%%%%%%%%%%%%%%%%%%%%%%%%%%%%%%%%
\subsection{Revision History}

%%%%%%%%%%%%%%%%%%%%%%%%%%%%%%%%%%%%%%%%
\paragraph{v2.0:} 2018/12/30

\begin{itemize}
\item
immediate forward processing
\item
added |\childdocby| mechanism
\item
manual restructured
\end{itemize}

%%%%%%%%%%%%%%%%%%%%%%%%%%%%%%%%%%%%%%%%
\paragraph{v1.6:} 2018/01/17

\begin{itemize}
\item
application for development of include files
\item
corrections to manual
\end{itemize}

%%%%%%%%%%%%%%%%%%%%%%%%%%%%%%%%%%%%%%%%
\paragraph{v1.5:} 2017/05/21

\begin{itemize}
\item
more complete structuring introduced
\item
|\childdocof| introduced
\item
|\childdoc| renamed to |\childdocmain|
\item
|\childredirect| renamed to |\childdocforward| and |\childdocforwardprefix|
and functionality expanded
\end{itemize}

%%%%%%%%%%%%%%%%%%%%%%%%%%%%%%%%%%%%%%%%
\paragraph{v1.0:} 2017/04/27

\begin{itemize}
\item
manual and install package
\item
first version published on CTAN
\end{itemize}

%%%%%%%%%%%%%%%%%%%%%%%%%%%%%%%%%%%%%%%%
\paragraph{v0.6:} 2017/04/26

\begin{itemize}
\item
redirection mechanism added
\end{itemize}

%%%%%%%%%%%%%%%%%%%%%%%%%%%%%%%%%%%%%%%%
\paragraph{v0.5:} 2017/04/26

\begin{itemize}
\item
functionality in definition file
\end{itemize}


%%%%%%%%%%%%%%%%%%%%%%%%%%%%%%%%%%%%%%%%%%%%%%%%%%%%%%%%%%%%%%%%%%%%%%%%%%%%%%%%
%%%%%%%%%%%%%%%%%%%%%%%%%%%%%%%%%%%%%%%%%%%%%%%%%%%%%%%%%%%%%%%%%%%%%%%%%%%%%%%%
%%%%%%%%%%%%%%%%%%%%%%%%%%%%%%%%%%%%%%%%%%%%%%%%%%%%%%%%%%%%%%%%%%%%%%%%%%%%%%%%
\appendix

\settowidth\MacroIndent{\rmfamily\scriptsize 000\ }

 \DocInput{childdoc.dtx}

\end{document}
%</driver>
% \fi
%
% %%%%%%%%%%%%%%%%%%%%%%%%%%%%%%%%%%%%%%%%%%%%%%%%%%%%%%%%%%%%%%%%%%%%%%%%%%%%%%
% %%%%%%%%%%%%%%%%%%%%%%%%%%%%%%%%%%%%%%%%%%%%%%%%%%%%%%%%%%%%%%%%%%%%%%%%%%%%%%
% \section{Sample}
%\iffalse
%<*samplemain>
%\fi
%
% The following presents a sample document
% with two chapters, two parts, a title page,
% a compile flag as well as three forwarding files to set the flag.
% It consists of eight |.tex| files:
% \begin{center}
% \begin{tabular}{ll}
% |cdocsamp.tex|&main file\\
% |cdocsch1.tex|&include file for chapter 1\\
% |cdocsch2.tex|&include file for chapter 2\\
% |cdocspt3.tex|&include file for part 3\\
% |cdocspt4.tex|&include file for part 4\\
% |cdocsdrf.tex|&forwarding file for main file in draft mode\\
% |cdocsfi1.tex|&forwarding file for final version of chapter 1\\
% |cdocsfi2.tex|&forwarding file for final version of chapter 2\\
% \end{tabular}
% \end{center}
% Each of the eight files can be compiled directly by the \LaTeX{} compiler.
%
% %%%%%%%%%%%%%%%%%%%%%%%%%%%%%%%%%%%%%%
% \paragraph{Main File.}
%
% The main file is called |cdocsamp.tex|.
%
% Load the \textsf{childdoc} definitions and
% declare the filename for the main document:
%    \begin{macrocode}
\input{childdoc.def}
\childdocmain{}
%    \end{macrocode}

% Optional override for |\version| flag:
%    \begin{macrocode}
%%\ifchilddoc\else\providecommand{\version}{draft}\fi
%    \end{macrocode}

% Define the default values for the |\version| flag
% (|final| for the main file and |draft| for childs):
%    \begin{macrocode}
\ifchilddoc
\providecommand{\version}{draft}
\else
\providecommand{\version}{final}
\fi
%    \end{macrocode}

% Load the standard document class:
%    \begin{macrocode}
\documentclass[12pt]{article}
%    \end{macrocode}

% Start the document body:
%    \begin{macrocode}
\begin{document}
%    \end{macrocode}

% Declare a title page.
% Print title, part of document being processed and version flag:
%    \begin{macrocode}
\addtocounter{page}{-1}
\begin{center}
{\LARGE\bfseries{}childdoc example\par}
\vspace{1cm}
\ifchilddoc
\ifchilddocmanual part\else chapter\fi:
`\childdocname' of `\childdocjob'\par
\else
main document: `\childdocjob'\par
\fi
version: \version\par
\end{center}
\newpage
%    \end{macrocode}

% Manually include selected file,
% otherwise process as usual:
%    \begin{macrocode}
\ifchilddocmanual
\section*{part `\childdocname'}
\input{\childdocname}
\else
%    \end{macrocode}

% Include the two chapters:
%    \begin{macrocode}
\include{cdocsch1}
\include{cdocsch2}
%    \end{macrocode}

% Include the two parts unless only chapters should be displayed:
%    \begin{macrocode}
\ifchilddoc\else
\section{part three}
\input{cdocspt3}
\section{part four}
\input{cdocspt4}
\fi
%    \end{macrocode}

% Process as usual until here:
%    \begin{macrocode}
\fi
%    \end{macrocode}

% End of document body:
%    \begin{macrocode}
\end{document}
%    \end{macrocode}
%\iffalse
%</samplemain>
%\fi
%
% %%%%%%%%%%%%%%%%%%%%%%%%%%%%%%%%%%%%%%
% \paragraph{Chapter Include Files.}
%
% The include files are called |cdocsch1.tex| and |cdocsch2.tex|.
%
%\iffalse
%<*samplechap1|samplechap2>
%\fi

% Optional override for |\version| flag:
%    \begin{macrocode}
%%\providecommand{\version}{final}
%    \end{macrocode}

% Include the main document:
%    \begin{macrocode}
\input{childdoc.def}
\childdocof{cdocsamp}
%    \end{macrocode}

%\iffalse
%</samplechap1|samplechap2>
%\fi
%
%\iffalse
%<*samplechap1>
%\fi
% Some text for chapter 1:
%    \begin{macrocode}
\section{one}
some text in chapter one
%    \end{macrocode}

%\iffalse
%</samplechap1>
%\fi
% Some text for chapter 2:
%\iffalse
%<*samplechap2>
%\fi
%    \begin{macrocode}
\section{two}
more text in chapter two
%    \end{macrocode}

%\iffalse
%</samplechap2>
%\fi
%
% %%%%%%%%%%%%%%%%%%%%%%%%%%%%%%%%%%%%%%
% \paragraph{Part Include Files.}
%
% The include files are called |cdocspt3.tex| and |cdocspt4.tex|.
%
%\iffalse
%<*samplepart3|samplepart4>
%\fi

% Optional override for |\version| flag:
%    \begin{macrocode}
%%\providecommand{\version}{final}
%    \end{macrocode}

% Include the main document:
%    \begin{macrocode}
\input{childdoc.def}
\childdocby{cdocsamp}
%    \end{macrocode}

%\iffalse
%</samplepart3|samplepart4>
%\fi
%
%\iffalse
%<*samplepart3>
%\fi
% Some text for part 3:
%    \begin{macrocode}
some text in part three
%    \end{macrocode}

%\iffalse
%</samplepart3>
%\fi
% Some text for part 4:
%\iffalse
%<*samplepart4>
%\fi
%    \begin{macrocode}
more text in part four
%    \end{macrocode}

%\iffalse
%</samplepart4>
%\fi
%
% %%%%%%%%%%%%%%%%%%%%%%%%%%%%%%%%%%%%%%
% \paragraph{Forwarding for a Complete Draft.}
%
% The following forwarding file |cdocsdrf.tex|
% compiles the main document in draft mode:
%\iffalse
%<*sampledraft>
%\fi
%    \begin{macrocode}
\def\version{draft}
\input{childdoc.def}
\childdocforward{cdocsamp}
%    \end{macrocode}

%\iffalse
%</sampledraft>
%\fi
%
% %%%%%%%%%%%%%%%%%%%%%%%%%%%%%%%%%%%%%%
% \paragraph{Forwarding for Final Version of the Chapters.}
%
% The following forwarding files |cdocsfn1.tex| and |cdocsfn2.tex|
% (with identical content)
% compile the final versions of the child documents
% |cdocsch1.tex| and |cdocsch2.tex|, respectively:
%\iffalse
%<*samplefinal>
%\fi
%    \begin{macrocode}
\def\version{final}
\input{childdoc.def}
\childdocforwardprefix[cdocsamp]{cdocsfn}{cdocsch}
%    \end{macrocode}

%\iffalse
%</samplefinal>
%\fi
%
% %%%%%%%%%%%%%%%%%%%%%%%%%%%%%%%%%%%%%%
% \paragraph{Command Line Processing.}
%
% The following three command lines generate the output files
% |cdocscld|, |cdocscl1| and |cdocscl2|
% which should be identical to
% |cdocsdrf|, |cdocsch1| and |cdocsfn2|, respectively:
% \begin{center}
% \begin{tabular}{l}
% |latex -jobname cdocscld \|\\
% |  "\def\version{draft}\input{childdoc.def}\childdocforward{cdocsamp}"|\\
% |latex -jobname cdocscl1 \|\\
% |  "\input{childdoc.def}\childdocforward[cdocsamp]{cdocsch1}"|\\
% |latex -jobname cdocscl2 \|\\
% |  "\def\version{final}\input{childdoc.def}\childdocforward{cdocsch2}"|
% \end{tabular}
% \end{center}
% Note that the trailing backslash on each first line
% merely continues the input to the second line
% (for convenient cut ant paste).
% Furthermore, the command |latex| can be replaced by any
% of its alternative versions such as |pdflatex|.
%
% %%%%%%%%%%%%%%%%%%%%%%%%%%%%%%%%%%%%%%%%%%%%%%%%%%%%%%%%%%%%%%%%%%%%%%%%%%%%%%
% %%%%%%%%%%%%%%%%%%%%%%%%%%%%%%%%%%%%%%%%%%%%%%%%%%%%%%%%%%%%%%%%%%%%%%%%%%%%%%
% \section{Implementation}
%\iffalse
%<*package>
%\fi
%
% This section describes the definitions file |childdoc.def|.

% The definitions cannot be loaded using |\usepackage| or |\RequirePackage|
% which has a mechanism to prevent loading a style file more than once.
% When loading the definitions by means of |\input|
% multiple instances have to be prevented manually:
%\iffalse
%This code needs to be before the `\ProvidesFile' directive
%which is defined at the beginning of this file.
%Therefore it is also placed there and commented out here.
%</package>
%<*discard>
%\fi
%    \begin{macrocode}
\ifdefined\childdocmain\endinput\fi
%    \end{macrocode}
%\iffalse
%</discard>
%<*package>
%\fi
%
% \macro{\ifchilddoc}
% \macro{\ifchilddocmanual}
% The conditional |\ifchilddoc| tells whether a
% child (true) or main (false) document is being compiled.
% The conditional |\ifchilddocmanual| tells whether
% the |\includeonly| mechanism is used (false) or
% the selection of child files must be performed manually (true).
% The definitions initialise to false:
%    \begin{macrocode}
\newif\ifchilddoc
\newif\ifchilddocmanual
%    \end{macrocode}

% \macro{\childdocname}
% \macro{\childdocjob}
% The macro |\childdocname| stores the name of the main document
% to be compiled. The macro |\childdocjob| stores the name of
% the document on which the \LaTeX{} compiler was originally invoked.
% The content of |\jobname| cannot be compared
% to filenames specified in the source due to different catcodes.
% The following code rescans |\jobname|, stores the result
% in |\childdocname| and saves a copy in |\childdocjob|:
%    \begin{macrocode}
\edef\childdocname{\scantokens\expandafter{\jobname\noexpand}}
\let\childdocjob\childdocname
%    \end{macrocode}

% \macro{\childdocdisable}
% The macro |\childdocdisable| prevents the main file
% from being processed more than once.
% At this stage, the main document command |\childdocmain|
% is assumed to be called once again where it should do nothing.
% Any subsequent call to it should prevent
% a secondary processing of the main document
% It overwrites the forwarding commands
% |\childdocof| and |\childdocforward|
% with empty macros to prevent further inclusions of the main document:
%    \begin{macrocode}
\newcommand{\childdocdisable}
{
  \renewcommand{\childdocmain}[1]{\renewcommand{\childdocmain}[1]{\endinput}}
  \renewcommand{\childdocof}[1]{}
  \renewcommand{\childdocby}[2][]{}
  \renewcommand{\childdocforward}[2][]{}
  \renewcommand{\childdocdisable}{}
}
%    \end{macrocode}

% \macro{\childdocmain}
% The macro |\childdocmain| is to be called at the top of the main file
% with nothing or the main filename (without extension) as argument.
% First, it breaks loops.
% If the argument is not empty and does not match |\childdocname|
% (which is set by the first inclusion of |childdoc.def|),
% |\ifchilddoc| is set to true, |\includeonly| is applied to the child file
% and |\jobname| is set to the main file
% (for proper handling of |.aux| files):
%    \begin{macrocode}
\newcommand{\childdocmain}[1]
{
  \childdocdisable\childdocmain{}
  \if?#1?\else
    \begingroup
      \def\childdoctmp{#1}
      \ifx\childdoctmp\childdocname
        \def\childdoctmp{}
      \else
        \def\childdoctmp
        {
          \childdoctrue
          \includeonly{\childdocname}
          \def\childdocjob{#1}
          \def\jobname{#1}
        }
      \fi
      \expandafter
    \endgroup
    \childdoctmp
  \fi
}
%    \end{macrocode}

% \macro{\childdocof}
% The command |\childdocof| redirects
% compilation to the main file |#1|.
%    \begin{macrocode}
\newcommand{\childdocof}[1]
{
  \childdocdisable
  \childdoctrue
  \includeonly{\childdocname}
  \def\jobname{#1}
  \def\childdocjob{#1}
  \input{#1}
}
%    \end{macrocode}

% \macro{\childdocby}
% The command |\childdocby| ....
%    \begin{macrocode}
\newcommand{\childdocby}[2][]
{
  \childdocdisable
  \childdoctrue
  \childdocmanualtrue
  \if?#1?\else
    \def\jobname{#2}
  \fi
  \def\childdocjob{#2}
  \input{#2}
  \endinput
}
%    \end{macrocode}

% \macro{\childdocforward}
% The command |\childdocforward| redirects
% compilation to the main file or
% (if the optional argument is given) a child file.
% Parameters are set as if the main file
% or a child file starting with |\childdocof| was compiled.
% Then compilation is handed over to the main file:
%    \begin{macrocode}
\newcommand{\childdocforward}[2][]
{
  \begingroup
    \if?#1?
      \def\childdoctmp
      {
        \def\childdocname{#2}
        \def\childdocjob{#2}
        \def\jobname{#2}
        \input{#2}
        \endinput
      }
    \else
      \def\childdoctmp
      {
        \childdocdisable
        \def\childdocname{#2}
        \childdoctrue
        \includeonly{#2}
        \def\childdocjob{#1}
        \def\jobname{#1}
        \input{#1}
        \endinput
      }
    \fi
    \expandafter
  \endgroup
  \childdoctmp
}
%    \end{macrocode}

% \macro{\childdocforwardprefix}
% The command |\childdocforwardprefix| redirects
% compilation to the main or a child file by means of a pattern.
% The prefix |#1| in the current filename is replaced by |#2|
% and the suffix of the current filename is kept
% (it is assumed that the filename does not contain the substring `|~~~|'
% which is used as a delimiter).
% Compilation is handed over to the new file by |\childdocforward|:
%    \begin{macrocode}
\newcommand{\childdocforwardprefix}[3][]
{
  \begingroup
    \def\childdocextract #2##1~~~{\def\childdoctmp{\childdocforward[#1]{#3##1}}}
    \expandafter\childdocextract\childdocname~~~
    \expandafter
  \endgroup
  \childdoctmp
}
%    \end{macrocode}

% \macro{\childdoc}
% The deprecated macro |\childdoc| is a legacy version of |\childdocmain|:
%    \begin{macrocode}
\newcommand{\childdoc}{\childdocmain}
%    \end{macrocode}

% \macro{\childdocredirect}
% The deprecated macro |\childdocredirect| is a legacy version
% of |\childdocforward| and |\childdocforwardprefix|:
%    \begin{macrocode}
\newcommand{\childdocredirect}[2][]
{
  \begingroup
    \if?#1?
      \def\childdoctmp{\childdocforward{#2}}
    \else
      \def\childdoctmp{\childdocforwardprefix{#1}{#2}}
    \fi
    \expandafter
  \endgroup
  \childdoctmp
}
%    \end{macrocode}

%\iffalse
%</package>
%\fi
%
\endinput
\childdocforward{cdocsch2}"|
% \end{tabular}
% \end{center}
% Note that the trailing backslash on each first line
% merely continues the input to the second line
% (for convenient cut ant paste).
% Furthermore, the command |latex| can be replaced by any
% of its alternative versions such as |pdflatex|.
%
% %%%%%%%%%%%%%%%%%%%%%%%%%%%%%%%%%%%%%%%%%%%%%%%%%%%%%%%%%%%%%%%%%%%%%%%%%%%%%%
% %%%%%%%%%%%%%%%%%%%%%%%%%%%%%%%%%%%%%%%%%%%%%%%%%%%%%%%%%%%%%%%%%%%%%%%%%%%%%%
% \section{Implementation}
%\iffalse
%<*package>
%\fi
%
% This section describes the definitions file |childdoc.def|.

% The definitions cannot be loaded using |\usepackage| or |\RequirePackage|
% which has a mechanism to prevent loading a style file more than once.
% When loading the definitions by means of |\input|
% multiple instances have to be prevented manually:
%\iffalse
%This code needs to be before the `\ProvidesFile' directive
%which is defined at the beginning of this file.
%Therefore it is also placed there and commented out here.
%</package>
%<*discard>
%\fi
%    \begin{macrocode}
\ifdefined\childdocmain\endinput\fi
%    \end{macrocode}
%\iffalse
%</discard>
%<*package>
%\fi
%
% \macro{\ifchilddoc}
% \macro{\ifchilddocmanual}
% The conditional |\ifchilddoc| tells whether a
% child (true) or main (false) document is being compiled.
% The conditional |\ifchilddocmanual| tells whether
% the |\includeonly| mechanism is used (false) or
% the selection of child files must be performed manually (true).
% The definitions initialise to false:
%    \begin{macrocode}
\newif\ifchilddoc
\newif\ifchilddocmanual
%    \end{macrocode}

% \macro{\childdocname}
% \macro{\childdocjob}
% The macro |\childdocname| stores the name of the main document
% to be compiled. The macro |\childdocjob| stores the name of
% the document on which the \LaTeX{} compiler was originally invoked.
% The content of |\jobname| cannot be compared
% to filenames specified in the source due to different catcodes.
% The following code rescans |\jobname|, stores the result
% in |\childdocname| and saves a copy in |\childdocjob|:
%    \begin{macrocode}
\edef\childdocname{\scantokens\expandafter{\jobname\noexpand}}
\let\childdocjob\childdocname
%    \end{macrocode}

% \macro{\childdocdisable}
% The macro |\childdocdisable| prevents the main file
% from being processed more than once.
% At this stage, the main document command |\childdocmain|
% is assumed to be called once again where it should do nothing.
% Any subsequent call to it should prevent
% a secondary processing of the main document
% It overwrites the forwarding commands
% |\childdocof| and |\childdocforward|
% with empty macros to prevent further inclusions of the main document:
%    \begin{macrocode}
\newcommand{\childdocdisable}
{
  \renewcommand{\childdocmain}[1]{\renewcommand{\childdocmain}[1]{\endinput}}
  \renewcommand{\childdocof}[1]{}
  \renewcommand{\childdocby}[2][]{}
  \renewcommand{\childdocforward}[2][]{}
  \renewcommand{\childdocdisable}{}
}
%    \end{macrocode}

% \macro{\childdocmain}
% The macro |\childdocmain| is to be called at the top of the main file
% with nothing or the main filename (without extension) as argument.
% First, it breaks loops.
% If the argument is not empty and does not match |\childdocname|
% (which is set by the first inclusion of |childdoc.def|),
% |\ifchilddoc| is set to true, |\includeonly| is applied to the child file
% and |\jobname| is set to the main file
% (for proper handling of |.aux| files):
%    \begin{macrocode}
\newcommand{\childdocmain}[1]
{
  \childdocdisable\childdocmain{}
  \if?#1?\else
    \begingroup
      \def\childdoctmp{#1}
      \ifx\childdoctmp\childdocname
        \def\childdoctmp{}
      \else
        \def\childdoctmp
        {
          \childdoctrue
          \includeonly{\childdocname}
          \def\childdocjob{#1}
          \def\jobname{#1}
        }
      \fi
      \expandafter
    \endgroup
    \childdoctmp
  \fi
}
%    \end{macrocode}

% \macro{\childdocof}
% The command |\childdocof| redirects
% compilation to the main file |#1|.
%    \begin{macrocode}
\newcommand{\childdocof}[1]
{
  \childdocdisable
  \childdoctrue
  \includeonly{\childdocname}
  \def\jobname{#1}
  \def\childdocjob{#1}
  \input{#1}
}
%    \end{macrocode}

% \macro{\childdocby}
% The command |\childdocby| ....
%    \begin{macrocode}
\newcommand{\childdocby}[2][]
{
  \childdocdisable
  \childdoctrue
  \childdocmanualtrue
  \if?#1?\else
    \def\jobname{#2}
  \fi
  \def\childdocjob{#2}
  \input{#2}
  \endinput
}
%    \end{macrocode}

% \macro{\childdocforward}
% The command |\childdocforward| redirects
% compilation to the main file or
% (if the optional argument is given) a child file.
% Parameters are set as if the main file
% or a child file starting with |\childdocof| was compiled.
% Then compilation is handed over to the main file:
%    \begin{macrocode}
\newcommand{\childdocforward}[2][]
{
  \begingroup
    \if?#1?
      \def\childdoctmp
      {
        \def\childdocname{#2}
        \def\childdocjob{#2}
        \def\jobname{#2}
        \input{#2}
        \endinput
      }
    \else
      \def\childdoctmp
      {
        \childdocdisable
        \def\childdocname{#2}
        \childdoctrue
        \includeonly{#2}
        \def\childdocjob{#1}
        \def\jobname{#1}
        \input{#1}
        \endinput
      }
    \fi
    \expandafter
  \endgroup
  \childdoctmp
}
%    \end{macrocode}

% \macro{\childdocforwardprefix}
% The command |\childdocforwardprefix| redirects
% compilation to the main or a child file by means of a pattern.
% The prefix |#1| in the current filename is replaced by |#2|
% and the suffix of the current filename is kept
% (it is assumed that the filename does not contain the substring `|~~~|'
% which is used as a delimiter).
% Compilation is handed over to the new file by |\childdocforward|:
%    \begin{macrocode}
\newcommand{\childdocforwardprefix}[3][]
{
  \begingroup
    \def\childdocextract #2##1~~~{\def\childdoctmp{\childdocforward[#1]{#3##1}}}
    \expandafter\childdocextract\childdocname~~~
    \expandafter
  \endgroup
  \childdoctmp
}
%    \end{macrocode}

% \macro{\childdoc}
% The deprecated macro |\childdoc| is a legacy version of |\childdocmain|:
%    \begin{macrocode}
\newcommand{\childdoc}{\childdocmain}
%    \end{macrocode}

% \macro{\childdocredirect}
% The deprecated macro |\childdocredirect| is a legacy version
% of |\childdocforward| and |\childdocforwardprefix|:
%    \begin{macrocode}
\newcommand{\childdocredirect}[2][]
{
  \begingroup
    \if?#1?
      \def\childdoctmp{\childdocforward{#2}}
    \else
      \def\childdoctmp{\childdocforwardprefix{#1}{#2}}
    \fi
    \expandafter
  \endgroup
  \childdoctmp
}
%    \end{macrocode}

%\iffalse
%</package>
%\fi
%
\endinput
|\\
|\childdocforward{|\textit{main}|}|
\end{tabular}
\end{center}
%
Likewise, the following files |final|\textit{nn}|.tex|
compile the final version of the child document
|child|\textit{nn}|.tex|:
%
\begin{center}
\begin{tabular}{l}
|\def\version{final}|\\
|% \iffalse
%
% childdoc.dtx Copyright (C) 2017-2018 Niklas Beisert
%
% This work may be distributed and/or modified under the
% conditions of the LaTeX Project Public License, either version 1.3
% of this license or (at your option) any later version.
% The latest version of this license is in
%   http://www.latex-project.org/lppl.txt
% and version 1.3 or later is part of all distributions of LaTeX
% version 2005/12/01 or later.
%
% This work has the LPPL maintenance status `maintained'.
%
% The Current Maintainer of this work is Niklas Beisert.
%
% This work consists of the files childdoc.dtx and childdoc.ins
% and the derived files childdoc.def and cdocsamp.tex with
% cdocsch1.tex, cdocsch2.tex, cdocsdrf.tex, cdocsfn1.tex, cdocsfn2.tex.
%
%<package>\ifdefined\childdocmain\endinput\fi
%<package>\ProvidesFile{childdoc.def}[2018/12/30 v2.0 child document driver]
%<samplemain>\ProvidesFile{cdocsamp.tex}[2018/12/30 v2.0 sample for childdoc]
%<*driver>
%\ProvidesFile{childdoc.drv}[2018/12/30 v2.0 childdoc reference manual file]
\PassOptionsToClass{10pt,a4paper}{article}
\documentclass{ltxdoc}

\usepackage[margin=35mm]{geometry}
\usepackage{hyperref}
\usepackage{hyperxmp}
\usepackage[usenames]{color}

\hypersetup{colorlinks=true}
\hypersetup{pdfstartview=FitH}
\hypersetup{pdfpagemode=UseNone}
\hypersetup{pdfsource={}}
\hypersetup{pdflang={en-UK}}
\hypersetup{pdfcopyright={Copyright 2017-2018 Niklas Beisert.
  This work may be distributed and/or modified under the
  conditions of the LaTeX Project Public License, either version 1.3
  of this license or (at your option) any later version.}}
\hypersetup{pdflicenseurl={http://www.latex-project.org/lppl.txt}}
\hypersetup{pdfcontactaddress={ETH Zurich, ITP, HIT K,
  Wolfgang-Pauli-Strasse 27}}
\hypersetup{pdfcontactpostcode={8093}}
\hypersetup{pdfcontactcity={Zurich}}
\hypersetup{pdfcontactcountry={Switzerland}}
\hypersetup{pdfcontactemail={nbeisert@itp.phys.ethz.ch}}
\hypersetup{pdfcontacturl={http://people.phys.ethz.ch/\xmptilde nbeisert/}}

\newcommand{\secref}[1]{\hyperref[#1]{section \ref*{#1}}}

\parskip1ex
\parindent0pt
\let\olditemize\itemize
\def\itemize{\olditemize\parskip0pt}

\begin{document}

\title{The \textsf{childdoc} Package}
\hypersetup{pdftitle={The childdoc Package}}
\author{Niklas Beisert\\[2ex]
  Institut f\"ur Theoretische Physik\\
  Eidgen\"ossische Technische Hochschule Z\"urich\\
  Wolfgang-Pauli-Strasse 27, 8093 Z\"urich, Switzerland\\[1ex]
  \href{mailto:nbeisert@itp.phys.ethz.ch}
  {\texttt{nbeisert@itp.phys.ethz.ch}}}
\hypersetup{pdfauthor={Niklas Beisert}}
\hypersetup{pdfsubject={Manual for the LaTeX2e Package childdoc}}
\date{30 December 2018, \textsf{v2.0}}
\maketitle

\begin{abstract}\noindent
\textsf{childdoc} is a \LaTeXe{} package
that enables the direct compilation
of document sections included by |\include|
to individual files.
\end{abstract}

\begingroup
\parskip0ex
\tableofcontents
\endgroup

%%%%%%%%%%%%%%%%%%%%%%%%%%%%%%%%%%%%%%%%%%%%%%%%%%%%%%%%%%%%%%%%%%%%%%%%%%%%%%%%
%%%%%%%%%%%%%%%%%%%%%%%%%%%%%%%%%%%%%%%%%%%%%%%%%%%%%%%%%%%%%%%%%%%%%%%%%%%%%%%%
\section{Introduction}

\LaTeX{} provides a mechanism to structure a large document (such as a book)
into a main file and several child files (containing the chapters)
using the |\include| command.
This mechanism is beneficial for documents
which span hundreds of pages in order to
make the source file(s) more manageable.
Moreover, compilation can be restricted to
selected child files by means of the |\includeonly| command.
The latter feature can be used to reduce the compilation time while editing
(this was significantly more useful in the earlier days of \LaTeX{})
or to generate a smaller document which is easier to navigate.
Another application of |\includeonly| is to generate
documents consisting of selected parts of the complete document.

However, there are a few drawbacks of the plain |\include| mechanism:
\begin{itemize}
\item
The child files cannot be compiled on their own,
they can only be compiled via the main file.
A naive editing environment
(such as a text editor with an option
to have the current file processed by \LaTeX)
may require one to switch to the main file before compiling;
attempting to compile the child file produces errors.
\item
The main file must be modified (each time)
to adjust the |\includeonly| command
to the present needs. This easily leaves the main file in a messy state.
\item
The generated document will always carry the filename
of the main document. This is inconvenient if
several child files are to be compiled and
to be kept for distribution.
\end{itemize}

The present package provides a simple interface
to make child files individually compilable by \LaTeX{}.
Compiling a child file then has the same effect as compiling
the main file with an |\includeonly| command
to select the appropriate child.
Moreover the generated document will carry the name of the child
rather than the main file.
This resolves all three above issues.

This feature is meant to make the editing of books,
thesis documents and lecture notes somewhat more convenient.
However, the package can also be used efficiently for
composing a series of documents (such as exercise sheets)
which are typically distributed individually.
It then assists the author in generating the individual documents
(potentially in different versions)
as well as a document containing the collected series.
Another application is in developing style files
or other kinds of included material
where compilation of the style file could redirect
to a sample or test file.

%%%%%%%%%%%%%%%%%%%%%%%%%%%%%%%%%%%%%%%%%%%%%%%%%%%%%%%%%%%%%%%%%%%%%%%%%%%%%%%%
%%%%%%%%%%%%%%%%%%%%%%%%%%%%%%%%%%%%%%%%%%%%%%%%%%%%%%%%%%%%%%%%%%%%%%%%%%%%%%%%
\section{Usage}

First of all, the package \textsf{childdoc} is \emph{not} a standard
\LaTeXe{} |.sty| style file! Therefore it needs to be invoked in
a non-standard way.

%%%%%%%%%%%%%%%%%%%%%%%%%%%%%%%%%%%%%%%%%%%%%%%%%%%%%%%%%%%%%%%%%%%%%%%%%%%%%%%%
\subsection{Included Files}
\label{sec:include}

%%%%%%%%%%%%%%%%%%%%%%%%%%%%%%%%%%%%%%%%
\DescribeMacro{\childdocmain}
To use the package, add the commands
\begin{center}
\begin{tabular}{l}
|% \iffalse
%
% childdoc.dtx Copyright (C) 2017-2018 Niklas Beisert
%
% This work may be distributed and/or modified under the
% conditions of the LaTeX Project Public License, either version 1.3
% of this license or (at your option) any later version.
% The latest version of this license is in
%   http://www.latex-project.org/lppl.txt
% and version 1.3 or later is part of all distributions of LaTeX
% version 2005/12/01 or later.
%
% This work has the LPPL maintenance status `maintained'.
%
% The Current Maintainer of this work is Niklas Beisert.
%
% This work consists of the files childdoc.dtx and childdoc.ins
% and the derived files childdoc.def and cdocsamp.tex with
% cdocsch1.tex, cdocsch2.tex, cdocsdrf.tex, cdocsfn1.tex, cdocsfn2.tex.
%
%<package>\ifdefined\childdocmain\endinput\fi
%<package>\ProvidesFile{childdoc.def}[2018/12/30 v2.0 child document driver]
%<samplemain>\ProvidesFile{cdocsamp.tex}[2018/12/30 v2.0 sample for childdoc]
%<*driver>
%\ProvidesFile{childdoc.drv}[2018/12/30 v2.0 childdoc reference manual file]
\PassOptionsToClass{10pt,a4paper}{article}
\documentclass{ltxdoc}

\usepackage[margin=35mm]{geometry}
\usepackage{hyperref}
\usepackage{hyperxmp}
\usepackage[usenames]{color}

\hypersetup{colorlinks=true}
\hypersetup{pdfstartview=FitH}
\hypersetup{pdfpagemode=UseNone}
\hypersetup{pdfsource={}}
\hypersetup{pdflang={en-UK}}
\hypersetup{pdfcopyright={Copyright 2017-2018 Niklas Beisert.
  This work may be distributed and/or modified under the
  conditions of the LaTeX Project Public License, either version 1.3
  of this license or (at your option) any later version.}}
\hypersetup{pdflicenseurl={http://www.latex-project.org/lppl.txt}}
\hypersetup{pdfcontactaddress={ETH Zurich, ITP, HIT K,
  Wolfgang-Pauli-Strasse 27}}
\hypersetup{pdfcontactpostcode={8093}}
\hypersetup{pdfcontactcity={Zurich}}
\hypersetup{pdfcontactcountry={Switzerland}}
\hypersetup{pdfcontactemail={nbeisert@itp.phys.ethz.ch}}
\hypersetup{pdfcontacturl={http://people.phys.ethz.ch/\xmptilde nbeisert/}}

\newcommand{\secref}[1]{\hyperref[#1]{section \ref*{#1}}}

\parskip1ex
\parindent0pt
\let\olditemize\itemize
\def\itemize{\olditemize\parskip0pt}

\begin{document}

\title{The \textsf{childdoc} Package}
\hypersetup{pdftitle={The childdoc Package}}
\author{Niklas Beisert\\[2ex]
  Institut f\"ur Theoretische Physik\\
  Eidgen\"ossische Technische Hochschule Z\"urich\\
  Wolfgang-Pauli-Strasse 27, 8093 Z\"urich, Switzerland\\[1ex]
  \href{mailto:nbeisert@itp.phys.ethz.ch}
  {\texttt{nbeisert@itp.phys.ethz.ch}}}
\hypersetup{pdfauthor={Niklas Beisert}}
\hypersetup{pdfsubject={Manual for the LaTeX2e Package childdoc}}
\date{30 December 2018, \textsf{v2.0}}
\maketitle

\begin{abstract}\noindent
\textsf{childdoc} is a \LaTeXe{} package
that enables the direct compilation
of document sections included by |\include|
to individual files.
\end{abstract}

\begingroup
\parskip0ex
\tableofcontents
\endgroup

%%%%%%%%%%%%%%%%%%%%%%%%%%%%%%%%%%%%%%%%%%%%%%%%%%%%%%%%%%%%%%%%%%%%%%%%%%%%%%%%
%%%%%%%%%%%%%%%%%%%%%%%%%%%%%%%%%%%%%%%%%%%%%%%%%%%%%%%%%%%%%%%%%%%%%%%%%%%%%%%%
\section{Introduction}

\LaTeX{} provides a mechanism to structure a large document (such as a book)
into a main file and several child files (containing the chapters)
using the |\include| command.
This mechanism is beneficial for documents
which span hundreds of pages in order to
make the source file(s) more manageable.
Moreover, compilation can be restricted to
selected child files by means of the |\includeonly| command.
The latter feature can be used to reduce the compilation time while editing
(this was significantly more useful in the earlier days of \LaTeX{})
or to generate a smaller document which is easier to navigate.
Another application of |\includeonly| is to generate
documents consisting of selected parts of the complete document.

However, there are a few drawbacks of the plain |\include| mechanism:
\begin{itemize}
\item
The child files cannot be compiled on their own,
they can only be compiled via the main file.
A naive editing environment
(such as a text editor with an option
to have the current file processed by \LaTeX)
may require one to switch to the main file before compiling;
attempting to compile the child file produces errors.
\item
The main file must be modified (each time)
to adjust the |\includeonly| command
to the present needs. This easily leaves the main file in a messy state.
\item
The generated document will always carry the filename
of the main document. This is inconvenient if
several child files are to be compiled and
to be kept for distribution.
\end{itemize}

The present package provides a simple interface
to make child files individually compilable by \LaTeX{}.
Compiling a child file then has the same effect as compiling
the main file with an |\includeonly| command
to select the appropriate child.
Moreover the generated document will carry the name of the child
rather than the main file.
This resolves all three above issues.

This feature is meant to make the editing of books,
thesis documents and lecture notes somewhat more convenient.
However, the package can also be used efficiently for
composing a series of documents (such as exercise sheets)
which are typically distributed individually.
It then assists the author in generating the individual documents
(potentially in different versions)
as well as a document containing the collected series.
Another application is in developing style files
or other kinds of included material
where compilation of the style file could redirect
to a sample or test file.

%%%%%%%%%%%%%%%%%%%%%%%%%%%%%%%%%%%%%%%%%%%%%%%%%%%%%%%%%%%%%%%%%%%%%%%%%%%%%%%%
%%%%%%%%%%%%%%%%%%%%%%%%%%%%%%%%%%%%%%%%%%%%%%%%%%%%%%%%%%%%%%%%%%%%%%%%%%%%%%%%
\section{Usage}

First of all, the package \textsf{childdoc} is \emph{not} a standard
\LaTeXe{} |.sty| style file! Therefore it needs to be invoked in
a non-standard way.

%%%%%%%%%%%%%%%%%%%%%%%%%%%%%%%%%%%%%%%%%%%%%%%%%%%%%%%%%%%%%%%%%%%%%%%%%%%%%%%%
\subsection{Included Files}
\label{sec:include}

%%%%%%%%%%%%%%%%%%%%%%%%%%%%%%%%%%%%%%%%
\DescribeMacro{\childdocmain}
To use the package, add the commands
\begin{center}
\begin{tabular}{l}
|\input{childdoc.def}|\\
|\childdocmain{}|\\
\end{tabular}
\end{center}
at the very top of the main \LaTeX{} file,
in particular \emph{before} the |\documentclass| statement!
The argument of |\childdocmain| should be left empty
(but it must be present).

%%%%%%%%%%%%%%%%%%%%%%%%%%%%%%%%%%%%%%%%
\DescribeMacro{\childdocof}
Furthermore, add the commands
\begin{center}
\begin{tabular}{l}
|\input{childdoc.def}|\\
|\childdocof{|\textit{main}|}|\\
\end{tabular}
\end{center}
at the top of every child file \textit{child}
which is included by |\include{|\textit{child}|}|
from within the main file
(or at least for those files to be compiled individually).
The argument \textit{main} must be the filename of the main file.

There are a couple of
considerations in setting up the main and child documents:

%%%%%%%%%%%%%%%%%%%%%%%%%%%%%%%%%%%%%%%%
\paragraph{Restrictions.}

Please note the following restrictions:
\begin{itemize}
\item
|\childdocmain| must be called with one argument \textit{main}
to ensure compatibility with earlier version of the package.
It must either be empty (|\childdocmain{}|)
or precisely match the filename of the main file in which it is specified.
See \secref{sec:detection} for further information.
\item
The filename \textit{main} must be specified without the |.tex| extension.
\item
The filename \textit{main} is case sensitive
(even in case-insensitive file systems)
due to internal string comparison.
\item
The argument \textit{main} should be fully expanded, it cannot be a macro.
\item
Subdirectories and special characters should be avoided in filenames.
\item
The command |\childdocmain{|\textit{main}|}| must be followed by a whitespace.
It should not be followed immediately by another command
or by a comment mark `|%|'.
This is because the \TeX{} parser reads the token immediately following
the argument of |\childdocmain| and puts it
at the beginning of every child section;
however, a white\-space is ignored.
\end{itemize}

%%%%%%%%%%%%%%%%%%%%%%%%%%%%%%%%%%%%%%%%
\paragraph{Content of Main File.}

It is advisable to place all content in the child files included by |\include|.
Any output contained in the main file will appear in all child documents
unless suppressed manually;
it cannot be suppressed automatically by the |\includeonly| directive
and thus should normally be avoided.
A method to include some content in the main file
by means of conditional processing is described in \secref{sec:conditional}.

%%%%%%%%%%%%%%%%%%%%%%%%%%%%%%%%%%%%%%%%
\paragraph{Page Numbering.}

When only a part of the document is compiled,
the appropriate numbering of pages
(as well as other status parameters)
is determined from the |.aux| files.
The latter contain information from previous passes.
However this information needs to propagate through
all intermediate child documents.
Therefore the page numbering in child documents may well
be inconsistent until the complete document is compiled at least once.

A useful (if unconventional) way to always ensure a consistent
page numbering is to restart the numbering in each child document
and denote the pages by `\textit{child}|.|\textit{page}'
where \textit{child} represents the chapter/section number of the child file.
This can be achieved by the command
|\numberwithin{page}{|\textit{child}|}|
of the \textsf{amsmath} package
where \textit{child} can be |chapter| or |section|
depending on the chosen structuring.
Alternatively, one can modify the macro |\thepage| appropriately
and reset the counter |page| at the start of each child file.

%%%%%%%%%%%%%%%%%%%%%%%%%%%%%%%%%%%%%%%%%%%%%%%%%%%%%%%%%%%%%%%%%%%%%%%%%%%%%%%%
\subsection{Conditional Processing}
\label{sec:conditional}

The package provides a mechanism to compile different versions
of a document. To customise the versions further some conditional processing
can come in handy to distinguish which version is being compiled.
The package provides two macros to describe the compilation context:

%%%%%%%%%%%%%%%%%%%%%%%%%%%%%%%%%%%%%%%%
\DescribeMacro{\ifchilddoc}
The conditional |\ifchilddoc| distinguishes between the compilation of
child documents and the main document:
%
\begin{center}
|\ifchilddoc |\textit{child-code}| |[|\||else |\textit{main-code}]| \||fi|
\end{center}

%%%%%%%%%%%%%%%%%%%%%%%%%%%%%%%%%%%%%%%%
\DescribeMacro{\childdocname}
\DescribeMacro{\childdocjob}
The macro |\childdocname| contains the filename (without extension)
of the main or child file being processed.
Note that |\childdocjob| will always contain the name of the main file.

%%%%%%%%%%%%%%%%%%%%%%%%%%%%%%%%%%%%%%%%
\paragraph{Title Page.}

Conditional processing can be used to include a title or banner page
in the main document when proper precautions are taken.
Importantly, the code in the main file should ensure that the page counter
(as well as other status parameters which are stored in the |.aux| files)
takes the same value after the conditional processing.
Otherwise the page numbers may take divergent values
depending on which part is compiled.

For example, a title page could be declared by:
%
\begin{center}
\begin{tabular}{l}
|\ifchilddoc\||else|\\
|\addtocounter{page}{-1}|\\
\textit{code for title page}\\
|\newpage|\\
|\||fi|
\end{tabular}
\end{center}
%
A banner page for the child documents can be generated by:
%
\begin{center}
\begin{tabular}{l}
|\ifchilddoc|\\
|\addtocounter{page}{-1}|\\
\textit{code for banner page}\\
|\newpage|\\
|\||fi|
\end{tabular}
\end{center}
%
Here one could write a message such as:
\begin{center}
|This is the part \childdocname{} of \childdocjob{}.|
\end{center}

%%%%%%%%%%%%%%%%%%%%%%%%%%%%%%%%%%%%%%%%%%%%%%%%%%%%%%%%%%%%%%%%%%%%%%%%%%%%%%%%
\subsection{Flags}
\label{sec:flags}

The package makes it easy to generate different versions
of the main or child documents.
To this end compilation flags can be defined
and assigned different default values.
They will be particularly useful in conjunction
with the forwarding mechanism described in \secref{sec:forward}.

For example, it may be useful to have a flag |\version|
which can be set to |draft| or |final|.
The document source will contain some conditional code
depending on the value of |\version|.
Suppose further, the flag should default to |final| for the main file
and to |draft| for child files
which is a natural assignment for editing the document.
This is achieved by placing the following code
in the preamble of the main document
(below the |\childdocmain| directive):
%
\begin{center}
\begin{tabular}{l}
|\ifchilddoc|\\
|\providecommand{\version}{draft}|\\
|\||else|\\
|\providecommand{\version}{final}|\\
|\||fi|
\end{tabular}
\end{center}
%
The definition by |\providecommand| makes sure
that previous definitions are not overwritten.
Further statements |\providecommand{\version}{...}|
can thus be added before the above code to override it.

For the main file, one might add a line
(between |\childdocmain| and the above block)
%
\begin{center}
|%\ifchilddoc\||else\providecommand{\version}{draft}\||fi|
\end{center}
%
which can be uncommented to produce a draft version.
Likewise one can add a line to the very top of a child file
(above the |\childdocof{|\textit{main}|}| directive)
%
\begin{center}
|%\providecommand{\version}{final}|
\end{center}
%
which can be uncommented to produce the final version of this child document.

%%%%%%%%%%%%%%%%%%%%%%%%%%%%%%%%%%%%%%%%%%%%%%%%%%%%%%%%%%%%%%%%%%%%%%%%%%%%%%%%
\subsection{Forwarding}
\label{sec:forward}

Different versions of the main or child documents
using compilation flags as described in \secref{sec:flags}
can be (permanently) stored in different files
for convenient compilation, viewing and distribution.
To this end, the package defines a command
to pass on compilation to a different file:

%%%%%%%%%%%%%%%%%%%%%%%%%%%%%%%%%%%%%%%%
\DescribeMacro{\childdocforward}
The command |\childdocforward| redirects processing to
another source file:
%
\begin{center}
\begin{tabular}{l}
|\input{childdoc.def}|\\
|\childdocforward[|\textit{main}|]{|\textit{dest}|}|\\
\end{tabular}
\end{center}
%
The argument \textit{dest} is the destination file
(without extension).
It should be the main file or one of the child files.
Note that further \textsf{childdoc} directives
such as |\childdocof| and |\childdocforward|
in the indicated file will be processed in this form.
The optional argument \textit{main}
passes on directly to the main file \textit{main}
while pretending to compile the child \textit{dest}.
This form behaves as if \textit{dest}
issues |\childdocof{|\textit{main}|}| right away,
and no further \textsf{childdoc} directives will be processed.

%%%%%%%%%%%%%%%%%%%%%%%%%%%%%%%%%%%%%%%%
\DescribeMacro{\...prefix}
In the alternative form |\childdocforwardprefix|,
%
\begin{center}
\begin{tabular}{l}
|\input{childdoc.def}|\\
|\childdocforwardprefix[|\textit{main}|]{|\textit{prefix}|}{|\textit{dest}|}|
\end{tabular}
\end{center}
%
the destination file is determined by a pattern
depending on the current file:
To make this work, the current file must be called
`{\textit{prefix}\hspace{0.2em}\textit{suffix}}'
with \textit{prefix} matching precisely the argument.
Processing is then passed on to the file
`{\textit{dest}\hspace{0.2em}\textit{suffix}}'.
Surely, the same effect is achieved by
directly specifying the
argument `{\textit{dest}\hspace{0.2em}\textit{suffix}}'
in the first form.
However, that requires to set up a different file
for each child. With the alternative form of the command
all these files can have exactly the same content
which simplifies setting them up and maintaining them.

For example, the following file |draft.tex|
with a compilation flag |\version| as described in \secref{sec:flags}
compiles the main document as a draft:
%
\begin{center}
\begin{tabular}{l}
|\def\version{draft}|\\
|\input{childdoc.def}|\\
|\childdocforward{|\textit{main}|}|
\end{tabular}
\end{center}
%
Likewise, the following files |final|\textit{nn}|.tex|
compile the final version of the child document
|child|\textit{nn}|.tex|:
%
\begin{center}
\begin{tabular}{l}
|\def\version{final}|\\
|\input{childdoc.def}|\\
|\childdocforwardprefix{final}{child}|
\end{tabular}
\end{center}
%

Note that when several versions of a main file and/or of each child file
are to be generated, it may be convenient to set up a |Makefile| or
shell script to automatise the process.

%%%%%%%%%%%%%%%%%%%%%%%%%%%%%%%%%%%%%%%%%%%%%%%%%%%%%%%%%%%%%%%%%%%%%%%%%%%%%%%%
\subsection{Command Line Processing}
\label{sec:commandline}

The effect of redirection files can also be achieved by invoking
the \LaTeX{} compiler with a more elaborate command line.
Most conveniently this should be done as part
of a shell script or a |Makefile|.

When using \textsf{childdoc} in the main file, the following
command lines effectively perform a redirection
(note that depending on the shell being used,
backslashes may have to be doubled: `|\|' $\to$ `|\\|'):
%
\begin{center}
|... -jobname "|\textit{target}|" |\\|"|[\textit{flags}]%
|\input{childdoc.def}\childdocforward[|\textit{main}|]{|\textit{dest}|}"|
\end{center}
%
Here \textit{target} is the name of the output file,
\textit{main} is the name of the main file
and \textit{dest} is the name of the main or child file to be processed
(all filenames without extensions).
The optional argument \textit{main} can be omitted
if \textit{main} matches \textit{dest}.
Optionally, compilation \textit{flags} can be defined via |\def| commands.
This command line makes the \TeX{} engine believe
it is compiling the file \textit{target}
whose content is specified as the latter parameter.
The provided code then forwards the processing to
\textit{main} or \textit{dest} as described in \secref{sec:forward}.

%%%%%%%%%%%%%%%%%%%%%%%%%%%%%%%%%%%%%%%%%%%%%%%%%%%%%%%%%%%%%%%%%%%%%%%%%%%%%%%%
\subsection{Include by Input}
\label{sec:input}

Including child documents by |\include| has some restrictions by design.
Most notably, the content of a child document always occupies
its own set of pages; pages cannot be shared between child documents.
Usually, this behaviour makes perfect sense
because each child document contain an essential part of the document.
However, in some situations it may be desirable to compose
a document from a collection of parts
without having mandatory page breaks between then.
For this case, the package
provides a mechanism to include parts
by |\input| which can also be processed individually.
However, by construction this mechanism
requires manual handling of the content to be output.

%%%%%%%%%%%%%%%%%%%%%%%%%%%%%%%%%%%%%%%%
\DescribeMacro{\ifchilddocmanual}
The main file should be prepared as usual, see \secref{sec:include}.
However, the document body must make a distinction
between processing of an individual part and of the main document, e.g.:
%
\begin{center}
\begin{tabular}{l}
|\ifchilddocmanual|\\
|\input{\childdocname}|\\
|\||else|\\
\textit{document body with }|\input{|\textit{part}|}|\\
|\||fi|
\end{tabular}
\end{center}
%
The conditional |\ifchilddocmanual| is true whenever
a part to be included by |\input| is being compiled,
and the name of the part is stored in |\childdocname|.

%%%%%%%%%%%%%%%%%%%%%%%%%%%%%%%%%%%%%%%%
\DescribeMacro{\childdocby}
Each part to be included by |\input| should start with:
%
\begin{center}
\begin{tabular}{l}
|\input{childdoc.def}|\\
|\childdocby{|\textit{main}|}|\\
\end{tabular}
\end{center}
%
The directive |\childdocby| is similar to |\childdocof|
described in \secref{sec:include},
but the subsequent selection of content must be done manually.
To that end, both |\ifchilddoc| and |\ifchilddocmanual|
will be true upon processing of a part,
and the name of the part is stored in |\childdocname|.
Note that |\jobname| will be set to the filename of the current part
so that each part receives an individual |.aux| file
that does not interfere with the |.aux| file(s) of the main document.
This behaviour can be altered by the alternative form
|\childdocby[*]{|\textit{main}|}| (with a non-empty optional argument)
which uses the |.aux| file of the main document
by setting |\jobname| to \textit{main}.

%%%%%%%%%%%%%%%%%%%%%%%%%%%%%%%%%%%%%%%%%%%%%%%%%%%%%%%%%%%%%%%%%%%%%%%%%%%%%%%%
\subsection{Driver Development}
\label{sec:driver}

The \textsf{childdoc} mechanism can also be use for the development
of definition files such as \LaTeX{} styles or classes.
This case differs from the above setup with multiple parts
included by |\include| in that no |\includeonly| should be invoked.
This can be achieved by starting the include file
(before |\ProvidesPackage|) with:
%
\begin{center}
\begin{tabular}{l}
|\input{childdoc.def}|\\
|\childdocforward{|\textit{main}|}|\\
\end{tabular}
\end{center}
%
or alternatively with:
%
\begin{center}
\begin{tabular}{l}
|\input{childdoc.def}|\\
|\childdocby{|\textit{main}|}|\\
\end{tabular}
\end{center}
%
Both forms have slightly different effects as described above.
The main file is prepared as usual, see \secref{sec:include}.

%%%%%%%%%%%%%%%%%%%%%%%%%%%%%%%%%%%%%%%%%%%%%%%%%%%%%%%%%%%%%%%%%%%%%%%%%%%%%%%%
\subsection{Legacy Detection}
\label{sec:detection}

The directive |\childdocmain| in the main file can detect
whether the complete document or merely a child is to be compiled
even without using the directive |\childdocof|.
This method is deprecated because it is less robust
and there is no compelling reason to use it;
it is merely provided for backward compatibility
and it may be removed in future versions.

If the detection mechanism is to be used,
it is mandatory to correctly specify
the filename of the main file as the argument of |\childdocmain|:
%
\begin{center}
\begin{tabular}{l}
|\input{childdoc.def}|\\
|\childdocmain{|\textit{main}|}|\\
\end{tabular}
\end{center}
%
If |\jobname| does not match the argument \textit{main} of |\childdocmain|,
it is assumed that |\jobname| points to the child file to be compiled.
When using |\childdocmain| with the main file specified as argument,
it suffices to start a child file
with just |\input{|\textit{main}|}|
without loading of the package and using |\childdocof|.
If instead all processing is done
with the appropriate \textsf{childdoc} directives,
the argument of \textit{main} of |\childdocmain| can be empty.

An alternative version of the command line processing described
in \secref{sec:commandline} using the detection mechanism reads:
%
\begin{center}
|... -jobname "|\textit{target}|" "|[\textit{flags}]%
[|\def\jobname{|\textit{dest}|}|]|\input{|\textit{main}|}"|
\end{center}

%%%%%%%%%%%%%%%%%%%%%%%%%%%%%%%%%%%%%%%%%%%%%%%%%%%%%%%%%%%%%%%%%%%%%%%%%%%%%%%%
\subsection{Manual Code}
\label{sec:manual}

In case one cannot be certain whether the definitions file |childdoc.def|
is installed on the target \TeX{} distribution
and one prefers not to ship it,
it is conceivable to paste a few relevant commands into the sources.

To that end, drop all statements |\input{childdoc.def}|
and perform the replacements as outlined below.
Instead of |\childdocmain{|\textit{main}|}| add the following code
to the top of the main file:
%
\begin{center}
\begin{tabular}{l}
|\||ifdefined\childdocname\endinput\||fi\newif\ifchilddoc|\\
|\edef\childdocname{\scantokens\expandafter{\jobname\noexpand}}|\\
|\def\childdocmain{|\textit{main}|}\||ifx\childdocmain\childdocname\||else|\\
|\childdoctrue\includeonly{\childdocname}\let\jobname\childdocmain\||fi|\\
\end{tabular}
\end{center}
%
Instead of |\childdocof{|\textit{main}|}| just include the main file
at the top of each child file:
%
\begin{center}
|\input{|\textit{main}|}|
\end{center}
%
A simple redirection |\childdocforward{|\textit{dest}|}| is achieved by:
%
\begin{center}
|\def\jobname{|\textit{dest}|}\input{\jobname}|
\end{center}
%
The redirection with prefix
|\childdocforwardprefix[|\textit{prefix}|]{|\textit{dest}|}|
is accomplished by:
%
\begin{center}
\begin{tabular}{l}
|{\edef\jobname{\scantokens\expandafter{\jobname\noexpand}}|\\
|\def\redirectjob |\textit{prefix}|#1~~~{\gdef\jobname{|\textit{dest}|#1}}|\\
|\expandafter\redirectjob\jobname~~~}\input{\jobname}|
\end{tabular}
\end{center}

In an alternative approach,
child documents can be compiled by a specific command line
without additional code or specific definitions:
%
\begin{center}
|... -jobname "|\textit{target}|" "|[\textit{flags}]%
|\includeonly{|\textit{dest}|}\input{|\textit{main}|}"|
\end{center}
%

%%%%%%%%%%%%%%%%%%%%%%%%%%%%%%%%%%%%%%%%%%%%%%%%%%%%%%%%%%%%%%%%%%%%%%%%%%%%%%%%
%%%%%%%%%%%%%%%%%%%%%%%%%%%%%%%%%%%%%%%%%%%%%%%%%%%%%%%%%%%%%%%%%%%%%%%%%%%%%%%%
\section{Information}

%%%%%%%%%%%%%%%%%%%%%%%%%%%%%%%%%%%%%%%%%%%%%%%%%%%%%%%%%%%%%%%%%%%%%%%%%%%%%%%%
\subsection{Copyright}

Copyright \copyright{} 2017--2018 Niklas Beisert

This work may be distributed and/or modified under the
conditions of the \LaTeX{} Project Public License, either version 1.3
of this license or (at your option) any later version.
The latest version of this license is in
  \url{http://www.latex-project.org/lppl.txt}
and version 1.3 or later is part of all distributions of \LaTeX{}
version 2005/12/01 or later.

This work has the LPPL maintenance status `maintained'.

The Current Maintainer of this work is Niklas Beisert.

This work consists of the files |README.txt|, |childdoc.ins| and |childdoc.dtx|
as well as the derived files |childdoc.def|, |cdocsamp.tex|
with |cdocsch1.tex|, |cdocsch2.tex|, |cdocspt3.tex|, |cdocspt4.tex|,
|cdocsdrf.tex|, |cdocsfn1.tex|, |cdocsfn2.tex|
as well as |childdoc.pdf|.

%%%%%%%%%%%%%%%%%%%%%%%%%%%%%%%%%%%%%%%%%%%%%%%%%%%%%%%%%%%%%%%%%%%%%%%%%%%%%%%%
\subsection{Files and Installation}

The package consists of the files:
%
\begin{center}
\begin{tabular}{ll}
    |README.txt|   & readme file \\
    |childdoc.ins| & installation file \\
    |childdoc.dtx| & source file \\
    |childdoc.def| & definition file \\
    |cdocsamp.tex| & sample main file \\
    |cdocsch1.tex| & sample include file \\
    |cdocsch2.tex| & sample include file \\
    |cdocspt3.tex| & sample part file \\
    |cdocspt4.tex| & sample part file \\
    |cdocsdrf.tex| & sample redirection file \\
    |cdocsfn1.tex| & sample redirection file \\
    |cdocsfn2.tex| & sample redirection file \\
    |childdoc.pdf| & manual
\end{tabular}
\end{center}
%
The distribution consists of the files
|README.txt|, |childdoc.ins| and |childdoc.dtx|.
%
\begin{itemize}
\item
Run (pdf)\LaTeX{} on |childdoc.dtx|
to compile the manual |childdoc.pdf| (this file).
\item
Run \LaTeX{} on |childdoc.ins| to create the definitions file |childdoc.def|
and the sample |cdocsamp.tex| with include files
|cdocsch1.tex|, |cdocsch2.tex|, |cdocspt3.tex|, |cdocspt4.tex|,
|cdocsdrf.tex|, |cdocsfn1.tex|, |cdocsfn2.tex|.
Then copy the file |childdoc.def| to an appropriate directory of your \LaTeX{}
distribution, e.g.\ \textit{texmf-root}|/tex/latex/childdoc|.
\end{itemize}

%%%%%%%%%%%%%%%%%%%%%%%%%%%%%%%%%%%%%%%%%%%%%%%%%%%%%%%%%%%%%%%%%%%%%%%%%%%%%%%%
\subsection{Related CTAN Packages}

There are several other packages which offer a similar functionality:
%
\begin{itemize}
\item
The packages
\href{http://ctan.org/pkg/docmute}{\textsf{docmute}},
\href{http://ctan.org/pkg/includex}{\textsf{includex}} and
\href{http://ctan.org/pkg/standalone}{\textsf{standalone}}
provide commands to include only the document body of
a child file thus allowing both files to be compiled individually.
\item
The packages \href{http://ctan.org/pkg/subdocs}{\textsf{subdocs}}
and \href{http://ctan.org/pkg/subfiles}{\textsf{subfiles}}
provide structures in which the main and child documents can be
encapsulated and allowing them to be compiled individually.
The inclusion mechanism is different from the conventional |\include|.
\item
The package \href{http://ctan.org/pkg/combine}{\textsf{combine}}
is an elaborate solution to combine several documents into one.
\end{itemize}
%
See also the CTAN topic \href{http://ctan.org/topic/subdocs}{\textsf{subdocs}}
for further related packages.
The present package differs from the above solutions in that
a document structure constructed with the conventional |\include| mechanism
just needs two extra commands at the top of every file
such that all constituent files can be compiled individually.

%%%%%%%%%%%%%%%%%%%%%%%%%%%%%%%%%%%%%%%%%%%%%%%%%%%%%%%%%%%%%%%%%%%%%%%%%%%%%%%%
%\subsection{Feature Suggestions}
%
%The following is a list of features which may be useful for future
%versions of this package:
%%
%\begin{itemize}
%\item
%\ldots
%\end{itemize}

%%%%%%%%%%%%%%%%%%%%%%%%%%%%%%%%%%%%%%%%%%%%%%%%%%%%%%%%%%%%%%%%%%%%%%%%%%%%%%%%
\subsection{Revision History}

%%%%%%%%%%%%%%%%%%%%%%%%%%%%%%%%%%%%%%%%
\paragraph{v2.0:} 2018/12/30

\begin{itemize}
\item
immediate forward processing
\item
added |\childdocby| mechanism
\item
manual restructured
\end{itemize}

%%%%%%%%%%%%%%%%%%%%%%%%%%%%%%%%%%%%%%%%
\paragraph{v1.6:} 2018/01/17

\begin{itemize}
\item
application for development of include files
\item
corrections to manual
\end{itemize}

%%%%%%%%%%%%%%%%%%%%%%%%%%%%%%%%%%%%%%%%
\paragraph{v1.5:} 2017/05/21

\begin{itemize}
\item
more complete structuring introduced
\item
|\childdocof| introduced
\item
|\childdoc| renamed to |\childdocmain|
\item
|\childredirect| renamed to |\childdocforward| and |\childdocforwardprefix|
and functionality expanded
\end{itemize}

%%%%%%%%%%%%%%%%%%%%%%%%%%%%%%%%%%%%%%%%
\paragraph{v1.0:} 2017/04/27

\begin{itemize}
\item
manual and install package
\item
first version published on CTAN
\end{itemize}

%%%%%%%%%%%%%%%%%%%%%%%%%%%%%%%%%%%%%%%%
\paragraph{v0.6:} 2017/04/26

\begin{itemize}
\item
redirection mechanism added
\end{itemize}

%%%%%%%%%%%%%%%%%%%%%%%%%%%%%%%%%%%%%%%%
\paragraph{v0.5:} 2017/04/26

\begin{itemize}
\item
functionality in definition file
\end{itemize}


%%%%%%%%%%%%%%%%%%%%%%%%%%%%%%%%%%%%%%%%%%%%%%%%%%%%%%%%%%%%%%%%%%%%%%%%%%%%%%%%
%%%%%%%%%%%%%%%%%%%%%%%%%%%%%%%%%%%%%%%%%%%%%%%%%%%%%%%%%%%%%%%%%%%%%%%%%%%%%%%%
%%%%%%%%%%%%%%%%%%%%%%%%%%%%%%%%%%%%%%%%%%%%%%%%%%%%%%%%%%%%%%%%%%%%%%%%%%%%%%%%
\appendix

\settowidth\MacroIndent{\rmfamily\scriptsize 000\ }

 \DocInput{childdoc.dtx}

\end{document}
%</driver>
% \fi
%
% %%%%%%%%%%%%%%%%%%%%%%%%%%%%%%%%%%%%%%%%%%%%%%%%%%%%%%%%%%%%%%%%%%%%%%%%%%%%%%
% %%%%%%%%%%%%%%%%%%%%%%%%%%%%%%%%%%%%%%%%%%%%%%%%%%%%%%%%%%%%%%%%%%%%%%%%%%%%%%
% \section{Sample}
%\iffalse
%<*samplemain>
%\fi
%
% The following presents a sample document
% with two chapters, two parts, a title page,
% a compile flag as well as three forwarding files to set the flag.
% It consists of eight |.tex| files:
% \begin{center}
% \begin{tabular}{ll}
% |cdocsamp.tex|&main file\\
% |cdocsch1.tex|&include file for chapter 1\\
% |cdocsch2.tex|&include file for chapter 2\\
% |cdocspt3.tex|&include file for part 3\\
% |cdocspt4.tex|&include file for part 4\\
% |cdocsdrf.tex|&forwarding file for main file in draft mode\\
% |cdocsfi1.tex|&forwarding file for final version of chapter 1\\
% |cdocsfi2.tex|&forwarding file for final version of chapter 2\\
% \end{tabular}
% \end{center}
% Each of the eight files can be compiled directly by the \LaTeX{} compiler.
%
% %%%%%%%%%%%%%%%%%%%%%%%%%%%%%%%%%%%%%%
% \paragraph{Main File.}
%
% The main file is called |cdocsamp.tex|.
%
% Load the \textsf{childdoc} definitions and
% declare the filename for the main document:
%    \begin{macrocode}
\input{childdoc.def}
\childdocmain{}
%    \end{macrocode}

% Optional override for |\version| flag:
%    \begin{macrocode}
%%\ifchilddoc\else\providecommand{\version}{draft}\fi
%    \end{macrocode}

% Define the default values for the |\version| flag
% (|final| for the main file and |draft| for childs):
%    \begin{macrocode}
\ifchilddoc
\providecommand{\version}{draft}
\else
\providecommand{\version}{final}
\fi
%    \end{macrocode}

% Load the standard document class:
%    \begin{macrocode}
\documentclass[12pt]{article}
%    \end{macrocode}

% Start the document body:
%    \begin{macrocode}
\begin{document}
%    \end{macrocode}

% Declare a title page.
% Print title, part of document being processed and version flag:
%    \begin{macrocode}
\addtocounter{page}{-1}
\begin{center}
{\LARGE\bfseries{}childdoc example\par}
\vspace{1cm}
\ifchilddoc
\ifchilddocmanual part\else chapter\fi:
`\childdocname' of `\childdocjob'\par
\else
main document: `\childdocjob'\par
\fi
version: \version\par
\end{center}
\newpage
%    \end{macrocode}

% Manually include selected file,
% otherwise process as usual:
%    \begin{macrocode}
\ifchilddocmanual
\section*{part `\childdocname'}
\input{\childdocname}
\else
%    \end{macrocode}

% Include the two chapters:
%    \begin{macrocode}
\include{cdocsch1}
\include{cdocsch2}
%    \end{macrocode}

% Include the two parts unless only chapters should be displayed:
%    \begin{macrocode}
\ifchilddoc\else
\section{part three}
\input{cdocspt3}
\section{part four}
\input{cdocspt4}
\fi
%    \end{macrocode}

% Process as usual until here:
%    \begin{macrocode}
\fi
%    \end{macrocode}

% End of document body:
%    \begin{macrocode}
\end{document}
%    \end{macrocode}
%\iffalse
%</samplemain>
%\fi
%
% %%%%%%%%%%%%%%%%%%%%%%%%%%%%%%%%%%%%%%
% \paragraph{Chapter Include Files.}
%
% The include files are called |cdocsch1.tex| and |cdocsch2.tex|.
%
%\iffalse
%<*samplechap1|samplechap2>
%\fi

% Optional override for |\version| flag:
%    \begin{macrocode}
%%\providecommand{\version}{final}
%    \end{macrocode}

% Include the main document:
%    \begin{macrocode}
\input{childdoc.def}
\childdocof{cdocsamp}
%    \end{macrocode}

%\iffalse
%</samplechap1|samplechap2>
%\fi
%
%\iffalse
%<*samplechap1>
%\fi
% Some text for chapter 1:
%    \begin{macrocode}
\section{one}
some text in chapter one
%    \end{macrocode}

%\iffalse
%</samplechap1>
%\fi
% Some text for chapter 2:
%\iffalse
%<*samplechap2>
%\fi
%    \begin{macrocode}
\section{two}
more text in chapter two
%    \end{macrocode}

%\iffalse
%</samplechap2>
%\fi
%
% %%%%%%%%%%%%%%%%%%%%%%%%%%%%%%%%%%%%%%
% \paragraph{Part Include Files.}
%
% The include files are called |cdocspt3.tex| and |cdocspt4.tex|.
%
%\iffalse
%<*samplepart3|samplepart4>
%\fi

% Optional override for |\version| flag:
%    \begin{macrocode}
%%\providecommand{\version}{final}
%    \end{macrocode}

% Include the main document:
%    \begin{macrocode}
\input{childdoc.def}
\childdocby{cdocsamp}
%    \end{macrocode}

%\iffalse
%</samplepart3|samplepart4>
%\fi
%
%\iffalse
%<*samplepart3>
%\fi
% Some text for part 3:
%    \begin{macrocode}
some text in part three
%    \end{macrocode}

%\iffalse
%</samplepart3>
%\fi
% Some text for part 4:
%\iffalse
%<*samplepart4>
%\fi
%    \begin{macrocode}
more text in part four
%    \end{macrocode}

%\iffalse
%</samplepart4>
%\fi
%
% %%%%%%%%%%%%%%%%%%%%%%%%%%%%%%%%%%%%%%
% \paragraph{Forwarding for a Complete Draft.}
%
% The following forwarding file |cdocsdrf.tex|
% compiles the main document in draft mode:
%\iffalse
%<*sampledraft>
%\fi
%    \begin{macrocode}
\def\version{draft}
\input{childdoc.def}
\childdocforward{cdocsamp}
%    \end{macrocode}

%\iffalse
%</sampledraft>
%\fi
%
% %%%%%%%%%%%%%%%%%%%%%%%%%%%%%%%%%%%%%%
% \paragraph{Forwarding for Final Version of the Chapters.}
%
% The following forwarding files |cdocsfn1.tex| and |cdocsfn2.tex|
% (with identical content)
% compile the final versions of the child documents
% |cdocsch1.tex| and |cdocsch2.tex|, respectively:
%\iffalse
%<*samplefinal>
%\fi
%    \begin{macrocode}
\def\version{final}
\input{childdoc.def}
\childdocforwardprefix[cdocsamp]{cdocsfn}{cdocsch}
%    \end{macrocode}

%\iffalse
%</samplefinal>
%\fi
%
% %%%%%%%%%%%%%%%%%%%%%%%%%%%%%%%%%%%%%%
% \paragraph{Command Line Processing.}
%
% The following three command lines generate the output files
% |cdocscld|, |cdocscl1| and |cdocscl2|
% which should be identical to
% |cdocsdrf|, |cdocsch1| and |cdocsfn2|, respectively:
% \begin{center}
% \begin{tabular}{l}
% |latex -jobname cdocscld \|\\
% |  "\def\version{draft}\input{childdoc.def}\childdocforward{cdocsamp}"|\\
% |latex -jobname cdocscl1 \|\\
% |  "\input{childdoc.def}\childdocforward[cdocsamp]{cdocsch1}"|\\
% |latex -jobname cdocscl2 \|\\
% |  "\def\version{final}\input{childdoc.def}\childdocforward{cdocsch2}"|
% \end{tabular}
% \end{center}
% Note that the trailing backslash on each first line
% merely continues the input to the second line
% (for convenient cut ant paste).
% Furthermore, the command |latex| can be replaced by any
% of its alternative versions such as |pdflatex|.
%
% %%%%%%%%%%%%%%%%%%%%%%%%%%%%%%%%%%%%%%%%%%%%%%%%%%%%%%%%%%%%%%%%%%%%%%%%%%%%%%
% %%%%%%%%%%%%%%%%%%%%%%%%%%%%%%%%%%%%%%%%%%%%%%%%%%%%%%%%%%%%%%%%%%%%%%%%%%%%%%
% \section{Implementation}
%\iffalse
%<*package>
%\fi
%
% This section describes the definitions file |childdoc.def|.

% The definitions cannot be loaded using |\usepackage| or |\RequirePackage|
% which has a mechanism to prevent loading a style file more than once.
% When loading the definitions by means of |\input|
% multiple instances have to be prevented manually:
%\iffalse
%This code needs to be before the `\ProvidesFile' directive
%which is defined at the beginning of this file.
%Therefore it is also placed there and commented out here.
%</package>
%<*discard>
%\fi
%    \begin{macrocode}
\ifdefined\childdocmain\endinput\fi
%    \end{macrocode}
%\iffalse
%</discard>
%<*package>
%\fi
%
% \macro{\ifchilddoc}
% \macro{\ifchilddocmanual}
% The conditional |\ifchilddoc| tells whether a
% child (true) or main (false) document is being compiled.
% The conditional |\ifchilddocmanual| tells whether
% the |\includeonly| mechanism is used (false) or
% the selection of child files must be performed manually (true).
% The definitions initialise to false:
%    \begin{macrocode}
\newif\ifchilddoc
\newif\ifchilddocmanual
%    \end{macrocode}

% \macro{\childdocname}
% \macro{\childdocjob}
% The macro |\childdocname| stores the name of the main document
% to be compiled. The macro |\childdocjob| stores the name of
% the document on which the \LaTeX{} compiler was originally invoked.
% The content of |\jobname| cannot be compared
% to filenames specified in the source due to different catcodes.
% The following code rescans |\jobname|, stores the result
% in |\childdocname| and saves a copy in |\childdocjob|:
%    \begin{macrocode}
\edef\childdocname{\scantokens\expandafter{\jobname\noexpand}}
\let\childdocjob\childdocname
%    \end{macrocode}

% \macro{\childdocdisable}
% The macro |\childdocdisable| prevents the main file
% from being processed more than once.
% At this stage, the main document command |\childdocmain|
% is assumed to be called once again where it should do nothing.
% Any subsequent call to it should prevent
% a secondary processing of the main document
% It overwrites the forwarding commands
% |\childdocof| and |\childdocforward|
% with empty macros to prevent further inclusions of the main document:
%    \begin{macrocode}
\newcommand{\childdocdisable}
{
  \renewcommand{\childdocmain}[1]{\renewcommand{\childdocmain}[1]{\endinput}}
  \renewcommand{\childdocof}[1]{}
  \renewcommand{\childdocby}[2][]{}
  \renewcommand{\childdocforward}[2][]{}
  \renewcommand{\childdocdisable}{}
}
%    \end{macrocode}

% \macro{\childdocmain}
% The macro |\childdocmain| is to be called at the top of the main file
% with nothing or the main filename (without extension) as argument.
% First, it breaks loops.
% If the argument is not empty and does not match |\childdocname|
% (which is set by the first inclusion of |childdoc.def|),
% |\ifchilddoc| is set to true, |\includeonly| is applied to the child file
% and |\jobname| is set to the main file
% (for proper handling of |.aux| files):
%    \begin{macrocode}
\newcommand{\childdocmain}[1]
{
  \childdocdisable\childdocmain{}
  \if?#1?\else
    \begingroup
      \def\childdoctmp{#1}
      \ifx\childdoctmp\childdocname
        \def\childdoctmp{}
      \else
        \def\childdoctmp
        {
          \childdoctrue
          \includeonly{\childdocname}
          \def\childdocjob{#1}
          \def\jobname{#1}
        }
      \fi
      \expandafter
    \endgroup
    \childdoctmp
  \fi
}
%    \end{macrocode}

% \macro{\childdocof}
% The command |\childdocof| redirects
% compilation to the main file |#1|.
%    \begin{macrocode}
\newcommand{\childdocof}[1]
{
  \childdocdisable
  \childdoctrue
  \includeonly{\childdocname}
  \def\jobname{#1}
  \def\childdocjob{#1}
  \input{#1}
}
%    \end{macrocode}

% \macro{\childdocby}
% The command |\childdocby| ....
%    \begin{macrocode}
\newcommand{\childdocby}[2][]
{
  \childdocdisable
  \childdoctrue
  \childdocmanualtrue
  \if?#1?\else
    \def\jobname{#2}
  \fi
  \def\childdocjob{#2}
  \input{#2}
  \endinput
}
%    \end{macrocode}

% \macro{\childdocforward}
% The command |\childdocforward| redirects
% compilation to the main file or
% (if the optional argument is given) a child file.
% Parameters are set as if the main file
% or a child file starting with |\childdocof| was compiled.
% Then compilation is handed over to the main file:
%    \begin{macrocode}
\newcommand{\childdocforward}[2][]
{
  \begingroup
    \if?#1?
      \def\childdoctmp
      {
        \def\childdocname{#2}
        \def\childdocjob{#2}
        \def\jobname{#2}
        \input{#2}
        \endinput
      }
    \else
      \def\childdoctmp
      {
        \childdocdisable
        \def\childdocname{#2}
        \childdoctrue
        \includeonly{#2}
        \def\childdocjob{#1}
        \def\jobname{#1}
        \input{#1}
        \endinput
      }
    \fi
    \expandafter
  \endgroup
  \childdoctmp
}
%    \end{macrocode}

% \macro{\childdocforwardprefix}
% The command |\childdocforwardprefix| redirects
% compilation to the main or a child file by means of a pattern.
% The prefix |#1| in the current filename is replaced by |#2|
% and the suffix of the current filename is kept
% (it is assumed that the filename does not contain the substring `|~~~|'
% which is used as a delimiter).
% Compilation is handed over to the new file by |\childdocforward|:
%    \begin{macrocode}
\newcommand{\childdocforwardprefix}[3][]
{
  \begingroup
    \def\childdocextract #2##1~~~{\def\childdoctmp{\childdocforward[#1]{#3##1}}}
    \expandafter\childdocextract\childdocname~~~
    \expandafter
  \endgroup
  \childdoctmp
}
%    \end{macrocode}

% \macro{\childdoc}
% The deprecated macro |\childdoc| is a legacy version of |\childdocmain|:
%    \begin{macrocode}
\newcommand{\childdoc}{\childdocmain}
%    \end{macrocode}

% \macro{\childdocredirect}
% The deprecated macro |\childdocredirect| is a legacy version
% of |\childdocforward| and |\childdocforwardprefix|:
%    \begin{macrocode}
\newcommand{\childdocredirect}[2][]
{
  \begingroup
    \if?#1?
      \def\childdoctmp{\childdocforward{#2}}
    \else
      \def\childdoctmp{\childdocforwardprefix{#1}{#2}}
    \fi
    \expandafter
  \endgroup
  \childdoctmp
}
%    \end{macrocode}

%\iffalse
%</package>
%\fi
%
\endinput
|\\
|\childdocmain{}|\\
\end{tabular}
\end{center}
at the very top of the main \LaTeX{} file,
in particular \emph{before} the |\documentclass| statement!
The argument of |\childdocmain| should be left empty
(but it must be present).

%%%%%%%%%%%%%%%%%%%%%%%%%%%%%%%%%%%%%%%%
\DescribeMacro{\childdocof}
Furthermore, add the commands
\begin{center}
\begin{tabular}{l}
|% \iffalse
%
% childdoc.dtx Copyright (C) 2017-2018 Niklas Beisert
%
% This work may be distributed and/or modified under the
% conditions of the LaTeX Project Public License, either version 1.3
% of this license or (at your option) any later version.
% The latest version of this license is in
%   http://www.latex-project.org/lppl.txt
% and version 1.3 or later is part of all distributions of LaTeX
% version 2005/12/01 or later.
%
% This work has the LPPL maintenance status `maintained'.
%
% The Current Maintainer of this work is Niklas Beisert.
%
% This work consists of the files childdoc.dtx and childdoc.ins
% and the derived files childdoc.def and cdocsamp.tex with
% cdocsch1.tex, cdocsch2.tex, cdocsdrf.tex, cdocsfn1.tex, cdocsfn2.tex.
%
%<package>\ifdefined\childdocmain\endinput\fi
%<package>\ProvidesFile{childdoc.def}[2018/12/30 v2.0 child document driver]
%<samplemain>\ProvidesFile{cdocsamp.tex}[2018/12/30 v2.0 sample for childdoc]
%<*driver>
%\ProvidesFile{childdoc.drv}[2018/12/30 v2.0 childdoc reference manual file]
\PassOptionsToClass{10pt,a4paper}{article}
\documentclass{ltxdoc}

\usepackage[margin=35mm]{geometry}
\usepackage{hyperref}
\usepackage{hyperxmp}
\usepackage[usenames]{color}

\hypersetup{colorlinks=true}
\hypersetup{pdfstartview=FitH}
\hypersetup{pdfpagemode=UseNone}
\hypersetup{pdfsource={}}
\hypersetup{pdflang={en-UK}}
\hypersetup{pdfcopyright={Copyright 2017-2018 Niklas Beisert.
  This work may be distributed and/or modified under the
  conditions of the LaTeX Project Public License, either version 1.3
  of this license or (at your option) any later version.}}
\hypersetup{pdflicenseurl={http://www.latex-project.org/lppl.txt}}
\hypersetup{pdfcontactaddress={ETH Zurich, ITP, HIT K,
  Wolfgang-Pauli-Strasse 27}}
\hypersetup{pdfcontactpostcode={8093}}
\hypersetup{pdfcontactcity={Zurich}}
\hypersetup{pdfcontactcountry={Switzerland}}
\hypersetup{pdfcontactemail={nbeisert@itp.phys.ethz.ch}}
\hypersetup{pdfcontacturl={http://people.phys.ethz.ch/\xmptilde nbeisert/}}

\newcommand{\secref}[1]{\hyperref[#1]{section \ref*{#1}}}

\parskip1ex
\parindent0pt
\let\olditemize\itemize
\def\itemize{\olditemize\parskip0pt}

\begin{document}

\title{The \textsf{childdoc} Package}
\hypersetup{pdftitle={The childdoc Package}}
\author{Niklas Beisert\\[2ex]
  Institut f\"ur Theoretische Physik\\
  Eidgen\"ossische Technische Hochschule Z\"urich\\
  Wolfgang-Pauli-Strasse 27, 8093 Z\"urich, Switzerland\\[1ex]
  \href{mailto:nbeisert@itp.phys.ethz.ch}
  {\texttt{nbeisert@itp.phys.ethz.ch}}}
\hypersetup{pdfauthor={Niklas Beisert}}
\hypersetup{pdfsubject={Manual for the LaTeX2e Package childdoc}}
\date{30 December 2018, \textsf{v2.0}}
\maketitle

\begin{abstract}\noindent
\textsf{childdoc} is a \LaTeXe{} package
that enables the direct compilation
of document sections included by |\include|
to individual files.
\end{abstract}

\begingroup
\parskip0ex
\tableofcontents
\endgroup

%%%%%%%%%%%%%%%%%%%%%%%%%%%%%%%%%%%%%%%%%%%%%%%%%%%%%%%%%%%%%%%%%%%%%%%%%%%%%%%%
%%%%%%%%%%%%%%%%%%%%%%%%%%%%%%%%%%%%%%%%%%%%%%%%%%%%%%%%%%%%%%%%%%%%%%%%%%%%%%%%
\section{Introduction}

\LaTeX{} provides a mechanism to structure a large document (such as a book)
into a main file and several child files (containing the chapters)
using the |\include| command.
This mechanism is beneficial for documents
which span hundreds of pages in order to
make the source file(s) more manageable.
Moreover, compilation can be restricted to
selected child files by means of the |\includeonly| command.
The latter feature can be used to reduce the compilation time while editing
(this was significantly more useful in the earlier days of \LaTeX{})
or to generate a smaller document which is easier to navigate.
Another application of |\includeonly| is to generate
documents consisting of selected parts of the complete document.

However, there are a few drawbacks of the plain |\include| mechanism:
\begin{itemize}
\item
The child files cannot be compiled on their own,
they can only be compiled via the main file.
A naive editing environment
(such as a text editor with an option
to have the current file processed by \LaTeX)
may require one to switch to the main file before compiling;
attempting to compile the child file produces errors.
\item
The main file must be modified (each time)
to adjust the |\includeonly| command
to the present needs. This easily leaves the main file in a messy state.
\item
The generated document will always carry the filename
of the main document. This is inconvenient if
several child files are to be compiled and
to be kept for distribution.
\end{itemize}

The present package provides a simple interface
to make child files individually compilable by \LaTeX{}.
Compiling a child file then has the same effect as compiling
the main file with an |\includeonly| command
to select the appropriate child.
Moreover the generated document will carry the name of the child
rather than the main file.
This resolves all three above issues.

This feature is meant to make the editing of books,
thesis documents and lecture notes somewhat more convenient.
However, the package can also be used efficiently for
composing a series of documents (such as exercise sheets)
which are typically distributed individually.
It then assists the author in generating the individual documents
(potentially in different versions)
as well as a document containing the collected series.
Another application is in developing style files
or other kinds of included material
where compilation of the style file could redirect
to a sample or test file.

%%%%%%%%%%%%%%%%%%%%%%%%%%%%%%%%%%%%%%%%%%%%%%%%%%%%%%%%%%%%%%%%%%%%%%%%%%%%%%%%
%%%%%%%%%%%%%%%%%%%%%%%%%%%%%%%%%%%%%%%%%%%%%%%%%%%%%%%%%%%%%%%%%%%%%%%%%%%%%%%%
\section{Usage}

First of all, the package \textsf{childdoc} is \emph{not} a standard
\LaTeXe{} |.sty| style file! Therefore it needs to be invoked in
a non-standard way.

%%%%%%%%%%%%%%%%%%%%%%%%%%%%%%%%%%%%%%%%%%%%%%%%%%%%%%%%%%%%%%%%%%%%%%%%%%%%%%%%
\subsection{Included Files}
\label{sec:include}

%%%%%%%%%%%%%%%%%%%%%%%%%%%%%%%%%%%%%%%%
\DescribeMacro{\childdocmain}
To use the package, add the commands
\begin{center}
\begin{tabular}{l}
|\input{childdoc.def}|\\
|\childdocmain{}|\\
\end{tabular}
\end{center}
at the very top of the main \LaTeX{} file,
in particular \emph{before} the |\documentclass| statement!
The argument of |\childdocmain| should be left empty
(but it must be present).

%%%%%%%%%%%%%%%%%%%%%%%%%%%%%%%%%%%%%%%%
\DescribeMacro{\childdocof}
Furthermore, add the commands
\begin{center}
\begin{tabular}{l}
|\input{childdoc.def}|\\
|\childdocof{|\textit{main}|}|\\
\end{tabular}
\end{center}
at the top of every child file \textit{child}
which is included by |\include{|\textit{child}|}|
from within the main file
(or at least for those files to be compiled individually).
The argument \textit{main} must be the filename of the main file.

There are a couple of
considerations in setting up the main and child documents:

%%%%%%%%%%%%%%%%%%%%%%%%%%%%%%%%%%%%%%%%
\paragraph{Restrictions.}

Please note the following restrictions:
\begin{itemize}
\item
|\childdocmain| must be called with one argument \textit{main}
to ensure compatibility with earlier version of the package.
It must either be empty (|\childdocmain{}|)
or precisely match the filename of the main file in which it is specified.
See \secref{sec:detection} for further information.
\item
The filename \textit{main} must be specified without the |.tex| extension.
\item
The filename \textit{main} is case sensitive
(even in case-insensitive file systems)
due to internal string comparison.
\item
The argument \textit{main} should be fully expanded, it cannot be a macro.
\item
Subdirectories and special characters should be avoided in filenames.
\item
The command |\childdocmain{|\textit{main}|}| must be followed by a whitespace.
It should not be followed immediately by another command
or by a comment mark `|%|'.
This is because the \TeX{} parser reads the token immediately following
the argument of |\childdocmain| and puts it
at the beginning of every child section;
however, a white\-space is ignored.
\end{itemize}

%%%%%%%%%%%%%%%%%%%%%%%%%%%%%%%%%%%%%%%%
\paragraph{Content of Main File.}

It is advisable to place all content in the child files included by |\include|.
Any output contained in the main file will appear in all child documents
unless suppressed manually;
it cannot be suppressed automatically by the |\includeonly| directive
and thus should normally be avoided.
A method to include some content in the main file
by means of conditional processing is described in \secref{sec:conditional}.

%%%%%%%%%%%%%%%%%%%%%%%%%%%%%%%%%%%%%%%%
\paragraph{Page Numbering.}

When only a part of the document is compiled,
the appropriate numbering of pages
(as well as other status parameters)
is determined from the |.aux| files.
The latter contain information from previous passes.
However this information needs to propagate through
all intermediate child documents.
Therefore the page numbering in child documents may well
be inconsistent until the complete document is compiled at least once.

A useful (if unconventional) way to always ensure a consistent
page numbering is to restart the numbering in each child document
and denote the pages by `\textit{child}|.|\textit{page}'
where \textit{child} represents the chapter/section number of the child file.
This can be achieved by the command
|\numberwithin{page}{|\textit{child}|}|
of the \textsf{amsmath} package
where \textit{child} can be |chapter| or |section|
depending on the chosen structuring.
Alternatively, one can modify the macro |\thepage| appropriately
and reset the counter |page| at the start of each child file.

%%%%%%%%%%%%%%%%%%%%%%%%%%%%%%%%%%%%%%%%%%%%%%%%%%%%%%%%%%%%%%%%%%%%%%%%%%%%%%%%
\subsection{Conditional Processing}
\label{sec:conditional}

The package provides a mechanism to compile different versions
of a document. To customise the versions further some conditional processing
can come in handy to distinguish which version is being compiled.
The package provides two macros to describe the compilation context:

%%%%%%%%%%%%%%%%%%%%%%%%%%%%%%%%%%%%%%%%
\DescribeMacro{\ifchilddoc}
The conditional |\ifchilddoc| distinguishes between the compilation of
child documents and the main document:
%
\begin{center}
|\ifchilddoc |\textit{child-code}| |[|\||else |\textit{main-code}]| \||fi|
\end{center}

%%%%%%%%%%%%%%%%%%%%%%%%%%%%%%%%%%%%%%%%
\DescribeMacro{\childdocname}
\DescribeMacro{\childdocjob}
The macro |\childdocname| contains the filename (without extension)
of the main or child file being processed.
Note that |\childdocjob| will always contain the name of the main file.

%%%%%%%%%%%%%%%%%%%%%%%%%%%%%%%%%%%%%%%%
\paragraph{Title Page.}

Conditional processing can be used to include a title or banner page
in the main document when proper precautions are taken.
Importantly, the code in the main file should ensure that the page counter
(as well as other status parameters which are stored in the |.aux| files)
takes the same value after the conditional processing.
Otherwise the page numbers may take divergent values
depending on which part is compiled.

For example, a title page could be declared by:
%
\begin{center}
\begin{tabular}{l}
|\ifchilddoc\||else|\\
|\addtocounter{page}{-1}|\\
\textit{code for title page}\\
|\newpage|\\
|\||fi|
\end{tabular}
\end{center}
%
A banner page for the child documents can be generated by:
%
\begin{center}
\begin{tabular}{l}
|\ifchilddoc|\\
|\addtocounter{page}{-1}|\\
\textit{code for banner page}\\
|\newpage|\\
|\||fi|
\end{tabular}
\end{center}
%
Here one could write a message such as:
\begin{center}
|This is the part \childdocname{} of \childdocjob{}.|
\end{center}

%%%%%%%%%%%%%%%%%%%%%%%%%%%%%%%%%%%%%%%%%%%%%%%%%%%%%%%%%%%%%%%%%%%%%%%%%%%%%%%%
\subsection{Flags}
\label{sec:flags}

The package makes it easy to generate different versions
of the main or child documents.
To this end compilation flags can be defined
and assigned different default values.
They will be particularly useful in conjunction
with the forwarding mechanism described in \secref{sec:forward}.

For example, it may be useful to have a flag |\version|
which can be set to |draft| or |final|.
The document source will contain some conditional code
depending on the value of |\version|.
Suppose further, the flag should default to |final| for the main file
and to |draft| for child files
which is a natural assignment for editing the document.
This is achieved by placing the following code
in the preamble of the main document
(below the |\childdocmain| directive):
%
\begin{center}
\begin{tabular}{l}
|\ifchilddoc|\\
|\providecommand{\version}{draft}|\\
|\||else|\\
|\providecommand{\version}{final}|\\
|\||fi|
\end{tabular}
\end{center}
%
The definition by |\providecommand| makes sure
that previous definitions are not overwritten.
Further statements |\providecommand{\version}{...}|
can thus be added before the above code to override it.

For the main file, one might add a line
(between |\childdocmain| and the above block)
%
\begin{center}
|%\ifchilddoc\||else\providecommand{\version}{draft}\||fi|
\end{center}
%
which can be uncommented to produce a draft version.
Likewise one can add a line to the very top of a child file
(above the |\childdocof{|\textit{main}|}| directive)
%
\begin{center}
|%\providecommand{\version}{final}|
\end{center}
%
which can be uncommented to produce the final version of this child document.

%%%%%%%%%%%%%%%%%%%%%%%%%%%%%%%%%%%%%%%%%%%%%%%%%%%%%%%%%%%%%%%%%%%%%%%%%%%%%%%%
\subsection{Forwarding}
\label{sec:forward}

Different versions of the main or child documents
using compilation flags as described in \secref{sec:flags}
can be (permanently) stored in different files
for convenient compilation, viewing and distribution.
To this end, the package defines a command
to pass on compilation to a different file:

%%%%%%%%%%%%%%%%%%%%%%%%%%%%%%%%%%%%%%%%
\DescribeMacro{\childdocforward}
The command |\childdocforward| redirects processing to
another source file:
%
\begin{center}
\begin{tabular}{l}
|\input{childdoc.def}|\\
|\childdocforward[|\textit{main}|]{|\textit{dest}|}|\\
\end{tabular}
\end{center}
%
The argument \textit{dest} is the destination file
(without extension).
It should be the main file or one of the child files.
Note that further \textsf{childdoc} directives
such as |\childdocof| and |\childdocforward|
in the indicated file will be processed in this form.
The optional argument \textit{main}
passes on directly to the main file \textit{main}
while pretending to compile the child \textit{dest}.
This form behaves as if \textit{dest}
issues |\childdocof{|\textit{main}|}| right away,
and no further \textsf{childdoc} directives will be processed.

%%%%%%%%%%%%%%%%%%%%%%%%%%%%%%%%%%%%%%%%
\DescribeMacro{\...prefix}
In the alternative form |\childdocforwardprefix|,
%
\begin{center}
\begin{tabular}{l}
|\input{childdoc.def}|\\
|\childdocforwardprefix[|\textit{main}|]{|\textit{prefix}|}{|\textit{dest}|}|
\end{tabular}
\end{center}
%
the destination file is determined by a pattern
depending on the current file:
To make this work, the current file must be called
`{\textit{prefix}\hspace{0.2em}\textit{suffix}}'
with \textit{prefix} matching precisely the argument.
Processing is then passed on to the file
`{\textit{dest}\hspace{0.2em}\textit{suffix}}'.
Surely, the same effect is achieved by
directly specifying the
argument `{\textit{dest}\hspace{0.2em}\textit{suffix}}'
in the first form.
However, that requires to set up a different file
for each child. With the alternative form of the command
all these files can have exactly the same content
which simplifies setting them up and maintaining them.

For example, the following file |draft.tex|
with a compilation flag |\version| as described in \secref{sec:flags}
compiles the main document as a draft:
%
\begin{center}
\begin{tabular}{l}
|\def\version{draft}|\\
|\input{childdoc.def}|\\
|\childdocforward{|\textit{main}|}|
\end{tabular}
\end{center}
%
Likewise, the following files |final|\textit{nn}|.tex|
compile the final version of the child document
|child|\textit{nn}|.tex|:
%
\begin{center}
\begin{tabular}{l}
|\def\version{final}|\\
|\input{childdoc.def}|\\
|\childdocforwardprefix{final}{child}|
\end{tabular}
\end{center}
%

Note that when several versions of a main file and/or of each child file
are to be generated, it may be convenient to set up a |Makefile| or
shell script to automatise the process.

%%%%%%%%%%%%%%%%%%%%%%%%%%%%%%%%%%%%%%%%%%%%%%%%%%%%%%%%%%%%%%%%%%%%%%%%%%%%%%%%
\subsection{Command Line Processing}
\label{sec:commandline}

The effect of redirection files can also be achieved by invoking
the \LaTeX{} compiler with a more elaborate command line.
Most conveniently this should be done as part
of a shell script or a |Makefile|.

When using \textsf{childdoc} in the main file, the following
command lines effectively perform a redirection
(note that depending on the shell being used,
backslashes may have to be doubled: `|\|' $\to$ `|\\|'):
%
\begin{center}
|... -jobname "|\textit{target}|" |\\|"|[\textit{flags}]%
|\input{childdoc.def}\childdocforward[|\textit{main}|]{|\textit{dest}|}"|
\end{center}
%
Here \textit{target} is the name of the output file,
\textit{main} is the name of the main file
and \textit{dest} is the name of the main or child file to be processed
(all filenames without extensions).
The optional argument \textit{main} can be omitted
if \textit{main} matches \textit{dest}.
Optionally, compilation \textit{flags} can be defined via |\def| commands.
This command line makes the \TeX{} engine believe
it is compiling the file \textit{target}
whose content is specified as the latter parameter.
The provided code then forwards the processing to
\textit{main} or \textit{dest} as described in \secref{sec:forward}.

%%%%%%%%%%%%%%%%%%%%%%%%%%%%%%%%%%%%%%%%%%%%%%%%%%%%%%%%%%%%%%%%%%%%%%%%%%%%%%%%
\subsection{Include by Input}
\label{sec:input}

Including child documents by |\include| has some restrictions by design.
Most notably, the content of a child document always occupies
its own set of pages; pages cannot be shared between child documents.
Usually, this behaviour makes perfect sense
because each child document contain an essential part of the document.
However, in some situations it may be desirable to compose
a document from a collection of parts
without having mandatory page breaks between then.
For this case, the package
provides a mechanism to include parts
by |\input| which can also be processed individually.
However, by construction this mechanism
requires manual handling of the content to be output.

%%%%%%%%%%%%%%%%%%%%%%%%%%%%%%%%%%%%%%%%
\DescribeMacro{\ifchilddocmanual}
The main file should be prepared as usual, see \secref{sec:include}.
However, the document body must make a distinction
between processing of an individual part and of the main document, e.g.:
%
\begin{center}
\begin{tabular}{l}
|\ifchilddocmanual|\\
|\input{\childdocname}|\\
|\||else|\\
\textit{document body with }|\input{|\textit{part}|}|\\
|\||fi|
\end{tabular}
\end{center}
%
The conditional |\ifchilddocmanual| is true whenever
a part to be included by |\input| is being compiled,
and the name of the part is stored in |\childdocname|.

%%%%%%%%%%%%%%%%%%%%%%%%%%%%%%%%%%%%%%%%
\DescribeMacro{\childdocby}
Each part to be included by |\input| should start with:
%
\begin{center}
\begin{tabular}{l}
|\input{childdoc.def}|\\
|\childdocby{|\textit{main}|}|\\
\end{tabular}
\end{center}
%
The directive |\childdocby| is similar to |\childdocof|
described in \secref{sec:include},
but the subsequent selection of content must be done manually.
To that end, both |\ifchilddoc| and |\ifchilddocmanual|
will be true upon processing of a part,
and the name of the part is stored in |\childdocname|.
Note that |\jobname| will be set to the filename of the current part
so that each part receives an individual |.aux| file
that does not interfere with the |.aux| file(s) of the main document.
This behaviour can be altered by the alternative form
|\childdocby[*]{|\textit{main}|}| (with a non-empty optional argument)
which uses the |.aux| file of the main document
by setting |\jobname| to \textit{main}.

%%%%%%%%%%%%%%%%%%%%%%%%%%%%%%%%%%%%%%%%%%%%%%%%%%%%%%%%%%%%%%%%%%%%%%%%%%%%%%%%
\subsection{Driver Development}
\label{sec:driver}

The \textsf{childdoc} mechanism can also be use for the development
of definition files such as \LaTeX{} styles or classes.
This case differs from the above setup with multiple parts
included by |\include| in that no |\includeonly| should be invoked.
This can be achieved by starting the include file
(before |\ProvidesPackage|) with:
%
\begin{center}
\begin{tabular}{l}
|\input{childdoc.def}|\\
|\childdocforward{|\textit{main}|}|\\
\end{tabular}
\end{center}
%
or alternatively with:
%
\begin{center}
\begin{tabular}{l}
|\input{childdoc.def}|\\
|\childdocby{|\textit{main}|}|\\
\end{tabular}
\end{center}
%
Both forms have slightly different effects as described above.
The main file is prepared as usual, see \secref{sec:include}.

%%%%%%%%%%%%%%%%%%%%%%%%%%%%%%%%%%%%%%%%%%%%%%%%%%%%%%%%%%%%%%%%%%%%%%%%%%%%%%%%
\subsection{Legacy Detection}
\label{sec:detection}

The directive |\childdocmain| in the main file can detect
whether the complete document or merely a child is to be compiled
even without using the directive |\childdocof|.
This method is deprecated because it is less robust
and there is no compelling reason to use it;
it is merely provided for backward compatibility
and it may be removed in future versions.

If the detection mechanism is to be used,
it is mandatory to correctly specify
the filename of the main file as the argument of |\childdocmain|:
%
\begin{center}
\begin{tabular}{l}
|\input{childdoc.def}|\\
|\childdocmain{|\textit{main}|}|\\
\end{tabular}
\end{center}
%
If |\jobname| does not match the argument \textit{main} of |\childdocmain|,
it is assumed that |\jobname| points to the child file to be compiled.
When using |\childdocmain| with the main file specified as argument,
it suffices to start a child file
with just |\input{|\textit{main}|}|
without loading of the package and using |\childdocof|.
If instead all processing is done
with the appropriate \textsf{childdoc} directives,
the argument of \textit{main} of |\childdocmain| can be empty.

An alternative version of the command line processing described
in \secref{sec:commandline} using the detection mechanism reads:
%
\begin{center}
|... -jobname "|\textit{target}|" "|[\textit{flags}]%
[|\def\jobname{|\textit{dest}|}|]|\input{|\textit{main}|}"|
\end{center}

%%%%%%%%%%%%%%%%%%%%%%%%%%%%%%%%%%%%%%%%%%%%%%%%%%%%%%%%%%%%%%%%%%%%%%%%%%%%%%%%
\subsection{Manual Code}
\label{sec:manual}

In case one cannot be certain whether the definitions file |childdoc.def|
is installed on the target \TeX{} distribution
and one prefers not to ship it,
it is conceivable to paste a few relevant commands into the sources.

To that end, drop all statements |\input{childdoc.def}|
and perform the replacements as outlined below.
Instead of |\childdocmain{|\textit{main}|}| add the following code
to the top of the main file:
%
\begin{center}
\begin{tabular}{l}
|\||ifdefined\childdocname\endinput\||fi\newif\ifchilddoc|\\
|\edef\childdocname{\scantokens\expandafter{\jobname\noexpand}}|\\
|\def\childdocmain{|\textit{main}|}\||ifx\childdocmain\childdocname\||else|\\
|\childdoctrue\includeonly{\childdocname}\let\jobname\childdocmain\||fi|\\
\end{tabular}
\end{center}
%
Instead of |\childdocof{|\textit{main}|}| just include the main file
at the top of each child file:
%
\begin{center}
|\input{|\textit{main}|}|
\end{center}
%
A simple redirection |\childdocforward{|\textit{dest}|}| is achieved by:
%
\begin{center}
|\def\jobname{|\textit{dest}|}\input{\jobname}|
\end{center}
%
The redirection with prefix
|\childdocforwardprefix[|\textit{prefix}|]{|\textit{dest}|}|
is accomplished by:
%
\begin{center}
\begin{tabular}{l}
|{\edef\jobname{\scantokens\expandafter{\jobname\noexpand}}|\\
|\def\redirectjob |\textit{prefix}|#1~~~{\gdef\jobname{|\textit{dest}|#1}}|\\
|\expandafter\redirectjob\jobname~~~}\input{\jobname}|
\end{tabular}
\end{center}

In an alternative approach,
child documents can be compiled by a specific command line
without additional code or specific definitions:
%
\begin{center}
|... -jobname "|\textit{target}|" "|[\textit{flags}]%
|\includeonly{|\textit{dest}|}\input{|\textit{main}|}"|
\end{center}
%

%%%%%%%%%%%%%%%%%%%%%%%%%%%%%%%%%%%%%%%%%%%%%%%%%%%%%%%%%%%%%%%%%%%%%%%%%%%%%%%%
%%%%%%%%%%%%%%%%%%%%%%%%%%%%%%%%%%%%%%%%%%%%%%%%%%%%%%%%%%%%%%%%%%%%%%%%%%%%%%%%
\section{Information}

%%%%%%%%%%%%%%%%%%%%%%%%%%%%%%%%%%%%%%%%%%%%%%%%%%%%%%%%%%%%%%%%%%%%%%%%%%%%%%%%
\subsection{Copyright}

Copyright \copyright{} 2017--2018 Niklas Beisert

This work may be distributed and/or modified under the
conditions of the \LaTeX{} Project Public License, either version 1.3
of this license or (at your option) any later version.
The latest version of this license is in
  \url{http://www.latex-project.org/lppl.txt}
and version 1.3 or later is part of all distributions of \LaTeX{}
version 2005/12/01 or later.

This work has the LPPL maintenance status `maintained'.

The Current Maintainer of this work is Niklas Beisert.

This work consists of the files |README.txt|, |childdoc.ins| and |childdoc.dtx|
as well as the derived files |childdoc.def|, |cdocsamp.tex|
with |cdocsch1.tex|, |cdocsch2.tex|, |cdocspt3.tex|, |cdocspt4.tex|,
|cdocsdrf.tex|, |cdocsfn1.tex|, |cdocsfn2.tex|
as well as |childdoc.pdf|.

%%%%%%%%%%%%%%%%%%%%%%%%%%%%%%%%%%%%%%%%%%%%%%%%%%%%%%%%%%%%%%%%%%%%%%%%%%%%%%%%
\subsection{Files and Installation}

The package consists of the files:
%
\begin{center}
\begin{tabular}{ll}
    |README.txt|   & readme file \\
    |childdoc.ins| & installation file \\
    |childdoc.dtx| & source file \\
    |childdoc.def| & definition file \\
    |cdocsamp.tex| & sample main file \\
    |cdocsch1.tex| & sample include file \\
    |cdocsch2.tex| & sample include file \\
    |cdocspt3.tex| & sample part file \\
    |cdocspt4.tex| & sample part file \\
    |cdocsdrf.tex| & sample redirection file \\
    |cdocsfn1.tex| & sample redirection file \\
    |cdocsfn2.tex| & sample redirection file \\
    |childdoc.pdf| & manual
\end{tabular}
\end{center}
%
The distribution consists of the files
|README.txt|, |childdoc.ins| and |childdoc.dtx|.
%
\begin{itemize}
\item
Run (pdf)\LaTeX{} on |childdoc.dtx|
to compile the manual |childdoc.pdf| (this file).
\item
Run \LaTeX{} on |childdoc.ins| to create the definitions file |childdoc.def|
and the sample |cdocsamp.tex| with include files
|cdocsch1.tex|, |cdocsch2.tex|, |cdocspt3.tex|, |cdocspt4.tex|,
|cdocsdrf.tex|, |cdocsfn1.tex|, |cdocsfn2.tex|.
Then copy the file |childdoc.def| to an appropriate directory of your \LaTeX{}
distribution, e.g.\ \textit{texmf-root}|/tex/latex/childdoc|.
\end{itemize}

%%%%%%%%%%%%%%%%%%%%%%%%%%%%%%%%%%%%%%%%%%%%%%%%%%%%%%%%%%%%%%%%%%%%%%%%%%%%%%%%
\subsection{Related CTAN Packages}

There are several other packages which offer a similar functionality:
%
\begin{itemize}
\item
The packages
\href{http://ctan.org/pkg/docmute}{\textsf{docmute}},
\href{http://ctan.org/pkg/includex}{\textsf{includex}} and
\href{http://ctan.org/pkg/standalone}{\textsf{standalone}}
provide commands to include only the document body of
a child file thus allowing both files to be compiled individually.
\item
The packages \href{http://ctan.org/pkg/subdocs}{\textsf{subdocs}}
and \href{http://ctan.org/pkg/subfiles}{\textsf{subfiles}}
provide structures in which the main and child documents can be
encapsulated and allowing them to be compiled individually.
The inclusion mechanism is different from the conventional |\include|.
\item
The package \href{http://ctan.org/pkg/combine}{\textsf{combine}}
is an elaborate solution to combine several documents into one.
\end{itemize}
%
See also the CTAN topic \href{http://ctan.org/topic/subdocs}{\textsf{subdocs}}
for further related packages.
The present package differs from the above solutions in that
a document structure constructed with the conventional |\include| mechanism
just needs two extra commands at the top of every file
such that all constituent files can be compiled individually.

%%%%%%%%%%%%%%%%%%%%%%%%%%%%%%%%%%%%%%%%%%%%%%%%%%%%%%%%%%%%%%%%%%%%%%%%%%%%%%%%
%\subsection{Feature Suggestions}
%
%The following is a list of features which may be useful for future
%versions of this package:
%%
%\begin{itemize}
%\item
%\ldots
%\end{itemize}

%%%%%%%%%%%%%%%%%%%%%%%%%%%%%%%%%%%%%%%%%%%%%%%%%%%%%%%%%%%%%%%%%%%%%%%%%%%%%%%%
\subsection{Revision History}

%%%%%%%%%%%%%%%%%%%%%%%%%%%%%%%%%%%%%%%%
\paragraph{v2.0:} 2018/12/30

\begin{itemize}
\item
immediate forward processing
\item
added |\childdocby| mechanism
\item
manual restructured
\end{itemize}

%%%%%%%%%%%%%%%%%%%%%%%%%%%%%%%%%%%%%%%%
\paragraph{v1.6:} 2018/01/17

\begin{itemize}
\item
application for development of include files
\item
corrections to manual
\end{itemize}

%%%%%%%%%%%%%%%%%%%%%%%%%%%%%%%%%%%%%%%%
\paragraph{v1.5:} 2017/05/21

\begin{itemize}
\item
more complete structuring introduced
\item
|\childdocof| introduced
\item
|\childdoc| renamed to |\childdocmain|
\item
|\childredirect| renamed to |\childdocforward| and |\childdocforwardprefix|
and functionality expanded
\end{itemize}

%%%%%%%%%%%%%%%%%%%%%%%%%%%%%%%%%%%%%%%%
\paragraph{v1.0:} 2017/04/27

\begin{itemize}
\item
manual and install package
\item
first version published on CTAN
\end{itemize}

%%%%%%%%%%%%%%%%%%%%%%%%%%%%%%%%%%%%%%%%
\paragraph{v0.6:} 2017/04/26

\begin{itemize}
\item
redirection mechanism added
\end{itemize}

%%%%%%%%%%%%%%%%%%%%%%%%%%%%%%%%%%%%%%%%
\paragraph{v0.5:} 2017/04/26

\begin{itemize}
\item
functionality in definition file
\end{itemize}


%%%%%%%%%%%%%%%%%%%%%%%%%%%%%%%%%%%%%%%%%%%%%%%%%%%%%%%%%%%%%%%%%%%%%%%%%%%%%%%%
%%%%%%%%%%%%%%%%%%%%%%%%%%%%%%%%%%%%%%%%%%%%%%%%%%%%%%%%%%%%%%%%%%%%%%%%%%%%%%%%
%%%%%%%%%%%%%%%%%%%%%%%%%%%%%%%%%%%%%%%%%%%%%%%%%%%%%%%%%%%%%%%%%%%%%%%%%%%%%%%%
\appendix

\settowidth\MacroIndent{\rmfamily\scriptsize 000\ }

 \DocInput{childdoc.dtx}

\end{document}
%</driver>
% \fi
%
% %%%%%%%%%%%%%%%%%%%%%%%%%%%%%%%%%%%%%%%%%%%%%%%%%%%%%%%%%%%%%%%%%%%%%%%%%%%%%%
% %%%%%%%%%%%%%%%%%%%%%%%%%%%%%%%%%%%%%%%%%%%%%%%%%%%%%%%%%%%%%%%%%%%%%%%%%%%%%%
% \section{Sample}
%\iffalse
%<*samplemain>
%\fi
%
% The following presents a sample document
% with two chapters, two parts, a title page,
% a compile flag as well as three forwarding files to set the flag.
% It consists of eight |.tex| files:
% \begin{center}
% \begin{tabular}{ll}
% |cdocsamp.tex|&main file\\
% |cdocsch1.tex|&include file for chapter 1\\
% |cdocsch2.tex|&include file for chapter 2\\
% |cdocspt3.tex|&include file for part 3\\
% |cdocspt4.tex|&include file for part 4\\
% |cdocsdrf.tex|&forwarding file for main file in draft mode\\
% |cdocsfi1.tex|&forwarding file for final version of chapter 1\\
% |cdocsfi2.tex|&forwarding file for final version of chapter 2\\
% \end{tabular}
% \end{center}
% Each of the eight files can be compiled directly by the \LaTeX{} compiler.
%
% %%%%%%%%%%%%%%%%%%%%%%%%%%%%%%%%%%%%%%
% \paragraph{Main File.}
%
% The main file is called |cdocsamp.tex|.
%
% Load the \textsf{childdoc} definitions and
% declare the filename for the main document:
%    \begin{macrocode}
\input{childdoc.def}
\childdocmain{}
%    \end{macrocode}

% Optional override for |\version| flag:
%    \begin{macrocode}
%%\ifchilddoc\else\providecommand{\version}{draft}\fi
%    \end{macrocode}

% Define the default values for the |\version| flag
% (|final| for the main file and |draft| for childs):
%    \begin{macrocode}
\ifchilddoc
\providecommand{\version}{draft}
\else
\providecommand{\version}{final}
\fi
%    \end{macrocode}

% Load the standard document class:
%    \begin{macrocode}
\documentclass[12pt]{article}
%    \end{macrocode}

% Start the document body:
%    \begin{macrocode}
\begin{document}
%    \end{macrocode}

% Declare a title page.
% Print title, part of document being processed and version flag:
%    \begin{macrocode}
\addtocounter{page}{-1}
\begin{center}
{\LARGE\bfseries{}childdoc example\par}
\vspace{1cm}
\ifchilddoc
\ifchilddocmanual part\else chapter\fi:
`\childdocname' of `\childdocjob'\par
\else
main document: `\childdocjob'\par
\fi
version: \version\par
\end{center}
\newpage
%    \end{macrocode}

% Manually include selected file,
% otherwise process as usual:
%    \begin{macrocode}
\ifchilddocmanual
\section*{part `\childdocname'}
\input{\childdocname}
\else
%    \end{macrocode}

% Include the two chapters:
%    \begin{macrocode}
\include{cdocsch1}
\include{cdocsch2}
%    \end{macrocode}

% Include the two parts unless only chapters should be displayed:
%    \begin{macrocode}
\ifchilddoc\else
\section{part three}
\input{cdocspt3}
\section{part four}
\input{cdocspt4}
\fi
%    \end{macrocode}

% Process as usual until here:
%    \begin{macrocode}
\fi
%    \end{macrocode}

% End of document body:
%    \begin{macrocode}
\end{document}
%    \end{macrocode}
%\iffalse
%</samplemain>
%\fi
%
% %%%%%%%%%%%%%%%%%%%%%%%%%%%%%%%%%%%%%%
% \paragraph{Chapter Include Files.}
%
% The include files are called |cdocsch1.tex| and |cdocsch2.tex|.
%
%\iffalse
%<*samplechap1|samplechap2>
%\fi

% Optional override for |\version| flag:
%    \begin{macrocode}
%%\providecommand{\version}{final}
%    \end{macrocode}

% Include the main document:
%    \begin{macrocode}
\input{childdoc.def}
\childdocof{cdocsamp}
%    \end{macrocode}

%\iffalse
%</samplechap1|samplechap2>
%\fi
%
%\iffalse
%<*samplechap1>
%\fi
% Some text for chapter 1:
%    \begin{macrocode}
\section{one}
some text in chapter one
%    \end{macrocode}

%\iffalse
%</samplechap1>
%\fi
% Some text for chapter 2:
%\iffalse
%<*samplechap2>
%\fi
%    \begin{macrocode}
\section{two}
more text in chapter two
%    \end{macrocode}

%\iffalse
%</samplechap2>
%\fi
%
% %%%%%%%%%%%%%%%%%%%%%%%%%%%%%%%%%%%%%%
% \paragraph{Part Include Files.}
%
% The include files are called |cdocspt3.tex| and |cdocspt4.tex|.
%
%\iffalse
%<*samplepart3|samplepart4>
%\fi

% Optional override for |\version| flag:
%    \begin{macrocode}
%%\providecommand{\version}{final}
%    \end{macrocode}

% Include the main document:
%    \begin{macrocode}
\input{childdoc.def}
\childdocby{cdocsamp}
%    \end{macrocode}

%\iffalse
%</samplepart3|samplepart4>
%\fi
%
%\iffalse
%<*samplepart3>
%\fi
% Some text for part 3:
%    \begin{macrocode}
some text in part three
%    \end{macrocode}

%\iffalse
%</samplepart3>
%\fi
% Some text for part 4:
%\iffalse
%<*samplepart4>
%\fi
%    \begin{macrocode}
more text in part four
%    \end{macrocode}

%\iffalse
%</samplepart4>
%\fi
%
% %%%%%%%%%%%%%%%%%%%%%%%%%%%%%%%%%%%%%%
% \paragraph{Forwarding for a Complete Draft.}
%
% The following forwarding file |cdocsdrf.tex|
% compiles the main document in draft mode:
%\iffalse
%<*sampledraft>
%\fi
%    \begin{macrocode}
\def\version{draft}
\input{childdoc.def}
\childdocforward{cdocsamp}
%    \end{macrocode}

%\iffalse
%</sampledraft>
%\fi
%
% %%%%%%%%%%%%%%%%%%%%%%%%%%%%%%%%%%%%%%
% \paragraph{Forwarding for Final Version of the Chapters.}
%
% The following forwarding files |cdocsfn1.tex| and |cdocsfn2.tex|
% (with identical content)
% compile the final versions of the child documents
% |cdocsch1.tex| and |cdocsch2.tex|, respectively:
%\iffalse
%<*samplefinal>
%\fi
%    \begin{macrocode}
\def\version{final}
\input{childdoc.def}
\childdocforwardprefix[cdocsamp]{cdocsfn}{cdocsch}
%    \end{macrocode}

%\iffalse
%</samplefinal>
%\fi
%
% %%%%%%%%%%%%%%%%%%%%%%%%%%%%%%%%%%%%%%
% \paragraph{Command Line Processing.}
%
% The following three command lines generate the output files
% |cdocscld|, |cdocscl1| and |cdocscl2|
% which should be identical to
% |cdocsdrf|, |cdocsch1| and |cdocsfn2|, respectively:
% \begin{center}
% \begin{tabular}{l}
% |latex -jobname cdocscld \|\\
% |  "\def\version{draft}\input{childdoc.def}\childdocforward{cdocsamp}"|\\
% |latex -jobname cdocscl1 \|\\
% |  "\input{childdoc.def}\childdocforward[cdocsamp]{cdocsch1}"|\\
% |latex -jobname cdocscl2 \|\\
% |  "\def\version{final}\input{childdoc.def}\childdocforward{cdocsch2}"|
% \end{tabular}
% \end{center}
% Note that the trailing backslash on each first line
% merely continues the input to the second line
% (for convenient cut ant paste).
% Furthermore, the command |latex| can be replaced by any
% of its alternative versions such as |pdflatex|.
%
% %%%%%%%%%%%%%%%%%%%%%%%%%%%%%%%%%%%%%%%%%%%%%%%%%%%%%%%%%%%%%%%%%%%%%%%%%%%%%%
% %%%%%%%%%%%%%%%%%%%%%%%%%%%%%%%%%%%%%%%%%%%%%%%%%%%%%%%%%%%%%%%%%%%%%%%%%%%%%%
% \section{Implementation}
%\iffalse
%<*package>
%\fi
%
% This section describes the definitions file |childdoc.def|.

% The definitions cannot be loaded using |\usepackage| or |\RequirePackage|
% which has a mechanism to prevent loading a style file more than once.
% When loading the definitions by means of |\input|
% multiple instances have to be prevented manually:
%\iffalse
%This code needs to be before the `\ProvidesFile' directive
%which is defined at the beginning of this file.
%Therefore it is also placed there and commented out here.
%</package>
%<*discard>
%\fi
%    \begin{macrocode}
\ifdefined\childdocmain\endinput\fi
%    \end{macrocode}
%\iffalse
%</discard>
%<*package>
%\fi
%
% \macro{\ifchilddoc}
% \macro{\ifchilddocmanual}
% The conditional |\ifchilddoc| tells whether a
% child (true) or main (false) document is being compiled.
% The conditional |\ifchilddocmanual| tells whether
% the |\includeonly| mechanism is used (false) or
% the selection of child files must be performed manually (true).
% The definitions initialise to false:
%    \begin{macrocode}
\newif\ifchilddoc
\newif\ifchilddocmanual
%    \end{macrocode}

% \macro{\childdocname}
% \macro{\childdocjob}
% The macro |\childdocname| stores the name of the main document
% to be compiled. The macro |\childdocjob| stores the name of
% the document on which the \LaTeX{} compiler was originally invoked.
% The content of |\jobname| cannot be compared
% to filenames specified in the source due to different catcodes.
% The following code rescans |\jobname|, stores the result
% in |\childdocname| and saves a copy in |\childdocjob|:
%    \begin{macrocode}
\edef\childdocname{\scantokens\expandafter{\jobname\noexpand}}
\let\childdocjob\childdocname
%    \end{macrocode}

% \macro{\childdocdisable}
% The macro |\childdocdisable| prevents the main file
% from being processed more than once.
% At this stage, the main document command |\childdocmain|
% is assumed to be called once again where it should do nothing.
% Any subsequent call to it should prevent
% a secondary processing of the main document
% It overwrites the forwarding commands
% |\childdocof| and |\childdocforward|
% with empty macros to prevent further inclusions of the main document:
%    \begin{macrocode}
\newcommand{\childdocdisable}
{
  \renewcommand{\childdocmain}[1]{\renewcommand{\childdocmain}[1]{\endinput}}
  \renewcommand{\childdocof}[1]{}
  \renewcommand{\childdocby}[2][]{}
  \renewcommand{\childdocforward}[2][]{}
  \renewcommand{\childdocdisable}{}
}
%    \end{macrocode}

% \macro{\childdocmain}
% The macro |\childdocmain| is to be called at the top of the main file
% with nothing or the main filename (without extension) as argument.
% First, it breaks loops.
% If the argument is not empty and does not match |\childdocname|
% (which is set by the first inclusion of |childdoc.def|),
% |\ifchilddoc| is set to true, |\includeonly| is applied to the child file
% and |\jobname| is set to the main file
% (for proper handling of |.aux| files):
%    \begin{macrocode}
\newcommand{\childdocmain}[1]
{
  \childdocdisable\childdocmain{}
  \if?#1?\else
    \begingroup
      \def\childdoctmp{#1}
      \ifx\childdoctmp\childdocname
        \def\childdoctmp{}
      \else
        \def\childdoctmp
        {
          \childdoctrue
          \includeonly{\childdocname}
          \def\childdocjob{#1}
          \def\jobname{#1}
        }
      \fi
      \expandafter
    \endgroup
    \childdoctmp
  \fi
}
%    \end{macrocode}

% \macro{\childdocof}
% The command |\childdocof| redirects
% compilation to the main file |#1|.
%    \begin{macrocode}
\newcommand{\childdocof}[1]
{
  \childdocdisable
  \childdoctrue
  \includeonly{\childdocname}
  \def\jobname{#1}
  \def\childdocjob{#1}
  \input{#1}
}
%    \end{macrocode}

% \macro{\childdocby}
% The command |\childdocby| ....
%    \begin{macrocode}
\newcommand{\childdocby}[2][]
{
  \childdocdisable
  \childdoctrue
  \childdocmanualtrue
  \if?#1?\else
    \def\jobname{#2}
  \fi
  \def\childdocjob{#2}
  \input{#2}
  \endinput
}
%    \end{macrocode}

% \macro{\childdocforward}
% The command |\childdocforward| redirects
% compilation to the main file or
% (if the optional argument is given) a child file.
% Parameters are set as if the main file
% or a child file starting with |\childdocof| was compiled.
% Then compilation is handed over to the main file:
%    \begin{macrocode}
\newcommand{\childdocforward}[2][]
{
  \begingroup
    \if?#1?
      \def\childdoctmp
      {
        \def\childdocname{#2}
        \def\childdocjob{#2}
        \def\jobname{#2}
        \input{#2}
        \endinput
      }
    \else
      \def\childdoctmp
      {
        \childdocdisable
        \def\childdocname{#2}
        \childdoctrue
        \includeonly{#2}
        \def\childdocjob{#1}
        \def\jobname{#1}
        \input{#1}
        \endinput
      }
    \fi
    \expandafter
  \endgroup
  \childdoctmp
}
%    \end{macrocode}

% \macro{\childdocforwardprefix}
% The command |\childdocforwardprefix| redirects
% compilation to the main or a child file by means of a pattern.
% The prefix |#1| in the current filename is replaced by |#2|
% and the suffix of the current filename is kept
% (it is assumed that the filename does not contain the substring `|~~~|'
% which is used as a delimiter).
% Compilation is handed over to the new file by |\childdocforward|:
%    \begin{macrocode}
\newcommand{\childdocforwardprefix}[3][]
{
  \begingroup
    \def\childdocextract #2##1~~~{\def\childdoctmp{\childdocforward[#1]{#3##1}}}
    \expandafter\childdocextract\childdocname~~~
    \expandafter
  \endgroup
  \childdoctmp
}
%    \end{macrocode}

% \macro{\childdoc}
% The deprecated macro |\childdoc| is a legacy version of |\childdocmain|:
%    \begin{macrocode}
\newcommand{\childdoc}{\childdocmain}
%    \end{macrocode}

% \macro{\childdocredirect}
% The deprecated macro |\childdocredirect| is a legacy version
% of |\childdocforward| and |\childdocforwardprefix|:
%    \begin{macrocode}
\newcommand{\childdocredirect}[2][]
{
  \begingroup
    \if?#1?
      \def\childdoctmp{\childdocforward{#2}}
    \else
      \def\childdoctmp{\childdocforwardprefix{#1}{#2}}
    \fi
    \expandafter
  \endgroup
  \childdoctmp
}
%    \end{macrocode}

%\iffalse
%</package>
%\fi
%
\endinput
|\\
|\childdocof{|\textit{main}|}|\\
\end{tabular}
\end{center}
at the top of every child file \textit{child}
which is included by |\include{|\textit{child}|}|
from within the main file
(or at least for those files to be compiled individually).
The argument \textit{main} must be the filename of the main file.

There are a couple of
considerations in setting up the main and child documents:

%%%%%%%%%%%%%%%%%%%%%%%%%%%%%%%%%%%%%%%%
\paragraph{Restrictions.}

Please note the following restrictions:
\begin{itemize}
\item
|\childdocmain| must be called with one argument \textit{main}
to ensure compatibility with earlier version of the package.
It must either be empty (|\childdocmain{}|)
or precisely match the filename of the main file in which it is specified.
See \secref{sec:detection} for further information.
\item
The filename \textit{main} must be specified without the |.tex| extension.
\item
The filename \textit{main} is case sensitive
(even in case-insensitive file systems)
due to internal string comparison.
\item
The argument \textit{main} should be fully expanded, it cannot be a macro.
\item
Subdirectories and special characters should be avoided in filenames.
\item
The command |\childdocmain{|\textit{main}|}| must be followed by a whitespace.
It should not be followed immediately by another command
or by a comment mark `|%|'.
This is because the \TeX{} parser reads the token immediately following
the argument of |\childdocmain| and puts it
at the beginning of every child section;
however, a white\-space is ignored.
\end{itemize}

%%%%%%%%%%%%%%%%%%%%%%%%%%%%%%%%%%%%%%%%
\paragraph{Content of Main File.}

It is advisable to place all content in the child files included by |\include|.
Any output contained in the main file will appear in all child documents
unless suppressed manually;
it cannot be suppressed automatically by the |\includeonly| directive
and thus should normally be avoided.
A method to include some content in the main file
by means of conditional processing is described in \secref{sec:conditional}.

%%%%%%%%%%%%%%%%%%%%%%%%%%%%%%%%%%%%%%%%
\paragraph{Page Numbering.}

When only a part of the document is compiled,
the appropriate numbering of pages
(as well as other status parameters)
is determined from the |.aux| files.
The latter contain information from previous passes.
However this information needs to propagate through
all intermediate child documents.
Therefore the page numbering in child documents may well
be inconsistent until the complete document is compiled at least once.

A useful (if unconventional) way to always ensure a consistent
page numbering is to restart the numbering in each child document
and denote the pages by `\textit{child}|.|\textit{page}'
where \textit{child} represents the chapter/section number of the child file.
This can be achieved by the command
|\numberwithin{page}{|\textit{child}|}|
of the \textsf{amsmath} package
where \textit{child} can be |chapter| or |section|
depending on the chosen structuring.
Alternatively, one can modify the macro |\thepage| appropriately
and reset the counter |page| at the start of each child file.

%%%%%%%%%%%%%%%%%%%%%%%%%%%%%%%%%%%%%%%%%%%%%%%%%%%%%%%%%%%%%%%%%%%%%%%%%%%%%%%%
\subsection{Conditional Processing}
\label{sec:conditional}

The package provides a mechanism to compile different versions
of a document. To customise the versions further some conditional processing
can come in handy to distinguish which version is being compiled.
The package provides two macros to describe the compilation context:

%%%%%%%%%%%%%%%%%%%%%%%%%%%%%%%%%%%%%%%%
\DescribeMacro{\ifchilddoc}
The conditional |\ifchilddoc| distinguishes between the compilation of
child documents and the main document:
%
\begin{center}
|\ifchilddoc |\textit{child-code}| |[|\||else |\textit{main-code}]| \||fi|
\end{center}

%%%%%%%%%%%%%%%%%%%%%%%%%%%%%%%%%%%%%%%%
\DescribeMacro{\childdocname}
\DescribeMacro{\childdocjob}
The macro |\childdocname| contains the filename (without extension)
of the main or child file being processed.
Note that |\childdocjob| will always contain the name of the main file.

%%%%%%%%%%%%%%%%%%%%%%%%%%%%%%%%%%%%%%%%
\paragraph{Title Page.}

Conditional processing can be used to include a title or banner page
in the main document when proper precautions are taken.
Importantly, the code in the main file should ensure that the page counter
(as well as other status parameters which are stored in the |.aux| files)
takes the same value after the conditional processing.
Otherwise the page numbers may take divergent values
depending on which part is compiled.

For example, a title page could be declared by:
%
\begin{center}
\begin{tabular}{l}
|\ifchilddoc\||else|\\
|\addtocounter{page}{-1}|\\
\textit{code for title page}\\
|\newpage|\\
|\||fi|
\end{tabular}
\end{center}
%
A banner page for the child documents can be generated by:
%
\begin{center}
\begin{tabular}{l}
|\ifchilddoc|\\
|\addtocounter{page}{-1}|\\
\textit{code for banner page}\\
|\newpage|\\
|\||fi|
\end{tabular}
\end{center}
%
Here one could write a message such as:
\begin{center}
|This is the part \childdocname{} of \childdocjob{}.|
\end{center}

%%%%%%%%%%%%%%%%%%%%%%%%%%%%%%%%%%%%%%%%%%%%%%%%%%%%%%%%%%%%%%%%%%%%%%%%%%%%%%%%
\subsection{Flags}
\label{sec:flags}

The package makes it easy to generate different versions
of the main or child documents.
To this end compilation flags can be defined
and assigned different default values.
They will be particularly useful in conjunction
with the forwarding mechanism described in \secref{sec:forward}.

For example, it may be useful to have a flag |\version|
which can be set to |draft| or |final|.
The document source will contain some conditional code
depending on the value of |\version|.
Suppose further, the flag should default to |final| for the main file
and to |draft| for child files
which is a natural assignment for editing the document.
This is achieved by placing the following code
in the preamble of the main document
(below the |\childdocmain| directive):
%
\begin{center}
\begin{tabular}{l}
|\ifchilddoc|\\
|\providecommand{\version}{draft}|\\
|\||else|\\
|\providecommand{\version}{final}|\\
|\||fi|
\end{tabular}
\end{center}
%
The definition by |\providecommand| makes sure
that previous definitions are not overwritten.
Further statements |\providecommand{\version}{...}|
can thus be added before the above code to override it.

For the main file, one might add a line
(between |\childdocmain| and the above block)
%
\begin{center}
|%\ifchilddoc\||else\providecommand{\version}{draft}\||fi|
\end{center}
%
which can be uncommented to produce a draft version.
Likewise one can add a line to the very top of a child file
(above the |\childdocof{|\textit{main}|}| directive)
%
\begin{center}
|%\providecommand{\version}{final}|
\end{center}
%
which can be uncommented to produce the final version of this child document.

%%%%%%%%%%%%%%%%%%%%%%%%%%%%%%%%%%%%%%%%%%%%%%%%%%%%%%%%%%%%%%%%%%%%%%%%%%%%%%%%
\subsection{Forwarding}
\label{sec:forward}

Different versions of the main or child documents
using compilation flags as described in \secref{sec:flags}
can be (permanently) stored in different files
for convenient compilation, viewing and distribution.
To this end, the package defines a command
to pass on compilation to a different file:

%%%%%%%%%%%%%%%%%%%%%%%%%%%%%%%%%%%%%%%%
\DescribeMacro{\childdocforward}
The command |\childdocforward| redirects processing to
another source file:
%
\begin{center}
\begin{tabular}{l}
|% \iffalse
%
% childdoc.dtx Copyright (C) 2017-2018 Niklas Beisert
%
% This work may be distributed and/or modified under the
% conditions of the LaTeX Project Public License, either version 1.3
% of this license or (at your option) any later version.
% The latest version of this license is in
%   http://www.latex-project.org/lppl.txt
% and version 1.3 or later is part of all distributions of LaTeX
% version 2005/12/01 or later.
%
% This work has the LPPL maintenance status `maintained'.
%
% The Current Maintainer of this work is Niklas Beisert.
%
% This work consists of the files childdoc.dtx and childdoc.ins
% and the derived files childdoc.def and cdocsamp.tex with
% cdocsch1.tex, cdocsch2.tex, cdocsdrf.tex, cdocsfn1.tex, cdocsfn2.tex.
%
%<package>\ifdefined\childdocmain\endinput\fi
%<package>\ProvidesFile{childdoc.def}[2018/12/30 v2.0 child document driver]
%<samplemain>\ProvidesFile{cdocsamp.tex}[2018/12/30 v2.0 sample for childdoc]
%<*driver>
%\ProvidesFile{childdoc.drv}[2018/12/30 v2.0 childdoc reference manual file]
\PassOptionsToClass{10pt,a4paper}{article}
\documentclass{ltxdoc}

\usepackage[margin=35mm]{geometry}
\usepackage{hyperref}
\usepackage{hyperxmp}
\usepackage[usenames]{color}

\hypersetup{colorlinks=true}
\hypersetup{pdfstartview=FitH}
\hypersetup{pdfpagemode=UseNone}
\hypersetup{pdfsource={}}
\hypersetup{pdflang={en-UK}}
\hypersetup{pdfcopyright={Copyright 2017-2018 Niklas Beisert.
  This work may be distributed and/or modified under the
  conditions of the LaTeX Project Public License, either version 1.3
  of this license or (at your option) any later version.}}
\hypersetup{pdflicenseurl={http://www.latex-project.org/lppl.txt}}
\hypersetup{pdfcontactaddress={ETH Zurich, ITP, HIT K,
  Wolfgang-Pauli-Strasse 27}}
\hypersetup{pdfcontactpostcode={8093}}
\hypersetup{pdfcontactcity={Zurich}}
\hypersetup{pdfcontactcountry={Switzerland}}
\hypersetup{pdfcontactemail={nbeisert@itp.phys.ethz.ch}}
\hypersetup{pdfcontacturl={http://people.phys.ethz.ch/\xmptilde nbeisert/}}

\newcommand{\secref}[1]{\hyperref[#1]{section \ref*{#1}}}

\parskip1ex
\parindent0pt
\let\olditemize\itemize
\def\itemize{\olditemize\parskip0pt}

\begin{document}

\title{The \textsf{childdoc} Package}
\hypersetup{pdftitle={The childdoc Package}}
\author{Niklas Beisert\\[2ex]
  Institut f\"ur Theoretische Physik\\
  Eidgen\"ossische Technische Hochschule Z\"urich\\
  Wolfgang-Pauli-Strasse 27, 8093 Z\"urich, Switzerland\\[1ex]
  \href{mailto:nbeisert@itp.phys.ethz.ch}
  {\texttt{nbeisert@itp.phys.ethz.ch}}}
\hypersetup{pdfauthor={Niklas Beisert}}
\hypersetup{pdfsubject={Manual for the LaTeX2e Package childdoc}}
\date{30 December 2018, \textsf{v2.0}}
\maketitle

\begin{abstract}\noindent
\textsf{childdoc} is a \LaTeXe{} package
that enables the direct compilation
of document sections included by |\include|
to individual files.
\end{abstract}

\begingroup
\parskip0ex
\tableofcontents
\endgroup

%%%%%%%%%%%%%%%%%%%%%%%%%%%%%%%%%%%%%%%%%%%%%%%%%%%%%%%%%%%%%%%%%%%%%%%%%%%%%%%%
%%%%%%%%%%%%%%%%%%%%%%%%%%%%%%%%%%%%%%%%%%%%%%%%%%%%%%%%%%%%%%%%%%%%%%%%%%%%%%%%
\section{Introduction}

\LaTeX{} provides a mechanism to structure a large document (such as a book)
into a main file and several child files (containing the chapters)
using the |\include| command.
This mechanism is beneficial for documents
which span hundreds of pages in order to
make the source file(s) more manageable.
Moreover, compilation can be restricted to
selected child files by means of the |\includeonly| command.
The latter feature can be used to reduce the compilation time while editing
(this was significantly more useful in the earlier days of \LaTeX{})
or to generate a smaller document which is easier to navigate.
Another application of |\includeonly| is to generate
documents consisting of selected parts of the complete document.

However, there are a few drawbacks of the plain |\include| mechanism:
\begin{itemize}
\item
The child files cannot be compiled on their own,
they can only be compiled via the main file.
A naive editing environment
(such as a text editor with an option
to have the current file processed by \LaTeX)
may require one to switch to the main file before compiling;
attempting to compile the child file produces errors.
\item
The main file must be modified (each time)
to adjust the |\includeonly| command
to the present needs. This easily leaves the main file in a messy state.
\item
The generated document will always carry the filename
of the main document. This is inconvenient if
several child files are to be compiled and
to be kept for distribution.
\end{itemize}

The present package provides a simple interface
to make child files individually compilable by \LaTeX{}.
Compiling a child file then has the same effect as compiling
the main file with an |\includeonly| command
to select the appropriate child.
Moreover the generated document will carry the name of the child
rather than the main file.
This resolves all three above issues.

This feature is meant to make the editing of books,
thesis documents and lecture notes somewhat more convenient.
However, the package can also be used efficiently for
composing a series of documents (such as exercise sheets)
which are typically distributed individually.
It then assists the author in generating the individual documents
(potentially in different versions)
as well as a document containing the collected series.
Another application is in developing style files
or other kinds of included material
where compilation of the style file could redirect
to a sample or test file.

%%%%%%%%%%%%%%%%%%%%%%%%%%%%%%%%%%%%%%%%%%%%%%%%%%%%%%%%%%%%%%%%%%%%%%%%%%%%%%%%
%%%%%%%%%%%%%%%%%%%%%%%%%%%%%%%%%%%%%%%%%%%%%%%%%%%%%%%%%%%%%%%%%%%%%%%%%%%%%%%%
\section{Usage}

First of all, the package \textsf{childdoc} is \emph{not} a standard
\LaTeXe{} |.sty| style file! Therefore it needs to be invoked in
a non-standard way.

%%%%%%%%%%%%%%%%%%%%%%%%%%%%%%%%%%%%%%%%%%%%%%%%%%%%%%%%%%%%%%%%%%%%%%%%%%%%%%%%
\subsection{Included Files}
\label{sec:include}

%%%%%%%%%%%%%%%%%%%%%%%%%%%%%%%%%%%%%%%%
\DescribeMacro{\childdocmain}
To use the package, add the commands
\begin{center}
\begin{tabular}{l}
|\input{childdoc.def}|\\
|\childdocmain{}|\\
\end{tabular}
\end{center}
at the very top of the main \LaTeX{} file,
in particular \emph{before} the |\documentclass| statement!
The argument of |\childdocmain| should be left empty
(but it must be present).

%%%%%%%%%%%%%%%%%%%%%%%%%%%%%%%%%%%%%%%%
\DescribeMacro{\childdocof}
Furthermore, add the commands
\begin{center}
\begin{tabular}{l}
|\input{childdoc.def}|\\
|\childdocof{|\textit{main}|}|\\
\end{tabular}
\end{center}
at the top of every child file \textit{child}
which is included by |\include{|\textit{child}|}|
from within the main file
(or at least for those files to be compiled individually).
The argument \textit{main} must be the filename of the main file.

There are a couple of
considerations in setting up the main and child documents:

%%%%%%%%%%%%%%%%%%%%%%%%%%%%%%%%%%%%%%%%
\paragraph{Restrictions.}

Please note the following restrictions:
\begin{itemize}
\item
|\childdocmain| must be called with one argument \textit{main}
to ensure compatibility with earlier version of the package.
It must either be empty (|\childdocmain{}|)
or precisely match the filename of the main file in which it is specified.
See \secref{sec:detection} for further information.
\item
The filename \textit{main} must be specified without the |.tex| extension.
\item
The filename \textit{main} is case sensitive
(even in case-insensitive file systems)
due to internal string comparison.
\item
The argument \textit{main} should be fully expanded, it cannot be a macro.
\item
Subdirectories and special characters should be avoided in filenames.
\item
The command |\childdocmain{|\textit{main}|}| must be followed by a whitespace.
It should not be followed immediately by another command
or by a comment mark `|%|'.
This is because the \TeX{} parser reads the token immediately following
the argument of |\childdocmain| and puts it
at the beginning of every child section;
however, a white\-space is ignored.
\end{itemize}

%%%%%%%%%%%%%%%%%%%%%%%%%%%%%%%%%%%%%%%%
\paragraph{Content of Main File.}

It is advisable to place all content in the child files included by |\include|.
Any output contained in the main file will appear in all child documents
unless suppressed manually;
it cannot be suppressed automatically by the |\includeonly| directive
and thus should normally be avoided.
A method to include some content in the main file
by means of conditional processing is described in \secref{sec:conditional}.

%%%%%%%%%%%%%%%%%%%%%%%%%%%%%%%%%%%%%%%%
\paragraph{Page Numbering.}

When only a part of the document is compiled,
the appropriate numbering of pages
(as well as other status parameters)
is determined from the |.aux| files.
The latter contain information from previous passes.
However this information needs to propagate through
all intermediate child documents.
Therefore the page numbering in child documents may well
be inconsistent until the complete document is compiled at least once.

A useful (if unconventional) way to always ensure a consistent
page numbering is to restart the numbering in each child document
and denote the pages by `\textit{child}|.|\textit{page}'
where \textit{child} represents the chapter/section number of the child file.
This can be achieved by the command
|\numberwithin{page}{|\textit{child}|}|
of the \textsf{amsmath} package
where \textit{child} can be |chapter| or |section|
depending on the chosen structuring.
Alternatively, one can modify the macro |\thepage| appropriately
and reset the counter |page| at the start of each child file.

%%%%%%%%%%%%%%%%%%%%%%%%%%%%%%%%%%%%%%%%%%%%%%%%%%%%%%%%%%%%%%%%%%%%%%%%%%%%%%%%
\subsection{Conditional Processing}
\label{sec:conditional}

The package provides a mechanism to compile different versions
of a document. To customise the versions further some conditional processing
can come in handy to distinguish which version is being compiled.
The package provides two macros to describe the compilation context:

%%%%%%%%%%%%%%%%%%%%%%%%%%%%%%%%%%%%%%%%
\DescribeMacro{\ifchilddoc}
The conditional |\ifchilddoc| distinguishes between the compilation of
child documents and the main document:
%
\begin{center}
|\ifchilddoc |\textit{child-code}| |[|\||else |\textit{main-code}]| \||fi|
\end{center}

%%%%%%%%%%%%%%%%%%%%%%%%%%%%%%%%%%%%%%%%
\DescribeMacro{\childdocname}
\DescribeMacro{\childdocjob}
The macro |\childdocname| contains the filename (without extension)
of the main or child file being processed.
Note that |\childdocjob| will always contain the name of the main file.

%%%%%%%%%%%%%%%%%%%%%%%%%%%%%%%%%%%%%%%%
\paragraph{Title Page.}

Conditional processing can be used to include a title or banner page
in the main document when proper precautions are taken.
Importantly, the code in the main file should ensure that the page counter
(as well as other status parameters which are stored in the |.aux| files)
takes the same value after the conditional processing.
Otherwise the page numbers may take divergent values
depending on which part is compiled.

For example, a title page could be declared by:
%
\begin{center}
\begin{tabular}{l}
|\ifchilddoc\||else|\\
|\addtocounter{page}{-1}|\\
\textit{code for title page}\\
|\newpage|\\
|\||fi|
\end{tabular}
\end{center}
%
A banner page for the child documents can be generated by:
%
\begin{center}
\begin{tabular}{l}
|\ifchilddoc|\\
|\addtocounter{page}{-1}|\\
\textit{code for banner page}\\
|\newpage|\\
|\||fi|
\end{tabular}
\end{center}
%
Here one could write a message such as:
\begin{center}
|This is the part \childdocname{} of \childdocjob{}.|
\end{center}

%%%%%%%%%%%%%%%%%%%%%%%%%%%%%%%%%%%%%%%%%%%%%%%%%%%%%%%%%%%%%%%%%%%%%%%%%%%%%%%%
\subsection{Flags}
\label{sec:flags}

The package makes it easy to generate different versions
of the main or child documents.
To this end compilation flags can be defined
and assigned different default values.
They will be particularly useful in conjunction
with the forwarding mechanism described in \secref{sec:forward}.

For example, it may be useful to have a flag |\version|
which can be set to |draft| or |final|.
The document source will contain some conditional code
depending on the value of |\version|.
Suppose further, the flag should default to |final| for the main file
and to |draft| for child files
which is a natural assignment for editing the document.
This is achieved by placing the following code
in the preamble of the main document
(below the |\childdocmain| directive):
%
\begin{center}
\begin{tabular}{l}
|\ifchilddoc|\\
|\providecommand{\version}{draft}|\\
|\||else|\\
|\providecommand{\version}{final}|\\
|\||fi|
\end{tabular}
\end{center}
%
The definition by |\providecommand| makes sure
that previous definitions are not overwritten.
Further statements |\providecommand{\version}{...}|
can thus be added before the above code to override it.

For the main file, one might add a line
(between |\childdocmain| and the above block)
%
\begin{center}
|%\ifchilddoc\||else\providecommand{\version}{draft}\||fi|
\end{center}
%
which can be uncommented to produce a draft version.
Likewise one can add a line to the very top of a child file
(above the |\childdocof{|\textit{main}|}| directive)
%
\begin{center}
|%\providecommand{\version}{final}|
\end{center}
%
which can be uncommented to produce the final version of this child document.

%%%%%%%%%%%%%%%%%%%%%%%%%%%%%%%%%%%%%%%%%%%%%%%%%%%%%%%%%%%%%%%%%%%%%%%%%%%%%%%%
\subsection{Forwarding}
\label{sec:forward}

Different versions of the main or child documents
using compilation flags as described in \secref{sec:flags}
can be (permanently) stored in different files
for convenient compilation, viewing and distribution.
To this end, the package defines a command
to pass on compilation to a different file:

%%%%%%%%%%%%%%%%%%%%%%%%%%%%%%%%%%%%%%%%
\DescribeMacro{\childdocforward}
The command |\childdocforward| redirects processing to
another source file:
%
\begin{center}
\begin{tabular}{l}
|\input{childdoc.def}|\\
|\childdocforward[|\textit{main}|]{|\textit{dest}|}|\\
\end{tabular}
\end{center}
%
The argument \textit{dest} is the destination file
(without extension).
It should be the main file or one of the child files.
Note that further \textsf{childdoc} directives
such as |\childdocof| and |\childdocforward|
in the indicated file will be processed in this form.
The optional argument \textit{main}
passes on directly to the main file \textit{main}
while pretending to compile the child \textit{dest}.
This form behaves as if \textit{dest}
issues |\childdocof{|\textit{main}|}| right away,
and no further \textsf{childdoc} directives will be processed.

%%%%%%%%%%%%%%%%%%%%%%%%%%%%%%%%%%%%%%%%
\DescribeMacro{\...prefix}
In the alternative form |\childdocforwardprefix|,
%
\begin{center}
\begin{tabular}{l}
|\input{childdoc.def}|\\
|\childdocforwardprefix[|\textit{main}|]{|\textit{prefix}|}{|\textit{dest}|}|
\end{tabular}
\end{center}
%
the destination file is determined by a pattern
depending on the current file:
To make this work, the current file must be called
`{\textit{prefix}\hspace{0.2em}\textit{suffix}}'
with \textit{prefix} matching precisely the argument.
Processing is then passed on to the file
`{\textit{dest}\hspace{0.2em}\textit{suffix}}'.
Surely, the same effect is achieved by
directly specifying the
argument `{\textit{dest}\hspace{0.2em}\textit{suffix}}'
in the first form.
However, that requires to set up a different file
for each child. With the alternative form of the command
all these files can have exactly the same content
which simplifies setting them up and maintaining them.

For example, the following file |draft.tex|
with a compilation flag |\version| as described in \secref{sec:flags}
compiles the main document as a draft:
%
\begin{center}
\begin{tabular}{l}
|\def\version{draft}|\\
|\input{childdoc.def}|\\
|\childdocforward{|\textit{main}|}|
\end{tabular}
\end{center}
%
Likewise, the following files |final|\textit{nn}|.tex|
compile the final version of the child document
|child|\textit{nn}|.tex|:
%
\begin{center}
\begin{tabular}{l}
|\def\version{final}|\\
|\input{childdoc.def}|\\
|\childdocforwardprefix{final}{child}|
\end{tabular}
\end{center}
%

Note that when several versions of a main file and/or of each child file
are to be generated, it may be convenient to set up a |Makefile| or
shell script to automatise the process.

%%%%%%%%%%%%%%%%%%%%%%%%%%%%%%%%%%%%%%%%%%%%%%%%%%%%%%%%%%%%%%%%%%%%%%%%%%%%%%%%
\subsection{Command Line Processing}
\label{sec:commandline}

The effect of redirection files can also be achieved by invoking
the \LaTeX{} compiler with a more elaborate command line.
Most conveniently this should be done as part
of a shell script or a |Makefile|.

When using \textsf{childdoc} in the main file, the following
command lines effectively perform a redirection
(note that depending on the shell being used,
backslashes may have to be doubled: `|\|' $\to$ `|\\|'):
%
\begin{center}
|... -jobname "|\textit{target}|" |\\|"|[\textit{flags}]%
|\input{childdoc.def}\childdocforward[|\textit{main}|]{|\textit{dest}|}"|
\end{center}
%
Here \textit{target} is the name of the output file,
\textit{main} is the name of the main file
and \textit{dest} is the name of the main or child file to be processed
(all filenames without extensions).
The optional argument \textit{main} can be omitted
if \textit{main} matches \textit{dest}.
Optionally, compilation \textit{flags} can be defined via |\def| commands.
This command line makes the \TeX{} engine believe
it is compiling the file \textit{target}
whose content is specified as the latter parameter.
The provided code then forwards the processing to
\textit{main} or \textit{dest} as described in \secref{sec:forward}.

%%%%%%%%%%%%%%%%%%%%%%%%%%%%%%%%%%%%%%%%%%%%%%%%%%%%%%%%%%%%%%%%%%%%%%%%%%%%%%%%
\subsection{Include by Input}
\label{sec:input}

Including child documents by |\include| has some restrictions by design.
Most notably, the content of a child document always occupies
its own set of pages; pages cannot be shared between child documents.
Usually, this behaviour makes perfect sense
because each child document contain an essential part of the document.
However, in some situations it may be desirable to compose
a document from a collection of parts
without having mandatory page breaks between then.
For this case, the package
provides a mechanism to include parts
by |\input| which can also be processed individually.
However, by construction this mechanism
requires manual handling of the content to be output.

%%%%%%%%%%%%%%%%%%%%%%%%%%%%%%%%%%%%%%%%
\DescribeMacro{\ifchilddocmanual}
The main file should be prepared as usual, see \secref{sec:include}.
However, the document body must make a distinction
between processing of an individual part and of the main document, e.g.:
%
\begin{center}
\begin{tabular}{l}
|\ifchilddocmanual|\\
|\input{\childdocname}|\\
|\||else|\\
\textit{document body with }|\input{|\textit{part}|}|\\
|\||fi|
\end{tabular}
\end{center}
%
The conditional |\ifchilddocmanual| is true whenever
a part to be included by |\input| is being compiled,
and the name of the part is stored in |\childdocname|.

%%%%%%%%%%%%%%%%%%%%%%%%%%%%%%%%%%%%%%%%
\DescribeMacro{\childdocby}
Each part to be included by |\input| should start with:
%
\begin{center}
\begin{tabular}{l}
|\input{childdoc.def}|\\
|\childdocby{|\textit{main}|}|\\
\end{tabular}
\end{center}
%
The directive |\childdocby| is similar to |\childdocof|
described in \secref{sec:include},
but the subsequent selection of content must be done manually.
To that end, both |\ifchilddoc| and |\ifchilddocmanual|
will be true upon processing of a part,
and the name of the part is stored in |\childdocname|.
Note that |\jobname| will be set to the filename of the current part
so that each part receives an individual |.aux| file
that does not interfere with the |.aux| file(s) of the main document.
This behaviour can be altered by the alternative form
|\childdocby[*]{|\textit{main}|}| (with a non-empty optional argument)
which uses the |.aux| file of the main document
by setting |\jobname| to \textit{main}.

%%%%%%%%%%%%%%%%%%%%%%%%%%%%%%%%%%%%%%%%%%%%%%%%%%%%%%%%%%%%%%%%%%%%%%%%%%%%%%%%
\subsection{Driver Development}
\label{sec:driver}

The \textsf{childdoc} mechanism can also be use for the development
of definition files such as \LaTeX{} styles or classes.
This case differs from the above setup with multiple parts
included by |\include| in that no |\includeonly| should be invoked.
This can be achieved by starting the include file
(before |\ProvidesPackage|) with:
%
\begin{center}
\begin{tabular}{l}
|\input{childdoc.def}|\\
|\childdocforward{|\textit{main}|}|\\
\end{tabular}
\end{center}
%
or alternatively with:
%
\begin{center}
\begin{tabular}{l}
|\input{childdoc.def}|\\
|\childdocby{|\textit{main}|}|\\
\end{tabular}
\end{center}
%
Both forms have slightly different effects as described above.
The main file is prepared as usual, see \secref{sec:include}.

%%%%%%%%%%%%%%%%%%%%%%%%%%%%%%%%%%%%%%%%%%%%%%%%%%%%%%%%%%%%%%%%%%%%%%%%%%%%%%%%
\subsection{Legacy Detection}
\label{sec:detection}

The directive |\childdocmain| in the main file can detect
whether the complete document or merely a child is to be compiled
even without using the directive |\childdocof|.
This method is deprecated because it is less robust
and there is no compelling reason to use it;
it is merely provided for backward compatibility
and it may be removed in future versions.

If the detection mechanism is to be used,
it is mandatory to correctly specify
the filename of the main file as the argument of |\childdocmain|:
%
\begin{center}
\begin{tabular}{l}
|\input{childdoc.def}|\\
|\childdocmain{|\textit{main}|}|\\
\end{tabular}
\end{center}
%
If |\jobname| does not match the argument \textit{main} of |\childdocmain|,
it is assumed that |\jobname| points to the child file to be compiled.
When using |\childdocmain| with the main file specified as argument,
it suffices to start a child file
with just |\input{|\textit{main}|}|
without loading of the package and using |\childdocof|.
If instead all processing is done
with the appropriate \textsf{childdoc} directives,
the argument of \textit{main} of |\childdocmain| can be empty.

An alternative version of the command line processing described
in \secref{sec:commandline} using the detection mechanism reads:
%
\begin{center}
|... -jobname "|\textit{target}|" "|[\textit{flags}]%
[|\def\jobname{|\textit{dest}|}|]|\input{|\textit{main}|}"|
\end{center}

%%%%%%%%%%%%%%%%%%%%%%%%%%%%%%%%%%%%%%%%%%%%%%%%%%%%%%%%%%%%%%%%%%%%%%%%%%%%%%%%
\subsection{Manual Code}
\label{sec:manual}

In case one cannot be certain whether the definitions file |childdoc.def|
is installed on the target \TeX{} distribution
and one prefers not to ship it,
it is conceivable to paste a few relevant commands into the sources.

To that end, drop all statements |\input{childdoc.def}|
and perform the replacements as outlined below.
Instead of |\childdocmain{|\textit{main}|}| add the following code
to the top of the main file:
%
\begin{center}
\begin{tabular}{l}
|\||ifdefined\childdocname\endinput\||fi\newif\ifchilddoc|\\
|\edef\childdocname{\scantokens\expandafter{\jobname\noexpand}}|\\
|\def\childdocmain{|\textit{main}|}\||ifx\childdocmain\childdocname\||else|\\
|\childdoctrue\includeonly{\childdocname}\let\jobname\childdocmain\||fi|\\
\end{tabular}
\end{center}
%
Instead of |\childdocof{|\textit{main}|}| just include the main file
at the top of each child file:
%
\begin{center}
|\input{|\textit{main}|}|
\end{center}
%
A simple redirection |\childdocforward{|\textit{dest}|}| is achieved by:
%
\begin{center}
|\def\jobname{|\textit{dest}|}\input{\jobname}|
\end{center}
%
The redirection with prefix
|\childdocforwardprefix[|\textit{prefix}|]{|\textit{dest}|}|
is accomplished by:
%
\begin{center}
\begin{tabular}{l}
|{\edef\jobname{\scantokens\expandafter{\jobname\noexpand}}|\\
|\def\redirectjob |\textit{prefix}|#1~~~{\gdef\jobname{|\textit{dest}|#1}}|\\
|\expandafter\redirectjob\jobname~~~}\input{\jobname}|
\end{tabular}
\end{center}

In an alternative approach,
child documents can be compiled by a specific command line
without additional code or specific definitions:
%
\begin{center}
|... -jobname "|\textit{target}|" "|[\textit{flags}]%
|\includeonly{|\textit{dest}|}\input{|\textit{main}|}"|
\end{center}
%

%%%%%%%%%%%%%%%%%%%%%%%%%%%%%%%%%%%%%%%%%%%%%%%%%%%%%%%%%%%%%%%%%%%%%%%%%%%%%%%%
%%%%%%%%%%%%%%%%%%%%%%%%%%%%%%%%%%%%%%%%%%%%%%%%%%%%%%%%%%%%%%%%%%%%%%%%%%%%%%%%
\section{Information}

%%%%%%%%%%%%%%%%%%%%%%%%%%%%%%%%%%%%%%%%%%%%%%%%%%%%%%%%%%%%%%%%%%%%%%%%%%%%%%%%
\subsection{Copyright}

Copyright \copyright{} 2017--2018 Niklas Beisert

This work may be distributed and/or modified under the
conditions of the \LaTeX{} Project Public License, either version 1.3
of this license or (at your option) any later version.
The latest version of this license is in
  \url{http://www.latex-project.org/lppl.txt}
and version 1.3 or later is part of all distributions of \LaTeX{}
version 2005/12/01 or later.

This work has the LPPL maintenance status `maintained'.

The Current Maintainer of this work is Niklas Beisert.

This work consists of the files |README.txt|, |childdoc.ins| and |childdoc.dtx|
as well as the derived files |childdoc.def|, |cdocsamp.tex|
with |cdocsch1.tex|, |cdocsch2.tex|, |cdocspt3.tex|, |cdocspt4.tex|,
|cdocsdrf.tex|, |cdocsfn1.tex|, |cdocsfn2.tex|
as well as |childdoc.pdf|.

%%%%%%%%%%%%%%%%%%%%%%%%%%%%%%%%%%%%%%%%%%%%%%%%%%%%%%%%%%%%%%%%%%%%%%%%%%%%%%%%
\subsection{Files and Installation}

The package consists of the files:
%
\begin{center}
\begin{tabular}{ll}
    |README.txt|   & readme file \\
    |childdoc.ins| & installation file \\
    |childdoc.dtx| & source file \\
    |childdoc.def| & definition file \\
    |cdocsamp.tex| & sample main file \\
    |cdocsch1.tex| & sample include file \\
    |cdocsch2.tex| & sample include file \\
    |cdocspt3.tex| & sample part file \\
    |cdocspt4.tex| & sample part file \\
    |cdocsdrf.tex| & sample redirection file \\
    |cdocsfn1.tex| & sample redirection file \\
    |cdocsfn2.tex| & sample redirection file \\
    |childdoc.pdf| & manual
\end{tabular}
\end{center}
%
The distribution consists of the files
|README.txt|, |childdoc.ins| and |childdoc.dtx|.
%
\begin{itemize}
\item
Run (pdf)\LaTeX{} on |childdoc.dtx|
to compile the manual |childdoc.pdf| (this file).
\item
Run \LaTeX{} on |childdoc.ins| to create the definitions file |childdoc.def|
and the sample |cdocsamp.tex| with include files
|cdocsch1.tex|, |cdocsch2.tex|, |cdocspt3.tex|, |cdocspt4.tex|,
|cdocsdrf.tex|, |cdocsfn1.tex|, |cdocsfn2.tex|.
Then copy the file |childdoc.def| to an appropriate directory of your \LaTeX{}
distribution, e.g.\ \textit{texmf-root}|/tex/latex/childdoc|.
\end{itemize}

%%%%%%%%%%%%%%%%%%%%%%%%%%%%%%%%%%%%%%%%%%%%%%%%%%%%%%%%%%%%%%%%%%%%%%%%%%%%%%%%
\subsection{Related CTAN Packages}

There are several other packages which offer a similar functionality:
%
\begin{itemize}
\item
The packages
\href{http://ctan.org/pkg/docmute}{\textsf{docmute}},
\href{http://ctan.org/pkg/includex}{\textsf{includex}} and
\href{http://ctan.org/pkg/standalone}{\textsf{standalone}}
provide commands to include only the document body of
a child file thus allowing both files to be compiled individually.
\item
The packages \href{http://ctan.org/pkg/subdocs}{\textsf{subdocs}}
and \href{http://ctan.org/pkg/subfiles}{\textsf{subfiles}}
provide structures in which the main and child documents can be
encapsulated and allowing them to be compiled individually.
The inclusion mechanism is different from the conventional |\include|.
\item
The package \href{http://ctan.org/pkg/combine}{\textsf{combine}}
is an elaborate solution to combine several documents into one.
\end{itemize}
%
See also the CTAN topic \href{http://ctan.org/topic/subdocs}{\textsf{subdocs}}
for further related packages.
The present package differs from the above solutions in that
a document structure constructed with the conventional |\include| mechanism
just needs two extra commands at the top of every file
such that all constituent files can be compiled individually.

%%%%%%%%%%%%%%%%%%%%%%%%%%%%%%%%%%%%%%%%%%%%%%%%%%%%%%%%%%%%%%%%%%%%%%%%%%%%%%%%
%\subsection{Feature Suggestions}
%
%The following is a list of features which may be useful for future
%versions of this package:
%%
%\begin{itemize}
%\item
%\ldots
%\end{itemize}

%%%%%%%%%%%%%%%%%%%%%%%%%%%%%%%%%%%%%%%%%%%%%%%%%%%%%%%%%%%%%%%%%%%%%%%%%%%%%%%%
\subsection{Revision History}

%%%%%%%%%%%%%%%%%%%%%%%%%%%%%%%%%%%%%%%%
\paragraph{v2.0:} 2018/12/30

\begin{itemize}
\item
immediate forward processing
\item
added |\childdocby| mechanism
\item
manual restructured
\end{itemize}

%%%%%%%%%%%%%%%%%%%%%%%%%%%%%%%%%%%%%%%%
\paragraph{v1.6:} 2018/01/17

\begin{itemize}
\item
application for development of include files
\item
corrections to manual
\end{itemize}

%%%%%%%%%%%%%%%%%%%%%%%%%%%%%%%%%%%%%%%%
\paragraph{v1.5:} 2017/05/21

\begin{itemize}
\item
more complete structuring introduced
\item
|\childdocof| introduced
\item
|\childdoc| renamed to |\childdocmain|
\item
|\childredirect| renamed to |\childdocforward| and |\childdocforwardprefix|
and functionality expanded
\end{itemize}

%%%%%%%%%%%%%%%%%%%%%%%%%%%%%%%%%%%%%%%%
\paragraph{v1.0:} 2017/04/27

\begin{itemize}
\item
manual and install package
\item
first version published on CTAN
\end{itemize}

%%%%%%%%%%%%%%%%%%%%%%%%%%%%%%%%%%%%%%%%
\paragraph{v0.6:} 2017/04/26

\begin{itemize}
\item
redirection mechanism added
\end{itemize}

%%%%%%%%%%%%%%%%%%%%%%%%%%%%%%%%%%%%%%%%
\paragraph{v0.5:} 2017/04/26

\begin{itemize}
\item
functionality in definition file
\end{itemize}


%%%%%%%%%%%%%%%%%%%%%%%%%%%%%%%%%%%%%%%%%%%%%%%%%%%%%%%%%%%%%%%%%%%%%%%%%%%%%%%%
%%%%%%%%%%%%%%%%%%%%%%%%%%%%%%%%%%%%%%%%%%%%%%%%%%%%%%%%%%%%%%%%%%%%%%%%%%%%%%%%
%%%%%%%%%%%%%%%%%%%%%%%%%%%%%%%%%%%%%%%%%%%%%%%%%%%%%%%%%%%%%%%%%%%%%%%%%%%%%%%%
\appendix

\settowidth\MacroIndent{\rmfamily\scriptsize 000\ }

 \DocInput{childdoc.dtx}

\end{document}
%</driver>
% \fi
%
% %%%%%%%%%%%%%%%%%%%%%%%%%%%%%%%%%%%%%%%%%%%%%%%%%%%%%%%%%%%%%%%%%%%%%%%%%%%%%%
% %%%%%%%%%%%%%%%%%%%%%%%%%%%%%%%%%%%%%%%%%%%%%%%%%%%%%%%%%%%%%%%%%%%%%%%%%%%%%%
% \section{Sample}
%\iffalse
%<*samplemain>
%\fi
%
% The following presents a sample document
% with two chapters, two parts, a title page,
% a compile flag as well as three forwarding files to set the flag.
% It consists of eight |.tex| files:
% \begin{center}
% \begin{tabular}{ll}
% |cdocsamp.tex|&main file\\
% |cdocsch1.tex|&include file for chapter 1\\
% |cdocsch2.tex|&include file for chapter 2\\
% |cdocspt3.tex|&include file for part 3\\
% |cdocspt4.tex|&include file for part 4\\
% |cdocsdrf.tex|&forwarding file for main file in draft mode\\
% |cdocsfi1.tex|&forwarding file for final version of chapter 1\\
% |cdocsfi2.tex|&forwarding file for final version of chapter 2\\
% \end{tabular}
% \end{center}
% Each of the eight files can be compiled directly by the \LaTeX{} compiler.
%
% %%%%%%%%%%%%%%%%%%%%%%%%%%%%%%%%%%%%%%
% \paragraph{Main File.}
%
% The main file is called |cdocsamp.tex|.
%
% Load the \textsf{childdoc} definitions and
% declare the filename for the main document:
%    \begin{macrocode}
\input{childdoc.def}
\childdocmain{}
%    \end{macrocode}

% Optional override for |\version| flag:
%    \begin{macrocode}
%%\ifchilddoc\else\providecommand{\version}{draft}\fi
%    \end{macrocode}

% Define the default values for the |\version| flag
% (|final| for the main file and |draft| for childs):
%    \begin{macrocode}
\ifchilddoc
\providecommand{\version}{draft}
\else
\providecommand{\version}{final}
\fi
%    \end{macrocode}

% Load the standard document class:
%    \begin{macrocode}
\documentclass[12pt]{article}
%    \end{macrocode}

% Start the document body:
%    \begin{macrocode}
\begin{document}
%    \end{macrocode}

% Declare a title page.
% Print title, part of document being processed and version flag:
%    \begin{macrocode}
\addtocounter{page}{-1}
\begin{center}
{\LARGE\bfseries{}childdoc example\par}
\vspace{1cm}
\ifchilddoc
\ifchilddocmanual part\else chapter\fi:
`\childdocname' of `\childdocjob'\par
\else
main document: `\childdocjob'\par
\fi
version: \version\par
\end{center}
\newpage
%    \end{macrocode}

% Manually include selected file,
% otherwise process as usual:
%    \begin{macrocode}
\ifchilddocmanual
\section*{part `\childdocname'}
\input{\childdocname}
\else
%    \end{macrocode}

% Include the two chapters:
%    \begin{macrocode}
\include{cdocsch1}
\include{cdocsch2}
%    \end{macrocode}

% Include the two parts unless only chapters should be displayed:
%    \begin{macrocode}
\ifchilddoc\else
\section{part three}
\input{cdocspt3}
\section{part four}
\input{cdocspt4}
\fi
%    \end{macrocode}

% Process as usual until here:
%    \begin{macrocode}
\fi
%    \end{macrocode}

% End of document body:
%    \begin{macrocode}
\end{document}
%    \end{macrocode}
%\iffalse
%</samplemain>
%\fi
%
% %%%%%%%%%%%%%%%%%%%%%%%%%%%%%%%%%%%%%%
% \paragraph{Chapter Include Files.}
%
% The include files are called |cdocsch1.tex| and |cdocsch2.tex|.
%
%\iffalse
%<*samplechap1|samplechap2>
%\fi

% Optional override for |\version| flag:
%    \begin{macrocode}
%%\providecommand{\version}{final}
%    \end{macrocode}

% Include the main document:
%    \begin{macrocode}
\input{childdoc.def}
\childdocof{cdocsamp}
%    \end{macrocode}

%\iffalse
%</samplechap1|samplechap2>
%\fi
%
%\iffalse
%<*samplechap1>
%\fi
% Some text for chapter 1:
%    \begin{macrocode}
\section{one}
some text in chapter one
%    \end{macrocode}

%\iffalse
%</samplechap1>
%\fi
% Some text for chapter 2:
%\iffalse
%<*samplechap2>
%\fi
%    \begin{macrocode}
\section{two}
more text in chapter two
%    \end{macrocode}

%\iffalse
%</samplechap2>
%\fi
%
% %%%%%%%%%%%%%%%%%%%%%%%%%%%%%%%%%%%%%%
% \paragraph{Part Include Files.}
%
% The include files are called |cdocspt3.tex| and |cdocspt4.tex|.
%
%\iffalse
%<*samplepart3|samplepart4>
%\fi

% Optional override for |\version| flag:
%    \begin{macrocode}
%%\providecommand{\version}{final}
%    \end{macrocode}

% Include the main document:
%    \begin{macrocode}
\input{childdoc.def}
\childdocby{cdocsamp}
%    \end{macrocode}

%\iffalse
%</samplepart3|samplepart4>
%\fi
%
%\iffalse
%<*samplepart3>
%\fi
% Some text for part 3:
%    \begin{macrocode}
some text in part three
%    \end{macrocode}

%\iffalse
%</samplepart3>
%\fi
% Some text for part 4:
%\iffalse
%<*samplepart4>
%\fi
%    \begin{macrocode}
more text in part four
%    \end{macrocode}

%\iffalse
%</samplepart4>
%\fi
%
% %%%%%%%%%%%%%%%%%%%%%%%%%%%%%%%%%%%%%%
% \paragraph{Forwarding for a Complete Draft.}
%
% The following forwarding file |cdocsdrf.tex|
% compiles the main document in draft mode:
%\iffalse
%<*sampledraft>
%\fi
%    \begin{macrocode}
\def\version{draft}
\input{childdoc.def}
\childdocforward{cdocsamp}
%    \end{macrocode}

%\iffalse
%</sampledraft>
%\fi
%
% %%%%%%%%%%%%%%%%%%%%%%%%%%%%%%%%%%%%%%
% \paragraph{Forwarding for Final Version of the Chapters.}
%
% The following forwarding files |cdocsfn1.tex| and |cdocsfn2.tex|
% (with identical content)
% compile the final versions of the child documents
% |cdocsch1.tex| and |cdocsch2.tex|, respectively:
%\iffalse
%<*samplefinal>
%\fi
%    \begin{macrocode}
\def\version{final}
\input{childdoc.def}
\childdocforwardprefix[cdocsamp]{cdocsfn}{cdocsch}
%    \end{macrocode}

%\iffalse
%</samplefinal>
%\fi
%
% %%%%%%%%%%%%%%%%%%%%%%%%%%%%%%%%%%%%%%
% \paragraph{Command Line Processing.}
%
% The following three command lines generate the output files
% |cdocscld|, |cdocscl1| and |cdocscl2|
% which should be identical to
% |cdocsdrf|, |cdocsch1| and |cdocsfn2|, respectively:
% \begin{center}
% \begin{tabular}{l}
% |latex -jobname cdocscld \|\\
% |  "\def\version{draft}\input{childdoc.def}\childdocforward{cdocsamp}"|\\
% |latex -jobname cdocscl1 \|\\
% |  "\input{childdoc.def}\childdocforward[cdocsamp]{cdocsch1}"|\\
% |latex -jobname cdocscl2 \|\\
% |  "\def\version{final}\input{childdoc.def}\childdocforward{cdocsch2}"|
% \end{tabular}
% \end{center}
% Note that the trailing backslash on each first line
% merely continues the input to the second line
% (for convenient cut ant paste).
% Furthermore, the command |latex| can be replaced by any
% of its alternative versions such as |pdflatex|.
%
% %%%%%%%%%%%%%%%%%%%%%%%%%%%%%%%%%%%%%%%%%%%%%%%%%%%%%%%%%%%%%%%%%%%%%%%%%%%%%%
% %%%%%%%%%%%%%%%%%%%%%%%%%%%%%%%%%%%%%%%%%%%%%%%%%%%%%%%%%%%%%%%%%%%%%%%%%%%%%%
% \section{Implementation}
%\iffalse
%<*package>
%\fi
%
% This section describes the definitions file |childdoc.def|.

% The definitions cannot be loaded using |\usepackage| or |\RequirePackage|
% which has a mechanism to prevent loading a style file more than once.
% When loading the definitions by means of |\input|
% multiple instances have to be prevented manually:
%\iffalse
%This code needs to be before the `\ProvidesFile' directive
%which is defined at the beginning of this file.
%Therefore it is also placed there and commented out here.
%</package>
%<*discard>
%\fi
%    \begin{macrocode}
\ifdefined\childdocmain\endinput\fi
%    \end{macrocode}
%\iffalse
%</discard>
%<*package>
%\fi
%
% \macro{\ifchilddoc}
% \macro{\ifchilddocmanual}
% The conditional |\ifchilddoc| tells whether a
% child (true) or main (false) document is being compiled.
% The conditional |\ifchilddocmanual| tells whether
% the |\includeonly| mechanism is used (false) or
% the selection of child files must be performed manually (true).
% The definitions initialise to false:
%    \begin{macrocode}
\newif\ifchilddoc
\newif\ifchilddocmanual
%    \end{macrocode}

% \macro{\childdocname}
% \macro{\childdocjob}
% The macro |\childdocname| stores the name of the main document
% to be compiled. The macro |\childdocjob| stores the name of
% the document on which the \LaTeX{} compiler was originally invoked.
% The content of |\jobname| cannot be compared
% to filenames specified in the source due to different catcodes.
% The following code rescans |\jobname|, stores the result
% in |\childdocname| and saves a copy in |\childdocjob|:
%    \begin{macrocode}
\edef\childdocname{\scantokens\expandafter{\jobname\noexpand}}
\let\childdocjob\childdocname
%    \end{macrocode}

% \macro{\childdocdisable}
% The macro |\childdocdisable| prevents the main file
% from being processed more than once.
% At this stage, the main document command |\childdocmain|
% is assumed to be called once again where it should do nothing.
% Any subsequent call to it should prevent
% a secondary processing of the main document
% It overwrites the forwarding commands
% |\childdocof| and |\childdocforward|
% with empty macros to prevent further inclusions of the main document:
%    \begin{macrocode}
\newcommand{\childdocdisable}
{
  \renewcommand{\childdocmain}[1]{\renewcommand{\childdocmain}[1]{\endinput}}
  \renewcommand{\childdocof}[1]{}
  \renewcommand{\childdocby}[2][]{}
  \renewcommand{\childdocforward}[2][]{}
  \renewcommand{\childdocdisable}{}
}
%    \end{macrocode}

% \macro{\childdocmain}
% The macro |\childdocmain| is to be called at the top of the main file
% with nothing or the main filename (without extension) as argument.
% First, it breaks loops.
% If the argument is not empty and does not match |\childdocname|
% (which is set by the first inclusion of |childdoc.def|),
% |\ifchilddoc| is set to true, |\includeonly| is applied to the child file
% and |\jobname| is set to the main file
% (for proper handling of |.aux| files):
%    \begin{macrocode}
\newcommand{\childdocmain}[1]
{
  \childdocdisable\childdocmain{}
  \if?#1?\else
    \begingroup
      \def\childdoctmp{#1}
      \ifx\childdoctmp\childdocname
        \def\childdoctmp{}
      \else
        \def\childdoctmp
        {
          \childdoctrue
          \includeonly{\childdocname}
          \def\childdocjob{#1}
          \def\jobname{#1}
        }
      \fi
      \expandafter
    \endgroup
    \childdoctmp
  \fi
}
%    \end{macrocode}

% \macro{\childdocof}
% The command |\childdocof| redirects
% compilation to the main file |#1|.
%    \begin{macrocode}
\newcommand{\childdocof}[1]
{
  \childdocdisable
  \childdoctrue
  \includeonly{\childdocname}
  \def\jobname{#1}
  \def\childdocjob{#1}
  \input{#1}
}
%    \end{macrocode}

% \macro{\childdocby}
% The command |\childdocby| ....
%    \begin{macrocode}
\newcommand{\childdocby}[2][]
{
  \childdocdisable
  \childdoctrue
  \childdocmanualtrue
  \if?#1?\else
    \def\jobname{#2}
  \fi
  \def\childdocjob{#2}
  \input{#2}
  \endinput
}
%    \end{macrocode}

% \macro{\childdocforward}
% The command |\childdocforward| redirects
% compilation to the main file or
% (if the optional argument is given) a child file.
% Parameters are set as if the main file
% or a child file starting with |\childdocof| was compiled.
% Then compilation is handed over to the main file:
%    \begin{macrocode}
\newcommand{\childdocforward}[2][]
{
  \begingroup
    \if?#1?
      \def\childdoctmp
      {
        \def\childdocname{#2}
        \def\childdocjob{#2}
        \def\jobname{#2}
        \input{#2}
        \endinput
      }
    \else
      \def\childdoctmp
      {
        \childdocdisable
        \def\childdocname{#2}
        \childdoctrue
        \includeonly{#2}
        \def\childdocjob{#1}
        \def\jobname{#1}
        \input{#1}
        \endinput
      }
    \fi
    \expandafter
  \endgroup
  \childdoctmp
}
%    \end{macrocode}

% \macro{\childdocforwardprefix}
% The command |\childdocforwardprefix| redirects
% compilation to the main or a child file by means of a pattern.
% The prefix |#1| in the current filename is replaced by |#2|
% and the suffix of the current filename is kept
% (it is assumed that the filename does not contain the substring `|~~~|'
% which is used as a delimiter).
% Compilation is handed over to the new file by |\childdocforward|:
%    \begin{macrocode}
\newcommand{\childdocforwardprefix}[3][]
{
  \begingroup
    \def\childdocextract #2##1~~~{\def\childdoctmp{\childdocforward[#1]{#3##1}}}
    \expandafter\childdocextract\childdocname~~~
    \expandafter
  \endgroup
  \childdoctmp
}
%    \end{macrocode}

% \macro{\childdoc}
% The deprecated macro |\childdoc| is a legacy version of |\childdocmain|:
%    \begin{macrocode}
\newcommand{\childdoc}{\childdocmain}
%    \end{macrocode}

% \macro{\childdocredirect}
% The deprecated macro |\childdocredirect| is a legacy version
% of |\childdocforward| and |\childdocforwardprefix|:
%    \begin{macrocode}
\newcommand{\childdocredirect}[2][]
{
  \begingroup
    \if?#1?
      \def\childdoctmp{\childdocforward{#2}}
    \else
      \def\childdoctmp{\childdocforwardprefix{#1}{#2}}
    \fi
    \expandafter
  \endgroup
  \childdoctmp
}
%    \end{macrocode}

%\iffalse
%</package>
%\fi
%
\endinput
|\\
|\childdocforward[|\textit{main}|]{|\textit{dest}|}|\\
\end{tabular}
\end{center}
%
The argument \textit{dest} is the destination file
(without extension).
It should be the main file or one of the child files.
Note that further \textsf{childdoc} directives
such as |\childdocof| and |\childdocforward|
in the indicated file will be processed in this form.
The optional argument \textit{main}
passes on directly to the main file \textit{main}
while pretending to compile the child \textit{dest}.
This form behaves as if \textit{dest}
issues |\childdocof{|\textit{main}|}| right away,
and no further \textsf{childdoc} directives will be processed.

%%%%%%%%%%%%%%%%%%%%%%%%%%%%%%%%%%%%%%%%
\DescribeMacro{\...prefix}
In the alternative form |\childdocforwardprefix|,
%
\begin{center}
\begin{tabular}{l}
|% \iffalse
%
% childdoc.dtx Copyright (C) 2017-2018 Niklas Beisert
%
% This work may be distributed and/or modified under the
% conditions of the LaTeX Project Public License, either version 1.3
% of this license or (at your option) any later version.
% The latest version of this license is in
%   http://www.latex-project.org/lppl.txt
% and version 1.3 or later is part of all distributions of LaTeX
% version 2005/12/01 or later.
%
% This work has the LPPL maintenance status `maintained'.
%
% The Current Maintainer of this work is Niklas Beisert.
%
% This work consists of the files childdoc.dtx and childdoc.ins
% and the derived files childdoc.def and cdocsamp.tex with
% cdocsch1.tex, cdocsch2.tex, cdocsdrf.tex, cdocsfn1.tex, cdocsfn2.tex.
%
%<package>\ifdefined\childdocmain\endinput\fi
%<package>\ProvidesFile{childdoc.def}[2018/12/30 v2.0 child document driver]
%<samplemain>\ProvidesFile{cdocsamp.tex}[2018/12/30 v2.0 sample for childdoc]
%<*driver>
%\ProvidesFile{childdoc.drv}[2018/12/30 v2.0 childdoc reference manual file]
\PassOptionsToClass{10pt,a4paper}{article}
\documentclass{ltxdoc}

\usepackage[margin=35mm]{geometry}
\usepackage{hyperref}
\usepackage{hyperxmp}
\usepackage[usenames]{color}

\hypersetup{colorlinks=true}
\hypersetup{pdfstartview=FitH}
\hypersetup{pdfpagemode=UseNone}
\hypersetup{pdfsource={}}
\hypersetup{pdflang={en-UK}}
\hypersetup{pdfcopyright={Copyright 2017-2018 Niklas Beisert.
  This work may be distributed and/or modified under the
  conditions of the LaTeX Project Public License, either version 1.3
  of this license or (at your option) any later version.}}
\hypersetup{pdflicenseurl={http://www.latex-project.org/lppl.txt}}
\hypersetup{pdfcontactaddress={ETH Zurich, ITP, HIT K,
  Wolfgang-Pauli-Strasse 27}}
\hypersetup{pdfcontactpostcode={8093}}
\hypersetup{pdfcontactcity={Zurich}}
\hypersetup{pdfcontactcountry={Switzerland}}
\hypersetup{pdfcontactemail={nbeisert@itp.phys.ethz.ch}}
\hypersetup{pdfcontacturl={http://people.phys.ethz.ch/\xmptilde nbeisert/}}

\newcommand{\secref}[1]{\hyperref[#1]{section \ref*{#1}}}

\parskip1ex
\parindent0pt
\let\olditemize\itemize
\def\itemize{\olditemize\parskip0pt}

\begin{document}

\title{The \textsf{childdoc} Package}
\hypersetup{pdftitle={The childdoc Package}}
\author{Niklas Beisert\\[2ex]
  Institut f\"ur Theoretische Physik\\
  Eidgen\"ossische Technische Hochschule Z\"urich\\
  Wolfgang-Pauli-Strasse 27, 8093 Z\"urich, Switzerland\\[1ex]
  \href{mailto:nbeisert@itp.phys.ethz.ch}
  {\texttt{nbeisert@itp.phys.ethz.ch}}}
\hypersetup{pdfauthor={Niklas Beisert}}
\hypersetup{pdfsubject={Manual for the LaTeX2e Package childdoc}}
\date{30 December 2018, \textsf{v2.0}}
\maketitle

\begin{abstract}\noindent
\textsf{childdoc} is a \LaTeXe{} package
that enables the direct compilation
of document sections included by |\include|
to individual files.
\end{abstract}

\begingroup
\parskip0ex
\tableofcontents
\endgroup

%%%%%%%%%%%%%%%%%%%%%%%%%%%%%%%%%%%%%%%%%%%%%%%%%%%%%%%%%%%%%%%%%%%%%%%%%%%%%%%%
%%%%%%%%%%%%%%%%%%%%%%%%%%%%%%%%%%%%%%%%%%%%%%%%%%%%%%%%%%%%%%%%%%%%%%%%%%%%%%%%
\section{Introduction}

\LaTeX{} provides a mechanism to structure a large document (such as a book)
into a main file and several child files (containing the chapters)
using the |\include| command.
This mechanism is beneficial for documents
which span hundreds of pages in order to
make the source file(s) more manageable.
Moreover, compilation can be restricted to
selected child files by means of the |\includeonly| command.
The latter feature can be used to reduce the compilation time while editing
(this was significantly more useful in the earlier days of \LaTeX{})
or to generate a smaller document which is easier to navigate.
Another application of |\includeonly| is to generate
documents consisting of selected parts of the complete document.

However, there are a few drawbacks of the plain |\include| mechanism:
\begin{itemize}
\item
The child files cannot be compiled on their own,
they can only be compiled via the main file.
A naive editing environment
(such as a text editor with an option
to have the current file processed by \LaTeX)
may require one to switch to the main file before compiling;
attempting to compile the child file produces errors.
\item
The main file must be modified (each time)
to adjust the |\includeonly| command
to the present needs. This easily leaves the main file in a messy state.
\item
The generated document will always carry the filename
of the main document. This is inconvenient if
several child files are to be compiled and
to be kept for distribution.
\end{itemize}

The present package provides a simple interface
to make child files individually compilable by \LaTeX{}.
Compiling a child file then has the same effect as compiling
the main file with an |\includeonly| command
to select the appropriate child.
Moreover the generated document will carry the name of the child
rather than the main file.
This resolves all three above issues.

This feature is meant to make the editing of books,
thesis documents and lecture notes somewhat more convenient.
However, the package can also be used efficiently for
composing a series of documents (such as exercise sheets)
which are typically distributed individually.
It then assists the author in generating the individual documents
(potentially in different versions)
as well as a document containing the collected series.
Another application is in developing style files
or other kinds of included material
where compilation of the style file could redirect
to a sample or test file.

%%%%%%%%%%%%%%%%%%%%%%%%%%%%%%%%%%%%%%%%%%%%%%%%%%%%%%%%%%%%%%%%%%%%%%%%%%%%%%%%
%%%%%%%%%%%%%%%%%%%%%%%%%%%%%%%%%%%%%%%%%%%%%%%%%%%%%%%%%%%%%%%%%%%%%%%%%%%%%%%%
\section{Usage}

First of all, the package \textsf{childdoc} is \emph{not} a standard
\LaTeXe{} |.sty| style file! Therefore it needs to be invoked in
a non-standard way.

%%%%%%%%%%%%%%%%%%%%%%%%%%%%%%%%%%%%%%%%%%%%%%%%%%%%%%%%%%%%%%%%%%%%%%%%%%%%%%%%
\subsection{Included Files}
\label{sec:include}

%%%%%%%%%%%%%%%%%%%%%%%%%%%%%%%%%%%%%%%%
\DescribeMacro{\childdocmain}
To use the package, add the commands
\begin{center}
\begin{tabular}{l}
|\input{childdoc.def}|\\
|\childdocmain{}|\\
\end{tabular}
\end{center}
at the very top of the main \LaTeX{} file,
in particular \emph{before} the |\documentclass| statement!
The argument of |\childdocmain| should be left empty
(but it must be present).

%%%%%%%%%%%%%%%%%%%%%%%%%%%%%%%%%%%%%%%%
\DescribeMacro{\childdocof}
Furthermore, add the commands
\begin{center}
\begin{tabular}{l}
|\input{childdoc.def}|\\
|\childdocof{|\textit{main}|}|\\
\end{tabular}
\end{center}
at the top of every child file \textit{child}
which is included by |\include{|\textit{child}|}|
from within the main file
(or at least for those files to be compiled individually).
The argument \textit{main} must be the filename of the main file.

There are a couple of
considerations in setting up the main and child documents:

%%%%%%%%%%%%%%%%%%%%%%%%%%%%%%%%%%%%%%%%
\paragraph{Restrictions.}

Please note the following restrictions:
\begin{itemize}
\item
|\childdocmain| must be called with one argument \textit{main}
to ensure compatibility with earlier version of the package.
It must either be empty (|\childdocmain{}|)
or precisely match the filename of the main file in which it is specified.
See \secref{sec:detection} for further information.
\item
The filename \textit{main} must be specified without the |.tex| extension.
\item
The filename \textit{main} is case sensitive
(even in case-insensitive file systems)
due to internal string comparison.
\item
The argument \textit{main} should be fully expanded, it cannot be a macro.
\item
Subdirectories and special characters should be avoided in filenames.
\item
The command |\childdocmain{|\textit{main}|}| must be followed by a whitespace.
It should not be followed immediately by another command
or by a comment mark `|%|'.
This is because the \TeX{} parser reads the token immediately following
the argument of |\childdocmain| and puts it
at the beginning of every child section;
however, a white\-space is ignored.
\end{itemize}

%%%%%%%%%%%%%%%%%%%%%%%%%%%%%%%%%%%%%%%%
\paragraph{Content of Main File.}

It is advisable to place all content in the child files included by |\include|.
Any output contained in the main file will appear in all child documents
unless suppressed manually;
it cannot be suppressed automatically by the |\includeonly| directive
and thus should normally be avoided.
A method to include some content in the main file
by means of conditional processing is described in \secref{sec:conditional}.

%%%%%%%%%%%%%%%%%%%%%%%%%%%%%%%%%%%%%%%%
\paragraph{Page Numbering.}

When only a part of the document is compiled,
the appropriate numbering of pages
(as well as other status parameters)
is determined from the |.aux| files.
The latter contain information from previous passes.
However this information needs to propagate through
all intermediate child documents.
Therefore the page numbering in child documents may well
be inconsistent until the complete document is compiled at least once.

A useful (if unconventional) way to always ensure a consistent
page numbering is to restart the numbering in each child document
and denote the pages by `\textit{child}|.|\textit{page}'
where \textit{child} represents the chapter/section number of the child file.
This can be achieved by the command
|\numberwithin{page}{|\textit{child}|}|
of the \textsf{amsmath} package
where \textit{child} can be |chapter| or |section|
depending on the chosen structuring.
Alternatively, one can modify the macro |\thepage| appropriately
and reset the counter |page| at the start of each child file.

%%%%%%%%%%%%%%%%%%%%%%%%%%%%%%%%%%%%%%%%%%%%%%%%%%%%%%%%%%%%%%%%%%%%%%%%%%%%%%%%
\subsection{Conditional Processing}
\label{sec:conditional}

The package provides a mechanism to compile different versions
of a document. To customise the versions further some conditional processing
can come in handy to distinguish which version is being compiled.
The package provides two macros to describe the compilation context:

%%%%%%%%%%%%%%%%%%%%%%%%%%%%%%%%%%%%%%%%
\DescribeMacro{\ifchilddoc}
The conditional |\ifchilddoc| distinguishes between the compilation of
child documents and the main document:
%
\begin{center}
|\ifchilddoc |\textit{child-code}| |[|\||else |\textit{main-code}]| \||fi|
\end{center}

%%%%%%%%%%%%%%%%%%%%%%%%%%%%%%%%%%%%%%%%
\DescribeMacro{\childdocname}
\DescribeMacro{\childdocjob}
The macro |\childdocname| contains the filename (without extension)
of the main or child file being processed.
Note that |\childdocjob| will always contain the name of the main file.

%%%%%%%%%%%%%%%%%%%%%%%%%%%%%%%%%%%%%%%%
\paragraph{Title Page.}

Conditional processing can be used to include a title or banner page
in the main document when proper precautions are taken.
Importantly, the code in the main file should ensure that the page counter
(as well as other status parameters which are stored in the |.aux| files)
takes the same value after the conditional processing.
Otherwise the page numbers may take divergent values
depending on which part is compiled.

For example, a title page could be declared by:
%
\begin{center}
\begin{tabular}{l}
|\ifchilddoc\||else|\\
|\addtocounter{page}{-1}|\\
\textit{code for title page}\\
|\newpage|\\
|\||fi|
\end{tabular}
\end{center}
%
A banner page for the child documents can be generated by:
%
\begin{center}
\begin{tabular}{l}
|\ifchilddoc|\\
|\addtocounter{page}{-1}|\\
\textit{code for banner page}\\
|\newpage|\\
|\||fi|
\end{tabular}
\end{center}
%
Here one could write a message such as:
\begin{center}
|This is the part \childdocname{} of \childdocjob{}.|
\end{center}

%%%%%%%%%%%%%%%%%%%%%%%%%%%%%%%%%%%%%%%%%%%%%%%%%%%%%%%%%%%%%%%%%%%%%%%%%%%%%%%%
\subsection{Flags}
\label{sec:flags}

The package makes it easy to generate different versions
of the main or child documents.
To this end compilation flags can be defined
and assigned different default values.
They will be particularly useful in conjunction
with the forwarding mechanism described in \secref{sec:forward}.

For example, it may be useful to have a flag |\version|
which can be set to |draft| or |final|.
The document source will contain some conditional code
depending on the value of |\version|.
Suppose further, the flag should default to |final| for the main file
and to |draft| for child files
which is a natural assignment for editing the document.
This is achieved by placing the following code
in the preamble of the main document
(below the |\childdocmain| directive):
%
\begin{center}
\begin{tabular}{l}
|\ifchilddoc|\\
|\providecommand{\version}{draft}|\\
|\||else|\\
|\providecommand{\version}{final}|\\
|\||fi|
\end{tabular}
\end{center}
%
The definition by |\providecommand| makes sure
that previous definitions are not overwritten.
Further statements |\providecommand{\version}{...}|
can thus be added before the above code to override it.

For the main file, one might add a line
(between |\childdocmain| and the above block)
%
\begin{center}
|%\ifchilddoc\||else\providecommand{\version}{draft}\||fi|
\end{center}
%
which can be uncommented to produce a draft version.
Likewise one can add a line to the very top of a child file
(above the |\childdocof{|\textit{main}|}| directive)
%
\begin{center}
|%\providecommand{\version}{final}|
\end{center}
%
which can be uncommented to produce the final version of this child document.

%%%%%%%%%%%%%%%%%%%%%%%%%%%%%%%%%%%%%%%%%%%%%%%%%%%%%%%%%%%%%%%%%%%%%%%%%%%%%%%%
\subsection{Forwarding}
\label{sec:forward}

Different versions of the main or child documents
using compilation flags as described in \secref{sec:flags}
can be (permanently) stored in different files
for convenient compilation, viewing and distribution.
To this end, the package defines a command
to pass on compilation to a different file:

%%%%%%%%%%%%%%%%%%%%%%%%%%%%%%%%%%%%%%%%
\DescribeMacro{\childdocforward}
The command |\childdocforward| redirects processing to
another source file:
%
\begin{center}
\begin{tabular}{l}
|\input{childdoc.def}|\\
|\childdocforward[|\textit{main}|]{|\textit{dest}|}|\\
\end{tabular}
\end{center}
%
The argument \textit{dest} is the destination file
(without extension).
It should be the main file or one of the child files.
Note that further \textsf{childdoc} directives
such as |\childdocof| and |\childdocforward|
in the indicated file will be processed in this form.
The optional argument \textit{main}
passes on directly to the main file \textit{main}
while pretending to compile the child \textit{dest}.
This form behaves as if \textit{dest}
issues |\childdocof{|\textit{main}|}| right away,
and no further \textsf{childdoc} directives will be processed.

%%%%%%%%%%%%%%%%%%%%%%%%%%%%%%%%%%%%%%%%
\DescribeMacro{\...prefix}
In the alternative form |\childdocforwardprefix|,
%
\begin{center}
\begin{tabular}{l}
|\input{childdoc.def}|\\
|\childdocforwardprefix[|\textit{main}|]{|\textit{prefix}|}{|\textit{dest}|}|
\end{tabular}
\end{center}
%
the destination file is determined by a pattern
depending on the current file:
To make this work, the current file must be called
`{\textit{prefix}\hspace{0.2em}\textit{suffix}}'
with \textit{prefix} matching precisely the argument.
Processing is then passed on to the file
`{\textit{dest}\hspace{0.2em}\textit{suffix}}'.
Surely, the same effect is achieved by
directly specifying the
argument `{\textit{dest}\hspace{0.2em}\textit{suffix}}'
in the first form.
However, that requires to set up a different file
for each child. With the alternative form of the command
all these files can have exactly the same content
which simplifies setting them up and maintaining them.

For example, the following file |draft.tex|
with a compilation flag |\version| as described in \secref{sec:flags}
compiles the main document as a draft:
%
\begin{center}
\begin{tabular}{l}
|\def\version{draft}|\\
|\input{childdoc.def}|\\
|\childdocforward{|\textit{main}|}|
\end{tabular}
\end{center}
%
Likewise, the following files |final|\textit{nn}|.tex|
compile the final version of the child document
|child|\textit{nn}|.tex|:
%
\begin{center}
\begin{tabular}{l}
|\def\version{final}|\\
|\input{childdoc.def}|\\
|\childdocforwardprefix{final}{child}|
\end{tabular}
\end{center}
%

Note that when several versions of a main file and/or of each child file
are to be generated, it may be convenient to set up a |Makefile| or
shell script to automatise the process.

%%%%%%%%%%%%%%%%%%%%%%%%%%%%%%%%%%%%%%%%%%%%%%%%%%%%%%%%%%%%%%%%%%%%%%%%%%%%%%%%
\subsection{Command Line Processing}
\label{sec:commandline}

The effect of redirection files can also be achieved by invoking
the \LaTeX{} compiler with a more elaborate command line.
Most conveniently this should be done as part
of a shell script or a |Makefile|.

When using \textsf{childdoc} in the main file, the following
command lines effectively perform a redirection
(note that depending on the shell being used,
backslashes may have to be doubled: `|\|' $\to$ `|\\|'):
%
\begin{center}
|... -jobname "|\textit{target}|" |\\|"|[\textit{flags}]%
|\input{childdoc.def}\childdocforward[|\textit{main}|]{|\textit{dest}|}"|
\end{center}
%
Here \textit{target} is the name of the output file,
\textit{main} is the name of the main file
and \textit{dest} is the name of the main or child file to be processed
(all filenames without extensions).
The optional argument \textit{main} can be omitted
if \textit{main} matches \textit{dest}.
Optionally, compilation \textit{flags} can be defined via |\def| commands.
This command line makes the \TeX{} engine believe
it is compiling the file \textit{target}
whose content is specified as the latter parameter.
The provided code then forwards the processing to
\textit{main} or \textit{dest} as described in \secref{sec:forward}.

%%%%%%%%%%%%%%%%%%%%%%%%%%%%%%%%%%%%%%%%%%%%%%%%%%%%%%%%%%%%%%%%%%%%%%%%%%%%%%%%
\subsection{Include by Input}
\label{sec:input}

Including child documents by |\include| has some restrictions by design.
Most notably, the content of a child document always occupies
its own set of pages; pages cannot be shared between child documents.
Usually, this behaviour makes perfect sense
because each child document contain an essential part of the document.
However, in some situations it may be desirable to compose
a document from a collection of parts
without having mandatory page breaks between then.
For this case, the package
provides a mechanism to include parts
by |\input| which can also be processed individually.
However, by construction this mechanism
requires manual handling of the content to be output.

%%%%%%%%%%%%%%%%%%%%%%%%%%%%%%%%%%%%%%%%
\DescribeMacro{\ifchilddocmanual}
The main file should be prepared as usual, see \secref{sec:include}.
However, the document body must make a distinction
between processing of an individual part and of the main document, e.g.:
%
\begin{center}
\begin{tabular}{l}
|\ifchilddocmanual|\\
|\input{\childdocname}|\\
|\||else|\\
\textit{document body with }|\input{|\textit{part}|}|\\
|\||fi|
\end{tabular}
\end{center}
%
The conditional |\ifchilddocmanual| is true whenever
a part to be included by |\input| is being compiled,
and the name of the part is stored in |\childdocname|.

%%%%%%%%%%%%%%%%%%%%%%%%%%%%%%%%%%%%%%%%
\DescribeMacro{\childdocby}
Each part to be included by |\input| should start with:
%
\begin{center}
\begin{tabular}{l}
|\input{childdoc.def}|\\
|\childdocby{|\textit{main}|}|\\
\end{tabular}
\end{center}
%
The directive |\childdocby| is similar to |\childdocof|
described in \secref{sec:include},
but the subsequent selection of content must be done manually.
To that end, both |\ifchilddoc| and |\ifchilddocmanual|
will be true upon processing of a part,
and the name of the part is stored in |\childdocname|.
Note that |\jobname| will be set to the filename of the current part
so that each part receives an individual |.aux| file
that does not interfere with the |.aux| file(s) of the main document.
This behaviour can be altered by the alternative form
|\childdocby[*]{|\textit{main}|}| (with a non-empty optional argument)
which uses the |.aux| file of the main document
by setting |\jobname| to \textit{main}.

%%%%%%%%%%%%%%%%%%%%%%%%%%%%%%%%%%%%%%%%%%%%%%%%%%%%%%%%%%%%%%%%%%%%%%%%%%%%%%%%
\subsection{Driver Development}
\label{sec:driver}

The \textsf{childdoc} mechanism can also be use for the development
of definition files such as \LaTeX{} styles or classes.
This case differs from the above setup with multiple parts
included by |\include| in that no |\includeonly| should be invoked.
This can be achieved by starting the include file
(before |\ProvidesPackage|) with:
%
\begin{center}
\begin{tabular}{l}
|\input{childdoc.def}|\\
|\childdocforward{|\textit{main}|}|\\
\end{tabular}
\end{center}
%
or alternatively with:
%
\begin{center}
\begin{tabular}{l}
|\input{childdoc.def}|\\
|\childdocby{|\textit{main}|}|\\
\end{tabular}
\end{center}
%
Both forms have slightly different effects as described above.
The main file is prepared as usual, see \secref{sec:include}.

%%%%%%%%%%%%%%%%%%%%%%%%%%%%%%%%%%%%%%%%%%%%%%%%%%%%%%%%%%%%%%%%%%%%%%%%%%%%%%%%
\subsection{Legacy Detection}
\label{sec:detection}

The directive |\childdocmain| in the main file can detect
whether the complete document or merely a child is to be compiled
even without using the directive |\childdocof|.
This method is deprecated because it is less robust
and there is no compelling reason to use it;
it is merely provided for backward compatibility
and it may be removed in future versions.

If the detection mechanism is to be used,
it is mandatory to correctly specify
the filename of the main file as the argument of |\childdocmain|:
%
\begin{center}
\begin{tabular}{l}
|\input{childdoc.def}|\\
|\childdocmain{|\textit{main}|}|\\
\end{tabular}
\end{center}
%
If |\jobname| does not match the argument \textit{main} of |\childdocmain|,
it is assumed that |\jobname| points to the child file to be compiled.
When using |\childdocmain| with the main file specified as argument,
it suffices to start a child file
with just |\input{|\textit{main}|}|
without loading of the package and using |\childdocof|.
If instead all processing is done
with the appropriate \textsf{childdoc} directives,
the argument of \textit{main} of |\childdocmain| can be empty.

An alternative version of the command line processing described
in \secref{sec:commandline} using the detection mechanism reads:
%
\begin{center}
|... -jobname "|\textit{target}|" "|[\textit{flags}]%
[|\def\jobname{|\textit{dest}|}|]|\input{|\textit{main}|}"|
\end{center}

%%%%%%%%%%%%%%%%%%%%%%%%%%%%%%%%%%%%%%%%%%%%%%%%%%%%%%%%%%%%%%%%%%%%%%%%%%%%%%%%
\subsection{Manual Code}
\label{sec:manual}

In case one cannot be certain whether the definitions file |childdoc.def|
is installed on the target \TeX{} distribution
and one prefers not to ship it,
it is conceivable to paste a few relevant commands into the sources.

To that end, drop all statements |\input{childdoc.def}|
and perform the replacements as outlined below.
Instead of |\childdocmain{|\textit{main}|}| add the following code
to the top of the main file:
%
\begin{center}
\begin{tabular}{l}
|\||ifdefined\childdocname\endinput\||fi\newif\ifchilddoc|\\
|\edef\childdocname{\scantokens\expandafter{\jobname\noexpand}}|\\
|\def\childdocmain{|\textit{main}|}\||ifx\childdocmain\childdocname\||else|\\
|\childdoctrue\includeonly{\childdocname}\let\jobname\childdocmain\||fi|\\
\end{tabular}
\end{center}
%
Instead of |\childdocof{|\textit{main}|}| just include the main file
at the top of each child file:
%
\begin{center}
|\input{|\textit{main}|}|
\end{center}
%
A simple redirection |\childdocforward{|\textit{dest}|}| is achieved by:
%
\begin{center}
|\def\jobname{|\textit{dest}|}\input{\jobname}|
\end{center}
%
The redirection with prefix
|\childdocforwardprefix[|\textit{prefix}|]{|\textit{dest}|}|
is accomplished by:
%
\begin{center}
\begin{tabular}{l}
|{\edef\jobname{\scantokens\expandafter{\jobname\noexpand}}|\\
|\def\redirectjob |\textit{prefix}|#1~~~{\gdef\jobname{|\textit{dest}|#1}}|\\
|\expandafter\redirectjob\jobname~~~}\input{\jobname}|
\end{tabular}
\end{center}

In an alternative approach,
child documents can be compiled by a specific command line
without additional code or specific definitions:
%
\begin{center}
|... -jobname "|\textit{target}|" "|[\textit{flags}]%
|\includeonly{|\textit{dest}|}\input{|\textit{main}|}"|
\end{center}
%

%%%%%%%%%%%%%%%%%%%%%%%%%%%%%%%%%%%%%%%%%%%%%%%%%%%%%%%%%%%%%%%%%%%%%%%%%%%%%%%%
%%%%%%%%%%%%%%%%%%%%%%%%%%%%%%%%%%%%%%%%%%%%%%%%%%%%%%%%%%%%%%%%%%%%%%%%%%%%%%%%
\section{Information}

%%%%%%%%%%%%%%%%%%%%%%%%%%%%%%%%%%%%%%%%%%%%%%%%%%%%%%%%%%%%%%%%%%%%%%%%%%%%%%%%
\subsection{Copyright}

Copyright \copyright{} 2017--2018 Niklas Beisert

This work may be distributed and/or modified under the
conditions of the \LaTeX{} Project Public License, either version 1.3
of this license or (at your option) any later version.
The latest version of this license is in
  \url{http://www.latex-project.org/lppl.txt}
and version 1.3 or later is part of all distributions of \LaTeX{}
version 2005/12/01 or later.

This work has the LPPL maintenance status `maintained'.

The Current Maintainer of this work is Niklas Beisert.

This work consists of the files |README.txt|, |childdoc.ins| and |childdoc.dtx|
as well as the derived files |childdoc.def|, |cdocsamp.tex|
with |cdocsch1.tex|, |cdocsch2.tex|, |cdocspt3.tex|, |cdocspt4.tex|,
|cdocsdrf.tex|, |cdocsfn1.tex|, |cdocsfn2.tex|
as well as |childdoc.pdf|.

%%%%%%%%%%%%%%%%%%%%%%%%%%%%%%%%%%%%%%%%%%%%%%%%%%%%%%%%%%%%%%%%%%%%%%%%%%%%%%%%
\subsection{Files and Installation}

The package consists of the files:
%
\begin{center}
\begin{tabular}{ll}
    |README.txt|   & readme file \\
    |childdoc.ins| & installation file \\
    |childdoc.dtx| & source file \\
    |childdoc.def| & definition file \\
    |cdocsamp.tex| & sample main file \\
    |cdocsch1.tex| & sample include file \\
    |cdocsch2.tex| & sample include file \\
    |cdocspt3.tex| & sample part file \\
    |cdocspt4.tex| & sample part file \\
    |cdocsdrf.tex| & sample redirection file \\
    |cdocsfn1.tex| & sample redirection file \\
    |cdocsfn2.tex| & sample redirection file \\
    |childdoc.pdf| & manual
\end{tabular}
\end{center}
%
The distribution consists of the files
|README.txt|, |childdoc.ins| and |childdoc.dtx|.
%
\begin{itemize}
\item
Run (pdf)\LaTeX{} on |childdoc.dtx|
to compile the manual |childdoc.pdf| (this file).
\item
Run \LaTeX{} on |childdoc.ins| to create the definitions file |childdoc.def|
and the sample |cdocsamp.tex| with include files
|cdocsch1.tex|, |cdocsch2.tex|, |cdocspt3.tex|, |cdocspt4.tex|,
|cdocsdrf.tex|, |cdocsfn1.tex|, |cdocsfn2.tex|.
Then copy the file |childdoc.def| to an appropriate directory of your \LaTeX{}
distribution, e.g.\ \textit{texmf-root}|/tex/latex/childdoc|.
\end{itemize}

%%%%%%%%%%%%%%%%%%%%%%%%%%%%%%%%%%%%%%%%%%%%%%%%%%%%%%%%%%%%%%%%%%%%%%%%%%%%%%%%
\subsection{Related CTAN Packages}

There are several other packages which offer a similar functionality:
%
\begin{itemize}
\item
The packages
\href{http://ctan.org/pkg/docmute}{\textsf{docmute}},
\href{http://ctan.org/pkg/includex}{\textsf{includex}} and
\href{http://ctan.org/pkg/standalone}{\textsf{standalone}}
provide commands to include only the document body of
a child file thus allowing both files to be compiled individually.
\item
The packages \href{http://ctan.org/pkg/subdocs}{\textsf{subdocs}}
and \href{http://ctan.org/pkg/subfiles}{\textsf{subfiles}}
provide structures in which the main and child documents can be
encapsulated and allowing them to be compiled individually.
The inclusion mechanism is different from the conventional |\include|.
\item
The package \href{http://ctan.org/pkg/combine}{\textsf{combine}}
is an elaborate solution to combine several documents into one.
\end{itemize}
%
See also the CTAN topic \href{http://ctan.org/topic/subdocs}{\textsf{subdocs}}
for further related packages.
The present package differs from the above solutions in that
a document structure constructed with the conventional |\include| mechanism
just needs two extra commands at the top of every file
such that all constituent files can be compiled individually.

%%%%%%%%%%%%%%%%%%%%%%%%%%%%%%%%%%%%%%%%%%%%%%%%%%%%%%%%%%%%%%%%%%%%%%%%%%%%%%%%
%\subsection{Feature Suggestions}
%
%The following is a list of features which may be useful for future
%versions of this package:
%%
%\begin{itemize}
%\item
%\ldots
%\end{itemize}

%%%%%%%%%%%%%%%%%%%%%%%%%%%%%%%%%%%%%%%%%%%%%%%%%%%%%%%%%%%%%%%%%%%%%%%%%%%%%%%%
\subsection{Revision History}

%%%%%%%%%%%%%%%%%%%%%%%%%%%%%%%%%%%%%%%%
\paragraph{v2.0:} 2018/12/30

\begin{itemize}
\item
immediate forward processing
\item
added |\childdocby| mechanism
\item
manual restructured
\end{itemize}

%%%%%%%%%%%%%%%%%%%%%%%%%%%%%%%%%%%%%%%%
\paragraph{v1.6:} 2018/01/17

\begin{itemize}
\item
application for development of include files
\item
corrections to manual
\end{itemize}

%%%%%%%%%%%%%%%%%%%%%%%%%%%%%%%%%%%%%%%%
\paragraph{v1.5:} 2017/05/21

\begin{itemize}
\item
more complete structuring introduced
\item
|\childdocof| introduced
\item
|\childdoc| renamed to |\childdocmain|
\item
|\childredirect| renamed to |\childdocforward| and |\childdocforwardprefix|
and functionality expanded
\end{itemize}

%%%%%%%%%%%%%%%%%%%%%%%%%%%%%%%%%%%%%%%%
\paragraph{v1.0:} 2017/04/27

\begin{itemize}
\item
manual and install package
\item
first version published on CTAN
\end{itemize}

%%%%%%%%%%%%%%%%%%%%%%%%%%%%%%%%%%%%%%%%
\paragraph{v0.6:} 2017/04/26

\begin{itemize}
\item
redirection mechanism added
\end{itemize}

%%%%%%%%%%%%%%%%%%%%%%%%%%%%%%%%%%%%%%%%
\paragraph{v0.5:} 2017/04/26

\begin{itemize}
\item
functionality in definition file
\end{itemize}


%%%%%%%%%%%%%%%%%%%%%%%%%%%%%%%%%%%%%%%%%%%%%%%%%%%%%%%%%%%%%%%%%%%%%%%%%%%%%%%%
%%%%%%%%%%%%%%%%%%%%%%%%%%%%%%%%%%%%%%%%%%%%%%%%%%%%%%%%%%%%%%%%%%%%%%%%%%%%%%%%
%%%%%%%%%%%%%%%%%%%%%%%%%%%%%%%%%%%%%%%%%%%%%%%%%%%%%%%%%%%%%%%%%%%%%%%%%%%%%%%%
\appendix

\settowidth\MacroIndent{\rmfamily\scriptsize 000\ }

 \DocInput{childdoc.dtx}

\end{document}
%</driver>
% \fi
%
% %%%%%%%%%%%%%%%%%%%%%%%%%%%%%%%%%%%%%%%%%%%%%%%%%%%%%%%%%%%%%%%%%%%%%%%%%%%%%%
% %%%%%%%%%%%%%%%%%%%%%%%%%%%%%%%%%%%%%%%%%%%%%%%%%%%%%%%%%%%%%%%%%%%%%%%%%%%%%%
% \section{Sample}
%\iffalse
%<*samplemain>
%\fi
%
% The following presents a sample document
% with two chapters, two parts, a title page,
% a compile flag as well as three forwarding files to set the flag.
% It consists of eight |.tex| files:
% \begin{center}
% \begin{tabular}{ll}
% |cdocsamp.tex|&main file\\
% |cdocsch1.tex|&include file for chapter 1\\
% |cdocsch2.tex|&include file for chapter 2\\
% |cdocspt3.tex|&include file for part 3\\
% |cdocspt4.tex|&include file for part 4\\
% |cdocsdrf.tex|&forwarding file for main file in draft mode\\
% |cdocsfi1.tex|&forwarding file for final version of chapter 1\\
% |cdocsfi2.tex|&forwarding file for final version of chapter 2\\
% \end{tabular}
% \end{center}
% Each of the eight files can be compiled directly by the \LaTeX{} compiler.
%
% %%%%%%%%%%%%%%%%%%%%%%%%%%%%%%%%%%%%%%
% \paragraph{Main File.}
%
% The main file is called |cdocsamp.tex|.
%
% Load the \textsf{childdoc} definitions and
% declare the filename for the main document:
%    \begin{macrocode}
\input{childdoc.def}
\childdocmain{}
%    \end{macrocode}

% Optional override for |\version| flag:
%    \begin{macrocode}
%%\ifchilddoc\else\providecommand{\version}{draft}\fi
%    \end{macrocode}

% Define the default values for the |\version| flag
% (|final| for the main file and |draft| for childs):
%    \begin{macrocode}
\ifchilddoc
\providecommand{\version}{draft}
\else
\providecommand{\version}{final}
\fi
%    \end{macrocode}

% Load the standard document class:
%    \begin{macrocode}
\documentclass[12pt]{article}
%    \end{macrocode}

% Start the document body:
%    \begin{macrocode}
\begin{document}
%    \end{macrocode}

% Declare a title page.
% Print title, part of document being processed and version flag:
%    \begin{macrocode}
\addtocounter{page}{-1}
\begin{center}
{\LARGE\bfseries{}childdoc example\par}
\vspace{1cm}
\ifchilddoc
\ifchilddocmanual part\else chapter\fi:
`\childdocname' of `\childdocjob'\par
\else
main document: `\childdocjob'\par
\fi
version: \version\par
\end{center}
\newpage
%    \end{macrocode}

% Manually include selected file,
% otherwise process as usual:
%    \begin{macrocode}
\ifchilddocmanual
\section*{part `\childdocname'}
\input{\childdocname}
\else
%    \end{macrocode}

% Include the two chapters:
%    \begin{macrocode}
\include{cdocsch1}
\include{cdocsch2}
%    \end{macrocode}

% Include the two parts unless only chapters should be displayed:
%    \begin{macrocode}
\ifchilddoc\else
\section{part three}
\input{cdocspt3}
\section{part four}
\input{cdocspt4}
\fi
%    \end{macrocode}

% Process as usual until here:
%    \begin{macrocode}
\fi
%    \end{macrocode}

% End of document body:
%    \begin{macrocode}
\end{document}
%    \end{macrocode}
%\iffalse
%</samplemain>
%\fi
%
% %%%%%%%%%%%%%%%%%%%%%%%%%%%%%%%%%%%%%%
% \paragraph{Chapter Include Files.}
%
% The include files are called |cdocsch1.tex| and |cdocsch2.tex|.
%
%\iffalse
%<*samplechap1|samplechap2>
%\fi

% Optional override for |\version| flag:
%    \begin{macrocode}
%%\providecommand{\version}{final}
%    \end{macrocode}

% Include the main document:
%    \begin{macrocode}
\input{childdoc.def}
\childdocof{cdocsamp}
%    \end{macrocode}

%\iffalse
%</samplechap1|samplechap2>
%\fi
%
%\iffalse
%<*samplechap1>
%\fi
% Some text for chapter 1:
%    \begin{macrocode}
\section{one}
some text in chapter one
%    \end{macrocode}

%\iffalse
%</samplechap1>
%\fi
% Some text for chapter 2:
%\iffalse
%<*samplechap2>
%\fi
%    \begin{macrocode}
\section{two}
more text in chapter two
%    \end{macrocode}

%\iffalse
%</samplechap2>
%\fi
%
% %%%%%%%%%%%%%%%%%%%%%%%%%%%%%%%%%%%%%%
% \paragraph{Part Include Files.}
%
% The include files are called |cdocspt3.tex| and |cdocspt4.tex|.
%
%\iffalse
%<*samplepart3|samplepart4>
%\fi

% Optional override for |\version| flag:
%    \begin{macrocode}
%%\providecommand{\version}{final}
%    \end{macrocode}

% Include the main document:
%    \begin{macrocode}
\input{childdoc.def}
\childdocby{cdocsamp}
%    \end{macrocode}

%\iffalse
%</samplepart3|samplepart4>
%\fi
%
%\iffalse
%<*samplepart3>
%\fi
% Some text for part 3:
%    \begin{macrocode}
some text in part three
%    \end{macrocode}

%\iffalse
%</samplepart3>
%\fi
% Some text for part 4:
%\iffalse
%<*samplepart4>
%\fi
%    \begin{macrocode}
more text in part four
%    \end{macrocode}

%\iffalse
%</samplepart4>
%\fi
%
% %%%%%%%%%%%%%%%%%%%%%%%%%%%%%%%%%%%%%%
% \paragraph{Forwarding for a Complete Draft.}
%
% The following forwarding file |cdocsdrf.tex|
% compiles the main document in draft mode:
%\iffalse
%<*sampledraft>
%\fi
%    \begin{macrocode}
\def\version{draft}
\input{childdoc.def}
\childdocforward{cdocsamp}
%    \end{macrocode}

%\iffalse
%</sampledraft>
%\fi
%
% %%%%%%%%%%%%%%%%%%%%%%%%%%%%%%%%%%%%%%
% \paragraph{Forwarding for Final Version of the Chapters.}
%
% The following forwarding files |cdocsfn1.tex| and |cdocsfn2.tex|
% (with identical content)
% compile the final versions of the child documents
% |cdocsch1.tex| and |cdocsch2.tex|, respectively:
%\iffalse
%<*samplefinal>
%\fi
%    \begin{macrocode}
\def\version{final}
\input{childdoc.def}
\childdocforwardprefix[cdocsamp]{cdocsfn}{cdocsch}
%    \end{macrocode}

%\iffalse
%</samplefinal>
%\fi
%
% %%%%%%%%%%%%%%%%%%%%%%%%%%%%%%%%%%%%%%
% \paragraph{Command Line Processing.}
%
% The following three command lines generate the output files
% |cdocscld|, |cdocscl1| and |cdocscl2|
% which should be identical to
% |cdocsdrf|, |cdocsch1| and |cdocsfn2|, respectively:
% \begin{center}
% \begin{tabular}{l}
% |latex -jobname cdocscld \|\\
% |  "\def\version{draft}\input{childdoc.def}\childdocforward{cdocsamp}"|\\
% |latex -jobname cdocscl1 \|\\
% |  "\input{childdoc.def}\childdocforward[cdocsamp]{cdocsch1}"|\\
% |latex -jobname cdocscl2 \|\\
% |  "\def\version{final}\input{childdoc.def}\childdocforward{cdocsch2}"|
% \end{tabular}
% \end{center}
% Note that the trailing backslash on each first line
% merely continues the input to the second line
% (for convenient cut ant paste).
% Furthermore, the command |latex| can be replaced by any
% of its alternative versions such as |pdflatex|.
%
% %%%%%%%%%%%%%%%%%%%%%%%%%%%%%%%%%%%%%%%%%%%%%%%%%%%%%%%%%%%%%%%%%%%%%%%%%%%%%%
% %%%%%%%%%%%%%%%%%%%%%%%%%%%%%%%%%%%%%%%%%%%%%%%%%%%%%%%%%%%%%%%%%%%%%%%%%%%%%%
% \section{Implementation}
%\iffalse
%<*package>
%\fi
%
% This section describes the definitions file |childdoc.def|.

% The definitions cannot be loaded using |\usepackage| or |\RequirePackage|
% which has a mechanism to prevent loading a style file more than once.
% When loading the definitions by means of |\input|
% multiple instances have to be prevented manually:
%\iffalse
%This code needs to be before the `\ProvidesFile' directive
%which is defined at the beginning of this file.
%Therefore it is also placed there and commented out here.
%</package>
%<*discard>
%\fi
%    \begin{macrocode}
\ifdefined\childdocmain\endinput\fi
%    \end{macrocode}
%\iffalse
%</discard>
%<*package>
%\fi
%
% \macro{\ifchilddoc}
% \macro{\ifchilddocmanual}
% The conditional |\ifchilddoc| tells whether a
% child (true) or main (false) document is being compiled.
% The conditional |\ifchilddocmanual| tells whether
% the |\includeonly| mechanism is used (false) or
% the selection of child files must be performed manually (true).
% The definitions initialise to false:
%    \begin{macrocode}
\newif\ifchilddoc
\newif\ifchilddocmanual
%    \end{macrocode}

% \macro{\childdocname}
% \macro{\childdocjob}
% The macro |\childdocname| stores the name of the main document
% to be compiled. The macro |\childdocjob| stores the name of
% the document on which the \LaTeX{} compiler was originally invoked.
% The content of |\jobname| cannot be compared
% to filenames specified in the source due to different catcodes.
% The following code rescans |\jobname|, stores the result
% in |\childdocname| and saves a copy in |\childdocjob|:
%    \begin{macrocode}
\edef\childdocname{\scantokens\expandafter{\jobname\noexpand}}
\let\childdocjob\childdocname
%    \end{macrocode}

% \macro{\childdocdisable}
% The macro |\childdocdisable| prevents the main file
% from being processed more than once.
% At this stage, the main document command |\childdocmain|
% is assumed to be called once again where it should do nothing.
% Any subsequent call to it should prevent
% a secondary processing of the main document
% It overwrites the forwarding commands
% |\childdocof| and |\childdocforward|
% with empty macros to prevent further inclusions of the main document:
%    \begin{macrocode}
\newcommand{\childdocdisable}
{
  \renewcommand{\childdocmain}[1]{\renewcommand{\childdocmain}[1]{\endinput}}
  \renewcommand{\childdocof}[1]{}
  \renewcommand{\childdocby}[2][]{}
  \renewcommand{\childdocforward}[2][]{}
  \renewcommand{\childdocdisable}{}
}
%    \end{macrocode}

% \macro{\childdocmain}
% The macro |\childdocmain| is to be called at the top of the main file
% with nothing or the main filename (without extension) as argument.
% First, it breaks loops.
% If the argument is not empty and does not match |\childdocname|
% (which is set by the first inclusion of |childdoc.def|),
% |\ifchilddoc| is set to true, |\includeonly| is applied to the child file
% and |\jobname| is set to the main file
% (for proper handling of |.aux| files):
%    \begin{macrocode}
\newcommand{\childdocmain}[1]
{
  \childdocdisable\childdocmain{}
  \if?#1?\else
    \begingroup
      \def\childdoctmp{#1}
      \ifx\childdoctmp\childdocname
        \def\childdoctmp{}
      \else
        \def\childdoctmp
        {
          \childdoctrue
          \includeonly{\childdocname}
          \def\childdocjob{#1}
          \def\jobname{#1}
        }
      \fi
      \expandafter
    \endgroup
    \childdoctmp
  \fi
}
%    \end{macrocode}

% \macro{\childdocof}
% The command |\childdocof| redirects
% compilation to the main file |#1|.
%    \begin{macrocode}
\newcommand{\childdocof}[1]
{
  \childdocdisable
  \childdoctrue
  \includeonly{\childdocname}
  \def\jobname{#1}
  \def\childdocjob{#1}
  \input{#1}
}
%    \end{macrocode}

% \macro{\childdocby}
% The command |\childdocby| ....
%    \begin{macrocode}
\newcommand{\childdocby}[2][]
{
  \childdocdisable
  \childdoctrue
  \childdocmanualtrue
  \if?#1?\else
    \def\jobname{#2}
  \fi
  \def\childdocjob{#2}
  \input{#2}
  \endinput
}
%    \end{macrocode}

% \macro{\childdocforward}
% The command |\childdocforward| redirects
% compilation to the main file or
% (if the optional argument is given) a child file.
% Parameters are set as if the main file
% or a child file starting with |\childdocof| was compiled.
% Then compilation is handed over to the main file:
%    \begin{macrocode}
\newcommand{\childdocforward}[2][]
{
  \begingroup
    \if?#1?
      \def\childdoctmp
      {
        \def\childdocname{#2}
        \def\childdocjob{#2}
        \def\jobname{#2}
        \input{#2}
        \endinput
      }
    \else
      \def\childdoctmp
      {
        \childdocdisable
        \def\childdocname{#2}
        \childdoctrue
        \includeonly{#2}
        \def\childdocjob{#1}
        \def\jobname{#1}
        \input{#1}
        \endinput
      }
    \fi
    \expandafter
  \endgroup
  \childdoctmp
}
%    \end{macrocode}

% \macro{\childdocforwardprefix}
% The command |\childdocforwardprefix| redirects
% compilation to the main or a child file by means of a pattern.
% The prefix |#1| in the current filename is replaced by |#2|
% and the suffix of the current filename is kept
% (it is assumed that the filename does not contain the substring `|~~~|'
% which is used as a delimiter).
% Compilation is handed over to the new file by |\childdocforward|:
%    \begin{macrocode}
\newcommand{\childdocforwardprefix}[3][]
{
  \begingroup
    \def\childdocextract #2##1~~~{\def\childdoctmp{\childdocforward[#1]{#3##1}}}
    \expandafter\childdocextract\childdocname~~~
    \expandafter
  \endgroup
  \childdoctmp
}
%    \end{macrocode}

% \macro{\childdoc}
% The deprecated macro |\childdoc| is a legacy version of |\childdocmain|:
%    \begin{macrocode}
\newcommand{\childdoc}{\childdocmain}
%    \end{macrocode}

% \macro{\childdocredirect}
% The deprecated macro |\childdocredirect| is a legacy version
% of |\childdocforward| and |\childdocforwardprefix|:
%    \begin{macrocode}
\newcommand{\childdocredirect}[2][]
{
  \begingroup
    \if?#1?
      \def\childdoctmp{\childdocforward{#2}}
    \else
      \def\childdoctmp{\childdocforwardprefix{#1}{#2}}
    \fi
    \expandafter
  \endgroup
  \childdoctmp
}
%    \end{macrocode}

%\iffalse
%</package>
%\fi
%
\endinput
|\\
|\childdocforwardprefix[|\textit{main}|]{|\textit{prefix}|}{|\textit{dest}|}|
\end{tabular}
\end{center}
%
the destination file is determined by a pattern
depending on the current file:
To make this work, the current file must be called
`{\textit{prefix}\hspace{0.2em}\textit{suffix}}'
with \textit{prefix} matching precisely the argument.
Processing is then passed on to the file
`{\textit{dest}\hspace{0.2em}\textit{suffix}}'.
Surely, the same effect is achieved by
directly specifying the
argument `{\textit{dest}\hspace{0.2em}\textit{suffix}}'
in the first form.
However, that requires to set up a different file
for each child. With the alternative form of the command
all these files can have exactly the same content
which simplifies setting them up and maintaining them.

For example, the following file |draft.tex|
with a compilation flag |\version| as described in \secref{sec:flags}
compiles the main document as a draft:
%
\begin{center}
\begin{tabular}{l}
|\def\version{draft}|\\
|% \iffalse
%
% childdoc.dtx Copyright (C) 2017-2018 Niklas Beisert
%
% This work may be distributed and/or modified under the
% conditions of the LaTeX Project Public License, either version 1.3
% of this license or (at your option) any later version.
% The latest version of this license is in
%   http://www.latex-project.org/lppl.txt
% and version 1.3 or later is part of all distributions of LaTeX
% version 2005/12/01 or later.
%
% This work has the LPPL maintenance status `maintained'.
%
% The Current Maintainer of this work is Niklas Beisert.
%
% This work consists of the files childdoc.dtx and childdoc.ins
% and the derived files childdoc.def and cdocsamp.tex with
% cdocsch1.tex, cdocsch2.tex, cdocsdrf.tex, cdocsfn1.tex, cdocsfn2.tex.
%
%<package>\ifdefined\childdocmain\endinput\fi
%<package>\ProvidesFile{childdoc.def}[2018/12/30 v2.0 child document driver]
%<samplemain>\ProvidesFile{cdocsamp.tex}[2018/12/30 v2.0 sample for childdoc]
%<*driver>
%\ProvidesFile{childdoc.drv}[2018/12/30 v2.0 childdoc reference manual file]
\PassOptionsToClass{10pt,a4paper}{article}
\documentclass{ltxdoc}

\usepackage[margin=35mm]{geometry}
\usepackage{hyperref}
\usepackage{hyperxmp}
\usepackage[usenames]{color}

\hypersetup{colorlinks=true}
\hypersetup{pdfstartview=FitH}
\hypersetup{pdfpagemode=UseNone}
\hypersetup{pdfsource={}}
\hypersetup{pdflang={en-UK}}
\hypersetup{pdfcopyright={Copyright 2017-2018 Niklas Beisert.
  This work may be distributed and/or modified under the
  conditions of the LaTeX Project Public License, either version 1.3
  of this license or (at your option) any later version.}}
\hypersetup{pdflicenseurl={http://www.latex-project.org/lppl.txt}}
\hypersetup{pdfcontactaddress={ETH Zurich, ITP, HIT K,
  Wolfgang-Pauli-Strasse 27}}
\hypersetup{pdfcontactpostcode={8093}}
\hypersetup{pdfcontactcity={Zurich}}
\hypersetup{pdfcontactcountry={Switzerland}}
\hypersetup{pdfcontactemail={nbeisert@itp.phys.ethz.ch}}
\hypersetup{pdfcontacturl={http://people.phys.ethz.ch/\xmptilde nbeisert/}}

\newcommand{\secref}[1]{\hyperref[#1]{section \ref*{#1}}}

\parskip1ex
\parindent0pt
\let\olditemize\itemize
\def\itemize{\olditemize\parskip0pt}

\begin{document}

\title{The \textsf{childdoc} Package}
\hypersetup{pdftitle={The childdoc Package}}
\author{Niklas Beisert\\[2ex]
  Institut f\"ur Theoretische Physik\\
  Eidgen\"ossische Technische Hochschule Z\"urich\\
  Wolfgang-Pauli-Strasse 27, 8093 Z\"urich, Switzerland\\[1ex]
  \href{mailto:nbeisert@itp.phys.ethz.ch}
  {\texttt{nbeisert@itp.phys.ethz.ch}}}
\hypersetup{pdfauthor={Niklas Beisert}}
\hypersetup{pdfsubject={Manual for the LaTeX2e Package childdoc}}
\date{30 December 2018, \textsf{v2.0}}
\maketitle

\begin{abstract}\noindent
\textsf{childdoc} is a \LaTeXe{} package
that enables the direct compilation
of document sections included by |\include|
to individual files.
\end{abstract}

\begingroup
\parskip0ex
\tableofcontents
\endgroup

%%%%%%%%%%%%%%%%%%%%%%%%%%%%%%%%%%%%%%%%%%%%%%%%%%%%%%%%%%%%%%%%%%%%%%%%%%%%%%%%
%%%%%%%%%%%%%%%%%%%%%%%%%%%%%%%%%%%%%%%%%%%%%%%%%%%%%%%%%%%%%%%%%%%%%%%%%%%%%%%%
\section{Introduction}

\LaTeX{} provides a mechanism to structure a large document (such as a book)
into a main file and several child files (containing the chapters)
using the |\include| command.
This mechanism is beneficial for documents
which span hundreds of pages in order to
make the source file(s) more manageable.
Moreover, compilation can be restricted to
selected child files by means of the |\includeonly| command.
The latter feature can be used to reduce the compilation time while editing
(this was significantly more useful in the earlier days of \LaTeX{})
or to generate a smaller document which is easier to navigate.
Another application of |\includeonly| is to generate
documents consisting of selected parts of the complete document.

However, there are a few drawbacks of the plain |\include| mechanism:
\begin{itemize}
\item
The child files cannot be compiled on their own,
they can only be compiled via the main file.
A naive editing environment
(such as a text editor with an option
to have the current file processed by \LaTeX)
may require one to switch to the main file before compiling;
attempting to compile the child file produces errors.
\item
The main file must be modified (each time)
to adjust the |\includeonly| command
to the present needs. This easily leaves the main file in a messy state.
\item
The generated document will always carry the filename
of the main document. This is inconvenient if
several child files are to be compiled and
to be kept for distribution.
\end{itemize}

The present package provides a simple interface
to make child files individually compilable by \LaTeX{}.
Compiling a child file then has the same effect as compiling
the main file with an |\includeonly| command
to select the appropriate child.
Moreover the generated document will carry the name of the child
rather than the main file.
This resolves all three above issues.

This feature is meant to make the editing of books,
thesis documents and lecture notes somewhat more convenient.
However, the package can also be used efficiently for
composing a series of documents (such as exercise sheets)
which are typically distributed individually.
It then assists the author in generating the individual documents
(potentially in different versions)
as well as a document containing the collected series.
Another application is in developing style files
or other kinds of included material
where compilation of the style file could redirect
to a sample or test file.

%%%%%%%%%%%%%%%%%%%%%%%%%%%%%%%%%%%%%%%%%%%%%%%%%%%%%%%%%%%%%%%%%%%%%%%%%%%%%%%%
%%%%%%%%%%%%%%%%%%%%%%%%%%%%%%%%%%%%%%%%%%%%%%%%%%%%%%%%%%%%%%%%%%%%%%%%%%%%%%%%
\section{Usage}

First of all, the package \textsf{childdoc} is \emph{not} a standard
\LaTeXe{} |.sty| style file! Therefore it needs to be invoked in
a non-standard way.

%%%%%%%%%%%%%%%%%%%%%%%%%%%%%%%%%%%%%%%%%%%%%%%%%%%%%%%%%%%%%%%%%%%%%%%%%%%%%%%%
\subsection{Included Files}
\label{sec:include}

%%%%%%%%%%%%%%%%%%%%%%%%%%%%%%%%%%%%%%%%
\DescribeMacro{\childdocmain}
To use the package, add the commands
\begin{center}
\begin{tabular}{l}
|\input{childdoc.def}|\\
|\childdocmain{}|\\
\end{tabular}
\end{center}
at the very top of the main \LaTeX{} file,
in particular \emph{before} the |\documentclass| statement!
The argument of |\childdocmain| should be left empty
(but it must be present).

%%%%%%%%%%%%%%%%%%%%%%%%%%%%%%%%%%%%%%%%
\DescribeMacro{\childdocof}
Furthermore, add the commands
\begin{center}
\begin{tabular}{l}
|\input{childdoc.def}|\\
|\childdocof{|\textit{main}|}|\\
\end{tabular}
\end{center}
at the top of every child file \textit{child}
which is included by |\include{|\textit{child}|}|
from within the main file
(or at least for those files to be compiled individually).
The argument \textit{main} must be the filename of the main file.

There are a couple of
considerations in setting up the main and child documents:

%%%%%%%%%%%%%%%%%%%%%%%%%%%%%%%%%%%%%%%%
\paragraph{Restrictions.}

Please note the following restrictions:
\begin{itemize}
\item
|\childdocmain| must be called with one argument \textit{main}
to ensure compatibility with earlier version of the package.
It must either be empty (|\childdocmain{}|)
or precisely match the filename of the main file in which it is specified.
See \secref{sec:detection} for further information.
\item
The filename \textit{main} must be specified without the |.tex| extension.
\item
The filename \textit{main} is case sensitive
(even in case-insensitive file systems)
due to internal string comparison.
\item
The argument \textit{main} should be fully expanded, it cannot be a macro.
\item
Subdirectories and special characters should be avoided in filenames.
\item
The command |\childdocmain{|\textit{main}|}| must be followed by a whitespace.
It should not be followed immediately by another command
or by a comment mark `|%|'.
This is because the \TeX{} parser reads the token immediately following
the argument of |\childdocmain| and puts it
at the beginning of every child section;
however, a white\-space is ignored.
\end{itemize}

%%%%%%%%%%%%%%%%%%%%%%%%%%%%%%%%%%%%%%%%
\paragraph{Content of Main File.}

It is advisable to place all content in the child files included by |\include|.
Any output contained in the main file will appear in all child documents
unless suppressed manually;
it cannot be suppressed automatically by the |\includeonly| directive
and thus should normally be avoided.
A method to include some content in the main file
by means of conditional processing is described in \secref{sec:conditional}.

%%%%%%%%%%%%%%%%%%%%%%%%%%%%%%%%%%%%%%%%
\paragraph{Page Numbering.}

When only a part of the document is compiled,
the appropriate numbering of pages
(as well as other status parameters)
is determined from the |.aux| files.
The latter contain information from previous passes.
However this information needs to propagate through
all intermediate child documents.
Therefore the page numbering in child documents may well
be inconsistent until the complete document is compiled at least once.

A useful (if unconventional) way to always ensure a consistent
page numbering is to restart the numbering in each child document
and denote the pages by `\textit{child}|.|\textit{page}'
where \textit{child} represents the chapter/section number of the child file.
This can be achieved by the command
|\numberwithin{page}{|\textit{child}|}|
of the \textsf{amsmath} package
where \textit{child} can be |chapter| or |section|
depending on the chosen structuring.
Alternatively, one can modify the macro |\thepage| appropriately
and reset the counter |page| at the start of each child file.

%%%%%%%%%%%%%%%%%%%%%%%%%%%%%%%%%%%%%%%%%%%%%%%%%%%%%%%%%%%%%%%%%%%%%%%%%%%%%%%%
\subsection{Conditional Processing}
\label{sec:conditional}

The package provides a mechanism to compile different versions
of a document. To customise the versions further some conditional processing
can come in handy to distinguish which version is being compiled.
The package provides two macros to describe the compilation context:

%%%%%%%%%%%%%%%%%%%%%%%%%%%%%%%%%%%%%%%%
\DescribeMacro{\ifchilddoc}
The conditional |\ifchilddoc| distinguishes between the compilation of
child documents and the main document:
%
\begin{center}
|\ifchilddoc |\textit{child-code}| |[|\||else |\textit{main-code}]| \||fi|
\end{center}

%%%%%%%%%%%%%%%%%%%%%%%%%%%%%%%%%%%%%%%%
\DescribeMacro{\childdocname}
\DescribeMacro{\childdocjob}
The macro |\childdocname| contains the filename (without extension)
of the main or child file being processed.
Note that |\childdocjob| will always contain the name of the main file.

%%%%%%%%%%%%%%%%%%%%%%%%%%%%%%%%%%%%%%%%
\paragraph{Title Page.}

Conditional processing can be used to include a title or banner page
in the main document when proper precautions are taken.
Importantly, the code in the main file should ensure that the page counter
(as well as other status parameters which are stored in the |.aux| files)
takes the same value after the conditional processing.
Otherwise the page numbers may take divergent values
depending on which part is compiled.

For example, a title page could be declared by:
%
\begin{center}
\begin{tabular}{l}
|\ifchilddoc\||else|\\
|\addtocounter{page}{-1}|\\
\textit{code for title page}\\
|\newpage|\\
|\||fi|
\end{tabular}
\end{center}
%
A banner page for the child documents can be generated by:
%
\begin{center}
\begin{tabular}{l}
|\ifchilddoc|\\
|\addtocounter{page}{-1}|\\
\textit{code for banner page}\\
|\newpage|\\
|\||fi|
\end{tabular}
\end{center}
%
Here one could write a message such as:
\begin{center}
|This is the part \childdocname{} of \childdocjob{}.|
\end{center}

%%%%%%%%%%%%%%%%%%%%%%%%%%%%%%%%%%%%%%%%%%%%%%%%%%%%%%%%%%%%%%%%%%%%%%%%%%%%%%%%
\subsection{Flags}
\label{sec:flags}

The package makes it easy to generate different versions
of the main or child documents.
To this end compilation flags can be defined
and assigned different default values.
They will be particularly useful in conjunction
with the forwarding mechanism described in \secref{sec:forward}.

For example, it may be useful to have a flag |\version|
which can be set to |draft| or |final|.
The document source will contain some conditional code
depending on the value of |\version|.
Suppose further, the flag should default to |final| for the main file
and to |draft| for child files
which is a natural assignment for editing the document.
This is achieved by placing the following code
in the preamble of the main document
(below the |\childdocmain| directive):
%
\begin{center}
\begin{tabular}{l}
|\ifchilddoc|\\
|\providecommand{\version}{draft}|\\
|\||else|\\
|\providecommand{\version}{final}|\\
|\||fi|
\end{tabular}
\end{center}
%
The definition by |\providecommand| makes sure
that previous definitions are not overwritten.
Further statements |\providecommand{\version}{...}|
can thus be added before the above code to override it.

For the main file, one might add a line
(between |\childdocmain| and the above block)
%
\begin{center}
|%\ifchilddoc\||else\providecommand{\version}{draft}\||fi|
\end{center}
%
which can be uncommented to produce a draft version.
Likewise one can add a line to the very top of a child file
(above the |\childdocof{|\textit{main}|}| directive)
%
\begin{center}
|%\providecommand{\version}{final}|
\end{center}
%
which can be uncommented to produce the final version of this child document.

%%%%%%%%%%%%%%%%%%%%%%%%%%%%%%%%%%%%%%%%%%%%%%%%%%%%%%%%%%%%%%%%%%%%%%%%%%%%%%%%
\subsection{Forwarding}
\label{sec:forward}

Different versions of the main or child documents
using compilation flags as described in \secref{sec:flags}
can be (permanently) stored in different files
for convenient compilation, viewing and distribution.
To this end, the package defines a command
to pass on compilation to a different file:

%%%%%%%%%%%%%%%%%%%%%%%%%%%%%%%%%%%%%%%%
\DescribeMacro{\childdocforward}
The command |\childdocforward| redirects processing to
another source file:
%
\begin{center}
\begin{tabular}{l}
|\input{childdoc.def}|\\
|\childdocforward[|\textit{main}|]{|\textit{dest}|}|\\
\end{tabular}
\end{center}
%
The argument \textit{dest} is the destination file
(without extension).
It should be the main file or one of the child files.
Note that further \textsf{childdoc} directives
such as |\childdocof| and |\childdocforward|
in the indicated file will be processed in this form.
The optional argument \textit{main}
passes on directly to the main file \textit{main}
while pretending to compile the child \textit{dest}.
This form behaves as if \textit{dest}
issues |\childdocof{|\textit{main}|}| right away,
and no further \textsf{childdoc} directives will be processed.

%%%%%%%%%%%%%%%%%%%%%%%%%%%%%%%%%%%%%%%%
\DescribeMacro{\...prefix}
In the alternative form |\childdocforwardprefix|,
%
\begin{center}
\begin{tabular}{l}
|\input{childdoc.def}|\\
|\childdocforwardprefix[|\textit{main}|]{|\textit{prefix}|}{|\textit{dest}|}|
\end{tabular}
\end{center}
%
the destination file is determined by a pattern
depending on the current file:
To make this work, the current file must be called
`{\textit{prefix}\hspace{0.2em}\textit{suffix}}'
with \textit{prefix} matching precisely the argument.
Processing is then passed on to the file
`{\textit{dest}\hspace{0.2em}\textit{suffix}}'.
Surely, the same effect is achieved by
directly specifying the
argument `{\textit{dest}\hspace{0.2em}\textit{suffix}}'
in the first form.
However, that requires to set up a different file
for each child. With the alternative form of the command
all these files can have exactly the same content
which simplifies setting them up and maintaining them.

For example, the following file |draft.tex|
with a compilation flag |\version| as described in \secref{sec:flags}
compiles the main document as a draft:
%
\begin{center}
\begin{tabular}{l}
|\def\version{draft}|\\
|\input{childdoc.def}|\\
|\childdocforward{|\textit{main}|}|
\end{tabular}
\end{center}
%
Likewise, the following files |final|\textit{nn}|.tex|
compile the final version of the child document
|child|\textit{nn}|.tex|:
%
\begin{center}
\begin{tabular}{l}
|\def\version{final}|\\
|\input{childdoc.def}|\\
|\childdocforwardprefix{final}{child}|
\end{tabular}
\end{center}
%

Note that when several versions of a main file and/or of each child file
are to be generated, it may be convenient to set up a |Makefile| or
shell script to automatise the process.

%%%%%%%%%%%%%%%%%%%%%%%%%%%%%%%%%%%%%%%%%%%%%%%%%%%%%%%%%%%%%%%%%%%%%%%%%%%%%%%%
\subsection{Command Line Processing}
\label{sec:commandline}

The effect of redirection files can also be achieved by invoking
the \LaTeX{} compiler with a more elaborate command line.
Most conveniently this should be done as part
of a shell script or a |Makefile|.

When using \textsf{childdoc} in the main file, the following
command lines effectively perform a redirection
(note that depending on the shell being used,
backslashes may have to be doubled: `|\|' $\to$ `|\\|'):
%
\begin{center}
|... -jobname "|\textit{target}|" |\\|"|[\textit{flags}]%
|\input{childdoc.def}\childdocforward[|\textit{main}|]{|\textit{dest}|}"|
\end{center}
%
Here \textit{target} is the name of the output file,
\textit{main} is the name of the main file
and \textit{dest} is the name of the main or child file to be processed
(all filenames without extensions).
The optional argument \textit{main} can be omitted
if \textit{main} matches \textit{dest}.
Optionally, compilation \textit{flags} can be defined via |\def| commands.
This command line makes the \TeX{} engine believe
it is compiling the file \textit{target}
whose content is specified as the latter parameter.
The provided code then forwards the processing to
\textit{main} or \textit{dest} as described in \secref{sec:forward}.

%%%%%%%%%%%%%%%%%%%%%%%%%%%%%%%%%%%%%%%%%%%%%%%%%%%%%%%%%%%%%%%%%%%%%%%%%%%%%%%%
\subsection{Include by Input}
\label{sec:input}

Including child documents by |\include| has some restrictions by design.
Most notably, the content of a child document always occupies
its own set of pages; pages cannot be shared between child documents.
Usually, this behaviour makes perfect sense
because each child document contain an essential part of the document.
However, in some situations it may be desirable to compose
a document from a collection of parts
without having mandatory page breaks between then.
For this case, the package
provides a mechanism to include parts
by |\input| which can also be processed individually.
However, by construction this mechanism
requires manual handling of the content to be output.

%%%%%%%%%%%%%%%%%%%%%%%%%%%%%%%%%%%%%%%%
\DescribeMacro{\ifchilddocmanual}
The main file should be prepared as usual, see \secref{sec:include}.
However, the document body must make a distinction
between processing of an individual part and of the main document, e.g.:
%
\begin{center}
\begin{tabular}{l}
|\ifchilddocmanual|\\
|\input{\childdocname}|\\
|\||else|\\
\textit{document body with }|\input{|\textit{part}|}|\\
|\||fi|
\end{tabular}
\end{center}
%
The conditional |\ifchilddocmanual| is true whenever
a part to be included by |\input| is being compiled,
and the name of the part is stored in |\childdocname|.

%%%%%%%%%%%%%%%%%%%%%%%%%%%%%%%%%%%%%%%%
\DescribeMacro{\childdocby}
Each part to be included by |\input| should start with:
%
\begin{center}
\begin{tabular}{l}
|\input{childdoc.def}|\\
|\childdocby{|\textit{main}|}|\\
\end{tabular}
\end{center}
%
The directive |\childdocby| is similar to |\childdocof|
described in \secref{sec:include},
but the subsequent selection of content must be done manually.
To that end, both |\ifchilddoc| and |\ifchilddocmanual|
will be true upon processing of a part,
and the name of the part is stored in |\childdocname|.
Note that |\jobname| will be set to the filename of the current part
so that each part receives an individual |.aux| file
that does not interfere with the |.aux| file(s) of the main document.
This behaviour can be altered by the alternative form
|\childdocby[*]{|\textit{main}|}| (with a non-empty optional argument)
which uses the |.aux| file of the main document
by setting |\jobname| to \textit{main}.

%%%%%%%%%%%%%%%%%%%%%%%%%%%%%%%%%%%%%%%%%%%%%%%%%%%%%%%%%%%%%%%%%%%%%%%%%%%%%%%%
\subsection{Driver Development}
\label{sec:driver}

The \textsf{childdoc} mechanism can also be use for the development
of definition files such as \LaTeX{} styles or classes.
This case differs from the above setup with multiple parts
included by |\include| in that no |\includeonly| should be invoked.
This can be achieved by starting the include file
(before |\ProvidesPackage|) with:
%
\begin{center}
\begin{tabular}{l}
|\input{childdoc.def}|\\
|\childdocforward{|\textit{main}|}|\\
\end{tabular}
\end{center}
%
or alternatively with:
%
\begin{center}
\begin{tabular}{l}
|\input{childdoc.def}|\\
|\childdocby{|\textit{main}|}|\\
\end{tabular}
\end{center}
%
Both forms have slightly different effects as described above.
The main file is prepared as usual, see \secref{sec:include}.

%%%%%%%%%%%%%%%%%%%%%%%%%%%%%%%%%%%%%%%%%%%%%%%%%%%%%%%%%%%%%%%%%%%%%%%%%%%%%%%%
\subsection{Legacy Detection}
\label{sec:detection}

The directive |\childdocmain| in the main file can detect
whether the complete document or merely a child is to be compiled
even without using the directive |\childdocof|.
This method is deprecated because it is less robust
and there is no compelling reason to use it;
it is merely provided for backward compatibility
and it may be removed in future versions.

If the detection mechanism is to be used,
it is mandatory to correctly specify
the filename of the main file as the argument of |\childdocmain|:
%
\begin{center}
\begin{tabular}{l}
|\input{childdoc.def}|\\
|\childdocmain{|\textit{main}|}|\\
\end{tabular}
\end{center}
%
If |\jobname| does not match the argument \textit{main} of |\childdocmain|,
it is assumed that |\jobname| points to the child file to be compiled.
When using |\childdocmain| with the main file specified as argument,
it suffices to start a child file
with just |\input{|\textit{main}|}|
without loading of the package and using |\childdocof|.
If instead all processing is done
with the appropriate \textsf{childdoc} directives,
the argument of \textit{main} of |\childdocmain| can be empty.

An alternative version of the command line processing described
in \secref{sec:commandline} using the detection mechanism reads:
%
\begin{center}
|... -jobname "|\textit{target}|" "|[\textit{flags}]%
[|\def\jobname{|\textit{dest}|}|]|\input{|\textit{main}|}"|
\end{center}

%%%%%%%%%%%%%%%%%%%%%%%%%%%%%%%%%%%%%%%%%%%%%%%%%%%%%%%%%%%%%%%%%%%%%%%%%%%%%%%%
\subsection{Manual Code}
\label{sec:manual}

In case one cannot be certain whether the definitions file |childdoc.def|
is installed on the target \TeX{} distribution
and one prefers not to ship it,
it is conceivable to paste a few relevant commands into the sources.

To that end, drop all statements |\input{childdoc.def}|
and perform the replacements as outlined below.
Instead of |\childdocmain{|\textit{main}|}| add the following code
to the top of the main file:
%
\begin{center}
\begin{tabular}{l}
|\||ifdefined\childdocname\endinput\||fi\newif\ifchilddoc|\\
|\edef\childdocname{\scantokens\expandafter{\jobname\noexpand}}|\\
|\def\childdocmain{|\textit{main}|}\||ifx\childdocmain\childdocname\||else|\\
|\childdoctrue\includeonly{\childdocname}\let\jobname\childdocmain\||fi|\\
\end{tabular}
\end{center}
%
Instead of |\childdocof{|\textit{main}|}| just include the main file
at the top of each child file:
%
\begin{center}
|\input{|\textit{main}|}|
\end{center}
%
A simple redirection |\childdocforward{|\textit{dest}|}| is achieved by:
%
\begin{center}
|\def\jobname{|\textit{dest}|}\input{\jobname}|
\end{center}
%
The redirection with prefix
|\childdocforwardprefix[|\textit{prefix}|]{|\textit{dest}|}|
is accomplished by:
%
\begin{center}
\begin{tabular}{l}
|{\edef\jobname{\scantokens\expandafter{\jobname\noexpand}}|\\
|\def\redirectjob |\textit{prefix}|#1~~~{\gdef\jobname{|\textit{dest}|#1}}|\\
|\expandafter\redirectjob\jobname~~~}\input{\jobname}|
\end{tabular}
\end{center}

In an alternative approach,
child documents can be compiled by a specific command line
without additional code or specific definitions:
%
\begin{center}
|... -jobname "|\textit{target}|" "|[\textit{flags}]%
|\includeonly{|\textit{dest}|}\input{|\textit{main}|}"|
\end{center}
%

%%%%%%%%%%%%%%%%%%%%%%%%%%%%%%%%%%%%%%%%%%%%%%%%%%%%%%%%%%%%%%%%%%%%%%%%%%%%%%%%
%%%%%%%%%%%%%%%%%%%%%%%%%%%%%%%%%%%%%%%%%%%%%%%%%%%%%%%%%%%%%%%%%%%%%%%%%%%%%%%%
\section{Information}

%%%%%%%%%%%%%%%%%%%%%%%%%%%%%%%%%%%%%%%%%%%%%%%%%%%%%%%%%%%%%%%%%%%%%%%%%%%%%%%%
\subsection{Copyright}

Copyright \copyright{} 2017--2018 Niklas Beisert

This work may be distributed and/or modified under the
conditions of the \LaTeX{} Project Public License, either version 1.3
of this license or (at your option) any later version.
The latest version of this license is in
  \url{http://www.latex-project.org/lppl.txt}
and version 1.3 or later is part of all distributions of \LaTeX{}
version 2005/12/01 or later.

This work has the LPPL maintenance status `maintained'.

The Current Maintainer of this work is Niklas Beisert.

This work consists of the files |README.txt|, |childdoc.ins| and |childdoc.dtx|
as well as the derived files |childdoc.def|, |cdocsamp.tex|
with |cdocsch1.tex|, |cdocsch2.tex|, |cdocspt3.tex|, |cdocspt4.tex|,
|cdocsdrf.tex|, |cdocsfn1.tex|, |cdocsfn2.tex|
as well as |childdoc.pdf|.

%%%%%%%%%%%%%%%%%%%%%%%%%%%%%%%%%%%%%%%%%%%%%%%%%%%%%%%%%%%%%%%%%%%%%%%%%%%%%%%%
\subsection{Files and Installation}

The package consists of the files:
%
\begin{center}
\begin{tabular}{ll}
    |README.txt|   & readme file \\
    |childdoc.ins| & installation file \\
    |childdoc.dtx| & source file \\
    |childdoc.def| & definition file \\
    |cdocsamp.tex| & sample main file \\
    |cdocsch1.tex| & sample include file \\
    |cdocsch2.tex| & sample include file \\
    |cdocspt3.tex| & sample part file \\
    |cdocspt4.tex| & sample part file \\
    |cdocsdrf.tex| & sample redirection file \\
    |cdocsfn1.tex| & sample redirection file \\
    |cdocsfn2.tex| & sample redirection file \\
    |childdoc.pdf| & manual
\end{tabular}
\end{center}
%
The distribution consists of the files
|README.txt|, |childdoc.ins| and |childdoc.dtx|.
%
\begin{itemize}
\item
Run (pdf)\LaTeX{} on |childdoc.dtx|
to compile the manual |childdoc.pdf| (this file).
\item
Run \LaTeX{} on |childdoc.ins| to create the definitions file |childdoc.def|
and the sample |cdocsamp.tex| with include files
|cdocsch1.tex|, |cdocsch2.tex|, |cdocspt3.tex|, |cdocspt4.tex|,
|cdocsdrf.tex|, |cdocsfn1.tex|, |cdocsfn2.tex|.
Then copy the file |childdoc.def| to an appropriate directory of your \LaTeX{}
distribution, e.g.\ \textit{texmf-root}|/tex/latex/childdoc|.
\end{itemize}

%%%%%%%%%%%%%%%%%%%%%%%%%%%%%%%%%%%%%%%%%%%%%%%%%%%%%%%%%%%%%%%%%%%%%%%%%%%%%%%%
\subsection{Related CTAN Packages}

There are several other packages which offer a similar functionality:
%
\begin{itemize}
\item
The packages
\href{http://ctan.org/pkg/docmute}{\textsf{docmute}},
\href{http://ctan.org/pkg/includex}{\textsf{includex}} and
\href{http://ctan.org/pkg/standalone}{\textsf{standalone}}
provide commands to include only the document body of
a child file thus allowing both files to be compiled individually.
\item
The packages \href{http://ctan.org/pkg/subdocs}{\textsf{subdocs}}
and \href{http://ctan.org/pkg/subfiles}{\textsf{subfiles}}
provide structures in which the main and child documents can be
encapsulated and allowing them to be compiled individually.
The inclusion mechanism is different from the conventional |\include|.
\item
The package \href{http://ctan.org/pkg/combine}{\textsf{combine}}
is an elaborate solution to combine several documents into one.
\end{itemize}
%
See also the CTAN topic \href{http://ctan.org/topic/subdocs}{\textsf{subdocs}}
for further related packages.
The present package differs from the above solutions in that
a document structure constructed with the conventional |\include| mechanism
just needs two extra commands at the top of every file
such that all constituent files can be compiled individually.

%%%%%%%%%%%%%%%%%%%%%%%%%%%%%%%%%%%%%%%%%%%%%%%%%%%%%%%%%%%%%%%%%%%%%%%%%%%%%%%%
%\subsection{Feature Suggestions}
%
%The following is a list of features which may be useful for future
%versions of this package:
%%
%\begin{itemize}
%\item
%\ldots
%\end{itemize}

%%%%%%%%%%%%%%%%%%%%%%%%%%%%%%%%%%%%%%%%%%%%%%%%%%%%%%%%%%%%%%%%%%%%%%%%%%%%%%%%
\subsection{Revision History}

%%%%%%%%%%%%%%%%%%%%%%%%%%%%%%%%%%%%%%%%
\paragraph{v2.0:} 2018/12/30

\begin{itemize}
\item
immediate forward processing
\item
added |\childdocby| mechanism
\item
manual restructured
\end{itemize}

%%%%%%%%%%%%%%%%%%%%%%%%%%%%%%%%%%%%%%%%
\paragraph{v1.6:} 2018/01/17

\begin{itemize}
\item
application for development of include files
\item
corrections to manual
\end{itemize}

%%%%%%%%%%%%%%%%%%%%%%%%%%%%%%%%%%%%%%%%
\paragraph{v1.5:} 2017/05/21

\begin{itemize}
\item
more complete structuring introduced
\item
|\childdocof| introduced
\item
|\childdoc| renamed to |\childdocmain|
\item
|\childredirect| renamed to |\childdocforward| and |\childdocforwardprefix|
and functionality expanded
\end{itemize}

%%%%%%%%%%%%%%%%%%%%%%%%%%%%%%%%%%%%%%%%
\paragraph{v1.0:} 2017/04/27

\begin{itemize}
\item
manual and install package
\item
first version published on CTAN
\end{itemize}

%%%%%%%%%%%%%%%%%%%%%%%%%%%%%%%%%%%%%%%%
\paragraph{v0.6:} 2017/04/26

\begin{itemize}
\item
redirection mechanism added
\end{itemize}

%%%%%%%%%%%%%%%%%%%%%%%%%%%%%%%%%%%%%%%%
\paragraph{v0.5:} 2017/04/26

\begin{itemize}
\item
functionality in definition file
\end{itemize}


%%%%%%%%%%%%%%%%%%%%%%%%%%%%%%%%%%%%%%%%%%%%%%%%%%%%%%%%%%%%%%%%%%%%%%%%%%%%%%%%
%%%%%%%%%%%%%%%%%%%%%%%%%%%%%%%%%%%%%%%%%%%%%%%%%%%%%%%%%%%%%%%%%%%%%%%%%%%%%%%%
%%%%%%%%%%%%%%%%%%%%%%%%%%%%%%%%%%%%%%%%%%%%%%%%%%%%%%%%%%%%%%%%%%%%%%%%%%%%%%%%
\appendix

\settowidth\MacroIndent{\rmfamily\scriptsize 000\ }

 \DocInput{childdoc.dtx}

\end{document}
%</driver>
% \fi
%
% %%%%%%%%%%%%%%%%%%%%%%%%%%%%%%%%%%%%%%%%%%%%%%%%%%%%%%%%%%%%%%%%%%%%%%%%%%%%%%
% %%%%%%%%%%%%%%%%%%%%%%%%%%%%%%%%%%%%%%%%%%%%%%%%%%%%%%%%%%%%%%%%%%%%%%%%%%%%%%
% \section{Sample}
%\iffalse
%<*samplemain>
%\fi
%
% The following presents a sample document
% with two chapters, two parts, a title page,
% a compile flag as well as three forwarding files to set the flag.
% It consists of eight |.tex| files:
% \begin{center}
% \begin{tabular}{ll}
% |cdocsamp.tex|&main file\\
% |cdocsch1.tex|&include file for chapter 1\\
% |cdocsch2.tex|&include file for chapter 2\\
% |cdocspt3.tex|&include file for part 3\\
% |cdocspt4.tex|&include file for part 4\\
% |cdocsdrf.tex|&forwarding file for main file in draft mode\\
% |cdocsfi1.tex|&forwarding file for final version of chapter 1\\
% |cdocsfi2.tex|&forwarding file for final version of chapter 2\\
% \end{tabular}
% \end{center}
% Each of the eight files can be compiled directly by the \LaTeX{} compiler.
%
% %%%%%%%%%%%%%%%%%%%%%%%%%%%%%%%%%%%%%%
% \paragraph{Main File.}
%
% The main file is called |cdocsamp.tex|.
%
% Load the \textsf{childdoc} definitions and
% declare the filename for the main document:
%    \begin{macrocode}
\input{childdoc.def}
\childdocmain{}
%    \end{macrocode}

% Optional override for |\version| flag:
%    \begin{macrocode}
%%\ifchilddoc\else\providecommand{\version}{draft}\fi
%    \end{macrocode}

% Define the default values for the |\version| flag
% (|final| for the main file and |draft| for childs):
%    \begin{macrocode}
\ifchilddoc
\providecommand{\version}{draft}
\else
\providecommand{\version}{final}
\fi
%    \end{macrocode}

% Load the standard document class:
%    \begin{macrocode}
\documentclass[12pt]{article}
%    \end{macrocode}

% Start the document body:
%    \begin{macrocode}
\begin{document}
%    \end{macrocode}

% Declare a title page.
% Print title, part of document being processed and version flag:
%    \begin{macrocode}
\addtocounter{page}{-1}
\begin{center}
{\LARGE\bfseries{}childdoc example\par}
\vspace{1cm}
\ifchilddoc
\ifchilddocmanual part\else chapter\fi:
`\childdocname' of `\childdocjob'\par
\else
main document: `\childdocjob'\par
\fi
version: \version\par
\end{center}
\newpage
%    \end{macrocode}

% Manually include selected file,
% otherwise process as usual:
%    \begin{macrocode}
\ifchilddocmanual
\section*{part `\childdocname'}
\input{\childdocname}
\else
%    \end{macrocode}

% Include the two chapters:
%    \begin{macrocode}
\include{cdocsch1}
\include{cdocsch2}
%    \end{macrocode}

% Include the two parts unless only chapters should be displayed:
%    \begin{macrocode}
\ifchilddoc\else
\section{part three}
\input{cdocspt3}
\section{part four}
\input{cdocspt4}
\fi
%    \end{macrocode}

% Process as usual until here:
%    \begin{macrocode}
\fi
%    \end{macrocode}

% End of document body:
%    \begin{macrocode}
\end{document}
%    \end{macrocode}
%\iffalse
%</samplemain>
%\fi
%
% %%%%%%%%%%%%%%%%%%%%%%%%%%%%%%%%%%%%%%
% \paragraph{Chapter Include Files.}
%
% The include files are called |cdocsch1.tex| and |cdocsch2.tex|.
%
%\iffalse
%<*samplechap1|samplechap2>
%\fi

% Optional override for |\version| flag:
%    \begin{macrocode}
%%\providecommand{\version}{final}
%    \end{macrocode}

% Include the main document:
%    \begin{macrocode}
\input{childdoc.def}
\childdocof{cdocsamp}
%    \end{macrocode}

%\iffalse
%</samplechap1|samplechap2>
%\fi
%
%\iffalse
%<*samplechap1>
%\fi
% Some text for chapter 1:
%    \begin{macrocode}
\section{one}
some text in chapter one
%    \end{macrocode}

%\iffalse
%</samplechap1>
%\fi
% Some text for chapter 2:
%\iffalse
%<*samplechap2>
%\fi
%    \begin{macrocode}
\section{two}
more text in chapter two
%    \end{macrocode}

%\iffalse
%</samplechap2>
%\fi
%
% %%%%%%%%%%%%%%%%%%%%%%%%%%%%%%%%%%%%%%
% \paragraph{Part Include Files.}
%
% The include files are called |cdocspt3.tex| and |cdocspt4.tex|.
%
%\iffalse
%<*samplepart3|samplepart4>
%\fi

% Optional override for |\version| flag:
%    \begin{macrocode}
%%\providecommand{\version}{final}
%    \end{macrocode}

% Include the main document:
%    \begin{macrocode}
\input{childdoc.def}
\childdocby{cdocsamp}
%    \end{macrocode}

%\iffalse
%</samplepart3|samplepart4>
%\fi
%
%\iffalse
%<*samplepart3>
%\fi
% Some text for part 3:
%    \begin{macrocode}
some text in part three
%    \end{macrocode}

%\iffalse
%</samplepart3>
%\fi
% Some text for part 4:
%\iffalse
%<*samplepart4>
%\fi
%    \begin{macrocode}
more text in part four
%    \end{macrocode}

%\iffalse
%</samplepart4>
%\fi
%
% %%%%%%%%%%%%%%%%%%%%%%%%%%%%%%%%%%%%%%
% \paragraph{Forwarding for a Complete Draft.}
%
% The following forwarding file |cdocsdrf.tex|
% compiles the main document in draft mode:
%\iffalse
%<*sampledraft>
%\fi
%    \begin{macrocode}
\def\version{draft}
\input{childdoc.def}
\childdocforward{cdocsamp}
%    \end{macrocode}

%\iffalse
%</sampledraft>
%\fi
%
% %%%%%%%%%%%%%%%%%%%%%%%%%%%%%%%%%%%%%%
% \paragraph{Forwarding for Final Version of the Chapters.}
%
% The following forwarding files |cdocsfn1.tex| and |cdocsfn2.tex|
% (with identical content)
% compile the final versions of the child documents
% |cdocsch1.tex| and |cdocsch2.tex|, respectively:
%\iffalse
%<*samplefinal>
%\fi
%    \begin{macrocode}
\def\version{final}
\input{childdoc.def}
\childdocforwardprefix[cdocsamp]{cdocsfn}{cdocsch}
%    \end{macrocode}

%\iffalse
%</samplefinal>
%\fi
%
% %%%%%%%%%%%%%%%%%%%%%%%%%%%%%%%%%%%%%%
% \paragraph{Command Line Processing.}
%
% The following three command lines generate the output files
% |cdocscld|, |cdocscl1| and |cdocscl2|
% which should be identical to
% |cdocsdrf|, |cdocsch1| and |cdocsfn2|, respectively:
% \begin{center}
% \begin{tabular}{l}
% |latex -jobname cdocscld \|\\
% |  "\def\version{draft}\input{childdoc.def}\childdocforward{cdocsamp}"|\\
% |latex -jobname cdocscl1 \|\\
% |  "\input{childdoc.def}\childdocforward[cdocsamp]{cdocsch1}"|\\
% |latex -jobname cdocscl2 \|\\
% |  "\def\version{final}\input{childdoc.def}\childdocforward{cdocsch2}"|
% \end{tabular}
% \end{center}
% Note that the trailing backslash on each first line
% merely continues the input to the second line
% (for convenient cut ant paste).
% Furthermore, the command |latex| can be replaced by any
% of its alternative versions such as |pdflatex|.
%
% %%%%%%%%%%%%%%%%%%%%%%%%%%%%%%%%%%%%%%%%%%%%%%%%%%%%%%%%%%%%%%%%%%%%%%%%%%%%%%
% %%%%%%%%%%%%%%%%%%%%%%%%%%%%%%%%%%%%%%%%%%%%%%%%%%%%%%%%%%%%%%%%%%%%%%%%%%%%%%
% \section{Implementation}
%\iffalse
%<*package>
%\fi
%
% This section describes the definitions file |childdoc.def|.

% The definitions cannot be loaded using |\usepackage| or |\RequirePackage|
% which has a mechanism to prevent loading a style file more than once.
% When loading the definitions by means of |\input|
% multiple instances have to be prevented manually:
%\iffalse
%This code needs to be before the `\ProvidesFile' directive
%which is defined at the beginning of this file.
%Therefore it is also placed there and commented out here.
%</package>
%<*discard>
%\fi
%    \begin{macrocode}
\ifdefined\childdocmain\endinput\fi
%    \end{macrocode}
%\iffalse
%</discard>
%<*package>
%\fi
%
% \macro{\ifchilddoc}
% \macro{\ifchilddocmanual}
% The conditional |\ifchilddoc| tells whether a
% child (true) or main (false) document is being compiled.
% The conditional |\ifchilddocmanual| tells whether
% the |\includeonly| mechanism is used (false) or
% the selection of child files must be performed manually (true).
% The definitions initialise to false:
%    \begin{macrocode}
\newif\ifchilddoc
\newif\ifchilddocmanual
%    \end{macrocode}

% \macro{\childdocname}
% \macro{\childdocjob}
% The macro |\childdocname| stores the name of the main document
% to be compiled. The macro |\childdocjob| stores the name of
% the document on which the \LaTeX{} compiler was originally invoked.
% The content of |\jobname| cannot be compared
% to filenames specified in the source due to different catcodes.
% The following code rescans |\jobname|, stores the result
% in |\childdocname| and saves a copy in |\childdocjob|:
%    \begin{macrocode}
\edef\childdocname{\scantokens\expandafter{\jobname\noexpand}}
\let\childdocjob\childdocname
%    \end{macrocode}

% \macro{\childdocdisable}
% The macro |\childdocdisable| prevents the main file
% from being processed more than once.
% At this stage, the main document command |\childdocmain|
% is assumed to be called once again where it should do nothing.
% Any subsequent call to it should prevent
% a secondary processing of the main document
% It overwrites the forwarding commands
% |\childdocof| and |\childdocforward|
% with empty macros to prevent further inclusions of the main document:
%    \begin{macrocode}
\newcommand{\childdocdisable}
{
  \renewcommand{\childdocmain}[1]{\renewcommand{\childdocmain}[1]{\endinput}}
  \renewcommand{\childdocof}[1]{}
  \renewcommand{\childdocby}[2][]{}
  \renewcommand{\childdocforward}[2][]{}
  \renewcommand{\childdocdisable}{}
}
%    \end{macrocode}

% \macro{\childdocmain}
% The macro |\childdocmain| is to be called at the top of the main file
% with nothing or the main filename (without extension) as argument.
% First, it breaks loops.
% If the argument is not empty and does not match |\childdocname|
% (which is set by the first inclusion of |childdoc.def|),
% |\ifchilddoc| is set to true, |\includeonly| is applied to the child file
% and |\jobname| is set to the main file
% (for proper handling of |.aux| files):
%    \begin{macrocode}
\newcommand{\childdocmain}[1]
{
  \childdocdisable\childdocmain{}
  \if?#1?\else
    \begingroup
      \def\childdoctmp{#1}
      \ifx\childdoctmp\childdocname
        \def\childdoctmp{}
      \else
        \def\childdoctmp
        {
          \childdoctrue
          \includeonly{\childdocname}
          \def\childdocjob{#1}
          \def\jobname{#1}
        }
      \fi
      \expandafter
    \endgroup
    \childdoctmp
  \fi
}
%    \end{macrocode}

% \macro{\childdocof}
% The command |\childdocof| redirects
% compilation to the main file |#1|.
%    \begin{macrocode}
\newcommand{\childdocof}[1]
{
  \childdocdisable
  \childdoctrue
  \includeonly{\childdocname}
  \def\jobname{#1}
  \def\childdocjob{#1}
  \input{#1}
}
%    \end{macrocode}

% \macro{\childdocby}
% The command |\childdocby| ....
%    \begin{macrocode}
\newcommand{\childdocby}[2][]
{
  \childdocdisable
  \childdoctrue
  \childdocmanualtrue
  \if?#1?\else
    \def\jobname{#2}
  \fi
  \def\childdocjob{#2}
  \input{#2}
  \endinput
}
%    \end{macrocode}

% \macro{\childdocforward}
% The command |\childdocforward| redirects
% compilation to the main file or
% (if the optional argument is given) a child file.
% Parameters are set as if the main file
% or a child file starting with |\childdocof| was compiled.
% Then compilation is handed over to the main file:
%    \begin{macrocode}
\newcommand{\childdocforward}[2][]
{
  \begingroup
    \if?#1?
      \def\childdoctmp
      {
        \def\childdocname{#2}
        \def\childdocjob{#2}
        \def\jobname{#2}
        \input{#2}
        \endinput
      }
    \else
      \def\childdoctmp
      {
        \childdocdisable
        \def\childdocname{#2}
        \childdoctrue
        \includeonly{#2}
        \def\childdocjob{#1}
        \def\jobname{#1}
        \input{#1}
        \endinput
      }
    \fi
    \expandafter
  \endgroup
  \childdoctmp
}
%    \end{macrocode}

% \macro{\childdocforwardprefix}
% The command |\childdocforwardprefix| redirects
% compilation to the main or a child file by means of a pattern.
% The prefix |#1| in the current filename is replaced by |#2|
% and the suffix of the current filename is kept
% (it is assumed that the filename does not contain the substring `|~~~|'
% which is used as a delimiter).
% Compilation is handed over to the new file by |\childdocforward|:
%    \begin{macrocode}
\newcommand{\childdocforwardprefix}[3][]
{
  \begingroup
    \def\childdocextract #2##1~~~{\def\childdoctmp{\childdocforward[#1]{#3##1}}}
    \expandafter\childdocextract\childdocname~~~
    \expandafter
  \endgroup
  \childdoctmp
}
%    \end{macrocode}

% \macro{\childdoc}
% The deprecated macro |\childdoc| is a legacy version of |\childdocmain|:
%    \begin{macrocode}
\newcommand{\childdoc}{\childdocmain}
%    \end{macrocode}

% \macro{\childdocredirect}
% The deprecated macro |\childdocredirect| is a legacy version
% of |\childdocforward| and |\childdocforwardprefix|:
%    \begin{macrocode}
\newcommand{\childdocredirect}[2][]
{
  \begingroup
    \if?#1?
      \def\childdoctmp{\childdocforward{#2}}
    \else
      \def\childdoctmp{\childdocforwardprefix{#1}{#2}}
    \fi
    \expandafter
  \endgroup
  \childdoctmp
}
%    \end{macrocode}

%\iffalse
%</package>
%\fi
%
\endinput
|\\
|\childdocforward{|\textit{main}|}|
\end{tabular}
\end{center}
%
Likewise, the following files |final|\textit{nn}|.tex|
compile the final version of the child document
|child|\textit{nn}|.tex|:
%
\begin{center}
\begin{tabular}{l}
|\def\version{final}|\\
|% \iffalse
%
% childdoc.dtx Copyright (C) 2017-2018 Niklas Beisert
%
% This work may be distributed and/or modified under the
% conditions of the LaTeX Project Public License, either version 1.3
% of this license or (at your option) any later version.
% The latest version of this license is in
%   http://www.latex-project.org/lppl.txt
% and version 1.3 or later is part of all distributions of LaTeX
% version 2005/12/01 or later.
%
% This work has the LPPL maintenance status `maintained'.
%
% The Current Maintainer of this work is Niklas Beisert.
%
% This work consists of the files childdoc.dtx and childdoc.ins
% and the derived files childdoc.def and cdocsamp.tex with
% cdocsch1.tex, cdocsch2.tex, cdocsdrf.tex, cdocsfn1.tex, cdocsfn2.tex.
%
%<package>\ifdefined\childdocmain\endinput\fi
%<package>\ProvidesFile{childdoc.def}[2018/12/30 v2.0 child document driver]
%<samplemain>\ProvidesFile{cdocsamp.tex}[2018/12/30 v2.0 sample for childdoc]
%<*driver>
%\ProvidesFile{childdoc.drv}[2018/12/30 v2.0 childdoc reference manual file]
\PassOptionsToClass{10pt,a4paper}{article}
\documentclass{ltxdoc}

\usepackage[margin=35mm]{geometry}
\usepackage{hyperref}
\usepackage{hyperxmp}
\usepackage[usenames]{color}

\hypersetup{colorlinks=true}
\hypersetup{pdfstartview=FitH}
\hypersetup{pdfpagemode=UseNone}
\hypersetup{pdfsource={}}
\hypersetup{pdflang={en-UK}}
\hypersetup{pdfcopyright={Copyright 2017-2018 Niklas Beisert.
  This work may be distributed and/or modified under the
  conditions of the LaTeX Project Public License, either version 1.3
  of this license or (at your option) any later version.}}
\hypersetup{pdflicenseurl={http://www.latex-project.org/lppl.txt}}
\hypersetup{pdfcontactaddress={ETH Zurich, ITP, HIT K,
  Wolfgang-Pauli-Strasse 27}}
\hypersetup{pdfcontactpostcode={8093}}
\hypersetup{pdfcontactcity={Zurich}}
\hypersetup{pdfcontactcountry={Switzerland}}
\hypersetup{pdfcontactemail={nbeisert@itp.phys.ethz.ch}}
\hypersetup{pdfcontacturl={http://people.phys.ethz.ch/\xmptilde nbeisert/}}

\newcommand{\secref}[1]{\hyperref[#1]{section \ref*{#1}}}

\parskip1ex
\parindent0pt
\let\olditemize\itemize
\def\itemize{\olditemize\parskip0pt}

\begin{document}

\title{The \textsf{childdoc} Package}
\hypersetup{pdftitle={The childdoc Package}}
\author{Niklas Beisert\\[2ex]
  Institut f\"ur Theoretische Physik\\
  Eidgen\"ossische Technische Hochschule Z\"urich\\
  Wolfgang-Pauli-Strasse 27, 8093 Z\"urich, Switzerland\\[1ex]
  \href{mailto:nbeisert@itp.phys.ethz.ch}
  {\texttt{nbeisert@itp.phys.ethz.ch}}}
\hypersetup{pdfauthor={Niklas Beisert}}
\hypersetup{pdfsubject={Manual for the LaTeX2e Package childdoc}}
\date{30 December 2018, \textsf{v2.0}}
\maketitle

\begin{abstract}\noindent
\textsf{childdoc} is a \LaTeXe{} package
that enables the direct compilation
of document sections included by |\include|
to individual files.
\end{abstract}

\begingroup
\parskip0ex
\tableofcontents
\endgroup

%%%%%%%%%%%%%%%%%%%%%%%%%%%%%%%%%%%%%%%%%%%%%%%%%%%%%%%%%%%%%%%%%%%%%%%%%%%%%%%%
%%%%%%%%%%%%%%%%%%%%%%%%%%%%%%%%%%%%%%%%%%%%%%%%%%%%%%%%%%%%%%%%%%%%%%%%%%%%%%%%
\section{Introduction}

\LaTeX{} provides a mechanism to structure a large document (such as a book)
into a main file and several child files (containing the chapters)
using the |\include| command.
This mechanism is beneficial for documents
which span hundreds of pages in order to
make the source file(s) more manageable.
Moreover, compilation can be restricted to
selected child files by means of the |\includeonly| command.
The latter feature can be used to reduce the compilation time while editing
(this was significantly more useful in the earlier days of \LaTeX{})
or to generate a smaller document which is easier to navigate.
Another application of |\includeonly| is to generate
documents consisting of selected parts of the complete document.

However, there are a few drawbacks of the plain |\include| mechanism:
\begin{itemize}
\item
The child files cannot be compiled on their own,
they can only be compiled via the main file.
A naive editing environment
(such as a text editor with an option
to have the current file processed by \LaTeX)
may require one to switch to the main file before compiling;
attempting to compile the child file produces errors.
\item
The main file must be modified (each time)
to adjust the |\includeonly| command
to the present needs. This easily leaves the main file in a messy state.
\item
The generated document will always carry the filename
of the main document. This is inconvenient if
several child files are to be compiled and
to be kept for distribution.
\end{itemize}

The present package provides a simple interface
to make child files individually compilable by \LaTeX{}.
Compiling a child file then has the same effect as compiling
the main file with an |\includeonly| command
to select the appropriate child.
Moreover the generated document will carry the name of the child
rather than the main file.
This resolves all three above issues.

This feature is meant to make the editing of books,
thesis documents and lecture notes somewhat more convenient.
However, the package can also be used efficiently for
composing a series of documents (such as exercise sheets)
which are typically distributed individually.
It then assists the author in generating the individual documents
(potentially in different versions)
as well as a document containing the collected series.
Another application is in developing style files
or other kinds of included material
where compilation of the style file could redirect
to a sample or test file.

%%%%%%%%%%%%%%%%%%%%%%%%%%%%%%%%%%%%%%%%%%%%%%%%%%%%%%%%%%%%%%%%%%%%%%%%%%%%%%%%
%%%%%%%%%%%%%%%%%%%%%%%%%%%%%%%%%%%%%%%%%%%%%%%%%%%%%%%%%%%%%%%%%%%%%%%%%%%%%%%%
\section{Usage}

First of all, the package \textsf{childdoc} is \emph{not} a standard
\LaTeXe{} |.sty| style file! Therefore it needs to be invoked in
a non-standard way.

%%%%%%%%%%%%%%%%%%%%%%%%%%%%%%%%%%%%%%%%%%%%%%%%%%%%%%%%%%%%%%%%%%%%%%%%%%%%%%%%
\subsection{Included Files}
\label{sec:include}

%%%%%%%%%%%%%%%%%%%%%%%%%%%%%%%%%%%%%%%%
\DescribeMacro{\childdocmain}
To use the package, add the commands
\begin{center}
\begin{tabular}{l}
|\input{childdoc.def}|\\
|\childdocmain{}|\\
\end{tabular}
\end{center}
at the very top of the main \LaTeX{} file,
in particular \emph{before} the |\documentclass| statement!
The argument of |\childdocmain| should be left empty
(but it must be present).

%%%%%%%%%%%%%%%%%%%%%%%%%%%%%%%%%%%%%%%%
\DescribeMacro{\childdocof}
Furthermore, add the commands
\begin{center}
\begin{tabular}{l}
|\input{childdoc.def}|\\
|\childdocof{|\textit{main}|}|\\
\end{tabular}
\end{center}
at the top of every child file \textit{child}
which is included by |\include{|\textit{child}|}|
from within the main file
(or at least for those files to be compiled individually).
The argument \textit{main} must be the filename of the main file.

There are a couple of
considerations in setting up the main and child documents:

%%%%%%%%%%%%%%%%%%%%%%%%%%%%%%%%%%%%%%%%
\paragraph{Restrictions.}

Please note the following restrictions:
\begin{itemize}
\item
|\childdocmain| must be called with one argument \textit{main}
to ensure compatibility with earlier version of the package.
It must either be empty (|\childdocmain{}|)
or precisely match the filename of the main file in which it is specified.
See \secref{sec:detection} for further information.
\item
The filename \textit{main} must be specified without the |.tex| extension.
\item
The filename \textit{main} is case sensitive
(even in case-insensitive file systems)
due to internal string comparison.
\item
The argument \textit{main} should be fully expanded, it cannot be a macro.
\item
Subdirectories and special characters should be avoided in filenames.
\item
The command |\childdocmain{|\textit{main}|}| must be followed by a whitespace.
It should not be followed immediately by another command
or by a comment mark `|%|'.
This is because the \TeX{} parser reads the token immediately following
the argument of |\childdocmain| and puts it
at the beginning of every child section;
however, a white\-space is ignored.
\end{itemize}

%%%%%%%%%%%%%%%%%%%%%%%%%%%%%%%%%%%%%%%%
\paragraph{Content of Main File.}

It is advisable to place all content in the child files included by |\include|.
Any output contained in the main file will appear in all child documents
unless suppressed manually;
it cannot be suppressed automatically by the |\includeonly| directive
and thus should normally be avoided.
A method to include some content in the main file
by means of conditional processing is described in \secref{sec:conditional}.

%%%%%%%%%%%%%%%%%%%%%%%%%%%%%%%%%%%%%%%%
\paragraph{Page Numbering.}

When only a part of the document is compiled,
the appropriate numbering of pages
(as well as other status parameters)
is determined from the |.aux| files.
The latter contain information from previous passes.
However this information needs to propagate through
all intermediate child documents.
Therefore the page numbering in child documents may well
be inconsistent until the complete document is compiled at least once.

A useful (if unconventional) way to always ensure a consistent
page numbering is to restart the numbering in each child document
and denote the pages by `\textit{child}|.|\textit{page}'
where \textit{child} represents the chapter/section number of the child file.
This can be achieved by the command
|\numberwithin{page}{|\textit{child}|}|
of the \textsf{amsmath} package
where \textit{child} can be |chapter| or |section|
depending on the chosen structuring.
Alternatively, one can modify the macro |\thepage| appropriately
and reset the counter |page| at the start of each child file.

%%%%%%%%%%%%%%%%%%%%%%%%%%%%%%%%%%%%%%%%%%%%%%%%%%%%%%%%%%%%%%%%%%%%%%%%%%%%%%%%
\subsection{Conditional Processing}
\label{sec:conditional}

The package provides a mechanism to compile different versions
of a document. To customise the versions further some conditional processing
can come in handy to distinguish which version is being compiled.
The package provides two macros to describe the compilation context:

%%%%%%%%%%%%%%%%%%%%%%%%%%%%%%%%%%%%%%%%
\DescribeMacro{\ifchilddoc}
The conditional |\ifchilddoc| distinguishes between the compilation of
child documents and the main document:
%
\begin{center}
|\ifchilddoc |\textit{child-code}| |[|\||else |\textit{main-code}]| \||fi|
\end{center}

%%%%%%%%%%%%%%%%%%%%%%%%%%%%%%%%%%%%%%%%
\DescribeMacro{\childdocname}
\DescribeMacro{\childdocjob}
The macro |\childdocname| contains the filename (without extension)
of the main or child file being processed.
Note that |\childdocjob| will always contain the name of the main file.

%%%%%%%%%%%%%%%%%%%%%%%%%%%%%%%%%%%%%%%%
\paragraph{Title Page.}

Conditional processing can be used to include a title or banner page
in the main document when proper precautions are taken.
Importantly, the code in the main file should ensure that the page counter
(as well as other status parameters which are stored in the |.aux| files)
takes the same value after the conditional processing.
Otherwise the page numbers may take divergent values
depending on which part is compiled.

For example, a title page could be declared by:
%
\begin{center}
\begin{tabular}{l}
|\ifchilddoc\||else|\\
|\addtocounter{page}{-1}|\\
\textit{code for title page}\\
|\newpage|\\
|\||fi|
\end{tabular}
\end{center}
%
A banner page for the child documents can be generated by:
%
\begin{center}
\begin{tabular}{l}
|\ifchilddoc|\\
|\addtocounter{page}{-1}|\\
\textit{code for banner page}\\
|\newpage|\\
|\||fi|
\end{tabular}
\end{center}
%
Here one could write a message such as:
\begin{center}
|This is the part \childdocname{} of \childdocjob{}.|
\end{center}

%%%%%%%%%%%%%%%%%%%%%%%%%%%%%%%%%%%%%%%%%%%%%%%%%%%%%%%%%%%%%%%%%%%%%%%%%%%%%%%%
\subsection{Flags}
\label{sec:flags}

The package makes it easy to generate different versions
of the main or child documents.
To this end compilation flags can be defined
and assigned different default values.
They will be particularly useful in conjunction
with the forwarding mechanism described in \secref{sec:forward}.

For example, it may be useful to have a flag |\version|
which can be set to |draft| or |final|.
The document source will contain some conditional code
depending on the value of |\version|.
Suppose further, the flag should default to |final| for the main file
and to |draft| for child files
which is a natural assignment for editing the document.
This is achieved by placing the following code
in the preamble of the main document
(below the |\childdocmain| directive):
%
\begin{center}
\begin{tabular}{l}
|\ifchilddoc|\\
|\providecommand{\version}{draft}|\\
|\||else|\\
|\providecommand{\version}{final}|\\
|\||fi|
\end{tabular}
\end{center}
%
The definition by |\providecommand| makes sure
that previous definitions are not overwritten.
Further statements |\providecommand{\version}{...}|
can thus be added before the above code to override it.

For the main file, one might add a line
(between |\childdocmain| and the above block)
%
\begin{center}
|%\ifchilddoc\||else\providecommand{\version}{draft}\||fi|
\end{center}
%
which can be uncommented to produce a draft version.
Likewise one can add a line to the very top of a child file
(above the |\childdocof{|\textit{main}|}| directive)
%
\begin{center}
|%\providecommand{\version}{final}|
\end{center}
%
which can be uncommented to produce the final version of this child document.

%%%%%%%%%%%%%%%%%%%%%%%%%%%%%%%%%%%%%%%%%%%%%%%%%%%%%%%%%%%%%%%%%%%%%%%%%%%%%%%%
\subsection{Forwarding}
\label{sec:forward}

Different versions of the main or child documents
using compilation flags as described in \secref{sec:flags}
can be (permanently) stored in different files
for convenient compilation, viewing and distribution.
To this end, the package defines a command
to pass on compilation to a different file:

%%%%%%%%%%%%%%%%%%%%%%%%%%%%%%%%%%%%%%%%
\DescribeMacro{\childdocforward}
The command |\childdocforward| redirects processing to
another source file:
%
\begin{center}
\begin{tabular}{l}
|\input{childdoc.def}|\\
|\childdocforward[|\textit{main}|]{|\textit{dest}|}|\\
\end{tabular}
\end{center}
%
The argument \textit{dest} is the destination file
(without extension).
It should be the main file or one of the child files.
Note that further \textsf{childdoc} directives
such as |\childdocof| and |\childdocforward|
in the indicated file will be processed in this form.
The optional argument \textit{main}
passes on directly to the main file \textit{main}
while pretending to compile the child \textit{dest}.
This form behaves as if \textit{dest}
issues |\childdocof{|\textit{main}|}| right away,
and no further \textsf{childdoc} directives will be processed.

%%%%%%%%%%%%%%%%%%%%%%%%%%%%%%%%%%%%%%%%
\DescribeMacro{\...prefix}
In the alternative form |\childdocforwardprefix|,
%
\begin{center}
\begin{tabular}{l}
|\input{childdoc.def}|\\
|\childdocforwardprefix[|\textit{main}|]{|\textit{prefix}|}{|\textit{dest}|}|
\end{tabular}
\end{center}
%
the destination file is determined by a pattern
depending on the current file:
To make this work, the current file must be called
`{\textit{prefix}\hspace{0.2em}\textit{suffix}}'
with \textit{prefix} matching precisely the argument.
Processing is then passed on to the file
`{\textit{dest}\hspace{0.2em}\textit{suffix}}'.
Surely, the same effect is achieved by
directly specifying the
argument `{\textit{dest}\hspace{0.2em}\textit{suffix}}'
in the first form.
However, that requires to set up a different file
for each child. With the alternative form of the command
all these files can have exactly the same content
which simplifies setting them up and maintaining them.

For example, the following file |draft.tex|
with a compilation flag |\version| as described in \secref{sec:flags}
compiles the main document as a draft:
%
\begin{center}
\begin{tabular}{l}
|\def\version{draft}|\\
|\input{childdoc.def}|\\
|\childdocforward{|\textit{main}|}|
\end{tabular}
\end{center}
%
Likewise, the following files |final|\textit{nn}|.tex|
compile the final version of the child document
|child|\textit{nn}|.tex|:
%
\begin{center}
\begin{tabular}{l}
|\def\version{final}|\\
|\input{childdoc.def}|\\
|\childdocforwardprefix{final}{child}|
\end{tabular}
\end{center}
%

Note that when several versions of a main file and/or of each child file
are to be generated, it may be convenient to set up a |Makefile| or
shell script to automatise the process.

%%%%%%%%%%%%%%%%%%%%%%%%%%%%%%%%%%%%%%%%%%%%%%%%%%%%%%%%%%%%%%%%%%%%%%%%%%%%%%%%
\subsection{Command Line Processing}
\label{sec:commandline}

The effect of redirection files can also be achieved by invoking
the \LaTeX{} compiler with a more elaborate command line.
Most conveniently this should be done as part
of a shell script or a |Makefile|.

When using \textsf{childdoc} in the main file, the following
command lines effectively perform a redirection
(note that depending on the shell being used,
backslashes may have to be doubled: `|\|' $\to$ `|\\|'):
%
\begin{center}
|... -jobname "|\textit{target}|" |\\|"|[\textit{flags}]%
|\input{childdoc.def}\childdocforward[|\textit{main}|]{|\textit{dest}|}"|
\end{center}
%
Here \textit{target} is the name of the output file,
\textit{main} is the name of the main file
and \textit{dest} is the name of the main or child file to be processed
(all filenames without extensions).
The optional argument \textit{main} can be omitted
if \textit{main} matches \textit{dest}.
Optionally, compilation \textit{flags} can be defined via |\def| commands.
This command line makes the \TeX{} engine believe
it is compiling the file \textit{target}
whose content is specified as the latter parameter.
The provided code then forwards the processing to
\textit{main} or \textit{dest} as described in \secref{sec:forward}.

%%%%%%%%%%%%%%%%%%%%%%%%%%%%%%%%%%%%%%%%%%%%%%%%%%%%%%%%%%%%%%%%%%%%%%%%%%%%%%%%
\subsection{Include by Input}
\label{sec:input}

Including child documents by |\include| has some restrictions by design.
Most notably, the content of a child document always occupies
its own set of pages; pages cannot be shared between child documents.
Usually, this behaviour makes perfect sense
because each child document contain an essential part of the document.
However, in some situations it may be desirable to compose
a document from a collection of parts
without having mandatory page breaks between then.
For this case, the package
provides a mechanism to include parts
by |\input| which can also be processed individually.
However, by construction this mechanism
requires manual handling of the content to be output.

%%%%%%%%%%%%%%%%%%%%%%%%%%%%%%%%%%%%%%%%
\DescribeMacro{\ifchilddocmanual}
The main file should be prepared as usual, see \secref{sec:include}.
However, the document body must make a distinction
between processing of an individual part and of the main document, e.g.:
%
\begin{center}
\begin{tabular}{l}
|\ifchilddocmanual|\\
|\input{\childdocname}|\\
|\||else|\\
\textit{document body with }|\input{|\textit{part}|}|\\
|\||fi|
\end{tabular}
\end{center}
%
The conditional |\ifchilddocmanual| is true whenever
a part to be included by |\input| is being compiled,
and the name of the part is stored in |\childdocname|.

%%%%%%%%%%%%%%%%%%%%%%%%%%%%%%%%%%%%%%%%
\DescribeMacro{\childdocby}
Each part to be included by |\input| should start with:
%
\begin{center}
\begin{tabular}{l}
|\input{childdoc.def}|\\
|\childdocby{|\textit{main}|}|\\
\end{tabular}
\end{center}
%
The directive |\childdocby| is similar to |\childdocof|
described in \secref{sec:include},
but the subsequent selection of content must be done manually.
To that end, both |\ifchilddoc| and |\ifchilddocmanual|
will be true upon processing of a part,
and the name of the part is stored in |\childdocname|.
Note that |\jobname| will be set to the filename of the current part
so that each part receives an individual |.aux| file
that does not interfere with the |.aux| file(s) of the main document.
This behaviour can be altered by the alternative form
|\childdocby[*]{|\textit{main}|}| (with a non-empty optional argument)
which uses the |.aux| file of the main document
by setting |\jobname| to \textit{main}.

%%%%%%%%%%%%%%%%%%%%%%%%%%%%%%%%%%%%%%%%%%%%%%%%%%%%%%%%%%%%%%%%%%%%%%%%%%%%%%%%
\subsection{Driver Development}
\label{sec:driver}

The \textsf{childdoc} mechanism can also be use for the development
of definition files such as \LaTeX{} styles or classes.
This case differs from the above setup with multiple parts
included by |\include| in that no |\includeonly| should be invoked.
This can be achieved by starting the include file
(before |\ProvidesPackage|) with:
%
\begin{center}
\begin{tabular}{l}
|\input{childdoc.def}|\\
|\childdocforward{|\textit{main}|}|\\
\end{tabular}
\end{center}
%
or alternatively with:
%
\begin{center}
\begin{tabular}{l}
|\input{childdoc.def}|\\
|\childdocby{|\textit{main}|}|\\
\end{tabular}
\end{center}
%
Both forms have slightly different effects as described above.
The main file is prepared as usual, see \secref{sec:include}.

%%%%%%%%%%%%%%%%%%%%%%%%%%%%%%%%%%%%%%%%%%%%%%%%%%%%%%%%%%%%%%%%%%%%%%%%%%%%%%%%
\subsection{Legacy Detection}
\label{sec:detection}

The directive |\childdocmain| in the main file can detect
whether the complete document or merely a child is to be compiled
even without using the directive |\childdocof|.
This method is deprecated because it is less robust
and there is no compelling reason to use it;
it is merely provided for backward compatibility
and it may be removed in future versions.

If the detection mechanism is to be used,
it is mandatory to correctly specify
the filename of the main file as the argument of |\childdocmain|:
%
\begin{center}
\begin{tabular}{l}
|\input{childdoc.def}|\\
|\childdocmain{|\textit{main}|}|\\
\end{tabular}
\end{center}
%
If |\jobname| does not match the argument \textit{main} of |\childdocmain|,
it is assumed that |\jobname| points to the child file to be compiled.
When using |\childdocmain| with the main file specified as argument,
it suffices to start a child file
with just |\input{|\textit{main}|}|
without loading of the package and using |\childdocof|.
If instead all processing is done
with the appropriate \textsf{childdoc} directives,
the argument of \textit{main} of |\childdocmain| can be empty.

An alternative version of the command line processing described
in \secref{sec:commandline} using the detection mechanism reads:
%
\begin{center}
|... -jobname "|\textit{target}|" "|[\textit{flags}]%
[|\def\jobname{|\textit{dest}|}|]|\input{|\textit{main}|}"|
\end{center}

%%%%%%%%%%%%%%%%%%%%%%%%%%%%%%%%%%%%%%%%%%%%%%%%%%%%%%%%%%%%%%%%%%%%%%%%%%%%%%%%
\subsection{Manual Code}
\label{sec:manual}

In case one cannot be certain whether the definitions file |childdoc.def|
is installed on the target \TeX{} distribution
and one prefers not to ship it,
it is conceivable to paste a few relevant commands into the sources.

To that end, drop all statements |\input{childdoc.def}|
and perform the replacements as outlined below.
Instead of |\childdocmain{|\textit{main}|}| add the following code
to the top of the main file:
%
\begin{center}
\begin{tabular}{l}
|\||ifdefined\childdocname\endinput\||fi\newif\ifchilddoc|\\
|\edef\childdocname{\scantokens\expandafter{\jobname\noexpand}}|\\
|\def\childdocmain{|\textit{main}|}\||ifx\childdocmain\childdocname\||else|\\
|\childdoctrue\includeonly{\childdocname}\let\jobname\childdocmain\||fi|\\
\end{tabular}
\end{center}
%
Instead of |\childdocof{|\textit{main}|}| just include the main file
at the top of each child file:
%
\begin{center}
|\input{|\textit{main}|}|
\end{center}
%
A simple redirection |\childdocforward{|\textit{dest}|}| is achieved by:
%
\begin{center}
|\def\jobname{|\textit{dest}|}\input{\jobname}|
\end{center}
%
The redirection with prefix
|\childdocforwardprefix[|\textit{prefix}|]{|\textit{dest}|}|
is accomplished by:
%
\begin{center}
\begin{tabular}{l}
|{\edef\jobname{\scantokens\expandafter{\jobname\noexpand}}|\\
|\def\redirectjob |\textit{prefix}|#1~~~{\gdef\jobname{|\textit{dest}|#1}}|\\
|\expandafter\redirectjob\jobname~~~}\input{\jobname}|
\end{tabular}
\end{center}

In an alternative approach,
child documents can be compiled by a specific command line
without additional code or specific definitions:
%
\begin{center}
|... -jobname "|\textit{target}|" "|[\textit{flags}]%
|\includeonly{|\textit{dest}|}\input{|\textit{main}|}"|
\end{center}
%

%%%%%%%%%%%%%%%%%%%%%%%%%%%%%%%%%%%%%%%%%%%%%%%%%%%%%%%%%%%%%%%%%%%%%%%%%%%%%%%%
%%%%%%%%%%%%%%%%%%%%%%%%%%%%%%%%%%%%%%%%%%%%%%%%%%%%%%%%%%%%%%%%%%%%%%%%%%%%%%%%
\section{Information}

%%%%%%%%%%%%%%%%%%%%%%%%%%%%%%%%%%%%%%%%%%%%%%%%%%%%%%%%%%%%%%%%%%%%%%%%%%%%%%%%
\subsection{Copyright}

Copyright \copyright{} 2017--2018 Niklas Beisert

This work may be distributed and/or modified under the
conditions of the \LaTeX{} Project Public License, either version 1.3
of this license or (at your option) any later version.
The latest version of this license is in
  \url{http://www.latex-project.org/lppl.txt}
and version 1.3 or later is part of all distributions of \LaTeX{}
version 2005/12/01 or later.

This work has the LPPL maintenance status `maintained'.

The Current Maintainer of this work is Niklas Beisert.

This work consists of the files |README.txt|, |childdoc.ins| and |childdoc.dtx|
as well as the derived files |childdoc.def|, |cdocsamp.tex|
with |cdocsch1.tex|, |cdocsch2.tex|, |cdocspt3.tex|, |cdocspt4.tex|,
|cdocsdrf.tex|, |cdocsfn1.tex|, |cdocsfn2.tex|
as well as |childdoc.pdf|.

%%%%%%%%%%%%%%%%%%%%%%%%%%%%%%%%%%%%%%%%%%%%%%%%%%%%%%%%%%%%%%%%%%%%%%%%%%%%%%%%
\subsection{Files and Installation}

The package consists of the files:
%
\begin{center}
\begin{tabular}{ll}
    |README.txt|   & readme file \\
    |childdoc.ins| & installation file \\
    |childdoc.dtx| & source file \\
    |childdoc.def| & definition file \\
    |cdocsamp.tex| & sample main file \\
    |cdocsch1.tex| & sample include file \\
    |cdocsch2.tex| & sample include file \\
    |cdocspt3.tex| & sample part file \\
    |cdocspt4.tex| & sample part file \\
    |cdocsdrf.tex| & sample redirection file \\
    |cdocsfn1.tex| & sample redirection file \\
    |cdocsfn2.tex| & sample redirection file \\
    |childdoc.pdf| & manual
\end{tabular}
\end{center}
%
The distribution consists of the files
|README.txt|, |childdoc.ins| and |childdoc.dtx|.
%
\begin{itemize}
\item
Run (pdf)\LaTeX{} on |childdoc.dtx|
to compile the manual |childdoc.pdf| (this file).
\item
Run \LaTeX{} on |childdoc.ins| to create the definitions file |childdoc.def|
and the sample |cdocsamp.tex| with include files
|cdocsch1.tex|, |cdocsch2.tex|, |cdocspt3.tex|, |cdocspt4.tex|,
|cdocsdrf.tex|, |cdocsfn1.tex|, |cdocsfn2.tex|.
Then copy the file |childdoc.def| to an appropriate directory of your \LaTeX{}
distribution, e.g.\ \textit{texmf-root}|/tex/latex/childdoc|.
\end{itemize}

%%%%%%%%%%%%%%%%%%%%%%%%%%%%%%%%%%%%%%%%%%%%%%%%%%%%%%%%%%%%%%%%%%%%%%%%%%%%%%%%
\subsection{Related CTAN Packages}

There are several other packages which offer a similar functionality:
%
\begin{itemize}
\item
The packages
\href{http://ctan.org/pkg/docmute}{\textsf{docmute}},
\href{http://ctan.org/pkg/includex}{\textsf{includex}} and
\href{http://ctan.org/pkg/standalone}{\textsf{standalone}}
provide commands to include only the document body of
a child file thus allowing both files to be compiled individually.
\item
The packages \href{http://ctan.org/pkg/subdocs}{\textsf{subdocs}}
and \href{http://ctan.org/pkg/subfiles}{\textsf{subfiles}}
provide structures in which the main and child documents can be
encapsulated and allowing them to be compiled individually.
The inclusion mechanism is different from the conventional |\include|.
\item
The package \href{http://ctan.org/pkg/combine}{\textsf{combine}}
is an elaborate solution to combine several documents into one.
\end{itemize}
%
See also the CTAN topic \href{http://ctan.org/topic/subdocs}{\textsf{subdocs}}
for further related packages.
The present package differs from the above solutions in that
a document structure constructed with the conventional |\include| mechanism
just needs two extra commands at the top of every file
such that all constituent files can be compiled individually.

%%%%%%%%%%%%%%%%%%%%%%%%%%%%%%%%%%%%%%%%%%%%%%%%%%%%%%%%%%%%%%%%%%%%%%%%%%%%%%%%
%\subsection{Feature Suggestions}
%
%The following is a list of features which may be useful for future
%versions of this package:
%%
%\begin{itemize}
%\item
%\ldots
%\end{itemize}

%%%%%%%%%%%%%%%%%%%%%%%%%%%%%%%%%%%%%%%%%%%%%%%%%%%%%%%%%%%%%%%%%%%%%%%%%%%%%%%%
\subsection{Revision History}

%%%%%%%%%%%%%%%%%%%%%%%%%%%%%%%%%%%%%%%%
\paragraph{v2.0:} 2018/12/30

\begin{itemize}
\item
immediate forward processing
\item
added |\childdocby| mechanism
\item
manual restructured
\end{itemize}

%%%%%%%%%%%%%%%%%%%%%%%%%%%%%%%%%%%%%%%%
\paragraph{v1.6:} 2018/01/17

\begin{itemize}
\item
application for development of include files
\item
corrections to manual
\end{itemize}

%%%%%%%%%%%%%%%%%%%%%%%%%%%%%%%%%%%%%%%%
\paragraph{v1.5:} 2017/05/21

\begin{itemize}
\item
more complete structuring introduced
\item
|\childdocof| introduced
\item
|\childdoc| renamed to |\childdocmain|
\item
|\childredirect| renamed to |\childdocforward| and |\childdocforwardprefix|
and functionality expanded
\end{itemize}

%%%%%%%%%%%%%%%%%%%%%%%%%%%%%%%%%%%%%%%%
\paragraph{v1.0:} 2017/04/27

\begin{itemize}
\item
manual and install package
\item
first version published on CTAN
\end{itemize}

%%%%%%%%%%%%%%%%%%%%%%%%%%%%%%%%%%%%%%%%
\paragraph{v0.6:} 2017/04/26

\begin{itemize}
\item
redirection mechanism added
\end{itemize}

%%%%%%%%%%%%%%%%%%%%%%%%%%%%%%%%%%%%%%%%
\paragraph{v0.5:} 2017/04/26

\begin{itemize}
\item
functionality in definition file
\end{itemize}


%%%%%%%%%%%%%%%%%%%%%%%%%%%%%%%%%%%%%%%%%%%%%%%%%%%%%%%%%%%%%%%%%%%%%%%%%%%%%%%%
%%%%%%%%%%%%%%%%%%%%%%%%%%%%%%%%%%%%%%%%%%%%%%%%%%%%%%%%%%%%%%%%%%%%%%%%%%%%%%%%
%%%%%%%%%%%%%%%%%%%%%%%%%%%%%%%%%%%%%%%%%%%%%%%%%%%%%%%%%%%%%%%%%%%%%%%%%%%%%%%%
\appendix

\settowidth\MacroIndent{\rmfamily\scriptsize 000\ }

 \DocInput{childdoc.dtx}

\end{document}
%</driver>
% \fi
%
% %%%%%%%%%%%%%%%%%%%%%%%%%%%%%%%%%%%%%%%%%%%%%%%%%%%%%%%%%%%%%%%%%%%%%%%%%%%%%%
% %%%%%%%%%%%%%%%%%%%%%%%%%%%%%%%%%%%%%%%%%%%%%%%%%%%%%%%%%%%%%%%%%%%%%%%%%%%%%%
% \section{Sample}
%\iffalse
%<*samplemain>
%\fi
%
% The following presents a sample document
% with two chapters, two parts, a title page,
% a compile flag as well as three forwarding files to set the flag.
% It consists of eight |.tex| files:
% \begin{center}
% \begin{tabular}{ll}
% |cdocsamp.tex|&main file\\
% |cdocsch1.tex|&include file for chapter 1\\
% |cdocsch2.tex|&include file for chapter 2\\
% |cdocspt3.tex|&include file for part 3\\
% |cdocspt4.tex|&include file for part 4\\
% |cdocsdrf.tex|&forwarding file for main file in draft mode\\
% |cdocsfi1.tex|&forwarding file for final version of chapter 1\\
% |cdocsfi2.tex|&forwarding file for final version of chapter 2\\
% \end{tabular}
% \end{center}
% Each of the eight files can be compiled directly by the \LaTeX{} compiler.
%
% %%%%%%%%%%%%%%%%%%%%%%%%%%%%%%%%%%%%%%
% \paragraph{Main File.}
%
% The main file is called |cdocsamp.tex|.
%
% Load the \textsf{childdoc} definitions and
% declare the filename for the main document:
%    \begin{macrocode}
\input{childdoc.def}
\childdocmain{}
%    \end{macrocode}

% Optional override for |\version| flag:
%    \begin{macrocode}
%%\ifchilddoc\else\providecommand{\version}{draft}\fi
%    \end{macrocode}

% Define the default values for the |\version| flag
% (|final| for the main file and |draft| for childs):
%    \begin{macrocode}
\ifchilddoc
\providecommand{\version}{draft}
\else
\providecommand{\version}{final}
\fi
%    \end{macrocode}

% Load the standard document class:
%    \begin{macrocode}
\documentclass[12pt]{article}
%    \end{macrocode}

% Start the document body:
%    \begin{macrocode}
\begin{document}
%    \end{macrocode}

% Declare a title page.
% Print title, part of document being processed and version flag:
%    \begin{macrocode}
\addtocounter{page}{-1}
\begin{center}
{\LARGE\bfseries{}childdoc example\par}
\vspace{1cm}
\ifchilddoc
\ifchilddocmanual part\else chapter\fi:
`\childdocname' of `\childdocjob'\par
\else
main document: `\childdocjob'\par
\fi
version: \version\par
\end{center}
\newpage
%    \end{macrocode}

% Manually include selected file,
% otherwise process as usual:
%    \begin{macrocode}
\ifchilddocmanual
\section*{part `\childdocname'}
\input{\childdocname}
\else
%    \end{macrocode}

% Include the two chapters:
%    \begin{macrocode}
\include{cdocsch1}
\include{cdocsch2}
%    \end{macrocode}

% Include the two parts unless only chapters should be displayed:
%    \begin{macrocode}
\ifchilddoc\else
\section{part three}
\input{cdocspt3}
\section{part four}
\input{cdocspt4}
\fi
%    \end{macrocode}

% Process as usual until here:
%    \begin{macrocode}
\fi
%    \end{macrocode}

% End of document body:
%    \begin{macrocode}
\end{document}
%    \end{macrocode}
%\iffalse
%</samplemain>
%\fi
%
% %%%%%%%%%%%%%%%%%%%%%%%%%%%%%%%%%%%%%%
% \paragraph{Chapter Include Files.}
%
% The include files are called |cdocsch1.tex| and |cdocsch2.tex|.
%
%\iffalse
%<*samplechap1|samplechap2>
%\fi

% Optional override for |\version| flag:
%    \begin{macrocode}
%%\providecommand{\version}{final}
%    \end{macrocode}

% Include the main document:
%    \begin{macrocode}
\input{childdoc.def}
\childdocof{cdocsamp}
%    \end{macrocode}

%\iffalse
%</samplechap1|samplechap2>
%\fi
%
%\iffalse
%<*samplechap1>
%\fi
% Some text for chapter 1:
%    \begin{macrocode}
\section{one}
some text in chapter one
%    \end{macrocode}

%\iffalse
%</samplechap1>
%\fi
% Some text for chapter 2:
%\iffalse
%<*samplechap2>
%\fi
%    \begin{macrocode}
\section{two}
more text in chapter two
%    \end{macrocode}

%\iffalse
%</samplechap2>
%\fi
%
% %%%%%%%%%%%%%%%%%%%%%%%%%%%%%%%%%%%%%%
% \paragraph{Part Include Files.}
%
% The include files are called |cdocspt3.tex| and |cdocspt4.tex|.
%
%\iffalse
%<*samplepart3|samplepart4>
%\fi

% Optional override for |\version| flag:
%    \begin{macrocode}
%%\providecommand{\version}{final}
%    \end{macrocode}

% Include the main document:
%    \begin{macrocode}
\input{childdoc.def}
\childdocby{cdocsamp}
%    \end{macrocode}

%\iffalse
%</samplepart3|samplepart4>
%\fi
%
%\iffalse
%<*samplepart3>
%\fi
% Some text for part 3:
%    \begin{macrocode}
some text in part three
%    \end{macrocode}

%\iffalse
%</samplepart3>
%\fi
% Some text for part 4:
%\iffalse
%<*samplepart4>
%\fi
%    \begin{macrocode}
more text in part four
%    \end{macrocode}

%\iffalse
%</samplepart4>
%\fi
%
% %%%%%%%%%%%%%%%%%%%%%%%%%%%%%%%%%%%%%%
% \paragraph{Forwarding for a Complete Draft.}
%
% The following forwarding file |cdocsdrf.tex|
% compiles the main document in draft mode:
%\iffalse
%<*sampledraft>
%\fi
%    \begin{macrocode}
\def\version{draft}
\input{childdoc.def}
\childdocforward{cdocsamp}
%    \end{macrocode}

%\iffalse
%</sampledraft>
%\fi
%
% %%%%%%%%%%%%%%%%%%%%%%%%%%%%%%%%%%%%%%
% \paragraph{Forwarding for Final Version of the Chapters.}
%
% The following forwarding files |cdocsfn1.tex| and |cdocsfn2.tex|
% (with identical content)
% compile the final versions of the child documents
% |cdocsch1.tex| and |cdocsch2.tex|, respectively:
%\iffalse
%<*samplefinal>
%\fi
%    \begin{macrocode}
\def\version{final}
\input{childdoc.def}
\childdocforwardprefix[cdocsamp]{cdocsfn}{cdocsch}
%    \end{macrocode}

%\iffalse
%</samplefinal>
%\fi
%
% %%%%%%%%%%%%%%%%%%%%%%%%%%%%%%%%%%%%%%
% \paragraph{Command Line Processing.}
%
% The following three command lines generate the output files
% |cdocscld|, |cdocscl1| and |cdocscl2|
% which should be identical to
% |cdocsdrf|, |cdocsch1| and |cdocsfn2|, respectively:
% \begin{center}
% \begin{tabular}{l}
% |latex -jobname cdocscld \|\\
% |  "\def\version{draft}\input{childdoc.def}\childdocforward{cdocsamp}"|\\
% |latex -jobname cdocscl1 \|\\
% |  "\input{childdoc.def}\childdocforward[cdocsamp]{cdocsch1}"|\\
% |latex -jobname cdocscl2 \|\\
% |  "\def\version{final}\input{childdoc.def}\childdocforward{cdocsch2}"|
% \end{tabular}
% \end{center}
% Note that the trailing backslash on each first line
% merely continues the input to the second line
% (for convenient cut ant paste).
% Furthermore, the command |latex| can be replaced by any
% of its alternative versions such as |pdflatex|.
%
% %%%%%%%%%%%%%%%%%%%%%%%%%%%%%%%%%%%%%%%%%%%%%%%%%%%%%%%%%%%%%%%%%%%%%%%%%%%%%%
% %%%%%%%%%%%%%%%%%%%%%%%%%%%%%%%%%%%%%%%%%%%%%%%%%%%%%%%%%%%%%%%%%%%%%%%%%%%%%%
% \section{Implementation}
%\iffalse
%<*package>
%\fi
%
% This section describes the definitions file |childdoc.def|.

% The definitions cannot be loaded using |\usepackage| or |\RequirePackage|
% which has a mechanism to prevent loading a style file more than once.
% When loading the definitions by means of |\input|
% multiple instances have to be prevented manually:
%\iffalse
%This code needs to be before the `\ProvidesFile' directive
%which is defined at the beginning of this file.
%Therefore it is also placed there and commented out here.
%</package>
%<*discard>
%\fi
%    \begin{macrocode}
\ifdefined\childdocmain\endinput\fi
%    \end{macrocode}
%\iffalse
%</discard>
%<*package>
%\fi
%
% \macro{\ifchilddoc}
% \macro{\ifchilddocmanual}
% The conditional |\ifchilddoc| tells whether a
% child (true) or main (false) document is being compiled.
% The conditional |\ifchilddocmanual| tells whether
% the |\includeonly| mechanism is used (false) or
% the selection of child files must be performed manually (true).
% The definitions initialise to false:
%    \begin{macrocode}
\newif\ifchilddoc
\newif\ifchilddocmanual
%    \end{macrocode}

% \macro{\childdocname}
% \macro{\childdocjob}
% The macro |\childdocname| stores the name of the main document
% to be compiled. The macro |\childdocjob| stores the name of
% the document on which the \LaTeX{} compiler was originally invoked.
% The content of |\jobname| cannot be compared
% to filenames specified in the source due to different catcodes.
% The following code rescans |\jobname|, stores the result
% in |\childdocname| and saves a copy in |\childdocjob|:
%    \begin{macrocode}
\edef\childdocname{\scantokens\expandafter{\jobname\noexpand}}
\let\childdocjob\childdocname
%    \end{macrocode}

% \macro{\childdocdisable}
% The macro |\childdocdisable| prevents the main file
% from being processed more than once.
% At this stage, the main document command |\childdocmain|
% is assumed to be called once again where it should do nothing.
% Any subsequent call to it should prevent
% a secondary processing of the main document
% It overwrites the forwarding commands
% |\childdocof| and |\childdocforward|
% with empty macros to prevent further inclusions of the main document:
%    \begin{macrocode}
\newcommand{\childdocdisable}
{
  \renewcommand{\childdocmain}[1]{\renewcommand{\childdocmain}[1]{\endinput}}
  \renewcommand{\childdocof}[1]{}
  \renewcommand{\childdocby}[2][]{}
  \renewcommand{\childdocforward}[2][]{}
  \renewcommand{\childdocdisable}{}
}
%    \end{macrocode}

% \macro{\childdocmain}
% The macro |\childdocmain| is to be called at the top of the main file
% with nothing or the main filename (without extension) as argument.
% First, it breaks loops.
% If the argument is not empty and does not match |\childdocname|
% (which is set by the first inclusion of |childdoc.def|),
% |\ifchilddoc| is set to true, |\includeonly| is applied to the child file
% and |\jobname| is set to the main file
% (for proper handling of |.aux| files):
%    \begin{macrocode}
\newcommand{\childdocmain}[1]
{
  \childdocdisable\childdocmain{}
  \if?#1?\else
    \begingroup
      \def\childdoctmp{#1}
      \ifx\childdoctmp\childdocname
        \def\childdoctmp{}
      \else
        \def\childdoctmp
        {
          \childdoctrue
          \includeonly{\childdocname}
          \def\childdocjob{#1}
          \def\jobname{#1}
        }
      \fi
      \expandafter
    \endgroup
    \childdoctmp
  \fi
}
%    \end{macrocode}

% \macro{\childdocof}
% The command |\childdocof| redirects
% compilation to the main file |#1|.
%    \begin{macrocode}
\newcommand{\childdocof}[1]
{
  \childdocdisable
  \childdoctrue
  \includeonly{\childdocname}
  \def\jobname{#1}
  \def\childdocjob{#1}
  \input{#1}
}
%    \end{macrocode}

% \macro{\childdocby}
% The command |\childdocby| ....
%    \begin{macrocode}
\newcommand{\childdocby}[2][]
{
  \childdocdisable
  \childdoctrue
  \childdocmanualtrue
  \if?#1?\else
    \def\jobname{#2}
  \fi
  \def\childdocjob{#2}
  \input{#2}
  \endinput
}
%    \end{macrocode}

% \macro{\childdocforward}
% The command |\childdocforward| redirects
% compilation to the main file or
% (if the optional argument is given) a child file.
% Parameters are set as if the main file
% or a child file starting with |\childdocof| was compiled.
% Then compilation is handed over to the main file:
%    \begin{macrocode}
\newcommand{\childdocforward}[2][]
{
  \begingroup
    \if?#1?
      \def\childdoctmp
      {
        \def\childdocname{#2}
        \def\childdocjob{#2}
        \def\jobname{#2}
        \input{#2}
        \endinput
      }
    \else
      \def\childdoctmp
      {
        \childdocdisable
        \def\childdocname{#2}
        \childdoctrue
        \includeonly{#2}
        \def\childdocjob{#1}
        \def\jobname{#1}
        \input{#1}
        \endinput
      }
    \fi
    \expandafter
  \endgroup
  \childdoctmp
}
%    \end{macrocode}

% \macro{\childdocforwardprefix}
% The command |\childdocforwardprefix| redirects
% compilation to the main or a child file by means of a pattern.
% The prefix |#1| in the current filename is replaced by |#2|
% and the suffix of the current filename is kept
% (it is assumed that the filename does not contain the substring `|~~~|'
% which is used as a delimiter).
% Compilation is handed over to the new file by |\childdocforward|:
%    \begin{macrocode}
\newcommand{\childdocforwardprefix}[3][]
{
  \begingroup
    \def\childdocextract #2##1~~~{\def\childdoctmp{\childdocforward[#1]{#3##1}}}
    \expandafter\childdocextract\childdocname~~~
    \expandafter
  \endgroup
  \childdoctmp
}
%    \end{macrocode}

% \macro{\childdoc}
% The deprecated macro |\childdoc| is a legacy version of |\childdocmain|:
%    \begin{macrocode}
\newcommand{\childdoc}{\childdocmain}
%    \end{macrocode}

% \macro{\childdocredirect}
% The deprecated macro |\childdocredirect| is a legacy version
% of |\childdocforward| and |\childdocforwardprefix|:
%    \begin{macrocode}
\newcommand{\childdocredirect}[2][]
{
  \begingroup
    \if?#1?
      \def\childdoctmp{\childdocforward{#2}}
    \else
      \def\childdoctmp{\childdocforwardprefix{#1}{#2}}
    \fi
    \expandafter
  \endgroup
  \childdoctmp
}
%    \end{macrocode}

%\iffalse
%</package>
%\fi
%
\endinput
|\\
|\childdocforwardprefix{final}{child}|
\end{tabular}
\end{center}
%

Note that when several versions of a main file and/or of each child file
are to be generated, it may be convenient to set up a |Makefile| or
shell script to automatise the process.

%%%%%%%%%%%%%%%%%%%%%%%%%%%%%%%%%%%%%%%%%%%%%%%%%%%%%%%%%%%%%%%%%%%%%%%%%%%%%%%%
\subsection{Command Line Processing}
\label{sec:commandline}

The effect of redirection files can also be achieved by invoking
the \LaTeX{} compiler with a more elaborate command line.
Most conveniently this should be done as part
of a shell script or a |Makefile|.

When using \textsf{childdoc} in the main file, the following
command lines effectively perform a redirection
(note that depending on the shell being used,
backslashes may have to be doubled: `|\|' $\to$ `|\\|'):
%
\begin{center}
|... -jobname "|\textit{target}|" |\\|"|[\textit{flags}]%
|% \iffalse
%
% childdoc.dtx Copyright (C) 2017-2018 Niklas Beisert
%
% This work may be distributed and/or modified under the
% conditions of the LaTeX Project Public License, either version 1.3
% of this license or (at your option) any later version.
% The latest version of this license is in
%   http://www.latex-project.org/lppl.txt
% and version 1.3 or later is part of all distributions of LaTeX
% version 2005/12/01 or later.
%
% This work has the LPPL maintenance status `maintained'.
%
% The Current Maintainer of this work is Niklas Beisert.
%
% This work consists of the files childdoc.dtx and childdoc.ins
% and the derived files childdoc.def and cdocsamp.tex with
% cdocsch1.tex, cdocsch2.tex, cdocsdrf.tex, cdocsfn1.tex, cdocsfn2.tex.
%
%<package>\ifdefined\childdocmain\endinput\fi
%<package>\ProvidesFile{childdoc.def}[2018/12/30 v2.0 child document driver]
%<samplemain>\ProvidesFile{cdocsamp.tex}[2018/12/30 v2.0 sample for childdoc]
%<*driver>
%\ProvidesFile{childdoc.drv}[2018/12/30 v2.0 childdoc reference manual file]
\PassOptionsToClass{10pt,a4paper}{article}
\documentclass{ltxdoc}

\usepackage[margin=35mm]{geometry}
\usepackage{hyperref}
\usepackage{hyperxmp}
\usepackage[usenames]{color}

\hypersetup{colorlinks=true}
\hypersetup{pdfstartview=FitH}
\hypersetup{pdfpagemode=UseNone}
\hypersetup{pdfsource={}}
\hypersetup{pdflang={en-UK}}
\hypersetup{pdfcopyright={Copyright 2017-2018 Niklas Beisert.
  This work may be distributed and/or modified under the
  conditions of the LaTeX Project Public License, either version 1.3
  of this license or (at your option) any later version.}}
\hypersetup{pdflicenseurl={http://www.latex-project.org/lppl.txt}}
\hypersetup{pdfcontactaddress={ETH Zurich, ITP, HIT K,
  Wolfgang-Pauli-Strasse 27}}
\hypersetup{pdfcontactpostcode={8093}}
\hypersetup{pdfcontactcity={Zurich}}
\hypersetup{pdfcontactcountry={Switzerland}}
\hypersetup{pdfcontactemail={nbeisert@itp.phys.ethz.ch}}
\hypersetup{pdfcontacturl={http://people.phys.ethz.ch/\xmptilde nbeisert/}}

\newcommand{\secref}[1]{\hyperref[#1]{section \ref*{#1}}}

\parskip1ex
\parindent0pt
\let\olditemize\itemize
\def\itemize{\olditemize\parskip0pt}

\begin{document}

\title{The \textsf{childdoc} Package}
\hypersetup{pdftitle={The childdoc Package}}
\author{Niklas Beisert\\[2ex]
  Institut f\"ur Theoretische Physik\\
  Eidgen\"ossische Technische Hochschule Z\"urich\\
  Wolfgang-Pauli-Strasse 27, 8093 Z\"urich, Switzerland\\[1ex]
  \href{mailto:nbeisert@itp.phys.ethz.ch}
  {\texttt{nbeisert@itp.phys.ethz.ch}}}
\hypersetup{pdfauthor={Niklas Beisert}}
\hypersetup{pdfsubject={Manual for the LaTeX2e Package childdoc}}
\date{30 December 2018, \textsf{v2.0}}
\maketitle

\begin{abstract}\noindent
\textsf{childdoc} is a \LaTeXe{} package
that enables the direct compilation
of document sections included by |\include|
to individual files.
\end{abstract}

\begingroup
\parskip0ex
\tableofcontents
\endgroup

%%%%%%%%%%%%%%%%%%%%%%%%%%%%%%%%%%%%%%%%%%%%%%%%%%%%%%%%%%%%%%%%%%%%%%%%%%%%%%%%
%%%%%%%%%%%%%%%%%%%%%%%%%%%%%%%%%%%%%%%%%%%%%%%%%%%%%%%%%%%%%%%%%%%%%%%%%%%%%%%%
\section{Introduction}

\LaTeX{} provides a mechanism to structure a large document (such as a book)
into a main file and several child files (containing the chapters)
using the |\include| command.
This mechanism is beneficial for documents
which span hundreds of pages in order to
make the source file(s) more manageable.
Moreover, compilation can be restricted to
selected child files by means of the |\includeonly| command.
The latter feature can be used to reduce the compilation time while editing
(this was significantly more useful in the earlier days of \LaTeX{})
or to generate a smaller document which is easier to navigate.
Another application of |\includeonly| is to generate
documents consisting of selected parts of the complete document.

However, there are a few drawbacks of the plain |\include| mechanism:
\begin{itemize}
\item
The child files cannot be compiled on their own,
they can only be compiled via the main file.
A naive editing environment
(such as a text editor with an option
to have the current file processed by \LaTeX)
may require one to switch to the main file before compiling;
attempting to compile the child file produces errors.
\item
The main file must be modified (each time)
to adjust the |\includeonly| command
to the present needs. This easily leaves the main file in a messy state.
\item
The generated document will always carry the filename
of the main document. This is inconvenient if
several child files are to be compiled and
to be kept for distribution.
\end{itemize}

The present package provides a simple interface
to make child files individually compilable by \LaTeX{}.
Compiling a child file then has the same effect as compiling
the main file with an |\includeonly| command
to select the appropriate child.
Moreover the generated document will carry the name of the child
rather than the main file.
This resolves all three above issues.

This feature is meant to make the editing of books,
thesis documents and lecture notes somewhat more convenient.
However, the package can also be used efficiently for
composing a series of documents (such as exercise sheets)
which are typically distributed individually.
It then assists the author in generating the individual documents
(potentially in different versions)
as well as a document containing the collected series.
Another application is in developing style files
or other kinds of included material
where compilation of the style file could redirect
to a sample or test file.

%%%%%%%%%%%%%%%%%%%%%%%%%%%%%%%%%%%%%%%%%%%%%%%%%%%%%%%%%%%%%%%%%%%%%%%%%%%%%%%%
%%%%%%%%%%%%%%%%%%%%%%%%%%%%%%%%%%%%%%%%%%%%%%%%%%%%%%%%%%%%%%%%%%%%%%%%%%%%%%%%
\section{Usage}

First of all, the package \textsf{childdoc} is \emph{not} a standard
\LaTeXe{} |.sty| style file! Therefore it needs to be invoked in
a non-standard way.

%%%%%%%%%%%%%%%%%%%%%%%%%%%%%%%%%%%%%%%%%%%%%%%%%%%%%%%%%%%%%%%%%%%%%%%%%%%%%%%%
\subsection{Included Files}
\label{sec:include}

%%%%%%%%%%%%%%%%%%%%%%%%%%%%%%%%%%%%%%%%
\DescribeMacro{\childdocmain}
To use the package, add the commands
\begin{center}
\begin{tabular}{l}
|\input{childdoc.def}|\\
|\childdocmain{}|\\
\end{tabular}
\end{center}
at the very top of the main \LaTeX{} file,
in particular \emph{before} the |\documentclass| statement!
The argument of |\childdocmain| should be left empty
(but it must be present).

%%%%%%%%%%%%%%%%%%%%%%%%%%%%%%%%%%%%%%%%
\DescribeMacro{\childdocof}
Furthermore, add the commands
\begin{center}
\begin{tabular}{l}
|\input{childdoc.def}|\\
|\childdocof{|\textit{main}|}|\\
\end{tabular}
\end{center}
at the top of every child file \textit{child}
which is included by |\include{|\textit{child}|}|
from within the main file
(or at least for those files to be compiled individually).
The argument \textit{main} must be the filename of the main file.

There are a couple of
considerations in setting up the main and child documents:

%%%%%%%%%%%%%%%%%%%%%%%%%%%%%%%%%%%%%%%%
\paragraph{Restrictions.}

Please note the following restrictions:
\begin{itemize}
\item
|\childdocmain| must be called with one argument \textit{main}
to ensure compatibility with earlier version of the package.
It must either be empty (|\childdocmain{}|)
or precisely match the filename of the main file in which it is specified.
See \secref{sec:detection} for further information.
\item
The filename \textit{main} must be specified without the |.tex| extension.
\item
The filename \textit{main} is case sensitive
(even in case-insensitive file systems)
due to internal string comparison.
\item
The argument \textit{main} should be fully expanded, it cannot be a macro.
\item
Subdirectories and special characters should be avoided in filenames.
\item
The command |\childdocmain{|\textit{main}|}| must be followed by a whitespace.
It should not be followed immediately by another command
or by a comment mark `|%|'.
This is because the \TeX{} parser reads the token immediately following
the argument of |\childdocmain| and puts it
at the beginning of every child section;
however, a white\-space is ignored.
\end{itemize}

%%%%%%%%%%%%%%%%%%%%%%%%%%%%%%%%%%%%%%%%
\paragraph{Content of Main File.}

It is advisable to place all content in the child files included by |\include|.
Any output contained in the main file will appear in all child documents
unless suppressed manually;
it cannot be suppressed automatically by the |\includeonly| directive
and thus should normally be avoided.
A method to include some content in the main file
by means of conditional processing is described in \secref{sec:conditional}.

%%%%%%%%%%%%%%%%%%%%%%%%%%%%%%%%%%%%%%%%
\paragraph{Page Numbering.}

When only a part of the document is compiled,
the appropriate numbering of pages
(as well as other status parameters)
is determined from the |.aux| files.
The latter contain information from previous passes.
However this information needs to propagate through
all intermediate child documents.
Therefore the page numbering in child documents may well
be inconsistent until the complete document is compiled at least once.

A useful (if unconventional) way to always ensure a consistent
page numbering is to restart the numbering in each child document
and denote the pages by `\textit{child}|.|\textit{page}'
where \textit{child} represents the chapter/section number of the child file.
This can be achieved by the command
|\numberwithin{page}{|\textit{child}|}|
of the \textsf{amsmath} package
where \textit{child} can be |chapter| or |section|
depending on the chosen structuring.
Alternatively, one can modify the macro |\thepage| appropriately
and reset the counter |page| at the start of each child file.

%%%%%%%%%%%%%%%%%%%%%%%%%%%%%%%%%%%%%%%%%%%%%%%%%%%%%%%%%%%%%%%%%%%%%%%%%%%%%%%%
\subsection{Conditional Processing}
\label{sec:conditional}

The package provides a mechanism to compile different versions
of a document. To customise the versions further some conditional processing
can come in handy to distinguish which version is being compiled.
The package provides two macros to describe the compilation context:

%%%%%%%%%%%%%%%%%%%%%%%%%%%%%%%%%%%%%%%%
\DescribeMacro{\ifchilddoc}
The conditional |\ifchilddoc| distinguishes between the compilation of
child documents and the main document:
%
\begin{center}
|\ifchilddoc |\textit{child-code}| |[|\||else |\textit{main-code}]| \||fi|
\end{center}

%%%%%%%%%%%%%%%%%%%%%%%%%%%%%%%%%%%%%%%%
\DescribeMacro{\childdocname}
\DescribeMacro{\childdocjob}
The macro |\childdocname| contains the filename (without extension)
of the main or child file being processed.
Note that |\childdocjob| will always contain the name of the main file.

%%%%%%%%%%%%%%%%%%%%%%%%%%%%%%%%%%%%%%%%
\paragraph{Title Page.}

Conditional processing can be used to include a title or banner page
in the main document when proper precautions are taken.
Importantly, the code in the main file should ensure that the page counter
(as well as other status parameters which are stored in the |.aux| files)
takes the same value after the conditional processing.
Otherwise the page numbers may take divergent values
depending on which part is compiled.

For example, a title page could be declared by:
%
\begin{center}
\begin{tabular}{l}
|\ifchilddoc\||else|\\
|\addtocounter{page}{-1}|\\
\textit{code for title page}\\
|\newpage|\\
|\||fi|
\end{tabular}
\end{center}
%
A banner page for the child documents can be generated by:
%
\begin{center}
\begin{tabular}{l}
|\ifchilddoc|\\
|\addtocounter{page}{-1}|\\
\textit{code for banner page}\\
|\newpage|\\
|\||fi|
\end{tabular}
\end{center}
%
Here one could write a message such as:
\begin{center}
|This is the part \childdocname{} of \childdocjob{}.|
\end{center}

%%%%%%%%%%%%%%%%%%%%%%%%%%%%%%%%%%%%%%%%%%%%%%%%%%%%%%%%%%%%%%%%%%%%%%%%%%%%%%%%
\subsection{Flags}
\label{sec:flags}

The package makes it easy to generate different versions
of the main or child documents.
To this end compilation flags can be defined
and assigned different default values.
They will be particularly useful in conjunction
with the forwarding mechanism described in \secref{sec:forward}.

For example, it may be useful to have a flag |\version|
which can be set to |draft| or |final|.
The document source will contain some conditional code
depending on the value of |\version|.
Suppose further, the flag should default to |final| for the main file
and to |draft| for child files
which is a natural assignment for editing the document.
This is achieved by placing the following code
in the preamble of the main document
(below the |\childdocmain| directive):
%
\begin{center}
\begin{tabular}{l}
|\ifchilddoc|\\
|\providecommand{\version}{draft}|\\
|\||else|\\
|\providecommand{\version}{final}|\\
|\||fi|
\end{tabular}
\end{center}
%
The definition by |\providecommand| makes sure
that previous definitions are not overwritten.
Further statements |\providecommand{\version}{...}|
can thus be added before the above code to override it.

For the main file, one might add a line
(between |\childdocmain| and the above block)
%
\begin{center}
|%\ifchilddoc\||else\providecommand{\version}{draft}\||fi|
\end{center}
%
which can be uncommented to produce a draft version.
Likewise one can add a line to the very top of a child file
(above the |\childdocof{|\textit{main}|}| directive)
%
\begin{center}
|%\providecommand{\version}{final}|
\end{center}
%
which can be uncommented to produce the final version of this child document.

%%%%%%%%%%%%%%%%%%%%%%%%%%%%%%%%%%%%%%%%%%%%%%%%%%%%%%%%%%%%%%%%%%%%%%%%%%%%%%%%
\subsection{Forwarding}
\label{sec:forward}

Different versions of the main or child documents
using compilation flags as described in \secref{sec:flags}
can be (permanently) stored in different files
for convenient compilation, viewing and distribution.
To this end, the package defines a command
to pass on compilation to a different file:

%%%%%%%%%%%%%%%%%%%%%%%%%%%%%%%%%%%%%%%%
\DescribeMacro{\childdocforward}
The command |\childdocforward| redirects processing to
another source file:
%
\begin{center}
\begin{tabular}{l}
|\input{childdoc.def}|\\
|\childdocforward[|\textit{main}|]{|\textit{dest}|}|\\
\end{tabular}
\end{center}
%
The argument \textit{dest} is the destination file
(without extension).
It should be the main file or one of the child files.
Note that further \textsf{childdoc} directives
such as |\childdocof| and |\childdocforward|
in the indicated file will be processed in this form.
The optional argument \textit{main}
passes on directly to the main file \textit{main}
while pretending to compile the child \textit{dest}.
This form behaves as if \textit{dest}
issues |\childdocof{|\textit{main}|}| right away,
and no further \textsf{childdoc} directives will be processed.

%%%%%%%%%%%%%%%%%%%%%%%%%%%%%%%%%%%%%%%%
\DescribeMacro{\...prefix}
In the alternative form |\childdocforwardprefix|,
%
\begin{center}
\begin{tabular}{l}
|\input{childdoc.def}|\\
|\childdocforwardprefix[|\textit{main}|]{|\textit{prefix}|}{|\textit{dest}|}|
\end{tabular}
\end{center}
%
the destination file is determined by a pattern
depending on the current file:
To make this work, the current file must be called
`{\textit{prefix}\hspace{0.2em}\textit{suffix}}'
with \textit{prefix} matching precisely the argument.
Processing is then passed on to the file
`{\textit{dest}\hspace{0.2em}\textit{suffix}}'.
Surely, the same effect is achieved by
directly specifying the
argument `{\textit{dest}\hspace{0.2em}\textit{suffix}}'
in the first form.
However, that requires to set up a different file
for each child. With the alternative form of the command
all these files can have exactly the same content
which simplifies setting them up and maintaining them.

For example, the following file |draft.tex|
with a compilation flag |\version| as described in \secref{sec:flags}
compiles the main document as a draft:
%
\begin{center}
\begin{tabular}{l}
|\def\version{draft}|\\
|\input{childdoc.def}|\\
|\childdocforward{|\textit{main}|}|
\end{tabular}
\end{center}
%
Likewise, the following files |final|\textit{nn}|.tex|
compile the final version of the child document
|child|\textit{nn}|.tex|:
%
\begin{center}
\begin{tabular}{l}
|\def\version{final}|\\
|\input{childdoc.def}|\\
|\childdocforwardprefix{final}{child}|
\end{tabular}
\end{center}
%

Note that when several versions of a main file and/or of each child file
are to be generated, it may be convenient to set up a |Makefile| or
shell script to automatise the process.

%%%%%%%%%%%%%%%%%%%%%%%%%%%%%%%%%%%%%%%%%%%%%%%%%%%%%%%%%%%%%%%%%%%%%%%%%%%%%%%%
\subsection{Command Line Processing}
\label{sec:commandline}

The effect of redirection files can also be achieved by invoking
the \LaTeX{} compiler with a more elaborate command line.
Most conveniently this should be done as part
of a shell script or a |Makefile|.

When using \textsf{childdoc} in the main file, the following
command lines effectively perform a redirection
(note that depending on the shell being used,
backslashes may have to be doubled: `|\|' $\to$ `|\\|'):
%
\begin{center}
|... -jobname "|\textit{target}|" |\\|"|[\textit{flags}]%
|\input{childdoc.def}\childdocforward[|\textit{main}|]{|\textit{dest}|}"|
\end{center}
%
Here \textit{target} is the name of the output file,
\textit{main} is the name of the main file
and \textit{dest} is the name of the main or child file to be processed
(all filenames without extensions).
The optional argument \textit{main} can be omitted
if \textit{main} matches \textit{dest}.
Optionally, compilation \textit{flags} can be defined via |\def| commands.
This command line makes the \TeX{} engine believe
it is compiling the file \textit{target}
whose content is specified as the latter parameter.
The provided code then forwards the processing to
\textit{main} or \textit{dest} as described in \secref{sec:forward}.

%%%%%%%%%%%%%%%%%%%%%%%%%%%%%%%%%%%%%%%%%%%%%%%%%%%%%%%%%%%%%%%%%%%%%%%%%%%%%%%%
\subsection{Include by Input}
\label{sec:input}

Including child documents by |\include| has some restrictions by design.
Most notably, the content of a child document always occupies
its own set of pages; pages cannot be shared between child documents.
Usually, this behaviour makes perfect sense
because each child document contain an essential part of the document.
However, in some situations it may be desirable to compose
a document from a collection of parts
without having mandatory page breaks between then.
For this case, the package
provides a mechanism to include parts
by |\input| which can also be processed individually.
However, by construction this mechanism
requires manual handling of the content to be output.

%%%%%%%%%%%%%%%%%%%%%%%%%%%%%%%%%%%%%%%%
\DescribeMacro{\ifchilddocmanual}
The main file should be prepared as usual, see \secref{sec:include}.
However, the document body must make a distinction
between processing of an individual part and of the main document, e.g.:
%
\begin{center}
\begin{tabular}{l}
|\ifchilddocmanual|\\
|\input{\childdocname}|\\
|\||else|\\
\textit{document body with }|\input{|\textit{part}|}|\\
|\||fi|
\end{tabular}
\end{center}
%
The conditional |\ifchilddocmanual| is true whenever
a part to be included by |\input| is being compiled,
and the name of the part is stored in |\childdocname|.

%%%%%%%%%%%%%%%%%%%%%%%%%%%%%%%%%%%%%%%%
\DescribeMacro{\childdocby}
Each part to be included by |\input| should start with:
%
\begin{center}
\begin{tabular}{l}
|\input{childdoc.def}|\\
|\childdocby{|\textit{main}|}|\\
\end{tabular}
\end{center}
%
The directive |\childdocby| is similar to |\childdocof|
described in \secref{sec:include},
but the subsequent selection of content must be done manually.
To that end, both |\ifchilddoc| and |\ifchilddocmanual|
will be true upon processing of a part,
and the name of the part is stored in |\childdocname|.
Note that |\jobname| will be set to the filename of the current part
so that each part receives an individual |.aux| file
that does not interfere with the |.aux| file(s) of the main document.
This behaviour can be altered by the alternative form
|\childdocby[*]{|\textit{main}|}| (with a non-empty optional argument)
which uses the |.aux| file of the main document
by setting |\jobname| to \textit{main}.

%%%%%%%%%%%%%%%%%%%%%%%%%%%%%%%%%%%%%%%%%%%%%%%%%%%%%%%%%%%%%%%%%%%%%%%%%%%%%%%%
\subsection{Driver Development}
\label{sec:driver}

The \textsf{childdoc} mechanism can also be use for the development
of definition files such as \LaTeX{} styles or classes.
This case differs from the above setup with multiple parts
included by |\include| in that no |\includeonly| should be invoked.
This can be achieved by starting the include file
(before |\ProvidesPackage|) with:
%
\begin{center}
\begin{tabular}{l}
|\input{childdoc.def}|\\
|\childdocforward{|\textit{main}|}|\\
\end{tabular}
\end{center}
%
or alternatively with:
%
\begin{center}
\begin{tabular}{l}
|\input{childdoc.def}|\\
|\childdocby{|\textit{main}|}|\\
\end{tabular}
\end{center}
%
Both forms have slightly different effects as described above.
The main file is prepared as usual, see \secref{sec:include}.

%%%%%%%%%%%%%%%%%%%%%%%%%%%%%%%%%%%%%%%%%%%%%%%%%%%%%%%%%%%%%%%%%%%%%%%%%%%%%%%%
\subsection{Legacy Detection}
\label{sec:detection}

The directive |\childdocmain| in the main file can detect
whether the complete document or merely a child is to be compiled
even without using the directive |\childdocof|.
This method is deprecated because it is less robust
and there is no compelling reason to use it;
it is merely provided for backward compatibility
and it may be removed in future versions.

If the detection mechanism is to be used,
it is mandatory to correctly specify
the filename of the main file as the argument of |\childdocmain|:
%
\begin{center}
\begin{tabular}{l}
|\input{childdoc.def}|\\
|\childdocmain{|\textit{main}|}|\\
\end{tabular}
\end{center}
%
If |\jobname| does not match the argument \textit{main} of |\childdocmain|,
it is assumed that |\jobname| points to the child file to be compiled.
When using |\childdocmain| with the main file specified as argument,
it suffices to start a child file
with just |\input{|\textit{main}|}|
without loading of the package and using |\childdocof|.
If instead all processing is done
with the appropriate \textsf{childdoc} directives,
the argument of \textit{main} of |\childdocmain| can be empty.

An alternative version of the command line processing described
in \secref{sec:commandline} using the detection mechanism reads:
%
\begin{center}
|... -jobname "|\textit{target}|" "|[\textit{flags}]%
[|\def\jobname{|\textit{dest}|}|]|\input{|\textit{main}|}"|
\end{center}

%%%%%%%%%%%%%%%%%%%%%%%%%%%%%%%%%%%%%%%%%%%%%%%%%%%%%%%%%%%%%%%%%%%%%%%%%%%%%%%%
\subsection{Manual Code}
\label{sec:manual}

In case one cannot be certain whether the definitions file |childdoc.def|
is installed on the target \TeX{} distribution
and one prefers not to ship it,
it is conceivable to paste a few relevant commands into the sources.

To that end, drop all statements |\input{childdoc.def}|
and perform the replacements as outlined below.
Instead of |\childdocmain{|\textit{main}|}| add the following code
to the top of the main file:
%
\begin{center}
\begin{tabular}{l}
|\||ifdefined\childdocname\endinput\||fi\newif\ifchilddoc|\\
|\edef\childdocname{\scantokens\expandafter{\jobname\noexpand}}|\\
|\def\childdocmain{|\textit{main}|}\||ifx\childdocmain\childdocname\||else|\\
|\childdoctrue\includeonly{\childdocname}\let\jobname\childdocmain\||fi|\\
\end{tabular}
\end{center}
%
Instead of |\childdocof{|\textit{main}|}| just include the main file
at the top of each child file:
%
\begin{center}
|\input{|\textit{main}|}|
\end{center}
%
A simple redirection |\childdocforward{|\textit{dest}|}| is achieved by:
%
\begin{center}
|\def\jobname{|\textit{dest}|}\input{\jobname}|
\end{center}
%
The redirection with prefix
|\childdocforwardprefix[|\textit{prefix}|]{|\textit{dest}|}|
is accomplished by:
%
\begin{center}
\begin{tabular}{l}
|{\edef\jobname{\scantokens\expandafter{\jobname\noexpand}}|\\
|\def\redirectjob |\textit{prefix}|#1~~~{\gdef\jobname{|\textit{dest}|#1}}|\\
|\expandafter\redirectjob\jobname~~~}\input{\jobname}|
\end{tabular}
\end{center}

In an alternative approach,
child documents can be compiled by a specific command line
without additional code or specific definitions:
%
\begin{center}
|... -jobname "|\textit{target}|" "|[\textit{flags}]%
|\includeonly{|\textit{dest}|}\input{|\textit{main}|}"|
\end{center}
%

%%%%%%%%%%%%%%%%%%%%%%%%%%%%%%%%%%%%%%%%%%%%%%%%%%%%%%%%%%%%%%%%%%%%%%%%%%%%%%%%
%%%%%%%%%%%%%%%%%%%%%%%%%%%%%%%%%%%%%%%%%%%%%%%%%%%%%%%%%%%%%%%%%%%%%%%%%%%%%%%%
\section{Information}

%%%%%%%%%%%%%%%%%%%%%%%%%%%%%%%%%%%%%%%%%%%%%%%%%%%%%%%%%%%%%%%%%%%%%%%%%%%%%%%%
\subsection{Copyright}

Copyright \copyright{} 2017--2018 Niklas Beisert

This work may be distributed and/or modified under the
conditions of the \LaTeX{} Project Public License, either version 1.3
of this license or (at your option) any later version.
The latest version of this license is in
  \url{http://www.latex-project.org/lppl.txt}
and version 1.3 or later is part of all distributions of \LaTeX{}
version 2005/12/01 or later.

This work has the LPPL maintenance status `maintained'.

The Current Maintainer of this work is Niklas Beisert.

This work consists of the files |README.txt|, |childdoc.ins| and |childdoc.dtx|
as well as the derived files |childdoc.def|, |cdocsamp.tex|
with |cdocsch1.tex|, |cdocsch2.tex|, |cdocspt3.tex|, |cdocspt4.tex|,
|cdocsdrf.tex|, |cdocsfn1.tex|, |cdocsfn2.tex|
as well as |childdoc.pdf|.

%%%%%%%%%%%%%%%%%%%%%%%%%%%%%%%%%%%%%%%%%%%%%%%%%%%%%%%%%%%%%%%%%%%%%%%%%%%%%%%%
\subsection{Files and Installation}

The package consists of the files:
%
\begin{center}
\begin{tabular}{ll}
    |README.txt|   & readme file \\
    |childdoc.ins| & installation file \\
    |childdoc.dtx| & source file \\
    |childdoc.def| & definition file \\
    |cdocsamp.tex| & sample main file \\
    |cdocsch1.tex| & sample include file \\
    |cdocsch2.tex| & sample include file \\
    |cdocspt3.tex| & sample part file \\
    |cdocspt4.tex| & sample part file \\
    |cdocsdrf.tex| & sample redirection file \\
    |cdocsfn1.tex| & sample redirection file \\
    |cdocsfn2.tex| & sample redirection file \\
    |childdoc.pdf| & manual
\end{tabular}
\end{center}
%
The distribution consists of the files
|README.txt|, |childdoc.ins| and |childdoc.dtx|.
%
\begin{itemize}
\item
Run (pdf)\LaTeX{} on |childdoc.dtx|
to compile the manual |childdoc.pdf| (this file).
\item
Run \LaTeX{} on |childdoc.ins| to create the definitions file |childdoc.def|
and the sample |cdocsamp.tex| with include files
|cdocsch1.tex|, |cdocsch2.tex|, |cdocspt3.tex|, |cdocspt4.tex|,
|cdocsdrf.tex|, |cdocsfn1.tex|, |cdocsfn2.tex|.
Then copy the file |childdoc.def| to an appropriate directory of your \LaTeX{}
distribution, e.g.\ \textit{texmf-root}|/tex/latex/childdoc|.
\end{itemize}

%%%%%%%%%%%%%%%%%%%%%%%%%%%%%%%%%%%%%%%%%%%%%%%%%%%%%%%%%%%%%%%%%%%%%%%%%%%%%%%%
\subsection{Related CTAN Packages}

There are several other packages which offer a similar functionality:
%
\begin{itemize}
\item
The packages
\href{http://ctan.org/pkg/docmute}{\textsf{docmute}},
\href{http://ctan.org/pkg/includex}{\textsf{includex}} and
\href{http://ctan.org/pkg/standalone}{\textsf{standalone}}
provide commands to include only the document body of
a child file thus allowing both files to be compiled individually.
\item
The packages \href{http://ctan.org/pkg/subdocs}{\textsf{subdocs}}
and \href{http://ctan.org/pkg/subfiles}{\textsf{subfiles}}
provide structures in which the main and child documents can be
encapsulated and allowing them to be compiled individually.
The inclusion mechanism is different from the conventional |\include|.
\item
The package \href{http://ctan.org/pkg/combine}{\textsf{combine}}
is an elaborate solution to combine several documents into one.
\end{itemize}
%
See also the CTAN topic \href{http://ctan.org/topic/subdocs}{\textsf{subdocs}}
for further related packages.
The present package differs from the above solutions in that
a document structure constructed with the conventional |\include| mechanism
just needs two extra commands at the top of every file
such that all constituent files can be compiled individually.

%%%%%%%%%%%%%%%%%%%%%%%%%%%%%%%%%%%%%%%%%%%%%%%%%%%%%%%%%%%%%%%%%%%%%%%%%%%%%%%%
%\subsection{Feature Suggestions}
%
%The following is a list of features which may be useful for future
%versions of this package:
%%
%\begin{itemize}
%\item
%\ldots
%\end{itemize}

%%%%%%%%%%%%%%%%%%%%%%%%%%%%%%%%%%%%%%%%%%%%%%%%%%%%%%%%%%%%%%%%%%%%%%%%%%%%%%%%
\subsection{Revision History}

%%%%%%%%%%%%%%%%%%%%%%%%%%%%%%%%%%%%%%%%
\paragraph{v2.0:} 2018/12/30

\begin{itemize}
\item
immediate forward processing
\item
added |\childdocby| mechanism
\item
manual restructured
\end{itemize}

%%%%%%%%%%%%%%%%%%%%%%%%%%%%%%%%%%%%%%%%
\paragraph{v1.6:} 2018/01/17

\begin{itemize}
\item
application for development of include files
\item
corrections to manual
\end{itemize}

%%%%%%%%%%%%%%%%%%%%%%%%%%%%%%%%%%%%%%%%
\paragraph{v1.5:} 2017/05/21

\begin{itemize}
\item
more complete structuring introduced
\item
|\childdocof| introduced
\item
|\childdoc| renamed to |\childdocmain|
\item
|\childredirect| renamed to |\childdocforward| and |\childdocforwardprefix|
and functionality expanded
\end{itemize}

%%%%%%%%%%%%%%%%%%%%%%%%%%%%%%%%%%%%%%%%
\paragraph{v1.0:} 2017/04/27

\begin{itemize}
\item
manual and install package
\item
first version published on CTAN
\end{itemize}

%%%%%%%%%%%%%%%%%%%%%%%%%%%%%%%%%%%%%%%%
\paragraph{v0.6:} 2017/04/26

\begin{itemize}
\item
redirection mechanism added
\end{itemize}

%%%%%%%%%%%%%%%%%%%%%%%%%%%%%%%%%%%%%%%%
\paragraph{v0.5:} 2017/04/26

\begin{itemize}
\item
functionality in definition file
\end{itemize}


%%%%%%%%%%%%%%%%%%%%%%%%%%%%%%%%%%%%%%%%%%%%%%%%%%%%%%%%%%%%%%%%%%%%%%%%%%%%%%%%
%%%%%%%%%%%%%%%%%%%%%%%%%%%%%%%%%%%%%%%%%%%%%%%%%%%%%%%%%%%%%%%%%%%%%%%%%%%%%%%%
%%%%%%%%%%%%%%%%%%%%%%%%%%%%%%%%%%%%%%%%%%%%%%%%%%%%%%%%%%%%%%%%%%%%%%%%%%%%%%%%
\appendix

\settowidth\MacroIndent{\rmfamily\scriptsize 000\ }

 \DocInput{childdoc.dtx}

\end{document}
%</driver>
% \fi
%
% %%%%%%%%%%%%%%%%%%%%%%%%%%%%%%%%%%%%%%%%%%%%%%%%%%%%%%%%%%%%%%%%%%%%%%%%%%%%%%
% %%%%%%%%%%%%%%%%%%%%%%%%%%%%%%%%%%%%%%%%%%%%%%%%%%%%%%%%%%%%%%%%%%%%%%%%%%%%%%
% \section{Sample}
%\iffalse
%<*samplemain>
%\fi
%
% The following presents a sample document
% with two chapters, two parts, a title page,
% a compile flag as well as three forwarding files to set the flag.
% It consists of eight |.tex| files:
% \begin{center}
% \begin{tabular}{ll}
% |cdocsamp.tex|&main file\\
% |cdocsch1.tex|&include file for chapter 1\\
% |cdocsch2.tex|&include file for chapter 2\\
% |cdocspt3.tex|&include file for part 3\\
% |cdocspt4.tex|&include file for part 4\\
% |cdocsdrf.tex|&forwarding file for main file in draft mode\\
% |cdocsfi1.tex|&forwarding file for final version of chapter 1\\
% |cdocsfi2.tex|&forwarding file for final version of chapter 2\\
% \end{tabular}
% \end{center}
% Each of the eight files can be compiled directly by the \LaTeX{} compiler.
%
% %%%%%%%%%%%%%%%%%%%%%%%%%%%%%%%%%%%%%%
% \paragraph{Main File.}
%
% The main file is called |cdocsamp.tex|.
%
% Load the \textsf{childdoc} definitions and
% declare the filename for the main document:
%    \begin{macrocode}
\input{childdoc.def}
\childdocmain{}
%    \end{macrocode}

% Optional override for |\version| flag:
%    \begin{macrocode}
%%\ifchilddoc\else\providecommand{\version}{draft}\fi
%    \end{macrocode}

% Define the default values for the |\version| flag
% (|final| for the main file and |draft| for childs):
%    \begin{macrocode}
\ifchilddoc
\providecommand{\version}{draft}
\else
\providecommand{\version}{final}
\fi
%    \end{macrocode}

% Load the standard document class:
%    \begin{macrocode}
\documentclass[12pt]{article}
%    \end{macrocode}

% Start the document body:
%    \begin{macrocode}
\begin{document}
%    \end{macrocode}

% Declare a title page.
% Print title, part of document being processed and version flag:
%    \begin{macrocode}
\addtocounter{page}{-1}
\begin{center}
{\LARGE\bfseries{}childdoc example\par}
\vspace{1cm}
\ifchilddoc
\ifchilddocmanual part\else chapter\fi:
`\childdocname' of `\childdocjob'\par
\else
main document: `\childdocjob'\par
\fi
version: \version\par
\end{center}
\newpage
%    \end{macrocode}

% Manually include selected file,
% otherwise process as usual:
%    \begin{macrocode}
\ifchilddocmanual
\section*{part `\childdocname'}
\input{\childdocname}
\else
%    \end{macrocode}

% Include the two chapters:
%    \begin{macrocode}
\include{cdocsch1}
\include{cdocsch2}
%    \end{macrocode}

% Include the two parts unless only chapters should be displayed:
%    \begin{macrocode}
\ifchilddoc\else
\section{part three}
\input{cdocspt3}
\section{part four}
\input{cdocspt4}
\fi
%    \end{macrocode}

% Process as usual until here:
%    \begin{macrocode}
\fi
%    \end{macrocode}

% End of document body:
%    \begin{macrocode}
\end{document}
%    \end{macrocode}
%\iffalse
%</samplemain>
%\fi
%
% %%%%%%%%%%%%%%%%%%%%%%%%%%%%%%%%%%%%%%
% \paragraph{Chapter Include Files.}
%
% The include files are called |cdocsch1.tex| and |cdocsch2.tex|.
%
%\iffalse
%<*samplechap1|samplechap2>
%\fi

% Optional override for |\version| flag:
%    \begin{macrocode}
%%\providecommand{\version}{final}
%    \end{macrocode}

% Include the main document:
%    \begin{macrocode}
\input{childdoc.def}
\childdocof{cdocsamp}
%    \end{macrocode}

%\iffalse
%</samplechap1|samplechap2>
%\fi
%
%\iffalse
%<*samplechap1>
%\fi
% Some text for chapter 1:
%    \begin{macrocode}
\section{one}
some text in chapter one
%    \end{macrocode}

%\iffalse
%</samplechap1>
%\fi
% Some text for chapter 2:
%\iffalse
%<*samplechap2>
%\fi
%    \begin{macrocode}
\section{two}
more text in chapter two
%    \end{macrocode}

%\iffalse
%</samplechap2>
%\fi
%
% %%%%%%%%%%%%%%%%%%%%%%%%%%%%%%%%%%%%%%
% \paragraph{Part Include Files.}
%
% The include files are called |cdocspt3.tex| and |cdocspt4.tex|.
%
%\iffalse
%<*samplepart3|samplepart4>
%\fi

% Optional override for |\version| flag:
%    \begin{macrocode}
%%\providecommand{\version}{final}
%    \end{macrocode}

% Include the main document:
%    \begin{macrocode}
\input{childdoc.def}
\childdocby{cdocsamp}
%    \end{macrocode}

%\iffalse
%</samplepart3|samplepart4>
%\fi
%
%\iffalse
%<*samplepart3>
%\fi
% Some text for part 3:
%    \begin{macrocode}
some text in part three
%    \end{macrocode}

%\iffalse
%</samplepart3>
%\fi
% Some text for part 4:
%\iffalse
%<*samplepart4>
%\fi
%    \begin{macrocode}
more text in part four
%    \end{macrocode}

%\iffalse
%</samplepart4>
%\fi
%
% %%%%%%%%%%%%%%%%%%%%%%%%%%%%%%%%%%%%%%
% \paragraph{Forwarding for a Complete Draft.}
%
% The following forwarding file |cdocsdrf.tex|
% compiles the main document in draft mode:
%\iffalse
%<*sampledraft>
%\fi
%    \begin{macrocode}
\def\version{draft}
\input{childdoc.def}
\childdocforward{cdocsamp}
%    \end{macrocode}

%\iffalse
%</sampledraft>
%\fi
%
% %%%%%%%%%%%%%%%%%%%%%%%%%%%%%%%%%%%%%%
% \paragraph{Forwarding for Final Version of the Chapters.}
%
% The following forwarding files |cdocsfn1.tex| and |cdocsfn2.tex|
% (with identical content)
% compile the final versions of the child documents
% |cdocsch1.tex| and |cdocsch2.tex|, respectively:
%\iffalse
%<*samplefinal>
%\fi
%    \begin{macrocode}
\def\version{final}
\input{childdoc.def}
\childdocforwardprefix[cdocsamp]{cdocsfn}{cdocsch}
%    \end{macrocode}

%\iffalse
%</samplefinal>
%\fi
%
% %%%%%%%%%%%%%%%%%%%%%%%%%%%%%%%%%%%%%%
% \paragraph{Command Line Processing.}
%
% The following three command lines generate the output files
% |cdocscld|, |cdocscl1| and |cdocscl2|
% which should be identical to
% |cdocsdrf|, |cdocsch1| and |cdocsfn2|, respectively:
% \begin{center}
% \begin{tabular}{l}
% |latex -jobname cdocscld \|\\
% |  "\def\version{draft}\input{childdoc.def}\childdocforward{cdocsamp}"|\\
% |latex -jobname cdocscl1 \|\\
% |  "\input{childdoc.def}\childdocforward[cdocsamp]{cdocsch1}"|\\
% |latex -jobname cdocscl2 \|\\
% |  "\def\version{final}\input{childdoc.def}\childdocforward{cdocsch2}"|
% \end{tabular}
% \end{center}
% Note that the trailing backslash on each first line
% merely continues the input to the second line
% (for convenient cut ant paste).
% Furthermore, the command |latex| can be replaced by any
% of its alternative versions such as |pdflatex|.
%
% %%%%%%%%%%%%%%%%%%%%%%%%%%%%%%%%%%%%%%%%%%%%%%%%%%%%%%%%%%%%%%%%%%%%%%%%%%%%%%
% %%%%%%%%%%%%%%%%%%%%%%%%%%%%%%%%%%%%%%%%%%%%%%%%%%%%%%%%%%%%%%%%%%%%%%%%%%%%%%
% \section{Implementation}
%\iffalse
%<*package>
%\fi
%
% This section describes the definitions file |childdoc.def|.

% The definitions cannot be loaded using |\usepackage| or |\RequirePackage|
% which has a mechanism to prevent loading a style file more than once.
% When loading the definitions by means of |\input|
% multiple instances have to be prevented manually:
%\iffalse
%This code needs to be before the `\ProvidesFile' directive
%which is defined at the beginning of this file.
%Therefore it is also placed there and commented out here.
%</package>
%<*discard>
%\fi
%    \begin{macrocode}
\ifdefined\childdocmain\endinput\fi
%    \end{macrocode}
%\iffalse
%</discard>
%<*package>
%\fi
%
% \macro{\ifchilddoc}
% \macro{\ifchilddocmanual}
% The conditional |\ifchilddoc| tells whether a
% child (true) or main (false) document is being compiled.
% The conditional |\ifchilddocmanual| tells whether
% the |\includeonly| mechanism is used (false) or
% the selection of child files must be performed manually (true).
% The definitions initialise to false:
%    \begin{macrocode}
\newif\ifchilddoc
\newif\ifchilddocmanual
%    \end{macrocode}

% \macro{\childdocname}
% \macro{\childdocjob}
% The macro |\childdocname| stores the name of the main document
% to be compiled. The macro |\childdocjob| stores the name of
% the document on which the \LaTeX{} compiler was originally invoked.
% The content of |\jobname| cannot be compared
% to filenames specified in the source due to different catcodes.
% The following code rescans |\jobname|, stores the result
% in |\childdocname| and saves a copy in |\childdocjob|:
%    \begin{macrocode}
\edef\childdocname{\scantokens\expandafter{\jobname\noexpand}}
\let\childdocjob\childdocname
%    \end{macrocode}

% \macro{\childdocdisable}
% The macro |\childdocdisable| prevents the main file
% from being processed more than once.
% At this stage, the main document command |\childdocmain|
% is assumed to be called once again where it should do nothing.
% Any subsequent call to it should prevent
% a secondary processing of the main document
% It overwrites the forwarding commands
% |\childdocof| and |\childdocforward|
% with empty macros to prevent further inclusions of the main document:
%    \begin{macrocode}
\newcommand{\childdocdisable}
{
  \renewcommand{\childdocmain}[1]{\renewcommand{\childdocmain}[1]{\endinput}}
  \renewcommand{\childdocof}[1]{}
  \renewcommand{\childdocby}[2][]{}
  \renewcommand{\childdocforward}[2][]{}
  \renewcommand{\childdocdisable}{}
}
%    \end{macrocode}

% \macro{\childdocmain}
% The macro |\childdocmain| is to be called at the top of the main file
% with nothing or the main filename (without extension) as argument.
% First, it breaks loops.
% If the argument is not empty and does not match |\childdocname|
% (which is set by the first inclusion of |childdoc.def|),
% |\ifchilddoc| is set to true, |\includeonly| is applied to the child file
% and |\jobname| is set to the main file
% (for proper handling of |.aux| files):
%    \begin{macrocode}
\newcommand{\childdocmain}[1]
{
  \childdocdisable\childdocmain{}
  \if?#1?\else
    \begingroup
      \def\childdoctmp{#1}
      \ifx\childdoctmp\childdocname
        \def\childdoctmp{}
      \else
        \def\childdoctmp
        {
          \childdoctrue
          \includeonly{\childdocname}
          \def\childdocjob{#1}
          \def\jobname{#1}
        }
      \fi
      \expandafter
    \endgroup
    \childdoctmp
  \fi
}
%    \end{macrocode}

% \macro{\childdocof}
% The command |\childdocof| redirects
% compilation to the main file |#1|.
%    \begin{macrocode}
\newcommand{\childdocof}[1]
{
  \childdocdisable
  \childdoctrue
  \includeonly{\childdocname}
  \def\jobname{#1}
  \def\childdocjob{#1}
  \input{#1}
}
%    \end{macrocode}

% \macro{\childdocby}
% The command |\childdocby| ....
%    \begin{macrocode}
\newcommand{\childdocby}[2][]
{
  \childdocdisable
  \childdoctrue
  \childdocmanualtrue
  \if?#1?\else
    \def\jobname{#2}
  \fi
  \def\childdocjob{#2}
  \input{#2}
  \endinput
}
%    \end{macrocode}

% \macro{\childdocforward}
% The command |\childdocforward| redirects
% compilation to the main file or
% (if the optional argument is given) a child file.
% Parameters are set as if the main file
% or a child file starting with |\childdocof| was compiled.
% Then compilation is handed over to the main file:
%    \begin{macrocode}
\newcommand{\childdocforward}[2][]
{
  \begingroup
    \if?#1?
      \def\childdoctmp
      {
        \def\childdocname{#2}
        \def\childdocjob{#2}
        \def\jobname{#2}
        \input{#2}
        \endinput
      }
    \else
      \def\childdoctmp
      {
        \childdocdisable
        \def\childdocname{#2}
        \childdoctrue
        \includeonly{#2}
        \def\childdocjob{#1}
        \def\jobname{#1}
        \input{#1}
        \endinput
      }
    \fi
    \expandafter
  \endgroup
  \childdoctmp
}
%    \end{macrocode}

% \macro{\childdocforwardprefix}
% The command |\childdocforwardprefix| redirects
% compilation to the main or a child file by means of a pattern.
% The prefix |#1| in the current filename is replaced by |#2|
% and the suffix of the current filename is kept
% (it is assumed that the filename does not contain the substring `|~~~|'
% which is used as a delimiter).
% Compilation is handed over to the new file by |\childdocforward|:
%    \begin{macrocode}
\newcommand{\childdocforwardprefix}[3][]
{
  \begingroup
    \def\childdocextract #2##1~~~{\def\childdoctmp{\childdocforward[#1]{#3##1}}}
    \expandafter\childdocextract\childdocname~~~
    \expandafter
  \endgroup
  \childdoctmp
}
%    \end{macrocode}

% \macro{\childdoc}
% The deprecated macro |\childdoc| is a legacy version of |\childdocmain|:
%    \begin{macrocode}
\newcommand{\childdoc}{\childdocmain}
%    \end{macrocode}

% \macro{\childdocredirect}
% The deprecated macro |\childdocredirect| is a legacy version
% of |\childdocforward| and |\childdocforwardprefix|:
%    \begin{macrocode}
\newcommand{\childdocredirect}[2][]
{
  \begingroup
    \if?#1?
      \def\childdoctmp{\childdocforward{#2}}
    \else
      \def\childdoctmp{\childdocforwardprefix{#1}{#2}}
    \fi
    \expandafter
  \endgroup
  \childdoctmp
}
%    \end{macrocode}

%\iffalse
%</package>
%\fi
%
\endinput
\childdocforward[|\textit{main}|]{|\textit{dest}|}"|
\end{center}
%
Here \textit{target} is the name of the output file,
\textit{main} is the name of the main file
and \textit{dest} is the name of the main or child file to be processed
(all filenames without extensions).
The optional argument \textit{main} can be omitted
if \textit{main} matches \textit{dest}.
Optionally, compilation \textit{flags} can be defined via |\def| commands.
This command line makes the \TeX{} engine believe
it is compiling the file \textit{target}
whose content is specified as the latter parameter.
The provided code then forwards the processing to
\textit{main} or \textit{dest} as described in \secref{sec:forward}.

%%%%%%%%%%%%%%%%%%%%%%%%%%%%%%%%%%%%%%%%%%%%%%%%%%%%%%%%%%%%%%%%%%%%%%%%%%%%%%%%
\subsection{Include by Input}
\label{sec:input}

Including child documents by |\include| has some restrictions by design.
Most notably, the content of a child document always occupies
its own set of pages; pages cannot be shared between child documents.
Usually, this behaviour makes perfect sense
because each child document contain an essential part of the document.
However, in some situations it may be desirable to compose
a document from a collection of parts
without having mandatory page breaks between then.
For this case, the package
provides a mechanism to include parts
by |\input| which can also be processed individually.
However, by construction this mechanism
requires manual handling of the content to be output.

%%%%%%%%%%%%%%%%%%%%%%%%%%%%%%%%%%%%%%%%
\DescribeMacro{\ifchilddocmanual}
The main file should be prepared as usual, see \secref{sec:include}.
However, the document body must make a distinction
between processing of an individual part and of the main document, e.g.:
%
\begin{center}
\begin{tabular}{l}
|\ifchilddocmanual|\\
|\input{\childdocname}|\\
|\||else|\\
\textit{document body with }|\input{|\textit{part}|}|\\
|\||fi|
\end{tabular}
\end{center}
%
The conditional |\ifchilddocmanual| is true whenever
a part to be included by |\input| is being compiled,
and the name of the part is stored in |\childdocname|.

%%%%%%%%%%%%%%%%%%%%%%%%%%%%%%%%%%%%%%%%
\DescribeMacro{\childdocby}
Each part to be included by |\input| should start with:
%
\begin{center}
\begin{tabular}{l}
|% \iffalse
%
% childdoc.dtx Copyright (C) 2017-2018 Niklas Beisert
%
% This work may be distributed and/or modified under the
% conditions of the LaTeX Project Public License, either version 1.3
% of this license or (at your option) any later version.
% The latest version of this license is in
%   http://www.latex-project.org/lppl.txt
% and version 1.3 or later is part of all distributions of LaTeX
% version 2005/12/01 or later.
%
% This work has the LPPL maintenance status `maintained'.
%
% The Current Maintainer of this work is Niklas Beisert.
%
% This work consists of the files childdoc.dtx and childdoc.ins
% and the derived files childdoc.def and cdocsamp.tex with
% cdocsch1.tex, cdocsch2.tex, cdocsdrf.tex, cdocsfn1.tex, cdocsfn2.tex.
%
%<package>\ifdefined\childdocmain\endinput\fi
%<package>\ProvidesFile{childdoc.def}[2018/12/30 v2.0 child document driver]
%<samplemain>\ProvidesFile{cdocsamp.tex}[2018/12/30 v2.0 sample for childdoc]
%<*driver>
%\ProvidesFile{childdoc.drv}[2018/12/30 v2.0 childdoc reference manual file]
\PassOptionsToClass{10pt,a4paper}{article}
\documentclass{ltxdoc}

\usepackage[margin=35mm]{geometry}
\usepackage{hyperref}
\usepackage{hyperxmp}
\usepackage[usenames]{color}

\hypersetup{colorlinks=true}
\hypersetup{pdfstartview=FitH}
\hypersetup{pdfpagemode=UseNone}
\hypersetup{pdfsource={}}
\hypersetup{pdflang={en-UK}}
\hypersetup{pdfcopyright={Copyright 2017-2018 Niklas Beisert.
  This work may be distributed and/or modified under the
  conditions of the LaTeX Project Public License, either version 1.3
  of this license or (at your option) any later version.}}
\hypersetup{pdflicenseurl={http://www.latex-project.org/lppl.txt}}
\hypersetup{pdfcontactaddress={ETH Zurich, ITP, HIT K,
  Wolfgang-Pauli-Strasse 27}}
\hypersetup{pdfcontactpostcode={8093}}
\hypersetup{pdfcontactcity={Zurich}}
\hypersetup{pdfcontactcountry={Switzerland}}
\hypersetup{pdfcontactemail={nbeisert@itp.phys.ethz.ch}}
\hypersetup{pdfcontacturl={http://people.phys.ethz.ch/\xmptilde nbeisert/}}

\newcommand{\secref}[1]{\hyperref[#1]{section \ref*{#1}}}

\parskip1ex
\parindent0pt
\let\olditemize\itemize
\def\itemize{\olditemize\parskip0pt}

\begin{document}

\title{The \textsf{childdoc} Package}
\hypersetup{pdftitle={The childdoc Package}}
\author{Niklas Beisert\\[2ex]
  Institut f\"ur Theoretische Physik\\
  Eidgen\"ossische Technische Hochschule Z\"urich\\
  Wolfgang-Pauli-Strasse 27, 8093 Z\"urich, Switzerland\\[1ex]
  \href{mailto:nbeisert@itp.phys.ethz.ch}
  {\texttt{nbeisert@itp.phys.ethz.ch}}}
\hypersetup{pdfauthor={Niklas Beisert}}
\hypersetup{pdfsubject={Manual for the LaTeX2e Package childdoc}}
\date{30 December 2018, \textsf{v2.0}}
\maketitle

\begin{abstract}\noindent
\textsf{childdoc} is a \LaTeXe{} package
that enables the direct compilation
of document sections included by |\include|
to individual files.
\end{abstract}

\begingroup
\parskip0ex
\tableofcontents
\endgroup

%%%%%%%%%%%%%%%%%%%%%%%%%%%%%%%%%%%%%%%%%%%%%%%%%%%%%%%%%%%%%%%%%%%%%%%%%%%%%%%%
%%%%%%%%%%%%%%%%%%%%%%%%%%%%%%%%%%%%%%%%%%%%%%%%%%%%%%%%%%%%%%%%%%%%%%%%%%%%%%%%
\section{Introduction}

\LaTeX{} provides a mechanism to structure a large document (such as a book)
into a main file and several child files (containing the chapters)
using the |\include| command.
This mechanism is beneficial for documents
which span hundreds of pages in order to
make the source file(s) more manageable.
Moreover, compilation can be restricted to
selected child files by means of the |\includeonly| command.
The latter feature can be used to reduce the compilation time while editing
(this was significantly more useful in the earlier days of \LaTeX{})
or to generate a smaller document which is easier to navigate.
Another application of |\includeonly| is to generate
documents consisting of selected parts of the complete document.

However, there are a few drawbacks of the plain |\include| mechanism:
\begin{itemize}
\item
The child files cannot be compiled on their own,
they can only be compiled via the main file.
A naive editing environment
(such as a text editor with an option
to have the current file processed by \LaTeX)
may require one to switch to the main file before compiling;
attempting to compile the child file produces errors.
\item
The main file must be modified (each time)
to adjust the |\includeonly| command
to the present needs. This easily leaves the main file in a messy state.
\item
The generated document will always carry the filename
of the main document. This is inconvenient if
several child files are to be compiled and
to be kept for distribution.
\end{itemize}

The present package provides a simple interface
to make child files individually compilable by \LaTeX{}.
Compiling a child file then has the same effect as compiling
the main file with an |\includeonly| command
to select the appropriate child.
Moreover the generated document will carry the name of the child
rather than the main file.
This resolves all three above issues.

This feature is meant to make the editing of books,
thesis documents and lecture notes somewhat more convenient.
However, the package can also be used efficiently for
composing a series of documents (such as exercise sheets)
which are typically distributed individually.
It then assists the author in generating the individual documents
(potentially in different versions)
as well as a document containing the collected series.
Another application is in developing style files
or other kinds of included material
where compilation of the style file could redirect
to a sample or test file.

%%%%%%%%%%%%%%%%%%%%%%%%%%%%%%%%%%%%%%%%%%%%%%%%%%%%%%%%%%%%%%%%%%%%%%%%%%%%%%%%
%%%%%%%%%%%%%%%%%%%%%%%%%%%%%%%%%%%%%%%%%%%%%%%%%%%%%%%%%%%%%%%%%%%%%%%%%%%%%%%%
\section{Usage}

First of all, the package \textsf{childdoc} is \emph{not} a standard
\LaTeXe{} |.sty| style file! Therefore it needs to be invoked in
a non-standard way.

%%%%%%%%%%%%%%%%%%%%%%%%%%%%%%%%%%%%%%%%%%%%%%%%%%%%%%%%%%%%%%%%%%%%%%%%%%%%%%%%
\subsection{Included Files}
\label{sec:include}

%%%%%%%%%%%%%%%%%%%%%%%%%%%%%%%%%%%%%%%%
\DescribeMacro{\childdocmain}
To use the package, add the commands
\begin{center}
\begin{tabular}{l}
|\input{childdoc.def}|\\
|\childdocmain{}|\\
\end{tabular}
\end{center}
at the very top of the main \LaTeX{} file,
in particular \emph{before} the |\documentclass| statement!
The argument of |\childdocmain| should be left empty
(but it must be present).

%%%%%%%%%%%%%%%%%%%%%%%%%%%%%%%%%%%%%%%%
\DescribeMacro{\childdocof}
Furthermore, add the commands
\begin{center}
\begin{tabular}{l}
|\input{childdoc.def}|\\
|\childdocof{|\textit{main}|}|\\
\end{tabular}
\end{center}
at the top of every child file \textit{child}
which is included by |\include{|\textit{child}|}|
from within the main file
(or at least for those files to be compiled individually).
The argument \textit{main} must be the filename of the main file.

There are a couple of
considerations in setting up the main and child documents:

%%%%%%%%%%%%%%%%%%%%%%%%%%%%%%%%%%%%%%%%
\paragraph{Restrictions.}

Please note the following restrictions:
\begin{itemize}
\item
|\childdocmain| must be called with one argument \textit{main}
to ensure compatibility with earlier version of the package.
It must either be empty (|\childdocmain{}|)
or precisely match the filename of the main file in which it is specified.
See \secref{sec:detection} for further information.
\item
The filename \textit{main} must be specified without the |.tex| extension.
\item
The filename \textit{main} is case sensitive
(even in case-insensitive file systems)
due to internal string comparison.
\item
The argument \textit{main} should be fully expanded, it cannot be a macro.
\item
Subdirectories and special characters should be avoided in filenames.
\item
The command |\childdocmain{|\textit{main}|}| must be followed by a whitespace.
It should not be followed immediately by another command
or by a comment mark `|%|'.
This is because the \TeX{} parser reads the token immediately following
the argument of |\childdocmain| and puts it
at the beginning of every child section;
however, a white\-space is ignored.
\end{itemize}

%%%%%%%%%%%%%%%%%%%%%%%%%%%%%%%%%%%%%%%%
\paragraph{Content of Main File.}

It is advisable to place all content in the child files included by |\include|.
Any output contained in the main file will appear in all child documents
unless suppressed manually;
it cannot be suppressed automatically by the |\includeonly| directive
and thus should normally be avoided.
A method to include some content in the main file
by means of conditional processing is described in \secref{sec:conditional}.

%%%%%%%%%%%%%%%%%%%%%%%%%%%%%%%%%%%%%%%%
\paragraph{Page Numbering.}

When only a part of the document is compiled,
the appropriate numbering of pages
(as well as other status parameters)
is determined from the |.aux| files.
The latter contain information from previous passes.
However this information needs to propagate through
all intermediate child documents.
Therefore the page numbering in child documents may well
be inconsistent until the complete document is compiled at least once.

A useful (if unconventional) way to always ensure a consistent
page numbering is to restart the numbering in each child document
and denote the pages by `\textit{child}|.|\textit{page}'
where \textit{child} represents the chapter/section number of the child file.
This can be achieved by the command
|\numberwithin{page}{|\textit{child}|}|
of the \textsf{amsmath} package
where \textit{child} can be |chapter| or |section|
depending on the chosen structuring.
Alternatively, one can modify the macro |\thepage| appropriately
and reset the counter |page| at the start of each child file.

%%%%%%%%%%%%%%%%%%%%%%%%%%%%%%%%%%%%%%%%%%%%%%%%%%%%%%%%%%%%%%%%%%%%%%%%%%%%%%%%
\subsection{Conditional Processing}
\label{sec:conditional}

The package provides a mechanism to compile different versions
of a document. To customise the versions further some conditional processing
can come in handy to distinguish which version is being compiled.
The package provides two macros to describe the compilation context:

%%%%%%%%%%%%%%%%%%%%%%%%%%%%%%%%%%%%%%%%
\DescribeMacro{\ifchilddoc}
The conditional |\ifchilddoc| distinguishes between the compilation of
child documents and the main document:
%
\begin{center}
|\ifchilddoc |\textit{child-code}| |[|\||else |\textit{main-code}]| \||fi|
\end{center}

%%%%%%%%%%%%%%%%%%%%%%%%%%%%%%%%%%%%%%%%
\DescribeMacro{\childdocname}
\DescribeMacro{\childdocjob}
The macro |\childdocname| contains the filename (without extension)
of the main or child file being processed.
Note that |\childdocjob| will always contain the name of the main file.

%%%%%%%%%%%%%%%%%%%%%%%%%%%%%%%%%%%%%%%%
\paragraph{Title Page.}

Conditional processing can be used to include a title or banner page
in the main document when proper precautions are taken.
Importantly, the code in the main file should ensure that the page counter
(as well as other status parameters which are stored in the |.aux| files)
takes the same value after the conditional processing.
Otherwise the page numbers may take divergent values
depending on which part is compiled.

For example, a title page could be declared by:
%
\begin{center}
\begin{tabular}{l}
|\ifchilddoc\||else|\\
|\addtocounter{page}{-1}|\\
\textit{code for title page}\\
|\newpage|\\
|\||fi|
\end{tabular}
\end{center}
%
A banner page for the child documents can be generated by:
%
\begin{center}
\begin{tabular}{l}
|\ifchilddoc|\\
|\addtocounter{page}{-1}|\\
\textit{code for banner page}\\
|\newpage|\\
|\||fi|
\end{tabular}
\end{center}
%
Here one could write a message such as:
\begin{center}
|This is the part \childdocname{} of \childdocjob{}.|
\end{center}

%%%%%%%%%%%%%%%%%%%%%%%%%%%%%%%%%%%%%%%%%%%%%%%%%%%%%%%%%%%%%%%%%%%%%%%%%%%%%%%%
\subsection{Flags}
\label{sec:flags}

The package makes it easy to generate different versions
of the main or child documents.
To this end compilation flags can be defined
and assigned different default values.
They will be particularly useful in conjunction
with the forwarding mechanism described in \secref{sec:forward}.

For example, it may be useful to have a flag |\version|
which can be set to |draft| or |final|.
The document source will contain some conditional code
depending on the value of |\version|.
Suppose further, the flag should default to |final| for the main file
and to |draft| for child files
which is a natural assignment for editing the document.
This is achieved by placing the following code
in the preamble of the main document
(below the |\childdocmain| directive):
%
\begin{center}
\begin{tabular}{l}
|\ifchilddoc|\\
|\providecommand{\version}{draft}|\\
|\||else|\\
|\providecommand{\version}{final}|\\
|\||fi|
\end{tabular}
\end{center}
%
The definition by |\providecommand| makes sure
that previous definitions are not overwritten.
Further statements |\providecommand{\version}{...}|
can thus be added before the above code to override it.

For the main file, one might add a line
(between |\childdocmain| and the above block)
%
\begin{center}
|%\ifchilddoc\||else\providecommand{\version}{draft}\||fi|
\end{center}
%
which can be uncommented to produce a draft version.
Likewise one can add a line to the very top of a child file
(above the |\childdocof{|\textit{main}|}| directive)
%
\begin{center}
|%\providecommand{\version}{final}|
\end{center}
%
which can be uncommented to produce the final version of this child document.

%%%%%%%%%%%%%%%%%%%%%%%%%%%%%%%%%%%%%%%%%%%%%%%%%%%%%%%%%%%%%%%%%%%%%%%%%%%%%%%%
\subsection{Forwarding}
\label{sec:forward}

Different versions of the main or child documents
using compilation flags as described in \secref{sec:flags}
can be (permanently) stored in different files
for convenient compilation, viewing and distribution.
To this end, the package defines a command
to pass on compilation to a different file:

%%%%%%%%%%%%%%%%%%%%%%%%%%%%%%%%%%%%%%%%
\DescribeMacro{\childdocforward}
The command |\childdocforward| redirects processing to
another source file:
%
\begin{center}
\begin{tabular}{l}
|\input{childdoc.def}|\\
|\childdocforward[|\textit{main}|]{|\textit{dest}|}|\\
\end{tabular}
\end{center}
%
The argument \textit{dest} is the destination file
(without extension).
It should be the main file or one of the child files.
Note that further \textsf{childdoc} directives
such as |\childdocof| and |\childdocforward|
in the indicated file will be processed in this form.
The optional argument \textit{main}
passes on directly to the main file \textit{main}
while pretending to compile the child \textit{dest}.
This form behaves as if \textit{dest}
issues |\childdocof{|\textit{main}|}| right away,
and no further \textsf{childdoc} directives will be processed.

%%%%%%%%%%%%%%%%%%%%%%%%%%%%%%%%%%%%%%%%
\DescribeMacro{\...prefix}
In the alternative form |\childdocforwardprefix|,
%
\begin{center}
\begin{tabular}{l}
|\input{childdoc.def}|\\
|\childdocforwardprefix[|\textit{main}|]{|\textit{prefix}|}{|\textit{dest}|}|
\end{tabular}
\end{center}
%
the destination file is determined by a pattern
depending on the current file:
To make this work, the current file must be called
`{\textit{prefix}\hspace{0.2em}\textit{suffix}}'
with \textit{prefix} matching precisely the argument.
Processing is then passed on to the file
`{\textit{dest}\hspace{0.2em}\textit{suffix}}'.
Surely, the same effect is achieved by
directly specifying the
argument `{\textit{dest}\hspace{0.2em}\textit{suffix}}'
in the first form.
However, that requires to set up a different file
for each child. With the alternative form of the command
all these files can have exactly the same content
which simplifies setting them up and maintaining them.

For example, the following file |draft.tex|
with a compilation flag |\version| as described in \secref{sec:flags}
compiles the main document as a draft:
%
\begin{center}
\begin{tabular}{l}
|\def\version{draft}|\\
|\input{childdoc.def}|\\
|\childdocforward{|\textit{main}|}|
\end{tabular}
\end{center}
%
Likewise, the following files |final|\textit{nn}|.tex|
compile the final version of the child document
|child|\textit{nn}|.tex|:
%
\begin{center}
\begin{tabular}{l}
|\def\version{final}|\\
|\input{childdoc.def}|\\
|\childdocforwardprefix{final}{child}|
\end{tabular}
\end{center}
%

Note that when several versions of a main file and/or of each child file
are to be generated, it may be convenient to set up a |Makefile| or
shell script to automatise the process.

%%%%%%%%%%%%%%%%%%%%%%%%%%%%%%%%%%%%%%%%%%%%%%%%%%%%%%%%%%%%%%%%%%%%%%%%%%%%%%%%
\subsection{Command Line Processing}
\label{sec:commandline}

The effect of redirection files can also be achieved by invoking
the \LaTeX{} compiler with a more elaborate command line.
Most conveniently this should be done as part
of a shell script or a |Makefile|.

When using \textsf{childdoc} in the main file, the following
command lines effectively perform a redirection
(note that depending on the shell being used,
backslashes may have to be doubled: `|\|' $\to$ `|\\|'):
%
\begin{center}
|... -jobname "|\textit{target}|" |\\|"|[\textit{flags}]%
|\input{childdoc.def}\childdocforward[|\textit{main}|]{|\textit{dest}|}"|
\end{center}
%
Here \textit{target} is the name of the output file,
\textit{main} is the name of the main file
and \textit{dest} is the name of the main or child file to be processed
(all filenames without extensions).
The optional argument \textit{main} can be omitted
if \textit{main} matches \textit{dest}.
Optionally, compilation \textit{flags} can be defined via |\def| commands.
This command line makes the \TeX{} engine believe
it is compiling the file \textit{target}
whose content is specified as the latter parameter.
The provided code then forwards the processing to
\textit{main} or \textit{dest} as described in \secref{sec:forward}.

%%%%%%%%%%%%%%%%%%%%%%%%%%%%%%%%%%%%%%%%%%%%%%%%%%%%%%%%%%%%%%%%%%%%%%%%%%%%%%%%
\subsection{Include by Input}
\label{sec:input}

Including child documents by |\include| has some restrictions by design.
Most notably, the content of a child document always occupies
its own set of pages; pages cannot be shared between child documents.
Usually, this behaviour makes perfect sense
because each child document contain an essential part of the document.
However, in some situations it may be desirable to compose
a document from a collection of parts
without having mandatory page breaks between then.
For this case, the package
provides a mechanism to include parts
by |\input| which can also be processed individually.
However, by construction this mechanism
requires manual handling of the content to be output.

%%%%%%%%%%%%%%%%%%%%%%%%%%%%%%%%%%%%%%%%
\DescribeMacro{\ifchilddocmanual}
The main file should be prepared as usual, see \secref{sec:include}.
However, the document body must make a distinction
between processing of an individual part and of the main document, e.g.:
%
\begin{center}
\begin{tabular}{l}
|\ifchilddocmanual|\\
|\input{\childdocname}|\\
|\||else|\\
\textit{document body with }|\input{|\textit{part}|}|\\
|\||fi|
\end{tabular}
\end{center}
%
The conditional |\ifchilddocmanual| is true whenever
a part to be included by |\input| is being compiled,
and the name of the part is stored in |\childdocname|.

%%%%%%%%%%%%%%%%%%%%%%%%%%%%%%%%%%%%%%%%
\DescribeMacro{\childdocby}
Each part to be included by |\input| should start with:
%
\begin{center}
\begin{tabular}{l}
|\input{childdoc.def}|\\
|\childdocby{|\textit{main}|}|\\
\end{tabular}
\end{center}
%
The directive |\childdocby| is similar to |\childdocof|
described in \secref{sec:include},
but the subsequent selection of content must be done manually.
To that end, both |\ifchilddoc| and |\ifchilddocmanual|
will be true upon processing of a part,
and the name of the part is stored in |\childdocname|.
Note that |\jobname| will be set to the filename of the current part
so that each part receives an individual |.aux| file
that does not interfere with the |.aux| file(s) of the main document.
This behaviour can be altered by the alternative form
|\childdocby[*]{|\textit{main}|}| (with a non-empty optional argument)
which uses the |.aux| file of the main document
by setting |\jobname| to \textit{main}.

%%%%%%%%%%%%%%%%%%%%%%%%%%%%%%%%%%%%%%%%%%%%%%%%%%%%%%%%%%%%%%%%%%%%%%%%%%%%%%%%
\subsection{Driver Development}
\label{sec:driver}

The \textsf{childdoc} mechanism can also be use for the development
of definition files such as \LaTeX{} styles or classes.
This case differs from the above setup with multiple parts
included by |\include| in that no |\includeonly| should be invoked.
This can be achieved by starting the include file
(before |\ProvidesPackage|) with:
%
\begin{center}
\begin{tabular}{l}
|\input{childdoc.def}|\\
|\childdocforward{|\textit{main}|}|\\
\end{tabular}
\end{center}
%
or alternatively with:
%
\begin{center}
\begin{tabular}{l}
|\input{childdoc.def}|\\
|\childdocby{|\textit{main}|}|\\
\end{tabular}
\end{center}
%
Both forms have slightly different effects as described above.
The main file is prepared as usual, see \secref{sec:include}.

%%%%%%%%%%%%%%%%%%%%%%%%%%%%%%%%%%%%%%%%%%%%%%%%%%%%%%%%%%%%%%%%%%%%%%%%%%%%%%%%
\subsection{Legacy Detection}
\label{sec:detection}

The directive |\childdocmain| in the main file can detect
whether the complete document or merely a child is to be compiled
even without using the directive |\childdocof|.
This method is deprecated because it is less robust
and there is no compelling reason to use it;
it is merely provided for backward compatibility
and it may be removed in future versions.

If the detection mechanism is to be used,
it is mandatory to correctly specify
the filename of the main file as the argument of |\childdocmain|:
%
\begin{center}
\begin{tabular}{l}
|\input{childdoc.def}|\\
|\childdocmain{|\textit{main}|}|\\
\end{tabular}
\end{center}
%
If |\jobname| does not match the argument \textit{main} of |\childdocmain|,
it is assumed that |\jobname| points to the child file to be compiled.
When using |\childdocmain| with the main file specified as argument,
it suffices to start a child file
with just |\input{|\textit{main}|}|
without loading of the package and using |\childdocof|.
If instead all processing is done
with the appropriate \textsf{childdoc} directives,
the argument of \textit{main} of |\childdocmain| can be empty.

An alternative version of the command line processing described
in \secref{sec:commandline} using the detection mechanism reads:
%
\begin{center}
|... -jobname "|\textit{target}|" "|[\textit{flags}]%
[|\def\jobname{|\textit{dest}|}|]|\input{|\textit{main}|}"|
\end{center}

%%%%%%%%%%%%%%%%%%%%%%%%%%%%%%%%%%%%%%%%%%%%%%%%%%%%%%%%%%%%%%%%%%%%%%%%%%%%%%%%
\subsection{Manual Code}
\label{sec:manual}

In case one cannot be certain whether the definitions file |childdoc.def|
is installed on the target \TeX{} distribution
and one prefers not to ship it,
it is conceivable to paste a few relevant commands into the sources.

To that end, drop all statements |\input{childdoc.def}|
and perform the replacements as outlined below.
Instead of |\childdocmain{|\textit{main}|}| add the following code
to the top of the main file:
%
\begin{center}
\begin{tabular}{l}
|\||ifdefined\childdocname\endinput\||fi\newif\ifchilddoc|\\
|\edef\childdocname{\scantokens\expandafter{\jobname\noexpand}}|\\
|\def\childdocmain{|\textit{main}|}\||ifx\childdocmain\childdocname\||else|\\
|\childdoctrue\includeonly{\childdocname}\let\jobname\childdocmain\||fi|\\
\end{tabular}
\end{center}
%
Instead of |\childdocof{|\textit{main}|}| just include the main file
at the top of each child file:
%
\begin{center}
|\input{|\textit{main}|}|
\end{center}
%
A simple redirection |\childdocforward{|\textit{dest}|}| is achieved by:
%
\begin{center}
|\def\jobname{|\textit{dest}|}\input{\jobname}|
\end{center}
%
The redirection with prefix
|\childdocforwardprefix[|\textit{prefix}|]{|\textit{dest}|}|
is accomplished by:
%
\begin{center}
\begin{tabular}{l}
|{\edef\jobname{\scantokens\expandafter{\jobname\noexpand}}|\\
|\def\redirectjob |\textit{prefix}|#1~~~{\gdef\jobname{|\textit{dest}|#1}}|\\
|\expandafter\redirectjob\jobname~~~}\input{\jobname}|
\end{tabular}
\end{center}

In an alternative approach,
child documents can be compiled by a specific command line
without additional code or specific definitions:
%
\begin{center}
|... -jobname "|\textit{target}|" "|[\textit{flags}]%
|\includeonly{|\textit{dest}|}\input{|\textit{main}|}"|
\end{center}
%

%%%%%%%%%%%%%%%%%%%%%%%%%%%%%%%%%%%%%%%%%%%%%%%%%%%%%%%%%%%%%%%%%%%%%%%%%%%%%%%%
%%%%%%%%%%%%%%%%%%%%%%%%%%%%%%%%%%%%%%%%%%%%%%%%%%%%%%%%%%%%%%%%%%%%%%%%%%%%%%%%
\section{Information}

%%%%%%%%%%%%%%%%%%%%%%%%%%%%%%%%%%%%%%%%%%%%%%%%%%%%%%%%%%%%%%%%%%%%%%%%%%%%%%%%
\subsection{Copyright}

Copyright \copyright{} 2017--2018 Niklas Beisert

This work may be distributed and/or modified under the
conditions of the \LaTeX{} Project Public License, either version 1.3
of this license or (at your option) any later version.
The latest version of this license is in
  \url{http://www.latex-project.org/lppl.txt}
and version 1.3 or later is part of all distributions of \LaTeX{}
version 2005/12/01 or later.

This work has the LPPL maintenance status `maintained'.

The Current Maintainer of this work is Niklas Beisert.

This work consists of the files |README.txt|, |childdoc.ins| and |childdoc.dtx|
as well as the derived files |childdoc.def|, |cdocsamp.tex|
with |cdocsch1.tex|, |cdocsch2.tex|, |cdocspt3.tex|, |cdocspt4.tex|,
|cdocsdrf.tex|, |cdocsfn1.tex|, |cdocsfn2.tex|
as well as |childdoc.pdf|.

%%%%%%%%%%%%%%%%%%%%%%%%%%%%%%%%%%%%%%%%%%%%%%%%%%%%%%%%%%%%%%%%%%%%%%%%%%%%%%%%
\subsection{Files and Installation}

The package consists of the files:
%
\begin{center}
\begin{tabular}{ll}
    |README.txt|   & readme file \\
    |childdoc.ins| & installation file \\
    |childdoc.dtx| & source file \\
    |childdoc.def| & definition file \\
    |cdocsamp.tex| & sample main file \\
    |cdocsch1.tex| & sample include file \\
    |cdocsch2.tex| & sample include file \\
    |cdocspt3.tex| & sample part file \\
    |cdocspt4.tex| & sample part file \\
    |cdocsdrf.tex| & sample redirection file \\
    |cdocsfn1.tex| & sample redirection file \\
    |cdocsfn2.tex| & sample redirection file \\
    |childdoc.pdf| & manual
\end{tabular}
\end{center}
%
The distribution consists of the files
|README.txt|, |childdoc.ins| and |childdoc.dtx|.
%
\begin{itemize}
\item
Run (pdf)\LaTeX{} on |childdoc.dtx|
to compile the manual |childdoc.pdf| (this file).
\item
Run \LaTeX{} on |childdoc.ins| to create the definitions file |childdoc.def|
and the sample |cdocsamp.tex| with include files
|cdocsch1.tex|, |cdocsch2.tex|, |cdocspt3.tex|, |cdocspt4.tex|,
|cdocsdrf.tex|, |cdocsfn1.tex|, |cdocsfn2.tex|.
Then copy the file |childdoc.def| to an appropriate directory of your \LaTeX{}
distribution, e.g.\ \textit{texmf-root}|/tex/latex/childdoc|.
\end{itemize}

%%%%%%%%%%%%%%%%%%%%%%%%%%%%%%%%%%%%%%%%%%%%%%%%%%%%%%%%%%%%%%%%%%%%%%%%%%%%%%%%
\subsection{Related CTAN Packages}

There are several other packages which offer a similar functionality:
%
\begin{itemize}
\item
The packages
\href{http://ctan.org/pkg/docmute}{\textsf{docmute}},
\href{http://ctan.org/pkg/includex}{\textsf{includex}} and
\href{http://ctan.org/pkg/standalone}{\textsf{standalone}}
provide commands to include only the document body of
a child file thus allowing both files to be compiled individually.
\item
The packages \href{http://ctan.org/pkg/subdocs}{\textsf{subdocs}}
and \href{http://ctan.org/pkg/subfiles}{\textsf{subfiles}}
provide structures in which the main and child documents can be
encapsulated and allowing them to be compiled individually.
The inclusion mechanism is different from the conventional |\include|.
\item
The package \href{http://ctan.org/pkg/combine}{\textsf{combine}}
is an elaborate solution to combine several documents into one.
\end{itemize}
%
See also the CTAN topic \href{http://ctan.org/topic/subdocs}{\textsf{subdocs}}
for further related packages.
The present package differs from the above solutions in that
a document structure constructed with the conventional |\include| mechanism
just needs two extra commands at the top of every file
such that all constituent files can be compiled individually.

%%%%%%%%%%%%%%%%%%%%%%%%%%%%%%%%%%%%%%%%%%%%%%%%%%%%%%%%%%%%%%%%%%%%%%%%%%%%%%%%
%\subsection{Feature Suggestions}
%
%The following is a list of features which may be useful for future
%versions of this package:
%%
%\begin{itemize}
%\item
%\ldots
%\end{itemize}

%%%%%%%%%%%%%%%%%%%%%%%%%%%%%%%%%%%%%%%%%%%%%%%%%%%%%%%%%%%%%%%%%%%%%%%%%%%%%%%%
\subsection{Revision History}

%%%%%%%%%%%%%%%%%%%%%%%%%%%%%%%%%%%%%%%%
\paragraph{v2.0:} 2018/12/30

\begin{itemize}
\item
immediate forward processing
\item
added |\childdocby| mechanism
\item
manual restructured
\end{itemize}

%%%%%%%%%%%%%%%%%%%%%%%%%%%%%%%%%%%%%%%%
\paragraph{v1.6:} 2018/01/17

\begin{itemize}
\item
application for development of include files
\item
corrections to manual
\end{itemize}

%%%%%%%%%%%%%%%%%%%%%%%%%%%%%%%%%%%%%%%%
\paragraph{v1.5:} 2017/05/21

\begin{itemize}
\item
more complete structuring introduced
\item
|\childdocof| introduced
\item
|\childdoc| renamed to |\childdocmain|
\item
|\childredirect| renamed to |\childdocforward| and |\childdocforwardprefix|
and functionality expanded
\end{itemize}

%%%%%%%%%%%%%%%%%%%%%%%%%%%%%%%%%%%%%%%%
\paragraph{v1.0:} 2017/04/27

\begin{itemize}
\item
manual and install package
\item
first version published on CTAN
\end{itemize}

%%%%%%%%%%%%%%%%%%%%%%%%%%%%%%%%%%%%%%%%
\paragraph{v0.6:} 2017/04/26

\begin{itemize}
\item
redirection mechanism added
\end{itemize}

%%%%%%%%%%%%%%%%%%%%%%%%%%%%%%%%%%%%%%%%
\paragraph{v0.5:} 2017/04/26

\begin{itemize}
\item
functionality in definition file
\end{itemize}


%%%%%%%%%%%%%%%%%%%%%%%%%%%%%%%%%%%%%%%%%%%%%%%%%%%%%%%%%%%%%%%%%%%%%%%%%%%%%%%%
%%%%%%%%%%%%%%%%%%%%%%%%%%%%%%%%%%%%%%%%%%%%%%%%%%%%%%%%%%%%%%%%%%%%%%%%%%%%%%%%
%%%%%%%%%%%%%%%%%%%%%%%%%%%%%%%%%%%%%%%%%%%%%%%%%%%%%%%%%%%%%%%%%%%%%%%%%%%%%%%%
\appendix

\settowidth\MacroIndent{\rmfamily\scriptsize 000\ }

 \DocInput{childdoc.dtx}

\end{document}
%</driver>
% \fi
%
% %%%%%%%%%%%%%%%%%%%%%%%%%%%%%%%%%%%%%%%%%%%%%%%%%%%%%%%%%%%%%%%%%%%%%%%%%%%%%%
% %%%%%%%%%%%%%%%%%%%%%%%%%%%%%%%%%%%%%%%%%%%%%%%%%%%%%%%%%%%%%%%%%%%%%%%%%%%%%%
% \section{Sample}
%\iffalse
%<*samplemain>
%\fi
%
% The following presents a sample document
% with two chapters, two parts, a title page,
% a compile flag as well as three forwarding files to set the flag.
% It consists of eight |.tex| files:
% \begin{center}
% \begin{tabular}{ll}
% |cdocsamp.tex|&main file\\
% |cdocsch1.tex|&include file for chapter 1\\
% |cdocsch2.tex|&include file for chapter 2\\
% |cdocspt3.tex|&include file for part 3\\
% |cdocspt4.tex|&include file for part 4\\
% |cdocsdrf.tex|&forwarding file for main file in draft mode\\
% |cdocsfi1.tex|&forwarding file for final version of chapter 1\\
% |cdocsfi2.tex|&forwarding file for final version of chapter 2\\
% \end{tabular}
% \end{center}
% Each of the eight files can be compiled directly by the \LaTeX{} compiler.
%
% %%%%%%%%%%%%%%%%%%%%%%%%%%%%%%%%%%%%%%
% \paragraph{Main File.}
%
% The main file is called |cdocsamp.tex|.
%
% Load the \textsf{childdoc} definitions and
% declare the filename for the main document:
%    \begin{macrocode}
\input{childdoc.def}
\childdocmain{}
%    \end{macrocode}

% Optional override for |\version| flag:
%    \begin{macrocode}
%%\ifchilddoc\else\providecommand{\version}{draft}\fi
%    \end{macrocode}

% Define the default values for the |\version| flag
% (|final| for the main file and |draft| for childs):
%    \begin{macrocode}
\ifchilddoc
\providecommand{\version}{draft}
\else
\providecommand{\version}{final}
\fi
%    \end{macrocode}

% Load the standard document class:
%    \begin{macrocode}
\documentclass[12pt]{article}
%    \end{macrocode}

% Start the document body:
%    \begin{macrocode}
\begin{document}
%    \end{macrocode}

% Declare a title page.
% Print title, part of document being processed and version flag:
%    \begin{macrocode}
\addtocounter{page}{-1}
\begin{center}
{\LARGE\bfseries{}childdoc example\par}
\vspace{1cm}
\ifchilddoc
\ifchilddocmanual part\else chapter\fi:
`\childdocname' of `\childdocjob'\par
\else
main document: `\childdocjob'\par
\fi
version: \version\par
\end{center}
\newpage
%    \end{macrocode}

% Manually include selected file,
% otherwise process as usual:
%    \begin{macrocode}
\ifchilddocmanual
\section*{part `\childdocname'}
\input{\childdocname}
\else
%    \end{macrocode}

% Include the two chapters:
%    \begin{macrocode}
\include{cdocsch1}
\include{cdocsch2}
%    \end{macrocode}

% Include the two parts unless only chapters should be displayed:
%    \begin{macrocode}
\ifchilddoc\else
\section{part three}
\input{cdocspt3}
\section{part four}
\input{cdocspt4}
\fi
%    \end{macrocode}

% Process as usual until here:
%    \begin{macrocode}
\fi
%    \end{macrocode}

% End of document body:
%    \begin{macrocode}
\end{document}
%    \end{macrocode}
%\iffalse
%</samplemain>
%\fi
%
% %%%%%%%%%%%%%%%%%%%%%%%%%%%%%%%%%%%%%%
% \paragraph{Chapter Include Files.}
%
% The include files are called |cdocsch1.tex| and |cdocsch2.tex|.
%
%\iffalse
%<*samplechap1|samplechap2>
%\fi

% Optional override for |\version| flag:
%    \begin{macrocode}
%%\providecommand{\version}{final}
%    \end{macrocode}

% Include the main document:
%    \begin{macrocode}
\input{childdoc.def}
\childdocof{cdocsamp}
%    \end{macrocode}

%\iffalse
%</samplechap1|samplechap2>
%\fi
%
%\iffalse
%<*samplechap1>
%\fi
% Some text for chapter 1:
%    \begin{macrocode}
\section{one}
some text in chapter one
%    \end{macrocode}

%\iffalse
%</samplechap1>
%\fi
% Some text for chapter 2:
%\iffalse
%<*samplechap2>
%\fi
%    \begin{macrocode}
\section{two}
more text in chapter two
%    \end{macrocode}

%\iffalse
%</samplechap2>
%\fi
%
% %%%%%%%%%%%%%%%%%%%%%%%%%%%%%%%%%%%%%%
% \paragraph{Part Include Files.}
%
% The include files are called |cdocspt3.tex| and |cdocspt4.tex|.
%
%\iffalse
%<*samplepart3|samplepart4>
%\fi

% Optional override for |\version| flag:
%    \begin{macrocode}
%%\providecommand{\version}{final}
%    \end{macrocode}

% Include the main document:
%    \begin{macrocode}
\input{childdoc.def}
\childdocby{cdocsamp}
%    \end{macrocode}

%\iffalse
%</samplepart3|samplepart4>
%\fi
%
%\iffalse
%<*samplepart3>
%\fi
% Some text for part 3:
%    \begin{macrocode}
some text in part three
%    \end{macrocode}

%\iffalse
%</samplepart3>
%\fi
% Some text for part 4:
%\iffalse
%<*samplepart4>
%\fi
%    \begin{macrocode}
more text in part four
%    \end{macrocode}

%\iffalse
%</samplepart4>
%\fi
%
% %%%%%%%%%%%%%%%%%%%%%%%%%%%%%%%%%%%%%%
% \paragraph{Forwarding for a Complete Draft.}
%
% The following forwarding file |cdocsdrf.tex|
% compiles the main document in draft mode:
%\iffalse
%<*sampledraft>
%\fi
%    \begin{macrocode}
\def\version{draft}
\input{childdoc.def}
\childdocforward{cdocsamp}
%    \end{macrocode}

%\iffalse
%</sampledraft>
%\fi
%
% %%%%%%%%%%%%%%%%%%%%%%%%%%%%%%%%%%%%%%
% \paragraph{Forwarding for Final Version of the Chapters.}
%
% The following forwarding files |cdocsfn1.tex| and |cdocsfn2.tex|
% (with identical content)
% compile the final versions of the child documents
% |cdocsch1.tex| and |cdocsch2.tex|, respectively:
%\iffalse
%<*samplefinal>
%\fi
%    \begin{macrocode}
\def\version{final}
\input{childdoc.def}
\childdocforwardprefix[cdocsamp]{cdocsfn}{cdocsch}
%    \end{macrocode}

%\iffalse
%</samplefinal>
%\fi
%
% %%%%%%%%%%%%%%%%%%%%%%%%%%%%%%%%%%%%%%
% \paragraph{Command Line Processing.}
%
% The following three command lines generate the output files
% |cdocscld|, |cdocscl1| and |cdocscl2|
% which should be identical to
% |cdocsdrf|, |cdocsch1| and |cdocsfn2|, respectively:
% \begin{center}
% \begin{tabular}{l}
% |latex -jobname cdocscld \|\\
% |  "\def\version{draft}\input{childdoc.def}\childdocforward{cdocsamp}"|\\
% |latex -jobname cdocscl1 \|\\
% |  "\input{childdoc.def}\childdocforward[cdocsamp]{cdocsch1}"|\\
% |latex -jobname cdocscl2 \|\\
% |  "\def\version{final}\input{childdoc.def}\childdocforward{cdocsch2}"|
% \end{tabular}
% \end{center}
% Note that the trailing backslash on each first line
% merely continues the input to the second line
% (for convenient cut ant paste).
% Furthermore, the command |latex| can be replaced by any
% of its alternative versions such as |pdflatex|.
%
% %%%%%%%%%%%%%%%%%%%%%%%%%%%%%%%%%%%%%%%%%%%%%%%%%%%%%%%%%%%%%%%%%%%%%%%%%%%%%%
% %%%%%%%%%%%%%%%%%%%%%%%%%%%%%%%%%%%%%%%%%%%%%%%%%%%%%%%%%%%%%%%%%%%%%%%%%%%%%%
% \section{Implementation}
%\iffalse
%<*package>
%\fi
%
% This section describes the definitions file |childdoc.def|.

% The definitions cannot be loaded using |\usepackage| or |\RequirePackage|
% which has a mechanism to prevent loading a style file more than once.
% When loading the definitions by means of |\input|
% multiple instances have to be prevented manually:
%\iffalse
%This code needs to be before the `\ProvidesFile' directive
%which is defined at the beginning of this file.
%Therefore it is also placed there and commented out here.
%</package>
%<*discard>
%\fi
%    \begin{macrocode}
\ifdefined\childdocmain\endinput\fi
%    \end{macrocode}
%\iffalse
%</discard>
%<*package>
%\fi
%
% \macro{\ifchilddoc}
% \macro{\ifchilddocmanual}
% The conditional |\ifchilddoc| tells whether a
% child (true) or main (false) document is being compiled.
% The conditional |\ifchilddocmanual| tells whether
% the |\includeonly| mechanism is used (false) or
% the selection of child files must be performed manually (true).
% The definitions initialise to false:
%    \begin{macrocode}
\newif\ifchilddoc
\newif\ifchilddocmanual
%    \end{macrocode}

% \macro{\childdocname}
% \macro{\childdocjob}
% The macro |\childdocname| stores the name of the main document
% to be compiled. The macro |\childdocjob| stores the name of
% the document on which the \LaTeX{} compiler was originally invoked.
% The content of |\jobname| cannot be compared
% to filenames specified in the source due to different catcodes.
% The following code rescans |\jobname|, stores the result
% in |\childdocname| and saves a copy in |\childdocjob|:
%    \begin{macrocode}
\edef\childdocname{\scantokens\expandafter{\jobname\noexpand}}
\let\childdocjob\childdocname
%    \end{macrocode}

% \macro{\childdocdisable}
% The macro |\childdocdisable| prevents the main file
% from being processed more than once.
% At this stage, the main document command |\childdocmain|
% is assumed to be called once again where it should do nothing.
% Any subsequent call to it should prevent
% a secondary processing of the main document
% It overwrites the forwarding commands
% |\childdocof| and |\childdocforward|
% with empty macros to prevent further inclusions of the main document:
%    \begin{macrocode}
\newcommand{\childdocdisable}
{
  \renewcommand{\childdocmain}[1]{\renewcommand{\childdocmain}[1]{\endinput}}
  \renewcommand{\childdocof}[1]{}
  \renewcommand{\childdocby}[2][]{}
  \renewcommand{\childdocforward}[2][]{}
  \renewcommand{\childdocdisable}{}
}
%    \end{macrocode}

% \macro{\childdocmain}
% The macro |\childdocmain| is to be called at the top of the main file
% with nothing or the main filename (without extension) as argument.
% First, it breaks loops.
% If the argument is not empty and does not match |\childdocname|
% (which is set by the first inclusion of |childdoc.def|),
% |\ifchilddoc| is set to true, |\includeonly| is applied to the child file
% and |\jobname| is set to the main file
% (for proper handling of |.aux| files):
%    \begin{macrocode}
\newcommand{\childdocmain}[1]
{
  \childdocdisable\childdocmain{}
  \if?#1?\else
    \begingroup
      \def\childdoctmp{#1}
      \ifx\childdoctmp\childdocname
        \def\childdoctmp{}
      \else
        \def\childdoctmp
        {
          \childdoctrue
          \includeonly{\childdocname}
          \def\childdocjob{#1}
          \def\jobname{#1}
        }
      \fi
      \expandafter
    \endgroup
    \childdoctmp
  \fi
}
%    \end{macrocode}

% \macro{\childdocof}
% The command |\childdocof| redirects
% compilation to the main file |#1|.
%    \begin{macrocode}
\newcommand{\childdocof}[1]
{
  \childdocdisable
  \childdoctrue
  \includeonly{\childdocname}
  \def\jobname{#1}
  \def\childdocjob{#1}
  \input{#1}
}
%    \end{macrocode}

% \macro{\childdocby}
% The command |\childdocby| ....
%    \begin{macrocode}
\newcommand{\childdocby}[2][]
{
  \childdocdisable
  \childdoctrue
  \childdocmanualtrue
  \if?#1?\else
    \def\jobname{#2}
  \fi
  \def\childdocjob{#2}
  \input{#2}
  \endinput
}
%    \end{macrocode}

% \macro{\childdocforward}
% The command |\childdocforward| redirects
% compilation to the main file or
% (if the optional argument is given) a child file.
% Parameters are set as if the main file
% or a child file starting with |\childdocof| was compiled.
% Then compilation is handed over to the main file:
%    \begin{macrocode}
\newcommand{\childdocforward}[2][]
{
  \begingroup
    \if?#1?
      \def\childdoctmp
      {
        \def\childdocname{#2}
        \def\childdocjob{#2}
        \def\jobname{#2}
        \input{#2}
        \endinput
      }
    \else
      \def\childdoctmp
      {
        \childdocdisable
        \def\childdocname{#2}
        \childdoctrue
        \includeonly{#2}
        \def\childdocjob{#1}
        \def\jobname{#1}
        \input{#1}
        \endinput
      }
    \fi
    \expandafter
  \endgroup
  \childdoctmp
}
%    \end{macrocode}

% \macro{\childdocforwardprefix}
% The command |\childdocforwardprefix| redirects
% compilation to the main or a child file by means of a pattern.
% The prefix |#1| in the current filename is replaced by |#2|
% and the suffix of the current filename is kept
% (it is assumed that the filename does not contain the substring `|~~~|'
% which is used as a delimiter).
% Compilation is handed over to the new file by |\childdocforward|:
%    \begin{macrocode}
\newcommand{\childdocforwardprefix}[3][]
{
  \begingroup
    \def\childdocextract #2##1~~~{\def\childdoctmp{\childdocforward[#1]{#3##1}}}
    \expandafter\childdocextract\childdocname~~~
    \expandafter
  \endgroup
  \childdoctmp
}
%    \end{macrocode}

% \macro{\childdoc}
% The deprecated macro |\childdoc| is a legacy version of |\childdocmain|:
%    \begin{macrocode}
\newcommand{\childdoc}{\childdocmain}
%    \end{macrocode}

% \macro{\childdocredirect}
% The deprecated macro |\childdocredirect| is a legacy version
% of |\childdocforward| and |\childdocforwardprefix|:
%    \begin{macrocode}
\newcommand{\childdocredirect}[2][]
{
  \begingroup
    \if?#1?
      \def\childdoctmp{\childdocforward{#2}}
    \else
      \def\childdoctmp{\childdocforwardprefix{#1}{#2}}
    \fi
    \expandafter
  \endgroup
  \childdoctmp
}
%    \end{macrocode}

%\iffalse
%</package>
%\fi
%
\endinput
|\\
|\childdocby{|\textit{main}|}|\\
\end{tabular}
\end{center}
%
The directive |\childdocby| is similar to |\childdocof|
described in \secref{sec:include},
but the subsequent selection of content must be done manually.
To that end, both |\ifchilddoc| and |\ifchilddocmanual|
will be true upon processing of a part,
and the name of the part is stored in |\childdocname|.
Note that |\jobname| will be set to the filename of the current part
so that each part receives an individual |.aux| file
that does not interfere with the |.aux| file(s) of the main document.
This behaviour can be altered by the alternative form
|\childdocby[*]{|\textit{main}|}| (with a non-empty optional argument)
which uses the |.aux| file of the main document
by setting |\jobname| to \textit{main}.

%%%%%%%%%%%%%%%%%%%%%%%%%%%%%%%%%%%%%%%%%%%%%%%%%%%%%%%%%%%%%%%%%%%%%%%%%%%%%%%%
\subsection{Driver Development}
\label{sec:driver}

The \textsf{childdoc} mechanism can also be use for the development
of definition files such as \LaTeX{} styles or classes.
This case differs from the above setup with multiple parts
included by |\include| in that no |\includeonly| should be invoked.
This can be achieved by starting the include file
(before |\ProvidesPackage|) with:
%
\begin{center}
\begin{tabular}{l}
|% \iffalse
%
% childdoc.dtx Copyright (C) 2017-2018 Niklas Beisert
%
% This work may be distributed and/or modified under the
% conditions of the LaTeX Project Public License, either version 1.3
% of this license or (at your option) any later version.
% The latest version of this license is in
%   http://www.latex-project.org/lppl.txt
% and version 1.3 or later is part of all distributions of LaTeX
% version 2005/12/01 or later.
%
% This work has the LPPL maintenance status `maintained'.
%
% The Current Maintainer of this work is Niklas Beisert.
%
% This work consists of the files childdoc.dtx and childdoc.ins
% and the derived files childdoc.def and cdocsamp.tex with
% cdocsch1.tex, cdocsch2.tex, cdocsdrf.tex, cdocsfn1.tex, cdocsfn2.tex.
%
%<package>\ifdefined\childdocmain\endinput\fi
%<package>\ProvidesFile{childdoc.def}[2018/12/30 v2.0 child document driver]
%<samplemain>\ProvidesFile{cdocsamp.tex}[2018/12/30 v2.0 sample for childdoc]
%<*driver>
%\ProvidesFile{childdoc.drv}[2018/12/30 v2.0 childdoc reference manual file]
\PassOptionsToClass{10pt,a4paper}{article}
\documentclass{ltxdoc}

\usepackage[margin=35mm]{geometry}
\usepackage{hyperref}
\usepackage{hyperxmp}
\usepackage[usenames]{color}

\hypersetup{colorlinks=true}
\hypersetup{pdfstartview=FitH}
\hypersetup{pdfpagemode=UseNone}
\hypersetup{pdfsource={}}
\hypersetup{pdflang={en-UK}}
\hypersetup{pdfcopyright={Copyright 2017-2018 Niklas Beisert.
  This work may be distributed and/or modified under the
  conditions of the LaTeX Project Public License, either version 1.3
  of this license or (at your option) any later version.}}
\hypersetup{pdflicenseurl={http://www.latex-project.org/lppl.txt}}
\hypersetup{pdfcontactaddress={ETH Zurich, ITP, HIT K,
  Wolfgang-Pauli-Strasse 27}}
\hypersetup{pdfcontactpostcode={8093}}
\hypersetup{pdfcontactcity={Zurich}}
\hypersetup{pdfcontactcountry={Switzerland}}
\hypersetup{pdfcontactemail={nbeisert@itp.phys.ethz.ch}}
\hypersetup{pdfcontacturl={http://people.phys.ethz.ch/\xmptilde nbeisert/}}

\newcommand{\secref}[1]{\hyperref[#1]{section \ref*{#1}}}

\parskip1ex
\parindent0pt
\let\olditemize\itemize
\def\itemize{\olditemize\parskip0pt}

\begin{document}

\title{The \textsf{childdoc} Package}
\hypersetup{pdftitle={The childdoc Package}}
\author{Niklas Beisert\\[2ex]
  Institut f\"ur Theoretische Physik\\
  Eidgen\"ossische Technische Hochschule Z\"urich\\
  Wolfgang-Pauli-Strasse 27, 8093 Z\"urich, Switzerland\\[1ex]
  \href{mailto:nbeisert@itp.phys.ethz.ch}
  {\texttt{nbeisert@itp.phys.ethz.ch}}}
\hypersetup{pdfauthor={Niklas Beisert}}
\hypersetup{pdfsubject={Manual for the LaTeX2e Package childdoc}}
\date{30 December 2018, \textsf{v2.0}}
\maketitle

\begin{abstract}\noindent
\textsf{childdoc} is a \LaTeXe{} package
that enables the direct compilation
of document sections included by |\include|
to individual files.
\end{abstract}

\begingroup
\parskip0ex
\tableofcontents
\endgroup

%%%%%%%%%%%%%%%%%%%%%%%%%%%%%%%%%%%%%%%%%%%%%%%%%%%%%%%%%%%%%%%%%%%%%%%%%%%%%%%%
%%%%%%%%%%%%%%%%%%%%%%%%%%%%%%%%%%%%%%%%%%%%%%%%%%%%%%%%%%%%%%%%%%%%%%%%%%%%%%%%
\section{Introduction}

\LaTeX{} provides a mechanism to structure a large document (such as a book)
into a main file and several child files (containing the chapters)
using the |\include| command.
This mechanism is beneficial for documents
which span hundreds of pages in order to
make the source file(s) more manageable.
Moreover, compilation can be restricted to
selected child files by means of the |\includeonly| command.
The latter feature can be used to reduce the compilation time while editing
(this was significantly more useful in the earlier days of \LaTeX{})
or to generate a smaller document which is easier to navigate.
Another application of |\includeonly| is to generate
documents consisting of selected parts of the complete document.

However, there are a few drawbacks of the plain |\include| mechanism:
\begin{itemize}
\item
The child files cannot be compiled on their own,
they can only be compiled via the main file.
A naive editing environment
(such as a text editor with an option
to have the current file processed by \LaTeX)
may require one to switch to the main file before compiling;
attempting to compile the child file produces errors.
\item
The main file must be modified (each time)
to adjust the |\includeonly| command
to the present needs. This easily leaves the main file in a messy state.
\item
The generated document will always carry the filename
of the main document. This is inconvenient if
several child files are to be compiled and
to be kept for distribution.
\end{itemize}

The present package provides a simple interface
to make child files individually compilable by \LaTeX{}.
Compiling a child file then has the same effect as compiling
the main file with an |\includeonly| command
to select the appropriate child.
Moreover the generated document will carry the name of the child
rather than the main file.
This resolves all three above issues.

This feature is meant to make the editing of books,
thesis documents and lecture notes somewhat more convenient.
However, the package can also be used efficiently for
composing a series of documents (such as exercise sheets)
which are typically distributed individually.
It then assists the author in generating the individual documents
(potentially in different versions)
as well as a document containing the collected series.
Another application is in developing style files
or other kinds of included material
where compilation of the style file could redirect
to a sample or test file.

%%%%%%%%%%%%%%%%%%%%%%%%%%%%%%%%%%%%%%%%%%%%%%%%%%%%%%%%%%%%%%%%%%%%%%%%%%%%%%%%
%%%%%%%%%%%%%%%%%%%%%%%%%%%%%%%%%%%%%%%%%%%%%%%%%%%%%%%%%%%%%%%%%%%%%%%%%%%%%%%%
\section{Usage}

First of all, the package \textsf{childdoc} is \emph{not} a standard
\LaTeXe{} |.sty| style file! Therefore it needs to be invoked in
a non-standard way.

%%%%%%%%%%%%%%%%%%%%%%%%%%%%%%%%%%%%%%%%%%%%%%%%%%%%%%%%%%%%%%%%%%%%%%%%%%%%%%%%
\subsection{Included Files}
\label{sec:include}

%%%%%%%%%%%%%%%%%%%%%%%%%%%%%%%%%%%%%%%%
\DescribeMacro{\childdocmain}
To use the package, add the commands
\begin{center}
\begin{tabular}{l}
|\input{childdoc.def}|\\
|\childdocmain{}|\\
\end{tabular}
\end{center}
at the very top of the main \LaTeX{} file,
in particular \emph{before} the |\documentclass| statement!
The argument of |\childdocmain| should be left empty
(but it must be present).

%%%%%%%%%%%%%%%%%%%%%%%%%%%%%%%%%%%%%%%%
\DescribeMacro{\childdocof}
Furthermore, add the commands
\begin{center}
\begin{tabular}{l}
|\input{childdoc.def}|\\
|\childdocof{|\textit{main}|}|\\
\end{tabular}
\end{center}
at the top of every child file \textit{child}
which is included by |\include{|\textit{child}|}|
from within the main file
(or at least for those files to be compiled individually).
The argument \textit{main} must be the filename of the main file.

There are a couple of
considerations in setting up the main and child documents:

%%%%%%%%%%%%%%%%%%%%%%%%%%%%%%%%%%%%%%%%
\paragraph{Restrictions.}

Please note the following restrictions:
\begin{itemize}
\item
|\childdocmain| must be called with one argument \textit{main}
to ensure compatibility with earlier version of the package.
It must either be empty (|\childdocmain{}|)
or precisely match the filename of the main file in which it is specified.
See \secref{sec:detection} for further information.
\item
The filename \textit{main} must be specified without the |.tex| extension.
\item
The filename \textit{main} is case sensitive
(even in case-insensitive file systems)
due to internal string comparison.
\item
The argument \textit{main} should be fully expanded, it cannot be a macro.
\item
Subdirectories and special characters should be avoided in filenames.
\item
The command |\childdocmain{|\textit{main}|}| must be followed by a whitespace.
It should not be followed immediately by another command
or by a comment mark `|%|'.
This is because the \TeX{} parser reads the token immediately following
the argument of |\childdocmain| and puts it
at the beginning of every child section;
however, a white\-space is ignored.
\end{itemize}

%%%%%%%%%%%%%%%%%%%%%%%%%%%%%%%%%%%%%%%%
\paragraph{Content of Main File.}

It is advisable to place all content in the child files included by |\include|.
Any output contained in the main file will appear in all child documents
unless suppressed manually;
it cannot be suppressed automatically by the |\includeonly| directive
and thus should normally be avoided.
A method to include some content in the main file
by means of conditional processing is described in \secref{sec:conditional}.

%%%%%%%%%%%%%%%%%%%%%%%%%%%%%%%%%%%%%%%%
\paragraph{Page Numbering.}

When only a part of the document is compiled,
the appropriate numbering of pages
(as well as other status parameters)
is determined from the |.aux| files.
The latter contain information from previous passes.
However this information needs to propagate through
all intermediate child documents.
Therefore the page numbering in child documents may well
be inconsistent until the complete document is compiled at least once.

A useful (if unconventional) way to always ensure a consistent
page numbering is to restart the numbering in each child document
and denote the pages by `\textit{child}|.|\textit{page}'
where \textit{child} represents the chapter/section number of the child file.
This can be achieved by the command
|\numberwithin{page}{|\textit{child}|}|
of the \textsf{amsmath} package
where \textit{child} can be |chapter| or |section|
depending on the chosen structuring.
Alternatively, one can modify the macro |\thepage| appropriately
and reset the counter |page| at the start of each child file.

%%%%%%%%%%%%%%%%%%%%%%%%%%%%%%%%%%%%%%%%%%%%%%%%%%%%%%%%%%%%%%%%%%%%%%%%%%%%%%%%
\subsection{Conditional Processing}
\label{sec:conditional}

The package provides a mechanism to compile different versions
of a document. To customise the versions further some conditional processing
can come in handy to distinguish which version is being compiled.
The package provides two macros to describe the compilation context:

%%%%%%%%%%%%%%%%%%%%%%%%%%%%%%%%%%%%%%%%
\DescribeMacro{\ifchilddoc}
The conditional |\ifchilddoc| distinguishes between the compilation of
child documents and the main document:
%
\begin{center}
|\ifchilddoc |\textit{child-code}| |[|\||else |\textit{main-code}]| \||fi|
\end{center}

%%%%%%%%%%%%%%%%%%%%%%%%%%%%%%%%%%%%%%%%
\DescribeMacro{\childdocname}
\DescribeMacro{\childdocjob}
The macro |\childdocname| contains the filename (without extension)
of the main or child file being processed.
Note that |\childdocjob| will always contain the name of the main file.

%%%%%%%%%%%%%%%%%%%%%%%%%%%%%%%%%%%%%%%%
\paragraph{Title Page.}

Conditional processing can be used to include a title or banner page
in the main document when proper precautions are taken.
Importantly, the code in the main file should ensure that the page counter
(as well as other status parameters which are stored in the |.aux| files)
takes the same value after the conditional processing.
Otherwise the page numbers may take divergent values
depending on which part is compiled.

For example, a title page could be declared by:
%
\begin{center}
\begin{tabular}{l}
|\ifchilddoc\||else|\\
|\addtocounter{page}{-1}|\\
\textit{code for title page}\\
|\newpage|\\
|\||fi|
\end{tabular}
\end{center}
%
A banner page for the child documents can be generated by:
%
\begin{center}
\begin{tabular}{l}
|\ifchilddoc|\\
|\addtocounter{page}{-1}|\\
\textit{code for banner page}\\
|\newpage|\\
|\||fi|
\end{tabular}
\end{center}
%
Here one could write a message such as:
\begin{center}
|This is the part \childdocname{} of \childdocjob{}.|
\end{center}

%%%%%%%%%%%%%%%%%%%%%%%%%%%%%%%%%%%%%%%%%%%%%%%%%%%%%%%%%%%%%%%%%%%%%%%%%%%%%%%%
\subsection{Flags}
\label{sec:flags}

The package makes it easy to generate different versions
of the main or child documents.
To this end compilation flags can be defined
and assigned different default values.
They will be particularly useful in conjunction
with the forwarding mechanism described in \secref{sec:forward}.

For example, it may be useful to have a flag |\version|
which can be set to |draft| or |final|.
The document source will contain some conditional code
depending on the value of |\version|.
Suppose further, the flag should default to |final| for the main file
and to |draft| for child files
which is a natural assignment for editing the document.
This is achieved by placing the following code
in the preamble of the main document
(below the |\childdocmain| directive):
%
\begin{center}
\begin{tabular}{l}
|\ifchilddoc|\\
|\providecommand{\version}{draft}|\\
|\||else|\\
|\providecommand{\version}{final}|\\
|\||fi|
\end{tabular}
\end{center}
%
The definition by |\providecommand| makes sure
that previous definitions are not overwritten.
Further statements |\providecommand{\version}{...}|
can thus be added before the above code to override it.

For the main file, one might add a line
(between |\childdocmain| and the above block)
%
\begin{center}
|%\ifchilddoc\||else\providecommand{\version}{draft}\||fi|
\end{center}
%
which can be uncommented to produce a draft version.
Likewise one can add a line to the very top of a child file
(above the |\childdocof{|\textit{main}|}| directive)
%
\begin{center}
|%\providecommand{\version}{final}|
\end{center}
%
which can be uncommented to produce the final version of this child document.

%%%%%%%%%%%%%%%%%%%%%%%%%%%%%%%%%%%%%%%%%%%%%%%%%%%%%%%%%%%%%%%%%%%%%%%%%%%%%%%%
\subsection{Forwarding}
\label{sec:forward}

Different versions of the main or child documents
using compilation flags as described in \secref{sec:flags}
can be (permanently) stored in different files
for convenient compilation, viewing and distribution.
To this end, the package defines a command
to pass on compilation to a different file:

%%%%%%%%%%%%%%%%%%%%%%%%%%%%%%%%%%%%%%%%
\DescribeMacro{\childdocforward}
The command |\childdocforward| redirects processing to
another source file:
%
\begin{center}
\begin{tabular}{l}
|\input{childdoc.def}|\\
|\childdocforward[|\textit{main}|]{|\textit{dest}|}|\\
\end{tabular}
\end{center}
%
The argument \textit{dest} is the destination file
(without extension).
It should be the main file or one of the child files.
Note that further \textsf{childdoc} directives
such as |\childdocof| and |\childdocforward|
in the indicated file will be processed in this form.
The optional argument \textit{main}
passes on directly to the main file \textit{main}
while pretending to compile the child \textit{dest}.
This form behaves as if \textit{dest}
issues |\childdocof{|\textit{main}|}| right away,
and no further \textsf{childdoc} directives will be processed.

%%%%%%%%%%%%%%%%%%%%%%%%%%%%%%%%%%%%%%%%
\DescribeMacro{\...prefix}
In the alternative form |\childdocforwardprefix|,
%
\begin{center}
\begin{tabular}{l}
|\input{childdoc.def}|\\
|\childdocforwardprefix[|\textit{main}|]{|\textit{prefix}|}{|\textit{dest}|}|
\end{tabular}
\end{center}
%
the destination file is determined by a pattern
depending on the current file:
To make this work, the current file must be called
`{\textit{prefix}\hspace{0.2em}\textit{suffix}}'
with \textit{prefix} matching precisely the argument.
Processing is then passed on to the file
`{\textit{dest}\hspace{0.2em}\textit{suffix}}'.
Surely, the same effect is achieved by
directly specifying the
argument `{\textit{dest}\hspace{0.2em}\textit{suffix}}'
in the first form.
However, that requires to set up a different file
for each child. With the alternative form of the command
all these files can have exactly the same content
which simplifies setting them up and maintaining them.

For example, the following file |draft.tex|
with a compilation flag |\version| as described in \secref{sec:flags}
compiles the main document as a draft:
%
\begin{center}
\begin{tabular}{l}
|\def\version{draft}|\\
|\input{childdoc.def}|\\
|\childdocforward{|\textit{main}|}|
\end{tabular}
\end{center}
%
Likewise, the following files |final|\textit{nn}|.tex|
compile the final version of the child document
|child|\textit{nn}|.tex|:
%
\begin{center}
\begin{tabular}{l}
|\def\version{final}|\\
|\input{childdoc.def}|\\
|\childdocforwardprefix{final}{child}|
\end{tabular}
\end{center}
%

Note that when several versions of a main file and/or of each child file
are to be generated, it may be convenient to set up a |Makefile| or
shell script to automatise the process.

%%%%%%%%%%%%%%%%%%%%%%%%%%%%%%%%%%%%%%%%%%%%%%%%%%%%%%%%%%%%%%%%%%%%%%%%%%%%%%%%
\subsection{Command Line Processing}
\label{sec:commandline}

The effect of redirection files can also be achieved by invoking
the \LaTeX{} compiler with a more elaborate command line.
Most conveniently this should be done as part
of a shell script or a |Makefile|.

When using \textsf{childdoc} in the main file, the following
command lines effectively perform a redirection
(note that depending on the shell being used,
backslashes may have to be doubled: `|\|' $\to$ `|\\|'):
%
\begin{center}
|... -jobname "|\textit{target}|" |\\|"|[\textit{flags}]%
|\input{childdoc.def}\childdocforward[|\textit{main}|]{|\textit{dest}|}"|
\end{center}
%
Here \textit{target} is the name of the output file,
\textit{main} is the name of the main file
and \textit{dest} is the name of the main or child file to be processed
(all filenames without extensions).
The optional argument \textit{main} can be omitted
if \textit{main} matches \textit{dest}.
Optionally, compilation \textit{flags} can be defined via |\def| commands.
This command line makes the \TeX{} engine believe
it is compiling the file \textit{target}
whose content is specified as the latter parameter.
The provided code then forwards the processing to
\textit{main} or \textit{dest} as described in \secref{sec:forward}.

%%%%%%%%%%%%%%%%%%%%%%%%%%%%%%%%%%%%%%%%%%%%%%%%%%%%%%%%%%%%%%%%%%%%%%%%%%%%%%%%
\subsection{Include by Input}
\label{sec:input}

Including child documents by |\include| has some restrictions by design.
Most notably, the content of a child document always occupies
its own set of pages; pages cannot be shared between child documents.
Usually, this behaviour makes perfect sense
because each child document contain an essential part of the document.
However, in some situations it may be desirable to compose
a document from a collection of parts
without having mandatory page breaks between then.
For this case, the package
provides a mechanism to include parts
by |\input| which can also be processed individually.
However, by construction this mechanism
requires manual handling of the content to be output.

%%%%%%%%%%%%%%%%%%%%%%%%%%%%%%%%%%%%%%%%
\DescribeMacro{\ifchilddocmanual}
The main file should be prepared as usual, see \secref{sec:include}.
However, the document body must make a distinction
between processing of an individual part and of the main document, e.g.:
%
\begin{center}
\begin{tabular}{l}
|\ifchilddocmanual|\\
|\input{\childdocname}|\\
|\||else|\\
\textit{document body with }|\input{|\textit{part}|}|\\
|\||fi|
\end{tabular}
\end{center}
%
The conditional |\ifchilddocmanual| is true whenever
a part to be included by |\input| is being compiled,
and the name of the part is stored in |\childdocname|.

%%%%%%%%%%%%%%%%%%%%%%%%%%%%%%%%%%%%%%%%
\DescribeMacro{\childdocby}
Each part to be included by |\input| should start with:
%
\begin{center}
\begin{tabular}{l}
|\input{childdoc.def}|\\
|\childdocby{|\textit{main}|}|\\
\end{tabular}
\end{center}
%
The directive |\childdocby| is similar to |\childdocof|
described in \secref{sec:include},
but the subsequent selection of content must be done manually.
To that end, both |\ifchilddoc| and |\ifchilddocmanual|
will be true upon processing of a part,
and the name of the part is stored in |\childdocname|.
Note that |\jobname| will be set to the filename of the current part
so that each part receives an individual |.aux| file
that does not interfere with the |.aux| file(s) of the main document.
This behaviour can be altered by the alternative form
|\childdocby[*]{|\textit{main}|}| (with a non-empty optional argument)
which uses the |.aux| file of the main document
by setting |\jobname| to \textit{main}.

%%%%%%%%%%%%%%%%%%%%%%%%%%%%%%%%%%%%%%%%%%%%%%%%%%%%%%%%%%%%%%%%%%%%%%%%%%%%%%%%
\subsection{Driver Development}
\label{sec:driver}

The \textsf{childdoc} mechanism can also be use for the development
of definition files such as \LaTeX{} styles or classes.
This case differs from the above setup with multiple parts
included by |\include| in that no |\includeonly| should be invoked.
This can be achieved by starting the include file
(before |\ProvidesPackage|) with:
%
\begin{center}
\begin{tabular}{l}
|\input{childdoc.def}|\\
|\childdocforward{|\textit{main}|}|\\
\end{tabular}
\end{center}
%
or alternatively with:
%
\begin{center}
\begin{tabular}{l}
|\input{childdoc.def}|\\
|\childdocby{|\textit{main}|}|\\
\end{tabular}
\end{center}
%
Both forms have slightly different effects as described above.
The main file is prepared as usual, see \secref{sec:include}.

%%%%%%%%%%%%%%%%%%%%%%%%%%%%%%%%%%%%%%%%%%%%%%%%%%%%%%%%%%%%%%%%%%%%%%%%%%%%%%%%
\subsection{Legacy Detection}
\label{sec:detection}

The directive |\childdocmain| in the main file can detect
whether the complete document or merely a child is to be compiled
even without using the directive |\childdocof|.
This method is deprecated because it is less robust
and there is no compelling reason to use it;
it is merely provided for backward compatibility
and it may be removed in future versions.

If the detection mechanism is to be used,
it is mandatory to correctly specify
the filename of the main file as the argument of |\childdocmain|:
%
\begin{center}
\begin{tabular}{l}
|\input{childdoc.def}|\\
|\childdocmain{|\textit{main}|}|\\
\end{tabular}
\end{center}
%
If |\jobname| does not match the argument \textit{main} of |\childdocmain|,
it is assumed that |\jobname| points to the child file to be compiled.
When using |\childdocmain| with the main file specified as argument,
it suffices to start a child file
with just |\input{|\textit{main}|}|
without loading of the package and using |\childdocof|.
If instead all processing is done
with the appropriate \textsf{childdoc} directives,
the argument of \textit{main} of |\childdocmain| can be empty.

An alternative version of the command line processing described
in \secref{sec:commandline} using the detection mechanism reads:
%
\begin{center}
|... -jobname "|\textit{target}|" "|[\textit{flags}]%
[|\def\jobname{|\textit{dest}|}|]|\input{|\textit{main}|}"|
\end{center}

%%%%%%%%%%%%%%%%%%%%%%%%%%%%%%%%%%%%%%%%%%%%%%%%%%%%%%%%%%%%%%%%%%%%%%%%%%%%%%%%
\subsection{Manual Code}
\label{sec:manual}

In case one cannot be certain whether the definitions file |childdoc.def|
is installed on the target \TeX{} distribution
and one prefers not to ship it,
it is conceivable to paste a few relevant commands into the sources.

To that end, drop all statements |\input{childdoc.def}|
and perform the replacements as outlined below.
Instead of |\childdocmain{|\textit{main}|}| add the following code
to the top of the main file:
%
\begin{center}
\begin{tabular}{l}
|\||ifdefined\childdocname\endinput\||fi\newif\ifchilddoc|\\
|\edef\childdocname{\scantokens\expandafter{\jobname\noexpand}}|\\
|\def\childdocmain{|\textit{main}|}\||ifx\childdocmain\childdocname\||else|\\
|\childdoctrue\includeonly{\childdocname}\let\jobname\childdocmain\||fi|\\
\end{tabular}
\end{center}
%
Instead of |\childdocof{|\textit{main}|}| just include the main file
at the top of each child file:
%
\begin{center}
|\input{|\textit{main}|}|
\end{center}
%
A simple redirection |\childdocforward{|\textit{dest}|}| is achieved by:
%
\begin{center}
|\def\jobname{|\textit{dest}|}\input{\jobname}|
\end{center}
%
The redirection with prefix
|\childdocforwardprefix[|\textit{prefix}|]{|\textit{dest}|}|
is accomplished by:
%
\begin{center}
\begin{tabular}{l}
|{\edef\jobname{\scantokens\expandafter{\jobname\noexpand}}|\\
|\def\redirectjob |\textit{prefix}|#1~~~{\gdef\jobname{|\textit{dest}|#1}}|\\
|\expandafter\redirectjob\jobname~~~}\input{\jobname}|
\end{tabular}
\end{center}

In an alternative approach,
child documents can be compiled by a specific command line
without additional code or specific definitions:
%
\begin{center}
|... -jobname "|\textit{target}|" "|[\textit{flags}]%
|\includeonly{|\textit{dest}|}\input{|\textit{main}|}"|
\end{center}
%

%%%%%%%%%%%%%%%%%%%%%%%%%%%%%%%%%%%%%%%%%%%%%%%%%%%%%%%%%%%%%%%%%%%%%%%%%%%%%%%%
%%%%%%%%%%%%%%%%%%%%%%%%%%%%%%%%%%%%%%%%%%%%%%%%%%%%%%%%%%%%%%%%%%%%%%%%%%%%%%%%
\section{Information}

%%%%%%%%%%%%%%%%%%%%%%%%%%%%%%%%%%%%%%%%%%%%%%%%%%%%%%%%%%%%%%%%%%%%%%%%%%%%%%%%
\subsection{Copyright}

Copyright \copyright{} 2017--2018 Niklas Beisert

This work may be distributed and/or modified under the
conditions of the \LaTeX{} Project Public License, either version 1.3
of this license or (at your option) any later version.
The latest version of this license is in
  \url{http://www.latex-project.org/lppl.txt}
and version 1.3 or later is part of all distributions of \LaTeX{}
version 2005/12/01 or later.

This work has the LPPL maintenance status `maintained'.

The Current Maintainer of this work is Niklas Beisert.

This work consists of the files |README.txt|, |childdoc.ins| and |childdoc.dtx|
as well as the derived files |childdoc.def|, |cdocsamp.tex|
with |cdocsch1.tex|, |cdocsch2.tex|, |cdocspt3.tex|, |cdocspt4.tex|,
|cdocsdrf.tex|, |cdocsfn1.tex|, |cdocsfn2.tex|
as well as |childdoc.pdf|.

%%%%%%%%%%%%%%%%%%%%%%%%%%%%%%%%%%%%%%%%%%%%%%%%%%%%%%%%%%%%%%%%%%%%%%%%%%%%%%%%
\subsection{Files and Installation}

The package consists of the files:
%
\begin{center}
\begin{tabular}{ll}
    |README.txt|   & readme file \\
    |childdoc.ins| & installation file \\
    |childdoc.dtx| & source file \\
    |childdoc.def| & definition file \\
    |cdocsamp.tex| & sample main file \\
    |cdocsch1.tex| & sample include file \\
    |cdocsch2.tex| & sample include file \\
    |cdocspt3.tex| & sample part file \\
    |cdocspt4.tex| & sample part file \\
    |cdocsdrf.tex| & sample redirection file \\
    |cdocsfn1.tex| & sample redirection file \\
    |cdocsfn2.tex| & sample redirection file \\
    |childdoc.pdf| & manual
\end{tabular}
\end{center}
%
The distribution consists of the files
|README.txt|, |childdoc.ins| and |childdoc.dtx|.
%
\begin{itemize}
\item
Run (pdf)\LaTeX{} on |childdoc.dtx|
to compile the manual |childdoc.pdf| (this file).
\item
Run \LaTeX{} on |childdoc.ins| to create the definitions file |childdoc.def|
and the sample |cdocsamp.tex| with include files
|cdocsch1.tex|, |cdocsch2.tex|, |cdocspt3.tex|, |cdocspt4.tex|,
|cdocsdrf.tex|, |cdocsfn1.tex|, |cdocsfn2.tex|.
Then copy the file |childdoc.def| to an appropriate directory of your \LaTeX{}
distribution, e.g.\ \textit{texmf-root}|/tex/latex/childdoc|.
\end{itemize}

%%%%%%%%%%%%%%%%%%%%%%%%%%%%%%%%%%%%%%%%%%%%%%%%%%%%%%%%%%%%%%%%%%%%%%%%%%%%%%%%
\subsection{Related CTAN Packages}

There are several other packages which offer a similar functionality:
%
\begin{itemize}
\item
The packages
\href{http://ctan.org/pkg/docmute}{\textsf{docmute}},
\href{http://ctan.org/pkg/includex}{\textsf{includex}} and
\href{http://ctan.org/pkg/standalone}{\textsf{standalone}}
provide commands to include only the document body of
a child file thus allowing both files to be compiled individually.
\item
The packages \href{http://ctan.org/pkg/subdocs}{\textsf{subdocs}}
and \href{http://ctan.org/pkg/subfiles}{\textsf{subfiles}}
provide structures in which the main and child documents can be
encapsulated and allowing them to be compiled individually.
The inclusion mechanism is different from the conventional |\include|.
\item
The package \href{http://ctan.org/pkg/combine}{\textsf{combine}}
is an elaborate solution to combine several documents into one.
\end{itemize}
%
See also the CTAN topic \href{http://ctan.org/topic/subdocs}{\textsf{subdocs}}
for further related packages.
The present package differs from the above solutions in that
a document structure constructed with the conventional |\include| mechanism
just needs two extra commands at the top of every file
such that all constituent files can be compiled individually.

%%%%%%%%%%%%%%%%%%%%%%%%%%%%%%%%%%%%%%%%%%%%%%%%%%%%%%%%%%%%%%%%%%%%%%%%%%%%%%%%
%\subsection{Feature Suggestions}
%
%The following is a list of features which may be useful for future
%versions of this package:
%%
%\begin{itemize}
%\item
%\ldots
%\end{itemize}

%%%%%%%%%%%%%%%%%%%%%%%%%%%%%%%%%%%%%%%%%%%%%%%%%%%%%%%%%%%%%%%%%%%%%%%%%%%%%%%%
\subsection{Revision History}

%%%%%%%%%%%%%%%%%%%%%%%%%%%%%%%%%%%%%%%%
\paragraph{v2.0:} 2018/12/30

\begin{itemize}
\item
immediate forward processing
\item
added |\childdocby| mechanism
\item
manual restructured
\end{itemize}

%%%%%%%%%%%%%%%%%%%%%%%%%%%%%%%%%%%%%%%%
\paragraph{v1.6:} 2018/01/17

\begin{itemize}
\item
application for development of include files
\item
corrections to manual
\end{itemize}

%%%%%%%%%%%%%%%%%%%%%%%%%%%%%%%%%%%%%%%%
\paragraph{v1.5:} 2017/05/21

\begin{itemize}
\item
more complete structuring introduced
\item
|\childdocof| introduced
\item
|\childdoc| renamed to |\childdocmain|
\item
|\childredirect| renamed to |\childdocforward| and |\childdocforwardprefix|
and functionality expanded
\end{itemize}

%%%%%%%%%%%%%%%%%%%%%%%%%%%%%%%%%%%%%%%%
\paragraph{v1.0:} 2017/04/27

\begin{itemize}
\item
manual and install package
\item
first version published on CTAN
\end{itemize}

%%%%%%%%%%%%%%%%%%%%%%%%%%%%%%%%%%%%%%%%
\paragraph{v0.6:} 2017/04/26

\begin{itemize}
\item
redirection mechanism added
\end{itemize}

%%%%%%%%%%%%%%%%%%%%%%%%%%%%%%%%%%%%%%%%
\paragraph{v0.5:} 2017/04/26

\begin{itemize}
\item
functionality in definition file
\end{itemize}


%%%%%%%%%%%%%%%%%%%%%%%%%%%%%%%%%%%%%%%%%%%%%%%%%%%%%%%%%%%%%%%%%%%%%%%%%%%%%%%%
%%%%%%%%%%%%%%%%%%%%%%%%%%%%%%%%%%%%%%%%%%%%%%%%%%%%%%%%%%%%%%%%%%%%%%%%%%%%%%%%
%%%%%%%%%%%%%%%%%%%%%%%%%%%%%%%%%%%%%%%%%%%%%%%%%%%%%%%%%%%%%%%%%%%%%%%%%%%%%%%%
\appendix

\settowidth\MacroIndent{\rmfamily\scriptsize 000\ }

 \DocInput{childdoc.dtx}

\end{document}
%</driver>
% \fi
%
% %%%%%%%%%%%%%%%%%%%%%%%%%%%%%%%%%%%%%%%%%%%%%%%%%%%%%%%%%%%%%%%%%%%%%%%%%%%%%%
% %%%%%%%%%%%%%%%%%%%%%%%%%%%%%%%%%%%%%%%%%%%%%%%%%%%%%%%%%%%%%%%%%%%%%%%%%%%%%%
% \section{Sample}
%\iffalse
%<*samplemain>
%\fi
%
% The following presents a sample document
% with two chapters, two parts, a title page,
% a compile flag as well as three forwarding files to set the flag.
% It consists of eight |.tex| files:
% \begin{center}
% \begin{tabular}{ll}
% |cdocsamp.tex|&main file\\
% |cdocsch1.tex|&include file for chapter 1\\
% |cdocsch2.tex|&include file for chapter 2\\
% |cdocspt3.tex|&include file for part 3\\
% |cdocspt4.tex|&include file for part 4\\
% |cdocsdrf.tex|&forwarding file for main file in draft mode\\
% |cdocsfi1.tex|&forwarding file for final version of chapter 1\\
% |cdocsfi2.tex|&forwarding file for final version of chapter 2\\
% \end{tabular}
% \end{center}
% Each of the eight files can be compiled directly by the \LaTeX{} compiler.
%
% %%%%%%%%%%%%%%%%%%%%%%%%%%%%%%%%%%%%%%
% \paragraph{Main File.}
%
% The main file is called |cdocsamp.tex|.
%
% Load the \textsf{childdoc} definitions and
% declare the filename for the main document:
%    \begin{macrocode}
\input{childdoc.def}
\childdocmain{}
%    \end{macrocode}

% Optional override for |\version| flag:
%    \begin{macrocode}
%%\ifchilddoc\else\providecommand{\version}{draft}\fi
%    \end{macrocode}

% Define the default values for the |\version| flag
% (|final| for the main file and |draft| for childs):
%    \begin{macrocode}
\ifchilddoc
\providecommand{\version}{draft}
\else
\providecommand{\version}{final}
\fi
%    \end{macrocode}

% Load the standard document class:
%    \begin{macrocode}
\documentclass[12pt]{article}
%    \end{macrocode}

% Start the document body:
%    \begin{macrocode}
\begin{document}
%    \end{macrocode}

% Declare a title page.
% Print title, part of document being processed and version flag:
%    \begin{macrocode}
\addtocounter{page}{-1}
\begin{center}
{\LARGE\bfseries{}childdoc example\par}
\vspace{1cm}
\ifchilddoc
\ifchilddocmanual part\else chapter\fi:
`\childdocname' of `\childdocjob'\par
\else
main document: `\childdocjob'\par
\fi
version: \version\par
\end{center}
\newpage
%    \end{macrocode}

% Manually include selected file,
% otherwise process as usual:
%    \begin{macrocode}
\ifchilddocmanual
\section*{part `\childdocname'}
\input{\childdocname}
\else
%    \end{macrocode}

% Include the two chapters:
%    \begin{macrocode}
\include{cdocsch1}
\include{cdocsch2}
%    \end{macrocode}

% Include the two parts unless only chapters should be displayed:
%    \begin{macrocode}
\ifchilddoc\else
\section{part three}
\input{cdocspt3}
\section{part four}
\input{cdocspt4}
\fi
%    \end{macrocode}

% Process as usual until here:
%    \begin{macrocode}
\fi
%    \end{macrocode}

% End of document body:
%    \begin{macrocode}
\end{document}
%    \end{macrocode}
%\iffalse
%</samplemain>
%\fi
%
% %%%%%%%%%%%%%%%%%%%%%%%%%%%%%%%%%%%%%%
% \paragraph{Chapter Include Files.}
%
% The include files are called |cdocsch1.tex| and |cdocsch2.tex|.
%
%\iffalse
%<*samplechap1|samplechap2>
%\fi

% Optional override for |\version| flag:
%    \begin{macrocode}
%%\providecommand{\version}{final}
%    \end{macrocode}

% Include the main document:
%    \begin{macrocode}
\input{childdoc.def}
\childdocof{cdocsamp}
%    \end{macrocode}

%\iffalse
%</samplechap1|samplechap2>
%\fi
%
%\iffalse
%<*samplechap1>
%\fi
% Some text for chapter 1:
%    \begin{macrocode}
\section{one}
some text in chapter one
%    \end{macrocode}

%\iffalse
%</samplechap1>
%\fi
% Some text for chapter 2:
%\iffalse
%<*samplechap2>
%\fi
%    \begin{macrocode}
\section{two}
more text in chapter two
%    \end{macrocode}

%\iffalse
%</samplechap2>
%\fi
%
% %%%%%%%%%%%%%%%%%%%%%%%%%%%%%%%%%%%%%%
% \paragraph{Part Include Files.}
%
% The include files are called |cdocspt3.tex| and |cdocspt4.tex|.
%
%\iffalse
%<*samplepart3|samplepart4>
%\fi

% Optional override for |\version| flag:
%    \begin{macrocode}
%%\providecommand{\version}{final}
%    \end{macrocode}

% Include the main document:
%    \begin{macrocode}
\input{childdoc.def}
\childdocby{cdocsamp}
%    \end{macrocode}

%\iffalse
%</samplepart3|samplepart4>
%\fi
%
%\iffalse
%<*samplepart3>
%\fi
% Some text for part 3:
%    \begin{macrocode}
some text in part three
%    \end{macrocode}

%\iffalse
%</samplepart3>
%\fi
% Some text for part 4:
%\iffalse
%<*samplepart4>
%\fi
%    \begin{macrocode}
more text in part four
%    \end{macrocode}

%\iffalse
%</samplepart4>
%\fi
%
% %%%%%%%%%%%%%%%%%%%%%%%%%%%%%%%%%%%%%%
% \paragraph{Forwarding for a Complete Draft.}
%
% The following forwarding file |cdocsdrf.tex|
% compiles the main document in draft mode:
%\iffalse
%<*sampledraft>
%\fi
%    \begin{macrocode}
\def\version{draft}
\input{childdoc.def}
\childdocforward{cdocsamp}
%    \end{macrocode}

%\iffalse
%</sampledraft>
%\fi
%
% %%%%%%%%%%%%%%%%%%%%%%%%%%%%%%%%%%%%%%
% \paragraph{Forwarding for Final Version of the Chapters.}
%
% The following forwarding files |cdocsfn1.tex| and |cdocsfn2.tex|
% (with identical content)
% compile the final versions of the child documents
% |cdocsch1.tex| and |cdocsch2.tex|, respectively:
%\iffalse
%<*samplefinal>
%\fi
%    \begin{macrocode}
\def\version{final}
\input{childdoc.def}
\childdocforwardprefix[cdocsamp]{cdocsfn}{cdocsch}
%    \end{macrocode}

%\iffalse
%</samplefinal>
%\fi
%
% %%%%%%%%%%%%%%%%%%%%%%%%%%%%%%%%%%%%%%
% \paragraph{Command Line Processing.}
%
% The following three command lines generate the output files
% |cdocscld|, |cdocscl1| and |cdocscl2|
% which should be identical to
% |cdocsdrf|, |cdocsch1| and |cdocsfn2|, respectively:
% \begin{center}
% \begin{tabular}{l}
% |latex -jobname cdocscld \|\\
% |  "\def\version{draft}\input{childdoc.def}\childdocforward{cdocsamp}"|\\
% |latex -jobname cdocscl1 \|\\
% |  "\input{childdoc.def}\childdocforward[cdocsamp]{cdocsch1}"|\\
% |latex -jobname cdocscl2 \|\\
% |  "\def\version{final}\input{childdoc.def}\childdocforward{cdocsch2}"|
% \end{tabular}
% \end{center}
% Note that the trailing backslash on each first line
% merely continues the input to the second line
% (for convenient cut ant paste).
% Furthermore, the command |latex| can be replaced by any
% of its alternative versions such as |pdflatex|.
%
% %%%%%%%%%%%%%%%%%%%%%%%%%%%%%%%%%%%%%%%%%%%%%%%%%%%%%%%%%%%%%%%%%%%%%%%%%%%%%%
% %%%%%%%%%%%%%%%%%%%%%%%%%%%%%%%%%%%%%%%%%%%%%%%%%%%%%%%%%%%%%%%%%%%%%%%%%%%%%%
% \section{Implementation}
%\iffalse
%<*package>
%\fi
%
% This section describes the definitions file |childdoc.def|.

% The definitions cannot be loaded using |\usepackage| or |\RequirePackage|
% which has a mechanism to prevent loading a style file more than once.
% When loading the definitions by means of |\input|
% multiple instances have to be prevented manually:
%\iffalse
%This code needs to be before the `\ProvidesFile' directive
%which is defined at the beginning of this file.
%Therefore it is also placed there and commented out here.
%</package>
%<*discard>
%\fi
%    \begin{macrocode}
\ifdefined\childdocmain\endinput\fi
%    \end{macrocode}
%\iffalse
%</discard>
%<*package>
%\fi
%
% \macro{\ifchilddoc}
% \macro{\ifchilddocmanual}
% The conditional |\ifchilddoc| tells whether a
% child (true) or main (false) document is being compiled.
% The conditional |\ifchilddocmanual| tells whether
% the |\includeonly| mechanism is used (false) or
% the selection of child files must be performed manually (true).
% The definitions initialise to false:
%    \begin{macrocode}
\newif\ifchilddoc
\newif\ifchilddocmanual
%    \end{macrocode}

% \macro{\childdocname}
% \macro{\childdocjob}
% The macro |\childdocname| stores the name of the main document
% to be compiled. The macro |\childdocjob| stores the name of
% the document on which the \LaTeX{} compiler was originally invoked.
% The content of |\jobname| cannot be compared
% to filenames specified in the source due to different catcodes.
% The following code rescans |\jobname|, stores the result
% in |\childdocname| and saves a copy in |\childdocjob|:
%    \begin{macrocode}
\edef\childdocname{\scantokens\expandafter{\jobname\noexpand}}
\let\childdocjob\childdocname
%    \end{macrocode}

% \macro{\childdocdisable}
% The macro |\childdocdisable| prevents the main file
% from being processed more than once.
% At this stage, the main document command |\childdocmain|
% is assumed to be called once again where it should do nothing.
% Any subsequent call to it should prevent
% a secondary processing of the main document
% It overwrites the forwarding commands
% |\childdocof| and |\childdocforward|
% with empty macros to prevent further inclusions of the main document:
%    \begin{macrocode}
\newcommand{\childdocdisable}
{
  \renewcommand{\childdocmain}[1]{\renewcommand{\childdocmain}[1]{\endinput}}
  \renewcommand{\childdocof}[1]{}
  \renewcommand{\childdocby}[2][]{}
  \renewcommand{\childdocforward}[2][]{}
  \renewcommand{\childdocdisable}{}
}
%    \end{macrocode}

% \macro{\childdocmain}
% The macro |\childdocmain| is to be called at the top of the main file
% with nothing or the main filename (without extension) as argument.
% First, it breaks loops.
% If the argument is not empty and does not match |\childdocname|
% (which is set by the first inclusion of |childdoc.def|),
% |\ifchilddoc| is set to true, |\includeonly| is applied to the child file
% and |\jobname| is set to the main file
% (for proper handling of |.aux| files):
%    \begin{macrocode}
\newcommand{\childdocmain}[1]
{
  \childdocdisable\childdocmain{}
  \if?#1?\else
    \begingroup
      \def\childdoctmp{#1}
      \ifx\childdoctmp\childdocname
        \def\childdoctmp{}
      \else
        \def\childdoctmp
        {
          \childdoctrue
          \includeonly{\childdocname}
          \def\childdocjob{#1}
          \def\jobname{#1}
        }
      \fi
      \expandafter
    \endgroup
    \childdoctmp
  \fi
}
%    \end{macrocode}

% \macro{\childdocof}
% The command |\childdocof| redirects
% compilation to the main file |#1|.
%    \begin{macrocode}
\newcommand{\childdocof}[1]
{
  \childdocdisable
  \childdoctrue
  \includeonly{\childdocname}
  \def\jobname{#1}
  \def\childdocjob{#1}
  \input{#1}
}
%    \end{macrocode}

% \macro{\childdocby}
% The command |\childdocby| ....
%    \begin{macrocode}
\newcommand{\childdocby}[2][]
{
  \childdocdisable
  \childdoctrue
  \childdocmanualtrue
  \if?#1?\else
    \def\jobname{#2}
  \fi
  \def\childdocjob{#2}
  \input{#2}
  \endinput
}
%    \end{macrocode}

% \macro{\childdocforward}
% The command |\childdocforward| redirects
% compilation to the main file or
% (if the optional argument is given) a child file.
% Parameters are set as if the main file
% or a child file starting with |\childdocof| was compiled.
% Then compilation is handed over to the main file:
%    \begin{macrocode}
\newcommand{\childdocforward}[2][]
{
  \begingroup
    \if?#1?
      \def\childdoctmp
      {
        \def\childdocname{#2}
        \def\childdocjob{#2}
        \def\jobname{#2}
        \input{#2}
        \endinput
      }
    \else
      \def\childdoctmp
      {
        \childdocdisable
        \def\childdocname{#2}
        \childdoctrue
        \includeonly{#2}
        \def\childdocjob{#1}
        \def\jobname{#1}
        \input{#1}
        \endinput
      }
    \fi
    \expandafter
  \endgroup
  \childdoctmp
}
%    \end{macrocode}

% \macro{\childdocforwardprefix}
% The command |\childdocforwardprefix| redirects
% compilation to the main or a child file by means of a pattern.
% The prefix |#1| in the current filename is replaced by |#2|
% and the suffix of the current filename is kept
% (it is assumed that the filename does not contain the substring `|~~~|'
% which is used as a delimiter).
% Compilation is handed over to the new file by |\childdocforward|:
%    \begin{macrocode}
\newcommand{\childdocforwardprefix}[3][]
{
  \begingroup
    \def\childdocextract #2##1~~~{\def\childdoctmp{\childdocforward[#1]{#3##1}}}
    \expandafter\childdocextract\childdocname~~~
    \expandafter
  \endgroup
  \childdoctmp
}
%    \end{macrocode}

% \macro{\childdoc}
% The deprecated macro |\childdoc| is a legacy version of |\childdocmain|:
%    \begin{macrocode}
\newcommand{\childdoc}{\childdocmain}
%    \end{macrocode}

% \macro{\childdocredirect}
% The deprecated macro |\childdocredirect| is a legacy version
% of |\childdocforward| and |\childdocforwardprefix|:
%    \begin{macrocode}
\newcommand{\childdocredirect}[2][]
{
  \begingroup
    \if?#1?
      \def\childdoctmp{\childdocforward{#2}}
    \else
      \def\childdoctmp{\childdocforwardprefix{#1}{#2}}
    \fi
    \expandafter
  \endgroup
  \childdoctmp
}
%    \end{macrocode}

%\iffalse
%</package>
%\fi
%
\endinput
|\\
|\childdocforward{|\textit{main}|}|\\
\end{tabular}
\end{center}
%
or alternatively with:
%
\begin{center}
\begin{tabular}{l}
|% \iffalse
%
% childdoc.dtx Copyright (C) 2017-2018 Niklas Beisert
%
% This work may be distributed and/or modified under the
% conditions of the LaTeX Project Public License, either version 1.3
% of this license or (at your option) any later version.
% The latest version of this license is in
%   http://www.latex-project.org/lppl.txt
% and version 1.3 or later is part of all distributions of LaTeX
% version 2005/12/01 or later.
%
% This work has the LPPL maintenance status `maintained'.
%
% The Current Maintainer of this work is Niklas Beisert.
%
% This work consists of the files childdoc.dtx and childdoc.ins
% and the derived files childdoc.def and cdocsamp.tex with
% cdocsch1.tex, cdocsch2.tex, cdocsdrf.tex, cdocsfn1.tex, cdocsfn2.tex.
%
%<package>\ifdefined\childdocmain\endinput\fi
%<package>\ProvidesFile{childdoc.def}[2018/12/30 v2.0 child document driver]
%<samplemain>\ProvidesFile{cdocsamp.tex}[2018/12/30 v2.0 sample for childdoc]
%<*driver>
%\ProvidesFile{childdoc.drv}[2018/12/30 v2.0 childdoc reference manual file]
\PassOptionsToClass{10pt,a4paper}{article}
\documentclass{ltxdoc}

\usepackage[margin=35mm]{geometry}
\usepackage{hyperref}
\usepackage{hyperxmp}
\usepackage[usenames]{color}

\hypersetup{colorlinks=true}
\hypersetup{pdfstartview=FitH}
\hypersetup{pdfpagemode=UseNone}
\hypersetup{pdfsource={}}
\hypersetup{pdflang={en-UK}}
\hypersetup{pdfcopyright={Copyright 2017-2018 Niklas Beisert.
  This work may be distributed and/or modified under the
  conditions of the LaTeX Project Public License, either version 1.3
  of this license or (at your option) any later version.}}
\hypersetup{pdflicenseurl={http://www.latex-project.org/lppl.txt}}
\hypersetup{pdfcontactaddress={ETH Zurich, ITP, HIT K,
  Wolfgang-Pauli-Strasse 27}}
\hypersetup{pdfcontactpostcode={8093}}
\hypersetup{pdfcontactcity={Zurich}}
\hypersetup{pdfcontactcountry={Switzerland}}
\hypersetup{pdfcontactemail={nbeisert@itp.phys.ethz.ch}}
\hypersetup{pdfcontacturl={http://people.phys.ethz.ch/\xmptilde nbeisert/}}

\newcommand{\secref}[1]{\hyperref[#1]{section \ref*{#1}}}

\parskip1ex
\parindent0pt
\let\olditemize\itemize
\def\itemize{\olditemize\parskip0pt}

\begin{document}

\title{The \textsf{childdoc} Package}
\hypersetup{pdftitle={The childdoc Package}}
\author{Niklas Beisert\\[2ex]
  Institut f\"ur Theoretische Physik\\
  Eidgen\"ossische Technische Hochschule Z\"urich\\
  Wolfgang-Pauli-Strasse 27, 8093 Z\"urich, Switzerland\\[1ex]
  \href{mailto:nbeisert@itp.phys.ethz.ch}
  {\texttt{nbeisert@itp.phys.ethz.ch}}}
\hypersetup{pdfauthor={Niklas Beisert}}
\hypersetup{pdfsubject={Manual for the LaTeX2e Package childdoc}}
\date{30 December 2018, \textsf{v2.0}}
\maketitle

\begin{abstract}\noindent
\textsf{childdoc} is a \LaTeXe{} package
that enables the direct compilation
of document sections included by |\include|
to individual files.
\end{abstract}

\begingroup
\parskip0ex
\tableofcontents
\endgroup

%%%%%%%%%%%%%%%%%%%%%%%%%%%%%%%%%%%%%%%%%%%%%%%%%%%%%%%%%%%%%%%%%%%%%%%%%%%%%%%%
%%%%%%%%%%%%%%%%%%%%%%%%%%%%%%%%%%%%%%%%%%%%%%%%%%%%%%%%%%%%%%%%%%%%%%%%%%%%%%%%
\section{Introduction}

\LaTeX{} provides a mechanism to structure a large document (such as a book)
into a main file and several child files (containing the chapters)
using the |\include| command.
This mechanism is beneficial for documents
which span hundreds of pages in order to
make the source file(s) more manageable.
Moreover, compilation can be restricted to
selected child files by means of the |\includeonly| command.
The latter feature can be used to reduce the compilation time while editing
(this was significantly more useful in the earlier days of \LaTeX{})
or to generate a smaller document which is easier to navigate.
Another application of |\includeonly| is to generate
documents consisting of selected parts of the complete document.

However, there are a few drawbacks of the plain |\include| mechanism:
\begin{itemize}
\item
The child files cannot be compiled on their own,
they can only be compiled via the main file.
A naive editing environment
(such as a text editor with an option
to have the current file processed by \LaTeX)
may require one to switch to the main file before compiling;
attempting to compile the child file produces errors.
\item
The main file must be modified (each time)
to adjust the |\includeonly| command
to the present needs. This easily leaves the main file in a messy state.
\item
The generated document will always carry the filename
of the main document. This is inconvenient if
several child files are to be compiled and
to be kept for distribution.
\end{itemize}

The present package provides a simple interface
to make child files individually compilable by \LaTeX{}.
Compiling a child file then has the same effect as compiling
the main file with an |\includeonly| command
to select the appropriate child.
Moreover the generated document will carry the name of the child
rather than the main file.
This resolves all three above issues.

This feature is meant to make the editing of books,
thesis documents and lecture notes somewhat more convenient.
However, the package can also be used efficiently for
composing a series of documents (such as exercise sheets)
which are typically distributed individually.
It then assists the author in generating the individual documents
(potentially in different versions)
as well as a document containing the collected series.
Another application is in developing style files
or other kinds of included material
where compilation of the style file could redirect
to a sample or test file.

%%%%%%%%%%%%%%%%%%%%%%%%%%%%%%%%%%%%%%%%%%%%%%%%%%%%%%%%%%%%%%%%%%%%%%%%%%%%%%%%
%%%%%%%%%%%%%%%%%%%%%%%%%%%%%%%%%%%%%%%%%%%%%%%%%%%%%%%%%%%%%%%%%%%%%%%%%%%%%%%%
\section{Usage}

First of all, the package \textsf{childdoc} is \emph{not} a standard
\LaTeXe{} |.sty| style file! Therefore it needs to be invoked in
a non-standard way.

%%%%%%%%%%%%%%%%%%%%%%%%%%%%%%%%%%%%%%%%%%%%%%%%%%%%%%%%%%%%%%%%%%%%%%%%%%%%%%%%
\subsection{Included Files}
\label{sec:include}

%%%%%%%%%%%%%%%%%%%%%%%%%%%%%%%%%%%%%%%%
\DescribeMacro{\childdocmain}
To use the package, add the commands
\begin{center}
\begin{tabular}{l}
|\input{childdoc.def}|\\
|\childdocmain{}|\\
\end{tabular}
\end{center}
at the very top of the main \LaTeX{} file,
in particular \emph{before} the |\documentclass| statement!
The argument of |\childdocmain| should be left empty
(but it must be present).

%%%%%%%%%%%%%%%%%%%%%%%%%%%%%%%%%%%%%%%%
\DescribeMacro{\childdocof}
Furthermore, add the commands
\begin{center}
\begin{tabular}{l}
|\input{childdoc.def}|\\
|\childdocof{|\textit{main}|}|\\
\end{tabular}
\end{center}
at the top of every child file \textit{child}
which is included by |\include{|\textit{child}|}|
from within the main file
(or at least for those files to be compiled individually).
The argument \textit{main} must be the filename of the main file.

There are a couple of
considerations in setting up the main and child documents:

%%%%%%%%%%%%%%%%%%%%%%%%%%%%%%%%%%%%%%%%
\paragraph{Restrictions.}

Please note the following restrictions:
\begin{itemize}
\item
|\childdocmain| must be called with one argument \textit{main}
to ensure compatibility with earlier version of the package.
It must either be empty (|\childdocmain{}|)
or precisely match the filename of the main file in which it is specified.
See \secref{sec:detection} for further information.
\item
The filename \textit{main} must be specified without the |.tex| extension.
\item
The filename \textit{main} is case sensitive
(even in case-insensitive file systems)
due to internal string comparison.
\item
The argument \textit{main} should be fully expanded, it cannot be a macro.
\item
Subdirectories and special characters should be avoided in filenames.
\item
The command |\childdocmain{|\textit{main}|}| must be followed by a whitespace.
It should not be followed immediately by another command
or by a comment mark `|%|'.
This is because the \TeX{} parser reads the token immediately following
the argument of |\childdocmain| and puts it
at the beginning of every child section;
however, a white\-space is ignored.
\end{itemize}

%%%%%%%%%%%%%%%%%%%%%%%%%%%%%%%%%%%%%%%%
\paragraph{Content of Main File.}

It is advisable to place all content in the child files included by |\include|.
Any output contained in the main file will appear in all child documents
unless suppressed manually;
it cannot be suppressed automatically by the |\includeonly| directive
and thus should normally be avoided.
A method to include some content in the main file
by means of conditional processing is described in \secref{sec:conditional}.

%%%%%%%%%%%%%%%%%%%%%%%%%%%%%%%%%%%%%%%%
\paragraph{Page Numbering.}

When only a part of the document is compiled,
the appropriate numbering of pages
(as well as other status parameters)
is determined from the |.aux| files.
The latter contain information from previous passes.
However this information needs to propagate through
all intermediate child documents.
Therefore the page numbering in child documents may well
be inconsistent until the complete document is compiled at least once.

A useful (if unconventional) way to always ensure a consistent
page numbering is to restart the numbering in each child document
and denote the pages by `\textit{child}|.|\textit{page}'
where \textit{child} represents the chapter/section number of the child file.
This can be achieved by the command
|\numberwithin{page}{|\textit{child}|}|
of the \textsf{amsmath} package
where \textit{child} can be |chapter| or |section|
depending on the chosen structuring.
Alternatively, one can modify the macro |\thepage| appropriately
and reset the counter |page| at the start of each child file.

%%%%%%%%%%%%%%%%%%%%%%%%%%%%%%%%%%%%%%%%%%%%%%%%%%%%%%%%%%%%%%%%%%%%%%%%%%%%%%%%
\subsection{Conditional Processing}
\label{sec:conditional}

The package provides a mechanism to compile different versions
of a document. To customise the versions further some conditional processing
can come in handy to distinguish which version is being compiled.
The package provides two macros to describe the compilation context:

%%%%%%%%%%%%%%%%%%%%%%%%%%%%%%%%%%%%%%%%
\DescribeMacro{\ifchilddoc}
The conditional |\ifchilddoc| distinguishes between the compilation of
child documents and the main document:
%
\begin{center}
|\ifchilddoc |\textit{child-code}| |[|\||else |\textit{main-code}]| \||fi|
\end{center}

%%%%%%%%%%%%%%%%%%%%%%%%%%%%%%%%%%%%%%%%
\DescribeMacro{\childdocname}
\DescribeMacro{\childdocjob}
The macro |\childdocname| contains the filename (without extension)
of the main or child file being processed.
Note that |\childdocjob| will always contain the name of the main file.

%%%%%%%%%%%%%%%%%%%%%%%%%%%%%%%%%%%%%%%%
\paragraph{Title Page.}

Conditional processing can be used to include a title or banner page
in the main document when proper precautions are taken.
Importantly, the code in the main file should ensure that the page counter
(as well as other status parameters which are stored in the |.aux| files)
takes the same value after the conditional processing.
Otherwise the page numbers may take divergent values
depending on which part is compiled.

For example, a title page could be declared by:
%
\begin{center}
\begin{tabular}{l}
|\ifchilddoc\||else|\\
|\addtocounter{page}{-1}|\\
\textit{code for title page}\\
|\newpage|\\
|\||fi|
\end{tabular}
\end{center}
%
A banner page for the child documents can be generated by:
%
\begin{center}
\begin{tabular}{l}
|\ifchilddoc|\\
|\addtocounter{page}{-1}|\\
\textit{code for banner page}\\
|\newpage|\\
|\||fi|
\end{tabular}
\end{center}
%
Here one could write a message such as:
\begin{center}
|This is the part \childdocname{} of \childdocjob{}.|
\end{center}

%%%%%%%%%%%%%%%%%%%%%%%%%%%%%%%%%%%%%%%%%%%%%%%%%%%%%%%%%%%%%%%%%%%%%%%%%%%%%%%%
\subsection{Flags}
\label{sec:flags}

The package makes it easy to generate different versions
of the main or child documents.
To this end compilation flags can be defined
and assigned different default values.
They will be particularly useful in conjunction
with the forwarding mechanism described in \secref{sec:forward}.

For example, it may be useful to have a flag |\version|
which can be set to |draft| or |final|.
The document source will contain some conditional code
depending on the value of |\version|.
Suppose further, the flag should default to |final| for the main file
and to |draft| for child files
which is a natural assignment for editing the document.
This is achieved by placing the following code
in the preamble of the main document
(below the |\childdocmain| directive):
%
\begin{center}
\begin{tabular}{l}
|\ifchilddoc|\\
|\providecommand{\version}{draft}|\\
|\||else|\\
|\providecommand{\version}{final}|\\
|\||fi|
\end{tabular}
\end{center}
%
The definition by |\providecommand| makes sure
that previous definitions are not overwritten.
Further statements |\providecommand{\version}{...}|
can thus be added before the above code to override it.

For the main file, one might add a line
(between |\childdocmain| and the above block)
%
\begin{center}
|%\ifchilddoc\||else\providecommand{\version}{draft}\||fi|
\end{center}
%
which can be uncommented to produce a draft version.
Likewise one can add a line to the very top of a child file
(above the |\childdocof{|\textit{main}|}| directive)
%
\begin{center}
|%\providecommand{\version}{final}|
\end{center}
%
which can be uncommented to produce the final version of this child document.

%%%%%%%%%%%%%%%%%%%%%%%%%%%%%%%%%%%%%%%%%%%%%%%%%%%%%%%%%%%%%%%%%%%%%%%%%%%%%%%%
\subsection{Forwarding}
\label{sec:forward}

Different versions of the main or child documents
using compilation flags as described in \secref{sec:flags}
can be (permanently) stored in different files
for convenient compilation, viewing and distribution.
To this end, the package defines a command
to pass on compilation to a different file:

%%%%%%%%%%%%%%%%%%%%%%%%%%%%%%%%%%%%%%%%
\DescribeMacro{\childdocforward}
The command |\childdocforward| redirects processing to
another source file:
%
\begin{center}
\begin{tabular}{l}
|\input{childdoc.def}|\\
|\childdocforward[|\textit{main}|]{|\textit{dest}|}|\\
\end{tabular}
\end{center}
%
The argument \textit{dest} is the destination file
(without extension).
It should be the main file or one of the child files.
Note that further \textsf{childdoc} directives
such as |\childdocof| and |\childdocforward|
in the indicated file will be processed in this form.
The optional argument \textit{main}
passes on directly to the main file \textit{main}
while pretending to compile the child \textit{dest}.
This form behaves as if \textit{dest}
issues |\childdocof{|\textit{main}|}| right away,
and no further \textsf{childdoc} directives will be processed.

%%%%%%%%%%%%%%%%%%%%%%%%%%%%%%%%%%%%%%%%
\DescribeMacro{\...prefix}
In the alternative form |\childdocforwardprefix|,
%
\begin{center}
\begin{tabular}{l}
|\input{childdoc.def}|\\
|\childdocforwardprefix[|\textit{main}|]{|\textit{prefix}|}{|\textit{dest}|}|
\end{tabular}
\end{center}
%
the destination file is determined by a pattern
depending on the current file:
To make this work, the current file must be called
`{\textit{prefix}\hspace{0.2em}\textit{suffix}}'
with \textit{prefix} matching precisely the argument.
Processing is then passed on to the file
`{\textit{dest}\hspace{0.2em}\textit{suffix}}'.
Surely, the same effect is achieved by
directly specifying the
argument `{\textit{dest}\hspace{0.2em}\textit{suffix}}'
in the first form.
However, that requires to set up a different file
for each child. With the alternative form of the command
all these files can have exactly the same content
which simplifies setting them up and maintaining them.

For example, the following file |draft.tex|
with a compilation flag |\version| as described in \secref{sec:flags}
compiles the main document as a draft:
%
\begin{center}
\begin{tabular}{l}
|\def\version{draft}|\\
|\input{childdoc.def}|\\
|\childdocforward{|\textit{main}|}|
\end{tabular}
\end{center}
%
Likewise, the following files |final|\textit{nn}|.tex|
compile the final version of the child document
|child|\textit{nn}|.tex|:
%
\begin{center}
\begin{tabular}{l}
|\def\version{final}|\\
|\input{childdoc.def}|\\
|\childdocforwardprefix{final}{child}|
\end{tabular}
\end{center}
%

Note that when several versions of a main file and/or of each child file
are to be generated, it may be convenient to set up a |Makefile| or
shell script to automatise the process.

%%%%%%%%%%%%%%%%%%%%%%%%%%%%%%%%%%%%%%%%%%%%%%%%%%%%%%%%%%%%%%%%%%%%%%%%%%%%%%%%
\subsection{Command Line Processing}
\label{sec:commandline}

The effect of redirection files can also be achieved by invoking
the \LaTeX{} compiler with a more elaborate command line.
Most conveniently this should be done as part
of a shell script or a |Makefile|.

When using \textsf{childdoc} in the main file, the following
command lines effectively perform a redirection
(note that depending on the shell being used,
backslashes may have to be doubled: `|\|' $\to$ `|\\|'):
%
\begin{center}
|... -jobname "|\textit{target}|" |\\|"|[\textit{flags}]%
|\input{childdoc.def}\childdocforward[|\textit{main}|]{|\textit{dest}|}"|
\end{center}
%
Here \textit{target} is the name of the output file,
\textit{main} is the name of the main file
and \textit{dest} is the name of the main or child file to be processed
(all filenames without extensions).
The optional argument \textit{main} can be omitted
if \textit{main} matches \textit{dest}.
Optionally, compilation \textit{flags} can be defined via |\def| commands.
This command line makes the \TeX{} engine believe
it is compiling the file \textit{target}
whose content is specified as the latter parameter.
The provided code then forwards the processing to
\textit{main} or \textit{dest} as described in \secref{sec:forward}.

%%%%%%%%%%%%%%%%%%%%%%%%%%%%%%%%%%%%%%%%%%%%%%%%%%%%%%%%%%%%%%%%%%%%%%%%%%%%%%%%
\subsection{Include by Input}
\label{sec:input}

Including child documents by |\include| has some restrictions by design.
Most notably, the content of a child document always occupies
its own set of pages; pages cannot be shared between child documents.
Usually, this behaviour makes perfect sense
because each child document contain an essential part of the document.
However, in some situations it may be desirable to compose
a document from a collection of parts
without having mandatory page breaks between then.
For this case, the package
provides a mechanism to include parts
by |\input| which can also be processed individually.
However, by construction this mechanism
requires manual handling of the content to be output.

%%%%%%%%%%%%%%%%%%%%%%%%%%%%%%%%%%%%%%%%
\DescribeMacro{\ifchilddocmanual}
The main file should be prepared as usual, see \secref{sec:include}.
However, the document body must make a distinction
between processing of an individual part and of the main document, e.g.:
%
\begin{center}
\begin{tabular}{l}
|\ifchilddocmanual|\\
|\input{\childdocname}|\\
|\||else|\\
\textit{document body with }|\input{|\textit{part}|}|\\
|\||fi|
\end{tabular}
\end{center}
%
The conditional |\ifchilddocmanual| is true whenever
a part to be included by |\input| is being compiled,
and the name of the part is stored in |\childdocname|.

%%%%%%%%%%%%%%%%%%%%%%%%%%%%%%%%%%%%%%%%
\DescribeMacro{\childdocby}
Each part to be included by |\input| should start with:
%
\begin{center}
\begin{tabular}{l}
|\input{childdoc.def}|\\
|\childdocby{|\textit{main}|}|\\
\end{tabular}
\end{center}
%
The directive |\childdocby| is similar to |\childdocof|
described in \secref{sec:include},
but the subsequent selection of content must be done manually.
To that end, both |\ifchilddoc| and |\ifchilddocmanual|
will be true upon processing of a part,
and the name of the part is stored in |\childdocname|.
Note that |\jobname| will be set to the filename of the current part
so that each part receives an individual |.aux| file
that does not interfere with the |.aux| file(s) of the main document.
This behaviour can be altered by the alternative form
|\childdocby[*]{|\textit{main}|}| (with a non-empty optional argument)
which uses the |.aux| file of the main document
by setting |\jobname| to \textit{main}.

%%%%%%%%%%%%%%%%%%%%%%%%%%%%%%%%%%%%%%%%%%%%%%%%%%%%%%%%%%%%%%%%%%%%%%%%%%%%%%%%
\subsection{Driver Development}
\label{sec:driver}

The \textsf{childdoc} mechanism can also be use for the development
of definition files such as \LaTeX{} styles or classes.
This case differs from the above setup with multiple parts
included by |\include| in that no |\includeonly| should be invoked.
This can be achieved by starting the include file
(before |\ProvidesPackage|) with:
%
\begin{center}
\begin{tabular}{l}
|\input{childdoc.def}|\\
|\childdocforward{|\textit{main}|}|\\
\end{tabular}
\end{center}
%
or alternatively with:
%
\begin{center}
\begin{tabular}{l}
|\input{childdoc.def}|\\
|\childdocby{|\textit{main}|}|\\
\end{tabular}
\end{center}
%
Both forms have slightly different effects as described above.
The main file is prepared as usual, see \secref{sec:include}.

%%%%%%%%%%%%%%%%%%%%%%%%%%%%%%%%%%%%%%%%%%%%%%%%%%%%%%%%%%%%%%%%%%%%%%%%%%%%%%%%
\subsection{Legacy Detection}
\label{sec:detection}

The directive |\childdocmain| in the main file can detect
whether the complete document or merely a child is to be compiled
even without using the directive |\childdocof|.
This method is deprecated because it is less robust
and there is no compelling reason to use it;
it is merely provided for backward compatibility
and it may be removed in future versions.

If the detection mechanism is to be used,
it is mandatory to correctly specify
the filename of the main file as the argument of |\childdocmain|:
%
\begin{center}
\begin{tabular}{l}
|\input{childdoc.def}|\\
|\childdocmain{|\textit{main}|}|\\
\end{tabular}
\end{center}
%
If |\jobname| does not match the argument \textit{main} of |\childdocmain|,
it is assumed that |\jobname| points to the child file to be compiled.
When using |\childdocmain| with the main file specified as argument,
it suffices to start a child file
with just |\input{|\textit{main}|}|
without loading of the package and using |\childdocof|.
If instead all processing is done
with the appropriate \textsf{childdoc} directives,
the argument of \textit{main} of |\childdocmain| can be empty.

An alternative version of the command line processing described
in \secref{sec:commandline} using the detection mechanism reads:
%
\begin{center}
|... -jobname "|\textit{target}|" "|[\textit{flags}]%
[|\def\jobname{|\textit{dest}|}|]|\input{|\textit{main}|}"|
\end{center}

%%%%%%%%%%%%%%%%%%%%%%%%%%%%%%%%%%%%%%%%%%%%%%%%%%%%%%%%%%%%%%%%%%%%%%%%%%%%%%%%
\subsection{Manual Code}
\label{sec:manual}

In case one cannot be certain whether the definitions file |childdoc.def|
is installed on the target \TeX{} distribution
and one prefers not to ship it,
it is conceivable to paste a few relevant commands into the sources.

To that end, drop all statements |\input{childdoc.def}|
and perform the replacements as outlined below.
Instead of |\childdocmain{|\textit{main}|}| add the following code
to the top of the main file:
%
\begin{center}
\begin{tabular}{l}
|\||ifdefined\childdocname\endinput\||fi\newif\ifchilddoc|\\
|\edef\childdocname{\scantokens\expandafter{\jobname\noexpand}}|\\
|\def\childdocmain{|\textit{main}|}\||ifx\childdocmain\childdocname\||else|\\
|\childdoctrue\includeonly{\childdocname}\let\jobname\childdocmain\||fi|\\
\end{tabular}
\end{center}
%
Instead of |\childdocof{|\textit{main}|}| just include the main file
at the top of each child file:
%
\begin{center}
|\input{|\textit{main}|}|
\end{center}
%
A simple redirection |\childdocforward{|\textit{dest}|}| is achieved by:
%
\begin{center}
|\def\jobname{|\textit{dest}|}\input{\jobname}|
\end{center}
%
The redirection with prefix
|\childdocforwardprefix[|\textit{prefix}|]{|\textit{dest}|}|
is accomplished by:
%
\begin{center}
\begin{tabular}{l}
|{\edef\jobname{\scantokens\expandafter{\jobname\noexpand}}|\\
|\def\redirectjob |\textit{prefix}|#1~~~{\gdef\jobname{|\textit{dest}|#1}}|\\
|\expandafter\redirectjob\jobname~~~}\input{\jobname}|
\end{tabular}
\end{center}

In an alternative approach,
child documents can be compiled by a specific command line
without additional code or specific definitions:
%
\begin{center}
|... -jobname "|\textit{target}|" "|[\textit{flags}]%
|\includeonly{|\textit{dest}|}\input{|\textit{main}|}"|
\end{center}
%

%%%%%%%%%%%%%%%%%%%%%%%%%%%%%%%%%%%%%%%%%%%%%%%%%%%%%%%%%%%%%%%%%%%%%%%%%%%%%%%%
%%%%%%%%%%%%%%%%%%%%%%%%%%%%%%%%%%%%%%%%%%%%%%%%%%%%%%%%%%%%%%%%%%%%%%%%%%%%%%%%
\section{Information}

%%%%%%%%%%%%%%%%%%%%%%%%%%%%%%%%%%%%%%%%%%%%%%%%%%%%%%%%%%%%%%%%%%%%%%%%%%%%%%%%
\subsection{Copyright}

Copyright \copyright{} 2017--2018 Niklas Beisert

This work may be distributed and/or modified under the
conditions of the \LaTeX{} Project Public License, either version 1.3
of this license or (at your option) any later version.
The latest version of this license is in
  \url{http://www.latex-project.org/lppl.txt}
and version 1.3 or later is part of all distributions of \LaTeX{}
version 2005/12/01 or later.

This work has the LPPL maintenance status `maintained'.

The Current Maintainer of this work is Niklas Beisert.

This work consists of the files |README.txt|, |childdoc.ins| and |childdoc.dtx|
as well as the derived files |childdoc.def|, |cdocsamp.tex|
with |cdocsch1.tex|, |cdocsch2.tex|, |cdocspt3.tex|, |cdocspt4.tex|,
|cdocsdrf.tex|, |cdocsfn1.tex|, |cdocsfn2.tex|
as well as |childdoc.pdf|.

%%%%%%%%%%%%%%%%%%%%%%%%%%%%%%%%%%%%%%%%%%%%%%%%%%%%%%%%%%%%%%%%%%%%%%%%%%%%%%%%
\subsection{Files and Installation}

The package consists of the files:
%
\begin{center}
\begin{tabular}{ll}
    |README.txt|   & readme file \\
    |childdoc.ins| & installation file \\
    |childdoc.dtx| & source file \\
    |childdoc.def| & definition file \\
    |cdocsamp.tex| & sample main file \\
    |cdocsch1.tex| & sample include file \\
    |cdocsch2.tex| & sample include file \\
    |cdocspt3.tex| & sample part file \\
    |cdocspt4.tex| & sample part file \\
    |cdocsdrf.tex| & sample redirection file \\
    |cdocsfn1.tex| & sample redirection file \\
    |cdocsfn2.tex| & sample redirection file \\
    |childdoc.pdf| & manual
\end{tabular}
\end{center}
%
The distribution consists of the files
|README.txt|, |childdoc.ins| and |childdoc.dtx|.
%
\begin{itemize}
\item
Run (pdf)\LaTeX{} on |childdoc.dtx|
to compile the manual |childdoc.pdf| (this file).
\item
Run \LaTeX{} on |childdoc.ins| to create the definitions file |childdoc.def|
and the sample |cdocsamp.tex| with include files
|cdocsch1.tex|, |cdocsch2.tex|, |cdocspt3.tex|, |cdocspt4.tex|,
|cdocsdrf.tex|, |cdocsfn1.tex|, |cdocsfn2.tex|.
Then copy the file |childdoc.def| to an appropriate directory of your \LaTeX{}
distribution, e.g.\ \textit{texmf-root}|/tex/latex/childdoc|.
\end{itemize}

%%%%%%%%%%%%%%%%%%%%%%%%%%%%%%%%%%%%%%%%%%%%%%%%%%%%%%%%%%%%%%%%%%%%%%%%%%%%%%%%
\subsection{Related CTAN Packages}

There are several other packages which offer a similar functionality:
%
\begin{itemize}
\item
The packages
\href{http://ctan.org/pkg/docmute}{\textsf{docmute}},
\href{http://ctan.org/pkg/includex}{\textsf{includex}} and
\href{http://ctan.org/pkg/standalone}{\textsf{standalone}}
provide commands to include only the document body of
a child file thus allowing both files to be compiled individually.
\item
The packages \href{http://ctan.org/pkg/subdocs}{\textsf{subdocs}}
and \href{http://ctan.org/pkg/subfiles}{\textsf{subfiles}}
provide structures in which the main and child documents can be
encapsulated and allowing them to be compiled individually.
The inclusion mechanism is different from the conventional |\include|.
\item
The package \href{http://ctan.org/pkg/combine}{\textsf{combine}}
is an elaborate solution to combine several documents into one.
\end{itemize}
%
See also the CTAN topic \href{http://ctan.org/topic/subdocs}{\textsf{subdocs}}
for further related packages.
The present package differs from the above solutions in that
a document structure constructed with the conventional |\include| mechanism
just needs two extra commands at the top of every file
such that all constituent files can be compiled individually.

%%%%%%%%%%%%%%%%%%%%%%%%%%%%%%%%%%%%%%%%%%%%%%%%%%%%%%%%%%%%%%%%%%%%%%%%%%%%%%%%
%\subsection{Feature Suggestions}
%
%The following is a list of features which may be useful for future
%versions of this package:
%%
%\begin{itemize}
%\item
%\ldots
%\end{itemize}

%%%%%%%%%%%%%%%%%%%%%%%%%%%%%%%%%%%%%%%%%%%%%%%%%%%%%%%%%%%%%%%%%%%%%%%%%%%%%%%%
\subsection{Revision History}

%%%%%%%%%%%%%%%%%%%%%%%%%%%%%%%%%%%%%%%%
\paragraph{v2.0:} 2018/12/30

\begin{itemize}
\item
immediate forward processing
\item
added |\childdocby| mechanism
\item
manual restructured
\end{itemize}

%%%%%%%%%%%%%%%%%%%%%%%%%%%%%%%%%%%%%%%%
\paragraph{v1.6:} 2018/01/17

\begin{itemize}
\item
application for development of include files
\item
corrections to manual
\end{itemize}

%%%%%%%%%%%%%%%%%%%%%%%%%%%%%%%%%%%%%%%%
\paragraph{v1.5:} 2017/05/21

\begin{itemize}
\item
more complete structuring introduced
\item
|\childdocof| introduced
\item
|\childdoc| renamed to |\childdocmain|
\item
|\childredirect| renamed to |\childdocforward| and |\childdocforwardprefix|
and functionality expanded
\end{itemize}

%%%%%%%%%%%%%%%%%%%%%%%%%%%%%%%%%%%%%%%%
\paragraph{v1.0:} 2017/04/27

\begin{itemize}
\item
manual and install package
\item
first version published on CTAN
\end{itemize}

%%%%%%%%%%%%%%%%%%%%%%%%%%%%%%%%%%%%%%%%
\paragraph{v0.6:} 2017/04/26

\begin{itemize}
\item
redirection mechanism added
\end{itemize}

%%%%%%%%%%%%%%%%%%%%%%%%%%%%%%%%%%%%%%%%
\paragraph{v0.5:} 2017/04/26

\begin{itemize}
\item
functionality in definition file
\end{itemize}


%%%%%%%%%%%%%%%%%%%%%%%%%%%%%%%%%%%%%%%%%%%%%%%%%%%%%%%%%%%%%%%%%%%%%%%%%%%%%%%%
%%%%%%%%%%%%%%%%%%%%%%%%%%%%%%%%%%%%%%%%%%%%%%%%%%%%%%%%%%%%%%%%%%%%%%%%%%%%%%%%
%%%%%%%%%%%%%%%%%%%%%%%%%%%%%%%%%%%%%%%%%%%%%%%%%%%%%%%%%%%%%%%%%%%%%%%%%%%%%%%%
\appendix

\settowidth\MacroIndent{\rmfamily\scriptsize 000\ }

 \DocInput{childdoc.dtx}

\end{document}
%</driver>
% \fi
%
% %%%%%%%%%%%%%%%%%%%%%%%%%%%%%%%%%%%%%%%%%%%%%%%%%%%%%%%%%%%%%%%%%%%%%%%%%%%%%%
% %%%%%%%%%%%%%%%%%%%%%%%%%%%%%%%%%%%%%%%%%%%%%%%%%%%%%%%%%%%%%%%%%%%%%%%%%%%%%%
% \section{Sample}
%\iffalse
%<*samplemain>
%\fi
%
% The following presents a sample document
% with two chapters, two parts, a title page,
% a compile flag as well as three forwarding files to set the flag.
% It consists of eight |.tex| files:
% \begin{center}
% \begin{tabular}{ll}
% |cdocsamp.tex|&main file\\
% |cdocsch1.tex|&include file for chapter 1\\
% |cdocsch2.tex|&include file for chapter 2\\
% |cdocspt3.tex|&include file for part 3\\
% |cdocspt4.tex|&include file for part 4\\
% |cdocsdrf.tex|&forwarding file for main file in draft mode\\
% |cdocsfi1.tex|&forwarding file for final version of chapter 1\\
% |cdocsfi2.tex|&forwarding file for final version of chapter 2\\
% \end{tabular}
% \end{center}
% Each of the eight files can be compiled directly by the \LaTeX{} compiler.
%
% %%%%%%%%%%%%%%%%%%%%%%%%%%%%%%%%%%%%%%
% \paragraph{Main File.}
%
% The main file is called |cdocsamp.tex|.
%
% Load the \textsf{childdoc} definitions and
% declare the filename for the main document:
%    \begin{macrocode}
\input{childdoc.def}
\childdocmain{}
%    \end{macrocode}

% Optional override for |\version| flag:
%    \begin{macrocode}
%%\ifchilddoc\else\providecommand{\version}{draft}\fi
%    \end{macrocode}

% Define the default values for the |\version| flag
% (|final| for the main file and |draft| for childs):
%    \begin{macrocode}
\ifchilddoc
\providecommand{\version}{draft}
\else
\providecommand{\version}{final}
\fi
%    \end{macrocode}

% Load the standard document class:
%    \begin{macrocode}
\documentclass[12pt]{article}
%    \end{macrocode}

% Start the document body:
%    \begin{macrocode}
\begin{document}
%    \end{macrocode}

% Declare a title page.
% Print title, part of document being processed and version flag:
%    \begin{macrocode}
\addtocounter{page}{-1}
\begin{center}
{\LARGE\bfseries{}childdoc example\par}
\vspace{1cm}
\ifchilddoc
\ifchilddocmanual part\else chapter\fi:
`\childdocname' of `\childdocjob'\par
\else
main document: `\childdocjob'\par
\fi
version: \version\par
\end{center}
\newpage
%    \end{macrocode}

% Manually include selected file,
% otherwise process as usual:
%    \begin{macrocode}
\ifchilddocmanual
\section*{part `\childdocname'}
\input{\childdocname}
\else
%    \end{macrocode}

% Include the two chapters:
%    \begin{macrocode}
\include{cdocsch1}
\include{cdocsch2}
%    \end{macrocode}

% Include the two parts unless only chapters should be displayed:
%    \begin{macrocode}
\ifchilddoc\else
\section{part three}
\input{cdocspt3}
\section{part four}
\input{cdocspt4}
\fi
%    \end{macrocode}

% Process as usual until here:
%    \begin{macrocode}
\fi
%    \end{macrocode}

% End of document body:
%    \begin{macrocode}
\end{document}
%    \end{macrocode}
%\iffalse
%</samplemain>
%\fi
%
% %%%%%%%%%%%%%%%%%%%%%%%%%%%%%%%%%%%%%%
% \paragraph{Chapter Include Files.}
%
% The include files are called |cdocsch1.tex| and |cdocsch2.tex|.
%
%\iffalse
%<*samplechap1|samplechap2>
%\fi

% Optional override for |\version| flag:
%    \begin{macrocode}
%%\providecommand{\version}{final}
%    \end{macrocode}

% Include the main document:
%    \begin{macrocode}
\input{childdoc.def}
\childdocof{cdocsamp}
%    \end{macrocode}

%\iffalse
%</samplechap1|samplechap2>
%\fi
%
%\iffalse
%<*samplechap1>
%\fi
% Some text for chapter 1:
%    \begin{macrocode}
\section{one}
some text in chapter one
%    \end{macrocode}

%\iffalse
%</samplechap1>
%\fi
% Some text for chapter 2:
%\iffalse
%<*samplechap2>
%\fi
%    \begin{macrocode}
\section{two}
more text in chapter two
%    \end{macrocode}

%\iffalse
%</samplechap2>
%\fi
%
% %%%%%%%%%%%%%%%%%%%%%%%%%%%%%%%%%%%%%%
% \paragraph{Part Include Files.}
%
% The include files are called |cdocspt3.tex| and |cdocspt4.tex|.
%
%\iffalse
%<*samplepart3|samplepart4>
%\fi

% Optional override for |\version| flag:
%    \begin{macrocode}
%%\providecommand{\version}{final}
%    \end{macrocode}

% Include the main document:
%    \begin{macrocode}
\input{childdoc.def}
\childdocby{cdocsamp}
%    \end{macrocode}

%\iffalse
%</samplepart3|samplepart4>
%\fi
%
%\iffalse
%<*samplepart3>
%\fi
% Some text for part 3:
%    \begin{macrocode}
some text in part three
%    \end{macrocode}

%\iffalse
%</samplepart3>
%\fi
% Some text for part 4:
%\iffalse
%<*samplepart4>
%\fi
%    \begin{macrocode}
more text in part four
%    \end{macrocode}

%\iffalse
%</samplepart4>
%\fi
%
% %%%%%%%%%%%%%%%%%%%%%%%%%%%%%%%%%%%%%%
% \paragraph{Forwarding for a Complete Draft.}
%
% The following forwarding file |cdocsdrf.tex|
% compiles the main document in draft mode:
%\iffalse
%<*sampledraft>
%\fi
%    \begin{macrocode}
\def\version{draft}
\input{childdoc.def}
\childdocforward{cdocsamp}
%    \end{macrocode}

%\iffalse
%</sampledraft>
%\fi
%
% %%%%%%%%%%%%%%%%%%%%%%%%%%%%%%%%%%%%%%
% \paragraph{Forwarding for Final Version of the Chapters.}
%
% The following forwarding files |cdocsfn1.tex| and |cdocsfn2.tex|
% (with identical content)
% compile the final versions of the child documents
% |cdocsch1.tex| and |cdocsch2.tex|, respectively:
%\iffalse
%<*samplefinal>
%\fi
%    \begin{macrocode}
\def\version{final}
\input{childdoc.def}
\childdocforwardprefix[cdocsamp]{cdocsfn}{cdocsch}
%    \end{macrocode}

%\iffalse
%</samplefinal>
%\fi
%
% %%%%%%%%%%%%%%%%%%%%%%%%%%%%%%%%%%%%%%
% \paragraph{Command Line Processing.}
%
% The following three command lines generate the output files
% |cdocscld|, |cdocscl1| and |cdocscl2|
% which should be identical to
% |cdocsdrf|, |cdocsch1| and |cdocsfn2|, respectively:
% \begin{center}
% \begin{tabular}{l}
% |latex -jobname cdocscld \|\\
% |  "\def\version{draft}\input{childdoc.def}\childdocforward{cdocsamp}"|\\
% |latex -jobname cdocscl1 \|\\
% |  "\input{childdoc.def}\childdocforward[cdocsamp]{cdocsch1}"|\\
% |latex -jobname cdocscl2 \|\\
% |  "\def\version{final}\input{childdoc.def}\childdocforward{cdocsch2}"|
% \end{tabular}
% \end{center}
% Note that the trailing backslash on each first line
% merely continues the input to the second line
% (for convenient cut ant paste).
% Furthermore, the command |latex| can be replaced by any
% of its alternative versions such as |pdflatex|.
%
% %%%%%%%%%%%%%%%%%%%%%%%%%%%%%%%%%%%%%%%%%%%%%%%%%%%%%%%%%%%%%%%%%%%%%%%%%%%%%%
% %%%%%%%%%%%%%%%%%%%%%%%%%%%%%%%%%%%%%%%%%%%%%%%%%%%%%%%%%%%%%%%%%%%%%%%%%%%%%%
% \section{Implementation}
%\iffalse
%<*package>
%\fi
%
% This section describes the definitions file |childdoc.def|.

% The definitions cannot be loaded using |\usepackage| or |\RequirePackage|
% which has a mechanism to prevent loading a style file more than once.
% When loading the definitions by means of |\input|
% multiple instances have to be prevented manually:
%\iffalse
%This code needs to be before the `\ProvidesFile' directive
%which is defined at the beginning of this file.
%Therefore it is also placed there and commented out here.
%</package>
%<*discard>
%\fi
%    \begin{macrocode}
\ifdefined\childdocmain\endinput\fi
%    \end{macrocode}
%\iffalse
%</discard>
%<*package>
%\fi
%
% \macro{\ifchilddoc}
% \macro{\ifchilddocmanual}
% The conditional |\ifchilddoc| tells whether a
% child (true) or main (false) document is being compiled.
% The conditional |\ifchilddocmanual| tells whether
% the |\includeonly| mechanism is used (false) or
% the selection of child files must be performed manually (true).
% The definitions initialise to false:
%    \begin{macrocode}
\newif\ifchilddoc
\newif\ifchilddocmanual
%    \end{macrocode}

% \macro{\childdocname}
% \macro{\childdocjob}
% The macro |\childdocname| stores the name of the main document
% to be compiled. The macro |\childdocjob| stores the name of
% the document on which the \LaTeX{} compiler was originally invoked.
% The content of |\jobname| cannot be compared
% to filenames specified in the source due to different catcodes.
% The following code rescans |\jobname|, stores the result
% in |\childdocname| and saves a copy in |\childdocjob|:
%    \begin{macrocode}
\edef\childdocname{\scantokens\expandafter{\jobname\noexpand}}
\let\childdocjob\childdocname
%    \end{macrocode}

% \macro{\childdocdisable}
% The macro |\childdocdisable| prevents the main file
% from being processed more than once.
% At this stage, the main document command |\childdocmain|
% is assumed to be called once again where it should do nothing.
% Any subsequent call to it should prevent
% a secondary processing of the main document
% It overwrites the forwarding commands
% |\childdocof| and |\childdocforward|
% with empty macros to prevent further inclusions of the main document:
%    \begin{macrocode}
\newcommand{\childdocdisable}
{
  \renewcommand{\childdocmain}[1]{\renewcommand{\childdocmain}[1]{\endinput}}
  \renewcommand{\childdocof}[1]{}
  \renewcommand{\childdocby}[2][]{}
  \renewcommand{\childdocforward}[2][]{}
  \renewcommand{\childdocdisable}{}
}
%    \end{macrocode}

% \macro{\childdocmain}
% The macro |\childdocmain| is to be called at the top of the main file
% with nothing or the main filename (without extension) as argument.
% First, it breaks loops.
% If the argument is not empty and does not match |\childdocname|
% (which is set by the first inclusion of |childdoc.def|),
% |\ifchilddoc| is set to true, |\includeonly| is applied to the child file
% and |\jobname| is set to the main file
% (for proper handling of |.aux| files):
%    \begin{macrocode}
\newcommand{\childdocmain}[1]
{
  \childdocdisable\childdocmain{}
  \if?#1?\else
    \begingroup
      \def\childdoctmp{#1}
      \ifx\childdoctmp\childdocname
        \def\childdoctmp{}
      \else
        \def\childdoctmp
        {
          \childdoctrue
          \includeonly{\childdocname}
          \def\childdocjob{#1}
          \def\jobname{#1}
        }
      \fi
      \expandafter
    \endgroup
    \childdoctmp
  \fi
}
%    \end{macrocode}

% \macro{\childdocof}
% The command |\childdocof| redirects
% compilation to the main file |#1|.
%    \begin{macrocode}
\newcommand{\childdocof}[1]
{
  \childdocdisable
  \childdoctrue
  \includeonly{\childdocname}
  \def\jobname{#1}
  \def\childdocjob{#1}
  \input{#1}
}
%    \end{macrocode}

% \macro{\childdocby}
% The command |\childdocby| ....
%    \begin{macrocode}
\newcommand{\childdocby}[2][]
{
  \childdocdisable
  \childdoctrue
  \childdocmanualtrue
  \if?#1?\else
    \def\jobname{#2}
  \fi
  \def\childdocjob{#2}
  \input{#2}
  \endinput
}
%    \end{macrocode}

% \macro{\childdocforward}
% The command |\childdocforward| redirects
% compilation to the main file or
% (if the optional argument is given) a child file.
% Parameters are set as if the main file
% or a child file starting with |\childdocof| was compiled.
% Then compilation is handed over to the main file:
%    \begin{macrocode}
\newcommand{\childdocforward}[2][]
{
  \begingroup
    \if?#1?
      \def\childdoctmp
      {
        \def\childdocname{#2}
        \def\childdocjob{#2}
        \def\jobname{#2}
        \input{#2}
        \endinput
      }
    \else
      \def\childdoctmp
      {
        \childdocdisable
        \def\childdocname{#2}
        \childdoctrue
        \includeonly{#2}
        \def\childdocjob{#1}
        \def\jobname{#1}
        \input{#1}
        \endinput
      }
    \fi
    \expandafter
  \endgroup
  \childdoctmp
}
%    \end{macrocode}

% \macro{\childdocforwardprefix}
% The command |\childdocforwardprefix| redirects
% compilation to the main or a child file by means of a pattern.
% The prefix |#1| in the current filename is replaced by |#2|
% and the suffix of the current filename is kept
% (it is assumed that the filename does not contain the substring `|~~~|'
% which is used as a delimiter).
% Compilation is handed over to the new file by |\childdocforward|:
%    \begin{macrocode}
\newcommand{\childdocforwardprefix}[3][]
{
  \begingroup
    \def\childdocextract #2##1~~~{\def\childdoctmp{\childdocforward[#1]{#3##1}}}
    \expandafter\childdocextract\childdocname~~~
    \expandafter
  \endgroup
  \childdoctmp
}
%    \end{macrocode}

% \macro{\childdoc}
% The deprecated macro |\childdoc| is a legacy version of |\childdocmain|:
%    \begin{macrocode}
\newcommand{\childdoc}{\childdocmain}
%    \end{macrocode}

% \macro{\childdocredirect}
% The deprecated macro |\childdocredirect| is a legacy version
% of |\childdocforward| and |\childdocforwardprefix|:
%    \begin{macrocode}
\newcommand{\childdocredirect}[2][]
{
  \begingroup
    \if?#1?
      \def\childdoctmp{\childdocforward{#2}}
    \else
      \def\childdoctmp{\childdocforwardprefix{#1}{#2}}
    \fi
    \expandafter
  \endgroup
  \childdoctmp
}
%    \end{macrocode}

%\iffalse
%</package>
%\fi
%
\endinput
|\\
|\childdocby{|\textit{main}|}|\\
\end{tabular}
\end{center}
%
Both forms have slightly different effects as described above.
The main file is prepared as usual, see \secref{sec:include}.

%%%%%%%%%%%%%%%%%%%%%%%%%%%%%%%%%%%%%%%%%%%%%%%%%%%%%%%%%%%%%%%%%%%%%%%%%%%%%%%%
\subsection{Legacy Detection}
\label{sec:detection}

The directive |\childdocmain| in the main file can detect
whether the complete document or merely a child is to be compiled
even without using the directive |\childdocof|.
This method is deprecated because it is less robust
and there is no compelling reason to use it;
it is merely provided for backward compatibility
and it may be removed in future versions.

If the detection mechanism is to be used,
it is mandatory to correctly specify
the filename of the main file as the argument of |\childdocmain|:
%
\begin{center}
\begin{tabular}{l}
|% \iffalse
%
% childdoc.dtx Copyright (C) 2017-2018 Niklas Beisert
%
% This work may be distributed and/or modified under the
% conditions of the LaTeX Project Public License, either version 1.3
% of this license or (at your option) any later version.
% The latest version of this license is in
%   http://www.latex-project.org/lppl.txt
% and version 1.3 or later is part of all distributions of LaTeX
% version 2005/12/01 or later.
%
% This work has the LPPL maintenance status `maintained'.
%
% The Current Maintainer of this work is Niklas Beisert.
%
% This work consists of the files childdoc.dtx and childdoc.ins
% and the derived files childdoc.def and cdocsamp.tex with
% cdocsch1.tex, cdocsch2.tex, cdocsdrf.tex, cdocsfn1.tex, cdocsfn2.tex.
%
%<package>\ifdefined\childdocmain\endinput\fi
%<package>\ProvidesFile{childdoc.def}[2018/12/30 v2.0 child document driver]
%<samplemain>\ProvidesFile{cdocsamp.tex}[2018/12/30 v2.0 sample for childdoc]
%<*driver>
%\ProvidesFile{childdoc.drv}[2018/12/30 v2.0 childdoc reference manual file]
\PassOptionsToClass{10pt,a4paper}{article}
\documentclass{ltxdoc}

\usepackage[margin=35mm]{geometry}
\usepackage{hyperref}
\usepackage{hyperxmp}
\usepackage[usenames]{color}

\hypersetup{colorlinks=true}
\hypersetup{pdfstartview=FitH}
\hypersetup{pdfpagemode=UseNone}
\hypersetup{pdfsource={}}
\hypersetup{pdflang={en-UK}}
\hypersetup{pdfcopyright={Copyright 2017-2018 Niklas Beisert.
  This work may be distributed and/or modified under the
  conditions of the LaTeX Project Public License, either version 1.3
  of this license or (at your option) any later version.}}
\hypersetup{pdflicenseurl={http://www.latex-project.org/lppl.txt}}
\hypersetup{pdfcontactaddress={ETH Zurich, ITP, HIT K,
  Wolfgang-Pauli-Strasse 27}}
\hypersetup{pdfcontactpostcode={8093}}
\hypersetup{pdfcontactcity={Zurich}}
\hypersetup{pdfcontactcountry={Switzerland}}
\hypersetup{pdfcontactemail={nbeisert@itp.phys.ethz.ch}}
\hypersetup{pdfcontacturl={http://people.phys.ethz.ch/\xmptilde nbeisert/}}

\newcommand{\secref}[1]{\hyperref[#1]{section \ref*{#1}}}

\parskip1ex
\parindent0pt
\let\olditemize\itemize
\def\itemize{\olditemize\parskip0pt}

\begin{document}

\title{The \textsf{childdoc} Package}
\hypersetup{pdftitle={The childdoc Package}}
\author{Niklas Beisert\\[2ex]
  Institut f\"ur Theoretische Physik\\
  Eidgen\"ossische Technische Hochschule Z\"urich\\
  Wolfgang-Pauli-Strasse 27, 8093 Z\"urich, Switzerland\\[1ex]
  \href{mailto:nbeisert@itp.phys.ethz.ch}
  {\texttt{nbeisert@itp.phys.ethz.ch}}}
\hypersetup{pdfauthor={Niklas Beisert}}
\hypersetup{pdfsubject={Manual for the LaTeX2e Package childdoc}}
\date{30 December 2018, \textsf{v2.0}}
\maketitle

\begin{abstract}\noindent
\textsf{childdoc} is a \LaTeXe{} package
that enables the direct compilation
of document sections included by |\include|
to individual files.
\end{abstract}

\begingroup
\parskip0ex
\tableofcontents
\endgroup

%%%%%%%%%%%%%%%%%%%%%%%%%%%%%%%%%%%%%%%%%%%%%%%%%%%%%%%%%%%%%%%%%%%%%%%%%%%%%%%%
%%%%%%%%%%%%%%%%%%%%%%%%%%%%%%%%%%%%%%%%%%%%%%%%%%%%%%%%%%%%%%%%%%%%%%%%%%%%%%%%
\section{Introduction}

\LaTeX{} provides a mechanism to structure a large document (such as a book)
into a main file and several child files (containing the chapters)
using the |\include| command.
This mechanism is beneficial for documents
which span hundreds of pages in order to
make the source file(s) more manageable.
Moreover, compilation can be restricted to
selected child files by means of the |\includeonly| command.
The latter feature can be used to reduce the compilation time while editing
(this was significantly more useful in the earlier days of \LaTeX{})
or to generate a smaller document which is easier to navigate.
Another application of |\includeonly| is to generate
documents consisting of selected parts of the complete document.

However, there are a few drawbacks of the plain |\include| mechanism:
\begin{itemize}
\item
The child files cannot be compiled on their own,
they can only be compiled via the main file.
A naive editing environment
(such as a text editor with an option
to have the current file processed by \LaTeX)
may require one to switch to the main file before compiling;
attempting to compile the child file produces errors.
\item
The main file must be modified (each time)
to adjust the |\includeonly| command
to the present needs. This easily leaves the main file in a messy state.
\item
The generated document will always carry the filename
of the main document. This is inconvenient if
several child files are to be compiled and
to be kept for distribution.
\end{itemize}

The present package provides a simple interface
to make child files individually compilable by \LaTeX{}.
Compiling a child file then has the same effect as compiling
the main file with an |\includeonly| command
to select the appropriate child.
Moreover the generated document will carry the name of the child
rather than the main file.
This resolves all three above issues.

This feature is meant to make the editing of books,
thesis documents and lecture notes somewhat more convenient.
However, the package can also be used efficiently for
composing a series of documents (such as exercise sheets)
which are typically distributed individually.
It then assists the author in generating the individual documents
(potentially in different versions)
as well as a document containing the collected series.
Another application is in developing style files
or other kinds of included material
where compilation of the style file could redirect
to a sample or test file.

%%%%%%%%%%%%%%%%%%%%%%%%%%%%%%%%%%%%%%%%%%%%%%%%%%%%%%%%%%%%%%%%%%%%%%%%%%%%%%%%
%%%%%%%%%%%%%%%%%%%%%%%%%%%%%%%%%%%%%%%%%%%%%%%%%%%%%%%%%%%%%%%%%%%%%%%%%%%%%%%%
\section{Usage}

First of all, the package \textsf{childdoc} is \emph{not} a standard
\LaTeXe{} |.sty| style file! Therefore it needs to be invoked in
a non-standard way.

%%%%%%%%%%%%%%%%%%%%%%%%%%%%%%%%%%%%%%%%%%%%%%%%%%%%%%%%%%%%%%%%%%%%%%%%%%%%%%%%
\subsection{Included Files}
\label{sec:include}

%%%%%%%%%%%%%%%%%%%%%%%%%%%%%%%%%%%%%%%%
\DescribeMacro{\childdocmain}
To use the package, add the commands
\begin{center}
\begin{tabular}{l}
|\input{childdoc.def}|\\
|\childdocmain{}|\\
\end{tabular}
\end{center}
at the very top of the main \LaTeX{} file,
in particular \emph{before} the |\documentclass| statement!
The argument of |\childdocmain| should be left empty
(but it must be present).

%%%%%%%%%%%%%%%%%%%%%%%%%%%%%%%%%%%%%%%%
\DescribeMacro{\childdocof}
Furthermore, add the commands
\begin{center}
\begin{tabular}{l}
|\input{childdoc.def}|\\
|\childdocof{|\textit{main}|}|\\
\end{tabular}
\end{center}
at the top of every child file \textit{child}
which is included by |\include{|\textit{child}|}|
from within the main file
(or at least for those files to be compiled individually).
The argument \textit{main} must be the filename of the main file.

There are a couple of
considerations in setting up the main and child documents:

%%%%%%%%%%%%%%%%%%%%%%%%%%%%%%%%%%%%%%%%
\paragraph{Restrictions.}

Please note the following restrictions:
\begin{itemize}
\item
|\childdocmain| must be called with one argument \textit{main}
to ensure compatibility with earlier version of the package.
It must either be empty (|\childdocmain{}|)
or precisely match the filename of the main file in which it is specified.
See \secref{sec:detection} for further information.
\item
The filename \textit{main} must be specified without the |.tex| extension.
\item
The filename \textit{main} is case sensitive
(even in case-insensitive file systems)
due to internal string comparison.
\item
The argument \textit{main} should be fully expanded, it cannot be a macro.
\item
Subdirectories and special characters should be avoided in filenames.
\item
The command |\childdocmain{|\textit{main}|}| must be followed by a whitespace.
It should not be followed immediately by another command
or by a comment mark `|%|'.
This is because the \TeX{} parser reads the token immediately following
the argument of |\childdocmain| and puts it
at the beginning of every child section;
however, a white\-space is ignored.
\end{itemize}

%%%%%%%%%%%%%%%%%%%%%%%%%%%%%%%%%%%%%%%%
\paragraph{Content of Main File.}

It is advisable to place all content in the child files included by |\include|.
Any output contained in the main file will appear in all child documents
unless suppressed manually;
it cannot be suppressed automatically by the |\includeonly| directive
and thus should normally be avoided.
A method to include some content in the main file
by means of conditional processing is described in \secref{sec:conditional}.

%%%%%%%%%%%%%%%%%%%%%%%%%%%%%%%%%%%%%%%%
\paragraph{Page Numbering.}

When only a part of the document is compiled,
the appropriate numbering of pages
(as well as other status parameters)
is determined from the |.aux| files.
The latter contain information from previous passes.
However this information needs to propagate through
all intermediate child documents.
Therefore the page numbering in child documents may well
be inconsistent until the complete document is compiled at least once.

A useful (if unconventional) way to always ensure a consistent
page numbering is to restart the numbering in each child document
and denote the pages by `\textit{child}|.|\textit{page}'
where \textit{child} represents the chapter/section number of the child file.
This can be achieved by the command
|\numberwithin{page}{|\textit{child}|}|
of the \textsf{amsmath} package
where \textit{child} can be |chapter| or |section|
depending on the chosen structuring.
Alternatively, one can modify the macro |\thepage| appropriately
and reset the counter |page| at the start of each child file.

%%%%%%%%%%%%%%%%%%%%%%%%%%%%%%%%%%%%%%%%%%%%%%%%%%%%%%%%%%%%%%%%%%%%%%%%%%%%%%%%
\subsection{Conditional Processing}
\label{sec:conditional}

The package provides a mechanism to compile different versions
of a document. To customise the versions further some conditional processing
can come in handy to distinguish which version is being compiled.
The package provides two macros to describe the compilation context:

%%%%%%%%%%%%%%%%%%%%%%%%%%%%%%%%%%%%%%%%
\DescribeMacro{\ifchilddoc}
The conditional |\ifchilddoc| distinguishes between the compilation of
child documents and the main document:
%
\begin{center}
|\ifchilddoc |\textit{child-code}| |[|\||else |\textit{main-code}]| \||fi|
\end{center}

%%%%%%%%%%%%%%%%%%%%%%%%%%%%%%%%%%%%%%%%
\DescribeMacro{\childdocname}
\DescribeMacro{\childdocjob}
The macro |\childdocname| contains the filename (without extension)
of the main or child file being processed.
Note that |\childdocjob| will always contain the name of the main file.

%%%%%%%%%%%%%%%%%%%%%%%%%%%%%%%%%%%%%%%%
\paragraph{Title Page.}

Conditional processing can be used to include a title or banner page
in the main document when proper precautions are taken.
Importantly, the code in the main file should ensure that the page counter
(as well as other status parameters which are stored in the |.aux| files)
takes the same value after the conditional processing.
Otherwise the page numbers may take divergent values
depending on which part is compiled.

For example, a title page could be declared by:
%
\begin{center}
\begin{tabular}{l}
|\ifchilddoc\||else|\\
|\addtocounter{page}{-1}|\\
\textit{code for title page}\\
|\newpage|\\
|\||fi|
\end{tabular}
\end{center}
%
A banner page for the child documents can be generated by:
%
\begin{center}
\begin{tabular}{l}
|\ifchilddoc|\\
|\addtocounter{page}{-1}|\\
\textit{code for banner page}\\
|\newpage|\\
|\||fi|
\end{tabular}
\end{center}
%
Here one could write a message such as:
\begin{center}
|This is the part \childdocname{} of \childdocjob{}.|
\end{center}

%%%%%%%%%%%%%%%%%%%%%%%%%%%%%%%%%%%%%%%%%%%%%%%%%%%%%%%%%%%%%%%%%%%%%%%%%%%%%%%%
\subsection{Flags}
\label{sec:flags}

The package makes it easy to generate different versions
of the main or child documents.
To this end compilation flags can be defined
and assigned different default values.
They will be particularly useful in conjunction
with the forwarding mechanism described in \secref{sec:forward}.

For example, it may be useful to have a flag |\version|
which can be set to |draft| or |final|.
The document source will contain some conditional code
depending on the value of |\version|.
Suppose further, the flag should default to |final| for the main file
and to |draft| for child files
which is a natural assignment for editing the document.
This is achieved by placing the following code
in the preamble of the main document
(below the |\childdocmain| directive):
%
\begin{center}
\begin{tabular}{l}
|\ifchilddoc|\\
|\providecommand{\version}{draft}|\\
|\||else|\\
|\providecommand{\version}{final}|\\
|\||fi|
\end{tabular}
\end{center}
%
The definition by |\providecommand| makes sure
that previous definitions are not overwritten.
Further statements |\providecommand{\version}{...}|
can thus be added before the above code to override it.

For the main file, one might add a line
(between |\childdocmain| and the above block)
%
\begin{center}
|%\ifchilddoc\||else\providecommand{\version}{draft}\||fi|
\end{center}
%
which can be uncommented to produce a draft version.
Likewise one can add a line to the very top of a child file
(above the |\childdocof{|\textit{main}|}| directive)
%
\begin{center}
|%\providecommand{\version}{final}|
\end{center}
%
which can be uncommented to produce the final version of this child document.

%%%%%%%%%%%%%%%%%%%%%%%%%%%%%%%%%%%%%%%%%%%%%%%%%%%%%%%%%%%%%%%%%%%%%%%%%%%%%%%%
\subsection{Forwarding}
\label{sec:forward}

Different versions of the main or child documents
using compilation flags as described in \secref{sec:flags}
can be (permanently) stored in different files
for convenient compilation, viewing and distribution.
To this end, the package defines a command
to pass on compilation to a different file:

%%%%%%%%%%%%%%%%%%%%%%%%%%%%%%%%%%%%%%%%
\DescribeMacro{\childdocforward}
The command |\childdocforward| redirects processing to
another source file:
%
\begin{center}
\begin{tabular}{l}
|\input{childdoc.def}|\\
|\childdocforward[|\textit{main}|]{|\textit{dest}|}|\\
\end{tabular}
\end{center}
%
The argument \textit{dest} is the destination file
(without extension).
It should be the main file or one of the child files.
Note that further \textsf{childdoc} directives
such as |\childdocof| and |\childdocforward|
in the indicated file will be processed in this form.
The optional argument \textit{main}
passes on directly to the main file \textit{main}
while pretending to compile the child \textit{dest}.
This form behaves as if \textit{dest}
issues |\childdocof{|\textit{main}|}| right away,
and no further \textsf{childdoc} directives will be processed.

%%%%%%%%%%%%%%%%%%%%%%%%%%%%%%%%%%%%%%%%
\DescribeMacro{\...prefix}
In the alternative form |\childdocforwardprefix|,
%
\begin{center}
\begin{tabular}{l}
|\input{childdoc.def}|\\
|\childdocforwardprefix[|\textit{main}|]{|\textit{prefix}|}{|\textit{dest}|}|
\end{tabular}
\end{center}
%
the destination file is determined by a pattern
depending on the current file:
To make this work, the current file must be called
`{\textit{prefix}\hspace{0.2em}\textit{suffix}}'
with \textit{prefix} matching precisely the argument.
Processing is then passed on to the file
`{\textit{dest}\hspace{0.2em}\textit{suffix}}'.
Surely, the same effect is achieved by
directly specifying the
argument `{\textit{dest}\hspace{0.2em}\textit{suffix}}'
in the first form.
However, that requires to set up a different file
for each child. With the alternative form of the command
all these files can have exactly the same content
which simplifies setting them up and maintaining them.

For example, the following file |draft.tex|
with a compilation flag |\version| as described in \secref{sec:flags}
compiles the main document as a draft:
%
\begin{center}
\begin{tabular}{l}
|\def\version{draft}|\\
|\input{childdoc.def}|\\
|\childdocforward{|\textit{main}|}|
\end{tabular}
\end{center}
%
Likewise, the following files |final|\textit{nn}|.tex|
compile the final version of the child document
|child|\textit{nn}|.tex|:
%
\begin{center}
\begin{tabular}{l}
|\def\version{final}|\\
|\input{childdoc.def}|\\
|\childdocforwardprefix{final}{child}|
\end{tabular}
\end{center}
%

Note that when several versions of a main file and/or of each child file
are to be generated, it may be convenient to set up a |Makefile| or
shell script to automatise the process.

%%%%%%%%%%%%%%%%%%%%%%%%%%%%%%%%%%%%%%%%%%%%%%%%%%%%%%%%%%%%%%%%%%%%%%%%%%%%%%%%
\subsection{Command Line Processing}
\label{sec:commandline}

The effect of redirection files can also be achieved by invoking
the \LaTeX{} compiler with a more elaborate command line.
Most conveniently this should be done as part
of a shell script or a |Makefile|.

When using \textsf{childdoc} in the main file, the following
command lines effectively perform a redirection
(note that depending on the shell being used,
backslashes may have to be doubled: `|\|' $\to$ `|\\|'):
%
\begin{center}
|... -jobname "|\textit{target}|" |\\|"|[\textit{flags}]%
|\input{childdoc.def}\childdocforward[|\textit{main}|]{|\textit{dest}|}"|
\end{center}
%
Here \textit{target} is the name of the output file,
\textit{main} is the name of the main file
and \textit{dest} is the name of the main or child file to be processed
(all filenames without extensions).
The optional argument \textit{main} can be omitted
if \textit{main} matches \textit{dest}.
Optionally, compilation \textit{flags} can be defined via |\def| commands.
This command line makes the \TeX{} engine believe
it is compiling the file \textit{target}
whose content is specified as the latter parameter.
The provided code then forwards the processing to
\textit{main} or \textit{dest} as described in \secref{sec:forward}.

%%%%%%%%%%%%%%%%%%%%%%%%%%%%%%%%%%%%%%%%%%%%%%%%%%%%%%%%%%%%%%%%%%%%%%%%%%%%%%%%
\subsection{Include by Input}
\label{sec:input}

Including child documents by |\include| has some restrictions by design.
Most notably, the content of a child document always occupies
its own set of pages; pages cannot be shared between child documents.
Usually, this behaviour makes perfect sense
because each child document contain an essential part of the document.
However, in some situations it may be desirable to compose
a document from a collection of parts
without having mandatory page breaks between then.
For this case, the package
provides a mechanism to include parts
by |\input| which can also be processed individually.
However, by construction this mechanism
requires manual handling of the content to be output.

%%%%%%%%%%%%%%%%%%%%%%%%%%%%%%%%%%%%%%%%
\DescribeMacro{\ifchilddocmanual}
The main file should be prepared as usual, see \secref{sec:include}.
However, the document body must make a distinction
between processing of an individual part and of the main document, e.g.:
%
\begin{center}
\begin{tabular}{l}
|\ifchilddocmanual|\\
|\input{\childdocname}|\\
|\||else|\\
\textit{document body with }|\input{|\textit{part}|}|\\
|\||fi|
\end{tabular}
\end{center}
%
The conditional |\ifchilddocmanual| is true whenever
a part to be included by |\input| is being compiled,
and the name of the part is stored in |\childdocname|.

%%%%%%%%%%%%%%%%%%%%%%%%%%%%%%%%%%%%%%%%
\DescribeMacro{\childdocby}
Each part to be included by |\input| should start with:
%
\begin{center}
\begin{tabular}{l}
|\input{childdoc.def}|\\
|\childdocby{|\textit{main}|}|\\
\end{tabular}
\end{center}
%
The directive |\childdocby| is similar to |\childdocof|
described in \secref{sec:include},
but the subsequent selection of content must be done manually.
To that end, both |\ifchilddoc| and |\ifchilddocmanual|
will be true upon processing of a part,
and the name of the part is stored in |\childdocname|.
Note that |\jobname| will be set to the filename of the current part
so that each part receives an individual |.aux| file
that does not interfere with the |.aux| file(s) of the main document.
This behaviour can be altered by the alternative form
|\childdocby[*]{|\textit{main}|}| (with a non-empty optional argument)
which uses the |.aux| file of the main document
by setting |\jobname| to \textit{main}.

%%%%%%%%%%%%%%%%%%%%%%%%%%%%%%%%%%%%%%%%%%%%%%%%%%%%%%%%%%%%%%%%%%%%%%%%%%%%%%%%
\subsection{Driver Development}
\label{sec:driver}

The \textsf{childdoc} mechanism can also be use for the development
of definition files such as \LaTeX{} styles or classes.
This case differs from the above setup with multiple parts
included by |\include| in that no |\includeonly| should be invoked.
This can be achieved by starting the include file
(before |\ProvidesPackage|) with:
%
\begin{center}
\begin{tabular}{l}
|\input{childdoc.def}|\\
|\childdocforward{|\textit{main}|}|\\
\end{tabular}
\end{center}
%
or alternatively with:
%
\begin{center}
\begin{tabular}{l}
|\input{childdoc.def}|\\
|\childdocby{|\textit{main}|}|\\
\end{tabular}
\end{center}
%
Both forms have slightly different effects as described above.
The main file is prepared as usual, see \secref{sec:include}.

%%%%%%%%%%%%%%%%%%%%%%%%%%%%%%%%%%%%%%%%%%%%%%%%%%%%%%%%%%%%%%%%%%%%%%%%%%%%%%%%
\subsection{Legacy Detection}
\label{sec:detection}

The directive |\childdocmain| in the main file can detect
whether the complete document or merely a child is to be compiled
even without using the directive |\childdocof|.
This method is deprecated because it is less robust
and there is no compelling reason to use it;
it is merely provided for backward compatibility
and it may be removed in future versions.

If the detection mechanism is to be used,
it is mandatory to correctly specify
the filename of the main file as the argument of |\childdocmain|:
%
\begin{center}
\begin{tabular}{l}
|\input{childdoc.def}|\\
|\childdocmain{|\textit{main}|}|\\
\end{tabular}
\end{center}
%
If |\jobname| does not match the argument \textit{main} of |\childdocmain|,
it is assumed that |\jobname| points to the child file to be compiled.
When using |\childdocmain| with the main file specified as argument,
it suffices to start a child file
with just |\input{|\textit{main}|}|
without loading of the package and using |\childdocof|.
If instead all processing is done
with the appropriate \textsf{childdoc} directives,
the argument of \textit{main} of |\childdocmain| can be empty.

An alternative version of the command line processing described
in \secref{sec:commandline} using the detection mechanism reads:
%
\begin{center}
|... -jobname "|\textit{target}|" "|[\textit{flags}]%
[|\def\jobname{|\textit{dest}|}|]|\input{|\textit{main}|}"|
\end{center}

%%%%%%%%%%%%%%%%%%%%%%%%%%%%%%%%%%%%%%%%%%%%%%%%%%%%%%%%%%%%%%%%%%%%%%%%%%%%%%%%
\subsection{Manual Code}
\label{sec:manual}

In case one cannot be certain whether the definitions file |childdoc.def|
is installed on the target \TeX{} distribution
and one prefers not to ship it,
it is conceivable to paste a few relevant commands into the sources.

To that end, drop all statements |\input{childdoc.def}|
and perform the replacements as outlined below.
Instead of |\childdocmain{|\textit{main}|}| add the following code
to the top of the main file:
%
\begin{center}
\begin{tabular}{l}
|\||ifdefined\childdocname\endinput\||fi\newif\ifchilddoc|\\
|\edef\childdocname{\scantokens\expandafter{\jobname\noexpand}}|\\
|\def\childdocmain{|\textit{main}|}\||ifx\childdocmain\childdocname\||else|\\
|\childdoctrue\includeonly{\childdocname}\let\jobname\childdocmain\||fi|\\
\end{tabular}
\end{center}
%
Instead of |\childdocof{|\textit{main}|}| just include the main file
at the top of each child file:
%
\begin{center}
|\input{|\textit{main}|}|
\end{center}
%
A simple redirection |\childdocforward{|\textit{dest}|}| is achieved by:
%
\begin{center}
|\def\jobname{|\textit{dest}|}\input{\jobname}|
\end{center}
%
The redirection with prefix
|\childdocforwardprefix[|\textit{prefix}|]{|\textit{dest}|}|
is accomplished by:
%
\begin{center}
\begin{tabular}{l}
|{\edef\jobname{\scantokens\expandafter{\jobname\noexpand}}|\\
|\def\redirectjob |\textit{prefix}|#1~~~{\gdef\jobname{|\textit{dest}|#1}}|\\
|\expandafter\redirectjob\jobname~~~}\input{\jobname}|
\end{tabular}
\end{center}

In an alternative approach,
child documents can be compiled by a specific command line
without additional code or specific definitions:
%
\begin{center}
|... -jobname "|\textit{target}|" "|[\textit{flags}]%
|\includeonly{|\textit{dest}|}\input{|\textit{main}|}"|
\end{center}
%

%%%%%%%%%%%%%%%%%%%%%%%%%%%%%%%%%%%%%%%%%%%%%%%%%%%%%%%%%%%%%%%%%%%%%%%%%%%%%%%%
%%%%%%%%%%%%%%%%%%%%%%%%%%%%%%%%%%%%%%%%%%%%%%%%%%%%%%%%%%%%%%%%%%%%%%%%%%%%%%%%
\section{Information}

%%%%%%%%%%%%%%%%%%%%%%%%%%%%%%%%%%%%%%%%%%%%%%%%%%%%%%%%%%%%%%%%%%%%%%%%%%%%%%%%
\subsection{Copyright}

Copyright \copyright{} 2017--2018 Niklas Beisert

This work may be distributed and/or modified under the
conditions of the \LaTeX{} Project Public License, either version 1.3
of this license or (at your option) any later version.
The latest version of this license is in
  \url{http://www.latex-project.org/lppl.txt}
and version 1.3 or later is part of all distributions of \LaTeX{}
version 2005/12/01 or later.

This work has the LPPL maintenance status `maintained'.

The Current Maintainer of this work is Niklas Beisert.

This work consists of the files |README.txt|, |childdoc.ins| and |childdoc.dtx|
as well as the derived files |childdoc.def|, |cdocsamp.tex|
with |cdocsch1.tex|, |cdocsch2.tex|, |cdocspt3.tex|, |cdocspt4.tex|,
|cdocsdrf.tex|, |cdocsfn1.tex|, |cdocsfn2.tex|
as well as |childdoc.pdf|.

%%%%%%%%%%%%%%%%%%%%%%%%%%%%%%%%%%%%%%%%%%%%%%%%%%%%%%%%%%%%%%%%%%%%%%%%%%%%%%%%
\subsection{Files and Installation}

The package consists of the files:
%
\begin{center}
\begin{tabular}{ll}
    |README.txt|   & readme file \\
    |childdoc.ins| & installation file \\
    |childdoc.dtx| & source file \\
    |childdoc.def| & definition file \\
    |cdocsamp.tex| & sample main file \\
    |cdocsch1.tex| & sample include file \\
    |cdocsch2.tex| & sample include file \\
    |cdocspt3.tex| & sample part file \\
    |cdocspt4.tex| & sample part file \\
    |cdocsdrf.tex| & sample redirection file \\
    |cdocsfn1.tex| & sample redirection file \\
    |cdocsfn2.tex| & sample redirection file \\
    |childdoc.pdf| & manual
\end{tabular}
\end{center}
%
The distribution consists of the files
|README.txt|, |childdoc.ins| and |childdoc.dtx|.
%
\begin{itemize}
\item
Run (pdf)\LaTeX{} on |childdoc.dtx|
to compile the manual |childdoc.pdf| (this file).
\item
Run \LaTeX{} on |childdoc.ins| to create the definitions file |childdoc.def|
and the sample |cdocsamp.tex| with include files
|cdocsch1.tex|, |cdocsch2.tex|, |cdocspt3.tex|, |cdocspt4.tex|,
|cdocsdrf.tex|, |cdocsfn1.tex|, |cdocsfn2.tex|.
Then copy the file |childdoc.def| to an appropriate directory of your \LaTeX{}
distribution, e.g.\ \textit{texmf-root}|/tex/latex/childdoc|.
\end{itemize}

%%%%%%%%%%%%%%%%%%%%%%%%%%%%%%%%%%%%%%%%%%%%%%%%%%%%%%%%%%%%%%%%%%%%%%%%%%%%%%%%
\subsection{Related CTAN Packages}

There are several other packages which offer a similar functionality:
%
\begin{itemize}
\item
The packages
\href{http://ctan.org/pkg/docmute}{\textsf{docmute}},
\href{http://ctan.org/pkg/includex}{\textsf{includex}} and
\href{http://ctan.org/pkg/standalone}{\textsf{standalone}}
provide commands to include only the document body of
a child file thus allowing both files to be compiled individually.
\item
The packages \href{http://ctan.org/pkg/subdocs}{\textsf{subdocs}}
and \href{http://ctan.org/pkg/subfiles}{\textsf{subfiles}}
provide structures in which the main and child documents can be
encapsulated and allowing them to be compiled individually.
The inclusion mechanism is different from the conventional |\include|.
\item
The package \href{http://ctan.org/pkg/combine}{\textsf{combine}}
is an elaborate solution to combine several documents into one.
\end{itemize}
%
See also the CTAN topic \href{http://ctan.org/topic/subdocs}{\textsf{subdocs}}
for further related packages.
The present package differs from the above solutions in that
a document structure constructed with the conventional |\include| mechanism
just needs two extra commands at the top of every file
such that all constituent files can be compiled individually.

%%%%%%%%%%%%%%%%%%%%%%%%%%%%%%%%%%%%%%%%%%%%%%%%%%%%%%%%%%%%%%%%%%%%%%%%%%%%%%%%
%\subsection{Feature Suggestions}
%
%The following is a list of features which may be useful for future
%versions of this package:
%%
%\begin{itemize}
%\item
%\ldots
%\end{itemize}

%%%%%%%%%%%%%%%%%%%%%%%%%%%%%%%%%%%%%%%%%%%%%%%%%%%%%%%%%%%%%%%%%%%%%%%%%%%%%%%%
\subsection{Revision History}

%%%%%%%%%%%%%%%%%%%%%%%%%%%%%%%%%%%%%%%%
\paragraph{v2.0:} 2018/12/30

\begin{itemize}
\item
immediate forward processing
\item
added |\childdocby| mechanism
\item
manual restructured
\end{itemize}

%%%%%%%%%%%%%%%%%%%%%%%%%%%%%%%%%%%%%%%%
\paragraph{v1.6:} 2018/01/17

\begin{itemize}
\item
application for development of include files
\item
corrections to manual
\end{itemize}

%%%%%%%%%%%%%%%%%%%%%%%%%%%%%%%%%%%%%%%%
\paragraph{v1.5:} 2017/05/21

\begin{itemize}
\item
more complete structuring introduced
\item
|\childdocof| introduced
\item
|\childdoc| renamed to |\childdocmain|
\item
|\childredirect| renamed to |\childdocforward| and |\childdocforwardprefix|
and functionality expanded
\end{itemize}

%%%%%%%%%%%%%%%%%%%%%%%%%%%%%%%%%%%%%%%%
\paragraph{v1.0:} 2017/04/27

\begin{itemize}
\item
manual and install package
\item
first version published on CTAN
\end{itemize}

%%%%%%%%%%%%%%%%%%%%%%%%%%%%%%%%%%%%%%%%
\paragraph{v0.6:} 2017/04/26

\begin{itemize}
\item
redirection mechanism added
\end{itemize}

%%%%%%%%%%%%%%%%%%%%%%%%%%%%%%%%%%%%%%%%
\paragraph{v0.5:} 2017/04/26

\begin{itemize}
\item
functionality in definition file
\end{itemize}


%%%%%%%%%%%%%%%%%%%%%%%%%%%%%%%%%%%%%%%%%%%%%%%%%%%%%%%%%%%%%%%%%%%%%%%%%%%%%%%%
%%%%%%%%%%%%%%%%%%%%%%%%%%%%%%%%%%%%%%%%%%%%%%%%%%%%%%%%%%%%%%%%%%%%%%%%%%%%%%%%
%%%%%%%%%%%%%%%%%%%%%%%%%%%%%%%%%%%%%%%%%%%%%%%%%%%%%%%%%%%%%%%%%%%%%%%%%%%%%%%%
\appendix

\settowidth\MacroIndent{\rmfamily\scriptsize 000\ }

 \DocInput{childdoc.dtx}

\end{document}
%</driver>
% \fi
%
% %%%%%%%%%%%%%%%%%%%%%%%%%%%%%%%%%%%%%%%%%%%%%%%%%%%%%%%%%%%%%%%%%%%%%%%%%%%%%%
% %%%%%%%%%%%%%%%%%%%%%%%%%%%%%%%%%%%%%%%%%%%%%%%%%%%%%%%%%%%%%%%%%%%%%%%%%%%%%%
% \section{Sample}
%\iffalse
%<*samplemain>
%\fi
%
% The following presents a sample document
% with two chapters, two parts, a title page,
% a compile flag as well as three forwarding files to set the flag.
% It consists of eight |.tex| files:
% \begin{center}
% \begin{tabular}{ll}
% |cdocsamp.tex|&main file\\
% |cdocsch1.tex|&include file for chapter 1\\
% |cdocsch2.tex|&include file for chapter 2\\
% |cdocspt3.tex|&include file for part 3\\
% |cdocspt4.tex|&include file for part 4\\
% |cdocsdrf.tex|&forwarding file for main file in draft mode\\
% |cdocsfi1.tex|&forwarding file for final version of chapter 1\\
% |cdocsfi2.tex|&forwarding file for final version of chapter 2\\
% \end{tabular}
% \end{center}
% Each of the eight files can be compiled directly by the \LaTeX{} compiler.
%
% %%%%%%%%%%%%%%%%%%%%%%%%%%%%%%%%%%%%%%
% \paragraph{Main File.}
%
% The main file is called |cdocsamp.tex|.
%
% Load the \textsf{childdoc} definitions and
% declare the filename for the main document:
%    \begin{macrocode}
\input{childdoc.def}
\childdocmain{}
%    \end{macrocode}

% Optional override for |\version| flag:
%    \begin{macrocode}
%%\ifchilddoc\else\providecommand{\version}{draft}\fi
%    \end{macrocode}

% Define the default values for the |\version| flag
% (|final| for the main file and |draft| for childs):
%    \begin{macrocode}
\ifchilddoc
\providecommand{\version}{draft}
\else
\providecommand{\version}{final}
\fi
%    \end{macrocode}

% Load the standard document class:
%    \begin{macrocode}
\documentclass[12pt]{article}
%    \end{macrocode}

% Start the document body:
%    \begin{macrocode}
\begin{document}
%    \end{macrocode}

% Declare a title page.
% Print title, part of document being processed and version flag:
%    \begin{macrocode}
\addtocounter{page}{-1}
\begin{center}
{\LARGE\bfseries{}childdoc example\par}
\vspace{1cm}
\ifchilddoc
\ifchilddocmanual part\else chapter\fi:
`\childdocname' of `\childdocjob'\par
\else
main document: `\childdocjob'\par
\fi
version: \version\par
\end{center}
\newpage
%    \end{macrocode}

% Manually include selected file,
% otherwise process as usual:
%    \begin{macrocode}
\ifchilddocmanual
\section*{part `\childdocname'}
\input{\childdocname}
\else
%    \end{macrocode}

% Include the two chapters:
%    \begin{macrocode}
\include{cdocsch1}
\include{cdocsch2}
%    \end{macrocode}

% Include the two parts unless only chapters should be displayed:
%    \begin{macrocode}
\ifchilddoc\else
\section{part three}
\input{cdocspt3}
\section{part four}
\input{cdocspt4}
\fi
%    \end{macrocode}

% Process as usual until here:
%    \begin{macrocode}
\fi
%    \end{macrocode}

% End of document body:
%    \begin{macrocode}
\end{document}
%    \end{macrocode}
%\iffalse
%</samplemain>
%\fi
%
% %%%%%%%%%%%%%%%%%%%%%%%%%%%%%%%%%%%%%%
% \paragraph{Chapter Include Files.}
%
% The include files are called |cdocsch1.tex| and |cdocsch2.tex|.
%
%\iffalse
%<*samplechap1|samplechap2>
%\fi

% Optional override for |\version| flag:
%    \begin{macrocode}
%%\providecommand{\version}{final}
%    \end{macrocode}

% Include the main document:
%    \begin{macrocode}
\input{childdoc.def}
\childdocof{cdocsamp}
%    \end{macrocode}

%\iffalse
%</samplechap1|samplechap2>
%\fi
%
%\iffalse
%<*samplechap1>
%\fi
% Some text for chapter 1:
%    \begin{macrocode}
\section{one}
some text in chapter one
%    \end{macrocode}

%\iffalse
%</samplechap1>
%\fi
% Some text for chapter 2:
%\iffalse
%<*samplechap2>
%\fi
%    \begin{macrocode}
\section{two}
more text in chapter two
%    \end{macrocode}

%\iffalse
%</samplechap2>
%\fi
%
% %%%%%%%%%%%%%%%%%%%%%%%%%%%%%%%%%%%%%%
% \paragraph{Part Include Files.}
%
% The include files are called |cdocspt3.tex| and |cdocspt4.tex|.
%
%\iffalse
%<*samplepart3|samplepart4>
%\fi

% Optional override for |\version| flag:
%    \begin{macrocode}
%%\providecommand{\version}{final}
%    \end{macrocode}

% Include the main document:
%    \begin{macrocode}
\input{childdoc.def}
\childdocby{cdocsamp}
%    \end{macrocode}

%\iffalse
%</samplepart3|samplepart4>
%\fi
%
%\iffalse
%<*samplepart3>
%\fi
% Some text for part 3:
%    \begin{macrocode}
some text in part three
%    \end{macrocode}

%\iffalse
%</samplepart3>
%\fi
% Some text for part 4:
%\iffalse
%<*samplepart4>
%\fi
%    \begin{macrocode}
more text in part four
%    \end{macrocode}

%\iffalse
%</samplepart4>
%\fi
%
% %%%%%%%%%%%%%%%%%%%%%%%%%%%%%%%%%%%%%%
% \paragraph{Forwarding for a Complete Draft.}
%
% The following forwarding file |cdocsdrf.tex|
% compiles the main document in draft mode:
%\iffalse
%<*sampledraft>
%\fi
%    \begin{macrocode}
\def\version{draft}
\input{childdoc.def}
\childdocforward{cdocsamp}
%    \end{macrocode}

%\iffalse
%</sampledraft>
%\fi
%
% %%%%%%%%%%%%%%%%%%%%%%%%%%%%%%%%%%%%%%
% \paragraph{Forwarding for Final Version of the Chapters.}
%
% The following forwarding files |cdocsfn1.tex| and |cdocsfn2.tex|
% (with identical content)
% compile the final versions of the child documents
% |cdocsch1.tex| and |cdocsch2.tex|, respectively:
%\iffalse
%<*samplefinal>
%\fi
%    \begin{macrocode}
\def\version{final}
\input{childdoc.def}
\childdocforwardprefix[cdocsamp]{cdocsfn}{cdocsch}
%    \end{macrocode}

%\iffalse
%</samplefinal>
%\fi
%
% %%%%%%%%%%%%%%%%%%%%%%%%%%%%%%%%%%%%%%
% \paragraph{Command Line Processing.}
%
% The following three command lines generate the output files
% |cdocscld|, |cdocscl1| and |cdocscl2|
% which should be identical to
% |cdocsdrf|, |cdocsch1| and |cdocsfn2|, respectively:
% \begin{center}
% \begin{tabular}{l}
% |latex -jobname cdocscld \|\\
% |  "\def\version{draft}\input{childdoc.def}\childdocforward{cdocsamp}"|\\
% |latex -jobname cdocscl1 \|\\
% |  "\input{childdoc.def}\childdocforward[cdocsamp]{cdocsch1}"|\\
% |latex -jobname cdocscl2 \|\\
% |  "\def\version{final}\input{childdoc.def}\childdocforward{cdocsch2}"|
% \end{tabular}
% \end{center}
% Note that the trailing backslash on each first line
% merely continues the input to the second line
% (for convenient cut ant paste).
% Furthermore, the command |latex| can be replaced by any
% of its alternative versions such as |pdflatex|.
%
% %%%%%%%%%%%%%%%%%%%%%%%%%%%%%%%%%%%%%%%%%%%%%%%%%%%%%%%%%%%%%%%%%%%%%%%%%%%%%%
% %%%%%%%%%%%%%%%%%%%%%%%%%%%%%%%%%%%%%%%%%%%%%%%%%%%%%%%%%%%%%%%%%%%%%%%%%%%%%%
% \section{Implementation}
%\iffalse
%<*package>
%\fi
%
% This section describes the definitions file |childdoc.def|.

% The definitions cannot be loaded using |\usepackage| or |\RequirePackage|
% which has a mechanism to prevent loading a style file more than once.
% When loading the definitions by means of |\input|
% multiple instances have to be prevented manually:
%\iffalse
%This code needs to be before the `\ProvidesFile' directive
%which is defined at the beginning of this file.
%Therefore it is also placed there and commented out here.
%</package>
%<*discard>
%\fi
%    \begin{macrocode}
\ifdefined\childdocmain\endinput\fi
%    \end{macrocode}
%\iffalse
%</discard>
%<*package>
%\fi
%
% \macro{\ifchilddoc}
% \macro{\ifchilddocmanual}
% The conditional |\ifchilddoc| tells whether a
% child (true) or main (false) document is being compiled.
% The conditional |\ifchilddocmanual| tells whether
% the |\includeonly| mechanism is used (false) or
% the selection of child files must be performed manually (true).
% The definitions initialise to false:
%    \begin{macrocode}
\newif\ifchilddoc
\newif\ifchilddocmanual
%    \end{macrocode}

% \macro{\childdocname}
% \macro{\childdocjob}
% The macro |\childdocname| stores the name of the main document
% to be compiled. The macro |\childdocjob| stores the name of
% the document on which the \LaTeX{} compiler was originally invoked.
% The content of |\jobname| cannot be compared
% to filenames specified in the source due to different catcodes.
% The following code rescans |\jobname|, stores the result
% in |\childdocname| and saves a copy in |\childdocjob|:
%    \begin{macrocode}
\edef\childdocname{\scantokens\expandafter{\jobname\noexpand}}
\let\childdocjob\childdocname
%    \end{macrocode}

% \macro{\childdocdisable}
% The macro |\childdocdisable| prevents the main file
% from being processed more than once.
% At this stage, the main document command |\childdocmain|
% is assumed to be called once again where it should do nothing.
% Any subsequent call to it should prevent
% a secondary processing of the main document
% It overwrites the forwarding commands
% |\childdocof| and |\childdocforward|
% with empty macros to prevent further inclusions of the main document:
%    \begin{macrocode}
\newcommand{\childdocdisable}
{
  \renewcommand{\childdocmain}[1]{\renewcommand{\childdocmain}[1]{\endinput}}
  \renewcommand{\childdocof}[1]{}
  \renewcommand{\childdocby}[2][]{}
  \renewcommand{\childdocforward}[2][]{}
  \renewcommand{\childdocdisable}{}
}
%    \end{macrocode}

% \macro{\childdocmain}
% The macro |\childdocmain| is to be called at the top of the main file
% with nothing or the main filename (without extension) as argument.
% First, it breaks loops.
% If the argument is not empty and does not match |\childdocname|
% (which is set by the first inclusion of |childdoc.def|),
% |\ifchilddoc| is set to true, |\includeonly| is applied to the child file
% and |\jobname| is set to the main file
% (for proper handling of |.aux| files):
%    \begin{macrocode}
\newcommand{\childdocmain}[1]
{
  \childdocdisable\childdocmain{}
  \if?#1?\else
    \begingroup
      \def\childdoctmp{#1}
      \ifx\childdoctmp\childdocname
        \def\childdoctmp{}
      \else
        \def\childdoctmp
        {
          \childdoctrue
          \includeonly{\childdocname}
          \def\childdocjob{#1}
          \def\jobname{#1}
        }
      \fi
      \expandafter
    \endgroup
    \childdoctmp
  \fi
}
%    \end{macrocode}

% \macro{\childdocof}
% The command |\childdocof| redirects
% compilation to the main file |#1|.
%    \begin{macrocode}
\newcommand{\childdocof}[1]
{
  \childdocdisable
  \childdoctrue
  \includeonly{\childdocname}
  \def\jobname{#1}
  \def\childdocjob{#1}
  \input{#1}
}
%    \end{macrocode}

% \macro{\childdocby}
% The command |\childdocby| ....
%    \begin{macrocode}
\newcommand{\childdocby}[2][]
{
  \childdocdisable
  \childdoctrue
  \childdocmanualtrue
  \if?#1?\else
    \def\jobname{#2}
  \fi
  \def\childdocjob{#2}
  \input{#2}
  \endinput
}
%    \end{macrocode}

% \macro{\childdocforward}
% The command |\childdocforward| redirects
% compilation to the main file or
% (if the optional argument is given) a child file.
% Parameters are set as if the main file
% or a child file starting with |\childdocof| was compiled.
% Then compilation is handed over to the main file:
%    \begin{macrocode}
\newcommand{\childdocforward}[2][]
{
  \begingroup
    \if?#1?
      \def\childdoctmp
      {
        \def\childdocname{#2}
        \def\childdocjob{#2}
        \def\jobname{#2}
        \input{#2}
        \endinput
      }
    \else
      \def\childdoctmp
      {
        \childdocdisable
        \def\childdocname{#2}
        \childdoctrue
        \includeonly{#2}
        \def\childdocjob{#1}
        \def\jobname{#1}
        \input{#1}
        \endinput
      }
    \fi
    \expandafter
  \endgroup
  \childdoctmp
}
%    \end{macrocode}

% \macro{\childdocforwardprefix}
% The command |\childdocforwardprefix| redirects
% compilation to the main or a child file by means of a pattern.
% The prefix |#1| in the current filename is replaced by |#2|
% and the suffix of the current filename is kept
% (it is assumed that the filename does not contain the substring `|~~~|'
% which is used as a delimiter).
% Compilation is handed over to the new file by |\childdocforward|:
%    \begin{macrocode}
\newcommand{\childdocforwardprefix}[3][]
{
  \begingroup
    \def\childdocextract #2##1~~~{\def\childdoctmp{\childdocforward[#1]{#3##1}}}
    \expandafter\childdocextract\childdocname~~~
    \expandafter
  \endgroup
  \childdoctmp
}
%    \end{macrocode}

% \macro{\childdoc}
% The deprecated macro |\childdoc| is a legacy version of |\childdocmain|:
%    \begin{macrocode}
\newcommand{\childdoc}{\childdocmain}
%    \end{macrocode}

% \macro{\childdocredirect}
% The deprecated macro |\childdocredirect| is a legacy version
% of |\childdocforward| and |\childdocforwardprefix|:
%    \begin{macrocode}
\newcommand{\childdocredirect}[2][]
{
  \begingroup
    \if?#1?
      \def\childdoctmp{\childdocforward{#2}}
    \else
      \def\childdoctmp{\childdocforwardprefix{#1}{#2}}
    \fi
    \expandafter
  \endgroup
  \childdoctmp
}
%    \end{macrocode}

%\iffalse
%</package>
%\fi
%
\endinput
|\\
|\childdocmain{|\textit{main}|}|\\
\end{tabular}
\end{center}
%
If |\jobname| does not match the argument \textit{main} of |\childdocmain|,
it is assumed that |\jobname| points to the child file to be compiled.
When using |\childdocmain| with the main file specified as argument,
it suffices to start a child file
with just |\input{|\textit{main}|}|
without loading of the package and using |\childdocof|.
If instead all processing is done
with the appropriate \textsf{childdoc} directives,
the argument of \textit{main} of |\childdocmain| can be empty.

An alternative version of the command line processing described
in \secref{sec:commandline} using the detection mechanism reads:
%
\begin{center}
|... -jobname "|\textit{target}|" "|[\textit{flags}]%
[|\def\jobname{|\textit{dest}|}|]|\input{|\textit{main}|}"|
\end{center}

%%%%%%%%%%%%%%%%%%%%%%%%%%%%%%%%%%%%%%%%%%%%%%%%%%%%%%%%%%%%%%%%%%%%%%%%%%%%%%%%
\subsection{Manual Code}
\label{sec:manual}

In case one cannot be certain whether the definitions file |childdoc.def|
is installed on the target \TeX{} distribution
and one prefers not to ship it,
it is conceivable to paste a few relevant commands into the sources.

To that end, drop all statements |% \iffalse
%
% childdoc.dtx Copyright (C) 2017-2018 Niklas Beisert
%
% This work may be distributed and/or modified under the
% conditions of the LaTeX Project Public License, either version 1.3
% of this license or (at your option) any later version.
% The latest version of this license is in
%   http://www.latex-project.org/lppl.txt
% and version 1.3 or later is part of all distributions of LaTeX
% version 2005/12/01 or later.
%
% This work has the LPPL maintenance status `maintained'.
%
% The Current Maintainer of this work is Niklas Beisert.
%
% This work consists of the files childdoc.dtx and childdoc.ins
% and the derived files childdoc.def and cdocsamp.tex with
% cdocsch1.tex, cdocsch2.tex, cdocsdrf.tex, cdocsfn1.tex, cdocsfn2.tex.
%
%<package>\ifdefined\childdocmain\endinput\fi
%<package>\ProvidesFile{childdoc.def}[2018/12/30 v2.0 child document driver]
%<samplemain>\ProvidesFile{cdocsamp.tex}[2018/12/30 v2.0 sample for childdoc]
%<*driver>
%\ProvidesFile{childdoc.drv}[2018/12/30 v2.0 childdoc reference manual file]
\PassOptionsToClass{10pt,a4paper}{article}
\documentclass{ltxdoc}

\usepackage[margin=35mm]{geometry}
\usepackage{hyperref}
\usepackage{hyperxmp}
\usepackage[usenames]{color}

\hypersetup{colorlinks=true}
\hypersetup{pdfstartview=FitH}
\hypersetup{pdfpagemode=UseNone}
\hypersetup{pdfsource={}}
\hypersetup{pdflang={en-UK}}
\hypersetup{pdfcopyright={Copyright 2017-2018 Niklas Beisert.
  This work may be distributed and/or modified under the
  conditions of the LaTeX Project Public License, either version 1.3
  of this license or (at your option) any later version.}}
\hypersetup{pdflicenseurl={http://www.latex-project.org/lppl.txt}}
\hypersetup{pdfcontactaddress={ETH Zurich, ITP, HIT K,
  Wolfgang-Pauli-Strasse 27}}
\hypersetup{pdfcontactpostcode={8093}}
\hypersetup{pdfcontactcity={Zurich}}
\hypersetup{pdfcontactcountry={Switzerland}}
\hypersetup{pdfcontactemail={nbeisert@itp.phys.ethz.ch}}
\hypersetup{pdfcontacturl={http://people.phys.ethz.ch/\xmptilde nbeisert/}}

\newcommand{\secref}[1]{\hyperref[#1]{section \ref*{#1}}}

\parskip1ex
\parindent0pt
\let\olditemize\itemize
\def\itemize{\olditemize\parskip0pt}

\begin{document}

\title{The \textsf{childdoc} Package}
\hypersetup{pdftitle={The childdoc Package}}
\author{Niklas Beisert\\[2ex]
  Institut f\"ur Theoretische Physik\\
  Eidgen\"ossische Technische Hochschule Z\"urich\\
  Wolfgang-Pauli-Strasse 27, 8093 Z\"urich, Switzerland\\[1ex]
  \href{mailto:nbeisert@itp.phys.ethz.ch}
  {\texttt{nbeisert@itp.phys.ethz.ch}}}
\hypersetup{pdfauthor={Niklas Beisert}}
\hypersetup{pdfsubject={Manual for the LaTeX2e Package childdoc}}
\date{30 December 2018, \textsf{v2.0}}
\maketitle

\begin{abstract}\noindent
\textsf{childdoc} is a \LaTeXe{} package
that enables the direct compilation
of document sections included by |\include|
to individual files.
\end{abstract}

\begingroup
\parskip0ex
\tableofcontents
\endgroup

%%%%%%%%%%%%%%%%%%%%%%%%%%%%%%%%%%%%%%%%%%%%%%%%%%%%%%%%%%%%%%%%%%%%%%%%%%%%%%%%
%%%%%%%%%%%%%%%%%%%%%%%%%%%%%%%%%%%%%%%%%%%%%%%%%%%%%%%%%%%%%%%%%%%%%%%%%%%%%%%%
\section{Introduction}

\LaTeX{} provides a mechanism to structure a large document (such as a book)
into a main file and several child files (containing the chapters)
using the |\include| command.
This mechanism is beneficial for documents
which span hundreds of pages in order to
make the source file(s) more manageable.
Moreover, compilation can be restricted to
selected child files by means of the |\includeonly| command.
The latter feature can be used to reduce the compilation time while editing
(this was significantly more useful in the earlier days of \LaTeX{})
or to generate a smaller document which is easier to navigate.
Another application of |\includeonly| is to generate
documents consisting of selected parts of the complete document.

However, there are a few drawbacks of the plain |\include| mechanism:
\begin{itemize}
\item
The child files cannot be compiled on their own,
they can only be compiled via the main file.
A naive editing environment
(such as a text editor with an option
to have the current file processed by \LaTeX)
may require one to switch to the main file before compiling;
attempting to compile the child file produces errors.
\item
The main file must be modified (each time)
to adjust the |\includeonly| command
to the present needs. This easily leaves the main file in a messy state.
\item
The generated document will always carry the filename
of the main document. This is inconvenient if
several child files are to be compiled and
to be kept for distribution.
\end{itemize}

The present package provides a simple interface
to make child files individually compilable by \LaTeX{}.
Compiling a child file then has the same effect as compiling
the main file with an |\includeonly| command
to select the appropriate child.
Moreover the generated document will carry the name of the child
rather than the main file.
This resolves all three above issues.

This feature is meant to make the editing of books,
thesis documents and lecture notes somewhat more convenient.
However, the package can also be used efficiently for
composing a series of documents (such as exercise sheets)
which are typically distributed individually.
It then assists the author in generating the individual documents
(potentially in different versions)
as well as a document containing the collected series.
Another application is in developing style files
or other kinds of included material
where compilation of the style file could redirect
to a sample or test file.

%%%%%%%%%%%%%%%%%%%%%%%%%%%%%%%%%%%%%%%%%%%%%%%%%%%%%%%%%%%%%%%%%%%%%%%%%%%%%%%%
%%%%%%%%%%%%%%%%%%%%%%%%%%%%%%%%%%%%%%%%%%%%%%%%%%%%%%%%%%%%%%%%%%%%%%%%%%%%%%%%
\section{Usage}

First of all, the package \textsf{childdoc} is \emph{not} a standard
\LaTeXe{} |.sty| style file! Therefore it needs to be invoked in
a non-standard way.

%%%%%%%%%%%%%%%%%%%%%%%%%%%%%%%%%%%%%%%%%%%%%%%%%%%%%%%%%%%%%%%%%%%%%%%%%%%%%%%%
\subsection{Included Files}
\label{sec:include}

%%%%%%%%%%%%%%%%%%%%%%%%%%%%%%%%%%%%%%%%
\DescribeMacro{\childdocmain}
To use the package, add the commands
\begin{center}
\begin{tabular}{l}
|\input{childdoc.def}|\\
|\childdocmain{}|\\
\end{tabular}
\end{center}
at the very top of the main \LaTeX{} file,
in particular \emph{before} the |\documentclass| statement!
The argument of |\childdocmain| should be left empty
(but it must be present).

%%%%%%%%%%%%%%%%%%%%%%%%%%%%%%%%%%%%%%%%
\DescribeMacro{\childdocof}
Furthermore, add the commands
\begin{center}
\begin{tabular}{l}
|\input{childdoc.def}|\\
|\childdocof{|\textit{main}|}|\\
\end{tabular}
\end{center}
at the top of every child file \textit{child}
which is included by |\include{|\textit{child}|}|
from within the main file
(or at least for those files to be compiled individually).
The argument \textit{main} must be the filename of the main file.

There are a couple of
considerations in setting up the main and child documents:

%%%%%%%%%%%%%%%%%%%%%%%%%%%%%%%%%%%%%%%%
\paragraph{Restrictions.}

Please note the following restrictions:
\begin{itemize}
\item
|\childdocmain| must be called with one argument \textit{main}
to ensure compatibility with earlier version of the package.
It must either be empty (|\childdocmain{}|)
or precisely match the filename of the main file in which it is specified.
See \secref{sec:detection} for further information.
\item
The filename \textit{main} must be specified without the |.tex| extension.
\item
The filename \textit{main} is case sensitive
(even in case-insensitive file systems)
due to internal string comparison.
\item
The argument \textit{main} should be fully expanded, it cannot be a macro.
\item
Subdirectories and special characters should be avoided in filenames.
\item
The command |\childdocmain{|\textit{main}|}| must be followed by a whitespace.
It should not be followed immediately by another command
or by a comment mark `|%|'.
This is because the \TeX{} parser reads the token immediately following
the argument of |\childdocmain| and puts it
at the beginning of every child section;
however, a white\-space is ignored.
\end{itemize}

%%%%%%%%%%%%%%%%%%%%%%%%%%%%%%%%%%%%%%%%
\paragraph{Content of Main File.}

It is advisable to place all content in the child files included by |\include|.
Any output contained in the main file will appear in all child documents
unless suppressed manually;
it cannot be suppressed automatically by the |\includeonly| directive
and thus should normally be avoided.
A method to include some content in the main file
by means of conditional processing is described in \secref{sec:conditional}.

%%%%%%%%%%%%%%%%%%%%%%%%%%%%%%%%%%%%%%%%
\paragraph{Page Numbering.}

When only a part of the document is compiled,
the appropriate numbering of pages
(as well as other status parameters)
is determined from the |.aux| files.
The latter contain information from previous passes.
However this information needs to propagate through
all intermediate child documents.
Therefore the page numbering in child documents may well
be inconsistent until the complete document is compiled at least once.

A useful (if unconventional) way to always ensure a consistent
page numbering is to restart the numbering in each child document
and denote the pages by `\textit{child}|.|\textit{page}'
where \textit{child} represents the chapter/section number of the child file.
This can be achieved by the command
|\numberwithin{page}{|\textit{child}|}|
of the \textsf{amsmath} package
where \textit{child} can be |chapter| or |section|
depending on the chosen structuring.
Alternatively, one can modify the macro |\thepage| appropriately
and reset the counter |page| at the start of each child file.

%%%%%%%%%%%%%%%%%%%%%%%%%%%%%%%%%%%%%%%%%%%%%%%%%%%%%%%%%%%%%%%%%%%%%%%%%%%%%%%%
\subsection{Conditional Processing}
\label{sec:conditional}

The package provides a mechanism to compile different versions
of a document. To customise the versions further some conditional processing
can come in handy to distinguish which version is being compiled.
The package provides two macros to describe the compilation context:

%%%%%%%%%%%%%%%%%%%%%%%%%%%%%%%%%%%%%%%%
\DescribeMacro{\ifchilddoc}
The conditional |\ifchilddoc| distinguishes between the compilation of
child documents and the main document:
%
\begin{center}
|\ifchilddoc |\textit{child-code}| |[|\||else |\textit{main-code}]| \||fi|
\end{center}

%%%%%%%%%%%%%%%%%%%%%%%%%%%%%%%%%%%%%%%%
\DescribeMacro{\childdocname}
\DescribeMacro{\childdocjob}
The macro |\childdocname| contains the filename (without extension)
of the main or child file being processed.
Note that |\childdocjob| will always contain the name of the main file.

%%%%%%%%%%%%%%%%%%%%%%%%%%%%%%%%%%%%%%%%
\paragraph{Title Page.}

Conditional processing can be used to include a title or banner page
in the main document when proper precautions are taken.
Importantly, the code in the main file should ensure that the page counter
(as well as other status parameters which are stored in the |.aux| files)
takes the same value after the conditional processing.
Otherwise the page numbers may take divergent values
depending on which part is compiled.

For example, a title page could be declared by:
%
\begin{center}
\begin{tabular}{l}
|\ifchilddoc\||else|\\
|\addtocounter{page}{-1}|\\
\textit{code for title page}\\
|\newpage|\\
|\||fi|
\end{tabular}
\end{center}
%
A banner page for the child documents can be generated by:
%
\begin{center}
\begin{tabular}{l}
|\ifchilddoc|\\
|\addtocounter{page}{-1}|\\
\textit{code for banner page}\\
|\newpage|\\
|\||fi|
\end{tabular}
\end{center}
%
Here one could write a message such as:
\begin{center}
|This is the part \childdocname{} of \childdocjob{}.|
\end{center}

%%%%%%%%%%%%%%%%%%%%%%%%%%%%%%%%%%%%%%%%%%%%%%%%%%%%%%%%%%%%%%%%%%%%%%%%%%%%%%%%
\subsection{Flags}
\label{sec:flags}

The package makes it easy to generate different versions
of the main or child documents.
To this end compilation flags can be defined
and assigned different default values.
They will be particularly useful in conjunction
with the forwarding mechanism described in \secref{sec:forward}.

For example, it may be useful to have a flag |\version|
which can be set to |draft| or |final|.
The document source will contain some conditional code
depending on the value of |\version|.
Suppose further, the flag should default to |final| for the main file
and to |draft| for child files
which is a natural assignment for editing the document.
This is achieved by placing the following code
in the preamble of the main document
(below the |\childdocmain| directive):
%
\begin{center}
\begin{tabular}{l}
|\ifchilddoc|\\
|\providecommand{\version}{draft}|\\
|\||else|\\
|\providecommand{\version}{final}|\\
|\||fi|
\end{tabular}
\end{center}
%
The definition by |\providecommand| makes sure
that previous definitions are not overwritten.
Further statements |\providecommand{\version}{...}|
can thus be added before the above code to override it.

For the main file, one might add a line
(between |\childdocmain| and the above block)
%
\begin{center}
|%\ifchilddoc\||else\providecommand{\version}{draft}\||fi|
\end{center}
%
which can be uncommented to produce a draft version.
Likewise one can add a line to the very top of a child file
(above the |\childdocof{|\textit{main}|}| directive)
%
\begin{center}
|%\providecommand{\version}{final}|
\end{center}
%
which can be uncommented to produce the final version of this child document.

%%%%%%%%%%%%%%%%%%%%%%%%%%%%%%%%%%%%%%%%%%%%%%%%%%%%%%%%%%%%%%%%%%%%%%%%%%%%%%%%
\subsection{Forwarding}
\label{sec:forward}

Different versions of the main or child documents
using compilation flags as described in \secref{sec:flags}
can be (permanently) stored in different files
for convenient compilation, viewing and distribution.
To this end, the package defines a command
to pass on compilation to a different file:

%%%%%%%%%%%%%%%%%%%%%%%%%%%%%%%%%%%%%%%%
\DescribeMacro{\childdocforward}
The command |\childdocforward| redirects processing to
another source file:
%
\begin{center}
\begin{tabular}{l}
|\input{childdoc.def}|\\
|\childdocforward[|\textit{main}|]{|\textit{dest}|}|\\
\end{tabular}
\end{center}
%
The argument \textit{dest} is the destination file
(without extension).
It should be the main file or one of the child files.
Note that further \textsf{childdoc} directives
such as |\childdocof| and |\childdocforward|
in the indicated file will be processed in this form.
The optional argument \textit{main}
passes on directly to the main file \textit{main}
while pretending to compile the child \textit{dest}.
This form behaves as if \textit{dest}
issues |\childdocof{|\textit{main}|}| right away,
and no further \textsf{childdoc} directives will be processed.

%%%%%%%%%%%%%%%%%%%%%%%%%%%%%%%%%%%%%%%%
\DescribeMacro{\...prefix}
In the alternative form |\childdocforwardprefix|,
%
\begin{center}
\begin{tabular}{l}
|\input{childdoc.def}|\\
|\childdocforwardprefix[|\textit{main}|]{|\textit{prefix}|}{|\textit{dest}|}|
\end{tabular}
\end{center}
%
the destination file is determined by a pattern
depending on the current file:
To make this work, the current file must be called
`{\textit{prefix}\hspace{0.2em}\textit{suffix}}'
with \textit{prefix} matching precisely the argument.
Processing is then passed on to the file
`{\textit{dest}\hspace{0.2em}\textit{suffix}}'.
Surely, the same effect is achieved by
directly specifying the
argument `{\textit{dest}\hspace{0.2em}\textit{suffix}}'
in the first form.
However, that requires to set up a different file
for each child. With the alternative form of the command
all these files can have exactly the same content
which simplifies setting them up and maintaining them.

For example, the following file |draft.tex|
with a compilation flag |\version| as described in \secref{sec:flags}
compiles the main document as a draft:
%
\begin{center}
\begin{tabular}{l}
|\def\version{draft}|\\
|\input{childdoc.def}|\\
|\childdocforward{|\textit{main}|}|
\end{tabular}
\end{center}
%
Likewise, the following files |final|\textit{nn}|.tex|
compile the final version of the child document
|child|\textit{nn}|.tex|:
%
\begin{center}
\begin{tabular}{l}
|\def\version{final}|\\
|\input{childdoc.def}|\\
|\childdocforwardprefix{final}{child}|
\end{tabular}
\end{center}
%

Note that when several versions of a main file and/or of each child file
are to be generated, it may be convenient to set up a |Makefile| or
shell script to automatise the process.

%%%%%%%%%%%%%%%%%%%%%%%%%%%%%%%%%%%%%%%%%%%%%%%%%%%%%%%%%%%%%%%%%%%%%%%%%%%%%%%%
\subsection{Command Line Processing}
\label{sec:commandline}

The effect of redirection files can also be achieved by invoking
the \LaTeX{} compiler with a more elaborate command line.
Most conveniently this should be done as part
of a shell script or a |Makefile|.

When using \textsf{childdoc} in the main file, the following
command lines effectively perform a redirection
(note that depending on the shell being used,
backslashes may have to be doubled: `|\|' $\to$ `|\\|'):
%
\begin{center}
|... -jobname "|\textit{target}|" |\\|"|[\textit{flags}]%
|\input{childdoc.def}\childdocforward[|\textit{main}|]{|\textit{dest}|}"|
\end{center}
%
Here \textit{target} is the name of the output file,
\textit{main} is the name of the main file
and \textit{dest} is the name of the main or child file to be processed
(all filenames without extensions).
The optional argument \textit{main} can be omitted
if \textit{main} matches \textit{dest}.
Optionally, compilation \textit{flags} can be defined via |\def| commands.
This command line makes the \TeX{} engine believe
it is compiling the file \textit{target}
whose content is specified as the latter parameter.
The provided code then forwards the processing to
\textit{main} or \textit{dest} as described in \secref{sec:forward}.

%%%%%%%%%%%%%%%%%%%%%%%%%%%%%%%%%%%%%%%%%%%%%%%%%%%%%%%%%%%%%%%%%%%%%%%%%%%%%%%%
\subsection{Include by Input}
\label{sec:input}

Including child documents by |\include| has some restrictions by design.
Most notably, the content of a child document always occupies
its own set of pages; pages cannot be shared between child documents.
Usually, this behaviour makes perfect sense
because each child document contain an essential part of the document.
However, in some situations it may be desirable to compose
a document from a collection of parts
without having mandatory page breaks between then.
For this case, the package
provides a mechanism to include parts
by |\input| which can also be processed individually.
However, by construction this mechanism
requires manual handling of the content to be output.

%%%%%%%%%%%%%%%%%%%%%%%%%%%%%%%%%%%%%%%%
\DescribeMacro{\ifchilddocmanual}
The main file should be prepared as usual, see \secref{sec:include}.
However, the document body must make a distinction
between processing of an individual part and of the main document, e.g.:
%
\begin{center}
\begin{tabular}{l}
|\ifchilddocmanual|\\
|\input{\childdocname}|\\
|\||else|\\
\textit{document body with }|\input{|\textit{part}|}|\\
|\||fi|
\end{tabular}
\end{center}
%
The conditional |\ifchilddocmanual| is true whenever
a part to be included by |\input| is being compiled,
and the name of the part is stored in |\childdocname|.

%%%%%%%%%%%%%%%%%%%%%%%%%%%%%%%%%%%%%%%%
\DescribeMacro{\childdocby}
Each part to be included by |\input| should start with:
%
\begin{center}
\begin{tabular}{l}
|\input{childdoc.def}|\\
|\childdocby{|\textit{main}|}|\\
\end{tabular}
\end{center}
%
The directive |\childdocby| is similar to |\childdocof|
described in \secref{sec:include},
but the subsequent selection of content must be done manually.
To that end, both |\ifchilddoc| and |\ifchilddocmanual|
will be true upon processing of a part,
and the name of the part is stored in |\childdocname|.
Note that |\jobname| will be set to the filename of the current part
so that each part receives an individual |.aux| file
that does not interfere with the |.aux| file(s) of the main document.
This behaviour can be altered by the alternative form
|\childdocby[*]{|\textit{main}|}| (with a non-empty optional argument)
which uses the |.aux| file of the main document
by setting |\jobname| to \textit{main}.

%%%%%%%%%%%%%%%%%%%%%%%%%%%%%%%%%%%%%%%%%%%%%%%%%%%%%%%%%%%%%%%%%%%%%%%%%%%%%%%%
\subsection{Driver Development}
\label{sec:driver}

The \textsf{childdoc} mechanism can also be use for the development
of definition files such as \LaTeX{} styles or classes.
This case differs from the above setup with multiple parts
included by |\include| in that no |\includeonly| should be invoked.
This can be achieved by starting the include file
(before |\ProvidesPackage|) with:
%
\begin{center}
\begin{tabular}{l}
|\input{childdoc.def}|\\
|\childdocforward{|\textit{main}|}|\\
\end{tabular}
\end{center}
%
or alternatively with:
%
\begin{center}
\begin{tabular}{l}
|\input{childdoc.def}|\\
|\childdocby{|\textit{main}|}|\\
\end{tabular}
\end{center}
%
Both forms have slightly different effects as described above.
The main file is prepared as usual, see \secref{sec:include}.

%%%%%%%%%%%%%%%%%%%%%%%%%%%%%%%%%%%%%%%%%%%%%%%%%%%%%%%%%%%%%%%%%%%%%%%%%%%%%%%%
\subsection{Legacy Detection}
\label{sec:detection}

The directive |\childdocmain| in the main file can detect
whether the complete document or merely a child is to be compiled
even without using the directive |\childdocof|.
This method is deprecated because it is less robust
and there is no compelling reason to use it;
it is merely provided for backward compatibility
and it may be removed in future versions.

If the detection mechanism is to be used,
it is mandatory to correctly specify
the filename of the main file as the argument of |\childdocmain|:
%
\begin{center}
\begin{tabular}{l}
|\input{childdoc.def}|\\
|\childdocmain{|\textit{main}|}|\\
\end{tabular}
\end{center}
%
If |\jobname| does not match the argument \textit{main} of |\childdocmain|,
it is assumed that |\jobname| points to the child file to be compiled.
When using |\childdocmain| with the main file specified as argument,
it suffices to start a child file
with just |\input{|\textit{main}|}|
without loading of the package and using |\childdocof|.
If instead all processing is done
with the appropriate \textsf{childdoc} directives,
the argument of \textit{main} of |\childdocmain| can be empty.

An alternative version of the command line processing described
in \secref{sec:commandline} using the detection mechanism reads:
%
\begin{center}
|... -jobname "|\textit{target}|" "|[\textit{flags}]%
[|\def\jobname{|\textit{dest}|}|]|\input{|\textit{main}|}"|
\end{center}

%%%%%%%%%%%%%%%%%%%%%%%%%%%%%%%%%%%%%%%%%%%%%%%%%%%%%%%%%%%%%%%%%%%%%%%%%%%%%%%%
\subsection{Manual Code}
\label{sec:manual}

In case one cannot be certain whether the definitions file |childdoc.def|
is installed on the target \TeX{} distribution
and one prefers not to ship it,
it is conceivable to paste a few relevant commands into the sources.

To that end, drop all statements |\input{childdoc.def}|
and perform the replacements as outlined below.
Instead of |\childdocmain{|\textit{main}|}| add the following code
to the top of the main file:
%
\begin{center}
\begin{tabular}{l}
|\||ifdefined\childdocname\endinput\||fi\newif\ifchilddoc|\\
|\edef\childdocname{\scantokens\expandafter{\jobname\noexpand}}|\\
|\def\childdocmain{|\textit{main}|}\||ifx\childdocmain\childdocname\||else|\\
|\childdoctrue\includeonly{\childdocname}\let\jobname\childdocmain\||fi|\\
\end{tabular}
\end{center}
%
Instead of |\childdocof{|\textit{main}|}| just include the main file
at the top of each child file:
%
\begin{center}
|\input{|\textit{main}|}|
\end{center}
%
A simple redirection |\childdocforward{|\textit{dest}|}| is achieved by:
%
\begin{center}
|\def\jobname{|\textit{dest}|}\input{\jobname}|
\end{center}
%
The redirection with prefix
|\childdocforwardprefix[|\textit{prefix}|]{|\textit{dest}|}|
is accomplished by:
%
\begin{center}
\begin{tabular}{l}
|{\edef\jobname{\scantokens\expandafter{\jobname\noexpand}}|\\
|\def\redirectjob |\textit{prefix}|#1~~~{\gdef\jobname{|\textit{dest}|#1}}|\\
|\expandafter\redirectjob\jobname~~~}\input{\jobname}|
\end{tabular}
\end{center}

In an alternative approach,
child documents can be compiled by a specific command line
without additional code or specific definitions:
%
\begin{center}
|... -jobname "|\textit{target}|" "|[\textit{flags}]%
|\includeonly{|\textit{dest}|}\input{|\textit{main}|}"|
\end{center}
%

%%%%%%%%%%%%%%%%%%%%%%%%%%%%%%%%%%%%%%%%%%%%%%%%%%%%%%%%%%%%%%%%%%%%%%%%%%%%%%%%
%%%%%%%%%%%%%%%%%%%%%%%%%%%%%%%%%%%%%%%%%%%%%%%%%%%%%%%%%%%%%%%%%%%%%%%%%%%%%%%%
\section{Information}

%%%%%%%%%%%%%%%%%%%%%%%%%%%%%%%%%%%%%%%%%%%%%%%%%%%%%%%%%%%%%%%%%%%%%%%%%%%%%%%%
\subsection{Copyright}

Copyright \copyright{} 2017--2018 Niklas Beisert

This work may be distributed and/or modified under the
conditions of the \LaTeX{} Project Public License, either version 1.3
of this license or (at your option) any later version.
The latest version of this license is in
  \url{http://www.latex-project.org/lppl.txt}
and version 1.3 or later is part of all distributions of \LaTeX{}
version 2005/12/01 or later.

This work has the LPPL maintenance status `maintained'.

The Current Maintainer of this work is Niklas Beisert.

This work consists of the files |README.txt|, |childdoc.ins| and |childdoc.dtx|
as well as the derived files |childdoc.def|, |cdocsamp.tex|
with |cdocsch1.tex|, |cdocsch2.tex|, |cdocspt3.tex|, |cdocspt4.tex|,
|cdocsdrf.tex|, |cdocsfn1.tex|, |cdocsfn2.tex|
as well as |childdoc.pdf|.

%%%%%%%%%%%%%%%%%%%%%%%%%%%%%%%%%%%%%%%%%%%%%%%%%%%%%%%%%%%%%%%%%%%%%%%%%%%%%%%%
\subsection{Files and Installation}

The package consists of the files:
%
\begin{center}
\begin{tabular}{ll}
    |README.txt|   & readme file \\
    |childdoc.ins| & installation file \\
    |childdoc.dtx| & source file \\
    |childdoc.def| & definition file \\
    |cdocsamp.tex| & sample main file \\
    |cdocsch1.tex| & sample include file \\
    |cdocsch2.tex| & sample include file \\
    |cdocspt3.tex| & sample part file \\
    |cdocspt4.tex| & sample part file \\
    |cdocsdrf.tex| & sample redirection file \\
    |cdocsfn1.tex| & sample redirection file \\
    |cdocsfn2.tex| & sample redirection file \\
    |childdoc.pdf| & manual
\end{tabular}
\end{center}
%
The distribution consists of the files
|README.txt|, |childdoc.ins| and |childdoc.dtx|.
%
\begin{itemize}
\item
Run (pdf)\LaTeX{} on |childdoc.dtx|
to compile the manual |childdoc.pdf| (this file).
\item
Run \LaTeX{} on |childdoc.ins| to create the definitions file |childdoc.def|
and the sample |cdocsamp.tex| with include files
|cdocsch1.tex|, |cdocsch2.tex|, |cdocspt3.tex|, |cdocspt4.tex|,
|cdocsdrf.tex|, |cdocsfn1.tex|, |cdocsfn2.tex|.
Then copy the file |childdoc.def| to an appropriate directory of your \LaTeX{}
distribution, e.g.\ \textit{texmf-root}|/tex/latex/childdoc|.
\end{itemize}

%%%%%%%%%%%%%%%%%%%%%%%%%%%%%%%%%%%%%%%%%%%%%%%%%%%%%%%%%%%%%%%%%%%%%%%%%%%%%%%%
\subsection{Related CTAN Packages}

There are several other packages which offer a similar functionality:
%
\begin{itemize}
\item
The packages
\href{http://ctan.org/pkg/docmute}{\textsf{docmute}},
\href{http://ctan.org/pkg/includex}{\textsf{includex}} and
\href{http://ctan.org/pkg/standalone}{\textsf{standalone}}
provide commands to include only the document body of
a child file thus allowing both files to be compiled individually.
\item
The packages \href{http://ctan.org/pkg/subdocs}{\textsf{subdocs}}
and \href{http://ctan.org/pkg/subfiles}{\textsf{subfiles}}
provide structures in which the main and child documents can be
encapsulated and allowing them to be compiled individually.
The inclusion mechanism is different from the conventional |\include|.
\item
The package \href{http://ctan.org/pkg/combine}{\textsf{combine}}
is an elaborate solution to combine several documents into one.
\end{itemize}
%
See also the CTAN topic \href{http://ctan.org/topic/subdocs}{\textsf{subdocs}}
for further related packages.
The present package differs from the above solutions in that
a document structure constructed with the conventional |\include| mechanism
just needs two extra commands at the top of every file
such that all constituent files can be compiled individually.

%%%%%%%%%%%%%%%%%%%%%%%%%%%%%%%%%%%%%%%%%%%%%%%%%%%%%%%%%%%%%%%%%%%%%%%%%%%%%%%%
%\subsection{Feature Suggestions}
%
%The following is a list of features which may be useful for future
%versions of this package:
%%
%\begin{itemize}
%\item
%\ldots
%\end{itemize}

%%%%%%%%%%%%%%%%%%%%%%%%%%%%%%%%%%%%%%%%%%%%%%%%%%%%%%%%%%%%%%%%%%%%%%%%%%%%%%%%
\subsection{Revision History}

%%%%%%%%%%%%%%%%%%%%%%%%%%%%%%%%%%%%%%%%
\paragraph{v2.0:} 2018/12/30

\begin{itemize}
\item
immediate forward processing
\item
added |\childdocby| mechanism
\item
manual restructured
\end{itemize}

%%%%%%%%%%%%%%%%%%%%%%%%%%%%%%%%%%%%%%%%
\paragraph{v1.6:} 2018/01/17

\begin{itemize}
\item
application for development of include files
\item
corrections to manual
\end{itemize}

%%%%%%%%%%%%%%%%%%%%%%%%%%%%%%%%%%%%%%%%
\paragraph{v1.5:} 2017/05/21

\begin{itemize}
\item
more complete structuring introduced
\item
|\childdocof| introduced
\item
|\childdoc| renamed to |\childdocmain|
\item
|\childredirect| renamed to |\childdocforward| and |\childdocforwardprefix|
and functionality expanded
\end{itemize}

%%%%%%%%%%%%%%%%%%%%%%%%%%%%%%%%%%%%%%%%
\paragraph{v1.0:} 2017/04/27

\begin{itemize}
\item
manual and install package
\item
first version published on CTAN
\end{itemize}

%%%%%%%%%%%%%%%%%%%%%%%%%%%%%%%%%%%%%%%%
\paragraph{v0.6:} 2017/04/26

\begin{itemize}
\item
redirection mechanism added
\end{itemize}

%%%%%%%%%%%%%%%%%%%%%%%%%%%%%%%%%%%%%%%%
\paragraph{v0.5:} 2017/04/26

\begin{itemize}
\item
functionality in definition file
\end{itemize}


%%%%%%%%%%%%%%%%%%%%%%%%%%%%%%%%%%%%%%%%%%%%%%%%%%%%%%%%%%%%%%%%%%%%%%%%%%%%%%%%
%%%%%%%%%%%%%%%%%%%%%%%%%%%%%%%%%%%%%%%%%%%%%%%%%%%%%%%%%%%%%%%%%%%%%%%%%%%%%%%%
%%%%%%%%%%%%%%%%%%%%%%%%%%%%%%%%%%%%%%%%%%%%%%%%%%%%%%%%%%%%%%%%%%%%%%%%%%%%%%%%
\appendix

\settowidth\MacroIndent{\rmfamily\scriptsize 000\ }

 \DocInput{childdoc.dtx}

\end{document}
%</driver>
% \fi
%
% %%%%%%%%%%%%%%%%%%%%%%%%%%%%%%%%%%%%%%%%%%%%%%%%%%%%%%%%%%%%%%%%%%%%%%%%%%%%%%
% %%%%%%%%%%%%%%%%%%%%%%%%%%%%%%%%%%%%%%%%%%%%%%%%%%%%%%%%%%%%%%%%%%%%%%%%%%%%%%
% \section{Sample}
%\iffalse
%<*samplemain>
%\fi
%
% The following presents a sample document
% with two chapters, two parts, a title page,
% a compile flag as well as three forwarding files to set the flag.
% It consists of eight |.tex| files:
% \begin{center}
% \begin{tabular}{ll}
% |cdocsamp.tex|&main file\\
% |cdocsch1.tex|&include file for chapter 1\\
% |cdocsch2.tex|&include file for chapter 2\\
% |cdocspt3.tex|&include file for part 3\\
% |cdocspt4.tex|&include file for part 4\\
% |cdocsdrf.tex|&forwarding file for main file in draft mode\\
% |cdocsfi1.tex|&forwarding file for final version of chapter 1\\
% |cdocsfi2.tex|&forwarding file for final version of chapter 2\\
% \end{tabular}
% \end{center}
% Each of the eight files can be compiled directly by the \LaTeX{} compiler.
%
% %%%%%%%%%%%%%%%%%%%%%%%%%%%%%%%%%%%%%%
% \paragraph{Main File.}
%
% The main file is called |cdocsamp.tex|.
%
% Load the \textsf{childdoc} definitions and
% declare the filename for the main document:
%    \begin{macrocode}
\input{childdoc.def}
\childdocmain{}
%    \end{macrocode}

% Optional override for |\version| flag:
%    \begin{macrocode}
%%\ifchilddoc\else\providecommand{\version}{draft}\fi
%    \end{macrocode}

% Define the default values for the |\version| flag
% (|final| for the main file and |draft| for childs):
%    \begin{macrocode}
\ifchilddoc
\providecommand{\version}{draft}
\else
\providecommand{\version}{final}
\fi
%    \end{macrocode}

% Load the standard document class:
%    \begin{macrocode}
\documentclass[12pt]{article}
%    \end{macrocode}

% Start the document body:
%    \begin{macrocode}
\begin{document}
%    \end{macrocode}

% Declare a title page.
% Print title, part of document being processed and version flag:
%    \begin{macrocode}
\addtocounter{page}{-1}
\begin{center}
{\LARGE\bfseries{}childdoc example\par}
\vspace{1cm}
\ifchilddoc
\ifchilddocmanual part\else chapter\fi:
`\childdocname' of `\childdocjob'\par
\else
main document: `\childdocjob'\par
\fi
version: \version\par
\end{center}
\newpage
%    \end{macrocode}

% Manually include selected file,
% otherwise process as usual:
%    \begin{macrocode}
\ifchilddocmanual
\section*{part `\childdocname'}
\input{\childdocname}
\else
%    \end{macrocode}

% Include the two chapters:
%    \begin{macrocode}
\include{cdocsch1}
\include{cdocsch2}
%    \end{macrocode}

% Include the two parts unless only chapters should be displayed:
%    \begin{macrocode}
\ifchilddoc\else
\section{part three}
\input{cdocspt3}
\section{part four}
\input{cdocspt4}
\fi
%    \end{macrocode}

% Process as usual until here:
%    \begin{macrocode}
\fi
%    \end{macrocode}

% End of document body:
%    \begin{macrocode}
\end{document}
%    \end{macrocode}
%\iffalse
%</samplemain>
%\fi
%
% %%%%%%%%%%%%%%%%%%%%%%%%%%%%%%%%%%%%%%
% \paragraph{Chapter Include Files.}
%
% The include files are called |cdocsch1.tex| and |cdocsch2.tex|.
%
%\iffalse
%<*samplechap1|samplechap2>
%\fi

% Optional override for |\version| flag:
%    \begin{macrocode}
%%\providecommand{\version}{final}
%    \end{macrocode}

% Include the main document:
%    \begin{macrocode}
\input{childdoc.def}
\childdocof{cdocsamp}
%    \end{macrocode}

%\iffalse
%</samplechap1|samplechap2>
%\fi
%
%\iffalse
%<*samplechap1>
%\fi
% Some text for chapter 1:
%    \begin{macrocode}
\section{one}
some text in chapter one
%    \end{macrocode}

%\iffalse
%</samplechap1>
%\fi
% Some text for chapter 2:
%\iffalse
%<*samplechap2>
%\fi
%    \begin{macrocode}
\section{two}
more text in chapter two
%    \end{macrocode}

%\iffalse
%</samplechap2>
%\fi
%
% %%%%%%%%%%%%%%%%%%%%%%%%%%%%%%%%%%%%%%
% \paragraph{Part Include Files.}
%
% The include files are called |cdocspt3.tex| and |cdocspt4.tex|.
%
%\iffalse
%<*samplepart3|samplepart4>
%\fi

% Optional override for |\version| flag:
%    \begin{macrocode}
%%\providecommand{\version}{final}
%    \end{macrocode}

% Include the main document:
%    \begin{macrocode}
\input{childdoc.def}
\childdocby{cdocsamp}
%    \end{macrocode}

%\iffalse
%</samplepart3|samplepart4>
%\fi
%
%\iffalse
%<*samplepart3>
%\fi
% Some text for part 3:
%    \begin{macrocode}
some text in part three
%    \end{macrocode}

%\iffalse
%</samplepart3>
%\fi
% Some text for part 4:
%\iffalse
%<*samplepart4>
%\fi
%    \begin{macrocode}
more text in part four
%    \end{macrocode}

%\iffalse
%</samplepart4>
%\fi
%
% %%%%%%%%%%%%%%%%%%%%%%%%%%%%%%%%%%%%%%
% \paragraph{Forwarding for a Complete Draft.}
%
% The following forwarding file |cdocsdrf.tex|
% compiles the main document in draft mode:
%\iffalse
%<*sampledraft>
%\fi
%    \begin{macrocode}
\def\version{draft}
\input{childdoc.def}
\childdocforward{cdocsamp}
%    \end{macrocode}

%\iffalse
%</sampledraft>
%\fi
%
% %%%%%%%%%%%%%%%%%%%%%%%%%%%%%%%%%%%%%%
% \paragraph{Forwarding for Final Version of the Chapters.}
%
% The following forwarding files |cdocsfn1.tex| and |cdocsfn2.tex|
% (with identical content)
% compile the final versions of the child documents
% |cdocsch1.tex| and |cdocsch2.tex|, respectively:
%\iffalse
%<*samplefinal>
%\fi
%    \begin{macrocode}
\def\version{final}
\input{childdoc.def}
\childdocforwardprefix[cdocsamp]{cdocsfn}{cdocsch}
%    \end{macrocode}

%\iffalse
%</samplefinal>
%\fi
%
% %%%%%%%%%%%%%%%%%%%%%%%%%%%%%%%%%%%%%%
% \paragraph{Command Line Processing.}
%
% The following three command lines generate the output files
% |cdocscld|, |cdocscl1| and |cdocscl2|
% which should be identical to
% |cdocsdrf|, |cdocsch1| and |cdocsfn2|, respectively:
% \begin{center}
% \begin{tabular}{l}
% |latex -jobname cdocscld \|\\
% |  "\def\version{draft}\input{childdoc.def}\childdocforward{cdocsamp}"|\\
% |latex -jobname cdocscl1 \|\\
% |  "\input{childdoc.def}\childdocforward[cdocsamp]{cdocsch1}"|\\
% |latex -jobname cdocscl2 \|\\
% |  "\def\version{final}\input{childdoc.def}\childdocforward{cdocsch2}"|
% \end{tabular}
% \end{center}
% Note that the trailing backslash on each first line
% merely continues the input to the second line
% (for convenient cut ant paste).
% Furthermore, the command |latex| can be replaced by any
% of its alternative versions such as |pdflatex|.
%
% %%%%%%%%%%%%%%%%%%%%%%%%%%%%%%%%%%%%%%%%%%%%%%%%%%%%%%%%%%%%%%%%%%%%%%%%%%%%%%
% %%%%%%%%%%%%%%%%%%%%%%%%%%%%%%%%%%%%%%%%%%%%%%%%%%%%%%%%%%%%%%%%%%%%%%%%%%%%%%
% \section{Implementation}
%\iffalse
%<*package>
%\fi
%
% This section describes the definitions file |childdoc.def|.

% The definitions cannot be loaded using |\usepackage| or |\RequirePackage|
% which has a mechanism to prevent loading a style file more than once.
% When loading the definitions by means of |\input|
% multiple instances have to be prevented manually:
%\iffalse
%This code needs to be before the `\ProvidesFile' directive
%which is defined at the beginning of this file.
%Therefore it is also placed there and commented out here.
%</package>
%<*discard>
%\fi
%    \begin{macrocode}
\ifdefined\childdocmain\endinput\fi
%    \end{macrocode}
%\iffalse
%</discard>
%<*package>
%\fi
%
% \macro{\ifchilddoc}
% \macro{\ifchilddocmanual}
% The conditional |\ifchilddoc| tells whether a
% child (true) or main (false) document is being compiled.
% The conditional |\ifchilddocmanual| tells whether
% the |\includeonly| mechanism is used (false) or
% the selection of child files must be performed manually (true).
% The definitions initialise to false:
%    \begin{macrocode}
\newif\ifchilddoc
\newif\ifchilddocmanual
%    \end{macrocode}

% \macro{\childdocname}
% \macro{\childdocjob}
% The macro |\childdocname| stores the name of the main document
% to be compiled. The macro |\childdocjob| stores the name of
% the document on which the \LaTeX{} compiler was originally invoked.
% The content of |\jobname| cannot be compared
% to filenames specified in the source due to different catcodes.
% The following code rescans |\jobname|, stores the result
% in |\childdocname| and saves a copy in |\childdocjob|:
%    \begin{macrocode}
\edef\childdocname{\scantokens\expandafter{\jobname\noexpand}}
\let\childdocjob\childdocname
%    \end{macrocode}

% \macro{\childdocdisable}
% The macro |\childdocdisable| prevents the main file
% from being processed more than once.
% At this stage, the main document command |\childdocmain|
% is assumed to be called once again where it should do nothing.
% Any subsequent call to it should prevent
% a secondary processing of the main document
% It overwrites the forwarding commands
% |\childdocof| and |\childdocforward|
% with empty macros to prevent further inclusions of the main document:
%    \begin{macrocode}
\newcommand{\childdocdisable}
{
  \renewcommand{\childdocmain}[1]{\renewcommand{\childdocmain}[1]{\endinput}}
  \renewcommand{\childdocof}[1]{}
  \renewcommand{\childdocby}[2][]{}
  \renewcommand{\childdocforward}[2][]{}
  \renewcommand{\childdocdisable}{}
}
%    \end{macrocode}

% \macro{\childdocmain}
% The macro |\childdocmain| is to be called at the top of the main file
% with nothing or the main filename (without extension) as argument.
% First, it breaks loops.
% If the argument is not empty and does not match |\childdocname|
% (which is set by the first inclusion of |childdoc.def|),
% |\ifchilddoc| is set to true, |\includeonly| is applied to the child file
% and |\jobname| is set to the main file
% (for proper handling of |.aux| files):
%    \begin{macrocode}
\newcommand{\childdocmain}[1]
{
  \childdocdisable\childdocmain{}
  \if?#1?\else
    \begingroup
      \def\childdoctmp{#1}
      \ifx\childdoctmp\childdocname
        \def\childdoctmp{}
      \else
        \def\childdoctmp
        {
          \childdoctrue
          \includeonly{\childdocname}
          \def\childdocjob{#1}
          \def\jobname{#1}
        }
      \fi
      \expandafter
    \endgroup
    \childdoctmp
  \fi
}
%    \end{macrocode}

% \macro{\childdocof}
% The command |\childdocof| redirects
% compilation to the main file |#1|.
%    \begin{macrocode}
\newcommand{\childdocof}[1]
{
  \childdocdisable
  \childdoctrue
  \includeonly{\childdocname}
  \def\jobname{#1}
  \def\childdocjob{#1}
  \input{#1}
}
%    \end{macrocode}

% \macro{\childdocby}
% The command |\childdocby| ....
%    \begin{macrocode}
\newcommand{\childdocby}[2][]
{
  \childdocdisable
  \childdoctrue
  \childdocmanualtrue
  \if?#1?\else
    \def\jobname{#2}
  \fi
  \def\childdocjob{#2}
  \input{#2}
  \endinput
}
%    \end{macrocode}

% \macro{\childdocforward}
% The command |\childdocforward| redirects
% compilation to the main file or
% (if the optional argument is given) a child file.
% Parameters are set as if the main file
% or a child file starting with |\childdocof| was compiled.
% Then compilation is handed over to the main file:
%    \begin{macrocode}
\newcommand{\childdocforward}[2][]
{
  \begingroup
    \if?#1?
      \def\childdoctmp
      {
        \def\childdocname{#2}
        \def\childdocjob{#2}
        \def\jobname{#2}
        \input{#2}
        \endinput
      }
    \else
      \def\childdoctmp
      {
        \childdocdisable
        \def\childdocname{#2}
        \childdoctrue
        \includeonly{#2}
        \def\childdocjob{#1}
        \def\jobname{#1}
        \input{#1}
        \endinput
      }
    \fi
    \expandafter
  \endgroup
  \childdoctmp
}
%    \end{macrocode}

% \macro{\childdocforwardprefix}
% The command |\childdocforwardprefix| redirects
% compilation to the main or a child file by means of a pattern.
% The prefix |#1| in the current filename is replaced by |#2|
% and the suffix of the current filename is kept
% (it is assumed that the filename does not contain the substring `|~~~|'
% which is used as a delimiter).
% Compilation is handed over to the new file by |\childdocforward|:
%    \begin{macrocode}
\newcommand{\childdocforwardprefix}[3][]
{
  \begingroup
    \def\childdocextract #2##1~~~{\def\childdoctmp{\childdocforward[#1]{#3##1}}}
    \expandafter\childdocextract\childdocname~~~
    \expandafter
  \endgroup
  \childdoctmp
}
%    \end{macrocode}

% \macro{\childdoc}
% The deprecated macro |\childdoc| is a legacy version of |\childdocmain|:
%    \begin{macrocode}
\newcommand{\childdoc}{\childdocmain}
%    \end{macrocode}

% \macro{\childdocredirect}
% The deprecated macro |\childdocredirect| is a legacy version
% of |\childdocforward| and |\childdocforwardprefix|:
%    \begin{macrocode}
\newcommand{\childdocredirect}[2][]
{
  \begingroup
    \if?#1?
      \def\childdoctmp{\childdocforward{#2}}
    \else
      \def\childdoctmp{\childdocforwardprefix{#1}{#2}}
    \fi
    \expandafter
  \endgroup
  \childdoctmp
}
%    \end{macrocode}

%\iffalse
%</package>
%\fi
%
\endinput
|
and perform the replacements as outlined below.
Instead of |\childdocmain{|\textit{main}|}| add the following code
to the top of the main file:
%
\begin{center}
\begin{tabular}{l}
|\||ifdefined\childdocname\endinput\||fi\newif\ifchilddoc|\\
|\edef\childdocname{\scantokens\expandafter{\jobname\noexpand}}|\\
|\def\childdocmain{|\textit{main}|}\||ifx\childdocmain\childdocname\||else|\\
|\childdoctrue\includeonly{\childdocname}\let\jobname\childdocmain\||fi|\\
\end{tabular}
\end{center}
%
Instead of |\childdocof{|\textit{main}|}| just include the main file
at the top of each child file:
%
\begin{center}
|\input{|\textit{main}|}|
\end{center}
%
A simple redirection |\childdocforward{|\textit{dest}|}| is achieved by:
%
\begin{center}
|\def\jobname{|\textit{dest}|}\input{\jobname}|
\end{center}
%
The redirection with prefix
|\childdocforwardprefix[|\textit{prefix}|]{|\textit{dest}|}|
is accomplished by:
%
\begin{center}
\begin{tabular}{l}
|{\edef\jobname{\scantokens\expandafter{\jobname\noexpand}}|\\
|\def\redirectjob |\textit{prefix}|#1~~~{\gdef\jobname{|\textit{dest}|#1}}|\\
|\expandafter\redirectjob\jobname~~~}\input{\jobname}|
\end{tabular}
\end{center}

In an alternative approach,
child documents can be compiled by a specific command line
without additional code or specific definitions:
%
\begin{center}
|... -jobname "|\textit{target}|" "|[\textit{flags}]%
|\includeonly{|\textit{dest}|}\input{|\textit{main}|}"|
\end{center}
%

%%%%%%%%%%%%%%%%%%%%%%%%%%%%%%%%%%%%%%%%%%%%%%%%%%%%%%%%%%%%%%%%%%%%%%%%%%%%%%%%
%%%%%%%%%%%%%%%%%%%%%%%%%%%%%%%%%%%%%%%%%%%%%%%%%%%%%%%%%%%%%%%%%%%%%%%%%%%%%%%%
\section{Information}

%%%%%%%%%%%%%%%%%%%%%%%%%%%%%%%%%%%%%%%%%%%%%%%%%%%%%%%%%%%%%%%%%%%%%%%%%%%%%%%%
\subsection{Copyright}

Copyright \copyright{} 2017--2018 Niklas Beisert

This work may be distributed and/or modified under the
conditions of the \LaTeX{} Project Public License, either version 1.3
of this license or (at your option) any later version.
The latest version of this license is in
  \url{http://www.latex-project.org/lppl.txt}
and version 1.3 or later is part of all distributions of \LaTeX{}
version 2005/12/01 or later.

This work has the LPPL maintenance status `maintained'.

The Current Maintainer of this work is Niklas Beisert.

This work consists of the files |README.txt|, |childdoc.ins| and |childdoc.dtx|
as well as the derived files |childdoc.def|, |cdocsamp.tex|
with |cdocsch1.tex|, |cdocsch2.tex|, |cdocspt3.tex|, |cdocspt4.tex|,
|cdocsdrf.tex|, |cdocsfn1.tex|, |cdocsfn2.tex|
as well as |childdoc.pdf|.

%%%%%%%%%%%%%%%%%%%%%%%%%%%%%%%%%%%%%%%%%%%%%%%%%%%%%%%%%%%%%%%%%%%%%%%%%%%%%%%%
\subsection{Files and Installation}

The package consists of the files:
%
\begin{center}
\begin{tabular}{ll}
    |README.txt|   & readme file \\
    |childdoc.ins| & installation file \\
    |childdoc.dtx| & source file \\
    |childdoc.def| & definition file \\
    |cdocsamp.tex| & sample main file \\
    |cdocsch1.tex| & sample include file \\
    |cdocsch2.tex| & sample include file \\
    |cdocspt3.tex| & sample part file \\
    |cdocspt4.tex| & sample part file \\
    |cdocsdrf.tex| & sample redirection file \\
    |cdocsfn1.tex| & sample redirection file \\
    |cdocsfn2.tex| & sample redirection file \\
    |childdoc.pdf| & manual
\end{tabular}
\end{center}
%
The distribution consists of the files
|README.txt|, |childdoc.ins| and |childdoc.dtx|.
%
\begin{itemize}
\item
Run (pdf)\LaTeX{} on |childdoc.dtx|
to compile the manual |childdoc.pdf| (this file).
\item
Run \LaTeX{} on |childdoc.ins| to create the definitions file |childdoc.def|
and the sample |cdocsamp.tex| with include files
|cdocsch1.tex|, |cdocsch2.tex|, |cdocspt3.tex|, |cdocspt4.tex|,
|cdocsdrf.tex|, |cdocsfn1.tex|, |cdocsfn2.tex|.
Then copy the file |childdoc.def| to an appropriate directory of your \LaTeX{}
distribution, e.g.\ \textit{texmf-root}|/tex/latex/childdoc|.
\end{itemize}

%%%%%%%%%%%%%%%%%%%%%%%%%%%%%%%%%%%%%%%%%%%%%%%%%%%%%%%%%%%%%%%%%%%%%%%%%%%%%%%%
\subsection{Related CTAN Packages}

There are several other packages which offer a similar functionality:
%
\begin{itemize}
\item
The packages
\href{http://ctan.org/pkg/docmute}{\textsf{docmute}},
\href{http://ctan.org/pkg/includex}{\textsf{includex}} and
\href{http://ctan.org/pkg/standalone}{\textsf{standalone}}
provide commands to include only the document body of
a child file thus allowing both files to be compiled individually.
\item
The packages \href{http://ctan.org/pkg/subdocs}{\textsf{subdocs}}
and \href{http://ctan.org/pkg/subfiles}{\textsf{subfiles}}
provide structures in which the main and child documents can be
encapsulated and allowing them to be compiled individually.
The inclusion mechanism is different from the conventional |\include|.
\item
The package \href{http://ctan.org/pkg/combine}{\textsf{combine}}
is an elaborate solution to combine several documents into one.
\end{itemize}
%
See also the CTAN topic \href{http://ctan.org/topic/subdocs}{\textsf{subdocs}}
for further related packages.
The present package differs from the above solutions in that
a document structure constructed with the conventional |\include| mechanism
just needs two extra commands at the top of every file
such that all constituent files can be compiled individually.

%%%%%%%%%%%%%%%%%%%%%%%%%%%%%%%%%%%%%%%%%%%%%%%%%%%%%%%%%%%%%%%%%%%%%%%%%%%%%%%%
%\subsection{Feature Suggestions}
%
%The following is a list of features which may be useful for future
%versions of this package:
%%
%\begin{itemize}
%\item
%\ldots
%\end{itemize}

%%%%%%%%%%%%%%%%%%%%%%%%%%%%%%%%%%%%%%%%%%%%%%%%%%%%%%%%%%%%%%%%%%%%%%%%%%%%%%%%
\subsection{Revision History}

%%%%%%%%%%%%%%%%%%%%%%%%%%%%%%%%%%%%%%%%
\paragraph{v2.0:} 2018/12/30

\begin{itemize}
\item
immediate forward processing
\item
added |\childdocby| mechanism
\item
manual restructured
\end{itemize}

%%%%%%%%%%%%%%%%%%%%%%%%%%%%%%%%%%%%%%%%
\paragraph{v1.6:} 2018/01/17

\begin{itemize}
\item
application for development of include files
\item
corrections to manual
\end{itemize}

%%%%%%%%%%%%%%%%%%%%%%%%%%%%%%%%%%%%%%%%
\paragraph{v1.5:} 2017/05/21

\begin{itemize}
\item
more complete structuring introduced
\item
|\childdocof| introduced
\item
|\childdoc| renamed to |\childdocmain|
\item
|\childredirect| renamed to |\childdocforward| and |\childdocforwardprefix|
and functionality expanded
\end{itemize}

%%%%%%%%%%%%%%%%%%%%%%%%%%%%%%%%%%%%%%%%
\paragraph{v1.0:} 2017/04/27

\begin{itemize}
\item
manual and install package
\item
first version published on CTAN
\end{itemize}

%%%%%%%%%%%%%%%%%%%%%%%%%%%%%%%%%%%%%%%%
\paragraph{v0.6:} 2017/04/26

\begin{itemize}
\item
redirection mechanism added
\end{itemize}

%%%%%%%%%%%%%%%%%%%%%%%%%%%%%%%%%%%%%%%%
\paragraph{v0.5:} 2017/04/26

\begin{itemize}
\item
functionality in definition file
\end{itemize}


%%%%%%%%%%%%%%%%%%%%%%%%%%%%%%%%%%%%%%%%%%%%%%%%%%%%%%%%%%%%%%%%%%%%%%%%%%%%%%%%
%%%%%%%%%%%%%%%%%%%%%%%%%%%%%%%%%%%%%%%%%%%%%%%%%%%%%%%%%%%%%%%%%%%%%%%%%%%%%%%%
%%%%%%%%%%%%%%%%%%%%%%%%%%%%%%%%%%%%%%%%%%%%%%%%%%%%%%%%%%%%%%%%%%%%%%%%%%%%%%%%
\appendix

\settowidth\MacroIndent{\rmfamily\scriptsize 000\ }

 \DocInput{childdoc.dtx}

\end{document}
%</driver>
% \fi
%
% %%%%%%%%%%%%%%%%%%%%%%%%%%%%%%%%%%%%%%%%%%%%%%%%%%%%%%%%%%%%%%%%%%%%%%%%%%%%%%
% %%%%%%%%%%%%%%%%%%%%%%%%%%%%%%%%%%%%%%%%%%%%%%%%%%%%%%%%%%%%%%%%%%%%%%%%%%%%%%
% \section{Sample}
%\iffalse
%<*samplemain>
%\fi
%
% The following presents a sample document
% with two chapters, two parts, a title page,
% a compile flag as well as three forwarding files to set the flag.
% It consists of eight |.tex| files:
% \begin{center}
% \begin{tabular}{ll}
% |cdocsamp.tex|&main file\\
% |cdocsch1.tex|&include file for chapter 1\\
% |cdocsch2.tex|&include file for chapter 2\\
% |cdocspt3.tex|&include file for part 3\\
% |cdocspt4.tex|&include file for part 4\\
% |cdocsdrf.tex|&forwarding file for main file in draft mode\\
% |cdocsfi1.tex|&forwarding file for final version of chapter 1\\
% |cdocsfi2.tex|&forwarding file for final version of chapter 2\\
% \end{tabular}
% \end{center}
% Each of the eight files can be compiled directly by the \LaTeX{} compiler.
%
% %%%%%%%%%%%%%%%%%%%%%%%%%%%%%%%%%%%%%%
% \paragraph{Main File.}
%
% The main file is called |cdocsamp.tex|.
%
% Load the \textsf{childdoc} definitions and
% declare the filename for the main document:
%    \begin{macrocode}
% \iffalse
%
% childdoc.dtx Copyright (C) 2017-2018 Niklas Beisert
%
% This work may be distributed and/or modified under the
% conditions of the LaTeX Project Public License, either version 1.3
% of this license or (at your option) any later version.
% The latest version of this license is in
%   http://www.latex-project.org/lppl.txt
% and version 1.3 or later is part of all distributions of LaTeX
% version 2005/12/01 or later.
%
% This work has the LPPL maintenance status `maintained'.
%
% The Current Maintainer of this work is Niklas Beisert.
%
% This work consists of the files childdoc.dtx and childdoc.ins
% and the derived files childdoc.def and cdocsamp.tex with
% cdocsch1.tex, cdocsch2.tex, cdocsdrf.tex, cdocsfn1.tex, cdocsfn2.tex.
%
%<package>\ifdefined\childdocmain\endinput\fi
%<package>\ProvidesFile{childdoc.def}[2018/12/30 v2.0 child document driver]
%<samplemain>\ProvidesFile{cdocsamp.tex}[2018/12/30 v2.0 sample for childdoc]
%<*driver>
%\ProvidesFile{childdoc.drv}[2018/12/30 v2.0 childdoc reference manual file]
\PassOptionsToClass{10pt,a4paper}{article}
\documentclass{ltxdoc}

\usepackage[margin=35mm]{geometry}
\usepackage{hyperref}
\usepackage{hyperxmp}
\usepackage[usenames]{color}

\hypersetup{colorlinks=true}
\hypersetup{pdfstartview=FitH}
\hypersetup{pdfpagemode=UseNone}
\hypersetup{pdfsource={}}
\hypersetup{pdflang={en-UK}}
\hypersetup{pdfcopyright={Copyright 2017-2018 Niklas Beisert.
  This work may be distributed and/or modified under the
  conditions of the LaTeX Project Public License, either version 1.3
  of this license or (at your option) any later version.}}
\hypersetup{pdflicenseurl={http://www.latex-project.org/lppl.txt}}
\hypersetup{pdfcontactaddress={ETH Zurich, ITP, HIT K,
  Wolfgang-Pauli-Strasse 27}}
\hypersetup{pdfcontactpostcode={8093}}
\hypersetup{pdfcontactcity={Zurich}}
\hypersetup{pdfcontactcountry={Switzerland}}
\hypersetup{pdfcontactemail={nbeisert@itp.phys.ethz.ch}}
\hypersetup{pdfcontacturl={http://people.phys.ethz.ch/\xmptilde nbeisert/}}

\newcommand{\secref}[1]{\hyperref[#1]{section \ref*{#1}}}

\parskip1ex
\parindent0pt
\let\olditemize\itemize
\def\itemize{\olditemize\parskip0pt}

\begin{document}

\title{The \textsf{childdoc} Package}
\hypersetup{pdftitle={The childdoc Package}}
\author{Niklas Beisert\\[2ex]
  Institut f\"ur Theoretische Physik\\
  Eidgen\"ossische Technische Hochschule Z\"urich\\
  Wolfgang-Pauli-Strasse 27, 8093 Z\"urich, Switzerland\\[1ex]
  \href{mailto:nbeisert@itp.phys.ethz.ch}
  {\texttt{nbeisert@itp.phys.ethz.ch}}}
\hypersetup{pdfauthor={Niklas Beisert}}
\hypersetup{pdfsubject={Manual for the LaTeX2e Package childdoc}}
\date{30 December 2018, \textsf{v2.0}}
\maketitle

\begin{abstract}\noindent
\textsf{childdoc} is a \LaTeXe{} package
that enables the direct compilation
of document sections included by |\include|
to individual files.
\end{abstract}

\begingroup
\parskip0ex
\tableofcontents
\endgroup

%%%%%%%%%%%%%%%%%%%%%%%%%%%%%%%%%%%%%%%%%%%%%%%%%%%%%%%%%%%%%%%%%%%%%%%%%%%%%%%%
%%%%%%%%%%%%%%%%%%%%%%%%%%%%%%%%%%%%%%%%%%%%%%%%%%%%%%%%%%%%%%%%%%%%%%%%%%%%%%%%
\section{Introduction}

\LaTeX{} provides a mechanism to structure a large document (such as a book)
into a main file and several child files (containing the chapters)
using the |\include| command.
This mechanism is beneficial for documents
which span hundreds of pages in order to
make the source file(s) more manageable.
Moreover, compilation can be restricted to
selected child files by means of the |\includeonly| command.
The latter feature can be used to reduce the compilation time while editing
(this was significantly more useful in the earlier days of \LaTeX{})
or to generate a smaller document which is easier to navigate.
Another application of |\includeonly| is to generate
documents consisting of selected parts of the complete document.

However, there are a few drawbacks of the plain |\include| mechanism:
\begin{itemize}
\item
The child files cannot be compiled on their own,
they can only be compiled via the main file.
A naive editing environment
(such as a text editor with an option
to have the current file processed by \LaTeX)
may require one to switch to the main file before compiling;
attempting to compile the child file produces errors.
\item
The main file must be modified (each time)
to adjust the |\includeonly| command
to the present needs. This easily leaves the main file in a messy state.
\item
The generated document will always carry the filename
of the main document. This is inconvenient if
several child files are to be compiled and
to be kept for distribution.
\end{itemize}

The present package provides a simple interface
to make child files individually compilable by \LaTeX{}.
Compiling a child file then has the same effect as compiling
the main file with an |\includeonly| command
to select the appropriate child.
Moreover the generated document will carry the name of the child
rather than the main file.
This resolves all three above issues.

This feature is meant to make the editing of books,
thesis documents and lecture notes somewhat more convenient.
However, the package can also be used efficiently for
composing a series of documents (such as exercise sheets)
which are typically distributed individually.
It then assists the author in generating the individual documents
(potentially in different versions)
as well as a document containing the collected series.
Another application is in developing style files
or other kinds of included material
where compilation of the style file could redirect
to a sample or test file.

%%%%%%%%%%%%%%%%%%%%%%%%%%%%%%%%%%%%%%%%%%%%%%%%%%%%%%%%%%%%%%%%%%%%%%%%%%%%%%%%
%%%%%%%%%%%%%%%%%%%%%%%%%%%%%%%%%%%%%%%%%%%%%%%%%%%%%%%%%%%%%%%%%%%%%%%%%%%%%%%%
\section{Usage}

First of all, the package \textsf{childdoc} is \emph{not} a standard
\LaTeXe{} |.sty| style file! Therefore it needs to be invoked in
a non-standard way.

%%%%%%%%%%%%%%%%%%%%%%%%%%%%%%%%%%%%%%%%%%%%%%%%%%%%%%%%%%%%%%%%%%%%%%%%%%%%%%%%
\subsection{Included Files}
\label{sec:include}

%%%%%%%%%%%%%%%%%%%%%%%%%%%%%%%%%%%%%%%%
\DescribeMacro{\childdocmain}
To use the package, add the commands
\begin{center}
\begin{tabular}{l}
|\input{childdoc.def}|\\
|\childdocmain{}|\\
\end{tabular}
\end{center}
at the very top of the main \LaTeX{} file,
in particular \emph{before} the |\documentclass| statement!
The argument of |\childdocmain| should be left empty
(but it must be present).

%%%%%%%%%%%%%%%%%%%%%%%%%%%%%%%%%%%%%%%%
\DescribeMacro{\childdocof}
Furthermore, add the commands
\begin{center}
\begin{tabular}{l}
|\input{childdoc.def}|\\
|\childdocof{|\textit{main}|}|\\
\end{tabular}
\end{center}
at the top of every child file \textit{child}
which is included by |\include{|\textit{child}|}|
from within the main file
(or at least for those files to be compiled individually).
The argument \textit{main} must be the filename of the main file.

There are a couple of
considerations in setting up the main and child documents:

%%%%%%%%%%%%%%%%%%%%%%%%%%%%%%%%%%%%%%%%
\paragraph{Restrictions.}

Please note the following restrictions:
\begin{itemize}
\item
|\childdocmain| must be called with one argument \textit{main}
to ensure compatibility with earlier version of the package.
It must either be empty (|\childdocmain{}|)
or precisely match the filename of the main file in which it is specified.
See \secref{sec:detection} for further information.
\item
The filename \textit{main} must be specified without the |.tex| extension.
\item
The filename \textit{main} is case sensitive
(even in case-insensitive file systems)
due to internal string comparison.
\item
The argument \textit{main} should be fully expanded, it cannot be a macro.
\item
Subdirectories and special characters should be avoided in filenames.
\item
The command |\childdocmain{|\textit{main}|}| must be followed by a whitespace.
It should not be followed immediately by another command
or by a comment mark `|%|'.
This is because the \TeX{} parser reads the token immediately following
the argument of |\childdocmain| and puts it
at the beginning of every child section;
however, a white\-space is ignored.
\end{itemize}

%%%%%%%%%%%%%%%%%%%%%%%%%%%%%%%%%%%%%%%%
\paragraph{Content of Main File.}

It is advisable to place all content in the child files included by |\include|.
Any output contained in the main file will appear in all child documents
unless suppressed manually;
it cannot be suppressed automatically by the |\includeonly| directive
and thus should normally be avoided.
A method to include some content in the main file
by means of conditional processing is described in \secref{sec:conditional}.

%%%%%%%%%%%%%%%%%%%%%%%%%%%%%%%%%%%%%%%%
\paragraph{Page Numbering.}

When only a part of the document is compiled,
the appropriate numbering of pages
(as well as other status parameters)
is determined from the |.aux| files.
The latter contain information from previous passes.
However this information needs to propagate through
all intermediate child documents.
Therefore the page numbering in child documents may well
be inconsistent until the complete document is compiled at least once.

A useful (if unconventional) way to always ensure a consistent
page numbering is to restart the numbering in each child document
and denote the pages by `\textit{child}|.|\textit{page}'
where \textit{child} represents the chapter/section number of the child file.
This can be achieved by the command
|\numberwithin{page}{|\textit{child}|}|
of the \textsf{amsmath} package
where \textit{child} can be |chapter| or |section|
depending on the chosen structuring.
Alternatively, one can modify the macro |\thepage| appropriately
and reset the counter |page| at the start of each child file.

%%%%%%%%%%%%%%%%%%%%%%%%%%%%%%%%%%%%%%%%%%%%%%%%%%%%%%%%%%%%%%%%%%%%%%%%%%%%%%%%
\subsection{Conditional Processing}
\label{sec:conditional}

The package provides a mechanism to compile different versions
of a document. To customise the versions further some conditional processing
can come in handy to distinguish which version is being compiled.
The package provides two macros to describe the compilation context:

%%%%%%%%%%%%%%%%%%%%%%%%%%%%%%%%%%%%%%%%
\DescribeMacro{\ifchilddoc}
The conditional |\ifchilddoc| distinguishes between the compilation of
child documents and the main document:
%
\begin{center}
|\ifchilddoc |\textit{child-code}| |[|\||else |\textit{main-code}]| \||fi|
\end{center}

%%%%%%%%%%%%%%%%%%%%%%%%%%%%%%%%%%%%%%%%
\DescribeMacro{\childdocname}
\DescribeMacro{\childdocjob}
The macro |\childdocname| contains the filename (without extension)
of the main or child file being processed.
Note that |\childdocjob| will always contain the name of the main file.

%%%%%%%%%%%%%%%%%%%%%%%%%%%%%%%%%%%%%%%%
\paragraph{Title Page.}

Conditional processing can be used to include a title or banner page
in the main document when proper precautions are taken.
Importantly, the code in the main file should ensure that the page counter
(as well as other status parameters which are stored in the |.aux| files)
takes the same value after the conditional processing.
Otherwise the page numbers may take divergent values
depending on which part is compiled.

For example, a title page could be declared by:
%
\begin{center}
\begin{tabular}{l}
|\ifchilddoc\||else|\\
|\addtocounter{page}{-1}|\\
\textit{code for title page}\\
|\newpage|\\
|\||fi|
\end{tabular}
\end{center}
%
A banner page for the child documents can be generated by:
%
\begin{center}
\begin{tabular}{l}
|\ifchilddoc|\\
|\addtocounter{page}{-1}|\\
\textit{code for banner page}\\
|\newpage|\\
|\||fi|
\end{tabular}
\end{center}
%
Here one could write a message such as:
\begin{center}
|This is the part \childdocname{} of \childdocjob{}.|
\end{center}

%%%%%%%%%%%%%%%%%%%%%%%%%%%%%%%%%%%%%%%%%%%%%%%%%%%%%%%%%%%%%%%%%%%%%%%%%%%%%%%%
\subsection{Flags}
\label{sec:flags}

The package makes it easy to generate different versions
of the main or child documents.
To this end compilation flags can be defined
and assigned different default values.
They will be particularly useful in conjunction
with the forwarding mechanism described in \secref{sec:forward}.

For example, it may be useful to have a flag |\version|
which can be set to |draft| or |final|.
The document source will contain some conditional code
depending on the value of |\version|.
Suppose further, the flag should default to |final| for the main file
and to |draft| for child files
which is a natural assignment for editing the document.
This is achieved by placing the following code
in the preamble of the main document
(below the |\childdocmain| directive):
%
\begin{center}
\begin{tabular}{l}
|\ifchilddoc|\\
|\providecommand{\version}{draft}|\\
|\||else|\\
|\providecommand{\version}{final}|\\
|\||fi|
\end{tabular}
\end{center}
%
The definition by |\providecommand| makes sure
that previous definitions are not overwritten.
Further statements |\providecommand{\version}{...}|
can thus be added before the above code to override it.

For the main file, one might add a line
(between |\childdocmain| and the above block)
%
\begin{center}
|%\ifchilddoc\||else\providecommand{\version}{draft}\||fi|
\end{center}
%
which can be uncommented to produce a draft version.
Likewise one can add a line to the very top of a child file
(above the |\childdocof{|\textit{main}|}| directive)
%
\begin{center}
|%\providecommand{\version}{final}|
\end{center}
%
which can be uncommented to produce the final version of this child document.

%%%%%%%%%%%%%%%%%%%%%%%%%%%%%%%%%%%%%%%%%%%%%%%%%%%%%%%%%%%%%%%%%%%%%%%%%%%%%%%%
\subsection{Forwarding}
\label{sec:forward}

Different versions of the main or child documents
using compilation flags as described in \secref{sec:flags}
can be (permanently) stored in different files
for convenient compilation, viewing and distribution.
To this end, the package defines a command
to pass on compilation to a different file:

%%%%%%%%%%%%%%%%%%%%%%%%%%%%%%%%%%%%%%%%
\DescribeMacro{\childdocforward}
The command |\childdocforward| redirects processing to
another source file:
%
\begin{center}
\begin{tabular}{l}
|\input{childdoc.def}|\\
|\childdocforward[|\textit{main}|]{|\textit{dest}|}|\\
\end{tabular}
\end{center}
%
The argument \textit{dest} is the destination file
(without extension).
It should be the main file or one of the child files.
Note that further \textsf{childdoc} directives
such as |\childdocof| and |\childdocforward|
in the indicated file will be processed in this form.
The optional argument \textit{main}
passes on directly to the main file \textit{main}
while pretending to compile the child \textit{dest}.
This form behaves as if \textit{dest}
issues |\childdocof{|\textit{main}|}| right away,
and no further \textsf{childdoc} directives will be processed.

%%%%%%%%%%%%%%%%%%%%%%%%%%%%%%%%%%%%%%%%
\DescribeMacro{\...prefix}
In the alternative form |\childdocforwardprefix|,
%
\begin{center}
\begin{tabular}{l}
|\input{childdoc.def}|\\
|\childdocforwardprefix[|\textit{main}|]{|\textit{prefix}|}{|\textit{dest}|}|
\end{tabular}
\end{center}
%
the destination file is determined by a pattern
depending on the current file:
To make this work, the current file must be called
`{\textit{prefix}\hspace{0.2em}\textit{suffix}}'
with \textit{prefix} matching precisely the argument.
Processing is then passed on to the file
`{\textit{dest}\hspace{0.2em}\textit{suffix}}'.
Surely, the same effect is achieved by
directly specifying the
argument `{\textit{dest}\hspace{0.2em}\textit{suffix}}'
in the first form.
However, that requires to set up a different file
for each child. With the alternative form of the command
all these files can have exactly the same content
which simplifies setting them up and maintaining them.

For example, the following file |draft.tex|
with a compilation flag |\version| as described in \secref{sec:flags}
compiles the main document as a draft:
%
\begin{center}
\begin{tabular}{l}
|\def\version{draft}|\\
|\input{childdoc.def}|\\
|\childdocforward{|\textit{main}|}|
\end{tabular}
\end{center}
%
Likewise, the following files |final|\textit{nn}|.tex|
compile the final version of the child document
|child|\textit{nn}|.tex|:
%
\begin{center}
\begin{tabular}{l}
|\def\version{final}|\\
|\input{childdoc.def}|\\
|\childdocforwardprefix{final}{child}|
\end{tabular}
\end{center}
%

Note that when several versions of a main file and/or of each child file
are to be generated, it may be convenient to set up a |Makefile| or
shell script to automatise the process.

%%%%%%%%%%%%%%%%%%%%%%%%%%%%%%%%%%%%%%%%%%%%%%%%%%%%%%%%%%%%%%%%%%%%%%%%%%%%%%%%
\subsection{Command Line Processing}
\label{sec:commandline}

The effect of redirection files can also be achieved by invoking
the \LaTeX{} compiler with a more elaborate command line.
Most conveniently this should be done as part
of a shell script or a |Makefile|.

When using \textsf{childdoc} in the main file, the following
command lines effectively perform a redirection
(note that depending on the shell being used,
backslashes may have to be doubled: `|\|' $\to$ `|\\|'):
%
\begin{center}
|... -jobname "|\textit{target}|" |\\|"|[\textit{flags}]%
|\input{childdoc.def}\childdocforward[|\textit{main}|]{|\textit{dest}|}"|
\end{center}
%
Here \textit{target} is the name of the output file,
\textit{main} is the name of the main file
and \textit{dest} is the name of the main or child file to be processed
(all filenames without extensions).
The optional argument \textit{main} can be omitted
if \textit{main} matches \textit{dest}.
Optionally, compilation \textit{flags} can be defined via |\def| commands.
This command line makes the \TeX{} engine believe
it is compiling the file \textit{target}
whose content is specified as the latter parameter.
The provided code then forwards the processing to
\textit{main} or \textit{dest} as described in \secref{sec:forward}.

%%%%%%%%%%%%%%%%%%%%%%%%%%%%%%%%%%%%%%%%%%%%%%%%%%%%%%%%%%%%%%%%%%%%%%%%%%%%%%%%
\subsection{Include by Input}
\label{sec:input}

Including child documents by |\include| has some restrictions by design.
Most notably, the content of a child document always occupies
its own set of pages; pages cannot be shared between child documents.
Usually, this behaviour makes perfect sense
because each child document contain an essential part of the document.
However, in some situations it may be desirable to compose
a document from a collection of parts
without having mandatory page breaks between then.
For this case, the package
provides a mechanism to include parts
by |\input| which can also be processed individually.
However, by construction this mechanism
requires manual handling of the content to be output.

%%%%%%%%%%%%%%%%%%%%%%%%%%%%%%%%%%%%%%%%
\DescribeMacro{\ifchilddocmanual}
The main file should be prepared as usual, see \secref{sec:include}.
However, the document body must make a distinction
between processing of an individual part and of the main document, e.g.:
%
\begin{center}
\begin{tabular}{l}
|\ifchilddocmanual|\\
|\input{\childdocname}|\\
|\||else|\\
\textit{document body with }|\input{|\textit{part}|}|\\
|\||fi|
\end{tabular}
\end{center}
%
The conditional |\ifchilddocmanual| is true whenever
a part to be included by |\input| is being compiled,
and the name of the part is stored in |\childdocname|.

%%%%%%%%%%%%%%%%%%%%%%%%%%%%%%%%%%%%%%%%
\DescribeMacro{\childdocby}
Each part to be included by |\input| should start with:
%
\begin{center}
\begin{tabular}{l}
|\input{childdoc.def}|\\
|\childdocby{|\textit{main}|}|\\
\end{tabular}
\end{center}
%
The directive |\childdocby| is similar to |\childdocof|
described in \secref{sec:include},
but the subsequent selection of content must be done manually.
To that end, both |\ifchilddoc| and |\ifchilddocmanual|
will be true upon processing of a part,
and the name of the part is stored in |\childdocname|.
Note that |\jobname| will be set to the filename of the current part
so that each part receives an individual |.aux| file
that does not interfere with the |.aux| file(s) of the main document.
This behaviour can be altered by the alternative form
|\childdocby[*]{|\textit{main}|}| (with a non-empty optional argument)
which uses the |.aux| file of the main document
by setting |\jobname| to \textit{main}.

%%%%%%%%%%%%%%%%%%%%%%%%%%%%%%%%%%%%%%%%%%%%%%%%%%%%%%%%%%%%%%%%%%%%%%%%%%%%%%%%
\subsection{Driver Development}
\label{sec:driver}

The \textsf{childdoc} mechanism can also be use for the development
of definition files such as \LaTeX{} styles or classes.
This case differs from the above setup with multiple parts
included by |\include| in that no |\includeonly| should be invoked.
This can be achieved by starting the include file
(before |\ProvidesPackage|) with:
%
\begin{center}
\begin{tabular}{l}
|\input{childdoc.def}|\\
|\childdocforward{|\textit{main}|}|\\
\end{tabular}
\end{center}
%
or alternatively with:
%
\begin{center}
\begin{tabular}{l}
|\input{childdoc.def}|\\
|\childdocby{|\textit{main}|}|\\
\end{tabular}
\end{center}
%
Both forms have slightly different effects as described above.
The main file is prepared as usual, see \secref{sec:include}.

%%%%%%%%%%%%%%%%%%%%%%%%%%%%%%%%%%%%%%%%%%%%%%%%%%%%%%%%%%%%%%%%%%%%%%%%%%%%%%%%
\subsection{Legacy Detection}
\label{sec:detection}

The directive |\childdocmain| in the main file can detect
whether the complete document or merely a child is to be compiled
even without using the directive |\childdocof|.
This method is deprecated because it is less robust
and there is no compelling reason to use it;
it is merely provided for backward compatibility
and it may be removed in future versions.

If the detection mechanism is to be used,
it is mandatory to correctly specify
the filename of the main file as the argument of |\childdocmain|:
%
\begin{center}
\begin{tabular}{l}
|\input{childdoc.def}|\\
|\childdocmain{|\textit{main}|}|\\
\end{tabular}
\end{center}
%
If |\jobname| does not match the argument \textit{main} of |\childdocmain|,
it is assumed that |\jobname| points to the child file to be compiled.
When using |\childdocmain| with the main file specified as argument,
it suffices to start a child file
with just |\input{|\textit{main}|}|
without loading of the package and using |\childdocof|.
If instead all processing is done
with the appropriate \textsf{childdoc} directives,
the argument of \textit{main} of |\childdocmain| can be empty.

An alternative version of the command line processing described
in \secref{sec:commandline} using the detection mechanism reads:
%
\begin{center}
|... -jobname "|\textit{target}|" "|[\textit{flags}]%
[|\def\jobname{|\textit{dest}|}|]|\input{|\textit{main}|}"|
\end{center}

%%%%%%%%%%%%%%%%%%%%%%%%%%%%%%%%%%%%%%%%%%%%%%%%%%%%%%%%%%%%%%%%%%%%%%%%%%%%%%%%
\subsection{Manual Code}
\label{sec:manual}

In case one cannot be certain whether the definitions file |childdoc.def|
is installed on the target \TeX{} distribution
and one prefers not to ship it,
it is conceivable to paste a few relevant commands into the sources.

To that end, drop all statements |\input{childdoc.def}|
and perform the replacements as outlined below.
Instead of |\childdocmain{|\textit{main}|}| add the following code
to the top of the main file:
%
\begin{center}
\begin{tabular}{l}
|\||ifdefined\childdocname\endinput\||fi\newif\ifchilddoc|\\
|\edef\childdocname{\scantokens\expandafter{\jobname\noexpand}}|\\
|\def\childdocmain{|\textit{main}|}\||ifx\childdocmain\childdocname\||else|\\
|\childdoctrue\includeonly{\childdocname}\let\jobname\childdocmain\||fi|\\
\end{tabular}
\end{center}
%
Instead of |\childdocof{|\textit{main}|}| just include the main file
at the top of each child file:
%
\begin{center}
|\input{|\textit{main}|}|
\end{center}
%
A simple redirection |\childdocforward{|\textit{dest}|}| is achieved by:
%
\begin{center}
|\def\jobname{|\textit{dest}|}\input{\jobname}|
\end{center}
%
The redirection with prefix
|\childdocforwardprefix[|\textit{prefix}|]{|\textit{dest}|}|
is accomplished by:
%
\begin{center}
\begin{tabular}{l}
|{\edef\jobname{\scantokens\expandafter{\jobname\noexpand}}|\\
|\def\redirectjob |\textit{prefix}|#1~~~{\gdef\jobname{|\textit{dest}|#1}}|\\
|\expandafter\redirectjob\jobname~~~}\input{\jobname}|
\end{tabular}
\end{center}

In an alternative approach,
child documents can be compiled by a specific command line
without additional code or specific definitions:
%
\begin{center}
|... -jobname "|\textit{target}|" "|[\textit{flags}]%
|\includeonly{|\textit{dest}|}\input{|\textit{main}|}"|
\end{center}
%

%%%%%%%%%%%%%%%%%%%%%%%%%%%%%%%%%%%%%%%%%%%%%%%%%%%%%%%%%%%%%%%%%%%%%%%%%%%%%%%%
%%%%%%%%%%%%%%%%%%%%%%%%%%%%%%%%%%%%%%%%%%%%%%%%%%%%%%%%%%%%%%%%%%%%%%%%%%%%%%%%
\section{Information}

%%%%%%%%%%%%%%%%%%%%%%%%%%%%%%%%%%%%%%%%%%%%%%%%%%%%%%%%%%%%%%%%%%%%%%%%%%%%%%%%
\subsection{Copyright}

Copyright \copyright{} 2017--2018 Niklas Beisert

This work may be distributed and/or modified under the
conditions of the \LaTeX{} Project Public License, either version 1.3
of this license or (at your option) any later version.
The latest version of this license is in
  \url{http://www.latex-project.org/lppl.txt}
and version 1.3 or later is part of all distributions of \LaTeX{}
version 2005/12/01 or later.

This work has the LPPL maintenance status `maintained'.

The Current Maintainer of this work is Niklas Beisert.

This work consists of the files |README.txt|, |childdoc.ins| and |childdoc.dtx|
as well as the derived files |childdoc.def|, |cdocsamp.tex|
with |cdocsch1.tex|, |cdocsch2.tex|, |cdocspt3.tex|, |cdocspt4.tex|,
|cdocsdrf.tex|, |cdocsfn1.tex|, |cdocsfn2.tex|
as well as |childdoc.pdf|.

%%%%%%%%%%%%%%%%%%%%%%%%%%%%%%%%%%%%%%%%%%%%%%%%%%%%%%%%%%%%%%%%%%%%%%%%%%%%%%%%
\subsection{Files and Installation}

The package consists of the files:
%
\begin{center}
\begin{tabular}{ll}
    |README.txt|   & readme file \\
    |childdoc.ins| & installation file \\
    |childdoc.dtx| & source file \\
    |childdoc.def| & definition file \\
    |cdocsamp.tex| & sample main file \\
    |cdocsch1.tex| & sample include file \\
    |cdocsch2.tex| & sample include file \\
    |cdocspt3.tex| & sample part file \\
    |cdocspt4.tex| & sample part file \\
    |cdocsdrf.tex| & sample redirection file \\
    |cdocsfn1.tex| & sample redirection file \\
    |cdocsfn2.tex| & sample redirection file \\
    |childdoc.pdf| & manual
\end{tabular}
\end{center}
%
The distribution consists of the files
|README.txt|, |childdoc.ins| and |childdoc.dtx|.
%
\begin{itemize}
\item
Run (pdf)\LaTeX{} on |childdoc.dtx|
to compile the manual |childdoc.pdf| (this file).
\item
Run \LaTeX{} on |childdoc.ins| to create the definitions file |childdoc.def|
and the sample |cdocsamp.tex| with include files
|cdocsch1.tex|, |cdocsch2.tex|, |cdocspt3.tex|, |cdocspt4.tex|,
|cdocsdrf.tex|, |cdocsfn1.tex|, |cdocsfn2.tex|.
Then copy the file |childdoc.def| to an appropriate directory of your \LaTeX{}
distribution, e.g.\ \textit{texmf-root}|/tex/latex/childdoc|.
\end{itemize}

%%%%%%%%%%%%%%%%%%%%%%%%%%%%%%%%%%%%%%%%%%%%%%%%%%%%%%%%%%%%%%%%%%%%%%%%%%%%%%%%
\subsection{Related CTAN Packages}

There are several other packages which offer a similar functionality:
%
\begin{itemize}
\item
The packages
\href{http://ctan.org/pkg/docmute}{\textsf{docmute}},
\href{http://ctan.org/pkg/includex}{\textsf{includex}} and
\href{http://ctan.org/pkg/standalone}{\textsf{standalone}}
provide commands to include only the document body of
a child file thus allowing both files to be compiled individually.
\item
The packages \href{http://ctan.org/pkg/subdocs}{\textsf{subdocs}}
and \href{http://ctan.org/pkg/subfiles}{\textsf{subfiles}}
provide structures in which the main and child documents can be
encapsulated and allowing them to be compiled individually.
The inclusion mechanism is different from the conventional |\include|.
\item
The package \href{http://ctan.org/pkg/combine}{\textsf{combine}}
is an elaborate solution to combine several documents into one.
\end{itemize}
%
See also the CTAN topic \href{http://ctan.org/topic/subdocs}{\textsf{subdocs}}
for further related packages.
The present package differs from the above solutions in that
a document structure constructed with the conventional |\include| mechanism
just needs two extra commands at the top of every file
such that all constituent files can be compiled individually.

%%%%%%%%%%%%%%%%%%%%%%%%%%%%%%%%%%%%%%%%%%%%%%%%%%%%%%%%%%%%%%%%%%%%%%%%%%%%%%%%
%\subsection{Feature Suggestions}
%
%The following is a list of features which may be useful for future
%versions of this package:
%%
%\begin{itemize}
%\item
%\ldots
%\end{itemize}

%%%%%%%%%%%%%%%%%%%%%%%%%%%%%%%%%%%%%%%%%%%%%%%%%%%%%%%%%%%%%%%%%%%%%%%%%%%%%%%%
\subsection{Revision History}

%%%%%%%%%%%%%%%%%%%%%%%%%%%%%%%%%%%%%%%%
\paragraph{v2.0:} 2018/12/30

\begin{itemize}
\item
immediate forward processing
\item
added |\childdocby| mechanism
\item
manual restructured
\end{itemize}

%%%%%%%%%%%%%%%%%%%%%%%%%%%%%%%%%%%%%%%%
\paragraph{v1.6:} 2018/01/17

\begin{itemize}
\item
application for development of include files
\item
corrections to manual
\end{itemize}

%%%%%%%%%%%%%%%%%%%%%%%%%%%%%%%%%%%%%%%%
\paragraph{v1.5:} 2017/05/21

\begin{itemize}
\item
more complete structuring introduced
\item
|\childdocof| introduced
\item
|\childdoc| renamed to |\childdocmain|
\item
|\childredirect| renamed to |\childdocforward| and |\childdocforwardprefix|
and functionality expanded
\end{itemize}

%%%%%%%%%%%%%%%%%%%%%%%%%%%%%%%%%%%%%%%%
\paragraph{v1.0:} 2017/04/27

\begin{itemize}
\item
manual and install package
\item
first version published on CTAN
\end{itemize}

%%%%%%%%%%%%%%%%%%%%%%%%%%%%%%%%%%%%%%%%
\paragraph{v0.6:} 2017/04/26

\begin{itemize}
\item
redirection mechanism added
\end{itemize}

%%%%%%%%%%%%%%%%%%%%%%%%%%%%%%%%%%%%%%%%
\paragraph{v0.5:} 2017/04/26

\begin{itemize}
\item
functionality in definition file
\end{itemize}


%%%%%%%%%%%%%%%%%%%%%%%%%%%%%%%%%%%%%%%%%%%%%%%%%%%%%%%%%%%%%%%%%%%%%%%%%%%%%%%%
%%%%%%%%%%%%%%%%%%%%%%%%%%%%%%%%%%%%%%%%%%%%%%%%%%%%%%%%%%%%%%%%%%%%%%%%%%%%%%%%
%%%%%%%%%%%%%%%%%%%%%%%%%%%%%%%%%%%%%%%%%%%%%%%%%%%%%%%%%%%%%%%%%%%%%%%%%%%%%%%%
\appendix

\settowidth\MacroIndent{\rmfamily\scriptsize 000\ }

 \DocInput{childdoc.dtx}

\end{document}
%</driver>
% \fi
%
% %%%%%%%%%%%%%%%%%%%%%%%%%%%%%%%%%%%%%%%%%%%%%%%%%%%%%%%%%%%%%%%%%%%%%%%%%%%%%%
% %%%%%%%%%%%%%%%%%%%%%%%%%%%%%%%%%%%%%%%%%%%%%%%%%%%%%%%%%%%%%%%%%%%%%%%%%%%%%%
% \section{Sample}
%\iffalse
%<*samplemain>
%\fi
%
% The following presents a sample document
% with two chapters, two parts, a title page,
% a compile flag as well as three forwarding files to set the flag.
% It consists of eight |.tex| files:
% \begin{center}
% \begin{tabular}{ll}
% |cdocsamp.tex|&main file\\
% |cdocsch1.tex|&include file for chapter 1\\
% |cdocsch2.tex|&include file for chapter 2\\
% |cdocspt3.tex|&include file for part 3\\
% |cdocspt4.tex|&include file for part 4\\
% |cdocsdrf.tex|&forwarding file for main file in draft mode\\
% |cdocsfi1.tex|&forwarding file for final version of chapter 1\\
% |cdocsfi2.tex|&forwarding file for final version of chapter 2\\
% \end{tabular}
% \end{center}
% Each of the eight files can be compiled directly by the \LaTeX{} compiler.
%
% %%%%%%%%%%%%%%%%%%%%%%%%%%%%%%%%%%%%%%
% \paragraph{Main File.}
%
% The main file is called |cdocsamp.tex|.
%
% Load the \textsf{childdoc} definitions and
% declare the filename for the main document:
%    \begin{macrocode}
\input{childdoc.def}
\childdocmain{}
%    \end{macrocode}

% Optional override for |\version| flag:
%    \begin{macrocode}
%%\ifchilddoc\else\providecommand{\version}{draft}\fi
%    \end{macrocode}

% Define the default values for the |\version| flag
% (|final| for the main file and |draft| for childs):
%    \begin{macrocode}
\ifchilddoc
\providecommand{\version}{draft}
\else
\providecommand{\version}{final}
\fi
%    \end{macrocode}

% Load the standard document class:
%    \begin{macrocode}
\documentclass[12pt]{article}
%    \end{macrocode}

% Start the document body:
%    \begin{macrocode}
\begin{document}
%    \end{macrocode}

% Declare a title page.
% Print title, part of document being processed and version flag:
%    \begin{macrocode}
\addtocounter{page}{-1}
\begin{center}
{\LARGE\bfseries{}childdoc example\par}
\vspace{1cm}
\ifchilddoc
\ifchilddocmanual part\else chapter\fi:
`\childdocname' of `\childdocjob'\par
\else
main document: `\childdocjob'\par
\fi
version: \version\par
\end{center}
\newpage
%    \end{macrocode}

% Manually include selected file,
% otherwise process as usual:
%    \begin{macrocode}
\ifchilddocmanual
\section*{part `\childdocname'}
\input{\childdocname}
\else
%    \end{macrocode}

% Include the two chapters:
%    \begin{macrocode}
\include{cdocsch1}
\include{cdocsch2}
%    \end{macrocode}

% Include the two parts unless only chapters should be displayed:
%    \begin{macrocode}
\ifchilddoc\else
\section{part three}
\input{cdocspt3}
\section{part four}
\input{cdocspt4}
\fi
%    \end{macrocode}

% Process as usual until here:
%    \begin{macrocode}
\fi
%    \end{macrocode}

% End of document body:
%    \begin{macrocode}
\end{document}
%    \end{macrocode}
%\iffalse
%</samplemain>
%\fi
%
% %%%%%%%%%%%%%%%%%%%%%%%%%%%%%%%%%%%%%%
% \paragraph{Chapter Include Files.}
%
% The include files are called |cdocsch1.tex| and |cdocsch2.tex|.
%
%\iffalse
%<*samplechap1|samplechap2>
%\fi

% Optional override for |\version| flag:
%    \begin{macrocode}
%%\providecommand{\version}{final}
%    \end{macrocode}

% Include the main document:
%    \begin{macrocode}
\input{childdoc.def}
\childdocof{cdocsamp}
%    \end{macrocode}

%\iffalse
%</samplechap1|samplechap2>
%\fi
%
%\iffalse
%<*samplechap1>
%\fi
% Some text for chapter 1:
%    \begin{macrocode}
\section{one}
some text in chapter one
%    \end{macrocode}

%\iffalse
%</samplechap1>
%\fi
% Some text for chapter 2:
%\iffalse
%<*samplechap2>
%\fi
%    \begin{macrocode}
\section{two}
more text in chapter two
%    \end{macrocode}

%\iffalse
%</samplechap2>
%\fi
%
% %%%%%%%%%%%%%%%%%%%%%%%%%%%%%%%%%%%%%%
% \paragraph{Part Include Files.}
%
% The include files are called |cdocspt3.tex| and |cdocspt4.tex|.
%
%\iffalse
%<*samplepart3|samplepart4>
%\fi

% Optional override for |\version| flag:
%    \begin{macrocode}
%%\providecommand{\version}{final}
%    \end{macrocode}

% Include the main document:
%    \begin{macrocode}
\input{childdoc.def}
\childdocby{cdocsamp}
%    \end{macrocode}

%\iffalse
%</samplepart3|samplepart4>
%\fi
%
%\iffalse
%<*samplepart3>
%\fi
% Some text for part 3:
%    \begin{macrocode}
some text in part three
%    \end{macrocode}

%\iffalse
%</samplepart3>
%\fi
% Some text for part 4:
%\iffalse
%<*samplepart4>
%\fi
%    \begin{macrocode}
more text in part four
%    \end{macrocode}

%\iffalse
%</samplepart4>
%\fi
%
% %%%%%%%%%%%%%%%%%%%%%%%%%%%%%%%%%%%%%%
% \paragraph{Forwarding for a Complete Draft.}
%
% The following forwarding file |cdocsdrf.tex|
% compiles the main document in draft mode:
%\iffalse
%<*sampledraft>
%\fi
%    \begin{macrocode}
\def\version{draft}
\input{childdoc.def}
\childdocforward{cdocsamp}
%    \end{macrocode}

%\iffalse
%</sampledraft>
%\fi
%
% %%%%%%%%%%%%%%%%%%%%%%%%%%%%%%%%%%%%%%
% \paragraph{Forwarding for Final Version of the Chapters.}
%
% The following forwarding files |cdocsfn1.tex| and |cdocsfn2.tex|
% (with identical content)
% compile the final versions of the child documents
% |cdocsch1.tex| and |cdocsch2.tex|, respectively:
%\iffalse
%<*samplefinal>
%\fi
%    \begin{macrocode}
\def\version{final}
\input{childdoc.def}
\childdocforwardprefix[cdocsamp]{cdocsfn}{cdocsch}
%    \end{macrocode}

%\iffalse
%</samplefinal>
%\fi
%
% %%%%%%%%%%%%%%%%%%%%%%%%%%%%%%%%%%%%%%
% \paragraph{Command Line Processing.}
%
% The following three command lines generate the output files
% |cdocscld|, |cdocscl1| and |cdocscl2|
% which should be identical to
% |cdocsdrf|, |cdocsch1| and |cdocsfn2|, respectively:
% \begin{center}
% \begin{tabular}{l}
% |latex -jobname cdocscld \|\\
% |  "\def\version{draft}\input{childdoc.def}\childdocforward{cdocsamp}"|\\
% |latex -jobname cdocscl1 \|\\
% |  "\input{childdoc.def}\childdocforward[cdocsamp]{cdocsch1}"|\\
% |latex -jobname cdocscl2 \|\\
% |  "\def\version{final}\input{childdoc.def}\childdocforward{cdocsch2}"|
% \end{tabular}
% \end{center}
% Note that the trailing backslash on each first line
% merely continues the input to the second line
% (for convenient cut ant paste).
% Furthermore, the command |latex| can be replaced by any
% of its alternative versions such as |pdflatex|.
%
% %%%%%%%%%%%%%%%%%%%%%%%%%%%%%%%%%%%%%%%%%%%%%%%%%%%%%%%%%%%%%%%%%%%%%%%%%%%%%%
% %%%%%%%%%%%%%%%%%%%%%%%%%%%%%%%%%%%%%%%%%%%%%%%%%%%%%%%%%%%%%%%%%%%%%%%%%%%%%%
% \section{Implementation}
%\iffalse
%<*package>
%\fi
%
% This section describes the definitions file |childdoc.def|.

% The definitions cannot be loaded using |\usepackage| or |\RequirePackage|
% which has a mechanism to prevent loading a style file more than once.
% When loading the definitions by means of |\input|
% multiple instances have to be prevented manually:
%\iffalse
%This code needs to be before the `\ProvidesFile' directive
%which is defined at the beginning of this file.
%Therefore it is also placed there and commented out here.
%</package>
%<*discard>
%\fi
%    \begin{macrocode}
\ifdefined\childdocmain\endinput\fi
%    \end{macrocode}
%\iffalse
%</discard>
%<*package>
%\fi
%
% \macro{\ifchilddoc}
% \macro{\ifchilddocmanual}
% The conditional |\ifchilddoc| tells whether a
% child (true) or main (false) document is being compiled.
% The conditional |\ifchilddocmanual| tells whether
% the |\includeonly| mechanism is used (false) or
% the selection of child files must be performed manually (true).
% The definitions initialise to false:
%    \begin{macrocode}
\newif\ifchilddoc
\newif\ifchilddocmanual
%    \end{macrocode}

% \macro{\childdocname}
% \macro{\childdocjob}
% The macro |\childdocname| stores the name of the main document
% to be compiled. The macro |\childdocjob| stores the name of
% the document on which the \LaTeX{} compiler was originally invoked.
% The content of |\jobname| cannot be compared
% to filenames specified in the source due to different catcodes.
% The following code rescans |\jobname|, stores the result
% in |\childdocname| and saves a copy in |\childdocjob|:
%    \begin{macrocode}
\edef\childdocname{\scantokens\expandafter{\jobname\noexpand}}
\let\childdocjob\childdocname
%    \end{macrocode}

% \macro{\childdocdisable}
% The macro |\childdocdisable| prevents the main file
% from being processed more than once.
% At this stage, the main document command |\childdocmain|
% is assumed to be called once again where it should do nothing.
% Any subsequent call to it should prevent
% a secondary processing of the main document
% It overwrites the forwarding commands
% |\childdocof| and |\childdocforward|
% with empty macros to prevent further inclusions of the main document:
%    \begin{macrocode}
\newcommand{\childdocdisable}
{
  \renewcommand{\childdocmain}[1]{\renewcommand{\childdocmain}[1]{\endinput}}
  \renewcommand{\childdocof}[1]{}
  \renewcommand{\childdocby}[2][]{}
  \renewcommand{\childdocforward}[2][]{}
  \renewcommand{\childdocdisable}{}
}
%    \end{macrocode}

% \macro{\childdocmain}
% The macro |\childdocmain| is to be called at the top of the main file
% with nothing or the main filename (without extension) as argument.
% First, it breaks loops.
% If the argument is not empty and does not match |\childdocname|
% (which is set by the first inclusion of |childdoc.def|),
% |\ifchilddoc| is set to true, |\includeonly| is applied to the child file
% and |\jobname| is set to the main file
% (for proper handling of |.aux| files):
%    \begin{macrocode}
\newcommand{\childdocmain}[1]
{
  \childdocdisable\childdocmain{}
  \if?#1?\else
    \begingroup
      \def\childdoctmp{#1}
      \ifx\childdoctmp\childdocname
        \def\childdoctmp{}
      \else
        \def\childdoctmp
        {
          \childdoctrue
          \includeonly{\childdocname}
          \def\childdocjob{#1}
          \def\jobname{#1}
        }
      \fi
      \expandafter
    \endgroup
    \childdoctmp
  \fi
}
%    \end{macrocode}

% \macro{\childdocof}
% The command |\childdocof| redirects
% compilation to the main file |#1|.
%    \begin{macrocode}
\newcommand{\childdocof}[1]
{
  \childdocdisable
  \childdoctrue
  \includeonly{\childdocname}
  \def\jobname{#1}
  \def\childdocjob{#1}
  \input{#1}
}
%    \end{macrocode}

% \macro{\childdocby}
% The command |\childdocby| ....
%    \begin{macrocode}
\newcommand{\childdocby}[2][]
{
  \childdocdisable
  \childdoctrue
  \childdocmanualtrue
  \if?#1?\else
    \def\jobname{#2}
  \fi
  \def\childdocjob{#2}
  \input{#2}
  \endinput
}
%    \end{macrocode}

% \macro{\childdocforward}
% The command |\childdocforward| redirects
% compilation to the main file or
% (if the optional argument is given) a child file.
% Parameters are set as if the main file
% or a child file starting with |\childdocof| was compiled.
% Then compilation is handed over to the main file:
%    \begin{macrocode}
\newcommand{\childdocforward}[2][]
{
  \begingroup
    \if?#1?
      \def\childdoctmp
      {
        \def\childdocname{#2}
        \def\childdocjob{#2}
        \def\jobname{#2}
        \input{#2}
        \endinput
      }
    \else
      \def\childdoctmp
      {
        \childdocdisable
        \def\childdocname{#2}
        \childdoctrue
        \includeonly{#2}
        \def\childdocjob{#1}
        \def\jobname{#1}
        \input{#1}
        \endinput
      }
    \fi
    \expandafter
  \endgroup
  \childdoctmp
}
%    \end{macrocode}

% \macro{\childdocforwardprefix}
% The command |\childdocforwardprefix| redirects
% compilation to the main or a child file by means of a pattern.
% The prefix |#1| in the current filename is replaced by |#2|
% and the suffix of the current filename is kept
% (it is assumed that the filename does not contain the substring `|~~~|'
% which is used as a delimiter).
% Compilation is handed over to the new file by |\childdocforward|:
%    \begin{macrocode}
\newcommand{\childdocforwardprefix}[3][]
{
  \begingroup
    \def\childdocextract #2##1~~~{\def\childdoctmp{\childdocforward[#1]{#3##1}}}
    \expandafter\childdocextract\childdocname~~~
    \expandafter
  \endgroup
  \childdoctmp
}
%    \end{macrocode}

% \macro{\childdoc}
% The deprecated macro |\childdoc| is a legacy version of |\childdocmain|:
%    \begin{macrocode}
\newcommand{\childdoc}{\childdocmain}
%    \end{macrocode}

% \macro{\childdocredirect}
% The deprecated macro |\childdocredirect| is a legacy version
% of |\childdocforward| and |\childdocforwardprefix|:
%    \begin{macrocode}
\newcommand{\childdocredirect}[2][]
{
  \begingroup
    \if?#1?
      \def\childdoctmp{\childdocforward{#2}}
    \else
      \def\childdoctmp{\childdocforwardprefix{#1}{#2}}
    \fi
    \expandafter
  \endgroup
  \childdoctmp
}
%    \end{macrocode}

%\iffalse
%</package>
%\fi
%
\endinput

\childdocmain{}
%    \end{macrocode}

% Optional override for |\version| flag:
%    \begin{macrocode}
%%\ifchilddoc\else\providecommand{\version}{draft}\fi
%    \end{macrocode}

% Define the default values for the |\version| flag
% (|final| for the main file and |draft| for childs):
%    \begin{macrocode}
\ifchilddoc
\providecommand{\version}{draft}
\else
\providecommand{\version}{final}
\fi
%    \end{macrocode}

% Load the standard document class:
%    \begin{macrocode}
\documentclass[12pt]{article}
%    \end{macrocode}

% Start the document body:
%    \begin{macrocode}
\begin{document}
%    \end{macrocode}

% Declare a title page.
% Print title, part of document being processed and version flag:
%    \begin{macrocode}
\addtocounter{page}{-1}
\begin{center}
{\LARGE\bfseries{}childdoc example\par}
\vspace{1cm}
\ifchilddoc
\ifchilddocmanual part\else chapter\fi:
`\childdocname' of `\childdocjob'\par
\else
main document: `\childdocjob'\par
\fi
version: \version\par
\end{center}
\newpage
%    \end{macrocode}

% Manually include selected file,
% otherwise process as usual:
%    \begin{macrocode}
\ifchilddocmanual
\section*{part `\childdocname'}
\input{\childdocname}
\else
%    \end{macrocode}

% Include the two chapters:
%    \begin{macrocode}
\include{cdocsch1}
\include{cdocsch2}
%    \end{macrocode}

% Include the two parts unless only chapters should be displayed:
%    \begin{macrocode}
\ifchilddoc\else
\section{part three}
\input{cdocspt3}
\section{part four}
\input{cdocspt4}
\fi
%    \end{macrocode}

% Process as usual until here:
%    \begin{macrocode}
\fi
%    \end{macrocode}

% End of document body:
%    \begin{macrocode}
\end{document}
%    \end{macrocode}
%\iffalse
%</samplemain>
%\fi
%
% %%%%%%%%%%%%%%%%%%%%%%%%%%%%%%%%%%%%%%
% \paragraph{Chapter Include Files.}
%
% The include files are called |cdocsch1.tex| and |cdocsch2.tex|.
%
%\iffalse
%<*samplechap1|samplechap2>
%\fi

% Optional override for |\version| flag:
%    \begin{macrocode}
%%\providecommand{\version}{final}
%    \end{macrocode}

% Include the main document:
%    \begin{macrocode}
% \iffalse
%
% childdoc.dtx Copyright (C) 2017-2018 Niklas Beisert
%
% This work may be distributed and/or modified under the
% conditions of the LaTeX Project Public License, either version 1.3
% of this license or (at your option) any later version.
% The latest version of this license is in
%   http://www.latex-project.org/lppl.txt
% and version 1.3 or later is part of all distributions of LaTeX
% version 2005/12/01 or later.
%
% This work has the LPPL maintenance status `maintained'.
%
% The Current Maintainer of this work is Niklas Beisert.
%
% This work consists of the files childdoc.dtx and childdoc.ins
% and the derived files childdoc.def and cdocsamp.tex with
% cdocsch1.tex, cdocsch2.tex, cdocsdrf.tex, cdocsfn1.tex, cdocsfn2.tex.
%
%<package>\ifdefined\childdocmain\endinput\fi
%<package>\ProvidesFile{childdoc.def}[2018/12/30 v2.0 child document driver]
%<samplemain>\ProvidesFile{cdocsamp.tex}[2018/12/30 v2.0 sample for childdoc]
%<*driver>
%\ProvidesFile{childdoc.drv}[2018/12/30 v2.0 childdoc reference manual file]
\PassOptionsToClass{10pt,a4paper}{article}
\documentclass{ltxdoc}

\usepackage[margin=35mm]{geometry}
\usepackage{hyperref}
\usepackage{hyperxmp}
\usepackage[usenames]{color}

\hypersetup{colorlinks=true}
\hypersetup{pdfstartview=FitH}
\hypersetup{pdfpagemode=UseNone}
\hypersetup{pdfsource={}}
\hypersetup{pdflang={en-UK}}
\hypersetup{pdfcopyright={Copyright 2017-2018 Niklas Beisert.
  This work may be distributed and/or modified under the
  conditions of the LaTeX Project Public License, either version 1.3
  of this license or (at your option) any later version.}}
\hypersetup{pdflicenseurl={http://www.latex-project.org/lppl.txt}}
\hypersetup{pdfcontactaddress={ETH Zurich, ITP, HIT K,
  Wolfgang-Pauli-Strasse 27}}
\hypersetup{pdfcontactpostcode={8093}}
\hypersetup{pdfcontactcity={Zurich}}
\hypersetup{pdfcontactcountry={Switzerland}}
\hypersetup{pdfcontactemail={nbeisert@itp.phys.ethz.ch}}
\hypersetup{pdfcontacturl={http://people.phys.ethz.ch/\xmptilde nbeisert/}}

\newcommand{\secref}[1]{\hyperref[#1]{section \ref*{#1}}}

\parskip1ex
\parindent0pt
\let\olditemize\itemize
\def\itemize{\olditemize\parskip0pt}

\begin{document}

\title{The \textsf{childdoc} Package}
\hypersetup{pdftitle={The childdoc Package}}
\author{Niklas Beisert\\[2ex]
  Institut f\"ur Theoretische Physik\\
  Eidgen\"ossische Technische Hochschule Z\"urich\\
  Wolfgang-Pauli-Strasse 27, 8093 Z\"urich, Switzerland\\[1ex]
  \href{mailto:nbeisert@itp.phys.ethz.ch}
  {\texttt{nbeisert@itp.phys.ethz.ch}}}
\hypersetup{pdfauthor={Niklas Beisert}}
\hypersetup{pdfsubject={Manual for the LaTeX2e Package childdoc}}
\date{30 December 2018, \textsf{v2.0}}
\maketitle

\begin{abstract}\noindent
\textsf{childdoc} is a \LaTeXe{} package
that enables the direct compilation
of document sections included by |\include|
to individual files.
\end{abstract}

\begingroup
\parskip0ex
\tableofcontents
\endgroup

%%%%%%%%%%%%%%%%%%%%%%%%%%%%%%%%%%%%%%%%%%%%%%%%%%%%%%%%%%%%%%%%%%%%%%%%%%%%%%%%
%%%%%%%%%%%%%%%%%%%%%%%%%%%%%%%%%%%%%%%%%%%%%%%%%%%%%%%%%%%%%%%%%%%%%%%%%%%%%%%%
\section{Introduction}

\LaTeX{} provides a mechanism to structure a large document (such as a book)
into a main file and several child files (containing the chapters)
using the |\include| command.
This mechanism is beneficial for documents
which span hundreds of pages in order to
make the source file(s) more manageable.
Moreover, compilation can be restricted to
selected child files by means of the |\includeonly| command.
The latter feature can be used to reduce the compilation time while editing
(this was significantly more useful in the earlier days of \LaTeX{})
or to generate a smaller document which is easier to navigate.
Another application of |\includeonly| is to generate
documents consisting of selected parts of the complete document.

However, there are a few drawbacks of the plain |\include| mechanism:
\begin{itemize}
\item
The child files cannot be compiled on their own,
they can only be compiled via the main file.
A naive editing environment
(such as a text editor with an option
to have the current file processed by \LaTeX)
may require one to switch to the main file before compiling;
attempting to compile the child file produces errors.
\item
The main file must be modified (each time)
to adjust the |\includeonly| command
to the present needs. This easily leaves the main file in a messy state.
\item
The generated document will always carry the filename
of the main document. This is inconvenient if
several child files are to be compiled and
to be kept for distribution.
\end{itemize}

The present package provides a simple interface
to make child files individually compilable by \LaTeX{}.
Compiling a child file then has the same effect as compiling
the main file with an |\includeonly| command
to select the appropriate child.
Moreover the generated document will carry the name of the child
rather than the main file.
This resolves all three above issues.

This feature is meant to make the editing of books,
thesis documents and lecture notes somewhat more convenient.
However, the package can also be used efficiently for
composing a series of documents (such as exercise sheets)
which are typically distributed individually.
It then assists the author in generating the individual documents
(potentially in different versions)
as well as a document containing the collected series.
Another application is in developing style files
or other kinds of included material
where compilation of the style file could redirect
to a sample or test file.

%%%%%%%%%%%%%%%%%%%%%%%%%%%%%%%%%%%%%%%%%%%%%%%%%%%%%%%%%%%%%%%%%%%%%%%%%%%%%%%%
%%%%%%%%%%%%%%%%%%%%%%%%%%%%%%%%%%%%%%%%%%%%%%%%%%%%%%%%%%%%%%%%%%%%%%%%%%%%%%%%
\section{Usage}

First of all, the package \textsf{childdoc} is \emph{not} a standard
\LaTeXe{} |.sty| style file! Therefore it needs to be invoked in
a non-standard way.

%%%%%%%%%%%%%%%%%%%%%%%%%%%%%%%%%%%%%%%%%%%%%%%%%%%%%%%%%%%%%%%%%%%%%%%%%%%%%%%%
\subsection{Included Files}
\label{sec:include}

%%%%%%%%%%%%%%%%%%%%%%%%%%%%%%%%%%%%%%%%
\DescribeMacro{\childdocmain}
To use the package, add the commands
\begin{center}
\begin{tabular}{l}
|\input{childdoc.def}|\\
|\childdocmain{}|\\
\end{tabular}
\end{center}
at the very top of the main \LaTeX{} file,
in particular \emph{before} the |\documentclass| statement!
The argument of |\childdocmain| should be left empty
(but it must be present).

%%%%%%%%%%%%%%%%%%%%%%%%%%%%%%%%%%%%%%%%
\DescribeMacro{\childdocof}
Furthermore, add the commands
\begin{center}
\begin{tabular}{l}
|\input{childdoc.def}|\\
|\childdocof{|\textit{main}|}|\\
\end{tabular}
\end{center}
at the top of every child file \textit{child}
which is included by |\include{|\textit{child}|}|
from within the main file
(or at least for those files to be compiled individually).
The argument \textit{main} must be the filename of the main file.

There are a couple of
considerations in setting up the main and child documents:

%%%%%%%%%%%%%%%%%%%%%%%%%%%%%%%%%%%%%%%%
\paragraph{Restrictions.}

Please note the following restrictions:
\begin{itemize}
\item
|\childdocmain| must be called with one argument \textit{main}
to ensure compatibility with earlier version of the package.
It must either be empty (|\childdocmain{}|)
or precisely match the filename of the main file in which it is specified.
See \secref{sec:detection} for further information.
\item
The filename \textit{main} must be specified without the |.tex| extension.
\item
The filename \textit{main} is case sensitive
(even in case-insensitive file systems)
due to internal string comparison.
\item
The argument \textit{main} should be fully expanded, it cannot be a macro.
\item
Subdirectories and special characters should be avoided in filenames.
\item
The command |\childdocmain{|\textit{main}|}| must be followed by a whitespace.
It should not be followed immediately by another command
or by a comment mark `|%|'.
This is because the \TeX{} parser reads the token immediately following
the argument of |\childdocmain| and puts it
at the beginning of every child section;
however, a white\-space is ignored.
\end{itemize}

%%%%%%%%%%%%%%%%%%%%%%%%%%%%%%%%%%%%%%%%
\paragraph{Content of Main File.}

It is advisable to place all content in the child files included by |\include|.
Any output contained in the main file will appear in all child documents
unless suppressed manually;
it cannot be suppressed automatically by the |\includeonly| directive
and thus should normally be avoided.
A method to include some content in the main file
by means of conditional processing is described in \secref{sec:conditional}.

%%%%%%%%%%%%%%%%%%%%%%%%%%%%%%%%%%%%%%%%
\paragraph{Page Numbering.}

When only a part of the document is compiled,
the appropriate numbering of pages
(as well as other status parameters)
is determined from the |.aux| files.
The latter contain information from previous passes.
However this information needs to propagate through
all intermediate child documents.
Therefore the page numbering in child documents may well
be inconsistent until the complete document is compiled at least once.

A useful (if unconventional) way to always ensure a consistent
page numbering is to restart the numbering in each child document
and denote the pages by `\textit{child}|.|\textit{page}'
where \textit{child} represents the chapter/section number of the child file.
This can be achieved by the command
|\numberwithin{page}{|\textit{child}|}|
of the \textsf{amsmath} package
where \textit{child} can be |chapter| or |section|
depending on the chosen structuring.
Alternatively, one can modify the macro |\thepage| appropriately
and reset the counter |page| at the start of each child file.

%%%%%%%%%%%%%%%%%%%%%%%%%%%%%%%%%%%%%%%%%%%%%%%%%%%%%%%%%%%%%%%%%%%%%%%%%%%%%%%%
\subsection{Conditional Processing}
\label{sec:conditional}

The package provides a mechanism to compile different versions
of a document. To customise the versions further some conditional processing
can come in handy to distinguish which version is being compiled.
The package provides two macros to describe the compilation context:

%%%%%%%%%%%%%%%%%%%%%%%%%%%%%%%%%%%%%%%%
\DescribeMacro{\ifchilddoc}
The conditional |\ifchilddoc| distinguishes between the compilation of
child documents and the main document:
%
\begin{center}
|\ifchilddoc |\textit{child-code}| |[|\||else |\textit{main-code}]| \||fi|
\end{center}

%%%%%%%%%%%%%%%%%%%%%%%%%%%%%%%%%%%%%%%%
\DescribeMacro{\childdocname}
\DescribeMacro{\childdocjob}
The macro |\childdocname| contains the filename (without extension)
of the main or child file being processed.
Note that |\childdocjob| will always contain the name of the main file.

%%%%%%%%%%%%%%%%%%%%%%%%%%%%%%%%%%%%%%%%
\paragraph{Title Page.}

Conditional processing can be used to include a title or banner page
in the main document when proper precautions are taken.
Importantly, the code in the main file should ensure that the page counter
(as well as other status parameters which are stored in the |.aux| files)
takes the same value after the conditional processing.
Otherwise the page numbers may take divergent values
depending on which part is compiled.

For example, a title page could be declared by:
%
\begin{center}
\begin{tabular}{l}
|\ifchilddoc\||else|\\
|\addtocounter{page}{-1}|\\
\textit{code for title page}\\
|\newpage|\\
|\||fi|
\end{tabular}
\end{center}
%
A banner page for the child documents can be generated by:
%
\begin{center}
\begin{tabular}{l}
|\ifchilddoc|\\
|\addtocounter{page}{-1}|\\
\textit{code for banner page}\\
|\newpage|\\
|\||fi|
\end{tabular}
\end{center}
%
Here one could write a message such as:
\begin{center}
|This is the part \childdocname{} of \childdocjob{}.|
\end{center}

%%%%%%%%%%%%%%%%%%%%%%%%%%%%%%%%%%%%%%%%%%%%%%%%%%%%%%%%%%%%%%%%%%%%%%%%%%%%%%%%
\subsection{Flags}
\label{sec:flags}

The package makes it easy to generate different versions
of the main or child documents.
To this end compilation flags can be defined
and assigned different default values.
They will be particularly useful in conjunction
with the forwarding mechanism described in \secref{sec:forward}.

For example, it may be useful to have a flag |\version|
which can be set to |draft| or |final|.
The document source will contain some conditional code
depending on the value of |\version|.
Suppose further, the flag should default to |final| for the main file
and to |draft| for child files
which is a natural assignment for editing the document.
This is achieved by placing the following code
in the preamble of the main document
(below the |\childdocmain| directive):
%
\begin{center}
\begin{tabular}{l}
|\ifchilddoc|\\
|\providecommand{\version}{draft}|\\
|\||else|\\
|\providecommand{\version}{final}|\\
|\||fi|
\end{tabular}
\end{center}
%
The definition by |\providecommand| makes sure
that previous definitions are not overwritten.
Further statements |\providecommand{\version}{...}|
can thus be added before the above code to override it.

For the main file, one might add a line
(between |\childdocmain| and the above block)
%
\begin{center}
|%\ifchilddoc\||else\providecommand{\version}{draft}\||fi|
\end{center}
%
which can be uncommented to produce a draft version.
Likewise one can add a line to the very top of a child file
(above the |\childdocof{|\textit{main}|}| directive)
%
\begin{center}
|%\providecommand{\version}{final}|
\end{center}
%
which can be uncommented to produce the final version of this child document.

%%%%%%%%%%%%%%%%%%%%%%%%%%%%%%%%%%%%%%%%%%%%%%%%%%%%%%%%%%%%%%%%%%%%%%%%%%%%%%%%
\subsection{Forwarding}
\label{sec:forward}

Different versions of the main or child documents
using compilation flags as described in \secref{sec:flags}
can be (permanently) stored in different files
for convenient compilation, viewing and distribution.
To this end, the package defines a command
to pass on compilation to a different file:

%%%%%%%%%%%%%%%%%%%%%%%%%%%%%%%%%%%%%%%%
\DescribeMacro{\childdocforward}
The command |\childdocforward| redirects processing to
another source file:
%
\begin{center}
\begin{tabular}{l}
|\input{childdoc.def}|\\
|\childdocforward[|\textit{main}|]{|\textit{dest}|}|\\
\end{tabular}
\end{center}
%
The argument \textit{dest} is the destination file
(without extension).
It should be the main file or one of the child files.
Note that further \textsf{childdoc} directives
such as |\childdocof| and |\childdocforward|
in the indicated file will be processed in this form.
The optional argument \textit{main}
passes on directly to the main file \textit{main}
while pretending to compile the child \textit{dest}.
This form behaves as if \textit{dest}
issues |\childdocof{|\textit{main}|}| right away,
and no further \textsf{childdoc} directives will be processed.

%%%%%%%%%%%%%%%%%%%%%%%%%%%%%%%%%%%%%%%%
\DescribeMacro{\...prefix}
In the alternative form |\childdocforwardprefix|,
%
\begin{center}
\begin{tabular}{l}
|\input{childdoc.def}|\\
|\childdocforwardprefix[|\textit{main}|]{|\textit{prefix}|}{|\textit{dest}|}|
\end{tabular}
\end{center}
%
the destination file is determined by a pattern
depending on the current file:
To make this work, the current file must be called
`{\textit{prefix}\hspace{0.2em}\textit{suffix}}'
with \textit{prefix} matching precisely the argument.
Processing is then passed on to the file
`{\textit{dest}\hspace{0.2em}\textit{suffix}}'.
Surely, the same effect is achieved by
directly specifying the
argument `{\textit{dest}\hspace{0.2em}\textit{suffix}}'
in the first form.
However, that requires to set up a different file
for each child. With the alternative form of the command
all these files can have exactly the same content
which simplifies setting them up and maintaining them.

For example, the following file |draft.tex|
with a compilation flag |\version| as described in \secref{sec:flags}
compiles the main document as a draft:
%
\begin{center}
\begin{tabular}{l}
|\def\version{draft}|\\
|\input{childdoc.def}|\\
|\childdocforward{|\textit{main}|}|
\end{tabular}
\end{center}
%
Likewise, the following files |final|\textit{nn}|.tex|
compile the final version of the child document
|child|\textit{nn}|.tex|:
%
\begin{center}
\begin{tabular}{l}
|\def\version{final}|\\
|\input{childdoc.def}|\\
|\childdocforwardprefix{final}{child}|
\end{tabular}
\end{center}
%

Note that when several versions of a main file and/or of each child file
are to be generated, it may be convenient to set up a |Makefile| or
shell script to automatise the process.

%%%%%%%%%%%%%%%%%%%%%%%%%%%%%%%%%%%%%%%%%%%%%%%%%%%%%%%%%%%%%%%%%%%%%%%%%%%%%%%%
\subsection{Command Line Processing}
\label{sec:commandline}

The effect of redirection files can also be achieved by invoking
the \LaTeX{} compiler with a more elaborate command line.
Most conveniently this should be done as part
of a shell script or a |Makefile|.

When using \textsf{childdoc} in the main file, the following
command lines effectively perform a redirection
(note that depending on the shell being used,
backslashes may have to be doubled: `|\|' $\to$ `|\\|'):
%
\begin{center}
|... -jobname "|\textit{target}|" |\\|"|[\textit{flags}]%
|\input{childdoc.def}\childdocforward[|\textit{main}|]{|\textit{dest}|}"|
\end{center}
%
Here \textit{target} is the name of the output file,
\textit{main} is the name of the main file
and \textit{dest} is the name of the main or child file to be processed
(all filenames without extensions).
The optional argument \textit{main} can be omitted
if \textit{main} matches \textit{dest}.
Optionally, compilation \textit{flags} can be defined via |\def| commands.
This command line makes the \TeX{} engine believe
it is compiling the file \textit{target}
whose content is specified as the latter parameter.
The provided code then forwards the processing to
\textit{main} or \textit{dest} as described in \secref{sec:forward}.

%%%%%%%%%%%%%%%%%%%%%%%%%%%%%%%%%%%%%%%%%%%%%%%%%%%%%%%%%%%%%%%%%%%%%%%%%%%%%%%%
\subsection{Include by Input}
\label{sec:input}

Including child documents by |\include| has some restrictions by design.
Most notably, the content of a child document always occupies
its own set of pages; pages cannot be shared between child documents.
Usually, this behaviour makes perfect sense
because each child document contain an essential part of the document.
However, in some situations it may be desirable to compose
a document from a collection of parts
without having mandatory page breaks between then.
For this case, the package
provides a mechanism to include parts
by |\input| which can also be processed individually.
However, by construction this mechanism
requires manual handling of the content to be output.

%%%%%%%%%%%%%%%%%%%%%%%%%%%%%%%%%%%%%%%%
\DescribeMacro{\ifchilddocmanual}
The main file should be prepared as usual, see \secref{sec:include}.
However, the document body must make a distinction
between processing of an individual part and of the main document, e.g.:
%
\begin{center}
\begin{tabular}{l}
|\ifchilddocmanual|\\
|\input{\childdocname}|\\
|\||else|\\
\textit{document body with }|\input{|\textit{part}|}|\\
|\||fi|
\end{tabular}
\end{center}
%
The conditional |\ifchilddocmanual| is true whenever
a part to be included by |\input| is being compiled,
and the name of the part is stored in |\childdocname|.

%%%%%%%%%%%%%%%%%%%%%%%%%%%%%%%%%%%%%%%%
\DescribeMacro{\childdocby}
Each part to be included by |\input| should start with:
%
\begin{center}
\begin{tabular}{l}
|\input{childdoc.def}|\\
|\childdocby{|\textit{main}|}|\\
\end{tabular}
\end{center}
%
The directive |\childdocby| is similar to |\childdocof|
described in \secref{sec:include},
but the subsequent selection of content must be done manually.
To that end, both |\ifchilddoc| and |\ifchilddocmanual|
will be true upon processing of a part,
and the name of the part is stored in |\childdocname|.
Note that |\jobname| will be set to the filename of the current part
so that each part receives an individual |.aux| file
that does not interfere with the |.aux| file(s) of the main document.
This behaviour can be altered by the alternative form
|\childdocby[*]{|\textit{main}|}| (with a non-empty optional argument)
which uses the |.aux| file of the main document
by setting |\jobname| to \textit{main}.

%%%%%%%%%%%%%%%%%%%%%%%%%%%%%%%%%%%%%%%%%%%%%%%%%%%%%%%%%%%%%%%%%%%%%%%%%%%%%%%%
\subsection{Driver Development}
\label{sec:driver}

The \textsf{childdoc} mechanism can also be use for the development
of definition files such as \LaTeX{} styles or classes.
This case differs from the above setup with multiple parts
included by |\include| in that no |\includeonly| should be invoked.
This can be achieved by starting the include file
(before |\ProvidesPackage|) with:
%
\begin{center}
\begin{tabular}{l}
|\input{childdoc.def}|\\
|\childdocforward{|\textit{main}|}|\\
\end{tabular}
\end{center}
%
or alternatively with:
%
\begin{center}
\begin{tabular}{l}
|\input{childdoc.def}|\\
|\childdocby{|\textit{main}|}|\\
\end{tabular}
\end{center}
%
Both forms have slightly different effects as described above.
The main file is prepared as usual, see \secref{sec:include}.

%%%%%%%%%%%%%%%%%%%%%%%%%%%%%%%%%%%%%%%%%%%%%%%%%%%%%%%%%%%%%%%%%%%%%%%%%%%%%%%%
\subsection{Legacy Detection}
\label{sec:detection}

The directive |\childdocmain| in the main file can detect
whether the complete document or merely a child is to be compiled
even without using the directive |\childdocof|.
This method is deprecated because it is less robust
and there is no compelling reason to use it;
it is merely provided for backward compatibility
and it may be removed in future versions.

If the detection mechanism is to be used,
it is mandatory to correctly specify
the filename of the main file as the argument of |\childdocmain|:
%
\begin{center}
\begin{tabular}{l}
|\input{childdoc.def}|\\
|\childdocmain{|\textit{main}|}|\\
\end{tabular}
\end{center}
%
If |\jobname| does not match the argument \textit{main} of |\childdocmain|,
it is assumed that |\jobname| points to the child file to be compiled.
When using |\childdocmain| with the main file specified as argument,
it suffices to start a child file
with just |\input{|\textit{main}|}|
without loading of the package and using |\childdocof|.
If instead all processing is done
with the appropriate \textsf{childdoc} directives,
the argument of \textit{main} of |\childdocmain| can be empty.

An alternative version of the command line processing described
in \secref{sec:commandline} using the detection mechanism reads:
%
\begin{center}
|... -jobname "|\textit{target}|" "|[\textit{flags}]%
[|\def\jobname{|\textit{dest}|}|]|\input{|\textit{main}|}"|
\end{center}

%%%%%%%%%%%%%%%%%%%%%%%%%%%%%%%%%%%%%%%%%%%%%%%%%%%%%%%%%%%%%%%%%%%%%%%%%%%%%%%%
\subsection{Manual Code}
\label{sec:manual}

In case one cannot be certain whether the definitions file |childdoc.def|
is installed on the target \TeX{} distribution
and one prefers not to ship it,
it is conceivable to paste a few relevant commands into the sources.

To that end, drop all statements |\input{childdoc.def}|
and perform the replacements as outlined below.
Instead of |\childdocmain{|\textit{main}|}| add the following code
to the top of the main file:
%
\begin{center}
\begin{tabular}{l}
|\||ifdefined\childdocname\endinput\||fi\newif\ifchilddoc|\\
|\edef\childdocname{\scantokens\expandafter{\jobname\noexpand}}|\\
|\def\childdocmain{|\textit{main}|}\||ifx\childdocmain\childdocname\||else|\\
|\childdoctrue\includeonly{\childdocname}\let\jobname\childdocmain\||fi|\\
\end{tabular}
\end{center}
%
Instead of |\childdocof{|\textit{main}|}| just include the main file
at the top of each child file:
%
\begin{center}
|\input{|\textit{main}|}|
\end{center}
%
A simple redirection |\childdocforward{|\textit{dest}|}| is achieved by:
%
\begin{center}
|\def\jobname{|\textit{dest}|}\input{\jobname}|
\end{center}
%
The redirection with prefix
|\childdocforwardprefix[|\textit{prefix}|]{|\textit{dest}|}|
is accomplished by:
%
\begin{center}
\begin{tabular}{l}
|{\edef\jobname{\scantokens\expandafter{\jobname\noexpand}}|\\
|\def\redirectjob |\textit{prefix}|#1~~~{\gdef\jobname{|\textit{dest}|#1}}|\\
|\expandafter\redirectjob\jobname~~~}\input{\jobname}|
\end{tabular}
\end{center}

In an alternative approach,
child documents can be compiled by a specific command line
without additional code or specific definitions:
%
\begin{center}
|... -jobname "|\textit{target}|" "|[\textit{flags}]%
|\includeonly{|\textit{dest}|}\input{|\textit{main}|}"|
\end{center}
%

%%%%%%%%%%%%%%%%%%%%%%%%%%%%%%%%%%%%%%%%%%%%%%%%%%%%%%%%%%%%%%%%%%%%%%%%%%%%%%%%
%%%%%%%%%%%%%%%%%%%%%%%%%%%%%%%%%%%%%%%%%%%%%%%%%%%%%%%%%%%%%%%%%%%%%%%%%%%%%%%%
\section{Information}

%%%%%%%%%%%%%%%%%%%%%%%%%%%%%%%%%%%%%%%%%%%%%%%%%%%%%%%%%%%%%%%%%%%%%%%%%%%%%%%%
\subsection{Copyright}

Copyright \copyright{} 2017--2018 Niklas Beisert

This work may be distributed and/or modified under the
conditions of the \LaTeX{} Project Public License, either version 1.3
of this license or (at your option) any later version.
The latest version of this license is in
  \url{http://www.latex-project.org/lppl.txt}
and version 1.3 or later is part of all distributions of \LaTeX{}
version 2005/12/01 or later.

This work has the LPPL maintenance status `maintained'.

The Current Maintainer of this work is Niklas Beisert.

This work consists of the files |README.txt|, |childdoc.ins| and |childdoc.dtx|
as well as the derived files |childdoc.def|, |cdocsamp.tex|
with |cdocsch1.tex|, |cdocsch2.tex|, |cdocspt3.tex|, |cdocspt4.tex|,
|cdocsdrf.tex|, |cdocsfn1.tex|, |cdocsfn2.tex|
as well as |childdoc.pdf|.

%%%%%%%%%%%%%%%%%%%%%%%%%%%%%%%%%%%%%%%%%%%%%%%%%%%%%%%%%%%%%%%%%%%%%%%%%%%%%%%%
\subsection{Files and Installation}

The package consists of the files:
%
\begin{center}
\begin{tabular}{ll}
    |README.txt|   & readme file \\
    |childdoc.ins| & installation file \\
    |childdoc.dtx| & source file \\
    |childdoc.def| & definition file \\
    |cdocsamp.tex| & sample main file \\
    |cdocsch1.tex| & sample include file \\
    |cdocsch2.tex| & sample include file \\
    |cdocspt3.tex| & sample part file \\
    |cdocspt4.tex| & sample part file \\
    |cdocsdrf.tex| & sample redirection file \\
    |cdocsfn1.tex| & sample redirection file \\
    |cdocsfn2.tex| & sample redirection file \\
    |childdoc.pdf| & manual
\end{tabular}
\end{center}
%
The distribution consists of the files
|README.txt|, |childdoc.ins| and |childdoc.dtx|.
%
\begin{itemize}
\item
Run (pdf)\LaTeX{} on |childdoc.dtx|
to compile the manual |childdoc.pdf| (this file).
\item
Run \LaTeX{} on |childdoc.ins| to create the definitions file |childdoc.def|
and the sample |cdocsamp.tex| with include files
|cdocsch1.tex|, |cdocsch2.tex|, |cdocspt3.tex|, |cdocspt4.tex|,
|cdocsdrf.tex|, |cdocsfn1.tex|, |cdocsfn2.tex|.
Then copy the file |childdoc.def| to an appropriate directory of your \LaTeX{}
distribution, e.g.\ \textit{texmf-root}|/tex/latex/childdoc|.
\end{itemize}

%%%%%%%%%%%%%%%%%%%%%%%%%%%%%%%%%%%%%%%%%%%%%%%%%%%%%%%%%%%%%%%%%%%%%%%%%%%%%%%%
\subsection{Related CTAN Packages}

There are several other packages which offer a similar functionality:
%
\begin{itemize}
\item
The packages
\href{http://ctan.org/pkg/docmute}{\textsf{docmute}},
\href{http://ctan.org/pkg/includex}{\textsf{includex}} and
\href{http://ctan.org/pkg/standalone}{\textsf{standalone}}
provide commands to include only the document body of
a child file thus allowing both files to be compiled individually.
\item
The packages \href{http://ctan.org/pkg/subdocs}{\textsf{subdocs}}
and \href{http://ctan.org/pkg/subfiles}{\textsf{subfiles}}
provide structures in which the main and child documents can be
encapsulated and allowing them to be compiled individually.
The inclusion mechanism is different from the conventional |\include|.
\item
The package \href{http://ctan.org/pkg/combine}{\textsf{combine}}
is an elaborate solution to combine several documents into one.
\end{itemize}
%
See also the CTAN topic \href{http://ctan.org/topic/subdocs}{\textsf{subdocs}}
for further related packages.
The present package differs from the above solutions in that
a document structure constructed with the conventional |\include| mechanism
just needs two extra commands at the top of every file
such that all constituent files can be compiled individually.

%%%%%%%%%%%%%%%%%%%%%%%%%%%%%%%%%%%%%%%%%%%%%%%%%%%%%%%%%%%%%%%%%%%%%%%%%%%%%%%%
%\subsection{Feature Suggestions}
%
%The following is a list of features which may be useful for future
%versions of this package:
%%
%\begin{itemize}
%\item
%\ldots
%\end{itemize}

%%%%%%%%%%%%%%%%%%%%%%%%%%%%%%%%%%%%%%%%%%%%%%%%%%%%%%%%%%%%%%%%%%%%%%%%%%%%%%%%
\subsection{Revision History}

%%%%%%%%%%%%%%%%%%%%%%%%%%%%%%%%%%%%%%%%
\paragraph{v2.0:} 2018/12/30

\begin{itemize}
\item
immediate forward processing
\item
added |\childdocby| mechanism
\item
manual restructured
\end{itemize}

%%%%%%%%%%%%%%%%%%%%%%%%%%%%%%%%%%%%%%%%
\paragraph{v1.6:} 2018/01/17

\begin{itemize}
\item
application for development of include files
\item
corrections to manual
\end{itemize}

%%%%%%%%%%%%%%%%%%%%%%%%%%%%%%%%%%%%%%%%
\paragraph{v1.5:} 2017/05/21

\begin{itemize}
\item
more complete structuring introduced
\item
|\childdocof| introduced
\item
|\childdoc| renamed to |\childdocmain|
\item
|\childredirect| renamed to |\childdocforward| and |\childdocforwardprefix|
and functionality expanded
\end{itemize}

%%%%%%%%%%%%%%%%%%%%%%%%%%%%%%%%%%%%%%%%
\paragraph{v1.0:} 2017/04/27

\begin{itemize}
\item
manual and install package
\item
first version published on CTAN
\end{itemize}

%%%%%%%%%%%%%%%%%%%%%%%%%%%%%%%%%%%%%%%%
\paragraph{v0.6:} 2017/04/26

\begin{itemize}
\item
redirection mechanism added
\end{itemize}

%%%%%%%%%%%%%%%%%%%%%%%%%%%%%%%%%%%%%%%%
\paragraph{v0.5:} 2017/04/26

\begin{itemize}
\item
functionality in definition file
\end{itemize}


%%%%%%%%%%%%%%%%%%%%%%%%%%%%%%%%%%%%%%%%%%%%%%%%%%%%%%%%%%%%%%%%%%%%%%%%%%%%%%%%
%%%%%%%%%%%%%%%%%%%%%%%%%%%%%%%%%%%%%%%%%%%%%%%%%%%%%%%%%%%%%%%%%%%%%%%%%%%%%%%%
%%%%%%%%%%%%%%%%%%%%%%%%%%%%%%%%%%%%%%%%%%%%%%%%%%%%%%%%%%%%%%%%%%%%%%%%%%%%%%%%
\appendix

\settowidth\MacroIndent{\rmfamily\scriptsize 000\ }

 \DocInput{childdoc.dtx}

\end{document}
%</driver>
% \fi
%
% %%%%%%%%%%%%%%%%%%%%%%%%%%%%%%%%%%%%%%%%%%%%%%%%%%%%%%%%%%%%%%%%%%%%%%%%%%%%%%
% %%%%%%%%%%%%%%%%%%%%%%%%%%%%%%%%%%%%%%%%%%%%%%%%%%%%%%%%%%%%%%%%%%%%%%%%%%%%%%
% \section{Sample}
%\iffalse
%<*samplemain>
%\fi
%
% The following presents a sample document
% with two chapters, two parts, a title page,
% a compile flag as well as three forwarding files to set the flag.
% It consists of eight |.tex| files:
% \begin{center}
% \begin{tabular}{ll}
% |cdocsamp.tex|&main file\\
% |cdocsch1.tex|&include file for chapter 1\\
% |cdocsch2.tex|&include file for chapter 2\\
% |cdocspt3.tex|&include file for part 3\\
% |cdocspt4.tex|&include file for part 4\\
% |cdocsdrf.tex|&forwarding file for main file in draft mode\\
% |cdocsfi1.tex|&forwarding file for final version of chapter 1\\
% |cdocsfi2.tex|&forwarding file for final version of chapter 2\\
% \end{tabular}
% \end{center}
% Each of the eight files can be compiled directly by the \LaTeX{} compiler.
%
% %%%%%%%%%%%%%%%%%%%%%%%%%%%%%%%%%%%%%%
% \paragraph{Main File.}
%
% The main file is called |cdocsamp.tex|.
%
% Load the \textsf{childdoc} definitions and
% declare the filename for the main document:
%    \begin{macrocode}
\input{childdoc.def}
\childdocmain{}
%    \end{macrocode}

% Optional override for |\version| flag:
%    \begin{macrocode}
%%\ifchilddoc\else\providecommand{\version}{draft}\fi
%    \end{macrocode}

% Define the default values for the |\version| flag
% (|final| for the main file and |draft| for childs):
%    \begin{macrocode}
\ifchilddoc
\providecommand{\version}{draft}
\else
\providecommand{\version}{final}
\fi
%    \end{macrocode}

% Load the standard document class:
%    \begin{macrocode}
\documentclass[12pt]{article}
%    \end{macrocode}

% Start the document body:
%    \begin{macrocode}
\begin{document}
%    \end{macrocode}

% Declare a title page.
% Print title, part of document being processed and version flag:
%    \begin{macrocode}
\addtocounter{page}{-1}
\begin{center}
{\LARGE\bfseries{}childdoc example\par}
\vspace{1cm}
\ifchilddoc
\ifchilddocmanual part\else chapter\fi:
`\childdocname' of `\childdocjob'\par
\else
main document: `\childdocjob'\par
\fi
version: \version\par
\end{center}
\newpage
%    \end{macrocode}

% Manually include selected file,
% otherwise process as usual:
%    \begin{macrocode}
\ifchilddocmanual
\section*{part `\childdocname'}
\input{\childdocname}
\else
%    \end{macrocode}

% Include the two chapters:
%    \begin{macrocode}
\include{cdocsch1}
\include{cdocsch2}
%    \end{macrocode}

% Include the two parts unless only chapters should be displayed:
%    \begin{macrocode}
\ifchilddoc\else
\section{part three}
\input{cdocspt3}
\section{part four}
\input{cdocspt4}
\fi
%    \end{macrocode}

% Process as usual until here:
%    \begin{macrocode}
\fi
%    \end{macrocode}

% End of document body:
%    \begin{macrocode}
\end{document}
%    \end{macrocode}
%\iffalse
%</samplemain>
%\fi
%
% %%%%%%%%%%%%%%%%%%%%%%%%%%%%%%%%%%%%%%
% \paragraph{Chapter Include Files.}
%
% The include files are called |cdocsch1.tex| and |cdocsch2.tex|.
%
%\iffalse
%<*samplechap1|samplechap2>
%\fi

% Optional override for |\version| flag:
%    \begin{macrocode}
%%\providecommand{\version}{final}
%    \end{macrocode}

% Include the main document:
%    \begin{macrocode}
\input{childdoc.def}
\childdocof{cdocsamp}
%    \end{macrocode}

%\iffalse
%</samplechap1|samplechap2>
%\fi
%
%\iffalse
%<*samplechap1>
%\fi
% Some text for chapter 1:
%    \begin{macrocode}
\section{one}
some text in chapter one
%    \end{macrocode}

%\iffalse
%</samplechap1>
%\fi
% Some text for chapter 2:
%\iffalse
%<*samplechap2>
%\fi
%    \begin{macrocode}
\section{two}
more text in chapter two
%    \end{macrocode}

%\iffalse
%</samplechap2>
%\fi
%
% %%%%%%%%%%%%%%%%%%%%%%%%%%%%%%%%%%%%%%
% \paragraph{Part Include Files.}
%
% The include files are called |cdocspt3.tex| and |cdocspt4.tex|.
%
%\iffalse
%<*samplepart3|samplepart4>
%\fi

% Optional override for |\version| flag:
%    \begin{macrocode}
%%\providecommand{\version}{final}
%    \end{macrocode}

% Include the main document:
%    \begin{macrocode}
\input{childdoc.def}
\childdocby{cdocsamp}
%    \end{macrocode}

%\iffalse
%</samplepart3|samplepart4>
%\fi
%
%\iffalse
%<*samplepart3>
%\fi
% Some text for part 3:
%    \begin{macrocode}
some text in part three
%    \end{macrocode}

%\iffalse
%</samplepart3>
%\fi
% Some text for part 4:
%\iffalse
%<*samplepart4>
%\fi
%    \begin{macrocode}
more text in part four
%    \end{macrocode}

%\iffalse
%</samplepart4>
%\fi
%
% %%%%%%%%%%%%%%%%%%%%%%%%%%%%%%%%%%%%%%
% \paragraph{Forwarding for a Complete Draft.}
%
% The following forwarding file |cdocsdrf.tex|
% compiles the main document in draft mode:
%\iffalse
%<*sampledraft>
%\fi
%    \begin{macrocode}
\def\version{draft}
\input{childdoc.def}
\childdocforward{cdocsamp}
%    \end{macrocode}

%\iffalse
%</sampledraft>
%\fi
%
% %%%%%%%%%%%%%%%%%%%%%%%%%%%%%%%%%%%%%%
% \paragraph{Forwarding for Final Version of the Chapters.}
%
% The following forwarding files |cdocsfn1.tex| and |cdocsfn2.tex|
% (with identical content)
% compile the final versions of the child documents
% |cdocsch1.tex| and |cdocsch2.tex|, respectively:
%\iffalse
%<*samplefinal>
%\fi
%    \begin{macrocode}
\def\version{final}
\input{childdoc.def}
\childdocforwardprefix[cdocsamp]{cdocsfn}{cdocsch}
%    \end{macrocode}

%\iffalse
%</samplefinal>
%\fi
%
% %%%%%%%%%%%%%%%%%%%%%%%%%%%%%%%%%%%%%%
% \paragraph{Command Line Processing.}
%
% The following three command lines generate the output files
% |cdocscld|, |cdocscl1| and |cdocscl2|
% which should be identical to
% |cdocsdrf|, |cdocsch1| and |cdocsfn2|, respectively:
% \begin{center}
% \begin{tabular}{l}
% |latex -jobname cdocscld \|\\
% |  "\def\version{draft}\input{childdoc.def}\childdocforward{cdocsamp}"|\\
% |latex -jobname cdocscl1 \|\\
% |  "\input{childdoc.def}\childdocforward[cdocsamp]{cdocsch1}"|\\
% |latex -jobname cdocscl2 \|\\
% |  "\def\version{final}\input{childdoc.def}\childdocforward{cdocsch2}"|
% \end{tabular}
% \end{center}
% Note that the trailing backslash on each first line
% merely continues the input to the second line
% (for convenient cut ant paste).
% Furthermore, the command |latex| can be replaced by any
% of its alternative versions such as |pdflatex|.
%
% %%%%%%%%%%%%%%%%%%%%%%%%%%%%%%%%%%%%%%%%%%%%%%%%%%%%%%%%%%%%%%%%%%%%%%%%%%%%%%
% %%%%%%%%%%%%%%%%%%%%%%%%%%%%%%%%%%%%%%%%%%%%%%%%%%%%%%%%%%%%%%%%%%%%%%%%%%%%%%
% \section{Implementation}
%\iffalse
%<*package>
%\fi
%
% This section describes the definitions file |childdoc.def|.

% The definitions cannot be loaded using |\usepackage| or |\RequirePackage|
% which has a mechanism to prevent loading a style file more than once.
% When loading the definitions by means of |\input|
% multiple instances have to be prevented manually:
%\iffalse
%This code needs to be before the `\ProvidesFile' directive
%which is defined at the beginning of this file.
%Therefore it is also placed there and commented out here.
%</package>
%<*discard>
%\fi
%    \begin{macrocode}
\ifdefined\childdocmain\endinput\fi
%    \end{macrocode}
%\iffalse
%</discard>
%<*package>
%\fi
%
% \macro{\ifchilddoc}
% \macro{\ifchilddocmanual}
% The conditional |\ifchilddoc| tells whether a
% child (true) or main (false) document is being compiled.
% The conditional |\ifchilddocmanual| tells whether
% the |\includeonly| mechanism is used (false) or
% the selection of child files must be performed manually (true).
% The definitions initialise to false:
%    \begin{macrocode}
\newif\ifchilddoc
\newif\ifchilddocmanual
%    \end{macrocode}

% \macro{\childdocname}
% \macro{\childdocjob}
% The macro |\childdocname| stores the name of the main document
% to be compiled. The macro |\childdocjob| stores the name of
% the document on which the \LaTeX{} compiler was originally invoked.
% The content of |\jobname| cannot be compared
% to filenames specified in the source due to different catcodes.
% The following code rescans |\jobname|, stores the result
% in |\childdocname| and saves a copy in |\childdocjob|:
%    \begin{macrocode}
\edef\childdocname{\scantokens\expandafter{\jobname\noexpand}}
\let\childdocjob\childdocname
%    \end{macrocode}

% \macro{\childdocdisable}
% The macro |\childdocdisable| prevents the main file
% from being processed more than once.
% At this stage, the main document command |\childdocmain|
% is assumed to be called once again where it should do nothing.
% Any subsequent call to it should prevent
% a secondary processing of the main document
% It overwrites the forwarding commands
% |\childdocof| and |\childdocforward|
% with empty macros to prevent further inclusions of the main document:
%    \begin{macrocode}
\newcommand{\childdocdisable}
{
  \renewcommand{\childdocmain}[1]{\renewcommand{\childdocmain}[1]{\endinput}}
  \renewcommand{\childdocof}[1]{}
  \renewcommand{\childdocby}[2][]{}
  \renewcommand{\childdocforward}[2][]{}
  \renewcommand{\childdocdisable}{}
}
%    \end{macrocode}

% \macro{\childdocmain}
% The macro |\childdocmain| is to be called at the top of the main file
% with nothing or the main filename (without extension) as argument.
% First, it breaks loops.
% If the argument is not empty and does not match |\childdocname|
% (which is set by the first inclusion of |childdoc.def|),
% |\ifchilddoc| is set to true, |\includeonly| is applied to the child file
% and |\jobname| is set to the main file
% (for proper handling of |.aux| files):
%    \begin{macrocode}
\newcommand{\childdocmain}[1]
{
  \childdocdisable\childdocmain{}
  \if?#1?\else
    \begingroup
      \def\childdoctmp{#1}
      \ifx\childdoctmp\childdocname
        \def\childdoctmp{}
      \else
        \def\childdoctmp
        {
          \childdoctrue
          \includeonly{\childdocname}
          \def\childdocjob{#1}
          \def\jobname{#1}
        }
      \fi
      \expandafter
    \endgroup
    \childdoctmp
  \fi
}
%    \end{macrocode}

% \macro{\childdocof}
% The command |\childdocof| redirects
% compilation to the main file |#1|.
%    \begin{macrocode}
\newcommand{\childdocof}[1]
{
  \childdocdisable
  \childdoctrue
  \includeonly{\childdocname}
  \def\jobname{#1}
  \def\childdocjob{#1}
  \input{#1}
}
%    \end{macrocode}

% \macro{\childdocby}
% The command |\childdocby| ....
%    \begin{macrocode}
\newcommand{\childdocby}[2][]
{
  \childdocdisable
  \childdoctrue
  \childdocmanualtrue
  \if?#1?\else
    \def\jobname{#2}
  \fi
  \def\childdocjob{#2}
  \input{#2}
  \endinput
}
%    \end{macrocode}

% \macro{\childdocforward}
% The command |\childdocforward| redirects
% compilation to the main file or
% (if the optional argument is given) a child file.
% Parameters are set as if the main file
% or a child file starting with |\childdocof| was compiled.
% Then compilation is handed over to the main file:
%    \begin{macrocode}
\newcommand{\childdocforward}[2][]
{
  \begingroup
    \if?#1?
      \def\childdoctmp
      {
        \def\childdocname{#2}
        \def\childdocjob{#2}
        \def\jobname{#2}
        \input{#2}
        \endinput
      }
    \else
      \def\childdoctmp
      {
        \childdocdisable
        \def\childdocname{#2}
        \childdoctrue
        \includeonly{#2}
        \def\childdocjob{#1}
        \def\jobname{#1}
        \input{#1}
        \endinput
      }
    \fi
    \expandafter
  \endgroup
  \childdoctmp
}
%    \end{macrocode}

% \macro{\childdocforwardprefix}
% The command |\childdocforwardprefix| redirects
% compilation to the main or a child file by means of a pattern.
% The prefix |#1| in the current filename is replaced by |#2|
% and the suffix of the current filename is kept
% (it is assumed that the filename does not contain the substring `|~~~|'
% which is used as a delimiter).
% Compilation is handed over to the new file by |\childdocforward|:
%    \begin{macrocode}
\newcommand{\childdocforwardprefix}[3][]
{
  \begingroup
    \def\childdocextract #2##1~~~{\def\childdoctmp{\childdocforward[#1]{#3##1}}}
    \expandafter\childdocextract\childdocname~~~
    \expandafter
  \endgroup
  \childdoctmp
}
%    \end{macrocode}

% \macro{\childdoc}
% The deprecated macro |\childdoc| is a legacy version of |\childdocmain|:
%    \begin{macrocode}
\newcommand{\childdoc}{\childdocmain}
%    \end{macrocode}

% \macro{\childdocredirect}
% The deprecated macro |\childdocredirect| is a legacy version
% of |\childdocforward| and |\childdocforwardprefix|:
%    \begin{macrocode}
\newcommand{\childdocredirect}[2][]
{
  \begingroup
    \if?#1?
      \def\childdoctmp{\childdocforward{#2}}
    \else
      \def\childdoctmp{\childdocforwardprefix{#1}{#2}}
    \fi
    \expandafter
  \endgroup
  \childdoctmp
}
%    \end{macrocode}

%\iffalse
%</package>
%\fi
%
\endinput

\childdocof{cdocsamp}
%    \end{macrocode}

%\iffalse
%</samplechap1|samplechap2>
%\fi
%
%\iffalse
%<*samplechap1>
%\fi
% Some text for chapter 1:
%    \begin{macrocode}
\section{one}
some text in chapter one
%    \end{macrocode}

%\iffalse
%</samplechap1>
%\fi
% Some text for chapter 2:
%\iffalse
%<*samplechap2>
%\fi
%    \begin{macrocode}
\section{two}
more text in chapter two
%    \end{macrocode}

%\iffalse
%</samplechap2>
%\fi
%
% %%%%%%%%%%%%%%%%%%%%%%%%%%%%%%%%%%%%%%
% \paragraph{Part Include Files.}
%
% The include files are called |cdocspt3.tex| and |cdocspt4.tex|.
%
%\iffalse
%<*samplepart3|samplepart4>
%\fi

% Optional override for |\version| flag:
%    \begin{macrocode}
%%\providecommand{\version}{final}
%    \end{macrocode}

% Include the main document:
%    \begin{macrocode}
% \iffalse
%
% childdoc.dtx Copyright (C) 2017-2018 Niklas Beisert
%
% This work may be distributed and/or modified under the
% conditions of the LaTeX Project Public License, either version 1.3
% of this license or (at your option) any later version.
% The latest version of this license is in
%   http://www.latex-project.org/lppl.txt
% and version 1.3 or later is part of all distributions of LaTeX
% version 2005/12/01 or later.
%
% This work has the LPPL maintenance status `maintained'.
%
% The Current Maintainer of this work is Niklas Beisert.
%
% This work consists of the files childdoc.dtx and childdoc.ins
% and the derived files childdoc.def and cdocsamp.tex with
% cdocsch1.tex, cdocsch2.tex, cdocsdrf.tex, cdocsfn1.tex, cdocsfn2.tex.
%
%<package>\ifdefined\childdocmain\endinput\fi
%<package>\ProvidesFile{childdoc.def}[2018/12/30 v2.0 child document driver]
%<samplemain>\ProvidesFile{cdocsamp.tex}[2018/12/30 v2.0 sample for childdoc]
%<*driver>
%\ProvidesFile{childdoc.drv}[2018/12/30 v2.0 childdoc reference manual file]
\PassOptionsToClass{10pt,a4paper}{article}
\documentclass{ltxdoc}

\usepackage[margin=35mm]{geometry}
\usepackage{hyperref}
\usepackage{hyperxmp}
\usepackage[usenames]{color}

\hypersetup{colorlinks=true}
\hypersetup{pdfstartview=FitH}
\hypersetup{pdfpagemode=UseNone}
\hypersetup{pdfsource={}}
\hypersetup{pdflang={en-UK}}
\hypersetup{pdfcopyright={Copyright 2017-2018 Niklas Beisert.
  This work may be distributed and/or modified under the
  conditions of the LaTeX Project Public License, either version 1.3
  of this license or (at your option) any later version.}}
\hypersetup{pdflicenseurl={http://www.latex-project.org/lppl.txt}}
\hypersetup{pdfcontactaddress={ETH Zurich, ITP, HIT K,
  Wolfgang-Pauli-Strasse 27}}
\hypersetup{pdfcontactpostcode={8093}}
\hypersetup{pdfcontactcity={Zurich}}
\hypersetup{pdfcontactcountry={Switzerland}}
\hypersetup{pdfcontactemail={nbeisert@itp.phys.ethz.ch}}
\hypersetup{pdfcontacturl={http://people.phys.ethz.ch/\xmptilde nbeisert/}}

\newcommand{\secref}[1]{\hyperref[#1]{section \ref*{#1}}}

\parskip1ex
\parindent0pt
\let\olditemize\itemize
\def\itemize{\olditemize\parskip0pt}

\begin{document}

\title{The \textsf{childdoc} Package}
\hypersetup{pdftitle={The childdoc Package}}
\author{Niklas Beisert\\[2ex]
  Institut f\"ur Theoretische Physik\\
  Eidgen\"ossische Technische Hochschule Z\"urich\\
  Wolfgang-Pauli-Strasse 27, 8093 Z\"urich, Switzerland\\[1ex]
  \href{mailto:nbeisert@itp.phys.ethz.ch}
  {\texttt{nbeisert@itp.phys.ethz.ch}}}
\hypersetup{pdfauthor={Niklas Beisert}}
\hypersetup{pdfsubject={Manual for the LaTeX2e Package childdoc}}
\date{30 December 2018, \textsf{v2.0}}
\maketitle

\begin{abstract}\noindent
\textsf{childdoc} is a \LaTeXe{} package
that enables the direct compilation
of document sections included by |\include|
to individual files.
\end{abstract}

\begingroup
\parskip0ex
\tableofcontents
\endgroup

%%%%%%%%%%%%%%%%%%%%%%%%%%%%%%%%%%%%%%%%%%%%%%%%%%%%%%%%%%%%%%%%%%%%%%%%%%%%%%%%
%%%%%%%%%%%%%%%%%%%%%%%%%%%%%%%%%%%%%%%%%%%%%%%%%%%%%%%%%%%%%%%%%%%%%%%%%%%%%%%%
\section{Introduction}

\LaTeX{} provides a mechanism to structure a large document (such as a book)
into a main file and several child files (containing the chapters)
using the |\include| command.
This mechanism is beneficial for documents
which span hundreds of pages in order to
make the source file(s) more manageable.
Moreover, compilation can be restricted to
selected child files by means of the |\includeonly| command.
The latter feature can be used to reduce the compilation time while editing
(this was significantly more useful in the earlier days of \LaTeX{})
or to generate a smaller document which is easier to navigate.
Another application of |\includeonly| is to generate
documents consisting of selected parts of the complete document.

However, there are a few drawbacks of the plain |\include| mechanism:
\begin{itemize}
\item
The child files cannot be compiled on their own,
they can only be compiled via the main file.
A naive editing environment
(such as a text editor with an option
to have the current file processed by \LaTeX)
may require one to switch to the main file before compiling;
attempting to compile the child file produces errors.
\item
The main file must be modified (each time)
to adjust the |\includeonly| command
to the present needs. This easily leaves the main file in a messy state.
\item
The generated document will always carry the filename
of the main document. This is inconvenient if
several child files are to be compiled and
to be kept for distribution.
\end{itemize}

The present package provides a simple interface
to make child files individually compilable by \LaTeX{}.
Compiling a child file then has the same effect as compiling
the main file with an |\includeonly| command
to select the appropriate child.
Moreover the generated document will carry the name of the child
rather than the main file.
This resolves all three above issues.

This feature is meant to make the editing of books,
thesis documents and lecture notes somewhat more convenient.
However, the package can also be used efficiently for
composing a series of documents (such as exercise sheets)
which are typically distributed individually.
It then assists the author in generating the individual documents
(potentially in different versions)
as well as a document containing the collected series.
Another application is in developing style files
or other kinds of included material
where compilation of the style file could redirect
to a sample or test file.

%%%%%%%%%%%%%%%%%%%%%%%%%%%%%%%%%%%%%%%%%%%%%%%%%%%%%%%%%%%%%%%%%%%%%%%%%%%%%%%%
%%%%%%%%%%%%%%%%%%%%%%%%%%%%%%%%%%%%%%%%%%%%%%%%%%%%%%%%%%%%%%%%%%%%%%%%%%%%%%%%
\section{Usage}

First of all, the package \textsf{childdoc} is \emph{not} a standard
\LaTeXe{} |.sty| style file! Therefore it needs to be invoked in
a non-standard way.

%%%%%%%%%%%%%%%%%%%%%%%%%%%%%%%%%%%%%%%%%%%%%%%%%%%%%%%%%%%%%%%%%%%%%%%%%%%%%%%%
\subsection{Included Files}
\label{sec:include}

%%%%%%%%%%%%%%%%%%%%%%%%%%%%%%%%%%%%%%%%
\DescribeMacro{\childdocmain}
To use the package, add the commands
\begin{center}
\begin{tabular}{l}
|\input{childdoc.def}|\\
|\childdocmain{}|\\
\end{tabular}
\end{center}
at the very top of the main \LaTeX{} file,
in particular \emph{before} the |\documentclass| statement!
The argument of |\childdocmain| should be left empty
(but it must be present).

%%%%%%%%%%%%%%%%%%%%%%%%%%%%%%%%%%%%%%%%
\DescribeMacro{\childdocof}
Furthermore, add the commands
\begin{center}
\begin{tabular}{l}
|\input{childdoc.def}|\\
|\childdocof{|\textit{main}|}|\\
\end{tabular}
\end{center}
at the top of every child file \textit{child}
which is included by |\include{|\textit{child}|}|
from within the main file
(or at least for those files to be compiled individually).
The argument \textit{main} must be the filename of the main file.

There are a couple of
considerations in setting up the main and child documents:

%%%%%%%%%%%%%%%%%%%%%%%%%%%%%%%%%%%%%%%%
\paragraph{Restrictions.}

Please note the following restrictions:
\begin{itemize}
\item
|\childdocmain| must be called with one argument \textit{main}
to ensure compatibility with earlier version of the package.
It must either be empty (|\childdocmain{}|)
or precisely match the filename of the main file in which it is specified.
See \secref{sec:detection} for further information.
\item
The filename \textit{main} must be specified without the |.tex| extension.
\item
The filename \textit{main} is case sensitive
(even in case-insensitive file systems)
due to internal string comparison.
\item
The argument \textit{main} should be fully expanded, it cannot be a macro.
\item
Subdirectories and special characters should be avoided in filenames.
\item
The command |\childdocmain{|\textit{main}|}| must be followed by a whitespace.
It should not be followed immediately by another command
or by a comment mark `|%|'.
This is because the \TeX{} parser reads the token immediately following
the argument of |\childdocmain| and puts it
at the beginning of every child section;
however, a white\-space is ignored.
\end{itemize}

%%%%%%%%%%%%%%%%%%%%%%%%%%%%%%%%%%%%%%%%
\paragraph{Content of Main File.}

It is advisable to place all content in the child files included by |\include|.
Any output contained in the main file will appear in all child documents
unless suppressed manually;
it cannot be suppressed automatically by the |\includeonly| directive
and thus should normally be avoided.
A method to include some content in the main file
by means of conditional processing is described in \secref{sec:conditional}.

%%%%%%%%%%%%%%%%%%%%%%%%%%%%%%%%%%%%%%%%
\paragraph{Page Numbering.}

When only a part of the document is compiled,
the appropriate numbering of pages
(as well as other status parameters)
is determined from the |.aux| files.
The latter contain information from previous passes.
However this information needs to propagate through
all intermediate child documents.
Therefore the page numbering in child documents may well
be inconsistent until the complete document is compiled at least once.

A useful (if unconventional) way to always ensure a consistent
page numbering is to restart the numbering in each child document
and denote the pages by `\textit{child}|.|\textit{page}'
where \textit{child} represents the chapter/section number of the child file.
This can be achieved by the command
|\numberwithin{page}{|\textit{child}|}|
of the \textsf{amsmath} package
where \textit{child} can be |chapter| or |section|
depending on the chosen structuring.
Alternatively, one can modify the macro |\thepage| appropriately
and reset the counter |page| at the start of each child file.

%%%%%%%%%%%%%%%%%%%%%%%%%%%%%%%%%%%%%%%%%%%%%%%%%%%%%%%%%%%%%%%%%%%%%%%%%%%%%%%%
\subsection{Conditional Processing}
\label{sec:conditional}

The package provides a mechanism to compile different versions
of a document. To customise the versions further some conditional processing
can come in handy to distinguish which version is being compiled.
The package provides two macros to describe the compilation context:

%%%%%%%%%%%%%%%%%%%%%%%%%%%%%%%%%%%%%%%%
\DescribeMacro{\ifchilddoc}
The conditional |\ifchilddoc| distinguishes between the compilation of
child documents and the main document:
%
\begin{center}
|\ifchilddoc |\textit{child-code}| |[|\||else |\textit{main-code}]| \||fi|
\end{center}

%%%%%%%%%%%%%%%%%%%%%%%%%%%%%%%%%%%%%%%%
\DescribeMacro{\childdocname}
\DescribeMacro{\childdocjob}
The macro |\childdocname| contains the filename (without extension)
of the main or child file being processed.
Note that |\childdocjob| will always contain the name of the main file.

%%%%%%%%%%%%%%%%%%%%%%%%%%%%%%%%%%%%%%%%
\paragraph{Title Page.}

Conditional processing can be used to include a title or banner page
in the main document when proper precautions are taken.
Importantly, the code in the main file should ensure that the page counter
(as well as other status parameters which are stored in the |.aux| files)
takes the same value after the conditional processing.
Otherwise the page numbers may take divergent values
depending on which part is compiled.

For example, a title page could be declared by:
%
\begin{center}
\begin{tabular}{l}
|\ifchilddoc\||else|\\
|\addtocounter{page}{-1}|\\
\textit{code for title page}\\
|\newpage|\\
|\||fi|
\end{tabular}
\end{center}
%
A banner page for the child documents can be generated by:
%
\begin{center}
\begin{tabular}{l}
|\ifchilddoc|\\
|\addtocounter{page}{-1}|\\
\textit{code for banner page}\\
|\newpage|\\
|\||fi|
\end{tabular}
\end{center}
%
Here one could write a message such as:
\begin{center}
|This is the part \childdocname{} of \childdocjob{}.|
\end{center}

%%%%%%%%%%%%%%%%%%%%%%%%%%%%%%%%%%%%%%%%%%%%%%%%%%%%%%%%%%%%%%%%%%%%%%%%%%%%%%%%
\subsection{Flags}
\label{sec:flags}

The package makes it easy to generate different versions
of the main or child documents.
To this end compilation flags can be defined
and assigned different default values.
They will be particularly useful in conjunction
with the forwarding mechanism described in \secref{sec:forward}.

For example, it may be useful to have a flag |\version|
which can be set to |draft| or |final|.
The document source will contain some conditional code
depending on the value of |\version|.
Suppose further, the flag should default to |final| for the main file
and to |draft| for child files
which is a natural assignment for editing the document.
This is achieved by placing the following code
in the preamble of the main document
(below the |\childdocmain| directive):
%
\begin{center}
\begin{tabular}{l}
|\ifchilddoc|\\
|\providecommand{\version}{draft}|\\
|\||else|\\
|\providecommand{\version}{final}|\\
|\||fi|
\end{tabular}
\end{center}
%
The definition by |\providecommand| makes sure
that previous definitions are not overwritten.
Further statements |\providecommand{\version}{...}|
can thus be added before the above code to override it.

For the main file, one might add a line
(between |\childdocmain| and the above block)
%
\begin{center}
|%\ifchilddoc\||else\providecommand{\version}{draft}\||fi|
\end{center}
%
which can be uncommented to produce a draft version.
Likewise one can add a line to the very top of a child file
(above the |\childdocof{|\textit{main}|}| directive)
%
\begin{center}
|%\providecommand{\version}{final}|
\end{center}
%
which can be uncommented to produce the final version of this child document.

%%%%%%%%%%%%%%%%%%%%%%%%%%%%%%%%%%%%%%%%%%%%%%%%%%%%%%%%%%%%%%%%%%%%%%%%%%%%%%%%
\subsection{Forwarding}
\label{sec:forward}

Different versions of the main or child documents
using compilation flags as described in \secref{sec:flags}
can be (permanently) stored in different files
for convenient compilation, viewing and distribution.
To this end, the package defines a command
to pass on compilation to a different file:

%%%%%%%%%%%%%%%%%%%%%%%%%%%%%%%%%%%%%%%%
\DescribeMacro{\childdocforward}
The command |\childdocforward| redirects processing to
another source file:
%
\begin{center}
\begin{tabular}{l}
|\input{childdoc.def}|\\
|\childdocforward[|\textit{main}|]{|\textit{dest}|}|\\
\end{tabular}
\end{center}
%
The argument \textit{dest} is the destination file
(without extension).
It should be the main file or one of the child files.
Note that further \textsf{childdoc} directives
such as |\childdocof| and |\childdocforward|
in the indicated file will be processed in this form.
The optional argument \textit{main}
passes on directly to the main file \textit{main}
while pretending to compile the child \textit{dest}.
This form behaves as if \textit{dest}
issues |\childdocof{|\textit{main}|}| right away,
and no further \textsf{childdoc} directives will be processed.

%%%%%%%%%%%%%%%%%%%%%%%%%%%%%%%%%%%%%%%%
\DescribeMacro{\...prefix}
In the alternative form |\childdocforwardprefix|,
%
\begin{center}
\begin{tabular}{l}
|\input{childdoc.def}|\\
|\childdocforwardprefix[|\textit{main}|]{|\textit{prefix}|}{|\textit{dest}|}|
\end{tabular}
\end{center}
%
the destination file is determined by a pattern
depending on the current file:
To make this work, the current file must be called
`{\textit{prefix}\hspace{0.2em}\textit{suffix}}'
with \textit{prefix} matching precisely the argument.
Processing is then passed on to the file
`{\textit{dest}\hspace{0.2em}\textit{suffix}}'.
Surely, the same effect is achieved by
directly specifying the
argument `{\textit{dest}\hspace{0.2em}\textit{suffix}}'
in the first form.
However, that requires to set up a different file
for each child. With the alternative form of the command
all these files can have exactly the same content
which simplifies setting them up and maintaining them.

For example, the following file |draft.tex|
with a compilation flag |\version| as described in \secref{sec:flags}
compiles the main document as a draft:
%
\begin{center}
\begin{tabular}{l}
|\def\version{draft}|\\
|\input{childdoc.def}|\\
|\childdocforward{|\textit{main}|}|
\end{tabular}
\end{center}
%
Likewise, the following files |final|\textit{nn}|.tex|
compile the final version of the child document
|child|\textit{nn}|.tex|:
%
\begin{center}
\begin{tabular}{l}
|\def\version{final}|\\
|\input{childdoc.def}|\\
|\childdocforwardprefix{final}{child}|
\end{tabular}
\end{center}
%

Note that when several versions of a main file and/or of each child file
are to be generated, it may be convenient to set up a |Makefile| or
shell script to automatise the process.

%%%%%%%%%%%%%%%%%%%%%%%%%%%%%%%%%%%%%%%%%%%%%%%%%%%%%%%%%%%%%%%%%%%%%%%%%%%%%%%%
\subsection{Command Line Processing}
\label{sec:commandline}

The effect of redirection files can also be achieved by invoking
the \LaTeX{} compiler with a more elaborate command line.
Most conveniently this should be done as part
of a shell script or a |Makefile|.

When using \textsf{childdoc} in the main file, the following
command lines effectively perform a redirection
(note that depending on the shell being used,
backslashes may have to be doubled: `|\|' $\to$ `|\\|'):
%
\begin{center}
|... -jobname "|\textit{target}|" |\\|"|[\textit{flags}]%
|\input{childdoc.def}\childdocforward[|\textit{main}|]{|\textit{dest}|}"|
\end{center}
%
Here \textit{target} is the name of the output file,
\textit{main} is the name of the main file
and \textit{dest} is the name of the main or child file to be processed
(all filenames without extensions).
The optional argument \textit{main} can be omitted
if \textit{main} matches \textit{dest}.
Optionally, compilation \textit{flags} can be defined via |\def| commands.
This command line makes the \TeX{} engine believe
it is compiling the file \textit{target}
whose content is specified as the latter parameter.
The provided code then forwards the processing to
\textit{main} or \textit{dest} as described in \secref{sec:forward}.

%%%%%%%%%%%%%%%%%%%%%%%%%%%%%%%%%%%%%%%%%%%%%%%%%%%%%%%%%%%%%%%%%%%%%%%%%%%%%%%%
\subsection{Include by Input}
\label{sec:input}

Including child documents by |\include| has some restrictions by design.
Most notably, the content of a child document always occupies
its own set of pages; pages cannot be shared between child documents.
Usually, this behaviour makes perfect sense
because each child document contain an essential part of the document.
However, in some situations it may be desirable to compose
a document from a collection of parts
without having mandatory page breaks between then.
For this case, the package
provides a mechanism to include parts
by |\input| which can also be processed individually.
However, by construction this mechanism
requires manual handling of the content to be output.

%%%%%%%%%%%%%%%%%%%%%%%%%%%%%%%%%%%%%%%%
\DescribeMacro{\ifchilddocmanual}
The main file should be prepared as usual, see \secref{sec:include}.
However, the document body must make a distinction
between processing of an individual part and of the main document, e.g.:
%
\begin{center}
\begin{tabular}{l}
|\ifchilddocmanual|\\
|\input{\childdocname}|\\
|\||else|\\
\textit{document body with }|\input{|\textit{part}|}|\\
|\||fi|
\end{tabular}
\end{center}
%
The conditional |\ifchilddocmanual| is true whenever
a part to be included by |\input| is being compiled,
and the name of the part is stored in |\childdocname|.

%%%%%%%%%%%%%%%%%%%%%%%%%%%%%%%%%%%%%%%%
\DescribeMacro{\childdocby}
Each part to be included by |\input| should start with:
%
\begin{center}
\begin{tabular}{l}
|\input{childdoc.def}|\\
|\childdocby{|\textit{main}|}|\\
\end{tabular}
\end{center}
%
The directive |\childdocby| is similar to |\childdocof|
described in \secref{sec:include},
but the subsequent selection of content must be done manually.
To that end, both |\ifchilddoc| and |\ifchilddocmanual|
will be true upon processing of a part,
and the name of the part is stored in |\childdocname|.
Note that |\jobname| will be set to the filename of the current part
so that each part receives an individual |.aux| file
that does not interfere with the |.aux| file(s) of the main document.
This behaviour can be altered by the alternative form
|\childdocby[*]{|\textit{main}|}| (with a non-empty optional argument)
which uses the |.aux| file of the main document
by setting |\jobname| to \textit{main}.

%%%%%%%%%%%%%%%%%%%%%%%%%%%%%%%%%%%%%%%%%%%%%%%%%%%%%%%%%%%%%%%%%%%%%%%%%%%%%%%%
\subsection{Driver Development}
\label{sec:driver}

The \textsf{childdoc} mechanism can also be use for the development
of definition files such as \LaTeX{} styles or classes.
This case differs from the above setup with multiple parts
included by |\include| in that no |\includeonly| should be invoked.
This can be achieved by starting the include file
(before |\ProvidesPackage|) with:
%
\begin{center}
\begin{tabular}{l}
|\input{childdoc.def}|\\
|\childdocforward{|\textit{main}|}|\\
\end{tabular}
\end{center}
%
or alternatively with:
%
\begin{center}
\begin{tabular}{l}
|\input{childdoc.def}|\\
|\childdocby{|\textit{main}|}|\\
\end{tabular}
\end{center}
%
Both forms have slightly different effects as described above.
The main file is prepared as usual, see \secref{sec:include}.

%%%%%%%%%%%%%%%%%%%%%%%%%%%%%%%%%%%%%%%%%%%%%%%%%%%%%%%%%%%%%%%%%%%%%%%%%%%%%%%%
\subsection{Legacy Detection}
\label{sec:detection}

The directive |\childdocmain| in the main file can detect
whether the complete document or merely a child is to be compiled
even without using the directive |\childdocof|.
This method is deprecated because it is less robust
and there is no compelling reason to use it;
it is merely provided for backward compatibility
and it may be removed in future versions.

If the detection mechanism is to be used,
it is mandatory to correctly specify
the filename of the main file as the argument of |\childdocmain|:
%
\begin{center}
\begin{tabular}{l}
|\input{childdoc.def}|\\
|\childdocmain{|\textit{main}|}|\\
\end{tabular}
\end{center}
%
If |\jobname| does not match the argument \textit{main} of |\childdocmain|,
it is assumed that |\jobname| points to the child file to be compiled.
When using |\childdocmain| with the main file specified as argument,
it suffices to start a child file
with just |\input{|\textit{main}|}|
without loading of the package and using |\childdocof|.
If instead all processing is done
with the appropriate \textsf{childdoc} directives,
the argument of \textit{main} of |\childdocmain| can be empty.

An alternative version of the command line processing described
in \secref{sec:commandline} using the detection mechanism reads:
%
\begin{center}
|... -jobname "|\textit{target}|" "|[\textit{flags}]%
[|\def\jobname{|\textit{dest}|}|]|\input{|\textit{main}|}"|
\end{center}

%%%%%%%%%%%%%%%%%%%%%%%%%%%%%%%%%%%%%%%%%%%%%%%%%%%%%%%%%%%%%%%%%%%%%%%%%%%%%%%%
\subsection{Manual Code}
\label{sec:manual}

In case one cannot be certain whether the definitions file |childdoc.def|
is installed on the target \TeX{} distribution
and one prefers not to ship it,
it is conceivable to paste a few relevant commands into the sources.

To that end, drop all statements |\input{childdoc.def}|
and perform the replacements as outlined below.
Instead of |\childdocmain{|\textit{main}|}| add the following code
to the top of the main file:
%
\begin{center}
\begin{tabular}{l}
|\||ifdefined\childdocname\endinput\||fi\newif\ifchilddoc|\\
|\edef\childdocname{\scantokens\expandafter{\jobname\noexpand}}|\\
|\def\childdocmain{|\textit{main}|}\||ifx\childdocmain\childdocname\||else|\\
|\childdoctrue\includeonly{\childdocname}\let\jobname\childdocmain\||fi|\\
\end{tabular}
\end{center}
%
Instead of |\childdocof{|\textit{main}|}| just include the main file
at the top of each child file:
%
\begin{center}
|\input{|\textit{main}|}|
\end{center}
%
A simple redirection |\childdocforward{|\textit{dest}|}| is achieved by:
%
\begin{center}
|\def\jobname{|\textit{dest}|}\input{\jobname}|
\end{center}
%
The redirection with prefix
|\childdocforwardprefix[|\textit{prefix}|]{|\textit{dest}|}|
is accomplished by:
%
\begin{center}
\begin{tabular}{l}
|{\edef\jobname{\scantokens\expandafter{\jobname\noexpand}}|\\
|\def\redirectjob |\textit{prefix}|#1~~~{\gdef\jobname{|\textit{dest}|#1}}|\\
|\expandafter\redirectjob\jobname~~~}\input{\jobname}|
\end{tabular}
\end{center}

In an alternative approach,
child documents can be compiled by a specific command line
without additional code or specific definitions:
%
\begin{center}
|... -jobname "|\textit{target}|" "|[\textit{flags}]%
|\includeonly{|\textit{dest}|}\input{|\textit{main}|}"|
\end{center}
%

%%%%%%%%%%%%%%%%%%%%%%%%%%%%%%%%%%%%%%%%%%%%%%%%%%%%%%%%%%%%%%%%%%%%%%%%%%%%%%%%
%%%%%%%%%%%%%%%%%%%%%%%%%%%%%%%%%%%%%%%%%%%%%%%%%%%%%%%%%%%%%%%%%%%%%%%%%%%%%%%%
\section{Information}

%%%%%%%%%%%%%%%%%%%%%%%%%%%%%%%%%%%%%%%%%%%%%%%%%%%%%%%%%%%%%%%%%%%%%%%%%%%%%%%%
\subsection{Copyright}

Copyright \copyright{} 2017--2018 Niklas Beisert

This work may be distributed and/or modified under the
conditions of the \LaTeX{} Project Public License, either version 1.3
of this license or (at your option) any later version.
The latest version of this license is in
  \url{http://www.latex-project.org/lppl.txt}
and version 1.3 or later is part of all distributions of \LaTeX{}
version 2005/12/01 or later.

This work has the LPPL maintenance status `maintained'.

The Current Maintainer of this work is Niklas Beisert.

This work consists of the files |README.txt|, |childdoc.ins| and |childdoc.dtx|
as well as the derived files |childdoc.def|, |cdocsamp.tex|
with |cdocsch1.tex|, |cdocsch2.tex|, |cdocspt3.tex|, |cdocspt4.tex|,
|cdocsdrf.tex|, |cdocsfn1.tex|, |cdocsfn2.tex|
as well as |childdoc.pdf|.

%%%%%%%%%%%%%%%%%%%%%%%%%%%%%%%%%%%%%%%%%%%%%%%%%%%%%%%%%%%%%%%%%%%%%%%%%%%%%%%%
\subsection{Files and Installation}

The package consists of the files:
%
\begin{center}
\begin{tabular}{ll}
    |README.txt|   & readme file \\
    |childdoc.ins| & installation file \\
    |childdoc.dtx| & source file \\
    |childdoc.def| & definition file \\
    |cdocsamp.tex| & sample main file \\
    |cdocsch1.tex| & sample include file \\
    |cdocsch2.tex| & sample include file \\
    |cdocspt3.tex| & sample part file \\
    |cdocspt4.tex| & sample part file \\
    |cdocsdrf.tex| & sample redirection file \\
    |cdocsfn1.tex| & sample redirection file \\
    |cdocsfn2.tex| & sample redirection file \\
    |childdoc.pdf| & manual
\end{tabular}
\end{center}
%
The distribution consists of the files
|README.txt|, |childdoc.ins| and |childdoc.dtx|.
%
\begin{itemize}
\item
Run (pdf)\LaTeX{} on |childdoc.dtx|
to compile the manual |childdoc.pdf| (this file).
\item
Run \LaTeX{} on |childdoc.ins| to create the definitions file |childdoc.def|
and the sample |cdocsamp.tex| with include files
|cdocsch1.tex|, |cdocsch2.tex|, |cdocspt3.tex|, |cdocspt4.tex|,
|cdocsdrf.tex|, |cdocsfn1.tex|, |cdocsfn2.tex|.
Then copy the file |childdoc.def| to an appropriate directory of your \LaTeX{}
distribution, e.g.\ \textit{texmf-root}|/tex/latex/childdoc|.
\end{itemize}

%%%%%%%%%%%%%%%%%%%%%%%%%%%%%%%%%%%%%%%%%%%%%%%%%%%%%%%%%%%%%%%%%%%%%%%%%%%%%%%%
\subsection{Related CTAN Packages}

There are several other packages which offer a similar functionality:
%
\begin{itemize}
\item
The packages
\href{http://ctan.org/pkg/docmute}{\textsf{docmute}},
\href{http://ctan.org/pkg/includex}{\textsf{includex}} and
\href{http://ctan.org/pkg/standalone}{\textsf{standalone}}
provide commands to include only the document body of
a child file thus allowing both files to be compiled individually.
\item
The packages \href{http://ctan.org/pkg/subdocs}{\textsf{subdocs}}
and \href{http://ctan.org/pkg/subfiles}{\textsf{subfiles}}
provide structures in which the main and child documents can be
encapsulated and allowing them to be compiled individually.
The inclusion mechanism is different from the conventional |\include|.
\item
The package \href{http://ctan.org/pkg/combine}{\textsf{combine}}
is an elaborate solution to combine several documents into one.
\end{itemize}
%
See also the CTAN topic \href{http://ctan.org/topic/subdocs}{\textsf{subdocs}}
for further related packages.
The present package differs from the above solutions in that
a document structure constructed with the conventional |\include| mechanism
just needs two extra commands at the top of every file
such that all constituent files can be compiled individually.

%%%%%%%%%%%%%%%%%%%%%%%%%%%%%%%%%%%%%%%%%%%%%%%%%%%%%%%%%%%%%%%%%%%%%%%%%%%%%%%%
%\subsection{Feature Suggestions}
%
%The following is a list of features which may be useful for future
%versions of this package:
%%
%\begin{itemize}
%\item
%\ldots
%\end{itemize}

%%%%%%%%%%%%%%%%%%%%%%%%%%%%%%%%%%%%%%%%%%%%%%%%%%%%%%%%%%%%%%%%%%%%%%%%%%%%%%%%
\subsection{Revision History}

%%%%%%%%%%%%%%%%%%%%%%%%%%%%%%%%%%%%%%%%
\paragraph{v2.0:} 2018/12/30

\begin{itemize}
\item
immediate forward processing
\item
added |\childdocby| mechanism
\item
manual restructured
\end{itemize}

%%%%%%%%%%%%%%%%%%%%%%%%%%%%%%%%%%%%%%%%
\paragraph{v1.6:} 2018/01/17

\begin{itemize}
\item
application for development of include files
\item
corrections to manual
\end{itemize}

%%%%%%%%%%%%%%%%%%%%%%%%%%%%%%%%%%%%%%%%
\paragraph{v1.5:} 2017/05/21

\begin{itemize}
\item
more complete structuring introduced
\item
|\childdocof| introduced
\item
|\childdoc| renamed to |\childdocmain|
\item
|\childredirect| renamed to |\childdocforward| and |\childdocforwardprefix|
and functionality expanded
\end{itemize}

%%%%%%%%%%%%%%%%%%%%%%%%%%%%%%%%%%%%%%%%
\paragraph{v1.0:} 2017/04/27

\begin{itemize}
\item
manual and install package
\item
first version published on CTAN
\end{itemize}

%%%%%%%%%%%%%%%%%%%%%%%%%%%%%%%%%%%%%%%%
\paragraph{v0.6:} 2017/04/26

\begin{itemize}
\item
redirection mechanism added
\end{itemize}

%%%%%%%%%%%%%%%%%%%%%%%%%%%%%%%%%%%%%%%%
\paragraph{v0.5:} 2017/04/26

\begin{itemize}
\item
functionality in definition file
\end{itemize}


%%%%%%%%%%%%%%%%%%%%%%%%%%%%%%%%%%%%%%%%%%%%%%%%%%%%%%%%%%%%%%%%%%%%%%%%%%%%%%%%
%%%%%%%%%%%%%%%%%%%%%%%%%%%%%%%%%%%%%%%%%%%%%%%%%%%%%%%%%%%%%%%%%%%%%%%%%%%%%%%%
%%%%%%%%%%%%%%%%%%%%%%%%%%%%%%%%%%%%%%%%%%%%%%%%%%%%%%%%%%%%%%%%%%%%%%%%%%%%%%%%
\appendix

\settowidth\MacroIndent{\rmfamily\scriptsize 000\ }

 \DocInput{childdoc.dtx}

\end{document}
%</driver>
% \fi
%
% %%%%%%%%%%%%%%%%%%%%%%%%%%%%%%%%%%%%%%%%%%%%%%%%%%%%%%%%%%%%%%%%%%%%%%%%%%%%%%
% %%%%%%%%%%%%%%%%%%%%%%%%%%%%%%%%%%%%%%%%%%%%%%%%%%%%%%%%%%%%%%%%%%%%%%%%%%%%%%
% \section{Sample}
%\iffalse
%<*samplemain>
%\fi
%
% The following presents a sample document
% with two chapters, two parts, a title page,
% a compile flag as well as three forwarding files to set the flag.
% It consists of eight |.tex| files:
% \begin{center}
% \begin{tabular}{ll}
% |cdocsamp.tex|&main file\\
% |cdocsch1.tex|&include file for chapter 1\\
% |cdocsch2.tex|&include file for chapter 2\\
% |cdocspt3.tex|&include file for part 3\\
% |cdocspt4.tex|&include file for part 4\\
% |cdocsdrf.tex|&forwarding file for main file in draft mode\\
% |cdocsfi1.tex|&forwarding file for final version of chapter 1\\
% |cdocsfi2.tex|&forwarding file for final version of chapter 2\\
% \end{tabular}
% \end{center}
% Each of the eight files can be compiled directly by the \LaTeX{} compiler.
%
% %%%%%%%%%%%%%%%%%%%%%%%%%%%%%%%%%%%%%%
% \paragraph{Main File.}
%
% The main file is called |cdocsamp.tex|.
%
% Load the \textsf{childdoc} definitions and
% declare the filename for the main document:
%    \begin{macrocode}
\input{childdoc.def}
\childdocmain{}
%    \end{macrocode}

% Optional override for |\version| flag:
%    \begin{macrocode}
%%\ifchilddoc\else\providecommand{\version}{draft}\fi
%    \end{macrocode}

% Define the default values for the |\version| flag
% (|final| for the main file and |draft| for childs):
%    \begin{macrocode}
\ifchilddoc
\providecommand{\version}{draft}
\else
\providecommand{\version}{final}
\fi
%    \end{macrocode}

% Load the standard document class:
%    \begin{macrocode}
\documentclass[12pt]{article}
%    \end{macrocode}

% Start the document body:
%    \begin{macrocode}
\begin{document}
%    \end{macrocode}

% Declare a title page.
% Print title, part of document being processed and version flag:
%    \begin{macrocode}
\addtocounter{page}{-1}
\begin{center}
{\LARGE\bfseries{}childdoc example\par}
\vspace{1cm}
\ifchilddoc
\ifchilddocmanual part\else chapter\fi:
`\childdocname' of `\childdocjob'\par
\else
main document: `\childdocjob'\par
\fi
version: \version\par
\end{center}
\newpage
%    \end{macrocode}

% Manually include selected file,
% otherwise process as usual:
%    \begin{macrocode}
\ifchilddocmanual
\section*{part `\childdocname'}
\input{\childdocname}
\else
%    \end{macrocode}

% Include the two chapters:
%    \begin{macrocode}
\include{cdocsch1}
\include{cdocsch2}
%    \end{macrocode}

% Include the two parts unless only chapters should be displayed:
%    \begin{macrocode}
\ifchilddoc\else
\section{part three}
\input{cdocspt3}
\section{part four}
\input{cdocspt4}
\fi
%    \end{macrocode}

% Process as usual until here:
%    \begin{macrocode}
\fi
%    \end{macrocode}

% End of document body:
%    \begin{macrocode}
\end{document}
%    \end{macrocode}
%\iffalse
%</samplemain>
%\fi
%
% %%%%%%%%%%%%%%%%%%%%%%%%%%%%%%%%%%%%%%
% \paragraph{Chapter Include Files.}
%
% The include files are called |cdocsch1.tex| and |cdocsch2.tex|.
%
%\iffalse
%<*samplechap1|samplechap2>
%\fi

% Optional override for |\version| flag:
%    \begin{macrocode}
%%\providecommand{\version}{final}
%    \end{macrocode}

% Include the main document:
%    \begin{macrocode}
\input{childdoc.def}
\childdocof{cdocsamp}
%    \end{macrocode}

%\iffalse
%</samplechap1|samplechap2>
%\fi
%
%\iffalse
%<*samplechap1>
%\fi
% Some text for chapter 1:
%    \begin{macrocode}
\section{one}
some text in chapter one
%    \end{macrocode}

%\iffalse
%</samplechap1>
%\fi
% Some text for chapter 2:
%\iffalse
%<*samplechap2>
%\fi
%    \begin{macrocode}
\section{two}
more text in chapter two
%    \end{macrocode}

%\iffalse
%</samplechap2>
%\fi
%
% %%%%%%%%%%%%%%%%%%%%%%%%%%%%%%%%%%%%%%
% \paragraph{Part Include Files.}
%
% The include files are called |cdocspt3.tex| and |cdocspt4.tex|.
%
%\iffalse
%<*samplepart3|samplepart4>
%\fi

% Optional override for |\version| flag:
%    \begin{macrocode}
%%\providecommand{\version}{final}
%    \end{macrocode}

% Include the main document:
%    \begin{macrocode}
\input{childdoc.def}
\childdocby{cdocsamp}
%    \end{macrocode}

%\iffalse
%</samplepart3|samplepart4>
%\fi
%
%\iffalse
%<*samplepart3>
%\fi
% Some text for part 3:
%    \begin{macrocode}
some text in part three
%    \end{macrocode}

%\iffalse
%</samplepart3>
%\fi
% Some text for part 4:
%\iffalse
%<*samplepart4>
%\fi
%    \begin{macrocode}
more text in part four
%    \end{macrocode}

%\iffalse
%</samplepart4>
%\fi
%
% %%%%%%%%%%%%%%%%%%%%%%%%%%%%%%%%%%%%%%
% \paragraph{Forwarding for a Complete Draft.}
%
% The following forwarding file |cdocsdrf.tex|
% compiles the main document in draft mode:
%\iffalse
%<*sampledraft>
%\fi
%    \begin{macrocode}
\def\version{draft}
\input{childdoc.def}
\childdocforward{cdocsamp}
%    \end{macrocode}

%\iffalse
%</sampledraft>
%\fi
%
% %%%%%%%%%%%%%%%%%%%%%%%%%%%%%%%%%%%%%%
% \paragraph{Forwarding for Final Version of the Chapters.}
%
% The following forwarding files |cdocsfn1.tex| and |cdocsfn2.tex|
% (with identical content)
% compile the final versions of the child documents
% |cdocsch1.tex| and |cdocsch2.tex|, respectively:
%\iffalse
%<*samplefinal>
%\fi
%    \begin{macrocode}
\def\version{final}
\input{childdoc.def}
\childdocforwardprefix[cdocsamp]{cdocsfn}{cdocsch}
%    \end{macrocode}

%\iffalse
%</samplefinal>
%\fi
%
% %%%%%%%%%%%%%%%%%%%%%%%%%%%%%%%%%%%%%%
% \paragraph{Command Line Processing.}
%
% The following three command lines generate the output files
% |cdocscld|, |cdocscl1| and |cdocscl2|
% which should be identical to
% |cdocsdrf|, |cdocsch1| and |cdocsfn2|, respectively:
% \begin{center}
% \begin{tabular}{l}
% |latex -jobname cdocscld \|\\
% |  "\def\version{draft}\input{childdoc.def}\childdocforward{cdocsamp}"|\\
% |latex -jobname cdocscl1 \|\\
% |  "\input{childdoc.def}\childdocforward[cdocsamp]{cdocsch1}"|\\
% |latex -jobname cdocscl2 \|\\
% |  "\def\version{final}\input{childdoc.def}\childdocforward{cdocsch2}"|
% \end{tabular}
% \end{center}
% Note that the trailing backslash on each first line
% merely continues the input to the second line
% (for convenient cut ant paste).
% Furthermore, the command |latex| can be replaced by any
% of its alternative versions such as |pdflatex|.
%
% %%%%%%%%%%%%%%%%%%%%%%%%%%%%%%%%%%%%%%%%%%%%%%%%%%%%%%%%%%%%%%%%%%%%%%%%%%%%%%
% %%%%%%%%%%%%%%%%%%%%%%%%%%%%%%%%%%%%%%%%%%%%%%%%%%%%%%%%%%%%%%%%%%%%%%%%%%%%%%
% \section{Implementation}
%\iffalse
%<*package>
%\fi
%
% This section describes the definitions file |childdoc.def|.

% The definitions cannot be loaded using |\usepackage| or |\RequirePackage|
% which has a mechanism to prevent loading a style file more than once.
% When loading the definitions by means of |\input|
% multiple instances have to be prevented manually:
%\iffalse
%This code needs to be before the `\ProvidesFile' directive
%which is defined at the beginning of this file.
%Therefore it is also placed there and commented out here.
%</package>
%<*discard>
%\fi
%    \begin{macrocode}
\ifdefined\childdocmain\endinput\fi
%    \end{macrocode}
%\iffalse
%</discard>
%<*package>
%\fi
%
% \macro{\ifchilddoc}
% \macro{\ifchilddocmanual}
% The conditional |\ifchilddoc| tells whether a
% child (true) or main (false) document is being compiled.
% The conditional |\ifchilddocmanual| tells whether
% the |\includeonly| mechanism is used (false) or
% the selection of child files must be performed manually (true).
% The definitions initialise to false:
%    \begin{macrocode}
\newif\ifchilddoc
\newif\ifchilddocmanual
%    \end{macrocode}

% \macro{\childdocname}
% \macro{\childdocjob}
% The macro |\childdocname| stores the name of the main document
% to be compiled. The macro |\childdocjob| stores the name of
% the document on which the \LaTeX{} compiler was originally invoked.
% The content of |\jobname| cannot be compared
% to filenames specified in the source due to different catcodes.
% The following code rescans |\jobname|, stores the result
% in |\childdocname| and saves a copy in |\childdocjob|:
%    \begin{macrocode}
\edef\childdocname{\scantokens\expandafter{\jobname\noexpand}}
\let\childdocjob\childdocname
%    \end{macrocode}

% \macro{\childdocdisable}
% The macro |\childdocdisable| prevents the main file
% from being processed more than once.
% At this stage, the main document command |\childdocmain|
% is assumed to be called once again where it should do nothing.
% Any subsequent call to it should prevent
% a secondary processing of the main document
% It overwrites the forwarding commands
% |\childdocof| and |\childdocforward|
% with empty macros to prevent further inclusions of the main document:
%    \begin{macrocode}
\newcommand{\childdocdisable}
{
  \renewcommand{\childdocmain}[1]{\renewcommand{\childdocmain}[1]{\endinput}}
  \renewcommand{\childdocof}[1]{}
  \renewcommand{\childdocby}[2][]{}
  \renewcommand{\childdocforward}[2][]{}
  \renewcommand{\childdocdisable}{}
}
%    \end{macrocode}

% \macro{\childdocmain}
% The macro |\childdocmain| is to be called at the top of the main file
% with nothing or the main filename (without extension) as argument.
% First, it breaks loops.
% If the argument is not empty and does not match |\childdocname|
% (which is set by the first inclusion of |childdoc.def|),
% |\ifchilddoc| is set to true, |\includeonly| is applied to the child file
% and |\jobname| is set to the main file
% (for proper handling of |.aux| files):
%    \begin{macrocode}
\newcommand{\childdocmain}[1]
{
  \childdocdisable\childdocmain{}
  \if?#1?\else
    \begingroup
      \def\childdoctmp{#1}
      \ifx\childdoctmp\childdocname
        \def\childdoctmp{}
      \else
        \def\childdoctmp
        {
          \childdoctrue
          \includeonly{\childdocname}
          \def\childdocjob{#1}
          \def\jobname{#1}
        }
      \fi
      \expandafter
    \endgroup
    \childdoctmp
  \fi
}
%    \end{macrocode}

% \macro{\childdocof}
% The command |\childdocof| redirects
% compilation to the main file |#1|.
%    \begin{macrocode}
\newcommand{\childdocof}[1]
{
  \childdocdisable
  \childdoctrue
  \includeonly{\childdocname}
  \def\jobname{#1}
  \def\childdocjob{#1}
  \input{#1}
}
%    \end{macrocode}

% \macro{\childdocby}
% The command |\childdocby| ....
%    \begin{macrocode}
\newcommand{\childdocby}[2][]
{
  \childdocdisable
  \childdoctrue
  \childdocmanualtrue
  \if?#1?\else
    \def\jobname{#2}
  \fi
  \def\childdocjob{#2}
  \input{#2}
  \endinput
}
%    \end{macrocode}

% \macro{\childdocforward}
% The command |\childdocforward| redirects
% compilation to the main file or
% (if the optional argument is given) a child file.
% Parameters are set as if the main file
% or a child file starting with |\childdocof| was compiled.
% Then compilation is handed over to the main file:
%    \begin{macrocode}
\newcommand{\childdocforward}[2][]
{
  \begingroup
    \if?#1?
      \def\childdoctmp
      {
        \def\childdocname{#2}
        \def\childdocjob{#2}
        \def\jobname{#2}
        \input{#2}
        \endinput
      }
    \else
      \def\childdoctmp
      {
        \childdocdisable
        \def\childdocname{#2}
        \childdoctrue
        \includeonly{#2}
        \def\childdocjob{#1}
        \def\jobname{#1}
        \input{#1}
        \endinput
      }
    \fi
    \expandafter
  \endgroup
  \childdoctmp
}
%    \end{macrocode}

% \macro{\childdocforwardprefix}
% The command |\childdocforwardprefix| redirects
% compilation to the main or a child file by means of a pattern.
% The prefix |#1| in the current filename is replaced by |#2|
% and the suffix of the current filename is kept
% (it is assumed that the filename does not contain the substring `|~~~|'
% which is used as a delimiter).
% Compilation is handed over to the new file by |\childdocforward|:
%    \begin{macrocode}
\newcommand{\childdocforwardprefix}[3][]
{
  \begingroup
    \def\childdocextract #2##1~~~{\def\childdoctmp{\childdocforward[#1]{#3##1}}}
    \expandafter\childdocextract\childdocname~~~
    \expandafter
  \endgroup
  \childdoctmp
}
%    \end{macrocode}

% \macro{\childdoc}
% The deprecated macro |\childdoc| is a legacy version of |\childdocmain|:
%    \begin{macrocode}
\newcommand{\childdoc}{\childdocmain}
%    \end{macrocode}

% \macro{\childdocredirect}
% The deprecated macro |\childdocredirect| is a legacy version
% of |\childdocforward| and |\childdocforwardprefix|:
%    \begin{macrocode}
\newcommand{\childdocredirect}[2][]
{
  \begingroup
    \if?#1?
      \def\childdoctmp{\childdocforward{#2}}
    \else
      \def\childdoctmp{\childdocforwardprefix{#1}{#2}}
    \fi
    \expandafter
  \endgroup
  \childdoctmp
}
%    \end{macrocode}

%\iffalse
%</package>
%\fi
%
\endinput

\childdocby{cdocsamp}
%    \end{macrocode}

%\iffalse
%</samplepart3|samplepart4>
%\fi
%
%\iffalse
%<*samplepart3>
%\fi
% Some text for part 3:
%    \begin{macrocode}
some text in part three
%    \end{macrocode}

%\iffalse
%</samplepart3>
%\fi
% Some text for part 4:
%\iffalse
%<*samplepart4>
%\fi
%    \begin{macrocode}
more text in part four
%    \end{macrocode}

%\iffalse
%</samplepart4>
%\fi
%
% %%%%%%%%%%%%%%%%%%%%%%%%%%%%%%%%%%%%%%
% \paragraph{Forwarding for a Complete Draft.}
%
% The following forwarding file |cdocsdrf.tex|
% compiles the main document in draft mode:
%\iffalse
%<*sampledraft>
%\fi
%    \begin{macrocode}
\def\version{draft}
% \iffalse
%
% childdoc.dtx Copyright (C) 2017-2018 Niklas Beisert
%
% This work may be distributed and/or modified under the
% conditions of the LaTeX Project Public License, either version 1.3
% of this license or (at your option) any later version.
% The latest version of this license is in
%   http://www.latex-project.org/lppl.txt
% and version 1.3 or later is part of all distributions of LaTeX
% version 2005/12/01 or later.
%
% This work has the LPPL maintenance status `maintained'.
%
% The Current Maintainer of this work is Niklas Beisert.
%
% This work consists of the files childdoc.dtx and childdoc.ins
% and the derived files childdoc.def and cdocsamp.tex with
% cdocsch1.tex, cdocsch2.tex, cdocsdrf.tex, cdocsfn1.tex, cdocsfn2.tex.
%
%<package>\ifdefined\childdocmain\endinput\fi
%<package>\ProvidesFile{childdoc.def}[2018/12/30 v2.0 child document driver]
%<samplemain>\ProvidesFile{cdocsamp.tex}[2018/12/30 v2.0 sample for childdoc]
%<*driver>
%\ProvidesFile{childdoc.drv}[2018/12/30 v2.0 childdoc reference manual file]
\PassOptionsToClass{10pt,a4paper}{article}
\documentclass{ltxdoc}

\usepackage[margin=35mm]{geometry}
\usepackage{hyperref}
\usepackage{hyperxmp}
\usepackage[usenames]{color}

\hypersetup{colorlinks=true}
\hypersetup{pdfstartview=FitH}
\hypersetup{pdfpagemode=UseNone}
\hypersetup{pdfsource={}}
\hypersetup{pdflang={en-UK}}
\hypersetup{pdfcopyright={Copyright 2017-2018 Niklas Beisert.
  This work may be distributed and/or modified under the
  conditions of the LaTeX Project Public License, either version 1.3
  of this license or (at your option) any later version.}}
\hypersetup{pdflicenseurl={http://www.latex-project.org/lppl.txt}}
\hypersetup{pdfcontactaddress={ETH Zurich, ITP, HIT K,
  Wolfgang-Pauli-Strasse 27}}
\hypersetup{pdfcontactpostcode={8093}}
\hypersetup{pdfcontactcity={Zurich}}
\hypersetup{pdfcontactcountry={Switzerland}}
\hypersetup{pdfcontactemail={nbeisert@itp.phys.ethz.ch}}
\hypersetup{pdfcontacturl={http://people.phys.ethz.ch/\xmptilde nbeisert/}}

\newcommand{\secref}[1]{\hyperref[#1]{section \ref*{#1}}}

\parskip1ex
\parindent0pt
\let\olditemize\itemize
\def\itemize{\olditemize\parskip0pt}

\begin{document}

\title{The \textsf{childdoc} Package}
\hypersetup{pdftitle={The childdoc Package}}
\author{Niklas Beisert\\[2ex]
  Institut f\"ur Theoretische Physik\\
  Eidgen\"ossische Technische Hochschule Z\"urich\\
  Wolfgang-Pauli-Strasse 27, 8093 Z\"urich, Switzerland\\[1ex]
  \href{mailto:nbeisert@itp.phys.ethz.ch}
  {\texttt{nbeisert@itp.phys.ethz.ch}}}
\hypersetup{pdfauthor={Niklas Beisert}}
\hypersetup{pdfsubject={Manual for the LaTeX2e Package childdoc}}
\date{30 December 2018, \textsf{v2.0}}
\maketitle

\begin{abstract}\noindent
\textsf{childdoc} is a \LaTeXe{} package
that enables the direct compilation
of document sections included by |\include|
to individual files.
\end{abstract}

\begingroup
\parskip0ex
\tableofcontents
\endgroup

%%%%%%%%%%%%%%%%%%%%%%%%%%%%%%%%%%%%%%%%%%%%%%%%%%%%%%%%%%%%%%%%%%%%%%%%%%%%%%%%
%%%%%%%%%%%%%%%%%%%%%%%%%%%%%%%%%%%%%%%%%%%%%%%%%%%%%%%%%%%%%%%%%%%%%%%%%%%%%%%%
\section{Introduction}

\LaTeX{} provides a mechanism to structure a large document (such as a book)
into a main file and several child files (containing the chapters)
using the |\include| command.
This mechanism is beneficial for documents
which span hundreds of pages in order to
make the source file(s) more manageable.
Moreover, compilation can be restricted to
selected child files by means of the |\includeonly| command.
The latter feature can be used to reduce the compilation time while editing
(this was significantly more useful in the earlier days of \LaTeX{})
or to generate a smaller document which is easier to navigate.
Another application of |\includeonly| is to generate
documents consisting of selected parts of the complete document.

However, there are a few drawbacks of the plain |\include| mechanism:
\begin{itemize}
\item
The child files cannot be compiled on their own,
they can only be compiled via the main file.
A naive editing environment
(such as a text editor with an option
to have the current file processed by \LaTeX)
may require one to switch to the main file before compiling;
attempting to compile the child file produces errors.
\item
The main file must be modified (each time)
to adjust the |\includeonly| command
to the present needs. This easily leaves the main file in a messy state.
\item
The generated document will always carry the filename
of the main document. This is inconvenient if
several child files are to be compiled and
to be kept for distribution.
\end{itemize}

The present package provides a simple interface
to make child files individually compilable by \LaTeX{}.
Compiling a child file then has the same effect as compiling
the main file with an |\includeonly| command
to select the appropriate child.
Moreover the generated document will carry the name of the child
rather than the main file.
This resolves all three above issues.

This feature is meant to make the editing of books,
thesis documents and lecture notes somewhat more convenient.
However, the package can also be used efficiently for
composing a series of documents (such as exercise sheets)
which are typically distributed individually.
It then assists the author in generating the individual documents
(potentially in different versions)
as well as a document containing the collected series.
Another application is in developing style files
or other kinds of included material
where compilation of the style file could redirect
to a sample or test file.

%%%%%%%%%%%%%%%%%%%%%%%%%%%%%%%%%%%%%%%%%%%%%%%%%%%%%%%%%%%%%%%%%%%%%%%%%%%%%%%%
%%%%%%%%%%%%%%%%%%%%%%%%%%%%%%%%%%%%%%%%%%%%%%%%%%%%%%%%%%%%%%%%%%%%%%%%%%%%%%%%
\section{Usage}

First of all, the package \textsf{childdoc} is \emph{not} a standard
\LaTeXe{} |.sty| style file! Therefore it needs to be invoked in
a non-standard way.

%%%%%%%%%%%%%%%%%%%%%%%%%%%%%%%%%%%%%%%%%%%%%%%%%%%%%%%%%%%%%%%%%%%%%%%%%%%%%%%%
\subsection{Included Files}
\label{sec:include}

%%%%%%%%%%%%%%%%%%%%%%%%%%%%%%%%%%%%%%%%
\DescribeMacro{\childdocmain}
To use the package, add the commands
\begin{center}
\begin{tabular}{l}
|\input{childdoc.def}|\\
|\childdocmain{}|\\
\end{tabular}
\end{center}
at the very top of the main \LaTeX{} file,
in particular \emph{before} the |\documentclass| statement!
The argument of |\childdocmain| should be left empty
(but it must be present).

%%%%%%%%%%%%%%%%%%%%%%%%%%%%%%%%%%%%%%%%
\DescribeMacro{\childdocof}
Furthermore, add the commands
\begin{center}
\begin{tabular}{l}
|\input{childdoc.def}|\\
|\childdocof{|\textit{main}|}|\\
\end{tabular}
\end{center}
at the top of every child file \textit{child}
which is included by |\include{|\textit{child}|}|
from within the main file
(or at least for those files to be compiled individually).
The argument \textit{main} must be the filename of the main file.

There are a couple of
considerations in setting up the main and child documents:

%%%%%%%%%%%%%%%%%%%%%%%%%%%%%%%%%%%%%%%%
\paragraph{Restrictions.}

Please note the following restrictions:
\begin{itemize}
\item
|\childdocmain| must be called with one argument \textit{main}
to ensure compatibility with earlier version of the package.
It must either be empty (|\childdocmain{}|)
or precisely match the filename of the main file in which it is specified.
See \secref{sec:detection} for further information.
\item
The filename \textit{main} must be specified without the |.tex| extension.
\item
The filename \textit{main} is case sensitive
(even in case-insensitive file systems)
due to internal string comparison.
\item
The argument \textit{main} should be fully expanded, it cannot be a macro.
\item
Subdirectories and special characters should be avoided in filenames.
\item
The command |\childdocmain{|\textit{main}|}| must be followed by a whitespace.
It should not be followed immediately by another command
or by a comment mark `|%|'.
This is because the \TeX{} parser reads the token immediately following
the argument of |\childdocmain| and puts it
at the beginning of every child section;
however, a white\-space is ignored.
\end{itemize}

%%%%%%%%%%%%%%%%%%%%%%%%%%%%%%%%%%%%%%%%
\paragraph{Content of Main File.}

It is advisable to place all content in the child files included by |\include|.
Any output contained in the main file will appear in all child documents
unless suppressed manually;
it cannot be suppressed automatically by the |\includeonly| directive
and thus should normally be avoided.
A method to include some content in the main file
by means of conditional processing is described in \secref{sec:conditional}.

%%%%%%%%%%%%%%%%%%%%%%%%%%%%%%%%%%%%%%%%
\paragraph{Page Numbering.}

When only a part of the document is compiled,
the appropriate numbering of pages
(as well as other status parameters)
is determined from the |.aux| files.
The latter contain information from previous passes.
However this information needs to propagate through
all intermediate child documents.
Therefore the page numbering in child documents may well
be inconsistent until the complete document is compiled at least once.

A useful (if unconventional) way to always ensure a consistent
page numbering is to restart the numbering in each child document
and denote the pages by `\textit{child}|.|\textit{page}'
where \textit{child} represents the chapter/section number of the child file.
This can be achieved by the command
|\numberwithin{page}{|\textit{child}|}|
of the \textsf{amsmath} package
where \textit{child} can be |chapter| or |section|
depending on the chosen structuring.
Alternatively, one can modify the macro |\thepage| appropriately
and reset the counter |page| at the start of each child file.

%%%%%%%%%%%%%%%%%%%%%%%%%%%%%%%%%%%%%%%%%%%%%%%%%%%%%%%%%%%%%%%%%%%%%%%%%%%%%%%%
\subsection{Conditional Processing}
\label{sec:conditional}

The package provides a mechanism to compile different versions
of a document. To customise the versions further some conditional processing
can come in handy to distinguish which version is being compiled.
The package provides two macros to describe the compilation context:

%%%%%%%%%%%%%%%%%%%%%%%%%%%%%%%%%%%%%%%%
\DescribeMacro{\ifchilddoc}
The conditional |\ifchilddoc| distinguishes between the compilation of
child documents and the main document:
%
\begin{center}
|\ifchilddoc |\textit{child-code}| |[|\||else |\textit{main-code}]| \||fi|
\end{center}

%%%%%%%%%%%%%%%%%%%%%%%%%%%%%%%%%%%%%%%%
\DescribeMacro{\childdocname}
\DescribeMacro{\childdocjob}
The macro |\childdocname| contains the filename (without extension)
of the main or child file being processed.
Note that |\childdocjob| will always contain the name of the main file.

%%%%%%%%%%%%%%%%%%%%%%%%%%%%%%%%%%%%%%%%
\paragraph{Title Page.}

Conditional processing can be used to include a title or banner page
in the main document when proper precautions are taken.
Importantly, the code in the main file should ensure that the page counter
(as well as other status parameters which are stored in the |.aux| files)
takes the same value after the conditional processing.
Otherwise the page numbers may take divergent values
depending on which part is compiled.

For example, a title page could be declared by:
%
\begin{center}
\begin{tabular}{l}
|\ifchilddoc\||else|\\
|\addtocounter{page}{-1}|\\
\textit{code for title page}\\
|\newpage|\\
|\||fi|
\end{tabular}
\end{center}
%
A banner page for the child documents can be generated by:
%
\begin{center}
\begin{tabular}{l}
|\ifchilddoc|\\
|\addtocounter{page}{-1}|\\
\textit{code for banner page}\\
|\newpage|\\
|\||fi|
\end{tabular}
\end{center}
%
Here one could write a message such as:
\begin{center}
|This is the part \childdocname{} of \childdocjob{}.|
\end{center}

%%%%%%%%%%%%%%%%%%%%%%%%%%%%%%%%%%%%%%%%%%%%%%%%%%%%%%%%%%%%%%%%%%%%%%%%%%%%%%%%
\subsection{Flags}
\label{sec:flags}

The package makes it easy to generate different versions
of the main or child documents.
To this end compilation flags can be defined
and assigned different default values.
They will be particularly useful in conjunction
with the forwarding mechanism described in \secref{sec:forward}.

For example, it may be useful to have a flag |\version|
which can be set to |draft| or |final|.
The document source will contain some conditional code
depending on the value of |\version|.
Suppose further, the flag should default to |final| for the main file
and to |draft| for child files
which is a natural assignment for editing the document.
This is achieved by placing the following code
in the preamble of the main document
(below the |\childdocmain| directive):
%
\begin{center}
\begin{tabular}{l}
|\ifchilddoc|\\
|\providecommand{\version}{draft}|\\
|\||else|\\
|\providecommand{\version}{final}|\\
|\||fi|
\end{tabular}
\end{center}
%
The definition by |\providecommand| makes sure
that previous definitions are not overwritten.
Further statements |\providecommand{\version}{...}|
can thus be added before the above code to override it.

For the main file, one might add a line
(between |\childdocmain| and the above block)
%
\begin{center}
|%\ifchilddoc\||else\providecommand{\version}{draft}\||fi|
\end{center}
%
which can be uncommented to produce a draft version.
Likewise one can add a line to the very top of a child file
(above the |\childdocof{|\textit{main}|}| directive)
%
\begin{center}
|%\providecommand{\version}{final}|
\end{center}
%
which can be uncommented to produce the final version of this child document.

%%%%%%%%%%%%%%%%%%%%%%%%%%%%%%%%%%%%%%%%%%%%%%%%%%%%%%%%%%%%%%%%%%%%%%%%%%%%%%%%
\subsection{Forwarding}
\label{sec:forward}

Different versions of the main or child documents
using compilation flags as described in \secref{sec:flags}
can be (permanently) stored in different files
for convenient compilation, viewing and distribution.
To this end, the package defines a command
to pass on compilation to a different file:

%%%%%%%%%%%%%%%%%%%%%%%%%%%%%%%%%%%%%%%%
\DescribeMacro{\childdocforward}
The command |\childdocforward| redirects processing to
another source file:
%
\begin{center}
\begin{tabular}{l}
|\input{childdoc.def}|\\
|\childdocforward[|\textit{main}|]{|\textit{dest}|}|\\
\end{tabular}
\end{center}
%
The argument \textit{dest} is the destination file
(without extension).
It should be the main file or one of the child files.
Note that further \textsf{childdoc} directives
such as |\childdocof| and |\childdocforward|
in the indicated file will be processed in this form.
The optional argument \textit{main}
passes on directly to the main file \textit{main}
while pretending to compile the child \textit{dest}.
This form behaves as if \textit{dest}
issues |\childdocof{|\textit{main}|}| right away,
and no further \textsf{childdoc} directives will be processed.

%%%%%%%%%%%%%%%%%%%%%%%%%%%%%%%%%%%%%%%%
\DescribeMacro{\...prefix}
In the alternative form |\childdocforwardprefix|,
%
\begin{center}
\begin{tabular}{l}
|\input{childdoc.def}|\\
|\childdocforwardprefix[|\textit{main}|]{|\textit{prefix}|}{|\textit{dest}|}|
\end{tabular}
\end{center}
%
the destination file is determined by a pattern
depending on the current file:
To make this work, the current file must be called
`{\textit{prefix}\hspace{0.2em}\textit{suffix}}'
with \textit{prefix} matching precisely the argument.
Processing is then passed on to the file
`{\textit{dest}\hspace{0.2em}\textit{suffix}}'.
Surely, the same effect is achieved by
directly specifying the
argument `{\textit{dest}\hspace{0.2em}\textit{suffix}}'
in the first form.
However, that requires to set up a different file
for each child. With the alternative form of the command
all these files can have exactly the same content
which simplifies setting them up and maintaining them.

For example, the following file |draft.tex|
with a compilation flag |\version| as described in \secref{sec:flags}
compiles the main document as a draft:
%
\begin{center}
\begin{tabular}{l}
|\def\version{draft}|\\
|\input{childdoc.def}|\\
|\childdocforward{|\textit{main}|}|
\end{tabular}
\end{center}
%
Likewise, the following files |final|\textit{nn}|.tex|
compile the final version of the child document
|child|\textit{nn}|.tex|:
%
\begin{center}
\begin{tabular}{l}
|\def\version{final}|\\
|\input{childdoc.def}|\\
|\childdocforwardprefix{final}{child}|
\end{tabular}
\end{center}
%

Note that when several versions of a main file and/or of each child file
are to be generated, it may be convenient to set up a |Makefile| or
shell script to automatise the process.

%%%%%%%%%%%%%%%%%%%%%%%%%%%%%%%%%%%%%%%%%%%%%%%%%%%%%%%%%%%%%%%%%%%%%%%%%%%%%%%%
\subsection{Command Line Processing}
\label{sec:commandline}

The effect of redirection files can also be achieved by invoking
the \LaTeX{} compiler with a more elaborate command line.
Most conveniently this should be done as part
of a shell script or a |Makefile|.

When using \textsf{childdoc} in the main file, the following
command lines effectively perform a redirection
(note that depending on the shell being used,
backslashes may have to be doubled: `|\|' $\to$ `|\\|'):
%
\begin{center}
|... -jobname "|\textit{target}|" |\\|"|[\textit{flags}]%
|\input{childdoc.def}\childdocforward[|\textit{main}|]{|\textit{dest}|}"|
\end{center}
%
Here \textit{target} is the name of the output file,
\textit{main} is the name of the main file
and \textit{dest} is the name of the main or child file to be processed
(all filenames without extensions).
The optional argument \textit{main} can be omitted
if \textit{main} matches \textit{dest}.
Optionally, compilation \textit{flags} can be defined via |\def| commands.
This command line makes the \TeX{} engine believe
it is compiling the file \textit{target}
whose content is specified as the latter parameter.
The provided code then forwards the processing to
\textit{main} or \textit{dest} as described in \secref{sec:forward}.

%%%%%%%%%%%%%%%%%%%%%%%%%%%%%%%%%%%%%%%%%%%%%%%%%%%%%%%%%%%%%%%%%%%%%%%%%%%%%%%%
\subsection{Include by Input}
\label{sec:input}

Including child documents by |\include| has some restrictions by design.
Most notably, the content of a child document always occupies
its own set of pages; pages cannot be shared between child documents.
Usually, this behaviour makes perfect sense
because each child document contain an essential part of the document.
However, in some situations it may be desirable to compose
a document from a collection of parts
without having mandatory page breaks between then.
For this case, the package
provides a mechanism to include parts
by |\input| which can also be processed individually.
However, by construction this mechanism
requires manual handling of the content to be output.

%%%%%%%%%%%%%%%%%%%%%%%%%%%%%%%%%%%%%%%%
\DescribeMacro{\ifchilddocmanual}
The main file should be prepared as usual, see \secref{sec:include}.
However, the document body must make a distinction
between processing of an individual part and of the main document, e.g.:
%
\begin{center}
\begin{tabular}{l}
|\ifchilddocmanual|\\
|\input{\childdocname}|\\
|\||else|\\
\textit{document body with }|\input{|\textit{part}|}|\\
|\||fi|
\end{tabular}
\end{center}
%
The conditional |\ifchilddocmanual| is true whenever
a part to be included by |\input| is being compiled,
and the name of the part is stored in |\childdocname|.

%%%%%%%%%%%%%%%%%%%%%%%%%%%%%%%%%%%%%%%%
\DescribeMacro{\childdocby}
Each part to be included by |\input| should start with:
%
\begin{center}
\begin{tabular}{l}
|\input{childdoc.def}|\\
|\childdocby{|\textit{main}|}|\\
\end{tabular}
\end{center}
%
The directive |\childdocby| is similar to |\childdocof|
described in \secref{sec:include},
but the subsequent selection of content must be done manually.
To that end, both |\ifchilddoc| and |\ifchilddocmanual|
will be true upon processing of a part,
and the name of the part is stored in |\childdocname|.
Note that |\jobname| will be set to the filename of the current part
so that each part receives an individual |.aux| file
that does not interfere with the |.aux| file(s) of the main document.
This behaviour can be altered by the alternative form
|\childdocby[*]{|\textit{main}|}| (with a non-empty optional argument)
which uses the |.aux| file of the main document
by setting |\jobname| to \textit{main}.

%%%%%%%%%%%%%%%%%%%%%%%%%%%%%%%%%%%%%%%%%%%%%%%%%%%%%%%%%%%%%%%%%%%%%%%%%%%%%%%%
\subsection{Driver Development}
\label{sec:driver}

The \textsf{childdoc} mechanism can also be use for the development
of definition files such as \LaTeX{} styles or classes.
This case differs from the above setup with multiple parts
included by |\include| in that no |\includeonly| should be invoked.
This can be achieved by starting the include file
(before |\ProvidesPackage|) with:
%
\begin{center}
\begin{tabular}{l}
|\input{childdoc.def}|\\
|\childdocforward{|\textit{main}|}|\\
\end{tabular}
\end{center}
%
or alternatively with:
%
\begin{center}
\begin{tabular}{l}
|\input{childdoc.def}|\\
|\childdocby{|\textit{main}|}|\\
\end{tabular}
\end{center}
%
Both forms have slightly different effects as described above.
The main file is prepared as usual, see \secref{sec:include}.

%%%%%%%%%%%%%%%%%%%%%%%%%%%%%%%%%%%%%%%%%%%%%%%%%%%%%%%%%%%%%%%%%%%%%%%%%%%%%%%%
\subsection{Legacy Detection}
\label{sec:detection}

The directive |\childdocmain| in the main file can detect
whether the complete document or merely a child is to be compiled
even without using the directive |\childdocof|.
This method is deprecated because it is less robust
and there is no compelling reason to use it;
it is merely provided for backward compatibility
and it may be removed in future versions.

If the detection mechanism is to be used,
it is mandatory to correctly specify
the filename of the main file as the argument of |\childdocmain|:
%
\begin{center}
\begin{tabular}{l}
|\input{childdoc.def}|\\
|\childdocmain{|\textit{main}|}|\\
\end{tabular}
\end{center}
%
If |\jobname| does not match the argument \textit{main} of |\childdocmain|,
it is assumed that |\jobname| points to the child file to be compiled.
When using |\childdocmain| with the main file specified as argument,
it suffices to start a child file
with just |\input{|\textit{main}|}|
without loading of the package and using |\childdocof|.
If instead all processing is done
with the appropriate \textsf{childdoc} directives,
the argument of \textit{main} of |\childdocmain| can be empty.

An alternative version of the command line processing described
in \secref{sec:commandline} using the detection mechanism reads:
%
\begin{center}
|... -jobname "|\textit{target}|" "|[\textit{flags}]%
[|\def\jobname{|\textit{dest}|}|]|\input{|\textit{main}|}"|
\end{center}

%%%%%%%%%%%%%%%%%%%%%%%%%%%%%%%%%%%%%%%%%%%%%%%%%%%%%%%%%%%%%%%%%%%%%%%%%%%%%%%%
\subsection{Manual Code}
\label{sec:manual}

In case one cannot be certain whether the definitions file |childdoc.def|
is installed on the target \TeX{} distribution
and one prefers not to ship it,
it is conceivable to paste a few relevant commands into the sources.

To that end, drop all statements |\input{childdoc.def}|
and perform the replacements as outlined below.
Instead of |\childdocmain{|\textit{main}|}| add the following code
to the top of the main file:
%
\begin{center}
\begin{tabular}{l}
|\||ifdefined\childdocname\endinput\||fi\newif\ifchilddoc|\\
|\edef\childdocname{\scantokens\expandafter{\jobname\noexpand}}|\\
|\def\childdocmain{|\textit{main}|}\||ifx\childdocmain\childdocname\||else|\\
|\childdoctrue\includeonly{\childdocname}\let\jobname\childdocmain\||fi|\\
\end{tabular}
\end{center}
%
Instead of |\childdocof{|\textit{main}|}| just include the main file
at the top of each child file:
%
\begin{center}
|\input{|\textit{main}|}|
\end{center}
%
A simple redirection |\childdocforward{|\textit{dest}|}| is achieved by:
%
\begin{center}
|\def\jobname{|\textit{dest}|}\input{\jobname}|
\end{center}
%
The redirection with prefix
|\childdocforwardprefix[|\textit{prefix}|]{|\textit{dest}|}|
is accomplished by:
%
\begin{center}
\begin{tabular}{l}
|{\edef\jobname{\scantokens\expandafter{\jobname\noexpand}}|\\
|\def\redirectjob |\textit{prefix}|#1~~~{\gdef\jobname{|\textit{dest}|#1}}|\\
|\expandafter\redirectjob\jobname~~~}\input{\jobname}|
\end{tabular}
\end{center}

In an alternative approach,
child documents can be compiled by a specific command line
without additional code or specific definitions:
%
\begin{center}
|... -jobname "|\textit{target}|" "|[\textit{flags}]%
|\includeonly{|\textit{dest}|}\input{|\textit{main}|}"|
\end{center}
%

%%%%%%%%%%%%%%%%%%%%%%%%%%%%%%%%%%%%%%%%%%%%%%%%%%%%%%%%%%%%%%%%%%%%%%%%%%%%%%%%
%%%%%%%%%%%%%%%%%%%%%%%%%%%%%%%%%%%%%%%%%%%%%%%%%%%%%%%%%%%%%%%%%%%%%%%%%%%%%%%%
\section{Information}

%%%%%%%%%%%%%%%%%%%%%%%%%%%%%%%%%%%%%%%%%%%%%%%%%%%%%%%%%%%%%%%%%%%%%%%%%%%%%%%%
\subsection{Copyright}

Copyright \copyright{} 2017--2018 Niklas Beisert

This work may be distributed and/or modified under the
conditions of the \LaTeX{} Project Public License, either version 1.3
of this license or (at your option) any later version.
The latest version of this license is in
  \url{http://www.latex-project.org/lppl.txt}
and version 1.3 or later is part of all distributions of \LaTeX{}
version 2005/12/01 or later.

This work has the LPPL maintenance status `maintained'.

The Current Maintainer of this work is Niklas Beisert.

This work consists of the files |README.txt|, |childdoc.ins| and |childdoc.dtx|
as well as the derived files |childdoc.def|, |cdocsamp.tex|
with |cdocsch1.tex|, |cdocsch2.tex|, |cdocspt3.tex|, |cdocspt4.tex|,
|cdocsdrf.tex|, |cdocsfn1.tex|, |cdocsfn2.tex|
as well as |childdoc.pdf|.

%%%%%%%%%%%%%%%%%%%%%%%%%%%%%%%%%%%%%%%%%%%%%%%%%%%%%%%%%%%%%%%%%%%%%%%%%%%%%%%%
\subsection{Files and Installation}

The package consists of the files:
%
\begin{center}
\begin{tabular}{ll}
    |README.txt|   & readme file \\
    |childdoc.ins| & installation file \\
    |childdoc.dtx| & source file \\
    |childdoc.def| & definition file \\
    |cdocsamp.tex| & sample main file \\
    |cdocsch1.tex| & sample include file \\
    |cdocsch2.tex| & sample include file \\
    |cdocspt3.tex| & sample part file \\
    |cdocspt4.tex| & sample part file \\
    |cdocsdrf.tex| & sample redirection file \\
    |cdocsfn1.tex| & sample redirection file \\
    |cdocsfn2.tex| & sample redirection file \\
    |childdoc.pdf| & manual
\end{tabular}
\end{center}
%
The distribution consists of the files
|README.txt|, |childdoc.ins| and |childdoc.dtx|.
%
\begin{itemize}
\item
Run (pdf)\LaTeX{} on |childdoc.dtx|
to compile the manual |childdoc.pdf| (this file).
\item
Run \LaTeX{} on |childdoc.ins| to create the definitions file |childdoc.def|
and the sample |cdocsamp.tex| with include files
|cdocsch1.tex|, |cdocsch2.tex|, |cdocspt3.tex|, |cdocspt4.tex|,
|cdocsdrf.tex|, |cdocsfn1.tex|, |cdocsfn2.tex|.
Then copy the file |childdoc.def| to an appropriate directory of your \LaTeX{}
distribution, e.g.\ \textit{texmf-root}|/tex/latex/childdoc|.
\end{itemize}

%%%%%%%%%%%%%%%%%%%%%%%%%%%%%%%%%%%%%%%%%%%%%%%%%%%%%%%%%%%%%%%%%%%%%%%%%%%%%%%%
\subsection{Related CTAN Packages}

There are several other packages which offer a similar functionality:
%
\begin{itemize}
\item
The packages
\href{http://ctan.org/pkg/docmute}{\textsf{docmute}},
\href{http://ctan.org/pkg/includex}{\textsf{includex}} and
\href{http://ctan.org/pkg/standalone}{\textsf{standalone}}
provide commands to include only the document body of
a child file thus allowing both files to be compiled individually.
\item
The packages \href{http://ctan.org/pkg/subdocs}{\textsf{subdocs}}
and \href{http://ctan.org/pkg/subfiles}{\textsf{subfiles}}
provide structures in which the main and child documents can be
encapsulated and allowing them to be compiled individually.
The inclusion mechanism is different from the conventional |\include|.
\item
The package \href{http://ctan.org/pkg/combine}{\textsf{combine}}
is an elaborate solution to combine several documents into one.
\end{itemize}
%
See also the CTAN topic \href{http://ctan.org/topic/subdocs}{\textsf{subdocs}}
for further related packages.
The present package differs from the above solutions in that
a document structure constructed with the conventional |\include| mechanism
just needs two extra commands at the top of every file
such that all constituent files can be compiled individually.

%%%%%%%%%%%%%%%%%%%%%%%%%%%%%%%%%%%%%%%%%%%%%%%%%%%%%%%%%%%%%%%%%%%%%%%%%%%%%%%%
%\subsection{Feature Suggestions}
%
%The following is a list of features which may be useful for future
%versions of this package:
%%
%\begin{itemize}
%\item
%\ldots
%\end{itemize}

%%%%%%%%%%%%%%%%%%%%%%%%%%%%%%%%%%%%%%%%%%%%%%%%%%%%%%%%%%%%%%%%%%%%%%%%%%%%%%%%
\subsection{Revision History}

%%%%%%%%%%%%%%%%%%%%%%%%%%%%%%%%%%%%%%%%
\paragraph{v2.0:} 2018/12/30

\begin{itemize}
\item
immediate forward processing
\item
added |\childdocby| mechanism
\item
manual restructured
\end{itemize}

%%%%%%%%%%%%%%%%%%%%%%%%%%%%%%%%%%%%%%%%
\paragraph{v1.6:} 2018/01/17

\begin{itemize}
\item
application for development of include files
\item
corrections to manual
\end{itemize}

%%%%%%%%%%%%%%%%%%%%%%%%%%%%%%%%%%%%%%%%
\paragraph{v1.5:} 2017/05/21

\begin{itemize}
\item
more complete structuring introduced
\item
|\childdocof| introduced
\item
|\childdoc| renamed to |\childdocmain|
\item
|\childredirect| renamed to |\childdocforward| and |\childdocforwardprefix|
and functionality expanded
\end{itemize}

%%%%%%%%%%%%%%%%%%%%%%%%%%%%%%%%%%%%%%%%
\paragraph{v1.0:} 2017/04/27

\begin{itemize}
\item
manual and install package
\item
first version published on CTAN
\end{itemize}

%%%%%%%%%%%%%%%%%%%%%%%%%%%%%%%%%%%%%%%%
\paragraph{v0.6:} 2017/04/26

\begin{itemize}
\item
redirection mechanism added
\end{itemize}

%%%%%%%%%%%%%%%%%%%%%%%%%%%%%%%%%%%%%%%%
\paragraph{v0.5:} 2017/04/26

\begin{itemize}
\item
functionality in definition file
\end{itemize}


%%%%%%%%%%%%%%%%%%%%%%%%%%%%%%%%%%%%%%%%%%%%%%%%%%%%%%%%%%%%%%%%%%%%%%%%%%%%%%%%
%%%%%%%%%%%%%%%%%%%%%%%%%%%%%%%%%%%%%%%%%%%%%%%%%%%%%%%%%%%%%%%%%%%%%%%%%%%%%%%%
%%%%%%%%%%%%%%%%%%%%%%%%%%%%%%%%%%%%%%%%%%%%%%%%%%%%%%%%%%%%%%%%%%%%%%%%%%%%%%%%
\appendix

\settowidth\MacroIndent{\rmfamily\scriptsize 000\ }

 \DocInput{childdoc.dtx}

\end{document}
%</driver>
% \fi
%
% %%%%%%%%%%%%%%%%%%%%%%%%%%%%%%%%%%%%%%%%%%%%%%%%%%%%%%%%%%%%%%%%%%%%%%%%%%%%%%
% %%%%%%%%%%%%%%%%%%%%%%%%%%%%%%%%%%%%%%%%%%%%%%%%%%%%%%%%%%%%%%%%%%%%%%%%%%%%%%
% \section{Sample}
%\iffalse
%<*samplemain>
%\fi
%
% The following presents a sample document
% with two chapters, two parts, a title page,
% a compile flag as well as three forwarding files to set the flag.
% It consists of eight |.tex| files:
% \begin{center}
% \begin{tabular}{ll}
% |cdocsamp.tex|&main file\\
% |cdocsch1.tex|&include file for chapter 1\\
% |cdocsch2.tex|&include file for chapter 2\\
% |cdocspt3.tex|&include file for part 3\\
% |cdocspt4.tex|&include file for part 4\\
% |cdocsdrf.tex|&forwarding file for main file in draft mode\\
% |cdocsfi1.tex|&forwarding file for final version of chapter 1\\
% |cdocsfi2.tex|&forwarding file for final version of chapter 2\\
% \end{tabular}
% \end{center}
% Each of the eight files can be compiled directly by the \LaTeX{} compiler.
%
% %%%%%%%%%%%%%%%%%%%%%%%%%%%%%%%%%%%%%%
% \paragraph{Main File.}
%
% The main file is called |cdocsamp.tex|.
%
% Load the \textsf{childdoc} definitions and
% declare the filename for the main document:
%    \begin{macrocode}
\input{childdoc.def}
\childdocmain{}
%    \end{macrocode}

% Optional override for |\version| flag:
%    \begin{macrocode}
%%\ifchilddoc\else\providecommand{\version}{draft}\fi
%    \end{macrocode}

% Define the default values for the |\version| flag
% (|final| for the main file and |draft| for childs):
%    \begin{macrocode}
\ifchilddoc
\providecommand{\version}{draft}
\else
\providecommand{\version}{final}
\fi
%    \end{macrocode}

% Load the standard document class:
%    \begin{macrocode}
\documentclass[12pt]{article}
%    \end{macrocode}

% Start the document body:
%    \begin{macrocode}
\begin{document}
%    \end{macrocode}

% Declare a title page.
% Print title, part of document being processed and version flag:
%    \begin{macrocode}
\addtocounter{page}{-1}
\begin{center}
{\LARGE\bfseries{}childdoc example\par}
\vspace{1cm}
\ifchilddoc
\ifchilddocmanual part\else chapter\fi:
`\childdocname' of `\childdocjob'\par
\else
main document: `\childdocjob'\par
\fi
version: \version\par
\end{center}
\newpage
%    \end{macrocode}

% Manually include selected file,
% otherwise process as usual:
%    \begin{macrocode}
\ifchilddocmanual
\section*{part `\childdocname'}
\input{\childdocname}
\else
%    \end{macrocode}

% Include the two chapters:
%    \begin{macrocode}
\include{cdocsch1}
\include{cdocsch2}
%    \end{macrocode}

% Include the two parts unless only chapters should be displayed:
%    \begin{macrocode}
\ifchilddoc\else
\section{part three}
\input{cdocspt3}
\section{part four}
\input{cdocspt4}
\fi
%    \end{macrocode}

% Process as usual until here:
%    \begin{macrocode}
\fi
%    \end{macrocode}

% End of document body:
%    \begin{macrocode}
\end{document}
%    \end{macrocode}
%\iffalse
%</samplemain>
%\fi
%
% %%%%%%%%%%%%%%%%%%%%%%%%%%%%%%%%%%%%%%
% \paragraph{Chapter Include Files.}
%
% The include files are called |cdocsch1.tex| and |cdocsch2.tex|.
%
%\iffalse
%<*samplechap1|samplechap2>
%\fi

% Optional override for |\version| flag:
%    \begin{macrocode}
%%\providecommand{\version}{final}
%    \end{macrocode}

% Include the main document:
%    \begin{macrocode}
\input{childdoc.def}
\childdocof{cdocsamp}
%    \end{macrocode}

%\iffalse
%</samplechap1|samplechap2>
%\fi
%
%\iffalse
%<*samplechap1>
%\fi
% Some text for chapter 1:
%    \begin{macrocode}
\section{one}
some text in chapter one
%    \end{macrocode}

%\iffalse
%</samplechap1>
%\fi
% Some text for chapter 2:
%\iffalse
%<*samplechap2>
%\fi
%    \begin{macrocode}
\section{two}
more text in chapter two
%    \end{macrocode}

%\iffalse
%</samplechap2>
%\fi
%
% %%%%%%%%%%%%%%%%%%%%%%%%%%%%%%%%%%%%%%
% \paragraph{Part Include Files.}
%
% The include files are called |cdocspt3.tex| and |cdocspt4.tex|.
%
%\iffalse
%<*samplepart3|samplepart4>
%\fi

% Optional override for |\version| flag:
%    \begin{macrocode}
%%\providecommand{\version}{final}
%    \end{macrocode}

% Include the main document:
%    \begin{macrocode}
\input{childdoc.def}
\childdocby{cdocsamp}
%    \end{macrocode}

%\iffalse
%</samplepart3|samplepart4>
%\fi
%
%\iffalse
%<*samplepart3>
%\fi
% Some text for part 3:
%    \begin{macrocode}
some text in part three
%    \end{macrocode}

%\iffalse
%</samplepart3>
%\fi
% Some text for part 4:
%\iffalse
%<*samplepart4>
%\fi
%    \begin{macrocode}
more text in part four
%    \end{macrocode}

%\iffalse
%</samplepart4>
%\fi
%
% %%%%%%%%%%%%%%%%%%%%%%%%%%%%%%%%%%%%%%
% \paragraph{Forwarding for a Complete Draft.}
%
% The following forwarding file |cdocsdrf.tex|
% compiles the main document in draft mode:
%\iffalse
%<*sampledraft>
%\fi
%    \begin{macrocode}
\def\version{draft}
\input{childdoc.def}
\childdocforward{cdocsamp}
%    \end{macrocode}

%\iffalse
%</sampledraft>
%\fi
%
% %%%%%%%%%%%%%%%%%%%%%%%%%%%%%%%%%%%%%%
% \paragraph{Forwarding for Final Version of the Chapters.}
%
% The following forwarding files |cdocsfn1.tex| and |cdocsfn2.tex|
% (with identical content)
% compile the final versions of the child documents
% |cdocsch1.tex| and |cdocsch2.tex|, respectively:
%\iffalse
%<*samplefinal>
%\fi
%    \begin{macrocode}
\def\version{final}
\input{childdoc.def}
\childdocforwardprefix[cdocsamp]{cdocsfn}{cdocsch}
%    \end{macrocode}

%\iffalse
%</samplefinal>
%\fi
%
% %%%%%%%%%%%%%%%%%%%%%%%%%%%%%%%%%%%%%%
% \paragraph{Command Line Processing.}
%
% The following three command lines generate the output files
% |cdocscld|, |cdocscl1| and |cdocscl2|
% which should be identical to
% |cdocsdrf|, |cdocsch1| and |cdocsfn2|, respectively:
% \begin{center}
% \begin{tabular}{l}
% |latex -jobname cdocscld \|\\
% |  "\def\version{draft}\input{childdoc.def}\childdocforward{cdocsamp}"|\\
% |latex -jobname cdocscl1 \|\\
% |  "\input{childdoc.def}\childdocforward[cdocsamp]{cdocsch1}"|\\
% |latex -jobname cdocscl2 \|\\
% |  "\def\version{final}\input{childdoc.def}\childdocforward{cdocsch2}"|
% \end{tabular}
% \end{center}
% Note that the trailing backslash on each first line
% merely continues the input to the second line
% (for convenient cut ant paste).
% Furthermore, the command |latex| can be replaced by any
% of its alternative versions such as |pdflatex|.
%
% %%%%%%%%%%%%%%%%%%%%%%%%%%%%%%%%%%%%%%%%%%%%%%%%%%%%%%%%%%%%%%%%%%%%%%%%%%%%%%
% %%%%%%%%%%%%%%%%%%%%%%%%%%%%%%%%%%%%%%%%%%%%%%%%%%%%%%%%%%%%%%%%%%%%%%%%%%%%%%
% \section{Implementation}
%\iffalse
%<*package>
%\fi
%
% This section describes the definitions file |childdoc.def|.

% The definitions cannot be loaded using |\usepackage| or |\RequirePackage|
% which has a mechanism to prevent loading a style file more than once.
% When loading the definitions by means of |\input|
% multiple instances have to be prevented manually:
%\iffalse
%This code needs to be before the `\ProvidesFile' directive
%which is defined at the beginning of this file.
%Therefore it is also placed there and commented out here.
%</package>
%<*discard>
%\fi
%    \begin{macrocode}
\ifdefined\childdocmain\endinput\fi
%    \end{macrocode}
%\iffalse
%</discard>
%<*package>
%\fi
%
% \macro{\ifchilddoc}
% \macro{\ifchilddocmanual}
% The conditional |\ifchilddoc| tells whether a
% child (true) or main (false) document is being compiled.
% The conditional |\ifchilddocmanual| tells whether
% the |\includeonly| mechanism is used (false) or
% the selection of child files must be performed manually (true).
% The definitions initialise to false:
%    \begin{macrocode}
\newif\ifchilddoc
\newif\ifchilddocmanual
%    \end{macrocode}

% \macro{\childdocname}
% \macro{\childdocjob}
% The macro |\childdocname| stores the name of the main document
% to be compiled. The macro |\childdocjob| stores the name of
% the document on which the \LaTeX{} compiler was originally invoked.
% The content of |\jobname| cannot be compared
% to filenames specified in the source due to different catcodes.
% The following code rescans |\jobname|, stores the result
% in |\childdocname| and saves a copy in |\childdocjob|:
%    \begin{macrocode}
\edef\childdocname{\scantokens\expandafter{\jobname\noexpand}}
\let\childdocjob\childdocname
%    \end{macrocode}

% \macro{\childdocdisable}
% The macro |\childdocdisable| prevents the main file
% from being processed more than once.
% At this stage, the main document command |\childdocmain|
% is assumed to be called once again where it should do nothing.
% Any subsequent call to it should prevent
% a secondary processing of the main document
% It overwrites the forwarding commands
% |\childdocof| and |\childdocforward|
% with empty macros to prevent further inclusions of the main document:
%    \begin{macrocode}
\newcommand{\childdocdisable}
{
  \renewcommand{\childdocmain}[1]{\renewcommand{\childdocmain}[1]{\endinput}}
  \renewcommand{\childdocof}[1]{}
  \renewcommand{\childdocby}[2][]{}
  \renewcommand{\childdocforward}[2][]{}
  \renewcommand{\childdocdisable}{}
}
%    \end{macrocode}

% \macro{\childdocmain}
% The macro |\childdocmain| is to be called at the top of the main file
% with nothing or the main filename (without extension) as argument.
% First, it breaks loops.
% If the argument is not empty and does not match |\childdocname|
% (which is set by the first inclusion of |childdoc.def|),
% |\ifchilddoc| is set to true, |\includeonly| is applied to the child file
% and |\jobname| is set to the main file
% (for proper handling of |.aux| files):
%    \begin{macrocode}
\newcommand{\childdocmain}[1]
{
  \childdocdisable\childdocmain{}
  \if?#1?\else
    \begingroup
      \def\childdoctmp{#1}
      \ifx\childdoctmp\childdocname
        \def\childdoctmp{}
      \else
        \def\childdoctmp
        {
          \childdoctrue
          \includeonly{\childdocname}
          \def\childdocjob{#1}
          \def\jobname{#1}
        }
      \fi
      \expandafter
    \endgroup
    \childdoctmp
  \fi
}
%    \end{macrocode}

% \macro{\childdocof}
% The command |\childdocof| redirects
% compilation to the main file |#1|.
%    \begin{macrocode}
\newcommand{\childdocof}[1]
{
  \childdocdisable
  \childdoctrue
  \includeonly{\childdocname}
  \def\jobname{#1}
  \def\childdocjob{#1}
  \input{#1}
}
%    \end{macrocode}

% \macro{\childdocby}
% The command |\childdocby| ....
%    \begin{macrocode}
\newcommand{\childdocby}[2][]
{
  \childdocdisable
  \childdoctrue
  \childdocmanualtrue
  \if?#1?\else
    \def\jobname{#2}
  \fi
  \def\childdocjob{#2}
  \input{#2}
  \endinput
}
%    \end{macrocode}

% \macro{\childdocforward}
% The command |\childdocforward| redirects
% compilation to the main file or
% (if the optional argument is given) a child file.
% Parameters are set as if the main file
% or a child file starting with |\childdocof| was compiled.
% Then compilation is handed over to the main file:
%    \begin{macrocode}
\newcommand{\childdocforward}[2][]
{
  \begingroup
    \if?#1?
      \def\childdoctmp
      {
        \def\childdocname{#2}
        \def\childdocjob{#2}
        \def\jobname{#2}
        \input{#2}
        \endinput
      }
    \else
      \def\childdoctmp
      {
        \childdocdisable
        \def\childdocname{#2}
        \childdoctrue
        \includeonly{#2}
        \def\childdocjob{#1}
        \def\jobname{#1}
        \input{#1}
        \endinput
      }
    \fi
    \expandafter
  \endgroup
  \childdoctmp
}
%    \end{macrocode}

% \macro{\childdocforwardprefix}
% The command |\childdocforwardprefix| redirects
% compilation to the main or a child file by means of a pattern.
% The prefix |#1| in the current filename is replaced by |#2|
% and the suffix of the current filename is kept
% (it is assumed that the filename does not contain the substring `|~~~|'
% which is used as a delimiter).
% Compilation is handed over to the new file by |\childdocforward|:
%    \begin{macrocode}
\newcommand{\childdocforwardprefix}[3][]
{
  \begingroup
    \def\childdocextract #2##1~~~{\def\childdoctmp{\childdocforward[#1]{#3##1}}}
    \expandafter\childdocextract\childdocname~~~
    \expandafter
  \endgroup
  \childdoctmp
}
%    \end{macrocode}

% \macro{\childdoc}
% The deprecated macro |\childdoc| is a legacy version of |\childdocmain|:
%    \begin{macrocode}
\newcommand{\childdoc}{\childdocmain}
%    \end{macrocode}

% \macro{\childdocredirect}
% The deprecated macro |\childdocredirect| is a legacy version
% of |\childdocforward| and |\childdocforwardprefix|:
%    \begin{macrocode}
\newcommand{\childdocredirect}[2][]
{
  \begingroup
    \if?#1?
      \def\childdoctmp{\childdocforward{#2}}
    \else
      \def\childdoctmp{\childdocforwardprefix{#1}{#2}}
    \fi
    \expandafter
  \endgroup
  \childdoctmp
}
%    \end{macrocode}

%\iffalse
%</package>
%\fi
%
\endinput

\childdocforward{cdocsamp}
%    \end{macrocode}

%\iffalse
%</sampledraft>
%\fi
%
% %%%%%%%%%%%%%%%%%%%%%%%%%%%%%%%%%%%%%%
% \paragraph{Forwarding for Final Version of the Chapters.}
%
% The following forwarding files |cdocsfn1.tex| and |cdocsfn2.tex|
% (with identical content)
% compile the final versions of the child documents
% |cdocsch1.tex| and |cdocsch2.tex|, respectively:
%\iffalse
%<*samplefinal>
%\fi
%    \begin{macrocode}
\def\version{final}
% \iffalse
%
% childdoc.dtx Copyright (C) 2017-2018 Niklas Beisert
%
% This work may be distributed and/or modified under the
% conditions of the LaTeX Project Public License, either version 1.3
% of this license or (at your option) any later version.
% The latest version of this license is in
%   http://www.latex-project.org/lppl.txt
% and version 1.3 or later is part of all distributions of LaTeX
% version 2005/12/01 or later.
%
% This work has the LPPL maintenance status `maintained'.
%
% The Current Maintainer of this work is Niklas Beisert.
%
% This work consists of the files childdoc.dtx and childdoc.ins
% and the derived files childdoc.def and cdocsamp.tex with
% cdocsch1.tex, cdocsch2.tex, cdocsdrf.tex, cdocsfn1.tex, cdocsfn2.tex.
%
%<package>\ifdefined\childdocmain\endinput\fi
%<package>\ProvidesFile{childdoc.def}[2018/12/30 v2.0 child document driver]
%<samplemain>\ProvidesFile{cdocsamp.tex}[2018/12/30 v2.0 sample for childdoc]
%<*driver>
%\ProvidesFile{childdoc.drv}[2018/12/30 v2.0 childdoc reference manual file]
\PassOptionsToClass{10pt,a4paper}{article}
\documentclass{ltxdoc}

\usepackage[margin=35mm]{geometry}
\usepackage{hyperref}
\usepackage{hyperxmp}
\usepackage[usenames]{color}

\hypersetup{colorlinks=true}
\hypersetup{pdfstartview=FitH}
\hypersetup{pdfpagemode=UseNone}
\hypersetup{pdfsource={}}
\hypersetup{pdflang={en-UK}}
\hypersetup{pdfcopyright={Copyright 2017-2018 Niklas Beisert.
  This work may be distributed and/or modified under the
  conditions of the LaTeX Project Public License, either version 1.3
  of this license or (at your option) any later version.}}
\hypersetup{pdflicenseurl={http://www.latex-project.org/lppl.txt}}
\hypersetup{pdfcontactaddress={ETH Zurich, ITP, HIT K,
  Wolfgang-Pauli-Strasse 27}}
\hypersetup{pdfcontactpostcode={8093}}
\hypersetup{pdfcontactcity={Zurich}}
\hypersetup{pdfcontactcountry={Switzerland}}
\hypersetup{pdfcontactemail={nbeisert@itp.phys.ethz.ch}}
\hypersetup{pdfcontacturl={http://people.phys.ethz.ch/\xmptilde nbeisert/}}

\newcommand{\secref}[1]{\hyperref[#1]{section \ref*{#1}}}

\parskip1ex
\parindent0pt
\let\olditemize\itemize
\def\itemize{\olditemize\parskip0pt}

\begin{document}

\title{The \textsf{childdoc} Package}
\hypersetup{pdftitle={The childdoc Package}}
\author{Niklas Beisert\\[2ex]
  Institut f\"ur Theoretische Physik\\
  Eidgen\"ossische Technische Hochschule Z\"urich\\
  Wolfgang-Pauli-Strasse 27, 8093 Z\"urich, Switzerland\\[1ex]
  \href{mailto:nbeisert@itp.phys.ethz.ch}
  {\texttt{nbeisert@itp.phys.ethz.ch}}}
\hypersetup{pdfauthor={Niklas Beisert}}
\hypersetup{pdfsubject={Manual for the LaTeX2e Package childdoc}}
\date{30 December 2018, \textsf{v2.0}}
\maketitle

\begin{abstract}\noindent
\textsf{childdoc} is a \LaTeXe{} package
that enables the direct compilation
of document sections included by |\include|
to individual files.
\end{abstract}

\begingroup
\parskip0ex
\tableofcontents
\endgroup

%%%%%%%%%%%%%%%%%%%%%%%%%%%%%%%%%%%%%%%%%%%%%%%%%%%%%%%%%%%%%%%%%%%%%%%%%%%%%%%%
%%%%%%%%%%%%%%%%%%%%%%%%%%%%%%%%%%%%%%%%%%%%%%%%%%%%%%%%%%%%%%%%%%%%%%%%%%%%%%%%
\section{Introduction}

\LaTeX{} provides a mechanism to structure a large document (such as a book)
into a main file and several child files (containing the chapters)
using the |\include| command.
This mechanism is beneficial for documents
which span hundreds of pages in order to
make the source file(s) more manageable.
Moreover, compilation can be restricted to
selected child files by means of the |\includeonly| command.
The latter feature can be used to reduce the compilation time while editing
(this was significantly more useful in the earlier days of \LaTeX{})
or to generate a smaller document which is easier to navigate.
Another application of |\includeonly| is to generate
documents consisting of selected parts of the complete document.

However, there are a few drawbacks of the plain |\include| mechanism:
\begin{itemize}
\item
The child files cannot be compiled on their own,
they can only be compiled via the main file.
A naive editing environment
(such as a text editor with an option
to have the current file processed by \LaTeX)
may require one to switch to the main file before compiling;
attempting to compile the child file produces errors.
\item
The main file must be modified (each time)
to adjust the |\includeonly| command
to the present needs. This easily leaves the main file in a messy state.
\item
The generated document will always carry the filename
of the main document. This is inconvenient if
several child files are to be compiled and
to be kept for distribution.
\end{itemize}

The present package provides a simple interface
to make child files individually compilable by \LaTeX{}.
Compiling a child file then has the same effect as compiling
the main file with an |\includeonly| command
to select the appropriate child.
Moreover the generated document will carry the name of the child
rather than the main file.
This resolves all three above issues.

This feature is meant to make the editing of books,
thesis documents and lecture notes somewhat more convenient.
However, the package can also be used efficiently for
composing a series of documents (such as exercise sheets)
which are typically distributed individually.
It then assists the author in generating the individual documents
(potentially in different versions)
as well as a document containing the collected series.
Another application is in developing style files
or other kinds of included material
where compilation of the style file could redirect
to a sample or test file.

%%%%%%%%%%%%%%%%%%%%%%%%%%%%%%%%%%%%%%%%%%%%%%%%%%%%%%%%%%%%%%%%%%%%%%%%%%%%%%%%
%%%%%%%%%%%%%%%%%%%%%%%%%%%%%%%%%%%%%%%%%%%%%%%%%%%%%%%%%%%%%%%%%%%%%%%%%%%%%%%%
\section{Usage}

First of all, the package \textsf{childdoc} is \emph{not} a standard
\LaTeXe{} |.sty| style file! Therefore it needs to be invoked in
a non-standard way.

%%%%%%%%%%%%%%%%%%%%%%%%%%%%%%%%%%%%%%%%%%%%%%%%%%%%%%%%%%%%%%%%%%%%%%%%%%%%%%%%
\subsection{Included Files}
\label{sec:include}

%%%%%%%%%%%%%%%%%%%%%%%%%%%%%%%%%%%%%%%%
\DescribeMacro{\childdocmain}
To use the package, add the commands
\begin{center}
\begin{tabular}{l}
|\input{childdoc.def}|\\
|\childdocmain{}|\\
\end{tabular}
\end{center}
at the very top of the main \LaTeX{} file,
in particular \emph{before} the |\documentclass| statement!
The argument of |\childdocmain| should be left empty
(but it must be present).

%%%%%%%%%%%%%%%%%%%%%%%%%%%%%%%%%%%%%%%%
\DescribeMacro{\childdocof}
Furthermore, add the commands
\begin{center}
\begin{tabular}{l}
|\input{childdoc.def}|\\
|\childdocof{|\textit{main}|}|\\
\end{tabular}
\end{center}
at the top of every child file \textit{child}
which is included by |\include{|\textit{child}|}|
from within the main file
(or at least for those files to be compiled individually).
The argument \textit{main} must be the filename of the main file.

There are a couple of
considerations in setting up the main and child documents:

%%%%%%%%%%%%%%%%%%%%%%%%%%%%%%%%%%%%%%%%
\paragraph{Restrictions.}

Please note the following restrictions:
\begin{itemize}
\item
|\childdocmain| must be called with one argument \textit{main}
to ensure compatibility with earlier version of the package.
It must either be empty (|\childdocmain{}|)
or precisely match the filename of the main file in which it is specified.
See \secref{sec:detection} for further information.
\item
The filename \textit{main} must be specified without the |.tex| extension.
\item
The filename \textit{main} is case sensitive
(even in case-insensitive file systems)
due to internal string comparison.
\item
The argument \textit{main} should be fully expanded, it cannot be a macro.
\item
Subdirectories and special characters should be avoided in filenames.
\item
The command |\childdocmain{|\textit{main}|}| must be followed by a whitespace.
It should not be followed immediately by another command
or by a comment mark `|%|'.
This is because the \TeX{} parser reads the token immediately following
the argument of |\childdocmain| and puts it
at the beginning of every child section;
however, a white\-space is ignored.
\end{itemize}

%%%%%%%%%%%%%%%%%%%%%%%%%%%%%%%%%%%%%%%%
\paragraph{Content of Main File.}

It is advisable to place all content in the child files included by |\include|.
Any output contained in the main file will appear in all child documents
unless suppressed manually;
it cannot be suppressed automatically by the |\includeonly| directive
and thus should normally be avoided.
A method to include some content in the main file
by means of conditional processing is described in \secref{sec:conditional}.

%%%%%%%%%%%%%%%%%%%%%%%%%%%%%%%%%%%%%%%%
\paragraph{Page Numbering.}

When only a part of the document is compiled,
the appropriate numbering of pages
(as well as other status parameters)
is determined from the |.aux| files.
The latter contain information from previous passes.
However this information needs to propagate through
all intermediate child documents.
Therefore the page numbering in child documents may well
be inconsistent until the complete document is compiled at least once.

A useful (if unconventional) way to always ensure a consistent
page numbering is to restart the numbering in each child document
and denote the pages by `\textit{child}|.|\textit{page}'
where \textit{child} represents the chapter/section number of the child file.
This can be achieved by the command
|\numberwithin{page}{|\textit{child}|}|
of the \textsf{amsmath} package
where \textit{child} can be |chapter| or |section|
depending on the chosen structuring.
Alternatively, one can modify the macro |\thepage| appropriately
and reset the counter |page| at the start of each child file.

%%%%%%%%%%%%%%%%%%%%%%%%%%%%%%%%%%%%%%%%%%%%%%%%%%%%%%%%%%%%%%%%%%%%%%%%%%%%%%%%
\subsection{Conditional Processing}
\label{sec:conditional}

The package provides a mechanism to compile different versions
of a document. To customise the versions further some conditional processing
can come in handy to distinguish which version is being compiled.
The package provides two macros to describe the compilation context:

%%%%%%%%%%%%%%%%%%%%%%%%%%%%%%%%%%%%%%%%
\DescribeMacro{\ifchilddoc}
The conditional |\ifchilddoc| distinguishes between the compilation of
child documents and the main document:
%
\begin{center}
|\ifchilddoc |\textit{child-code}| |[|\||else |\textit{main-code}]| \||fi|
\end{center}

%%%%%%%%%%%%%%%%%%%%%%%%%%%%%%%%%%%%%%%%
\DescribeMacro{\childdocname}
\DescribeMacro{\childdocjob}
The macro |\childdocname| contains the filename (without extension)
of the main or child file being processed.
Note that |\childdocjob| will always contain the name of the main file.

%%%%%%%%%%%%%%%%%%%%%%%%%%%%%%%%%%%%%%%%
\paragraph{Title Page.}

Conditional processing can be used to include a title or banner page
in the main document when proper precautions are taken.
Importantly, the code in the main file should ensure that the page counter
(as well as other status parameters which are stored in the |.aux| files)
takes the same value after the conditional processing.
Otherwise the page numbers may take divergent values
depending on which part is compiled.

For example, a title page could be declared by:
%
\begin{center}
\begin{tabular}{l}
|\ifchilddoc\||else|\\
|\addtocounter{page}{-1}|\\
\textit{code for title page}\\
|\newpage|\\
|\||fi|
\end{tabular}
\end{center}
%
A banner page for the child documents can be generated by:
%
\begin{center}
\begin{tabular}{l}
|\ifchilddoc|\\
|\addtocounter{page}{-1}|\\
\textit{code for banner page}\\
|\newpage|\\
|\||fi|
\end{tabular}
\end{center}
%
Here one could write a message such as:
\begin{center}
|This is the part \childdocname{} of \childdocjob{}.|
\end{center}

%%%%%%%%%%%%%%%%%%%%%%%%%%%%%%%%%%%%%%%%%%%%%%%%%%%%%%%%%%%%%%%%%%%%%%%%%%%%%%%%
\subsection{Flags}
\label{sec:flags}

The package makes it easy to generate different versions
of the main or child documents.
To this end compilation flags can be defined
and assigned different default values.
They will be particularly useful in conjunction
with the forwarding mechanism described in \secref{sec:forward}.

For example, it may be useful to have a flag |\version|
which can be set to |draft| or |final|.
The document source will contain some conditional code
depending on the value of |\version|.
Suppose further, the flag should default to |final| for the main file
and to |draft| for child files
which is a natural assignment for editing the document.
This is achieved by placing the following code
in the preamble of the main document
(below the |\childdocmain| directive):
%
\begin{center}
\begin{tabular}{l}
|\ifchilddoc|\\
|\providecommand{\version}{draft}|\\
|\||else|\\
|\providecommand{\version}{final}|\\
|\||fi|
\end{tabular}
\end{center}
%
The definition by |\providecommand| makes sure
that previous definitions are not overwritten.
Further statements |\providecommand{\version}{...}|
can thus be added before the above code to override it.

For the main file, one might add a line
(between |\childdocmain| and the above block)
%
\begin{center}
|%\ifchilddoc\||else\providecommand{\version}{draft}\||fi|
\end{center}
%
which can be uncommented to produce a draft version.
Likewise one can add a line to the very top of a child file
(above the |\childdocof{|\textit{main}|}| directive)
%
\begin{center}
|%\providecommand{\version}{final}|
\end{center}
%
which can be uncommented to produce the final version of this child document.

%%%%%%%%%%%%%%%%%%%%%%%%%%%%%%%%%%%%%%%%%%%%%%%%%%%%%%%%%%%%%%%%%%%%%%%%%%%%%%%%
\subsection{Forwarding}
\label{sec:forward}

Different versions of the main or child documents
using compilation flags as described in \secref{sec:flags}
can be (permanently) stored in different files
for convenient compilation, viewing and distribution.
To this end, the package defines a command
to pass on compilation to a different file:

%%%%%%%%%%%%%%%%%%%%%%%%%%%%%%%%%%%%%%%%
\DescribeMacro{\childdocforward}
The command |\childdocforward| redirects processing to
another source file:
%
\begin{center}
\begin{tabular}{l}
|\input{childdoc.def}|\\
|\childdocforward[|\textit{main}|]{|\textit{dest}|}|\\
\end{tabular}
\end{center}
%
The argument \textit{dest} is the destination file
(without extension).
It should be the main file or one of the child files.
Note that further \textsf{childdoc} directives
such as |\childdocof| and |\childdocforward|
in the indicated file will be processed in this form.
The optional argument \textit{main}
passes on directly to the main file \textit{main}
while pretending to compile the child \textit{dest}.
This form behaves as if \textit{dest}
issues |\childdocof{|\textit{main}|}| right away,
and no further \textsf{childdoc} directives will be processed.

%%%%%%%%%%%%%%%%%%%%%%%%%%%%%%%%%%%%%%%%
\DescribeMacro{\...prefix}
In the alternative form |\childdocforwardprefix|,
%
\begin{center}
\begin{tabular}{l}
|\input{childdoc.def}|\\
|\childdocforwardprefix[|\textit{main}|]{|\textit{prefix}|}{|\textit{dest}|}|
\end{tabular}
\end{center}
%
the destination file is determined by a pattern
depending on the current file:
To make this work, the current file must be called
`{\textit{prefix}\hspace{0.2em}\textit{suffix}}'
with \textit{prefix} matching precisely the argument.
Processing is then passed on to the file
`{\textit{dest}\hspace{0.2em}\textit{suffix}}'.
Surely, the same effect is achieved by
directly specifying the
argument `{\textit{dest}\hspace{0.2em}\textit{suffix}}'
in the first form.
However, that requires to set up a different file
for each child. With the alternative form of the command
all these files can have exactly the same content
which simplifies setting them up and maintaining them.

For example, the following file |draft.tex|
with a compilation flag |\version| as described in \secref{sec:flags}
compiles the main document as a draft:
%
\begin{center}
\begin{tabular}{l}
|\def\version{draft}|\\
|\input{childdoc.def}|\\
|\childdocforward{|\textit{main}|}|
\end{tabular}
\end{center}
%
Likewise, the following files |final|\textit{nn}|.tex|
compile the final version of the child document
|child|\textit{nn}|.tex|:
%
\begin{center}
\begin{tabular}{l}
|\def\version{final}|\\
|\input{childdoc.def}|\\
|\childdocforwardprefix{final}{child}|
\end{tabular}
\end{center}
%

Note that when several versions of a main file and/or of each child file
are to be generated, it may be convenient to set up a |Makefile| or
shell script to automatise the process.

%%%%%%%%%%%%%%%%%%%%%%%%%%%%%%%%%%%%%%%%%%%%%%%%%%%%%%%%%%%%%%%%%%%%%%%%%%%%%%%%
\subsection{Command Line Processing}
\label{sec:commandline}

The effect of redirection files can also be achieved by invoking
the \LaTeX{} compiler with a more elaborate command line.
Most conveniently this should be done as part
of a shell script or a |Makefile|.

When using \textsf{childdoc} in the main file, the following
command lines effectively perform a redirection
(note that depending on the shell being used,
backslashes may have to be doubled: `|\|' $\to$ `|\\|'):
%
\begin{center}
|... -jobname "|\textit{target}|" |\\|"|[\textit{flags}]%
|\input{childdoc.def}\childdocforward[|\textit{main}|]{|\textit{dest}|}"|
\end{center}
%
Here \textit{target} is the name of the output file,
\textit{main} is the name of the main file
and \textit{dest} is the name of the main or child file to be processed
(all filenames without extensions).
The optional argument \textit{main} can be omitted
if \textit{main} matches \textit{dest}.
Optionally, compilation \textit{flags} can be defined via |\def| commands.
This command line makes the \TeX{} engine believe
it is compiling the file \textit{target}
whose content is specified as the latter parameter.
The provided code then forwards the processing to
\textit{main} or \textit{dest} as described in \secref{sec:forward}.

%%%%%%%%%%%%%%%%%%%%%%%%%%%%%%%%%%%%%%%%%%%%%%%%%%%%%%%%%%%%%%%%%%%%%%%%%%%%%%%%
\subsection{Include by Input}
\label{sec:input}

Including child documents by |\include| has some restrictions by design.
Most notably, the content of a child document always occupies
its own set of pages; pages cannot be shared between child documents.
Usually, this behaviour makes perfect sense
because each child document contain an essential part of the document.
However, in some situations it may be desirable to compose
a document from a collection of parts
without having mandatory page breaks between then.
For this case, the package
provides a mechanism to include parts
by |\input| which can also be processed individually.
However, by construction this mechanism
requires manual handling of the content to be output.

%%%%%%%%%%%%%%%%%%%%%%%%%%%%%%%%%%%%%%%%
\DescribeMacro{\ifchilddocmanual}
The main file should be prepared as usual, see \secref{sec:include}.
However, the document body must make a distinction
between processing of an individual part and of the main document, e.g.:
%
\begin{center}
\begin{tabular}{l}
|\ifchilddocmanual|\\
|\input{\childdocname}|\\
|\||else|\\
\textit{document body with }|\input{|\textit{part}|}|\\
|\||fi|
\end{tabular}
\end{center}
%
The conditional |\ifchilddocmanual| is true whenever
a part to be included by |\input| is being compiled,
and the name of the part is stored in |\childdocname|.

%%%%%%%%%%%%%%%%%%%%%%%%%%%%%%%%%%%%%%%%
\DescribeMacro{\childdocby}
Each part to be included by |\input| should start with:
%
\begin{center}
\begin{tabular}{l}
|\input{childdoc.def}|\\
|\childdocby{|\textit{main}|}|\\
\end{tabular}
\end{center}
%
The directive |\childdocby| is similar to |\childdocof|
described in \secref{sec:include},
but the subsequent selection of content must be done manually.
To that end, both |\ifchilddoc| and |\ifchilddocmanual|
will be true upon processing of a part,
and the name of the part is stored in |\childdocname|.
Note that |\jobname| will be set to the filename of the current part
so that each part receives an individual |.aux| file
that does not interfere with the |.aux| file(s) of the main document.
This behaviour can be altered by the alternative form
|\childdocby[*]{|\textit{main}|}| (with a non-empty optional argument)
which uses the |.aux| file of the main document
by setting |\jobname| to \textit{main}.

%%%%%%%%%%%%%%%%%%%%%%%%%%%%%%%%%%%%%%%%%%%%%%%%%%%%%%%%%%%%%%%%%%%%%%%%%%%%%%%%
\subsection{Driver Development}
\label{sec:driver}

The \textsf{childdoc} mechanism can also be use for the development
of definition files such as \LaTeX{} styles or classes.
This case differs from the above setup with multiple parts
included by |\include| in that no |\includeonly| should be invoked.
This can be achieved by starting the include file
(before |\ProvidesPackage|) with:
%
\begin{center}
\begin{tabular}{l}
|\input{childdoc.def}|\\
|\childdocforward{|\textit{main}|}|\\
\end{tabular}
\end{center}
%
or alternatively with:
%
\begin{center}
\begin{tabular}{l}
|\input{childdoc.def}|\\
|\childdocby{|\textit{main}|}|\\
\end{tabular}
\end{center}
%
Both forms have slightly different effects as described above.
The main file is prepared as usual, see \secref{sec:include}.

%%%%%%%%%%%%%%%%%%%%%%%%%%%%%%%%%%%%%%%%%%%%%%%%%%%%%%%%%%%%%%%%%%%%%%%%%%%%%%%%
\subsection{Legacy Detection}
\label{sec:detection}

The directive |\childdocmain| in the main file can detect
whether the complete document or merely a child is to be compiled
even without using the directive |\childdocof|.
This method is deprecated because it is less robust
and there is no compelling reason to use it;
it is merely provided for backward compatibility
and it may be removed in future versions.

If the detection mechanism is to be used,
it is mandatory to correctly specify
the filename of the main file as the argument of |\childdocmain|:
%
\begin{center}
\begin{tabular}{l}
|\input{childdoc.def}|\\
|\childdocmain{|\textit{main}|}|\\
\end{tabular}
\end{center}
%
If |\jobname| does not match the argument \textit{main} of |\childdocmain|,
it is assumed that |\jobname| points to the child file to be compiled.
When using |\childdocmain| with the main file specified as argument,
it suffices to start a child file
with just |\input{|\textit{main}|}|
without loading of the package and using |\childdocof|.
If instead all processing is done
with the appropriate \textsf{childdoc} directives,
the argument of \textit{main} of |\childdocmain| can be empty.

An alternative version of the command line processing described
in \secref{sec:commandline} using the detection mechanism reads:
%
\begin{center}
|... -jobname "|\textit{target}|" "|[\textit{flags}]%
[|\def\jobname{|\textit{dest}|}|]|\input{|\textit{main}|}"|
\end{center}

%%%%%%%%%%%%%%%%%%%%%%%%%%%%%%%%%%%%%%%%%%%%%%%%%%%%%%%%%%%%%%%%%%%%%%%%%%%%%%%%
\subsection{Manual Code}
\label{sec:manual}

In case one cannot be certain whether the definitions file |childdoc.def|
is installed on the target \TeX{} distribution
and one prefers not to ship it,
it is conceivable to paste a few relevant commands into the sources.

To that end, drop all statements |\input{childdoc.def}|
and perform the replacements as outlined below.
Instead of |\childdocmain{|\textit{main}|}| add the following code
to the top of the main file:
%
\begin{center}
\begin{tabular}{l}
|\||ifdefined\childdocname\endinput\||fi\newif\ifchilddoc|\\
|\edef\childdocname{\scantokens\expandafter{\jobname\noexpand}}|\\
|\def\childdocmain{|\textit{main}|}\||ifx\childdocmain\childdocname\||else|\\
|\childdoctrue\includeonly{\childdocname}\let\jobname\childdocmain\||fi|\\
\end{tabular}
\end{center}
%
Instead of |\childdocof{|\textit{main}|}| just include the main file
at the top of each child file:
%
\begin{center}
|\input{|\textit{main}|}|
\end{center}
%
A simple redirection |\childdocforward{|\textit{dest}|}| is achieved by:
%
\begin{center}
|\def\jobname{|\textit{dest}|}\input{\jobname}|
\end{center}
%
The redirection with prefix
|\childdocforwardprefix[|\textit{prefix}|]{|\textit{dest}|}|
is accomplished by:
%
\begin{center}
\begin{tabular}{l}
|{\edef\jobname{\scantokens\expandafter{\jobname\noexpand}}|\\
|\def\redirectjob |\textit{prefix}|#1~~~{\gdef\jobname{|\textit{dest}|#1}}|\\
|\expandafter\redirectjob\jobname~~~}\input{\jobname}|
\end{tabular}
\end{center}

In an alternative approach,
child documents can be compiled by a specific command line
without additional code or specific definitions:
%
\begin{center}
|... -jobname "|\textit{target}|" "|[\textit{flags}]%
|\includeonly{|\textit{dest}|}\input{|\textit{main}|}"|
\end{center}
%

%%%%%%%%%%%%%%%%%%%%%%%%%%%%%%%%%%%%%%%%%%%%%%%%%%%%%%%%%%%%%%%%%%%%%%%%%%%%%%%%
%%%%%%%%%%%%%%%%%%%%%%%%%%%%%%%%%%%%%%%%%%%%%%%%%%%%%%%%%%%%%%%%%%%%%%%%%%%%%%%%
\section{Information}

%%%%%%%%%%%%%%%%%%%%%%%%%%%%%%%%%%%%%%%%%%%%%%%%%%%%%%%%%%%%%%%%%%%%%%%%%%%%%%%%
\subsection{Copyright}

Copyright \copyright{} 2017--2018 Niklas Beisert

This work may be distributed and/or modified under the
conditions of the \LaTeX{} Project Public License, either version 1.3
of this license or (at your option) any later version.
The latest version of this license is in
  \url{http://www.latex-project.org/lppl.txt}
and version 1.3 or later is part of all distributions of \LaTeX{}
version 2005/12/01 or later.

This work has the LPPL maintenance status `maintained'.

The Current Maintainer of this work is Niklas Beisert.

This work consists of the files |README.txt|, |childdoc.ins| and |childdoc.dtx|
as well as the derived files |childdoc.def|, |cdocsamp.tex|
with |cdocsch1.tex|, |cdocsch2.tex|, |cdocspt3.tex|, |cdocspt4.tex|,
|cdocsdrf.tex|, |cdocsfn1.tex|, |cdocsfn2.tex|
as well as |childdoc.pdf|.

%%%%%%%%%%%%%%%%%%%%%%%%%%%%%%%%%%%%%%%%%%%%%%%%%%%%%%%%%%%%%%%%%%%%%%%%%%%%%%%%
\subsection{Files and Installation}

The package consists of the files:
%
\begin{center}
\begin{tabular}{ll}
    |README.txt|   & readme file \\
    |childdoc.ins| & installation file \\
    |childdoc.dtx| & source file \\
    |childdoc.def| & definition file \\
    |cdocsamp.tex| & sample main file \\
    |cdocsch1.tex| & sample include file \\
    |cdocsch2.tex| & sample include file \\
    |cdocspt3.tex| & sample part file \\
    |cdocspt4.tex| & sample part file \\
    |cdocsdrf.tex| & sample redirection file \\
    |cdocsfn1.tex| & sample redirection file \\
    |cdocsfn2.tex| & sample redirection file \\
    |childdoc.pdf| & manual
\end{tabular}
\end{center}
%
The distribution consists of the files
|README.txt|, |childdoc.ins| and |childdoc.dtx|.
%
\begin{itemize}
\item
Run (pdf)\LaTeX{} on |childdoc.dtx|
to compile the manual |childdoc.pdf| (this file).
\item
Run \LaTeX{} on |childdoc.ins| to create the definitions file |childdoc.def|
and the sample |cdocsamp.tex| with include files
|cdocsch1.tex|, |cdocsch2.tex|, |cdocspt3.tex|, |cdocspt4.tex|,
|cdocsdrf.tex|, |cdocsfn1.tex|, |cdocsfn2.tex|.
Then copy the file |childdoc.def| to an appropriate directory of your \LaTeX{}
distribution, e.g.\ \textit{texmf-root}|/tex/latex/childdoc|.
\end{itemize}

%%%%%%%%%%%%%%%%%%%%%%%%%%%%%%%%%%%%%%%%%%%%%%%%%%%%%%%%%%%%%%%%%%%%%%%%%%%%%%%%
\subsection{Related CTAN Packages}

There are several other packages which offer a similar functionality:
%
\begin{itemize}
\item
The packages
\href{http://ctan.org/pkg/docmute}{\textsf{docmute}},
\href{http://ctan.org/pkg/includex}{\textsf{includex}} and
\href{http://ctan.org/pkg/standalone}{\textsf{standalone}}
provide commands to include only the document body of
a child file thus allowing both files to be compiled individually.
\item
The packages \href{http://ctan.org/pkg/subdocs}{\textsf{subdocs}}
and \href{http://ctan.org/pkg/subfiles}{\textsf{subfiles}}
provide structures in which the main and child documents can be
encapsulated and allowing them to be compiled individually.
The inclusion mechanism is different from the conventional |\include|.
\item
The package \href{http://ctan.org/pkg/combine}{\textsf{combine}}
is an elaborate solution to combine several documents into one.
\end{itemize}
%
See also the CTAN topic \href{http://ctan.org/topic/subdocs}{\textsf{subdocs}}
for further related packages.
The present package differs from the above solutions in that
a document structure constructed with the conventional |\include| mechanism
just needs two extra commands at the top of every file
such that all constituent files can be compiled individually.

%%%%%%%%%%%%%%%%%%%%%%%%%%%%%%%%%%%%%%%%%%%%%%%%%%%%%%%%%%%%%%%%%%%%%%%%%%%%%%%%
%\subsection{Feature Suggestions}
%
%The following is a list of features which may be useful for future
%versions of this package:
%%
%\begin{itemize}
%\item
%\ldots
%\end{itemize}

%%%%%%%%%%%%%%%%%%%%%%%%%%%%%%%%%%%%%%%%%%%%%%%%%%%%%%%%%%%%%%%%%%%%%%%%%%%%%%%%
\subsection{Revision History}

%%%%%%%%%%%%%%%%%%%%%%%%%%%%%%%%%%%%%%%%
\paragraph{v2.0:} 2018/12/30

\begin{itemize}
\item
immediate forward processing
\item
added |\childdocby| mechanism
\item
manual restructured
\end{itemize}

%%%%%%%%%%%%%%%%%%%%%%%%%%%%%%%%%%%%%%%%
\paragraph{v1.6:} 2018/01/17

\begin{itemize}
\item
application for development of include files
\item
corrections to manual
\end{itemize}

%%%%%%%%%%%%%%%%%%%%%%%%%%%%%%%%%%%%%%%%
\paragraph{v1.5:} 2017/05/21

\begin{itemize}
\item
more complete structuring introduced
\item
|\childdocof| introduced
\item
|\childdoc| renamed to |\childdocmain|
\item
|\childredirect| renamed to |\childdocforward| and |\childdocforwardprefix|
and functionality expanded
\end{itemize}

%%%%%%%%%%%%%%%%%%%%%%%%%%%%%%%%%%%%%%%%
\paragraph{v1.0:} 2017/04/27

\begin{itemize}
\item
manual and install package
\item
first version published on CTAN
\end{itemize}

%%%%%%%%%%%%%%%%%%%%%%%%%%%%%%%%%%%%%%%%
\paragraph{v0.6:} 2017/04/26

\begin{itemize}
\item
redirection mechanism added
\end{itemize}

%%%%%%%%%%%%%%%%%%%%%%%%%%%%%%%%%%%%%%%%
\paragraph{v0.5:} 2017/04/26

\begin{itemize}
\item
functionality in definition file
\end{itemize}


%%%%%%%%%%%%%%%%%%%%%%%%%%%%%%%%%%%%%%%%%%%%%%%%%%%%%%%%%%%%%%%%%%%%%%%%%%%%%%%%
%%%%%%%%%%%%%%%%%%%%%%%%%%%%%%%%%%%%%%%%%%%%%%%%%%%%%%%%%%%%%%%%%%%%%%%%%%%%%%%%
%%%%%%%%%%%%%%%%%%%%%%%%%%%%%%%%%%%%%%%%%%%%%%%%%%%%%%%%%%%%%%%%%%%%%%%%%%%%%%%%
\appendix

\settowidth\MacroIndent{\rmfamily\scriptsize 000\ }

 \DocInput{childdoc.dtx}

\end{document}
%</driver>
% \fi
%
% %%%%%%%%%%%%%%%%%%%%%%%%%%%%%%%%%%%%%%%%%%%%%%%%%%%%%%%%%%%%%%%%%%%%%%%%%%%%%%
% %%%%%%%%%%%%%%%%%%%%%%%%%%%%%%%%%%%%%%%%%%%%%%%%%%%%%%%%%%%%%%%%%%%%%%%%%%%%%%
% \section{Sample}
%\iffalse
%<*samplemain>
%\fi
%
% The following presents a sample document
% with two chapters, two parts, a title page,
% a compile flag as well as three forwarding files to set the flag.
% It consists of eight |.tex| files:
% \begin{center}
% \begin{tabular}{ll}
% |cdocsamp.tex|&main file\\
% |cdocsch1.tex|&include file for chapter 1\\
% |cdocsch2.tex|&include file for chapter 2\\
% |cdocspt3.tex|&include file for part 3\\
% |cdocspt4.tex|&include file for part 4\\
% |cdocsdrf.tex|&forwarding file for main file in draft mode\\
% |cdocsfi1.tex|&forwarding file for final version of chapter 1\\
% |cdocsfi2.tex|&forwarding file for final version of chapter 2\\
% \end{tabular}
% \end{center}
% Each of the eight files can be compiled directly by the \LaTeX{} compiler.
%
% %%%%%%%%%%%%%%%%%%%%%%%%%%%%%%%%%%%%%%
% \paragraph{Main File.}
%
% The main file is called |cdocsamp.tex|.
%
% Load the \textsf{childdoc} definitions and
% declare the filename for the main document:
%    \begin{macrocode}
\input{childdoc.def}
\childdocmain{}
%    \end{macrocode}

% Optional override for |\version| flag:
%    \begin{macrocode}
%%\ifchilddoc\else\providecommand{\version}{draft}\fi
%    \end{macrocode}

% Define the default values for the |\version| flag
% (|final| for the main file and |draft| for childs):
%    \begin{macrocode}
\ifchilddoc
\providecommand{\version}{draft}
\else
\providecommand{\version}{final}
\fi
%    \end{macrocode}

% Load the standard document class:
%    \begin{macrocode}
\documentclass[12pt]{article}
%    \end{macrocode}

% Start the document body:
%    \begin{macrocode}
\begin{document}
%    \end{macrocode}

% Declare a title page.
% Print title, part of document being processed and version flag:
%    \begin{macrocode}
\addtocounter{page}{-1}
\begin{center}
{\LARGE\bfseries{}childdoc example\par}
\vspace{1cm}
\ifchilddoc
\ifchilddocmanual part\else chapter\fi:
`\childdocname' of `\childdocjob'\par
\else
main document: `\childdocjob'\par
\fi
version: \version\par
\end{center}
\newpage
%    \end{macrocode}

% Manually include selected file,
% otherwise process as usual:
%    \begin{macrocode}
\ifchilddocmanual
\section*{part `\childdocname'}
\input{\childdocname}
\else
%    \end{macrocode}

% Include the two chapters:
%    \begin{macrocode}
\include{cdocsch1}
\include{cdocsch2}
%    \end{macrocode}

% Include the two parts unless only chapters should be displayed:
%    \begin{macrocode}
\ifchilddoc\else
\section{part three}
\input{cdocspt3}
\section{part four}
\input{cdocspt4}
\fi
%    \end{macrocode}

% Process as usual until here:
%    \begin{macrocode}
\fi
%    \end{macrocode}

% End of document body:
%    \begin{macrocode}
\end{document}
%    \end{macrocode}
%\iffalse
%</samplemain>
%\fi
%
% %%%%%%%%%%%%%%%%%%%%%%%%%%%%%%%%%%%%%%
% \paragraph{Chapter Include Files.}
%
% The include files are called |cdocsch1.tex| and |cdocsch2.tex|.
%
%\iffalse
%<*samplechap1|samplechap2>
%\fi

% Optional override for |\version| flag:
%    \begin{macrocode}
%%\providecommand{\version}{final}
%    \end{macrocode}

% Include the main document:
%    \begin{macrocode}
\input{childdoc.def}
\childdocof{cdocsamp}
%    \end{macrocode}

%\iffalse
%</samplechap1|samplechap2>
%\fi
%
%\iffalse
%<*samplechap1>
%\fi
% Some text for chapter 1:
%    \begin{macrocode}
\section{one}
some text in chapter one
%    \end{macrocode}

%\iffalse
%</samplechap1>
%\fi
% Some text for chapter 2:
%\iffalse
%<*samplechap2>
%\fi
%    \begin{macrocode}
\section{two}
more text in chapter two
%    \end{macrocode}

%\iffalse
%</samplechap2>
%\fi
%
% %%%%%%%%%%%%%%%%%%%%%%%%%%%%%%%%%%%%%%
% \paragraph{Part Include Files.}
%
% The include files are called |cdocspt3.tex| and |cdocspt4.tex|.
%
%\iffalse
%<*samplepart3|samplepart4>
%\fi

% Optional override for |\version| flag:
%    \begin{macrocode}
%%\providecommand{\version}{final}
%    \end{macrocode}

% Include the main document:
%    \begin{macrocode}
\input{childdoc.def}
\childdocby{cdocsamp}
%    \end{macrocode}

%\iffalse
%</samplepart3|samplepart4>
%\fi
%
%\iffalse
%<*samplepart3>
%\fi
% Some text for part 3:
%    \begin{macrocode}
some text in part three
%    \end{macrocode}

%\iffalse
%</samplepart3>
%\fi
% Some text for part 4:
%\iffalse
%<*samplepart4>
%\fi
%    \begin{macrocode}
more text in part four
%    \end{macrocode}

%\iffalse
%</samplepart4>
%\fi
%
% %%%%%%%%%%%%%%%%%%%%%%%%%%%%%%%%%%%%%%
% \paragraph{Forwarding for a Complete Draft.}
%
% The following forwarding file |cdocsdrf.tex|
% compiles the main document in draft mode:
%\iffalse
%<*sampledraft>
%\fi
%    \begin{macrocode}
\def\version{draft}
\input{childdoc.def}
\childdocforward{cdocsamp}
%    \end{macrocode}

%\iffalse
%</sampledraft>
%\fi
%
% %%%%%%%%%%%%%%%%%%%%%%%%%%%%%%%%%%%%%%
% \paragraph{Forwarding for Final Version of the Chapters.}
%
% The following forwarding files |cdocsfn1.tex| and |cdocsfn2.tex|
% (with identical content)
% compile the final versions of the child documents
% |cdocsch1.tex| and |cdocsch2.tex|, respectively:
%\iffalse
%<*samplefinal>
%\fi
%    \begin{macrocode}
\def\version{final}
\input{childdoc.def}
\childdocforwardprefix[cdocsamp]{cdocsfn}{cdocsch}
%    \end{macrocode}

%\iffalse
%</samplefinal>
%\fi
%
% %%%%%%%%%%%%%%%%%%%%%%%%%%%%%%%%%%%%%%
% \paragraph{Command Line Processing.}
%
% The following three command lines generate the output files
% |cdocscld|, |cdocscl1| and |cdocscl2|
% which should be identical to
% |cdocsdrf|, |cdocsch1| and |cdocsfn2|, respectively:
% \begin{center}
% \begin{tabular}{l}
% |latex -jobname cdocscld \|\\
% |  "\def\version{draft}\input{childdoc.def}\childdocforward{cdocsamp}"|\\
% |latex -jobname cdocscl1 \|\\
% |  "\input{childdoc.def}\childdocforward[cdocsamp]{cdocsch1}"|\\
% |latex -jobname cdocscl2 \|\\
% |  "\def\version{final}\input{childdoc.def}\childdocforward{cdocsch2}"|
% \end{tabular}
% \end{center}
% Note that the trailing backslash on each first line
% merely continues the input to the second line
% (for convenient cut ant paste).
% Furthermore, the command |latex| can be replaced by any
% of its alternative versions such as |pdflatex|.
%
% %%%%%%%%%%%%%%%%%%%%%%%%%%%%%%%%%%%%%%%%%%%%%%%%%%%%%%%%%%%%%%%%%%%%%%%%%%%%%%
% %%%%%%%%%%%%%%%%%%%%%%%%%%%%%%%%%%%%%%%%%%%%%%%%%%%%%%%%%%%%%%%%%%%%%%%%%%%%%%
% \section{Implementation}
%\iffalse
%<*package>
%\fi
%
% This section describes the definitions file |childdoc.def|.

% The definitions cannot be loaded using |\usepackage| or |\RequirePackage|
% which has a mechanism to prevent loading a style file more than once.
% When loading the definitions by means of |\input|
% multiple instances have to be prevented manually:
%\iffalse
%This code needs to be before the `\ProvidesFile' directive
%which is defined at the beginning of this file.
%Therefore it is also placed there and commented out here.
%</package>
%<*discard>
%\fi
%    \begin{macrocode}
\ifdefined\childdocmain\endinput\fi
%    \end{macrocode}
%\iffalse
%</discard>
%<*package>
%\fi
%
% \macro{\ifchilddoc}
% \macro{\ifchilddocmanual}
% The conditional |\ifchilddoc| tells whether a
% child (true) or main (false) document is being compiled.
% The conditional |\ifchilddocmanual| tells whether
% the |\includeonly| mechanism is used (false) or
% the selection of child files must be performed manually (true).
% The definitions initialise to false:
%    \begin{macrocode}
\newif\ifchilddoc
\newif\ifchilddocmanual
%    \end{macrocode}

% \macro{\childdocname}
% \macro{\childdocjob}
% The macro |\childdocname| stores the name of the main document
% to be compiled. The macro |\childdocjob| stores the name of
% the document on which the \LaTeX{} compiler was originally invoked.
% The content of |\jobname| cannot be compared
% to filenames specified in the source due to different catcodes.
% The following code rescans |\jobname|, stores the result
% in |\childdocname| and saves a copy in |\childdocjob|:
%    \begin{macrocode}
\edef\childdocname{\scantokens\expandafter{\jobname\noexpand}}
\let\childdocjob\childdocname
%    \end{macrocode}

% \macro{\childdocdisable}
% The macro |\childdocdisable| prevents the main file
% from being processed more than once.
% At this stage, the main document command |\childdocmain|
% is assumed to be called once again where it should do nothing.
% Any subsequent call to it should prevent
% a secondary processing of the main document
% It overwrites the forwarding commands
% |\childdocof| and |\childdocforward|
% with empty macros to prevent further inclusions of the main document:
%    \begin{macrocode}
\newcommand{\childdocdisable}
{
  \renewcommand{\childdocmain}[1]{\renewcommand{\childdocmain}[1]{\endinput}}
  \renewcommand{\childdocof}[1]{}
  \renewcommand{\childdocby}[2][]{}
  \renewcommand{\childdocforward}[2][]{}
  \renewcommand{\childdocdisable}{}
}
%    \end{macrocode}

% \macro{\childdocmain}
% The macro |\childdocmain| is to be called at the top of the main file
% with nothing or the main filename (without extension) as argument.
% First, it breaks loops.
% If the argument is not empty and does not match |\childdocname|
% (which is set by the first inclusion of |childdoc.def|),
% |\ifchilddoc| is set to true, |\includeonly| is applied to the child file
% and |\jobname| is set to the main file
% (for proper handling of |.aux| files):
%    \begin{macrocode}
\newcommand{\childdocmain}[1]
{
  \childdocdisable\childdocmain{}
  \if?#1?\else
    \begingroup
      \def\childdoctmp{#1}
      \ifx\childdoctmp\childdocname
        \def\childdoctmp{}
      \else
        \def\childdoctmp
        {
          \childdoctrue
          \includeonly{\childdocname}
          \def\childdocjob{#1}
          \def\jobname{#1}
        }
      \fi
      \expandafter
    \endgroup
    \childdoctmp
  \fi
}
%    \end{macrocode}

% \macro{\childdocof}
% The command |\childdocof| redirects
% compilation to the main file |#1|.
%    \begin{macrocode}
\newcommand{\childdocof}[1]
{
  \childdocdisable
  \childdoctrue
  \includeonly{\childdocname}
  \def\jobname{#1}
  \def\childdocjob{#1}
  \input{#1}
}
%    \end{macrocode}

% \macro{\childdocby}
% The command |\childdocby| ....
%    \begin{macrocode}
\newcommand{\childdocby}[2][]
{
  \childdocdisable
  \childdoctrue
  \childdocmanualtrue
  \if?#1?\else
    \def\jobname{#2}
  \fi
  \def\childdocjob{#2}
  \input{#2}
  \endinput
}
%    \end{macrocode}

% \macro{\childdocforward}
% The command |\childdocforward| redirects
% compilation to the main file or
% (if the optional argument is given) a child file.
% Parameters are set as if the main file
% or a child file starting with |\childdocof| was compiled.
% Then compilation is handed over to the main file:
%    \begin{macrocode}
\newcommand{\childdocforward}[2][]
{
  \begingroup
    \if?#1?
      \def\childdoctmp
      {
        \def\childdocname{#2}
        \def\childdocjob{#2}
        \def\jobname{#2}
        \input{#2}
        \endinput
      }
    \else
      \def\childdoctmp
      {
        \childdocdisable
        \def\childdocname{#2}
        \childdoctrue
        \includeonly{#2}
        \def\childdocjob{#1}
        \def\jobname{#1}
        \input{#1}
        \endinput
      }
    \fi
    \expandafter
  \endgroup
  \childdoctmp
}
%    \end{macrocode}

% \macro{\childdocforwardprefix}
% The command |\childdocforwardprefix| redirects
% compilation to the main or a child file by means of a pattern.
% The prefix |#1| in the current filename is replaced by |#2|
% and the suffix of the current filename is kept
% (it is assumed that the filename does not contain the substring `|~~~|'
% which is used as a delimiter).
% Compilation is handed over to the new file by |\childdocforward|:
%    \begin{macrocode}
\newcommand{\childdocforwardprefix}[3][]
{
  \begingroup
    \def\childdocextract #2##1~~~{\def\childdoctmp{\childdocforward[#1]{#3##1}}}
    \expandafter\childdocextract\childdocname~~~
    \expandafter
  \endgroup
  \childdoctmp
}
%    \end{macrocode}

% \macro{\childdoc}
% The deprecated macro |\childdoc| is a legacy version of |\childdocmain|:
%    \begin{macrocode}
\newcommand{\childdoc}{\childdocmain}
%    \end{macrocode}

% \macro{\childdocredirect}
% The deprecated macro |\childdocredirect| is a legacy version
% of |\childdocforward| and |\childdocforwardprefix|:
%    \begin{macrocode}
\newcommand{\childdocredirect}[2][]
{
  \begingroup
    \if?#1?
      \def\childdoctmp{\childdocforward{#2}}
    \else
      \def\childdoctmp{\childdocforwardprefix{#1}{#2}}
    \fi
    \expandafter
  \endgroup
  \childdoctmp
}
%    \end{macrocode}

%\iffalse
%</package>
%\fi
%
\endinput

\childdocforwardprefix[cdocsamp]{cdocsfn}{cdocsch}
%    \end{macrocode}

%\iffalse
%</samplefinal>
%\fi
%
% %%%%%%%%%%%%%%%%%%%%%%%%%%%%%%%%%%%%%%
% \paragraph{Command Line Processing.}
%
% The following three command lines generate the output files
% |cdocscld|, |cdocscl1| and |cdocscl2|
% which should be identical to
% |cdocsdrf|, |cdocsch1| and |cdocsfn2|, respectively:
% \begin{center}
% \begin{tabular}{l}
% |latex -jobname cdocscld \|\\
% |  "\def\version{draft}% \iffalse
%
% childdoc.dtx Copyright (C) 2017-2018 Niklas Beisert
%
% This work may be distributed and/or modified under the
% conditions of the LaTeX Project Public License, either version 1.3
% of this license or (at your option) any later version.
% The latest version of this license is in
%   http://www.latex-project.org/lppl.txt
% and version 1.3 or later is part of all distributions of LaTeX
% version 2005/12/01 or later.
%
% This work has the LPPL maintenance status `maintained'.
%
% The Current Maintainer of this work is Niklas Beisert.
%
% This work consists of the files childdoc.dtx and childdoc.ins
% and the derived files childdoc.def and cdocsamp.tex with
% cdocsch1.tex, cdocsch2.tex, cdocsdrf.tex, cdocsfn1.tex, cdocsfn2.tex.
%
%<package>\ifdefined\childdocmain\endinput\fi
%<package>\ProvidesFile{childdoc.def}[2018/12/30 v2.0 child document driver]
%<samplemain>\ProvidesFile{cdocsamp.tex}[2018/12/30 v2.0 sample for childdoc]
%<*driver>
%\ProvidesFile{childdoc.drv}[2018/12/30 v2.0 childdoc reference manual file]
\PassOptionsToClass{10pt,a4paper}{article}
\documentclass{ltxdoc}

\usepackage[margin=35mm]{geometry}
\usepackage{hyperref}
\usepackage{hyperxmp}
\usepackage[usenames]{color}

\hypersetup{colorlinks=true}
\hypersetup{pdfstartview=FitH}
\hypersetup{pdfpagemode=UseNone}
\hypersetup{pdfsource={}}
\hypersetup{pdflang={en-UK}}
\hypersetup{pdfcopyright={Copyright 2017-2018 Niklas Beisert.
  This work may be distributed and/or modified under the
  conditions of the LaTeX Project Public License, either version 1.3
  of this license or (at your option) any later version.}}
\hypersetup{pdflicenseurl={http://www.latex-project.org/lppl.txt}}
\hypersetup{pdfcontactaddress={ETH Zurich, ITP, HIT K,
  Wolfgang-Pauli-Strasse 27}}
\hypersetup{pdfcontactpostcode={8093}}
\hypersetup{pdfcontactcity={Zurich}}
\hypersetup{pdfcontactcountry={Switzerland}}
\hypersetup{pdfcontactemail={nbeisert@itp.phys.ethz.ch}}
\hypersetup{pdfcontacturl={http://people.phys.ethz.ch/\xmptilde nbeisert/}}

\newcommand{\secref}[1]{\hyperref[#1]{section \ref*{#1}}}

\parskip1ex
\parindent0pt
\let\olditemize\itemize
\def\itemize{\olditemize\parskip0pt}

\begin{document}

\title{The \textsf{childdoc} Package}
\hypersetup{pdftitle={The childdoc Package}}
\author{Niklas Beisert\\[2ex]
  Institut f\"ur Theoretische Physik\\
  Eidgen\"ossische Technische Hochschule Z\"urich\\
  Wolfgang-Pauli-Strasse 27, 8093 Z\"urich, Switzerland\\[1ex]
  \href{mailto:nbeisert@itp.phys.ethz.ch}
  {\texttt{nbeisert@itp.phys.ethz.ch}}}
\hypersetup{pdfauthor={Niklas Beisert}}
\hypersetup{pdfsubject={Manual for the LaTeX2e Package childdoc}}
\date{30 December 2018, \textsf{v2.0}}
\maketitle

\begin{abstract}\noindent
\textsf{childdoc} is a \LaTeXe{} package
that enables the direct compilation
of document sections included by |\include|
to individual files.
\end{abstract}

\begingroup
\parskip0ex
\tableofcontents
\endgroup

%%%%%%%%%%%%%%%%%%%%%%%%%%%%%%%%%%%%%%%%%%%%%%%%%%%%%%%%%%%%%%%%%%%%%%%%%%%%%%%%
%%%%%%%%%%%%%%%%%%%%%%%%%%%%%%%%%%%%%%%%%%%%%%%%%%%%%%%%%%%%%%%%%%%%%%%%%%%%%%%%
\section{Introduction}

\LaTeX{} provides a mechanism to structure a large document (such as a book)
into a main file and several child files (containing the chapters)
using the |\include| command.
This mechanism is beneficial for documents
which span hundreds of pages in order to
make the source file(s) more manageable.
Moreover, compilation can be restricted to
selected child files by means of the |\includeonly| command.
The latter feature can be used to reduce the compilation time while editing
(this was significantly more useful in the earlier days of \LaTeX{})
or to generate a smaller document which is easier to navigate.
Another application of |\includeonly| is to generate
documents consisting of selected parts of the complete document.

However, there are a few drawbacks of the plain |\include| mechanism:
\begin{itemize}
\item
The child files cannot be compiled on their own,
they can only be compiled via the main file.
A naive editing environment
(such as a text editor with an option
to have the current file processed by \LaTeX)
may require one to switch to the main file before compiling;
attempting to compile the child file produces errors.
\item
The main file must be modified (each time)
to adjust the |\includeonly| command
to the present needs. This easily leaves the main file in a messy state.
\item
The generated document will always carry the filename
of the main document. This is inconvenient if
several child files are to be compiled and
to be kept for distribution.
\end{itemize}

The present package provides a simple interface
to make child files individually compilable by \LaTeX{}.
Compiling a child file then has the same effect as compiling
the main file with an |\includeonly| command
to select the appropriate child.
Moreover the generated document will carry the name of the child
rather than the main file.
This resolves all three above issues.

This feature is meant to make the editing of books,
thesis documents and lecture notes somewhat more convenient.
However, the package can also be used efficiently for
composing a series of documents (such as exercise sheets)
which are typically distributed individually.
It then assists the author in generating the individual documents
(potentially in different versions)
as well as a document containing the collected series.
Another application is in developing style files
or other kinds of included material
where compilation of the style file could redirect
to a sample or test file.

%%%%%%%%%%%%%%%%%%%%%%%%%%%%%%%%%%%%%%%%%%%%%%%%%%%%%%%%%%%%%%%%%%%%%%%%%%%%%%%%
%%%%%%%%%%%%%%%%%%%%%%%%%%%%%%%%%%%%%%%%%%%%%%%%%%%%%%%%%%%%%%%%%%%%%%%%%%%%%%%%
\section{Usage}

First of all, the package \textsf{childdoc} is \emph{not} a standard
\LaTeXe{} |.sty| style file! Therefore it needs to be invoked in
a non-standard way.

%%%%%%%%%%%%%%%%%%%%%%%%%%%%%%%%%%%%%%%%%%%%%%%%%%%%%%%%%%%%%%%%%%%%%%%%%%%%%%%%
\subsection{Included Files}
\label{sec:include}

%%%%%%%%%%%%%%%%%%%%%%%%%%%%%%%%%%%%%%%%
\DescribeMacro{\childdocmain}
To use the package, add the commands
\begin{center}
\begin{tabular}{l}
|\input{childdoc.def}|\\
|\childdocmain{}|\\
\end{tabular}
\end{center}
at the very top of the main \LaTeX{} file,
in particular \emph{before} the |\documentclass| statement!
The argument of |\childdocmain| should be left empty
(but it must be present).

%%%%%%%%%%%%%%%%%%%%%%%%%%%%%%%%%%%%%%%%
\DescribeMacro{\childdocof}
Furthermore, add the commands
\begin{center}
\begin{tabular}{l}
|\input{childdoc.def}|\\
|\childdocof{|\textit{main}|}|\\
\end{tabular}
\end{center}
at the top of every child file \textit{child}
which is included by |\include{|\textit{child}|}|
from within the main file
(or at least for those files to be compiled individually).
The argument \textit{main} must be the filename of the main file.

There are a couple of
considerations in setting up the main and child documents:

%%%%%%%%%%%%%%%%%%%%%%%%%%%%%%%%%%%%%%%%
\paragraph{Restrictions.}

Please note the following restrictions:
\begin{itemize}
\item
|\childdocmain| must be called with one argument \textit{main}
to ensure compatibility with earlier version of the package.
It must either be empty (|\childdocmain{}|)
or precisely match the filename of the main file in which it is specified.
See \secref{sec:detection} for further information.
\item
The filename \textit{main} must be specified without the |.tex| extension.
\item
The filename \textit{main} is case sensitive
(even in case-insensitive file systems)
due to internal string comparison.
\item
The argument \textit{main} should be fully expanded, it cannot be a macro.
\item
Subdirectories and special characters should be avoided in filenames.
\item
The command |\childdocmain{|\textit{main}|}| must be followed by a whitespace.
It should not be followed immediately by another command
or by a comment mark `|%|'.
This is because the \TeX{} parser reads the token immediately following
the argument of |\childdocmain| and puts it
at the beginning of every child section;
however, a white\-space is ignored.
\end{itemize}

%%%%%%%%%%%%%%%%%%%%%%%%%%%%%%%%%%%%%%%%
\paragraph{Content of Main File.}

It is advisable to place all content in the child files included by |\include|.
Any output contained in the main file will appear in all child documents
unless suppressed manually;
it cannot be suppressed automatically by the |\includeonly| directive
and thus should normally be avoided.
A method to include some content in the main file
by means of conditional processing is described in \secref{sec:conditional}.

%%%%%%%%%%%%%%%%%%%%%%%%%%%%%%%%%%%%%%%%
\paragraph{Page Numbering.}

When only a part of the document is compiled,
the appropriate numbering of pages
(as well as other status parameters)
is determined from the |.aux| files.
The latter contain information from previous passes.
However this information needs to propagate through
all intermediate child documents.
Therefore the page numbering in child documents may well
be inconsistent until the complete document is compiled at least once.

A useful (if unconventional) way to always ensure a consistent
page numbering is to restart the numbering in each child document
and denote the pages by `\textit{child}|.|\textit{page}'
where \textit{child} represents the chapter/section number of the child file.
This can be achieved by the command
|\numberwithin{page}{|\textit{child}|}|
of the \textsf{amsmath} package
where \textit{child} can be |chapter| or |section|
depending on the chosen structuring.
Alternatively, one can modify the macro |\thepage| appropriately
and reset the counter |page| at the start of each child file.

%%%%%%%%%%%%%%%%%%%%%%%%%%%%%%%%%%%%%%%%%%%%%%%%%%%%%%%%%%%%%%%%%%%%%%%%%%%%%%%%
\subsection{Conditional Processing}
\label{sec:conditional}

The package provides a mechanism to compile different versions
of a document. To customise the versions further some conditional processing
can come in handy to distinguish which version is being compiled.
The package provides two macros to describe the compilation context:

%%%%%%%%%%%%%%%%%%%%%%%%%%%%%%%%%%%%%%%%
\DescribeMacro{\ifchilddoc}
The conditional |\ifchilddoc| distinguishes between the compilation of
child documents and the main document:
%
\begin{center}
|\ifchilddoc |\textit{child-code}| |[|\||else |\textit{main-code}]| \||fi|
\end{center}

%%%%%%%%%%%%%%%%%%%%%%%%%%%%%%%%%%%%%%%%
\DescribeMacro{\childdocname}
\DescribeMacro{\childdocjob}
The macro |\childdocname| contains the filename (without extension)
of the main or child file being processed.
Note that |\childdocjob| will always contain the name of the main file.

%%%%%%%%%%%%%%%%%%%%%%%%%%%%%%%%%%%%%%%%
\paragraph{Title Page.}

Conditional processing can be used to include a title or banner page
in the main document when proper precautions are taken.
Importantly, the code in the main file should ensure that the page counter
(as well as other status parameters which are stored in the |.aux| files)
takes the same value after the conditional processing.
Otherwise the page numbers may take divergent values
depending on which part is compiled.

For example, a title page could be declared by:
%
\begin{center}
\begin{tabular}{l}
|\ifchilddoc\||else|\\
|\addtocounter{page}{-1}|\\
\textit{code for title page}\\
|\newpage|\\
|\||fi|
\end{tabular}
\end{center}
%
A banner page for the child documents can be generated by:
%
\begin{center}
\begin{tabular}{l}
|\ifchilddoc|\\
|\addtocounter{page}{-1}|\\
\textit{code for banner page}\\
|\newpage|\\
|\||fi|
\end{tabular}
\end{center}
%
Here one could write a message such as:
\begin{center}
|This is the part \childdocname{} of \childdocjob{}.|
\end{center}

%%%%%%%%%%%%%%%%%%%%%%%%%%%%%%%%%%%%%%%%%%%%%%%%%%%%%%%%%%%%%%%%%%%%%%%%%%%%%%%%
\subsection{Flags}
\label{sec:flags}

The package makes it easy to generate different versions
of the main or child documents.
To this end compilation flags can be defined
and assigned different default values.
They will be particularly useful in conjunction
with the forwarding mechanism described in \secref{sec:forward}.

For example, it may be useful to have a flag |\version|
which can be set to |draft| or |final|.
The document source will contain some conditional code
depending on the value of |\version|.
Suppose further, the flag should default to |final| for the main file
and to |draft| for child files
which is a natural assignment for editing the document.
This is achieved by placing the following code
in the preamble of the main document
(below the |\childdocmain| directive):
%
\begin{center}
\begin{tabular}{l}
|\ifchilddoc|\\
|\providecommand{\version}{draft}|\\
|\||else|\\
|\providecommand{\version}{final}|\\
|\||fi|
\end{tabular}
\end{center}
%
The definition by |\providecommand| makes sure
that previous definitions are not overwritten.
Further statements |\providecommand{\version}{...}|
can thus be added before the above code to override it.

For the main file, one might add a line
(between |\childdocmain| and the above block)
%
\begin{center}
|%\ifchilddoc\||else\providecommand{\version}{draft}\||fi|
\end{center}
%
which can be uncommented to produce a draft version.
Likewise one can add a line to the very top of a child file
(above the |\childdocof{|\textit{main}|}| directive)
%
\begin{center}
|%\providecommand{\version}{final}|
\end{center}
%
which can be uncommented to produce the final version of this child document.

%%%%%%%%%%%%%%%%%%%%%%%%%%%%%%%%%%%%%%%%%%%%%%%%%%%%%%%%%%%%%%%%%%%%%%%%%%%%%%%%
\subsection{Forwarding}
\label{sec:forward}

Different versions of the main or child documents
using compilation flags as described in \secref{sec:flags}
can be (permanently) stored in different files
for convenient compilation, viewing and distribution.
To this end, the package defines a command
to pass on compilation to a different file:

%%%%%%%%%%%%%%%%%%%%%%%%%%%%%%%%%%%%%%%%
\DescribeMacro{\childdocforward}
The command |\childdocforward| redirects processing to
another source file:
%
\begin{center}
\begin{tabular}{l}
|\input{childdoc.def}|\\
|\childdocforward[|\textit{main}|]{|\textit{dest}|}|\\
\end{tabular}
\end{center}
%
The argument \textit{dest} is the destination file
(without extension).
It should be the main file or one of the child files.
Note that further \textsf{childdoc} directives
such as |\childdocof| and |\childdocforward|
in the indicated file will be processed in this form.
The optional argument \textit{main}
passes on directly to the main file \textit{main}
while pretending to compile the child \textit{dest}.
This form behaves as if \textit{dest}
issues |\childdocof{|\textit{main}|}| right away,
and no further \textsf{childdoc} directives will be processed.

%%%%%%%%%%%%%%%%%%%%%%%%%%%%%%%%%%%%%%%%
\DescribeMacro{\...prefix}
In the alternative form |\childdocforwardprefix|,
%
\begin{center}
\begin{tabular}{l}
|\input{childdoc.def}|\\
|\childdocforwardprefix[|\textit{main}|]{|\textit{prefix}|}{|\textit{dest}|}|
\end{tabular}
\end{center}
%
the destination file is determined by a pattern
depending on the current file:
To make this work, the current file must be called
`{\textit{prefix}\hspace{0.2em}\textit{suffix}}'
with \textit{prefix} matching precisely the argument.
Processing is then passed on to the file
`{\textit{dest}\hspace{0.2em}\textit{suffix}}'.
Surely, the same effect is achieved by
directly specifying the
argument `{\textit{dest}\hspace{0.2em}\textit{suffix}}'
in the first form.
However, that requires to set up a different file
for each child. With the alternative form of the command
all these files can have exactly the same content
which simplifies setting them up and maintaining them.

For example, the following file |draft.tex|
with a compilation flag |\version| as described in \secref{sec:flags}
compiles the main document as a draft:
%
\begin{center}
\begin{tabular}{l}
|\def\version{draft}|\\
|\input{childdoc.def}|\\
|\childdocforward{|\textit{main}|}|
\end{tabular}
\end{center}
%
Likewise, the following files |final|\textit{nn}|.tex|
compile the final version of the child document
|child|\textit{nn}|.tex|:
%
\begin{center}
\begin{tabular}{l}
|\def\version{final}|\\
|\input{childdoc.def}|\\
|\childdocforwardprefix{final}{child}|
\end{tabular}
\end{center}
%

Note that when several versions of a main file and/or of each child file
are to be generated, it may be convenient to set up a |Makefile| or
shell script to automatise the process.

%%%%%%%%%%%%%%%%%%%%%%%%%%%%%%%%%%%%%%%%%%%%%%%%%%%%%%%%%%%%%%%%%%%%%%%%%%%%%%%%
\subsection{Command Line Processing}
\label{sec:commandline}

The effect of redirection files can also be achieved by invoking
the \LaTeX{} compiler with a more elaborate command line.
Most conveniently this should be done as part
of a shell script or a |Makefile|.

When using \textsf{childdoc} in the main file, the following
command lines effectively perform a redirection
(note that depending on the shell being used,
backslashes may have to be doubled: `|\|' $\to$ `|\\|'):
%
\begin{center}
|... -jobname "|\textit{target}|" |\\|"|[\textit{flags}]%
|\input{childdoc.def}\childdocforward[|\textit{main}|]{|\textit{dest}|}"|
\end{center}
%
Here \textit{target} is the name of the output file,
\textit{main} is the name of the main file
and \textit{dest} is the name of the main or child file to be processed
(all filenames without extensions).
The optional argument \textit{main} can be omitted
if \textit{main} matches \textit{dest}.
Optionally, compilation \textit{flags} can be defined via |\def| commands.
This command line makes the \TeX{} engine believe
it is compiling the file \textit{target}
whose content is specified as the latter parameter.
The provided code then forwards the processing to
\textit{main} or \textit{dest} as described in \secref{sec:forward}.

%%%%%%%%%%%%%%%%%%%%%%%%%%%%%%%%%%%%%%%%%%%%%%%%%%%%%%%%%%%%%%%%%%%%%%%%%%%%%%%%
\subsection{Include by Input}
\label{sec:input}

Including child documents by |\include| has some restrictions by design.
Most notably, the content of a child document always occupies
its own set of pages; pages cannot be shared between child documents.
Usually, this behaviour makes perfect sense
because each child document contain an essential part of the document.
However, in some situations it may be desirable to compose
a document from a collection of parts
without having mandatory page breaks between then.
For this case, the package
provides a mechanism to include parts
by |\input| which can also be processed individually.
However, by construction this mechanism
requires manual handling of the content to be output.

%%%%%%%%%%%%%%%%%%%%%%%%%%%%%%%%%%%%%%%%
\DescribeMacro{\ifchilddocmanual}
The main file should be prepared as usual, see \secref{sec:include}.
However, the document body must make a distinction
between processing of an individual part and of the main document, e.g.:
%
\begin{center}
\begin{tabular}{l}
|\ifchilddocmanual|\\
|\input{\childdocname}|\\
|\||else|\\
\textit{document body with }|\input{|\textit{part}|}|\\
|\||fi|
\end{tabular}
\end{center}
%
The conditional |\ifchilddocmanual| is true whenever
a part to be included by |\input| is being compiled,
and the name of the part is stored in |\childdocname|.

%%%%%%%%%%%%%%%%%%%%%%%%%%%%%%%%%%%%%%%%
\DescribeMacro{\childdocby}
Each part to be included by |\input| should start with:
%
\begin{center}
\begin{tabular}{l}
|\input{childdoc.def}|\\
|\childdocby{|\textit{main}|}|\\
\end{tabular}
\end{center}
%
The directive |\childdocby| is similar to |\childdocof|
described in \secref{sec:include},
but the subsequent selection of content must be done manually.
To that end, both |\ifchilddoc| and |\ifchilddocmanual|
will be true upon processing of a part,
and the name of the part is stored in |\childdocname|.
Note that |\jobname| will be set to the filename of the current part
so that each part receives an individual |.aux| file
that does not interfere with the |.aux| file(s) of the main document.
This behaviour can be altered by the alternative form
|\childdocby[*]{|\textit{main}|}| (with a non-empty optional argument)
which uses the |.aux| file of the main document
by setting |\jobname| to \textit{main}.

%%%%%%%%%%%%%%%%%%%%%%%%%%%%%%%%%%%%%%%%%%%%%%%%%%%%%%%%%%%%%%%%%%%%%%%%%%%%%%%%
\subsection{Driver Development}
\label{sec:driver}

The \textsf{childdoc} mechanism can also be use for the development
of definition files such as \LaTeX{} styles or classes.
This case differs from the above setup with multiple parts
included by |\include| in that no |\includeonly| should be invoked.
This can be achieved by starting the include file
(before |\ProvidesPackage|) with:
%
\begin{center}
\begin{tabular}{l}
|\input{childdoc.def}|\\
|\childdocforward{|\textit{main}|}|\\
\end{tabular}
\end{center}
%
or alternatively with:
%
\begin{center}
\begin{tabular}{l}
|\input{childdoc.def}|\\
|\childdocby{|\textit{main}|}|\\
\end{tabular}
\end{center}
%
Both forms have slightly different effects as described above.
The main file is prepared as usual, see \secref{sec:include}.

%%%%%%%%%%%%%%%%%%%%%%%%%%%%%%%%%%%%%%%%%%%%%%%%%%%%%%%%%%%%%%%%%%%%%%%%%%%%%%%%
\subsection{Legacy Detection}
\label{sec:detection}

The directive |\childdocmain| in the main file can detect
whether the complete document or merely a child is to be compiled
even without using the directive |\childdocof|.
This method is deprecated because it is less robust
and there is no compelling reason to use it;
it is merely provided for backward compatibility
and it may be removed in future versions.

If the detection mechanism is to be used,
it is mandatory to correctly specify
the filename of the main file as the argument of |\childdocmain|:
%
\begin{center}
\begin{tabular}{l}
|\input{childdoc.def}|\\
|\childdocmain{|\textit{main}|}|\\
\end{tabular}
\end{center}
%
If |\jobname| does not match the argument \textit{main} of |\childdocmain|,
it is assumed that |\jobname| points to the child file to be compiled.
When using |\childdocmain| with the main file specified as argument,
it suffices to start a child file
with just |\input{|\textit{main}|}|
without loading of the package and using |\childdocof|.
If instead all processing is done
with the appropriate \textsf{childdoc} directives,
the argument of \textit{main} of |\childdocmain| can be empty.

An alternative version of the command line processing described
in \secref{sec:commandline} using the detection mechanism reads:
%
\begin{center}
|... -jobname "|\textit{target}|" "|[\textit{flags}]%
[|\def\jobname{|\textit{dest}|}|]|\input{|\textit{main}|}"|
\end{center}

%%%%%%%%%%%%%%%%%%%%%%%%%%%%%%%%%%%%%%%%%%%%%%%%%%%%%%%%%%%%%%%%%%%%%%%%%%%%%%%%
\subsection{Manual Code}
\label{sec:manual}

In case one cannot be certain whether the definitions file |childdoc.def|
is installed on the target \TeX{} distribution
and one prefers not to ship it,
it is conceivable to paste a few relevant commands into the sources.

To that end, drop all statements |\input{childdoc.def}|
and perform the replacements as outlined below.
Instead of |\childdocmain{|\textit{main}|}| add the following code
to the top of the main file:
%
\begin{center}
\begin{tabular}{l}
|\||ifdefined\childdocname\endinput\||fi\newif\ifchilddoc|\\
|\edef\childdocname{\scantokens\expandafter{\jobname\noexpand}}|\\
|\def\childdocmain{|\textit{main}|}\||ifx\childdocmain\childdocname\||else|\\
|\childdoctrue\includeonly{\childdocname}\let\jobname\childdocmain\||fi|\\
\end{tabular}
\end{center}
%
Instead of |\childdocof{|\textit{main}|}| just include the main file
at the top of each child file:
%
\begin{center}
|\input{|\textit{main}|}|
\end{center}
%
A simple redirection |\childdocforward{|\textit{dest}|}| is achieved by:
%
\begin{center}
|\def\jobname{|\textit{dest}|}\input{\jobname}|
\end{center}
%
The redirection with prefix
|\childdocforwardprefix[|\textit{prefix}|]{|\textit{dest}|}|
is accomplished by:
%
\begin{center}
\begin{tabular}{l}
|{\edef\jobname{\scantokens\expandafter{\jobname\noexpand}}|\\
|\def\redirectjob |\textit{prefix}|#1~~~{\gdef\jobname{|\textit{dest}|#1}}|\\
|\expandafter\redirectjob\jobname~~~}\input{\jobname}|
\end{tabular}
\end{center}

In an alternative approach,
child documents can be compiled by a specific command line
without additional code or specific definitions:
%
\begin{center}
|... -jobname "|\textit{target}|" "|[\textit{flags}]%
|\includeonly{|\textit{dest}|}\input{|\textit{main}|}"|
\end{center}
%

%%%%%%%%%%%%%%%%%%%%%%%%%%%%%%%%%%%%%%%%%%%%%%%%%%%%%%%%%%%%%%%%%%%%%%%%%%%%%%%%
%%%%%%%%%%%%%%%%%%%%%%%%%%%%%%%%%%%%%%%%%%%%%%%%%%%%%%%%%%%%%%%%%%%%%%%%%%%%%%%%
\section{Information}

%%%%%%%%%%%%%%%%%%%%%%%%%%%%%%%%%%%%%%%%%%%%%%%%%%%%%%%%%%%%%%%%%%%%%%%%%%%%%%%%
\subsection{Copyright}

Copyright \copyright{} 2017--2018 Niklas Beisert

This work may be distributed and/or modified under the
conditions of the \LaTeX{} Project Public License, either version 1.3
of this license or (at your option) any later version.
The latest version of this license is in
  \url{http://www.latex-project.org/lppl.txt}
and version 1.3 or later is part of all distributions of \LaTeX{}
version 2005/12/01 or later.

This work has the LPPL maintenance status `maintained'.

The Current Maintainer of this work is Niklas Beisert.

This work consists of the files |README.txt|, |childdoc.ins| and |childdoc.dtx|
as well as the derived files |childdoc.def|, |cdocsamp.tex|
with |cdocsch1.tex|, |cdocsch2.tex|, |cdocspt3.tex|, |cdocspt4.tex|,
|cdocsdrf.tex|, |cdocsfn1.tex|, |cdocsfn2.tex|
as well as |childdoc.pdf|.

%%%%%%%%%%%%%%%%%%%%%%%%%%%%%%%%%%%%%%%%%%%%%%%%%%%%%%%%%%%%%%%%%%%%%%%%%%%%%%%%
\subsection{Files and Installation}

The package consists of the files:
%
\begin{center}
\begin{tabular}{ll}
    |README.txt|   & readme file \\
    |childdoc.ins| & installation file \\
    |childdoc.dtx| & source file \\
    |childdoc.def| & definition file \\
    |cdocsamp.tex| & sample main file \\
    |cdocsch1.tex| & sample include file \\
    |cdocsch2.tex| & sample include file \\
    |cdocspt3.tex| & sample part file \\
    |cdocspt4.tex| & sample part file \\
    |cdocsdrf.tex| & sample redirection file \\
    |cdocsfn1.tex| & sample redirection file \\
    |cdocsfn2.tex| & sample redirection file \\
    |childdoc.pdf| & manual
\end{tabular}
\end{center}
%
The distribution consists of the files
|README.txt|, |childdoc.ins| and |childdoc.dtx|.
%
\begin{itemize}
\item
Run (pdf)\LaTeX{} on |childdoc.dtx|
to compile the manual |childdoc.pdf| (this file).
\item
Run \LaTeX{} on |childdoc.ins| to create the definitions file |childdoc.def|
and the sample |cdocsamp.tex| with include files
|cdocsch1.tex|, |cdocsch2.tex|, |cdocspt3.tex|, |cdocspt4.tex|,
|cdocsdrf.tex|, |cdocsfn1.tex|, |cdocsfn2.tex|.
Then copy the file |childdoc.def| to an appropriate directory of your \LaTeX{}
distribution, e.g.\ \textit{texmf-root}|/tex/latex/childdoc|.
\end{itemize}

%%%%%%%%%%%%%%%%%%%%%%%%%%%%%%%%%%%%%%%%%%%%%%%%%%%%%%%%%%%%%%%%%%%%%%%%%%%%%%%%
\subsection{Related CTAN Packages}

There are several other packages which offer a similar functionality:
%
\begin{itemize}
\item
The packages
\href{http://ctan.org/pkg/docmute}{\textsf{docmute}},
\href{http://ctan.org/pkg/includex}{\textsf{includex}} and
\href{http://ctan.org/pkg/standalone}{\textsf{standalone}}
provide commands to include only the document body of
a child file thus allowing both files to be compiled individually.
\item
The packages \href{http://ctan.org/pkg/subdocs}{\textsf{subdocs}}
and \href{http://ctan.org/pkg/subfiles}{\textsf{subfiles}}
provide structures in which the main and child documents can be
encapsulated and allowing them to be compiled individually.
The inclusion mechanism is different from the conventional |\include|.
\item
The package \href{http://ctan.org/pkg/combine}{\textsf{combine}}
is an elaborate solution to combine several documents into one.
\end{itemize}
%
See also the CTAN topic \href{http://ctan.org/topic/subdocs}{\textsf{subdocs}}
for further related packages.
The present package differs from the above solutions in that
a document structure constructed with the conventional |\include| mechanism
just needs two extra commands at the top of every file
such that all constituent files can be compiled individually.

%%%%%%%%%%%%%%%%%%%%%%%%%%%%%%%%%%%%%%%%%%%%%%%%%%%%%%%%%%%%%%%%%%%%%%%%%%%%%%%%
%\subsection{Feature Suggestions}
%
%The following is a list of features which may be useful for future
%versions of this package:
%%
%\begin{itemize}
%\item
%\ldots
%\end{itemize}

%%%%%%%%%%%%%%%%%%%%%%%%%%%%%%%%%%%%%%%%%%%%%%%%%%%%%%%%%%%%%%%%%%%%%%%%%%%%%%%%
\subsection{Revision History}

%%%%%%%%%%%%%%%%%%%%%%%%%%%%%%%%%%%%%%%%
\paragraph{v2.0:} 2018/12/30

\begin{itemize}
\item
immediate forward processing
\item
added |\childdocby| mechanism
\item
manual restructured
\end{itemize}

%%%%%%%%%%%%%%%%%%%%%%%%%%%%%%%%%%%%%%%%
\paragraph{v1.6:} 2018/01/17

\begin{itemize}
\item
application for development of include files
\item
corrections to manual
\end{itemize}

%%%%%%%%%%%%%%%%%%%%%%%%%%%%%%%%%%%%%%%%
\paragraph{v1.5:} 2017/05/21

\begin{itemize}
\item
more complete structuring introduced
\item
|\childdocof| introduced
\item
|\childdoc| renamed to |\childdocmain|
\item
|\childredirect| renamed to |\childdocforward| and |\childdocforwardprefix|
and functionality expanded
\end{itemize}

%%%%%%%%%%%%%%%%%%%%%%%%%%%%%%%%%%%%%%%%
\paragraph{v1.0:} 2017/04/27

\begin{itemize}
\item
manual and install package
\item
first version published on CTAN
\end{itemize}

%%%%%%%%%%%%%%%%%%%%%%%%%%%%%%%%%%%%%%%%
\paragraph{v0.6:} 2017/04/26

\begin{itemize}
\item
redirection mechanism added
\end{itemize}

%%%%%%%%%%%%%%%%%%%%%%%%%%%%%%%%%%%%%%%%
\paragraph{v0.5:} 2017/04/26

\begin{itemize}
\item
functionality in definition file
\end{itemize}


%%%%%%%%%%%%%%%%%%%%%%%%%%%%%%%%%%%%%%%%%%%%%%%%%%%%%%%%%%%%%%%%%%%%%%%%%%%%%%%%
%%%%%%%%%%%%%%%%%%%%%%%%%%%%%%%%%%%%%%%%%%%%%%%%%%%%%%%%%%%%%%%%%%%%%%%%%%%%%%%%
%%%%%%%%%%%%%%%%%%%%%%%%%%%%%%%%%%%%%%%%%%%%%%%%%%%%%%%%%%%%%%%%%%%%%%%%%%%%%%%%
\appendix

\settowidth\MacroIndent{\rmfamily\scriptsize 000\ }

 \DocInput{childdoc.dtx}

\end{document}
%</driver>
% \fi
%
% %%%%%%%%%%%%%%%%%%%%%%%%%%%%%%%%%%%%%%%%%%%%%%%%%%%%%%%%%%%%%%%%%%%%%%%%%%%%%%
% %%%%%%%%%%%%%%%%%%%%%%%%%%%%%%%%%%%%%%%%%%%%%%%%%%%%%%%%%%%%%%%%%%%%%%%%%%%%%%
% \section{Sample}
%\iffalse
%<*samplemain>
%\fi
%
% The following presents a sample document
% with two chapters, two parts, a title page,
% a compile flag as well as three forwarding files to set the flag.
% It consists of eight |.tex| files:
% \begin{center}
% \begin{tabular}{ll}
% |cdocsamp.tex|&main file\\
% |cdocsch1.tex|&include file for chapter 1\\
% |cdocsch2.tex|&include file for chapter 2\\
% |cdocspt3.tex|&include file for part 3\\
% |cdocspt4.tex|&include file for part 4\\
% |cdocsdrf.tex|&forwarding file for main file in draft mode\\
% |cdocsfi1.tex|&forwarding file for final version of chapter 1\\
% |cdocsfi2.tex|&forwarding file for final version of chapter 2\\
% \end{tabular}
% \end{center}
% Each of the eight files can be compiled directly by the \LaTeX{} compiler.
%
% %%%%%%%%%%%%%%%%%%%%%%%%%%%%%%%%%%%%%%
% \paragraph{Main File.}
%
% The main file is called |cdocsamp.tex|.
%
% Load the \textsf{childdoc} definitions and
% declare the filename for the main document:
%    \begin{macrocode}
\input{childdoc.def}
\childdocmain{}
%    \end{macrocode}

% Optional override for |\version| flag:
%    \begin{macrocode}
%%\ifchilddoc\else\providecommand{\version}{draft}\fi
%    \end{macrocode}

% Define the default values for the |\version| flag
% (|final| for the main file and |draft| for childs):
%    \begin{macrocode}
\ifchilddoc
\providecommand{\version}{draft}
\else
\providecommand{\version}{final}
\fi
%    \end{macrocode}

% Load the standard document class:
%    \begin{macrocode}
\documentclass[12pt]{article}
%    \end{macrocode}

% Start the document body:
%    \begin{macrocode}
\begin{document}
%    \end{macrocode}

% Declare a title page.
% Print title, part of document being processed and version flag:
%    \begin{macrocode}
\addtocounter{page}{-1}
\begin{center}
{\LARGE\bfseries{}childdoc example\par}
\vspace{1cm}
\ifchilddoc
\ifchilddocmanual part\else chapter\fi:
`\childdocname' of `\childdocjob'\par
\else
main document: `\childdocjob'\par
\fi
version: \version\par
\end{center}
\newpage
%    \end{macrocode}

% Manually include selected file,
% otherwise process as usual:
%    \begin{macrocode}
\ifchilddocmanual
\section*{part `\childdocname'}
\input{\childdocname}
\else
%    \end{macrocode}

% Include the two chapters:
%    \begin{macrocode}
\include{cdocsch1}
\include{cdocsch2}
%    \end{macrocode}

% Include the two parts unless only chapters should be displayed:
%    \begin{macrocode}
\ifchilddoc\else
\section{part three}
\input{cdocspt3}
\section{part four}
\input{cdocspt4}
\fi
%    \end{macrocode}

% Process as usual until here:
%    \begin{macrocode}
\fi
%    \end{macrocode}

% End of document body:
%    \begin{macrocode}
\end{document}
%    \end{macrocode}
%\iffalse
%</samplemain>
%\fi
%
% %%%%%%%%%%%%%%%%%%%%%%%%%%%%%%%%%%%%%%
% \paragraph{Chapter Include Files.}
%
% The include files are called |cdocsch1.tex| and |cdocsch2.tex|.
%
%\iffalse
%<*samplechap1|samplechap2>
%\fi

% Optional override for |\version| flag:
%    \begin{macrocode}
%%\providecommand{\version}{final}
%    \end{macrocode}

% Include the main document:
%    \begin{macrocode}
\input{childdoc.def}
\childdocof{cdocsamp}
%    \end{macrocode}

%\iffalse
%</samplechap1|samplechap2>
%\fi
%
%\iffalse
%<*samplechap1>
%\fi
% Some text for chapter 1:
%    \begin{macrocode}
\section{one}
some text in chapter one
%    \end{macrocode}

%\iffalse
%</samplechap1>
%\fi
% Some text for chapter 2:
%\iffalse
%<*samplechap2>
%\fi
%    \begin{macrocode}
\section{two}
more text in chapter two
%    \end{macrocode}

%\iffalse
%</samplechap2>
%\fi
%
% %%%%%%%%%%%%%%%%%%%%%%%%%%%%%%%%%%%%%%
% \paragraph{Part Include Files.}
%
% The include files are called |cdocspt3.tex| and |cdocspt4.tex|.
%
%\iffalse
%<*samplepart3|samplepart4>
%\fi

% Optional override for |\version| flag:
%    \begin{macrocode}
%%\providecommand{\version}{final}
%    \end{macrocode}

% Include the main document:
%    \begin{macrocode}
\input{childdoc.def}
\childdocby{cdocsamp}
%    \end{macrocode}

%\iffalse
%</samplepart3|samplepart4>
%\fi
%
%\iffalse
%<*samplepart3>
%\fi
% Some text for part 3:
%    \begin{macrocode}
some text in part three
%    \end{macrocode}

%\iffalse
%</samplepart3>
%\fi
% Some text for part 4:
%\iffalse
%<*samplepart4>
%\fi
%    \begin{macrocode}
more text in part four
%    \end{macrocode}

%\iffalse
%</samplepart4>
%\fi
%
% %%%%%%%%%%%%%%%%%%%%%%%%%%%%%%%%%%%%%%
% \paragraph{Forwarding for a Complete Draft.}
%
% The following forwarding file |cdocsdrf.tex|
% compiles the main document in draft mode:
%\iffalse
%<*sampledraft>
%\fi
%    \begin{macrocode}
\def\version{draft}
\input{childdoc.def}
\childdocforward{cdocsamp}
%    \end{macrocode}

%\iffalse
%</sampledraft>
%\fi
%
% %%%%%%%%%%%%%%%%%%%%%%%%%%%%%%%%%%%%%%
% \paragraph{Forwarding for Final Version of the Chapters.}
%
% The following forwarding files |cdocsfn1.tex| and |cdocsfn2.tex|
% (with identical content)
% compile the final versions of the child documents
% |cdocsch1.tex| and |cdocsch2.tex|, respectively:
%\iffalse
%<*samplefinal>
%\fi
%    \begin{macrocode}
\def\version{final}
\input{childdoc.def}
\childdocforwardprefix[cdocsamp]{cdocsfn}{cdocsch}
%    \end{macrocode}

%\iffalse
%</samplefinal>
%\fi
%
% %%%%%%%%%%%%%%%%%%%%%%%%%%%%%%%%%%%%%%
% \paragraph{Command Line Processing.}
%
% The following three command lines generate the output files
% |cdocscld|, |cdocscl1| and |cdocscl2|
% which should be identical to
% |cdocsdrf|, |cdocsch1| and |cdocsfn2|, respectively:
% \begin{center}
% \begin{tabular}{l}
% |latex -jobname cdocscld \|\\
% |  "\def\version{draft}\input{childdoc.def}\childdocforward{cdocsamp}"|\\
% |latex -jobname cdocscl1 \|\\
% |  "\input{childdoc.def}\childdocforward[cdocsamp]{cdocsch1}"|\\
% |latex -jobname cdocscl2 \|\\
% |  "\def\version{final}\input{childdoc.def}\childdocforward{cdocsch2}"|
% \end{tabular}
% \end{center}
% Note that the trailing backslash on each first line
% merely continues the input to the second line
% (for convenient cut ant paste).
% Furthermore, the command |latex| can be replaced by any
% of its alternative versions such as |pdflatex|.
%
% %%%%%%%%%%%%%%%%%%%%%%%%%%%%%%%%%%%%%%%%%%%%%%%%%%%%%%%%%%%%%%%%%%%%%%%%%%%%%%
% %%%%%%%%%%%%%%%%%%%%%%%%%%%%%%%%%%%%%%%%%%%%%%%%%%%%%%%%%%%%%%%%%%%%%%%%%%%%%%
% \section{Implementation}
%\iffalse
%<*package>
%\fi
%
% This section describes the definitions file |childdoc.def|.

% The definitions cannot be loaded using |\usepackage| or |\RequirePackage|
% which has a mechanism to prevent loading a style file more than once.
% When loading the definitions by means of |\input|
% multiple instances have to be prevented manually:
%\iffalse
%This code needs to be before the `\ProvidesFile' directive
%which is defined at the beginning of this file.
%Therefore it is also placed there and commented out here.
%</package>
%<*discard>
%\fi
%    \begin{macrocode}
\ifdefined\childdocmain\endinput\fi
%    \end{macrocode}
%\iffalse
%</discard>
%<*package>
%\fi
%
% \macro{\ifchilddoc}
% \macro{\ifchilddocmanual}
% The conditional |\ifchilddoc| tells whether a
% child (true) or main (false) document is being compiled.
% The conditional |\ifchilddocmanual| tells whether
% the |\includeonly| mechanism is used (false) or
% the selection of child files must be performed manually (true).
% The definitions initialise to false:
%    \begin{macrocode}
\newif\ifchilddoc
\newif\ifchilddocmanual
%    \end{macrocode}

% \macro{\childdocname}
% \macro{\childdocjob}
% The macro |\childdocname| stores the name of the main document
% to be compiled. The macro |\childdocjob| stores the name of
% the document on which the \LaTeX{} compiler was originally invoked.
% The content of |\jobname| cannot be compared
% to filenames specified in the source due to different catcodes.
% The following code rescans |\jobname|, stores the result
% in |\childdocname| and saves a copy in |\childdocjob|:
%    \begin{macrocode}
\edef\childdocname{\scantokens\expandafter{\jobname\noexpand}}
\let\childdocjob\childdocname
%    \end{macrocode}

% \macro{\childdocdisable}
% The macro |\childdocdisable| prevents the main file
% from being processed more than once.
% At this stage, the main document command |\childdocmain|
% is assumed to be called once again where it should do nothing.
% Any subsequent call to it should prevent
% a secondary processing of the main document
% It overwrites the forwarding commands
% |\childdocof| and |\childdocforward|
% with empty macros to prevent further inclusions of the main document:
%    \begin{macrocode}
\newcommand{\childdocdisable}
{
  \renewcommand{\childdocmain}[1]{\renewcommand{\childdocmain}[1]{\endinput}}
  \renewcommand{\childdocof}[1]{}
  \renewcommand{\childdocby}[2][]{}
  \renewcommand{\childdocforward}[2][]{}
  \renewcommand{\childdocdisable}{}
}
%    \end{macrocode}

% \macro{\childdocmain}
% The macro |\childdocmain| is to be called at the top of the main file
% with nothing or the main filename (without extension) as argument.
% First, it breaks loops.
% If the argument is not empty and does not match |\childdocname|
% (which is set by the first inclusion of |childdoc.def|),
% |\ifchilddoc| is set to true, |\includeonly| is applied to the child file
% and |\jobname| is set to the main file
% (for proper handling of |.aux| files):
%    \begin{macrocode}
\newcommand{\childdocmain}[1]
{
  \childdocdisable\childdocmain{}
  \if?#1?\else
    \begingroup
      \def\childdoctmp{#1}
      \ifx\childdoctmp\childdocname
        \def\childdoctmp{}
      \else
        \def\childdoctmp
        {
          \childdoctrue
          \includeonly{\childdocname}
          \def\childdocjob{#1}
          \def\jobname{#1}
        }
      \fi
      \expandafter
    \endgroup
    \childdoctmp
  \fi
}
%    \end{macrocode}

% \macro{\childdocof}
% The command |\childdocof| redirects
% compilation to the main file |#1|.
%    \begin{macrocode}
\newcommand{\childdocof}[1]
{
  \childdocdisable
  \childdoctrue
  \includeonly{\childdocname}
  \def\jobname{#1}
  \def\childdocjob{#1}
  \input{#1}
}
%    \end{macrocode}

% \macro{\childdocby}
% The command |\childdocby| ....
%    \begin{macrocode}
\newcommand{\childdocby}[2][]
{
  \childdocdisable
  \childdoctrue
  \childdocmanualtrue
  \if?#1?\else
    \def\jobname{#2}
  \fi
  \def\childdocjob{#2}
  \input{#2}
  \endinput
}
%    \end{macrocode}

% \macro{\childdocforward}
% The command |\childdocforward| redirects
% compilation to the main file or
% (if the optional argument is given) a child file.
% Parameters are set as if the main file
% or a child file starting with |\childdocof| was compiled.
% Then compilation is handed over to the main file:
%    \begin{macrocode}
\newcommand{\childdocforward}[2][]
{
  \begingroup
    \if?#1?
      \def\childdoctmp
      {
        \def\childdocname{#2}
        \def\childdocjob{#2}
        \def\jobname{#2}
        \input{#2}
        \endinput
      }
    \else
      \def\childdoctmp
      {
        \childdocdisable
        \def\childdocname{#2}
        \childdoctrue
        \includeonly{#2}
        \def\childdocjob{#1}
        \def\jobname{#1}
        \input{#1}
        \endinput
      }
    \fi
    \expandafter
  \endgroup
  \childdoctmp
}
%    \end{macrocode}

% \macro{\childdocforwardprefix}
% The command |\childdocforwardprefix| redirects
% compilation to the main or a child file by means of a pattern.
% The prefix |#1| in the current filename is replaced by |#2|
% and the suffix of the current filename is kept
% (it is assumed that the filename does not contain the substring `|~~~|'
% which is used as a delimiter).
% Compilation is handed over to the new file by |\childdocforward|:
%    \begin{macrocode}
\newcommand{\childdocforwardprefix}[3][]
{
  \begingroup
    \def\childdocextract #2##1~~~{\def\childdoctmp{\childdocforward[#1]{#3##1}}}
    \expandafter\childdocextract\childdocname~~~
    \expandafter
  \endgroup
  \childdoctmp
}
%    \end{macrocode}

% \macro{\childdoc}
% The deprecated macro |\childdoc| is a legacy version of |\childdocmain|:
%    \begin{macrocode}
\newcommand{\childdoc}{\childdocmain}
%    \end{macrocode}

% \macro{\childdocredirect}
% The deprecated macro |\childdocredirect| is a legacy version
% of |\childdocforward| and |\childdocforwardprefix|:
%    \begin{macrocode}
\newcommand{\childdocredirect}[2][]
{
  \begingroup
    \if?#1?
      \def\childdoctmp{\childdocforward{#2}}
    \else
      \def\childdoctmp{\childdocforwardprefix{#1}{#2}}
    \fi
    \expandafter
  \endgroup
  \childdoctmp
}
%    \end{macrocode}

%\iffalse
%</package>
%\fi
%
\endinput
\childdocforward{cdocsamp}"|\\
% |latex -jobname cdocscl1 \|\\
% |  "% \iffalse
%
% childdoc.dtx Copyright (C) 2017-2018 Niklas Beisert
%
% This work may be distributed and/or modified under the
% conditions of the LaTeX Project Public License, either version 1.3
% of this license or (at your option) any later version.
% The latest version of this license is in
%   http://www.latex-project.org/lppl.txt
% and version 1.3 or later is part of all distributions of LaTeX
% version 2005/12/01 or later.
%
% This work has the LPPL maintenance status `maintained'.
%
% The Current Maintainer of this work is Niklas Beisert.
%
% This work consists of the files childdoc.dtx and childdoc.ins
% and the derived files childdoc.def and cdocsamp.tex with
% cdocsch1.tex, cdocsch2.tex, cdocsdrf.tex, cdocsfn1.tex, cdocsfn2.tex.
%
%<package>\ifdefined\childdocmain\endinput\fi
%<package>\ProvidesFile{childdoc.def}[2018/12/30 v2.0 child document driver]
%<samplemain>\ProvidesFile{cdocsamp.tex}[2018/12/30 v2.0 sample for childdoc]
%<*driver>
%\ProvidesFile{childdoc.drv}[2018/12/30 v2.0 childdoc reference manual file]
\PassOptionsToClass{10pt,a4paper}{article}
\documentclass{ltxdoc}

\usepackage[margin=35mm]{geometry}
\usepackage{hyperref}
\usepackage{hyperxmp}
\usepackage[usenames]{color}

\hypersetup{colorlinks=true}
\hypersetup{pdfstartview=FitH}
\hypersetup{pdfpagemode=UseNone}
\hypersetup{pdfsource={}}
\hypersetup{pdflang={en-UK}}
\hypersetup{pdfcopyright={Copyright 2017-2018 Niklas Beisert.
  This work may be distributed and/or modified under the
  conditions of the LaTeX Project Public License, either version 1.3
  of this license or (at your option) any later version.}}
\hypersetup{pdflicenseurl={http://www.latex-project.org/lppl.txt}}
\hypersetup{pdfcontactaddress={ETH Zurich, ITP, HIT K,
  Wolfgang-Pauli-Strasse 27}}
\hypersetup{pdfcontactpostcode={8093}}
\hypersetup{pdfcontactcity={Zurich}}
\hypersetup{pdfcontactcountry={Switzerland}}
\hypersetup{pdfcontactemail={nbeisert@itp.phys.ethz.ch}}
\hypersetup{pdfcontacturl={http://people.phys.ethz.ch/\xmptilde nbeisert/}}

\newcommand{\secref}[1]{\hyperref[#1]{section \ref*{#1}}}

\parskip1ex
\parindent0pt
\let\olditemize\itemize
\def\itemize{\olditemize\parskip0pt}

\begin{document}

\title{The \textsf{childdoc} Package}
\hypersetup{pdftitle={The childdoc Package}}
\author{Niklas Beisert\\[2ex]
  Institut f\"ur Theoretische Physik\\
  Eidgen\"ossische Technische Hochschule Z\"urich\\
  Wolfgang-Pauli-Strasse 27, 8093 Z\"urich, Switzerland\\[1ex]
  \href{mailto:nbeisert@itp.phys.ethz.ch}
  {\texttt{nbeisert@itp.phys.ethz.ch}}}
\hypersetup{pdfauthor={Niklas Beisert}}
\hypersetup{pdfsubject={Manual for the LaTeX2e Package childdoc}}
\date{30 December 2018, \textsf{v2.0}}
\maketitle

\begin{abstract}\noindent
\textsf{childdoc} is a \LaTeXe{} package
that enables the direct compilation
of document sections included by |\include|
to individual files.
\end{abstract}

\begingroup
\parskip0ex
\tableofcontents
\endgroup

%%%%%%%%%%%%%%%%%%%%%%%%%%%%%%%%%%%%%%%%%%%%%%%%%%%%%%%%%%%%%%%%%%%%%%%%%%%%%%%%
%%%%%%%%%%%%%%%%%%%%%%%%%%%%%%%%%%%%%%%%%%%%%%%%%%%%%%%%%%%%%%%%%%%%%%%%%%%%%%%%
\section{Introduction}

\LaTeX{} provides a mechanism to structure a large document (such as a book)
into a main file and several child files (containing the chapters)
using the |\include| command.
This mechanism is beneficial for documents
which span hundreds of pages in order to
make the source file(s) more manageable.
Moreover, compilation can be restricted to
selected child files by means of the |\includeonly| command.
The latter feature can be used to reduce the compilation time while editing
(this was significantly more useful in the earlier days of \LaTeX{})
or to generate a smaller document which is easier to navigate.
Another application of |\includeonly| is to generate
documents consisting of selected parts of the complete document.

However, there are a few drawbacks of the plain |\include| mechanism:
\begin{itemize}
\item
The child files cannot be compiled on their own,
they can only be compiled via the main file.
A naive editing environment
(such as a text editor with an option
to have the current file processed by \LaTeX)
may require one to switch to the main file before compiling;
attempting to compile the child file produces errors.
\item
The main file must be modified (each time)
to adjust the |\includeonly| command
to the present needs. This easily leaves the main file in a messy state.
\item
The generated document will always carry the filename
of the main document. This is inconvenient if
several child files are to be compiled and
to be kept for distribution.
\end{itemize}

The present package provides a simple interface
to make child files individually compilable by \LaTeX{}.
Compiling a child file then has the same effect as compiling
the main file with an |\includeonly| command
to select the appropriate child.
Moreover the generated document will carry the name of the child
rather than the main file.
This resolves all three above issues.

This feature is meant to make the editing of books,
thesis documents and lecture notes somewhat more convenient.
However, the package can also be used efficiently for
composing a series of documents (such as exercise sheets)
which are typically distributed individually.
It then assists the author in generating the individual documents
(potentially in different versions)
as well as a document containing the collected series.
Another application is in developing style files
or other kinds of included material
where compilation of the style file could redirect
to a sample or test file.

%%%%%%%%%%%%%%%%%%%%%%%%%%%%%%%%%%%%%%%%%%%%%%%%%%%%%%%%%%%%%%%%%%%%%%%%%%%%%%%%
%%%%%%%%%%%%%%%%%%%%%%%%%%%%%%%%%%%%%%%%%%%%%%%%%%%%%%%%%%%%%%%%%%%%%%%%%%%%%%%%
\section{Usage}

First of all, the package \textsf{childdoc} is \emph{not} a standard
\LaTeXe{} |.sty| style file! Therefore it needs to be invoked in
a non-standard way.

%%%%%%%%%%%%%%%%%%%%%%%%%%%%%%%%%%%%%%%%%%%%%%%%%%%%%%%%%%%%%%%%%%%%%%%%%%%%%%%%
\subsection{Included Files}
\label{sec:include}

%%%%%%%%%%%%%%%%%%%%%%%%%%%%%%%%%%%%%%%%
\DescribeMacro{\childdocmain}
To use the package, add the commands
\begin{center}
\begin{tabular}{l}
|\input{childdoc.def}|\\
|\childdocmain{}|\\
\end{tabular}
\end{center}
at the very top of the main \LaTeX{} file,
in particular \emph{before} the |\documentclass| statement!
The argument of |\childdocmain| should be left empty
(but it must be present).

%%%%%%%%%%%%%%%%%%%%%%%%%%%%%%%%%%%%%%%%
\DescribeMacro{\childdocof}
Furthermore, add the commands
\begin{center}
\begin{tabular}{l}
|\input{childdoc.def}|\\
|\childdocof{|\textit{main}|}|\\
\end{tabular}
\end{center}
at the top of every child file \textit{child}
which is included by |\include{|\textit{child}|}|
from within the main file
(or at least for those files to be compiled individually).
The argument \textit{main} must be the filename of the main file.

There are a couple of
considerations in setting up the main and child documents:

%%%%%%%%%%%%%%%%%%%%%%%%%%%%%%%%%%%%%%%%
\paragraph{Restrictions.}

Please note the following restrictions:
\begin{itemize}
\item
|\childdocmain| must be called with one argument \textit{main}
to ensure compatibility with earlier version of the package.
It must either be empty (|\childdocmain{}|)
or precisely match the filename of the main file in which it is specified.
See \secref{sec:detection} for further information.
\item
The filename \textit{main} must be specified without the |.tex| extension.
\item
The filename \textit{main} is case sensitive
(even in case-insensitive file systems)
due to internal string comparison.
\item
The argument \textit{main} should be fully expanded, it cannot be a macro.
\item
Subdirectories and special characters should be avoided in filenames.
\item
The command |\childdocmain{|\textit{main}|}| must be followed by a whitespace.
It should not be followed immediately by another command
or by a comment mark `|%|'.
This is because the \TeX{} parser reads the token immediately following
the argument of |\childdocmain| and puts it
at the beginning of every child section;
however, a white\-space is ignored.
\end{itemize}

%%%%%%%%%%%%%%%%%%%%%%%%%%%%%%%%%%%%%%%%
\paragraph{Content of Main File.}

It is advisable to place all content in the child files included by |\include|.
Any output contained in the main file will appear in all child documents
unless suppressed manually;
it cannot be suppressed automatically by the |\includeonly| directive
and thus should normally be avoided.
A method to include some content in the main file
by means of conditional processing is described in \secref{sec:conditional}.

%%%%%%%%%%%%%%%%%%%%%%%%%%%%%%%%%%%%%%%%
\paragraph{Page Numbering.}

When only a part of the document is compiled,
the appropriate numbering of pages
(as well as other status parameters)
is determined from the |.aux| files.
The latter contain information from previous passes.
However this information needs to propagate through
all intermediate child documents.
Therefore the page numbering in child documents may well
be inconsistent until the complete document is compiled at least once.

A useful (if unconventional) way to always ensure a consistent
page numbering is to restart the numbering in each child document
and denote the pages by `\textit{child}|.|\textit{page}'
where \textit{child} represents the chapter/section number of the child file.
This can be achieved by the command
|\numberwithin{page}{|\textit{child}|}|
of the \textsf{amsmath} package
where \textit{child} can be |chapter| or |section|
depending on the chosen structuring.
Alternatively, one can modify the macro |\thepage| appropriately
and reset the counter |page| at the start of each child file.

%%%%%%%%%%%%%%%%%%%%%%%%%%%%%%%%%%%%%%%%%%%%%%%%%%%%%%%%%%%%%%%%%%%%%%%%%%%%%%%%
\subsection{Conditional Processing}
\label{sec:conditional}

The package provides a mechanism to compile different versions
of a document. To customise the versions further some conditional processing
can come in handy to distinguish which version is being compiled.
The package provides two macros to describe the compilation context:

%%%%%%%%%%%%%%%%%%%%%%%%%%%%%%%%%%%%%%%%
\DescribeMacro{\ifchilddoc}
The conditional |\ifchilddoc| distinguishes between the compilation of
child documents and the main document:
%
\begin{center}
|\ifchilddoc |\textit{child-code}| |[|\||else |\textit{main-code}]| \||fi|
\end{center}

%%%%%%%%%%%%%%%%%%%%%%%%%%%%%%%%%%%%%%%%
\DescribeMacro{\childdocname}
\DescribeMacro{\childdocjob}
The macro |\childdocname| contains the filename (without extension)
of the main or child file being processed.
Note that |\childdocjob| will always contain the name of the main file.

%%%%%%%%%%%%%%%%%%%%%%%%%%%%%%%%%%%%%%%%
\paragraph{Title Page.}

Conditional processing can be used to include a title or banner page
in the main document when proper precautions are taken.
Importantly, the code in the main file should ensure that the page counter
(as well as other status parameters which are stored in the |.aux| files)
takes the same value after the conditional processing.
Otherwise the page numbers may take divergent values
depending on which part is compiled.

For example, a title page could be declared by:
%
\begin{center}
\begin{tabular}{l}
|\ifchilddoc\||else|\\
|\addtocounter{page}{-1}|\\
\textit{code for title page}\\
|\newpage|\\
|\||fi|
\end{tabular}
\end{center}
%
A banner page for the child documents can be generated by:
%
\begin{center}
\begin{tabular}{l}
|\ifchilddoc|\\
|\addtocounter{page}{-1}|\\
\textit{code for banner page}\\
|\newpage|\\
|\||fi|
\end{tabular}
\end{center}
%
Here one could write a message such as:
\begin{center}
|This is the part \childdocname{} of \childdocjob{}.|
\end{center}

%%%%%%%%%%%%%%%%%%%%%%%%%%%%%%%%%%%%%%%%%%%%%%%%%%%%%%%%%%%%%%%%%%%%%%%%%%%%%%%%
\subsection{Flags}
\label{sec:flags}

The package makes it easy to generate different versions
of the main or child documents.
To this end compilation flags can be defined
and assigned different default values.
They will be particularly useful in conjunction
with the forwarding mechanism described in \secref{sec:forward}.

For example, it may be useful to have a flag |\version|
which can be set to |draft| or |final|.
The document source will contain some conditional code
depending on the value of |\version|.
Suppose further, the flag should default to |final| for the main file
and to |draft| for child files
which is a natural assignment for editing the document.
This is achieved by placing the following code
in the preamble of the main document
(below the |\childdocmain| directive):
%
\begin{center}
\begin{tabular}{l}
|\ifchilddoc|\\
|\providecommand{\version}{draft}|\\
|\||else|\\
|\providecommand{\version}{final}|\\
|\||fi|
\end{tabular}
\end{center}
%
The definition by |\providecommand| makes sure
that previous definitions are not overwritten.
Further statements |\providecommand{\version}{...}|
can thus be added before the above code to override it.

For the main file, one might add a line
(between |\childdocmain| and the above block)
%
\begin{center}
|%\ifchilddoc\||else\providecommand{\version}{draft}\||fi|
\end{center}
%
which can be uncommented to produce a draft version.
Likewise one can add a line to the very top of a child file
(above the |\childdocof{|\textit{main}|}| directive)
%
\begin{center}
|%\providecommand{\version}{final}|
\end{center}
%
which can be uncommented to produce the final version of this child document.

%%%%%%%%%%%%%%%%%%%%%%%%%%%%%%%%%%%%%%%%%%%%%%%%%%%%%%%%%%%%%%%%%%%%%%%%%%%%%%%%
\subsection{Forwarding}
\label{sec:forward}

Different versions of the main or child documents
using compilation flags as described in \secref{sec:flags}
can be (permanently) stored in different files
for convenient compilation, viewing and distribution.
To this end, the package defines a command
to pass on compilation to a different file:

%%%%%%%%%%%%%%%%%%%%%%%%%%%%%%%%%%%%%%%%
\DescribeMacro{\childdocforward}
The command |\childdocforward| redirects processing to
another source file:
%
\begin{center}
\begin{tabular}{l}
|\input{childdoc.def}|\\
|\childdocforward[|\textit{main}|]{|\textit{dest}|}|\\
\end{tabular}
\end{center}
%
The argument \textit{dest} is the destination file
(without extension).
It should be the main file or one of the child files.
Note that further \textsf{childdoc} directives
such as |\childdocof| and |\childdocforward|
in the indicated file will be processed in this form.
The optional argument \textit{main}
passes on directly to the main file \textit{main}
while pretending to compile the child \textit{dest}.
This form behaves as if \textit{dest}
issues |\childdocof{|\textit{main}|}| right away,
and no further \textsf{childdoc} directives will be processed.

%%%%%%%%%%%%%%%%%%%%%%%%%%%%%%%%%%%%%%%%
\DescribeMacro{\...prefix}
In the alternative form |\childdocforwardprefix|,
%
\begin{center}
\begin{tabular}{l}
|\input{childdoc.def}|\\
|\childdocforwardprefix[|\textit{main}|]{|\textit{prefix}|}{|\textit{dest}|}|
\end{tabular}
\end{center}
%
the destination file is determined by a pattern
depending on the current file:
To make this work, the current file must be called
`{\textit{prefix}\hspace{0.2em}\textit{suffix}}'
with \textit{prefix} matching precisely the argument.
Processing is then passed on to the file
`{\textit{dest}\hspace{0.2em}\textit{suffix}}'.
Surely, the same effect is achieved by
directly specifying the
argument `{\textit{dest}\hspace{0.2em}\textit{suffix}}'
in the first form.
However, that requires to set up a different file
for each child. With the alternative form of the command
all these files can have exactly the same content
which simplifies setting them up and maintaining them.

For example, the following file |draft.tex|
with a compilation flag |\version| as described in \secref{sec:flags}
compiles the main document as a draft:
%
\begin{center}
\begin{tabular}{l}
|\def\version{draft}|\\
|\input{childdoc.def}|\\
|\childdocforward{|\textit{main}|}|
\end{tabular}
\end{center}
%
Likewise, the following files |final|\textit{nn}|.tex|
compile the final version of the child document
|child|\textit{nn}|.tex|:
%
\begin{center}
\begin{tabular}{l}
|\def\version{final}|\\
|\input{childdoc.def}|\\
|\childdocforwardprefix{final}{child}|
\end{tabular}
\end{center}
%

Note that when several versions of a main file and/or of each child file
are to be generated, it may be convenient to set up a |Makefile| or
shell script to automatise the process.

%%%%%%%%%%%%%%%%%%%%%%%%%%%%%%%%%%%%%%%%%%%%%%%%%%%%%%%%%%%%%%%%%%%%%%%%%%%%%%%%
\subsection{Command Line Processing}
\label{sec:commandline}

The effect of redirection files can also be achieved by invoking
the \LaTeX{} compiler with a more elaborate command line.
Most conveniently this should be done as part
of a shell script or a |Makefile|.

When using \textsf{childdoc} in the main file, the following
command lines effectively perform a redirection
(note that depending on the shell being used,
backslashes may have to be doubled: `|\|' $\to$ `|\\|'):
%
\begin{center}
|... -jobname "|\textit{target}|" |\\|"|[\textit{flags}]%
|\input{childdoc.def}\childdocforward[|\textit{main}|]{|\textit{dest}|}"|
\end{center}
%
Here \textit{target} is the name of the output file,
\textit{main} is the name of the main file
and \textit{dest} is the name of the main or child file to be processed
(all filenames without extensions).
The optional argument \textit{main} can be omitted
if \textit{main} matches \textit{dest}.
Optionally, compilation \textit{flags} can be defined via |\def| commands.
This command line makes the \TeX{} engine believe
it is compiling the file \textit{target}
whose content is specified as the latter parameter.
The provided code then forwards the processing to
\textit{main} or \textit{dest} as described in \secref{sec:forward}.

%%%%%%%%%%%%%%%%%%%%%%%%%%%%%%%%%%%%%%%%%%%%%%%%%%%%%%%%%%%%%%%%%%%%%%%%%%%%%%%%
\subsection{Include by Input}
\label{sec:input}

Including child documents by |\include| has some restrictions by design.
Most notably, the content of a child document always occupies
its own set of pages; pages cannot be shared between child documents.
Usually, this behaviour makes perfect sense
because each child document contain an essential part of the document.
However, in some situations it may be desirable to compose
a document from a collection of parts
without having mandatory page breaks between then.
For this case, the package
provides a mechanism to include parts
by |\input| which can also be processed individually.
However, by construction this mechanism
requires manual handling of the content to be output.

%%%%%%%%%%%%%%%%%%%%%%%%%%%%%%%%%%%%%%%%
\DescribeMacro{\ifchilddocmanual}
The main file should be prepared as usual, see \secref{sec:include}.
However, the document body must make a distinction
between processing of an individual part and of the main document, e.g.:
%
\begin{center}
\begin{tabular}{l}
|\ifchilddocmanual|\\
|\input{\childdocname}|\\
|\||else|\\
\textit{document body with }|\input{|\textit{part}|}|\\
|\||fi|
\end{tabular}
\end{center}
%
The conditional |\ifchilddocmanual| is true whenever
a part to be included by |\input| is being compiled,
and the name of the part is stored in |\childdocname|.

%%%%%%%%%%%%%%%%%%%%%%%%%%%%%%%%%%%%%%%%
\DescribeMacro{\childdocby}
Each part to be included by |\input| should start with:
%
\begin{center}
\begin{tabular}{l}
|\input{childdoc.def}|\\
|\childdocby{|\textit{main}|}|\\
\end{tabular}
\end{center}
%
The directive |\childdocby| is similar to |\childdocof|
described in \secref{sec:include},
but the subsequent selection of content must be done manually.
To that end, both |\ifchilddoc| and |\ifchilddocmanual|
will be true upon processing of a part,
and the name of the part is stored in |\childdocname|.
Note that |\jobname| will be set to the filename of the current part
so that each part receives an individual |.aux| file
that does not interfere with the |.aux| file(s) of the main document.
This behaviour can be altered by the alternative form
|\childdocby[*]{|\textit{main}|}| (with a non-empty optional argument)
which uses the |.aux| file of the main document
by setting |\jobname| to \textit{main}.

%%%%%%%%%%%%%%%%%%%%%%%%%%%%%%%%%%%%%%%%%%%%%%%%%%%%%%%%%%%%%%%%%%%%%%%%%%%%%%%%
\subsection{Driver Development}
\label{sec:driver}

The \textsf{childdoc} mechanism can also be use for the development
of definition files such as \LaTeX{} styles or classes.
This case differs from the above setup with multiple parts
included by |\include| in that no |\includeonly| should be invoked.
This can be achieved by starting the include file
(before |\ProvidesPackage|) with:
%
\begin{center}
\begin{tabular}{l}
|\input{childdoc.def}|\\
|\childdocforward{|\textit{main}|}|\\
\end{tabular}
\end{center}
%
or alternatively with:
%
\begin{center}
\begin{tabular}{l}
|\input{childdoc.def}|\\
|\childdocby{|\textit{main}|}|\\
\end{tabular}
\end{center}
%
Both forms have slightly different effects as described above.
The main file is prepared as usual, see \secref{sec:include}.

%%%%%%%%%%%%%%%%%%%%%%%%%%%%%%%%%%%%%%%%%%%%%%%%%%%%%%%%%%%%%%%%%%%%%%%%%%%%%%%%
\subsection{Legacy Detection}
\label{sec:detection}

The directive |\childdocmain| in the main file can detect
whether the complete document or merely a child is to be compiled
even without using the directive |\childdocof|.
This method is deprecated because it is less robust
and there is no compelling reason to use it;
it is merely provided for backward compatibility
and it may be removed in future versions.

If the detection mechanism is to be used,
it is mandatory to correctly specify
the filename of the main file as the argument of |\childdocmain|:
%
\begin{center}
\begin{tabular}{l}
|\input{childdoc.def}|\\
|\childdocmain{|\textit{main}|}|\\
\end{tabular}
\end{center}
%
If |\jobname| does not match the argument \textit{main} of |\childdocmain|,
it is assumed that |\jobname| points to the child file to be compiled.
When using |\childdocmain| with the main file specified as argument,
it suffices to start a child file
with just |\input{|\textit{main}|}|
without loading of the package and using |\childdocof|.
If instead all processing is done
with the appropriate \textsf{childdoc} directives,
the argument of \textit{main} of |\childdocmain| can be empty.

An alternative version of the command line processing described
in \secref{sec:commandline} using the detection mechanism reads:
%
\begin{center}
|... -jobname "|\textit{target}|" "|[\textit{flags}]%
[|\def\jobname{|\textit{dest}|}|]|\input{|\textit{main}|}"|
\end{center}

%%%%%%%%%%%%%%%%%%%%%%%%%%%%%%%%%%%%%%%%%%%%%%%%%%%%%%%%%%%%%%%%%%%%%%%%%%%%%%%%
\subsection{Manual Code}
\label{sec:manual}

In case one cannot be certain whether the definitions file |childdoc.def|
is installed on the target \TeX{} distribution
and one prefers not to ship it,
it is conceivable to paste a few relevant commands into the sources.

To that end, drop all statements |\input{childdoc.def}|
and perform the replacements as outlined below.
Instead of |\childdocmain{|\textit{main}|}| add the following code
to the top of the main file:
%
\begin{center}
\begin{tabular}{l}
|\||ifdefined\childdocname\endinput\||fi\newif\ifchilddoc|\\
|\edef\childdocname{\scantokens\expandafter{\jobname\noexpand}}|\\
|\def\childdocmain{|\textit{main}|}\||ifx\childdocmain\childdocname\||else|\\
|\childdoctrue\includeonly{\childdocname}\let\jobname\childdocmain\||fi|\\
\end{tabular}
\end{center}
%
Instead of |\childdocof{|\textit{main}|}| just include the main file
at the top of each child file:
%
\begin{center}
|\input{|\textit{main}|}|
\end{center}
%
A simple redirection |\childdocforward{|\textit{dest}|}| is achieved by:
%
\begin{center}
|\def\jobname{|\textit{dest}|}\input{\jobname}|
\end{center}
%
The redirection with prefix
|\childdocforwardprefix[|\textit{prefix}|]{|\textit{dest}|}|
is accomplished by:
%
\begin{center}
\begin{tabular}{l}
|{\edef\jobname{\scantokens\expandafter{\jobname\noexpand}}|\\
|\def\redirectjob |\textit{prefix}|#1~~~{\gdef\jobname{|\textit{dest}|#1}}|\\
|\expandafter\redirectjob\jobname~~~}\input{\jobname}|
\end{tabular}
\end{center}

In an alternative approach,
child documents can be compiled by a specific command line
without additional code or specific definitions:
%
\begin{center}
|... -jobname "|\textit{target}|" "|[\textit{flags}]%
|\includeonly{|\textit{dest}|}\input{|\textit{main}|}"|
\end{center}
%

%%%%%%%%%%%%%%%%%%%%%%%%%%%%%%%%%%%%%%%%%%%%%%%%%%%%%%%%%%%%%%%%%%%%%%%%%%%%%%%%
%%%%%%%%%%%%%%%%%%%%%%%%%%%%%%%%%%%%%%%%%%%%%%%%%%%%%%%%%%%%%%%%%%%%%%%%%%%%%%%%
\section{Information}

%%%%%%%%%%%%%%%%%%%%%%%%%%%%%%%%%%%%%%%%%%%%%%%%%%%%%%%%%%%%%%%%%%%%%%%%%%%%%%%%
\subsection{Copyright}

Copyright \copyright{} 2017--2018 Niklas Beisert

This work may be distributed and/or modified under the
conditions of the \LaTeX{} Project Public License, either version 1.3
of this license or (at your option) any later version.
The latest version of this license is in
  \url{http://www.latex-project.org/lppl.txt}
and version 1.3 or later is part of all distributions of \LaTeX{}
version 2005/12/01 or later.

This work has the LPPL maintenance status `maintained'.

The Current Maintainer of this work is Niklas Beisert.

This work consists of the files |README.txt|, |childdoc.ins| and |childdoc.dtx|
as well as the derived files |childdoc.def|, |cdocsamp.tex|
with |cdocsch1.tex|, |cdocsch2.tex|, |cdocspt3.tex|, |cdocspt4.tex|,
|cdocsdrf.tex|, |cdocsfn1.tex|, |cdocsfn2.tex|
as well as |childdoc.pdf|.

%%%%%%%%%%%%%%%%%%%%%%%%%%%%%%%%%%%%%%%%%%%%%%%%%%%%%%%%%%%%%%%%%%%%%%%%%%%%%%%%
\subsection{Files and Installation}

The package consists of the files:
%
\begin{center}
\begin{tabular}{ll}
    |README.txt|   & readme file \\
    |childdoc.ins| & installation file \\
    |childdoc.dtx| & source file \\
    |childdoc.def| & definition file \\
    |cdocsamp.tex| & sample main file \\
    |cdocsch1.tex| & sample include file \\
    |cdocsch2.tex| & sample include file \\
    |cdocspt3.tex| & sample part file \\
    |cdocspt4.tex| & sample part file \\
    |cdocsdrf.tex| & sample redirection file \\
    |cdocsfn1.tex| & sample redirection file \\
    |cdocsfn2.tex| & sample redirection file \\
    |childdoc.pdf| & manual
\end{tabular}
\end{center}
%
The distribution consists of the files
|README.txt|, |childdoc.ins| and |childdoc.dtx|.
%
\begin{itemize}
\item
Run (pdf)\LaTeX{} on |childdoc.dtx|
to compile the manual |childdoc.pdf| (this file).
\item
Run \LaTeX{} on |childdoc.ins| to create the definitions file |childdoc.def|
and the sample |cdocsamp.tex| with include files
|cdocsch1.tex|, |cdocsch2.tex|, |cdocspt3.tex|, |cdocspt4.tex|,
|cdocsdrf.tex|, |cdocsfn1.tex|, |cdocsfn2.tex|.
Then copy the file |childdoc.def| to an appropriate directory of your \LaTeX{}
distribution, e.g.\ \textit{texmf-root}|/tex/latex/childdoc|.
\end{itemize}

%%%%%%%%%%%%%%%%%%%%%%%%%%%%%%%%%%%%%%%%%%%%%%%%%%%%%%%%%%%%%%%%%%%%%%%%%%%%%%%%
\subsection{Related CTAN Packages}

There are several other packages which offer a similar functionality:
%
\begin{itemize}
\item
The packages
\href{http://ctan.org/pkg/docmute}{\textsf{docmute}},
\href{http://ctan.org/pkg/includex}{\textsf{includex}} and
\href{http://ctan.org/pkg/standalone}{\textsf{standalone}}
provide commands to include only the document body of
a child file thus allowing both files to be compiled individually.
\item
The packages \href{http://ctan.org/pkg/subdocs}{\textsf{subdocs}}
and \href{http://ctan.org/pkg/subfiles}{\textsf{subfiles}}
provide structures in which the main and child documents can be
encapsulated and allowing them to be compiled individually.
The inclusion mechanism is different from the conventional |\include|.
\item
The package \href{http://ctan.org/pkg/combine}{\textsf{combine}}
is an elaborate solution to combine several documents into one.
\end{itemize}
%
See also the CTAN topic \href{http://ctan.org/topic/subdocs}{\textsf{subdocs}}
for further related packages.
The present package differs from the above solutions in that
a document structure constructed with the conventional |\include| mechanism
just needs two extra commands at the top of every file
such that all constituent files can be compiled individually.

%%%%%%%%%%%%%%%%%%%%%%%%%%%%%%%%%%%%%%%%%%%%%%%%%%%%%%%%%%%%%%%%%%%%%%%%%%%%%%%%
%\subsection{Feature Suggestions}
%
%The following is a list of features which may be useful for future
%versions of this package:
%%
%\begin{itemize}
%\item
%\ldots
%\end{itemize}

%%%%%%%%%%%%%%%%%%%%%%%%%%%%%%%%%%%%%%%%%%%%%%%%%%%%%%%%%%%%%%%%%%%%%%%%%%%%%%%%
\subsection{Revision History}

%%%%%%%%%%%%%%%%%%%%%%%%%%%%%%%%%%%%%%%%
\paragraph{v2.0:} 2018/12/30

\begin{itemize}
\item
immediate forward processing
\item
added |\childdocby| mechanism
\item
manual restructured
\end{itemize}

%%%%%%%%%%%%%%%%%%%%%%%%%%%%%%%%%%%%%%%%
\paragraph{v1.6:} 2018/01/17

\begin{itemize}
\item
application for development of include files
\item
corrections to manual
\end{itemize}

%%%%%%%%%%%%%%%%%%%%%%%%%%%%%%%%%%%%%%%%
\paragraph{v1.5:} 2017/05/21

\begin{itemize}
\item
more complete structuring introduced
\item
|\childdocof| introduced
\item
|\childdoc| renamed to |\childdocmain|
\item
|\childredirect| renamed to |\childdocforward| and |\childdocforwardprefix|
and functionality expanded
\end{itemize}

%%%%%%%%%%%%%%%%%%%%%%%%%%%%%%%%%%%%%%%%
\paragraph{v1.0:} 2017/04/27

\begin{itemize}
\item
manual and install package
\item
first version published on CTAN
\end{itemize}

%%%%%%%%%%%%%%%%%%%%%%%%%%%%%%%%%%%%%%%%
\paragraph{v0.6:} 2017/04/26

\begin{itemize}
\item
redirection mechanism added
\end{itemize}

%%%%%%%%%%%%%%%%%%%%%%%%%%%%%%%%%%%%%%%%
\paragraph{v0.5:} 2017/04/26

\begin{itemize}
\item
functionality in definition file
\end{itemize}


%%%%%%%%%%%%%%%%%%%%%%%%%%%%%%%%%%%%%%%%%%%%%%%%%%%%%%%%%%%%%%%%%%%%%%%%%%%%%%%%
%%%%%%%%%%%%%%%%%%%%%%%%%%%%%%%%%%%%%%%%%%%%%%%%%%%%%%%%%%%%%%%%%%%%%%%%%%%%%%%%
%%%%%%%%%%%%%%%%%%%%%%%%%%%%%%%%%%%%%%%%%%%%%%%%%%%%%%%%%%%%%%%%%%%%%%%%%%%%%%%%
\appendix

\settowidth\MacroIndent{\rmfamily\scriptsize 000\ }

 \DocInput{childdoc.dtx}

\end{document}
%</driver>
% \fi
%
% %%%%%%%%%%%%%%%%%%%%%%%%%%%%%%%%%%%%%%%%%%%%%%%%%%%%%%%%%%%%%%%%%%%%%%%%%%%%%%
% %%%%%%%%%%%%%%%%%%%%%%%%%%%%%%%%%%%%%%%%%%%%%%%%%%%%%%%%%%%%%%%%%%%%%%%%%%%%%%
% \section{Sample}
%\iffalse
%<*samplemain>
%\fi
%
% The following presents a sample document
% with two chapters, two parts, a title page,
% a compile flag as well as three forwarding files to set the flag.
% It consists of eight |.tex| files:
% \begin{center}
% \begin{tabular}{ll}
% |cdocsamp.tex|&main file\\
% |cdocsch1.tex|&include file for chapter 1\\
% |cdocsch2.tex|&include file for chapter 2\\
% |cdocspt3.tex|&include file for part 3\\
% |cdocspt4.tex|&include file for part 4\\
% |cdocsdrf.tex|&forwarding file for main file in draft mode\\
% |cdocsfi1.tex|&forwarding file for final version of chapter 1\\
% |cdocsfi2.tex|&forwarding file for final version of chapter 2\\
% \end{tabular}
% \end{center}
% Each of the eight files can be compiled directly by the \LaTeX{} compiler.
%
% %%%%%%%%%%%%%%%%%%%%%%%%%%%%%%%%%%%%%%
% \paragraph{Main File.}
%
% The main file is called |cdocsamp.tex|.
%
% Load the \textsf{childdoc} definitions and
% declare the filename for the main document:
%    \begin{macrocode}
\input{childdoc.def}
\childdocmain{}
%    \end{macrocode}

% Optional override for |\version| flag:
%    \begin{macrocode}
%%\ifchilddoc\else\providecommand{\version}{draft}\fi
%    \end{macrocode}

% Define the default values for the |\version| flag
% (|final| for the main file and |draft| for childs):
%    \begin{macrocode}
\ifchilddoc
\providecommand{\version}{draft}
\else
\providecommand{\version}{final}
\fi
%    \end{macrocode}

% Load the standard document class:
%    \begin{macrocode}
\documentclass[12pt]{article}
%    \end{macrocode}

% Start the document body:
%    \begin{macrocode}
\begin{document}
%    \end{macrocode}

% Declare a title page.
% Print title, part of document being processed and version flag:
%    \begin{macrocode}
\addtocounter{page}{-1}
\begin{center}
{\LARGE\bfseries{}childdoc example\par}
\vspace{1cm}
\ifchilddoc
\ifchilddocmanual part\else chapter\fi:
`\childdocname' of `\childdocjob'\par
\else
main document: `\childdocjob'\par
\fi
version: \version\par
\end{center}
\newpage
%    \end{macrocode}

% Manually include selected file,
% otherwise process as usual:
%    \begin{macrocode}
\ifchilddocmanual
\section*{part `\childdocname'}
\input{\childdocname}
\else
%    \end{macrocode}

% Include the two chapters:
%    \begin{macrocode}
\include{cdocsch1}
\include{cdocsch2}
%    \end{macrocode}

% Include the two parts unless only chapters should be displayed:
%    \begin{macrocode}
\ifchilddoc\else
\section{part three}
\input{cdocspt3}
\section{part four}
\input{cdocspt4}
\fi
%    \end{macrocode}

% Process as usual until here:
%    \begin{macrocode}
\fi
%    \end{macrocode}

% End of document body:
%    \begin{macrocode}
\end{document}
%    \end{macrocode}
%\iffalse
%</samplemain>
%\fi
%
% %%%%%%%%%%%%%%%%%%%%%%%%%%%%%%%%%%%%%%
% \paragraph{Chapter Include Files.}
%
% The include files are called |cdocsch1.tex| and |cdocsch2.tex|.
%
%\iffalse
%<*samplechap1|samplechap2>
%\fi

% Optional override for |\version| flag:
%    \begin{macrocode}
%%\providecommand{\version}{final}
%    \end{macrocode}

% Include the main document:
%    \begin{macrocode}
\input{childdoc.def}
\childdocof{cdocsamp}
%    \end{macrocode}

%\iffalse
%</samplechap1|samplechap2>
%\fi
%
%\iffalse
%<*samplechap1>
%\fi
% Some text for chapter 1:
%    \begin{macrocode}
\section{one}
some text in chapter one
%    \end{macrocode}

%\iffalse
%</samplechap1>
%\fi
% Some text for chapter 2:
%\iffalse
%<*samplechap2>
%\fi
%    \begin{macrocode}
\section{two}
more text in chapter two
%    \end{macrocode}

%\iffalse
%</samplechap2>
%\fi
%
% %%%%%%%%%%%%%%%%%%%%%%%%%%%%%%%%%%%%%%
% \paragraph{Part Include Files.}
%
% The include files are called |cdocspt3.tex| and |cdocspt4.tex|.
%
%\iffalse
%<*samplepart3|samplepart4>
%\fi

% Optional override for |\version| flag:
%    \begin{macrocode}
%%\providecommand{\version}{final}
%    \end{macrocode}

% Include the main document:
%    \begin{macrocode}
\input{childdoc.def}
\childdocby{cdocsamp}
%    \end{macrocode}

%\iffalse
%</samplepart3|samplepart4>
%\fi
%
%\iffalse
%<*samplepart3>
%\fi
% Some text for part 3:
%    \begin{macrocode}
some text in part three
%    \end{macrocode}

%\iffalse
%</samplepart3>
%\fi
% Some text for part 4:
%\iffalse
%<*samplepart4>
%\fi
%    \begin{macrocode}
more text in part four
%    \end{macrocode}

%\iffalse
%</samplepart4>
%\fi
%
% %%%%%%%%%%%%%%%%%%%%%%%%%%%%%%%%%%%%%%
% \paragraph{Forwarding for a Complete Draft.}
%
% The following forwarding file |cdocsdrf.tex|
% compiles the main document in draft mode:
%\iffalse
%<*sampledraft>
%\fi
%    \begin{macrocode}
\def\version{draft}
\input{childdoc.def}
\childdocforward{cdocsamp}
%    \end{macrocode}

%\iffalse
%</sampledraft>
%\fi
%
% %%%%%%%%%%%%%%%%%%%%%%%%%%%%%%%%%%%%%%
% \paragraph{Forwarding for Final Version of the Chapters.}
%
% The following forwarding files |cdocsfn1.tex| and |cdocsfn2.tex|
% (with identical content)
% compile the final versions of the child documents
% |cdocsch1.tex| and |cdocsch2.tex|, respectively:
%\iffalse
%<*samplefinal>
%\fi
%    \begin{macrocode}
\def\version{final}
\input{childdoc.def}
\childdocforwardprefix[cdocsamp]{cdocsfn}{cdocsch}
%    \end{macrocode}

%\iffalse
%</samplefinal>
%\fi
%
% %%%%%%%%%%%%%%%%%%%%%%%%%%%%%%%%%%%%%%
% \paragraph{Command Line Processing.}
%
% The following three command lines generate the output files
% |cdocscld|, |cdocscl1| and |cdocscl2|
% which should be identical to
% |cdocsdrf|, |cdocsch1| and |cdocsfn2|, respectively:
% \begin{center}
% \begin{tabular}{l}
% |latex -jobname cdocscld \|\\
% |  "\def\version{draft}\input{childdoc.def}\childdocforward{cdocsamp}"|\\
% |latex -jobname cdocscl1 \|\\
% |  "\input{childdoc.def}\childdocforward[cdocsamp]{cdocsch1}"|\\
% |latex -jobname cdocscl2 \|\\
% |  "\def\version{final}\input{childdoc.def}\childdocforward{cdocsch2}"|
% \end{tabular}
% \end{center}
% Note that the trailing backslash on each first line
% merely continues the input to the second line
% (for convenient cut ant paste).
% Furthermore, the command |latex| can be replaced by any
% of its alternative versions such as |pdflatex|.
%
% %%%%%%%%%%%%%%%%%%%%%%%%%%%%%%%%%%%%%%%%%%%%%%%%%%%%%%%%%%%%%%%%%%%%%%%%%%%%%%
% %%%%%%%%%%%%%%%%%%%%%%%%%%%%%%%%%%%%%%%%%%%%%%%%%%%%%%%%%%%%%%%%%%%%%%%%%%%%%%
% \section{Implementation}
%\iffalse
%<*package>
%\fi
%
% This section describes the definitions file |childdoc.def|.

% The definitions cannot be loaded using |\usepackage| or |\RequirePackage|
% which has a mechanism to prevent loading a style file more than once.
% When loading the definitions by means of |\input|
% multiple instances have to be prevented manually:
%\iffalse
%This code needs to be before the `\ProvidesFile' directive
%which is defined at the beginning of this file.
%Therefore it is also placed there and commented out here.
%</package>
%<*discard>
%\fi
%    \begin{macrocode}
\ifdefined\childdocmain\endinput\fi
%    \end{macrocode}
%\iffalse
%</discard>
%<*package>
%\fi
%
% \macro{\ifchilddoc}
% \macro{\ifchilddocmanual}
% The conditional |\ifchilddoc| tells whether a
% child (true) or main (false) document is being compiled.
% The conditional |\ifchilddocmanual| tells whether
% the |\includeonly| mechanism is used (false) or
% the selection of child files must be performed manually (true).
% The definitions initialise to false:
%    \begin{macrocode}
\newif\ifchilddoc
\newif\ifchilddocmanual
%    \end{macrocode}

% \macro{\childdocname}
% \macro{\childdocjob}
% The macro |\childdocname| stores the name of the main document
% to be compiled. The macro |\childdocjob| stores the name of
% the document on which the \LaTeX{} compiler was originally invoked.
% The content of |\jobname| cannot be compared
% to filenames specified in the source due to different catcodes.
% The following code rescans |\jobname|, stores the result
% in |\childdocname| and saves a copy in |\childdocjob|:
%    \begin{macrocode}
\edef\childdocname{\scantokens\expandafter{\jobname\noexpand}}
\let\childdocjob\childdocname
%    \end{macrocode}

% \macro{\childdocdisable}
% The macro |\childdocdisable| prevents the main file
% from being processed more than once.
% At this stage, the main document command |\childdocmain|
% is assumed to be called once again where it should do nothing.
% Any subsequent call to it should prevent
% a secondary processing of the main document
% It overwrites the forwarding commands
% |\childdocof| and |\childdocforward|
% with empty macros to prevent further inclusions of the main document:
%    \begin{macrocode}
\newcommand{\childdocdisable}
{
  \renewcommand{\childdocmain}[1]{\renewcommand{\childdocmain}[1]{\endinput}}
  \renewcommand{\childdocof}[1]{}
  \renewcommand{\childdocby}[2][]{}
  \renewcommand{\childdocforward}[2][]{}
  \renewcommand{\childdocdisable}{}
}
%    \end{macrocode}

% \macro{\childdocmain}
% The macro |\childdocmain| is to be called at the top of the main file
% with nothing or the main filename (without extension) as argument.
% First, it breaks loops.
% If the argument is not empty and does not match |\childdocname|
% (which is set by the first inclusion of |childdoc.def|),
% |\ifchilddoc| is set to true, |\includeonly| is applied to the child file
% and |\jobname| is set to the main file
% (for proper handling of |.aux| files):
%    \begin{macrocode}
\newcommand{\childdocmain}[1]
{
  \childdocdisable\childdocmain{}
  \if?#1?\else
    \begingroup
      \def\childdoctmp{#1}
      \ifx\childdoctmp\childdocname
        \def\childdoctmp{}
      \else
        \def\childdoctmp
        {
          \childdoctrue
          \includeonly{\childdocname}
          \def\childdocjob{#1}
          \def\jobname{#1}
        }
      \fi
      \expandafter
    \endgroup
    \childdoctmp
  \fi
}
%    \end{macrocode}

% \macro{\childdocof}
% The command |\childdocof| redirects
% compilation to the main file |#1|.
%    \begin{macrocode}
\newcommand{\childdocof}[1]
{
  \childdocdisable
  \childdoctrue
  \includeonly{\childdocname}
  \def\jobname{#1}
  \def\childdocjob{#1}
  \input{#1}
}
%    \end{macrocode}

% \macro{\childdocby}
% The command |\childdocby| ....
%    \begin{macrocode}
\newcommand{\childdocby}[2][]
{
  \childdocdisable
  \childdoctrue
  \childdocmanualtrue
  \if?#1?\else
    \def\jobname{#2}
  \fi
  \def\childdocjob{#2}
  \input{#2}
  \endinput
}
%    \end{macrocode}

% \macro{\childdocforward}
% The command |\childdocforward| redirects
% compilation to the main file or
% (if the optional argument is given) a child file.
% Parameters are set as if the main file
% or a child file starting with |\childdocof| was compiled.
% Then compilation is handed over to the main file:
%    \begin{macrocode}
\newcommand{\childdocforward}[2][]
{
  \begingroup
    \if?#1?
      \def\childdoctmp
      {
        \def\childdocname{#2}
        \def\childdocjob{#2}
        \def\jobname{#2}
        \input{#2}
        \endinput
      }
    \else
      \def\childdoctmp
      {
        \childdocdisable
        \def\childdocname{#2}
        \childdoctrue
        \includeonly{#2}
        \def\childdocjob{#1}
        \def\jobname{#1}
        \input{#1}
        \endinput
      }
    \fi
    \expandafter
  \endgroup
  \childdoctmp
}
%    \end{macrocode}

% \macro{\childdocforwardprefix}
% The command |\childdocforwardprefix| redirects
% compilation to the main or a child file by means of a pattern.
% The prefix |#1| in the current filename is replaced by |#2|
% and the suffix of the current filename is kept
% (it is assumed that the filename does not contain the substring `|~~~|'
% which is used as a delimiter).
% Compilation is handed over to the new file by |\childdocforward|:
%    \begin{macrocode}
\newcommand{\childdocforwardprefix}[3][]
{
  \begingroup
    \def\childdocextract #2##1~~~{\def\childdoctmp{\childdocforward[#1]{#3##1}}}
    \expandafter\childdocextract\childdocname~~~
    \expandafter
  \endgroup
  \childdoctmp
}
%    \end{macrocode}

% \macro{\childdoc}
% The deprecated macro |\childdoc| is a legacy version of |\childdocmain|:
%    \begin{macrocode}
\newcommand{\childdoc}{\childdocmain}
%    \end{macrocode}

% \macro{\childdocredirect}
% The deprecated macro |\childdocredirect| is a legacy version
% of |\childdocforward| and |\childdocforwardprefix|:
%    \begin{macrocode}
\newcommand{\childdocredirect}[2][]
{
  \begingroup
    \if?#1?
      \def\childdoctmp{\childdocforward{#2}}
    \else
      \def\childdoctmp{\childdocforwardprefix{#1}{#2}}
    \fi
    \expandafter
  \endgroup
  \childdoctmp
}
%    \end{macrocode}

%\iffalse
%</package>
%\fi
%
\endinput
\childdocforward[cdocsamp]{cdocsch1}"|\\
% |latex -jobname cdocscl2 \|\\
% |  "\def\version{final}% \iffalse
%
% childdoc.dtx Copyright (C) 2017-2018 Niklas Beisert
%
% This work may be distributed and/or modified under the
% conditions of the LaTeX Project Public License, either version 1.3
% of this license or (at your option) any later version.
% The latest version of this license is in
%   http://www.latex-project.org/lppl.txt
% and version 1.3 or later is part of all distributions of LaTeX
% version 2005/12/01 or later.
%
% This work has the LPPL maintenance status `maintained'.
%
% The Current Maintainer of this work is Niklas Beisert.
%
% This work consists of the files childdoc.dtx and childdoc.ins
% and the derived files childdoc.def and cdocsamp.tex with
% cdocsch1.tex, cdocsch2.tex, cdocsdrf.tex, cdocsfn1.tex, cdocsfn2.tex.
%
%<package>\ifdefined\childdocmain\endinput\fi
%<package>\ProvidesFile{childdoc.def}[2018/12/30 v2.0 child document driver]
%<samplemain>\ProvidesFile{cdocsamp.tex}[2018/12/30 v2.0 sample for childdoc]
%<*driver>
%\ProvidesFile{childdoc.drv}[2018/12/30 v2.0 childdoc reference manual file]
\PassOptionsToClass{10pt,a4paper}{article}
\documentclass{ltxdoc}

\usepackage[margin=35mm]{geometry}
\usepackage{hyperref}
\usepackage{hyperxmp}
\usepackage[usenames]{color}

\hypersetup{colorlinks=true}
\hypersetup{pdfstartview=FitH}
\hypersetup{pdfpagemode=UseNone}
\hypersetup{pdfsource={}}
\hypersetup{pdflang={en-UK}}
\hypersetup{pdfcopyright={Copyright 2017-2018 Niklas Beisert.
  This work may be distributed and/or modified under the
  conditions of the LaTeX Project Public License, either version 1.3
  of this license or (at your option) any later version.}}
\hypersetup{pdflicenseurl={http://www.latex-project.org/lppl.txt}}
\hypersetup{pdfcontactaddress={ETH Zurich, ITP, HIT K,
  Wolfgang-Pauli-Strasse 27}}
\hypersetup{pdfcontactpostcode={8093}}
\hypersetup{pdfcontactcity={Zurich}}
\hypersetup{pdfcontactcountry={Switzerland}}
\hypersetup{pdfcontactemail={nbeisert@itp.phys.ethz.ch}}
\hypersetup{pdfcontacturl={http://people.phys.ethz.ch/\xmptilde nbeisert/}}

\newcommand{\secref}[1]{\hyperref[#1]{section \ref*{#1}}}

\parskip1ex
\parindent0pt
\let\olditemize\itemize
\def\itemize{\olditemize\parskip0pt}

\begin{document}

\title{The \textsf{childdoc} Package}
\hypersetup{pdftitle={The childdoc Package}}
\author{Niklas Beisert\\[2ex]
  Institut f\"ur Theoretische Physik\\
  Eidgen\"ossische Technische Hochschule Z\"urich\\
  Wolfgang-Pauli-Strasse 27, 8093 Z\"urich, Switzerland\\[1ex]
  \href{mailto:nbeisert@itp.phys.ethz.ch}
  {\texttt{nbeisert@itp.phys.ethz.ch}}}
\hypersetup{pdfauthor={Niklas Beisert}}
\hypersetup{pdfsubject={Manual for the LaTeX2e Package childdoc}}
\date{30 December 2018, \textsf{v2.0}}
\maketitle

\begin{abstract}\noindent
\textsf{childdoc} is a \LaTeXe{} package
that enables the direct compilation
of document sections included by |\include|
to individual files.
\end{abstract}

\begingroup
\parskip0ex
\tableofcontents
\endgroup

%%%%%%%%%%%%%%%%%%%%%%%%%%%%%%%%%%%%%%%%%%%%%%%%%%%%%%%%%%%%%%%%%%%%%%%%%%%%%%%%
%%%%%%%%%%%%%%%%%%%%%%%%%%%%%%%%%%%%%%%%%%%%%%%%%%%%%%%%%%%%%%%%%%%%%%%%%%%%%%%%
\section{Introduction}

\LaTeX{} provides a mechanism to structure a large document (such as a book)
into a main file and several child files (containing the chapters)
using the |\include| command.
This mechanism is beneficial for documents
which span hundreds of pages in order to
make the source file(s) more manageable.
Moreover, compilation can be restricted to
selected child files by means of the |\includeonly| command.
The latter feature can be used to reduce the compilation time while editing
(this was significantly more useful in the earlier days of \LaTeX{})
or to generate a smaller document which is easier to navigate.
Another application of |\includeonly| is to generate
documents consisting of selected parts of the complete document.

However, there are a few drawbacks of the plain |\include| mechanism:
\begin{itemize}
\item
The child files cannot be compiled on their own,
they can only be compiled via the main file.
A naive editing environment
(such as a text editor with an option
to have the current file processed by \LaTeX)
may require one to switch to the main file before compiling;
attempting to compile the child file produces errors.
\item
The main file must be modified (each time)
to adjust the |\includeonly| command
to the present needs. This easily leaves the main file in a messy state.
\item
The generated document will always carry the filename
of the main document. This is inconvenient if
several child files are to be compiled and
to be kept for distribution.
\end{itemize}

The present package provides a simple interface
to make child files individually compilable by \LaTeX{}.
Compiling a child file then has the same effect as compiling
the main file with an |\includeonly| command
to select the appropriate child.
Moreover the generated document will carry the name of the child
rather than the main file.
This resolves all three above issues.

This feature is meant to make the editing of books,
thesis documents and lecture notes somewhat more convenient.
However, the package can also be used efficiently for
composing a series of documents (such as exercise sheets)
which are typically distributed individually.
It then assists the author in generating the individual documents
(potentially in different versions)
as well as a document containing the collected series.
Another application is in developing style files
or other kinds of included material
where compilation of the style file could redirect
to a sample or test file.

%%%%%%%%%%%%%%%%%%%%%%%%%%%%%%%%%%%%%%%%%%%%%%%%%%%%%%%%%%%%%%%%%%%%%%%%%%%%%%%%
%%%%%%%%%%%%%%%%%%%%%%%%%%%%%%%%%%%%%%%%%%%%%%%%%%%%%%%%%%%%%%%%%%%%%%%%%%%%%%%%
\section{Usage}

First of all, the package \textsf{childdoc} is \emph{not} a standard
\LaTeXe{} |.sty| style file! Therefore it needs to be invoked in
a non-standard way.

%%%%%%%%%%%%%%%%%%%%%%%%%%%%%%%%%%%%%%%%%%%%%%%%%%%%%%%%%%%%%%%%%%%%%%%%%%%%%%%%
\subsection{Included Files}
\label{sec:include}

%%%%%%%%%%%%%%%%%%%%%%%%%%%%%%%%%%%%%%%%
\DescribeMacro{\childdocmain}
To use the package, add the commands
\begin{center}
\begin{tabular}{l}
|\input{childdoc.def}|\\
|\childdocmain{}|\\
\end{tabular}
\end{center}
at the very top of the main \LaTeX{} file,
in particular \emph{before} the |\documentclass| statement!
The argument of |\childdocmain| should be left empty
(but it must be present).

%%%%%%%%%%%%%%%%%%%%%%%%%%%%%%%%%%%%%%%%
\DescribeMacro{\childdocof}
Furthermore, add the commands
\begin{center}
\begin{tabular}{l}
|\input{childdoc.def}|\\
|\childdocof{|\textit{main}|}|\\
\end{tabular}
\end{center}
at the top of every child file \textit{child}
which is included by |\include{|\textit{child}|}|
from within the main file
(or at least for those files to be compiled individually).
The argument \textit{main} must be the filename of the main file.

There are a couple of
considerations in setting up the main and child documents:

%%%%%%%%%%%%%%%%%%%%%%%%%%%%%%%%%%%%%%%%
\paragraph{Restrictions.}

Please note the following restrictions:
\begin{itemize}
\item
|\childdocmain| must be called with one argument \textit{main}
to ensure compatibility with earlier version of the package.
It must either be empty (|\childdocmain{}|)
or precisely match the filename of the main file in which it is specified.
See \secref{sec:detection} for further information.
\item
The filename \textit{main} must be specified without the |.tex| extension.
\item
The filename \textit{main} is case sensitive
(even in case-insensitive file systems)
due to internal string comparison.
\item
The argument \textit{main} should be fully expanded, it cannot be a macro.
\item
Subdirectories and special characters should be avoided in filenames.
\item
The command |\childdocmain{|\textit{main}|}| must be followed by a whitespace.
It should not be followed immediately by another command
or by a comment mark `|%|'.
This is because the \TeX{} parser reads the token immediately following
the argument of |\childdocmain| and puts it
at the beginning of every child section;
however, a white\-space is ignored.
\end{itemize}

%%%%%%%%%%%%%%%%%%%%%%%%%%%%%%%%%%%%%%%%
\paragraph{Content of Main File.}

It is advisable to place all content in the child files included by |\include|.
Any output contained in the main file will appear in all child documents
unless suppressed manually;
it cannot be suppressed automatically by the |\includeonly| directive
and thus should normally be avoided.
A method to include some content in the main file
by means of conditional processing is described in \secref{sec:conditional}.

%%%%%%%%%%%%%%%%%%%%%%%%%%%%%%%%%%%%%%%%
\paragraph{Page Numbering.}

When only a part of the document is compiled,
the appropriate numbering of pages
(as well as other status parameters)
is determined from the |.aux| files.
The latter contain information from previous passes.
However this information needs to propagate through
all intermediate child documents.
Therefore the page numbering in child documents may well
be inconsistent until the complete document is compiled at least once.

A useful (if unconventional) way to always ensure a consistent
page numbering is to restart the numbering in each child document
and denote the pages by `\textit{child}|.|\textit{page}'
where \textit{child} represents the chapter/section number of the child file.
This can be achieved by the command
|\numberwithin{page}{|\textit{child}|}|
of the \textsf{amsmath} package
where \textit{child} can be |chapter| or |section|
depending on the chosen structuring.
Alternatively, one can modify the macro |\thepage| appropriately
and reset the counter |page| at the start of each child file.

%%%%%%%%%%%%%%%%%%%%%%%%%%%%%%%%%%%%%%%%%%%%%%%%%%%%%%%%%%%%%%%%%%%%%%%%%%%%%%%%
\subsection{Conditional Processing}
\label{sec:conditional}

The package provides a mechanism to compile different versions
of a document. To customise the versions further some conditional processing
can come in handy to distinguish which version is being compiled.
The package provides two macros to describe the compilation context:

%%%%%%%%%%%%%%%%%%%%%%%%%%%%%%%%%%%%%%%%
\DescribeMacro{\ifchilddoc}
The conditional |\ifchilddoc| distinguishes between the compilation of
child documents and the main document:
%
\begin{center}
|\ifchilddoc |\textit{child-code}| |[|\||else |\textit{main-code}]| \||fi|
\end{center}

%%%%%%%%%%%%%%%%%%%%%%%%%%%%%%%%%%%%%%%%
\DescribeMacro{\childdocname}
\DescribeMacro{\childdocjob}
The macro |\childdocname| contains the filename (without extension)
of the main or child file being processed.
Note that |\childdocjob| will always contain the name of the main file.

%%%%%%%%%%%%%%%%%%%%%%%%%%%%%%%%%%%%%%%%
\paragraph{Title Page.}

Conditional processing can be used to include a title or banner page
in the main document when proper precautions are taken.
Importantly, the code in the main file should ensure that the page counter
(as well as other status parameters which are stored in the |.aux| files)
takes the same value after the conditional processing.
Otherwise the page numbers may take divergent values
depending on which part is compiled.

For example, a title page could be declared by:
%
\begin{center}
\begin{tabular}{l}
|\ifchilddoc\||else|\\
|\addtocounter{page}{-1}|\\
\textit{code for title page}\\
|\newpage|\\
|\||fi|
\end{tabular}
\end{center}
%
A banner page for the child documents can be generated by:
%
\begin{center}
\begin{tabular}{l}
|\ifchilddoc|\\
|\addtocounter{page}{-1}|\\
\textit{code for banner page}\\
|\newpage|\\
|\||fi|
\end{tabular}
\end{center}
%
Here one could write a message such as:
\begin{center}
|This is the part \childdocname{} of \childdocjob{}.|
\end{center}

%%%%%%%%%%%%%%%%%%%%%%%%%%%%%%%%%%%%%%%%%%%%%%%%%%%%%%%%%%%%%%%%%%%%%%%%%%%%%%%%
\subsection{Flags}
\label{sec:flags}

The package makes it easy to generate different versions
of the main or child documents.
To this end compilation flags can be defined
and assigned different default values.
They will be particularly useful in conjunction
with the forwarding mechanism described in \secref{sec:forward}.

For example, it may be useful to have a flag |\version|
which can be set to |draft| or |final|.
The document source will contain some conditional code
depending on the value of |\version|.
Suppose further, the flag should default to |final| for the main file
and to |draft| for child files
which is a natural assignment for editing the document.
This is achieved by placing the following code
in the preamble of the main document
(below the |\childdocmain| directive):
%
\begin{center}
\begin{tabular}{l}
|\ifchilddoc|\\
|\providecommand{\version}{draft}|\\
|\||else|\\
|\providecommand{\version}{final}|\\
|\||fi|
\end{tabular}
\end{center}
%
The definition by |\providecommand| makes sure
that previous definitions are not overwritten.
Further statements |\providecommand{\version}{...}|
can thus be added before the above code to override it.

For the main file, one might add a line
(between |\childdocmain| and the above block)
%
\begin{center}
|%\ifchilddoc\||else\providecommand{\version}{draft}\||fi|
\end{center}
%
which can be uncommented to produce a draft version.
Likewise one can add a line to the very top of a child file
(above the |\childdocof{|\textit{main}|}| directive)
%
\begin{center}
|%\providecommand{\version}{final}|
\end{center}
%
which can be uncommented to produce the final version of this child document.

%%%%%%%%%%%%%%%%%%%%%%%%%%%%%%%%%%%%%%%%%%%%%%%%%%%%%%%%%%%%%%%%%%%%%%%%%%%%%%%%
\subsection{Forwarding}
\label{sec:forward}

Different versions of the main or child documents
using compilation flags as described in \secref{sec:flags}
can be (permanently) stored in different files
for convenient compilation, viewing and distribution.
To this end, the package defines a command
to pass on compilation to a different file:

%%%%%%%%%%%%%%%%%%%%%%%%%%%%%%%%%%%%%%%%
\DescribeMacro{\childdocforward}
The command |\childdocforward| redirects processing to
another source file:
%
\begin{center}
\begin{tabular}{l}
|\input{childdoc.def}|\\
|\childdocforward[|\textit{main}|]{|\textit{dest}|}|\\
\end{tabular}
\end{center}
%
The argument \textit{dest} is the destination file
(without extension).
It should be the main file or one of the child files.
Note that further \textsf{childdoc} directives
such as |\childdocof| and |\childdocforward|
in the indicated file will be processed in this form.
The optional argument \textit{main}
passes on directly to the main file \textit{main}
while pretending to compile the child \textit{dest}.
This form behaves as if \textit{dest}
issues |\childdocof{|\textit{main}|}| right away,
and no further \textsf{childdoc} directives will be processed.

%%%%%%%%%%%%%%%%%%%%%%%%%%%%%%%%%%%%%%%%
\DescribeMacro{\...prefix}
In the alternative form |\childdocforwardprefix|,
%
\begin{center}
\begin{tabular}{l}
|\input{childdoc.def}|\\
|\childdocforwardprefix[|\textit{main}|]{|\textit{prefix}|}{|\textit{dest}|}|
\end{tabular}
\end{center}
%
the destination file is determined by a pattern
depending on the current file:
To make this work, the current file must be called
`{\textit{prefix}\hspace{0.2em}\textit{suffix}}'
with \textit{prefix} matching precisely the argument.
Processing is then passed on to the file
`{\textit{dest}\hspace{0.2em}\textit{suffix}}'.
Surely, the same effect is achieved by
directly specifying the
argument `{\textit{dest}\hspace{0.2em}\textit{suffix}}'
in the first form.
However, that requires to set up a different file
for each child. With the alternative form of the command
all these files can have exactly the same content
which simplifies setting them up and maintaining them.

For example, the following file |draft.tex|
with a compilation flag |\version| as described in \secref{sec:flags}
compiles the main document as a draft:
%
\begin{center}
\begin{tabular}{l}
|\def\version{draft}|\\
|\input{childdoc.def}|\\
|\childdocforward{|\textit{main}|}|
\end{tabular}
\end{center}
%
Likewise, the following files |final|\textit{nn}|.tex|
compile the final version of the child document
|child|\textit{nn}|.tex|:
%
\begin{center}
\begin{tabular}{l}
|\def\version{final}|\\
|\input{childdoc.def}|\\
|\childdocforwardprefix{final}{child}|
\end{tabular}
\end{center}
%

Note that when several versions of a main file and/or of each child file
are to be generated, it may be convenient to set up a |Makefile| or
shell script to automatise the process.

%%%%%%%%%%%%%%%%%%%%%%%%%%%%%%%%%%%%%%%%%%%%%%%%%%%%%%%%%%%%%%%%%%%%%%%%%%%%%%%%
\subsection{Command Line Processing}
\label{sec:commandline}

The effect of redirection files can also be achieved by invoking
the \LaTeX{} compiler with a more elaborate command line.
Most conveniently this should be done as part
of a shell script or a |Makefile|.

When using \textsf{childdoc} in the main file, the following
command lines effectively perform a redirection
(note that depending on the shell being used,
backslashes may have to be doubled: `|\|' $\to$ `|\\|'):
%
\begin{center}
|... -jobname "|\textit{target}|" |\\|"|[\textit{flags}]%
|\input{childdoc.def}\childdocforward[|\textit{main}|]{|\textit{dest}|}"|
\end{center}
%
Here \textit{target} is the name of the output file,
\textit{main} is the name of the main file
and \textit{dest} is the name of the main or child file to be processed
(all filenames without extensions).
The optional argument \textit{main} can be omitted
if \textit{main} matches \textit{dest}.
Optionally, compilation \textit{flags} can be defined via |\def| commands.
This command line makes the \TeX{} engine believe
it is compiling the file \textit{target}
whose content is specified as the latter parameter.
The provided code then forwards the processing to
\textit{main} or \textit{dest} as described in \secref{sec:forward}.

%%%%%%%%%%%%%%%%%%%%%%%%%%%%%%%%%%%%%%%%%%%%%%%%%%%%%%%%%%%%%%%%%%%%%%%%%%%%%%%%
\subsection{Include by Input}
\label{sec:input}

Including child documents by |\include| has some restrictions by design.
Most notably, the content of a child document always occupies
its own set of pages; pages cannot be shared between child documents.
Usually, this behaviour makes perfect sense
because each child document contain an essential part of the document.
However, in some situations it may be desirable to compose
a document from a collection of parts
without having mandatory page breaks between then.
For this case, the package
provides a mechanism to include parts
by |\input| which can also be processed individually.
However, by construction this mechanism
requires manual handling of the content to be output.

%%%%%%%%%%%%%%%%%%%%%%%%%%%%%%%%%%%%%%%%
\DescribeMacro{\ifchilddocmanual}
The main file should be prepared as usual, see \secref{sec:include}.
However, the document body must make a distinction
between processing of an individual part and of the main document, e.g.:
%
\begin{center}
\begin{tabular}{l}
|\ifchilddocmanual|\\
|\input{\childdocname}|\\
|\||else|\\
\textit{document body with }|\input{|\textit{part}|}|\\
|\||fi|
\end{tabular}
\end{center}
%
The conditional |\ifchilddocmanual| is true whenever
a part to be included by |\input| is being compiled,
and the name of the part is stored in |\childdocname|.

%%%%%%%%%%%%%%%%%%%%%%%%%%%%%%%%%%%%%%%%
\DescribeMacro{\childdocby}
Each part to be included by |\input| should start with:
%
\begin{center}
\begin{tabular}{l}
|\input{childdoc.def}|\\
|\childdocby{|\textit{main}|}|\\
\end{tabular}
\end{center}
%
The directive |\childdocby| is similar to |\childdocof|
described in \secref{sec:include},
but the subsequent selection of content must be done manually.
To that end, both |\ifchilddoc| and |\ifchilddocmanual|
will be true upon processing of a part,
and the name of the part is stored in |\childdocname|.
Note that |\jobname| will be set to the filename of the current part
so that each part receives an individual |.aux| file
that does not interfere with the |.aux| file(s) of the main document.
This behaviour can be altered by the alternative form
|\childdocby[*]{|\textit{main}|}| (with a non-empty optional argument)
which uses the |.aux| file of the main document
by setting |\jobname| to \textit{main}.

%%%%%%%%%%%%%%%%%%%%%%%%%%%%%%%%%%%%%%%%%%%%%%%%%%%%%%%%%%%%%%%%%%%%%%%%%%%%%%%%
\subsection{Driver Development}
\label{sec:driver}

The \textsf{childdoc} mechanism can also be use for the development
of definition files such as \LaTeX{} styles or classes.
This case differs from the above setup with multiple parts
included by |\include| in that no |\includeonly| should be invoked.
This can be achieved by starting the include file
(before |\ProvidesPackage|) with:
%
\begin{center}
\begin{tabular}{l}
|\input{childdoc.def}|\\
|\childdocforward{|\textit{main}|}|\\
\end{tabular}
\end{center}
%
or alternatively with:
%
\begin{center}
\begin{tabular}{l}
|\input{childdoc.def}|\\
|\childdocby{|\textit{main}|}|\\
\end{tabular}
\end{center}
%
Both forms have slightly different effects as described above.
The main file is prepared as usual, see \secref{sec:include}.

%%%%%%%%%%%%%%%%%%%%%%%%%%%%%%%%%%%%%%%%%%%%%%%%%%%%%%%%%%%%%%%%%%%%%%%%%%%%%%%%
\subsection{Legacy Detection}
\label{sec:detection}

The directive |\childdocmain| in the main file can detect
whether the complete document or merely a child is to be compiled
even without using the directive |\childdocof|.
This method is deprecated because it is less robust
and there is no compelling reason to use it;
it is merely provided for backward compatibility
and it may be removed in future versions.

If the detection mechanism is to be used,
it is mandatory to correctly specify
the filename of the main file as the argument of |\childdocmain|:
%
\begin{center}
\begin{tabular}{l}
|\input{childdoc.def}|\\
|\childdocmain{|\textit{main}|}|\\
\end{tabular}
\end{center}
%
If |\jobname| does not match the argument \textit{main} of |\childdocmain|,
it is assumed that |\jobname| points to the child file to be compiled.
When using |\childdocmain| with the main file specified as argument,
it suffices to start a child file
with just |\input{|\textit{main}|}|
without loading of the package and using |\childdocof|.
If instead all processing is done
with the appropriate \textsf{childdoc} directives,
the argument of \textit{main} of |\childdocmain| can be empty.

An alternative version of the command line processing described
in \secref{sec:commandline} using the detection mechanism reads:
%
\begin{center}
|... -jobname "|\textit{target}|" "|[\textit{flags}]%
[|\def\jobname{|\textit{dest}|}|]|\input{|\textit{main}|}"|
\end{center}

%%%%%%%%%%%%%%%%%%%%%%%%%%%%%%%%%%%%%%%%%%%%%%%%%%%%%%%%%%%%%%%%%%%%%%%%%%%%%%%%
\subsection{Manual Code}
\label{sec:manual}

In case one cannot be certain whether the definitions file |childdoc.def|
is installed on the target \TeX{} distribution
and one prefers not to ship it,
it is conceivable to paste a few relevant commands into the sources.

To that end, drop all statements |\input{childdoc.def}|
and perform the replacements as outlined below.
Instead of |\childdocmain{|\textit{main}|}| add the following code
to the top of the main file:
%
\begin{center}
\begin{tabular}{l}
|\||ifdefined\childdocname\endinput\||fi\newif\ifchilddoc|\\
|\edef\childdocname{\scantokens\expandafter{\jobname\noexpand}}|\\
|\def\childdocmain{|\textit{main}|}\||ifx\childdocmain\childdocname\||else|\\
|\childdoctrue\includeonly{\childdocname}\let\jobname\childdocmain\||fi|\\
\end{tabular}
\end{center}
%
Instead of |\childdocof{|\textit{main}|}| just include the main file
at the top of each child file:
%
\begin{center}
|\input{|\textit{main}|}|
\end{center}
%
A simple redirection |\childdocforward{|\textit{dest}|}| is achieved by:
%
\begin{center}
|\def\jobname{|\textit{dest}|}\input{\jobname}|
\end{center}
%
The redirection with prefix
|\childdocforwardprefix[|\textit{prefix}|]{|\textit{dest}|}|
is accomplished by:
%
\begin{center}
\begin{tabular}{l}
|{\edef\jobname{\scantokens\expandafter{\jobname\noexpand}}|\\
|\def\redirectjob |\textit{prefix}|#1~~~{\gdef\jobname{|\textit{dest}|#1}}|\\
|\expandafter\redirectjob\jobname~~~}\input{\jobname}|
\end{tabular}
\end{center}

In an alternative approach,
child documents can be compiled by a specific command line
without additional code or specific definitions:
%
\begin{center}
|... -jobname "|\textit{target}|" "|[\textit{flags}]%
|\includeonly{|\textit{dest}|}\input{|\textit{main}|}"|
\end{center}
%

%%%%%%%%%%%%%%%%%%%%%%%%%%%%%%%%%%%%%%%%%%%%%%%%%%%%%%%%%%%%%%%%%%%%%%%%%%%%%%%%
%%%%%%%%%%%%%%%%%%%%%%%%%%%%%%%%%%%%%%%%%%%%%%%%%%%%%%%%%%%%%%%%%%%%%%%%%%%%%%%%
\section{Information}

%%%%%%%%%%%%%%%%%%%%%%%%%%%%%%%%%%%%%%%%%%%%%%%%%%%%%%%%%%%%%%%%%%%%%%%%%%%%%%%%
\subsection{Copyright}

Copyright \copyright{} 2017--2018 Niklas Beisert

This work may be distributed and/or modified under the
conditions of the \LaTeX{} Project Public License, either version 1.3
of this license or (at your option) any later version.
The latest version of this license is in
  \url{http://www.latex-project.org/lppl.txt}
and version 1.3 or later is part of all distributions of \LaTeX{}
version 2005/12/01 or later.

This work has the LPPL maintenance status `maintained'.

The Current Maintainer of this work is Niklas Beisert.

This work consists of the files |README.txt|, |childdoc.ins| and |childdoc.dtx|
as well as the derived files |childdoc.def|, |cdocsamp.tex|
with |cdocsch1.tex|, |cdocsch2.tex|, |cdocspt3.tex|, |cdocspt4.tex|,
|cdocsdrf.tex|, |cdocsfn1.tex|, |cdocsfn2.tex|
as well as |childdoc.pdf|.

%%%%%%%%%%%%%%%%%%%%%%%%%%%%%%%%%%%%%%%%%%%%%%%%%%%%%%%%%%%%%%%%%%%%%%%%%%%%%%%%
\subsection{Files and Installation}

The package consists of the files:
%
\begin{center}
\begin{tabular}{ll}
    |README.txt|   & readme file \\
    |childdoc.ins| & installation file \\
    |childdoc.dtx| & source file \\
    |childdoc.def| & definition file \\
    |cdocsamp.tex| & sample main file \\
    |cdocsch1.tex| & sample include file \\
    |cdocsch2.tex| & sample include file \\
    |cdocspt3.tex| & sample part file \\
    |cdocspt4.tex| & sample part file \\
    |cdocsdrf.tex| & sample redirection file \\
    |cdocsfn1.tex| & sample redirection file \\
    |cdocsfn2.tex| & sample redirection file \\
    |childdoc.pdf| & manual
\end{tabular}
\end{center}
%
The distribution consists of the files
|README.txt|, |childdoc.ins| and |childdoc.dtx|.
%
\begin{itemize}
\item
Run (pdf)\LaTeX{} on |childdoc.dtx|
to compile the manual |childdoc.pdf| (this file).
\item
Run \LaTeX{} on |childdoc.ins| to create the definitions file |childdoc.def|
and the sample |cdocsamp.tex| with include files
|cdocsch1.tex|, |cdocsch2.tex|, |cdocspt3.tex|, |cdocspt4.tex|,
|cdocsdrf.tex|, |cdocsfn1.tex|, |cdocsfn2.tex|.
Then copy the file |childdoc.def| to an appropriate directory of your \LaTeX{}
distribution, e.g.\ \textit{texmf-root}|/tex/latex/childdoc|.
\end{itemize}

%%%%%%%%%%%%%%%%%%%%%%%%%%%%%%%%%%%%%%%%%%%%%%%%%%%%%%%%%%%%%%%%%%%%%%%%%%%%%%%%
\subsection{Related CTAN Packages}

There are several other packages which offer a similar functionality:
%
\begin{itemize}
\item
The packages
\href{http://ctan.org/pkg/docmute}{\textsf{docmute}},
\href{http://ctan.org/pkg/includex}{\textsf{includex}} and
\href{http://ctan.org/pkg/standalone}{\textsf{standalone}}
provide commands to include only the document body of
a child file thus allowing both files to be compiled individually.
\item
The packages \href{http://ctan.org/pkg/subdocs}{\textsf{subdocs}}
and \href{http://ctan.org/pkg/subfiles}{\textsf{subfiles}}
provide structures in which the main and child documents can be
encapsulated and allowing them to be compiled individually.
The inclusion mechanism is different from the conventional |\include|.
\item
The package \href{http://ctan.org/pkg/combine}{\textsf{combine}}
is an elaborate solution to combine several documents into one.
\end{itemize}
%
See also the CTAN topic \href{http://ctan.org/topic/subdocs}{\textsf{subdocs}}
for further related packages.
The present package differs from the above solutions in that
a document structure constructed with the conventional |\include| mechanism
just needs two extra commands at the top of every file
such that all constituent files can be compiled individually.

%%%%%%%%%%%%%%%%%%%%%%%%%%%%%%%%%%%%%%%%%%%%%%%%%%%%%%%%%%%%%%%%%%%%%%%%%%%%%%%%
%\subsection{Feature Suggestions}
%
%The following is a list of features which may be useful for future
%versions of this package:
%%
%\begin{itemize}
%\item
%\ldots
%\end{itemize}

%%%%%%%%%%%%%%%%%%%%%%%%%%%%%%%%%%%%%%%%%%%%%%%%%%%%%%%%%%%%%%%%%%%%%%%%%%%%%%%%
\subsection{Revision History}

%%%%%%%%%%%%%%%%%%%%%%%%%%%%%%%%%%%%%%%%
\paragraph{v2.0:} 2018/12/30

\begin{itemize}
\item
immediate forward processing
\item
added |\childdocby| mechanism
\item
manual restructured
\end{itemize}

%%%%%%%%%%%%%%%%%%%%%%%%%%%%%%%%%%%%%%%%
\paragraph{v1.6:} 2018/01/17

\begin{itemize}
\item
application for development of include files
\item
corrections to manual
\end{itemize}

%%%%%%%%%%%%%%%%%%%%%%%%%%%%%%%%%%%%%%%%
\paragraph{v1.5:} 2017/05/21

\begin{itemize}
\item
more complete structuring introduced
\item
|\childdocof| introduced
\item
|\childdoc| renamed to |\childdocmain|
\item
|\childredirect| renamed to |\childdocforward| and |\childdocforwardprefix|
and functionality expanded
\end{itemize}

%%%%%%%%%%%%%%%%%%%%%%%%%%%%%%%%%%%%%%%%
\paragraph{v1.0:} 2017/04/27

\begin{itemize}
\item
manual and install package
\item
first version published on CTAN
\end{itemize}

%%%%%%%%%%%%%%%%%%%%%%%%%%%%%%%%%%%%%%%%
\paragraph{v0.6:} 2017/04/26

\begin{itemize}
\item
redirection mechanism added
\end{itemize}

%%%%%%%%%%%%%%%%%%%%%%%%%%%%%%%%%%%%%%%%
\paragraph{v0.5:} 2017/04/26

\begin{itemize}
\item
functionality in definition file
\end{itemize}


%%%%%%%%%%%%%%%%%%%%%%%%%%%%%%%%%%%%%%%%%%%%%%%%%%%%%%%%%%%%%%%%%%%%%%%%%%%%%%%%
%%%%%%%%%%%%%%%%%%%%%%%%%%%%%%%%%%%%%%%%%%%%%%%%%%%%%%%%%%%%%%%%%%%%%%%%%%%%%%%%
%%%%%%%%%%%%%%%%%%%%%%%%%%%%%%%%%%%%%%%%%%%%%%%%%%%%%%%%%%%%%%%%%%%%%%%%%%%%%%%%
\appendix

\settowidth\MacroIndent{\rmfamily\scriptsize 000\ }

 \DocInput{childdoc.dtx}

\end{document}
%</driver>
% \fi
%
% %%%%%%%%%%%%%%%%%%%%%%%%%%%%%%%%%%%%%%%%%%%%%%%%%%%%%%%%%%%%%%%%%%%%%%%%%%%%%%
% %%%%%%%%%%%%%%%%%%%%%%%%%%%%%%%%%%%%%%%%%%%%%%%%%%%%%%%%%%%%%%%%%%%%%%%%%%%%%%
% \section{Sample}
%\iffalse
%<*samplemain>
%\fi
%
% The following presents a sample document
% with two chapters, two parts, a title page,
% a compile flag as well as three forwarding files to set the flag.
% It consists of eight |.tex| files:
% \begin{center}
% \begin{tabular}{ll}
% |cdocsamp.tex|&main file\\
% |cdocsch1.tex|&include file for chapter 1\\
% |cdocsch2.tex|&include file for chapter 2\\
% |cdocspt3.tex|&include file for part 3\\
% |cdocspt4.tex|&include file for part 4\\
% |cdocsdrf.tex|&forwarding file for main file in draft mode\\
% |cdocsfi1.tex|&forwarding file for final version of chapter 1\\
% |cdocsfi2.tex|&forwarding file for final version of chapter 2\\
% \end{tabular}
% \end{center}
% Each of the eight files can be compiled directly by the \LaTeX{} compiler.
%
% %%%%%%%%%%%%%%%%%%%%%%%%%%%%%%%%%%%%%%
% \paragraph{Main File.}
%
% The main file is called |cdocsamp.tex|.
%
% Load the \textsf{childdoc} definitions and
% declare the filename for the main document:
%    \begin{macrocode}
\input{childdoc.def}
\childdocmain{}
%    \end{macrocode}

% Optional override for |\version| flag:
%    \begin{macrocode}
%%\ifchilddoc\else\providecommand{\version}{draft}\fi
%    \end{macrocode}

% Define the default values for the |\version| flag
% (|final| for the main file and |draft| for childs):
%    \begin{macrocode}
\ifchilddoc
\providecommand{\version}{draft}
\else
\providecommand{\version}{final}
\fi
%    \end{macrocode}

% Load the standard document class:
%    \begin{macrocode}
\documentclass[12pt]{article}
%    \end{macrocode}

% Start the document body:
%    \begin{macrocode}
\begin{document}
%    \end{macrocode}

% Declare a title page.
% Print title, part of document being processed and version flag:
%    \begin{macrocode}
\addtocounter{page}{-1}
\begin{center}
{\LARGE\bfseries{}childdoc example\par}
\vspace{1cm}
\ifchilddoc
\ifchilddocmanual part\else chapter\fi:
`\childdocname' of `\childdocjob'\par
\else
main document: `\childdocjob'\par
\fi
version: \version\par
\end{center}
\newpage
%    \end{macrocode}

% Manually include selected file,
% otherwise process as usual:
%    \begin{macrocode}
\ifchilddocmanual
\section*{part `\childdocname'}
\input{\childdocname}
\else
%    \end{macrocode}

% Include the two chapters:
%    \begin{macrocode}
\include{cdocsch1}
\include{cdocsch2}
%    \end{macrocode}

% Include the two parts unless only chapters should be displayed:
%    \begin{macrocode}
\ifchilddoc\else
\section{part three}
\input{cdocspt3}
\section{part four}
\input{cdocspt4}
\fi
%    \end{macrocode}

% Process as usual until here:
%    \begin{macrocode}
\fi
%    \end{macrocode}

% End of document body:
%    \begin{macrocode}
\end{document}
%    \end{macrocode}
%\iffalse
%</samplemain>
%\fi
%
% %%%%%%%%%%%%%%%%%%%%%%%%%%%%%%%%%%%%%%
% \paragraph{Chapter Include Files.}
%
% The include files are called |cdocsch1.tex| and |cdocsch2.tex|.
%
%\iffalse
%<*samplechap1|samplechap2>
%\fi

% Optional override for |\version| flag:
%    \begin{macrocode}
%%\providecommand{\version}{final}
%    \end{macrocode}

% Include the main document:
%    \begin{macrocode}
\input{childdoc.def}
\childdocof{cdocsamp}
%    \end{macrocode}

%\iffalse
%</samplechap1|samplechap2>
%\fi
%
%\iffalse
%<*samplechap1>
%\fi
% Some text for chapter 1:
%    \begin{macrocode}
\section{one}
some text in chapter one
%    \end{macrocode}

%\iffalse
%</samplechap1>
%\fi
% Some text for chapter 2:
%\iffalse
%<*samplechap2>
%\fi
%    \begin{macrocode}
\section{two}
more text in chapter two
%    \end{macrocode}

%\iffalse
%</samplechap2>
%\fi
%
% %%%%%%%%%%%%%%%%%%%%%%%%%%%%%%%%%%%%%%
% \paragraph{Part Include Files.}
%
% The include files are called |cdocspt3.tex| and |cdocspt4.tex|.
%
%\iffalse
%<*samplepart3|samplepart4>
%\fi

% Optional override for |\version| flag:
%    \begin{macrocode}
%%\providecommand{\version}{final}
%    \end{macrocode}

% Include the main document:
%    \begin{macrocode}
\input{childdoc.def}
\childdocby{cdocsamp}
%    \end{macrocode}

%\iffalse
%</samplepart3|samplepart4>
%\fi
%
%\iffalse
%<*samplepart3>
%\fi
% Some text for part 3:
%    \begin{macrocode}
some text in part three
%    \end{macrocode}

%\iffalse
%</samplepart3>
%\fi
% Some text for part 4:
%\iffalse
%<*samplepart4>
%\fi
%    \begin{macrocode}
more text in part four
%    \end{macrocode}

%\iffalse
%</samplepart4>
%\fi
%
% %%%%%%%%%%%%%%%%%%%%%%%%%%%%%%%%%%%%%%
% \paragraph{Forwarding for a Complete Draft.}
%
% The following forwarding file |cdocsdrf.tex|
% compiles the main document in draft mode:
%\iffalse
%<*sampledraft>
%\fi
%    \begin{macrocode}
\def\version{draft}
\input{childdoc.def}
\childdocforward{cdocsamp}
%    \end{macrocode}

%\iffalse
%</sampledraft>
%\fi
%
% %%%%%%%%%%%%%%%%%%%%%%%%%%%%%%%%%%%%%%
% \paragraph{Forwarding for Final Version of the Chapters.}
%
% The following forwarding files |cdocsfn1.tex| and |cdocsfn2.tex|
% (with identical content)
% compile the final versions of the child documents
% |cdocsch1.tex| and |cdocsch2.tex|, respectively:
%\iffalse
%<*samplefinal>
%\fi
%    \begin{macrocode}
\def\version{final}
\input{childdoc.def}
\childdocforwardprefix[cdocsamp]{cdocsfn}{cdocsch}
%    \end{macrocode}

%\iffalse
%</samplefinal>
%\fi
%
% %%%%%%%%%%%%%%%%%%%%%%%%%%%%%%%%%%%%%%
% \paragraph{Command Line Processing.}
%
% The following three command lines generate the output files
% |cdocscld|, |cdocscl1| and |cdocscl2|
% which should be identical to
% |cdocsdrf|, |cdocsch1| and |cdocsfn2|, respectively:
% \begin{center}
% \begin{tabular}{l}
% |latex -jobname cdocscld \|\\
% |  "\def\version{draft}\input{childdoc.def}\childdocforward{cdocsamp}"|\\
% |latex -jobname cdocscl1 \|\\
% |  "\input{childdoc.def}\childdocforward[cdocsamp]{cdocsch1}"|\\
% |latex -jobname cdocscl2 \|\\
% |  "\def\version{final}\input{childdoc.def}\childdocforward{cdocsch2}"|
% \end{tabular}
% \end{center}
% Note that the trailing backslash on each first line
% merely continues the input to the second line
% (for convenient cut ant paste).
% Furthermore, the command |latex| can be replaced by any
% of its alternative versions such as |pdflatex|.
%
% %%%%%%%%%%%%%%%%%%%%%%%%%%%%%%%%%%%%%%%%%%%%%%%%%%%%%%%%%%%%%%%%%%%%%%%%%%%%%%
% %%%%%%%%%%%%%%%%%%%%%%%%%%%%%%%%%%%%%%%%%%%%%%%%%%%%%%%%%%%%%%%%%%%%%%%%%%%%%%
% \section{Implementation}
%\iffalse
%<*package>
%\fi
%
% This section describes the definitions file |childdoc.def|.

% The definitions cannot be loaded using |\usepackage| or |\RequirePackage|
% which has a mechanism to prevent loading a style file more than once.
% When loading the definitions by means of |\input|
% multiple instances have to be prevented manually:
%\iffalse
%This code needs to be before the `\ProvidesFile' directive
%which is defined at the beginning of this file.
%Therefore it is also placed there and commented out here.
%</package>
%<*discard>
%\fi
%    \begin{macrocode}
\ifdefined\childdocmain\endinput\fi
%    \end{macrocode}
%\iffalse
%</discard>
%<*package>
%\fi
%
% \macro{\ifchilddoc}
% \macro{\ifchilddocmanual}
% The conditional |\ifchilddoc| tells whether a
% child (true) or main (false) document is being compiled.
% The conditional |\ifchilddocmanual| tells whether
% the |\includeonly| mechanism is used (false) or
% the selection of child files must be performed manually (true).
% The definitions initialise to false:
%    \begin{macrocode}
\newif\ifchilddoc
\newif\ifchilddocmanual
%    \end{macrocode}

% \macro{\childdocname}
% \macro{\childdocjob}
% The macro |\childdocname| stores the name of the main document
% to be compiled. The macro |\childdocjob| stores the name of
% the document on which the \LaTeX{} compiler was originally invoked.
% The content of |\jobname| cannot be compared
% to filenames specified in the source due to different catcodes.
% The following code rescans |\jobname|, stores the result
% in |\childdocname| and saves a copy in |\childdocjob|:
%    \begin{macrocode}
\edef\childdocname{\scantokens\expandafter{\jobname\noexpand}}
\let\childdocjob\childdocname
%    \end{macrocode}

% \macro{\childdocdisable}
% The macro |\childdocdisable| prevents the main file
% from being processed more than once.
% At this stage, the main document command |\childdocmain|
% is assumed to be called once again where it should do nothing.
% Any subsequent call to it should prevent
% a secondary processing of the main document
% It overwrites the forwarding commands
% |\childdocof| and |\childdocforward|
% with empty macros to prevent further inclusions of the main document:
%    \begin{macrocode}
\newcommand{\childdocdisable}
{
  \renewcommand{\childdocmain}[1]{\renewcommand{\childdocmain}[1]{\endinput}}
  \renewcommand{\childdocof}[1]{}
  \renewcommand{\childdocby}[2][]{}
  \renewcommand{\childdocforward}[2][]{}
  \renewcommand{\childdocdisable}{}
}
%    \end{macrocode}

% \macro{\childdocmain}
% The macro |\childdocmain| is to be called at the top of the main file
% with nothing or the main filename (without extension) as argument.
% First, it breaks loops.
% If the argument is not empty and does not match |\childdocname|
% (which is set by the first inclusion of |childdoc.def|),
% |\ifchilddoc| is set to true, |\includeonly| is applied to the child file
% and |\jobname| is set to the main file
% (for proper handling of |.aux| files):
%    \begin{macrocode}
\newcommand{\childdocmain}[1]
{
  \childdocdisable\childdocmain{}
  \if?#1?\else
    \begingroup
      \def\childdoctmp{#1}
      \ifx\childdoctmp\childdocname
        \def\childdoctmp{}
      \else
        \def\childdoctmp
        {
          \childdoctrue
          \includeonly{\childdocname}
          \def\childdocjob{#1}
          \def\jobname{#1}
        }
      \fi
      \expandafter
    \endgroup
    \childdoctmp
  \fi
}
%    \end{macrocode}

% \macro{\childdocof}
% The command |\childdocof| redirects
% compilation to the main file |#1|.
%    \begin{macrocode}
\newcommand{\childdocof}[1]
{
  \childdocdisable
  \childdoctrue
  \includeonly{\childdocname}
  \def\jobname{#1}
  \def\childdocjob{#1}
  \input{#1}
}
%    \end{macrocode}

% \macro{\childdocby}
% The command |\childdocby| ....
%    \begin{macrocode}
\newcommand{\childdocby}[2][]
{
  \childdocdisable
  \childdoctrue
  \childdocmanualtrue
  \if?#1?\else
    \def\jobname{#2}
  \fi
  \def\childdocjob{#2}
  \input{#2}
  \endinput
}
%    \end{macrocode}

% \macro{\childdocforward}
% The command |\childdocforward| redirects
% compilation to the main file or
% (if the optional argument is given) a child file.
% Parameters are set as if the main file
% or a child file starting with |\childdocof| was compiled.
% Then compilation is handed over to the main file:
%    \begin{macrocode}
\newcommand{\childdocforward}[2][]
{
  \begingroup
    \if?#1?
      \def\childdoctmp
      {
        \def\childdocname{#2}
        \def\childdocjob{#2}
        \def\jobname{#2}
        \input{#2}
        \endinput
      }
    \else
      \def\childdoctmp
      {
        \childdocdisable
        \def\childdocname{#2}
        \childdoctrue
        \includeonly{#2}
        \def\childdocjob{#1}
        \def\jobname{#1}
        \input{#1}
        \endinput
      }
    \fi
    \expandafter
  \endgroup
  \childdoctmp
}
%    \end{macrocode}

% \macro{\childdocforwardprefix}
% The command |\childdocforwardprefix| redirects
% compilation to the main or a child file by means of a pattern.
% The prefix |#1| in the current filename is replaced by |#2|
% and the suffix of the current filename is kept
% (it is assumed that the filename does not contain the substring `|~~~|'
% which is used as a delimiter).
% Compilation is handed over to the new file by |\childdocforward|:
%    \begin{macrocode}
\newcommand{\childdocforwardprefix}[3][]
{
  \begingroup
    \def\childdocextract #2##1~~~{\def\childdoctmp{\childdocforward[#1]{#3##1}}}
    \expandafter\childdocextract\childdocname~~~
    \expandafter
  \endgroup
  \childdoctmp
}
%    \end{macrocode}

% \macro{\childdoc}
% The deprecated macro |\childdoc| is a legacy version of |\childdocmain|:
%    \begin{macrocode}
\newcommand{\childdoc}{\childdocmain}
%    \end{macrocode}

% \macro{\childdocredirect}
% The deprecated macro |\childdocredirect| is a legacy version
% of |\childdocforward| and |\childdocforwardprefix|:
%    \begin{macrocode}
\newcommand{\childdocredirect}[2][]
{
  \begingroup
    \if?#1?
      \def\childdoctmp{\childdocforward{#2}}
    \else
      \def\childdoctmp{\childdocforwardprefix{#1}{#2}}
    \fi
    \expandafter
  \endgroup
  \childdoctmp
}
%    \end{macrocode}

%\iffalse
%</package>
%\fi
%
\endinput
\childdocforward{cdocsch2}"|
% \end{tabular}
% \end{center}
% Note that the trailing backslash on each first line
% merely continues the input to the second line
% (for convenient cut ant paste).
% Furthermore, the command |latex| can be replaced by any
% of its alternative versions such as |pdflatex|.
%
% %%%%%%%%%%%%%%%%%%%%%%%%%%%%%%%%%%%%%%%%%%%%%%%%%%%%%%%%%%%%%%%%%%%%%%%%%%%%%%
% %%%%%%%%%%%%%%%%%%%%%%%%%%%%%%%%%%%%%%%%%%%%%%%%%%%%%%%%%%%%%%%%%%%%%%%%%%%%%%
% \section{Implementation}
%\iffalse
%<*package>
%\fi
%
% This section describes the definitions file |childdoc.def|.

% The definitions cannot be loaded using |\usepackage| or |\RequirePackage|
% which has a mechanism to prevent loading a style file more than once.
% When loading the definitions by means of |\input|
% multiple instances have to be prevented manually:
%\iffalse
%This code needs to be before the `\ProvidesFile' directive
%which is defined at the beginning of this file.
%Therefore it is also placed there and commented out here.
%</package>
%<*discard>
%\fi
%    \begin{macrocode}
\ifdefined\childdocmain\endinput\fi
%    \end{macrocode}
%\iffalse
%</discard>
%<*package>
%\fi
%
% \macro{\ifchilddoc}
% \macro{\ifchilddocmanual}
% The conditional |\ifchilddoc| tells whether a
% child (true) or main (false) document is being compiled.
% The conditional |\ifchilddocmanual| tells whether
% the |\includeonly| mechanism is used (false) or
% the selection of child files must be performed manually (true).
% The definitions initialise to false:
%    \begin{macrocode}
\newif\ifchilddoc
\newif\ifchilddocmanual
%    \end{macrocode}

% \macro{\childdocname}
% \macro{\childdocjob}
% The macro |\childdocname| stores the name of the main document
% to be compiled. The macro |\childdocjob| stores the name of
% the document on which the \LaTeX{} compiler was originally invoked.
% The content of |\jobname| cannot be compared
% to filenames specified in the source due to different catcodes.
% The following code rescans |\jobname|, stores the result
% in |\childdocname| and saves a copy in |\childdocjob|:
%    \begin{macrocode}
\edef\childdocname{\scantokens\expandafter{\jobname\noexpand}}
\let\childdocjob\childdocname
%    \end{macrocode}

% \macro{\childdocdisable}
% The macro |\childdocdisable| prevents the main file
% from being processed more than once.
% At this stage, the main document command |\childdocmain|
% is assumed to be called once again where it should do nothing.
% Any subsequent call to it should prevent
% a secondary processing of the main document
% It overwrites the forwarding commands
% |\childdocof| and |\childdocforward|
% with empty macros to prevent further inclusions of the main document:
%    \begin{macrocode}
\newcommand{\childdocdisable}
{
  \renewcommand{\childdocmain}[1]{\renewcommand{\childdocmain}[1]{\endinput}}
  \renewcommand{\childdocof}[1]{}
  \renewcommand{\childdocby}[2][]{}
  \renewcommand{\childdocforward}[2][]{}
  \renewcommand{\childdocdisable}{}
}
%    \end{macrocode}

% \macro{\childdocmain}
% The macro |\childdocmain| is to be called at the top of the main file
% with nothing or the main filename (without extension) as argument.
% First, it breaks loops.
% If the argument is not empty and does not match |\childdocname|
% (which is set by the first inclusion of |childdoc.def|),
% |\ifchilddoc| is set to true, |\includeonly| is applied to the child file
% and |\jobname| is set to the main file
% (for proper handling of |.aux| files):
%    \begin{macrocode}
\newcommand{\childdocmain}[1]
{
  \childdocdisable\childdocmain{}
  \if?#1?\else
    \begingroup
      \def\childdoctmp{#1}
      \ifx\childdoctmp\childdocname
        \def\childdoctmp{}
      \else
        \def\childdoctmp
        {
          \childdoctrue
          \includeonly{\childdocname}
          \def\childdocjob{#1}
          \def\jobname{#1}
        }
      \fi
      \expandafter
    \endgroup
    \childdoctmp
  \fi
}
%    \end{macrocode}

% \macro{\childdocof}
% The command |\childdocof| redirects
% compilation to the main file |#1|.
%    \begin{macrocode}
\newcommand{\childdocof}[1]
{
  \childdocdisable
  \childdoctrue
  \includeonly{\childdocname}
  \def\jobname{#1}
  \def\childdocjob{#1}
  \input{#1}
}
%    \end{macrocode}

% \macro{\childdocby}
% The command |\childdocby| ....
%    \begin{macrocode}
\newcommand{\childdocby}[2][]
{
  \childdocdisable
  \childdoctrue
  \childdocmanualtrue
  \if?#1?\else
    \def\jobname{#2}
  \fi
  \def\childdocjob{#2}
  \input{#2}
  \endinput
}
%    \end{macrocode}

% \macro{\childdocforward}
% The command |\childdocforward| redirects
% compilation to the main file or
% (if the optional argument is given) a child file.
% Parameters are set as if the main file
% or a child file starting with |\childdocof| was compiled.
% Then compilation is handed over to the main file:
%    \begin{macrocode}
\newcommand{\childdocforward}[2][]
{
  \begingroup
    \if?#1?
      \def\childdoctmp
      {
        \def\childdocname{#2}
        \def\childdocjob{#2}
        \def\jobname{#2}
        \input{#2}
        \endinput
      }
    \else
      \def\childdoctmp
      {
        \childdocdisable
        \def\childdocname{#2}
        \childdoctrue
        \includeonly{#2}
        \def\childdocjob{#1}
        \def\jobname{#1}
        \input{#1}
        \endinput
      }
    \fi
    \expandafter
  \endgroup
  \childdoctmp
}
%    \end{macrocode}

% \macro{\childdocforwardprefix}
% The command |\childdocforwardprefix| redirects
% compilation to the main or a child file by means of a pattern.
% The prefix |#1| in the current filename is replaced by |#2|
% and the suffix of the current filename is kept
% (it is assumed that the filename does not contain the substring `|~~~|'
% which is used as a delimiter).
% Compilation is handed over to the new file by |\childdocforward|:
%    \begin{macrocode}
\newcommand{\childdocforwardprefix}[3][]
{
  \begingroup
    \def\childdocextract #2##1~~~{\def\childdoctmp{\childdocforward[#1]{#3##1}}}
    \expandafter\childdocextract\childdocname~~~
    \expandafter
  \endgroup
  \childdoctmp
}
%    \end{macrocode}

% \macro{\childdoc}
% The deprecated macro |\childdoc| is a legacy version of |\childdocmain|:
%    \begin{macrocode}
\newcommand{\childdoc}{\childdocmain}
%    \end{macrocode}

% \macro{\childdocredirect}
% The deprecated macro |\childdocredirect| is a legacy version
% of |\childdocforward| and |\childdocforwardprefix|:
%    \begin{macrocode}
\newcommand{\childdocredirect}[2][]
{
  \begingroup
    \if?#1?
      \def\childdoctmp{\childdocforward{#2}}
    \else
      \def\childdoctmp{\childdocforwardprefix{#1}{#2}}
    \fi
    \expandafter
  \endgroup
  \childdoctmp
}
%    \end{macrocode}

%\iffalse
%</package>
%\fi
%
\endinput
|\\
|\childdocforwardprefix{final}{child}|
\end{tabular}
\end{center}
%

Note that when several versions of a main file and/or of each child file
are to be generated, it may be convenient to set up a |Makefile| or
shell script to automatise the process.

%%%%%%%%%%%%%%%%%%%%%%%%%%%%%%%%%%%%%%%%%%%%%%%%%%%%%%%%%%%%%%%%%%%%%%%%%%%%%%%%
\subsection{Command Line Processing}
\label{sec:commandline}

The effect of redirection files can also be achieved by invoking
the \LaTeX{} compiler with a more elaborate command line.
Most conveniently this should be done as part
of a shell script or a |Makefile|.

When using \textsf{childdoc} in the main file, the following
command lines effectively perform a redirection
(note that depending on the shell being used,
backslashes may have to be doubled: `|\|' $\to$ `|\\|'):
%
\begin{center}
|... -jobname "|\textit{target}|" |\\|"|[\textit{flags}]%
|% \iffalse
%
% childdoc.dtx Copyright (C) 2017-2018 Niklas Beisert
%
% This work may be distributed and/or modified under the
% conditions of the LaTeX Project Public License, either version 1.3
% of this license or (at your option) any later version.
% The latest version of this license is in
%   http://www.latex-project.org/lppl.txt
% and version 1.3 or later is part of all distributions of LaTeX
% version 2005/12/01 or later.
%
% This work has the LPPL maintenance status `maintained'.
%
% The Current Maintainer of this work is Niklas Beisert.
%
% This work consists of the files childdoc.dtx and childdoc.ins
% and the derived files childdoc.def and cdocsamp.tex with
% cdocsch1.tex, cdocsch2.tex, cdocsdrf.tex, cdocsfn1.tex, cdocsfn2.tex.
%
%<package>\ifdefined\childdocmain\endinput\fi
%<package>\ProvidesFile{childdoc.def}[2018/12/30 v2.0 child document driver]
%<samplemain>\ProvidesFile{cdocsamp.tex}[2018/12/30 v2.0 sample for childdoc]
%<*driver>
%\ProvidesFile{childdoc.drv}[2018/12/30 v2.0 childdoc reference manual file]
\PassOptionsToClass{10pt,a4paper}{article}
\documentclass{ltxdoc}

\usepackage[margin=35mm]{geometry}
\usepackage{hyperref}
\usepackage{hyperxmp}
\usepackage[usenames]{color}

\hypersetup{colorlinks=true}
\hypersetup{pdfstartview=FitH}
\hypersetup{pdfpagemode=UseNone}
\hypersetup{pdfsource={}}
\hypersetup{pdflang={en-UK}}
\hypersetup{pdfcopyright={Copyright 2017-2018 Niklas Beisert.
  This work may be distributed and/or modified under the
  conditions of the LaTeX Project Public License, either version 1.3
  of this license or (at your option) any later version.}}
\hypersetup{pdflicenseurl={http://www.latex-project.org/lppl.txt}}
\hypersetup{pdfcontactaddress={ETH Zurich, ITP, HIT K,
  Wolfgang-Pauli-Strasse 27}}
\hypersetup{pdfcontactpostcode={8093}}
\hypersetup{pdfcontactcity={Zurich}}
\hypersetup{pdfcontactcountry={Switzerland}}
\hypersetup{pdfcontactemail={nbeisert@itp.phys.ethz.ch}}
\hypersetup{pdfcontacturl={http://people.phys.ethz.ch/\xmptilde nbeisert/}}

\newcommand{\secref}[1]{\hyperref[#1]{section \ref*{#1}}}

\parskip1ex
\parindent0pt
\let\olditemize\itemize
\def\itemize{\olditemize\parskip0pt}

\begin{document}

\title{The \textsf{childdoc} Package}
\hypersetup{pdftitle={The childdoc Package}}
\author{Niklas Beisert\\[2ex]
  Institut f\"ur Theoretische Physik\\
  Eidgen\"ossische Technische Hochschule Z\"urich\\
  Wolfgang-Pauli-Strasse 27, 8093 Z\"urich, Switzerland\\[1ex]
  \href{mailto:nbeisert@itp.phys.ethz.ch}
  {\texttt{nbeisert@itp.phys.ethz.ch}}}
\hypersetup{pdfauthor={Niklas Beisert}}
\hypersetup{pdfsubject={Manual for the LaTeX2e Package childdoc}}
\date{30 December 2018, \textsf{v2.0}}
\maketitle

\begin{abstract}\noindent
\textsf{childdoc} is a \LaTeXe{} package
that enables the direct compilation
of document sections included by |\include|
to individual files.
\end{abstract}

\begingroup
\parskip0ex
\tableofcontents
\endgroup

%%%%%%%%%%%%%%%%%%%%%%%%%%%%%%%%%%%%%%%%%%%%%%%%%%%%%%%%%%%%%%%%%%%%%%%%%%%%%%%%
%%%%%%%%%%%%%%%%%%%%%%%%%%%%%%%%%%%%%%%%%%%%%%%%%%%%%%%%%%%%%%%%%%%%%%%%%%%%%%%%
\section{Introduction}

\LaTeX{} provides a mechanism to structure a large document (such as a book)
into a main file and several child files (containing the chapters)
using the |\include| command.
This mechanism is beneficial for documents
which span hundreds of pages in order to
make the source file(s) more manageable.
Moreover, compilation can be restricted to
selected child files by means of the |\includeonly| command.
The latter feature can be used to reduce the compilation time while editing
(this was significantly more useful in the earlier days of \LaTeX{})
or to generate a smaller document which is easier to navigate.
Another application of |\includeonly| is to generate
documents consisting of selected parts of the complete document.

However, there are a few drawbacks of the plain |\include| mechanism:
\begin{itemize}
\item
The child files cannot be compiled on their own,
they can only be compiled via the main file.
A naive editing environment
(such as a text editor with an option
to have the current file processed by \LaTeX)
may require one to switch to the main file before compiling;
attempting to compile the child file produces errors.
\item
The main file must be modified (each time)
to adjust the |\includeonly| command
to the present needs. This easily leaves the main file in a messy state.
\item
The generated document will always carry the filename
of the main document. This is inconvenient if
several child files are to be compiled and
to be kept for distribution.
\end{itemize}

The present package provides a simple interface
to make child files individually compilable by \LaTeX{}.
Compiling a child file then has the same effect as compiling
the main file with an |\includeonly| command
to select the appropriate child.
Moreover the generated document will carry the name of the child
rather than the main file.
This resolves all three above issues.

This feature is meant to make the editing of books,
thesis documents and lecture notes somewhat more convenient.
However, the package can also be used efficiently for
composing a series of documents (such as exercise sheets)
which are typically distributed individually.
It then assists the author in generating the individual documents
(potentially in different versions)
as well as a document containing the collected series.
Another application is in developing style files
or other kinds of included material
where compilation of the style file could redirect
to a sample or test file.

%%%%%%%%%%%%%%%%%%%%%%%%%%%%%%%%%%%%%%%%%%%%%%%%%%%%%%%%%%%%%%%%%%%%%%%%%%%%%%%%
%%%%%%%%%%%%%%%%%%%%%%%%%%%%%%%%%%%%%%%%%%%%%%%%%%%%%%%%%%%%%%%%%%%%%%%%%%%%%%%%
\section{Usage}

First of all, the package \textsf{childdoc} is \emph{not} a standard
\LaTeXe{} |.sty| style file! Therefore it needs to be invoked in
a non-standard way.

%%%%%%%%%%%%%%%%%%%%%%%%%%%%%%%%%%%%%%%%%%%%%%%%%%%%%%%%%%%%%%%%%%%%%%%%%%%%%%%%
\subsection{Included Files}
\label{sec:include}

%%%%%%%%%%%%%%%%%%%%%%%%%%%%%%%%%%%%%%%%
\DescribeMacro{\childdocmain}
To use the package, add the commands
\begin{center}
\begin{tabular}{l}
|% \iffalse
%
% childdoc.dtx Copyright (C) 2017-2018 Niklas Beisert
%
% This work may be distributed and/or modified under the
% conditions of the LaTeX Project Public License, either version 1.3
% of this license or (at your option) any later version.
% The latest version of this license is in
%   http://www.latex-project.org/lppl.txt
% and version 1.3 or later is part of all distributions of LaTeX
% version 2005/12/01 or later.
%
% This work has the LPPL maintenance status `maintained'.
%
% The Current Maintainer of this work is Niklas Beisert.
%
% This work consists of the files childdoc.dtx and childdoc.ins
% and the derived files childdoc.def and cdocsamp.tex with
% cdocsch1.tex, cdocsch2.tex, cdocsdrf.tex, cdocsfn1.tex, cdocsfn2.tex.
%
%<package>\ifdefined\childdocmain\endinput\fi
%<package>\ProvidesFile{childdoc.def}[2018/12/30 v2.0 child document driver]
%<samplemain>\ProvidesFile{cdocsamp.tex}[2018/12/30 v2.0 sample for childdoc]
%<*driver>
%\ProvidesFile{childdoc.drv}[2018/12/30 v2.0 childdoc reference manual file]
\PassOptionsToClass{10pt,a4paper}{article}
\documentclass{ltxdoc}

\usepackage[margin=35mm]{geometry}
\usepackage{hyperref}
\usepackage{hyperxmp}
\usepackage[usenames]{color}

\hypersetup{colorlinks=true}
\hypersetup{pdfstartview=FitH}
\hypersetup{pdfpagemode=UseNone}
\hypersetup{pdfsource={}}
\hypersetup{pdflang={en-UK}}
\hypersetup{pdfcopyright={Copyright 2017-2018 Niklas Beisert.
  This work may be distributed and/or modified under the
  conditions of the LaTeX Project Public License, either version 1.3
  of this license or (at your option) any later version.}}
\hypersetup{pdflicenseurl={http://www.latex-project.org/lppl.txt}}
\hypersetup{pdfcontactaddress={ETH Zurich, ITP, HIT K,
  Wolfgang-Pauli-Strasse 27}}
\hypersetup{pdfcontactpostcode={8093}}
\hypersetup{pdfcontactcity={Zurich}}
\hypersetup{pdfcontactcountry={Switzerland}}
\hypersetup{pdfcontactemail={nbeisert@itp.phys.ethz.ch}}
\hypersetup{pdfcontacturl={http://people.phys.ethz.ch/\xmptilde nbeisert/}}

\newcommand{\secref}[1]{\hyperref[#1]{section \ref*{#1}}}

\parskip1ex
\parindent0pt
\let\olditemize\itemize
\def\itemize{\olditemize\parskip0pt}

\begin{document}

\title{The \textsf{childdoc} Package}
\hypersetup{pdftitle={The childdoc Package}}
\author{Niklas Beisert\\[2ex]
  Institut f\"ur Theoretische Physik\\
  Eidgen\"ossische Technische Hochschule Z\"urich\\
  Wolfgang-Pauli-Strasse 27, 8093 Z\"urich, Switzerland\\[1ex]
  \href{mailto:nbeisert@itp.phys.ethz.ch}
  {\texttt{nbeisert@itp.phys.ethz.ch}}}
\hypersetup{pdfauthor={Niklas Beisert}}
\hypersetup{pdfsubject={Manual for the LaTeX2e Package childdoc}}
\date{30 December 2018, \textsf{v2.0}}
\maketitle

\begin{abstract}\noindent
\textsf{childdoc} is a \LaTeXe{} package
that enables the direct compilation
of document sections included by |\include|
to individual files.
\end{abstract}

\begingroup
\parskip0ex
\tableofcontents
\endgroup

%%%%%%%%%%%%%%%%%%%%%%%%%%%%%%%%%%%%%%%%%%%%%%%%%%%%%%%%%%%%%%%%%%%%%%%%%%%%%%%%
%%%%%%%%%%%%%%%%%%%%%%%%%%%%%%%%%%%%%%%%%%%%%%%%%%%%%%%%%%%%%%%%%%%%%%%%%%%%%%%%
\section{Introduction}

\LaTeX{} provides a mechanism to structure a large document (such as a book)
into a main file and several child files (containing the chapters)
using the |\include| command.
This mechanism is beneficial for documents
which span hundreds of pages in order to
make the source file(s) more manageable.
Moreover, compilation can be restricted to
selected child files by means of the |\includeonly| command.
The latter feature can be used to reduce the compilation time while editing
(this was significantly more useful in the earlier days of \LaTeX{})
or to generate a smaller document which is easier to navigate.
Another application of |\includeonly| is to generate
documents consisting of selected parts of the complete document.

However, there are a few drawbacks of the plain |\include| mechanism:
\begin{itemize}
\item
The child files cannot be compiled on their own,
they can only be compiled via the main file.
A naive editing environment
(such as a text editor with an option
to have the current file processed by \LaTeX)
may require one to switch to the main file before compiling;
attempting to compile the child file produces errors.
\item
The main file must be modified (each time)
to adjust the |\includeonly| command
to the present needs. This easily leaves the main file in a messy state.
\item
The generated document will always carry the filename
of the main document. This is inconvenient if
several child files are to be compiled and
to be kept for distribution.
\end{itemize}

The present package provides a simple interface
to make child files individually compilable by \LaTeX{}.
Compiling a child file then has the same effect as compiling
the main file with an |\includeonly| command
to select the appropriate child.
Moreover the generated document will carry the name of the child
rather than the main file.
This resolves all three above issues.

This feature is meant to make the editing of books,
thesis documents and lecture notes somewhat more convenient.
However, the package can also be used efficiently for
composing a series of documents (such as exercise sheets)
which are typically distributed individually.
It then assists the author in generating the individual documents
(potentially in different versions)
as well as a document containing the collected series.
Another application is in developing style files
or other kinds of included material
where compilation of the style file could redirect
to a sample or test file.

%%%%%%%%%%%%%%%%%%%%%%%%%%%%%%%%%%%%%%%%%%%%%%%%%%%%%%%%%%%%%%%%%%%%%%%%%%%%%%%%
%%%%%%%%%%%%%%%%%%%%%%%%%%%%%%%%%%%%%%%%%%%%%%%%%%%%%%%%%%%%%%%%%%%%%%%%%%%%%%%%
\section{Usage}

First of all, the package \textsf{childdoc} is \emph{not} a standard
\LaTeXe{} |.sty| style file! Therefore it needs to be invoked in
a non-standard way.

%%%%%%%%%%%%%%%%%%%%%%%%%%%%%%%%%%%%%%%%%%%%%%%%%%%%%%%%%%%%%%%%%%%%%%%%%%%%%%%%
\subsection{Included Files}
\label{sec:include}

%%%%%%%%%%%%%%%%%%%%%%%%%%%%%%%%%%%%%%%%
\DescribeMacro{\childdocmain}
To use the package, add the commands
\begin{center}
\begin{tabular}{l}
|\input{childdoc.def}|\\
|\childdocmain{}|\\
\end{tabular}
\end{center}
at the very top of the main \LaTeX{} file,
in particular \emph{before} the |\documentclass| statement!
The argument of |\childdocmain| should be left empty
(but it must be present).

%%%%%%%%%%%%%%%%%%%%%%%%%%%%%%%%%%%%%%%%
\DescribeMacro{\childdocof}
Furthermore, add the commands
\begin{center}
\begin{tabular}{l}
|\input{childdoc.def}|\\
|\childdocof{|\textit{main}|}|\\
\end{tabular}
\end{center}
at the top of every child file \textit{child}
which is included by |\include{|\textit{child}|}|
from within the main file
(or at least for those files to be compiled individually).
The argument \textit{main} must be the filename of the main file.

There are a couple of
considerations in setting up the main and child documents:

%%%%%%%%%%%%%%%%%%%%%%%%%%%%%%%%%%%%%%%%
\paragraph{Restrictions.}

Please note the following restrictions:
\begin{itemize}
\item
|\childdocmain| must be called with one argument \textit{main}
to ensure compatibility with earlier version of the package.
It must either be empty (|\childdocmain{}|)
or precisely match the filename of the main file in which it is specified.
See \secref{sec:detection} for further information.
\item
The filename \textit{main} must be specified without the |.tex| extension.
\item
The filename \textit{main} is case sensitive
(even in case-insensitive file systems)
due to internal string comparison.
\item
The argument \textit{main} should be fully expanded, it cannot be a macro.
\item
Subdirectories and special characters should be avoided in filenames.
\item
The command |\childdocmain{|\textit{main}|}| must be followed by a whitespace.
It should not be followed immediately by another command
or by a comment mark `|%|'.
This is because the \TeX{} parser reads the token immediately following
the argument of |\childdocmain| and puts it
at the beginning of every child section;
however, a white\-space is ignored.
\end{itemize}

%%%%%%%%%%%%%%%%%%%%%%%%%%%%%%%%%%%%%%%%
\paragraph{Content of Main File.}

It is advisable to place all content in the child files included by |\include|.
Any output contained in the main file will appear in all child documents
unless suppressed manually;
it cannot be suppressed automatically by the |\includeonly| directive
and thus should normally be avoided.
A method to include some content in the main file
by means of conditional processing is described in \secref{sec:conditional}.

%%%%%%%%%%%%%%%%%%%%%%%%%%%%%%%%%%%%%%%%
\paragraph{Page Numbering.}

When only a part of the document is compiled,
the appropriate numbering of pages
(as well as other status parameters)
is determined from the |.aux| files.
The latter contain information from previous passes.
However this information needs to propagate through
all intermediate child documents.
Therefore the page numbering in child documents may well
be inconsistent until the complete document is compiled at least once.

A useful (if unconventional) way to always ensure a consistent
page numbering is to restart the numbering in each child document
and denote the pages by `\textit{child}|.|\textit{page}'
where \textit{child} represents the chapter/section number of the child file.
This can be achieved by the command
|\numberwithin{page}{|\textit{child}|}|
of the \textsf{amsmath} package
where \textit{child} can be |chapter| or |section|
depending on the chosen structuring.
Alternatively, one can modify the macro |\thepage| appropriately
and reset the counter |page| at the start of each child file.

%%%%%%%%%%%%%%%%%%%%%%%%%%%%%%%%%%%%%%%%%%%%%%%%%%%%%%%%%%%%%%%%%%%%%%%%%%%%%%%%
\subsection{Conditional Processing}
\label{sec:conditional}

The package provides a mechanism to compile different versions
of a document. To customise the versions further some conditional processing
can come in handy to distinguish which version is being compiled.
The package provides two macros to describe the compilation context:

%%%%%%%%%%%%%%%%%%%%%%%%%%%%%%%%%%%%%%%%
\DescribeMacro{\ifchilddoc}
The conditional |\ifchilddoc| distinguishes between the compilation of
child documents and the main document:
%
\begin{center}
|\ifchilddoc |\textit{child-code}| |[|\||else |\textit{main-code}]| \||fi|
\end{center}

%%%%%%%%%%%%%%%%%%%%%%%%%%%%%%%%%%%%%%%%
\DescribeMacro{\childdocname}
\DescribeMacro{\childdocjob}
The macro |\childdocname| contains the filename (without extension)
of the main or child file being processed.
Note that |\childdocjob| will always contain the name of the main file.

%%%%%%%%%%%%%%%%%%%%%%%%%%%%%%%%%%%%%%%%
\paragraph{Title Page.}

Conditional processing can be used to include a title or banner page
in the main document when proper precautions are taken.
Importantly, the code in the main file should ensure that the page counter
(as well as other status parameters which are stored in the |.aux| files)
takes the same value after the conditional processing.
Otherwise the page numbers may take divergent values
depending on which part is compiled.

For example, a title page could be declared by:
%
\begin{center}
\begin{tabular}{l}
|\ifchilddoc\||else|\\
|\addtocounter{page}{-1}|\\
\textit{code for title page}\\
|\newpage|\\
|\||fi|
\end{tabular}
\end{center}
%
A banner page for the child documents can be generated by:
%
\begin{center}
\begin{tabular}{l}
|\ifchilddoc|\\
|\addtocounter{page}{-1}|\\
\textit{code for banner page}\\
|\newpage|\\
|\||fi|
\end{tabular}
\end{center}
%
Here one could write a message such as:
\begin{center}
|This is the part \childdocname{} of \childdocjob{}.|
\end{center}

%%%%%%%%%%%%%%%%%%%%%%%%%%%%%%%%%%%%%%%%%%%%%%%%%%%%%%%%%%%%%%%%%%%%%%%%%%%%%%%%
\subsection{Flags}
\label{sec:flags}

The package makes it easy to generate different versions
of the main or child documents.
To this end compilation flags can be defined
and assigned different default values.
They will be particularly useful in conjunction
with the forwarding mechanism described in \secref{sec:forward}.

For example, it may be useful to have a flag |\version|
which can be set to |draft| or |final|.
The document source will contain some conditional code
depending on the value of |\version|.
Suppose further, the flag should default to |final| for the main file
and to |draft| for child files
which is a natural assignment for editing the document.
This is achieved by placing the following code
in the preamble of the main document
(below the |\childdocmain| directive):
%
\begin{center}
\begin{tabular}{l}
|\ifchilddoc|\\
|\providecommand{\version}{draft}|\\
|\||else|\\
|\providecommand{\version}{final}|\\
|\||fi|
\end{tabular}
\end{center}
%
The definition by |\providecommand| makes sure
that previous definitions are not overwritten.
Further statements |\providecommand{\version}{...}|
can thus be added before the above code to override it.

For the main file, one might add a line
(between |\childdocmain| and the above block)
%
\begin{center}
|%\ifchilddoc\||else\providecommand{\version}{draft}\||fi|
\end{center}
%
which can be uncommented to produce a draft version.
Likewise one can add a line to the very top of a child file
(above the |\childdocof{|\textit{main}|}| directive)
%
\begin{center}
|%\providecommand{\version}{final}|
\end{center}
%
which can be uncommented to produce the final version of this child document.

%%%%%%%%%%%%%%%%%%%%%%%%%%%%%%%%%%%%%%%%%%%%%%%%%%%%%%%%%%%%%%%%%%%%%%%%%%%%%%%%
\subsection{Forwarding}
\label{sec:forward}

Different versions of the main or child documents
using compilation flags as described in \secref{sec:flags}
can be (permanently) stored in different files
for convenient compilation, viewing and distribution.
To this end, the package defines a command
to pass on compilation to a different file:

%%%%%%%%%%%%%%%%%%%%%%%%%%%%%%%%%%%%%%%%
\DescribeMacro{\childdocforward}
The command |\childdocforward| redirects processing to
another source file:
%
\begin{center}
\begin{tabular}{l}
|\input{childdoc.def}|\\
|\childdocforward[|\textit{main}|]{|\textit{dest}|}|\\
\end{tabular}
\end{center}
%
The argument \textit{dest} is the destination file
(without extension).
It should be the main file or one of the child files.
Note that further \textsf{childdoc} directives
such as |\childdocof| and |\childdocforward|
in the indicated file will be processed in this form.
The optional argument \textit{main}
passes on directly to the main file \textit{main}
while pretending to compile the child \textit{dest}.
This form behaves as if \textit{dest}
issues |\childdocof{|\textit{main}|}| right away,
and no further \textsf{childdoc} directives will be processed.

%%%%%%%%%%%%%%%%%%%%%%%%%%%%%%%%%%%%%%%%
\DescribeMacro{\...prefix}
In the alternative form |\childdocforwardprefix|,
%
\begin{center}
\begin{tabular}{l}
|\input{childdoc.def}|\\
|\childdocforwardprefix[|\textit{main}|]{|\textit{prefix}|}{|\textit{dest}|}|
\end{tabular}
\end{center}
%
the destination file is determined by a pattern
depending on the current file:
To make this work, the current file must be called
`{\textit{prefix}\hspace{0.2em}\textit{suffix}}'
with \textit{prefix} matching precisely the argument.
Processing is then passed on to the file
`{\textit{dest}\hspace{0.2em}\textit{suffix}}'.
Surely, the same effect is achieved by
directly specifying the
argument `{\textit{dest}\hspace{0.2em}\textit{suffix}}'
in the first form.
However, that requires to set up a different file
for each child. With the alternative form of the command
all these files can have exactly the same content
which simplifies setting them up and maintaining them.

For example, the following file |draft.tex|
with a compilation flag |\version| as described in \secref{sec:flags}
compiles the main document as a draft:
%
\begin{center}
\begin{tabular}{l}
|\def\version{draft}|\\
|\input{childdoc.def}|\\
|\childdocforward{|\textit{main}|}|
\end{tabular}
\end{center}
%
Likewise, the following files |final|\textit{nn}|.tex|
compile the final version of the child document
|child|\textit{nn}|.tex|:
%
\begin{center}
\begin{tabular}{l}
|\def\version{final}|\\
|\input{childdoc.def}|\\
|\childdocforwardprefix{final}{child}|
\end{tabular}
\end{center}
%

Note that when several versions of a main file and/or of each child file
are to be generated, it may be convenient to set up a |Makefile| or
shell script to automatise the process.

%%%%%%%%%%%%%%%%%%%%%%%%%%%%%%%%%%%%%%%%%%%%%%%%%%%%%%%%%%%%%%%%%%%%%%%%%%%%%%%%
\subsection{Command Line Processing}
\label{sec:commandline}

The effect of redirection files can also be achieved by invoking
the \LaTeX{} compiler with a more elaborate command line.
Most conveniently this should be done as part
of a shell script or a |Makefile|.

When using \textsf{childdoc} in the main file, the following
command lines effectively perform a redirection
(note that depending on the shell being used,
backslashes may have to be doubled: `|\|' $\to$ `|\\|'):
%
\begin{center}
|... -jobname "|\textit{target}|" |\\|"|[\textit{flags}]%
|\input{childdoc.def}\childdocforward[|\textit{main}|]{|\textit{dest}|}"|
\end{center}
%
Here \textit{target} is the name of the output file,
\textit{main} is the name of the main file
and \textit{dest} is the name of the main or child file to be processed
(all filenames without extensions).
The optional argument \textit{main} can be omitted
if \textit{main} matches \textit{dest}.
Optionally, compilation \textit{flags} can be defined via |\def| commands.
This command line makes the \TeX{} engine believe
it is compiling the file \textit{target}
whose content is specified as the latter parameter.
The provided code then forwards the processing to
\textit{main} or \textit{dest} as described in \secref{sec:forward}.

%%%%%%%%%%%%%%%%%%%%%%%%%%%%%%%%%%%%%%%%%%%%%%%%%%%%%%%%%%%%%%%%%%%%%%%%%%%%%%%%
\subsection{Include by Input}
\label{sec:input}

Including child documents by |\include| has some restrictions by design.
Most notably, the content of a child document always occupies
its own set of pages; pages cannot be shared between child documents.
Usually, this behaviour makes perfect sense
because each child document contain an essential part of the document.
However, in some situations it may be desirable to compose
a document from a collection of parts
without having mandatory page breaks between then.
For this case, the package
provides a mechanism to include parts
by |\input| which can also be processed individually.
However, by construction this mechanism
requires manual handling of the content to be output.

%%%%%%%%%%%%%%%%%%%%%%%%%%%%%%%%%%%%%%%%
\DescribeMacro{\ifchilddocmanual}
The main file should be prepared as usual, see \secref{sec:include}.
However, the document body must make a distinction
between processing of an individual part and of the main document, e.g.:
%
\begin{center}
\begin{tabular}{l}
|\ifchilddocmanual|\\
|\input{\childdocname}|\\
|\||else|\\
\textit{document body with }|\input{|\textit{part}|}|\\
|\||fi|
\end{tabular}
\end{center}
%
The conditional |\ifchilddocmanual| is true whenever
a part to be included by |\input| is being compiled,
and the name of the part is stored in |\childdocname|.

%%%%%%%%%%%%%%%%%%%%%%%%%%%%%%%%%%%%%%%%
\DescribeMacro{\childdocby}
Each part to be included by |\input| should start with:
%
\begin{center}
\begin{tabular}{l}
|\input{childdoc.def}|\\
|\childdocby{|\textit{main}|}|\\
\end{tabular}
\end{center}
%
The directive |\childdocby| is similar to |\childdocof|
described in \secref{sec:include},
but the subsequent selection of content must be done manually.
To that end, both |\ifchilddoc| and |\ifchilddocmanual|
will be true upon processing of a part,
and the name of the part is stored in |\childdocname|.
Note that |\jobname| will be set to the filename of the current part
so that each part receives an individual |.aux| file
that does not interfere with the |.aux| file(s) of the main document.
This behaviour can be altered by the alternative form
|\childdocby[*]{|\textit{main}|}| (with a non-empty optional argument)
which uses the |.aux| file of the main document
by setting |\jobname| to \textit{main}.

%%%%%%%%%%%%%%%%%%%%%%%%%%%%%%%%%%%%%%%%%%%%%%%%%%%%%%%%%%%%%%%%%%%%%%%%%%%%%%%%
\subsection{Driver Development}
\label{sec:driver}

The \textsf{childdoc} mechanism can also be use for the development
of definition files such as \LaTeX{} styles or classes.
This case differs from the above setup with multiple parts
included by |\include| in that no |\includeonly| should be invoked.
This can be achieved by starting the include file
(before |\ProvidesPackage|) with:
%
\begin{center}
\begin{tabular}{l}
|\input{childdoc.def}|\\
|\childdocforward{|\textit{main}|}|\\
\end{tabular}
\end{center}
%
or alternatively with:
%
\begin{center}
\begin{tabular}{l}
|\input{childdoc.def}|\\
|\childdocby{|\textit{main}|}|\\
\end{tabular}
\end{center}
%
Both forms have slightly different effects as described above.
The main file is prepared as usual, see \secref{sec:include}.

%%%%%%%%%%%%%%%%%%%%%%%%%%%%%%%%%%%%%%%%%%%%%%%%%%%%%%%%%%%%%%%%%%%%%%%%%%%%%%%%
\subsection{Legacy Detection}
\label{sec:detection}

The directive |\childdocmain| in the main file can detect
whether the complete document or merely a child is to be compiled
even without using the directive |\childdocof|.
This method is deprecated because it is less robust
and there is no compelling reason to use it;
it is merely provided for backward compatibility
and it may be removed in future versions.

If the detection mechanism is to be used,
it is mandatory to correctly specify
the filename of the main file as the argument of |\childdocmain|:
%
\begin{center}
\begin{tabular}{l}
|\input{childdoc.def}|\\
|\childdocmain{|\textit{main}|}|\\
\end{tabular}
\end{center}
%
If |\jobname| does not match the argument \textit{main} of |\childdocmain|,
it is assumed that |\jobname| points to the child file to be compiled.
When using |\childdocmain| with the main file specified as argument,
it suffices to start a child file
with just |\input{|\textit{main}|}|
without loading of the package and using |\childdocof|.
If instead all processing is done
with the appropriate \textsf{childdoc} directives,
the argument of \textit{main} of |\childdocmain| can be empty.

An alternative version of the command line processing described
in \secref{sec:commandline} using the detection mechanism reads:
%
\begin{center}
|... -jobname "|\textit{target}|" "|[\textit{flags}]%
[|\def\jobname{|\textit{dest}|}|]|\input{|\textit{main}|}"|
\end{center}

%%%%%%%%%%%%%%%%%%%%%%%%%%%%%%%%%%%%%%%%%%%%%%%%%%%%%%%%%%%%%%%%%%%%%%%%%%%%%%%%
\subsection{Manual Code}
\label{sec:manual}

In case one cannot be certain whether the definitions file |childdoc.def|
is installed on the target \TeX{} distribution
and one prefers not to ship it,
it is conceivable to paste a few relevant commands into the sources.

To that end, drop all statements |\input{childdoc.def}|
and perform the replacements as outlined below.
Instead of |\childdocmain{|\textit{main}|}| add the following code
to the top of the main file:
%
\begin{center}
\begin{tabular}{l}
|\||ifdefined\childdocname\endinput\||fi\newif\ifchilddoc|\\
|\edef\childdocname{\scantokens\expandafter{\jobname\noexpand}}|\\
|\def\childdocmain{|\textit{main}|}\||ifx\childdocmain\childdocname\||else|\\
|\childdoctrue\includeonly{\childdocname}\let\jobname\childdocmain\||fi|\\
\end{tabular}
\end{center}
%
Instead of |\childdocof{|\textit{main}|}| just include the main file
at the top of each child file:
%
\begin{center}
|\input{|\textit{main}|}|
\end{center}
%
A simple redirection |\childdocforward{|\textit{dest}|}| is achieved by:
%
\begin{center}
|\def\jobname{|\textit{dest}|}\input{\jobname}|
\end{center}
%
The redirection with prefix
|\childdocforwardprefix[|\textit{prefix}|]{|\textit{dest}|}|
is accomplished by:
%
\begin{center}
\begin{tabular}{l}
|{\edef\jobname{\scantokens\expandafter{\jobname\noexpand}}|\\
|\def\redirectjob |\textit{prefix}|#1~~~{\gdef\jobname{|\textit{dest}|#1}}|\\
|\expandafter\redirectjob\jobname~~~}\input{\jobname}|
\end{tabular}
\end{center}

In an alternative approach,
child documents can be compiled by a specific command line
without additional code or specific definitions:
%
\begin{center}
|... -jobname "|\textit{target}|" "|[\textit{flags}]%
|\includeonly{|\textit{dest}|}\input{|\textit{main}|}"|
\end{center}
%

%%%%%%%%%%%%%%%%%%%%%%%%%%%%%%%%%%%%%%%%%%%%%%%%%%%%%%%%%%%%%%%%%%%%%%%%%%%%%%%%
%%%%%%%%%%%%%%%%%%%%%%%%%%%%%%%%%%%%%%%%%%%%%%%%%%%%%%%%%%%%%%%%%%%%%%%%%%%%%%%%
\section{Information}

%%%%%%%%%%%%%%%%%%%%%%%%%%%%%%%%%%%%%%%%%%%%%%%%%%%%%%%%%%%%%%%%%%%%%%%%%%%%%%%%
\subsection{Copyright}

Copyright \copyright{} 2017--2018 Niklas Beisert

This work may be distributed and/or modified under the
conditions of the \LaTeX{} Project Public License, either version 1.3
of this license or (at your option) any later version.
The latest version of this license is in
  \url{http://www.latex-project.org/lppl.txt}
and version 1.3 or later is part of all distributions of \LaTeX{}
version 2005/12/01 or later.

This work has the LPPL maintenance status `maintained'.

The Current Maintainer of this work is Niklas Beisert.

This work consists of the files |README.txt|, |childdoc.ins| and |childdoc.dtx|
as well as the derived files |childdoc.def|, |cdocsamp.tex|
with |cdocsch1.tex|, |cdocsch2.tex|, |cdocspt3.tex|, |cdocspt4.tex|,
|cdocsdrf.tex|, |cdocsfn1.tex|, |cdocsfn2.tex|
as well as |childdoc.pdf|.

%%%%%%%%%%%%%%%%%%%%%%%%%%%%%%%%%%%%%%%%%%%%%%%%%%%%%%%%%%%%%%%%%%%%%%%%%%%%%%%%
\subsection{Files and Installation}

The package consists of the files:
%
\begin{center}
\begin{tabular}{ll}
    |README.txt|   & readme file \\
    |childdoc.ins| & installation file \\
    |childdoc.dtx| & source file \\
    |childdoc.def| & definition file \\
    |cdocsamp.tex| & sample main file \\
    |cdocsch1.tex| & sample include file \\
    |cdocsch2.tex| & sample include file \\
    |cdocspt3.tex| & sample part file \\
    |cdocspt4.tex| & sample part file \\
    |cdocsdrf.tex| & sample redirection file \\
    |cdocsfn1.tex| & sample redirection file \\
    |cdocsfn2.tex| & sample redirection file \\
    |childdoc.pdf| & manual
\end{tabular}
\end{center}
%
The distribution consists of the files
|README.txt|, |childdoc.ins| and |childdoc.dtx|.
%
\begin{itemize}
\item
Run (pdf)\LaTeX{} on |childdoc.dtx|
to compile the manual |childdoc.pdf| (this file).
\item
Run \LaTeX{} on |childdoc.ins| to create the definitions file |childdoc.def|
and the sample |cdocsamp.tex| with include files
|cdocsch1.tex|, |cdocsch2.tex|, |cdocspt3.tex|, |cdocspt4.tex|,
|cdocsdrf.tex|, |cdocsfn1.tex|, |cdocsfn2.tex|.
Then copy the file |childdoc.def| to an appropriate directory of your \LaTeX{}
distribution, e.g.\ \textit{texmf-root}|/tex/latex/childdoc|.
\end{itemize}

%%%%%%%%%%%%%%%%%%%%%%%%%%%%%%%%%%%%%%%%%%%%%%%%%%%%%%%%%%%%%%%%%%%%%%%%%%%%%%%%
\subsection{Related CTAN Packages}

There are several other packages which offer a similar functionality:
%
\begin{itemize}
\item
The packages
\href{http://ctan.org/pkg/docmute}{\textsf{docmute}},
\href{http://ctan.org/pkg/includex}{\textsf{includex}} and
\href{http://ctan.org/pkg/standalone}{\textsf{standalone}}
provide commands to include only the document body of
a child file thus allowing both files to be compiled individually.
\item
The packages \href{http://ctan.org/pkg/subdocs}{\textsf{subdocs}}
and \href{http://ctan.org/pkg/subfiles}{\textsf{subfiles}}
provide structures in which the main and child documents can be
encapsulated and allowing them to be compiled individually.
The inclusion mechanism is different from the conventional |\include|.
\item
The package \href{http://ctan.org/pkg/combine}{\textsf{combine}}
is an elaborate solution to combine several documents into one.
\end{itemize}
%
See also the CTAN topic \href{http://ctan.org/topic/subdocs}{\textsf{subdocs}}
for further related packages.
The present package differs from the above solutions in that
a document structure constructed with the conventional |\include| mechanism
just needs two extra commands at the top of every file
such that all constituent files can be compiled individually.

%%%%%%%%%%%%%%%%%%%%%%%%%%%%%%%%%%%%%%%%%%%%%%%%%%%%%%%%%%%%%%%%%%%%%%%%%%%%%%%%
%\subsection{Feature Suggestions}
%
%The following is a list of features which may be useful for future
%versions of this package:
%%
%\begin{itemize}
%\item
%\ldots
%\end{itemize}

%%%%%%%%%%%%%%%%%%%%%%%%%%%%%%%%%%%%%%%%%%%%%%%%%%%%%%%%%%%%%%%%%%%%%%%%%%%%%%%%
\subsection{Revision History}

%%%%%%%%%%%%%%%%%%%%%%%%%%%%%%%%%%%%%%%%
\paragraph{v2.0:} 2018/12/30

\begin{itemize}
\item
immediate forward processing
\item
added |\childdocby| mechanism
\item
manual restructured
\end{itemize}

%%%%%%%%%%%%%%%%%%%%%%%%%%%%%%%%%%%%%%%%
\paragraph{v1.6:} 2018/01/17

\begin{itemize}
\item
application for development of include files
\item
corrections to manual
\end{itemize}

%%%%%%%%%%%%%%%%%%%%%%%%%%%%%%%%%%%%%%%%
\paragraph{v1.5:} 2017/05/21

\begin{itemize}
\item
more complete structuring introduced
\item
|\childdocof| introduced
\item
|\childdoc| renamed to |\childdocmain|
\item
|\childredirect| renamed to |\childdocforward| and |\childdocforwardprefix|
and functionality expanded
\end{itemize}

%%%%%%%%%%%%%%%%%%%%%%%%%%%%%%%%%%%%%%%%
\paragraph{v1.0:} 2017/04/27

\begin{itemize}
\item
manual and install package
\item
first version published on CTAN
\end{itemize}

%%%%%%%%%%%%%%%%%%%%%%%%%%%%%%%%%%%%%%%%
\paragraph{v0.6:} 2017/04/26

\begin{itemize}
\item
redirection mechanism added
\end{itemize}

%%%%%%%%%%%%%%%%%%%%%%%%%%%%%%%%%%%%%%%%
\paragraph{v0.5:} 2017/04/26

\begin{itemize}
\item
functionality in definition file
\end{itemize}


%%%%%%%%%%%%%%%%%%%%%%%%%%%%%%%%%%%%%%%%%%%%%%%%%%%%%%%%%%%%%%%%%%%%%%%%%%%%%%%%
%%%%%%%%%%%%%%%%%%%%%%%%%%%%%%%%%%%%%%%%%%%%%%%%%%%%%%%%%%%%%%%%%%%%%%%%%%%%%%%%
%%%%%%%%%%%%%%%%%%%%%%%%%%%%%%%%%%%%%%%%%%%%%%%%%%%%%%%%%%%%%%%%%%%%%%%%%%%%%%%%
\appendix

\settowidth\MacroIndent{\rmfamily\scriptsize 000\ }

 \DocInput{childdoc.dtx}

\end{document}
%</driver>
% \fi
%
% %%%%%%%%%%%%%%%%%%%%%%%%%%%%%%%%%%%%%%%%%%%%%%%%%%%%%%%%%%%%%%%%%%%%%%%%%%%%%%
% %%%%%%%%%%%%%%%%%%%%%%%%%%%%%%%%%%%%%%%%%%%%%%%%%%%%%%%%%%%%%%%%%%%%%%%%%%%%%%
% \section{Sample}
%\iffalse
%<*samplemain>
%\fi
%
% The following presents a sample document
% with two chapters, two parts, a title page,
% a compile flag as well as three forwarding files to set the flag.
% It consists of eight |.tex| files:
% \begin{center}
% \begin{tabular}{ll}
% |cdocsamp.tex|&main file\\
% |cdocsch1.tex|&include file for chapter 1\\
% |cdocsch2.tex|&include file for chapter 2\\
% |cdocspt3.tex|&include file for part 3\\
% |cdocspt4.tex|&include file for part 4\\
% |cdocsdrf.tex|&forwarding file for main file in draft mode\\
% |cdocsfi1.tex|&forwarding file for final version of chapter 1\\
% |cdocsfi2.tex|&forwarding file for final version of chapter 2\\
% \end{tabular}
% \end{center}
% Each of the eight files can be compiled directly by the \LaTeX{} compiler.
%
% %%%%%%%%%%%%%%%%%%%%%%%%%%%%%%%%%%%%%%
% \paragraph{Main File.}
%
% The main file is called |cdocsamp.tex|.
%
% Load the \textsf{childdoc} definitions and
% declare the filename for the main document:
%    \begin{macrocode}
\input{childdoc.def}
\childdocmain{}
%    \end{macrocode}

% Optional override for |\version| flag:
%    \begin{macrocode}
%%\ifchilddoc\else\providecommand{\version}{draft}\fi
%    \end{macrocode}

% Define the default values for the |\version| flag
% (|final| for the main file and |draft| for childs):
%    \begin{macrocode}
\ifchilddoc
\providecommand{\version}{draft}
\else
\providecommand{\version}{final}
\fi
%    \end{macrocode}

% Load the standard document class:
%    \begin{macrocode}
\documentclass[12pt]{article}
%    \end{macrocode}

% Start the document body:
%    \begin{macrocode}
\begin{document}
%    \end{macrocode}

% Declare a title page.
% Print title, part of document being processed and version flag:
%    \begin{macrocode}
\addtocounter{page}{-1}
\begin{center}
{\LARGE\bfseries{}childdoc example\par}
\vspace{1cm}
\ifchilddoc
\ifchilddocmanual part\else chapter\fi:
`\childdocname' of `\childdocjob'\par
\else
main document: `\childdocjob'\par
\fi
version: \version\par
\end{center}
\newpage
%    \end{macrocode}

% Manually include selected file,
% otherwise process as usual:
%    \begin{macrocode}
\ifchilddocmanual
\section*{part `\childdocname'}
\input{\childdocname}
\else
%    \end{macrocode}

% Include the two chapters:
%    \begin{macrocode}
\include{cdocsch1}
\include{cdocsch2}
%    \end{macrocode}

% Include the two parts unless only chapters should be displayed:
%    \begin{macrocode}
\ifchilddoc\else
\section{part three}
\input{cdocspt3}
\section{part four}
\input{cdocspt4}
\fi
%    \end{macrocode}

% Process as usual until here:
%    \begin{macrocode}
\fi
%    \end{macrocode}

% End of document body:
%    \begin{macrocode}
\end{document}
%    \end{macrocode}
%\iffalse
%</samplemain>
%\fi
%
% %%%%%%%%%%%%%%%%%%%%%%%%%%%%%%%%%%%%%%
% \paragraph{Chapter Include Files.}
%
% The include files are called |cdocsch1.tex| and |cdocsch2.tex|.
%
%\iffalse
%<*samplechap1|samplechap2>
%\fi

% Optional override for |\version| flag:
%    \begin{macrocode}
%%\providecommand{\version}{final}
%    \end{macrocode}

% Include the main document:
%    \begin{macrocode}
\input{childdoc.def}
\childdocof{cdocsamp}
%    \end{macrocode}

%\iffalse
%</samplechap1|samplechap2>
%\fi
%
%\iffalse
%<*samplechap1>
%\fi
% Some text for chapter 1:
%    \begin{macrocode}
\section{one}
some text in chapter one
%    \end{macrocode}

%\iffalse
%</samplechap1>
%\fi
% Some text for chapter 2:
%\iffalse
%<*samplechap2>
%\fi
%    \begin{macrocode}
\section{two}
more text in chapter two
%    \end{macrocode}

%\iffalse
%</samplechap2>
%\fi
%
% %%%%%%%%%%%%%%%%%%%%%%%%%%%%%%%%%%%%%%
% \paragraph{Part Include Files.}
%
% The include files are called |cdocspt3.tex| and |cdocspt4.tex|.
%
%\iffalse
%<*samplepart3|samplepart4>
%\fi

% Optional override for |\version| flag:
%    \begin{macrocode}
%%\providecommand{\version}{final}
%    \end{macrocode}

% Include the main document:
%    \begin{macrocode}
\input{childdoc.def}
\childdocby{cdocsamp}
%    \end{macrocode}

%\iffalse
%</samplepart3|samplepart4>
%\fi
%
%\iffalse
%<*samplepart3>
%\fi
% Some text for part 3:
%    \begin{macrocode}
some text in part three
%    \end{macrocode}

%\iffalse
%</samplepart3>
%\fi
% Some text for part 4:
%\iffalse
%<*samplepart4>
%\fi
%    \begin{macrocode}
more text in part four
%    \end{macrocode}

%\iffalse
%</samplepart4>
%\fi
%
% %%%%%%%%%%%%%%%%%%%%%%%%%%%%%%%%%%%%%%
% \paragraph{Forwarding for a Complete Draft.}
%
% The following forwarding file |cdocsdrf.tex|
% compiles the main document in draft mode:
%\iffalse
%<*sampledraft>
%\fi
%    \begin{macrocode}
\def\version{draft}
\input{childdoc.def}
\childdocforward{cdocsamp}
%    \end{macrocode}

%\iffalse
%</sampledraft>
%\fi
%
% %%%%%%%%%%%%%%%%%%%%%%%%%%%%%%%%%%%%%%
% \paragraph{Forwarding for Final Version of the Chapters.}
%
% The following forwarding files |cdocsfn1.tex| and |cdocsfn2.tex|
% (with identical content)
% compile the final versions of the child documents
% |cdocsch1.tex| and |cdocsch2.tex|, respectively:
%\iffalse
%<*samplefinal>
%\fi
%    \begin{macrocode}
\def\version{final}
\input{childdoc.def}
\childdocforwardprefix[cdocsamp]{cdocsfn}{cdocsch}
%    \end{macrocode}

%\iffalse
%</samplefinal>
%\fi
%
% %%%%%%%%%%%%%%%%%%%%%%%%%%%%%%%%%%%%%%
% \paragraph{Command Line Processing.}
%
% The following three command lines generate the output files
% |cdocscld|, |cdocscl1| and |cdocscl2|
% which should be identical to
% |cdocsdrf|, |cdocsch1| and |cdocsfn2|, respectively:
% \begin{center}
% \begin{tabular}{l}
% |latex -jobname cdocscld \|\\
% |  "\def\version{draft}\input{childdoc.def}\childdocforward{cdocsamp}"|\\
% |latex -jobname cdocscl1 \|\\
% |  "\input{childdoc.def}\childdocforward[cdocsamp]{cdocsch1}"|\\
% |latex -jobname cdocscl2 \|\\
% |  "\def\version{final}\input{childdoc.def}\childdocforward{cdocsch2}"|
% \end{tabular}
% \end{center}
% Note that the trailing backslash on each first line
% merely continues the input to the second line
% (for convenient cut ant paste).
% Furthermore, the command |latex| can be replaced by any
% of its alternative versions such as |pdflatex|.
%
% %%%%%%%%%%%%%%%%%%%%%%%%%%%%%%%%%%%%%%%%%%%%%%%%%%%%%%%%%%%%%%%%%%%%%%%%%%%%%%
% %%%%%%%%%%%%%%%%%%%%%%%%%%%%%%%%%%%%%%%%%%%%%%%%%%%%%%%%%%%%%%%%%%%%%%%%%%%%%%
% \section{Implementation}
%\iffalse
%<*package>
%\fi
%
% This section describes the definitions file |childdoc.def|.

% The definitions cannot be loaded using |\usepackage| or |\RequirePackage|
% which has a mechanism to prevent loading a style file more than once.
% When loading the definitions by means of |\input|
% multiple instances have to be prevented manually:
%\iffalse
%This code needs to be before the `\ProvidesFile' directive
%which is defined at the beginning of this file.
%Therefore it is also placed there and commented out here.
%</package>
%<*discard>
%\fi
%    \begin{macrocode}
\ifdefined\childdocmain\endinput\fi
%    \end{macrocode}
%\iffalse
%</discard>
%<*package>
%\fi
%
% \macro{\ifchilddoc}
% \macro{\ifchilddocmanual}
% The conditional |\ifchilddoc| tells whether a
% child (true) or main (false) document is being compiled.
% The conditional |\ifchilddocmanual| tells whether
% the |\includeonly| mechanism is used (false) or
% the selection of child files must be performed manually (true).
% The definitions initialise to false:
%    \begin{macrocode}
\newif\ifchilddoc
\newif\ifchilddocmanual
%    \end{macrocode}

% \macro{\childdocname}
% \macro{\childdocjob}
% The macro |\childdocname| stores the name of the main document
% to be compiled. The macro |\childdocjob| stores the name of
% the document on which the \LaTeX{} compiler was originally invoked.
% The content of |\jobname| cannot be compared
% to filenames specified in the source due to different catcodes.
% The following code rescans |\jobname|, stores the result
% in |\childdocname| and saves a copy in |\childdocjob|:
%    \begin{macrocode}
\edef\childdocname{\scantokens\expandafter{\jobname\noexpand}}
\let\childdocjob\childdocname
%    \end{macrocode}

% \macro{\childdocdisable}
% The macro |\childdocdisable| prevents the main file
% from being processed more than once.
% At this stage, the main document command |\childdocmain|
% is assumed to be called once again where it should do nothing.
% Any subsequent call to it should prevent
% a secondary processing of the main document
% It overwrites the forwarding commands
% |\childdocof| and |\childdocforward|
% with empty macros to prevent further inclusions of the main document:
%    \begin{macrocode}
\newcommand{\childdocdisable}
{
  \renewcommand{\childdocmain}[1]{\renewcommand{\childdocmain}[1]{\endinput}}
  \renewcommand{\childdocof}[1]{}
  \renewcommand{\childdocby}[2][]{}
  \renewcommand{\childdocforward}[2][]{}
  \renewcommand{\childdocdisable}{}
}
%    \end{macrocode}

% \macro{\childdocmain}
% The macro |\childdocmain| is to be called at the top of the main file
% with nothing or the main filename (without extension) as argument.
% First, it breaks loops.
% If the argument is not empty and does not match |\childdocname|
% (which is set by the first inclusion of |childdoc.def|),
% |\ifchilddoc| is set to true, |\includeonly| is applied to the child file
% and |\jobname| is set to the main file
% (for proper handling of |.aux| files):
%    \begin{macrocode}
\newcommand{\childdocmain}[1]
{
  \childdocdisable\childdocmain{}
  \if?#1?\else
    \begingroup
      \def\childdoctmp{#1}
      \ifx\childdoctmp\childdocname
        \def\childdoctmp{}
      \else
        \def\childdoctmp
        {
          \childdoctrue
          \includeonly{\childdocname}
          \def\childdocjob{#1}
          \def\jobname{#1}
        }
      \fi
      \expandafter
    \endgroup
    \childdoctmp
  \fi
}
%    \end{macrocode}

% \macro{\childdocof}
% The command |\childdocof| redirects
% compilation to the main file |#1|.
%    \begin{macrocode}
\newcommand{\childdocof}[1]
{
  \childdocdisable
  \childdoctrue
  \includeonly{\childdocname}
  \def\jobname{#1}
  \def\childdocjob{#1}
  \input{#1}
}
%    \end{macrocode}

% \macro{\childdocby}
% The command |\childdocby| ....
%    \begin{macrocode}
\newcommand{\childdocby}[2][]
{
  \childdocdisable
  \childdoctrue
  \childdocmanualtrue
  \if?#1?\else
    \def\jobname{#2}
  \fi
  \def\childdocjob{#2}
  \input{#2}
  \endinput
}
%    \end{macrocode}

% \macro{\childdocforward}
% The command |\childdocforward| redirects
% compilation to the main file or
% (if the optional argument is given) a child file.
% Parameters are set as if the main file
% or a child file starting with |\childdocof| was compiled.
% Then compilation is handed over to the main file:
%    \begin{macrocode}
\newcommand{\childdocforward}[2][]
{
  \begingroup
    \if?#1?
      \def\childdoctmp
      {
        \def\childdocname{#2}
        \def\childdocjob{#2}
        \def\jobname{#2}
        \input{#2}
        \endinput
      }
    \else
      \def\childdoctmp
      {
        \childdocdisable
        \def\childdocname{#2}
        \childdoctrue
        \includeonly{#2}
        \def\childdocjob{#1}
        \def\jobname{#1}
        \input{#1}
        \endinput
      }
    \fi
    \expandafter
  \endgroup
  \childdoctmp
}
%    \end{macrocode}

% \macro{\childdocforwardprefix}
% The command |\childdocforwardprefix| redirects
% compilation to the main or a child file by means of a pattern.
% The prefix |#1| in the current filename is replaced by |#2|
% and the suffix of the current filename is kept
% (it is assumed that the filename does not contain the substring `|~~~|'
% which is used as a delimiter).
% Compilation is handed over to the new file by |\childdocforward|:
%    \begin{macrocode}
\newcommand{\childdocforwardprefix}[3][]
{
  \begingroup
    \def\childdocextract #2##1~~~{\def\childdoctmp{\childdocforward[#1]{#3##1}}}
    \expandafter\childdocextract\childdocname~~~
    \expandafter
  \endgroup
  \childdoctmp
}
%    \end{macrocode}

% \macro{\childdoc}
% The deprecated macro |\childdoc| is a legacy version of |\childdocmain|:
%    \begin{macrocode}
\newcommand{\childdoc}{\childdocmain}
%    \end{macrocode}

% \macro{\childdocredirect}
% The deprecated macro |\childdocredirect| is a legacy version
% of |\childdocforward| and |\childdocforwardprefix|:
%    \begin{macrocode}
\newcommand{\childdocredirect}[2][]
{
  \begingroup
    \if?#1?
      \def\childdoctmp{\childdocforward{#2}}
    \else
      \def\childdoctmp{\childdocforwardprefix{#1}{#2}}
    \fi
    \expandafter
  \endgroup
  \childdoctmp
}
%    \end{macrocode}

%\iffalse
%</package>
%\fi
%
\endinput
|\\
|\childdocmain{}|\\
\end{tabular}
\end{center}
at the very top of the main \LaTeX{} file,
in particular \emph{before} the |\documentclass| statement!
The argument of |\childdocmain| should be left empty
(but it must be present).

%%%%%%%%%%%%%%%%%%%%%%%%%%%%%%%%%%%%%%%%
\DescribeMacro{\childdocof}
Furthermore, add the commands
\begin{center}
\begin{tabular}{l}
|% \iffalse
%
% childdoc.dtx Copyright (C) 2017-2018 Niklas Beisert
%
% This work may be distributed and/or modified under the
% conditions of the LaTeX Project Public License, either version 1.3
% of this license or (at your option) any later version.
% The latest version of this license is in
%   http://www.latex-project.org/lppl.txt
% and version 1.3 or later is part of all distributions of LaTeX
% version 2005/12/01 or later.
%
% This work has the LPPL maintenance status `maintained'.
%
% The Current Maintainer of this work is Niklas Beisert.
%
% This work consists of the files childdoc.dtx and childdoc.ins
% and the derived files childdoc.def and cdocsamp.tex with
% cdocsch1.tex, cdocsch2.tex, cdocsdrf.tex, cdocsfn1.tex, cdocsfn2.tex.
%
%<package>\ifdefined\childdocmain\endinput\fi
%<package>\ProvidesFile{childdoc.def}[2018/12/30 v2.0 child document driver]
%<samplemain>\ProvidesFile{cdocsamp.tex}[2018/12/30 v2.0 sample for childdoc]
%<*driver>
%\ProvidesFile{childdoc.drv}[2018/12/30 v2.0 childdoc reference manual file]
\PassOptionsToClass{10pt,a4paper}{article}
\documentclass{ltxdoc}

\usepackage[margin=35mm]{geometry}
\usepackage{hyperref}
\usepackage{hyperxmp}
\usepackage[usenames]{color}

\hypersetup{colorlinks=true}
\hypersetup{pdfstartview=FitH}
\hypersetup{pdfpagemode=UseNone}
\hypersetup{pdfsource={}}
\hypersetup{pdflang={en-UK}}
\hypersetup{pdfcopyright={Copyright 2017-2018 Niklas Beisert.
  This work may be distributed and/or modified under the
  conditions of the LaTeX Project Public License, either version 1.3
  of this license or (at your option) any later version.}}
\hypersetup{pdflicenseurl={http://www.latex-project.org/lppl.txt}}
\hypersetup{pdfcontactaddress={ETH Zurich, ITP, HIT K,
  Wolfgang-Pauli-Strasse 27}}
\hypersetup{pdfcontactpostcode={8093}}
\hypersetup{pdfcontactcity={Zurich}}
\hypersetup{pdfcontactcountry={Switzerland}}
\hypersetup{pdfcontactemail={nbeisert@itp.phys.ethz.ch}}
\hypersetup{pdfcontacturl={http://people.phys.ethz.ch/\xmptilde nbeisert/}}

\newcommand{\secref}[1]{\hyperref[#1]{section \ref*{#1}}}

\parskip1ex
\parindent0pt
\let\olditemize\itemize
\def\itemize{\olditemize\parskip0pt}

\begin{document}

\title{The \textsf{childdoc} Package}
\hypersetup{pdftitle={The childdoc Package}}
\author{Niklas Beisert\\[2ex]
  Institut f\"ur Theoretische Physik\\
  Eidgen\"ossische Technische Hochschule Z\"urich\\
  Wolfgang-Pauli-Strasse 27, 8093 Z\"urich, Switzerland\\[1ex]
  \href{mailto:nbeisert@itp.phys.ethz.ch}
  {\texttt{nbeisert@itp.phys.ethz.ch}}}
\hypersetup{pdfauthor={Niklas Beisert}}
\hypersetup{pdfsubject={Manual for the LaTeX2e Package childdoc}}
\date{30 December 2018, \textsf{v2.0}}
\maketitle

\begin{abstract}\noindent
\textsf{childdoc} is a \LaTeXe{} package
that enables the direct compilation
of document sections included by |\include|
to individual files.
\end{abstract}

\begingroup
\parskip0ex
\tableofcontents
\endgroup

%%%%%%%%%%%%%%%%%%%%%%%%%%%%%%%%%%%%%%%%%%%%%%%%%%%%%%%%%%%%%%%%%%%%%%%%%%%%%%%%
%%%%%%%%%%%%%%%%%%%%%%%%%%%%%%%%%%%%%%%%%%%%%%%%%%%%%%%%%%%%%%%%%%%%%%%%%%%%%%%%
\section{Introduction}

\LaTeX{} provides a mechanism to structure a large document (such as a book)
into a main file and several child files (containing the chapters)
using the |\include| command.
This mechanism is beneficial for documents
which span hundreds of pages in order to
make the source file(s) more manageable.
Moreover, compilation can be restricted to
selected child files by means of the |\includeonly| command.
The latter feature can be used to reduce the compilation time while editing
(this was significantly more useful in the earlier days of \LaTeX{})
or to generate a smaller document which is easier to navigate.
Another application of |\includeonly| is to generate
documents consisting of selected parts of the complete document.

However, there are a few drawbacks of the plain |\include| mechanism:
\begin{itemize}
\item
The child files cannot be compiled on their own,
they can only be compiled via the main file.
A naive editing environment
(such as a text editor with an option
to have the current file processed by \LaTeX)
may require one to switch to the main file before compiling;
attempting to compile the child file produces errors.
\item
The main file must be modified (each time)
to adjust the |\includeonly| command
to the present needs. This easily leaves the main file in a messy state.
\item
The generated document will always carry the filename
of the main document. This is inconvenient if
several child files are to be compiled and
to be kept for distribution.
\end{itemize}

The present package provides a simple interface
to make child files individually compilable by \LaTeX{}.
Compiling a child file then has the same effect as compiling
the main file with an |\includeonly| command
to select the appropriate child.
Moreover the generated document will carry the name of the child
rather than the main file.
This resolves all three above issues.

This feature is meant to make the editing of books,
thesis documents and lecture notes somewhat more convenient.
However, the package can also be used efficiently for
composing a series of documents (such as exercise sheets)
which are typically distributed individually.
It then assists the author in generating the individual documents
(potentially in different versions)
as well as a document containing the collected series.
Another application is in developing style files
or other kinds of included material
where compilation of the style file could redirect
to a sample or test file.

%%%%%%%%%%%%%%%%%%%%%%%%%%%%%%%%%%%%%%%%%%%%%%%%%%%%%%%%%%%%%%%%%%%%%%%%%%%%%%%%
%%%%%%%%%%%%%%%%%%%%%%%%%%%%%%%%%%%%%%%%%%%%%%%%%%%%%%%%%%%%%%%%%%%%%%%%%%%%%%%%
\section{Usage}

First of all, the package \textsf{childdoc} is \emph{not} a standard
\LaTeXe{} |.sty| style file! Therefore it needs to be invoked in
a non-standard way.

%%%%%%%%%%%%%%%%%%%%%%%%%%%%%%%%%%%%%%%%%%%%%%%%%%%%%%%%%%%%%%%%%%%%%%%%%%%%%%%%
\subsection{Included Files}
\label{sec:include}

%%%%%%%%%%%%%%%%%%%%%%%%%%%%%%%%%%%%%%%%
\DescribeMacro{\childdocmain}
To use the package, add the commands
\begin{center}
\begin{tabular}{l}
|\input{childdoc.def}|\\
|\childdocmain{}|\\
\end{tabular}
\end{center}
at the very top of the main \LaTeX{} file,
in particular \emph{before} the |\documentclass| statement!
The argument of |\childdocmain| should be left empty
(but it must be present).

%%%%%%%%%%%%%%%%%%%%%%%%%%%%%%%%%%%%%%%%
\DescribeMacro{\childdocof}
Furthermore, add the commands
\begin{center}
\begin{tabular}{l}
|\input{childdoc.def}|\\
|\childdocof{|\textit{main}|}|\\
\end{tabular}
\end{center}
at the top of every child file \textit{child}
which is included by |\include{|\textit{child}|}|
from within the main file
(or at least for those files to be compiled individually).
The argument \textit{main} must be the filename of the main file.

There are a couple of
considerations in setting up the main and child documents:

%%%%%%%%%%%%%%%%%%%%%%%%%%%%%%%%%%%%%%%%
\paragraph{Restrictions.}

Please note the following restrictions:
\begin{itemize}
\item
|\childdocmain| must be called with one argument \textit{main}
to ensure compatibility with earlier version of the package.
It must either be empty (|\childdocmain{}|)
or precisely match the filename of the main file in which it is specified.
See \secref{sec:detection} for further information.
\item
The filename \textit{main} must be specified without the |.tex| extension.
\item
The filename \textit{main} is case sensitive
(even in case-insensitive file systems)
due to internal string comparison.
\item
The argument \textit{main} should be fully expanded, it cannot be a macro.
\item
Subdirectories and special characters should be avoided in filenames.
\item
The command |\childdocmain{|\textit{main}|}| must be followed by a whitespace.
It should not be followed immediately by another command
or by a comment mark `|%|'.
This is because the \TeX{} parser reads the token immediately following
the argument of |\childdocmain| and puts it
at the beginning of every child section;
however, a white\-space is ignored.
\end{itemize}

%%%%%%%%%%%%%%%%%%%%%%%%%%%%%%%%%%%%%%%%
\paragraph{Content of Main File.}

It is advisable to place all content in the child files included by |\include|.
Any output contained in the main file will appear in all child documents
unless suppressed manually;
it cannot be suppressed automatically by the |\includeonly| directive
and thus should normally be avoided.
A method to include some content in the main file
by means of conditional processing is described in \secref{sec:conditional}.

%%%%%%%%%%%%%%%%%%%%%%%%%%%%%%%%%%%%%%%%
\paragraph{Page Numbering.}

When only a part of the document is compiled,
the appropriate numbering of pages
(as well as other status parameters)
is determined from the |.aux| files.
The latter contain information from previous passes.
However this information needs to propagate through
all intermediate child documents.
Therefore the page numbering in child documents may well
be inconsistent until the complete document is compiled at least once.

A useful (if unconventional) way to always ensure a consistent
page numbering is to restart the numbering in each child document
and denote the pages by `\textit{child}|.|\textit{page}'
where \textit{child} represents the chapter/section number of the child file.
This can be achieved by the command
|\numberwithin{page}{|\textit{child}|}|
of the \textsf{amsmath} package
where \textit{child} can be |chapter| or |section|
depending on the chosen structuring.
Alternatively, one can modify the macro |\thepage| appropriately
and reset the counter |page| at the start of each child file.

%%%%%%%%%%%%%%%%%%%%%%%%%%%%%%%%%%%%%%%%%%%%%%%%%%%%%%%%%%%%%%%%%%%%%%%%%%%%%%%%
\subsection{Conditional Processing}
\label{sec:conditional}

The package provides a mechanism to compile different versions
of a document. To customise the versions further some conditional processing
can come in handy to distinguish which version is being compiled.
The package provides two macros to describe the compilation context:

%%%%%%%%%%%%%%%%%%%%%%%%%%%%%%%%%%%%%%%%
\DescribeMacro{\ifchilddoc}
The conditional |\ifchilddoc| distinguishes between the compilation of
child documents and the main document:
%
\begin{center}
|\ifchilddoc |\textit{child-code}| |[|\||else |\textit{main-code}]| \||fi|
\end{center}

%%%%%%%%%%%%%%%%%%%%%%%%%%%%%%%%%%%%%%%%
\DescribeMacro{\childdocname}
\DescribeMacro{\childdocjob}
The macro |\childdocname| contains the filename (without extension)
of the main or child file being processed.
Note that |\childdocjob| will always contain the name of the main file.

%%%%%%%%%%%%%%%%%%%%%%%%%%%%%%%%%%%%%%%%
\paragraph{Title Page.}

Conditional processing can be used to include a title or banner page
in the main document when proper precautions are taken.
Importantly, the code in the main file should ensure that the page counter
(as well as other status parameters which are stored in the |.aux| files)
takes the same value after the conditional processing.
Otherwise the page numbers may take divergent values
depending on which part is compiled.

For example, a title page could be declared by:
%
\begin{center}
\begin{tabular}{l}
|\ifchilddoc\||else|\\
|\addtocounter{page}{-1}|\\
\textit{code for title page}\\
|\newpage|\\
|\||fi|
\end{tabular}
\end{center}
%
A banner page for the child documents can be generated by:
%
\begin{center}
\begin{tabular}{l}
|\ifchilddoc|\\
|\addtocounter{page}{-1}|\\
\textit{code for banner page}\\
|\newpage|\\
|\||fi|
\end{tabular}
\end{center}
%
Here one could write a message such as:
\begin{center}
|This is the part \childdocname{} of \childdocjob{}.|
\end{center}

%%%%%%%%%%%%%%%%%%%%%%%%%%%%%%%%%%%%%%%%%%%%%%%%%%%%%%%%%%%%%%%%%%%%%%%%%%%%%%%%
\subsection{Flags}
\label{sec:flags}

The package makes it easy to generate different versions
of the main or child documents.
To this end compilation flags can be defined
and assigned different default values.
They will be particularly useful in conjunction
with the forwarding mechanism described in \secref{sec:forward}.

For example, it may be useful to have a flag |\version|
which can be set to |draft| or |final|.
The document source will contain some conditional code
depending on the value of |\version|.
Suppose further, the flag should default to |final| for the main file
and to |draft| for child files
which is a natural assignment for editing the document.
This is achieved by placing the following code
in the preamble of the main document
(below the |\childdocmain| directive):
%
\begin{center}
\begin{tabular}{l}
|\ifchilddoc|\\
|\providecommand{\version}{draft}|\\
|\||else|\\
|\providecommand{\version}{final}|\\
|\||fi|
\end{tabular}
\end{center}
%
The definition by |\providecommand| makes sure
that previous definitions are not overwritten.
Further statements |\providecommand{\version}{...}|
can thus be added before the above code to override it.

For the main file, one might add a line
(between |\childdocmain| and the above block)
%
\begin{center}
|%\ifchilddoc\||else\providecommand{\version}{draft}\||fi|
\end{center}
%
which can be uncommented to produce a draft version.
Likewise one can add a line to the very top of a child file
(above the |\childdocof{|\textit{main}|}| directive)
%
\begin{center}
|%\providecommand{\version}{final}|
\end{center}
%
which can be uncommented to produce the final version of this child document.

%%%%%%%%%%%%%%%%%%%%%%%%%%%%%%%%%%%%%%%%%%%%%%%%%%%%%%%%%%%%%%%%%%%%%%%%%%%%%%%%
\subsection{Forwarding}
\label{sec:forward}

Different versions of the main or child documents
using compilation flags as described in \secref{sec:flags}
can be (permanently) stored in different files
for convenient compilation, viewing and distribution.
To this end, the package defines a command
to pass on compilation to a different file:

%%%%%%%%%%%%%%%%%%%%%%%%%%%%%%%%%%%%%%%%
\DescribeMacro{\childdocforward}
The command |\childdocforward| redirects processing to
another source file:
%
\begin{center}
\begin{tabular}{l}
|\input{childdoc.def}|\\
|\childdocforward[|\textit{main}|]{|\textit{dest}|}|\\
\end{tabular}
\end{center}
%
The argument \textit{dest} is the destination file
(without extension).
It should be the main file or one of the child files.
Note that further \textsf{childdoc} directives
such as |\childdocof| and |\childdocforward|
in the indicated file will be processed in this form.
The optional argument \textit{main}
passes on directly to the main file \textit{main}
while pretending to compile the child \textit{dest}.
This form behaves as if \textit{dest}
issues |\childdocof{|\textit{main}|}| right away,
and no further \textsf{childdoc} directives will be processed.

%%%%%%%%%%%%%%%%%%%%%%%%%%%%%%%%%%%%%%%%
\DescribeMacro{\...prefix}
In the alternative form |\childdocforwardprefix|,
%
\begin{center}
\begin{tabular}{l}
|\input{childdoc.def}|\\
|\childdocforwardprefix[|\textit{main}|]{|\textit{prefix}|}{|\textit{dest}|}|
\end{tabular}
\end{center}
%
the destination file is determined by a pattern
depending on the current file:
To make this work, the current file must be called
`{\textit{prefix}\hspace{0.2em}\textit{suffix}}'
with \textit{prefix} matching precisely the argument.
Processing is then passed on to the file
`{\textit{dest}\hspace{0.2em}\textit{suffix}}'.
Surely, the same effect is achieved by
directly specifying the
argument `{\textit{dest}\hspace{0.2em}\textit{suffix}}'
in the first form.
However, that requires to set up a different file
for each child. With the alternative form of the command
all these files can have exactly the same content
which simplifies setting them up and maintaining them.

For example, the following file |draft.tex|
with a compilation flag |\version| as described in \secref{sec:flags}
compiles the main document as a draft:
%
\begin{center}
\begin{tabular}{l}
|\def\version{draft}|\\
|\input{childdoc.def}|\\
|\childdocforward{|\textit{main}|}|
\end{tabular}
\end{center}
%
Likewise, the following files |final|\textit{nn}|.tex|
compile the final version of the child document
|child|\textit{nn}|.tex|:
%
\begin{center}
\begin{tabular}{l}
|\def\version{final}|\\
|\input{childdoc.def}|\\
|\childdocforwardprefix{final}{child}|
\end{tabular}
\end{center}
%

Note that when several versions of a main file and/or of each child file
are to be generated, it may be convenient to set up a |Makefile| or
shell script to automatise the process.

%%%%%%%%%%%%%%%%%%%%%%%%%%%%%%%%%%%%%%%%%%%%%%%%%%%%%%%%%%%%%%%%%%%%%%%%%%%%%%%%
\subsection{Command Line Processing}
\label{sec:commandline}

The effect of redirection files can also be achieved by invoking
the \LaTeX{} compiler with a more elaborate command line.
Most conveniently this should be done as part
of a shell script or a |Makefile|.

When using \textsf{childdoc} in the main file, the following
command lines effectively perform a redirection
(note that depending on the shell being used,
backslashes may have to be doubled: `|\|' $\to$ `|\\|'):
%
\begin{center}
|... -jobname "|\textit{target}|" |\\|"|[\textit{flags}]%
|\input{childdoc.def}\childdocforward[|\textit{main}|]{|\textit{dest}|}"|
\end{center}
%
Here \textit{target} is the name of the output file,
\textit{main} is the name of the main file
and \textit{dest} is the name of the main or child file to be processed
(all filenames without extensions).
The optional argument \textit{main} can be omitted
if \textit{main} matches \textit{dest}.
Optionally, compilation \textit{flags} can be defined via |\def| commands.
This command line makes the \TeX{} engine believe
it is compiling the file \textit{target}
whose content is specified as the latter parameter.
The provided code then forwards the processing to
\textit{main} or \textit{dest} as described in \secref{sec:forward}.

%%%%%%%%%%%%%%%%%%%%%%%%%%%%%%%%%%%%%%%%%%%%%%%%%%%%%%%%%%%%%%%%%%%%%%%%%%%%%%%%
\subsection{Include by Input}
\label{sec:input}

Including child documents by |\include| has some restrictions by design.
Most notably, the content of a child document always occupies
its own set of pages; pages cannot be shared between child documents.
Usually, this behaviour makes perfect sense
because each child document contain an essential part of the document.
However, in some situations it may be desirable to compose
a document from a collection of parts
without having mandatory page breaks between then.
For this case, the package
provides a mechanism to include parts
by |\input| which can also be processed individually.
However, by construction this mechanism
requires manual handling of the content to be output.

%%%%%%%%%%%%%%%%%%%%%%%%%%%%%%%%%%%%%%%%
\DescribeMacro{\ifchilddocmanual}
The main file should be prepared as usual, see \secref{sec:include}.
However, the document body must make a distinction
between processing of an individual part and of the main document, e.g.:
%
\begin{center}
\begin{tabular}{l}
|\ifchilddocmanual|\\
|\input{\childdocname}|\\
|\||else|\\
\textit{document body with }|\input{|\textit{part}|}|\\
|\||fi|
\end{tabular}
\end{center}
%
The conditional |\ifchilddocmanual| is true whenever
a part to be included by |\input| is being compiled,
and the name of the part is stored in |\childdocname|.

%%%%%%%%%%%%%%%%%%%%%%%%%%%%%%%%%%%%%%%%
\DescribeMacro{\childdocby}
Each part to be included by |\input| should start with:
%
\begin{center}
\begin{tabular}{l}
|\input{childdoc.def}|\\
|\childdocby{|\textit{main}|}|\\
\end{tabular}
\end{center}
%
The directive |\childdocby| is similar to |\childdocof|
described in \secref{sec:include},
but the subsequent selection of content must be done manually.
To that end, both |\ifchilddoc| and |\ifchilddocmanual|
will be true upon processing of a part,
and the name of the part is stored in |\childdocname|.
Note that |\jobname| will be set to the filename of the current part
so that each part receives an individual |.aux| file
that does not interfere with the |.aux| file(s) of the main document.
This behaviour can be altered by the alternative form
|\childdocby[*]{|\textit{main}|}| (with a non-empty optional argument)
which uses the |.aux| file of the main document
by setting |\jobname| to \textit{main}.

%%%%%%%%%%%%%%%%%%%%%%%%%%%%%%%%%%%%%%%%%%%%%%%%%%%%%%%%%%%%%%%%%%%%%%%%%%%%%%%%
\subsection{Driver Development}
\label{sec:driver}

The \textsf{childdoc} mechanism can also be use for the development
of definition files such as \LaTeX{} styles or classes.
This case differs from the above setup with multiple parts
included by |\include| in that no |\includeonly| should be invoked.
This can be achieved by starting the include file
(before |\ProvidesPackage|) with:
%
\begin{center}
\begin{tabular}{l}
|\input{childdoc.def}|\\
|\childdocforward{|\textit{main}|}|\\
\end{tabular}
\end{center}
%
or alternatively with:
%
\begin{center}
\begin{tabular}{l}
|\input{childdoc.def}|\\
|\childdocby{|\textit{main}|}|\\
\end{tabular}
\end{center}
%
Both forms have slightly different effects as described above.
The main file is prepared as usual, see \secref{sec:include}.

%%%%%%%%%%%%%%%%%%%%%%%%%%%%%%%%%%%%%%%%%%%%%%%%%%%%%%%%%%%%%%%%%%%%%%%%%%%%%%%%
\subsection{Legacy Detection}
\label{sec:detection}

The directive |\childdocmain| in the main file can detect
whether the complete document or merely a child is to be compiled
even without using the directive |\childdocof|.
This method is deprecated because it is less robust
and there is no compelling reason to use it;
it is merely provided for backward compatibility
and it may be removed in future versions.

If the detection mechanism is to be used,
it is mandatory to correctly specify
the filename of the main file as the argument of |\childdocmain|:
%
\begin{center}
\begin{tabular}{l}
|\input{childdoc.def}|\\
|\childdocmain{|\textit{main}|}|\\
\end{tabular}
\end{center}
%
If |\jobname| does not match the argument \textit{main} of |\childdocmain|,
it is assumed that |\jobname| points to the child file to be compiled.
When using |\childdocmain| with the main file specified as argument,
it suffices to start a child file
with just |\input{|\textit{main}|}|
without loading of the package and using |\childdocof|.
If instead all processing is done
with the appropriate \textsf{childdoc} directives,
the argument of \textit{main} of |\childdocmain| can be empty.

An alternative version of the command line processing described
in \secref{sec:commandline} using the detection mechanism reads:
%
\begin{center}
|... -jobname "|\textit{target}|" "|[\textit{flags}]%
[|\def\jobname{|\textit{dest}|}|]|\input{|\textit{main}|}"|
\end{center}

%%%%%%%%%%%%%%%%%%%%%%%%%%%%%%%%%%%%%%%%%%%%%%%%%%%%%%%%%%%%%%%%%%%%%%%%%%%%%%%%
\subsection{Manual Code}
\label{sec:manual}

In case one cannot be certain whether the definitions file |childdoc.def|
is installed on the target \TeX{} distribution
and one prefers not to ship it,
it is conceivable to paste a few relevant commands into the sources.

To that end, drop all statements |\input{childdoc.def}|
and perform the replacements as outlined below.
Instead of |\childdocmain{|\textit{main}|}| add the following code
to the top of the main file:
%
\begin{center}
\begin{tabular}{l}
|\||ifdefined\childdocname\endinput\||fi\newif\ifchilddoc|\\
|\edef\childdocname{\scantokens\expandafter{\jobname\noexpand}}|\\
|\def\childdocmain{|\textit{main}|}\||ifx\childdocmain\childdocname\||else|\\
|\childdoctrue\includeonly{\childdocname}\let\jobname\childdocmain\||fi|\\
\end{tabular}
\end{center}
%
Instead of |\childdocof{|\textit{main}|}| just include the main file
at the top of each child file:
%
\begin{center}
|\input{|\textit{main}|}|
\end{center}
%
A simple redirection |\childdocforward{|\textit{dest}|}| is achieved by:
%
\begin{center}
|\def\jobname{|\textit{dest}|}\input{\jobname}|
\end{center}
%
The redirection with prefix
|\childdocforwardprefix[|\textit{prefix}|]{|\textit{dest}|}|
is accomplished by:
%
\begin{center}
\begin{tabular}{l}
|{\edef\jobname{\scantokens\expandafter{\jobname\noexpand}}|\\
|\def\redirectjob |\textit{prefix}|#1~~~{\gdef\jobname{|\textit{dest}|#1}}|\\
|\expandafter\redirectjob\jobname~~~}\input{\jobname}|
\end{tabular}
\end{center}

In an alternative approach,
child documents can be compiled by a specific command line
without additional code or specific definitions:
%
\begin{center}
|... -jobname "|\textit{target}|" "|[\textit{flags}]%
|\includeonly{|\textit{dest}|}\input{|\textit{main}|}"|
\end{center}
%

%%%%%%%%%%%%%%%%%%%%%%%%%%%%%%%%%%%%%%%%%%%%%%%%%%%%%%%%%%%%%%%%%%%%%%%%%%%%%%%%
%%%%%%%%%%%%%%%%%%%%%%%%%%%%%%%%%%%%%%%%%%%%%%%%%%%%%%%%%%%%%%%%%%%%%%%%%%%%%%%%
\section{Information}

%%%%%%%%%%%%%%%%%%%%%%%%%%%%%%%%%%%%%%%%%%%%%%%%%%%%%%%%%%%%%%%%%%%%%%%%%%%%%%%%
\subsection{Copyright}

Copyright \copyright{} 2017--2018 Niklas Beisert

This work may be distributed and/or modified under the
conditions of the \LaTeX{} Project Public License, either version 1.3
of this license or (at your option) any later version.
The latest version of this license is in
  \url{http://www.latex-project.org/lppl.txt}
and version 1.3 or later is part of all distributions of \LaTeX{}
version 2005/12/01 or later.

This work has the LPPL maintenance status `maintained'.

The Current Maintainer of this work is Niklas Beisert.

This work consists of the files |README.txt|, |childdoc.ins| and |childdoc.dtx|
as well as the derived files |childdoc.def|, |cdocsamp.tex|
with |cdocsch1.tex|, |cdocsch2.tex|, |cdocspt3.tex|, |cdocspt4.tex|,
|cdocsdrf.tex|, |cdocsfn1.tex|, |cdocsfn2.tex|
as well as |childdoc.pdf|.

%%%%%%%%%%%%%%%%%%%%%%%%%%%%%%%%%%%%%%%%%%%%%%%%%%%%%%%%%%%%%%%%%%%%%%%%%%%%%%%%
\subsection{Files and Installation}

The package consists of the files:
%
\begin{center}
\begin{tabular}{ll}
    |README.txt|   & readme file \\
    |childdoc.ins| & installation file \\
    |childdoc.dtx| & source file \\
    |childdoc.def| & definition file \\
    |cdocsamp.tex| & sample main file \\
    |cdocsch1.tex| & sample include file \\
    |cdocsch2.tex| & sample include file \\
    |cdocspt3.tex| & sample part file \\
    |cdocspt4.tex| & sample part file \\
    |cdocsdrf.tex| & sample redirection file \\
    |cdocsfn1.tex| & sample redirection file \\
    |cdocsfn2.tex| & sample redirection file \\
    |childdoc.pdf| & manual
\end{tabular}
\end{center}
%
The distribution consists of the files
|README.txt|, |childdoc.ins| and |childdoc.dtx|.
%
\begin{itemize}
\item
Run (pdf)\LaTeX{} on |childdoc.dtx|
to compile the manual |childdoc.pdf| (this file).
\item
Run \LaTeX{} on |childdoc.ins| to create the definitions file |childdoc.def|
and the sample |cdocsamp.tex| with include files
|cdocsch1.tex|, |cdocsch2.tex|, |cdocspt3.tex|, |cdocspt4.tex|,
|cdocsdrf.tex|, |cdocsfn1.tex|, |cdocsfn2.tex|.
Then copy the file |childdoc.def| to an appropriate directory of your \LaTeX{}
distribution, e.g.\ \textit{texmf-root}|/tex/latex/childdoc|.
\end{itemize}

%%%%%%%%%%%%%%%%%%%%%%%%%%%%%%%%%%%%%%%%%%%%%%%%%%%%%%%%%%%%%%%%%%%%%%%%%%%%%%%%
\subsection{Related CTAN Packages}

There are several other packages which offer a similar functionality:
%
\begin{itemize}
\item
The packages
\href{http://ctan.org/pkg/docmute}{\textsf{docmute}},
\href{http://ctan.org/pkg/includex}{\textsf{includex}} and
\href{http://ctan.org/pkg/standalone}{\textsf{standalone}}
provide commands to include only the document body of
a child file thus allowing both files to be compiled individually.
\item
The packages \href{http://ctan.org/pkg/subdocs}{\textsf{subdocs}}
and \href{http://ctan.org/pkg/subfiles}{\textsf{subfiles}}
provide structures in which the main and child documents can be
encapsulated and allowing them to be compiled individually.
The inclusion mechanism is different from the conventional |\include|.
\item
The package \href{http://ctan.org/pkg/combine}{\textsf{combine}}
is an elaborate solution to combine several documents into one.
\end{itemize}
%
See also the CTAN topic \href{http://ctan.org/topic/subdocs}{\textsf{subdocs}}
for further related packages.
The present package differs from the above solutions in that
a document structure constructed with the conventional |\include| mechanism
just needs two extra commands at the top of every file
such that all constituent files can be compiled individually.

%%%%%%%%%%%%%%%%%%%%%%%%%%%%%%%%%%%%%%%%%%%%%%%%%%%%%%%%%%%%%%%%%%%%%%%%%%%%%%%%
%\subsection{Feature Suggestions}
%
%The following is a list of features which may be useful for future
%versions of this package:
%%
%\begin{itemize}
%\item
%\ldots
%\end{itemize}

%%%%%%%%%%%%%%%%%%%%%%%%%%%%%%%%%%%%%%%%%%%%%%%%%%%%%%%%%%%%%%%%%%%%%%%%%%%%%%%%
\subsection{Revision History}

%%%%%%%%%%%%%%%%%%%%%%%%%%%%%%%%%%%%%%%%
\paragraph{v2.0:} 2018/12/30

\begin{itemize}
\item
immediate forward processing
\item
added |\childdocby| mechanism
\item
manual restructured
\end{itemize}

%%%%%%%%%%%%%%%%%%%%%%%%%%%%%%%%%%%%%%%%
\paragraph{v1.6:} 2018/01/17

\begin{itemize}
\item
application for development of include files
\item
corrections to manual
\end{itemize}

%%%%%%%%%%%%%%%%%%%%%%%%%%%%%%%%%%%%%%%%
\paragraph{v1.5:} 2017/05/21

\begin{itemize}
\item
more complete structuring introduced
\item
|\childdocof| introduced
\item
|\childdoc| renamed to |\childdocmain|
\item
|\childredirect| renamed to |\childdocforward| and |\childdocforwardprefix|
and functionality expanded
\end{itemize}

%%%%%%%%%%%%%%%%%%%%%%%%%%%%%%%%%%%%%%%%
\paragraph{v1.0:} 2017/04/27

\begin{itemize}
\item
manual and install package
\item
first version published on CTAN
\end{itemize}

%%%%%%%%%%%%%%%%%%%%%%%%%%%%%%%%%%%%%%%%
\paragraph{v0.6:} 2017/04/26

\begin{itemize}
\item
redirection mechanism added
\end{itemize}

%%%%%%%%%%%%%%%%%%%%%%%%%%%%%%%%%%%%%%%%
\paragraph{v0.5:} 2017/04/26

\begin{itemize}
\item
functionality in definition file
\end{itemize}


%%%%%%%%%%%%%%%%%%%%%%%%%%%%%%%%%%%%%%%%%%%%%%%%%%%%%%%%%%%%%%%%%%%%%%%%%%%%%%%%
%%%%%%%%%%%%%%%%%%%%%%%%%%%%%%%%%%%%%%%%%%%%%%%%%%%%%%%%%%%%%%%%%%%%%%%%%%%%%%%%
%%%%%%%%%%%%%%%%%%%%%%%%%%%%%%%%%%%%%%%%%%%%%%%%%%%%%%%%%%%%%%%%%%%%%%%%%%%%%%%%
\appendix

\settowidth\MacroIndent{\rmfamily\scriptsize 000\ }

 \DocInput{childdoc.dtx}

\end{document}
%</driver>
% \fi
%
% %%%%%%%%%%%%%%%%%%%%%%%%%%%%%%%%%%%%%%%%%%%%%%%%%%%%%%%%%%%%%%%%%%%%%%%%%%%%%%
% %%%%%%%%%%%%%%%%%%%%%%%%%%%%%%%%%%%%%%%%%%%%%%%%%%%%%%%%%%%%%%%%%%%%%%%%%%%%%%
% \section{Sample}
%\iffalse
%<*samplemain>
%\fi
%
% The following presents a sample document
% with two chapters, two parts, a title page,
% a compile flag as well as three forwarding files to set the flag.
% It consists of eight |.tex| files:
% \begin{center}
% \begin{tabular}{ll}
% |cdocsamp.tex|&main file\\
% |cdocsch1.tex|&include file for chapter 1\\
% |cdocsch2.tex|&include file for chapter 2\\
% |cdocspt3.tex|&include file for part 3\\
% |cdocspt4.tex|&include file for part 4\\
% |cdocsdrf.tex|&forwarding file for main file in draft mode\\
% |cdocsfi1.tex|&forwarding file for final version of chapter 1\\
% |cdocsfi2.tex|&forwarding file for final version of chapter 2\\
% \end{tabular}
% \end{center}
% Each of the eight files can be compiled directly by the \LaTeX{} compiler.
%
% %%%%%%%%%%%%%%%%%%%%%%%%%%%%%%%%%%%%%%
% \paragraph{Main File.}
%
% The main file is called |cdocsamp.tex|.
%
% Load the \textsf{childdoc} definitions and
% declare the filename for the main document:
%    \begin{macrocode}
\input{childdoc.def}
\childdocmain{}
%    \end{macrocode}

% Optional override for |\version| flag:
%    \begin{macrocode}
%%\ifchilddoc\else\providecommand{\version}{draft}\fi
%    \end{macrocode}

% Define the default values for the |\version| flag
% (|final| for the main file and |draft| for childs):
%    \begin{macrocode}
\ifchilddoc
\providecommand{\version}{draft}
\else
\providecommand{\version}{final}
\fi
%    \end{macrocode}

% Load the standard document class:
%    \begin{macrocode}
\documentclass[12pt]{article}
%    \end{macrocode}

% Start the document body:
%    \begin{macrocode}
\begin{document}
%    \end{macrocode}

% Declare a title page.
% Print title, part of document being processed and version flag:
%    \begin{macrocode}
\addtocounter{page}{-1}
\begin{center}
{\LARGE\bfseries{}childdoc example\par}
\vspace{1cm}
\ifchilddoc
\ifchilddocmanual part\else chapter\fi:
`\childdocname' of `\childdocjob'\par
\else
main document: `\childdocjob'\par
\fi
version: \version\par
\end{center}
\newpage
%    \end{macrocode}

% Manually include selected file,
% otherwise process as usual:
%    \begin{macrocode}
\ifchilddocmanual
\section*{part `\childdocname'}
\input{\childdocname}
\else
%    \end{macrocode}

% Include the two chapters:
%    \begin{macrocode}
\include{cdocsch1}
\include{cdocsch2}
%    \end{macrocode}

% Include the two parts unless only chapters should be displayed:
%    \begin{macrocode}
\ifchilddoc\else
\section{part three}
\input{cdocspt3}
\section{part four}
\input{cdocspt4}
\fi
%    \end{macrocode}

% Process as usual until here:
%    \begin{macrocode}
\fi
%    \end{macrocode}

% End of document body:
%    \begin{macrocode}
\end{document}
%    \end{macrocode}
%\iffalse
%</samplemain>
%\fi
%
% %%%%%%%%%%%%%%%%%%%%%%%%%%%%%%%%%%%%%%
% \paragraph{Chapter Include Files.}
%
% The include files are called |cdocsch1.tex| and |cdocsch2.tex|.
%
%\iffalse
%<*samplechap1|samplechap2>
%\fi

% Optional override for |\version| flag:
%    \begin{macrocode}
%%\providecommand{\version}{final}
%    \end{macrocode}

% Include the main document:
%    \begin{macrocode}
\input{childdoc.def}
\childdocof{cdocsamp}
%    \end{macrocode}

%\iffalse
%</samplechap1|samplechap2>
%\fi
%
%\iffalse
%<*samplechap1>
%\fi
% Some text for chapter 1:
%    \begin{macrocode}
\section{one}
some text in chapter one
%    \end{macrocode}

%\iffalse
%</samplechap1>
%\fi
% Some text for chapter 2:
%\iffalse
%<*samplechap2>
%\fi
%    \begin{macrocode}
\section{two}
more text in chapter two
%    \end{macrocode}

%\iffalse
%</samplechap2>
%\fi
%
% %%%%%%%%%%%%%%%%%%%%%%%%%%%%%%%%%%%%%%
% \paragraph{Part Include Files.}
%
% The include files are called |cdocspt3.tex| and |cdocspt4.tex|.
%
%\iffalse
%<*samplepart3|samplepart4>
%\fi

% Optional override for |\version| flag:
%    \begin{macrocode}
%%\providecommand{\version}{final}
%    \end{macrocode}

% Include the main document:
%    \begin{macrocode}
\input{childdoc.def}
\childdocby{cdocsamp}
%    \end{macrocode}

%\iffalse
%</samplepart3|samplepart4>
%\fi
%
%\iffalse
%<*samplepart3>
%\fi
% Some text for part 3:
%    \begin{macrocode}
some text in part three
%    \end{macrocode}

%\iffalse
%</samplepart3>
%\fi
% Some text for part 4:
%\iffalse
%<*samplepart4>
%\fi
%    \begin{macrocode}
more text in part four
%    \end{macrocode}

%\iffalse
%</samplepart4>
%\fi
%
% %%%%%%%%%%%%%%%%%%%%%%%%%%%%%%%%%%%%%%
% \paragraph{Forwarding for a Complete Draft.}
%
% The following forwarding file |cdocsdrf.tex|
% compiles the main document in draft mode:
%\iffalse
%<*sampledraft>
%\fi
%    \begin{macrocode}
\def\version{draft}
\input{childdoc.def}
\childdocforward{cdocsamp}
%    \end{macrocode}

%\iffalse
%</sampledraft>
%\fi
%
% %%%%%%%%%%%%%%%%%%%%%%%%%%%%%%%%%%%%%%
% \paragraph{Forwarding for Final Version of the Chapters.}
%
% The following forwarding files |cdocsfn1.tex| and |cdocsfn2.tex|
% (with identical content)
% compile the final versions of the child documents
% |cdocsch1.tex| and |cdocsch2.tex|, respectively:
%\iffalse
%<*samplefinal>
%\fi
%    \begin{macrocode}
\def\version{final}
\input{childdoc.def}
\childdocforwardprefix[cdocsamp]{cdocsfn}{cdocsch}
%    \end{macrocode}

%\iffalse
%</samplefinal>
%\fi
%
% %%%%%%%%%%%%%%%%%%%%%%%%%%%%%%%%%%%%%%
% \paragraph{Command Line Processing.}
%
% The following three command lines generate the output files
% |cdocscld|, |cdocscl1| and |cdocscl2|
% which should be identical to
% |cdocsdrf|, |cdocsch1| and |cdocsfn2|, respectively:
% \begin{center}
% \begin{tabular}{l}
% |latex -jobname cdocscld \|\\
% |  "\def\version{draft}\input{childdoc.def}\childdocforward{cdocsamp}"|\\
% |latex -jobname cdocscl1 \|\\
% |  "\input{childdoc.def}\childdocforward[cdocsamp]{cdocsch1}"|\\
% |latex -jobname cdocscl2 \|\\
% |  "\def\version{final}\input{childdoc.def}\childdocforward{cdocsch2}"|
% \end{tabular}
% \end{center}
% Note that the trailing backslash on each first line
% merely continues the input to the second line
% (for convenient cut ant paste).
% Furthermore, the command |latex| can be replaced by any
% of its alternative versions such as |pdflatex|.
%
% %%%%%%%%%%%%%%%%%%%%%%%%%%%%%%%%%%%%%%%%%%%%%%%%%%%%%%%%%%%%%%%%%%%%%%%%%%%%%%
% %%%%%%%%%%%%%%%%%%%%%%%%%%%%%%%%%%%%%%%%%%%%%%%%%%%%%%%%%%%%%%%%%%%%%%%%%%%%%%
% \section{Implementation}
%\iffalse
%<*package>
%\fi
%
% This section describes the definitions file |childdoc.def|.

% The definitions cannot be loaded using |\usepackage| or |\RequirePackage|
% which has a mechanism to prevent loading a style file more than once.
% When loading the definitions by means of |\input|
% multiple instances have to be prevented manually:
%\iffalse
%This code needs to be before the `\ProvidesFile' directive
%which is defined at the beginning of this file.
%Therefore it is also placed there and commented out here.
%</package>
%<*discard>
%\fi
%    \begin{macrocode}
\ifdefined\childdocmain\endinput\fi
%    \end{macrocode}
%\iffalse
%</discard>
%<*package>
%\fi
%
% \macro{\ifchilddoc}
% \macro{\ifchilddocmanual}
% The conditional |\ifchilddoc| tells whether a
% child (true) or main (false) document is being compiled.
% The conditional |\ifchilddocmanual| tells whether
% the |\includeonly| mechanism is used (false) or
% the selection of child files must be performed manually (true).
% The definitions initialise to false:
%    \begin{macrocode}
\newif\ifchilddoc
\newif\ifchilddocmanual
%    \end{macrocode}

% \macro{\childdocname}
% \macro{\childdocjob}
% The macro |\childdocname| stores the name of the main document
% to be compiled. The macro |\childdocjob| stores the name of
% the document on which the \LaTeX{} compiler was originally invoked.
% The content of |\jobname| cannot be compared
% to filenames specified in the source due to different catcodes.
% The following code rescans |\jobname|, stores the result
% in |\childdocname| and saves a copy in |\childdocjob|:
%    \begin{macrocode}
\edef\childdocname{\scantokens\expandafter{\jobname\noexpand}}
\let\childdocjob\childdocname
%    \end{macrocode}

% \macro{\childdocdisable}
% The macro |\childdocdisable| prevents the main file
% from being processed more than once.
% At this stage, the main document command |\childdocmain|
% is assumed to be called once again where it should do nothing.
% Any subsequent call to it should prevent
% a secondary processing of the main document
% It overwrites the forwarding commands
% |\childdocof| and |\childdocforward|
% with empty macros to prevent further inclusions of the main document:
%    \begin{macrocode}
\newcommand{\childdocdisable}
{
  \renewcommand{\childdocmain}[1]{\renewcommand{\childdocmain}[1]{\endinput}}
  \renewcommand{\childdocof}[1]{}
  \renewcommand{\childdocby}[2][]{}
  \renewcommand{\childdocforward}[2][]{}
  \renewcommand{\childdocdisable}{}
}
%    \end{macrocode}

% \macro{\childdocmain}
% The macro |\childdocmain| is to be called at the top of the main file
% with nothing or the main filename (without extension) as argument.
% First, it breaks loops.
% If the argument is not empty and does not match |\childdocname|
% (which is set by the first inclusion of |childdoc.def|),
% |\ifchilddoc| is set to true, |\includeonly| is applied to the child file
% and |\jobname| is set to the main file
% (for proper handling of |.aux| files):
%    \begin{macrocode}
\newcommand{\childdocmain}[1]
{
  \childdocdisable\childdocmain{}
  \if?#1?\else
    \begingroup
      \def\childdoctmp{#1}
      \ifx\childdoctmp\childdocname
        \def\childdoctmp{}
      \else
        \def\childdoctmp
        {
          \childdoctrue
          \includeonly{\childdocname}
          \def\childdocjob{#1}
          \def\jobname{#1}
        }
      \fi
      \expandafter
    \endgroup
    \childdoctmp
  \fi
}
%    \end{macrocode}

% \macro{\childdocof}
% The command |\childdocof| redirects
% compilation to the main file |#1|.
%    \begin{macrocode}
\newcommand{\childdocof}[1]
{
  \childdocdisable
  \childdoctrue
  \includeonly{\childdocname}
  \def\jobname{#1}
  \def\childdocjob{#1}
  \input{#1}
}
%    \end{macrocode}

% \macro{\childdocby}
% The command |\childdocby| ....
%    \begin{macrocode}
\newcommand{\childdocby}[2][]
{
  \childdocdisable
  \childdoctrue
  \childdocmanualtrue
  \if?#1?\else
    \def\jobname{#2}
  \fi
  \def\childdocjob{#2}
  \input{#2}
  \endinput
}
%    \end{macrocode}

% \macro{\childdocforward}
% The command |\childdocforward| redirects
% compilation to the main file or
% (if the optional argument is given) a child file.
% Parameters are set as if the main file
% or a child file starting with |\childdocof| was compiled.
% Then compilation is handed over to the main file:
%    \begin{macrocode}
\newcommand{\childdocforward}[2][]
{
  \begingroup
    \if?#1?
      \def\childdoctmp
      {
        \def\childdocname{#2}
        \def\childdocjob{#2}
        \def\jobname{#2}
        \input{#2}
        \endinput
      }
    \else
      \def\childdoctmp
      {
        \childdocdisable
        \def\childdocname{#2}
        \childdoctrue
        \includeonly{#2}
        \def\childdocjob{#1}
        \def\jobname{#1}
        \input{#1}
        \endinput
      }
    \fi
    \expandafter
  \endgroup
  \childdoctmp
}
%    \end{macrocode}

% \macro{\childdocforwardprefix}
% The command |\childdocforwardprefix| redirects
% compilation to the main or a child file by means of a pattern.
% The prefix |#1| in the current filename is replaced by |#2|
% and the suffix of the current filename is kept
% (it is assumed that the filename does not contain the substring `|~~~|'
% which is used as a delimiter).
% Compilation is handed over to the new file by |\childdocforward|:
%    \begin{macrocode}
\newcommand{\childdocforwardprefix}[3][]
{
  \begingroup
    \def\childdocextract #2##1~~~{\def\childdoctmp{\childdocforward[#1]{#3##1}}}
    \expandafter\childdocextract\childdocname~~~
    \expandafter
  \endgroup
  \childdoctmp
}
%    \end{macrocode}

% \macro{\childdoc}
% The deprecated macro |\childdoc| is a legacy version of |\childdocmain|:
%    \begin{macrocode}
\newcommand{\childdoc}{\childdocmain}
%    \end{macrocode}

% \macro{\childdocredirect}
% The deprecated macro |\childdocredirect| is a legacy version
% of |\childdocforward| and |\childdocforwardprefix|:
%    \begin{macrocode}
\newcommand{\childdocredirect}[2][]
{
  \begingroup
    \if?#1?
      \def\childdoctmp{\childdocforward{#2}}
    \else
      \def\childdoctmp{\childdocforwardprefix{#1}{#2}}
    \fi
    \expandafter
  \endgroup
  \childdoctmp
}
%    \end{macrocode}

%\iffalse
%</package>
%\fi
%
\endinput
|\\
|\childdocof{|\textit{main}|}|\\
\end{tabular}
\end{center}
at the top of every child file \textit{child}
which is included by |\include{|\textit{child}|}|
from within the main file
(or at least for those files to be compiled individually).
The argument \textit{main} must be the filename of the main file.

There are a couple of
considerations in setting up the main and child documents:

%%%%%%%%%%%%%%%%%%%%%%%%%%%%%%%%%%%%%%%%
\paragraph{Restrictions.}

Please note the following restrictions:
\begin{itemize}
\item
|\childdocmain| must be called with one argument \textit{main}
to ensure compatibility with earlier version of the package.
It must either be empty (|\childdocmain{}|)
or precisely match the filename of the main file in which it is specified.
See \secref{sec:detection} for further information.
\item
The filename \textit{main} must be specified without the |.tex| extension.
\item
The filename \textit{main} is case sensitive
(even in case-insensitive file systems)
due to internal string comparison.
\item
The argument \textit{main} should be fully expanded, it cannot be a macro.
\item
Subdirectories and special characters should be avoided in filenames.
\item
The command |\childdocmain{|\textit{main}|}| must be followed by a whitespace.
It should not be followed immediately by another command
or by a comment mark `|%|'.
This is because the \TeX{} parser reads the token immediately following
the argument of |\childdocmain| and puts it
at the beginning of every child section;
however, a white\-space is ignored.
\end{itemize}

%%%%%%%%%%%%%%%%%%%%%%%%%%%%%%%%%%%%%%%%
\paragraph{Content of Main File.}

It is advisable to place all content in the child files included by |\include|.
Any output contained in the main file will appear in all child documents
unless suppressed manually;
it cannot be suppressed automatically by the |\includeonly| directive
and thus should normally be avoided.
A method to include some content in the main file
by means of conditional processing is described in \secref{sec:conditional}.

%%%%%%%%%%%%%%%%%%%%%%%%%%%%%%%%%%%%%%%%
\paragraph{Page Numbering.}

When only a part of the document is compiled,
the appropriate numbering of pages
(as well as other status parameters)
is determined from the |.aux| files.
The latter contain information from previous passes.
However this information needs to propagate through
all intermediate child documents.
Therefore the page numbering in child documents may well
be inconsistent until the complete document is compiled at least once.

A useful (if unconventional) way to always ensure a consistent
page numbering is to restart the numbering in each child document
and denote the pages by `\textit{child}|.|\textit{page}'
where \textit{child} represents the chapter/section number of the child file.
This can be achieved by the command
|\numberwithin{page}{|\textit{child}|}|
of the \textsf{amsmath} package
where \textit{child} can be |chapter| or |section|
depending on the chosen structuring.
Alternatively, one can modify the macro |\thepage| appropriately
and reset the counter |page| at the start of each child file.

%%%%%%%%%%%%%%%%%%%%%%%%%%%%%%%%%%%%%%%%%%%%%%%%%%%%%%%%%%%%%%%%%%%%%%%%%%%%%%%%
\subsection{Conditional Processing}
\label{sec:conditional}

The package provides a mechanism to compile different versions
of a document. To customise the versions further some conditional processing
can come in handy to distinguish which version is being compiled.
The package provides two macros to describe the compilation context:

%%%%%%%%%%%%%%%%%%%%%%%%%%%%%%%%%%%%%%%%
\DescribeMacro{\ifchilddoc}
The conditional |\ifchilddoc| distinguishes between the compilation of
child documents and the main document:
%
\begin{center}
|\ifchilddoc |\textit{child-code}| |[|\||else |\textit{main-code}]| \||fi|
\end{center}

%%%%%%%%%%%%%%%%%%%%%%%%%%%%%%%%%%%%%%%%
\DescribeMacro{\childdocname}
\DescribeMacro{\childdocjob}
The macro |\childdocname| contains the filename (without extension)
of the main or child file being processed.
Note that |\childdocjob| will always contain the name of the main file.

%%%%%%%%%%%%%%%%%%%%%%%%%%%%%%%%%%%%%%%%
\paragraph{Title Page.}

Conditional processing can be used to include a title or banner page
in the main document when proper precautions are taken.
Importantly, the code in the main file should ensure that the page counter
(as well as other status parameters which are stored in the |.aux| files)
takes the same value after the conditional processing.
Otherwise the page numbers may take divergent values
depending on which part is compiled.

For example, a title page could be declared by:
%
\begin{center}
\begin{tabular}{l}
|\ifchilddoc\||else|\\
|\addtocounter{page}{-1}|\\
\textit{code for title page}\\
|\newpage|\\
|\||fi|
\end{tabular}
\end{center}
%
A banner page for the child documents can be generated by:
%
\begin{center}
\begin{tabular}{l}
|\ifchilddoc|\\
|\addtocounter{page}{-1}|\\
\textit{code for banner page}\\
|\newpage|\\
|\||fi|
\end{tabular}
\end{center}
%
Here one could write a message such as:
\begin{center}
|This is the part \childdocname{} of \childdocjob{}.|
\end{center}

%%%%%%%%%%%%%%%%%%%%%%%%%%%%%%%%%%%%%%%%%%%%%%%%%%%%%%%%%%%%%%%%%%%%%%%%%%%%%%%%
\subsection{Flags}
\label{sec:flags}

The package makes it easy to generate different versions
of the main or child documents.
To this end compilation flags can be defined
and assigned different default values.
They will be particularly useful in conjunction
with the forwarding mechanism described in \secref{sec:forward}.

For example, it may be useful to have a flag |\version|
which can be set to |draft| or |final|.
The document source will contain some conditional code
depending on the value of |\version|.
Suppose further, the flag should default to |final| for the main file
and to |draft| for child files
which is a natural assignment for editing the document.
This is achieved by placing the following code
in the preamble of the main document
(below the |\childdocmain| directive):
%
\begin{center}
\begin{tabular}{l}
|\ifchilddoc|\\
|\providecommand{\version}{draft}|\\
|\||else|\\
|\providecommand{\version}{final}|\\
|\||fi|
\end{tabular}
\end{center}
%
The definition by |\providecommand| makes sure
that previous definitions are not overwritten.
Further statements |\providecommand{\version}{...}|
can thus be added before the above code to override it.

For the main file, one might add a line
(between |\childdocmain| and the above block)
%
\begin{center}
|%\ifchilddoc\||else\providecommand{\version}{draft}\||fi|
\end{center}
%
which can be uncommented to produce a draft version.
Likewise one can add a line to the very top of a child file
(above the |\childdocof{|\textit{main}|}| directive)
%
\begin{center}
|%\providecommand{\version}{final}|
\end{center}
%
which can be uncommented to produce the final version of this child document.

%%%%%%%%%%%%%%%%%%%%%%%%%%%%%%%%%%%%%%%%%%%%%%%%%%%%%%%%%%%%%%%%%%%%%%%%%%%%%%%%
\subsection{Forwarding}
\label{sec:forward}

Different versions of the main or child documents
using compilation flags as described in \secref{sec:flags}
can be (permanently) stored in different files
for convenient compilation, viewing and distribution.
To this end, the package defines a command
to pass on compilation to a different file:

%%%%%%%%%%%%%%%%%%%%%%%%%%%%%%%%%%%%%%%%
\DescribeMacro{\childdocforward}
The command |\childdocforward| redirects processing to
another source file:
%
\begin{center}
\begin{tabular}{l}
|% \iffalse
%
% childdoc.dtx Copyright (C) 2017-2018 Niklas Beisert
%
% This work may be distributed and/or modified under the
% conditions of the LaTeX Project Public License, either version 1.3
% of this license or (at your option) any later version.
% The latest version of this license is in
%   http://www.latex-project.org/lppl.txt
% and version 1.3 or later is part of all distributions of LaTeX
% version 2005/12/01 or later.
%
% This work has the LPPL maintenance status `maintained'.
%
% The Current Maintainer of this work is Niklas Beisert.
%
% This work consists of the files childdoc.dtx and childdoc.ins
% and the derived files childdoc.def and cdocsamp.tex with
% cdocsch1.tex, cdocsch2.tex, cdocsdrf.tex, cdocsfn1.tex, cdocsfn2.tex.
%
%<package>\ifdefined\childdocmain\endinput\fi
%<package>\ProvidesFile{childdoc.def}[2018/12/30 v2.0 child document driver]
%<samplemain>\ProvidesFile{cdocsamp.tex}[2018/12/30 v2.0 sample for childdoc]
%<*driver>
%\ProvidesFile{childdoc.drv}[2018/12/30 v2.0 childdoc reference manual file]
\PassOptionsToClass{10pt,a4paper}{article}
\documentclass{ltxdoc}

\usepackage[margin=35mm]{geometry}
\usepackage{hyperref}
\usepackage{hyperxmp}
\usepackage[usenames]{color}

\hypersetup{colorlinks=true}
\hypersetup{pdfstartview=FitH}
\hypersetup{pdfpagemode=UseNone}
\hypersetup{pdfsource={}}
\hypersetup{pdflang={en-UK}}
\hypersetup{pdfcopyright={Copyright 2017-2018 Niklas Beisert.
  This work may be distributed and/or modified under the
  conditions of the LaTeX Project Public License, either version 1.3
  of this license or (at your option) any later version.}}
\hypersetup{pdflicenseurl={http://www.latex-project.org/lppl.txt}}
\hypersetup{pdfcontactaddress={ETH Zurich, ITP, HIT K,
  Wolfgang-Pauli-Strasse 27}}
\hypersetup{pdfcontactpostcode={8093}}
\hypersetup{pdfcontactcity={Zurich}}
\hypersetup{pdfcontactcountry={Switzerland}}
\hypersetup{pdfcontactemail={nbeisert@itp.phys.ethz.ch}}
\hypersetup{pdfcontacturl={http://people.phys.ethz.ch/\xmptilde nbeisert/}}

\newcommand{\secref}[1]{\hyperref[#1]{section \ref*{#1}}}

\parskip1ex
\parindent0pt
\let\olditemize\itemize
\def\itemize{\olditemize\parskip0pt}

\begin{document}

\title{The \textsf{childdoc} Package}
\hypersetup{pdftitle={The childdoc Package}}
\author{Niklas Beisert\\[2ex]
  Institut f\"ur Theoretische Physik\\
  Eidgen\"ossische Technische Hochschule Z\"urich\\
  Wolfgang-Pauli-Strasse 27, 8093 Z\"urich, Switzerland\\[1ex]
  \href{mailto:nbeisert@itp.phys.ethz.ch}
  {\texttt{nbeisert@itp.phys.ethz.ch}}}
\hypersetup{pdfauthor={Niklas Beisert}}
\hypersetup{pdfsubject={Manual for the LaTeX2e Package childdoc}}
\date{30 December 2018, \textsf{v2.0}}
\maketitle

\begin{abstract}\noindent
\textsf{childdoc} is a \LaTeXe{} package
that enables the direct compilation
of document sections included by |\include|
to individual files.
\end{abstract}

\begingroup
\parskip0ex
\tableofcontents
\endgroup

%%%%%%%%%%%%%%%%%%%%%%%%%%%%%%%%%%%%%%%%%%%%%%%%%%%%%%%%%%%%%%%%%%%%%%%%%%%%%%%%
%%%%%%%%%%%%%%%%%%%%%%%%%%%%%%%%%%%%%%%%%%%%%%%%%%%%%%%%%%%%%%%%%%%%%%%%%%%%%%%%
\section{Introduction}

\LaTeX{} provides a mechanism to structure a large document (such as a book)
into a main file and several child files (containing the chapters)
using the |\include| command.
This mechanism is beneficial for documents
which span hundreds of pages in order to
make the source file(s) more manageable.
Moreover, compilation can be restricted to
selected child files by means of the |\includeonly| command.
The latter feature can be used to reduce the compilation time while editing
(this was significantly more useful in the earlier days of \LaTeX{})
or to generate a smaller document which is easier to navigate.
Another application of |\includeonly| is to generate
documents consisting of selected parts of the complete document.

However, there are a few drawbacks of the plain |\include| mechanism:
\begin{itemize}
\item
The child files cannot be compiled on their own,
they can only be compiled via the main file.
A naive editing environment
(such as a text editor with an option
to have the current file processed by \LaTeX)
may require one to switch to the main file before compiling;
attempting to compile the child file produces errors.
\item
The main file must be modified (each time)
to adjust the |\includeonly| command
to the present needs. This easily leaves the main file in a messy state.
\item
The generated document will always carry the filename
of the main document. This is inconvenient if
several child files are to be compiled and
to be kept for distribution.
\end{itemize}

The present package provides a simple interface
to make child files individually compilable by \LaTeX{}.
Compiling a child file then has the same effect as compiling
the main file with an |\includeonly| command
to select the appropriate child.
Moreover the generated document will carry the name of the child
rather than the main file.
This resolves all three above issues.

This feature is meant to make the editing of books,
thesis documents and lecture notes somewhat more convenient.
However, the package can also be used efficiently for
composing a series of documents (such as exercise sheets)
which are typically distributed individually.
It then assists the author in generating the individual documents
(potentially in different versions)
as well as a document containing the collected series.
Another application is in developing style files
or other kinds of included material
where compilation of the style file could redirect
to a sample or test file.

%%%%%%%%%%%%%%%%%%%%%%%%%%%%%%%%%%%%%%%%%%%%%%%%%%%%%%%%%%%%%%%%%%%%%%%%%%%%%%%%
%%%%%%%%%%%%%%%%%%%%%%%%%%%%%%%%%%%%%%%%%%%%%%%%%%%%%%%%%%%%%%%%%%%%%%%%%%%%%%%%
\section{Usage}

First of all, the package \textsf{childdoc} is \emph{not} a standard
\LaTeXe{} |.sty| style file! Therefore it needs to be invoked in
a non-standard way.

%%%%%%%%%%%%%%%%%%%%%%%%%%%%%%%%%%%%%%%%%%%%%%%%%%%%%%%%%%%%%%%%%%%%%%%%%%%%%%%%
\subsection{Included Files}
\label{sec:include}

%%%%%%%%%%%%%%%%%%%%%%%%%%%%%%%%%%%%%%%%
\DescribeMacro{\childdocmain}
To use the package, add the commands
\begin{center}
\begin{tabular}{l}
|\input{childdoc.def}|\\
|\childdocmain{}|\\
\end{tabular}
\end{center}
at the very top of the main \LaTeX{} file,
in particular \emph{before} the |\documentclass| statement!
The argument of |\childdocmain| should be left empty
(but it must be present).

%%%%%%%%%%%%%%%%%%%%%%%%%%%%%%%%%%%%%%%%
\DescribeMacro{\childdocof}
Furthermore, add the commands
\begin{center}
\begin{tabular}{l}
|\input{childdoc.def}|\\
|\childdocof{|\textit{main}|}|\\
\end{tabular}
\end{center}
at the top of every child file \textit{child}
which is included by |\include{|\textit{child}|}|
from within the main file
(or at least for those files to be compiled individually).
The argument \textit{main} must be the filename of the main file.

There are a couple of
considerations in setting up the main and child documents:

%%%%%%%%%%%%%%%%%%%%%%%%%%%%%%%%%%%%%%%%
\paragraph{Restrictions.}

Please note the following restrictions:
\begin{itemize}
\item
|\childdocmain| must be called with one argument \textit{main}
to ensure compatibility with earlier version of the package.
It must either be empty (|\childdocmain{}|)
or precisely match the filename of the main file in which it is specified.
See \secref{sec:detection} for further information.
\item
The filename \textit{main} must be specified without the |.tex| extension.
\item
The filename \textit{main} is case sensitive
(even in case-insensitive file systems)
due to internal string comparison.
\item
The argument \textit{main} should be fully expanded, it cannot be a macro.
\item
Subdirectories and special characters should be avoided in filenames.
\item
The command |\childdocmain{|\textit{main}|}| must be followed by a whitespace.
It should not be followed immediately by another command
or by a comment mark `|%|'.
This is because the \TeX{} parser reads the token immediately following
the argument of |\childdocmain| and puts it
at the beginning of every child section;
however, a white\-space is ignored.
\end{itemize}

%%%%%%%%%%%%%%%%%%%%%%%%%%%%%%%%%%%%%%%%
\paragraph{Content of Main File.}

It is advisable to place all content in the child files included by |\include|.
Any output contained in the main file will appear in all child documents
unless suppressed manually;
it cannot be suppressed automatically by the |\includeonly| directive
and thus should normally be avoided.
A method to include some content in the main file
by means of conditional processing is described in \secref{sec:conditional}.

%%%%%%%%%%%%%%%%%%%%%%%%%%%%%%%%%%%%%%%%
\paragraph{Page Numbering.}

When only a part of the document is compiled,
the appropriate numbering of pages
(as well as other status parameters)
is determined from the |.aux| files.
The latter contain information from previous passes.
However this information needs to propagate through
all intermediate child documents.
Therefore the page numbering in child documents may well
be inconsistent until the complete document is compiled at least once.

A useful (if unconventional) way to always ensure a consistent
page numbering is to restart the numbering in each child document
and denote the pages by `\textit{child}|.|\textit{page}'
where \textit{child} represents the chapter/section number of the child file.
This can be achieved by the command
|\numberwithin{page}{|\textit{child}|}|
of the \textsf{amsmath} package
where \textit{child} can be |chapter| or |section|
depending on the chosen structuring.
Alternatively, one can modify the macro |\thepage| appropriately
and reset the counter |page| at the start of each child file.

%%%%%%%%%%%%%%%%%%%%%%%%%%%%%%%%%%%%%%%%%%%%%%%%%%%%%%%%%%%%%%%%%%%%%%%%%%%%%%%%
\subsection{Conditional Processing}
\label{sec:conditional}

The package provides a mechanism to compile different versions
of a document. To customise the versions further some conditional processing
can come in handy to distinguish which version is being compiled.
The package provides two macros to describe the compilation context:

%%%%%%%%%%%%%%%%%%%%%%%%%%%%%%%%%%%%%%%%
\DescribeMacro{\ifchilddoc}
The conditional |\ifchilddoc| distinguishes between the compilation of
child documents and the main document:
%
\begin{center}
|\ifchilddoc |\textit{child-code}| |[|\||else |\textit{main-code}]| \||fi|
\end{center}

%%%%%%%%%%%%%%%%%%%%%%%%%%%%%%%%%%%%%%%%
\DescribeMacro{\childdocname}
\DescribeMacro{\childdocjob}
The macro |\childdocname| contains the filename (without extension)
of the main or child file being processed.
Note that |\childdocjob| will always contain the name of the main file.

%%%%%%%%%%%%%%%%%%%%%%%%%%%%%%%%%%%%%%%%
\paragraph{Title Page.}

Conditional processing can be used to include a title or banner page
in the main document when proper precautions are taken.
Importantly, the code in the main file should ensure that the page counter
(as well as other status parameters which are stored in the |.aux| files)
takes the same value after the conditional processing.
Otherwise the page numbers may take divergent values
depending on which part is compiled.

For example, a title page could be declared by:
%
\begin{center}
\begin{tabular}{l}
|\ifchilddoc\||else|\\
|\addtocounter{page}{-1}|\\
\textit{code for title page}\\
|\newpage|\\
|\||fi|
\end{tabular}
\end{center}
%
A banner page for the child documents can be generated by:
%
\begin{center}
\begin{tabular}{l}
|\ifchilddoc|\\
|\addtocounter{page}{-1}|\\
\textit{code for banner page}\\
|\newpage|\\
|\||fi|
\end{tabular}
\end{center}
%
Here one could write a message such as:
\begin{center}
|This is the part \childdocname{} of \childdocjob{}.|
\end{center}

%%%%%%%%%%%%%%%%%%%%%%%%%%%%%%%%%%%%%%%%%%%%%%%%%%%%%%%%%%%%%%%%%%%%%%%%%%%%%%%%
\subsection{Flags}
\label{sec:flags}

The package makes it easy to generate different versions
of the main or child documents.
To this end compilation flags can be defined
and assigned different default values.
They will be particularly useful in conjunction
with the forwarding mechanism described in \secref{sec:forward}.

For example, it may be useful to have a flag |\version|
which can be set to |draft| or |final|.
The document source will contain some conditional code
depending on the value of |\version|.
Suppose further, the flag should default to |final| for the main file
and to |draft| for child files
which is a natural assignment for editing the document.
This is achieved by placing the following code
in the preamble of the main document
(below the |\childdocmain| directive):
%
\begin{center}
\begin{tabular}{l}
|\ifchilddoc|\\
|\providecommand{\version}{draft}|\\
|\||else|\\
|\providecommand{\version}{final}|\\
|\||fi|
\end{tabular}
\end{center}
%
The definition by |\providecommand| makes sure
that previous definitions are not overwritten.
Further statements |\providecommand{\version}{...}|
can thus be added before the above code to override it.

For the main file, one might add a line
(between |\childdocmain| and the above block)
%
\begin{center}
|%\ifchilddoc\||else\providecommand{\version}{draft}\||fi|
\end{center}
%
which can be uncommented to produce a draft version.
Likewise one can add a line to the very top of a child file
(above the |\childdocof{|\textit{main}|}| directive)
%
\begin{center}
|%\providecommand{\version}{final}|
\end{center}
%
which can be uncommented to produce the final version of this child document.

%%%%%%%%%%%%%%%%%%%%%%%%%%%%%%%%%%%%%%%%%%%%%%%%%%%%%%%%%%%%%%%%%%%%%%%%%%%%%%%%
\subsection{Forwarding}
\label{sec:forward}

Different versions of the main or child documents
using compilation flags as described in \secref{sec:flags}
can be (permanently) stored in different files
for convenient compilation, viewing and distribution.
To this end, the package defines a command
to pass on compilation to a different file:

%%%%%%%%%%%%%%%%%%%%%%%%%%%%%%%%%%%%%%%%
\DescribeMacro{\childdocforward}
The command |\childdocforward| redirects processing to
another source file:
%
\begin{center}
\begin{tabular}{l}
|\input{childdoc.def}|\\
|\childdocforward[|\textit{main}|]{|\textit{dest}|}|\\
\end{tabular}
\end{center}
%
The argument \textit{dest} is the destination file
(without extension).
It should be the main file or one of the child files.
Note that further \textsf{childdoc} directives
such as |\childdocof| and |\childdocforward|
in the indicated file will be processed in this form.
The optional argument \textit{main}
passes on directly to the main file \textit{main}
while pretending to compile the child \textit{dest}.
This form behaves as if \textit{dest}
issues |\childdocof{|\textit{main}|}| right away,
and no further \textsf{childdoc} directives will be processed.

%%%%%%%%%%%%%%%%%%%%%%%%%%%%%%%%%%%%%%%%
\DescribeMacro{\...prefix}
In the alternative form |\childdocforwardprefix|,
%
\begin{center}
\begin{tabular}{l}
|\input{childdoc.def}|\\
|\childdocforwardprefix[|\textit{main}|]{|\textit{prefix}|}{|\textit{dest}|}|
\end{tabular}
\end{center}
%
the destination file is determined by a pattern
depending on the current file:
To make this work, the current file must be called
`{\textit{prefix}\hspace{0.2em}\textit{suffix}}'
with \textit{prefix} matching precisely the argument.
Processing is then passed on to the file
`{\textit{dest}\hspace{0.2em}\textit{suffix}}'.
Surely, the same effect is achieved by
directly specifying the
argument `{\textit{dest}\hspace{0.2em}\textit{suffix}}'
in the first form.
However, that requires to set up a different file
for each child. With the alternative form of the command
all these files can have exactly the same content
which simplifies setting them up and maintaining them.

For example, the following file |draft.tex|
with a compilation flag |\version| as described in \secref{sec:flags}
compiles the main document as a draft:
%
\begin{center}
\begin{tabular}{l}
|\def\version{draft}|\\
|\input{childdoc.def}|\\
|\childdocforward{|\textit{main}|}|
\end{tabular}
\end{center}
%
Likewise, the following files |final|\textit{nn}|.tex|
compile the final version of the child document
|child|\textit{nn}|.tex|:
%
\begin{center}
\begin{tabular}{l}
|\def\version{final}|\\
|\input{childdoc.def}|\\
|\childdocforwardprefix{final}{child}|
\end{tabular}
\end{center}
%

Note that when several versions of a main file and/or of each child file
are to be generated, it may be convenient to set up a |Makefile| or
shell script to automatise the process.

%%%%%%%%%%%%%%%%%%%%%%%%%%%%%%%%%%%%%%%%%%%%%%%%%%%%%%%%%%%%%%%%%%%%%%%%%%%%%%%%
\subsection{Command Line Processing}
\label{sec:commandline}

The effect of redirection files can also be achieved by invoking
the \LaTeX{} compiler with a more elaborate command line.
Most conveniently this should be done as part
of a shell script or a |Makefile|.

When using \textsf{childdoc} in the main file, the following
command lines effectively perform a redirection
(note that depending on the shell being used,
backslashes may have to be doubled: `|\|' $\to$ `|\\|'):
%
\begin{center}
|... -jobname "|\textit{target}|" |\\|"|[\textit{flags}]%
|\input{childdoc.def}\childdocforward[|\textit{main}|]{|\textit{dest}|}"|
\end{center}
%
Here \textit{target} is the name of the output file,
\textit{main} is the name of the main file
and \textit{dest} is the name of the main or child file to be processed
(all filenames without extensions).
The optional argument \textit{main} can be omitted
if \textit{main} matches \textit{dest}.
Optionally, compilation \textit{flags} can be defined via |\def| commands.
This command line makes the \TeX{} engine believe
it is compiling the file \textit{target}
whose content is specified as the latter parameter.
The provided code then forwards the processing to
\textit{main} or \textit{dest} as described in \secref{sec:forward}.

%%%%%%%%%%%%%%%%%%%%%%%%%%%%%%%%%%%%%%%%%%%%%%%%%%%%%%%%%%%%%%%%%%%%%%%%%%%%%%%%
\subsection{Include by Input}
\label{sec:input}

Including child documents by |\include| has some restrictions by design.
Most notably, the content of a child document always occupies
its own set of pages; pages cannot be shared between child documents.
Usually, this behaviour makes perfect sense
because each child document contain an essential part of the document.
However, in some situations it may be desirable to compose
a document from a collection of parts
without having mandatory page breaks between then.
For this case, the package
provides a mechanism to include parts
by |\input| which can also be processed individually.
However, by construction this mechanism
requires manual handling of the content to be output.

%%%%%%%%%%%%%%%%%%%%%%%%%%%%%%%%%%%%%%%%
\DescribeMacro{\ifchilddocmanual}
The main file should be prepared as usual, see \secref{sec:include}.
However, the document body must make a distinction
between processing of an individual part and of the main document, e.g.:
%
\begin{center}
\begin{tabular}{l}
|\ifchilddocmanual|\\
|\input{\childdocname}|\\
|\||else|\\
\textit{document body with }|\input{|\textit{part}|}|\\
|\||fi|
\end{tabular}
\end{center}
%
The conditional |\ifchilddocmanual| is true whenever
a part to be included by |\input| is being compiled,
and the name of the part is stored in |\childdocname|.

%%%%%%%%%%%%%%%%%%%%%%%%%%%%%%%%%%%%%%%%
\DescribeMacro{\childdocby}
Each part to be included by |\input| should start with:
%
\begin{center}
\begin{tabular}{l}
|\input{childdoc.def}|\\
|\childdocby{|\textit{main}|}|\\
\end{tabular}
\end{center}
%
The directive |\childdocby| is similar to |\childdocof|
described in \secref{sec:include},
but the subsequent selection of content must be done manually.
To that end, both |\ifchilddoc| and |\ifchilddocmanual|
will be true upon processing of a part,
and the name of the part is stored in |\childdocname|.
Note that |\jobname| will be set to the filename of the current part
so that each part receives an individual |.aux| file
that does not interfere with the |.aux| file(s) of the main document.
This behaviour can be altered by the alternative form
|\childdocby[*]{|\textit{main}|}| (with a non-empty optional argument)
which uses the |.aux| file of the main document
by setting |\jobname| to \textit{main}.

%%%%%%%%%%%%%%%%%%%%%%%%%%%%%%%%%%%%%%%%%%%%%%%%%%%%%%%%%%%%%%%%%%%%%%%%%%%%%%%%
\subsection{Driver Development}
\label{sec:driver}

The \textsf{childdoc} mechanism can also be use for the development
of definition files such as \LaTeX{} styles or classes.
This case differs from the above setup with multiple parts
included by |\include| in that no |\includeonly| should be invoked.
This can be achieved by starting the include file
(before |\ProvidesPackage|) with:
%
\begin{center}
\begin{tabular}{l}
|\input{childdoc.def}|\\
|\childdocforward{|\textit{main}|}|\\
\end{tabular}
\end{center}
%
or alternatively with:
%
\begin{center}
\begin{tabular}{l}
|\input{childdoc.def}|\\
|\childdocby{|\textit{main}|}|\\
\end{tabular}
\end{center}
%
Both forms have slightly different effects as described above.
The main file is prepared as usual, see \secref{sec:include}.

%%%%%%%%%%%%%%%%%%%%%%%%%%%%%%%%%%%%%%%%%%%%%%%%%%%%%%%%%%%%%%%%%%%%%%%%%%%%%%%%
\subsection{Legacy Detection}
\label{sec:detection}

The directive |\childdocmain| in the main file can detect
whether the complete document or merely a child is to be compiled
even without using the directive |\childdocof|.
This method is deprecated because it is less robust
and there is no compelling reason to use it;
it is merely provided for backward compatibility
and it may be removed in future versions.

If the detection mechanism is to be used,
it is mandatory to correctly specify
the filename of the main file as the argument of |\childdocmain|:
%
\begin{center}
\begin{tabular}{l}
|\input{childdoc.def}|\\
|\childdocmain{|\textit{main}|}|\\
\end{tabular}
\end{center}
%
If |\jobname| does not match the argument \textit{main} of |\childdocmain|,
it is assumed that |\jobname| points to the child file to be compiled.
When using |\childdocmain| with the main file specified as argument,
it suffices to start a child file
with just |\input{|\textit{main}|}|
without loading of the package and using |\childdocof|.
If instead all processing is done
with the appropriate \textsf{childdoc} directives,
the argument of \textit{main} of |\childdocmain| can be empty.

An alternative version of the command line processing described
in \secref{sec:commandline} using the detection mechanism reads:
%
\begin{center}
|... -jobname "|\textit{target}|" "|[\textit{flags}]%
[|\def\jobname{|\textit{dest}|}|]|\input{|\textit{main}|}"|
\end{center}

%%%%%%%%%%%%%%%%%%%%%%%%%%%%%%%%%%%%%%%%%%%%%%%%%%%%%%%%%%%%%%%%%%%%%%%%%%%%%%%%
\subsection{Manual Code}
\label{sec:manual}

In case one cannot be certain whether the definitions file |childdoc.def|
is installed on the target \TeX{} distribution
and one prefers not to ship it,
it is conceivable to paste a few relevant commands into the sources.

To that end, drop all statements |\input{childdoc.def}|
and perform the replacements as outlined below.
Instead of |\childdocmain{|\textit{main}|}| add the following code
to the top of the main file:
%
\begin{center}
\begin{tabular}{l}
|\||ifdefined\childdocname\endinput\||fi\newif\ifchilddoc|\\
|\edef\childdocname{\scantokens\expandafter{\jobname\noexpand}}|\\
|\def\childdocmain{|\textit{main}|}\||ifx\childdocmain\childdocname\||else|\\
|\childdoctrue\includeonly{\childdocname}\let\jobname\childdocmain\||fi|\\
\end{tabular}
\end{center}
%
Instead of |\childdocof{|\textit{main}|}| just include the main file
at the top of each child file:
%
\begin{center}
|\input{|\textit{main}|}|
\end{center}
%
A simple redirection |\childdocforward{|\textit{dest}|}| is achieved by:
%
\begin{center}
|\def\jobname{|\textit{dest}|}\input{\jobname}|
\end{center}
%
The redirection with prefix
|\childdocforwardprefix[|\textit{prefix}|]{|\textit{dest}|}|
is accomplished by:
%
\begin{center}
\begin{tabular}{l}
|{\edef\jobname{\scantokens\expandafter{\jobname\noexpand}}|\\
|\def\redirectjob |\textit{prefix}|#1~~~{\gdef\jobname{|\textit{dest}|#1}}|\\
|\expandafter\redirectjob\jobname~~~}\input{\jobname}|
\end{tabular}
\end{center}

In an alternative approach,
child documents can be compiled by a specific command line
without additional code or specific definitions:
%
\begin{center}
|... -jobname "|\textit{target}|" "|[\textit{flags}]%
|\includeonly{|\textit{dest}|}\input{|\textit{main}|}"|
\end{center}
%

%%%%%%%%%%%%%%%%%%%%%%%%%%%%%%%%%%%%%%%%%%%%%%%%%%%%%%%%%%%%%%%%%%%%%%%%%%%%%%%%
%%%%%%%%%%%%%%%%%%%%%%%%%%%%%%%%%%%%%%%%%%%%%%%%%%%%%%%%%%%%%%%%%%%%%%%%%%%%%%%%
\section{Information}

%%%%%%%%%%%%%%%%%%%%%%%%%%%%%%%%%%%%%%%%%%%%%%%%%%%%%%%%%%%%%%%%%%%%%%%%%%%%%%%%
\subsection{Copyright}

Copyright \copyright{} 2017--2018 Niklas Beisert

This work may be distributed and/or modified under the
conditions of the \LaTeX{} Project Public License, either version 1.3
of this license or (at your option) any later version.
The latest version of this license is in
  \url{http://www.latex-project.org/lppl.txt}
and version 1.3 or later is part of all distributions of \LaTeX{}
version 2005/12/01 or later.

This work has the LPPL maintenance status `maintained'.

The Current Maintainer of this work is Niklas Beisert.

This work consists of the files |README.txt|, |childdoc.ins| and |childdoc.dtx|
as well as the derived files |childdoc.def|, |cdocsamp.tex|
with |cdocsch1.tex|, |cdocsch2.tex|, |cdocspt3.tex|, |cdocspt4.tex|,
|cdocsdrf.tex|, |cdocsfn1.tex|, |cdocsfn2.tex|
as well as |childdoc.pdf|.

%%%%%%%%%%%%%%%%%%%%%%%%%%%%%%%%%%%%%%%%%%%%%%%%%%%%%%%%%%%%%%%%%%%%%%%%%%%%%%%%
\subsection{Files and Installation}

The package consists of the files:
%
\begin{center}
\begin{tabular}{ll}
    |README.txt|   & readme file \\
    |childdoc.ins| & installation file \\
    |childdoc.dtx| & source file \\
    |childdoc.def| & definition file \\
    |cdocsamp.tex| & sample main file \\
    |cdocsch1.tex| & sample include file \\
    |cdocsch2.tex| & sample include file \\
    |cdocspt3.tex| & sample part file \\
    |cdocspt4.tex| & sample part file \\
    |cdocsdrf.tex| & sample redirection file \\
    |cdocsfn1.tex| & sample redirection file \\
    |cdocsfn2.tex| & sample redirection file \\
    |childdoc.pdf| & manual
\end{tabular}
\end{center}
%
The distribution consists of the files
|README.txt|, |childdoc.ins| and |childdoc.dtx|.
%
\begin{itemize}
\item
Run (pdf)\LaTeX{} on |childdoc.dtx|
to compile the manual |childdoc.pdf| (this file).
\item
Run \LaTeX{} on |childdoc.ins| to create the definitions file |childdoc.def|
and the sample |cdocsamp.tex| with include files
|cdocsch1.tex|, |cdocsch2.tex|, |cdocspt3.tex|, |cdocspt4.tex|,
|cdocsdrf.tex|, |cdocsfn1.tex|, |cdocsfn2.tex|.
Then copy the file |childdoc.def| to an appropriate directory of your \LaTeX{}
distribution, e.g.\ \textit{texmf-root}|/tex/latex/childdoc|.
\end{itemize}

%%%%%%%%%%%%%%%%%%%%%%%%%%%%%%%%%%%%%%%%%%%%%%%%%%%%%%%%%%%%%%%%%%%%%%%%%%%%%%%%
\subsection{Related CTAN Packages}

There are several other packages which offer a similar functionality:
%
\begin{itemize}
\item
The packages
\href{http://ctan.org/pkg/docmute}{\textsf{docmute}},
\href{http://ctan.org/pkg/includex}{\textsf{includex}} and
\href{http://ctan.org/pkg/standalone}{\textsf{standalone}}
provide commands to include only the document body of
a child file thus allowing both files to be compiled individually.
\item
The packages \href{http://ctan.org/pkg/subdocs}{\textsf{subdocs}}
and \href{http://ctan.org/pkg/subfiles}{\textsf{subfiles}}
provide structures in which the main and child documents can be
encapsulated and allowing them to be compiled individually.
The inclusion mechanism is different from the conventional |\include|.
\item
The package \href{http://ctan.org/pkg/combine}{\textsf{combine}}
is an elaborate solution to combine several documents into one.
\end{itemize}
%
See also the CTAN topic \href{http://ctan.org/topic/subdocs}{\textsf{subdocs}}
for further related packages.
The present package differs from the above solutions in that
a document structure constructed with the conventional |\include| mechanism
just needs two extra commands at the top of every file
such that all constituent files can be compiled individually.

%%%%%%%%%%%%%%%%%%%%%%%%%%%%%%%%%%%%%%%%%%%%%%%%%%%%%%%%%%%%%%%%%%%%%%%%%%%%%%%%
%\subsection{Feature Suggestions}
%
%The following is a list of features which may be useful for future
%versions of this package:
%%
%\begin{itemize}
%\item
%\ldots
%\end{itemize}

%%%%%%%%%%%%%%%%%%%%%%%%%%%%%%%%%%%%%%%%%%%%%%%%%%%%%%%%%%%%%%%%%%%%%%%%%%%%%%%%
\subsection{Revision History}

%%%%%%%%%%%%%%%%%%%%%%%%%%%%%%%%%%%%%%%%
\paragraph{v2.0:} 2018/12/30

\begin{itemize}
\item
immediate forward processing
\item
added |\childdocby| mechanism
\item
manual restructured
\end{itemize}

%%%%%%%%%%%%%%%%%%%%%%%%%%%%%%%%%%%%%%%%
\paragraph{v1.6:} 2018/01/17

\begin{itemize}
\item
application for development of include files
\item
corrections to manual
\end{itemize}

%%%%%%%%%%%%%%%%%%%%%%%%%%%%%%%%%%%%%%%%
\paragraph{v1.5:} 2017/05/21

\begin{itemize}
\item
more complete structuring introduced
\item
|\childdocof| introduced
\item
|\childdoc| renamed to |\childdocmain|
\item
|\childredirect| renamed to |\childdocforward| and |\childdocforwardprefix|
and functionality expanded
\end{itemize}

%%%%%%%%%%%%%%%%%%%%%%%%%%%%%%%%%%%%%%%%
\paragraph{v1.0:} 2017/04/27

\begin{itemize}
\item
manual and install package
\item
first version published on CTAN
\end{itemize}

%%%%%%%%%%%%%%%%%%%%%%%%%%%%%%%%%%%%%%%%
\paragraph{v0.6:} 2017/04/26

\begin{itemize}
\item
redirection mechanism added
\end{itemize}

%%%%%%%%%%%%%%%%%%%%%%%%%%%%%%%%%%%%%%%%
\paragraph{v0.5:} 2017/04/26

\begin{itemize}
\item
functionality in definition file
\end{itemize}


%%%%%%%%%%%%%%%%%%%%%%%%%%%%%%%%%%%%%%%%%%%%%%%%%%%%%%%%%%%%%%%%%%%%%%%%%%%%%%%%
%%%%%%%%%%%%%%%%%%%%%%%%%%%%%%%%%%%%%%%%%%%%%%%%%%%%%%%%%%%%%%%%%%%%%%%%%%%%%%%%
%%%%%%%%%%%%%%%%%%%%%%%%%%%%%%%%%%%%%%%%%%%%%%%%%%%%%%%%%%%%%%%%%%%%%%%%%%%%%%%%
\appendix

\settowidth\MacroIndent{\rmfamily\scriptsize 000\ }

 \DocInput{childdoc.dtx}

\end{document}
%</driver>
% \fi
%
% %%%%%%%%%%%%%%%%%%%%%%%%%%%%%%%%%%%%%%%%%%%%%%%%%%%%%%%%%%%%%%%%%%%%%%%%%%%%%%
% %%%%%%%%%%%%%%%%%%%%%%%%%%%%%%%%%%%%%%%%%%%%%%%%%%%%%%%%%%%%%%%%%%%%%%%%%%%%%%
% \section{Sample}
%\iffalse
%<*samplemain>
%\fi
%
% The following presents a sample document
% with two chapters, two parts, a title page,
% a compile flag as well as three forwarding files to set the flag.
% It consists of eight |.tex| files:
% \begin{center}
% \begin{tabular}{ll}
% |cdocsamp.tex|&main file\\
% |cdocsch1.tex|&include file for chapter 1\\
% |cdocsch2.tex|&include file for chapter 2\\
% |cdocspt3.tex|&include file for part 3\\
% |cdocspt4.tex|&include file for part 4\\
% |cdocsdrf.tex|&forwarding file for main file in draft mode\\
% |cdocsfi1.tex|&forwarding file for final version of chapter 1\\
% |cdocsfi2.tex|&forwarding file for final version of chapter 2\\
% \end{tabular}
% \end{center}
% Each of the eight files can be compiled directly by the \LaTeX{} compiler.
%
% %%%%%%%%%%%%%%%%%%%%%%%%%%%%%%%%%%%%%%
% \paragraph{Main File.}
%
% The main file is called |cdocsamp.tex|.
%
% Load the \textsf{childdoc} definitions and
% declare the filename for the main document:
%    \begin{macrocode}
\input{childdoc.def}
\childdocmain{}
%    \end{macrocode}

% Optional override for |\version| flag:
%    \begin{macrocode}
%%\ifchilddoc\else\providecommand{\version}{draft}\fi
%    \end{macrocode}

% Define the default values for the |\version| flag
% (|final| for the main file and |draft| for childs):
%    \begin{macrocode}
\ifchilddoc
\providecommand{\version}{draft}
\else
\providecommand{\version}{final}
\fi
%    \end{macrocode}

% Load the standard document class:
%    \begin{macrocode}
\documentclass[12pt]{article}
%    \end{macrocode}

% Start the document body:
%    \begin{macrocode}
\begin{document}
%    \end{macrocode}

% Declare a title page.
% Print title, part of document being processed and version flag:
%    \begin{macrocode}
\addtocounter{page}{-1}
\begin{center}
{\LARGE\bfseries{}childdoc example\par}
\vspace{1cm}
\ifchilddoc
\ifchilddocmanual part\else chapter\fi:
`\childdocname' of `\childdocjob'\par
\else
main document: `\childdocjob'\par
\fi
version: \version\par
\end{center}
\newpage
%    \end{macrocode}

% Manually include selected file,
% otherwise process as usual:
%    \begin{macrocode}
\ifchilddocmanual
\section*{part `\childdocname'}
\input{\childdocname}
\else
%    \end{macrocode}

% Include the two chapters:
%    \begin{macrocode}
\include{cdocsch1}
\include{cdocsch2}
%    \end{macrocode}

% Include the two parts unless only chapters should be displayed:
%    \begin{macrocode}
\ifchilddoc\else
\section{part three}
\input{cdocspt3}
\section{part four}
\input{cdocspt4}
\fi
%    \end{macrocode}

% Process as usual until here:
%    \begin{macrocode}
\fi
%    \end{macrocode}

% End of document body:
%    \begin{macrocode}
\end{document}
%    \end{macrocode}
%\iffalse
%</samplemain>
%\fi
%
% %%%%%%%%%%%%%%%%%%%%%%%%%%%%%%%%%%%%%%
% \paragraph{Chapter Include Files.}
%
% The include files are called |cdocsch1.tex| and |cdocsch2.tex|.
%
%\iffalse
%<*samplechap1|samplechap2>
%\fi

% Optional override for |\version| flag:
%    \begin{macrocode}
%%\providecommand{\version}{final}
%    \end{macrocode}

% Include the main document:
%    \begin{macrocode}
\input{childdoc.def}
\childdocof{cdocsamp}
%    \end{macrocode}

%\iffalse
%</samplechap1|samplechap2>
%\fi
%
%\iffalse
%<*samplechap1>
%\fi
% Some text for chapter 1:
%    \begin{macrocode}
\section{one}
some text in chapter one
%    \end{macrocode}

%\iffalse
%</samplechap1>
%\fi
% Some text for chapter 2:
%\iffalse
%<*samplechap2>
%\fi
%    \begin{macrocode}
\section{two}
more text in chapter two
%    \end{macrocode}

%\iffalse
%</samplechap2>
%\fi
%
% %%%%%%%%%%%%%%%%%%%%%%%%%%%%%%%%%%%%%%
% \paragraph{Part Include Files.}
%
% The include files are called |cdocspt3.tex| and |cdocspt4.tex|.
%
%\iffalse
%<*samplepart3|samplepart4>
%\fi

% Optional override for |\version| flag:
%    \begin{macrocode}
%%\providecommand{\version}{final}
%    \end{macrocode}

% Include the main document:
%    \begin{macrocode}
\input{childdoc.def}
\childdocby{cdocsamp}
%    \end{macrocode}

%\iffalse
%</samplepart3|samplepart4>
%\fi
%
%\iffalse
%<*samplepart3>
%\fi
% Some text for part 3:
%    \begin{macrocode}
some text in part three
%    \end{macrocode}

%\iffalse
%</samplepart3>
%\fi
% Some text for part 4:
%\iffalse
%<*samplepart4>
%\fi
%    \begin{macrocode}
more text in part four
%    \end{macrocode}

%\iffalse
%</samplepart4>
%\fi
%
% %%%%%%%%%%%%%%%%%%%%%%%%%%%%%%%%%%%%%%
% \paragraph{Forwarding for a Complete Draft.}
%
% The following forwarding file |cdocsdrf.tex|
% compiles the main document in draft mode:
%\iffalse
%<*sampledraft>
%\fi
%    \begin{macrocode}
\def\version{draft}
\input{childdoc.def}
\childdocforward{cdocsamp}
%    \end{macrocode}

%\iffalse
%</sampledraft>
%\fi
%
% %%%%%%%%%%%%%%%%%%%%%%%%%%%%%%%%%%%%%%
% \paragraph{Forwarding for Final Version of the Chapters.}
%
% The following forwarding files |cdocsfn1.tex| and |cdocsfn2.tex|
% (with identical content)
% compile the final versions of the child documents
% |cdocsch1.tex| and |cdocsch2.tex|, respectively:
%\iffalse
%<*samplefinal>
%\fi
%    \begin{macrocode}
\def\version{final}
\input{childdoc.def}
\childdocforwardprefix[cdocsamp]{cdocsfn}{cdocsch}
%    \end{macrocode}

%\iffalse
%</samplefinal>
%\fi
%
% %%%%%%%%%%%%%%%%%%%%%%%%%%%%%%%%%%%%%%
% \paragraph{Command Line Processing.}
%
% The following three command lines generate the output files
% |cdocscld|, |cdocscl1| and |cdocscl2|
% which should be identical to
% |cdocsdrf|, |cdocsch1| and |cdocsfn2|, respectively:
% \begin{center}
% \begin{tabular}{l}
% |latex -jobname cdocscld \|\\
% |  "\def\version{draft}\input{childdoc.def}\childdocforward{cdocsamp}"|\\
% |latex -jobname cdocscl1 \|\\
% |  "\input{childdoc.def}\childdocforward[cdocsamp]{cdocsch1}"|\\
% |latex -jobname cdocscl2 \|\\
% |  "\def\version{final}\input{childdoc.def}\childdocforward{cdocsch2}"|
% \end{tabular}
% \end{center}
% Note that the trailing backslash on each first line
% merely continues the input to the second line
% (for convenient cut ant paste).
% Furthermore, the command |latex| can be replaced by any
% of its alternative versions such as |pdflatex|.
%
% %%%%%%%%%%%%%%%%%%%%%%%%%%%%%%%%%%%%%%%%%%%%%%%%%%%%%%%%%%%%%%%%%%%%%%%%%%%%%%
% %%%%%%%%%%%%%%%%%%%%%%%%%%%%%%%%%%%%%%%%%%%%%%%%%%%%%%%%%%%%%%%%%%%%%%%%%%%%%%
% \section{Implementation}
%\iffalse
%<*package>
%\fi
%
% This section describes the definitions file |childdoc.def|.

% The definitions cannot be loaded using |\usepackage| or |\RequirePackage|
% which has a mechanism to prevent loading a style file more than once.
% When loading the definitions by means of |\input|
% multiple instances have to be prevented manually:
%\iffalse
%This code needs to be before the `\ProvidesFile' directive
%which is defined at the beginning of this file.
%Therefore it is also placed there and commented out here.
%</package>
%<*discard>
%\fi
%    \begin{macrocode}
\ifdefined\childdocmain\endinput\fi
%    \end{macrocode}
%\iffalse
%</discard>
%<*package>
%\fi
%
% \macro{\ifchilddoc}
% \macro{\ifchilddocmanual}
% The conditional |\ifchilddoc| tells whether a
% child (true) or main (false) document is being compiled.
% The conditional |\ifchilddocmanual| tells whether
% the |\includeonly| mechanism is used (false) or
% the selection of child files must be performed manually (true).
% The definitions initialise to false:
%    \begin{macrocode}
\newif\ifchilddoc
\newif\ifchilddocmanual
%    \end{macrocode}

% \macro{\childdocname}
% \macro{\childdocjob}
% The macro |\childdocname| stores the name of the main document
% to be compiled. The macro |\childdocjob| stores the name of
% the document on which the \LaTeX{} compiler was originally invoked.
% The content of |\jobname| cannot be compared
% to filenames specified in the source due to different catcodes.
% The following code rescans |\jobname|, stores the result
% in |\childdocname| and saves a copy in |\childdocjob|:
%    \begin{macrocode}
\edef\childdocname{\scantokens\expandafter{\jobname\noexpand}}
\let\childdocjob\childdocname
%    \end{macrocode}

% \macro{\childdocdisable}
% The macro |\childdocdisable| prevents the main file
% from being processed more than once.
% At this stage, the main document command |\childdocmain|
% is assumed to be called once again where it should do nothing.
% Any subsequent call to it should prevent
% a secondary processing of the main document
% It overwrites the forwarding commands
% |\childdocof| and |\childdocforward|
% with empty macros to prevent further inclusions of the main document:
%    \begin{macrocode}
\newcommand{\childdocdisable}
{
  \renewcommand{\childdocmain}[1]{\renewcommand{\childdocmain}[1]{\endinput}}
  \renewcommand{\childdocof}[1]{}
  \renewcommand{\childdocby}[2][]{}
  \renewcommand{\childdocforward}[2][]{}
  \renewcommand{\childdocdisable}{}
}
%    \end{macrocode}

% \macro{\childdocmain}
% The macro |\childdocmain| is to be called at the top of the main file
% with nothing or the main filename (without extension) as argument.
% First, it breaks loops.
% If the argument is not empty and does not match |\childdocname|
% (which is set by the first inclusion of |childdoc.def|),
% |\ifchilddoc| is set to true, |\includeonly| is applied to the child file
% and |\jobname| is set to the main file
% (for proper handling of |.aux| files):
%    \begin{macrocode}
\newcommand{\childdocmain}[1]
{
  \childdocdisable\childdocmain{}
  \if?#1?\else
    \begingroup
      \def\childdoctmp{#1}
      \ifx\childdoctmp\childdocname
        \def\childdoctmp{}
      \else
        \def\childdoctmp
        {
          \childdoctrue
          \includeonly{\childdocname}
          \def\childdocjob{#1}
          \def\jobname{#1}
        }
      \fi
      \expandafter
    \endgroup
    \childdoctmp
  \fi
}
%    \end{macrocode}

% \macro{\childdocof}
% The command |\childdocof| redirects
% compilation to the main file |#1|.
%    \begin{macrocode}
\newcommand{\childdocof}[1]
{
  \childdocdisable
  \childdoctrue
  \includeonly{\childdocname}
  \def\jobname{#1}
  \def\childdocjob{#1}
  \input{#1}
}
%    \end{macrocode}

% \macro{\childdocby}
% The command |\childdocby| ....
%    \begin{macrocode}
\newcommand{\childdocby}[2][]
{
  \childdocdisable
  \childdoctrue
  \childdocmanualtrue
  \if?#1?\else
    \def\jobname{#2}
  \fi
  \def\childdocjob{#2}
  \input{#2}
  \endinput
}
%    \end{macrocode}

% \macro{\childdocforward}
% The command |\childdocforward| redirects
% compilation to the main file or
% (if the optional argument is given) a child file.
% Parameters are set as if the main file
% or a child file starting with |\childdocof| was compiled.
% Then compilation is handed over to the main file:
%    \begin{macrocode}
\newcommand{\childdocforward}[2][]
{
  \begingroup
    \if?#1?
      \def\childdoctmp
      {
        \def\childdocname{#2}
        \def\childdocjob{#2}
        \def\jobname{#2}
        \input{#2}
        \endinput
      }
    \else
      \def\childdoctmp
      {
        \childdocdisable
        \def\childdocname{#2}
        \childdoctrue
        \includeonly{#2}
        \def\childdocjob{#1}
        \def\jobname{#1}
        \input{#1}
        \endinput
      }
    \fi
    \expandafter
  \endgroup
  \childdoctmp
}
%    \end{macrocode}

% \macro{\childdocforwardprefix}
% The command |\childdocforwardprefix| redirects
% compilation to the main or a child file by means of a pattern.
% The prefix |#1| in the current filename is replaced by |#2|
% and the suffix of the current filename is kept
% (it is assumed that the filename does not contain the substring `|~~~|'
% which is used as a delimiter).
% Compilation is handed over to the new file by |\childdocforward|:
%    \begin{macrocode}
\newcommand{\childdocforwardprefix}[3][]
{
  \begingroup
    \def\childdocextract #2##1~~~{\def\childdoctmp{\childdocforward[#1]{#3##1}}}
    \expandafter\childdocextract\childdocname~~~
    \expandafter
  \endgroup
  \childdoctmp
}
%    \end{macrocode}

% \macro{\childdoc}
% The deprecated macro |\childdoc| is a legacy version of |\childdocmain|:
%    \begin{macrocode}
\newcommand{\childdoc}{\childdocmain}
%    \end{macrocode}

% \macro{\childdocredirect}
% The deprecated macro |\childdocredirect| is a legacy version
% of |\childdocforward| and |\childdocforwardprefix|:
%    \begin{macrocode}
\newcommand{\childdocredirect}[2][]
{
  \begingroup
    \if?#1?
      \def\childdoctmp{\childdocforward{#2}}
    \else
      \def\childdoctmp{\childdocforwardprefix{#1}{#2}}
    \fi
    \expandafter
  \endgroup
  \childdoctmp
}
%    \end{macrocode}

%\iffalse
%</package>
%\fi
%
\endinput
|\\
|\childdocforward[|\textit{main}|]{|\textit{dest}|}|\\
\end{tabular}
\end{center}
%
The argument \textit{dest} is the destination file
(without extension).
It should be the main file or one of the child files.
Note that further \textsf{childdoc} directives
such as |\childdocof| and |\childdocforward|
in the indicated file will be processed in this form.
The optional argument \textit{main}
passes on directly to the main file \textit{main}
while pretending to compile the child \textit{dest}.
This form behaves as if \textit{dest}
issues |\childdocof{|\textit{main}|}| right away,
and no further \textsf{childdoc} directives will be processed.

%%%%%%%%%%%%%%%%%%%%%%%%%%%%%%%%%%%%%%%%
\DescribeMacro{\...prefix}
In the alternative form |\childdocforwardprefix|,
%
\begin{center}
\begin{tabular}{l}
|% \iffalse
%
% childdoc.dtx Copyright (C) 2017-2018 Niklas Beisert
%
% This work may be distributed and/or modified under the
% conditions of the LaTeX Project Public License, either version 1.3
% of this license or (at your option) any later version.
% The latest version of this license is in
%   http://www.latex-project.org/lppl.txt
% and version 1.3 or later is part of all distributions of LaTeX
% version 2005/12/01 or later.
%
% This work has the LPPL maintenance status `maintained'.
%
% The Current Maintainer of this work is Niklas Beisert.
%
% This work consists of the files childdoc.dtx and childdoc.ins
% and the derived files childdoc.def and cdocsamp.tex with
% cdocsch1.tex, cdocsch2.tex, cdocsdrf.tex, cdocsfn1.tex, cdocsfn2.tex.
%
%<package>\ifdefined\childdocmain\endinput\fi
%<package>\ProvidesFile{childdoc.def}[2018/12/30 v2.0 child document driver]
%<samplemain>\ProvidesFile{cdocsamp.tex}[2018/12/30 v2.0 sample for childdoc]
%<*driver>
%\ProvidesFile{childdoc.drv}[2018/12/30 v2.0 childdoc reference manual file]
\PassOptionsToClass{10pt,a4paper}{article}
\documentclass{ltxdoc}

\usepackage[margin=35mm]{geometry}
\usepackage{hyperref}
\usepackage{hyperxmp}
\usepackage[usenames]{color}

\hypersetup{colorlinks=true}
\hypersetup{pdfstartview=FitH}
\hypersetup{pdfpagemode=UseNone}
\hypersetup{pdfsource={}}
\hypersetup{pdflang={en-UK}}
\hypersetup{pdfcopyright={Copyright 2017-2018 Niklas Beisert.
  This work may be distributed and/or modified under the
  conditions of the LaTeX Project Public License, either version 1.3
  of this license or (at your option) any later version.}}
\hypersetup{pdflicenseurl={http://www.latex-project.org/lppl.txt}}
\hypersetup{pdfcontactaddress={ETH Zurich, ITP, HIT K,
  Wolfgang-Pauli-Strasse 27}}
\hypersetup{pdfcontactpostcode={8093}}
\hypersetup{pdfcontactcity={Zurich}}
\hypersetup{pdfcontactcountry={Switzerland}}
\hypersetup{pdfcontactemail={nbeisert@itp.phys.ethz.ch}}
\hypersetup{pdfcontacturl={http://people.phys.ethz.ch/\xmptilde nbeisert/}}

\newcommand{\secref}[1]{\hyperref[#1]{section \ref*{#1}}}

\parskip1ex
\parindent0pt
\let\olditemize\itemize
\def\itemize{\olditemize\parskip0pt}

\begin{document}

\title{The \textsf{childdoc} Package}
\hypersetup{pdftitle={The childdoc Package}}
\author{Niklas Beisert\\[2ex]
  Institut f\"ur Theoretische Physik\\
  Eidgen\"ossische Technische Hochschule Z\"urich\\
  Wolfgang-Pauli-Strasse 27, 8093 Z\"urich, Switzerland\\[1ex]
  \href{mailto:nbeisert@itp.phys.ethz.ch}
  {\texttt{nbeisert@itp.phys.ethz.ch}}}
\hypersetup{pdfauthor={Niklas Beisert}}
\hypersetup{pdfsubject={Manual for the LaTeX2e Package childdoc}}
\date{30 December 2018, \textsf{v2.0}}
\maketitle

\begin{abstract}\noindent
\textsf{childdoc} is a \LaTeXe{} package
that enables the direct compilation
of document sections included by |\include|
to individual files.
\end{abstract}

\begingroup
\parskip0ex
\tableofcontents
\endgroup

%%%%%%%%%%%%%%%%%%%%%%%%%%%%%%%%%%%%%%%%%%%%%%%%%%%%%%%%%%%%%%%%%%%%%%%%%%%%%%%%
%%%%%%%%%%%%%%%%%%%%%%%%%%%%%%%%%%%%%%%%%%%%%%%%%%%%%%%%%%%%%%%%%%%%%%%%%%%%%%%%
\section{Introduction}

\LaTeX{} provides a mechanism to structure a large document (such as a book)
into a main file and several child files (containing the chapters)
using the |\include| command.
This mechanism is beneficial for documents
which span hundreds of pages in order to
make the source file(s) more manageable.
Moreover, compilation can be restricted to
selected child files by means of the |\includeonly| command.
The latter feature can be used to reduce the compilation time while editing
(this was significantly more useful in the earlier days of \LaTeX{})
or to generate a smaller document which is easier to navigate.
Another application of |\includeonly| is to generate
documents consisting of selected parts of the complete document.

However, there are a few drawbacks of the plain |\include| mechanism:
\begin{itemize}
\item
The child files cannot be compiled on their own,
they can only be compiled via the main file.
A naive editing environment
(such as a text editor with an option
to have the current file processed by \LaTeX)
may require one to switch to the main file before compiling;
attempting to compile the child file produces errors.
\item
The main file must be modified (each time)
to adjust the |\includeonly| command
to the present needs. This easily leaves the main file in a messy state.
\item
The generated document will always carry the filename
of the main document. This is inconvenient if
several child files are to be compiled and
to be kept for distribution.
\end{itemize}

The present package provides a simple interface
to make child files individually compilable by \LaTeX{}.
Compiling a child file then has the same effect as compiling
the main file with an |\includeonly| command
to select the appropriate child.
Moreover the generated document will carry the name of the child
rather than the main file.
This resolves all three above issues.

This feature is meant to make the editing of books,
thesis documents and lecture notes somewhat more convenient.
However, the package can also be used efficiently for
composing a series of documents (such as exercise sheets)
which are typically distributed individually.
It then assists the author in generating the individual documents
(potentially in different versions)
as well as a document containing the collected series.
Another application is in developing style files
or other kinds of included material
where compilation of the style file could redirect
to a sample or test file.

%%%%%%%%%%%%%%%%%%%%%%%%%%%%%%%%%%%%%%%%%%%%%%%%%%%%%%%%%%%%%%%%%%%%%%%%%%%%%%%%
%%%%%%%%%%%%%%%%%%%%%%%%%%%%%%%%%%%%%%%%%%%%%%%%%%%%%%%%%%%%%%%%%%%%%%%%%%%%%%%%
\section{Usage}

First of all, the package \textsf{childdoc} is \emph{not} a standard
\LaTeXe{} |.sty| style file! Therefore it needs to be invoked in
a non-standard way.

%%%%%%%%%%%%%%%%%%%%%%%%%%%%%%%%%%%%%%%%%%%%%%%%%%%%%%%%%%%%%%%%%%%%%%%%%%%%%%%%
\subsection{Included Files}
\label{sec:include}

%%%%%%%%%%%%%%%%%%%%%%%%%%%%%%%%%%%%%%%%
\DescribeMacro{\childdocmain}
To use the package, add the commands
\begin{center}
\begin{tabular}{l}
|\input{childdoc.def}|\\
|\childdocmain{}|\\
\end{tabular}
\end{center}
at the very top of the main \LaTeX{} file,
in particular \emph{before} the |\documentclass| statement!
The argument of |\childdocmain| should be left empty
(but it must be present).

%%%%%%%%%%%%%%%%%%%%%%%%%%%%%%%%%%%%%%%%
\DescribeMacro{\childdocof}
Furthermore, add the commands
\begin{center}
\begin{tabular}{l}
|\input{childdoc.def}|\\
|\childdocof{|\textit{main}|}|\\
\end{tabular}
\end{center}
at the top of every child file \textit{child}
which is included by |\include{|\textit{child}|}|
from within the main file
(or at least for those files to be compiled individually).
The argument \textit{main} must be the filename of the main file.

There are a couple of
considerations in setting up the main and child documents:

%%%%%%%%%%%%%%%%%%%%%%%%%%%%%%%%%%%%%%%%
\paragraph{Restrictions.}

Please note the following restrictions:
\begin{itemize}
\item
|\childdocmain| must be called with one argument \textit{main}
to ensure compatibility with earlier version of the package.
It must either be empty (|\childdocmain{}|)
or precisely match the filename of the main file in which it is specified.
See \secref{sec:detection} for further information.
\item
The filename \textit{main} must be specified without the |.tex| extension.
\item
The filename \textit{main} is case sensitive
(even in case-insensitive file systems)
due to internal string comparison.
\item
The argument \textit{main} should be fully expanded, it cannot be a macro.
\item
Subdirectories and special characters should be avoided in filenames.
\item
The command |\childdocmain{|\textit{main}|}| must be followed by a whitespace.
It should not be followed immediately by another command
or by a comment mark `|%|'.
This is because the \TeX{} parser reads the token immediately following
the argument of |\childdocmain| and puts it
at the beginning of every child section;
however, a white\-space is ignored.
\end{itemize}

%%%%%%%%%%%%%%%%%%%%%%%%%%%%%%%%%%%%%%%%
\paragraph{Content of Main File.}

It is advisable to place all content in the child files included by |\include|.
Any output contained in the main file will appear in all child documents
unless suppressed manually;
it cannot be suppressed automatically by the |\includeonly| directive
and thus should normally be avoided.
A method to include some content in the main file
by means of conditional processing is described in \secref{sec:conditional}.

%%%%%%%%%%%%%%%%%%%%%%%%%%%%%%%%%%%%%%%%
\paragraph{Page Numbering.}

When only a part of the document is compiled,
the appropriate numbering of pages
(as well as other status parameters)
is determined from the |.aux| files.
The latter contain information from previous passes.
However this information needs to propagate through
all intermediate child documents.
Therefore the page numbering in child documents may well
be inconsistent until the complete document is compiled at least once.

A useful (if unconventional) way to always ensure a consistent
page numbering is to restart the numbering in each child document
and denote the pages by `\textit{child}|.|\textit{page}'
where \textit{child} represents the chapter/section number of the child file.
This can be achieved by the command
|\numberwithin{page}{|\textit{child}|}|
of the \textsf{amsmath} package
where \textit{child} can be |chapter| or |section|
depending on the chosen structuring.
Alternatively, one can modify the macro |\thepage| appropriately
and reset the counter |page| at the start of each child file.

%%%%%%%%%%%%%%%%%%%%%%%%%%%%%%%%%%%%%%%%%%%%%%%%%%%%%%%%%%%%%%%%%%%%%%%%%%%%%%%%
\subsection{Conditional Processing}
\label{sec:conditional}

The package provides a mechanism to compile different versions
of a document. To customise the versions further some conditional processing
can come in handy to distinguish which version is being compiled.
The package provides two macros to describe the compilation context:

%%%%%%%%%%%%%%%%%%%%%%%%%%%%%%%%%%%%%%%%
\DescribeMacro{\ifchilddoc}
The conditional |\ifchilddoc| distinguishes between the compilation of
child documents and the main document:
%
\begin{center}
|\ifchilddoc |\textit{child-code}| |[|\||else |\textit{main-code}]| \||fi|
\end{center}

%%%%%%%%%%%%%%%%%%%%%%%%%%%%%%%%%%%%%%%%
\DescribeMacro{\childdocname}
\DescribeMacro{\childdocjob}
The macro |\childdocname| contains the filename (without extension)
of the main or child file being processed.
Note that |\childdocjob| will always contain the name of the main file.

%%%%%%%%%%%%%%%%%%%%%%%%%%%%%%%%%%%%%%%%
\paragraph{Title Page.}

Conditional processing can be used to include a title or banner page
in the main document when proper precautions are taken.
Importantly, the code in the main file should ensure that the page counter
(as well as other status parameters which are stored in the |.aux| files)
takes the same value after the conditional processing.
Otherwise the page numbers may take divergent values
depending on which part is compiled.

For example, a title page could be declared by:
%
\begin{center}
\begin{tabular}{l}
|\ifchilddoc\||else|\\
|\addtocounter{page}{-1}|\\
\textit{code for title page}\\
|\newpage|\\
|\||fi|
\end{tabular}
\end{center}
%
A banner page for the child documents can be generated by:
%
\begin{center}
\begin{tabular}{l}
|\ifchilddoc|\\
|\addtocounter{page}{-1}|\\
\textit{code for banner page}\\
|\newpage|\\
|\||fi|
\end{tabular}
\end{center}
%
Here one could write a message such as:
\begin{center}
|This is the part \childdocname{} of \childdocjob{}.|
\end{center}

%%%%%%%%%%%%%%%%%%%%%%%%%%%%%%%%%%%%%%%%%%%%%%%%%%%%%%%%%%%%%%%%%%%%%%%%%%%%%%%%
\subsection{Flags}
\label{sec:flags}

The package makes it easy to generate different versions
of the main or child documents.
To this end compilation flags can be defined
and assigned different default values.
They will be particularly useful in conjunction
with the forwarding mechanism described in \secref{sec:forward}.

For example, it may be useful to have a flag |\version|
which can be set to |draft| or |final|.
The document source will contain some conditional code
depending on the value of |\version|.
Suppose further, the flag should default to |final| for the main file
and to |draft| for child files
which is a natural assignment for editing the document.
This is achieved by placing the following code
in the preamble of the main document
(below the |\childdocmain| directive):
%
\begin{center}
\begin{tabular}{l}
|\ifchilddoc|\\
|\providecommand{\version}{draft}|\\
|\||else|\\
|\providecommand{\version}{final}|\\
|\||fi|
\end{tabular}
\end{center}
%
The definition by |\providecommand| makes sure
that previous definitions are not overwritten.
Further statements |\providecommand{\version}{...}|
can thus be added before the above code to override it.

For the main file, one might add a line
(between |\childdocmain| and the above block)
%
\begin{center}
|%\ifchilddoc\||else\providecommand{\version}{draft}\||fi|
\end{center}
%
which can be uncommented to produce a draft version.
Likewise one can add a line to the very top of a child file
(above the |\childdocof{|\textit{main}|}| directive)
%
\begin{center}
|%\providecommand{\version}{final}|
\end{center}
%
which can be uncommented to produce the final version of this child document.

%%%%%%%%%%%%%%%%%%%%%%%%%%%%%%%%%%%%%%%%%%%%%%%%%%%%%%%%%%%%%%%%%%%%%%%%%%%%%%%%
\subsection{Forwarding}
\label{sec:forward}

Different versions of the main or child documents
using compilation flags as described in \secref{sec:flags}
can be (permanently) stored in different files
for convenient compilation, viewing and distribution.
To this end, the package defines a command
to pass on compilation to a different file:

%%%%%%%%%%%%%%%%%%%%%%%%%%%%%%%%%%%%%%%%
\DescribeMacro{\childdocforward}
The command |\childdocforward| redirects processing to
another source file:
%
\begin{center}
\begin{tabular}{l}
|\input{childdoc.def}|\\
|\childdocforward[|\textit{main}|]{|\textit{dest}|}|\\
\end{tabular}
\end{center}
%
The argument \textit{dest} is the destination file
(without extension).
It should be the main file or one of the child files.
Note that further \textsf{childdoc} directives
such as |\childdocof| and |\childdocforward|
in the indicated file will be processed in this form.
The optional argument \textit{main}
passes on directly to the main file \textit{main}
while pretending to compile the child \textit{dest}.
This form behaves as if \textit{dest}
issues |\childdocof{|\textit{main}|}| right away,
and no further \textsf{childdoc} directives will be processed.

%%%%%%%%%%%%%%%%%%%%%%%%%%%%%%%%%%%%%%%%
\DescribeMacro{\...prefix}
In the alternative form |\childdocforwardprefix|,
%
\begin{center}
\begin{tabular}{l}
|\input{childdoc.def}|\\
|\childdocforwardprefix[|\textit{main}|]{|\textit{prefix}|}{|\textit{dest}|}|
\end{tabular}
\end{center}
%
the destination file is determined by a pattern
depending on the current file:
To make this work, the current file must be called
`{\textit{prefix}\hspace{0.2em}\textit{suffix}}'
with \textit{prefix} matching precisely the argument.
Processing is then passed on to the file
`{\textit{dest}\hspace{0.2em}\textit{suffix}}'.
Surely, the same effect is achieved by
directly specifying the
argument `{\textit{dest}\hspace{0.2em}\textit{suffix}}'
in the first form.
However, that requires to set up a different file
for each child. With the alternative form of the command
all these files can have exactly the same content
which simplifies setting them up and maintaining them.

For example, the following file |draft.tex|
with a compilation flag |\version| as described in \secref{sec:flags}
compiles the main document as a draft:
%
\begin{center}
\begin{tabular}{l}
|\def\version{draft}|\\
|\input{childdoc.def}|\\
|\childdocforward{|\textit{main}|}|
\end{tabular}
\end{center}
%
Likewise, the following files |final|\textit{nn}|.tex|
compile the final version of the child document
|child|\textit{nn}|.tex|:
%
\begin{center}
\begin{tabular}{l}
|\def\version{final}|\\
|\input{childdoc.def}|\\
|\childdocforwardprefix{final}{child}|
\end{tabular}
\end{center}
%

Note that when several versions of a main file and/or of each child file
are to be generated, it may be convenient to set up a |Makefile| or
shell script to automatise the process.

%%%%%%%%%%%%%%%%%%%%%%%%%%%%%%%%%%%%%%%%%%%%%%%%%%%%%%%%%%%%%%%%%%%%%%%%%%%%%%%%
\subsection{Command Line Processing}
\label{sec:commandline}

The effect of redirection files can also be achieved by invoking
the \LaTeX{} compiler with a more elaborate command line.
Most conveniently this should be done as part
of a shell script or a |Makefile|.

When using \textsf{childdoc} in the main file, the following
command lines effectively perform a redirection
(note that depending on the shell being used,
backslashes may have to be doubled: `|\|' $\to$ `|\\|'):
%
\begin{center}
|... -jobname "|\textit{target}|" |\\|"|[\textit{flags}]%
|\input{childdoc.def}\childdocforward[|\textit{main}|]{|\textit{dest}|}"|
\end{center}
%
Here \textit{target} is the name of the output file,
\textit{main} is the name of the main file
and \textit{dest} is the name of the main or child file to be processed
(all filenames without extensions).
The optional argument \textit{main} can be omitted
if \textit{main} matches \textit{dest}.
Optionally, compilation \textit{flags} can be defined via |\def| commands.
This command line makes the \TeX{} engine believe
it is compiling the file \textit{target}
whose content is specified as the latter parameter.
The provided code then forwards the processing to
\textit{main} or \textit{dest} as described in \secref{sec:forward}.

%%%%%%%%%%%%%%%%%%%%%%%%%%%%%%%%%%%%%%%%%%%%%%%%%%%%%%%%%%%%%%%%%%%%%%%%%%%%%%%%
\subsection{Include by Input}
\label{sec:input}

Including child documents by |\include| has some restrictions by design.
Most notably, the content of a child document always occupies
its own set of pages; pages cannot be shared between child documents.
Usually, this behaviour makes perfect sense
because each child document contain an essential part of the document.
However, in some situations it may be desirable to compose
a document from a collection of parts
without having mandatory page breaks between then.
For this case, the package
provides a mechanism to include parts
by |\input| which can also be processed individually.
However, by construction this mechanism
requires manual handling of the content to be output.

%%%%%%%%%%%%%%%%%%%%%%%%%%%%%%%%%%%%%%%%
\DescribeMacro{\ifchilddocmanual}
The main file should be prepared as usual, see \secref{sec:include}.
However, the document body must make a distinction
between processing of an individual part and of the main document, e.g.:
%
\begin{center}
\begin{tabular}{l}
|\ifchilddocmanual|\\
|\input{\childdocname}|\\
|\||else|\\
\textit{document body with }|\input{|\textit{part}|}|\\
|\||fi|
\end{tabular}
\end{center}
%
The conditional |\ifchilddocmanual| is true whenever
a part to be included by |\input| is being compiled,
and the name of the part is stored in |\childdocname|.

%%%%%%%%%%%%%%%%%%%%%%%%%%%%%%%%%%%%%%%%
\DescribeMacro{\childdocby}
Each part to be included by |\input| should start with:
%
\begin{center}
\begin{tabular}{l}
|\input{childdoc.def}|\\
|\childdocby{|\textit{main}|}|\\
\end{tabular}
\end{center}
%
The directive |\childdocby| is similar to |\childdocof|
described in \secref{sec:include},
but the subsequent selection of content must be done manually.
To that end, both |\ifchilddoc| and |\ifchilddocmanual|
will be true upon processing of a part,
and the name of the part is stored in |\childdocname|.
Note that |\jobname| will be set to the filename of the current part
so that each part receives an individual |.aux| file
that does not interfere with the |.aux| file(s) of the main document.
This behaviour can be altered by the alternative form
|\childdocby[*]{|\textit{main}|}| (with a non-empty optional argument)
which uses the |.aux| file of the main document
by setting |\jobname| to \textit{main}.

%%%%%%%%%%%%%%%%%%%%%%%%%%%%%%%%%%%%%%%%%%%%%%%%%%%%%%%%%%%%%%%%%%%%%%%%%%%%%%%%
\subsection{Driver Development}
\label{sec:driver}

The \textsf{childdoc} mechanism can also be use for the development
of definition files such as \LaTeX{} styles or classes.
This case differs from the above setup with multiple parts
included by |\include| in that no |\includeonly| should be invoked.
This can be achieved by starting the include file
(before |\ProvidesPackage|) with:
%
\begin{center}
\begin{tabular}{l}
|\input{childdoc.def}|\\
|\childdocforward{|\textit{main}|}|\\
\end{tabular}
\end{center}
%
or alternatively with:
%
\begin{center}
\begin{tabular}{l}
|\input{childdoc.def}|\\
|\childdocby{|\textit{main}|}|\\
\end{tabular}
\end{center}
%
Both forms have slightly different effects as described above.
The main file is prepared as usual, see \secref{sec:include}.

%%%%%%%%%%%%%%%%%%%%%%%%%%%%%%%%%%%%%%%%%%%%%%%%%%%%%%%%%%%%%%%%%%%%%%%%%%%%%%%%
\subsection{Legacy Detection}
\label{sec:detection}

The directive |\childdocmain| in the main file can detect
whether the complete document or merely a child is to be compiled
even without using the directive |\childdocof|.
This method is deprecated because it is less robust
and there is no compelling reason to use it;
it is merely provided for backward compatibility
and it may be removed in future versions.

If the detection mechanism is to be used,
it is mandatory to correctly specify
the filename of the main file as the argument of |\childdocmain|:
%
\begin{center}
\begin{tabular}{l}
|\input{childdoc.def}|\\
|\childdocmain{|\textit{main}|}|\\
\end{tabular}
\end{center}
%
If |\jobname| does not match the argument \textit{main} of |\childdocmain|,
it is assumed that |\jobname| points to the child file to be compiled.
When using |\childdocmain| with the main file specified as argument,
it suffices to start a child file
with just |\input{|\textit{main}|}|
without loading of the package and using |\childdocof|.
If instead all processing is done
with the appropriate \textsf{childdoc} directives,
the argument of \textit{main} of |\childdocmain| can be empty.

An alternative version of the command line processing described
in \secref{sec:commandline} using the detection mechanism reads:
%
\begin{center}
|... -jobname "|\textit{target}|" "|[\textit{flags}]%
[|\def\jobname{|\textit{dest}|}|]|\input{|\textit{main}|}"|
\end{center}

%%%%%%%%%%%%%%%%%%%%%%%%%%%%%%%%%%%%%%%%%%%%%%%%%%%%%%%%%%%%%%%%%%%%%%%%%%%%%%%%
\subsection{Manual Code}
\label{sec:manual}

In case one cannot be certain whether the definitions file |childdoc.def|
is installed on the target \TeX{} distribution
and one prefers not to ship it,
it is conceivable to paste a few relevant commands into the sources.

To that end, drop all statements |\input{childdoc.def}|
and perform the replacements as outlined below.
Instead of |\childdocmain{|\textit{main}|}| add the following code
to the top of the main file:
%
\begin{center}
\begin{tabular}{l}
|\||ifdefined\childdocname\endinput\||fi\newif\ifchilddoc|\\
|\edef\childdocname{\scantokens\expandafter{\jobname\noexpand}}|\\
|\def\childdocmain{|\textit{main}|}\||ifx\childdocmain\childdocname\||else|\\
|\childdoctrue\includeonly{\childdocname}\let\jobname\childdocmain\||fi|\\
\end{tabular}
\end{center}
%
Instead of |\childdocof{|\textit{main}|}| just include the main file
at the top of each child file:
%
\begin{center}
|\input{|\textit{main}|}|
\end{center}
%
A simple redirection |\childdocforward{|\textit{dest}|}| is achieved by:
%
\begin{center}
|\def\jobname{|\textit{dest}|}\input{\jobname}|
\end{center}
%
The redirection with prefix
|\childdocforwardprefix[|\textit{prefix}|]{|\textit{dest}|}|
is accomplished by:
%
\begin{center}
\begin{tabular}{l}
|{\edef\jobname{\scantokens\expandafter{\jobname\noexpand}}|\\
|\def\redirectjob |\textit{prefix}|#1~~~{\gdef\jobname{|\textit{dest}|#1}}|\\
|\expandafter\redirectjob\jobname~~~}\input{\jobname}|
\end{tabular}
\end{center}

In an alternative approach,
child documents can be compiled by a specific command line
without additional code or specific definitions:
%
\begin{center}
|... -jobname "|\textit{target}|" "|[\textit{flags}]%
|\includeonly{|\textit{dest}|}\input{|\textit{main}|}"|
\end{center}
%

%%%%%%%%%%%%%%%%%%%%%%%%%%%%%%%%%%%%%%%%%%%%%%%%%%%%%%%%%%%%%%%%%%%%%%%%%%%%%%%%
%%%%%%%%%%%%%%%%%%%%%%%%%%%%%%%%%%%%%%%%%%%%%%%%%%%%%%%%%%%%%%%%%%%%%%%%%%%%%%%%
\section{Information}

%%%%%%%%%%%%%%%%%%%%%%%%%%%%%%%%%%%%%%%%%%%%%%%%%%%%%%%%%%%%%%%%%%%%%%%%%%%%%%%%
\subsection{Copyright}

Copyright \copyright{} 2017--2018 Niklas Beisert

This work may be distributed and/or modified under the
conditions of the \LaTeX{} Project Public License, either version 1.3
of this license or (at your option) any later version.
The latest version of this license is in
  \url{http://www.latex-project.org/lppl.txt}
and version 1.3 or later is part of all distributions of \LaTeX{}
version 2005/12/01 or later.

This work has the LPPL maintenance status `maintained'.

The Current Maintainer of this work is Niklas Beisert.

This work consists of the files |README.txt|, |childdoc.ins| and |childdoc.dtx|
as well as the derived files |childdoc.def|, |cdocsamp.tex|
with |cdocsch1.tex|, |cdocsch2.tex|, |cdocspt3.tex|, |cdocspt4.tex|,
|cdocsdrf.tex|, |cdocsfn1.tex|, |cdocsfn2.tex|
as well as |childdoc.pdf|.

%%%%%%%%%%%%%%%%%%%%%%%%%%%%%%%%%%%%%%%%%%%%%%%%%%%%%%%%%%%%%%%%%%%%%%%%%%%%%%%%
\subsection{Files and Installation}

The package consists of the files:
%
\begin{center}
\begin{tabular}{ll}
    |README.txt|   & readme file \\
    |childdoc.ins| & installation file \\
    |childdoc.dtx| & source file \\
    |childdoc.def| & definition file \\
    |cdocsamp.tex| & sample main file \\
    |cdocsch1.tex| & sample include file \\
    |cdocsch2.tex| & sample include file \\
    |cdocspt3.tex| & sample part file \\
    |cdocspt4.tex| & sample part file \\
    |cdocsdrf.tex| & sample redirection file \\
    |cdocsfn1.tex| & sample redirection file \\
    |cdocsfn2.tex| & sample redirection file \\
    |childdoc.pdf| & manual
\end{tabular}
\end{center}
%
The distribution consists of the files
|README.txt|, |childdoc.ins| and |childdoc.dtx|.
%
\begin{itemize}
\item
Run (pdf)\LaTeX{} on |childdoc.dtx|
to compile the manual |childdoc.pdf| (this file).
\item
Run \LaTeX{} on |childdoc.ins| to create the definitions file |childdoc.def|
and the sample |cdocsamp.tex| with include files
|cdocsch1.tex|, |cdocsch2.tex|, |cdocspt3.tex|, |cdocspt4.tex|,
|cdocsdrf.tex|, |cdocsfn1.tex|, |cdocsfn2.tex|.
Then copy the file |childdoc.def| to an appropriate directory of your \LaTeX{}
distribution, e.g.\ \textit{texmf-root}|/tex/latex/childdoc|.
\end{itemize}

%%%%%%%%%%%%%%%%%%%%%%%%%%%%%%%%%%%%%%%%%%%%%%%%%%%%%%%%%%%%%%%%%%%%%%%%%%%%%%%%
\subsection{Related CTAN Packages}

There are several other packages which offer a similar functionality:
%
\begin{itemize}
\item
The packages
\href{http://ctan.org/pkg/docmute}{\textsf{docmute}},
\href{http://ctan.org/pkg/includex}{\textsf{includex}} and
\href{http://ctan.org/pkg/standalone}{\textsf{standalone}}
provide commands to include only the document body of
a child file thus allowing both files to be compiled individually.
\item
The packages \href{http://ctan.org/pkg/subdocs}{\textsf{subdocs}}
and \href{http://ctan.org/pkg/subfiles}{\textsf{subfiles}}
provide structures in which the main and child documents can be
encapsulated and allowing them to be compiled individually.
The inclusion mechanism is different from the conventional |\include|.
\item
The package \href{http://ctan.org/pkg/combine}{\textsf{combine}}
is an elaborate solution to combine several documents into one.
\end{itemize}
%
See also the CTAN topic \href{http://ctan.org/topic/subdocs}{\textsf{subdocs}}
for further related packages.
The present package differs from the above solutions in that
a document structure constructed with the conventional |\include| mechanism
just needs two extra commands at the top of every file
such that all constituent files can be compiled individually.

%%%%%%%%%%%%%%%%%%%%%%%%%%%%%%%%%%%%%%%%%%%%%%%%%%%%%%%%%%%%%%%%%%%%%%%%%%%%%%%%
%\subsection{Feature Suggestions}
%
%The following is a list of features which may be useful for future
%versions of this package:
%%
%\begin{itemize}
%\item
%\ldots
%\end{itemize}

%%%%%%%%%%%%%%%%%%%%%%%%%%%%%%%%%%%%%%%%%%%%%%%%%%%%%%%%%%%%%%%%%%%%%%%%%%%%%%%%
\subsection{Revision History}

%%%%%%%%%%%%%%%%%%%%%%%%%%%%%%%%%%%%%%%%
\paragraph{v2.0:} 2018/12/30

\begin{itemize}
\item
immediate forward processing
\item
added |\childdocby| mechanism
\item
manual restructured
\end{itemize}

%%%%%%%%%%%%%%%%%%%%%%%%%%%%%%%%%%%%%%%%
\paragraph{v1.6:} 2018/01/17

\begin{itemize}
\item
application for development of include files
\item
corrections to manual
\end{itemize}

%%%%%%%%%%%%%%%%%%%%%%%%%%%%%%%%%%%%%%%%
\paragraph{v1.5:} 2017/05/21

\begin{itemize}
\item
more complete structuring introduced
\item
|\childdocof| introduced
\item
|\childdoc| renamed to |\childdocmain|
\item
|\childredirect| renamed to |\childdocforward| and |\childdocforwardprefix|
and functionality expanded
\end{itemize}

%%%%%%%%%%%%%%%%%%%%%%%%%%%%%%%%%%%%%%%%
\paragraph{v1.0:} 2017/04/27

\begin{itemize}
\item
manual and install package
\item
first version published on CTAN
\end{itemize}

%%%%%%%%%%%%%%%%%%%%%%%%%%%%%%%%%%%%%%%%
\paragraph{v0.6:} 2017/04/26

\begin{itemize}
\item
redirection mechanism added
\end{itemize}

%%%%%%%%%%%%%%%%%%%%%%%%%%%%%%%%%%%%%%%%
\paragraph{v0.5:} 2017/04/26

\begin{itemize}
\item
functionality in definition file
\end{itemize}


%%%%%%%%%%%%%%%%%%%%%%%%%%%%%%%%%%%%%%%%%%%%%%%%%%%%%%%%%%%%%%%%%%%%%%%%%%%%%%%%
%%%%%%%%%%%%%%%%%%%%%%%%%%%%%%%%%%%%%%%%%%%%%%%%%%%%%%%%%%%%%%%%%%%%%%%%%%%%%%%%
%%%%%%%%%%%%%%%%%%%%%%%%%%%%%%%%%%%%%%%%%%%%%%%%%%%%%%%%%%%%%%%%%%%%%%%%%%%%%%%%
\appendix

\settowidth\MacroIndent{\rmfamily\scriptsize 000\ }

 \DocInput{childdoc.dtx}

\end{document}
%</driver>
% \fi
%
% %%%%%%%%%%%%%%%%%%%%%%%%%%%%%%%%%%%%%%%%%%%%%%%%%%%%%%%%%%%%%%%%%%%%%%%%%%%%%%
% %%%%%%%%%%%%%%%%%%%%%%%%%%%%%%%%%%%%%%%%%%%%%%%%%%%%%%%%%%%%%%%%%%%%%%%%%%%%%%
% \section{Sample}
%\iffalse
%<*samplemain>
%\fi
%
% The following presents a sample document
% with two chapters, two parts, a title page,
% a compile flag as well as three forwarding files to set the flag.
% It consists of eight |.tex| files:
% \begin{center}
% \begin{tabular}{ll}
% |cdocsamp.tex|&main file\\
% |cdocsch1.tex|&include file for chapter 1\\
% |cdocsch2.tex|&include file for chapter 2\\
% |cdocspt3.tex|&include file for part 3\\
% |cdocspt4.tex|&include file for part 4\\
% |cdocsdrf.tex|&forwarding file for main file in draft mode\\
% |cdocsfi1.tex|&forwarding file for final version of chapter 1\\
% |cdocsfi2.tex|&forwarding file for final version of chapter 2\\
% \end{tabular}
% \end{center}
% Each of the eight files can be compiled directly by the \LaTeX{} compiler.
%
% %%%%%%%%%%%%%%%%%%%%%%%%%%%%%%%%%%%%%%
% \paragraph{Main File.}
%
% The main file is called |cdocsamp.tex|.
%
% Load the \textsf{childdoc} definitions and
% declare the filename for the main document:
%    \begin{macrocode}
\input{childdoc.def}
\childdocmain{}
%    \end{macrocode}

% Optional override for |\version| flag:
%    \begin{macrocode}
%%\ifchilddoc\else\providecommand{\version}{draft}\fi
%    \end{macrocode}

% Define the default values for the |\version| flag
% (|final| for the main file and |draft| for childs):
%    \begin{macrocode}
\ifchilddoc
\providecommand{\version}{draft}
\else
\providecommand{\version}{final}
\fi
%    \end{macrocode}

% Load the standard document class:
%    \begin{macrocode}
\documentclass[12pt]{article}
%    \end{macrocode}

% Start the document body:
%    \begin{macrocode}
\begin{document}
%    \end{macrocode}

% Declare a title page.
% Print title, part of document being processed and version flag:
%    \begin{macrocode}
\addtocounter{page}{-1}
\begin{center}
{\LARGE\bfseries{}childdoc example\par}
\vspace{1cm}
\ifchilddoc
\ifchilddocmanual part\else chapter\fi:
`\childdocname' of `\childdocjob'\par
\else
main document: `\childdocjob'\par
\fi
version: \version\par
\end{center}
\newpage
%    \end{macrocode}

% Manually include selected file,
% otherwise process as usual:
%    \begin{macrocode}
\ifchilddocmanual
\section*{part `\childdocname'}
\input{\childdocname}
\else
%    \end{macrocode}

% Include the two chapters:
%    \begin{macrocode}
\include{cdocsch1}
\include{cdocsch2}
%    \end{macrocode}

% Include the two parts unless only chapters should be displayed:
%    \begin{macrocode}
\ifchilddoc\else
\section{part three}
\input{cdocspt3}
\section{part four}
\input{cdocspt4}
\fi
%    \end{macrocode}

% Process as usual until here:
%    \begin{macrocode}
\fi
%    \end{macrocode}

% End of document body:
%    \begin{macrocode}
\end{document}
%    \end{macrocode}
%\iffalse
%</samplemain>
%\fi
%
% %%%%%%%%%%%%%%%%%%%%%%%%%%%%%%%%%%%%%%
% \paragraph{Chapter Include Files.}
%
% The include files are called |cdocsch1.tex| and |cdocsch2.tex|.
%
%\iffalse
%<*samplechap1|samplechap2>
%\fi

% Optional override for |\version| flag:
%    \begin{macrocode}
%%\providecommand{\version}{final}
%    \end{macrocode}

% Include the main document:
%    \begin{macrocode}
\input{childdoc.def}
\childdocof{cdocsamp}
%    \end{macrocode}

%\iffalse
%</samplechap1|samplechap2>
%\fi
%
%\iffalse
%<*samplechap1>
%\fi
% Some text for chapter 1:
%    \begin{macrocode}
\section{one}
some text in chapter one
%    \end{macrocode}

%\iffalse
%</samplechap1>
%\fi
% Some text for chapter 2:
%\iffalse
%<*samplechap2>
%\fi
%    \begin{macrocode}
\section{two}
more text in chapter two
%    \end{macrocode}

%\iffalse
%</samplechap2>
%\fi
%
% %%%%%%%%%%%%%%%%%%%%%%%%%%%%%%%%%%%%%%
% \paragraph{Part Include Files.}
%
% The include files are called |cdocspt3.tex| and |cdocspt4.tex|.
%
%\iffalse
%<*samplepart3|samplepart4>
%\fi

% Optional override for |\version| flag:
%    \begin{macrocode}
%%\providecommand{\version}{final}
%    \end{macrocode}

% Include the main document:
%    \begin{macrocode}
\input{childdoc.def}
\childdocby{cdocsamp}
%    \end{macrocode}

%\iffalse
%</samplepart3|samplepart4>
%\fi
%
%\iffalse
%<*samplepart3>
%\fi
% Some text for part 3:
%    \begin{macrocode}
some text in part three
%    \end{macrocode}

%\iffalse
%</samplepart3>
%\fi
% Some text for part 4:
%\iffalse
%<*samplepart4>
%\fi
%    \begin{macrocode}
more text in part four
%    \end{macrocode}

%\iffalse
%</samplepart4>
%\fi
%
% %%%%%%%%%%%%%%%%%%%%%%%%%%%%%%%%%%%%%%
% \paragraph{Forwarding for a Complete Draft.}
%
% The following forwarding file |cdocsdrf.tex|
% compiles the main document in draft mode:
%\iffalse
%<*sampledraft>
%\fi
%    \begin{macrocode}
\def\version{draft}
\input{childdoc.def}
\childdocforward{cdocsamp}
%    \end{macrocode}

%\iffalse
%</sampledraft>
%\fi
%
% %%%%%%%%%%%%%%%%%%%%%%%%%%%%%%%%%%%%%%
% \paragraph{Forwarding for Final Version of the Chapters.}
%
% The following forwarding files |cdocsfn1.tex| and |cdocsfn2.tex|
% (with identical content)
% compile the final versions of the child documents
% |cdocsch1.tex| and |cdocsch2.tex|, respectively:
%\iffalse
%<*samplefinal>
%\fi
%    \begin{macrocode}
\def\version{final}
\input{childdoc.def}
\childdocforwardprefix[cdocsamp]{cdocsfn}{cdocsch}
%    \end{macrocode}

%\iffalse
%</samplefinal>
%\fi
%
% %%%%%%%%%%%%%%%%%%%%%%%%%%%%%%%%%%%%%%
% \paragraph{Command Line Processing.}
%
% The following three command lines generate the output files
% |cdocscld|, |cdocscl1| and |cdocscl2|
% which should be identical to
% |cdocsdrf|, |cdocsch1| and |cdocsfn2|, respectively:
% \begin{center}
% \begin{tabular}{l}
% |latex -jobname cdocscld \|\\
% |  "\def\version{draft}\input{childdoc.def}\childdocforward{cdocsamp}"|\\
% |latex -jobname cdocscl1 \|\\
% |  "\input{childdoc.def}\childdocforward[cdocsamp]{cdocsch1}"|\\
% |latex -jobname cdocscl2 \|\\
% |  "\def\version{final}\input{childdoc.def}\childdocforward{cdocsch2}"|
% \end{tabular}
% \end{center}
% Note that the trailing backslash on each first line
% merely continues the input to the second line
% (for convenient cut ant paste).
% Furthermore, the command |latex| can be replaced by any
% of its alternative versions such as |pdflatex|.
%
% %%%%%%%%%%%%%%%%%%%%%%%%%%%%%%%%%%%%%%%%%%%%%%%%%%%%%%%%%%%%%%%%%%%%%%%%%%%%%%
% %%%%%%%%%%%%%%%%%%%%%%%%%%%%%%%%%%%%%%%%%%%%%%%%%%%%%%%%%%%%%%%%%%%%%%%%%%%%%%
% \section{Implementation}
%\iffalse
%<*package>
%\fi
%
% This section describes the definitions file |childdoc.def|.

% The definitions cannot be loaded using |\usepackage| or |\RequirePackage|
% which has a mechanism to prevent loading a style file more than once.
% When loading the definitions by means of |\input|
% multiple instances have to be prevented manually:
%\iffalse
%This code needs to be before the `\ProvidesFile' directive
%which is defined at the beginning of this file.
%Therefore it is also placed there and commented out here.
%</package>
%<*discard>
%\fi
%    \begin{macrocode}
\ifdefined\childdocmain\endinput\fi
%    \end{macrocode}
%\iffalse
%</discard>
%<*package>
%\fi
%
% \macro{\ifchilddoc}
% \macro{\ifchilddocmanual}
% The conditional |\ifchilddoc| tells whether a
% child (true) or main (false) document is being compiled.
% The conditional |\ifchilddocmanual| tells whether
% the |\includeonly| mechanism is used (false) or
% the selection of child files must be performed manually (true).
% The definitions initialise to false:
%    \begin{macrocode}
\newif\ifchilddoc
\newif\ifchilddocmanual
%    \end{macrocode}

% \macro{\childdocname}
% \macro{\childdocjob}
% The macro |\childdocname| stores the name of the main document
% to be compiled. The macro |\childdocjob| stores the name of
% the document on which the \LaTeX{} compiler was originally invoked.
% The content of |\jobname| cannot be compared
% to filenames specified in the source due to different catcodes.
% The following code rescans |\jobname|, stores the result
% in |\childdocname| and saves a copy in |\childdocjob|:
%    \begin{macrocode}
\edef\childdocname{\scantokens\expandafter{\jobname\noexpand}}
\let\childdocjob\childdocname
%    \end{macrocode}

% \macro{\childdocdisable}
% The macro |\childdocdisable| prevents the main file
% from being processed more than once.
% At this stage, the main document command |\childdocmain|
% is assumed to be called once again where it should do nothing.
% Any subsequent call to it should prevent
% a secondary processing of the main document
% It overwrites the forwarding commands
% |\childdocof| and |\childdocforward|
% with empty macros to prevent further inclusions of the main document:
%    \begin{macrocode}
\newcommand{\childdocdisable}
{
  \renewcommand{\childdocmain}[1]{\renewcommand{\childdocmain}[1]{\endinput}}
  \renewcommand{\childdocof}[1]{}
  \renewcommand{\childdocby}[2][]{}
  \renewcommand{\childdocforward}[2][]{}
  \renewcommand{\childdocdisable}{}
}
%    \end{macrocode}

% \macro{\childdocmain}
% The macro |\childdocmain| is to be called at the top of the main file
% with nothing or the main filename (without extension) as argument.
% First, it breaks loops.
% If the argument is not empty and does not match |\childdocname|
% (which is set by the first inclusion of |childdoc.def|),
% |\ifchilddoc| is set to true, |\includeonly| is applied to the child file
% and |\jobname| is set to the main file
% (for proper handling of |.aux| files):
%    \begin{macrocode}
\newcommand{\childdocmain}[1]
{
  \childdocdisable\childdocmain{}
  \if?#1?\else
    \begingroup
      \def\childdoctmp{#1}
      \ifx\childdoctmp\childdocname
        \def\childdoctmp{}
      \else
        \def\childdoctmp
        {
          \childdoctrue
          \includeonly{\childdocname}
          \def\childdocjob{#1}
          \def\jobname{#1}
        }
      \fi
      \expandafter
    \endgroup
    \childdoctmp
  \fi
}
%    \end{macrocode}

% \macro{\childdocof}
% The command |\childdocof| redirects
% compilation to the main file |#1|.
%    \begin{macrocode}
\newcommand{\childdocof}[1]
{
  \childdocdisable
  \childdoctrue
  \includeonly{\childdocname}
  \def\jobname{#1}
  \def\childdocjob{#1}
  \input{#1}
}
%    \end{macrocode}

% \macro{\childdocby}
% The command |\childdocby| ....
%    \begin{macrocode}
\newcommand{\childdocby}[2][]
{
  \childdocdisable
  \childdoctrue
  \childdocmanualtrue
  \if?#1?\else
    \def\jobname{#2}
  \fi
  \def\childdocjob{#2}
  \input{#2}
  \endinput
}
%    \end{macrocode}

% \macro{\childdocforward}
% The command |\childdocforward| redirects
% compilation to the main file or
% (if the optional argument is given) a child file.
% Parameters are set as if the main file
% or a child file starting with |\childdocof| was compiled.
% Then compilation is handed over to the main file:
%    \begin{macrocode}
\newcommand{\childdocforward}[2][]
{
  \begingroup
    \if?#1?
      \def\childdoctmp
      {
        \def\childdocname{#2}
        \def\childdocjob{#2}
        \def\jobname{#2}
        \input{#2}
        \endinput
      }
    \else
      \def\childdoctmp
      {
        \childdocdisable
        \def\childdocname{#2}
        \childdoctrue
        \includeonly{#2}
        \def\childdocjob{#1}
        \def\jobname{#1}
        \input{#1}
        \endinput
      }
    \fi
    \expandafter
  \endgroup
  \childdoctmp
}
%    \end{macrocode}

% \macro{\childdocforwardprefix}
% The command |\childdocforwardprefix| redirects
% compilation to the main or a child file by means of a pattern.
% The prefix |#1| in the current filename is replaced by |#2|
% and the suffix of the current filename is kept
% (it is assumed that the filename does not contain the substring `|~~~|'
% which is used as a delimiter).
% Compilation is handed over to the new file by |\childdocforward|:
%    \begin{macrocode}
\newcommand{\childdocforwardprefix}[3][]
{
  \begingroup
    \def\childdocextract #2##1~~~{\def\childdoctmp{\childdocforward[#1]{#3##1}}}
    \expandafter\childdocextract\childdocname~~~
    \expandafter
  \endgroup
  \childdoctmp
}
%    \end{macrocode}

% \macro{\childdoc}
% The deprecated macro |\childdoc| is a legacy version of |\childdocmain|:
%    \begin{macrocode}
\newcommand{\childdoc}{\childdocmain}
%    \end{macrocode}

% \macro{\childdocredirect}
% The deprecated macro |\childdocredirect| is a legacy version
% of |\childdocforward| and |\childdocforwardprefix|:
%    \begin{macrocode}
\newcommand{\childdocredirect}[2][]
{
  \begingroup
    \if?#1?
      \def\childdoctmp{\childdocforward{#2}}
    \else
      \def\childdoctmp{\childdocforwardprefix{#1}{#2}}
    \fi
    \expandafter
  \endgroup
  \childdoctmp
}
%    \end{macrocode}

%\iffalse
%</package>
%\fi
%
\endinput
|\\
|\childdocforwardprefix[|\textit{main}|]{|\textit{prefix}|}{|\textit{dest}|}|
\end{tabular}
\end{center}
%
the destination file is determined by a pattern
depending on the current file:
To make this work, the current file must be called
`{\textit{prefix}\hspace{0.2em}\textit{suffix}}'
with \textit{prefix} matching precisely the argument.
Processing is then passed on to the file
`{\textit{dest}\hspace{0.2em}\textit{suffix}}'.
Surely, the same effect is achieved by
directly specifying the
argument `{\textit{dest}\hspace{0.2em}\textit{suffix}}'
in the first form.
However, that requires to set up a different file
for each child. With the alternative form of the command
all these files can have exactly the same content
which simplifies setting them up and maintaining them.

For example, the following file |draft.tex|
with a compilation flag |\version| as described in \secref{sec:flags}
compiles the main document as a draft:
%
\begin{center}
\begin{tabular}{l}
|\def\version{draft}|\\
|% \iffalse
%
% childdoc.dtx Copyright (C) 2017-2018 Niklas Beisert
%
% This work may be distributed and/or modified under the
% conditions of the LaTeX Project Public License, either version 1.3
% of this license or (at your option) any later version.
% The latest version of this license is in
%   http://www.latex-project.org/lppl.txt
% and version 1.3 or later is part of all distributions of LaTeX
% version 2005/12/01 or later.
%
% This work has the LPPL maintenance status `maintained'.
%
% The Current Maintainer of this work is Niklas Beisert.
%
% This work consists of the files childdoc.dtx and childdoc.ins
% and the derived files childdoc.def and cdocsamp.tex with
% cdocsch1.tex, cdocsch2.tex, cdocsdrf.tex, cdocsfn1.tex, cdocsfn2.tex.
%
%<package>\ifdefined\childdocmain\endinput\fi
%<package>\ProvidesFile{childdoc.def}[2018/12/30 v2.0 child document driver]
%<samplemain>\ProvidesFile{cdocsamp.tex}[2018/12/30 v2.0 sample for childdoc]
%<*driver>
%\ProvidesFile{childdoc.drv}[2018/12/30 v2.0 childdoc reference manual file]
\PassOptionsToClass{10pt,a4paper}{article}
\documentclass{ltxdoc}

\usepackage[margin=35mm]{geometry}
\usepackage{hyperref}
\usepackage{hyperxmp}
\usepackage[usenames]{color}

\hypersetup{colorlinks=true}
\hypersetup{pdfstartview=FitH}
\hypersetup{pdfpagemode=UseNone}
\hypersetup{pdfsource={}}
\hypersetup{pdflang={en-UK}}
\hypersetup{pdfcopyright={Copyright 2017-2018 Niklas Beisert.
  This work may be distributed and/or modified under the
  conditions of the LaTeX Project Public License, either version 1.3
  of this license or (at your option) any later version.}}
\hypersetup{pdflicenseurl={http://www.latex-project.org/lppl.txt}}
\hypersetup{pdfcontactaddress={ETH Zurich, ITP, HIT K,
  Wolfgang-Pauli-Strasse 27}}
\hypersetup{pdfcontactpostcode={8093}}
\hypersetup{pdfcontactcity={Zurich}}
\hypersetup{pdfcontactcountry={Switzerland}}
\hypersetup{pdfcontactemail={nbeisert@itp.phys.ethz.ch}}
\hypersetup{pdfcontacturl={http://people.phys.ethz.ch/\xmptilde nbeisert/}}

\newcommand{\secref}[1]{\hyperref[#1]{section \ref*{#1}}}

\parskip1ex
\parindent0pt
\let\olditemize\itemize
\def\itemize{\olditemize\parskip0pt}

\begin{document}

\title{The \textsf{childdoc} Package}
\hypersetup{pdftitle={The childdoc Package}}
\author{Niklas Beisert\\[2ex]
  Institut f\"ur Theoretische Physik\\
  Eidgen\"ossische Technische Hochschule Z\"urich\\
  Wolfgang-Pauli-Strasse 27, 8093 Z\"urich, Switzerland\\[1ex]
  \href{mailto:nbeisert@itp.phys.ethz.ch}
  {\texttt{nbeisert@itp.phys.ethz.ch}}}
\hypersetup{pdfauthor={Niklas Beisert}}
\hypersetup{pdfsubject={Manual for the LaTeX2e Package childdoc}}
\date{30 December 2018, \textsf{v2.0}}
\maketitle

\begin{abstract}\noindent
\textsf{childdoc} is a \LaTeXe{} package
that enables the direct compilation
of document sections included by |\include|
to individual files.
\end{abstract}

\begingroup
\parskip0ex
\tableofcontents
\endgroup

%%%%%%%%%%%%%%%%%%%%%%%%%%%%%%%%%%%%%%%%%%%%%%%%%%%%%%%%%%%%%%%%%%%%%%%%%%%%%%%%
%%%%%%%%%%%%%%%%%%%%%%%%%%%%%%%%%%%%%%%%%%%%%%%%%%%%%%%%%%%%%%%%%%%%%%%%%%%%%%%%
\section{Introduction}

\LaTeX{} provides a mechanism to structure a large document (such as a book)
into a main file and several child files (containing the chapters)
using the |\include| command.
This mechanism is beneficial for documents
which span hundreds of pages in order to
make the source file(s) more manageable.
Moreover, compilation can be restricted to
selected child files by means of the |\includeonly| command.
The latter feature can be used to reduce the compilation time while editing
(this was significantly more useful in the earlier days of \LaTeX{})
or to generate a smaller document which is easier to navigate.
Another application of |\includeonly| is to generate
documents consisting of selected parts of the complete document.

However, there are a few drawbacks of the plain |\include| mechanism:
\begin{itemize}
\item
The child files cannot be compiled on their own,
they can only be compiled via the main file.
A naive editing environment
(such as a text editor with an option
to have the current file processed by \LaTeX)
may require one to switch to the main file before compiling;
attempting to compile the child file produces errors.
\item
The main file must be modified (each time)
to adjust the |\includeonly| command
to the present needs. This easily leaves the main file in a messy state.
\item
The generated document will always carry the filename
of the main document. This is inconvenient if
several child files are to be compiled and
to be kept for distribution.
\end{itemize}

The present package provides a simple interface
to make child files individually compilable by \LaTeX{}.
Compiling a child file then has the same effect as compiling
the main file with an |\includeonly| command
to select the appropriate child.
Moreover the generated document will carry the name of the child
rather than the main file.
This resolves all three above issues.

This feature is meant to make the editing of books,
thesis documents and lecture notes somewhat more convenient.
However, the package can also be used efficiently for
composing a series of documents (such as exercise sheets)
which are typically distributed individually.
It then assists the author in generating the individual documents
(potentially in different versions)
as well as a document containing the collected series.
Another application is in developing style files
or other kinds of included material
where compilation of the style file could redirect
to a sample or test file.

%%%%%%%%%%%%%%%%%%%%%%%%%%%%%%%%%%%%%%%%%%%%%%%%%%%%%%%%%%%%%%%%%%%%%%%%%%%%%%%%
%%%%%%%%%%%%%%%%%%%%%%%%%%%%%%%%%%%%%%%%%%%%%%%%%%%%%%%%%%%%%%%%%%%%%%%%%%%%%%%%
\section{Usage}

First of all, the package \textsf{childdoc} is \emph{not} a standard
\LaTeXe{} |.sty| style file! Therefore it needs to be invoked in
a non-standard way.

%%%%%%%%%%%%%%%%%%%%%%%%%%%%%%%%%%%%%%%%%%%%%%%%%%%%%%%%%%%%%%%%%%%%%%%%%%%%%%%%
\subsection{Included Files}
\label{sec:include}

%%%%%%%%%%%%%%%%%%%%%%%%%%%%%%%%%%%%%%%%
\DescribeMacro{\childdocmain}
To use the package, add the commands
\begin{center}
\begin{tabular}{l}
|\input{childdoc.def}|\\
|\childdocmain{}|\\
\end{tabular}
\end{center}
at the very top of the main \LaTeX{} file,
in particular \emph{before} the |\documentclass| statement!
The argument of |\childdocmain| should be left empty
(but it must be present).

%%%%%%%%%%%%%%%%%%%%%%%%%%%%%%%%%%%%%%%%
\DescribeMacro{\childdocof}
Furthermore, add the commands
\begin{center}
\begin{tabular}{l}
|\input{childdoc.def}|\\
|\childdocof{|\textit{main}|}|\\
\end{tabular}
\end{center}
at the top of every child file \textit{child}
which is included by |\include{|\textit{child}|}|
from within the main file
(or at least for those files to be compiled individually).
The argument \textit{main} must be the filename of the main file.

There are a couple of
considerations in setting up the main and child documents:

%%%%%%%%%%%%%%%%%%%%%%%%%%%%%%%%%%%%%%%%
\paragraph{Restrictions.}

Please note the following restrictions:
\begin{itemize}
\item
|\childdocmain| must be called with one argument \textit{main}
to ensure compatibility with earlier version of the package.
It must either be empty (|\childdocmain{}|)
or precisely match the filename of the main file in which it is specified.
See \secref{sec:detection} for further information.
\item
The filename \textit{main} must be specified without the |.tex| extension.
\item
The filename \textit{main} is case sensitive
(even in case-insensitive file systems)
due to internal string comparison.
\item
The argument \textit{main} should be fully expanded, it cannot be a macro.
\item
Subdirectories and special characters should be avoided in filenames.
\item
The command |\childdocmain{|\textit{main}|}| must be followed by a whitespace.
It should not be followed immediately by another command
or by a comment mark `|%|'.
This is because the \TeX{} parser reads the token immediately following
the argument of |\childdocmain| and puts it
at the beginning of every child section;
however, a white\-space is ignored.
\end{itemize}

%%%%%%%%%%%%%%%%%%%%%%%%%%%%%%%%%%%%%%%%
\paragraph{Content of Main File.}

It is advisable to place all content in the child files included by |\include|.
Any output contained in the main file will appear in all child documents
unless suppressed manually;
it cannot be suppressed automatically by the |\includeonly| directive
and thus should normally be avoided.
A method to include some content in the main file
by means of conditional processing is described in \secref{sec:conditional}.

%%%%%%%%%%%%%%%%%%%%%%%%%%%%%%%%%%%%%%%%
\paragraph{Page Numbering.}

When only a part of the document is compiled,
the appropriate numbering of pages
(as well as other status parameters)
is determined from the |.aux| files.
The latter contain information from previous passes.
However this information needs to propagate through
all intermediate child documents.
Therefore the page numbering in child documents may well
be inconsistent until the complete document is compiled at least once.

A useful (if unconventional) way to always ensure a consistent
page numbering is to restart the numbering in each child document
and denote the pages by `\textit{child}|.|\textit{page}'
where \textit{child} represents the chapter/section number of the child file.
This can be achieved by the command
|\numberwithin{page}{|\textit{child}|}|
of the \textsf{amsmath} package
where \textit{child} can be |chapter| or |section|
depending on the chosen structuring.
Alternatively, one can modify the macro |\thepage| appropriately
and reset the counter |page| at the start of each child file.

%%%%%%%%%%%%%%%%%%%%%%%%%%%%%%%%%%%%%%%%%%%%%%%%%%%%%%%%%%%%%%%%%%%%%%%%%%%%%%%%
\subsection{Conditional Processing}
\label{sec:conditional}

The package provides a mechanism to compile different versions
of a document. To customise the versions further some conditional processing
can come in handy to distinguish which version is being compiled.
The package provides two macros to describe the compilation context:

%%%%%%%%%%%%%%%%%%%%%%%%%%%%%%%%%%%%%%%%
\DescribeMacro{\ifchilddoc}
The conditional |\ifchilddoc| distinguishes between the compilation of
child documents and the main document:
%
\begin{center}
|\ifchilddoc |\textit{child-code}| |[|\||else |\textit{main-code}]| \||fi|
\end{center}

%%%%%%%%%%%%%%%%%%%%%%%%%%%%%%%%%%%%%%%%
\DescribeMacro{\childdocname}
\DescribeMacro{\childdocjob}
The macro |\childdocname| contains the filename (without extension)
of the main or child file being processed.
Note that |\childdocjob| will always contain the name of the main file.

%%%%%%%%%%%%%%%%%%%%%%%%%%%%%%%%%%%%%%%%
\paragraph{Title Page.}

Conditional processing can be used to include a title or banner page
in the main document when proper precautions are taken.
Importantly, the code in the main file should ensure that the page counter
(as well as other status parameters which are stored in the |.aux| files)
takes the same value after the conditional processing.
Otherwise the page numbers may take divergent values
depending on which part is compiled.

For example, a title page could be declared by:
%
\begin{center}
\begin{tabular}{l}
|\ifchilddoc\||else|\\
|\addtocounter{page}{-1}|\\
\textit{code for title page}\\
|\newpage|\\
|\||fi|
\end{tabular}
\end{center}
%
A banner page for the child documents can be generated by:
%
\begin{center}
\begin{tabular}{l}
|\ifchilddoc|\\
|\addtocounter{page}{-1}|\\
\textit{code for banner page}\\
|\newpage|\\
|\||fi|
\end{tabular}
\end{center}
%
Here one could write a message such as:
\begin{center}
|This is the part \childdocname{} of \childdocjob{}.|
\end{center}

%%%%%%%%%%%%%%%%%%%%%%%%%%%%%%%%%%%%%%%%%%%%%%%%%%%%%%%%%%%%%%%%%%%%%%%%%%%%%%%%
\subsection{Flags}
\label{sec:flags}

The package makes it easy to generate different versions
of the main or child documents.
To this end compilation flags can be defined
and assigned different default values.
They will be particularly useful in conjunction
with the forwarding mechanism described in \secref{sec:forward}.

For example, it may be useful to have a flag |\version|
which can be set to |draft| or |final|.
The document source will contain some conditional code
depending on the value of |\version|.
Suppose further, the flag should default to |final| for the main file
and to |draft| for child files
which is a natural assignment for editing the document.
This is achieved by placing the following code
in the preamble of the main document
(below the |\childdocmain| directive):
%
\begin{center}
\begin{tabular}{l}
|\ifchilddoc|\\
|\providecommand{\version}{draft}|\\
|\||else|\\
|\providecommand{\version}{final}|\\
|\||fi|
\end{tabular}
\end{center}
%
The definition by |\providecommand| makes sure
that previous definitions are not overwritten.
Further statements |\providecommand{\version}{...}|
can thus be added before the above code to override it.

For the main file, one might add a line
(between |\childdocmain| and the above block)
%
\begin{center}
|%\ifchilddoc\||else\providecommand{\version}{draft}\||fi|
\end{center}
%
which can be uncommented to produce a draft version.
Likewise one can add a line to the very top of a child file
(above the |\childdocof{|\textit{main}|}| directive)
%
\begin{center}
|%\providecommand{\version}{final}|
\end{center}
%
which can be uncommented to produce the final version of this child document.

%%%%%%%%%%%%%%%%%%%%%%%%%%%%%%%%%%%%%%%%%%%%%%%%%%%%%%%%%%%%%%%%%%%%%%%%%%%%%%%%
\subsection{Forwarding}
\label{sec:forward}

Different versions of the main or child documents
using compilation flags as described in \secref{sec:flags}
can be (permanently) stored in different files
for convenient compilation, viewing and distribution.
To this end, the package defines a command
to pass on compilation to a different file:

%%%%%%%%%%%%%%%%%%%%%%%%%%%%%%%%%%%%%%%%
\DescribeMacro{\childdocforward}
The command |\childdocforward| redirects processing to
another source file:
%
\begin{center}
\begin{tabular}{l}
|\input{childdoc.def}|\\
|\childdocforward[|\textit{main}|]{|\textit{dest}|}|\\
\end{tabular}
\end{center}
%
The argument \textit{dest} is the destination file
(without extension).
It should be the main file or one of the child files.
Note that further \textsf{childdoc} directives
such as |\childdocof| and |\childdocforward|
in the indicated file will be processed in this form.
The optional argument \textit{main}
passes on directly to the main file \textit{main}
while pretending to compile the child \textit{dest}.
This form behaves as if \textit{dest}
issues |\childdocof{|\textit{main}|}| right away,
and no further \textsf{childdoc} directives will be processed.

%%%%%%%%%%%%%%%%%%%%%%%%%%%%%%%%%%%%%%%%
\DescribeMacro{\...prefix}
In the alternative form |\childdocforwardprefix|,
%
\begin{center}
\begin{tabular}{l}
|\input{childdoc.def}|\\
|\childdocforwardprefix[|\textit{main}|]{|\textit{prefix}|}{|\textit{dest}|}|
\end{tabular}
\end{center}
%
the destination file is determined by a pattern
depending on the current file:
To make this work, the current file must be called
`{\textit{prefix}\hspace{0.2em}\textit{suffix}}'
with \textit{prefix} matching precisely the argument.
Processing is then passed on to the file
`{\textit{dest}\hspace{0.2em}\textit{suffix}}'.
Surely, the same effect is achieved by
directly specifying the
argument `{\textit{dest}\hspace{0.2em}\textit{suffix}}'
in the first form.
However, that requires to set up a different file
for each child. With the alternative form of the command
all these files can have exactly the same content
which simplifies setting them up and maintaining them.

For example, the following file |draft.tex|
with a compilation flag |\version| as described in \secref{sec:flags}
compiles the main document as a draft:
%
\begin{center}
\begin{tabular}{l}
|\def\version{draft}|\\
|\input{childdoc.def}|\\
|\childdocforward{|\textit{main}|}|
\end{tabular}
\end{center}
%
Likewise, the following files |final|\textit{nn}|.tex|
compile the final version of the child document
|child|\textit{nn}|.tex|:
%
\begin{center}
\begin{tabular}{l}
|\def\version{final}|\\
|\input{childdoc.def}|\\
|\childdocforwardprefix{final}{child}|
\end{tabular}
\end{center}
%

Note that when several versions of a main file and/or of each child file
are to be generated, it may be convenient to set up a |Makefile| or
shell script to automatise the process.

%%%%%%%%%%%%%%%%%%%%%%%%%%%%%%%%%%%%%%%%%%%%%%%%%%%%%%%%%%%%%%%%%%%%%%%%%%%%%%%%
\subsection{Command Line Processing}
\label{sec:commandline}

The effect of redirection files can also be achieved by invoking
the \LaTeX{} compiler with a more elaborate command line.
Most conveniently this should be done as part
of a shell script or a |Makefile|.

When using \textsf{childdoc} in the main file, the following
command lines effectively perform a redirection
(note that depending on the shell being used,
backslashes may have to be doubled: `|\|' $\to$ `|\\|'):
%
\begin{center}
|... -jobname "|\textit{target}|" |\\|"|[\textit{flags}]%
|\input{childdoc.def}\childdocforward[|\textit{main}|]{|\textit{dest}|}"|
\end{center}
%
Here \textit{target} is the name of the output file,
\textit{main} is the name of the main file
and \textit{dest} is the name of the main or child file to be processed
(all filenames without extensions).
The optional argument \textit{main} can be omitted
if \textit{main} matches \textit{dest}.
Optionally, compilation \textit{flags} can be defined via |\def| commands.
This command line makes the \TeX{} engine believe
it is compiling the file \textit{target}
whose content is specified as the latter parameter.
The provided code then forwards the processing to
\textit{main} or \textit{dest} as described in \secref{sec:forward}.

%%%%%%%%%%%%%%%%%%%%%%%%%%%%%%%%%%%%%%%%%%%%%%%%%%%%%%%%%%%%%%%%%%%%%%%%%%%%%%%%
\subsection{Include by Input}
\label{sec:input}

Including child documents by |\include| has some restrictions by design.
Most notably, the content of a child document always occupies
its own set of pages; pages cannot be shared between child documents.
Usually, this behaviour makes perfect sense
because each child document contain an essential part of the document.
However, in some situations it may be desirable to compose
a document from a collection of parts
without having mandatory page breaks between then.
For this case, the package
provides a mechanism to include parts
by |\input| which can also be processed individually.
However, by construction this mechanism
requires manual handling of the content to be output.

%%%%%%%%%%%%%%%%%%%%%%%%%%%%%%%%%%%%%%%%
\DescribeMacro{\ifchilddocmanual}
The main file should be prepared as usual, see \secref{sec:include}.
However, the document body must make a distinction
between processing of an individual part and of the main document, e.g.:
%
\begin{center}
\begin{tabular}{l}
|\ifchilddocmanual|\\
|\input{\childdocname}|\\
|\||else|\\
\textit{document body with }|\input{|\textit{part}|}|\\
|\||fi|
\end{tabular}
\end{center}
%
The conditional |\ifchilddocmanual| is true whenever
a part to be included by |\input| is being compiled,
and the name of the part is stored in |\childdocname|.

%%%%%%%%%%%%%%%%%%%%%%%%%%%%%%%%%%%%%%%%
\DescribeMacro{\childdocby}
Each part to be included by |\input| should start with:
%
\begin{center}
\begin{tabular}{l}
|\input{childdoc.def}|\\
|\childdocby{|\textit{main}|}|\\
\end{tabular}
\end{center}
%
The directive |\childdocby| is similar to |\childdocof|
described in \secref{sec:include},
but the subsequent selection of content must be done manually.
To that end, both |\ifchilddoc| and |\ifchilddocmanual|
will be true upon processing of a part,
and the name of the part is stored in |\childdocname|.
Note that |\jobname| will be set to the filename of the current part
so that each part receives an individual |.aux| file
that does not interfere with the |.aux| file(s) of the main document.
This behaviour can be altered by the alternative form
|\childdocby[*]{|\textit{main}|}| (with a non-empty optional argument)
which uses the |.aux| file of the main document
by setting |\jobname| to \textit{main}.

%%%%%%%%%%%%%%%%%%%%%%%%%%%%%%%%%%%%%%%%%%%%%%%%%%%%%%%%%%%%%%%%%%%%%%%%%%%%%%%%
\subsection{Driver Development}
\label{sec:driver}

The \textsf{childdoc} mechanism can also be use for the development
of definition files such as \LaTeX{} styles or classes.
This case differs from the above setup with multiple parts
included by |\include| in that no |\includeonly| should be invoked.
This can be achieved by starting the include file
(before |\ProvidesPackage|) with:
%
\begin{center}
\begin{tabular}{l}
|\input{childdoc.def}|\\
|\childdocforward{|\textit{main}|}|\\
\end{tabular}
\end{center}
%
or alternatively with:
%
\begin{center}
\begin{tabular}{l}
|\input{childdoc.def}|\\
|\childdocby{|\textit{main}|}|\\
\end{tabular}
\end{center}
%
Both forms have slightly different effects as described above.
The main file is prepared as usual, see \secref{sec:include}.

%%%%%%%%%%%%%%%%%%%%%%%%%%%%%%%%%%%%%%%%%%%%%%%%%%%%%%%%%%%%%%%%%%%%%%%%%%%%%%%%
\subsection{Legacy Detection}
\label{sec:detection}

The directive |\childdocmain| in the main file can detect
whether the complete document or merely a child is to be compiled
even without using the directive |\childdocof|.
This method is deprecated because it is less robust
and there is no compelling reason to use it;
it is merely provided for backward compatibility
and it may be removed in future versions.

If the detection mechanism is to be used,
it is mandatory to correctly specify
the filename of the main file as the argument of |\childdocmain|:
%
\begin{center}
\begin{tabular}{l}
|\input{childdoc.def}|\\
|\childdocmain{|\textit{main}|}|\\
\end{tabular}
\end{center}
%
If |\jobname| does not match the argument \textit{main} of |\childdocmain|,
it is assumed that |\jobname| points to the child file to be compiled.
When using |\childdocmain| with the main file specified as argument,
it suffices to start a child file
with just |\input{|\textit{main}|}|
without loading of the package and using |\childdocof|.
If instead all processing is done
with the appropriate \textsf{childdoc} directives,
the argument of \textit{main} of |\childdocmain| can be empty.

An alternative version of the command line processing described
in \secref{sec:commandline} using the detection mechanism reads:
%
\begin{center}
|... -jobname "|\textit{target}|" "|[\textit{flags}]%
[|\def\jobname{|\textit{dest}|}|]|\input{|\textit{main}|}"|
\end{center}

%%%%%%%%%%%%%%%%%%%%%%%%%%%%%%%%%%%%%%%%%%%%%%%%%%%%%%%%%%%%%%%%%%%%%%%%%%%%%%%%
\subsection{Manual Code}
\label{sec:manual}

In case one cannot be certain whether the definitions file |childdoc.def|
is installed on the target \TeX{} distribution
and one prefers not to ship it,
it is conceivable to paste a few relevant commands into the sources.

To that end, drop all statements |\input{childdoc.def}|
and perform the replacements as outlined below.
Instead of |\childdocmain{|\textit{main}|}| add the following code
to the top of the main file:
%
\begin{center}
\begin{tabular}{l}
|\||ifdefined\childdocname\endinput\||fi\newif\ifchilddoc|\\
|\edef\childdocname{\scantokens\expandafter{\jobname\noexpand}}|\\
|\def\childdocmain{|\textit{main}|}\||ifx\childdocmain\childdocname\||else|\\
|\childdoctrue\includeonly{\childdocname}\let\jobname\childdocmain\||fi|\\
\end{tabular}
\end{center}
%
Instead of |\childdocof{|\textit{main}|}| just include the main file
at the top of each child file:
%
\begin{center}
|\input{|\textit{main}|}|
\end{center}
%
A simple redirection |\childdocforward{|\textit{dest}|}| is achieved by:
%
\begin{center}
|\def\jobname{|\textit{dest}|}\input{\jobname}|
\end{center}
%
The redirection with prefix
|\childdocforwardprefix[|\textit{prefix}|]{|\textit{dest}|}|
is accomplished by:
%
\begin{center}
\begin{tabular}{l}
|{\edef\jobname{\scantokens\expandafter{\jobname\noexpand}}|\\
|\def\redirectjob |\textit{prefix}|#1~~~{\gdef\jobname{|\textit{dest}|#1}}|\\
|\expandafter\redirectjob\jobname~~~}\input{\jobname}|
\end{tabular}
\end{center}

In an alternative approach,
child documents can be compiled by a specific command line
without additional code or specific definitions:
%
\begin{center}
|... -jobname "|\textit{target}|" "|[\textit{flags}]%
|\includeonly{|\textit{dest}|}\input{|\textit{main}|}"|
\end{center}
%

%%%%%%%%%%%%%%%%%%%%%%%%%%%%%%%%%%%%%%%%%%%%%%%%%%%%%%%%%%%%%%%%%%%%%%%%%%%%%%%%
%%%%%%%%%%%%%%%%%%%%%%%%%%%%%%%%%%%%%%%%%%%%%%%%%%%%%%%%%%%%%%%%%%%%%%%%%%%%%%%%
\section{Information}

%%%%%%%%%%%%%%%%%%%%%%%%%%%%%%%%%%%%%%%%%%%%%%%%%%%%%%%%%%%%%%%%%%%%%%%%%%%%%%%%
\subsection{Copyright}

Copyright \copyright{} 2017--2018 Niklas Beisert

This work may be distributed and/or modified under the
conditions of the \LaTeX{} Project Public License, either version 1.3
of this license or (at your option) any later version.
The latest version of this license is in
  \url{http://www.latex-project.org/lppl.txt}
and version 1.3 or later is part of all distributions of \LaTeX{}
version 2005/12/01 or later.

This work has the LPPL maintenance status `maintained'.

The Current Maintainer of this work is Niklas Beisert.

This work consists of the files |README.txt|, |childdoc.ins| and |childdoc.dtx|
as well as the derived files |childdoc.def|, |cdocsamp.tex|
with |cdocsch1.tex|, |cdocsch2.tex|, |cdocspt3.tex|, |cdocspt4.tex|,
|cdocsdrf.tex|, |cdocsfn1.tex|, |cdocsfn2.tex|
as well as |childdoc.pdf|.

%%%%%%%%%%%%%%%%%%%%%%%%%%%%%%%%%%%%%%%%%%%%%%%%%%%%%%%%%%%%%%%%%%%%%%%%%%%%%%%%
\subsection{Files and Installation}

The package consists of the files:
%
\begin{center}
\begin{tabular}{ll}
    |README.txt|   & readme file \\
    |childdoc.ins| & installation file \\
    |childdoc.dtx| & source file \\
    |childdoc.def| & definition file \\
    |cdocsamp.tex| & sample main file \\
    |cdocsch1.tex| & sample include file \\
    |cdocsch2.tex| & sample include file \\
    |cdocspt3.tex| & sample part file \\
    |cdocspt4.tex| & sample part file \\
    |cdocsdrf.tex| & sample redirection file \\
    |cdocsfn1.tex| & sample redirection file \\
    |cdocsfn2.tex| & sample redirection file \\
    |childdoc.pdf| & manual
\end{tabular}
\end{center}
%
The distribution consists of the files
|README.txt|, |childdoc.ins| and |childdoc.dtx|.
%
\begin{itemize}
\item
Run (pdf)\LaTeX{} on |childdoc.dtx|
to compile the manual |childdoc.pdf| (this file).
\item
Run \LaTeX{} on |childdoc.ins| to create the definitions file |childdoc.def|
and the sample |cdocsamp.tex| with include files
|cdocsch1.tex|, |cdocsch2.tex|, |cdocspt3.tex|, |cdocspt4.tex|,
|cdocsdrf.tex|, |cdocsfn1.tex|, |cdocsfn2.tex|.
Then copy the file |childdoc.def| to an appropriate directory of your \LaTeX{}
distribution, e.g.\ \textit{texmf-root}|/tex/latex/childdoc|.
\end{itemize}

%%%%%%%%%%%%%%%%%%%%%%%%%%%%%%%%%%%%%%%%%%%%%%%%%%%%%%%%%%%%%%%%%%%%%%%%%%%%%%%%
\subsection{Related CTAN Packages}

There are several other packages which offer a similar functionality:
%
\begin{itemize}
\item
The packages
\href{http://ctan.org/pkg/docmute}{\textsf{docmute}},
\href{http://ctan.org/pkg/includex}{\textsf{includex}} and
\href{http://ctan.org/pkg/standalone}{\textsf{standalone}}
provide commands to include only the document body of
a child file thus allowing both files to be compiled individually.
\item
The packages \href{http://ctan.org/pkg/subdocs}{\textsf{subdocs}}
and \href{http://ctan.org/pkg/subfiles}{\textsf{subfiles}}
provide structures in which the main and child documents can be
encapsulated and allowing them to be compiled individually.
The inclusion mechanism is different from the conventional |\include|.
\item
The package \href{http://ctan.org/pkg/combine}{\textsf{combine}}
is an elaborate solution to combine several documents into one.
\end{itemize}
%
See also the CTAN topic \href{http://ctan.org/topic/subdocs}{\textsf{subdocs}}
for further related packages.
The present package differs from the above solutions in that
a document structure constructed with the conventional |\include| mechanism
just needs two extra commands at the top of every file
such that all constituent files can be compiled individually.

%%%%%%%%%%%%%%%%%%%%%%%%%%%%%%%%%%%%%%%%%%%%%%%%%%%%%%%%%%%%%%%%%%%%%%%%%%%%%%%%
%\subsection{Feature Suggestions}
%
%The following is a list of features which may be useful for future
%versions of this package:
%%
%\begin{itemize}
%\item
%\ldots
%\end{itemize}

%%%%%%%%%%%%%%%%%%%%%%%%%%%%%%%%%%%%%%%%%%%%%%%%%%%%%%%%%%%%%%%%%%%%%%%%%%%%%%%%
\subsection{Revision History}

%%%%%%%%%%%%%%%%%%%%%%%%%%%%%%%%%%%%%%%%
\paragraph{v2.0:} 2018/12/30

\begin{itemize}
\item
immediate forward processing
\item
added |\childdocby| mechanism
\item
manual restructured
\end{itemize}

%%%%%%%%%%%%%%%%%%%%%%%%%%%%%%%%%%%%%%%%
\paragraph{v1.6:} 2018/01/17

\begin{itemize}
\item
application for development of include files
\item
corrections to manual
\end{itemize}

%%%%%%%%%%%%%%%%%%%%%%%%%%%%%%%%%%%%%%%%
\paragraph{v1.5:} 2017/05/21

\begin{itemize}
\item
more complete structuring introduced
\item
|\childdocof| introduced
\item
|\childdoc| renamed to |\childdocmain|
\item
|\childredirect| renamed to |\childdocforward| and |\childdocforwardprefix|
and functionality expanded
\end{itemize}

%%%%%%%%%%%%%%%%%%%%%%%%%%%%%%%%%%%%%%%%
\paragraph{v1.0:} 2017/04/27

\begin{itemize}
\item
manual and install package
\item
first version published on CTAN
\end{itemize}

%%%%%%%%%%%%%%%%%%%%%%%%%%%%%%%%%%%%%%%%
\paragraph{v0.6:} 2017/04/26

\begin{itemize}
\item
redirection mechanism added
\end{itemize}

%%%%%%%%%%%%%%%%%%%%%%%%%%%%%%%%%%%%%%%%
\paragraph{v0.5:} 2017/04/26

\begin{itemize}
\item
functionality in definition file
\end{itemize}


%%%%%%%%%%%%%%%%%%%%%%%%%%%%%%%%%%%%%%%%%%%%%%%%%%%%%%%%%%%%%%%%%%%%%%%%%%%%%%%%
%%%%%%%%%%%%%%%%%%%%%%%%%%%%%%%%%%%%%%%%%%%%%%%%%%%%%%%%%%%%%%%%%%%%%%%%%%%%%%%%
%%%%%%%%%%%%%%%%%%%%%%%%%%%%%%%%%%%%%%%%%%%%%%%%%%%%%%%%%%%%%%%%%%%%%%%%%%%%%%%%
\appendix

\settowidth\MacroIndent{\rmfamily\scriptsize 000\ }

 \DocInput{childdoc.dtx}

\end{document}
%</driver>
% \fi
%
% %%%%%%%%%%%%%%%%%%%%%%%%%%%%%%%%%%%%%%%%%%%%%%%%%%%%%%%%%%%%%%%%%%%%%%%%%%%%%%
% %%%%%%%%%%%%%%%%%%%%%%%%%%%%%%%%%%%%%%%%%%%%%%%%%%%%%%%%%%%%%%%%%%%%%%%%%%%%%%
% \section{Sample}
%\iffalse
%<*samplemain>
%\fi
%
% The following presents a sample document
% with two chapters, two parts, a title page,
% a compile flag as well as three forwarding files to set the flag.
% It consists of eight |.tex| files:
% \begin{center}
% \begin{tabular}{ll}
% |cdocsamp.tex|&main file\\
% |cdocsch1.tex|&include file for chapter 1\\
% |cdocsch2.tex|&include file for chapter 2\\
% |cdocspt3.tex|&include file for part 3\\
% |cdocspt4.tex|&include file for part 4\\
% |cdocsdrf.tex|&forwarding file for main file in draft mode\\
% |cdocsfi1.tex|&forwarding file for final version of chapter 1\\
% |cdocsfi2.tex|&forwarding file for final version of chapter 2\\
% \end{tabular}
% \end{center}
% Each of the eight files can be compiled directly by the \LaTeX{} compiler.
%
% %%%%%%%%%%%%%%%%%%%%%%%%%%%%%%%%%%%%%%
% \paragraph{Main File.}
%
% The main file is called |cdocsamp.tex|.
%
% Load the \textsf{childdoc} definitions and
% declare the filename for the main document:
%    \begin{macrocode}
\input{childdoc.def}
\childdocmain{}
%    \end{macrocode}

% Optional override for |\version| flag:
%    \begin{macrocode}
%%\ifchilddoc\else\providecommand{\version}{draft}\fi
%    \end{macrocode}

% Define the default values for the |\version| flag
% (|final| for the main file and |draft| for childs):
%    \begin{macrocode}
\ifchilddoc
\providecommand{\version}{draft}
\else
\providecommand{\version}{final}
\fi
%    \end{macrocode}

% Load the standard document class:
%    \begin{macrocode}
\documentclass[12pt]{article}
%    \end{macrocode}

% Start the document body:
%    \begin{macrocode}
\begin{document}
%    \end{macrocode}

% Declare a title page.
% Print title, part of document being processed and version flag:
%    \begin{macrocode}
\addtocounter{page}{-1}
\begin{center}
{\LARGE\bfseries{}childdoc example\par}
\vspace{1cm}
\ifchilddoc
\ifchilddocmanual part\else chapter\fi:
`\childdocname' of `\childdocjob'\par
\else
main document: `\childdocjob'\par
\fi
version: \version\par
\end{center}
\newpage
%    \end{macrocode}

% Manually include selected file,
% otherwise process as usual:
%    \begin{macrocode}
\ifchilddocmanual
\section*{part `\childdocname'}
\input{\childdocname}
\else
%    \end{macrocode}

% Include the two chapters:
%    \begin{macrocode}
\include{cdocsch1}
\include{cdocsch2}
%    \end{macrocode}

% Include the two parts unless only chapters should be displayed:
%    \begin{macrocode}
\ifchilddoc\else
\section{part three}
\input{cdocspt3}
\section{part four}
\input{cdocspt4}
\fi
%    \end{macrocode}

% Process as usual until here:
%    \begin{macrocode}
\fi
%    \end{macrocode}

% End of document body:
%    \begin{macrocode}
\end{document}
%    \end{macrocode}
%\iffalse
%</samplemain>
%\fi
%
% %%%%%%%%%%%%%%%%%%%%%%%%%%%%%%%%%%%%%%
% \paragraph{Chapter Include Files.}
%
% The include files are called |cdocsch1.tex| and |cdocsch2.tex|.
%
%\iffalse
%<*samplechap1|samplechap2>
%\fi

% Optional override for |\version| flag:
%    \begin{macrocode}
%%\providecommand{\version}{final}
%    \end{macrocode}

% Include the main document:
%    \begin{macrocode}
\input{childdoc.def}
\childdocof{cdocsamp}
%    \end{macrocode}

%\iffalse
%</samplechap1|samplechap2>
%\fi
%
%\iffalse
%<*samplechap1>
%\fi
% Some text for chapter 1:
%    \begin{macrocode}
\section{one}
some text in chapter one
%    \end{macrocode}

%\iffalse
%</samplechap1>
%\fi
% Some text for chapter 2:
%\iffalse
%<*samplechap2>
%\fi
%    \begin{macrocode}
\section{two}
more text in chapter two
%    \end{macrocode}

%\iffalse
%</samplechap2>
%\fi
%
% %%%%%%%%%%%%%%%%%%%%%%%%%%%%%%%%%%%%%%
% \paragraph{Part Include Files.}
%
% The include files are called |cdocspt3.tex| and |cdocspt4.tex|.
%
%\iffalse
%<*samplepart3|samplepart4>
%\fi

% Optional override for |\version| flag:
%    \begin{macrocode}
%%\providecommand{\version}{final}
%    \end{macrocode}

% Include the main document:
%    \begin{macrocode}
\input{childdoc.def}
\childdocby{cdocsamp}
%    \end{macrocode}

%\iffalse
%</samplepart3|samplepart4>
%\fi
%
%\iffalse
%<*samplepart3>
%\fi
% Some text for part 3:
%    \begin{macrocode}
some text in part three
%    \end{macrocode}

%\iffalse
%</samplepart3>
%\fi
% Some text for part 4:
%\iffalse
%<*samplepart4>
%\fi
%    \begin{macrocode}
more text in part four
%    \end{macrocode}

%\iffalse
%</samplepart4>
%\fi
%
% %%%%%%%%%%%%%%%%%%%%%%%%%%%%%%%%%%%%%%
% \paragraph{Forwarding for a Complete Draft.}
%
% The following forwarding file |cdocsdrf.tex|
% compiles the main document in draft mode:
%\iffalse
%<*sampledraft>
%\fi
%    \begin{macrocode}
\def\version{draft}
\input{childdoc.def}
\childdocforward{cdocsamp}
%    \end{macrocode}

%\iffalse
%</sampledraft>
%\fi
%
% %%%%%%%%%%%%%%%%%%%%%%%%%%%%%%%%%%%%%%
% \paragraph{Forwarding for Final Version of the Chapters.}
%
% The following forwarding files |cdocsfn1.tex| and |cdocsfn2.tex|
% (with identical content)
% compile the final versions of the child documents
% |cdocsch1.tex| and |cdocsch2.tex|, respectively:
%\iffalse
%<*samplefinal>
%\fi
%    \begin{macrocode}
\def\version{final}
\input{childdoc.def}
\childdocforwardprefix[cdocsamp]{cdocsfn}{cdocsch}
%    \end{macrocode}

%\iffalse
%</samplefinal>
%\fi
%
% %%%%%%%%%%%%%%%%%%%%%%%%%%%%%%%%%%%%%%
% \paragraph{Command Line Processing.}
%
% The following three command lines generate the output files
% |cdocscld|, |cdocscl1| and |cdocscl2|
% which should be identical to
% |cdocsdrf|, |cdocsch1| and |cdocsfn2|, respectively:
% \begin{center}
% \begin{tabular}{l}
% |latex -jobname cdocscld \|\\
% |  "\def\version{draft}\input{childdoc.def}\childdocforward{cdocsamp}"|\\
% |latex -jobname cdocscl1 \|\\
% |  "\input{childdoc.def}\childdocforward[cdocsamp]{cdocsch1}"|\\
% |latex -jobname cdocscl2 \|\\
% |  "\def\version{final}\input{childdoc.def}\childdocforward{cdocsch2}"|
% \end{tabular}
% \end{center}
% Note that the trailing backslash on each first line
% merely continues the input to the second line
% (for convenient cut ant paste).
% Furthermore, the command |latex| can be replaced by any
% of its alternative versions such as |pdflatex|.
%
% %%%%%%%%%%%%%%%%%%%%%%%%%%%%%%%%%%%%%%%%%%%%%%%%%%%%%%%%%%%%%%%%%%%%%%%%%%%%%%
% %%%%%%%%%%%%%%%%%%%%%%%%%%%%%%%%%%%%%%%%%%%%%%%%%%%%%%%%%%%%%%%%%%%%%%%%%%%%%%
% \section{Implementation}
%\iffalse
%<*package>
%\fi
%
% This section describes the definitions file |childdoc.def|.

% The definitions cannot be loaded using |\usepackage| or |\RequirePackage|
% which has a mechanism to prevent loading a style file more than once.
% When loading the definitions by means of |\input|
% multiple instances have to be prevented manually:
%\iffalse
%This code needs to be before the `\ProvidesFile' directive
%which is defined at the beginning of this file.
%Therefore it is also placed there and commented out here.
%</package>
%<*discard>
%\fi
%    \begin{macrocode}
\ifdefined\childdocmain\endinput\fi
%    \end{macrocode}
%\iffalse
%</discard>
%<*package>
%\fi
%
% \macro{\ifchilddoc}
% \macro{\ifchilddocmanual}
% The conditional |\ifchilddoc| tells whether a
% child (true) or main (false) document is being compiled.
% The conditional |\ifchilddocmanual| tells whether
% the |\includeonly| mechanism is used (false) or
% the selection of child files must be performed manually (true).
% The definitions initialise to false:
%    \begin{macrocode}
\newif\ifchilddoc
\newif\ifchilddocmanual
%    \end{macrocode}

% \macro{\childdocname}
% \macro{\childdocjob}
% The macro |\childdocname| stores the name of the main document
% to be compiled. The macro |\childdocjob| stores the name of
% the document on which the \LaTeX{} compiler was originally invoked.
% The content of |\jobname| cannot be compared
% to filenames specified in the source due to different catcodes.
% The following code rescans |\jobname|, stores the result
% in |\childdocname| and saves a copy in |\childdocjob|:
%    \begin{macrocode}
\edef\childdocname{\scantokens\expandafter{\jobname\noexpand}}
\let\childdocjob\childdocname
%    \end{macrocode}

% \macro{\childdocdisable}
% The macro |\childdocdisable| prevents the main file
% from being processed more than once.
% At this stage, the main document command |\childdocmain|
% is assumed to be called once again where it should do nothing.
% Any subsequent call to it should prevent
% a secondary processing of the main document
% It overwrites the forwarding commands
% |\childdocof| and |\childdocforward|
% with empty macros to prevent further inclusions of the main document:
%    \begin{macrocode}
\newcommand{\childdocdisable}
{
  \renewcommand{\childdocmain}[1]{\renewcommand{\childdocmain}[1]{\endinput}}
  \renewcommand{\childdocof}[1]{}
  \renewcommand{\childdocby}[2][]{}
  \renewcommand{\childdocforward}[2][]{}
  \renewcommand{\childdocdisable}{}
}
%    \end{macrocode}

% \macro{\childdocmain}
% The macro |\childdocmain| is to be called at the top of the main file
% with nothing or the main filename (without extension) as argument.
% First, it breaks loops.
% If the argument is not empty and does not match |\childdocname|
% (which is set by the first inclusion of |childdoc.def|),
% |\ifchilddoc| is set to true, |\includeonly| is applied to the child file
% and |\jobname| is set to the main file
% (for proper handling of |.aux| files):
%    \begin{macrocode}
\newcommand{\childdocmain}[1]
{
  \childdocdisable\childdocmain{}
  \if?#1?\else
    \begingroup
      \def\childdoctmp{#1}
      \ifx\childdoctmp\childdocname
        \def\childdoctmp{}
      \else
        \def\childdoctmp
        {
          \childdoctrue
          \includeonly{\childdocname}
          \def\childdocjob{#1}
          \def\jobname{#1}
        }
      \fi
      \expandafter
    \endgroup
    \childdoctmp
  \fi
}
%    \end{macrocode}

% \macro{\childdocof}
% The command |\childdocof| redirects
% compilation to the main file |#1|.
%    \begin{macrocode}
\newcommand{\childdocof}[1]
{
  \childdocdisable
  \childdoctrue
  \includeonly{\childdocname}
  \def\jobname{#1}
  \def\childdocjob{#1}
  \input{#1}
}
%    \end{macrocode}

% \macro{\childdocby}
% The command |\childdocby| ....
%    \begin{macrocode}
\newcommand{\childdocby}[2][]
{
  \childdocdisable
  \childdoctrue
  \childdocmanualtrue
  \if?#1?\else
    \def\jobname{#2}
  \fi
  \def\childdocjob{#2}
  \input{#2}
  \endinput
}
%    \end{macrocode}

% \macro{\childdocforward}
% The command |\childdocforward| redirects
% compilation to the main file or
% (if the optional argument is given) a child file.
% Parameters are set as if the main file
% or a child file starting with |\childdocof| was compiled.
% Then compilation is handed over to the main file:
%    \begin{macrocode}
\newcommand{\childdocforward}[2][]
{
  \begingroup
    \if?#1?
      \def\childdoctmp
      {
        \def\childdocname{#2}
        \def\childdocjob{#2}
        \def\jobname{#2}
        \input{#2}
        \endinput
      }
    \else
      \def\childdoctmp
      {
        \childdocdisable
        \def\childdocname{#2}
        \childdoctrue
        \includeonly{#2}
        \def\childdocjob{#1}
        \def\jobname{#1}
        \input{#1}
        \endinput
      }
    \fi
    \expandafter
  \endgroup
  \childdoctmp
}
%    \end{macrocode}

% \macro{\childdocforwardprefix}
% The command |\childdocforwardprefix| redirects
% compilation to the main or a child file by means of a pattern.
% The prefix |#1| in the current filename is replaced by |#2|
% and the suffix of the current filename is kept
% (it is assumed that the filename does not contain the substring `|~~~|'
% which is used as a delimiter).
% Compilation is handed over to the new file by |\childdocforward|:
%    \begin{macrocode}
\newcommand{\childdocforwardprefix}[3][]
{
  \begingroup
    \def\childdocextract #2##1~~~{\def\childdoctmp{\childdocforward[#1]{#3##1}}}
    \expandafter\childdocextract\childdocname~~~
    \expandafter
  \endgroup
  \childdoctmp
}
%    \end{macrocode}

% \macro{\childdoc}
% The deprecated macro |\childdoc| is a legacy version of |\childdocmain|:
%    \begin{macrocode}
\newcommand{\childdoc}{\childdocmain}
%    \end{macrocode}

% \macro{\childdocredirect}
% The deprecated macro |\childdocredirect| is a legacy version
% of |\childdocforward| and |\childdocforwardprefix|:
%    \begin{macrocode}
\newcommand{\childdocredirect}[2][]
{
  \begingroup
    \if?#1?
      \def\childdoctmp{\childdocforward{#2}}
    \else
      \def\childdoctmp{\childdocforwardprefix{#1}{#2}}
    \fi
    \expandafter
  \endgroup
  \childdoctmp
}
%    \end{macrocode}

%\iffalse
%</package>
%\fi
%
\endinput
|\\
|\childdocforward{|\textit{main}|}|
\end{tabular}
\end{center}
%
Likewise, the following files |final|\textit{nn}|.tex|
compile the final version of the child document
|child|\textit{nn}|.tex|:
%
\begin{center}
\begin{tabular}{l}
|\def\version{final}|\\
|% \iffalse
%
% childdoc.dtx Copyright (C) 2017-2018 Niklas Beisert
%
% This work may be distributed and/or modified under the
% conditions of the LaTeX Project Public License, either version 1.3
% of this license or (at your option) any later version.
% The latest version of this license is in
%   http://www.latex-project.org/lppl.txt
% and version 1.3 or later is part of all distributions of LaTeX
% version 2005/12/01 or later.
%
% This work has the LPPL maintenance status `maintained'.
%
% The Current Maintainer of this work is Niklas Beisert.
%
% This work consists of the files childdoc.dtx and childdoc.ins
% and the derived files childdoc.def and cdocsamp.tex with
% cdocsch1.tex, cdocsch2.tex, cdocsdrf.tex, cdocsfn1.tex, cdocsfn2.tex.
%
%<package>\ifdefined\childdocmain\endinput\fi
%<package>\ProvidesFile{childdoc.def}[2018/12/30 v2.0 child document driver]
%<samplemain>\ProvidesFile{cdocsamp.tex}[2018/12/30 v2.0 sample for childdoc]
%<*driver>
%\ProvidesFile{childdoc.drv}[2018/12/30 v2.0 childdoc reference manual file]
\PassOptionsToClass{10pt,a4paper}{article}
\documentclass{ltxdoc}

\usepackage[margin=35mm]{geometry}
\usepackage{hyperref}
\usepackage{hyperxmp}
\usepackage[usenames]{color}

\hypersetup{colorlinks=true}
\hypersetup{pdfstartview=FitH}
\hypersetup{pdfpagemode=UseNone}
\hypersetup{pdfsource={}}
\hypersetup{pdflang={en-UK}}
\hypersetup{pdfcopyright={Copyright 2017-2018 Niklas Beisert.
  This work may be distributed and/or modified under the
  conditions of the LaTeX Project Public License, either version 1.3
  of this license or (at your option) any later version.}}
\hypersetup{pdflicenseurl={http://www.latex-project.org/lppl.txt}}
\hypersetup{pdfcontactaddress={ETH Zurich, ITP, HIT K,
  Wolfgang-Pauli-Strasse 27}}
\hypersetup{pdfcontactpostcode={8093}}
\hypersetup{pdfcontactcity={Zurich}}
\hypersetup{pdfcontactcountry={Switzerland}}
\hypersetup{pdfcontactemail={nbeisert@itp.phys.ethz.ch}}
\hypersetup{pdfcontacturl={http://people.phys.ethz.ch/\xmptilde nbeisert/}}

\newcommand{\secref}[1]{\hyperref[#1]{section \ref*{#1}}}

\parskip1ex
\parindent0pt
\let\olditemize\itemize
\def\itemize{\olditemize\parskip0pt}

\begin{document}

\title{The \textsf{childdoc} Package}
\hypersetup{pdftitle={The childdoc Package}}
\author{Niklas Beisert\\[2ex]
  Institut f\"ur Theoretische Physik\\
  Eidgen\"ossische Technische Hochschule Z\"urich\\
  Wolfgang-Pauli-Strasse 27, 8093 Z\"urich, Switzerland\\[1ex]
  \href{mailto:nbeisert@itp.phys.ethz.ch}
  {\texttt{nbeisert@itp.phys.ethz.ch}}}
\hypersetup{pdfauthor={Niklas Beisert}}
\hypersetup{pdfsubject={Manual for the LaTeX2e Package childdoc}}
\date{30 December 2018, \textsf{v2.0}}
\maketitle

\begin{abstract}\noindent
\textsf{childdoc} is a \LaTeXe{} package
that enables the direct compilation
of document sections included by |\include|
to individual files.
\end{abstract}

\begingroup
\parskip0ex
\tableofcontents
\endgroup

%%%%%%%%%%%%%%%%%%%%%%%%%%%%%%%%%%%%%%%%%%%%%%%%%%%%%%%%%%%%%%%%%%%%%%%%%%%%%%%%
%%%%%%%%%%%%%%%%%%%%%%%%%%%%%%%%%%%%%%%%%%%%%%%%%%%%%%%%%%%%%%%%%%%%%%%%%%%%%%%%
\section{Introduction}

\LaTeX{} provides a mechanism to structure a large document (such as a book)
into a main file and several child files (containing the chapters)
using the |\include| command.
This mechanism is beneficial for documents
which span hundreds of pages in order to
make the source file(s) more manageable.
Moreover, compilation can be restricted to
selected child files by means of the |\includeonly| command.
The latter feature can be used to reduce the compilation time while editing
(this was significantly more useful in the earlier days of \LaTeX{})
or to generate a smaller document which is easier to navigate.
Another application of |\includeonly| is to generate
documents consisting of selected parts of the complete document.

However, there are a few drawbacks of the plain |\include| mechanism:
\begin{itemize}
\item
The child files cannot be compiled on their own,
they can only be compiled via the main file.
A naive editing environment
(such as a text editor with an option
to have the current file processed by \LaTeX)
may require one to switch to the main file before compiling;
attempting to compile the child file produces errors.
\item
The main file must be modified (each time)
to adjust the |\includeonly| command
to the present needs. This easily leaves the main file in a messy state.
\item
The generated document will always carry the filename
of the main document. This is inconvenient if
several child files are to be compiled and
to be kept for distribution.
\end{itemize}

The present package provides a simple interface
to make child files individually compilable by \LaTeX{}.
Compiling a child file then has the same effect as compiling
the main file with an |\includeonly| command
to select the appropriate child.
Moreover the generated document will carry the name of the child
rather than the main file.
This resolves all three above issues.

This feature is meant to make the editing of books,
thesis documents and lecture notes somewhat more convenient.
However, the package can also be used efficiently for
composing a series of documents (such as exercise sheets)
which are typically distributed individually.
It then assists the author in generating the individual documents
(potentially in different versions)
as well as a document containing the collected series.
Another application is in developing style files
or other kinds of included material
where compilation of the style file could redirect
to a sample or test file.

%%%%%%%%%%%%%%%%%%%%%%%%%%%%%%%%%%%%%%%%%%%%%%%%%%%%%%%%%%%%%%%%%%%%%%%%%%%%%%%%
%%%%%%%%%%%%%%%%%%%%%%%%%%%%%%%%%%%%%%%%%%%%%%%%%%%%%%%%%%%%%%%%%%%%%%%%%%%%%%%%
\section{Usage}

First of all, the package \textsf{childdoc} is \emph{not} a standard
\LaTeXe{} |.sty| style file! Therefore it needs to be invoked in
a non-standard way.

%%%%%%%%%%%%%%%%%%%%%%%%%%%%%%%%%%%%%%%%%%%%%%%%%%%%%%%%%%%%%%%%%%%%%%%%%%%%%%%%
\subsection{Included Files}
\label{sec:include}

%%%%%%%%%%%%%%%%%%%%%%%%%%%%%%%%%%%%%%%%
\DescribeMacro{\childdocmain}
To use the package, add the commands
\begin{center}
\begin{tabular}{l}
|\input{childdoc.def}|\\
|\childdocmain{}|\\
\end{tabular}
\end{center}
at the very top of the main \LaTeX{} file,
in particular \emph{before} the |\documentclass| statement!
The argument of |\childdocmain| should be left empty
(but it must be present).

%%%%%%%%%%%%%%%%%%%%%%%%%%%%%%%%%%%%%%%%
\DescribeMacro{\childdocof}
Furthermore, add the commands
\begin{center}
\begin{tabular}{l}
|\input{childdoc.def}|\\
|\childdocof{|\textit{main}|}|\\
\end{tabular}
\end{center}
at the top of every child file \textit{child}
which is included by |\include{|\textit{child}|}|
from within the main file
(or at least for those files to be compiled individually).
The argument \textit{main} must be the filename of the main file.

There are a couple of
considerations in setting up the main and child documents:

%%%%%%%%%%%%%%%%%%%%%%%%%%%%%%%%%%%%%%%%
\paragraph{Restrictions.}

Please note the following restrictions:
\begin{itemize}
\item
|\childdocmain| must be called with one argument \textit{main}
to ensure compatibility with earlier version of the package.
It must either be empty (|\childdocmain{}|)
or precisely match the filename of the main file in which it is specified.
See \secref{sec:detection} for further information.
\item
The filename \textit{main} must be specified without the |.tex| extension.
\item
The filename \textit{main} is case sensitive
(even in case-insensitive file systems)
due to internal string comparison.
\item
The argument \textit{main} should be fully expanded, it cannot be a macro.
\item
Subdirectories and special characters should be avoided in filenames.
\item
The command |\childdocmain{|\textit{main}|}| must be followed by a whitespace.
It should not be followed immediately by another command
or by a comment mark `|%|'.
This is because the \TeX{} parser reads the token immediately following
the argument of |\childdocmain| and puts it
at the beginning of every child section;
however, a white\-space is ignored.
\end{itemize}

%%%%%%%%%%%%%%%%%%%%%%%%%%%%%%%%%%%%%%%%
\paragraph{Content of Main File.}

It is advisable to place all content in the child files included by |\include|.
Any output contained in the main file will appear in all child documents
unless suppressed manually;
it cannot be suppressed automatically by the |\includeonly| directive
and thus should normally be avoided.
A method to include some content in the main file
by means of conditional processing is described in \secref{sec:conditional}.

%%%%%%%%%%%%%%%%%%%%%%%%%%%%%%%%%%%%%%%%
\paragraph{Page Numbering.}

When only a part of the document is compiled,
the appropriate numbering of pages
(as well as other status parameters)
is determined from the |.aux| files.
The latter contain information from previous passes.
However this information needs to propagate through
all intermediate child documents.
Therefore the page numbering in child documents may well
be inconsistent until the complete document is compiled at least once.

A useful (if unconventional) way to always ensure a consistent
page numbering is to restart the numbering in each child document
and denote the pages by `\textit{child}|.|\textit{page}'
where \textit{child} represents the chapter/section number of the child file.
This can be achieved by the command
|\numberwithin{page}{|\textit{child}|}|
of the \textsf{amsmath} package
where \textit{child} can be |chapter| or |section|
depending on the chosen structuring.
Alternatively, one can modify the macro |\thepage| appropriately
and reset the counter |page| at the start of each child file.

%%%%%%%%%%%%%%%%%%%%%%%%%%%%%%%%%%%%%%%%%%%%%%%%%%%%%%%%%%%%%%%%%%%%%%%%%%%%%%%%
\subsection{Conditional Processing}
\label{sec:conditional}

The package provides a mechanism to compile different versions
of a document. To customise the versions further some conditional processing
can come in handy to distinguish which version is being compiled.
The package provides two macros to describe the compilation context:

%%%%%%%%%%%%%%%%%%%%%%%%%%%%%%%%%%%%%%%%
\DescribeMacro{\ifchilddoc}
The conditional |\ifchilddoc| distinguishes between the compilation of
child documents and the main document:
%
\begin{center}
|\ifchilddoc |\textit{child-code}| |[|\||else |\textit{main-code}]| \||fi|
\end{center}

%%%%%%%%%%%%%%%%%%%%%%%%%%%%%%%%%%%%%%%%
\DescribeMacro{\childdocname}
\DescribeMacro{\childdocjob}
The macro |\childdocname| contains the filename (without extension)
of the main or child file being processed.
Note that |\childdocjob| will always contain the name of the main file.

%%%%%%%%%%%%%%%%%%%%%%%%%%%%%%%%%%%%%%%%
\paragraph{Title Page.}

Conditional processing can be used to include a title or banner page
in the main document when proper precautions are taken.
Importantly, the code in the main file should ensure that the page counter
(as well as other status parameters which are stored in the |.aux| files)
takes the same value after the conditional processing.
Otherwise the page numbers may take divergent values
depending on which part is compiled.

For example, a title page could be declared by:
%
\begin{center}
\begin{tabular}{l}
|\ifchilddoc\||else|\\
|\addtocounter{page}{-1}|\\
\textit{code for title page}\\
|\newpage|\\
|\||fi|
\end{tabular}
\end{center}
%
A banner page for the child documents can be generated by:
%
\begin{center}
\begin{tabular}{l}
|\ifchilddoc|\\
|\addtocounter{page}{-1}|\\
\textit{code for banner page}\\
|\newpage|\\
|\||fi|
\end{tabular}
\end{center}
%
Here one could write a message such as:
\begin{center}
|This is the part \childdocname{} of \childdocjob{}.|
\end{center}

%%%%%%%%%%%%%%%%%%%%%%%%%%%%%%%%%%%%%%%%%%%%%%%%%%%%%%%%%%%%%%%%%%%%%%%%%%%%%%%%
\subsection{Flags}
\label{sec:flags}

The package makes it easy to generate different versions
of the main or child documents.
To this end compilation flags can be defined
and assigned different default values.
They will be particularly useful in conjunction
with the forwarding mechanism described in \secref{sec:forward}.

For example, it may be useful to have a flag |\version|
which can be set to |draft| or |final|.
The document source will contain some conditional code
depending on the value of |\version|.
Suppose further, the flag should default to |final| for the main file
and to |draft| for child files
which is a natural assignment for editing the document.
This is achieved by placing the following code
in the preamble of the main document
(below the |\childdocmain| directive):
%
\begin{center}
\begin{tabular}{l}
|\ifchilddoc|\\
|\providecommand{\version}{draft}|\\
|\||else|\\
|\providecommand{\version}{final}|\\
|\||fi|
\end{tabular}
\end{center}
%
The definition by |\providecommand| makes sure
that previous definitions are not overwritten.
Further statements |\providecommand{\version}{...}|
can thus be added before the above code to override it.

For the main file, one might add a line
(between |\childdocmain| and the above block)
%
\begin{center}
|%\ifchilddoc\||else\providecommand{\version}{draft}\||fi|
\end{center}
%
which can be uncommented to produce a draft version.
Likewise one can add a line to the very top of a child file
(above the |\childdocof{|\textit{main}|}| directive)
%
\begin{center}
|%\providecommand{\version}{final}|
\end{center}
%
which can be uncommented to produce the final version of this child document.

%%%%%%%%%%%%%%%%%%%%%%%%%%%%%%%%%%%%%%%%%%%%%%%%%%%%%%%%%%%%%%%%%%%%%%%%%%%%%%%%
\subsection{Forwarding}
\label{sec:forward}

Different versions of the main or child documents
using compilation flags as described in \secref{sec:flags}
can be (permanently) stored in different files
for convenient compilation, viewing and distribution.
To this end, the package defines a command
to pass on compilation to a different file:

%%%%%%%%%%%%%%%%%%%%%%%%%%%%%%%%%%%%%%%%
\DescribeMacro{\childdocforward}
The command |\childdocforward| redirects processing to
another source file:
%
\begin{center}
\begin{tabular}{l}
|\input{childdoc.def}|\\
|\childdocforward[|\textit{main}|]{|\textit{dest}|}|\\
\end{tabular}
\end{center}
%
The argument \textit{dest} is the destination file
(without extension).
It should be the main file or one of the child files.
Note that further \textsf{childdoc} directives
such as |\childdocof| and |\childdocforward|
in the indicated file will be processed in this form.
The optional argument \textit{main}
passes on directly to the main file \textit{main}
while pretending to compile the child \textit{dest}.
This form behaves as if \textit{dest}
issues |\childdocof{|\textit{main}|}| right away,
and no further \textsf{childdoc} directives will be processed.

%%%%%%%%%%%%%%%%%%%%%%%%%%%%%%%%%%%%%%%%
\DescribeMacro{\...prefix}
In the alternative form |\childdocforwardprefix|,
%
\begin{center}
\begin{tabular}{l}
|\input{childdoc.def}|\\
|\childdocforwardprefix[|\textit{main}|]{|\textit{prefix}|}{|\textit{dest}|}|
\end{tabular}
\end{center}
%
the destination file is determined by a pattern
depending on the current file:
To make this work, the current file must be called
`{\textit{prefix}\hspace{0.2em}\textit{suffix}}'
with \textit{prefix} matching precisely the argument.
Processing is then passed on to the file
`{\textit{dest}\hspace{0.2em}\textit{suffix}}'.
Surely, the same effect is achieved by
directly specifying the
argument `{\textit{dest}\hspace{0.2em}\textit{suffix}}'
in the first form.
However, that requires to set up a different file
for each child. With the alternative form of the command
all these files can have exactly the same content
which simplifies setting them up and maintaining them.

For example, the following file |draft.tex|
with a compilation flag |\version| as described in \secref{sec:flags}
compiles the main document as a draft:
%
\begin{center}
\begin{tabular}{l}
|\def\version{draft}|\\
|\input{childdoc.def}|\\
|\childdocforward{|\textit{main}|}|
\end{tabular}
\end{center}
%
Likewise, the following files |final|\textit{nn}|.tex|
compile the final version of the child document
|child|\textit{nn}|.tex|:
%
\begin{center}
\begin{tabular}{l}
|\def\version{final}|\\
|\input{childdoc.def}|\\
|\childdocforwardprefix{final}{child}|
\end{tabular}
\end{center}
%

Note that when several versions of a main file and/or of each child file
are to be generated, it may be convenient to set up a |Makefile| or
shell script to automatise the process.

%%%%%%%%%%%%%%%%%%%%%%%%%%%%%%%%%%%%%%%%%%%%%%%%%%%%%%%%%%%%%%%%%%%%%%%%%%%%%%%%
\subsection{Command Line Processing}
\label{sec:commandline}

The effect of redirection files can also be achieved by invoking
the \LaTeX{} compiler with a more elaborate command line.
Most conveniently this should be done as part
of a shell script or a |Makefile|.

When using \textsf{childdoc} in the main file, the following
command lines effectively perform a redirection
(note that depending on the shell being used,
backslashes may have to be doubled: `|\|' $\to$ `|\\|'):
%
\begin{center}
|... -jobname "|\textit{target}|" |\\|"|[\textit{flags}]%
|\input{childdoc.def}\childdocforward[|\textit{main}|]{|\textit{dest}|}"|
\end{center}
%
Here \textit{target} is the name of the output file,
\textit{main} is the name of the main file
and \textit{dest} is the name of the main or child file to be processed
(all filenames without extensions).
The optional argument \textit{main} can be omitted
if \textit{main} matches \textit{dest}.
Optionally, compilation \textit{flags} can be defined via |\def| commands.
This command line makes the \TeX{} engine believe
it is compiling the file \textit{target}
whose content is specified as the latter parameter.
The provided code then forwards the processing to
\textit{main} or \textit{dest} as described in \secref{sec:forward}.

%%%%%%%%%%%%%%%%%%%%%%%%%%%%%%%%%%%%%%%%%%%%%%%%%%%%%%%%%%%%%%%%%%%%%%%%%%%%%%%%
\subsection{Include by Input}
\label{sec:input}

Including child documents by |\include| has some restrictions by design.
Most notably, the content of a child document always occupies
its own set of pages; pages cannot be shared between child documents.
Usually, this behaviour makes perfect sense
because each child document contain an essential part of the document.
However, in some situations it may be desirable to compose
a document from a collection of parts
without having mandatory page breaks between then.
For this case, the package
provides a mechanism to include parts
by |\input| which can also be processed individually.
However, by construction this mechanism
requires manual handling of the content to be output.

%%%%%%%%%%%%%%%%%%%%%%%%%%%%%%%%%%%%%%%%
\DescribeMacro{\ifchilddocmanual}
The main file should be prepared as usual, see \secref{sec:include}.
However, the document body must make a distinction
between processing of an individual part and of the main document, e.g.:
%
\begin{center}
\begin{tabular}{l}
|\ifchilddocmanual|\\
|\input{\childdocname}|\\
|\||else|\\
\textit{document body with }|\input{|\textit{part}|}|\\
|\||fi|
\end{tabular}
\end{center}
%
The conditional |\ifchilddocmanual| is true whenever
a part to be included by |\input| is being compiled,
and the name of the part is stored in |\childdocname|.

%%%%%%%%%%%%%%%%%%%%%%%%%%%%%%%%%%%%%%%%
\DescribeMacro{\childdocby}
Each part to be included by |\input| should start with:
%
\begin{center}
\begin{tabular}{l}
|\input{childdoc.def}|\\
|\childdocby{|\textit{main}|}|\\
\end{tabular}
\end{center}
%
The directive |\childdocby| is similar to |\childdocof|
described in \secref{sec:include},
but the subsequent selection of content must be done manually.
To that end, both |\ifchilddoc| and |\ifchilddocmanual|
will be true upon processing of a part,
and the name of the part is stored in |\childdocname|.
Note that |\jobname| will be set to the filename of the current part
so that each part receives an individual |.aux| file
that does not interfere with the |.aux| file(s) of the main document.
This behaviour can be altered by the alternative form
|\childdocby[*]{|\textit{main}|}| (with a non-empty optional argument)
which uses the |.aux| file of the main document
by setting |\jobname| to \textit{main}.

%%%%%%%%%%%%%%%%%%%%%%%%%%%%%%%%%%%%%%%%%%%%%%%%%%%%%%%%%%%%%%%%%%%%%%%%%%%%%%%%
\subsection{Driver Development}
\label{sec:driver}

The \textsf{childdoc} mechanism can also be use for the development
of definition files such as \LaTeX{} styles or classes.
This case differs from the above setup with multiple parts
included by |\include| in that no |\includeonly| should be invoked.
This can be achieved by starting the include file
(before |\ProvidesPackage|) with:
%
\begin{center}
\begin{tabular}{l}
|\input{childdoc.def}|\\
|\childdocforward{|\textit{main}|}|\\
\end{tabular}
\end{center}
%
or alternatively with:
%
\begin{center}
\begin{tabular}{l}
|\input{childdoc.def}|\\
|\childdocby{|\textit{main}|}|\\
\end{tabular}
\end{center}
%
Both forms have slightly different effects as described above.
The main file is prepared as usual, see \secref{sec:include}.

%%%%%%%%%%%%%%%%%%%%%%%%%%%%%%%%%%%%%%%%%%%%%%%%%%%%%%%%%%%%%%%%%%%%%%%%%%%%%%%%
\subsection{Legacy Detection}
\label{sec:detection}

The directive |\childdocmain| in the main file can detect
whether the complete document or merely a child is to be compiled
even without using the directive |\childdocof|.
This method is deprecated because it is less robust
and there is no compelling reason to use it;
it is merely provided for backward compatibility
and it may be removed in future versions.

If the detection mechanism is to be used,
it is mandatory to correctly specify
the filename of the main file as the argument of |\childdocmain|:
%
\begin{center}
\begin{tabular}{l}
|\input{childdoc.def}|\\
|\childdocmain{|\textit{main}|}|\\
\end{tabular}
\end{center}
%
If |\jobname| does not match the argument \textit{main} of |\childdocmain|,
it is assumed that |\jobname| points to the child file to be compiled.
When using |\childdocmain| with the main file specified as argument,
it suffices to start a child file
with just |\input{|\textit{main}|}|
without loading of the package and using |\childdocof|.
If instead all processing is done
with the appropriate \textsf{childdoc} directives,
the argument of \textit{main} of |\childdocmain| can be empty.

An alternative version of the command line processing described
in \secref{sec:commandline} using the detection mechanism reads:
%
\begin{center}
|... -jobname "|\textit{target}|" "|[\textit{flags}]%
[|\def\jobname{|\textit{dest}|}|]|\input{|\textit{main}|}"|
\end{center}

%%%%%%%%%%%%%%%%%%%%%%%%%%%%%%%%%%%%%%%%%%%%%%%%%%%%%%%%%%%%%%%%%%%%%%%%%%%%%%%%
\subsection{Manual Code}
\label{sec:manual}

In case one cannot be certain whether the definitions file |childdoc.def|
is installed on the target \TeX{} distribution
and one prefers not to ship it,
it is conceivable to paste a few relevant commands into the sources.

To that end, drop all statements |\input{childdoc.def}|
and perform the replacements as outlined below.
Instead of |\childdocmain{|\textit{main}|}| add the following code
to the top of the main file:
%
\begin{center}
\begin{tabular}{l}
|\||ifdefined\childdocname\endinput\||fi\newif\ifchilddoc|\\
|\edef\childdocname{\scantokens\expandafter{\jobname\noexpand}}|\\
|\def\childdocmain{|\textit{main}|}\||ifx\childdocmain\childdocname\||else|\\
|\childdoctrue\includeonly{\childdocname}\let\jobname\childdocmain\||fi|\\
\end{tabular}
\end{center}
%
Instead of |\childdocof{|\textit{main}|}| just include the main file
at the top of each child file:
%
\begin{center}
|\input{|\textit{main}|}|
\end{center}
%
A simple redirection |\childdocforward{|\textit{dest}|}| is achieved by:
%
\begin{center}
|\def\jobname{|\textit{dest}|}\input{\jobname}|
\end{center}
%
The redirection with prefix
|\childdocforwardprefix[|\textit{prefix}|]{|\textit{dest}|}|
is accomplished by:
%
\begin{center}
\begin{tabular}{l}
|{\edef\jobname{\scantokens\expandafter{\jobname\noexpand}}|\\
|\def\redirectjob |\textit{prefix}|#1~~~{\gdef\jobname{|\textit{dest}|#1}}|\\
|\expandafter\redirectjob\jobname~~~}\input{\jobname}|
\end{tabular}
\end{center}

In an alternative approach,
child documents can be compiled by a specific command line
without additional code or specific definitions:
%
\begin{center}
|... -jobname "|\textit{target}|" "|[\textit{flags}]%
|\includeonly{|\textit{dest}|}\input{|\textit{main}|}"|
\end{center}
%

%%%%%%%%%%%%%%%%%%%%%%%%%%%%%%%%%%%%%%%%%%%%%%%%%%%%%%%%%%%%%%%%%%%%%%%%%%%%%%%%
%%%%%%%%%%%%%%%%%%%%%%%%%%%%%%%%%%%%%%%%%%%%%%%%%%%%%%%%%%%%%%%%%%%%%%%%%%%%%%%%
\section{Information}

%%%%%%%%%%%%%%%%%%%%%%%%%%%%%%%%%%%%%%%%%%%%%%%%%%%%%%%%%%%%%%%%%%%%%%%%%%%%%%%%
\subsection{Copyright}

Copyright \copyright{} 2017--2018 Niklas Beisert

This work may be distributed and/or modified under the
conditions of the \LaTeX{} Project Public License, either version 1.3
of this license or (at your option) any later version.
The latest version of this license is in
  \url{http://www.latex-project.org/lppl.txt}
and version 1.3 or later is part of all distributions of \LaTeX{}
version 2005/12/01 or later.

This work has the LPPL maintenance status `maintained'.

The Current Maintainer of this work is Niklas Beisert.

This work consists of the files |README.txt|, |childdoc.ins| and |childdoc.dtx|
as well as the derived files |childdoc.def|, |cdocsamp.tex|
with |cdocsch1.tex|, |cdocsch2.tex|, |cdocspt3.tex|, |cdocspt4.tex|,
|cdocsdrf.tex|, |cdocsfn1.tex|, |cdocsfn2.tex|
as well as |childdoc.pdf|.

%%%%%%%%%%%%%%%%%%%%%%%%%%%%%%%%%%%%%%%%%%%%%%%%%%%%%%%%%%%%%%%%%%%%%%%%%%%%%%%%
\subsection{Files and Installation}

The package consists of the files:
%
\begin{center}
\begin{tabular}{ll}
    |README.txt|   & readme file \\
    |childdoc.ins| & installation file \\
    |childdoc.dtx| & source file \\
    |childdoc.def| & definition file \\
    |cdocsamp.tex| & sample main file \\
    |cdocsch1.tex| & sample include file \\
    |cdocsch2.tex| & sample include file \\
    |cdocspt3.tex| & sample part file \\
    |cdocspt4.tex| & sample part file \\
    |cdocsdrf.tex| & sample redirection file \\
    |cdocsfn1.tex| & sample redirection file \\
    |cdocsfn2.tex| & sample redirection file \\
    |childdoc.pdf| & manual
\end{tabular}
\end{center}
%
The distribution consists of the files
|README.txt|, |childdoc.ins| and |childdoc.dtx|.
%
\begin{itemize}
\item
Run (pdf)\LaTeX{} on |childdoc.dtx|
to compile the manual |childdoc.pdf| (this file).
\item
Run \LaTeX{} on |childdoc.ins| to create the definitions file |childdoc.def|
and the sample |cdocsamp.tex| with include files
|cdocsch1.tex|, |cdocsch2.tex|, |cdocspt3.tex|, |cdocspt4.tex|,
|cdocsdrf.tex|, |cdocsfn1.tex|, |cdocsfn2.tex|.
Then copy the file |childdoc.def| to an appropriate directory of your \LaTeX{}
distribution, e.g.\ \textit{texmf-root}|/tex/latex/childdoc|.
\end{itemize}

%%%%%%%%%%%%%%%%%%%%%%%%%%%%%%%%%%%%%%%%%%%%%%%%%%%%%%%%%%%%%%%%%%%%%%%%%%%%%%%%
\subsection{Related CTAN Packages}

There are several other packages which offer a similar functionality:
%
\begin{itemize}
\item
The packages
\href{http://ctan.org/pkg/docmute}{\textsf{docmute}},
\href{http://ctan.org/pkg/includex}{\textsf{includex}} and
\href{http://ctan.org/pkg/standalone}{\textsf{standalone}}
provide commands to include only the document body of
a child file thus allowing both files to be compiled individually.
\item
The packages \href{http://ctan.org/pkg/subdocs}{\textsf{subdocs}}
and \href{http://ctan.org/pkg/subfiles}{\textsf{subfiles}}
provide structures in which the main and child documents can be
encapsulated and allowing them to be compiled individually.
The inclusion mechanism is different from the conventional |\include|.
\item
The package \href{http://ctan.org/pkg/combine}{\textsf{combine}}
is an elaborate solution to combine several documents into one.
\end{itemize}
%
See also the CTAN topic \href{http://ctan.org/topic/subdocs}{\textsf{subdocs}}
for further related packages.
The present package differs from the above solutions in that
a document structure constructed with the conventional |\include| mechanism
just needs two extra commands at the top of every file
such that all constituent files can be compiled individually.

%%%%%%%%%%%%%%%%%%%%%%%%%%%%%%%%%%%%%%%%%%%%%%%%%%%%%%%%%%%%%%%%%%%%%%%%%%%%%%%%
%\subsection{Feature Suggestions}
%
%The following is a list of features which may be useful for future
%versions of this package:
%%
%\begin{itemize}
%\item
%\ldots
%\end{itemize}

%%%%%%%%%%%%%%%%%%%%%%%%%%%%%%%%%%%%%%%%%%%%%%%%%%%%%%%%%%%%%%%%%%%%%%%%%%%%%%%%
\subsection{Revision History}

%%%%%%%%%%%%%%%%%%%%%%%%%%%%%%%%%%%%%%%%
\paragraph{v2.0:} 2018/12/30

\begin{itemize}
\item
immediate forward processing
\item
added |\childdocby| mechanism
\item
manual restructured
\end{itemize}

%%%%%%%%%%%%%%%%%%%%%%%%%%%%%%%%%%%%%%%%
\paragraph{v1.6:} 2018/01/17

\begin{itemize}
\item
application for development of include files
\item
corrections to manual
\end{itemize}

%%%%%%%%%%%%%%%%%%%%%%%%%%%%%%%%%%%%%%%%
\paragraph{v1.5:} 2017/05/21

\begin{itemize}
\item
more complete structuring introduced
\item
|\childdocof| introduced
\item
|\childdoc| renamed to |\childdocmain|
\item
|\childredirect| renamed to |\childdocforward| and |\childdocforwardprefix|
and functionality expanded
\end{itemize}

%%%%%%%%%%%%%%%%%%%%%%%%%%%%%%%%%%%%%%%%
\paragraph{v1.0:} 2017/04/27

\begin{itemize}
\item
manual and install package
\item
first version published on CTAN
\end{itemize}

%%%%%%%%%%%%%%%%%%%%%%%%%%%%%%%%%%%%%%%%
\paragraph{v0.6:} 2017/04/26

\begin{itemize}
\item
redirection mechanism added
\end{itemize}

%%%%%%%%%%%%%%%%%%%%%%%%%%%%%%%%%%%%%%%%
\paragraph{v0.5:} 2017/04/26

\begin{itemize}
\item
functionality in definition file
\end{itemize}


%%%%%%%%%%%%%%%%%%%%%%%%%%%%%%%%%%%%%%%%%%%%%%%%%%%%%%%%%%%%%%%%%%%%%%%%%%%%%%%%
%%%%%%%%%%%%%%%%%%%%%%%%%%%%%%%%%%%%%%%%%%%%%%%%%%%%%%%%%%%%%%%%%%%%%%%%%%%%%%%%
%%%%%%%%%%%%%%%%%%%%%%%%%%%%%%%%%%%%%%%%%%%%%%%%%%%%%%%%%%%%%%%%%%%%%%%%%%%%%%%%
\appendix

\settowidth\MacroIndent{\rmfamily\scriptsize 000\ }

 \DocInput{childdoc.dtx}

\end{document}
%</driver>
% \fi
%
% %%%%%%%%%%%%%%%%%%%%%%%%%%%%%%%%%%%%%%%%%%%%%%%%%%%%%%%%%%%%%%%%%%%%%%%%%%%%%%
% %%%%%%%%%%%%%%%%%%%%%%%%%%%%%%%%%%%%%%%%%%%%%%%%%%%%%%%%%%%%%%%%%%%%%%%%%%%%%%
% \section{Sample}
%\iffalse
%<*samplemain>
%\fi
%
% The following presents a sample document
% with two chapters, two parts, a title page,
% a compile flag as well as three forwarding files to set the flag.
% It consists of eight |.tex| files:
% \begin{center}
% \begin{tabular}{ll}
% |cdocsamp.tex|&main file\\
% |cdocsch1.tex|&include file for chapter 1\\
% |cdocsch2.tex|&include file for chapter 2\\
% |cdocspt3.tex|&include file for part 3\\
% |cdocspt4.tex|&include file for part 4\\
% |cdocsdrf.tex|&forwarding file for main file in draft mode\\
% |cdocsfi1.tex|&forwarding file for final version of chapter 1\\
% |cdocsfi2.tex|&forwarding file for final version of chapter 2\\
% \end{tabular}
% \end{center}
% Each of the eight files can be compiled directly by the \LaTeX{} compiler.
%
% %%%%%%%%%%%%%%%%%%%%%%%%%%%%%%%%%%%%%%
% \paragraph{Main File.}
%
% The main file is called |cdocsamp.tex|.
%
% Load the \textsf{childdoc} definitions and
% declare the filename for the main document:
%    \begin{macrocode}
\input{childdoc.def}
\childdocmain{}
%    \end{macrocode}

% Optional override for |\version| flag:
%    \begin{macrocode}
%%\ifchilddoc\else\providecommand{\version}{draft}\fi
%    \end{macrocode}

% Define the default values for the |\version| flag
% (|final| for the main file and |draft| for childs):
%    \begin{macrocode}
\ifchilddoc
\providecommand{\version}{draft}
\else
\providecommand{\version}{final}
\fi
%    \end{macrocode}

% Load the standard document class:
%    \begin{macrocode}
\documentclass[12pt]{article}
%    \end{macrocode}

% Start the document body:
%    \begin{macrocode}
\begin{document}
%    \end{macrocode}

% Declare a title page.
% Print title, part of document being processed and version flag:
%    \begin{macrocode}
\addtocounter{page}{-1}
\begin{center}
{\LARGE\bfseries{}childdoc example\par}
\vspace{1cm}
\ifchilddoc
\ifchilddocmanual part\else chapter\fi:
`\childdocname' of `\childdocjob'\par
\else
main document: `\childdocjob'\par
\fi
version: \version\par
\end{center}
\newpage
%    \end{macrocode}

% Manually include selected file,
% otherwise process as usual:
%    \begin{macrocode}
\ifchilddocmanual
\section*{part `\childdocname'}
\input{\childdocname}
\else
%    \end{macrocode}

% Include the two chapters:
%    \begin{macrocode}
\include{cdocsch1}
\include{cdocsch2}
%    \end{macrocode}

% Include the two parts unless only chapters should be displayed:
%    \begin{macrocode}
\ifchilddoc\else
\section{part three}
\input{cdocspt3}
\section{part four}
\input{cdocspt4}
\fi
%    \end{macrocode}

% Process as usual until here:
%    \begin{macrocode}
\fi
%    \end{macrocode}

% End of document body:
%    \begin{macrocode}
\end{document}
%    \end{macrocode}
%\iffalse
%</samplemain>
%\fi
%
% %%%%%%%%%%%%%%%%%%%%%%%%%%%%%%%%%%%%%%
% \paragraph{Chapter Include Files.}
%
% The include files are called |cdocsch1.tex| and |cdocsch2.tex|.
%
%\iffalse
%<*samplechap1|samplechap2>
%\fi

% Optional override for |\version| flag:
%    \begin{macrocode}
%%\providecommand{\version}{final}
%    \end{macrocode}

% Include the main document:
%    \begin{macrocode}
\input{childdoc.def}
\childdocof{cdocsamp}
%    \end{macrocode}

%\iffalse
%</samplechap1|samplechap2>
%\fi
%
%\iffalse
%<*samplechap1>
%\fi
% Some text for chapter 1:
%    \begin{macrocode}
\section{one}
some text in chapter one
%    \end{macrocode}

%\iffalse
%</samplechap1>
%\fi
% Some text for chapter 2:
%\iffalse
%<*samplechap2>
%\fi
%    \begin{macrocode}
\section{two}
more text in chapter two
%    \end{macrocode}

%\iffalse
%</samplechap2>
%\fi
%
% %%%%%%%%%%%%%%%%%%%%%%%%%%%%%%%%%%%%%%
% \paragraph{Part Include Files.}
%
% The include files are called |cdocspt3.tex| and |cdocspt4.tex|.
%
%\iffalse
%<*samplepart3|samplepart4>
%\fi

% Optional override for |\version| flag:
%    \begin{macrocode}
%%\providecommand{\version}{final}
%    \end{macrocode}

% Include the main document:
%    \begin{macrocode}
\input{childdoc.def}
\childdocby{cdocsamp}
%    \end{macrocode}

%\iffalse
%</samplepart3|samplepart4>
%\fi
%
%\iffalse
%<*samplepart3>
%\fi
% Some text for part 3:
%    \begin{macrocode}
some text in part three
%    \end{macrocode}

%\iffalse
%</samplepart3>
%\fi
% Some text for part 4:
%\iffalse
%<*samplepart4>
%\fi
%    \begin{macrocode}
more text in part four
%    \end{macrocode}

%\iffalse
%</samplepart4>
%\fi
%
% %%%%%%%%%%%%%%%%%%%%%%%%%%%%%%%%%%%%%%
% \paragraph{Forwarding for a Complete Draft.}
%
% The following forwarding file |cdocsdrf.tex|
% compiles the main document in draft mode:
%\iffalse
%<*sampledraft>
%\fi
%    \begin{macrocode}
\def\version{draft}
\input{childdoc.def}
\childdocforward{cdocsamp}
%    \end{macrocode}

%\iffalse
%</sampledraft>
%\fi
%
% %%%%%%%%%%%%%%%%%%%%%%%%%%%%%%%%%%%%%%
% \paragraph{Forwarding for Final Version of the Chapters.}
%
% The following forwarding files |cdocsfn1.tex| and |cdocsfn2.tex|
% (with identical content)
% compile the final versions of the child documents
% |cdocsch1.tex| and |cdocsch2.tex|, respectively:
%\iffalse
%<*samplefinal>
%\fi
%    \begin{macrocode}
\def\version{final}
\input{childdoc.def}
\childdocforwardprefix[cdocsamp]{cdocsfn}{cdocsch}
%    \end{macrocode}

%\iffalse
%</samplefinal>
%\fi
%
% %%%%%%%%%%%%%%%%%%%%%%%%%%%%%%%%%%%%%%
% \paragraph{Command Line Processing.}
%
% The following three command lines generate the output files
% |cdocscld|, |cdocscl1| and |cdocscl2|
% which should be identical to
% |cdocsdrf|, |cdocsch1| and |cdocsfn2|, respectively:
% \begin{center}
% \begin{tabular}{l}
% |latex -jobname cdocscld \|\\
% |  "\def\version{draft}\input{childdoc.def}\childdocforward{cdocsamp}"|\\
% |latex -jobname cdocscl1 \|\\
% |  "\input{childdoc.def}\childdocforward[cdocsamp]{cdocsch1}"|\\
% |latex -jobname cdocscl2 \|\\
% |  "\def\version{final}\input{childdoc.def}\childdocforward{cdocsch2}"|
% \end{tabular}
% \end{center}
% Note that the trailing backslash on each first line
% merely continues the input to the second line
% (for convenient cut ant paste).
% Furthermore, the command |latex| can be replaced by any
% of its alternative versions such as |pdflatex|.
%
% %%%%%%%%%%%%%%%%%%%%%%%%%%%%%%%%%%%%%%%%%%%%%%%%%%%%%%%%%%%%%%%%%%%%%%%%%%%%%%
% %%%%%%%%%%%%%%%%%%%%%%%%%%%%%%%%%%%%%%%%%%%%%%%%%%%%%%%%%%%%%%%%%%%%%%%%%%%%%%
% \section{Implementation}
%\iffalse
%<*package>
%\fi
%
% This section describes the definitions file |childdoc.def|.

% The definitions cannot be loaded using |\usepackage| or |\RequirePackage|
% which has a mechanism to prevent loading a style file more than once.
% When loading the definitions by means of |\input|
% multiple instances have to be prevented manually:
%\iffalse
%This code needs to be before the `\ProvidesFile' directive
%which is defined at the beginning of this file.
%Therefore it is also placed there and commented out here.
%</package>
%<*discard>
%\fi
%    \begin{macrocode}
\ifdefined\childdocmain\endinput\fi
%    \end{macrocode}
%\iffalse
%</discard>
%<*package>
%\fi
%
% \macro{\ifchilddoc}
% \macro{\ifchilddocmanual}
% The conditional |\ifchilddoc| tells whether a
% child (true) or main (false) document is being compiled.
% The conditional |\ifchilddocmanual| tells whether
% the |\includeonly| mechanism is used (false) or
% the selection of child files must be performed manually (true).
% The definitions initialise to false:
%    \begin{macrocode}
\newif\ifchilddoc
\newif\ifchilddocmanual
%    \end{macrocode}

% \macro{\childdocname}
% \macro{\childdocjob}
% The macro |\childdocname| stores the name of the main document
% to be compiled. The macro |\childdocjob| stores the name of
% the document on which the \LaTeX{} compiler was originally invoked.
% The content of |\jobname| cannot be compared
% to filenames specified in the source due to different catcodes.
% The following code rescans |\jobname|, stores the result
% in |\childdocname| and saves a copy in |\childdocjob|:
%    \begin{macrocode}
\edef\childdocname{\scantokens\expandafter{\jobname\noexpand}}
\let\childdocjob\childdocname
%    \end{macrocode}

% \macro{\childdocdisable}
% The macro |\childdocdisable| prevents the main file
% from being processed more than once.
% At this stage, the main document command |\childdocmain|
% is assumed to be called once again where it should do nothing.
% Any subsequent call to it should prevent
% a secondary processing of the main document
% It overwrites the forwarding commands
% |\childdocof| and |\childdocforward|
% with empty macros to prevent further inclusions of the main document:
%    \begin{macrocode}
\newcommand{\childdocdisable}
{
  \renewcommand{\childdocmain}[1]{\renewcommand{\childdocmain}[1]{\endinput}}
  \renewcommand{\childdocof}[1]{}
  \renewcommand{\childdocby}[2][]{}
  \renewcommand{\childdocforward}[2][]{}
  \renewcommand{\childdocdisable}{}
}
%    \end{macrocode}

% \macro{\childdocmain}
% The macro |\childdocmain| is to be called at the top of the main file
% with nothing or the main filename (without extension) as argument.
% First, it breaks loops.
% If the argument is not empty and does not match |\childdocname|
% (which is set by the first inclusion of |childdoc.def|),
% |\ifchilddoc| is set to true, |\includeonly| is applied to the child file
% and |\jobname| is set to the main file
% (for proper handling of |.aux| files):
%    \begin{macrocode}
\newcommand{\childdocmain}[1]
{
  \childdocdisable\childdocmain{}
  \if?#1?\else
    \begingroup
      \def\childdoctmp{#1}
      \ifx\childdoctmp\childdocname
        \def\childdoctmp{}
      \else
        \def\childdoctmp
        {
          \childdoctrue
          \includeonly{\childdocname}
          \def\childdocjob{#1}
          \def\jobname{#1}
        }
      \fi
      \expandafter
    \endgroup
    \childdoctmp
  \fi
}
%    \end{macrocode}

% \macro{\childdocof}
% The command |\childdocof| redirects
% compilation to the main file |#1|.
%    \begin{macrocode}
\newcommand{\childdocof}[1]
{
  \childdocdisable
  \childdoctrue
  \includeonly{\childdocname}
  \def\jobname{#1}
  \def\childdocjob{#1}
  \input{#1}
}
%    \end{macrocode}

% \macro{\childdocby}
% The command |\childdocby| ....
%    \begin{macrocode}
\newcommand{\childdocby}[2][]
{
  \childdocdisable
  \childdoctrue
  \childdocmanualtrue
  \if?#1?\else
    \def\jobname{#2}
  \fi
  \def\childdocjob{#2}
  \input{#2}
  \endinput
}
%    \end{macrocode}

% \macro{\childdocforward}
% The command |\childdocforward| redirects
% compilation to the main file or
% (if the optional argument is given) a child file.
% Parameters are set as if the main file
% or a child file starting with |\childdocof| was compiled.
% Then compilation is handed over to the main file:
%    \begin{macrocode}
\newcommand{\childdocforward}[2][]
{
  \begingroup
    \if?#1?
      \def\childdoctmp
      {
        \def\childdocname{#2}
        \def\childdocjob{#2}
        \def\jobname{#2}
        \input{#2}
        \endinput
      }
    \else
      \def\childdoctmp
      {
        \childdocdisable
        \def\childdocname{#2}
        \childdoctrue
        \includeonly{#2}
        \def\childdocjob{#1}
        \def\jobname{#1}
        \input{#1}
        \endinput
      }
    \fi
    \expandafter
  \endgroup
  \childdoctmp
}
%    \end{macrocode}

% \macro{\childdocforwardprefix}
% The command |\childdocforwardprefix| redirects
% compilation to the main or a child file by means of a pattern.
% The prefix |#1| in the current filename is replaced by |#2|
% and the suffix of the current filename is kept
% (it is assumed that the filename does not contain the substring `|~~~|'
% which is used as a delimiter).
% Compilation is handed over to the new file by |\childdocforward|:
%    \begin{macrocode}
\newcommand{\childdocforwardprefix}[3][]
{
  \begingroup
    \def\childdocextract #2##1~~~{\def\childdoctmp{\childdocforward[#1]{#3##1}}}
    \expandafter\childdocextract\childdocname~~~
    \expandafter
  \endgroup
  \childdoctmp
}
%    \end{macrocode}

% \macro{\childdoc}
% The deprecated macro |\childdoc| is a legacy version of |\childdocmain|:
%    \begin{macrocode}
\newcommand{\childdoc}{\childdocmain}
%    \end{macrocode}

% \macro{\childdocredirect}
% The deprecated macro |\childdocredirect| is a legacy version
% of |\childdocforward| and |\childdocforwardprefix|:
%    \begin{macrocode}
\newcommand{\childdocredirect}[2][]
{
  \begingroup
    \if?#1?
      \def\childdoctmp{\childdocforward{#2}}
    \else
      \def\childdoctmp{\childdocforwardprefix{#1}{#2}}
    \fi
    \expandafter
  \endgroup
  \childdoctmp
}
%    \end{macrocode}

%\iffalse
%</package>
%\fi
%
\endinput
|\\
|\childdocforwardprefix{final}{child}|
\end{tabular}
\end{center}
%

Note that when several versions of a main file and/or of each child file
are to be generated, it may be convenient to set up a |Makefile| or
shell script to automatise the process.

%%%%%%%%%%%%%%%%%%%%%%%%%%%%%%%%%%%%%%%%%%%%%%%%%%%%%%%%%%%%%%%%%%%%%%%%%%%%%%%%
\subsection{Command Line Processing}
\label{sec:commandline}

The effect of redirection files can also be achieved by invoking
the \LaTeX{} compiler with a more elaborate command line.
Most conveniently this should be done as part
of a shell script or a |Makefile|.

When using \textsf{childdoc} in the main file, the following
command lines effectively perform a redirection
(note that depending on the shell being used,
backslashes may have to be doubled: `|\|' $\to$ `|\\|'):
%
\begin{center}
|... -jobname "|\textit{target}|" |\\|"|[\textit{flags}]%
|% \iffalse
%
% childdoc.dtx Copyright (C) 2017-2018 Niklas Beisert
%
% This work may be distributed and/or modified under the
% conditions of the LaTeX Project Public License, either version 1.3
% of this license or (at your option) any later version.
% The latest version of this license is in
%   http://www.latex-project.org/lppl.txt
% and version 1.3 or later is part of all distributions of LaTeX
% version 2005/12/01 or later.
%
% This work has the LPPL maintenance status `maintained'.
%
% The Current Maintainer of this work is Niklas Beisert.
%
% This work consists of the files childdoc.dtx and childdoc.ins
% and the derived files childdoc.def and cdocsamp.tex with
% cdocsch1.tex, cdocsch2.tex, cdocsdrf.tex, cdocsfn1.tex, cdocsfn2.tex.
%
%<package>\ifdefined\childdocmain\endinput\fi
%<package>\ProvidesFile{childdoc.def}[2018/12/30 v2.0 child document driver]
%<samplemain>\ProvidesFile{cdocsamp.tex}[2018/12/30 v2.0 sample for childdoc]
%<*driver>
%\ProvidesFile{childdoc.drv}[2018/12/30 v2.0 childdoc reference manual file]
\PassOptionsToClass{10pt,a4paper}{article}
\documentclass{ltxdoc}

\usepackage[margin=35mm]{geometry}
\usepackage{hyperref}
\usepackage{hyperxmp}
\usepackage[usenames]{color}

\hypersetup{colorlinks=true}
\hypersetup{pdfstartview=FitH}
\hypersetup{pdfpagemode=UseNone}
\hypersetup{pdfsource={}}
\hypersetup{pdflang={en-UK}}
\hypersetup{pdfcopyright={Copyright 2017-2018 Niklas Beisert.
  This work may be distributed and/or modified under the
  conditions of the LaTeX Project Public License, either version 1.3
  of this license or (at your option) any later version.}}
\hypersetup{pdflicenseurl={http://www.latex-project.org/lppl.txt}}
\hypersetup{pdfcontactaddress={ETH Zurich, ITP, HIT K,
  Wolfgang-Pauli-Strasse 27}}
\hypersetup{pdfcontactpostcode={8093}}
\hypersetup{pdfcontactcity={Zurich}}
\hypersetup{pdfcontactcountry={Switzerland}}
\hypersetup{pdfcontactemail={nbeisert@itp.phys.ethz.ch}}
\hypersetup{pdfcontacturl={http://people.phys.ethz.ch/\xmptilde nbeisert/}}

\newcommand{\secref}[1]{\hyperref[#1]{section \ref*{#1}}}

\parskip1ex
\parindent0pt
\let\olditemize\itemize
\def\itemize{\olditemize\parskip0pt}

\begin{document}

\title{The \textsf{childdoc} Package}
\hypersetup{pdftitle={The childdoc Package}}
\author{Niklas Beisert\\[2ex]
  Institut f\"ur Theoretische Physik\\
  Eidgen\"ossische Technische Hochschule Z\"urich\\
  Wolfgang-Pauli-Strasse 27, 8093 Z\"urich, Switzerland\\[1ex]
  \href{mailto:nbeisert@itp.phys.ethz.ch}
  {\texttt{nbeisert@itp.phys.ethz.ch}}}
\hypersetup{pdfauthor={Niklas Beisert}}
\hypersetup{pdfsubject={Manual for the LaTeX2e Package childdoc}}
\date{30 December 2018, \textsf{v2.0}}
\maketitle

\begin{abstract}\noindent
\textsf{childdoc} is a \LaTeXe{} package
that enables the direct compilation
of document sections included by |\include|
to individual files.
\end{abstract}

\begingroup
\parskip0ex
\tableofcontents
\endgroup

%%%%%%%%%%%%%%%%%%%%%%%%%%%%%%%%%%%%%%%%%%%%%%%%%%%%%%%%%%%%%%%%%%%%%%%%%%%%%%%%
%%%%%%%%%%%%%%%%%%%%%%%%%%%%%%%%%%%%%%%%%%%%%%%%%%%%%%%%%%%%%%%%%%%%%%%%%%%%%%%%
\section{Introduction}

\LaTeX{} provides a mechanism to structure a large document (such as a book)
into a main file and several child files (containing the chapters)
using the |\include| command.
This mechanism is beneficial for documents
which span hundreds of pages in order to
make the source file(s) more manageable.
Moreover, compilation can be restricted to
selected child files by means of the |\includeonly| command.
The latter feature can be used to reduce the compilation time while editing
(this was significantly more useful in the earlier days of \LaTeX{})
or to generate a smaller document which is easier to navigate.
Another application of |\includeonly| is to generate
documents consisting of selected parts of the complete document.

However, there are a few drawbacks of the plain |\include| mechanism:
\begin{itemize}
\item
The child files cannot be compiled on their own,
they can only be compiled via the main file.
A naive editing environment
(such as a text editor with an option
to have the current file processed by \LaTeX)
may require one to switch to the main file before compiling;
attempting to compile the child file produces errors.
\item
The main file must be modified (each time)
to adjust the |\includeonly| command
to the present needs. This easily leaves the main file in a messy state.
\item
The generated document will always carry the filename
of the main document. This is inconvenient if
several child files are to be compiled and
to be kept for distribution.
\end{itemize}

The present package provides a simple interface
to make child files individually compilable by \LaTeX{}.
Compiling a child file then has the same effect as compiling
the main file with an |\includeonly| command
to select the appropriate child.
Moreover the generated document will carry the name of the child
rather than the main file.
This resolves all three above issues.

This feature is meant to make the editing of books,
thesis documents and lecture notes somewhat more convenient.
However, the package can also be used efficiently for
composing a series of documents (such as exercise sheets)
which are typically distributed individually.
It then assists the author in generating the individual documents
(potentially in different versions)
as well as a document containing the collected series.
Another application is in developing style files
or other kinds of included material
where compilation of the style file could redirect
to a sample or test file.

%%%%%%%%%%%%%%%%%%%%%%%%%%%%%%%%%%%%%%%%%%%%%%%%%%%%%%%%%%%%%%%%%%%%%%%%%%%%%%%%
%%%%%%%%%%%%%%%%%%%%%%%%%%%%%%%%%%%%%%%%%%%%%%%%%%%%%%%%%%%%%%%%%%%%%%%%%%%%%%%%
\section{Usage}

First of all, the package \textsf{childdoc} is \emph{not} a standard
\LaTeXe{} |.sty| style file! Therefore it needs to be invoked in
a non-standard way.

%%%%%%%%%%%%%%%%%%%%%%%%%%%%%%%%%%%%%%%%%%%%%%%%%%%%%%%%%%%%%%%%%%%%%%%%%%%%%%%%
\subsection{Included Files}
\label{sec:include}

%%%%%%%%%%%%%%%%%%%%%%%%%%%%%%%%%%%%%%%%
\DescribeMacro{\childdocmain}
To use the package, add the commands
\begin{center}
\begin{tabular}{l}
|\input{childdoc.def}|\\
|\childdocmain{}|\\
\end{tabular}
\end{center}
at the very top of the main \LaTeX{} file,
in particular \emph{before} the |\documentclass| statement!
The argument of |\childdocmain| should be left empty
(but it must be present).

%%%%%%%%%%%%%%%%%%%%%%%%%%%%%%%%%%%%%%%%
\DescribeMacro{\childdocof}
Furthermore, add the commands
\begin{center}
\begin{tabular}{l}
|\input{childdoc.def}|\\
|\childdocof{|\textit{main}|}|\\
\end{tabular}
\end{center}
at the top of every child file \textit{child}
which is included by |\include{|\textit{child}|}|
from within the main file
(or at least for those files to be compiled individually).
The argument \textit{main} must be the filename of the main file.

There are a couple of
considerations in setting up the main and child documents:

%%%%%%%%%%%%%%%%%%%%%%%%%%%%%%%%%%%%%%%%
\paragraph{Restrictions.}

Please note the following restrictions:
\begin{itemize}
\item
|\childdocmain| must be called with one argument \textit{main}
to ensure compatibility with earlier version of the package.
It must either be empty (|\childdocmain{}|)
or precisely match the filename of the main file in which it is specified.
See \secref{sec:detection} for further information.
\item
The filename \textit{main} must be specified without the |.tex| extension.
\item
The filename \textit{main} is case sensitive
(even in case-insensitive file systems)
due to internal string comparison.
\item
The argument \textit{main} should be fully expanded, it cannot be a macro.
\item
Subdirectories and special characters should be avoided in filenames.
\item
The command |\childdocmain{|\textit{main}|}| must be followed by a whitespace.
It should not be followed immediately by another command
or by a comment mark `|%|'.
This is because the \TeX{} parser reads the token immediately following
the argument of |\childdocmain| and puts it
at the beginning of every child section;
however, a white\-space is ignored.
\end{itemize}

%%%%%%%%%%%%%%%%%%%%%%%%%%%%%%%%%%%%%%%%
\paragraph{Content of Main File.}

It is advisable to place all content in the child files included by |\include|.
Any output contained in the main file will appear in all child documents
unless suppressed manually;
it cannot be suppressed automatically by the |\includeonly| directive
and thus should normally be avoided.
A method to include some content in the main file
by means of conditional processing is described in \secref{sec:conditional}.

%%%%%%%%%%%%%%%%%%%%%%%%%%%%%%%%%%%%%%%%
\paragraph{Page Numbering.}

When only a part of the document is compiled,
the appropriate numbering of pages
(as well as other status parameters)
is determined from the |.aux| files.
The latter contain information from previous passes.
However this information needs to propagate through
all intermediate child documents.
Therefore the page numbering in child documents may well
be inconsistent until the complete document is compiled at least once.

A useful (if unconventional) way to always ensure a consistent
page numbering is to restart the numbering in each child document
and denote the pages by `\textit{child}|.|\textit{page}'
where \textit{child} represents the chapter/section number of the child file.
This can be achieved by the command
|\numberwithin{page}{|\textit{child}|}|
of the \textsf{amsmath} package
where \textit{child} can be |chapter| or |section|
depending on the chosen structuring.
Alternatively, one can modify the macro |\thepage| appropriately
and reset the counter |page| at the start of each child file.

%%%%%%%%%%%%%%%%%%%%%%%%%%%%%%%%%%%%%%%%%%%%%%%%%%%%%%%%%%%%%%%%%%%%%%%%%%%%%%%%
\subsection{Conditional Processing}
\label{sec:conditional}

The package provides a mechanism to compile different versions
of a document. To customise the versions further some conditional processing
can come in handy to distinguish which version is being compiled.
The package provides two macros to describe the compilation context:

%%%%%%%%%%%%%%%%%%%%%%%%%%%%%%%%%%%%%%%%
\DescribeMacro{\ifchilddoc}
The conditional |\ifchilddoc| distinguishes between the compilation of
child documents and the main document:
%
\begin{center}
|\ifchilddoc |\textit{child-code}| |[|\||else |\textit{main-code}]| \||fi|
\end{center}

%%%%%%%%%%%%%%%%%%%%%%%%%%%%%%%%%%%%%%%%
\DescribeMacro{\childdocname}
\DescribeMacro{\childdocjob}
The macro |\childdocname| contains the filename (without extension)
of the main or child file being processed.
Note that |\childdocjob| will always contain the name of the main file.

%%%%%%%%%%%%%%%%%%%%%%%%%%%%%%%%%%%%%%%%
\paragraph{Title Page.}

Conditional processing can be used to include a title or banner page
in the main document when proper precautions are taken.
Importantly, the code in the main file should ensure that the page counter
(as well as other status parameters which are stored in the |.aux| files)
takes the same value after the conditional processing.
Otherwise the page numbers may take divergent values
depending on which part is compiled.

For example, a title page could be declared by:
%
\begin{center}
\begin{tabular}{l}
|\ifchilddoc\||else|\\
|\addtocounter{page}{-1}|\\
\textit{code for title page}\\
|\newpage|\\
|\||fi|
\end{tabular}
\end{center}
%
A banner page for the child documents can be generated by:
%
\begin{center}
\begin{tabular}{l}
|\ifchilddoc|\\
|\addtocounter{page}{-1}|\\
\textit{code for banner page}\\
|\newpage|\\
|\||fi|
\end{tabular}
\end{center}
%
Here one could write a message such as:
\begin{center}
|This is the part \childdocname{} of \childdocjob{}.|
\end{center}

%%%%%%%%%%%%%%%%%%%%%%%%%%%%%%%%%%%%%%%%%%%%%%%%%%%%%%%%%%%%%%%%%%%%%%%%%%%%%%%%
\subsection{Flags}
\label{sec:flags}

The package makes it easy to generate different versions
of the main or child documents.
To this end compilation flags can be defined
and assigned different default values.
They will be particularly useful in conjunction
with the forwarding mechanism described in \secref{sec:forward}.

For example, it may be useful to have a flag |\version|
which can be set to |draft| or |final|.
The document source will contain some conditional code
depending on the value of |\version|.
Suppose further, the flag should default to |final| for the main file
and to |draft| for child files
which is a natural assignment for editing the document.
This is achieved by placing the following code
in the preamble of the main document
(below the |\childdocmain| directive):
%
\begin{center}
\begin{tabular}{l}
|\ifchilddoc|\\
|\providecommand{\version}{draft}|\\
|\||else|\\
|\providecommand{\version}{final}|\\
|\||fi|
\end{tabular}
\end{center}
%
The definition by |\providecommand| makes sure
that previous definitions are not overwritten.
Further statements |\providecommand{\version}{...}|
can thus be added before the above code to override it.

For the main file, one might add a line
(between |\childdocmain| and the above block)
%
\begin{center}
|%\ifchilddoc\||else\providecommand{\version}{draft}\||fi|
\end{center}
%
which can be uncommented to produce a draft version.
Likewise one can add a line to the very top of a child file
(above the |\childdocof{|\textit{main}|}| directive)
%
\begin{center}
|%\providecommand{\version}{final}|
\end{center}
%
which can be uncommented to produce the final version of this child document.

%%%%%%%%%%%%%%%%%%%%%%%%%%%%%%%%%%%%%%%%%%%%%%%%%%%%%%%%%%%%%%%%%%%%%%%%%%%%%%%%
\subsection{Forwarding}
\label{sec:forward}

Different versions of the main or child documents
using compilation flags as described in \secref{sec:flags}
can be (permanently) stored in different files
for convenient compilation, viewing and distribution.
To this end, the package defines a command
to pass on compilation to a different file:

%%%%%%%%%%%%%%%%%%%%%%%%%%%%%%%%%%%%%%%%
\DescribeMacro{\childdocforward}
The command |\childdocforward| redirects processing to
another source file:
%
\begin{center}
\begin{tabular}{l}
|\input{childdoc.def}|\\
|\childdocforward[|\textit{main}|]{|\textit{dest}|}|\\
\end{tabular}
\end{center}
%
The argument \textit{dest} is the destination file
(without extension).
It should be the main file or one of the child files.
Note that further \textsf{childdoc} directives
such as |\childdocof| and |\childdocforward|
in the indicated file will be processed in this form.
The optional argument \textit{main}
passes on directly to the main file \textit{main}
while pretending to compile the child \textit{dest}.
This form behaves as if \textit{dest}
issues |\childdocof{|\textit{main}|}| right away,
and no further \textsf{childdoc} directives will be processed.

%%%%%%%%%%%%%%%%%%%%%%%%%%%%%%%%%%%%%%%%
\DescribeMacro{\...prefix}
In the alternative form |\childdocforwardprefix|,
%
\begin{center}
\begin{tabular}{l}
|\input{childdoc.def}|\\
|\childdocforwardprefix[|\textit{main}|]{|\textit{prefix}|}{|\textit{dest}|}|
\end{tabular}
\end{center}
%
the destination file is determined by a pattern
depending on the current file:
To make this work, the current file must be called
`{\textit{prefix}\hspace{0.2em}\textit{suffix}}'
with \textit{prefix} matching precisely the argument.
Processing is then passed on to the file
`{\textit{dest}\hspace{0.2em}\textit{suffix}}'.
Surely, the same effect is achieved by
directly specifying the
argument `{\textit{dest}\hspace{0.2em}\textit{suffix}}'
in the first form.
However, that requires to set up a different file
for each child. With the alternative form of the command
all these files can have exactly the same content
which simplifies setting them up and maintaining them.

For example, the following file |draft.tex|
with a compilation flag |\version| as described in \secref{sec:flags}
compiles the main document as a draft:
%
\begin{center}
\begin{tabular}{l}
|\def\version{draft}|\\
|\input{childdoc.def}|\\
|\childdocforward{|\textit{main}|}|
\end{tabular}
\end{center}
%
Likewise, the following files |final|\textit{nn}|.tex|
compile the final version of the child document
|child|\textit{nn}|.tex|:
%
\begin{center}
\begin{tabular}{l}
|\def\version{final}|\\
|\input{childdoc.def}|\\
|\childdocforwardprefix{final}{child}|
\end{tabular}
\end{center}
%

Note that when several versions of a main file and/or of each child file
are to be generated, it may be convenient to set up a |Makefile| or
shell script to automatise the process.

%%%%%%%%%%%%%%%%%%%%%%%%%%%%%%%%%%%%%%%%%%%%%%%%%%%%%%%%%%%%%%%%%%%%%%%%%%%%%%%%
\subsection{Command Line Processing}
\label{sec:commandline}

The effect of redirection files can also be achieved by invoking
the \LaTeX{} compiler with a more elaborate command line.
Most conveniently this should be done as part
of a shell script or a |Makefile|.

When using \textsf{childdoc} in the main file, the following
command lines effectively perform a redirection
(note that depending on the shell being used,
backslashes may have to be doubled: `|\|' $\to$ `|\\|'):
%
\begin{center}
|... -jobname "|\textit{target}|" |\\|"|[\textit{flags}]%
|\input{childdoc.def}\childdocforward[|\textit{main}|]{|\textit{dest}|}"|
\end{center}
%
Here \textit{target} is the name of the output file,
\textit{main} is the name of the main file
and \textit{dest} is the name of the main or child file to be processed
(all filenames without extensions).
The optional argument \textit{main} can be omitted
if \textit{main} matches \textit{dest}.
Optionally, compilation \textit{flags} can be defined via |\def| commands.
This command line makes the \TeX{} engine believe
it is compiling the file \textit{target}
whose content is specified as the latter parameter.
The provided code then forwards the processing to
\textit{main} or \textit{dest} as described in \secref{sec:forward}.

%%%%%%%%%%%%%%%%%%%%%%%%%%%%%%%%%%%%%%%%%%%%%%%%%%%%%%%%%%%%%%%%%%%%%%%%%%%%%%%%
\subsection{Include by Input}
\label{sec:input}

Including child documents by |\include| has some restrictions by design.
Most notably, the content of a child document always occupies
its own set of pages; pages cannot be shared between child documents.
Usually, this behaviour makes perfect sense
because each child document contain an essential part of the document.
However, in some situations it may be desirable to compose
a document from a collection of parts
without having mandatory page breaks between then.
For this case, the package
provides a mechanism to include parts
by |\input| which can also be processed individually.
However, by construction this mechanism
requires manual handling of the content to be output.

%%%%%%%%%%%%%%%%%%%%%%%%%%%%%%%%%%%%%%%%
\DescribeMacro{\ifchilddocmanual}
The main file should be prepared as usual, see \secref{sec:include}.
However, the document body must make a distinction
between processing of an individual part and of the main document, e.g.:
%
\begin{center}
\begin{tabular}{l}
|\ifchilddocmanual|\\
|\input{\childdocname}|\\
|\||else|\\
\textit{document body with }|\input{|\textit{part}|}|\\
|\||fi|
\end{tabular}
\end{center}
%
The conditional |\ifchilddocmanual| is true whenever
a part to be included by |\input| is being compiled,
and the name of the part is stored in |\childdocname|.

%%%%%%%%%%%%%%%%%%%%%%%%%%%%%%%%%%%%%%%%
\DescribeMacro{\childdocby}
Each part to be included by |\input| should start with:
%
\begin{center}
\begin{tabular}{l}
|\input{childdoc.def}|\\
|\childdocby{|\textit{main}|}|\\
\end{tabular}
\end{center}
%
The directive |\childdocby| is similar to |\childdocof|
described in \secref{sec:include},
but the subsequent selection of content must be done manually.
To that end, both |\ifchilddoc| and |\ifchilddocmanual|
will be true upon processing of a part,
and the name of the part is stored in |\childdocname|.
Note that |\jobname| will be set to the filename of the current part
so that each part receives an individual |.aux| file
that does not interfere with the |.aux| file(s) of the main document.
This behaviour can be altered by the alternative form
|\childdocby[*]{|\textit{main}|}| (with a non-empty optional argument)
which uses the |.aux| file of the main document
by setting |\jobname| to \textit{main}.

%%%%%%%%%%%%%%%%%%%%%%%%%%%%%%%%%%%%%%%%%%%%%%%%%%%%%%%%%%%%%%%%%%%%%%%%%%%%%%%%
\subsection{Driver Development}
\label{sec:driver}

The \textsf{childdoc} mechanism can also be use for the development
of definition files such as \LaTeX{} styles or classes.
This case differs from the above setup with multiple parts
included by |\include| in that no |\includeonly| should be invoked.
This can be achieved by starting the include file
(before |\ProvidesPackage|) with:
%
\begin{center}
\begin{tabular}{l}
|\input{childdoc.def}|\\
|\childdocforward{|\textit{main}|}|\\
\end{tabular}
\end{center}
%
or alternatively with:
%
\begin{center}
\begin{tabular}{l}
|\input{childdoc.def}|\\
|\childdocby{|\textit{main}|}|\\
\end{tabular}
\end{center}
%
Both forms have slightly different effects as described above.
The main file is prepared as usual, see \secref{sec:include}.

%%%%%%%%%%%%%%%%%%%%%%%%%%%%%%%%%%%%%%%%%%%%%%%%%%%%%%%%%%%%%%%%%%%%%%%%%%%%%%%%
\subsection{Legacy Detection}
\label{sec:detection}

The directive |\childdocmain| in the main file can detect
whether the complete document or merely a child is to be compiled
even without using the directive |\childdocof|.
This method is deprecated because it is less robust
and there is no compelling reason to use it;
it is merely provided for backward compatibility
and it may be removed in future versions.

If the detection mechanism is to be used,
it is mandatory to correctly specify
the filename of the main file as the argument of |\childdocmain|:
%
\begin{center}
\begin{tabular}{l}
|\input{childdoc.def}|\\
|\childdocmain{|\textit{main}|}|\\
\end{tabular}
\end{center}
%
If |\jobname| does not match the argument \textit{main} of |\childdocmain|,
it is assumed that |\jobname| points to the child file to be compiled.
When using |\childdocmain| with the main file specified as argument,
it suffices to start a child file
with just |\input{|\textit{main}|}|
without loading of the package and using |\childdocof|.
If instead all processing is done
with the appropriate \textsf{childdoc} directives,
the argument of \textit{main} of |\childdocmain| can be empty.

An alternative version of the command line processing described
in \secref{sec:commandline} using the detection mechanism reads:
%
\begin{center}
|... -jobname "|\textit{target}|" "|[\textit{flags}]%
[|\def\jobname{|\textit{dest}|}|]|\input{|\textit{main}|}"|
\end{center}

%%%%%%%%%%%%%%%%%%%%%%%%%%%%%%%%%%%%%%%%%%%%%%%%%%%%%%%%%%%%%%%%%%%%%%%%%%%%%%%%
\subsection{Manual Code}
\label{sec:manual}

In case one cannot be certain whether the definitions file |childdoc.def|
is installed on the target \TeX{} distribution
and one prefers not to ship it,
it is conceivable to paste a few relevant commands into the sources.

To that end, drop all statements |\input{childdoc.def}|
and perform the replacements as outlined below.
Instead of |\childdocmain{|\textit{main}|}| add the following code
to the top of the main file:
%
\begin{center}
\begin{tabular}{l}
|\||ifdefined\childdocname\endinput\||fi\newif\ifchilddoc|\\
|\edef\childdocname{\scantokens\expandafter{\jobname\noexpand}}|\\
|\def\childdocmain{|\textit{main}|}\||ifx\childdocmain\childdocname\||else|\\
|\childdoctrue\includeonly{\childdocname}\let\jobname\childdocmain\||fi|\\
\end{tabular}
\end{center}
%
Instead of |\childdocof{|\textit{main}|}| just include the main file
at the top of each child file:
%
\begin{center}
|\input{|\textit{main}|}|
\end{center}
%
A simple redirection |\childdocforward{|\textit{dest}|}| is achieved by:
%
\begin{center}
|\def\jobname{|\textit{dest}|}\input{\jobname}|
\end{center}
%
The redirection with prefix
|\childdocforwardprefix[|\textit{prefix}|]{|\textit{dest}|}|
is accomplished by:
%
\begin{center}
\begin{tabular}{l}
|{\edef\jobname{\scantokens\expandafter{\jobname\noexpand}}|\\
|\def\redirectjob |\textit{prefix}|#1~~~{\gdef\jobname{|\textit{dest}|#1}}|\\
|\expandafter\redirectjob\jobname~~~}\input{\jobname}|
\end{tabular}
\end{center}

In an alternative approach,
child documents can be compiled by a specific command line
without additional code or specific definitions:
%
\begin{center}
|... -jobname "|\textit{target}|" "|[\textit{flags}]%
|\includeonly{|\textit{dest}|}\input{|\textit{main}|}"|
\end{center}
%

%%%%%%%%%%%%%%%%%%%%%%%%%%%%%%%%%%%%%%%%%%%%%%%%%%%%%%%%%%%%%%%%%%%%%%%%%%%%%%%%
%%%%%%%%%%%%%%%%%%%%%%%%%%%%%%%%%%%%%%%%%%%%%%%%%%%%%%%%%%%%%%%%%%%%%%%%%%%%%%%%
\section{Information}

%%%%%%%%%%%%%%%%%%%%%%%%%%%%%%%%%%%%%%%%%%%%%%%%%%%%%%%%%%%%%%%%%%%%%%%%%%%%%%%%
\subsection{Copyright}

Copyright \copyright{} 2017--2018 Niklas Beisert

This work may be distributed and/or modified under the
conditions of the \LaTeX{} Project Public License, either version 1.3
of this license or (at your option) any later version.
The latest version of this license is in
  \url{http://www.latex-project.org/lppl.txt}
and version 1.3 or later is part of all distributions of \LaTeX{}
version 2005/12/01 or later.

This work has the LPPL maintenance status `maintained'.

The Current Maintainer of this work is Niklas Beisert.

This work consists of the files |README.txt|, |childdoc.ins| and |childdoc.dtx|
as well as the derived files |childdoc.def|, |cdocsamp.tex|
with |cdocsch1.tex|, |cdocsch2.tex|, |cdocspt3.tex|, |cdocspt4.tex|,
|cdocsdrf.tex|, |cdocsfn1.tex|, |cdocsfn2.tex|
as well as |childdoc.pdf|.

%%%%%%%%%%%%%%%%%%%%%%%%%%%%%%%%%%%%%%%%%%%%%%%%%%%%%%%%%%%%%%%%%%%%%%%%%%%%%%%%
\subsection{Files and Installation}

The package consists of the files:
%
\begin{center}
\begin{tabular}{ll}
    |README.txt|   & readme file \\
    |childdoc.ins| & installation file \\
    |childdoc.dtx| & source file \\
    |childdoc.def| & definition file \\
    |cdocsamp.tex| & sample main file \\
    |cdocsch1.tex| & sample include file \\
    |cdocsch2.tex| & sample include file \\
    |cdocspt3.tex| & sample part file \\
    |cdocspt4.tex| & sample part file \\
    |cdocsdrf.tex| & sample redirection file \\
    |cdocsfn1.tex| & sample redirection file \\
    |cdocsfn2.tex| & sample redirection file \\
    |childdoc.pdf| & manual
\end{tabular}
\end{center}
%
The distribution consists of the files
|README.txt|, |childdoc.ins| and |childdoc.dtx|.
%
\begin{itemize}
\item
Run (pdf)\LaTeX{} on |childdoc.dtx|
to compile the manual |childdoc.pdf| (this file).
\item
Run \LaTeX{} on |childdoc.ins| to create the definitions file |childdoc.def|
and the sample |cdocsamp.tex| with include files
|cdocsch1.tex|, |cdocsch2.tex|, |cdocspt3.tex|, |cdocspt4.tex|,
|cdocsdrf.tex|, |cdocsfn1.tex|, |cdocsfn2.tex|.
Then copy the file |childdoc.def| to an appropriate directory of your \LaTeX{}
distribution, e.g.\ \textit{texmf-root}|/tex/latex/childdoc|.
\end{itemize}

%%%%%%%%%%%%%%%%%%%%%%%%%%%%%%%%%%%%%%%%%%%%%%%%%%%%%%%%%%%%%%%%%%%%%%%%%%%%%%%%
\subsection{Related CTAN Packages}

There are several other packages which offer a similar functionality:
%
\begin{itemize}
\item
The packages
\href{http://ctan.org/pkg/docmute}{\textsf{docmute}},
\href{http://ctan.org/pkg/includex}{\textsf{includex}} and
\href{http://ctan.org/pkg/standalone}{\textsf{standalone}}
provide commands to include only the document body of
a child file thus allowing both files to be compiled individually.
\item
The packages \href{http://ctan.org/pkg/subdocs}{\textsf{subdocs}}
and \href{http://ctan.org/pkg/subfiles}{\textsf{subfiles}}
provide structures in which the main and child documents can be
encapsulated and allowing them to be compiled individually.
The inclusion mechanism is different from the conventional |\include|.
\item
The package \href{http://ctan.org/pkg/combine}{\textsf{combine}}
is an elaborate solution to combine several documents into one.
\end{itemize}
%
See also the CTAN topic \href{http://ctan.org/topic/subdocs}{\textsf{subdocs}}
for further related packages.
The present package differs from the above solutions in that
a document structure constructed with the conventional |\include| mechanism
just needs two extra commands at the top of every file
such that all constituent files can be compiled individually.

%%%%%%%%%%%%%%%%%%%%%%%%%%%%%%%%%%%%%%%%%%%%%%%%%%%%%%%%%%%%%%%%%%%%%%%%%%%%%%%%
%\subsection{Feature Suggestions}
%
%The following is a list of features which may be useful for future
%versions of this package:
%%
%\begin{itemize}
%\item
%\ldots
%\end{itemize}

%%%%%%%%%%%%%%%%%%%%%%%%%%%%%%%%%%%%%%%%%%%%%%%%%%%%%%%%%%%%%%%%%%%%%%%%%%%%%%%%
\subsection{Revision History}

%%%%%%%%%%%%%%%%%%%%%%%%%%%%%%%%%%%%%%%%
\paragraph{v2.0:} 2018/12/30

\begin{itemize}
\item
immediate forward processing
\item
added |\childdocby| mechanism
\item
manual restructured
\end{itemize}

%%%%%%%%%%%%%%%%%%%%%%%%%%%%%%%%%%%%%%%%
\paragraph{v1.6:} 2018/01/17

\begin{itemize}
\item
application for development of include files
\item
corrections to manual
\end{itemize}

%%%%%%%%%%%%%%%%%%%%%%%%%%%%%%%%%%%%%%%%
\paragraph{v1.5:} 2017/05/21

\begin{itemize}
\item
more complete structuring introduced
\item
|\childdocof| introduced
\item
|\childdoc| renamed to |\childdocmain|
\item
|\childredirect| renamed to |\childdocforward| and |\childdocforwardprefix|
and functionality expanded
\end{itemize}

%%%%%%%%%%%%%%%%%%%%%%%%%%%%%%%%%%%%%%%%
\paragraph{v1.0:} 2017/04/27

\begin{itemize}
\item
manual and install package
\item
first version published on CTAN
\end{itemize}

%%%%%%%%%%%%%%%%%%%%%%%%%%%%%%%%%%%%%%%%
\paragraph{v0.6:} 2017/04/26

\begin{itemize}
\item
redirection mechanism added
\end{itemize}

%%%%%%%%%%%%%%%%%%%%%%%%%%%%%%%%%%%%%%%%
\paragraph{v0.5:} 2017/04/26

\begin{itemize}
\item
functionality in definition file
\end{itemize}


%%%%%%%%%%%%%%%%%%%%%%%%%%%%%%%%%%%%%%%%%%%%%%%%%%%%%%%%%%%%%%%%%%%%%%%%%%%%%%%%
%%%%%%%%%%%%%%%%%%%%%%%%%%%%%%%%%%%%%%%%%%%%%%%%%%%%%%%%%%%%%%%%%%%%%%%%%%%%%%%%
%%%%%%%%%%%%%%%%%%%%%%%%%%%%%%%%%%%%%%%%%%%%%%%%%%%%%%%%%%%%%%%%%%%%%%%%%%%%%%%%
\appendix

\settowidth\MacroIndent{\rmfamily\scriptsize 000\ }

 \DocInput{childdoc.dtx}

\end{document}
%</driver>
% \fi
%
% %%%%%%%%%%%%%%%%%%%%%%%%%%%%%%%%%%%%%%%%%%%%%%%%%%%%%%%%%%%%%%%%%%%%%%%%%%%%%%
% %%%%%%%%%%%%%%%%%%%%%%%%%%%%%%%%%%%%%%%%%%%%%%%%%%%%%%%%%%%%%%%%%%%%%%%%%%%%%%
% \section{Sample}
%\iffalse
%<*samplemain>
%\fi
%
% The following presents a sample document
% with two chapters, two parts, a title page,
% a compile flag as well as three forwarding files to set the flag.
% It consists of eight |.tex| files:
% \begin{center}
% \begin{tabular}{ll}
% |cdocsamp.tex|&main file\\
% |cdocsch1.tex|&include file for chapter 1\\
% |cdocsch2.tex|&include file for chapter 2\\
% |cdocspt3.tex|&include file for part 3\\
% |cdocspt4.tex|&include file for part 4\\
% |cdocsdrf.tex|&forwarding file for main file in draft mode\\
% |cdocsfi1.tex|&forwarding file for final version of chapter 1\\
% |cdocsfi2.tex|&forwarding file for final version of chapter 2\\
% \end{tabular}
% \end{center}
% Each of the eight files can be compiled directly by the \LaTeX{} compiler.
%
% %%%%%%%%%%%%%%%%%%%%%%%%%%%%%%%%%%%%%%
% \paragraph{Main File.}
%
% The main file is called |cdocsamp.tex|.
%
% Load the \textsf{childdoc} definitions and
% declare the filename for the main document:
%    \begin{macrocode}
\input{childdoc.def}
\childdocmain{}
%    \end{macrocode}

% Optional override for |\version| flag:
%    \begin{macrocode}
%%\ifchilddoc\else\providecommand{\version}{draft}\fi
%    \end{macrocode}

% Define the default values for the |\version| flag
% (|final| for the main file and |draft| for childs):
%    \begin{macrocode}
\ifchilddoc
\providecommand{\version}{draft}
\else
\providecommand{\version}{final}
\fi
%    \end{macrocode}

% Load the standard document class:
%    \begin{macrocode}
\documentclass[12pt]{article}
%    \end{macrocode}

% Start the document body:
%    \begin{macrocode}
\begin{document}
%    \end{macrocode}

% Declare a title page.
% Print title, part of document being processed and version flag:
%    \begin{macrocode}
\addtocounter{page}{-1}
\begin{center}
{\LARGE\bfseries{}childdoc example\par}
\vspace{1cm}
\ifchilddoc
\ifchilddocmanual part\else chapter\fi:
`\childdocname' of `\childdocjob'\par
\else
main document: `\childdocjob'\par
\fi
version: \version\par
\end{center}
\newpage
%    \end{macrocode}

% Manually include selected file,
% otherwise process as usual:
%    \begin{macrocode}
\ifchilddocmanual
\section*{part `\childdocname'}
\input{\childdocname}
\else
%    \end{macrocode}

% Include the two chapters:
%    \begin{macrocode}
\include{cdocsch1}
\include{cdocsch2}
%    \end{macrocode}

% Include the two parts unless only chapters should be displayed:
%    \begin{macrocode}
\ifchilddoc\else
\section{part three}
\input{cdocspt3}
\section{part four}
\input{cdocspt4}
\fi
%    \end{macrocode}

% Process as usual until here:
%    \begin{macrocode}
\fi
%    \end{macrocode}

% End of document body:
%    \begin{macrocode}
\end{document}
%    \end{macrocode}
%\iffalse
%</samplemain>
%\fi
%
% %%%%%%%%%%%%%%%%%%%%%%%%%%%%%%%%%%%%%%
% \paragraph{Chapter Include Files.}
%
% The include files are called |cdocsch1.tex| and |cdocsch2.tex|.
%
%\iffalse
%<*samplechap1|samplechap2>
%\fi

% Optional override for |\version| flag:
%    \begin{macrocode}
%%\providecommand{\version}{final}
%    \end{macrocode}

% Include the main document:
%    \begin{macrocode}
\input{childdoc.def}
\childdocof{cdocsamp}
%    \end{macrocode}

%\iffalse
%</samplechap1|samplechap2>
%\fi
%
%\iffalse
%<*samplechap1>
%\fi
% Some text for chapter 1:
%    \begin{macrocode}
\section{one}
some text in chapter one
%    \end{macrocode}

%\iffalse
%</samplechap1>
%\fi
% Some text for chapter 2:
%\iffalse
%<*samplechap2>
%\fi
%    \begin{macrocode}
\section{two}
more text in chapter two
%    \end{macrocode}

%\iffalse
%</samplechap2>
%\fi
%
% %%%%%%%%%%%%%%%%%%%%%%%%%%%%%%%%%%%%%%
% \paragraph{Part Include Files.}
%
% The include files are called |cdocspt3.tex| and |cdocspt4.tex|.
%
%\iffalse
%<*samplepart3|samplepart4>
%\fi

% Optional override for |\version| flag:
%    \begin{macrocode}
%%\providecommand{\version}{final}
%    \end{macrocode}

% Include the main document:
%    \begin{macrocode}
\input{childdoc.def}
\childdocby{cdocsamp}
%    \end{macrocode}

%\iffalse
%</samplepart3|samplepart4>
%\fi
%
%\iffalse
%<*samplepart3>
%\fi
% Some text for part 3:
%    \begin{macrocode}
some text in part three
%    \end{macrocode}

%\iffalse
%</samplepart3>
%\fi
% Some text for part 4:
%\iffalse
%<*samplepart4>
%\fi
%    \begin{macrocode}
more text in part four
%    \end{macrocode}

%\iffalse
%</samplepart4>
%\fi
%
% %%%%%%%%%%%%%%%%%%%%%%%%%%%%%%%%%%%%%%
% \paragraph{Forwarding for a Complete Draft.}
%
% The following forwarding file |cdocsdrf.tex|
% compiles the main document in draft mode:
%\iffalse
%<*sampledraft>
%\fi
%    \begin{macrocode}
\def\version{draft}
\input{childdoc.def}
\childdocforward{cdocsamp}
%    \end{macrocode}

%\iffalse
%</sampledraft>
%\fi
%
% %%%%%%%%%%%%%%%%%%%%%%%%%%%%%%%%%%%%%%
% \paragraph{Forwarding for Final Version of the Chapters.}
%
% The following forwarding files |cdocsfn1.tex| and |cdocsfn2.tex|
% (with identical content)
% compile the final versions of the child documents
% |cdocsch1.tex| and |cdocsch2.tex|, respectively:
%\iffalse
%<*samplefinal>
%\fi
%    \begin{macrocode}
\def\version{final}
\input{childdoc.def}
\childdocforwardprefix[cdocsamp]{cdocsfn}{cdocsch}
%    \end{macrocode}

%\iffalse
%</samplefinal>
%\fi
%
% %%%%%%%%%%%%%%%%%%%%%%%%%%%%%%%%%%%%%%
% \paragraph{Command Line Processing.}
%
% The following three command lines generate the output files
% |cdocscld|, |cdocscl1| and |cdocscl2|
% which should be identical to
% |cdocsdrf|, |cdocsch1| and |cdocsfn2|, respectively:
% \begin{center}
% \begin{tabular}{l}
% |latex -jobname cdocscld \|\\
% |  "\def\version{draft}\input{childdoc.def}\childdocforward{cdocsamp}"|\\
% |latex -jobname cdocscl1 \|\\
% |  "\input{childdoc.def}\childdocforward[cdocsamp]{cdocsch1}"|\\
% |latex -jobname cdocscl2 \|\\
% |  "\def\version{final}\input{childdoc.def}\childdocforward{cdocsch2}"|
% \end{tabular}
% \end{center}
% Note that the trailing backslash on each first line
% merely continues the input to the second line
% (for convenient cut ant paste).
% Furthermore, the command |latex| can be replaced by any
% of its alternative versions such as |pdflatex|.
%
% %%%%%%%%%%%%%%%%%%%%%%%%%%%%%%%%%%%%%%%%%%%%%%%%%%%%%%%%%%%%%%%%%%%%%%%%%%%%%%
% %%%%%%%%%%%%%%%%%%%%%%%%%%%%%%%%%%%%%%%%%%%%%%%%%%%%%%%%%%%%%%%%%%%%%%%%%%%%%%
% \section{Implementation}
%\iffalse
%<*package>
%\fi
%
% This section describes the definitions file |childdoc.def|.

% The definitions cannot be loaded using |\usepackage| or |\RequirePackage|
% which has a mechanism to prevent loading a style file more than once.
% When loading the definitions by means of |\input|
% multiple instances have to be prevented manually:
%\iffalse
%This code needs to be before the `\ProvidesFile' directive
%which is defined at the beginning of this file.
%Therefore it is also placed there and commented out here.
%</package>
%<*discard>
%\fi
%    \begin{macrocode}
\ifdefined\childdocmain\endinput\fi
%    \end{macrocode}
%\iffalse
%</discard>
%<*package>
%\fi
%
% \macro{\ifchilddoc}
% \macro{\ifchilddocmanual}
% The conditional |\ifchilddoc| tells whether a
% child (true) or main (false) document is being compiled.
% The conditional |\ifchilddocmanual| tells whether
% the |\includeonly| mechanism is used (false) or
% the selection of child files must be performed manually (true).
% The definitions initialise to false:
%    \begin{macrocode}
\newif\ifchilddoc
\newif\ifchilddocmanual
%    \end{macrocode}

% \macro{\childdocname}
% \macro{\childdocjob}
% The macro |\childdocname| stores the name of the main document
% to be compiled. The macro |\childdocjob| stores the name of
% the document on which the \LaTeX{} compiler was originally invoked.
% The content of |\jobname| cannot be compared
% to filenames specified in the source due to different catcodes.
% The following code rescans |\jobname|, stores the result
% in |\childdocname| and saves a copy in |\childdocjob|:
%    \begin{macrocode}
\edef\childdocname{\scantokens\expandafter{\jobname\noexpand}}
\let\childdocjob\childdocname
%    \end{macrocode}

% \macro{\childdocdisable}
% The macro |\childdocdisable| prevents the main file
% from being processed more than once.
% At this stage, the main document command |\childdocmain|
% is assumed to be called once again where it should do nothing.
% Any subsequent call to it should prevent
% a secondary processing of the main document
% It overwrites the forwarding commands
% |\childdocof| and |\childdocforward|
% with empty macros to prevent further inclusions of the main document:
%    \begin{macrocode}
\newcommand{\childdocdisable}
{
  \renewcommand{\childdocmain}[1]{\renewcommand{\childdocmain}[1]{\endinput}}
  \renewcommand{\childdocof}[1]{}
  \renewcommand{\childdocby}[2][]{}
  \renewcommand{\childdocforward}[2][]{}
  \renewcommand{\childdocdisable}{}
}
%    \end{macrocode}

% \macro{\childdocmain}
% The macro |\childdocmain| is to be called at the top of the main file
% with nothing or the main filename (without extension) as argument.
% First, it breaks loops.
% If the argument is not empty and does not match |\childdocname|
% (which is set by the first inclusion of |childdoc.def|),
% |\ifchilddoc| is set to true, |\includeonly| is applied to the child file
% and |\jobname| is set to the main file
% (for proper handling of |.aux| files):
%    \begin{macrocode}
\newcommand{\childdocmain}[1]
{
  \childdocdisable\childdocmain{}
  \if?#1?\else
    \begingroup
      \def\childdoctmp{#1}
      \ifx\childdoctmp\childdocname
        \def\childdoctmp{}
      \else
        \def\childdoctmp
        {
          \childdoctrue
          \includeonly{\childdocname}
          \def\childdocjob{#1}
          \def\jobname{#1}
        }
      \fi
      \expandafter
    \endgroup
    \childdoctmp
  \fi
}
%    \end{macrocode}

% \macro{\childdocof}
% The command |\childdocof| redirects
% compilation to the main file |#1|.
%    \begin{macrocode}
\newcommand{\childdocof}[1]
{
  \childdocdisable
  \childdoctrue
  \includeonly{\childdocname}
  \def\jobname{#1}
  \def\childdocjob{#1}
  \input{#1}
}
%    \end{macrocode}

% \macro{\childdocby}
% The command |\childdocby| ....
%    \begin{macrocode}
\newcommand{\childdocby}[2][]
{
  \childdocdisable
  \childdoctrue
  \childdocmanualtrue
  \if?#1?\else
    \def\jobname{#2}
  \fi
  \def\childdocjob{#2}
  \input{#2}
  \endinput
}
%    \end{macrocode}

% \macro{\childdocforward}
% The command |\childdocforward| redirects
% compilation to the main file or
% (if the optional argument is given) a child file.
% Parameters are set as if the main file
% or a child file starting with |\childdocof| was compiled.
% Then compilation is handed over to the main file:
%    \begin{macrocode}
\newcommand{\childdocforward}[2][]
{
  \begingroup
    \if?#1?
      \def\childdoctmp
      {
        \def\childdocname{#2}
        \def\childdocjob{#2}
        \def\jobname{#2}
        \input{#2}
        \endinput
      }
    \else
      \def\childdoctmp
      {
        \childdocdisable
        \def\childdocname{#2}
        \childdoctrue
        \includeonly{#2}
        \def\childdocjob{#1}
        \def\jobname{#1}
        \input{#1}
        \endinput
      }
    \fi
    \expandafter
  \endgroup
  \childdoctmp
}
%    \end{macrocode}

% \macro{\childdocforwardprefix}
% The command |\childdocforwardprefix| redirects
% compilation to the main or a child file by means of a pattern.
% The prefix |#1| in the current filename is replaced by |#2|
% and the suffix of the current filename is kept
% (it is assumed that the filename does not contain the substring `|~~~|'
% which is used as a delimiter).
% Compilation is handed over to the new file by |\childdocforward|:
%    \begin{macrocode}
\newcommand{\childdocforwardprefix}[3][]
{
  \begingroup
    \def\childdocextract #2##1~~~{\def\childdoctmp{\childdocforward[#1]{#3##1}}}
    \expandafter\childdocextract\childdocname~~~
    \expandafter
  \endgroup
  \childdoctmp
}
%    \end{macrocode}

% \macro{\childdoc}
% The deprecated macro |\childdoc| is a legacy version of |\childdocmain|:
%    \begin{macrocode}
\newcommand{\childdoc}{\childdocmain}
%    \end{macrocode}

% \macro{\childdocredirect}
% The deprecated macro |\childdocredirect| is a legacy version
% of |\childdocforward| and |\childdocforwardprefix|:
%    \begin{macrocode}
\newcommand{\childdocredirect}[2][]
{
  \begingroup
    \if?#1?
      \def\childdoctmp{\childdocforward{#2}}
    \else
      \def\childdoctmp{\childdocforwardprefix{#1}{#2}}
    \fi
    \expandafter
  \endgroup
  \childdoctmp
}
%    \end{macrocode}

%\iffalse
%</package>
%\fi
%
\endinput
\childdocforward[|\textit{main}|]{|\textit{dest}|}"|
\end{center}
%
Here \textit{target} is the name of the output file,
\textit{main} is the name of the main file
and \textit{dest} is the name of the main or child file to be processed
(all filenames without extensions).
The optional argument \textit{main} can be omitted
if \textit{main} matches \textit{dest}.
Optionally, compilation \textit{flags} can be defined via |\def| commands.
This command line makes the \TeX{} engine believe
it is compiling the file \textit{target}
whose content is specified as the latter parameter.
The provided code then forwards the processing to
\textit{main} or \textit{dest} as described in \secref{sec:forward}.

%%%%%%%%%%%%%%%%%%%%%%%%%%%%%%%%%%%%%%%%%%%%%%%%%%%%%%%%%%%%%%%%%%%%%%%%%%%%%%%%
\subsection{Include by Input}
\label{sec:input}

Including child documents by |\include| has some restrictions by design.
Most notably, the content of a child document always occupies
its own set of pages; pages cannot be shared between child documents.
Usually, this behaviour makes perfect sense
because each child document contain an essential part of the document.
However, in some situations it may be desirable to compose
a document from a collection of parts
without having mandatory page breaks between then.
For this case, the package
provides a mechanism to include parts
by |\input| which can also be processed individually.
However, by construction this mechanism
requires manual handling of the content to be output.

%%%%%%%%%%%%%%%%%%%%%%%%%%%%%%%%%%%%%%%%
\DescribeMacro{\ifchilddocmanual}
The main file should be prepared as usual, see \secref{sec:include}.
However, the document body must make a distinction
between processing of an individual part and of the main document, e.g.:
%
\begin{center}
\begin{tabular}{l}
|\ifchilddocmanual|\\
|\input{\childdocname}|\\
|\||else|\\
\textit{document body with }|\input{|\textit{part}|}|\\
|\||fi|
\end{tabular}
\end{center}
%
The conditional |\ifchilddocmanual| is true whenever
a part to be included by |\input| is being compiled,
and the name of the part is stored in |\childdocname|.

%%%%%%%%%%%%%%%%%%%%%%%%%%%%%%%%%%%%%%%%
\DescribeMacro{\childdocby}
Each part to be included by |\input| should start with:
%
\begin{center}
\begin{tabular}{l}
|% \iffalse
%
% childdoc.dtx Copyright (C) 2017-2018 Niklas Beisert
%
% This work may be distributed and/or modified under the
% conditions of the LaTeX Project Public License, either version 1.3
% of this license or (at your option) any later version.
% The latest version of this license is in
%   http://www.latex-project.org/lppl.txt
% and version 1.3 or later is part of all distributions of LaTeX
% version 2005/12/01 or later.
%
% This work has the LPPL maintenance status `maintained'.
%
% The Current Maintainer of this work is Niklas Beisert.
%
% This work consists of the files childdoc.dtx and childdoc.ins
% and the derived files childdoc.def and cdocsamp.tex with
% cdocsch1.tex, cdocsch2.tex, cdocsdrf.tex, cdocsfn1.tex, cdocsfn2.tex.
%
%<package>\ifdefined\childdocmain\endinput\fi
%<package>\ProvidesFile{childdoc.def}[2018/12/30 v2.0 child document driver]
%<samplemain>\ProvidesFile{cdocsamp.tex}[2018/12/30 v2.0 sample for childdoc]
%<*driver>
%\ProvidesFile{childdoc.drv}[2018/12/30 v2.0 childdoc reference manual file]
\PassOptionsToClass{10pt,a4paper}{article}
\documentclass{ltxdoc}

\usepackage[margin=35mm]{geometry}
\usepackage{hyperref}
\usepackage{hyperxmp}
\usepackage[usenames]{color}

\hypersetup{colorlinks=true}
\hypersetup{pdfstartview=FitH}
\hypersetup{pdfpagemode=UseNone}
\hypersetup{pdfsource={}}
\hypersetup{pdflang={en-UK}}
\hypersetup{pdfcopyright={Copyright 2017-2018 Niklas Beisert.
  This work may be distributed and/or modified under the
  conditions of the LaTeX Project Public License, either version 1.3
  of this license or (at your option) any later version.}}
\hypersetup{pdflicenseurl={http://www.latex-project.org/lppl.txt}}
\hypersetup{pdfcontactaddress={ETH Zurich, ITP, HIT K,
  Wolfgang-Pauli-Strasse 27}}
\hypersetup{pdfcontactpostcode={8093}}
\hypersetup{pdfcontactcity={Zurich}}
\hypersetup{pdfcontactcountry={Switzerland}}
\hypersetup{pdfcontactemail={nbeisert@itp.phys.ethz.ch}}
\hypersetup{pdfcontacturl={http://people.phys.ethz.ch/\xmptilde nbeisert/}}

\newcommand{\secref}[1]{\hyperref[#1]{section \ref*{#1}}}

\parskip1ex
\parindent0pt
\let\olditemize\itemize
\def\itemize{\olditemize\parskip0pt}

\begin{document}

\title{The \textsf{childdoc} Package}
\hypersetup{pdftitle={The childdoc Package}}
\author{Niklas Beisert\\[2ex]
  Institut f\"ur Theoretische Physik\\
  Eidgen\"ossische Technische Hochschule Z\"urich\\
  Wolfgang-Pauli-Strasse 27, 8093 Z\"urich, Switzerland\\[1ex]
  \href{mailto:nbeisert@itp.phys.ethz.ch}
  {\texttt{nbeisert@itp.phys.ethz.ch}}}
\hypersetup{pdfauthor={Niklas Beisert}}
\hypersetup{pdfsubject={Manual for the LaTeX2e Package childdoc}}
\date{30 December 2018, \textsf{v2.0}}
\maketitle

\begin{abstract}\noindent
\textsf{childdoc} is a \LaTeXe{} package
that enables the direct compilation
of document sections included by |\include|
to individual files.
\end{abstract}

\begingroup
\parskip0ex
\tableofcontents
\endgroup

%%%%%%%%%%%%%%%%%%%%%%%%%%%%%%%%%%%%%%%%%%%%%%%%%%%%%%%%%%%%%%%%%%%%%%%%%%%%%%%%
%%%%%%%%%%%%%%%%%%%%%%%%%%%%%%%%%%%%%%%%%%%%%%%%%%%%%%%%%%%%%%%%%%%%%%%%%%%%%%%%
\section{Introduction}

\LaTeX{} provides a mechanism to structure a large document (such as a book)
into a main file and several child files (containing the chapters)
using the |\include| command.
This mechanism is beneficial for documents
which span hundreds of pages in order to
make the source file(s) more manageable.
Moreover, compilation can be restricted to
selected child files by means of the |\includeonly| command.
The latter feature can be used to reduce the compilation time while editing
(this was significantly more useful in the earlier days of \LaTeX{})
or to generate a smaller document which is easier to navigate.
Another application of |\includeonly| is to generate
documents consisting of selected parts of the complete document.

However, there are a few drawbacks of the plain |\include| mechanism:
\begin{itemize}
\item
The child files cannot be compiled on their own,
they can only be compiled via the main file.
A naive editing environment
(such as a text editor with an option
to have the current file processed by \LaTeX)
may require one to switch to the main file before compiling;
attempting to compile the child file produces errors.
\item
The main file must be modified (each time)
to adjust the |\includeonly| command
to the present needs. This easily leaves the main file in a messy state.
\item
The generated document will always carry the filename
of the main document. This is inconvenient if
several child files are to be compiled and
to be kept for distribution.
\end{itemize}

The present package provides a simple interface
to make child files individually compilable by \LaTeX{}.
Compiling a child file then has the same effect as compiling
the main file with an |\includeonly| command
to select the appropriate child.
Moreover the generated document will carry the name of the child
rather than the main file.
This resolves all three above issues.

This feature is meant to make the editing of books,
thesis documents and lecture notes somewhat more convenient.
However, the package can also be used efficiently for
composing a series of documents (such as exercise sheets)
which are typically distributed individually.
It then assists the author in generating the individual documents
(potentially in different versions)
as well as a document containing the collected series.
Another application is in developing style files
or other kinds of included material
where compilation of the style file could redirect
to a sample or test file.

%%%%%%%%%%%%%%%%%%%%%%%%%%%%%%%%%%%%%%%%%%%%%%%%%%%%%%%%%%%%%%%%%%%%%%%%%%%%%%%%
%%%%%%%%%%%%%%%%%%%%%%%%%%%%%%%%%%%%%%%%%%%%%%%%%%%%%%%%%%%%%%%%%%%%%%%%%%%%%%%%
\section{Usage}

First of all, the package \textsf{childdoc} is \emph{not} a standard
\LaTeXe{} |.sty| style file! Therefore it needs to be invoked in
a non-standard way.

%%%%%%%%%%%%%%%%%%%%%%%%%%%%%%%%%%%%%%%%%%%%%%%%%%%%%%%%%%%%%%%%%%%%%%%%%%%%%%%%
\subsection{Included Files}
\label{sec:include}

%%%%%%%%%%%%%%%%%%%%%%%%%%%%%%%%%%%%%%%%
\DescribeMacro{\childdocmain}
To use the package, add the commands
\begin{center}
\begin{tabular}{l}
|\input{childdoc.def}|\\
|\childdocmain{}|\\
\end{tabular}
\end{center}
at the very top of the main \LaTeX{} file,
in particular \emph{before} the |\documentclass| statement!
The argument of |\childdocmain| should be left empty
(but it must be present).

%%%%%%%%%%%%%%%%%%%%%%%%%%%%%%%%%%%%%%%%
\DescribeMacro{\childdocof}
Furthermore, add the commands
\begin{center}
\begin{tabular}{l}
|\input{childdoc.def}|\\
|\childdocof{|\textit{main}|}|\\
\end{tabular}
\end{center}
at the top of every child file \textit{child}
which is included by |\include{|\textit{child}|}|
from within the main file
(or at least for those files to be compiled individually).
The argument \textit{main} must be the filename of the main file.

There are a couple of
considerations in setting up the main and child documents:

%%%%%%%%%%%%%%%%%%%%%%%%%%%%%%%%%%%%%%%%
\paragraph{Restrictions.}

Please note the following restrictions:
\begin{itemize}
\item
|\childdocmain| must be called with one argument \textit{main}
to ensure compatibility with earlier version of the package.
It must either be empty (|\childdocmain{}|)
or precisely match the filename of the main file in which it is specified.
See \secref{sec:detection} for further information.
\item
The filename \textit{main} must be specified without the |.tex| extension.
\item
The filename \textit{main} is case sensitive
(even in case-insensitive file systems)
due to internal string comparison.
\item
The argument \textit{main} should be fully expanded, it cannot be a macro.
\item
Subdirectories and special characters should be avoided in filenames.
\item
The command |\childdocmain{|\textit{main}|}| must be followed by a whitespace.
It should not be followed immediately by another command
or by a comment mark `|%|'.
This is because the \TeX{} parser reads the token immediately following
the argument of |\childdocmain| and puts it
at the beginning of every child section;
however, a white\-space is ignored.
\end{itemize}

%%%%%%%%%%%%%%%%%%%%%%%%%%%%%%%%%%%%%%%%
\paragraph{Content of Main File.}

It is advisable to place all content in the child files included by |\include|.
Any output contained in the main file will appear in all child documents
unless suppressed manually;
it cannot be suppressed automatically by the |\includeonly| directive
and thus should normally be avoided.
A method to include some content in the main file
by means of conditional processing is described in \secref{sec:conditional}.

%%%%%%%%%%%%%%%%%%%%%%%%%%%%%%%%%%%%%%%%
\paragraph{Page Numbering.}

When only a part of the document is compiled,
the appropriate numbering of pages
(as well as other status parameters)
is determined from the |.aux| files.
The latter contain information from previous passes.
However this information needs to propagate through
all intermediate child documents.
Therefore the page numbering in child documents may well
be inconsistent until the complete document is compiled at least once.

A useful (if unconventional) way to always ensure a consistent
page numbering is to restart the numbering in each child document
and denote the pages by `\textit{child}|.|\textit{page}'
where \textit{child} represents the chapter/section number of the child file.
This can be achieved by the command
|\numberwithin{page}{|\textit{child}|}|
of the \textsf{amsmath} package
where \textit{child} can be |chapter| or |section|
depending on the chosen structuring.
Alternatively, one can modify the macro |\thepage| appropriately
and reset the counter |page| at the start of each child file.

%%%%%%%%%%%%%%%%%%%%%%%%%%%%%%%%%%%%%%%%%%%%%%%%%%%%%%%%%%%%%%%%%%%%%%%%%%%%%%%%
\subsection{Conditional Processing}
\label{sec:conditional}

The package provides a mechanism to compile different versions
of a document. To customise the versions further some conditional processing
can come in handy to distinguish which version is being compiled.
The package provides two macros to describe the compilation context:

%%%%%%%%%%%%%%%%%%%%%%%%%%%%%%%%%%%%%%%%
\DescribeMacro{\ifchilddoc}
The conditional |\ifchilddoc| distinguishes between the compilation of
child documents and the main document:
%
\begin{center}
|\ifchilddoc |\textit{child-code}| |[|\||else |\textit{main-code}]| \||fi|
\end{center}

%%%%%%%%%%%%%%%%%%%%%%%%%%%%%%%%%%%%%%%%
\DescribeMacro{\childdocname}
\DescribeMacro{\childdocjob}
The macro |\childdocname| contains the filename (without extension)
of the main or child file being processed.
Note that |\childdocjob| will always contain the name of the main file.

%%%%%%%%%%%%%%%%%%%%%%%%%%%%%%%%%%%%%%%%
\paragraph{Title Page.}

Conditional processing can be used to include a title or banner page
in the main document when proper precautions are taken.
Importantly, the code in the main file should ensure that the page counter
(as well as other status parameters which are stored in the |.aux| files)
takes the same value after the conditional processing.
Otherwise the page numbers may take divergent values
depending on which part is compiled.

For example, a title page could be declared by:
%
\begin{center}
\begin{tabular}{l}
|\ifchilddoc\||else|\\
|\addtocounter{page}{-1}|\\
\textit{code for title page}\\
|\newpage|\\
|\||fi|
\end{tabular}
\end{center}
%
A banner page for the child documents can be generated by:
%
\begin{center}
\begin{tabular}{l}
|\ifchilddoc|\\
|\addtocounter{page}{-1}|\\
\textit{code for banner page}\\
|\newpage|\\
|\||fi|
\end{tabular}
\end{center}
%
Here one could write a message such as:
\begin{center}
|This is the part \childdocname{} of \childdocjob{}.|
\end{center}

%%%%%%%%%%%%%%%%%%%%%%%%%%%%%%%%%%%%%%%%%%%%%%%%%%%%%%%%%%%%%%%%%%%%%%%%%%%%%%%%
\subsection{Flags}
\label{sec:flags}

The package makes it easy to generate different versions
of the main or child documents.
To this end compilation flags can be defined
and assigned different default values.
They will be particularly useful in conjunction
with the forwarding mechanism described in \secref{sec:forward}.

For example, it may be useful to have a flag |\version|
which can be set to |draft| or |final|.
The document source will contain some conditional code
depending on the value of |\version|.
Suppose further, the flag should default to |final| for the main file
and to |draft| for child files
which is a natural assignment for editing the document.
This is achieved by placing the following code
in the preamble of the main document
(below the |\childdocmain| directive):
%
\begin{center}
\begin{tabular}{l}
|\ifchilddoc|\\
|\providecommand{\version}{draft}|\\
|\||else|\\
|\providecommand{\version}{final}|\\
|\||fi|
\end{tabular}
\end{center}
%
The definition by |\providecommand| makes sure
that previous definitions are not overwritten.
Further statements |\providecommand{\version}{...}|
can thus be added before the above code to override it.

For the main file, one might add a line
(between |\childdocmain| and the above block)
%
\begin{center}
|%\ifchilddoc\||else\providecommand{\version}{draft}\||fi|
\end{center}
%
which can be uncommented to produce a draft version.
Likewise one can add a line to the very top of a child file
(above the |\childdocof{|\textit{main}|}| directive)
%
\begin{center}
|%\providecommand{\version}{final}|
\end{center}
%
which can be uncommented to produce the final version of this child document.

%%%%%%%%%%%%%%%%%%%%%%%%%%%%%%%%%%%%%%%%%%%%%%%%%%%%%%%%%%%%%%%%%%%%%%%%%%%%%%%%
\subsection{Forwarding}
\label{sec:forward}

Different versions of the main or child documents
using compilation flags as described in \secref{sec:flags}
can be (permanently) stored in different files
for convenient compilation, viewing and distribution.
To this end, the package defines a command
to pass on compilation to a different file:

%%%%%%%%%%%%%%%%%%%%%%%%%%%%%%%%%%%%%%%%
\DescribeMacro{\childdocforward}
The command |\childdocforward| redirects processing to
another source file:
%
\begin{center}
\begin{tabular}{l}
|\input{childdoc.def}|\\
|\childdocforward[|\textit{main}|]{|\textit{dest}|}|\\
\end{tabular}
\end{center}
%
The argument \textit{dest} is the destination file
(without extension).
It should be the main file or one of the child files.
Note that further \textsf{childdoc} directives
such as |\childdocof| and |\childdocforward|
in the indicated file will be processed in this form.
The optional argument \textit{main}
passes on directly to the main file \textit{main}
while pretending to compile the child \textit{dest}.
This form behaves as if \textit{dest}
issues |\childdocof{|\textit{main}|}| right away,
and no further \textsf{childdoc} directives will be processed.

%%%%%%%%%%%%%%%%%%%%%%%%%%%%%%%%%%%%%%%%
\DescribeMacro{\...prefix}
In the alternative form |\childdocforwardprefix|,
%
\begin{center}
\begin{tabular}{l}
|\input{childdoc.def}|\\
|\childdocforwardprefix[|\textit{main}|]{|\textit{prefix}|}{|\textit{dest}|}|
\end{tabular}
\end{center}
%
the destination file is determined by a pattern
depending on the current file:
To make this work, the current file must be called
`{\textit{prefix}\hspace{0.2em}\textit{suffix}}'
with \textit{prefix} matching precisely the argument.
Processing is then passed on to the file
`{\textit{dest}\hspace{0.2em}\textit{suffix}}'.
Surely, the same effect is achieved by
directly specifying the
argument `{\textit{dest}\hspace{0.2em}\textit{suffix}}'
in the first form.
However, that requires to set up a different file
for each child. With the alternative form of the command
all these files can have exactly the same content
which simplifies setting them up and maintaining them.

For example, the following file |draft.tex|
with a compilation flag |\version| as described in \secref{sec:flags}
compiles the main document as a draft:
%
\begin{center}
\begin{tabular}{l}
|\def\version{draft}|\\
|\input{childdoc.def}|\\
|\childdocforward{|\textit{main}|}|
\end{tabular}
\end{center}
%
Likewise, the following files |final|\textit{nn}|.tex|
compile the final version of the child document
|child|\textit{nn}|.tex|:
%
\begin{center}
\begin{tabular}{l}
|\def\version{final}|\\
|\input{childdoc.def}|\\
|\childdocforwardprefix{final}{child}|
\end{tabular}
\end{center}
%

Note that when several versions of a main file and/or of each child file
are to be generated, it may be convenient to set up a |Makefile| or
shell script to automatise the process.

%%%%%%%%%%%%%%%%%%%%%%%%%%%%%%%%%%%%%%%%%%%%%%%%%%%%%%%%%%%%%%%%%%%%%%%%%%%%%%%%
\subsection{Command Line Processing}
\label{sec:commandline}

The effect of redirection files can also be achieved by invoking
the \LaTeX{} compiler with a more elaborate command line.
Most conveniently this should be done as part
of a shell script or a |Makefile|.

When using \textsf{childdoc} in the main file, the following
command lines effectively perform a redirection
(note that depending on the shell being used,
backslashes may have to be doubled: `|\|' $\to$ `|\\|'):
%
\begin{center}
|... -jobname "|\textit{target}|" |\\|"|[\textit{flags}]%
|\input{childdoc.def}\childdocforward[|\textit{main}|]{|\textit{dest}|}"|
\end{center}
%
Here \textit{target} is the name of the output file,
\textit{main} is the name of the main file
and \textit{dest} is the name of the main or child file to be processed
(all filenames without extensions).
The optional argument \textit{main} can be omitted
if \textit{main} matches \textit{dest}.
Optionally, compilation \textit{flags} can be defined via |\def| commands.
This command line makes the \TeX{} engine believe
it is compiling the file \textit{target}
whose content is specified as the latter parameter.
The provided code then forwards the processing to
\textit{main} or \textit{dest} as described in \secref{sec:forward}.

%%%%%%%%%%%%%%%%%%%%%%%%%%%%%%%%%%%%%%%%%%%%%%%%%%%%%%%%%%%%%%%%%%%%%%%%%%%%%%%%
\subsection{Include by Input}
\label{sec:input}

Including child documents by |\include| has some restrictions by design.
Most notably, the content of a child document always occupies
its own set of pages; pages cannot be shared between child documents.
Usually, this behaviour makes perfect sense
because each child document contain an essential part of the document.
However, in some situations it may be desirable to compose
a document from a collection of parts
without having mandatory page breaks between then.
For this case, the package
provides a mechanism to include parts
by |\input| which can also be processed individually.
However, by construction this mechanism
requires manual handling of the content to be output.

%%%%%%%%%%%%%%%%%%%%%%%%%%%%%%%%%%%%%%%%
\DescribeMacro{\ifchilddocmanual}
The main file should be prepared as usual, see \secref{sec:include}.
However, the document body must make a distinction
between processing of an individual part and of the main document, e.g.:
%
\begin{center}
\begin{tabular}{l}
|\ifchilddocmanual|\\
|\input{\childdocname}|\\
|\||else|\\
\textit{document body with }|\input{|\textit{part}|}|\\
|\||fi|
\end{tabular}
\end{center}
%
The conditional |\ifchilddocmanual| is true whenever
a part to be included by |\input| is being compiled,
and the name of the part is stored in |\childdocname|.

%%%%%%%%%%%%%%%%%%%%%%%%%%%%%%%%%%%%%%%%
\DescribeMacro{\childdocby}
Each part to be included by |\input| should start with:
%
\begin{center}
\begin{tabular}{l}
|\input{childdoc.def}|\\
|\childdocby{|\textit{main}|}|\\
\end{tabular}
\end{center}
%
The directive |\childdocby| is similar to |\childdocof|
described in \secref{sec:include},
but the subsequent selection of content must be done manually.
To that end, both |\ifchilddoc| and |\ifchilddocmanual|
will be true upon processing of a part,
and the name of the part is stored in |\childdocname|.
Note that |\jobname| will be set to the filename of the current part
so that each part receives an individual |.aux| file
that does not interfere with the |.aux| file(s) of the main document.
This behaviour can be altered by the alternative form
|\childdocby[*]{|\textit{main}|}| (with a non-empty optional argument)
which uses the |.aux| file of the main document
by setting |\jobname| to \textit{main}.

%%%%%%%%%%%%%%%%%%%%%%%%%%%%%%%%%%%%%%%%%%%%%%%%%%%%%%%%%%%%%%%%%%%%%%%%%%%%%%%%
\subsection{Driver Development}
\label{sec:driver}

The \textsf{childdoc} mechanism can also be use for the development
of definition files such as \LaTeX{} styles or classes.
This case differs from the above setup with multiple parts
included by |\include| in that no |\includeonly| should be invoked.
This can be achieved by starting the include file
(before |\ProvidesPackage|) with:
%
\begin{center}
\begin{tabular}{l}
|\input{childdoc.def}|\\
|\childdocforward{|\textit{main}|}|\\
\end{tabular}
\end{center}
%
or alternatively with:
%
\begin{center}
\begin{tabular}{l}
|\input{childdoc.def}|\\
|\childdocby{|\textit{main}|}|\\
\end{tabular}
\end{center}
%
Both forms have slightly different effects as described above.
The main file is prepared as usual, see \secref{sec:include}.

%%%%%%%%%%%%%%%%%%%%%%%%%%%%%%%%%%%%%%%%%%%%%%%%%%%%%%%%%%%%%%%%%%%%%%%%%%%%%%%%
\subsection{Legacy Detection}
\label{sec:detection}

The directive |\childdocmain| in the main file can detect
whether the complete document or merely a child is to be compiled
even without using the directive |\childdocof|.
This method is deprecated because it is less robust
and there is no compelling reason to use it;
it is merely provided for backward compatibility
and it may be removed in future versions.

If the detection mechanism is to be used,
it is mandatory to correctly specify
the filename of the main file as the argument of |\childdocmain|:
%
\begin{center}
\begin{tabular}{l}
|\input{childdoc.def}|\\
|\childdocmain{|\textit{main}|}|\\
\end{tabular}
\end{center}
%
If |\jobname| does not match the argument \textit{main} of |\childdocmain|,
it is assumed that |\jobname| points to the child file to be compiled.
When using |\childdocmain| with the main file specified as argument,
it suffices to start a child file
with just |\input{|\textit{main}|}|
without loading of the package and using |\childdocof|.
If instead all processing is done
with the appropriate \textsf{childdoc} directives,
the argument of \textit{main} of |\childdocmain| can be empty.

An alternative version of the command line processing described
in \secref{sec:commandline} using the detection mechanism reads:
%
\begin{center}
|... -jobname "|\textit{target}|" "|[\textit{flags}]%
[|\def\jobname{|\textit{dest}|}|]|\input{|\textit{main}|}"|
\end{center}

%%%%%%%%%%%%%%%%%%%%%%%%%%%%%%%%%%%%%%%%%%%%%%%%%%%%%%%%%%%%%%%%%%%%%%%%%%%%%%%%
\subsection{Manual Code}
\label{sec:manual}

In case one cannot be certain whether the definitions file |childdoc.def|
is installed on the target \TeX{} distribution
and one prefers not to ship it,
it is conceivable to paste a few relevant commands into the sources.

To that end, drop all statements |\input{childdoc.def}|
and perform the replacements as outlined below.
Instead of |\childdocmain{|\textit{main}|}| add the following code
to the top of the main file:
%
\begin{center}
\begin{tabular}{l}
|\||ifdefined\childdocname\endinput\||fi\newif\ifchilddoc|\\
|\edef\childdocname{\scantokens\expandafter{\jobname\noexpand}}|\\
|\def\childdocmain{|\textit{main}|}\||ifx\childdocmain\childdocname\||else|\\
|\childdoctrue\includeonly{\childdocname}\let\jobname\childdocmain\||fi|\\
\end{tabular}
\end{center}
%
Instead of |\childdocof{|\textit{main}|}| just include the main file
at the top of each child file:
%
\begin{center}
|\input{|\textit{main}|}|
\end{center}
%
A simple redirection |\childdocforward{|\textit{dest}|}| is achieved by:
%
\begin{center}
|\def\jobname{|\textit{dest}|}\input{\jobname}|
\end{center}
%
The redirection with prefix
|\childdocforwardprefix[|\textit{prefix}|]{|\textit{dest}|}|
is accomplished by:
%
\begin{center}
\begin{tabular}{l}
|{\edef\jobname{\scantokens\expandafter{\jobname\noexpand}}|\\
|\def\redirectjob |\textit{prefix}|#1~~~{\gdef\jobname{|\textit{dest}|#1}}|\\
|\expandafter\redirectjob\jobname~~~}\input{\jobname}|
\end{tabular}
\end{center}

In an alternative approach,
child documents can be compiled by a specific command line
without additional code or specific definitions:
%
\begin{center}
|... -jobname "|\textit{target}|" "|[\textit{flags}]%
|\includeonly{|\textit{dest}|}\input{|\textit{main}|}"|
\end{center}
%

%%%%%%%%%%%%%%%%%%%%%%%%%%%%%%%%%%%%%%%%%%%%%%%%%%%%%%%%%%%%%%%%%%%%%%%%%%%%%%%%
%%%%%%%%%%%%%%%%%%%%%%%%%%%%%%%%%%%%%%%%%%%%%%%%%%%%%%%%%%%%%%%%%%%%%%%%%%%%%%%%
\section{Information}

%%%%%%%%%%%%%%%%%%%%%%%%%%%%%%%%%%%%%%%%%%%%%%%%%%%%%%%%%%%%%%%%%%%%%%%%%%%%%%%%
\subsection{Copyright}

Copyright \copyright{} 2017--2018 Niklas Beisert

This work may be distributed and/or modified under the
conditions of the \LaTeX{} Project Public License, either version 1.3
of this license or (at your option) any later version.
The latest version of this license is in
  \url{http://www.latex-project.org/lppl.txt}
and version 1.3 or later is part of all distributions of \LaTeX{}
version 2005/12/01 or later.

This work has the LPPL maintenance status `maintained'.

The Current Maintainer of this work is Niklas Beisert.

This work consists of the files |README.txt|, |childdoc.ins| and |childdoc.dtx|
as well as the derived files |childdoc.def|, |cdocsamp.tex|
with |cdocsch1.tex|, |cdocsch2.tex|, |cdocspt3.tex|, |cdocspt4.tex|,
|cdocsdrf.tex|, |cdocsfn1.tex|, |cdocsfn2.tex|
as well as |childdoc.pdf|.

%%%%%%%%%%%%%%%%%%%%%%%%%%%%%%%%%%%%%%%%%%%%%%%%%%%%%%%%%%%%%%%%%%%%%%%%%%%%%%%%
\subsection{Files and Installation}

The package consists of the files:
%
\begin{center}
\begin{tabular}{ll}
    |README.txt|   & readme file \\
    |childdoc.ins| & installation file \\
    |childdoc.dtx| & source file \\
    |childdoc.def| & definition file \\
    |cdocsamp.tex| & sample main file \\
    |cdocsch1.tex| & sample include file \\
    |cdocsch2.tex| & sample include file \\
    |cdocspt3.tex| & sample part file \\
    |cdocspt4.tex| & sample part file \\
    |cdocsdrf.tex| & sample redirection file \\
    |cdocsfn1.tex| & sample redirection file \\
    |cdocsfn2.tex| & sample redirection file \\
    |childdoc.pdf| & manual
\end{tabular}
\end{center}
%
The distribution consists of the files
|README.txt|, |childdoc.ins| and |childdoc.dtx|.
%
\begin{itemize}
\item
Run (pdf)\LaTeX{} on |childdoc.dtx|
to compile the manual |childdoc.pdf| (this file).
\item
Run \LaTeX{} on |childdoc.ins| to create the definitions file |childdoc.def|
and the sample |cdocsamp.tex| with include files
|cdocsch1.tex|, |cdocsch2.tex|, |cdocspt3.tex|, |cdocspt4.tex|,
|cdocsdrf.tex|, |cdocsfn1.tex|, |cdocsfn2.tex|.
Then copy the file |childdoc.def| to an appropriate directory of your \LaTeX{}
distribution, e.g.\ \textit{texmf-root}|/tex/latex/childdoc|.
\end{itemize}

%%%%%%%%%%%%%%%%%%%%%%%%%%%%%%%%%%%%%%%%%%%%%%%%%%%%%%%%%%%%%%%%%%%%%%%%%%%%%%%%
\subsection{Related CTAN Packages}

There are several other packages which offer a similar functionality:
%
\begin{itemize}
\item
The packages
\href{http://ctan.org/pkg/docmute}{\textsf{docmute}},
\href{http://ctan.org/pkg/includex}{\textsf{includex}} and
\href{http://ctan.org/pkg/standalone}{\textsf{standalone}}
provide commands to include only the document body of
a child file thus allowing both files to be compiled individually.
\item
The packages \href{http://ctan.org/pkg/subdocs}{\textsf{subdocs}}
and \href{http://ctan.org/pkg/subfiles}{\textsf{subfiles}}
provide structures in which the main and child documents can be
encapsulated and allowing them to be compiled individually.
The inclusion mechanism is different from the conventional |\include|.
\item
The package \href{http://ctan.org/pkg/combine}{\textsf{combine}}
is an elaborate solution to combine several documents into one.
\end{itemize}
%
See also the CTAN topic \href{http://ctan.org/topic/subdocs}{\textsf{subdocs}}
for further related packages.
The present package differs from the above solutions in that
a document structure constructed with the conventional |\include| mechanism
just needs two extra commands at the top of every file
such that all constituent files can be compiled individually.

%%%%%%%%%%%%%%%%%%%%%%%%%%%%%%%%%%%%%%%%%%%%%%%%%%%%%%%%%%%%%%%%%%%%%%%%%%%%%%%%
%\subsection{Feature Suggestions}
%
%The following is a list of features which may be useful for future
%versions of this package:
%%
%\begin{itemize}
%\item
%\ldots
%\end{itemize}

%%%%%%%%%%%%%%%%%%%%%%%%%%%%%%%%%%%%%%%%%%%%%%%%%%%%%%%%%%%%%%%%%%%%%%%%%%%%%%%%
\subsection{Revision History}

%%%%%%%%%%%%%%%%%%%%%%%%%%%%%%%%%%%%%%%%
\paragraph{v2.0:} 2018/12/30

\begin{itemize}
\item
immediate forward processing
\item
added |\childdocby| mechanism
\item
manual restructured
\end{itemize}

%%%%%%%%%%%%%%%%%%%%%%%%%%%%%%%%%%%%%%%%
\paragraph{v1.6:} 2018/01/17

\begin{itemize}
\item
application for development of include files
\item
corrections to manual
\end{itemize}

%%%%%%%%%%%%%%%%%%%%%%%%%%%%%%%%%%%%%%%%
\paragraph{v1.5:} 2017/05/21

\begin{itemize}
\item
more complete structuring introduced
\item
|\childdocof| introduced
\item
|\childdoc| renamed to |\childdocmain|
\item
|\childredirect| renamed to |\childdocforward| and |\childdocforwardprefix|
and functionality expanded
\end{itemize}

%%%%%%%%%%%%%%%%%%%%%%%%%%%%%%%%%%%%%%%%
\paragraph{v1.0:} 2017/04/27

\begin{itemize}
\item
manual and install package
\item
first version published on CTAN
\end{itemize}

%%%%%%%%%%%%%%%%%%%%%%%%%%%%%%%%%%%%%%%%
\paragraph{v0.6:} 2017/04/26

\begin{itemize}
\item
redirection mechanism added
\end{itemize}

%%%%%%%%%%%%%%%%%%%%%%%%%%%%%%%%%%%%%%%%
\paragraph{v0.5:} 2017/04/26

\begin{itemize}
\item
functionality in definition file
\end{itemize}


%%%%%%%%%%%%%%%%%%%%%%%%%%%%%%%%%%%%%%%%%%%%%%%%%%%%%%%%%%%%%%%%%%%%%%%%%%%%%%%%
%%%%%%%%%%%%%%%%%%%%%%%%%%%%%%%%%%%%%%%%%%%%%%%%%%%%%%%%%%%%%%%%%%%%%%%%%%%%%%%%
%%%%%%%%%%%%%%%%%%%%%%%%%%%%%%%%%%%%%%%%%%%%%%%%%%%%%%%%%%%%%%%%%%%%%%%%%%%%%%%%
\appendix

\settowidth\MacroIndent{\rmfamily\scriptsize 000\ }

 \DocInput{childdoc.dtx}

\end{document}
%</driver>
% \fi
%
% %%%%%%%%%%%%%%%%%%%%%%%%%%%%%%%%%%%%%%%%%%%%%%%%%%%%%%%%%%%%%%%%%%%%%%%%%%%%%%
% %%%%%%%%%%%%%%%%%%%%%%%%%%%%%%%%%%%%%%%%%%%%%%%%%%%%%%%%%%%%%%%%%%%%%%%%%%%%%%
% \section{Sample}
%\iffalse
%<*samplemain>
%\fi
%
% The following presents a sample document
% with two chapters, two parts, a title page,
% a compile flag as well as three forwarding files to set the flag.
% It consists of eight |.tex| files:
% \begin{center}
% \begin{tabular}{ll}
% |cdocsamp.tex|&main file\\
% |cdocsch1.tex|&include file for chapter 1\\
% |cdocsch2.tex|&include file for chapter 2\\
% |cdocspt3.tex|&include file for part 3\\
% |cdocspt4.tex|&include file for part 4\\
% |cdocsdrf.tex|&forwarding file for main file in draft mode\\
% |cdocsfi1.tex|&forwarding file for final version of chapter 1\\
% |cdocsfi2.tex|&forwarding file for final version of chapter 2\\
% \end{tabular}
% \end{center}
% Each of the eight files can be compiled directly by the \LaTeX{} compiler.
%
% %%%%%%%%%%%%%%%%%%%%%%%%%%%%%%%%%%%%%%
% \paragraph{Main File.}
%
% The main file is called |cdocsamp.tex|.
%
% Load the \textsf{childdoc} definitions and
% declare the filename for the main document:
%    \begin{macrocode}
\input{childdoc.def}
\childdocmain{}
%    \end{macrocode}

% Optional override for |\version| flag:
%    \begin{macrocode}
%%\ifchilddoc\else\providecommand{\version}{draft}\fi
%    \end{macrocode}

% Define the default values for the |\version| flag
% (|final| for the main file and |draft| for childs):
%    \begin{macrocode}
\ifchilddoc
\providecommand{\version}{draft}
\else
\providecommand{\version}{final}
\fi
%    \end{macrocode}

% Load the standard document class:
%    \begin{macrocode}
\documentclass[12pt]{article}
%    \end{macrocode}

% Start the document body:
%    \begin{macrocode}
\begin{document}
%    \end{macrocode}

% Declare a title page.
% Print title, part of document being processed and version flag:
%    \begin{macrocode}
\addtocounter{page}{-1}
\begin{center}
{\LARGE\bfseries{}childdoc example\par}
\vspace{1cm}
\ifchilddoc
\ifchilddocmanual part\else chapter\fi:
`\childdocname' of `\childdocjob'\par
\else
main document: `\childdocjob'\par
\fi
version: \version\par
\end{center}
\newpage
%    \end{macrocode}

% Manually include selected file,
% otherwise process as usual:
%    \begin{macrocode}
\ifchilddocmanual
\section*{part `\childdocname'}
\input{\childdocname}
\else
%    \end{macrocode}

% Include the two chapters:
%    \begin{macrocode}
\include{cdocsch1}
\include{cdocsch2}
%    \end{macrocode}

% Include the two parts unless only chapters should be displayed:
%    \begin{macrocode}
\ifchilddoc\else
\section{part three}
\input{cdocspt3}
\section{part four}
\input{cdocspt4}
\fi
%    \end{macrocode}

% Process as usual until here:
%    \begin{macrocode}
\fi
%    \end{macrocode}

% End of document body:
%    \begin{macrocode}
\end{document}
%    \end{macrocode}
%\iffalse
%</samplemain>
%\fi
%
% %%%%%%%%%%%%%%%%%%%%%%%%%%%%%%%%%%%%%%
% \paragraph{Chapter Include Files.}
%
% The include files are called |cdocsch1.tex| and |cdocsch2.tex|.
%
%\iffalse
%<*samplechap1|samplechap2>
%\fi

% Optional override for |\version| flag:
%    \begin{macrocode}
%%\providecommand{\version}{final}
%    \end{macrocode}

% Include the main document:
%    \begin{macrocode}
\input{childdoc.def}
\childdocof{cdocsamp}
%    \end{macrocode}

%\iffalse
%</samplechap1|samplechap2>
%\fi
%
%\iffalse
%<*samplechap1>
%\fi
% Some text for chapter 1:
%    \begin{macrocode}
\section{one}
some text in chapter one
%    \end{macrocode}

%\iffalse
%</samplechap1>
%\fi
% Some text for chapter 2:
%\iffalse
%<*samplechap2>
%\fi
%    \begin{macrocode}
\section{two}
more text in chapter two
%    \end{macrocode}

%\iffalse
%</samplechap2>
%\fi
%
% %%%%%%%%%%%%%%%%%%%%%%%%%%%%%%%%%%%%%%
% \paragraph{Part Include Files.}
%
% The include files are called |cdocspt3.tex| and |cdocspt4.tex|.
%
%\iffalse
%<*samplepart3|samplepart4>
%\fi

% Optional override for |\version| flag:
%    \begin{macrocode}
%%\providecommand{\version}{final}
%    \end{macrocode}

% Include the main document:
%    \begin{macrocode}
\input{childdoc.def}
\childdocby{cdocsamp}
%    \end{macrocode}

%\iffalse
%</samplepart3|samplepart4>
%\fi
%
%\iffalse
%<*samplepart3>
%\fi
% Some text for part 3:
%    \begin{macrocode}
some text in part three
%    \end{macrocode}

%\iffalse
%</samplepart3>
%\fi
% Some text for part 4:
%\iffalse
%<*samplepart4>
%\fi
%    \begin{macrocode}
more text in part four
%    \end{macrocode}

%\iffalse
%</samplepart4>
%\fi
%
% %%%%%%%%%%%%%%%%%%%%%%%%%%%%%%%%%%%%%%
% \paragraph{Forwarding for a Complete Draft.}
%
% The following forwarding file |cdocsdrf.tex|
% compiles the main document in draft mode:
%\iffalse
%<*sampledraft>
%\fi
%    \begin{macrocode}
\def\version{draft}
\input{childdoc.def}
\childdocforward{cdocsamp}
%    \end{macrocode}

%\iffalse
%</sampledraft>
%\fi
%
% %%%%%%%%%%%%%%%%%%%%%%%%%%%%%%%%%%%%%%
% \paragraph{Forwarding for Final Version of the Chapters.}
%
% The following forwarding files |cdocsfn1.tex| and |cdocsfn2.tex|
% (with identical content)
% compile the final versions of the child documents
% |cdocsch1.tex| and |cdocsch2.tex|, respectively:
%\iffalse
%<*samplefinal>
%\fi
%    \begin{macrocode}
\def\version{final}
\input{childdoc.def}
\childdocforwardprefix[cdocsamp]{cdocsfn}{cdocsch}
%    \end{macrocode}

%\iffalse
%</samplefinal>
%\fi
%
% %%%%%%%%%%%%%%%%%%%%%%%%%%%%%%%%%%%%%%
% \paragraph{Command Line Processing.}
%
% The following three command lines generate the output files
% |cdocscld|, |cdocscl1| and |cdocscl2|
% which should be identical to
% |cdocsdrf|, |cdocsch1| and |cdocsfn2|, respectively:
% \begin{center}
% \begin{tabular}{l}
% |latex -jobname cdocscld \|\\
% |  "\def\version{draft}\input{childdoc.def}\childdocforward{cdocsamp}"|\\
% |latex -jobname cdocscl1 \|\\
% |  "\input{childdoc.def}\childdocforward[cdocsamp]{cdocsch1}"|\\
% |latex -jobname cdocscl2 \|\\
% |  "\def\version{final}\input{childdoc.def}\childdocforward{cdocsch2}"|
% \end{tabular}
% \end{center}
% Note that the trailing backslash on each first line
% merely continues the input to the second line
% (for convenient cut ant paste).
% Furthermore, the command |latex| can be replaced by any
% of its alternative versions such as |pdflatex|.
%
% %%%%%%%%%%%%%%%%%%%%%%%%%%%%%%%%%%%%%%%%%%%%%%%%%%%%%%%%%%%%%%%%%%%%%%%%%%%%%%
% %%%%%%%%%%%%%%%%%%%%%%%%%%%%%%%%%%%%%%%%%%%%%%%%%%%%%%%%%%%%%%%%%%%%%%%%%%%%%%
% \section{Implementation}
%\iffalse
%<*package>
%\fi
%
% This section describes the definitions file |childdoc.def|.

% The definitions cannot be loaded using |\usepackage| or |\RequirePackage|
% which has a mechanism to prevent loading a style file more than once.
% When loading the definitions by means of |\input|
% multiple instances have to be prevented manually:
%\iffalse
%This code needs to be before the `\ProvidesFile' directive
%which is defined at the beginning of this file.
%Therefore it is also placed there and commented out here.
%</package>
%<*discard>
%\fi
%    \begin{macrocode}
\ifdefined\childdocmain\endinput\fi
%    \end{macrocode}
%\iffalse
%</discard>
%<*package>
%\fi
%
% \macro{\ifchilddoc}
% \macro{\ifchilddocmanual}
% The conditional |\ifchilddoc| tells whether a
% child (true) or main (false) document is being compiled.
% The conditional |\ifchilddocmanual| tells whether
% the |\includeonly| mechanism is used (false) or
% the selection of child files must be performed manually (true).
% The definitions initialise to false:
%    \begin{macrocode}
\newif\ifchilddoc
\newif\ifchilddocmanual
%    \end{macrocode}

% \macro{\childdocname}
% \macro{\childdocjob}
% The macro |\childdocname| stores the name of the main document
% to be compiled. The macro |\childdocjob| stores the name of
% the document on which the \LaTeX{} compiler was originally invoked.
% The content of |\jobname| cannot be compared
% to filenames specified in the source due to different catcodes.
% The following code rescans |\jobname|, stores the result
% in |\childdocname| and saves a copy in |\childdocjob|:
%    \begin{macrocode}
\edef\childdocname{\scantokens\expandafter{\jobname\noexpand}}
\let\childdocjob\childdocname
%    \end{macrocode}

% \macro{\childdocdisable}
% The macro |\childdocdisable| prevents the main file
% from being processed more than once.
% At this stage, the main document command |\childdocmain|
% is assumed to be called once again where it should do nothing.
% Any subsequent call to it should prevent
% a secondary processing of the main document
% It overwrites the forwarding commands
% |\childdocof| and |\childdocforward|
% with empty macros to prevent further inclusions of the main document:
%    \begin{macrocode}
\newcommand{\childdocdisable}
{
  \renewcommand{\childdocmain}[1]{\renewcommand{\childdocmain}[1]{\endinput}}
  \renewcommand{\childdocof}[1]{}
  \renewcommand{\childdocby}[2][]{}
  \renewcommand{\childdocforward}[2][]{}
  \renewcommand{\childdocdisable}{}
}
%    \end{macrocode}

% \macro{\childdocmain}
% The macro |\childdocmain| is to be called at the top of the main file
% with nothing or the main filename (without extension) as argument.
% First, it breaks loops.
% If the argument is not empty and does not match |\childdocname|
% (which is set by the first inclusion of |childdoc.def|),
% |\ifchilddoc| is set to true, |\includeonly| is applied to the child file
% and |\jobname| is set to the main file
% (for proper handling of |.aux| files):
%    \begin{macrocode}
\newcommand{\childdocmain}[1]
{
  \childdocdisable\childdocmain{}
  \if?#1?\else
    \begingroup
      \def\childdoctmp{#1}
      \ifx\childdoctmp\childdocname
        \def\childdoctmp{}
      \else
        \def\childdoctmp
        {
          \childdoctrue
          \includeonly{\childdocname}
          \def\childdocjob{#1}
          \def\jobname{#1}
        }
      \fi
      \expandafter
    \endgroup
    \childdoctmp
  \fi
}
%    \end{macrocode}

% \macro{\childdocof}
% The command |\childdocof| redirects
% compilation to the main file |#1|.
%    \begin{macrocode}
\newcommand{\childdocof}[1]
{
  \childdocdisable
  \childdoctrue
  \includeonly{\childdocname}
  \def\jobname{#1}
  \def\childdocjob{#1}
  \input{#1}
}
%    \end{macrocode}

% \macro{\childdocby}
% The command |\childdocby| ....
%    \begin{macrocode}
\newcommand{\childdocby}[2][]
{
  \childdocdisable
  \childdoctrue
  \childdocmanualtrue
  \if?#1?\else
    \def\jobname{#2}
  \fi
  \def\childdocjob{#2}
  \input{#2}
  \endinput
}
%    \end{macrocode}

% \macro{\childdocforward}
% The command |\childdocforward| redirects
% compilation to the main file or
% (if the optional argument is given) a child file.
% Parameters are set as if the main file
% or a child file starting with |\childdocof| was compiled.
% Then compilation is handed over to the main file:
%    \begin{macrocode}
\newcommand{\childdocforward}[2][]
{
  \begingroup
    \if?#1?
      \def\childdoctmp
      {
        \def\childdocname{#2}
        \def\childdocjob{#2}
        \def\jobname{#2}
        \input{#2}
        \endinput
      }
    \else
      \def\childdoctmp
      {
        \childdocdisable
        \def\childdocname{#2}
        \childdoctrue
        \includeonly{#2}
        \def\childdocjob{#1}
        \def\jobname{#1}
        \input{#1}
        \endinput
      }
    \fi
    \expandafter
  \endgroup
  \childdoctmp
}
%    \end{macrocode}

% \macro{\childdocforwardprefix}
% The command |\childdocforwardprefix| redirects
% compilation to the main or a child file by means of a pattern.
% The prefix |#1| in the current filename is replaced by |#2|
% and the suffix of the current filename is kept
% (it is assumed that the filename does not contain the substring `|~~~|'
% which is used as a delimiter).
% Compilation is handed over to the new file by |\childdocforward|:
%    \begin{macrocode}
\newcommand{\childdocforwardprefix}[3][]
{
  \begingroup
    \def\childdocextract #2##1~~~{\def\childdoctmp{\childdocforward[#1]{#3##1}}}
    \expandafter\childdocextract\childdocname~~~
    \expandafter
  \endgroup
  \childdoctmp
}
%    \end{macrocode}

% \macro{\childdoc}
% The deprecated macro |\childdoc| is a legacy version of |\childdocmain|:
%    \begin{macrocode}
\newcommand{\childdoc}{\childdocmain}
%    \end{macrocode}

% \macro{\childdocredirect}
% The deprecated macro |\childdocredirect| is a legacy version
% of |\childdocforward| and |\childdocforwardprefix|:
%    \begin{macrocode}
\newcommand{\childdocredirect}[2][]
{
  \begingroup
    \if?#1?
      \def\childdoctmp{\childdocforward{#2}}
    \else
      \def\childdoctmp{\childdocforwardprefix{#1}{#2}}
    \fi
    \expandafter
  \endgroup
  \childdoctmp
}
%    \end{macrocode}

%\iffalse
%</package>
%\fi
%
\endinput
|\\
|\childdocby{|\textit{main}|}|\\
\end{tabular}
\end{center}
%
The directive |\childdocby| is similar to |\childdocof|
described in \secref{sec:include},
but the subsequent selection of content must be done manually.
To that end, both |\ifchilddoc| and |\ifchilddocmanual|
will be true upon processing of a part,
and the name of the part is stored in |\childdocname|.
Note that |\jobname| will be set to the filename of the current part
so that each part receives an individual |.aux| file
that does not interfere with the |.aux| file(s) of the main document.
This behaviour can be altered by the alternative form
|\childdocby[*]{|\textit{main}|}| (with a non-empty optional argument)
which uses the |.aux| file of the main document
by setting |\jobname| to \textit{main}.

%%%%%%%%%%%%%%%%%%%%%%%%%%%%%%%%%%%%%%%%%%%%%%%%%%%%%%%%%%%%%%%%%%%%%%%%%%%%%%%%
\subsection{Driver Development}
\label{sec:driver}

The \textsf{childdoc} mechanism can also be use for the development
of definition files such as \LaTeX{} styles or classes.
This case differs from the above setup with multiple parts
included by |\include| in that no |\includeonly| should be invoked.
This can be achieved by starting the include file
(before |\ProvidesPackage|) with:
%
\begin{center}
\begin{tabular}{l}
|% \iffalse
%
% childdoc.dtx Copyright (C) 2017-2018 Niklas Beisert
%
% This work may be distributed and/or modified under the
% conditions of the LaTeX Project Public License, either version 1.3
% of this license or (at your option) any later version.
% The latest version of this license is in
%   http://www.latex-project.org/lppl.txt
% and version 1.3 or later is part of all distributions of LaTeX
% version 2005/12/01 or later.
%
% This work has the LPPL maintenance status `maintained'.
%
% The Current Maintainer of this work is Niklas Beisert.
%
% This work consists of the files childdoc.dtx and childdoc.ins
% and the derived files childdoc.def and cdocsamp.tex with
% cdocsch1.tex, cdocsch2.tex, cdocsdrf.tex, cdocsfn1.tex, cdocsfn2.tex.
%
%<package>\ifdefined\childdocmain\endinput\fi
%<package>\ProvidesFile{childdoc.def}[2018/12/30 v2.0 child document driver]
%<samplemain>\ProvidesFile{cdocsamp.tex}[2018/12/30 v2.0 sample for childdoc]
%<*driver>
%\ProvidesFile{childdoc.drv}[2018/12/30 v2.0 childdoc reference manual file]
\PassOptionsToClass{10pt,a4paper}{article}
\documentclass{ltxdoc}

\usepackage[margin=35mm]{geometry}
\usepackage{hyperref}
\usepackage{hyperxmp}
\usepackage[usenames]{color}

\hypersetup{colorlinks=true}
\hypersetup{pdfstartview=FitH}
\hypersetup{pdfpagemode=UseNone}
\hypersetup{pdfsource={}}
\hypersetup{pdflang={en-UK}}
\hypersetup{pdfcopyright={Copyright 2017-2018 Niklas Beisert.
  This work may be distributed and/or modified under the
  conditions of the LaTeX Project Public License, either version 1.3
  of this license or (at your option) any later version.}}
\hypersetup{pdflicenseurl={http://www.latex-project.org/lppl.txt}}
\hypersetup{pdfcontactaddress={ETH Zurich, ITP, HIT K,
  Wolfgang-Pauli-Strasse 27}}
\hypersetup{pdfcontactpostcode={8093}}
\hypersetup{pdfcontactcity={Zurich}}
\hypersetup{pdfcontactcountry={Switzerland}}
\hypersetup{pdfcontactemail={nbeisert@itp.phys.ethz.ch}}
\hypersetup{pdfcontacturl={http://people.phys.ethz.ch/\xmptilde nbeisert/}}

\newcommand{\secref}[1]{\hyperref[#1]{section \ref*{#1}}}

\parskip1ex
\parindent0pt
\let\olditemize\itemize
\def\itemize{\olditemize\parskip0pt}

\begin{document}

\title{The \textsf{childdoc} Package}
\hypersetup{pdftitle={The childdoc Package}}
\author{Niklas Beisert\\[2ex]
  Institut f\"ur Theoretische Physik\\
  Eidgen\"ossische Technische Hochschule Z\"urich\\
  Wolfgang-Pauli-Strasse 27, 8093 Z\"urich, Switzerland\\[1ex]
  \href{mailto:nbeisert@itp.phys.ethz.ch}
  {\texttt{nbeisert@itp.phys.ethz.ch}}}
\hypersetup{pdfauthor={Niklas Beisert}}
\hypersetup{pdfsubject={Manual for the LaTeX2e Package childdoc}}
\date{30 December 2018, \textsf{v2.0}}
\maketitle

\begin{abstract}\noindent
\textsf{childdoc} is a \LaTeXe{} package
that enables the direct compilation
of document sections included by |\include|
to individual files.
\end{abstract}

\begingroup
\parskip0ex
\tableofcontents
\endgroup

%%%%%%%%%%%%%%%%%%%%%%%%%%%%%%%%%%%%%%%%%%%%%%%%%%%%%%%%%%%%%%%%%%%%%%%%%%%%%%%%
%%%%%%%%%%%%%%%%%%%%%%%%%%%%%%%%%%%%%%%%%%%%%%%%%%%%%%%%%%%%%%%%%%%%%%%%%%%%%%%%
\section{Introduction}

\LaTeX{} provides a mechanism to structure a large document (such as a book)
into a main file and several child files (containing the chapters)
using the |\include| command.
This mechanism is beneficial for documents
which span hundreds of pages in order to
make the source file(s) more manageable.
Moreover, compilation can be restricted to
selected child files by means of the |\includeonly| command.
The latter feature can be used to reduce the compilation time while editing
(this was significantly more useful in the earlier days of \LaTeX{})
or to generate a smaller document which is easier to navigate.
Another application of |\includeonly| is to generate
documents consisting of selected parts of the complete document.

However, there are a few drawbacks of the plain |\include| mechanism:
\begin{itemize}
\item
The child files cannot be compiled on their own,
they can only be compiled via the main file.
A naive editing environment
(such as a text editor with an option
to have the current file processed by \LaTeX)
may require one to switch to the main file before compiling;
attempting to compile the child file produces errors.
\item
The main file must be modified (each time)
to adjust the |\includeonly| command
to the present needs. This easily leaves the main file in a messy state.
\item
The generated document will always carry the filename
of the main document. This is inconvenient if
several child files are to be compiled and
to be kept for distribution.
\end{itemize}

The present package provides a simple interface
to make child files individually compilable by \LaTeX{}.
Compiling a child file then has the same effect as compiling
the main file with an |\includeonly| command
to select the appropriate child.
Moreover the generated document will carry the name of the child
rather than the main file.
This resolves all three above issues.

This feature is meant to make the editing of books,
thesis documents and lecture notes somewhat more convenient.
However, the package can also be used efficiently for
composing a series of documents (such as exercise sheets)
which are typically distributed individually.
It then assists the author in generating the individual documents
(potentially in different versions)
as well as a document containing the collected series.
Another application is in developing style files
or other kinds of included material
where compilation of the style file could redirect
to a sample or test file.

%%%%%%%%%%%%%%%%%%%%%%%%%%%%%%%%%%%%%%%%%%%%%%%%%%%%%%%%%%%%%%%%%%%%%%%%%%%%%%%%
%%%%%%%%%%%%%%%%%%%%%%%%%%%%%%%%%%%%%%%%%%%%%%%%%%%%%%%%%%%%%%%%%%%%%%%%%%%%%%%%
\section{Usage}

First of all, the package \textsf{childdoc} is \emph{not} a standard
\LaTeXe{} |.sty| style file! Therefore it needs to be invoked in
a non-standard way.

%%%%%%%%%%%%%%%%%%%%%%%%%%%%%%%%%%%%%%%%%%%%%%%%%%%%%%%%%%%%%%%%%%%%%%%%%%%%%%%%
\subsection{Included Files}
\label{sec:include}

%%%%%%%%%%%%%%%%%%%%%%%%%%%%%%%%%%%%%%%%
\DescribeMacro{\childdocmain}
To use the package, add the commands
\begin{center}
\begin{tabular}{l}
|\input{childdoc.def}|\\
|\childdocmain{}|\\
\end{tabular}
\end{center}
at the very top of the main \LaTeX{} file,
in particular \emph{before} the |\documentclass| statement!
The argument of |\childdocmain| should be left empty
(but it must be present).

%%%%%%%%%%%%%%%%%%%%%%%%%%%%%%%%%%%%%%%%
\DescribeMacro{\childdocof}
Furthermore, add the commands
\begin{center}
\begin{tabular}{l}
|\input{childdoc.def}|\\
|\childdocof{|\textit{main}|}|\\
\end{tabular}
\end{center}
at the top of every child file \textit{child}
which is included by |\include{|\textit{child}|}|
from within the main file
(or at least for those files to be compiled individually).
The argument \textit{main} must be the filename of the main file.

There are a couple of
considerations in setting up the main and child documents:

%%%%%%%%%%%%%%%%%%%%%%%%%%%%%%%%%%%%%%%%
\paragraph{Restrictions.}

Please note the following restrictions:
\begin{itemize}
\item
|\childdocmain| must be called with one argument \textit{main}
to ensure compatibility with earlier version of the package.
It must either be empty (|\childdocmain{}|)
or precisely match the filename of the main file in which it is specified.
See \secref{sec:detection} for further information.
\item
The filename \textit{main} must be specified without the |.tex| extension.
\item
The filename \textit{main} is case sensitive
(even in case-insensitive file systems)
due to internal string comparison.
\item
The argument \textit{main} should be fully expanded, it cannot be a macro.
\item
Subdirectories and special characters should be avoided in filenames.
\item
The command |\childdocmain{|\textit{main}|}| must be followed by a whitespace.
It should not be followed immediately by another command
or by a comment mark `|%|'.
This is because the \TeX{} parser reads the token immediately following
the argument of |\childdocmain| and puts it
at the beginning of every child section;
however, a white\-space is ignored.
\end{itemize}

%%%%%%%%%%%%%%%%%%%%%%%%%%%%%%%%%%%%%%%%
\paragraph{Content of Main File.}

It is advisable to place all content in the child files included by |\include|.
Any output contained in the main file will appear in all child documents
unless suppressed manually;
it cannot be suppressed automatically by the |\includeonly| directive
and thus should normally be avoided.
A method to include some content in the main file
by means of conditional processing is described in \secref{sec:conditional}.

%%%%%%%%%%%%%%%%%%%%%%%%%%%%%%%%%%%%%%%%
\paragraph{Page Numbering.}

When only a part of the document is compiled,
the appropriate numbering of pages
(as well as other status parameters)
is determined from the |.aux| files.
The latter contain information from previous passes.
However this information needs to propagate through
all intermediate child documents.
Therefore the page numbering in child documents may well
be inconsistent until the complete document is compiled at least once.

A useful (if unconventional) way to always ensure a consistent
page numbering is to restart the numbering in each child document
and denote the pages by `\textit{child}|.|\textit{page}'
where \textit{child} represents the chapter/section number of the child file.
This can be achieved by the command
|\numberwithin{page}{|\textit{child}|}|
of the \textsf{amsmath} package
where \textit{child} can be |chapter| or |section|
depending on the chosen structuring.
Alternatively, one can modify the macro |\thepage| appropriately
and reset the counter |page| at the start of each child file.

%%%%%%%%%%%%%%%%%%%%%%%%%%%%%%%%%%%%%%%%%%%%%%%%%%%%%%%%%%%%%%%%%%%%%%%%%%%%%%%%
\subsection{Conditional Processing}
\label{sec:conditional}

The package provides a mechanism to compile different versions
of a document. To customise the versions further some conditional processing
can come in handy to distinguish which version is being compiled.
The package provides two macros to describe the compilation context:

%%%%%%%%%%%%%%%%%%%%%%%%%%%%%%%%%%%%%%%%
\DescribeMacro{\ifchilddoc}
The conditional |\ifchilddoc| distinguishes between the compilation of
child documents and the main document:
%
\begin{center}
|\ifchilddoc |\textit{child-code}| |[|\||else |\textit{main-code}]| \||fi|
\end{center}

%%%%%%%%%%%%%%%%%%%%%%%%%%%%%%%%%%%%%%%%
\DescribeMacro{\childdocname}
\DescribeMacro{\childdocjob}
The macro |\childdocname| contains the filename (without extension)
of the main or child file being processed.
Note that |\childdocjob| will always contain the name of the main file.

%%%%%%%%%%%%%%%%%%%%%%%%%%%%%%%%%%%%%%%%
\paragraph{Title Page.}

Conditional processing can be used to include a title or banner page
in the main document when proper precautions are taken.
Importantly, the code in the main file should ensure that the page counter
(as well as other status parameters which are stored in the |.aux| files)
takes the same value after the conditional processing.
Otherwise the page numbers may take divergent values
depending on which part is compiled.

For example, a title page could be declared by:
%
\begin{center}
\begin{tabular}{l}
|\ifchilddoc\||else|\\
|\addtocounter{page}{-1}|\\
\textit{code for title page}\\
|\newpage|\\
|\||fi|
\end{tabular}
\end{center}
%
A banner page for the child documents can be generated by:
%
\begin{center}
\begin{tabular}{l}
|\ifchilddoc|\\
|\addtocounter{page}{-1}|\\
\textit{code for banner page}\\
|\newpage|\\
|\||fi|
\end{tabular}
\end{center}
%
Here one could write a message such as:
\begin{center}
|This is the part \childdocname{} of \childdocjob{}.|
\end{center}

%%%%%%%%%%%%%%%%%%%%%%%%%%%%%%%%%%%%%%%%%%%%%%%%%%%%%%%%%%%%%%%%%%%%%%%%%%%%%%%%
\subsection{Flags}
\label{sec:flags}

The package makes it easy to generate different versions
of the main or child documents.
To this end compilation flags can be defined
and assigned different default values.
They will be particularly useful in conjunction
with the forwarding mechanism described in \secref{sec:forward}.

For example, it may be useful to have a flag |\version|
which can be set to |draft| or |final|.
The document source will contain some conditional code
depending on the value of |\version|.
Suppose further, the flag should default to |final| for the main file
and to |draft| for child files
which is a natural assignment for editing the document.
This is achieved by placing the following code
in the preamble of the main document
(below the |\childdocmain| directive):
%
\begin{center}
\begin{tabular}{l}
|\ifchilddoc|\\
|\providecommand{\version}{draft}|\\
|\||else|\\
|\providecommand{\version}{final}|\\
|\||fi|
\end{tabular}
\end{center}
%
The definition by |\providecommand| makes sure
that previous definitions are not overwritten.
Further statements |\providecommand{\version}{...}|
can thus be added before the above code to override it.

For the main file, one might add a line
(between |\childdocmain| and the above block)
%
\begin{center}
|%\ifchilddoc\||else\providecommand{\version}{draft}\||fi|
\end{center}
%
which can be uncommented to produce a draft version.
Likewise one can add a line to the very top of a child file
(above the |\childdocof{|\textit{main}|}| directive)
%
\begin{center}
|%\providecommand{\version}{final}|
\end{center}
%
which can be uncommented to produce the final version of this child document.

%%%%%%%%%%%%%%%%%%%%%%%%%%%%%%%%%%%%%%%%%%%%%%%%%%%%%%%%%%%%%%%%%%%%%%%%%%%%%%%%
\subsection{Forwarding}
\label{sec:forward}

Different versions of the main or child documents
using compilation flags as described in \secref{sec:flags}
can be (permanently) stored in different files
for convenient compilation, viewing and distribution.
To this end, the package defines a command
to pass on compilation to a different file:

%%%%%%%%%%%%%%%%%%%%%%%%%%%%%%%%%%%%%%%%
\DescribeMacro{\childdocforward}
The command |\childdocforward| redirects processing to
another source file:
%
\begin{center}
\begin{tabular}{l}
|\input{childdoc.def}|\\
|\childdocforward[|\textit{main}|]{|\textit{dest}|}|\\
\end{tabular}
\end{center}
%
The argument \textit{dest} is the destination file
(without extension).
It should be the main file or one of the child files.
Note that further \textsf{childdoc} directives
such as |\childdocof| and |\childdocforward|
in the indicated file will be processed in this form.
The optional argument \textit{main}
passes on directly to the main file \textit{main}
while pretending to compile the child \textit{dest}.
This form behaves as if \textit{dest}
issues |\childdocof{|\textit{main}|}| right away,
and no further \textsf{childdoc} directives will be processed.

%%%%%%%%%%%%%%%%%%%%%%%%%%%%%%%%%%%%%%%%
\DescribeMacro{\...prefix}
In the alternative form |\childdocforwardprefix|,
%
\begin{center}
\begin{tabular}{l}
|\input{childdoc.def}|\\
|\childdocforwardprefix[|\textit{main}|]{|\textit{prefix}|}{|\textit{dest}|}|
\end{tabular}
\end{center}
%
the destination file is determined by a pattern
depending on the current file:
To make this work, the current file must be called
`{\textit{prefix}\hspace{0.2em}\textit{suffix}}'
with \textit{prefix} matching precisely the argument.
Processing is then passed on to the file
`{\textit{dest}\hspace{0.2em}\textit{suffix}}'.
Surely, the same effect is achieved by
directly specifying the
argument `{\textit{dest}\hspace{0.2em}\textit{suffix}}'
in the first form.
However, that requires to set up a different file
for each child. With the alternative form of the command
all these files can have exactly the same content
which simplifies setting them up and maintaining them.

For example, the following file |draft.tex|
with a compilation flag |\version| as described in \secref{sec:flags}
compiles the main document as a draft:
%
\begin{center}
\begin{tabular}{l}
|\def\version{draft}|\\
|\input{childdoc.def}|\\
|\childdocforward{|\textit{main}|}|
\end{tabular}
\end{center}
%
Likewise, the following files |final|\textit{nn}|.tex|
compile the final version of the child document
|child|\textit{nn}|.tex|:
%
\begin{center}
\begin{tabular}{l}
|\def\version{final}|\\
|\input{childdoc.def}|\\
|\childdocforwardprefix{final}{child}|
\end{tabular}
\end{center}
%

Note that when several versions of a main file and/or of each child file
are to be generated, it may be convenient to set up a |Makefile| or
shell script to automatise the process.

%%%%%%%%%%%%%%%%%%%%%%%%%%%%%%%%%%%%%%%%%%%%%%%%%%%%%%%%%%%%%%%%%%%%%%%%%%%%%%%%
\subsection{Command Line Processing}
\label{sec:commandline}

The effect of redirection files can also be achieved by invoking
the \LaTeX{} compiler with a more elaborate command line.
Most conveniently this should be done as part
of a shell script or a |Makefile|.

When using \textsf{childdoc} in the main file, the following
command lines effectively perform a redirection
(note that depending on the shell being used,
backslashes may have to be doubled: `|\|' $\to$ `|\\|'):
%
\begin{center}
|... -jobname "|\textit{target}|" |\\|"|[\textit{flags}]%
|\input{childdoc.def}\childdocforward[|\textit{main}|]{|\textit{dest}|}"|
\end{center}
%
Here \textit{target} is the name of the output file,
\textit{main} is the name of the main file
and \textit{dest} is the name of the main or child file to be processed
(all filenames without extensions).
The optional argument \textit{main} can be omitted
if \textit{main} matches \textit{dest}.
Optionally, compilation \textit{flags} can be defined via |\def| commands.
This command line makes the \TeX{} engine believe
it is compiling the file \textit{target}
whose content is specified as the latter parameter.
The provided code then forwards the processing to
\textit{main} or \textit{dest} as described in \secref{sec:forward}.

%%%%%%%%%%%%%%%%%%%%%%%%%%%%%%%%%%%%%%%%%%%%%%%%%%%%%%%%%%%%%%%%%%%%%%%%%%%%%%%%
\subsection{Include by Input}
\label{sec:input}

Including child documents by |\include| has some restrictions by design.
Most notably, the content of a child document always occupies
its own set of pages; pages cannot be shared between child documents.
Usually, this behaviour makes perfect sense
because each child document contain an essential part of the document.
However, in some situations it may be desirable to compose
a document from a collection of parts
without having mandatory page breaks between then.
For this case, the package
provides a mechanism to include parts
by |\input| which can also be processed individually.
However, by construction this mechanism
requires manual handling of the content to be output.

%%%%%%%%%%%%%%%%%%%%%%%%%%%%%%%%%%%%%%%%
\DescribeMacro{\ifchilddocmanual}
The main file should be prepared as usual, see \secref{sec:include}.
However, the document body must make a distinction
between processing of an individual part and of the main document, e.g.:
%
\begin{center}
\begin{tabular}{l}
|\ifchilddocmanual|\\
|\input{\childdocname}|\\
|\||else|\\
\textit{document body with }|\input{|\textit{part}|}|\\
|\||fi|
\end{tabular}
\end{center}
%
The conditional |\ifchilddocmanual| is true whenever
a part to be included by |\input| is being compiled,
and the name of the part is stored in |\childdocname|.

%%%%%%%%%%%%%%%%%%%%%%%%%%%%%%%%%%%%%%%%
\DescribeMacro{\childdocby}
Each part to be included by |\input| should start with:
%
\begin{center}
\begin{tabular}{l}
|\input{childdoc.def}|\\
|\childdocby{|\textit{main}|}|\\
\end{tabular}
\end{center}
%
The directive |\childdocby| is similar to |\childdocof|
described in \secref{sec:include},
but the subsequent selection of content must be done manually.
To that end, both |\ifchilddoc| and |\ifchilddocmanual|
will be true upon processing of a part,
and the name of the part is stored in |\childdocname|.
Note that |\jobname| will be set to the filename of the current part
so that each part receives an individual |.aux| file
that does not interfere with the |.aux| file(s) of the main document.
This behaviour can be altered by the alternative form
|\childdocby[*]{|\textit{main}|}| (with a non-empty optional argument)
which uses the |.aux| file of the main document
by setting |\jobname| to \textit{main}.

%%%%%%%%%%%%%%%%%%%%%%%%%%%%%%%%%%%%%%%%%%%%%%%%%%%%%%%%%%%%%%%%%%%%%%%%%%%%%%%%
\subsection{Driver Development}
\label{sec:driver}

The \textsf{childdoc} mechanism can also be use for the development
of definition files such as \LaTeX{} styles or classes.
This case differs from the above setup with multiple parts
included by |\include| in that no |\includeonly| should be invoked.
This can be achieved by starting the include file
(before |\ProvidesPackage|) with:
%
\begin{center}
\begin{tabular}{l}
|\input{childdoc.def}|\\
|\childdocforward{|\textit{main}|}|\\
\end{tabular}
\end{center}
%
or alternatively with:
%
\begin{center}
\begin{tabular}{l}
|\input{childdoc.def}|\\
|\childdocby{|\textit{main}|}|\\
\end{tabular}
\end{center}
%
Both forms have slightly different effects as described above.
The main file is prepared as usual, see \secref{sec:include}.

%%%%%%%%%%%%%%%%%%%%%%%%%%%%%%%%%%%%%%%%%%%%%%%%%%%%%%%%%%%%%%%%%%%%%%%%%%%%%%%%
\subsection{Legacy Detection}
\label{sec:detection}

The directive |\childdocmain| in the main file can detect
whether the complete document or merely a child is to be compiled
even without using the directive |\childdocof|.
This method is deprecated because it is less robust
and there is no compelling reason to use it;
it is merely provided for backward compatibility
and it may be removed in future versions.

If the detection mechanism is to be used,
it is mandatory to correctly specify
the filename of the main file as the argument of |\childdocmain|:
%
\begin{center}
\begin{tabular}{l}
|\input{childdoc.def}|\\
|\childdocmain{|\textit{main}|}|\\
\end{tabular}
\end{center}
%
If |\jobname| does not match the argument \textit{main} of |\childdocmain|,
it is assumed that |\jobname| points to the child file to be compiled.
When using |\childdocmain| with the main file specified as argument,
it suffices to start a child file
with just |\input{|\textit{main}|}|
without loading of the package and using |\childdocof|.
If instead all processing is done
with the appropriate \textsf{childdoc} directives,
the argument of \textit{main} of |\childdocmain| can be empty.

An alternative version of the command line processing described
in \secref{sec:commandline} using the detection mechanism reads:
%
\begin{center}
|... -jobname "|\textit{target}|" "|[\textit{flags}]%
[|\def\jobname{|\textit{dest}|}|]|\input{|\textit{main}|}"|
\end{center}

%%%%%%%%%%%%%%%%%%%%%%%%%%%%%%%%%%%%%%%%%%%%%%%%%%%%%%%%%%%%%%%%%%%%%%%%%%%%%%%%
\subsection{Manual Code}
\label{sec:manual}

In case one cannot be certain whether the definitions file |childdoc.def|
is installed on the target \TeX{} distribution
and one prefers not to ship it,
it is conceivable to paste a few relevant commands into the sources.

To that end, drop all statements |\input{childdoc.def}|
and perform the replacements as outlined below.
Instead of |\childdocmain{|\textit{main}|}| add the following code
to the top of the main file:
%
\begin{center}
\begin{tabular}{l}
|\||ifdefined\childdocname\endinput\||fi\newif\ifchilddoc|\\
|\edef\childdocname{\scantokens\expandafter{\jobname\noexpand}}|\\
|\def\childdocmain{|\textit{main}|}\||ifx\childdocmain\childdocname\||else|\\
|\childdoctrue\includeonly{\childdocname}\let\jobname\childdocmain\||fi|\\
\end{tabular}
\end{center}
%
Instead of |\childdocof{|\textit{main}|}| just include the main file
at the top of each child file:
%
\begin{center}
|\input{|\textit{main}|}|
\end{center}
%
A simple redirection |\childdocforward{|\textit{dest}|}| is achieved by:
%
\begin{center}
|\def\jobname{|\textit{dest}|}\input{\jobname}|
\end{center}
%
The redirection with prefix
|\childdocforwardprefix[|\textit{prefix}|]{|\textit{dest}|}|
is accomplished by:
%
\begin{center}
\begin{tabular}{l}
|{\edef\jobname{\scantokens\expandafter{\jobname\noexpand}}|\\
|\def\redirectjob |\textit{prefix}|#1~~~{\gdef\jobname{|\textit{dest}|#1}}|\\
|\expandafter\redirectjob\jobname~~~}\input{\jobname}|
\end{tabular}
\end{center}

In an alternative approach,
child documents can be compiled by a specific command line
without additional code or specific definitions:
%
\begin{center}
|... -jobname "|\textit{target}|" "|[\textit{flags}]%
|\includeonly{|\textit{dest}|}\input{|\textit{main}|}"|
\end{center}
%

%%%%%%%%%%%%%%%%%%%%%%%%%%%%%%%%%%%%%%%%%%%%%%%%%%%%%%%%%%%%%%%%%%%%%%%%%%%%%%%%
%%%%%%%%%%%%%%%%%%%%%%%%%%%%%%%%%%%%%%%%%%%%%%%%%%%%%%%%%%%%%%%%%%%%%%%%%%%%%%%%
\section{Information}

%%%%%%%%%%%%%%%%%%%%%%%%%%%%%%%%%%%%%%%%%%%%%%%%%%%%%%%%%%%%%%%%%%%%%%%%%%%%%%%%
\subsection{Copyright}

Copyright \copyright{} 2017--2018 Niklas Beisert

This work may be distributed and/or modified under the
conditions of the \LaTeX{} Project Public License, either version 1.3
of this license or (at your option) any later version.
The latest version of this license is in
  \url{http://www.latex-project.org/lppl.txt}
and version 1.3 or later is part of all distributions of \LaTeX{}
version 2005/12/01 or later.

This work has the LPPL maintenance status `maintained'.

The Current Maintainer of this work is Niklas Beisert.

This work consists of the files |README.txt|, |childdoc.ins| and |childdoc.dtx|
as well as the derived files |childdoc.def|, |cdocsamp.tex|
with |cdocsch1.tex|, |cdocsch2.tex|, |cdocspt3.tex|, |cdocspt4.tex|,
|cdocsdrf.tex|, |cdocsfn1.tex|, |cdocsfn2.tex|
as well as |childdoc.pdf|.

%%%%%%%%%%%%%%%%%%%%%%%%%%%%%%%%%%%%%%%%%%%%%%%%%%%%%%%%%%%%%%%%%%%%%%%%%%%%%%%%
\subsection{Files and Installation}

The package consists of the files:
%
\begin{center}
\begin{tabular}{ll}
    |README.txt|   & readme file \\
    |childdoc.ins| & installation file \\
    |childdoc.dtx| & source file \\
    |childdoc.def| & definition file \\
    |cdocsamp.tex| & sample main file \\
    |cdocsch1.tex| & sample include file \\
    |cdocsch2.tex| & sample include file \\
    |cdocspt3.tex| & sample part file \\
    |cdocspt4.tex| & sample part file \\
    |cdocsdrf.tex| & sample redirection file \\
    |cdocsfn1.tex| & sample redirection file \\
    |cdocsfn2.tex| & sample redirection file \\
    |childdoc.pdf| & manual
\end{tabular}
\end{center}
%
The distribution consists of the files
|README.txt|, |childdoc.ins| and |childdoc.dtx|.
%
\begin{itemize}
\item
Run (pdf)\LaTeX{} on |childdoc.dtx|
to compile the manual |childdoc.pdf| (this file).
\item
Run \LaTeX{} on |childdoc.ins| to create the definitions file |childdoc.def|
and the sample |cdocsamp.tex| with include files
|cdocsch1.tex|, |cdocsch2.tex|, |cdocspt3.tex|, |cdocspt4.tex|,
|cdocsdrf.tex|, |cdocsfn1.tex|, |cdocsfn2.tex|.
Then copy the file |childdoc.def| to an appropriate directory of your \LaTeX{}
distribution, e.g.\ \textit{texmf-root}|/tex/latex/childdoc|.
\end{itemize}

%%%%%%%%%%%%%%%%%%%%%%%%%%%%%%%%%%%%%%%%%%%%%%%%%%%%%%%%%%%%%%%%%%%%%%%%%%%%%%%%
\subsection{Related CTAN Packages}

There are several other packages which offer a similar functionality:
%
\begin{itemize}
\item
The packages
\href{http://ctan.org/pkg/docmute}{\textsf{docmute}},
\href{http://ctan.org/pkg/includex}{\textsf{includex}} and
\href{http://ctan.org/pkg/standalone}{\textsf{standalone}}
provide commands to include only the document body of
a child file thus allowing both files to be compiled individually.
\item
The packages \href{http://ctan.org/pkg/subdocs}{\textsf{subdocs}}
and \href{http://ctan.org/pkg/subfiles}{\textsf{subfiles}}
provide structures in which the main and child documents can be
encapsulated and allowing them to be compiled individually.
The inclusion mechanism is different from the conventional |\include|.
\item
The package \href{http://ctan.org/pkg/combine}{\textsf{combine}}
is an elaborate solution to combine several documents into one.
\end{itemize}
%
See also the CTAN topic \href{http://ctan.org/topic/subdocs}{\textsf{subdocs}}
for further related packages.
The present package differs from the above solutions in that
a document structure constructed with the conventional |\include| mechanism
just needs two extra commands at the top of every file
such that all constituent files can be compiled individually.

%%%%%%%%%%%%%%%%%%%%%%%%%%%%%%%%%%%%%%%%%%%%%%%%%%%%%%%%%%%%%%%%%%%%%%%%%%%%%%%%
%\subsection{Feature Suggestions}
%
%The following is a list of features which may be useful for future
%versions of this package:
%%
%\begin{itemize}
%\item
%\ldots
%\end{itemize}

%%%%%%%%%%%%%%%%%%%%%%%%%%%%%%%%%%%%%%%%%%%%%%%%%%%%%%%%%%%%%%%%%%%%%%%%%%%%%%%%
\subsection{Revision History}

%%%%%%%%%%%%%%%%%%%%%%%%%%%%%%%%%%%%%%%%
\paragraph{v2.0:} 2018/12/30

\begin{itemize}
\item
immediate forward processing
\item
added |\childdocby| mechanism
\item
manual restructured
\end{itemize}

%%%%%%%%%%%%%%%%%%%%%%%%%%%%%%%%%%%%%%%%
\paragraph{v1.6:} 2018/01/17

\begin{itemize}
\item
application for development of include files
\item
corrections to manual
\end{itemize}

%%%%%%%%%%%%%%%%%%%%%%%%%%%%%%%%%%%%%%%%
\paragraph{v1.5:} 2017/05/21

\begin{itemize}
\item
more complete structuring introduced
\item
|\childdocof| introduced
\item
|\childdoc| renamed to |\childdocmain|
\item
|\childredirect| renamed to |\childdocforward| and |\childdocforwardprefix|
and functionality expanded
\end{itemize}

%%%%%%%%%%%%%%%%%%%%%%%%%%%%%%%%%%%%%%%%
\paragraph{v1.0:} 2017/04/27

\begin{itemize}
\item
manual and install package
\item
first version published on CTAN
\end{itemize}

%%%%%%%%%%%%%%%%%%%%%%%%%%%%%%%%%%%%%%%%
\paragraph{v0.6:} 2017/04/26

\begin{itemize}
\item
redirection mechanism added
\end{itemize}

%%%%%%%%%%%%%%%%%%%%%%%%%%%%%%%%%%%%%%%%
\paragraph{v0.5:} 2017/04/26

\begin{itemize}
\item
functionality in definition file
\end{itemize}


%%%%%%%%%%%%%%%%%%%%%%%%%%%%%%%%%%%%%%%%%%%%%%%%%%%%%%%%%%%%%%%%%%%%%%%%%%%%%%%%
%%%%%%%%%%%%%%%%%%%%%%%%%%%%%%%%%%%%%%%%%%%%%%%%%%%%%%%%%%%%%%%%%%%%%%%%%%%%%%%%
%%%%%%%%%%%%%%%%%%%%%%%%%%%%%%%%%%%%%%%%%%%%%%%%%%%%%%%%%%%%%%%%%%%%%%%%%%%%%%%%
\appendix

\settowidth\MacroIndent{\rmfamily\scriptsize 000\ }

 \DocInput{childdoc.dtx}

\end{document}
%</driver>
% \fi
%
% %%%%%%%%%%%%%%%%%%%%%%%%%%%%%%%%%%%%%%%%%%%%%%%%%%%%%%%%%%%%%%%%%%%%%%%%%%%%%%
% %%%%%%%%%%%%%%%%%%%%%%%%%%%%%%%%%%%%%%%%%%%%%%%%%%%%%%%%%%%%%%%%%%%%%%%%%%%%%%
% \section{Sample}
%\iffalse
%<*samplemain>
%\fi
%
% The following presents a sample document
% with two chapters, two parts, a title page,
% a compile flag as well as three forwarding files to set the flag.
% It consists of eight |.tex| files:
% \begin{center}
% \begin{tabular}{ll}
% |cdocsamp.tex|&main file\\
% |cdocsch1.tex|&include file for chapter 1\\
% |cdocsch2.tex|&include file for chapter 2\\
% |cdocspt3.tex|&include file for part 3\\
% |cdocspt4.tex|&include file for part 4\\
% |cdocsdrf.tex|&forwarding file for main file in draft mode\\
% |cdocsfi1.tex|&forwarding file for final version of chapter 1\\
% |cdocsfi2.tex|&forwarding file for final version of chapter 2\\
% \end{tabular}
% \end{center}
% Each of the eight files can be compiled directly by the \LaTeX{} compiler.
%
% %%%%%%%%%%%%%%%%%%%%%%%%%%%%%%%%%%%%%%
% \paragraph{Main File.}
%
% The main file is called |cdocsamp.tex|.
%
% Load the \textsf{childdoc} definitions and
% declare the filename for the main document:
%    \begin{macrocode}
\input{childdoc.def}
\childdocmain{}
%    \end{macrocode}

% Optional override for |\version| flag:
%    \begin{macrocode}
%%\ifchilddoc\else\providecommand{\version}{draft}\fi
%    \end{macrocode}

% Define the default values for the |\version| flag
% (|final| for the main file and |draft| for childs):
%    \begin{macrocode}
\ifchilddoc
\providecommand{\version}{draft}
\else
\providecommand{\version}{final}
\fi
%    \end{macrocode}

% Load the standard document class:
%    \begin{macrocode}
\documentclass[12pt]{article}
%    \end{macrocode}

% Start the document body:
%    \begin{macrocode}
\begin{document}
%    \end{macrocode}

% Declare a title page.
% Print title, part of document being processed and version flag:
%    \begin{macrocode}
\addtocounter{page}{-1}
\begin{center}
{\LARGE\bfseries{}childdoc example\par}
\vspace{1cm}
\ifchilddoc
\ifchilddocmanual part\else chapter\fi:
`\childdocname' of `\childdocjob'\par
\else
main document: `\childdocjob'\par
\fi
version: \version\par
\end{center}
\newpage
%    \end{macrocode}

% Manually include selected file,
% otherwise process as usual:
%    \begin{macrocode}
\ifchilddocmanual
\section*{part `\childdocname'}
\input{\childdocname}
\else
%    \end{macrocode}

% Include the two chapters:
%    \begin{macrocode}
\include{cdocsch1}
\include{cdocsch2}
%    \end{macrocode}

% Include the two parts unless only chapters should be displayed:
%    \begin{macrocode}
\ifchilddoc\else
\section{part three}
\input{cdocspt3}
\section{part four}
\input{cdocspt4}
\fi
%    \end{macrocode}

% Process as usual until here:
%    \begin{macrocode}
\fi
%    \end{macrocode}

% End of document body:
%    \begin{macrocode}
\end{document}
%    \end{macrocode}
%\iffalse
%</samplemain>
%\fi
%
% %%%%%%%%%%%%%%%%%%%%%%%%%%%%%%%%%%%%%%
% \paragraph{Chapter Include Files.}
%
% The include files are called |cdocsch1.tex| and |cdocsch2.tex|.
%
%\iffalse
%<*samplechap1|samplechap2>
%\fi

% Optional override for |\version| flag:
%    \begin{macrocode}
%%\providecommand{\version}{final}
%    \end{macrocode}

% Include the main document:
%    \begin{macrocode}
\input{childdoc.def}
\childdocof{cdocsamp}
%    \end{macrocode}

%\iffalse
%</samplechap1|samplechap2>
%\fi
%
%\iffalse
%<*samplechap1>
%\fi
% Some text for chapter 1:
%    \begin{macrocode}
\section{one}
some text in chapter one
%    \end{macrocode}

%\iffalse
%</samplechap1>
%\fi
% Some text for chapter 2:
%\iffalse
%<*samplechap2>
%\fi
%    \begin{macrocode}
\section{two}
more text in chapter two
%    \end{macrocode}

%\iffalse
%</samplechap2>
%\fi
%
% %%%%%%%%%%%%%%%%%%%%%%%%%%%%%%%%%%%%%%
% \paragraph{Part Include Files.}
%
% The include files are called |cdocspt3.tex| and |cdocspt4.tex|.
%
%\iffalse
%<*samplepart3|samplepart4>
%\fi

% Optional override for |\version| flag:
%    \begin{macrocode}
%%\providecommand{\version}{final}
%    \end{macrocode}

% Include the main document:
%    \begin{macrocode}
\input{childdoc.def}
\childdocby{cdocsamp}
%    \end{macrocode}

%\iffalse
%</samplepart3|samplepart4>
%\fi
%
%\iffalse
%<*samplepart3>
%\fi
% Some text for part 3:
%    \begin{macrocode}
some text in part three
%    \end{macrocode}

%\iffalse
%</samplepart3>
%\fi
% Some text for part 4:
%\iffalse
%<*samplepart4>
%\fi
%    \begin{macrocode}
more text in part four
%    \end{macrocode}

%\iffalse
%</samplepart4>
%\fi
%
% %%%%%%%%%%%%%%%%%%%%%%%%%%%%%%%%%%%%%%
% \paragraph{Forwarding for a Complete Draft.}
%
% The following forwarding file |cdocsdrf.tex|
% compiles the main document in draft mode:
%\iffalse
%<*sampledraft>
%\fi
%    \begin{macrocode}
\def\version{draft}
\input{childdoc.def}
\childdocforward{cdocsamp}
%    \end{macrocode}

%\iffalse
%</sampledraft>
%\fi
%
% %%%%%%%%%%%%%%%%%%%%%%%%%%%%%%%%%%%%%%
% \paragraph{Forwarding for Final Version of the Chapters.}
%
% The following forwarding files |cdocsfn1.tex| and |cdocsfn2.tex|
% (with identical content)
% compile the final versions of the child documents
% |cdocsch1.tex| and |cdocsch2.tex|, respectively:
%\iffalse
%<*samplefinal>
%\fi
%    \begin{macrocode}
\def\version{final}
\input{childdoc.def}
\childdocforwardprefix[cdocsamp]{cdocsfn}{cdocsch}
%    \end{macrocode}

%\iffalse
%</samplefinal>
%\fi
%
% %%%%%%%%%%%%%%%%%%%%%%%%%%%%%%%%%%%%%%
% \paragraph{Command Line Processing.}
%
% The following three command lines generate the output files
% |cdocscld|, |cdocscl1| and |cdocscl2|
% which should be identical to
% |cdocsdrf|, |cdocsch1| and |cdocsfn2|, respectively:
% \begin{center}
% \begin{tabular}{l}
% |latex -jobname cdocscld \|\\
% |  "\def\version{draft}\input{childdoc.def}\childdocforward{cdocsamp}"|\\
% |latex -jobname cdocscl1 \|\\
% |  "\input{childdoc.def}\childdocforward[cdocsamp]{cdocsch1}"|\\
% |latex -jobname cdocscl2 \|\\
% |  "\def\version{final}\input{childdoc.def}\childdocforward{cdocsch2}"|
% \end{tabular}
% \end{center}
% Note that the trailing backslash on each first line
% merely continues the input to the second line
% (for convenient cut ant paste).
% Furthermore, the command |latex| can be replaced by any
% of its alternative versions such as |pdflatex|.
%
% %%%%%%%%%%%%%%%%%%%%%%%%%%%%%%%%%%%%%%%%%%%%%%%%%%%%%%%%%%%%%%%%%%%%%%%%%%%%%%
% %%%%%%%%%%%%%%%%%%%%%%%%%%%%%%%%%%%%%%%%%%%%%%%%%%%%%%%%%%%%%%%%%%%%%%%%%%%%%%
% \section{Implementation}
%\iffalse
%<*package>
%\fi
%
% This section describes the definitions file |childdoc.def|.

% The definitions cannot be loaded using |\usepackage| or |\RequirePackage|
% which has a mechanism to prevent loading a style file more than once.
% When loading the definitions by means of |\input|
% multiple instances have to be prevented manually:
%\iffalse
%This code needs to be before the `\ProvidesFile' directive
%which is defined at the beginning of this file.
%Therefore it is also placed there and commented out here.
%</package>
%<*discard>
%\fi
%    \begin{macrocode}
\ifdefined\childdocmain\endinput\fi
%    \end{macrocode}
%\iffalse
%</discard>
%<*package>
%\fi
%
% \macro{\ifchilddoc}
% \macro{\ifchilddocmanual}
% The conditional |\ifchilddoc| tells whether a
% child (true) or main (false) document is being compiled.
% The conditional |\ifchilddocmanual| tells whether
% the |\includeonly| mechanism is used (false) or
% the selection of child files must be performed manually (true).
% The definitions initialise to false:
%    \begin{macrocode}
\newif\ifchilddoc
\newif\ifchilddocmanual
%    \end{macrocode}

% \macro{\childdocname}
% \macro{\childdocjob}
% The macro |\childdocname| stores the name of the main document
% to be compiled. The macro |\childdocjob| stores the name of
% the document on which the \LaTeX{} compiler was originally invoked.
% The content of |\jobname| cannot be compared
% to filenames specified in the source due to different catcodes.
% The following code rescans |\jobname|, stores the result
% in |\childdocname| and saves a copy in |\childdocjob|:
%    \begin{macrocode}
\edef\childdocname{\scantokens\expandafter{\jobname\noexpand}}
\let\childdocjob\childdocname
%    \end{macrocode}

% \macro{\childdocdisable}
% The macro |\childdocdisable| prevents the main file
% from being processed more than once.
% At this stage, the main document command |\childdocmain|
% is assumed to be called once again where it should do nothing.
% Any subsequent call to it should prevent
% a secondary processing of the main document
% It overwrites the forwarding commands
% |\childdocof| and |\childdocforward|
% with empty macros to prevent further inclusions of the main document:
%    \begin{macrocode}
\newcommand{\childdocdisable}
{
  \renewcommand{\childdocmain}[1]{\renewcommand{\childdocmain}[1]{\endinput}}
  \renewcommand{\childdocof}[1]{}
  \renewcommand{\childdocby}[2][]{}
  \renewcommand{\childdocforward}[2][]{}
  \renewcommand{\childdocdisable}{}
}
%    \end{macrocode}

% \macro{\childdocmain}
% The macro |\childdocmain| is to be called at the top of the main file
% with nothing or the main filename (without extension) as argument.
% First, it breaks loops.
% If the argument is not empty and does not match |\childdocname|
% (which is set by the first inclusion of |childdoc.def|),
% |\ifchilddoc| is set to true, |\includeonly| is applied to the child file
% and |\jobname| is set to the main file
% (for proper handling of |.aux| files):
%    \begin{macrocode}
\newcommand{\childdocmain}[1]
{
  \childdocdisable\childdocmain{}
  \if?#1?\else
    \begingroup
      \def\childdoctmp{#1}
      \ifx\childdoctmp\childdocname
        \def\childdoctmp{}
      \else
        \def\childdoctmp
        {
          \childdoctrue
          \includeonly{\childdocname}
          \def\childdocjob{#1}
          \def\jobname{#1}
        }
      \fi
      \expandafter
    \endgroup
    \childdoctmp
  \fi
}
%    \end{macrocode}

% \macro{\childdocof}
% The command |\childdocof| redirects
% compilation to the main file |#1|.
%    \begin{macrocode}
\newcommand{\childdocof}[1]
{
  \childdocdisable
  \childdoctrue
  \includeonly{\childdocname}
  \def\jobname{#1}
  \def\childdocjob{#1}
  \input{#1}
}
%    \end{macrocode}

% \macro{\childdocby}
% The command |\childdocby| ....
%    \begin{macrocode}
\newcommand{\childdocby}[2][]
{
  \childdocdisable
  \childdoctrue
  \childdocmanualtrue
  \if?#1?\else
    \def\jobname{#2}
  \fi
  \def\childdocjob{#2}
  \input{#2}
  \endinput
}
%    \end{macrocode}

% \macro{\childdocforward}
% The command |\childdocforward| redirects
% compilation to the main file or
% (if the optional argument is given) a child file.
% Parameters are set as if the main file
% or a child file starting with |\childdocof| was compiled.
% Then compilation is handed over to the main file:
%    \begin{macrocode}
\newcommand{\childdocforward}[2][]
{
  \begingroup
    \if?#1?
      \def\childdoctmp
      {
        \def\childdocname{#2}
        \def\childdocjob{#2}
        \def\jobname{#2}
        \input{#2}
        \endinput
      }
    \else
      \def\childdoctmp
      {
        \childdocdisable
        \def\childdocname{#2}
        \childdoctrue
        \includeonly{#2}
        \def\childdocjob{#1}
        \def\jobname{#1}
        \input{#1}
        \endinput
      }
    \fi
    \expandafter
  \endgroup
  \childdoctmp
}
%    \end{macrocode}

% \macro{\childdocforwardprefix}
% The command |\childdocforwardprefix| redirects
% compilation to the main or a child file by means of a pattern.
% The prefix |#1| in the current filename is replaced by |#2|
% and the suffix of the current filename is kept
% (it is assumed that the filename does not contain the substring `|~~~|'
% which is used as a delimiter).
% Compilation is handed over to the new file by |\childdocforward|:
%    \begin{macrocode}
\newcommand{\childdocforwardprefix}[3][]
{
  \begingroup
    \def\childdocextract #2##1~~~{\def\childdoctmp{\childdocforward[#1]{#3##1}}}
    \expandafter\childdocextract\childdocname~~~
    \expandafter
  \endgroup
  \childdoctmp
}
%    \end{macrocode}

% \macro{\childdoc}
% The deprecated macro |\childdoc| is a legacy version of |\childdocmain|:
%    \begin{macrocode}
\newcommand{\childdoc}{\childdocmain}
%    \end{macrocode}

% \macro{\childdocredirect}
% The deprecated macro |\childdocredirect| is a legacy version
% of |\childdocforward| and |\childdocforwardprefix|:
%    \begin{macrocode}
\newcommand{\childdocredirect}[2][]
{
  \begingroup
    \if?#1?
      \def\childdoctmp{\childdocforward{#2}}
    \else
      \def\childdoctmp{\childdocforwardprefix{#1}{#2}}
    \fi
    \expandafter
  \endgroup
  \childdoctmp
}
%    \end{macrocode}

%\iffalse
%</package>
%\fi
%
\endinput
|\\
|\childdocforward{|\textit{main}|}|\\
\end{tabular}
\end{center}
%
or alternatively with:
%
\begin{center}
\begin{tabular}{l}
|% \iffalse
%
% childdoc.dtx Copyright (C) 2017-2018 Niklas Beisert
%
% This work may be distributed and/or modified under the
% conditions of the LaTeX Project Public License, either version 1.3
% of this license or (at your option) any later version.
% The latest version of this license is in
%   http://www.latex-project.org/lppl.txt
% and version 1.3 or later is part of all distributions of LaTeX
% version 2005/12/01 or later.
%
% This work has the LPPL maintenance status `maintained'.
%
% The Current Maintainer of this work is Niklas Beisert.
%
% This work consists of the files childdoc.dtx and childdoc.ins
% and the derived files childdoc.def and cdocsamp.tex with
% cdocsch1.tex, cdocsch2.tex, cdocsdrf.tex, cdocsfn1.tex, cdocsfn2.tex.
%
%<package>\ifdefined\childdocmain\endinput\fi
%<package>\ProvidesFile{childdoc.def}[2018/12/30 v2.0 child document driver]
%<samplemain>\ProvidesFile{cdocsamp.tex}[2018/12/30 v2.0 sample for childdoc]
%<*driver>
%\ProvidesFile{childdoc.drv}[2018/12/30 v2.0 childdoc reference manual file]
\PassOptionsToClass{10pt,a4paper}{article}
\documentclass{ltxdoc}

\usepackage[margin=35mm]{geometry}
\usepackage{hyperref}
\usepackage{hyperxmp}
\usepackage[usenames]{color}

\hypersetup{colorlinks=true}
\hypersetup{pdfstartview=FitH}
\hypersetup{pdfpagemode=UseNone}
\hypersetup{pdfsource={}}
\hypersetup{pdflang={en-UK}}
\hypersetup{pdfcopyright={Copyright 2017-2018 Niklas Beisert.
  This work may be distributed and/or modified under the
  conditions of the LaTeX Project Public License, either version 1.3
  of this license or (at your option) any later version.}}
\hypersetup{pdflicenseurl={http://www.latex-project.org/lppl.txt}}
\hypersetup{pdfcontactaddress={ETH Zurich, ITP, HIT K,
  Wolfgang-Pauli-Strasse 27}}
\hypersetup{pdfcontactpostcode={8093}}
\hypersetup{pdfcontactcity={Zurich}}
\hypersetup{pdfcontactcountry={Switzerland}}
\hypersetup{pdfcontactemail={nbeisert@itp.phys.ethz.ch}}
\hypersetup{pdfcontacturl={http://people.phys.ethz.ch/\xmptilde nbeisert/}}

\newcommand{\secref}[1]{\hyperref[#1]{section \ref*{#1}}}

\parskip1ex
\parindent0pt
\let\olditemize\itemize
\def\itemize{\olditemize\parskip0pt}

\begin{document}

\title{The \textsf{childdoc} Package}
\hypersetup{pdftitle={The childdoc Package}}
\author{Niklas Beisert\\[2ex]
  Institut f\"ur Theoretische Physik\\
  Eidgen\"ossische Technische Hochschule Z\"urich\\
  Wolfgang-Pauli-Strasse 27, 8093 Z\"urich, Switzerland\\[1ex]
  \href{mailto:nbeisert@itp.phys.ethz.ch}
  {\texttt{nbeisert@itp.phys.ethz.ch}}}
\hypersetup{pdfauthor={Niklas Beisert}}
\hypersetup{pdfsubject={Manual for the LaTeX2e Package childdoc}}
\date{30 December 2018, \textsf{v2.0}}
\maketitle

\begin{abstract}\noindent
\textsf{childdoc} is a \LaTeXe{} package
that enables the direct compilation
of document sections included by |\include|
to individual files.
\end{abstract}

\begingroup
\parskip0ex
\tableofcontents
\endgroup

%%%%%%%%%%%%%%%%%%%%%%%%%%%%%%%%%%%%%%%%%%%%%%%%%%%%%%%%%%%%%%%%%%%%%%%%%%%%%%%%
%%%%%%%%%%%%%%%%%%%%%%%%%%%%%%%%%%%%%%%%%%%%%%%%%%%%%%%%%%%%%%%%%%%%%%%%%%%%%%%%
\section{Introduction}

\LaTeX{} provides a mechanism to structure a large document (such as a book)
into a main file and several child files (containing the chapters)
using the |\include| command.
This mechanism is beneficial for documents
which span hundreds of pages in order to
make the source file(s) more manageable.
Moreover, compilation can be restricted to
selected child files by means of the |\includeonly| command.
The latter feature can be used to reduce the compilation time while editing
(this was significantly more useful in the earlier days of \LaTeX{})
or to generate a smaller document which is easier to navigate.
Another application of |\includeonly| is to generate
documents consisting of selected parts of the complete document.

However, there are a few drawbacks of the plain |\include| mechanism:
\begin{itemize}
\item
The child files cannot be compiled on their own,
they can only be compiled via the main file.
A naive editing environment
(such as a text editor with an option
to have the current file processed by \LaTeX)
may require one to switch to the main file before compiling;
attempting to compile the child file produces errors.
\item
The main file must be modified (each time)
to adjust the |\includeonly| command
to the present needs. This easily leaves the main file in a messy state.
\item
The generated document will always carry the filename
of the main document. This is inconvenient if
several child files are to be compiled and
to be kept for distribution.
\end{itemize}

The present package provides a simple interface
to make child files individually compilable by \LaTeX{}.
Compiling a child file then has the same effect as compiling
the main file with an |\includeonly| command
to select the appropriate child.
Moreover the generated document will carry the name of the child
rather than the main file.
This resolves all three above issues.

This feature is meant to make the editing of books,
thesis documents and lecture notes somewhat more convenient.
However, the package can also be used efficiently for
composing a series of documents (such as exercise sheets)
which are typically distributed individually.
It then assists the author in generating the individual documents
(potentially in different versions)
as well as a document containing the collected series.
Another application is in developing style files
or other kinds of included material
where compilation of the style file could redirect
to a sample or test file.

%%%%%%%%%%%%%%%%%%%%%%%%%%%%%%%%%%%%%%%%%%%%%%%%%%%%%%%%%%%%%%%%%%%%%%%%%%%%%%%%
%%%%%%%%%%%%%%%%%%%%%%%%%%%%%%%%%%%%%%%%%%%%%%%%%%%%%%%%%%%%%%%%%%%%%%%%%%%%%%%%
\section{Usage}

First of all, the package \textsf{childdoc} is \emph{not} a standard
\LaTeXe{} |.sty| style file! Therefore it needs to be invoked in
a non-standard way.

%%%%%%%%%%%%%%%%%%%%%%%%%%%%%%%%%%%%%%%%%%%%%%%%%%%%%%%%%%%%%%%%%%%%%%%%%%%%%%%%
\subsection{Included Files}
\label{sec:include}

%%%%%%%%%%%%%%%%%%%%%%%%%%%%%%%%%%%%%%%%
\DescribeMacro{\childdocmain}
To use the package, add the commands
\begin{center}
\begin{tabular}{l}
|\input{childdoc.def}|\\
|\childdocmain{}|\\
\end{tabular}
\end{center}
at the very top of the main \LaTeX{} file,
in particular \emph{before} the |\documentclass| statement!
The argument of |\childdocmain| should be left empty
(but it must be present).

%%%%%%%%%%%%%%%%%%%%%%%%%%%%%%%%%%%%%%%%
\DescribeMacro{\childdocof}
Furthermore, add the commands
\begin{center}
\begin{tabular}{l}
|\input{childdoc.def}|\\
|\childdocof{|\textit{main}|}|\\
\end{tabular}
\end{center}
at the top of every child file \textit{child}
which is included by |\include{|\textit{child}|}|
from within the main file
(or at least for those files to be compiled individually).
The argument \textit{main} must be the filename of the main file.

There are a couple of
considerations in setting up the main and child documents:

%%%%%%%%%%%%%%%%%%%%%%%%%%%%%%%%%%%%%%%%
\paragraph{Restrictions.}

Please note the following restrictions:
\begin{itemize}
\item
|\childdocmain| must be called with one argument \textit{main}
to ensure compatibility with earlier version of the package.
It must either be empty (|\childdocmain{}|)
or precisely match the filename of the main file in which it is specified.
See \secref{sec:detection} for further information.
\item
The filename \textit{main} must be specified without the |.tex| extension.
\item
The filename \textit{main} is case sensitive
(even in case-insensitive file systems)
due to internal string comparison.
\item
The argument \textit{main} should be fully expanded, it cannot be a macro.
\item
Subdirectories and special characters should be avoided in filenames.
\item
The command |\childdocmain{|\textit{main}|}| must be followed by a whitespace.
It should not be followed immediately by another command
or by a comment mark `|%|'.
This is because the \TeX{} parser reads the token immediately following
the argument of |\childdocmain| and puts it
at the beginning of every child section;
however, a white\-space is ignored.
\end{itemize}

%%%%%%%%%%%%%%%%%%%%%%%%%%%%%%%%%%%%%%%%
\paragraph{Content of Main File.}

It is advisable to place all content in the child files included by |\include|.
Any output contained in the main file will appear in all child documents
unless suppressed manually;
it cannot be suppressed automatically by the |\includeonly| directive
and thus should normally be avoided.
A method to include some content in the main file
by means of conditional processing is described in \secref{sec:conditional}.

%%%%%%%%%%%%%%%%%%%%%%%%%%%%%%%%%%%%%%%%
\paragraph{Page Numbering.}

When only a part of the document is compiled,
the appropriate numbering of pages
(as well as other status parameters)
is determined from the |.aux| files.
The latter contain information from previous passes.
However this information needs to propagate through
all intermediate child documents.
Therefore the page numbering in child documents may well
be inconsistent until the complete document is compiled at least once.

A useful (if unconventional) way to always ensure a consistent
page numbering is to restart the numbering in each child document
and denote the pages by `\textit{child}|.|\textit{page}'
where \textit{child} represents the chapter/section number of the child file.
This can be achieved by the command
|\numberwithin{page}{|\textit{child}|}|
of the \textsf{amsmath} package
where \textit{child} can be |chapter| or |section|
depending on the chosen structuring.
Alternatively, one can modify the macro |\thepage| appropriately
and reset the counter |page| at the start of each child file.

%%%%%%%%%%%%%%%%%%%%%%%%%%%%%%%%%%%%%%%%%%%%%%%%%%%%%%%%%%%%%%%%%%%%%%%%%%%%%%%%
\subsection{Conditional Processing}
\label{sec:conditional}

The package provides a mechanism to compile different versions
of a document. To customise the versions further some conditional processing
can come in handy to distinguish which version is being compiled.
The package provides two macros to describe the compilation context:

%%%%%%%%%%%%%%%%%%%%%%%%%%%%%%%%%%%%%%%%
\DescribeMacro{\ifchilddoc}
The conditional |\ifchilddoc| distinguishes between the compilation of
child documents and the main document:
%
\begin{center}
|\ifchilddoc |\textit{child-code}| |[|\||else |\textit{main-code}]| \||fi|
\end{center}

%%%%%%%%%%%%%%%%%%%%%%%%%%%%%%%%%%%%%%%%
\DescribeMacro{\childdocname}
\DescribeMacro{\childdocjob}
The macro |\childdocname| contains the filename (without extension)
of the main or child file being processed.
Note that |\childdocjob| will always contain the name of the main file.

%%%%%%%%%%%%%%%%%%%%%%%%%%%%%%%%%%%%%%%%
\paragraph{Title Page.}

Conditional processing can be used to include a title or banner page
in the main document when proper precautions are taken.
Importantly, the code in the main file should ensure that the page counter
(as well as other status parameters which are stored in the |.aux| files)
takes the same value after the conditional processing.
Otherwise the page numbers may take divergent values
depending on which part is compiled.

For example, a title page could be declared by:
%
\begin{center}
\begin{tabular}{l}
|\ifchilddoc\||else|\\
|\addtocounter{page}{-1}|\\
\textit{code for title page}\\
|\newpage|\\
|\||fi|
\end{tabular}
\end{center}
%
A banner page for the child documents can be generated by:
%
\begin{center}
\begin{tabular}{l}
|\ifchilddoc|\\
|\addtocounter{page}{-1}|\\
\textit{code for banner page}\\
|\newpage|\\
|\||fi|
\end{tabular}
\end{center}
%
Here one could write a message such as:
\begin{center}
|This is the part \childdocname{} of \childdocjob{}.|
\end{center}

%%%%%%%%%%%%%%%%%%%%%%%%%%%%%%%%%%%%%%%%%%%%%%%%%%%%%%%%%%%%%%%%%%%%%%%%%%%%%%%%
\subsection{Flags}
\label{sec:flags}

The package makes it easy to generate different versions
of the main or child documents.
To this end compilation flags can be defined
and assigned different default values.
They will be particularly useful in conjunction
with the forwarding mechanism described in \secref{sec:forward}.

For example, it may be useful to have a flag |\version|
which can be set to |draft| or |final|.
The document source will contain some conditional code
depending on the value of |\version|.
Suppose further, the flag should default to |final| for the main file
and to |draft| for child files
which is a natural assignment for editing the document.
This is achieved by placing the following code
in the preamble of the main document
(below the |\childdocmain| directive):
%
\begin{center}
\begin{tabular}{l}
|\ifchilddoc|\\
|\providecommand{\version}{draft}|\\
|\||else|\\
|\providecommand{\version}{final}|\\
|\||fi|
\end{tabular}
\end{center}
%
The definition by |\providecommand| makes sure
that previous definitions are not overwritten.
Further statements |\providecommand{\version}{...}|
can thus be added before the above code to override it.

For the main file, one might add a line
(between |\childdocmain| and the above block)
%
\begin{center}
|%\ifchilddoc\||else\providecommand{\version}{draft}\||fi|
\end{center}
%
which can be uncommented to produce a draft version.
Likewise one can add a line to the very top of a child file
(above the |\childdocof{|\textit{main}|}| directive)
%
\begin{center}
|%\providecommand{\version}{final}|
\end{center}
%
which can be uncommented to produce the final version of this child document.

%%%%%%%%%%%%%%%%%%%%%%%%%%%%%%%%%%%%%%%%%%%%%%%%%%%%%%%%%%%%%%%%%%%%%%%%%%%%%%%%
\subsection{Forwarding}
\label{sec:forward}

Different versions of the main or child documents
using compilation flags as described in \secref{sec:flags}
can be (permanently) stored in different files
for convenient compilation, viewing and distribution.
To this end, the package defines a command
to pass on compilation to a different file:

%%%%%%%%%%%%%%%%%%%%%%%%%%%%%%%%%%%%%%%%
\DescribeMacro{\childdocforward}
The command |\childdocforward| redirects processing to
another source file:
%
\begin{center}
\begin{tabular}{l}
|\input{childdoc.def}|\\
|\childdocforward[|\textit{main}|]{|\textit{dest}|}|\\
\end{tabular}
\end{center}
%
The argument \textit{dest} is the destination file
(without extension).
It should be the main file or one of the child files.
Note that further \textsf{childdoc} directives
such as |\childdocof| and |\childdocforward|
in the indicated file will be processed in this form.
The optional argument \textit{main}
passes on directly to the main file \textit{main}
while pretending to compile the child \textit{dest}.
This form behaves as if \textit{dest}
issues |\childdocof{|\textit{main}|}| right away,
and no further \textsf{childdoc} directives will be processed.

%%%%%%%%%%%%%%%%%%%%%%%%%%%%%%%%%%%%%%%%
\DescribeMacro{\...prefix}
In the alternative form |\childdocforwardprefix|,
%
\begin{center}
\begin{tabular}{l}
|\input{childdoc.def}|\\
|\childdocforwardprefix[|\textit{main}|]{|\textit{prefix}|}{|\textit{dest}|}|
\end{tabular}
\end{center}
%
the destination file is determined by a pattern
depending on the current file:
To make this work, the current file must be called
`{\textit{prefix}\hspace{0.2em}\textit{suffix}}'
with \textit{prefix} matching precisely the argument.
Processing is then passed on to the file
`{\textit{dest}\hspace{0.2em}\textit{suffix}}'.
Surely, the same effect is achieved by
directly specifying the
argument `{\textit{dest}\hspace{0.2em}\textit{suffix}}'
in the first form.
However, that requires to set up a different file
for each child. With the alternative form of the command
all these files can have exactly the same content
which simplifies setting them up and maintaining them.

For example, the following file |draft.tex|
with a compilation flag |\version| as described in \secref{sec:flags}
compiles the main document as a draft:
%
\begin{center}
\begin{tabular}{l}
|\def\version{draft}|\\
|\input{childdoc.def}|\\
|\childdocforward{|\textit{main}|}|
\end{tabular}
\end{center}
%
Likewise, the following files |final|\textit{nn}|.tex|
compile the final version of the child document
|child|\textit{nn}|.tex|:
%
\begin{center}
\begin{tabular}{l}
|\def\version{final}|\\
|\input{childdoc.def}|\\
|\childdocforwardprefix{final}{child}|
\end{tabular}
\end{center}
%

Note that when several versions of a main file and/or of each child file
are to be generated, it may be convenient to set up a |Makefile| or
shell script to automatise the process.

%%%%%%%%%%%%%%%%%%%%%%%%%%%%%%%%%%%%%%%%%%%%%%%%%%%%%%%%%%%%%%%%%%%%%%%%%%%%%%%%
\subsection{Command Line Processing}
\label{sec:commandline}

The effect of redirection files can also be achieved by invoking
the \LaTeX{} compiler with a more elaborate command line.
Most conveniently this should be done as part
of a shell script or a |Makefile|.

When using \textsf{childdoc} in the main file, the following
command lines effectively perform a redirection
(note that depending on the shell being used,
backslashes may have to be doubled: `|\|' $\to$ `|\\|'):
%
\begin{center}
|... -jobname "|\textit{target}|" |\\|"|[\textit{flags}]%
|\input{childdoc.def}\childdocforward[|\textit{main}|]{|\textit{dest}|}"|
\end{center}
%
Here \textit{target} is the name of the output file,
\textit{main} is the name of the main file
and \textit{dest} is the name of the main or child file to be processed
(all filenames without extensions).
The optional argument \textit{main} can be omitted
if \textit{main} matches \textit{dest}.
Optionally, compilation \textit{flags} can be defined via |\def| commands.
This command line makes the \TeX{} engine believe
it is compiling the file \textit{target}
whose content is specified as the latter parameter.
The provided code then forwards the processing to
\textit{main} or \textit{dest} as described in \secref{sec:forward}.

%%%%%%%%%%%%%%%%%%%%%%%%%%%%%%%%%%%%%%%%%%%%%%%%%%%%%%%%%%%%%%%%%%%%%%%%%%%%%%%%
\subsection{Include by Input}
\label{sec:input}

Including child documents by |\include| has some restrictions by design.
Most notably, the content of a child document always occupies
its own set of pages; pages cannot be shared between child documents.
Usually, this behaviour makes perfect sense
because each child document contain an essential part of the document.
However, in some situations it may be desirable to compose
a document from a collection of parts
without having mandatory page breaks between then.
For this case, the package
provides a mechanism to include parts
by |\input| which can also be processed individually.
However, by construction this mechanism
requires manual handling of the content to be output.

%%%%%%%%%%%%%%%%%%%%%%%%%%%%%%%%%%%%%%%%
\DescribeMacro{\ifchilddocmanual}
The main file should be prepared as usual, see \secref{sec:include}.
However, the document body must make a distinction
between processing of an individual part and of the main document, e.g.:
%
\begin{center}
\begin{tabular}{l}
|\ifchilddocmanual|\\
|\input{\childdocname}|\\
|\||else|\\
\textit{document body with }|\input{|\textit{part}|}|\\
|\||fi|
\end{tabular}
\end{center}
%
The conditional |\ifchilddocmanual| is true whenever
a part to be included by |\input| is being compiled,
and the name of the part is stored in |\childdocname|.

%%%%%%%%%%%%%%%%%%%%%%%%%%%%%%%%%%%%%%%%
\DescribeMacro{\childdocby}
Each part to be included by |\input| should start with:
%
\begin{center}
\begin{tabular}{l}
|\input{childdoc.def}|\\
|\childdocby{|\textit{main}|}|\\
\end{tabular}
\end{center}
%
The directive |\childdocby| is similar to |\childdocof|
described in \secref{sec:include},
but the subsequent selection of content must be done manually.
To that end, both |\ifchilddoc| and |\ifchilddocmanual|
will be true upon processing of a part,
and the name of the part is stored in |\childdocname|.
Note that |\jobname| will be set to the filename of the current part
so that each part receives an individual |.aux| file
that does not interfere with the |.aux| file(s) of the main document.
This behaviour can be altered by the alternative form
|\childdocby[*]{|\textit{main}|}| (with a non-empty optional argument)
which uses the |.aux| file of the main document
by setting |\jobname| to \textit{main}.

%%%%%%%%%%%%%%%%%%%%%%%%%%%%%%%%%%%%%%%%%%%%%%%%%%%%%%%%%%%%%%%%%%%%%%%%%%%%%%%%
\subsection{Driver Development}
\label{sec:driver}

The \textsf{childdoc} mechanism can also be use for the development
of definition files such as \LaTeX{} styles or classes.
This case differs from the above setup with multiple parts
included by |\include| in that no |\includeonly| should be invoked.
This can be achieved by starting the include file
(before |\ProvidesPackage|) with:
%
\begin{center}
\begin{tabular}{l}
|\input{childdoc.def}|\\
|\childdocforward{|\textit{main}|}|\\
\end{tabular}
\end{center}
%
or alternatively with:
%
\begin{center}
\begin{tabular}{l}
|\input{childdoc.def}|\\
|\childdocby{|\textit{main}|}|\\
\end{tabular}
\end{center}
%
Both forms have slightly different effects as described above.
The main file is prepared as usual, see \secref{sec:include}.

%%%%%%%%%%%%%%%%%%%%%%%%%%%%%%%%%%%%%%%%%%%%%%%%%%%%%%%%%%%%%%%%%%%%%%%%%%%%%%%%
\subsection{Legacy Detection}
\label{sec:detection}

The directive |\childdocmain| in the main file can detect
whether the complete document or merely a child is to be compiled
even without using the directive |\childdocof|.
This method is deprecated because it is less robust
and there is no compelling reason to use it;
it is merely provided for backward compatibility
and it may be removed in future versions.

If the detection mechanism is to be used,
it is mandatory to correctly specify
the filename of the main file as the argument of |\childdocmain|:
%
\begin{center}
\begin{tabular}{l}
|\input{childdoc.def}|\\
|\childdocmain{|\textit{main}|}|\\
\end{tabular}
\end{center}
%
If |\jobname| does not match the argument \textit{main} of |\childdocmain|,
it is assumed that |\jobname| points to the child file to be compiled.
When using |\childdocmain| with the main file specified as argument,
it suffices to start a child file
with just |\input{|\textit{main}|}|
without loading of the package and using |\childdocof|.
If instead all processing is done
with the appropriate \textsf{childdoc} directives,
the argument of \textit{main} of |\childdocmain| can be empty.

An alternative version of the command line processing described
in \secref{sec:commandline} using the detection mechanism reads:
%
\begin{center}
|... -jobname "|\textit{target}|" "|[\textit{flags}]%
[|\def\jobname{|\textit{dest}|}|]|\input{|\textit{main}|}"|
\end{center}

%%%%%%%%%%%%%%%%%%%%%%%%%%%%%%%%%%%%%%%%%%%%%%%%%%%%%%%%%%%%%%%%%%%%%%%%%%%%%%%%
\subsection{Manual Code}
\label{sec:manual}

In case one cannot be certain whether the definitions file |childdoc.def|
is installed on the target \TeX{} distribution
and one prefers not to ship it,
it is conceivable to paste a few relevant commands into the sources.

To that end, drop all statements |\input{childdoc.def}|
and perform the replacements as outlined below.
Instead of |\childdocmain{|\textit{main}|}| add the following code
to the top of the main file:
%
\begin{center}
\begin{tabular}{l}
|\||ifdefined\childdocname\endinput\||fi\newif\ifchilddoc|\\
|\edef\childdocname{\scantokens\expandafter{\jobname\noexpand}}|\\
|\def\childdocmain{|\textit{main}|}\||ifx\childdocmain\childdocname\||else|\\
|\childdoctrue\includeonly{\childdocname}\let\jobname\childdocmain\||fi|\\
\end{tabular}
\end{center}
%
Instead of |\childdocof{|\textit{main}|}| just include the main file
at the top of each child file:
%
\begin{center}
|\input{|\textit{main}|}|
\end{center}
%
A simple redirection |\childdocforward{|\textit{dest}|}| is achieved by:
%
\begin{center}
|\def\jobname{|\textit{dest}|}\input{\jobname}|
\end{center}
%
The redirection with prefix
|\childdocforwardprefix[|\textit{prefix}|]{|\textit{dest}|}|
is accomplished by:
%
\begin{center}
\begin{tabular}{l}
|{\edef\jobname{\scantokens\expandafter{\jobname\noexpand}}|\\
|\def\redirectjob |\textit{prefix}|#1~~~{\gdef\jobname{|\textit{dest}|#1}}|\\
|\expandafter\redirectjob\jobname~~~}\input{\jobname}|
\end{tabular}
\end{center}

In an alternative approach,
child documents can be compiled by a specific command line
without additional code or specific definitions:
%
\begin{center}
|... -jobname "|\textit{target}|" "|[\textit{flags}]%
|\includeonly{|\textit{dest}|}\input{|\textit{main}|}"|
\end{center}
%

%%%%%%%%%%%%%%%%%%%%%%%%%%%%%%%%%%%%%%%%%%%%%%%%%%%%%%%%%%%%%%%%%%%%%%%%%%%%%%%%
%%%%%%%%%%%%%%%%%%%%%%%%%%%%%%%%%%%%%%%%%%%%%%%%%%%%%%%%%%%%%%%%%%%%%%%%%%%%%%%%
\section{Information}

%%%%%%%%%%%%%%%%%%%%%%%%%%%%%%%%%%%%%%%%%%%%%%%%%%%%%%%%%%%%%%%%%%%%%%%%%%%%%%%%
\subsection{Copyright}

Copyright \copyright{} 2017--2018 Niklas Beisert

This work may be distributed and/or modified under the
conditions of the \LaTeX{} Project Public License, either version 1.3
of this license or (at your option) any later version.
The latest version of this license is in
  \url{http://www.latex-project.org/lppl.txt}
and version 1.3 or later is part of all distributions of \LaTeX{}
version 2005/12/01 or later.

This work has the LPPL maintenance status `maintained'.

The Current Maintainer of this work is Niklas Beisert.

This work consists of the files |README.txt|, |childdoc.ins| and |childdoc.dtx|
as well as the derived files |childdoc.def|, |cdocsamp.tex|
with |cdocsch1.tex|, |cdocsch2.tex|, |cdocspt3.tex|, |cdocspt4.tex|,
|cdocsdrf.tex|, |cdocsfn1.tex|, |cdocsfn2.tex|
as well as |childdoc.pdf|.

%%%%%%%%%%%%%%%%%%%%%%%%%%%%%%%%%%%%%%%%%%%%%%%%%%%%%%%%%%%%%%%%%%%%%%%%%%%%%%%%
\subsection{Files and Installation}

The package consists of the files:
%
\begin{center}
\begin{tabular}{ll}
    |README.txt|   & readme file \\
    |childdoc.ins| & installation file \\
    |childdoc.dtx| & source file \\
    |childdoc.def| & definition file \\
    |cdocsamp.tex| & sample main file \\
    |cdocsch1.tex| & sample include file \\
    |cdocsch2.tex| & sample include file \\
    |cdocspt3.tex| & sample part file \\
    |cdocspt4.tex| & sample part file \\
    |cdocsdrf.tex| & sample redirection file \\
    |cdocsfn1.tex| & sample redirection file \\
    |cdocsfn2.tex| & sample redirection file \\
    |childdoc.pdf| & manual
\end{tabular}
\end{center}
%
The distribution consists of the files
|README.txt|, |childdoc.ins| and |childdoc.dtx|.
%
\begin{itemize}
\item
Run (pdf)\LaTeX{} on |childdoc.dtx|
to compile the manual |childdoc.pdf| (this file).
\item
Run \LaTeX{} on |childdoc.ins| to create the definitions file |childdoc.def|
and the sample |cdocsamp.tex| with include files
|cdocsch1.tex|, |cdocsch2.tex|, |cdocspt3.tex|, |cdocspt4.tex|,
|cdocsdrf.tex|, |cdocsfn1.tex|, |cdocsfn2.tex|.
Then copy the file |childdoc.def| to an appropriate directory of your \LaTeX{}
distribution, e.g.\ \textit{texmf-root}|/tex/latex/childdoc|.
\end{itemize}

%%%%%%%%%%%%%%%%%%%%%%%%%%%%%%%%%%%%%%%%%%%%%%%%%%%%%%%%%%%%%%%%%%%%%%%%%%%%%%%%
\subsection{Related CTAN Packages}

There are several other packages which offer a similar functionality:
%
\begin{itemize}
\item
The packages
\href{http://ctan.org/pkg/docmute}{\textsf{docmute}},
\href{http://ctan.org/pkg/includex}{\textsf{includex}} and
\href{http://ctan.org/pkg/standalone}{\textsf{standalone}}
provide commands to include only the document body of
a child file thus allowing both files to be compiled individually.
\item
The packages \href{http://ctan.org/pkg/subdocs}{\textsf{subdocs}}
and \href{http://ctan.org/pkg/subfiles}{\textsf{subfiles}}
provide structures in which the main and child documents can be
encapsulated and allowing them to be compiled individually.
The inclusion mechanism is different from the conventional |\include|.
\item
The package \href{http://ctan.org/pkg/combine}{\textsf{combine}}
is an elaborate solution to combine several documents into one.
\end{itemize}
%
See also the CTAN topic \href{http://ctan.org/topic/subdocs}{\textsf{subdocs}}
for further related packages.
The present package differs from the above solutions in that
a document structure constructed with the conventional |\include| mechanism
just needs two extra commands at the top of every file
such that all constituent files can be compiled individually.

%%%%%%%%%%%%%%%%%%%%%%%%%%%%%%%%%%%%%%%%%%%%%%%%%%%%%%%%%%%%%%%%%%%%%%%%%%%%%%%%
%\subsection{Feature Suggestions}
%
%The following is a list of features which may be useful for future
%versions of this package:
%%
%\begin{itemize}
%\item
%\ldots
%\end{itemize}

%%%%%%%%%%%%%%%%%%%%%%%%%%%%%%%%%%%%%%%%%%%%%%%%%%%%%%%%%%%%%%%%%%%%%%%%%%%%%%%%
\subsection{Revision History}

%%%%%%%%%%%%%%%%%%%%%%%%%%%%%%%%%%%%%%%%
\paragraph{v2.0:} 2018/12/30

\begin{itemize}
\item
immediate forward processing
\item
added |\childdocby| mechanism
\item
manual restructured
\end{itemize}

%%%%%%%%%%%%%%%%%%%%%%%%%%%%%%%%%%%%%%%%
\paragraph{v1.6:} 2018/01/17

\begin{itemize}
\item
application for development of include files
\item
corrections to manual
\end{itemize}

%%%%%%%%%%%%%%%%%%%%%%%%%%%%%%%%%%%%%%%%
\paragraph{v1.5:} 2017/05/21

\begin{itemize}
\item
more complete structuring introduced
\item
|\childdocof| introduced
\item
|\childdoc| renamed to |\childdocmain|
\item
|\childredirect| renamed to |\childdocforward| and |\childdocforwardprefix|
and functionality expanded
\end{itemize}

%%%%%%%%%%%%%%%%%%%%%%%%%%%%%%%%%%%%%%%%
\paragraph{v1.0:} 2017/04/27

\begin{itemize}
\item
manual and install package
\item
first version published on CTAN
\end{itemize}

%%%%%%%%%%%%%%%%%%%%%%%%%%%%%%%%%%%%%%%%
\paragraph{v0.6:} 2017/04/26

\begin{itemize}
\item
redirection mechanism added
\end{itemize}

%%%%%%%%%%%%%%%%%%%%%%%%%%%%%%%%%%%%%%%%
\paragraph{v0.5:} 2017/04/26

\begin{itemize}
\item
functionality in definition file
\end{itemize}


%%%%%%%%%%%%%%%%%%%%%%%%%%%%%%%%%%%%%%%%%%%%%%%%%%%%%%%%%%%%%%%%%%%%%%%%%%%%%%%%
%%%%%%%%%%%%%%%%%%%%%%%%%%%%%%%%%%%%%%%%%%%%%%%%%%%%%%%%%%%%%%%%%%%%%%%%%%%%%%%%
%%%%%%%%%%%%%%%%%%%%%%%%%%%%%%%%%%%%%%%%%%%%%%%%%%%%%%%%%%%%%%%%%%%%%%%%%%%%%%%%
\appendix

\settowidth\MacroIndent{\rmfamily\scriptsize 000\ }

 \DocInput{childdoc.dtx}

\end{document}
%</driver>
% \fi
%
% %%%%%%%%%%%%%%%%%%%%%%%%%%%%%%%%%%%%%%%%%%%%%%%%%%%%%%%%%%%%%%%%%%%%%%%%%%%%%%
% %%%%%%%%%%%%%%%%%%%%%%%%%%%%%%%%%%%%%%%%%%%%%%%%%%%%%%%%%%%%%%%%%%%%%%%%%%%%%%
% \section{Sample}
%\iffalse
%<*samplemain>
%\fi
%
% The following presents a sample document
% with two chapters, two parts, a title page,
% a compile flag as well as three forwarding files to set the flag.
% It consists of eight |.tex| files:
% \begin{center}
% \begin{tabular}{ll}
% |cdocsamp.tex|&main file\\
% |cdocsch1.tex|&include file for chapter 1\\
% |cdocsch2.tex|&include file for chapter 2\\
% |cdocspt3.tex|&include file for part 3\\
% |cdocspt4.tex|&include file for part 4\\
% |cdocsdrf.tex|&forwarding file for main file in draft mode\\
% |cdocsfi1.tex|&forwarding file for final version of chapter 1\\
% |cdocsfi2.tex|&forwarding file for final version of chapter 2\\
% \end{tabular}
% \end{center}
% Each of the eight files can be compiled directly by the \LaTeX{} compiler.
%
% %%%%%%%%%%%%%%%%%%%%%%%%%%%%%%%%%%%%%%
% \paragraph{Main File.}
%
% The main file is called |cdocsamp.tex|.
%
% Load the \textsf{childdoc} definitions and
% declare the filename for the main document:
%    \begin{macrocode}
\input{childdoc.def}
\childdocmain{}
%    \end{macrocode}

% Optional override for |\version| flag:
%    \begin{macrocode}
%%\ifchilddoc\else\providecommand{\version}{draft}\fi
%    \end{macrocode}

% Define the default values for the |\version| flag
% (|final| for the main file and |draft| for childs):
%    \begin{macrocode}
\ifchilddoc
\providecommand{\version}{draft}
\else
\providecommand{\version}{final}
\fi
%    \end{macrocode}

% Load the standard document class:
%    \begin{macrocode}
\documentclass[12pt]{article}
%    \end{macrocode}

% Start the document body:
%    \begin{macrocode}
\begin{document}
%    \end{macrocode}

% Declare a title page.
% Print title, part of document being processed and version flag:
%    \begin{macrocode}
\addtocounter{page}{-1}
\begin{center}
{\LARGE\bfseries{}childdoc example\par}
\vspace{1cm}
\ifchilddoc
\ifchilddocmanual part\else chapter\fi:
`\childdocname' of `\childdocjob'\par
\else
main document: `\childdocjob'\par
\fi
version: \version\par
\end{center}
\newpage
%    \end{macrocode}

% Manually include selected file,
% otherwise process as usual:
%    \begin{macrocode}
\ifchilddocmanual
\section*{part `\childdocname'}
\input{\childdocname}
\else
%    \end{macrocode}

% Include the two chapters:
%    \begin{macrocode}
\include{cdocsch1}
\include{cdocsch2}
%    \end{macrocode}

% Include the two parts unless only chapters should be displayed:
%    \begin{macrocode}
\ifchilddoc\else
\section{part three}
\input{cdocspt3}
\section{part four}
\input{cdocspt4}
\fi
%    \end{macrocode}

% Process as usual until here:
%    \begin{macrocode}
\fi
%    \end{macrocode}

% End of document body:
%    \begin{macrocode}
\end{document}
%    \end{macrocode}
%\iffalse
%</samplemain>
%\fi
%
% %%%%%%%%%%%%%%%%%%%%%%%%%%%%%%%%%%%%%%
% \paragraph{Chapter Include Files.}
%
% The include files are called |cdocsch1.tex| and |cdocsch2.tex|.
%
%\iffalse
%<*samplechap1|samplechap2>
%\fi

% Optional override for |\version| flag:
%    \begin{macrocode}
%%\providecommand{\version}{final}
%    \end{macrocode}

% Include the main document:
%    \begin{macrocode}
\input{childdoc.def}
\childdocof{cdocsamp}
%    \end{macrocode}

%\iffalse
%</samplechap1|samplechap2>
%\fi
%
%\iffalse
%<*samplechap1>
%\fi
% Some text for chapter 1:
%    \begin{macrocode}
\section{one}
some text in chapter one
%    \end{macrocode}

%\iffalse
%</samplechap1>
%\fi
% Some text for chapter 2:
%\iffalse
%<*samplechap2>
%\fi
%    \begin{macrocode}
\section{two}
more text in chapter two
%    \end{macrocode}

%\iffalse
%</samplechap2>
%\fi
%
% %%%%%%%%%%%%%%%%%%%%%%%%%%%%%%%%%%%%%%
% \paragraph{Part Include Files.}
%
% The include files are called |cdocspt3.tex| and |cdocspt4.tex|.
%
%\iffalse
%<*samplepart3|samplepart4>
%\fi

% Optional override for |\version| flag:
%    \begin{macrocode}
%%\providecommand{\version}{final}
%    \end{macrocode}

% Include the main document:
%    \begin{macrocode}
\input{childdoc.def}
\childdocby{cdocsamp}
%    \end{macrocode}

%\iffalse
%</samplepart3|samplepart4>
%\fi
%
%\iffalse
%<*samplepart3>
%\fi
% Some text for part 3:
%    \begin{macrocode}
some text in part three
%    \end{macrocode}

%\iffalse
%</samplepart3>
%\fi
% Some text for part 4:
%\iffalse
%<*samplepart4>
%\fi
%    \begin{macrocode}
more text in part four
%    \end{macrocode}

%\iffalse
%</samplepart4>
%\fi
%
% %%%%%%%%%%%%%%%%%%%%%%%%%%%%%%%%%%%%%%
% \paragraph{Forwarding for a Complete Draft.}
%
% The following forwarding file |cdocsdrf.tex|
% compiles the main document in draft mode:
%\iffalse
%<*sampledraft>
%\fi
%    \begin{macrocode}
\def\version{draft}
\input{childdoc.def}
\childdocforward{cdocsamp}
%    \end{macrocode}

%\iffalse
%</sampledraft>
%\fi
%
% %%%%%%%%%%%%%%%%%%%%%%%%%%%%%%%%%%%%%%
% \paragraph{Forwarding for Final Version of the Chapters.}
%
% The following forwarding files |cdocsfn1.tex| and |cdocsfn2.tex|
% (with identical content)
% compile the final versions of the child documents
% |cdocsch1.tex| and |cdocsch2.tex|, respectively:
%\iffalse
%<*samplefinal>
%\fi
%    \begin{macrocode}
\def\version{final}
\input{childdoc.def}
\childdocforwardprefix[cdocsamp]{cdocsfn}{cdocsch}
%    \end{macrocode}

%\iffalse
%</samplefinal>
%\fi
%
% %%%%%%%%%%%%%%%%%%%%%%%%%%%%%%%%%%%%%%
% \paragraph{Command Line Processing.}
%
% The following three command lines generate the output files
% |cdocscld|, |cdocscl1| and |cdocscl2|
% which should be identical to
% |cdocsdrf|, |cdocsch1| and |cdocsfn2|, respectively:
% \begin{center}
% \begin{tabular}{l}
% |latex -jobname cdocscld \|\\
% |  "\def\version{draft}\input{childdoc.def}\childdocforward{cdocsamp}"|\\
% |latex -jobname cdocscl1 \|\\
% |  "\input{childdoc.def}\childdocforward[cdocsamp]{cdocsch1}"|\\
% |latex -jobname cdocscl2 \|\\
% |  "\def\version{final}\input{childdoc.def}\childdocforward{cdocsch2}"|
% \end{tabular}
% \end{center}
% Note that the trailing backslash on each first line
% merely continues the input to the second line
% (for convenient cut ant paste).
% Furthermore, the command |latex| can be replaced by any
% of its alternative versions such as |pdflatex|.
%
% %%%%%%%%%%%%%%%%%%%%%%%%%%%%%%%%%%%%%%%%%%%%%%%%%%%%%%%%%%%%%%%%%%%%%%%%%%%%%%
% %%%%%%%%%%%%%%%%%%%%%%%%%%%%%%%%%%%%%%%%%%%%%%%%%%%%%%%%%%%%%%%%%%%%%%%%%%%%%%
% \section{Implementation}
%\iffalse
%<*package>
%\fi
%
% This section describes the definitions file |childdoc.def|.

% The definitions cannot be loaded using |\usepackage| or |\RequirePackage|
% which has a mechanism to prevent loading a style file more than once.
% When loading the definitions by means of |\input|
% multiple instances have to be prevented manually:
%\iffalse
%This code needs to be before the `\ProvidesFile' directive
%which is defined at the beginning of this file.
%Therefore it is also placed there and commented out here.
%</package>
%<*discard>
%\fi
%    \begin{macrocode}
\ifdefined\childdocmain\endinput\fi
%    \end{macrocode}
%\iffalse
%</discard>
%<*package>
%\fi
%
% \macro{\ifchilddoc}
% \macro{\ifchilddocmanual}
% The conditional |\ifchilddoc| tells whether a
% child (true) or main (false) document is being compiled.
% The conditional |\ifchilddocmanual| tells whether
% the |\includeonly| mechanism is used (false) or
% the selection of child files must be performed manually (true).
% The definitions initialise to false:
%    \begin{macrocode}
\newif\ifchilddoc
\newif\ifchilddocmanual
%    \end{macrocode}

% \macro{\childdocname}
% \macro{\childdocjob}
% The macro |\childdocname| stores the name of the main document
% to be compiled. The macro |\childdocjob| stores the name of
% the document on which the \LaTeX{} compiler was originally invoked.
% The content of |\jobname| cannot be compared
% to filenames specified in the source due to different catcodes.
% The following code rescans |\jobname|, stores the result
% in |\childdocname| and saves a copy in |\childdocjob|:
%    \begin{macrocode}
\edef\childdocname{\scantokens\expandafter{\jobname\noexpand}}
\let\childdocjob\childdocname
%    \end{macrocode}

% \macro{\childdocdisable}
% The macro |\childdocdisable| prevents the main file
% from being processed more than once.
% At this stage, the main document command |\childdocmain|
% is assumed to be called once again where it should do nothing.
% Any subsequent call to it should prevent
% a secondary processing of the main document
% It overwrites the forwarding commands
% |\childdocof| and |\childdocforward|
% with empty macros to prevent further inclusions of the main document:
%    \begin{macrocode}
\newcommand{\childdocdisable}
{
  \renewcommand{\childdocmain}[1]{\renewcommand{\childdocmain}[1]{\endinput}}
  \renewcommand{\childdocof}[1]{}
  \renewcommand{\childdocby}[2][]{}
  \renewcommand{\childdocforward}[2][]{}
  \renewcommand{\childdocdisable}{}
}
%    \end{macrocode}

% \macro{\childdocmain}
% The macro |\childdocmain| is to be called at the top of the main file
% with nothing or the main filename (without extension) as argument.
% First, it breaks loops.
% If the argument is not empty and does not match |\childdocname|
% (which is set by the first inclusion of |childdoc.def|),
% |\ifchilddoc| is set to true, |\includeonly| is applied to the child file
% and |\jobname| is set to the main file
% (for proper handling of |.aux| files):
%    \begin{macrocode}
\newcommand{\childdocmain}[1]
{
  \childdocdisable\childdocmain{}
  \if?#1?\else
    \begingroup
      \def\childdoctmp{#1}
      \ifx\childdoctmp\childdocname
        \def\childdoctmp{}
      \else
        \def\childdoctmp
        {
          \childdoctrue
          \includeonly{\childdocname}
          \def\childdocjob{#1}
          \def\jobname{#1}
        }
      \fi
      \expandafter
    \endgroup
    \childdoctmp
  \fi
}
%    \end{macrocode}

% \macro{\childdocof}
% The command |\childdocof| redirects
% compilation to the main file |#1|.
%    \begin{macrocode}
\newcommand{\childdocof}[1]
{
  \childdocdisable
  \childdoctrue
  \includeonly{\childdocname}
  \def\jobname{#1}
  \def\childdocjob{#1}
  \input{#1}
}
%    \end{macrocode}

% \macro{\childdocby}
% The command |\childdocby| ....
%    \begin{macrocode}
\newcommand{\childdocby}[2][]
{
  \childdocdisable
  \childdoctrue
  \childdocmanualtrue
  \if?#1?\else
    \def\jobname{#2}
  \fi
  \def\childdocjob{#2}
  \input{#2}
  \endinput
}
%    \end{macrocode}

% \macro{\childdocforward}
% The command |\childdocforward| redirects
% compilation to the main file or
% (if the optional argument is given) a child file.
% Parameters are set as if the main file
% or a child file starting with |\childdocof| was compiled.
% Then compilation is handed over to the main file:
%    \begin{macrocode}
\newcommand{\childdocforward}[2][]
{
  \begingroup
    \if?#1?
      \def\childdoctmp
      {
        \def\childdocname{#2}
        \def\childdocjob{#2}
        \def\jobname{#2}
        \input{#2}
        \endinput
      }
    \else
      \def\childdoctmp
      {
        \childdocdisable
        \def\childdocname{#2}
        \childdoctrue
        \includeonly{#2}
        \def\childdocjob{#1}
        \def\jobname{#1}
        \input{#1}
        \endinput
      }
    \fi
    \expandafter
  \endgroup
  \childdoctmp
}
%    \end{macrocode}

% \macro{\childdocforwardprefix}
% The command |\childdocforwardprefix| redirects
% compilation to the main or a child file by means of a pattern.
% The prefix |#1| in the current filename is replaced by |#2|
% and the suffix of the current filename is kept
% (it is assumed that the filename does not contain the substring `|~~~|'
% which is used as a delimiter).
% Compilation is handed over to the new file by |\childdocforward|:
%    \begin{macrocode}
\newcommand{\childdocforwardprefix}[3][]
{
  \begingroup
    \def\childdocextract #2##1~~~{\def\childdoctmp{\childdocforward[#1]{#3##1}}}
    \expandafter\childdocextract\childdocname~~~
    \expandafter
  \endgroup
  \childdoctmp
}
%    \end{macrocode}

% \macro{\childdoc}
% The deprecated macro |\childdoc| is a legacy version of |\childdocmain|:
%    \begin{macrocode}
\newcommand{\childdoc}{\childdocmain}
%    \end{macrocode}

% \macro{\childdocredirect}
% The deprecated macro |\childdocredirect| is a legacy version
% of |\childdocforward| and |\childdocforwardprefix|:
%    \begin{macrocode}
\newcommand{\childdocredirect}[2][]
{
  \begingroup
    \if?#1?
      \def\childdoctmp{\childdocforward{#2}}
    \else
      \def\childdoctmp{\childdocforwardprefix{#1}{#2}}
    \fi
    \expandafter
  \endgroup
  \childdoctmp
}
%    \end{macrocode}

%\iffalse
%</package>
%\fi
%
\endinput
|\\
|\childdocby{|\textit{main}|}|\\
\end{tabular}
\end{center}
%
Both forms have slightly different effects as described above.
The main file is prepared as usual, see \secref{sec:include}.

%%%%%%%%%%%%%%%%%%%%%%%%%%%%%%%%%%%%%%%%%%%%%%%%%%%%%%%%%%%%%%%%%%%%%%%%%%%%%%%%
\subsection{Legacy Detection}
\label{sec:detection}

The directive |\childdocmain| in the main file can detect
whether the complete document or merely a child is to be compiled
even without using the directive |\childdocof|.
This method is deprecated because it is less robust
and there is no compelling reason to use it;
it is merely provided for backward compatibility
and it may be removed in future versions.

If the detection mechanism is to be used,
it is mandatory to correctly specify
the filename of the main file as the argument of |\childdocmain|:
%
\begin{center}
\begin{tabular}{l}
|% \iffalse
%
% childdoc.dtx Copyright (C) 2017-2018 Niklas Beisert
%
% This work may be distributed and/or modified under the
% conditions of the LaTeX Project Public License, either version 1.3
% of this license or (at your option) any later version.
% The latest version of this license is in
%   http://www.latex-project.org/lppl.txt
% and version 1.3 or later is part of all distributions of LaTeX
% version 2005/12/01 or later.
%
% This work has the LPPL maintenance status `maintained'.
%
% The Current Maintainer of this work is Niklas Beisert.
%
% This work consists of the files childdoc.dtx and childdoc.ins
% and the derived files childdoc.def and cdocsamp.tex with
% cdocsch1.tex, cdocsch2.tex, cdocsdrf.tex, cdocsfn1.tex, cdocsfn2.tex.
%
%<package>\ifdefined\childdocmain\endinput\fi
%<package>\ProvidesFile{childdoc.def}[2018/12/30 v2.0 child document driver]
%<samplemain>\ProvidesFile{cdocsamp.tex}[2018/12/30 v2.0 sample for childdoc]
%<*driver>
%\ProvidesFile{childdoc.drv}[2018/12/30 v2.0 childdoc reference manual file]
\PassOptionsToClass{10pt,a4paper}{article}
\documentclass{ltxdoc}

\usepackage[margin=35mm]{geometry}
\usepackage{hyperref}
\usepackage{hyperxmp}
\usepackage[usenames]{color}

\hypersetup{colorlinks=true}
\hypersetup{pdfstartview=FitH}
\hypersetup{pdfpagemode=UseNone}
\hypersetup{pdfsource={}}
\hypersetup{pdflang={en-UK}}
\hypersetup{pdfcopyright={Copyright 2017-2018 Niklas Beisert.
  This work may be distributed and/or modified under the
  conditions of the LaTeX Project Public License, either version 1.3
  of this license or (at your option) any later version.}}
\hypersetup{pdflicenseurl={http://www.latex-project.org/lppl.txt}}
\hypersetup{pdfcontactaddress={ETH Zurich, ITP, HIT K,
  Wolfgang-Pauli-Strasse 27}}
\hypersetup{pdfcontactpostcode={8093}}
\hypersetup{pdfcontactcity={Zurich}}
\hypersetup{pdfcontactcountry={Switzerland}}
\hypersetup{pdfcontactemail={nbeisert@itp.phys.ethz.ch}}
\hypersetup{pdfcontacturl={http://people.phys.ethz.ch/\xmptilde nbeisert/}}

\newcommand{\secref}[1]{\hyperref[#1]{section \ref*{#1}}}

\parskip1ex
\parindent0pt
\let\olditemize\itemize
\def\itemize{\olditemize\parskip0pt}

\begin{document}

\title{The \textsf{childdoc} Package}
\hypersetup{pdftitle={The childdoc Package}}
\author{Niklas Beisert\\[2ex]
  Institut f\"ur Theoretische Physik\\
  Eidgen\"ossische Technische Hochschule Z\"urich\\
  Wolfgang-Pauli-Strasse 27, 8093 Z\"urich, Switzerland\\[1ex]
  \href{mailto:nbeisert@itp.phys.ethz.ch}
  {\texttt{nbeisert@itp.phys.ethz.ch}}}
\hypersetup{pdfauthor={Niklas Beisert}}
\hypersetup{pdfsubject={Manual for the LaTeX2e Package childdoc}}
\date{30 December 2018, \textsf{v2.0}}
\maketitle

\begin{abstract}\noindent
\textsf{childdoc} is a \LaTeXe{} package
that enables the direct compilation
of document sections included by |\include|
to individual files.
\end{abstract}

\begingroup
\parskip0ex
\tableofcontents
\endgroup

%%%%%%%%%%%%%%%%%%%%%%%%%%%%%%%%%%%%%%%%%%%%%%%%%%%%%%%%%%%%%%%%%%%%%%%%%%%%%%%%
%%%%%%%%%%%%%%%%%%%%%%%%%%%%%%%%%%%%%%%%%%%%%%%%%%%%%%%%%%%%%%%%%%%%%%%%%%%%%%%%
\section{Introduction}

\LaTeX{} provides a mechanism to structure a large document (such as a book)
into a main file and several child files (containing the chapters)
using the |\include| command.
This mechanism is beneficial for documents
which span hundreds of pages in order to
make the source file(s) more manageable.
Moreover, compilation can be restricted to
selected child files by means of the |\includeonly| command.
The latter feature can be used to reduce the compilation time while editing
(this was significantly more useful in the earlier days of \LaTeX{})
or to generate a smaller document which is easier to navigate.
Another application of |\includeonly| is to generate
documents consisting of selected parts of the complete document.

However, there are a few drawbacks of the plain |\include| mechanism:
\begin{itemize}
\item
The child files cannot be compiled on their own,
they can only be compiled via the main file.
A naive editing environment
(such as a text editor with an option
to have the current file processed by \LaTeX)
may require one to switch to the main file before compiling;
attempting to compile the child file produces errors.
\item
The main file must be modified (each time)
to adjust the |\includeonly| command
to the present needs. This easily leaves the main file in a messy state.
\item
The generated document will always carry the filename
of the main document. This is inconvenient if
several child files are to be compiled and
to be kept for distribution.
\end{itemize}

The present package provides a simple interface
to make child files individually compilable by \LaTeX{}.
Compiling a child file then has the same effect as compiling
the main file with an |\includeonly| command
to select the appropriate child.
Moreover the generated document will carry the name of the child
rather than the main file.
This resolves all three above issues.

This feature is meant to make the editing of books,
thesis documents and lecture notes somewhat more convenient.
However, the package can also be used efficiently for
composing a series of documents (such as exercise sheets)
which are typically distributed individually.
It then assists the author in generating the individual documents
(potentially in different versions)
as well as a document containing the collected series.
Another application is in developing style files
or other kinds of included material
where compilation of the style file could redirect
to a sample or test file.

%%%%%%%%%%%%%%%%%%%%%%%%%%%%%%%%%%%%%%%%%%%%%%%%%%%%%%%%%%%%%%%%%%%%%%%%%%%%%%%%
%%%%%%%%%%%%%%%%%%%%%%%%%%%%%%%%%%%%%%%%%%%%%%%%%%%%%%%%%%%%%%%%%%%%%%%%%%%%%%%%
\section{Usage}

First of all, the package \textsf{childdoc} is \emph{not} a standard
\LaTeXe{} |.sty| style file! Therefore it needs to be invoked in
a non-standard way.

%%%%%%%%%%%%%%%%%%%%%%%%%%%%%%%%%%%%%%%%%%%%%%%%%%%%%%%%%%%%%%%%%%%%%%%%%%%%%%%%
\subsection{Included Files}
\label{sec:include}

%%%%%%%%%%%%%%%%%%%%%%%%%%%%%%%%%%%%%%%%
\DescribeMacro{\childdocmain}
To use the package, add the commands
\begin{center}
\begin{tabular}{l}
|\input{childdoc.def}|\\
|\childdocmain{}|\\
\end{tabular}
\end{center}
at the very top of the main \LaTeX{} file,
in particular \emph{before} the |\documentclass| statement!
The argument of |\childdocmain| should be left empty
(but it must be present).

%%%%%%%%%%%%%%%%%%%%%%%%%%%%%%%%%%%%%%%%
\DescribeMacro{\childdocof}
Furthermore, add the commands
\begin{center}
\begin{tabular}{l}
|\input{childdoc.def}|\\
|\childdocof{|\textit{main}|}|\\
\end{tabular}
\end{center}
at the top of every child file \textit{child}
which is included by |\include{|\textit{child}|}|
from within the main file
(or at least for those files to be compiled individually).
The argument \textit{main} must be the filename of the main file.

There are a couple of
considerations in setting up the main and child documents:

%%%%%%%%%%%%%%%%%%%%%%%%%%%%%%%%%%%%%%%%
\paragraph{Restrictions.}

Please note the following restrictions:
\begin{itemize}
\item
|\childdocmain| must be called with one argument \textit{main}
to ensure compatibility with earlier version of the package.
It must either be empty (|\childdocmain{}|)
or precisely match the filename of the main file in which it is specified.
See \secref{sec:detection} for further information.
\item
The filename \textit{main} must be specified without the |.tex| extension.
\item
The filename \textit{main} is case sensitive
(even in case-insensitive file systems)
due to internal string comparison.
\item
The argument \textit{main} should be fully expanded, it cannot be a macro.
\item
Subdirectories and special characters should be avoided in filenames.
\item
The command |\childdocmain{|\textit{main}|}| must be followed by a whitespace.
It should not be followed immediately by another command
or by a comment mark `|%|'.
This is because the \TeX{} parser reads the token immediately following
the argument of |\childdocmain| and puts it
at the beginning of every child section;
however, a white\-space is ignored.
\end{itemize}

%%%%%%%%%%%%%%%%%%%%%%%%%%%%%%%%%%%%%%%%
\paragraph{Content of Main File.}

It is advisable to place all content in the child files included by |\include|.
Any output contained in the main file will appear in all child documents
unless suppressed manually;
it cannot be suppressed automatically by the |\includeonly| directive
and thus should normally be avoided.
A method to include some content in the main file
by means of conditional processing is described in \secref{sec:conditional}.

%%%%%%%%%%%%%%%%%%%%%%%%%%%%%%%%%%%%%%%%
\paragraph{Page Numbering.}

When only a part of the document is compiled,
the appropriate numbering of pages
(as well as other status parameters)
is determined from the |.aux| files.
The latter contain information from previous passes.
However this information needs to propagate through
all intermediate child documents.
Therefore the page numbering in child documents may well
be inconsistent until the complete document is compiled at least once.

A useful (if unconventional) way to always ensure a consistent
page numbering is to restart the numbering in each child document
and denote the pages by `\textit{child}|.|\textit{page}'
where \textit{child} represents the chapter/section number of the child file.
This can be achieved by the command
|\numberwithin{page}{|\textit{child}|}|
of the \textsf{amsmath} package
where \textit{child} can be |chapter| or |section|
depending on the chosen structuring.
Alternatively, one can modify the macro |\thepage| appropriately
and reset the counter |page| at the start of each child file.

%%%%%%%%%%%%%%%%%%%%%%%%%%%%%%%%%%%%%%%%%%%%%%%%%%%%%%%%%%%%%%%%%%%%%%%%%%%%%%%%
\subsection{Conditional Processing}
\label{sec:conditional}

The package provides a mechanism to compile different versions
of a document. To customise the versions further some conditional processing
can come in handy to distinguish which version is being compiled.
The package provides two macros to describe the compilation context:

%%%%%%%%%%%%%%%%%%%%%%%%%%%%%%%%%%%%%%%%
\DescribeMacro{\ifchilddoc}
The conditional |\ifchilddoc| distinguishes between the compilation of
child documents and the main document:
%
\begin{center}
|\ifchilddoc |\textit{child-code}| |[|\||else |\textit{main-code}]| \||fi|
\end{center}

%%%%%%%%%%%%%%%%%%%%%%%%%%%%%%%%%%%%%%%%
\DescribeMacro{\childdocname}
\DescribeMacro{\childdocjob}
The macro |\childdocname| contains the filename (without extension)
of the main or child file being processed.
Note that |\childdocjob| will always contain the name of the main file.

%%%%%%%%%%%%%%%%%%%%%%%%%%%%%%%%%%%%%%%%
\paragraph{Title Page.}

Conditional processing can be used to include a title or banner page
in the main document when proper precautions are taken.
Importantly, the code in the main file should ensure that the page counter
(as well as other status parameters which are stored in the |.aux| files)
takes the same value after the conditional processing.
Otherwise the page numbers may take divergent values
depending on which part is compiled.

For example, a title page could be declared by:
%
\begin{center}
\begin{tabular}{l}
|\ifchilddoc\||else|\\
|\addtocounter{page}{-1}|\\
\textit{code for title page}\\
|\newpage|\\
|\||fi|
\end{tabular}
\end{center}
%
A banner page for the child documents can be generated by:
%
\begin{center}
\begin{tabular}{l}
|\ifchilddoc|\\
|\addtocounter{page}{-1}|\\
\textit{code for banner page}\\
|\newpage|\\
|\||fi|
\end{tabular}
\end{center}
%
Here one could write a message such as:
\begin{center}
|This is the part \childdocname{} of \childdocjob{}.|
\end{center}

%%%%%%%%%%%%%%%%%%%%%%%%%%%%%%%%%%%%%%%%%%%%%%%%%%%%%%%%%%%%%%%%%%%%%%%%%%%%%%%%
\subsection{Flags}
\label{sec:flags}

The package makes it easy to generate different versions
of the main or child documents.
To this end compilation flags can be defined
and assigned different default values.
They will be particularly useful in conjunction
with the forwarding mechanism described in \secref{sec:forward}.

For example, it may be useful to have a flag |\version|
which can be set to |draft| or |final|.
The document source will contain some conditional code
depending on the value of |\version|.
Suppose further, the flag should default to |final| for the main file
and to |draft| for child files
which is a natural assignment for editing the document.
This is achieved by placing the following code
in the preamble of the main document
(below the |\childdocmain| directive):
%
\begin{center}
\begin{tabular}{l}
|\ifchilddoc|\\
|\providecommand{\version}{draft}|\\
|\||else|\\
|\providecommand{\version}{final}|\\
|\||fi|
\end{tabular}
\end{center}
%
The definition by |\providecommand| makes sure
that previous definitions are not overwritten.
Further statements |\providecommand{\version}{...}|
can thus be added before the above code to override it.

For the main file, one might add a line
(between |\childdocmain| and the above block)
%
\begin{center}
|%\ifchilddoc\||else\providecommand{\version}{draft}\||fi|
\end{center}
%
which can be uncommented to produce a draft version.
Likewise one can add a line to the very top of a child file
(above the |\childdocof{|\textit{main}|}| directive)
%
\begin{center}
|%\providecommand{\version}{final}|
\end{center}
%
which can be uncommented to produce the final version of this child document.

%%%%%%%%%%%%%%%%%%%%%%%%%%%%%%%%%%%%%%%%%%%%%%%%%%%%%%%%%%%%%%%%%%%%%%%%%%%%%%%%
\subsection{Forwarding}
\label{sec:forward}

Different versions of the main or child documents
using compilation flags as described in \secref{sec:flags}
can be (permanently) stored in different files
for convenient compilation, viewing and distribution.
To this end, the package defines a command
to pass on compilation to a different file:

%%%%%%%%%%%%%%%%%%%%%%%%%%%%%%%%%%%%%%%%
\DescribeMacro{\childdocforward}
The command |\childdocforward| redirects processing to
another source file:
%
\begin{center}
\begin{tabular}{l}
|\input{childdoc.def}|\\
|\childdocforward[|\textit{main}|]{|\textit{dest}|}|\\
\end{tabular}
\end{center}
%
The argument \textit{dest} is the destination file
(without extension).
It should be the main file or one of the child files.
Note that further \textsf{childdoc} directives
such as |\childdocof| and |\childdocforward|
in the indicated file will be processed in this form.
The optional argument \textit{main}
passes on directly to the main file \textit{main}
while pretending to compile the child \textit{dest}.
This form behaves as if \textit{dest}
issues |\childdocof{|\textit{main}|}| right away,
and no further \textsf{childdoc} directives will be processed.

%%%%%%%%%%%%%%%%%%%%%%%%%%%%%%%%%%%%%%%%
\DescribeMacro{\...prefix}
In the alternative form |\childdocforwardprefix|,
%
\begin{center}
\begin{tabular}{l}
|\input{childdoc.def}|\\
|\childdocforwardprefix[|\textit{main}|]{|\textit{prefix}|}{|\textit{dest}|}|
\end{tabular}
\end{center}
%
the destination file is determined by a pattern
depending on the current file:
To make this work, the current file must be called
`{\textit{prefix}\hspace{0.2em}\textit{suffix}}'
with \textit{prefix} matching precisely the argument.
Processing is then passed on to the file
`{\textit{dest}\hspace{0.2em}\textit{suffix}}'.
Surely, the same effect is achieved by
directly specifying the
argument `{\textit{dest}\hspace{0.2em}\textit{suffix}}'
in the first form.
However, that requires to set up a different file
for each child. With the alternative form of the command
all these files can have exactly the same content
which simplifies setting them up and maintaining them.

For example, the following file |draft.tex|
with a compilation flag |\version| as described in \secref{sec:flags}
compiles the main document as a draft:
%
\begin{center}
\begin{tabular}{l}
|\def\version{draft}|\\
|\input{childdoc.def}|\\
|\childdocforward{|\textit{main}|}|
\end{tabular}
\end{center}
%
Likewise, the following files |final|\textit{nn}|.tex|
compile the final version of the child document
|child|\textit{nn}|.tex|:
%
\begin{center}
\begin{tabular}{l}
|\def\version{final}|\\
|\input{childdoc.def}|\\
|\childdocforwardprefix{final}{child}|
\end{tabular}
\end{center}
%

Note that when several versions of a main file and/or of each child file
are to be generated, it may be convenient to set up a |Makefile| or
shell script to automatise the process.

%%%%%%%%%%%%%%%%%%%%%%%%%%%%%%%%%%%%%%%%%%%%%%%%%%%%%%%%%%%%%%%%%%%%%%%%%%%%%%%%
\subsection{Command Line Processing}
\label{sec:commandline}

The effect of redirection files can also be achieved by invoking
the \LaTeX{} compiler with a more elaborate command line.
Most conveniently this should be done as part
of a shell script or a |Makefile|.

When using \textsf{childdoc} in the main file, the following
command lines effectively perform a redirection
(note that depending on the shell being used,
backslashes may have to be doubled: `|\|' $\to$ `|\\|'):
%
\begin{center}
|... -jobname "|\textit{target}|" |\\|"|[\textit{flags}]%
|\input{childdoc.def}\childdocforward[|\textit{main}|]{|\textit{dest}|}"|
\end{center}
%
Here \textit{target} is the name of the output file,
\textit{main} is the name of the main file
and \textit{dest} is the name of the main or child file to be processed
(all filenames without extensions).
The optional argument \textit{main} can be omitted
if \textit{main} matches \textit{dest}.
Optionally, compilation \textit{flags} can be defined via |\def| commands.
This command line makes the \TeX{} engine believe
it is compiling the file \textit{target}
whose content is specified as the latter parameter.
The provided code then forwards the processing to
\textit{main} or \textit{dest} as described in \secref{sec:forward}.

%%%%%%%%%%%%%%%%%%%%%%%%%%%%%%%%%%%%%%%%%%%%%%%%%%%%%%%%%%%%%%%%%%%%%%%%%%%%%%%%
\subsection{Include by Input}
\label{sec:input}

Including child documents by |\include| has some restrictions by design.
Most notably, the content of a child document always occupies
its own set of pages; pages cannot be shared between child documents.
Usually, this behaviour makes perfect sense
because each child document contain an essential part of the document.
However, in some situations it may be desirable to compose
a document from a collection of parts
without having mandatory page breaks between then.
For this case, the package
provides a mechanism to include parts
by |\input| which can also be processed individually.
However, by construction this mechanism
requires manual handling of the content to be output.

%%%%%%%%%%%%%%%%%%%%%%%%%%%%%%%%%%%%%%%%
\DescribeMacro{\ifchilddocmanual}
The main file should be prepared as usual, see \secref{sec:include}.
However, the document body must make a distinction
between processing of an individual part and of the main document, e.g.:
%
\begin{center}
\begin{tabular}{l}
|\ifchilddocmanual|\\
|\input{\childdocname}|\\
|\||else|\\
\textit{document body with }|\input{|\textit{part}|}|\\
|\||fi|
\end{tabular}
\end{center}
%
The conditional |\ifchilddocmanual| is true whenever
a part to be included by |\input| is being compiled,
and the name of the part is stored in |\childdocname|.

%%%%%%%%%%%%%%%%%%%%%%%%%%%%%%%%%%%%%%%%
\DescribeMacro{\childdocby}
Each part to be included by |\input| should start with:
%
\begin{center}
\begin{tabular}{l}
|\input{childdoc.def}|\\
|\childdocby{|\textit{main}|}|\\
\end{tabular}
\end{center}
%
The directive |\childdocby| is similar to |\childdocof|
described in \secref{sec:include},
but the subsequent selection of content must be done manually.
To that end, both |\ifchilddoc| and |\ifchilddocmanual|
will be true upon processing of a part,
and the name of the part is stored in |\childdocname|.
Note that |\jobname| will be set to the filename of the current part
so that each part receives an individual |.aux| file
that does not interfere with the |.aux| file(s) of the main document.
This behaviour can be altered by the alternative form
|\childdocby[*]{|\textit{main}|}| (with a non-empty optional argument)
which uses the |.aux| file of the main document
by setting |\jobname| to \textit{main}.

%%%%%%%%%%%%%%%%%%%%%%%%%%%%%%%%%%%%%%%%%%%%%%%%%%%%%%%%%%%%%%%%%%%%%%%%%%%%%%%%
\subsection{Driver Development}
\label{sec:driver}

The \textsf{childdoc} mechanism can also be use for the development
of definition files such as \LaTeX{} styles or classes.
This case differs from the above setup with multiple parts
included by |\include| in that no |\includeonly| should be invoked.
This can be achieved by starting the include file
(before |\ProvidesPackage|) with:
%
\begin{center}
\begin{tabular}{l}
|\input{childdoc.def}|\\
|\childdocforward{|\textit{main}|}|\\
\end{tabular}
\end{center}
%
or alternatively with:
%
\begin{center}
\begin{tabular}{l}
|\input{childdoc.def}|\\
|\childdocby{|\textit{main}|}|\\
\end{tabular}
\end{center}
%
Both forms have slightly different effects as described above.
The main file is prepared as usual, see \secref{sec:include}.

%%%%%%%%%%%%%%%%%%%%%%%%%%%%%%%%%%%%%%%%%%%%%%%%%%%%%%%%%%%%%%%%%%%%%%%%%%%%%%%%
\subsection{Legacy Detection}
\label{sec:detection}

The directive |\childdocmain| in the main file can detect
whether the complete document or merely a child is to be compiled
even without using the directive |\childdocof|.
This method is deprecated because it is less robust
and there is no compelling reason to use it;
it is merely provided for backward compatibility
and it may be removed in future versions.

If the detection mechanism is to be used,
it is mandatory to correctly specify
the filename of the main file as the argument of |\childdocmain|:
%
\begin{center}
\begin{tabular}{l}
|\input{childdoc.def}|\\
|\childdocmain{|\textit{main}|}|\\
\end{tabular}
\end{center}
%
If |\jobname| does not match the argument \textit{main} of |\childdocmain|,
it is assumed that |\jobname| points to the child file to be compiled.
When using |\childdocmain| with the main file specified as argument,
it suffices to start a child file
with just |\input{|\textit{main}|}|
without loading of the package and using |\childdocof|.
If instead all processing is done
with the appropriate \textsf{childdoc} directives,
the argument of \textit{main} of |\childdocmain| can be empty.

An alternative version of the command line processing described
in \secref{sec:commandline} using the detection mechanism reads:
%
\begin{center}
|... -jobname "|\textit{target}|" "|[\textit{flags}]%
[|\def\jobname{|\textit{dest}|}|]|\input{|\textit{main}|}"|
\end{center}

%%%%%%%%%%%%%%%%%%%%%%%%%%%%%%%%%%%%%%%%%%%%%%%%%%%%%%%%%%%%%%%%%%%%%%%%%%%%%%%%
\subsection{Manual Code}
\label{sec:manual}

In case one cannot be certain whether the definitions file |childdoc.def|
is installed on the target \TeX{} distribution
and one prefers not to ship it,
it is conceivable to paste a few relevant commands into the sources.

To that end, drop all statements |\input{childdoc.def}|
and perform the replacements as outlined below.
Instead of |\childdocmain{|\textit{main}|}| add the following code
to the top of the main file:
%
\begin{center}
\begin{tabular}{l}
|\||ifdefined\childdocname\endinput\||fi\newif\ifchilddoc|\\
|\edef\childdocname{\scantokens\expandafter{\jobname\noexpand}}|\\
|\def\childdocmain{|\textit{main}|}\||ifx\childdocmain\childdocname\||else|\\
|\childdoctrue\includeonly{\childdocname}\let\jobname\childdocmain\||fi|\\
\end{tabular}
\end{center}
%
Instead of |\childdocof{|\textit{main}|}| just include the main file
at the top of each child file:
%
\begin{center}
|\input{|\textit{main}|}|
\end{center}
%
A simple redirection |\childdocforward{|\textit{dest}|}| is achieved by:
%
\begin{center}
|\def\jobname{|\textit{dest}|}\input{\jobname}|
\end{center}
%
The redirection with prefix
|\childdocforwardprefix[|\textit{prefix}|]{|\textit{dest}|}|
is accomplished by:
%
\begin{center}
\begin{tabular}{l}
|{\edef\jobname{\scantokens\expandafter{\jobname\noexpand}}|\\
|\def\redirectjob |\textit{prefix}|#1~~~{\gdef\jobname{|\textit{dest}|#1}}|\\
|\expandafter\redirectjob\jobname~~~}\input{\jobname}|
\end{tabular}
\end{center}

In an alternative approach,
child documents can be compiled by a specific command line
without additional code or specific definitions:
%
\begin{center}
|... -jobname "|\textit{target}|" "|[\textit{flags}]%
|\includeonly{|\textit{dest}|}\input{|\textit{main}|}"|
\end{center}
%

%%%%%%%%%%%%%%%%%%%%%%%%%%%%%%%%%%%%%%%%%%%%%%%%%%%%%%%%%%%%%%%%%%%%%%%%%%%%%%%%
%%%%%%%%%%%%%%%%%%%%%%%%%%%%%%%%%%%%%%%%%%%%%%%%%%%%%%%%%%%%%%%%%%%%%%%%%%%%%%%%
\section{Information}

%%%%%%%%%%%%%%%%%%%%%%%%%%%%%%%%%%%%%%%%%%%%%%%%%%%%%%%%%%%%%%%%%%%%%%%%%%%%%%%%
\subsection{Copyright}

Copyright \copyright{} 2017--2018 Niklas Beisert

This work may be distributed and/or modified under the
conditions of the \LaTeX{} Project Public License, either version 1.3
of this license or (at your option) any later version.
The latest version of this license is in
  \url{http://www.latex-project.org/lppl.txt}
and version 1.3 or later is part of all distributions of \LaTeX{}
version 2005/12/01 or later.

This work has the LPPL maintenance status `maintained'.

The Current Maintainer of this work is Niklas Beisert.

This work consists of the files |README.txt|, |childdoc.ins| and |childdoc.dtx|
as well as the derived files |childdoc.def|, |cdocsamp.tex|
with |cdocsch1.tex|, |cdocsch2.tex|, |cdocspt3.tex|, |cdocspt4.tex|,
|cdocsdrf.tex|, |cdocsfn1.tex|, |cdocsfn2.tex|
as well as |childdoc.pdf|.

%%%%%%%%%%%%%%%%%%%%%%%%%%%%%%%%%%%%%%%%%%%%%%%%%%%%%%%%%%%%%%%%%%%%%%%%%%%%%%%%
\subsection{Files and Installation}

The package consists of the files:
%
\begin{center}
\begin{tabular}{ll}
    |README.txt|   & readme file \\
    |childdoc.ins| & installation file \\
    |childdoc.dtx| & source file \\
    |childdoc.def| & definition file \\
    |cdocsamp.tex| & sample main file \\
    |cdocsch1.tex| & sample include file \\
    |cdocsch2.tex| & sample include file \\
    |cdocspt3.tex| & sample part file \\
    |cdocspt4.tex| & sample part file \\
    |cdocsdrf.tex| & sample redirection file \\
    |cdocsfn1.tex| & sample redirection file \\
    |cdocsfn2.tex| & sample redirection file \\
    |childdoc.pdf| & manual
\end{tabular}
\end{center}
%
The distribution consists of the files
|README.txt|, |childdoc.ins| and |childdoc.dtx|.
%
\begin{itemize}
\item
Run (pdf)\LaTeX{} on |childdoc.dtx|
to compile the manual |childdoc.pdf| (this file).
\item
Run \LaTeX{} on |childdoc.ins| to create the definitions file |childdoc.def|
and the sample |cdocsamp.tex| with include files
|cdocsch1.tex|, |cdocsch2.tex|, |cdocspt3.tex|, |cdocspt4.tex|,
|cdocsdrf.tex|, |cdocsfn1.tex|, |cdocsfn2.tex|.
Then copy the file |childdoc.def| to an appropriate directory of your \LaTeX{}
distribution, e.g.\ \textit{texmf-root}|/tex/latex/childdoc|.
\end{itemize}

%%%%%%%%%%%%%%%%%%%%%%%%%%%%%%%%%%%%%%%%%%%%%%%%%%%%%%%%%%%%%%%%%%%%%%%%%%%%%%%%
\subsection{Related CTAN Packages}

There are several other packages which offer a similar functionality:
%
\begin{itemize}
\item
The packages
\href{http://ctan.org/pkg/docmute}{\textsf{docmute}},
\href{http://ctan.org/pkg/includex}{\textsf{includex}} and
\href{http://ctan.org/pkg/standalone}{\textsf{standalone}}
provide commands to include only the document body of
a child file thus allowing both files to be compiled individually.
\item
The packages \href{http://ctan.org/pkg/subdocs}{\textsf{subdocs}}
and \href{http://ctan.org/pkg/subfiles}{\textsf{subfiles}}
provide structures in which the main and child documents can be
encapsulated and allowing them to be compiled individually.
The inclusion mechanism is different from the conventional |\include|.
\item
The package \href{http://ctan.org/pkg/combine}{\textsf{combine}}
is an elaborate solution to combine several documents into one.
\end{itemize}
%
See also the CTAN topic \href{http://ctan.org/topic/subdocs}{\textsf{subdocs}}
for further related packages.
The present package differs from the above solutions in that
a document structure constructed with the conventional |\include| mechanism
just needs two extra commands at the top of every file
such that all constituent files can be compiled individually.

%%%%%%%%%%%%%%%%%%%%%%%%%%%%%%%%%%%%%%%%%%%%%%%%%%%%%%%%%%%%%%%%%%%%%%%%%%%%%%%%
%\subsection{Feature Suggestions}
%
%The following is a list of features which may be useful for future
%versions of this package:
%%
%\begin{itemize}
%\item
%\ldots
%\end{itemize}

%%%%%%%%%%%%%%%%%%%%%%%%%%%%%%%%%%%%%%%%%%%%%%%%%%%%%%%%%%%%%%%%%%%%%%%%%%%%%%%%
\subsection{Revision History}

%%%%%%%%%%%%%%%%%%%%%%%%%%%%%%%%%%%%%%%%
\paragraph{v2.0:} 2018/12/30

\begin{itemize}
\item
immediate forward processing
\item
added |\childdocby| mechanism
\item
manual restructured
\end{itemize}

%%%%%%%%%%%%%%%%%%%%%%%%%%%%%%%%%%%%%%%%
\paragraph{v1.6:} 2018/01/17

\begin{itemize}
\item
application for development of include files
\item
corrections to manual
\end{itemize}

%%%%%%%%%%%%%%%%%%%%%%%%%%%%%%%%%%%%%%%%
\paragraph{v1.5:} 2017/05/21

\begin{itemize}
\item
more complete structuring introduced
\item
|\childdocof| introduced
\item
|\childdoc| renamed to |\childdocmain|
\item
|\childredirect| renamed to |\childdocforward| and |\childdocforwardprefix|
and functionality expanded
\end{itemize}

%%%%%%%%%%%%%%%%%%%%%%%%%%%%%%%%%%%%%%%%
\paragraph{v1.0:} 2017/04/27

\begin{itemize}
\item
manual and install package
\item
first version published on CTAN
\end{itemize}

%%%%%%%%%%%%%%%%%%%%%%%%%%%%%%%%%%%%%%%%
\paragraph{v0.6:} 2017/04/26

\begin{itemize}
\item
redirection mechanism added
\end{itemize}

%%%%%%%%%%%%%%%%%%%%%%%%%%%%%%%%%%%%%%%%
\paragraph{v0.5:} 2017/04/26

\begin{itemize}
\item
functionality in definition file
\end{itemize}


%%%%%%%%%%%%%%%%%%%%%%%%%%%%%%%%%%%%%%%%%%%%%%%%%%%%%%%%%%%%%%%%%%%%%%%%%%%%%%%%
%%%%%%%%%%%%%%%%%%%%%%%%%%%%%%%%%%%%%%%%%%%%%%%%%%%%%%%%%%%%%%%%%%%%%%%%%%%%%%%%
%%%%%%%%%%%%%%%%%%%%%%%%%%%%%%%%%%%%%%%%%%%%%%%%%%%%%%%%%%%%%%%%%%%%%%%%%%%%%%%%
\appendix

\settowidth\MacroIndent{\rmfamily\scriptsize 000\ }

 \DocInput{childdoc.dtx}

\end{document}
%</driver>
% \fi
%
% %%%%%%%%%%%%%%%%%%%%%%%%%%%%%%%%%%%%%%%%%%%%%%%%%%%%%%%%%%%%%%%%%%%%%%%%%%%%%%
% %%%%%%%%%%%%%%%%%%%%%%%%%%%%%%%%%%%%%%%%%%%%%%%%%%%%%%%%%%%%%%%%%%%%%%%%%%%%%%
% \section{Sample}
%\iffalse
%<*samplemain>
%\fi
%
% The following presents a sample document
% with two chapters, two parts, a title page,
% a compile flag as well as three forwarding files to set the flag.
% It consists of eight |.tex| files:
% \begin{center}
% \begin{tabular}{ll}
% |cdocsamp.tex|&main file\\
% |cdocsch1.tex|&include file for chapter 1\\
% |cdocsch2.tex|&include file for chapter 2\\
% |cdocspt3.tex|&include file for part 3\\
% |cdocspt4.tex|&include file for part 4\\
% |cdocsdrf.tex|&forwarding file for main file in draft mode\\
% |cdocsfi1.tex|&forwarding file for final version of chapter 1\\
% |cdocsfi2.tex|&forwarding file for final version of chapter 2\\
% \end{tabular}
% \end{center}
% Each of the eight files can be compiled directly by the \LaTeX{} compiler.
%
% %%%%%%%%%%%%%%%%%%%%%%%%%%%%%%%%%%%%%%
% \paragraph{Main File.}
%
% The main file is called |cdocsamp.tex|.
%
% Load the \textsf{childdoc} definitions and
% declare the filename for the main document:
%    \begin{macrocode}
\input{childdoc.def}
\childdocmain{}
%    \end{macrocode}

% Optional override for |\version| flag:
%    \begin{macrocode}
%%\ifchilddoc\else\providecommand{\version}{draft}\fi
%    \end{macrocode}

% Define the default values for the |\version| flag
% (|final| for the main file and |draft| for childs):
%    \begin{macrocode}
\ifchilddoc
\providecommand{\version}{draft}
\else
\providecommand{\version}{final}
\fi
%    \end{macrocode}

% Load the standard document class:
%    \begin{macrocode}
\documentclass[12pt]{article}
%    \end{macrocode}

% Start the document body:
%    \begin{macrocode}
\begin{document}
%    \end{macrocode}

% Declare a title page.
% Print title, part of document being processed and version flag:
%    \begin{macrocode}
\addtocounter{page}{-1}
\begin{center}
{\LARGE\bfseries{}childdoc example\par}
\vspace{1cm}
\ifchilddoc
\ifchilddocmanual part\else chapter\fi:
`\childdocname' of `\childdocjob'\par
\else
main document: `\childdocjob'\par
\fi
version: \version\par
\end{center}
\newpage
%    \end{macrocode}

% Manually include selected file,
% otherwise process as usual:
%    \begin{macrocode}
\ifchilddocmanual
\section*{part `\childdocname'}
\input{\childdocname}
\else
%    \end{macrocode}

% Include the two chapters:
%    \begin{macrocode}
\include{cdocsch1}
\include{cdocsch2}
%    \end{macrocode}

% Include the two parts unless only chapters should be displayed:
%    \begin{macrocode}
\ifchilddoc\else
\section{part three}
\input{cdocspt3}
\section{part four}
\input{cdocspt4}
\fi
%    \end{macrocode}

% Process as usual until here:
%    \begin{macrocode}
\fi
%    \end{macrocode}

% End of document body:
%    \begin{macrocode}
\end{document}
%    \end{macrocode}
%\iffalse
%</samplemain>
%\fi
%
% %%%%%%%%%%%%%%%%%%%%%%%%%%%%%%%%%%%%%%
% \paragraph{Chapter Include Files.}
%
% The include files are called |cdocsch1.tex| and |cdocsch2.tex|.
%
%\iffalse
%<*samplechap1|samplechap2>
%\fi

% Optional override for |\version| flag:
%    \begin{macrocode}
%%\providecommand{\version}{final}
%    \end{macrocode}

% Include the main document:
%    \begin{macrocode}
\input{childdoc.def}
\childdocof{cdocsamp}
%    \end{macrocode}

%\iffalse
%</samplechap1|samplechap2>
%\fi
%
%\iffalse
%<*samplechap1>
%\fi
% Some text for chapter 1:
%    \begin{macrocode}
\section{one}
some text in chapter one
%    \end{macrocode}

%\iffalse
%</samplechap1>
%\fi
% Some text for chapter 2:
%\iffalse
%<*samplechap2>
%\fi
%    \begin{macrocode}
\section{two}
more text in chapter two
%    \end{macrocode}

%\iffalse
%</samplechap2>
%\fi
%
% %%%%%%%%%%%%%%%%%%%%%%%%%%%%%%%%%%%%%%
% \paragraph{Part Include Files.}
%
% The include files are called |cdocspt3.tex| and |cdocspt4.tex|.
%
%\iffalse
%<*samplepart3|samplepart4>
%\fi

% Optional override for |\version| flag:
%    \begin{macrocode}
%%\providecommand{\version}{final}
%    \end{macrocode}

% Include the main document:
%    \begin{macrocode}
\input{childdoc.def}
\childdocby{cdocsamp}
%    \end{macrocode}

%\iffalse
%</samplepart3|samplepart4>
%\fi
%
%\iffalse
%<*samplepart3>
%\fi
% Some text for part 3:
%    \begin{macrocode}
some text in part three
%    \end{macrocode}

%\iffalse
%</samplepart3>
%\fi
% Some text for part 4:
%\iffalse
%<*samplepart4>
%\fi
%    \begin{macrocode}
more text in part four
%    \end{macrocode}

%\iffalse
%</samplepart4>
%\fi
%
% %%%%%%%%%%%%%%%%%%%%%%%%%%%%%%%%%%%%%%
% \paragraph{Forwarding for a Complete Draft.}
%
% The following forwarding file |cdocsdrf.tex|
% compiles the main document in draft mode:
%\iffalse
%<*sampledraft>
%\fi
%    \begin{macrocode}
\def\version{draft}
\input{childdoc.def}
\childdocforward{cdocsamp}
%    \end{macrocode}

%\iffalse
%</sampledraft>
%\fi
%
% %%%%%%%%%%%%%%%%%%%%%%%%%%%%%%%%%%%%%%
% \paragraph{Forwarding for Final Version of the Chapters.}
%
% The following forwarding files |cdocsfn1.tex| and |cdocsfn2.tex|
% (with identical content)
% compile the final versions of the child documents
% |cdocsch1.tex| and |cdocsch2.tex|, respectively:
%\iffalse
%<*samplefinal>
%\fi
%    \begin{macrocode}
\def\version{final}
\input{childdoc.def}
\childdocforwardprefix[cdocsamp]{cdocsfn}{cdocsch}
%    \end{macrocode}

%\iffalse
%</samplefinal>
%\fi
%
% %%%%%%%%%%%%%%%%%%%%%%%%%%%%%%%%%%%%%%
% \paragraph{Command Line Processing.}
%
% The following three command lines generate the output files
% |cdocscld|, |cdocscl1| and |cdocscl2|
% which should be identical to
% |cdocsdrf|, |cdocsch1| and |cdocsfn2|, respectively:
% \begin{center}
% \begin{tabular}{l}
% |latex -jobname cdocscld \|\\
% |  "\def\version{draft}\input{childdoc.def}\childdocforward{cdocsamp}"|\\
% |latex -jobname cdocscl1 \|\\
% |  "\input{childdoc.def}\childdocforward[cdocsamp]{cdocsch1}"|\\
% |latex -jobname cdocscl2 \|\\
% |  "\def\version{final}\input{childdoc.def}\childdocforward{cdocsch2}"|
% \end{tabular}
% \end{center}
% Note that the trailing backslash on each first line
% merely continues the input to the second line
% (for convenient cut ant paste).
% Furthermore, the command |latex| can be replaced by any
% of its alternative versions such as |pdflatex|.
%
% %%%%%%%%%%%%%%%%%%%%%%%%%%%%%%%%%%%%%%%%%%%%%%%%%%%%%%%%%%%%%%%%%%%%%%%%%%%%%%
% %%%%%%%%%%%%%%%%%%%%%%%%%%%%%%%%%%%%%%%%%%%%%%%%%%%%%%%%%%%%%%%%%%%%%%%%%%%%%%
% \section{Implementation}
%\iffalse
%<*package>
%\fi
%
% This section describes the definitions file |childdoc.def|.

% The definitions cannot be loaded using |\usepackage| or |\RequirePackage|
% which has a mechanism to prevent loading a style file more than once.
% When loading the definitions by means of |\input|
% multiple instances have to be prevented manually:
%\iffalse
%This code needs to be before the `\ProvidesFile' directive
%which is defined at the beginning of this file.
%Therefore it is also placed there and commented out here.
%</package>
%<*discard>
%\fi
%    \begin{macrocode}
\ifdefined\childdocmain\endinput\fi
%    \end{macrocode}
%\iffalse
%</discard>
%<*package>
%\fi
%
% \macro{\ifchilddoc}
% \macro{\ifchilddocmanual}
% The conditional |\ifchilddoc| tells whether a
% child (true) or main (false) document is being compiled.
% The conditional |\ifchilddocmanual| tells whether
% the |\includeonly| mechanism is used (false) or
% the selection of child files must be performed manually (true).
% The definitions initialise to false:
%    \begin{macrocode}
\newif\ifchilddoc
\newif\ifchilddocmanual
%    \end{macrocode}

% \macro{\childdocname}
% \macro{\childdocjob}
% The macro |\childdocname| stores the name of the main document
% to be compiled. The macro |\childdocjob| stores the name of
% the document on which the \LaTeX{} compiler was originally invoked.
% The content of |\jobname| cannot be compared
% to filenames specified in the source due to different catcodes.
% The following code rescans |\jobname|, stores the result
% in |\childdocname| and saves a copy in |\childdocjob|:
%    \begin{macrocode}
\edef\childdocname{\scantokens\expandafter{\jobname\noexpand}}
\let\childdocjob\childdocname
%    \end{macrocode}

% \macro{\childdocdisable}
% The macro |\childdocdisable| prevents the main file
% from being processed more than once.
% At this stage, the main document command |\childdocmain|
% is assumed to be called once again where it should do nothing.
% Any subsequent call to it should prevent
% a secondary processing of the main document
% It overwrites the forwarding commands
% |\childdocof| and |\childdocforward|
% with empty macros to prevent further inclusions of the main document:
%    \begin{macrocode}
\newcommand{\childdocdisable}
{
  \renewcommand{\childdocmain}[1]{\renewcommand{\childdocmain}[1]{\endinput}}
  \renewcommand{\childdocof}[1]{}
  \renewcommand{\childdocby}[2][]{}
  \renewcommand{\childdocforward}[2][]{}
  \renewcommand{\childdocdisable}{}
}
%    \end{macrocode}

% \macro{\childdocmain}
% The macro |\childdocmain| is to be called at the top of the main file
% with nothing or the main filename (without extension) as argument.
% First, it breaks loops.
% If the argument is not empty and does not match |\childdocname|
% (which is set by the first inclusion of |childdoc.def|),
% |\ifchilddoc| is set to true, |\includeonly| is applied to the child file
% and |\jobname| is set to the main file
% (for proper handling of |.aux| files):
%    \begin{macrocode}
\newcommand{\childdocmain}[1]
{
  \childdocdisable\childdocmain{}
  \if?#1?\else
    \begingroup
      \def\childdoctmp{#1}
      \ifx\childdoctmp\childdocname
        \def\childdoctmp{}
      \else
        \def\childdoctmp
        {
          \childdoctrue
          \includeonly{\childdocname}
          \def\childdocjob{#1}
          \def\jobname{#1}
        }
      \fi
      \expandafter
    \endgroup
    \childdoctmp
  \fi
}
%    \end{macrocode}

% \macro{\childdocof}
% The command |\childdocof| redirects
% compilation to the main file |#1|.
%    \begin{macrocode}
\newcommand{\childdocof}[1]
{
  \childdocdisable
  \childdoctrue
  \includeonly{\childdocname}
  \def\jobname{#1}
  \def\childdocjob{#1}
  \input{#1}
}
%    \end{macrocode}

% \macro{\childdocby}
% The command |\childdocby| ....
%    \begin{macrocode}
\newcommand{\childdocby}[2][]
{
  \childdocdisable
  \childdoctrue
  \childdocmanualtrue
  \if?#1?\else
    \def\jobname{#2}
  \fi
  \def\childdocjob{#2}
  \input{#2}
  \endinput
}
%    \end{macrocode}

% \macro{\childdocforward}
% The command |\childdocforward| redirects
% compilation to the main file or
% (if the optional argument is given) a child file.
% Parameters are set as if the main file
% or a child file starting with |\childdocof| was compiled.
% Then compilation is handed over to the main file:
%    \begin{macrocode}
\newcommand{\childdocforward}[2][]
{
  \begingroup
    \if?#1?
      \def\childdoctmp
      {
        \def\childdocname{#2}
        \def\childdocjob{#2}
        \def\jobname{#2}
        \input{#2}
        \endinput
      }
    \else
      \def\childdoctmp
      {
        \childdocdisable
        \def\childdocname{#2}
        \childdoctrue
        \includeonly{#2}
        \def\childdocjob{#1}
        \def\jobname{#1}
        \input{#1}
        \endinput
      }
    \fi
    \expandafter
  \endgroup
  \childdoctmp
}
%    \end{macrocode}

% \macro{\childdocforwardprefix}
% The command |\childdocforwardprefix| redirects
% compilation to the main or a child file by means of a pattern.
% The prefix |#1| in the current filename is replaced by |#2|
% and the suffix of the current filename is kept
% (it is assumed that the filename does not contain the substring `|~~~|'
% which is used as a delimiter).
% Compilation is handed over to the new file by |\childdocforward|:
%    \begin{macrocode}
\newcommand{\childdocforwardprefix}[3][]
{
  \begingroup
    \def\childdocextract #2##1~~~{\def\childdoctmp{\childdocforward[#1]{#3##1}}}
    \expandafter\childdocextract\childdocname~~~
    \expandafter
  \endgroup
  \childdoctmp
}
%    \end{macrocode}

% \macro{\childdoc}
% The deprecated macro |\childdoc| is a legacy version of |\childdocmain|:
%    \begin{macrocode}
\newcommand{\childdoc}{\childdocmain}
%    \end{macrocode}

% \macro{\childdocredirect}
% The deprecated macro |\childdocredirect| is a legacy version
% of |\childdocforward| and |\childdocforwardprefix|:
%    \begin{macrocode}
\newcommand{\childdocredirect}[2][]
{
  \begingroup
    \if?#1?
      \def\childdoctmp{\childdocforward{#2}}
    \else
      \def\childdoctmp{\childdocforwardprefix{#1}{#2}}
    \fi
    \expandafter
  \endgroup
  \childdoctmp
}
%    \end{macrocode}

%\iffalse
%</package>
%\fi
%
\endinput
|\\
|\childdocmain{|\textit{main}|}|\\
\end{tabular}
\end{center}
%
If |\jobname| does not match the argument \textit{main} of |\childdocmain|,
it is assumed that |\jobname| points to the child file to be compiled.
When using |\childdocmain| with the main file specified as argument,
it suffices to start a child file
with just |\input{|\textit{main}|}|
without loading of the package and using |\childdocof|.
If instead all processing is done
with the appropriate \textsf{childdoc} directives,
the argument of \textit{main} of |\childdocmain| can be empty.

An alternative version of the command line processing described
in \secref{sec:commandline} using the detection mechanism reads:
%
\begin{center}
|... -jobname "|\textit{target}|" "|[\textit{flags}]%
[|\def\jobname{|\textit{dest}|}|]|\input{|\textit{main}|}"|
\end{center}

%%%%%%%%%%%%%%%%%%%%%%%%%%%%%%%%%%%%%%%%%%%%%%%%%%%%%%%%%%%%%%%%%%%%%%%%%%%%%%%%
\subsection{Manual Code}
\label{sec:manual}

In case one cannot be certain whether the definitions file |childdoc.def|
is installed on the target \TeX{} distribution
and one prefers not to ship it,
it is conceivable to paste a few relevant commands into the sources.

To that end, drop all statements |% \iffalse
%
% childdoc.dtx Copyright (C) 2017-2018 Niklas Beisert
%
% This work may be distributed and/or modified under the
% conditions of the LaTeX Project Public License, either version 1.3
% of this license or (at your option) any later version.
% The latest version of this license is in
%   http://www.latex-project.org/lppl.txt
% and version 1.3 or later is part of all distributions of LaTeX
% version 2005/12/01 or later.
%
% This work has the LPPL maintenance status `maintained'.
%
% The Current Maintainer of this work is Niklas Beisert.
%
% This work consists of the files childdoc.dtx and childdoc.ins
% and the derived files childdoc.def and cdocsamp.tex with
% cdocsch1.tex, cdocsch2.tex, cdocsdrf.tex, cdocsfn1.tex, cdocsfn2.tex.
%
%<package>\ifdefined\childdocmain\endinput\fi
%<package>\ProvidesFile{childdoc.def}[2018/12/30 v2.0 child document driver]
%<samplemain>\ProvidesFile{cdocsamp.tex}[2018/12/30 v2.0 sample for childdoc]
%<*driver>
%\ProvidesFile{childdoc.drv}[2018/12/30 v2.0 childdoc reference manual file]
\PassOptionsToClass{10pt,a4paper}{article}
\documentclass{ltxdoc}

\usepackage[margin=35mm]{geometry}
\usepackage{hyperref}
\usepackage{hyperxmp}
\usepackage[usenames]{color}

\hypersetup{colorlinks=true}
\hypersetup{pdfstartview=FitH}
\hypersetup{pdfpagemode=UseNone}
\hypersetup{pdfsource={}}
\hypersetup{pdflang={en-UK}}
\hypersetup{pdfcopyright={Copyright 2017-2018 Niklas Beisert.
  This work may be distributed and/or modified under the
  conditions of the LaTeX Project Public License, either version 1.3
  of this license or (at your option) any later version.}}
\hypersetup{pdflicenseurl={http://www.latex-project.org/lppl.txt}}
\hypersetup{pdfcontactaddress={ETH Zurich, ITP, HIT K,
  Wolfgang-Pauli-Strasse 27}}
\hypersetup{pdfcontactpostcode={8093}}
\hypersetup{pdfcontactcity={Zurich}}
\hypersetup{pdfcontactcountry={Switzerland}}
\hypersetup{pdfcontactemail={nbeisert@itp.phys.ethz.ch}}
\hypersetup{pdfcontacturl={http://people.phys.ethz.ch/\xmptilde nbeisert/}}

\newcommand{\secref}[1]{\hyperref[#1]{section \ref*{#1}}}

\parskip1ex
\parindent0pt
\let\olditemize\itemize
\def\itemize{\olditemize\parskip0pt}

\begin{document}

\title{The \textsf{childdoc} Package}
\hypersetup{pdftitle={The childdoc Package}}
\author{Niklas Beisert\\[2ex]
  Institut f\"ur Theoretische Physik\\
  Eidgen\"ossische Technische Hochschule Z\"urich\\
  Wolfgang-Pauli-Strasse 27, 8093 Z\"urich, Switzerland\\[1ex]
  \href{mailto:nbeisert@itp.phys.ethz.ch}
  {\texttt{nbeisert@itp.phys.ethz.ch}}}
\hypersetup{pdfauthor={Niklas Beisert}}
\hypersetup{pdfsubject={Manual for the LaTeX2e Package childdoc}}
\date{30 December 2018, \textsf{v2.0}}
\maketitle

\begin{abstract}\noindent
\textsf{childdoc} is a \LaTeXe{} package
that enables the direct compilation
of document sections included by |\include|
to individual files.
\end{abstract}

\begingroup
\parskip0ex
\tableofcontents
\endgroup

%%%%%%%%%%%%%%%%%%%%%%%%%%%%%%%%%%%%%%%%%%%%%%%%%%%%%%%%%%%%%%%%%%%%%%%%%%%%%%%%
%%%%%%%%%%%%%%%%%%%%%%%%%%%%%%%%%%%%%%%%%%%%%%%%%%%%%%%%%%%%%%%%%%%%%%%%%%%%%%%%
\section{Introduction}

\LaTeX{} provides a mechanism to structure a large document (such as a book)
into a main file and several child files (containing the chapters)
using the |\include| command.
This mechanism is beneficial for documents
which span hundreds of pages in order to
make the source file(s) more manageable.
Moreover, compilation can be restricted to
selected child files by means of the |\includeonly| command.
The latter feature can be used to reduce the compilation time while editing
(this was significantly more useful in the earlier days of \LaTeX{})
or to generate a smaller document which is easier to navigate.
Another application of |\includeonly| is to generate
documents consisting of selected parts of the complete document.

However, there are a few drawbacks of the plain |\include| mechanism:
\begin{itemize}
\item
The child files cannot be compiled on their own,
they can only be compiled via the main file.
A naive editing environment
(such as a text editor with an option
to have the current file processed by \LaTeX)
may require one to switch to the main file before compiling;
attempting to compile the child file produces errors.
\item
The main file must be modified (each time)
to adjust the |\includeonly| command
to the present needs. This easily leaves the main file in a messy state.
\item
The generated document will always carry the filename
of the main document. This is inconvenient if
several child files are to be compiled and
to be kept for distribution.
\end{itemize}

The present package provides a simple interface
to make child files individually compilable by \LaTeX{}.
Compiling a child file then has the same effect as compiling
the main file with an |\includeonly| command
to select the appropriate child.
Moreover the generated document will carry the name of the child
rather than the main file.
This resolves all three above issues.

This feature is meant to make the editing of books,
thesis documents and lecture notes somewhat more convenient.
However, the package can also be used efficiently for
composing a series of documents (such as exercise sheets)
which are typically distributed individually.
It then assists the author in generating the individual documents
(potentially in different versions)
as well as a document containing the collected series.
Another application is in developing style files
or other kinds of included material
where compilation of the style file could redirect
to a sample or test file.

%%%%%%%%%%%%%%%%%%%%%%%%%%%%%%%%%%%%%%%%%%%%%%%%%%%%%%%%%%%%%%%%%%%%%%%%%%%%%%%%
%%%%%%%%%%%%%%%%%%%%%%%%%%%%%%%%%%%%%%%%%%%%%%%%%%%%%%%%%%%%%%%%%%%%%%%%%%%%%%%%
\section{Usage}

First of all, the package \textsf{childdoc} is \emph{not} a standard
\LaTeXe{} |.sty| style file! Therefore it needs to be invoked in
a non-standard way.

%%%%%%%%%%%%%%%%%%%%%%%%%%%%%%%%%%%%%%%%%%%%%%%%%%%%%%%%%%%%%%%%%%%%%%%%%%%%%%%%
\subsection{Included Files}
\label{sec:include}

%%%%%%%%%%%%%%%%%%%%%%%%%%%%%%%%%%%%%%%%
\DescribeMacro{\childdocmain}
To use the package, add the commands
\begin{center}
\begin{tabular}{l}
|\input{childdoc.def}|\\
|\childdocmain{}|\\
\end{tabular}
\end{center}
at the very top of the main \LaTeX{} file,
in particular \emph{before} the |\documentclass| statement!
The argument of |\childdocmain| should be left empty
(but it must be present).

%%%%%%%%%%%%%%%%%%%%%%%%%%%%%%%%%%%%%%%%
\DescribeMacro{\childdocof}
Furthermore, add the commands
\begin{center}
\begin{tabular}{l}
|\input{childdoc.def}|\\
|\childdocof{|\textit{main}|}|\\
\end{tabular}
\end{center}
at the top of every child file \textit{child}
which is included by |\include{|\textit{child}|}|
from within the main file
(or at least for those files to be compiled individually).
The argument \textit{main} must be the filename of the main file.

There are a couple of
considerations in setting up the main and child documents:

%%%%%%%%%%%%%%%%%%%%%%%%%%%%%%%%%%%%%%%%
\paragraph{Restrictions.}

Please note the following restrictions:
\begin{itemize}
\item
|\childdocmain| must be called with one argument \textit{main}
to ensure compatibility with earlier version of the package.
It must either be empty (|\childdocmain{}|)
or precisely match the filename of the main file in which it is specified.
See \secref{sec:detection} for further information.
\item
The filename \textit{main} must be specified without the |.tex| extension.
\item
The filename \textit{main} is case sensitive
(even in case-insensitive file systems)
due to internal string comparison.
\item
The argument \textit{main} should be fully expanded, it cannot be a macro.
\item
Subdirectories and special characters should be avoided in filenames.
\item
The command |\childdocmain{|\textit{main}|}| must be followed by a whitespace.
It should not be followed immediately by another command
or by a comment mark `|%|'.
This is because the \TeX{} parser reads the token immediately following
the argument of |\childdocmain| and puts it
at the beginning of every child section;
however, a white\-space is ignored.
\end{itemize}

%%%%%%%%%%%%%%%%%%%%%%%%%%%%%%%%%%%%%%%%
\paragraph{Content of Main File.}

It is advisable to place all content in the child files included by |\include|.
Any output contained in the main file will appear in all child documents
unless suppressed manually;
it cannot be suppressed automatically by the |\includeonly| directive
and thus should normally be avoided.
A method to include some content in the main file
by means of conditional processing is described in \secref{sec:conditional}.

%%%%%%%%%%%%%%%%%%%%%%%%%%%%%%%%%%%%%%%%
\paragraph{Page Numbering.}

When only a part of the document is compiled,
the appropriate numbering of pages
(as well as other status parameters)
is determined from the |.aux| files.
The latter contain information from previous passes.
However this information needs to propagate through
all intermediate child documents.
Therefore the page numbering in child documents may well
be inconsistent until the complete document is compiled at least once.

A useful (if unconventional) way to always ensure a consistent
page numbering is to restart the numbering in each child document
and denote the pages by `\textit{child}|.|\textit{page}'
where \textit{child} represents the chapter/section number of the child file.
This can be achieved by the command
|\numberwithin{page}{|\textit{child}|}|
of the \textsf{amsmath} package
where \textit{child} can be |chapter| or |section|
depending on the chosen structuring.
Alternatively, one can modify the macro |\thepage| appropriately
and reset the counter |page| at the start of each child file.

%%%%%%%%%%%%%%%%%%%%%%%%%%%%%%%%%%%%%%%%%%%%%%%%%%%%%%%%%%%%%%%%%%%%%%%%%%%%%%%%
\subsection{Conditional Processing}
\label{sec:conditional}

The package provides a mechanism to compile different versions
of a document. To customise the versions further some conditional processing
can come in handy to distinguish which version is being compiled.
The package provides two macros to describe the compilation context:

%%%%%%%%%%%%%%%%%%%%%%%%%%%%%%%%%%%%%%%%
\DescribeMacro{\ifchilddoc}
The conditional |\ifchilddoc| distinguishes between the compilation of
child documents and the main document:
%
\begin{center}
|\ifchilddoc |\textit{child-code}| |[|\||else |\textit{main-code}]| \||fi|
\end{center}

%%%%%%%%%%%%%%%%%%%%%%%%%%%%%%%%%%%%%%%%
\DescribeMacro{\childdocname}
\DescribeMacro{\childdocjob}
The macro |\childdocname| contains the filename (without extension)
of the main or child file being processed.
Note that |\childdocjob| will always contain the name of the main file.

%%%%%%%%%%%%%%%%%%%%%%%%%%%%%%%%%%%%%%%%
\paragraph{Title Page.}

Conditional processing can be used to include a title or banner page
in the main document when proper precautions are taken.
Importantly, the code in the main file should ensure that the page counter
(as well as other status parameters which are stored in the |.aux| files)
takes the same value after the conditional processing.
Otherwise the page numbers may take divergent values
depending on which part is compiled.

For example, a title page could be declared by:
%
\begin{center}
\begin{tabular}{l}
|\ifchilddoc\||else|\\
|\addtocounter{page}{-1}|\\
\textit{code for title page}\\
|\newpage|\\
|\||fi|
\end{tabular}
\end{center}
%
A banner page for the child documents can be generated by:
%
\begin{center}
\begin{tabular}{l}
|\ifchilddoc|\\
|\addtocounter{page}{-1}|\\
\textit{code for banner page}\\
|\newpage|\\
|\||fi|
\end{tabular}
\end{center}
%
Here one could write a message such as:
\begin{center}
|This is the part \childdocname{} of \childdocjob{}.|
\end{center}

%%%%%%%%%%%%%%%%%%%%%%%%%%%%%%%%%%%%%%%%%%%%%%%%%%%%%%%%%%%%%%%%%%%%%%%%%%%%%%%%
\subsection{Flags}
\label{sec:flags}

The package makes it easy to generate different versions
of the main or child documents.
To this end compilation flags can be defined
and assigned different default values.
They will be particularly useful in conjunction
with the forwarding mechanism described in \secref{sec:forward}.

For example, it may be useful to have a flag |\version|
which can be set to |draft| or |final|.
The document source will contain some conditional code
depending on the value of |\version|.
Suppose further, the flag should default to |final| for the main file
and to |draft| for child files
which is a natural assignment for editing the document.
This is achieved by placing the following code
in the preamble of the main document
(below the |\childdocmain| directive):
%
\begin{center}
\begin{tabular}{l}
|\ifchilddoc|\\
|\providecommand{\version}{draft}|\\
|\||else|\\
|\providecommand{\version}{final}|\\
|\||fi|
\end{tabular}
\end{center}
%
The definition by |\providecommand| makes sure
that previous definitions are not overwritten.
Further statements |\providecommand{\version}{...}|
can thus be added before the above code to override it.

For the main file, one might add a line
(between |\childdocmain| and the above block)
%
\begin{center}
|%\ifchilddoc\||else\providecommand{\version}{draft}\||fi|
\end{center}
%
which can be uncommented to produce a draft version.
Likewise one can add a line to the very top of a child file
(above the |\childdocof{|\textit{main}|}| directive)
%
\begin{center}
|%\providecommand{\version}{final}|
\end{center}
%
which can be uncommented to produce the final version of this child document.

%%%%%%%%%%%%%%%%%%%%%%%%%%%%%%%%%%%%%%%%%%%%%%%%%%%%%%%%%%%%%%%%%%%%%%%%%%%%%%%%
\subsection{Forwarding}
\label{sec:forward}

Different versions of the main or child documents
using compilation flags as described in \secref{sec:flags}
can be (permanently) stored in different files
for convenient compilation, viewing and distribution.
To this end, the package defines a command
to pass on compilation to a different file:

%%%%%%%%%%%%%%%%%%%%%%%%%%%%%%%%%%%%%%%%
\DescribeMacro{\childdocforward}
The command |\childdocforward| redirects processing to
another source file:
%
\begin{center}
\begin{tabular}{l}
|\input{childdoc.def}|\\
|\childdocforward[|\textit{main}|]{|\textit{dest}|}|\\
\end{tabular}
\end{center}
%
The argument \textit{dest} is the destination file
(without extension).
It should be the main file or one of the child files.
Note that further \textsf{childdoc} directives
such as |\childdocof| and |\childdocforward|
in the indicated file will be processed in this form.
The optional argument \textit{main}
passes on directly to the main file \textit{main}
while pretending to compile the child \textit{dest}.
This form behaves as if \textit{dest}
issues |\childdocof{|\textit{main}|}| right away,
and no further \textsf{childdoc} directives will be processed.

%%%%%%%%%%%%%%%%%%%%%%%%%%%%%%%%%%%%%%%%
\DescribeMacro{\...prefix}
In the alternative form |\childdocforwardprefix|,
%
\begin{center}
\begin{tabular}{l}
|\input{childdoc.def}|\\
|\childdocforwardprefix[|\textit{main}|]{|\textit{prefix}|}{|\textit{dest}|}|
\end{tabular}
\end{center}
%
the destination file is determined by a pattern
depending on the current file:
To make this work, the current file must be called
`{\textit{prefix}\hspace{0.2em}\textit{suffix}}'
with \textit{prefix} matching precisely the argument.
Processing is then passed on to the file
`{\textit{dest}\hspace{0.2em}\textit{suffix}}'.
Surely, the same effect is achieved by
directly specifying the
argument `{\textit{dest}\hspace{0.2em}\textit{suffix}}'
in the first form.
However, that requires to set up a different file
for each child. With the alternative form of the command
all these files can have exactly the same content
which simplifies setting them up and maintaining them.

For example, the following file |draft.tex|
with a compilation flag |\version| as described in \secref{sec:flags}
compiles the main document as a draft:
%
\begin{center}
\begin{tabular}{l}
|\def\version{draft}|\\
|\input{childdoc.def}|\\
|\childdocforward{|\textit{main}|}|
\end{tabular}
\end{center}
%
Likewise, the following files |final|\textit{nn}|.tex|
compile the final version of the child document
|child|\textit{nn}|.tex|:
%
\begin{center}
\begin{tabular}{l}
|\def\version{final}|\\
|\input{childdoc.def}|\\
|\childdocforwardprefix{final}{child}|
\end{tabular}
\end{center}
%

Note that when several versions of a main file and/or of each child file
are to be generated, it may be convenient to set up a |Makefile| or
shell script to automatise the process.

%%%%%%%%%%%%%%%%%%%%%%%%%%%%%%%%%%%%%%%%%%%%%%%%%%%%%%%%%%%%%%%%%%%%%%%%%%%%%%%%
\subsection{Command Line Processing}
\label{sec:commandline}

The effect of redirection files can also be achieved by invoking
the \LaTeX{} compiler with a more elaborate command line.
Most conveniently this should be done as part
of a shell script or a |Makefile|.

When using \textsf{childdoc} in the main file, the following
command lines effectively perform a redirection
(note that depending on the shell being used,
backslashes may have to be doubled: `|\|' $\to$ `|\\|'):
%
\begin{center}
|... -jobname "|\textit{target}|" |\\|"|[\textit{flags}]%
|\input{childdoc.def}\childdocforward[|\textit{main}|]{|\textit{dest}|}"|
\end{center}
%
Here \textit{target} is the name of the output file,
\textit{main} is the name of the main file
and \textit{dest} is the name of the main or child file to be processed
(all filenames without extensions).
The optional argument \textit{main} can be omitted
if \textit{main} matches \textit{dest}.
Optionally, compilation \textit{flags} can be defined via |\def| commands.
This command line makes the \TeX{} engine believe
it is compiling the file \textit{target}
whose content is specified as the latter parameter.
The provided code then forwards the processing to
\textit{main} or \textit{dest} as described in \secref{sec:forward}.

%%%%%%%%%%%%%%%%%%%%%%%%%%%%%%%%%%%%%%%%%%%%%%%%%%%%%%%%%%%%%%%%%%%%%%%%%%%%%%%%
\subsection{Include by Input}
\label{sec:input}

Including child documents by |\include| has some restrictions by design.
Most notably, the content of a child document always occupies
its own set of pages; pages cannot be shared between child documents.
Usually, this behaviour makes perfect sense
because each child document contain an essential part of the document.
However, in some situations it may be desirable to compose
a document from a collection of parts
without having mandatory page breaks between then.
For this case, the package
provides a mechanism to include parts
by |\input| which can also be processed individually.
However, by construction this mechanism
requires manual handling of the content to be output.

%%%%%%%%%%%%%%%%%%%%%%%%%%%%%%%%%%%%%%%%
\DescribeMacro{\ifchilddocmanual}
The main file should be prepared as usual, see \secref{sec:include}.
However, the document body must make a distinction
between processing of an individual part and of the main document, e.g.:
%
\begin{center}
\begin{tabular}{l}
|\ifchilddocmanual|\\
|\input{\childdocname}|\\
|\||else|\\
\textit{document body with }|\input{|\textit{part}|}|\\
|\||fi|
\end{tabular}
\end{center}
%
The conditional |\ifchilddocmanual| is true whenever
a part to be included by |\input| is being compiled,
and the name of the part is stored in |\childdocname|.

%%%%%%%%%%%%%%%%%%%%%%%%%%%%%%%%%%%%%%%%
\DescribeMacro{\childdocby}
Each part to be included by |\input| should start with:
%
\begin{center}
\begin{tabular}{l}
|\input{childdoc.def}|\\
|\childdocby{|\textit{main}|}|\\
\end{tabular}
\end{center}
%
The directive |\childdocby| is similar to |\childdocof|
described in \secref{sec:include},
but the subsequent selection of content must be done manually.
To that end, both |\ifchilddoc| and |\ifchilddocmanual|
will be true upon processing of a part,
and the name of the part is stored in |\childdocname|.
Note that |\jobname| will be set to the filename of the current part
so that each part receives an individual |.aux| file
that does not interfere with the |.aux| file(s) of the main document.
This behaviour can be altered by the alternative form
|\childdocby[*]{|\textit{main}|}| (with a non-empty optional argument)
which uses the |.aux| file of the main document
by setting |\jobname| to \textit{main}.

%%%%%%%%%%%%%%%%%%%%%%%%%%%%%%%%%%%%%%%%%%%%%%%%%%%%%%%%%%%%%%%%%%%%%%%%%%%%%%%%
\subsection{Driver Development}
\label{sec:driver}

The \textsf{childdoc} mechanism can also be use for the development
of definition files such as \LaTeX{} styles or classes.
This case differs from the above setup with multiple parts
included by |\include| in that no |\includeonly| should be invoked.
This can be achieved by starting the include file
(before |\ProvidesPackage|) with:
%
\begin{center}
\begin{tabular}{l}
|\input{childdoc.def}|\\
|\childdocforward{|\textit{main}|}|\\
\end{tabular}
\end{center}
%
or alternatively with:
%
\begin{center}
\begin{tabular}{l}
|\input{childdoc.def}|\\
|\childdocby{|\textit{main}|}|\\
\end{tabular}
\end{center}
%
Both forms have slightly different effects as described above.
The main file is prepared as usual, see \secref{sec:include}.

%%%%%%%%%%%%%%%%%%%%%%%%%%%%%%%%%%%%%%%%%%%%%%%%%%%%%%%%%%%%%%%%%%%%%%%%%%%%%%%%
\subsection{Legacy Detection}
\label{sec:detection}

The directive |\childdocmain| in the main file can detect
whether the complete document or merely a child is to be compiled
even without using the directive |\childdocof|.
This method is deprecated because it is less robust
and there is no compelling reason to use it;
it is merely provided for backward compatibility
and it may be removed in future versions.

If the detection mechanism is to be used,
it is mandatory to correctly specify
the filename of the main file as the argument of |\childdocmain|:
%
\begin{center}
\begin{tabular}{l}
|\input{childdoc.def}|\\
|\childdocmain{|\textit{main}|}|\\
\end{tabular}
\end{center}
%
If |\jobname| does not match the argument \textit{main} of |\childdocmain|,
it is assumed that |\jobname| points to the child file to be compiled.
When using |\childdocmain| with the main file specified as argument,
it suffices to start a child file
with just |\input{|\textit{main}|}|
without loading of the package and using |\childdocof|.
If instead all processing is done
with the appropriate \textsf{childdoc} directives,
the argument of \textit{main} of |\childdocmain| can be empty.

An alternative version of the command line processing described
in \secref{sec:commandline} using the detection mechanism reads:
%
\begin{center}
|... -jobname "|\textit{target}|" "|[\textit{flags}]%
[|\def\jobname{|\textit{dest}|}|]|\input{|\textit{main}|}"|
\end{center}

%%%%%%%%%%%%%%%%%%%%%%%%%%%%%%%%%%%%%%%%%%%%%%%%%%%%%%%%%%%%%%%%%%%%%%%%%%%%%%%%
\subsection{Manual Code}
\label{sec:manual}

In case one cannot be certain whether the definitions file |childdoc.def|
is installed on the target \TeX{} distribution
and one prefers not to ship it,
it is conceivable to paste a few relevant commands into the sources.

To that end, drop all statements |\input{childdoc.def}|
and perform the replacements as outlined below.
Instead of |\childdocmain{|\textit{main}|}| add the following code
to the top of the main file:
%
\begin{center}
\begin{tabular}{l}
|\||ifdefined\childdocname\endinput\||fi\newif\ifchilddoc|\\
|\edef\childdocname{\scantokens\expandafter{\jobname\noexpand}}|\\
|\def\childdocmain{|\textit{main}|}\||ifx\childdocmain\childdocname\||else|\\
|\childdoctrue\includeonly{\childdocname}\let\jobname\childdocmain\||fi|\\
\end{tabular}
\end{center}
%
Instead of |\childdocof{|\textit{main}|}| just include the main file
at the top of each child file:
%
\begin{center}
|\input{|\textit{main}|}|
\end{center}
%
A simple redirection |\childdocforward{|\textit{dest}|}| is achieved by:
%
\begin{center}
|\def\jobname{|\textit{dest}|}\input{\jobname}|
\end{center}
%
The redirection with prefix
|\childdocforwardprefix[|\textit{prefix}|]{|\textit{dest}|}|
is accomplished by:
%
\begin{center}
\begin{tabular}{l}
|{\edef\jobname{\scantokens\expandafter{\jobname\noexpand}}|\\
|\def\redirectjob |\textit{prefix}|#1~~~{\gdef\jobname{|\textit{dest}|#1}}|\\
|\expandafter\redirectjob\jobname~~~}\input{\jobname}|
\end{tabular}
\end{center}

In an alternative approach,
child documents can be compiled by a specific command line
without additional code or specific definitions:
%
\begin{center}
|... -jobname "|\textit{target}|" "|[\textit{flags}]%
|\includeonly{|\textit{dest}|}\input{|\textit{main}|}"|
\end{center}
%

%%%%%%%%%%%%%%%%%%%%%%%%%%%%%%%%%%%%%%%%%%%%%%%%%%%%%%%%%%%%%%%%%%%%%%%%%%%%%%%%
%%%%%%%%%%%%%%%%%%%%%%%%%%%%%%%%%%%%%%%%%%%%%%%%%%%%%%%%%%%%%%%%%%%%%%%%%%%%%%%%
\section{Information}

%%%%%%%%%%%%%%%%%%%%%%%%%%%%%%%%%%%%%%%%%%%%%%%%%%%%%%%%%%%%%%%%%%%%%%%%%%%%%%%%
\subsection{Copyright}

Copyright \copyright{} 2017--2018 Niklas Beisert

This work may be distributed and/or modified under the
conditions of the \LaTeX{} Project Public License, either version 1.3
of this license or (at your option) any later version.
The latest version of this license is in
  \url{http://www.latex-project.org/lppl.txt}
and version 1.3 or later is part of all distributions of \LaTeX{}
version 2005/12/01 or later.

This work has the LPPL maintenance status `maintained'.

The Current Maintainer of this work is Niklas Beisert.

This work consists of the files |README.txt|, |childdoc.ins| and |childdoc.dtx|
as well as the derived files |childdoc.def|, |cdocsamp.tex|
with |cdocsch1.tex|, |cdocsch2.tex|, |cdocspt3.tex|, |cdocspt4.tex|,
|cdocsdrf.tex|, |cdocsfn1.tex|, |cdocsfn2.tex|
as well as |childdoc.pdf|.

%%%%%%%%%%%%%%%%%%%%%%%%%%%%%%%%%%%%%%%%%%%%%%%%%%%%%%%%%%%%%%%%%%%%%%%%%%%%%%%%
\subsection{Files and Installation}

The package consists of the files:
%
\begin{center}
\begin{tabular}{ll}
    |README.txt|   & readme file \\
    |childdoc.ins| & installation file \\
    |childdoc.dtx| & source file \\
    |childdoc.def| & definition file \\
    |cdocsamp.tex| & sample main file \\
    |cdocsch1.tex| & sample include file \\
    |cdocsch2.tex| & sample include file \\
    |cdocspt3.tex| & sample part file \\
    |cdocspt4.tex| & sample part file \\
    |cdocsdrf.tex| & sample redirection file \\
    |cdocsfn1.tex| & sample redirection file \\
    |cdocsfn2.tex| & sample redirection file \\
    |childdoc.pdf| & manual
\end{tabular}
\end{center}
%
The distribution consists of the files
|README.txt|, |childdoc.ins| and |childdoc.dtx|.
%
\begin{itemize}
\item
Run (pdf)\LaTeX{} on |childdoc.dtx|
to compile the manual |childdoc.pdf| (this file).
\item
Run \LaTeX{} on |childdoc.ins| to create the definitions file |childdoc.def|
and the sample |cdocsamp.tex| with include files
|cdocsch1.tex|, |cdocsch2.tex|, |cdocspt3.tex|, |cdocspt4.tex|,
|cdocsdrf.tex|, |cdocsfn1.tex|, |cdocsfn2.tex|.
Then copy the file |childdoc.def| to an appropriate directory of your \LaTeX{}
distribution, e.g.\ \textit{texmf-root}|/tex/latex/childdoc|.
\end{itemize}

%%%%%%%%%%%%%%%%%%%%%%%%%%%%%%%%%%%%%%%%%%%%%%%%%%%%%%%%%%%%%%%%%%%%%%%%%%%%%%%%
\subsection{Related CTAN Packages}

There are several other packages which offer a similar functionality:
%
\begin{itemize}
\item
The packages
\href{http://ctan.org/pkg/docmute}{\textsf{docmute}},
\href{http://ctan.org/pkg/includex}{\textsf{includex}} and
\href{http://ctan.org/pkg/standalone}{\textsf{standalone}}
provide commands to include only the document body of
a child file thus allowing both files to be compiled individually.
\item
The packages \href{http://ctan.org/pkg/subdocs}{\textsf{subdocs}}
and \href{http://ctan.org/pkg/subfiles}{\textsf{subfiles}}
provide structures in which the main and child documents can be
encapsulated and allowing them to be compiled individually.
The inclusion mechanism is different from the conventional |\include|.
\item
The package \href{http://ctan.org/pkg/combine}{\textsf{combine}}
is an elaborate solution to combine several documents into one.
\end{itemize}
%
See also the CTAN topic \href{http://ctan.org/topic/subdocs}{\textsf{subdocs}}
for further related packages.
The present package differs from the above solutions in that
a document structure constructed with the conventional |\include| mechanism
just needs two extra commands at the top of every file
such that all constituent files can be compiled individually.

%%%%%%%%%%%%%%%%%%%%%%%%%%%%%%%%%%%%%%%%%%%%%%%%%%%%%%%%%%%%%%%%%%%%%%%%%%%%%%%%
%\subsection{Feature Suggestions}
%
%The following is a list of features which may be useful for future
%versions of this package:
%%
%\begin{itemize}
%\item
%\ldots
%\end{itemize}

%%%%%%%%%%%%%%%%%%%%%%%%%%%%%%%%%%%%%%%%%%%%%%%%%%%%%%%%%%%%%%%%%%%%%%%%%%%%%%%%
\subsection{Revision History}

%%%%%%%%%%%%%%%%%%%%%%%%%%%%%%%%%%%%%%%%
\paragraph{v2.0:} 2018/12/30

\begin{itemize}
\item
immediate forward processing
\item
added |\childdocby| mechanism
\item
manual restructured
\end{itemize}

%%%%%%%%%%%%%%%%%%%%%%%%%%%%%%%%%%%%%%%%
\paragraph{v1.6:} 2018/01/17

\begin{itemize}
\item
application for development of include files
\item
corrections to manual
\end{itemize}

%%%%%%%%%%%%%%%%%%%%%%%%%%%%%%%%%%%%%%%%
\paragraph{v1.5:} 2017/05/21

\begin{itemize}
\item
more complete structuring introduced
\item
|\childdocof| introduced
\item
|\childdoc| renamed to |\childdocmain|
\item
|\childredirect| renamed to |\childdocforward| and |\childdocforwardprefix|
and functionality expanded
\end{itemize}

%%%%%%%%%%%%%%%%%%%%%%%%%%%%%%%%%%%%%%%%
\paragraph{v1.0:} 2017/04/27

\begin{itemize}
\item
manual and install package
\item
first version published on CTAN
\end{itemize}

%%%%%%%%%%%%%%%%%%%%%%%%%%%%%%%%%%%%%%%%
\paragraph{v0.6:} 2017/04/26

\begin{itemize}
\item
redirection mechanism added
\end{itemize}

%%%%%%%%%%%%%%%%%%%%%%%%%%%%%%%%%%%%%%%%
\paragraph{v0.5:} 2017/04/26

\begin{itemize}
\item
functionality in definition file
\end{itemize}


%%%%%%%%%%%%%%%%%%%%%%%%%%%%%%%%%%%%%%%%%%%%%%%%%%%%%%%%%%%%%%%%%%%%%%%%%%%%%%%%
%%%%%%%%%%%%%%%%%%%%%%%%%%%%%%%%%%%%%%%%%%%%%%%%%%%%%%%%%%%%%%%%%%%%%%%%%%%%%%%%
%%%%%%%%%%%%%%%%%%%%%%%%%%%%%%%%%%%%%%%%%%%%%%%%%%%%%%%%%%%%%%%%%%%%%%%%%%%%%%%%
\appendix

\settowidth\MacroIndent{\rmfamily\scriptsize 000\ }

 \DocInput{childdoc.dtx}

\end{document}
%</driver>
% \fi
%
% %%%%%%%%%%%%%%%%%%%%%%%%%%%%%%%%%%%%%%%%%%%%%%%%%%%%%%%%%%%%%%%%%%%%%%%%%%%%%%
% %%%%%%%%%%%%%%%%%%%%%%%%%%%%%%%%%%%%%%%%%%%%%%%%%%%%%%%%%%%%%%%%%%%%%%%%%%%%%%
% \section{Sample}
%\iffalse
%<*samplemain>
%\fi
%
% The following presents a sample document
% with two chapters, two parts, a title page,
% a compile flag as well as three forwarding files to set the flag.
% It consists of eight |.tex| files:
% \begin{center}
% \begin{tabular}{ll}
% |cdocsamp.tex|&main file\\
% |cdocsch1.tex|&include file for chapter 1\\
% |cdocsch2.tex|&include file for chapter 2\\
% |cdocspt3.tex|&include file for part 3\\
% |cdocspt4.tex|&include file for part 4\\
% |cdocsdrf.tex|&forwarding file for main file in draft mode\\
% |cdocsfi1.tex|&forwarding file for final version of chapter 1\\
% |cdocsfi2.tex|&forwarding file for final version of chapter 2\\
% \end{tabular}
% \end{center}
% Each of the eight files can be compiled directly by the \LaTeX{} compiler.
%
% %%%%%%%%%%%%%%%%%%%%%%%%%%%%%%%%%%%%%%
% \paragraph{Main File.}
%
% The main file is called |cdocsamp.tex|.
%
% Load the \textsf{childdoc} definitions and
% declare the filename for the main document:
%    \begin{macrocode}
\input{childdoc.def}
\childdocmain{}
%    \end{macrocode}

% Optional override for |\version| flag:
%    \begin{macrocode}
%%\ifchilddoc\else\providecommand{\version}{draft}\fi
%    \end{macrocode}

% Define the default values for the |\version| flag
% (|final| for the main file and |draft| for childs):
%    \begin{macrocode}
\ifchilddoc
\providecommand{\version}{draft}
\else
\providecommand{\version}{final}
\fi
%    \end{macrocode}

% Load the standard document class:
%    \begin{macrocode}
\documentclass[12pt]{article}
%    \end{macrocode}

% Start the document body:
%    \begin{macrocode}
\begin{document}
%    \end{macrocode}

% Declare a title page.
% Print title, part of document being processed and version flag:
%    \begin{macrocode}
\addtocounter{page}{-1}
\begin{center}
{\LARGE\bfseries{}childdoc example\par}
\vspace{1cm}
\ifchilddoc
\ifchilddocmanual part\else chapter\fi:
`\childdocname' of `\childdocjob'\par
\else
main document: `\childdocjob'\par
\fi
version: \version\par
\end{center}
\newpage
%    \end{macrocode}

% Manually include selected file,
% otherwise process as usual:
%    \begin{macrocode}
\ifchilddocmanual
\section*{part `\childdocname'}
\input{\childdocname}
\else
%    \end{macrocode}

% Include the two chapters:
%    \begin{macrocode}
\include{cdocsch1}
\include{cdocsch2}
%    \end{macrocode}

% Include the two parts unless only chapters should be displayed:
%    \begin{macrocode}
\ifchilddoc\else
\section{part three}
\input{cdocspt3}
\section{part four}
\input{cdocspt4}
\fi
%    \end{macrocode}

% Process as usual until here:
%    \begin{macrocode}
\fi
%    \end{macrocode}

% End of document body:
%    \begin{macrocode}
\end{document}
%    \end{macrocode}
%\iffalse
%</samplemain>
%\fi
%
% %%%%%%%%%%%%%%%%%%%%%%%%%%%%%%%%%%%%%%
% \paragraph{Chapter Include Files.}
%
% The include files are called |cdocsch1.tex| and |cdocsch2.tex|.
%
%\iffalse
%<*samplechap1|samplechap2>
%\fi

% Optional override for |\version| flag:
%    \begin{macrocode}
%%\providecommand{\version}{final}
%    \end{macrocode}

% Include the main document:
%    \begin{macrocode}
\input{childdoc.def}
\childdocof{cdocsamp}
%    \end{macrocode}

%\iffalse
%</samplechap1|samplechap2>
%\fi
%
%\iffalse
%<*samplechap1>
%\fi
% Some text for chapter 1:
%    \begin{macrocode}
\section{one}
some text in chapter one
%    \end{macrocode}

%\iffalse
%</samplechap1>
%\fi
% Some text for chapter 2:
%\iffalse
%<*samplechap2>
%\fi
%    \begin{macrocode}
\section{two}
more text in chapter two
%    \end{macrocode}

%\iffalse
%</samplechap2>
%\fi
%
% %%%%%%%%%%%%%%%%%%%%%%%%%%%%%%%%%%%%%%
% \paragraph{Part Include Files.}
%
% The include files are called |cdocspt3.tex| and |cdocspt4.tex|.
%
%\iffalse
%<*samplepart3|samplepart4>
%\fi

% Optional override for |\version| flag:
%    \begin{macrocode}
%%\providecommand{\version}{final}
%    \end{macrocode}

% Include the main document:
%    \begin{macrocode}
\input{childdoc.def}
\childdocby{cdocsamp}
%    \end{macrocode}

%\iffalse
%</samplepart3|samplepart4>
%\fi
%
%\iffalse
%<*samplepart3>
%\fi
% Some text for part 3:
%    \begin{macrocode}
some text in part three
%    \end{macrocode}

%\iffalse
%</samplepart3>
%\fi
% Some text for part 4:
%\iffalse
%<*samplepart4>
%\fi
%    \begin{macrocode}
more text in part four
%    \end{macrocode}

%\iffalse
%</samplepart4>
%\fi
%
% %%%%%%%%%%%%%%%%%%%%%%%%%%%%%%%%%%%%%%
% \paragraph{Forwarding for a Complete Draft.}
%
% The following forwarding file |cdocsdrf.tex|
% compiles the main document in draft mode:
%\iffalse
%<*sampledraft>
%\fi
%    \begin{macrocode}
\def\version{draft}
\input{childdoc.def}
\childdocforward{cdocsamp}
%    \end{macrocode}

%\iffalse
%</sampledraft>
%\fi
%
% %%%%%%%%%%%%%%%%%%%%%%%%%%%%%%%%%%%%%%
% \paragraph{Forwarding for Final Version of the Chapters.}
%
% The following forwarding files |cdocsfn1.tex| and |cdocsfn2.tex|
% (with identical content)
% compile the final versions of the child documents
% |cdocsch1.tex| and |cdocsch2.tex|, respectively:
%\iffalse
%<*samplefinal>
%\fi
%    \begin{macrocode}
\def\version{final}
\input{childdoc.def}
\childdocforwardprefix[cdocsamp]{cdocsfn}{cdocsch}
%    \end{macrocode}

%\iffalse
%</samplefinal>
%\fi
%
% %%%%%%%%%%%%%%%%%%%%%%%%%%%%%%%%%%%%%%
% \paragraph{Command Line Processing.}
%
% The following three command lines generate the output files
% |cdocscld|, |cdocscl1| and |cdocscl2|
% which should be identical to
% |cdocsdrf|, |cdocsch1| and |cdocsfn2|, respectively:
% \begin{center}
% \begin{tabular}{l}
% |latex -jobname cdocscld \|\\
% |  "\def\version{draft}\input{childdoc.def}\childdocforward{cdocsamp}"|\\
% |latex -jobname cdocscl1 \|\\
% |  "\input{childdoc.def}\childdocforward[cdocsamp]{cdocsch1}"|\\
% |latex -jobname cdocscl2 \|\\
% |  "\def\version{final}\input{childdoc.def}\childdocforward{cdocsch2}"|
% \end{tabular}
% \end{center}
% Note that the trailing backslash on each first line
% merely continues the input to the second line
% (for convenient cut ant paste).
% Furthermore, the command |latex| can be replaced by any
% of its alternative versions such as |pdflatex|.
%
% %%%%%%%%%%%%%%%%%%%%%%%%%%%%%%%%%%%%%%%%%%%%%%%%%%%%%%%%%%%%%%%%%%%%%%%%%%%%%%
% %%%%%%%%%%%%%%%%%%%%%%%%%%%%%%%%%%%%%%%%%%%%%%%%%%%%%%%%%%%%%%%%%%%%%%%%%%%%%%
% \section{Implementation}
%\iffalse
%<*package>
%\fi
%
% This section describes the definitions file |childdoc.def|.

% The definitions cannot be loaded using |\usepackage| or |\RequirePackage|
% which has a mechanism to prevent loading a style file more than once.
% When loading the definitions by means of |\input|
% multiple instances have to be prevented manually:
%\iffalse
%This code needs to be before the `\ProvidesFile' directive
%which is defined at the beginning of this file.
%Therefore it is also placed there and commented out here.
%</package>
%<*discard>
%\fi
%    \begin{macrocode}
\ifdefined\childdocmain\endinput\fi
%    \end{macrocode}
%\iffalse
%</discard>
%<*package>
%\fi
%
% \macro{\ifchilddoc}
% \macro{\ifchilddocmanual}
% The conditional |\ifchilddoc| tells whether a
% child (true) or main (false) document is being compiled.
% The conditional |\ifchilddocmanual| tells whether
% the |\includeonly| mechanism is used (false) or
% the selection of child files must be performed manually (true).
% The definitions initialise to false:
%    \begin{macrocode}
\newif\ifchilddoc
\newif\ifchilddocmanual
%    \end{macrocode}

% \macro{\childdocname}
% \macro{\childdocjob}
% The macro |\childdocname| stores the name of the main document
% to be compiled. The macro |\childdocjob| stores the name of
% the document on which the \LaTeX{} compiler was originally invoked.
% The content of |\jobname| cannot be compared
% to filenames specified in the source due to different catcodes.
% The following code rescans |\jobname|, stores the result
% in |\childdocname| and saves a copy in |\childdocjob|:
%    \begin{macrocode}
\edef\childdocname{\scantokens\expandafter{\jobname\noexpand}}
\let\childdocjob\childdocname
%    \end{macrocode}

% \macro{\childdocdisable}
% The macro |\childdocdisable| prevents the main file
% from being processed more than once.
% At this stage, the main document command |\childdocmain|
% is assumed to be called once again where it should do nothing.
% Any subsequent call to it should prevent
% a secondary processing of the main document
% It overwrites the forwarding commands
% |\childdocof| and |\childdocforward|
% with empty macros to prevent further inclusions of the main document:
%    \begin{macrocode}
\newcommand{\childdocdisable}
{
  \renewcommand{\childdocmain}[1]{\renewcommand{\childdocmain}[1]{\endinput}}
  \renewcommand{\childdocof}[1]{}
  \renewcommand{\childdocby}[2][]{}
  \renewcommand{\childdocforward}[2][]{}
  \renewcommand{\childdocdisable}{}
}
%    \end{macrocode}

% \macro{\childdocmain}
% The macro |\childdocmain| is to be called at the top of the main file
% with nothing or the main filename (without extension) as argument.
% First, it breaks loops.
% If the argument is not empty and does not match |\childdocname|
% (which is set by the first inclusion of |childdoc.def|),
% |\ifchilddoc| is set to true, |\includeonly| is applied to the child file
% and |\jobname| is set to the main file
% (for proper handling of |.aux| files):
%    \begin{macrocode}
\newcommand{\childdocmain}[1]
{
  \childdocdisable\childdocmain{}
  \if?#1?\else
    \begingroup
      \def\childdoctmp{#1}
      \ifx\childdoctmp\childdocname
        \def\childdoctmp{}
      \else
        \def\childdoctmp
        {
          \childdoctrue
          \includeonly{\childdocname}
          \def\childdocjob{#1}
          \def\jobname{#1}
        }
      \fi
      \expandafter
    \endgroup
    \childdoctmp
  \fi
}
%    \end{macrocode}

% \macro{\childdocof}
% The command |\childdocof| redirects
% compilation to the main file |#1|.
%    \begin{macrocode}
\newcommand{\childdocof}[1]
{
  \childdocdisable
  \childdoctrue
  \includeonly{\childdocname}
  \def\jobname{#1}
  \def\childdocjob{#1}
  \input{#1}
}
%    \end{macrocode}

% \macro{\childdocby}
% The command |\childdocby| ....
%    \begin{macrocode}
\newcommand{\childdocby}[2][]
{
  \childdocdisable
  \childdoctrue
  \childdocmanualtrue
  \if?#1?\else
    \def\jobname{#2}
  \fi
  \def\childdocjob{#2}
  \input{#2}
  \endinput
}
%    \end{macrocode}

% \macro{\childdocforward}
% The command |\childdocforward| redirects
% compilation to the main file or
% (if the optional argument is given) a child file.
% Parameters are set as if the main file
% or a child file starting with |\childdocof| was compiled.
% Then compilation is handed over to the main file:
%    \begin{macrocode}
\newcommand{\childdocforward}[2][]
{
  \begingroup
    \if?#1?
      \def\childdoctmp
      {
        \def\childdocname{#2}
        \def\childdocjob{#2}
        \def\jobname{#2}
        \input{#2}
        \endinput
      }
    \else
      \def\childdoctmp
      {
        \childdocdisable
        \def\childdocname{#2}
        \childdoctrue
        \includeonly{#2}
        \def\childdocjob{#1}
        \def\jobname{#1}
        \input{#1}
        \endinput
      }
    \fi
    \expandafter
  \endgroup
  \childdoctmp
}
%    \end{macrocode}

% \macro{\childdocforwardprefix}
% The command |\childdocforwardprefix| redirects
% compilation to the main or a child file by means of a pattern.
% The prefix |#1| in the current filename is replaced by |#2|
% and the suffix of the current filename is kept
% (it is assumed that the filename does not contain the substring `|~~~|'
% which is used as a delimiter).
% Compilation is handed over to the new file by |\childdocforward|:
%    \begin{macrocode}
\newcommand{\childdocforwardprefix}[3][]
{
  \begingroup
    \def\childdocextract #2##1~~~{\def\childdoctmp{\childdocforward[#1]{#3##1}}}
    \expandafter\childdocextract\childdocname~~~
    \expandafter
  \endgroup
  \childdoctmp
}
%    \end{macrocode}

% \macro{\childdoc}
% The deprecated macro |\childdoc| is a legacy version of |\childdocmain|:
%    \begin{macrocode}
\newcommand{\childdoc}{\childdocmain}
%    \end{macrocode}

% \macro{\childdocredirect}
% The deprecated macro |\childdocredirect| is a legacy version
% of |\childdocforward| and |\childdocforwardprefix|:
%    \begin{macrocode}
\newcommand{\childdocredirect}[2][]
{
  \begingroup
    \if?#1?
      \def\childdoctmp{\childdocforward{#2}}
    \else
      \def\childdoctmp{\childdocforwardprefix{#1}{#2}}
    \fi
    \expandafter
  \endgroup
  \childdoctmp
}
%    \end{macrocode}

%\iffalse
%</package>
%\fi
%
\endinput
|
and perform the replacements as outlined below.
Instead of |\childdocmain{|\textit{main}|}| add the following code
to the top of the main file:
%
\begin{center}
\begin{tabular}{l}
|\||ifdefined\childdocname\endinput\||fi\newif\ifchilddoc|\\
|\edef\childdocname{\scantokens\expandafter{\jobname\noexpand}}|\\
|\def\childdocmain{|\textit{main}|}\||ifx\childdocmain\childdocname\||else|\\
|\childdoctrue\includeonly{\childdocname}\let\jobname\childdocmain\||fi|\\
\end{tabular}
\end{center}
%
Instead of |\childdocof{|\textit{main}|}| just include the main file
at the top of each child file:
%
\begin{center}
|\input{|\textit{main}|}|
\end{center}
%
A simple redirection |\childdocforward{|\textit{dest}|}| is achieved by:
%
\begin{center}
|\def\jobname{|\textit{dest}|}\input{\jobname}|
\end{center}
%
The redirection with prefix
|\childdocforwardprefix[|\textit{prefix}|]{|\textit{dest}|}|
is accomplished by:
%
\begin{center}
\begin{tabular}{l}
|{\edef\jobname{\scantokens\expandafter{\jobname\noexpand}}|\\
|\def\redirectjob |\textit{prefix}|#1~~~{\gdef\jobname{|\textit{dest}|#1}}|\\
|\expandafter\redirectjob\jobname~~~}\input{\jobname}|
\end{tabular}
\end{center}

In an alternative approach,
child documents can be compiled by a specific command line
without additional code or specific definitions:
%
\begin{center}
|... -jobname "|\textit{target}|" "|[\textit{flags}]%
|\includeonly{|\textit{dest}|}\input{|\textit{main}|}"|
\end{center}
%

%%%%%%%%%%%%%%%%%%%%%%%%%%%%%%%%%%%%%%%%%%%%%%%%%%%%%%%%%%%%%%%%%%%%%%%%%%%%%%%%
%%%%%%%%%%%%%%%%%%%%%%%%%%%%%%%%%%%%%%%%%%%%%%%%%%%%%%%%%%%%%%%%%%%%%%%%%%%%%%%%
\section{Information}

%%%%%%%%%%%%%%%%%%%%%%%%%%%%%%%%%%%%%%%%%%%%%%%%%%%%%%%%%%%%%%%%%%%%%%%%%%%%%%%%
\subsection{Copyright}

Copyright \copyright{} 2017--2018 Niklas Beisert

This work may be distributed and/or modified under the
conditions of the \LaTeX{} Project Public License, either version 1.3
of this license or (at your option) any later version.
The latest version of this license is in
  \url{http://www.latex-project.org/lppl.txt}
and version 1.3 or later is part of all distributions of \LaTeX{}
version 2005/12/01 or later.

This work has the LPPL maintenance status `maintained'.

The Current Maintainer of this work is Niklas Beisert.

This work consists of the files |README.txt|, |childdoc.ins| and |childdoc.dtx|
as well as the derived files |childdoc.def|, |cdocsamp.tex|
with |cdocsch1.tex|, |cdocsch2.tex|, |cdocspt3.tex|, |cdocspt4.tex|,
|cdocsdrf.tex|, |cdocsfn1.tex|, |cdocsfn2.tex|
as well as |childdoc.pdf|.

%%%%%%%%%%%%%%%%%%%%%%%%%%%%%%%%%%%%%%%%%%%%%%%%%%%%%%%%%%%%%%%%%%%%%%%%%%%%%%%%
\subsection{Files and Installation}

The package consists of the files:
%
\begin{center}
\begin{tabular}{ll}
    |README.txt|   & readme file \\
    |childdoc.ins| & installation file \\
    |childdoc.dtx| & source file \\
    |childdoc.def| & definition file \\
    |cdocsamp.tex| & sample main file \\
    |cdocsch1.tex| & sample include file \\
    |cdocsch2.tex| & sample include file \\
    |cdocspt3.tex| & sample part file \\
    |cdocspt4.tex| & sample part file \\
    |cdocsdrf.tex| & sample redirection file \\
    |cdocsfn1.tex| & sample redirection file \\
    |cdocsfn2.tex| & sample redirection file \\
    |childdoc.pdf| & manual
\end{tabular}
\end{center}
%
The distribution consists of the files
|README.txt|, |childdoc.ins| and |childdoc.dtx|.
%
\begin{itemize}
\item
Run (pdf)\LaTeX{} on |childdoc.dtx|
to compile the manual |childdoc.pdf| (this file).
\item
Run \LaTeX{} on |childdoc.ins| to create the definitions file |childdoc.def|
and the sample |cdocsamp.tex| with include files
|cdocsch1.tex|, |cdocsch2.tex|, |cdocspt3.tex|, |cdocspt4.tex|,
|cdocsdrf.tex|, |cdocsfn1.tex|, |cdocsfn2.tex|.
Then copy the file |childdoc.def| to an appropriate directory of your \LaTeX{}
distribution, e.g.\ \textit{texmf-root}|/tex/latex/childdoc|.
\end{itemize}

%%%%%%%%%%%%%%%%%%%%%%%%%%%%%%%%%%%%%%%%%%%%%%%%%%%%%%%%%%%%%%%%%%%%%%%%%%%%%%%%
\subsection{Related CTAN Packages}

There are several other packages which offer a similar functionality:
%
\begin{itemize}
\item
The packages
\href{http://ctan.org/pkg/docmute}{\textsf{docmute}},
\href{http://ctan.org/pkg/includex}{\textsf{includex}} and
\href{http://ctan.org/pkg/standalone}{\textsf{standalone}}
provide commands to include only the document body of
a child file thus allowing both files to be compiled individually.
\item
The packages \href{http://ctan.org/pkg/subdocs}{\textsf{subdocs}}
and \href{http://ctan.org/pkg/subfiles}{\textsf{subfiles}}
provide structures in which the main and child documents can be
encapsulated and allowing them to be compiled individually.
The inclusion mechanism is different from the conventional |\include|.
\item
The package \href{http://ctan.org/pkg/combine}{\textsf{combine}}
is an elaborate solution to combine several documents into one.
\end{itemize}
%
See also the CTAN topic \href{http://ctan.org/topic/subdocs}{\textsf{subdocs}}
for further related packages.
The present package differs from the above solutions in that
a document structure constructed with the conventional |\include| mechanism
just needs two extra commands at the top of every file
such that all constituent files can be compiled individually.

%%%%%%%%%%%%%%%%%%%%%%%%%%%%%%%%%%%%%%%%%%%%%%%%%%%%%%%%%%%%%%%%%%%%%%%%%%%%%%%%
%\subsection{Feature Suggestions}
%
%The following is a list of features which may be useful for future
%versions of this package:
%%
%\begin{itemize}
%\item
%\ldots
%\end{itemize}

%%%%%%%%%%%%%%%%%%%%%%%%%%%%%%%%%%%%%%%%%%%%%%%%%%%%%%%%%%%%%%%%%%%%%%%%%%%%%%%%
\subsection{Revision History}

%%%%%%%%%%%%%%%%%%%%%%%%%%%%%%%%%%%%%%%%
\paragraph{v2.0:} 2018/12/30

\begin{itemize}
\item
immediate forward processing
\item
added |\childdocby| mechanism
\item
manual restructured
\end{itemize}

%%%%%%%%%%%%%%%%%%%%%%%%%%%%%%%%%%%%%%%%
\paragraph{v1.6:} 2018/01/17

\begin{itemize}
\item
application for development of include files
\item
corrections to manual
\end{itemize}

%%%%%%%%%%%%%%%%%%%%%%%%%%%%%%%%%%%%%%%%
\paragraph{v1.5:} 2017/05/21

\begin{itemize}
\item
more complete structuring introduced
\item
|\childdocof| introduced
\item
|\childdoc| renamed to |\childdocmain|
\item
|\childredirect| renamed to |\childdocforward| and |\childdocforwardprefix|
and functionality expanded
\end{itemize}

%%%%%%%%%%%%%%%%%%%%%%%%%%%%%%%%%%%%%%%%
\paragraph{v1.0:} 2017/04/27

\begin{itemize}
\item
manual and install package
\item
first version published on CTAN
\end{itemize}

%%%%%%%%%%%%%%%%%%%%%%%%%%%%%%%%%%%%%%%%
\paragraph{v0.6:} 2017/04/26

\begin{itemize}
\item
redirection mechanism added
\end{itemize}

%%%%%%%%%%%%%%%%%%%%%%%%%%%%%%%%%%%%%%%%
\paragraph{v0.5:} 2017/04/26

\begin{itemize}
\item
functionality in definition file
\end{itemize}


%%%%%%%%%%%%%%%%%%%%%%%%%%%%%%%%%%%%%%%%%%%%%%%%%%%%%%%%%%%%%%%%%%%%%%%%%%%%%%%%
%%%%%%%%%%%%%%%%%%%%%%%%%%%%%%%%%%%%%%%%%%%%%%%%%%%%%%%%%%%%%%%%%%%%%%%%%%%%%%%%
%%%%%%%%%%%%%%%%%%%%%%%%%%%%%%%%%%%%%%%%%%%%%%%%%%%%%%%%%%%%%%%%%%%%%%%%%%%%%%%%
\appendix

\settowidth\MacroIndent{\rmfamily\scriptsize 000\ }

 \DocInput{childdoc.dtx}

\end{document}
%</driver>
% \fi
%
% %%%%%%%%%%%%%%%%%%%%%%%%%%%%%%%%%%%%%%%%%%%%%%%%%%%%%%%%%%%%%%%%%%%%%%%%%%%%%%
% %%%%%%%%%%%%%%%%%%%%%%%%%%%%%%%%%%%%%%%%%%%%%%%%%%%%%%%%%%%%%%%%%%%%%%%%%%%%%%
% \section{Sample}
%\iffalse
%<*samplemain>
%\fi
%
% The following presents a sample document
% with two chapters, two parts, a title page,
% a compile flag as well as three forwarding files to set the flag.
% It consists of eight |.tex| files:
% \begin{center}
% \begin{tabular}{ll}
% |cdocsamp.tex|&main file\\
% |cdocsch1.tex|&include file for chapter 1\\
% |cdocsch2.tex|&include file for chapter 2\\
% |cdocspt3.tex|&include file for part 3\\
% |cdocspt4.tex|&include file for part 4\\
% |cdocsdrf.tex|&forwarding file for main file in draft mode\\
% |cdocsfi1.tex|&forwarding file for final version of chapter 1\\
% |cdocsfi2.tex|&forwarding file for final version of chapter 2\\
% \end{tabular}
% \end{center}
% Each of the eight files can be compiled directly by the \LaTeX{} compiler.
%
% %%%%%%%%%%%%%%%%%%%%%%%%%%%%%%%%%%%%%%
% \paragraph{Main File.}
%
% The main file is called |cdocsamp.tex|.
%
% Load the \textsf{childdoc} definitions and
% declare the filename for the main document:
%    \begin{macrocode}
% \iffalse
%
% childdoc.dtx Copyright (C) 2017-2018 Niklas Beisert
%
% This work may be distributed and/or modified under the
% conditions of the LaTeX Project Public License, either version 1.3
% of this license or (at your option) any later version.
% The latest version of this license is in
%   http://www.latex-project.org/lppl.txt
% and version 1.3 or later is part of all distributions of LaTeX
% version 2005/12/01 or later.
%
% This work has the LPPL maintenance status `maintained'.
%
% The Current Maintainer of this work is Niklas Beisert.
%
% This work consists of the files childdoc.dtx and childdoc.ins
% and the derived files childdoc.def and cdocsamp.tex with
% cdocsch1.tex, cdocsch2.tex, cdocsdrf.tex, cdocsfn1.tex, cdocsfn2.tex.
%
%<package>\ifdefined\childdocmain\endinput\fi
%<package>\ProvidesFile{childdoc.def}[2018/12/30 v2.0 child document driver]
%<samplemain>\ProvidesFile{cdocsamp.tex}[2018/12/30 v2.0 sample for childdoc]
%<*driver>
%\ProvidesFile{childdoc.drv}[2018/12/30 v2.0 childdoc reference manual file]
\PassOptionsToClass{10pt,a4paper}{article}
\documentclass{ltxdoc}

\usepackage[margin=35mm]{geometry}
\usepackage{hyperref}
\usepackage{hyperxmp}
\usepackage[usenames]{color}

\hypersetup{colorlinks=true}
\hypersetup{pdfstartview=FitH}
\hypersetup{pdfpagemode=UseNone}
\hypersetup{pdfsource={}}
\hypersetup{pdflang={en-UK}}
\hypersetup{pdfcopyright={Copyright 2017-2018 Niklas Beisert.
  This work may be distributed and/or modified under the
  conditions of the LaTeX Project Public License, either version 1.3
  of this license or (at your option) any later version.}}
\hypersetup{pdflicenseurl={http://www.latex-project.org/lppl.txt}}
\hypersetup{pdfcontactaddress={ETH Zurich, ITP, HIT K,
  Wolfgang-Pauli-Strasse 27}}
\hypersetup{pdfcontactpostcode={8093}}
\hypersetup{pdfcontactcity={Zurich}}
\hypersetup{pdfcontactcountry={Switzerland}}
\hypersetup{pdfcontactemail={nbeisert@itp.phys.ethz.ch}}
\hypersetup{pdfcontacturl={http://people.phys.ethz.ch/\xmptilde nbeisert/}}

\newcommand{\secref}[1]{\hyperref[#1]{section \ref*{#1}}}

\parskip1ex
\parindent0pt
\let\olditemize\itemize
\def\itemize{\olditemize\parskip0pt}

\begin{document}

\title{The \textsf{childdoc} Package}
\hypersetup{pdftitle={The childdoc Package}}
\author{Niklas Beisert\\[2ex]
  Institut f\"ur Theoretische Physik\\
  Eidgen\"ossische Technische Hochschule Z\"urich\\
  Wolfgang-Pauli-Strasse 27, 8093 Z\"urich, Switzerland\\[1ex]
  \href{mailto:nbeisert@itp.phys.ethz.ch}
  {\texttt{nbeisert@itp.phys.ethz.ch}}}
\hypersetup{pdfauthor={Niklas Beisert}}
\hypersetup{pdfsubject={Manual for the LaTeX2e Package childdoc}}
\date{30 December 2018, \textsf{v2.0}}
\maketitle

\begin{abstract}\noindent
\textsf{childdoc} is a \LaTeXe{} package
that enables the direct compilation
of document sections included by |\include|
to individual files.
\end{abstract}

\begingroup
\parskip0ex
\tableofcontents
\endgroup

%%%%%%%%%%%%%%%%%%%%%%%%%%%%%%%%%%%%%%%%%%%%%%%%%%%%%%%%%%%%%%%%%%%%%%%%%%%%%%%%
%%%%%%%%%%%%%%%%%%%%%%%%%%%%%%%%%%%%%%%%%%%%%%%%%%%%%%%%%%%%%%%%%%%%%%%%%%%%%%%%
\section{Introduction}

\LaTeX{} provides a mechanism to structure a large document (such as a book)
into a main file and several child files (containing the chapters)
using the |\include| command.
This mechanism is beneficial for documents
which span hundreds of pages in order to
make the source file(s) more manageable.
Moreover, compilation can be restricted to
selected child files by means of the |\includeonly| command.
The latter feature can be used to reduce the compilation time while editing
(this was significantly more useful in the earlier days of \LaTeX{})
or to generate a smaller document which is easier to navigate.
Another application of |\includeonly| is to generate
documents consisting of selected parts of the complete document.

However, there are a few drawbacks of the plain |\include| mechanism:
\begin{itemize}
\item
The child files cannot be compiled on their own,
they can only be compiled via the main file.
A naive editing environment
(such as a text editor with an option
to have the current file processed by \LaTeX)
may require one to switch to the main file before compiling;
attempting to compile the child file produces errors.
\item
The main file must be modified (each time)
to adjust the |\includeonly| command
to the present needs. This easily leaves the main file in a messy state.
\item
The generated document will always carry the filename
of the main document. This is inconvenient if
several child files are to be compiled and
to be kept for distribution.
\end{itemize}

The present package provides a simple interface
to make child files individually compilable by \LaTeX{}.
Compiling a child file then has the same effect as compiling
the main file with an |\includeonly| command
to select the appropriate child.
Moreover the generated document will carry the name of the child
rather than the main file.
This resolves all three above issues.

This feature is meant to make the editing of books,
thesis documents and lecture notes somewhat more convenient.
However, the package can also be used efficiently for
composing a series of documents (such as exercise sheets)
which are typically distributed individually.
It then assists the author in generating the individual documents
(potentially in different versions)
as well as a document containing the collected series.
Another application is in developing style files
or other kinds of included material
where compilation of the style file could redirect
to a sample or test file.

%%%%%%%%%%%%%%%%%%%%%%%%%%%%%%%%%%%%%%%%%%%%%%%%%%%%%%%%%%%%%%%%%%%%%%%%%%%%%%%%
%%%%%%%%%%%%%%%%%%%%%%%%%%%%%%%%%%%%%%%%%%%%%%%%%%%%%%%%%%%%%%%%%%%%%%%%%%%%%%%%
\section{Usage}

First of all, the package \textsf{childdoc} is \emph{not} a standard
\LaTeXe{} |.sty| style file! Therefore it needs to be invoked in
a non-standard way.

%%%%%%%%%%%%%%%%%%%%%%%%%%%%%%%%%%%%%%%%%%%%%%%%%%%%%%%%%%%%%%%%%%%%%%%%%%%%%%%%
\subsection{Included Files}
\label{sec:include}

%%%%%%%%%%%%%%%%%%%%%%%%%%%%%%%%%%%%%%%%
\DescribeMacro{\childdocmain}
To use the package, add the commands
\begin{center}
\begin{tabular}{l}
|\input{childdoc.def}|\\
|\childdocmain{}|\\
\end{tabular}
\end{center}
at the very top of the main \LaTeX{} file,
in particular \emph{before} the |\documentclass| statement!
The argument of |\childdocmain| should be left empty
(but it must be present).

%%%%%%%%%%%%%%%%%%%%%%%%%%%%%%%%%%%%%%%%
\DescribeMacro{\childdocof}
Furthermore, add the commands
\begin{center}
\begin{tabular}{l}
|\input{childdoc.def}|\\
|\childdocof{|\textit{main}|}|\\
\end{tabular}
\end{center}
at the top of every child file \textit{child}
which is included by |\include{|\textit{child}|}|
from within the main file
(or at least for those files to be compiled individually).
The argument \textit{main} must be the filename of the main file.

There are a couple of
considerations in setting up the main and child documents:

%%%%%%%%%%%%%%%%%%%%%%%%%%%%%%%%%%%%%%%%
\paragraph{Restrictions.}

Please note the following restrictions:
\begin{itemize}
\item
|\childdocmain| must be called with one argument \textit{main}
to ensure compatibility with earlier version of the package.
It must either be empty (|\childdocmain{}|)
or precisely match the filename of the main file in which it is specified.
See \secref{sec:detection} for further information.
\item
The filename \textit{main} must be specified without the |.tex| extension.
\item
The filename \textit{main} is case sensitive
(even in case-insensitive file systems)
due to internal string comparison.
\item
The argument \textit{main} should be fully expanded, it cannot be a macro.
\item
Subdirectories and special characters should be avoided in filenames.
\item
The command |\childdocmain{|\textit{main}|}| must be followed by a whitespace.
It should not be followed immediately by another command
or by a comment mark `|%|'.
This is because the \TeX{} parser reads the token immediately following
the argument of |\childdocmain| and puts it
at the beginning of every child section;
however, a white\-space is ignored.
\end{itemize}

%%%%%%%%%%%%%%%%%%%%%%%%%%%%%%%%%%%%%%%%
\paragraph{Content of Main File.}

It is advisable to place all content in the child files included by |\include|.
Any output contained in the main file will appear in all child documents
unless suppressed manually;
it cannot be suppressed automatically by the |\includeonly| directive
and thus should normally be avoided.
A method to include some content in the main file
by means of conditional processing is described in \secref{sec:conditional}.

%%%%%%%%%%%%%%%%%%%%%%%%%%%%%%%%%%%%%%%%
\paragraph{Page Numbering.}

When only a part of the document is compiled,
the appropriate numbering of pages
(as well as other status parameters)
is determined from the |.aux| files.
The latter contain information from previous passes.
However this information needs to propagate through
all intermediate child documents.
Therefore the page numbering in child documents may well
be inconsistent until the complete document is compiled at least once.

A useful (if unconventional) way to always ensure a consistent
page numbering is to restart the numbering in each child document
and denote the pages by `\textit{child}|.|\textit{page}'
where \textit{child} represents the chapter/section number of the child file.
This can be achieved by the command
|\numberwithin{page}{|\textit{child}|}|
of the \textsf{amsmath} package
where \textit{child} can be |chapter| or |section|
depending on the chosen structuring.
Alternatively, one can modify the macro |\thepage| appropriately
and reset the counter |page| at the start of each child file.

%%%%%%%%%%%%%%%%%%%%%%%%%%%%%%%%%%%%%%%%%%%%%%%%%%%%%%%%%%%%%%%%%%%%%%%%%%%%%%%%
\subsection{Conditional Processing}
\label{sec:conditional}

The package provides a mechanism to compile different versions
of a document. To customise the versions further some conditional processing
can come in handy to distinguish which version is being compiled.
The package provides two macros to describe the compilation context:

%%%%%%%%%%%%%%%%%%%%%%%%%%%%%%%%%%%%%%%%
\DescribeMacro{\ifchilddoc}
The conditional |\ifchilddoc| distinguishes between the compilation of
child documents and the main document:
%
\begin{center}
|\ifchilddoc |\textit{child-code}| |[|\||else |\textit{main-code}]| \||fi|
\end{center}

%%%%%%%%%%%%%%%%%%%%%%%%%%%%%%%%%%%%%%%%
\DescribeMacro{\childdocname}
\DescribeMacro{\childdocjob}
The macro |\childdocname| contains the filename (without extension)
of the main or child file being processed.
Note that |\childdocjob| will always contain the name of the main file.

%%%%%%%%%%%%%%%%%%%%%%%%%%%%%%%%%%%%%%%%
\paragraph{Title Page.}

Conditional processing can be used to include a title or banner page
in the main document when proper precautions are taken.
Importantly, the code in the main file should ensure that the page counter
(as well as other status parameters which are stored in the |.aux| files)
takes the same value after the conditional processing.
Otherwise the page numbers may take divergent values
depending on which part is compiled.

For example, a title page could be declared by:
%
\begin{center}
\begin{tabular}{l}
|\ifchilddoc\||else|\\
|\addtocounter{page}{-1}|\\
\textit{code for title page}\\
|\newpage|\\
|\||fi|
\end{tabular}
\end{center}
%
A banner page for the child documents can be generated by:
%
\begin{center}
\begin{tabular}{l}
|\ifchilddoc|\\
|\addtocounter{page}{-1}|\\
\textit{code for banner page}\\
|\newpage|\\
|\||fi|
\end{tabular}
\end{center}
%
Here one could write a message such as:
\begin{center}
|This is the part \childdocname{} of \childdocjob{}.|
\end{center}

%%%%%%%%%%%%%%%%%%%%%%%%%%%%%%%%%%%%%%%%%%%%%%%%%%%%%%%%%%%%%%%%%%%%%%%%%%%%%%%%
\subsection{Flags}
\label{sec:flags}

The package makes it easy to generate different versions
of the main or child documents.
To this end compilation flags can be defined
and assigned different default values.
They will be particularly useful in conjunction
with the forwarding mechanism described in \secref{sec:forward}.

For example, it may be useful to have a flag |\version|
which can be set to |draft| or |final|.
The document source will contain some conditional code
depending on the value of |\version|.
Suppose further, the flag should default to |final| for the main file
and to |draft| for child files
which is a natural assignment for editing the document.
This is achieved by placing the following code
in the preamble of the main document
(below the |\childdocmain| directive):
%
\begin{center}
\begin{tabular}{l}
|\ifchilddoc|\\
|\providecommand{\version}{draft}|\\
|\||else|\\
|\providecommand{\version}{final}|\\
|\||fi|
\end{tabular}
\end{center}
%
The definition by |\providecommand| makes sure
that previous definitions are not overwritten.
Further statements |\providecommand{\version}{...}|
can thus be added before the above code to override it.

For the main file, one might add a line
(between |\childdocmain| and the above block)
%
\begin{center}
|%\ifchilddoc\||else\providecommand{\version}{draft}\||fi|
\end{center}
%
which can be uncommented to produce a draft version.
Likewise one can add a line to the very top of a child file
(above the |\childdocof{|\textit{main}|}| directive)
%
\begin{center}
|%\providecommand{\version}{final}|
\end{center}
%
which can be uncommented to produce the final version of this child document.

%%%%%%%%%%%%%%%%%%%%%%%%%%%%%%%%%%%%%%%%%%%%%%%%%%%%%%%%%%%%%%%%%%%%%%%%%%%%%%%%
\subsection{Forwarding}
\label{sec:forward}

Different versions of the main or child documents
using compilation flags as described in \secref{sec:flags}
can be (permanently) stored in different files
for convenient compilation, viewing and distribution.
To this end, the package defines a command
to pass on compilation to a different file:

%%%%%%%%%%%%%%%%%%%%%%%%%%%%%%%%%%%%%%%%
\DescribeMacro{\childdocforward}
The command |\childdocforward| redirects processing to
another source file:
%
\begin{center}
\begin{tabular}{l}
|\input{childdoc.def}|\\
|\childdocforward[|\textit{main}|]{|\textit{dest}|}|\\
\end{tabular}
\end{center}
%
The argument \textit{dest} is the destination file
(without extension).
It should be the main file or one of the child files.
Note that further \textsf{childdoc} directives
such as |\childdocof| and |\childdocforward|
in the indicated file will be processed in this form.
The optional argument \textit{main}
passes on directly to the main file \textit{main}
while pretending to compile the child \textit{dest}.
This form behaves as if \textit{dest}
issues |\childdocof{|\textit{main}|}| right away,
and no further \textsf{childdoc} directives will be processed.

%%%%%%%%%%%%%%%%%%%%%%%%%%%%%%%%%%%%%%%%
\DescribeMacro{\...prefix}
In the alternative form |\childdocforwardprefix|,
%
\begin{center}
\begin{tabular}{l}
|\input{childdoc.def}|\\
|\childdocforwardprefix[|\textit{main}|]{|\textit{prefix}|}{|\textit{dest}|}|
\end{tabular}
\end{center}
%
the destination file is determined by a pattern
depending on the current file:
To make this work, the current file must be called
`{\textit{prefix}\hspace{0.2em}\textit{suffix}}'
with \textit{prefix} matching precisely the argument.
Processing is then passed on to the file
`{\textit{dest}\hspace{0.2em}\textit{suffix}}'.
Surely, the same effect is achieved by
directly specifying the
argument `{\textit{dest}\hspace{0.2em}\textit{suffix}}'
in the first form.
However, that requires to set up a different file
for each child. With the alternative form of the command
all these files can have exactly the same content
which simplifies setting them up and maintaining them.

For example, the following file |draft.tex|
with a compilation flag |\version| as described in \secref{sec:flags}
compiles the main document as a draft:
%
\begin{center}
\begin{tabular}{l}
|\def\version{draft}|\\
|\input{childdoc.def}|\\
|\childdocforward{|\textit{main}|}|
\end{tabular}
\end{center}
%
Likewise, the following files |final|\textit{nn}|.tex|
compile the final version of the child document
|child|\textit{nn}|.tex|:
%
\begin{center}
\begin{tabular}{l}
|\def\version{final}|\\
|\input{childdoc.def}|\\
|\childdocforwardprefix{final}{child}|
\end{tabular}
\end{center}
%

Note that when several versions of a main file and/or of each child file
are to be generated, it may be convenient to set up a |Makefile| or
shell script to automatise the process.

%%%%%%%%%%%%%%%%%%%%%%%%%%%%%%%%%%%%%%%%%%%%%%%%%%%%%%%%%%%%%%%%%%%%%%%%%%%%%%%%
\subsection{Command Line Processing}
\label{sec:commandline}

The effect of redirection files can also be achieved by invoking
the \LaTeX{} compiler with a more elaborate command line.
Most conveniently this should be done as part
of a shell script or a |Makefile|.

When using \textsf{childdoc} in the main file, the following
command lines effectively perform a redirection
(note that depending on the shell being used,
backslashes may have to be doubled: `|\|' $\to$ `|\\|'):
%
\begin{center}
|... -jobname "|\textit{target}|" |\\|"|[\textit{flags}]%
|\input{childdoc.def}\childdocforward[|\textit{main}|]{|\textit{dest}|}"|
\end{center}
%
Here \textit{target} is the name of the output file,
\textit{main} is the name of the main file
and \textit{dest} is the name of the main or child file to be processed
(all filenames without extensions).
The optional argument \textit{main} can be omitted
if \textit{main} matches \textit{dest}.
Optionally, compilation \textit{flags} can be defined via |\def| commands.
This command line makes the \TeX{} engine believe
it is compiling the file \textit{target}
whose content is specified as the latter parameter.
The provided code then forwards the processing to
\textit{main} or \textit{dest} as described in \secref{sec:forward}.

%%%%%%%%%%%%%%%%%%%%%%%%%%%%%%%%%%%%%%%%%%%%%%%%%%%%%%%%%%%%%%%%%%%%%%%%%%%%%%%%
\subsection{Include by Input}
\label{sec:input}

Including child documents by |\include| has some restrictions by design.
Most notably, the content of a child document always occupies
its own set of pages; pages cannot be shared between child documents.
Usually, this behaviour makes perfect sense
because each child document contain an essential part of the document.
However, in some situations it may be desirable to compose
a document from a collection of parts
without having mandatory page breaks between then.
For this case, the package
provides a mechanism to include parts
by |\input| which can also be processed individually.
However, by construction this mechanism
requires manual handling of the content to be output.

%%%%%%%%%%%%%%%%%%%%%%%%%%%%%%%%%%%%%%%%
\DescribeMacro{\ifchilddocmanual}
The main file should be prepared as usual, see \secref{sec:include}.
However, the document body must make a distinction
between processing of an individual part and of the main document, e.g.:
%
\begin{center}
\begin{tabular}{l}
|\ifchilddocmanual|\\
|\input{\childdocname}|\\
|\||else|\\
\textit{document body with }|\input{|\textit{part}|}|\\
|\||fi|
\end{tabular}
\end{center}
%
The conditional |\ifchilddocmanual| is true whenever
a part to be included by |\input| is being compiled,
and the name of the part is stored in |\childdocname|.

%%%%%%%%%%%%%%%%%%%%%%%%%%%%%%%%%%%%%%%%
\DescribeMacro{\childdocby}
Each part to be included by |\input| should start with:
%
\begin{center}
\begin{tabular}{l}
|\input{childdoc.def}|\\
|\childdocby{|\textit{main}|}|\\
\end{tabular}
\end{center}
%
The directive |\childdocby| is similar to |\childdocof|
described in \secref{sec:include},
but the subsequent selection of content must be done manually.
To that end, both |\ifchilddoc| and |\ifchilddocmanual|
will be true upon processing of a part,
and the name of the part is stored in |\childdocname|.
Note that |\jobname| will be set to the filename of the current part
so that each part receives an individual |.aux| file
that does not interfere with the |.aux| file(s) of the main document.
This behaviour can be altered by the alternative form
|\childdocby[*]{|\textit{main}|}| (with a non-empty optional argument)
which uses the |.aux| file of the main document
by setting |\jobname| to \textit{main}.

%%%%%%%%%%%%%%%%%%%%%%%%%%%%%%%%%%%%%%%%%%%%%%%%%%%%%%%%%%%%%%%%%%%%%%%%%%%%%%%%
\subsection{Driver Development}
\label{sec:driver}

The \textsf{childdoc} mechanism can also be use for the development
of definition files such as \LaTeX{} styles or classes.
This case differs from the above setup with multiple parts
included by |\include| in that no |\includeonly| should be invoked.
This can be achieved by starting the include file
(before |\ProvidesPackage|) with:
%
\begin{center}
\begin{tabular}{l}
|\input{childdoc.def}|\\
|\childdocforward{|\textit{main}|}|\\
\end{tabular}
\end{center}
%
or alternatively with:
%
\begin{center}
\begin{tabular}{l}
|\input{childdoc.def}|\\
|\childdocby{|\textit{main}|}|\\
\end{tabular}
\end{center}
%
Both forms have slightly different effects as described above.
The main file is prepared as usual, see \secref{sec:include}.

%%%%%%%%%%%%%%%%%%%%%%%%%%%%%%%%%%%%%%%%%%%%%%%%%%%%%%%%%%%%%%%%%%%%%%%%%%%%%%%%
\subsection{Legacy Detection}
\label{sec:detection}

The directive |\childdocmain| in the main file can detect
whether the complete document or merely a child is to be compiled
even without using the directive |\childdocof|.
This method is deprecated because it is less robust
and there is no compelling reason to use it;
it is merely provided for backward compatibility
and it may be removed in future versions.

If the detection mechanism is to be used,
it is mandatory to correctly specify
the filename of the main file as the argument of |\childdocmain|:
%
\begin{center}
\begin{tabular}{l}
|\input{childdoc.def}|\\
|\childdocmain{|\textit{main}|}|\\
\end{tabular}
\end{center}
%
If |\jobname| does not match the argument \textit{main} of |\childdocmain|,
it is assumed that |\jobname| points to the child file to be compiled.
When using |\childdocmain| with the main file specified as argument,
it suffices to start a child file
with just |\input{|\textit{main}|}|
without loading of the package and using |\childdocof|.
If instead all processing is done
with the appropriate \textsf{childdoc} directives,
the argument of \textit{main} of |\childdocmain| can be empty.

An alternative version of the command line processing described
in \secref{sec:commandline} using the detection mechanism reads:
%
\begin{center}
|... -jobname "|\textit{target}|" "|[\textit{flags}]%
[|\def\jobname{|\textit{dest}|}|]|\input{|\textit{main}|}"|
\end{center}

%%%%%%%%%%%%%%%%%%%%%%%%%%%%%%%%%%%%%%%%%%%%%%%%%%%%%%%%%%%%%%%%%%%%%%%%%%%%%%%%
\subsection{Manual Code}
\label{sec:manual}

In case one cannot be certain whether the definitions file |childdoc.def|
is installed on the target \TeX{} distribution
and one prefers not to ship it,
it is conceivable to paste a few relevant commands into the sources.

To that end, drop all statements |\input{childdoc.def}|
and perform the replacements as outlined below.
Instead of |\childdocmain{|\textit{main}|}| add the following code
to the top of the main file:
%
\begin{center}
\begin{tabular}{l}
|\||ifdefined\childdocname\endinput\||fi\newif\ifchilddoc|\\
|\edef\childdocname{\scantokens\expandafter{\jobname\noexpand}}|\\
|\def\childdocmain{|\textit{main}|}\||ifx\childdocmain\childdocname\||else|\\
|\childdoctrue\includeonly{\childdocname}\let\jobname\childdocmain\||fi|\\
\end{tabular}
\end{center}
%
Instead of |\childdocof{|\textit{main}|}| just include the main file
at the top of each child file:
%
\begin{center}
|\input{|\textit{main}|}|
\end{center}
%
A simple redirection |\childdocforward{|\textit{dest}|}| is achieved by:
%
\begin{center}
|\def\jobname{|\textit{dest}|}\input{\jobname}|
\end{center}
%
The redirection with prefix
|\childdocforwardprefix[|\textit{prefix}|]{|\textit{dest}|}|
is accomplished by:
%
\begin{center}
\begin{tabular}{l}
|{\edef\jobname{\scantokens\expandafter{\jobname\noexpand}}|\\
|\def\redirectjob |\textit{prefix}|#1~~~{\gdef\jobname{|\textit{dest}|#1}}|\\
|\expandafter\redirectjob\jobname~~~}\input{\jobname}|
\end{tabular}
\end{center}

In an alternative approach,
child documents can be compiled by a specific command line
without additional code or specific definitions:
%
\begin{center}
|... -jobname "|\textit{target}|" "|[\textit{flags}]%
|\includeonly{|\textit{dest}|}\input{|\textit{main}|}"|
\end{center}
%

%%%%%%%%%%%%%%%%%%%%%%%%%%%%%%%%%%%%%%%%%%%%%%%%%%%%%%%%%%%%%%%%%%%%%%%%%%%%%%%%
%%%%%%%%%%%%%%%%%%%%%%%%%%%%%%%%%%%%%%%%%%%%%%%%%%%%%%%%%%%%%%%%%%%%%%%%%%%%%%%%
\section{Information}

%%%%%%%%%%%%%%%%%%%%%%%%%%%%%%%%%%%%%%%%%%%%%%%%%%%%%%%%%%%%%%%%%%%%%%%%%%%%%%%%
\subsection{Copyright}

Copyright \copyright{} 2017--2018 Niklas Beisert

This work may be distributed and/or modified under the
conditions of the \LaTeX{} Project Public License, either version 1.3
of this license or (at your option) any later version.
The latest version of this license is in
  \url{http://www.latex-project.org/lppl.txt}
and version 1.3 or later is part of all distributions of \LaTeX{}
version 2005/12/01 or later.

This work has the LPPL maintenance status `maintained'.

The Current Maintainer of this work is Niklas Beisert.

This work consists of the files |README.txt|, |childdoc.ins| and |childdoc.dtx|
as well as the derived files |childdoc.def|, |cdocsamp.tex|
with |cdocsch1.tex|, |cdocsch2.tex|, |cdocspt3.tex|, |cdocspt4.tex|,
|cdocsdrf.tex|, |cdocsfn1.tex|, |cdocsfn2.tex|
as well as |childdoc.pdf|.

%%%%%%%%%%%%%%%%%%%%%%%%%%%%%%%%%%%%%%%%%%%%%%%%%%%%%%%%%%%%%%%%%%%%%%%%%%%%%%%%
\subsection{Files and Installation}

The package consists of the files:
%
\begin{center}
\begin{tabular}{ll}
    |README.txt|   & readme file \\
    |childdoc.ins| & installation file \\
    |childdoc.dtx| & source file \\
    |childdoc.def| & definition file \\
    |cdocsamp.tex| & sample main file \\
    |cdocsch1.tex| & sample include file \\
    |cdocsch2.tex| & sample include file \\
    |cdocspt3.tex| & sample part file \\
    |cdocspt4.tex| & sample part file \\
    |cdocsdrf.tex| & sample redirection file \\
    |cdocsfn1.tex| & sample redirection file \\
    |cdocsfn2.tex| & sample redirection file \\
    |childdoc.pdf| & manual
\end{tabular}
\end{center}
%
The distribution consists of the files
|README.txt|, |childdoc.ins| and |childdoc.dtx|.
%
\begin{itemize}
\item
Run (pdf)\LaTeX{} on |childdoc.dtx|
to compile the manual |childdoc.pdf| (this file).
\item
Run \LaTeX{} on |childdoc.ins| to create the definitions file |childdoc.def|
and the sample |cdocsamp.tex| with include files
|cdocsch1.tex|, |cdocsch2.tex|, |cdocspt3.tex|, |cdocspt4.tex|,
|cdocsdrf.tex|, |cdocsfn1.tex|, |cdocsfn2.tex|.
Then copy the file |childdoc.def| to an appropriate directory of your \LaTeX{}
distribution, e.g.\ \textit{texmf-root}|/tex/latex/childdoc|.
\end{itemize}

%%%%%%%%%%%%%%%%%%%%%%%%%%%%%%%%%%%%%%%%%%%%%%%%%%%%%%%%%%%%%%%%%%%%%%%%%%%%%%%%
\subsection{Related CTAN Packages}

There are several other packages which offer a similar functionality:
%
\begin{itemize}
\item
The packages
\href{http://ctan.org/pkg/docmute}{\textsf{docmute}},
\href{http://ctan.org/pkg/includex}{\textsf{includex}} and
\href{http://ctan.org/pkg/standalone}{\textsf{standalone}}
provide commands to include only the document body of
a child file thus allowing both files to be compiled individually.
\item
The packages \href{http://ctan.org/pkg/subdocs}{\textsf{subdocs}}
and \href{http://ctan.org/pkg/subfiles}{\textsf{subfiles}}
provide structures in which the main and child documents can be
encapsulated and allowing them to be compiled individually.
The inclusion mechanism is different from the conventional |\include|.
\item
The package \href{http://ctan.org/pkg/combine}{\textsf{combine}}
is an elaborate solution to combine several documents into one.
\end{itemize}
%
See also the CTAN topic \href{http://ctan.org/topic/subdocs}{\textsf{subdocs}}
for further related packages.
The present package differs from the above solutions in that
a document structure constructed with the conventional |\include| mechanism
just needs two extra commands at the top of every file
such that all constituent files can be compiled individually.

%%%%%%%%%%%%%%%%%%%%%%%%%%%%%%%%%%%%%%%%%%%%%%%%%%%%%%%%%%%%%%%%%%%%%%%%%%%%%%%%
%\subsection{Feature Suggestions}
%
%The following is a list of features which may be useful for future
%versions of this package:
%%
%\begin{itemize}
%\item
%\ldots
%\end{itemize}

%%%%%%%%%%%%%%%%%%%%%%%%%%%%%%%%%%%%%%%%%%%%%%%%%%%%%%%%%%%%%%%%%%%%%%%%%%%%%%%%
\subsection{Revision History}

%%%%%%%%%%%%%%%%%%%%%%%%%%%%%%%%%%%%%%%%
\paragraph{v2.0:} 2018/12/30

\begin{itemize}
\item
immediate forward processing
\item
added |\childdocby| mechanism
\item
manual restructured
\end{itemize}

%%%%%%%%%%%%%%%%%%%%%%%%%%%%%%%%%%%%%%%%
\paragraph{v1.6:} 2018/01/17

\begin{itemize}
\item
application for development of include files
\item
corrections to manual
\end{itemize}

%%%%%%%%%%%%%%%%%%%%%%%%%%%%%%%%%%%%%%%%
\paragraph{v1.5:} 2017/05/21

\begin{itemize}
\item
more complete structuring introduced
\item
|\childdocof| introduced
\item
|\childdoc| renamed to |\childdocmain|
\item
|\childredirect| renamed to |\childdocforward| and |\childdocforwardprefix|
and functionality expanded
\end{itemize}

%%%%%%%%%%%%%%%%%%%%%%%%%%%%%%%%%%%%%%%%
\paragraph{v1.0:} 2017/04/27

\begin{itemize}
\item
manual and install package
\item
first version published on CTAN
\end{itemize}

%%%%%%%%%%%%%%%%%%%%%%%%%%%%%%%%%%%%%%%%
\paragraph{v0.6:} 2017/04/26

\begin{itemize}
\item
redirection mechanism added
\end{itemize}

%%%%%%%%%%%%%%%%%%%%%%%%%%%%%%%%%%%%%%%%
\paragraph{v0.5:} 2017/04/26

\begin{itemize}
\item
functionality in definition file
\end{itemize}


%%%%%%%%%%%%%%%%%%%%%%%%%%%%%%%%%%%%%%%%%%%%%%%%%%%%%%%%%%%%%%%%%%%%%%%%%%%%%%%%
%%%%%%%%%%%%%%%%%%%%%%%%%%%%%%%%%%%%%%%%%%%%%%%%%%%%%%%%%%%%%%%%%%%%%%%%%%%%%%%%
%%%%%%%%%%%%%%%%%%%%%%%%%%%%%%%%%%%%%%%%%%%%%%%%%%%%%%%%%%%%%%%%%%%%%%%%%%%%%%%%
\appendix

\settowidth\MacroIndent{\rmfamily\scriptsize 000\ }

 \DocInput{childdoc.dtx}

\end{document}
%</driver>
% \fi
%
% %%%%%%%%%%%%%%%%%%%%%%%%%%%%%%%%%%%%%%%%%%%%%%%%%%%%%%%%%%%%%%%%%%%%%%%%%%%%%%
% %%%%%%%%%%%%%%%%%%%%%%%%%%%%%%%%%%%%%%%%%%%%%%%%%%%%%%%%%%%%%%%%%%%%%%%%%%%%%%
% \section{Sample}
%\iffalse
%<*samplemain>
%\fi
%
% The following presents a sample document
% with two chapters, two parts, a title page,
% a compile flag as well as three forwarding files to set the flag.
% It consists of eight |.tex| files:
% \begin{center}
% \begin{tabular}{ll}
% |cdocsamp.tex|&main file\\
% |cdocsch1.tex|&include file for chapter 1\\
% |cdocsch2.tex|&include file for chapter 2\\
% |cdocspt3.tex|&include file for part 3\\
% |cdocspt4.tex|&include file for part 4\\
% |cdocsdrf.tex|&forwarding file for main file in draft mode\\
% |cdocsfi1.tex|&forwarding file for final version of chapter 1\\
% |cdocsfi2.tex|&forwarding file for final version of chapter 2\\
% \end{tabular}
% \end{center}
% Each of the eight files can be compiled directly by the \LaTeX{} compiler.
%
% %%%%%%%%%%%%%%%%%%%%%%%%%%%%%%%%%%%%%%
% \paragraph{Main File.}
%
% The main file is called |cdocsamp.tex|.
%
% Load the \textsf{childdoc} definitions and
% declare the filename for the main document:
%    \begin{macrocode}
\input{childdoc.def}
\childdocmain{}
%    \end{macrocode}

% Optional override for |\version| flag:
%    \begin{macrocode}
%%\ifchilddoc\else\providecommand{\version}{draft}\fi
%    \end{macrocode}

% Define the default values for the |\version| flag
% (|final| for the main file and |draft| for childs):
%    \begin{macrocode}
\ifchilddoc
\providecommand{\version}{draft}
\else
\providecommand{\version}{final}
\fi
%    \end{macrocode}

% Load the standard document class:
%    \begin{macrocode}
\documentclass[12pt]{article}
%    \end{macrocode}

% Start the document body:
%    \begin{macrocode}
\begin{document}
%    \end{macrocode}

% Declare a title page.
% Print title, part of document being processed and version flag:
%    \begin{macrocode}
\addtocounter{page}{-1}
\begin{center}
{\LARGE\bfseries{}childdoc example\par}
\vspace{1cm}
\ifchilddoc
\ifchilddocmanual part\else chapter\fi:
`\childdocname' of `\childdocjob'\par
\else
main document: `\childdocjob'\par
\fi
version: \version\par
\end{center}
\newpage
%    \end{macrocode}

% Manually include selected file,
% otherwise process as usual:
%    \begin{macrocode}
\ifchilddocmanual
\section*{part `\childdocname'}
\input{\childdocname}
\else
%    \end{macrocode}

% Include the two chapters:
%    \begin{macrocode}
\include{cdocsch1}
\include{cdocsch2}
%    \end{macrocode}

% Include the two parts unless only chapters should be displayed:
%    \begin{macrocode}
\ifchilddoc\else
\section{part three}
\input{cdocspt3}
\section{part four}
\input{cdocspt4}
\fi
%    \end{macrocode}

% Process as usual until here:
%    \begin{macrocode}
\fi
%    \end{macrocode}

% End of document body:
%    \begin{macrocode}
\end{document}
%    \end{macrocode}
%\iffalse
%</samplemain>
%\fi
%
% %%%%%%%%%%%%%%%%%%%%%%%%%%%%%%%%%%%%%%
% \paragraph{Chapter Include Files.}
%
% The include files are called |cdocsch1.tex| and |cdocsch2.tex|.
%
%\iffalse
%<*samplechap1|samplechap2>
%\fi

% Optional override for |\version| flag:
%    \begin{macrocode}
%%\providecommand{\version}{final}
%    \end{macrocode}

% Include the main document:
%    \begin{macrocode}
\input{childdoc.def}
\childdocof{cdocsamp}
%    \end{macrocode}

%\iffalse
%</samplechap1|samplechap2>
%\fi
%
%\iffalse
%<*samplechap1>
%\fi
% Some text for chapter 1:
%    \begin{macrocode}
\section{one}
some text in chapter one
%    \end{macrocode}

%\iffalse
%</samplechap1>
%\fi
% Some text for chapter 2:
%\iffalse
%<*samplechap2>
%\fi
%    \begin{macrocode}
\section{two}
more text in chapter two
%    \end{macrocode}

%\iffalse
%</samplechap2>
%\fi
%
% %%%%%%%%%%%%%%%%%%%%%%%%%%%%%%%%%%%%%%
% \paragraph{Part Include Files.}
%
% The include files are called |cdocspt3.tex| and |cdocspt4.tex|.
%
%\iffalse
%<*samplepart3|samplepart4>
%\fi

% Optional override for |\version| flag:
%    \begin{macrocode}
%%\providecommand{\version}{final}
%    \end{macrocode}

% Include the main document:
%    \begin{macrocode}
\input{childdoc.def}
\childdocby{cdocsamp}
%    \end{macrocode}

%\iffalse
%</samplepart3|samplepart4>
%\fi
%
%\iffalse
%<*samplepart3>
%\fi
% Some text for part 3:
%    \begin{macrocode}
some text in part three
%    \end{macrocode}

%\iffalse
%</samplepart3>
%\fi
% Some text for part 4:
%\iffalse
%<*samplepart4>
%\fi
%    \begin{macrocode}
more text in part four
%    \end{macrocode}

%\iffalse
%</samplepart4>
%\fi
%
% %%%%%%%%%%%%%%%%%%%%%%%%%%%%%%%%%%%%%%
% \paragraph{Forwarding for a Complete Draft.}
%
% The following forwarding file |cdocsdrf.tex|
% compiles the main document in draft mode:
%\iffalse
%<*sampledraft>
%\fi
%    \begin{macrocode}
\def\version{draft}
\input{childdoc.def}
\childdocforward{cdocsamp}
%    \end{macrocode}

%\iffalse
%</sampledraft>
%\fi
%
% %%%%%%%%%%%%%%%%%%%%%%%%%%%%%%%%%%%%%%
% \paragraph{Forwarding for Final Version of the Chapters.}
%
% The following forwarding files |cdocsfn1.tex| and |cdocsfn2.tex|
% (with identical content)
% compile the final versions of the child documents
% |cdocsch1.tex| and |cdocsch2.tex|, respectively:
%\iffalse
%<*samplefinal>
%\fi
%    \begin{macrocode}
\def\version{final}
\input{childdoc.def}
\childdocforwardprefix[cdocsamp]{cdocsfn}{cdocsch}
%    \end{macrocode}

%\iffalse
%</samplefinal>
%\fi
%
% %%%%%%%%%%%%%%%%%%%%%%%%%%%%%%%%%%%%%%
% \paragraph{Command Line Processing.}
%
% The following three command lines generate the output files
% |cdocscld|, |cdocscl1| and |cdocscl2|
% which should be identical to
% |cdocsdrf|, |cdocsch1| and |cdocsfn2|, respectively:
% \begin{center}
% \begin{tabular}{l}
% |latex -jobname cdocscld \|\\
% |  "\def\version{draft}\input{childdoc.def}\childdocforward{cdocsamp}"|\\
% |latex -jobname cdocscl1 \|\\
% |  "\input{childdoc.def}\childdocforward[cdocsamp]{cdocsch1}"|\\
% |latex -jobname cdocscl2 \|\\
% |  "\def\version{final}\input{childdoc.def}\childdocforward{cdocsch2}"|
% \end{tabular}
% \end{center}
% Note that the trailing backslash on each first line
% merely continues the input to the second line
% (for convenient cut ant paste).
% Furthermore, the command |latex| can be replaced by any
% of its alternative versions such as |pdflatex|.
%
% %%%%%%%%%%%%%%%%%%%%%%%%%%%%%%%%%%%%%%%%%%%%%%%%%%%%%%%%%%%%%%%%%%%%%%%%%%%%%%
% %%%%%%%%%%%%%%%%%%%%%%%%%%%%%%%%%%%%%%%%%%%%%%%%%%%%%%%%%%%%%%%%%%%%%%%%%%%%%%
% \section{Implementation}
%\iffalse
%<*package>
%\fi
%
% This section describes the definitions file |childdoc.def|.

% The definitions cannot be loaded using |\usepackage| or |\RequirePackage|
% which has a mechanism to prevent loading a style file more than once.
% When loading the definitions by means of |\input|
% multiple instances have to be prevented manually:
%\iffalse
%This code needs to be before the `\ProvidesFile' directive
%which is defined at the beginning of this file.
%Therefore it is also placed there and commented out here.
%</package>
%<*discard>
%\fi
%    \begin{macrocode}
\ifdefined\childdocmain\endinput\fi
%    \end{macrocode}
%\iffalse
%</discard>
%<*package>
%\fi
%
% \macro{\ifchilddoc}
% \macro{\ifchilddocmanual}
% The conditional |\ifchilddoc| tells whether a
% child (true) or main (false) document is being compiled.
% The conditional |\ifchilddocmanual| tells whether
% the |\includeonly| mechanism is used (false) or
% the selection of child files must be performed manually (true).
% The definitions initialise to false:
%    \begin{macrocode}
\newif\ifchilddoc
\newif\ifchilddocmanual
%    \end{macrocode}

% \macro{\childdocname}
% \macro{\childdocjob}
% The macro |\childdocname| stores the name of the main document
% to be compiled. The macro |\childdocjob| stores the name of
% the document on which the \LaTeX{} compiler was originally invoked.
% The content of |\jobname| cannot be compared
% to filenames specified in the source due to different catcodes.
% The following code rescans |\jobname|, stores the result
% in |\childdocname| and saves a copy in |\childdocjob|:
%    \begin{macrocode}
\edef\childdocname{\scantokens\expandafter{\jobname\noexpand}}
\let\childdocjob\childdocname
%    \end{macrocode}

% \macro{\childdocdisable}
% The macro |\childdocdisable| prevents the main file
% from being processed more than once.
% At this stage, the main document command |\childdocmain|
% is assumed to be called once again where it should do nothing.
% Any subsequent call to it should prevent
% a secondary processing of the main document
% It overwrites the forwarding commands
% |\childdocof| and |\childdocforward|
% with empty macros to prevent further inclusions of the main document:
%    \begin{macrocode}
\newcommand{\childdocdisable}
{
  \renewcommand{\childdocmain}[1]{\renewcommand{\childdocmain}[1]{\endinput}}
  \renewcommand{\childdocof}[1]{}
  \renewcommand{\childdocby}[2][]{}
  \renewcommand{\childdocforward}[2][]{}
  \renewcommand{\childdocdisable}{}
}
%    \end{macrocode}

% \macro{\childdocmain}
% The macro |\childdocmain| is to be called at the top of the main file
% with nothing or the main filename (without extension) as argument.
% First, it breaks loops.
% If the argument is not empty and does not match |\childdocname|
% (which is set by the first inclusion of |childdoc.def|),
% |\ifchilddoc| is set to true, |\includeonly| is applied to the child file
% and |\jobname| is set to the main file
% (for proper handling of |.aux| files):
%    \begin{macrocode}
\newcommand{\childdocmain}[1]
{
  \childdocdisable\childdocmain{}
  \if?#1?\else
    \begingroup
      \def\childdoctmp{#1}
      \ifx\childdoctmp\childdocname
        \def\childdoctmp{}
      \else
        \def\childdoctmp
        {
          \childdoctrue
          \includeonly{\childdocname}
          \def\childdocjob{#1}
          \def\jobname{#1}
        }
      \fi
      \expandafter
    \endgroup
    \childdoctmp
  \fi
}
%    \end{macrocode}

% \macro{\childdocof}
% The command |\childdocof| redirects
% compilation to the main file |#1|.
%    \begin{macrocode}
\newcommand{\childdocof}[1]
{
  \childdocdisable
  \childdoctrue
  \includeonly{\childdocname}
  \def\jobname{#1}
  \def\childdocjob{#1}
  \input{#1}
}
%    \end{macrocode}

% \macro{\childdocby}
% The command |\childdocby| ....
%    \begin{macrocode}
\newcommand{\childdocby}[2][]
{
  \childdocdisable
  \childdoctrue
  \childdocmanualtrue
  \if?#1?\else
    \def\jobname{#2}
  \fi
  \def\childdocjob{#2}
  \input{#2}
  \endinput
}
%    \end{macrocode}

% \macro{\childdocforward}
% The command |\childdocforward| redirects
% compilation to the main file or
% (if the optional argument is given) a child file.
% Parameters are set as if the main file
% or a child file starting with |\childdocof| was compiled.
% Then compilation is handed over to the main file:
%    \begin{macrocode}
\newcommand{\childdocforward}[2][]
{
  \begingroup
    \if?#1?
      \def\childdoctmp
      {
        \def\childdocname{#2}
        \def\childdocjob{#2}
        \def\jobname{#2}
        \input{#2}
        \endinput
      }
    \else
      \def\childdoctmp
      {
        \childdocdisable
        \def\childdocname{#2}
        \childdoctrue
        \includeonly{#2}
        \def\childdocjob{#1}
        \def\jobname{#1}
        \input{#1}
        \endinput
      }
    \fi
    \expandafter
  \endgroup
  \childdoctmp
}
%    \end{macrocode}

% \macro{\childdocforwardprefix}
% The command |\childdocforwardprefix| redirects
% compilation to the main or a child file by means of a pattern.
% The prefix |#1| in the current filename is replaced by |#2|
% and the suffix of the current filename is kept
% (it is assumed that the filename does not contain the substring `|~~~|'
% which is used as a delimiter).
% Compilation is handed over to the new file by |\childdocforward|:
%    \begin{macrocode}
\newcommand{\childdocforwardprefix}[3][]
{
  \begingroup
    \def\childdocextract #2##1~~~{\def\childdoctmp{\childdocforward[#1]{#3##1}}}
    \expandafter\childdocextract\childdocname~~~
    \expandafter
  \endgroup
  \childdoctmp
}
%    \end{macrocode}

% \macro{\childdoc}
% The deprecated macro |\childdoc| is a legacy version of |\childdocmain|:
%    \begin{macrocode}
\newcommand{\childdoc}{\childdocmain}
%    \end{macrocode}

% \macro{\childdocredirect}
% The deprecated macro |\childdocredirect| is a legacy version
% of |\childdocforward| and |\childdocforwardprefix|:
%    \begin{macrocode}
\newcommand{\childdocredirect}[2][]
{
  \begingroup
    \if?#1?
      \def\childdoctmp{\childdocforward{#2}}
    \else
      \def\childdoctmp{\childdocforwardprefix{#1}{#2}}
    \fi
    \expandafter
  \endgroup
  \childdoctmp
}
%    \end{macrocode}

%\iffalse
%</package>
%\fi
%
\endinput

\childdocmain{}
%    \end{macrocode}

% Optional override for |\version| flag:
%    \begin{macrocode}
%%\ifchilddoc\else\providecommand{\version}{draft}\fi
%    \end{macrocode}

% Define the default values for the |\version| flag
% (|final| for the main file and |draft| for childs):
%    \begin{macrocode}
\ifchilddoc
\providecommand{\version}{draft}
\else
\providecommand{\version}{final}
\fi
%    \end{macrocode}

% Load the standard document class:
%    \begin{macrocode}
\documentclass[12pt]{article}
%    \end{macrocode}

% Start the document body:
%    \begin{macrocode}
\begin{document}
%    \end{macrocode}

% Declare a title page.
% Print title, part of document being processed and version flag:
%    \begin{macrocode}
\addtocounter{page}{-1}
\begin{center}
{\LARGE\bfseries{}childdoc example\par}
\vspace{1cm}
\ifchilddoc
\ifchilddocmanual part\else chapter\fi:
`\childdocname' of `\childdocjob'\par
\else
main document: `\childdocjob'\par
\fi
version: \version\par
\end{center}
\newpage
%    \end{macrocode}

% Manually include selected file,
% otherwise process as usual:
%    \begin{macrocode}
\ifchilddocmanual
\section*{part `\childdocname'}
\input{\childdocname}
\else
%    \end{macrocode}

% Include the two chapters:
%    \begin{macrocode}
\include{cdocsch1}
\include{cdocsch2}
%    \end{macrocode}

% Include the two parts unless only chapters should be displayed:
%    \begin{macrocode}
\ifchilddoc\else
\section{part three}
\input{cdocspt3}
\section{part four}
\input{cdocspt4}
\fi
%    \end{macrocode}

% Process as usual until here:
%    \begin{macrocode}
\fi
%    \end{macrocode}

% End of document body:
%    \begin{macrocode}
\end{document}
%    \end{macrocode}
%\iffalse
%</samplemain>
%\fi
%
% %%%%%%%%%%%%%%%%%%%%%%%%%%%%%%%%%%%%%%
% \paragraph{Chapter Include Files.}
%
% The include files are called |cdocsch1.tex| and |cdocsch2.tex|.
%
%\iffalse
%<*samplechap1|samplechap2>
%\fi

% Optional override for |\version| flag:
%    \begin{macrocode}
%%\providecommand{\version}{final}
%    \end{macrocode}

% Include the main document:
%    \begin{macrocode}
% \iffalse
%
% childdoc.dtx Copyright (C) 2017-2018 Niklas Beisert
%
% This work may be distributed and/or modified under the
% conditions of the LaTeX Project Public License, either version 1.3
% of this license or (at your option) any later version.
% The latest version of this license is in
%   http://www.latex-project.org/lppl.txt
% and version 1.3 or later is part of all distributions of LaTeX
% version 2005/12/01 or later.
%
% This work has the LPPL maintenance status `maintained'.
%
% The Current Maintainer of this work is Niklas Beisert.
%
% This work consists of the files childdoc.dtx and childdoc.ins
% and the derived files childdoc.def and cdocsamp.tex with
% cdocsch1.tex, cdocsch2.tex, cdocsdrf.tex, cdocsfn1.tex, cdocsfn2.tex.
%
%<package>\ifdefined\childdocmain\endinput\fi
%<package>\ProvidesFile{childdoc.def}[2018/12/30 v2.0 child document driver]
%<samplemain>\ProvidesFile{cdocsamp.tex}[2018/12/30 v2.0 sample for childdoc]
%<*driver>
%\ProvidesFile{childdoc.drv}[2018/12/30 v2.0 childdoc reference manual file]
\PassOptionsToClass{10pt,a4paper}{article}
\documentclass{ltxdoc}

\usepackage[margin=35mm]{geometry}
\usepackage{hyperref}
\usepackage{hyperxmp}
\usepackage[usenames]{color}

\hypersetup{colorlinks=true}
\hypersetup{pdfstartview=FitH}
\hypersetup{pdfpagemode=UseNone}
\hypersetup{pdfsource={}}
\hypersetup{pdflang={en-UK}}
\hypersetup{pdfcopyright={Copyright 2017-2018 Niklas Beisert.
  This work may be distributed and/or modified under the
  conditions of the LaTeX Project Public License, either version 1.3
  of this license or (at your option) any later version.}}
\hypersetup{pdflicenseurl={http://www.latex-project.org/lppl.txt}}
\hypersetup{pdfcontactaddress={ETH Zurich, ITP, HIT K,
  Wolfgang-Pauli-Strasse 27}}
\hypersetup{pdfcontactpostcode={8093}}
\hypersetup{pdfcontactcity={Zurich}}
\hypersetup{pdfcontactcountry={Switzerland}}
\hypersetup{pdfcontactemail={nbeisert@itp.phys.ethz.ch}}
\hypersetup{pdfcontacturl={http://people.phys.ethz.ch/\xmptilde nbeisert/}}

\newcommand{\secref}[1]{\hyperref[#1]{section \ref*{#1}}}

\parskip1ex
\parindent0pt
\let\olditemize\itemize
\def\itemize{\olditemize\parskip0pt}

\begin{document}

\title{The \textsf{childdoc} Package}
\hypersetup{pdftitle={The childdoc Package}}
\author{Niklas Beisert\\[2ex]
  Institut f\"ur Theoretische Physik\\
  Eidgen\"ossische Technische Hochschule Z\"urich\\
  Wolfgang-Pauli-Strasse 27, 8093 Z\"urich, Switzerland\\[1ex]
  \href{mailto:nbeisert@itp.phys.ethz.ch}
  {\texttt{nbeisert@itp.phys.ethz.ch}}}
\hypersetup{pdfauthor={Niklas Beisert}}
\hypersetup{pdfsubject={Manual for the LaTeX2e Package childdoc}}
\date{30 December 2018, \textsf{v2.0}}
\maketitle

\begin{abstract}\noindent
\textsf{childdoc} is a \LaTeXe{} package
that enables the direct compilation
of document sections included by |\include|
to individual files.
\end{abstract}

\begingroup
\parskip0ex
\tableofcontents
\endgroup

%%%%%%%%%%%%%%%%%%%%%%%%%%%%%%%%%%%%%%%%%%%%%%%%%%%%%%%%%%%%%%%%%%%%%%%%%%%%%%%%
%%%%%%%%%%%%%%%%%%%%%%%%%%%%%%%%%%%%%%%%%%%%%%%%%%%%%%%%%%%%%%%%%%%%%%%%%%%%%%%%
\section{Introduction}

\LaTeX{} provides a mechanism to structure a large document (such as a book)
into a main file and several child files (containing the chapters)
using the |\include| command.
This mechanism is beneficial for documents
which span hundreds of pages in order to
make the source file(s) more manageable.
Moreover, compilation can be restricted to
selected child files by means of the |\includeonly| command.
The latter feature can be used to reduce the compilation time while editing
(this was significantly more useful in the earlier days of \LaTeX{})
or to generate a smaller document which is easier to navigate.
Another application of |\includeonly| is to generate
documents consisting of selected parts of the complete document.

However, there are a few drawbacks of the plain |\include| mechanism:
\begin{itemize}
\item
The child files cannot be compiled on their own,
they can only be compiled via the main file.
A naive editing environment
(such as a text editor with an option
to have the current file processed by \LaTeX)
may require one to switch to the main file before compiling;
attempting to compile the child file produces errors.
\item
The main file must be modified (each time)
to adjust the |\includeonly| command
to the present needs. This easily leaves the main file in a messy state.
\item
The generated document will always carry the filename
of the main document. This is inconvenient if
several child files are to be compiled and
to be kept for distribution.
\end{itemize}

The present package provides a simple interface
to make child files individually compilable by \LaTeX{}.
Compiling a child file then has the same effect as compiling
the main file with an |\includeonly| command
to select the appropriate child.
Moreover the generated document will carry the name of the child
rather than the main file.
This resolves all three above issues.

This feature is meant to make the editing of books,
thesis documents and lecture notes somewhat more convenient.
However, the package can also be used efficiently for
composing a series of documents (such as exercise sheets)
which are typically distributed individually.
It then assists the author in generating the individual documents
(potentially in different versions)
as well as a document containing the collected series.
Another application is in developing style files
or other kinds of included material
where compilation of the style file could redirect
to a sample or test file.

%%%%%%%%%%%%%%%%%%%%%%%%%%%%%%%%%%%%%%%%%%%%%%%%%%%%%%%%%%%%%%%%%%%%%%%%%%%%%%%%
%%%%%%%%%%%%%%%%%%%%%%%%%%%%%%%%%%%%%%%%%%%%%%%%%%%%%%%%%%%%%%%%%%%%%%%%%%%%%%%%
\section{Usage}

First of all, the package \textsf{childdoc} is \emph{not} a standard
\LaTeXe{} |.sty| style file! Therefore it needs to be invoked in
a non-standard way.

%%%%%%%%%%%%%%%%%%%%%%%%%%%%%%%%%%%%%%%%%%%%%%%%%%%%%%%%%%%%%%%%%%%%%%%%%%%%%%%%
\subsection{Included Files}
\label{sec:include}

%%%%%%%%%%%%%%%%%%%%%%%%%%%%%%%%%%%%%%%%
\DescribeMacro{\childdocmain}
To use the package, add the commands
\begin{center}
\begin{tabular}{l}
|\input{childdoc.def}|\\
|\childdocmain{}|\\
\end{tabular}
\end{center}
at the very top of the main \LaTeX{} file,
in particular \emph{before} the |\documentclass| statement!
The argument of |\childdocmain| should be left empty
(but it must be present).

%%%%%%%%%%%%%%%%%%%%%%%%%%%%%%%%%%%%%%%%
\DescribeMacro{\childdocof}
Furthermore, add the commands
\begin{center}
\begin{tabular}{l}
|\input{childdoc.def}|\\
|\childdocof{|\textit{main}|}|\\
\end{tabular}
\end{center}
at the top of every child file \textit{child}
which is included by |\include{|\textit{child}|}|
from within the main file
(or at least for those files to be compiled individually).
The argument \textit{main} must be the filename of the main file.

There are a couple of
considerations in setting up the main and child documents:

%%%%%%%%%%%%%%%%%%%%%%%%%%%%%%%%%%%%%%%%
\paragraph{Restrictions.}

Please note the following restrictions:
\begin{itemize}
\item
|\childdocmain| must be called with one argument \textit{main}
to ensure compatibility with earlier version of the package.
It must either be empty (|\childdocmain{}|)
or precisely match the filename of the main file in which it is specified.
See \secref{sec:detection} for further information.
\item
The filename \textit{main} must be specified without the |.tex| extension.
\item
The filename \textit{main} is case sensitive
(even in case-insensitive file systems)
due to internal string comparison.
\item
The argument \textit{main} should be fully expanded, it cannot be a macro.
\item
Subdirectories and special characters should be avoided in filenames.
\item
The command |\childdocmain{|\textit{main}|}| must be followed by a whitespace.
It should not be followed immediately by another command
or by a comment mark `|%|'.
This is because the \TeX{} parser reads the token immediately following
the argument of |\childdocmain| and puts it
at the beginning of every child section;
however, a white\-space is ignored.
\end{itemize}

%%%%%%%%%%%%%%%%%%%%%%%%%%%%%%%%%%%%%%%%
\paragraph{Content of Main File.}

It is advisable to place all content in the child files included by |\include|.
Any output contained in the main file will appear in all child documents
unless suppressed manually;
it cannot be suppressed automatically by the |\includeonly| directive
and thus should normally be avoided.
A method to include some content in the main file
by means of conditional processing is described in \secref{sec:conditional}.

%%%%%%%%%%%%%%%%%%%%%%%%%%%%%%%%%%%%%%%%
\paragraph{Page Numbering.}

When only a part of the document is compiled,
the appropriate numbering of pages
(as well as other status parameters)
is determined from the |.aux| files.
The latter contain information from previous passes.
However this information needs to propagate through
all intermediate child documents.
Therefore the page numbering in child documents may well
be inconsistent until the complete document is compiled at least once.

A useful (if unconventional) way to always ensure a consistent
page numbering is to restart the numbering in each child document
and denote the pages by `\textit{child}|.|\textit{page}'
where \textit{child} represents the chapter/section number of the child file.
This can be achieved by the command
|\numberwithin{page}{|\textit{child}|}|
of the \textsf{amsmath} package
where \textit{child} can be |chapter| or |section|
depending on the chosen structuring.
Alternatively, one can modify the macro |\thepage| appropriately
and reset the counter |page| at the start of each child file.

%%%%%%%%%%%%%%%%%%%%%%%%%%%%%%%%%%%%%%%%%%%%%%%%%%%%%%%%%%%%%%%%%%%%%%%%%%%%%%%%
\subsection{Conditional Processing}
\label{sec:conditional}

The package provides a mechanism to compile different versions
of a document. To customise the versions further some conditional processing
can come in handy to distinguish which version is being compiled.
The package provides two macros to describe the compilation context:

%%%%%%%%%%%%%%%%%%%%%%%%%%%%%%%%%%%%%%%%
\DescribeMacro{\ifchilddoc}
The conditional |\ifchilddoc| distinguishes between the compilation of
child documents and the main document:
%
\begin{center}
|\ifchilddoc |\textit{child-code}| |[|\||else |\textit{main-code}]| \||fi|
\end{center}

%%%%%%%%%%%%%%%%%%%%%%%%%%%%%%%%%%%%%%%%
\DescribeMacro{\childdocname}
\DescribeMacro{\childdocjob}
The macro |\childdocname| contains the filename (without extension)
of the main or child file being processed.
Note that |\childdocjob| will always contain the name of the main file.

%%%%%%%%%%%%%%%%%%%%%%%%%%%%%%%%%%%%%%%%
\paragraph{Title Page.}

Conditional processing can be used to include a title or banner page
in the main document when proper precautions are taken.
Importantly, the code in the main file should ensure that the page counter
(as well as other status parameters which are stored in the |.aux| files)
takes the same value after the conditional processing.
Otherwise the page numbers may take divergent values
depending on which part is compiled.

For example, a title page could be declared by:
%
\begin{center}
\begin{tabular}{l}
|\ifchilddoc\||else|\\
|\addtocounter{page}{-1}|\\
\textit{code for title page}\\
|\newpage|\\
|\||fi|
\end{tabular}
\end{center}
%
A banner page for the child documents can be generated by:
%
\begin{center}
\begin{tabular}{l}
|\ifchilddoc|\\
|\addtocounter{page}{-1}|\\
\textit{code for banner page}\\
|\newpage|\\
|\||fi|
\end{tabular}
\end{center}
%
Here one could write a message such as:
\begin{center}
|This is the part \childdocname{} of \childdocjob{}.|
\end{center}

%%%%%%%%%%%%%%%%%%%%%%%%%%%%%%%%%%%%%%%%%%%%%%%%%%%%%%%%%%%%%%%%%%%%%%%%%%%%%%%%
\subsection{Flags}
\label{sec:flags}

The package makes it easy to generate different versions
of the main or child documents.
To this end compilation flags can be defined
and assigned different default values.
They will be particularly useful in conjunction
with the forwarding mechanism described in \secref{sec:forward}.

For example, it may be useful to have a flag |\version|
which can be set to |draft| or |final|.
The document source will contain some conditional code
depending on the value of |\version|.
Suppose further, the flag should default to |final| for the main file
and to |draft| for child files
which is a natural assignment for editing the document.
This is achieved by placing the following code
in the preamble of the main document
(below the |\childdocmain| directive):
%
\begin{center}
\begin{tabular}{l}
|\ifchilddoc|\\
|\providecommand{\version}{draft}|\\
|\||else|\\
|\providecommand{\version}{final}|\\
|\||fi|
\end{tabular}
\end{center}
%
The definition by |\providecommand| makes sure
that previous definitions are not overwritten.
Further statements |\providecommand{\version}{...}|
can thus be added before the above code to override it.

For the main file, one might add a line
(between |\childdocmain| and the above block)
%
\begin{center}
|%\ifchilddoc\||else\providecommand{\version}{draft}\||fi|
\end{center}
%
which can be uncommented to produce a draft version.
Likewise one can add a line to the very top of a child file
(above the |\childdocof{|\textit{main}|}| directive)
%
\begin{center}
|%\providecommand{\version}{final}|
\end{center}
%
which can be uncommented to produce the final version of this child document.

%%%%%%%%%%%%%%%%%%%%%%%%%%%%%%%%%%%%%%%%%%%%%%%%%%%%%%%%%%%%%%%%%%%%%%%%%%%%%%%%
\subsection{Forwarding}
\label{sec:forward}

Different versions of the main or child documents
using compilation flags as described in \secref{sec:flags}
can be (permanently) stored in different files
for convenient compilation, viewing and distribution.
To this end, the package defines a command
to pass on compilation to a different file:

%%%%%%%%%%%%%%%%%%%%%%%%%%%%%%%%%%%%%%%%
\DescribeMacro{\childdocforward}
The command |\childdocforward| redirects processing to
another source file:
%
\begin{center}
\begin{tabular}{l}
|\input{childdoc.def}|\\
|\childdocforward[|\textit{main}|]{|\textit{dest}|}|\\
\end{tabular}
\end{center}
%
The argument \textit{dest} is the destination file
(without extension).
It should be the main file or one of the child files.
Note that further \textsf{childdoc} directives
such as |\childdocof| and |\childdocforward|
in the indicated file will be processed in this form.
The optional argument \textit{main}
passes on directly to the main file \textit{main}
while pretending to compile the child \textit{dest}.
This form behaves as if \textit{dest}
issues |\childdocof{|\textit{main}|}| right away,
and no further \textsf{childdoc} directives will be processed.

%%%%%%%%%%%%%%%%%%%%%%%%%%%%%%%%%%%%%%%%
\DescribeMacro{\...prefix}
In the alternative form |\childdocforwardprefix|,
%
\begin{center}
\begin{tabular}{l}
|\input{childdoc.def}|\\
|\childdocforwardprefix[|\textit{main}|]{|\textit{prefix}|}{|\textit{dest}|}|
\end{tabular}
\end{center}
%
the destination file is determined by a pattern
depending on the current file:
To make this work, the current file must be called
`{\textit{prefix}\hspace{0.2em}\textit{suffix}}'
with \textit{prefix} matching precisely the argument.
Processing is then passed on to the file
`{\textit{dest}\hspace{0.2em}\textit{suffix}}'.
Surely, the same effect is achieved by
directly specifying the
argument `{\textit{dest}\hspace{0.2em}\textit{suffix}}'
in the first form.
However, that requires to set up a different file
for each child. With the alternative form of the command
all these files can have exactly the same content
which simplifies setting them up and maintaining them.

For example, the following file |draft.tex|
with a compilation flag |\version| as described in \secref{sec:flags}
compiles the main document as a draft:
%
\begin{center}
\begin{tabular}{l}
|\def\version{draft}|\\
|\input{childdoc.def}|\\
|\childdocforward{|\textit{main}|}|
\end{tabular}
\end{center}
%
Likewise, the following files |final|\textit{nn}|.tex|
compile the final version of the child document
|child|\textit{nn}|.tex|:
%
\begin{center}
\begin{tabular}{l}
|\def\version{final}|\\
|\input{childdoc.def}|\\
|\childdocforwardprefix{final}{child}|
\end{tabular}
\end{center}
%

Note that when several versions of a main file and/or of each child file
are to be generated, it may be convenient to set up a |Makefile| or
shell script to automatise the process.

%%%%%%%%%%%%%%%%%%%%%%%%%%%%%%%%%%%%%%%%%%%%%%%%%%%%%%%%%%%%%%%%%%%%%%%%%%%%%%%%
\subsection{Command Line Processing}
\label{sec:commandline}

The effect of redirection files can also be achieved by invoking
the \LaTeX{} compiler with a more elaborate command line.
Most conveniently this should be done as part
of a shell script or a |Makefile|.

When using \textsf{childdoc} in the main file, the following
command lines effectively perform a redirection
(note that depending on the shell being used,
backslashes may have to be doubled: `|\|' $\to$ `|\\|'):
%
\begin{center}
|... -jobname "|\textit{target}|" |\\|"|[\textit{flags}]%
|\input{childdoc.def}\childdocforward[|\textit{main}|]{|\textit{dest}|}"|
\end{center}
%
Here \textit{target} is the name of the output file,
\textit{main} is the name of the main file
and \textit{dest} is the name of the main or child file to be processed
(all filenames without extensions).
The optional argument \textit{main} can be omitted
if \textit{main} matches \textit{dest}.
Optionally, compilation \textit{flags} can be defined via |\def| commands.
This command line makes the \TeX{} engine believe
it is compiling the file \textit{target}
whose content is specified as the latter parameter.
The provided code then forwards the processing to
\textit{main} or \textit{dest} as described in \secref{sec:forward}.

%%%%%%%%%%%%%%%%%%%%%%%%%%%%%%%%%%%%%%%%%%%%%%%%%%%%%%%%%%%%%%%%%%%%%%%%%%%%%%%%
\subsection{Include by Input}
\label{sec:input}

Including child documents by |\include| has some restrictions by design.
Most notably, the content of a child document always occupies
its own set of pages; pages cannot be shared between child documents.
Usually, this behaviour makes perfect sense
because each child document contain an essential part of the document.
However, in some situations it may be desirable to compose
a document from a collection of parts
without having mandatory page breaks between then.
For this case, the package
provides a mechanism to include parts
by |\input| which can also be processed individually.
However, by construction this mechanism
requires manual handling of the content to be output.

%%%%%%%%%%%%%%%%%%%%%%%%%%%%%%%%%%%%%%%%
\DescribeMacro{\ifchilddocmanual}
The main file should be prepared as usual, see \secref{sec:include}.
However, the document body must make a distinction
between processing of an individual part and of the main document, e.g.:
%
\begin{center}
\begin{tabular}{l}
|\ifchilddocmanual|\\
|\input{\childdocname}|\\
|\||else|\\
\textit{document body with }|\input{|\textit{part}|}|\\
|\||fi|
\end{tabular}
\end{center}
%
The conditional |\ifchilddocmanual| is true whenever
a part to be included by |\input| is being compiled,
and the name of the part is stored in |\childdocname|.

%%%%%%%%%%%%%%%%%%%%%%%%%%%%%%%%%%%%%%%%
\DescribeMacro{\childdocby}
Each part to be included by |\input| should start with:
%
\begin{center}
\begin{tabular}{l}
|\input{childdoc.def}|\\
|\childdocby{|\textit{main}|}|\\
\end{tabular}
\end{center}
%
The directive |\childdocby| is similar to |\childdocof|
described in \secref{sec:include},
but the subsequent selection of content must be done manually.
To that end, both |\ifchilddoc| and |\ifchilddocmanual|
will be true upon processing of a part,
and the name of the part is stored in |\childdocname|.
Note that |\jobname| will be set to the filename of the current part
so that each part receives an individual |.aux| file
that does not interfere with the |.aux| file(s) of the main document.
This behaviour can be altered by the alternative form
|\childdocby[*]{|\textit{main}|}| (with a non-empty optional argument)
which uses the |.aux| file of the main document
by setting |\jobname| to \textit{main}.

%%%%%%%%%%%%%%%%%%%%%%%%%%%%%%%%%%%%%%%%%%%%%%%%%%%%%%%%%%%%%%%%%%%%%%%%%%%%%%%%
\subsection{Driver Development}
\label{sec:driver}

The \textsf{childdoc} mechanism can also be use for the development
of definition files such as \LaTeX{} styles or classes.
This case differs from the above setup with multiple parts
included by |\include| in that no |\includeonly| should be invoked.
This can be achieved by starting the include file
(before |\ProvidesPackage|) with:
%
\begin{center}
\begin{tabular}{l}
|\input{childdoc.def}|\\
|\childdocforward{|\textit{main}|}|\\
\end{tabular}
\end{center}
%
or alternatively with:
%
\begin{center}
\begin{tabular}{l}
|\input{childdoc.def}|\\
|\childdocby{|\textit{main}|}|\\
\end{tabular}
\end{center}
%
Both forms have slightly different effects as described above.
The main file is prepared as usual, see \secref{sec:include}.

%%%%%%%%%%%%%%%%%%%%%%%%%%%%%%%%%%%%%%%%%%%%%%%%%%%%%%%%%%%%%%%%%%%%%%%%%%%%%%%%
\subsection{Legacy Detection}
\label{sec:detection}

The directive |\childdocmain| in the main file can detect
whether the complete document or merely a child is to be compiled
even without using the directive |\childdocof|.
This method is deprecated because it is less robust
and there is no compelling reason to use it;
it is merely provided for backward compatibility
and it may be removed in future versions.

If the detection mechanism is to be used,
it is mandatory to correctly specify
the filename of the main file as the argument of |\childdocmain|:
%
\begin{center}
\begin{tabular}{l}
|\input{childdoc.def}|\\
|\childdocmain{|\textit{main}|}|\\
\end{tabular}
\end{center}
%
If |\jobname| does not match the argument \textit{main} of |\childdocmain|,
it is assumed that |\jobname| points to the child file to be compiled.
When using |\childdocmain| with the main file specified as argument,
it suffices to start a child file
with just |\input{|\textit{main}|}|
without loading of the package and using |\childdocof|.
If instead all processing is done
with the appropriate \textsf{childdoc} directives,
the argument of \textit{main} of |\childdocmain| can be empty.

An alternative version of the command line processing described
in \secref{sec:commandline} using the detection mechanism reads:
%
\begin{center}
|... -jobname "|\textit{target}|" "|[\textit{flags}]%
[|\def\jobname{|\textit{dest}|}|]|\input{|\textit{main}|}"|
\end{center}

%%%%%%%%%%%%%%%%%%%%%%%%%%%%%%%%%%%%%%%%%%%%%%%%%%%%%%%%%%%%%%%%%%%%%%%%%%%%%%%%
\subsection{Manual Code}
\label{sec:manual}

In case one cannot be certain whether the definitions file |childdoc.def|
is installed on the target \TeX{} distribution
and one prefers not to ship it,
it is conceivable to paste a few relevant commands into the sources.

To that end, drop all statements |\input{childdoc.def}|
and perform the replacements as outlined below.
Instead of |\childdocmain{|\textit{main}|}| add the following code
to the top of the main file:
%
\begin{center}
\begin{tabular}{l}
|\||ifdefined\childdocname\endinput\||fi\newif\ifchilddoc|\\
|\edef\childdocname{\scantokens\expandafter{\jobname\noexpand}}|\\
|\def\childdocmain{|\textit{main}|}\||ifx\childdocmain\childdocname\||else|\\
|\childdoctrue\includeonly{\childdocname}\let\jobname\childdocmain\||fi|\\
\end{tabular}
\end{center}
%
Instead of |\childdocof{|\textit{main}|}| just include the main file
at the top of each child file:
%
\begin{center}
|\input{|\textit{main}|}|
\end{center}
%
A simple redirection |\childdocforward{|\textit{dest}|}| is achieved by:
%
\begin{center}
|\def\jobname{|\textit{dest}|}\input{\jobname}|
\end{center}
%
The redirection with prefix
|\childdocforwardprefix[|\textit{prefix}|]{|\textit{dest}|}|
is accomplished by:
%
\begin{center}
\begin{tabular}{l}
|{\edef\jobname{\scantokens\expandafter{\jobname\noexpand}}|\\
|\def\redirectjob |\textit{prefix}|#1~~~{\gdef\jobname{|\textit{dest}|#1}}|\\
|\expandafter\redirectjob\jobname~~~}\input{\jobname}|
\end{tabular}
\end{center}

In an alternative approach,
child documents can be compiled by a specific command line
without additional code or specific definitions:
%
\begin{center}
|... -jobname "|\textit{target}|" "|[\textit{flags}]%
|\includeonly{|\textit{dest}|}\input{|\textit{main}|}"|
\end{center}
%

%%%%%%%%%%%%%%%%%%%%%%%%%%%%%%%%%%%%%%%%%%%%%%%%%%%%%%%%%%%%%%%%%%%%%%%%%%%%%%%%
%%%%%%%%%%%%%%%%%%%%%%%%%%%%%%%%%%%%%%%%%%%%%%%%%%%%%%%%%%%%%%%%%%%%%%%%%%%%%%%%
\section{Information}

%%%%%%%%%%%%%%%%%%%%%%%%%%%%%%%%%%%%%%%%%%%%%%%%%%%%%%%%%%%%%%%%%%%%%%%%%%%%%%%%
\subsection{Copyright}

Copyright \copyright{} 2017--2018 Niklas Beisert

This work may be distributed and/or modified under the
conditions of the \LaTeX{} Project Public License, either version 1.3
of this license or (at your option) any later version.
The latest version of this license is in
  \url{http://www.latex-project.org/lppl.txt}
and version 1.3 or later is part of all distributions of \LaTeX{}
version 2005/12/01 or later.

This work has the LPPL maintenance status `maintained'.

The Current Maintainer of this work is Niklas Beisert.

This work consists of the files |README.txt|, |childdoc.ins| and |childdoc.dtx|
as well as the derived files |childdoc.def|, |cdocsamp.tex|
with |cdocsch1.tex|, |cdocsch2.tex|, |cdocspt3.tex|, |cdocspt4.tex|,
|cdocsdrf.tex|, |cdocsfn1.tex|, |cdocsfn2.tex|
as well as |childdoc.pdf|.

%%%%%%%%%%%%%%%%%%%%%%%%%%%%%%%%%%%%%%%%%%%%%%%%%%%%%%%%%%%%%%%%%%%%%%%%%%%%%%%%
\subsection{Files and Installation}

The package consists of the files:
%
\begin{center}
\begin{tabular}{ll}
    |README.txt|   & readme file \\
    |childdoc.ins| & installation file \\
    |childdoc.dtx| & source file \\
    |childdoc.def| & definition file \\
    |cdocsamp.tex| & sample main file \\
    |cdocsch1.tex| & sample include file \\
    |cdocsch2.tex| & sample include file \\
    |cdocspt3.tex| & sample part file \\
    |cdocspt4.tex| & sample part file \\
    |cdocsdrf.tex| & sample redirection file \\
    |cdocsfn1.tex| & sample redirection file \\
    |cdocsfn2.tex| & sample redirection file \\
    |childdoc.pdf| & manual
\end{tabular}
\end{center}
%
The distribution consists of the files
|README.txt|, |childdoc.ins| and |childdoc.dtx|.
%
\begin{itemize}
\item
Run (pdf)\LaTeX{} on |childdoc.dtx|
to compile the manual |childdoc.pdf| (this file).
\item
Run \LaTeX{} on |childdoc.ins| to create the definitions file |childdoc.def|
and the sample |cdocsamp.tex| with include files
|cdocsch1.tex|, |cdocsch2.tex|, |cdocspt3.tex|, |cdocspt4.tex|,
|cdocsdrf.tex|, |cdocsfn1.tex|, |cdocsfn2.tex|.
Then copy the file |childdoc.def| to an appropriate directory of your \LaTeX{}
distribution, e.g.\ \textit{texmf-root}|/tex/latex/childdoc|.
\end{itemize}

%%%%%%%%%%%%%%%%%%%%%%%%%%%%%%%%%%%%%%%%%%%%%%%%%%%%%%%%%%%%%%%%%%%%%%%%%%%%%%%%
\subsection{Related CTAN Packages}

There are several other packages which offer a similar functionality:
%
\begin{itemize}
\item
The packages
\href{http://ctan.org/pkg/docmute}{\textsf{docmute}},
\href{http://ctan.org/pkg/includex}{\textsf{includex}} and
\href{http://ctan.org/pkg/standalone}{\textsf{standalone}}
provide commands to include only the document body of
a child file thus allowing both files to be compiled individually.
\item
The packages \href{http://ctan.org/pkg/subdocs}{\textsf{subdocs}}
and \href{http://ctan.org/pkg/subfiles}{\textsf{subfiles}}
provide structures in which the main and child documents can be
encapsulated and allowing them to be compiled individually.
The inclusion mechanism is different from the conventional |\include|.
\item
The package \href{http://ctan.org/pkg/combine}{\textsf{combine}}
is an elaborate solution to combine several documents into one.
\end{itemize}
%
See also the CTAN topic \href{http://ctan.org/topic/subdocs}{\textsf{subdocs}}
for further related packages.
The present package differs from the above solutions in that
a document structure constructed with the conventional |\include| mechanism
just needs two extra commands at the top of every file
such that all constituent files can be compiled individually.

%%%%%%%%%%%%%%%%%%%%%%%%%%%%%%%%%%%%%%%%%%%%%%%%%%%%%%%%%%%%%%%%%%%%%%%%%%%%%%%%
%\subsection{Feature Suggestions}
%
%The following is a list of features which may be useful for future
%versions of this package:
%%
%\begin{itemize}
%\item
%\ldots
%\end{itemize}

%%%%%%%%%%%%%%%%%%%%%%%%%%%%%%%%%%%%%%%%%%%%%%%%%%%%%%%%%%%%%%%%%%%%%%%%%%%%%%%%
\subsection{Revision History}

%%%%%%%%%%%%%%%%%%%%%%%%%%%%%%%%%%%%%%%%
\paragraph{v2.0:} 2018/12/30

\begin{itemize}
\item
immediate forward processing
\item
added |\childdocby| mechanism
\item
manual restructured
\end{itemize}

%%%%%%%%%%%%%%%%%%%%%%%%%%%%%%%%%%%%%%%%
\paragraph{v1.6:} 2018/01/17

\begin{itemize}
\item
application for development of include files
\item
corrections to manual
\end{itemize}

%%%%%%%%%%%%%%%%%%%%%%%%%%%%%%%%%%%%%%%%
\paragraph{v1.5:} 2017/05/21

\begin{itemize}
\item
more complete structuring introduced
\item
|\childdocof| introduced
\item
|\childdoc| renamed to |\childdocmain|
\item
|\childredirect| renamed to |\childdocforward| and |\childdocforwardprefix|
and functionality expanded
\end{itemize}

%%%%%%%%%%%%%%%%%%%%%%%%%%%%%%%%%%%%%%%%
\paragraph{v1.0:} 2017/04/27

\begin{itemize}
\item
manual and install package
\item
first version published on CTAN
\end{itemize}

%%%%%%%%%%%%%%%%%%%%%%%%%%%%%%%%%%%%%%%%
\paragraph{v0.6:} 2017/04/26

\begin{itemize}
\item
redirection mechanism added
\end{itemize}

%%%%%%%%%%%%%%%%%%%%%%%%%%%%%%%%%%%%%%%%
\paragraph{v0.5:} 2017/04/26

\begin{itemize}
\item
functionality in definition file
\end{itemize}


%%%%%%%%%%%%%%%%%%%%%%%%%%%%%%%%%%%%%%%%%%%%%%%%%%%%%%%%%%%%%%%%%%%%%%%%%%%%%%%%
%%%%%%%%%%%%%%%%%%%%%%%%%%%%%%%%%%%%%%%%%%%%%%%%%%%%%%%%%%%%%%%%%%%%%%%%%%%%%%%%
%%%%%%%%%%%%%%%%%%%%%%%%%%%%%%%%%%%%%%%%%%%%%%%%%%%%%%%%%%%%%%%%%%%%%%%%%%%%%%%%
\appendix

\settowidth\MacroIndent{\rmfamily\scriptsize 000\ }

 \DocInput{childdoc.dtx}

\end{document}
%</driver>
% \fi
%
% %%%%%%%%%%%%%%%%%%%%%%%%%%%%%%%%%%%%%%%%%%%%%%%%%%%%%%%%%%%%%%%%%%%%%%%%%%%%%%
% %%%%%%%%%%%%%%%%%%%%%%%%%%%%%%%%%%%%%%%%%%%%%%%%%%%%%%%%%%%%%%%%%%%%%%%%%%%%%%
% \section{Sample}
%\iffalse
%<*samplemain>
%\fi
%
% The following presents a sample document
% with two chapters, two parts, a title page,
% a compile flag as well as three forwarding files to set the flag.
% It consists of eight |.tex| files:
% \begin{center}
% \begin{tabular}{ll}
% |cdocsamp.tex|&main file\\
% |cdocsch1.tex|&include file for chapter 1\\
% |cdocsch2.tex|&include file for chapter 2\\
% |cdocspt3.tex|&include file for part 3\\
% |cdocspt4.tex|&include file for part 4\\
% |cdocsdrf.tex|&forwarding file for main file in draft mode\\
% |cdocsfi1.tex|&forwarding file for final version of chapter 1\\
% |cdocsfi2.tex|&forwarding file for final version of chapter 2\\
% \end{tabular}
% \end{center}
% Each of the eight files can be compiled directly by the \LaTeX{} compiler.
%
% %%%%%%%%%%%%%%%%%%%%%%%%%%%%%%%%%%%%%%
% \paragraph{Main File.}
%
% The main file is called |cdocsamp.tex|.
%
% Load the \textsf{childdoc} definitions and
% declare the filename for the main document:
%    \begin{macrocode}
\input{childdoc.def}
\childdocmain{}
%    \end{macrocode}

% Optional override for |\version| flag:
%    \begin{macrocode}
%%\ifchilddoc\else\providecommand{\version}{draft}\fi
%    \end{macrocode}

% Define the default values for the |\version| flag
% (|final| for the main file and |draft| for childs):
%    \begin{macrocode}
\ifchilddoc
\providecommand{\version}{draft}
\else
\providecommand{\version}{final}
\fi
%    \end{macrocode}

% Load the standard document class:
%    \begin{macrocode}
\documentclass[12pt]{article}
%    \end{macrocode}

% Start the document body:
%    \begin{macrocode}
\begin{document}
%    \end{macrocode}

% Declare a title page.
% Print title, part of document being processed and version flag:
%    \begin{macrocode}
\addtocounter{page}{-1}
\begin{center}
{\LARGE\bfseries{}childdoc example\par}
\vspace{1cm}
\ifchilddoc
\ifchilddocmanual part\else chapter\fi:
`\childdocname' of `\childdocjob'\par
\else
main document: `\childdocjob'\par
\fi
version: \version\par
\end{center}
\newpage
%    \end{macrocode}

% Manually include selected file,
% otherwise process as usual:
%    \begin{macrocode}
\ifchilddocmanual
\section*{part `\childdocname'}
\input{\childdocname}
\else
%    \end{macrocode}

% Include the two chapters:
%    \begin{macrocode}
\include{cdocsch1}
\include{cdocsch2}
%    \end{macrocode}

% Include the two parts unless only chapters should be displayed:
%    \begin{macrocode}
\ifchilddoc\else
\section{part three}
\input{cdocspt3}
\section{part four}
\input{cdocspt4}
\fi
%    \end{macrocode}

% Process as usual until here:
%    \begin{macrocode}
\fi
%    \end{macrocode}

% End of document body:
%    \begin{macrocode}
\end{document}
%    \end{macrocode}
%\iffalse
%</samplemain>
%\fi
%
% %%%%%%%%%%%%%%%%%%%%%%%%%%%%%%%%%%%%%%
% \paragraph{Chapter Include Files.}
%
% The include files are called |cdocsch1.tex| and |cdocsch2.tex|.
%
%\iffalse
%<*samplechap1|samplechap2>
%\fi

% Optional override for |\version| flag:
%    \begin{macrocode}
%%\providecommand{\version}{final}
%    \end{macrocode}

% Include the main document:
%    \begin{macrocode}
\input{childdoc.def}
\childdocof{cdocsamp}
%    \end{macrocode}

%\iffalse
%</samplechap1|samplechap2>
%\fi
%
%\iffalse
%<*samplechap1>
%\fi
% Some text for chapter 1:
%    \begin{macrocode}
\section{one}
some text in chapter one
%    \end{macrocode}

%\iffalse
%</samplechap1>
%\fi
% Some text for chapter 2:
%\iffalse
%<*samplechap2>
%\fi
%    \begin{macrocode}
\section{two}
more text in chapter two
%    \end{macrocode}

%\iffalse
%</samplechap2>
%\fi
%
% %%%%%%%%%%%%%%%%%%%%%%%%%%%%%%%%%%%%%%
% \paragraph{Part Include Files.}
%
% The include files are called |cdocspt3.tex| and |cdocspt4.tex|.
%
%\iffalse
%<*samplepart3|samplepart4>
%\fi

% Optional override for |\version| flag:
%    \begin{macrocode}
%%\providecommand{\version}{final}
%    \end{macrocode}

% Include the main document:
%    \begin{macrocode}
\input{childdoc.def}
\childdocby{cdocsamp}
%    \end{macrocode}

%\iffalse
%</samplepart3|samplepart4>
%\fi
%
%\iffalse
%<*samplepart3>
%\fi
% Some text for part 3:
%    \begin{macrocode}
some text in part three
%    \end{macrocode}

%\iffalse
%</samplepart3>
%\fi
% Some text for part 4:
%\iffalse
%<*samplepart4>
%\fi
%    \begin{macrocode}
more text in part four
%    \end{macrocode}

%\iffalse
%</samplepart4>
%\fi
%
% %%%%%%%%%%%%%%%%%%%%%%%%%%%%%%%%%%%%%%
% \paragraph{Forwarding for a Complete Draft.}
%
% The following forwarding file |cdocsdrf.tex|
% compiles the main document in draft mode:
%\iffalse
%<*sampledraft>
%\fi
%    \begin{macrocode}
\def\version{draft}
\input{childdoc.def}
\childdocforward{cdocsamp}
%    \end{macrocode}

%\iffalse
%</sampledraft>
%\fi
%
% %%%%%%%%%%%%%%%%%%%%%%%%%%%%%%%%%%%%%%
% \paragraph{Forwarding for Final Version of the Chapters.}
%
% The following forwarding files |cdocsfn1.tex| and |cdocsfn2.tex|
% (with identical content)
% compile the final versions of the child documents
% |cdocsch1.tex| and |cdocsch2.tex|, respectively:
%\iffalse
%<*samplefinal>
%\fi
%    \begin{macrocode}
\def\version{final}
\input{childdoc.def}
\childdocforwardprefix[cdocsamp]{cdocsfn}{cdocsch}
%    \end{macrocode}

%\iffalse
%</samplefinal>
%\fi
%
% %%%%%%%%%%%%%%%%%%%%%%%%%%%%%%%%%%%%%%
% \paragraph{Command Line Processing.}
%
% The following three command lines generate the output files
% |cdocscld|, |cdocscl1| and |cdocscl2|
% which should be identical to
% |cdocsdrf|, |cdocsch1| and |cdocsfn2|, respectively:
% \begin{center}
% \begin{tabular}{l}
% |latex -jobname cdocscld \|\\
% |  "\def\version{draft}\input{childdoc.def}\childdocforward{cdocsamp}"|\\
% |latex -jobname cdocscl1 \|\\
% |  "\input{childdoc.def}\childdocforward[cdocsamp]{cdocsch1}"|\\
% |latex -jobname cdocscl2 \|\\
% |  "\def\version{final}\input{childdoc.def}\childdocforward{cdocsch2}"|
% \end{tabular}
% \end{center}
% Note that the trailing backslash on each first line
% merely continues the input to the second line
% (for convenient cut ant paste).
% Furthermore, the command |latex| can be replaced by any
% of its alternative versions such as |pdflatex|.
%
% %%%%%%%%%%%%%%%%%%%%%%%%%%%%%%%%%%%%%%%%%%%%%%%%%%%%%%%%%%%%%%%%%%%%%%%%%%%%%%
% %%%%%%%%%%%%%%%%%%%%%%%%%%%%%%%%%%%%%%%%%%%%%%%%%%%%%%%%%%%%%%%%%%%%%%%%%%%%%%
% \section{Implementation}
%\iffalse
%<*package>
%\fi
%
% This section describes the definitions file |childdoc.def|.

% The definitions cannot be loaded using |\usepackage| or |\RequirePackage|
% which has a mechanism to prevent loading a style file more than once.
% When loading the definitions by means of |\input|
% multiple instances have to be prevented manually:
%\iffalse
%This code needs to be before the `\ProvidesFile' directive
%which is defined at the beginning of this file.
%Therefore it is also placed there and commented out here.
%</package>
%<*discard>
%\fi
%    \begin{macrocode}
\ifdefined\childdocmain\endinput\fi
%    \end{macrocode}
%\iffalse
%</discard>
%<*package>
%\fi
%
% \macro{\ifchilddoc}
% \macro{\ifchilddocmanual}
% The conditional |\ifchilddoc| tells whether a
% child (true) or main (false) document is being compiled.
% The conditional |\ifchilddocmanual| tells whether
% the |\includeonly| mechanism is used (false) or
% the selection of child files must be performed manually (true).
% The definitions initialise to false:
%    \begin{macrocode}
\newif\ifchilddoc
\newif\ifchilddocmanual
%    \end{macrocode}

% \macro{\childdocname}
% \macro{\childdocjob}
% The macro |\childdocname| stores the name of the main document
% to be compiled. The macro |\childdocjob| stores the name of
% the document on which the \LaTeX{} compiler was originally invoked.
% The content of |\jobname| cannot be compared
% to filenames specified in the source due to different catcodes.
% The following code rescans |\jobname|, stores the result
% in |\childdocname| and saves a copy in |\childdocjob|:
%    \begin{macrocode}
\edef\childdocname{\scantokens\expandafter{\jobname\noexpand}}
\let\childdocjob\childdocname
%    \end{macrocode}

% \macro{\childdocdisable}
% The macro |\childdocdisable| prevents the main file
% from being processed more than once.
% At this stage, the main document command |\childdocmain|
% is assumed to be called once again where it should do nothing.
% Any subsequent call to it should prevent
% a secondary processing of the main document
% It overwrites the forwarding commands
% |\childdocof| and |\childdocforward|
% with empty macros to prevent further inclusions of the main document:
%    \begin{macrocode}
\newcommand{\childdocdisable}
{
  \renewcommand{\childdocmain}[1]{\renewcommand{\childdocmain}[1]{\endinput}}
  \renewcommand{\childdocof}[1]{}
  \renewcommand{\childdocby}[2][]{}
  \renewcommand{\childdocforward}[2][]{}
  \renewcommand{\childdocdisable}{}
}
%    \end{macrocode}

% \macro{\childdocmain}
% The macro |\childdocmain| is to be called at the top of the main file
% with nothing or the main filename (without extension) as argument.
% First, it breaks loops.
% If the argument is not empty and does not match |\childdocname|
% (which is set by the first inclusion of |childdoc.def|),
% |\ifchilddoc| is set to true, |\includeonly| is applied to the child file
% and |\jobname| is set to the main file
% (for proper handling of |.aux| files):
%    \begin{macrocode}
\newcommand{\childdocmain}[1]
{
  \childdocdisable\childdocmain{}
  \if?#1?\else
    \begingroup
      \def\childdoctmp{#1}
      \ifx\childdoctmp\childdocname
        \def\childdoctmp{}
      \else
        \def\childdoctmp
        {
          \childdoctrue
          \includeonly{\childdocname}
          \def\childdocjob{#1}
          \def\jobname{#1}
        }
      \fi
      \expandafter
    \endgroup
    \childdoctmp
  \fi
}
%    \end{macrocode}

% \macro{\childdocof}
% The command |\childdocof| redirects
% compilation to the main file |#1|.
%    \begin{macrocode}
\newcommand{\childdocof}[1]
{
  \childdocdisable
  \childdoctrue
  \includeonly{\childdocname}
  \def\jobname{#1}
  \def\childdocjob{#1}
  \input{#1}
}
%    \end{macrocode}

% \macro{\childdocby}
% The command |\childdocby| ....
%    \begin{macrocode}
\newcommand{\childdocby}[2][]
{
  \childdocdisable
  \childdoctrue
  \childdocmanualtrue
  \if?#1?\else
    \def\jobname{#2}
  \fi
  \def\childdocjob{#2}
  \input{#2}
  \endinput
}
%    \end{macrocode}

% \macro{\childdocforward}
% The command |\childdocforward| redirects
% compilation to the main file or
% (if the optional argument is given) a child file.
% Parameters are set as if the main file
% or a child file starting with |\childdocof| was compiled.
% Then compilation is handed over to the main file:
%    \begin{macrocode}
\newcommand{\childdocforward}[2][]
{
  \begingroup
    \if?#1?
      \def\childdoctmp
      {
        \def\childdocname{#2}
        \def\childdocjob{#2}
        \def\jobname{#2}
        \input{#2}
        \endinput
      }
    \else
      \def\childdoctmp
      {
        \childdocdisable
        \def\childdocname{#2}
        \childdoctrue
        \includeonly{#2}
        \def\childdocjob{#1}
        \def\jobname{#1}
        \input{#1}
        \endinput
      }
    \fi
    \expandafter
  \endgroup
  \childdoctmp
}
%    \end{macrocode}

% \macro{\childdocforwardprefix}
% The command |\childdocforwardprefix| redirects
% compilation to the main or a child file by means of a pattern.
% The prefix |#1| in the current filename is replaced by |#2|
% and the suffix of the current filename is kept
% (it is assumed that the filename does not contain the substring `|~~~|'
% which is used as a delimiter).
% Compilation is handed over to the new file by |\childdocforward|:
%    \begin{macrocode}
\newcommand{\childdocforwardprefix}[3][]
{
  \begingroup
    \def\childdocextract #2##1~~~{\def\childdoctmp{\childdocforward[#1]{#3##1}}}
    \expandafter\childdocextract\childdocname~~~
    \expandafter
  \endgroup
  \childdoctmp
}
%    \end{macrocode}

% \macro{\childdoc}
% The deprecated macro |\childdoc| is a legacy version of |\childdocmain|:
%    \begin{macrocode}
\newcommand{\childdoc}{\childdocmain}
%    \end{macrocode}

% \macro{\childdocredirect}
% The deprecated macro |\childdocredirect| is a legacy version
% of |\childdocforward| and |\childdocforwardprefix|:
%    \begin{macrocode}
\newcommand{\childdocredirect}[2][]
{
  \begingroup
    \if?#1?
      \def\childdoctmp{\childdocforward{#2}}
    \else
      \def\childdoctmp{\childdocforwardprefix{#1}{#2}}
    \fi
    \expandafter
  \endgroup
  \childdoctmp
}
%    \end{macrocode}

%\iffalse
%</package>
%\fi
%
\endinput

\childdocof{cdocsamp}
%    \end{macrocode}

%\iffalse
%</samplechap1|samplechap2>
%\fi
%
%\iffalse
%<*samplechap1>
%\fi
% Some text for chapter 1:
%    \begin{macrocode}
\section{one}
some text in chapter one
%    \end{macrocode}

%\iffalse
%</samplechap1>
%\fi
% Some text for chapter 2:
%\iffalse
%<*samplechap2>
%\fi
%    \begin{macrocode}
\section{two}
more text in chapter two
%    \end{macrocode}

%\iffalse
%</samplechap2>
%\fi
%
% %%%%%%%%%%%%%%%%%%%%%%%%%%%%%%%%%%%%%%
% \paragraph{Part Include Files.}
%
% The include files are called |cdocspt3.tex| and |cdocspt4.tex|.
%
%\iffalse
%<*samplepart3|samplepart4>
%\fi

% Optional override for |\version| flag:
%    \begin{macrocode}
%%\providecommand{\version}{final}
%    \end{macrocode}

% Include the main document:
%    \begin{macrocode}
% \iffalse
%
% childdoc.dtx Copyright (C) 2017-2018 Niklas Beisert
%
% This work may be distributed and/or modified under the
% conditions of the LaTeX Project Public License, either version 1.3
% of this license or (at your option) any later version.
% The latest version of this license is in
%   http://www.latex-project.org/lppl.txt
% and version 1.3 or later is part of all distributions of LaTeX
% version 2005/12/01 or later.
%
% This work has the LPPL maintenance status `maintained'.
%
% The Current Maintainer of this work is Niklas Beisert.
%
% This work consists of the files childdoc.dtx and childdoc.ins
% and the derived files childdoc.def and cdocsamp.tex with
% cdocsch1.tex, cdocsch2.tex, cdocsdrf.tex, cdocsfn1.tex, cdocsfn2.tex.
%
%<package>\ifdefined\childdocmain\endinput\fi
%<package>\ProvidesFile{childdoc.def}[2018/12/30 v2.0 child document driver]
%<samplemain>\ProvidesFile{cdocsamp.tex}[2018/12/30 v2.0 sample for childdoc]
%<*driver>
%\ProvidesFile{childdoc.drv}[2018/12/30 v2.0 childdoc reference manual file]
\PassOptionsToClass{10pt,a4paper}{article}
\documentclass{ltxdoc}

\usepackage[margin=35mm]{geometry}
\usepackage{hyperref}
\usepackage{hyperxmp}
\usepackage[usenames]{color}

\hypersetup{colorlinks=true}
\hypersetup{pdfstartview=FitH}
\hypersetup{pdfpagemode=UseNone}
\hypersetup{pdfsource={}}
\hypersetup{pdflang={en-UK}}
\hypersetup{pdfcopyright={Copyright 2017-2018 Niklas Beisert.
  This work may be distributed and/or modified under the
  conditions of the LaTeX Project Public License, either version 1.3
  of this license or (at your option) any later version.}}
\hypersetup{pdflicenseurl={http://www.latex-project.org/lppl.txt}}
\hypersetup{pdfcontactaddress={ETH Zurich, ITP, HIT K,
  Wolfgang-Pauli-Strasse 27}}
\hypersetup{pdfcontactpostcode={8093}}
\hypersetup{pdfcontactcity={Zurich}}
\hypersetup{pdfcontactcountry={Switzerland}}
\hypersetup{pdfcontactemail={nbeisert@itp.phys.ethz.ch}}
\hypersetup{pdfcontacturl={http://people.phys.ethz.ch/\xmptilde nbeisert/}}

\newcommand{\secref}[1]{\hyperref[#1]{section \ref*{#1}}}

\parskip1ex
\parindent0pt
\let\olditemize\itemize
\def\itemize{\olditemize\parskip0pt}

\begin{document}

\title{The \textsf{childdoc} Package}
\hypersetup{pdftitle={The childdoc Package}}
\author{Niklas Beisert\\[2ex]
  Institut f\"ur Theoretische Physik\\
  Eidgen\"ossische Technische Hochschule Z\"urich\\
  Wolfgang-Pauli-Strasse 27, 8093 Z\"urich, Switzerland\\[1ex]
  \href{mailto:nbeisert@itp.phys.ethz.ch}
  {\texttt{nbeisert@itp.phys.ethz.ch}}}
\hypersetup{pdfauthor={Niklas Beisert}}
\hypersetup{pdfsubject={Manual for the LaTeX2e Package childdoc}}
\date{30 December 2018, \textsf{v2.0}}
\maketitle

\begin{abstract}\noindent
\textsf{childdoc} is a \LaTeXe{} package
that enables the direct compilation
of document sections included by |\include|
to individual files.
\end{abstract}

\begingroup
\parskip0ex
\tableofcontents
\endgroup

%%%%%%%%%%%%%%%%%%%%%%%%%%%%%%%%%%%%%%%%%%%%%%%%%%%%%%%%%%%%%%%%%%%%%%%%%%%%%%%%
%%%%%%%%%%%%%%%%%%%%%%%%%%%%%%%%%%%%%%%%%%%%%%%%%%%%%%%%%%%%%%%%%%%%%%%%%%%%%%%%
\section{Introduction}

\LaTeX{} provides a mechanism to structure a large document (such as a book)
into a main file and several child files (containing the chapters)
using the |\include| command.
This mechanism is beneficial for documents
which span hundreds of pages in order to
make the source file(s) more manageable.
Moreover, compilation can be restricted to
selected child files by means of the |\includeonly| command.
The latter feature can be used to reduce the compilation time while editing
(this was significantly more useful in the earlier days of \LaTeX{})
or to generate a smaller document which is easier to navigate.
Another application of |\includeonly| is to generate
documents consisting of selected parts of the complete document.

However, there are a few drawbacks of the plain |\include| mechanism:
\begin{itemize}
\item
The child files cannot be compiled on their own,
they can only be compiled via the main file.
A naive editing environment
(such as a text editor with an option
to have the current file processed by \LaTeX)
may require one to switch to the main file before compiling;
attempting to compile the child file produces errors.
\item
The main file must be modified (each time)
to adjust the |\includeonly| command
to the present needs. This easily leaves the main file in a messy state.
\item
The generated document will always carry the filename
of the main document. This is inconvenient if
several child files are to be compiled and
to be kept for distribution.
\end{itemize}

The present package provides a simple interface
to make child files individually compilable by \LaTeX{}.
Compiling a child file then has the same effect as compiling
the main file with an |\includeonly| command
to select the appropriate child.
Moreover the generated document will carry the name of the child
rather than the main file.
This resolves all three above issues.

This feature is meant to make the editing of books,
thesis documents and lecture notes somewhat more convenient.
However, the package can also be used efficiently for
composing a series of documents (such as exercise sheets)
which are typically distributed individually.
It then assists the author in generating the individual documents
(potentially in different versions)
as well as a document containing the collected series.
Another application is in developing style files
or other kinds of included material
where compilation of the style file could redirect
to a sample or test file.

%%%%%%%%%%%%%%%%%%%%%%%%%%%%%%%%%%%%%%%%%%%%%%%%%%%%%%%%%%%%%%%%%%%%%%%%%%%%%%%%
%%%%%%%%%%%%%%%%%%%%%%%%%%%%%%%%%%%%%%%%%%%%%%%%%%%%%%%%%%%%%%%%%%%%%%%%%%%%%%%%
\section{Usage}

First of all, the package \textsf{childdoc} is \emph{not} a standard
\LaTeXe{} |.sty| style file! Therefore it needs to be invoked in
a non-standard way.

%%%%%%%%%%%%%%%%%%%%%%%%%%%%%%%%%%%%%%%%%%%%%%%%%%%%%%%%%%%%%%%%%%%%%%%%%%%%%%%%
\subsection{Included Files}
\label{sec:include}

%%%%%%%%%%%%%%%%%%%%%%%%%%%%%%%%%%%%%%%%
\DescribeMacro{\childdocmain}
To use the package, add the commands
\begin{center}
\begin{tabular}{l}
|\input{childdoc.def}|\\
|\childdocmain{}|\\
\end{tabular}
\end{center}
at the very top of the main \LaTeX{} file,
in particular \emph{before} the |\documentclass| statement!
The argument of |\childdocmain| should be left empty
(but it must be present).

%%%%%%%%%%%%%%%%%%%%%%%%%%%%%%%%%%%%%%%%
\DescribeMacro{\childdocof}
Furthermore, add the commands
\begin{center}
\begin{tabular}{l}
|\input{childdoc.def}|\\
|\childdocof{|\textit{main}|}|\\
\end{tabular}
\end{center}
at the top of every child file \textit{child}
which is included by |\include{|\textit{child}|}|
from within the main file
(or at least for those files to be compiled individually).
The argument \textit{main} must be the filename of the main file.

There are a couple of
considerations in setting up the main and child documents:

%%%%%%%%%%%%%%%%%%%%%%%%%%%%%%%%%%%%%%%%
\paragraph{Restrictions.}

Please note the following restrictions:
\begin{itemize}
\item
|\childdocmain| must be called with one argument \textit{main}
to ensure compatibility with earlier version of the package.
It must either be empty (|\childdocmain{}|)
or precisely match the filename of the main file in which it is specified.
See \secref{sec:detection} for further information.
\item
The filename \textit{main} must be specified without the |.tex| extension.
\item
The filename \textit{main} is case sensitive
(even in case-insensitive file systems)
due to internal string comparison.
\item
The argument \textit{main} should be fully expanded, it cannot be a macro.
\item
Subdirectories and special characters should be avoided in filenames.
\item
The command |\childdocmain{|\textit{main}|}| must be followed by a whitespace.
It should not be followed immediately by another command
or by a comment mark `|%|'.
This is because the \TeX{} parser reads the token immediately following
the argument of |\childdocmain| and puts it
at the beginning of every child section;
however, a white\-space is ignored.
\end{itemize}

%%%%%%%%%%%%%%%%%%%%%%%%%%%%%%%%%%%%%%%%
\paragraph{Content of Main File.}

It is advisable to place all content in the child files included by |\include|.
Any output contained in the main file will appear in all child documents
unless suppressed manually;
it cannot be suppressed automatically by the |\includeonly| directive
and thus should normally be avoided.
A method to include some content in the main file
by means of conditional processing is described in \secref{sec:conditional}.

%%%%%%%%%%%%%%%%%%%%%%%%%%%%%%%%%%%%%%%%
\paragraph{Page Numbering.}

When only a part of the document is compiled,
the appropriate numbering of pages
(as well as other status parameters)
is determined from the |.aux| files.
The latter contain information from previous passes.
However this information needs to propagate through
all intermediate child documents.
Therefore the page numbering in child documents may well
be inconsistent until the complete document is compiled at least once.

A useful (if unconventional) way to always ensure a consistent
page numbering is to restart the numbering in each child document
and denote the pages by `\textit{child}|.|\textit{page}'
where \textit{child} represents the chapter/section number of the child file.
This can be achieved by the command
|\numberwithin{page}{|\textit{child}|}|
of the \textsf{amsmath} package
where \textit{child} can be |chapter| or |section|
depending on the chosen structuring.
Alternatively, one can modify the macro |\thepage| appropriately
and reset the counter |page| at the start of each child file.

%%%%%%%%%%%%%%%%%%%%%%%%%%%%%%%%%%%%%%%%%%%%%%%%%%%%%%%%%%%%%%%%%%%%%%%%%%%%%%%%
\subsection{Conditional Processing}
\label{sec:conditional}

The package provides a mechanism to compile different versions
of a document. To customise the versions further some conditional processing
can come in handy to distinguish which version is being compiled.
The package provides two macros to describe the compilation context:

%%%%%%%%%%%%%%%%%%%%%%%%%%%%%%%%%%%%%%%%
\DescribeMacro{\ifchilddoc}
The conditional |\ifchilddoc| distinguishes between the compilation of
child documents and the main document:
%
\begin{center}
|\ifchilddoc |\textit{child-code}| |[|\||else |\textit{main-code}]| \||fi|
\end{center}

%%%%%%%%%%%%%%%%%%%%%%%%%%%%%%%%%%%%%%%%
\DescribeMacro{\childdocname}
\DescribeMacro{\childdocjob}
The macro |\childdocname| contains the filename (without extension)
of the main or child file being processed.
Note that |\childdocjob| will always contain the name of the main file.

%%%%%%%%%%%%%%%%%%%%%%%%%%%%%%%%%%%%%%%%
\paragraph{Title Page.}

Conditional processing can be used to include a title or banner page
in the main document when proper precautions are taken.
Importantly, the code in the main file should ensure that the page counter
(as well as other status parameters which are stored in the |.aux| files)
takes the same value after the conditional processing.
Otherwise the page numbers may take divergent values
depending on which part is compiled.

For example, a title page could be declared by:
%
\begin{center}
\begin{tabular}{l}
|\ifchilddoc\||else|\\
|\addtocounter{page}{-1}|\\
\textit{code for title page}\\
|\newpage|\\
|\||fi|
\end{tabular}
\end{center}
%
A banner page for the child documents can be generated by:
%
\begin{center}
\begin{tabular}{l}
|\ifchilddoc|\\
|\addtocounter{page}{-1}|\\
\textit{code for banner page}\\
|\newpage|\\
|\||fi|
\end{tabular}
\end{center}
%
Here one could write a message such as:
\begin{center}
|This is the part \childdocname{} of \childdocjob{}.|
\end{center}

%%%%%%%%%%%%%%%%%%%%%%%%%%%%%%%%%%%%%%%%%%%%%%%%%%%%%%%%%%%%%%%%%%%%%%%%%%%%%%%%
\subsection{Flags}
\label{sec:flags}

The package makes it easy to generate different versions
of the main or child documents.
To this end compilation flags can be defined
and assigned different default values.
They will be particularly useful in conjunction
with the forwarding mechanism described in \secref{sec:forward}.

For example, it may be useful to have a flag |\version|
which can be set to |draft| or |final|.
The document source will contain some conditional code
depending on the value of |\version|.
Suppose further, the flag should default to |final| for the main file
and to |draft| for child files
which is a natural assignment for editing the document.
This is achieved by placing the following code
in the preamble of the main document
(below the |\childdocmain| directive):
%
\begin{center}
\begin{tabular}{l}
|\ifchilddoc|\\
|\providecommand{\version}{draft}|\\
|\||else|\\
|\providecommand{\version}{final}|\\
|\||fi|
\end{tabular}
\end{center}
%
The definition by |\providecommand| makes sure
that previous definitions are not overwritten.
Further statements |\providecommand{\version}{...}|
can thus be added before the above code to override it.

For the main file, one might add a line
(between |\childdocmain| and the above block)
%
\begin{center}
|%\ifchilddoc\||else\providecommand{\version}{draft}\||fi|
\end{center}
%
which can be uncommented to produce a draft version.
Likewise one can add a line to the very top of a child file
(above the |\childdocof{|\textit{main}|}| directive)
%
\begin{center}
|%\providecommand{\version}{final}|
\end{center}
%
which can be uncommented to produce the final version of this child document.

%%%%%%%%%%%%%%%%%%%%%%%%%%%%%%%%%%%%%%%%%%%%%%%%%%%%%%%%%%%%%%%%%%%%%%%%%%%%%%%%
\subsection{Forwarding}
\label{sec:forward}

Different versions of the main or child documents
using compilation flags as described in \secref{sec:flags}
can be (permanently) stored in different files
for convenient compilation, viewing and distribution.
To this end, the package defines a command
to pass on compilation to a different file:

%%%%%%%%%%%%%%%%%%%%%%%%%%%%%%%%%%%%%%%%
\DescribeMacro{\childdocforward}
The command |\childdocforward| redirects processing to
another source file:
%
\begin{center}
\begin{tabular}{l}
|\input{childdoc.def}|\\
|\childdocforward[|\textit{main}|]{|\textit{dest}|}|\\
\end{tabular}
\end{center}
%
The argument \textit{dest} is the destination file
(without extension).
It should be the main file or one of the child files.
Note that further \textsf{childdoc} directives
such as |\childdocof| and |\childdocforward|
in the indicated file will be processed in this form.
The optional argument \textit{main}
passes on directly to the main file \textit{main}
while pretending to compile the child \textit{dest}.
This form behaves as if \textit{dest}
issues |\childdocof{|\textit{main}|}| right away,
and no further \textsf{childdoc} directives will be processed.

%%%%%%%%%%%%%%%%%%%%%%%%%%%%%%%%%%%%%%%%
\DescribeMacro{\...prefix}
In the alternative form |\childdocforwardprefix|,
%
\begin{center}
\begin{tabular}{l}
|\input{childdoc.def}|\\
|\childdocforwardprefix[|\textit{main}|]{|\textit{prefix}|}{|\textit{dest}|}|
\end{tabular}
\end{center}
%
the destination file is determined by a pattern
depending on the current file:
To make this work, the current file must be called
`{\textit{prefix}\hspace{0.2em}\textit{suffix}}'
with \textit{prefix} matching precisely the argument.
Processing is then passed on to the file
`{\textit{dest}\hspace{0.2em}\textit{suffix}}'.
Surely, the same effect is achieved by
directly specifying the
argument `{\textit{dest}\hspace{0.2em}\textit{suffix}}'
in the first form.
However, that requires to set up a different file
for each child. With the alternative form of the command
all these files can have exactly the same content
which simplifies setting them up and maintaining them.

For example, the following file |draft.tex|
with a compilation flag |\version| as described in \secref{sec:flags}
compiles the main document as a draft:
%
\begin{center}
\begin{tabular}{l}
|\def\version{draft}|\\
|\input{childdoc.def}|\\
|\childdocforward{|\textit{main}|}|
\end{tabular}
\end{center}
%
Likewise, the following files |final|\textit{nn}|.tex|
compile the final version of the child document
|child|\textit{nn}|.tex|:
%
\begin{center}
\begin{tabular}{l}
|\def\version{final}|\\
|\input{childdoc.def}|\\
|\childdocforwardprefix{final}{child}|
\end{tabular}
\end{center}
%

Note that when several versions of a main file and/or of each child file
are to be generated, it may be convenient to set up a |Makefile| or
shell script to automatise the process.

%%%%%%%%%%%%%%%%%%%%%%%%%%%%%%%%%%%%%%%%%%%%%%%%%%%%%%%%%%%%%%%%%%%%%%%%%%%%%%%%
\subsection{Command Line Processing}
\label{sec:commandline}

The effect of redirection files can also be achieved by invoking
the \LaTeX{} compiler with a more elaborate command line.
Most conveniently this should be done as part
of a shell script or a |Makefile|.

When using \textsf{childdoc} in the main file, the following
command lines effectively perform a redirection
(note that depending on the shell being used,
backslashes may have to be doubled: `|\|' $\to$ `|\\|'):
%
\begin{center}
|... -jobname "|\textit{target}|" |\\|"|[\textit{flags}]%
|\input{childdoc.def}\childdocforward[|\textit{main}|]{|\textit{dest}|}"|
\end{center}
%
Here \textit{target} is the name of the output file,
\textit{main} is the name of the main file
and \textit{dest} is the name of the main or child file to be processed
(all filenames without extensions).
The optional argument \textit{main} can be omitted
if \textit{main} matches \textit{dest}.
Optionally, compilation \textit{flags} can be defined via |\def| commands.
This command line makes the \TeX{} engine believe
it is compiling the file \textit{target}
whose content is specified as the latter parameter.
The provided code then forwards the processing to
\textit{main} or \textit{dest} as described in \secref{sec:forward}.

%%%%%%%%%%%%%%%%%%%%%%%%%%%%%%%%%%%%%%%%%%%%%%%%%%%%%%%%%%%%%%%%%%%%%%%%%%%%%%%%
\subsection{Include by Input}
\label{sec:input}

Including child documents by |\include| has some restrictions by design.
Most notably, the content of a child document always occupies
its own set of pages; pages cannot be shared between child documents.
Usually, this behaviour makes perfect sense
because each child document contain an essential part of the document.
However, in some situations it may be desirable to compose
a document from a collection of parts
without having mandatory page breaks between then.
For this case, the package
provides a mechanism to include parts
by |\input| which can also be processed individually.
However, by construction this mechanism
requires manual handling of the content to be output.

%%%%%%%%%%%%%%%%%%%%%%%%%%%%%%%%%%%%%%%%
\DescribeMacro{\ifchilddocmanual}
The main file should be prepared as usual, see \secref{sec:include}.
However, the document body must make a distinction
between processing of an individual part and of the main document, e.g.:
%
\begin{center}
\begin{tabular}{l}
|\ifchilddocmanual|\\
|\input{\childdocname}|\\
|\||else|\\
\textit{document body with }|\input{|\textit{part}|}|\\
|\||fi|
\end{tabular}
\end{center}
%
The conditional |\ifchilddocmanual| is true whenever
a part to be included by |\input| is being compiled,
and the name of the part is stored in |\childdocname|.

%%%%%%%%%%%%%%%%%%%%%%%%%%%%%%%%%%%%%%%%
\DescribeMacro{\childdocby}
Each part to be included by |\input| should start with:
%
\begin{center}
\begin{tabular}{l}
|\input{childdoc.def}|\\
|\childdocby{|\textit{main}|}|\\
\end{tabular}
\end{center}
%
The directive |\childdocby| is similar to |\childdocof|
described in \secref{sec:include},
but the subsequent selection of content must be done manually.
To that end, both |\ifchilddoc| and |\ifchilddocmanual|
will be true upon processing of a part,
and the name of the part is stored in |\childdocname|.
Note that |\jobname| will be set to the filename of the current part
so that each part receives an individual |.aux| file
that does not interfere with the |.aux| file(s) of the main document.
This behaviour can be altered by the alternative form
|\childdocby[*]{|\textit{main}|}| (with a non-empty optional argument)
which uses the |.aux| file of the main document
by setting |\jobname| to \textit{main}.

%%%%%%%%%%%%%%%%%%%%%%%%%%%%%%%%%%%%%%%%%%%%%%%%%%%%%%%%%%%%%%%%%%%%%%%%%%%%%%%%
\subsection{Driver Development}
\label{sec:driver}

The \textsf{childdoc} mechanism can also be use for the development
of definition files such as \LaTeX{} styles or classes.
This case differs from the above setup with multiple parts
included by |\include| in that no |\includeonly| should be invoked.
This can be achieved by starting the include file
(before |\ProvidesPackage|) with:
%
\begin{center}
\begin{tabular}{l}
|\input{childdoc.def}|\\
|\childdocforward{|\textit{main}|}|\\
\end{tabular}
\end{center}
%
or alternatively with:
%
\begin{center}
\begin{tabular}{l}
|\input{childdoc.def}|\\
|\childdocby{|\textit{main}|}|\\
\end{tabular}
\end{center}
%
Both forms have slightly different effects as described above.
The main file is prepared as usual, see \secref{sec:include}.

%%%%%%%%%%%%%%%%%%%%%%%%%%%%%%%%%%%%%%%%%%%%%%%%%%%%%%%%%%%%%%%%%%%%%%%%%%%%%%%%
\subsection{Legacy Detection}
\label{sec:detection}

The directive |\childdocmain| in the main file can detect
whether the complete document or merely a child is to be compiled
even without using the directive |\childdocof|.
This method is deprecated because it is less robust
and there is no compelling reason to use it;
it is merely provided for backward compatibility
and it may be removed in future versions.

If the detection mechanism is to be used,
it is mandatory to correctly specify
the filename of the main file as the argument of |\childdocmain|:
%
\begin{center}
\begin{tabular}{l}
|\input{childdoc.def}|\\
|\childdocmain{|\textit{main}|}|\\
\end{tabular}
\end{center}
%
If |\jobname| does not match the argument \textit{main} of |\childdocmain|,
it is assumed that |\jobname| points to the child file to be compiled.
When using |\childdocmain| with the main file specified as argument,
it suffices to start a child file
with just |\input{|\textit{main}|}|
without loading of the package and using |\childdocof|.
If instead all processing is done
with the appropriate \textsf{childdoc} directives,
the argument of \textit{main} of |\childdocmain| can be empty.

An alternative version of the command line processing described
in \secref{sec:commandline} using the detection mechanism reads:
%
\begin{center}
|... -jobname "|\textit{target}|" "|[\textit{flags}]%
[|\def\jobname{|\textit{dest}|}|]|\input{|\textit{main}|}"|
\end{center}

%%%%%%%%%%%%%%%%%%%%%%%%%%%%%%%%%%%%%%%%%%%%%%%%%%%%%%%%%%%%%%%%%%%%%%%%%%%%%%%%
\subsection{Manual Code}
\label{sec:manual}

In case one cannot be certain whether the definitions file |childdoc.def|
is installed on the target \TeX{} distribution
and one prefers not to ship it,
it is conceivable to paste a few relevant commands into the sources.

To that end, drop all statements |\input{childdoc.def}|
and perform the replacements as outlined below.
Instead of |\childdocmain{|\textit{main}|}| add the following code
to the top of the main file:
%
\begin{center}
\begin{tabular}{l}
|\||ifdefined\childdocname\endinput\||fi\newif\ifchilddoc|\\
|\edef\childdocname{\scantokens\expandafter{\jobname\noexpand}}|\\
|\def\childdocmain{|\textit{main}|}\||ifx\childdocmain\childdocname\||else|\\
|\childdoctrue\includeonly{\childdocname}\let\jobname\childdocmain\||fi|\\
\end{tabular}
\end{center}
%
Instead of |\childdocof{|\textit{main}|}| just include the main file
at the top of each child file:
%
\begin{center}
|\input{|\textit{main}|}|
\end{center}
%
A simple redirection |\childdocforward{|\textit{dest}|}| is achieved by:
%
\begin{center}
|\def\jobname{|\textit{dest}|}\input{\jobname}|
\end{center}
%
The redirection with prefix
|\childdocforwardprefix[|\textit{prefix}|]{|\textit{dest}|}|
is accomplished by:
%
\begin{center}
\begin{tabular}{l}
|{\edef\jobname{\scantokens\expandafter{\jobname\noexpand}}|\\
|\def\redirectjob |\textit{prefix}|#1~~~{\gdef\jobname{|\textit{dest}|#1}}|\\
|\expandafter\redirectjob\jobname~~~}\input{\jobname}|
\end{tabular}
\end{center}

In an alternative approach,
child documents can be compiled by a specific command line
without additional code or specific definitions:
%
\begin{center}
|... -jobname "|\textit{target}|" "|[\textit{flags}]%
|\includeonly{|\textit{dest}|}\input{|\textit{main}|}"|
\end{center}
%

%%%%%%%%%%%%%%%%%%%%%%%%%%%%%%%%%%%%%%%%%%%%%%%%%%%%%%%%%%%%%%%%%%%%%%%%%%%%%%%%
%%%%%%%%%%%%%%%%%%%%%%%%%%%%%%%%%%%%%%%%%%%%%%%%%%%%%%%%%%%%%%%%%%%%%%%%%%%%%%%%
\section{Information}

%%%%%%%%%%%%%%%%%%%%%%%%%%%%%%%%%%%%%%%%%%%%%%%%%%%%%%%%%%%%%%%%%%%%%%%%%%%%%%%%
\subsection{Copyright}

Copyright \copyright{} 2017--2018 Niklas Beisert

This work may be distributed and/or modified under the
conditions of the \LaTeX{} Project Public License, either version 1.3
of this license or (at your option) any later version.
The latest version of this license is in
  \url{http://www.latex-project.org/lppl.txt}
and version 1.3 or later is part of all distributions of \LaTeX{}
version 2005/12/01 or later.

This work has the LPPL maintenance status `maintained'.

The Current Maintainer of this work is Niklas Beisert.

This work consists of the files |README.txt|, |childdoc.ins| and |childdoc.dtx|
as well as the derived files |childdoc.def|, |cdocsamp.tex|
with |cdocsch1.tex|, |cdocsch2.tex|, |cdocspt3.tex|, |cdocspt4.tex|,
|cdocsdrf.tex|, |cdocsfn1.tex|, |cdocsfn2.tex|
as well as |childdoc.pdf|.

%%%%%%%%%%%%%%%%%%%%%%%%%%%%%%%%%%%%%%%%%%%%%%%%%%%%%%%%%%%%%%%%%%%%%%%%%%%%%%%%
\subsection{Files and Installation}

The package consists of the files:
%
\begin{center}
\begin{tabular}{ll}
    |README.txt|   & readme file \\
    |childdoc.ins| & installation file \\
    |childdoc.dtx| & source file \\
    |childdoc.def| & definition file \\
    |cdocsamp.tex| & sample main file \\
    |cdocsch1.tex| & sample include file \\
    |cdocsch2.tex| & sample include file \\
    |cdocspt3.tex| & sample part file \\
    |cdocspt4.tex| & sample part file \\
    |cdocsdrf.tex| & sample redirection file \\
    |cdocsfn1.tex| & sample redirection file \\
    |cdocsfn2.tex| & sample redirection file \\
    |childdoc.pdf| & manual
\end{tabular}
\end{center}
%
The distribution consists of the files
|README.txt|, |childdoc.ins| and |childdoc.dtx|.
%
\begin{itemize}
\item
Run (pdf)\LaTeX{} on |childdoc.dtx|
to compile the manual |childdoc.pdf| (this file).
\item
Run \LaTeX{} on |childdoc.ins| to create the definitions file |childdoc.def|
and the sample |cdocsamp.tex| with include files
|cdocsch1.tex|, |cdocsch2.tex|, |cdocspt3.tex|, |cdocspt4.tex|,
|cdocsdrf.tex|, |cdocsfn1.tex|, |cdocsfn2.tex|.
Then copy the file |childdoc.def| to an appropriate directory of your \LaTeX{}
distribution, e.g.\ \textit{texmf-root}|/tex/latex/childdoc|.
\end{itemize}

%%%%%%%%%%%%%%%%%%%%%%%%%%%%%%%%%%%%%%%%%%%%%%%%%%%%%%%%%%%%%%%%%%%%%%%%%%%%%%%%
\subsection{Related CTAN Packages}

There are several other packages which offer a similar functionality:
%
\begin{itemize}
\item
The packages
\href{http://ctan.org/pkg/docmute}{\textsf{docmute}},
\href{http://ctan.org/pkg/includex}{\textsf{includex}} and
\href{http://ctan.org/pkg/standalone}{\textsf{standalone}}
provide commands to include only the document body of
a child file thus allowing both files to be compiled individually.
\item
The packages \href{http://ctan.org/pkg/subdocs}{\textsf{subdocs}}
and \href{http://ctan.org/pkg/subfiles}{\textsf{subfiles}}
provide structures in which the main and child documents can be
encapsulated and allowing them to be compiled individually.
The inclusion mechanism is different from the conventional |\include|.
\item
The package \href{http://ctan.org/pkg/combine}{\textsf{combine}}
is an elaborate solution to combine several documents into one.
\end{itemize}
%
See also the CTAN topic \href{http://ctan.org/topic/subdocs}{\textsf{subdocs}}
for further related packages.
The present package differs from the above solutions in that
a document structure constructed with the conventional |\include| mechanism
just needs two extra commands at the top of every file
such that all constituent files can be compiled individually.

%%%%%%%%%%%%%%%%%%%%%%%%%%%%%%%%%%%%%%%%%%%%%%%%%%%%%%%%%%%%%%%%%%%%%%%%%%%%%%%%
%\subsection{Feature Suggestions}
%
%The following is a list of features which may be useful for future
%versions of this package:
%%
%\begin{itemize}
%\item
%\ldots
%\end{itemize}

%%%%%%%%%%%%%%%%%%%%%%%%%%%%%%%%%%%%%%%%%%%%%%%%%%%%%%%%%%%%%%%%%%%%%%%%%%%%%%%%
\subsection{Revision History}

%%%%%%%%%%%%%%%%%%%%%%%%%%%%%%%%%%%%%%%%
\paragraph{v2.0:} 2018/12/30

\begin{itemize}
\item
immediate forward processing
\item
added |\childdocby| mechanism
\item
manual restructured
\end{itemize}

%%%%%%%%%%%%%%%%%%%%%%%%%%%%%%%%%%%%%%%%
\paragraph{v1.6:} 2018/01/17

\begin{itemize}
\item
application for development of include files
\item
corrections to manual
\end{itemize}

%%%%%%%%%%%%%%%%%%%%%%%%%%%%%%%%%%%%%%%%
\paragraph{v1.5:} 2017/05/21

\begin{itemize}
\item
more complete structuring introduced
\item
|\childdocof| introduced
\item
|\childdoc| renamed to |\childdocmain|
\item
|\childredirect| renamed to |\childdocforward| and |\childdocforwardprefix|
and functionality expanded
\end{itemize}

%%%%%%%%%%%%%%%%%%%%%%%%%%%%%%%%%%%%%%%%
\paragraph{v1.0:} 2017/04/27

\begin{itemize}
\item
manual and install package
\item
first version published on CTAN
\end{itemize}

%%%%%%%%%%%%%%%%%%%%%%%%%%%%%%%%%%%%%%%%
\paragraph{v0.6:} 2017/04/26

\begin{itemize}
\item
redirection mechanism added
\end{itemize}

%%%%%%%%%%%%%%%%%%%%%%%%%%%%%%%%%%%%%%%%
\paragraph{v0.5:} 2017/04/26

\begin{itemize}
\item
functionality in definition file
\end{itemize}


%%%%%%%%%%%%%%%%%%%%%%%%%%%%%%%%%%%%%%%%%%%%%%%%%%%%%%%%%%%%%%%%%%%%%%%%%%%%%%%%
%%%%%%%%%%%%%%%%%%%%%%%%%%%%%%%%%%%%%%%%%%%%%%%%%%%%%%%%%%%%%%%%%%%%%%%%%%%%%%%%
%%%%%%%%%%%%%%%%%%%%%%%%%%%%%%%%%%%%%%%%%%%%%%%%%%%%%%%%%%%%%%%%%%%%%%%%%%%%%%%%
\appendix

\settowidth\MacroIndent{\rmfamily\scriptsize 000\ }

 \DocInput{childdoc.dtx}

\end{document}
%</driver>
% \fi
%
% %%%%%%%%%%%%%%%%%%%%%%%%%%%%%%%%%%%%%%%%%%%%%%%%%%%%%%%%%%%%%%%%%%%%%%%%%%%%%%
% %%%%%%%%%%%%%%%%%%%%%%%%%%%%%%%%%%%%%%%%%%%%%%%%%%%%%%%%%%%%%%%%%%%%%%%%%%%%%%
% \section{Sample}
%\iffalse
%<*samplemain>
%\fi
%
% The following presents a sample document
% with two chapters, two parts, a title page,
% a compile flag as well as three forwarding files to set the flag.
% It consists of eight |.tex| files:
% \begin{center}
% \begin{tabular}{ll}
% |cdocsamp.tex|&main file\\
% |cdocsch1.tex|&include file for chapter 1\\
% |cdocsch2.tex|&include file for chapter 2\\
% |cdocspt3.tex|&include file for part 3\\
% |cdocspt4.tex|&include file for part 4\\
% |cdocsdrf.tex|&forwarding file for main file in draft mode\\
% |cdocsfi1.tex|&forwarding file for final version of chapter 1\\
% |cdocsfi2.tex|&forwarding file for final version of chapter 2\\
% \end{tabular}
% \end{center}
% Each of the eight files can be compiled directly by the \LaTeX{} compiler.
%
% %%%%%%%%%%%%%%%%%%%%%%%%%%%%%%%%%%%%%%
% \paragraph{Main File.}
%
% The main file is called |cdocsamp.tex|.
%
% Load the \textsf{childdoc} definitions and
% declare the filename for the main document:
%    \begin{macrocode}
\input{childdoc.def}
\childdocmain{}
%    \end{macrocode}

% Optional override for |\version| flag:
%    \begin{macrocode}
%%\ifchilddoc\else\providecommand{\version}{draft}\fi
%    \end{macrocode}

% Define the default values for the |\version| flag
% (|final| for the main file and |draft| for childs):
%    \begin{macrocode}
\ifchilddoc
\providecommand{\version}{draft}
\else
\providecommand{\version}{final}
\fi
%    \end{macrocode}

% Load the standard document class:
%    \begin{macrocode}
\documentclass[12pt]{article}
%    \end{macrocode}

% Start the document body:
%    \begin{macrocode}
\begin{document}
%    \end{macrocode}

% Declare a title page.
% Print title, part of document being processed and version flag:
%    \begin{macrocode}
\addtocounter{page}{-1}
\begin{center}
{\LARGE\bfseries{}childdoc example\par}
\vspace{1cm}
\ifchilddoc
\ifchilddocmanual part\else chapter\fi:
`\childdocname' of `\childdocjob'\par
\else
main document: `\childdocjob'\par
\fi
version: \version\par
\end{center}
\newpage
%    \end{macrocode}

% Manually include selected file,
% otherwise process as usual:
%    \begin{macrocode}
\ifchilddocmanual
\section*{part `\childdocname'}
\input{\childdocname}
\else
%    \end{macrocode}

% Include the two chapters:
%    \begin{macrocode}
\include{cdocsch1}
\include{cdocsch2}
%    \end{macrocode}

% Include the two parts unless only chapters should be displayed:
%    \begin{macrocode}
\ifchilddoc\else
\section{part three}
\input{cdocspt3}
\section{part four}
\input{cdocspt4}
\fi
%    \end{macrocode}

% Process as usual until here:
%    \begin{macrocode}
\fi
%    \end{macrocode}

% End of document body:
%    \begin{macrocode}
\end{document}
%    \end{macrocode}
%\iffalse
%</samplemain>
%\fi
%
% %%%%%%%%%%%%%%%%%%%%%%%%%%%%%%%%%%%%%%
% \paragraph{Chapter Include Files.}
%
% The include files are called |cdocsch1.tex| and |cdocsch2.tex|.
%
%\iffalse
%<*samplechap1|samplechap2>
%\fi

% Optional override for |\version| flag:
%    \begin{macrocode}
%%\providecommand{\version}{final}
%    \end{macrocode}

% Include the main document:
%    \begin{macrocode}
\input{childdoc.def}
\childdocof{cdocsamp}
%    \end{macrocode}

%\iffalse
%</samplechap1|samplechap2>
%\fi
%
%\iffalse
%<*samplechap1>
%\fi
% Some text for chapter 1:
%    \begin{macrocode}
\section{one}
some text in chapter one
%    \end{macrocode}

%\iffalse
%</samplechap1>
%\fi
% Some text for chapter 2:
%\iffalse
%<*samplechap2>
%\fi
%    \begin{macrocode}
\section{two}
more text in chapter two
%    \end{macrocode}

%\iffalse
%</samplechap2>
%\fi
%
% %%%%%%%%%%%%%%%%%%%%%%%%%%%%%%%%%%%%%%
% \paragraph{Part Include Files.}
%
% The include files are called |cdocspt3.tex| and |cdocspt4.tex|.
%
%\iffalse
%<*samplepart3|samplepart4>
%\fi

% Optional override for |\version| flag:
%    \begin{macrocode}
%%\providecommand{\version}{final}
%    \end{macrocode}

% Include the main document:
%    \begin{macrocode}
\input{childdoc.def}
\childdocby{cdocsamp}
%    \end{macrocode}

%\iffalse
%</samplepart3|samplepart4>
%\fi
%
%\iffalse
%<*samplepart3>
%\fi
% Some text for part 3:
%    \begin{macrocode}
some text in part three
%    \end{macrocode}

%\iffalse
%</samplepart3>
%\fi
% Some text for part 4:
%\iffalse
%<*samplepart4>
%\fi
%    \begin{macrocode}
more text in part four
%    \end{macrocode}

%\iffalse
%</samplepart4>
%\fi
%
% %%%%%%%%%%%%%%%%%%%%%%%%%%%%%%%%%%%%%%
% \paragraph{Forwarding for a Complete Draft.}
%
% The following forwarding file |cdocsdrf.tex|
% compiles the main document in draft mode:
%\iffalse
%<*sampledraft>
%\fi
%    \begin{macrocode}
\def\version{draft}
\input{childdoc.def}
\childdocforward{cdocsamp}
%    \end{macrocode}

%\iffalse
%</sampledraft>
%\fi
%
% %%%%%%%%%%%%%%%%%%%%%%%%%%%%%%%%%%%%%%
% \paragraph{Forwarding for Final Version of the Chapters.}
%
% The following forwarding files |cdocsfn1.tex| and |cdocsfn2.tex|
% (with identical content)
% compile the final versions of the child documents
% |cdocsch1.tex| and |cdocsch2.tex|, respectively:
%\iffalse
%<*samplefinal>
%\fi
%    \begin{macrocode}
\def\version{final}
\input{childdoc.def}
\childdocforwardprefix[cdocsamp]{cdocsfn}{cdocsch}
%    \end{macrocode}

%\iffalse
%</samplefinal>
%\fi
%
% %%%%%%%%%%%%%%%%%%%%%%%%%%%%%%%%%%%%%%
% \paragraph{Command Line Processing.}
%
% The following three command lines generate the output files
% |cdocscld|, |cdocscl1| and |cdocscl2|
% which should be identical to
% |cdocsdrf|, |cdocsch1| and |cdocsfn2|, respectively:
% \begin{center}
% \begin{tabular}{l}
% |latex -jobname cdocscld \|\\
% |  "\def\version{draft}\input{childdoc.def}\childdocforward{cdocsamp}"|\\
% |latex -jobname cdocscl1 \|\\
% |  "\input{childdoc.def}\childdocforward[cdocsamp]{cdocsch1}"|\\
% |latex -jobname cdocscl2 \|\\
% |  "\def\version{final}\input{childdoc.def}\childdocforward{cdocsch2}"|
% \end{tabular}
% \end{center}
% Note that the trailing backslash on each first line
% merely continues the input to the second line
% (for convenient cut ant paste).
% Furthermore, the command |latex| can be replaced by any
% of its alternative versions such as |pdflatex|.
%
% %%%%%%%%%%%%%%%%%%%%%%%%%%%%%%%%%%%%%%%%%%%%%%%%%%%%%%%%%%%%%%%%%%%%%%%%%%%%%%
% %%%%%%%%%%%%%%%%%%%%%%%%%%%%%%%%%%%%%%%%%%%%%%%%%%%%%%%%%%%%%%%%%%%%%%%%%%%%%%
% \section{Implementation}
%\iffalse
%<*package>
%\fi
%
% This section describes the definitions file |childdoc.def|.

% The definitions cannot be loaded using |\usepackage| or |\RequirePackage|
% which has a mechanism to prevent loading a style file more than once.
% When loading the definitions by means of |\input|
% multiple instances have to be prevented manually:
%\iffalse
%This code needs to be before the `\ProvidesFile' directive
%which is defined at the beginning of this file.
%Therefore it is also placed there and commented out here.
%</package>
%<*discard>
%\fi
%    \begin{macrocode}
\ifdefined\childdocmain\endinput\fi
%    \end{macrocode}
%\iffalse
%</discard>
%<*package>
%\fi
%
% \macro{\ifchilddoc}
% \macro{\ifchilddocmanual}
% The conditional |\ifchilddoc| tells whether a
% child (true) or main (false) document is being compiled.
% The conditional |\ifchilddocmanual| tells whether
% the |\includeonly| mechanism is used (false) or
% the selection of child files must be performed manually (true).
% The definitions initialise to false:
%    \begin{macrocode}
\newif\ifchilddoc
\newif\ifchilddocmanual
%    \end{macrocode}

% \macro{\childdocname}
% \macro{\childdocjob}
% The macro |\childdocname| stores the name of the main document
% to be compiled. The macro |\childdocjob| stores the name of
% the document on which the \LaTeX{} compiler was originally invoked.
% The content of |\jobname| cannot be compared
% to filenames specified in the source due to different catcodes.
% The following code rescans |\jobname|, stores the result
% in |\childdocname| and saves a copy in |\childdocjob|:
%    \begin{macrocode}
\edef\childdocname{\scantokens\expandafter{\jobname\noexpand}}
\let\childdocjob\childdocname
%    \end{macrocode}

% \macro{\childdocdisable}
% The macro |\childdocdisable| prevents the main file
% from being processed more than once.
% At this stage, the main document command |\childdocmain|
% is assumed to be called once again where it should do nothing.
% Any subsequent call to it should prevent
% a secondary processing of the main document
% It overwrites the forwarding commands
% |\childdocof| and |\childdocforward|
% with empty macros to prevent further inclusions of the main document:
%    \begin{macrocode}
\newcommand{\childdocdisable}
{
  \renewcommand{\childdocmain}[1]{\renewcommand{\childdocmain}[1]{\endinput}}
  \renewcommand{\childdocof}[1]{}
  \renewcommand{\childdocby}[2][]{}
  \renewcommand{\childdocforward}[2][]{}
  \renewcommand{\childdocdisable}{}
}
%    \end{macrocode}

% \macro{\childdocmain}
% The macro |\childdocmain| is to be called at the top of the main file
% with nothing or the main filename (without extension) as argument.
% First, it breaks loops.
% If the argument is not empty and does not match |\childdocname|
% (which is set by the first inclusion of |childdoc.def|),
% |\ifchilddoc| is set to true, |\includeonly| is applied to the child file
% and |\jobname| is set to the main file
% (for proper handling of |.aux| files):
%    \begin{macrocode}
\newcommand{\childdocmain}[1]
{
  \childdocdisable\childdocmain{}
  \if?#1?\else
    \begingroup
      \def\childdoctmp{#1}
      \ifx\childdoctmp\childdocname
        \def\childdoctmp{}
      \else
        \def\childdoctmp
        {
          \childdoctrue
          \includeonly{\childdocname}
          \def\childdocjob{#1}
          \def\jobname{#1}
        }
      \fi
      \expandafter
    \endgroup
    \childdoctmp
  \fi
}
%    \end{macrocode}

% \macro{\childdocof}
% The command |\childdocof| redirects
% compilation to the main file |#1|.
%    \begin{macrocode}
\newcommand{\childdocof}[1]
{
  \childdocdisable
  \childdoctrue
  \includeonly{\childdocname}
  \def\jobname{#1}
  \def\childdocjob{#1}
  \input{#1}
}
%    \end{macrocode}

% \macro{\childdocby}
% The command |\childdocby| ....
%    \begin{macrocode}
\newcommand{\childdocby}[2][]
{
  \childdocdisable
  \childdoctrue
  \childdocmanualtrue
  \if?#1?\else
    \def\jobname{#2}
  \fi
  \def\childdocjob{#2}
  \input{#2}
  \endinput
}
%    \end{macrocode}

% \macro{\childdocforward}
% The command |\childdocforward| redirects
% compilation to the main file or
% (if the optional argument is given) a child file.
% Parameters are set as if the main file
% or a child file starting with |\childdocof| was compiled.
% Then compilation is handed over to the main file:
%    \begin{macrocode}
\newcommand{\childdocforward}[2][]
{
  \begingroup
    \if?#1?
      \def\childdoctmp
      {
        \def\childdocname{#2}
        \def\childdocjob{#2}
        \def\jobname{#2}
        \input{#2}
        \endinput
      }
    \else
      \def\childdoctmp
      {
        \childdocdisable
        \def\childdocname{#2}
        \childdoctrue
        \includeonly{#2}
        \def\childdocjob{#1}
        \def\jobname{#1}
        \input{#1}
        \endinput
      }
    \fi
    \expandafter
  \endgroup
  \childdoctmp
}
%    \end{macrocode}

% \macro{\childdocforwardprefix}
% The command |\childdocforwardprefix| redirects
% compilation to the main or a child file by means of a pattern.
% The prefix |#1| in the current filename is replaced by |#2|
% and the suffix of the current filename is kept
% (it is assumed that the filename does not contain the substring `|~~~|'
% which is used as a delimiter).
% Compilation is handed over to the new file by |\childdocforward|:
%    \begin{macrocode}
\newcommand{\childdocforwardprefix}[3][]
{
  \begingroup
    \def\childdocextract #2##1~~~{\def\childdoctmp{\childdocforward[#1]{#3##1}}}
    \expandafter\childdocextract\childdocname~~~
    \expandafter
  \endgroup
  \childdoctmp
}
%    \end{macrocode}

% \macro{\childdoc}
% The deprecated macro |\childdoc| is a legacy version of |\childdocmain|:
%    \begin{macrocode}
\newcommand{\childdoc}{\childdocmain}
%    \end{macrocode}

% \macro{\childdocredirect}
% The deprecated macro |\childdocredirect| is a legacy version
% of |\childdocforward| and |\childdocforwardprefix|:
%    \begin{macrocode}
\newcommand{\childdocredirect}[2][]
{
  \begingroup
    \if?#1?
      \def\childdoctmp{\childdocforward{#2}}
    \else
      \def\childdoctmp{\childdocforwardprefix{#1}{#2}}
    \fi
    \expandafter
  \endgroup
  \childdoctmp
}
%    \end{macrocode}

%\iffalse
%</package>
%\fi
%
\endinput

\childdocby{cdocsamp}
%    \end{macrocode}

%\iffalse
%</samplepart3|samplepart4>
%\fi
%
%\iffalse
%<*samplepart3>
%\fi
% Some text for part 3:
%    \begin{macrocode}
some text in part three
%    \end{macrocode}

%\iffalse
%</samplepart3>
%\fi
% Some text for part 4:
%\iffalse
%<*samplepart4>
%\fi
%    \begin{macrocode}
more text in part four
%    \end{macrocode}

%\iffalse
%</samplepart4>
%\fi
%
% %%%%%%%%%%%%%%%%%%%%%%%%%%%%%%%%%%%%%%
% \paragraph{Forwarding for a Complete Draft.}
%
% The following forwarding file |cdocsdrf.tex|
% compiles the main document in draft mode:
%\iffalse
%<*sampledraft>
%\fi
%    \begin{macrocode}
\def\version{draft}
% \iffalse
%
% childdoc.dtx Copyright (C) 2017-2018 Niklas Beisert
%
% This work may be distributed and/or modified under the
% conditions of the LaTeX Project Public License, either version 1.3
% of this license or (at your option) any later version.
% The latest version of this license is in
%   http://www.latex-project.org/lppl.txt
% and version 1.3 or later is part of all distributions of LaTeX
% version 2005/12/01 or later.
%
% This work has the LPPL maintenance status `maintained'.
%
% The Current Maintainer of this work is Niklas Beisert.
%
% This work consists of the files childdoc.dtx and childdoc.ins
% and the derived files childdoc.def and cdocsamp.tex with
% cdocsch1.tex, cdocsch2.tex, cdocsdrf.tex, cdocsfn1.tex, cdocsfn2.tex.
%
%<package>\ifdefined\childdocmain\endinput\fi
%<package>\ProvidesFile{childdoc.def}[2018/12/30 v2.0 child document driver]
%<samplemain>\ProvidesFile{cdocsamp.tex}[2018/12/30 v2.0 sample for childdoc]
%<*driver>
%\ProvidesFile{childdoc.drv}[2018/12/30 v2.0 childdoc reference manual file]
\PassOptionsToClass{10pt,a4paper}{article}
\documentclass{ltxdoc}

\usepackage[margin=35mm]{geometry}
\usepackage{hyperref}
\usepackage{hyperxmp}
\usepackage[usenames]{color}

\hypersetup{colorlinks=true}
\hypersetup{pdfstartview=FitH}
\hypersetup{pdfpagemode=UseNone}
\hypersetup{pdfsource={}}
\hypersetup{pdflang={en-UK}}
\hypersetup{pdfcopyright={Copyright 2017-2018 Niklas Beisert.
  This work may be distributed and/or modified under the
  conditions of the LaTeX Project Public License, either version 1.3
  of this license or (at your option) any later version.}}
\hypersetup{pdflicenseurl={http://www.latex-project.org/lppl.txt}}
\hypersetup{pdfcontactaddress={ETH Zurich, ITP, HIT K,
  Wolfgang-Pauli-Strasse 27}}
\hypersetup{pdfcontactpostcode={8093}}
\hypersetup{pdfcontactcity={Zurich}}
\hypersetup{pdfcontactcountry={Switzerland}}
\hypersetup{pdfcontactemail={nbeisert@itp.phys.ethz.ch}}
\hypersetup{pdfcontacturl={http://people.phys.ethz.ch/\xmptilde nbeisert/}}

\newcommand{\secref}[1]{\hyperref[#1]{section \ref*{#1}}}

\parskip1ex
\parindent0pt
\let\olditemize\itemize
\def\itemize{\olditemize\parskip0pt}

\begin{document}

\title{The \textsf{childdoc} Package}
\hypersetup{pdftitle={The childdoc Package}}
\author{Niklas Beisert\\[2ex]
  Institut f\"ur Theoretische Physik\\
  Eidgen\"ossische Technische Hochschule Z\"urich\\
  Wolfgang-Pauli-Strasse 27, 8093 Z\"urich, Switzerland\\[1ex]
  \href{mailto:nbeisert@itp.phys.ethz.ch}
  {\texttt{nbeisert@itp.phys.ethz.ch}}}
\hypersetup{pdfauthor={Niklas Beisert}}
\hypersetup{pdfsubject={Manual for the LaTeX2e Package childdoc}}
\date{30 December 2018, \textsf{v2.0}}
\maketitle

\begin{abstract}\noindent
\textsf{childdoc} is a \LaTeXe{} package
that enables the direct compilation
of document sections included by |\include|
to individual files.
\end{abstract}

\begingroup
\parskip0ex
\tableofcontents
\endgroup

%%%%%%%%%%%%%%%%%%%%%%%%%%%%%%%%%%%%%%%%%%%%%%%%%%%%%%%%%%%%%%%%%%%%%%%%%%%%%%%%
%%%%%%%%%%%%%%%%%%%%%%%%%%%%%%%%%%%%%%%%%%%%%%%%%%%%%%%%%%%%%%%%%%%%%%%%%%%%%%%%
\section{Introduction}

\LaTeX{} provides a mechanism to structure a large document (such as a book)
into a main file and several child files (containing the chapters)
using the |\include| command.
This mechanism is beneficial for documents
which span hundreds of pages in order to
make the source file(s) more manageable.
Moreover, compilation can be restricted to
selected child files by means of the |\includeonly| command.
The latter feature can be used to reduce the compilation time while editing
(this was significantly more useful in the earlier days of \LaTeX{})
or to generate a smaller document which is easier to navigate.
Another application of |\includeonly| is to generate
documents consisting of selected parts of the complete document.

However, there are a few drawbacks of the plain |\include| mechanism:
\begin{itemize}
\item
The child files cannot be compiled on their own,
they can only be compiled via the main file.
A naive editing environment
(such as a text editor with an option
to have the current file processed by \LaTeX)
may require one to switch to the main file before compiling;
attempting to compile the child file produces errors.
\item
The main file must be modified (each time)
to adjust the |\includeonly| command
to the present needs. This easily leaves the main file in a messy state.
\item
The generated document will always carry the filename
of the main document. This is inconvenient if
several child files are to be compiled and
to be kept for distribution.
\end{itemize}

The present package provides a simple interface
to make child files individually compilable by \LaTeX{}.
Compiling a child file then has the same effect as compiling
the main file with an |\includeonly| command
to select the appropriate child.
Moreover the generated document will carry the name of the child
rather than the main file.
This resolves all three above issues.

This feature is meant to make the editing of books,
thesis documents and lecture notes somewhat more convenient.
However, the package can also be used efficiently for
composing a series of documents (such as exercise sheets)
which are typically distributed individually.
It then assists the author in generating the individual documents
(potentially in different versions)
as well as a document containing the collected series.
Another application is in developing style files
or other kinds of included material
where compilation of the style file could redirect
to a sample or test file.

%%%%%%%%%%%%%%%%%%%%%%%%%%%%%%%%%%%%%%%%%%%%%%%%%%%%%%%%%%%%%%%%%%%%%%%%%%%%%%%%
%%%%%%%%%%%%%%%%%%%%%%%%%%%%%%%%%%%%%%%%%%%%%%%%%%%%%%%%%%%%%%%%%%%%%%%%%%%%%%%%
\section{Usage}

First of all, the package \textsf{childdoc} is \emph{not} a standard
\LaTeXe{} |.sty| style file! Therefore it needs to be invoked in
a non-standard way.

%%%%%%%%%%%%%%%%%%%%%%%%%%%%%%%%%%%%%%%%%%%%%%%%%%%%%%%%%%%%%%%%%%%%%%%%%%%%%%%%
\subsection{Included Files}
\label{sec:include}

%%%%%%%%%%%%%%%%%%%%%%%%%%%%%%%%%%%%%%%%
\DescribeMacro{\childdocmain}
To use the package, add the commands
\begin{center}
\begin{tabular}{l}
|\input{childdoc.def}|\\
|\childdocmain{}|\\
\end{tabular}
\end{center}
at the very top of the main \LaTeX{} file,
in particular \emph{before} the |\documentclass| statement!
The argument of |\childdocmain| should be left empty
(but it must be present).

%%%%%%%%%%%%%%%%%%%%%%%%%%%%%%%%%%%%%%%%
\DescribeMacro{\childdocof}
Furthermore, add the commands
\begin{center}
\begin{tabular}{l}
|\input{childdoc.def}|\\
|\childdocof{|\textit{main}|}|\\
\end{tabular}
\end{center}
at the top of every child file \textit{child}
which is included by |\include{|\textit{child}|}|
from within the main file
(or at least for those files to be compiled individually).
The argument \textit{main} must be the filename of the main file.

There are a couple of
considerations in setting up the main and child documents:

%%%%%%%%%%%%%%%%%%%%%%%%%%%%%%%%%%%%%%%%
\paragraph{Restrictions.}

Please note the following restrictions:
\begin{itemize}
\item
|\childdocmain| must be called with one argument \textit{main}
to ensure compatibility with earlier version of the package.
It must either be empty (|\childdocmain{}|)
or precisely match the filename of the main file in which it is specified.
See \secref{sec:detection} for further information.
\item
The filename \textit{main} must be specified without the |.tex| extension.
\item
The filename \textit{main} is case sensitive
(even in case-insensitive file systems)
due to internal string comparison.
\item
The argument \textit{main} should be fully expanded, it cannot be a macro.
\item
Subdirectories and special characters should be avoided in filenames.
\item
The command |\childdocmain{|\textit{main}|}| must be followed by a whitespace.
It should not be followed immediately by another command
or by a comment mark `|%|'.
This is because the \TeX{} parser reads the token immediately following
the argument of |\childdocmain| and puts it
at the beginning of every child section;
however, a white\-space is ignored.
\end{itemize}

%%%%%%%%%%%%%%%%%%%%%%%%%%%%%%%%%%%%%%%%
\paragraph{Content of Main File.}

It is advisable to place all content in the child files included by |\include|.
Any output contained in the main file will appear in all child documents
unless suppressed manually;
it cannot be suppressed automatically by the |\includeonly| directive
and thus should normally be avoided.
A method to include some content in the main file
by means of conditional processing is described in \secref{sec:conditional}.

%%%%%%%%%%%%%%%%%%%%%%%%%%%%%%%%%%%%%%%%
\paragraph{Page Numbering.}

When only a part of the document is compiled,
the appropriate numbering of pages
(as well as other status parameters)
is determined from the |.aux| files.
The latter contain information from previous passes.
However this information needs to propagate through
all intermediate child documents.
Therefore the page numbering in child documents may well
be inconsistent until the complete document is compiled at least once.

A useful (if unconventional) way to always ensure a consistent
page numbering is to restart the numbering in each child document
and denote the pages by `\textit{child}|.|\textit{page}'
where \textit{child} represents the chapter/section number of the child file.
This can be achieved by the command
|\numberwithin{page}{|\textit{child}|}|
of the \textsf{amsmath} package
where \textit{child} can be |chapter| or |section|
depending on the chosen structuring.
Alternatively, one can modify the macro |\thepage| appropriately
and reset the counter |page| at the start of each child file.

%%%%%%%%%%%%%%%%%%%%%%%%%%%%%%%%%%%%%%%%%%%%%%%%%%%%%%%%%%%%%%%%%%%%%%%%%%%%%%%%
\subsection{Conditional Processing}
\label{sec:conditional}

The package provides a mechanism to compile different versions
of a document. To customise the versions further some conditional processing
can come in handy to distinguish which version is being compiled.
The package provides two macros to describe the compilation context:

%%%%%%%%%%%%%%%%%%%%%%%%%%%%%%%%%%%%%%%%
\DescribeMacro{\ifchilddoc}
The conditional |\ifchilddoc| distinguishes between the compilation of
child documents and the main document:
%
\begin{center}
|\ifchilddoc |\textit{child-code}| |[|\||else |\textit{main-code}]| \||fi|
\end{center}

%%%%%%%%%%%%%%%%%%%%%%%%%%%%%%%%%%%%%%%%
\DescribeMacro{\childdocname}
\DescribeMacro{\childdocjob}
The macro |\childdocname| contains the filename (without extension)
of the main or child file being processed.
Note that |\childdocjob| will always contain the name of the main file.

%%%%%%%%%%%%%%%%%%%%%%%%%%%%%%%%%%%%%%%%
\paragraph{Title Page.}

Conditional processing can be used to include a title or banner page
in the main document when proper precautions are taken.
Importantly, the code in the main file should ensure that the page counter
(as well as other status parameters which are stored in the |.aux| files)
takes the same value after the conditional processing.
Otherwise the page numbers may take divergent values
depending on which part is compiled.

For example, a title page could be declared by:
%
\begin{center}
\begin{tabular}{l}
|\ifchilddoc\||else|\\
|\addtocounter{page}{-1}|\\
\textit{code for title page}\\
|\newpage|\\
|\||fi|
\end{tabular}
\end{center}
%
A banner page for the child documents can be generated by:
%
\begin{center}
\begin{tabular}{l}
|\ifchilddoc|\\
|\addtocounter{page}{-1}|\\
\textit{code for banner page}\\
|\newpage|\\
|\||fi|
\end{tabular}
\end{center}
%
Here one could write a message such as:
\begin{center}
|This is the part \childdocname{} of \childdocjob{}.|
\end{center}

%%%%%%%%%%%%%%%%%%%%%%%%%%%%%%%%%%%%%%%%%%%%%%%%%%%%%%%%%%%%%%%%%%%%%%%%%%%%%%%%
\subsection{Flags}
\label{sec:flags}

The package makes it easy to generate different versions
of the main or child documents.
To this end compilation flags can be defined
and assigned different default values.
They will be particularly useful in conjunction
with the forwarding mechanism described in \secref{sec:forward}.

For example, it may be useful to have a flag |\version|
which can be set to |draft| or |final|.
The document source will contain some conditional code
depending on the value of |\version|.
Suppose further, the flag should default to |final| for the main file
and to |draft| for child files
which is a natural assignment for editing the document.
This is achieved by placing the following code
in the preamble of the main document
(below the |\childdocmain| directive):
%
\begin{center}
\begin{tabular}{l}
|\ifchilddoc|\\
|\providecommand{\version}{draft}|\\
|\||else|\\
|\providecommand{\version}{final}|\\
|\||fi|
\end{tabular}
\end{center}
%
The definition by |\providecommand| makes sure
that previous definitions are not overwritten.
Further statements |\providecommand{\version}{...}|
can thus be added before the above code to override it.

For the main file, one might add a line
(between |\childdocmain| and the above block)
%
\begin{center}
|%\ifchilddoc\||else\providecommand{\version}{draft}\||fi|
\end{center}
%
which can be uncommented to produce a draft version.
Likewise one can add a line to the very top of a child file
(above the |\childdocof{|\textit{main}|}| directive)
%
\begin{center}
|%\providecommand{\version}{final}|
\end{center}
%
which can be uncommented to produce the final version of this child document.

%%%%%%%%%%%%%%%%%%%%%%%%%%%%%%%%%%%%%%%%%%%%%%%%%%%%%%%%%%%%%%%%%%%%%%%%%%%%%%%%
\subsection{Forwarding}
\label{sec:forward}

Different versions of the main or child documents
using compilation flags as described in \secref{sec:flags}
can be (permanently) stored in different files
for convenient compilation, viewing and distribution.
To this end, the package defines a command
to pass on compilation to a different file:

%%%%%%%%%%%%%%%%%%%%%%%%%%%%%%%%%%%%%%%%
\DescribeMacro{\childdocforward}
The command |\childdocforward| redirects processing to
another source file:
%
\begin{center}
\begin{tabular}{l}
|\input{childdoc.def}|\\
|\childdocforward[|\textit{main}|]{|\textit{dest}|}|\\
\end{tabular}
\end{center}
%
The argument \textit{dest} is the destination file
(without extension).
It should be the main file or one of the child files.
Note that further \textsf{childdoc} directives
such as |\childdocof| and |\childdocforward|
in the indicated file will be processed in this form.
The optional argument \textit{main}
passes on directly to the main file \textit{main}
while pretending to compile the child \textit{dest}.
This form behaves as if \textit{dest}
issues |\childdocof{|\textit{main}|}| right away,
and no further \textsf{childdoc} directives will be processed.

%%%%%%%%%%%%%%%%%%%%%%%%%%%%%%%%%%%%%%%%
\DescribeMacro{\...prefix}
In the alternative form |\childdocforwardprefix|,
%
\begin{center}
\begin{tabular}{l}
|\input{childdoc.def}|\\
|\childdocforwardprefix[|\textit{main}|]{|\textit{prefix}|}{|\textit{dest}|}|
\end{tabular}
\end{center}
%
the destination file is determined by a pattern
depending on the current file:
To make this work, the current file must be called
`{\textit{prefix}\hspace{0.2em}\textit{suffix}}'
with \textit{prefix} matching precisely the argument.
Processing is then passed on to the file
`{\textit{dest}\hspace{0.2em}\textit{suffix}}'.
Surely, the same effect is achieved by
directly specifying the
argument `{\textit{dest}\hspace{0.2em}\textit{suffix}}'
in the first form.
However, that requires to set up a different file
for each child. With the alternative form of the command
all these files can have exactly the same content
which simplifies setting them up and maintaining them.

For example, the following file |draft.tex|
with a compilation flag |\version| as described in \secref{sec:flags}
compiles the main document as a draft:
%
\begin{center}
\begin{tabular}{l}
|\def\version{draft}|\\
|\input{childdoc.def}|\\
|\childdocforward{|\textit{main}|}|
\end{tabular}
\end{center}
%
Likewise, the following files |final|\textit{nn}|.tex|
compile the final version of the child document
|child|\textit{nn}|.tex|:
%
\begin{center}
\begin{tabular}{l}
|\def\version{final}|\\
|\input{childdoc.def}|\\
|\childdocforwardprefix{final}{child}|
\end{tabular}
\end{center}
%

Note that when several versions of a main file and/or of each child file
are to be generated, it may be convenient to set up a |Makefile| or
shell script to automatise the process.

%%%%%%%%%%%%%%%%%%%%%%%%%%%%%%%%%%%%%%%%%%%%%%%%%%%%%%%%%%%%%%%%%%%%%%%%%%%%%%%%
\subsection{Command Line Processing}
\label{sec:commandline}

The effect of redirection files can also be achieved by invoking
the \LaTeX{} compiler with a more elaborate command line.
Most conveniently this should be done as part
of a shell script or a |Makefile|.

When using \textsf{childdoc} in the main file, the following
command lines effectively perform a redirection
(note that depending on the shell being used,
backslashes may have to be doubled: `|\|' $\to$ `|\\|'):
%
\begin{center}
|... -jobname "|\textit{target}|" |\\|"|[\textit{flags}]%
|\input{childdoc.def}\childdocforward[|\textit{main}|]{|\textit{dest}|}"|
\end{center}
%
Here \textit{target} is the name of the output file,
\textit{main} is the name of the main file
and \textit{dest} is the name of the main or child file to be processed
(all filenames without extensions).
The optional argument \textit{main} can be omitted
if \textit{main} matches \textit{dest}.
Optionally, compilation \textit{flags} can be defined via |\def| commands.
This command line makes the \TeX{} engine believe
it is compiling the file \textit{target}
whose content is specified as the latter parameter.
The provided code then forwards the processing to
\textit{main} or \textit{dest} as described in \secref{sec:forward}.

%%%%%%%%%%%%%%%%%%%%%%%%%%%%%%%%%%%%%%%%%%%%%%%%%%%%%%%%%%%%%%%%%%%%%%%%%%%%%%%%
\subsection{Include by Input}
\label{sec:input}

Including child documents by |\include| has some restrictions by design.
Most notably, the content of a child document always occupies
its own set of pages; pages cannot be shared between child documents.
Usually, this behaviour makes perfect sense
because each child document contain an essential part of the document.
However, in some situations it may be desirable to compose
a document from a collection of parts
without having mandatory page breaks between then.
For this case, the package
provides a mechanism to include parts
by |\input| which can also be processed individually.
However, by construction this mechanism
requires manual handling of the content to be output.

%%%%%%%%%%%%%%%%%%%%%%%%%%%%%%%%%%%%%%%%
\DescribeMacro{\ifchilddocmanual}
The main file should be prepared as usual, see \secref{sec:include}.
However, the document body must make a distinction
between processing of an individual part and of the main document, e.g.:
%
\begin{center}
\begin{tabular}{l}
|\ifchilddocmanual|\\
|\input{\childdocname}|\\
|\||else|\\
\textit{document body with }|\input{|\textit{part}|}|\\
|\||fi|
\end{tabular}
\end{center}
%
The conditional |\ifchilddocmanual| is true whenever
a part to be included by |\input| is being compiled,
and the name of the part is stored in |\childdocname|.

%%%%%%%%%%%%%%%%%%%%%%%%%%%%%%%%%%%%%%%%
\DescribeMacro{\childdocby}
Each part to be included by |\input| should start with:
%
\begin{center}
\begin{tabular}{l}
|\input{childdoc.def}|\\
|\childdocby{|\textit{main}|}|\\
\end{tabular}
\end{center}
%
The directive |\childdocby| is similar to |\childdocof|
described in \secref{sec:include},
but the subsequent selection of content must be done manually.
To that end, both |\ifchilddoc| and |\ifchilddocmanual|
will be true upon processing of a part,
and the name of the part is stored in |\childdocname|.
Note that |\jobname| will be set to the filename of the current part
so that each part receives an individual |.aux| file
that does not interfere with the |.aux| file(s) of the main document.
This behaviour can be altered by the alternative form
|\childdocby[*]{|\textit{main}|}| (with a non-empty optional argument)
which uses the |.aux| file of the main document
by setting |\jobname| to \textit{main}.

%%%%%%%%%%%%%%%%%%%%%%%%%%%%%%%%%%%%%%%%%%%%%%%%%%%%%%%%%%%%%%%%%%%%%%%%%%%%%%%%
\subsection{Driver Development}
\label{sec:driver}

The \textsf{childdoc} mechanism can also be use for the development
of definition files such as \LaTeX{} styles or classes.
This case differs from the above setup with multiple parts
included by |\include| in that no |\includeonly| should be invoked.
This can be achieved by starting the include file
(before |\ProvidesPackage|) with:
%
\begin{center}
\begin{tabular}{l}
|\input{childdoc.def}|\\
|\childdocforward{|\textit{main}|}|\\
\end{tabular}
\end{center}
%
or alternatively with:
%
\begin{center}
\begin{tabular}{l}
|\input{childdoc.def}|\\
|\childdocby{|\textit{main}|}|\\
\end{tabular}
\end{center}
%
Both forms have slightly different effects as described above.
The main file is prepared as usual, see \secref{sec:include}.

%%%%%%%%%%%%%%%%%%%%%%%%%%%%%%%%%%%%%%%%%%%%%%%%%%%%%%%%%%%%%%%%%%%%%%%%%%%%%%%%
\subsection{Legacy Detection}
\label{sec:detection}

The directive |\childdocmain| in the main file can detect
whether the complete document or merely a child is to be compiled
even without using the directive |\childdocof|.
This method is deprecated because it is less robust
and there is no compelling reason to use it;
it is merely provided for backward compatibility
and it may be removed in future versions.

If the detection mechanism is to be used,
it is mandatory to correctly specify
the filename of the main file as the argument of |\childdocmain|:
%
\begin{center}
\begin{tabular}{l}
|\input{childdoc.def}|\\
|\childdocmain{|\textit{main}|}|\\
\end{tabular}
\end{center}
%
If |\jobname| does not match the argument \textit{main} of |\childdocmain|,
it is assumed that |\jobname| points to the child file to be compiled.
When using |\childdocmain| with the main file specified as argument,
it suffices to start a child file
with just |\input{|\textit{main}|}|
without loading of the package and using |\childdocof|.
If instead all processing is done
with the appropriate \textsf{childdoc} directives,
the argument of \textit{main} of |\childdocmain| can be empty.

An alternative version of the command line processing described
in \secref{sec:commandline} using the detection mechanism reads:
%
\begin{center}
|... -jobname "|\textit{target}|" "|[\textit{flags}]%
[|\def\jobname{|\textit{dest}|}|]|\input{|\textit{main}|}"|
\end{center}

%%%%%%%%%%%%%%%%%%%%%%%%%%%%%%%%%%%%%%%%%%%%%%%%%%%%%%%%%%%%%%%%%%%%%%%%%%%%%%%%
\subsection{Manual Code}
\label{sec:manual}

In case one cannot be certain whether the definitions file |childdoc.def|
is installed on the target \TeX{} distribution
and one prefers not to ship it,
it is conceivable to paste a few relevant commands into the sources.

To that end, drop all statements |\input{childdoc.def}|
and perform the replacements as outlined below.
Instead of |\childdocmain{|\textit{main}|}| add the following code
to the top of the main file:
%
\begin{center}
\begin{tabular}{l}
|\||ifdefined\childdocname\endinput\||fi\newif\ifchilddoc|\\
|\edef\childdocname{\scantokens\expandafter{\jobname\noexpand}}|\\
|\def\childdocmain{|\textit{main}|}\||ifx\childdocmain\childdocname\||else|\\
|\childdoctrue\includeonly{\childdocname}\let\jobname\childdocmain\||fi|\\
\end{tabular}
\end{center}
%
Instead of |\childdocof{|\textit{main}|}| just include the main file
at the top of each child file:
%
\begin{center}
|\input{|\textit{main}|}|
\end{center}
%
A simple redirection |\childdocforward{|\textit{dest}|}| is achieved by:
%
\begin{center}
|\def\jobname{|\textit{dest}|}\input{\jobname}|
\end{center}
%
The redirection with prefix
|\childdocforwardprefix[|\textit{prefix}|]{|\textit{dest}|}|
is accomplished by:
%
\begin{center}
\begin{tabular}{l}
|{\edef\jobname{\scantokens\expandafter{\jobname\noexpand}}|\\
|\def\redirectjob |\textit{prefix}|#1~~~{\gdef\jobname{|\textit{dest}|#1}}|\\
|\expandafter\redirectjob\jobname~~~}\input{\jobname}|
\end{tabular}
\end{center}

In an alternative approach,
child documents can be compiled by a specific command line
without additional code or specific definitions:
%
\begin{center}
|... -jobname "|\textit{target}|" "|[\textit{flags}]%
|\includeonly{|\textit{dest}|}\input{|\textit{main}|}"|
\end{center}
%

%%%%%%%%%%%%%%%%%%%%%%%%%%%%%%%%%%%%%%%%%%%%%%%%%%%%%%%%%%%%%%%%%%%%%%%%%%%%%%%%
%%%%%%%%%%%%%%%%%%%%%%%%%%%%%%%%%%%%%%%%%%%%%%%%%%%%%%%%%%%%%%%%%%%%%%%%%%%%%%%%
\section{Information}

%%%%%%%%%%%%%%%%%%%%%%%%%%%%%%%%%%%%%%%%%%%%%%%%%%%%%%%%%%%%%%%%%%%%%%%%%%%%%%%%
\subsection{Copyright}

Copyright \copyright{} 2017--2018 Niklas Beisert

This work may be distributed and/or modified under the
conditions of the \LaTeX{} Project Public License, either version 1.3
of this license or (at your option) any later version.
The latest version of this license is in
  \url{http://www.latex-project.org/lppl.txt}
and version 1.3 or later is part of all distributions of \LaTeX{}
version 2005/12/01 or later.

This work has the LPPL maintenance status `maintained'.

The Current Maintainer of this work is Niklas Beisert.

This work consists of the files |README.txt|, |childdoc.ins| and |childdoc.dtx|
as well as the derived files |childdoc.def|, |cdocsamp.tex|
with |cdocsch1.tex|, |cdocsch2.tex|, |cdocspt3.tex|, |cdocspt4.tex|,
|cdocsdrf.tex|, |cdocsfn1.tex|, |cdocsfn2.tex|
as well as |childdoc.pdf|.

%%%%%%%%%%%%%%%%%%%%%%%%%%%%%%%%%%%%%%%%%%%%%%%%%%%%%%%%%%%%%%%%%%%%%%%%%%%%%%%%
\subsection{Files and Installation}

The package consists of the files:
%
\begin{center}
\begin{tabular}{ll}
    |README.txt|   & readme file \\
    |childdoc.ins| & installation file \\
    |childdoc.dtx| & source file \\
    |childdoc.def| & definition file \\
    |cdocsamp.tex| & sample main file \\
    |cdocsch1.tex| & sample include file \\
    |cdocsch2.tex| & sample include file \\
    |cdocspt3.tex| & sample part file \\
    |cdocspt4.tex| & sample part file \\
    |cdocsdrf.tex| & sample redirection file \\
    |cdocsfn1.tex| & sample redirection file \\
    |cdocsfn2.tex| & sample redirection file \\
    |childdoc.pdf| & manual
\end{tabular}
\end{center}
%
The distribution consists of the files
|README.txt|, |childdoc.ins| and |childdoc.dtx|.
%
\begin{itemize}
\item
Run (pdf)\LaTeX{} on |childdoc.dtx|
to compile the manual |childdoc.pdf| (this file).
\item
Run \LaTeX{} on |childdoc.ins| to create the definitions file |childdoc.def|
and the sample |cdocsamp.tex| with include files
|cdocsch1.tex|, |cdocsch2.tex|, |cdocspt3.tex|, |cdocspt4.tex|,
|cdocsdrf.tex|, |cdocsfn1.tex|, |cdocsfn2.tex|.
Then copy the file |childdoc.def| to an appropriate directory of your \LaTeX{}
distribution, e.g.\ \textit{texmf-root}|/tex/latex/childdoc|.
\end{itemize}

%%%%%%%%%%%%%%%%%%%%%%%%%%%%%%%%%%%%%%%%%%%%%%%%%%%%%%%%%%%%%%%%%%%%%%%%%%%%%%%%
\subsection{Related CTAN Packages}

There are several other packages which offer a similar functionality:
%
\begin{itemize}
\item
The packages
\href{http://ctan.org/pkg/docmute}{\textsf{docmute}},
\href{http://ctan.org/pkg/includex}{\textsf{includex}} and
\href{http://ctan.org/pkg/standalone}{\textsf{standalone}}
provide commands to include only the document body of
a child file thus allowing both files to be compiled individually.
\item
The packages \href{http://ctan.org/pkg/subdocs}{\textsf{subdocs}}
and \href{http://ctan.org/pkg/subfiles}{\textsf{subfiles}}
provide structures in which the main and child documents can be
encapsulated and allowing them to be compiled individually.
The inclusion mechanism is different from the conventional |\include|.
\item
The package \href{http://ctan.org/pkg/combine}{\textsf{combine}}
is an elaborate solution to combine several documents into one.
\end{itemize}
%
See also the CTAN topic \href{http://ctan.org/topic/subdocs}{\textsf{subdocs}}
for further related packages.
The present package differs from the above solutions in that
a document structure constructed with the conventional |\include| mechanism
just needs two extra commands at the top of every file
such that all constituent files can be compiled individually.

%%%%%%%%%%%%%%%%%%%%%%%%%%%%%%%%%%%%%%%%%%%%%%%%%%%%%%%%%%%%%%%%%%%%%%%%%%%%%%%%
%\subsection{Feature Suggestions}
%
%The following is a list of features which may be useful for future
%versions of this package:
%%
%\begin{itemize}
%\item
%\ldots
%\end{itemize}

%%%%%%%%%%%%%%%%%%%%%%%%%%%%%%%%%%%%%%%%%%%%%%%%%%%%%%%%%%%%%%%%%%%%%%%%%%%%%%%%
\subsection{Revision History}

%%%%%%%%%%%%%%%%%%%%%%%%%%%%%%%%%%%%%%%%
\paragraph{v2.0:} 2018/12/30

\begin{itemize}
\item
immediate forward processing
\item
added |\childdocby| mechanism
\item
manual restructured
\end{itemize}

%%%%%%%%%%%%%%%%%%%%%%%%%%%%%%%%%%%%%%%%
\paragraph{v1.6:} 2018/01/17

\begin{itemize}
\item
application for development of include files
\item
corrections to manual
\end{itemize}

%%%%%%%%%%%%%%%%%%%%%%%%%%%%%%%%%%%%%%%%
\paragraph{v1.5:} 2017/05/21

\begin{itemize}
\item
more complete structuring introduced
\item
|\childdocof| introduced
\item
|\childdoc| renamed to |\childdocmain|
\item
|\childredirect| renamed to |\childdocforward| and |\childdocforwardprefix|
and functionality expanded
\end{itemize}

%%%%%%%%%%%%%%%%%%%%%%%%%%%%%%%%%%%%%%%%
\paragraph{v1.0:} 2017/04/27

\begin{itemize}
\item
manual and install package
\item
first version published on CTAN
\end{itemize}

%%%%%%%%%%%%%%%%%%%%%%%%%%%%%%%%%%%%%%%%
\paragraph{v0.6:} 2017/04/26

\begin{itemize}
\item
redirection mechanism added
\end{itemize}

%%%%%%%%%%%%%%%%%%%%%%%%%%%%%%%%%%%%%%%%
\paragraph{v0.5:} 2017/04/26

\begin{itemize}
\item
functionality in definition file
\end{itemize}


%%%%%%%%%%%%%%%%%%%%%%%%%%%%%%%%%%%%%%%%%%%%%%%%%%%%%%%%%%%%%%%%%%%%%%%%%%%%%%%%
%%%%%%%%%%%%%%%%%%%%%%%%%%%%%%%%%%%%%%%%%%%%%%%%%%%%%%%%%%%%%%%%%%%%%%%%%%%%%%%%
%%%%%%%%%%%%%%%%%%%%%%%%%%%%%%%%%%%%%%%%%%%%%%%%%%%%%%%%%%%%%%%%%%%%%%%%%%%%%%%%
\appendix

\settowidth\MacroIndent{\rmfamily\scriptsize 000\ }

 \DocInput{childdoc.dtx}

\end{document}
%</driver>
% \fi
%
% %%%%%%%%%%%%%%%%%%%%%%%%%%%%%%%%%%%%%%%%%%%%%%%%%%%%%%%%%%%%%%%%%%%%%%%%%%%%%%
% %%%%%%%%%%%%%%%%%%%%%%%%%%%%%%%%%%%%%%%%%%%%%%%%%%%%%%%%%%%%%%%%%%%%%%%%%%%%%%
% \section{Sample}
%\iffalse
%<*samplemain>
%\fi
%
% The following presents a sample document
% with two chapters, two parts, a title page,
% a compile flag as well as three forwarding files to set the flag.
% It consists of eight |.tex| files:
% \begin{center}
% \begin{tabular}{ll}
% |cdocsamp.tex|&main file\\
% |cdocsch1.tex|&include file for chapter 1\\
% |cdocsch2.tex|&include file for chapter 2\\
% |cdocspt3.tex|&include file for part 3\\
% |cdocspt4.tex|&include file for part 4\\
% |cdocsdrf.tex|&forwarding file for main file in draft mode\\
% |cdocsfi1.tex|&forwarding file for final version of chapter 1\\
% |cdocsfi2.tex|&forwarding file for final version of chapter 2\\
% \end{tabular}
% \end{center}
% Each of the eight files can be compiled directly by the \LaTeX{} compiler.
%
% %%%%%%%%%%%%%%%%%%%%%%%%%%%%%%%%%%%%%%
% \paragraph{Main File.}
%
% The main file is called |cdocsamp.tex|.
%
% Load the \textsf{childdoc} definitions and
% declare the filename for the main document:
%    \begin{macrocode}
\input{childdoc.def}
\childdocmain{}
%    \end{macrocode}

% Optional override for |\version| flag:
%    \begin{macrocode}
%%\ifchilddoc\else\providecommand{\version}{draft}\fi
%    \end{macrocode}

% Define the default values for the |\version| flag
% (|final| for the main file and |draft| for childs):
%    \begin{macrocode}
\ifchilddoc
\providecommand{\version}{draft}
\else
\providecommand{\version}{final}
\fi
%    \end{macrocode}

% Load the standard document class:
%    \begin{macrocode}
\documentclass[12pt]{article}
%    \end{macrocode}

% Start the document body:
%    \begin{macrocode}
\begin{document}
%    \end{macrocode}

% Declare a title page.
% Print title, part of document being processed and version flag:
%    \begin{macrocode}
\addtocounter{page}{-1}
\begin{center}
{\LARGE\bfseries{}childdoc example\par}
\vspace{1cm}
\ifchilddoc
\ifchilddocmanual part\else chapter\fi:
`\childdocname' of `\childdocjob'\par
\else
main document: `\childdocjob'\par
\fi
version: \version\par
\end{center}
\newpage
%    \end{macrocode}

% Manually include selected file,
% otherwise process as usual:
%    \begin{macrocode}
\ifchilddocmanual
\section*{part `\childdocname'}
\input{\childdocname}
\else
%    \end{macrocode}

% Include the two chapters:
%    \begin{macrocode}
\include{cdocsch1}
\include{cdocsch2}
%    \end{macrocode}

% Include the two parts unless only chapters should be displayed:
%    \begin{macrocode}
\ifchilddoc\else
\section{part three}
\input{cdocspt3}
\section{part four}
\input{cdocspt4}
\fi
%    \end{macrocode}

% Process as usual until here:
%    \begin{macrocode}
\fi
%    \end{macrocode}

% End of document body:
%    \begin{macrocode}
\end{document}
%    \end{macrocode}
%\iffalse
%</samplemain>
%\fi
%
% %%%%%%%%%%%%%%%%%%%%%%%%%%%%%%%%%%%%%%
% \paragraph{Chapter Include Files.}
%
% The include files are called |cdocsch1.tex| and |cdocsch2.tex|.
%
%\iffalse
%<*samplechap1|samplechap2>
%\fi

% Optional override for |\version| flag:
%    \begin{macrocode}
%%\providecommand{\version}{final}
%    \end{macrocode}

% Include the main document:
%    \begin{macrocode}
\input{childdoc.def}
\childdocof{cdocsamp}
%    \end{macrocode}

%\iffalse
%</samplechap1|samplechap2>
%\fi
%
%\iffalse
%<*samplechap1>
%\fi
% Some text for chapter 1:
%    \begin{macrocode}
\section{one}
some text in chapter one
%    \end{macrocode}

%\iffalse
%</samplechap1>
%\fi
% Some text for chapter 2:
%\iffalse
%<*samplechap2>
%\fi
%    \begin{macrocode}
\section{two}
more text in chapter two
%    \end{macrocode}

%\iffalse
%</samplechap2>
%\fi
%
% %%%%%%%%%%%%%%%%%%%%%%%%%%%%%%%%%%%%%%
% \paragraph{Part Include Files.}
%
% The include files are called |cdocspt3.tex| and |cdocspt4.tex|.
%
%\iffalse
%<*samplepart3|samplepart4>
%\fi

% Optional override for |\version| flag:
%    \begin{macrocode}
%%\providecommand{\version}{final}
%    \end{macrocode}

% Include the main document:
%    \begin{macrocode}
\input{childdoc.def}
\childdocby{cdocsamp}
%    \end{macrocode}

%\iffalse
%</samplepart3|samplepart4>
%\fi
%
%\iffalse
%<*samplepart3>
%\fi
% Some text for part 3:
%    \begin{macrocode}
some text in part three
%    \end{macrocode}

%\iffalse
%</samplepart3>
%\fi
% Some text for part 4:
%\iffalse
%<*samplepart4>
%\fi
%    \begin{macrocode}
more text in part four
%    \end{macrocode}

%\iffalse
%</samplepart4>
%\fi
%
% %%%%%%%%%%%%%%%%%%%%%%%%%%%%%%%%%%%%%%
% \paragraph{Forwarding for a Complete Draft.}
%
% The following forwarding file |cdocsdrf.tex|
% compiles the main document in draft mode:
%\iffalse
%<*sampledraft>
%\fi
%    \begin{macrocode}
\def\version{draft}
\input{childdoc.def}
\childdocforward{cdocsamp}
%    \end{macrocode}

%\iffalse
%</sampledraft>
%\fi
%
% %%%%%%%%%%%%%%%%%%%%%%%%%%%%%%%%%%%%%%
% \paragraph{Forwarding for Final Version of the Chapters.}
%
% The following forwarding files |cdocsfn1.tex| and |cdocsfn2.tex|
% (with identical content)
% compile the final versions of the child documents
% |cdocsch1.tex| and |cdocsch2.tex|, respectively:
%\iffalse
%<*samplefinal>
%\fi
%    \begin{macrocode}
\def\version{final}
\input{childdoc.def}
\childdocforwardprefix[cdocsamp]{cdocsfn}{cdocsch}
%    \end{macrocode}

%\iffalse
%</samplefinal>
%\fi
%
% %%%%%%%%%%%%%%%%%%%%%%%%%%%%%%%%%%%%%%
% \paragraph{Command Line Processing.}
%
% The following three command lines generate the output files
% |cdocscld|, |cdocscl1| and |cdocscl2|
% which should be identical to
% |cdocsdrf|, |cdocsch1| and |cdocsfn2|, respectively:
% \begin{center}
% \begin{tabular}{l}
% |latex -jobname cdocscld \|\\
% |  "\def\version{draft}\input{childdoc.def}\childdocforward{cdocsamp}"|\\
% |latex -jobname cdocscl1 \|\\
% |  "\input{childdoc.def}\childdocforward[cdocsamp]{cdocsch1}"|\\
% |latex -jobname cdocscl2 \|\\
% |  "\def\version{final}\input{childdoc.def}\childdocforward{cdocsch2}"|
% \end{tabular}
% \end{center}
% Note that the trailing backslash on each first line
% merely continues the input to the second line
% (for convenient cut ant paste).
% Furthermore, the command |latex| can be replaced by any
% of its alternative versions such as |pdflatex|.
%
% %%%%%%%%%%%%%%%%%%%%%%%%%%%%%%%%%%%%%%%%%%%%%%%%%%%%%%%%%%%%%%%%%%%%%%%%%%%%%%
% %%%%%%%%%%%%%%%%%%%%%%%%%%%%%%%%%%%%%%%%%%%%%%%%%%%%%%%%%%%%%%%%%%%%%%%%%%%%%%
% \section{Implementation}
%\iffalse
%<*package>
%\fi
%
% This section describes the definitions file |childdoc.def|.

% The definitions cannot be loaded using |\usepackage| or |\RequirePackage|
% which has a mechanism to prevent loading a style file more than once.
% When loading the definitions by means of |\input|
% multiple instances have to be prevented manually:
%\iffalse
%This code needs to be before the `\ProvidesFile' directive
%which is defined at the beginning of this file.
%Therefore it is also placed there and commented out here.
%</package>
%<*discard>
%\fi
%    \begin{macrocode}
\ifdefined\childdocmain\endinput\fi
%    \end{macrocode}
%\iffalse
%</discard>
%<*package>
%\fi
%
% \macro{\ifchilddoc}
% \macro{\ifchilddocmanual}
% The conditional |\ifchilddoc| tells whether a
% child (true) or main (false) document is being compiled.
% The conditional |\ifchilddocmanual| tells whether
% the |\includeonly| mechanism is used (false) or
% the selection of child files must be performed manually (true).
% The definitions initialise to false:
%    \begin{macrocode}
\newif\ifchilddoc
\newif\ifchilddocmanual
%    \end{macrocode}

% \macro{\childdocname}
% \macro{\childdocjob}
% The macro |\childdocname| stores the name of the main document
% to be compiled. The macro |\childdocjob| stores the name of
% the document on which the \LaTeX{} compiler was originally invoked.
% The content of |\jobname| cannot be compared
% to filenames specified in the source due to different catcodes.
% The following code rescans |\jobname|, stores the result
% in |\childdocname| and saves a copy in |\childdocjob|:
%    \begin{macrocode}
\edef\childdocname{\scantokens\expandafter{\jobname\noexpand}}
\let\childdocjob\childdocname
%    \end{macrocode}

% \macro{\childdocdisable}
% The macro |\childdocdisable| prevents the main file
% from being processed more than once.
% At this stage, the main document command |\childdocmain|
% is assumed to be called once again where it should do nothing.
% Any subsequent call to it should prevent
% a secondary processing of the main document
% It overwrites the forwarding commands
% |\childdocof| and |\childdocforward|
% with empty macros to prevent further inclusions of the main document:
%    \begin{macrocode}
\newcommand{\childdocdisable}
{
  \renewcommand{\childdocmain}[1]{\renewcommand{\childdocmain}[1]{\endinput}}
  \renewcommand{\childdocof}[1]{}
  \renewcommand{\childdocby}[2][]{}
  \renewcommand{\childdocforward}[2][]{}
  \renewcommand{\childdocdisable}{}
}
%    \end{macrocode}

% \macro{\childdocmain}
% The macro |\childdocmain| is to be called at the top of the main file
% with nothing or the main filename (without extension) as argument.
% First, it breaks loops.
% If the argument is not empty and does not match |\childdocname|
% (which is set by the first inclusion of |childdoc.def|),
% |\ifchilddoc| is set to true, |\includeonly| is applied to the child file
% and |\jobname| is set to the main file
% (for proper handling of |.aux| files):
%    \begin{macrocode}
\newcommand{\childdocmain}[1]
{
  \childdocdisable\childdocmain{}
  \if?#1?\else
    \begingroup
      \def\childdoctmp{#1}
      \ifx\childdoctmp\childdocname
        \def\childdoctmp{}
      \else
        \def\childdoctmp
        {
          \childdoctrue
          \includeonly{\childdocname}
          \def\childdocjob{#1}
          \def\jobname{#1}
        }
      \fi
      \expandafter
    \endgroup
    \childdoctmp
  \fi
}
%    \end{macrocode}

% \macro{\childdocof}
% The command |\childdocof| redirects
% compilation to the main file |#1|.
%    \begin{macrocode}
\newcommand{\childdocof}[1]
{
  \childdocdisable
  \childdoctrue
  \includeonly{\childdocname}
  \def\jobname{#1}
  \def\childdocjob{#1}
  \input{#1}
}
%    \end{macrocode}

% \macro{\childdocby}
% The command |\childdocby| ....
%    \begin{macrocode}
\newcommand{\childdocby}[2][]
{
  \childdocdisable
  \childdoctrue
  \childdocmanualtrue
  \if?#1?\else
    \def\jobname{#2}
  \fi
  \def\childdocjob{#2}
  \input{#2}
  \endinput
}
%    \end{macrocode}

% \macro{\childdocforward}
% The command |\childdocforward| redirects
% compilation to the main file or
% (if the optional argument is given) a child file.
% Parameters are set as if the main file
% or a child file starting with |\childdocof| was compiled.
% Then compilation is handed over to the main file:
%    \begin{macrocode}
\newcommand{\childdocforward}[2][]
{
  \begingroup
    \if?#1?
      \def\childdoctmp
      {
        \def\childdocname{#2}
        \def\childdocjob{#2}
        \def\jobname{#2}
        \input{#2}
        \endinput
      }
    \else
      \def\childdoctmp
      {
        \childdocdisable
        \def\childdocname{#2}
        \childdoctrue
        \includeonly{#2}
        \def\childdocjob{#1}
        \def\jobname{#1}
        \input{#1}
        \endinput
      }
    \fi
    \expandafter
  \endgroup
  \childdoctmp
}
%    \end{macrocode}

% \macro{\childdocforwardprefix}
% The command |\childdocforwardprefix| redirects
% compilation to the main or a child file by means of a pattern.
% The prefix |#1| in the current filename is replaced by |#2|
% and the suffix of the current filename is kept
% (it is assumed that the filename does not contain the substring `|~~~|'
% which is used as a delimiter).
% Compilation is handed over to the new file by |\childdocforward|:
%    \begin{macrocode}
\newcommand{\childdocforwardprefix}[3][]
{
  \begingroup
    \def\childdocextract #2##1~~~{\def\childdoctmp{\childdocforward[#1]{#3##1}}}
    \expandafter\childdocextract\childdocname~~~
    \expandafter
  \endgroup
  \childdoctmp
}
%    \end{macrocode}

% \macro{\childdoc}
% The deprecated macro |\childdoc| is a legacy version of |\childdocmain|:
%    \begin{macrocode}
\newcommand{\childdoc}{\childdocmain}
%    \end{macrocode}

% \macro{\childdocredirect}
% The deprecated macro |\childdocredirect| is a legacy version
% of |\childdocforward| and |\childdocforwardprefix|:
%    \begin{macrocode}
\newcommand{\childdocredirect}[2][]
{
  \begingroup
    \if?#1?
      \def\childdoctmp{\childdocforward{#2}}
    \else
      \def\childdoctmp{\childdocforwardprefix{#1}{#2}}
    \fi
    \expandafter
  \endgroup
  \childdoctmp
}
%    \end{macrocode}

%\iffalse
%</package>
%\fi
%
\endinput

\childdocforward{cdocsamp}
%    \end{macrocode}

%\iffalse
%</sampledraft>
%\fi
%
% %%%%%%%%%%%%%%%%%%%%%%%%%%%%%%%%%%%%%%
% \paragraph{Forwarding for Final Version of the Chapters.}
%
% The following forwarding files |cdocsfn1.tex| and |cdocsfn2.tex|
% (with identical content)
% compile the final versions of the child documents
% |cdocsch1.tex| and |cdocsch2.tex|, respectively:
%\iffalse
%<*samplefinal>
%\fi
%    \begin{macrocode}
\def\version{final}
% \iffalse
%
% childdoc.dtx Copyright (C) 2017-2018 Niklas Beisert
%
% This work may be distributed and/or modified under the
% conditions of the LaTeX Project Public License, either version 1.3
% of this license or (at your option) any later version.
% The latest version of this license is in
%   http://www.latex-project.org/lppl.txt
% and version 1.3 or later is part of all distributions of LaTeX
% version 2005/12/01 or later.
%
% This work has the LPPL maintenance status `maintained'.
%
% The Current Maintainer of this work is Niklas Beisert.
%
% This work consists of the files childdoc.dtx and childdoc.ins
% and the derived files childdoc.def and cdocsamp.tex with
% cdocsch1.tex, cdocsch2.tex, cdocsdrf.tex, cdocsfn1.tex, cdocsfn2.tex.
%
%<package>\ifdefined\childdocmain\endinput\fi
%<package>\ProvidesFile{childdoc.def}[2018/12/30 v2.0 child document driver]
%<samplemain>\ProvidesFile{cdocsamp.tex}[2018/12/30 v2.0 sample for childdoc]
%<*driver>
%\ProvidesFile{childdoc.drv}[2018/12/30 v2.0 childdoc reference manual file]
\PassOptionsToClass{10pt,a4paper}{article}
\documentclass{ltxdoc}

\usepackage[margin=35mm]{geometry}
\usepackage{hyperref}
\usepackage{hyperxmp}
\usepackage[usenames]{color}

\hypersetup{colorlinks=true}
\hypersetup{pdfstartview=FitH}
\hypersetup{pdfpagemode=UseNone}
\hypersetup{pdfsource={}}
\hypersetup{pdflang={en-UK}}
\hypersetup{pdfcopyright={Copyright 2017-2018 Niklas Beisert.
  This work may be distributed and/or modified under the
  conditions of the LaTeX Project Public License, either version 1.3
  of this license or (at your option) any later version.}}
\hypersetup{pdflicenseurl={http://www.latex-project.org/lppl.txt}}
\hypersetup{pdfcontactaddress={ETH Zurich, ITP, HIT K,
  Wolfgang-Pauli-Strasse 27}}
\hypersetup{pdfcontactpostcode={8093}}
\hypersetup{pdfcontactcity={Zurich}}
\hypersetup{pdfcontactcountry={Switzerland}}
\hypersetup{pdfcontactemail={nbeisert@itp.phys.ethz.ch}}
\hypersetup{pdfcontacturl={http://people.phys.ethz.ch/\xmptilde nbeisert/}}

\newcommand{\secref}[1]{\hyperref[#1]{section \ref*{#1}}}

\parskip1ex
\parindent0pt
\let\olditemize\itemize
\def\itemize{\olditemize\parskip0pt}

\begin{document}

\title{The \textsf{childdoc} Package}
\hypersetup{pdftitle={The childdoc Package}}
\author{Niklas Beisert\\[2ex]
  Institut f\"ur Theoretische Physik\\
  Eidgen\"ossische Technische Hochschule Z\"urich\\
  Wolfgang-Pauli-Strasse 27, 8093 Z\"urich, Switzerland\\[1ex]
  \href{mailto:nbeisert@itp.phys.ethz.ch}
  {\texttt{nbeisert@itp.phys.ethz.ch}}}
\hypersetup{pdfauthor={Niklas Beisert}}
\hypersetup{pdfsubject={Manual for the LaTeX2e Package childdoc}}
\date{30 December 2018, \textsf{v2.0}}
\maketitle

\begin{abstract}\noindent
\textsf{childdoc} is a \LaTeXe{} package
that enables the direct compilation
of document sections included by |\include|
to individual files.
\end{abstract}

\begingroup
\parskip0ex
\tableofcontents
\endgroup

%%%%%%%%%%%%%%%%%%%%%%%%%%%%%%%%%%%%%%%%%%%%%%%%%%%%%%%%%%%%%%%%%%%%%%%%%%%%%%%%
%%%%%%%%%%%%%%%%%%%%%%%%%%%%%%%%%%%%%%%%%%%%%%%%%%%%%%%%%%%%%%%%%%%%%%%%%%%%%%%%
\section{Introduction}

\LaTeX{} provides a mechanism to structure a large document (such as a book)
into a main file and several child files (containing the chapters)
using the |\include| command.
This mechanism is beneficial for documents
which span hundreds of pages in order to
make the source file(s) more manageable.
Moreover, compilation can be restricted to
selected child files by means of the |\includeonly| command.
The latter feature can be used to reduce the compilation time while editing
(this was significantly more useful in the earlier days of \LaTeX{})
or to generate a smaller document which is easier to navigate.
Another application of |\includeonly| is to generate
documents consisting of selected parts of the complete document.

However, there are a few drawbacks of the plain |\include| mechanism:
\begin{itemize}
\item
The child files cannot be compiled on their own,
they can only be compiled via the main file.
A naive editing environment
(such as a text editor with an option
to have the current file processed by \LaTeX)
may require one to switch to the main file before compiling;
attempting to compile the child file produces errors.
\item
The main file must be modified (each time)
to adjust the |\includeonly| command
to the present needs. This easily leaves the main file in a messy state.
\item
The generated document will always carry the filename
of the main document. This is inconvenient if
several child files are to be compiled and
to be kept for distribution.
\end{itemize}

The present package provides a simple interface
to make child files individually compilable by \LaTeX{}.
Compiling a child file then has the same effect as compiling
the main file with an |\includeonly| command
to select the appropriate child.
Moreover the generated document will carry the name of the child
rather than the main file.
This resolves all three above issues.

This feature is meant to make the editing of books,
thesis documents and lecture notes somewhat more convenient.
However, the package can also be used efficiently for
composing a series of documents (such as exercise sheets)
which are typically distributed individually.
It then assists the author in generating the individual documents
(potentially in different versions)
as well as a document containing the collected series.
Another application is in developing style files
or other kinds of included material
where compilation of the style file could redirect
to a sample or test file.

%%%%%%%%%%%%%%%%%%%%%%%%%%%%%%%%%%%%%%%%%%%%%%%%%%%%%%%%%%%%%%%%%%%%%%%%%%%%%%%%
%%%%%%%%%%%%%%%%%%%%%%%%%%%%%%%%%%%%%%%%%%%%%%%%%%%%%%%%%%%%%%%%%%%%%%%%%%%%%%%%
\section{Usage}

First of all, the package \textsf{childdoc} is \emph{not} a standard
\LaTeXe{} |.sty| style file! Therefore it needs to be invoked in
a non-standard way.

%%%%%%%%%%%%%%%%%%%%%%%%%%%%%%%%%%%%%%%%%%%%%%%%%%%%%%%%%%%%%%%%%%%%%%%%%%%%%%%%
\subsection{Included Files}
\label{sec:include}

%%%%%%%%%%%%%%%%%%%%%%%%%%%%%%%%%%%%%%%%
\DescribeMacro{\childdocmain}
To use the package, add the commands
\begin{center}
\begin{tabular}{l}
|\input{childdoc.def}|\\
|\childdocmain{}|\\
\end{tabular}
\end{center}
at the very top of the main \LaTeX{} file,
in particular \emph{before} the |\documentclass| statement!
The argument of |\childdocmain| should be left empty
(but it must be present).

%%%%%%%%%%%%%%%%%%%%%%%%%%%%%%%%%%%%%%%%
\DescribeMacro{\childdocof}
Furthermore, add the commands
\begin{center}
\begin{tabular}{l}
|\input{childdoc.def}|\\
|\childdocof{|\textit{main}|}|\\
\end{tabular}
\end{center}
at the top of every child file \textit{child}
which is included by |\include{|\textit{child}|}|
from within the main file
(or at least for those files to be compiled individually).
The argument \textit{main} must be the filename of the main file.

There are a couple of
considerations in setting up the main and child documents:

%%%%%%%%%%%%%%%%%%%%%%%%%%%%%%%%%%%%%%%%
\paragraph{Restrictions.}

Please note the following restrictions:
\begin{itemize}
\item
|\childdocmain| must be called with one argument \textit{main}
to ensure compatibility with earlier version of the package.
It must either be empty (|\childdocmain{}|)
or precisely match the filename of the main file in which it is specified.
See \secref{sec:detection} for further information.
\item
The filename \textit{main} must be specified without the |.tex| extension.
\item
The filename \textit{main} is case sensitive
(even in case-insensitive file systems)
due to internal string comparison.
\item
The argument \textit{main} should be fully expanded, it cannot be a macro.
\item
Subdirectories and special characters should be avoided in filenames.
\item
The command |\childdocmain{|\textit{main}|}| must be followed by a whitespace.
It should not be followed immediately by another command
or by a comment mark `|%|'.
This is because the \TeX{} parser reads the token immediately following
the argument of |\childdocmain| and puts it
at the beginning of every child section;
however, a white\-space is ignored.
\end{itemize}

%%%%%%%%%%%%%%%%%%%%%%%%%%%%%%%%%%%%%%%%
\paragraph{Content of Main File.}

It is advisable to place all content in the child files included by |\include|.
Any output contained in the main file will appear in all child documents
unless suppressed manually;
it cannot be suppressed automatically by the |\includeonly| directive
and thus should normally be avoided.
A method to include some content in the main file
by means of conditional processing is described in \secref{sec:conditional}.

%%%%%%%%%%%%%%%%%%%%%%%%%%%%%%%%%%%%%%%%
\paragraph{Page Numbering.}

When only a part of the document is compiled,
the appropriate numbering of pages
(as well as other status parameters)
is determined from the |.aux| files.
The latter contain information from previous passes.
However this information needs to propagate through
all intermediate child documents.
Therefore the page numbering in child documents may well
be inconsistent until the complete document is compiled at least once.

A useful (if unconventional) way to always ensure a consistent
page numbering is to restart the numbering in each child document
and denote the pages by `\textit{child}|.|\textit{page}'
where \textit{child} represents the chapter/section number of the child file.
This can be achieved by the command
|\numberwithin{page}{|\textit{child}|}|
of the \textsf{amsmath} package
where \textit{child} can be |chapter| or |section|
depending on the chosen structuring.
Alternatively, one can modify the macro |\thepage| appropriately
and reset the counter |page| at the start of each child file.

%%%%%%%%%%%%%%%%%%%%%%%%%%%%%%%%%%%%%%%%%%%%%%%%%%%%%%%%%%%%%%%%%%%%%%%%%%%%%%%%
\subsection{Conditional Processing}
\label{sec:conditional}

The package provides a mechanism to compile different versions
of a document. To customise the versions further some conditional processing
can come in handy to distinguish which version is being compiled.
The package provides two macros to describe the compilation context:

%%%%%%%%%%%%%%%%%%%%%%%%%%%%%%%%%%%%%%%%
\DescribeMacro{\ifchilddoc}
The conditional |\ifchilddoc| distinguishes between the compilation of
child documents and the main document:
%
\begin{center}
|\ifchilddoc |\textit{child-code}| |[|\||else |\textit{main-code}]| \||fi|
\end{center}

%%%%%%%%%%%%%%%%%%%%%%%%%%%%%%%%%%%%%%%%
\DescribeMacro{\childdocname}
\DescribeMacro{\childdocjob}
The macro |\childdocname| contains the filename (without extension)
of the main or child file being processed.
Note that |\childdocjob| will always contain the name of the main file.

%%%%%%%%%%%%%%%%%%%%%%%%%%%%%%%%%%%%%%%%
\paragraph{Title Page.}

Conditional processing can be used to include a title or banner page
in the main document when proper precautions are taken.
Importantly, the code in the main file should ensure that the page counter
(as well as other status parameters which are stored in the |.aux| files)
takes the same value after the conditional processing.
Otherwise the page numbers may take divergent values
depending on which part is compiled.

For example, a title page could be declared by:
%
\begin{center}
\begin{tabular}{l}
|\ifchilddoc\||else|\\
|\addtocounter{page}{-1}|\\
\textit{code for title page}\\
|\newpage|\\
|\||fi|
\end{tabular}
\end{center}
%
A banner page for the child documents can be generated by:
%
\begin{center}
\begin{tabular}{l}
|\ifchilddoc|\\
|\addtocounter{page}{-1}|\\
\textit{code for banner page}\\
|\newpage|\\
|\||fi|
\end{tabular}
\end{center}
%
Here one could write a message such as:
\begin{center}
|This is the part \childdocname{} of \childdocjob{}.|
\end{center}

%%%%%%%%%%%%%%%%%%%%%%%%%%%%%%%%%%%%%%%%%%%%%%%%%%%%%%%%%%%%%%%%%%%%%%%%%%%%%%%%
\subsection{Flags}
\label{sec:flags}

The package makes it easy to generate different versions
of the main or child documents.
To this end compilation flags can be defined
and assigned different default values.
They will be particularly useful in conjunction
with the forwarding mechanism described in \secref{sec:forward}.

For example, it may be useful to have a flag |\version|
which can be set to |draft| or |final|.
The document source will contain some conditional code
depending on the value of |\version|.
Suppose further, the flag should default to |final| for the main file
and to |draft| for child files
which is a natural assignment for editing the document.
This is achieved by placing the following code
in the preamble of the main document
(below the |\childdocmain| directive):
%
\begin{center}
\begin{tabular}{l}
|\ifchilddoc|\\
|\providecommand{\version}{draft}|\\
|\||else|\\
|\providecommand{\version}{final}|\\
|\||fi|
\end{tabular}
\end{center}
%
The definition by |\providecommand| makes sure
that previous definitions are not overwritten.
Further statements |\providecommand{\version}{...}|
can thus be added before the above code to override it.

For the main file, one might add a line
(between |\childdocmain| and the above block)
%
\begin{center}
|%\ifchilddoc\||else\providecommand{\version}{draft}\||fi|
\end{center}
%
which can be uncommented to produce a draft version.
Likewise one can add a line to the very top of a child file
(above the |\childdocof{|\textit{main}|}| directive)
%
\begin{center}
|%\providecommand{\version}{final}|
\end{center}
%
which can be uncommented to produce the final version of this child document.

%%%%%%%%%%%%%%%%%%%%%%%%%%%%%%%%%%%%%%%%%%%%%%%%%%%%%%%%%%%%%%%%%%%%%%%%%%%%%%%%
\subsection{Forwarding}
\label{sec:forward}

Different versions of the main or child documents
using compilation flags as described in \secref{sec:flags}
can be (permanently) stored in different files
for convenient compilation, viewing and distribution.
To this end, the package defines a command
to pass on compilation to a different file:

%%%%%%%%%%%%%%%%%%%%%%%%%%%%%%%%%%%%%%%%
\DescribeMacro{\childdocforward}
The command |\childdocforward| redirects processing to
another source file:
%
\begin{center}
\begin{tabular}{l}
|\input{childdoc.def}|\\
|\childdocforward[|\textit{main}|]{|\textit{dest}|}|\\
\end{tabular}
\end{center}
%
The argument \textit{dest} is the destination file
(without extension).
It should be the main file or one of the child files.
Note that further \textsf{childdoc} directives
such as |\childdocof| and |\childdocforward|
in the indicated file will be processed in this form.
The optional argument \textit{main}
passes on directly to the main file \textit{main}
while pretending to compile the child \textit{dest}.
This form behaves as if \textit{dest}
issues |\childdocof{|\textit{main}|}| right away,
and no further \textsf{childdoc} directives will be processed.

%%%%%%%%%%%%%%%%%%%%%%%%%%%%%%%%%%%%%%%%
\DescribeMacro{\...prefix}
In the alternative form |\childdocforwardprefix|,
%
\begin{center}
\begin{tabular}{l}
|\input{childdoc.def}|\\
|\childdocforwardprefix[|\textit{main}|]{|\textit{prefix}|}{|\textit{dest}|}|
\end{tabular}
\end{center}
%
the destination file is determined by a pattern
depending on the current file:
To make this work, the current file must be called
`{\textit{prefix}\hspace{0.2em}\textit{suffix}}'
with \textit{prefix} matching precisely the argument.
Processing is then passed on to the file
`{\textit{dest}\hspace{0.2em}\textit{suffix}}'.
Surely, the same effect is achieved by
directly specifying the
argument `{\textit{dest}\hspace{0.2em}\textit{suffix}}'
in the first form.
However, that requires to set up a different file
for each child. With the alternative form of the command
all these files can have exactly the same content
which simplifies setting them up and maintaining them.

For example, the following file |draft.tex|
with a compilation flag |\version| as described in \secref{sec:flags}
compiles the main document as a draft:
%
\begin{center}
\begin{tabular}{l}
|\def\version{draft}|\\
|\input{childdoc.def}|\\
|\childdocforward{|\textit{main}|}|
\end{tabular}
\end{center}
%
Likewise, the following files |final|\textit{nn}|.tex|
compile the final version of the child document
|child|\textit{nn}|.tex|:
%
\begin{center}
\begin{tabular}{l}
|\def\version{final}|\\
|\input{childdoc.def}|\\
|\childdocforwardprefix{final}{child}|
\end{tabular}
\end{center}
%

Note that when several versions of a main file and/or of each child file
are to be generated, it may be convenient to set up a |Makefile| or
shell script to automatise the process.

%%%%%%%%%%%%%%%%%%%%%%%%%%%%%%%%%%%%%%%%%%%%%%%%%%%%%%%%%%%%%%%%%%%%%%%%%%%%%%%%
\subsection{Command Line Processing}
\label{sec:commandline}

The effect of redirection files can also be achieved by invoking
the \LaTeX{} compiler with a more elaborate command line.
Most conveniently this should be done as part
of a shell script or a |Makefile|.

When using \textsf{childdoc} in the main file, the following
command lines effectively perform a redirection
(note that depending on the shell being used,
backslashes may have to be doubled: `|\|' $\to$ `|\\|'):
%
\begin{center}
|... -jobname "|\textit{target}|" |\\|"|[\textit{flags}]%
|\input{childdoc.def}\childdocforward[|\textit{main}|]{|\textit{dest}|}"|
\end{center}
%
Here \textit{target} is the name of the output file,
\textit{main} is the name of the main file
and \textit{dest} is the name of the main or child file to be processed
(all filenames without extensions).
The optional argument \textit{main} can be omitted
if \textit{main} matches \textit{dest}.
Optionally, compilation \textit{flags} can be defined via |\def| commands.
This command line makes the \TeX{} engine believe
it is compiling the file \textit{target}
whose content is specified as the latter parameter.
The provided code then forwards the processing to
\textit{main} or \textit{dest} as described in \secref{sec:forward}.

%%%%%%%%%%%%%%%%%%%%%%%%%%%%%%%%%%%%%%%%%%%%%%%%%%%%%%%%%%%%%%%%%%%%%%%%%%%%%%%%
\subsection{Include by Input}
\label{sec:input}

Including child documents by |\include| has some restrictions by design.
Most notably, the content of a child document always occupies
its own set of pages; pages cannot be shared between child documents.
Usually, this behaviour makes perfect sense
because each child document contain an essential part of the document.
However, in some situations it may be desirable to compose
a document from a collection of parts
without having mandatory page breaks between then.
For this case, the package
provides a mechanism to include parts
by |\input| which can also be processed individually.
However, by construction this mechanism
requires manual handling of the content to be output.

%%%%%%%%%%%%%%%%%%%%%%%%%%%%%%%%%%%%%%%%
\DescribeMacro{\ifchilddocmanual}
The main file should be prepared as usual, see \secref{sec:include}.
However, the document body must make a distinction
between processing of an individual part and of the main document, e.g.:
%
\begin{center}
\begin{tabular}{l}
|\ifchilddocmanual|\\
|\input{\childdocname}|\\
|\||else|\\
\textit{document body with }|\input{|\textit{part}|}|\\
|\||fi|
\end{tabular}
\end{center}
%
The conditional |\ifchilddocmanual| is true whenever
a part to be included by |\input| is being compiled,
and the name of the part is stored in |\childdocname|.

%%%%%%%%%%%%%%%%%%%%%%%%%%%%%%%%%%%%%%%%
\DescribeMacro{\childdocby}
Each part to be included by |\input| should start with:
%
\begin{center}
\begin{tabular}{l}
|\input{childdoc.def}|\\
|\childdocby{|\textit{main}|}|\\
\end{tabular}
\end{center}
%
The directive |\childdocby| is similar to |\childdocof|
described in \secref{sec:include},
but the subsequent selection of content must be done manually.
To that end, both |\ifchilddoc| and |\ifchilddocmanual|
will be true upon processing of a part,
and the name of the part is stored in |\childdocname|.
Note that |\jobname| will be set to the filename of the current part
so that each part receives an individual |.aux| file
that does not interfere with the |.aux| file(s) of the main document.
This behaviour can be altered by the alternative form
|\childdocby[*]{|\textit{main}|}| (with a non-empty optional argument)
which uses the |.aux| file of the main document
by setting |\jobname| to \textit{main}.

%%%%%%%%%%%%%%%%%%%%%%%%%%%%%%%%%%%%%%%%%%%%%%%%%%%%%%%%%%%%%%%%%%%%%%%%%%%%%%%%
\subsection{Driver Development}
\label{sec:driver}

The \textsf{childdoc} mechanism can also be use for the development
of definition files such as \LaTeX{} styles or classes.
This case differs from the above setup with multiple parts
included by |\include| in that no |\includeonly| should be invoked.
This can be achieved by starting the include file
(before |\ProvidesPackage|) with:
%
\begin{center}
\begin{tabular}{l}
|\input{childdoc.def}|\\
|\childdocforward{|\textit{main}|}|\\
\end{tabular}
\end{center}
%
or alternatively with:
%
\begin{center}
\begin{tabular}{l}
|\input{childdoc.def}|\\
|\childdocby{|\textit{main}|}|\\
\end{tabular}
\end{center}
%
Both forms have slightly different effects as described above.
The main file is prepared as usual, see \secref{sec:include}.

%%%%%%%%%%%%%%%%%%%%%%%%%%%%%%%%%%%%%%%%%%%%%%%%%%%%%%%%%%%%%%%%%%%%%%%%%%%%%%%%
\subsection{Legacy Detection}
\label{sec:detection}

The directive |\childdocmain| in the main file can detect
whether the complete document or merely a child is to be compiled
even without using the directive |\childdocof|.
This method is deprecated because it is less robust
and there is no compelling reason to use it;
it is merely provided for backward compatibility
and it may be removed in future versions.

If the detection mechanism is to be used,
it is mandatory to correctly specify
the filename of the main file as the argument of |\childdocmain|:
%
\begin{center}
\begin{tabular}{l}
|\input{childdoc.def}|\\
|\childdocmain{|\textit{main}|}|\\
\end{tabular}
\end{center}
%
If |\jobname| does not match the argument \textit{main} of |\childdocmain|,
it is assumed that |\jobname| points to the child file to be compiled.
When using |\childdocmain| with the main file specified as argument,
it suffices to start a child file
with just |\input{|\textit{main}|}|
without loading of the package and using |\childdocof|.
If instead all processing is done
with the appropriate \textsf{childdoc} directives,
the argument of \textit{main} of |\childdocmain| can be empty.

An alternative version of the command line processing described
in \secref{sec:commandline} using the detection mechanism reads:
%
\begin{center}
|... -jobname "|\textit{target}|" "|[\textit{flags}]%
[|\def\jobname{|\textit{dest}|}|]|\input{|\textit{main}|}"|
\end{center}

%%%%%%%%%%%%%%%%%%%%%%%%%%%%%%%%%%%%%%%%%%%%%%%%%%%%%%%%%%%%%%%%%%%%%%%%%%%%%%%%
\subsection{Manual Code}
\label{sec:manual}

In case one cannot be certain whether the definitions file |childdoc.def|
is installed on the target \TeX{} distribution
and one prefers not to ship it,
it is conceivable to paste a few relevant commands into the sources.

To that end, drop all statements |\input{childdoc.def}|
and perform the replacements as outlined below.
Instead of |\childdocmain{|\textit{main}|}| add the following code
to the top of the main file:
%
\begin{center}
\begin{tabular}{l}
|\||ifdefined\childdocname\endinput\||fi\newif\ifchilddoc|\\
|\edef\childdocname{\scantokens\expandafter{\jobname\noexpand}}|\\
|\def\childdocmain{|\textit{main}|}\||ifx\childdocmain\childdocname\||else|\\
|\childdoctrue\includeonly{\childdocname}\let\jobname\childdocmain\||fi|\\
\end{tabular}
\end{center}
%
Instead of |\childdocof{|\textit{main}|}| just include the main file
at the top of each child file:
%
\begin{center}
|\input{|\textit{main}|}|
\end{center}
%
A simple redirection |\childdocforward{|\textit{dest}|}| is achieved by:
%
\begin{center}
|\def\jobname{|\textit{dest}|}\input{\jobname}|
\end{center}
%
The redirection with prefix
|\childdocforwardprefix[|\textit{prefix}|]{|\textit{dest}|}|
is accomplished by:
%
\begin{center}
\begin{tabular}{l}
|{\edef\jobname{\scantokens\expandafter{\jobname\noexpand}}|\\
|\def\redirectjob |\textit{prefix}|#1~~~{\gdef\jobname{|\textit{dest}|#1}}|\\
|\expandafter\redirectjob\jobname~~~}\input{\jobname}|
\end{tabular}
\end{center}

In an alternative approach,
child documents can be compiled by a specific command line
without additional code or specific definitions:
%
\begin{center}
|... -jobname "|\textit{target}|" "|[\textit{flags}]%
|\includeonly{|\textit{dest}|}\input{|\textit{main}|}"|
\end{center}
%

%%%%%%%%%%%%%%%%%%%%%%%%%%%%%%%%%%%%%%%%%%%%%%%%%%%%%%%%%%%%%%%%%%%%%%%%%%%%%%%%
%%%%%%%%%%%%%%%%%%%%%%%%%%%%%%%%%%%%%%%%%%%%%%%%%%%%%%%%%%%%%%%%%%%%%%%%%%%%%%%%
\section{Information}

%%%%%%%%%%%%%%%%%%%%%%%%%%%%%%%%%%%%%%%%%%%%%%%%%%%%%%%%%%%%%%%%%%%%%%%%%%%%%%%%
\subsection{Copyright}

Copyright \copyright{} 2017--2018 Niklas Beisert

This work may be distributed and/or modified under the
conditions of the \LaTeX{} Project Public License, either version 1.3
of this license or (at your option) any later version.
The latest version of this license is in
  \url{http://www.latex-project.org/lppl.txt}
and version 1.3 or later is part of all distributions of \LaTeX{}
version 2005/12/01 or later.

This work has the LPPL maintenance status `maintained'.

The Current Maintainer of this work is Niklas Beisert.

This work consists of the files |README.txt|, |childdoc.ins| and |childdoc.dtx|
as well as the derived files |childdoc.def|, |cdocsamp.tex|
with |cdocsch1.tex|, |cdocsch2.tex|, |cdocspt3.tex|, |cdocspt4.tex|,
|cdocsdrf.tex|, |cdocsfn1.tex|, |cdocsfn2.tex|
as well as |childdoc.pdf|.

%%%%%%%%%%%%%%%%%%%%%%%%%%%%%%%%%%%%%%%%%%%%%%%%%%%%%%%%%%%%%%%%%%%%%%%%%%%%%%%%
\subsection{Files and Installation}

The package consists of the files:
%
\begin{center}
\begin{tabular}{ll}
    |README.txt|   & readme file \\
    |childdoc.ins| & installation file \\
    |childdoc.dtx| & source file \\
    |childdoc.def| & definition file \\
    |cdocsamp.tex| & sample main file \\
    |cdocsch1.tex| & sample include file \\
    |cdocsch2.tex| & sample include file \\
    |cdocspt3.tex| & sample part file \\
    |cdocspt4.tex| & sample part file \\
    |cdocsdrf.tex| & sample redirection file \\
    |cdocsfn1.tex| & sample redirection file \\
    |cdocsfn2.tex| & sample redirection file \\
    |childdoc.pdf| & manual
\end{tabular}
\end{center}
%
The distribution consists of the files
|README.txt|, |childdoc.ins| and |childdoc.dtx|.
%
\begin{itemize}
\item
Run (pdf)\LaTeX{} on |childdoc.dtx|
to compile the manual |childdoc.pdf| (this file).
\item
Run \LaTeX{} on |childdoc.ins| to create the definitions file |childdoc.def|
and the sample |cdocsamp.tex| with include files
|cdocsch1.tex|, |cdocsch2.tex|, |cdocspt3.tex|, |cdocspt4.tex|,
|cdocsdrf.tex|, |cdocsfn1.tex|, |cdocsfn2.tex|.
Then copy the file |childdoc.def| to an appropriate directory of your \LaTeX{}
distribution, e.g.\ \textit{texmf-root}|/tex/latex/childdoc|.
\end{itemize}

%%%%%%%%%%%%%%%%%%%%%%%%%%%%%%%%%%%%%%%%%%%%%%%%%%%%%%%%%%%%%%%%%%%%%%%%%%%%%%%%
\subsection{Related CTAN Packages}

There are several other packages which offer a similar functionality:
%
\begin{itemize}
\item
The packages
\href{http://ctan.org/pkg/docmute}{\textsf{docmute}},
\href{http://ctan.org/pkg/includex}{\textsf{includex}} and
\href{http://ctan.org/pkg/standalone}{\textsf{standalone}}
provide commands to include only the document body of
a child file thus allowing both files to be compiled individually.
\item
The packages \href{http://ctan.org/pkg/subdocs}{\textsf{subdocs}}
and \href{http://ctan.org/pkg/subfiles}{\textsf{subfiles}}
provide structures in which the main and child documents can be
encapsulated and allowing them to be compiled individually.
The inclusion mechanism is different from the conventional |\include|.
\item
The package \href{http://ctan.org/pkg/combine}{\textsf{combine}}
is an elaborate solution to combine several documents into one.
\end{itemize}
%
See also the CTAN topic \href{http://ctan.org/topic/subdocs}{\textsf{subdocs}}
for further related packages.
The present package differs from the above solutions in that
a document structure constructed with the conventional |\include| mechanism
just needs two extra commands at the top of every file
such that all constituent files can be compiled individually.

%%%%%%%%%%%%%%%%%%%%%%%%%%%%%%%%%%%%%%%%%%%%%%%%%%%%%%%%%%%%%%%%%%%%%%%%%%%%%%%%
%\subsection{Feature Suggestions}
%
%The following is a list of features which may be useful for future
%versions of this package:
%%
%\begin{itemize}
%\item
%\ldots
%\end{itemize}

%%%%%%%%%%%%%%%%%%%%%%%%%%%%%%%%%%%%%%%%%%%%%%%%%%%%%%%%%%%%%%%%%%%%%%%%%%%%%%%%
\subsection{Revision History}

%%%%%%%%%%%%%%%%%%%%%%%%%%%%%%%%%%%%%%%%
\paragraph{v2.0:} 2018/12/30

\begin{itemize}
\item
immediate forward processing
\item
added |\childdocby| mechanism
\item
manual restructured
\end{itemize}

%%%%%%%%%%%%%%%%%%%%%%%%%%%%%%%%%%%%%%%%
\paragraph{v1.6:} 2018/01/17

\begin{itemize}
\item
application for development of include files
\item
corrections to manual
\end{itemize}

%%%%%%%%%%%%%%%%%%%%%%%%%%%%%%%%%%%%%%%%
\paragraph{v1.5:} 2017/05/21

\begin{itemize}
\item
more complete structuring introduced
\item
|\childdocof| introduced
\item
|\childdoc| renamed to |\childdocmain|
\item
|\childredirect| renamed to |\childdocforward| and |\childdocforwardprefix|
and functionality expanded
\end{itemize}

%%%%%%%%%%%%%%%%%%%%%%%%%%%%%%%%%%%%%%%%
\paragraph{v1.0:} 2017/04/27

\begin{itemize}
\item
manual and install package
\item
first version published on CTAN
\end{itemize}

%%%%%%%%%%%%%%%%%%%%%%%%%%%%%%%%%%%%%%%%
\paragraph{v0.6:} 2017/04/26

\begin{itemize}
\item
redirection mechanism added
\end{itemize}

%%%%%%%%%%%%%%%%%%%%%%%%%%%%%%%%%%%%%%%%
\paragraph{v0.5:} 2017/04/26

\begin{itemize}
\item
functionality in definition file
\end{itemize}


%%%%%%%%%%%%%%%%%%%%%%%%%%%%%%%%%%%%%%%%%%%%%%%%%%%%%%%%%%%%%%%%%%%%%%%%%%%%%%%%
%%%%%%%%%%%%%%%%%%%%%%%%%%%%%%%%%%%%%%%%%%%%%%%%%%%%%%%%%%%%%%%%%%%%%%%%%%%%%%%%
%%%%%%%%%%%%%%%%%%%%%%%%%%%%%%%%%%%%%%%%%%%%%%%%%%%%%%%%%%%%%%%%%%%%%%%%%%%%%%%%
\appendix

\settowidth\MacroIndent{\rmfamily\scriptsize 000\ }

 \DocInput{childdoc.dtx}

\end{document}
%</driver>
% \fi
%
% %%%%%%%%%%%%%%%%%%%%%%%%%%%%%%%%%%%%%%%%%%%%%%%%%%%%%%%%%%%%%%%%%%%%%%%%%%%%%%
% %%%%%%%%%%%%%%%%%%%%%%%%%%%%%%%%%%%%%%%%%%%%%%%%%%%%%%%%%%%%%%%%%%%%%%%%%%%%%%
% \section{Sample}
%\iffalse
%<*samplemain>
%\fi
%
% The following presents a sample document
% with two chapters, two parts, a title page,
% a compile flag as well as three forwarding files to set the flag.
% It consists of eight |.tex| files:
% \begin{center}
% \begin{tabular}{ll}
% |cdocsamp.tex|&main file\\
% |cdocsch1.tex|&include file for chapter 1\\
% |cdocsch2.tex|&include file for chapter 2\\
% |cdocspt3.tex|&include file for part 3\\
% |cdocspt4.tex|&include file for part 4\\
% |cdocsdrf.tex|&forwarding file for main file in draft mode\\
% |cdocsfi1.tex|&forwarding file for final version of chapter 1\\
% |cdocsfi2.tex|&forwarding file for final version of chapter 2\\
% \end{tabular}
% \end{center}
% Each of the eight files can be compiled directly by the \LaTeX{} compiler.
%
% %%%%%%%%%%%%%%%%%%%%%%%%%%%%%%%%%%%%%%
% \paragraph{Main File.}
%
% The main file is called |cdocsamp.tex|.
%
% Load the \textsf{childdoc} definitions and
% declare the filename for the main document:
%    \begin{macrocode}
\input{childdoc.def}
\childdocmain{}
%    \end{macrocode}

% Optional override for |\version| flag:
%    \begin{macrocode}
%%\ifchilddoc\else\providecommand{\version}{draft}\fi
%    \end{macrocode}

% Define the default values for the |\version| flag
% (|final| for the main file and |draft| for childs):
%    \begin{macrocode}
\ifchilddoc
\providecommand{\version}{draft}
\else
\providecommand{\version}{final}
\fi
%    \end{macrocode}

% Load the standard document class:
%    \begin{macrocode}
\documentclass[12pt]{article}
%    \end{macrocode}

% Start the document body:
%    \begin{macrocode}
\begin{document}
%    \end{macrocode}

% Declare a title page.
% Print title, part of document being processed and version flag:
%    \begin{macrocode}
\addtocounter{page}{-1}
\begin{center}
{\LARGE\bfseries{}childdoc example\par}
\vspace{1cm}
\ifchilddoc
\ifchilddocmanual part\else chapter\fi:
`\childdocname' of `\childdocjob'\par
\else
main document: `\childdocjob'\par
\fi
version: \version\par
\end{center}
\newpage
%    \end{macrocode}

% Manually include selected file,
% otherwise process as usual:
%    \begin{macrocode}
\ifchilddocmanual
\section*{part `\childdocname'}
\input{\childdocname}
\else
%    \end{macrocode}

% Include the two chapters:
%    \begin{macrocode}
\include{cdocsch1}
\include{cdocsch2}
%    \end{macrocode}

% Include the two parts unless only chapters should be displayed:
%    \begin{macrocode}
\ifchilddoc\else
\section{part three}
\input{cdocspt3}
\section{part four}
\input{cdocspt4}
\fi
%    \end{macrocode}

% Process as usual until here:
%    \begin{macrocode}
\fi
%    \end{macrocode}

% End of document body:
%    \begin{macrocode}
\end{document}
%    \end{macrocode}
%\iffalse
%</samplemain>
%\fi
%
% %%%%%%%%%%%%%%%%%%%%%%%%%%%%%%%%%%%%%%
% \paragraph{Chapter Include Files.}
%
% The include files are called |cdocsch1.tex| and |cdocsch2.tex|.
%
%\iffalse
%<*samplechap1|samplechap2>
%\fi

% Optional override for |\version| flag:
%    \begin{macrocode}
%%\providecommand{\version}{final}
%    \end{macrocode}

% Include the main document:
%    \begin{macrocode}
\input{childdoc.def}
\childdocof{cdocsamp}
%    \end{macrocode}

%\iffalse
%</samplechap1|samplechap2>
%\fi
%
%\iffalse
%<*samplechap1>
%\fi
% Some text for chapter 1:
%    \begin{macrocode}
\section{one}
some text in chapter one
%    \end{macrocode}

%\iffalse
%</samplechap1>
%\fi
% Some text for chapter 2:
%\iffalse
%<*samplechap2>
%\fi
%    \begin{macrocode}
\section{two}
more text in chapter two
%    \end{macrocode}

%\iffalse
%</samplechap2>
%\fi
%
% %%%%%%%%%%%%%%%%%%%%%%%%%%%%%%%%%%%%%%
% \paragraph{Part Include Files.}
%
% The include files are called |cdocspt3.tex| and |cdocspt4.tex|.
%
%\iffalse
%<*samplepart3|samplepart4>
%\fi

% Optional override for |\version| flag:
%    \begin{macrocode}
%%\providecommand{\version}{final}
%    \end{macrocode}

% Include the main document:
%    \begin{macrocode}
\input{childdoc.def}
\childdocby{cdocsamp}
%    \end{macrocode}

%\iffalse
%</samplepart3|samplepart4>
%\fi
%
%\iffalse
%<*samplepart3>
%\fi
% Some text for part 3:
%    \begin{macrocode}
some text in part three
%    \end{macrocode}

%\iffalse
%</samplepart3>
%\fi
% Some text for part 4:
%\iffalse
%<*samplepart4>
%\fi
%    \begin{macrocode}
more text in part four
%    \end{macrocode}

%\iffalse
%</samplepart4>
%\fi
%
% %%%%%%%%%%%%%%%%%%%%%%%%%%%%%%%%%%%%%%
% \paragraph{Forwarding for a Complete Draft.}
%
% The following forwarding file |cdocsdrf.tex|
% compiles the main document in draft mode:
%\iffalse
%<*sampledraft>
%\fi
%    \begin{macrocode}
\def\version{draft}
\input{childdoc.def}
\childdocforward{cdocsamp}
%    \end{macrocode}

%\iffalse
%</sampledraft>
%\fi
%
% %%%%%%%%%%%%%%%%%%%%%%%%%%%%%%%%%%%%%%
% \paragraph{Forwarding for Final Version of the Chapters.}
%
% The following forwarding files |cdocsfn1.tex| and |cdocsfn2.tex|
% (with identical content)
% compile the final versions of the child documents
% |cdocsch1.tex| and |cdocsch2.tex|, respectively:
%\iffalse
%<*samplefinal>
%\fi
%    \begin{macrocode}
\def\version{final}
\input{childdoc.def}
\childdocforwardprefix[cdocsamp]{cdocsfn}{cdocsch}
%    \end{macrocode}

%\iffalse
%</samplefinal>
%\fi
%
% %%%%%%%%%%%%%%%%%%%%%%%%%%%%%%%%%%%%%%
% \paragraph{Command Line Processing.}
%
% The following three command lines generate the output files
% |cdocscld|, |cdocscl1| and |cdocscl2|
% which should be identical to
% |cdocsdrf|, |cdocsch1| and |cdocsfn2|, respectively:
% \begin{center}
% \begin{tabular}{l}
% |latex -jobname cdocscld \|\\
% |  "\def\version{draft}\input{childdoc.def}\childdocforward{cdocsamp}"|\\
% |latex -jobname cdocscl1 \|\\
% |  "\input{childdoc.def}\childdocforward[cdocsamp]{cdocsch1}"|\\
% |latex -jobname cdocscl2 \|\\
% |  "\def\version{final}\input{childdoc.def}\childdocforward{cdocsch2}"|
% \end{tabular}
% \end{center}
% Note that the trailing backslash on each first line
% merely continues the input to the second line
% (for convenient cut ant paste).
% Furthermore, the command |latex| can be replaced by any
% of its alternative versions such as |pdflatex|.
%
% %%%%%%%%%%%%%%%%%%%%%%%%%%%%%%%%%%%%%%%%%%%%%%%%%%%%%%%%%%%%%%%%%%%%%%%%%%%%%%
% %%%%%%%%%%%%%%%%%%%%%%%%%%%%%%%%%%%%%%%%%%%%%%%%%%%%%%%%%%%%%%%%%%%%%%%%%%%%%%
% \section{Implementation}
%\iffalse
%<*package>
%\fi
%
% This section describes the definitions file |childdoc.def|.

% The definitions cannot be loaded using |\usepackage| or |\RequirePackage|
% which has a mechanism to prevent loading a style file more than once.
% When loading the definitions by means of |\input|
% multiple instances have to be prevented manually:
%\iffalse
%This code needs to be before the `\ProvidesFile' directive
%which is defined at the beginning of this file.
%Therefore it is also placed there and commented out here.
%</package>
%<*discard>
%\fi
%    \begin{macrocode}
\ifdefined\childdocmain\endinput\fi
%    \end{macrocode}
%\iffalse
%</discard>
%<*package>
%\fi
%
% \macro{\ifchilddoc}
% \macro{\ifchilddocmanual}
% The conditional |\ifchilddoc| tells whether a
% child (true) or main (false) document is being compiled.
% The conditional |\ifchilddocmanual| tells whether
% the |\includeonly| mechanism is used (false) or
% the selection of child files must be performed manually (true).
% The definitions initialise to false:
%    \begin{macrocode}
\newif\ifchilddoc
\newif\ifchilddocmanual
%    \end{macrocode}

% \macro{\childdocname}
% \macro{\childdocjob}
% The macro |\childdocname| stores the name of the main document
% to be compiled. The macro |\childdocjob| stores the name of
% the document on which the \LaTeX{} compiler was originally invoked.
% The content of |\jobname| cannot be compared
% to filenames specified in the source due to different catcodes.
% The following code rescans |\jobname|, stores the result
% in |\childdocname| and saves a copy in |\childdocjob|:
%    \begin{macrocode}
\edef\childdocname{\scantokens\expandafter{\jobname\noexpand}}
\let\childdocjob\childdocname
%    \end{macrocode}

% \macro{\childdocdisable}
% The macro |\childdocdisable| prevents the main file
% from being processed more than once.
% At this stage, the main document command |\childdocmain|
% is assumed to be called once again where it should do nothing.
% Any subsequent call to it should prevent
% a secondary processing of the main document
% It overwrites the forwarding commands
% |\childdocof| and |\childdocforward|
% with empty macros to prevent further inclusions of the main document:
%    \begin{macrocode}
\newcommand{\childdocdisable}
{
  \renewcommand{\childdocmain}[1]{\renewcommand{\childdocmain}[1]{\endinput}}
  \renewcommand{\childdocof}[1]{}
  \renewcommand{\childdocby}[2][]{}
  \renewcommand{\childdocforward}[2][]{}
  \renewcommand{\childdocdisable}{}
}
%    \end{macrocode}

% \macro{\childdocmain}
% The macro |\childdocmain| is to be called at the top of the main file
% with nothing or the main filename (without extension) as argument.
% First, it breaks loops.
% If the argument is not empty and does not match |\childdocname|
% (which is set by the first inclusion of |childdoc.def|),
% |\ifchilddoc| is set to true, |\includeonly| is applied to the child file
% and |\jobname| is set to the main file
% (for proper handling of |.aux| files):
%    \begin{macrocode}
\newcommand{\childdocmain}[1]
{
  \childdocdisable\childdocmain{}
  \if?#1?\else
    \begingroup
      \def\childdoctmp{#1}
      \ifx\childdoctmp\childdocname
        \def\childdoctmp{}
      \else
        \def\childdoctmp
        {
          \childdoctrue
          \includeonly{\childdocname}
          \def\childdocjob{#1}
          \def\jobname{#1}
        }
      \fi
      \expandafter
    \endgroup
    \childdoctmp
  \fi
}
%    \end{macrocode}

% \macro{\childdocof}
% The command |\childdocof| redirects
% compilation to the main file |#1|.
%    \begin{macrocode}
\newcommand{\childdocof}[1]
{
  \childdocdisable
  \childdoctrue
  \includeonly{\childdocname}
  \def\jobname{#1}
  \def\childdocjob{#1}
  \input{#1}
}
%    \end{macrocode}

% \macro{\childdocby}
% The command |\childdocby| ....
%    \begin{macrocode}
\newcommand{\childdocby}[2][]
{
  \childdocdisable
  \childdoctrue
  \childdocmanualtrue
  \if?#1?\else
    \def\jobname{#2}
  \fi
  \def\childdocjob{#2}
  \input{#2}
  \endinput
}
%    \end{macrocode}

% \macro{\childdocforward}
% The command |\childdocforward| redirects
% compilation to the main file or
% (if the optional argument is given) a child file.
% Parameters are set as if the main file
% or a child file starting with |\childdocof| was compiled.
% Then compilation is handed over to the main file:
%    \begin{macrocode}
\newcommand{\childdocforward}[2][]
{
  \begingroup
    \if?#1?
      \def\childdoctmp
      {
        \def\childdocname{#2}
        \def\childdocjob{#2}
        \def\jobname{#2}
        \input{#2}
        \endinput
      }
    \else
      \def\childdoctmp
      {
        \childdocdisable
        \def\childdocname{#2}
        \childdoctrue
        \includeonly{#2}
        \def\childdocjob{#1}
        \def\jobname{#1}
        \input{#1}
        \endinput
      }
    \fi
    \expandafter
  \endgroup
  \childdoctmp
}
%    \end{macrocode}

% \macro{\childdocforwardprefix}
% The command |\childdocforwardprefix| redirects
% compilation to the main or a child file by means of a pattern.
% The prefix |#1| in the current filename is replaced by |#2|
% and the suffix of the current filename is kept
% (it is assumed that the filename does not contain the substring `|~~~|'
% which is used as a delimiter).
% Compilation is handed over to the new file by |\childdocforward|:
%    \begin{macrocode}
\newcommand{\childdocforwardprefix}[3][]
{
  \begingroup
    \def\childdocextract #2##1~~~{\def\childdoctmp{\childdocforward[#1]{#3##1}}}
    \expandafter\childdocextract\childdocname~~~
    \expandafter
  \endgroup
  \childdoctmp
}
%    \end{macrocode}

% \macro{\childdoc}
% The deprecated macro |\childdoc| is a legacy version of |\childdocmain|:
%    \begin{macrocode}
\newcommand{\childdoc}{\childdocmain}
%    \end{macrocode}

% \macro{\childdocredirect}
% The deprecated macro |\childdocredirect| is a legacy version
% of |\childdocforward| and |\childdocforwardprefix|:
%    \begin{macrocode}
\newcommand{\childdocredirect}[2][]
{
  \begingroup
    \if?#1?
      \def\childdoctmp{\childdocforward{#2}}
    \else
      \def\childdoctmp{\childdocforwardprefix{#1}{#2}}
    \fi
    \expandafter
  \endgroup
  \childdoctmp
}
%    \end{macrocode}

%\iffalse
%</package>
%\fi
%
\endinput

\childdocforwardprefix[cdocsamp]{cdocsfn}{cdocsch}
%    \end{macrocode}

%\iffalse
%</samplefinal>
%\fi
%
% %%%%%%%%%%%%%%%%%%%%%%%%%%%%%%%%%%%%%%
% \paragraph{Command Line Processing.}
%
% The following three command lines generate the output files
% |cdocscld|, |cdocscl1| and |cdocscl2|
% which should be identical to
% |cdocsdrf|, |cdocsch1| and |cdocsfn2|, respectively:
% \begin{center}
% \begin{tabular}{l}
% |latex -jobname cdocscld \|\\
% |  "\def\version{draft}% \iffalse
%
% childdoc.dtx Copyright (C) 2017-2018 Niklas Beisert
%
% This work may be distributed and/or modified under the
% conditions of the LaTeX Project Public License, either version 1.3
% of this license or (at your option) any later version.
% The latest version of this license is in
%   http://www.latex-project.org/lppl.txt
% and version 1.3 or later is part of all distributions of LaTeX
% version 2005/12/01 or later.
%
% This work has the LPPL maintenance status `maintained'.
%
% The Current Maintainer of this work is Niklas Beisert.
%
% This work consists of the files childdoc.dtx and childdoc.ins
% and the derived files childdoc.def and cdocsamp.tex with
% cdocsch1.tex, cdocsch2.tex, cdocsdrf.tex, cdocsfn1.tex, cdocsfn2.tex.
%
%<package>\ifdefined\childdocmain\endinput\fi
%<package>\ProvidesFile{childdoc.def}[2018/12/30 v2.0 child document driver]
%<samplemain>\ProvidesFile{cdocsamp.tex}[2018/12/30 v2.0 sample for childdoc]
%<*driver>
%\ProvidesFile{childdoc.drv}[2018/12/30 v2.0 childdoc reference manual file]
\PassOptionsToClass{10pt,a4paper}{article}
\documentclass{ltxdoc}

\usepackage[margin=35mm]{geometry}
\usepackage{hyperref}
\usepackage{hyperxmp}
\usepackage[usenames]{color}

\hypersetup{colorlinks=true}
\hypersetup{pdfstartview=FitH}
\hypersetup{pdfpagemode=UseNone}
\hypersetup{pdfsource={}}
\hypersetup{pdflang={en-UK}}
\hypersetup{pdfcopyright={Copyright 2017-2018 Niklas Beisert.
  This work may be distributed and/or modified under the
  conditions of the LaTeX Project Public License, either version 1.3
  of this license or (at your option) any later version.}}
\hypersetup{pdflicenseurl={http://www.latex-project.org/lppl.txt}}
\hypersetup{pdfcontactaddress={ETH Zurich, ITP, HIT K,
  Wolfgang-Pauli-Strasse 27}}
\hypersetup{pdfcontactpostcode={8093}}
\hypersetup{pdfcontactcity={Zurich}}
\hypersetup{pdfcontactcountry={Switzerland}}
\hypersetup{pdfcontactemail={nbeisert@itp.phys.ethz.ch}}
\hypersetup{pdfcontacturl={http://people.phys.ethz.ch/\xmptilde nbeisert/}}

\newcommand{\secref}[1]{\hyperref[#1]{section \ref*{#1}}}

\parskip1ex
\parindent0pt
\let\olditemize\itemize
\def\itemize{\olditemize\parskip0pt}

\begin{document}

\title{The \textsf{childdoc} Package}
\hypersetup{pdftitle={The childdoc Package}}
\author{Niklas Beisert\\[2ex]
  Institut f\"ur Theoretische Physik\\
  Eidgen\"ossische Technische Hochschule Z\"urich\\
  Wolfgang-Pauli-Strasse 27, 8093 Z\"urich, Switzerland\\[1ex]
  \href{mailto:nbeisert@itp.phys.ethz.ch}
  {\texttt{nbeisert@itp.phys.ethz.ch}}}
\hypersetup{pdfauthor={Niklas Beisert}}
\hypersetup{pdfsubject={Manual for the LaTeX2e Package childdoc}}
\date{30 December 2018, \textsf{v2.0}}
\maketitle

\begin{abstract}\noindent
\textsf{childdoc} is a \LaTeXe{} package
that enables the direct compilation
of document sections included by |\include|
to individual files.
\end{abstract}

\begingroup
\parskip0ex
\tableofcontents
\endgroup

%%%%%%%%%%%%%%%%%%%%%%%%%%%%%%%%%%%%%%%%%%%%%%%%%%%%%%%%%%%%%%%%%%%%%%%%%%%%%%%%
%%%%%%%%%%%%%%%%%%%%%%%%%%%%%%%%%%%%%%%%%%%%%%%%%%%%%%%%%%%%%%%%%%%%%%%%%%%%%%%%
\section{Introduction}

\LaTeX{} provides a mechanism to structure a large document (such as a book)
into a main file and several child files (containing the chapters)
using the |\include| command.
This mechanism is beneficial for documents
which span hundreds of pages in order to
make the source file(s) more manageable.
Moreover, compilation can be restricted to
selected child files by means of the |\includeonly| command.
The latter feature can be used to reduce the compilation time while editing
(this was significantly more useful in the earlier days of \LaTeX{})
or to generate a smaller document which is easier to navigate.
Another application of |\includeonly| is to generate
documents consisting of selected parts of the complete document.

However, there are a few drawbacks of the plain |\include| mechanism:
\begin{itemize}
\item
The child files cannot be compiled on their own,
they can only be compiled via the main file.
A naive editing environment
(such as a text editor with an option
to have the current file processed by \LaTeX)
may require one to switch to the main file before compiling;
attempting to compile the child file produces errors.
\item
The main file must be modified (each time)
to adjust the |\includeonly| command
to the present needs. This easily leaves the main file in a messy state.
\item
The generated document will always carry the filename
of the main document. This is inconvenient if
several child files are to be compiled and
to be kept for distribution.
\end{itemize}

The present package provides a simple interface
to make child files individually compilable by \LaTeX{}.
Compiling a child file then has the same effect as compiling
the main file with an |\includeonly| command
to select the appropriate child.
Moreover the generated document will carry the name of the child
rather than the main file.
This resolves all three above issues.

This feature is meant to make the editing of books,
thesis documents and lecture notes somewhat more convenient.
However, the package can also be used efficiently for
composing a series of documents (such as exercise sheets)
which are typically distributed individually.
It then assists the author in generating the individual documents
(potentially in different versions)
as well as a document containing the collected series.
Another application is in developing style files
or other kinds of included material
where compilation of the style file could redirect
to a sample or test file.

%%%%%%%%%%%%%%%%%%%%%%%%%%%%%%%%%%%%%%%%%%%%%%%%%%%%%%%%%%%%%%%%%%%%%%%%%%%%%%%%
%%%%%%%%%%%%%%%%%%%%%%%%%%%%%%%%%%%%%%%%%%%%%%%%%%%%%%%%%%%%%%%%%%%%%%%%%%%%%%%%
\section{Usage}

First of all, the package \textsf{childdoc} is \emph{not} a standard
\LaTeXe{} |.sty| style file! Therefore it needs to be invoked in
a non-standard way.

%%%%%%%%%%%%%%%%%%%%%%%%%%%%%%%%%%%%%%%%%%%%%%%%%%%%%%%%%%%%%%%%%%%%%%%%%%%%%%%%
\subsection{Included Files}
\label{sec:include}

%%%%%%%%%%%%%%%%%%%%%%%%%%%%%%%%%%%%%%%%
\DescribeMacro{\childdocmain}
To use the package, add the commands
\begin{center}
\begin{tabular}{l}
|\input{childdoc.def}|\\
|\childdocmain{}|\\
\end{tabular}
\end{center}
at the very top of the main \LaTeX{} file,
in particular \emph{before} the |\documentclass| statement!
The argument of |\childdocmain| should be left empty
(but it must be present).

%%%%%%%%%%%%%%%%%%%%%%%%%%%%%%%%%%%%%%%%
\DescribeMacro{\childdocof}
Furthermore, add the commands
\begin{center}
\begin{tabular}{l}
|\input{childdoc.def}|\\
|\childdocof{|\textit{main}|}|\\
\end{tabular}
\end{center}
at the top of every child file \textit{child}
which is included by |\include{|\textit{child}|}|
from within the main file
(or at least for those files to be compiled individually).
The argument \textit{main} must be the filename of the main file.

There are a couple of
considerations in setting up the main and child documents:

%%%%%%%%%%%%%%%%%%%%%%%%%%%%%%%%%%%%%%%%
\paragraph{Restrictions.}

Please note the following restrictions:
\begin{itemize}
\item
|\childdocmain| must be called with one argument \textit{main}
to ensure compatibility with earlier version of the package.
It must either be empty (|\childdocmain{}|)
or precisely match the filename of the main file in which it is specified.
See \secref{sec:detection} for further information.
\item
The filename \textit{main} must be specified without the |.tex| extension.
\item
The filename \textit{main} is case sensitive
(even in case-insensitive file systems)
due to internal string comparison.
\item
The argument \textit{main} should be fully expanded, it cannot be a macro.
\item
Subdirectories and special characters should be avoided in filenames.
\item
The command |\childdocmain{|\textit{main}|}| must be followed by a whitespace.
It should not be followed immediately by another command
or by a comment mark `|%|'.
This is because the \TeX{} parser reads the token immediately following
the argument of |\childdocmain| and puts it
at the beginning of every child section;
however, a white\-space is ignored.
\end{itemize}

%%%%%%%%%%%%%%%%%%%%%%%%%%%%%%%%%%%%%%%%
\paragraph{Content of Main File.}

It is advisable to place all content in the child files included by |\include|.
Any output contained in the main file will appear in all child documents
unless suppressed manually;
it cannot be suppressed automatically by the |\includeonly| directive
and thus should normally be avoided.
A method to include some content in the main file
by means of conditional processing is described in \secref{sec:conditional}.

%%%%%%%%%%%%%%%%%%%%%%%%%%%%%%%%%%%%%%%%
\paragraph{Page Numbering.}

When only a part of the document is compiled,
the appropriate numbering of pages
(as well as other status parameters)
is determined from the |.aux| files.
The latter contain information from previous passes.
However this information needs to propagate through
all intermediate child documents.
Therefore the page numbering in child documents may well
be inconsistent until the complete document is compiled at least once.

A useful (if unconventional) way to always ensure a consistent
page numbering is to restart the numbering in each child document
and denote the pages by `\textit{child}|.|\textit{page}'
where \textit{child} represents the chapter/section number of the child file.
This can be achieved by the command
|\numberwithin{page}{|\textit{child}|}|
of the \textsf{amsmath} package
where \textit{child} can be |chapter| or |section|
depending on the chosen structuring.
Alternatively, one can modify the macro |\thepage| appropriately
and reset the counter |page| at the start of each child file.

%%%%%%%%%%%%%%%%%%%%%%%%%%%%%%%%%%%%%%%%%%%%%%%%%%%%%%%%%%%%%%%%%%%%%%%%%%%%%%%%
\subsection{Conditional Processing}
\label{sec:conditional}

The package provides a mechanism to compile different versions
of a document. To customise the versions further some conditional processing
can come in handy to distinguish which version is being compiled.
The package provides two macros to describe the compilation context:

%%%%%%%%%%%%%%%%%%%%%%%%%%%%%%%%%%%%%%%%
\DescribeMacro{\ifchilddoc}
The conditional |\ifchilddoc| distinguishes between the compilation of
child documents and the main document:
%
\begin{center}
|\ifchilddoc |\textit{child-code}| |[|\||else |\textit{main-code}]| \||fi|
\end{center}

%%%%%%%%%%%%%%%%%%%%%%%%%%%%%%%%%%%%%%%%
\DescribeMacro{\childdocname}
\DescribeMacro{\childdocjob}
The macro |\childdocname| contains the filename (without extension)
of the main or child file being processed.
Note that |\childdocjob| will always contain the name of the main file.

%%%%%%%%%%%%%%%%%%%%%%%%%%%%%%%%%%%%%%%%
\paragraph{Title Page.}

Conditional processing can be used to include a title or banner page
in the main document when proper precautions are taken.
Importantly, the code in the main file should ensure that the page counter
(as well as other status parameters which are stored in the |.aux| files)
takes the same value after the conditional processing.
Otherwise the page numbers may take divergent values
depending on which part is compiled.

For example, a title page could be declared by:
%
\begin{center}
\begin{tabular}{l}
|\ifchilddoc\||else|\\
|\addtocounter{page}{-1}|\\
\textit{code for title page}\\
|\newpage|\\
|\||fi|
\end{tabular}
\end{center}
%
A banner page for the child documents can be generated by:
%
\begin{center}
\begin{tabular}{l}
|\ifchilddoc|\\
|\addtocounter{page}{-1}|\\
\textit{code for banner page}\\
|\newpage|\\
|\||fi|
\end{tabular}
\end{center}
%
Here one could write a message such as:
\begin{center}
|This is the part \childdocname{} of \childdocjob{}.|
\end{center}

%%%%%%%%%%%%%%%%%%%%%%%%%%%%%%%%%%%%%%%%%%%%%%%%%%%%%%%%%%%%%%%%%%%%%%%%%%%%%%%%
\subsection{Flags}
\label{sec:flags}

The package makes it easy to generate different versions
of the main or child documents.
To this end compilation flags can be defined
and assigned different default values.
They will be particularly useful in conjunction
with the forwarding mechanism described in \secref{sec:forward}.

For example, it may be useful to have a flag |\version|
which can be set to |draft| or |final|.
The document source will contain some conditional code
depending on the value of |\version|.
Suppose further, the flag should default to |final| for the main file
and to |draft| for child files
which is a natural assignment for editing the document.
This is achieved by placing the following code
in the preamble of the main document
(below the |\childdocmain| directive):
%
\begin{center}
\begin{tabular}{l}
|\ifchilddoc|\\
|\providecommand{\version}{draft}|\\
|\||else|\\
|\providecommand{\version}{final}|\\
|\||fi|
\end{tabular}
\end{center}
%
The definition by |\providecommand| makes sure
that previous definitions are not overwritten.
Further statements |\providecommand{\version}{...}|
can thus be added before the above code to override it.

For the main file, one might add a line
(between |\childdocmain| and the above block)
%
\begin{center}
|%\ifchilddoc\||else\providecommand{\version}{draft}\||fi|
\end{center}
%
which can be uncommented to produce a draft version.
Likewise one can add a line to the very top of a child file
(above the |\childdocof{|\textit{main}|}| directive)
%
\begin{center}
|%\providecommand{\version}{final}|
\end{center}
%
which can be uncommented to produce the final version of this child document.

%%%%%%%%%%%%%%%%%%%%%%%%%%%%%%%%%%%%%%%%%%%%%%%%%%%%%%%%%%%%%%%%%%%%%%%%%%%%%%%%
\subsection{Forwarding}
\label{sec:forward}

Different versions of the main or child documents
using compilation flags as described in \secref{sec:flags}
can be (permanently) stored in different files
for convenient compilation, viewing and distribution.
To this end, the package defines a command
to pass on compilation to a different file:

%%%%%%%%%%%%%%%%%%%%%%%%%%%%%%%%%%%%%%%%
\DescribeMacro{\childdocforward}
The command |\childdocforward| redirects processing to
another source file:
%
\begin{center}
\begin{tabular}{l}
|\input{childdoc.def}|\\
|\childdocforward[|\textit{main}|]{|\textit{dest}|}|\\
\end{tabular}
\end{center}
%
The argument \textit{dest} is the destination file
(without extension).
It should be the main file or one of the child files.
Note that further \textsf{childdoc} directives
such as |\childdocof| and |\childdocforward|
in the indicated file will be processed in this form.
The optional argument \textit{main}
passes on directly to the main file \textit{main}
while pretending to compile the child \textit{dest}.
This form behaves as if \textit{dest}
issues |\childdocof{|\textit{main}|}| right away,
and no further \textsf{childdoc} directives will be processed.

%%%%%%%%%%%%%%%%%%%%%%%%%%%%%%%%%%%%%%%%
\DescribeMacro{\...prefix}
In the alternative form |\childdocforwardprefix|,
%
\begin{center}
\begin{tabular}{l}
|\input{childdoc.def}|\\
|\childdocforwardprefix[|\textit{main}|]{|\textit{prefix}|}{|\textit{dest}|}|
\end{tabular}
\end{center}
%
the destination file is determined by a pattern
depending on the current file:
To make this work, the current file must be called
`{\textit{prefix}\hspace{0.2em}\textit{suffix}}'
with \textit{prefix} matching precisely the argument.
Processing is then passed on to the file
`{\textit{dest}\hspace{0.2em}\textit{suffix}}'.
Surely, the same effect is achieved by
directly specifying the
argument `{\textit{dest}\hspace{0.2em}\textit{suffix}}'
in the first form.
However, that requires to set up a different file
for each child. With the alternative form of the command
all these files can have exactly the same content
which simplifies setting them up and maintaining them.

For example, the following file |draft.tex|
with a compilation flag |\version| as described in \secref{sec:flags}
compiles the main document as a draft:
%
\begin{center}
\begin{tabular}{l}
|\def\version{draft}|\\
|\input{childdoc.def}|\\
|\childdocforward{|\textit{main}|}|
\end{tabular}
\end{center}
%
Likewise, the following files |final|\textit{nn}|.tex|
compile the final version of the child document
|child|\textit{nn}|.tex|:
%
\begin{center}
\begin{tabular}{l}
|\def\version{final}|\\
|\input{childdoc.def}|\\
|\childdocforwardprefix{final}{child}|
\end{tabular}
\end{center}
%

Note that when several versions of a main file and/or of each child file
are to be generated, it may be convenient to set up a |Makefile| or
shell script to automatise the process.

%%%%%%%%%%%%%%%%%%%%%%%%%%%%%%%%%%%%%%%%%%%%%%%%%%%%%%%%%%%%%%%%%%%%%%%%%%%%%%%%
\subsection{Command Line Processing}
\label{sec:commandline}

The effect of redirection files can also be achieved by invoking
the \LaTeX{} compiler with a more elaborate command line.
Most conveniently this should be done as part
of a shell script or a |Makefile|.

When using \textsf{childdoc} in the main file, the following
command lines effectively perform a redirection
(note that depending on the shell being used,
backslashes may have to be doubled: `|\|' $\to$ `|\\|'):
%
\begin{center}
|... -jobname "|\textit{target}|" |\\|"|[\textit{flags}]%
|\input{childdoc.def}\childdocforward[|\textit{main}|]{|\textit{dest}|}"|
\end{center}
%
Here \textit{target} is the name of the output file,
\textit{main} is the name of the main file
and \textit{dest} is the name of the main or child file to be processed
(all filenames without extensions).
The optional argument \textit{main} can be omitted
if \textit{main} matches \textit{dest}.
Optionally, compilation \textit{flags} can be defined via |\def| commands.
This command line makes the \TeX{} engine believe
it is compiling the file \textit{target}
whose content is specified as the latter parameter.
The provided code then forwards the processing to
\textit{main} or \textit{dest} as described in \secref{sec:forward}.

%%%%%%%%%%%%%%%%%%%%%%%%%%%%%%%%%%%%%%%%%%%%%%%%%%%%%%%%%%%%%%%%%%%%%%%%%%%%%%%%
\subsection{Include by Input}
\label{sec:input}

Including child documents by |\include| has some restrictions by design.
Most notably, the content of a child document always occupies
its own set of pages; pages cannot be shared between child documents.
Usually, this behaviour makes perfect sense
because each child document contain an essential part of the document.
However, in some situations it may be desirable to compose
a document from a collection of parts
without having mandatory page breaks between then.
For this case, the package
provides a mechanism to include parts
by |\input| which can also be processed individually.
However, by construction this mechanism
requires manual handling of the content to be output.

%%%%%%%%%%%%%%%%%%%%%%%%%%%%%%%%%%%%%%%%
\DescribeMacro{\ifchilddocmanual}
The main file should be prepared as usual, see \secref{sec:include}.
However, the document body must make a distinction
between processing of an individual part and of the main document, e.g.:
%
\begin{center}
\begin{tabular}{l}
|\ifchilddocmanual|\\
|\input{\childdocname}|\\
|\||else|\\
\textit{document body with }|\input{|\textit{part}|}|\\
|\||fi|
\end{tabular}
\end{center}
%
The conditional |\ifchilddocmanual| is true whenever
a part to be included by |\input| is being compiled,
and the name of the part is stored in |\childdocname|.

%%%%%%%%%%%%%%%%%%%%%%%%%%%%%%%%%%%%%%%%
\DescribeMacro{\childdocby}
Each part to be included by |\input| should start with:
%
\begin{center}
\begin{tabular}{l}
|\input{childdoc.def}|\\
|\childdocby{|\textit{main}|}|\\
\end{tabular}
\end{center}
%
The directive |\childdocby| is similar to |\childdocof|
described in \secref{sec:include},
but the subsequent selection of content must be done manually.
To that end, both |\ifchilddoc| and |\ifchilddocmanual|
will be true upon processing of a part,
and the name of the part is stored in |\childdocname|.
Note that |\jobname| will be set to the filename of the current part
so that each part receives an individual |.aux| file
that does not interfere with the |.aux| file(s) of the main document.
This behaviour can be altered by the alternative form
|\childdocby[*]{|\textit{main}|}| (with a non-empty optional argument)
which uses the |.aux| file of the main document
by setting |\jobname| to \textit{main}.

%%%%%%%%%%%%%%%%%%%%%%%%%%%%%%%%%%%%%%%%%%%%%%%%%%%%%%%%%%%%%%%%%%%%%%%%%%%%%%%%
\subsection{Driver Development}
\label{sec:driver}

The \textsf{childdoc} mechanism can also be use for the development
of definition files such as \LaTeX{} styles or classes.
This case differs from the above setup with multiple parts
included by |\include| in that no |\includeonly| should be invoked.
This can be achieved by starting the include file
(before |\ProvidesPackage|) with:
%
\begin{center}
\begin{tabular}{l}
|\input{childdoc.def}|\\
|\childdocforward{|\textit{main}|}|\\
\end{tabular}
\end{center}
%
or alternatively with:
%
\begin{center}
\begin{tabular}{l}
|\input{childdoc.def}|\\
|\childdocby{|\textit{main}|}|\\
\end{tabular}
\end{center}
%
Both forms have slightly different effects as described above.
The main file is prepared as usual, see \secref{sec:include}.

%%%%%%%%%%%%%%%%%%%%%%%%%%%%%%%%%%%%%%%%%%%%%%%%%%%%%%%%%%%%%%%%%%%%%%%%%%%%%%%%
\subsection{Legacy Detection}
\label{sec:detection}

The directive |\childdocmain| in the main file can detect
whether the complete document or merely a child is to be compiled
even without using the directive |\childdocof|.
This method is deprecated because it is less robust
and there is no compelling reason to use it;
it is merely provided for backward compatibility
and it may be removed in future versions.

If the detection mechanism is to be used,
it is mandatory to correctly specify
the filename of the main file as the argument of |\childdocmain|:
%
\begin{center}
\begin{tabular}{l}
|\input{childdoc.def}|\\
|\childdocmain{|\textit{main}|}|\\
\end{tabular}
\end{center}
%
If |\jobname| does not match the argument \textit{main} of |\childdocmain|,
it is assumed that |\jobname| points to the child file to be compiled.
When using |\childdocmain| with the main file specified as argument,
it suffices to start a child file
with just |\input{|\textit{main}|}|
without loading of the package and using |\childdocof|.
If instead all processing is done
with the appropriate \textsf{childdoc} directives,
the argument of \textit{main} of |\childdocmain| can be empty.

An alternative version of the command line processing described
in \secref{sec:commandline} using the detection mechanism reads:
%
\begin{center}
|... -jobname "|\textit{target}|" "|[\textit{flags}]%
[|\def\jobname{|\textit{dest}|}|]|\input{|\textit{main}|}"|
\end{center}

%%%%%%%%%%%%%%%%%%%%%%%%%%%%%%%%%%%%%%%%%%%%%%%%%%%%%%%%%%%%%%%%%%%%%%%%%%%%%%%%
\subsection{Manual Code}
\label{sec:manual}

In case one cannot be certain whether the definitions file |childdoc.def|
is installed on the target \TeX{} distribution
and one prefers not to ship it,
it is conceivable to paste a few relevant commands into the sources.

To that end, drop all statements |\input{childdoc.def}|
and perform the replacements as outlined below.
Instead of |\childdocmain{|\textit{main}|}| add the following code
to the top of the main file:
%
\begin{center}
\begin{tabular}{l}
|\||ifdefined\childdocname\endinput\||fi\newif\ifchilddoc|\\
|\edef\childdocname{\scantokens\expandafter{\jobname\noexpand}}|\\
|\def\childdocmain{|\textit{main}|}\||ifx\childdocmain\childdocname\||else|\\
|\childdoctrue\includeonly{\childdocname}\let\jobname\childdocmain\||fi|\\
\end{tabular}
\end{center}
%
Instead of |\childdocof{|\textit{main}|}| just include the main file
at the top of each child file:
%
\begin{center}
|\input{|\textit{main}|}|
\end{center}
%
A simple redirection |\childdocforward{|\textit{dest}|}| is achieved by:
%
\begin{center}
|\def\jobname{|\textit{dest}|}\input{\jobname}|
\end{center}
%
The redirection with prefix
|\childdocforwardprefix[|\textit{prefix}|]{|\textit{dest}|}|
is accomplished by:
%
\begin{center}
\begin{tabular}{l}
|{\edef\jobname{\scantokens\expandafter{\jobname\noexpand}}|\\
|\def\redirectjob |\textit{prefix}|#1~~~{\gdef\jobname{|\textit{dest}|#1}}|\\
|\expandafter\redirectjob\jobname~~~}\input{\jobname}|
\end{tabular}
\end{center}

In an alternative approach,
child documents can be compiled by a specific command line
without additional code or specific definitions:
%
\begin{center}
|... -jobname "|\textit{target}|" "|[\textit{flags}]%
|\includeonly{|\textit{dest}|}\input{|\textit{main}|}"|
\end{center}
%

%%%%%%%%%%%%%%%%%%%%%%%%%%%%%%%%%%%%%%%%%%%%%%%%%%%%%%%%%%%%%%%%%%%%%%%%%%%%%%%%
%%%%%%%%%%%%%%%%%%%%%%%%%%%%%%%%%%%%%%%%%%%%%%%%%%%%%%%%%%%%%%%%%%%%%%%%%%%%%%%%
\section{Information}

%%%%%%%%%%%%%%%%%%%%%%%%%%%%%%%%%%%%%%%%%%%%%%%%%%%%%%%%%%%%%%%%%%%%%%%%%%%%%%%%
\subsection{Copyright}

Copyright \copyright{} 2017--2018 Niklas Beisert

This work may be distributed and/or modified under the
conditions of the \LaTeX{} Project Public License, either version 1.3
of this license or (at your option) any later version.
The latest version of this license is in
  \url{http://www.latex-project.org/lppl.txt}
and version 1.3 or later is part of all distributions of \LaTeX{}
version 2005/12/01 or later.

This work has the LPPL maintenance status `maintained'.

The Current Maintainer of this work is Niklas Beisert.

This work consists of the files |README.txt|, |childdoc.ins| and |childdoc.dtx|
as well as the derived files |childdoc.def|, |cdocsamp.tex|
with |cdocsch1.tex|, |cdocsch2.tex|, |cdocspt3.tex|, |cdocspt4.tex|,
|cdocsdrf.tex|, |cdocsfn1.tex|, |cdocsfn2.tex|
as well as |childdoc.pdf|.

%%%%%%%%%%%%%%%%%%%%%%%%%%%%%%%%%%%%%%%%%%%%%%%%%%%%%%%%%%%%%%%%%%%%%%%%%%%%%%%%
\subsection{Files and Installation}

The package consists of the files:
%
\begin{center}
\begin{tabular}{ll}
    |README.txt|   & readme file \\
    |childdoc.ins| & installation file \\
    |childdoc.dtx| & source file \\
    |childdoc.def| & definition file \\
    |cdocsamp.tex| & sample main file \\
    |cdocsch1.tex| & sample include file \\
    |cdocsch2.tex| & sample include file \\
    |cdocspt3.tex| & sample part file \\
    |cdocspt4.tex| & sample part file \\
    |cdocsdrf.tex| & sample redirection file \\
    |cdocsfn1.tex| & sample redirection file \\
    |cdocsfn2.tex| & sample redirection file \\
    |childdoc.pdf| & manual
\end{tabular}
\end{center}
%
The distribution consists of the files
|README.txt|, |childdoc.ins| and |childdoc.dtx|.
%
\begin{itemize}
\item
Run (pdf)\LaTeX{} on |childdoc.dtx|
to compile the manual |childdoc.pdf| (this file).
\item
Run \LaTeX{} on |childdoc.ins| to create the definitions file |childdoc.def|
and the sample |cdocsamp.tex| with include files
|cdocsch1.tex|, |cdocsch2.tex|, |cdocspt3.tex|, |cdocspt4.tex|,
|cdocsdrf.tex|, |cdocsfn1.tex|, |cdocsfn2.tex|.
Then copy the file |childdoc.def| to an appropriate directory of your \LaTeX{}
distribution, e.g.\ \textit{texmf-root}|/tex/latex/childdoc|.
\end{itemize}

%%%%%%%%%%%%%%%%%%%%%%%%%%%%%%%%%%%%%%%%%%%%%%%%%%%%%%%%%%%%%%%%%%%%%%%%%%%%%%%%
\subsection{Related CTAN Packages}

There are several other packages which offer a similar functionality:
%
\begin{itemize}
\item
The packages
\href{http://ctan.org/pkg/docmute}{\textsf{docmute}},
\href{http://ctan.org/pkg/includex}{\textsf{includex}} and
\href{http://ctan.org/pkg/standalone}{\textsf{standalone}}
provide commands to include only the document body of
a child file thus allowing both files to be compiled individually.
\item
The packages \href{http://ctan.org/pkg/subdocs}{\textsf{subdocs}}
and \href{http://ctan.org/pkg/subfiles}{\textsf{subfiles}}
provide structures in which the main and child documents can be
encapsulated and allowing them to be compiled individually.
The inclusion mechanism is different from the conventional |\include|.
\item
The package \href{http://ctan.org/pkg/combine}{\textsf{combine}}
is an elaborate solution to combine several documents into one.
\end{itemize}
%
See also the CTAN topic \href{http://ctan.org/topic/subdocs}{\textsf{subdocs}}
for further related packages.
The present package differs from the above solutions in that
a document structure constructed with the conventional |\include| mechanism
just needs two extra commands at the top of every file
such that all constituent files can be compiled individually.

%%%%%%%%%%%%%%%%%%%%%%%%%%%%%%%%%%%%%%%%%%%%%%%%%%%%%%%%%%%%%%%%%%%%%%%%%%%%%%%%
%\subsection{Feature Suggestions}
%
%The following is a list of features which may be useful for future
%versions of this package:
%%
%\begin{itemize}
%\item
%\ldots
%\end{itemize}

%%%%%%%%%%%%%%%%%%%%%%%%%%%%%%%%%%%%%%%%%%%%%%%%%%%%%%%%%%%%%%%%%%%%%%%%%%%%%%%%
\subsection{Revision History}

%%%%%%%%%%%%%%%%%%%%%%%%%%%%%%%%%%%%%%%%
\paragraph{v2.0:} 2018/12/30

\begin{itemize}
\item
immediate forward processing
\item
added |\childdocby| mechanism
\item
manual restructured
\end{itemize}

%%%%%%%%%%%%%%%%%%%%%%%%%%%%%%%%%%%%%%%%
\paragraph{v1.6:} 2018/01/17

\begin{itemize}
\item
application for development of include files
\item
corrections to manual
\end{itemize}

%%%%%%%%%%%%%%%%%%%%%%%%%%%%%%%%%%%%%%%%
\paragraph{v1.5:} 2017/05/21

\begin{itemize}
\item
more complete structuring introduced
\item
|\childdocof| introduced
\item
|\childdoc| renamed to |\childdocmain|
\item
|\childredirect| renamed to |\childdocforward| and |\childdocforwardprefix|
and functionality expanded
\end{itemize}

%%%%%%%%%%%%%%%%%%%%%%%%%%%%%%%%%%%%%%%%
\paragraph{v1.0:} 2017/04/27

\begin{itemize}
\item
manual and install package
\item
first version published on CTAN
\end{itemize}

%%%%%%%%%%%%%%%%%%%%%%%%%%%%%%%%%%%%%%%%
\paragraph{v0.6:} 2017/04/26

\begin{itemize}
\item
redirection mechanism added
\end{itemize}

%%%%%%%%%%%%%%%%%%%%%%%%%%%%%%%%%%%%%%%%
\paragraph{v0.5:} 2017/04/26

\begin{itemize}
\item
functionality in definition file
\end{itemize}


%%%%%%%%%%%%%%%%%%%%%%%%%%%%%%%%%%%%%%%%%%%%%%%%%%%%%%%%%%%%%%%%%%%%%%%%%%%%%%%%
%%%%%%%%%%%%%%%%%%%%%%%%%%%%%%%%%%%%%%%%%%%%%%%%%%%%%%%%%%%%%%%%%%%%%%%%%%%%%%%%
%%%%%%%%%%%%%%%%%%%%%%%%%%%%%%%%%%%%%%%%%%%%%%%%%%%%%%%%%%%%%%%%%%%%%%%%%%%%%%%%
\appendix

\settowidth\MacroIndent{\rmfamily\scriptsize 000\ }

 \DocInput{childdoc.dtx}

\end{document}
%</driver>
% \fi
%
% %%%%%%%%%%%%%%%%%%%%%%%%%%%%%%%%%%%%%%%%%%%%%%%%%%%%%%%%%%%%%%%%%%%%%%%%%%%%%%
% %%%%%%%%%%%%%%%%%%%%%%%%%%%%%%%%%%%%%%%%%%%%%%%%%%%%%%%%%%%%%%%%%%%%%%%%%%%%%%
% \section{Sample}
%\iffalse
%<*samplemain>
%\fi
%
% The following presents a sample document
% with two chapters, two parts, a title page,
% a compile flag as well as three forwarding files to set the flag.
% It consists of eight |.tex| files:
% \begin{center}
% \begin{tabular}{ll}
% |cdocsamp.tex|&main file\\
% |cdocsch1.tex|&include file for chapter 1\\
% |cdocsch2.tex|&include file for chapter 2\\
% |cdocspt3.tex|&include file for part 3\\
% |cdocspt4.tex|&include file for part 4\\
% |cdocsdrf.tex|&forwarding file for main file in draft mode\\
% |cdocsfi1.tex|&forwarding file for final version of chapter 1\\
% |cdocsfi2.tex|&forwarding file for final version of chapter 2\\
% \end{tabular}
% \end{center}
% Each of the eight files can be compiled directly by the \LaTeX{} compiler.
%
% %%%%%%%%%%%%%%%%%%%%%%%%%%%%%%%%%%%%%%
% \paragraph{Main File.}
%
% The main file is called |cdocsamp.tex|.
%
% Load the \textsf{childdoc} definitions and
% declare the filename for the main document:
%    \begin{macrocode}
\input{childdoc.def}
\childdocmain{}
%    \end{macrocode}

% Optional override for |\version| flag:
%    \begin{macrocode}
%%\ifchilddoc\else\providecommand{\version}{draft}\fi
%    \end{macrocode}

% Define the default values for the |\version| flag
% (|final| for the main file and |draft| for childs):
%    \begin{macrocode}
\ifchilddoc
\providecommand{\version}{draft}
\else
\providecommand{\version}{final}
\fi
%    \end{macrocode}

% Load the standard document class:
%    \begin{macrocode}
\documentclass[12pt]{article}
%    \end{macrocode}

% Start the document body:
%    \begin{macrocode}
\begin{document}
%    \end{macrocode}

% Declare a title page.
% Print title, part of document being processed and version flag:
%    \begin{macrocode}
\addtocounter{page}{-1}
\begin{center}
{\LARGE\bfseries{}childdoc example\par}
\vspace{1cm}
\ifchilddoc
\ifchilddocmanual part\else chapter\fi:
`\childdocname' of `\childdocjob'\par
\else
main document: `\childdocjob'\par
\fi
version: \version\par
\end{center}
\newpage
%    \end{macrocode}

% Manually include selected file,
% otherwise process as usual:
%    \begin{macrocode}
\ifchilddocmanual
\section*{part `\childdocname'}
\input{\childdocname}
\else
%    \end{macrocode}

% Include the two chapters:
%    \begin{macrocode}
\include{cdocsch1}
\include{cdocsch2}
%    \end{macrocode}

% Include the two parts unless only chapters should be displayed:
%    \begin{macrocode}
\ifchilddoc\else
\section{part three}
\input{cdocspt3}
\section{part four}
\input{cdocspt4}
\fi
%    \end{macrocode}

% Process as usual until here:
%    \begin{macrocode}
\fi
%    \end{macrocode}

% End of document body:
%    \begin{macrocode}
\end{document}
%    \end{macrocode}
%\iffalse
%</samplemain>
%\fi
%
% %%%%%%%%%%%%%%%%%%%%%%%%%%%%%%%%%%%%%%
% \paragraph{Chapter Include Files.}
%
% The include files are called |cdocsch1.tex| and |cdocsch2.tex|.
%
%\iffalse
%<*samplechap1|samplechap2>
%\fi

% Optional override for |\version| flag:
%    \begin{macrocode}
%%\providecommand{\version}{final}
%    \end{macrocode}

% Include the main document:
%    \begin{macrocode}
\input{childdoc.def}
\childdocof{cdocsamp}
%    \end{macrocode}

%\iffalse
%</samplechap1|samplechap2>
%\fi
%
%\iffalse
%<*samplechap1>
%\fi
% Some text for chapter 1:
%    \begin{macrocode}
\section{one}
some text in chapter one
%    \end{macrocode}

%\iffalse
%</samplechap1>
%\fi
% Some text for chapter 2:
%\iffalse
%<*samplechap2>
%\fi
%    \begin{macrocode}
\section{two}
more text in chapter two
%    \end{macrocode}

%\iffalse
%</samplechap2>
%\fi
%
% %%%%%%%%%%%%%%%%%%%%%%%%%%%%%%%%%%%%%%
% \paragraph{Part Include Files.}
%
% The include files are called |cdocspt3.tex| and |cdocspt4.tex|.
%
%\iffalse
%<*samplepart3|samplepart4>
%\fi

% Optional override for |\version| flag:
%    \begin{macrocode}
%%\providecommand{\version}{final}
%    \end{macrocode}

% Include the main document:
%    \begin{macrocode}
\input{childdoc.def}
\childdocby{cdocsamp}
%    \end{macrocode}

%\iffalse
%</samplepart3|samplepart4>
%\fi
%
%\iffalse
%<*samplepart3>
%\fi
% Some text for part 3:
%    \begin{macrocode}
some text in part three
%    \end{macrocode}

%\iffalse
%</samplepart3>
%\fi
% Some text for part 4:
%\iffalse
%<*samplepart4>
%\fi
%    \begin{macrocode}
more text in part four
%    \end{macrocode}

%\iffalse
%</samplepart4>
%\fi
%
% %%%%%%%%%%%%%%%%%%%%%%%%%%%%%%%%%%%%%%
% \paragraph{Forwarding for a Complete Draft.}
%
% The following forwarding file |cdocsdrf.tex|
% compiles the main document in draft mode:
%\iffalse
%<*sampledraft>
%\fi
%    \begin{macrocode}
\def\version{draft}
\input{childdoc.def}
\childdocforward{cdocsamp}
%    \end{macrocode}

%\iffalse
%</sampledraft>
%\fi
%
% %%%%%%%%%%%%%%%%%%%%%%%%%%%%%%%%%%%%%%
% \paragraph{Forwarding for Final Version of the Chapters.}
%
% The following forwarding files |cdocsfn1.tex| and |cdocsfn2.tex|
% (with identical content)
% compile the final versions of the child documents
% |cdocsch1.tex| and |cdocsch2.tex|, respectively:
%\iffalse
%<*samplefinal>
%\fi
%    \begin{macrocode}
\def\version{final}
\input{childdoc.def}
\childdocforwardprefix[cdocsamp]{cdocsfn}{cdocsch}
%    \end{macrocode}

%\iffalse
%</samplefinal>
%\fi
%
% %%%%%%%%%%%%%%%%%%%%%%%%%%%%%%%%%%%%%%
% \paragraph{Command Line Processing.}
%
% The following three command lines generate the output files
% |cdocscld|, |cdocscl1| and |cdocscl2|
% which should be identical to
% |cdocsdrf|, |cdocsch1| and |cdocsfn2|, respectively:
% \begin{center}
% \begin{tabular}{l}
% |latex -jobname cdocscld \|\\
% |  "\def\version{draft}\input{childdoc.def}\childdocforward{cdocsamp}"|\\
% |latex -jobname cdocscl1 \|\\
% |  "\input{childdoc.def}\childdocforward[cdocsamp]{cdocsch1}"|\\
% |latex -jobname cdocscl2 \|\\
% |  "\def\version{final}\input{childdoc.def}\childdocforward{cdocsch2}"|
% \end{tabular}
% \end{center}
% Note that the trailing backslash on each first line
% merely continues the input to the second line
% (for convenient cut ant paste).
% Furthermore, the command |latex| can be replaced by any
% of its alternative versions such as |pdflatex|.
%
% %%%%%%%%%%%%%%%%%%%%%%%%%%%%%%%%%%%%%%%%%%%%%%%%%%%%%%%%%%%%%%%%%%%%%%%%%%%%%%
% %%%%%%%%%%%%%%%%%%%%%%%%%%%%%%%%%%%%%%%%%%%%%%%%%%%%%%%%%%%%%%%%%%%%%%%%%%%%%%
% \section{Implementation}
%\iffalse
%<*package>
%\fi
%
% This section describes the definitions file |childdoc.def|.

% The definitions cannot be loaded using |\usepackage| or |\RequirePackage|
% which has a mechanism to prevent loading a style file more than once.
% When loading the definitions by means of |\input|
% multiple instances have to be prevented manually:
%\iffalse
%This code needs to be before the `\ProvidesFile' directive
%which is defined at the beginning of this file.
%Therefore it is also placed there and commented out here.
%</package>
%<*discard>
%\fi
%    \begin{macrocode}
\ifdefined\childdocmain\endinput\fi
%    \end{macrocode}
%\iffalse
%</discard>
%<*package>
%\fi
%
% \macro{\ifchilddoc}
% \macro{\ifchilddocmanual}
% The conditional |\ifchilddoc| tells whether a
% child (true) or main (false) document is being compiled.
% The conditional |\ifchilddocmanual| tells whether
% the |\includeonly| mechanism is used (false) or
% the selection of child files must be performed manually (true).
% The definitions initialise to false:
%    \begin{macrocode}
\newif\ifchilddoc
\newif\ifchilddocmanual
%    \end{macrocode}

% \macro{\childdocname}
% \macro{\childdocjob}
% The macro |\childdocname| stores the name of the main document
% to be compiled. The macro |\childdocjob| stores the name of
% the document on which the \LaTeX{} compiler was originally invoked.
% The content of |\jobname| cannot be compared
% to filenames specified in the source due to different catcodes.
% The following code rescans |\jobname|, stores the result
% in |\childdocname| and saves a copy in |\childdocjob|:
%    \begin{macrocode}
\edef\childdocname{\scantokens\expandafter{\jobname\noexpand}}
\let\childdocjob\childdocname
%    \end{macrocode}

% \macro{\childdocdisable}
% The macro |\childdocdisable| prevents the main file
% from being processed more than once.
% At this stage, the main document command |\childdocmain|
% is assumed to be called once again where it should do nothing.
% Any subsequent call to it should prevent
% a secondary processing of the main document
% It overwrites the forwarding commands
% |\childdocof| and |\childdocforward|
% with empty macros to prevent further inclusions of the main document:
%    \begin{macrocode}
\newcommand{\childdocdisable}
{
  \renewcommand{\childdocmain}[1]{\renewcommand{\childdocmain}[1]{\endinput}}
  \renewcommand{\childdocof}[1]{}
  \renewcommand{\childdocby}[2][]{}
  \renewcommand{\childdocforward}[2][]{}
  \renewcommand{\childdocdisable}{}
}
%    \end{macrocode}

% \macro{\childdocmain}
% The macro |\childdocmain| is to be called at the top of the main file
% with nothing or the main filename (without extension) as argument.
% First, it breaks loops.
% If the argument is not empty and does not match |\childdocname|
% (which is set by the first inclusion of |childdoc.def|),
% |\ifchilddoc| is set to true, |\includeonly| is applied to the child file
% and |\jobname| is set to the main file
% (for proper handling of |.aux| files):
%    \begin{macrocode}
\newcommand{\childdocmain}[1]
{
  \childdocdisable\childdocmain{}
  \if?#1?\else
    \begingroup
      \def\childdoctmp{#1}
      \ifx\childdoctmp\childdocname
        \def\childdoctmp{}
      \else
        \def\childdoctmp
        {
          \childdoctrue
          \includeonly{\childdocname}
          \def\childdocjob{#1}
          \def\jobname{#1}
        }
      \fi
      \expandafter
    \endgroup
    \childdoctmp
  \fi
}
%    \end{macrocode}

% \macro{\childdocof}
% The command |\childdocof| redirects
% compilation to the main file |#1|.
%    \begin{macrocode}
\newcommand{\childdocof}[1]
{
  \childdocdisable
  \childdoctrue
  \includeonly{\childdocname}
  \def\jobname{#1}
  \def\childdocjob{#1}
  \input{#1}
}
%    \end{macrocode}

% \macro{\childdocby}
% The command |\childdocby| ....
%    \begin{macrocode}
\newcommand{\childdocby}[2][]
{
  \childdocdisable
  \childdoctrue
  \childdocmanualtrue
  \if?#1?\else
    \def\jobname{#2}
  \fi
  \def\childdocjob{#2}
  \input{#2}
  \endinput
}
%    \end{macrocode}

% \macro{\childdocforward}
% The command |\childdocforward| redirects
% compilation to the main file or
% (if the optional argument is given) a child file.
% Parameters are set as if the main file
% or a child file starting with |\childdocof| was compiled.
% Then compilation is handed over to the main file:
%    \begin{macrocode}
\newcommand{\childdocforward}[2][]
{
  \begingroup
    \if?#1?
      \def\childdoctmp
      {
        \def\childdocname{#2}
        \def\childdocjob{#2}
        \def\jobname{#2}
        \input{#2}
        \endinput
      }
    \else
      \def\childdoctmp
      {
        \childdocdisable
        \def\childdocname{#2}
        \childdoctrue
        \includeonly{#2}
        \def\childdocjob{#1}
        \def\jobname{#1}
        \input{#1}
        \endinput
      }
    \fi
    \expandafter
  \endgroup
  \childdoctmp
}
%    \end{macrocode}

% \macro{\childdocforwardprefix}
% The command |\childdocforwardprefix| redirects
% compilation to the main or a child file by means of a pattern.
% The prefix |#1| in the current filename is replaced by |#2|
% and the suffix of the current filename is kept
% (it is assumed that the filename does not contain the substring `|~~~|'
% which is used as a delimiter).
% Compilation is handed over to the new file by |\childdocforward|:
%    \begin{macrocode}
\newcommand{\childdocforwardprefix}[3][]
{
  \begingroup
    \def\childdocextract #2##1~~~{\def\childdoctmp{\childdocforward[#1]{#3##1}}}
    \expandafter\childdocextract\childdocname~~~
    \expandafter
  \endgroup
  \childdoctmp
}
%    \end{macrocode}

% \macro{\childdoc}
% The deprecated macro |\childdoc| is a legacy version of |\childdocmain|:
%    \begin{macrocode}
\newcommand{\childdoc}{\childdocmain}
%    \end{macrocode}

% \macro{\childdocredirect}
% The deprecated macro |\childdocredirect| is a legacy version
% of |\childdocforward| and |\childdocforwardprefix|:
%    \begin{macrocode}
\newcommand{\childdocredirect}[2][]
{
  \begingroup
    \if?#1?
      \def\childdoctmp{\childdocforward{#2}}
    \else
      \def\childdoctmp{\childdocforwardprefix{#1}{#2}}
    \fi
    \expandafter
  \endgroup
  \childdoctmp
}
%    \end{macrocode}

%\iffalse
%</package>
%\fi
%
\endinput
\childdocforward{cdocsamp}"|\\
% |latex -jobname cdocscl1 \|\\
% |  "% \iffalse
%
% childdoc.dtx Copyright (C) 2017-2018 Niklas Beisert
%
% This work may be distributed and/or modified under the
% conditions of the LaTeX Project Public License, either version 1.3
% of this license or (at your option) any later version.
% The latest version of this license is in
%   http://www.latex-project.org/lppl.txt
% and version 1.3 or later is part of all distributions of LaTeX
% version 2005/12/01 or later.
%
% This work has the LPPL maintenance status `maintained'.
%
% The Current Maintainer of this work is Niklas Beisert.
%
% This work consists of the files childdoc.dtx and childdoc.ins
% and the derived files childdoc.def and cdocsamp.tex with
% cdocsch1.tex, cdocsch2.tex, cdocsdrf.tex, cdocsfn1.tex, cdocsfn2.tex.
%
%<package>\ifdefined\childdocmain\endinput\fi
%<package>\ProvidesFile{childdoc.def}[2018/12/30 v2.0 child document driver]
%<samplemain>\ProvidesFile{cdocsamp.tex}[2018/12/30 v2.0 sample for childdoc]
%<*driver>
%\ProvidesFile{childdoc.drv}[2018/12/30 v2.0 childdoc reference manual file]
\PassOptionsToClass{10pt,a4paper}{article}
\documentclass{ltxdoc}

\usepackage[margin=35mm]{geometry}
\usepackage{hyperref}
\usepackage{hyperxmp}
\usepackage[usenames]{color}

\hypersetup{colorlinks=true}
\hypersetup{pdfstartview=FitH}
\hypersetup{pdfpagemode=UseNone}
\hypersetup{pdfsource={}}
\hypersetup{pdflang={en-UK}}
\hypersetup{pdfcopyright={Copyright 2017-2018 Niklas Beisert.
  This work may be distributed and/or modified under the
  conditions of the LaTeX Project Public License, either version 1.3
  of this license or (at your option) any later version.}}
\hypersetup{pdflicenseurl={http://www.latex-project.org/lppl.txt}}
\hypersetup{pdfcontactaddress={ETH Zurich, ITP, HIT K,
  Wolfgang-Pauli-Strasse 27}}
\hypersetup{pdfcontactpostcode={8093}}
\hypersetup{pdfcontactcity={Zurich}}
\hypersetup{pdfcontactcountry={Switzerland}}
\hypersetup{pdfcontactemail={nbeisert@itp.phys.ethz.ch}}
\hypersetup{pdfcontacturl={http://people.phys.ethz.ch/\xmptilde nbeisert/}}

\newcommand{\secref}[1]{\hyperref[#1]{section \ref*{#1}}}

\parskip1ex
\parindent0pt
\let\olditemize\itemize
\def\itemize{\olditemize\parskip0pt}

\begin{document}

\title{The \textsf{childdoc} Package}
\hypersetup{pdftitle={The childdoc Package}}
\author{Niklas Beisert\\[2ex]
  Institut f\"ur Theoretische Physik\\
  Eidgen\"ossische Technische Hochschule Z\"urich\\
  Wolfgang-Pauli-Strasse 27, 8093 Z\"urich, Switzerland\\[1ex]
  \href{mailto:nbeisert@itp.phys.ethz.ch}
  {\texttt{nbeisert@itp.phys.ethz.ch}}}
\hypersetup{pdfauthor={Niklas Beisert}}
\hypersetup{pdfsubject={Manual for the LaTeX2e Package childdoc}}
\date{30 December 2018, \textsf{v2.0}}
\maketitle

\begin{abstract}\noindent
\textsf{childdoc} is a \LaTeXe{} package
that enables the direct compilation
of document sections included by |\include|
to individual files.
\end{abstract}

\begingroup
\parskip0ex
\tableofcontents
\endgroup

%%%%%%%%%%%%%%%%%%%%%%%%%%%%%%%%%%%%%%%%%%%%%%%%%%%%%%%%%%%%%%%%%%%%%%%%%%%%%%%%
%%%%%%%%%%%%%%%%%%%%%%%%%%%%%%%%%%%%%%%%%%%%%%%%%%%%%%%%%%%%%%%%%%%%%%%%%%%%%%%%
\section{Introduction}

\LaTeX{} provides a mechanism to structure a large document (such as a book)
into a main file and several child files (containing the chapters)
using the |\include| command.
This mechanism is beneficial for documents
which span hundreds of pages in order to
make the source file(s) more manageable.
Moreover, compilation can be restricted to
selected child files by means of the |\includeonly| command.
The latter feature can be used to reduce the compilation time while editing
(this was significantly more useful in the earlier days of \LaTeX{})
or to generate a smaller document which is easier to navigate.
Another application of |\includeonly| is to generate
documents consisting of selected parts of the complete document.

However, there are a few drawbacks of the plain |\include| mechanism:
\begin{itemize}
\item
The child files cannot be compiled on their own,
they can only be compiled via the main file.
A naive editing environment
(such as a text editor with an option
to have the current file processed by \LaTeX)
may require one to switch to the main file before compiling;
attempting to compile the child file produces errors.
\item
The main file must be modified (each time)
to adjust the |\includeonly| command
to the present needs. This easily leaves the main file in a messy state.
\item
The generated document will always carry the filename
of the main document. This is inconvenient if
several child files are to be compiled and
to be kept for distribution.
\end{itemize}

The present package provides a simple interface
to make child files individually compilable by \LaTeX{}.
Compiling a child file then has the same effect as compiling
the main file with an |\includeonly| command
to select the appropriate child.
Moreover the generated document will carry the name of the child
rather than the main file.
This resolves all three above issues.

This feature is meant to make the editing of books,
thesis documents and lecture notes somewhat more convenient.
However, the package can also be used efficiently for
composing a series of documents (such as exercise sheets)
which are typically distributed individually.
It then assists the author in generating the individual documents
(potentially in different versions)
as well as a document containing the collected series.
Another application is in developing style files
or other kinds of included material
where compilation of the style file could redirect
to a sample or test file.

%%%%%%%%%%%%%%%%%%%%%%%%%%%%%%%%%%%%%%%%%%%%%%%%%%%%%%%%%%%%%%%%%%%%%%%%%%%%%%%%
%%%%%%%%%%%%%%%%%%%%%%%%%%%%%%%%%%%%%%%%%%%%%%%%%%%%%%%%%%%%%%%%%%%%%%%%%%%%%%%%
\section{Usage}

First of all, the package \textsf{childdoc} is \emph{not} a standard
\LaTeXe{} |.sty| style file! Therefore it needs to be invoked in
a non-standard way.

%%%%%%%%%%%%%%%%%%%%%%%%%%%%%%%%%%%%%%%%%%%%%%%%%%%%%%%%%%%%%%%%%%%%%%%%%%%%%%%%
\subsection{Included Files}
\label{sec:include}

%%%%%%%%%%%%%%%%%%%%%%%%%%%%%%%%%%%%%%%%
\DescribeMacro{\childdocmain}
To use the package, add the commands
\begin{center}
\begin{tabular}{l}
|\input{childdoc.def}|\\
|\childdocmain{}|\\
\end{tabular}
\end{center}
at the very top of the main \LaTeX{} file,
in particular \emph{before} the |\documentclass| statement!
The argument of |\childdocmain| should be left empty
(but it must be present).

%%%%%%%%%%%%%%%%%%%%%%%%%%%%%%%%%%%%%%%%
\DescribeMacro{\childdocof}
Furthermore, add the commands
\begin{center}
\begin{tabular}{l}
|\input{childdoc.def}|\\
|\childdocof{|\textit{main}|}|\\
\end{tabular}
\end{center}
at the top of every child file \textit{child}
which is included by |\include{|\textit{child}|}|
from within the main file
(or at least for those files to be compiled individually).
The argument \textit{main} must be the filename of the main file.

There are a couple of
considerations in setting up the main and child documents:

%%%%%%%%%%%%%%%%%%%%%%%%%%%%%%%%%%%%%%%%
\paragraph{Restrictions.}

Please note the following restrictions:
\begin{itemize}
\item
|\childdocmain| must be called with one argument \textit{main}
to ensure compatibility with earlier version of the package.
It must either be empty (|\childdocmain{}|)
or precisely match the filename of the main file in which it is specified.
See \secref{sec:detection} for further information.
\item
The filename \textit{main} must be specified without the |.tex| extension.
\item
The filename \textit{main} is case sensitive
(even in case-insensitive file systems)
due to internal string comparison.
\item
The argument \textit{main} should be fully expanded, it cannot be a macro.
\item
Subdirectories and special characters should be avoided in filenames.
\item
The command |\childdocmain{|\textit{main}|}| must be followed by a whitespace.
It should not be followed immediately by another command
or by a comment mark `|%|'.
This is because the \TeX{} parser reads the token immediately following
the argument of |\childdocmain| and puts it
at the beginning of every child section;
however, a white\-space is ignored.
\end{itemize}

%%%%%%%%%%%%%%%%%%%%%%%%%%%%%%%%%%%%%%%%
\paragraph{Content of Main File.}

It is advisable to place all content in the child files included by |\include|.
Any output contained in the main file will appear in all child documents
unless suppressed manually;
it cannot be suppressed automatically by the |\includeonly| directive
and thus should normally be avoided.
A method to include some content in the main file
by means of conditional processing is described in \secref{sec:conditional}.

%%%%%%%%%%%%%%%%%%%%%%%%%%%%%%%%%%%%%%%%
\paragraph{Page Numbering.}

When only a part of the document is compiled,
the appropriate numbering of pages
(as well as other status parameters)
is determined from the |.aux| files.
The latter contain information from previous passes.
However this information needs to propagate through
all intermediate child documents.
Therefore the page numbering in child documents may well
be inconsistent until the complete document is compiled at least once.

A useful (if unconventional) way to always ensure a consistent
page numbering is to restart the numbering in each child document
and denote the pages by `\textit{child}|.|\textit{page}'
where \textit{child} represents the chapter/section number of the child file.
This can be achieved by the command
|\numberwithin{page}{|\textit{child}|}|
of the \textsf{amsmath} package
where \textit{child} can be |chapter| or |section|
depending on the chosen structuring.
Alternatively, one can modify the macro |\thepage| appropriately
and reset the counter |page| at the start of each child file.

%%%%%%%%%%%%%%%%%%%%%%%%%%%%%%%%%%%%%%%%%%%%%%%%%%%%%%%%%%%%%%%%%%%%%%%%%%%%%%%%
\subsection{Conditional Processing}
\label{sec:conditional}

The package provides a mechanism to compile different versions
of a document. To customise the versions further some conditional processing
can come in handy to distinguish which version is being compiled.
The package provides two macros to describe the compilation context:

%%%%%%%%%%%%%%%%%%%%%%%%%%%%%%%%%%%%%%%%
\DescribeMacro{\ifchilddoc}
The conditional |\ifchilddoc| distinguishes between the compilation of
child documents and the main document:
%
\begin{center}
|\ifchilddoc |\textit{child-code}| |[|\||else |\textit{main-code}]| \||fi|
\end{center}

%%%%%%%%%%%%%%%%%%%%%%%%%%%%%%%%%%%%%%%%
\DescribeMacro{\childdocname}
\DescribeMacro{\childdocjob}
The macro |\childdocname| contains the filename (without extension)
of the main or child file being processed.
Note that |\childdocjob| will always contain the name of the main file.

%%%%%%%%%%%%%%%%%%%%%%%%%%%%%%%%%%%%%%%%
\paragraph{Title Page.}

Conditional processing can be used to include a title or banner page
in the main document when proper precautions are taken.
Importantly, the code in the main file should ensure that the page counter
(as well as other status parameters which are stored in the |.aux| files)
takes the same value after the conditional processing.
Otherwise the page numbers may take divergent values
depending on which part is compiled.

For example, a title page could be declared by:
%
\begin{center}
\begin{tabular}{l}
|\ifchilddoc\||else|\\
|\addtocounter{page}{-1}|\\
\textit{code for title page}\\
|\newpage|\\
|\||fi|
\end{tabular}
\end{center}
%
A banner page for the child documents can be generated by:
%
\begin{center}
\begin{tabular}{l}
|\ifchilddoc|\\
|\addtocounter{page}{-1}|\\
\textit{code for banner page}\\
|\newpage|\\
|\||fi|
\end{tabular}
\end{center}
%
Here one could write a message such as:
\begin{center}
|This is the part \childdocname{} of \childdocjob{}.|
\end{center}

%%%%%%%%%%%%%%%%%%%%%%%%%%%%%%%%%%%%%%%%%%%%%%%%%%%%%%%%%%%%%%%%%%%%%%%%%%%%%%%%
\subsection{Flags}
\label{sec:flags}

The package makes it easy to generate different versions
of the main or child documents.
To this end compilation flags can be defined
and assigned different default values.
They will be particularly useful in conjunction
with the forwarding mechanism described in \secref{sec:forward}.

For example, it may be useful to have a flag |\version|
which can be set to |draft| or |final|.
The document source will contain some conditional code
depending on the value of |\version|.
Suppose further, the flag should default to |final| for the main file
and to |draft| for child files
which is a natural assignment for editing the document.
This is achieved by placing the following code
in the preamble of the main document
(below the |\childdocmain| directive):
%
\begin{center}
\begin{tabular}{l}
|\ifchilddoc|\\
|\providecommand{\version}{draft}|\\
|\||else|\\
|\providecommand{\version}{final}|\\
|\||fi|
\end{tabular}
\end{center}
%
The definition by |\providecommand| makes sure
that previous definitions are not overwritten.
Further statements |\providecommand{\version}{...}|
can thus be added before the above code to override it.

For the main file, one might add a line
(between |\childdocmain| and the above block)
%
\begin{center}
|%\ifchilddoc\||else\providecommand{\version}{draft}\||fi|
\end{center}
%
which can be uncommented to produce a draft version.
Likewise one can add a line to the very top of a child file
(above the |\childdocof{|\textit{main}|}| directive)
%
\begin{center}
|%\providecommand{\version}{final}|
\end{center}
%
which can be uncommented to produce the final version of this child document.

%%%%%%%%%%%%%%%%%%%%%%%%%%%%%%%%%%%%%%%%%%%%%%%%%%%%%%%%%%%%%%%%%%%%%%%%%%%%%%%%
\subsection{Forwarding}
\label{sec:forward}

Different versions of the main or child documents
using compilation flags as described in \secref{sec:flags}
can be (permanently) stored in different files
for convenient compilation, viewing and distribution.
To this end, the package defines a command
to pass on compilation to a different file:

%%%%%%%%%%%%%%%%%%%%%%%%%%%%%%%%%%%%%%%%
\DescribeMacro{\childdocforward}
The command |\childdocforward| redirects processing to
another source file:
%
\begin{center}
\begin{tabular}{l}
|\input{childdoc.def}|\\
|\childdocforward[|\textit{main}|]{|\textit{dest}|}|\\
\end{tabular}
\end{center}
%
The argument \textit{dest} is the destination file
(without extension).
It should be the main file or one of the child files.
Note that further \textsf{childdoc} directives
such as |\childdocof| and |\childdocforward|
in the indicated file will be processed in this form.
The optional argument \textit{main}
passes on directly to the main file \textit{main}
while pretending to compile the child \textit{dest}.
This form behaves as if \textit{dest}
issues |\childdocof{|\textit{main}|}| right away,
and no further \textsf{childdoc} directives will be processed.

%%%%%%%%%%%%%%%%%%%%%%%%%%%%%%%%%%%%%%%%
\DescribeMacro{\...prefix}
In the alternative form |\childdocforwardprefix|,
%
\begin{center}
\begin{tabular}{l}
|\input{childdoc.def}|\\
|\childdocforwardprefix[|\textit{main}|]{|\textit{prefix}|}{|\textit{dest}|}|
\end{tabular}
\end{center}
%
the destination file is determined by a pattern
depending on the current file:
To make this work, the current file must be called
`{\textit{prefix}\hspace{0.2em}\textit{suffix}}'
with \textit{prefix} matching precisely the argument.
Processing is then passed on to the file
`{\textit{dest}\hspace{0.2em}\textit{suffix}}'.
Surely, the same effect is achieved by
directly specifying the
argument `{\textit{dest}\hspace{0.2em}\textit{suffix}}'
in the first form.
However, that requires to set up a different file
for each child. With the alternative form of the command
all these files can have exactly the same content
which simplifies setting them up and maintaining them.

For example, the following file |draft.tex|
with a compilation flag |\version| as described in \secref{sec:flags}
compiles the main document as a draft:
%
\begin{center}
\begin{tabular}{l}
|\def\version{draft}|\\
|\input{childdoc.def}|\\
|\childdocforward{|\textit{main}|}|
\end{tabular}
\end{center}
%
Likewise, the following files |final|\textit{nn}|.tex|
compile the final version of the child document
|child|\textit{nn}|.tex|:
%
\begin{center}
\begin{tabular}{l}
|\def\version{final}|\\
|\input{childdoc.def}|\\
|\childdocforwardprefix{final}{child}|
\end{tabular}
\end{center}
%

Note that when several versions of a main file and/or of each child file
are to be generated, it may be convenient to set up a |Makefile| or
shell script to automatise the process.

%%%%%%%%%%%%%%%%%%%%%%%%%%%%%%%%%%%%%%%%%%%%%%%%%%%%%%%%%%%%%%%%%%%%%%%%%%%%%%%%
\subsection{Command Line Processing}
\label{sec:commandline}

The effect of redirection files can also be achieved by invoking
the \LaTeX{} compiler with a more elaborate command line.
Most conveniently this should be done as part
of a shell script or a |Makefile|.

When using \textsf{childdoc} in the main file, the following
command lines effectively perform a redirection
(note that depending on the shell being used,
backslashes may have to be doubled: `|\|' $\to$ `|\\|'):
%
\begin{center}
|... -jobname "|\textit{target}|" |\\|"|[\textit{flags}]%
|\input{childdoc.def}\childdocforward[|\textit{main}|]{|\textit{dest}|}"|
\end{center}
%
Here \textit{target} is the name of the output file,
\textit{main} is the name of the main file
and \textit{dest} is the name of the main or child file to be processed
(all filenames without extensions).
The optional argument \textit{main} can be omitted
if \textit{main} matches \textit{dest}.
Optionally, compilation \textit{flags} can be defined via |\def| commands.
This command line makes the \TeX{} engine believe
it is compiling the file \textit{target}
whose content is specified as the latter parameter.
The provided code then forwards the processing to
\textit{main} or \textit{dest} as described in \secref{sec:forward}.

%%%%%%%%%%%%%%%%%%%%%%%%%%%%%%%%%%%%%%%%%%%%%%%%%%%%%%%%%%%%%%%%%%%%%%%%%%%%%%%%
\subsection{Include by Input}
\label{sec:input}

Including child documents by |\include| has some restrictions by design.
Most notably, the content of a child document always occupies
its own set of pages; pages cannot be shared between child documents.
Usually, this behaviour makes perfect sense
because each child document contain an essential part of the document.
However, in some situations it may be desirable to compose
a document from a collection of parts
without having mandatory page breaks between then.
For this case, the package
provides a mechanism to include parts
by |\input| which can also be processed individually.
However, by construction this mechanism
requires manual handling of the content to be output.

%%%%%%%%%%%%%%%%%%%%%%%%%%%%%%%%%%%%%%%%
\DescribeMacro{\ifchilddocmanual}
The main file should be prepared as usual, see \secref{sec:include}.
However, the document body must make a distinction
between processing of an individual part and of the main document, e.g.:
%
\begin{center}
\begin{tabular}{l}
|\ifchilddocmanual|\\
|\input{\childdocname}|\\
|\||else|\\
\textit{document body with }|\input{|\textit{part}|}|\\
|\||fi|
\end{tabular}
\end{center}
%
The conditional |\ifchilddocmanual| is true whenever
a part to be included by |\input| is being compiled,
and the name of the part is stored in |\childdocname|.

%%%%%%%%%%%%%%%%%%%%%%%%%%%%%%%%%%%%%%%%
\DescribeMacro{\childdocby}
Each part to be included by |\input| should start with:
%
\begin{center}
\begin{tabular}{l}
|\input{childdoc.def}|\\
|\childdocby{|\textit{main}|}|\\
\end{tabular}
\end{center}
%
The directive |\childdocby| is similar to |\childdocof|
described in \secref{sec:include},
but the subsequent selection of content must be done manually.
To that end, both |\ifchilddoc| and |\ifchilddocmanual|
will be true upon processing of a part,
and the name of the part is stored in |\childdocname|.
Note that |\jobname| will be set to the filename of the current part
so that each part receives an individual |.aux| file
that does not interfere with the |.aux| file(s) of the main document.
This behaviour can be altered by the alternative form
|\childdocby[*]{|\textit{main}|}| (with a non-empty optional argument)
which uses the |.aux| file of the main document
by setting |\jobname| to \textit{main}.

%%%%%%%%%%%%%%%%%%%%%%%%%%%%%%%%%%%%%%%%%%%%%%%%%%%%%%%%%%%%%%%%%%%%%%%%%%%%%%%%
\subsection{Driver Development}
\label{sec:driver}

The \textsf{childdoc} mechanism can also be use for the development
of definition files such as \LaTeX{} styles or classes.
This case differs from the above setup with multiple parts
included by |\include| in that no |\includeonly| should be invoked.
This can be achieved by starting the include file
(before |\ProvidesPackage|) with:
%
\begin{center}
\begin{tabular}{l}
|\input{childdoc.def}|\\
|\childdocforward{|\textit{main}|}|\\
\end{tabular}
\end{center}
%
or alternatively with:
%
\begin{center}
\begin{tabular}{l}
|\input{childdoc.def}|\\
|\childdocby{|\textit{main}|}|\\
\end{tabular}
\end{center}
%
Both forms have slightly different effects as described above.
The main file is prepared as usual, see \secref{sec:include}.

%%%%%%%%%%%%%%%%%%%%%%%%%%%%%%%%%%%%%%%%%%%%%%%%%%%%%%%%%%%%%%%%%%%%%%%%%%%%%%%%
\subsection{Legacy Detection}
\label{sec:detection}

The directive |\childdocmain| in the main file can detect
whether the complete document or merely a child is to be compiled
even without using the directive |\childdocof|.
This method is deprecated because it is less robust
and there is no compelling reason to use it;
it is merely provided for backward compatibility
and it may be removed in future versions.

If the detection mechanism is to be used,
it is mandatory to correctly specify
the filename of the main file as the argument of |\childdocmain|:
%
\begin{center}
\begin{tabular}{l}
|\input{childdoc.def}|\\
|\childdocmain{|\textit{main}|}|\\
\end{tabular}
\end{center}
%
If |\jobname| does not match the argument \textit{main} of |\childdocmain|,
it is assumed that |\jobname| points to the child file to be compiled.
When using |\childdocmain| with the main file specified as argument,
it suffices to start a child file
with just |\input{|\textit{main}|}|
without loading of the package and using |\childdocof|.
If instead all processing is done
with the appropriate \textsf{childdoc} directives,
the argument of \textit{main} of |\childdocmain| can be empty.

An alternative version of the command line processing described
in \secref{sec:commandline} using the detection mechanism reads:
%
\begin{center}
|... -jobname "|\textit{target}|" "|[\textit{flags}]%
[|\def\jobname{|\textit{dest}|}|]|\input{|\textit{main}|}"|
\end{center}

%%%%%%%%%%%%%%%%%%%%%%%%%%%%%%%%%%%%%%%%%%%%%%%%%%%%%%%%%%%%%%%%%%%%%%%%%%%%%%%%
\subsection{Manual Code}
\label{sec:manual}

In case one cannot be certain whether the definitions file |childdoc.def|
is installed on the target \TeX{} distribution
and one prefers not to ship it,
it is conceivable to paste a few relevant commands into the sources.

To that end, drop all statements |\input{childdoc.def}|
and perform the replacements as outlined below.
Instead of |\childdocmain{|\textit{main}|}| add the following code
to the top of the main file:
%
\begin{center}
\begin{tabular}{l}
|\||ifdefined\childdocname\endinput\||fi\newif\ifchilddoc|\\
|\edef\childdocname{\scantokens\expandafter{\jobname\noexpand}}|\\
|\def\childdocmain{|\textit{main}|}\||ifx\childdocmain\childdocname\||else|\\
|\childdoctrue\includeonly{\childdocname}\let\jobname\childdocmain\||fi|\\
\end{tabular}
\end{center}
%
Instead of |\childdocof{|\textit{main}|}| just include the main file
at the top of each child file:
%
\begin{center}
|\input{|\textit{main}|}|
\end{center}
%
A simple redirection |\childdocforward{|\textit{dest}|}| is achieved by:
%
\begin{center}
|\def\jobname{|\textit{dest}|}\input{\jobname}|
\end{center}
%
The redirection with prefix
|\childdocforwardprefix[|\textit{prefix}|]{|\textit{dest}|}|
is accomplished by:
%
\begin{center}
\begin{tabular}{l}
|{\edef\jobname{\scantokens\expandafter{\jobname\noexpand}}|\\
|\def\redirectjob |\textit{prefix}|#1~~~{\gdef\jobname{|\textit{dest}|#1}}|\\
|\expandafter\redirectjob\jobname~~~}\input{\jobname}|
\end{tabular}
\end{center}

In an alternative approach,
child documents can be compiled by a specific command line
without additional code or specific definitions:
%
\begin{center}
|... -jobname "|\textit{target}|" "|[\textit{flags}]%
|\includeonly{|\textit{dest}|}\input{|\textit{main}|}"|
\end{center}
%

%%%%%%%%%%%%%%%%%%%%%%%%%%%%%%%%%%%%%%%%%%%%%%%%%%%%%%%%%%%%%%%%%%%%%%%%%%%%%%%%
%%%%%%%%%%%%%%%%%%%%%%%%%%%%%%%%%%%%%%%%%%%%%%%%%%%%%%%%%%%%%%%%%%%%%%%%%%%%%%%%
\section{Information}

%%%%%%%%%%%%%%%%%%%%%%%%%%%%%%%%%%%%%%%%%%%%%%%%%%%%%%%%%%%%%%%%%%%%%%%%%%%%%%%%
\subsection{Copyright}

Copyright \copyright{} 2017--2018 Niklas Beisert

This work may be distributed and/or modified under the
conditions of the \LaTeX{} Project Public License, either version 1.3
of this license or (at your option) any later version.
The latest version of this license is in
  \url{http://www.latex-project.org/lppl.txt}
and version 1.3 or later is part of all distributions of \LaTeX{}
version 2005/12/01 or later.

This work has the LPPL maintenance status `maintained'.

The Current Maintainer of this work is Niklas Beisert.

This work consists of the files |README.txt|, |childdoc.ins| and |childdoc.dtx|
as well as the derived files |childdoc.def|, |cdocsamp.tex|
with |cdocsch1.tex|, |cdocsch2.tex|, |cdocspt3.tex|, |cdocspt4.tex|,
|cdocsdrf.tex|, |cdocsfn1.tex|, |cdocsfn2.tex|
as well as |childdoc.pdf|.

%%%%%%%%%%%%%%%%%%%%%%%%%%%%%%%%%%%%%%%%%%%%%%%%%%%%%%%%%%%%%%%%%%%%%%%%%%%%%%%%
\subsection{Files and Installation}

The package consists of the files:
%
\begin{center}
\begin{tabular}{ll}
    |README.txt|   & readme file \\
    |childdoc.ins| & installation file \\
    |childdoc.dtx| & source file \\
    |childdoc.def| & definition file \\
    |cdocsamp.tex| & sample main file \\
    |cdocsch1.tex| & sample include file \\
    |cdocsch2.tex| & sample include file \\
    |cdocspt3.tex| & sample part file \\
    |cdocspt4.tex| & sample part file \\
    |cdocsdrf.tex| & sample redirection file \\
    |cdocsfn1.tex| & sample redirection file \\
    |cdocsfn2.tex| & sample redirection file \\
    |childdoc.pdf| & manual
\end{tabular}
\end{center}
%
The distribution consists of the files
|README.txt|, |childdoc.ins| and |childdoc.dtx|.
%
\begin{itemize}
\item
Run (pdf)\LaTeX{} on |childdoc.dtx|
to compile the manual |childdoc.pdf| (this file).
\item
Run \LaTeX{} on |childdoc.ins| to create the definitions file |childdoc.def|
and the sample |cdocsamp.tex| with include files
|cdocsch1.tex|, |cdocsch2.tex|, |cdocspt3.tex|, |cdocspt4.tex|,
|cdocsdrf.tex|, |cdocsfn1.tex|, |cdocsfn2.tex|.
Then copy the file |childdoc.def| to an appropriate directory of your \LaTeX{}
distribution, e.g.\ \textit{texmf-root}|/tex/latex/childdoc|.
\end{itemize}

%%%%%%%%%%%%%%%%%%%%%%%%%%%%%%%%%%%%%%%%%%%%%%%%%%%%%%%%%%%%%%%%%%%%%%%%%%%%%%%%
\subsection{Related CTAN Packages}

There are several other packages which offer a similar functionality:
%
\begin{itemize}
\item
The packages
\href{http://ctan.org/pkg/docmute}{\textsf{docmute}},
\href{http://ctan.org/pkg/includex}{\textsf{includex}} and
\href{http://ctan.org/pkg/standalone}{\textsf{standalone}}
provide commands to include only the document body of
a child file thus allowing both files to be compiled individually.
\item
The packages \href{http://ctan.org/pkg/subdocs}{\textsf{subdocs}}
and \href{http://ctan.org/pkg/subfiles}{\textsf{subfiles}}
provide structures in which the main and child documents can be
encapsulated and allowing them to be compiled individually.
The inclusion mechanism is different from the conventional |\include|.
\item
The package \href{http://ctan.org/pkg/combine}{\textsf{combine}}
is an elaborate solution to combine several documents into one.
\end{itemize}
%
See also the CTAN topic \href{http://ctan.org/topic/subdocs}{\textsf{subdocs}}
for further related packages.
The present package differs from the above solutions in that
a document structure constructed with the conventional |\include| mechanism
just needs two extra commands at the top of every file
such that all constituent files can be compiled individually.

%%%%%%%%%%%%%%%%%%%%%%%%%%%%%%%%%%%%%%%%%%%%%%%%%%%%%%%%%%%%%%%%%%%%%%%%%%%%%%%%
%\subsection{Feature Suggestions}
%
%The following is a list of features which may be useful for future
%versions of this package:
%%
%\begin{itemize}
%\item
%\ldots
%\end{itemize}

%%%%%%%%%%%%%%%%%%%%%%%%%%%%%%%%%%%%%%%%%%%%%%%%%%%%%%%%%%%%%%%%%%%%%%%%%%%%%%%%
\subsection{Revision History}

%%%%%%%%%%%%%%%%%%%%%%%%%%%%%%%%%%%%%%%%
\paragraph{v2.0:} 2018/12/30

\begin{itemize}
\item
immediate forward processing
\item
added |\childdocby| mechanism
\item
manual restructured
\end{itemize}

%%%%%%%%%%%%%%%%%%%%%%%%%%%%%%%%%%%%%%%%
\paragraph{v1.6:} 2018/01/17

\begin{itemize}
\item
application for development of include files
\item
corrections to manual
\end{itemize}

%%%%%%%%%%%%%%%%%%%%%%%%%%%%%%%%%%%%%%%%
\paragraph{v1.5:} 2017/05/21

\begin{itemize}
\item
more complete structuring introduced
\item
|\childdocof| introduced
\item
|\childdoc| renamed to |\childdocmain|
\item
|\childredirect| renamed to |\childdocforward| and |\childdocforwardprefix|
and functionality expanded
\end{itemize}

%%%%%%%%%%%%%%%%%%%%%%%%%%%%%%%%%%%%%%%%
\paragraph{v1.0:} 2017/04/27

\begin{itemize}
\item
manual and install package
\item
first version published on CTAN
\end{itemize}

%%%%%%%%%%%%%%%%%%%%%%%%%%%%%%%%%%%%%%%%
\paragraph{v0.6:} 2017/04/26

\begin{itemize}
\item
redirection mechanism added
\end{itemize}

%%%%%%%%%%%%%%%%%%%%%%%%%%%%%%%%%%%%%%%%
\paragraph{v0.5:} 2017/04/26

\begin{itemize}
\item
functionality in definition file
\end{itemize}


%%%%%%%%%%%%%%%%%%%%%%%%%%%%%%%%%%%%%%%%%%%%%%%%%%%%%%%%%%%%%%%%%%%%%%%%%%%%%%%%
%%%%%%%%%%%%%%%%%%%%%%%%%%%%%%%%%%%%%%%%%%%%%%%%%%%%%%%%%%%%%%%%%%%%%%%%%%%%%%%%
%%%%%%%%%%%%%%%%%%%%%%%%%%%%%%%%%%%%%%%%%%%%%%%%%%%%%%%%%%%%%%%%%%%%%%%%%%%%%%%%
\appendix

\settowidth\MacroIndent{\rmfamily\scriptsize 000\ }

 \DocInput{childdoc.dtx}

\end{document}
%</driver>
% \fi
%
% %%%%%%%%%%%%%%%%%%%%%%%%%%%%%%%%%%%%%%%%%%%%%%%%%%%%%%%%%%%%%%%%%%%%%%%%%%%%%%
% %%%%%%%%%%%%%%%%%%%%%%%%%%%%%%%%%%%%%%%%%%%%%%%%%%%%%%%%%%%%%%%%%%%%%%%%%%%%%%
% \section{Sample}
%\iffalse
%<*samplemain>
%\fi
%
% The following presents a sample document
% with two chapters, two parts, a title page,
% a compile flag as well as three forwarding files to set the flag.
% It consists of eight |.tex| files:
% \begin{center}
% \begin{tabular}{ll}
% |cdocsamp.tex|&main file\\
% |cdocsch1.tex|&include file for chapter 1\\
% |cdocsch2.tex|&include file for chapter 2\\
% |cdocspt3.tex|&include file for part 3\\
% |cdocspt4.tex|&include file for part 4\\
% |cdocsdrf.tex|&forwarding file for main file in draft mode\\
% |cdocsfi1.tex|&forwarding file for final version of chapter 1\\
% |cdocsfi2.tex|&forwarding file for final version of chapter 2\\
% \end{tabular}
% \end{center}
% Each of the eight files can be compiled directly by the \LaTeX{} compiler.
%
% %%%%%%%%%%%%%%%%%%%%%%%%%%%%%%%%%%%%%%
% \paragraph{Main File.}
%
% The main file is called |cdocsamp.tex|.
%
% Load the \textsf{childdoc} definitions and
% declare the filename for the main document:
%    \begin{macrocode}
\input{childdoc.def}
\childdocmain{}
%    \end{macrocode}

% Optional override for |\version| flag:
%    \begin{macrocode}
%%\ifchilddoc\else\providecommand{\version}{draft}\fi
%    \end{macrocode}

% Define the default values for the |\version| flag
% (|final| for the main file and |draft| for childs):
%    \begin{macrocode}
\ifchilddoc
\providecommand{\version}{draft}
\else
\providecommand{\version}{final}
\fi
%    \end{macrocode}

% Load the standard document class:
%    \begin{macrocode}
\documentclass[12pt]{article}
%    \end{macrocode}

% Start the document body:
%    \begin{macrocode}
\begin{document}
%    \end{macrocode}

% Declare a title page.
% Print title, part of document being processed and version flag:
%    \begin{macrocode}
\addtocounter{page}{-1}
\begin{center}
{\LARGE\bfseries{}childdoc example\par}
\vspace{1cm}
\ifchilddoc
\ifchilddocmanual part\else chapter\fi:
`\childdocname' of `\childdocjob'\par
\else
main document: `\childdocjob'\par
\fi
version: \version\par
\end{center}
\newpage
%    \end{macrocode}

% Manually include selected file,
% otherwise process as usual:
%    \begin{macrocode}
\ifchilddocmanual
\section*{part `\childdocname'}
\input{\childdocname}
\else
%    \end{macrocode}

% Include the two chapters:
%    \begin{macrocode}
\include{cdocsch1}
\include{cdocsch2}
%    \end{macrocode}

% Include the two parts unless only chapters should be displayed:
%    \begin{macrocode}
\ifchilddoc\else
\section{part three}
\input{cdocspt3}
\section{part four}
\input{cdocspt4}
\fi
%    \end{macrocode}

% Process as usual until here:
%    \begin{macrocode}
\fi
%    \end{macrocode}

% End of document body:
%    \begin{macrocode}
\end{document}
%    \end{macrocode}
%\iffalse
%</samplemain>
%\fi
%
% %%%%%%%%%%%%%%%%%%%%%%%%%%%%%%%%%%%%%%
% \paragraph{Chapter Include Files.}
%
% The include files are called |cdocsch1.tex| and |cdocsch2.tex|.
%
%\iffalse
%<*samplechap1|samplechap2>
%\fi

% Optional override for |\version| flag:
%    \begin{macrocode}
%%\providecommand{\version}{final}
%    \end{macrocode}

% Include the main document:
%    \begin{macrocode}
\input{childdoc.def}
\childdocof{cdocsamp}
%    \end{macrocode}

%\iffalse
%</samplechap1|samplechap2>
%\fi
%
%\iffalse
%<*samplechap1>
%\fi
% Some text for chapter 1:
%    \begin{macrocode}
\section{one}
some text in chapter one
%    \end{macrocode}

%\iffalse
%</samplechap1>
%\fi
% Some text for chapter 2:
%\iffalse
%<*samplechap2>
%\fi
%    \begin{macrocode}
\section{two}
more text in chapter two
%    \end{macrocode}

%\iffalse
%</samplechap2>
%\fi
%
% %%%%%%%%%%%%%%%%%%%%%%%%%%%%%%%%%%%%%%
% \paragraph{Part Include Files.}
%
% The include files are called |cdocspt3.tex| and |cdocspt4.tex|.
%
%\iffalse
%<*samplepart3|samplepart4>
%\fi

% Optional override for |\version| flag:
%    \begin{macrocode}
%%\providecommand{\version}{final}
%    \end{macrocode}

% Include the main document:
%    \begin{macrocode}
\input{childdoc.def}
\childdocby{cdocsamp}
%    \end{macrocode}

%\iffalse
%</samplepart3|samplepart4>
%\fi
%
%\iffalse
%<*samplepart3>
%\fi
% Some text for part 3:
%    \begin{macrocode}
some text in part three
%    \end{macrocode}

%\iffalse
%</samplepart3>
%\fi
% Some text for part 4:
%\iffalse
%<*samplepart4>
%\fi
%    \begin{macrocode}
more text in part four
%    \end{macrocode}

%\iffalse
%</samplepart4>
%\fi
%
% %%%%%%%%%%%%%%%%%%%%%%%%%%%%%%%%%%%%%%
% \paragraph{Forwarding for a Complete Draft.}
%
% The following forwarding file |cdocsdrf.tex|
% compiles the main document in draft mode:
%\iffalse
%<*sampledraft>
%\fi
%    \begin{macrocode}
\def\version{draft}
\input{childdoc.def}
\childdocforward{cdocsamp}
%    \end{macrocode}

%\iffalse
%</sampledraft>
%\fi
%
% %%%%%%%%%%%%%%%%%%%%%%%%%%%%%%%%%%%%%%
% \paragraph{Forwarding for Final Version of the Chapters.}
%
% The following forwarding files |cdocsfn1.tex| and |cdocsfn2.tex|
% (with identical content)
% compile the final versions of the child documents
% |cdocsch1.tex| and |cdocsch2.tex|, respectively:
%\iffalse
%<*samplefinal>
%\fi
%    \begin{macrocode}
\def\version{final}
\input{childdoc.def}
\childdocforwardprefix[cdocsamp]{cdocsfn}{cdocsch}
%    \end{macrocode}

%\iffalse
%</samplefinal>
%\fi
%
% %%%%%%%%%%%%%%%%%%%%%%%%%%%%%%%%%%%%%%
% \paragraph{Command Line Processing.}
%
% The following three command lines generate the output files
% |cdocscld|, |cdocscl1| and |cdocscl2|
% which should be identical to
% |cdocsdrf|, |cdocsch1| and |cdocsfn2|, respectively:
% \begin{center}
% \begin{tabular}{l}
% |latex -jobname cdocscld \|\\
% |  "\def\version{draft}\input{childdoc.def}\childdocforward{cdocsamp}"|\\
% |latex -jobname cdocscl1 \|\\
% |  "\input{childdoc.def}\childdocforward[cdocsamp]{cdocsch1}"|\\
% |latex -jobname cdocscl2 \|\\
% |  "\def\version{final}\input{childdoc.def}\childdocforward{cdocsch2}"|
% \end{tabular}
% \end{center}
% Note that the trailing backslash on each first line
% merely continues the input to the second line
% (for convenient cut ant paste).
% Furthermore, the command |latex| can be replaced by any
% of its alternative versions such as |pdflatex|.
%
% %%%%%%%%%%%%%%%%%%%%%%%%%%%%%%%%%%%%%%%%%%%%%%%%%%%%%%%%%%%%%%%%%%%%%%%%%%%%%%
% %%%%%%%%%%%%%%%%%%%%%%%%%%%%%%%%%%%%%%%%%%%%%%%%%%%%%%%%%%%%%%%%%%%%%%%%%%%%%%
% \section{Implementation}
%\iffalse
%<*package>
%\fi
%
% This section describes the definitions file |childdoc.def|.

% The definitions cannot be loaded using |\usepackage| or |\RequirePackage|
% which has a mechanism to prevent loading a style file more than once.
% When loading the definitions by means of |\input|
% multiple instances have to be prevented manually:
%\iffalse
%This code needs to be before the `\ProvidesFile' directive
%which is defined at the beginning of this file.
%Therefore it is also placed there and commented out here.
%</package>
%<*discard>
%\fi
%    \begin{macrocode}
\ifdefined\childdocmain\endinput\fi
%    \end{macrocode}
%\iffalse
%</discard>
%<*package>
%\fi
%
% \macro{\ifchilddoc}
% \macro{\ifchilddocmanual}
% The conditional |\ifchilddoc| tells whether a
% child (true) or main (false) document is being compiled.
% The conditional |\ifchilddocmanual| tells whether
% the |\includeonly| mechanism is used (false) or
% the selection of child files must be performed manually (true).
% The definitions initialise to false:
%    \begin{macrocode}
\newif\ifchilddoc
\newif\ifchilddocmanual
%    \end{macrocode}

% \macro{\childdocname}
% \macro{\childdocjob}
% The macro |\childdocname| stores the name of the main document
% to be compiled. The macro |\childdocjob| stores the name of
% the document on which the \LaTeX{} compiler was originally invoked.
% The content of |\jobname| cannot be compared
% to filenames specified in the source due to different catcodes.
% The following code rescans |\jobname|, stores the result
% in |\childdocname| and saves a copy in |\childdocjob|:
%    \begin{macrocode}
\edef\childdocname{\scantokens\expandafter{\jobname\noexpand}}
\let\childdocjob\childdocname
%    \end{macrocode}

% \macro{\childdocdisable}
% The macro |\childdocdisable| prevents the main file
% from being processed more than once.
% At this stage, the main document command |\childdocmain|
% is assumed to be called once again where it should do nothing.
% Any subsequent call to it should prevent
% a secondary processing of the main document
% It overwrites the forwarding commands
% |\childdocof| and |\childdocforward|
% with empty macros to prevent further inclusions of the main document:
%    \begin{macrocode}
\newcommand{\childdocdisable}
{
  \renewcommand{\childdocmain}[1]{\renewcommand{\childdocmain}[1]{\endinput}}
  \renewcommand{\childdocof}[1]{}
  \renewcommand{\childdocby}[2][]{}
  \renewcommand{\childdocforward}[2][]{}
  \renewcommand{\childdocdisable}{}
}
%    \end{macrocode}

% \macro{\childdocmain}
% The macro |\childdocmain| is to be called at the top of the main file
% with nothing or the main filename (without extension) as argument.
% First, it breaks loops.
% If the argument is not empty and does not match |\childdocname|
% (which is set by the first inclusion of |childdoc.def|),
% |\ifchilddoc| is set to true, |\includeonly| is applied to the child file
% and |\jobname| is set to the main file
% (for proper handling of |.aux| files):
%    \begin{macrocode}
\newcommand{\childdocmain}[1]
{
  \childdocdisable\childdocmain{}
  \if?#1?\else
    \begingroup
      \def\childdoctmp{#1}
      \ifx\childdoctmp\childdocname
        \def\childdoctmp{}
      \else
        \def\childdoctmp
        {
          \childdoctrue
          \includeonly{\childdocname}
          \def\childdocjob{#1}
          \def\jobname{#1}
        }
      \fi
      \expandafter
    \endgroup
    \childdoctmp
  \fi
}
%    \end{macrocode}

% \macro{\childdocof}
% The command |\childdocof| redirects
% compilation to the main file |#1|.
%    \begin{macrocode}
\newcommand{\childdocof}[1]
{
  \childdocdisable
  \childdoctrue
  \includeonly{\childdocname}
  \def\jobname{#1}
  \def\childdocjob{#1}
  \input{#1}
}
%    \end{macrocode}

% \macro{\childdocby}
% The command |\childdocby| ....
%    \begin{macrocode}
\newcommand{\childdocby}[2][]
{
  \childdocdisable
  \childdoctrue
  \childdocmanualtrue
  \if?#1?\else
    \def\jobname{#2}
  \fi
  \def\childdocjob{#2}
  \input{#2}
  \endinput
}
%    \end{macrocode}

% \macro{\childdocforward}
% The command |\childdocforward| redirects
% compilation to the main file or
% (if the optional argument is given) a child file.
% Parameters are set as if the main file
% or a child file starting with |\childdocof| was compiled.
% Then compilation is handed over to the main file:
%    \begin{macrocode}
\newcommand{\childdocforward}[2][]
{
  \begingroup
    \if?#1?
      \def\childdoctmp
      {
        \def\childdocname{#2}
        \def\childdocjob{#2}
        \def\jobname{#2}
        \input{#2}
        \endinput
      }
    \else
      \def\childdoctmp
      {
        \childdocdisable
        \def\childdocname{#2}
        \childdoctrue
        \includeonly{#2}
        \def\childdocjob{#1}
        \def\jobname{#1}
        \input{#1}
        \endinput
      }
    \fi
    \expandafter
  \endgroup
  \childdoctmp
}
%    \end{macrocode}

% \macro{\childdocforwardprefix}
% The command |\childdocforwardprefix| redirects
% compilation to the main or a child file by means of a pattern.
% The prefix |#1| in the current filename is replaced by |#2|
% and the suffix of the current filename is kept
% (it is assumed that the filename does not contain the substring `|~~~|'
% which is used as a delimiter).
% Compilation is handed over to the new file by |\childdocforward|:
%    \begin{macrocode}
\newcommand{\childdocforwardprefix}[3][]
{
  \begingroup
    \def\childdocextract #2##1~~~{\def\childdoctmp{\childdocforward[#1]{#3##1}}}
    \expandafter\childdocextract\childdocname~~~
    \expandafter
  \endgroup
  \childdoctmp
}
%    \end{macrocode}

% \macro{\childdoc}
% The deprecated macro |\childdoc| is a legacy version of |\childdocmain|:
%    \begin{macrocode}
\newcommand{\childdoc}{\childdocmain}
%    \end{macrocode}

% \macro{\childdocredirect}
% The deprecated macro |\childdocredirect| is a legacy version
% of |\childdocforward| and |\childdocforwardprefix|:
%    \begin{macrocode}
\newcommand{\childdocredirect}[2][]
{
  \begingroup
    \if?#1?
      \def\childdoctmp{\childdocforward{#2}}
    \else
      \def\childdoctmp{\childdocforwardprefix{#1}{#2}}
    \fi
    \expandafter
  \endgroup
  \childdoctmp
}
%    \end{macrocode}

%\iffalse
%</package>
%\fi
%
\endinput
\childdocforward[cdocsamp]{cdocsch1}"|\\
% |latex -jobname cdocscl2 \|\\
% |  "\def\version{final}% \iffalse
%
% childdoc.dtx Copyright (C) 2017-2018 Niklas Beisert
%
% This work may be distributed and/or modified under the
% conditions of the LaTeX Project Public License, either version 1.3
% of this license or (at your option) any later version.
% The latest version of this license is in
%   http://www.latex-project.org/lppl.txt
% and version 1.3 or later is part of all distributions of LaTeX
% version 2005/12/01 or later.
%
% This work has the LPPL maintenance status `maintained'.
%
% The Current Maintainer of this work is Niklas Beisert.
%
% This work consists of the files childdoc.dtx and childdoc.ins
% and the derived files childdoc.def and cdocsamp.tex with
% cdocsch1.tex, cdocsch2.tex, cdocsdrf.tex, cdocsfn1.tex, cdocsfn2.tex.
%
%<package>\ifdefined\childdocmain\endinput\fi
%<package>\ProvidesFile{childdoc.def}[2018/12/30 v2.0 child document driver]
%<samplemain>\ProvidesFile{cdocsamp.tex}[2018/12/30 v2.0 sample for childdoc]
%<*driver>
%\ProvidesFile{childdoc.drv}[2018/12/30 v2.0 childdoc reference manual file]
\PassOptionsToClass{10pt,a4paper}{article}
\documentclass{ltxdoc}

\usepackage[margin=35mm]{geometry}
\usepackage{hyperref}
\usepackage{hyperxmp}
\usepackage[usenames]{color}

\hypersetup{colorlinks=true}
\hypersetup{pdfstartview=FitH}
\hypersetup{pdfpagemode=UseNone}
\hypersetup{pdfsource={}}
\hypersetup{pdflang={en-UK}}
\hypersetup{pdfcopyright={Copyright 2017-2018 Niklas Beisert.
  This work may be distributed and/or modified under the
  conditions of the LaTeX Project Public License, either version 1.3
  of this license or (at your option) any later version.}}
\hypersetup{pdflicenseurl={http://www.latex-project.org/lppl.txt}}
\hypersetup{pdfcontactaddress={ETH Zurich, ITP, HIT K,
  Wolfgang-Pauli-Strasse 27}}
\hypersetup{pdfcontactpostcode={8093}}
\hypersetup{pdfcontactcity={Zurich}}
\hypersetup{pdfcontactcountry={Switzerland}}
\hypersetup{pdfcontactemail={nbeisert@itp.phys.ethz.ch}}
\hypersetup{pdfcontacturl={http://people.phys.ethz.ch/\xmptilde nbeisert/}}

\newcommand{\secref}[1]{\hyperref[#1]{section \ref*{#1}}}

\parskip1ex
\parindent0pt
\let\olditemize\itemize
\def\itemize{\olditemize\parskip0pt}

\begin{document}

\title{The \textsf{childdoc} Package}
\hypersetup{pdftitle={The childdoc Package}}
\author{Niklas Beisert\\[2ex]
  Institut f\"ur Theoretische Physik\\
  Eidgen\"ossische Technische Hochschule Z\"urich\\
  Wolfgang-Pauli-Strasse 27, 8093 Z\"urich, Switzerland\\[1ex]
  \href{mailto:nbeisert@itp.phys.ethz.ch}
  {\texttt{nbeisert@itp.phys.ethz.ch}}}
\hypersetup{pdfauthor={Niklas Beisert}}
\hypersetup{pdfsubject={Manual for the LaTeX2e Package childdoc}}
\date{30 December 2018, \textsf{v2.0}}
\maketitle

\begin{abstract}\noindent
\textsf{childdoc} is a \LaTeXe{} package
that enables the direct compilation
of document sections included by |\include|
to individual files.
\end{abstract}

\begingroup
\parskip0ex
\tableofcontents
\endgroup

%%%%%%%%%%%%%%%%%%%%%%%%%%%%%%%%%%%%%%%%%%%%%%%%%%%%%%%%%%%%%%%%%%%%%%%%%%%%%%%%
%%%%%%%%%%%%%%%%%%%%%%%%%%%%%%%%%%%%%%%%%%%%%%%%%%%%%%%%%%%%%%%%%%%%%%%%%%%%%%%%
\section{Introduction}

\LaTeX{} provides a mechanism to structure a large document (such as a book)
into a main file and several child files (containing the chapters)
using the |\include| command.
This mechanism is beneficial for documents
which span hundreds of pages in order to
make the source file(s) more manageable.
Moreover, compilation can be restricted to
selected child files by means of the |\includeonly| command.
The latter feature can be used to reduce the compilation time while editing
(this was significantly more useful in the earlier days of \LaTeX{})
or to generate a smaller document which is easier to navigate.
Another application of |\includeonly| is to generate
documents consisting of selected parts of the complete document.

However, there are a few drawbacks of the plain |\include| mechanism:
\begin{itemize}
\item
The child files cannot be compiled on their own,
they can only be compiled via the main file.
A naive editing environment
(such as a text editor with an option
to have the current file processed by \LaTeX)
may require one to switch to the main file before compiling;
attempting to compile the child file produces errors.
\item
The main file must be modified (each time)
to adjust the |\includeonly| command
to the present needs. This easily leaves the main file in a messy state.
\item
The generated document will always carry the filename
of the main document. This is inconvenient if
several child files are to be compiled and
to be kept for distribution.
\end{itemize}

The present package provides a simple interface
to make child files individually compilable by \LaTeX{}.
Compiling a child file then has the same effect as compiling
the main file with an |\includeonly| command
to select the appropriate child.
Moreover the generated document will carry the name of the child
rather than the main file.
This resolves all three above issues.

This feature is meant to make the editing of books,
thesis documents and lecture notes somewhat more convenient.
However, the package can also be used efficiently for
composing a series of documents (such as exercise sheets)
which are typically distributed individually.
It then assists the author in generating the individual documents
(potentially in different versions)
as well as a document containing the collected series.
Another application is in developing style files
or other kinds of included material
where compilation of the style file could redirect
to a sample or test file.

%%%%%%%%%%%%%%%%%%%%%%%%%%%%%%%%%%%%%%%%%%%%%%%%%%%%%%%%%%%%%%%%%%%%%%%%%%%%%%%%
%%%%%%%%%%%%%%%%%%%%%%%%%%%%%%%%%%%%%%%%%%%%%%%%%%%%%%%%%%%%%%%%%%%%%%%%%%%%%%%%
\section{Usage}

First of all, the package \textsf{childdoc} is \emph{not} a standard
\LaTeXe{} |.sty| style file! Therefore it needs to be invoked in
a non-standard way.

%%%%%%%%%%%%%%%%%%%%%%%%%%%%%%%%%%%%%%%%%%%%%%%%%%%%%%%%%%%%%%%%%%%%%%%%%%%%%%%%
\subsection{Included Files}
\label{sec:include}

%%%%%%%%%%%%%%%%%%%%%%%%%%%%%%%%%%%%%%%%
\DescribeMacro{\childdocmain}
To use the package, add the commands
\begin{center}
\begin{tabular}{l}
|\input{childdoc.def}|\\
|\childdocmain{}|\\
\end{tabular}
\end{center}
at the very top of the main \LaTeX{} file,
in particular \emph{before} the |\documentclass| statement!
The argument of |\childdocmain| should be left empty
(but it must be present).

%%%%%%%%%%%%%%%%%%%%%%%%%%%%%%%%%%%%%%%%
\DescribeMacro{\childdocof}
Furthermore, add the commands
\begin{center}
\begin{tabular}{l}
|\input{childdoc.def}|\\
|\childdocof{|\textit{main}|}|\\
\end{tabular}
\end{center}
at the top of every child file \textit{child}
which is included by |\include{|\textit{child}|}|
from within the main file
(or at least for those files to be compiled individually).
The argument \textit{main} must be the filename of the main file.

There are a couple of
considerations in setting up the main and child documents:

%%%%%%%%%%%%%%%%%%%%%%%%%%%%%%%%%%%%%%%%
\paragraph{Restrictions.}

Please note the following restrictions:
\begin{itemize}
\item
|\childdocmain| must be called with one argument \textit{main}
to ensure compatibility with earlier version of the package.
It must either be empty (|\childdocmain{}|)
or precisely match the filename of the main file in which it is specified.
See \secref{sec:detection} for further information.
\item
The filename \textit{main} must be specified without the |.tex| extension.
\item
The filename \textit{main} is case sensitive
(even in case-insensitive file systems)
due to internal string comparison.
\item
The argument \textit{main} should be fully expanded, it cannot be a macro.
\item
Subdirectories and special characters should be avoided in filenames.
\item
The command |\childdocmain{|\textit{main}|}| must be followed by a whitespace.
It should not be followed immediately by another command
or by a comment mark `|%|'.
This is because the \TeX{} parser reads the token immediately following
the argument of |\childdocmain| and puts it
at the beginning of every child section;
however, a white\-space is ignored.
\end{itemize}

%%%%%%%%%%%%%%%%%%%%%%%%%%%%%%%%%%%%%%%%
\paragraph{Content of Main File.}

It is advisable to place all content in the child files included by |\include|.
Any output contained in the main file will appear in all child documents
unless suppressed manually;
it cannot be suppressed automatically by the |\includeonly| directive
and thus should normally be avoided.
A method to include some content in the main file
by means of conditional processing is described in \secref{sec:conditional}.

%%%%%%%%%%%%%%%%%%%%%%%%%%%%%%%%%%%%%%%%
\paragraph{Page Numbering.}

When only a part of the document is compiled,
the appropriate numbering of pages
(as well as other status parameters)
is determined from the |.aux| files.
The latter contain information from previous passes.
However this information needs to propagate through
all intermediate child documents.
Therefore the page numbering in child documents may well
be inconsistent until the complete document is compiled at least once.

A useful (if unconventional) way to always ensure a consistent
page numbering is to restart the numbering in each child document
and denote the pages by `\textit{child}|.|\textit{page}'
where \textit{child} represents the chapter/section number of the child file.
This can be achieved by the command
|\numberwithin{page}{|\textit{child}|}|
of the \textsf{amsmath} package
where \textit{child} can be |chapter| or |section|
depending on the chosen structuring.
Alternatively, one can modify the macro |\thepage| appropriately
and reset the counter |page| at the start of each child file.

%%%%%%%%%%%%%%%%%%%%%%%%%%%%%%%%%%%%%%%%%%%%%%%%%%%%%%%%%%%%%%%%%%%%%%%%%%%%%%%%
\subsection{Conditional Processing}
\label{sec:conditional}

The package provides a mechanism to compile different versions
of a document. To customise the versions further some conditional processing
can come in handy to distinguish which version is being compiled.
The package provides two macros to describe the compilation context:

%%%%%%%%%%%%%%%%%%%%%%%%%%%%%%%%%%%%%%%%
\DescribeMacro{\ifchilddoc}
The conditional |\ifchilddoc| distinguishes between the compilation of
child documents and the main document:
%
\begin{center}
|\ifchilddoc |\textit{child-code}| |[|\||else |\textit{main-code}]| \||fi|
\end{center}

%%%%%%%%%%%%%%%%%%%%%%%%%%%%%%%%%%%%%%%%
\DescribeMacro{\childdocname}
\DescribeMacro{\childdocjob}
The macro |\childdocname| contains the filename (without extension)
of the main or child file being processed.
Note that |\childdocjob| will always contain the name of the main file.

%%%%%%%%%%%%%%%%%%%%%%%%%%%%%%%%%%%%%%%%
\paragraph{Title Page.}

Conditional processing can be used to include a title or banner page
in the main document when proper precautions are taken.
Importantly, the code in the main file should ensure that the page counter
(as well as other status parameters which are stored in the |.aux| files)
takes the same value after the conditional processing.
Otherwise the page numbers may take divergent values
depending on which part is compiled.

For example, a title page could be declared by:
%
\begin{center}
\begin{tabular}{l}
|\ifchilddoc\||else|\\
|\addtocounter{page}{-1}|\\
\textit{code for title page}\\
|\newpage|\\
|\||fi|
\end{tabular}
\end{center}
%
A banner page for the child documents can be generated by:
%
\begin{center}
\begin{tabular}{l}
|\ifchilddoc|\\
|\addtocounter{page}{-1}|\\
\textit{code for banner page}\\
|\newpage|\\
|\||fi|
\end{tabular}
\end{center}
%
Here one could write a message such as:
\begin{center}
|This is the part \childdocname{} of \childdocjob{}.|
\end{center}

%%%%%%%%%%%%%%%%%%%%%%%%%%%%%%%%%%%%%%%%%%%%%%%%%%%%%%%%%%%%%%%%%%%%%%%%%%%%%%%%
\subsection{Flags}
\label{sec:flags}

The package makes it easy to generate different versions
of the main or child documents.
To this end compilation flags can be defined
and assigned different default values.
They will be particularly useful in conjunction
with the forwarding mechanism described in \secref{sec:forward}.

For example, it may be useful to have a flag |\version|
which can be set to |draft| or |final|.
The document source will contain some conditional code
depending on the value of |\version|.
Suppose further, the flag should default to |final| for the main file
and to |draft| for child files
which is a natural assignment for editing the document.
This is achieved by placing the following code
in the preamble of the main document
(below the |\childdocmain| directive):
%
\begin{center}
\begin{tabular}{l}
|\ifchilddoc|\\
|\providecommand{\version}{draft}|\\
|\||else|\\
|\providecommand{\version}{final}|\\
|\||fi|
\end{tabular}
\end{center}
%
The definition by |\providecommand| makes sure
that previous definitions are not overwritten.
Further statements |\providecommand{\version}{...}|
can thus be added before the above code to override it.

For the main file, one might add a line
(between |\childdocmain| and the above block)
%
\begin{center}
|%\ifchilddoc\||else\providecommand{\version}{draft}\||fi|
\end{center}
%
which can be uncommented to produce a draft version.
Likewise one can add a line to the very top of a child file
(above the |\childdocof{|\textit{main}|}| directive)
%
\begin{center}
|%\providecommand{\version}{final}|
\end{center}
%
which can be uncommented to produce the final version of this child document.

%%%%%%%%%%%%%%%%%%%%%%%%%%%%%%%%%%%%%%%%%%%%%%%%%%%%%%%%%%%%%%%%%%%%%%%%%%%%%%%%
\subsection{Forwarding}
\label{sec:forward}

Different versions of the main or child documents
using compilation flags as described in \secref{sec:flags}
can be (permanently) stored in different files
for convenient compilation, viewing and distribution.
To this end, the package defines a command
to pass on compilation to a different file:

%%%%%%%%%%%%%%%%%%%%%%%%%%%%%%%%%%%%%%%%
\DescribeMacro{\childdocforward}
The command |\childdocforward| redirects processing to
another source file:
%
\begin{center}
\begin{tabular}{l}
|\input{childdoc.def}|\\
|\childdocforward[|\textit{main}|]{|\textit{dest}|}|\\
\end{tabular}
\end{center}
%
The argument \textit{dest} is the destination file
(without extension).
It should be the main file or one of the child files.
Note that further \textsf{childdoc} directives
such as |\childdocof| and |\childdocforward|
in the indicated file will be processed in this form.
The optional argument \textit{main}
passes on directly to the main file \textit{main}
while pretending to compile the child \textit{dest}.
This form behaves as if \textit{dest}
issues |\childdocof{|\textit{main}|}| right away,
and no further \textsf{childdoc} directives will be processed.

%%%%%%%%%%%%%%%%%%%%%%%%%%%%%%%%%%%%%%%%
\DescribeMacro{\...prefix}
In the alternative form |\childdocforwardprefix|,
%
\begin{center}
\begin{tabular}{l}
|\input{childdoc.def}|\\
|\childdocforwardprefix[|\textit{main}|]{|\textit{prefix}|}{|\textit{dest}|}|
\end{tabular}
\end{center}
%
the destination file is determined by a pattern
depending on the current file:
To make this work, the current file must be called
`{\textit{prefix}\hspace{0.2em}\textit{suffix}}'
with \textit{prefix} matching precisely the argument.
Processing is then passed on to the file
`{\textit{dest}\hspace{0.2em}\textit{suffix}}'.
Surely, the same effect is achieved by
directly specifying the
argument `{\textit{dest}\hspace{0.2em}\textit{suffix}}'
in the first form.
However, that requires to set up a different file
for each child. With the alternative form of the command
all these files can have exactly the same content
which simplifies setting them up and maintaining them.

For example, the following file |draft.tex|
with a compilation flag |\version| as described in \secref{sec:flags}
compiles the main document as a draft:
%
\begin{center}
\begin{tabular}{l}
|\def\version{draft}|\\
|\input{childdoc.def}|\\
|\childdocforward{|\textit{main}|}|
\end{tabular}
\end{center}
%
Likewise, the following files |final|\textit{nn}|.tex|
compile the final version of the child document
|child|\textit{nn}|.tex|:
%
\begin{center}
\begin{tabular}{l}
|\def\version{final}|\\
|\input{childdoc.def}|\\
|\childdocforwardprefix{final}{child}|
\end{tabular}
\end{center}
%

Note that when several versions of a main file and/or of each child file
are to be generated, it may be convenient to set up a |Makefile| or
shell script to automatise the process.

%%%%%%%%%%%%%%%%%%%%%%%%%%%%%%%%%%%%%%%%%%%%%%%%%%%%%%%%%%%%%%%%%%%%%%%%%%%%%%%%
\subsection{Command Line Processing}
\label{sec:commandline}

The effect of redirection files can also be achieved by invoking
the \LaTeX{} compiler with a more elaborate command line.
Most conveniently this should be done as part
of a shell script or a |Makefile|.

When using \textsf{childdoc} in the main file, the following
command lines effectively perform a redirection
(note that depending on the shell being used,
backslashes may have to be doubled: `|\|' $\to$ `|\\|'):
%
\begin{center}
|... -jobname "|\textit{target}|" |\\|"|[\textit{flags}]%
|\input{childdoc.def}\childdocforward[|\textit{main}|]{|\textit{dest}|}"|
\end{center}
%
Here \textit{target} is the name of the output file,
\textit{main} is the name of the main file
and \textit{dest} is the name of the main or child file to be processed
(all filenames without extensions).
The optional argument \textit{main} can be omitted
if \textit{main} matches \textit{dest}.
Optionally, compilation \textit{flags} can be defined via |\def| commands.
This command line makes the \TeX{} engine believe
it is compiling the file \textit{target}
whose content is specified as the latter parameter.
The provided code then forwards the processing to
\textit{main} or \textit{dest} as described in \secref{sec:forward}.

%%%%%%%%%%%%%%%%%%%%%%%%%%%%%%%%%%%%%%%%%%%%%%%%%%%%%%%%%%%%%%%%%%%%%%%%%%%%%%%%
\subsection{Include by Input}
\label{sec:input}

Including child documents by |\include| has some restrictions by design.
Most notably, the content of a child document always occupies
its own set of pages; pages cannot be shared between child documents.
Usually, this behaviour makes perfect sense
because each child document contain an essential part of the document.
However, in some situations it may be desirable to compose
a document from a collection of parts
without having mandatory page breaks between then.
For this case, the package
provides a mechanism to include parts
by |\input| which can also be processed individually.
However, by construction this mechanism
requires manual handling of the content to be output.

%%%%%%%%%%%%%%%%%%%%%%%%%%%%%%%%%%%%%%%%
\DescribeMacro{\ifchilddocmanual}
The main file should be prepared as usual, see \secref{sec:include}.
However, the document body must make a distinction
between processing of an individual part and of the main document, e.g.:
%
\begin{center}
\begin{tabular}{l}
|\ifchilddocmanual|\\
|\input{\childdocname}|\\
|\||else|\\
\textit{document body with }|\input{|\textit{part}|}|\\
|\||fi|
\end{tabular}
\end{center}
%
The conditional |\ifchilddocmanual| is true whenever
a part to be included by |\input| is being compiled,
and the name of the part is stored in |\childdocname|.

%%%%%%%%%%%%%%%%%%%%%%%%%%%%%%%%%%%%%%%%
\DescribeMacro{\childdocby}
Each part to be included by |\input| should start with:
%
\begin{center}
\begin{tabular}{l}
|\input{childdoc.def}|\\
|\childdocby{|\textit{main}|}|\\
\end{tabular}
\end{center}
%
The directive |\childdocby| is similar to |\childdocof|
described in \secref{sec:include},
but the subsequent selection of content must be done manually.
To that end, both |\ifchilddoc| and |\ifchilddocmanual|
will be true upon processing of a part,
and the name of the part is stored in |\childdocname|.
Note that |\jobname| will be set to the filename of the current part
so that each part receives an individual |.aux| file
that does not interfere with the |.aux| file(s) of the main document.
This behaviour can be altered by the alternative form
|\childdocby[*]{|\textit{main}|}| (with a non-empty optional argument)
which uses the |.aux| file of the main document
by setting |\jobname| to \textit{main}.

%%%%%%%%%%%%%%%%%%%%%%%%%%%%%%%%%%%%%%%%%%%%%%%%%%%%%%%%%%%%%%%%%%%%%%%%%%%%%%%%
\subsection{Driver Development}
\label{sec:driver}

The \textsf{childdoc} mechanism can also be use for the development
of definition files such as \LaTeX{} styles or classes.
This case differs from the above setup with multiple parts
included by |\include| in that no |\includeonly| should be invoked.
This can be achieved by starting the include file
(before |\ProvidesPackage|) with:
%
\begin{center}
\begin{tabular}{l}
|\input{childdoc.def}|\\
|\childdocforward{|\textit{main}|}|\\
\end{tabular}
\end{center}
%
or alternatively with:
%
\begin{center}
\begin{tabular}{l}
|\input{childdoc.def}|\\
|\childdocby{|\textit{main}|}|\\
\end{tabular}
\end{center}
%
Both forms have slightly different effects as described above.
The main file is prepared as usual, see \secref{sec:include}.

%%%%%%%%%%%%%%%%%%%%%%%%%%%%%%%%%%%%%%%%%%%%%%%%%%%%%%%%%%%%%%%%%%%%%%%%%%%%%%%%
\subsection{Legacy Detection}
\label{sec:detection}

The directive |\childdocmain| in the main file can detect
whether the complete document or merely a child is to be compiled
even without using the directive |\childdocof|.
This method is deprecated because it is less robust
and there is no compelling reason to use it;
it is merely provided for backward compatibility
and it may be removed in future versions.

If the detection mechanism is to be used,
it is mandatory to correctly specify
the filename of the main file as the argument of |\childdocmain|:
%
\begin{center}
\begin{tabular}{l}
|\input{childdoc.def}|\\
|\childdocmain{|\textit{main}|}|\\
\end{tabular}
\end{center}
%
If |\jobname| does not match the argument \textit{main} of |\childdocmain|,
it is assumed that |\jobname| points to the child file to be compiled.
When using |\childdocmain| with the main file specified as argument,
it suffices to start a child file
with just |\input{|\textit{main}|}|
without loading of the package and using |\childdocof|.
If instead all processing is done
with the appropriate \textsf{childdoc} directives,
the argument of \textit{main} of |\childdocmain| can be empty.

An alternative version of the command line processing described
in \secref{sec:commandline} using the detection mechanism reads:
%
\begin{center}
|... -jobname "|\textit{target}|" "|[\textit{flags}]%
[|\def\jobname{|\textit{dest}|}|]|\input{|\textit{main}|}"|
\end{center}

%%%%%%%%%%%%%%%%%%%%%%%%%%%%%%%%%%%%%%%%%%%%%%%%%%%%%%%%%%%%%%%%%%%%%%%%%%%%%%%%
\subsection{Manual Code}
\label{sec:manual}

In case one cannot be certain whether the definitions file |childdoc.def|
is installed on the target \TeX{} distribution
and one prefers not to ship it,
it is conceivable to paste a few relevant commands into the sources.

To that end, drop all statements |\input{childdoc.def}|
and perform the replacements as outlined below.
Instead of |\childdocmain{|\textit{main}|}| add the following code
to the top of the main file:
%
\begin{center}
\begin{tabular}{l}
|\||ifdefined\childdocname\endinput\||fi\newif\ifchilddoc|\\
|\edef\childdocname{\scantokens\expandafter{\jobname\noexpand}}|\\
|\def\childdocmain{|\textit{main}|}\||ifx\childdocmain\childdocname\||else|\\
|\childdoctrue\includeonly{\childdocname}\let\jobname\childdocmain\||fi|\\
\end{tabular}
\end{center}
%
Instead of |\childdocof{|\textit{main}|}| just include the main file
at the top of each child file:
%
\begin{center}
|\input{|\textit{main}|}|
\end{center}
%
A simple redirection |\childdocforward{|\textit{dest}|}| is achieved by:
%
\begin{center}
|\def\jobname{|\textit{dest}|}\input{\jobname}|
\end{center}
%
The redirection with prefix
|\childdocforwardprefix[|\textit{prefix}|]{|\textit{dest}|}|
is accomplished by:
%
\begin{center}
\begin{tabular}{l}
|{\edef\jobname{\scantokens\expandafter{\jobname\noexpand}}|\\
|\def\redirectjob |\textit{prefix}|#1~~~{\gdef\jobname{|\textit{dest}|#1}}|\\
|\expandafter\redirectjob\jobname~~~}\input{\jobname}|
\end{tabular}
\end{center}

In an alternative approach,
child documents can be compiled by a specific command line
without additional code or specific definitions:
%
\begin{center}
|... -jobname "|\textit{target}|" "|[\textit{flags}]%
|\includeonly{|\textit{dest}|}\input{|\textit{main}|}"|
\end{center}
%

%%%%%%%%%%%%%%%%%%%%%%%%%%%%%%%%%%%%%%%%%%%%%%%%%%%%%%%%%%%%%%%%%%%%%%%%%%%%%%%%
%%%%%%%%%%%%%%%%%%%%%%%%%%%%%%%%%%%%%%%%%%%%%%%%%%%%%%%%%%%%%%%%%%%%%%%%%%%%%%%%
\section{Information}

%%%%%%%%%%%%%%%%%%%%%%%%%%%%%%%%%%%%%%%%%%%%%%%%%%%%%%%%%%%%%%%%%%%%%%%%%%%%%%%%
\subsection{Copyright}

Copyright \copyright{} 2017--2018 Niklas Beisert

This work may be distributed and/or modified under the
conditions of the \LaTeX{} Project Public License, either version 1.3
of this license or (at your option) any later version.
The latest version of this license is in
  \url{http://www.latex-project.org/lppl.txt}
and version 1.3 or later is part of all distributions of \LaTeX{}
version 2005/12/01 or later.

This work has the LPPL maintenance status `maintained'.

The Current Maintainer of this work is Niklas Beisert.

This work consists of the files |README.txt|, |childdoc.ins| and |childdoc.dtx|
as well as the derived files |childdoc.def|, |cdocsamp.tex|
with |cdocsch1.tex|, |cdocsch2.tex|, |cdocspt3.tex|, |cdocspt4.tex|,
|cdocsdrf.tex|, |cdocsfn1.tex|, |cdocsfn2.tex|
as well as |childdoc.pdf|.

%%%%%%%%%%%%%%%%%%%%%%%%%%%%%%%%%%%%%%%%%%%%%%%%%%%%%%%%%%%%%%%%%%%%%%%%%%%%%%%%
\subsection{Files and Installation}

The package consists of the files:
%
\begin{center}
\begin{tabular}{ll}
    |README.txt|   & readme file \\
    |childdoc.ins| & installation file \\
    |childdoc.dtx| & source file \\
    |childdoc.def| & definition file \\
    |cdocsamp.tex| & sample main file \\
    |cdocsch1.tex| & sample include file \\
    |cdocsch2.tex| & sample include file \\
    |cdocspt3.tex| & sample part file \\
    |cdocspt4.tex| & sample part file \\
    |cdocsdrf.tex| & sample redirection file \\
    |cdocsfn1.tex| & sample redirection file \\
    |cdocsfn2.tex| & sample redirection file \\
    |childdoc.pdf| & manual
\end{tabular}
\end{center}
%
The distribution consists of the files
|README.txt|, |childdoc.ins| and |childdoc.dtx|.
%
\begin{itemize}
\item
Run (pdf)\LaTeX{} on |childdoc.dtx|
to compile the manual |childdoc.pdf| (this file).
\item
Run \LaTeX{} on |childdoc.ins| to create the definitions file |childdoc.def|
and the sample |cdocsamp.tex| with include files
|cdocsch1.tex|, |cdocsch2.tex|, |cdocspt3.tex|, |cdocspt4.tex|,
|cdocsdrf.tex|, |cdocsfn1.tex|, |cdocsfn2.tex|.
Then copy the file |childdoc.def| to an appropriate directory of your \LaTeX{}
distribution, e.g.\ \textit{texmf-root}|/tex/latex/childdoc|.
\end{itemize}

%%%%%%%%%%%%%%%%%%%%%%%%%%%%%%%%%%%%%%%%%%%%%%%%%%%%%%%%%%%%%%%%%%%%%%%%%%%%%%%%
\subsection{Related CTAN Packages}

There are several other packages which offer a similar functionality:
%
\begin{itemize}
\item
The packages
\href{http://ctan.org/pkg/docmute}{\textsf{docmute}},
\href{http://ctan.org/pkg/includex}{\textsf{includex}} and
\href{http://ctan.org/pkg/standalone}{\textsf{standalone}}
provide commands to include only the document body of
a child file thus allowing both files to be compiled individually.
\item
The packages \href{http://ctan.org/pkg/subdocs}{\textsf{subdocs}}
and \href{http://ctan.org/pkg/subfiles}{\textsf{subfiles}}
provide structures in which the main and child documents can be
encapsulated and allowing them to be compiled individually.
The inclusion mechanism is different from the conventional |\include|.
\item
The package \href{http://ctan.org/pkg/combine}{\textsf{combine}}
is an elaborate solution to combine several documents into one.
\end{itemize}
%
See also the CTAN topic \href{http://ctan.org/topic/subdocs}{\textsf{subdocs}}
for further related packages.
The present package differs from the above solutions in that
a document structure constructed with the conventional |\include| mechanism
just needs two extra commands at the top of every file
such that all constituent files can be compiled individually.

%%%%%%%%%%%%%%%%%%%%%%%%%%%%%%%%%%%%%%%%%%%%%%%%%%%%%%%%%%%%%%%%%%%%%%%%%%%%%%%%
%\subsection{Feature Suggestions}
%
%The following is a list of features which may be useful for future
%versions of this package:
%%
%\begin{itemize}
%\item
%\ldots
%\end{itemize}

%%%%%%%%%%%%%%%%%%%%%%%%%%%%%%%%%%%%%%%%%%%%%%%%%%%%%%%%%%%%%%%%%%%%%%%%%%%%%%%%
\subsection{Revision History}

%%%%%%%%%%%%%%%%%%%%%%%%%%%%%%%%%%%%%%%%
\paragraph{v2.0:} 2018/12/30

\begin{itemize}
\item
immediate forward processing
\item
added |\childdocby| mechanism
\item
manual restructured
\end{itemize}

%%%%%%%%%%%%%%%%%%%%%%%%%%%%%%%%%%%%%%%%
\paragraph{v1.6:} 2018/01/17

\begin{itemize}
\item
application for development of include files
\item
corrections to manual
\end{itemize}

%%%%%%%%%%%%%%%%%%%%%%%%%%%%%%%%%%%%%%%%
\paragraph{v1.5:} 2017/05/21

\begin{itemize}
\item
more complete structuring introduced
\item
|\childdocof| introduced
\item
|\childdoc| renamed to |\childdocmain|
\item
|\childredirect| renamed to |\childdocforward| and |\childdocforwardprefix|
and functionality expanded
\end{itemize}

%%%%%%%%%%%%%%%%%%%%%%%%%%%%%%%%%%%%%%%%
\paragraph{v1.0:} 2017/04/27

\begin{itemize}
\item
manual and install package
\item
first version published on CTAN
\end{itemize}

%%%%%%%%%%%%%%%%%%%%%%%%%%%%%%%%%%%%%%%%
\paragraph{v0.6:} 2017/04/26

\begin{itemize}
\item
redirection mechanism added
\end{itemize}

%%%%%%%%%%%%%%%%%%%%%%%%%%%%%%%%%%%%%%%%
\paragraph{v0.5:} 2017/04/26

\begin{itemize}
\item
functionality in definition file
\end{itemize}


%%%%%%%%%%%%%%%%%%%%%%%%%%%%%%%%%%%%%%%%%%%%%%%%%%%%%%%%%%%%%%%%%%%%%%%%%%%%%%%%
%%%%%%%%%%%%%%%%%%%%%%%%%%%%%%%%%%%%%%%%%%%%%%%%%%%%%%%%%%%%%%%%%%%%%%%%%%%%%%%%
%%%%%%%%%%%%%%%%%%%%%%%%%%%%%%%%%%%%%%%%%%%%%%%%%%%%%%%%%%%%%%%%%%%%%%%%%%%%%%%%
\appendix

\settowidth\MacroIndent{\rmfamily\scriptsize 000\ }

 \DocInput{childdoc.dtx}

\end{document}
%</driver>
% \fi
%
% %%%%%%%%%%%%%%%%%%%%%%%%%%%%%%%%%%%%%%%%%%%%%%%%%%%%%%%%%%%%%%%%%%%%%%%%%%%%%%
% %%%%%%%%%%%%%%%%%%%%%%%%%%%%%%%%%%%%%%%%%%%%%%%%%%%%%%%%%%%%%%%%%%%%%%%%%%%%%%
% \section{Sample}
%\iffalse
%<*samplemain>
%\fi
%
% The following presents a sample document
% with two chapters, two parts, a title page,
% a compile flag as well as three forwarding files to set the flag.
% It consists of eight |.tex| files:
% \begin{center}
% \begin{tabular}{ll}
% |cdocsamp.tex|&main file\\
% |cdocsch1.tex|&include file for chapter 1\\
% |cdocsch2.tex|&include file for chapter 2\\
% |cdocspt3.tex|&include file for part 3\\
% |cdocspt4.tex|&include file for part 4\\
% |cdocsdrf.tex|&forwarding file for main file in draft mode\\
% |cdocsfi1.tex|&forwarding file for final version of chapter 1\\
% |cdocsfi2.tex|&forwarding file for final version of chapter 2\\
% \end{tabular}
% \end{center}
% Each of the eight files can be compiled directly by the \LaTeX{} compiler.
%
% %%%%%%%%%%%%%%%%%%%%%%%%%%%%%%%%%%%%%%
% \paragraph{Main File.}
%
% The main file is called |cdocsamp.tex|.
%
% Load the \textsf{childdoc} definitions and
% declare the filename for the main document:
%    \begin{macrocode}
\input{childdoc.def}
\childdocmain{}
%    \end{macrocode}

% Optional override for |\version| flag:
%    \begin{macrocode}
%%\ifchilddoc\else\providecommand{\version}{draft}\fi
%    \end{macrocode}

% Define the default values for the |\version| flag
% (|final| for the main file and |draft| for childs):
%    \begin{macrocode}
\ifchilddoc
\providecommand{\version}{draft}
\else
\providecommand{\version}{final}
\fi
%    \end{macrocode}

% Load the standard document class:
%    \begin{macrocode}
\documentclass[12pt]{article}
%    \end{macrocode}

% Start the document body:
%    \begin{macrocode}
\begin{document}
%    \end{macrocode}

% Declare a title page.
% Print title, part of document being processed and version flag:
%    \begin{macrocode}
\addtocounter{page}{-1}
\begin{center}
{\LARGE\bfseries{}childdoc example\par}
\vspace{1cm}
\ifchilddoc
\ifchilddocmanual part\else chapter\fi:
`\childdocname' of `\childdocjob'\par
\else
main document: `\childdocjob'\par
\fi
version: \version\par
\end{center}
\newpage
%    \end{macrocode}

% Manually include selected file,
% otherwise process as usual:
%    \begin{macrocode}
\ifchilddocmanual
\section*{part `\childdocname'}
\input{\childdocname}
\else
%    \end{macrocode}

% Include the two chapters:
%    \begin{macrocode}
\include{cdocsch1}
\include{cdocsch2}
%    \end{macrocode}

% Include the two parts unless only chapters should be displayed:
%    \begin{macrocode}
\ifchilddoc\else
\section{part three}
\input{cdocspt3}
\section{part four}
\input{cdocspt4}
\fi
%    \end{macrocode}

% Process as usual until here:
%    \begin{macrocode}
\fi
%    \end{macrocode}

% End of document body:
%    \begin{macrocode}
\end{document}
%    \end{macrocode}
%\iffalse
%</samplemain>
%\fi
%
% %%%%%%%%%%%%%%%%%%%%%%%%%%%%%%%%%%%%%%
% \paragraph{Chapter Include Files.}
%
% The include files are called |cdocsch1.tex| and |cdocsch2.tex|.
%
%\iffalse
%<*samplechap1|samplechap2>
%\fi

% Optional override for |\version| flag:
%    \begin{macrocode}
%%\providecommand{\version}{final}
%    \end{macrocode}

% Include the main document:
%    \begin{macrocode}
\input{childdoc.def}
\childdocof{cdocsamp}
%    \end{macrocode}

%\iffalse
%</samplechap1|samplechap2>
%\fi
%
%\iffalse
%<*samplechap1>
%\fi
% Some text for chapter 1:
%    \begin{macrocode}
\section{one}
some text in chapter one
%    \end{macrocode}

%\iffalse
%</samplechap1>
%\fi
% Some text for chapter 2:
%\iffalse
%<*samplechap2>
%\fi
%    \begin{macrocode}
\section{two}
more text in chapter two
%    \end{macrocode}

%\iffalse
%</samplechap2>
%\fi
%
% %%%%%%%%%%%%%%%%%%%%%%%%%%%%%%%%%%%%%%
% \paragraph{Part Include Files.}
%
% The include files are called |cdocspt3.tex| and |cdocspt4.tex|.
%
%\iffalse
%<*samplepart3|samplepart4>
%\fi

% Optional override for |\version| flag:
%    \begin{macrocode}
%%\providecommand{\version}{final}
%    \end{macrocode}

% Include the main document:
%    \begin{macrocode}
\input{childdoc.def}
\childdocby{cdocsamp}
%    \end{macrocode}

%\iffalse
%</samplepart3|samplepart4>
%\fi
%
%\iffalse
%<*samplepart3>
%\fi
% Some text for part 3:
%    \begin{macrocode}
some text in part three
%    \end{macrocode}

%\iffalse
%</samplepart3>
%\fi
% Some text for part 4:
%\iffalse
%<*samplepart4>
%\fi
%    \begin{macrocode}
more text in part four
%    \end{macrocode}

%\iffalse
%</samplepart4>
%\fi
%
% %%%%%%%%%%%%%%%%%%%%%%%%%%%%%%%%%%%%%%
% \paragraph{Forwarding for a Complete Draft.}
%
% The following forwarding file |cdocsdrf.tex|
% compiles the main document in draft mode:
%\iffalse
%<*sampledraft>
%\fi
%    \begin{macrocode}
\def\version{draft}
\input{childdoc.def}
\childdocforward{cdocsamp}
%    \end{macrocode}

%\iffalse
%</sampledraft>
%\fi
%
% %%%%%%%%%%%%%%%%%%%%%%%%%%%%%%%%%%%%%%
% \paragraph{Forwarding for Final Version of the Chapters.}
%
% The following forwarding files |cdocsfn1.tex| and |cdocsfn2.tex|
% (with identical content)
% compile the final versions of the child documents
% |cdocsch1.tex| and |cdocsch2.tex|, respectively:
%\iffalse
%<*samplefinal>
%\fi
%    \begin{macrocode}
\def\version{final}
\input{childdoc.def}
\childdocforwardprefix[cdocsamp]{cdocsfn}{cdocsch}
%    \end{macrocode}

%\iffalse
%</samplefinal>
%\fi
%
% %%%%%%%%%%%%%%%%%%%%%%%%%%%%%%%%%%%%%%
% \paragraph{Command Line Processing.}
%
% The following three command lines generate the output files
% |cdocscld|, |cdocscl1| and |cdocscl2|
% which should be identical to
% |cdocsdrf|, |cdocsch1| and |cdocsfn2|, respectively:
% \begin{center}
% \begin{tabular}{l}
% |latex -jobname cdocscld \|\\
% |  "\def\version{draft}\input{childdoc.def}\childdocforward{cdocsamp}"|\\
% |latex -jobname cdocscl1 \|\\
% |  "\input{childdoc.def}\childdocforward[cdocsamp]{cdocsch1}"|\\
% |latex -jobname cdocscl2 \|\\
% |  "\def\version{final}\input{childdoc.def}\childdocforward{cdocsch2}"|
% \end{tabular}
% \end{center}
% Note that the trailing backslash on each first line
% merely continues the input to the second line
% (for convenient cut ant paste).
% Furthermore, the command |latex| can be replaced by any
% of its alternative versions such as |pdflatex|.
%
% %%%%%%%%%%%%%%%%%%%%%%%%%%%%%%%%%%%%%%%%%%%%%%%%%%%%%%%%%%%%%%%%%%%%%%%%%%%%%%
% %%%%%%%%%%%%%%%%%%%%%%%%%%%%%%%%%%%%%%%%%%%%%%%%%%%%%%%%%%%%%%%%%%%%%%%%%%%%%%
% \section{Implementation}
%\iffalse
%<*package>
%\fi
%
% This section describes the definitions file |childdoc.def|.

% The definitions cannot be loaded using |\usepackage| or |\RequirePackage|
% which has a mechanism to prevent loading a style file more than once.
% When loading the definitions by means of |\input|
% multiple instances have to be prevented manually:
%\iffalse
%This code needs to be before the `\ProvidesFile' directive
%which is defined at the beginning of this file.
%Therefore it is also placed there and commented out here.
%</package>
%<*discard>
%\fi
%    \begin{macrocode}
\ifdefined\childdocmain\endinput\fi
%    \end{macrocode}
%\iffalse
%</discard>
%<*package>
%\fi
%
% \macro{\ifchilddoc}
% \macro{\ifchilddocmanual}
% The conditional |\ifchilddoc| tells whether a
% child (true) or main (false) document is being compiled.
% The conditional |\ifchilddocmanual| tells whether
% the |\includeonly| mechanism is used (false) or
% the selection of child files must be performed manually (true).
% The definitions initialise to false:
%    \begin{macrocode}
\newif\ifchilddoc
\newif\ifchilddocmanual
%    \end{macrocode}

% \macro{\childdocname}
% \macro{\childdocjob}
% The macro |\childdocname| stores the name of the main document
% to be compiled. The macro |\childdocjob| stores the name of
% the document on which the \LaTeX{} compiler was originally invoked.
% The content of |\jobname| cannot be compared
% to filenames specified in the source due to different catcodes.
% The following code rescans |\jobname|, stores the result
% in |\childdocname| and saves a copy in |\childdocjob|:
%    \begin{macrocode}
\edef\childdocname{\scantokens\expandafter{\jobname\noexpand}}
\let\childdocjob\childdocname
%    \end{macrocode}

% \macro{\childdocdisable}
% The macro |\childdocdisable| prevents the main file
% from being processed more than once.
% At this stage, the main document command |\childdocmain|
% is assumed to be called once again where it should do nothing.
% Any subsequent call to it should prevent
% a secondary processing of the main document
% It overwrites the forwarding commands
% |\childdocof| and |\childdocforward|
% with empty macros to prevent further inclusions of the main document:
%    \begin{macrocode}
\newcommand{\childdocdisable}
{
  \renewcommand{\childdocmain}[1]{\renewcommand{\childdocmain}[1]{\endinput}}
  \renewcommand{\childdocof}[1]{}
  \renewcommand{\childdocby}[2][]{}
  \renewcommand{\childdocforward}[2][]{}
  \renewcommand{\childdocdisable}{}
}
%    \end{macrocode}

% \macro{\childdocmain}
% The macro |\childdocmain| is to be called at the top of the main file
% with nothing or the main filename (without extension) as argument.
% First, it breaks loops.
% If the argument is not empty and does not match |\childdocname|
% (which is set by the first inclusion of |childdoc.def|),
% |\ifchilddoc| is set to true, |\includeonly| is applied to the child file
% and |\jobname| is set to the main file
% (for proper handling of |.aux| files):
%    \begin{macrocode}
\newcommand{\childdocmain}[1]
{
  \childdocdisable\childdocmain{}
  \if?#1?\else
    \begingroup
      \def\childdoctmp{#1}
      \ifx\childdoctmp\childdocname
        \def\childdoctmp{}
      \else
        \def\childdoctmp
        {
          \childdoctrue
          \includeonly{\childdocname}
          \def\childdocjob{#1}
          \def\jobname{#1}
        }
      \fi
      \expandafter
    \endgroup
    \childdoctmp
  \fi
}
%    \end{macrocode}

% \macro{\childdocof}
% The command |\childdocof| redirects
% compilation to the main file |#1|.
%    \begin{macrocode}
\newcommand{\childdocof}[1]
{
  \childdocdisable
  \childdoctrue
  \includeonly{\childdocname}
  \def\jobname{#1}
  \def\childdocjob{#1}
  \input{#1}
}
%    \end{macrocode}

% \macro{\childdocby}
% The command |\childdocby| ....
%    \begin{macrocode}
\newcommand{\childdocby}[2][]
{
  \childdocdisable
  \childdoctrue
  \childdocmanualtrue
  \if?#1?\else
    \def\jobname{#2}
  \fi
  \def\childdocjob{#2}
  \input{#2}
  \endinput
}
%    \end{macrocode}

% \macro{\childdocforward}
% The command |\childdocforward| redirects
% compilation to the main file or
% (if the optional argument is given) a child file.
% Parameters are set as if the main file
% or a child file starting with |\childdocof| was compiled.
% Then compilation is handed over to the main file:
%    \begin{macrocode}
\newcommand{\childdocforward}[2][]
{
  \begingroup
    \if?#1?
      \def\childdoctmp
      {
        \def\childdocname{#2}
        \def\childdocjob{#2}
        \def\jobname{#2}
        \input{#2}
        \endinput
      }
    \else
      \def\childdoctmp
      {
        \childdocdisable
        \def\childdocname{#2}
        \childdoctrue
        \includeonly{#2}
        \def\childdocjob{#1}
        \def\jobname{#1}
        \input{#1}
        \endinput
      }
    \fi
    \expandafter
  \endgroup
  \childdoctmp
}
%    \end{macrocode}

% \macro{\childdocforwardprefix}
% The command |\childdocforwardprefix| redirects
% compilation to the main or a child file by means of a pattern.
% The prefix |#1| in the current filename is replaced by |#2|
% and the suffix of the current filename is kept
% (it is assumed that the filename does not contain the substring `|~~~|'
% which is used as a delimiter).
% Compilation is handed over to the new file by |\childdocforward|:
%    \begin{macrocode}
\newcommand{\childdocforwardprefix}[3][]
{
  \begingroup
    \def\childdocextract #2##1~~~{\def\childdoctmp{\childdocforward[#1]{#3##1}}}
    \expandafter\childdocextract\childdocname~~~
    \expandafter
  \endgroup
  \childdoctmp
}
%    \end{macrocode}

% \macro{\childdoc}
% The deprecated macro |\childdoc| is a legacy version of |\childdocmain|:
%    \begin{macrocode}
\newcommand{\childdoc}{\childdocmain}
%    \end{macrocode}

% \macro{\childdocredirect}
% The deprecated macro |\childdocredirect| is a legacy version
% of |\childdocforward| and |\childdocforwardprefix|:
%    \begin{macrocode}
\newcommand{\childdocredirect}[2][]
{
  \begingroup
    \if?#1?
      \def\childdoctmp{\childdocforward{#2}}
    \else
      \def\childdoctmp{\childdocforwardprefix{#1}{#2}}
    \fi
    \expandafter
  \endgroup
  \childdoctmp
}
%    \end{macrocode}

%\iffalse
%</package>
%\fi
%
\endinput
\childdocforward{cdocsch2}"|
% \end{tabular}
% \end{center}
% Note that the trailing backslash on each first line
% merely continues the input to the second line
% (for convenient cut ant paste).
% Furthermore, the command |latex| can be replaced by any
% of its alternative versions such as |pdflatex|.
%
% %%%%%%%%%%%%%%%%%%%%%%%%%%%%%%%%%%%%%%%%%%%%%%%%%%%%%%%%%%%%%%%%%%%%%%%%%%%%%%
% %%%%%%%%%%%%%%%%%%%%%%%%%%%%%%%%%%%%%%%%%%%%%%%%%%%%%%%%%%%%%%%%%%%%%%%%%%%%%%
% \section{Implementation}
%\iffalse
%<*package>
%\fi
%
% This section describes the definitions file |childdoc.def|.

% The definitions cannot be loaded using |\usepackage| or |\RequirePackage|
% which has a mechanism to prevent loading a style file more than once.
% When loading the definitions by means of |\input|
% multiple instances have to be prevented manually:
%\iffalse
%This code needs to be before the `\ProvidesFile' directive
%which is defined at the beginning of this file.
%Therefore it is also placed there and commented out here.
%</package>
%<*discard>
%\fi
%    \begin{macrocode}
\ifdefined\childdocmain\endinput\fi
%    \end{macrocode}
%\iffalse
%</discard>
%<*package>
%\fi
%
% \macro{\ifchilddoc}
% \macro{\ifchilddocmanual}
% The conditional |\ifchilddoc| tells whether a
% child (true) or main (false) document is being compiled.
% The conditional |\ifchilddocmanual| tells whether
% the |\includeonly| mechanism is used (false) or
% the selection of child files must be performed manually (true).
% The definitions initialise to false:
%    \begin{macrocode}
\newif\ifchilddoc
\newif\ifchilddocmanual
%    \end{macrocode}

% \macro{\childdocname}
% \macro{\childdocjob}
% The macro |\childdocname| stores the name of the main document
% to be compiled. The macro |\childdocjob| stores the name of
% the document on which the \LaTeX{} compiler was originally invoked.
% The content of |\jobname| cannot be compared
% to filenames specified in the source due to different catcodes.
% The following code rescans |\jobname|, stores the result
% in |\childdocname| and saves a copy in |\childdocjob|:
%    \begin{macrocode}
\edef\childdocname{\scantokens\expandafter{\jobname\noexpand}}
\let\childdocjob\childdocname
%    \end{macrocode}

% \macro{\childdocdisable}
% The macro |\childdocdisable| prevents the main file
% from being processed more than once.
% At this stage, the main document command |\childdocmain|
% is assumed to be called once again where it should do nothing.
% Any subsequent call to it should prevent
% a secondary processing of the main document
% It overwrites the forwarding commands
% |\childdocof| and |\childdocforward|
% with empty macros to prevent further inclusions of the main document:
%    \begin{macrocode}
\newcommand{\childdocdisable}
{
  \renewcommand{\childdocmain}[1]{\renewcommand{\childdocmain}[1]{\endinput}}
  \renewcommand{\childdocof}[1]{}
  \renewcommand{\childdocby}[2][]{}
  \renewcommand{\childdocforward}[2][]{}
  \renewcommand{\childdocdisable}{}
}
%    \end{macrocode}

% \macro{\childdocmain}
% The macro |\childdocmain| is to be called at the top of the main file
% with nothing or the main filename (without extension) as argument.
% First, it breaks loops.
% If the argument is not empty and does not match |\childdocname|
% (which is set by the first inclusion of |childdoc.def|),
% |\ifchilddoc| is set to true, |\includeonly| is applied to the child file
% and |\jobname| is set to the main file
% (for proper handling of |.aux| files):
%    \begin{macrocode}
\newcommand{\childdocmain}[1]
{
  \childdocdisable\childdocmain{}
  \if?#1?\else
    \begingroup
      \def\childdoctmp{#1}
      \ifx\childdoctmp\childdocname
        \def\childdoctmp{}
      \else
        \def\childdoctmp
        {
          \childdoctrue
          \includeonly{\childdocname}
          \def\childdocjob{#1}
          \def\jobname{#1}
        }
      \fi
      \expandafter
    \endgroup
    \childdoctmp
  \fi
}
%    \end{macrocode}

% \macro{\childdocof}
% The command |\childdocof| redirects
% compilation to the main file |#1|.
%    \begin{macrocode}
\newcommand{\childdocof}[1]
{
  \childdocdisable
  \childdoctrue
  \includeonly{\childdocname}
  \def\jobname{#1}
  \def\childdocjob{#1}
  \input{#1}
}
%    \end{macrocode}

% \macro{\childdocby}
% The command |\childdocby| ....
%    \begin{macrocode}
\newcommand{\childdocby}[2][]
{
  \childdocdisable
  \childdoctrue
  \childdocmanualtrue
  \if?#1?\else
    \def\jobname{#2}
  \fi
  \def\childdocjob{#2}
  \input{#2}
  \endinput
}
%    \end{macrocode}

% \macro{\childdocforward}
% The command |\childdocforward| redirects
% compilation to the main file or
% (if the optional argument is given) a child file.
% Parameters are set as if the main file
% or a child file starting with |\childdocof| was compiled.
% Then compilation is handed over to the main file:
%    \begin{macrocode}
\newcommand{\childdocforward}[2][]
{
  \begingroup
    \if?#1?
      \def\childdoctmp
      {
        \def\childdocname{#2}
        \def\childdocjob{#2}
        \def\jobname{#2}
        \input{#2}
        \endinput
      }
    \else
      \def\childdoctmp
      {
        \childdocdisable
        \def\childdocname{#2}
        \childdoctrue
        \includeonly{#2}
        \def\childdocjob{#1}
        \def\jobname{#1}
        \input{#1}
        \endinput
      }
    \fi
    \expandafter
  \endgroup
  \childdoctmp
}
%    \end{macrocode}

% \macro{\childdocforwardprefix}
% The command |\childdocforwardprefix| redirects
% compilation to the main or a child file by means of a pattern.
% The prefix |#1| in the current filename is replaced by |#2|
% and the suffix of the current filename is kept
% (it is assumed that the filename does not contain the substring `|~~~|'
% which is used as a delimiter).
% Compilation is handed over to the new file by |\childdocforward|:
%    \begin{macrocode}
\newcommand{\childdocforwardprefix}[3][]
{
  \begingroup
    \def\childdocextract #2##1~~~{\def\childdoctmp{\childdocforward[#1]{#3##1}}}
    \expandafter\childdocextract\childdocname~~~
    \expandafter
  \endgroup
  \childdoctmp
}
%    \end{macrocode}

% \macro{\childdoc}
% The deprecated macro |\childdoc| is a legacy version of |\childdocmain|:
%    \begin{macrocode}
\newcommand{\childdoc}{\childdocmain}
%    \end{macrocode}

% \macro{\childdocredirect}
% The deprecated macro |\childdocredirect| is a legacy version
% of |\childdocforward| and |\childdocforwardprefix|:
%    \begin{macrocode}
\newcommand{\childdocredirect}[2][]
{
  \begingroup
    \if?#1?
      \def\childdoctmp{\childdocforward{#2}}
    \else
      \def\childdoctmp{\childdocforwardprefix{#1}{#2}}
    \fi
    \expandafter
  \endgroup
  \childdoctmp
}
%    \end{macrocode}

%\iffalse
%</package>
%\fi
%
\endinput
\childdocforward[|\textit{main}|]{|\textit{dest}|}"|
\end{center}
%
Here \textit{target} is the name of the output file,
\textit{main} is the name of the main file
and \textit{dest} is the name of the main or child file to be processed
(all filenames without extensions).
The optional argument \textit{main} can be omitted
if \textit{main} matches \textit{dest}.
Optionally, compilation \textit{flags} can be defined via |\def| commands.
This command line makes the \TeX{} engine believe
it is compiling the file \textit{target}
whose content is specified as the latter parameter.
The provided code then forwards the processing to
\textit{main} or \textit{dest} as described in \secref{sec:forward}.

%%%%%%%%%%%%%%%%%%%%%%%%%%%%%%%%%%%%%%%%%%%%%%%%%%%%%%%%%%%%%%%%%%%%%%%%%%%%%%%%
\subsection{Include by Input}
\label{sec:input}

Including child documents by |\include| has some restrictions by design.
Most notably, the content of a child document always occupies
its own set of pages; pages cannot be shared between child documents.
Usually, this behaviour makes perfect sense
because each child document contain an essential part of the document.
However, in some situations it may be desirable to compose
a document from a collection of parts
without having mandatory page breaks between then.
For this case, the package
provides a mechanism to include parts
by |\input| which can also be processed individually.
However, by construction this mechanism
requires manual handling of the content to be output.

%%%%%%%%%%%%%%%%%%%%%%%%%%%%%%%%%%%%%%%%
\DescribeMacro{\ifchilddocmanual}
The main file should be prepared as usual, see \secref{sec:include}.
However, the document body must make a distinction
between processing of an individual part and of the main document, e.g.:
%
\begin{center}
\begin{tabular}{l}
|\ifchilddocmanual|\\
|\input{\childdocname}|\\
|\||else|\\
\textit{document body with }|\input{|\textit{part}|}|\\
|\||fi|
\end{tabular}
\end{center}
%
The conditional |\ifchilddocmanual| is true whenever
a part to be included by |\input| is being compiled,
and the name of the part is stored in |\childdocname|.

%%%%%%%%%%%%%%%%%%%%%%%%%%%%%%%%%%%%%%%%
\DescribeMacro{\childdocby}
Each part to be included by |\input| should start with:
%
\begin{center}
\begin{tabular}{l}
|% \iffalse
%
% childdoc.dtx Copyright (C) 2017-2018 Niklas Beisert
%
% This work may be distributed and/or modified under the
% conditions of the LaTeX Project Public License, either version 1.3
% of this license or (at your option) any later version.
% The latest version of this license is in
%   http://www.latex-project.org/lppl.txt
% and version 1.3 or later is part of all distributions of LaTeX
% version 2005/12/01 or later.
%
% This work has the LPPL maintenance status `maintained'.
%
% The Current Maintainer of this work is Niklas Beisert.
%
% This work consists of the files childdoc.dtx and childdoc.ins
% and the derived files childdoc.def and cdocsamp.tex with
% cdocsch1.tex, cdocsch2.tex, cdocsdrf.tex, cdocsfn1.tex, cdocsfn2.tex.
%
%<package>\ifdefined\childdocmain\endinput\fi
%<package>\ProvidesFile{childdoc.def}[2018/12/30 v2.0 child document driver]
%<samplemain>\ProvidesFile{cdocsamp.tex}[2018/12/30 v2.0 sample for childdoc]
%<*driver>
%\ProvidesFile{childdoc.drv}[2018/12/30 v2.0 childdoc reference manual file]
\PassOptionsToClass{10pt,a4paper}{article}
\documentclass{ltxdoc}

\usepackage[margin=35mm]{geometry}
\usepackage{hyperref}
\usepackage{hyperxmp}
\usepackage[usenames]{color}

\hypersetup{colorlinks=true}
\hypersetup{pdfstartview=FitH}
\hypersetup{pdfpagemode=UseNone}
\hypersetup{pdfsource={}}
\hypersetup{pdflang={en-UK}}
\hypersetup{pdfcopyright={Copyright 2017-2018 Niklas Beisert.
  This work may be distributed and/or modified under the
  conditions of the LaTeX Project Public License, either version 1.3
  of this license or (at your option) any later version.}}
\hypersetup{pdflicenseurl={http://www.latex-project.org/lppl.txt}}
\hypersetup{pdfcontactaddress={ETH Zurich, ITP, HIT K,
  Wolfgang-Pauli-Strasse 27}}
\hypersetup{pdfcontactpostcode={8093}}
\hypersetup{pdfcontactcity={Zurich}}
\hypersetup{pdfcontactcountry={Switzerland}}
\hypersetup{pdfcontactemail={nbeisert@itp.phys.ethz.ch}}
\hypersetup{pdfcontacturl={http://people.phys.ethz.ch/\xmptilde nbeisert/}}

\newcommand{\secref}[1]{\hyperref[#1]{section \ref*{#1}}}

\parskip1ex
\parindent0pt
\let\olditemize\itemize
\def\itemize{\olditemize\parskip0pt}

\begin{document}

\title{The \textsf{childdoc} Package}
\hypersetup{pdftitle={The childdoc Package}}
\author{Niklas Beisert\\[2ex]
  Institut f\"ur Theoretische Physik\\
  Eidgen\"ossische Technische Hochschule Z\"urich\\
  Wolfgang-Pauli-Strasse 27, 8093 Z\"urich, Switzerland\\[1ex]
  \href{mailto:nbeisert@itp.phys.ethz.ch}
  {\texttt{nbeisert@itp.phys.ethz.ch}}}
\hypersetup{pdfauthor={Niklas Beisert}}
\hypersetup{pdfsubject={Manual for the LaTeX2e Package childdoc}}
\date{30 December 2018, \textsf{v2.0}}
\maketitle

\begin{abstract}\noindent
\textsf{childdoc} is a \LaTeXe{} package
that enables the direct compilation
of document sections included by |\include|
to individual files.
\end{abstract}

\begingroup
\parskip0ex
\tableofcontents
\endgroup

%%%%%%%%%%%%%%%%%%%%%%%%%%%%%%%%%%%%%%%%%%%%%%%%%%%%%%%%%%%%%%%%%%%%%%%%%%%%%%%%
%%%%%%%%%%%%%%%%%%%%%%%%%%%%%%%%%%%%%%%%%%%%%%%%%%%%%%%%%%%%%%%%%%%%%%%%%%%%%%%%
\section{Introduction}

\LaTeX{} provides a mechanism to structure a large document (such as a book)
into a main file and several child files (containing the chapters)
using the |\include| command.
This mechanism is beneficial for documents
which span hundreds of pages in order to
make the source file(s) more manageable.
Moreover, compilation can be restricted to
selected child files by means of the |\includeonly| command.
The latter feature can be used to reduce the compilation time while editing
(this was significantly more useful in the earlier days of \LaTeX{})
or to generate a smaller document which is easier to navigate.
Another application of |\includeonly| is to generate
documents consisting of selected parts of the complete document.

However, there are a few drawbacks of the plain |\include| mechanism:
\begin{itemize}
\item
The child files cannot be compiled on their own,
they can only be compiled via the main file.
A naive editing environment
(such as a text editor with an option
to have the current file processed by \LaTeX)
may require one to switch to the main file before compiling;
attempting to compile the child file produces errors.
\item
The main file must be modified (each time)
to adjust the |\includeonly| command
to the present needs. This easily leaves the main file in a messy state.
\item
The generated document will always carry the filename
of the main document. This is inconvenient if
several child files are to be compiled and
to be kept for distribution.
\end{itemize}

The present package provides a simple interface
to make child files individually compilable by \LaTeX{}.
Compiling a child file then has the same effect as compiling
the main file with an |\includeonly| command
to select the appropriate child.
Moreover the generated document will carry the name of the child
rather than the main file.
This resolves all three above issues.

This feature is meant to make the editing of books,
thesis documents and lecture notes somewhat more convenient.
However, the package can also be used efficiently for
composing a series of documents (such as exercise sheets)
which are typically distributed individually.
It then assists the author in generating the individual documents
(potentially in different versions)
as well as a document containing the collected series.
Another application is in developing style files
or other kinds of included material
where compilation of the style file could redirect
to a sample or test file.

%%%%%%%%%%%%%%%%%%%%%%%%%%%%%%%%%%%%%%%%%%%%%%%%%%%%%%%%%%%%%%%%%%%%%%%%%%%%%%%%
%%%%%%%%%%%%%%%%%%%%%%%%%%%%%%%%%%%%%%%%%%%%%%%%%%%%%%%%%%%%%%%%%%%%%%%%%%%%%%%%
\section{Usage}

First of all, the package \textsf{childdoc} is \emph{not} a standard
\LaTeXe{} |.sty| style file! Therefore it needs to be invoked in
a non-standard way.

%%%%%%%%%%%%%%%%%%%%%%%%%%%%%%%%%%%%%%%%%%%%%%%%%%%%%%%%%%%%%%%%%%%%%%%%%%%%%%%%
\subsection{Included Files}
\label{sec:include}

%%%%%%%%%%%%%%%%%%%%%%%%%%%%%%%%%%%%%%%%
\DescribeMacro{\childdocmain}
To use the package, add the commands
\begin{center}
\begin{tabular}{l}
|% \iffalse
%
% childdoc.dtx Copyright (C) 2017-2018 Niklas Beisert
%
% This work may be distributed and/or modified under the
% conditions of the LaTeX Project Public License, either version 1.3
% of this license or (at your option) any later version.
% The latest version of this license is in
%   http://www.latex-project.org/lppl.txt
% and version 1.3 or later is part of all distributions of LaTeX
% version 2005/12/01 or later.
%
% This work has the LPPL maintenance status `maintained'.
%
% The Current Maintainer of this work is Niklas Beisert.
%
% This work consists of the files childdoc.dtx and childdoc.ins
% and the derived files childdoc.def and cdocsamp.tex with
% cdocsch1.tex, cdocsch2.tex, cdocsdrf.tex, cdocsfn1.tex, cdocsfn2.tex.
%
%<package>\ifdefined\childdocmain\endinput\fi
%<package>\ProvidesFile{childdoc.def}[2018/12/30 v2.0 child document driver]
%<samplemain>\ProvidesFile{cdocsamp.tex}[2018/12/30 v2.0 sample for childdoc]
%<*driver>
%\ProvidesFile{childdoc.drv}[2018/12/30 v2.0 childdoc reference manual file]
\PassOptionsToClass{10pt,a4paper}{article}
\documentclass{ltxdoc}

\usepackage[margin=35mm]{geometry}
\usepackage{hyperref}
\usepackage{hyperxmp}
\usepackage[usenames]{color}

\hypersetup{colorlinks=true}
\hypersetup{pdfstartview=FitH}
\hypersetup{pdfpagemode=UseNone}
\hypersetup{pdfsource={}}
\hypersetup{pdflang={en-UK}}
\hypersetup{pdfcopyright={Copyright 2017-2018 Niklas Beisert.
  This work may be distributed and/or modified under the
  conditions of the LaTeX Project Public License, either version 1.3
  of this license or (at your option) any later version.}}
\hypersetup{pdflicenseurl={http://www.latex-project.org/lppl.txt}}
\hypersetup{pdfcontactaddress={ETH Zurich, ITP, HIT K,
  Wolfgang-Pauli-Strasse 27}}
\hypersetup{pdfcontactpostcode={8093}}
\hypersetup{pdfcontactcity={Zurich}}
\hypersetup{pdfcontactcountry={Switzerland}}
\hypersetup{pdfcontactemail={nbeisert@itp.phys.ethz.ch}}
\hypersetup{pdfcontacturl={http://people.phys.ethz.ch/\xmptilde nbeisert/}}

\newcommand{\secref}[1]{\hyperref[#1]{section \ref*{#1}}}

\parskip1ex
\parindent0pt
\let\olditemize\itemize
\def\itemize{\olditemize\parskip0pt}

\begin{document}

\title{The \textsf{childdoc} Package}
\hypersetup{pdftitle={The childdoc Package}}
\author{Niklas Beisert\\[2ex]
  Institut f\"ur Theoretische Physik\\
  Eidgen\"ossische Technische Hochschule Z\"urich\\
  Wolfgang-Pauli-Strasse 27, 8093 Z\"urich, Switzerland\\[1ex]
  \href{mailto:nbeisert@itp.phys.ethz.ch}
  {\texttt{nbeisert@itp.phys.ethz.ch}}}
\hypersetup{pdfauthor={Niklas Beisert}}
\hypersetup{pdfsubject={Manual for the LaTeX2e Package childdoc}}
\date{30 December 2018, \textsf{v2.0}}
\maketitle

\begin{abstract}\noindent
\textsf{childdoc} is a \LaTeXe{} package
that enables the direct compilation
of document sections included by |\include|
to individual files.
\end{abstract}

\begingroup
\parskip0ex
\tableofcontents
\endgroup

%%%%%%%%%%%%%%%%%%%%%%%%%%%%%%%%%%%%%%%%%%%%%%%%%%%%%%%%%%%%%%%%%%%%%%%%%%%%%%%%
%%%%%%%%%%%%%%%%%%%%%%%%%%%%%%%%%%%%%%%%%%%%%%%%%%%%%%%%%%%%%%%%%%%%%%%%%%%%%%%%
\section{Introduction}

\LaTeX{} provides a mechanism to structure a large document (such as a book)
into a main file and several child files (containing the chapters)
using the |\include| command.
This mechanism is beneficial for documents
which span hundreds of pages in order to
make the source file(s) more manageable.
Moreover, compilation can be restricted to
selected child files by means of the |\includeonly| command.
The latter feature can be used to reduce the compilation time while editing
(this was significantly more useful in the earlier days of \LaTeX{})
or to generate a smaller document which is easier to navigate.
Another application of |\includeonly| is to generate
documents consisting of selected parts of the complete document.

However, there are a few drawbacks of the plain |\include| mechanism:
\begin{itemize}
\item
The child files cannot be compiled on their own,
they can only be compiled via the main file.
A naive editing environment
(such as a text editor with an option
to have the current file processed by \LaTeX)
may require one to switch to the main file before compiling;
attempting to compile the child file produces errors.
\item
The main file must be modified (each time)
to adjust the |\includeonly| command
to the present needs. This easily leaves the main file in a messy state.
\item
The generated document will always carry the filename
of the main document. This is inconvenient if
several child files are to be compiled and
to be kept for distribution.
\end{itemize}

The present package provides a simple interface
to make child files individually compilable by \LaTeX{}.
Compiling a child file then has the same effect as compiling
the main file with an |\includeonly| command
to select the appropriate child.
Moreover the generated document will carry the name of the child
rather than the main file.
This resolves all three above issues.

This feature is meant to make the editing of books,
thesis documents and lecture notes somewhat more convenient.
However, the package can also be used efficiently for
composing a series of documents (such as exercise sheets)
which are typically distributed individually.
It then assists the author in generating the individual documents
(potentially in different versions)
as well as a document containing the collected series.
Another application is in developing style files
or other kinds of included material
where compilation of the style file could redirect
to a sample or test file.

%%%%%%%%%%%%%%%%%%%%%%%%%%%%%%%%%%%%%%%%%%%%%%%%%%%%%%%%%%%%%%%%%%%%%%%%%%%%%%%%
%%%%%%%%%%%%%%%%%%%%%%%%%%%%%%%%%%%%%%%%%%%%%%%%%%%%%%%%%%%%%%%%%%%%%%%%%%%%%%%%
\section{Usage}

First of all, the package \textsf{childdoc} is \emph{not} a standard
\LaTeXe{} |.sty| style file! Therefore it needs to be invoked in
a non-standard way.

%%%%%%%%%%%%%%%%%%%%%%%%%%%%%%%%%%%%%%%%%%%%%%%%%%%%%%%%%%%%%%%%%%%%%%%%%%%%%%%%
\subsection{Included Files}
\label{sec:include}

%%%%%%%%%%%%%%%%%%%%%%%%%%%%%%%%%%%%%%%%
\DescribeMacro{\childdocmain}
To use the package, add the commands
\begin{center}
\begin{tabular}{l}
|\input{childdoc.def}|\\
|\childdocmain{}|\\
\end{tabular}
\end{center}
at the very top of the main \LaTeX{} file,
in particular \emph{before} the |\documentclass| statement!
The argument of |\childdocmain| should be left empty
(but it must be present).

%%%%%%%%%%%%%%%%%%%%%%%%%%%%%%%%%%%%%%%%
\DescribeMacro{\childdocof}
Furthermore, add the commands
\begin{center}
\begin{tabular}{l}
|\input{childdoc.def}|\\
|\childdocof{|\textit{main}|}|\\
\end{tabular}
\end{center}
at the top of every child file \textit{child}
which is included by |\include{|\textit{child}|}|
from within the main file
(or at least for those files to be compiled individually).
The argument \textit{main} must be the filename of the main file.

There are a couple of
considerations in setting up the main and child documents:

%%%%%%%%%%%%%%%%%%%%%%%%%%%%%%%%%%%%%%%%
\paragraph{Restrictions.}

Please note the following restrictions:
\begin{itemize}
\item
|\childdocmain| must be called with one argument \textit{main}
to ensure compatibility with earlier version of the package.
It must either be empty (|\childdocmain{}|)
or precisely match the filename of the main file in which it is specified.
See \secref{sec:detection} for further information.
\item
The filename \textit{main} must be specified without the |.tex| extension.
\item
The filename \textit{main} is case sensitive
(even in case-insensitive file systems)
due to internal string comparison.
\item
The argument \textit{main} should be fully expanded, it cannot be a macro.
\item
Subdirectories and special characters should be avoided in filenames.
\item
The command |\childdocmain{|\textit{main}|}| must be followed by a whitespace.
It should not be followed immediately by another command
or by a comment mark `|%|'.
This is because the \TeX{} parser reads the token immediately following
the argument of |\childdocmain| and puts it
at the beginning of every child section;
however, a white\-space is ignored.
\end{itemize}

%%%%%%%%%%%%%%%%%%%%%%%%%%%%%%%%%%%%%%%%
\paragraph{Content of Main File.}

It is advisable to place all content in the child files included by |\include|.
Any output contained in the main file will appear in all child documents
unless suppressed manually;
it cannot be suppressed automatically by the |\includeonly| directive
and thus should normally be avoided.
A method to include some content in the main file
by means of conditional processing is described in \secref{sec:conditional}.

%%%%%%%%%%%%%%%%%%%%%%%%%%%%%%%%%%%%%%%%
\paragraph{Page Numbering.}

When only a part of the document is compiled,
the appropriate numbering of pages
(as well as other status parameters)
is determined from the |.aux| files.
The latter contain information from previous passes.
However this information needs to propagate through
all intermediate child documents.
Therefore the page numbering in child documents may well
be inconsistent until the complete document is compiled at least once.

A useful (if unconventional) way to always ensure a consistent
page numbering is to restart the numbering in each child document
and denote the pages by `\textit{child}|.|\textit{page}'
where \textit{child} represents the chapter/section number of the child file.
This can be achieved by the command
|\numberwithin{page}{|\textit{child}|}|
of the \textsf{amsmath} package
where \textit{child} can be |chapter| or |section|
depending on the chosen structuring.
Alternatively, one can modify the macro |\thepage| appropriately
and reset the counter |page| at the start of each child file.

%%%%%%%%%%%%%%%%%%%%%%%%%%%%%%%%%%%%%%%%%%%%%%%%%%%%%%%%%%%%%%%%%%%%%%%%%%%%%%%%
\subsection{Conditional Processing}
\label{sec:conditional}

The package provides a mechanism to compile different versions
of a document. To customise the versions further some conditional processing
can come in handy to distinguish which version is being compiled.
The package provides two macros to describe the compilation context:

%%%%%%%%%%%%%%%%%%%%%%%%%%%%%%%%%%%%%%%%
\DescribeMacro{\ifchilddoc}
The conditional |\ifchilddoc| distinguishes between the compilation of
child documents and the main document:
%
\begin{center}
|\ifchilddoc |\textit{child-code}| |[|\||else |\textit{main-code}]| \||fi|
\end{center}

%%%%%%%%%%%%%%%%%%%%%%%%%%%%%%%%%%%%%%%%
\DescribeMacro{\childdocname}
\DescribeMacro{\childdocjob}
The macro |\childdocname| contains the filename (without extension)
of the main or child file being processed.
Note that |\childdocjob| will always contain the name of the main file.

%%%%%%%%%%%%%%%%%%%%%%%%%%%%%%%%%%%%%%%%
\paragraph{Title Page.}

Conditional processing can be used to include a title or banner page
in the main document when proper precautions are taken.
Importantly, the code in the main file should ensure that the page counter
(as well as other status parameters which are stored in the |.aux| files)
takes the same value after the conditional processing.
Otherwise the page numbers may take divergent values
depending on which part is compiled.

For example, a title page could be declared by:
%
\begin{center}
\begin{tabular}{l}
|\ifchilddoc\||else|\\
|\addtocounter{page}{-1}|\\
\textit{code for title page}\\
|\newpage|\\
|\||fi|
\end{tabular}
\end{center}
%
A banner page for the child documents can be generated by:
%
\begin{center}
\begin{tabular}{l}
|\ifchilddoc|\\
|\addtocounter{page}{-1}|\\
\textit{code for banner page}\\
|\newpage|\\
|\||fi|
\end{tabular}
\end{center}
%
Here one could write a message such as:
\begin{center}
|This is the part \childdocname{} of \childdocjob{}.|
\end{center}

%%%%%%%%%%%%%%%%%%%%%%%%%%%%%%%%%%%%%%%%%%%%%%%%%%%%%%%%%%%%%%%%%%%%%%%%%%%%%%%%
\subsection{Flags}
\label{sec:flags}

The package makes it easy to generate different versions
of the main or child documents.
To this end compilation flags can be defined
and assigned different default values.
They will be particularly useful in conjunction
with the forwarding mechanism described in \secref{sec:forward}.

For example, it may be useful to have a flag |\version|
which can be set to |draft| or |final|.
The document source will contain some conditional code
depending on the value of |\version|.
Suppose further, the flag should default to |final| for the main file
and to |draft| for child files
which is a natural assignment for editing the document.
This is achieved by placing the following code
in the preamble of the main document
(below the |\childdocmain| directive):
%
\begin{center}
\begin{tabular}{l}
|\ifchilddoc|\\
|\providecommand{\version}{draft}|\\
|\||else|\\
|\providecommand{\version}{final}|\\
|\||fi|
\end{tabular}
\end{center}
%
The definition by |\providecommand| makes sure
that previous definitions are not overwritten.
Further statements |\providecommand{\version}{...}|
can thus be added before the above code to override it.

For the main file, one might add a line
(between |\childdocmain| and the above block)
%
\begin{center}
|%\ifchilddoc\||else\providecommand{\version}{draft}\||fi|
\end{center}
%
which can be uncommented to produce a draft version.
Likewise one can add a line to the very top of a child file
(above the |\childdocof{|\textit{main}|}| directive)
%
\begin{center}
|%\providecommand{\version}{final}|
\end{center}
%
which can be uncommented to produce the final version of this child document.

%%%%%%%%%%%%%%%%%%%%%%%%%%%%%%%%%%%%%%%%%%%%%%%%%%%%%%%%%%%%%%%%%%%%%%%%%%%%%%%%
\subsection{Forwarding}
\label{sec:forward}

Different versions of the main or child documents
using compilation flags as described in \secref{sec:flags}
can be (permanently) stored in different files
for convenient compilation, viewing and distribution.
To this end, the package defines a command
to pass on compilation to a different file:

%%%%%%%%%%%%%%%%%%%%%%%%%%%%%%%%%%%%%%%%
\DescribeMacro{\childdocforward}
The command |\childdocforward| redirects processing to
another source file:
%
\begin{center}
\begin{tabular}{l}
|\input{childdoc.def}|\\
|\childdocforward[|\textit{main}|]{|\textit{dest}|}|\\
\end{tabular}
\end{center}
%
The argument \textit{dest} is the destination file
(without extension).
It should be the main file or one of the child files.
Note that further \textsf{childdoc} directives
such as |\childdocof| and |\childdocforward|
in the indicated file will be processed in this form.
The optional argument \textit{main}
passes on directly to the main file \textit{main}
while pretending to compile the child \textit{dest}.
This form behaves as if \textit{dest}
issues |\childdocof{|\textit{main}|}| right away,
and no further \textsf{childdoc} directives will be processed.

%%%%%%%%%%%%%%%%%%%%%%%%%%%%%%%%%%%%%%%%
\DescribeMacro{\...prefix}
In the alternative form |\childdocforwardprefix|,
%
\begin{center}
\begin{tabular}{l}
|\input{childdoc.def}|\\
|\childdocforwardprefix[|\textit{main}|]{|\textit{prefix}|}{|\textit{dest}|}|
\end{tabular}
\end{center}
%
the destination file is determined by a pattern
depending on the current file:
To make this work, the current file must be called
`{\textit{prefix}\hspace{0.2em}\textit{suffix}}'
with \textit{prefix} matching precisely the argument.
Processing is then passed on to the file
`{\textit{dest}\hspace{0.2em}\textit{suffix}}'.
Surely, the same effect is achieved by
directly specifying the
argument `{\textit{dest}\hspace{0.2em}\textit{suffix}}'
in the first form.
However, that requires to set up a different file
for each child. With the alternative form of the command
all these files can have exactly the same content
which simplifies setting them up and maintaining them.

For example, the following file |draft.tex|
with a compilation flag |\version| as described in \secref{sec:flags}
compiles the main document as a draft:
%
\begin{center}
\begin{tabular}{l}
|\def\version{draft}|\\
|\input{childdoc.def}|\\
|\childdocforward{|\textit{main}|}|
\end{tabular}
\end{center}
%
Likewise, the following files |final|\textit{nn}|.tex|
compile the final version of the child document
|child|\textit{nn}|.tex|:
%
\begin{center}
\begin{tabular}{l}
|\def\version{final}|\\
|\input{childdoc.def}|\\
|\childdocforwardprefix{final}{child}|
\end{tabular}
\end{center}
%

Note that when several versions of a main file and/or of each child file
are to be generated, it may be convenient to set up a |Makefile| or
shell script to automatise the process.

%%%%%%%%%%%%%%%%%%%%%%%%%%%%%%%%%%%%%%%%%%%%%%%%%%%%%%%%%%%%%%%%%%%%%%%%%%%%%%%%
\subsection{Command Line Processing}
\label{sec:commandline}

The effect of redirection files can also be achieved by invoking
the \LaTeX{} compiler with a more elaborate command line.
Most conveniently this should be done as part
of a shell script or a |Makefile|.

When using \textsf{childdoc} in the main file, the following
command lines effectively perform a redirection
(note that depending on the shell being used,
backslashes may have to be doubled: `|\|' $\to$ `|\\|'):
%
\begin{center}
|... -jobname "|\textit{target}|" |\\|"|[\textit{flags}]%
|\input{childdoc.def}\childdocforward[|\textit{main}|]{|\textit{dest}|}"|
\end{center}
%
Here \textit{target} is the name of the output file,
\textit{main} is the name of the main file
and \textit{dest} is the name of the main or child file to be processed
(all filenames without extensions).
The optional argument \textit{main} can be omitted
if \textit{main} matches \textit{dest}.
Optionally, compilation \textit{flags} can be defined via |\def| commands.
This command line makes the \TeX{} engine believe
it is compiling the file \textit{target}
whose content is specified as the latter parameter.
The provided code then forwards the processing to
\textit{main} or \textit{dest} as described in \secref{sec:forward}.

%%%%%%%%%%%%%%%%%%%%%%%%%%%%%%%%%%%%%%%%%%%%%%%%%%%%%%%%%%%%%%%%%%%%%%%%%%%%%%%%
\subsection{Include by Input}
\label{sec:input}

Including child documents by |\include| has some restrictions by design.
Most notably, the content of a child document always occupies
its own set of pages; pages cannot be shared between child documents.
Usually, this behaviour makes perfect sense
because each child document contain an essential part of the document.
However, in some situations it may be desirable to compose
a document from a collection of parts
without having mandatory page breaks between then.
For this case, the package
provides a mechanism to include parts
by |\input| which can also be processed individually.
However, by construction this mechanism
requires manual handling of the content to be output.

%%%%%%%%%%%%%%%%%%%%%%%%%%%%%%%%%%%%%%%%
\DescribeMacro{\ifchilddocmanual}
The main file should be prepared as usual, see \secref{sec:include}.
However, the document body must make a distinction
between processing of an individual part and of the main document, e.g.:
%
\begin{center}
\begin{tabular}{l}
|\ifchilddocmanual|\\
|\input{\childdocname}|\\
|\||else|\\
\textit{document body with }|\input{|\textit{part}|}|\\
|\||fi|
\end{tabular}
\end{center}
%
The conditional |\ifchilddocmanual| is true whenever
a part to be included by |\input| is being compiled,
and the name of the part is stored in |\childdocname|.

%%%%%%%%%%%%%%%%%%%%%%%%%%%%%%%%%%%%%%%%
\DescribeMacro{\childdocby}
Each part to be included by |\input| should start with:
%
\begin{center}
\begin{tabular}{l}
|\input{childdoc.def}|\\
|\childdocby{|\textit{main}|}|\\
\end{tabular}
\end{center}
%
The directive |\childdocby| is similar to |\childdocof|
described in \secref{sec:include},
but the subsequent selection of content must be done manually.
To that end, both |\ifchilddoc| and |\ifchilddocmanual|
will be true upon processing of a part,
and the name of the part is stored in |\childdocname|.
Note that |\jobname| will be set to the filename of the current part
so that each part receives an individual |.aux| file
that does not interfere with the |.aux| file(s) of the main document.
This behaviour can be altered by the alternative form
|\childdocby[*]{|\textit{main}|}| (with a non-empty optional argument)
which uses the |.aux| file of the main document
by setting |\jobname| to \textit{main}.

%%%%%%%%%%%%%%%%%%%%%%%%%%%%%%%%%%%%%%%%%%%%%%%%%%%%%%%%%%%%%%%%%%%%%%%%%%%%%%%%
\subsection{Driver Development}
\label{sec:driver}

The \textsf{childdoc} mechanism can also be use for the development
of definition files such as \LaTeX{} styles or classes.
This case differs from the above setup with multiple parts
included by |\include| in that no |\includeonly| should be invoked.
This can be achieved by starting the include file
(before |\ProvidesPackage|) with:
%
\begin{center}
\begin{tabular}{l}
|\input{childdoc.def}|\\
|\childdocforward{|\textit{main}|}|\\
\end{tabular}
\end{center}
%
or alternatively with:
%
\begin{center}
\begin{tabular}{l}
|\input{childdoc.def}|\\
|\childdocby{|\textit{main}|}|\\
\end{tabular}
\end{center}
%
Both forms have slightly different effects as described above.
The main file is prepared as usual, see \secref{sec:include}.

%%%%%%%%%%%%%%%%%%%%%%%%%%%%%%%%%%%%%%%%%%%%%%%%%%%%%%%%%%%%%%%%%%%%%%%%%%%%%%%%
\subsection{Legacy Detection}
\label{sec:detection}

The directive |\childdocmain| in the main file can detect
whether the complete document or merely a child is to be compiled
even without using the directive |\childdocof|.
This method is deprecated because it is less robust
and there is no compelling reason to use it;
it is merely provided for backward compatibility
and it may be removed in future versions.

If the detection mechanism is to be used,
it is mandatory to correctly specify
the filename of the main file as the argument of |\childdocmain|:
%
\begin{center}
\begin{tabular}{l}
|\input{childdoc.def}|\\
|\childdocmain{|\textit{main}|}|\\
\end{tabular}
\end{center}
%
If |\jobname| does not match the argument \textit{main} of |\childdocmain|,
it is assumed that |\jobname| points to the child file to be compiled.
When using |\childdocmain| with the main file specified as argument,
it suffices to start a child file
with just |\input{|\textit{main}|}|
without loading of the package and using |\childdocof|.
If instead all processing is done
with the appropriate \textsf{childdoc} directives,
the argument of \textit{main} of |\childdocmain| can be empty.

An alternative version of the command line processing described
in \secref{sec:commandline} using the detection mechanism reads:
%
\begin{center}
|... -jobname "|\textit{target}|" "|[\textit{flags}]%
[|\def\jobname{|\textit{dest}|}|]|\input{|\textit{main}|}"|
\end{center}

%%%%%%%%%%%%%%%%%%%%%%%%%%%%%%%%%%%%%%%%%%%%%%%%%%%%%%%%%%%%%%%%%%%%%%%%%%%%%%%%
\subsection{Manual Code}
\label{sec:manual}

In case one cannot be certain whether the definitions file |childdoc.def|
is installed on the target \TeX{} distribution
and one prefers not to ship it,
it is conceivable to paste a few relevant commands into the sources.

To that end, drop all statements |\input{childdoc.def}|
and perform the replacements as outlined below.
Instead of |\childdocmain{|\textit{main}|}| add the following code
to the top of the main file:
%
\begin{center}
\begin{tabular}{l}
|\||ifdefined\childdocname\endinput\||fi\newif\ifchilddoc|\\
|\edef\childdocname{\scantokens\expandafter{\jobname\noexpand}}|\\
|\def\childdocmain{|\textit{main}|}\||ifx\childdocmain\childdocname\||else|\\
|\childdoctrue\includeonly{\childdocname}\let\jobname\childdocmain\||fi|\\
\end{tabular}
\end{center}
%
Instead of |\childdocof{|\textit{main}|}| just include the main file
at the top of each child file:
%
\begin{center}
|\input{|\textit{main}|}|
\end{center}
%
A simple redirection |\childdocforward{|\textit{dest}|}| is achieved by:
%
\begin{center}
|\def\jobname{|\textit{dest}|}\input{\jobname}|
\end{center}
%
The redirection with prefix
|\childdocforwardprefix[|\textit{prefix}|]{|\textit{dest}|}|
is accomplished by:
%
\begin{center}
\begin{tabular}{l}
|{\edef\jobname{\scantokens\expandafter{\jobname\noexpand}}|\\
|\def\redirectjob |\textit{prefix}|#1~~~{\gdef\jobname{|\textit{dest}|#1}}|\\
|\expandafter\redirectjob\jobname~~~}\input{\jobname}|
\end{tabular}
\end{center}

In an alternative approach,
child documents can be compiled by a specific command line
without additional code or specific definitions:
%
\begin{center}
|... -jobname "|\textit{target}|" "|[\textit{flags}]%
|\includeonly{|\textit{dest}|}\input{|\textit{main}|}"|
\end{center}
%

%%%%%%%%%%%%%%%%%%%%%%%%%%%%%%%%%%%%%%%%%%%%%%%%%%%%%%%%%%%%%%%%%%%%%%%%%%%%%%%%
%%%%%%%%%%%%%%%%%%%%%%%%%%%%%%%%%%%%%%%%%%%%%%%%%%%%%%%%%%%%%%%%%%%%%%%%%%%%%%%%
\section{Information}

%%%%%%%%%%%%%%%%%%%%%%%%%%%%%%%%%%%%%%%%%%%%%%%%%%%%%%%%%%%%%%%%%%%%%%%%%%%%%%%%
\subsection{Copyright}

Copyright \copyright{} 2017--2018 Niklas Beisert

This work may be distributed and/or modified under the
conditions of the \LaTeX{} Project Public License, either version 1.3
of this license or (at your option) any later version.
The latest version of this license is in
  \url{http://www.latex-project.org/lppl.txt}
and version 1.3 or later is part of all distributions of \LaTeX{}
version 2005/12/01 or later.

This work has the LPPL maintenance status `maintained'.

The Current Maintainer of this work is Niklas Beisert.

This work consists of the files |README.txt|, |childdoc.ins| and |childdoc.dtx|
as well as the derived files |childdoc.def|, |cdocsamp.tex|
with |cdocsch1.tex|, |cdocsch2.tex|, |cdocspt3.tex|, |cdocspt4.tex|,
|cdocsdrf.tex|, |cdocsfn1.tex|, |cdocsfn2.tex|
as well as |childdoc.pdf|.

%%%%%%%%%%%%%%%%%%%%%%%%%%%%%%%%%%%%%%%%%%%%%%%%%%%%%%%%%%%%%%%%%%%%%%%%%%%%%%%%
\subsection{Files and Installation}

The package consists of the files:
%
\begin{center}
\begin{tabular}{ll}
    |README.txt|   & readme file \\
    |childdoc.ins| & installation file \\
    |childdoc.dtx| & source file \\
    |childdoc.def| & definition file \\
    |cdocsamp.tex| & sample main file \\
    |cdocsch1.tex| & sample include file \\
    |cdocsch2.tex| & sample include file \\
    |cdocspt3.tex| & sample part file \\
    |cdocspt4.tex| & sample part file \\
    |cdocsdrf.tex| & sample redirection file \\
    |cdocsfn1.tex| & sample redirection file \\
    |cdocsfn2.tex| & sample redirection file \\
    |childdoc.pdf| & manual
\end{tabular}
\end{center}
%
The distribution consists of the files
|README.txt|, |childdoc.ins| and |childdoc.dtx|.
%
\begin{itemize}
\item
Run (pdf)\LaTeX{} on |childdoc.dtx|
to compile the manual |childdoc.pdf| (this file).
\item
Run \LaTeX{} on |childdoc.ins| to create the definitions file |childdoc.def|
and the sample |cdocsamp.tex| with include files
|cdocsch1.tex|, |cdocsch2.tex|, |cdocspt3.tex|, |cdocspt4.tex|,
|cdocsdrf.tex|, |cdocsfn1.tex|, |cdocsfn2.tex|.
Then copy the file |childdoc.def| to an appropriate directory of your \LaTeX{}
distribution, e.g.\ \textit{texmf-root}|/tex/latex/childdoc|.
\end{itemize}

%%%%%%%%%%%%%%%%%%%%%%%%%%%%%%%%%%%%%%%%%%%%%%%%%%%%%%%%%%%%%%%%%%%%%%%%%%%%%%%%
\subsection{Related CTAN Packages}

There are several other packages which offer a similar functionality:
%
\begin{itemize}
\item
The packages
\href{http://ctan.org/pkg/docmute}{\textsf{docmute}},
\href{http://ctan.org/pkg/includex}{\textsf{includex}} and
\href{http://ctan.org/pkg/standalone}{\textsf{standalone}}
provide commands to include only the document body of
a child file thus allowing both files to be compiled individually.
\item
The packages \href{http://ctan.org/pkg/subdocs}{\textsf{subdocs}}
and \href{http://ctan.org/pkg/subfiles}{\textsf{subfiles}}
provide structures in which the main and child documents can be
encapsulated and allowing them to be compiled individually.
The inclusion mechanism is different from the conventional |\include|.
\item
The package \href{http://ctan.org/pkg/combine}{\textsf{combine}}
is an elaborate solution to combine several documents into one.
\end{itemize}
%
See also the CTAN topic \href{http://ctan.org/topic/subdocs}{\textsf{subdocs}}
for further related packages.
The present package differs from the above solutions in that
a document structure constructed with the conventional |\include| mechanism
just needs two extra commands at the top of every file
such that all constituent files can be compiled individually.

%%%%%%%%%%%%%%%%%%%%%%%%%%%%%%%%%%%%%%%%%%%%%%%%%%%%%%%%%%%%%%%%%%%%%%%%%%%%%%%%
%\subsection{Feature Suggestions}
%
%The following is a list of features which may be useful for future
%versions of this package:
%%
%\begin{itemize}
%\item
%\ldots
%\end{itemize}

%%%%%%%%%%%%%%%%%%%%%%%%%%%%%%%%%%%%%%%%%%%%%%%%%%%%%%%%%%%%%%%%%%%%%%%%%%%%%%%%
\subsection{Revision History}

%%%%%%%%%%%%%%%%%%%%%%%%%%%%%%%%%%%%%%%%
\paragraph{v2.0:} 2018/12/30

\begin{itemize}
\item
immediate forward processing
\item
added |\childdocby| mechanism
\item
manual restructured
\end{itemize}

%%%%%%%%%%%%%%%%%%%%%%%%%%%%%%%%%%%%%%%%
\paragraph{v1.6:} 2018/01/17

\begin{itemize}
\item
application for development of include files
\item
corrections to manual
\end{itemize}

%%%%%%%%%%%%%%%%%%%%%%%%%%%%%%%%%%%%%%%%
\paragraph{v1.5:} 2017/05/21

\begin{itemize}
\item
more complete structuring introduced
\item
|\childdocof| introduced
\item
|\childdoc| renamed to |\childdocmain|
\item
|\childredirect| renamed to |\childdocforward| and |\childdocforwardprefix|
and functionality expanded
\end{itemize}

%%%%%%%%%%%%%%%%%%%%%%%%%%%%%%%%%%%%%%%%
\paragraph{v1.0:} 2017/04/27

\begin{itemize}
\item
manual and install package
\item
first version published on CTAN
\end{itemize}

%%%%%%%%%%%%%%%%%%%%%%%%%%%%%%%%%%%%%%%%
\paragraph{v0.6:} 2017/04/26

\begin{itemize}
\item
redirection mechanism added
\end{itemize}

%%%%%%%%%%%%%%%%%%%%%%%%%%%%%%%%%%%%%%%%
\paragraph{v0.5:} 2017/04/26

\begin{itemize}
\item
functionality in definition file
\end{itemize}


%%%%%%%%%%%%%%%%%%%%%%%%%%%%%%%%%%%%%%%%%%%%%%%%%%%%%%%%%%%%%%%%%%%%%%%%%%%%%%%%
%%%%%%%%%%%%%%%%%%%%%%%%%%%%%%%%%%%%%%%%%%%%%%%%%%%%%%%%%%%%%%%%%%%%%%%%%%%%%%%%
%%%%%%%%%%%%%%%%%%%%%%%%%%%%%%%%%%%%%%%%%%%%%%%%%%%%%%%%%%%%%%%%%%%%%%%%%%%%%%%%
\appendix

\settowidth\MacroIndent{\rmfamily\scriptsize 000\ }

 \DocInput{childdoc.dtx}

\end{document}
%</driver>
% \fi
%
% %%%%%%%%%%%%%%%%%%%%%%%%%%%%%%%%%%%%%%%%%%%%%%%%%%%%%%%%%%%%%%%%%%%%%%%%%%%%%%
% %%%%%%%%%%%%%%%%%%%%%%%%%%%%%%%%%%%%%%%%%%%%%%%%%%%%%%%%%%%%%%%%%%%%%%%%%%%%%%
% \section{Sample}
%\iffalse
%<*samplemain>
%\fi
%
% The following presents a sample document
% with two chapters, two parts, a title page,
% a compile flag as well as three forwarding files to set the flag.
% It consists of eight |.tex| files:
% \begin{center}
% \begin{tabular}{ll}
% |cdocsamp.tex|&main file\\
% |cdocsch1.tex|&include file for chapter 1\\
% |cdocsch2.tex|&include file for chapter 2\\
% |cdocspt3.tex|&include file for part 3\\
% |cdocspt4.tex|&include file for part 4\\
% |cdocsdrf.tex|&forwarding file for main file in draft mode\\
% |cdocsfi1.tex|&forwarding file for final version of chapter 1\\
% |cdocsfi2.tex|&forwarding file for final version of chapter 2\\
% \end{tabular}
% \end{center}
% Each of the eight files can be compiled directly by the \LaTeX{} compiler.
%
% %%%%%%%%%%%%%%%%%%%%%%%%%%%%%%%%%%%%%%
% \paragraph{Main File.}
%
% The main file is called |cdocsamp.tex|.
%
% Load the \textsf{childdoc} definitions and
% declare the filename for the main document:
%    \begin{macrocode}
\input{childdoc.def}
\childdocmain{}
%    \end{macrocode}

% Optional override for |\version| flag:
%    \begin{macrocode}
%%\ifchilddoc\else\providecommand{\version}{draft}\fi
%    \end{macrocode}

% Define the default values for the |\version| flag
% (|final| for the main file and |draft| for childs):
%    \begin{macrocode}
\ifchilddoc
\providecommand{\version}{draft}
\else
\providecommand{\version}{final}
\fi
%    \end{macrocode}

% Load the standard document class:
%    \begin{macrocode}
\documentclass[12pt]{article}
%    \end{macrocode}

% Start the document body:
%    \begin{macrocode}
\begin{document}
%    \end{macrocode}

% Declare a title page.
% Print title, part of document being processed and version flag:
%    \begin{macrocode}
\addtocounter{page}{-1}
\begin{center}
{\LARGE\bfseries{}childdoc example\par}
\vspace{1cm}
\ifchilddoc
\ifchilddocmanual part\else chapter\fi:
`\childdocname' of `\childdocjob'\par
\else
main document: `\childdocjob'\par
\fi
version: \version\par
\end{center}
\newpage
%    \end{macrocode}

% Manually include selected file,
% otherwise process as usual:
%    \begin{macrocode}
\ifchilddocmanual
\section*{part `\childdocname'}
\input{\childdocname}
\else
%    \end{macrocode}

% Include the two chapters:
%    \begin{macrocode}
\include{cdocsch1}
\include{cdocsch2}
%    \end{macrocode}

% Include the two parts unless only chapters should be displayed:
%    \begin{macrocode}
\ifchilddoc\else
\section{part three}
\input{cdocspt3}
\section{part four}
\input{cdocspt4}
\fi
%    \end{macrocode}

% Process as usual until here:
%    \begin{macrocode}
\fi
%    \end{macrocode}

% End of document body:
%    \begin{macrocode}
\end{document}
%    \end{macrocode}
%\iffalse
%</samplemain>
%\fi
%
% %%%%%%%%%%%%%%%%%%%%%%%%%%%%%%%%%%%%%%
% \paragraph{Chapter Include Files.}
%
% The include files are called |cdocsch1.tex| and |cdocsch2.tex|.
%
%\iffalse
%<*samplechap1|samplechap2>
%\fi

% Optional override for |\version| flag:
%    \begin{macrocode}
%%\providecommand{\version}{final}
%    \end{macrocode}

% Include the main document:
%    \begin{macrocode}
\input{childdoc.def}
\childdocof{cdocsamp}
%    \end{macrocode}

%\iffalse
%</samplechap1|samplechap2>
%\fi
%
%\iffalse
%<*samplechap1>
%\fi
% Some text for chapter 1:
%    \begin{macrocode}
\section{one}
some text in chapter one
%    \end{macrocode}

%\iffalse
%</samplechap1>
%\fi
% Some text for chapter 2:
%\iffalse
%<*samplechap2>
%\fi
%    \begin{macrocode}
\section{two}
more text in chapter two
%    \end{macrocode}

%\iffalse
%</samplechap2>
%\fi
%
% %%%%%%%%%%%%%%%%%%%%%%%%%%%%%%%%%%%%%%
% \paragraph{Part Include Files.}
%
% The include files are called |cdocspt3.tex| and |cdocspt4.tex|.
%
%\iffalse
%<*samplepart3|samplepart4>
%\fi

% Optional override for |\version| flag:
%    \begin{macrocode}
%%\providecommand{\version}{final}
%    \end{macrocode}

% Include the main document:
%    \begin{macrocode}
\input{childdoc.def}
\childdocby{cdocsamp}
%    \end{macrocode}

%\iffalse
%</samplepart3|samplepart4>
%\fi
%
%\iffalse
%<*samplepart3>
%\fi
% Some text for part 3:
%    \begin{macrocode}
some text in part three
%    \end{macrocode}

%\iffalse
%</samplepart3>
%\fi
% Some text for part 4:
%\iffalse
%<*samplepart4>
%\fi
%    \begin{macrocode}
more text in part four
%    \end{macrocode}

%\iffalse
%</samplepart4>
%\fi
%
% %%%%%%%%%%%%%%%%%%%%%%%%%%%%%%%%%%%%%%
% \paragraph{Forwarding for a Complete Draft.}
%
% The following forwarding file |cdocsdrf.tex|
% compiles the main document in draft mode:
%\iffalse
%<*sampledraft>
%\fi
%    \begin{macrocode}
\def\version{draft}
\input{childdoc.def}
\childdocforward{cdocsamp}
%    \end{macrocode}

%\iffalse
%</sampledraft>
%\fi
%
% %%%%%%%%%%%%%%%%%%%%%%%%%%%%%%%%%%%%%%
% \paragraph{Forwarding for Final Version of the Chapters.}
%
% The following forwarding files |cdocsfn1.tex| and |cdocsfn2.tex|
% (with identical content)
% compile the final versions of the child documents
% |cdocsch1.tex| and |cdocsch2.tex|, respectively:
%\iffalse
%<*samplefinal>
%\fi
%    \begin{macrocode}
\def\version{final}
\input{childdoc.def}
\childdocforwardprefix[cdocsamp]{cdocsfn}{cdocsch}
%    \end{macrocode}

%\iffalse
%</samplefinal>
%\fi
%
% %%%%%%%%%%%%%%%%%%%%%%%%%%%%%%%%%%%%%%
% \paragraph{Command Line Processing.}
%
% The following three command lines generate the output files
% |cdocscld|, |cdocscl1| and |cdocscl2|
% which should be identical to
% |cdocsdrf|, |cdocsch1| and |cdocsfn2|, respectively:
% \begin{center}
% \begin{tabular}{l}
% |latex -jobname cdocscld \|\\
% |  "\def\version{draft}\input{childdoc.def}\childdocforward{cdocsamp}"|\\
% |latex -jobname cdocscl1 \|\\
% |  "\input{childdoc.def}\childdocforward[cdocsamp]{cdocsch1}"|\\
% |latex -jobname cdocscl2 \|\\
% |  "\def\version{final}\input{childdoc.def}\childdocforward{cdocsch2}"|
% \end{tabular}
% \end{center}
% Note that the trailing backslash on each first line
% merely continues the input to the second line
% (for convenient cut ant paste).
% Furthermore, the command |latex| can be replaced by any
% of its alternative versions such as |pdflatex|.
%
% %%%%%%%%%%%%%%%%%%%%%%%%%%%%%%%%%%%%%%%%%%%%%%%%%%%%%%%%%%%%%%%%%%%%%%%%%%%%%%
% %%%%%%%%%%%%%%%%%%%%%%%%%%%%%%%%%%%%%%%%%%%%%%%%%%%%%%%%%%%%%%%%%%%%%%%%%%%%%%
% \section{Implementation}
%\iffalse
%<*package>
%\fi
%
% This section describes the definitions file |childdoc.def|.

% The definitions cannot be loaded using |\usepackage| or |\RequirePackage|
% which has a mechanism to prevent loading a style file more than once.
% When loading the definitions by means of |\input|
% multiple instances have to be prevented manually:
%\iffalse
%This code needs to be before the `\ProvidesFile' directive
%which is defined at the beginning of this file.
%Therefore it is also placed there and commented out here.
%</package>
%<*discard>
%\fi
%    \begin{macrocode}
\ifdefined\childdocmain\endinput\fi
%    \end{macrocode}
%\iffalse
%</discard>
%<*package>
%\fi
%
% \macro{\ifchilddoc}
% \macro{\ifchilddocmanual}
% The conditional |\ifchilddoc| tells whether a
% child (true) or main (false) document is being compiled.
% The conditional |\ifchilddocmanual| tells whether
% the |\includeonly| mechanism is used (false) or
% the selection of child files must be performed manually (true).
% The definitions initialise to false:
%    \begin{macrocode}
\newif\ifchilddoc
\newif\ifchilddocmanual
%    \end{macrocode}

% \macro{\childdocname}
% \macro{\childdocjob}
% The macro |\childdocname| stores the name of the main document
% to be compiled. The macro |\childdocjob| stores the name of
% the document on which the \LaTeX{} compiler was originally invoked.
% The content of |\jobname| cannot be compared
% to filenames specified in the source due to different catcodes.
% The following code rescans |\jobname|, stores the result
% in |\childdocname| and saves a copy in |\childdocjob|:
%    \begin{macrocode}
\edef\childdocname{\scantokens\expandafter{\jobname\noexpand}}
\let\childdocjob\childdocname
%    \end{macrocode}

% \macro{\childdocdisable}
% The macro |\childdocdisable| prevents the main file
% from being processed more than once.
% At this stage, the main document command |\childdocmain|
% is assumed to be called once again where it should do nothing.
% Any subsequent call to it should prevent
% a secondary processing of the main document
% It overwrites the forwarding commands
% |\childdocof| and |\childdocforward|
% with empty macros to prevent further inclusions of the main document:
%    \begin{macrocode}
\newcommand{\childdocdisable}
{
  \renewcommand{\childdocmain}[1]{\renewcommand{\childdocmain}[1]{\endinput}}
  \renewcommand{\childdocof}[1]{}
  \renewcommand{\childdocby}[2][]{}
  \renewcommand{\childdocforward}[2][]{}
  \renewcommand{\childdocdisable}{}
}
%    \end{macrocode}

% \macro{\childdocmain}
% The macro |\childdocmain| is to be called at the top of the main file
% with nothing or the main filename (without extension) as argument.
% First, it breaks loops.
% If the argument is not empty and does not match |\childdocname|
% (which is set by the first inclusion of |childdoc.def|),
% |\ifchilddoc| is set to true, |\includeonly| is applied to the child file
% and |\jobname| is set to the main file
% (for proper handling of |.aux| files):
%    \begin{macrocode}
\newcommand{\childdocmain}[1]
{
  \childdocdisable\childdocmain{}
  \if?#1?\else
    \begingroup
      \def\childdoctmp{#1}
      \ifx\childdoctmp\childdocname
        \def\childdoctmp{}
      \else
        \def\childdoctmp
        {
          \childdoctrue
          \includeonly{\childdocname}
          \def\childdocjob{#1}
          \def\jobname{#1}
        }
      \fi
      \expandafter
    \endgroup
    \childdoctmp
  \fi
}
%    \end{macrocode}

% \macro{\childdocof}
% The command |\childdocof| redirects
% compilation to the main file |#1|.
%    \begin{macrocode}
\newcommand{\childdocof}[1]
{
  \childdocdisable
  \childdoctrue
  \includeonly{\childdocname}
  \def\jobname{#1}
  \def\childdocjob{#1}
  \input{#1}
}
%    \end{macrocode}

% \macro{\childdocby}
% The command |\childdocby| ....
%    \begin{macrocode}
\newcommand{\childdocby}[2][]
{
  \childdocdisable
  \childdoctrue
  \childdocmanualtrue
  \if?#1?\else
    \def\jobname{#2}
  \fi
  \def\childdocjob{#2}
  \input{#2}
  \endinput
}
%    \end{macrocode}

% \macro{\childdocforward}
% The command |\childdocforward| redirects
% compilation to the main file or
% (if the optional argument is given) a child file.
% Parameters are set as if the main file
% or a child file starting with |\childdocof| was compiled.
% Then compilation is handed over to the main file:
%    \begin{macrocode}
\newcommand{\childdocforward}[2][]
{
  \begingroup
    \if?#1?
      \def\childdoctmp
      {
        \def\childdocname{#2}
        \def\childdocjob{#2}
        \def\jobname{#2}
        \input{#2}
        \endinput
      }
    \else
      \def\childdoctmp
      {
        \childdocdisable
        \def\childdocname{#2}
        \childdoctrue
        \includeonly{#2}
        \def\childdocjob{#1}
        \def\jobname{#1}
        \input{#1}
        \endinput
      }
    \fi
    \expandafter
  \endgroup
  \childdoctmp
}
%    \end{macrocode}

% \macro{\childdocforwardprefix}
% The command |\childdocforwardprefix| redirects
% compilation to the main or a child file by means of a pattern.
% The prefix |#1| in the current filename is replaced by |#2|
% and the suffix of the current filename is kept
% (it is assumed that the filename does not contain the substring `|~~~|'
% which is used as a delimiter).
% Compilation is handed over to the new file by |\childdocforward|:
%    \begin{macrocode}
\newcommand{\childdocforwardprefix}[3][]
{
  \begingroup
    \def\childdocextract #2##1~~~{\def\childdoctmp{\childdocforward[#1]{#3##1}}}
    \expandafter\childdocextract\childdocname~~~
    \expandafter
  \endgroup
  \childdoctmp
}
%    \end{macrocode}

% \macro{\childdoc}
% The deprecated macro |\childdoc| is a legacy version of |\childdocmain|:
%    \begin{macrocode}
\newcommand{\childdoc}{\childdocmain}
%    \end{macrocode}

% \macro{\childdocredirect}
% The deprecated macro |\childdocredirect| is a legacy version
% of |\childdocforward| and |\childdocforwardprefix|:
%    \begin{macrocode}
\newcommand{\childdocredirect}[2][]
{
  \begingroup
    \if?#1?
      \def\childdoctmp{\childdocforward{#2}}
    \else
      \def\childdoctmp{\childdocforwardprefix{#1}{#2}}
    \fi
    \expandafter
  \endgroup
  \childdoctmp
}
%    \end{macrocode}

%\iffalse
%</package>
%\fi
%
\endinput
|\\
|\childdocmain{}|\\
\end{tabular}
\end{center}
at the very top of the main \LaTeX{} file,
in particular \emph{before} the |\documentclass| statement!
The argument of |\childdocmain| should be left empty
(but it must be present).

%%%%%%%%%%%%%%%%%%%%%%%%%%%%%%%%%%%%%%%%
\DescribeMacro{\childdocof}
Furthermore, add the commands
\begin{center}
\begin{tabular}{l}
|% \iffalse
%
% childdoc.dtx Copyright (C) 2017-2018 Niklas Beisert
%
% This work may be distributed and/or modified under the
% conditions of the LaTeX Project Public License, either version 1.3
% of this license or (at your option) any later version.
% The latest version of this license is in
%   http://www.latex-project.org/lppl.txt
% and version 1.3 or later is part of all distributions of LaTeX
% version 2005/12/01 or later.
%
% This work has the LPPL maintenance status `maintained'.
%
% The Current Maintainer of this work is Niklas Beisert.
%
% This work consists of the files childdoc.dtx and childdoc.ins
% and the derived files childdoc.def and cdocsamp.tex with
% cdocsch1.tex, cdocsch2.tex, cdocsdrf.tex, cdocsfn1.tex, cdocsfn2.tex.
%
%<package>\ifdefined\childdocmain\endinput\fi
%<package>\ProvidesFile{childdoc.def}[2018/12/30 v2.0 child document driver]
%<samplemain>\ProvidesFile{cdocsamp.tex}[2018/12/30 v2.0 sample for childdoc]
%<*driver>
%\ProvidesFile{childdoc.drv}[2018/12/30 v2.0 childdoc reference manual file]
\PassOptionsToClass{10pt,a4paper}{article}
\documentclass{ltxdoc}

\usepackage[margin=35mm]{geometry}
\usepackage{hyperref}
\usepackage{hyperxmp}
\usepackage[usenames]{color}

\hypersetup{colorlinks=true}
\hypersetup{pdfstartview=FitH}
\hypersetup{pdfpagemode=UseNone}
\hypersetup{pdfsource={}}
\hypersetup{pdflang={en-UK}}
\hypersetup{pdfcopyright={Copyright 2017-2018 Niklas Beisert.
  This work may be distributed and/or modified under the
  conditions of the LaTeX Project Public License, either version 1.3
  of this license or (at your option) any later version.}}
\hypersetup{pdflicenseurl={http://www.latex-project.org/lppl.txt}}
\hypersetup{pdfcontactaddress={ETH Zurich, ITP, HIT K,
  Wolfgang-Pauli-Strasse 27}}
\hypersetup{pdfcontactpostcode={8093}}
\hypersetup{pdfcontactcity={Zurich}}
\hypersetup{pdfcontactcountry={Switzerland}}
\hypersetup{pdfcontactemail={nbeisert@itp.phys.ethz.ch}}
\hypersetup{pdfcontacturl={http://people.phys.ethz.ch/\xmptilde nbeisert/}}

\newcommand{\secref}[1]{\hyperref[#1]{section \ref*{#1}}}

\parskip1ex
\parindent0pt
\let\olditemize\itemize
\def\itemize{\olditemize\parskip0pt}

\begin{document}

\title{The \textsf{childdoc} Package}
\hypersetup{pdftitle={The childdoc Package}}
\author{Niklas Beisert\\[2ex]
  Institut f\"ur Theoretische Physik\\
  Eidgen\"ossische Technische Hochschule Z\"urich\\
  Wolfgang-Pauli-Strasse 27, 8093 Z\"urich, Switzerland\\[1ex]
  \href{mailto:nbeisert@itp.phys.ethz.ch}
  {\texttt{nbeisert@itp.phys.ethz.ch}}}
\hypersetup{pdfauthor={Niklas Beisert}}
\hypersetup{pdfsubject={Manual for the LaTeX2e Package childdoc}}
\date{30 December 2018, \textsf{v2.0}}
\maketitle

\begin{abstract}\noindent
\textsf{childdoc} is a \LaTeXe{} package
that enables the direct compilation
of document sections included by |\include|
to individual files.
\end{abstract}

\begingroup
\parskip0ex
\tableofcontents
\endgroup

%%%%%%%%%%%%%%%%%%%%%%%%%%%%%%%%%%%%%%%%%%%%%%%%%%%%%%%%%%%%%%%%%%%%%%%%%%%%%%%%
%%%%%%%%%%%%%%%%%%%%%%%%%%%%%%%%%%%%%%%%%%%%%%%%%%%%%%%%%%%%%%%%%%%%%%%%%%%%%%%%
\section{Introduction}

\LaTeX{} provides a mechanism to structure a large document (such as a book)
into a main file and several child files (containing the chapters)
using the |\include| command.
This mechanism is beneficial for documents
which span hundreds of pages in order to
make the source file(s) more manageable.
Moreover, compilation can be restricted to
selected child files by means of the |\includeonly| command.
The latter feature can be used to reduce the compilation time while editing
(this was significantly more useful in the earlier days of \LaTeX{})
or to generate a smaller document which is easier to navigate.
Another application of |\includeonly| is to generate
documents consisting of selected parts of the complete document.

However, there are a few drawbacks of the plain |\include| mechanism:
\begin{itemize}
\item
The child files cannot be compiled on their own,
they can only be compiled via the main file.
A naive editing environment
(such as a text editor with an option
to have the current file processed by \LaTeX)
may require one to switch to the main file before compiling;
attempting to compile the child file produces errors.
\item
The main file must be modified (each time)
to adjust the |\includeonly| command
to the present needs. This easily leaves the main file in a messy state.
\item
The generated document will always carry the filename
of the main document. This is inconvenient if
several child files are to be compiled and
to be kept for distribution.
\end{itemize}

The present package provides a simple interface
to make child files individually compilable by \LaTeX{}.
Compiling a child file then has the same effect as compiling
the main file with an |\includeonly| command
to select the appropriate child.
Moreover the generated document will carry the name of the child
rather than the main file.
This resolves all three above issues.

This feature is meant to make the editing of books,
thesis documents and lecture notes somewhat more convenient.
However, the package can also be used efficiently for
composing a series of documents (such as exercise sheets)
which are typically distributed individually.
It then assists the author in generating the individual documents
(potentially in different versions)
as well as a document containing the collected series.
Another application is in developing style files
or other kinds of included material
where compilation of the style file could redirect
to a sample or test file.

%%%%%%%%%%%%%%%%%%%%%%%%%%%%%%%%%%%%%%%%%%%%%%%%%%%%%%%%%%%%%%%%%%%%%%%%%%%%%%%%
%%%%%%%%%%%%%%%%%%%%%%%%%%%%%%%%%%%%%%%%%%%%%%%%%%%%%%%%%%%%%%%%%%%%%%%%%%%%%%%%
\section{Usage}

First of all, the package \textsf{childdoc} is \emph{not} a standard
\LaTeXe{} |.sty| style file! Therefore it needs to be invoked in
a non-standard way.

%%%%%%%%%%%%%%%%%%%%%%%%%%%%%%%%%%%%%%%%%%%%%%%%%%%%%%%%%%%%%%%%%%%%%%%%%%%%%%%%
\subsection{Included Files}
\label{sec:include}

%%%%%%%%%%%%%%%%%%%%%%%%%%%%%%%%%%%%%%%%
\DescribeMacro{\childdocmain}
To use the package, add the commands
\begin{center}
\begin{tabular}{l}
|\input{childdoc.def}|\\
|\childdocmain{}|\\
\end{tabular}
\end{center}
at the very top of the main \LaTeX{} file,
in particular \emph{before} the |\documentclass| statement!
The argument of |\childdocmain| should be left empty
(but it must be present).

%%%%%%%%%%%%%%%%%%%%%%%%%%%%%%%%%%%%%%%%
\DescribeMacro{\childdocof}
Furthermore, add the commands
\begin{center}
\begin{tabular}{l}
|\input{childdoc.def}|\\
|\childdocof{|\textit{main}|}|\\
\end{tabular}
\end{center}
at the top of every child file \textit{child}
which is included by |\include{|\textit{child}|}|
from within the main file
(or at least for those files to be compiled individually).
The argument \textit{main} must be the filename of the main file.

There are a couple of
considerations in setting up the main and child documents:

%%%%%%%%%%%%%%%%%%%%%%%%%%%%%%%%%%%%%%%%
\paragraph{Restrictions.}

Please note the following restrictions:
\begin{itemize}
\item
|\childdocmain| must be called with one argument \textit{main}
to ensure compatibility with earlier version of the package.
It must either be empty (|\childdocmain{}|)
or precisely match the filename of the main file in which it is specified.
See \secref{sec:detection} for further information.
\item
The filename \textit{main} must be specified without the |.tex| extension.
\item
The filename \textit{main} is case sensitive
(even in case-insensitive file systems)
due to internal string comparison.
\item
The argument \textit{main} should be fully expanded, it cannot be a macro.
\item
Subdirectories and special characters should be avoided in filenames.
\item
The command |\childdocmain{|\textit{main}|}| must be followed by a whitespace.
It should not be followed immediately by another command
or by a comment mark `|%|'.
This is because the \TeX{} parser reads the token immediately following
the argument of |\childdocmain| and puts it
at the beginning of every child section;
however, a white\-space is ignored.
\end{itemize}

%%%%%%%%%%%%%%%%%%%%%%%%%%%%%%%%%%%%%%%%
\paragraph{Content of Main File.}

It is advisable to place all content in the child files included by |\include|.
Any output contained in the main file will appear in all child documents
unless suppressed manually;
it cannot be suppressed automatically by the |\includeonly| directive
and thus should normally be avoided.
A method to include some content in the main file
by means of conditional processing is described in \secref{sec:conditional}.

%%%%%%%%%%%%%%%%%%%%%%%%%%%%%%%%%%%%%%%%
\paragraph{Page Numbering.}

When only a part of the document is compiled,
the appropriate numbering of pages
(as well as other status parameters)
is determined from the |.aux| files.
The latter contain information from previous passes.
However this information needs to propagate through
all intermediate child documents.
Therefore the page numbering in child documents may well
be inconsistent until the complete document is compiled at least once.

A useful (if unconventional) way to always ensure a consistent
page numbering is to restart the numbering in each child document
and denote the pages by `\textit{child}|.|\textit{page}'
where \textit{child} represents the chapter/section number of the child file.
This can be achieved by the command
|\numberwithin{page}{|\textit{child}|}|
of the \textsf{amsmath} package
where \textit{child} can be |chapter| or |section|
depending on the chosen structuring.
Alternatively, one can modify the macro |\thepage| appropriately
and reset the counter |page| at the start of each child file.

%%%%%%%%%%%%%%%%%%%%%%%%%%%%%%%%%%%%%%%%%%%%%%%%%%%%%%%%%%%%%%%%%%%%%%%%%%%%%%%%
\subsection{Conditional Processing}
\label{sec:conditional}

The package provides a mechanism to compile different versions
of a document. To customise the versions further some conditional processing
can come in handy to distinguish which version is being compiled.
The package provides two macros to describe the compilation context:

%%%%%%%%%%%%%%%%%%%%%%%%%%%%%%%%%%%%%%%%
\DescribeMacro{\ifchilddoc}
The conditional |\ifchilddoc| distinguishes between the compilation of
child documents and the main document:
%
\begin{center}
|\ifchilddoc |\textit{child-code}| |[|\||else |\textit{main-code}]| \||fi|
\end{center}

%%%%%%%%%%%%%%%%%%%%%%%%%%%%%%%%%%%%%%%%
\DescribeMacro{\childdocname}
\DescribeMacro{\childdocjob}
The macro |\childdocname| contains the filename (without extension)
of the main or child file being processed.
Note that |\childdocjob| will always contain the name of the main file.

%%%%%%%%%%%%%%%%%%%%%%%%%%%%%%%%%%%%%%%%
\paragraph{Title Page.}

Conditional processing can be used to include a title or banner page
in the main document when proper precautions are taken.
Importantly, the code in the main file should ensure that the page counter
(as well as other status parameters which are stored in the |.aux| files)
takes the same value after the conditional processing.
Otherwise the page numbers may take divergent values
depending on which part is compiled.

For example, a title page could be declared by:
%
\begin{center}
\begin{tabular}{l}
|\ifchilddoc\||else|\\
|\addtocounter{page}{-1}|\\
\textit{code for title page}\\
|\newpage|\\
|\||fi|
\end{tabular}
\end{center}
%
A banner page for the child documents can be generated by:
%
\begin{center}
\begin{tabular}{l}
|\ifchilddoc|\\
|\addtocounter{page}{-1}|\\
\textit{code for banner page}\\
|\newpage|\\
|\||fi|
\end{tabular}
\end{center}
%
Here one could write a message such as:
\begin{center}
|This is the part \childdocname{} of \childdocjob{}.|
\end{center}

%%%%%%%%%%%%%%%%%%%%%%%%%%%%%%%%%%%%%%%%%%%%%%%%%%%%%%%%%%%%%%%%%%%%%%%%%%%%%%%%
\subsection{Flags}
\label{sec:flags}

The package makes it easy to generate different versions
of the main or child documents.
To this end compilation flags can be defined
and assigned different default values.
They will be particularly useful in conjunction
with the forwarding mechanism described in \secref{sec:forward}.

For example, it may be useful to have a flag |\version|
which can be set to |draft| or |final|.
The document source will contain some conditional code
depending on the value of |\version|.
Suppose further, the flag should default to |final| for the main file
and to |draft| for child files
which is a natural assignment for editing the document.
This is achieved by placing the following code
in the preamble of the main document
(below the |\childdocmain| directive):
%
\begin{center}
\begin{tabular}{l}
|\ifchilddoc|\\
|\providecommand{\version}{draft}|\\
|\||else|\\
|\providecommand{\version}{final}|\\
|\||fi|
\end{tabular}
\end{center}
%
The definition by |\providecommand| makes sure
that previous definitions are not overwritten.
Further statements |\providecommand{\version}{...}|
can thus be added before the above code to override it.

For the main file, one might add a line
(between |\childdocmain| and the above block)
%
\begin{center}
|%\ifchilddoc\||else\providecommand{\version}{draft}\||fi|
\end{center}
%
which can be uncommented to produce a draft version.
Likewise one can add a line to the very top of a child file
(above the |\childdocof{|\textit{main}|}| directive)
%
\begin{center}
|%\providecommand{\version}{final}|
\end{center}
%
which can be uncommented to produce the final version of this child document.

%%%%%%%%%%%%%%%%%%%%%%%%%%%%%%%%%%%%%%%%%%%%%%%%%%%%%%%%%%%%%%%%%%%%%%%%%%%%%%%%
\subsection{Forwarding}
\label{sec:forward}

Different versions of the main or child documents
using compilation flags as described in \secref{sec:flags}
can be (permanently) stored in different files
for convenient compilation, viewing and distribution.
To this end, the package defines a command
to pass on compilation to a different file:

%%%%%%%%%%%%%%%%%%%%%%%%%%%%%%%%%%%%%%%%
\DescribeMacro{\childdocforward}
The command |\childdocforward| redirects processing to
another source file:
%
\begin{center}
\begin{tabular}{l}
|\input{childdoc.def}|\\
|\childdocforward[|\textit{main}|]{|\textit{dest}|}|\\
\end{tabular}
\end{center}
%
The argument \textit{dest} is the destination file
(without extension).
It should be the main file or one of the child files.
Note that further \textsf{childdoc} directives
such as |\childdocof| and |\childdocforward|
in the indicated file will be processed in this form.
The optional argument \textit{main}
passes on directly to the main file \textit{main}
while pretending to compile the child \textit{dest}.
This form behaves as if \textit{dest}
issues |\childdocof{|\textit{main}|}| right away,
and no further \textsf{childdoc} directives will be processed.

%%%%%%%%%%%%%%%%%%%%%%%%%%%%%%%%%%%%%%%%
\DescribeMacro{\...prefix}
In the alternative form |\childdocforwardprefix|,
%
\begin{center}
\begin{tabular}{l}
|\input{childdoc.def}|\\
|\childdocforwardprefix[|\textit{main}|]{|\textit{prefix}|}{|\textit{dest}|}|
\end{tabular}
\end{center}
%
the destination file is determined by a pattern
depending on the current file:
To make this work, the current file must be called
`{\textit{prefix}\hspace{0.2em}\textit{suffix}}'
with \textit{prefix} matching precisely the argument.
Processing is then passed on to the file
`{\textit{dest}\hspace{0.2em}\textit{suffix}}'.
Surely, the same effect is achieved by
directly specifying the
argument `{\textit{dest}\hspace{0.2em}\textit{suffix}}'
in the first form.
However, that requires to set up a different file
for each child. With the alternative form of the command
all these files can have exactly the same content
which simplifies setting them up and maintaining them.

For example, the following file |draft.tex|
with a compilation flag |\version| as described in \secref{sec:flags}
compiles the main document as a draft:
%
\begin{center}
\begin{tabular}{l}
|\def\version{draft}|\\
|\input{childdoc.def}|\\
|\childdocforward{|\textit{main}|}|
\end{tabular}
\end{center}
%
Likewise, the following files |final|\textit{nn}|.tex|
compile the final version of the child document
|child|\textit{nn}|.tex|:
%
\begin{center}
\begin{tabular}{l}
|\def\version{final}|\\
|\input{childdoc.def}|\\
|\childdocforwardprefix{final}{child}|
\end{tabular}
\end{center}
%

Note that when several versions of a main file and/or of each child file
are to be generated, it may be convenient to set up a |Makefile| or
shell script to automatise the process.

%%%%%%%%%%%%%%%%%%%%%%%%%%%%%%%%%%%%%%%%%%%%%%%%%%%%%%%%%%%%%%%%%%%%%%%%%%%%%%%%
\subsection{Command Line Processing}
\label{sec:commandline}

The effect of redirection files can also be achieved by invoking
the \LaTeX{} compiler with a more elaborate command line.
Most conveniently this should be done as part
of a shell script or a |Makefile|.

When using \textsf{childdoc} in the main file, the following
command lines effectively perform a redirection
(note that depending on the shell being used,
backslashes may have to be doubled: `|\|' $\to$ `|\\|'):
%
\begin{center}
|... -jobname "|\textit{target}|" |\\|"|[\textit{flags}]%
|\input{childdoc.def}\childdocforward[|\textit{main}|]{|\textit{dest}|}"|
\end{center}
%
Here \textit{target} is the name of the output file,
\textit{main} is the name of the main file
and \textit{dest} is the name of the main or child file to be processed
(all filenames without extensions).
The optional argument \textit{main} can be omitted
if \textit{main} matches \textit{dest}.
Optionally, compilation \textit{flags} can be defined via |\def| commands.
This command line makes the \TeX{} engine believe
it is compiling the file \textit{target}
whose content is specified as the latter parameter.
The provided code then forwards the processing to
\textit{main} or \textit{dest} as described in \secref{sec:forward}.

%%%%%%%%%%%%%%%%%%%%%%%%%%%%%%%%%%%%%%%%%%%%%%%%%%%%%%%%%%%%%%%%%%%%%%%%%%%%%%%%
\subsection{Include by Input}
\label{sec:input}

Including child documents by |\include| has some restrictions by design.
Most notably, the content of a child document always occupies
its own set of pages; pages cannot be shared between child documents.
Usually, this behaviour makes perfect sense
because each child document contain an essential part of the document.
However, in some situations it may be desirable to compose
a document from a collection of parts
without having mandatory page breaks between then.
For this case, the package
provides a mechanism to include parts
by |\input| which can also be processed individually.
However, by construction this mechanism
requires manual handling of the content to be output.

%%%%%%%%%%%%%%%%%%%%%%%%%%%%%%%%%%%%%%%%
\DescribeMacro{\ifchilddocmanual}
The main file should be prepared as usual, see \secref{sec:include}.
However, the document body must make a distinction
between processing of an individual part and of the main document, e.g.:
%
\begin{center}
\begin{tabular}{l}
|\ifchilddocmanual|\\
|\input{\childdocname}|\\
|\||else|\\
\textit{document body with }|\input{|\textit{part}|}|\\
|\||fi|
\end{tabular}
\end{center}
%
The conditional |\ifchilddocmanual| is true whenever
a part to be included by |\input| is being compiled,
and the name of the part is stored in |\childdocname|.

%%%%%%%%%%%%%%%%%%%%%%%%%%%%%%%%%%%%%%%%
\DescribeMacro{\childdocby}
Each part to be included by |\input| should start with:
%
\begin{center}
\begin{tabular}{l}
|\input{childdoc.def}|\\
|\childdocby{|\textit{main}|}|\\
\end{tabular}
\end{center}
%
The directive |\childdocby| is similar to |\childdocof|
described in \secref{sec:include},
but the subsequent selection of content must be done manually.
To that end, both |\ifchilddoc| and |\ifchilddocmanual|
will be true upon processing of a part,
and the name of the part is stored in |\childdocname|.
Note that |\jobname| will be set to the filename of the current part
so that each part receives an individual |.aux| file
that does not interfere with the |.aux| file(s) of the main document.
This behaviour can be altered by the alternative form
|\childdocby[*]{|\textit{main}|}| (with a non-empty optional argument)
which uses the |.aux| file of the main document
by setting |\jobname| to \textit{main}.

%%%%%%%%%%%%%%%%%%%%%%%%%%%%%%%%%%%%%%%%%%%%%%%%%%%%%%%%%%%%%%%%%%%%%%%%%%%%%%%%
\subsection{Driver Development}
\label{sec:driver}

The \textsf{childdoc} mechanism can also be use for the development
of definition files such as \LaTeX{} styles or classes.
This case differs from the above setup with multiple parts
included by |\include| in that no |\includeonly| should be invoked.
This can be achieved by starting the include file
(before |\ProvidesPackage|) with:
%
\begin{center}
\begin{tabular}{l}
|\input{childdoc.def}|\\
|\childdocforward{|\textit{main}|}|\\
\end{tabular}
\end{center}
%
or alternatively with:
%
\begin{center}
\begin{tabular}{l}
|\input{childdoc.def}|\\
|\childdocby{|\textit{main}|}|\\
\end{tabular}
\end{center}
%
Both forms have slightly different effects as described above.
The main file is prepared as usual, see \secref{sec:include}.

%%%%%%%%%%%%%%%%%%%%%%%%%%%%%%%%%%%%%%%%%%%%%%%%%%%%%%%%%%%%%%%%%%%%%%%%%%%%%%%%
\subsection{Legacy Detection}
\label{sec:detection}

The directive |\childdocmain| in the main file can detect
whether the complete document or merely a child is to be compiled
even without using the directive |\childdocof|.
This method is deprecated because it is less robust
and there is no compelling reason to use it;
it is merely provided for backward compatibility
and it may be removed in future versions.

If the detection mechanism is to be used,
it is mandatory to correctly specify
the filename of the main file as the argument of |\childdocmain|:
%
\begin{center}
\begin{tabular}{l}
|\input{childdoc.def}|\\
|\childdocmain{|\textit{main}|}|\\
\end{tabular}
\end{center}
%
If |\jobname| does not match the argument \textit{main} of |\childdocmain|,
it is assumed that |\jobname| points to the child file to be compiled.
When using |\childdocmain| with the main file specified as argument,
it suffices to start a child file
with just |\input{|\textit{main}|}|
without loading of the package and using |\childdocof|.
If instead all processing is done
with the appropriate \textsf{childdoc} directives,
the argument of \textit{main} of |\childdocmain| can be empty.

An alternative version of the command line processing described
in \secref{sec:commandline} using the detection mechanism reads:
%
\begin{center}
|... -jobname "|\textit{target}|" "|[\textit{flags}]%
[|\def\jobname{|\textit{dest}|}|]|\input{|\textit{main}|}"|
\end{center}

%%%%%%%%%%%%%%%%%%%%%%%%%%%%%%%%%%%%%%%%%%%%%%%%%%%%%%%%%%%%%%%%%%%%%%%%%%%%%%%%
\subsection{Manual Code}
\label{sec:manual}

In case one cannot be certain whether the definitions file |childdoc.def|
is installed on the target \TeX{} distribution
and one prefers not to ship it,
it is conceivable to paste a few relevant commands into the sources.

To that end, drop all statements |\input{childdoc.def}|
and perform the replacements as outlined below.
Instead of |\childdocmain{|\textit{main}|}| add the following code
to the top of the main file:
%
\begin{center}
\begin{tabular}{l}
|\||ifdefined\childdocname\endinput\||fi\newif\ifchilddoc|\\
|\edef\childdocname{\scantokens\expandafter{\jobname\noexpand}}|\\
|\def\childdocmain{|\textit{main}|}\||ifx\childdocmain\childdocname\||else|\\
|\childdoctrue\includeonly{\childdocname}\let\jobname\childdocmain\||fi|\\
\end{tabular}
\end{center}
%
Instead of |\childdocof{|\textit{main}|}| just include the main file
at the top of each child file:
%
\begin{center}
|\input{|\textit{main}|}|
\end{center}
%
A simple redirection |\childdocforward{|\textit{dest}|}| is achieved by:
%
\begin{center}
|\def\jobname{|\textit{dest}|}\input{\jobname}|
\end{center}
%
The redirection with prefix
|\childdocforwardprefix[|\textit{prefix}|]{|\textit{dest}|}|
is accomplished by:
%
\begin{center}
\begin{tabular}{l}
|{\edef\jobname{\scantokens\expandafter{\jobname\noexpand}}|\\
|\def\redirectjob |\textit{prefix}|#1~~~{\gdef\jobname{|\textit{dest}|#1}}|\\
|\expandafter\redirectjob\jobname~~~}\input{\jobname}|
\end{tabular}
\end{center}

In an alternative approach,
child documents can be compiled by a specific command line
without additional code or specific definitions:
%
\begin{center}
|... -jobname "|\textit{target}|" "|[\textit{flags}]%
|\includeonly{|\textit{dest}|}\input{|\textit{main}|}"|
\end{center}
%

%%%%%%%%%%%%%%%%%%%%%%%%%%%%%%%%%%%%%%%%%%%%%%%%%%%%%%%%%%%%%%%%%%%%%%%%%%%%%%%%
%%%%%%%%%%%%%%%%%%%%%%%%%%%%%%%%%%%%%%%%%%%%%%%%%%%%%%%%%%%%%%%%%%%%%%%%%%%%%%%%
\section{Information}

%%%%%%%%%%%%%%%%%%%%%%%%%%%%%%%%%%%%%%%%%%%%%%%%%%%%%%%%%%%%%%%%%%%%%%%%%%%%%%%%
\subsection{Copyright}

Copyright \copyright{} 2017--2018 Niklas Beisert

This work may be distributed and/or modified under the
conditions of the \LaTeX{} Project Public License, either version 1.3
of this license or (at your option) any later version.
The latest version of this license is in
  \url{http://www.latex-project.org/lppl.txt}
and version 1.3 or later is part of all distributions of \LaTeX{}
version 2005/12/01 or later.

This work has the LPPL maintenance status `maintained'.

The Current Maintainer of this work is Niklas Beisert.

This work consists of the files |README.txt|, |childdoc.ins| and |childdoc.dtx|
as well as the derived files |childdoc.def|, |cdocsamp.tex|
with |cdocsch1.tex|, |cdocsch2.tex|, |cdocspt3.tex|, |cdocspt4.tex|,
|cdocsdrf.tex|, |cdocsfn1.tex|, |cdocsfn2.tex|
as well as |childdoc.pdf|.

%%%%%%%%%%%%%%%%%%%%%%%%%%%%%%%%%%%%%%%%%%%%%%%%%%%%%%%%%%%%%%%%%%%%%%%%%%%%%%%%
\subsection{Files and Installation}

The package consists of the files:
%
\begin{center}
\begin{tabular}{ll}
    |README.txt|   & readme file \\
    |childdoc.ins| & installation file \\
    |childdoc.dtx| & source file \\
    |childdoc.def| & definition file \\
    |cdocsamp.tex| & sample main file \\
    |cdocsch1.tex| & sample include file \\
    |cdocsch2.tex| & sample include file \\
    |cdocspt3.tex| & sample part file \\
    |cdocspt4.tex| & sample part file \\
    |cdocsdrf.tex| & sample redirection file \\
    |cdocsfn1.tex| & sample redirection file \\
    |cdocsfn2.tex| & sample redirection file \\
    |childdoc.pdf| & manual
\end{tabular}
\end{center}
%
The distribution consists of the files
|README.txt|, |childdoc.ins| and |childdoc.dtx|.
%
\begin{itemize}
\item
Run (pdf)\LaTeX{} on |childdoc.dtx|
to compile the manual |childdoc.pdf| (this file).
\item
Run \LaTeX{} on |childdoc.ins| to create the definitions file |childdoc.def|
and the sample |cdocsamp.tex| with include files
|cdocsch1.tex|, |cdocsch2.tex|, |cdocspt3.tex|, |cdocspt4.tex|,
|cdocsdrf.tex|, |cdocsfn1.tex|, |cdocsfn2.tex|.
Then copy the file |childdoc.def| to an appropriate directory of your \LaTeX{}
distribution, e.g.\ \textit{texmf-root}|/tex/latex/childdoc|.
\end{itemize}

%%%%%%%%%%%%%%%%%%%%%%%%%%%%%%%%%%%%%%%%%%%%%%%%%%%%%%%%%%%%%%%%%%%%%%%%%%%%%%%%
\subsection{Related CTAN Packages}

There are several other packages which offer a similar functionality:
%
\begin{itemize}
\item
The packages
\href{http://ctan.org/pkg/docmute}{\textsf{docmute}},
\href{http://ctan.org/pkg/includex}{\textsf{includex}} and
\href{http://ctan.org/pkg/standalone}{\textsf{standalone}}
provide commands to include only the document body of
a child file thus allowing both files to be compiled individually.
\item
The packages \href{http://ctan.org/pkg/subdocs}{\textsf{subdocs}}
and \href{http://ctan.org/pkg/subfiles}{\textsf{subfiles}}
provide structures in which the main and child documents can be
encapsulated and allowing them to be compiled individually.
The inclusion mechanism is different from the conventional |\include|.
\item
The package \href{http://ctan.org/pkg/combine}{\textsf{combine}}
is an elaborate solution to combine several documents into one.
\end{itemize}
%
See also the CTAN topic \href{http://ctan.org/topic/subdocs}{\textsf{subdocs}}
for further related packages.
The present package differs from the above solutions in that
a document structure constructed with the conventional |\include| mechanism
just needs two extra commands at the top of every file
such that all constituent files can be compiled individually.

%%%%%%%%%%%%%%%%%%%%%%%%%%%%%%%%%%%%%%%%%%%%%%%%%%%%%%%%%%%%%%%%%%%%%%%%%%%%%%%%
%\subsection{Feature Suggestions}
%
%The following is a list of features which may be useful for future
%versions of this package:
%%
%\begin{itemize}
%\item
%\ldots
%\end{itemize}

%%%%%%%%%%%%%%%%%%%%%%%%%%%%%%%%%%%%%%%%%%%%%%%%%%%%%%%%%%%%%%%%%%%%%%%%%%%%%%%%
\subsection{Revision History}

%%%%%%%%%%%%%%%%%%%%%%%%%%%%%%%%%%%%%%%%
\paragraph{v2.0:} 2018/12/30

\begin{itemize}
\item
immediate forward processing
\item
added |\childdocby| mechanism
\item
manual restructured
\end{itemize}

%%%%%%%%%%%%%%%%%%%%%%%%%%%%%%%%%%%%%%%%
\paragraph{v1.6:} 2018/01/17

\begin{itemize}
\item
application for development of include files
\item
corrections to manual
\end{itemize}

%%%%%%%%%%%%%%%%%%%%%%%%%%%%%%%%%%%%%%%%
\paragraph{v1.5:} 2017/05/21

\begin{itemize}
\item
more complete structuring introduced
\item
|\childdocof| introduced
\item
|\childdoc| renamed to |\childdocmain|
\item
|\childredirect| renamed to |\childdocforward| and |\childdocforwardprefix|
and functionality expanded
\end{itemize}

%%%%%%%%%%%%%%%%%%%%%%%%%%%%%%%%%%%%%%%%
\paragraph{v1.0:} 2017/04/27

\begin{itemize}
\item
manual and install package
\item
first version published on CTAN
\end{itemize}

%%%%%%%%%%%%%%%%%%%%%%%%%%%%%%%%%%%%%%%%
\paragraph{v0.6:} 2017/04/26

\begin{itemize}
\item
redirection mechanism added
\end{itemize}

%%%%%%%%%%%%%%%%%%%%%%%%%%%%%%%%%%%%%%%%
\paragraph{v0.5:} 2017/04/26

\begin{itemize}
\item
functionality in definition file
\end{itemize}


%%%%%%%%%%%%%%%%%%%%%%%%%%%%%%%%%%%%%%%%%%%%%%%%%%%%%%%%%%%%%%%%%%%%%%%%%%%%%%%%
%%%%%%%%%%%%%%%%%%%%%%%%%%%%%%%%%%%%%%%%%%%%%%%%%%%%%%%%%%%%%%%%%%%%%%%%%%%%%%%%
%%%%%%%%%%%%%%%%%%%%%%%%%%%%%%%%%%%%%%%%%%%%%%%%%%%%%%%%%%%%%%%%%%%%%%%%%%%%%%%%
\appendix

\settowidth\MacroIndent{\rmfamily\scriptsize 000\ }

 \DocInput{childdoc.dtx}

\end{document}
%</driver>
% \fi
%
% %%%%%%%%%%%%%%%%%%%%%%%%%%%%%%%%%%%%%%%%%%%%%%%%%%%%%%%%%%%%%%%%%%%%%%%%%%%%%%
% %%%%%%%%%%%%%%%%%%%%%%%%%%%%%%%%%%%%%%%%%%%%%%%%%%%%%%%%%%%%%%%%%%%%%%%%%%%%%%
% \section{Sample}
%\iffalse
%<*samplemain>
%\fi
%
% The following presents a sample document
% with two chapters, two parts, a title page,
% a compile flag as well as three forwarding files to set the flag.
% It consists of eight |.tex| files:
% \begin{center}
% \begin{tabular}{ll}
% |cdocsamp.tex|&main file\\
% |cdocsch1.tex|&include file for chapter 1\\
% |cdocsch2.tex|&include file for chapter 2\\
% |cdocspt3.tex|&include file for part 3\\
% |cdocspt4.tex|&include file for part 4\\
% |cdocsdrf.tex|&forwarding file for main file in draft mode\\
% |cdocsfi1.tex|&forwarding file for final version of chapter 1\\
% |cdocsfi2.tex|&forwarding file for final version of chapter 2\\
% \end{tabular}
% \end{center}
% Each of the eight files can be compiled directly by the \LaTeX{} compiler.
%
% %%%%%%%%%%%%%%%%%%%%%%%%%%%%%%%%%%%%%%
% \paragraph{Main File.}
%
% The main file is called |cdocsamp.tex|.
%
% Load the \textsf{childdoc} definitions and
% declare the filename for the main document:
%    \begin{macrocode}
\input{childdoc.def}
\childdocmain{}
%    \end{macrocode}

% Optional override for |\version| flag:
%    \begin{macrocode}
%%\ifchilddoc\else\providecommand{\version}{draft}\fi
%    \end{macrocode}

% Define the default values for the |\version| flag
% (|final| for the main file and |draft| for childs):
%    \begin{macrocode}
\ifchilddoc
\providecommand{\version}{draft}
\else
\providecommand{\version}{final}
\fi
%    \end{macrocode}

% Load the standard document class:
%    \begin{macrocode}
\documentclass[12pt]{article}
%    \end{macrocode}

% Start the document body:
%    \begin{macrocode}
\begin{document}
%    \end{macrocode}

% Declare a title page.
% Print title, part of document being processed and version flag:
%    \begin{macrocode}
\addtocounter{page}{-1}
\begin{center}
{\LARGE\bfseries{}childdoc example\par}
\vspace{1cm}
\ifchilddoc
\ifchilddocmanual part\else chapter\fi:
`\childdocname' of `\childdocjob'\par
\else
main document: `\childdocjob'\par
\fi
version: \version\par
\end{center}
\newpage
%    \end{macrocode}

% Manually include selected file,
% otherwise process as usual:
%    \begin{macrocode}
\ifchilddocmanual
\section*{part `\childdocname'}
\input{\childdocname}
\else
%    \end{macrocode}

% Include the two chapters:
%    \begin{macrocode}
\include{cdocsch1}
\include{cdocsch2}
%    \end{macrocode}

% Include the two parts unless only chapters should be displayed:
%    \begin{macrocode}
\ifchilddoc\else
\section{part three}
\input{cdocspt3}
\section{part four}
\input{cdocspt4}
\fi
%    \end{macrocode}

% Process as usual until here:
%    \begin{macrocode}
\fi
%    \end{macrocode}

% End of document body:
%    \begin{macrocode}
\end{document}
%    \end{macrocode}
%\iffalse
%</samplemain>
%\fi
%
% %%%%%%%%%%%%%%%%%%%%%%%%%%%%%%%%%%%%%%
% \paragraph{Chapter Include Files.}
%
% The include files are called |cdocsch1.tex| and |cdocsch2.tex|.
%
%\iffalse
%<*samplechap1|samplechap2>
%\fi

% Optional override for |\version| flag:
%    \begin{macrocode}
%%\providecommand{\version}{final}
%    \end{macrocode}

% Include the main document:
%    \begin{macrocode}
\input{childdoc.def}
\childdocof{cdocsamp}
%    \end{macrocode}

%\iffalse
%</samplechap1|samplechap2>
%\fi
%
%\iffalse
%<*samplechap1>
%\fi
% Some text for chapter 1:
%    \begin{macrocode}
\section{one}
some text in chapter one
%    \end{macrocode}

%\iffalse
%</samplechap1>
%\fi
% Some text for chapter 2:
%\iffalse
%<*samplechap2>
%\fi
%    \begin{macrocode}
\section{two}
more text in chapter two
%    \end{macrocode}

%\iffalse
%</samplechap2>
%\fi
%
% %%%%%%%%%%%%%%%%%%%%%%%%%%%%%%%%%%%%%%
% \paragraph{Part Include Files.}
%
% The include files are called |cdocspt3.tex| and |cdocspt4.tex|.
%
%\iffalse
%<*samplepart3|samplepart4>
%\fi

% Optional override for |\version| flag:
%    \begin{macrocode}
%%\providecommand{\version}{final}
%    \end{macrocode}

% Include the main document:
%    \begin{macrocode}
\input{childdoc.def}
\childdocby{cdocsamp}
%    \end{macrocode}

%\iffalse
%</samplepart3|samplepart4>
%\fi
%
%\iffalse
%<*samplepart3>
%\fi
% Some text for part 3:
%    \begin{macrocode}
some text in part three
%    \end{macrocode}

%\iffalse
%</samplepart3>
%\fi
% Some text for part 4:
%\iffalse
%<*samplepart4>
%\fi
%    \begin{macrocode}
more text in part four
%    \end{macrocode}

%\iffalse
%</samplepart4>
%\fi
%
% %%%%%%%%%%%%%%%%%%%%%%%%%%%%%%%%%%%%%%
% \paragraph{Forwarding for a Complete Draft.}
%
% The following forwarding file |cdocsdrf.tex|
% compiles the main document in draft mode:
%\iffalse
%<*sampledraft>
%\fi
%    \begin{macrocode}
\def\version{draft}
\input{childdoc.def}
\childdocforward{cdocsamp}
%    \end{macrocode}

%\iffalse
%</sampledraft>
%\fi
%
% %%%%%%%%%%%%%%%%%%%%%%%%%%%%%%%%%%%%%%
% \paragraph{Forwarding for Final Version of the Chapters.}
%
% The following forwarding files |cdocsfn1.tex| and |cdocsfn2.tex|
% (with identical content)
% compile the final versions of the child documents
% |cdocsch1.tex| and |cdocsch2.tex|, respectively:
%\iffalse
%<*samplefinal>
%\fi
%    \begin{macrocode}
\def\version{final}
\input{childdoc.def}
\childdocforwardprefix[cdocsamp]{cdocsfn}{cdocsch}
%    \end{macrocode}

%\iffalse
%</samplefinal>
%\fi
%
% %%%%%%%%%%%%%%%%%%%%%%%%%%%%%%%%%%%%%%
% \paragraph{Command Line Processing.}
%
% The following three command lines generate the output files
% |cdocscld|, |cdocscl1| and |cdocscl2|
% which should be identical to
% |cdocsdrf|, |cdocsch1| and |cdocsfn2|, respectively:
% \begin{center}
% \begin{tabular}{l}
% |latex -jobname cdocscld \|\\
% |  "\def\version{draft}\input{childdoc.def}\childdocforward{cdocsamp}"|\\
% |latex -jobname cdocscl1 \|\\
% |  "\input{childdoc.def}\childdocforward[cdocsamp]{cdocsch1}"|\\
% |latex -jobname cdocscl2 \|\\
% |  "\def\version{final}\input{childdoc.def}\childdocforward{cdocsch2}"|
% \end{tabular}
% \end{center}
% Note that the trailing backslash on each first line
% merely continues the input to the second line
% (for convenient cut ant paste).
% Furthermore, the command |latex| can be replaced by any
% of its alternative versions such as |pdflatex|.
%
% %%%%%%%%%%%%%%%%%%%%%%%%%%%%%%%%%%%%%%%%%%%%%%%%%%%%%%%%%%%%%%%%%%%%%%%%%%%%%%
% %%%%%%%%%%%%%%%%%%%%%%%%%%%%%%%%%%%%%%%%%%%%%%%%%%%%%%%%%%%%%%%%%%%%%%%%%%%%%%
% \section{Implementation}
%\iffalse
%<*package>
%\fi
%
% This section describes the definitions file |childdoc.def|.

% The definitions cannot be loaded using |\usepackage| or |\RequirePackage|
% which has a mechanism to prevent loading a style file more than once.
% When loading the definitions by means of |\input|
% multiple instances have to be prevented manually:
%\iffalse
%This code needs to be before the `\ProvidesFile' directive
%which is defined at the beginning of this file.
%Therefore it is also placed there and commented out here.
%</package>
%<*discard>
%\fi
%    \begin{macrocode}
\ifdefined\childdocmain\endinput\fi
%    \end{macrocode}
%\iffalse
%</discard>
%<*package>
%\fi
%
% \macro{\ifchilddoc}
% \macro{\ifchilddocmanual}
% The conditional |\ifchilddoc| tells whether a
% child (true) or main (false) document is being compiled.
% The conditional |\ifchilddocmanual| tells whether
% the |\includeonly| mechanism is used (false) or
% the selection of child files must be performed manually (true).
% The definitions initialise to false:
%    \begin{macrocode}
\newif\ifchilddoc
\newif\ifchilddocmanual
%    \end{macrocode}

% \macro{\childdocname}
% \macro{\childdocjob}
% The macro |\childdocname| stores the name of the main document
% to be compiled. The macro |\childdocjob| stores the name of
% the document on which the \LaTeX{} compiler was originally invoked.
% The content of |\jobname| cannot be compared
% to filenames specified in the source due to different catcodes.
% The following code rescans |\jobname|, stores the result
% in |\childdocname| and saves a copy in |\childdocjob|:
%    \begin{macrocode}
\edef\childdocname{\scantokens\expandafter{\jobname\noexpand}}
\let\childdocjob\childdocname
%    \end{macrocode}

% \macro{\childdocdisable}
% The macro |\childdocdisable| prevents the main file
% from being processed more than once.
% At this stage, the main document command |\childdocmain|
% is assumed to be called once again where it should do nothing.
% Any subsequent call to it should prevent
% a secondary processing of the main document
% It overwrites the forwarding commands
% |\childdocof| and |\childdocforward|
% with empty macros to prevent further inclusions of the main document:
%    \begin{macrocode}
\newcommand{\childdocdisable}
{
  \renewcommand{\childdocmain}[1]{\renewcommand{\childdocmain}[1]{\endinput}}
  \renewcommand{\childdocof}[1]{}
  \renewcommand{\childdocby}[2][]{}
  \renewcommand{\childdocforward}[2][]{}
  \renewcommand{\childdocdisable}{}
}
%    \end{macrocode}

% \macro{\childdocmain}
% The macro |\childdocmain| is to be called at the top of the main file
% with nothing or the main filename (without extension) as argument.
% First, it breaks loops.
% If the argument is not empty and does not match |\childdocname|
% (which is set by the first inclusion of |childdoc.def|),
% |\ifchilddoc| is set to true, |\includeonly| is applied to the child file
% and |\jobname| is set to the main file
% (for proper handling of |.aux| files):
%    \begin{macrocode}
\newcommand{\childdocmain}[1]
{
  \childdocdisable\childdocmain{}
  \if?#1?\else
    \begingroup
      \def\childdoctmp{#1}
      \ifx\childdoctmp\childdocname
        \def\childdoctmp{}
      \else
        \def\childdoctmp
        {
          \childdoctrue
          \includeonly{\childdocname}
          \def\childdocjob{#1}
          \def\jobname{#1}
        }
      \fi
      \expandafter
    \endgroup
    \childdoctmp
  \fi
}
%    \end{macrocode}

% \macro{\childdocof}
% The command |\childdocof| redirects
% compilation to the main file |#1|.
%    \begin{macrocode}
\newcommand{\childdocof}[1]
{
  \childdocdisable
  \childdoctrue
  \includeonly{\childdocname}
  \def\jobname{#1}
  \def\childdocjob{#1}
  \input{#1}
}
%    \end{macrocode}

% \macro{\childdocby}
% The command |\childdocby| ....
%    \begin{macrocode}
\newcommand{\childdocby}[2][]
{
  \childdocdisable
  \childdoctrue
  \childdocmanualtrue
  \if?#1?\else
    \def\jobname{#2}
  \fi
  \def\childdocjob{#2}
  \input{#2}
  \endinput
}
%    \end{macrocode}

% \macro{\childdocforward}
% The command |\childdocforward| redirects
% compilation to the main file or
% (if the optional argument is given) a child file.
% Parameters are set as if the main file
% or a child file starting with |\childdocof| was compiled.
% Then compilation is handed over to the main file:
%    \begin{macrocode}
\newcommand{\childdocforward}[2][]
{
  \begingroup
    \if?#1?
      \def\childdoctmp
      {
        \def\childdocname{#2}
        \def\childdocjob{#2}
        \def\jobname{#2}
        \input{#2}
        \endinput
      }
    \else
      \def\childdoctmp
      {
        \childdocdisable
        \def\childdocname{#2}
        \childdoctrue
        \includeonly{#2}
        \def\childdocjob{#1}
        \def\jobname{#1}
        \input{#1}
        \endinput
      }
    \fi
    \expandafter
  \endgroup
  \childdoctmp
}
%    \end{macrocode}

% \macro{\childdocforwardprefix}
% The command |\childdocforwardprefix| redirects
% compilation to the main or a child file by means of a pattern.
% The prefix |#1| in the current filename is replaced by |#2|
% and the suffix of the current filename is kept
% (it is assumed that the filename does not contain the substring `|~~~|'
% which is used as a delimiter).
% Compilation is handed over to the new file by |\childdocforward|:
%    \begin{macrocode}
\newcommand{\childdocforwardprefix}[3][]
{
  \begingroup
    \def\childdocextract #2##1~~~{\def\childdoctmp{\childdocforward[#1]{#3##1}}}
    \expandafter\childdocextract\childdocname~~~
    \expandafter
  \endgroup
  \childdoctmp
}
%    \end{macrocode}

% \macro{\childdoc}
% The deprecated macro |\childdoc| is a legacy version of |\childdocmain|:
%    \begin{macrocode}
\newcommand{\childdoc}{\childdocmain}
%    \end{macrocode}

% \macro{\childdocredirect}
% The deprecated macro |\childdocredirect| is a legacy version
% of |\childdocforward| and |\childdocforwardprefix|:
%    \begin{macrocode}
\newcommand{\childdocredirect}[2][]
{
  \begingroup
    \if?#1?
      \def\childdoctmp{\childdocforward{#2}}
    \else
      \def\childdoctmp{\childdocforwardprefix{#1}{#2}}
    \fi
    \expandafter
  \endgroup
  \childdoctmp
}
%    \end{macrocode}

%\iffalse
%</package>
%\fi
%
\endinput
|\\
|\childdocof{|\textit{main}|}|\\
\end{tabular}
\end{center}
at the top of every child file \textit{child}
which is included by |\include{|\textit{child}|}|
from within the main file
(or at least for those files to be compiled individually).
The argument \textit{main} must be the filename of the main file.

There are a couple of
considerations in setting up the main and child documents:

%%%%%%%%%%%%%%%%%%%%%%%%%%%%%%%%%%%%%%%%
\paragraph{Restrictions.}

Please note the following restrictions:
\begin{itemize}
\item
|\childdocmain| must be called with one argument \textit{main}
to ensure compatibility with earlier version of the package.
It must either be empty (|\childdocmain{}|)
or precisely match the filename of the main file in which it is specified.
See \secref{sec:detection} for further information.
\item
The filename \textit{main} must be specified without the |.tex| extension.
\item
The filename \textit{main} is case sensitive
(even in case-insensitive file systems)
due to internal string comparison.
\item
The argument \textit{main} should be fully expanded, it cannot be a macro.
\item
Subdirectories and special characters should be avoided in filenames.
\item
The command |\childdocmain{|\textit{main}|}| must be followed by a whitespace.
It should not be followed immediately by another command
or by a comment mark `|%|'.
This is because the \TeX{} parser reads the token immediately following
the argument of |\childdocmain| and puts it
at the beginning of every child section;
however, a white\-space is ignored.
\end{itemize}

%%%%%%%%%%%%%%%%%%%%%%%%%%%%%%%%%%%%%%%%
\paragraph{Content of Main File.}

It is advisable to place all content in the child files included by |\include|.
Any output contained in the main file will appear in all child documents
unless suppressed manually;
it cannot be suppressed automatically by the |\includeonly| directive
and thus should normally be avoided.
A method to include some content in the main file
by means of conditional processing is described in \secref{sec:conditional}.

%%%%%%%%%%%%%%%%%%%%%%%%%%%%%%%%%%%%%%%%
\paragraph{Page Numbering.}

When only a part of the document is compiled,
the appropriate numbering of pages
(as well as other status parameters)
is determined from the |.aux| files.
The latter contain information from previous passes.
However this information needs to propagate through
all intermediate child documents.
Therefore the page numbering in child documents may well
be inconsistent until the complete document is compiled at least once.

A useful (if unconventional) way to always ensure a consistent
page numbering is to restart the numbering in each child document
and denote the pages by `\textit{child}|.|\textit{page}'
where \textit{child} represents the chapter/section number of the child file.
This can be achieved by the command
|\numberwithin{page}{|\textit{child}|}|
of the \textsf{amsmath} package
where \textit{child} can be |chapter| or |section|
depending on the chosen structuring.
Alternatively, one can modify the macro |\thepage| appropriately
and reset the counter |page| at the start of each child file.

%%%%%%%%%%%%%%%%%%%%%%%%%%%%%%%%%%%%%%%%%%%%%%%%%%%%%%%%%%%%%%%%%%%%%%%%%%%%%%%%
\subsection{Conditional Processing}
\label{sec:conditional}

The package provides a mechanism to compile different versions
of a document. To customise the versions further some conditional processing
can come in handy to distinguish which version is being compiled.
The package provides two macros to describe the compilation context:

%%%%%%%%%%%%%%%%%%%%%%%%%%%%%%%%%%%%%%%%
\DescribeMacro{\ifchilddoc}
The conditional |\ifchilddoc| distinguishes between the compilation of
child documents and the main document:
%
\begin{center}
|\ifchilddoc |\textit{child-code}| |[|\||else |\textit{main-code}]| \||fi|
\end{center}

%%%%%%%%%%%%%%%%%%%%%%%%%%%%%%%%%%%%%%%%
\DescribeMacro{\childdocname}
\DescribeMacro{\childdocjob}
The macro |\childdocname| contains the filename (without extension)
of the main or child file being processed.
Note that |\childdocjob| will always contain the name of the main file.

%%%%%%%%%%%%%%%%%%%%%%%%%%%%%%%%%%%%%%%%
\paragraph{Title Page.}

Conditional processing can be used to include a title or banner page
in the main document when proper precautions are taken.
Importantly, the code in the main file should ensure that the page counter
(as well as other status parameters which are stored in the |.aux| files)
takes the same value after the conditional processing.
Otherwise the page numbers may take divergent values
depending on which part is compiled.

For example, a title page could be declared by:
%
\begin{center}
\begin{tabular}{l}
|\ifchilddoc\||else|\\
|\addtocounter{page}{-1}|\\
\textit{code for title page}\\
|\newpage|\\
|\||fi|
\end{tabular}
\end{center}
%
A banner page for the child documents can be generated by:
%
\begin{center}
\begin{tabular}{l}
|\ifchilddoc|\\
|\addtocounter{page}{-1}|\\
\textit{code for banner page}\\
|\newpage|\\
|\||fi|
\end{tabular}
\end{center}
%
Here one could write a message such as:
\begin{center}
|This is the part \childdocname{} of \childdocjob{}.|
\end{center}

%%%%%%%%%%%%%%%%%%%%%%%%%%%%%%%%%%%%%%%%%%%%%%%%%%%%%%%%%%%%%%%%%%%%%%%%%%%%%%%%
\subsection{Flags}
\label{sec:flags}

The package makes it easy to generate different versions
of the main or child documents.
To this end compilation flags can be defined
and assigned different default values.
They will be particularly useful in conjunction
with the forwarding mechanism described in \secref{sec:forward}.

For example, it may be useful to have a flag |\version|
which can be set to |draft| or |final|.
The document source will contain some conditional code
depending on the value of |\version|.
Suppose further, the flag should default to |final| for the main file
and to |draft| for child files
which is a natural assignment for editing the document.
This is achieved by placing the following code
in the preamble of the main document
(below the |\childdocmain| directive):
%
\begin{center}
\begin{tabular}{l}
|\ifchilddoc|\\
|\providecommand{\version}{draft}|\\
|\||else|\\
|\providecommand{\version}{final}|\\
|\||fi|
\end{tabular}
\end{center}
%
The definition by |\providecommand| makes sure
that previous definitions are not overwritten.
Further statements |\providecommand{\version}{...}|
can thus be added before the above code to override it.

For the main file, one might add a line
(between |\childdocmain| and the above block)
%
\begin{center}
|%\ifchilddoc\||else\providecommand{\version}{draft}\||fi|
\end{center}
%
which can be uncommented to produce a draft version.
Likewise one can add a line to the very top of a child file
(above the |\childdocof{|\textit{main}|}| directive)
%
\begin{center}
|%\providecommand{\version}{final}|
\end{center}
%
which can be uncommented to produce the final version of this child document.

%%%%%%%%%%%%%%%%%%%%%%%%%%%%%%%%%%%%%%%%%%%%%%%%%%%%%%%%%%%%%%%%%%%%%%%%%%%%%%%%
\subsection{Forwarding}
\label{sec:forward}

Different versions of the main or child documents
using compilation flags as described in \secref{sec:flags}
can be (permanently) stored in different files
for convenient compilation, viewing and distribution.
To this end, the package defines a command
to pass on compilation to a different file:

%%%%%%%%%%%%%%%%%%%%%%%%%%%%%%%%%%%%%%%%
\DescribeMacro{\childdocforward}
The command |\childdocforward| redirects processing to
another source file:
%
\begin{center}
\begin{tabular}{l}
|% \iffalse
%
% childdoc.dtx Copyright (C) 2017-2018 Niklas Beisert
%
% This work may be distributed and/or modified under the
% conditions of the LaTeX Project Public License, either version 1.3
% of this license or (at your option) any later version.
% The latest version of this license is in
%   http://www.latex-project.org/lppl.txt
% and version 1.3 or later is part of all distributions of LaTeX
% version 2005/12/01 or later.
%
% This work has the LPPL maintenance status `maintained'.
%
% The Current Maintainer of this work is Niklas Beisert.
%
% This work consists of the files childdoc.dtx and childdoc.ins
% and the derived files childdoc.def and cdocsamp.tex with
% cdocsch1.tex, cdocsch2.tex, cdocsdrf.tex, cdocsfn1.tex, cdocsfn2.tex.
%
%<package>\ifdefined\childdocmain\endinput\fi
%<package>\ProvidesFile{childdoc.def}[2018/12/30 v2.0 child document driver]
%<samplemain>\ProvidesFile{cdocsamp.tex}[2018/12/30 v2.0 sample for childdoc]
%<*driver>
%\ProvidesFile{childdoc.drv}[2018/12/30 v2.0 childdoc reference manual file]
\PassOptionsToClass{10pt,a4paper}{article}
\documentclass{ltxdoc}

\usepackage[margin=35mm]{geometry}
\usepackage{hyperref}
\usepackage{hyperxmp}
\usepackage[usenames]{color}

\hypersetup{colorlinks=true}
\hypersetup{pdfstartview=FitH}
\hypersetup{pdfpagemode=UseNone}
\hypersetup{pdfsource={}}
\hypersetup{pdflang={en-UK}}
\hypersetup{pdfcopyright={Copyright 2017-2018 Niklas Beisert.
  This work may be distributed and/or modified under the
  conditions of the LaTeX Project Public License, either version 1.3
  of this license or (at your option) any later version.}}
\hypersetup{pdflicenseurl={http://www.latex-project.org/lppl.txt}}
\hypersetup{pdfcontactaddress={ETH Zurich, ITP, HIT K,
  Wolfgang-Pauli-Strasse 27}}
\hypersetup{pdfcontactpostcode={8093}}
\hypersetup{pdfcontactcity={Zurich}}
\hypersetup{pdfcontactcountry={Switzerland}}
\hypersetup{pdfcontactemail={nbeisert@itp.phys.ethz.ch}}
\hypersetup{pdfcontacturl={http://people.phys.ethz.ch/\xmptilde nbeisert/}}

\newcommand{\secref}[1]{\hyperref[#1]{section \ref*{#1}}}

\parskip1ex
\parindent0pt
\let\olditemize\itemize
\def\itemize{\olditemize\parskip0pt}

\begin{document}

\title{The \textsf{childdoc} Package}
\hypersetup{pdftitle={The childdoc Package}}
\author{Niklas Beisert\\[2ex]
  Institut f\"ur Theoretische Physik\\
  Eidgen\"ossische Technische Hochschule Z\"urich\\
  Wolfgang-Pauli-Strasse 27, 8093 Z\"urich, Switzerland\\[1ex]
  \href{mailto:nbeisert@itp.phys.ethz.ch}
  {\texttt{nbeisert@itp.phys.ethz.ch}}}
\hypersetup{pdfauthor={Niklas Beisert}}
\hypersetup{pdfsubject={Manual for the LaTeX2e Package childdoc}}
\date{30 December 2018, \textsf{v2.0}}
\maketitle

\begin{abstract}\noindent
\textsf{childdoc} is a \LaTeXe{} package
that enables the direct compilation
of document sections included by |\include|
to individual files.
\end{abstract}

\begingroup
\parskip0ex
\tableofcontents
\endgroup

%%%%%%%%%%%%%%%%%%%%%%%%%%%%%%%%%%%%%%%%%%%%%%%%%%%%%%%%%%%%%%%%%%%%%%%%%%%%%%%%
%%%%%%%%%%%%%%%%%%%%%%%%%%%%%%%%%%%%%%%%%%%%%%%%%%%%%%%%%%%%%%%%%%%%%%%%%%%%%%%%
\section{Introduction}

\LaTeX{} provides a mechanism to structure a large document (such as a book)
into a main file and several child files (containing the chapters)
using the |\include| command.
This mechanism is beneficial for documents
which span hundreds of pages in order to
make the source file(s) more manageable.
Moreover, compilation can be restricted to
selected child files by means of the |\includeonly| command.
The latter feature can be used to reduce the compilation time while editing
(this was significantly more useful in the earlier days of \LaTeX{})
or to generate a smaller document which is easier to navigate.
Another application of |\includeonly| is to generate
documents consisting of selected parts of the complete document.

However, there are a few drawbacks of the plain |\include| mechanism:
\begin{itemize}
\item
The child files cannot be compiled on their own,
they can only be compiled via the main file.
A naive editing environment
(such as a text editor with an option
to have the current file processed by \LaTeX)
may require one to switch to the main file before compiling;
attempting to compile the child file produces errors.
\item
The main file must be modified (each time)
to adjust the |\includeonly| command
to the present needs. This easily leaves the main file in a messy state.
\item
The generated document will always carry the filename
of the main document. This is inconvenient if
several child files are to be compiled and
to be kept for distribution.
\end{itemize}

The present package provides a simple interface
to make child files individually compilable by \LaTeX{}.
Compiling a child file then has the same effect as compiling
the main file with an |\includeonly| command
to select the appropriate child.
Moreover the generated document will carry the name of the child
rather than the main file.
This resolves all three above issues.

This feature is meant to make the editing of books,
thesis documents and lecture notes somewhat more convenient.
However, the package can also be used efficiently for
composing a series of documents (such as exercise sheets)
which are typically distributed individually.
It then assists the author in generating the individual documents
(potentially in different versions)
as well as a document containing the collected series.
Another application is in developing style files
or other kinds of included material
where compilation of the style file could redirect
to a sample or test file.

%%%%%%%%%%%%%%%%%%%%%%%%%%%%%%%%%%%%%%%%%%%%%%%%%%%%%%%%%%%%%%%%%%%%%%%%%%%%%%%%
%%%%%%%%%%%%%%%%%%%%%%%%%%%%%%%%%%%%%%%%%%%%%%%%%%%%%%%%%%%%%%%%%%%%%%%%%%%%%%%%
\section{Usage}

First of all, the package \textsf{childdoc} is \emph{not} a standard
\LaTeXe{} |.sty| style file! Therefore it needs to be invoked in
a non-standard way.

%%%%%%%%%%%%%%%%%%%%%%%%%%%%%%%%%%%%%%%%%%%%%%%%%%%%%%%%%%%%%%%%%%%%%%%%%%%%%%%%
\subsection{Included Files}
\label{sec:include}

%%%%%%%%%%%%%%%%%%%%%%%%%%%%%%%%%%%%%%%%
\DescribeMacro{\childdocmain}
To use the package, add the commands
\begin{center}
\begin{tabular}{l}
|\input{childdoc.def}|\\
|\childdocmain{}|\\
\end{tabular}
\end{center}
at the very top of the main \LaTeX{} file,
in particular \emph{before} the |\documentclass| statement!
The argument of |\childdocmain| should be left empty
(but it must be present).

%%%%%%%%%%%%%%%%%%%%%%%%%%%%%%%%%%%%%%%%
\DescribeMacro{\childdocof}
Furthermore, add the commands
\begin{center}
\begin{tabular}{l}
|\input{childdoc.def}|\\
|\childdocof{|\textit{main}|}|\\
\end{tabular}
\end{center}
at the top of every child file \textit{child}
which is included by |\include{|\textit{child}|}|
from within the main file
(or at least for those files to be compiled individually).
The argument \textit{main} must be the filename of the main file.

There are a couple of
considerations in setting up the main and child documents:

%%%%%%%%%%%%%%%%%%%%%%%%%%%%%%%%%%%%%%%%
\paragraph{Restrictions.}

Please note the following restrictions:
\begin{itemize}
\item
|\childdocmain| must be called with one argument \textit{main}
to ensure compatibility with earlier version of the package.
It must either be empty (|\childdocmain{}|)
or precisely match the filename of the main file in which it is specified.
See \secref{sec:detection} for further information.
\item
The filename \textit{main} must be specified without the |.tex| extension.
\item
The filename \textit{main} is case sensitive
(even in case-insensitive file systems)
due to internal string comparison.
\item
The argument \textit{main} should be fully expanded, it cannot be a macro.
\item
Subdirectories and special characters should be avoided in filenames.
\item
The command |\childdocmain{|\textit{main}|}| must be followed by a whitespace.
It should not be followed immediately by another command
or by a comment mark `|%|'.
This is because the \TeX{} parser reads the token immediately following
the argument of |\childdocmain| and puts it
at the beginning of every child section;
however, a white\-space is ignored.
\end{itemize}

%%%%%%%%%%%%%%%%%%%%%%%%%%%%%%%%%%%%%%%%
\paragraph{Content of Main File.}

It is advisable to place all content in the child files included by |\include|.
Any output contained in the main file will appear in all child documents
unless suppressed manually;
it cannot be suppressed automatically by the |\includeonly| directive
and thus should normally be avoided.
A method to include some content in the main file
by means of conditional processing is described in \secref{sec:conditional}.

%%%%%%%%%%%%%%%%%%%%%%%%%%%%%%%%%%%%%%%%
\paragraph{Page Numbering.}

When only a part of the document is compiled,
the appropriate numbering of pages
(as well as other status parameters)
is determined from the |.aux| files.
The latter contain information from previous passes.
However this information needs to propagate through
all intermediate child documents.
Therefore the page numbering in child documents may well
be inconsistent until the complete document is compiled at least once.

A useful (if unconventional) way to always ensure a consistent
page numbering is to restart the numbering in each child document
and denote the pages by `\textit{child}|.|\textit{page}'
where \textit{child} represents the chapter/section number of the child file.
This can be achieved by the command
|\numberwithin{page}{|\textit{child}|}|
of the \textsf{amsmath} package
where \textit{child} can be |chapter| or |section|
depending on the chosen structuring.
Alternatively, one can modify the macro |\thepage| appropriately
and reset the counter |page| at the start of each child file.

%%%%%%%%%%%%%%%%%%%%%%%%%%%%%%%%%%%%%%%%%%%%%%%%%%%%%%%%%%%%%%%%%%%%%%%%%%%%%%%%
\subsection{Conditional Processing}
\label{sec:conditional}

The package provides a mechanism to compile different versions
of a document. To customise the versions further some conditional processing
can come in handy to distinguish which version is being compiled.
The package provides two macros to describe the compilation context:

%%%%%%%%%%%%%%%%%%%%%%%%%%%%%%%%%%%%%%%%
\DescribeMacro{\ifchilddoc}
The conditional |\ifchilddoc| distinguishes between the compilation of
child documents and the main document:
%
\begin{center}
|\ifchilddoc |\textit{child-code}| |[|\||else |\textit{main-code}]| \||fi|
\end{center}

%%%%%%%%%%%%%%%%%%%%%%%%%%%%%%%%%%%%%%%%
\DescribeMacro{\childdocname}
\DescribeMacro{\childdocjob}
The macro |\childdocname| contains the filename (without extension)
of the main or child file being processed.
Note that |\childdocjob| will always contain the name of the main file.

%%%%%%%%%%%%%%%%%%%%%%%%%%%%%%%%%%%%%%%%
\paragraph{Title Page.}

Conditional processing can be used to include a title or banner page
in the main document when proper precautions are taken.
Importantly, the code in the main file should ensure that the page counter
(as well as other status parameters which are stored in the |.aux| files)
takes the same value after the conditional processing.
Otherwise the page numbers may take divergent values
depending on which part is compiled.

For example, a title page could be declared by:
%
\begin{center}
\begin{tabular}{l}
|\ifchilddoc\||else|\\
|\addtocounter{page}{-1}|\\
\textit{code for title page}\\
|\newpage|\\
|\||fi|
\end{tabular}
\end{center}
%
A banner page for the child documents can be generated by:
%
\begin{center}
\begin{tabular}{l}
|\ifchilddoc|\\
|\addtocounter{page}{-1}|\\
\textit{code for banner page}\\
|\newpage|\\
|\||fi|
\end{tabular}
\end{center}
%
Here one could write a message such as:
\begin{center}
|This is the part \childdocname{} of \childdocjob{}.|
\end{center}

%%%%%%%%%%%%%%%%%%%%%%%%%%%%%%%%%%%%%%%%%%%%%%%%%%%%%%%%%%%%%%%%%%%%%%%%%%%%%%%%
\subsection{Flags}
\label{sec:flags}

The package makes it easy to generate different versions
of the main or child documents.
To this end compilation flags can be defined
and assigned different default values.
They will be particularly useful in conjunction
with the forwarding mechanism described in \secref{sec:forward}.

For example, it may be useful to have a flag |\version|
which can be set to |draft| or |final|.
The document source will contain some conditional code
depending on the value of |\version|.
Suppose further, the flag should default to |final| for the main file
and to |draft| for child files
which is a natural assignment for editing the document.
This is achieved by placing the following code
in the preamble of the main document
(below the |\childdocmain| directive):
%
\begin{center}
\begin{tabular}{l}
|\ifchilddoc|\\
|\providecommand{\version}{draft}|\\
|\||else|\\
|\providecommand{\version}{final}|\\
|\||fi|
\end{tabular}
\end{center}
%
The definition by |\providecommand| makes sure
that previous definitions are not overwritten.
Further statements |\providecommand{\version}{...}|
can thus be added before the above code to override it.

For the main file, one might add a line
(between |\childdocmain| and the above block)
%
\begin{center}
|%\ifchilddoc\||else\providecommand{\version}{draft}\||fi|
\end{center}
%
which can be uncommented to produce a draft version.
Likewise one can add a line to the very top of a child file
(above the |\childdocof{|\textit{main}|}| directive)
%
\begin{center}
|%\providecommand{\version}{final}|
\end{center}
%
which can be uncommented to produce the final version of this child document.

%%%%%%%%%%%%%%%%%%%%%%%%%%%%%%%%%%%%%%%%%%%%%%%%%%%%%%%%%%%%%%%%%%%%%%%%%%%%%%%%
\subsection{Forwarding}
\label{sec:forward}

Different versions of the main or child documents
using compilation flags as described in \secref{sec:flags}
can be (permanently) stored in different files
for convenient compilation, viewing and distribution.
To this end, the package defines a command
to pass on compilation to a different file:

%%%%%%%%%%%%%%%%%%%%%%%%%%%%%%%%%%%%%%%%
\DescribeMacro{\childdocforward}
The command |\childdocforward| redirects processing to
another source file:
%
\begin{center}
\begin{tabular}{l}
|\input{childdoc.def}|\\
|\childdocforward[|\textit{main}|]{|\textit{dest}|}|\\
\end{tabular}
\end{center}
%
The argument \textit{dest} is the destination file
(without extension).
It should be the main file or one of the child files.
Note that further \textsf{childdoc} directives
such as |\childdocof| and |\childdocforward|
in the indicated file will be processed in this form.
The optional argument \textit{main}
passes on directly to the main file \textit{main}
while pretending to compile the child \textit{dest}.
This form behaves as if \textit{dest}
issues |\childdocof{|\textit{main}|}| right away,
and no further \textsf{childdoc} directives will be processed.

%%%%%%%%%%%%%%%%%%%%%%%%%%%%%%%%%%%%%%%%
\DescribeMacro{\...prefix}
In the alternative form |\childdocforwardprefix|,
%
\begin{center}
\begin{tabular}{l}
|\input{childdoc.def}|\\
|\childdocforwardprefix[|\textit{main}|]{|\textit{prefix}|}{|\textit{dest}|}|
\end{tabular}
\end{center}
%
the destination file is determined by a pattern
depending on the current file:
To make this work, the current file must be called
`{\textit{prefix}\hspace{0.2em}\textit{suffix}}'
with \textit{prefix} matching precisely the argument.
Processing is then passed on to the file
`{\textit{dest}\hspace{0.2em}\textit{suffix}}'.
Surely, the same effect is achieved by
directly specifying the
argument `{\textit{dest}\hspace{0.2em}\textit{suffix}}'
in the first form.
However, that requires to set up a different file
for each child. With the alternative form of the command
all these files can have exactly the same content
which simplifies setting them up and maintaining them.

For example, the following file |draft.tex|
with a compilation flag |\version| as described in \secref{sec:flags}
compiles the main document as a draft:
%
\begin{center}
\begin{tabular}{l}
|\def\version{draft}|\\
|\input{childdoc.def}|\\
|\childdocforward{|\textit{main}|}|
\end{tabular}
\end{center}
%
Likewise, the following files |final|\textit{nn}|.tex|
compile the final version of the child document
|child|\textit{nn}|.tex|:
%
\begin{center}
\begin{tabular}{l}
|\def\version{final}|\\
|\input{childdoc.def}|\\
|\childdocforwardprefix{final}{child}|
\end{tabular}
\end{center}
%

Note that when several versions of a main file and/or of each child file
are to be generated, it may be convenient to set up a |Makefile| or
shell script to automatise the process.

%%%%%%%%%%%%%%%%%%%%%%%%%%%%%%%%%%%%%%%%%%%%%%%%%%%%%%%%%%%%%%%%%%%%%%%%%%%%%%%%
\subsection{Command Line Processing}
\label{sec:commandline}

The effect of redirection files can also be achieved by invoking
the \LaTeX{} compiler with a more elaborate command line.
Most conveniently this should be done as part
of a shell script or a |Makefile|.

When using \textsf{childdoc} in the main file, the following
command lines effectively perform a redirection
(note that depending on the shell being used,
backslashes may have to be doubled: `|\|' $\to$ `|\\|'):
%
\begin{center}
|... -jobname "|\textit{target}|" |\\|"|[\textit{flags}]%
|\input{childdoc.def}\childdocforward[|\textit{main}|]{|\textit{dest}|}"|
\end{center}
%
Here \textit{target} is the name of the output file,
\textit{main} is the name of the main file
and \textit{dest} is the name of the main or child file to be processed
(all filenames without extensions).
The optional argument \textit{main} can be omitted
if \textit{main} matches \textit{dest}.
Optionally, compilation \textit{flags} can be defined via |\def| commands.
This command line makes the \TeX{} engine believe
it is compiling the file \textit{target}
whose content is specified as the latter parameter.
The provided code then forwards the processing to
\textit{main} or \textit{dest} as described in \secref{sec:forward}.

%%%%%%%%%%%%%%%%%%%%%%%%%%%%%%%%%%%%%%%%%%%%%%%%%%%%%%%%%%%%%%%%%%%%%%%%%%%%%%%%
\subsection{Include by Input}
\label{sec:input}

Including child documents by |\include| has some restrictions by design.
Most notably, the content of a child document always occupies
its own set of pages; pages cannot be shared between child documents.
Usually, this behaviour makes perfect sense
because each child document contain an essential part of the document.
However, in some situations it may be desirable to compose
a document from a collection of parts
without having mandatory page breaks between then.
For this case, the package
provides a mechanism to include parts
by |\input| which can also be processed individually.
However, by construction this mechanism
requires manual handling of the content to be output.

%%%%%%%%%%%%%%%%%%%%%%%%%%%%%%%%%%%%%%%%
\DescribeMacro{\ifchilddocmanual}
The main file should be prepared as usual, see \secref{sec:include}.
However, the document body must make a distinction
between processing of an individual part and of the main document, e.g.:
%
\begin{center}
\begin{tabular}{l}
|\ifchilddocmanual|\\
|\input{\childdocname}|\\
|\||else|\\
\textit{document body with }|\input{|\textit{part}|}|\\
|\||fi|
\end{tabular}
\end{center}
%
The conditional |\ifchilddocmanual| is true whenever
a part to be included by |\input| is being compiled,
and the name of the part is stored in |\childdocname|.

%%%%%%%%%%%%%%%%%%%%%%%%%%%%%%%%%%%%%%%%
\DescribeMacro{\childdocby}
Each part to be included by |\input| should start with:
%
\begin{center}
\begin{tabular}{l}
|\input{childdoc.def}|\\
|\childdocby{|\textit{main}|}|\\
\end{tabular}
\end{center}
%
The directive |\childdocby| is similar to |\childdocof|
described in \secref{sec:include},
but the subsequent selection of content must be done manually.
To that end, both |\ifchilddoc| and |\ifchilddocmanual|
will be true upon processing of a part,
and the name of the part is stored in |\childdocname|.
Note that |\jobname| will be set to the filename of the current part
so that each part receives an individual |.aux| file
that does not interfere with the |.aux| file(s) of the main document.
This behaviour can be altered by the alternative form
|\childdocby[*]{|\textit{main}|}| (with a non-empty optional argument)
which uses the |.aux| file of the main document
by setting |\jobname| to \textit{main}.

%%%%%%%%%%%%%%%%%%%%%%%%%%%%%%%%%%%%%%%%%%%%%%%%%%%%%%%%%%%%%%%%%%%%%%%%%%%%%%%%
\subsection{Driver Development}
\label{sec:driver}

The \textsf{childdoc} mechanism can also be use for the development
of definition files such as \LaTeX{} styles or classes.
This case differs from the above setup with multiple parts
included by |\include| in that no |\includeonly| should be invoked.
This can be achieved by starting the include file
(before |\ProvidesPackage|) with:
%
\begin{center}
\begin{tabular}{l}
|\input{childdoc.def}|\\
|\childdocforward{|\textit{main}|}|\\
\end{tabular}
\end{center}
%
or alternatively with:
%
\begin{center}
\begin{tabular}{l}
|\input{childdoc.def}|\\
|\childdocby{|\textit{main}|}|\\
\end{tabular}
\end{center}
%
Both forms have slightly different effects as described above.
The main file is prepared as usual, see \secref{sec:include}.

%%%%%%%%%%%%%%%%%%%%%%%%%%%%%%%%%%%%%%%%%%%%%%%%%%%%%%%%%%%%%%%%%%%%%%%%%%%%%%%%
\subsection{Legacy Detection}
\label{sec:detection}

The directive |\childdocmain| in the main file can detect
whether the complete document or merely a child is to be compiled
even without using the directive |\childdocof|.
This method is deprecated because it is less robust
and there is no compelling reason to use it;
it is merely provided for backward compatibility
and it may be removed in future versions.

If the detection mechanism is to be used,
it is mandatory to correctly specify
the filename of the main file as the argument of |\childdocmain|:
%
\begin{center}
\begin{tabular}{l}
|\input{childdoc.def}|\\
|\childdocmain{|\textit{main}|}|\\
\end{tabular}
\end{center}
%
If |\jobname| does not match the argument \textit{main} of |\childdocmain|,
it is assumed that |\jobname| points to the child file to be compiled.
When using |\childdocmain| with the main file specified as argument,
it suffices to start a child file
with just |\input{|\textit{main}|}|
without loading of the package and using |\childdocof|.
If instead all processing is done
with the appropriate \textsf{childdoc} directives,
the argument of \textit{main} of |\childdocmain| can be empty.

An alternative version of the command line processing described
in \secref{sec:commandline} using the detection mechanism reads:
%
\begin{center}
|... -jobname "|\textit{target}|" "|[\textit{flags}]%
[|\def\jobname{|\textit{dest}|}|]|\input{|\textit{main}|}"|
\end{center}

%%%%%%%%%%%%%%%%%%%%%%%%%%%%%%%%%%%%%%%%%%%%%%%%%%%%%%%%%%%%%%%%%%%%%%%%%%%%%%%%
\subsection{Manual Code}
\label{sec:manual}

In case one cannot be certain whether the definitions file |childdoc.def|
is installed on the target \TeX{} distribution
and one prefers not to ship it,
it is conceivable to paste a few relevant commands into the sources.

To that end, drop all statements |\input{childdoc.def}|
and perform the replacements as outlined below.
Instead of |\childdocmain{|\textit{main}|}| add the following code
to the top of the main file:
%
\begin{center}
\begin{tabular}{l}
|\||ifdefined\childdocname\endinput\||fi\newif\ifchilddoc|\\
|\edef\childdocname{\scantokens\expandafter{\jobname\noexpand}}|\\
|\def\childdocmain{|\textit{main}|}\||ifx\childdocmain\childdocname\||else|\\
|\childdoctrue\includeonly{\childdocname}\let\jobname\childdocmain\||fi|\\
\end{tabular}
\end{center}
%
Instead of |\childdocof{|\textit{main}|}| just include the main file
at the top of each child file:
%
\begin{center}
|\input{|\textit{main}|}|
\end{center}
%
A simple redirection |\childdocforward{|\textit{dest}|}| is achieved by:
%
\begin{center}
|\def\jobname{|\textit{dest}|}\input{\jobname}|
\end{center}
%
The redirection with prefix
|\childdocforwardprefix[|\textit{prefix}|]{|\textit{dest}|}|
is accomplished by:
%
\begin{center}
\begin{tabular}{l}
|{\edef\jobname{\scantokens\expandafter{\jobname\noexpand}}|\\
|\def\redirectjob |\textit{prefix}|#1~~~{\gdef\jobname{|\textit{dest}|#1}}|\\
|\expandafter\redirectjob\jobname~~~}\input{\jobname}|
\end{tabular}
\end{center}

In an alternative approach,
child documents can be compiled by a specific command line
without additional code or specific definitions:
%
\begin{center}
|... -jobname "|\textit{target}|" "|[\textit{flags}]%
|\includeonly{|\textit{dest}|}\input{|\textit{main}|}"|
\end{center}
%

%%%%%%%%%%%%%%%%%%%%%%%%%%%%%%%%%%%%%%%%%%%%%%%%%%%%%%%%%%%%%%%%%%%%%%%%%%%%%%%%
%%%%%%%%%%%%%%%%%%%%%%%%%%%%%%%%%%%%%%%%%%%%%%%%%%%%%%%%%%%%%%%%%%%%%%%%%%%%%%%%
\section{Information}

%%%%%%%%%%%%%%%%%%%%%%%%%%%%%%%%%%%%%%%%%%%%%%%%%%%%%%%%%%%%%%%%%%%%%%%%%%%%%%%%
\subsection{Copyright}

Copyright \copyright{} 2017--2018 Niklas Beisert

This work may be distributed and/or modified under the
conditions of the \LaTeX{} Project Public License, either version 1.3
of this license or (at your option) any later version.
The latest version of this license is in
  \url{http://www.latex-project.org/lppl.txt}
and version 1.3 or later is part of all distributions of \LaTeX{}
version 2005/12/01 or later.

This work has the LPPL maintenance status `maintained'.

The Current Maintainer of this work is Niklas Beisert.

This work consists of the files |README.txt|, |childdoc.ins| and |childdoc.dtx|
as well as the derived files |childdoc.def|, |cdocsamp.tex|
with |cdocsch1.tex|, |cdocsch2.tex|, |cdocspt3.tex|, |cdocspt4.tex|,
|cdocsdrf.tex|, |cdocsfn1.tex|, |cdocsfn2.tex|
as well as |childdoc.pdf|.

%%%%%%%%%%%%%%%%%%%%%%%%%%%%%%%%%%%%%%%%%%%%%%%%%%%%%%%%%%%%%%%%%%%%%%%%%%%%%%%%
\subsection{Files and Installation}

The package consists of the files:
%
\begin{center}
\begin{tabular}{ll}
    |README.txt|   & readme file \\
    |childdoc.ins| & installation file \\
    |childdoc.dtx| & source file \\
    |childdoc.def| & definition file \\
    |cdocsamp.tex| & sample main file \\
    |cdocsch1.tex| & sample include file \\
    |cdocsch2.tex| & sample include file \\
    |cdocspt3.tex| & sample part file \\
    |cdocspt4.tex| & sample part file \\
    |cdocsdrf.tex| & sample redirection file \\
    |cdocsfn1.tex| & sample redirection file \\
    |cdocsfn2.tex| & sample redirection file \\
    |childdoc.pdf| & manual
\end{tabular}
\end{center}
%
The distribution consists of the files
|README.txt|, |childdoc.ins| and |childdoc.dtx|.
%
\begin{itemize}
\item
Run (pdf)\LaTeX{} on |childdoc.dtx|
to compile the manual |childdoc.pdf| (this file).
\item
Run \LaTeX{} on |childdoc.ins| to create the definitions file |childdoc.def|
and the sample |cdocsamp.tex| with include files
|cdocsch1.tex|, |cdocsch2.tex|, |cdocspt3.tex|, |cdocspt4.tex|,
|cdocsdrf.tex|, |cdocsfn1.tex|, |cdocsfn2.tex|.
Then copy the file |childdoc.def| to an appropriate directory of your \LaTeX{}
distribution, e.g.\ \textit{texmf-root}|/tex/latex/childdoc|.
\end{itemize}

%%%%%%%%%%%%%%%%%%%%%%%%%%%%%%%%%%%%%%%%%%%%%%%%%%%%%%%%%%%%%%%%%%%%%%%%%%%%%%%%
\subsection{Related CTAN Packages}

There are several other packages which offer a similar functionality:
%
\begin{itemize}
\item
The packages
\href{http://ctan.org/pkg/docmute}{\textsf{docmute}},
\href{http://ctan.org/pkg/includex}{\textsf{includex}} and
\href{http://ctan.org/pkg/standalone}{\textsf{standalone}}
provide commands to include only the document body of
a child file thus allowing both files to be compiled individually.
\item
The packages \href{http://ctan.org/pkg/subdocs}{\textsf{subdocs}}
and \href{http://ctan.org/pkg/subfiles}{\textsf{subfiles}}
provide structures in which the main and child documents can be
encapsulated and allowing them to be compiled individually.
The inclusion mechanism is different from the conventional |\include|.
\item
The package \href{http://ctan.org/pkg/combine}{\textsf{combine}}
is an elaborate solution to combine several documents into one.
\end{itemize}
%
See also the CTAN topic \href{http://ctan.org/topic/subdocs}{\textsf{subdocs}}
for further related packages.
The present package differs from the above solutions in that
a document structure constructed with the conventional |\include| mechanism
just needs two extra commands at the top of every file
such that all constituent files can be compiled individually.

%%%%%%%%%%%%%%%%%%%%%%%%%%%%%%%%%%%%%%%%%%%%%%%%%%%%%%%%%%%%%%%%%%%%%%%%%%%%%%%%
%\subsection{Feature Suggestions}
%
%The following is a list of features which may be useful for future
%versions of this package:
%%
%\begin{itemize}
%\item
%\ldots
%\end{itemize}

%%%%%%%%%%%%%%%%%%%%%%%%%%%%%%%%%%%%%%%%%%%%%%%%%%%%%%%%%%%%%%%%%%%%%%%%%%%%%%%%
\subsection{Revision History}

%%%%%%%%%%%%%%%%%%%%%%%%%%%%%%%%%%%%%%%%
\paragraph{v2.0:} 2018/12/30

\begin{itemize}
\item
immediate forward processing
\item
added |\childdocby| mechanism
\item
manual restructured
\end{itemize}

%%%%%%%%%%%%%%%%%%%%%%%%%%%%%%%%%%%%%%%%
\paragraph{v1.6:} 2018/01/17

\begin{itemize}
\item
application for development of include files
\item
corrections to manual
\end{itemize}

%%%%%%%%%%%%%%%%%%%%%%%%%%%%%%%%%%%%%%%%
\paragraph{v1.5:} 2017/05/21

\begin{itemize}
\item
more complete structuring introduced
\item
|\childdocof| introduced
\item
|\childdoc| renamed to |\childdocmain|
\item
|\childredirect| renamed to |\childdocforward| and |\childdocforwardprefix|
and functionality expanded
\end{itemize}

%%%%%%%%%%%%%%%%%%%%%%%%%%%%%%%%%%%%%%%%
\paragraph{v1.0:} 2017/04/27

\begin{itemize}
\item
manual and install package
\item
first version published on CTAN
\end{itemize}

%%%%%%%%%%%%%%%%%%%%%%%%%%%%%%%%%%%%%%%%
\paragraph{v0.6:} 2017/04/26

\begin{itemize}
\item
redirection mechanism added
\end{itemize}

%%%%%%%%%%%%%%%%%%%%%%%%%%%%%%%%%%%%%%%%
\paragraph{v0.5:} 2017/04/26

\begin{itemize}
\item
functionality in definition file
\end{itemize}


%%%%%%%%%%%%%%%%%%%%%%%%%%%%%%%%%%%%%%%%%%%%%%%%%%%%%%%%%%%%%%%%%%%%%%%%%%%%%%%%
%%%%%%%%%%%%%%%%%%%%%%%%%%%%%%%%%%%%%%%%%%%%%%%%%%%%%%%%%%%%%%%%%%%%%%%%%%%%%%%%
%%%%%%%%%%%%%%%%%%%%%%%%%%%%%%%%%%%%%%%%%%%%%%%%%%%%%%%%%%%%%%%%%%%%%%%%%%%%%%%%
\appendix

\settowidth\MacroIndent{\rmfamily\scriptsize 000\ }

 \DocInput{childdoc.dtx}

\end{document}
%</driver>
% \fi
%
% %%%%%%%%%%%%%%%%%%%%%%%%%%%%%%%%%%%%%%%%%%%%%%%%%%%%%%%%%%%%%%%%%%%%%%%%%%%%%%
% %%%%%%%%%%%%%%%%%%%%%%%%%%%%%%%%%%%%%%%%%%%%%%%%%%%%%%%%%%%%%%%%%%%%%%%%%%%%%%
% \section{Sample}
%\iffalse
%<*samplemain>
%\fi
%
% The following presents a sample document
% with two chapters, two parts, a title page,
% a compile flag as well as three forwarding files to set the flag.
% It consists of eight |.tex| files:
% \begin{center}
% \begin{tabular}{ll}
% |cdocsamp.tex|&main file\\
% |cdocsch1.tex|&include file for chapter 1\\
% |cdocsch2.tex|&include file for chapter 2\\
% |cdocspt3.tex|&include file for part 3\\
% |cdocspt4.tex|&include file for part 4\\
% |cdocsdrf.tex|&forwarding file for main file in draft mode\\
% |cdocsfi1.tex|&forwarding file for final version of chapter 1\\
% |cdocsfi2.tex|&forwarding file for final version of chapter 2\\
% \end{tabular}
% \end{center}
% Each of the eight files can be compiled directly by the \LaTeX{} compiler.
%
% %%%%%%%%%%%%%%%%%%%%%%%%%%%%%%%%%%%%%%
% \paragraph{Main File.}
%
% The main file is called |cdocsamp.tex|.
%
% Load the \textsf{childdoc} definitions and
% declare the filename for the main document:
%    \begin{macrocode}
\input{childdoc.def}
\childdocmain{}
%    \end{macrocode}

% Optional override for |\version| flag:
%    \begin{macrocode}
%%\ifchilddoc\else\providecommand{\version}{draft}\fi
%    \end{macrocode}

% Define the default values for the |\version| flag
% (|final| for the main file and |draft| for childs):
%    \begin{macrocode}
\ifchilddoc
\providecommand{\version}{draft}
\else
\providecommand{\version}{final}
\fi
%    \end{macrocode}

% Load the standard document class:
%    \begin{macrocode}
\documentclass[12pt]{article}
%    \end{macrocode}

% Start the document body:
%    \begin{macrocode}
\begin{document}
%    \end{macrocode}

% Declare a title page.
% Print title, part of document being processed and version flag:
%    \begin{macrocode}
\addtocounter{page}{-1}
\begin{center}
{\LARGE\bfseries{}childdoc example\par}
\vspace{1cm}
\ifchilddoc
\ifchilddocmanual part\else chapter\fi:
`\childdocname' of `\childdocjob'\par
\else
main document: `\childdocjob'\par
\fi
version: \version\par
\end{center}
\newpage
%    \end{macrocode}

% Manually include selected file,
% otherwise process as usual:
%    \begin{macrocode}
\ifchilddocmanual
\section*{part `\childdocname'}
\input{\childdocname}
\else
%    \end{macrocode}

% Include the two chapters:
%    \begin{macrocode}
\include{cdocsch1}
\include{cdocsch2}
%    \end{macrocode}

% Include the two parts unless only chapters should be displayed:
%    \begin{macrocode}
\ifchilddoc\else
\section{part three}
\input{cdocspt3}
\section{part four}
\input{cdocspt4}
\fi
%    \end{macrocode}

% Process as usual until here:
%    \begin{macrocode}
\fi
%    \end{macrocode}

% End of document body:
%    \begin{macrocode}
\end{document}
%    \end{macrocode}
%\iffalse
%</samplemain>
%\fi
%
% %%%%%%%%%%%%%%%%%%%%%%%%%%%%%%%%%%%%%%
% \paragraph{Chapter Include Files.}
%
% The include files are called |cdocsch1.tex| and |cdocsch2.tex|.
%
%\iffalse
%<*samplechap1|samplechap2>
%\fi

% Optional override for |\version| flag:
%    \begin{macrocode}
%%\providecommand{\version}{final}
%    \end{macrocode}

% Include the main document:
%    \begin{macrocode}
\input{childdoc.def}
\childdocof{cdocsamp}
%    \end{macrocode}

%\iffalse
%</samplechap1|samplechap2>
%\fi
%
%\iffalse
%<*samplechap1>
%\fi
% Some text for chapter 1:
%    \begin{macrocode}
\section{one}
some text in chapter one
%    \end{macrocode}

%\iffalse
%</samplechap1>
%\fi
% Some text for chapter 2:
%\iffalse
%<*samplechap2>
%\fi
%    \begin{macrocode}
\section{two}
more text in chapter two
%    \end{macrocode}

%\iffalse
%</samplechap2>
%\fi
%
% %%%%%%%%%%%%%%%%%%%%%%%%%%%%%%%%%%%%%%
% \paragraph{Part Include Files.}
%
% The include files are called |cdocspt3.tex| and |cdocspt4.tex|.
%
%\iffalse
%<*samplepart3|samplepart4>
%\fi

% Optional override for |\version| flag:
%    \begin{macrocode}
%%\providecommand{\version}{final}
%    \end{macrocode}

% Include the main document:
%    \begin{macrocode}
\input{childdoc.def}
\childdocby{cdocsamp}
%    \end{macrocode}

%\iffalse
%</samplepart3|samplepart4>
%\fi
%
%\iffalse
%<*samplepart3>
%\fi
% Some text for part 3:
%    \begin{macrocode}
some text in part three
%    \end{macrocode}

%\iffalse
%</samplepart3>
%\fi
% Some text for part 4:
%\iffalse
%<*samplepart4>
%\fi
%    \begin{macrocode}
more text in part four
%    \end{macrocode}

%\iffalse
%</samplepart4>
%\fi
%
% %%%%%%%%%%%%%%%%%%%%%%%%%%%%%%%%%%%%%%
% \paragraph{Forwarding for a Complete Draft.}
%
% The following forwarding file |cdocsdrf.tex|
% compiles the main document in draft mode:
%\iffalse
%<*sampledraft>
%\fi
%    \begin{macrocode}
\def\version{draft}
\input{childdoc.def}
\childdocforward{cdocsamp}
%    \end{macrocode}

%\iffalse
%</sampledraft>
%\fi
%
% %%%%%%%%%%%%%%%%%%%%%%%%%%%%%%%%%%%%%%
% \paragraph{Forwarding for Final Version of the Chapters.}
%
% The following forwarding files |cdocsfn1.tex| and |cdocsfn2.tex|
% (with identical content)
% compile the final versions of the child documents
% |cdocsch1.tex| and |cdocsch2.tex|, respectively:
%\iffalse
%<*samplefinal>
%\fi
%    \begin{macrocode}
\def\version{final}
\input{childdoc.def}
\childdocforwardprefix[cdocsamp]{cdocsfn}{cdocsch}
%    \end{macrocode}

%\iffalse
%</samplefinal>
%\fi
%
% %%%%%%%%%%%%%%%%%%%%%%%%%%%%%%%%%%%%%%
% \paragraph{Command Line Processing.}
%
% The following three command lines generate the output files
% |cdocscld|, |cdocscl1| and |cdocscl2|
% which should be identical to
% |cdocsdrf|, |cdocsch1| and |cdocsfn2|, respectively:
% \begin{center}
% \begin{tabular}{l}
% |latex -jobname cdocscld \|\\
% |  "\def\version{draft}\input{childdoc.def}\childdocforward{cdocsamp}"|\\
% |latex -jobname cdocscl1 \|\\
% |  "\input{childdoc.def}\childdocforward[cdocsamp]{cdocsch1}"|\\
% |latex -jobname cdocscl2 \|\\
% |  "\def\version{final}\input{childdoc.def}\childdocforward{cdocsch2}"|
% \end{tabular}
% \end{center}
% Note that the trailing backslash on each first line
% merely continues the input to the second line
% (for convenient cut ant paste).
% Furthermore, the command |latex| can be replaced by any
% of its alternative versions such as |pdflatex|.
%
% %%%%%%%%%%%%%%%%%%%%%%%%%%%%%%%%%%%%%%%%%%%%%%%%%%%%%%%%%%%%%%%%%%%%%%%%%%%%%%
% %%%%%%%%%%%%%%%%%%%%%%%%%%%%%%%%%%%%%%%%%%%%%%%%%%%%%%%%%%%%%%%%%%%%%%%%%%%%%%
% \section{Implementation}
%\iffalse
%<*package>
%\fi
%
% This section describes the definitions file |childdoc.def|.

% The definitions cannot be loaded using |\usepackage| or |\RequirePackage|
% which has a mechanism to prevent loading a style file more than once.
% When loading the definitions by means of |\input|
% multiple instances have to be prevented manually:
%\iffalse
%This code needs to be before the `\ProvidesFile' directive
%which is defined at the beginning of this file.
%Therefore it is also placed there and commented out here.
%</package>
%<*discard>
%\fi
%    \begin{macrocode}
\ifdefined\childdocmain\endinput\fi
%    \end{macrocode}
%\iffalse
%</discard>
%<*package>
%\fi
%
% \macro{\ifchilddoc}
% \macro{\ifchilddocmanual}
% The conditional |\ifchilddoc| tells whether a
% child (true) or main (false) document is being compiled.
% The conditional |\ifchilddocmanual| tells whether
% the |\includeonly| mechanism is used (false) or
% the selection of child files must be performed manually (true).
% The definitions initialise to false:
%    \begin{macrocode}
\newif\ifchilddoc
\newif\ifchilddocmanual
%    \end{macrocode}

% \macro{\childdocname}
% \macro{\childdocjob}
% The macro |\childdocname| stores the name of the main document
% to be compiled. The macro |\childdocjob| stores the name of
% the document on which the \LaTeX{} compiler was originally invoked.
% The content of |\jobname| cannot be compared
% to filenames specified in the source due to different catcodes.
% The following code rescans |\jobname|, stores the result
% in |\childdocname| and saves a copy in |\childdocjob|:
%    \begin{macrocode}
\edef\childdocname{\scantokens\expandafter{\jobname\noexpand}}
\let\childdocjob\childdocname
%    \end{macrocode}

% \macro{\childdocdisable}
% The macro |\childdocdisable| prevents the main file
% from being processed more than once.
% At this stage, the main document command |\childdocmain|
% is assumed to be called once again where it should do nothing.
% Any subsequent call to it should prevent
% a secondary processing of the main document
% It overwrites the forwarding commands
% |\childdocof| and |\childdocforward|
% with empty macros to prevent further inclusions of the main document:
%    \begin{macrocode}
\newcommand{\childdocdisable}
{
  \renewcommand{\childdocmain}[1]{\renewcommand{\childdocmain}[1]{\endinput}}
  \renewcommand{\childdocof}[1]{}
  \renewcommand{\childdocby}[2][]{}
  \renewcommand{\childdocforward}[2][]{}
  \renewcommand{\childdocdisable}{}
}
%    \end{macrocode}

% \macro{\childdocmain}
% The macro |\childdocmain| is to be called at the top of the main file
% with nothing or the main filename (without extension) as argument.
% First, it breaks loops.
% If the argument is not empty and does not match |\childdocname|
% (which is set by the first inclusion of |childdoc.def|),
% |\ifchilddoc| is set to true, |\includeonly| is applied to the child file
% and |\jobname| is set to the main file
% (for proper handling of |.aux| files):
%    \begin{macrocode}
\newcommand{\childdocmain}[1]
{
  \childdocdisable\childdocmain{}
  \if?#1?\else
    \begingroup
      \def\childdoctmp{#1}
      \ifx\childdoctmp\childdocname
        \def\childdoctmp{}
      \else
        \def\childdoctmp
        {
          \childdoctrue
          \includeonly{\childdocname}
          \def\childdocjob{#1}
          \def\jobname{#1}
        }
      \fi
      \expandafter
    \endgroup
    \childdoctmp
  \fi
}
%    \end{macrocode}

% \macro{\childdocof}
% The command |\childdocof| redirects
% compilation to the main file |#1|.
%    \begin{macrocode}
\newcommand{\childdocof}[1]
{
  \childdocdisable
  \childdoctrue
  \includeonly{\childdocname}
  \def\jobname{#1}
  \def\childdocjob{#1}
  \input{#1}
}
%    \end{macrocode}

% \macro{\childdocby}
% The command |\childdocby| ....
%    \begin{macrocode}
\newcommand{\childdocby}[2][]
{
  \childdocdisable
  \childdoctrue
  \childdocmanualtrue
  \if?#1?\else
    \def\jobname{#2}
  \fi
  \def\childdocjob{#2}
  \input{#2}
  \endinput
}
%    \end{macrocode}

% \macro{\childdocforward}
% The command |\childdocforward| redirects
% compilation to the main file or
% (if the optional argument is given) a child file.
% Parameters are set as if the main file
% or a child file starting with |\childdocof| was compiled.
% Then compilation is handed over to the main file:
%    \begin{macrocode}
\newcommand{\childdocforward}[2][]
{
  \begingroup
    \if?#1?
      \def\childdoctmp
      {
        \def\childdocname{#2}
        \def\childdocjob{#2}
        \def\jobname{#2}
        \input{#2}
        \endinput
      }
    \else
      \def\childdoctmp
      {
        \childdocdisable
        \def\childdocname{#2}
        \childdoctrue
        \includeonly{#2}
        \def\childdocjob{#1}
        \def\jobname{#1}
        \input{#1}
        \endinput
      }
    \fi
    \expandafter
  \endgroup
  \childdoctmp
}
%    \end{macrocode}

% \macro{\childdocforwardprefix}
% The command |\childdocforwardprefix| redirects
% compilation to the main or a child file by means of a pattern.
% The prefix |#1| in the current filename is replaced by |#2|
% and the suffix of the current filename is kept
% (it is assumed that the filename does not contain the substring `|~~~|'
% which is used as a delimiter).
% Compilation is handed over to the new file by |\childdocforward|:
%    \begin{macrocode}
\newcommand{\childdocforwardprefix}[3][]
{
  \begingroup
    \def\childdocextract #2##1~~~{\def\childdoctmp{\childdocforward[#1]{#3##1}}}
    \expandafter\childdocextract\childdocname~~~
    \expandafter
  \endgroup
  \childdoctmp
}
%    \end{macrocode}

% \macro{\childdoc}
% The deprecated macro |\childdoc| is a legacy version of |\childdocmain|:
%    \begin{macrocode}
\newcommand{\childdoc}{\childdocmain}
%    \end{macrocode}

% \macro{\childdocredirect}
% The deprecated macro |\childdocredirect| is a legacy version
% of |\childdocforward| and |\childdocforwardprefix|:
%    \begin{macrocode}
\newcommand{\childdocredirect}[2][]
{
  \begingroup
    \if?#1?
      \def\childdoctmp{\childdocforward{#2}}
    \else
      \def\childdoctmp{\childdocforwardprefix{#1}{#2}}
    \fi
    \expandafter
  \endgroup
  \childdoctmp
}
%    \end{macrocode}

%\iffalse
%</package>
%\fi
%
\endinput
|\\
|\childdocforward[|\textit{main}|]{|\textit{dest}|}|\\
\end{tabular}
\end{center}
%
The argument \textit{dest} is the destination file
(without extension).
It should be the main file or one of the child files.
Note that further \textsf{childdoc} directives
such as |\childdocof| and |\childdocforward|
in the indicated file will be processed in this form.
The optional argument \textit{main}
passes on directly to the main file \textit{main}
while pretending to compile the child \textit{dest}.
This form behaves as if \textit{dest}
issues |\childdocof{|\textit{main}|}| right away,
and no further \textsf{childdoc} directives will be processed.

%%%%%%%%%%%%%%%%%%%%%%%%%%%%%%%%%%%%%%%%
\DescribeMacro{\...prefix}
In the alternative form |\childdocforwardprefix|,
%
\begin{center}
\begin{tabular}{l}
|% \iffalse
%
% childdoc.dtx Copyright (C) 2017-2018 Niklas Beisert
%
% This work may be distributed and/or modified under the
% conditions of the LaTeX Project Public License, either version 1.3
% of this license or (at your option) any later version.
% The latest version of this license is in
%   http://www.latex-project.org/lppl.txt
% and version 1.3 or later is part of all distributions of LaTeX
% version 2005/12/01 or later.
%
% This work has the LPPL maintenance status `maintained'.
%
% The Current Maintainer of this work is Niklas Beisert.
%
% This work consists of the files childdoc.dtx and childdoc.ins
% and the derived files childdoc.def and cdocsamp.tex with
% cdocsch1.tex, cdocsch2.tex, cdocsdrf.tex, cdocsfn1.tex, cdocsfn2.tex.
%
%<package>\ifdefined\childdocmain\endinput\fi
%<package>\ProvidesFile{childdoc.def}[2018/12/30 v2.0 child document driver]
%<samplemain>\ProvidesFile{cdocsamp.tex}[2018/12/30 v2.0 sample for childdoc]
%<*driver>
%\ProvidesFile{childdoc.drv}[2018/12/30 v2.0 childdoc reference manual file]
\PassOptionsToClass{10pt,a4paper}{article}
\documentclass{ltxdoc}

\usepackage[margin=35mm]{geometry}
\usepackage{hyperref}
\usepackage{hyperxmp}
\usepackage[usenames]{color}

\hypersetup{colorlinks=true}
\hypersetup{pdfstartview=FitH}
\hypersetup{pdfpagemode=UseNone}
\hypersetup{pdfsource={}}
\hypersetup{pdflang={en-UK}}
\hypersetup{pdfcopyright={Copyright 2017-2018 Niklas Beisert.
  This work may be distributed and/or modified under the
  conditions of the LaTeX Project Public License, either version 1.3
  of this license or (at your option) any later version.}}
\hypersetup{pdflicenseurl={http://www.latex-project.org/lppl.txt}}
\hypersetup{pdfcontactaddress={ETH Zurich, ITP, HIT K,
  Wolfgang-Pauli-Strasse 27}}
\hypersetup{pdfcontactpostcode={8093}}
\hypersetup{pdfcontactcity={Zurich}}
\hypersetup{pdfcontactcountry={Switzerland}}
\hypersetup{pdfcontactemail={nbeisert@itp.phys.ethz.ch}}
\hypersetup{pdfcontacturl={http://people.phys.ethz.ch/\xmptilde nbeisert/}}

\newcommand{\secref}[1]{\hyperref[#1]{section \ref*{#1}}}

\parskip1ex
\parindent0pt
\let\olditemize\itemize
\def\itemize{\olditemize\parskip0pt}

\begin{document}

\title{The \textsf{childdoc} Package}
\hypersetup{pdftitle={The childdoc Package}}
\author{Niklas Beisert\\[2ex]
  Institut f\"ur Theoretische Physik\\
  Eidgen\"ossische Technische Hochschule Z\"urich\\
  Wolfgang-Pauli-Strasse 27, 8093 Z\"urich, Switzerland\\[1ex]
  \href{mailto:nbeisert@itp.phys.ethz.ch}
  {\texttt{nbeisert@itp.phys.ethz.ch}}}
\hypersetup{pdfauthor={Niklas Beisert}}
\hypersetup{pdfsubject={Manual for the LaTeX2e Package childdoc}}
\date{30 December 2018, \textsf{v2.0}}
\maketitle

\begin{abstract}\noindent
\textsf{childdoc} is a \LaTeXe{} package
that enables the direct compilation
of document sections included by |\include|
to individual files.
\end{abstract}

\begingroup
\parskip0ex
\tableofcontents
\endgroup

%%%%%%%%%%%%%%%%%%%%%%%%%%%%%%%%%%%%%%%%%%%%%%%%%%%%%%%%%%%%%%%%%%%%%%%%%%%%%%%%
%%%%%%%%%%%%%%%%%%%%%%%%%%%%%%%%%%%%%%%%%%%%%%%%%%%%%%%%%%%%%%%%%%%%%%%%%%%%%%%%
\section{Introduction}

\LaTeX{} provides a mechanism to structure a large document (such as a book)
into a main file and several child files (containing the chapters)
using the |\include| command.
This mechanism is beneficial for documents
which span hundreds of pages in order to
make the source file(s) more manageable.
Moreover, compilation can be restricted to
selected child files by means of the |\includeonly| command.
The latter feature can be used to reduce the compilation time while editing
(this was significantly more useful in the earlier days of \LaTeX{})
or to generate a smaller document which is easier to navigate.
Another application of |\includeonly| is to generate
documents consisting of selected parts of the complete document.

However, there are a few drawbacks of the plain |\include| mechanism:
\begin{itemize}
\item
The child files cannot be compiled on their own,
they can only be compiled via the main file.
A naive editing environment
(such as a text editor with an option
to have the current file processed by \LaTeX)
may require one to switch to the main file before compiling;
attempting to compile the child file produces errors.
\item
The main file must be modified (each time)
to adjust the |\includeonly| command
to the present needs. This easily leaves the main file in a messy state.
\item
The generated document will always carry the filename
of the main document. This is inconvenient if
several child files are to be compiled and
to be kept for distribution.
\end{itemize}

The present package provides a simple interface
to make child files individually compilable by \LaTeX{}.
Compiling a child file then has the same effect as compiling
the main file with an |\includeonly| command
to select the appropriate child.
Moreover the generated document will carry the name of the child
rather than the main file.
This resolves all three above issues.

This feature is meant to make the editing of books,
thesis documents and lecture notes somewhat more convenient.
However, the package can also be used efficiently for
composing a series of documents (such as exercise sheets)
which are typically distributed individually.
It then assists the author in generating the individual documents
(potentially in different versions)
as well as a document containing the collected series.
Another application is in developing style files
or other kinds of included material
where compilation of the style file could redirect
to a sample or test file.

%%%%%%%%%%%%%%%%%%%%%%%%%%%%%%%%%%%%%%%%%%%%%%%%%%%%%%%%%%%%%%%%%%%%%%%%%%%%%%%%
%%%%%%%%%%%%%%%%%%%%%%%%%%%%%%%%%%%%%%%%%%%%%%%%%%%%%%%%%%%%%%%%%%%%%%%%%%%%%%%%
\section{Usage}

First of all, the package \textsf{childdoc} is \emph{not} a standard
\LaTeXe{} |.sty| style file! Therefore it needs to be invoked in
a non-standard way.

%%%%%%%%%%%%%%%%%%%%%%%%%%%%%%%%%%%%%%%%%%%%%%%%%%%%%%%%%%%%%%%%%%%%%%%%%%%%%%%%
\subsection{Included Files}
\label{sec:include}

%%%%%%%%%%%%%%%%%%%%%%%%%%%%%%%%%%%%%%%%
\DescribeMacro{\childdocmain}
To use the package, add the commands
\begin{center}
\begin{tabular}{l}
|\input{childdoc.def}|\\
|\childdocmain{}|\\
\end{tabular}
\end{center}
at the very top of the main \LaTeX{} file,
in particular \emph{before} the |\documentclass| statement!
The argument of |\childdocmain| should be left empty
(but it must be present).

%%%%%%%%%%%%%%%%%%%%%%%%%%%%%%%%%%%%%%%%
\DescribeMacro{\childdocof}
Furthermore, add the commands
\begin{center}
\begin{tabular}{l}
|\input{childdoc.def}|\\
|\childdocof{|\textit{main}|}|\\
\end{tabular}
\end{center}
at the top of every child file \textit{child}
which is included by |\include{|\textit{child}|}|
from within the main file
(or at least for those files to be compiled individually).
The argument \textit{main} must be the filename of the main file.

There are a couple of
considerations in setting up the main and child documents:

%%%%%%%%%%%%%%%%%%%%%%%%%%%%%%%%%%%%%%%%
\paragraph{Restrictions.}

Please note the following restrictions:
\begin{itemize}
\item
|\childdocmain| must be called with one argument \textit{main}
to ensure compatibility with earlier version of the package.
It must either be empty (|\childdocmain{}|)
or precisely match the filename of the main file in which it is specified.
See \secref{sec:detection} for further information.
\item
The filename \textit{main} must be specified without the |.tex| extension.
\item
The filename \textit{main} is case sensitive
(even in case-insensitive file systems)
due to internal string comparison.
\item
The argument \textit{main} should be fully expanded, it cannot be a macro.
\item
Subdirectories and special characters should be avoided in filenames.
\item
The command |\childdocmain{|\textit{main}|}| must be followed by a whitespace.
It should not be followed immediately by another command
or by a comment mark `|%|'.
This is because the \TeX{} parser reads the token immediately following
the argument of |\childdocmain| and puts it
at the beginning of every child section;
however, a white\-space is ignored.
\end{itemize}

%%%%%%%%%%%%%%%%%%%%%%%%%%%%%%%%%%%%%%%%
\paragraph{Content of Main File.}

It is advisable to place all content in the child files included by |\include|.
Any output contained in the main file will appear in all child documents
unless suppressed manually;
it cannot be suppressed automatically by the |\includeonly| directive
and thus should normally be avoided.
A method to include some content in the main file
by means of conditional processing is described in \secref{sec:conditional}.

%%%%%%%%%%%%%%%%%%%%%%%%%%%%%%%%%%%%%%%%
\paragraph{Page Numbering.}

When only a part of the document is compiled,
the appropriate numbering of pages
(as well as other status parameters)
is determined from the |.aux| files.
The latter contain information from previous passes.
However this information needs to propagate through
all intermediate child documents.
Therefore the page numbering in child documents may well
be inconsistent until the complete document is compiled at least once.

A useful (if unconventional) way to always ensure a consistent
page numbering is to restart the numbering in each child document
and denote the pages by `\textit{child}|.|\textit{page}'
where \textit{child} represents the chapter/section number of the child file.
This can be achieved by the command
|\numberwithin{page}{|\textit{child}|}|
of the \textsf{amsmath} package
where \textit{child} can be |chapter| or |section|
depending on the chosen structuring.
Alternatively, one can modify the macro |\thepage| appropriately
and reset the counter |page| at the start of each child file.

%%%%%%%%%%%%%%%%%%%%%%%%%%%%%%%%%%%%%%%%%%%%%%%%%%%%%%%%%%%%%%%%%%%%%%%%%%%%%%%%
\subsection{Conditional Processing}
\label{sec:conditional}

The package provides a mechanism to compile different versions
of a document. To customise the versions further some conditional processing
can come in handy to distinguish which version is being compiled.
The package provides two macros to describe the compilation context:

%%%%%%%%%%%%%%%%%%%%%%%%%%%%%%%%%%%%%%%%
\DescribeMacro{\ifchilddoc}
The conditional |\ifchilddoc| distinguishes between the compilation of
child documents and the main document:
%
\begin{center}
|\ifchilddoc |\textit{child-code}| |[|\||else |\textit{main-code}]| \||fi|
\end{center}

%%%%%%%%%%%%%%%%%%%%%%%%%%%%%%%%%%%%%%%%
\DescribeMacro{\childdocname}
\DescribeMacro{\childdocjob}
The macro |\childdocname| contains the filename (without extension)
of the main or child file being processed.
Note that |\childdocjob| will always contain the name of the main file.

%%%%%%%%%%%%%%%%%%%%%%%%%%%%%%%%%%%%%%%%
\paragraph{Title Page.}

Conditional processing can be used to include a title or banner page
in the main document when proper precautions are taken.
Importantly, the code in the main file should ensure that the page counter
(as well as other status parameters which are stored in the |.aux| files)
takes the same value after the conditional processing.
Otherwise the page numbers may take divergent values
depending on which part is compiled.

For example, a title page could be declared by:
%
\begin{center}
\begin{tabular}{l}
|\ifchilddoc\||else|\\
|\addtocounter{page}{-1}|\\
\textit{code for title page}\\
|\newpage|\\
|\||fi|
\end{tabular}
\end{center}
%
A banner page for the child documents can be generated by:
%
\begin{center}
\begin{tabular}{l}
|\ifchilddoc|\\
|\addtocounter{page}{-1}|\\
\textit{code for banner page}\\
|\newpage|\\
|\||fi|
\end{tabular}
\end{center}
%
Here one could write a message such as:
\begin{center}
|This is the part \childdocname{} of \childdocjob{}.|
\end{center}

%%%%%%%%%%%%%%%%%%%%%%%%%%%%%%%%%%%%%%%%%%%%%%%%%%%%%%%%%%%%%%%%%%%%%%%%%%%%%%%%
\subsection{Flags}
\label{sec:flags}

The package makes it easy to generate different versions
of the main or child documents.
To this end compilation flags can be defined
and assigned different default values.
They will be particularly useful in conjunction
with the forwarding mechanism described in \secref{sec:forward}.

For example, it may be useful to have a flag |\version|
which can be set to |draft| or |final|.
The document source will contain some conditional code
depending on the value of |\version|.
Suppose further, the flag should default to |final| for the main file
and to |draft| for child files
which is a natural assignment for editing the document.
This is achieved by placing the following code
in the preamble of the main document
(below the |\childdocmain| directive):
%
\begin{center}
\begin{tabular}{l}
|\ifchilddoc|\\
|\providecommand{\version}{draft}|\\
|\||else|\\
|\providecommand{\version}{final}|\\
|\||fi|
\end{tabular}
\end{center}
%
The definition by |\providecommand| makes sure
that previous definitions are not overwritten.
Further statements |\providecommand{\version}{...}|
can thus be added before the above code to override it.

For the main file, one might add a line
(between |\childdocmain| and the above block)
%
\begin{center}
|%\ifchilddoc\||else\providecommand{\version}{draft}\||fi|
\end{center}
%
which can be uncommented to produce a draft version.
Likewise one can add a line to the very top of a child file
(above the |\childdocof{|\textit{main}|}| directive)
%
\begin{center}
|%\providecommand{\version}{final}|
\end{center}
%
which can be uncommented to produce the final version of this child document.

%%%%%%%%%%%%%%%%%%%%%%%%%%%%%%%%%%%%%%%%%%%%%%%%%%%%%%%%%%%%%%%%%%%%%%%%%%%%%%%%
\subsection{Forwarding}
\label{sec:forward}

Different versions of the main or child documents
using compilation flags as described in \secref{sec:flags}
can be (permanently) stored in different files
for convenient compilation, viewing and distribution.
To this end, the package defines a command
to pass on compilation to a different file:

%%%%%%%%%%%%%%%%%%%%%%%%%%%%%%%%%%%%%%%%
\DescribeMacro{\childdocforward}
The command |\childdocforward| redirects processing to
another source file:
%
\begin{center}
\begin{tabular}{l}
|\input{childdoc.def}|\\
|\childdocforward[|\textit{main}|]{|\textit{dest}|}|\\
\end{tabular}
\end{center}
%
The argument \textit{dest} is the destination file
(without extension).
It should be the main file or one of the child files.
Note that further \textsf{childdoc} directives
such as |\childdocof| and |\childdocforward|
in the indicated file will be processed in this form.
The optional argument \textit{main}
passes on directly to the main file \textit{main}
while pretending to compile the child \textit{dest}.
This form behaves as if \textit{dest}
issues |\childdocof{|\textit{main}|}| right away,
and no further \textsf{childdoc} directives will be processed.

%%%%%%%%%%%%%%%%%%%%%%%%%%%%%%%%%%%%%%%%
\DescribeMacro{\...prefix}
In the alternative form |\childdocforwardprefix|,
%
\begin{center}
\begin{tabular}{l}
|\input{childdoc.def}|\\
|\childdocforwardprefix[|\textit{main}|]{|\textit{prefix}|}{|\textit{dest}|}|
\end{tabular}
\end{center}
%
the destination file is determined by a pattern
depending on the current file:
To make this work, the current file must be called
`{\textit{prefix}\hspace{0.2em}\textit{suffix}}'
with \textit{prefix} matching precisely the argument.
Processing is then passed on to the file
`{\textit{dest}\hspace{0.2em}\textit{suffix}}'.
Surely, the same effect is achieved by
directly specifying the
argument `{\textit{dest}\hspace{0.2em}\textit{suffix}}'
in the first form.
However, that requires to set up a different file
for each child. With the alternative form of the command
all these files can have exactly the same content
which simplifies setting them up and maintaining them.

For example, the following file |draft.tex|
with a compilation flag |\version| as described in \secref{sec:flags}
compiles the main document as a draft:
%
\begin{center}
\begin{tabular}{l}
|\def\version{draft}|\\
|\input{childdoc.def}|\\
|\childdocforward{|\textit{main}|}|
\end{tabular}
\end{center}
%
Likewise, the following files |final|\textit{nn}|.tex|
compile the final version of the child document
|child|\textit{nn}|.tex|:
%
\begin{center}
\begin{tabular}{l}
|\def\version{final}|\\
|\input{childdoc.def}|\\
|\childdocforwardprefix{final}{child}|
\end{tabular}
\end{center}
%

Note that when several versions of a main file and/or of each child file
are to be generated, it may be convenient to set up a |Makefile| or
shell script to automatise the process.

%%%%%%%%%%%%%%%%%%%%%%%%%%%%%%%%%%%%%%%%%%%%%%%%%%%%%%%%%%%%%%%%%%%%%%%%%%%%%%%%
\subsection{Command Line Processing}
\label{sec:commandline}

The effect of redirection files can also be achieved by invoking
the \LaTeX{} compiler with a more elaborate command line.
Most conveniently this should be done as part
of a shell script or a |Makefile|.

When using \textsf{childdoc} in the main file, the following
command lines effectively perform a redirection
(note that depending on the shell being used,
backslashes may have to be doubled: `|\|' $\to$ `|\\|'):
%
\begin{center}
|... -jobname "|\textit{target}|" |\\|"|[\textit{flags}]%
|\input{childdoc.def}\childdocforward[|\textit{main}|]{|\textit{dest}|}"|
\end{center}
%
Here \textit{target} is the name of the output file,
\textit{main} is the name of the main file
and \textit{dest} is the name of the main or child file to be processed
(all filenames without extensions).
The optional argument \textit{main} can be omitted
if \textit{main} matches \textit{dest}.
Optionally, compilation \textit{flags} can be defined via |\def| commands.
This command line makes the \TeX{} engine believe
it is compiling the file \textit{target}
whose content is specified as the latter parameter.
The provided code then forwards the processing to
\textit{main} or \textit{dest} as described in \secref{sec:forward}.

%%%%%%%%%%%%%%%%%%%%%%%%%%%%%%%%%%%%%%%%%%%%%%%%%%%%%%%%%%%%%%%%%%%%%%%%%%%%%%%%
\subsection{Include by Input}
\label{sec:input}

Including child documents by |\include| has some restrictions by design.
Most notably, the content of a child document always occupies
its own set of pages; pages cannot be shared between child documents.
Usually, this behaviour makes perfect sense
because each child document contain an essential part of the document.
However, in some situations it may be desirable to compose
a document from a collection of parts
without having mandatory page breaks between then.
For this case, the package
provides a mechanism to include parts
by |\input| which can also be processed individually.
However, by construction this mechanism
requires manual handling of the content to be output.

%%%%%%%%%%%%%%%%%%%%%%%%%%%%%%%%%%%%%%%%
\DescribeMacro{\ifchilddocmanual}
The main file should be prepared as usual, see \secref{sec:include}.
However, the document body must make a distinction
between processing of an individual part and of the main document, e.g.:
%
\begin{center}
\begin{tabular}{l}
|\ifchilddocmanual|\\
|\input{\childdocname}|\\
|\||else|\\
\textit{document body with }|\input{|\textit{part}|}|\\
|\||fi|
\end{tabular}
\end{center}
%
The conditional |\ifchilddocmanual| is true whenever
a part to be included by |\input| is being compiled,
and the name of the part is stored in |\childdocname|.

%%%%%%%%%%%%%%%%%%%%%%%%%%%%%%%%%%%%%%%%
\DescribeMacro{\childdocby}
Each part to be included by |\input| should start with:
%
\begin{center}
\begin{tabular}{l}
|\input{childdoc.def}|\\
|\childdocby{|\textit{main}|}|\\
\end{tabular}
\end{center}
%
The directive |\childdocby| is similar to |\childdocof|
described in \secref{sec:include},
but the subsequent selection of content must be done manually.
To that end, both |\ifchilddoc| and |\ifchilddocmanual|
will be true upon processing of a part,
and the name of the part is stored in |\childdocname|.
Note that |\jobname| will be set to the filename of the current part
so that each part receives an individual |.aux| file
that does not interfere with the |.aux| file(s) of the main document.
This behaviour can be altered by the alternative form
|\childdocby[*]{|\textit{main}|}| (with a non-empty optional argument)
which uses the |.aux| file of the main document
by setting |\jobname| to \textit{main}.

%%%%%%%%%%%%%%%%%%%%%%%%%%%%%%%%%%%%%%%%%%%%%%%%%%%%%%%%%%%%%%%%%%%%%%%%%%%%%%%%
\subsection{Driver Development}
\label{sec:driver}

The \textsf{childdoc} mechanism can also be use for the development
of definition files such as \LaTeX{} styles or classes.
This case differs from the above setup with multiple parts
included by |\include| in that no |\includeonly| should be invoked.
This can be achieved by starting the include file
(before |\ProvidesPackage|) with:
%
\begin{center}
\begin{tabular}{l}
|\input{childdoc.def}|\\
|\childdocforward{|\textit{main}|}|\\
\end{tabular}
\end{center}
%
or alternatively with:
%
\begin{center}
\begin{tabular}{l}
|\input{childdoc.def}|\\
|\childdocby{|\textit{main}|}|\\
\end{tabular}
\end{center}
%
Both forms have slightly different effects as described above.
The main file is prepared as usual, see \secref{sec:include}.

%%%%%%%%%%%%%%%%%%%%%%%%%%%%%%%%%%%%%%%%%%%%%%%%%%%%%%%%%%%%%%%%%%%%%%%%%%%%%%%%
\subsection{Legacy Detection}
\label{sec:detection}

The directive |\childdocmain| in the main file can detect
whether the complete document or merely a child is to be compiled
even without using the directive |\childdocof|.
This method is deprecated because it is less robust
and there is no compelling reason to use it;
it is merely provided for backward compatibility
and it may be removed in future versions.

If the detection mechanism is to be used,
it is mandatory to correctly specify
the filename of the main file as the argument of |\childdocmain|:
%
\begin{center}
\begin{tabular}{l}
|\input{childdoc.def}|\\
|\childdocmain{|\textit{main}|}|\\
\end{tabular}
\end{center}
%
If |\jobname| does not match the argument \textit{main} of |\childdocmain|,
it is assumed that |\jobname| points to the child file to be compiled.
When using |\childdocmain| with the main file specified as argument,
it suffices to start a child file
with just |\input{|\textit{main}|}|
without loading of the package and using |\childdocof|.
If instead all processing is done
with the appropriate \textsf{childdoc} directives,
the argument of \textit{main} of |\childdocmain| can be empty.

An alternative version of the command line processing described
in \secref{sec:commandline} using the detection mechanism reads:
%
\begin{center}
|... -jobname "|\textit{target}|" "|[\textit{flags}]%
[|\def\jobname{|\textit{dest}|}|]|\input{|\textit{main}|}"|
\end{center}

%%%%%%%%%%%%%%%%%%%%%%%%%%%%%%%%%%%%%%%%%%%%%%%%%%%%%%%%%%%%%%%%%%%%%%%%%%%%%%%%
\subsection{Manual Code}
\label{sec:manual}

In case one cannot be certain whether the definitions file |childdoc.def|
is installed on the target \TeX{} distribution
and one prefers not to ship it,
it is conceivable to paste a few relevant commands into the sources.

To that end, drop all statements |\input{childdoc.def}|
and perform the replacements as outlined below.
Instead of |\childdocmain{|\textit{main}|}| add the following code
to the top of the main file:
%
\begin{center}
\begin{tabular}{l}
|\||ifdefined\childdocname\endinput\||fi\newif\ifchilddoc|\\
|\edef\childdocname{\scantokens\expandafter{\jobname\noexpand}}|\\
|\def\childdocmain{|\textit{main}|}\||ifx\childdocmain\childdocname\||else|\\
|\childdoctrue\includeonly{\childdocname}\let\jobname\childdocmain\||fi|\\
\end{tabular}
\end{center}
%
Instead of |\childdocof{|\textit{main}|}| just include the main file
at the top of each child file:
%
\begin{center}
|\input{|\textit{main}|}|
\end{center}
%
A simple redirection |\childdocforward{|\textit{dest}|}| is achieved by:
%
\begin{center}
|\def\jobname{|\textit{dest}|}\input{\jobname}|
\end{center}
%
The redirection with prefix
|\childdocforwardprefix[|\textit{prefix}|]{|\textit{dest}|}|
is accomplished by:
%
\begin{center}
\begin{tabular}{l}
|{\edef\jobname{\scantokens\expandafter{\jobname\noexpand}}|\\
|\def\redirectjob |\textit{prefix}|#1~~~{\gdef\jobname{|\textit{dest}|#1}}|\\
|\expandafter\redirectjob\jobname~~~}\input{\jobname}|
\end{tabular}
\end{center}

In an alternative approach,
child documents can be compiled by a specific command line
without additional code or specific definitions:
%
\begin{center}
|... -jobname "|\textit{target}|" "|[\textit{flags}]%
|\includeonly{|\textit{dest}|}\input{|\textit{main}|}"|
\end{center}
%

%%%%%%%%%%%%%%%%%%%%%%%%%%%%%%%%%%%%%%%%%%%%%%%%%%%%%%%%%%%%%%%%%%%%%%%%%%%%%%%%
%%%%%%%%%%%%%%%%%%%%%%%%%%%%%%%%%%%%%%%%%%%%%%%%%%%%%%%%%%%%%%%%%%%%%%%%%%%%%%%%
\section{Information}

%%%%%%%%%%%%%%%%%%%%%%%%%%%%%%%%%%%%%%%%%%%%%%%%%%%%%%%%%%%%%%%%%%%%%%%%%%%%%%%%
\subsection{Copyright}

Copyright \copyright{} 2017--2018 Niklas Beisert

This work may be distributed and/or modified under the
conditions of the \LaTeX{} Project Public License, either version 1.3
of this license or (at your option) any later version.
The latest version of this license is in
  \url{http://www.latex-project.org/lppl.txt}
and version 1.3 or later is part of all distributions of \LaTeX{}
version 2005/12/01 or later.

This work has the LPPL maintenance status `maintained'.

The Current Maintainer of this work is Niklas Beisert.

This work consists of the files |README.txt|, |childdoc.ins| and |childdoc.dtx|
as well as the derived files |childdoc.def|, |cdocsamp.tex|
with |cdocsch1.tex|, |cdocsch2.tex|, |cdocspt3.tex|, |cdocspt4.tex|,
|cdocsdrf.tex|, |cdocsfn1.tex|, |cdocsfn2.tex|
as well as |childdoc.pdf|.

%%%%%%%%%%%%%%%%%%%%%%%%%%%%%%%%%%%%%%%%%%%%%%%%%%%%%%%%%%%%%%%%%%%%%%%%%%%%%%%%
\subsection{Files and Installation}

The package consists of the files:
%
\begin{center}
\begin{tabular}{ll}
    |README.txt|   & readme file \\
    |childdoc.ins| & installation file \\
    |childdoc.dtx| & source file \\
    |childdoc.def| & definition file \\
    |cdocsamp.tex| & sample main file \\
    |cdocsch1.tex| & sample include file \\
    |cdocsch2.tex| & sample include file \\
    |cdocspt3.tex| & sample part file \\
    |cdocspt4.tex| & sample part file \\
    |cdocsdrf.tex| & sample redirection file \\
    |cdocsfn1.tex| & sample redirection file \\
    |cdocsfn2.tex| & sample redirection file \\
    |childdoc.pdf| & manual
\end{tabular}
\end{center}
%
The distribution consists of the files
|README.txt|, |childdoc.ins| and |childdoc.dtx|.
%
\begin{itemize}
\item
Run (pdf)\LaTeX{} on |childdoc.dtx|
to compile the manual |childdoc.pdf| (this file).
\item
Run \LaTeX{} on |childdoc.ins| to create the definitions file |childdoc.def|
and the sample |cdocsamp.tex| with include files
|cdocsch1.tex|, |cdocsch2.tex|, |cdocspt3.tex|, |cdocspt4.tex|,
|cdocsdrf.tex|, |cdocsfn1.tex|, |cdocsfn2.tex|.
Then copy the file |childdoc.def| to an appropriate directory of your \LaTeX{}
distribution, e.g.\ \textit{texmf-root}|/tex/latex/childdoc|.
\end{itemize}

%%%%%%%%%%%%%%%%%%%%%%%%%%%%%%%%%%%%%%%%%%%%%%%%%%%%%%%%%%%%%%%%%%%%%%%%%%%%%%%%
\subsection{Related CTAN Packages}

There are several other packages which offer a similar functionality:
%
\begin{itemize}
\item
The packages
\href{http://ctan.org/pkg/docmute}{\textsf{docmute}},
\href{http://ctan.org/pkg/includex}{\textsf{includex}} and
\href{http://ctan.org/pkg/standalone}{\textsf{standalone}}
provide commands to include only the document body of
a child file thus allowing both files to be compiled individually.
\item
The packages \href{http://ctan.org/pkg/subdocs}{\textsf{subdocs}}
and \href{http://ctan.org/pkg/subfiles}{\textsf{subfiles}}
provide structures in which the main and child documents can be
encapsulated and allowing them to be compiled individually.
The inclusion mechanism is different from the conventional |\include|.
\item
The package \href{http://ctan.org/pkg/combine}{\textsf{combine}}
is an elaborate solution to combine several documents into one.
\end{itemize}
%
See also the CTAN topic \href{http://ctan.org/topic/subdocs}{\textsf{subdocs}}
for further related packages.
The present package differs from the above solutions in that
a document structure constructed with the conventional |\include| mechanism
just needs two extra commands at the top of every file
such that all constituent files can be compiled individually.

%%%%%%%%%%%%%%%%%%%%%%%%%%%%%%%%%%%%%%%%%%%%%%%%%%%%%%%%%%%%%%%%%%%%%%%%%%%%%%%%
%\subsection{Feature Suggestions}
%
%The following is a list of features which may be useful for future
%versions of this package:
%%
%\begin{itemize}
%\item
%\ldots
%\end{itemize}

%%%%%%%%%%%%%%%%%%%%%%%%%%%%%%%%%%%%%%%%%%%%%%%%%%%%%%%%%%%%%%%%%%%%%%%%%%%%%%%%
\subsection{Revision History}

%%%%%%%%%%%%%%%%%%%%%%%%%%%%%%%%%%%%%%%%
\paragraph{v2.0:} 2018/12/30

\begin{itemize}
\item
immediate forward processing
\item
added |\childdocby| mechanism
\item
manual restructured
\end{itemize}

%%%%%%%%%%%%%%%%%%%%%%%%%%%%%%%%%%%%%%%%
\paragraph{v1.6:} 2018/01/17

\begin{itemize}
\item
application for development of include files
\item
corrections to manual
\end{itemize}

%%%%%%%%%%%%%%%%%%%%%%%%%%%%%%%%%%%%%%%%
\paragraph{v1.5:} 2017/05/21

\begin{itemize}
\item
more complete structuring introduced
\item
|\childdocof| introduced
\item
|\childdoc| renamed to |\childdocmain|
\item
|\childredirect| renamed to |\childdocforward| and |\childdocforwardprefix|
and functionality expanded
\end{itemize}

%%%%%%%%%%%%%%%%%%%%%%%%%%%%%%%%%%%%%%%%
\paragraph{v1.0:} 2017/04/27

\begin{itemize}
\item
manual and install package
\item
first version published on CTAN
\end{itemize}

%%%%%%%%%%%%%%%%%%%%%%%%%%%%%%%%%%%%%%%%
\paragraph{v0.6:} 2017/04/26

\begin{itemize}
\item
redirection mechanism added
\end{itemize}

%%%%%%%%%%%%%%%%%%%%%%%%%%%%%%%%%%%%%%%%
\paragraph{v0.5:} 2017/04/26

\begin{itemize}
\item
functionality in definition file
\end{itemize}


%%%%%%%%%%%%%%%%%%%%%%%%%%%%%%%%%%%%%%%%%%%%%%%%%%%%%%%%%%%%%%%%%%%%%%%%%%%%%%%%
%%%%%%%%%%%%%%%%%%%%%%%%%%%%%%%%%%%%%%%%%%%%%%%%%%%%%%%%%%%%%%%%%%%%%%%%%%%%%%%%
%%%%%%%%%%%%%%%%%%%%%%%%%%%%%%%%%%%%%%%%%%%%%%%%%%%%%%%%%%%%%%%%%%%%%%%%%%%%%%%%
\appendix

\settowidth\MacroIndent{\rmfamily\scriptsize 000\ }

 \DocInput{childdoc.dtx}

\end{document}
%</driver>
% \fi
%
% %%%%%%%%%%%%%%%%%%%%%%%%%%%%%%%%%%%%%%%%%%%%%%%%%%%%%%%%%%%%%%%%%%%%%%%%%%%%%%
% %%%%%%%%%%%%%%%%%%%%%%%%%%%%%%%%%%%%%%%%%%%%%%%%%%%%%%%%%%%%%%%%%%%%%%%%%%%%%%
% \section{Sample}
%\iffalse
%<*samplemain>
%\fi
%
% The following presents a sample document
% with two chapters, two parts, a title page,
% a compile flag as well as three forwarding files to set the flag.
% It consists of eight |.tex| files:
% \begin{center}
% \begin{tabular}{ll}
% |cdocsamp.tex|&main file\\
% |cdocsch1.tex|&include file for chapter 1\\
% |cdocsch2.tex|&include file for chapter 2\\
% |cdocspt3.tex|&include file for part 3\\
% |cdocspt4.tex|&include file for part 4\\
% |cdocsdrf.tex|&forwarding file for main file in draft mode\\
% |cdocsfi1.tex|&forwarding file for final version of chapter 1\\
% |cdocsfi2.tex|&forwarding file for final version of chapter 2\\
% \end{tabular}
% \end{center}
% Each of the eight files can be compiled directly by the \LaTeX{} compiler.
%
% %%%%%%%%%%%%%%%%%%%%%%%%%%%%%%%%%%%%%%
% \paragraph{Main File.}
%
% The main file is called |cdocsamp.tex|.
%
% Load the \textsf{childdoc} definitions and
% declare the filename for the main document:
%    \begin{macrocode}
\input{childdoc.def}
\childdocmain{}
%    \end{macrocode}

% Optional override for |\version| flag:
%    \begin{macrocode}
%%\ifchilddoc\else\providecommand{\version}{draft}\fi
%    \end{macrocode}

% Define the default values for the |\version| flag
% (|final| for the main file and |draft| for childs):
%    \begin{macrocode}
\ifchilddoc
\providecommand{\version}{draft}
\else
\providecommand{\version}{final}
\fi
%    \end{macrocode}

% Load the standard document class:
%    \begin{macrocode}
\documentclass[12pt]{article}
%    \end{macrocode}

% Start the document body:
%    \begin{macrocode}
\begin{document}
%    \end{macrocode}

% Declare a title page.
% Print title, part of document being processed and version flag:
%    \begin{macrocode}
\addtocounter{page}{-1}
\begin{center}
{\LARGE\bfseries{}childdoc example\par}
\vspace{1cm}
\ifchilddoc
\ifchilddocmanual part\else chapter\fi:
`\childdocname' of `\childdocjob'\par
\else
main document: `\childdocjob'\par
\fi
version: \version\par
\end{center}
\newpage
%    \end{macrocode}

% Manually include selected file,
% otherwise process as usual:
%    \begin{macrocode}
\ifchilddocmanual
\section*{part `\childdocname'}
\input{\childdocname}
\else
%    \end{macrocode}

% Include the two chapters:
%    \begin{macrocode}
\include{cdocsch1}
\include{cdocsch2}
%    \end{macrocode}

% Include the two parts unless only chapters should be displayed:
%    \begin{macrocode}
\ifchilddoc\else
\section{part three}
\input{cdocspt3}
\section{part four}
\input{cdocspt4}
\fi
%    \end{macrocode}

% Process as usual until here:
%    \begin{macrocode}
\fi
%    \end{macrocode}

% End of document body:
%    \begin{macrocode}
\end{document}
%    \end{macrocode}
%\iffalse
%</samplemain>
%\fi
%
% %%%%%%%%%%%%%%%%%%%%%%%%%%%%%%%%%%%%%%
% \paragraph{Chapter Include Files.}
%
% The include files are called |cdocsch1.tex| and |cdocsch2.tex|.
%
%\iffalse
%<*samplechap1|samplechap2>
%\fi

% Optional override for |\version| flag:
%    \begin{macrocode}
%%\providecommand{\version}{final}
%    \end{macrocode}

% Include the main document:
%    \begin{macrocode}
\input{childdoc.def}
\childdocof{cdocsamp}
%    \end{macrocode}

%\iffalse
%</samplechap1|samplechap2>
%\fi
%
%\iffalse
%<*samplechap1>
%\fi
% Some text for chapter 1:
%    \begin{macrocode}
\section{one}
some text in chapter one
%    \end{macrocode}

%\iffalse
%</samplechap1>
%\fi
% Some text for chapter 2:
%\iffalse
%<*samplechap2>
%\fi
%    \begin{macrocode}
\section{two}
more text in chapter two
%    \end{macrocode}

%\iffalse
%</samplechap2>
%\fi
%
% %%%%%%%%%%%%%%%%%%%%%%%%%%%%%%%%%%%%%%
% \paragraph{Part Include Files.}
%
% The include files are called |cdocspt3.tex| and |cdocspt4.tex|.
%
%\iffalse
%<*samplepart3|samplepart4>
%\fi

% Optional override for |\version| flag:
%    \begin{macrocode}
%%\providecommand{\version}{final}
%    \end{macrocode}

% Include the main document:
%    \begin{macrocode}
\input{childdoc.def}
\childdocby{cdocsamp}
%    \end{macrocode}

%\iffalse
%</samplepart3|samplepart4>
%\fi
%
%\iffalse
%<*samplepart3>
%\fi
% Some text for part 3:
%    \begin{macrocode}
some text in part three
%    \end{macrocode}

%\iffalse
%</samplepart3>
%\fi
% Some text for part 4:
%\iffalse
%<*samplepart4>
%\fi
%    \begin{macrocode}
more text in part four
%    \end{macrocode}

%\iffalse
%</samplepart4>
%\fi
%
% %%%%%%%%%%%%%%%%%%%%%%%%%%%%%%%%%%%%%%
% \paragraph{Forwarding for a Complete Draft.}
%
% The following forwarding file |cdocsdrf.tex|
% compiles the main document in draft mode:
%\iffalse
%<*sampledraft>
%\fi
%    \begin{macrocode}
\def\version{draft}
\input{childdoc.def}
\childdocforward{cdocsamp}
%    \end{macrocode}

%\iffalse
%</sampledraft>
%\fi
%
% %%%%%%%%%%%%%%%%%%%%%%%%%%%%%%%%%%%%%%
% \paragraph{Forwarding for Final Version of the Chapters.}
%
% The following forwarding files |cdocsfn1.tex| and |cdocsfn2.tex|
% (with identical content)
% compile the final versions of the child documents
% |cdocsch1.tex| and |cdocsch2.tex|, respectively:
%\iffalse
%<*samplefinal>
%\fi
%    \begin{macrocode}
\def\version{final}
\input{childdoc.def}
\childdocforwardprefix[cdocsamp]{cdocsfn}{cdocsch}
%    \end{macrocode}

%\iffalse
%</samplefinal>
%\fi
%
% %%%%%%%%%%%%%%%%%%%%%%%%%%%%%%%%%%%%%%
% \paragraph{Command Line Processing.}
%
% The following three command lines generate the output files
% |cdocscld|, |cdocscl1| and |cdocscl2|
% which should be identical to
% |cdocsdrf|, |cdocsch1| and |cdocsfn2|, respectively:
% \begin{center}
% \begin{tabular}{l}
% |latex -jobname cdocscld \|\\
% |  "\def\version{draft}\input{childdoc.def}\childdocforward{cdocsamp}"|\\
% |latex -jobname cdocscl1 \|\\
% |  "\input{childdoc.def}\childdocforward[cdocsamp]{cdocsch1}"|\\
% |latex -jobname cdocscl2 \|\\
% |  "\def\version{final}\input{childdoc.def}\childdocforward{cdocsch2}"|
% \end{tabular}
% \end{center}
% Note that the trailing backslash on each first line
% merely continues the input to the second line
% (for convenient cut ant paste).
% Furthermore, the command |latex| can be replaced by any
% of its alternative versions such as |pdflatex|.
%
% %%%%%%%%%%%%%%%%%%%%%%%%%%%%%%%%%%%%%%%%%%%%%%%%%%%%%%%%%%%%%%%%%%%%%%%%%%%%%%
% %%%%%%%%%%%%%%%%%%%%%%%%%%%%%%%%%%%%%%%%%%%%%%%%%%%%%%%%%%%%%%%%%%%%%%%%%%%%%%
% \section{Implementation}
%\iffalse
%<*package>
%\fi
%
% This section describes the definitions file |childdoc.def|.

% The definitions cannot be loaded using |\usepackage| or |\RequirePackage|
% which has a mechanism to prevent loading a style file more than once.
% When loading the definitions by means of |\input|
% multiple instances have to be prevented manually:
%\iffalse
%This code needs to be before the `\ProvidesFile' directive
%which is defined at the beginning of this file.
%Therefore it is also placed there and commented out here.
%</package>
%<*discard>
%\fi
%    \begin{macrocode}
\ifdefined\childdocmain\endinput\fi
%    \end{macrocode}
%\iffalse
%</discard>
%<*package>
%\fi
%
% \macro{\ifchilddoc}
% \macro{\ifchilddocmanual}
% The conditional |\ifchilddoc| tells whether a
% child (true) or main (false) document is being compiled.
% The conditional |\ifchilddocmanual| tells whether
% the |\includeonly| mechanism is used (false) or
% the selection of child files must be performed manually (true).
% The definitions initialise to false:
%    \begin{macrocode}
\newif\ifchilddoc
\newif\ifchilddocmanual
%    \end{macrocode}

% \macro{\childdocname}
% \macro{\childdocjob}
% The macro |\childdocname| stores the name of the main document
% to be compiled. The macro |\childdocjob| stores the name of
% the document on which the \LaTeX{} compiler was originally invoked.
% The content of |\jobname| cannot be compared
% to filenames specified in the source due to different catcodes.
% The following code rescans |\jobname|, stores the result
% in |\childdocname| and saves a copy in |\childdocjob|:
%    \begin{macrocode}
\edef\childdocname{\scantokens\expandafter{\jobname\noexpand}}
\let\childdocjob\childdocname
%    \end{macrocode}

% \macro{\childdocdisable}
% The macro |\childdocdisable| prevents the main file
% from being processed more than once.
% At this stage, the main document command |\childdocmain|
% is assumed to be called once again where it should do nothing.
% Any subsequent call to it should prevent
% a secondary processing of the main document
% It overwrites the forwarding commands
% |\childdocof| and |\childdocforward|
% with empty macros to prevent further inclusions of the main document:
%    \begin{macrocode}
\newcommand{\childdocdisable}
{
  \renewcommand{\childdocmain}[1]{\renewcommand{\childdocmain}[1]{\endinput}}
  \renewcommand{\childdocof}[1]{}
  \renewcommand{\childdocby}[2][]{}
  \renewcommand{\childdocforward}[2][]{}
  \renewcommand{\childdocdisable}{}
}
%    \end{macrocode}

% \macro{\childdocmain}
% The macro |\childdocmain| is to be called at the top of the main file
% with nothing or the main filename (without extension) as argument.
% First, it breaks loops.
% If the argument is not empty and does not match |\childdocname|
% (which is set by the first inclusion of |childdoc.def|),
% |\ifchilddoc| is set to true, |\includeonly| is applied to the child file
% and |\jobname| is set to the main file
% (for proper handling of |.aux| files):
%    \begin{macrocode}
\newcommand{\childdocmain}[1]
{
  \childdocdisable\childdocmain{}
  \if?#1?\else
    \begingroup
      \def\childdoctmp{#1}
      \ifx\childdoctmp\childdocname
        \def\childdoctmp{}
      \else
        \def\childdoctmp
        {
          \childdoctrue
          \includeonly{\childdocname}
          \def\childdocjob{#1}
          \def\jobname{#1}
        }
      \fi
      \expandafter
    \endgroup
    \childdoctmp
  \fi
}
%    \end{macrocode}

% \macro{\childdocof}
% The command |\childdocof| redirects
% compilation to the main file |#1|.
%    \begin{macrocode}
\newcommand{\childdocof}[1]
{
  \childdocdisable
  \childdoctrue
  \includeonly{\childdocname}
  \def\jobname{#1}
  \def\childdocjob{#1}
  \input{#1}
}
%    \end{macrocode}

% \macro{\childdocby}
% The command |\childdocby| ....
%    \begin{macrocode}
\newcommand{\childdocby}[2][]
{
  \childdocdisable
  \childdoctrue
  \childdocmanualtrue
  \if?#1?\else
    \def\jobname{#2}
  \fi
  \def\childdocjob{#2}
  \input{#2}
  \endinput
}
%    \end{macrocode}

% \macro{\childdocforward}
% The command |\childdocforward| redirects
% compilation to the main file or
% (if the optional argument is given) a child file.
% Parameters are set as if the main file
% or a child file starting with |\childdocof| was compiled.
% Then compilation is handed over to the main file:
%    \begin{macrocode}
\newcommand{\childdocforward}[2][]
{
  \begingroup
    \if?#1?
      \def\childdoctmp
      {
        \def\childdocname{#2}
        \def\childdocjob{#2}
        \def\jobname{#2}
        \input{#2}
        \endinput
      }
    \else
      \def\childdoctmp
      {
        \childdocdisable
        \def\childdocname{#2}
        \childdoctrue
        \includeonly{#2}
        \def\childdocjob{#1}
        \def\jobname{#1}
        \input{#1}
        \endinput
      }
    \fi
    \expandafter
  \endgroup
  \childdoctmp
}
%    \end{macrocode}

% \macro{\childdocforwardprefix}
% The command |\childdocforwardprefix| redirects
% compilation to the main or a child file by means of a pattern.
% The prefix |#1| in the current filename is replaced by |#2|
% and the suffix of the current filename is kept
% (it is assumed that the filename does not contain the substring `|~~~|'
% which is used as a delimiter).
% Compilation is handed over to the new file by |\childdocforward|:
%    \begin{macrocode}
\newcommand{\childdocforwardprefix}[3][]
{
  \begingroup
    \def\childdocextract #2##1~~~{\def\childdoctmp{\childdocforward[#1]{#3##1}}}
    \expandafter\childdocextract\childdocname~~~
    \expandafter
  \endgroup
  \childdoctmp
}
%    \end{macrocode}

% \macro{\childdoc}
% The deprecated macro |\childdoc| is a legacy version of |\childdocmain|:
%    \begin{macrocode}
\newcommand{\childdoc}{\childdocmain}
%    \end{macrocode}

% \macro{\childdocredirect}
% The deprecated macro |\childdocredirect| is a legacy version
% of |\childdocforward| and |\childdocforwardprefix|:
%    \begin{macrocode}
\newcommand{\childdocredirect}[2][]
{
  \begingroup
    \if?#1?
      \def\childdoctmp{\childdocforward{#2}}
    \else
      \def\childdoctmp{\childdocforwardprefix{#1}{#2}}
    \fi
    \expandafter
  \endgroup
  \childdoctmp
}
%    \end{macrocode}

%\iffalse
%</package>
%\fi
%
\endinput
|\\
|\childdocforwardprefix[|\textit{main}|]{|\textit{prefix}|}{|\textit{dest}|}|
\end{tabular}
\end{center}
%
the destination file is determined by a pattern
depending on the current file:
To make this work, the current file must be called
`{\textit{prefix}\hspace{0.2em}\textit{suffix}}'
with \textit{prefix} matching precisely the argument.
Processing is then passed on to the file
`{\textit{dest}\hspace{0.2em}\textit{suffix}}'.
Surely, the same effect is achieved by
directly specifying the
argument `{\textit{dest}\hspace{0.2em}\textit{suffix}}'
in the first form.
However, that requires to set up a different file
for each child. With the alternative form of the command
all these files can have exactly the same content
which simplifies setting them up and maintaining them.

For example, the following file |draft.tex|
with a compilation flag |\version| as described in \secref{sec:flags}
compiles the main document as a draft:
%
\begin{center}
\begin{tabular}{l}
|\def\version{draft}|\\
|% \iffalse
%
% childdoc.dtx Copyright (C) 2017-2018 Niklas Beisert
%
% This work may be distributed and/or modified under the
% conditions of the LaTeX Project Public License, either version 1.3
% of this license or (at your option) any later version.
% The latest version of this license is in
%   http://www.latex-project.org/lppl.txt
% and version 1.3 or later is part of all distributions of LaTeX
% version 2005/12/01 or later.
%
% This work has the LPPL maintenance status `maintained'.
%
% The Current Maintainer of this work is Niklas Beisert.
%
% This work consists of the files childdoc.dtx and childdoc.ins
% and the derived files childdoc.def and cdocsamp.tex with
% cdocsch1.tex, cdocsch2.tex, cdocsdrf.tex, cdocsfn1.tex, cdocsfn2.tex.
%
%<package>\ifdefined\childdocmain\endinput\fi
%<package>\ProvidesFile{childdoc.def}[2018/12/30 v2.0 child document driver]
%<samplemain>\ProvidesFile{cdocsamp.tex}[2018/12/30 v2.0 sample for childdoc]
%<*driver>
%\ProvidesFile{childdoc.drv}[2018/12/30 v2.0 childdoc reference manual file]
\PassOptionsToClass{10pt,a4paper}{article}
\documentclass{ltxdoc}

\usepackage[margin=35mm]{geometry}
\usepackage{hyperref}
\usepackage{hyperxmp}
\usepackage[usenames]{color}

\hypersetup{colorlinks=true}
\hypersetup{pdfstartview=FitH}
\hypersetup{pdfpagemode=UseNone}
\hypersetup{pdfsource={}}
\hypersetup{pdflang={en-UK}}
\hypersetup{pdfcopyright={Copyright 2017-2018 Niklas Beisert.
  This work may be distributed and/or modified under the
  conditions of the LaTeX Project Public License, either version 1.3
  of this license or (at your option) any later version.}}
\hypersetup{pdflicenseurl={http://www.latex-project.org/lppl.txt}}
\hypersetup{pdfcontactaddress={ETH Zurich, ITP, HIT K,
  Wolfgang-Pauli-Strasse 27}}
\hypersetup{pdfcontactpostcode={8093}}
\hypersetup{pdfcontactcity={Zurich}}
\hypersetup{pdfcontactcountry={Switzerland}}
\hypersetup{pdfcontactemail={nbeisert@itp.phys.ethz.ch}}
\hypersetup{pdfcontacturl={http://people.phys.ethz.ch/\xmptilde nbeisert/}}

\newcommand{\secref}[1]{\hyperref[#1]{section \ref*{#1}}}

\parskip1ex
\parindent0pt
\let\olditemize\itemize
\def\itemize{\olditemize\parskip0pt}

\begin{document}

\title{The \textsf{childdoc} Package}
\hypersetup{pdftitle={The childdoc Package}}
\author{Niklas Beisert\\[2ex]
  Institut f\"ur Theoretische Physik\\
  Eidgen\"ossische Technische Hochschule Z\"urich\\
  Wolfgang-Pauli-Strasse 27, 8093 Z\"urich, Switzerland\\[1ex]
  \href{mailto:nbeisert@itp.phys.ethz.ch}
  {\texttt{nbeisert@itp.phys.ethz.ch}}}
\hypersetup{pdfauthor={Niklas Beisert}}
\hypersetup{pdfsubject={Manual for the LaTeX2e Package childdoc}}
\date{30 December 2018, \textsf{v2.0}}
\maketitle

\begin{abstract}\noindent
\textsf{childdoc} is a \LaTeXe{} package
that enables the direct compilation
of document sections included by |\include|
to individual files.
\end{abstract}

\begingroup
\parskip0ex
\tableofcontents
\endgroup

%%%%%%%%%%%%%%%%%%%%%%%%%%%%%%%%%%%%%%%%%%%%%%%%%%%%%%%%%%%%%%%%%%%%%%%%%%%%%%%%
%%%%%%%%%%%%%%%%%%%%%%%%%%%%%%%%%%%%%%%%%%%%%%%%%%%%%%%%%%%%%%%%%%%%%%%%%%%%%%%%
\section{Introduction}

\LaTeX{} provides a mechanism to structure a large document (such as a book)
into a main file and several child files (containing the chapters)
using the |\include| command.
This mechanism is beneficial for documents
which span hundreds of pages in order to
make the source file(s) more manageable.
Moreover, compilation can be restricted to
selected child files by means of the |\includeonly| command.
The latter feature can be used to reduce the compilation time while editing
(this was significantly more useful in the earlier days of \LaTeX{})
or to generate a smaller document which is easier to navigate.
Another application of |\includeonly| is to generate
documents consisting of selected parts of the complete document.

However, there are a few drawbacks of the plain |\include| mechanism:
\begin{itemize}
\item
The child files cannot be compiled on their own,
they can only be compiled via the main file.
A naive editing environment
(such as a text editor with an option
to have the current file processed by \LaTeX)
may require one to switch to the main file before compiling;
attempting to compile the child file produces errors.
\item
The main file must be modified (each time)
to adjust the |\includeonly| command
to the present needs. This easily leaves the main file in a messy state.
\item
The generated document will always carry the filename
of the main document. This is inconvenient if
several child files are to be compiled and
to be kept for distribution.
\end{itemize}

The present package provides a simple interface
to make child files individually compilable by \LaTeX{}.
Compiling a child file then has the same effect as compiling
the main file with an |\includeonly| command
to select the appropriate child.
Moreover the generated document will carry the name of the child
rather than the main file.
This resolves all three above issues.

This feature is meant to make the editing of books,
thesis documents and lecture notes somewhat more convenient.
However, the package can also be used efficiently for
composing a series of documents (such as exercise sheets)
which are typically distributed individually.
It then assists the author in generating the individual documents
(potentially in different versions)
as well as a document containing the collected series.
Another application is in developing style files
or other kinds of included material
where compilation of the style file could redirect
to a sample or test file.

%%%%%%%%%%%%%%%%%%%%%%%%%%%%%%%%%%%%%%%%%%%%%%%%%%%%%%%%%%%%%%%%%%%%%%%%%%%%%%%%
%%%%%%%%%%%%%%%%%%%%%%%%%%%%%%%%%%%%%%%%%%%%%%%%%%%%%%%%%%%%%%%%%%%%%%%%%%%%%%%%
\section{Usage}

First of all, the package \textsf{childdoc} is \emph{not} a standard
\LaTeXe{} |.sty| style file! Therefore it needs to be invoked in
a non-standard way.

%%%%%%%%%%%%%%%%%%%%%%%%%%%%%%%%%%%%%%%%%%%%%%%%%%%%%%%%%%%%%%%%%%%%%%%%%%%%%%%%
\subsection{Included Files}
\label{sec:include}

%%%%%%%%%%%%%%%%%%%%%%%%%%%%%%%%%%%%%%%%
\DescribeMacro{\childdocmain}
To use the package, add the commands
\begin{center}
\begin{tabular}{l}
|\input{childdoc.def}|\\
|\childdocmain{}|\\
\end{tabular}
\end{center}
at the very top of the main \LaTeX{} file,
in particular \emph{before} the |\documentclass| statement!
The argument of |\childdocmain| should be left empty
(but it must be present).

%%%%%%%%%%%%%%%%%%%%%%%%%%%%%%%%%%%%%%%%
\DescribeMacro{\childdocof}
Furthermore, add the commands
\begin{center}
\begin{tabular}{l}
|\input{childdoc.def}|\\
|\childdocof{|\textit{main}|}|\\
\end{tabular}
\end{center}
at the top of every child file \textit{child}
which is included by |\include{|\textit{child}|}|
from within the main file
(or at least for those files to be compiled individually).
The argument \textit{main} must be the filename of the main file.

There are a couple of
considerations in setting up the main and child documents:

%%%%%%%%%%%%%%%%%%%%%%%%%%%%%%%%%%%%%%%%
\paragraph{Restrictions.}

Please note the following restrictions:
\begin{itemize}
\item
|\childdocmain| must be called with one argument \textit{main}
to ensure compatibility with earlier version of the package.
It must either be empty (|\childdocmain{}|)
or precisely match the filename of the main file in which it is specified.
See \secref{sec:detection} for further information.
\item
The filename \textit{main} must be specified without the |.tex| extension.
\item
The filename \textit{main} is case sensitive
(even in case-insensitive file systems)
due to internal string comparison.
\item
The argument \textit{main} should be fully expanded, it cannot be a macro.
\item
Subdirectories and special characters should be avoided in filenames.
\item
The command |\childdocmain{|\textit{main}|}| must be followed by a whitespace.
It should not be followed immediately by another command
or by a comment mark `|%|'.
This is because the \TeX{} parser reads the token immediately following
the argument of |\childdocmain| and puts it
at the beginning of every child section;
however, a white\-space is ignored.
\end{itemize}

%%%%%%%%%%%%%%%%%%%%%%%%%%%%%%%%%%%%%%%%
\paragraph{Content of Main File.}

It is advisable to place all content in the child files included by |\include|.
Any output contained in the main file will appear in all child documents
unless suppressed manually;
it cannot be suppressed automatically by the |\includeonly| directive
and thus should normally be avoided.
A method to include some content in the main file
by means of conditional processing is described in \secref{sec:conditional}.

%%%%%%%%%%%%%%%%%%%%%%%%%%%%%%%%%%%%%%%%
\paragraph{Page Numbering.}

When only a part of the document is compiled,
the appropriate numbering of pages
(as well as other status parameters)
is determined from the |.aux| files.
The latter contain information from previous passes.
However this information needs to propagate through
all intermediate child documents.
Therefore the page numbering in child documents may well
be inconsistent until the complete document is compiled at least once.

A useful (if unconventional) way to always ensure a consistent
page numbering is to restart the numbering in each child document
and denote the pages by `\textit{child}|.|\textit{page}'
where \textit{child} represents the chapter/section number of the child file.
This can be achieved by the command
|\numberwithin{page}{|\textit{child}|}|
of the \textsf{amsmath} package
where \textit{child} can be |chapter| or |section|
depending on the chosen structuring.
Alternatively, one can modify the macro |\thepage| appropriately
and reset the counter |page| at the start of each child file.

%%%%%%%%%%%%%%%%%%%%%%%%%%%%%%%%%%%%%%%%%%%%%%%%%%%%%%%%%%%%%%%%%%%%%%%%%%%%%%%%
\subsection{Conditional Processing}
\label{sec:conditional}

The package provides a mechanism to compile different versions
of a document. To customise the versions further some conditional processing
can come in handy to distinguish which version is being compiled.
The package provides two macros to describe the compilation context:

%%%%%%%%%%%%%%%%%%%%%%%%%%%%%%%%%%%%%%%%
\DescribeMacro{\ifchilddoc}
The conditional |\ifchilddoc| distinguishes between the compilation of
child documents and the main document:
%
\begin{center}
|\ifchilddoc |\textit{child-code}| |[|\||else |\textit{main-code}]| \||fi|
\end{center}

%%%%%%%%%%%%%%%%%%%%%%%%%%%%%%%%%%%%%%%%
\DescribeMacro{\childdocname}
\DescribeMacro{\childdocjob}
The macro |\childdocname| contains the filename (without extension)
of the main or child file being processed.
Note that |\childdocjob| will always contain the name of the main file.

%%%%%%%%%%%%%%%%%%%%%%%%%%%%%%%%%%%%%%%%
\paragraph{Title Page.}

Conditional processing can be used to include a title or banner page
in the main document when proper precautions are taken.
Importantly, the code in the main file should ensure that the page counter
(as well as other status parameters which are stored in the |.aux| files)
takes the same value after the conditional processing.
Otherwise the page numbers may take divergent values
depending on which part is compiled.

For example, a title page could be declared by:
%
\begin{center}
\begin{tabular}{l}
|\ifchilddoc\||else|\\
|\addtocounter{page}{-1}|\\
\textit{code for title page}\\
|\newpage|\\
|\||fi|
\end{tabular}
\end{center}
%
A banner page for the child documents can be generated by:
%
\begin{center}
\begin{tabular}{l}
|\ifchilddoc|\\
|\addtocounter{page}{-1}|\\
\textit{code for banner page}\\
|\newpage|\\
|\||fi|
\end{tabular}
\end{center}
%
Here one could write a message such as:
\begin{center}
|This is the part \childdocname{} of \childdocjob{}.|
\end{center}

%%%%%%%%%%%%%%%%%%%%%%%%%%%%%%%%%%%%%%%%%%%%%%%%%%%%%%%%%%%%%%%%%%%%%%%%%%%%%%%%
\subsection{Flags}
\label{sec:flags}

The package makes it easy to generate different versions
of the main or child documents.
To this end compilation flags can be defined
and assigned different default values.
They will be particularly useful in conjunction
with the forwarding mechanism described in \secref{sec:forward}.

For example, it may be useful to have a flag |\version|
which can be set to |draft| or |final|.
The document source will contain some conditional code
depending on the value of |\version|.
Suppose further, the flag should default to |final| for the main file
and to |draft| for child files
which is a natural assignment for editing the document.
This is achieved by placing the following code
in the preamble of the main document
(below the |\childdocmain| directive):
%
\begin{center}
\begin{tabular}{l}
|\ifchilddoc|\\
|\providecommand{\version}{draft}|\\
|\||else|\\
|\providecommand{\version}{final}|\\
|\||fi|
\end{tabular}
\end{center}
%
The definition by |\providecommand| makes sure
that previous definitions are not overwritten.
Further statements |\providecommand{\version}{...}|
can thus be added before the above code to override it.

For the main file, one might add a line
(between |\childdocmain| and the above block)
%
\begin{center}
|%\ifchilddoc\||else\providecommand{\version}{draft}\||fi|
\end{center}
%
which can be uncommented to produce a draft version.
Likewise one can add a line to the very top of a child file
(above the |\childdocof{|\textit{main}|}| directive)
%
\begin{center}
|%\providecommand{\version}{final}|
\end{center}
%
which can be uncommented to produce the final version of this child document.

%%%%%%%%%%%%%%%%%%%%%%%%%%%%%%%%%%%%%%%%%%%%%%%%%%%%%%%%%%%%%%%%%%%%%%%%%%%%%%%%
\subsection{Forwarding}
\label{sec:forward}

Different versions of the main or child documents
using compilation flags as described in \secref{sec:flags}
can be (permanently) stored in different files
for convenient compilation, viewing and distribution.
To this end, the package defines a command
to pass on compilation to a different file:

%%%%%%%%%%%%%%%%%%%%%%%%%%%%%%%%%%%%%%%%
\DescribeMacro{\childdocforward}
The command |\childdocforward| redirects processing to
another source file:
%
\begin{center}
\begin{tabular}{l}
|\input{childdoc.def}|\\
|\childdocforward[|\textit{main}|]{|\textit{dest}|}|\\
\end{tabular}
\end{center}
%
The argument \textit{dest} is the destination file
(without extension).
It should be the main file or one of the child files.
Note that further \textsf{childdoc} directives
such as |\childdocof| and |\childdocforward|
in the indicated file will be processed in this form.
The optional argument \textit{main}
passes on directly to the main file \textit{main}
while pretending to compile the child \textit{dest}.
This form behaves as if \textit{dest}
issues |\childdocof{|\textit{main}|}| right away,
and no further \textsf{childdoc} directives will be processed.

%%%%%%%%%%%%%%%%%%%%%%%%%%%%%%%%%%%%%%%%
\DescribeMacro{\...prefix}
In the alternative form |\childdocforwardprefix|,
%
\begin{center}
\begin{tabular}{l}
|\input{childdoc.def}|\\
|\childdocforwardprefix[|\textit{main}|]{|\textit{prefix}|}{|\textit{dest}|}|
\end{tabular}
\end{center}
%
the destination file is determined by a pattern
depending on the current file:
To make this work, the current file must be called
`{\textit{prefix}\hspace{0.2em}\textit{suffix}}'
with \textit{prefix} matching precisely the argument.
Processing is then passed on to the file
`{\textit{dest}\hspace{0.2em}\textit{suffix}}'.
Surely, the same effect is achieved by
directly specifying the
argument `{\textit{dest}\hspace{0.2em}\textit{suffix}}'
in the first form.
However, that requires to set up a different file
for each child. With the alternative form of the command
all these files can have exactly the same content
which simplifies setting them up and maintaining them.

For example, the following file |draft.tex|
with a compilation flag |\version| as described in \secref{sec:flags}
compiles the main document as a draft:
%
\begin{center}
\begin{tabular}{l}
|\def\version{draft}|\\
|\input{childdoc.def}|\\
|\childdocforward{|\textit{main}|}|
\end{tabular}
\end{center}
%
Likewise, the following files |final|\textit{nn}|.tex|
compile the final version of the child document
|child|\textit{nn}|.tex|:
%
\begin{center}
\begin{tabular}{l}
|\def\version{final}|\\
|\input{childdoc.def}|\\
|\childdocforwardprefix{final}{child}|
\end{tabular}
\end{center}
%

Note that when several versions of a main file and/or of each child file
are to be generated, it may be convenient to set up a |Makefile| or
shell script to automatise the process.

%%%%%%%%%%%%%%%%%%%%%%%%%%%%%%%%%%%%%%%%%%%%%%%%%%%%%%%%%%%%%%%%%%%%%%%%%%%%%%%%
\subsection{Command Line Processing}
\label{sec:commandline}

The effect of redirection files can also be achieved by invoking
the \LaTeX{} compiler with a more elaborate command line.
Most conveniently this should be done as part
of a shell script or a |Makefile|.

When using \textsf{childdoc} in the main file, the following
command lines effectively perform a redirection
(note that depending on the shell being used,
backslashes may have to be doubled: `|\|' $\to$ `|\\|'):
%
\begin{center}
|... -jobname "|\textit{target}|" |\\|"|[\textit{flags}]%
|\input{childdoc.def}\childdocforward[|\textit{main}|]{|\textit{dest}|}"|
\end{center}
%
Here \textit{target} is the name of the output file,
\textit{main} is the name of the main file
and \textit{dest} is the name of the main or child file to be processed
(all filenames without extensions).
The optional argument \textit{main} can be omitted
if \textit{main} matches \textit{dest}.
Optionally, compilation \textit{flags} can be defined via |\def| commands.
This command line makes the \TeX{} engine believe
it is compiling the file \textit{target}
whose content is specified as the latter parameter.
The provided code then forwards the processing to
\textit{main} or \textit{dest} as described in \secref{sec:forward}.

%%%%%%%%%%%%%%%%%%%%%%%%%%%%%%%%%%%%%%%%%%%%%%%%%%%%%%%%%%%%%%%%%%%%%%%%%%%%%%%%
\subsection{Include by Input}
\label{sec:input}

Including child documents by |\include| has some restrictions by design.
Most notably, the content of a child document always occupies
its own set of pages; pages cannot be shared between child documents.
Usually, this behaviour makes perfect sense
because each child document contain an essential part of the document.
However, in some situations it may be desirable to compose
a document from a collection of parts
without having mandatory page breaks between then.
For this case, the package
provides a mechanism to include parts
by |\input| which can also be processed individually.
However, by construction this mechanism
requires manual handling of the content to be output.

%%%%%%%%%%%%%%%%%%%%%%%%%%%%%%%%%%%%%%%%
\DescribeMacro{\ifchilddocmanual}
The main file should be prepared as usual, see \secref{sec:include}.
However, the document body must make a distinction
between processing of an individual part and of the main document, e.g.:
%
\begin{center}
\begin{tabular}{l}
|\ifchilddocmanual|\\
|\input{\childdocname}|\\
|\||else|\\
\textit{document body with }|\input{|\textit{part}|}|\\
|\||fi|
\end{tabular}
\end{center}
%
The conditional |\ifchilddocmanual| is true whenever
a part to be included by |\input| is being compiled,
and the name of the part is stored in |\childdocname|.

%%%%%%%%%%%%%%%%%%%%%%%%%%%%%%%%%%%%%%%%
\DescribeMacro{\childdocby}
Each part to be included by |\input| should start with:
%
\begin{center}
\begin{tabular}{l}
|\input{childdoc.def}|\\
|\childdocby{|\textit{main}|}|\\
\end{tabular}
\end{center}
%
The directive |\childdocby| is similar to |\childdocof|
described in \secref{sec:include},
but the subsequent selection of content must be done manually.
To that end, both |\ifchilddoc| and |\ifchilddocmanual|
will be true upon processing of a part,
and the name of the part is stored in |\childdocname|.
Note that |\jobname| will be set to the filename of the current part
so that each part receives an individual |.aux| file
that does not interfere with the |.aux| file(s) of the main document.
This behaviour can be altered by the alternative form
|\childdocby[*]{|\textit{main}|}| (with a non-empty optional argument)
which uses the |.aux| file of the main document
by setting |\jobname| to \textit{main}.

%%%%%%%%%%%%%%%%%%%%%%%%%%%%%%%%%%%%%%%%%%%%%%%%%%%%%%%%%%%%%%%%%%%%%%%%%%%%%%%%
\subsection{Driver Development}
\label{sec:driver}

The \textsf{childdoc} mechanism can also be use for the development
of definition files such as \LaTeX{} styles or classes.
This case differs from the above setup with multiple parts
included by |\include| in that no |\includeonly| should be invoked.
This can be achieved by starting the include file
(before |\ProvidesPackage|) with:
%
\begin{center}
\begin{tabular}{l}
|\input{childdoc.def}|\\
|\childdocforward{|\textit{main}|}|\\
\end{tabular}
\end{center}
%
or alternatively with:
%
\begin{center}
\begin{tabular}{l}
|\input{childdoc.def}|\\
|\childdocby{|\textit{main}|}|\\
\end{tabular}
\end{center}
%
Both forms have slightly different effects as described above.
The main file is prepared as usual, see \secref{sec:include}.

%%%%%%%%%%%%%%%%%%%%%%%%%%%%%%%%%%%%%%%%%%%%%%%%%%%%%%%%%%%%%%%%%%%%%%%%%%%%%%%%
\subsection{Legacy Detection}
\label{sec:detection}

The directive |\childdocmain| in the main file can detect
whether the complete document or merely a child is to be compiled
even without using the directive |\childdocof|.
This method is deprecated because it is less robust
and there is no compelling reason to use it;
it is merely provided for backward compatibility
and it may be removed in future versions.

If the detection mechanism is to be used,
it is mandatory to correctly specify
the filename of the main file as the argument of |\childdocmain|:
%
\begin{center}
\begin{tabular}{l}
|\input{childdoc.def}|\\
|\childdocmain{|\textit{main}|}|\\
\end{tabular}
\end{center}
%
If |\jobname| does not match the argument \textit{main} of |\childdocmain|,
it is assumed that |\jobname| points to the child file to be compiled.
When using |\childdocmain| with the main file specified as argument,
it suffices to start a child file
with just |\input{|\textit{main}|}|
without loading of the package and using |\childdocof|.
If instead all processing is done
with the appropriate \textsf{childdoc} directives,
the argument of \textit{main} of |\childdocmain| can be empty.

An alternative version of the command line processing described
in \secref{sec:commandline} using the detection mechanism reads:
%
\begin{center}
|... -jobname "|\textit{target}|" "|[\textit{flags}]%
[|\def\jobname{|\textit{dest}|}|]|\input{|\textit{main}|}"|
\end{center}

%%%%%%%%%%%%%%%%%%%%%%%%%%%%%%%%%%%%%%%%%%%%%%%%%%%%%%%%%%%%%%%%%%%%%%%%%%%%%%%%
\subsection{Manual Code}
\label{sec:manual}

In case one cannot be certain whether the definitions file |childdoc.def|
is installed on the target \TeX{} distribution
and one prefers not to ship it,
it is conceivable to paste a few relevant commands into the sources.

To that end, drop all statements |\input{childdoc.def}|
and perform the replacements as outlined below.
Instead of |\childdocmain{|\textit{main}|}| add the following code
to the top of the main file:
%
\begin{center}
\begin{tabular}{l}
|\||ifdefined\childdocname\endinput\||fi\newif\ifchilddoc|\\
|\edef\childdocname{\scantokens\expandafter{\jobname\noexpand}}|\\
|\def\childdocmain{|\textit{main}|}\||ifx\childdocmain\childdocname\||else|\\
|\childdoctrue\includeonly{\childdocname}\let\jobname\childdocmain\||fi|\\
\end{tabular}
\end{center}
%
Instead of |\childdocof{|\textit{main}|}| just include the main file
at the top of each child file:
%
\begin{center}
|\input{|\textit{main}|}|
\end{center}
%
A simple redirection |\childdocforward{|\textit{dest}|}| is achieved by:
%
\begin{center}
|\def\jobname{|\textit{dest}|}\input{\jobname}|
\end{center}
%
The redirection with prefix
|\childdocforwardprefix[|\textit{prefix}|]{|\textit{dest}|}|
is accomplished by:
%
\begin{center}
\begin{tabular}{l}
|{\edef\jobname{\scantokens\expandafter{\jobname\noexpand}}|\\
|\def\redirectjob |\textit{prefix}|#1~~~{\gdef\jobname{|\textit{dest}|#1}}|\\
|\expandafter\redirectjob\jobname~~~}\input{\jobname}|
\end{tabular}
\end{center}

In an alternative approach,
child documents can be compiled by a specific command line
without additional code or specific definitions:
%
\begin{center}
|... -jobname "|\textit{target}|" "|[\textit{flags}]%
|\includeonly{|\textit{dest}|}\input{|\textit{main}|}"|
\end{center}
%

%%%%%%%%%%%%%%%%%%%%%%%%%%%%%%%%%%%%%%%%%%%%%%%%%%%%%%%%%%%%%%%%%%%%%%%%%%%%%%%%
%%%%%%%%%%%%%%%%%%%%%%%%%%%%%%%%%%%%%%%%%%%%%%%%%%%%%%%%%%%%%%%%%%%%%%%%%%%%%%%%
\section{Information}

%%%%%%%%%%%%%%%%%%%%%%%%%%%%%%%%%%%%%%%%%%%%%%%%%%%%%%%%%%%%%%%%%%%%%%%%%%%%%%%%
\subsection{Copyright}

Copyright \copyright{} 2017--2018 Niklas Beisert

This work may be distributed and/or modified under the
conditions of the \LaTeX{} Project Public License, either version 1.3
of this license or (at your option) any later version.
The latest version of this license is in
  \url{http://www.latex-project.org/lppl.txt}
and version 1.3 or later is part of all distributions of \LaTeX{}
version 2005/12/01 or later.

This work has the LPPL maintenance status `maintained'.

The Current Maintainer of this work is Niklas Beisert.

This work consists of the files |README.txt|, |childdoc.ins| and |childdoc.dtx|
as well as the derived files |childdoc.def|, |cdocsamp.tex|
with |cdocsch1.tex|, |cdocsch2.tex|, |cdocspt3.tex|, |cdocspt4.tex|,
|cdocsdrf.tex|, |cdocsfn1.tex|, |cdocsfn2.tex|
as well as |childdoc.pdf|.

%%%%%%%%%%%%%%%%%%%%%%%%%%%%%%%%%%%%%%%%%%%%%%%%%%%%%%%%%%%%%%%%%%%%%%%%%%%%%%%%
\subsection{Files and Installation}

The package consists of the files:
%
\begin{center}
\begin{tabular}{ll}
    |README.txt|   & readme file \\
    |childdoc.ins| & installation file \\
    |childdoc.dtx| & source file \\
    |childdoc.def| & definition file \\
    |cdocsamp.tex| & sample main file \\
    |cdocsch1.tex| & sample include file \\
    |cdocsch2.tex| & sample include file \\
    |cdocspt3.tex| & sample part file \\
    |cdocspt4.tex| & sample part file \\
    |cdocsdrf.tex| & sample redirection file \\
    |cdocsfn1.tex| & sample redirection file \\
    |cdocsfn2.tex| & sample redirection file \\
    |childdoc.pdf| & manual
\end{tabular}
\end{center}
%
The distribution consists of the files
|README.txt|, |childdoc.ins| and |childdoc.dtx|.
%
\begin{itemize}
\item
Run (pdf)\LaTeX{} on |childdoc.dtx|
to compile the manual |childdoc.pdf| (this file).
\item
Run \LaTeX{} on |childdoc.ins| to create the definitions file |childdoc.def|
and the sample |cdocsamp.tex| with include files
|cdocsch1.tex|, |cdocsch2.tex|, |cdocspt3.tex|, |cdocspt4.tex|,
|cdocsdrf.tex|, |cdocsfn1.tex|, |cdocsfn2.tex|.
Then copy the file |childdoc.def| to an appropriate directory of your \LaTeX{}
distribution, e.g.\ \textit{texmf-root}|/tex/latex/childdoc|.
\end{itemize}

%%%%%%%%%%%%%%%%%%%%%%%%%%%%%%%%%%%%%%%%%%%%%%%%%%%%%%%%%%%%%%%%%%%%%%%%%%%%%%%%
\subsection{Related CTAN Packages}

There are several other packages which offer a similar functionality:
%
\begin{itemize}
\item
The packages
\href{http://ctan.org/pkg/docmute}{\textsf{docmute}},
\href{http://ctan.org/pkg/includex}{\textsf{includex}} and
\href{http://ctan.org/pkg/standalone}{\textsf{standalone}}
provide commands to include only the document body of
a child file thus allowing both files to be compiled individually.
\item
The packages \href{http://ctan.org/pkg/subdocs}{\textsf{subdocs}}
and \href{http://ctan.org/pkg/subfiles}{\textsf{subfiles}}
provide structures in which the main and child documents can be
encapsulated and allowing them to be compiled individually.
The inclusion mechanism is different from the conventional |\include|.
\item
The package \href{http://ctan.org/pkg/combine}{\textsf{combine}}
is an elaborate solution to combine several documents into one.
\end{itemize}
%
See also the CTAN topic \href{http://ctan.org/topic/subdocs}{\textsf{subdocs}}
for further related packages.
The present package differs from the above solutions in that
a document structure constructed with the conventional |\include| mechanism
just needs two extra commands at the top of every file
such that all constituent files can be compiled individually.

%%%%%%%%%%%%%%%%%%%%%%%%%%%%%%%%%%%%%%%%%%%%%%%%%%%%%%%%%%%%%%%%%%%%%%%%%%%%%%%%
%\subsection{Feature Suggestions}
%
%The following is a list of features which may be useful for future
%versions of this package:
%%
%\begin{itemize}
%\item
%\ldots
%\end{itemize}

%%%%%%%%%%%%%%%%%%%%%%%%%%%%%%%%%%%%%%%%%%%%%%%%%%%%%%%%%%%%%%%%%%%%%%%%%%%%%%%%
\subsection{Revision History}

%%%%%%%%%%%%%%%%%%%%%%%%%%%%%%%%%%%%%%%%
\paragraph{v2.0:} 2018/12/30

\begin{itemize}
\item
immediate forward processing
\item
added |\childdocby| mechanism
\item
manual restructured
\end{itemize}

%%%%%%%%%%%%%%%%%%%%%%%%%%%%%%%%%%%%%%%%
\paragraph{v1.6:} 2018/01/17

\begin{itemize}
\item
application for development of include files
\item
corrections to manual
\end{itemize}

%%%%%%%%%%%%%%%%%%%%%%%%%%%%%%%%%%%%%%%%
\paragraph{v1.5:} 2017/05/21

\begin{itemize}
\item
more complete structuring introduced
\item
|\childdocof| introduced
\item
|\childdoc| renamed to |\childdocmain|
\item
|\childredirect| renamed to |\childdocforward| and |\childdocforwardprefix|
and functionality expanded
\end{itemize}

%%%%%%%%%%%%%%%%%%%%%%%%%%%%%%%%%%%%%%%%
\paragraph{v1.0:} 2017/04/27

\begin{itemize}
\item
manual and install package
\item
first version published on CTAN
\end{itemize}

%%%%%%%%%%%%%%%%%%%%%%%%%%%%%%%%%%%%%%%%
\paragraph{v0.6:} 2017/04/26

\begin{itemize}
\item
redirection mechanism added
\end{itemize}

%%%%%%%%%%%%%%%%%%%%%%%%%%%%%%%%%%%%%%%%
\paragraph{v0.5:} 2017/04/26

\begin{itemize}
\item
functionality in definition file
\end{itemize}


%%%%%%%%%%%%%%%%%%%%%%%%%%%%%%%%%%%%%%%%%%%%%%%%%%%%%%%%%%%%%%%%%%%%%%%%%%%%%%%%
%%%%%%%%%%%%%%%%%%%%%%%%%%%%%%%%%%%%%%%%%%%%%%%%%%%%%%%%%%%%%%%%%%%%%%%%%%%%%%%%
%%%%%%%%%%%%%%%%%%%%%%%%%%%%%%%%%%%%%%%%%%%%%%%%%%%%%%%%%%%%%%%%%%%%%%%%%%%%%%%%
\appendix

\settowidth\MacroIndent{\rmfamily\scriptsize 000\ }

 \DocInput{childdoc.dtx}

\end{document}
%</driver>
% \fi
%
% %%%%%%%%%%%%%%%%%%%%%%%%%%%%%%%%%%%%%%%%%%%%%%%%%%%%%%%%%%%%%%%%%%%%%%%%%%%%%%
% %%%%%%%%%%%%%%%%%%%%%%%%%%%%%%%%%%%%%%%%%%%%%%%%%%%%%%%%%%%%%%%%%%%%%%%%%%%%%%
% \section{Sample}
%\iffalse
%<*samplemain>
%\fi
%
% The following presents a sample document
% with two chapters, two parts, a title page,
% a compile flag as well as three forwarding files to set the flag.
% It consists of eight |.tex| files:
% \begin{center}
% \begin{tabular}{ll}
% |cdocsamp.tex|&main file\\
% |cdocsch1.tex|&include file for chapter 1\\
% |cdocsch2.tex|&include file for chapter 2\\
% |cdocspt3.tex|&include file for part 3\\
% |cdocspt4.tex|&include file for part 4\\
% |cdocsdrf.tex|&forwarding file for main file in draft mode\\
% |cdocsfi1.tex|&forwarding file for final version of chapter 1\\
% |cdocsfi2.tex|&forwarding file for final version of chapter 2\\
% \end{tabular}
% \end{center}
% Each of the eight files can be compiled directly by the \LaTeX{} compiler.
%
% %%%%%%%%%%%%%%%%%%%%%%%%%%%%%%%%%%%%%%
% \paragraph{Main File.}
%
% The main file is called |cdocsamp.tex|.
%
% Load the \textsf{childdoc} definitions and
% declare the filename for the main document:
%    \begin{macrocode}
\input{childdoc.def}
\childdocmain{}
%    \end{macrocode}

% Optional override for |\version| flag:
%    \begin{macrocode}
%%\ifchilddoc\else\providecommand{\version}{draft}\fi
%    \end{macrocode}

% Define the default values for the |\version| flag
% (|final| for the main file and |draft| for childs):
%    \begin{macrocode}
\ifchilddoc
\providecommand{\version}{draft}
\else
\providecommand{\version}{final}
\fi
%    \end{macrocode}

% Load the standard document class:
%    \begin{macrocode}
\documentclass[12pt]{article}
%    \end{macrocode}

% Start the document body:
%    \begin{macrocode}
\begin{document}
%    \end{macrocode}

% Declare a title page.
% Print title, part of document being processed and version flag:
%    \begin{macrocode}
\addtocounter{page}{-1}
\begin{center}
{\LARGE\bfseries{}childdoc example\par}
\vspace{1cm}
\ifchilddoc
\ifchilddocmanual part\else chapter\fi:
`\childdocname' of `\childdocjob'\par
\else
main document: `\childdocjob'\par
\fi
version: \version\par
\end{center}
\newpage
%    \end{macrocode}

% Manually include selected file,
% otherwise process as usual:
%    \begin{macrocode}
\ifchilddocmanual
\section*{part `\childdocname'}
\input{\childdocname}
\else
%    \end{macrocode}

% Include the two chapters:
%    \begin{macrocode}
\include{cdocsch1}
\include{cdocsch2}
%    \end{macrocode}

% Include the two parts unless only chapters should be displayed:
%    \begin{macrocode}
\ifchilddoc\else
\section{part three}
\input{cdocspt3}
\section{part four}
\input{cdocspt4}
\fi
%    \end{macrocode}

% Process as usual until here:
%    \begin{macrocode}
\fi
%    \end{macrocode}

% End of document body:
%    \begin{macrocode}
\end{document}
%    \end{macrocode}
%\iffalse
%</samplemain>
%\fi
%
% %%%%%%%%%%%%%%%%%%%%%%%%%%%%%%%%%%%%%%
% \paragraph{Chapter Include Files.}
%
% The include files are called |cdocsch1.tex| and |cdocsch2.tex|.
%
%\iffalse
%<*samplechap1|samplechap2>
%\fi

% Optional override for |\version| flag:
%    \begin{macrocode}
%%\providecommand{\version}{final}
%    \end{macrocode}

% Include the main document:
%    \begin{macrocode}
\input{childdoc.def}
\childdocof{cdocsamp}
%    \end{macrocode}

%\iffalse
%</samplechap1|samplechap2>
%\fi
%
%\iffalse
%<*samplechap1>
%\fi
% Some text for chapter 1:
%    \begin{macrocode}
\section{one}
some text in chapter one
%    \end{macrocode}

%\iffalse
%</samplechap1>
%\fi
% Some text for chapter 2:
%\iffalse
%<*samplechap2>
%\fi
%    \begin{macrocode}
\section{two}
more text in chapter two
%    \end{macrocode}

%\iffalse
%</samplechap2>
%\fi
%
% %%%%%%%%%%%%%%%%%%%%%%%%%%%%%%%%%%%%%%
% \paragraph{Part Include Files.}
%
% The include files are called |cdocspt3.tex| and |cdocspt4.tex|.
%
%\iffalse
%<*samplepart3|samplepart4>
%\fi

% Optional override for |\version| flag:
%    \begin{macrocode}
%%\providecommand{\version}{final}
%    \end{macrocode}

% Include the main document:
%    \begin{macrocode}
\input{childdoc.def}
\childdocby{cdocsamp}
%    \end{macrocode}

%\iffalse
%</samplepart3|samplepart4>
%\fi
%
%\iffalse
%<*samplepart3>
%\fi
% Some text for part 3:
%    \begin{macrocode}
some text in part three
%    \end{macrocode}

%\iffalse
%</samplepart3>
%\fi
% Some text for part 4:
%\iffalse
%<*samplepart4>
%\fi
%    \begin{macrocode}
more text in part four
%    \end{macrocode}

%\iffalse
%</samplepart4>
%\fi
%
% %%%%%%%%%%%%%%%%%%%%%%%%%%%%%%%%%%%%%%
% \paragraph{Forwarding for a Complete Draft.}
%
% The following forwarding file |cdocsdrf.tex|
% compiles the main document in draft mode:
%\iffalse
%<*sampledraft>
%\fi
%    \begin{macrocode}
\def\version{draft}
\input{childdoc.def}
\childdocforward{cdocsamp}
%    \end{macrocode}

%\iffalse
%</sampledraft>
%\fi
%
% %%%%%%%%%%%%%%%%%%%%%%%%%%%%%%%%%%%%%%
% \paragraph{Forwarding for Final Version of the Chapters.}
%
% The following forwarding files |cdocsfn1.tex| and |cdocsfn2.tex|
% (with identical content)
% compile the final versions of the child documents
% |cdocsch1.tex| and |cdocsch2.tex|, respectively:
%\iffalse
%<*samplefinal>
%\fi
%    \begin{macrocode}
\def\version{final}
\input{childdoc.def}
\childdocforwardprefix[cdocsamp]{cdocsfn}{cdocsch}
%    \end{macrocode}

%\iffalse
%</samplefinal>
%\fi
%
% %%%%%%%%%%%%%%%%%%%%%%%%%%%%%%%%%%%%%%
% \paragraph{Command Line Processing.}
%
% The following three command lines generate the output files
% |cdocscld|, |cdocscl1| and |cdocscl2|
% which should be identical to
% |cdocsdrf|, |cdocsch1| and |cdocsfn2|, respectively:
% \begin{center}
% \begin{tabular}{l}
% |latex -jobname cdocscld \|\\
% |  "\def\version{draft}\input{childdoc.def}\childdocforward{cdocsamp}"|\\
% |latex -jobname cdocscl1 \|\\
% |  "\input{childdoc.def}\childdocforward[cdocsamp]{cdocsch1}"|\\
% |latex -jobname cdocscl2 \|\\
% |  "\def\version{final}\input{childdoc.def}\childdocforward{cdocsch2}"|
% \end{tabular}
% \end{center}
% Note that the trailing backslash on each first line
% merely continues the input to the second line
% (for convenient cut ant paste).
% Furthermore, the command |latex| can be replaced by any
% of its alternative versions such as |pdflatex|.
%
% %%%%%%%%%%%%%%%%%%%%%%%%%%%%%%%%%%%%%%%%%%%%%%%%%%%%%%%%%%%%%%%%%%%%%%%%%%%%%%
% %%%%%%%%%%%%%%%%%%%%%%%%%%%%%%%%%%%%%%%%%%%%%%%%%%%%%%%%%%%%%%%%%%%%%%%%%%%%%%
% \section{Implementation}
%\iffalse
%<*package>
%\fi
%
% This section describes the definitions file |childdoc.def|.

% The definitions cannot be loaded using |\usepackage| or |\RequirePackage|
% which has a mechanism to prevent loading a style file more than once.
% When loading the definitions by means of |\input|
% multiple instances have to be prevented manually:
%\iffalse
%This code needs to be before the `\ProvidesFile' directive
%which is defined at the beginning of this file.
%Therefore it is also placed there and commented out here.
%</package>
%<*discard>
%\fi
%    \begin{macrocode}
\ifdefined\childdocmain\endinput\fi
%    \end{macrocode}
%\iffalse
%</discard>
%<*package>
%\fi
%
% \macro{\ifchilddoc}
% \macro{\ifchilddocmanual}
% The conditional |\ifchilddoc| tells whether a
% child (true) or main (false) document is being compiled.
% The conditional |\ifchilddocmanual| tells whether
% the |\includeonly| mechanism is used (false) or
% the selection of child files must be performed manually (true).
% The definitions initialise to false:
%    \begin{macrocode}
\newif\ifchilddoc
\newif\ifchilddocmanual
%    \end{macrocode}

% \macro{\childdocname}
% \macro{\childdocjob}
% The macro |\childdocname| stores the name of the main document
% to be compiled. The macro |\childdocjob| stores the name of
% the document on which the \LaTeX{} compiler was originally invoked.
% The content of |\jobname| cannot be compared
% to filenames specified in the source due to different catcodes.
% The following code rescans |\jobname|, stores the result
% in |\childdocname| and saves a copy in |\childdocjob|:
%    \begin{macrocode}
\edef\childdocname{\scantokens\expandafter{\jobname\noexpand}}
\let\childdocjob\childdocname
%    \end{macrocode}

% \macro{\childdocdisable}
% The macro |\childdocdisable| prevents the main file
% from being processed more than once.
% At this stage, the main document command |\childdocmain|
% is assumed to be called once again where it should do nothing.
% Any subsequent call to it should prevent
% a secondary processing of the main document
% It overwrites the forwarding commands
% |\childdocof| and |\childdocforward|
% with empty macros to prevent further inclusions of the main document:
%    \begin{macrocode}
\newcommand{\childdocdisable}
{
  \renewcommand{\childdocmain}[1]{\renewcommand{\childdocmain}[1]{\endinput}}
  \renewcommand{\childdocof}[1]{}
  \renewcommand{\childdocby}[2][]{}
  \renewcommand{\childdocforward}[2][]{}
  \renewcommand{\childdocdisable}{}
}
%    \end{macrocode}

% \macro{\childdocmain}
% The macro |\childdocmain| is to be called at the top of the main file
% with nothing or the main filename (without extension) as argument.
% First, it breaks loops.
% If the argument is not empty and does not match |\childdocname|
% (which is set by the first inclusion of |childdoc.def|),
% |\ifchilddoc| is set to true, |\includeonly| is applied to the child file
% and |\jobname| is set to the main file
% (for proper handling of |.aux| files):
%    \begin{macrocode}
\newcommand{\childdocmain}[1]
{
  \childdocdisable\childdocmain{}
  \if?#1?\else
    \begingroup
      \def\childdoctmp{#1}
      \ifx\childdoctmp\childdocname
        \def\childdoctmp{}
      \else
        \def\childdoctmp
        {
          \childdoctrue
          \includeonly{\childdocname}
          \def\childdocjob{#1}
          \def\jobname{#1}
        }
      \fi
      \expandafter
    \endgroup
    \childdoctmp
  \fi
}
%    \end{macrocode}

% \macro{\childdocof}
% The command |\childdocof| redirects
% compilation to the main file |#1|.
%    \begin{macrocode}
\newcommand{\childdocof}[1]
{
  \childdocdisable
  \childdoctrue
  \includeonly{\childdocname}
  \def\jobname{#1}
  \def\childdocjob{#1}
  \input{#1}
}
%    \end{macrocode}

% \macro{\childdocby}
% The command |\childdocby| ....
%    \begin{macrocode}
\newcommand{\childdocby}[2][]
{
  \childdocdisable
  \childdoctrue
  \childdocmanualtrue
  \if?#1?\else
    \def\jobname{#2}
  \fi
  \def\childdocjob{#2}
  \input{#2}
  \endinput
}
%    \end{macrocode}

% \macro{\childdocforward}
% The command |\childdocforward| redirects
% compilation to the main file or
% (if the optional argument is given) a child file.
% Parameters are set as if the main file
% or a child file starting with |\childdocof| was compiled.
% Then compilation is handed over to the main file:
%    \begin{macrocode}
\newcommand{\childdocforward}[2][]
{
  \begingroup
    \if?#1?
      \def\childdoctmp
      {
        \def\childdocname{#2}
        \def\childdocjob{#2}
        \def\jobname{#2}
        \input{#2}
        \endinput
      }
    \else
      \def\childdoctmp
      {
        \childdocdisable
        \def\childdocname{#2}
        \childdoctrue
        \includeonly{#2}
        \def\childdocjob{#1}
        \def\jobname{#1}
        \input{#1}
        \endinput
      }
    \fi
    \expandafter
  \endgroup
  \childdoctmp
}
%    \end{macrocode}

% \macro{\childdocforwardprefix}
% The command |\childdocforwardprefix| redirects
% compilation to the main or a child file by means of a pattern.
% The prefix |#1| in the current filename is replaced by |#2|
% and the suffix of the current filename is kept
% (it is assumed that the filename does not contain the substring `|~~~|'
% which is used as a delimiter).
% Compilation is handed over to the new file by |\childdocforward|:
%    \begin{macrocode}
\newcommand{\childdocforwardprefix}[3][]
{
  \begingroup
    \def\childdocextract #2##1~~~{\def\childdoctmp{\childdocforward[#1]{#3##1}}}
    \expandafter\childdocextract\childdocname~~~
    \expandafter
  \endgroup
  \childdoctmp
}
%    \end{macrocode}

% \macro{\childdoc}
% The deprecated macro |\childdoc| is a legacy version of |\childdocmain|:
%    \begin{macrocode}
\newcommand{\childdoc}{\childdocmain}
%    \end{macrocode}

% \macro{\childdocredirect}
% The deprecated macro |\childdocredirect| is a legacy version
% of |\childdocforward| and |\childdocforwardprefix|:
%    \begin{macrocode}
\newcommand{\childdocredirect}[2][]
{
  \begingroup
    \if?#1?
      \def\childdoctmp{\childdocforward{#2}}
    \else
      \def\childdoctmp{\childdocforwardprefix{#1}{#2}}
    \fi
    \expandafter
  \endgroup
  \childdoctmp
}
%    \end{macrocode}

%\iffalse
%</package>
%\fi
%
\endinput
|\\
|\childdocforward{|\textit{main}|}|
\end{tabular}
\end{center}
%
Likewise, the following files |final|\textit{nn}|.tex|
compile the final version of the child document
|child|\textit{nn}|.tex|:
%
\begin{center}
\begin{tabular}{l}
|\def\version{final}|\\
|% \iffalse
%
% childdoc.dtx Copyright (C) 2017-2018 Niklas Beisert
%
% This work may be distributed and/or modified under the
% conditions of the LaTeX Project Public License, either version 1.3
% of this license or (at your option) any later version.
% The latest version of this license is in
%   http://www.latex-project.org/lppl.txt
% and version 1.3 or later is part of all distributions of LaTeX
% version 2005/12/01 or later.
%
% This work has the LPPL maintenance status `maintained'.
%
% The Current Maintainer of this work is Niklas Beisert.
%
% This work consists of the files childdoc.dtx and childdoc.ins
% and the derived files childdoc.def and cdocsamp.tex with
% cdocsch1.tex, cdocsch2.tex, cdocsdrf.tex, cdocsfn1.tex, cdocsfn2.tex.
%
%<package>\ifdefined\childdocmain\endinput\fi
%<package>\ProvidesFile{childdoc.def}[2018/12/30 v2.0 child document driver]
%<samplemain>\ProvidesFile{cdocsamp.tex}[2018/12/30 v2.0 sample for childdoc]
%<*driver>
%\ProvidesFile{childdoc.drv}[2018/12/30 v2.0 childdoc reference manual file]
\PassOptionsToClass{10pt,a4paper}{article}
\documentclass{ltxdoc}

\usepackage[margin=35mm]{geometry}
\usepackage{hyperref}
\usepackage{hyperxmp}
\usepackage[usenames]{color}

\hypersetup{colorlinks=true}
\hypersetup{pdfstartview=FitH}
\hypersetup{pdfpagemode=UseNone}
\hypersetup{pdfsource={}}
\hypersetup{pdflang={en-UK}}
\hypersetup{pdfcopyright={Copyright 2017-2018 Niklas Beisert.
  This work may be distributed and/or modified under the
  conditions of the LaTeX Project Public License, either version 1.3
  of this license or (at your option) any later version.}}
\hypersetup{pdflicenseurl={http://www.latex-project.org/lppl.txt}}
\hypersetup{pdfcontactaddress={ETH Zurich, ITP, HIT K,
  Wolfgang-Pauli-Strasse 27}}
\hypersetup{pdfcontactpostcode={8093}}
\hypersetup{pdfcontactcity={Zurich}}
\hypersetup{pdfcontactcountry={Switzerland}}
\hypersetup{pdfcontactemail={nbeisert@itp.phys.ethz.ch}}
\hypersetup{pdfcontacturl={http://people.phys.ethz.ch/\xmptilde nbeisert/}}

\newcommand{\secref}[1]{\hyperref[#1]{section \ref*{#1}}}

\parskip1ex
\parindent0pt
\let\olditemize\itemize
\def\itemize{\olditemize\parskip0pt}

\begin{document}

\title{The \textsf{childdoc} Package}
\hypersetup{pdftitle={The childdoc Package}}
\author{Niklas Beisert\\[2ex]
  Institut f\"ur Theoretische Physik\\
  Eidgen\"ossische Technische Hochschule Z\"urich\\
  Wolfgang-Pauli-Strasse 27, 8093 Z\"urich, Switzerland\\[1ex]
  \href{mailto:nbeisert@itp.phys.ethz.ch}
  {\texttt{nbeisert@itp.phys.ethz.ch}}}
\hypersetup{pdfauthor={Niklas Beisert}}
\hypersetup{pdfsubject={Manual for the LaTeX2e Package childdoc}}
\date{30 December 2018, \textsf{v2.0}}
\maketitle

\begin{abstract}\noindent
\textsf{childdoc} is a \LaTeXe{} package
that enables the direct compilation
of document sections included by |\include|
to individual files.
\end{abstract}

\begingroup
\parskip0ex
\tableofcontents
\endgroup

%%%%%%%%%%%%%%%%%%%%%%%%%%%%%%%%%%%%%%%%%%%%%%%%%%%%%%%%%%%%%%%%%%%%%%%%%%%%%%%%
%%%%%%%%%%%%%%%%%%%%%%%%%%%%%%%%%%%%%%%%%%%%%%%%%%%%%%%%%%%%%%%%%%%%%%%%%%%%%%%%
\section{Introduction}

\LaTeX{} provides a mechanism to structure a large document (such as a book)
into a main file and several child files (containing the chapters)
using the |\include| command.
This mechanism is beneficial for documents
which span hundreds of pages in order to
make the source file(s) more manageable.
Moreover, compilation can be restricted to
selected child files by means of the |\includeonly| command.
The latter feature can be used to reduce the compilation time while editing
(this was significantly more useful in the earlier days of \LaTeX{})
or to generate a smaller document which is easier to navigate.
Another application of |\includeonly| is to generate
documents consisting of selected parts of the complete document.

However, there are a few drawbacks of the plain |\include| mechanism:
\begin{itemize}
\item
The child files cannot be compiled on their own,
they can only be compiled via the main file.
A naive editing environment
(such as a text editor with an option
to have the current file processed by \LaTeX)
may require one to switch to the main file before compiling;
attempting to compile the child file produces errors.
\item
The main file must be modified (each time)
to adjust the |\includeonly| command
to the present needs. This easily leaves the main file in a messy state.
\item
The generated document will always carry the filename
of the main document. This is inconvenient if
several child files are to be compiled and
to be kept for distribution.
\end{itemize}

The present package provides a simple interface
to make child files individually compilable by \LaTeX{}.
Compiling a child file then has the same effect as compiling
the main file with an |\includeonly| command
to select the appropriate child.
Moreover the generated document will carry the name of the child
rather than the main file.
This resolves all three above issues.

This feature is meant to make the editing of books,
thesis documents and lecture notes somewhat more convenient.
However, the package can also be used efficiently for
composing a series of documents (such as exercise sheets)
which are typically distributed individually.
It then assists the author in generating the individual documents
(potentially in different versions)
as well as a document containing the collected series.
Another application is in developing style files
or other kinds of included material
where compilation of the style file could redirect
to a sample or test file.

%%%%%%%%%%%%%%%%%%%%%%%%%%%%%%%%%%%%%%%%%%%%%%%%%%%%%%%%%%%%%%%%%%%%%%%%%%%%%%%%
%%%%%%%%%%%%%%%%%%%%%%%%%%%%%%%%%%%%%%%%%%%%%%%%%%%%%%%%%%%%%%%%%%%%%%%%%%%%%%%%
\section{Usage}

First of all, the package \textsf{childdoc} is \emph{not} a standard
\LaTeXe{} |.sty| style file! Therefore it needs to be invoked in
a non-standard way.

%%%%%%%%%%%%%%%%%%%%%%%%%%%%%%%%%%%%%%%%%%%%%%%%%%%%%%%%%%%%%%%%%%%%%%%%%%%%%%%%
\subsection{Included Files}
\label{sec:include}

%%%%%%%%%%%%%%%%%%%%%%%%%%%%%%%%%%%%%%%%
\DescribeMacro{\childdocmain}
To use the package, add the commands
\begin{center}
\begin{tabular}{l}
|\input{childdoc.def}|\\
|\childdocmain{}|\\
\end{tabular}
\end{center}
at the very top of the main \LaTeX{} file,
in particular \emph{before} the |\documentclass| statement!
The argument of |\childdocmain| should be left empty
(but it must be present).

%%%%%%%%%%%%%%%%%%%%%%%%%%%%%%%%%%%%%%%%
\DescribeMacro{\childdocof}
Furthermore, add the commands
\begin{center}
\begin{tabular}{l}
|\input{childdoc.def}|\\
|\childdocof{|\textit{main}|}|\\
\end{tabular}
\end{center}
at the top of every child file \textit{child}
which is included by |\include{|\textit{child}|}|
from within the main file
(or at least for those files to be compiled individually).
The argument \textit{main} must be the filename of the main file.

There are a couple of
considerations in setting up the main and child documents:

%%%%%%%%%%%%%%%%%%%%%%%%%%%%%%%%%%%%%%%%
\paragraph{Restrictions.}

Please note the following restrictions:
\begin{itemize}
\item
|\childdocmain| must be called with one argument \textit{main}
to ensure compatibility with earlier version of the package.
It must either be empty (|\childdocmain{}|)
or precisely match the filename of the main file in which it is specified.
See \secref{sec:detection} for further information.
\item
The filename \textit{main} must be specified without the |.tex| extension.
\item
The filename \textit{main} is case sensitive
(even in case-insensitive file systems)
due to internal string comparison.
\item
The argument \textit{main} should be fully expanded, it cannot be a macro.
\item
Subdirectories and special characters should be avoided in filenames.
\item
The command |\childdocmain{|\textit{main}|}| must be followed by a whitespace.
It should not be followed immediately by another command
or by a comment mark `|%|'.
This is because the \TeX{} parser reads the token immediately following
the argument of |\childdocmain| and puts it
at the beginning of every child section;
however, a white\-space is ignored.
\end{itemize}

%%%%%%%%%%%%%%%%%%%%%%%%%%%%%%%%%%%%%%%%
\paragraph{Content of Main File.}

It is advisable to place all content in the child files included by |\include|.
Any output contained in the main file will appear in all child documents
unless suppressed manually;
it cannot be suppressed automatically by the |\includeonly| directive
and thus should normally be avoided.
A method to include some content in the main file
by means of conditional processing is described in \secref{sec:conditional}.

%%%%%%%%%%%%%%%%%%%%%%%%%%%%%%%%%%%%%%%%
\paragraph{Page Numbering.}

When only a part of the document is compiled,
the appropriate numbering of pages
(as well as other status parameters)
is determined from the |.aux| files.
The latter contain information from previous passes.
However this information needs to propagate through
all intermediate child documents.
Therefore the page numbering in child documents may well
be inconsistent until the complete document is compiled at least once.

A useful (if unconventional) way to always ensure a consistent
page numbering is to restart the numbering in each child document
and denote the pages by `\textit{child}|.|\textit{page}'
where \textit{child} represents the chapter/section number of the child file.
This can be achieved by the command
|\numberwithin{page}{|\textit{child}|}|
of the \textsf{amsmath} package
where \textit{child} can be |chapter| or |section|
depending on the chosen structuring.
Alternatively, one can modify the macro |\thepage| appropriately
and reset the counter |page| at the start of each child file.

%%%%%%%%%%%%%%%%%%%%%%%%%%%%%%%%%%%%%%%%%%%%%%%%%%%%%%%%%%%%%%%%%%%%%%%%%%%%%%%%
\subsection{Conditional Processing}
\label{sec:conditional}

The package provides a mechanism to compile different versions
of a document. To customise the versions further some conditional processing
can come in handy to distinguish which version is being compiled.
The package provides two macros to describe the compilation context:

%%%%%%%%%%%%%%%%%%%%%%%%%%%%%%%%%%%%%%%%
\DescribeMacro{\ifchilddoc}
The conditional |\ifchilddoc| distinguishes between the compilation of
child documents and the main document:
%
\begin{center}
|\ifchilddoc |\textit{child-code}| |[|\||else |\textit{main-code}]| \||fi|
\end{center}

%%%%%%%%%%%%%%%%%%%%%%%%%%%%%%%%%%%%%%%%
\DescribeMacro{\childdocname}
\DescribeMacro{\childdocjob}
The macro |\childdocname| contains the filename (without extension)
of the main or child file being processed.
Note that |\childdocjob| will always contain the name of the main file.

%%%%%%%%%%%%%%%%%%%%%%%%%%%%%%%%%%%%%%%%
\paragraph{Title Page.}

Conditional processing can be used to include a title or banner page
in the main document when proper precautions are taken.
Importantly, the code in the main file should ensure that the page counter
(as well as other status parameters which are stored in the |.aux| files)
takes the same value after the conditional processing.
Otherwise the page numbers may take divergent values
depending on which part is compiled.

For example, a title page could be declared by:
%
\begin{center}
\begin{tabular}{l}
|\ifchilddoc\||else|\\
|\addtocounter{page}{-1}|\\
\textit{code for title page}\\
|\newpage|\\
|\||fi|
\end{tabular}
\end{center}
%
A banner page for the child documents can be generated by:
%
\begin{center}
\begin{tabular}{l}
|\ifchilddoc|\\
|\addtocounter{page}{-1}|\\
\textit{code for banner page}\\
|\newpage|\\
|\||fi|
\end{tabular}
\end{center}
%
Here one could write a message such as:
\begin{center}
|This is the part \childdocname{} of \childdocjob{}.|
\end{center}

%%%%%%%%%%%%%%%%%%%%%%%%%%%%%%%%%%%%%%%%%%%%%%%%%%%%%%%%%%%%%%%%%%%%%%%%%%%%%%%%
\subsection{Flags}
\label{sec:flags}

The package makes it easy to generate different versions
of the main or child documents.
To this end compilation flags can be defined
and assigned different default values.
They will be particularly useful in conjunction
with the forwarding mechanism described in \secref{sec:forward}.

For example, it may be useful to have a flag |\version|
which can be set to |draft| or |final|.
The document source will contain some conditional code
depending on the value of |\version|.
Suppose further, the flag should default to |final| for the main file
and to |draft| for child files
which is a natural assignment for editing the document.
This is achieved by placing the following code
in the preamble of the main document
(below the |\childdocmain| directive):
%
\begin{center}
\begin{tabular}{l}
|\ifchilddoc|\\
|\providecommand{\version}{draft}|\\
|\||else|\\
|\providecommand{\version}{final}|\\
|\||fi|
\end{tabular}
\end{center}
%
The definition by |\providecommand| makes sure
that previous definitions are not overwritten.
Further statements |\providecommand{\version}{...}|
can thus be added before the above code to override it.

For the main file, one might add a line
(between |\childdocmain| and the above block)
%
\begin{center}
|%\ifchilddoc\||else\providecommand{\version}{draft}\||fi|
\end{center}
%
which can be uncommented to produce a draft version.
Likewise one can add a line to the very top of a child file
(above the |\childdocof{|\textit{main}|}| directive)
%
\begin{center}
|%\providecommand{\version}{final}|
\end{center}
%
which can be uncommented to produce the final version of this child document.

%%%%%%%%%%%%%%%%%%%%%%%%%%%%%%%%%%%%%%%%%%%%%%%%%%%%%%%%%%%%%%%%%%%%%%%%%%%%%%%%
\subsection{Forwarding}
\label{sec:forward}

Different versions of the main or child documents
using compilation flags as described in \secref{sec:flags}
can be (permanently) stored in different files
for convenient compilation, viewing and distribution.
To this end, the package defines a command
to pass on compilation to a different file:

%%%%%%%%%%%%%%%%%%%%%%%%%%%%%%%%%%%%%%%%
\DescribeMacro{\childdocforward}
The command |\childdocforward| redirects processing to
another source file:
%
\begin{center}
\begin{tabular}{l}
|\input{childdoc.def}|\\
|\childdocforward[|\textit{main}|]{|\textit{dest}|}|\\
\end{tabular}
\end{center}
%
The argument \textit{dest} is the destination file
(without extension).
It should be the main file or one of the child files.
Note that further \textsf{childdoc} directives
such as |\childdocof| and |\childdocforward|
in the indicated file will be processed in this form.
The optional argument \textit{main}
passes on directly to the main file \textit{main}
while pretending to compile the child \textit{dest}.
This form behaves as if \textit{dest}
issues |\childdocof{|\textit{main}|}| right away,
and no further \textsf{childdoc} directives will be processed.

%%%%%%%%%%%%%%%%%%%%%%%%%%%%%%%%%%%%%%%%
\DescribeMacro{\...prefix}
In the alternative form |\childdocforwardprefix|,
%
\begin{center}
\begin{tabular}{l}
|\input{childdoc.def}|\\
|\childdocforwardprefix[|\textit{main}|]{|\textit{prefix}|}{|\textit{dest}|}|
\end{tabular}
\end{center}
%
the destination file is determined by a pattern
depending on the current file:
To make this work, the current file must be called
`{\textit{prefix}\hspace{0.2em}\textit{suffix}}'
with \textit{prefix} matching precisely the argument.
Processing is then passed on to the file
`{\textit{dest}\hspace{0.2em}\textit{suffix}}'.
Surely, the same effect is achieved by
directly specifying the
argument `{\textit{dest}\hspace{0.2em}\textit{suffix}}'
in the first form.
However, that requires to set up a different file
for each child. With the alternative form of the command
all these files can have exactly the same content
which simplifies setting them up and maintaining them.

For example, the following file |draft.tex|
with a compilation flag |\version| as described in \secref{sec:flags}
compiles the main document as a draft:
%
\begin{center}
\begin{tabular}{l}
|\def\version{draft}|\\
|\input{childdoc.def}|\\
|\childdocforward{|\textit{main}|}|
\end{tabular}
\end{center}
%
Likewise, the following files |final|\textit{nn}|.tex|
compile the final version of the child document
|child|\textit{nn}|.tex|:
%
\begin{center}
\begin{tabular}{l}
|\def\version{final}|\\
|\input{childdoc.def}|\\
|\childdocforwardprefix{final}{child}|
\end{tabular}
\end{center}
%

Note that when several versions of a main file and/or of each child file
are to be generated, it may be convenient to set up a |Makefile| or
shell script to automatise the process.

%%%%%%%%%%%%%%%%%%%%%%%%%%%%%%%%%%%%%%%%%%%%%%%%%%%%%%%%%%%%%%%%%%%%%%%%%%%%%%%%
\subsection{Command Line Processing}
\label{sec:commandline}

The effect of redirection files can also be achieved by invoking
the \LaTeX{} compiler with a more elaborate command line.
Most conveniently this should be done as part
of a shell script or a |Makefile|.

When using \textsf{childdoc} in the main file, the following
command lines effectively perform a redirection
(note that depending on the shell being used,
backslashes may have to be doubled: `|\|' $\to$ `|\\|'):
%
\begin{center}
|... -jobname "|\textit{target}|" |\\|"|[\textit{flags}]%
|\input{childdoc.def}\childdocforward[|\textit{main}|]{|\textit{dest}|}"|
\end{center}
%
Here \textit{target} is the name of the output file,
\textit{main} is the name of the main file
and \textit{dest} is the name of the main or child file to be processed
(all filenames without extensions).
The optional argument \textit{main} can be omitted
if \textit{main} matches \textit{dest}.
Optionally, compilation \textit{flags} can be defined via |\def| commands.
This command line makes the \TeX{} engine believe
it is compiling the file \textit{target}
whose content is specified as the latter parameter.
The provided code then forwards the processing to
\textit{main} or \textit{dest} as described in \secref{sec:forward}.

%%%%%%%%%%%%%%%%%%%%%%%%%%%%%%%%%%%%%%%%%%%%%%%%%%%%%%%%%%%%%%%%%%%%%%%%%%%%%%%%
\subsection{Include by Input}
\label{sec:input}

Including child documents by |\include| has some restrictions by design.
Most notably, the content of a child document always occupies
its own set of pages; pages cannot be shared between child documents.
Usually, this behaviour makes perfect sense
because each child document contain an essential part of the document.
However, in some situations it may be desirable to compose
a document from a collection of parts
without having mandatory page breaks between then.
For this case, the package
provides a mechanism to include parts
by |\input| which can also be processed individually.
However, by construction this mechanism
requires manual handling of the content to be output.

%%%%%%%%%%%%%%%%%%%%%%%%%%%%%%%%%%%%%%%%
\DescribeMacro{\ifchilddocmanual}
The main file should be prepared as usual, see \secref{sec:include}.
However, the document body must make a distinction
between processing of an individual part and of the main document, e.g.:
%
\begin{center}
\begin{tabular}{l}
|\ifchilddocmanual|\\
|\input{\childdocname}|\\
|\||else|\\
\textit{document body with }|\input{|\textit{part}|}|\\
|\||fi|
\end{tabular}
\end{center}
%
The conditional |\ifchilddocmanual| is true whenever
a part to be included by |\input| is being compiled,
and the name of the part is stored in |\childdocname|.

%%%%%%%%%%%%%%%%%%%%%%%%%%%%%%%%%%%%%%%%
\DescribeMacro{\childdocby}
Each part to be included by |\input| should start with:
%
\begin{center}
\begin{tabular}{l}
|\input{childdoc.def}|\\
|\childdocby{|\textit{main}|}|\\
\end{tabular}
\end{center}
%
The directive |\childdocby| is similar to |\childdocof|
described in \secref{sec:include},
but the subsequent selection of content must be done manually.
To that end, both |\ifchilddoc| and |\ifchilddocmanual|
will be true upon processing of a part,
and the name of the part is stored in |\childdocname|.
Note that |\jobname| will be set to the filename of the current part
so that each part receives an individual |.aux| file
that does not interfere with the |.aux| file(s) of the main document.
This behaviour can be altered by the alternative form
|\childdocby[*]{|\textit{main}|}| (with a non-empty optional argument)
which uses the |.aux| file of the main document
by setting |\jobname| to \textit{main}.

%%%%%%%%%%%%%%%%%%%%%%%%%%%%%%%%%%%%%%%%%%%%%%%%%%%%%%%%%%%%%%%%%%%%%%%%%%%%%%%%
\subsection{Driver Development}
\label{sec:driver}

The \textsf{childdoc} mechanism can also be use for the development
of definition files such as \LaTeX{} styles or classes.
This case differs from the above setup with multiple parts
included by |\include| in that no |\includeonly| should be invoked.
This can be achieved by starting the include file
(before |\ProvidesPackage|) with:
%
\begin{center}
\begin{tabular}{l}
|\input{childdoc.def}|\\
|\childdocforward{|\textit{main}|}|\\
\end{tabular}
\end{center}
%
or alternatively with:
%
\begin{center}
\begin{tabular}{l}
|\input{childdoc.def}|\\
|\childdocby{|\textit{main}|}|\\
\end{tabular}
\end{center}
%
Both forms have slightly different effects as described above.
The main file is prepared as usual, see \secref{sec:include}.

%%%%%%%%%%%%%%%%%%%%%%%%%%%%%%%%%%%%%%%%%%%%%%%%%%%%%%%%%%%%%%%%%%%%%%%%%%%%%%%%
\subsection{Legacy Detection}
\label{sec:detection}

The directive |\childdocmain| in the main file can detect
whether the complete document or merely a child is to be compiled
even without using the directive |\childdocof|.
This method is deprecated because it is less robust
and there is no compelling reason to use it;
it is merely provided for backward compatibility
and it may be removed in future versions.

If the detection mechanism is to be used,
it is mandatory to correctly specify
the filename of the main file as the argument of |\childdocmain|:
%
\begin{center}
\begin{tabular}{l}
|\input{childdoc.def}|\\
|\childdocmain{|\textit{main}|}|\\
\end{tabular}
\end{center}
%
If |\jobname| does not match the argument \textit{main} of |\childdocmain|,
it is assumed that |\jobname| points to the child file to be compiled.
When using |\childdocmain| with the main file specified as argument,
it suffices to start a child file
with just |\input{|\textit{main}|}|
without loading of the package and using |\childdocof|.
If instead all processing is done
with the appropriate \textsf{childdoc} directives,
the argument of \textit{main} of |\childdocmain| can be empty.

An alternative version of the command line processing described
in \secref{sec:commandline} using the detection mechanism reads:
%
\begin{center}
|... -jobname "|\textit{target}|" "|[\textit{flags}]%
[|\def\jobname{|\textit{dest}|}|]|\input{|\textit{main}|}"|
\end{center}

%%%%%%%%%%%%%%%%%%%%%%%%%%%%%%%%%%%%%%%%%%%%%%%%%%%%%%%%%%%%%%%%%%%%%%%%%%%%%%%%
\subsection{Manual Code}
\label{sec:manual}

In case one cannot be certain whether the definitions file |childdoc.def|
is installed on the target \TeX{} distribution
and one prefers not to ship it,
it is conceivable to paste a few relevant commands into the sources.

To that end, drop all statements |\input{childdoc.def}|
and perform the replacements as outlined below.
Instead of |\childdocmain{|\textit{main}|}| add the following code
to the top of the main file:
%
\begin{center}
\begin{tabular}{l}
|\||ifdefined\childdocname\endinput\||fi\newif\ifchilddoc|\\
|\edef\childdocname{\scantokens\expandafter{\jobname\noexpand}}|\\
|\def\childdocmain{|\textit{main}|}\||ifx\childdocmain\childdocname\||else|\\
|\childdoctrue\includeonly{\childdocname}\let\jobname\childdocmain\||fi|\\
\end{tabular}
\end{center}
%
Instead of |\childdocof{|\textit{main}|}| just include the main file
at the top of each child file:
%
\begin{center}
|\input{|\textit{main}|}|
\end{center}
%
A simple redirection |\childdocforward{|\textit{dest}|}| is achieved by:
%
\begin{center}
|\def\jobname{|\textit{dest}|}\input{\jobname}|
\end{center}
%
The redirection with prefix
|\childdocforwardprefix[|\textit{prefix}|]{|\textit{dest}|}|
is accomplished by:
%
\begin{center}
\begin{tabular}{l}
|{\edef\jobname{\scantokens\expandafter{\jobname\noexpand}}|\\
|\def\redirectjob |\textit{prefix}|#1~~~{\gdef\jobname{|\textit{dest}|#1}}|\\
|\expandafter\redirectjob\jobname~~~}\input{\jobname}|
\end{tabular}
\end{center}

In an alternative approach,
child documents can be compiled by a specific command line
without additional code or specific definitions:
%
\begin{center}
|... -jobname "|\textit{target}|" "|[\textit{flags}]%
|\includeonly{|\textit{dest}|}\input{|\textit{main}|}"|
\end{center}
%

%%%%%%%%%%%%%%%%%%%%%%%%%%%%%%%%%%%%%%%%%%%%%%%%%%%%%%%%%%%%%%%%%%%%%%%%%%%%%%%%
%%%%%%%%%%%%%%%%%%%%%%%%%%%%%%%%%%%%%%%%%%%%%%%%%%%%%%%%%%%%%%%%%%%%%%%%%%%%%%%%
\section{Information}

%%%%%%%%%%%%%%%%%%%%%%%%%%%%%%%%%%%%%%%%%%%%%%%%%%%%%%%%%%%%%%%%%%%%%%%%%%%%%%%%
\subsection{Copyright}

Copyright \copyright{} 2017--2018 Niklas Beisert

This work may be distributed and/or modified under the
conditions of the \LaTeX{} Project Public License, either version 1.3
of this license or (at your option) any later version.
The latest version of this license is in
  \url{http://www.latex-project.org/lppl.txt}
and version 1.3 or later is part of all distributions of \LaTeX{}
version 2005/12/01 or later.

This work has the LPPL maintenance status `maintained'.

The Current Maintainer of this work is Niklas Beisert.

This work consists of the files |README.txt|, |childdoc.ins| and |childdoc.dtx|
as well as the derived files |childdoc.def|, |cdocsamp.tex|
with |cdocsch1.tex|, |cdocsch2.tex|, |cdocspt3.tex|, |cdocspt4.tex|,
|cdocsdrf.tex|, |cdocsfn1.tex|, |cdocsfn2.tex|
as well as |childdoc.pdf|.

%%%%%%%%%%%%%%%%%%%%%%%%%%%%%%%%%%%%%%%%%%%%%%%%%%%%%%%%%%%%%%%%%%%%%%%%%%%%%%%%
\subsection{Files and Installation}

The package consists of the files:
%
\begin{center}
\begin{tabular}{ll}
    |README.txt|   & readme file \\
    |childdoc.ins| & installation file \\
    |childdoc.dtx| & source file \\
    |childdoc.def| & definition file \\
    |cdocsamp.tex| & sample main file \\
    |cdocsch1.tex| & sample include file \\
    |cdocsch2.tex| & sample include file \\
    |cdocspt3.tex| & sample part file \\
    |cdocspt4.tex| & sample part file \\
    |cdocsdrf.tex| & sample redirection file \\
    |cdocsfn1.tex| & sample redirection file \\
    |cdocsfn2.tex| & sample redirection file \\
    |childdoc.pdf| & manual
\end{tabular}
\end{center}
%
The distribution consists of the files
|README.txt|, |childdoc.ins| and |childdoc.dtx|.
%
\begin{itemize}
\item
Run (pdf)\LaTeX{} on |childdoc.dtx|
to compile the manual |childdoc.pdf| (this file).
\item
Run \LaTeX{} on |childdoc.ins| to create the definitions file |childdoc.def|
and the sample |cdocsamp.tex| with include files
|cdocsch1.tex|, |cdocsch2.tex|, |cdocspt3.tex|, |cdocspt4.tex|,
|cdocsdrf.tex|, |cdocsfn1.tex|, |cdocsfn2.tex|.
Then copy the file |childdoc.def| to an appropriate directory of your \LaTeX{}
distribution, e.g.\ \textit{texmf-root}|/tex/latex/childdoc|.
\end{itemize}

%%%%%%%%%%%%%%%%%%%%%%%%%%%%%%%%%%%%%%%%%%%%%%%%%%%%%%%%%%%%%%%%%%%%%%%%%%%%%%%%
\subsection{Related CTAN Packages}

There are several other packages which offer a similar functionality:
%
\begin{itemize}
\item
The packages
\href{http://ctan.org/pkg/docmute}{\textsf{docmute}},
\href{http://ctan.org/pkg/includex}{\textsf{includex}} and
\href{http://ctan.org/pkg/standalone}{\textsf{standalone}}
provide commands to include only the document body of
a child file thus allowing both files to be compiled individually.
\item
The packages \href{http://ctan.org/pkg/subdocs}{\textsf{subdocs}}
and \href{http://ctan.org/pkg/subfiles}{\textsf{subfiles}}
provide structures in which the main and child documents can be
encapsulated and allowing them to be compiled individually.
The inclusion mechanism is different from the conventional |\include|.
\item
The package \href{http://ctan.org/pkg/combine}{\textsf{combine}}
is an elaborate solution to combine several documents into one.
\end{itemize}
%
See also the CTAN topic \href{http://ctan.org/topic/subdocs}{\textsf{subdocs}}
for further related packages.
The present package differs from the above solutions in that
a document structure constructed with the conventional |\include| mechanism
just needs two extra commands at the top of every file
such that all constituent files can be compiled individually.

%%%%%%%%%%%%%%%%%%%%%%%%%%%%%%%%%%%%%%%%%%%%%%%%%%%%%%%%%%%%%%%%%%%%%%%%%%%%%%%%
%\subsection{Feature Suggestions}
%
%The following is a list of features which may be useful for future
%versions of this package:
%%
%\begin{itemize}
%\item
%\ldots
%\end{itemize}

%%%%%%%%%%%%%%%%%%%%%%%%%%%%%%%%%%%%%%%%%%%%%%%%%%%%%%%%%%%%%%%%%%%%%%%%%%%%%%%%
\subsection{Revision History}

%%%%%%%%%%%%%%%%%%%%%%%%%%%%%%%%%%%%%%%%
\paragraph{v2.0:} 2018/12/30

\begin{itemize}
\item
immediate forward processing
\item
added |\childdocby| mechanism
\item
manual restructured
\end{itemize}

%%%%%%%%%%%%%%%%%%%%%%%%%%%%%%%%%%%%%%%%
\paragraph{v1.6:} 2018/01/17

\begin{itemize}
\item
application for development of include files
\item
corrections to manual
\end{itemize}

%%%%%%%%%%%%%%%%%%%%%%%%%%%%%%%%%%%%%%%%
\paragraph{v1.5:} 2017/05/21

\begin{itemize}
\item
more complete structuring introduced
\item
|\childdocof| introduced
\item
|\childdoc| renamed to |\childdocmain|
\item
|\childredirect| renamed to |\childdocforward| and |\childdocforwardprefix|
and functionality expanded
\end{itemize}

%%%%%%%%%%%%%%%%%%%%%%%%%%%%%%%%%%%%%%%%
\paragraph{v1.0:} 2017/04/27

\begin{itemize}
\item
manual and install package
\item
first version published on CTAN
\end{itemize}

%%%%%%%%%%%%%%%%%%%%%%%%%%%%%%%%%%%%%%%%
\paragraph{v0.6:} 2017/04/26

\begin{itemize}
\item
redirection mechanism added
\end{itemize}

%%%%%%%%%%%%%%%%%%%%%%%%%%%%%%%%%%%%%%%%
\paragraph{v0.5:} 2017/04/26

\begin{itemize}
\item
functionality in definition file
\end{itemize}


%%%%%%%%%%%%%%%%%%%%%%%%%%%%%%%%%%%%%%%%%%%%%%%%%%%%%%%%%%%%%%%%%%%%%%%%%%%%%%%%
%%%%%%%%%%%%%%%%%%%%%%%%%%%%%%%%%%%%%%%%%%%%%%%%%%%%%%%%%%%%%%%%%%%%%%%%%%%%%%%%
%%%%%%%%%%%%%%%%%%%%%%%%%%%%%%%%%%%%%%%%%%%%%%%%%%%%%%%%%%%%%%%%%%%%%%%%%%%%%%%%
\appendix

\settowidth\MacroIndent{\rmfamily\scriptsize 000\ }

 \DocInput{childdoc.dtx}

\end{document}
%</driver>
% \fi
%
% %%%%%%%%%%%%%%%%%%%%%%%%%%%%%%%%%%%%%%%%%%%%%%%%%%%%%%%%%%%%%%%%%%%%%%%%%%%%%%
% %%%%%%%%%%%%%%%%%%%%%%%%%%%%%%%%%%%%%%%%%%%%%%%%%%%%%%%%%%%%%%%%%%%%%%%%%%%%%%
% \section{Sample}
%\iffalse
%<*samplemain>
%\fi
%
% The following presents a sample document
% with two chapters, two parts, a title page,
% a compile flag as well as three forwarding files to set the flag.
% It consists of eight |.tex| files:
% \begin{center}
% \begin{tabular}{ll}
% |cdocsamp.tex|&main file\\
% |cdocsch1.tex|&include file for chapter 1\\
% |cdocsch2.tex|&include file for chapter 2\\
% |cdocspt3.tex|&include file for part 3\\
% |cdocspt4.tex|&include file for part 4\\
% |cdocsdrf.tex|&forwarding file for main file in draft mode\\
% |cdocsfi1.tex|&forwarding file for final version of chapter 1\\
% |cdocsfi2.tex|&forwarding file for final version of chapter 2\\
% \end{tabular}
% \end{center}
% Each of the eight files can be compiled directly by the \LaTeX{} compiler.
%
% %%%%%%%%%%%%%%%%%%%%%%%%%%%%%%%%%%%%%%
% \paragraph{Main File.}
%
% The main file is called |cdocsamp.tex|.
%
% Load the \textsf{childdoc} definitions and
% declare the filename for the main document:
%    \begin{macrocode}
\input{childdoc.def}
\childdocmain{}
%    \end{macrocode}

% Optional override for |\version| flag:
%    \begin{macrocode}
%%\ifchilddoc\else\providecommand{\version}{draft}\fi
%    \end{macrocode}

% Define the default values for the |\version| flag
% (|final| for the main file and |draft| for childs):
%    \begin{macrocode}
\ifchilddoc
\providecommand{\version}{draft}
\else
\providecommand{\version}{final}
\fi
%    \end{macrocode}

% Load the standard document class:
%    \begin{macrocode}
\documentclass[12pt]{article}
%    \end{macrocode}

% Start the document body:
%    \begin{macrocode}
\begin{document}
%    \end{macrocode}

% Declare a title page.
% Print title, part of document being processed and version flag:
%    \begin{macrocode}
\addtocounter{page}{-1}
\begin{center}
{\LARGE\bfseries{}childdoc example\par}
\vspace{1cm}
\ifchilddoc
\ifchilddocmanual part\else chapter\fi:
`\childdocname' of `\childdocjob'\par
\else
main document: `\childdocjob'\par
\fi
version: \version\par
\end{center}
\newpage
%    \end{macrocode}

% Manually include selected file,
% otherwise process as usual:
%    \begin{macrocode}
\ifchilddocmanual
\section*{part `\childdocname'}
\input{\childdocname}
\else
%    \end{macrocode}

% Include the two chapters:
%    \begin{macrocode}
\include{cdocsch1}
\include{cdocsch2}
%    \end{macrocode}

% Include the two parts unless only chapters should be displayed:
%    \begin{macrocode}
\ifchilddoc\else
\section{part three}
\input{cdocspt3}
\section{part four}
\input{cdocspt4}
\fi
%    \end{macrocode}

% Process as usual until here:
%    \begin{macrocode}
\fi
%    \end{macrocode}

% End of document body:
%    \begin{macrocode}
\end{document}
%    \end{macrocode}
%\iffalse
%</samplemain>
%\fi
%
% %%%%%%%%%%%%%%%%%%%%%%%%%%%%%%%%%%%%%%
% \paragraph{Chapter Include Files.}
%
% The include files are called |cdocsch1.tex| and |cdocsch2.tex|.
%
%\iffalse
%<*samplechap1|samplechap2>
%\fi

% Optional override for |\version| flag:
%    \begin{macrocode}
%%\providecommand{\version}{final}
%    \end{macrocode}

% Include the main document:
%    \begin{macrocode}
\input{childdoc.def}
\childdocof{cdocsamp}
%    \end{macrocode}

%\iffalse
%</samplechap1|samplechap2>
%\fi
%
%\iffalse
%<*samplechap1>
%\fi
% Some text for chapter 1:
%    \begin{macrocode}
\section{one}
some text in chapter one
%    \end{macrocode}

%\iffalse
%</samplechap1>
%\fi
% Some text for chapter 2:
%\iffalse
%<*samplechap2>
%\fi
%    \begin{macrocode}
\section{two}
more text in chapter two
%    \end{macrocode}

%\iffalse
%</samplechap2>
%\fi
%
% %%%%%%%%%%%%%%%%%%%%%%%%%%%%%%%%%%%%%%
% \paragraph{Part Include Files.}
%
% The include files are called |cdocspt3.tex| and |cdocspt4.tex|.
%
%\iffalse
%<*samplepart3|samplepart4>
%\fi

% Optional override for |\version| flag:
%    \begin{macrocode}
%%\providecommand{\version}{final}
%    \end{macrocode}

% Include the main document:
%    \begin{macrocode}
\input{childdoc.def}
\childdocby{cdocsamp}
%    \end{macrocode}

%\iffalse
%</samplepart3|samplepart4>
%\fi
%
%\iffalse
%<*samplepart3>
%\fi
% Some text for part 3:
%    \begin{macrocode}
some text in part three
%    \end{macrocode}

%\iffalse
%</samplepart3>
%\fi
% Some text for part 4:
%\iffalse
%<*samplepart4>
%\fi
%    \begin{macrocode}
more text in part four
%    \end{macrocode}

%\iffalse
%</samplepart4>
%\fi
%
% %%%%%%%%%%%%%%%%%%%%%%%%%%%%%%%%%%%%%%
% \paragraph{Forwarding for a Complete Draft.}
%
% The following forwarding file |cdocsdrf.tex|
% compiles the main document in draft mode:
%\iffalse
%<*sampledraft>
%\fi
%    \begin{macrocode}
\def\version{draft}
\input{childdoc.def}
\childdocforward{cdocsamp}
%    \end{macrocode}

%\iffalse
%</sampledraft>
%\fi
%
% %%%%%%%%%%%%%%%%%%%%%%%%%%%%%%%%%%%%%%
% \paragraph{Forwarding for Final Version of the Chapters.}
%
% The following forwarding files |cdocsfn1.tex| and |cdocsfn2.tex|
% (with identical content)
% compile the final versions of the child documents
% |cdocsch1.tex| and |cdocsch2.tex|, respectively:
%\iffalse
%<*samplefinal>
%\fi
%    \begin{macrocode}
\def\version{final}
\input{childdoc.def}
\childdocforwardprefix[cdocsamp]{cdocsfn}{cdocsch}
%    \end{macrocode}

%\iffalse
%</samplefinal>
%\fi
%
% %%%%%%%%%%%%%%%%%%%%%%%%%%%%%%%%%%%%%%
% \paragraph{Command Line Processing.}
%
% The following three command lines generate the output files
% |cdocscld|, |cdocscl1| and |cdocscl2|
% which should be identical to
% |cdocsdrf|, |cdocsch1| and |cdocsfn2|, respectively:
% \begin{center}
% \begin{tabular}{l}
% |latex -jobname cdocscld \|\\
% |  "\def\version{draft}\input{childdoc.def}\childdocforward{cdocsamp}"|\\
% |latex -jobname cdocscl1 \|\\
% |  "\input{childdoc.def}\childdocforward[cdocsamp]{cdocsch1}"|\\
% |latex -jobname cdocscl2 \|\\
% |  "\def\version{final}\input{childdoc.def}\childdocforward{cdocsch2}"|
% \end{tabular}
% \end{center}
% Note that the trailing backslash on each first line
% merely continues the input to the second line
% (for convenient cut ant paste).
% Furthermore, the command |latex| can be replaced by any
% of its alternative versions such as |pdflatex|.
%
% %%%%%%%%%%%%%%%%%%%%%%%%%%%%%%%%%%%%%%%%%%%%%%%%%%%%%%%%%%%%%%%%%%%%%%%%%%%%%%
% %%%%%%%%%%%%%%%%%%%%%%%%%%%%%%%%%%%%%%%%%%%%%%%%%%%%%%%%%%%%%%%%%%%%%%%%%%%%%%
% \section{Implementation}
%\iffalse
%<*package>
%\fi
%
% This section describes the definitions file |childdoc.def|.

% The definitions cannot be loaded using |\usepackage| or |\RequirePackage|
% which has a mechanism to prevent loading a style file more than once.
% When loading the definitions by means of |\input|
% multiple instances have to be prevented manually:
%\iffalse
%This code needs to be before the `\ProvidesFile' directive
%which is defined at the beginning of this file.
%Therefore it is also placed there and commented out here.
%</package>
%<*discard>
%\fi
%    \begin{macrocode}
\ifdefined\childdocmain\endinput\fi
%    \end{macrocode}
%\iffalse
%</discard>
%<*package>
%\fi
%
% \macro{\ifchilddoc}
% \macro{\ifchilddocmanual}
% The conditional |\ifchilddoc| tells whether a
% child (true) or main (false) document is being compiled.
% The conditional |\ifchilddocmanual| tells whether
% the |\includeonly| mechanism is used (false) or
% the selection of child files must be performed manually (true).
% The definitions initialise to false:
%    \begin{macrocode}
\newif\ifchilddoc
\newif\ifchilddocmanual
%    \end{macrocode}

% \macro{\childdocname}
% \macro{\childdocjob}
% The macro |\childdocname| stores the name of the main document
% to be compiled. The macro |\childdocjob| stores the name of
% the document on which the \LaTeX{} compiler was originally invoked.
% The content of |\jobname| cannot be compared
% to filenames specified in the source due to different catcodes.
% The following code rescans |\jobname|, stores the result
% in |\childdocname| and saves a copy in |\childdocjob|:
%    \begin{macrocode}
\edef\childdocname{\scantokens\expandafter{\jobname\noexpand}}
\let\childdocjob\childdocname
%    \end{macrocode}

% \macro{\childdocdisable}
% The macro |\childdocdisable| prevents the main file
% from being processed more than once.
% At this stage, the main document command |\childdocmain|
% is assumed to be called once again where it should do nothing.
% Any subsequent call to it should prevent
% a secondary processing of the main document
% It overwrites the forwarding commands
% |\childdocof| and |\childdocforward|
% with empty macros to prevent further inclusions of the main document:
%    \begin{macrocode}
\newcommand{\childdocdisable}
{
  \renewcommand{\childdocmain}[1]{\renewcommand{\childdocmain}[1]{\endinput}}
  \renewcommand{\childdocof}[1]{}
  \renewcommand{\childdocby}[2][]{}
  \renewcommand{\childdocforward}[2][]{}
  \renewcommand{\childdocdisable}{}
}
%    \end{macrocode}

% \macro{\childdocmain}
% The macro |\childdocmain| is to be called at the top of the main file
% with nothing or the main filename (without extension) as argument.
% First, it breaks loops.
% If the argument is not empty and does not match |\childdocname|
% (which is set by the first inclusion of |childdoc.def|),
% |\ifchilddoc| is set to true, |\includeonly| is applied to the child file
% and |\jobname| is set to the main file
% (for proper handling of |.aux| files):
%    \begin{macrocode}
\newcommand{\childdocmain}[1]
{
  \childdocdisable\childdocmain{}
  \if?#1?\else
    \begingroup
      \def\childdoctmp{#1}
      \ifx\childdoctmp\childdocname
        \def\childdoctmp{}
      \else
        \def\childdoctmp
        {
          \childdoctrue
          \includeonly{\childdocname}
          \def\childdocjob{#1}
          \def\jobname{#1}
        }
      \fi
      \expandafter
    \endgroup
    \childdoctmp
  \fi
}
%    \end{macrocode}

% \macro{\childdocof}
% The command |\childdocof| redirects
% compilation to the main file |#1|.
%    \begin{macrocode}
\newcommand{\childdocof}[1]
{
  \childdocdisable
  \childdoctrue
  \includeonly{\childdocname}
  \def\jobname{#1}
  \def\childdocjob{#1}
  \input{#1}
}
%    \end{macrocode}

% \macro{\childdocby}
% The command |\childdocby| ....
%    \begin{macrocode}
\newcommand{\childdocby}[2][]
{
  \childdocdisable
  \childdoctrue
  \childdocmanualtrue
  \if?#1?\else
    \def\jobname{#2}
  \fi
  \def\childdocjob{#2}
  \input{#2}
  \endinput
}
%    \end{macrocode}

% \macro{\childdocforward}
% The command |\childdocforward| redirects
% compilation to the main file or
% (if the optional argument is given) a child file.
% Parameters are set as if the main file
% or a child file starting with |\childdocof| was compiled.
% Then compilation is handed over to the main file:
%    \begin{macrocode}
\newcommand{\childdocforward}[2][]
{
  \begingroup
    \if?#1?
      \def\childdoctmp
      {
        \def\childdocname{#2}
        \def\childdocjob{#2}
        \def\jobname{#2}
        \input{#2}
        \endinput
      }
    \else
      \def\childdoctmp
      {
        \childdocdisable
        \def\childdocname{#2}
        \childdoctrue
        \includeonly{#2}
        \def\childdocjob{#1}
        \def\jobname{#1}
        \input{#1}
        \endinput
      }
    \fi
    \expandafter
  \endgroup
  \childdoctmp
}
%    \end{macrocode}

% \macro{\childdocforwardprefix}
% The command |\childdocforwardprefix| redirects
% compilation to the main or a child file by means of a pattern.
% The prefix |#1| in the current filename is replaced by |#2|
% and the suffix of the current filename is kept
% (it is assumed that the filename does not contain the substring `|~~~|'
% which is used as a delimiter).
% Compilation is handed over to the new file by |\childdocforward|:
%    \begin{macrocode}
\newcommand{\childdocforwardprefix}[3][]
{
  \begingroup
    \def\childdocextract #2##1~~~{\def\childdoctmp{\childdocforward[#1]{#3##1}}}
    \expandafter\childdocextract\childdocname~~~
    \expandafter
  \endgroup
  \childdoctmp
}
%    \end{macrocode}

% \macro{\childdoc}
% The deprecated macro |\childdoc| is a legacy version of |\childdocmain|:
%    \begin{macrocode}
\newcommand{\childdoc}{\childdocmain}
%    \end{macrocode}

% \macro{\childdocredirect}
% The deprecated macro |\childdocredirect| is a legacy version
% of |\childdocforward| and |\childdocforwardprefix|:
%    \begin{macrocode}
\newcommand{\childdocredirect}[2][]
{
  \begingroup
    \if?#1?
      \def\childdoctmp{\childdocforward{#2}}
    \else
      \def\childdoctmp{\childdocforwardprefix{#1}{#2}}
    \fi
    \expandafter
  \endgroup
  \childdoctmp
}
%    \end{macrocode}

%\iffalse
%</package>
%\fi
%
\endinput
|\\
|\childdocforwardprefix{final}{child}|
\end{tabular}
\end{center}
%

Note that when several versions of a main file and/or of each child file
are to be generated, it may be convenient to set up a |Makefile| or
shell script to automatise the process.

%%%%%%%%%%%%%%%%%%%%%%%%%%%%%%%%%%%%%%%%%%%%%%%%%%%%%%%%%%%%%%%%%%%%%%%%%%%%%%%%
\subsection{Command Line Processing}
\label{sec:commandline}

The effect of redirection files can also be achieved by invoking
the \LaTeX{} compiler with a more elaborate command line.
Most conveniently this should be done as part
of a shell script or a |Makefile|.

When using \textsf{childdoc} in the main file, the following
command lines effectively perform a redirection
(note that depending on the shell being used,
backslashes may have to be doubled: `|\|' $\to$ `|\\|'):
%
\begin{center}
|... -jobname "|\textit{target}|" |\\|"|[\textit{flags}]%
|% \iffalse
%
% childdoc.dtx Copyright (C) 2017-2018 Niklas Beisert
%
% This work may be distributed and/or modified under the
% conditions of the LaTeX Project Public License, either version 1.3
% of this license or (at your option) any later version.
% The latest version of this license is in
%   http://www.latex-project.org/lppl.txt
% and version 1.3 or later is part of all distributions of LaTeX
% version 2005/12/01 or later.
%
% This work has the LPPL maintenance status `maintained'.
%
% The Current Maintainer of this work is Niklas Beisert.
%
% This work consists of the files childdoc.dtx and childdoc.ins
% and the derived files childdoc.def and cdocsamp.tex with
% cdocsch1.tex, cdocsch2.tex, cdocsdrf.tex, cdocsfn1.tex, cdocsfn2.tex.
%
%<package>\ifdefined\childdocmain\endinput\fi
%<package>\ProvidesFile{childdoc.def}[2018/12/30 v2.0 child document driver]
%<samplemain>\ProvidesFile{cdocsamp.tex}[2018/12/30 v2.0 sample for childdoc]
%<*driver>
%\ProvidesFile{childdoc.drv}[2018/12/30 v2.0 childdoc reference manual file]
\PassOptionsToClass{10pt,a4paper}{article}
\documentclass{ltxdoc}

\usepackage[margin=35mm]{geometry}
\usepackage{hyperref}
\usepackage{hyperxmp}
\usepackage[usenames]{color}

\hypersetup{colorlinks=true}
\hypersetup{pdfstartview=FitH}
\hypersetup{pdfpagemode=UseNone}
\hypersetup{pdfsource={}}
\hypersetup{pdflang={en-UK}}
\hypersetup{pdfcopyright={Copyright 2017-2018 Niklas Beisert.
  This work may be distributed and/or modified under the
  conditions of the LaTeX Project Public License, either version 1.3
  of this license or (at your option) any later version.}}
\hypersetup{pdflicenseurl={http://www.latex-project.org/lppl.txt}}
\hypersetup{pdfcontactaddress={ETH Zurich, ITP, HIT K,
  Wolfgang-Pauli-Strasse 27}}
\hypersetup{pdfcontactpostcode={8093}}
\hypersetup{pdfcontactcity={Zurich}}
\hypersetup{pdfcontactcountry={Switzerland}}
\hypersetup{pdfcontactemail={nbeisert@itp.phys.ethz.ch}}
\hypersetup{pdfcontacturl={http://people.phys.ethz.ch/\xmptilde nbeisert/}}

\newcommand{\secref}[1]{\hyperref[#1]{section \ref*{#1}}}

\parskip1ex
\parindent0pt
\let\olditemize\itemize
\def\itemize{\olditemize\parskip0pt}

\begin{document}

\title{The \textsf{childdoc} Package}
\hypersetup{pdftitle={The childdoc Package}}
\author{Niklas Beisert\\[2ex]
  Institut f\"ur Theoretische Physik\\
  Eidgen\"ossische Technische Hochschule Z\"urich\\
  Wolfgang-Pauli-Strasse 27, 8093 Z\"urich, Switzerland\\[1ex]
  \href{mailto:nbeisert@itp.phys.ethz.ch}
  {\texttt{nbeisert@itp.phys.ethz.ch}}}
\hypersetup{pdfauthor={Niklas Beisert}}
\hypersetup{pdfsubject={Manual for the LaTeX2e Package childdoc}}
\date{30 December 2018, \textsf{v2.0}}
\maketitle

\begin{abstract}\noindent
\textsf{childdoc} is a \LaTeXe{} package
that enables the direct compilation
of document sections included by |\include|
to individual files.
\end{abstract}

\begingroup
\parskip0ex
\tableofcontents
\endgroup

%%%%%%%%%%%%%%%%%%%%%%%%%%%%%%%%%%%%%%%%%%%%%%%%%%%%%%%%%%%%%%%%%%%%%%%%%%%%%%%%
%%%%%%%%%%%%%%%%%%%%%%%%%%%%%%%%%%%%%%%%%%%%%%%%%%%%%%%%%%%%%%%%%%%%%%%%%%%%%%%%
\section{Introduction}

\LaTeX{} provides a mechanism to structure a large document (such as a book)
into a main file and several child files (containing the chapters)
using the |\include| command.
This mechanism is beneficial for documents
which span hundreds of pages in order to
make the source file(s) more manageable.
Moreover, compilation can be restricted to
selected child files by means of the |\includeonly| command.
The latter feature can be used to reduce the compilation time while editing
(this was significantly more useful in the earlier days of \LaTeX{})
or to generate a smaller document which is easier to navigate.
Another application of |\includeonly| is to generate
documents consisting of selected parts of the complete document.

However, there are a few drawbacks of the plain |\include| mechanism:
\begin{itemize}
\item
The child files cannot be compiled on their own,
they can only be compiled via the main file.
A naive editing environment
(such as a text editor with an option
to have the current file processed by \LaTeX)
may require one to switch to the main file before compiling;
attempting to compile the child file produces errors.
\item
The main file must be modified (each time)
to adjust the |\includeonly| command
to the present needs. This easily leaves the main file in a messy state.
\item
The generated document will always carry the filename
of the main document. This is inconvenient if
several child files are to be compiled and
to be kept for distribution.
\end{itemize}

The present package provides a simple interface
to make child files individually compilable by \LaTeX{}.
Compiling a child file then has the same effect as compiling
the main file with an |\includeonly| command
to select the appropriate child.
Moreover the generated document will carry the name of the child
rather than the main file.
This resolves all three above issues.

This feature is meant to make the editing of books,
thesis documents and lecture notes somewhat more convenient.
However, the package can also be used efficiently for
composing a series of documents (such as exercise sheets)
which are typically distributed individually.
It then assists the author in generating the individual documents
(potentially in different versions)
as well as a document containing the collected series.
Another application is in developing style files
or other kinds of included material
where compilation of the style file could redirect
to a sample or test file.

%%%%%%%%%%%%%%%%%%%%%%%%%%%%%%%%%%%%%%%%%%%%%%%%%%%%%%%%%%%%%%%%%%%%%%%%%%%%%%%%
%%%%%%%%%%%%%%%%%%%%%%%%%%%%%%%%%%%%%%%%%%%%%%%%%%%%%%%%%%%%%%%%%%%%%%%%%%%%%%%%
\section{Usage}

First of all, the package \textsf{childdoc} is \emph{not} a standard
\LaTeXe{} |.sty| style file! Therefore it needs to be invoked in
a non-standard way.

%%%%%%%%%%%%%%%%%%%%%%%%%%%%%%%%%%%%%%%%%%%%%%%%%%%%%%%%%%%%%%%%%%%%%%%%%%%%%%%%
\subsection{Included Files}
\label{sec:include}

%%%%%%%%%%%%%%%%%%%%%%%%%%%%%%%%%%%%%%%%
\DescribeMacro{\childdocmain}
To use the package, add the commands
\begin{center}
\begin{tabular}{l}
|\input{childdoc.def}|\\
|\childdocmain{}|\\
\end{tabular}
\end{center}
at the very top of the main \LaTeX{} file,
in particular \emph{before} the |\documentclass| statement!
The argument of |\childdocmain| should be left empty
(but it must be present).

%%%%%%%%%%%%%%%%%%%%%%%%%%%%%%%%%%%%%%%%
\DescribeMacro{\childdocof}
Furthermore, add the commands
\begin{center}
\begin{tabular}{l}
|\input{childdoc.def}|\\
|\childdocof{|\textit{main}|}|\\
\end{tabular}
\end{center}
at the top of every child file \textit{child}
which is included by |\include{|\textit{child}|}|
from within the main file
(or at least for those files to be compiled individually).
The argument \textit{main} must be the filename of the main file.

There are a couple of
considerations in setting up the main and child documents:

%%%%%%%%%%%%%%%%%%%%%%%%%%%%%%%%%%%%%%%%
\paragraph{Restrictions.}

Please note the following restrictions:
\begin{itemize}
\item
|\childdocmain| must be called with one argument \textit{main}
to ensure compatibility with earlier version of the package.
It must either be empty (|\childdocmain{}|)
or precisely match the filename of the main file in which it is specified.
See \secref{sec:detection} for further information.
\item
The filename \textit{main} must be specified without the |.tex| extension.
\item
The filename \textit{main} is case sensitive
(even in case-insensitive file systems)
due to internal string comparison.
\item
The argument \textit{main} should be fully expanded, it cannot be a macro.
\item
Subdirectories and special characters should be avoided in filenames.
\item
The command |\childdocmain{|\textit{main}|}| must be followed by a whitespace.
It should not be followed immediately by another command
or by a comment mark `|%|'.
This is because the \TeX{} parser reads the token immediately following
the argument of |\childdocmain| and puts it
at the beginning of every child section;
however, a white\-space is ignored.
\end{itemize}

%%%%%%%%%%%%%%%%%%%%%%%%%%%%%%%%%%%%%%%%
\paragraph{Content of Main File.}

It is advisable to place all content in the child files included by |\include|.
Any output contained in the main file will appear in all child documents
unless suppressed manually;
it cannot be suppressed automatically by the |\includeonly| directive
and thus should normally be avoided.
A method to include some content in the main file
by means of conditional processing is described in \secref{sec:conditional}.

%%%%%%%%%%%%%%%%%%%%%%%%%%%%%%%%%%%%%%%%
\paragraph{Page Numbering.}

When only a part of the document is compiled,
the appropriate numbering of pages
(as well as other status parameters)
is determined from the |.aux| files.
The latter contain information from previous passes.
However this information needs to propagate through
all intermediate child documents.
Therefore the page numbering in child documents may well
be inconsistent until the complete document is compiled at least once.

A useful (if unconventional) way to always ensure a consistent
page numbering is to restart the numbering in each child document
and denote the pages by `\textit{child}|.|\textit{page}'
where \textit{child} represents the chapter/section number of the child file.
This can be achieved by the command
|\numberwithin{page}{|\textit{child}|}|
of the \textsf{amsmath} package
where \textit{child} can be |chapter| or |section|
depending on the chosen structuring.
Alternatively, one can modify the macro |\thepage| appropriately
and reset the counter |page| at the start of each child file.

%%%%%%%%%%%%%%%%%%%%%%%%%%%%%%%%%%%%%%%%%%%%%%%%%%%%%%%%%%%%%%%%%%%%%%%%%%%%%%%%
\subsection{Conditional Processing}
\label{sec:conditional}

The package provides a mechanism to compile different versions
of a document. To customise the versions further some conditional processing
can come in handy to distinguish which version is being compiled.
The package provides two macros to describe the compilation context:

%%%%%%%%%%%%%%%%%%%%%%%%%%%%%%%%%%%%%%%%
\DescribeMacro{\ifchilddoc}
The conditional |\ifchilddoc| distinguishes between the compilation of
child documents and the main document:
%
\begin{center}
|\ifchilddoc |\textit{child-code}| |[|\||else |\textit{main-code}]| \||fi|
\end{center}

%%%%%%%%%%%%%%%%%%%%%%%%%%%%%%%%%%%%%%%%
\DescribeMacro{\childdocname}
\DescribeMacro{\childdocjob}
The macro |\childdocname| contains the filename (without extension)
of the main or child file being processed.
Note that |\childdocjob| will always contain the name of the main file.

%%%%%%%%%%%%%%%%%%%%%%%%%%%%%%%%%%%%%%%%
\paragraph{Title Page.}

Conditional processing can be used to include a title or banner page
in the main document when proper precautions are taken.
Importantly, the code in the main file should ensure that the page counter
(as well as other status parameters which are stored in the |.aux| files)
takes the same value after the conditional processing.
Otherwise the page numbers may take divergent values
depending on which part is compiled.

For example, a title page could be declared by:
%
\begin{center}
\begin{tabular}{l}
|\ifchilddoc\||else|\\
|\addtocounter{page}{-1}|\\
\textit{code for title page}\\
|\newpage|\\
|\||fi|
\end{tabular}
\end{center}
%
A banner page for the child documents can be generated by:
%
\begin{center}
\begin{tabular}{l}
|\ifchilddoc|\\
|\addtocounter{page}{-1}|\\
\textit{code for banner page}\\
|\newpage|\\
|\||fi|
\end{tabular}
\end{center}
%
Here one could write a message such as:
\begin{center}
|This is the part \childdocname{} of \childdocjob{}.|
\end{center}

%%%%%%%%%%%%%%%%%%%%%%%%%%%%%%%%%%%%%%%%%%%%%%%%%%%%%%%%%%%%%%%%%%%%%%%%%%%%%%%%
\subsection{Flags}
\label{sec:flags}

The package makes it easy to generate different versions
of the main or child documents.
To this end compilation flags can be defined
and assigned different default values.
They will be particularly useful in conjunction
with the forwarding mechanism described in \secref{sec:forward}.

For example, it may be useful to have a flag |\version|
which can be set to |draft| or |final|.
The document source will contain some conditional code
depending on the value of |\version|.
Suppose further, the flag should default to |final| for the main file
and to |draft| for child files
which is a natural assignment for editing the document.
This is achieved by placing the following code
in the preamble of the main document
(below the |\childdocmain| directive):
%
\begin{center}
\begin{tabular}{l}
|\ifchilddoc|\\
|\providecommand{\version}{draft}|\\
|\||else|\\
|\providecommand{\version}{final}|\\
|\||fi|
\end{tabular}
\end{center}
%
The definition by |\providecommand| makes sure
that previous definitions are not overwritten.
Further statements |\providecommand{\version}{...}|
can thus be added before the above code to override it.

For the main file, one might add a line
(between |\childdocmain| and the above block)
%
\begin{center}
|%\ifchilddoc\||else\providecommand{\version}{draft}\||fi|
\end{center}
%
which can be uncommented to produce a draft version.
Likewise one can add a line to the very top of a child file
(above the |\childdocof{|\textit{main}|}| directive)
%
\begin{center}
|%\providecommand{\version}{final}|
\end{center}
%
which can be uncommented to produce the final version of this child document.

%%%%%%%%%%%%%%%%%%%%%%%%%%%%%%%%%%%%%%%%%%%%%%%%%%%%%%%%%%%%%%%%%%%%%%%%%%%%%%%%
\subsection{Forwarding}
\label{sec:forward}

Different versions of the main or child documents
using compilation flags as described in \secref{sec:flags}
can be (permanently) stored in different files
for convenient compilation, viewing and distribution.
To this end, the package defines a command
to pass on compilation to a different file:

%%%%%%%%%%%%%%%%%%%%%%%%%%%%%%%%%%%%%%%%
\DescribeMacro{\childdocforward}
The command |\childdocforward| redirects processing to
another source file:
%
\begin{center}
\begin{tabular}{l}
|\input{childdoc.def}|\\
|\childdocforward[|\textit{main}|]{|\textit{dest}|}|\\
\end{tabular}
\end{center}
%
The argument \textit{dest} is the destination file
(without extension).
It should be the main file or one of the child files.
Note that further \textsf{childdoc} directives
such as |\childdocof| and |\childdocforward|
in the indicated file will be processed in this form.
The optional argument \textit{main}
passes on directly to the main file \textit{main}
while pretending to compile the child \textit{dest}.
This form behaves as if \textit{dest}
issues |\childdocof{|\textit{main}|}| right away,
and no further \textsf{childdoc} directives will be processed.

%%%%%%%%%%%%%%%%%%%%%%%%%%%%%%%%%%%%%%%%
\DescribeMacro{\...prefix}
In the alternative form |\childdocforwardprefix|,
%
\begin{center}
\begin{tabular}{l}
|\input{childdoc.def}|\\
|\childdocforwardprefix[|\textit{main}|]{|\textit{prefix}|}{|\textit{dest}|}|
\end{tabular}
\end{center}
%
the destination file is determined by a pattern
depending on the current file:
To make this work, the current file must be called
`{\textit{prefix}\hspace{0.2em}\textit{suffix}}'
with \textit{prefix} matching precisely the argument.
Processing is then passed on to the file
`{\textit{dest}\hspace{0.2em}\textit{suffix}}'.
Surely, the same effect is achieved by
directly specifying the
argument `{\textit{dest}\hspace{0.2em}\textit{suffix}}'
in the first form.
However, that requires to set up a different file
for each child. With the alternative form of the command
all these files can have exactly the same content
which simplifies setting them up and maintaining them.

For example, the following file |draft.tex|
with a compilation flag |\version| as described in \secref{sec:flags}
compiles the main document as a draft:
%
\begin{center}
\begin{tabular}{l}
|\def\version{draft}|\\
|\input{childdoc.def}|\\
|\childdocforward{|\textit{main}|}|
\end{tabular}
\end{center}
%
Likewise, the following files |final|\textit{nn}|.tex|
compile the final version of the child document
|child|\textit{nn}|.tex|:
%
\begin{center}
\begin{tabular}{l}
|\def\version{final}|\\
|\input{childdoc.def}|\\
|\childdocforwardprefix{final}{child}|
\end{tabular}
\end{center}
%

Note that when several versions of a main file and/or of each child file
are to be generated, it may be convenient to set up a |Makefile| or
shell script to automatise the process.

%%%%%%%%%%%%%%%%%%%%%%%%%%%%%%%%%%%%%%%%%%%%%%%%%%%%%%%%%%%%%%%%%%%%%%%%%%%%%%%%
\subsection{Command Line Processing}
\label{sec:commandline}

The effect of redirection files can also be achieved by invoking
the \LaTeX{} compiler with a more elaborate command line.
Most conveniently this should be done as part
of a shell script or a |Makefile|.

When using \textsf{childdoc} in the main file, the following
command lines effectively perform a redirection
(note that depending on the shell being used,
backslashes may have to be doubled: `|\|' $\to$ `|\\|'):
%
\begin{center}
|... -jobname "|\textit{target}|" |\\|"|[\textit{flags}]%
|\input{childdoc.def}\childdocforward[|\textit{main}|]{|\textit{dest}|}"|
\end{center}
%
Here \textit{target} is the name of the output file,
\textit{main} is the name of the main file
and \textit{dest} is the name of the main or child file to be processed
(all filenames without extensions).
The optional argument \textit{main} can be omitted
if \textit{main} matches \textit{dest}.
Optionally, compilation \textit{flags} can be defined via |\def| commands.
This command line makes the \TeX{} engine believe
it is compiling the file \textit{target}
whose content is specified as the latter parameter.
The provided code then forwards the processing to
\textit{main} or \textit{dest} as described in \secref{sec:forward}.

%%%%%%%%%%%%%%%%%%%%%%%%%%%%%%%%%%%%%%%%%%%%%%%%%%%%%%%%%%%%%%%%%%%%%%%%%%%%%%%%
\subsection{Include by Input}
\label{sec:input}

Including child documents by |\include| has some restrictions by design.
Most notably, the content of a child document always occupies
its own set of pages; pages cannot be shared between child documents.
Usually, this behaviour makes perfect sense
because each child document contain an essential part of the document.
However, in some situations it may be desirable to compose
a document from a collection of parts
without having mandatory page breaks between then.
For this case, the package
provides a mechanism to include parts
by |\input| which can also be processed individually.
However, by construction this mechanism
requires manual handling of the content to be output.

%%%%%%%%%%%%%%%%%%%%%%%%%%%%%%%%%%%%%%%%
\DescribeMacro{\ifchilddocmanual}
The main file should be prepared as usual, see \secref{sec:include}.
However, the document body must make a distinction
between processing of an individual part and of the main document, e.g.:
%
\begin{center}
\begin{tabular}{l}
|\ifchilddocmanual|\\
|\input{\childdocname}|\\
|\||else|\\
\textit{document body with }|\input{|\textit{part}|}|\\
|\||fi|
\end{tabular}
\end{center}
%
The conditional |\ifchilddocmanual| is true whenever
a part to be included by |\input| is being compiled,
and the name of the part is stored in |\childdocname|.

%%%%%%%%%%%%%%%%%%%%%%%%%%%%%%%%%%%%%%%%
\DescribeMacro{\childdocby}
Each part to be included by |\input| should start with:
%
\begin{center}
\begin{tabular}{l}
|\input{childdoc.def}|\\
|\childdocby{|\textit{main}|}|\\
\end{tabular}
\end{center}
%
The directive |\childdocby| is similar to |\childdocof|
described in \secref{sec:include},
but the subsequent selection of content must be done manually.
To that end, both |\ifchilddoc| and |\ifchilddocmanual|
will be true upon processing of a part,
and the name of the part is stored in |\childdocname|.
Note that |\jobname| will be set to the filename of the current part
so that each part receives an individual |.aux| file
that does not interfere with the |.aux| file(s) of the main document.
This behaviour can be altered by the alternative form
|\childdocby[*]{|\textit{main}|}| (with a non-empty optional argument)
which uses the |.aux| file of the main document
by setting |\jobname| to \textit{main}.

%%%%%%%%%%%%%%%%%%%%%%%%%%%%%%%%%%%%%%%%%%%%%%%%%%%%%%%%%%%%%%%%%%%%%%%%%%%%%%%%
\subsection{Driver Development}
\label{sec:driver}

The \textsf{childdoc} mechanism can also be use for the development
of definition files such as \LaTeX{} styles or classes.
This case differs from the above setup with multiple parts
included by |\include| in that no |\includeonly| should be invoked.
This can be achieved by starting the include file
(before |\ProvidesPackage|) with:
%
\begin{center}
\begin{tabular}{l}
|\input{childdoc.def}|\\
|\childdocforward{|\textit{main}|}|\\
\end{tabular}
\end{center}
%
or alternatively with:
%
\begin{center}
\begin{tabular}{l}
|\input{childdoc.def}|\\
|\childdocby{|\textit{main}|}|\\
\end{tabular}
\end{center}
%
Both forms have slightly different effects as described above.
The main file is prepared as usual, see \secref{sec:include}.

%%%%%%%%%%%%%%%%%%%%%%%%%%%%%%%%%%%%%%%%%%%%%%%%%%%%%%%%%%%%%%%%%%%%%%%%%%%%%%%%
\subsection{Legacy Detection}
\label{sec:detection}

The directive |\childdocmain| in the main file can detect
whether the complete document or merely a child is to be compiled
even without using the directive |\childdocof|.
This method is deprecated because it is less robust
and there is no compelling reason to use it;
it is merely provided for backward compatibility
and it may be removed in future versions.

If the detection mechanism is to be used,
it is mandatory to correctly specify
the filename of the main file as the argument of |\childdocmain|:
%
\begin{center}
\begin{tabular}{l}
|\input{childdoc.def}|\\
|\childdocmain{|\textit{main}|}|\\
\end{tabular}
\end{center}
%
If |\jobname| does not match the argument \textit{main} of |\childdocmain|,
it is assumed that |\jobname| points to the child file to be compiled.
When using |\childdocmain| with the main file specified as argument,
it suffices to start a child file
with just |\input{|\textit{main}|}|
without loading of the package and using |\childdocof|.
If instead all processing is done
with the appropriate \textsf{childdoc} directives,
the argument of \textit{main} of |\childdocmain| can be empty.

An alternative version of the command line processing described
in \secref{sec:commandline} using the detection mechanism reads:
%
\begin{center}
|... -jobname "|\textit{target}|" "|[\textit{flags}]%
[|\def\jobname{|\textit{dest}|}|]|\input{|\textit{main}|}"|
\end{center}

%%%%%%%%%%%%%%%%%%%%%%%%%%%%%%%%%%%%%%%%%%%%%%%%%%%%%%%%%%%%%%%%%%%%%%%%%%%%%%%%
\subsection{Manual Code}
\label{sec:manual}

In case one cannot be certain whether the definitions file |childdoc.def|
is installed on the target \TeX{} distribution
and one prefers not to ship it,
it is conceivable to paste a few relevant commands into the sources.

To that end, drop all statements |\input{childdoc.def}|
and perform the replacements as outlined below.
Instead of |\childdocmain{|\textit{main}|}| add the following code
to the top of the main file:
%
\begin{center}
\begin{tabular}{l}
|\||ifdefined\childdocname\endinput\||fi\newif\ifchilddoc|\\
|\edef\childdocname{\scantokens\expandafter{\jobname\noexpand}}|\\
|\def\childdocmain{|\textit{main}|}\||ifx\childdocmain\childdocname\||else|\\
|\childdoctrue\includeonly{\childdocname}\let\jobname\childdocmain\||fi|\\
\end{tabular}
\end{center}
%
Instead of |\childdocof{|\textit{main}|}| just include the main file
at the top of each child file:
%
\begin{center}
|\input{|\textit{main}|}|
\end{center}
%
A simple redirection |\childdocforward{|\textit{dest}|}| is achieved by:
%
\begin{center}
|\def\jobname{|\textit{dest}|}\input{\jobname}|
\end{center}
%
The redirection with prefix
|\childdocforwardprefix[|\textit{prefix}|]{|\textit{dest}|}|
is accomplished by:
%
\begin{center}
\begin{tabular}{l}
|{\edef\jobname{\scantokens\expandafter{\jobname\noexpand}}|\\
|\def\redirectjob |\textit{prefix}|#1~~~{\gdef\jobname{|\textit{dest}|#1}}|\\
|\expandafter\redirectjob\jobname~~~}\input{\jobname}|
\end{tabular}
\end{center}

In an alternative approach,
child documents can be compiled by a specific command line
without additional code or specific definitions:
%
\begin{center}
|... -jobname "|\textit{target}|" "|[\textit{flags}]%
|\includeonly{|\textit{dest}|}\input{|\textit{main}|}"|
\end{center}
%

%%%%%%%%%%%%%%%%%%%%%%%%%%%%%%%%%%%%%%%%%%%%%%%%%%%%%%%%%%%%%%%%%%%%%%%%%%%%%%%%
%%%%%%%%%%%%%%%%%%%%%%%%%%%%%%%%%%%%%%%%%%%%%%%%%%%%%%%%%%%%%%%%%%%%%%%%%%%%%%%%
\section{Information}

%%%%%%%%%%%%%%%%%%%%%%%%%%%%%%%%%%%%%%%%%%%%%%%%%%%%%%%%%%%%%%%%%%%%%%%%%%%%%%%%
\subsection{Copyright}

Copyright \copyright{} 2017--2018 Niklas Beisert

This work may be distributed and/or modified under the
conditions of the \LaTeX{} Project Public License, either version 1.3
of this license or (at your option) any later version.
The latest version of this license is in
  \url{http://www.latex-project.org/lppl.txt}
and version 1.3 or later is part of all distributions of \LaTeX{}
version 2005/12/01 or later.

This work has the LPPL maintenance status `maintained'.

The Current Maintainer of this work is Niklas Beisert.

This work consists of the files |README.txt|, |childdoc.ins| and |childdoc.dtx|
as well as the derived files |childdoc.def|, |cdocsamp.tex|
with |cdocsch1.tex|, |cdocsch2.tex|, |cdocspt3.tex|, |cdocspt4.tex|,
|cdocsdrf.tex|, |cdocsfn1.tex|, |cdocsfn2.tex|
as well as |childdoc.pdf|.

%%%%%%%%%%%%%%%%%%%%%%%%%%%%%%%%%%%%%%%%%%%%%%%%%%%%%%%%%%%%%%%%%%%%%%%%%%%%%%%%
\subsection{Files and Installation}

The package consists of the files:
%
\begin{center}
\begin{tabular}{ll}
    |README.txt|   & readme file \\
    |childdoc.ins| & installation file \\
    |childdoc.dtx| & source file \\
    |childdoc.def| & definition file \\
    |cdocsamp.tex| & sample main file \\
    |cdocsch1.tex| & sample include file \\
    |cdocsch2.tex| & sample include file \\
    |cdocspt3.tex| & sample part file \\
    |cdocspt4.tex| & sample part file \\
    |cdocsdrf.tex| & sample redirection file \\
    |cdocsfn1.tex| & sample redirection file \\
    |cdocsfn2.tex| & sample redirection file \\
    |childdoc.pdf| & manual
\end{tabular}
\end{center}
%
The distribution consists of the files
|README.txt|, |childdoc.ins| and |childdoc.dtx|.
%
\begin{itemize}
\item
Run (pdf)\LaTeX{} on |childdoc.dtx|
to compile the manual |childdoc.pdf| (this file).
\item
Run \LaTeX{} on |childdoc.ins| to create the definitions file |childdoc.def|
and the sample |cdocsamp.tex| with include files
|cdocsch1.tex|, |cdocsch2.tex|, |cdocspt3.tex|, |cdocspt4.tex|,
|cdocsdrf.tex|, |cdocsfn1.tex|, |cdocsfn2.tex|.
Then copy the file |childdoc.def| to an appropriate directory of your \LaTeX{}
distribution, e.g.\ \textit{texmf-root}|/tex/latex/childdoc|.
\end{itemize}

%%%%%%%%%%%%%%%%%%%%%%%%%%%%%%%%%%%%%%%%%%%%%%%%%%%%%%%%%%%%%%%%%%%%%%%%%%%%%%%%
\subsection{Related CTAN Packages}

There are several other packages which offer a similar functionality:
%
\begin{itemize}
\item
The packages
\href{http://ctan.org/pkg/docmute}{\textsf{docmute}},
\href{http://ctan.org/pkg/includex}{\textsf{includex}} and
\href{http://ctan.org/pkg/standalone}{\textsf{standalone}}
provide commands to include only the document body of
a child file thus allowing both files to be compiled individually.
\item
The packages \href{http://ctan.org/pkg/subdocs}{\textsf{subdocs}}
and \href{http://ctan.org/pkg/subfiles}{\textsf{subfiles}}
provide structures in which the main and child documents can be
encapsulated and allowing them to be compiled individually.
The inclusion mechanism is different from the conventional |\include|.
\item
The package \href{http://ctan.org/pkg/combine}{\textsf{combine}}
is an elaborate solution to combine several documents into one.
\end{itemize}
%
See also the CTAN topic \href{http://ctan.org/topic/subdocs}{\textsf{subdocs}}
for further related packages.
The present package differs from the above solutions in that
a document structure constructed with the conventional |\include| mechanism
just needs two extra commands at the top of every file
such that all constituent files can be compiled individually.

%%%%%%%%%%%%%%%%%%%%%%%%%%%%%%%%%%%%%%%%%%%%%%%%%%%%%%%%%%%%%%%%%%%%%%%%%%%%%%%%
%\subsection{Feature Suggestions}
%
%The following is a list of features which may be useful for future
%versions of this package:
%%
%\begin{itemize}
%\item
%\ldots
%\end{itemize}

%%%%%%%%%%%%%%%%%%%%%%%%%%%%%%%%%%%%%%%%%%%%%%%%%%%%%%%%%%%%%%%%%%%%%%%%%%%%%%%%
\subsection{Revision History}

%%%%%%%%%%%%%%%%%%%%%%%%%%%%%%%%%%%%%%%%
\paragraph{v2.0:} 2018/12/30

\begin{itemize}
\item
immediate forward processing
\item
added |\childdocby| mechanism
\item
manual restructured
\end{itemize}

%%%%%%%%%%%%%%%%%%%%%%%%%%%%%%%%%%%%%%%%
\paragraph{v1.6:} 2018/01/17

\begin{itemize}
\item
application for development of include files
\item
corrections to manual
\end{itemize}

%%%%%%%%%%%%%%%%%%%%%%%%%%%%%%%%%%%%%%%%
\paragraph{v1.5:} 2017/05/21

\begin{itemize}
\item
more complete structuring introduced
\item
|\childdocof| introduced
\item
|\childdoc| renamed to |\childdocmain|
\item
|\childredirect| renamed to |\childdocforward| and |\childdocforwardprefix|
and functionality expanded
\end{itemize}

%%%%%%%%%%%%%%%%%%%%%%%%%%%%%%%%%%%%%%%%
\paragraph{v1.0:} 2017/04/27

\begin{itemize}
\item
manual and install package
\item
first version published on CTAN
\end{itemize}

%%%%%%%%%%%%%%%%%%%%%%%%%%%%%%%%%%%%%%%%
\paragraph{v0.6:} 2017/04/26

\begin{itemize}
\item
redirection mechanism added
\end{itemize}

%%%%%%%%%%%%%%%%%%%%%%%%%%%%%%%%%%%%%%%%
\paragraph{v0.5:} 2017/04/26

\begin{itemize}
\item
functionality in definition file
\end{itemize}


%%%%%%%%%%%%%%%%%%%%%%%%%%%%%%%%%%%%%%%%%%%%%%%%%%%%%%%%%%%%%%%%%%%%%%%%%%%%%%%%
%%%%%%%%%%%%%%%%%%%%%%%%%%%%%%%%%%%%%%%%%%%%%%%%%%%%%%%%%%%%%%%%%%%%%%%%%%%%%%%%
%%%%%%%%%%%%%%%%%%%%%%%%%%%%%%%%%%%%%%%%%%%%%%%%%%%%%%%%%%%%%%%%%%%%%%%%%%%%%%%%
\appendix

\settowidth\MacroIndent{\rmfamily\scriptsize 000\ }

 \DocInput{childdoc.dtx}

\end{document}
%</driver>
% \fi
%
% %%%%%%%%%%%%%%%%%%%%%%%%%%%%%%%%%%%%%%%%%%%%%%%%%%%%%%%%%%%%%%%%%%%%%%%%%%%%%%
% %%%%%%%%%%%%%%%%%%%%%%%%%%%%%%%%%%%%%%%%%%%%%%%%%%%%%%%%%%%%%%%%%%%%%%%%%%%%%%
% \section{Sample}
%\iffalse
%<*samplemain>
%\fi
%
% The following presents a sample document
% with two chapters, two parts, a title page,
% a compile flag as well as three forwarding files to set the flag.
% It consists of eight |.tex| files:
% \begin{center}
% \begin{tabular}{ll}
% |cdocsamp.tex|&main file\\
% |cdocsch1.tex|&include file for chapter 1\\
% |cdocsch2.tex|&include file for chapter 2\\
% |cdocspt3.tex|&include file for part 3\\
% |cdocspt4.tex|&include file for part 4\\
% |cdocsdrf.tex|&forwarding file for main file in draft mode\\
% |cdocsfi1.tex|&forwarding file for final version of chapter 1\\
% |cdocsfi2.tex|&forwarding file for final version of chapter 2\\
% \end{tabular}
% \end{center}
% Each of the eight files can be compiled directly by the \LaTeX{} compiler.
%
% %%%%%%%%%%%%%%%%%%%%%%%%%%%%%%%%%%%%%%
% \paragraph{Main File.}
%
% The main file is called |cdocsamp.tex|.
%
% Load the \textsf{childdoc} definitions and
% declare the filename for the main document:
%    \begin{macrocode}
\input{childdoc.def}
\childdocmain{}
%    \end{macrocode}

% Optional override for |\version| flag:
%    \begin{macrocode}
%%\ifchilddoc\else\providecommand{\version}{draft}\fi
%    \end{macrocode}

% Define the default values for the |\version| flag
% (|final| for the main file and |draft| for childs):
%    \begin{macrocode}
\ifchilddoc
\providecommand{\version}{draft}
\else
\providecommand{\version}{final}
\fi
%    \end{macrocode}

% Load the standard document class:
%    \begin{macrocode}
\documentclass[12pt]{article}
%    \end{macrocode}

% Start the document body:
%    \begin{macrocode}
\begin{document}
%    \end{macrocode}

% Declare a title page.
% Print title, part of document being processed and version flag:
%    \begin{macrocode}
\addtocounter{page}{-1}
\begin{center}
{\LARGE\bfseries{}childdoc example\par}
\vspace{1cm}
\ifchilddoc
\ifchilddocmanual part\else chapter\fi:
`\childdocname' of `\childdocjob'\par
\else
main document: `\childdocjob'\par
\fi
version: \version\par
\end{center}
\newpage
%    \end{macrocode}

% Manually include selected file,
% otherwise process as usual:
%    \begin{macrocode}
\ifchilddocmanual
\section*{part `\childdocname'}
\input{\childdocname}
\else
%    \end{macrocode}

% Include the two chapters:
%    \begin{macrocode}
\include{cdocsch1}
\include{cdocsch2}
%    \end{macrocode}

% Include the two parts unless only chapters should be displayed:
%    \begin{macrocode}
\ifchilddoc\else
\section{part three}
\input{cdocspt3}
\section{part four}
\input{cdocspt4}
\fi
%    \end{macrocode}

% Process as usual until here:
%    \begin{macrocode}
\fi
%    \end{macrocode}

% End of document body:
%    \begin{macrocode}
\end{document}
%    \end{macrocode}
%\iffalse
%</samplemain>
%\fi
%
% %%%%%%%%%%%%%%%%%%%%%%%%%%%%%%%%%%%%%%
% \paragraph{Chapter Include Files.}
%
% The include files are called |cdocsch1.tex| and |cdocsch2.tex|.
%
%\iffalse
%<*samplechap1|samplechap2>
%\fi

% Optional override for |\version| flag:
%    \begin{macrocode}
%%\providecommand{\version}{final}
%    \end{macrocode}

% Include the main document:
%    \begin{macrocode}
\input{childdoc.def}
\childdocof{cdocsamp}
%    \end{macrocode}

%\iffalse
%</samplechap1|samplechap2>
%\fi
%
%\iffalse
%<*samplechap1>
%\fi
% Some text for chapter 1:
%    \begin{macrocode}
\section{one}
some text in chapter one
%    \end{macrocode}

%\iffalse
%</samplechap1>
%\fi
% Some text for chapter 2:
%\iffalse
%<*samplechap2>
%\fi
%    \begin{macrocode}
\section{two}
more text in chapter two
%    \end{macrocode}

%\iffalse
%</samplechap2>
%\fi
%
% %%%%%%%%%%%%%%%%%%%%%%%%%%%%%%%%%%%%%%
% \paragraph{Part Include Files.}
%
% The include files are called |cdocspt3.tex| and |cdocspt4.tex|.
%
%\iffalse
%<*samplepart3|samplepart4>
%\fi

% Optional override for |\version| flag:
%    \begin{macrocode}
%%\providecommand{\version}{final}
%    \end{macrocode}

% Include the main document:
%    \begin{macrocode}
\input{childdoc.def}
\childdocby{cdocsamp}
%    \end{macrocode}

%\iffalse
%</samplepart3|samplepart4>
%\fi
%
%\iffalse
%<*samplepart3>
%\fi
% Some text for part 3:
%    \begin{macrocode}
some text in part three
%    \end{macrocode}

%\iffalse
%</samplepart3>
%\fi
% Some text for part 4:
%\iffalse
%<*samplepart4>
%\fi
%    \begin{macrocode}
more text in part four
%    \end{macrocode}

%\iffalse
%</samplepart4>
%\fi
%
% %%%%%%%%%%%%%%%%%%%%%%%%%%%%%%%%%%%%%%
% \paragraph{Forwarding for a Complete Draft.}
%
% The following forwarding file |cdocsdrf.tex|
% compiles the main document in draft mode:
%\iffalse
%<*sampledraft>
%\fi
%    \begin{macrocode}
\def\version{draft}
\input{childdoc.def}
\childdocforward{cdocsamp}
%    \end{macrocode}

%\iffalse
%</sampledraft>
%\fi
%
% %%%%%%%%%%%%%%%%%%%%%%%%%%%%%%%%%%%%%%
% \paragraph{Forwarding for Final Version of the Chapters.}
%
% The following forwarding files |cdocsfn1.tex| and |cdocsfn2.tex|
% (with identical content)
% compile the final versions of the child documents
% |cdocsch1.tex| and |cdocsch2.tex|, respectively:
%\iffalse
%<*samplefinal>
%\fi
%    \begin{macrocode}
\def\version{final}
\input{childdoc.def}
\childdocforwardprefix[cdocsamp]{cdocsfn}{cdocsch}
%    \end{macrocode}

%\iffalse
%</samplefinal>
%\fi
%
% %%%%%%%%%%%%%%%%%%%%%%%%%%%%%%%%%%%%%%
% \paragraph{Command Line Processing.}
%
% The following three command lines generate the output files
% |cdocscld|, |cdocscl1| and |cdocscl2|
% which should be identical to
% |cdocsdrf|, |cdocsch1| and |cdocsfn2|, respectively:
% \begin{center}
% \begin{tabular}{l}
% |latex -jobname cdocscld \|\\
% |  "\def\version{draft}\input{childdoc.def}\childdocforward{cdocsamp}"|\\
% |latex -jobname cdocscl1 \|\\
% |  "\input{childdoc.def}\childdocforward[cdocsamp]{cdocsch1}"|\\
% |latex -jobname cdocscl2 \|\\
% |  "\def\version{final}\input{childdoc.def}\childdocforward{cdocsch2}"|
% \end{tabular}
% \end{center}
% Note that the trailing backslash on each first line
% merely continues the input to the second line
% (for convenient cut ant paste).
% Furthermore, the command |latex| can be replaced by any
% of its alternative versions such as |pdflatex|.
%
% %%%%%%%%%%%%%%%%%%%%%%%%%%%%%%%%%%%%%%%%%%%%%%%%%%%%%%%%%%%%%%%%%%%%%%%%%%%%%%
% %%%%%%%%%%%%%%%%%%%%%%%%%%%%%%%%%%%%%%%%%%%%%%%%%%%%%%%%%%%%%%%%%%%%%%%%%%%%%%
% \section{Implementation}
%\iffalse
%<*package>
%\fi
%
% This section describes the definitions file |childdoc.def|.

% The definitions cannot be loaded using |\usepackage| or |\RequirePackage|
% which has a mechanism to prevent loading a style file more than once.
% When loading the definitions by means of |\input|
% multiple instances have to be prevented manually:
%\iffalse
%This code needs to be before the `\ProvidesFile' directive
%which is defined at the beginning of this file.
%Therefore it is also placed there and commented out here.
%</package>
%<*discard>
%\fi
%    \begin{macrocode}
\ifdefined\childdocmain\endinput\fi
%    \end{macrocode}
%\iffalse
%</discard>
%<*package>
%\fi
%
% \macro{\ifchilddoc}
% \macro{\ifchilddocmanual}
% The conditional |\ifchilddoc| tells whether a
% child (true) or main (false) document is being compiled.
% The conditional |\ifchilddocmanual| tells whether
% the |\includeonly| mechanism is used (false) or
% the selection of child files must be performed manually (true).
% The definitions initialise to false:
%    \begin{macrocode}
\newif\ifchilddoc
\newif\ifchilddocmanual
%    \end{macrocode}

% \macro{\childdocname}
% \macro{\childdocjob}
% The macro |\childdocname| stores the name of the main document
% to be compiled. The macro |\childdocjob| stores the name of
% the document on which the \LaTeX{} compiler was originally invoked.
% The content of |\jobname| cannot be compared
% to filenames specified in the source due to different catcodes.
% The following code rescans |\jobname|, stores the result
% in |\childdocname| and saves a copy in |\childdocjob|:
%    \begin{macrocode}
\edef\childdocname{\scantokens\expandafter{\jobname\noexpand}}
\let\childdocjob\childdocname
%    \end{macrocode}

% \macro{\childdocdisable}
% The macro |\childdocdisable| prevents the main file
% from being processed more than once.
% At this stage, the main document command |\childdocmain|
% is assumed to be called once again where it should do nothing.
% Any subsequent call to it should prevent
% a secondary processing of the main document
% It overwrites the forwarding commands
% |\childdocof| and |\childdocforward|
% with empty macros to prevent further inclusions of the main document:
%    \begin{macrocode}
\newcommand{\childdocdisable}
{
  \renewcommand{\childdocmain}[1]{\renewcommand{\childdocmain}[1]{\endinput}}
  \renewcommand{\childdocof}[1]{}
  \renewcommand{\childdocby}[2][]{}
  \renewcommand{\childdocforward}[2][]{}
  \renewcommand{\childdocdisable}{}
}
%    \end{macrocode}

% \macro{\childdocmain}
% The macro |\childdocmain| is to be called at the top of the main file
% with nothing or the main filename (without extension) as argument.
% First, it breaks loops.
% If the argument is not empty and does not match |\childdocname|
% (which is set by the first inclusion of |childdoc.def|),
% |\ifchilddoc| is set to true, |\includeonly| is applied to the child file
% and |\jobname| is set to the main file
% (for proper handling of |.aux| files):
%    \begin{macrocode}
\newcommand{\childdocmain}[1]
{
  \childdocdisable\childdocmain{}
  \if?#1?\else
    \begingroup
      \def\childdoctmp{#1}
      \ifx\childdoctmp\childdocname
        \def\childdoctmp{}
      \else
        \def\childdoctmp
        {
          \childdoctrue
          \includeonly{\childdocname}
          \def\childdocjob{#1}
          \def\jobname{#1}
        }
      \fi
      \expandafter
    \endgroup
    \childdoctmp
  \fi
}
%    \end{macrocode}

% \macro{\childdocof}
% The command |\childdocof| redirects
% compilation to the main file |#1|.
%    \begin{macrocode}
\newcommand{\childdocof}[1]
{
  \childdocdisable
  \childdoctrue
  \includeonly{\childdocname}
  \def\jobname{#1}
  \def\childdocjob{#1}
  \input{#1}
}
%    \end{macrocode}

% \macro{\childdocby}
% The command |\childdocby| ....
%    \begin{macrocode}
\newcommand{\childdocby}[2][]
{
  \childdocdisable
  \childdoctrue
  \childdocmanualtrue
  \if?#1?\else
    \def\jobname{#2}
  \fi
  \def\childdocjob{#2}
  \input{#2}
  \endinput
}
%    \end{macrocode}

% \macro{\childdocforward}
% The command |\childdocforward| redirects
% compilation to the main file or
% (if the optional argument is given) a child file.
% Parameters are set as if the main file
% or a child file starting with |\childdocof| was compiled.
% Then compilation is handed over to the main file:
%    \begin{macrocode}
\newcommand{\childdocforward}[2][]
{
  \begingroup
    \if?#1?
      \def\childdoctmp
      {
        \def\childdocname{#2}
        \def\childdocjob{#2}
        \def\jobname{#2}
        \input{#2}
        \endinput
      }
    \else
      \def\childdoctmp
      {
        \childdocdisable
        \def\childdocname{#2}
        \childdoctrue
        \includeonly{#2}
        \def\childdocjob{#1}
        \def\jobname{#1}
        \input{#1}
        \endinput
      }
    \fi
    \expandafter
  \endgroup
  \childdoctmp
}
%    \end{macrocode}

% \macro{\childdocforwardprefix}
% The command |\childdocforwardprefix| redirects
% compilation to the main or a child file by means of a pattern.
% The prefix |#1| in the current filename is replaced by |#2|
% and the suffix of the current filename is kept
% (it is assumed that the filename does not contain the substring `|~~~|'
% which is used as a delimiter).
% Compilation is handed over to the new file by |\childdocforward|:
%    \begin{macrocode}
\newcommand{\childdocforwardprefix}[3][]
{
  \begingroup
    \def\childdocextract #2##1~~~{\def\childdoctmp{\childdocforward[#1]{#3##1}}}
    \expandafter\childdocextract\childdocname~~~
    \expandafter
  \endgroup
  \childdoctmp
}
%    \end{macrocode}

% \macro{\childdoc}
% The deprecated macro |\childdoc| is a legacy version of |\childdocmain|:
%    \begin{macrocode}
\newcommand{\childdoc}{\childdocmain}
%    \end{macrocode}

% \macro{\childdocredirect}
% The deprecated macro |\childdocredirect| is a legacy version
% of |\childdocforward| and |\childdocforwardprefix|:
%    \begin{macrocode}
\newcommand{\childdocredirect}[2][]
{
  \begingroup
    \if?#1?
      \def\childdoctmp{\childdocforward{#2}}
    \else
      \def\childdoctmp{\childdocforwardprefix{#1}{#2}}
    \fi
    \expandafter
  \endgroup
  \childdoctmp
}
%    \end{macrocode}

%\iffalse
%</package>
%\fi
%
\endinput
\childdocforward[|\textit{main}|]{|\textit{dest}|}"|
\end{center}
%
Here \textit{target} is the name of the output file,
\textit{main} is the name of the main file
and \textit{dest} is the name of the main or child file to be processed
(all filenames without extensions).
The optional argument \textit{main} can be omitted
if \textit{main} matches \textit{dest}.
Optionally, compilation \textit{flags} can be defined via |\def| commands.
This command line makes the \TeX{} engine believe
it is compiling the file \textit{target}
whose content is specified as the latter parameter.
The provided code then forwards the processing to
\textit{main} or \textit{dest} as described in \secref{sec:forward}.

%%%%%%%%%%%%%%%%%%%%%%%%%%%%%%%%%%%%%%%%%%%%%%%%%%%%%%%%%%%%%%%%%%%%%%%%%%%%%%%%
\subsection{Include by Input}
\label{sec:input}

Including child documents by |\include| has some restrictions by design.
Most notably, the content of a child document always occupies
its own set of pages; pages cannot be shared between child documents.
Usually, this behaviour makes perfect sense
because each child document contain an essential part of the document.
However, in some situations it may be desirable to compose
a document from a collection of parts
without having mandatory page breaks between then.
For this case, the package
provides a mechanism to include parts
by |\input| which can also be processed individually.
However, by construction this mechanism
requires manual handling of the content to be output.

%%%%%%%%%%%%%%%%%%%%%%%%%%%%%%%%%%%%%%%%
\DescribeMacro{\ifchilddocmanual}
The main file should be prepared as usual, see \secref{sec:include}.
However, the document body must make a distinction
between processing of an individual part and of the main document, e.g.:
%
\begin{center}
\begin{tabular}{l}
|\ifchilddocmanual|\\
|\input{\childdocname}|\\
|\||else|\\
\textit{document body with }|\input{|\textit{part}|}|\\
|\||fi|
\end{tabular}
\end{center}
%
The conditional |\ifchilddocmanual| is true whenever
a part to be included by |\input| is being compiled,
and the name of the part is stored in |\childdocname|.

%%%%%%%%%%%%%%%%%%%%%%%%%%%%%%%%%%%%%%%%
\DescribeMacro{\childdocby}
Each part to be included by |\input| should start with:
%
\begin{center}
\begin{tabular}{l}
|% \iffalse
%
% childdoc.dtx Copyright (C) 2017-2018 Niklas Beisert
%
% This work may be distributed and/or modified under the
% conditions of the LaTeX Project Public License, either version 1.3
% of this license or (at your option) any later version.
% The latest version of this license is in
%   http://www.latex-project.org/lppl.txt
% and version 1.3 or later is part of all distributions of LaTeX
% version 2005/12/01 or later.
%
% This work has the LPPL maintenance status `maintained'.
%
% The Current Maintainer of this work is Niklas Beisert.
%
% This work consists of the files childdoc.dtx and childdoc.ins
% and the derived files childdoc.def and cdocsamp.tex with
% cdocsch1.tex, cdocsch2.tex, cdocsdrf.tex, cdocsfn1.tex, cdocsfn2.tex.
%
%<package>\ifdefined\childdocmain\endinput\fi
%<package>\ProvidesFile{childdoc.def}[2018/12/30 v2.0 child document driver]
%<samplemain>\ProvidesFile{cdocsamp.tex}[2018/12/30 v2.0 sample for childdoc]
%<*driver>
%\ProvidesFile{childdoc.drv}[2018/12/30 v2.0 childdoc reference manual file]
\PassOptionsToClass{10pt,a4paper}{article}
\documentclass{ltxdoc}

\usepackage[margin=35mm]{geometry}
\usepackage{hyperref}
\usepackage{hyperxmp}
\usepackage[usenames]{color}

\hypersetup{colorlinks=true}
\hypersetup{pdfstartview=FitH}
\hypersetup{pdfpagemode=UseNone}
\hypersetup{pdfsource={}}
\hypersetup{pdflang={en-UK}}
\hypersetup{pdfcopyright={Copyright 2017-2018 Niklas Beisert.
  This work may be distributed and/or modified under the
  conditions of the LaTeX Project Public License, either version 1.3
  of this license or (at your option) any later version.}}
\hypersetup{pdflicenseurl={http://www.latex-project.org/lppl.txt}}
\hypersetup{pdfcontactaddress={ETH Zurich, ITP, HIT K,
  Wolfgang-Pauli-Strasse 27}}
\hypersetup{pdfcontactpostcode={8093}}
\hypersetup{pdfcontactcity={Zurich}}
\hypersetup{pdfcontactcountry={Switzerland}}
\hypersetup{pdfcontactemail={nbeisert@itp.phys.ethz.ch}}
\hypersetup{pdfcontacturl={http://people.phys.ethz.ch/\xmptilde nbeisert/}}

\newcommand{\secref}[1]{\hyperref[#1]{section \ref*{#1}}}

\parskip1ex
\parindent0pt
\let\olditemize\itemize
\def\itemize{\olditemize\parskip0pt}

\begin{document}

\title{The \textsf{childdoc} Package}
\hypersetup{pdftitle={The childdoc Package}}
\author{Niklas Beisert\\[2ex]
  Institut f\"ur Theoretische Physik\\
  Eidgen\"ossische Technische Hochschule Z\"urich\\
  Wolfgang-Pauli-Strasse 27, 8093 Z\"urich, Switzerland\\[1ex]
  \href{mailto:nbeisert@itp.phys.ethz.ch}
  {\texttt{nbeisert@itp.phys.ethz.ch}}}
\hypersetup{pdfauthor={Niklas Beisert}}
\hypersetup{pdfsubject={Manual for the LaTeX2e Package childdoc}}
\date{30 December 2018, \textsf{v2.0}}
\maketitle

\begin{abstract}\noindent
\textsf{childdoc} is a \LaTeXe{} package
that enables the direct compilation
of document sections included by |\include|
to individual files.
\end{abstract}

\begingroup
\parskip0ex
\tableofcontents
\endgroup

%%%%%%%%%%%%%%%%%%%%%%%%%%%%%%%%%%%%%%%%%%%%%%%%%%%%%%%%%%%%%%%%%%%%%%%%%%%%%%%%
%%%%%%%%%%%%%%%%%%%%%%%%%%%%%%%%%%%%%%%%%%%%%%%%%%%%%%%%%%%%%%%%%%%%%%%%%%%%%%%%
\section{Introduction}

\LaTeX{} provides a mechanism to structure a large document (such as a book)
into a main file and several child files (containing the chapters)
using the |\include| command.
This mechanism is beneficial for documents
which span hundreds of pages in order to
make the source file(s) more manageable.
Moreover, compilation can be restricted to
selected child files by means of the |\includeonly| command.
The latter feature can be used to reduce the compilation time while editing
(this was significantly more useful in the earlier days of \LaTeX{})
or to generate a smaller document which is easier to navigate.
Another application of |\includeonly| is to generate
documents consisting of selected parts of the complete document.

However, there are a few drawbacks of the plain |\include| mechanism:
\begin{itemize}
\item
The child files cannot be compiled on their own,
they can only be compiled via the main file.
A naive editing environment
(such as a text editor with an option
to have the current file processed by \LaTeX)
may require one to switch to the main file before compiling;
attempting to compile the child file produces errors.
\item
The main file must be modified (each time)
to adjust the |\includeonly| command
to the present needs. This easily leaves the main file in a messy state.
\item
The generated document will always carry the filename
of the main document. This is inconvenient if
several child files are to be compiled and
to be kept for distribution.
\end{itemize}

The present package provides a simple interface
to make child files individually compilable by \LaTeX{}.
Compiling a child file then has the same effect as compiling
the main file with an |\includeonly| command
to select the appropriate child.
Moreover the generated document will carry the name of the child
rather than the main file.
This resolves all three above issues.

This feature is meant to make the editing of books,
thesis documents and lecture notes somewhat more convenient.
However, the package can also be used efficiently for
composing a series of documents (such as exercise sheets)
which are typically distributed individually.
It then assists the author in generating the individual documents
(potentially in different versions)
as well as a document containing the collected series.
Another application is in developing style files
or other kinds of included material
where compilation of the style file could redirect
to a sample or test file.

%%%%%%%%%%%%%%%%%%%%%%%%%%%%%%%%%%%%%%%%%%%%%%%%%%%%%%%%%%%%%%%%%%%%%%%%%%%%%%%%
%%%%%%%%%%%%%%%%%%%%%%%%%%%%%%%%%%%%%%%%%%%%%%%%%%%%%%%%%%%%%%%%%%%%%%%%%%%%%%%%
\section{Usage}

First of all, the package \textsf{childdoc} is \emph{not} a standard
\LaTeXe{} |.sty| style file! Therefore it needs to be invoked in
a non-standard way.

%%%%%%%%%%%%%%%%%%%%%%%%%%%%%%%%%%%%%%%%%%%%%%%%%%%%%%%%%%%%%%%%%%%%%%%%%%%%%%%%
\subsection{Included Files}
\label{sec:include}

%%%%%%%%%%%%%%%%%%%%%%%%%%%%%%%%%%%%%%%%
\DescribeMacro{\childdocmain}
To use the package, add the commands
\begin{center}
\begin{tabular}{l}
|\input{childdoc.def}|\\
|\childdocmain{}|\\
\end{tabular}
\end{center}
at the very top of the main \LaTeX{} file,
in particular \emph{before} the |\documentclass| statement!
The argument of |\childdocmain| should be left empty
(but it must be present).

%%%%%%%%%%%%%%%%%%%%%%%%%%%%%%%%%%%%%%%%
\DescribeMacro{\childdocof}
Furthermore, add the commands
\begin{center}
\begin{tabular}{l}
|\input{childdoc.def}|\\
|\childdocof{|\textit{main}|}|\\
\end{tabular}
\end{center}
at the top of every child file \textit{child}
which is included by |\include{|\textit{child}|}|
from within the main file
(or at least for those files to be compiled individually).
The argument \textit{main} must be the filename of the main file.

There are a couple of
considerations in setting up the main and child documents:

%%%%%%%%%%%%%%%%%%%%%%%%%%%%%%%%%%%%%%%%
\paragraph{Restrictions.}

Please note the following restrictions:
\begin{itemize}
\item
|\childdocmain| must be called with one argument \textit{main}
to ensure compatibility with earlier version of the package.
It must either be empty (|\childdocmain{}|)
or precisely match the filename of the main file in which it is specified.
See \secref{sec:detection} for further information.
\item
The filename \textit{main} must be specified without the |.tex| extension.
\item
The filename \textit{main} is case sensitive
(even in case-insensitive file systems)
due to internal string comparison.
\item
The argument \textit{main} should be fully expanded, it cannot be a macro.
\item
Subdirectories and special characters should be avoided in filenames.
\item
The command |\childdocmain{|\textit{main}|}| must be followed by a whitespace.
It should not be followed immediately by another command
or by a comment mark `|%|'.
This is because the \TeX{} parser reads the token immediately following
the argument of |\childdocmain| and puts it
at the beginning of every child section;
however, a white\-space is ignored.
\end{itemize}

%%%%%%%%%%%%%%%%%%%%%%%%%%%%%%%%%%%%%%%%
\paragraph{Content of Main File.}

It is advisable to place all content in the child files included by |\include|.
Any output contained in the main file will appear in all child documents
unless suppressed manually;
it cannot be suppressed automatically by the |\includeonly| directive
and thus should normally be avoided.
A method to include some content in the main file
by means of conditional processing is described in \secref{sec:conditional}.

%%%%%%%%%%%%%%%%%%%%%%%%%%%%%%%%%%%%%%%%
\paragraph{Page Numbering.}

When only a part of the document is compiled,
the appropriate numbering of pages
(as well as other status parameters)
is determined from the |.aux| files.
The latter contain information from previous passes.
However this information needs to propagate through
all intermediate child documents.
Therefore the page numbering in child documents may well
be inconsistent until the complete document is compiled at least once.

A useful (if unconventional) way to always ensure a consistent
page numbering is to restart the numbering in each child document
and denote the pages by `\textit{child}|.|\textit{page}'
where \textit{child} represents the chapter/section number of the child file.
This can be achieved by the command
|\numberwithin{page}{|\textit{child}|}|
of the \textsf{amsmath} package
where \textit{child} can be |chapter| or |section|
depending on the chosen structuring.
Alternatively, one can modify the macro |\thepage| appropriately
and reset the counter |page| at the start of each child file.

%%%%%%%%%%%%%%%%%%%%%%%%%%%%%%%%%%%%%%%%%%%%%%%%%%%%%%%%%%%%%%%%%%%%%%%%%%%%%%%%
\subsection{Conditional Processing}
\label{sec:conditional}

The package provides a mechanism to compile different versions
of a document. To customise the versions further some conditional processing
can come in handy to distinguish which version is being compiled.
The package provides two macros to describe the compilation context:

%%%%%%%%%%%%%%%%%%%%%%%%%%%%%%%%%%%%%%%%
\DescribeMacro{\ifchilddoc}
The conditional |\ifchilddoc| distinguishes between the compilation of
child documents and the main document:
%
\begin{center}
|\ifchilddoc |\textit{child-code}| |[|\||else |\textit{main-code}]| \||fi|
\end{center}

%%%%%%%%%%%%%%%%%%%%%%%%%%%%%%%%%%%%%%%%
\DescribeMacro{\childdocname}
\DescribeMacro{\childdocjob}
The macro |\childdocname| contains the filename (without extension)
of the main or child file being processed.
Note that |\childdocjob| will always contain the name of the main file.

%%%%%%%%%%%%%%%%%%%%%%%%%%%%%%%%%%%%%%%%
\paragraph{Title Page.}

Conditional processing can be used to include a title or banner page
in the main document when proper precautions are taken.
Importantly, the code in the main file should ensure that the page counter
(as well as other status parameters which are stored in the |.aux| files)
takes the same value after the conditional processing.
Otherwise the page numbers may take divergent values
depending on which part is compiled.

For example, a title page could be declared by:
%
\begin{center}
\begin{tabular}{l}
|\ifchilddoc\||else|\\
|\addtocounter{page}{-1}|\\
\textit{code for title page}\\
|\newpage|\\
|\||fi|
\end{tabular}
\end{center}
%
A banner page for the child documents can be generated by:
%
\begin{center}
\begin{tabular}{l}
|\ifchilddoc|\\
|\addtocounter{page}{-1}|\\
\textit{code for banner page}\\
|\newpage|\\
|\||fi|
\end{tabular}
\end{center}
%
Here one could write a message such as:
\begin{center}
|This is the part \childdocname{} of \childdocjob{}.|
\end{center}

%%%%%%%%%%%%%%%%%%%%%%%%%%%%%%%%%%%%%%%%%%%%%%%%%%%%%%%%%%%%%%%%%%%%%%%%%%%%%%%%
\subsection{Flags}
\label{sec:flags}

The package makes it easy to generate different versions
of the main or child documents.
To this end compilation flags can be defined
and assigned different default values.
They will be particularly useful in conjunction
with the forwarding mechanism described in \secref{sec:forward}.

For example, it may be useful to have a flag |\version|
which can be set to |draft| or |final|.
The document source will contain some conditional code
depending on the value of |\version|.
Suppose further, the flag should default to |final| for the main file
and to |draft| for child files
which is a natural assignment for editing the document.
This is achieved by placing the following code
in the preamble of the main document
(below the |\childdocmain| directive):
%
\begin{center}
\begin{tabular}{l}
|\ifchilddoc|\\
|\providecommand{\version}{draft}|\\
|\||else|\\
|\providecommand{\version}{final}|\\
|\||fi|
\end{tabular}
\end{center}
%
The definition by |\providecommand| makes sure
that previous definitions are not overwritten.
Further statements |\providecommand{\version}{...}|
can thus be added before the above code to override it.

For the main file, one might add a line
(between |\childdocmain| and the above block)
%
\begin{center}
|%\ifchilddoc\||else\providecommand{\version}{draft}\||fi|
\end{center}
%
which can be uncommented to produce a draft version.
Likewise one can add a line to the very top of a child file
(above the |\childdocof{|\textit{main}|}| directive)
%
\begin{center}
|%\providecommand{\version}{final}|
\end{center}
%
which can be uncommented to produce the final version of this child document.

%%%%%%%%%%%%%%%%%%%%%%%%%%%%%%%%%%%%%%%%%%%%%%%%%%%%%%%%%%%%%%%%%%%%%%%%%%%%%%%%
\subsection{Forwarding}
\label{sec:forward}

Different versions of the main or child documents
using compilation flags as described in \secref{sec:flags}
can be (permanently) stored in different files
for convenient compilation, viewing and distribution.
To this end, the package defines a command
to pass on compilation to a different file:

%%%%%%%%%%%%%%%%%%%%%%%%%%%%%%%%%%%%%%%%
\DescribeMacro{\childdocforward}
The command |\childdocforward| redirects processing to
another source file:
%
\begin{center}
\begin{tabular}{l}
|\input{childdoc.def}|\\
|\childdocforward[|\textit{main}|]{|\textit{dest}|}|\\
\end{tabular}
\end{center}
%
The argument \textit{dest} is the destination file
(without extension).
It should be the main file or one of the child files.
Note that further \textsf{childdoc} directives
such as |\childdocof| and |\childdocforward|
in the indicated file will be processed in this form.
The optional argument \textit{main}
passes on directly to the main file \textit{main}
while pretending to compile the child \textit{dest}.
This form behaves as if \textit{dest}
issues |\childdocof{|\textit{main}|}| right away,
and no further \textsf{childdoc} directives will be processed.

%%%%%%%%%%%%%%%%%%%%%%%%%%%%%%%%%%%%%%%%
\DescribeMacro{\...prefix}
In the alternative form |\childdocforwardprefix|,
%
\begin{center}
\begin{tabular}{l}
|\input{childdoc.def}|\\
|\childdocforwardprefix[|\textit{main}|]{|\textit{prefix}|}{|\textit{dest}|}|
\end{tabular}
\end{center}
%
the destination file is determined by a pattern
depending on the current file:
To make this work, the current file must be called
`{\textit{prefix}\hspace{0.2em}\textit{suffix}}'
with \textit{prefix} matching precisely the argument.
Processing is then passed on to the file
`{\textit{dest}\hspace{0.2em}\textit{suffix}}'.
Surely, the same effect is achieved by
directly specifying the
argument `{\textit{dest}\hspace{0.2em}\textit{suffix}}'
in the first form.
However, that requires to set up a different file
for each child. With the alternative form of the command
all these files can have exactly the same content
which simplifies setting them up and maintaining them.

For example, the following file |draft.tex|
with a compilation flag |\version| as described in \secref{sec:flags}
compiles the main document as a draft:
%
\begin{center}
\begin{tabular}{l}
|\def\version{draft}|\\
|\input{childdoc.def}|\\
|\childdocforward{|\textit{main}|}|
\end{tabular}
\end{center}
%
Likewise, the following files |final|\textit{nn}|.tex|
compile the final version of the child document
|child|\textit{nn}|.tex|:
%
\begin{center}
\begin{tabular}{l}
|\def\version{final}|\\
|\input{childdoc.def}|\\
|\childdocforwardprefix{final}{child}|
\end{tabular}
\end{center}
%

Note that when several versions of a main file and/or of each child file
are to be generated, it may be convenient to set up a |Makefile| or
shell script to automatise the process.

%%%%%%%%%%%%%%%%%%%%%%%%%%%%%%%%%%%%%%%%%%%%%%%%%%%%%%%%%%%%%%%%%%%%%%%%%%%%%%%%
\subsection{Command Line Processing}
\label{sec:commandline}

The effect of redirection files can also be achieved by invoking
the \LaTeX{} compiler with a more elaborate command line.
Most conveniently this should be done as part
of a shell script or a |Makefile|.

When using \textsf{childdoc} in the main file, the following
command lines effectively perform a redirection
(note that depending on the shell being used,
backslashes may have to be doubled: `|\|' $\to$ `|\\|'):
%
\begin{center}
|... -jobname "|\textit{target}|" |\\|"|[\textit{flags}]%
|\input{childdoc.def}\childdocforward[|\textit{main}|]{|\textit{dest}|}"|
\end{center}
%
Here \textit{target} is the name of the output file,
\textit{main} is the name of the main file
and \textit{dest} is the name of the main or child file to be processed
(all filenames without extensions).
The optional argument \textit{main} can be omitted
if \textit{main} matches \textit{dest}.
Optionally, compilation \textit{flags} can be defined via |\def| commands.
This command line makes the \TeX{} engine believe
it is compiling the file \textit{target}
whose content is specified as the latter parameter.
The provided code then forwards the processing to
\textit{main} or \textit{dest} as described in \secref{sec:forward}.

%%%%%%%%%%%%%%%%%%%%%%%%%%%%%%%%%%%%%%%%%%%%%%%%%%%%%%%%%%%%%%%%%%%%%%%%%%%%%%%%
\subsection{Include by Input}
\label{sec:input}

Including child documents by |\include| has some restrictions by design.
Most notably, the content of a child document always occupies
its own set of pages; pages cannot be shared between child documents.
Usually, this behaviour makes perfect sense
because each child document contain an essential part of the document.
However, in some situations it may be desirable to compose
a document from a collection of parts
without having mandatory page breaks between then.
For this case, the package
provides a mechanism to include parts
by |\input| which can also be processed individually.
However, by construction this mechanism
requires manual handling of the content to be output.

%%%%%%%%%%%%%%%%%%%%%%%%%%%%%%%%%%%%%%%%
\DescribeMacro{\ifchilddocmanual}
The main file should be prepared as usual, see \secref{sec:include}.
However, the document body must make a distinction
between processing of an individual part and of the main document, e.g.:
%
\begin{center}
\begin{tabular}{l}
|\ifchilddocmanual|\\
|\input{\childdocname}|\\
|\||else|\\
\textit{document body with }|\input{|\textit{part}|}|\\
|\||fi|
\end{tabular}
\end{center}
%
The conditional |\ifchilddocmanual| is true whenever
a part to be included by |\input| is being compiled,
and the name of the part is stored in |\childdocname|.

%%%%%%%%%%%%%%%%%%%%%%%%%%%%%%%%%%%%%%%%
\DescribeMacro{\childdocby}
Each part to be included by |\input| should start with:
%
\begin{center}
\begin{tabular}{l}
|\input{childdoc.def}|\\
|\childdocby{|\textit{main}|}|\\
\end{tabular}
\end{center}
%
The directive |\childdocby| is similar to |\childdocof|
described in \secref{sec:include},
but the subsequent selection of content must be done manually.
To that end, both |\ifchilddoc| and |\ifchilddocmanual|
will be true upon processing of a part,
and the name of the part is stored in |\childdocname|.
Note that |\jobname| will be set to the filename of the current part
so that each part receives an individual |.aux| file
that does not interfere with the |.aux| file(s) of the main document.
This behaviour can be altered by the alternative form
|\childdocby[*]{|\textit{main}|}| (with a non-empty optional argument)
which uses the |.aux| file of the main document
by setting |\jobname| to \textit{main}.

%%%%%%%%%%%%%%%%%%%%%%%%%%%%%%%%%%%%%%%%%%%%%%%%%%%%%%%%%%%%%%%%%%%%%%%%%%%%%%%%
\subsection{Driver Development}
\label{sec:driver}

The \textsf{childdoc} mechanism can also be use for the development
of definition files such as \LaTeX{} styles or classes.
This case differs from the above setup with multiple parts
included by |\include| in that no |\includeonly| should be invoked.
This can be achieved by starting the include file
(before |\ProvidesPackage|) with:
%
\begin{center}
\begin{tabular}{l}
|\input{childdoc.def}|\\
|\childdocforward{|\textit{main}|}|\\
\end{tabular}
\end{center}
%
or alternatively with:
%
\begin{center}
\begin{tabular}{l}
|\input{childdoc.def}|\\
|\childdocby{|\textit{main}|}|\\
\end{tabular}
\end{center}
%
Both forms have slightly different effects as described above.
The main file is prepared as usual, see \secref{sec:include}.

%%%%%%%%%%%%%%%%%%%%%%%%%%%%%%%%%%%%%%%%%%%%%%%%%%%%%%%%%%%%%%%%%%%%%%%%%%%%%%%%
\subsection{Legacy Detection}
\label{sec:detection}

The directive |\childdocmain| in the main file can detect
whether the complete document or merely a child is to be compiled
even without using the directive |\childdocof|.
This method is deprecated because it is less robust
and there is no compelling reason to use it;
it is merely provided for backward compatibility
and it may be removed in future versions.

If the detection mechanism is to be used,
it is mandatory to correctly specify
the filename of the main file as the argument of |\childdocmain|:
%
\begin{center}
\begin{tabular}{l}
|\input{childdoc.def}|\\
|\childdocmain{|\textit{main}|}|\\
\end{tabular}
\end{center}
%
If |\jobname| does not match the argument \textit{main} of |\childdocmain|,
it is assumed that |\jobname| points to the child file to be compiled.
When using |\childdocmain| with the main file specified as argument,
it suffices to start a child file
with just |\input{|\textit{main}|}|
without loading of the package and using |\childdocof|.
If instead all processing is done
with the appropriate \textsf{childdoc} directives,
the argument of \textit{main} of |\childdocmain| can be empty.

An alternative version of the command line processing described
in \secref{sec:commandline} using the detection mechanism reads:
%
\begin{center}
|... -jobname "|\textit{target}|" "|[\textit{flags}]%
[|\def\jobname{|\textit{dest}|}|]|\input{|\textit{main}|}"|
\end{center}

%%%%%%%%%%%%%%%%%%%%%%%%%%%%%%%%%%%%%%%%%%%%%%%%%%%%%%%%%%%%%%%%%%%%%%%%%%%%%%%%
\subsection{Manual Code}
\label{sec:manual}

In case one cannot be certain whether the definitions file |childdoc.def|
is installed on the target \TeX{} distribution
and one prefers not to ship it,
it is conceivable to paste a few relevant commands into the sources.

To that end, drop all statements |\input{childdoc.def}|
and perform the replacements as outlined below.
Instead of |\childdocmain{|\textit{main}|}| add the following code
to the top of the main file:
%
\begin{center}
\begin{tabular}{l}
|\||ifdefined\childdocname\endinput\||fi\newif\ifchilddoc|\\
|\edef\childdocname{\scantokens\expandafter{\jobname\noexpand}}|\\
|\def\childdocmain{|\textit{main}|}\||ifx\childdocmain\childdocname\||else|\\
|\childdoctrue\includeonly{\childdocname}\let\jobname\childdocmain\||fi|\\
\end{tabular}
\end{center}
%
Instead of |\childdocof{|\textit{main}|}| just include the main file
at the top of each child file:
%
\begin{center}
|\input{|\textit{main}|}|
\end{center}
%
A simple redirection |\childdocforward{|\textit{dest}|}| is achieved by:
%
\begin{center}
|\def\jobname{|\textit{dest}|}\input{\jobname}|
\end{center}
%
The redirection with prefix
|\childdocforwardprefix[|\textit{prefix}|]{|\textit{dest}|}|
is accomplished by:
%
\begin{center}
\begin{tabular}{l}
|{\edef\jobname{\scantokens\expandafter{\jobname\noexpand}}|\\
|\def\redirectjob |\textit{prefix}|#1~~~{\gdef\jobname{|\textit{dest}|#1}}|\\
|\expandafter\redirectjob\jobname~~~}\input{\jobname}|
\end{tabular}
\end{center}

In an alternative approach,
child documents can be compiled by a specific command line
without additional code or specific definitions:
%
\begin{center}
|... -jobname "|\textit{target}|" "|[\textit{flags}]%
|\includeonly{|\textit{dest}|}\input{|\textit{main}|}"|
\end{center}
%

%%%%%%%%%%%%%%%%%%%%%%%%%%%%%%%%%%%%%%%%%%%%%%%%%%%%%%%%%%%%%%%%%%%%%%%%%%%%%%%%
%%%%%%%%%%%%%%%%%%%%%%%%%%%%%%%%%%%%%%%%%%%%%%%%%%%%%%%%%%%%%%%%%%%%%%%%%%%%%%%%
\section{Information}

%%%%%%%%%%%%%%%%%%%%%%%%%%%%%%%%%%%%%%%%%%%%%%%%%%%%%%%%%%%%%%%%%%%%%%%%%%%%%%%%
\subsection{Copyright}

Copyright \copyright{} 2017--2018 Niklas Beisert

This work may be distributed and/or modified under the
conditions of the \LaTeX{} Project Public License, either version 1.3
of this license or (at your option) any later version.
The latest version of this license is in
  \url{http://www.latex-project.org/lppl.txt}
and version 1.3 or later is part of all distributions of \LaTeX{}
version 2005/12/01 or later.

This work has the LPPL maintenance status `maintained'.

The Current Maintainer of this work is Niklas Beisert.

This work consists of the files |README.txt|, |childdoc.ins| and |childdoc.dtx|
as well as the derived files |childdoc.def|, |cdocsamp.tex|
with |cdocsch1.tex|, |cdocsch2.tex|, |cdocspt3.tex|, |cdocspt4.tex|,
|cdocsdrf.tex|, |cdocsfn1.tex|, |cdocsfn2.tex|
as well as |childdoc.pdf|.

%%%%%%%%%%%%%%%%%%%%%%%%%%%%%%%%%%%%%%%%%%%%%%%%%%%%%%%%%%%%%%%%%%%%%%%%%%%%%%%%
\subsection{Files and Installation}

The package consists of the files:
%
\begin{center}
\begin{tabular}{ll}
    |README.txt|   & readme file \\
    |childdoc.ins| & installation file \\
    |childdoc.dtx| & source file \\
    |childdoc.def| & definition file \\
    |cdocsamp.tex| & sample main file \\
    |cdocsch1.tex| & sample include file \\
    |cdocsch2.tex| & sample include file \\
    |cdocspt3.tex| & sample part file \\
    |cdocspt4.tex| & sample part file \\
    |cdocsdrf.tex| & sample redirection file \\
    |cdocsfn1.tex| & sample redirection file \\
    |cdocsfn2.tex| & sample redirection file \\
    |childdoc.pdf| & manual
\end{tabular}
\end{center}
%
The distribution consists of the files
|README.txt|, |childdoc.ins| and |childdoc.dtx|.
%
\begin{itemize}
\item
Run (pdf)\LaTeX{} on |childdoc.dtx|
to compile the manual |childdoc.pdf| (this file).
\item
Run \LaTeX{} on |childdoc.ins| to create the definitions file |childdoc.def|
and the sample |cdocsamp.tex| with include files
|cdocsch1.tex|, |cdocsch2.tex|, |cdocspt3.tex|, |cdocspt4.tex|,
|cdocsdrf.tex|, |cdocsfn1.tex|, |cdocsfn2.tex|.
Then copy the file |childdoc.def| to an appropriate directory of your \LaTeX{}
distribution, e.g.\ \textit{texmf-root}|/tex/latex/childdoc|.
\end{itemize}

%%%%%%%%%%%%%%%%%%%%%%%%%%%%%%%%%%%%%%%%%%%%%%%%%%%%%%%%%%%%%%%%%%%%%%%%%%%%%%%%
\subsection{Related CTAN Packages}

There are several other packages which offer a similar functionality:
%
\begin{itemize}
\item
The packages
\href{http://ctan.org/pkg/docmute}{\textsf{docmute}},
\href{http://ctan.org/pkg/includex}{\textsf{includex}} and
\href{http://ctan.org/pkg/standalone}{\textsf{standalone}}
provide commands to include only the document body of
a child file thus allowing both files to be compiled individually.
\item
The packages \href{http://ctan.org/pkg/subdocs}{\textsf{subdocs}}
and \href{http://ctan.org/pkg/subfiles}{\textsf{subfiles}}
provide structures in which the main and child documents can be
encapsulated and allowing them to be compiled individually.
The inclusion mechanism is different from the conventional |\include|.
\item
The package \href{http://ctan.org/pkg/combine}{\textsf{combine}}
is an elaborate solution to combine several documents into one.
\end{itemize}
%
See also the CTAN topic \href{http://ctan.org/topic/subdocs}{\textsf{subdocs}}
for further related packages.
The present package differs from the above solutions in that
a document structure constructed with the conventional |\include| mechanism
just needs two extra commands at the top of every file
such that all constituent files can be compiled individually.

%%%%%%%%%%%%%%%%%%%%%%%%%%%%%%%%%%%%%%%%%%%%%%%%%%%%%%%%%%%%%%%%%%%%%%%%%%%%%%%%
%\subsection{Feature Suggestions}
%
%The following is a list of features which may be useful for future
%versions of this package:
%%
%\begin{itemize}
%\item
%\ldots
%\end{itemize}

%%%%%%%%%%%%%%%%%%%%%%%%%%%%%%%%%%%%%%%%%%%%%%%%%%%%%%%%%%%%%%%%%%%%%%%%%%%%%%%%
\subsection{Revision History}

%%%%%%%%%%%%%%%%%%%%%%%%%%%%%%%%%%%%%%%%
\paragraph{v2.0:} 2018/12/30

\begin{itemize}
\item
immediate forward processing
\item
added |\childdocby| mechanism
\item
manual restructured
\end{itemize}

%%%%%%%%%%%%%%%%%%%%%%%%%%%%%%%%%%%%%%%%
\paragraph{v1.6:} 2018/01/17

\begin{itemize}
\item
application for development of include files
\item
corrections to manual
\end{itemize}

%%%%%%%%%%%%%%%%%%%%%%%%%%%%%%%%%%%%%%%%
\paragraph{v1.5:} 2017/05/21

\begin{itemize}
\item
more complete structuring introduced
\item
|\childdocof| introduced
\item
|\childdoc| renamed to |\childdocmain|
\item
|\childredirect| renamed to |\childdocforward| and |\childdocforwardprefix|
and functionality expanded
\end{itemize}

%%%%%%%%%%%%%%%%%%%%%%%%%%%%%%%%%%%%%%%%
\paragraph{v1.0:} 2017/04/27

\begin{itemize}
\item
manual and install package
\item
first version published on CTAN
\end{itemize}

%%%%%%%%%%%%%%%%%%%%%%%%%%%%%%%%%%%%%%%%
\paragraph{v0.6:} 2017/04/26

\begin{itemize}
\item
redirection mechanism added
\end{itemize}

%%%%%%%%%%%%%%%%%%%%%%%%%%%%%%%%%%%%%%%%
\paragraph{v0.5:} 2017/04/26

\begin{itemize}
\item
functionality in definition file
\end{itemize}


%%%%%%%%%%%%%%%%%%%%%%%%%%%%%%%%%%%%%%%%%%%%%%%%%%%%%%%%%%%%%%%%%%%%%%%%%%%%%%%%
%%%%%%%%%%%%%%%%%%%%%%%%%%%%%%%%%%%%%%%%%%%%%%%%%%%%%%%%%%%%%%%%%%%%%%%%%%%%%%%%
%%%%%%%%%%%%%%%%%%%%%%%%%%%%%%%%%%%%%%%%%%%%%%%%%%%%%%%%%%%%%%%%%%%%%%%%%%%%%%%%
\appendix

\settowidth\MacroIndent{\rmfamily\scriptsize 000\ }

 \DocInput{childdoc.dtx}

\end{document}
%</driver>
% \fi
%
% %%%%%%%%%%%%%%%%%%%%%%%%%%%%%%%%%%%%%%%%%%%%%%%%%%%%%%%%%%%%%%%%%%%%%%%%%%%%%%
% %%%%%%%%%%%%%%%%%%%%%%%%%%%%%%%%%%%%%%%%%%%%%%%%%%%%%%%%%%%%%%%%%%%%%%%%%%%%%%
% \section{Sample}
%\iffalse
%<*samplemain>
%\fi
%
% The following presents a sample document
% with two chapters, two parts, a title page,
% a compile flag as well as three forwarding files to set the flag.
% It consists of eight |.tex| files:
% \begin{center}
% \begin{tabular}{ll}
% |cdocsamp.tex|&main file\\
% |cdocsch1.tex|&include file for chapter 1\\
% |cdocsch2.tex|&include file for chapter 2\\
% |cdocspt3.tex|&include file for part 3\\
% |cdocspt4.tex|&include file for part 4\\
% |cdocsdrf.tex|&forwarding file for main file in draft mode\\
% |cdocsfi1.tex|&forwarding file for final version of chapter 1\\
% |cdocsfi2.tex|&forwarding file for final version of chapter 2\\
% \end{tabular}
% \end{center}
% Each of the eight files can be compiled directly by the \LaTeX{} compiler.
%
% %%%%%%%%%%%%%%%%%%%%%%%%%%%%%%%%%%%%%%
% \paragraph{Main File.}
%
% The main file is called |cdocsamp.tex|.
%
% Load the \textsf{childdoc} definitions and
% declare the filename for the main document:
%    \begin{macrocode}
\input{childdoc.def}
\childdocmain{}
%    \end{macrocode}

% Optional override for |\version| flag:
%    \begin{macrocode}
%%\ifchilddoc\else\providecommand{\version}{draft}\fi
%    \end{macrocode}

% Define the default values for the |\version| flag
% (|final| for the main file and |draft| for childs):
%    \begin{macrocode}
\ifchilddoc
\providecommand{\version}{draft}
\else
\providecommand{\version}{final}
\fi
%    \end{macrocode}

% Load the standard document class:
%    \begin{macrocode}
\documentclass[12pt]{article}
%    \end{macrocode}

% Start the document body:
%    \begin{macrocode}
\begin{document}
%    \end{macrocode}

% Declare a title page.
% Print title, part of document being processed and version flag:
%    \begin{macrocode}
\addtocounter{page}{-1}
\begin{center}
{\LARGE\bfseries{}childdoc example\par}
\vspace{1cm}
\ifchilddoc
\ifchilddocmanual part\else chapter\fi:
`\childdocname' of `\childdocjob'\par
\else
main document: `\childdocjob'\par
\fi
version: \version\par
\end{center}
\newpage
%    \end{macrocode}

% Manually include selected file,
% otherwise process as usual:
%    \begin{macrocode}
\ifchilddocmanual
\section*{part `\childdocname'}
\input{\childdocname}
\else
%    \end{macrocode}

% Include the two chapters:
%    \begin{macrocode}
\include{cdocsch1}
\include{cdocsch2}
%    \end{macrocode}

% Include the two parts unless only chapters should be displayed:
%    \begin{macrocode}
\ifchilddoc\else
\section{part three}
\input{cdocspt3}
\section{part four}
\input{cdocspt4}
\fi
%    \end{macrocode}

% Process as usual until here:
%    \begin{macrocode}
\fi
%    \end{macrocode}

% End of document body:
%    \begin{macrocode}
\end{document}
%    \end{macrocode}
%\iffalse
%</samplemain>
%\fi
%
% %%%%%%%%%%%%%%%%%%%%%%%%%%%%%%%%%%%%%%
% \paragraph{Chapter Include Files.}
%
% The include files are called |cdocsch1.tex| and |cdocsch2.tex|.
%
%\iffalse
%<*samplechap1|samplechap2>
%\fi

% Optional override for |\version| flag:
%    \begin{macrocode}
%%\providecommand{\version}{final}
%    \end{macrocode}

% Include the main document:
%    \begin{macrocode}
\input{childdoc.def}
\childdocof{cdocsamp}
%    \end{macrocode}

%\iffalse
%</samplechap1|samplechap2>
%\fi
%
%\iffalse
%<*samplechap1>
%\fi
% Some text for chapter 1:
%    \begin{macrocode}
\section{one}
some text in chapter one
%    \end{macrocode}

%\iffalse
%</samplechap1>
%\fi
% Some text for chapter 2:
%\iffalse
%<*samplechap2>
%\fi
%    \begin{macrocode}
\section{two}
more text in chapter two
%    \end{macrocode}

%\iffalse
%</samplechap2>
%\fi
%
% %%%%%%%%%%%%%%%%%%%%%%%%%%%%%%%%%%%%%%
% \paragraph{Part Include Files.}
%
% The include files are called |cdocspt3.tex| and |cdocspt4.tex|.
%
%\iffalse
%<*samplepart3|samplepart4>
%\fi

% Optional override for |\version| flag:
%    \begin{macrocode}
%%\providecommand{\version}{final}
%    \end{macrocode}

% Include the main document:
%    \begin{macrocode}
\input{childdoc.def}
\childdocby{cdocsamp}
%    \end{macrocode}

%\iffalse
%</samplepart3|samplepart4>
%\fi
%
%\iffalse
%<*samplepart3>
%\fi
% Some text for part 3:
%    \begin{macrocode}
some text in part three
%    \end{macrocode}

%\iffalse
%</samplepart3>
%\fi
% Some text for part 4:
%\iffalse
%<*samplepart4>
%\fi
%    \begin{macrocode}
more text in part four
%    \end{macrocode}

%\iffalse
%</samplepart4>
%\fi
%
% %%%%%%%%%%%%%%%%%%%%%%%%%%%%%%%%%%%%%%
% \paragraph{Forwarding for a Complete Draft.}
%
% The following forwarding file |cdocsdrf.tex|
% compiles the main document in draft mode:
%\iffalse
%<*sampledraft>
%\fi
%    \begin{macrocode}
\def\version{draft}
\input{childdoc.def}
\childdocforward{cdocsamp}
%    \end{macrocode}

%\iffalse
%</sampledraft>
%\fi
%
% %%%%%%%%%%%%%%%%%%%%%%%%%%%%%%%%%%%%%%
% \paragraph{Forwarding for Final Version of the Chapters.}
%
% The following forwarding files |cdocsfn1.tex| and |cdocsfn2.tex|
% (with identical content)
% compile the final versions of the child documents
% |cdocsch1.tex| and |cdocsch2.tex|, respectively:
%\iffalse
%<*samplefinal>
%\fi
%    \begin{macrocode}
\def\version{final}
\input{childdoc.def}
\childdocforwardprefix[cdocsamp]{cdocsfn}{cdocsch}
%    \end{macrocode}

%\iffalse
%</samplefinal>
%\fi
%
% %%%%%%%%%%%%%%%%%%%%%%%%%%%%%%%%%%%%%%
% \paragraph{Command Line Processing.}
%
% The following three command lines generate the output files
% |cdocscld|, |cdocscl1| and |cdocscl2|
% which should be identical to
% |cdocsdrf|, |cdocsch1| and |cdocsfn2|, respectively:
% \begin{center}
% \begin{tabular}{l}
% |latex -jobname cdocscld \|\\
% |  "\def\version{draft}\input{childdoc.def}\childdocforward{cdocsamp}"|\\
% |latex -jobname cdocscl1 \|\\
% |  "\input{childdoc.def}\childdocforward[cdocsamp]{cdocsch1}"|\\
% |latex -jobname cdocscl2 \|\\
% |  "\def\version{final}\input{childdoc.def}\childdocforward{cdocsch2}"|
% \end{tabular}
% \end{center}
% Note that the trailing backslash on each first line
% merely continues the input to the second line
% (for convenient cut ant paste).
% Furthermore, the command |latex| can be replaced by any
% of its alternative versions such as |pdflatex|.
%
% %%%%%%%%%%%%%%%%%%%%%%%%%%%%%%%%%%%%%%%%%%%%%%%%%%%%%%%%%%%%%%%%%%%%%%%%%%%%%%
% %%%%%%%%%%%%%%%%%%%%%%%%%%%%%%%%%%%%%%%%%%%%%%%%%%%%%%%%%%%%%%%%%%%%%%%%%%%%%%
% \section{Implementation}
%\iffalse
%<*package>
%\fi
%
% This section describes the definitions file |childdoc.def|.

% The definitions cannot be loaded using |\usepackage| or |\RequirePackage|
% which has a mechanism to prevent loading a style file more than once.
% When loading the definitions by means of |\input|
% multiple instances have to be prevented manually:
%\iffalse
%This code needs to be before the `\ProvidesFile' directive
%which is defined at the beginning of this file.
%Therefore it is also placed there and commented out here.
%</package>
%<*discard>
%\fi
%    \begin{macrocode}
\ifdefined\childdocmain\endinput\fi
%    \end{macrocode}
%\iffalse
%</discard>
%<*package>
%\fi
%
% \macro{\ifchilddoc}
% \macro{\ifchilddocmanual}
% The conditional |\ifchilddoc| tells whether a
% child (true) or main (false) document is being compiled.
% The conditional |\ifchilddocmanual| tells whether
% the |\includeonly| mechanism is used (false) or
% the selection of child files must be performed manually (true).
% The definitions initialise to false:
%    \begin{macrocode}
\newif\ifchilddoc
\newif\ifchilddocmanual
%    \end{macrocode}

% \macro{\childdocname}
% \macro{\childdocjob}
% The macro |\childdocname| stores the name of the main document
% to be compiled. The macro |\childdocjob| stores the name of
% the document on which the \LaTeX{} compiler was originally invoked.
% The content of |\jobname| cannot be compared
% to filenames specified in the source due to different catcodes.
% The following code rescans |\jobname|, stores the result
% in |\childdocname| and saves a copy in |\childdocjob|:
%    \begin{macrocode}
\edef\childdocname{\scantokens\expandafter{\jobname\noexpand}}
\let\childdocjob\childdocname
%    \end{macrocode}

% \macro{\childdocdisable}
% The macro |\childdocdisable| prevents the main file
% from being processed more than once.
% At this stage, the main document command |\childdocmain|
% is assumed to be called once again where it should do nothing.
% Any subsequent call to it should prevent
% a secondary processing of the main document
% It overwrites the forwarding commands
% |\childdocof| and |\childdocforward|
% with empty macros to prevent further inclusions of the main document:
%    \begin{macrocode}
\newcommand{\childdocdisable}
{
  \renewcommand{\childdocmain}[1]{\renewcommand{\childdocmain}[1]{\endinput}}
  \renewcommand{\childdocof}[1]{}
  \renewcommand{\childdocby}[2][]{}
  \renewcommand{\childdocforward}[2][]{}
  \renewcommand{\childdocdisable}{}
}
%    \end{macrocode}

% \macro{\childdocmain}
% The macro |\childdocmain| is to be called at the top of the main file
% with nothing or the main filename (without extension) as argument.
% First, it breaks loops.
% If the argument is not empty and does not match |\childdocname|
% (which is set by the first inclusion of |childdoc.def|),
% |\ifchilddoc| is set to true, |\includeonly| is applied to the child file
% and |\jobname| is set to the main file
% (for proper handling of |.aux| files):
%    \begin{macrocode}
\newcommand{\childdocmain}[1]
{
  \childdocdisable\childdocmain{}
  \if?#1?\else
    \begingroup
      \def\childdoctmp{#1}
      \ifx\childdoctmp\childdocname
        \def\childdoctmp{}
      \else
        \def\childdoctmp
        {
          \childdoctrue
          \includeonly{\childdocname}
          \def\childdocjob{#1}
          \def\jobname{#1}
        }
      \fi
      \expandafter
    \endgroup
    \childdoctmp
  \fi
}
%    \end{macrocode}

% \macro{\childdocof}
% The command |\childdocof| redirects
% compilation to the main file |#1|.
%    \begin{macrocode}
\newcommand{\childdocof}[1]
{
  \childdocdisable
  \childdoctrue
  \includeonly{\childdocname}
  \def\jobname{#1}
  \def\childdocjob{#1}
  \input{#1}
}
%    \end{macrocode}

% \macro{\childdocby}
% The command |\childdocby| ....
%    \begin{macrocode}
\newcommand{\childdocby}[2][]
{
  \childdocdisable
  \childdoctrue
  \childdocmanualtrue
  \if?#1?\else
    \def\jobname{#2}
  \fi
  \def\childdocjob{#2}
  \input{#2}
  \endinput
}
%    \end{macrocode}

% \macro{\childdocforward}
% The command |\childdocforward| redirects
% compilation to the main file or
% (if the optional argument is given) a child file.
% Parameters are set as if the main file
% or a child file starting with |\childdocof| was compiled.
% Then compilation is handed over to the main file:
%    \begin{macrocode}
\newcommand{\childdocforward}[2][]
{
  \begingroup
    \if?#1?
      \def\childdoctmp
      {
        \def\childdocname{#2}
        \def\childdocjob{#2}
        \def\jobname{#2}
        \input{#2}
        \endinput
      }
    \else
      \def\childdoctmp
      {
        \childdocdisable
        \def\childdocname{#2}
        \childdoctrue
        \includeonly{#2}
        \def\childdocjob{#1}
        \def\jobname{#1}
        \input{#1}
        \endinput
      }
    \fi
    \expandafter
  \endgroup
  \childdoctmp
}
%    \end{macrocode}

% \macro{\childdocforwardprefix}
% The command |\childdocforwardprefix| redirects
% compilation to the main or a child file by means of a pattern.
% The prefix |#1| in the current filename is replaced by |#2|
% and the suffix of the current filename is kept
% (it is assumed that the filename does not contain the substring `|~~~|'
% which is used as a delimiter).
% Compilation is handed over to the new file by |\childdocforward|:
%    \begin{macrocode}
\newcommand{\childdocforwardprefix}[3][]
{
  \begingroup
    \def\childdocextract #2##1~~~{\def\childdoctmp{\childdocforward[#1]{#3##1}}}
    \expandafter\childdocextract\childdocname~~~
    \expandafter
  \endgroup
  \childdoctmp
}
%    \end{macrocode}

% \macro{\childdoc}
% The deprecated macro |\childdoc| is a legacy version of |\childdocmain|:
%    \begin{macrocode}
\newcommand{\childdoc}{\childdocmain}
%    \end{macrocode}

% \macro{\childdocredirect}
% The deprecated macro |\childdocredirect| is a legacy version
% of |\childdocforward| and |\childdocforwardprefix|:
%    \begin{macrocode}
\newcommand{\childdocredirect}[2][]
{
  \begingroup
    \if?#1?
      \def\childdoctmp{\childdocforward{#2}}
    \else
      \def\childdoctmp{\childdocforwardprefix{#1}{#2}}
    \fi
    \expandafter
  \endgroup
  \childdoctmp
}
%    \end{macrocode}

%\iffalse
%</package>
%\fi
%
\endinput
|\\
|\childdocby{|\textit{main}|}|\\
\end{tabular}
\end{center}
%
The directive |\childdocby| is similar to |\childdocof|
described in \secref{sec:include},
but the subsequent selection of content must be done manually.
To that end, both |\ifchilddoc| and |\ifchilddocmanual|
will be true upon processing of a part,
and the name of the part is stored in |\childdocname|.
Note that |\jobname| will be set to the filename of the current part
so that each part receives an individual |.aux| file
that does not interfere with the |.aux| file(s) of the main document.
This behaviour can be altered by the alternative form
|\childdocby[*]{|\textit{main}|}| (with a non-empty optional argument)
which uses the |.aux| file of the main document
by setting |\jobname| to \textit{main}.

%%%%%%%%%%%%%%%%%%%%%%%%%%%%%%%%%%%%%%%%%%%%%%%%%%%%%%%%%%%%%%%%%%%%%%%%%%%%%%%%
\subsection{Driver Development}
\label{sec:driver}

The \textsf{childdoc} mechanism can also be use for the development
of definition files such as \LaTeX{} styles or classes.
This case differs from the above setup with multiple parts
included by |\include| in that no |\includeonly| should be invoked.
This can be achieved by starting the include file
(before |\ProvidesPackage|) with:
%
\begin{center}
\begin{tabular}{l}
|% \iffalse
%
% childdoc.dtx Copyright (C) 2017-2018 Niklas Beisert
%
% This work may be distributed and/or modified under the
% conditions of the LaTeX Project Public License, either version 1.3
% of this license or (at your option) any later version.
% The latest version of this license is in
%   http://www.latex-project.org/lppl.txt
% and version 1.3 or later is part of all distributions of LaTeX
% version 2005/12/01 or later.
%
% This work has the LPPL maintenance status `maintained'.
%
% The Current Maintainer of this work is Niklas Beisert.
%
% This work consists of the files childdoc.dtx and childdoc.ins
% and the derived files childdoc.def and cdocsamp.tex with
% cdocsch1.tex, cdocsch2.tex, cdocsdrf.tex, cdocsfn1.tex, cdocsfn2.tex.
%
%<package>\ifdefined\childdocmain\endinput\fi
%<package>\ProvidesFile{childdoc.def}[2018/12/30 v2.0 child document driver]
%<samplemain>\ProvidesFile{cdocsamp.tex}[2018/12/30 v2.0 sample for childdoc]
%<*driver>
%\ProvidesFile{childdoc.drv}[2018/12/30 v2.0 childdoc reference manual file]
\PassOptionsToClass{10pt,a4paper}{article}
\documentclass{ltxdoc}

\usepackage[margin=35mm]{geometry}
\usepackage{hyperref}
\usepackage{hyperxmp}
\usepackage[usenames]{color}

\hypersetup{colorlinks=true}
\hypersetup{pdfstartview=FitH}
\hypersetup{pdfpagemode=UseNone}
\hypersetup{pdfsource={}}
\hypersetup{pdflang={en-UK}}
\hypersetup{pdfcopyright={Copyright 2017-2018 Niklas Beisert.
  This work may be distributed and/or modified under the
  conditions of the LaTeX Project Public License, either version 1.3
  of this license or (at your option) any later version.}}
\hypersetup{pdflicenseurl={http://www.latex-project.org/lppl.txt}}
\hypersetup{pdfcontactaddress={ETH Zurich, ITP, HIT K,
  Wolfgang-Pauli-Strasse 27}}
\hypersetup{pdfcontactpostcode={8093}}
\hypersetup{pdfcontactcity={Zurich}}
\hypersetup{pdfcontactcountry={Switzerland}}
\hypersetup{pdfcontactemail={nbeisert@itp.phys.ethz.ch}}
\hypersetup{pdfcontacturl={http://people.phys.ethz.ch/\xmptilde nbeisert/}}

\newcommand{\secref}[1]{\hyperref[#1]{section \ref*{#1}}}

\parskip1ex
\parindent0pt
\let\olditemize\itemize
\def\itemize{\olditemize\parskip0pt}

\begin{document}

\title{The \textsf{childdoc} Package}
\hypersetup{pdftitle={The childdoc Package}}
\author{Niklas Beisert\\[2ex]
  Institut f\"ur Theoretische Physik\\
  Eidgen\"ossische Technische Hochschule Z\"urich\\
  Wolfgang-Pauli-Strasse 27, 8093 Z\"urich, Switzerland\\[1ex]
  \href{mailto:nbeisert@itp.phys.ethz.ch}
  {\texttt{nbeisert@itp.phys.ethz.ch}}}
\hypersetup{pdfauthor={Niklas Beisert}}
\hypersetup{pdfsubject={Manual for the LaTeX2e Package childdoc}}
\date{30 December 2018, \textsf{v2.0}}
\maketitle

\begin{abstract}\noindent
\textsf{childdoc} is a \LaTeXe{} package
that enables the direct compilation
of document sections included by |\include|
to individual files.
\end{abstract}

\begingroup
\parskip0ex
\tableofcontents
\endgroup

%%%%%%%%%%%%%%%%%%%%%%%%%%%%%%%%%%%%%%%%%%%%%%%%%%%%%%%%%%%%%%%%%%%%%%%%%%%%%%%%
%%%%%%%%%%%%%%%%%%%%%%%%%%%%%%%%%%%%%%%%%%%%%%%%%%%%%%%%%%%%%%%%%%%%%%%%%%%%%%%%
\section{Introduction}

\LaTeX{} provides a mechanism to structure a large document (such as a book)
into a main file and several child files (containing the chapters)
using the |\include| command.
This mechanism is beneficial for documents
which span hundreds of pages in order to
make the source file(s) more manageable.
Moreover, compilation can be restricted to
selected child files by means of the |\includeonly| command.
The latter feature can be used to reduce the compilation time while editing
(this was significantly more useful in the earlier days of \LaTeX{})
or to generate a smaller document which is easier to navigate.
Another application of |\includeonly| is to generate
documents consisting of selected parts of the complete document.

However, there are a few drawbacks of the plain |\include| mechanism:
\begin{itemize}
\item
The child files cannot be compiled on their own,
they can only be compiled via the main file.
A naive editing environment
(such as a text editor with an option
to have the current file processed by \LaTeX)
may require one to switch to the main file before compiling;
attempting to compile the child file produces errors.
\item
The main file must be modified (each time)
to adjust the |\includeonly| command
to the present needs. This easily leaves the main file in a messy state.
\item
The generated document will always carry the filename
of the main document. This is inconvenient if
several child files are to be compiled and
to be kept for distribution.
\end{itemize}

The present package provides a simple interface
to make child files individually compilable by \LaTeX{}.
Compiling a child file then has the same effect as compiling
the main file with an |\includeonly| command
to select the appropriate child.
Moreover the generated document will carry the name of the child
rather than the main file.
This resolves all three above issues.

This feature is meant to make the editing of books,
thesis documents and lecture notes somewhat more convenient.
However, the package can also be used efficiently for
composing a series of documents (such as exercise sheets)
which are typically distributed individually.
It then assists the author in generating the individual documents
(potentially in different versions)
as well as a document containing the collected series.
Another application is in developing style files
or other kinds of included material
where compilation of the style file could redirect
to a sample or test file.

%%%%%%%%%%%%%%%%%%%%%%%%%%%%%%%%%%%%%%%%%%%%%%%%%%%%%%%%%%%%%%%%%%%%%%%%%%%%%%%%
%%%%%%%%%%%%%%%%%%%%%%%%%%%%%%%%%%%%%%%%%%%%%%%%%%%%%%%%%%%%%%%%%%%%%%%%%%%%%%%%
\section{Usage}

First of all, the package \textsf{childdoc} is \emph{not} a standard
\LaTeXe{} |.sty| style file! Therefore it needs to be invoked in
a non-standard way.

%%%%%%%%%%%%%%%%%%%%%%%%%%%%%%%%%%%%%%%%%%%%%%%%%%%%%%%%%%%%%%%%%%%%%%%%%%%%%%%%
\subsection{Included Files}
\label{sec:include}

%%%%%%%%%%%%%%%%%%%%%%%%%%%%%%%%%%%%%%%%
\DescribeMacro{\childdocmain}
To use the package, add the commands
\begin{center}
\begin{tabular}{l}
|\input{childdoc.def}|\\
|\childdocmain{}|\\
\end{tabular}
\end{center}
at the very top of the main \LaTeX{} file,
in particular \emph{before} the |\documentclass| statement!
The argument of |\childdocmain| should be left empty
(but it must be present).

%%%%%%%%%%%%%%%%%%%%%%%%%%%%%%%%%%%%%%%%
\DescribeMacro{\childdocof}
Furthermore, add the commands
\begin{center}
\begin{tabular}{l}
|\input{childdoc.def}|\\
|\childdocof{|\textit{main}|}|\\
\end{tabular}
\end{center}
at the top of every child file \textit{child}
which is included by |\include{|\textit{child}|}|
from within the main file
(or at least for those files to be compiled individually).
The argument \textit{main} must be the filename of the main file.

There are a couple of
considerations in setting up the main and child documents:

%%%%%%%%%%%%%%%%%%%%%%%%%%%%%%%%%%%%%%%%
\paragraph{Restrictions.}

Please note the following restrictions:
\begin{itemize}
\item
|\childdocmain| must be called with one argument \textit{main}
to ensure compatibility with earlier version of the package.
It must either be empty (|\childdocmain{}|)
or precisely match the filename of the main file in which it is specified.
See \secref{sec:detection} for further information.
\item
The filename \textit{main} must be specified without the |.tex| extension.
\item
The filename \textit{main} is case sensitive
(even in case-insensitive file systems)
due to internal string comparison.
\item
The argument \textit{main} should be fully expanded, it cannot be a macro.
\item
Subdirectories and special characters should be avoided in filenames.
\item
The command |\childdocmain{|\textit{main}|}| must be followed by a whitespace.
It should not be followed immediately by another command
or by a comment mark `|%|'.
This is because the \TeX{} parser reads the token immediately following
the argument of |\childdocmain| and puts it
at the beginning of every child section;
however, a white\-space is ignored.
\end{itemize}

%%%%%%%%%%%%%%%%%%%%%%%%%%%%%%%%%%%%%%%%
\paragraph{Content of Main File.}

It is advisable to place all content in the child files included by |\include|.
Any output contained in the main file will appear in all child documents
unless suppressed manually;
it cannot be suppressed automatically by the |\includeonly| directive
and thus should normally be avoided.
A method to include some content in the main file
by means of conditional processing is described in \secref{sec:conditional}.

%%%%%%%%%%%%%%%%%%%%%%%%%%%%%%%%%%%%%%%%
\paragraph{Page Numbering.}

When only a part of the document is compiled,
the appropriate numbering of pages
(as well as other status parameters)
is determined from the |.aux| files.
The latter contain information from previous passes.
However this information needs to propagate through
all intermediate child documents.
Therefore the page numbering in child documents may well
be inconsistent until the complete document is compiled at least once.

A useful (if unconventional) way to always ensure a consistent
page numbering is to restart the numbering in each child document
and denote the pages by `\textit{child}|.|\textit{page}'
where \textit{child} represents the chapter/section number of the child file.
This can be achieved by the command
|\numberwithin{page}{|\textit{child}|}|
of the \textsf{amsmath} package
where \textit{child} can be |chapter| or |section|
depending on the chosen structuring.
Alternatively, one can modify the macro |\thepage| appropriately
and reset the counter |page| at the start of each child file.

%%%%%%%%%%%%%%%%%%%%%%%%%%%%%%%%%%%%%%%%%%%%%%%%%%%%%%%%%%%%%%%%%%%%%%%%%%%%%%%%
\subsection{Conditional Processing}
\label{sec:conditional}

The package provides a mechanism to compile different versions
of a document. To customise the versions further some conditional processing
can come in handy to distinguish which version is being compiled.
The package provides two macros to describe the compilation context:

%%%%%%%%%%%%%%%%%%%%%%%%%%%%%%%%%%%%%%%%
\DescribeMacro{\ifchilddoc}
The conditional |\ifchilddoc| distinguishes between the compilation of
child documents and the main document:
%
\begin{center}
|\ifchilddoc |\textit{child-code}| |[|\||else |\textit{main-code}]| \||fi|
\end{center}

%%%%%%%%%%%%%%%%%%%%%%%%%%%%%%%%%%%%%%%%
\DescribeMacro{\childdocname}
\DescribeMacro{\childdocjob}
The macro |\childdocname| contains the filename (without extension)
of the main or child file being processed.
Note that |\childdocjob| will always contain the name of the main file.

%%%%%%%%%%%%%%%%%%%%%%%%%%%%%%%%%%%%%%%%
\paragraph{Title Page.}

Conditional processing can be used to include a title or banner page
in the main document when proper precautions are taken.
Importantly, the code in the main file should ensure that the page counter
(as well as other status parameters which are stored in the |.aux| files)
takes the same value after the conditional processing.
Otherwise the page numbers may take divergent values
depending on which part is compiled.

For example, a title page could be declared by:
%
\begin{center}
\begin{tabular}{l}
|\ifchilddoc\||else|\\
|\addtocounter{page}{-1}|\\
\textit{code for title page}\\
|\newpage|\\
|\||fi|
\end{tabular}
\end{center}
%
A banner page for the child documents can be generated by:
%
\begin{center}
\begin{tabular}{l}
|\ifchilddoc|\\
|\addtocounter{page}{-1}|\\
\textit{code for banner page}\\
|\newpage|\\
|\||fi|
\end{tabular}
\end{center}
%
Here one could write a message such as:
\begin{center}
|This is the part \childdocname{} of \childdocjob{}.|
\end{center}

%%%%%%%%%%%%%%%%%%%%%%%%%%%%%%%%%%%%%%%%%%%%%%%%%%%%%%%%%%%%%%%%%%%%%%%%%%%%%%%%
\subsection{Flags}
\label{sec:flags}

The package makes it easy to generate different versions
of the main or child documents.
To this end compilation flags can be defined
and assigned different default values.
They will be particularly useful in conjunction
with the forwarding mechanism described in \secref{sec:forward}.

For example, it may be useful to have a flag |\version|
which can be set to |draft| or |final|.
The document source will contain some conditional code
depending on the value of |\version|.
Suppose further, the flag should default to |final| for the main file
and to |draft| for child files
which is a natural assignment for editing the document.
This is achieved by placing the following code
in the preamble of the main document
(below the |\childdocmain| directive):
%
\begin{center}
\begin{tabular}{l}
|\ifchilddoc|\\
|\providecommand{\version}{draft}|\\
|\||else|\\
|\providecommand{\version}{final}|\\
|\||fi|
\end{tabular}
\end{center}
%
The definition by |\providecommand| makes sure
that previous definitions are not overwritten.
Further statements |\providecommand{\version}{...}|
can thus be added before the above code to override it.

For the main file, one might add a line
(between |\childdocmain| and the above block)
%
\begin{center}
|%\ifchilddoc\||else\providecommand{\version}{draft}\||fi|
\end{center}
%
which can be uncommented to produce a draft version.
Likewise one can add a line to the very top of a child file
(above the |\childdocof{|\textit{main}|}| directive)
%
\begin{center}
|%\providecommand{\version}{final}|
\end{center}
%
which can be uncommented to produce the final version of this child document.

%%%%%%%%%%%%%%%%%%%%%%%%%%%%%%%%%%%%%%%%%%%%%%%%%%%%%%%%%%%%%%%%%%%%%%%%%%%%%%%%
\subsection{Forwarding}
\label{sec:forward}

Different versions of the main or child documents
using compilation flags as described in \secref{sec:flags}
can be (permanently) stored in different files
for convenient compilation, viewing and distribution.
To this end, the package defines a command
to pass on compilation to a different file:

%%%%%%%%%%%%%%%%%%%%%%%%%%%%%%%%%%%%%%%%
\DescribeMacro{\childdocforward}
The command |\childdocforward| redirects processing to
another source file:
%
\begin{center}
\begin{tabular}{l}
|\input{childdoc.def}|\\
|\childdocforward[|\textit{main}|]{|\textit{dest}|}|\\
\end{tabular}
\end{center}
%
The argument \textit{dest} is the destination file
(without extension).
It should be the main file or one of the child files.
Note that further \textsf{childdoc} directives
such as |\childdocof| and |\childdocforward|
in the indicated file will be processed in this form.
The optional argument \textit{main}
passes on directly to the main file \textit{main}
while pretending to compile the child \textit{dest}.
This form behaves as if \textit{dest}
issues |\childdocof{|\textit{main}|}| right away,
and no further \textsf{childdoc} directives will be processed.

%%%%%%%%%%%%%%%%%%%%%%%%%%%%%%%%%%%%%%%%
\DescribeMacro{\...prefix}
In the alternative form |\childdocforwardprefix|,
%
\begin{center}
\begin{tabular}{l}
|\input{childdoc.def}|\\
|\childdocforwardprefix[|\textit{main}|]{|\textit{prefix}|}{|\textit{dest}|}|
\end{tabular}
\end{center}
%
the destination file is determined by a pattern
depending on the current file:
To make this work, the current file must be called
`{\textit{prefix}\hspace{0.2em}\textit{suffix}}'
with \textit{prefix} matching precisely the argument.
Processing is then passed on to the file
`{\textit{dest}\hspace{0.2em}\textit{suffix}}'.
Surely, the same effect is achieved by
directly specifying the
argument `{\textit{dest}\hspace{0.2em}\textit{suffix}}'
in the first form.
However, that requires to set up a different file
for each child. With the alternative form of the command
all these files can have exactly the same content
which simplifies setting them up and maintaining them.

For example, the following file |draft.tex|
with a compilation flag |\version| as described in \secref{sec:flags}
compiles the main document as a draft:
%
\begin{center}
\begin{tabular}{l}
|\def\version{draft}|\\
|\input{childdoc.def}|\\
|\childdocforward{|\textit{main}|}|
\end{tabular}
\end{center}
%
Likewise, the following files |final|\textit{nn}|.tex|
compile the final version of the child document
|child|\textit{nn}|.tex|:
%
\begin{center}
\begin{tabular}{l}
|\def\version{final}|\\
|\input{childdoc.def}|\\
|\childdocforwardprefix{final}{child}|
\end{tabular}
\end{center}
%

Note that when several versions of a main file and/or of each child file
are to be generated, it may be convenient to set up a |Makefile| or
shell script to automatise the process.

%%%%%%%%%%%%%%%%%%%%%%%%%%%%%%%%%%%%%%%%%%%%%%%%%%%%%%%%%%%%%%%%%%%%%%%%%%%%%%%%
\subsection{Command Line Processing}
\label{sec:commandline}

The effect of redirection files can also be achieved by invoking
the \LaTeX{} compiler with a more elaborate command line.
Most conveniently this should be done as part
of a shell script or a |Makefile|.

When using \textsf{childdoc} in the main file, the following
command lines effectively perform a redirection
(note that depending on the shell being used,
backslashes may have to be doubled: `|\|' $\to$ `|\\|'):
%
\begin{center}
|... -jobname "|\textit{target}|" |\\|"|[\textit{flags}]%
|\input{childdoc.def}\childdocforward[|\textit{main}|]{|\textit{dest}|}"|
\end{center}
%
Here \textit{target} is the name of the output file,
\textit{main} is the name of the main file
and \textit{dest} is the name of the main or child file to be processed
(all filenames without extensions).
The optional argument \textit{main} can be omitted
if \textit{main} matches \textit{dest}.
Optionally, compilation \textit{flags} can be defined via |\def| commands.
This command line makes the \TeX{} engine believe
it is compiling the file \textit{target}
whose content is specified as the latter parameter.
The provided code then forwards the processing to
\textit{main} or \textit{dest} as described in \secref{sec:forward}.

%%%%%%%%%%%%%%%%%%%%%%%%%%%%%%%%%%%%%%%%%%%%%%%%%%%%%%%%%%%%%%%%%%%%%%%%%%%%%%%%
\subsection{Include by Input}
\label{sec:input}

Including child documents by |\include| has some restrictions by design.
Most notably, the content of a child document always occupies
its own set of pages; pages cannot be shared between child documents.
Usually, this behaviour makes perfect sense
because each child document contain an essential part of the document.
However, in some situations it may be desirable to compose
a document from a collection of parts
without having mandatory page breaks between then.
For this case, the package
provides a mechanism to include parts
by |\input| which can also be processed individually.
However, by construction this mechanism
requires manual handling of the content to be output.

%%%%%%%%%%%%%%%%%%%%%%%%%%%%%%%%%%%%%%%%
\DescribeMacro{\ifchilddocmanual}
The main file should be prepared as usual, see \secref{sec:include}.
However, the document body must make a distinction
between processing of an individual part and of the main document, e.g.:
%
\begin{center}
\begin{tabular}{l}
|\ifchilddocmanual|\\
|\input{\childdocname}|\\
|\||else|\\
\textit{document body with }|\input{|\textit{part}|}|\\
|\||fi|
\end{tabular}
\end{center}
%
The conditional |\ifchilddocmanual| is true whenever
a part to be included by |\input| is being compiled,
and the name of the part is stored in |\childdocname|.

%%%%%%%%%%%%%%%%%%%%%%%%%%%%%%%%%%%%%%%%
\DescribeMacro{\childdocby}
Each part to be included by |\input| should start with:
%
\begin{center}
\begin{tabular}{l}
|\input{childdoc.def}|\\
|\childdocby{|\textit{main}|}|\\
\end{tabular}
\end{center}
%
The directive |\childdocby| is similar to |\childdocof|
described in \secref{sec:include},
but the subsequent selection of content must be done manually.
To that end, both |\ifchilddoc| and |\ifchilddocmanual|
will be true upon processing of a part,
and the name of the part is stored in |\childdocname|.
Note that |\jobname| will be set to the filename of the current part
so that each part receives an individual |.aux| file
that does not interfere with the |.aux| file(s) of the main document.
This behaviour can be altered by the alternative form
|\childdocby[*]{|\textit{main}|}| (with a non-empty optional argument)
which uses the |.aux| file of the main document
by setting |\jobname| to \textit{main}.

%%%%%%%%%%%%%%%%%%%%%%%%%%%%%%%%%%%%%%%%%%%%%%%%%%%%%%%%%%%%%%%%%%%%%%%%%%%%%%%%
\subsection{Driver Development}
\label{sec:driver}

The \textsf{childdoc} mechanism can also be use for the development
of definition files such as \LaTeX{} styles or classes.
This case differs from the above setup with multiple parts
included by |\include| in that no |\includeonly| should be invoked.
This can be achieved by starting the include file
(before |\ProvidesPackage|) with:
%
\begin{center}
\begin{tabular}{l}
|\input{childdoc.def}|\\
|\childdocforward{|\textit{main}|}|\\
\end{tabular}
\end{center}
%
or alternatively with:
%
\begin{center}
\begin{tabular}{l}
|\input{childdoc.def}|\\
|\childdocby{|\textit{main}|}|\\
\end{tabular}
\end{center}
%
Both forms have slightly different effects as described above.
The main file is prepared as usual, see \secref{sec:include}.

%%%%%%%%%%%%%%%%%%%%%%%%%%%%%%%%%%%%%%%%%%%%%%%%%%%%%%%%%%%%%%%%%%%%%%%%%%%%%%%%
\subsection{Legacy Detection}
\label{sec:detection}

The directive |\childdocmain| in the main file can detect
whether the complete document or merely a child is to be compiled
even without using the directive |\childdocof|.
This method is deprecated because it is less robust
and there is no compelling reason to use it;
it is merely provided for backward compatibility
and it may be removed in future versions.

If the detection mechanism is to be used,
it is mandatory to correctly specify
the filename of the main file as the argument of |\childdocmain|:
%
\begin{center}
\begin{tabular}{l}
|\input{childdoc.def}|\\
|\childdocmain{|\textit{main}|}|\\
\end{tabular}
\end{center}
%
If |\jobname| does not match the argument \textit{main} of |\childdocmain|,
it is assumed that |\jobname| points to the child file to be compiled.
When using |\childdocmain| with the main file specified as argument,
it suffices to start a child file
with just |\input{|\textit{main}|}|
without loading of the package and using |\childdocof|.
If instead all processing is done
with the appropriate \textsf{childdoc} directives,
the argument of \textit{main} of |\childdocmain| can be empty.

An alternative version of the command line processing described
in \secref{sec:commandline} using the detection mechanism reads:
%
\begin{center}
|... -jobname "|\textit{target}|" "|[\textit{flags}]%
[|\def\jobname{|\textit{dest}|}|]|\input{|\textit{main}|}"|
\end{center}

%%%%%%%%%%%%%%%%%%%%%%%%%%%%%%%%%%%%%%%%%%%%%%%%%%%%%%%%%%%%%%%%%%%%%%%%%%%%%%%%
\subsection{Manual Code}
\label{sec:manual}

In case one cannot be certain whether the definitions file |childdoc.def|
is installed on the target \TeX{} distribution
and one prefers not to ship it,
it is conceivable to paste a few relevant commands into the sources.

To that end, drop all statements |\input{childdoc.def}|
and perform the replacements as outlined below.
Instead of |\childdocmain{|\textit{main}|}| add the following code
to the top of the main file:
%
\begin{center}
\begin{tabular}{l}
|\||ifdefined\childdocname\endinput\||fi\newif\ifchilddoc|\\
|\edef\childdocname{\scantokens\expandafter{\jobname\noexpand}}|\\
|\def\childdocmain{|\textit{main}|}\||ifx\childdocmain\childdocname\||else|\\
|\childdoctrue\includeonly{\childdocname}\let\jobname\childdocmain\||fi|\\
\end{tabular}
\end{center}
%
Instead of |\childdocof{|\textit{main}|}| just include the main file
at the top of each child file:
%
\begin{center}
|\input{|\textit{main}|}|
\end{center}
%
A simple redirection |\childdocforward{|\textit{dest}|}| is achieved by:
%
\begin{center}
|\def\jobname{|\textit{dest}|}\input{\jobname}|
\end{center}
%
The redirection with prefix
|\childdocforwardprefix[|\textit{prefix}|]{|\textit{dest}|}|
is accomplished by:
%
\begin{center}
\begin{tabular}{l}
|{\edef\jobname{\scantokens\expandafter{\jobname\noexpand}}|\\
|\def\redirectjob |\textit{prefix}|#1~~~{\gdef\jobname{|\textit{dest}|#1}}|\\
|\expandafter\redirectjob\jobname~~~}\input{\jobname}|
\end{tabular}
\end{center}

In an alternative approach,
child documents can be compiled by a specific command line
without additional code or specific definitions:
%
\begin{center}
|... -jobname "|\textit{target}|" "|[\textit{flags}]%
|\includeonly{|\textit{dest}|}\input{|\textit{main}|}"|
\end{center}
%

%%%%%%%%%%%%%%%%%%%%%%%%%%%%%%%%%%%%%%%%%%%%%%%%%%%%%%%%%%%%%%%%%%%%%%%%%%%%%%%%
%%%%%%%%%%%%%%%%%%%%%%%%%%%%%%%%%%%%%%%%%%%%%%%%%%%%%%%%%%%%%%%%%%%%%%%%%%%%%%%%
\section{Information}

%%%%%%%%%%%%%%%%%%%%%%%%%%%%%%%%%%%%%%%%%%%%%%%%%%%%%%%%%%%%%%%%%%%%%%%%%%%%%%%%
\subsection{Copyright}

Copyright \copyright{} 2017--2018 Niklas Beisert

This work may be distributed and/or modified under the
conditions of the \LaTeX{} Project Public License, either version 1.3
of this license or (at your option) any later version.
The latest version of this license is in
  \url{http://www.latex-project.org/lppl.txt}
and version 1.3 or later is part of all distributions of \LaTeX{}
version 2005/12/01 or later.

This work has the LPPL maintenance status `maintained'.

The Current Maintainer of this work is Niklas Beisert.

This work consists of the files |README.txt|, |childdoc.ins| and |childdoc.dtx|
as well as the derived files |childdoc.def|, |cdocsamp.tex|
with |cdocsch1.tex|, |cdocsch2.tex|, |cdocspt3.tex|, |cdocspt4.tex|,
|cdocsdrf.tex|, |cdocsfn1.tex|, |cdocsfn2.tex|
as well as |childdoc.pdf|.

%%%%%%%%%%%%%%%%%%%%%%%%%%%%%%%%%%%%%%%%%%%%%%%%%%%%%%%%%%%%%%%%%%%%%%%%%%%%%%%%
\subsection{Files and Installation}

The package consists of the files:
%
\begin{center}
\begin{tabular}{ll}
    |README.txt|   & readme file \\
    |childdoc.ins| & installation file \\
    |childdoc.dtx| & source file \\
    |childdoc.def| & definition file \\
    |cdocsamp.tex| & sample main file \\
    |cdocsch1.tex| & sample include file \\
    |cdocsch2.tex| & sample include file \\
    |cdocspt3.tex| & sample part file \\
    |cdocspt4.tex| & sample part file \\
    |cdocsdrf.tex| & sample redirection file \\
    |cdocsfn1.tex| & sample redirection file \\
    |cdocsfn2.tex| & sample redirection file \\
    |childdoc.pdf| & manual
\end{tabular}
\end{center}
%
The distribution consists of the files
|README.txt|, |childdoc.ins| and |childdoc.dtx|.
%
\begin{itemize}
\item
Run (pdf)\LaTeX{} on |childdoc.dtx|
to compile the manual |childdoc.pdf| (this file).
\item
Run \LaTeX{} on |childdoc.ins| to create the definitions file |childdoc.def|
and the sample |cdocsamp.tex| with include files
|cdocsch1.tex|, |cdocsch2.tex|, |cdocspt3.tex|, |cdocspt4.tex|,
|cdocsdrf.tex|, |cdocsfn1.tex|, |cdocsfn2.tex|.
Then copy the file |childdoc.def| to an appropriate directory of your \LaTeX{}
distribution, e.g.\ \textit{texmf-root}|/tex/latex/childdoc|.
\end{itemize}

%%%%%%%%%%%%%%%%%%%%%%%%%%%%%%%%%%%%%%%%%%%%%%%%%%%%%%%%%%%%%%%%%%%%%%%%%%%%%%%%
\subsection{Related CTAN Packages}

There are several other packages which offer a similar functionality:
%
\begin{itemize}
\item
The packages
\href{http://ctan.org/pkg/docmute}{\textsf{docmute}},
\href{http://ctan.org/pkg/includex}{\textsf{includex}} and
\href{http://ctan.org/pkg/standalone}{\textsf{standalone}}
provide commands to include only the document body of
a child file thus allowing both files to be compiled individually.
\item
The packages \href{http://ctan.org/pkg/subdocs}{\textsf{subdocs}}
and \href{http://ctan.org/pkg/subfiles}{\textsf{subfiles}}
provide structures in which the main and child documents can be
encapsulated and allowing them to be compiled individually.
The inclusion mechanism is different from the conventional |\include|.
\item
The package \href{http://ctan.org/pkg/combine}{\textsf{combine}}
is an elaborate solution to combine several documents into one.
\end{itemize}
%
See also the CTAN topic \href{http://ctan.org/topic/subdocs}{\textsf{subdocs}}
for further related packages.
The present package differs from the above solutions in that
a document structure constructed with the conventional |\include| mechanism
just needs two extra commands at the top of every file
such that all constituent files can be compiled individually.

%%%%%%%%%%%%%%%%%%%%%%%%%%%%%%%%%%%%%%%%%%%%%%%%%%%%%%%%%%%%%%%%%%%%%%%%%%%%%%%%
%\subsection{Feature Suggestions}
%
%The following is a list of features which may be useful for future
%versions of this package:
%%
%\begin{itemize}
%\item
%\ldots
%\end{itemize}

%%%%%%%%%%%%%%%%%%%%%%%%%%%%%%%%%%%%%%%%%%%%%%%%%%%%%%%%%%%%%%%%%%%%%%%%%%%%%%%%
\subsection{Revision History}

%%%%%%%%%%%%%%%%%%%%%%%%%%%%%%%%%%%%%%%%
\paragraph{v2.0:} 2018/12/30

\begin{itemize}
\item
immediate forward processing
\item
added |\childdocby| mechanism
\item
manual restructured
\end{itemize}

%%%%%%%%%%%%%%%%%%%%%%%%%%%%%%%%%%%%%%%%
\paragraph{v1.6:} 2018/01/17

\begin{itemize}
\item
application for development of include files
\item
corrections to manual
\end{itemize}

%%%%%%%%%%%%%%%%%%%%%%%%%%%%%%%%%%%%%%%%
\paragraph{v1.5:} 2017/05/21

\begin{itemize}
\item
more complete structuring introduced
\item
|\childdocof| introduced
\item
|\childdoc| renamed to |\childdocmain|
\item
|\childredirect| renamed to |\childdocforward| and |\childdocforwardprefix|
and functionality expanded
\end{itemize}

%%%%%%%%%%%%%%%%%%%%%%%%%%%%%%%%%%%%%%%%
\paragraph{v1.0:} 2017/04/27

\begin{itemize}
\item
manual and install package
\item
first version published on CTAN
\end{itemize}

%%%%%%%%%%%%%%%%%%%%%%%%%%%%%%%%%%%%%%%%
\paragraph{v0.6:} 2017/04/26

\begin{itemize}
\item
redirection mechanism added
\end{itemize}

%%%%%%%%%%%%%%%%%%%%%%%%%%%%%%%%%%%%%%%%
\paragraph{v0.5:} 2017/04/26

\begin{itemize}
\item
functionality in definition file
\end{itemize}


%%%%%%%%%%%%%%%%%%%%%%%%%%%%%%%%%%%%%%%%%%%%%%%%%%%%%%%%%%%%%%%%%%%%%%%%%%%%%%%%
%%%%%%%%%%%%%%%%%%%%%%%%%%%%%%%%%%%%%%%%%%%%%%%%%%%%%%%%%%%%%%%%%%%%%%%%%%%%%%%%
%%%%%%%%%%%%%%%%%%%%%%%%%%%%%%%%%%%%%%%%%%%%%%%%%%%%%%%%%%%%%%%%%%%%%%%%%%%%%%%%
\appendix

\settowidth\MacroIndent{\rmfamily\scriptsize 000\ }

 \DocInput{childdoc.dtx}

\end{document}
%</driver>
% \fi
%
% %%%%%%%%%%%%%%%%%%%%%%%%%%%%%%%%%%%%%%%%%%%%%%%%%%%%%%%%%%%%%%%%%%%%%%%%%%%%%%
% %%%%%%%%%%%%%%%%%%%%%%%%%%%%%%%%%%%%%%%%%%%%%%%%%%%%%%%%%%%%%%%%%%%%%%%%%%%%%%
% \section{Sample}
%\iffalse
%<*samplemain>
%\fi
%
% The following presents a sample document
% with two chapters, two parts, a title page,
% a compile flag as well as three forwarding files to set the flag.
% It consists of eight |.tex| files:
% \begin{center}
% \begin{tabular}{ll}
% |cdocsamp.tex|&main file\\
% |cdocsch1.tex|&include file for chapter 1\\
% |cdocsch2.tex|&include file for chapter 2\\
% |cdocspt3.tex|&include file for part 3\\
% |cdocspt4.tex|&include file for part 4\\
% |cdocsdrf.tex|&forwarding file for main file in draft mode\\
% |cdocsfi1.tex|&forwarding file for final version of chapter 1\\
% |cdocsfi2.tex|&forwarding file for final version of chapter 2\\
% \end{tabular}
% \end{center}
% Each of the eight files can be compiled directly by the \LaTeX{} compiler.
%
% %%%%%%%%%%%%%%%%%%%%%%%%%%%%%%%%%%%%%%
% \paragraph{Main File.}
%
% The main file is called |cdocsamp.tex|.
%
% Load the \textsf{childdoc} definitions and
% declare the filename for the main document:
%    \begin{macrocode}
\input{childdoc.def}
\childdocmain{}
%    \end{macrocode}

% Optional override for |\version| flag:
%    \begin{macrocode}
%%\ifchilddoc\else\providecommand{\version}{draft}\fi
%    \end{macrocode}

% Define the default values for the |\version| flag
% (|final| for the main file and |draft| for childs):
%    \begin{macrocode}
\ifchilddoc
\providecommand{\version}{draft}
\else
\providecommand{\version}{final}
\fi
%    \end{macrocode}

% Load the standard document class:
%    \begin{macrocode}
\documentclass[12pt]{article}
%    \end{macrocode}

% Start the document body:
%    \begin{macrocode}
\begin{document}
%    \end{macrocode}

% Declare a title page.
% Print title, part of document being processed and version flag:
%    \begin{macrocode}
\addtocounter{page}{-1}
\begin{center}
{\LARGE\bfseries{}childdoc example\par}
\vspace{1cm}
\ifchilddoc
\ifchilddocmanual part\else chapter\fi:
`\childdocname' of `\childdocjob'\par
\else
main document: `\childdocjob'\par
\fi
version: \version\par
\end{center}
\newpage
%    \end{macrocode}

% Manually include selected file,
% otherwise process as usual:
%    \begin{macrocode}
\ifchilddocmanual
\section*{part `\childdocname'}
\input{\childdocname}
\else
%    \end{macrocode}

% Include the two chapters:
%    \begin{macrocode}
\include{cdocsch1}
\include{cdocsch2}
%    \end{macrocode}

% Include the two parts unless only chapters should be displayed:
%    \begin{macrocode}
\ifchilddoc\else
\section{part three}
\input{cdocspt3}
\section{part four}
\input{cdocspt4}
\fi
%    \end{macrocode}

% Process as usual until here:
%    \begin{macrocode}
\fi
%    \end{macrocode}

% End of document body:
%    \begin{macrocode}
\end{document}
%    \end{macrocode}
%\iffalse
%</samplemain>
%\fi
%
% %%%%%%%%%%%%%%%%%%%%%%%%%%%%%%%%%%%%%%
% \paragraph{Chapter Include Files.}
%
% The include files are called |cdocsch1.tex| and |cdocsch2.tex|.
%
%\iffalse
%<*samplechap1|samplechap2>
%\fi

% Optional override for |\version| flag:
%    \begin{macrocode}
%%\providecommand{\version}{final}
%    \end{macrocode}

% Include the main document:
%    \begin{macrocode}
\input{childdoc.def}
\childdocof{cdocsamp}
%    \end{macrocode}

%\iffalse
%</samplechap1|samplechap2>
%\fi
%
%\iffalse
%<*samplechap1>
%\fi
% Some text for chapter 1:
%    \begin{macrocode}
\section{one}
some text in chapter one
%    \end{macrocode}

%\iffalse
%</samplechap1>
%\fi
% Some text for chapter 2:
%\iffalse
%<*samplechap2>
%\fi
%    \begin{macrocode}
\section{two}
more text in chapter two
%    \end{macrocode}

%\iffalse
%</samplechap2>
%\fi
%
% %%%%%%%%%%%%%%%%%%%%%%%%%%%%%%%%%%%%%%
% \paragraph{Part Include Files.}
%
% The include files are called |cdocspt3.tex| and |cdocspt4.tex|.
%
%\iffalse
%<*samplepart3|samplepart4>
%\fi

% Optional override for |\version| flag:
%    \begin{macrocode}
%%\providecommand{\version}{final}
%    \end{macrocode}

% Include the main document:
%    \begin{macrocode}
\input{childdoc.def}
\childdocby{cdocsamp}
%    \end{macrocode}

%\iffalse
%</samplepart3|samplepart4>
%\fi
%
%\iffalse
%<*samplepart3>
%\fi
% Some text for part 3:
%    \begin{macrocode}
some text in part three
%    \end{macrocode}

%\iffalse
%</samplepart3>
%\fi
% Some text for part 4:
%\iffalse
%<*samplepart4>
%\fi
%    \begin{macrocode}
more text in part four
%    \end{macrocode}

%\iffalse
%</samplepart4>
%\fi
%
% %%%%%%%%%%%%%%%%%%%%%%%%%%%%%%%%%%%%%%
% \paragraph{Forwarding for a Complete Draft.}
%
% The following forwarding file |cdocsdrf.tex|
% compiles the main document in draft mode:
%\iffalse
%<*sampledraft>
%\fi
%    \begin{macrocode}
\def\version{draft}
\input{childdoc.def}
\childdocforward{cdocsamp}
%    \end{macrocode}

%\iffalse
%</sampledraft>
%\fi
%
% %%%%%%%%%%%%%%%%%%%%%%%%%%%%%%%%%%%%%%
% \paragraph{Forwarding for Final Version of the Chapters.}
%
% The following forwarding files |cdocsfn1.tex| and |cdocsfn2.tex|
% (with identical content)
% compile the final versions of the child documents
% |cdocsch1.tex| and |cdocsch2.tex|, respectively:
%\iffalse
%<*samplefinal>
%\fi
%    \begin{macrocode}
\def\version{final}
\input{childdoc.def}
\childdocforwardprefix[cdocsamp]{cdocsfn}{cdocsch}
%    \end{macrocode}

%\iffalse
%</samplefinal>
%\fi
%
% %%%%%%%%%%%%%%%%%%%%%%%%%%%%%%%%%%%%%%
% \paragraph{Command Line Processing.}
%
% The following three command lines generate the output files
% |cdocscld|, |cdocscl1| and |cdocscl2|
% which should be identical to
% |cdocsdrf|, |cdocsch1| and |cdocsfn2|, respectively:
% \begin{center}
% \begin{tabular}{l}
% |latex -jobname cdocscld \|\\
% |  "\def\version{draft}\input{childdoc.def}\childdocforward{cdocsamp}"|\\
% |latex -jobname cdocscl1 \|\\
% |  "\input{childdoc.def}\childdocforward[cdocsamp]{cdocsch1}"|\\
% |latex -jobname cdocscl2 \|\\
% |  "\def\version{final}\input{childdoc.def}\childdocforward{cdocsch2}"|
% \end{tabular}
% \end{center}
% Note that the trailing backslash on each first line
% merely continues the input to the second line
% (for convenient cut ant paste).
% Furthermore, the command |latex| can be replaced by any
% of its alternative versions such as |pdflatex|.
%
% %%%%%%%%%%%%%%%%%%%%%%%%%%%%%%%%%%%%%%%%%%%%%%%%%%%%%%%%%%%%%%%%%%%%%%%%%%%%%%
% %%%%%%%%%%%%%%%%%%%%%%%%%%%%%%%%%%%%%%%%%%%%%%%%%%%%%%%%%%%%%%%%%%%%%%%%%%%%%%
% \section{Implementation}
%\iffalse
%<*package>
%\fi
%
% This section describes the definitions file |childdoc.def|.

% The definitions cannot be loaded using |\usepackage| or |\RequirePackage|
% which has a mechanism to prevent loading a style file more than once.
% When loading the definitions by means of |\input|
% multiple instances have to be prevented manually:
%\iffalse
%This code needs to be before the `\ProvidesFile' directive
%which is defined at the beginning of this file.
%Therefore it is also placed there and commented out here.
%</package>
%<*discard>
%\fi
%    \begin{macrocode}
\ifdefined\childdocmain\endinput\fi
%    \end{macrocode}
%\iffalse
%</discard>
%<*package>
%\fi
%
% \macro{\ifchilddoc}
% \macro{\ifchilddocmanual}
% The conditional |\ifchilddoc| tells whether a
% child (true) or main (false) document is being compiled.
% The conditional |\ifchilddocmanual| tells whether
% the |\includeonly| mechanism is used (false) or
% the selection of child files must be performed manually (true).
% The definitions initialise to false:
%    \begin{macrocode}
\newif\ifchilddoc
\newif\ifchilddocmanual
%    \end{macrocode}

% \macro{\childdocname}
% \macro{\childdocjob}
% The macro |\childdocname| stores the name of the main document
% to be compiled. The macro |\childdocjob| stores the name of
% the document on which the \LaTeX{} compiler was originally invoked.
% The content of |\jobname| cannot be compared
% to filenames specified in the source due to different catcodes.
% The following code rescans |\jobname|, stores the result
% in |\childdocname| and saves a copy in |\childdocjob|:
%    \begin{macrocode}
\edef\childdocname{\scantokens\expandafter{\jobname\noexpand}}
\let\childdocjob\childdocname
%    \end{macrocode}

% \macro{\childdocdisable}
% The macro |\childdocdisable| prevents the main file
% from being processed more than once.
% At this stage, the main document command |\childdocmain|
% is assumed to be called once again where it should do nothing.
% Any subsequent call to it should prevent
% a secondary processing of the main document
% It overwrites the forwarding commands
% |\childdocof| and |\childdocforward|
% with empty macros to prevent further inclusions of the main document:
%    \begin{macrocode}
\newcommand{\childdocdisable}
{
  \renewcommand{\childdocmain}[1]{\renewcommand{\childdocmain}[1]{\endinput}}
  \renewcommand{\childdocof}[1]{}
  \renewcommand{\childdocby}[2][]{}
  \renewcommand{\childdocforward}[2][]{}
  \renewcommand{\childdocdisable}{}
}
%    \end{macrocode}

% \macro{\childdocmain}
% The macro |\childdocmain| is to be called at the top of the main file
% with nothing or the main filename (without extension) as argument.
% First, it breaks loops.
% If the argument is not empty and does not match |\childdocname|
% (which is set by the first inclusion of |childdoc.def|),
% |\ifchilddoc| is set to true, |\includeonly| is applied to the child file
% and |\jobname| is set to the main file
% (for proper handling of |.aux| files):
%    \begin{macrocode}
\newcommand{\childdocmain}[1]
{
  \childdocdisable\childdocmain{}
  \if?#1?\else
    \begingroup
      \def\childdoctmp{#1}
      \ifx\childdoctmp\childdocname
        \def\childdoctmp{}
      \else
        \def\childdoctmp
        {
          \childdoctrue
          \includeonly{\childdocname}
          \def\childdocjob{#1}
          \def\jobname{#1}
        }
      \fi
      \expandafter
    \endgroup
    \childdoctmp
  \fi
}
%    \end{macrocode}

% \macro{\childdocof}
% The command |\childdocof| redirects
% compilation to the main file |#1|.
%    \begin{macrocode}
\newcommand{\childdocof}[1]
{
  \childdocdisable
  \childdoctrue
  \includeonly{\childdocname}
  \def\jobname{#1}
  \def\childdocjob{#1}
  \input{#1}
}
%    \end{macrocode}

% \macro{\childdocby}
% The command |\childdocby| ....
%    \begin{macrocode}
\newcommand{\childdocby}[2][]
{
  \childdocdisable
  \childdoctrue
  \childdocmanualtrue
  \if?#1?\else
    \def\jobname{#2}
  \fi
  \def\childdocjob{#2}
  \input{#2}
  \endinput
}
%    \end{macrocode}

% \macro{\childdocforward}
% The command |\childdocforward| redirects
% compilation to the main file or
% (if the optional argument is given) a child file.
% Parameters are set as if the main file
% or a child file starting with |\childdocof| was compiled.
% Then compilation is handed over to the main file:
%    \begin{macrocode}
\newcommand{\childdocforward}[2][]
{
  \begingroup
    \if?#1?
      \def\childdoctmp
      {
        \def\childdocname{#2}
        \def\childdocjob{#2}
        \def\jobname{#2}
        \input{#2}
        \endinput
      }
    \else
      \def\childdoctmp
      {
        \childdocdisable
        \def\childdocname{#2}
        \childdoctrue
        \includeonly{#2}
        \def\childdocjob{#1}
        \def\jobname{#1}
        \input{#1}
        \endinput
      }
    \fi
    \expandafter
  \endgroup
  \childdoctmp
}
%    \end{macrocode}

% \macro{\childdocforwardprefix}
% The command |\childdocforwardprefix| redirects
% compilation to the main or a child file by means of a pattern.
% The prefix |#1| in the current filename is replaced by |#2|
% and the suffix of the current filename is kept
% (it is assumed that the filename does not contain the substring `|~~~|'
% which is used as a delimiter).
% Compilation is handed over to the new file by |\childdocforward|:
%    \begin{macrocode}
\newcommand{\childdocforwardprefix}[3][]
{
  \begingroup
    \def\childdocextract #2##1~~~{\def\childdoctmp{\childdocforward[#1]{#3##1}}}
    \expandafter\childdocextract\childdocname~~~
    \expandafter
  \endgroup
  \childdoctmp
}
%    \end{macrocode}

% \macro{\childdoc}
% The deprecated macro |\childdoc| is a legacy version of |\childdocmain|:
%    \begin{macrocode}
\newcommand{\childdoc}{\childdocmain}
%    \end{macrocode}

% \macro{\childdocredirect}
% The deprecated macro |\childdocredirect| is a legacy version
% of |\childdocforward| and |\childdocforwardprefix|:
%    \begin{macrocode}
\newcommand{\childdocredirect}[2][]
{
  \begingroup
    \if?#1?
      \def\childdoctmp{\childdocforward{#2}}
    \else
      \def\childdoctmp{\childdocforwardprefix{#1}{#2}}
    \fi
    \expandafter
  \endgroup
  \childdoctmp
}
%    \end{macrocode}

%\iffalse
%</package>
%\fi
%
\endinput
|\\
|\childdocforward{|\textit{main}|}|\\
\end{tabular}
\end{center}
%
or alternatively with:
%
\begin{center}
\begin{tabular}{l}
|% \iffalse
%
% childdoc.dtx Copyright (C) 2017-2018 Niklas Beisert
%
% This work may be distributed and/or modified under the
% conditions of the LaTeX Project Public License, either version 1.3
% of this license or (at your option) any later version.
% The latest version of this license is in
%   http://www.latex-project.org/lppl.txt
% and version 1.3 or later is part of all distributions of LaTeX
% version 2005/12/01 or later.
%
% This work has the LPPL maintenance status `maintained'.
%
% The Current Maintainer of this work is Niklas Beisert.
%
% This work consists of the files childdoc.dtx and childdoc.ins
% and the derived files childdoc.def and cdocsamp.tex with
% cdocsch1.tex, cdocsch2.tex, cdocsdrf.tex, cdocsfn1.tex, cdocsfn2.tex.
%
%<package>\ifdefined\childdocmain\endinput\fi
%<package>\ProvidesFile{childdoc.def}[2018/12/30 v2.0 child document driver]
%<samplemain>\ProvidesFile{cdocsamp.tex}[2018/12/30 v2.0 sample for childdoc]
%<*driver>
%\ProvidesFile{childdoc.drv}[2018/12/30 v2.0 childdoc reference manual file]
\PassOptionsToClass{10pt,a4paper}{article}
\documentclass{ltxdoc}

\usepackage[margin=35mm]{geometry}
\usepackage{hyperref}
\usepackage{hyperxmp}
\usepackage[usenames]{color}

\hypersetup{colorlinks=true}
\hypersetup{pdfstartview=FitH}
\hypersetup{pdfpagemode=UseNone}
\hypersetup{pdfsource={}}
\hypersetup{pdflang={en-UK}}
\hypersetup{pdfcopyright={Copyright 2017-2018 Niklas Beisert.
  This work may be distributed and/or modified under the
  conditions of the LaTeX Project Public License, either version 1.3
  of this license or (at your option) any later version.}}
\hypersetup{pdflicenseurl={http://www.latex-project.org/lppl.txt}}
\hypersetup{pdfcontactaddress={ETH Zurich, ITP, HIT K,
  Wolfgang-Pauli-Strasse 27}}
\hypersetup{pdfcontactpostcode={8093}}
\hypersetup{pdfcontactcity={Zurich}}
\hypersetup{pdfcontactcountry={Switzerland}}
\hypersetup{pdfcontactemail={nbeisert@itp.phys.ethz.ch}}
\hypersetup{pdfcontacturl={http://people.phys.ethz.ch/\xmptilde nbeisert/}}

\newcommand{\secref}[1]{\hyperref[#1]{section \ref*{#1}}}

\parskip1ex
\parindent0pt
\let\olditemize\itemize
\def\itemize{\olditemize\parskip0pt}

\begin{document}

\title{The \textsf{childdoc} Package}
\hypersetup{pdftitle={The childdoc Package}}
\author{Niklas Beisert\\[2ex]
  Institut f\"ur Theoretische Physik\\
  Eidgen\"ossische Technische Hochschule Z\"urich\\
  Wolfgang-Pauli-Strasse 27, 8093 Z\"urich, Switzerland\\[1ex]
  \href{mailto:nbeisert@itp.phys.ethz.ch}
  {\texttt{nbeisert@itp.phys.ethz.ch}}}
\hypersetup{pdfauthor={Niklas Beisert}}
\hypersetup{pdfsubject={Manual for the LaTeX2e Package childdoc}}
\date{30 December 2018, \textsf{v2.0}}
\maketitle

\begin{abstract}\noindent
\textsf{childdoc} is a \LaTeXe{} package
that enables the direct compilation
of document sections included by |\include|
to individual files.
\end{abstract}

\begingroup
\parskip0ex
\tableofcontents
\endgroup

%%%%%%%%%%%%%%%%%%%%%%%%%%%%%%%%%%%%%%%%%%%%%%%%%%%%%%%%%%%%%%%%%%%%%%%%%%%%%%%%
%%%%%%%%%%%%%%%%%%%%%%%%%%%%%%%%%%%%%%%%%%%%%%%%%%%%%%%%%%%%%%%%%%%%%%%%%%%%%%%%
\section{Introduction}

\LaTeX{} provides a mechanism to structure a large document (such as a book)
into a main file and several child files (containing the chapters)
using the |\include| command.
This mechanism is beneficial for documents
which span hundreds of pages in order to
make the source file(s) more manageable.
Moreover, compilation can be restricted to
selected child files by means of the |\includeonly| command.
The latter feature can be used to reduce the compilation time while editing
(this was significantly more useful in the earlier days of \LaTeX{})
or to generate a smaller document which is easier to navigate.
Another application of |\includeonly| is to generate
documents consisting of selected parts of the complete document.

However, there are a few drawbacks of the plain |\include| mechanism:
\begin{itemize}
\item
The child files cannot be compiled on their own,
they can only be compiled via the main file.
A naive editing environment
(such as a text editor with an option
to have the current file processed by \LaTeX)
may require one to switch to the main file before compiling;
attempting to compile the child file produces errors.
\item
The main file must be modified (each time)
to adjust the |\includeonly| command
to the present needs. This easily leaves the main file in a messy state.
\item
The generated document will always carry the filename
of the main document. This is inconvenient if
several child files are to be compiled and
to be kept for distribution.
\end{itemize}

The present package provides a simple interface
to make child files individually compilable by \LaTeX{}.
Compiling a child file then has the same effect as compiling
the main file with an |\includeonly| command
to select the appropriate child.
Moreover the generated document will carry the name of the child
rather than the main file.
This resolves all three above issues.

This feature is meant to make the editing of books,
thesis documents and lecture notes somewhat more convenient.
However, the package can also be used efficiently for
composing a series of documents (such as exercise sheets)
which are typically distributed individually.
It then assists the author in generating the individual documents
(potentially in different versions)
as well as a document containing the collected series.
Another application is in developing style files
or other kinds of included material
where compilation of the style file could redirect
to a sample or test file.

%%%%%%%%%%%%%%%%%%%%%%%%%%%%%%%%%%%%%%%%%%%%%%%%%%%%%%%%%%%%%%%%%%%%%%%%%%%%%%%%
%%%%%%%%%%%%%%%%%%%%%%%%%%%%%%%%%%%%%%%%%%%%%%%%%%%%%%%%%%%%%%%%%%%%%%%%%%%%%%%%
\section{Usage}

First of all, the package \textsf{childdoc} is \emph{not} a standard
\LaTeXe{} |.sty| style file! Therefore it needs to be invoked in
a non-standard way.

%%%%%%%%%%%%%%%%%%%%%%%%%%%%%%%%%%%%%%%%%%%%%%%%%%%%%%%%%%%%%%%%%%%%%%%%%%%%%%%%
\subsection{Included Files}
\label{sec:include}

%%%%%%%%%%%%%%%%%%%%%%%%%%%%%%%%%%%%%%%%
\DescribeMacro{\childdocmain}
To use the package, add the commands
\begin{center}
\begin{tabular}{l}
|\input{childdoc.def}|\\
|\childdocmain{}|\\
\end{tabular}
\end{center}
at the very top of the main \LaTeX{} file,
in particular \emph{before} the |\documentclass| statement!
The argument of |\childdocmain| should be left empty
(but it must be present).

%%%%%%%%%%%%%%%%%%%%%%%%%%%%%%%%%%%%%%%%
\DescribeMacro{\childdocof}
Furthermore, add the commands
\begin{center}
\begin{tabular}{l}
|\input{childdoc.def}|\\
|\childdocof{|\textit{main}|}|\\
\end{tabular}
\end{center}
at the top of every child file \textit{child}
which is included by |\include{|\textit{child}|}|
from within the main file
(or at least for those files to be compiled individually).
The argument \textit{main} must be the filename of the main file.

There are a couple of
considerations in setting up the main and child documents:

%%%%%%%%%%%%%%%%%%%%%%%%%%%%%%%%%%%%%%%%
\paragraph{Restrictions.}

Please note the following restrictions:
\begin{itemize}
\item
|\childdocmain| must be called with one argument \textit{main}
to ensure compatibility with earlier version of the package.
It must either be empty (|\childdocmain{}|)
or precisely match the filename of the main file in which it is specified.
See \secref{sec:detection} for further information.
\item
The filename \textit{main} must be specified without the |.tex| extension.
\item
The filename \textit{main} is case sensitive
(even in case-insensitive file systems)
due to internal string comparison.
\item
The argument \textit{main} should be fully expanded, it cannot be a macro.
\item
Subdirectories and special characters should be avoided in filenames.
\item
The command |\childdocmain{|\textit{main}|}| must be followed by a whitespace.
It should not be followed immediately by another command
or by a comment mark `|%|'.
This is because the \TeX{} parser reads the token immediately following
the argument of |\childdocmain| and puts it
at the beginning of every child section;
however, a white\-space is ignored.
\end{itemize}

%%%%%%%%%%%%%%%%%%%%%%%%%%%%%%%%%%%%%%%%
\paragraph{Content of Main File.}

It is advisable to place all content in the child files included by |\include|.
Any output contained in the main file will appear in all child documents
unless suppressed manually;
it cannot be suppressed automatically by the |\includeonly| directive
and thus should normally be avoided.
A method to include some content in the main file
by means of conditional processing is described in \secref{sec:conditional}.

%%%%%%%%%%%%%%%%%%%%%%%%%%%%%%%%%%%%%%%%
\paragraph{Page Numbering.}

When only a part of the document is compiled,
the appropriate numbering of pages
(as well as other status parameters)
is determined from the |.aux| files.
The latter contain information from previous passes.
However this information needs to propagate through
all intermediate child documents.
Therefore the page numbering in child documents may well
be inconsistent until the complete document is compiled at least once.

A useful (if unconventional) way to always ensure a consistent
page numbering is to restart the numbering in each child document
and denote the pages by `\textit{child}|.|\textit{page}'
where \textit{child} represents the chapter/section number of the child file.
This can be achieved by the command
|\numberwithin{page}{|\textit{child}|}|
of the \textsf{amsmath} package
where \textit{child} can be |chapter| or |section|
depending on the chosen structuring.
Alternatively, one can modify the macro |\thepage| appropriately
and reset the counter |page| at the start of each child file.

%%%%%%%%%%%%%%%%%%%%%%%%%%%%%%%%%%%%%%%%%%%%%%%%%%%%%%%%%%%%%%%%%%%%%%%%%%%%%%%%
\subsection{Conditional Processing}
\label{sec:conditional}

The package provides a mechanism to compile different versions
of a document. To customise the versions further some conditional processing
can come in handy to distinguish which version is being compiled.
The package provides two macros to describe the compilation context:

%%%%%%%%%%%%%%%%%%%%%%%%%%%%%%%%%%%%%%%%
\DescribeMacro{\ifchilddoc}
The conditional |\ifchilddoc| distinguishes between the compilation of
child documents and the main document:
%
\begin{center}
|\ifchilddoc |\textit{child-code}| |[|\||else |\textit{main-code}]| \||fi|
\end{center}

%%%%%%%%%%%%%%%%%%%%%%%%%%%%%%%%%%%%%%%%
\DescribeMacro{\childdocname}
\DescribeMacro{\childdocjob}
The macro |\childdocname| contains the filename (without extension)
of the main or child file being processed.
Note that |\childdocjob| will always contain the name of the main file.

%%%%%%%%%%%%%%%%%%%%%%%%%%%%%%%%%%%%%%%%
\paragraph{Title Page.}

Conditional processing can be used to include a title or banner page
in the main document when proper precautions are taken.
Importantly, the code in the main file should ensure that the page counter
(as well as other status parameters which are stored in the |.aux| files)
takes the same value after the conditional processing.
Otherwise the page numbers may take divergent values
depending on which part is compiled.

For example, a title page could be declared by:
%
\begin{center}
\begin{tabular}{l}
|\ifchilddoc\||else|\\
|\addtocounter{page}{-1}|\\
\textit{code for title page}\\
|\newpage|\\
|\||fi|
\end{tabular}
\end{center}
%
A banner page for the child documents can be generated by:
%
\begin{center}
\begin{tabular}{l}
|\ifchilddoc|\\
|\addtocounter{page}{-1}|\\
\textit{code for banner page}\\
|\newpage|\\
|\||fi|
\end{tabular}
\end{center}
%
Here one could write a message such as:
\begin{center}
|This is the part \childdocname{} of \childdocjob{}.|
\end{center}

%%%%%%%%%%%%%%%%%%%%%%%%%%%%%%%%%%%%%%%%%%%%%%%%%%%%%%%%%%%%%%%%%%%%%%%%%%%%%%%%
\subsection{Flags}
\label{sec:flags}

The package makes it easy to generate different versions
of the main or child documents.
To this end compilation flags can be defined
and assigned different default values.
They will be particularly useful in conjunction
with the forwarding mechanism described in \secref{sec:forward}.

For example, it may be useful to have a flag |\version|
which can be set to |draft| or |final|.
The document source will contain some conditional code
depending on the value of |\version|.
Suppose further, the flag should default to |final| for the main file
and to |draft| for child files
which is a natural assignment for editing the document.
This is achieved by placing the following code
in the preamble of the main document
(below the |\childdocmain| directive):
%
\begin{center}
\begin{tabular}{l}
|\ifchilddoc|\\
|\providecommand{\version}{draft}|\\
|\||else|\\
|\providecommand{\version}{final}|\\
|\||fi|
\end{tabular}
\end{center}
%
The definition by |\providecommand| makes sure
that previous definitions are not overwritten.
Further statements |\providecommand{\version}{...}|
can thus be added before the above code to override it.

For the main file, one might add a line
(between |\childdocmain| and the above block)
%
\begin{center}
|%\ifchilddoc\||else\providecommand{\version}{draft}\||fi|
\end{center}
%
which can be uncommented to produce a draft version.
Likewise one can add a line to the very top of a child file
(above the |\childdocof{|\textit{main}|}| directive)
%
\begin{center}
|%\providecommand{\version}{final}|
\end{center}
%
which can be uncommented to produce the final version of this child document.

%%%%%%%%%%%%%%%%%%%%%%%%%%%%%%%%%%%%%%%%%%%%%%%%%%%%%%%%%%%%%%%%%%%%%%%%%%%%%%%%
\subsection{Forwarding}
\label{sec:forward}

Different versions of the main or child documents
using compilation flags as described in \secref{sec:flags}
can be (permanently) stored in different files
for convenient compilation, viewing and distribution.
To this end, the package defines a command
to pass on compilation to a different file:

%%%%%%%%%%%%%%%%%%%%%%%%%%%%%%%%%%%%%%%%
\DescribeMacro{\childdocforward}
The command |\childdocforward| redirects processing to
another source file:
%
\begin{center}
\begin{tabular}{l}
|\input{childdoc.def}|\\
|\childdocforward[|\textit{main}|]{|\textit{dest}|}|\\
\end{tabular}
\end{center}
%
The argument \textit{dest} is the destination file
(without extension).
It should be the main file or one of the child files.
Note that further \textsf{childdoc} directives
such as |\childdocof| and |\childdocforward|
in the indicated file will be processed in this form.
The optional argument \textit{main}
passes on directly to the main file \textit{main}
while pretending to compile the child \textit{dest}.
This form behaves as if \textit{dest}
issues |\childdocof{|\textit{main}|}| right away,
and no further \textsf{childdoc} directives will be processed.

%%%%%%%%%%%%%%%%%%%%%%%%%%%%%%%%%%%%%%%%
\DescribeMacro{\...prefix}
In the alternative form |\childdocforwardprefix|,
%
\begin{center}
\begin{tabular}{l}
|\input{childdoc.def}|\\
|\childdocforwardprefix[|\textit{main}|]{|\textit{prefix}|}{|\textit{dest}|}|
\end{tabular}
\end{center}
%
the destination file is determined by a pattern
depending on the current file:
To make this work, the current file must be called
`{\textit{prefix}\hspace{0.2em}\textit{suffix}}'
with \textit{prefix} matching precisely the argument.
Processing is then passed on to the file
`{\textit{dest}\hspace{0.2em}\textit{suffix}}'.
Surely, the same effect is achieved by
directly specifying the
argument `{\textit{dest}\hspace{0.2em}\textit{suffix}}'
in the first form.
However, that requires to set up a different file
for each child. With the alternative form of the command
all these files can have exactly the same content
which simplifies setting them up and maintaining them.

For example, the following file |draft.tex|
with a compilation flag |\version| as described in \secref{sec:flags}
compiles the main document as a draft:
%
\begin{center}
\begin{tabular}{l}
|\def\version{draft}|\\
|\input{childdoc.def}|\\
|\childdocforward{|\textit{main}|}|
\end{tabular}
\end{center}
%
Likewise, the following files |final|\textit{nn}|.tex|
compile the final version of the child document
|child|\textit{nn}|.tex|:
%
\begin{center}
\begin{tabular}{l}
|\def\version{final}|\\
|\input{childdoc.def}|\\
|\childdocforwardprefix{final}{child}|
\end{tabular}
\end{center}
%

Note that when several versions of a main file and/or of each child file
are to be generated, it may be convenient to set up a |Makefile| or
shell script to automatise the process.

%%%%%%%%%%%%%%%%%%%%%%%%%%%%%%%%%%%%%%%%%%%%%%%%%%%%%%%%%%%%%%%%%%%%%%%%%%%%%%%%
\subsection{Command Line Processing}
\label{sec:commandline}

The effect of redirection files can also be achieved by invoking
the \LaTeX{} compiler with a more elaborate command line.
Most conveniently this should be done as part
of a shell script or a |Makefile|.

When using \textsf{childdoc} in the main file, the following
command lines effectively perform a redirection
(note that depending on the shell being used,
backslashes may have to be doubled: `|\|' $\to$ `|\\|'):
%
\begin{center}
|... -jobname "|\textit{target}|" |\\|"|[\textit{flags}]%
|\input{childdoc.def}\childdocforward[|\textit{main}|]{|\textit{dest}|}"|
\end{center}
%
Here \textit{target} is the name of the output file,
\textit{main} is the name of the main file
and \textit{dest} is the name of the main or child file to be processed
(all filenames without extensions).
The optional argument \textit{main} can be omitted
if \textit{main} matches \textit{dest}.
Optionally, compilation \textit{flags} can be defined via |\def| commands.
This command line makes the \TeX{} engine believe
it is compiling the file \textit{target}
whose content is specified as the latter parameter.
The provided code then forwards the processing to
\textit{main} or \textit{dest} as described in \secref{sec:forward}.

%%%%%%%%%%%%%%%%%%%%%%%%%%%%%%%%%%%%%%%%%%%%%%%%%%%%%%%%%%%%%%%%%%%%%%%%%%%%%%%%
\subsection{Include by Input}
\label{sec:input}

Including child documents by |\include| has some restrictions by design.
Most notably, the content of a child document always occupies
its own set of pages; pages cannot be shared between child documents.
Usually, this behaviour makes perfect sense
because each child document contain an essential part of the document.
However, in some situations it may be desirable to compose
a document from a collection of parts
without having mandatory page breaks between then.
For this case, the package
provides a mechanism to include parts
by |\input| which can also be processed individually.
However, by construction this mechanism
requires manual handling of the content to be output.

%%%%%%%%%%%%%%%%%%%%%%%%%%%%%%%%%%%%%%%%
\DescribeMacro{\ifchilddocmanual}
The main file should be prepared as usual, see \secref{sec:include}.
However, the document body must make a distinction
between processing of an individual part and of the main document, e.g.:
%
\begin{center}
\begin{tabular}{l}
|\ifchilddocmanual|\\
|\input{\childdocname}|\\
|\||else|\\
\textit{document body with }|\input{|\textit{part}|}|\\
|\||fi|
\end{tabular}
\end{center}
%
The conditional |\ifchilddocmanual| is true whenever
a part to be included by |\input| is being compiled,
and the name of the part is stored in |\childdocname|.

%%%%%%%%%%%%%%%%%%%%%%%%%%%%%%%%%%%%%%%%
\DescribeMacro{\childdocby}
Each part to be included by |\input| should start with:
%
\begin{center}
\begin{tabular}{l}
|\input{childdoc.def}|\\
|\childdocby{|\textit{main}|}|\\
\end{tabular}
\end{center}
%
The directive |\childdocby| is similar to |\childdocof|
described in \secref{sec:include},
but the subsequent selection of content must be done manually.
To that end, both |\ifchilddoc| and |\ifchilddocmanual|
will be true upon processing of a part,
and the name of the part is stored in |\childdocname|.
Note that |\jobname| will be set to the filename of the current part
so that each part receives an individual |.aux| file
that does not interfere with the |.aux| file(s) of the main document.
This behaviour can be altered by the alternative form
|\childdocby[*]{|\textit{main}|}| (with a non-empty optional argument)
which uses the |.aux| file of the main document
by setting |\jobname| to \textit{main}.

%%%%%%%%%%%%%%%%%%%%%%%%%%%%%%%%%%%%%%%%%%%%%%%%%%%%%%%%%%%%%%%%%%%%%%%%%%%%%%%%
\subsection{Driver Development}
\label{sec:driver}

The \textsf{childdoc} mechanism can also be use for the development
of definition files such as \LaTeX{} styles or classes.
This case differs from the above setup with multiple parts
included by |\include| in that no |\includeonly| should be invoked.
This can be achieved by starting the include file
(before |\ProvidesPackage|) with:
%
\begin{center}
\begin{tabular}{l}
|\input{childdoc.def}|\\
|\childdocforward{|\textit{main}|}|\\
\end{tabular}
\end{center}
%
or alternatively with:
%
\begin{center}
\begin{tabular}{l}
|\input{childdoc.def}|\\
|\childdocby{|\textit{main}|}|\\
\end{tabular}
\end{center}
%
Both forms have slightly different effects as described above.
The main file is prepared as usual, see \secref{sec:include}.

%%%%%%%%%%%%%%%%%%%%%%%%%%%%%%%%%%%%%%%%%%%%%%%%%%%%%%%%%%%%%%%%%%%%%%%%%%%%%%%%
\subsection{Legacy Detection}
\label{sec:detection}

The directive |\childdocmain| in the main file can detect
whether the complete document or merely a child is to be compiled
even without using the directive |\childdocof|.
This method is deprecated because it is less robust
and there is no compelling reason to use it;
it is merely provided for backward compatibility
and it may be removed in future versions.

If the detection mechanism is to be used,
it is mandatory to correctly specify
the filename of the main file as the argument of |\childdocmain|:
%
\begin{center}
\begin{tabular}{l}
|\input{childdoc.def}|\\
|\childdocmain{|\textit{main}|}|\\
\end{tabular}
\end{center}
%
If |\jobname| does not match the argument \textit{main} of |\childdocmain|,
it is assumed that |\jobname| points to the child file to be compiled.
When using |\childdocmain| with the main file specified as argument,
it suffices to start a child file
with just |\input{|\textit{main}|}|
without loading of the package and using |\childdocof|.
If instead all processing is done
with the appropriate \textsf{childdoc} directives,
the argument of \textit{main} of |\childdocmain| can be empty.

An alternative version of the command line processing described
in \secref{sec:commandline} using the detection mechanism reads:
%
\begin{center}
|... -jobname "|\textit{target}|" "|[\textit{flags}]%
[|\def\jobname{|\textit{dest}|}|]|\input{|\textit{main}|}"|
\end{center}

%%%%%%%%%%%%%%%%%%%%%%%%%%%%%%%%%%%%%%%%%%%%%%%%%%%%%%%%%%%%%%%%%%%%%%%%%%%%%%%%
\subsection{Manual Code}
\label{sec:manual}

In case one cannot be certain whether the definitions file |childdoc.def|
is installed on the target \TeX{} distribution
and one prefers not to ship it,
it is conceivable to paste a few relevant commands into the sources.

To that end, drop all statements |\input{childdoc.def}|
and perform the replacements as outlined below.
Instead of |\childdocmain{|\textit{main}|}| add the following code
to the top of the main file:
%
\begin{center}
\begin{tabular}{l}
|\||ifdefined\childdocname\endinput\||fi\newif\ifchilddoc|\\
|\edef\childdocname{\scantokens\expandafter{\jobname\noexpand}}|\\
|\def\childdocmain{|\textit{main}|}\||ifx\childdocmain\childdocname\||else|\\
|\childdoctrue\includeonly{\childdocname}\let\jobname\childdocmain\||fi|\\
\end{tabular}
\end{center}
%
Instead of |\childdocof{|\textit{main}|}| just include the main file
at the top of each child file:
%
\begin{center}
|\input{|\textit{main}|}|
\end{center}
%
A simple redirection |\childdocforward{|\textit{dest}|}| is achieved by:
%
\begin{center}
|\def\jobname{|\textit{dest}|}\input{\jobname}|
\end{center}
%
The redirection with prefix
|\childdocforwardprefix[|\textit{prefix}|]{|\textit{dest}|}|
is accomplished by:
%
\begin{center}
\begin{tabular}{l}
|{\edef\jobname{\scantokens\expandafter{\jobname\noexpand}}|\\
|\def\redirectjob |\textit{prefix}|#1~~~{\gdef\jobname{|\textit{dest}|#1}}|\\
|\expandafter\redirectjob\jobname~~~}\input{\jobname}|
\end{tabular}
\end{center}

In an alternative approach,
child documents can be compiled by a specific command line
without additional code or specific definitions:
%
\begin{center}
|... -jobname "|\textit{target}|" "|[\textit{flags}]%
|\includeonly{|\textit{dest}|}\input{|\textit{main}|}"|
\end{center}
%

%%%%%%%%%%%%%%%%%%%%%%%%%%%%%%%%%%%%%%%%%%%%%%%%%%%%%%%%%%%%%%%%%%%%%%%%%%%%%%%%
%%%%%%%%%%%%%%%%%%%%%%%%%%%%%%%%%%%%%%%%%%%%%%%%%%%%%%%%%%%%%%%%%%%%%%%%%%%%%%%%
\section{Information}

%%%%%%%%%%%%%%%%%%%%%%%%%%%%%%%%%%%%%%%%%%%%%%%%%%%%%%%%%%%%%%%%%%%%%%%%%%%%%%%%
\subsection{Copyright}

Copyright \copyright{} 2017--2018 Niklas Beisert

This work may be distributed and/or modified under the
conditions of the \LaTeX{} Project Public License, either version 1.3
of this license or (at your option) any later version.
The latest version of this license is in
  \url{http://www.latex-project.org/lppl.txt}
and version 1.3 or later is part of all distributions of \LaTeX{}
version 2005/12/01 or later.

This work has the LPPL maintenance status `maintained'.

The Current Maintainer of this work is Niklas Beisert.

This work consists of the files |README.txt|, |childdoc.ins| and |childdoc.dtx|
as well as the derived files |childdoc.def|, |cdocsamp.tex|
with |cdocsch1.tex|, |cdocsch2.tex|, |cdocspt3.tex|, |cdocspt4.tex|,
|cdocsdrf.tex|, |cdocsfn1.tex|, |cdocsfn2.tex|
as well as |childdoc.pdf|.

%%%%%%%%%%%%%%%%%%%%%%%%%%%%%%%%%%%%%%%%%%%%%%%%%%%%%%%%%%%%%%%%%%%%%%%%%%%%%%%%
\subsection{Files and Installation}

The package consists of the files:
%
\begin{center}
\begin{tabular}{ll}
    |README.txt|   & readme file \\
    |childdoc.ins| & installation file \\
    |childdoc.dtx| & source file \\
    |childdoc.def| & definition file \\
    |cdocsamp.tex| & sample main file \\
    |cdocsch1.tex| & sample include file \\
    |cdocsch2.tex| & sample include file \\
    |cdocspt3.tex| & sample part file \\
    |cdocspt4.tex| & sample part file \\
    |cdocsdrf.tex| & sample redirection file \\
    |cdocsfn1.tex| & sample redirection file \\
    |cdocsfn2.tex| & sample redirection file \\
    |childdoc.pdf| & manual
\end{tabular}
\end{center}
%
The distribution consists of the files
|README.txt|, |childdoc.ins| and |childdoc.dtx|.
%
\begin{itemize}
\item
Run (pdf)\LaTeX{} on |childdoc.dtx|
to compile the manual |childdoc.pdf| (this file).
\item
Run \LaTeX{} on |childdoc.ins| to create the definitions file |childdoc.def|
and the sample |cdocsamp.tex| with include files
|cdocsch1.tex|, |cdocsch2.tex|, |cdocspt3.tex|, |cdocspt4.tex|,
|cdocsdrf.tex|, |cdocsfn1.tex|, |cdocsfn2.tex|.
Then copy the file |childdoc.def| to an appropriate directory of your \LaTeX{}
distribution, e.g.\ \textit{texmf-root}|/tex/latex/childdoc|.
\end{itemize}

%%%%%%%%%%%%%%%%%%%%%%%%%%%%%%%%%%%%%%%%%%%%%%%%%%%%%%%%%%%%%%%%%%%%%%%%%%%%%%%%
\subsection{Related CTAN Packages}

There are several other packages which offer a similar functionality:
%
\begin{itemize}
\item
The packages
\href{http://ctan.org/pkg/docmute}{\textsf{docmute}},
\href{http://ctan.org/pkg/includex}{\textsf{includex}} and
\href{http://ctan.org/pkg/standalone}{\textsf{standalone}}
provide commands to include only the document body of
a child file thus allowing both files to be compiled individually.
\item
The packages \href{http://ctan.org/pkg/subdocs}{\textsf{subdocs}}
and \href{http://ctan.org/pkg/subfiles}{\textsf{subfiles}}
provide structures in which the main and child documents can be
encapsulated and allowing them to be compiled individually.
The inclusion mechanism is different from the conventional |\include|.
\item
The package \href{http://ctan.org/pkg/combine}{\textsf{combine}}
is an elaborate solution to combine several documents into one.
\end{itemize}
%
See also the CTAN topic \href{http://ctan.org/topic/subdocs}{\textsf{subdocs}}
for further related packages.
The present package differs from the above solutions in that
a document structure constructed with the conventional |\include| mechanism
just needs two extra commands at the top of every file
such that all constituent files can be compiled individually.

%%%%%%%%%%%%%%%%%%%%%%%%%%%%%%%%%%%%%%%%%%%%%%%%%%%%%%%%%%%%%%%%%%%%%%%%%%%%%%%%
%\subsection{Feature Suggestions}
%
%The following is a list of features which may be useful for future
%versions of this package:
%%
%\begin{itemize}
%\item
%\ldots
%\end{itemize}

%%%%%%%%%%%%%%%%%%%%%%%%%%%%%%%%%%%%%%%%%%%%%%%%%%%%%%%%%%%%%%%%%%%%%%%%%%%%%%%%
\subsection{Revision History}

%%%%%%%%%%%%%%%%%%%%%%%%%%%%%%%%%%%%%%%%
\paragraph{v2.0:} 2018/12/30

\begin{itemize}
\item
immediate forward processing
\item
added |\childdocby| mechanism
\item
manual restructured
\end{itemize}

%%%%%%%%%%%%%%%%%%%%%%%%%%%%%%%%%%%%%%%%
\paragraph{v1.6:} 2018/01/17

\begin{itemize}
\item
application for development of include files
\item
corrections to manual
\end{itemize}

%%%%%%%%%%%%%%%%%%%%%%%%%%%%%%%%%%%%%%%%
\paragraph{v1.5:} 2017/05/21

\begin{itemize}
\item
more complete structuring introduced
\item
|\childdocof| introduced
\item
|\childdoc| renamed to |\childdocmain|
\item
|\childredirect| renamed to |\childdocforward| and |\childdocforwardprefix|
and functionality expanded
\end{itemize}

%%%%%%%%%%%%%%%%%%%%%%%%%%%%%%%%%%%%%%%%
\paragraph{v1.0:} 2017/04/27

\begin{itemize}
\item
manual and install package
\item
first version published on CTAN
\end{itemize}

%%%%%%%%%%%%%%%%%%%%%%%%%%%%%%%%%%%%%%%%
\paragraph{v0.6:} 2017/04/26

\begin{itemize}
\item
redirection mechanism added
\end{itemize}

%%%%%%%%%%%%%%%%%%%%%%%%%%%%%%%%%%%%%%%%
\paragraph{v0.5:} 2017/04/26

\begin{itemize}
\item
functionality in definition file
\end{itemize}


%%%%%%%%%%%%%%%%%%%%%%%%%%%%%%%%%%%%%%%%%%%%%%%%%%%%%%%%%%%%%%%%%%%%%%%%%%%%%%%%
%%%%%%%%%%%%%%%%%%%%%%%%%%%%%%%%%%%%%%%%%%%%%%%%%%%%%%%%%%%%%%%%%%%%%%%%%%%%%%%%
%%%%%%%%%%%%%%%%%%%%%%%%%%%%%%%%%%%%%%%%%%%%%%%%%%%%%%%%%%%%%%%%%%%%%%%%%%%%%%%%
\appendix

\settowidth\MacroIndent{\rmfamily\scriptsize 000\ }

 \DocInput{childdoc.dtx}

\end{document}
%</driver>
% \fi
%
% %%%%%%%%%%%%%%%%%%%%%%%%%%%%%%%%%%%%%%%%%%%%%%%%%%%%%%%%%%%%%%%%%%%%%%%%%%%%%%
% %%%%%%%%%%%%%%%%%%%%%%%%%%%%%%%%%%%%%%%%%%%%%%%%%%%%%%%%%%%%%%%%%%%%%%%%%%%%%%
% \section{Sample}
%\iffalse
%<*samplemain>
%\fi
%
% The following presents a sample document
% with two chapters, two parts, a title page,
% a compile flag as well as three forwarding files to set the flag.
% It consists of eight |.tex| files:
% \begin{center}
% \begin{tabular}{ll}
% |cdocsamp.tex|&main file\\
% |cdocsch1.tex|&include file for chapter 1\\
% |cdocsch2.tex|&include file for chapter 2\\
% |cdocspt3.tex|&include file for part 3\\
% |cdocspt4.tex|&include file for part 4\\
% |cdocsdrf.tex|&forwarding file for main file in draft mode\\
% |cdocsfi1.tex|&forwarding file for final version of chapter 1\\
% |cdocsfi2.tex|&forwarding file for final version of chapter 2\\
% \end{tabular}
% \end{center}
% Each of the eight files can be compiled directly by the \LaTeX{} compiler.
%
% %%%%%%%%%%%%%%%%%%%%%%%%%%%%%%%%%%%%%%
% \paragraph{Main File.}
%
% The main file is called |cdocsamp.tex|.
%
% Load the \textsf{childdoc} definitions and
% declare the filename for the main document:
%    \begin{macrocode}
\input{childdoc.def}
\childdocmain{}
%    \end{macrocode}

% Optional override for |\version| flag:
%    \begin{macrocode}
%%\ifchilddoc\else\providecommand{\version}{draft}\fi
%    \end{macrocode}

% Define the default values for the |\version| flag
% (|final| for the main file and |draft| for childs):
%    \begin{macrocode}
\ifchilddoc
\providecommand{\version}{draft}
\else
\providecommand{\version}{final}
\fi
%    \end{macrocode}

% Load the standard document class:
%    \begin{macrocode}
\documentclass[12pt]{article}
%    \end{macrocode}

% Start the document body:
%    \begin{macrocode}
\begin{document}
%    \end{macrocode}

% Declare a title page.
% Print title, part of document being processed and version flag:
%    \begin{macrocode}
\addtocounter{page}{-1}
\begin{center}
{\LARGE\bfseries{}childdoc example\par}
\vspace{1cm}
\ifchilddoc
\ifchilddocmanual part\else chapter\fi:
`\childdocname' of `\childdocjob'\par
\else
main document: `\childdocjob'\par
\fi
version: \version\par
\end{center}
\newpage
%    \end{macrocode}

% Manually include selected file,
% otherwise process as usual:
%    \begin{macrocode}
\ifchilddocmanual
\section*{part `\childdocname'}
\input{\childdocname}
\else
%    \end{macrocode}

% Include the two chapters:
%    \begin{macrocode}
\include{cdocsch1}
\include{cdocsch2}
%    \end{macrocode}

% Include the two parts unless only chapters should be displayed:
%    \begin{macrocode}
\ifchilddoc\else
\section{part three}
\input{cdocspt3}
\section{part four}
\input{cdocspt4}
\fi
%    \end{macrocode}

% Process as usual until here:
%    \begin{macrocode}
\fi
%    \end{macrocode}

% End of document body:
%    \begin{macrocode}
\end{document}
%    \end{macrocode}
%\iffalse
%</samplemain>
%\fi
%
% %%%%%%%%%%%%%%%%%%%%%%%%%%%%%%%%%%%%%%
% \paragraph{Chapter Include Files.}
%
% The include files are called |cdocsch1.tex| and |cdocsch2.tex|.
%
%\iffalse
%<*samplechap1|samplechap2>
%\fi

% Optional override for |\version| flag:
%    \begin{macrocode}
%%\providecommand{\version}{final}
%    \end{macrocode}

% Include the main document:
%    \begin{macrocode}
\input{childdoc.def}
\childdocof{cdocsamp}
%    \end{macrocode}

%\iffalse
%</samplechap1|samplechap2>
%\fi
%
%\iffalse
%<*samplechap1>
%\fi
% Some text for chapter 1:
%    \begin{macrocode}
\section{one}
some text in chapter one
%    \end{macrocode}

%\iffalse
%</samplechap1>
%\fi
% Some text for chapter 2:
%\iffalse
%<*samplechap2>
%\fi
%    \begin{macrocode}
\section{two}
more text in chapter two
%    \end{macrocode}

%\iffalse
%</samplechap2>
%\fi
%
% %%%%%%%%%%%%%%%%%%%%%%%%%%%%%%%%%%%%%%
% \paragraph{Part Include Files.}
%
% The include files are called |cdocspt3.tex| and |cdocspt4.tex|.
%
%\iffalse
%<*samplepart3|samplepart4>
%\fi

% Optional override for |\version| flag:
%    \begin{macrocode}
%%\providecommand{\version}{final}
%    \end{macrocode}

% Include the main document:
%    \begin{macrocode}
\input{childdoc.def}
\childdocby{cdocsamp}
%    \end{macrocode}

%\iffalse
%</samplepart3|samplepart4>
%\fi
%
%\iffalse
%<*samplepart3>
%\fi
% Some text for part 3:
%    \begin{macrocode}
some text in part three
%    \end{macrocode}

%\iffalse
%</samplepart3>
%\fi
% Some text for part 4:
%\iffalse
%<*samplepart4>
%\fi
%    \begin{macrocode}
more text in part four
%    \end{macrocode}

%\iffalse
%</samplepart4>
%\fi
%
% %%%%%%%%%%%%%%%%%%%%%%%%%%%%%%%%%%%%%%
% \paragraph{Forwarding for a Complete Draft.}
%
% The following forwarding file |cdocsdrf.tex|
% compiles the main document in draft mode:
%\iffalse
%<*sampledraft>
%\fi
%    \begin{macrocode}
\def\version{draft}
\input{childdoc.def}
\childdocforward{cdocsamp}
%    \end{macrocode}

%\iffalse
%</sampledraft>
%\fi
%
% %%%%%%%%%%%%%%%%%%%%%%%%%%%%%%%%%%%%%%
% \paragraph{Forwarding for Final Version of the Chapters.}
%
% The following forwarding files |cdocsfn1.tex| and |cdocsfn2.tex|
% (with identical content)
% compile the final versions of the child documents
% |cdocsch1.tex| and |cdocsch2.tex|, respectively:
%\iffalse
%<*samplefinal>
%\fi
%    \begin{macrocode}
\def\version{final}
\input{childdoc.def}
\childdocforwardprefix[cdocsamp]{cdocsfn}{cdocsch}
%    \end{macrocode}

%\iffalse
%</samplefinal>
%\fi
%
% %%%%%%%%%%%%%%%%%%%%%%%%%%%%%%%%%%%%%%
% \paragraph{Command Line Processing.}
%
% The following three command lines generate the output files
% |cdocscld|, |cdocscl1| and |cdocscl2|
% which should be identical to
% |cdocsdrf|, |cdocsch1| and |cdocsfn2|, respectively:
% \begin{center}
% \begin{tabular}{l}
% |latex -jobname cdocscld \|\\
% |  "\def\version{draft}\input{childdoc.def}\childdocforward{cdocsamp}"|\\
% |latex -jobname cdocscl1 \|\\
% |  "\input{childdoc.def}\childdocforward[cdocsamp]{cdocsch1}"|\\
% |latex -jobname cdocscl2 \|\\
% |  "\def\version{final}\input{childdoc.def}\childdocforward{cdocsch2}"|
% \end{tabular}
% \end{center}
% Note that the trailing backslash on each first line
% merely continues the input to the second line
% (for convenient cut ant paste).
% Furthermore, the command |latex| can be replaced by any
% of its alternative versions such as |pdflatex|.
%
% %%%%%%%%%%%%%%%%%%%%%%%%%%%%%%%%%%%%%%%%%%%%%%%%%%%%%%%%%%%%%%%%%%%%%%%%%%%%%%
% %%%%%%%%%%%%%%%%%%%%%%%%%%%%%%%%%%%%%%%%%%%%%%%%%%%%%%%%%%%%%%%%%%%%%%%%%%%%%%
% \section{Implementation}
%\iffalse
%<*package>
%\fi
%
% This section describes the definitions file |childdoc.def|.

% The definitions cannot be loaded using |\usepackage| or |\RequirePackage|
% which has a mechanism to prevent loading a style file more than once.
% When loading the definitions by means of |\input|
% multiple instances have to be prevented manually:
%\iffalse
%This code needs to be before the `\ProvidesFile' directive
%which is defined at the beginning of this file.
%Therefore it is also placed there and commented out here.
%</package>
%<*discard>
%\fi
%    \begin{macrocode}
\ifdefined\childdocmain\endinput\fi
%    \end{macrocode}
%\iffalse
%</discard>
%<*package>
%\fi
%
% \macro{\ifchilddoc}
% \macro{\ifchilddocmanual}
% The conditional |\ifchilddoc| tells whether a
% child (true) or main (false) document is being compiled.
% The conditional |\ifchilddocmanual| tells whether
% the |\includeonly| mechanism is used (false) or
% the selection of child files must be performed manually (true).
% The definitions initialise to false:
%    \begin{macrocode}
\newif\ifchilddoc
\newif\ifchilddocmanual
%    \end{macrocode}

% \macro{\childdocname}
% \macro{\childdocjob}
% The macro |\childdocname| stores the name of the main document
% to be compiled. The macro |\childdocjob| stores the name of
% the document on which the \LaTeX{} compiler was originally invoked.
% The content of |\jobname| cannot be compared
% to filenames specified in the source due to different catcodes.
% The following code rescans |\jobname|, stores the result
% in |\childdocname| and saves a copy in |\childdocjob|:
%    \begin{macrocode}
\edef\childdocname{\scantokens\expandafter{\jobname\noexpand}}
\let\childdocjob\childdocname
%    \end{macrocode}

% \macro{\childdocdisable}
% The macro |\childdocdisable| prevents the main file
% from being processed more than once.
% At this stage, the main document command |\childdocmain|
% is assumed to be called once again where it should do nothing.
% Any subsequent call to it should prevent
% a secondary processing of the main document
% It overwrites the forwarding commands
% |\childdocof| and |\childdocforward|
% with empty macros to prevent further inclusions of the main document:
%    \begin{macrocode}
\newcommand{\childdocdisable}
{
  \renewcommand{\childdocmain}[1]{\renewcommand{\childdocmain}[1]{\endinput}}
  \renewcommand{\childdocof}[1]{}
  \renewcommand{\childdocby}[2][]{}
  \renewcommand{\childdocforward}[2][]{}
  \renewcommand{\childdocdisable}{}
}
%    \end{macrocode}

% \macro{\childdocmain}
% The macro |\childdocmain| is to be called at the top of the main file
% with nothing or the main filename (without extension) as argument.
% First, it breaks loops.
% If the argument is not empty and does not match |\childdocname|
% (which is set by the first inclusion of |childdoc.def|),
% |\ifchilddoc| is set to true, |\includeonly| is applied to the child file
% and |\jobname| is set to the main file
% (for proper handling of |.aux| files):
%    \begin{macrocode}
\newcommand{\childdocmain}[1]
{
  \childdocdisable\childdocmain{}
  \if?#1?\else
    \begingroup
      \def\childdoctmp{#1}
      \ifx\childdoctmp\childdocname
        \def\childdoctmp{}
      \else
        \def\childdoctmp
        {
          \childdoctrue
          \includeonly{\childdocname}
          \def\childdocjob{#1}
          \def\jobname{#1}
        }
      \fi
      \expandafter
    \endgroup
    \childdoctmp
  \fi
}
%    \end{macrocode}

% \macro{\childdocof}
% The command |\childdocof| redirects
% compilation to the main file |#1|.
%    \begin{macrocode}
\newcommand{\childdocof}[1]
{
  \childdocdisable
  \childdoctrue
  \includeonly{\childdocname}
  \def\jobname{#1}
  \def\childdocjob{#1}
  \input{#1}
}
%    \end{macrocode}

% \macro{\childdocby}
% The command |\childdocby| ....
%    \begin{macrocode}
\newcommand{\childdocby}[2][]
{
  \childdocdisable
  \childdoctrue
  \childdocmanualtrue
  \if?#1?\else
    \def\jobname{#2}
  \fi
  \def\childdocjob{#2}
  \input{#2}
  \endinput
}
%    \end{macrocode}

% \macro{\childdocforward}
% The command |\childdocforward| redirects
% compilation to the main file or
% (if the optional argument is given) a child file.
% Parameters are set as if the main file
% or a child file starting with |\childdocof| was compiled.
% Then compilation is handed over to the main file:
%    \begin{macrocode}
\newcommand{\childdocforward}[2][]
{
  \begingroup
    \if?#1?
      \def\childdoctmp
      {
        \def\childdocname{#2}
        \def\childdocjob{#2}
        \def\jobname{#2}
        \input{#2}
        \endinput
      }
    \else
      \def\childdoctmp
      {
        \childdocdisable
        \def\childdocname{#2}
        \childdoctrue
        \includeonly{#2}
        \def\childdocjob{#1}
        \def\jobname{#1}
        \input{#1}
        \endinput
      }
    \fi
    \expandafter
  \endgroup
  \childdoctmp
}
%    \end{macrocode}

% \macro{\childdocforwardprefix}
% The command |\childdocforwardprefix| redirects
% compilation to the main or a child file by means of a pattern.
% The prefix |#1| in the current filename is replaced by |#2|
% and the suffix of the current filename is kept
% (it is assumed that the filename does not contain the substring `|~~~|'
% which is used as a delimiter).
% Compilation is handed over to the new file by |\childdocforward|:
%    \begin{macrocode}
\newcommand{\childdocforwardprefix}[3][]
{
  \begingroup
    \def\childdocextract #2##1~~~{\def\childdoctmp{\childdocforward[#1]{#3##1}}}
    \expandafter\childdocextract\childdocname~~~
    \expandafter
  \endgroup
  \childdoctmp
}
%    \end{macrocode}

% \macro{\childdoc}
% The deprecated macro |\childdoc| is a legacy version of |\childdocmain|:
%    \begin{macrocode}
\newcommand{\childdoc}{\childdocmain}
%    \end{macrocode}

% \macro{\childdocredirect}
% The deprecated macro |\childdocredirect| is a legacy version
% of |\childdocforward| and |\childdocforwardprefix|:
%    \begin{macrocode}
\newcommand{\childdocredirect}[2][]
{
  \begingroup
    \if?#1?
      \def\childdoctmp{\childdocforward{#2}}
    \else
      \def\childdoctmp{\childdocforwardprefix{#1}{#2}}
    \fi
    \expandafter
  \endgroup
  \childdoctmp
}
%    \end{macrocode}

%\iffalse
%</package>
%\fi
%
\endinput
|\\
|\childdocby{|\textit{main}|}|\\
\end{tabular}
\end{center}
%
Both forms have slightly different effects as described above.
The main file is prepared as usual, see \secref{sec:include}.

%%%%%%%%%%%%%%%%%%%%%%%%%%%%%%%%%%%%%%%%%%%%%%%%%%%%%%%%%%%%%%%%%%%%%%%%%%%%%%%%
\subsection{Legacy Detection}
\label{sec:detection}

The directive |\childdocmain| in the main file can detect
whether the complete document or merely a child is to be compiled
even without using the directive |\childdocof|.
This method is deprecated because it is less robust
and there is no compelling reason to use it;
it is merely provided for backward compatibility
and it may be removed in future versions.

If the detection mechanism is to be used,
it is mandatory to correctly specify
the filename of the main file as the argument of |\childdocmain|:
%
\begin{center}
\begin{tabular}{l}
|% \iffalse
%
% childdoc.dtx Copyright (C) 2017-2018 Niklas Beisert
%
% This work may be distributed and/or modified under the
% conditions of the LaTeX Project Public License, either version 1.3
% of this license or (at your option) any later version.
% The latest version of this license is in
%   http://www.latex-project.org/lppl.txt
% and version 1.3 or later is part of all distributions of LaTeX
% version 2005/12/01 or later.
%
% This work has the LPPL maintenance status `maintained'.
%
% The Current Maintainer of this work is Niklas Beisert.
%
% This work consists of the files childdoc.dtx and childdoc.ins
% and the derived files childdoc.def and cdocsamp.tex with
% cdocsch1.tex, cdocsch2.tex, cdocsdrf.tex, cdocsfn1.tex, cdocsfn2.tex.
%
%<package>\ifdefined\childdocmain\endinput\fi
%<package>\ProvidesFile{childdoc.def}[2018/12/30 v2.0 child document driver]
%<samplemain>\ProvidesFile{cdocsamp.tex}[2018/12/30 v2.0 sample for childdoc]
%<*driver>
%\ProvidesFile{childdoc.drv}[2018/12/30 v2.0 childdoc reference manual file]
\PassOptionsToClass{10pt,a4paper}{article}
\documentclass{ltxdoc}

\usepackage[margin=35mm]{geometry}
\usepackage{hyperref}
\usepackage{hyperxmp}
\usepackage[usenames]{color}

\hypersetup{colorlinks=true}
\hypersetup{pdfstartview=FitH}
\hypersetup{pdfpagemode=UseNone}
\hypersetup{pdfsource={}}
\hypersetup{pdflang={en-UK}}
\hypersetup{pdfcopyright={Copyright 2017-2018 Niklas Beisert.
  This work may be distributed and/or modified under the
  conditions of the LaTeX Project Public License, either version 1.3
  of this license or (at your option) any later version.}}
\hypersetup{pdflicenseurl={http://www.latex-project.org/lppl.txt}}
\hypersetup{pdfcontactaddress={ETH Zurich, ITP, HIT K,
  Wolfgang-Pauli-Strasse 27}}
\hypersetup{pdfcontactpostcode={8093}}
\hypersetup{pdfcontactcity={Zurich}}
\hypersetup{pdfcontactcountry={Switzerland}}
\hypersetup{pdfcontactemail={nbeisert@itp.phys.ethz.ch}}
\hypersetup{pdfcontacturl={http://people.phys.ethz.ch/\xmptilde nbeisert/}}

\newcommand{\secref}[1]{\hyperref[#1]{section \ref*{#1}}}

\parskip1ex
\parindent0pt
\let\olditemize\itemize
\def\itemize{\olditemize\parskip0pt}

\begin{document}

\title{The \textsf{childdoc} Package}
\hypersetup{pdftitle={The childdoc Package}}
\author{Niklas Beisert\\[2ex]
  Institut f\"ur Theoretische Physik\\
  Eidgen\"ossische Technische Hochschule Z\"urich\\
  Wolfgang-Pauli-Strasse 27, 8093 Z\"urich, Switzerland\\[1ex]
  \href{mailto:nbeisert@itp.phys.ethz.ch}
  {\texttt{nbeisert@itp.phys.ethz.ch}}}
\hypersetup{pdfauthor={Niklas Beisert}}
\hypersetup{pdfsubject={Manual for the LaTeX2e Package childdoc}}
\date{30 December 2018, \textsf{v2.0}}
\maketitle

\begin{abstract}\noindent
\textsf{childdoc} is a \LaTeXe{} package
that enables the direct compilation
of document sections included by |\include|
to individual files.
\end{abstract}

\begingroup
\parskip0ex
\tableofcontents
\endgroup

%%%%%%%%%%%%%%%%%%%%%%%%%%%%%%%%%%%%%%%%%%%%%%%%%%%%%%%%%%%%%%%%%%%%%%%%%%%%%%%%
%%%%%%%%%%%%%%%%%%%%%%%%%%%%%%%%%%%%%%%%%%%%%%%%%%%%%%%%%%%%%%%%%%%%%%%%%%%%%%%%
\section{Introduction}

\LaTeX{} provides a mechanism to structure a large document (such as a book)
into a main file and several child files (containing the chapters)
using the |\include| command.
This mechanism is beneficial for documents
which span hundreds of pages in order to
make the source file(s) more manageable.
Moreover, compilation can be restricted to
selected child files by means of the |\includeonly| command.
The latter feature can be used to reduce the compilation time while editing
(this was significantly more useful in the earlier days of \LaTeX{})
or to generate a smaller document which is easier to navigate.
Another application of |\includeonly| is to generate
documents consisting of selected parts of the complete document.

However, there are a few drawbacks of the plain |\include| mechanism:
\begin{itemize}
\item
The child files cannot be compiled on their own,
they can only be compiled via the main file.
A naive editing environment
(such as a text editor with an option
to have the current file processed by \LaTeX)
may require one to switch to the main file before compiling;
attempting to compile the child file produces errors.
\item
The main file must be modified (each time)
to adjust the |\includeonly| command
to the present needs. This easily leaves the main file in a messy state.
\item
The generated document will always carry the filename
of the main document. This is inconvenient if
several child files are to be compiled and
to be kept for distribution.
\end{itemize}

The present package provides a simple interface
to make child files individually compilable by \LaTeX{}.
Compiling a child file then has the same effect as compiling
the main file with an |\includeonly| command
to select the appropriate child.
Moreover the generated document will carry the name of the child
rather than the main file.
This resolves all three above issues.

This feature is meant to make the editing of books,
thesis documents and lecture notes somewhat more convenient.
However, the package can also be used efficiently for
composing a series of documents (such as exercise sheets)
which are typically distributed individually.
It then assists the author in generating the individual documents
(potentially in different versions)
as well as a document containing the collected series.
Another application is in developing style files
or other kinds of included material
where compilation of the style file could redirect
to a sample or test file.

%%%%%%%%%%%%%%%%%%%%%%%%%%%%%%%%%%%%%%%%%%%%%%%%%%%%%%%%%%%%%%%%%%%%%%%%%%%%%%%%
%%%%%%%%%%%%%%%%%%%%%%%%%%%%%%%%%%%%%%%%%%%%%%%%%%%%%%%%%%%%%%%%%%%%%%%%%%%%%%%%
\section{Usage}

First of all, the package \textsf{childdoc} is \emph{not} a standard
\LaTeXe{} |.sty| style file! Therefore it needs to be invoked in
a non-standard way.

%%%%%%%%%%%%%%%%%%%%%%%%%%%%%%%%%%%%%%%%%%%%%%%%%%%%%%%%%%%%%%%%%%%%%%%%%%%%%%%%
\subsection{Included Files}
\label{sec:include}

%%%%%%%%%%%%%%%%%%%%%%%%%%%%%%%%%%%%%%%%
\DescribeMacro{\childdocmain}
To use the package, add the commands
\begin{center}
\begin{tabular}{l}
|\input{childdoc.def}|\\
|\childdocmain{}|\\
\end{tabular}
\end{center}
at the very top of the main \LaTeX{} file,
in particular \emph{before} the |\documentclass| statement!
The argument of |\childdocmain| should be left empty
(but it must be present).

%%%%%%%%%%%%%%%%%%%%%%%%%%%%%%%%%%%%%%%%
\DescribeMacro{\childdocof}
Furthermore, add the commands
\begin{center}
\begin{tabular}{l}
|\input{childdoc.def}|\\
|\childdocof{|\textit{main}|}|\\
\end{tabular}
\end{center}
at the top of every child file \textit{child}
which is included by |\include{|\textit{child}|}|
from within the main file
(or at least for those files to be compiled individually).
The argument \textit{main} must be the filename of the main file.

There are a couple of
considerations in setting up the main and child documents:

%%%%%%%%%%%%%%%%%%%%%%%%%%%%%%%%%%%%%%%%
\paragraph{Restrictions.}

Please note the following restrictions:
\begin{itemize}
\item
|\childdocmain| must be called with one argument \textit{main}
to ensure compatibility with earlier version of the package.
It must either be empty (|\childdocmain{}|)
or precisely match the filename of the main file in which it is specified.
See \secref{sec:detection} for further information.
\item
The filename \textit{main} must be specified without the |.tex| extension.
\item
The filename \textit{main} is case sensitive
(even in case-insensitive file systems)
due to internal string comparison.
\item
The argument \textit{main} should be fully expanded, it cannot be a macro.
\item
Subdirectories and special characters should be avoided in filenames.
\item
The command |\childdocmain{|\textit{main}|}| must be followed by a whitespace.
It should not be followed immediately by another command
or by a comment mark `|%|'.
This is because the \TeX{} parser reads the token immediately following
the argument of |\childdocmain| and puts it
at the beginning of every child section;
however, a white\-space is ignored.
\end{itemize}

%%%%%%%%%%%%%%%%%%%%%%%%%%%%%%%%%%%%%%%%
\paragraph{Content of Main File.}

It is advisable to place all content in the child files included by |\include|.
Any output contained in the main file will appear in all child documents
unless suppressed manually;
it cannot be suppressed automatically by the |\includeonly| directive
and thus should normally be avoided.
A method to include some content in the main file
by means of conditional processing is described in \secref{sec:conditional}.

%%%%%%%%%%%%%%%%%%%%%%%%%%%%%%%%%%%%%%%%
\paragraph{Page Numbering.}

When only a part of the document is compiled,
the appropriate numbering of pages
(as well as other status parameters)
is determined from the |.aux| files.
The latter contain information from previous passes.
However this information needs to propagate through
all intermediate child documents.
Therefore the page numbering in child documents may well
be inconsistent until the complete document is compiled at least once.

A useful (if unconventional) way to always ensure a consistent
page numbering is to restart the numbering in each child document
and denote the pages by `\textit{child}|.|\textit{page}'
where \textit{child} represents the chapter/section number of the child file.
This can be achieved by the command
|\numberwithin{page}{|\textit{child}|}|
of the \textsf{amsmath} package
where \textit{child} can be |chapter| or |section|
depending on the chosen structuring.
Alternatively, one can modify the macro |\thepage| appropriately
and reset the counter |page| at the start of each child file.

%%%%%%%%%%%%%%%%%%%%%%%%%%%%%%%%%%%%%%%%%%%%%%%%%%%%%%%%%%%%%%%%%%%%%%%%%%%%%%%%
\subsection{Conditional Processing}
\label{sec:conditional}

The package provides a mechanism to compile different versions
of a document. To customise the versions further some conditional processing
can come in handy to distinguish which version is being compiled.
The package provides two macros to describe the compilation context:

%%%%%%%%%%%%%%%%%%%%%%%%%%%%%%%%%%%%%%%%
\DescribeMacro{\ifchilddoc}
The conditional |\ifchilddoc| distinguishes between the compilation of
child documents and the main document:
%
\begin{center}
|\ifchilddoc |\textit{child-code}| |[|\||else |\textit{main-code}]| \||fi|
\end{center}

%%%%%%%%%%%%%%%%%%%%%%%%%%%%%%%%%%%%%%%%
\DescribeMacro{\childdocname}
\DescribeMacro{\childdocjob}
The macro |\childdocname| contains the filename (without extension)
of the main or child file being processed.
Note that |\childdocjob| will always contain the name of the main file.

%%%%%%%%%%%%%%%%%%%%%%%%%%%%%%%%%%%%%%%%
\paragraph{Title Page.}

Conditional processing can be used to include a title or banner page
in the main document when proper precautions are taken.
Importantly, the code in the main file should ensure that the page counter
(as well as other status parameters which are stored in the |.aux| files)
takes the same value after the conditional processing.
Otherwise the page numbers may take divergent values
depending on which part is compiled.

For example, a title page could be declared by:
%
\begin{center}
\begin{tabular}{l}
|\ifchilddoc\||else|\\
|\addtocounter{page}{-1}|\\
\textit{code for title page}\\
|\newpage|\\
|\||fi|
\end{tabular}
\end{center}
%
A banner page for the child documents can be generated by:
%
\begin{center}
\begin{tabular}{l}
|\ifchilddoc|\\
|\addtocounter{page}{-1}|\\
\textit{code for banner page}\\
|\newpage|\\
|\||fi|
\end{tabular}
\end{center}
%
Here one could write a message such as:
\begin{center}
|This is the part \childdocname{} of \childdocjob{}.|
\end{center}

%%%%%%%%%%%%%%%%%%%%%%%%%%%%%%%%%%%%%%%%%%%%%%%%%%%%%%%%%%%%%%%%%%%%%%%%%%%%%%%%
\subsection{Flags}
\label{sec:flags}

The package makes it easy to generate different versions
of the main or child documents.
To this end compilation flags can be defined
and assigned different default values.
They will be particularly useful in conjunction
with the forwarding mechanism described in \secref{sec:forward}.

For example, it may be useful to have a flag |\version|
which can be set to |draft| or |final|.
The document source will contain some conditional code
depending on the value of |\version|.
Suppose further, the flag should default to |final| for the main file
and to |draft| for child files
which is a natural assignment for editing the document.
This is achieved by placing the following code
in the preamble of the main document
(below the |\childdocmain| directive):
%
\begin{center}
\begin{tabular}{l}
|\ifchilddoc|\\
|\providecommand{\version}{draft}|\\
|\||else|\\
|\providecommand{\version}{final}|\\
|\||fi|
\end{tabular}
\end{center}
%
The definition by |\providecommand| makes sure
that previous definitions are not overwritten.
Further statements |\providecommand{\version}{...}|
can thus be added before the above code to override it.

For the main file, one might add a line
(between |\childdocmain| and the above block)
%
\begin{center}
|%\ifchilddoc\||else\providecommand{\version}{draft}\||fi|
\end{center}
%
which can be uncommented to produce a draft version.
Likewise one can add a line to the very top of a child file
(above the |\childdocof{|\textit{main}|}| directive)
%
\begin{center}
|%\providecommand{\version}{final}|
\end{center}
%
which can be uncommented to produce the final version of this child document.

%%%%%%%%%%%%%%%%%%%%%%%%%%%%%%%%%%%%%%%%%%%%%%%%%%%%%%%%%%%%%%%%%%%%%%%%%%%%%%%%
\subsection{Forwarding}
\label{sec:forward}

Different versions of the main or child documents
using compilation flags as described in \secref{sec:flags}
can be (permanently) stored in different files
for convenient compilation, viewing and distribution.
To this end, the package defines a command
to pass on compilation to a different file:

%%%%%%%%%%%%%%%%%%%%%%%%%%%%%%%%%%%%%%%%
\DescribeMacro{\childdocforward}
The command |\childdocforward| redirects processing to
another source file:
%
\begin{center}
\begin{tabular}{l}
|\input{childdoc.def}|\\
|\childdocforward[|\textit{main}|]{|\textit{dest}|}|\\
\end{tabular}
\end{center}
%
The argument \textit{dest} is the destination file
(without extension).
It should be the main file or one of the child files.
Note that further \textsf{childdoc} directives
such as |\childdocof| and |\childdocforward|
in the indicated file will be processed in this form.
The optional argument \textit{main}
passes on directly to the main file \textit{main}
while pretending to compile the child \textit{dest}.
This form behaves as if \textit{dest}
issues |\childdocof{|\textit{main}|}| right away,
and no further \textsf{childdoc} directives will be processed.

%%%%%%%%%%%%%%%%%%%%%%%%%%%%%%%%%%%%%%%%
\DescribeMacro{\...prefix}
In the alternative form |\childdocforwardprefix|,
%
\begin{center}
\begin{tabular}{l}
|\input{childdoc.def}|\\
|\childdocforwardprefix[|\textit{main}|]{|\textit{prefix}|}{|\textit{dest}|}|
\end{tabular}
\end{center}
%
the destination file is determined by a pattern
depending on the current file:
To make this work, the current file must be called
`{\textit{prefix}\hspace{0.2em}\textit{suffix}}'
with \textit{prefix} matching precisely the argument.
Processing is then passed on to the file
`{\textit{dest}\hspace{0.2em}\textit{suffix}}'.
Surely, the same effect is achieved by
directly specifying the
argument `{\textit{dest}\hspace{0.2em}\textit{suffix}}'
in the first form.
However, that requires to set up a different file
for each child. With the alternative form of the command
all these files can have exactly the same content
which simplifies setting them up and maintaining them.

For example, the following file |draft.tex|
with a compilation flag |\version| as described in \secref{sec:flags}
compiles the main document as a draft:
%
\begin{center}
\begin{tabular}{l}
|\def\version{draft}|\\
|\input{childdoc.def}|\\
|\childdocforward{|\textit{main}|}|
\end{tabular}
\end{center}
%
Likewise, the following files |final|\textit{nn}|.tex|
compile the final version of the child document
|child|\textit{nn}|.tex|:
%
\begin{center}
\begin{tabular}{l}
|\def\version{final}|\\
|\input{childdoc.def}|\\
|\childdocforwardprefix{final}{child}|
\end{tabular}
\end{center}
%

Note that when several versions of a main file and/or of each child file
are to be generated, it may be convenient to set up a |Makefile| or
shell script to automatise the process.

%%%%%%%%%%%%%%%%%%%%%%%%%%%%%%%%%%%%%%%%%%%%%%%%%%%%%%%%%%%%%%%%%%%%%%%%%%%%%%%%
\subsection{Command Line Processing}
\label{sec:commandline}

The effect of redirection files can also be achieved by invoking
the \LaTeX{} compiler with a more elaborate command line.
Most conveniently this should be done as part
of a shell script or a |Makefile|.

When using \textsf{childdoc} in the main file, the following
command lines effectively perform a redirection
(note that depending on the shell being used,
backslashes may have to be doubled: `|\|' $\to$ `|\\|'):
%
\begin{center}
|... -jobname "|\textit{target}|" |\\|"|[\textit{flags}]%
|\input{childdoc.def}\childdocforward[|\textit{main}|]{|\textit{dest}|}"|
\end{center}
%
Here \textit{target} is the name of the output file,
\textit{main} is the name of the main file
and \textit{dest} is the name of the main or child file to be processed
(all filenames without extensions).
The optional argument \textit{main} can be omitted
if \textit{main} matches \textit{dest}.
Optionally, compilation \textit{flags} can be defined via |\def| commands.
This command line makes the \TeX{} engine believe
it is compiling the file \textit{target}
whose content is specified as the latter parameter.
The provided code then forwards the processing to
\textit{main} or \textit{dest} as described in \secref{sec:forward}.

%%%%%%%%%%%%%%%%%%%%%%%%%%%%%%%%%%%%%%%%%%%%%%%%%%%%%%%%%%%%%%%%%%%%%%%%%%%%%%%%
\subsection{Include by Input}
\label{sec:input}

Including child documents by |\include| has some restrictions by design.
Most notably, the content of a child document always occupies
its own set of pages; pages cannot be shared between child documents.
Usually, this behaviour makes perfect sense
because each child document contain an essential part of the document.
However, in some situations it may be desirable to compose
a document from a collection of parts
without having mandatory page breaks between then.
For this case, the package
provides a mechanism to include parts
by |\input| which can also be processed individually.
However, by construction this mechanism
requires manual handling of the content to be output.

%%%%%%%%%%%%%%%%%%%%%%%%%%%%%%%%%%%%%%%%
\DescribeMacro{\ifchilddocmanual}
The main file should be prepared as usual, see \secref{sec:include}.
However, the document body must make a distinction
between processing of an individual part and of the main document, e.g.:
%
\begin{center}
\begin{tabular}{l}
|\ifchilddocmanual|\\
|\input{\childdocname}|\\
|\||else|\\
\textit{document body with }|\input{|\textit{part}|}|\\
|\||fi|
\end{tabular}
\end{center}
%
The conditional |\ifchilddocmanual| is true whenever
a part to be included by |\input| is being compiled,
and the name of the part is stored in |\childdocname|.

%%%%%%%%%%%%%%%%%%%%%%%%%%%%%%%%%%%%%%%%
\DescribeMacro{\childdocby}
Each part to be included by |\input| should start with:
%
\begin{center}
\begin{tabular}{l}
|\input{childdoc.def}|\\
|\childdocby{|\textit{main}|}|\\
\end{tabular}
\end{center}
%
The directive |\childdocby| is similar to |\childdocof|
described in \secref{sec:include},
but the subsequent selection of content must be done manually.
To that end, both |\ifchilddoc| and |\ifchilddocmanual|
will be true upon processing of a part,
and the name of the part is stored in |\childdocname|.
Note that |\jobname| will be set to the filename of the current part
so that each part receives an individual |.aux| file
that does not interfere with the |.aux| file(s) of the main document.
This behaviour can be altered by the alternative form
|\childdocby[*]{|\textit{main}|}| (with a non-empty optional argument)
which uses the |.aux| file of the main document
by setting |\jobname| to \textit{main}.

%%%%%%%%%%%%%%%%%%%%%%%%%%%%%%%%%%%%%%%%%%%%%%%%%%%%%%%%%%%%%%%%%%%%%%%%%%%%%%%%
\subsection{Driver Development}
\label{sec:driver}

The \textsf{childdoc} mechanism can also be use for the development
of definition files such as \LaTeX{} styles or classes.
This case differs from the above setup with multiple parts
included by |\include| in that no |\includeonly| should be invoked.
This can be achieved by starting the include file
(before |\ProvidesPackage|) with:
%
\begin{center}
\begin{tabular}{l}
|\input{childdoc.def}|\\
|\childdocforward{|\textit{main}|}|\\
\end{tabular}
\end{center}
%
or alternatively with:
%
\begin{center}
\begin{tabular}{l}
|\input{childdoc.def}|\\
|\childdocby{|\textit{main}|}|\\
\end{tabular}
\end{center}
%
Both forms have slightly different effects as described above.
The main file is prepared as usual, see \secref{sec:include}.

%%%%%%%%%%%%%%%%%%%%%%%%%%%%%%%%%%%%%%%%%%%%%%%%%%%%%%%%%%%%%%%%%%%%%%%%%%%%%%%%
\subsection{Legacy Detection}
\label{sec:detection}

The directive |\childdocmain| in the main file can detect
whether the complete document or merely a child is to be compiled
even without using the directive |\childdocof|.
This method is deprecated because it is less robust
and there is no compelling reason to use it;
it is merely provided for backward compatibility
and it may be removed in future versions.

If the detection mechanism is to be used,
it is mandatory to correctly specify
the filename of the main file as the argument of |\childdocmain|:
%
\begin{center}
\begin{tabular}{l}
|\input{childdoc.def}|\\
|\childdocmain{|\textit{main}|}|\\
\end{tabular}
\end{center}
%
If |\jobname| does not match the argument \textit{main} of |\childdocmain|,
it is assumed that |\jobname| points to the child file to be compiled.
When using |\childdocmain| with the main file specified as argument,
it suffices to start a child file
with just |\input{|\textit{main}|}|
without loading of the package and using |\childdocof|.
If instead all processing is done
with the appropriate \textsf{childdoc} directives,
the argument of \textit{main} of |\childdocmain| can be empty.

An alternative version of the command line processing described
in \secref{sec:commandline} using the detection mechanism reads:
%
\begin{center}
|... -jobname "|\textit{target}|" "|[\textit{flags}]%
[|\def\jobname{|\textit{dest}|}|]|\input{|\textit{main}|}"|
\end{center}

%%%%%%%%%%%%%%%%%%%%%%%%%%%%%%%%%%%%%%%%%%%%%%%%%%%%%%%%%%%%%%%%%%%%%%%%%%%%%%%%
\subsection{Manual Code}
\label{sec:manual}

In case one cannot be certain whether the definitions file |childdoc.def|
is installed on the target \TeX{} distribution
and one prefers not to ship it,
it is conceivable to paste a few relevant commands into the sources.

To that end, drop all statements |\input{childdoc.def}|
and perform the replacements as outlined below.
Instead of |\childdocmain{|\textit{main}|}| add the following code
to the top of the main file:
%
\begin{center}
\begin{tabular}{l}
|\||ifdefined\childdocname\endinput\||fi\newif\ifchilddoc|\\
|\edef\childdocname{\scantokens\expandafter{\jobname\noexpand}}|\\
|\def\childdocmain{|\textit{main}|}\||ifx\childdocmain\childdocname\||else|\\
|\childdoctrue\includeonly{\childdocname}\let\jobname\childdocmain\||fi|\\
\end{tabular}
\end{center}
%
Instead of |\childdocof{|\textit{main}|}| just include the main file
at the top of each child file:
%
\begin{center}
|\input{|\textit{main}|}|
\end{center}
%
A simple redirection |\childdocforward{|\textit{dest}|}| is achieved by:
%
\begin{center}
|\def\jobname{|\textit{dest}|}\input{\jobname}|
\end{center}
%
The redirection with prefix
|\childdocforwardprefix[|\textit{prefix}|]{|\textit{dest}|}|
is accomplished by:
%
\begin{center}
\begin{tabular}{l}
|{\edef\jobname{\scantokens\expandafter{\jobname\noexpand}}|\\
|\def\redirectjob |\textit{prefix}|#1~~~{\gdef\jobname{|\textit{dest}|#1}}|\\
|\expandafter\redirectjob\jobname~~~}\input{\jobname}|
\end{tabular}
\end{center}

In an alternative approach,
child documents can be compiled by a specific command line
without additional code or specific definitions:
%
\begin{center}
|... -jobname "|\textit{target}|" "|[\textit{flags}]%
|\includeonly{|\textit{dest}|}\input{|\textit{main}|}"|
\end{center}
%

%%%%%%%%%%%%%%%%%%%%%%%%%%%%%%%%%%%%%%%%%%%%%%%%%%%%%%%%%%%%%%%%%%%%%%%%%%%%%%%%
%%%%%%%%%%%%%%%%%%%%%%%%%%%%%%%%%%%%%%%%%%%%%%%%%%%%%%%%%%%%%%%%%%%%%%%%%%%%%%%%
\section{Information}

%%%%%%%%%%%%%%%%%%%%%%%%%%%%%%%%%%%%%%%%%%%%%%%%%%%%%%%%%%%%%%%%%%%%%%%%%%%%%%%%
\subsection{Copyright}

Copyright \copyright{} 2017--2018 Niklas Beisert

This work may be distributed and/or modified under the
conditions of the \LaTeX{} Project Public License, either version 1.3
of this license or (at your option) any later version.
The latest version of this license is in
  \url{http://www.latex-project.org/lppl.txt}
and version 1.3 or later is part of all distributions of \LaTeX{}
version 2005/12/01 or later.

This work has the LPPL maintenance status `maintained'.

The Current Maintainer of this work is Niklas Beisert.

This work consists of the files |README.txt|, |childdoc.ins| and |childdoc.dtx|
as well as the derived files |childdoc.def|, |cdocsamp.tex|
with |cdocsch1.tex|, |cdocsch2.tex|, |cdocspt3.tex|, |cdocspt4.tex|,
|cdocsdrf.tex|, |cdocsfn1.tex|, |cdocsfn2.tex|
as well as |childdoc.pdf|.

%%%%%%%%%%%%%%%%%%%%%%%%%%%%%%%%%%%%%%%%%%%%%%%%%%%%%%%%%%%%%%%%%%%%%%%%%%%%%%%%
\subsection{Files and Installation}

The package consists of the files:
%
\begin{center}
\begin{tabular}{ll}
    |README.txt|   & readme file \\
    |childdoc.ins| & installation file \\
    |childdoc.dtx| & source file \\
    |childdoc.def| & definition file \\
    |cdocsamp.tex| & sample main file \\
    |cdocsch1.tex| & sample include file \\
    |cdocsch2.tex| & sample include file \\
    |cdocspt3.tex| & sample part file \\
    |cdocspt4.tex| & sample part file \\
    |cdocsdrf.tex| & sample redirection file \\
    |cdocsfn1.tex| & sample redirection file \\
    |cdocsfn2.tex| & sample redirection file \\
    |childdoc.pdf| & manual
\end{tabular}
\end{center}
%
The distribution consists of the files
|README.txt|, |childdoc.ins| and |childdoc.dtx|.
%
\begin{itemize}
\item
Run (pdf)\LaTeX{} on |childdoc.dtx|
to compile the manual |childdoc.pdf| (this file).
\item
Run \LaTeX{} on |childdoc.ins| to create the definitions file |childdoc.def|
and the sample |cdocsamp.tex| with include files
|cdocsch1.tex|, |cdocsch2.tex|, |cdocspt3.tex|, |cdocspt4.tex|,
|cdocsdrf.tex|, |cdocsfn1.tex|, |cdocsfn2.tex|.
Then copy the file |childdoc.def| to an appropriate directory of your \LaTeX{}
distribution, e.g.\ \textit{texmf-root}|/tex/latex/childdoc|.
\end{itemize}

%%%%%%%%%%%%%%%%%%%%%%%%%%%%%%%%%%%%%%%%%%%%%%%%%%%%%%%%%%%%%%%%%%%%%%%%%%%%%%%%
\subsection{Related CTAN Packages}

There are several other packages which offer a similar functionality:
%
\begin{itemize}
\item
The packages
\href{http://ctan.org/pkg/docmute}{\textsf{docmute}},
\href{http://ctan.org/pkg/includex}{\textsf{includex}} and
\href{http://ctan.org/pkg/standalone}{\textsf{standalone}}
provide commands to include only the document body of
a child file thus allowing both files to be compiled individually.
\item
The packages \href{http://ctan.org/pkg/subdocs}{\textsf{subdocs}}
and \href{http://ctan.org/pkg/subfiles}{\textsf{subfiles}}
provide structures in which the main and child documents can be
encapsulated and allowing them to be compiled individually.
The inclusion mechanism is different from the conventional |\include|.
\item
The package \href{http://ctan.org/pkg/combine}{\textsf{combine}}
is an elaborate solution to combine several documents into one.
\end{itemize}
%
See also the CTAN topic \href{http://ctan.org/topic/subdocs}{\textsf{subdocs}}
for further related packages.
The present package differs from the above solutions in that
a document structure constructed with the conventional |\include| mechanism
just needs two extra commands at the top of every file
such that all constituent files can be compiled individually.

%%%%%%%%%%%%%%%%%%%%%%%%%%%%%%%%%%%%%%%%%%%%%%%%%%%%%%%%%%%%%%%%%%%%%%%%%%%%%%%%
%\subsection{Feature Suggestions}
%
%The following is a list of features which may be useful for future
%versions of this package:
%%
%\begin{itemize}
%\item
%\ldots
%\end{itemize}

%%%%%%%%%%%%%%%%%%%%%%%%%%%%%%%%%%%%%%%%%%%%%%%%%%%%%%%%%%%%%%%%%%%%%%%%%%%%%%%%
\subsection{Revision History}

%%%%%%%%%%%%%%%%%%%%%%%%%%%%%%%%%%%%%%%%
\paragraph{v2.0:} 2018/12/30

\begin{itemize}
\item
immediate forward processing
\item
added |\childdocby| mechanism
\item
manual restructured
\end{itemize}

%%%%%%%%%%%%%%%%%%%%%%%%%%%%%%%%%%%%%%%%
\paragraph{v1.6:} 2018/01/17

\begin{itemize}
\item
application for development of include files
\item
corrections to manual
\end{itemize}

%%%%%%%%%%%%%%%%%%%%%%%%%%%%%%%%%%%%%%%%
\paragraph{v1.5:} 2017/05/21

\begin{itemize}
\item
more complete structuring introduced
\item
|\childdocof| introduced
\item
|\childdoc| renamed to |\childdocmain|
\item
|\childredirect| renamed to |\childdocforward| and |\childdocforwardprefix|
and functionality expanded
\end{itemize}

%%%%%%%%%%%%%%%%%%%%%%%%%%%%%%%%%%%%%%%%
\paragraph{v1.0:} 2017/04/27

\begin{itemize}
\item
manual and install package
\item
first version published on CTAN
\end{itemize}

%%%%%%%%%%%%%%%%%%%%%%%%%%%%%%%%%%%%%%%%
\paragraph{v0.6:} 2017/04/26

\begin{itemize}
\item
redirection mechanism added
\end{itemize}

%%%%%%%%%%%%%%%%%%%%%%%%%%%%%%%%%%%%%%%%
\paragraph{v0.5:} 2017/04/26

\begin{itemize}
\item
functionality in definition file
\end{itemize}


%%%%%%%%%%%%%%%%%%%%%%%%%%%%%%%%%%%%%%%%%%%%%%%%%%%%%%%%%%%%%%%%%%%%%%%%%%%%%%%%
%%%%%%%%%%%%%%%%%%%%%%%%%%%%%%%%%%%%%%%%%%%%%%%%%%%%%%%%%%%%%%%%%%%%%%%%%%%%%%%%
%%%%%%%%%%%%%%%%%%%%%%%%%%%%%%%%%%%%%%%%%%%%%%%%%%%%%%%%%%%%%%%%%%%%%%%%%%%%%%%%
\appendix

\settowidth\MacroIndent{\rmfamily\scriptsize 000\ }

 \DocInput{childdoc.dtx}

\end{document}
%</driver>
% \fi
%
% %%%%%%%%%%%%%%%%%%%%%%%%%%%%%%%%%%%%%%%%%%%%%%%%%%%%%%%%%%%%%%%%%%%%%%%%%%%%%%
% %%%%%%%%%%%%%%%%%%%%%%%%%%%%%%%%%%%%%%%%%%%%%%%%%%%%%%%%%%%%%%%%%%%%%%%%%%%%%%
% \section{Sample}
%\iffalse
%<*samplemain>
%\fi
%
% The following presents a sample document
% with two chapters, two parts, a title page,
% a compile flag as well as three forwarding files to set the flag.
% It consists of eight |.tex| files:
% \begin{center}
% \begin{tabular}{ll}
% |cdocsamp.tex|&main file\\
% |cdocsch1.tex|&include file for chapter 1\\
% |cdocsch2.tex|&include file for chapter 2\\
% |cdocspt3.tex|&include file for part 3\\
% |cdocspt4.tex|&include file for part 4\\
% |cdocsdrf.tex|&forwarding file for main file in draft mode\\
% |cdocsfi1.tex|&forwarding file for final version of chapter 1\\
% |cdocsfi2.tex|&forwarding file for final version of chapter 2\\
% \end{tabular}
% \end{center}
% Each of the eight files can be compiled directly by the \LaTeX{} compiler.
%
% %%%%%%%%%%%%%%%%%%%%%%%%%%%%%%%%%%%%%%
% \paragraph{Main File.}
%
% The main file is called |cdocsamp.tex|.
%
% Load the \textsf{childdoc} definitions and
% declare the filename for the main document:
%    \begin{macrocode}
\input{childdoc.def}
\childdocmain{}
%    \end{macrocode}

% Optional override for |\version| flag:
%    \begin{macrocode}
%%\ifchilddoc\else\providecommand{\version}{draft}\fi
%    \end{macrocode}

% Define the default values for the |\version| flag
% (|final| for the main file and |draft| for childs):
%    \begin{macrocode}
\ifchilddoc
\providecommand{\version}{draft}
\else
\providecommand{\version}{final}
\fi
%    \end{macrocode}

% Load the standard document class:
%    \begin{macrocode}
\documentclass[12pt]{article}
%    \end{macrocode}

% Start the document body:
%    \begin{macrocode}
\begin{document}
%    \end{macrocode}

% Declare a title page.
% Print title, part of document being processed and version flag:
%    \begin{macrocode}
\addtocounter{page}{-1}
\begin{center}
{\LARGE\bfseries{}childdoc example\par}
\vspace{1cm}
\ifchilddoc
\ifchilddocmanual part\else chapter\fi:
`\childdocname' of `\childdocjob'\par
\else
main document: `\childdocjob'\par
\fi
version: \version\par
\end{center}
\newpage
%    \end{macrocode}

% Manually include selected file,
% otherwise process as usual:
%    \begin{macrocode}
\ifchilddocmanual
\section*{part `\childdocname'}
\input{\childdocname}
\else
%    \end{macrocode}

% Include the two chapters:
%    \begin{macrocode}
\include{cdocsch1}
\include{cdocsch2}
%    \end{macrocode}

% Include the two parts unless only chapters should be displayed:
%    \begin{macrocode}
\ifchilddoc\else
\section{part three}
\input{cdocspt3}
\section{part four}
\input{cdocspt4}
\fi
%    \end{macrocode}

% Process as usual until here:
%    \begin{macrocode}
\fi
%    \end{macrocode}

% End of document body:
%    \begin{macrocode}
\end{document}
%    \end{macrocode}
%\iffalse
%</samplemain>
%\fi
%
% %%%%%%%%%%%%%%%%%%%%%%%%%%%%%%%%%%%%%%
% \paragraph{Chapter Include Files.}
%
% The include files are called |cdocsch1.tex| and |cdocsch2.tex|.
%
%\iffalse
%<*samplechap1|samplechap2>
%\fi

% Optional override for |\version| flag:
%    \begin{macrocode}
%%\providecommand{\version}{final}
%    \end{macrocode}

% Include the main document:
%    \begin{macrocode}
\input{childdoc.def}
\childdocof{cdocsamp}
%    \end{macrocode}

%\iffalse
%</samplechap1|samplechap2>
%\fi
%
%\iffalse
%<*samplechap1>
%\fi
% Some text for chapter 1:
%    \begin{macrocode}
\section{one}
some text in chapter one
%    \end{macrocode}

%\iffalse
%</samplechap1>
%\fi
% Some text for chapter 2:
%\iffalse
%<*samplechap2>
%\fi
%    \begin{macrocode}
\section{two}
more text in chapter two
%    \end{macrocode}

%\iffalse
%</samplechap2>
%\fi
%
% %%%%%%%%%%%%%%%%%%%%%%%%%%%%%%%%%%%%%%
% \paragraph{Part Include Files.}
%
% The include files are called |cdocspt3.tex| and |cdocspt4.tex|.
%
%\iffalse
%<*samplepart3|samplepart4>
%\fi

% Optional override for |\version| flag:
%    \begin{macrocode}
%%\providecommand{\version}{final}
%    \end{macrocode}

% Include the main document:
%    \begin{macrocode}
\input{childdoc.def}
\childdocby{cdocsamp}
%    \end{macrocode}

%\iffalse
%</samplepart3|samplepart4>
%\fi
%
%\iffalse
%<*samplepart3>
%\fi
% Some text for part 3:
%    \begin{macrocode}
some text in part three
%    \end{macrocode}

%\iffalse
%</samplepart3>
%\fi
% Some text for part 4:
%\iffalse
%<*samplepart4>
%\fi
%    \begin{macrocode}
more text in part four
%    \end{macrocode}

%\iffalse
%</samplepart4>
%\fi
%
% %%%%%%%%%%%%%%%%%%%%%%%%%%%%%%%%%%%%%%
% \paragraph{Forwarding for a Complete Draft.}
%
% The following forwarding file |cdocsdrf.tex|
% compiles the main document in draft mode:
%\iffalse
%<*sampledraft>
%\fi
%    \begin{macrocode}
\def\version{draft}
\input{childdoc.def}
\childdocforward{cdocsamp}
%    \end{macrocode}

%\iffalse
%</sampledraft>
%\fi
%
% %%%%%%%%%%%%%%%%%%%%%%%%%%%%%%%%%%%%%%
% \paragraph{Forwarding for Final Version of the Chapters.}
%
% The following forwarding files |cdocsfn1.tex| and |cdocsfn2.tex|
% (with identical content)
% compile the final versions of the child documents
% |cdocsch1.tex| and |cdocsch2.tex|, respectively:
%\iffalse
%<*samplefinal>
%\fi
%    \begin{macrocode}
\def\version{final}
\input{childdoc.def}
\childdocforwardprefix[cdocsamp]{cdocsfn}{cdocsch}
%    \end{macrocode}

%\iffalse
%</samplefinal>
%\fi
%
% %%%%%%%%%%%%%%%%%%%%%%%%%%%%%%%%%%%%%%
% \paragraph{Command Line Processing.}
%
% The following three command lines generate the output files
% |cdocscld|, |cdocscl1| and |cdocscl2|
% which should be identical to
% |cdocsdrf|, |cdocsch1| and |cdocsfn2|, respectively:
% \begin{center}
% \begin{tabular}{l}
% |latex -jobname cdocscld \|\\
% |  "\def\version{draft}\input{childdoc.def}\childdocforward{cdocsamp}"|\\
% |latex -jobname cdocscl1 \|\\
% |  "\input{childdoc.def}\childdocforward[cdocsamp]{cdocsch1}"|\\
% |latex -jobname cdocscl2 \|\\
% |  "\def\version{final}\input{childdoc.def}\childdocforward{cdocsch2}"|
% \end{tabular}
% \end{center}
% Note that the trailing backslash on each first line
% merely continues the input to the second line
% (for convenient cut ant paste).
% Furthermore, the command |latex| can be replaced by any
% of its alternative versions such as |pdflatex|.
%
% %%%%%%%%%%%%%%%%%%%%%%%%%%%%%%%%%%%%%%%%%%%%%%%%%%%%%%%%%%%%%%%%%%%%%%%%%%%%%%
% %%%%%%%%%%%%%%%%%%%%%%%%%%%%%%%%%%%%%%%%%%%%%%%%%%%%%%%%%%%%%%%%%%%%%%%%%%%%%%
% \section{Implementation}
%\iffalse
%<*package>
%\fi
%
% This section describes the definitions file |childdoc.def|.

% The definitions cannot be loaded using |\usepackage| or |\RequirePackage|
% which has a mechanism to prevent loading a style file more than once.
% When loading the definitions by means of |\input|
% multiple instances have to be prevented manually:
%\iffalse
%This code needs to be before the `\ProvidesFile' directive
%which is defined at the beginning of this file.
%Therefore it is also placed there and commented out here.
%</package>
%<*discard>
%\fi
%    \begin{macrocode}
\ifdefined\childdocmain\endinput\fi
%    \end{macrocode}
%\iffalse
%</discard>
%<*package>
%\fi
%
% \macro{\ifchilddoc}
% \macro{\ifchilddocmanual}
% The conditional |\ifchilddoc| tells whether a
% child (true) or main (false) document is being compiled.
% The conditional |\ifchilddocmanual| tells whether
% the |\includeonly| mechanism is used (false) or
% the selection of child files must be performed manually (true).
% The definitions initialise to false:
%    \begin{macrocode}
\newif\ifchilddoc
\newif\ifchilddocmanual
%    \end{macrocode}

% \macro{\childdocname}
% \macro{\childdocjob}
% The macro |\childdocname| stores the name of the main document
% to be compiled. The macro |\childdocjob| stores the name of
% the document on which the \LaTeX{} compiler was originally invoked.
% The content of |\jobname| cannot be compared
% to filenames specified in the source due to different catcodes.
% The following code rescans |\jobname|, stores the result
% in |\childdocname| and saves a copy in |\childdocjob|:
%    \begin{macrocode}
\edef\childdocname{\scantokens\expandafter{\jobname\noexpand}}
\let\childdocjob\childdocname
%    \end{macrocode}

% \macro{\childdocdisable}
% The macro |\childdocdisable| prevents the main file
% from being processed more than once.
% At this stage, the main document command |\childdocmain|
% is assumed to be called once again where it should do nothing.
% Any subsequent call to it should prevent
% a secondary processing of the main document
% It overwrites the forwarding commands
% |\childdocof| and |\childdocforward|
% with empty macros to prevent further inclusions of the main document:
%    \begin{macrocode}
\newcommand{\childdocdisable}
{
  \renewcommand{\childdocmain}[1]{\renewcommand{\childdocmain}[1]{\endinput}}
  \renewcommand{\childdocof}[1]{}
  \renewcommand{\childdocby}[2][]{}
  \renewcommand{\childdocforward}[2][]{}
  \renewcommand{\childdocdisable}{}
}
%    \end{macrocode}

% \macro{\childdocmain}
% The macro |\childdocmain| is to be called at the top of the main file
% with nothing or the main filename (without extension) as argument.
% First, it breaks loops.
% If the argument is not empty and does not match |\childdocname|
% (which is set by the first inclusion of |childdoc.def|),
% |\ifchilddoc| is set to true, |\includeonly| is applied to the child file
% and |\jobname| is set to the main file
% (for proper handling of |.aux| files):
%    \begin{macrocode}
\newcommand{\childdocmain}[1]
{
  \childdocdisable\childdocmain{}
  \if?#1?\else
    \begingroup
      \def\childdoctmp{#1}
      \ifx\childdoctmp\childdocname
        \def\childdoctmp{}
      \else
        \def\childdoctmp
        {
          \childdoctrue
          \includeonly{\childdocname}
          \def\childdocjob{#1}
          \def\jobname{#1}
        }
      \fi
      \expandafter
    \endgroup
    \childdoctmp
  \fi
}
%    \end{macrocode}

% \macro{\childdocof}
% The command |\childdocof| redirects
% compilation to the main file |#1|.
%    \begin{macrocode}
\newcommand{\childdocof}[1]
{
  \childdocdisable
  \childdoctrue
  \includeonly{\childdocname}
  \def\jobname{#1}
  \def\childdocjob{#1}
  \input{#1}
}
%    \end{macrocode}

% \macro{\childdocby}
% The command |\childdocby| ....
%    \begin{macrocode}
\newcommand{\childdocby}[2][]
{
  \childdocdisable
  \childdoctrue
  \childdocmanualtrue
  \if?#1?\else
    \def\jobname{#2}
  \fi
  \def\childdocjob{#2}
  \input{#2}
  \endinput
}
%    \end{macrocode}

% \macro{\childdocforward}
% The command |\childdocforward| redirects
% compilation to the main file or
% (if the optional argument is given) a child file.
% Parameters are set as if the main file
% or a child file starting with |\childdocof| was compiled.
% Then compilation is handed over to the main file:
%    \begin{macrocode}
\newcommand{\childdocforward}[2][]
{
  \begingroup
    \if?#1?
      \def\childdoctmp
      {
        \def\childdocname{#2}
        \def\childdocjob{#2}
        \def\jobname{#2}
        \input{#2}
        \endinput
      }
    \else
      \def\childdoctmp
      {
        \childdocdisable
        \def\childdocname{#2}
        \childdoctrue
        \includeonly{#2}
        \def\childdocjob{#1}
        \def\jobname{#1}
        \input{#1}
        \endinput
      }
    \fi
    \expandafter
  \endgroup
  \childdoctmp
}
%    \end{macrocode}

% \macro{\childdocforwardprefix}
% The command |\childdocforwardprefix| redirects
% compilation to the main or a child file by means of a pattern.
% The prefix |#1| in the current filename is replaced by |#2|
% and the suffix of the current filename is kept
% (it is assumed that the filename does not contain the substring `|~~~|'
% which is used as a delimiter).
% Compilation is handed over to the new file by |\childdocforward|:
%    \begin{macrocode}
\newcommand{\childdocforwardprefix}[3][]
{
  \begingroup
    \def\childdocextract #2##1~~~{\def\childdoctmp{\childdocforward[#1]{#3##1}}}
    \expandafter\childdocextract\childdocname~~~
    \expandafter
  \endgroup
  \childdoctmp
}
%    \end{macrocode}

% \macro{\childdoc}
% The deprecated macro |\childdoc| is a legacy version of |\childdocmain|:
%    \begin{macrocode}
\newcommand{\childdoc}{\childdocmain}
%    \end{macrocode}

% \macro{\childdocredirect}
% The deprecated macro |\childdocredirect| is a legacy version
% of |\childdocforward| and |\childdocforwardprefix|:
%    \begin{macrocode}
\newcommand{\childdocredirect}[2][]
{
  \begingroup
    \if?#1?
      \def\childdoctmp{\childdocforward{#2}}
    \else
      \def\childdoctmp{\childdocforwardprefix{#1}{#2}}
    \fi
    \expandafter
  \endgroup
  \childdoctmp
}
%    \end{macrocode}

%\iffalse
%</package>
%\fi
%
\endinput
|\\
|\childdocmain{|\textit{main}|}|\\
\end{tabular}
\end{center}
%
If |\jobname| does not match the argument \textit{main} of |\childdocmain|,
it is assumed that |\jobname| points to the child file to be compiled.
When using |\childdocmain| with the main file specified as argument,
it suffices to start a child file
with just |\input{|\textit{main}|}|
without loading of the package and using |\childdocof|.
If instead all processing is done
with the appropriate \textsf{childdoc} directives,
the argument of \textit{main} of |\childdocmain| can be empty.

An alternative version of the command line processing described
in \secref{sec:commandline} using the detection mechanism reads:
%
\begin{center}
|... -jobname "|\textit{target}|" "|[\textit{flags}]%
[|\def\jobname{|\textit{dest}|}|]|\input{|\textit{main}|}"|
\end{center}

%%%%%%%%%%%%%%%%%%%%%%%%%%%%%%%%%%%%%%%%%%%%%%%%%%%%%%%%%%%%%%%%%%%%%%%%%%%%%%%%
\subsection{Manual Code}
\label{sec:manual}

In case one cannot be certain whether the definitions file |childdoc.def|
is installed on the target \TeX{} distribution
and one prefers not to ship it,
it is conceivable to paste a few relevant commands into the sources.

To that end, drop all statements |% \iffalse
%
% childdoc.dtx Copyright (C) 2017-2018 Niklas Beisert
%
% This work may be distributed and/or modified under the
% conditions of the LaTeX Project Public License, either version 1.3
% of this license or (at your option) any later version.
% The latest version of this license is in
%   http://www.latex-project.org/lppl.txt
% and version 1.3 or later is part of all distributions of LaTeX
% version 2005/12/01 or later.
%
% This work has the LPPL maintenance status `maintained'.
%
% The Current Maintainer of this work is Niklas Beisert.
%
% This work consists of the files childdoc.dtx and childdoc.ins
% and the derived files childdoc.def and cdocsamp.tex with
% cdocsch1.tex, cdocsch2.tex, cdocsdrf.tex, cdocsfn1.tex, cdocsfn2.tex.
%
%<package>\ifdefined\childdocmain\endinput\fi
%<package>\ProvidesFile{childdoc.def}[2018/12/30 v2.0 child document driver]
%<samplemain>\ProvidesFile{cdocsamp.tex}[2018/12/30 v2.0 sample for childdoc]
%<*driver>
%\ProvidesFile{childdoc.drv}[2018/12/30 v2.0 childdoc reference manual file]
\PassOptionsToClass{10pt,a4paper}{article}
\documentclass{ltxdoc}

\usepackage[margin=35mm]{geometry}
\usepackage{hyperref}
\usepackage{hyperxmp}
\usepackage[usenames]{color}

\hypersetup{colorlinks=true}
\hypersetup{pdfstartview=FitH}
\hypersetup{pdfpagemode=UseNone}
\hypersetup{pdfsource={}}
\hypersetup{pdflang={en-UK}}
\hypersetup{pdfcopyright={Copyright 2017-2018 Niklas Beisert.
  This work may be distributed and/or modified under the
  conditions of the LaTeX Project Public License, either version 1.3
  of this license or (at your option) any later version.}}
\hypersetup{pdflicenseurl={http://www.latex-project.org/lppl.txt}}
\hypersetup{pdfcontactaddress={ETH Zurich, ITP, HIT K,
  Wolfgang-Pauli-Strasse 27}}
\hypersetup{pdfcontactpostcode={8093}}
\hypersetup{pdfcontactcity={Zurich}}
\hypersetup{pdfcontactcountry={Switzerland}}
\hypersetup{pdfcontactemail={nbeisert@itp.phys.ethz.ch}}
\hypersetup{pdfcontacturl={http://people.phys.ethz.ch/\xmptilde nbeisert/}}

\newcommand{\secref}[1]{\hyperref[#1]{section \ref*{#1}}}

\parskip1ex
\parindent0pt
\let\olditemize\itemize
\def\itemize{\olditemize\parskip0pt}

\begin{document}

\title{The \textsf{childdoc} Package}
\hypersetup{pdftitle={The childdoc Package}}
\author{Niklas Beisert\\[2ex]
  Institut f\"ur Theoretische Physik\\
  Eidgen\"ossische Technische Hochschule Z\"urich\\
  Wolfgang-Pauli-Strasse 27, 8093 Z\"urich, Switzerland\\[1ex]
  \href{mailto:nbeisert@itp.phys.ethz.ch}
  {\texttt{nbeisert@itp.phys.ethz.ch}}}
\hypersetup{pdfauthor={Niklas Beisert}}
\hypersetup{pdfsubject={Manual for the LaTeX2e Package childdoc}}
\date{30 December 2018, \textsf{v2.0}}
\maketitle

\begin{abstract}\noindent
\textsf{childdoc} is a \LaTeXe{} package
that enables the direct compilation
of document sections included by |\include|
to individual files.
\end{abstract}

\begingroup
\parskip0ex
\tableofcontents
\endgroup

%%%%%%%%%%%%%%%%%%%%%%%%%%%%%%%%%%%%%%%%%%%%%%%%%%%%%%%%%%%%%%%%%%%%%%%%%%%%%%%%
%%%%%%%%%%%%%%%%%%%%%%%%%%%%%%%%%%%%%%%%%%%%%%%%%%%%%%%%%%%%%%%%%%%%%%%%%%%%%%%%
\section{Introduction}

\LaTeX{} provides a mechanism to structure a large document (such as a book)
into a main file and several child files (containing the chapters)
using the |\include| command.
This mechanism is beneficial for documents
which span hundreds of pages in order to
make the source file(s) more manageable.
Moreover, compilation can be restricted to
selected child files by means of the |\includeonly| command.
The latter feature can be used to reduce the compilation time while editing
(this was significantly more useful in the earlier days of \LaTeX{})
or to generate a smaller document which is easier to navigate.
Another application of |\includeonly| is to generate
documents consisting of selected parts of the complete document.

However, there are a few drawbacks of the plain |\include| mechanism:
\begin{itemize}
\item
The child files cannot be compiled on their own,
they can only be compiled via the main file.
A naive editing environment
(such as a text editor with an option
to have the current file processed by \LaTeX)
may require one to switch to the main file before compiling;
attempting to compile the child file produces errors.
\item
The main file must be modified (each time)
to adjust the |\includeonly| command
to the present needs. This easily leaves the main file in a messy state.
\item
The generated document will always carry the filename
of the main document. This is inconvenient if
several child files are to be compiled and
to be kept for distribution.
\end{itemize}

The present package provides a simple interface
to make child files individually compilable by \LaTeX{}.
Compiling a child file then has the same effect as compiling
the main file with an |\includeonly| command
to select the appropriate child.
Moreover the generated document will carry the name of the child
rather than the main file.
This resolves all three above issues.

This feature is meant to make the editing of books,
thesis documents and lecture notes somewhat more convenient.
However, the package can also be used efficiently for
composing a series of documents (such as exercise sheets)
which are typically distributed individually.
It then assists the author in generating the individual documents
(potentially in different versions)
as well as a document containing the collected series.
Another application is in developing style files
or other kinds of included material
where compilation of the style file could redirect
to a sample or test file.

%%%%%%%%%%%%%%%%%%%%%%%%%%%%%%%%%%%%%%%%%%%%%%%%%%%%%%%%%%%%%%%%%%%%%%%%%%%%%%%%
%%%%%%%%%%%%%%%%%%%%%%%%%%%%%%%%%%%%%%%%%%%%%%%%%%%%%%%%%%%%%%%%%%%%%%%%%%%%%%%%
\section{Usage}

First of all, the package \textsf{childdoc} is \emph{not} a standard
\LaTeXe{} |.sty| style file! Therefore it needs to be invoked in
a non-standard way.

%%%%%%%%%%%%%%%%%%%%%%%%%%%%%%%%%%%%%%%%%%%%%%%%%%%%%%%%%%%%%%%%%%%%%%%%%%%%%%%%
\subsection{Included Files}
\label{sec:include}

%%%%%%%%%%%%%%%%%%%%%%%%%%%%%%%%%%%%%%%%
\DescribeMacro{\childdocmain}
To use the package, add the commands
\begin{center}
\begin{tabular}{l}
|\input{childdoc.def}|\\
|\childdocmain{}|\\
\end{tabular}
\end{center}
at the very top of the main \LaTeX{} file,
in particular \emph{before} the |\documentclass| statement!
The argument of |\childdocmain| should be left empty
(but it must be present).

%%%%%%%%%%%%%%%%%%%%%%%%%%%%%%%%%%%%%%%%
\DescribeMacro{\childdocof}
Furthermore, add the commands
\begin{center}
\begin{tabular}{l}
|\input{childdoc.def}|\\
|\childdocof{|\textit{main}|}|\\
\end{tabular}
\end{center}
at the top of every child file \textit{child}
which is included by |\include{|\textit{child}|}|
from within the main file
(or at least for those files to be compiled individually).
The argument \textit{main} must be the filename of the main file.

There are a couple of
considerations in setting up the main and child documents:

%%%%%%%%%%%%%%%%%%%%%%%%%%%%%%%%%%%%%%%%
\paragraph{Restrictions.}

Please note the following restrictions:
\begin{itemize}
\item
|\childdocmain| must be called with one argument \textit{main}
to ensure compatibility with earlier version of the package.
It must either be empty (|\childdocmain{}|)
or precisely match the filename of the main file in which it is specified.
See \secref{sec:detection} for further information.
\item
The filename \textit{main} must be specified without the |.tex| extension.
\item
The filename \textit{main} is case sensitive
(even in case-insensitive file systems)
due to internal string comparison.
\item
The argument \textit{main} should be fully expanded, it cannot be a macro.
\item
Subdirectories and special characters should be avoided in filenames.
\item
The command |\childdocmain{|\textit{main}|}| must be followed by a whitespace.
It should not be followed immediately by another command
or by a comment mark `|%|'.
This is because the \TeX{} parser reads the token immediately following
the argument of |\childdocmain| and puts it
at the beginning of every child section;
however, a white\-space is ignored.
\end{itemize}

%%%%%%%%%%%%%%%%%%%%%%%%%%%%%%%%%%%%%%%%
\paragraph{Content of Main File.}

It is advisable to place all content in the child files included by |\include|.
Any output contained in the main file will appear in all child documents
unless suppressed manually;
it cannot be suppressed automatically by the |\includeonly| directive
and thus should normally be avoided.
A method to include some content in the main file
by means of conditional processing is described in \secref{sec:conditional}.

%%%%%%%%%%%%%%%%%%%%%%%%%%%%%%%%%%%%%%%%
\paragraph{Page Numbering.}

When only a part of the document is compiled,
the appropriate numbering of pages
(as well as other status parameters)
is determined from the |.aux| files.
The latter contain information from previous passes.
However this information needs to propagate through
all intermediate child documents.
Therefore the page numbering in child documents may well
be inconsistent until the complete document is compiled at least once.

A useful (if unconventional) way to always ensure a consistent
page numbering is to restart the numbering in each child document
and denote the pages by `\textit{child}|.|\textit{page}'
where \textit{child} represents the chapter/section number of the child file.
This can be achieved by the command
|\numberwithin{page}{|\textit{child}|}|
of the \textsf{amsmath} package
where \textit{child} can be |chapter| or |section|
depending on the chosen structuring.
Alternatively, one can modify the macro |\thepage| appropriately
and reset the counter |page| at the start of each child file.

%%%%%%%%%%%%%%%%%%%%%%%%%%%%%%%%%%%%%%%%%%%%%%%%%%%%%%%%%%%%%%%%%%%%%%%%%%%%%%%%
\subsection{Conditional Processing}
\label{sec:conditional}

The package provides a mechanism to compile different versions
of a document. To customise the versions further some conditional processing
can come in handy to distinguish which version is being compiled.
The package provides two macros to describe the compilation context:

%%%%%%%%%%%%%%%%%%%%%%%%%%%%%%%%%%%%%%%%
\DescribeMacro{\ifchilddoc}
The conditional |\ifchilddoc| distinguishes between the compilation of
child documents and the main document:
%
\begin{center}
|\ifchilddoc |\textit{child-code}| |[|\||else |\textit{main-code}]| \||fi|
\end{center}

%%%%%%%%%%%%%%%%%%%%%%%%%%%%%%%%%%%%%%%%
\DescribeMacro{\childdocname}
\DescribeMacro{\childdocjob}
The macro |\childdocname| contains the filename (without extension)
of the main or child file being processed.
Note that |\childdocjob| will always contain the name of the main file.

%%%%%%%%%%%%%%%%%%%%%%%%%%%%%%%%%%%%%%%%
\paragraph{Title Page.}

Conditional processing can be used to include a title or banner page
in the main document when proper precautions are taken.
Importantly, the code in the main file should ensure that the page counter
(as well as other status parameters which are stored in the |.aux| files)
takes the same value after the conditional processing.
Otherwise the page numbers may take divergent values
depending on which part is compiled.

For example, a title page could be declared by:
%
\begin{center}
\begin{tabular}{l}
|\ifchilddoc\||else|\\
|\addtocounter{page}{-1}|\\
\textit{code for title page}\\
|\newpage|\\
|\||fi|
\end{tabular}
\end{center}
%
A banner page for the child documents can be generated by:
%
\begin{center}
\begin{tabular}{l}
|\ifchilddoc|\\
|\addtocounter{page}{-1}|\\
\textit{code for banner page}\\
|\newpage|\\
|\||fi|
\end{tabular}
\end{center}
%
Here one could write a message such as:
\begin{center}
|This is the part \childdocname{} of \childdocjob{}.|
\end{center}

%%%%%%%%%%%%%%%%%%%%%%%%%%%%%%%%%%%%%%%%%%%%%%%%%%%%%%%%%%%%%%%%%%%%%%%%%%%%%%%%
\subsection{Flags}
\label{sec:flags}

The package makes it easy to generate different versions
of the main or child documents.
To this end compilation flags can be defined
and assigned different default values.
They will be particularly useful in conjunction
with the forwarding mechanism described in \secref{sec:forward}.

For example, it may be useful to have a flag |\version|
which can be set to |draft| or |final|.
The document source will contain some conditional code
depending on the value of |\version|.
Suppose further, the flag should default to |final| for the main file
and to |draft| for child files
which is a natural assignment for editing the document.
This is achieved by placing the following code
in the preamble of the main document
(below the |\childdocmain| directive):
%
\begin{center}
\begin{tabular}{l}
|\ifchilddoc|\\
|\providecommand{\version}{draft}|\\
|\||else|\\
|\providecommand{\version}{final}|\\
|\||fi|
\end{tabular}
\end{center}
%
The definition by |\providecommand| makes sure
that previous definitions are not overwritten.
Further statements |\providecommand{\version}{...}|
can thus be added before the above code to override it.

For the main file, one might add a line
(between |\childdocmain| and the above block)
%
\begin{center}
|%\ifchilddoc\||else\providecommand{\version}{draft}\||fi|
\end{center}
%
which can be uncommented to produce a draft version.
Likewise one can add a line to the very top of a child file
(above the |\childdocof{|\textit{main}|}| directive)
%
\begin{center}
|%\providecommand{\version}{final}|
\end{center}
%
which can be uncommented to produce the final version of this child document.

%%%%%%%%%%%%%%%%%%%%%%%%%%%%%%%%%%%%%%%%%%%%%%%%%%%%%%%%%%%%%%%%%%%%%%%%%%%%%%%%
\subsection{Forwarding}
\label{sec:forward}

Different versions of the main or child documents
using compilation flags as described in \secref{sec:flags}
can be (permanently) stored in different files
for convenient compilation, viewing and distribution.
To this end, the package defines a command
to pass on compilation to a different file:

%%%%%%%%%%%%%%%%%%%%%%%%%%%%%%%%%%%%%%%%
\DescribeMacro{\childdocforward}
The command |\childdocforward| redirects processing to
another source file:
%
\begin{center}
\begin{tabular}{l}
|\input{childdoc.def}|\\
|\childdocforward[|\textit{main}|]{|\textit{dest}|}|\\
\end{tabular}
\end{center}
%
The argument \textit{dest} is the destination file
(without extension).
It should be the main file or one of the child files.
Note that further \textsf{childdoc} directives
such as |\childdocof| and |\childdocforward|
in the indicated file will be processed in this form.
The optional argument \textit{main}
passes on directly to the main file \textit{main}
while pretending to compile the child \textit{dest}.
This form behaves as if \textit{dest}
issues |\childdocof{|\textit{main}|}| right away,
and no further \textsf{childdoc} directives will be processed.

%%%%%%%%%%%%%%%%%%%%%%%%%%%%%%%%%%%%%%%%
\DescribeMacro{\...prefix}
In the alternative form |\childdocforwardprefix|,
%
\begin{center}
\begin{tabular}{l}
|\input{childdoc.def}|\\
|\childdocforwardprefix[|\textit{main}|]{|\textit{prefix}|}{|\textit{dest}|}|
\end{tabular}
\end{center}
%
the destination file is determined by a pattern
depending on the current file:
To make this work, the current file must be called
`{\textit{prefix}\hspace{0.2em}\textit{suffix}}'
with \textit{prefix} matching precisely the argument.
Processing is then passed on to the file
`{\textit{dest}\hspace{0.2em}\textit{suffix}}'.
Surely, the same effect is achieved by
directly specifying the
argument `{\textit{dest}\hspace{0.2em}\textit{suffix}}'
in the first form.
However, that requires to set up a different file
for each child. With the alternative form of the command
all these files can have exactly the same content
which simplifies setting them up and maintaining them.

For example, the following file |draft.tex|
with a compilation flag |\version| as described in \secref{sec:flags}
compiles the main document as a draft:
%
\begin{center}
\begin{tabular}{l}
|\def\version{draft}|\\
|\input{childdoc.def}|\\
|\childdocforward{|\textit{main}|}|
\end{tabular}
\end{center}
%
Likewise, the following files |final|\textit{nn}|.tex|
compile the final version of the child document
|child|\textit{nn}|.tex|:
%
\begin{center}
\begin{tabular}{l}
|\def\version{final}|\\
|\input{childdoc.def}|\\
|\childdocforwardprefix{final}{child}|
\end{tabular}
\end{center}
%

Note that when several versions of a main file and/or of each child file
are to be generated, it may be convenient to set up a |Makefile| or
shell script to automatise the process.

%%%%%%%%%%%%%%%%%%%%%%%%%%%%%%%%%%%%%%%%%%%%%%%%%%%%%%%%%%%%%%%%%%%%%%%%%%%%%%%%
\subsection{Command Line Processing}
\label{sec:commandline}

The effect of redirection files can also be achieved by invoking
the \LaTeX{} compiler with a more elaborate command line.
Most conveniently this should be done as part
of a shell script or a |Makefile|.

When using \textsf{childdoc} in the main file, the following
command lines effectively perform a redirection
(note that depending on the shell being used,
backslashes may have to be doubled: `|\|' $\to$ `|\\|'):
%
\begin{center}
|... -jobname "|\textit{target}|" |\\|"|[\textit{flags}]%
|\input{childdoc.def}\childdocforward[|\textit{main}|]{|\textit{dest}|}"|
\end{center}
%
Here \textit{target} is the name of the output file,
\textit{main} is the name of the main file
and \textit{dest} is the name of the main or child file to be processed
(all filenames without extensions).
The optional argument \textit{main} can be omitted
if \textit{main} matches \textit{dest}.
Optionally, compilation \textit{flags} can be defined via |\def| commands.
This command line makes the \TeX{} engine believe
it is compiling the file \textit{target}
whose content is specified as the latter parameter.
The provided code then forwards the processing to
\textit{main} or \textit{dest} as described in \secref{sec:forward}.

%%%%%%%%%%%%%%%%%%%%%%%%%%%%%%%%%%%%%%%%%%%%%%%%%%%%%%%%%%%%%%%%%%%%%%%%%%%%%%%%
\subsection{Include by Input}
\label{sec:input}

Including child documents by |\include| has some restrictions by design.
Most notably, the content of a child document always occupies
its own set of pages; pages cannot be shared between child documents.
Usually, this behaviour makes perfect sense
because each child document contain an essential part of the document.
However, in some situations it may be desirable to compose
a document from a collection of parts
without having mandatory page breaks between then.
For this case, the package
provides a mechanism to include parts
by |\input| which can also be processed individually.
However, by construction this mechanism
requires manual handling of the content to be output.

%%%%%%%%%%%%%%%%%%%%%%%%%%%%%%%%%%%%%%%%
\DescribeMacro{\ifchilddocmanual}
The main file should be prepared as usual, see \secref{sec:include}.
However, the document body must make a distinction
between processing of an individual part and of the main document, e.g.:
%
\begin{center}
\begin{tabular}{l}
|\ifchilddocmanual|\\
|\input{\childdocname}|\\
|\||else|\\
\textit{document body with }|\input{|\textit{part}|}|\\
|\||fi|
\end{tabular}
\end{center}
%
The conditional |\ifchilddocmanual| is true whenever
a part to be included by |\input| is being compiled,
and the name of the part is stored in |\childdocname|.

%%%%%%%%%%%%%%%%%%%%%%%%%%%%%%%%%%%%%%%%
\DescribeMacro{\childdocby}
Each part to be included by |\input| should start with:
%
\begin{center}
\begin{tabular}{l}
|\input{childdoc.def}|\\
|\childdocby{|\textit{main}|}|\\
\end{tabular}
\end{center}
%
The directive |\childdocby| is similar to |\childdocof|
described in \secref{sec:include},
but the subsequent selection of content must be done manually.
To that end, both |\ifchilddoc| and |\ifchilddocmanual|
will be true upon processing of a part,
and the name of the part is stored in |\childdocname|.
Note that |\jobname| will be set to the filename of the current part
so that each part receives an individual |.aux| file
that does not interfere with the |.aux| file(s) of the main document.
This behaviour can be altered by the alternative form
|\childdocby[*]{|\textit{main}|}| (with a non-empty optional argument)
which uses the |.aux| file of the main document
by setting |\jobname| to \textit{main}.

%%%%%%%%%%%%%%%%%%%%%%%%%%%%%%%%%%%%%%%%%%%%%%%%%%%%%%%%%%%%%%%%%%%%%%%%%%%%%%%%
\subsection{Driver Development}
\label{sec:driver}

The \textsf{childdoc} mechanism can also be use for the development
of definition files such as \LaTeX{} styles or classes.
This case differs from the above setup with multiple parts
included by |\include| in that no |\includeonly| should be invoked.
This can be achieved by starting the include file
(before |\ProvidesPackage|) with:
%
\begin{center}
\begin{tabular}{l}
|\input{childdoc.def}|\\
|\childdocforward{|\textit{main}|}|\\
\end{tabular}
\end{center}
%
or alternatively with:
%
\begin{center}
\begin{tabular}{l}
|\input{childdoc.def}|\\
|\childdocby{|\textit{main}|}|\\
\end{tabular}
\end{center}
%
Both forms have slightly different effects as described above.
The main file is prepared as usual, see \secref{sec:include}.

%%%%%%%%%%%%%%%%%%%%%%%%%%%%%%%%%%%%%%%%%%%%%%%%%%%%%%%%%%%%%%%%%%%%%%%%%%%%%%%%
\subsection{Legacy Detection}
\label{sec:detection}

The directive |\childdocmain| in the main file can detect
whether the complete document or merely a child is to be compiled
even without using the directive |\childdocof|.
This method is deprecated because it is less robust
and there is no compelling reason to use it;
it is merely provided for backward compatibility
and it may be removed in future versions.

If the detection mechanism is to be used,
it is mandatory to correctly specify
the filename of the main file as the argument of |\childdocmain|:
%
\begin{center}
\begin{tabular}{l}
|\input{childdoc.def}|\\
|\childdocmain{|\textit{main}|}|\\
\end{tabular}
\end{center}
%
If |\jobname| does not match the argument \textit{main} of |\childdocmain|,
it is assumed that |\jobname| points to the child file to be compiled.
When using |\childdocmain| with the main file specified as argument,
it suffices to start a child file
with just |\input{|\textit{main}|}|
without loading of the package and using |\childdocof|.
If instead all processing is done
with the appropriate \textsf{childdoc} directives,
the argument of \textit{main} of |\childdocmain| can be empty.

An alternative version of the command line processing described
in \secref{sec:commandline} using the detection mechanism reads:
%
\begin{center}
|... -jobname "|\textit{target}|" "|[\textit{flags}]%
[|\def\jobname{|\textit{dest}|}|]|\input{|\textit{main}|}"|
\end{center}

%%%%%%%%%%%%%%%%%%%%%%%%%%%%%%%%%%%%%%%%%%%%%%%%%%%%%%%%%%%%%%%%%%%%%%%%%%%%%%%%
\subsection{Manual Code}
\label{sec:manual}

In case one cannot be certain whether the definitions file |childdoc.def|
is installed on the target \TeX{} distribution
and one prefers not to ship it,
it is conceivable to paste a few relevant commands into the sources.

To that end, drop all statements |\input{childdoc.def}|
and perform the replacements as outlined below.
Instead of |\childdocmain{|\textit{main}|}| add the following code
to the top of the main file:
%
\begin{center}
\begin{tabular}{l}
|\||ifdefined\childdocname\endinput\||fi\newif\ifchilddoc|\\
|\edef\childdocname{\scantokens\expandafter{\jobname\noexpand}}|\\
|\def\childdocmain{|\textit{main}|}\||ifx\childdocmain\childdocname\||else|\\
|\childdoctrue\includeonly{\childdocname}\let\jobname\childdocmain\||fi|\\
\end{tabular}
\end{center}
%
Instead of |\childdocof{|\textit{main}|}| just include the main file
at the top of each child file:
%
\begin{center}
|\input{|\textit{main}|}|
\end{center}
%
A simple redirection |\childdocforward{|\textit{dest}|}| is achieved by:
%
\begin{center}
|\def\jobname{|\textit{dest}|}\input{\jobname}|
\end{center}
%
The redirection with prefix
|\childdocforwardprefix[|\textit{prefix}|]{|\textit{dest}|}|
is accomplished by:
%
\begin{center}
\begin{tabular}{l}
|{\edef\jobname{\scantokens\expandafter{\jobname\noexpand}}|\\
|\def\redirectjob |\textit{prefix}|#1~~~{\gdef\jobname{|\textit{dest}|#1}}|\\
|\expandafter\redirectjob\jobname~~~}\input{\jobname}|
\end{tabular}
\end{center}

In an alternative approach,
child documents can be compiled by a specific command line
without additional code or specific definitions:
%
\begin{center}
|... -jobname "|\textit{target}|" "|[\textit{flags}]%
|\includeonly{|\textit{dest}|}\input{|\textit{main}|}"|
\end{center}
%

%%%%%%%%%%%%%%%%%%%%%%%%%%%%%%%%%%%%%%%%%%%%%%%%%%%%%%%%%%%%%%%%%%%%%%%%%%%%%%%%
%%%%%%%%%%%%%%%%%%%%%%%%%%%%%%%%%%%%%%%%%%%%%%%%%%%%%%%%%%%%%%%%%%%%%%%%%%%%%%%%
\section{Information}

%%%%%%%%%%%%%%%%%%%%%%%%%%%%%%%%%%%%%%%%%%%%%%%%%%%%%%%%%%%%%%%%%%%%%%%%%%%%%%%%
\subsection{Copyright}

Copyright \copyright{} 2017--2018 Niklas Beisert

This work may be distributed and/or modified under the
conditions of the \LaTeX{} Project Public License, either version 1.3
of this license or (at your option) any later version.
The latest version of this license is in
  \url{http://www.latex-project.org/lppl.txt}
and version 1.3 or later is part of all distributions of \LaTeX{}
version 2005/12/01 or later.

This work has the LPPL maintenance status `maintained'.

The Current Maintainer of this work is Niklas Beisert.

This work consists of the files |README.txt|, |childdoc.ins| and |childdoc.dtx|
as well as the derived files |childdoc.def|, |cdocsamp.tex|
with |cdocsch1.tex|, |cdocsch2.tex|, |cdocspt3.tex|, |cdocspt4.tex|,
|cdocsdrf.tex|, |cdocsfn1.tex|, |cdocsfn2.tex|
as well as |childdoc.pdf|.

%%%%%%%%%%%%%%%%%%%%%%%%%%%%%%%%%%%%%%%%%%%%%%%%%%%%%%%%%%%%%%%%%%%%%%%%%%%%%%%%
\subsection{Files and Installation}

The package consists of the files:
%
\begin{center}
\begin{tabular}{ll}
    |README.txt|   & readme file \\
    |childdoc.ins| & installation file \\
    |childdoc.dtx| & source file \\
    |childdoc.def| & definition file \\
    |cdocsamp.tex| & sample main file \\
    |cdocsch1.tex| & sample include file \\
    |cdocsch2.tex| & sample include file \\
    |cdocspt3.tex| & sample part file \\
    |cdocspt4.tex| & sample part file \\
    |cdocsdrf.tex| & sample redirection file \\
    |cdocsfn1.tex| & sample redirection file \\
    |cdocsfn2.tex| & sample redirection file \\
    |childdoc.pdf| & manual
\end{tabular}
\end{center}
%
The distribution consists of the files
|README.txt|, |childdoc.ins| and |childdoc.dtx|.
%
\begin{itemize}
\item
Run (pdf)\LaTeX{} on |childdoc.dtx|
to compile the manual |childdoc.pdf| (this file).
\item
Run \LaTeX{} on |childdoc.ins| to create the definitions file |childdoc.def|
and the sample |cdocsamp.tex| with include files
|cdocsch1.tex|, |cdocsch2.tex|, |cdocspt3.tex|, |cdocspt4.tex|,
|cdocsdrf.tex|, |cdocsfn1.tex|, |cdocsfn2.tex|.
Then copy the file |childdoc.def| to an appropriate directory of your \LaTeX{}
distribution, e.g.\ \textit{texmf-root}|/tex/latex/childdoc|.
\end{itemize}

%%%%%%%%%%%%%%%%%%%%%%%%%%%%%%%%%%%%%%%%%%%%%%%%%%%%%%%%%%%%%%%%%%%%%%%%%%%%%%%%
\subsection{Related CTAN Packages}

There are several other packages which offer a similar functionality:
%
\begin{itemize}
\item
The packages
\href{http://ctan.org/pkg/docmute}{\textsf{docmute}},
\href{http://ctan.org/pkg/includex}{\textsf{includex}} and
\href{http://ctan.org/pkg/standalone}{\textsf{standalone}}
provide commands to include only the document body of
a child file thus allowing both files to be compiled individually.
\item
The packages \href{http://ctan.org/pkg/subdocs}{\textsf{subdocs}}
and \href{http://ctan.org/pkg/subfiles}{\textsf{subfiles}}
provide structures in which the main and child documents can be
encapsulated and allowing them to be compiled individually.
The inclusion mechanism is different from the conventional |\include|.
\item
The package \href{http://ctan.org/pkg/combine}{\textsf{combine}}
is an elaborate solution to combine several documents into one.
\end{itemize}
%
See also the CTAN topic \href{http://ctan.org/topic/subdocs}{\textsf{subdocs}}
for further related packages.
The present package differs from the above solutions in that
a document structure constructed with the conventional |\include| mechanism
just needs two extra commands at the top of every file
such that all constituent files can be compiled individually.

%%%%%%%%%%%%%%%%%%%%%%%%%%%%%%%%%%%%%%%%%%%%%%%%%%%%%%%%%%%%%%%%%%%%%%%%%%%%%%%%
%\subsection{Feature Suggestions}
%
%The following is a list of features which may be useful for future
%versions of this package:
%%
%\begin{itemize}
%\item
%\ldots
%\end{itemize}

%%%%%%%%%%%%%%%%%%%%%%%%%%%%%%%%%%%%%%%%%%%%%%%%%%%%%%%%%%%%%%%%%%%%%%%%%%%%%%%%
\subsection{Revision History}

%%%%%%%%%%%%%%%%%%%%%%%%%%%%%%%%%%%%%%%%
\paragraph{v2.0:} 2018/12/30

\begin{itemize}
\item
immediate forward processing
\item
added |\childdocby| mechanism
\item
manual restructured
\end{itemize}

%%%%%%%%%%%%%%%%%%%%%%%%%%%%%%%%%%%%%%%%
\paragraph{v1.6:} 2018/01/17

\begin{itemize}
\item
application for development of include files
\item
corrections to manual
\end{itemize}

%%%%%%%%%%%%%%%%%%%%%%%%%%%%%%%%%%%%%%%%
\paragraph{v1.5:} 2017/05/21

\begin{itemize}
\item
more complete structuring introduced
\item
|\childdocof| introduced
\item
|\childdoc| renamed to |\childdocmain|
\item
|\childredirect| renamed to |\childdocforward| and |\childdocforwardprefix|
and functionality expanded
\end{itemize}

%%%%%%%%%%%%%%%%%%%%%%%%%%%%%%%%%%%%%%%%
\paragraph{v1.0:} 2017/04/27

\begin{itemize}
\item
manual and install package
\item
first version published on CTAN
\end{itemize}

%%%%%%%%%%%%%%%%%%%%%%%%%%%%%%%%%%%%%%%%
\paragraph{v0.6:} 2017/04/26

\begin{itemize}
\item
redirection mechanism added
\end{itemize}

%%%%%%%%%%%%%%%%%%%%%%%%%%%%%%%%%%%%%%%%
\paragraph{v0.5:} 2017/04/26

\begin{itemize}
\item
functionality in definition file
\end{itemize}


%%%%%%%%%%%%%%%%%%%%%%%%%%%%%%%%%%%%%%%%%%%%%%%%%%%%%%%%%%%%%%%%%%%%%%%%%%%%%%%%
%%%%%%%%%%%%%%%%%%%%%%%%%%%%%%%%%%%%%%%%%%%%%%%%%%%%%%%%%%%%%%%%%%%%%%%%%%%%%%%%
%%%%%%%%%%%%%%%%%%%%%%%%%%%%%%%%%%%%%%%%%%%%%%%%%%%%%%%%%%%%%%%%%%%%%%%%%%%%%%%%
\appendix

\settowidth\MacroIndent{\rmfamily\scriptsize 000\ }

 \DocInput{childdoc.dtx}

\end{document}
%</driver>
% \fi
%
% %%%%%%%%%%%%%%%%%%%%%%%%%%%%%%%%%%%%%%%%%%%%%%%%%%%%%%%%%%%%%%%%%%%%%%%%%%%%%%
% %%%%%%%%%%%%%%%%%%%%%%%%%%%%%%%%%%%%%%%%%%%%%%%%%%%%%%%%%%%%%%%%%%%%%%%%%%%%%%
% \section{Sample}
%\iffalse
%<*samplemain>
%\fi
%
% The following presents a sample document
% with two chapters, two parts, a title page,
% a compile flag as well as three forwarding files to set the flag.
% It consists of eight |.tex| files:
% \begin{center}
% \begin{tabular}{ll}
% |cdocsamp.tex|&main file\\
% |cdocsch1.tex|&include file for chapter 1\\
% |cdocsch2.tex|&include file for chapter 2\\
% |cdocspt3.tex|&include file for part 3\\
% |cdocspt4.tex|&include file for part 4\\
% |cdocsdrf.tex|&forwarding file for main file in draft mode\\
% |cdocsfi1.tex|&forwarding file for final version of chapter 1\\
% |cdocsfi2.tex|&forwarding file for final version of chapter 2\\
% \end{tabular}
% \end{center}
% Each of the eight files can be compiled directly by the \LaTeX{} compiler.
%
% %%%%%%%%%%%%%%%%%%%%%%%%%%%%%%%%%%%%%%
% \paragraph{Main File.}
%
% The main file is called |cdocsamp.tex|.
%
% Load the \textsf{childdoc} definitions and
% declare the filename for the main document:
%    \begin{macrocode}
\input{childdoc.def}
\childdocmain{}
%    \end{macrocode}

% Optional override for |\version| flag:
%    \begin{macrocode}
%%\ifchilddoc\else\providecommand{\version}{draft}\fi
%    \end{macrocode}

% Define the default values for the |\version| flag
% (|final| for the main file and |draft| for childs):
%    \begin{macrocode}
\ifchilddoc
\providecommand{\version}{draft}
\else
\providecommand{\version}{final}
\fi
%    \end{macrocode}

% Load the standard document class:
%    \begin{macrocode}
\documentclass[12pt]{article}
%    \end{macrocode}

% Start the document body:
%    \begin{macrocode}
\begin{document}
%    \end{macrocode}

% Declare a title page.
% Print title, part of document being processed and version flag:
%    \begin{macrocode}
\addtocounter{page}{-1}
\begin{center}
{\LARGE\bfseries{}childdoc example\par}
\vspace{1cm}
\ifchilddoc
\ifchilddocmanual part\else chapter\fi:
`\childdocname' of `\childdocjob'\par
\else
main document: `\childdocjob'\par
\fi
version: \version\par
\end{center}
\newpage
%    \end{macrocode}

% Manually include selected file,
% otherwise process as usual:
%    \begin{macrocode}
\ifchilddocmanual
\section*{part `\childdocname'}
\input{\childdocname}
\else
%    \end{macrocode}

% Include the two chapters:
%    \begin{macrocode}
\include{cdocsch1}
\include{cdocsch2}
%    \end{macrocode}

% Include the two parts unless only chapters should be displayed:
%    \begin{macrocode}
\ifchilddoc\else
\section{part three}
\input{cdocspt3}
\section{part four}
\input{cdocspt4}
\fi
%    \end{macrocode}

% Process as usual until here:
%    \begin{macrocode}
\fi
%    \end{macrocode}

% End of document body:
%    \begin{macrocode}
\end{document}
%    \end{macrocode}
%\iffalse
%</samplemain>
%\fi
%
% %%%%%%%%%%%%%%%%%%%%%%%%%%%%%%%%%%%%%%
% \paragraph{Chapter Include Files.}
%
% The include files are called |cdocsch1.tex| and |cdocsch2.tex|.
%
%\iffalse
%<*samplechap1|samplechap2>
%\fi

% Optional override for |\version| flag:
%    \begin{macrocode}
%%\providecommand{\version}{final}
%    \end{macrocode}

% Include the main document:
%    \begin{macrocode}
\input{childdoc.def}
\childdocof{cdocsamp}
%    \end{macrocode}

%\iffalse
%</samplechap1|samplechap2>
%\fi
%
%\iffalse
%<*samplechap1>
%\fi
% Some text for chapter 1:
%    \begin{macrocode}
\section{one}
some text in chapter one
%    \end{macrocode}

%\iffalse
%</samplechap1>
%\fi
% Some text for chapter 2:
%\iffalse
%<*samplechap2>
%\fi
%    \begin{macrocode}
\section{two}
more text in chapter two
%    \end{macrocode}

%\iffalse
%</samplechap2>
%\fi
%
% %%%%%%%%%%%%%%%%%%%%%%%%%%%%%%%%%%%%%%
% \paragraph{Part Include Files.}
%
% The include files are called |cdocspt3.tex| and |cdocspt4.tex|.
%
%\iffalse
%<*samplepart3|samplepart4>
%\fi

% Optional override for |\version| flag:
%    \begin{macrocode}
%%\providecommand{\version}{final}
%    \end{macrocode}

% Include the main document:
%    \begin{macrocode}
\input{childdoc.def}
\childdocby{cdocsamp}
%    \end{macrocode}

%\iffalse
%</samplepart3|samplepart4>
%\fi
%
%\iffalse
%<*samplepart3>
%\fi
% Some text for part 3:
%    \begin{macrocode}
some text in part three
%    \end{macrocode}

%\iffalse
%</samplepart3>
%\fi
% Some text for part 4:
%\iffalse
%<*samplepart4>
%\fi
%    \begin{macrocode}
more text in part four
%    \end{macrocode}

%\iffalse
%</samplepart4>
%\fi
%
% %%%%%%%%%%%%%%%%%%%%%%%%%%%%%%%%%%%%%%
% \paragraph{Forwarding for a Complete Draft.}
%
% The following forwarding file |cdocsdrf.tex|
% compiles the main document in draft mode:
%\iffalse
%<*sampledraft>
%\fi
%    \begin{macrocode}
\def\version{draft}
\input{childdoc.def}
\childdocforward{cdocsamp}
%    \end{macrocode}

%\iffalse
%</sampledraft>
%\fi
%
% %%%%%%%%%%%%%%%%%%%%%%%%%%%%%%%%%%%%%%
% \paragraph{Forwarding for Final Version of the Chapters.}
%
% The following forwarding files |cdocsfn1.tex| and |cdocsfn2.tex|
% (with identical content)
% compile the final versions of the child documents
% |cdocsch1.tex| and |cdocsch2.tex|, respectively:
%\iffalse
%<*samplefinal>
%\fi
%    \begin{macrocode}
\def\version{final}
\input{childdoc.def}
\childdocforwardprefix[cdocsamp]{cdocsfn}{cdocsch}
%    \end{macrocode}

%\iffalse
%</samplefinal>
%\fi
%
% %%%%%%%%%%%%%%%%%%%%%%%%%%%%%%%%%%%%%%
% \paragraph{Command Line Processing.}
%
% The following three command lines generate the output files
% |cdocscld|, |cdocscl1| and |cdocscl2|
% which should be identical to
% |cdocsdrf|, |cdocsch1| and |cdocsfn2|, respectively:
% \begin{center}
% \begin{tabular}{l}
% |latex -jobname cdocscld \|\\
% |  "\def\version{draft}\input{childdoc.def}\childdocforward{cdocsamp}"|\\
% |latex -jobname cdocscl1 \|\\
% |  "\input{childdoc.def}\childdocforward[cdocsamp]{cdocsch1}"|\\
% |latex -jobname cdocscl2 \|\\
% |  "\def\version{final}\input{childdoc.def}\childdocforward{cdocsch2}"|
% \end{tabular}
% \end{center}
% Note that the trailing backslash on each first line
% merely continues the input to the second line
% (for convenient cut ant paste).
% Furthermore, the command |latex| can be replaced by any
% of its alternative versions such as |pdflatex|.
%
% %%%%%%%%%%%%%%%%%%%%%%%%%%%%%%%%%%%%%%%%%%%%%%%%%%%%%%%%%%%%%%%%%%%%%%%%%%%%%%
% %%%%%%%%%%%%%%%%%%%%%%%%%%%%%%%%%%%%%%%%%%%%%%%%%%%%%%%%%%%%%%%%%%%%%%%%%%%%%%
% \section{Implementation}
%\iffalse
%<*package>
%\fi
%
% This section describes the definitions file |childdoc.def|.

% The definitions cannot be loaded using |\usepackage| or |\RequirePackage|
% which has a mechanism to prevent loading a style file more than once.
% When loading the definitions by means of |\input|
% multiple instances have to be prevented manually:
%\iffalse
%This code needs to be before the `\ProvidesFile' directive
%which is defined at the beginning of this file.
%Therefore it is also placed there and commented out here.
%</package>
%<*discard>
%\fi
%    \begin{macrocode}
\ifdefined\childdocmain\endinput\fi
%    \end{macrocode}
%\iffalse
%</discard>
%<*package>
%\fi
%
% \macro{\ifchilddoc}
% \macro{\ifchilddocmanual}
% The conditional |\ifchilddoc| tells whether a
% child (true) or main (false) document is being compiled.
% The conditional |\ifchilddocmanual| tells whether
% the |\includeonly| mechanism is used (false) or
% the selection of child files must be performed manually (true).
% The definitions initialise to false:
%    \begin{macrocode}
\newif\ifchilddoc
\newif\ifchilddocmanual
%    \end{macrocode}

% \macro{\childdocname}
% \macro{\childdocjob}
% The macro |\childdocname| stores the name of the main document
% to be compiled. The macro |\childdocjob| stores the name of
% the document on which the \LaTeX{} compiler was originally invoked.
% The content of |\jobname| cannot be compared
% to filenames specified in the source due to different catcodes.
% The following code rescans |\jobname|, stores the result
% in |\childdocname| and saves a copy in |\childdocjob|:
%    \begin{macrocode}
\edef\childdocname{\scantokens\expandafter{\jobname\noexpand}}
\let\childdocjob\childdocname
%    \end{macrocode}

% \macro{\childdocdisable}
% The macro |\childdocdisable| prevents the main file
% from being processed more than once.
% At this stage, the main document command |\childdocmain|
% is assumed to be called once again where it should do nothing.
% Any subsequent call to it should prevent
% a secondary processing of the main document
% It overwrites the forwarding commands
% |\childdocof| and |\childdocforward|
% with empty macros to prevent further inclusions of the main document:
%    \begin{macrocode}
\newcommand{\childdocdisable}
{
  \renewcommand{\childdocmain}[1]{\renewcommand{\childdocmain}[1]{\endinput}}
  \renewcommand{\childdocof}[1]{}
  \renewcommand{\childdocby}[2][]{}
  \renewcommand{\childdocforward}[2][]{}
  \renewcommand{\childdocdisable}{}
}
%    \end{macrocode}

% \macro{\childdocmain}
% The macro |\childdocmain| is to be called at the top of the main file
% with nothing or the main filename (without extension) as argument.
% First, it breaks loops.
% If the argument is not empty and does not match |\childdocname|
% (which is set by the first inclusion of |childdoc.def|),
% |\ifchilddoc| is set to true, |\includeonly| is applied to the child file
% and |\jobname| is set to the main file
% (for proper handling of |.aux| files):
%    \begin{macrocode}
\newcommand{\childdocmain}[1]
{
  \childdocdisable\childdocmain{}
  \if?#1?\else
    \begingroup
      \def\childdoctmp{#1}
      \ifx\childdoctmp\childdocname
        \def\childdoctmp{}
      \else
        \def\childdoctmp
        {
          \childdoctrue
          \includeonly{\childdocname}
          \def\childdocjob{#1}
          \def\jobname{#1}
        }
      \fi
      \expandafter
    \endgroup
    \childdoctmp
  \fi
}
%    \end{macrocode}

% \macro{\childdocof}
% The command |\childdocof| redirects
% compilation to the main file |#1|.
%    \begin{macrocode}
\newcommand{\childdocof}[1]
{
  \childdocdisable
  \childdoctrue
  \includeonly{\childdocname}
  \def\jobname{#1}
  \def\childdocjob{#1}
  \input{#1}
}
%    \end{macrocode}

% \macro{\childdocby}
% The command |\childdocby| ....
%    \begin{macrocode}
\newcommand{\childdocby}[2][]
{
  \childdocdisable
  \childdoctrue
  \childdocmanualtrue
  \if?#1?\else
    \def\jobname{#2}
  \fi
  \def\childdocjob{#2}
  \input{#2}
  \endinput
}
%    \end{macrocode}

% \macro{\childdocforward}
% The command |\childdocforward| redirects
% compilation to the main file or
% (if the optional argument is given) a child file.
% Parameters are set as if the main file
% or a child file starting with |\childdocof| was compiled.
% Then compilation is handed over to the main file:
%    \begin{macrocode}
\newcommand{\childdocforward}[2][]
{
  \begingroup
    \if?#1?
      \def\childdoctmp
      {
        \def\childdocname{#2}
        \def\childdocjob{#2}
        \def\jobname{#2}
        \input{#2}
        \endinput
      }
    \else
      \def\childdoctmp
      {
        \childdocdisable
        \def\childdocname{#2}
        \childdoctrue
        \includeonly{#2}
        \def\childdocjob{#1}
        \def\jobname{#1}
        \input{#1}
        \endinput
      }
    \fi
    \expandafter
  \endgroup
  \childdoctmp
}
%    \end{macrocode}

% \macro{\childdocforwardprefix}
% The command |\childdocforwardprefix| redirects
% compilation to the main or a child file by means of a pattern.
% The prefix |#1| in the current filename is replaced by |#2|
% and the suffix of the current filename is kept
% (it is assumed that the filename does not contain the substring `|~~~|'
% which is used as a delimiter).
% Compilation is handed over to the new file by |\childdocforward|:
%    \begin{macrocode}
\newcommand{\childdocforwardprefix}[3][]
{
  \begingroup
    \def\childdocextract #2##1~~~{\def\childdoctmp{\childdocforward[#1]{#3##1}}}
    \expandafter\childdocextract\childdocname~~~
    \expandafter
  \endgroup
  \childdoctmp
}
%    \end{macrocode}

% \macro{\childdoc}
% The deprecated macro |\childdoc| is a legacy version of |\childdocmain|:
%    \begin{macrocode}
\newcommand{\childdoc}{\childdocmain}
%    \end{macrocode}

% \macro{\childdocredirect}
% The deprecated macro |\childdocredirect| is a legacy version
% of |\childdocforward| and |\childdocforwardprefix|:
%    \begin{macrocode}
\newcommand{\childdocredirect}[2][]
{
  \begingroup
    \if?#1?
      \def\childdoctmp{\childdocforward{#2}}
    \else
      \def\childdoctmp{\childdocforwardprefix{#1}{#2}}
    \fi
    \expandafter
  \endgroup
  \childdoctmp
}
%    \end{macrocode}

%\iffalse
%</package>
%\fi
%
\endinput
|
and perform the replacements as outlined below.
Instead of |\childdocmain{|\textit{main}|}| add the following code
to the top of the main file:
%
\begin{center}
\begin{tabular}{l}
|\||ifdefined\childdocname\endinput\||fi\newif\ifchilddoc|\\
|\edef\childdocname{\scantokens\expandafter{\jobname\noexpand}}|\\
|\def\childdocmain{|\textit{main}|}\||ifx\childdocmain\childdocname\||else|\\
|\childdoctrue\includeonly{\childdocname}\let\jobname\childdocmain\||fi|\\
\end{tabular}
\end{center}
%
Instead of |\childdocof{|\textit{main}|}| just include the main file
at the top of each child file:
%
\begin{center}
|\input{|\textit{main}|}|
\end{center}
%
A simple redirection |\childdocforward{|\textit{dest}|}| is achieved by:
%
\begin{center}
|\def\jobname{|\textit{dest}|}\input{\jobname}|
\end{center}
%
The redirection with prefix
|\childdocforwardprefix[|\textit{prefix}|]{|\textit{dest}|}|
is accomplished by:
%
\begin{center}
\begin{tabular}{l}
|{\edef\jobname{\scantokens\expandafter{\jobname\noexpand}}|\\
|\def\redirectjob |\textit{prefix}|#1~~~{\gdef\jobname{|\textit{dest}|#1}}|\\
|\expandafter\redirectjob\jobname~~~}\input{\jobname}|
\end{tabular}
\end{center}

In an alternative approach,
child documents can be compiled by a specific command line
without additional code or specific definitions:
%
\begin{center}
|... -jobname "|\textit{target}|" "|[\textit{flags}]%
|\includeonly{|\textit{dest}|}\input{|\textit{main}|}"|
\end{center}
%

%%%%%%%%%%%%%%%%%%%%%%%%%%%%%%%%%%%%%%%%%%%%%%%%%%%%%%%%%%%%%%%%%%%%%%%%%%%%%%%%
%%%%%%%%%%%%%%%%%%%%%%%%%%%%%%%%%%%%%%%%%%%%%%%%%%%%%%%%%%%%%%%%%%%%%%%%%%%%%%%%
\section{Information}

%%%%%%%%%%%%%%%%%%%%%%%%%%%%%%%%%%%%%%%%%%%%%%%%%%%%%%%%%%%%%%%%%%%%%%%%%%%%%%%%
\subsection{Copyright}

Copyright \copyright{} 2017--2018 Niklas Beisert

This work may be distributed and/or modified under the
conditions of the \LaTeX{} Project Public License, either version 1.3
of this license or (at your option) any later version.
The latest version of this license is in
  \url{http://www.latex-project.org/lppl.txt}
and version 1.3 or later is part of all distributions of \LaTeX{}
version 2005/12/01 or later.

This work has the LPPL maintenance status `maintained'.

The Current Maintainer of this work is Niklas Beisert.

This work consists of the files |README.txt|, |childdoc.ins| and |childdoc.dtx|
as well as the derived files |childdoc.def|, |cdocsamp.tex|
with |cdocsch1.tex|, |cdocsch2.tex|, |cdocspt3.tex|, |cdocspt4.tex|,
|cdocsdrf.tex|, |cdocsfn1.tex|, |cdocsfn2.tex|
as well as |childdoc.pdf|.

%%%%%%%%%%%%%%%%%%%%%%%%%%%%%%%%%%%%%%%%%%%%%%%%%%%%%%%%%%%%%%%%%%%%%%%%%%%%%%%%
\subsection{Files and Installation}

The package consists of the files:
%
\begin{center}
\begin{tabular}{ll}
    |README.txt|   & readme file \\
    |childdoc.ins| & installation file \\
    |childdoc.dtx| & source file \\
    |childdoc.def| & definition file \\
    |cdocsamp.tex| & sample main file \\
    |cdocsch1.tex| & sample include file \\
    |cdocsch2.tex| & sample include file \\
    |cdocspt3.tex| & sample part file \\
    |cdocspt4.tex| & sample part file \\
    |cdocsdrf.tex| & sample redirection file \\
    |cdocsfn1.tex| & sample redirection file \\
    |cdocsfn2.tex| & sample redirection file \\
    |childdoc.pdf| & manual
\end{tabular}
\end{center}
%
The distribution consists of the files
|README.txt|, |childdoc.ins| and |childdoc.dtx|.
%
\begin{itemize}
\item
Run (pdf)\LaTeX{} on |childdoc.dtx|
to compile the manual |childdoc.pdf| (this file).
\item
Run \LaTeX{} on |childdoc.ins| to create the definitions file |childdoc.def|
and the sample |cdocsamp.tex| with include files
|cdocsch1.tex|, |cdocsch2.tex|, |cdocspt3.tex|, |cdocspt4.tex|,
|cdocsdrf.tex|, |cdocsfn1.tex|, |cdocsfn2.tex|.
Then copy the file |childdoc.def| to an appropriate directory of your \LaTeX{}
distribution, e.g.\ \textit{texmf-root}|/tex/latex/childdoc|.
\end{itemize}

%%%%%%%%%%%%%%%%%%%%%%%%%%%%%%%%%%%%%%%%%%%%%%%%%%%%%%%%%%%%%%%%%%%%%%%%%%%%%%%%
\subsection{Related CTAN Packages}

There are several other packages which offer a similar functionality:
%
\begin{itemize}
\item
The packages
\href{http://ctan.org/pkg/docmute}{\textsf{docmute}},
\href{http://ctan.org/pkg/includex}{\textsf{includex}} and
\href{http://ctan.org/pkg/standalone}{\textsf{standalone}}
provide commands to include only the document body of
a child file thus allowing both files to be compiled individually.
\item
The packages \href{http://ctan.org/pkg/subdocs}{\textsf{subdocs}}
and \href{http://ctan.org/pkg/subfiles}{\textsf{subfiles}}
provide structures in which the main and child documents can be
encapsulated and allowing them to be compiled individually.
The inclusion mechanism is different from the conventional |\include|.
\item
The package \href{http://ctan.org/pkg/combine}{\textsf{combine}}
is an elaborate solution to combine several documents into one.
\end{itemize}
%
See also the CTAN topic \href{http://ctan.org/topic/subdocs}{\textsf{subdocs}}
for further related packages.
The present package differs from the above solutions in that
a document structure constructed with the conventional |\include| mechanism
just needs two extra commands at the top of every file
such that all constituent files can be compiled individually.

%%%%%%%%%%%%%%%%%%%%%%%%%%%%%%%%%%%%%%%%%%%%%%%%%%%%%%%%%%%%%%%%%%%%%%%%%%%%%%%%
%\subsection{Feature Suggestions}
%
%The following is a list of features which may be useful for future
%versions of this package:
%%
%\begin{itemize}
%\item
%\ldots
%\end{itemize}

%%%%%%%%%%%%%%%%%%%%%%%%%%%%%%%%%%%%%%%%%%%%%%%%%%%%%%%%%%%%%%%%%%%%%%%%%%%%%%%%
\subsection{Revision History}

%%%%%%%%%%%%%%%%%%%%%%%%%%%%%%%%%%%%%%%%
\paragraph{v2.0:} 2018/12/30

\begin{itemize}
\item
immediate forward processing
\item
added |\childdocby| mechanism
\item
manual restructured
\end{itemize}

%%%%%%%%%%%%%%%%%%%%%%%%%%%%%%%%%%%%%%%%
\paragraph{v1.6:} 2018/01/17

\begin{itemize}
\item
application for development of include files
\item
corrections to manual
\end{itemize}

%%%%%%%%%%%%%%%%%%%%%%%%%%%%%%%%%%%%%%%%
\paragraph{v1.5:} 2017/05/21

\begin{itemize}
\item
more complete structuring introduced
\item
|\childdocof| introduced
\item
|\childdoc| renamed to |\childdocmain|
\item
|\childredirect| renamed to |\childdocforward| and |\childdocforwardprefix|
and functionality expanded
\end{itemize}

%%%%%%%%%%%%%%%%%%%%%%%%%%%%%%%%%%%%%%%%
\paragraph{v1.0:} 2017/04/27

\begin{itemize}
\item
manual and install package
\item
first version published on CTAN
\end{itemize}

%%%%%%%%%%%%%%%%%%%%%%%%%%%%%%%%%%%%%%%%
\paragraph{v0.6:} 2017/04/26

\begin{itemize}
\item
redirection mechanism added
\end{itemize}

%%%%%%%%%%%%%%%%%%%%%%%%%%%%%%%%%%%%%%%%
\paragraph{v0.5:} 2017/04/26

\begin{itemize}
\item
functionality in definition file
\end{itemize}


%%%%%%%%%%%%%%%%%%%%%%%%%%%%%%%%%%%%%%%%%%%%%%%%%%%%%%%%%%%%%%%%%%%%%%%%%%%%%%%%
%%%%%%%%%%%%%%%%%%%%%%%%%%%%%%%%%%%%%%%%%%%%%%%%%%%%%%%%%%%%%%%%%%%%%%%%%%%%%%%%
%%%%%%%%%%%%%%%%%%%%%%%%%%%%%%%%%%%%%%%%%%%%%%%%%%%%%%%%%%%%%%%%%%%%%%%%%%%%%%%%
\appendix

\settowidth\MacroIndent{\rmfamily\scriptsize 000\ }

 \DocInput{childdoc.dtx}

\end{document}
%</driver>
% \fi
%
% %%%%%%%%%%%%%%%%%%%%%%%%%%%%%%%%%%%%%%%%%%%%%%%%%%%%%%%%%%%%%%%%%%%%%%%%%%%%%%
% %%%%%%%%%%%%%%%%%%%%%%%%%%%%%%%%%%%%%%%%%%%%%%%%%%%%%%%%%%%%%%%%%%%%%%%%%%%%%%
% \section{Sample}
%\iffalse
%<*samplemain>
%\fi
%
% The following presents a sample document
% with two chapters, two parts, a title page,
% a compile flag as well as three forwarding files to set the flag.
% It consists of eight |.tex| files:
% \begin{center}
% \begin{tabular}{ll}
% |cdocsamp.tex|&main file\\
% |cdocsch1.tex|&include file for chapter 1\\
% |cdocsch2.tex|&include file for chapter 2\\
% |cdocspt3.tex|&include file for part 3\\
% |cdocspt4.tex|&include file for part 4\\
% |cdocsdrf.tex|&forwarding file for main file in draft mode\\
% |cdocsfi1.tex|&forwarding file for final version of chapter 1\\
% |cdocsfi2.tex|&forwarding file for final version of chapter 2\\
% \end{tabular}
% \end{center}
% Each of the eight files can be compiled directly by the \LaTeX{} compiler.
%
% %%%%%%%%%%%%%%%%%%%%%%%%%%%%%%%%%%%%%%
% \paragraph{Main File.}
%
% The main file is called |cdocsamp.tex|.
%
% Load the \textsf{childdoc} definitions and
% declare the filename for the main document:
%    \begin{macrocode}
% \iffalse
%
% childdoc.dtx Copyright (C) 2017-2018 Niklas Beisert
%
% This work may be distributed and/or modified under the
% conditions of the LaTeX Project Public License, either version 1.3
% of this license or (at your option) any later version.
% The latest version of this license is in
%   http://www.latex-project.org/lppl.txt
% and version 1.3 or later is part of all distributions of LaTeX
% version 2005/12/01 or later.
%
% This work has the LPPL maintenance status `maintained'.
%
% The Current Maintainer of this work is Niklas Beisert.
%
% This work consists of the files childdoc.dtx and childdoc.ins
% and the derived files childdoc.def and cdocsamp.tex with
% cdocsch1.tex, cdocsch2.tex, cdocsdrf.tex, cdocsfn1.tex, cdocsfn2.tex.
%
%<package>\ifdefined\childdocmain\endinput\fi
%<package>\ProvidesFile{childdoc.def}[2018/12/30 v2.0 child document driver]
%<samplemain>\ProvidesFile{cdocsamp.tex}[2018/12/30 v2.0 sample for childdoc]
%<*driver>
%\ProvidesFile{childdoc.drv}[2018/12/30 v2.0 childdoc reference manual file]
\PassOptionsToClass{10pt,a4paper}{article}
\documentclass{ltxdoc}

\usepackage[margin=35mm]{geometry}
\usepackage{hyperref}
\usepackage{hyperxmp}
\usepackage[usenames]{color}

\hypersetup{colorlinks=true}
\hypersetup{pdfstartview=FitH}
\hypersetup{pdfpagemode=UseNone}
\hypersetup{pdfsource={}}
\hypersetup{pdflang={en-UK}}
\hypersetup{pdfcopyright={Copyright 2017-2018 Niklas Beisert.
  This work may be distributed and/or modified under the
  conditions of the LaTeX Project Public License, either version 1.3
  of this license or (at your option) any later version.}}
\hypersetup{pdflicenseurl={http://www.latex-project.org/lppl.txt}}
\hypersetup{pdfcontactaddress={ETH Zurich, ITP, HIT K,
  Wolfgang-Pauli-Strasse 27}}
\hypersetup{pdfcontactpostcode={8093}}
\hypersetup{pdfcontactcity={Zurich}}
\hypersetup{pdfcontactcountry={Switzerland}}
\hypersetup{pdfcontactemail={nbeisert@itp.phys.ethz.ch}}
\hypersetup{pdfcontacturl={http://people.phys.ethz.ch/\xmptilde nbeisert/}}

\newcommand{\secref}[1]{\hyperref[#1]{section \ref*{#1}}}

\parskip1ex
\parindent0pt
\let\olditemize\itemize
\def\itemize{\olditemize\parskip0pt}

\begin{document}

\title{The \textsf{childdoc} Package}
\hypersetup{pdftitle={The childdoc Package}}
\author{Niklas Beisert\\[2ex]
  Institut f\"ur Theoretische Physik\\
  Eidgen\"ossische Technische Hochschule Z\"urich\\
  Wolfgang-Pauli-Strasse 27, 8093 Z\"urich, Switzerland\\[1ex]
  \href{mailto:nbeisert@itp.phys.ethz.ch}
  {\texttt{nbeisert@itp.phys.ethz.ch}}}
\hypersetup{pdfauthor={Niklas Beisert}}
\hypersetup{pdfsubject={Manual for the LaTeX2e Package childdoc}}
\date{30 December 2018, \textsf{v2.0}}
\maketitle

\begin{abstract}\noindent
\textsf{childdoc} is a \LaTeXe{} package
that enables the direct compilation
of document sections included by |\include|
to individual files.
\end{abstract}

\begingroup
\parskip0ex
\tableofcontents
\endgroup

%%%%%%%%%%%%%%%%%%%%%%%%%%%%%%%%%%%%%%%%%%%%%%%%%%%%%%%%%%%%%%%%%%%%%%%%%%%%%%%%
%%%%%%%%%%%%%%%%%%%%%%%%%%%%%%%%%%%%%%%%%%%%%%%%%%%%%%%%%%%%%%%%%%%%%%%%%%%%%%%%
\section{Introduction}

\LaTeX{} provides a mechanism to structure a large document (such as a book)
into a main file and several child files (containing the chapters)
using the |\include| command.
This mechanism is beneficial for documents
which span hundreds of pages in order to
make the source file(s) more manageable.
Moreover, compilation can be restricted to
selected child files by means of the |\includeonly| command.
The latter feature can be used to reduce the compilation time while editing
(this was significantly more useful in the earlier days of \LaTeX{})
or to generate a smaller document which is easier to navigate.
Another application of |\includeonly| is to generate
documents consisting of selected parts of the complete document.

However, there are a few drawbacks of the plain |\include| mechanism:
\begin{itemize}
\item
The child files cannot be compiled on their own,
they can only be compiled via the main file.
A naive editing environment
(such as a text editor with an option
to have the current file processed by \LaTeX)
may require one to switch to the main file before compiling;
attempting to compile the child file produces errors.
\item
The main file must be modified (each time)
to adjust the |\includeonly| command
to the present needs. This easily leaves the main file in a messy state.
\item
The generated document will always carry the filename
of the main document. This is inconvenient if
several child files are to be compiled and
to be kept for distribution.
\end{itemize}

The present package provides a simple interface
to make child files individually compilable by \LaTeX{}.
Compiling a child file then has the same effect as compiling
the main file with an |\includeonly| command
to select the appropriate child.
Moreover the generated document will carry the name of the child
rather than the main file.
This resolves all three above issues.

This feature is meant to make the editing of books,
thesis documents and lecture notes somewhat more convenient.
However, the package can also be used efficiently for
composing a series of documents (such as exercise sheets)
which are typically distributed individually.
It then assists the author in generating the individual documents
(potentially in different versions)
as well as a document containing the collected series.
Another application is in developing style files
or other kinds of included material
where compilation of the style file could redirect
to a sample or test file.

%%%%%%%%%%%%%%%%%%%%%%%%%%%%%%%%%%%%%%%%%%%%%%%%%%%%%%%%%%%%%%%%%%%%%%%%%%%%%%%%
%%%%%%%%%%%%%%%%%%%%%%%%%%%%%%%%%%%%%%%%%%%%%%%%%%%%%%%%%%%%%%%%%%%%%%%%%%%%%%%%
\section{Usage}

First of all, the package \textsf{childdoc} is \emph{not} a standard
\LaTeXe{} |.sty| style file! Therefore it needs to be invoked in
a non-standard way.

%%%%%%%%%%%%%%%%%%%%%%%%%%%%%%%%%%%%%%%%%%%%%%%%%%%%%%%%%%%%%%%%%%%%%%%%%%%%%%%%
\subsection{Included Files}
\label{sec:include}

%%%%%%%%%%%%%%%%%%%%%%%%%%%%%%%%%%%%%%%%
\DescribeMacro{\childdocmain}
To use the package, add the commands
\begin{center}
\begin{tabular}{l}
|\input{childdoc.def}|\\
|\childdocmain{}|\\
\end{tabular}
\end{center}
at the very top of the main \LaTeX{} file,
in particular \emph{before} the |\documentclass| statement!
The argument of |\childdocmain| should be left empty
(but it must be present).

%%%%%%%%%%%%%%%%%%%%%%%%%%%%%%%%%%%%%%%%
\DescribeMacro{\childdocof}
Furthermore, add the commands
\begin{center}
\begin{tabular}{l}
|\input{childdoc.def}|\\
|\childdocof{|\textit{main}|}|\\
\end{tabular}
\end{center}
at the top of every child file \textit{child}
which is included by |\include{|\textit{child}|}|
from within the main file
(or at least for those files to be compiled individually).
The argument \textit{main} must be the filename of the main file.

There are a couple of
considerations in setting up the main and child documents:

%%%%%%%%%%%%%%%%%%%%%%%%%%%%%%%%%%%%%%%%
\paragraph{Restrictions.}

Please note the following restrictions:
\begin{itemize}
\item
|\childdocmain| must be called with one argument \textit{main}
to ensure compatibility with earlier version of the package.
It must either be empty (|\childdocmain{}|)
or precisely match the filename of the main file in which it is specified.
See \secref{sec:detection} for further information.
\item
The filename \textit{main} must be specified without the |.tex| extension.
\item
The filename \textit{main} is case sensitive
(even in case-insensitive file systems)
due to internal string comparison.
\item
The argument \textit{main} should be fully expanded, it cannot be a macro.
\item
Subdirectories and special characters should be avoided in filenames.
\item
The command |\childdocmain{|\textit{main}|}| must be followed by a whitespace.
It should not be followed immediately by another command
or by a comment mark `|%|'.
This is because the \TeX{} parser reads the token immediately following
the argument of |\childdocmain| and puts it
at the beginning of every child section;
however, a white\-space is ignored.
\end{itemize}

%%%%%%%%%%%%%%%%%%%%%%%%%%%%%%%%%%%%%%%%
\paragraph{Content of Main File.}

It is advisable to place all content in the child files included by |\include|.
Any output contained in the main file will appear in all child documents
unless suppressed manually;
it cannot be suppressed automatically by the |\includeonly| directive
and thus should normally be avoided.
A method to include some content in the main file
by means of conditional processing is described in \secref{sec:conditional}.

%%%%%%%%%%%%%%%%%%%%%%%%%%%%%%%%%%%%%%%%
\paragraph{Page Numbering.}

When only a part of the document is compiled,
the appropriate numbering of pages
(as well as other status parameters)
is determined from the |.aux| files.
The latter contain information from previous passes.
However this information needs to propagate through
all intermediate child documents.
Therefore the page numbering in child documents may well
be inconsistent until the complete document is compiled at least once.

A useful (if unconventional) way to always ensure a consistent
page numbering is to restart the numbering in each child document
and denote the pages by `\textit{child}|.|\textit{page}'
where \textit{child} represents the chapter/section number of the child file.
This can be achieved by the command
|\numberwithin{page}{|\textit{child}|}|
of the \textsf{amsmath} package
where \textit{child} can be |chapter| or |section|
depending on the chosen structuring.
Alternatively, one can modify the macro |\thepage| appropriately
and reset the counter |page| at the start of each child file.

%%%%%%%%%%%%%%%%%%%%%%%%%%%%%%%%%%%%%%%%%%%%%%%%%%%%%%%%%%%%%%%%%%%%%%%%%%%%%%%%
\subsection{Conditional Processing}
\label{sec:conditional}

The package provides a mechanism to compile different versions
of a document. To customise the versions further some conditional processing
can come in handy to distinguish which version is being compiled.
The package provides two macros to describe the compilation context:

%%%%%%%%%%%%%%%%%%%%%%%%%%%%%%%%%%%%%%%%
\DescribeMacro{\ifchilddoc}
The conditional |\ifchilddoc| distinguishes between the compilation of
child documents and the main document:
%
\begin{center}
|\ifchilddoc |\textit{child-code}| |[|\||else |\textit{main-code}]| \||fi|
\end{center}

%%%%%%%%%%%%%%%%%%%%%%%%%%%%%%%%%%%%%%%%
\DescribeMacro{\childdocname}
\DescribeMacro{\childdocjob}
The macro |\childdocname| contains the filename (without extension)
of the main or child file being processed.
Note that |\childdocjob| will always contain the name of the main file.

%%%%%%%%%%%%%%%%%%%%%%%%%%%%%%%%%%%%%%%%
\paragraph{Title Page.}

Conditional processing can be used to include a title or banner page
in the main document when proper precautions are taken.
Importantly, the code in the main file should ensure that the page counter
(as well as other status parameters which are stored in the |.aux| files)
takes the same value after the conditional processing.
Otherwise the page numbers may take divergent values
depending on which part is compiled.

For example, a title page could be declared by:
%
\begin{center}
\begin{tabular}{l}
|\ifchilddoc\||else|\\
|\addtocounter{page}{-1}|\\
\textit{code for title page}\\
|\newpage|\\
|\||fi|
\end{tabular}
\end{center}
%
A banner page for the child documents can be generated by:
%
\begin{center}
\begin{tabular}{l}
|\ifchilddoc|\\
|\addtocounter{page}{-1}|\\
\textit{code for banner page}\\
|\newpage|\\
|\||fi|
\end{tabular}
\end{center}
%
Here one could write a message such as:
\begin{center}
|This is the part \childdocname{} of \childdocjob{}.|
\end{center}

%%%%%%%%%%%%%%%%%%%%%%%%%%%%%%%%%%%%%%%%%%%%%%%%%%%%%%%%%%%%%%%%%%%%%%%%%%%%%%%%
\subsection{Flags}
\label{sec:flags}

The package makes it easy to generate different versions
of the main or child documents.
To this end compilation flags can be defined
and assigned different default values.
They will be particularly useful in conjunction
with the forwarding mechanism described in \secref{sec:forward}.

For example, it may be useful to have a flag |\version|
which can be set to |draft| or |final|.
The document source will contain some conditional code
depending on the value of |\version|.
Suppose further, the flag should default to |final| for the main file
and to |draft| for child files
which is a natural assignment for editing the document.
This is achieved by placing the following code
in the preamble of the main document
(below the |\childdocmain| directive):
%
\begin{center}
\begin{tabular}{l}
|\ifchilddoc|\\
|\providecommand{\version}{draft}|\\
|\||else|\\
|\providecommand{\version}{final}|\\
|\||fi|
\end{tabular}
\end{center}
%
The definition by |\providecommand| makes sure
that previous definitions are not overwritten.
Further statements |\providecommand{\version}{...}|
can thus be added before the above code to override it.

For the main file, one might add a line
(between |\childdocmain| and the above block)
%
\begin{center}
|%\ifchilddoc\||else\providecommand{\version}{draft}\||fi|
\end{center}
%
which can be uncommented to produce a draft version.
Likewise one can add a line to the very top of a child file
(above the |\childdocof{|\textit{main}|}| directive)
%
\begin{center}
|%\providecommand{\version}{final}|
\end{center}
%
which can be uncommented to produce the final version of this child document.

%%%%%%%%%%%%%%%%%%%%%%%%%%%%%%%%%%%%%%%%%%%%%%%%%%%%%%%%%%%%%%%%%%%%%%%%%%%%%%%%
\subsection{Forwarding}
\label{sec:forward}

Different versions of the main or child documents
using compilation flags as described in \secref{sec:flags}
can be (permanently) stored in different files
for convenient compilation, viewing and distribution.
To this end, the package defines a command
to pass on compilation to a different file:

%%%%%%%%%%%%%%%%%%%%%%%%%%%%%%%%%%%%%%%%
\DescribeMacro{\childdocforward}
The command |\childdocforward| redirects processing to
another source file:
%
\begin{center}
\begin{tabular}{l}
|\input{childdoc.def}|\\
|\childdocforward[|\textit{main}|]{|\textit{dest}|}|\\
\end{tabular}
\end{center}
%
The argument \textit{dest} is the destination file
(without extension).
It should be the main file or one of the child files.
Note that further \textsf{childdoc} directives
such as |\childdocof| and |\childdocforward|
in the indicated file will be processed in this form.
The optional argument \textit{main}
passes on directly to the main file \textit{main}
while pretending to compile the child \textit{dest}.
This form behaves as if \textit{dest}
issues |\childdocof{|\textit{main}|}| right away,
and no further \textsf{childdoc} directives will be processed.

%%%%%%%%%%%%%%%%%%%%%%%%%%%%%%%%%%%%%%%%
\DescribeMacro{\...prefix}
In the alternative form |\childdocforwardprefix|,
%
\begin{center}
\begin{tabular}{l}
|\input{childdoc.def}|\\
|\childdocforwardprefix[|\textit{main}|]{|\textit{prefix}|}{|\textit{dest}|}|
\end{tabular}
\end{center}
%
the destination file is determined by a pattern
depending on the current file:
To make this work, the current file must be called
`{\textit{prefix}\hspace{0.2em}\textit{suffix}}'
with \textit{prefix} matching precisely the argument.
Processing is then passed on to the file
`{\textit{dest}\hspace{0.2em}\textit{suffix}}'.
Surely, the same effect is achieved by
directly specifying the
argument `{\textit{dest}\hspace{0.2em}\textit{suffix}}'
in the first form.
However, that requires to set up a different file
for each child. With the alternative form of the command
all these files can have exactly the same content
which simplifies setting them up and maintaining them.

For example, the following file |draft.tex|
with a compilation flag |\version| as described in \secref{sec:flags}
compiles the main document as a draft:
%
\begin{center}
\begin{tabular}{l}
|\def\version{draft}|\\
|\input{childdoc.def}|\\
|\childdocforward{|\textit{main}|}|
\end{tabular}
\end{center}
%
Likewise, the following files |final|\textit{nn}|.tex|
compile the final version of the child document
|child|\textit{nn}|.tex|:
%
\begin{center}
\begin{tabular}{l}
|\def\version{final}|\\
|\input{childdoc.def}|\\
|\childdocforwardprefix{final}{child}|
\end{tabular}
\end{center}
%

Note that when several versions of a main file and/or of each child file
are to be generated, it may be convenient to set up a |Makefile| or
shell script to automatise the process.

%%%%%%%%%%%%%%%%%%%%%%%%%%%%%%%%%%%%%%%%%%%%%%%%%%%%%%%%%%%%%%%%%%%%%%%%%%%%%%%%
\subsection{Command Line Processing}
\label{sec:commandline}

The effect of redirection files can also be achieved by invoking
the \LaTeX{} compiler with a more elaborate command line.
Most conveniently this should be done as part
of a shell script or a |Makefile|.

When using \textsf{childdoc} in the main file, the following
command lines effectively perform a redirection
(note that depending on the shell being used,
backslashes may have to be doubled: `|\|' $\to$ `|\\|'):
%
\begin{center}
|... -jobname "|\textit{target}|" |\\|"|[\textit{flags}]%
|\input{childdoc.def}\childdocforward[|\textit{main}|]{|\textit{dest}|}"|
\end{center}
%
Here \textit{target} is the name of the output file,
\textit{main} is the name of the main file
and \textit{dest} is the name of the main or child file to be processed
(all filenames without extensions).
The optional argument \textit{main} can be omitted
if \textit{main} matches \textit{dest}.
Optionally, compilation \textit{flags} can be defined via |\def| commands.
This command line makes the \TeX{} engine believe
it is compiling the file \textit{target}
whose content is specified as the latter parameter.
The provided code then forwards the processing to
\textit{main} or \textit{dest} as described in \secref{sec:forward}.

%%%%%%%%%%%%%%%%%%%%%%%%%%%%%%%%%%%%%%%%%%%%%%%%%%%%%%%%%%%%%%%%%%%%%%%%%%%%%%%%
\subsection{Include by Input}
\label{sec:input}

Including child documents by |\include| has some restrictions by design.
Most notably, the content of a child document always occupies
its own set of pages; pages cannot be shared between child documents.
Usually, this behaviour makes perfect sense
because each child document contain an essential part of the document.
However, in some situations it may be desirable to compose
a document from a collection of parts
without having mandatory page breaks between then.
For this case, the package
provides a mechanism to include parts
by |\input| which can also be processed individually.
However, by construction this mechanism
requires manual handling of the content to be output.

%%%%%%%%%%%%%%%%%%%%%%%%%%%%%%%%%%%%%%%%
\DescribeMacro{\ifchilddocmanual}
The main file should be prepared as usual, see \secref{sec:include}.
However, the document body must make a distinction
between processing of an individual part and of the main document, e.g.:
%
\begin{center}
\begin{tabular}{l}
|\ifchilddocmanual|\\
|\input{\childdocname}|\\
|\||else|\\
\textit{document body with }|\input{|\textit{part}|}|\\
|\||fi|
\end{tabular}
\end{center}
%
The conditional |\ifchilddocmanual| is true whenever
a part to be included by |\input| is being compiled,
and the name of the part is stored in |\childdocname|.

%%%%%%%%%%%%%%%%%%%%%%%%%%%%%%%%%%%%%%%%
\DescribeMacro{\childdocby}
Each part to be included by |\input| should start with:
%
\begin{center}
\begin{tabular}{l}
|\input{childdoc.def}|\\
|\childdocby{|\textit{main}|}|\\
\end{tabular}
\end{center}
%
The directive |\childdocby| is similar to |\childdocof|
described in \secref{sec:include},
but the subsequent selection of content must be done manually.
To that end, both |\ifchilddoc| and |\ifchilddocmanual|
will be true upon processing of a part,
and the name of the part is stored in |\childdocname|.
Note that |\jobname| will be set to the filename of the current part
so that each part receives an individual |.aux| file
that does not interfere with the |.aux| file(s) of the main document.
This behaviour can be altered by the alternative form
|\childdocby[*]{|\textit{main}|}| (with a non-empty optional argument)
which uses the |.aux| file of the main document
by setting |\jobname| to \textit{main}.

%%%%%%%%%%%%%%%%%%%%%%%%%%%%%%%%%%%%%%%%%%%%%%%%%%%%%%%%%%%%%%%%%%%%%%%%%%%%%%%%
\subsection{Driver Development}
\label{sec:driver}

The \textsf{childdoc} mechanism can also be use for the development
of definition files such as \LaTeX{} styles or classes.
This case differs from the above setup with multiple parts
included by |\include| in that no |\includeonly| should be invoked.
This can be achieved by starting the include file
(before |\ProvidesPackage|) with:
%
\begin{center}
\begin{tabular}{l}
|\input{childdoc.def}|\\
|\childdocforward{|\textit{main}|}|\\
\end{tabular}
\end{center}
%
or alternatively with:
%
\begin{center}
\begin{tabular}{l}
|\input{childdoc.def}|\\
|\childdocby{|\textit{main}|}|\\
\end{tabular}
\end{center}
%
Both forms have slightly different effects as described above.
The main file is prepared as usual, see \secref{sec:include}.

%%%%%%%%%%%%%%%%%%%%%%%%%%%%%%%%%%%%%%%%%%%%%%%%%%%%%%%%%%%%%%%%%%%%%%%%%%%%%%%%
\subsection{Legacy Detection}
\label{sec:detection}

The directive |\childdocmain| in the main file can detect
whether the complete document or merely a child is to be compiled
even without using the directive |\childdocof|.
This method is deprecated because it is less robust
and there is no compelling reason to use it;
it is merely provided for backward compatibility
and it may be removed in future versions.

If the detection mechanism is to be used,
it is mandatory to correctly specify
the filename of the main file as the argument of |\childdocmain|:
%
\begin{center}
\begin{tabular}{l}
|\input{childdoc.def}|\\
|\childdocmain{|\textit{main}|}|\\
\end{tabular}
\end{center}
%
If |\jobname| does not match the argument \textit{main} of |\childdocmain|,
it is assumed that |\jobname| points to the child file to be compiled.
When using |\childdocmain| with the main file specified as argument,
it suffices to start a child file
with just |\input{|\textit{main}|}|
without loading of the package and using |\childdocof|.
If instead all processing is done
with the appropriate \textsf{childdoc} directives,
the argument of \textit{main} of |\childdocmain| can be empty.

An alternative version of the command line processing described
in \secref{sec:commandline} using the detection mechanism reads:
%
\begin{center}
|... -jobname "|\textit{target}|" "|[\textit{flags}]%
[|\def\jobname{|\textit{dest}|}|]|\input{|\textit{main}|}"|
\end{center}

%%%%%%%%%%%%%%%%%%%%%%%%%%%%%%%%%%%%%%%%%%%%%%%%%%%%%%%%%%%%%%%%%%%%%%%%%%%%%%%%
\subsection{Manual Code}
\label{sec:manual}

In case one cannot be certain whether the definitions file |childdoc.def|
is installed on the target \TeX{} distribution
and one prefers not to ship it,
it is conceivable to paste a few relevant commands into the sources.

To that end, drop all statements |\input{childdoc.def}|
and perform the replacements as outlined below.
Instead of |\childdocmain{|\textit{main}|}| add the following code
to the top of the main file:
%
\begin{center}
\begin{tabular}{l}
|\||ifdefined\childdocname\endinput\||fi\newif\ifchilddoc|\\
|\edef\childdocname{\scantokens\expandafter{\jobname\noexpand}}|\\
|\def\childdocmain{|\textit{main}|}\||ifx\childdocmain\childdocname\||else|\\
|\childdoctrue\includeonly{\childdocname}\let\jobname\childdocmain\||fi|\\
\end{tabular}
\end{center}
%
Instead of |\childdocof{|\textit{main}|}| just include the main file
at the top of each child file:
%
\begin{center}
|\input{|\textit{main}|}|
\end{center}
%
A simple redirection |\childdocforward{|\textit{dest}|}| is achieved by:
%
\begin{center}
|\def\jobname{|\textit{dest}|}\input{\jobname}|
\end{center}
%
The redirection with prefix
|\childdocforwardprefix[|\textit{prefix}|]{|\textit{dest}|}|
is accomplished by:
%
\begin{center}
\begin{tabular}{l}
|{\edef\jobname{\scantokens\expandafter{\jobname\noexpand}}|\\
|\def\redirectjob |\textit{prefix}|#1~~~{\gdef\jobname{|\textit{dest}|#1}}|\\
|\expandafter\redirectjob\jobname~~~}\input{\jobname}|
\end{tabular}
\end{center}

In an alternative approach,
child documents can be compiled by a specific command line
without additional code or specific definitions:
%
\begin{center}
|... -jobname "|\textit{target}|" "|[\textit{flags}]%
|\includeonly{|\textit{dest}|}\input{|\textit{main}|}"|
\end{center}
%

%%%%%%%%%%%%%%%%%%%%%%%%%%%%%%%%%%%%%%%%%%%%%%%%%%%%%%%%%%%%%%%%%%%%%%%%%%%%%%%%
%%%%%%%%%%%%%%%%%%%%%%%%%%%%%%%%%%%%%%%%%%%%%%%%%%%%%%%%%%%%%%%%%%%%%%%%%%%%%%%%
\section{Information}

%%%%%%%%%%%%%%%%%%%%%%%%%%%%%%%%%%%%%%%%%%%%%%%%%%%%%%%%%%%%%%%%%%%%%%%%%%%%%%%%
\subsection{Copyright}

Copyright \copyright{} 2017--2018 Niklas Beisert

This work may be distributed and/or modified under the
conditions of the \LaTeX{} Project Public License, either version 1.3
of this license or (at your option) any later version.
The latest version of this license is in
  \url{http://www.latex-project.org/lppl.txt}
and version 1.3 or later is part of all distributions of \LaTeX{}
version 2005/12/01 or later.

This work has the LPPL maintenance status `maintained'.

The Current Maintainer of this work is Niklas Beisert.

This work consists of the files |README.txt|, |childdoc.ins| and |childdoc.dtx|
as well as the derived files |childdoc.def|, |cdocsamp.tex|
with |cdocsch1.tex|, |cdocsch2.tex|, |cdocspt3.tex|, |cdocspt4.tex|,
|cdocsdrf.tex|, |cdocsfn1.tex|, |cdocsfn2.tex|
as well as |childdoc.pdf|.

%%%%%%%%%%%%%%%%%%%%%%%%%%%%%%%%%%%%%%%%%%%%%%%%%%%%%%%%%%%%%%%%%%%%%%%%%%%%%%%%
\subsection{Files and Installation}

The package consists of the files:
%
\begin{center}
\begin{tabular}{ll}
    |README.txt|   & readme file \\
    |childdoc.ins| & installation file \\
    |childdoc.dtx| & source file \\
    |childdoc.def| & definition file \\
    |cdocsamp.tex| & sample main file \\
    |cdocsch1.tex| & sample include file \\
    |cdocsch2.tex| & sample include file \\
    |cdocspt3.tex| & sample part file \\
    |cdocspt4.tex| & sample part file \\
    |cdocsdrf.tex| & sample redirection file \\
    |cdocsfn1.tex| & sample redirection file \\
    |cdocsfn2.tex| & sample redirection file \\
    |childdoc.pdf| & manual
\end{tabular}
\end{center}
%
The distribution consists of the files
|README.txt|, |childdoc.ins| and |childdoc.dtx|.
%
\begin{itemize}
\item
Run (pdf)\LaTeX{} on |childdoc.dtx|
to compile the manual |childdoc.pdf| (this file).
\item
Run \LaTeX{} on |childdoc.ins| to create the definitions file |childdoc.def|
and the sample |cdocsamp.tex| with include files
|cdocsch1.tex|, |cdocsch2.tex|, |cdocspt3.tex|, |cdocspt4.tex|,
|cdocsdrf.tex|, |cdocsfn1.tex|, |cdocsfn2.tex|.
Then copy the file |childdoc.def| to an appropriate directory of your \LaTeX{}
distribution, e.g.\ \textit{texmf-root}|/tex/latex/childdoc|.
\end{itemize}

%%%%%%%%%%%%%%%%%%%%%%%%%%%%%%%%%%%%%%%%%%%%%%%%%%%%%%%%%%%%%%%%%%%%%%%%%%%%%%%%
\subsection{Related CTAN Packages}

There are several other packages which offer a similar functionality:
%
\begin{itemize}
\item
The packages
\href{http://ctan.org/pkg/docmute}{\textsf{docmute}},
\href{http://ctan.org/pkg/includex}{\textsf{includex}} and
\href{http://ctan.org/pkg/standalone}{\textsf{standalone}}
provide commands to include only the document body of
a child file thus allowing both files to be compiled individually.
\item
The packages \href{http://ctan.org/pkg/subdocs}{\textsf{subdocs}}
and \href{http://ctan.org/pkg/subfiles}{\textsf{subfiles}}
provide structures in which the main and child documents can be
encapsulated and allowing them to be compiled individually.
The inclusion mechanism is different from the conventional |\include|.
\item
The package \href{http://ctan.org/pkg/combine}{\textsf{combine}}
is an elaborate solution to combine several documents into one.
\end{itemize}
%
See also the CTAN topic \href{http://ctan.org/topic/subdocs}{\textsf{subdocs}}
for further related packages.
The present package differs from the above solutions in that
a document structure constructed with the conventional |\include| mechanism
just needs two extra commands at the top of every file
such that all constituent files can be compiled individually.

%%%%%%%%%%%%%%%%%%%%%%%%%%%%%%%%%%%%%%%%%%%%%%%%%%%%%%%%%%%%%%%%%%%%%%%%%%%%%%%%
%\subsection{Feature Suggestions}
%
%The following is a list of features which may be useful for future
%versions of this package:
%%
%\begin{itemize}
%\item
%\ldots
%\end{itemize}

%%%%%%%%%%%%%%%%%%%%%%%%%%%%%%%%%%%%%%%%%%%%%%%%%%%%%%%%%%%%%%%%%%%%%%%%%%%%%%%%
\subsection{Revision History}

%%%%%%%%%%%%%%%%%%%%%%%%%%%%%%%%%%%%%%%%
\paragraph{v2.0:} 2018/12/30

\begin{itemize}
\item
immediate forward processing
\item
added |\childdocby| mechanism
\item
manual restructured
\end{itemize}

%%%%%%%%%%%%%%%%%%%%%%%%%%%%%%%%%%%%%%%%
\paragraph{v1.6:} 2018/01/17

\begin{itemize}
\item
application for development of include files
\item
corrections to manual
\end{itemize}

%%%%%%%%%%%%%%%%%%%%%%%%%%%%%%%%%%%%%%%%
\paragraph{v1.5:} 2017/05/21

\begin{itemize}
\item
more complete structuring introduced
\item
|\childdocof| introduced
\item
|\childdoc| renamed to |\childdocmain|
\item
|\childredirect| renamed to |\childdocforward| and |\childdocforwardprefix|
and functionality expanded
\end{itemize}

%%%%%%%%%%%%%%%%%%%%%%%%%%%%%%%%%%%%%%%%
\paragraph{v1.0:} 2017/04/27

\begin{itemize}
\item
manual and install package
\item
first version published on CTAN
\end{itemize}

%%%%%%%%%%%%%%%%%%%%%%%%%%%%%%%%%%%%%%%%
\paragraph{v0.6:} 2017/04/26

\begin{itemize}
\item
redirection mechanism added
\end{itemize}

%%%%%%%%%%%%%%%%%%%%%%%%%%%%%%%%%%%%%%%%
\paragraph{v0.5:} 2017/04/26

\begin{itemize}
\item
functionality in definition file
\end{itemize}


%%%%%%%%%%%%%%%%%%%%%%%%%%%%%%%%%%%%%%%%%%%%%%%%%%%%%%%%%%%%%%%%%%%%%%%%%%%%%%%%
%%%%%%%%%%%%%%%%%%%%%%%%%%%%%%%%%%%%%%%%%%%%%%%%%%%%%%%%%%%%%%%%%%%%%%%%%%%%%%%%
%%%%%%%%%%%%%%%%%%%%%%%%%%%%%%%%%%%%%%%%%%%%%%%%%%%%%%%%%%%%%%%%%%%%%%%%%%%%%%%%
\appendix

\settowidth\MacroIndent{\rmfamily\scriptsize 000\ }

 \DocInput{childdoc.dtx}

\end{document}
%</driver>
% \fi
%
% %%%%%%%%%%%%%%%%%%%%%%%%%%%%%%%%%%%%%%%%%%%%%%%%%%%%%%%%%%%%%%%%%%%%%%%%%%%%%%
% %%%%%%%%%%%%%%%%%%%%%%%%%%%%%%%%%%%%%%%%%%%%%%%%%%%%%%%%%%%%%%%%%%%%%%%%%%%%%%
% \section{Sample}
%\iffalse
%<*samplemain>
%\fi
%
% The following presents a sample document
% with two chapters, two parts, a title page,
% a compile flag as well as three forwarding files to set the flag.
% It consists of eight |.tex| files:
% \begin{center}
% \begin{tabular}{ll}
% |cdocsamp.tex|&main file\\
% |cdocsch1.tex|&include file for chapter 1\\
% |cdocsch2.tex|&include file for chapter 2\\
% |cdocspt3.tex|&include file for part 3\\
% |cdocspt4.tex|&include file for part 4\\
% |cdocsdrf.tex|&forwarding file for main file in draft mode\\
% |cdocsfi1.tex|&forwarding file for final version of chapter 1\\
% |cdocsfi2.tex|&forwarding file for final version of chapter 2\\
% \end{tabular}
% \end{center}
% Each of the eight files can be compiled directly by the \LaTeX{} compiler.
%
% %%%%%%%%%%%%%%%%%%%%%%%%%%%%%%%%%%%%%%
% \paragraph{Main File.}
%
% The main file is called |cdocsamp.tex|.
%
% Load the \textsf{childdoc} definitions and
% declare the filename for the main document:
%    \begin{macrocode}
\input{childdoc.def}
\childdocmain{}
%    \end{macrocode}

% Optional override for |\version| flag:
%    \begin{macrocode}
%%\ifchilddoc\else\providecommand{\version}{draft}\fi
%    \end{macrocode}

% Define the default values for the |\version| flag
% (|final| for the main file and |draft| for childs):
%    \begin{macrocode}
\ifchilddoc
\providecommand{\version}{draft}
\else
\providecommand{\version}{final}
\fi
%    \end{macrocode}

% Load the standard document class:
%    \begin{macrocode}
\documentclass[12pt]{article}
%    \end{macrocode}

% Start the document body:
%    \begin{macrocode}
\begin{document}
%    \end{macrocode}

% Declare a title page.
% Print title, part of document being processed and version flag:
%    \begin{macrocode}
\addtocounter{page}{-1}
\begin{center}
{\LARGE\bfseries{}childdoc example\par}
\vspace{1cm}
\ifchilddoc
\ifchilddocmanual part\else chapter\fi:
`\childdocname' of `\childdocjob'\par
\else
main document: `\childdocjob'\par
\fi
version: \version\par
\end{center}
\newpage
%    \end{macrocode}

% Manually include selected file,
% otherwise process as usual:
%    \begin{macrocode}
\ifchilddocmanual
\section*{part `\childdocname'}
\input{\childdocname}
\else
%    \end{macrocode}

% Include the two chapters:
%    \begin{macrocode}
\include{cdocsch1}
\include{cdocsch2}
%    \end{macrocode}

% Include the two parts unless only chapters should be displayed:
%    \begin{macrocode}
\ifchilddoc\else
\section{part three}
\input{cdocspt3}
\section{part four}
\input{cdocspt4}
\fi
%    \end{macrocode}

% Process as usual until here:
%    \begin{macrocode}
\fi
%    \end{macrocode}

% End of document body:
%    \begin{macrocode}
\end{document}
%    \end{macrocode}
%\iffalse
%</samplemain>
%\fi
%
% %%%%%%%%%%%%%%%%%%%%%%%%%%%%%%%%%%%%%%
% \paragraph{Chapter Include Files.}
%
% The include files are called |cdocsch1.tex| and |cdocsch2.tex|.
%
%\iffalse
%<*samplechap1|samplechap2>
%\fi

% Optional override for |\version| flag:
%    \begin{macrocode}
%%\providecommand{\version}{final}
%    \end{macrocode}

% Include the main document:
%    \begin{macrocode}
\input{childdoc.def}
\childdocof{cdocsamp}
%    \end{macrocode}

%\iffalse
%</samplechap1|samplechap2>
%\fi
%
%\iffalse
%<*samplechap1>
%\fi
% Some text for chapter 1:
%    \begin{macrocode}
\section{one}
some text in chapter one
%    \end{macrocode}

%\iffalse
%</samplechap1>
%\fi
% Some text for chapter 2:
%\iffalse
%<*samplechap2>
%\fi
%    \begin{macrocode}
\section{two}
more text in chapter two
%    \end{macrocode}

%\iffalse
%</samplechap2>
%\fi
%
% %%%%%%%%%%%%%%%%%%%%%%%%%%%%%%%%%%%%%%
% \paragraph{Part Include Files.}
%
% The include files are called |cdocspt3.tex| and |cdocspt4.tex|.
%
%\iffalse
%<*samplepart3|samplepart4>
%\fi

% Optional override for |\version| flag:
%    \begin{macrocode}
%%\providecommand{\version}{final}
%    \end{macrocode}

% Include the main document:
%    \begin{macrocode}
\input{childdoc.def}
\childdocby{cdocsamp}
%    \end{macrocode}

%\iffalse
%</samplepart3|samplepart4>
%\fi
%
%\iffalse
%<*samplepart3>
%\fi
% Some text for part 3:
%    \begin{macrocode}
some text in part three
%    \end{macrocode}

%\iffalse
%</samplepart3>
%\fi
% Some text for part 4:
%\iffalse
%<*samplepart4>
%\fi
%    \begin{macrocode}
more text in part four
%    \end{macrocode}

%\iffalse
%</samplepart4>
%\fi
%
% %%%%%%%%%%%%%%%%%%%%%%%%%%%%%%%%%%%%%%
% \paragraph{Forwarding for a Complete Draft.}
%
% The following forwarding file |cdocsdrf.tex|
% compiles the main document in draft mode:
%\iffalse
%<*sampledraft>
%\fi
%    \begin{macrocode}
\def\version{draft}
\input{childdoc.def}
\childdocforward{cdocsamp}
%    \end{macrocode}

%\iffalse
%</sampledraft>
%\fi
%
% %%%%%%%%%%%%%%%%%%%%%%%%%%%%%%%%%%%%%%
% \paragraph{Forwarding for Final Version of the Chapters.}
%
% The following forwarding files |cdocsfn1.tex| and |cdocsfn2.tex|
% (with identical content)
% compile the final versions of the child documents
% |cdocsch1.tex| and |cdocsch2.tex|, respectively:
%\iffalse
%<*samplefinal>
%\fi
%    \begin{macrocode}
\def\version{final}
\input{childdoc.def}
\childdocforwardprefix[cdocsamp]{cdocsfn}{cdocsch}
%    \end{macrocode}

%\iffalse
%</samplefinal>
%\fi
%
% %%%%%%%%%%%%%%%%%%%%%%%%%%%%%%%%%%%%%%
% \paragraph{Command Line Processing.}
%
% The following three command lines generate the output files
% |cdocscld|, |cdocscl1| and |cdocscl2|
% which should be identical to
% |cdocsdrf|, |cdocsch1| and |cdocsfn2|, respectively:
% \begin{center}
% \begin{tabular}{l}
% |latex -jobname cdocscld \|\\
% |  "\def\version{draft}\input{childdoc.def}\childdocforward{cdocsamp}"|\\
% |latex -jobname cdocscl1 \|\\
% |  "\input{childdoc.def}\childdocforward[cdocsamp]{cdocsch1}"|\\
% |latex -jobname cdocscl2 \|\\
% |  "\def\version{final}\input{childdoc.def}\childdocforward{cdocsch2}"|
% \end{tabular}
% \end{center}
% Note that the trailing backslash on each first line
% merely continues the input to the second line
% (for convenient cut ant paste).
% Furthermore, the command |latex| can be replaced by any
% of its alternative versions such as |pdflatex|.
%
% %%%%%%%%%%%%%%%%%%%%%%%%%%%%%%%%%%%%%%%%%%%%%%%%%%%%%%%%%%%%%%%%%%%%%%%%%%%%%%
% %%%%%%%%%%%%%%%%%%%%%%%%%%%%%%%%%%%%%%%%%%%%%%%%%%%%%%%%%%%%%%%%%%%%%%%%%%%%%%
% \section{Implementation}
%\iffalse
%<*package>
%\fi
%
% This section describes the definitions file |childdoc.def|.

% The definitions cannot be loaded using |\usepackage| or |\RequirePackage|
% which has a mechanism to prevent loading a style file more than once.
% When loading the definitions by means of |\input|
% multiple instances have to be prevented manually:
%\iffalse
%This code needs to be before the `\ProvidesFile' directive
%which is defined at the beginning of this file.
%Therefore it is also placed there and commented out here.
%</package>
%<*discard>
%\fi
%    \begin{macrocode}
\ifdefined\childdocmain\endinput\fi
%    \end{macrocode}
%\iffalse
%</discard>
%<*package>
%\fi
%
% \macro{\ifchilddoc}
% \macro{\ifchilddocmanual}
% The conditional |\ifchilddoc| tells whether a
% child (true) or main (false) document is being compiled.
% The conditional |\ifchilddocmanual| tells whether
% the |\includeonly| mechanism is used (false) or
% the selection of child files must be performed manually (true).
% The definitions initialise to false:
%    \begin{macrocode}
\newif\ifchilddoc
\newif\ifchilddocmanual
%    \end{macrocode}

% \macro{\childdocname}
% \macro{\childdocjob}
% The macro |\childdocname| stores the name of the main document
% to be compiled. The macro |\childdocjob| stores the name of
% the document on which the \LaTeX{} compiler was originally invoked.
% The content of |\jobname| cannot be compared
% to filenames specified in the source due to different catcodes.
% The following code rescans |\jobname|, stores the result
% in |\childdocname| and saves a copy in |\childdocjob|:
%    \begin{macrocode}
\edef\childdocname{\scantokens\expandafter{\jobname\noexpand}}
\let\childdocjob\childdocname
%    \end{macrocode}

% \macro{\childdocdisable}
% The macro |\childdocdisable| prevents the main file
% from being processed more than once.
% At this stage, the main document command |\childdocmain|
% is assumed to be called once again where it should do nothing.
% Any subsequent call to it should prevent
% a secondary processing of the main document
% It overwrites the forwarding commands
% |\childdocof| and |\childdocforward|
% with empty macros to prevent further inclusions of the main document:
%    \begin{macrocode}
\newcommand{\childdocdisable}
{
  \renewcommand{\childdocmain}[1]{\renewcommand{\childdocmain}[1]{\endinput}}
  \renewcommand{\childdocof}[1]{}
  \renewcommand{\childdocby}[2][]{}
  \renewcommand{\childdocforward}[2][]{}
  \renewcommand{\childdocdisable}{}
}
%    \end{macrocode}

% \macro{\childdocmain}
% The macro |\childdocmain| is to be called at the top of the main file
% with nothing or the main filename (without extension) as argument.
% First, it breaks loops.
% If the argument is not empty and does not match |\childdocname|
% (which is set by the first inclusion of |childdoc.def|),
% |\ifchilddoc| is set to true, |\includeonly| is applied to the child file
% and |\jobname| is set to the main file
% (for proper handling of |.aux| files):
%    \begin{macrocode}
\newcommand{\childdocmain}[1]
{
  \childdocdisable\childdocmain{}
  \if?#1?\else
    \begingroup
      \def\childdoctmp{#1}
      \ifx\childdoctmp\childdocname
        \def\childdoctmp{}
      \else
        \def\childdoctmp
        {
          \childdoctrue
          \includeonly{\childdocname}
          \def\childdocjob{#1}
          \def\jobname{#1}
        }
      \fi
      \expandafter
    \endgroup
    \childdoctmp
  \fi
}
%    \end{macrocode}

% \macro{\childdocof}
% The command |\childdocof| redirects
% compilation to the main file |#1|.
%    \begin{macrocode}
\newcommand{\childdocof}[1]
{
  \childdocdisable
  \childdoctrue
  \includeonly{\childdocname}
  \def\jobname{#1}
  \def\childdocjob{#1}
  \input{#1}
}
%    \end{macrocode}

% \macro{\childdocby}
% The command |\childdocby| ....
%    \begin{macrocode}
\newcommand{\childdocby}[2][]
{
  \childdocdisable
  \childdoctrue
  \childdocmanualtrue
  \if?#1?\else
    \def\jobname{#2}
  \fi
  \def\childdocjob{#2}
  \input{#2}
  \endinput
}
%    \end{macrocode}

% \macro{\childdocforward}
% The command |\childdocforward| redirects
% compilation to the main file or
% (if the optional argument is given) a child file.
% Parameters are set as if the main file
% or a child file starting with |\childdocof| was compiled.
% Then compilation is handed over to the main file:
%    \begin{macrocode}
\newcommand{\childdocforward}[2][]
{
  \begingroup
    \if?#1?
      \def\childdoctmp
      {
        \def\childdocname{#2}
        \def\childdocjob{#2}
        \def\jobname{#2}
        \input{#2}
        \endinput
      }
    \else
      \def\childdoctmp
      {
        \childdocdisable
        \def\childdocname{#2}
        \childdoctrue
        \includeonly{#2}
        \def\childdocjob{#1}
        \def\jobname{#1}
        \input{#1}
        \endinput
      }
    \fi
    \expandafter
  \endgroup
  \childdoctmp
}
%    \end{macrocode}

% \macro{\childdocforwardprefix}
% The command |\childdocforwardprefix| redirects
% compilation to the main or a child file by means of a pattern.
% The prefix |#1| in the current filename is replaced by |#2|
% and the suffix of the current filename is kept
% (it is assumed that the filename does not contain the substring `|~~~|'
% which is used as a delimiter).
% Compilation is handed over to the new file by |\childdocforward|:
%    \begin{macrocode}
\newcommand{\childdocforwardprefix}[3][]
{
  \begingroup
    \def\childdocextract #2##1~~~{\def\childdoctmp{\childdocforward[#1]{#3##1}}}
    \expandafter\childdocextract\childdocname~~~
    \expandafter
  \endgroup
  \childdoctmp
}
%    \end{macrocode}

% \macro{\childdoc}
% The deprecated macro |\childdoc| is a legacy version of |\childdocmain|:
%    \begin{macrocode}
\newcommand{\childdoc}{\childdocmain}
%    \end{macrocode}

% \macro{\childdocredirect}
% The deprecated macro |\childdocredirect| is a legacy version
% of |\childdocforward| and |\childdocforwardprefix|:
%    \begin{macrocode}
\newcommand{\childdocredirect}[2][]
{
  \begingroup
    \if?#1?
      \def\childdoctmp{\childdocforward{#2}}
    \else
      \def\childdoctmp{\childdocforwardprefix{#1}{#2}}
    \fi
    \expandafter
  \endgroup
  \childdoctmp
}
%    \end{macrocode}

%\iffalse
%</package>
%\fi
%
\endinput

\childdocmain{}
%    \end{macrocode}

% Optional override for |\version| flag:
%    \begin{macrocode}
%%\ifchilddoc\else\providecommand{\version}{draft}\fi
%    \end{macrocode}

% Define the default values for the |\version| flag
% (|final| for the main file and |draft| for childs):
%    \begin{macrocode}
\ifchilddoc
\providecommand{\version}{draft}
\else
\providecommand{\version}{final}
\fi
%    \end{macrocode}

% Load the standard document class:
%    \begin{macrocode}
\documentclass[12pt]{article}
%    \end{macrocode}

% Start the document body:
%    \begin{macrocode}
\begin{document}
%    \end{macrocode}

% Declare a title page.
% Print title, part of document being processed and version flag:
%    \begin{macrocode}
\addtocounter{page}{-1}
\begin{center}
{\LARGE\bfseries{}childdoc example\par}
\vspace{1cm}
\ifchilddoc
\ifchilddocmanual part\else chapter\fi:
`\childdocname' of `\childdocjob'\par
\else
main document: `\childdocjob'\par
\fi
version: \version\par
\end{center}
\newpage
%    \end{macrocode}

% Manually include selected file,
% otherwise process as usual:
%    \begin{macrocode}
\ifchilddocmanual
\section*{part `\childdocname'}
\input{\childdocname}
\else
%    \end{macrocode}

% Include the two chapters:
%    \begin{macrocode}
\include{cdocsch1}
\include{cdocsch2}
%    \end{macrocode}

% Include the two parts unless only chapters should be displayed:
%    \begin{macrocode}
\ifchilddoc\else
\section{part three}
\input{cdocspt3}
\section{part four}
\input{cdocspt4}
\fi
%    \end{macrocode}

% Process as usual until here:
%    \begin{macrocode}
\fi
%    \end{macrocode}

% End of document body:
%    \begin{macrocode}
\end{document}
%    \end{macrocode}
%\iffalse
%</samplemain>
%\fi
%
% %%%%%%%%%%%%%%%%%%%%%%%%%%%%%%%%%%%%%%
% \paragraph{Chapter Include Files.}
%
% The include files are called |cdocsch1.tex| and |cdocsch2.tex|.
%
%\iffalse
%<*samplechap1|samplechap2>
%\fi

% Optional override for |\version| flag:
%    \begin{macrocode}
%%\providecommand{\version}{final}
%    \end{macrocode}

% Include the main document:
%    \begin{macrocode}
% \iffalse
%
% childdoc.dtx Copyright (C) 2017-2018 Niklas Beisert
%
% This work may be distributed and/or modified under the
% conditions of the LaTeX Project Public License, either version 1.3
% of this license or (at your option) any later version.
% The latest version of this license is in
%   http://www.latex-project.org/lppl.txt
% and version 1.3 or later is part of all distributions of LaTeX
% version 2005/12/01 or later.
%
% This work has the LPPL maintenance status `maintained'.
%
% The Current Maintainer of this work is Niklas Beisert.
%
% This work consists of the files childdoc.dtx and childdoc.ins
% and the derived files childdoc.def and cdocsamp.tex with
% cdocsch1.tex, cdocsch2.tex, cdocsdrf.tex, cdocsfn1.tex, cdocsfn2.tex.
%
%<package>\ifdefined\childdocmain\endinput\fi
%<package>\ProvidesFile{childdoc.def}[2018/12/30 v2.0 child document driver]
%<samplemain>\ProvidesFile{cdocsamp.tex}[2018/12/30 v2.0 sample for childdoc]
%<*driver>
%\ProvidesFile{childdoc.drv}[2018/12/30 v2.0 childdoc reference manual file]
\PassOptionsToClass{10pt,a4paper}{article}
\documentclass{ltxdoc}

\usepackage[margin=35mm]{geometry}
\usepackage{hyperref}
\usepackage{hyperxmp}
\usepackage[usenames]{color}

\hypersetup{colorlinks=true}
\hypersetup{pdfstartview=FitH}
\hypersetup{pdfpagemode=UseNone}
\hypersetup{pdfsource={}}
\hypersetup{pdflang={en-UK}}
\hypersetup{pdfcopyright={Copyright 2017-2018 Niklas Beisert.
  This work may be distributed and/or modified under the
  conditions of the LaTeX Project Public License, either version 1.3
  of this license or (at your option) any later version.}}
\hypersetup{pdflicenseurl={http://www.latex-project.org/lppl.txt}}
\hypersetup{pdfcontactaddress={ETH Zurich, ITP, HIT K,
  Wolfgang-Pauli-Strasse 27}}
\hypersetup{pdfcontactpostcode={8093}}
\hypersetup{pdfcontactcity={Zurich}}
\hypersetup{pdfcontactcountry={Switzerland}}
\hypersetup{pdfcontactemail={nbeisert@itp.phys.ethz.ch}}
\hypersetup{pdfcontacturl={http://people.phys.ethz.ch/\xmptilde nbeisert/}}

\newcommand{\secref}[1]{\hyperref[#1]{section \ref*{#1}}}

\parskip1ex
\parindent0pt
\let\olditemize\itemize
\def\itemize{\olditemize\parskip0pt}

\begin{document}

\title{The \textsf{childdoc} Package}
\hypersetup{pdftitle={The childdoc Package}}
\author{Niklas Beisert\\[2ex]
  Institut f\"ur Theoretische Physik\\
  Eidgen\"ossische Technische Hochschule Z\"urich\\
  Wolfgang-Pauli-Strasse 27, 8093 Z\"urich, Switzerland\\[1ex]
  \href{mailto:nbeisert@itp.phys.ethz.ch}
  {\texttt{nbeisert@itp.phys.ethz.ch}}}
\hypersetup{pdfauthor={Niklas Beisert}}
\hypersetup{pdfsubject={Manual for the LaTeX2e Package childdoc}}
\date{30 December 2018, \textsf{v2.0}}
\maketitle

\begin{abstract}\noindent
\textsf{childdoc} is a \LaTeXe{} package
that enables the direct compilation
of document sections included by |\include|
to individual files.
\end{abstract}

\begingroup
\parskip0ex
\tableofcontents
\endgroup

%%%%%%%%%%%%%%%%%%%%%%%%%%%%%%%%%%%%%%%%%%%%%%%%%%%%%%%%%%%%%%%%%%%%%%%%%%%%%%%%
%%%%%%%%%%%%%%%%%%%%%%%%%%%%%%%%%%%%%%%%%%%%%%%%%%%%%%%%%%%%%%%%%%%%%%%%%%%%%%%%
\section{Introduction}

\LaTeX{} provides a mechanism to structure a large document (such as a book)
into a main file and several child files (containing the chapters)
using the |\include| command.
This mechanism is beneficial for documents
which span hundreds of pages in order to
make the source file(s) more manageable.
Moreover, compilation can be restricted to
selected child files by means of the |\includeonly| command.
The latter feature can be used to reduce the compilation time while editing
(this was significantly more useful in the earlier days of \LaTeX{})
or to generate a smaller document which is easier to navigate.
Another application of |\includeonly| is to generate
documents consisting of selected parts of the complete document.

However, there are a few drawbacks of the plain |\include| mechanism:
\begin{itemize}
\item
The child files cannot be compiled on their own,
they can only be compiled via the main file.
A naive editing environment
(such as a text editor with an option
to have the current file processed by \LaTeX)
may require one to switch to the main file before compiling;
attempting to compile the child file produces errors.
\item
The main file must be modified (each time)
to adjust the |\includeonly| command
to the present needs. This easily leaves the main file in a messy state.
\item
The generated document will always carry the filename
of the main document. This is inconvenient if
several child files are to be compiled and
to be kept for distribution.
\end{itemize}

The present package provides a simple interface
to make child files individually compilable by \LaTeX{}.
Compiling a child file then has the same effect as compiling
the main file with an |\includeonly| command
to select the appropriate child.
Moreover the generated document will carry the name of the child
rather than the main file.
This resolves all three above issues.

This feature is meant to make the editing of books,
thesis documents and lecture notes somewhat more convenient.
However, the package can also be used efficiently for
composing a series of documents (such as exercise sheets)
which are typically distributed individually.
It then assists the author in generating the individual documents
(potentially in different versions)
as well as a document containing the collected series.
Another application is in developing style files
or other kinds of included material
where compilation of the style file could redirect
to a sample or test file.

%%%%%%%%%%%%%%%%%%%%%%%%%%%%%%%%%%%%%%%%%%%%%%%%%%%%%%%%%%%%%%%%%%%%%%%%%%%%%%%%
%%%%%%%%%%%%%%%%%%%%%%%%%%%%%%%%%%%%%%%%%%%%%%%%%%%%%%%%%%%%%%%%%%%%%%%%%%%%%%%%
\section{Usage}

First of all, the package \textsf{childdoc} is \emph{not} a standard
\LaTeXe{} |.sty| style file! Therefore it needs to be invoked in
a non-standard way.

%%%%%%%%%%%%%%%%%%%%%%%%%%%%%%%%%%%%%%%%%%%%%%%%%%%%%%%%%%%%%%%%%%%%%%%%%%%%%%%%
\subsection{Included Files}
\label{sec:include}

%%%%%%%%%%%%%%%%%%%%%%%%%%%%%%%%%%%%%%%%
\DescribeMacro{\childdocmain}
To use the package, add the commands
\begin{center}
\begin{tabular}{l}
|\input{childdoc.def}|\\
|\childdocmain{}|\\
\end{tabular}
\end{center}
at the very top of the main \LaTeX{} file,
in particular \emph{before} the |\documentclass| statement!
The argument of |\childdocmain| should be left empty
(but it must be present).

%%%%%%%%%%%%%%%%%%%%%%%%%%%%%%%%%%%%%%%%
\DescribeMacro{\childdocof}
Furthermore, add the commands
\begin{center}
\begin{tabular}{l}
|\input{childdoc.def}|\\
|\childdocof{|\textit{main}|}|\\
\end{tabular}
\end{center}
at the top of every child file \textit{child}
which is included by |\include{|\textit{child}|}|
from within the main file
(or at least for those files to be compiled individually).
The argument \textit{main} must be the filename of the main file.

There are a couple of
considerations in setting up the main and child documents:

%%%%%%%%%%%%%%%%%%%%%%%%%%%%%%%%%%%%%%%%
\paragraph{Restrictions.}

Please note the following restrictions:
\begin{itemize}
\item
|\childdocmain| must be called with one argument \textit{main}
to ensure compatibility with earlier version of the package.
It must either be empty (|\childdocmain{}|)
or precisely match the filename of the main file in which it is specified.
See \secref{sec:detection} for further information.
\item
The filename \textit{main} must be specified without the |.tex| extension.
\item
The filename \textit{main} is case sensitive
(even in case-insensitive file systems)
due to internal string comparison.
\item
The argument \textit{main} should be fully expanded, it cannot be a macro.
\item
Subdirectories and special characters should be avoided in filenames.
\item
The command |\childdocmain{|\textit{main}|}| must be followed by a whitespace.
It should not be followed immediately by another command
or by a comment mark `|%|'.
This is because the \TeX{} parser reads the token immediately following
the argument of |\childdocmain| and puts it
at the beginning of every child section;
however, a white\-space is ignored.
\end{itemize}

%%%%%%%%%%%%%%%%%%%%%%%%%%%%%%%%%%%%%%%%
\paragraph{Content of Main File.}

It is advisable to place all content in the child files included by |\include|.
Any output contained in the main file will appear in all child documents
unless suppressed manually;
it cannot be suppressed automatically by the |\includeonly| directive
and thus should normally be avoided.
A method to include some content in the main file
by means of conditional processing is described in \secref{sec:conditional}.

%%%%%%%%%%%%%%%%%%%%%%%%%%%%%%%%%%%%%%%%
\paragraph{Page Numbering.}

When only a part of the document is compiled,
the appropriate numbering of pages
(as well as other status parameters)
is determined from the |.aux| files.
The latter contain information from previous passes.
However this information needs to propagate through
all intermediate child documents.
Therefore the page numbering in child documents may well
be inconsistent until the complete document is compiled at least once.

A useful (if unconventional) way to always ensure a consistent
page numbering is to restart the numbering in each child document
and denote the pages by `\textit{child}|.|\textit{page}'
where \textit{child} represents the chapter/section number of the child file.
This can be achieved by the command
|\numberwithin{page}{|\textit{child}|}|
of the \textsf{amsmath} package
where \textit{child} can be |chapter| or |section|
depending on the chosen structuring.
Alternatively, one can modify the macro |\thepage| appropriately
and reset the counter |page| at the start of each child file.

%%%%%%%%%%%%%%%%%%%%%%%%%%%%%%%%%%%%%%%%%%%%%%%%%%%%%%%%%%%%%%%%%%%%%%%%%%%%%%%%
\subsection{Conditional Processing}
\label{sec:conditional}

The package provides a mechanism to compile different versions
of a document. To customise the versions further some conditional processing
can come in handy to distinguish which version is being compiled.
The package provides two macros to describe the compilation context:

%%%%%%%%%%%%%%%%%%%%%%%%%%%%%%%%%%%%%%%%
\DescribeMacro{\ifchilddoc}
The conditional |\ifchilddoc| distinguishes between the compilation of
child documents and the main document:
%
\begin{center}
|\ifchilddoc |\textit{child-code}| |[|\||else |\textit{main-code}]| \||fi|
\end{center}

%%%%%%%%%%%%%%%%%%%%%%%%%%%%%%%%%%%%%%%%
\DescribeMacro{\childdocname}
\DescribeMacro{\childdocjob}
The macro |\childdocname| contains the filename (without extension)
of the main or child file being processed.
Note that |\childdocjob| will always contain the name of the main file.

%%%%%%%%%%%%%%%%%%%%%%%%%%%%%%%%%%%%%%%%
\paragraph{Title Page.}

Conditional processing can be used to include a title or banner page
in the main document when proper precautions are taken.
Importantly, the code in the main file should ensure that the page counter
(as well as other status parameters which are stored in the |.aux| files)
takes the same value after the conditional processing.
Otherwise the page numbers may take divergent values
depending on which part is compiled.

For example, a title page could be declared by:
%
\begin{center}
\begin{tabular}{l}
|\ifchilddoc\||else|\\
|\addtocounter{page}{-1}|\\
\textit{code for title page}\\
|\newpage|\\
|\||fi|
\end{tabular}
\end{center}
%
A banner page for the child documents can be generated by:
%
\begin{center}
\begin{tabular}{l}
|\ifchilddoc|\\
|\addtocounter{page}{-1}|\\
\textit{code for banner page}\\
|\newpage|\\
|\||fi|
\end{tabular}
\end{center}
%
Here one could write a message such as:
\begin{center}
|This is the part \childdocname{} of \childdocjob{}.|
\end{center}

%%%%%%%%%%%%%%%%%%%%%%%%%%%%%%%%%%%%%%%%%%%%%%%%%%%%%%%%%%%%%%%%%%%%%%%%%%%%%%%%
\subsection{Flags}
\label{sec:flags}

The package makes it easy to generate different versions
of the main or child documents.
To this end compilation flags can be defined
and assigned different default values.
They will be particularly useful in conjunction
with the forwarding mechanism described in \secref{sec:forward}.

For example, it may be useful to have a flag |\version|
which can be set to |draft| or |final|.
The document source will contain some conditional code
depending on the value of |\version|.
Suppose further, the flag should default to |final| for the main file
and to |draft| for child files
which is a natural assignment for editing the document.
This is achieved by placing the following code
in the preamble of the main document
(below the |\childdocmain| directive):
%
\begin{center}
\begin{tabular}{l}
|\ifchilddoc|\\
|\providecommand{\version}{draft}|\\
|\||else|\\
|\providecommand{\version}{final}|\\
|\||fi|
\end{tabular}
\end{center}
%
The definition by |\providecommand| makes sure
that previous definitions are not overwritten.
Further statements |\providecommand{\version}{...}|
can thus be added before the above code to override it.

For the main file, one might add a line
(between |\childdocmain| and the above block)
%
\begin{center}
|%\ifchilddoc\||else\providecommand{\version}{draft}\||fi|
\end{center}
%
which can be uncommented to produce a draft version.
Likewise one can add a line to the very top of a child file
(above the |\childdocof{|\textit{main}|}| directive)
%
\begin{center}
|%\providecommand{\version}{final}|
\end{center}
%
which can be uncommented to produce the final version of this child document.

%%%%%%%%%%%%%%%%%%%%%%%%%%%%%%%%%%%%%%%%%%%%%%%%%%%%%%%%%%%%%%%%%%%%%%%%%%%%%%%%
\subsection{Forwarding}
\label{sec:forward}

Different versions of the main or child documents
using compilation flags as described in \secref{sec:flags}
can be (permanently) stored in different files
for convenient compilation, viewing and distribution.
To this end, the package defines a command
to pass on compilation to a different file:

%%%%%%%%%%%%%%%%%%%%%%%%%%%%%%%%%%%%%%%%
\DescribeMacro{\childdocforward}
The command |\childdocforward| redirects processing to
another source file:
%
\begin{center}
\begin{tabular}{l}
|\input{childdoc.def}|\\
|\childdocforward[|\textit{main}|]{|\textit{dest}|}|\\
\end{tabular}
\end{center}
%
The argument \textit{dest} is the destination file
(without extension).
It should be the main file or one of the child files.
Note that further \textsf{childdoc} directives
such as |\childdocof| and |\childdocforward|
in the indicated file will be processed in this form.
The optional argument \textit{main}
passes on directly to the main file \textit{main}
while pretending to compile the child \textit{dest}.
This form behaves as if \textit{dest}
issues |\childdocof{|\textit{main}|}| right away,
and no further \textsf{childdoc} directives will be processed.

%%%%%%%%%%%%%%%%%%%%%%%%%%%%%%%%%%%%%%%%
\DescribeMacro{\...prefix}
In the alternative form |\childdocforwardprefix|,
%
\begin{center}
\begin{tabular}{l}
|\input{childdoc.def}|\\
|\childdocforwardprefix[|\textit{main}|]{|\textit{prefix}|}{|\textit{dest}|}|
\end{tabular}
\end{center}
%
the destination file is determined by a pattern
depending on the current file:
To make this work, the current file must be called
`{\textit{prefix}\hspace{0.2em}\textit{suffix}}'
with \textit{prefix} matching precisely the argument.
Processing is then passed on to the file
`{\textit{dest}\hspace{0.2em}\textit{suffix}}'.
Surely, the same effect is achieved by
directly specifying the
argument `{\textit{dest}\hspace{0.2em}\textit{suffix}}'
in the first form.
However, that requires to set up a different file
for each child. With the alternative form of the command
all these files can have exactly the same content
which simplifies setting them up and maintaining them.

For example, the following file |draft.tex|
with a compilation flag |\version| as described in \secref{sec:flags}
compiles the main document as a draft:
%
\begin{center}
\begin{tabular}{l}
|\def\version{draft}|\\
|\input{childdoc.def}|\\
|\childdocforward{|\textit{main}|}|
\end{tabular}
\end{center}
%
Likewise, the following files |final|\textit{nn}|.tex|
compile the final version of the child document
|child|\textit{nn}|.tex|:
%
\begin{center}
\begin{tabular}{l}
|\def\version{final}|\\
|\input{childdoc.def}|\\
|\childdocforwardprefix{final}{child}|
\end{tabular}
\end{center}
%

Note that when several versions of a main file and/or of each child file
are to be generated, it may be convenient to set up a |Makefile| or
shell script to automatise the process.

%%%%%%%%%%%%%%%%%%%%%%%%%%%%%%%%%%%%%%%%%%%%%%%%%%%%%%%%%%%%%%%%%%%%%%%%%%%%%%%%
\subsection{Command Line Processing}
\label{sec:commandline}

The effect of redirection files can also be achieved by invoking
the \LaTeX{} compiler with a more elaborate command line.
Most conveniently this should be done as part
of a shell script or a |Makefile|.

When using \textsf{childdoc} in the main file, the following
command lines effectively perform a redirection
(note that depending on the shell being used,
backslashes may have to be doubled: `|\|' $\to$ `|\\|'):
%
\begin{center}
|... -jobname "|\textit{target}|" |\\|"|[\textit{flags}]%
|\input{childdoc.def}\childdocforward[|\textit{main}|]{|\textit{dest}|}"|
\end{center}
%
Here \textit{target} is the name of the output file,
\textit{main} is the name of the main file
and \textit{dest} is the name of the main or child file to be processed
(all filenames without extensions).
The optional argument \textit{main} can be omitted
if \textit{main} matches \textit{dest}.
Optionally, compilation \textit{flags} can be defined via |\def| commands.
This command line makes the \TeX{} engine believe
it is compiling the file \textit{target}
whose content is specified as the latter parameter.
The provided code then forwards the processing to
\textit{main} or \textit{dest} as described in \secref{sec:forward}.

%%%%%%%%%%%%%%%%%%%%%%%%%%%%%%%%%%%%%%%%%%%%%%%%%%%%%%%%%%%%%%%%%%%%%%%%%%%%%%%%
\subsection{Include by Input}
\label{sec:input}

Including child documents by |\include| has some restrictions by design.
Most notably, the content of a child document always occupies
its own set of pages; pages cannot be shared between child documents.
Usually, this behaviour makes perfect sense
because each child document contain an essential part of the document.
However, in some situations it may be desirable to compose
a document from a collection of parts
without having mandatory page breaks between then.
For this case, the package
provides a mechanism to include parts
by |\input| which can also be processed individually.
However, by construction this mechanism
requires manual handling of the content to be output.

%%%%%%%%%%%%%%%%%%%%%%%%%%%%%%%%%%%%%%%%
\DescribeMacro{\ifchilddocmanual}
The main file should be prepared as usual, see \secref{sec:include}.
However, the document body must make a distinction
between processing of an individual part and of the main document, e.g.:
%
\begin{center}
\begin{tabular}{l}
|\ifchilddocmanual|\\
|\input{\childdocname}|\\
|\||else|\\
\textit{document body with }|\input{|\textit{part}|}|\\
|\||fi|
\end{tabular}
\end{center}
%
The conditional |\ifchilddocmanual| is true whenever
a part to be included by |\input| is being compiled,
and the name of the part is stored in |\childdocname|.

%%%%%%%%%%%%%%%%%%%%%%%%%%%%%%%%%%%%%%%%
\DescribeMacro{\childdocby}
Each part to be included by |\input| should start with:
%
\begin{center}
\begin{tabular}{l}
|\input{childdoc.def}|\\
|\childdocby{|\textit{main}|}|\\
\end{tabular}
\end{center}
%
The directive |\childdocby| is similar to |\childdocof|
described in \secref{sec:include},
but the subsequent selection of content must be done manually.
To that end, both |\ifchilddoc| and |\ifchilddocmanual|
will be true upon processing of a part,
and the name of the part is stored in |\childdocname|.
Note that |\jobname| will be set to the filename of the current part
so that each part receives an individual |.aux| file
that does not interfere with the |.aux| file(s) of the main document.
This behaviour can be altered by the alternative form
|\childdocby[*]{|\textit{main}|}| (with a non-empty optional argument)
which uses the |.aux| file of the main document
by setting |\jobname| to \textit{main}.

%%%%%%%%%%%%%%%%%%%%%%%%%%%%%%%%%%%%%%%%%%%%%%%%%%%%%%%%%%%%%%%%%%%%%%%%%%%%%%%%
\subsection{Driver Development}
\label{sec:driver}

The \textsf{childdoc} mechanism can also be use for the development
of definition files such as \LaTeX{} styles or classes.
This case differs from the above setup with multiple parts
included by |\include| in that no |\includeonly| should be invoked.
This can be achieved by starting the include file
(before |\ProvidesPackage|) with:
%
\begin{center}
\begin{tabular}{l}
|\input{childdoc.def}|\\
|\childdocforward{|\textit{main}|}|\\
\end{tabular}
\end{center}
%
or alternatively with:
%
\begin{center}
\begin{tabular}{l}
|\input{childdoc.def}|\\
|\childdocby{|\textit{main}|}|\\
\end{tabular}
\end{center}
%
Both forms have slightly different effects as described above.
The main file is prepared as usual, see \secref{sec:include}.

%%%%%%%%%%%%%%%%%%%%%%%%%%%%%%%%%%%%%%%%%%%%%%%%%%%%%%%%%%%%%%%%%%%%%%%%%%%%%%%%
\subsection{Legacy Detection}
\label{sec:detection}

The directive |\childdocmain| in the main file can detect
whether the complete document or merely a child is to be compiled
even without using the directive |\childdocof|.
This method is deprecated because it is less robust
and there is no compelling reason to use it;
it is merely provided for backward compatibility
and it may be removed in future versions.

If the detection mechanism is to be used,
it is mandatory to correctly specify
the filename of the main file as the argument of |\childdocmain|:
%
\begin{center}
\begin{tabular}{l}
|\input{childdoc.def}|\\
|\childdocmain{|\textit{main}|}|\\
\end{tabular}
\end{center}
%
If |\jobname| does not match the argument \textit{main} of |\childdocmain|,
it is assumed that |\jobname| points to the child file to be compiled.
When using |\childdocmain| with the main file specified as argument,
it suffices to start a child file
with just |\input{|\textit{main}|}|
without loading of the package and using |\childdocof|.
If instead all processing is done
with the appropriate \textsf{childdoc} directives,
the argument of \textit{main} of |\childdocmain| can be empty.

An alternative version of the command line processing described
in \secref{sec:commandline} using the detection mechanism reads:
%
\begin{center}
|... -jobname "|\textit{target}|" "|[\textit{flags}]%
[|\def\jobname{|\textit{dest}|}|]|\input{|\textit{main}|}"|
\end{center}

%%%%%%%%%%%%%%%%%%%%%%%%%%%%%%%%%%%%%%%%%%%%%%%%%%%%%%%%%%%%%%%%%%%%%%%%%%%%%%%%
\subsection{Manual Code}
\label{sec:manual}

In case one cannot be certain whether the definitions file |childdoc.def|
is installed on the target \TeX{} distribution
and one prefers not to ship it,
it is conceivable to paste a few relevant commands into the sources.

To that end, drop all statements |\input{childdoc.def}|
and perform the replacements as outlined below.
Instead of |\childdocmain{|\textit{main}|}| add the following code
to the top of the main file:
%
\begin{center}
\begin{tabular}{l}
|\||ifdefined\childdocname\endinput\||fi\newif\ifchilddoc|\\
|\edef\childdocname{\scantokens\expandafter{\jobname\noexpand}}|\\
|\def\childdocmain{|\textit{main}|}\||ifx\childdocmain\childdocname\||else|\\
|\childdoctrue\includeonly{\childdocname}\let\jobname\childdocmain\||fi|\\
\end{tabular}
\end{center}
%
Instead of |\childdocof{|\textit{main}|}| just include the main file
at the top of each child file:
%
\begin{center}
|\input{|\textit{main}|}|
\end{center}
%
A simple redirection |\childdocforward{|\textit{dest}|}| is achieved by:
%
\begin{center}
|\def\jobname{|\textit{dest}|}\input{\jobname}|
\end{center}
%
The redirection with prefix
|\childdocforwardprefix[|\textit{prefix}|]{|\textit{dest}|}|
is accomplished by:
%
\begin{center}
\begin{tabular}{l}
|{\edef\jobname{\scantokens\expandafter{\jobname\noexpand}}|\\
|\def\redirectjob |\textit{prefix}|#1~~~{\gdef\jobname{|\textit{dest}|#1}}|\\
|\expandafter\redirectjob\jobname~~~}\input{\jobname}|
\end{tabular}
\end{center}

In an alternative approach,
child documents can be compiled by a specific command line
without additional code or specific definitions:
%
\begin{center}
|... -jobname "|\textit{target}|" "|[\textit{flags}]%
|\includeonly{|\textit{dest}|}\input{|\textit{main}|}"|
\end{center}
%

%%%%%%%%%%%%%%%%%%%%%%%%%%%%%%%%%%%%%%%%%%%%%%%%%%%%%%%%%%%%%%%%%%%%%%%%%%%%%%%%
%%%%%%%%%%%%%%%%%%%%%%%%%%%%%%%%%%%%%%%%%%%%%%%%%%%%%%%%%%%%%%%%%%%%%%%%%%%%%%%%
\section{Information}

%%%%%%%%%%%%%%%%%%%%%%%%%%%%%%%%%%%%%%%%%%%%%%%%%%%%%%%%%%%%%%%%%%%%%%%%%%%%%%%%
\subsection{Copyright}

Copyright \copyright{} 2017--2018 Niklas Beisert

This work may be distributed and/or modified under the
conditions of the \LaTeX{} Project Public License, either version 1.3
of this license or (at your option) any later version.
The latest version of this license is in
  \url{http://www.latex-project.org/lppl.txt}
and version 1.3 or later is part of all distributions of \LaTeX{}
version 2005/12/01 or later.

This work has the LPPL maintenance status `maintained'.

The Current Maintainer of this work is Niklas Beisert.

This work consists of the files |README.txt|, |childdoc.ins| and |childdoc.dtx|
as well as the derived files |childdoc.def|, |cdocsamp.tex|
with |cdocsch1.tex|, |cdocsch2.tex|, |cdocspt3.tex|, |cdocspt4.tex|,
|cdocsdrf.tex|, |cdocsfn1.tex|, |cdocsfn2.tex|
as well as |childdoc.pdf|.

%%%%%%%%%%%%%%%%%%%%%%%%%%%%%%%%%%%%%%%%%%%%%%%%%%%%%%%%%%%%%%%%%%%%%%%%%%%%%%%%
\subsection{Files and Installation}

The package consists of the files:
%
\begin{center}
\begin{tabular}{ll}
    |README.txt|   & readme file \\
    |childdoc.ins| & installation file \\
    |childdoc.dtx| & source file \\
    |childdoc.def| & definition file \\
    |cdocsamp.tex| & sample main file \\
    |cdocsch1.tex| & sample include file \\
    |cdocsch2.tex| & sample include file \\
    |cdocspt3.tex| & sample part file \\
    |cdocspt4.tex| & sample part file \\
    |cdocsdrf.tex| & sample redirection file \\
    |cdocsfn1.tex| & sample redirection file \\
    |cdocsfn2.tex| & sample redirection file \\
    |childdoc.pdf| & manual
\end{tabular}
\end{center}
%
The distribution consists of the files
|README.txt|, |childdoc.ins| and |childdoc.dtx|.
%
\begin{itemize}
\item
Run (pdf)\LaTeX{} on |childdoc.dtx|
to compile the manual |childdoc.pdf| (this file).
\item
Run \LaTeX{} on |childdoc.ins| to create the definitions file |childdoc.def|
and the sample |cdocsamp.tex| with include files
|cdocsch1.tex|, |cdocsch2.tex|, |cdocspt3.tex|, |cdocspt4.tex|,
|cdocsdrf.tex|, |cdocsfn1.tex|, |cdocsfn2.tex|.
Then copy the file |childdoc.def| to an appropriate directory of your \LaTeX{}
distribution, e.g.\ \textit{texmf-root}|/tex/latex/childdoc|.
\end{itemize}

%%%%%%%%%%%%%%%%%%%%%%%%%%%%%%%%%%%%%%%%%%%%%%%%%%%%%%%%%%%%%%%%%%%%%%%%%%%%%%%%
\subsection{Related CTAN Packages}

There are several other packages which offer a similar functionality:
%
\begin{itemize}
\item
The packages
\href{http://ctan.org/pkg/docmute}{\textsf{docmute}},
\href{http://ctan.org/pkg/includex}{\textsf{includex}} and
\href{http://ctan.org/pkg/standalone}{\textsf{standalone}}
provide commands to include only the document body of
a child file thus allowing both files to be compiled individually.
\item
The packages \href{http://ctan.org/pkg/subdocs}{\textsf{subdocs}}
and \href{http://ctan.org/pkg/subfiles}{\textsf{subfiles}}
provide structures in which the main and child documents can be
encapsulated and allowing them to be compiled individually.
The inclusion mechanism is different from the conventional |\include|.
\item
The package \href{http://ctan.org/pkg/combine}{\textsf{combine}}
is an elaborate solution to combine several documents into one.
\end{itemize}
%
See also the CTAN topic \href{http://ctan.org/topic/subdocs}{\textsf{subdocs}}
for further related packages.
The present package differs from the above solutions in that
a document structure constructed with the conventional |\include| mechanism
just needs two extra commands at the top of every file
such that all constituent files can be compiled individually.

%%%%%%%%%%%%%%%%%%%%%%%%%%%%%%%%%%%%%%%%%%%%%%%%%%%%%%%%%%%%%%%%%%%%%%%%%%%%%%%%
%\subsection{Feature Suggestions}
%
%The following is a list of features which may be useful for future
%versions of this package:
%%
%\begin{itemize}
%\item
%\ldots
%\end{itemize}

%%%%%%%%%%%%%%%%%%%%%%%%%%%%%%%%%%%%%%%%%%%%%%%%%%%%%%%%%%%%%%%%%%%%%%%%%%%%%%%%
\subsection{Revision History}

%%%%%%%%%%%%%%%%%%%%%%%%%%%%%%%%%%%%%%%%
\paragraph{v2.0:} 2018/12/30

\begin{itemize}
\item
immediate forward processing
\item
added |\childdocby| mechanism
\item
manual restructured
\end{itemize}

%%%%%%%%%%%%%%%%%%%%%%%%%%%%%%%%%%%%%%%%
\paragraph{v1.6:} 2018/01/17

\begin{itemize}
\item
application for development of include files
\item
corrections to manual
\end{itemize}

%%%%%%%%%%%%%%%%%%%%%%%%%%%%%%%%%%%%%%%%
\paragraph{v1.5:} 2017/05/21

\begin{itemize}
\item
more complete structuring introduced
\item
|\childdocof| introduced
\item
|\childdoc| renamed to |\childdocmain|
\item
|\childredirect| renamed to |\childdocforward| and |\childdocforwardprefix|
and functionality expanded
\end{itemize}

%%%%%%%%%%%%%%%%%%%%%%%%%%%%%%%%%%%%%%%%
\paragraph{v1.0:} 2017/04/27

\begin{itemize}
\item
manual and install package
\item
first version published on CTAN
\end{itemize}

%%%%%%%%%%%%%%%%%%%%%%%%%%%%%%%%%%%%%%%%
\paragraph{v0.6:} 2017/04/26

\begin{itemize}
\item
redirection mechanism added
\end{itemize}

%%%%%%%%%%%%%%%%%%%%%%%%%%%%%%%%%%%%%%%%
\paragraph{v0.5:} 2017/04/26

\begin{itemize}
\item
functionality in definition file
\end{itemize}


%%%%%%%%%%%%%%%%%%%%%%%%%%%%%%%%%%%%%%%%%%%%%%%%%%%%%%%%%%%%%%%%%%%%%%%%%%%%%%%%
%%%%%%%%%%%%%%%%%%%%%%%%%%%%%%%%%%%%%%%%%%%%%%%%%%%%%%%%%%%%%%%%%%%%%%%%%%%%%%%%
%%%%%%%%%%%%%%%%%%%%%%%%%%%%%%%%%%%%%%%%%%%%%%%%%%%%%%%%%%%%%%%%%%%%%%%%%%%%%%%%
\appendix

\settowidth\MacroIndent{\rmfamily\scriptsize 000\ }

 \DocInput{childdoc.dtx}

\end{document}
%</driver>
% \fi
%
% %%%%%%%%%%%%%%%%%%%%%%%%%%%%%%%%%%%%%%%%%%%%%%%%%%%%%%%%%%%%%%%%%%%%%%%%%%%%%%
% %%%%%%%%%%%%%%%%%%%%%%%%%%%%%%%%%%%%%%%%%%%%%%%%%%%%%%%%%%%%%%%%%%%%%%%%%%%%%%
% \section{Sample}
%\iffalse
%<*samplemain>
%\fi
%
% The following presents a sample document
% with two chapters, two parts, a title page,
% a compile flag as well as three forwarding files to set the flag.
% It consists of eight |.tex| files:
% \begin{center}
% \begin{tabular}{ll}
% |cdocsamp.tex|&main file\\
% |cdocsch1.tex|&include file for chapter 1\\
% |cdocsch2.tex|&include file for chapter 2\\
% |cdocspt3.tex|&include file for part 3\\
% |cdocspt4.tex|&include file for part 4\\
% |cdocsdrf.tex|&forwarding file for main file in draft mode\\
% |cdocsfi1.tex|&forwarding file for final version of chapter 1\\
% |cdocsfi2.tex|&forwarding file for final version of chapter 2\\
% \end{tabular}
% \end{center}
% Each of the eight files can be compiled directly by the \LaTeX{} compiler.
%
% %%%%%%%%%%%%%%%%%%%%%%%%%%%%%%%%%%%%%%
% \paragraph{Main File.}
%
% The main file is called |cdocsamp.tex|.
%
% Load the \textsf{childdoc} definitions and
% declare the filename for the main document:
%    \begin{macrocode}
\input{childdoc.def}
\childdocmain{}
%    \end{macrocode}

% Optional override for |\version| flag:
%    \begin{macrocode}
%%\ifchilddoc\else\providecommand{\version}{draft}\fi
%    \end{macrocode}

% Define the default values for the |\version| flag
% (|final| for the main file and |draft| for childs):
%    \begin{macrocode}
\ifchilddoc
\providecommand{\version}{draft}
\else
\providecommand{\version}{final}
\fi
%    \end{macrocode}

% Load the standard document class:
%    \begin{macrocode}
\documentclass[12pt]{article}
%    \end{macrocode}

% Start the document body:
%    \begin{macrocode}
\begin{document}
%    \end{macrocode}

% Declare a title page.
% Print title, part of document being processed and version flag:
%    \begin{macrocode}
\addtocounter{page}{-1}
\begin{center}
{\LARGE\bfseries{}childdoc example\par}
\vspace{1cm}
\ifchilddoc
\ifchilddocmanual part\else chapter\fi:
`\childdocname' of `\childdocjob'\par
\else
main document: `\childdocjob'\par
\fi
version: \version\par
\end{center}
\newpage
%    \end{macrocode}

% Manually include selected file,
% otherwise process as usual:
%    \begin{macrocode}
\ifchilddocmanual
\section*{part `\childdocname'}
\input{\childdocname}
\else
%    \end{macrocode}

% Include the two chapters:
%    \begin{macrocode}
\include{cdocsch1}
\include{cdocsch2}
%    \end{macrocode}

% Include the two parts unless only chapters should be displayed:
%    \begin{macrocode}
\ifchilddoc\else
\section{part three}
\input{cdocspt3}
\section{part four}
\input{cdocspt4}
\fi
%    \end{macrocode}

% Process as usual until here:
%    \begin{macrocode}
\fi
%    \end{macrocode}

% End of document body:
%    \begin{macrocode}
\end{document}
%    \end{macrocode}
%\iffalse
%</samplemain>
%\fi
%
% %%%%%%%%%%%%%%%%%%%%%%%%%%%%%%%%%%%%%%
% \paragraph{Chapter Include Files.}
%
% The include files are called |cdocsch1.tex| and |cdocsch2.tex|.
%
%\iffalse
%<*samplechap1|samplechap2>
%\fi

% Optional override for |\version| flag:
%    \begin{macrocode}
%%\providecommand{\version}{final}
%    \end{macrocode}

% Include the main document:
%    \begin{macrocode}
\input{childdoc.def}
\childdocof{cdocsamp}
%    \end{macrocode}

%\iffalse
%</samplechap1|samplechap2>
%\fi
%
%\iffalse
%<*samplechap1>
%\fi
% Some text for chapter 1:
%    \begin{macrocode}
\section{one}
some text in chapter one
%    \end{macrocode}

%\iffalse
%</samplechap1>
%\fi
% Some text for chapter 2:
%\iffalse
%<*samplechap2>
%\fi
%    \begin{macrocode}
\section{two}
more text in chapter two
%    \end{macrocode}

%\iffalse
%</samplechap2>
%\fi
%
% %%%%%%%%%%%%%%%%%%%%%%%%%%%%%%%%%%%%%%
% \paragraph{Part Include Files.}
%
% The include files are called |cdocspt3.tex| and |cdocspt4.tex|.
%
%\iffalse
%<*samplepart3|samplepart4>
%\fi

% Optional override for |\version| flag:
%    \begin{macrocode}
%%\providecommand{\version}{final}
%    \end{macrocode}

% Include the main document:
%    \begin{macrocode}
\input{childdoc.def}
\childdocby{cdocsamp}
%    \end{macrocode}

%\iffalse
%</samplepart3|samplepart4>
%\fi
%
%\iffalse
%<*samplepart3>
%\fi
% Some text for part 3:
%    \begin{macrocode}
some text in part three
%    \end{macrocode}

%\iffalse
%</samplepart3>
%\fi
% Some text for part 4:
%\iffalse
%<*samplepart4>
%\fi
%    \begin{macrocode}
more text in part four
%    \end{macrocode}

%\iffalse
%</samplepart4>
%\fi
%
% %%%%%%%%%%%%%%%%%%%%%%%%%%%%%%%%%%%%%%
% \paragraph{Forwarding for a Complete Draft.}
%
% The following forwarding file |cdocsdrf.tex|
% compiles the main document in draft mode:
%\iffalse
%<*sampledraft>
%\fi
%    \begin{macrocode}
\def\version{draft}
\input{childdoc.def}
\childdocforward{cdocsamp}
%    \end{macrocode}

%\iffalse
%</sampledraft>
%\fi
%
% %%%%%%%%%%%%%%%%%%%%%%%%%%%%%%%%%%%%%%
% \paragraph{Forwarding for Final Version of the Chapters.}
%
% The following forwarding files |cdocsfn1.tex| and |cdocsfn2.tex|
% (with identical content)
% compile the final versions of the child documents
% |cdocsch1.tex| and |cdocsch2.tex|, respectively:
%\iffalse
%<*samplefinal>
%\fi
%    \begin{macrocode}
\def\version{final}
\input{childdoc.def}
\childdocforwardprefix[cdocsamp]{cdocsfn}{cdocsch}
%    \end{macrocode}

%\iffalse
%</samplefinal>
%\fi
%
% %%%%%%%%%%%%%%%%%%%%%%%%%%%%%%%%%%%%%%
% \paragraph{Command Line Processing.}
%
% The following three command lines generate the output files
% |cdocscld|, |cdocscl1| and |cdocscl2|
% which should be identical to
% |cdocsdrf|, |cdocsch1| and |cdocsfn2|, respectively:
% \begin{center}
% \begin{tabular}{l}
% |latex -jobname cdocscld \|\\
% |  "\def\version{draft}\input{childdoc.def}\childdocforward{cdocsamp}"|\\
% |latex -jobname cdocscl1 \|\\
% |  "\input{childdoc.def}\childdocforward[cdocsamp]{cdocsch1}"|\\
% |latex -jobname cdocscl2 \|\\
% |  "\def\version{final}\input{childdoc.def}\childdocforward{cdocsch2}"|
% \end{tabular}
% \end{center}
% Note that the trailing backslash on each first line
% merely continues the input to the second line
% (for convenient cut ant paste).
% Furthermore, the command |latex| can be replaced by any
% of its alternative versions such as |pdflatex|.
%
% %%%%%%%%%%%%%%%%%%%%%%%%%%%%%%%%%%%%%%%%%%%%%%%%%%%%%%%%%%%%%%%%%%%%%%%%%%%%%%
% %%%%%%%%%%%%%%%%%%%%%%%%%%%%%%%%%%%%%%%%%%%%%%%%%%%%%%%%%%%%%%%%%%%%%%%%%%%%%%
% \section{Implementation}
%\iffalse
%<*package>
%\fi
%
% This section describes the definitions file |childdoc.def|.

% The definitions cannot be loaded using |\usepackage| or |\RequirePackage|
% which has a mechanism to prevent loading a style file more than once.
% When loading the definitions by means of |\input|
% multiple instances have to be prevented manually:
%\iffalse
%This code needs to be before the `\ProvidesFile' directive
%which is defined at the beginning of this file.
%Therefore it is also placed there and commented out here.
%</package>
%<*discard>
%\fi
%    \begin{macrocode}
\ifdefined\childdocmain\endinput\fi
%    \end{macrocode}
%\iffalse
%</discard>
%<*package>
%\fi
%
% \macro{\ifchilddoc}
% \macro{\ifchilddocmanual}
% The conditional |\ifchilddoc| tells whether a
% child (true) or main (false) document is being compiled.
% The conditional |\ifchilddocmanual| tells whether
% the |\includeonly| mechanism is used (false) or
% the selection of child files must be performed manually (true).
% The definitions initialise to false:
%    \begin{macrocode}
\newif\ifchilddoc
\newif\ifchilddocmanual
%    \end{macrocode}

% \macro{\childdocname}
% \macro{\childdocjob}
% The macro |\childdocname| stores the name of the main document
% to be compiled. The macro |\childdocjob| stores the name of
% the document on which the \LaTeX{} compiler was originally invoked.
% The content of |\jobname| cannot be compared
% to filenames specified in the source due to different catcodes.
% The following code rescans |\jobname|, stores the result
% in |\childdocname| and saves a copy in |\childdocjob|:
%    \begin{macrocode}
\edef\childdocname{\scantokens\expandafter{\jobname\noexpand}}
\let\childdocjob\childdocname
%    \end{macrocode}

% \macro{\childdocdisable}
% The macro |\childdocdisable| prevents the main file
% from being processed more than once.
% At this stage, the main document command |\childdocmain|
% is assumed to be called once again where it should do nothing.
% Any subsequent call to it should prevent
% a secondary processing of the main document
% It overwrites the forwarding commands
% |\childdocof| and |\childdocforward|
% with empty macros to prevent further inclusions of the main document:
%    \begin{macrocode}
\newcommand{\childdocdisable}
{
  \renewcommand{\childdocmain}[1]{\renewcommand{\childdocmain}[1]{\endinput}}
  \renewcommand{\childdocof}[1]{}
  \renewcommand{\childdocby}[2][]{}
  \renewcommand{\childdocforward}[2][]{}
  \renewcommand{\childdocdisable}{}
}
%    \end{macrocode}

% \macro{\childdocmain}
% The macro |\childdocmain| is to be called at the top of the main file
% with nothing or the main filename (without extension) as argument.
% First, it breaks loops.
% If the argument is not empty and does not match |\childdocname|
% (which is set by the first inclusion of |childdoc.def|),
% |\ifchilddoc| is set to true, |\includeonly| is applied to the child file
% and |\jobname| is set to the main file
% (for proper handling of |.aux| files):
%    \begin{macrocode}
\newcommand{\childdocmain}[1]
{
  \childdocdisable\childdocmain{}
  \if?#1?\else
    \begingroup
      \def\childdoctmp{#1}
      \ifx\childdoctmp\childdocname
        \def\childdoctmp{}
      \else
        \def\childdoctmp
        {
          \childdoctrue
          \includeonly{\childdocname}
          \def\childdocjob{#1}
          \def\jobname{#1}
        }
      \fi
      \expandafter
    \endgroup
    \childdoctmp
  \fi
}
%    \end{macrocode}

% \macro{\childdocof}
% The command |\childdocof| redirects
% compilation to the main file |#1|.
%    \begin{macrocode}
\newcommand{\childdocof}[1]
{
  \childdocdisable
  \childdoctrue
  \includeonly{\childdocname}
  \def\jobname{#1}
  \def\childdocjob{#1}
  \input{#1}
}
%    \end{macrocode}

% \macro{\childdocby}
% The command |\childdocby| ....
%    \begin{macrocode}
\newcommand{\childdocby}[2][]
{
  \childdocdisable
  \childdoctrue
  \childdocmanualtrue
  \if?#1?\else
    \def\jobname{#2}
  \fi
  \def\childdocjob{#2}
  \input{#2}
  \endinput
}
%    \end{macrocode}

% \macro{\childdocforward}
% The command |\childdocforward| redirects
% compilation to the main file or
% (if the optional argument is given) a child file.
% Parameters are set as if the main file
% or a child file starting with |\childdocof| was compiled.
% Then compilation is handed over to the main file:
%    \begin{macrocode}
\newcommand{\childdocforward}[2][]
{
  \begingroup
    \if?#1?
      \def\childdoctmp
      {
        \def\childdocname{#2}
        \def\childdocjob{#2}
        \def\jobname{#2}
        \input{#2}
        \endinput
      }
    \else
      \def\childdoctmp
      {
        \childdocdisable
        \def\childdocname{#2}
        \childdoctrue
        \includeonly{#2}
        \def\childdocjob{#1}
        \def\jobname{#1}
        \input{#1}
        \endinput
      }
    \fi
    \expandafter
  \endgroup
  \childdoctmp
}
%    \end{macrocode}

% \macro{\childdocforwardprefix}
% The command |\childdocforwardprefix| redirects
% compilation to the main or a child file by means of a pattern.
% The prefix |#1| in the current filename is replaced by |#2|
% and the suffix of the current filename is kept
% (it is assumed that the filename does not contain the substring `|~~~|'
% which is used as a delimiter).
% Compilation is handed over to the new file by |\childdocforward|:
%    \begin{macrocode}
\newcommand{\childdocforwardprefix}[3][]
{
  \begingroup
    \def\childdocextract #2##1~~~{\def\childdoctmp{\childdocforward[#1]{#3##1}}}
    \expandafter\childdocextract\childdocname~~~
    \expandafter
  \endgroup
  \childdoctmp
}
%    \end{macrocode}

% \macro{\childdoc}
% The deprecated macro |\childdoc| is a legacy version of |\childdocmain|:
%    \begin{macrocode}
\newcommand{\childdoc}{\childdocmain}
%    \end{macrocode}

% \macro{\childdocredirect}
% The deprecated macro |\childdocredirect| is a legacy version
% of |\childdocforward| and |\childdocforwardprefix|:
%    \begin{macrocode}
\newcommand{\childdocredirect}[2][]
{
  \begingroup
    \if?#1?
      \def\childdoctmp{\childdocforward{#2}}
    \else
      \def\childdoctmp{\childdocforwardprefix{#1}{#2}}
    \fi
    \expandafter
  \endgroup
  \childdoctmp
}
%    \end{macrocode}

%\iffalse
%</package>
%\fi
%
\endinput

\childdocof{cdocsamp}
%    \end{macrocode}

%\iffalse
%</samplechap1|samplechap2>
%\fi
%
%\iffalse
%<*samplechap1>
%\fi
% Some text for chapter 1:
%    \begin{macrocode}
\section{one}
some text in chapter one
%    \end{macrocode}

%\iffalse
%</samplechap1>
%\fi
% Some text for chapter 2:
%\iffalse
%<*samplechap2>
%\fi
%    \begin{macrocode}
\section{two}
more text in chapter two
%    \end{macrocode}

%\iffalse
%</samplechap2>
%\fi
%
% %%%%%%%%%%%%%%%%%%%%%%%%%%%%%%%%%%%%%%
% \paragraph{Part Include Files.}
%
% The include files are called |cdocspt3.tex| and |cdocspt4.tex|.
%
%\iffalse
%<*samplepart3|samplepart4>
%\fi

% Optional override for |\version| flag:
%    \begin{macrocode}
%%\providecommand{\version}{final}
%    \end{macrocode}

% Include the main document:
%    \begin{macrocode}
% \iffalse
%
% childdoc.dtx Copyright (C) 2017-2018 Niklas Beisert
%
% This work may be distributed and/or modified under the
% conditions of the LaTeX Project Public License, either version 1.3
% of this license or (at your option) any later version.
% The latest version of this license is in
%   http://www.latex-project.org/lppl.txt
% and version 1.3 or later is part of all distributions of LaTeX
% version 2005/12/01 or later.
%
% This work has the LPPL maintenance status `maintained'.
%
% The Current Maintainer of this work is Niklas Beisert.
%
% This work consists of the files childdoc.dtx and childdoc.ins
% and the derived files childdoc.def and cdocsamp.tex with
% cdocsch1.tex, cdocsch2.tex, cdocsdrf.tex, cdocsfn1.tex, cdocsfn2.tex.
%
%<package>\ifdefined\childdocmain\endinput\fi
%<package>\ProvidesFile{childdoc.def}[2018/12/30 v2.0 child document driver]
%<samplemain>\ProvidesFile{cdocsamp.tex}[2018/12/30 v2.0 sample for childdoc]
%<*driver>
%\ProvidesFile{childdoc.drv}[2018/12/30 v2.0 childdoc reference manual file]
\PassOptionsToClass{10pt,a4paper}{article}
\documentclass{ltxdoc}

\usepackage[margin=35mm]{geometry}
\usepackage{hyperref}
\usepackage{hyperxmp}
\usepackage[usenames]{color}

\hypersetup{colorlinks=true}
\hypersetup{pdfstartview=FitH}
\hypersetup{pdfpagemode=UseNone}
\hypersetup{pdfsource={}}
\hypersetup{pdflang={en-UK}}
\hypersetup{pdfcopyright={Copyright 2017-2018 Niklas Beisert.
  This work may be distributed and/or modified under the
  conditions of the LaTeX Project Public License, either version 1.3
  of this license or (at your option) any later version.}}
\hypersetup{pdflicenseurl={http://www.latex-project.org/lppl.txt}}
\hypersetup{pdfcontactaddress={ETH Zurich, ITP, HIT K,
  Wolfgang-Pauli-Strasse 27}}
\hypersetup{pdfcontactpostcode={8093}}
\hypersetup{pdfcontactcity={Zurich}}
\hypersetup{pdfcontactcountry={Switzerland}}
\hypersetup{pdfcontactemail={nbeisert@itp.phys.ethz.ch}}
\hypersetup{pdfcontacturl={http://people.phys.ethz.ch/\xmptilde nbeisert/}}

\newcommand{\secref}[1]{\hyperref[#1]{section \ref*{#1}}}

\parskip1ex
\parindent0pt
\let\olditemize\itemize
\def\itemize{\olditemize\parskip0pt}

\begin{document}

\title{The \textsf{childdoc} Package}
\hypersetup{pdftitle={The childdoc Package}}
\author{Niklas Beisert\\[2ex]
  Institut f\"ur Theoretische Physik\\
  Eidgen\"ossische Technische Hochschule Z\"urich\\
  Wolfgang-Pauli-Strasse 27, 8093 Z\"urich, Switzerland\\[1ex]
  \href{mailto:nbeisert@itp.phys.ethz.ch}
  {\texttt{nbeisert@itp.phys.ethz.ch}}}
\hypersetup{pdfauthor={Niklas Beisert}}
\hypersetup{pdfsubject={Manual for the LaTeX2e Package childdoc}}
\date{30 December 2018, \textsf{v2.0}}
\maketitle

\begin{abstract}\noindent
\textsf{childdoc} is a \LaTeXe{} package
that enables the direct compilation
of document sections included by |\include|
to individual files.
\end{abstract}

\begingroup
\parskip0ex
\tableofcontents
\endgroup

%%%%%%%%%%%%%%%%%%%%%%%%%%%%%%%%%%%%%%%%%%%%%%%%%%%%%%%%%%%%%%%%%%%%%%%%%%%%%%%%
%%%%%%%%%%%%%%%%%%%%%%%%%%%%%%%%%%%%%%%%%%%%%%%%%%%%%%%%%%%%%%%%%%%%%%%%%%%%%%%%
\section{Introduction}

\LaTeX{} provides a mechanism to structure a large document (such as a book)
into a main file and several child files (containing the chapters)
using the |\include| command.
This mechanism is beneficial for documents
which span hundreds of pages in order to
make the source file(s) more manageable.
Moreover, compilation can be restricted to
selected child files by means of the |\includeonly| command.
The latter feature can be used to reduce the compilation time while editing
(this was significantly more useful in the earlier days of \LaTeX{})
or to generate a smaller document which is easier to navigate.
Another application of |\includeonly| is to generate
documents consisting of selected parts of the complete document.

However, there are a few drawbacks of the plain |\include| mechanism:
\begin{itemize}
\item
The child files cannot be compiled on their own,
they can only be compiled via the main file.
A naive editing environment
(such as a text editor with an option
to have the current file processed by \LaTeX)
may require one to switch to the main file before compiling;
attempting to compile the child file produces errors.
\item
The main file must be modified (each time)
to adjust the |\includeonly| command
to the present needs. This easily leaves the main file in a messy state.
\item
The generated document will always carry the filename
of the main document. This is inconvenient if
several child files are to be compiled and
to be kept for distribution.
\end{itemize}

The present package provides a simple interface
to make child files individually compilable by \LaTeX{}.
Compiling a child file then has the same effect as compiling
the main file with an |\includeonly| command
to select the appropriate child.
Moreover the generated document will carry the name of the child
rather than the main file.
This resolves all three above issues.

This feature is meant to make the editing of books,
thesis documents and lecture notes somewhat more convenient.
However, the package can also be used efficiently for
composing a series of documents (such as exercise sheets)
which are typically distributed individually.
It then assists the author in generating the individual documents
(potentially in different versions)
as well as a document containing the collected series.
Another application is in developing style files
or other kinds of included material
where compilation of the style file could redirect
to a sample or test file.

%%%%%%%%%%%%%%%%%%%%%%%%%%%%%%%%%%%%%%%%%%%%%%%%%%%%%%%%%%%%%%%%%%%%%%%%%%%%%%%%
%%%%%%%%%%%%%%%%%%%%%%%%%%%%%%%%%%%%%%%%%%%%%%%%%%%%%%%%%%%%%%%%%%%%%%%%%%%%%%%%
\section{Usage}

First of all, the package \textsf{childdoc} is \emph{not} a standard
\LaTeXe{} |.sty| style file! Therefore it needs to be invoked in
a non-standard way.

%%%%%%%%%%%%%%%%%%%%%%%%%%%%%%%%%%%%%%%%%%%%%%%%%%%%%%%%%%%%%%%%%%%%%%%%%%%%%%%%
\subsection{Included Files}
\label{sec:include}

%%%%%%%%%%%%%%%%%%%%%%%%%%%%%%%%%%%%%%%%
\DescribeMacro{\childdocmain}
To use the package, add the commands
\begin{center}
\begin{tabular}{l}
|\input{childdoc.def}|\\
|\childdocmain{}|\\
\end{tabular}
\end{center}
at the very top of the main \LaTeX{} file,
in particular \emph{before} the |\documentclass| statement!
The argument of |\childdocmain| should be left empty
(but it must be present).

%%%%%%%%%%%%%%%%%%%%%%%%%%%%%%%%%%%%%%%%
\DescribeMacro{\childdocof}
Furthermore, add the commands
\begin{center}
\begin{tabular}{l}
|\input{childdoc.def}|\\
|\childdocof{|\textit{main}|}|\\
\end{tabular}
\end{center}
at the top of every child file \textit{child}
which is included by |\include{|\textit{child}|}|
from within the main file
(or at least for those files to be compiled individually).
The argument \textit{main} must be the filename of the main file.

There are a couple of
considerations in setting up the main and child documents:

%%%%%%%%%%%%%%%%%%%%%%%%%%%%%%%%%%%%%%%%
\paragraph{Restrictions.}

Please note the following restrictions:
\begin{itemize}
\item
|\childdocmain| must be called with one argument \textit{main}
to ensure compatibility with earlier version of the package.
It must either be empty (|\childdocmain{}|)
or precisely match the filename of the main file in which it is specified.
See \secref{sec:detection} for further information.
\item
The filename \textit{main} must be specified without the |.tex| extension.
\item
The filename \textit{main} is case sensitive
(even in case-insensitive file systems)
due to internal string comparison.
\item
The argument \textit{main} should be fully expanded, it cannot be a macro.
\item
Subdirectories and special characters should be avoided in filenames.
\item
The command |\childdocmain{|\textit{main}|}| must be followed by a whitespace.
It should not be followed immediately by another command
or by a comment mark `|%|'.
This is because the \TeX{} parser reads the token immediately following
the argument of |\childdocmain| and puts it
at the beginning of every child section;
however, a white\-space is ignored.
\end{itemize}

%%%%%%%%%%%%%%%%%%%%%%%%%%%%%%%%%%%%%%%%
\paragraph{Content of Main File.}

It is advisable to place all content in the child files included by |\include|.
Any output contained in the main file will appear in all child documents
unless suppressed manually;
it cannot be suppressed automatically by the |\includeonly| directive
and thus should normally be avoided.
A method to include some content in the main file
by means of conditional processing is described in \secref{sec:conditional}.

%%%%%%%%%%%%%%%%%%%%%%%%%%%%%%%%%%%%%%%%
\paragraph{Page Numbering.}

When only a part of the document is compiled,
the appropriate numbering of pages
(as well as other status parameters)
is determined from the |.aux| files.
The latter contain information from previous passes.
However this information needs to propagate through
all intermediate child documents.
Therefore the page numbering in child documents may well
be inconsistent until the complete document is compiled at least once.

A useful (if unconventional) way to always ensure a consistent
page numbering is to restart the numbering in each child document
and denote the pages by `\textit{child}|.|\textit{page}'
where \textit{child} represents the chapter/section number of the child file.
This can be achieved by the command
|\numberwithin{page}{|\textit{child}|}|
of the \textsf{amsmath} package
where \textit{child} can be |chapter| or |section|
depending on the chosen structuring.
Alternatively, one can modify the macro |\thepage| appropriately
and reset the counter |page| at the start of each child file.

%%%%%%%%%%%%%%%%%%%%%%%%%%%%%%%%%%%%%%%%%%%%%%%%%%%%%%%%%%%%%%%%%%%%%%%%%%%%%%%%
\subsection{Conditional Processing}
\label{sec:conditional}

The package provides a mechanism to compile different versions
of a document. To customise the versions further some conditional processing
can come in handy to distinguish which version is being compiled.
The package provides two macros to describe the compilation context:

%%%%%%%%%%%%%%%%%%%%%%%%%%%%%%%%%%%%%%%%
\DescribeMacro{\ifchilddoc}
The conditional |\ifchilddoc| distinguishes between the compilation of
child documents and the main document:
%
\begin{center}
|\ifchilddoc |\textit{child-code}| |[|\||else |\textit{main-code}]| \||fi|
\end{center}

%%%%%%%%%%%%%%%%%%%%%%%%%%%%%%%%%%%%%%%%
\DescribeMacro{\childdocname}
\DescribeMacro{\childdocjob}
The macro |\childdocname| contains the filename (without extension)
of the main or child file being processed.
Note that |\childdocjob| will always contain the name of the main file.

%%%%%%%%%%%%%%%%%%%%%%%%%%%%%%%%%%%%%%%%
\paragraph{Title Page.}

Conditional processing can be used to include a title or banner page
in the main document when proper precautions are taken.
Importantly, the code in the main file should ensure that the page counter
(as well as other status parameters which are stored in the |.aux| files)
takes the same value after the conditional processing.
Otherwise the page numbers may take divergent values
depending on which part is compiled.

For example, a title page could be declared by:
%
\begin{center}
\begin{tabular}{l}
|\ifchilddoc\||else|\\
|\addtocounter{page}{-1}|\\
\textit{code for title page}\\
|\newpage|\\
|\||fi|
\end{tabular}
\end{center}
%
A banner page for the child documents can be generated by:
%
\begin{center}
\begin{tabular}{l}
|\ifchilddoc|\\
|\addtocounter{page}{-1}|\\
\textit{code for banner page}\\
|\newpage|\\
|\||fi|
\end{tabular}
\end{center}
%
Here one could write a message such as:
\begin{center}
|This is the part \childdocname{} of \childdocjob{}.|
\end{center}

%%%%%%%%%%%%%%%%%%%%%%%%%%%%%%%%%%%%%%%%%%%%%%%%%%%%%%%%%%%%%%%%%%%%%%%%%%%%%%%%
\subsection{Flags}
\label{sec:flags}

The package makes it easy to generate different versions
of the main or child documents.
To this end compilation flags can be defined
and assigned different default values.
They will be particularly useful in conjunction
with the forwarding mechanism described in \secref{sec:forward}.

For example, it may be useful to have a flag |\version|
which can be set to |draft| or |final|.
The document source will contain some conditional code
depending on the value of |\version|.
Suppose further, the flag should default to |final| for the main file
and to |draft| for child files
which is a natural assignment for editing the document.
This is achieved by placing the following code
in the preamble of the main document
(below the |\childdocmain| directive):
%
\begin{center}
\begin{tabular}{l}
|\ifchilddoc|\\
|\providecommand{\version}{draft}|\\
|\||else|\\
|\providecommand{\version}{final}|\\
|\||fi|
\end{tabular}
\end{center}
%
The definition by |\providecommand| makes sure
that previous definitions are not overwritten.
Further statements |\providecommand{\version}{...}|
can thus be added before the above code to override it.

For the main file, one might add a line
(between |\childdocmain| and the above block)
%
\begin{center}
|%\ifchilddoc\||else\providecommand{\version}{draft}\||fi|
\end{center}
%
which can be uncommented to produce a draft version.
Likewise one can add a line to the very top of a child file
(above the |\childdocof{|\textit{main}|}| directive)
%
\begin{center}
|%\providecommand{\version}{final}|
\end{center}
%
which can be uncommented to produce the final version of this child document.

%%%%%%%%%%%%%%%%%%%%%%%%%%%%%%%%%%%%%%%%%%%%%%%%%%%%%%%%%%%%%%%%%%%%%%%%%%%%%%%%
\subsection{Forwarding}
\label{sec:forward}

Different versions of the main or child documents
using compilation flags as described in \secref{sec:flags}
can be (permanently) stored in different files
for convenient compilation, viewing and distribution.
To this end, the package defines a command
to pass on compilation to a different file:

%%%%%%%%%%%%%%%%%%%%%%%%%%%%%%%%%%%%%%%%
\DescribeMacro{\childdocforward}
The command |\childdocforward| redirects processing to
another source file:
%
\begin{center}
\begin{tabular}{l}
|\input{childdoc.def}|\\
|\childdocforward[|\textit{main}|]{|\textit{dest}|}|\\
\end{tabular}
\end{center}
%
The argument \textit{dest} is the destination file
(without extension).
It should be the main file or one of the child files.
Note that further \textsf{childdoc} directives
such as |\childdocof| and |\childdocforward|
in the indicated file will be processed in this form.
The optional argument \textit{main}
passes on directly to the main file \textit{main}
while pretending to compile the child \textit{dest}.
This form behaves as if \textit{dest}
issues |\childdocof{|\textit{main}|}| right away,
and no further \textsf{childdoc} directives will be processed.

%%%%%%%%%%%%%%%%%%%%%%%%%%%%%%%%%%%%%%%%
\DescribeMacro{\...prefix}
In the alternative form |\childdocforwardprefix|,
%
\begin{center}
\begin{tabular}{l}
|\input{childdoc.def}|\\
|\childdocforwardprefix[|\textit{main}|]{|\textit{prefix}|}{|\textit{dest}|}|
\end{tabular}
\end{center}
%
the destination file is determined by a pattern
depending on the current file:
To make this work, the current file must be called
`{\textit{prefix}\hspace{0.2em}\textit{suffix}}'
with \textit{prefix} matching precisely the argument.
Processing is then passed on to the file
`{\textit{dest}\hspace{0.2em}\textit{suffix}}'.
Surely, the same effect is achieved by
directly specifying the
argument `{\textit{dest}\hspace{0.2em}\textit{suffix}}'
in the first form.
However, that requires to set up a different file
for each child. With the alternative form of the command
all these files can have exactly the same content
which simplifies setting them up and maintaining them.

For example, the following file |draft.tex|
with a compilation flag |\version| as described in \secref{sec:flags}
compiles the main document as a draft:
%
\begin{center}
\begin{tabular}{l}
|\def\version{draft}|\\
|\input{childdoc.def}|\\
|\childdocforward{|\textit{main}|}|
\end{tabular}
\end{center}
%
Likewise, the following files |final|\textit{nn}|.tex|
compile the final version of the child document
|child|\textit{nn}|.tex|:
%
\begin{center}
\begin{tabular}{l}
|\def\version{final}|\\
|\input{childdoc.def}|\\
|\childdocforwardprefix{final}{child}|
\end{tabular}
\end{center}
%

Note that when several versions of a main file and/or of each child file
are to be generated, it may be convenient to set up a |Makefile| or
shell script to automatise the process.

%%%%%%%%%%%%%%%%%%%%%%%%%%%%%%%%%%%%%%%%%%%%%%%%%%%%%%%%%%%%%%%%%%%%%%%%%%%%%%%%
\subsection{Command Line Processing}
\label{sec:commandline}

The effect of redirection files can also be achieved by invoking
the \LaTeX{} compiler with a more elaborate command line.
Most conveniently this should be done as part
of a shell script or a |Makefile|.

When using \textsf{childdoc} in the main file, the following
command lines effectively perform a redirection
(note that depending on the shell being used,
backslashes may have to be doubled: `|\|' $\to$ `|\\|'):
%
\begin{center}
|... -jobname "|\textit{target}|" |\\|"|[\textit{flags}]%
|\input{childdoc.def}\childdocforward[|\textit{main}|]{|\textit{dest}|}"|
\end{center}
%
Here \textit{target} is the name of the output file,
\textit{main} is the name of the main file
and \textit{dest} is the name of the main or child file to be processed
(all filenames without extensions).
The optional argument \textit{main} can be omitted
if \textit{main} matches \textit{dest}.
Optionally, compilation \textit{flags} can be defined via |\def| commands.
This command line makes the \TeX{} engine believe
it is compiling the file \textit{target}
whose content is specified as the latter parameter.
The provided code then forwards the processing to
\textit{main} or \textit{dest} as described in \secref{sec:forward}.

%%%%%%%%%%%%%%%%%%%%%%%%%%%%%%%%%%%%%%%%%%%%%%%%%%%%%%%%%%%%%%%%%%%%%%%%%%%%%%%%
\subsection{Include by Input}
\label{sec:input}

Including child documents by |\include| has some restrictions by design.
Most notably, the content of a child document always occupies
its own set of pages; pages cannot be shared between child documents.
Usually, this behaviour makes perfect sense
because each child document contain an essential part of the document.
However, in some situations it may be desirable to compose
a document from a collection of parts
without having mandatory page breaks between then.
For this case, the package
provides a mechanism to include parts
by |\input| which can also be processed individually.
However, by construction this mechanism
requires manual handling of the content to be output.

%%%%%%%%%%%%%%%%%%%%%%%%%%%%%%%%%%%%%%%%
\DescribeMacro{\ifchilddocmanual}
The main file should be prepared as usual, see \secref{sec:include}.
However, the document body must make a distinction
between processing of an individual part and of the main document, e.g.:
%
\begin{center}
\begin{tabular}{l}
|\ifchilddocmanual|\\
|\input{\childdocname}|\\
|\||else|\\
\textit{document body with }|\input{|\textit{part}|}|\\
|\||fi|
\end{tabular}
\end{center}
%
The conditional |\ifchilddocmanual| is true whenever
a part to be included by |\input| is being compiled,
and the name of the part is stored in |\childdocname|.

%%%%%%%%%%%%%%%%%%%%%%%%%%%%%%%%%%%%%%%%
\DescribeMacro{\childdocby}
Each part to be included by |\input| should start with:
%
\begin{center}
\begin{tabular}{l}
|\input{childdoc.def}|\\
|\childdocby{|\textit{main}|}|\\
\end{tabular}
\end{center}
%
The directive |\childdocby| is similar to |\childdocof|
described in \secref{sec:include},
but the subsequent selection of content must be done manually.
To that end, both |\ifchilddoc| and |\ifchilddocmanual|
will be true upon processing of a part,
and the name of the part is stored in |\childdocname|.
Note that |\jobname| will be set to the filename of the current part
so that each part receives an individual |.aux| file
that does not interfere with the |.aux| file(s) of the main document.
This behaviour can be altered by the alternative form
|\childdocby[*]{|\textit{main}|}| (with a non-empty optional argument)
which uses the |.aux| file of the main document
by setting |\jobname| to \textit{main}.

%%%%%%%%%%%%%%%%%%%%%%%%%%%%%%%%%%%%%%%%%%%%%%%%%%%%%%%%%%%%%%%%%%%%%%%%%%%%%%%%
\subsection{Driver Development}
\label{sec:driver}

The \textsf{childdoc} mechanism can also be use for the development
of definition files such as \LaTeX{} styles or classes.
This case differs from the above setup with multiple parts
included by |\include| in that no |\includeonly| should be invoked.
This can be achieved by starting the include file
(before |\ProvidesPackage|) with:
%
\begin{center}
\begin{tabular}{l}
|\input{childdoc.def}|\\
|\childdocforward{|\textit{main}|}|\\
\end{tabular}
\end{center}
%
or alternatively with:
%
\begin{center}
\begin{tabular}{l}
|\input{childdoc.def}|\\
|\childdocby{|\textit{main}|}|\\
\end{tabular}
\end{center}
%
Both forms have slightly different effects as described above.
The main file is prepared as usual, see \secref{sec:include}.

%%%%%%%%%%%%%%%%%%%%%%%%%%%%%%%%%%%%%%%%%%%%%%%%%%%%%%%%%%%%%%%%%%%%%%%%%%%%%%%%
\subsection{Legacy Detection}
\label{sec:detection}

The directive |\childdocmain| in the main file can detect
whether the complete document or merely a child is to be compiled
even without using the directive |\childdocof|.
This method is deprecated because it is less robust
and there is no compelling reason to use it;
it is merely provided for backward compatibility
and it may be removed in future versions.

If the detection mechanism is to be used,
it is mandatory to correctly specify
the filename of the main file as the argument of |\childdocmain|:
%
\begin{center}
\begin{tabular}{l}
|\input{childdoc.def}|\\
|\childdocmain{|\textit{main}|}|\\
\end{tabular}
\end{center}
%
If |\jobname| does not match the argument \textit{main} of |\childdocmain|,
it is assumed that |\jobname| points to the child file to be compiled.
When using |\childdocmain| with the main file specified as argument,
it suffices to start a child file
with just |\input{|\textit{main}|}|
without loading of the package and using |\childdocof|.
If instead all processing is done
with the appropriate \textsf{childdoc} directives,
the argument of \textit{main} of |\childdocmain| can be empty.

An alternative version of the command line processing described
in \secref{sec:commandline} using the detection mechanism reads:
%
\begin{center}
|... -jobname "|\textit{target}|" "|[\textit{flags}]%
[|\def\jobname{|\textit{dest}|}|]|\input{|\textit{main}|}"|
\end{center}

%%%%%%%%%%%%%%%%%%%%%%%%%%%%%%%%%%%%%%%%%%%%%%%%%%%%%%%%%%%%%%%%%%%%%%%%%%%%%%%%
\subsection{Manual Code}
\label{sec:manual}

In case one cannot be certain whether the definitions file |childdoc.def|
is installed on the target \TeX{} distribution
and one prefers not to ship it,
it is conceivable to paste a few relevant commands into the sources.

To that end, drop all statements |\input{childdoc.def}|
and perform the replacements as outlined below.
Instead of |\childdocmain{|\textit{main}|}| add the following code
to the top of the main file:
%
\begin{center}
\begin{tabular}{l}
|\||ifdefined\childdocname\endinput\||fi\newif\ifchilddoc|\\
|\edef\childdocname{\scantokens\expandafter{\jobname\noexpand}}|\\
|\def\childdocmain{|\textit{main}|}\||ifx\childdocmain\childdocname\||else|\\
|\childdoctrue\includeonly{\childdocname}\let\jobname\childdocmain\||fi|\\
\end{tabular}
\end{center}
%
Instead of |\childdocof{|\textit{main}|}| just include the main file
at the top of each child file:
%
\begin{center}
|\input{|\textit{main}|}|
\end{center}
%
A simple redirection |\childdocforward{|\textit{dest}|}| is achieved by:
%
\begin{center}
|\def\jobname{|\textit{dest}|}\input{\jobname}|
\end{center}
%
The redirection with prefix
|\childdocforwardprefix[|\textit{prefix}|]{|\textit{dest}|}|
is accomplished by:
%
\begin{center}
\begin{tabular}{l}
|{\edef\jobname{\scantokens\expandafter{\jobname\noexpand}}|\\
|\def\redirectjob |\textit{prefix}|#1~~~{\gdef\jobname{|\textit{dest}|#1}}|\\
|\expandafter\redirectjob\jobname~~~}\input{\jobname}|
\end{tabular}
\end{center}

In an alternative approach,
child documents can be compiled by a specific command line
without additional code or specific definitions:
%
\begin{center}
|... -jobname "|\textit{target}|" "|[\textit{flags}]%
|\includeonly{|\textit{dest}|}\input{|\textit{main}|}"|
\end{center}
%

%%%%%%%%%%%%%%%%%%%%%%%%%%%%%%%%%%%%%%%%%%%%%%%%%%%%%%%%%%%%%%%%%%%%%%%%%%%%%%%%
%%%%%%%%%%%%%%%%%%%%%%%%%%%%%%%%%%%%%%%%%%%%%%%%%%%%%%%%%%%%%%%%%%%%%%%%%%%%%%%%
\section{Information}

%%%%%%%%%%%%%%%%%%%%%%%%%%%%%%%%%%%%%%%%%%%%%%%%%%%%%%%%%%%%%%%%%%%%%%%%%%%%%%%%
\subsection{Copyright}

Copyright \copyright{} 2017--2018 Niklas Beisert

This work may be distributed and/or modified under the
conditions of the \LaTeX{} Project Public License, either version 1.3
of this license or (at your option) any later version.
The latest version of this license is in
  \url{http://www.latex-project.org/lppl.txt}
and version 1.3 or later is part of all distributions of \LaTeX{}
version 2005/12/01 or later.

This work has the LPPL maintenance status `maintained'.

The Current Maintainer of this work is Niklas Beisert.

This work consists of the files |README.txt|, |childdoc.ins| and |childdoc.dtx|
as well as the derived files |childdoc.def|, |cdocsamp.tex|
with |cdocsch1.tex|, |cdocsch2.tex|, |cdocspt3.tex|, |cdocspt4.tex|,
|cdocsdrf.tex|, |cdocsfn1.tex|, |cdocsfn2.tex|
as well as |childdoc.pdf|.

%%%%%%%%%%%%%%%%%%%%%%%%%%%%%%%%%%%%%%%%%%%%%%%%%%%%%%%%%%%%%%%%%%%%%%%%%%%%%%%%
\subsection{Files and Installation}

The package consists of the files:
%
\begin{center}
\begin{tabular}{ll}
    |README.txt|   & readme file \\
    |childdoc.ins| & installation file \\
    |childdoc.dtx| & source file \\
    |childdoc.def| & definition file \\
    |cdocsamp.tex| & sample main file \\
    |cdocsch1.tex| & sample include file \\
    |cdocsch2.tex| & sample include file \\
    |cdocspt3.tex| & sample part file \\
    |cdocspt4.tex| & sample part file \\
    |cdocsdrf.tex| & sample redirection file \\
    |cdocsfn1.tex| & sample redirection file \\
    |cdocsfn2.tex| & sample redirection file \\
    |childdoc.pdf| & manual
\end{tabular}
\end{center}
%
The distribution consists of the files
|README.txt|, |childdoc.ins| and |childdoc.dtx|.
%
\begin{itemize}
\item
Run (pdf)\LaTeX{} on |childdoc.dtx|
to compile the manual |childdoc.pdf| (this file).
\item
Run \LaTeX{} on |childdoc.ins| to create the definitions file |childdoc.def|
and the sample |cdocsamp.tex| with include files
|cdocsch1.tex|, |cdocsch2.tex|, |cdocspt3.tex|, |cdocspt4.tex|,
|cdocsdrf.tex|, |cdocsfn1.tex|, |cdocsfn2.tex|.
Then copy the file |childdoc.def| to an appropriate directory of your \LaTeX{}
distribution, e.g.\ \textit{texmf-root}|/tex/latex/childdoc|.
\end{itemize}

%%%%%%%%%%%%%%%%%%%%%%%%%%%%%%%%%%%%%%%%%%%%%%%%%%%%%%%%%%%%%%%%%%%%%%%%%%%%%%%%
\subsection{Related CTAN Packages}

There are several other packages which offer a similar functionality:
%
\begin{itemize}
\item
The packages
\href{http://ctan.org/pkg/docmute}{\textsf{docmute}},
\href{http://ctan.org/pkg/includex}{\textsf{includex}} and
\href{http://ctan.org/pkg/standalone}{\textsf{standalone}}
provide commands to include only the document body of
a child file thus allowing both files to be compiled individually.
\item
The packages \href{http://ctan.org/pkg/subdocs}{\textsf{subdocs}}
and \href{http://ctan.org/pkg/subfiles}{\textsf{subfiles}}
provide structures in which the main and child documents can be
encapsulated and allowing them to be compiled individually.
The inclusion mechanism is different from the conventional |\include|.
\item
The package \href{http://ctan.org/pkg/combine}{\textsf{combine}}
is an elaborate solution to combine several documents into one.
\end{itemize}
%
See also the CTAN topic \href{http://ctan.org/topic/subdocs}{\textsf{subdocs}}
for further related packages.
The present package differs from the above solutions in that
a document structure constructed with the conventional |\include| mechanism
just needs two extra commands at the top of every file
such that all constituent files can be compiled individually.

%%%%%%%%%%%%%%%%%%%%%%%%%%%%%%%%%%%%%%%%%%%%%%%%%%%%%%%%%%%%%%%%%%%%%%%%%%%%%%%%
%\subsection{Feature Suggestions}
%
%The following is a list of features which may be useful for future
%versions of this package:
%%
%\begin{itemize}
%\item
%\ldots
%\end{itemize}

%%%%%%%%%%%%%%%%%%%%%%%%%%%%%%%%%%%%%%%%%%%%%%%%%%%%%%%%%%%%%%%%%%%%%%%%%%%%%%%%
\subsection{Revision History}

%%%%%%%%%%%%%%%%%%%%%%%%%%%%%%%%%%%%%%%%
\paragraph{v2.0:} 2018/12/30

\begin{itemize}
\item
immediate forward processing
\item
added |\childdocby| mechanism
\item
manual restructured
\end{itemize}

%%%%%%%%%%%%%%%%%%%%%%%%%%%%%%%%%%%%%%%%
\paragraph{v1.6:} 2018/01/17

\begin{itemize}
\item
application for development of include files
\item
corrections to manual
\end{itemize}

%%%%%%%%%%%%%%%%%%%%%%%%%%%%%%%%%%%%%%%%
\paragraph{v1.5:} 2017/05/21

\begin{itemize}
\item
more complete structuring introduced
\item
|\childdocof| introduced
\item
|\childdoc| renamed to |\childdocmain|
\item
|\childredirect| renamed to |\childdocforward| and |\childdocforwardprefix|
and functionality expanded
\end{itemize}

%%%%%%%%%%%%%%%%%%%%%%%%%%%%%%%%%%%%%%%%
\paragraph{v1.0:} 2017/04/27

\begin{itemize}
\item
manual and install package
\item
first version published on CTAN
\end{itemize}

%%%%%%%%%%%%%%%%%%%%%%%%%%%%%%%%%%%%%%%%
\paragraph{v0.6:} 2017/04/26

\begin{itemize}
\item
redirection mechanism added
\end{itemize}

%%%%%%%%%%%%%%%%%%%%%%%%%%%%%%%%%%%%%%%%
\paragraph{v0.5:} 2017/04/26

\begin{itemize}
\item
functionality in definition file
\end{itemize}


%%%%%%%%%%%%%%%%%%%%%%%%%%%%%%%%%%%%%%%%%%%%%%%%%%%%%%%%%%%%%%%%%%%%%%%%%%%%%%%%
%%%%%%%%%%%%%%%%%%%%%%%%%%%%%%%%%%%%%%%%%%%%%%%%%%%%%%%%%%%%%%%%%%%%%%%%%%%%%%%%
%%%%%%%%%%%%%%%%%%%%%%%%%%%%%%%%%%%%%%%%%%%%%%%%%%%%%%%%%%%%%%%%%%%%%%%%%%%%%%%%
\appendix

\settowidth\MacroIndent{\rmfamily\scriptsize 000\ }

 \DocInput{childdoc.dtx}

\end{document}
%</driver>
% \fi
%
% %%%%%%%%%%%%%%%%%%%%%%%%%%%%%%%%%%%%%%%%%%%%%%%%%%%%%%%%%%%%%%%%%%%%%%%%%%%%%%
% %%%%%%%%%%%%%%%%%%%%%%%%%%%%%%%%%%%%%%%%%%%%%%%%%%%%%%%%%%%%%%%%%%%%%%%%%%%%%%
% \section{Sample}
%\iffalse
%<*samplemain>
%\fi
%
% The following presents a sample document
% with two chapters, two parts, a title page,
% a compile flag as well as three forwarding files to set the flag.
% It consists of eight |.tex| files:
% \begin{center}
% \begin{tabular}{ll}
% |cdocsamp.tex|&main file\\
% |cdocsch1.tex|&include file for chapter 1\\
% |cdocsch2.tex|&include file for chapter 2\\
% |cdocspt3.tex|&include file for part 3\\
% |cdocspt4.tex|&include file for part 4\\
% |cdocsdrf.tex|&forwarding file for main file in draft mode\\
% |cdocsfi1.tex|&forwarding file for final version of chapter 1\\
% |cdocsfi2.tex|&forwarding file for final version of chapter 2\\
% \end{tabular}
% \end{center}
% Each of the eight files can be compiled directly by the \LaTeX{} compiler.
%
% %%%%%%%%%%%%%%%%%%%%%%%%%%%%%%%%%%%%%%
% \paragraph{Main File.}
%
% The main file is called |cdocsamp.tex|.
%
% Load the \textsf{childdoc} definitions and
% declare the filename for the main document:
%    \begin{macrocode}
\input{childdoc.def}
\childdocmain{}
%    \end{macrocode}

% Optional override for |\version| flag:
%    \begin{macrocode}
%%\ifchilddoc\else\providecommand{\version}{draft}\fi
%    \end{macrocode}

% Define the default values for the |\version| flag
% (|final| for the main file and |draft| for childs):
%    \begin{macrocode}
\ifchilddoc
\providecommand{\version}{draft}
\else
\providecommand{\version}{final}
\fi
%    \end{macrocode}

% Load the standard document class:
%    \begin{macrocode}
\documentclass[12pt]{article}
%    \end{macrocode}

% Start the document body:
%    \begin{macrocode}
\begin{document}
%    \end{macrocode}

% Declare a title page.
% Print title, part of document being processed and version flag:
%    \begin{macrocode}
\addtocounter{page}{-1}
\begin{center}
{\LARGE\bfseries{}childdoc example\par}
\vspace{1cm}
\ifchilddoc
\ifchilddocmanual part\else chapter\fi:
`\childdocname' of `\childdocjob'\par
\else
main document: `\childdocjob'\par
\fi
version: \version\par
\end{center}
\newpage
%    \end{macrocode}

% Manually include selected file,
% otherwise process as usual:
%    \begin{macrocode}
\ifchilddocmanual
\section*{part `\childdocname'}
\input{\childdocname}
\else
%    \end{macrocode}

% Include the two chapters:
%    \begin{macrocode}
\include{cdocsch1}
\include{cdocsch2}
%    \end{macrocode}

% Include the two parts unless only chapters should be displayed:
%    \begin{macrocode}
\ifchilddoc\else
\section{part three}
\input{cdocspt3}
\section{part four}
\input{cdocspt4}
\fi
%    \end{macrocode}

% Process as usual until here:
%    \begin{macrocode}
\fi
%    \end{macrocode}

% End of document body:
%    \begin{macrocode}
\end{document}
%    \end{macrocode}
%\iffalse
%</samplemain>
%\fi
%
% %%%%%%%%%%%%%%%%%%%%%%%%%%%%%%%%%%%%%%
% \paragraph{Chapter Include Files.}
%
% The include files are called |cdocsch1.tex| and |cdocsch2.tex|.
%
%\iffalse
%<*samplechap1|samplechap2>
%\fi

% Optional override for |\version| flag:
%    \begin{macrocode}
%%\providecommand{\version}{final}
%    \end{macrocode}

% Include the main document:
%    \begin{macrocode}
\input{childdoc.def}
\childdocof{cdocsamp}
%    \end{macrocode}

%\iffalse
%</samplechap1|samplechap2>
%\fi
%
%\iffalse
%<*samplechap1>
%\fi
% Some text for chapter 1:
%    \begin{macrocode}
\section{one}
some text in chapter one
%    \end{macrocode}

%\iffalse
%</samplechap1>
%\fi
% Some text for chapter 2:
%\iffalse
%<*samplechap2>
%\fi
%    \begin{macrocode}
\section{two}
more text in chapter two
%    \end{macrocode}

%\iffalse
%</samplechap2>
%\fi
%
% %%%%%%%%%%%%%%%%%%%%%%%%%%%%%%%%%%%%%%
% \paragraph{Part Include Files.}
%
% The include files are called |cdocspt3.tex| and |cdocspt4.tex|.
%
%\iffalse
%<*samplepart3|samplepart4>
%\fi

% Optional override for |\version| flag:
%    \begin{macrocode}
%%\providecommand{\version}{final}
%    \end{macrocode}

% Include the main document:
%    \begin{macrocode}
\input{childdoc.def}
\childdocby{cdocsamp}
%    \end{macrocode}

%\iffalse
%</samplepart3|samplepart4>
%\fi
%
%\iffalse
%<*samplepart3>
%\fi
% Some text for part 3:
%    \begin{macrocode}
some text in part three
%    \end{macrocode}

%\iffalse
%</samplepart3>
%\fi
% Some text for part 4:
%\iffalse
%<*samplepart4>
%\fi
%    \begin{macrocode}
more text in part four
%    \end{macrocode}

%\iffalse
%</samplepart4>
%\fi
%
% %%%%%%%%%%%%%%%%%%%%%%%%%%%%%%%%%%%%%%
% \paragraph{Forwarding for a Complete Draft.}
%
% The following forwarding file |cdocsdrf.tex|
% compiles the main document in draft mode:
%\iffalse
%<*sampledraft>
%\fi
%    \begin{macrocode}
\def\version{draft}
\input{childdoc.def}
\childdocforward{cdocsamp}
%    \end{macrocode}

%\iffalse
%</sampledraft>
%\fi
%
% %%%%%%%%%%%%%%%%%%%%%%%%%%%%%%%%%%%%%%
% \paragraph{Forwarding for Final Version of the Chapters.}
%
% The following forwarding files |cdocsfn1.tex| and |cdocsfn2.tex|
% (with identical content)
% compile the final versions of the child documents
% |cdocsch1.tex| and |cdocsch2.tex|, respectively:
%\iffalse
%<*samplefinal>
%\fi
%    \begin{macrocode}
\def\version{final}
\input{childdoc.def}
\childdocforwardprefix[cdocsamp]{cdocsfn}{cdocsch}
%    \end{macrocode}

%\iffalse
%</samplefinal>
%\fi
%
% %%%%%%%%%%%%%%%%%%%%%%%%%%%%%%%%%%%%%%
% \paragraph{Command Line Processing.}
%
% The following three command lines generate the output files
% |cdocscld|, |cdocscl1| and |cdocscl2|
% which should be identical to
% |cdocsdrf|, |cdocsch1| and |cdocsfn2|, respectively:
% \begin{center}
% \begin{tabular}{l}
% |latex -jobname cdocscld \|\\
% |  "\def\version{draft}\input{childdoc.def}\childdocforward{cdocsamp}"|\\
% |latex -jobname cdocscl1 \|\\
% |  "\input{childdoc.def}\childdocforward[cdocsamp]{cdocsch1}"|\\
% |latex -jobname cdocscl2 \|\\
% |  "\def\version{final}\input{childdoc.def}\childdocforward{cdocsch2}"|
% \end{tabular}
% \end{center}
% Note that the trailing backslash on each first line
% merely continues the input to the second line
% (for convenient cut ant paste).
% Furthermore, the command |latex| can be replaced by any
% of its alternative versions such as |pdflatex|.
%
% %%%%%%%%%%%%%%%%%%%%%%%%%%%%%%%%%%%%%%%%%%%%%%%%%%%%%%%%%%%%%%%%%%%%%%%%%%%%%%
% %%%%%%%%%%%%%%%%%%%%%%%%%%%%%%%%%%%%%%%%%%%%%%%%%%%%%%%%%%%%%%%%%%%%%%%%%%%%%%
% \section{Implementation}
%\iffalse
%<*package>
%\fi
%
% This section describes the definitions file |childdoc.def|.

% The definitions cannot be loaded using |\usepackage| or |\RequirePackage|
% which has a mechanism to prevent loading a style file more than once.
% When loading the definitions by means of |\input|
% multiple instances have to be prevented manually:
%\iffalse
%This code needs to be before the `\ProvidesFile' directive
%which is defined at the beginning of this file.
%Therefore it is also placed there and commented out here.
%</package>
%<*discard>
%\fi
%    \begin{macrocode}
\ifdefined\childdocmain\endinput\fi
%    \end{macrocode}
%\iffalse
%</discard>
%<*package>
%\fi
%
% \macro{\ifchilddoc}
% \macro{\ifchilddocmanual}
% The conditional |\ifchilddoc| tells whether a
% child (true) or main (false) document is being compiled.
% The conditional |\ifchilddocmanual| tells whether
% the |\includeonly| mechanism is used (false) or
% the selection of child files must be performed manually (true).
% The definitions initialise to false:
%    \begin{macrocode}
\newif\ifchilddoc
\newif\ifchilddocmanual
%    \end{macrocode}

% \macro{\childdocname}
% \macro{\childdocjob}
% The macro |\childdocname| stores the name of the main document
% to be compiled. The macro |\childdocjob| stores the name of
% the document on which the \LaTeX{} compiler was originally invoked.
% The content of |\jobname| cannot be compared
% to filenames specified in the source due to different catcodes.
% The following code rescans |\jobname|, stores the result
% in |\childdocname| and saves a copy in |\childdocjob|:
%    \begin{macrocode}
\edef\childdocname{\scantokens\expandafter{\jobname\noexpand}}
\let\childdocjob\childdocname
%    \end{macrocode}

% \macro{\childdocdisable}
% The macro |\childdocdisable| prevents the main file
% from being processed more than once.
% At this stage, the main document command |\childdocmain|
% is assumed to be called once again where it should do nothing.
% Any subsequent call to it should prevent
% a secondary processing of the main document
% It overwrites the forwarding commands
% |\childdocof| and |\childdocforward|
% with empty macros to prevent further inclusions of the main document:
%    \begin{macrocode}
\newcommand{\childdocdisable}
{
  \renewcommand{\childdocmain}[1]{\renewcommand{\childdocmain}[1]{\endinput}}
  \renewcommand{\childdocof}[1]{}
  \renewcommand{\childdocby}[2][]{}
  \renewcommand{\childdocforward}[2][]{}
  \renewcommand{\childdocdisable}{}
}
%    \end{macrocode}

% \macro{\childdocmain}
% The macro |\childdocmain| is to be called at the top of the main file
% with nothing or the main filename (without extension) as argument.
% First, it breaks loops.
% If the argument is not empty and does not match |\childdocname|
% (which is set by the first inclusion of |childdoc.def|),
% |\ifchilddoc| is set to true, |\includeonly| is applied to the child file
% and |\jobname| is set to the main file
% (for proper handling of |.aux| files):
%    \begin{macrocode}
\newcommand{\childdocmain}[1]
{
  \childdocdisable\childdocmain{}
  \if?#1?\else
    \begingroup
      \def\childdoctmp{#1}
      \ifx\childdoctmp\childdocname
        \def\childdoctmp{}
      \else
        \def\childdoctmp
        {
          \childdoctrue
          \includeonly{\childdocname}
          \def\childdocjob{#1}
          \def\jobname{#1}
        }
      \fi
      \expandafter
    \endgroup
    \childdoctmp
  \fi
}
%    \end{macrocode}

% \macro{\childdocof}
% The command |\childdocof| redirects
% compilation to the main file |#1|.
%    \begin{macrocode}
\newcommand{\childdocof}[1]
{
  \childdocdisable
  \childdoctrue
  \includeonly{\childdocname}
  \def\jobname{#1}
  \def\childdocjob{#1}
  \input{#1}
}
%    \end{macrocode}

% \macro{\childdocby}
% The command |\childdocby| ....
%    \begin{macrocode}
\newcommand{\childdocby}[2][]
{
  \childdocdisable
  \childdoctrue
  \childdocmanualtrue
  \if?#1?\else
    \def\jobname{#2}
  \fi
  \def\childdocjob{#2}
  \input{#2}
  \endinput
}
%    \end{macrocode}

% \macro{\childdocforward}
% The command |\childdocforward| redirects
% compilation to the main file or
% (if the optional argument is given) a child file.
% Parameters are set as if the main file
% or a child file starting with |\childdocof| was compiled.
% Then compilation is handed over to the main file:
%    \begin{macrocode}
\newcommand{\childdocforward}[2][]
{
  \begingroup
    \if?#1?
      \def\childdoctmp
      {
        \def\childdocname{#2}
        \def\childdocjob{#2}
        \def\jobname{#2}
        \input{#2}
        \endinput
      }
    \else
      \def\childdoctmp
      {
        \childdocdisable
        \def\childdocname{#2}
        \childdoctrue
        \includeonly{#2}
        \def\childdocjob{#1}
        \def\jobname{#1}
        \input{#1}
        \endinput
      }
    \fi
    \expandafter
  \endgroup
  \childdoctmp
}
%    \end{macrocode}

% \macro{\childdocforwardprefix}
% The command |\childdocforwardprefix| redirects
% compilation to the main or a child file by means of a pattern.
% The prefix |#1| in the current filename is replaced by |#2|
% and the suffix of the current filename is kept
% (it is assumed that the filename does not contain the substring `|~~~|'
% which is used as a delimiter).
% Compilation is handed over to the new file by |\childdocforward|:
%    \begin{macrocode}
\newcommand{\childdocforwardprefix}[3][]
{
  \begingroup
    \def\childdocextract #2##1~~~{\def\childdoctmp{\childdocforward[#1]{#3##1}}}
    \expandafter\childdocextract\childdocname~~~
    \expandafter
  \endgroup
  \childdoctmp
}
%    \end{macrocode}

% \macro{\childdoc}
% The deprecated macro |\childdoc| is a legacy version of |\childdocmain|:
%    \begin{macrocode}
\newcommand{\childdoc}{\childdocmain}
%    \end{macrocode}

% \macro{\childdocredirect}
% The deprecated macro |\childdocredirect| is a legacy version
% of |\childdocforward| and |\childdocforwardprefix|:
%    \begin{macrocode}
\newcommand{\childdocredirect}[2][]
{
  \begingroup
    \if?#1?
      \def\childdoctmp{\childdocforward{#2}}
    \else
      \def\childdoctmp{\childdocforwardprefix{#1}{#2}}
    \fi
    \expandafter
  \endgroup
  \childdoctmp
}
%    \end{macrocode}

%\iffalse
%</package>
%\fi
%
\endinput

\childdocby{cdocsamp}
%    \end{macrocode}

%\iffalse
%</samplepart3|samplepart4>
%\fi
%
%\iffalse
%<*samplepart3>
%\fi
% Some text for part 3:
%    \begin{macrocode}
some text in part three
%    \end{macrocode}

%\iffalse
%</samplepart3>
%\fi
% Some text for part 4:
%\iffalse
%<*samplepart4>
%\fi
%    \begin{macrocode}
more text in part four
%    \end{macrocode}

%\iffalse
%</samplepart4>
%\fi
%
% %%%%%%%%%%%%%%%%%%%%%%%%%%%%%%%%%%%%%%
% \paragraph{Forwarding for a Complete Draft.}
%
% The following forwarding file |cdocsdrf.tex|
% compiles the main document in draft mode:
%\iffalse
%<*sampledraft>
%\fi
%    \begin{macrocode}
\def\version{draft}
% \iffalse
%
% childdoc.dtx Copyright (C) 2017-2018 Niklas Beisert
%
% This work may be distributed and/or modified under the
% conditions of the LaTeX Project Public License, either version 1.3
% of this license or (at your option) any later version.
% The latest version of this license is in
%   http://www.latex-project.org/lppl.txt
% and version 1.3 or later is part of all distributions of LaTeX
% version 2005/12/01 or later.
%
% This work has the LPPL maintenance status `maintained'.
%
% The Current Maintainer of this work is Niklas Beisert.
%
% This work consists of the files childdoc.dtx and childdoc.ins
% and the derived files childdoc.def and cdocsamp.tex with
% cdocsch1.tex, cdocsch2.tex, cdocsdrf.tex, cdocsfn1.tex, cdocsfn2.tex.
%
%<package>\ifdefined\childdocmain\endinput\fi
%<package>\ProvidesFile{childdoc.def}[2018/12/30 v2.0 child document driver]
%<samplemain>\ProvidesFile{cdocsamp.tex}[2018/12/30 v2.0 sample for childdoc]
%<*driver>
%\ProvidesFile{childdoc.drv}[2018/12/30 v2.0 childdoc reference manual file]
\PassOptionsToClass{10pt,a4paper}{article}
\documentclass{ltxdoc}

\usepackage[margin=35mm]{geometry}
\usepackage{hyperref}
\usepackage{hyperxmp}
\usepackage[usenames]{color}

\hypersetup{colorlinks=true}
\hypersetup{pdfstartview=FitH}
\hypersetup{pdfpagemode=UseNone}
\hypersetup{pdfsource={}}
\hypersetup{pdflang={en-UK}}
\hypersetup{pdfcopyright={Copyright 2017-2018 Niklas Beisert.
  This work may be distributed and/or modified under the
  conditions of the LaTeX Project Public License, either version 1.3
  of this license or (at your option) any later version.}}
\hypersetup{pdflicenseurl={http://www.latex-project.org/lppl.txt}}
\hypersetup{pdfcontactaddress={ETH Zurich, ITP, HIT K,
  Wolfgang-Pauli-Strasse 27}}
\hypersetup{pdfcontactpostcode={8093}}
\hypersetup{pdfcontactcity={Zurich}}
\hypersetup{pdfcontactcountry={Switzerland}}
\hypersetup{pdfcontactemail={nbeisert@itp.phys.ethz.ch}}
\hypersetup{pdfcontacturl={http://people.phys.ethz.ch/\xmptilde nbeisert/}}

\newcommand{\secref}[1]{\hyperref[#1]{section \ref*{#1}}}

\parskip1ex
\parindent0pt
\let\olditemize\itemize
\def\itemize{\olditemize\parskip0pt}

\begin{document}

\title{The \textsf{childdoc} Package}
\hypersetup{pdftitle={The childdoc Package}}
\author{Niklas Beisert\\[2ex]
  Institut f\"ur Theoretische Physik\\
  Eidgen\"ossische Technische Hochschule Z\"urich\\
  Wolfgang-Pauli-Strasse 27, 8093 Z\"urich, Switzerland\\[1ex]
  \href{mailto:nbeisert@itp.phys.ethz.ch}
  {\texttt{nbeisert@itp.phys.ethz.ch}}}
\hypersetup{pdfauthor={Niklas Beisert}}
\hypersetup{pdfsubject={Manual for the LaTeX2e Package childdoc}}
\date{30 December 2018, \textsf{v2.0}}
\maketitle

\begin{abstract}\noindent
\textsf{childdoc} is a \LaTeXe{} package
that enables the direct compilation
of document sections included by |\include|
to individual files.
\end{abstract}

\begingroup
\parskip0ex
\tableofcontents
\endgroup

%%%%%%%%%%%%%%%%%%%%%%%%%%%%%%%%%%%%%%%%%%%%%%%%%%%%%%%%%%%%%%%%%%%%%%%%%%%%%%%%
%%%%%%%%%%%%%%%%%%%%%%%%%%%%%%%%%%%%%%%%%%%%%%%%%%%%%%%%%%%%%%%%%%%%%%%%%%%%%%%%
\section{Introduction}

\LaTeX{} provides a mechanism to structure a large document (such as a book)
into a main file and several child files (containing the chapters)
using the |\include| command.
This mechanism is beneficial for documents
which span hundreds of pages in order to
make the source file(s) more manageable.
Moreover, compilation can be restricted to
selected child files by means of the |\includeonly| command.
The latter feature can be used to reduce the compilation time while editing
(this was significantly more useful in the earlier days of \LaTeX{})
or to generate a smaller document which is easier to navigate.
Another application of |\includeonly| is to generate
documents consisting of selected parts of the complete document.

However, there are a few drawbacks of the plain |\include| mechanism:
\begin{itemize}
\item
The child files cannot be compiled on their own,
they can only be compiled via the main file.
A naive editing environment
(such as a text editor with an option
to have the current file processed by \LaTeX)
may require one to switch to the main file before compiling;
attempting to compile the child file produces errors.
\item
The main file must be modified (each time)
to adjust the |\includeonly| command
to the present needs. This easily leaves the main file in a messy state.
\item
The generated document will always carry the filename
of the main document. This is inconvenient if
several child files are to be compiled and
to be kept for distribution.
\end{itemize}

The present package provides a simple interface
to make child files individually compilable by \LaTeX{}.
Compiling a child file then has the same effect as compiling
the main file with an |\includeonly| command
to select the appropriate child.
Moreover the generated document will carry the name of the child
rather than the main file.
This resolves all three above issues.

This feature is meant to make the editing of books,
thesis documents and lecture notes somewhat more convenient.
However, the package can also be used efficiently for
composing a series of documents (such as exercise sheets)
which are typically distributed individually.
It then assists the author in generating the individual documents
(potentially in different versions)
as well as a document containing the collected series.
Another application is in developing style files
or other kinds of included material
where compilation of the style file could redirect
to a sample or test file.

%%%%%%%%%%%%%%%%%%%%%%%%%%%%%%%%%%%%%%%%%%%%%%%%%%%%%%%%%%%%%%%%%%%%%%%%%%%%%%%%
%%%%%%%%%%%%%%%%%%%%%%%%%%%%%%%%%%%%%%%%%%%%%%%%%%%%%%%%%%%%%%%%%%%%%%%%%%%%%%%%
\section{Usage}

First of all, the package \textsf{childdoc} is \emph{not} a standard
\LaTeXe{} |.sty| style file! Therefore it needs to be invoked in
a non-standard way.

%%%%%%%%%%%%%%%%%%%%%%%%%%%%%%%%%%%%%%%%%%%%%%%%%%%%%%%%%%%%%%%%%%%%%%%%%%%%%%%%
\subsection{Included Files}
\label{sec:include}

%%%%%%%%%%%%%%%%%%%%%%%%%%%%%%%%%%%%%%%%
\DescribeMacro{\childdocmain}
To use the package, add the commands
\begin{center}
\begin{tabular}{l}
|\input{childdoc.def}|\\
|\childdocmain{}|\\
\end{tabular}
\end{center}
at the very top of the main \LaTeX{} file,
in particular \emph{before} the |\documentclass| statement!
The argument of |\childdocmain| should be left empty
(but it must be present).

%%%%%%%%%%%%%%%%%%%%%%%%%%%%%%%%%%%%%%%%
\DescribeMacro{\childdocof}
Furthermore, add the commands
\begin{center}
\begin{tabular}{l}
|\input{childdoc.def}|\\
|\childdocof{|\textit{main}|}|\\
\end{tabular}
\end{center}
at the top of every child file \textit{child}
which is included by |\include{|\textit{child}|}|
from within the main file
(or at least for those files to be compiled individually).
The argument \textit{main} must be the filename of the main file.

There are a couple of
considerations in setting up the main and child documents:

%%%%%%%%%%%%%%%%%%%%%%%%%%%%%%%%%%%%%%%%
\paragraph{Restrictions.}

Please note the following restrictions:
\begin{itemize}
\item
|\childdocmain| must be called with one argument \textit{main}
to ensure compatibility with earlier version of the package.
It must either be empty (|\childdocmain{}|)
or precisely match the filename of the main file in which it is specified.
See \secref{sec:detection} for further information.
\item
The filename \textit{main} must be specified without the |.tex| extension.
\item
The filename \textit{main} is case sensitive
(even in case-insensitive file systems)
due to internal string comparison.
\item
The argument \textit{main} should be fully expanded, it cannot be a macro.
\item
Subdirectories and special characters should be avoided in filenames.
\item
The command |\childdocmain{|\textit{main}|}| must be followed by a whitespace.
It should not be followed immediately by another command
or by a comment mark `|%|'.
This is because the \TeX{} parser reads the token immediately following
the argument of |\childdocmain| and puts it
at the beginning of every child section;
however, a white\-space is ignored.
\end{itemize}

%%%%%%%%%%%%%%%%%%%%%%%%%%%%%%%%%%%%%%%%
\paragraph{Content of Main File.}

It is advisable to place all content in the child files included by |\include|.
Any output contained in the main file will appear in all child documents
unless suppressed manually;
it cannot be suppressed automatically by the |\includeonly| directive
and thus should normally be avoided.
A method to include some content in the main file
by means of conditional processing is described in \secref{sec:conditional}.

%%%%%%%%%%%%%%%%%%%%%%%%%%%%%%%%%%%%%%%%
\paragraph{Page Numbering.}

When only a part of the document is compiled,
the appropriate numbering of pages
(as well as other status parameters)
is determined from the |.aux| files.
The latter contain information from previous passes.
However this information needs to propagate through
all intermediate child documents.
Therefore the page numbering in child documents may well
be inconsistent until the complete document is compiled at least once.

A useful (if unconventional) way to always ensure a consistent
page numbering is to restart the numbering in each child document
and denote the pages by `\textit{child}|.|\textit{page}'
where \textit{child} represents the chapter/section number of the child file.
This can be achieved by the command
|\numberwithin{page}{|\textit{child}|}|
of the \textsf{amsmath} package
where \textit{child} can be |chapter| or |section|
depending on the chosen structuring.
Alternatively, one can modify the macro |\thepage| appropriately
and reset the counter |page| at the start of each child file.

%%%%%%%%%%%%%%%%%%%%%%%%%%%%%%%%%%%%%%%%%%%%%%%%%%%%%%%%%%%%%%%%%%%%%%%%%%%%%%%%
\subsection{Conditional Processing}
\label{sec:conditional}

The package provides a mechanism to compile different versions
of a document. To customise the versions further some conditional processing
can come in handy to distinguish which version is being compiled.
The package provides two macros to describe the compilation context:

%%%%%%%%%%%%%%%%%%%%%%%%%%%%%%%%%%%%%%%%
\DescribeMacro{\ifchilddoc}
The conditional |\ifchilddoc| distinguishes between the compilation of
child documents and the main document:
%
\begin{center}
|\ifchilddoc |\textit{child-code}| |[|\||else |\textit{main-code}]| \||fi|
\end{center}

%%%%%%%%%%%%%%%%%%%%%%%%%%%%%%%%%%%%%%%%
\DescribeMacro{\childdocname}
\DescribeMacro{\childdocjob}
The macro |\childdocname| contains the filename (without extension)
of the main or child file being processed.
Note that |\childdocjob| will always contain the name of the main file.

%%%%%%%%%%%%%%%%%%%%%%%%%%%%%%%%%%%%%%%%
\paragraph{Title Page.}

Conditional processing can be used to include a title or banner page
in the main document when proper precautions are taken.
Importantly, the code in the main file should ensure that the page counter
(as well as other status parameters which are stored in the |.aux| files)
takes the same value after the conditional processing.
Otherwise the page numbers may take divergent values
depending on which part is compiled.

For example, a title page could be declared by:
%
\begin{center}
\begin{tabular}{l}
|\ifchilddoc\||else|\\
|\addtocounter{page}{-1}|\\
\textit{code for title page}\\
|\newpage|\\
|\||fi|
\end{tabular}
\end{center}
%
A banner page for the child documents can be generated by:
%
\begin{center}
\begin{tabular}{l}
|\ifchilddoc|\\
|\addtocounter{page}{-1}|\\
\textit{code for banner page}\\
|\newpage|\\
|\||fi|
\end{tabular}
\end{center}
%
Here one could write a message such as:
\begin{center}
|This is the part \childdocname{} of \childdocjob{}.|
\end{center}

%%%%%%%%%%%%%%%%%%%%%%%%%%%%%%%%%%%%%%%%%%%%%%%%%%%%%%%%%%%%%%%%%%%%%%%%%%%%%%%%
\subsection{Flags}
\label{sec:flags}

The package makes it easy to generate different versions
of the main or child documents.
To this end compilation flags can be defined
and assigned different default values.
They will be particularly useful in conjunction
with the forwarding mechanism described in \secref{sec:forward}.

For example, it may be useful to have a flag |\version|
which can be set to |draft| or |final|.
The document source will contain some conditional code
depending on the value of |\version|.
Suppose further, the flag should default to |final| for the main file
and to |draft| for child files
which is a natural assignment for editing the document.
This is achieved by placing the following code
in the preamble of the main document
(below the |\childdocmain| directive):
%
\begin{center}
\begin{tabular}{l}
|\ifchilddoc|\\
|\providecommand{\version}{draft}|\\
|\||else|\\
|\providecommand{\version}{final}|\\
|\||fi|
\end{tabular}
\end{center}
%
The definition by |\providecommand| makes sure
that previous definitions are not overwritten.
Further statements |\providecommand{\version}{...}|
can thus be added before the above code to override it.

For the main file, one might add a line
(between |\childdocmain| and the above block)
%
\begin{center}
|%\ifchilddoc\||else\providecommand{\version}{draft}\||fi|
\end{center}
%
which can be uncommented to produce a draft version.
Likewise one can add a line to the very top of a child file
(above the |\childdocof{|\textit{main}|}| directive)
%
\begin{center}
|%\providecommand{\version}{final}|
\end{center}
%
which can be uncommented to produce the final version of this child document.

%%%%%%%%%%%%%%%%%%%%%%%%%%%%%%%%%%%%%%%%%%%%%%%%%%%%%%%%%%%%%%%%%%%%%%%%%%%%%%%%
\subsection{Forwarding}
\label{sec:forward}

Different versions of the main or child documents
using compilation flags as described in \secref{sec:flags}
can be (permanently) stored in different files
for convenient compilation, viewing and distribution.
To this end, the package defines a command
to pass on compilation to a different file:

%%%%%%%%%%%%%%%%%%%%%%%%%%%%%%%%%%%%%%%%
\DescribeMacro{\childdocforward}
The command |\childdocforward| redirects processing to
another source file:
%
\begin{center}
\begin{tabular}{l}
|\input{childdoc.def}|\\
|\childdocforward[|\textit{main}|]{|\textit{dest}|}|\\
\end{tabular}
\end{center}
%
The argument \textit{dest} is the destination file
(without extension).
It should be the main file or one of the child files.
Note that further \textsf{childdoc} directives
such as |\childdocof| and |\childdocforward|
in the indicated file will be processed in this form.
The optional argument \textit{main}
passes on directly to the main file \textit{main}
while pretending to compile the child \textit{dest}.
This form behaves as if \textit{dest}
issues |\childdocof{|\textit{main}|}| right away,
and no further \textsf{childdoc} directives will be processed.

%%%%%%%%%%%%%%%%%%%%%%%%%%%%%%%%%%%%%%%%
\DescribeMacro{\...prefix}
In the alternative form |\childdocforwardprefix|,
%
\begin{center}
\begin{tabular}{l}
|\input{childdoc.def}|\\
|\childdocforwardprefix[|\textit{main}|]{|\textit{prefix}|}{|\textit{dest}|}|
\end{tabular}
\end{center}
%
the destination file is determined by a pattern
depending on the current file:
To make this work, the current file must be called
`{\textit{prefix}\hspace{0.2em}\textit{suffix}}'
with \textit{prefix} matching precisely the argument.
Processing is then passed on to the file
`{\textit{dest}\hspace{0.2em}\textit{suffix}}'.
Surely, the same effect is achieved by
directly specifying the
argument `{\textit{dest}\hspace{0.2em}\textit{suffix}}'
in the first form.
However, that requires to set up a different file
for each child. With the alternative form of the command
all these files can have exactly the same content
which simplifies setting them up and maintaining them.

For example, the following file |draft.tex|
with a compilation flag |\version| as described in \secref{sec:flags}
compiles the main document as a draft:
%
\begin{center}
\begin{tabular}{l}
|\def\version{draft}|\\
|\input{childdoc.def}|\\
|\childdocforward{|\textit{main}|}|
\end{tabular}
\end{center}
%
Likewise, the following files |final|\textit{nn}|.tex|
compile the final version of the child document
|child|\textit{nn}|.tex|:
%
\begin{center}
\begin{tabular}{l}
|\def\version{final}|\\
|\input{childdoc.def}|\\
|\childdocforwardprefix{final}{child}|
\end{tabular}
\end{center}
%

Note that when several versions of a main file and/or of each child file
are to be generated, it may be convenient to set up a |Makefile| or
shell script to automatise the process.

%%%%%%%%%%%%%%%%%%%%%%%%%%%%%%%%%%%%%%%%%%%%%%%%%%%%%%%%%%%%%%%%%%%%%%%%%%%%%%%%
\subsection{Command Line Processing}
\label{sec:commandline}

The effect of redirection files can also be achieved by invoking
the \LaTeX{} compiler with a more elaborate command line.
Most conveniently this should be done as part
of a shell script or a |Makefile|.

When using \textsf{childdoc} in the main file, the following
command lines effectively perform a redirection
(note that depending on the shell being used,
backslashes may have to be doubled: `|\|' $\to$ `|\\|'):
%
\begin{center}
|... -jobname "|\textit{target}|" |\\|"|[\textit{flags}]%
|\input{childdoc.def}\childdocforward[|\textit{main}|]{|\textit{dest}|}"|
\end{center}
%
Here \textit{target} is the name of the output file,
\textit{main} is the name of the main file
and \textit{dest} is the name of the main or child file to be processed
(all filenames without extensions).
The optional argument \textit{main} can be omitted
if \textit{main} matches \textit{dest}.
Optionally, compilation \textit{flags} can be defined via |\def| commands.
This command line makes the \TeX{} engine believe
it is compiling the file \textit{target}
whose content is specified as the latter parameter.
The provided code then forwards the processing to
\textit{main} or \textit{dest} as described in \secref{sec:forward}.

%%%%%%%%%%%%%%%%%%%%%%%%%%%%%%%%%%%%%%%%%%%%%%%%%%%%%%%%%%%%%%%%%%%%%%%%%%%%%%%%
\subsection{Include by Input}
\label{sec:input}

Including child documents by |\include| has some restrictions by design.
Most notably, the content of a child document always occupies
its own set of pages; pages cannot be shared between child documents.
Usually, this behaviour makes perfect sense
because each child document contain an essential part of the document.
However, in some situations it may be desirable to compose
a document from a collection of parts
without having mandatory page breaks between then.
For this case, the package
provides a mechanism to include parts
by |\input| which can also be processed individually.
However, by construction this mechanism
requires manual handling of the content to be output.

%%%%%%%%%%%%%%%%%%%%%%%%%%%%%%%%%%%%%%%%
\DescribeMacro{\ifchilddocmanual}
The main file should be prepared as usual, see \secref{sec:include}.
However, the document body must make a distinction
between processing of an individual part and of the main document, e.g.:
%
\begin{center}
\begin{tabular}{l}
|\ifchilddocmanual|\\
|\input{\childdocname}|\\
|\||else|\\
\textit{document body with }|\input{|\textit{part}|}|\\
|\||fi|
\end{tabular}
\end{center}
%
The conditional |\ifchilddocmanual| is true whenever
a part to be included by |\input| is being compiled,
and the name of the part is stored in |\childdocname|.

%%%%%%%%%%%%%%%%%%%%%%%%%%%%%%%%%%%%%%%%
\DescribeMacro{\childdocby}
Each part to be included by |\input| should start with:
%
\begin{center}
\begin{tabular}{l}
|\input{childdoc.def}|\\
|\childdocby{|\textit{main}|}|\\
\end{tabular}
\end{center}
%
The directive |\childdocby| is similar to |\childdocof|
described in \secref{sec:include},
but the subsequent selection of content must be done manually.
To that end, both |\ifchilddoc| and |\ifchilddocmanual|
will be true upon processing of a part,
and the name of the part is stored in |\childdocname|.
Note that |\jobname| will be set to the filename of the current part
so that each part receives an individual |.aux| file
that does not interfere with the |.aux| file(s) of the main document.
This behaviour can be altered by the alternative form
|\childdocby[*]{|\textit{main}|}| (with a non-empty optional argument)
which uses the |.aux| file of the main document
by setting |\jobname| to \textit{main}.

%%%%%%%%%%%%%%%%%%%%%%%%%%%%%%%%%%%%%%%%%%%%%%%%%%%%%%%%%%%%%%%%%%%%%%%%%%%%%%%%
\subsection{Driver Development}
\label{sec:driver}

The \textsf{childdoc} mechanism can also be use for the development
of definition files such as \LaTeX{} styles or classes.
This case differs from the above setup with multiple parts
included by |\include| in that no |\includeonly| should be invoked.
This can be achieved by starting the include file
(before |\ProvidesPackage|) with:
%
\begin{center}
\begin{tabular}{l}
|\input{childdoc.def}|\\
|\childdocforward{|\textit{main}|}|\\
\end{tabular}
\end{center}
%
or alternatively with:
%
\begin{center}
\begin{tabular}{l}
|\input{childdoc.def}|\\
|\childdocby{|\textit{main}|}|\\
\end{tabular}
\end{center}
%
Both forms have slightly different effects as described above.
The main file is prepared as usual, see \secref{sec:include}.

%%%%%%%%%%%%%%%%%%%%%%%%%%%%%%%%%%%%%%%%%%%%%%%%%%%%%%%%%%%%%%%%%%%%%%%%%%%%%%%%
\subsection{Legacy Detection}
\label{sec:detection}

The directive |\childdocmain| in the main file can detect
whether the complete document or merely a child is to be compiled
even without using the directive |\childdocof|.
This method is deprecated because it is less robust
and there is no compelling reason to use it;
it is merely provided for backward compatibility
and it may be removed in future versions.

If the detection mechanism is to be used,
it is mandatory to correctly specify
the filename of the main file as the argument of |\childdocmain|:
%
\begin{center}
\begin{tabular}{l}
|\input{childdoc.def}|\\
|\childdocmain{|\textit{main}|}|\\
\end{tabular}
\end{center}
%
If |\jobname| does not match the argument \textit{main} of |\childdocmain|,
it is assumed that |\jobname| points to the child file to be compiled.
When using |\childdocmain| with the main file specified as argument,
it suffices to start a child file
with just |\input{|\textit{main}|}|
without loading of the package and using |\childdocof|.
If instead all processing is done
with the appropriate \textsf{childdoc} directives,
the argument of \textit{main} of |\childdocmain| can be empty.

An alternative version of the command line processing described
in \secref{sec:commandline} using the detection mechanism reads:
%
\begin{center}
|... -jobname "|\textit{target}|" "|[\textit{flags}]%
[|\def\jobname{|\textit{dest}|}|]|\input{|\textit{main}|}"|
\end{center}

%%%%%%%%%%%%%%%%%%%%%%%%%%%%%%%%%%%%%%%%%%%%%%%%%%%%%%%%%%%%%%%%%%%%%%%%%%%%%%%%
\subsection{Manual Code}
\label{sec:manual}

In case one cannot be certain whether the definitions file |childdoc.def|
is installed on the target \TeX{} distribution
and one prefers not to ship it,
it is conceivable to paste a few relevant commands into the sources.

To that end, drop all statements |\input{childdoc.def}|
and perform the replacements as outlined below.
Instead of |\childdocmain{|\textit{main}|}| add the following code
to the top of the main file:
%
\begin{center}
\begin{tabular}{l}
|\||ifdefined\childdocname\endinput\||fi\newif\ifchilddoc|\\
|\edef\childdocname{\scantokens\expandafter{\jobname\noexpand}}|\\
|\def\childdocmain{|\textit{main}|}\||ifx\childdocmain\childdocname\||else|\\
|\childdoctrue\includeonly{\childdocname}\let\jobname\childdocmain\||fi|\\
\end{tabular}
\end{center}
%
Instead of |\childdocof{|\textit{main}|}| just include the main file
at the top of each child file:
%
\begin{center}
|\input{|\textit{main}|}|
\end{center}
%
A simple redirection |\childdocforward{|\textit{dest}|}| is achieved by:
%
\begin{center}
|\def\jobname{|\textit{dest}|}\input{\jobname}|
\end{center}
%
The redirection with prefix
|\childdocforwardprefix[|\textit{prefix}|]{|\textit{dest}|}|
is accomplished by:
%
\begin{center}
\begin{tabular}{l}
|{\edef\jobname{\scantokens\expandafter{\jobname\noexpand}}|\\
|\def\redirectjob |\textit{prefix}|#1~~~{\gdef\jobname{|\textit{dest}|#1}}|\\
|\expandafter\redirectjob\jobname~~~}\input{\jobname}|
\end{tabular}
\end{center}

In an alternative approach,
child documents can be compiled by a specific command line
without additional code or specific definitions:
%
\begin{center}
|... -jobname "|\textit{target}|" "|[\textit{flags}]%
|\includeonly{|\textit{dest}|}\input{|\textit{main}|}"|
\end{center}
%

%%%%%%%%%%%%%%%%%%%%%%%%%%%%%%%%%%%%%%%%%%%%%%%%%%%%%%%%%%%%%%%%%%%%%%%%%%%%%%%%
%%%%%%%%%%%%%%%%%%%%%%%%%%%%%%%%%%%%%%%%%%%%%%%%%%%%%%%%%%%%%%%%%%%%%%%%%%%%%%%%
\section{Information}

%%%%%%%%%%%%%%%%%%%%%%%%%%%%%%%%%%%%%%%%%%%%%%%%%%%%%%%%%%%%%%%%%%%%%%%%%%%%%%%%
\subsection{Copyright}

Copyright \copyright{} 2017--2018 Niklas Beisert

This work may be distributed and/or modified under the
conditions of the \LaTeX{} Project Public License, either version 1.3
of this license or (at your option) any later version.
The latest version of this license is in
  \url{http://www.latex-project.org/lppl.txt}
and version 1.3 or later is part of all distributions of \LaTeX{}
version 2005/12/01 or later.

This work has the LPPL maintenance status `maintained'.

The Current Maintainer of this work is Niklas Beisert.

This work consists of the files |README.txt|, |childdoc.ins| and |childdoc.dtx|
as well as the derived files |childdoc.def|, |cdocsamp.tex|
with |cdocsch1.tex|, |cdocsch2.tex|, |cdocspt3.tex|, |cdocspt4.tex|,
|cdocsdrf.tex|, |cdocsfn1.tex|, |cdocsfn2.tex|
as well as |childdoc.pdf|.

%%%%%%%%%%%%%%%%%%%%%%%%%%%%%%%%%%%%%%%%%%%%%%%%%%%%%%%%%%%%%%%%%%%%%%%%%%%%%%%%
\subsection{Files and Installation}

The package consists of the files:
%
\begin{center}
\begin{tabular}{ll}
    |README.txt|   & readme file \\
    |childdoc.ins| & installation file \\
    |childdoc.dtx| & source file \\
    |childdoc.def| & definition file \\
    |cdocsamp.tex| & sample main file \\
    |cdocsch1.tex| & sample include file \\
    |cdocsch2.tex| & sample include file \\
    |cdocspt3.tex| & sample part file \\
    |cdocspt4.tex| & sample part file \\
    |cdocsdrf.tex| & sample redirection file \\
    |cdocsfn1.tex| & sample redirection file \\
    |cdocsfn2.tex| & sample redirection file \\
    |childdoc.pdf| & manual
\end{tabular}
\end{center}
%
The distribution consists of the files
|README.txt|, |childdoc.ins| and |childdoc.dtx|.
%
\begin{itemize}
\item
Run (pdf)\LaTeX{} on |childdoc.dtx|
to compile the manual |childdoc.pdf| (this file).
\item
Run \LaTeX{} on |childdoc.ins| to create the definitions file |childdoc.def|
and the sample |cdocsamp.tex| with include files
|cdocsch1.tex|, |cdocsch2.tex|, |cdocspt3.tex|, |cdocspt4.tex|,
|cdocsdrf.tex|, |cdocsfn1.tex|, |cdocsfn2.tex|.
Then copy the file |childdoc.def| to an appropriate directory of your \LaTeX{}
distribution, e.g.\ \textit{texmf-root}|/tex/latex/childdoc|.
\end{itemize}

%%%%%%%%%%%%%%%%%%%%%%%%%%%%%%%%%%%%%%%%%%%%%%%%%%%%%%%%%%%%%%%%%%%%%%%%%%%%%%%%
\subsection{Related CTAN Packages}

There are several other packages which offer a similar functionality:
%
\begin{itemize}
\item
The packages
\href{http://ctan.org/pkg/docmute}{\textsf{docmute}},
\href{http://ctan.org/pkg/includex}{\textsf{includex}} and
\href{http://ctan.org/pkg/standalone}{\textsf{standalone}}
provide commands to include only the document body of
a child file thus allowing both files to be compiled individually.
\item
The packages \href{http://ctan.org/pkg/subdocs}{\textsf{subdocs}}
and \href{http://ctan.org/pkg/subfiles}{\textsf{subfiles}}
provide structures in which the main and child documents can be
encapsulated and allowing them to be compiled individually.
The inclusion mechanism is different from the conventional |\include|.
\item
The package \href{http://ctan.org/pkg/combine}{\textsf{combine}}
is an elaborate solution to combine several documents into one.
\end{itemize}
%
See also the CTAN topic \href{http://ctan.org/topic/subdocs}{\textsf{subdocs}}
for further related packages.
The present package differs from the above solutions in that
a document structure constructed with the conventional |\include| mechanism
just needs two extra commands at the top of every file
such that all constituent files can be compiled individually.

%%%%%%%%%%%%%%%%%%%%%%%%%%%%%%%%%%%%%%%%%%%%%%%%%%%%%%%%%%%%%%%%%%%%%%%%%%%%%%%%
%\subsection{Feature Suggestions}
%
%The following is a list of features which may be useful for future
%versions of this package:
%%
%\begin{itemize}
%\item
%\ldots
%\end{itemize}

%%%%%%%%%%%%%%%%%%%%%%%%%%%%%%%%%%%%%%%%%%%%%%%%%%%%%%%%%%%%%%%%%%%%%%%%%%%%%%%%
\subsection{Revision History}

%%%%%%%%%%%%%%%%%%%%%%%%%%%%%%%%%%%%%%%%
\paragraph{v2.0:} 2018/12/30

\begin{itemize}
\item
immediate forward processing
\item
added |\childdocby| mechanism
\item
manual restructured
\end{itemize}

%%%%%%%%%%%%%%%%%%%%%%%%%%%%%%%%%%%%%%%%
\paragraph{v1.6:} 2018/01/17

\begin{itemize}
\item
application for development of include files
\item
corrections to manual
\end{itemize}

%%%%%%%%%%%%%%%%%%%%%%%%%%%%%%%%%%%%%%%%
\paragraph{v1.5:} 2017/05/21

\begin{itemize}
\item
more complete structuring introduced
\item
|\childdocof| introduced
\item
|\childdoc| renamed to |\childdocmain|
\item
|\childredirect| renamed to |\childdocforward| and |\childdocforwardprefix|
and functionality expanded
\end{itemize}

%%%%%%%%%%%%%%%%%%%%%%%%%%%%%%%%%%%%%%%%
\paragraph{v1.0:} 2017/04/27

\begin{itemize}
\item
manual and install package
\item
first version published on CTAN
\end{itemize}

%%%%%%%%%%%%%%%%%%%%%%%%%%%%%%%%%%%%%%%%
\paragraph{v0.6:} 2017/04/26

\begin{itemize}
\item
redirection mechanism added
\end{itemize}

%%%%%%%%%%%%%%%%%%%%%%%%%%%%%%%%%%%%%%%%
\paragraph{v0.5:} 2017/04/26

\begin{itemize}
\item
functionality in definition file
\end{itemize}


%%%%%%%%%%%%%%%%%%%%%%%%%%%%%%%%%%%%%%%%%%%%%%%%%%%%%%%%%%%%%%%%%%%%%%%%%%%%%%%%
%%%%%%%%%%%%%%%%%%%%%%%%%%%%%%%%%%%%%%%%%%%%%%%%%%%%%%%%%%%%%%%%%%%%%%%%%%%%%%%%
%%%%%%%%%%%%%%%%%%%%%%%%%%%%%%%%%%%%%%%%%%%%%%%%%%%%%%%%%%%%%%%%%%%%%%%%%%%%%%%%
\appendix

\settowidth\MacroIndent{\rmfamily\scriptsize 000\ }

 \DocInput{childdoc.dtx}

\end{document}
%</driver>
% \fi
%
% %%%%%%%%%%%%%%%%%%%%%%%%%%%%%%%%%%%%%%%%%%%%%%%%%%%%%%%%%%%%%%%%%%%%%%%%%%%%%%
% %%%%%%%%%%%%%%%%%%%%%%%%%%%%%%%%%%%%%%%%%%%%%%%%%%%%%%%%%%%%%%%%%%%%%%%%%%%%%%
% \section{Sample}
%\iffalse
%<*samplemain>
%\fi
%
% The following presents a sample document
% with two chapters, two parts, a title page,
% a compile flag as well as three forwarding files to set the flag.
% It consists of eight |.tex| files:
% \begin{center}
% \begin{tabular}{ll}
% |cdocsamp.tex|&main file\\
% |cdocsch1.tex|&include file for chapter 1\\
% |cdocsch2.tex|&include file for chapter 2\\
% |cdocspt3.tex|&include file for part 3\\
% |cdocspt4.tex|&include file for part 4\\
% |cdocsdrf.tex|&forwarding file for main file in draft mode\\
% |cdocsfi1.tex|&forwarding file for final version of chapter 1\\
% |cdocsfi2.tex|&forwarding file for final version of chapter 2\\
% \end{tabular}
% \end{center}
% Each of the eight files can be compiled directly by the \LaTeX{} compiler.
%
% %%%%%%%%%%%%%%%%%%%%%%%%%%%%%%%%%%%%%%
% \paragraph{Main File.}
%
% The main file is called |cdocsamp.tex|.
%
% Load the \textsf{childdoc} definitions and
% declare the filename for the main document:
%    \begin{macrocode}
\input{childdoc.def}
\childdocmain{}
%    \end{macrocode}

% Optional override for |\version| flag:
%    \begin{macrocode}
%%\ifchilddoc\else\providecommand{\version}{draft}\fi
%    \end{macrocode}

% Define the default values for the |\version| flag
% (|final| for the main file and |draft| for childs):
%    \begin{macrocode}
\ifchilddoc
\providecommand{\version}{draft}
\else
\providecommand{\version}{final}
\fi
%    \end{macrocode}

% Load the standard document class:
%    \begin{macrocode}
\documentclass[12pt]{article}
%    \end{macrocode}

% Start the document body:
%    \begin{macrocode}
\begin{document}
%    \end{macrocode}

% Declare a title page.
% Print title, part of document being processed and version flag:
%    \begin{macrocode}
\addtocounter{page}{-1}
\begin{center}
{\LARGE\bfseries{}childdoc example\par}
\vspace{1cm}
\ifchilddoc
\ifchilddocmanual part\else chapter\fi:
`\childdocname' of `\childdocjob'\par
\else
main document: `\childdocjob'\par
\fi
version: \version\par
\end{center}
\newpage
%    \end{macrocode}

% Manually include selected file,
% otherwise process as usual:
%    \begin{macrocode}
\ifchilddocmanual
\section*{part `\childdocname'}
\input{\childdocname}
\else
%    \end{macrocode}

% Include the two chapters:
%    \begin{macrocode}
\include{cdocsch1}
\include{cdocsch2}
%    \end{macrocode}

% Include the two parts unless only chapters should be displayed:
%    \begin{macrocode}
\ifchilddoc\else
\section{part three}
\input{cdocspt3}
\section{part four}
\input{cdocspt4}
\fi
%    \end{macrocode}

% Process as usual until here:
%    \begin{macrocode}
\fi
%    \end{macrocode}

% End of document body:
%    \begin{macrocode}
\end{document}
%    \end{macrocode}
%\iffalse
%</samplemain>
%\fi
%
% %%%%%%%%%%%%%%%%%%%%%%%%%%%%%%%%%%%%%%
% \paragraph{Chapter Include Files.}
%
% The include files are called |cdocsch1.tex| and |cdocsch2.tex|.
%
%\iffalse
%<*samplechap1|samplechap2>
%\fi

% Optional override for |\version| flag:
%    \begin{macrocode}
%%\providecommand{\version}{final}
%    \end{macrocode}

% Include the main document:
%    \begin{macrocode}
\input{childdoc.def}
\childdocof{cdocsamp}
%    \end{macrocode}

%\iffalse
%</samplechap1|samplechap2>
%\fi
%
%\iffalse
%<*samplechap1>
%\fi
% Some text for chapter 1:
%    \begin{macrocode}
\section{one}
some text in chapter one
%    \end{macrocode}

%\iffalse
%</samplechap1>
%\fi
% Some text for chapter 2:
%\iffalse
%<*samplechap2>
%\fi
%    \begin{macrocode}
\section{two}
more text in chapter two
%    \end{macrocode}

%\iffalse
%</samplechap2>
%\fi
%
% %%%%%%%%%%%%%%%%%%%%%%%%%%%%%%%%%%%%%%
% \paragraph{Part Include Files.}
%
% The include files are called |cdocspt3.tex| and |cdocspt4.tex|.
%
%\iffalse
%<*samplepart3|samplepart4>
%\fi

% Optional override for |\version| flag:
%    \begin{macrocode}
%%\providecommand{\version}{final}
%    \end{macrocode}

% Include the main document:
%    \begin{macrocode}
\input{childdoc.def}
\childdocby{cdocsamp}
%    \end{macrocode}

%\iffalse
%</samplepart3|samplepart4>
%\fi
%
%\iffalse
%<*samplepart3>
%\fi
% Some text for part 3:
%    \begin{macrocode}
some text in part three
%    \end{macrocode}

%\iffalse
%</samplepart3>
%\fi
% Some text for part 4:
%\iffalse
%<*samplepart4>
%\fi
%    \begin{macrocode}
more text in part four
%    \end{macrocode}

%\iffalse
%</samplepart4>
%\fi
%
% %%%%%%%%%%%%%%%%%%%%%%%%%%%%%%%%%%%%%%
% \paragraph{Forwarding for a Complete Draft.}
%
% The following forwarding file |cdocsdrf.tex|
% compiles the main document in draft mode:
%\iffalse
%<*sampledraft>
%\fi
%    \begin{macrocode}
\def\version{draft}
\input{childdoc.def}
\childdocforward{cdocsamp}
%    \end{macrocode}

%\iffalse
%</sampledraft>
%\fi
%
% %%%%%%%%%%%%%%%%%%%%%%%%%%%%%%%%%%%%%%
% \paragraph{Forwarding for Final Version of the Chapters.}
%
% The following forwarding files |cdocsfn1.tex| and |cdocsfn2.tex|
% (with identical content)
% compile the final versions of the child documents
% |cdocsch1.tex| and |cdocsch2.tex|, respectively:
%\iffalse
%<*samplefinal>
%\fi
%    \begin{macrocode}
\def\version{final}
\input{childdoc.def}
\childdocforwardprefix[cdocsamp]{cdocsfn}{cdocsch}
%    \end{macrocode}

%\iffalse
%</samplefinal>
%\fi
%
% %%%%%%%%%%%%%%%%%%%%%%%%%%%%%%%%%%%%%%
% \paragraph{Command Line Processing.}
%
% The following three command lines generate the output files
% |cdocscld|, |cdocscl1| and |cdocscl2|
% which should be identical to
% |cdocsdrf|, |cdocsch1| and |cdocsfn2|, respectively:
% \begin{center}
% \begin{tabular}{l}
% |latex -jobname cdocscld \|\\
% |  "\def\version{draft}\input{childdoc.def}\childdocforward{cdocsamp}"|\\
% |latex -jobname cdocscl1 \|\\
% |  "\input{childdoc.def}\childdocforward[cdocsamp]{cdocsch1}"|\\
% |latex -jobname cdocscl2 \|\\
% |  "\def\version{final}\input{childdoc.def}\childdocforward{cdocsch2}"|
% \end{tabular}
% \end{center}
% Note that the trailing backslash on each first line
% merely continues the input to the second line
% (for convenient cut ant paste).
% Furthermore, the command |latex| can be replaced by any
% of its alternative versions such as |pdflatex|.
%
% %%%%%%%%%%%%%%%%%%%%%%%%%%%%%%%%%%%%%%%%%%%%%%%%%%%%%%%%%%%%%%%%%%%%%%%%%%%%%%
% %%%%%%%%%%%%%%%%%%%%%%%%%%%%%%%%%%%%%%%%%%%%%%%%%%%%%%%%%%%%%%%%%%%%%%%%%%%%%%
% \section{Implementation}
%\iffalse
%<*package>
%\fi
%
% This section describes the definitions file |childdoc.def|.

% The definitions cannot be loaded using |\usepackage| or |\RequirePackage|
% which has a mechanism to prevent loading a style file more than once.
% When loading the definitions by means of |\input|
% multiple instances have to be prevented manually:
%\iffalse
%This code needs to be before the `\ProvidesFile' directive
%which is defined at the beginning of this file.
%Therefore it is also placed there and commented out here.
%</package>
%<*discard>
%\fi
%    \begin{macrocode}
\ifdefined\childdocmain\endinput\fi
%    \end{macrocode}
%\iffalse
%</discard>
%<*package>
%\fi
%
% \macro{\ifchilddoc}
% \macro{\ifchilddocmanual}
% The conditional |\ifchilddoc| tells whether a
% child (true) or main (false) document is being compiled.
% The conditional |\ifchilddocmanual| tells whether
% the |\includeonly| mechanism is used (false) or
% the selection of child files must be performed manually (true).
% The definitions initialise to false:
%    \begin{macrocode}
\newif\ifchilddoc
\newif\ifchilddocmanual
%    \end{macrocode}

% \macro{\childdocname}
% \macro{\childdocjob}
% The macro |\childdocname| stores the name of the main document
% to be compiled. The macro |\childdocjob| stores the name of
% the document on which the \LaTeX{} compiler was originally invoked.
% The content of |\jobname| cannot be compared
% to filenames specified in the source due to different catcodes.
% The following code rescans |\jobname|, stores the result
% in |\childdocname| and saves a copy in |\childdocjob|:
%    \begin{macrocode}
\edef\childdocname{\scantokens\expandafter{\jobname\noexpand}}
\let\childdocjob\childdocname
%    \end{macrocode}

% \macro{\childdocdisable}
% The macro |\childdocdisable| prevents the main file
% from being processed more than once.
% At this stage, the main document command |\childdocmain|
% is assumed to be called once again where it should do nothing.
% Any subsequent call to it should prevent
% a secondary processing of the main document
% It overwrites the forwarding commands
% |\childdocof| and |\childdocforward|
% with empty macros to prevent further inclusions of the main document:
%    \begin{macrocode}
\newcommand{\childdocdisable}
{
  \renewcommand{\childdocmain}[1]{\renewcommand{\childdocmain}[1]{\endinput}}
  \renewcommand{\childdocof}[1]{}
  \renewcommand{\childdocby}[2][]{}
  \renewcommand{\childdocforward}[2][]{}
  \renewcommand{\childdocdisable}{}
}
%    \end{macrocode}

% \macro{\childdocmain}
% The macro |\childdocmain| is to be called at the top of the main file
% with nothing or the main filename (without extension) as argument.
% First, it breaks loops.
% If the argument is not empty and does not match |\childdocname|
% (which is set by the first inclusion of |childdoc.def|),
% |\ifchilddoc| is set to true, |\includeonly| is applied to the child file
% and |\jobname| is set to the main file
% (for proper handling of |.aux| files):
%    \begin{macrocode}
\newcommand{\childdocmain}[1]
{
  \childdocdisable\childdocmain{}
  \if?#1?\else
    \begingroup
      \def\childdoctmp{#1}
      \ifx\childdoctmp\childdocname
        \def\childdoctmp{}
      \else
        \def\childdoctmp
        {
          \childdoctrue
          \includeonly{\childdocname}
          \def\childdocjob{#1}
          \def\jobname{#1}
        }
      \fi
      \expandafter
    \endgroup
    \childdoctmp
  \fi
}
%    \end{macrocode}

% \macro{\childdocof}
% The command |\childdocof| redirects
% compilation to the main file |#1|.
%    \begin{macrocode}
\newcommand{\childdocof}[1]
{
  \childdocdisable
  \childdoctrue
  \includeonly{\childdocname}
  \def\jobname{#1}
  \def\childdocjob{#1}
  \input{#1}
}
%    \end{macrocode}

% \macro{\childdocby}
% The command |\childdocby| ....
%    \begin{macrocode}
\newcommand{\childdocby}[2][]
{
  \childdocdisable
  \childdoctrue
  \childdocmanualtrue
  \if?#1?\else
    \def\jobname{#2}
  \fi
  \def\childdocjob{#2}
  \input{#2}
  \endinput
}
%    \end{macrocode}

% \macro{\childdocforward}
% The command |\childdocforward| redirects
% compilation to the main file or
% (if the optional argument is given) a child file.
% Parameters are set as if the main file
% or a child file starting with |\childdocof| was compiled.
% Then compilation is handed over to the main file:
%    \begin{macrocode}
\newcommand{\childdocforward}[2][]
{
  \begingroup
    \if?#1?
      \def\childdoctmp
      {
        \def\childdocname{#2}
        \def\childdocjob{#2}
        \def\jobname{#2}
        \input{#2}
        \endinput
      }
    \else
      \def\childdoctmp
      {
        \childdocdisable
        \def\childdocname{#2}
        \childdoctrue
        \includeonly{#2}
        \def\childdocjob{#1}
        \def\jobname{#1}
        \input{#1}
        \endinput
      }
    \fi
    \expandafter
  \endgroup
  \childdoctmp
}
%    \end{macrocode}

% \macro{\childdocforwardprefix}
% The command |\childdocforwardprefix| redirects
% compilation to the main or a child file by means of a pattern.
% The prefix |#1| in the current filename is replaced by |#2|
% and the suffix of the current filename is kept
% (it is assumed that the filename does not contain the substring `|~~~|'
% which is used as a delimiter).
% Compilation is handed over to the new file by |\childdocforward|:
%    \begin{macrocode}
\newcommand{\childdocforwardprefix}[3][]
{
  \begingroup
    \def\childdocextract #2##1~~~{\def\childdoctmp{\childdocforward[#1]{#3##1}}}
    \expandafter\childdocextract\childdocname~~~
    \expandafter
  \endgroup
  \childdoctmp
}
%    \end{macrocode}

% \macro{\childdoc}
% The deprecated macro |\childdoc| is a legacy version of |\childdocmain|:
%    \begin{macrocode}
\newcommand{\childdoc}{\childdocmain}
%    \end{macrocode}

% \macro{\childdocredirect}
% The deprecated macro |\childdocredirect| is a legacy version
% of |\childdocforward| and |\childdocforwardprefix|:
%    \begin{macrocode}
\newcommand{\childdocredirect}[2][]
{
  \begingroup
    \if?#1?
      \def\childdoctmp{\childdocforward{#2}}
    \else
      \def\childdoctmp{\childdocforwardprefix{#1}{#2}}
    \fi
    \expandafter
  \endgroup
  \childdoctmp
}
%    \end{macrocode}

%\iffalse
%</package>
%\fi
%
\endinput

\childdocforward{cdocsamp}
%    \end{macrocode}

%\iffalse
%</sampledraft>
%\fi
%
% %%%%%%%%%%%%%%%%%%%%%%%%%%%%%%%%%%%%%%
% \paragraph{Forwarding for Final Version of the Chapters.}
%
% The following forwarding files |cdocsfn1.tex| and |cdocsfn2.tex|
% (with identical content)
% compile the final versions of the child documents
% |cdocsch1.tex| and |cdocsch2.tex|, respectively:
%\iffalse
%<*samplefinal>
%\fi
%    \begin{macrocode}
\def\version{final}
% \iffalse
%
% childdoc.dtx Copyright (C) 2017-2018 Niklas Beisert
%
% This work may be distributed and/or modified under the
% conditions of the LaTeX Project Public License, either version 1.3
% of this license or (at your option) any later version.
% The latest version of this license is in
%   http://www.latex-project.org/lppl.txt
% and version 1.3 or later is part of all distributions of LaTeX
% version 2005/12/01 or later.
%
% This work has the LPPL maintenance status `maintained'.
%
% The Current Maintainer of this work is Niklas Beisert.
%
% This work consists of the files childdoc.dtx and childdoc.ins
% and the derived files childdoc.def and cdocsamp.tex with
% cdocsch1.tex, cdocsch2.tex, cdocsdrf.tex, cdocsfn1.tex, cdocsfn2.tex.
%
%<package>\ifdefined\childdocmain\endinput\fi
%<package>\ProvidesFile{childdoc.def}[2018/12/30 v2.0 child document driver]
%<samplemain>\ProvidesFile{cdocsamp.tex}[2018/12/30 v2.0 sample for childdoc]
%<*driver>
%\ProvidesFile{childdoc.drv}[2018/12/30 v2.0 childdoc reference manual file]
\PassOptionsToClass{10pt,a4paper}{article}
\documentclass{ltxdoc}

\usepackage[margin=35mm]{geometry}
\usepackage{hyperref}
\usepackage{hyperxmp}
\usepackage[usenames]{color}

\hypersetup{colorlinks=true}
\hypersetup{pdfstartview=FitH}
\hypersetup{pdfpagemode=UseNone}
\hypersetup{pdfsource={}}
\hypersetup{pdflang={en-UK}}
\hypersetup{pdfcopyright={Copyright 2017-2018 Niklas Beisert.
  This work may be distributed and/or modified under the
  conditions of the LaTeX Project Public License, either version 1.3
  of this license or (at your option) any later version.}}
\hypersetup{pdflicenseurl={http://www.latex-project.org/lppl.txt}}
\hypersetup{pdfcontactaddress={ETH Zurich, ITP, HIT K,
  Wolfgang-Pauli-Strasse 27}}
\hypersetup{pdfcontactpostcode={8093}}
\hypersetup{pdfcontactcity={Zurich}}
\hypersetup{pdfcontactcountry={Switzerland}}
\hypersetup{pdfcontactemail={nbeisert@itp.phys.ethz.ch}}
\hypersetup{pdfcontacturl={http://people.phys.ethz.ch/\xmptilde nbeisert/}}

\newcommand{\secref}[1]{\hyperref[#1]{section \ref*{#1}}}

\parskip1ex
\parindent0pt
\let\olditemize\itemize
\def\itemize{\olditemize\parskip0pt}

\begin{document}

\title{The \textsf{childdoc} Package}
\hypersetup{pdftitle={The childdoc Package}}
\author{Niklas Beisert\\[2ex]
  Institut f\"ur Theoretische Physik\\
  Eidgen\"ossische Technische Hochschule Z\"urich\\
  Wolfgang-Pauli-Strasse 27, 8093 Z\"urich, Switzerland\\[1ex]
  \href{mailto:nbeisert@itp.phys.ethz.ch}
  {\texttt{nbeisert@itp.phys.ethz.ch}}}
\hypersetup{pdfauthor={Niklas Beisert}}
\hypersetup{pdfsubject={Manual for the LaTeX2e Package childdoc}}
\date{30 December 2018, \textsf{v2.0}}
\maketitle

\begin{abstract}\noindent
\textsf{childdoc} is a \LaTeXe{} package
that enables the direct compilation
of document sections included by |\include|
to individual files.
\end{abstract}

\begingroup
\parskip0ex
\tableofcontents
\endgroup

%%%%%%%%%%%%%%%%%%%%%%%%%%%%%%%%%%%%%%%%%%%%%%%%%%%%%%%%%%%%%%%%%%%%%%%%%%%%%%%%
%%%%%%%%%%%%%%%%%%%%%%%%%%%%%%%%%%%%%%%%%%%%%%%%%%%%%%%%%%%%%%%%%%%%%%%%%%%%%%%%
\section{Introduction}

\LaTeX{} provides a mechanism to structure a large document (such as a book)
into a main file and several child files (containing the chapters)
using the |\include| command.
This mechanism is beneficial for documents
which span hundreds of pages in order to
make the source file(s) more manageable.
Moreover, compilation can be restricted to
selected child files by means of the |\includeonly| command.
The latter feature can be used to reduce the compilation time while editing
(this was significantly more useful in the earlier days of \LaTeX{})
or to generate a smaller document which is easier to navigate.
Another application of |\includeonly| is to generate
documents consisting of selected parts of the complete document.

However, there are a few drawbacks of the plain |\include| mechanism:
\begin{itemize}
\item
The child files cannot be compiled on their own,
they can only be compiled via the main file.
A naive editing environment
(such as a text editor with an option
to have the current file processed by \LaTeX)
may require one to switch to the main file before compiling;
attempting to compile the child file produces errors.
\item
The main file must be modified (each time)
to adjust the |\includeonly| command
to the present needs. This easily leaves the main file in a messy state.
\item
The generated document will always carry the filename
of the main document. This is inconvenient if
several child files are to be compiled and
to be kept for distribution.
\end{itemize}

The present package provides a simple interface
to make child files individually compilable by \LaTeX{}.
Compiling a child file then has the same effect as compiling
the main file with an |\includeonly| command
to select the appropriate child.
Moreover the generated document will carry the name of the child
rather than the main file.
This resolves all three above issues.

This feature is meant to make the editing of books,
thesis documents and lecture notes somewhat more convenient.
However, the package can also be used efficiently for
composing a series of documents (such as exercise sheets)
which are typically distributed individually.
It then assists the author in generating the individual documents
(potentially in different versions)
as well as a document containing the collected series.
Another application is in developing style files
or other kinds of included material
where compilation of the style file could redirect
to a sample or test file.

%%%%%%%%%%%%%%%%%%%%%%%%%%%%%%%%%%%%%%%%%%%%%%%%%%%%%%%%%%%%%%%%%%%%%%%%%%%%%%%%
%%%%%%%%%%%%%%%%%%%%%%%%%%%%%%%%%%%%%%%%%%%%%%%%%%%%%%%%%%%%%%%%%%%%%%%%%%%%%%%%
\section{Usage}

First of all, the package \textsf{childdoc} is \emph{not} a standard
\LaTeXe{} |.sty| style file! Therefore it needs to be invoked in
a non-standard way.

%%%%%%%%%%%%%%%%%%%%%%%%%%%%%%%%%%%%%%%%%%%%%%%%%%%%%%%%%%%%%%%%%%%%%%%%%%%%%%%%
\subsection{Included Files}
\label{sec:include}

%%%%%%%%%%%%%%%%%%%%%%%%%%%%%%%%%%%%%%%%
\DescribeMacro{\childdocmain}
To use the package, add the commands
\begin{center}
\begin{tabular}{l}
|\input{childdoc.def}|\\
|\childdocmain{}|\\
\end{tabular}
\end{center}
at the very top of the main \LaTeX{} file,
in particular \emph{before} the |\documentclass| statement!
The argument of |\childdocmain| should be left empty
(but it must be present).

%%%%%%%%%%%%%%%%%%%%%%%%%%%%%%%%%%%%%%%%
\DescribeMacro{\childdocof}
Furthermore, add the commands
\begin{center}
\begin{tabular}{l}
|\input{childdoc.def}|\\
|\childdocof{|\textit{main}|}|\\
\end{tabular}
\end{center}
at the top of every child file \textit{child}
which is included by |\include{|\textit{child}|}|
from within the main file
(or at least for those files to be compiled individually).
The argument \textit{main} must be the filename of the main file.

There are a couple of
considerations in setting up the main and child documents:

%%%%%%%%%%%%%%%%%%%%%%%%%%%%%%%%%%%%%%%%
\paragraph{Restrictions.}

Please note the following restrictions:
\begin{itemize}
\item
|\childdocmain| must be called with one argument \textit{main}
to ensure compatibility with earlier version of the package.
It must either be empty (|\childdocmain{}|)
or precisely match the filename of the main file in which it is specified.
See \secref{sec:detection} for further information.
\item
The filename \textit{main} must be specified without the |.tex| extension.
\item
The filename \textit{main} is case sensitive
(even in case-insensitive file systems)
due to internal string comparison.
\item
The argument \textit{main} should be fully expanded, it cannot be a macro.
\item
Subdirectories and special characters should be avoided in filenames.
\item
The command |\childdocmain{|\textit{main}|}| must be followed by a whitespace.
It should not be followed immediately by another command
or by a comment mark `|%|'.
This is because the \TeX{} parser reads the token immediately following
the argument of |\childdocmain| and puts it
at the beginning of every child section;
however, a white\-space is ignored.
\end{itemize}

%%%%%%%%%%%%%%%%%%%%%%%%%%%%%%%%%%%%%%%%
\paragraph{Content of Main File.}

It is advisable to place all content in the child files included by |\include|.
Any output contained in the main file will appear in all child documents
unless suppressed manually;
it cannot be suppressed automatically by the |\includeonly| directive
and thus should normally be avoided.
A method to include some content in the main file
by means of conditional processing is described in \secref{sec:conditional}.

%%%%%%%%%%%%%%%%%%%%%%%%%%%%%%%%%%%%%%%%
\paragraph{Page Numbering.}

When only a part of the document is compiled,
the appropriate numbering of pages
(as well as other status parameters)
is determined from the |.aux| files.
The latter contain information from previous passes.
However this information needs to propagate through
all intermediate child documents.
Therefore the page numbering in child documents may well
be inconsistent until the complete document is compiled at least once.

A useful (if unconventional) way to always ensure a consistent
page numbering is to restart the numbering in each child document
and denote the pages by `\textit{child}|.|\textit{page}'
where \textit{child} represents the chapter/section number of the child file.
This can be achieved by the command
|\numberwithin{page}{|\textit{child}|}|
of the \textsf{amsmath} package
where \textit{child} can be |chapter| or |section|
depending on the chosen structuring.
Alternatively, one can modify the macro |\thepage| appropriately
and reset the counter |page| at the start of each child file.

%%%%%%%%%%%%%%%%%%%%%%%%%%%%%%%%%%%%%%%%%%%%%%%%%%%%%%%%%%%%%%%%%%%%%%%%%%%%%%%%
\subsection{Conditional Processing}
\label{sec:conditional}

The package provides a mechanism to compile different versions
of a document. To customise the versions further some conditional processing
can come in handy to distinguish which version is being compiled.
The package provides two macros to describe the compilation context:

%%%%%%%%%%%%%%%%%%%%%%%%%%%%%%%%%%%%%%%%
\DescribeMacro{\ifchilddoc}
The conditional |\ifchilddoc| distinguishes between the compilation of
child documents and the main document:
%
\begin{center}
|\ifchilddoc |\textit{child-code}| |[|\||else |\textit{main-code}]| \||fi|
\end{center}

%%%%%%%%%%%%%%%%%%%%%%%%%%%%%%%%%%%%%%%%
\DescribeMacro{\childdocname}
\DescribeMacro{\childdocjob}
The macro |\childdocname| contains the filename (without extension)
of the main or child file being processed.
Note that |\childdocjob| will always contain the name of the main file.

%%%%%%%%%%%%%%%%%%%%%%%%%%%%%%%%%%%%%%%%
\paragraph{Title Page.}

Conditional processing can be used to include a title or banner page
in the main document when proper precautions are taken.
Importantly, the code in the main file should ensure that the page counter
(as well as other status parameters which are stored in the |.aux| files)
takes the same value after the conditional processing.
Otherwise the page numbers may take divergent values
depending on which part is compiled.

For example, a title page could be declared by:
%
\begin{center}
\begin{tabular}{l}
|\ifchilddoc\||else|\\
|\addtocounter{page}{-1}|\\
\textit{code for title page}\\
|\newpage|\\
|\||fi|
\end{tabular}
\end{center}
%
A banner page for the child documents can be generated by:
%
\begin{center}
\begin{tabular}{l}
|\ifchilddoc|\\
|\addtocounter{page}{-1}|\\
\textit{code for banner page}\\
|\newpage|\\
|\||fi|
\end{tabular}
\end{center}
%
Here one could write a message such as:
\begin{center}
|This is the part \childdocname{} of \childdocjob{}.|
\end{center}

%%%%%%%%%%%%%%%%%%%%%%%%%%%%%%%%%%%%%%%%%%%%%%%%%%%%%%%%%%%%%%%%%%%%%%%%%%%%%%%%
\subsection{Flags}
\label{sec:flags}

The package makes it easy to generate different versions
of the main or child documents.
To this end compilation flags can be defined
and assigned different default values.
They will be particularly useful in conjunction
with the forwarding mechanism described in \secref{sec:forward}.

For example, it may be useful to have a flag |\version|
which can be set to |draft| or |final|.
The document source will contain some conditional code
depending on the value of |\version|.
Suppose further, the flag should default to |final| for the main file
and to |draft| for child files
which is a natural assignment for editing the document.
This is achieved by placing the following code
in the preamble of the main document
(below the |\childdocmain| directive):
%
\begin{center}
\begin{tabular}{l}
|\ifchilddoc|\\
|\providecommand{\version}{draft}|\\
|\||else|\\
|\providecommand{\version}{final}|\\
|\||fi|
\end{tabular}
\end{center}
%
The definition by |\providecommand| makes sure
that previous definitions are not overwritten.
Further statements |\providecommand{\version}{...}|
can thus be added before the above code to override it.

For the main file, one might add a line
(between |\childdocmain| and the above block)
%
\begin{center}
|%\ifchilddoc\||else\providecommand{\version}{draft}\||fi|
\end{center}
%
which can be uncommented to produce a draft version.
Likewise one can add a line to the very top of a child file
(above the |\childdocof{|\textit{main}|}| directive)
%
\begin{center}
|%\providecommand{\version}{final}|
\end{center}
%
which can be uncommented to produce the final version of this child document.

%%%%%%%%%%%%%%%%%%%%%%%%%%%%%%%%%%%%%%%%%%%%%%%%%%%%%%%%%%%%%%%%%%%%%%%%%%%%%%%%
\subsection{Forwarding}
\label{sec:forward}

Different versions of the main or child documents
using compilation flags as described in \secref{sec:flags}
can be (permanently) stored in different files
for convenient compilation, viewing and distribution.
To this end, the package defines a command
to pass on compilation to a different file:

%%%%%%%%%%%%%%%%%%%%%%%%%%%%%%%%%%%%%%%%
\DescribeMacro{\childdocforward}
The command |\childdocforward| redirects processing to
another source file:
%
\begin{center}
\begin{tabular}{l}
|\input{childdoc.def}|\\
|\childdocforward[|\textit{main}|]{|\textit{dest}|}|\\
\end{tabular}
\end{center}
%
The argument \textit{dest} is the destination file
(without extension).
It should be the main file or one of the child files.
Note that further \textsf{childdoc} directives
such as |\childdocof| and |\childdocforward|
in the indicated file will be processed in this form.
The optional argument \textit{main}
passes on directly to the main file \textit{main}
while pretending to compile the child \textit{dest}.
This form behaves as if \textit{dest}
issues |\childdocof{|\textit{main}|}| right away,
and no further \textsf{childdoc} directives will be processed.

%%%%%%%%%%%%%%%%%%%%%%%%%%%%%%%%%%%%%%%%
\DescribeMacro{\...prefix}
In the alternative form |\childdocforwardprefix|,
%
\begin{center}
\begin{tabular}{l}
|\input{childdoc.def}|\\
|\childdocforwardprefix[|\textit{main}|]{|\textit{prefix}|}{|\textit{dest}|}|
\end{tabular}
\end{center}
%
the destination file is determined by a pattern
depending on the current file:
To make this work, the current file must be called
`{\textit{prefix}\hspace{0.2em}\textit{suffix}}'
with \textit{prefix} matching precisely the argument.
Processing is then passed on to the file
`{\textit{dest}\hspace{0.2em}\textit{suffix}}'.
Surely, the same effect is achieved by
directly specifying the
argument `{\textit{dest}\hspace{0.2em}\textit{suffix}}'
in the first form.
However, that requires to set up a different file
for each child. With the alternative form of the command
all these files can have exactly the same content
which simplifies setting them up and maintaining them.

For example, the following file |draft.tex|
with a compilation flag |\version| as described in \secref{sec:flags}
compiles the main document as a draft:
%
\begin{center}
\begin{tabular}{l}
|\def\version{draft}|\\
|\input{childdoc.def}|\\
|\childdocforward{|\textit{main}|}|
\end{tabular}
\end{center}
%
Likewise, the following files |final|\textit{nn}|.tex|
compile the final version of the child document
|child|\textit{nn}|.tex|:
%
\begin{center}
\begin{tabular}{l}
|\def\version{final}|\\
|\input{childdoc.def}|\\
|\childdocforwardprefix{final}{child}|
\end{tabular}
\end{center}
%

Note that when several versions of a main file and/or of each child file
are to be generated, it may be convenient to set up a |Makefile| or
shell script to automatise the process.

%%%%%%%%%%%%%%%%%%%%%%%%%%%%%%%%%%%%%%%%%%%%%%%%%%%%%%%%%%%%%%%%%%%%%%%%%%%%%%%%
\subsection{Command Line Processing}
\label{sec:commandline}

The effect of redirection files can also be achieved by invoking
the \LaTeX{} compiler with a more elaborate command line.
Most conveniently this should be done as part
of a shell script or a |Makefile|.

When using \textsf{childdoc} in the main file, the following
command lines effectively perform a redirection
(note that depending on the shell being used,
backslashes may have to be doubled: `|\|' $\to$ `|\\|'):
%
\begin{center}
|... -jobname "|\textit{target}|" |\\|"|[\textit{flags}]%
|\input{childdoc.def}\childdocforward[|\textit{main}|]{|\textit{dest}|}"|
\end{center}
%
Here \textit{target} is the name of the output file,
\textit{main} is the name of the main file
and \textit{dest} is the name of the main or child file to be processed
(all filenames without extensions).
The optional argument \textit{main} can be omitted
if \textit{main} matches \textit{dest}.
Optionally, compilation \textit{flags} can be defined via |\def| commands.
This command line makes the \TeX{} engine believe
it is compiling the file \textit{target}
whose content is specified as the latter parameter.
The provided code then forwards the processing to
\textit{main} or \textit{dest} as described in \secref{sec:forward}.

%%%%%%%%%%%%%%%%%%%%%%%%%%%%%%%%%%%%%%%%%%%%%%%%%%%%%%%%%%%%%%%%%%%%%%%%%%%%%%%%
\subsection{Include by Input}
\label{sec:input}

Including child documents by |\include| has some restrictions by design.
Most notably, the content of a child document always occupies
its own set of pages; pages cannot be shared between child documents.
Usually, this behaviour makes perfect sense
because each child document contain an essential part of the document.
However, in some situations it may be desirable to compose
a document from a collection of parts
without having mandatory page breaks between then.
For this case, the package
provides a mechanism to include parts
by |\input| which can also be processed individually.
However, by construction this mechanism
requires manual handling of the content to be output.

%%%%%%%%%%%%%%%%%%%%%%%%%%%%%%%%%%%%%%%%
\DescribeMacro{\ifchilddocmanual}
The main file should be prepared as usual, see \secref{sec:include}.
However, the document body must make a distinction
between processing of an individual part and of the main document, e.g.:
%
\begin{center}
\begin{tabular}{l}
|\ifchilddocmanual|\\
|\input{\childdocname}|\\
|\||else|\\
\textit{document body with }|\input{|\textit{part}|}|\\
|\||fi|
\end{tabular}
\end{center}
%
The conditional |\ifchilddocmanual| is true whenever
a part to be included by |\input| is being compiled,
and the name of the part is stored in |\childdocname|.

%%%%%%%%%%%%%%%%%%%%%%%%%%%%%%%%%%%%%%%%
\DescribeMacro{\childdocby}
Each part to be included by |\input| should start with:
%
\begin{center}
\begin{tabular}{l}
|\input{childdoc.def}|\\
|\childdocby{|\textit{main}|}|\\
\end{tabular}
\end{center}
%
The directive |\childdocby| is similar to |\childdocof|
described in \secref{sec:include},
but the subsequent selection of content must be done manually.
To that end, both |\ifchilddoc| and |\ifchilddocmanual|
will be true upon processing of a part,
and the name of the part is stored in |\childdocname|.
Note that |\jobname| will be set to the filename of the current part
so that each part receives an individual |.aux| file
that does not interfere with the |.aux| file(s) of the main document.
This behaviour can be altered by the alternative form
|\childdocby[*]{|\textit{main}|}| (with a non-empty optional argument)
which uses the |.aux| file of the main document
by setting |\jobname| to \textit{main}.

%%%%%%%%%%%%%%%%%%%%%%%%%%%%%%%%%%%%%%%%%%%%%%%%%%%%%%%%%%%%%%%%%%%%%%%%%%%%%%%%
\subsection{Driver Development}
\label{sec:driver}

The \textsf{childdoc} mechanism can also be use for the development
of definition files such as \LaTeX{} styles or classes.
This case differs from the above setup with multiple parts
included by |\include| in that no |\includeonly| should be invoked.
This can be achieved by starting the include file
(before |\ProvidesPackage|) with:
%
\begin{center}
\begin{tabular}{l}
|\input{childdoc.def}|\\
|\childdocforward{|\textit{main}|}|\\
\end{tabular}
\end{center}
%
or alternatively with:
%
\begin{center}
\begin{tabular}{l}
|\input{childdoc.def}|\\
|\childdocby{|\textit{main}|}|\\
\end{tabular}
\end{center}
%
Both forms have slightly different effects as described above.
The main file is prepared as usual, see \secref{sec:include}.

%%%%%%%%%%%%%%%%%%%%%%%%%%%%%%%%%%%%%%%%%%%%%%%%%%%%%%%%%%%%%%%%%%%%%%%%%%%%%%%%
\subsection{Legacy Detection}
\label{sec:detection}

The directive |\childdocmain| in the main file can detect
whether the complete document or merely a child is to be compiled
even without using the directive |\childdocof|.
This method is deprecated because it is less robust
and there is no compelling reason to use it;
it is merely provided for backward compatibility
and it may be removed in future versions.

If the detection mechanism is to be used,
it is mandatory to correctly specify
the filename of the main file as the argument of |\childdocmain|:
%
\begin{center}
\begin{tabular}{l}
|\input{childdoc.def}|\\
|\childdocmain{|\textit{main}|}|\\
\end{tabular}
\end{center}
%
If |\jobname| does not match the argument \textit{main} of |\childdocmain|,
it is assumed that |\jobname| points to the child file to be compiled.
When using |\childdocmain| with the main file specified as argument,
it suffices to start a child file
with just |\input{|\textit{main}|}|
without loading of the package and using |\childdocof|.
If instead all processing is done
with the appropriate \textsf{childdoc} directives,
the argument of \textit{main} of |\childdocmain| can be empty.

An alternative version of the command line processing described
in \secref{sec:commandline} using the detection mechanism reads:
%
\begin{center}
|... -jobname "|\textit{target}|" "|[\textit{flags}]%
[|\def\jobname{|\textit{dest}|}|]|\input{|\textit{main}|}"|
\end{center}

%%%%%%%%%%%%%%%%%%%%%%%%%%%%%%%%%%%%%%%%%%%%%%%%%%%%%%%%%%%%%%%%%%%%%%%%%%%%%%%%
\subsection{Manual Code}
\label{sec:manual}

In case one cannot be certain whether the definitions file |childdoc.def|
is installed on the target \TeX{} distribution
and one prefers not to ship it,
it is conceivable to paste a few relevant commands into the sources.

To that end, drop all statements |\input{childdoc.def}|
and perform the replacements as outlined below.
Instead of |\childdocmain{|\textit{main}|}| add the following code
to the top of the main file:
%
\begin{center}
\begin{tabular}{l}
|\||ifdefined\childdocname\endinput\||fi\newif\ifchilddoc|\\
|\edef\childdocname{\scantokens\expandafter{\jobname\noexpand}}|\\
|\def\childdocmain{|\textit{main}|}\||ifx\childdocmain\childdocname\||else|\\
|\childdoctrue\includeonly{\childdocname}\let\jobname\childdocmain\||fi|\\
\end{tabular}
\end{center}
%
Instead of |\childdocof{|\textit{main}|}| just include the main file
at the top of each child file:
%
\begin{center}
|\input{|\textit{main}|}|
\end{center}
%
A simple redirection |\childdocforward{|\textit{dest}|}| is achieved by:
%
\begin{center}
|\def\jobname{|\textit{dest}|}\input{\jobname}|
\end{center}
%
The redirection with prefix
|\childdocforwardprefix[|\textit{prefix}|]{|\textit{dest}|}|
is accomplished by:
%
\begin{center}
\begin{tabular}{l}
|{\edef\jobname{\scantokens\expandafter{\jobname\noexpand}}|\\
|\def\redirectjob |\textit{prefix}|#1~~~{\gdef\jobname{|\textit{dest}|#1}}|\\
|\expandafter\redirectjob\jobname~~~}\input{\jobname}|
\end{tabular}
\end{center}

In an alternative approach,
child documents can be compiled by a specific command line
without additional code or specific definitions:
%
\begin{center}
|... -jobname "|\textit{target}|" "|[\textit{flags}]%
|\includeonly{|\textit{dest}|}\input{|\textit{main}|}"|
\end{center}
%

%%%%%%%%%%%%%%%%%%%%%%%%%%%%%%%%%%%%%%%%%%%%%%%%%%%%%%%%%%%%%%%%%%%%%%%%%%%%%%%%
%%%%%%%%%%%%%%%%%%%%%%%%%%%%%%%%%%%%%%%%%%%%%%%%%%%%%%%%%%%%%%%%%%%%%%%%%%%%%%%%
\section{Information}

%%%%%%%%%%%%%%%%%%%%%%%%%%%%%%%%%%%%%%%%%%%%%%%%%%%%%%%%%%%%%%%%%%%%%%%%%%%%%%%%
\subsection{Copyright}

Copyright \copyright{} 2017--2018 Niklas Beisert

This work may be distributed and/or modified under the
conditions of the \LaTeX{} Project Public License, either version 1.3
of this license or (at your option) any later version.
The latest version of this license is in
  \url{http://www.latex-project.org/lppl.txt}
and version 1.3 or later is part of all distributions of \LaTeX{}
version 2005/12/01 or later.

This work has the LPPL maintenance status `maintained'.

The Current Maintainer of this work is Niklas Beisert.

This work consists of the files |README.txt|, |childdoc.ins| and |childdoc.dtx|
as well as the derived files |childdoc.def|, |cdocsamp.tex|
with |cdocsch1.tex|, |cdocsch2.tex|, |cdocspt3.tex|, |cdocspt4.tex|,
|cdocsdrf.tex|, |cdocsfn1.tex|, |cdocsfn2.tex|
as well as |childdoc.pdf|.

%%%%%%%%%%%%%%%%%%%%%%%%%%%%%%%%%%%%%%%%%%%%%%%%%%%%%%%%%%%%%%%%%%%%%%%%%%%%%%%%
\subsection{Files and Installation}

The package consists of the files:
%
\begin{center}
\begin{tabular}{ll}
    |README.txt|   & readme file \\
    |childdoc.ins| & installation file \\
    |childdoc.dtx| & source file \\
    |childdoc.def| & definition file \\
    |cdocsamp.tex| & sample main file \\
    |cdocsch1.tex| & sample include file \\
    |cdocsch2.tex| & sample include file \\
    |cdocspt3.tex| & sample part file \\
    |cdocspt4.tex| & sample part file \\
    |cdocsdrf.tex| & sample redirection file \\
    |cdocsfn1.tex| & sample redirection file \\
    |cdocsfn2.tex| & sample redirection file \\
    |childdoc.pdf| & manual
\end{tabular}
\end{center}
%
The distribution consists of the files
|README.txt|, |childdoc.ins| and |childdoc.dtx|.
%
\begin{itemize}
\item
Run (pdf)\LaTeX{} on |childdoc.dtx|
to compile the manual |childdoc.pdf| (this file).
\item
Run \LaTeX{} on |childdoc.ins| to create the definitions file |childdoc.def|
and the sample |cdocsamp.tex| with include files
|cdocsch1.tex|, |cdocsch2.tex|, |cdocspt3.tex|, |cdocspt4.tex|,
|cdocsdrf.tex|, |cdocsfn1.tex|, |cdocsfn2.tex|.
Then copy the file |childdoc.def| to an appropriate directory of your \LaTeX{}
distribution, e.g.\ \textit{texmf-root}|/tex/latex/childdoc|.
\end{itemize}

%%%%%%%%%%%%%%%%%%%%%%%%%%%%%%%%%%%%%%%%%%%%%%%%%%%%%%%%%%%%%%%%%%%%%%%%%%%%%%%%
\subsection{Related CTAN Packages}

There are several other packages which offer a similar functionality:
%
\begin{itemize}
\item
The packages
\href{http://ctan.org/pkg/docmute}{\textsf{docmute}},
\href{http://ctan.org/pkg/includex}{\textsf{includex}} and
\href{http://ctan.org/pkg/standalone}{\textsf{standalone}}
provide commands to include only the document body of
a child file thus allowing both files to be compiled individually.
\item
The packages \href{http://ctan.org/pkg/subdocs}{\textsf{subdocs}}
and \href{http://ctan.org/pkg/subfiles}{\textsf{subfiles}}
provide structures in which the main and child documents can be
encapsulated and allowing them to be compiled individually.
The inclusion mechanism is different from the conventional |\include|.
\item
The package \href{http://ctan.org/pkg/combine}{\textsf{combine}}
is an elaborate solution to combine several documents into one.
\end{itemize}
%
See also the CTAN topic \href{http://ctan.org/topic/subdocs}{\textsf{subdocs}}
for further related packages.
The present package differs from the above solutions in that
a document structure constructed with the conventional |\include| mechanism
just needs two extra commands at the top of every file
such that all constituent files can be compiled individually.

%%%%%%%%%%%%%%%%%%%%%%%%%%%%%%%%%%%%%%%%%%%%%%%%%%%%%%%%%%%%%%%%%%%%%%%%%%%%%%%%
%\subsection{Feature Suggestions}
%
%The following is a list of features which may be useful for future
%versions of this package:
%%
%\begin{itemize}
%\item
%\ldots
%\end{itemize}

%%%%%%%%%%%%%%%%%%%%%%%%%%%%%%%%%%%%%%%%%%%%%%%%%%%%%%%%%%%%%%%%%%%%%%%%%%%%%%%%
\subsection{Revision History}

%%%%%%%%%%%%%%%%%%%%%%%%%%%%%%%%%%%%%%%%
\paragraph{v2.0:} 2018/12/30

\begin{itemize}
\item
immediate forward processing
\item
added |\childdocby| mechanism
\item
manual restructured
\end{itemize}

%%%%%%%%%%%%%%%%%%%%%%%%%%%%%%%%%%%%%%%%
\paragraph{v1.6:} 2018/01/17

\begin{itemize}
\item
application for development of include files
\item
corrections to manual
\end{itemize}

%%%%%%%%%%%%%%%%%%%%%%%%%%%%%%%%%%%%%%%%
\paragraph{v1.5:} 2017/05/21

\begin{itemize}
\item
more complete structuring introduced
\item
|\childdocof| introduced
\item
|\childdoc| renamed to |\childdocmain|
\item
|\childredirect| renamed to |\childdocforward| and |\childdocforwardprefix|
and functionality expanded
\end{itemize}

%%%%%%%%%%%%%%%%%%%%%%%%%%%%%%%%%%%%%%%%
\paragraph{v1.0:} 2017/04/27

\begin{itemize}
\item
manual and install package
\item
first version published on CTAN
\end{itemize}

%%%%%%%%%%%%%%%%%%%%%%%%%%%%%%%%%%%%%%%%
\paragraph{v0.6:} 2017/04/26

\begin{itemize}
\item
redirection mechanism added
\end{itemize}

%%%%%%%%%%%%%%%%%%%%%%%%%%%%%%%%%%%%%%%%
\paragraph{v0.5:} 2017/04/26

\begin{itemize}
\item
functionality in definition file
\end{itemize}


%%%%%%%%%%%%%%%%%%%%%%%%%%%%%%%%%%%%%%%%%%%%%%%%%%%%%%%%%%%%%%%%%%%%%%%%%%%%%%%%
%%%%%%%%%%%%%%%%%%%%%%%%%%%%%%%%%%%%%%%%%%%%%%%%%%%%%%%%%%%%%%%%%%%%%%%%%%%%%%%%
%%%%%%%%%%%%%%%%%%%%%%%%%%%%%%%%%%%%%%%%%%%%%%%%%%%%%%%%%%%%%%%%%%%%%%%%%%%%%%%%
\appendix

\settowidth\MacroIndent{\rmfamily\scriptsize 000\ }

 \DocInput{childdoc.dtx}

\end{document}
%</driver>
% \fi
%
% %%%%%%%%%%%%%%%%%%%%%%%%%%%%%%%%%%%%%%%%%%%%%%%%%%%%%%%%%%%%%%%%%%%%%%%%%%%%%%
% %%%%%%%%%%%%%%%%%%%%%%%%%%%%%%%%%%%%%%%%%%%%%%%%%%%%%%%%%%%%%%%%%%%%%%%%%%%%%%
% \section{Sample}
%\iffalse
%<*samplemain>
%\fi
%
% The following presents a sample document
% with two chapters, two parts, a title page,
% a compile flag as well as three forwarding files to set the flag.
% It consists of eight |.tex| files:
% \begin{center}
% \begin{tabular}{ll}
% |cdocsamp.tex|&main file\\
% |cdocsch1.tex|&include file for chapter 1\\
% |cdocsch2.tex|&include file for chapter 2\\
% |cdocspt3.tex|&include file for part 3\\
% |cdocspt4.tex|&include file for part 4\\
% |cdocsdrf.tex|&forwarding file for main file in draft mode\\
% |cdocsfi1.tex|&forwarding file for final version of chapter 1\\
% |cdocsfi2.tex|&forwarding file for final version of chapter 2\\
% \end{tabular}
% \end{center}
% Each of the eight files can be compiled directly by the \LaTeX{} compiler.
%
% %%%%%%%%%%%%%%%%%%%%%%%%%%%%%%%%%%%%%%
% \paragraph{Main File.}
%
% The main file is called |cdocsamp.tex|.
%
% Load the \textsf{childdoc} definitions and
% declare the filename for the main document:
%    \begin{macrocode}
\input{childdoc.def}
\childdocmain{}
%    \end{macrocode}

% Optional override for |\version| flag:
%    \begin{macrocode}
%%\ifchilddoc\else\providecommand{\version}{draft}\fi
%    \end{macrocode}

% Define the default values for the |\version| flag
% (|final| for the main file and |draft| for childs):
%    \begin{macrocode}
\ifchilddoc
\providecommand{\version}{draft}
\else
\providecommand{\version}{final}
\fi
%    \end{macrocode}

% Load the standard document class:
%    \begin{macrocode}
\documentclass[12pt]{article}
%    \end{macrocode}

% Start the document body:
%    \begin{macrocode}
\begin{document}
%    \end{macrocode}

% Declare a title page.
% Print title, part of document being processed and version flag:
%    \begin{macrocode}
\addtocounter{page}{-1}
\begin{center}
{\LARGE\bfseries{}childdoc example\par}
\vspace{1cm}
\ifchilddoc
\ifchilddocmanual part\else chapter\fi:
`\childdocname' of `\childdocjob'\par
\else
main document: `\childdocjob'\par
\fi
version: \version\par
\end{center}
\newpage
%    \end{macrocode}

% Manually include selected file,
% otherwise process as usual:
%    \begin{macrocode}
\ifchilddocmanual
\section*{part `\childdocname'}
\input{\childdocname}
\else
%    \end{macrocode}

% Include the two chapters:
%    \begin{macrocode}
\include{cdocsch1}
\include{cdocsch2}
%    \end{macrocode}

% Include the two parts unless only chapters should be displayed:
%    \begin{macrocode}
\ifchilddoc\else
\section{part three}
\input{cdocspt3}
\section{part four}
\input{cdocspt4}
\fi
%    \end{macrocode}

% Process as usual until here:
%    \begin{macrocode}
\fi
%    \end{macrocode}

% End of document body:
%    \begin{macrocode}
\end{document}
%    \end{macrocode}
%\iffalse
%</samplemain>
%\fi
%
% %%%%%%%%%%%%%%%%%%%%%%%%%%%%%%%%%%%%%%
% \paragraph{Chapter Include Files.}
%
% The include files are called |cdocsch1.tex| and |cdocsch2.tex|.
%
%\iffalse
%<*samplechap1|samplechap2>
%\fi

% Optional override for |\version| flag:
%    \begin{macrocode}
%%\providecommand{\version}{final}
%    \end{macrocode}

% Include the main document:
%    \begin{macrocode}
\input{childdoc.def}
\childdocof{cdocsamp}
%    \end{macrocode}

%\iffalse
%</samplechap1|samplechap2>
%\fi
%
%\iffalse
%<*samplechap1>
%\fi
% Some text for chapter 1:
%    \begin{macrocode}
\section{one}
some text in chapter one
%    \end{macrocode}

%\iffalse
%</samplechap1>
%\fi
% Some text for chapter 2:
%\iffalse
%<*samplechap2>
%\fi
%    \begin{macrocode}
\section{two}
more text in chapter two
%    \end{macrocode}

%\iffalse
%</samplechap2>
%\fi
%
% %%%%%%%%%%%%%%%%%%%%%%%%%%%%%%%%%%%%%%
% \paragraph{Part Include Files.}
%
% The include files are called |cdocspt3.tex| and |cdocspt4.tex|.
%
%\iffalse
%<*samplepart3|samplepart4>
%\fi

% Optional override for |\version| flag:
%    \begin{macrocode}
%%\providecommand{\version}{final}
%    \end{macrocode}

% Include the main document:
%    \begin{macrocode}
\input{childdoc.def}
\childdocby{cdocsamp}
%    \end{macrocode}

%\iffalse
%</samplepart3|samplepart4>
%\fi
%
%\iffalse
%<*samplepart3>
%\fi
% Some text for part 3:
%    \begin{macrocode}
some text in part three
%    \end{macrocode}

%\iffalse
%</samplepart3>
%\fi
% Some text for part 4:
%\iffalse
%<*samplepart4>
%\fi
%    \begin{macrocode}
more text in part four
%    \end{macrocode}

%\iffalse
%</samplepart4>
%\fi
%
% %%%%%%%%%%%%%%%%%%%%%%%%%%%%%%%%%%%%%%
% \paragraph{Forwarding for a Complete Draft.}
%
% The following forwarding file |cdocsdrf.tex|
% compiles the main document in draft mode:
%\iffalse
%<*sampledraft>
%\fi
%    \begin{macrocode}
\def\version{draft}
\input{childdoc.def}
\childdocforward{cdocsamp}
%    \end{macrocode}

%\iffalse
%</sampledraft>
%\fi
%
% %%%%%%%%%%%%%%%%%%%%%%%%%%%%%%%%%%%%%%
% \paragraph{Forwarding for Final Version of the Chapters.}
%
% The following forwarding files |cdocsfn1.tex| and |cdocsfn2.tex|
% (with identical content)
% compile the final versions of the child documents
% |cdocsch1.tex| and |cdocsch2.tex|, respectively:
%\iffalse
%<*samplefinal>
%\fi
%    \begin{macrocode}
\def\version{final}
\input{childdoc.def}
\childdocforwardprefix[cdocsamp]{cdocsfn}{cdocsch}
%    \end{macrocode}

%\iffalse
%</samplefinal>
%\fi
%
% %%%%%%%%%%%%%%%%%%%%%%%%%%%%%%%%%%%%%%
% \paragraph{Command Line Processing.}
%
% The following three command lines generate the output files
% |cdocscld|, |cdocscl1| and |cdocscl2|
% which should be identical to
% |cdocsdrf|, |cdocsch1| and |cdocsfn2|, respectively:
% \begin{center}
% \begin{tabular}{l}
% |latex -jobname cdocscld \|\\
% |  "\def\version{draft}\input{childdoc.def}\childdocforward{cdocsamp}"|\\
% |latex -jobname cdocscl1 \|\\
% |  "\input{childdoc.def}\childdocforward[cdocsamp]{cdocsch1}"|\\
% |latex -jobname cdocscl2 \|\\
% |  "\def\version{final}\input{childdoc.def}\childdocforward{cdocsch2}"|
% \end{tabular}
% \end{center}
% Note that the trailing backslash on each first line
% merely continues the input to the second line
% (for convenient cut ant paste).
% Furthermore, the command |latex| can be replaced by any
% of its alternative versions such as |pdflatex|.
%
% %%%%%%%%%%%%%%%%%%%%%%%%%%%%%%%%%%%%%%%%%%%%%%%%%%%%%%%%%%%%%%%%%%%%%%%%%%%%%%
% %%%%%%%%%%%%%%%%%%%%%%%%%%%%%%%%%%%%%%%%%%%%%%%%%%%%%%%%%%%%%%%%%%%%%%%%%%%%%%
% \section{Implementation}
%\iffalse
%<*package>
%\fi
%
% This section describes the definitions file |childdoc.def|.

% The definitions cannot be loaded using |\usepackage| or |\RequirePackage|
% which has a mechanism to prevent loading a style file more than once.
% When loading the definitions by means of |\input|
% multiple instances have to be prevented manually:
%\iffalse
%This code needs to be before the `\ProvidesFile' directive
%which is defined at the beginning of this file.
%Therefore it is also placed there and commented out here.
%</package>
%<*discard>
%\fi
%    \begin{macrocode}
\ifdefined\childdocmain\endinput\fi
%    \end{macrocode}
%\iffalse
%</discard>
%<*package>
%\fi
%
% \macro{\ifchilddoc}
% \macro{\ifchilddocmanual}
% The conditional |\ifchilddoc| tells whether a
% child (true) or main (false) document is being compiled.
% The conditional |\ifchilddocmanual| tells whether
% the |\includeonly| mechanism is used (false) or
% the selection of child files must be performed manually (true).
% The definitions initialise to false:
%    \begin{macrocode}
\newif\ifchilddoc
\newif\ifchilddocmanual
%    \end{macrocode}

% \macro{\childdocname}
% \macro{\childdocjob}
% The macro |\childdocname| stores the name of the main document
% to be compiled. The macro |\childdocjob| stores the name of
% the document on which the \LaTeX{} compiler was originally invoked.
% The content of |\jobname| cannot be compared
% to filenames specified in the source due to different catcodes.
% The following code rescans |\jobname|, stores the result
% in |\childdocname| and saves a copy in |\childdocjob|:
%    \begin{macrocode}
\edef\childdocname{\scantokens\expandafter{\jobname\noexpand}}
\let\childdocjob\childdocname
%    \end{macrocode}

% \macro{\childdocdisable}
% The macro |\childdocdisable| prevents the main file
% from being processed more than once.
% At this stage, the main document command |\childdocmain|
% is assumed to be called once again where it should do nothing.
% Any subsequent call to it should prevent
% a secondary processing of the main document
% It overwrites the forwarding commands
% |\childdocof| and |\childdocforward|
% with empty macros to prevent further inclusions of the main document:
%    \begin{macrocode}
\newcommand{\childdocdisable}
{
  \renewcommand{\childdocmain}[1]{\renewcommand{\childdocmain}[1]{\endinput}}
  \renewcommand{\childdocof}[1]{}
  \renewcommand{\childdocby}[2][]{}
  \renewcommand{\childdocforward}[2][]{}
  \renewcommand{\childdocdisable}{}
}
%    \end{macrocode}

% \macro{\childdocmain}
% The macro |\childdocmain| is to be called at the top of the main file
% with nothing or the main filename (without extension) as argument.
% First, it breaks loops.
% If the argument is not empty and does not match |\childdocname|
% (which is set by the first inclusion of |childdoc.def|),
% |\ifchilddoc| is set to true, |\includeonly| is applied to the child file
% and |\jobname| is set to the main file
% (for proper handling of |.aux| files):
%    \begin{macrocode}
\newcommand{\childdocmain}[1]
{
  \childdocdisable\childdocmain{}
  \if?#1?\else
    \begingroup
      \def\childdoctmp{#1}
      \ifx\childdoctmp\childdocname
        \def\childdoctmp{}
      \else
        \def\childdoctmp
        {
          \childdoctrue
          \includeonly{\childdocname}
          \def\childdocjob{#1}
          \def\jobname{#1}
        }
      \fi
      \expandafter
    \endgroup
    \childdoctmp
  \fi
}
%    \end{macrocode}

% \macro{\childdocof}
% The command |\childdocof| redirects
% compilation to the main file |#1|.
%    \begin{macrocode}
\newcommand{\childdocof}[1]
{
  \childdocdisable
  \childdoctrue
  \includeonly{\childdocname}
  \def\jobname{#1}
  \def\childdocjob{#1}
  \input{#1}
}
%    \end{macrocode}

% \macro{\childdocby}
% The command |\childdocby| ....
%    \begin{macrocode}
\newcommand{\childdocby}[2][]
{
  \childdocdisable
  \childdoctrue
  \childdocmanualtrue
  \if?#1?\else
    \def\jobname{#2}
  \fi
  \def\childdocjob{#2}
  \input{#2}
  \endinput
}
%    \end{macrocode}

% \macro{\childdocforward}
% The command |\childdocforward| redirects
% compilation to the main file or
% (if the optional argument is given) a child file.
% Parameters are set as if the main file
% or a child file starting with |\childdocof| was compiled.
% Then compilation is handed over to the main file:
%    \begin{macrocode}
\newcommand{\childdocforward}[2][]
{
  \begingroup
    \if?#1?
      \def\childdoctmp
      {
        \def\childdocname{#2}
        \def\childdocjob{#2}
        \def\jobname{#2}
        \input{#2}
        \endinput
      }
    \else
      \def\childdoctmp
      {
        \childdocdisable
        \def\childdocname{#2}
        \childdoctrue
        \includeonly{#2}
        \def\childdocjob{#1}
        \def\jobname{#1}
        \input{#1}
        \endinput
      }
    \fi
    \expandafter
  \endgroup
  \childdoctmp
}
%    \end{macrocode}

% \macro{\childdocforwardprefix}
% The command |\childdocforwardprefix| redirects
% compilation to the main or a child file by means of a pattern.
% The prefix |#1| in the current filename is replaced by |#2|
% and the suffix of the current filename is kept
% (it is assumed that the filename does not contain the substring `|~~~|'
% which is used as a delimiter).
% Compilation is handed over to the new file by |\childdocforward|:
%    \begin{macrocode}
\newcommand{\childdocforwardprefix}[3][]
{
  \begingroup
    \def\childdocextract #2##1~~~{\def\childdoctmp{\childdocforward[#1]{#3##1}}}
    \expandafter\childdocextract\childdocname~~~
    \expandafter
  \endgroup
  \childdoctmp
}
%    \end{macrocode}

% \macro{\childdoc}
% The deprecated macro |\childdoc| is a legacy version of |\childdocmain|:
%    \begin{macrocode}
\newcommand{\childdoc}{\childdocmain}
%    \end{macrocode}

% \macro{\childdocredirect}
% The deprecated macro |\childdocredirect| is a legacy version
% of |\childdocforward| and |\childdocforwardprefix|:
%    \begin{macrocode}
\newcommand{\childdocredirect}[2][]
{
  \begingroup
    \if?#1?
      \def\childdoctmp{\childdocforward{#2}}
    \else
      \def\childdoctmp{\childdocforwardprefix{#1}{#2}}
    \fi
    \expandafter
  \endgroup
  \childdoctmp
}
%    \end{macrocode}

%\iffalse
%</package>
%\fi
%
\endinput

\childdocforwardprefix[cdocsamp]{cdocsfn}{cdocsch}
%    \end{macrocode}

%\iffalse
%</samplefinal>
%\fi
%
% %%%%%%%%%%%%%%%%%%%%%%%%%%%%%%%%%%%%%%
% \paragraph{Command Line Processing.}
%
% The following three command lines generate the output files
% |cdocscld|, |cdocscl1| and |cdocscl2|
% which should be identical to
% |cdocsdrf|, |cdocsch1| and |cdocsfn2|, respectively:
% \begin{center}
% \begin{tabular}{l}
% |latex -jobname cdocscld \|\\
% |  "\def\version{draft}% \iffalse
%
% childdoc.dtx Copyright (C) 2017-2018 Niklas Beisert
%
% This work may be distributed and/or modified under the
% conditions of the LaTeX Project Public License, either version 1.3
% of this license or (at your option) any later version.
% The latest version of this license is in
%   http://www.latex-project.org/lppl.txt
% and version 1.3 or later is part of all distributions of LaTeX
% version 2005/12/01 or later.
%
% This work has the LPPL maintenance status `maintained'.
%
% The Current Maintainer of this work is Niklas Beisert.
%
% This work consists of the files childdoc.dtx and childdoc.ins
% and the derived files childdoc.def and cdocsamp.tex with
% cdocsch1.tex, cdocsch2.tex, cdocsdrf.tex, cdocsfn1.tex, cdocsfn2.tex.
%
%<package>\ifdefined\childdocmain\endinput\fi
%<package>\ProvidesFile{childdoc.def}[2018/12/30 v2.0 child document driver]
%<samplemain>\ProvidesFile{cdocsamp.tex}[2018/12/30 v2.0 sample for childdoc]
%<*driver>
%\ProvidesFile{childdoc.drv}[2018/12/30 v2.0 childdoc reference manual file]
\PassOptionsToClass{10pt,a4paper}{article}
\documentclass{ltxdoc}

\usepackage[margin=35mm]{geometry}
\usepackage{hyperref}
\usepackage{hyperxmp}
\usepackage[usenames]{color}

\hypersetup{colorlinks=true}
\hypersetup{pdfstartview=FitH}
\hypersetup{pdfpagemode=UseNone}
\hypersetup{pdfsource={}}
\hypersetup{pdflang={en-UK}}
\hypersetup{pdfcopyright={Copyright 2017-2018 Niklas Beisert.
  This work may be distributed and/or modified under the
  conditions of the LaTeX Project Public License, either version 1.3
  of this license or (at your option) any later version.}}
\hypersetup{pdflicenseurl={http://www.latex-project.org/lppl.txt}}
\hypersetup{pdfcontactaddress={ETH Zurich, ITP, HIT K,
  Wolfgang-Pauli-Strasse 27}}
\hypersetup{pdfcontactpostcode={8093}}
\hypersetup{pdfcontactcity={Zurich}}
\hypersetup{pdfcontactcountry={Switzerland}}
\hypersetup{pdfcontactemail={nbeisert@itp.phys.ethz.ch}}
\hypersetup{pdfcontacturl={http://people.phys.ethz.ch/\xmptilde nbeisert/}}

\newcommand{\secref}[1]{\hyperref[#1]{section \ref*{#1}}}

\parskip1ex
\parindent0pt
\let\olditemize\itemize
\def\itemize{\olditemize\parskip0pt}

\begin{document}

\title{The \textsf{childdoc} Package}
\hypersetup{pdftitle={The childdoc Package}}
\author{Niklas Beisert\\[2ex]
  Institut f\"ur Theoretische Physik\\
  Eidgen\"ossische Technische Hochschule Z\"urich\\
  Wolfgang-Pauli-Strasse 27, 8093 Z\"urich, Switzerland\\[1ex]
  \href{mailto:nbeisert@itp.phys.ethz.ch}
  {\texttt{nbeisert@itp.phys.ethz.ch}}}
\hypersetup{pdfauthor={Niklas Beisert}}
\hypersetup{pdfsubject={Manual for the LaTeX2e Package childdoc}}
\date{30 December 2018, \textsf{v2.0}}
\maketitle

\begin{abstract}\noindent
\textsf{childdoc} is a \LaTeXe{} package
that enables the direct compilation
of document sections included by |\include|
to individual files.
\end{abstract}

\begingroup
\parskip0ex
\tableofcontents
\endgroup

%%%%%%%%%%%%%%%%%%%%%%%%%%%%%%%%%%%%%%%%%%%%%%%%%%%%%%%%%%%%%%%%%%%%%%%%%%%%%%%%
%%%%%%%%%%%%%%%%%%%%%%%%%%%%%%%%%%%%%%%%%%%%%%%%%%%%%%%%%%%%%%%%%%%%%%%%%%%%%%%%
\section{Introduction}

\LaTeX{} provides a mechanism to structure a large document (such as a book)
into a main file and several child files (containing the chapters)
using the |\include| command.
This mechanism is beneficial for documents
which span hundreds of pages in order to
make the source file(s) more manageable.
Moreover, compilation can be restricted to
selected child files by means of the |\includeonly| command.
The latter feature can be used to reduce the compilation time while editing
(this was significantly more useful in the earlier days of \LaTeX{})
or to generate a smaller document which is easier to navigate.
Another application of |\includeonly| is to generate
documents consisting of selected parts of the complete document.

However, there are a few drawbacks of the plain |\include| mechanism:
\begin{itemize}
\item
The child files cannot be compiled on their own,
they can only be compiled via the main file.
A naive editing environment
(such as a text editor with an option
to have the current file processed by \LaTeX)
may require one to switch to the main file before compiling;
attempting to compile the child file produces errors.
\item
The main file must be modified (each time)
to adjust the |\includeonly| command
to the present needs. This easily leaves the main file in a messy state.
\item
The generated document will always carry the filename
of the main document. This is inconvenient if
several child files are to be compiled and
to be kept for distribution.
\end{itemize}

The present package provides a simple interface
to make child files individually compilable by \LaTeX{}.
Compiling a child file then has the same effect as compiling
the main file with an |\includeonly| command
to select the appropriate child.
Moreover the generated document will carry the name of the child
rather than the main file.
This resolves all three above issues.

This feature is meant to make the editing of books,
thesis documents and lecture notes somewhat more convenient.
However, the package can also be used efficiently for
composing a series of documents (such as exercise sheets)
which are typically distributed individually.
It then assists the author in generating the individual documents
(potentially in different versions)
as well as a document containing the collected series.
Another application is in developing style files
or other kinds of included material
where compilation of the style file could redirect
to a sample or test file.

%%%%%%%%%%%%%%%%%%%%%%%%%%%%%%%%%%%%%%%%%%%%%%%%%%%%%%%%%%%%%%%%%%%%%%%%%%%%%%%%
%%%%%%%%%%%%%%%%%%%%%%%%%%%%%%%%%%%%%%%%%%%%%%%%%%%%%%%%%%%%%%%%%%%%%%%%%%%%%%%%
\section{Usage}

First of all, the package \textsf{childdoc} is \emph{not} a standard
\LaTeXe{} |.sty| style file! Therefore it needs to be invoked in
a non-standard way.

%%%%%%%%%%%%%%%%%%%%%%%%%%%%%%%%%%%%%%%%%%%%%%%%%%%%%%%%%%%%%%%%%%%%%%%%%%%%%%%%
\subsection{Included Files}
\label{sec:include}

%%%%%%%%%%%%%%%%%%%%%%%%%%%%%%%%%%%%%%%%
\DescribeMacro{\childdocmain}
To use the package, add the commands
\begin{center}
\begin{tabular}{l}
|\input{childdoc.def}|\\
|\childdocmain{}|\\
\end{tabular}
\end{center}
at the very top of the main \LaTeX{} file,
in particular \emph{before} the |\documentclass| statement!
The argument of |\childdocmain| should be left empty
(but it must be present).

%%%%%%%%%%%%%%%%%%%%%%%%%%%%%%%%%%%%%%%%
\DescribeMacro{\childdocof}
Furthermore, add the commands
\begin{center}
\begin{tabular}{l}
|\input{childdoc.def}|\\
|\childdocof{|\textit{main}|}|\\
\end{tabular}
\end{center}
at the top of every child file \textit{child}
which is included by |\include{|\textit{child}|}|
from within the main file
(or at least for those files to be compiled individually).
The argument \textit{main} must be the filename of the main file.

There are a couple of
considerations in setting up the main and child documents:

%%%%%%%%%%%%%%%%%%%%%%%%%%%%%%%%%%%%%%%%
\paragraph{Restrictions.}

Please note the following restrictions:
\begin{itemize}
\item
|\childdocmain| must be called with one argument \textit{main}
to ensure compatibility with earlier version of the package.
It must either be empty (|\childdocmain{}|)
or precisely match the filename of the main file in which it is specified.
See \secref{sec:detection} for further information.
\item
The filename \textit{main} must be specified without the |.tex| extension.
\item
The filename \textit{main} is case sensitive
(even in case-insensitive file systems)
due to internal string comparison.
\item
The argument \textit{main} should be fully expanded, it cannot be a macro.
\item
Subdirectories and special characters should be avoided in filenames.
\item
The command |\childdocmain{|\textit{main}|}| must be followed by a whitespace.
It should not be followed immediately by another command
or by a comment mark `|%|'.
This is because the \TeX{} parser reads the token immediately following
the argument of |\childdocmain| and puts it
at the beginning of every child section;
however, a white\-space is ignored.
\end{itemize}

%%%%%%%%%%%%%%%%%%%%%%%%%%%%%%%%%%%%%%%%
\paragraph{Content of Main File.}

It is advisable to place all content in the child files included by |\include|.
Any output contained in the main file will appear in all child documents
unless suppressed manually;
it cannot be suppressed automatically by the |\includeonly| directive
and thus should normally be avoided.
A method to include some content in the main file
by means of conditional processing is described in \secref{sec:conditional}.

%%%%%%%%%%%%%%%%%%%%%%%%%%%%%%%%%%%%%%%%
\paragraph{Page Numbering.}

When only a part of the document is compiled,
the appropriate numbering of pages
(as well as other status parameters)
is determined from the |.aux| files.
The latter contain information from previous passes.
However this information needs to propagate through
all intermediate child documents.
Therefore the page numbering in child documents may well
be inconsistent until the complete document is compiled at least once.

A useful (if unconventional) way to always ensure a consistent
page numbering is to restart the numbering in each child document
and denote the pages by `\textit{child}|.|\textit{page}'
where \textit{child} represents the chapter/section number of the child file.
This can be achieved by the command
|\numberwithin{page}{|\textit{child}|}|
of the \textsf{amsmath} package
where \textit{child} can be |chapter| or |section|
depending on the chosen structuring.
Alternatively, one can modify the macro |\thepage| appropriately
and reset the counter |page| at the start of each child file.

%%%%%%%%%%%%%%%%%%%%%%%%%%%%%%%%%%%%%%%%%%%%%%%%%%%%%%%%%%%%%%%%%%%%%%%%%%%%%%%%
\subsection{Conditional Processing}
\label{sec:conditional}

The package provides a mechanism to compile different versions
of a document. To customise the versions further some conditional processing
can come in handy to distinguish which version is being compiled.
The package provides two macros to describe the compilation context:

%%%%%%%%%%%%%%%%%%%%%%%%%%%%%%%%%%%%%%%%
\DescribeMacro{\ifchilddoc}
The conditional |\ifchilddoc| distinguishes between the compilation of
child documents and the main document:
%
\begin{center}
|\ifchilddoc |\textit{child-code}| |[|\||else |\textit{main-code}]| \||fi|
\end{center}

%%%%%%%%%%%%%%%%%%%%%%%%%%%%%%%%%%%%%%%%
\DescribeMacro{\childdocname}
\DescribeMacro{\childdocjob}
The macro |\childdocname| contains the filename (without extension)
of the main or child file being processed.
Note that |\childdocjob| will always contain the name of the main file.

%%%%%%%%%%%%%%%%%%%%%%%%%%%%%%%%%%%%%%%%
\paragraph{Title Page.}

Conditional processing can be used to include a title or banner page
in the main document when proper precautions are taken.
Importantly, the code in the main file should ensure that the page counter
(as well as other status parameters which are stored in the |.aux| files)
takes the same value after the conditional processing.
Otherwise the page numbers may take divergent values
depending on which part is compiled.

For example, a title page could be declared by:
%
\begin{center}
\begin{tabular}{l}
|\ifchilddoc\||else|\\
|\addtocounter{page}{-1}|\\
\textit{code for title page}\\
|\newpage|\\
|\||fi|
\end{tabular}
\end{center}
%
A banner page for the child documents can be generated by:
%
\begin{center}
\begin{tabular}{l}
|\ifchilddoc|\\
|\addtocounter{page}{-1}|\\
\textit{code for banner page}\\
|\newpage|\\
|\||fi|
\end{tabular}
\end{center}
%
Here one could write a message such as:
\begin{center}
|This is the part \childdocname{} of \childdocjob{}.|
\end{center}

%%%%%%%%%%%%%%%%%%%%%%%%%%%%%%%%%%%%%%%%%%%%%%%%%%%%%%%%%%%%%%%%%%%%%%%%%%%%%%%%
\subsection{Flags}
\label{sec:flags}

The package makes it easy to generate different versions
of the main or child documents.
To this end compilation flags can be defined
and assigned different default values.
They will be particularly useful in conjunction
with the forwarding mechanism described in \secref{sec:forward}.

For example, it may be useful to have a flag |\version|
which can be set to |draft| or |final|.
The document source will contain some conditional code
depending on the value of |\version|.
Suppose further, the flag should default to |final| for the main file
and to |draft| for child files
which is a natural assignment for editing the document.
This is achieved by placing the following code
in the preamble of the main document
(below the |\childdocmain| directive):
%
\begin{center}
\begin{tabular}{l}
|\ifchilddoc|\\
|\providecommand{\version}{draft}|\\
|\||else|\\
|\providecommand{\version}{final}|\\
|\||fi|
\end{tabular}
\end{center}
%
The definition by |\providecommand| makes sure
that previous definitions are not overwritten.
Further statements |\providecommand{\version}{...}|
can thus be added before the above code to override it.

For the main file, one might add a line
(between |\childdocmain| and the above block)
%
\begin{center}
|%\ifchilddoc\||else\providecommand{\version}{draft}\||fi|
\end{center}
%
which can be uncommented to produce a draft version.
Likewise one can add a line to the very top of a child file
(above the |\childdocof{|\textit{main}|}| directive)
%
\begin{center}
|%\providecommand{\version}{final}|
\end{center}
%
which can be uncommented to produce the final version of this child document.

%%%%%%%%%%%%%%%%%%%%%%%%%%%%%%%%%%%%%%%%%%%%%%%%%%%%%%%%%%%%%%%%%%%%%%%%%%%%%%%%
\subsection{Forwarding}
\label{sec:forward}

Different versions of the main or child documents
using compilation flags as described in \secref{sec:flags}
can be (permanently) stored in different files
for convenient compilation, viewing and distribution.
To this end, the package defines a command
to pass on compilation to a different file:

%%%%%%%%%%%%%%%%%%%%%%%%%%%%%%%%%%%%%%%%
\DescribeMacro{\childdocforward}
The command |\childdocforward| redirects processing to
another source file:
%
\begin{center}
\begin{tabular}{l}
|\input{childdoc.def}|\\
|\childdocforward[|\textit{main}|]{|\textit{dest}|}|\\
\end{tabular}
\end{center}
%
The argument \textit{dest} is the destination file
(without extension).
It should be the main file or one of the child files.
Note that further \textsf{childdoc} directives
such as |\childdocof| and |\childdocforward|
in the indicated file will be processed in this form.
The optional argument \textit{main}
passes on directly to the main file \textit{main}
while pretending to compile the child \textit{dest}.
This form behaves as if \textit{dest}
issues |\childdocof{|\textit{main}|}| right away,
and no further \textsf{childdoc} directives will be processed.

%%%%%%%%%%%%%%%%%%%%%%%%%%%%%%%%%%%%%%%%
\DescribeMacro{\...prefix}
In the alternative form |\childdocforwardprefix|,
%
\begin{center}
\begin{tabular}{l}
|\input{childdoc.def}|\\
|\childdocforwardprefix[|\textit{main}|]{|\textit{prefix}|}{|\textit{dest}|}|
\end{tabular}
\end{center}
%
the destination file is determined by a pattern
depending on the current file:
To make this work, the current file must be called
`{\textit{prefix}\hspace{0.2em}\textit{suffix}}'
with \textit{prefix} matching precisely the argument.
Processing is then passed on to the file
`{\textit{dest}\hspace{0.2em}\textit{suffix}}'.
Surely, the same effect is achieved by
directly specifying the
argument `{\textit{dest}\hspace{0.2em}\textit{suffix}}'
in the first form.
However, that requires to set up a different file
for each child. With the alternative form of the command
all these files can have exactly the same content
which simplifies setting them up and maintaining them.

For example, the following file |draft.tex|
with a compilation flag |\version| as described in \secref{sec:flags}
compiles the main document as a draft:
%
\begin{center}
\begin{tabular}{l}
|\def\version{draft}|\\
|\input{childdoc.def}|\\
|\childdocforward{|\textit{main}|}|
\end{tabular}
\end{center}
%
Likewise, the following files |final|\textit{nn}|.tex|
compile the final version of the child document
|child|\textit{nn}|.tex|:
%
\begin{center}
\begin{tabular}{l}
|\def\version{final}|\\
|\input{childdoc.def}|\\
|\childdocforwardprefix{final}{child}|
\end{tabular}
\end{center}
%

Note that when several versions of a main file and/or of each child file
are to be generated, it may be convenient to set up a |Makefile| or
shell script to automatise the process.

%%%%%%%%%%%%%%%%%%%%%%%%%%%%%%%%%%%%%%%%%%%%%%%%%%%%%%%%%%%%%%%%%%%%%%%%%%%%%%%%
\subsection{Command Line Processing}
\label{sec:commandline}

The effect of redirection files can also be achieved by invoking
the \LaTeX{} compiler with a more elaborate command line.
Most conveniently this should be done as part
of a shell script or a |Makefile|.

When using \textsf{childdoc} in the main file, the following
command lines effectively perform a redirection
(note that depending on the shell being used,
backslashes may have to be doubled: `|\|' $\to$ `|\\|'):
%
\begin{center}
|... -jobname "|\textit{target}|" |\\|"|[\textit{flags}]%
|\input{childdoc.def}\childdocforward[|\textit{main}|]{|\textit{dest}|}"|
\end{center}
%
Here \textit{target} is the name of the output file,
\textit{main} is the name of the main file
and \textit{dest} is the name of the main or child file to be processed
(all filenames without extensions).
The optional argument \textit{main} can be omitted
if \textit{main} matches \textit{dest}.
Optionally, compilation \textit{flags} can be defined via |\def| commands.
This command line makes the \TeX{} engine believe
it is compiling the file \textit{target}
whose content is specified as the latter parameter.
The provided code then forwards the processing to
\textit{main} or \textit{dest} as described in \secref{sec:forward}.

%%%%%%%%%%%%%%%%%%%%%%%%%%%%%%%%%%%%%%%%%%%%%%%%%%%%%%%%%%%%%%%%%%%%%%%%%%%%%%%%
\subsection{Include by Input}
\label{sec:input}

Including child documents by |\include| has some restrictions by design.
Most notably, the content of a child document always occupies
its own set of pages; pages cannot be shared between child documents.
Usually, this behaviour makes perfect sense
because each child document contain an essential part of the document.
However, in some situations it may be desirable to compose
a document from a collection of parts
without having mandatory page breaks between then.
For this case, the package
provides a mechanism to include parts
by |\input| which can also be processed individually.
However, by construction this mechanism
requires manual handling of the content to be output.

%%%%%%%%%%%%%%%%%%%%%%%%%%%%%%%%%%%%%%%%
\DescribeMacro{\ifchilddocmanual}
The main file should be prepared as usual, see \secref{sec:include}.
However, the document body must make a distinction
between processing of an individual part and of the main document, e.g.:
%
\begin{center}
\begin{tabular}{l}
|\ifchilddocmanual|\\
|\input{\childdocname}|\\
|\||else|\\
\textit{document body with }|\input{|\textit{part}|}|\\
|\||fi|
\end{tabular}
\end{center}
%
The conditional |\ifchilddocmanual| is true whenever
a part to be included by |\input| is being compiled,
and the name of the part is stored in |\childdocname|.

%%%%%%%%%%%%%%%%%%%%%%%%%%%%%%%%%%%%%%%%
\DescribeMacro{\childdocby}
Each part to be included by |\input| should start with:
%
\begin{center}
\begin{tabular}{l}
|\input{childdoc.def}|\\
|\childdocby{|\textit{main}|}|\\
\end{tabular}
\end{center}
%
The directive |\childdocby| is similar to |\childdocof|
described in \secref{sec:include},
but the subsequent selection of content must be done manually.
To that end, both |\ifchilddoc| and |\ifchilddocmanual|
will be true upon processing of a part,
and the name of the part is stored in |\childdocname|.
Note that |\jobname| will be set to the filename of the current part
so that each part receives an individual |.aux| file
that does not interfere with the |.aux| file(s) of the main document.
This behaviour can be altered by the alternative form
|\childdocby[*]{|\textit{main}|}| (with a non-empty optional argument)
which uses the |.aux| file of the main document
by setting |\jobname| to \textit{main}.

%%%%%%%%%%%%%%%%%%%%%%%%%%%%%%%%%%%%%%%%%%%%%%%%%%%%%%%%%%%%%%%%%%%%%%%%%%%%%%%%
\subsection{Driver Development}
\label{sec:driver}

The \textsf{childdoc} mechanism can also be use for the development
of definition files such as \LaTeX{} styles or classes.
This case differs from the above setup with multiple parts
included by |\include| in that no |\includeonly| should be invoked.
This can be achieved by starting the include file
(before |\ProvidesPackage|) with:
%
\begin{center}
\begin{tabular}{l}
|\input{childdoc.def}|\\
|\childdocforward{|\textit{main}|}|\\
\end{tabular}
\end{center}
%
or alternatively with:
%
\begin{center}
\begin{tabular}{l}
|\input{childdoc.def}|\\
|\childdocby{|\textit{main}|}|\\
\end{tabular}
\end{center}
%
Both forms have slightly different effects as described above.
The main file is prepared as usual, see \secref{sec:include}.

%%%%%%%%%%%%%%%%%%%%%%%%%%%%%%%%%%%%%%%%%%%%%%%%%%%%%%%%%%%%%%%%%%%%%%%%%%%%%%%%
\subsection{Legacy Detection}
\label{sec:detection}

The directive |\childdocmain| in the main file can detect
whether the complete document or merely a child is to be compiled
even without using the directive |\childdocof|.
This method is deprecated because it is less robust
and there is no compelling reason to use it;
it is merely provided for backward compatibility
and it may be removed in future versions.

If the detection mechanism is to be used,
it is mandatory to correctly specify
the filename of the main file as the argument of |\childdocmain|:
%
\begin{center}
\begin{tabular}{l}
|\input{childdoc.def}|\\
|\childdocmain{|\textit{main}|}|\\
\end{tabular}
\end{center}
%
If |\jobname| does not match the argument \textit{main} of |\childdocmain|,
it is assumed that |\jobname| points to the child file to be compiled.
When using |\childdocmain| with the main file specified as argument,
it suffices to start a child file
with just |\input{|\textit{main}|}|
without loading of the package and using |\childdocof|.
If instead all processing is done
with the appropriate \textsf{childdoc} directives,
the argument of \textit{main} of |\childdocmain| can be empty.

An alternative version of the command line processing described
in \secref{sec:commandline} using the detection mechanism reads:
%
\begin{center}
|... -jobname "|\textit{target}|" "|[\textit{flags}]%
[|\def\jobname{|\textit{dest}|}|]|\input{|\textit{main}|}"|
\end{center}

%%%%%%%%%%%%%%%%%%%%%%%%%%%%%%%%%%%%%%%%%%%%%%%%%%%%%%%%%%%%%%%%%%%%%%%%%%%%%%%%
\subsection{Manual Code}
\label{sec:manual}

In case one cannot be certain whether the definitions file |childdoc.def|
is installed on the target \TeX{} distribution
and one prefers not to ship it,
it is conceivable to paste a few relevant commands into the sources.

To that end, drop all statements |\input{childdoc.def}|
and perform the replacements as outlined below.
Instead of |\childdocmain{|\textit{main}|}| add the following code
to the top of the main file:
%
\begin{center}
\begin{tabular}{l}
|\||ifdefined\childdocname\endinput\||fi\newif\ifchilddoc|\\
|\edef\childdocname{\scantokens\expandafter{\jobname\noexpand}}|\\
|\def\childdocmain{|\textit{main}|}\||ifx\childdocmain\childdocname\||else|\\
|\childdoctrue\includeonly{\childdocname}\let\jobname\childdocmain\||fi|\\
\end{tabular}
\end{center}
%
Instead of |\childdocof{|\textit{main}|}| just include the main file
at the top of each child file:
%
\begin{center}
|\input{|\textit{main}|}|
\end{center}
%
A simple redirection |\childdocforward{|\textit{dest}|}| is achieved by:
%
\begin{center}
|\def\jobname{|\textit{dest}|}\input{\jobname}|
\end{center}
%
The redirection with prefix
|\childdocforwardprefix[|\textit{prefix}|]{|\textit{dest}|}|
is accomplished by:
%
\begin{center}
\begin{tabular}{l}
|{\edef\jobname{\scantokens\expandafter{\jobname\noexpand}}|\\
|\def\redirectjob |\textit{prefix}|#1~~~{\gdef\jobname{|\textit{dest}|#1}}|\\
|\expandafter\redirectjob\jobname~~~}\input{\jobname}|
\end{tabular}
\end{center}

In an alternative approach,
child documents can be compiled by a specific command line
without additional code or specific definitions:
%
\begin{center}
|... -jobname "|\textit{target}|" "|[\textit{flags}]%
|\includeonly{|\textit{dest}|}\input{|\textit{main}|}"|
\end{center}
%

%%%%%%%%%%%%%%%%%%%%%%%%%%%%%%%%%%%%%%%%%%%%%%%%%%%%%%%%%%%%%%%%%%%%%%%%%%%%%%%%
%%%%%%%%%%%%%%%%%%%%%%%%%%%%%%%%%%%%%%%%%%%%%%%%%%%%%%%%%%%%%%%%%%%%%%%%%%%%%%%%
\section{Information}

%%%%%%%%%%%%%%%%%%%%%%%%%%%%%%%%%%%%%%%%%%%%%%%%%%%%%%%%%%%%%%%%%%%%%%%%%%%%%%%%
\subsection{Copyright}

Copyright \copyright{} 2017--2018 Niklas Beisert

This work may be distributed and/or modified under the
conditions of the \LaTeX{} Project Public License, either version 1.3
of this license or (at your option) any later version.
The latest version of this license is in
  \url{http://www.latex-project.org/lppl.txt}
and version 1.3 or later is part of all distributions of \LaTeX{}
version 2005/12/01 or later.

This work has the LPPL maintenance status `maintained'.

The Current Maintainer of this work is Niklas Beisert.

This work consists of the files |README.txt|, |childdoc.ins| and |childdoc.dtx|
as well as the derived files |childdoc.def|, |cdocsamp.tex|
with |cdocsch1.tex|, |cdocsch2.tex|, |cdocspt3.tex|, |cdocspt4.tex|,
|cdocsdrf.tex|, |cdocsfn1.tex|, |cdocsfn2.tex|
as well as |childdoc.pdf|.

%%%%%%%%%%%%%%%%%%%%%%%%%%%%%%%%%%%%%%%%%%%%%%%%%%%%%%%%%%%%%%%%%%%%%%%%%%%%%%%%
\subsection{Files and Installation}

The package consists of the files:
%
\begin{center}
\begin{tabular}{ll}
    |README.txt|   & readme file \\
    |childdoc.ins| & installation file \\
    |childdoc.dtx| & source file \\
    |childdoc.def| & definition file \\
    |cdocsamp.tex| & sample main file \\
    |cdocsch1.tex| & sample include file \\
    |cdocsch2.tex| & sample include file \\
    |cdocspt3.tex| & sample part file \\
    |cdocspt4.tex| & sample part file \\
    |cdocsdrf.tex| & sample redirection file \\
    |cdocsfn1.tex| & sample redirection file \\
    |cdocsfn2.tex| & sample redirection file \\
    |childdoc.pdf| & manual
\end{tabular}
\end{center}
%
The distribution consists of the files
|README.txt|, |childdoc.ins| and |childdoc.dtx|.
%
\begin{itemize}
\item
Run (pdf)\LaTeX{} on |childdoc.dtx|
to compile the manual |childdoc.pdf| (this file).
\item
Run \LaTeX{} on |childdoc.ins| to create the definitions file |childdoc.def|
and the sample |cdocsamp.tex| with include files
|cdocsch1.tex|, |cdocsch2.tex|, |cdocspt3.tex|, |cdocspt4.tex|,
|cdocsdrf.tex|, |cdocsfn1.tex|, |cdocsfn2.tex|.
Then copy the file |childdoc.def| to an appropriate directory of your \LaTeX{}
distribution, e.g.\ \textit{texmf-root}|/tex/latex/childdoc|.
\end{itemize}

%%%%%%%%%%%%%%%%%%%%%%%%%%%%%%%%%%%%%%%%%%%%%%%%%%%%%%%%%%%%%%%%%%%%%%%%%%%%%%%%
\subsection{Related CTAN Packages}

There are several other packages which offer a similar functionality:
%
\begin{itemize}
\item
The packages
\href{http://ctan.org/pkg/docmute}{\textsf{docmute}},
\href{http://ctan.org/pkg/includex}{\textsf{includex}} and
\href{http://ctan.org/pkg/standalone}{\textsf{standalone}}
provide commands to include only the document body of
a child file thus allowing both files to be compiled individually.
\item
The packages \href{http://ctan.org/pkg/subdocs}{\textsf{subdocs}}
and \href{http://ctan.org/pkg/subfiles}{\textsf{subfiles}}
provide structures in which the main and child documents can be
encapsulated and allowing them to be compiled individually.
The inclusion mechanism is different from the conventional |\include|.
\item
The package \href{http://ctan.org/pkg/combine}{\textsf{combine}}
is an elaborate solution to combine several documents into one.
\end{itemize}
%
See also the CTAN topic \href{http://ctan.org/topic/subdocs}{\textsf{subdocs}}
for further related packages.
The present package differs from the above solutions in that
a document structure constructed with the conventional |\include| mechanism
just needs two extra commands at the top of every file
such that all constituent files can be compiled individually.

%%%%%%%%%%%%%%%%%%%%%%%%%%%%%%%%%%%%%%%%%%%%%%%%%%%%%%%%%%%%%%%%%%%%%%%%%%%%%%%%
%\subsection{Feature Suggestions}
%
%The following is a list of features which may be useful for future
%versions of this package:
%%
%\begin{itemize}
%\item
%\ldots
%\end{itemize}

%%%%%%%%%%%%%%%%%%%%%%%%%%%%%%%%%%%%%%%%%%%%%%%%%%%%%%%%%%%%%%%%%%%%%%%%%%%%%%%%
\subsection{Revision History}

%%%%%%%%%%%%%%%%%%%%%%%%%%%%%%%%%%%%%%%%
\paragraph{v2.0:} 2018/12/30

\begin{itemize}
\item
immediate forward processing
\item
added |\childdocby| mechanism
\item
manual restructured
\end{itemize}

%%%%%%%%%%%%%%%%%%%%%%%%%%%%%%%%%%%%%%%%
\paragraph{v1.6:} 2018/01/17

\begin{itemize}
\item
application for development of include files
\item
corrections to manual
\end{itemize}

%%%%%%%%%%%%%%%%%%%%%%%%%%%%%%%%%%%%%%%%
\paragraph{v1.5:} 2017/05/21

\begin{itemize}
\item
more complete structuring introduced
\item
|\childdocof| introduced
\item
|\childdoc| renamed to |\childdocmain|
\item
|\childredirect| renamed to |\childdocforward| and |\childdocforwardprefix|
and functionality expanded
\end{itemize}

%%%%%%%%%%%%%%%%%%%%%%%%%%%%%%%%%%%%%%%%
\paragraph{v1.0:} 2017/04/27

\begin{itemize}
\item
manual and install package
\item
first version published on CTAN
\end{itemize}

%%%%%%%%%%%%%%%%%%%%%%%%%%%%%%%%%%%%%%%%
\paragraph{v0.6:} 2017/04/26

\begin{itemize}
\item
redirection mechanism added
\end{itemize}

%%%%%%%%%%%%%%%%%%%%%%%%%%%%%%%%%%%%%%%%
\paragraph{v0.5:} 2017/04/26

\begin{itemize}
\item
functionality in definition file
\end{itemize}


%%%%%%%%%%%%%%%%%%%%%%%%%%%%%%%%%%%%%%%%%%%%%%%%%%%%%%%%%%%%%%%%%%%%%%%%%%%%%%%%
%%%%%%%%%%%%%%%%%%%%%%%%%%%%%%%%%%%%%%%%%%%%%%%%%%%%%%%%%%%%%%%%%%%%%%%%%%%%%%%%
%%%%%%%%%%%%%%%%%%%%%%%%%%%%%%%%%%%%%%%%%%%%%%%%%%%%%%%%%%%%%%%%%%%%%%%%%%%%%%%%
\appendix

\settowidth\MacroIndent{\rmfamily\scriptsize 000\ }

 \DocInput{childdoc.dtx}

\end{document}
%</driver>
% \fi
%
% %%%%%%%%%%%%%%%%%%%%%%%%%%%%%%%%%%%%%%%%%%%%%%%%%%%%%%%%%%%%%%%%%%%%%%%%%%%%%%
% %%%%%%%%%%%%%%%%%%%%%%%%%%%%%%%%%%%%%%%%%%%%%%%%%%%%%%%%%%%%%%%%%%%%%%%%%%%%%%
% \section{Sample}
%\iffalse
%<*samplemain>
%\fi
%
% The following presents a sample document
% with two chapters, two parts, a title page,
% a compile flag as well as three forwarding files to set the flag.
% It consists of eight |.tex| files:
% \begin{center}
% \begin{tabular}{ll}
% |cdocsamp.tex|&main file\\
% |cdocsch1.tex|&include file for chapter 1\\
% |cdocsch2.tex|&include file for chapter 2\\
% |cdocspt3.tex|&include file for part 3\\
% |cdocspt4.tex|&include file for part 4\\
% |cdocsdrf.tex|&forwarding file for main file in draft mode\\
% |cdocsfi1.tex|&forwarding file for final version of chapter 1\\
% |cdocsfi2.tex|&forwarding file for final version of chapter 2\\
% \end{tabular}
% \end{center}
% Each of the eight files can be compiled directly by the \LaTeX{} compiler.
%
% %%%%%%%%%%%%%%%%%%%%%%%%%%%%%%%%%%%%%%
% \paragraph{Main File.}
%
% The main file is called |cdocsamp.tex|.
%
% Load the \textsf{childdoc} definitions and
% declare the filename for the main document:
%    \begin{macrocode}
\input{childdoc.def}
\childdocmain{}
%    \end{macrocode}

% Optional override for |\version| flag:
%    \begin{macrocode}
%%\ifchilddoc\else\providecommand{\version}{draft}\fi
%    \end{macrocode}

% Define the default values for the |\version| flag
% (|final| for the main file and |draft| for childs):
%    \begin{macrocode}
\ifchilddoc
\providecommand{\version}{draft}
\else
\providecommand{\version}{final}
\fi
%    \end{macrocode}

% Load the standard document class:
%    \begin{macrocode}
\documentclass[12pt]{article}
%    \end{macrocode}

% Start the document body:
%    \begin{macrocode}
\begin{document}
%    \end{macrocode}

% Declare a title page.
% Print title, part of document being processed and version flag:
%    \begin{macrocode}
\addtocounter{page}{-1}
\begin{center}
{\LARGE\bfseries{}childdoc example\par}
\vspace{1cm}
\ifchilddoc
\ifchilddocmanual part\else chapter\fi:
`\childdocname' of `\childdocjob'\par
\else
main document: `\childdocjob'\par
\fi
version: \version\par
\end{center}
\newpage
%    \end{macrocode}

% Manually include selected file,
% otherwise process as usual:
%    \begin{macrocode}
\ifchilddocmanual
\section*{part `\childdocname'}
\input{\childdocname}
\else
%    \end{macrocode}

% Include the two chapters:
%    \begin{macrocode}
\include{cdocsch1}
\include{cdocsch2}
%    \end{macrocode}

% Include the two parts unless only chapters should be displayed:
%    \begin{macrocode}
\ifchilddoc\else
\section{part three}
\input{cdocspt3}
\section{part four}
\input{cdocspt4}
\fi
%    \end{macrocode}

% Process as usual until here:
%    \begin{macrocode}
\fi
%    \end{macrocode}

% End of document body:
%    \begin{macrocode}
\end{document}
%    \end{macrocode}
%\iffalse
%</samplemain>
%\fi
%
% %%%%%%%%%%%%%%%%%%%%%%%%%%%%%%%%%%%%%%
% \paragraph{Chapter Include Files.}
%
% The include files are called |cdocsch1.tex| and |cdocsch2.tex|.
%
%\iffalse
%<*samplechap1|samplechap2>
%\fi

% Optional override for |\version| flag:
%    \begin{macrocode}
%%\providecommand{\version}{final}
%    \end{macrocode}

% Include the main document:
%    \begin{macrocode}
\input{childdoc.def}
\childdocof{cdocsamp}
%    \end{macrocode}

%\iffalse
%</samplechap1|samplechap2>
%\fi
%
%\iffalse
%<*samplechap1>
%\fi
% Some text for chapter 1:
%    \begin{macrocode}
\section{one}
some text in chapter one
%    \end{macrocode}

%\iffalse
%</samplechap1>
%\fi
% Some text for chapter 2:
%\iffalse
%<*samplechap2>
%\fi
%    \begin{macrocode}
\section{two}
more text in chapter two
%    \end{macrocode}

%\iffalse
%</samplechap2>
%\fi
%
% %%%%%%%%%%%%%%%%%%%%%%%%%%%%%%%%%%%%%%
% \paragraph{Part Include Files.}
%
% The include files are called |cdocspt3.tex| and |cdocspt4.tex|.
%
%\iffalse
%<*samplepart3|samplepart4>
%\fi

% Optional override for |\version| flag:
%    \begin{macrocode}
%%\providecommand{\version}{final}
%    \end{macrocode}

% Include the main document:
%    \begin{macrocode}
\input{childdoc.def}
\childdocby{cdocsamp}
%    \end{macrocode}

%\iffalse
%</samplepart3|samplepart4>
%\fi
%
%\iffalse
%<*samplepart3>
%\fi
% Some text for part 3:
%    \begin{macrocode}
some text in part three
%    \end{macrocode}

%\iffalse
%</samplepart3>
%\fi
% Some text for part 4:
%\iffalse
%<*samplepart4>
%\fi
%    \begin{macrocode}
more text in part four
%    \end{macrocode}

%\iffalse
%</samplepart4>
%\fi
%
% %%%%%%%%%%%%%%%%%%%%%%%%%%%%%%%%%%%%%%
% \paragraph{Forwarding for a Complete Draft.}
%
% The following forwarding file |cdocsdrf.tex|
% compiles the main document in draft mode:
%\iffalse
%<*sampledraft>
%\fi
%    \begin{macrocode}
\def\version{draft}
\input{childdoc.def}
\childdocforward{cdocsamp}
%    \end{macrocode}

%\iffalse
%</sampledraft>
%\fi
%
% %%%%%%%%%%%%%%%%%%%%%%%%%%%%%%%%%%%%%%
% \paragraph{Forwarding for Final Version of the Chapters.}
%
% The following forwarding files |cdocsfn1.tex| and |cdocsfn2.tex|
% (with identical content)
% compile the final versions of the child documents
% |cdocsch1.tex| and |cdocsch2.tex|, respectively:
%\iffalse
%<*samplefinal>
%\fi
%    \begin{macrocode}
\def\version{final}
\input{childdoc.def}
\childdocforwardprefix[cdocsamp]{cdocsfn}{cdocsch}
%    \end{macrocode}

%\iffalse
%</samplefinal>
%\fi
%
% %%%%%%%%%%%%%%%%%%%%%%%%%%%%%%%%%%%%%%
% \paragraph{Command Line Processing.}
%
% The following three command lines generate the output files
% |cdocscld|, |cdocscl1| and |cdocscl2|
% which should be identical to
% |cdocsdrf|, |cdocsch1| and |cdocsfn2|, respectively:
% \begin{center}
% \begin{tabular}{l}
% |latex -jobname cdocscld \|\\
% |  "\def\version{draft}\input{childdoc.def}\childdocforward{cdocsamp}"|\\
% |latex -jobname cdocscl1 \|\\
% |  "\input{childdoc.def}\childdocforward[cdocsamp]{cdocsch1}"|\\
% |latex -jobname cdocscl2 \|\\
% |  "\def\version{final}\input{childdoc.def}\childdocforward{cdocsch2}"|
% \end{tabular}
% \end{center}
% Note that the trailing backslash on each first line
% merely continues the input to the second line
% (for convenient cut ant paste).
% Furthermore, the command |latex| can be replaced by any
% of its alternative versions such as |pdflatex|.
%
% %%%%%%%%%%%%%%%%%%%%%%%%%%%%%%%%%%%%%%%%%%%%%%%%%%%%%%%%%%%%%%%%%%%%%%%%%%%%%%
% %%%%%%%%%%%%%%%%%%%%%%%%%%%%%%%%%%%%%%%%%%%%%%%%%%%%%%%%%%%%%%%%%%%%%%%%%%%%%%
% \section{Implementation}
%\iffalse
%<*package>
%\fi
%
% This section describes the definitions file |childdoc.def|.

% The definitions cannot be loaded using |\usepackage| or |\RequirePackage|
% which has a mechanism to prevent loading a style file more than once.
% When loading the definitions by means of |\input|
% multiple instances have to be prevented manually:
%\iffalse
%This code needs to be before the `\ProvidesFile' directive
%which is defined at the beginning of this file.
%Therefore it is also placed there and commented out here.
%</package>
%<*discard>
%\fi
%    \begin{macrocode}
\ifdefined\childdocmain\endinput\fi
%    \end{macrocode}
%\iffalse
%</discard>
%<*package>
%\fi
%
% \macro{\ifchilddoc}
% \macro{\ifchilddocmanual}
% The conditional |\ifchilddoc| tells whether a
% child (true) or main (false) document is being compiled.
% The conditional |\ifchilddocmanual| tells whether
% the |\includeonly| mechanism is used (false) or
% the selection of child files must be performed manually (true).
% The definitions initialise to false:
%    \begin{macrocode}
\newif\ifchilddoc
\newif\ifchilddocmanual
%    \end{macrocode}

% \macro{\childdocname}
% \macro{\childdocjob}
% The macro |\childdocname| stores the name of the main document
% to be compiled. The macro |\childdocjob| stores the name of
% the document on which the \LaTeX{} compiler was originally invoked.
% The content of |\jobname| cannot be compared
% to filenames specified in the source due to different catcodes.
% The following code rescans |\jobname|, stores the result
% in |\childdocname| and saves a copy in |\childdocjob|:
%    \begin{macrocode}
\edef\childdocname{\scantokens\expandafter{\jobname\noexpand}}
\let\childdocjob\childdocname
%    \end{macrocode}

% \macro{\childdocdisable}
% The macro |\childdocdisable| prevents the main file
% from being processed more than once.
% At this stage, the main document command |\childdocmain|
% is assumed to be called once again where it should do nothing.
% Any subsequent call to it should prevent
% a secondary processing of the main document
% It overwrites the forwarding commands
% |\childdocof| and |\childdocforward|
% with empty macros to prevent further inclusions of the main document:
%    \begin{macrocode}
\newcommand{\childdocdisable}
{
  \renewcommand{\childdocmain}[1]{\renewcommand{\childdocmain}[1]{\endinput}}
  \renewcommand{\childdocof}[1]{}
  \renewcommand{\childdocby}[2][]{}
  \renewcommand{\childdocforward}[2][]{}
  \renewcommand{\childdocdisable}{}
}
%    \end{macrocode}

% \macro{\childdocmain}
% The macro |\childdocmain| is to be called at the top of the main file
% with nothing or the main filename (without extension) as argument.
% First, it breaks loops.
% If the argument is not empty and does not match |\childdocname|
% (which is set by the first inclusion of |childdoc.def|),
% |\ifchilddoc| is set to true, |\includeonly| is applied to the child file
% and |\jobname| is set to the main file
% (for proper handling of |.aux| files):
%    \begin{macrocode}
\newcommand{\childdocmain}[1]
{
  \childdocdisable\childdocmain{}
  \if?#1?\else
    \begingroup
      \def\childdoctmp{#1}
      \ifx\childdoctmp\childdocname
        \def\childdoctmp{}
      \else
        \def\childdoctmp
        {
          \childdoctrue
          \includeonly{\childdocname}
          \def\childdocjob{#1}
          \def\jobname{#1}
        }
      \fi
      \expandafter
    \endgroup
    \childdoctmp
  \fi
}
%    \end{macrocode}

% \macro{\childdocof}
% The command |\childdocof| redirects
% compilation to the main file |#1|.
%    \begin{macrocode}
\newcommand{\childdocof}[1]
{
  \childdocdisable
  \childdoctrue
  \includeonly{\childdocname}
  \def\jobname{#1}
  \def\childdocjob{#1}
  \input{#1}
}
%    \end{macrocode}

% \macro{\childdocby}
% The command |\childdocby| ....
%    \begin{macrocode}
\newcommand{\childdocby}[2][]
{
  \childdocdisable
  \childdoctrue
  \childdocmanualtrue
  \if?#1?\else
    \def\jobname{#2}
  \fi
  \def\childdocjob{#2}
  \input{#2}
  \endinput
}
%    \end{macrocode}

% \macro{\childdocforward}
% The command |\childdocforward| redirects
% compilation to the main file or
% (if the optional argument is given) a child file.
% Parameters are set as if the main file
% or a child file starting with |\childdocof| was compiled.
% Then compilation is handed over to the main file:
%    \begin{macrocode}
\newcommand{\childdocforward}[2][]
{
  \begingroup
    \if?#1?
      \def\childdoctmp
      {
        \def\childdocname{#2}
        \def\childdocjob{#2}
        \def\jobname{#2}
        \input{#2}
        \endinput
      }
    \else
      \def\childdoctmp
      {
        \childdocdisable
        \def\childdocname{#2}
        \childdoctrue
        \includeonly{#2}
        \def\childdocjob{#1}
        \def\jobname{#1}
        \input{#1}
        \endinput
      }
    \fi
    \expandafter
  \endgroup
  \childdoctmp
}
%    \end{macrocode}

% \macro{\childdocforwardprefix}
% The command |\childdocforwardprefix| redirects
% compilation to the main or a child file by means of a pattern.
% The prefix |#1| in the current filename is replaced by |#2|
% and the suffix of the current filename is kept
% (it is assumed that the filename does not contain the substring `|~~~|'
% which is used as a delimiter).
% Compilation is handed over to the new file by |\childdocforward|:
%    \begin{macrocode}
\newcommand{\childdocforwardprefix}[3][]
{
  \begingroup
    \def\childdocextract #2##1~~~{\def\childdoctmp{\childdocforward[#1]{#3##1}}}
    \expandafter\childdocextract\childdocname~~~
    \expandafter
  \endgroup
  \childdoctmp
}
%    \end{macrocode}

% \macro{\childdoc}
% The deprecated macro |\childdoc| is a legacy version of |\childdocmain|:
%    \begin{macrocode}
\newcommand{\childdoc}{\childdocmain}
%    \end{macrocode}

% \macro{\childdocredirect}
% The deprecated macro |\childdocredirect| is a legacy version
% of |\childdocforward| and |\childdocforwardprefix|:
%    \begin{macrocode}
\newcommand{\childdocredirect}[2][]
{
  \begingroup
    \if?#1?
      \def\childdoctmp{\childdocforward{#2}}
    \else
      \def\childdoctmp{\childdocforwardprefix{#1}{#2}}
    \fi
    \expandafter
  \endgroup
  \childdoctmp
}
%    \end{macrocode}

%\iffalse
%</package>
%\fi
%
\endinput
\childdocforward{cdocsamp}"|\\
% |latex -jobname cdocscl1 \|\\
% |  "% \iffalse
%
% childdoc.dtx Copyright (C) 2017-2018 Niklas Beisert
%
% This work may be distributed and/or modified under the
% conditions of the LaTeX Project Public License, either version 1.3
% of this license or (at your option) any later version.
% The latest version of this license is in
%   http://www.latex-project.org/lppl.txt
% and version 1.3 or later is part of all distributions of LaTeX
% version 2005/12/01 or later.
%
% This work has the LPPL maintenance status `maintained'.
%
% The Current Maintainer of this work is Niklas Beisert.
%
% This work consists of the files childdoc.dtx and childdoc.ins
% and the derived files childdoc.def and cdocsamp.tex with
% cdocsch1.tex, cdocsch2.tex, cdocsdrf.tex, cdocsfn1.tex, cdocsfn2.tex.
%
%<package>\ifdefined\childdocmain\endinput\fi
%<package>\ProvidesFile{childdoc.def}[2018/12/30 v2.0 child document driver]
%<samplemain>\ProvidesFile{cdocsamp.tex}[2018/12/30 v2.0 sample for childdoc]
%<*driver>
%\ProvidesFile{childdoc.drv}[2018/12/30 v2.0 childdoc reference manual file]
\PassOptionsToClass{10pt,a4paper}{article}
\documentclass{ltxdoc}

\usepackage[margin=35mm]{geometry}
\usepackage{hyperref}
\usepackage{hyperxmp}
\usepackage[usenames]{color}

\hypersetup{colorlinks=true}
\hypersetup{pdfstartview=FitH}
\hypersetup{pdfpagemode=UseNone}
\hypersetup{pdfsource={}}
\hypersetup{pdflang={en-UK}}
\hypersetup{pdfcopyright={Copyright 2017-2018 Niklas Beisert.
  This work may be distributed and/or modified under the
  conditions of the LaTeX Project Public License, either version 1.3
  of this license or (at your option) any later version.}}
\hypersetup{pdflicenseurl={http://www.latex-project.org/lppl.txt}}
\hypersetup{pdfcontactaddress={ETH Zurich, ITP, HIT K,
  Wolfgang-Pauli-Strasse 27}}
\hypersetup{pdfcontactpostcode={8093}}
\hypersetup{pdfcontactcity={Zurich}}
\hypersetup{pdfcontactcountry={Switzerland}}
\hypersetup{pdfcontactemail={nbeisert@itp.phys.ethz.ch}}
\hypersetup{pdfcontacturl={http://people.phys.ethz.ch/\xmptilde nbeisert/}}

\newcommand{\secref}[1]{\hyperref[#1]{section \ref*{#1}}}

\parskip1ex
\parindent0pt
\let\olditemize\itemize
\def\itemize{\olditemize\parskip0pt}

\begin{document}

\title{The \textsf{childdoc} Package}
\hypersetup{pdftitle={The childdoc Package}}
\author{Niklas Beisert\\[2ex]
  Institut f\"ur Theoretische Physik\\
  Eidgen\"ossische Technische Hochschule Z\"urich\\
  Wolfgang-Pauli-Strasse 27, 8093 Z\"urich, Switzerland\\[1ex]
  \href{mailto:nbeisert@itp.phys.ethz.ch}
  {\texttt{nbeisert@itp.phys.ethz.ch}}}
\hypersetup{pdfauthor={Niklas Beisert}}
\hypersetup{pdfsubject={Manual for the LaTeX2e Package childdoc}}
\date{30 December 2018, \textsf{v2.0}}
\maketitle

\begin{abstract}\noindent
\textsf{childdoc} is a \LaTeXe{} package
that enables the direct compilation
of document sections included by |\include|
to individual files.
\end{abstract}

\begingroup
\parskip0ex
\tableofcontents
\endgroup

%%%%%%%%%%%%%%%%%%%%%%%%%%%%%%%%%%%%%%%%%%%%%%%%%%%%%%%%%%%%%%%%%%%%%%%%%%%%%%%%
%%%%%%%%%%%%%%%%%%%%%%%%%%%%%%%%%%%%%%%%%%%%%%%%%%%%%%%%%%%%%%%%%%%%%%%%%%%%%%%%
\section{Introduction}

\LaTeX{} provides a mechanism to structure a large document (such as a book)
into a main file and several child files (containing the chapters)
using the |\include| command.
This mechanism is beneficial for documents
which span hundreds of pages in order to
make the source file(s) more manageable.
Moreover, compilation can be restricted to
selected child files by means of the |\includeonly| command.
The latter feature can be used to reduce the compilation time while editing
(this was significantly more useful in the earlier days of \LaTeX{})
or to generate a smaller document which is easier to navigate.
Another application of |\includeonly| is to generate
documents consisting of selected parts of the complete document.

However, there are a few drawbacks of the plain |\include| mechanism:
\begin{itemize}
\item
The child files cannot be compiled on their own,
they can only be compiled via the main file.
A naive editing environment
(such as a text editor with an option
to have the current file processed by \LaTeX)
may require one to switch to the main file before compiling;
attempting to compile the child file produces errors.
\item
The main file must be modified (each time)
to adjust the |\includeonly| command
to the present needs. This easily leaves the main file in a messy state.
\item
The generated document will always carry the filename
of the main document. This is inconvenient if
several child files are to be compiled and
to be kept for distribution.
\end{itemize}

The present package provides a simple interface
to make child files individually compilable by \LaTeX{}.
Compiling a child file then has the same effect as compiling
the main file with an |\includeonly| command
to select the appropriate child.
Moreover the generated document will carry the name of the child
rather than the main file.
This resolves all three above issues.

This feature is meant to make the editing of books,
thesis documents and lecture notes somewhat more convenient.
However, the package can also be used efficiently for
composing a series of documents (such as exercise sheets)
which are typically distributed individually.
It then assists the author in generating the individual documents
(potentially in different versions)
as well as a document containing the collected series.
Another application is in developing style files
or other kinds of included material
where compilation of the style file could redirect
to a sample or test file.

%%%%%%%%%%%%%%%%%%%%%%%%%%%%%%%%%%%%%%%%%%%%%%%%%%%%%%%%%%%%%%%%%%%%%%%%%%%%%%%%
%%%%%%%%%%%%%%%%%%%%%%%%%%%%%%%%%%%%%%%%%%%%%%%%%%%%%%%%%%%%%%%%%%%%%%%%%%%%%%%%
\section{Usage}

First of all, the package \textsf{childdoc} is \emph{not} a standard
\LaTeXe{} |.sty| style file! Therefore it needs to be invoked in
a non-standard way.

%%%%%%%%%%%%%%%%%%%%%%%%%%%%%%%%%%%%%%%%%%%%%%%%%%%%%%%%%%%%%%%%%%%%%%%%%%%%%%%%
\subsection{Included Files}
\label{sec:include}

%%%%%%%%%%%%%%%%%%%%%%%%%%%%%%%%%%%%%%%%
\DescribeMacro{\childdocmain}
To use the package, add the commands
\begin{center}
\begin{tabular}{l}
|\input{childdoc.def}|\\
|\childdocmain{}|\\
\end{tabular}
\end{center}
at the very top of the main \LaTeX{} file,
in particular \emph{before} the |\documentclass| statement!
The argument of |\childdocmain| should be left empty
(but it must be present).

%%%%%%%%%%%%%%%%%%%%%%%%%%%%%%%%%%%%%%%%
\DescribeMacro{\childdocof}
Furthermore, add the commands
\begin{center}
\begin{tabular}{l}
|\input{childdoc.def}|\\
|\childdocof{|\textit{main}|}|\\
\end{tabular}
\end{center}
at the top of every child file \textit{child}
which is included by |\include{|\textit{child}|}|
from within the main file
(or at least for those files to be compiled individually).
The argument \textit{main} must be the filename of the main file.

There are a couple of
considerations in setting up the main and child documents:

%%%%%%%%%%%%%%%%%%%%%%%%%%%%%%%%%%%%%%%%
\paragraph{Restrictions.}

Please note the following restrictions:
\begin{itemize}
\item
|\childdocmain| must be called with one argument \textit{main}
to ensure compatibility with earlier version of the package.
It must either be empty (|\childdocmain{}|)
or precisely match the filename of the main file in which it is specified.
See \secref{sec:detection} for further information.
\item
The filename \textit{main} must be specified without the |.tex| extension.
\item
The filename \textit{main} is case sensitive
(even in case-insensitive file systems)
due to internal string comparison.
\item
The argument \textit{main} should be fully expanded, it cannot be a macro.
\item
Subdirectories and special characters should be avoided in filenames.
\item
The command |\childdocmain{|\textit{main}|}| must be followed by a whitespace.
It should not be followed immediately by another command
or by a comment mark `|%|'.
This is because the \TeX{} parser reads the token immediately following
the argument of |\childdocmain| and puts it
at the beginning of every child section;
however, a white\-space is ignored.
\end{itemize}

%%%%%%%%%%%%%%%%%%%%%%%%%%%%%%%%%%%%%%%%
\paragraph{Content of Main File.}

It is advisable to place all content in the child files included by |\include|.
Any output contained in the main file will appear in all child documents
unless suppressed manually;
it cannot be suppressed automatically by the |\includeonly| directive
and thus should normally be avoided.
A method to include some content in the main file
by means of conditional processing is described in \secref{sec:conditional}.

%%%%%%%%%%%%%%%%%%%%%%%%%%%%%%%%%%%%%%%%
\paragraph{Page Numbering.}

When only a part of the document is compiled,
the appropriate numbering of pages
(as well as other status parameters)
is determined from the |.aux| files.
The latter contain information from previous passes.
However this information needs to propagate through
all intermediate child documents.
Therefore the page numbering in child documents may well
be inconsistent until the complete document is compiled at least once.

A useful (if unconventional) way to always ensure a consistent
page numbering is to restart the numbering in each child document
and denote the pages by `\textit{child}|.|\textit{page}'
where \textit{child} represents the chapter/section number of the child file.
This can be achieved by the command
|\numberwithin{page}{|\textit{child}|}|
of the \textsf{amsmath} package
where \textit{child} can be |chapter| or |section|
depending on the chosen structuring.
Alternatively, one can modify the macro |\thepage| appropriately
and reset the counter |page| at the start of each child file.

%%%%%%%%%%%%%%%%%%%%%%%%%%%%%%%%%%%%%%%%%%%%%%%%%%%%%%%%%%%%%%%%%%%%%%%%%%%%%%%%
\subsection{Conditional Processing}
\label{sec:conditional}

The package provides a mechanism to compile different versions
of a document. To customise the versions further some conditional processing
can come in handy to distinguish which version is being compiled.
The package provides two macros to describe the compilation context:

%%%%%%%%%%%%%%%%%%%%%%%%%%%%%%%%%%%%%%%%
\DescribeMacro{\ifchilddoc}
The conditional |\ifchilddoc| distinguishes between the compilation of
child documents and the main document:
%
\begin{center}
|\ifchilddoc |\textit{child-code}| |[|\||else |\textit{main-code}]| \||fi|
\end{center}

%%%%%%%%%%%%%%%%%%%%%%%%%%%%%%%%%%%%%%%%
\DescribeMacro{\childdocname}
\DescribeMacro{\childdocjob}
The macro |\childdocname| contains the filename (without extension)
of the main or child file being processed.
Note that |\childdocjob| will always contain the name of the main file.

%%%%%%%%%%%%%%%%%%%%%%%%%%%%%%%%%%%%%%%%
\paragraph{Title Page.}

Conditional processing can be used to include a title or banner page
in the main document when proper precautions are taken.
Importantly, the code in the main file should ensure that the page counter
(as well as other status parameters which are stored in the |.aux| files)
takes the same value after the conditional processing.
Otherwise the page numbers may take divergent values
depending on which part is compiled.

For example, a title page could be declared by:
%
\begin{center}
\begin{tabular}{l}
|\ifchilddoc\||else|\\
|\addtocounter{page}{-1}|\\
\textit{code for title page}\\
|\newpage|\\
|\||fi|
\end{tabular}
\end{center}
%
A banner page for the child documents can be generated by:
%
\begin{center}
\begin{tabular}{l}
|\ifchilddoc|\\
|\addtocounter{page}{-1}|\\
\textit{code for banner page}\\
|\newpage|\\
|\||fi|
\end{tabular}
\end{center}
%
Here one could write a message such as:
\begin{center}
|This is the part \childdocname{} of \childdocjob{}.|
\end{center}

%%%%%%%%%%%%%%%%%%%%%%%%%%%%%%%%%%%%%%%%%%%%%%%%%%%%%%%%%%%%%%%%%%%%%%%%%%%%%%%%
\subsection{Flags}
\label{sec:flags}

The package makes it easy to generate different versions
of the main or child documents.
To this end compilation flags can be defined
and assigned different default values.
They will be particularly useful in conjunction
with the forwarding mechanism described in \secref{sec:forward}.

For example, it may be useful to have a flag |\version|
which can be set to |draft| or |final|.
The document source will contain some conditional code
depending on the value of |\version|.
Suppose further, the flag should default to |final| for the main file
and to |draft| for child files
which is a natural assignment for editing the document.
This is achieved by placing the following code
in the preamble of the main document
(below the |\childdocmain| directive):
%
\begin{center}
\begin{tabular}{l}
|\ifchilddoc|\\
|\providecommand{\version}{draft}|\\
|\||else|\\
|\providecommand{\version}{final}|\\
|\||fi|
\end{tabular}
\end{center}
%
The definition by |\providecommand| makes sure
that previous definitions are not overwritten.
Further statements |\providecommand{\version}{...}|
can thus be added before the above code to override it.

For the main file, one might add a line
(between |\childdocmain| and the above block)
%
\begin{center}
|%\ifchilddoc\||else\providecommand{\version}{draft}\||fi|
\end{center}
%
which can be uncommented to produce a draft version.
Likewise one can add a line to the very top of a child file
(above the |\childdocof{|\textit{main}|}| directive)
%
\begin{center}
|%\providecommand{\version}{final}|
\end{center}
%
which can be uncommented to produce the final version of this child document.

%%%%%%%%%%%%%%%%%%%%%%%%%%%%%%%%%%%%%%%%%%%%%%%%%%%%%%%%%%%%%%%%%%%%%%%%%%%%%%%%
\subsection{Forwarding}
\label{sec:forward}

Different versions of the main or child documents
using compilation flags as described in \secref{sec:flags}
can be (permanently) stored in different files
for convenient compilation, viewing and distribution.
To this end, the package defines a command
to pass on compilation to a different file:

%%%%%%%%%%%%%%%%%%%%%%%%%%%%%%%%%%%%%%%%
\DescribeMacro{\childdocforward}
The command |\childdocforward| redirects processing to
another source file:
%
\begin{center}
\begin{tabular}{l}
|\input{childdoc.def}|\\
|\childdocforward[|\textit{main}|]{|\textit{dest}|}|\\
\end{tabular}
\end{center}
%
The argument \textit{dest} is the destination file
(without extension).
It should be the main file or one of the child files.
Note that further \textsf{childdoc} directives
such as |\childdocof| and |\childdocforward|
in the indicated file will be processed in this form.
The optional argument \textit{main}
passes on directly to the main file \textit{main}
while pretending to compile the child \textit{dest}.
This form behaves as if \textit{dest}
issues |\childdocof{|\textit{main}|}| right away,
and no further \textsf{childdoc} directives will be processed.

%%%%%%%%%%%%%%%%%%%%%%%%%%%%%%%%%%%%%%%%
\DescribeMacro{\...prefix}
In the alternative form |\childdocforwardprefix|,
%
\begin{center}
\begin{tabular}{l}
|\input{childdoc.def}|\\
|\childdocforwardprefix[|\textit{main}|]{|\textit{prefix}|}{|\textit{dest}|}|
\end{tabular}
\end{center}
%
the destination file is determined by a pattern
depending on the current file:
To make this work, the current file must be called
`{\textit{prefix}\hspace{0.2em}\textit{suffix}}'
with \textit{prefix} matching precisely the argument.
Processing is then passed on to the file
`{\textit{dest}\hspace{0.2em}\textit{suffix}}'.
Surely, the same effect is achieved by
directly specifying the
argument `{\textit{dest}\hspace{0.2em}\textit{suffix}}'
in the first form.
However, that requires to set up a different file
for each child. With the alternative form of the command
all these files can have exactly the same content
which simplifies setting them up and maintaining them.

For example, the following file |draft.tex|
with a compilation flag |\version| as described in \secref{sec:flags}
compiles the main document as a draft:
%
\begin{center}
\begin{tabular}{l}
|\def\version{draft}|\\
|\input{childdoc.def}|\\
|\childdocforward{|\textit{main}|}|
\end{tabular}
\end{center}
%
Likewise, the following files |final|\textit{nn}|.tex|
compile the final version of the child document
|child|\textit{nn}|.tex|:
%
\begin{center}
\begin{tabular}{l}
|\def\version{final}|\\
|\input{childdoc.def}|\\
|\childdocforwardprefix{final}{child}|
\end{tabular}
\end{center}
%

Note that when several versions of a main file and/or of each child file
are to be generated, it may be convenient to set up a |Makefile| or
shell script to automatise the process.

%%%%%%%%%%%%%%%%%%%%%%%%%%%%%%%%%%%%%%%%%%%%%%%%%%%%%%%%%%%%%%%%%%%%%%%%%%%%%%%%
\subsection{Command Line Processing}
\label{sec:commandline}

The effect of redirection files can also be achieved by invoking
the \LaTeX{} compiler with a more elaborate command line.
Most conveniently this should be done as part
of a shell script or a |Makefile|.

When using \textsf{childdoc} in the main file, the following
command lines effectively perform a redirection
(note that depending on the shell being used,
backslashes may have to be doubled: `|\|' $\to$ `|\\|'):
%
\begin{center}
|... -jobname "|\textit{target}|" |\\|"|[\textit{flags}]%
|\input{childdoc.def}\childdocforward[|\textit{main}|]{|\textit{dest}|}"|
\end{center}
%
Here \textit{target} is the name of the output file,
\textit{main} is the name of the main file
and \textit{dest} is the name of the main or child file to be processed
(all filenames without extensions).
The optional argument \textit{main} can be omitted
if \textit{main} matches \textit{dest}.
Optionally, compilation \textit{flags} can be defined via |\def| commands.
This command line makes the \TeX{} engine believe
it is compiling the file \textit{target}
whose content is specified as the latter parameter.
The provided code then forwards the processing to
\textit{main} or \textit{dest} as described in \secref{sec:forward}.

%%%%%%%%%%%%%%%%%%%%%%%%%%%%%%%%%%%%%%%%%%%%%%%%%%%%%%%%%%%%%%%%%%%%%%%%%%%%%%%%
\subsection{Include by Input}
\label{sec:input}

Including child documents by |\include| has some restrictions by design.
Most notably, the content of a child document always occupies
its own set of pages; pages cannot be shared between child documents.
Usually, this behaviour makes perfect sense
because each child document contain an essential part of the document.
However, in some situations it may be desirable to compose
a document from a collection of parts
without having mandatory page breaks between then.
For this case, the package
provides a mechanism to include parts
by |\input| which can also be processed individually.
However, by construction this mechanism
requires manual handling of the content to be output.

%%%%%%%%%%%%%%%%%%%%%%%%%%%%%%%%%%%%%%%%
\DescribeMacro{\ifchilddocmanual}
The main file should be prepared as usual, see \secref{sec:include}.
However, the document body must make a distinction
between processing of an individual part and of the main document, e.g.:
%
\begin{center}
\begin{tabular}{l}
|\ifchilddocmanual|\\
|\input{\childdocname}|\\
|\||else|\\
\textit{document body with }|\input{|\textit{part}|}|\\
|\||fi|
\end{tabular}
\end{center}
%
The conditional |\ifchilddocmanual| is true whenever
a part to be included by |\input| is being compiled,
and the name of the part is stored in |\childdocname|.

%%%%%%%%%%%%%%%%%%%%%%%%%%%%%%%%%%%%%%%%
\DescribeMacro{\childdocby}
Each part to be included by |\input| should start with:
%
\begin{center}
\begin{tabular}{l}
|\input{childdoc.def}|\\
|\childdocby{|\textit{main}|}|\\
\end{tabular}
\end{center}
%
The directive |\childdocby| is similar to |\childdocof|
described in \secref{sec:include},
but the subsequent selection of content must be done manually.
To that end, both |\ifchilddoc| and |\ifchilddocmanual|
will be true upon processing of a part,
and the name of the part is stored in |\childdocname|.
Note that |\jobname| will be set to the filename of the current part
so that each part receives an individual |.aux| file
that does not interfere with the |.aux| file(s) of the main document.
This behaviour can be altered by the alternative form
|\childdocby[*]{|\textit{main}|}| (with a non-empty optional argument)
which uses the |.aux| file of the main document
by setting |\jobname| to \textit{main}.

%%%%%%%%%%%%%%%%%%%%%%%%%%%%%%%%%%%%%%%%%%%%%%%%%%%%%%%%%%%%%%%%%%%%%%%%%%%%%%%%
\subsection{Driver Development}
\label{sec:driver}

The \textsf{childdoc} mechanism can also be use for the development
of definition files such as \LaTeX{} styles or classes.
This case differs from the above setup with multiple parts
included by |\include| in that no |\includeonly| should be invoked.
This can be achieved by starting the include file
(before |\ProvidesPackage|) with:
%
\begin{center}
\begin{tabular}{l}
|\input{childdoc.def}|\\
|\childdocforward{|\textit{main}|}|\\
\end{tabular}
\end{center}
%
or alternatively with:
%
\begin{center}
\begin{tabular}{l}
|\input{childdoc.def}|\\
|\childdocby{|\textit{main}|}|\\
\end{tabular}
\end{center}
%
Both forms have slightly different effects as described above.
The main file is prepared as usual, see \secref{sec:include}.

%%%%%%%%%%%%%%%%%%%%%%%%%%%%%%%%%%%%%%%%%%%%%%%%%%%%%%%%%%%%%%%%%%%%%%%%%%%%%%%%
\subsection{Legacy Detection}
\label{sec:detection}

The directive |\childdocmain| in the main file can detect
whether the complete document or merely a child is to be compiled
even without using the directive |\childdocof|.
This method is deprecated because it is less robust
and there is no compelling reason to use it;
it is merely provided for backward compatibility
and it may be removed in future versions.

If the detection mechanism is to be used,
it is mandatory to correctly specify
the filename of the main file as the argument of |\childdocmain|:
%
\begin{center}
\begin{tabular}{l}
|\input{childdoc.def}|\\
|\childdocmain{|\textit{main}|}|\\
\end{tabular}
\end{center}
%
If |\jobname| does not match the argument \textit{main} of |\childdocmain|,
it is assumed that |\jobname| points to the child file to be compiled.
When using |\childdocmain| with the main file specified as argument,
it suffices to start a child file
with just |\input{|\textit{main}|}|
without loading of the package and using |\childdocof|.
If instead all processing is done
with the appropriate \textsf{childdoc} directives,
the argument of \textit{main} of |\childdocmain| can be empty.

An alternative version of the command line processing described
in \secref{sec:commandline} using the detection mechanism reads:
%
\begin{center}
|... -jobname "|\textit{target}|" "|[\textit{flags}]%
[|\def\jobname{|\textit{dest}|}|]|\input{|\textit{main}|}"|
\end{center}

%%%%%%%%%%%%%%%%%%%%%%%%%%%%%%%%%%%%%%%%%%%%%%%%%%%%%%%%%%%%%%%%%%%%%%%%%%%%%%%%
\subsection{Manual Code}
\label{sec:manual}

In case one cannot be certain whether the definitions file |childdoc.def|
is installed on the target \TeX{} distribution
and one prefers not to ship it,
it is conceivable to paste a few relevant commands into the sources.

To that end, drop all statements |\input{childdoc.def}|
and perform the replacements as outlined below.
Instead of |\childdocmain{|\textit{main}|}| add the following code
to the top of the main file:
%
\begin{center}
\begin{tabular}{l}
|\||ifdefined\childdocname\endinput\||fi\newif\ifchilddoc|\\
|\edef\childdocname{\scantokens\expandafter{\jobname\noexpand}}|\\
|\def\childdocmain{|\textit{main}|}\||ifx\childdocmain\childdocname\||else|\\
|\childdoctrue\includeonly{\childdocname}\let\jobname\childdocmain\||fi|\\
\end{tabular}
\end{center}
%
Instead of |\childdocof{|\textit{main}|}| just include the main file
at the top of each child file:
%
\begin{center}
|\input{|\textit{main}|}|
\end{center}
%
A simple redirection |\childdocforward{|\textit{dest}|}| is achieved by:
%
\begin{center}
|\def\jobname{|\textit{dest}|}\input{\jobname}|
\end{center}
%
The redirection with prefix
|\childdocforwardprefix[|\textit{prefix}|]{|\textit{dest}|}|
is accomplished by:
%
\begin{center}
\begin{tabular}{l}
|{\edef\jobname{\scantokens\expandafter{\jobname\noexpand}}|\\
|\def\redirectjob |\textit{prefix}|#1~~~{\gdef\jobname{|\textit{dest}|#1}}|\\
|\expandafter\redirectjob\jobname~~~}\input{\jobname}|
\end{tabular}
\end{center}

In an alternative approach,
child documents can be compiled by a specific command line
without additional code or specific definitions:
%
\begin{center}
|... -jobname "|\textit{target}|" "|[\textit{flags}]%
|\includeonly{|\textit{dest}|}\input{|\textit{main}|}"|
\end{center}
%

%%%%%%%%%%%%%%%%%%%%%%%%%%%%%%%%%%%%%%%%%%%%%%%%%%%%%%%%%%%%%%%%%%%%%%%%%%%%%%%%
%%%%%%%%%%%%%%%%%%%%%%%%%%%%%%%%%%%%%%%%%%%%%%%%%%%%%%%%%%%%%%%%%%%%%%%%%%%%%%%%
\section{Information}

%%%%%%%%%%%%%%%%%%%%%%%%%%%%%%%%%%%%%%%%%%%%%%%%%%%%%%%%%%%%%%%%%%%%%%%%%%%%%%%%
\subsection{Copyright}

Copyright \copyright{} 2017--2018 Niklas Beisert

This work may be distributed and/or modified under the
conditions of the \LaTeX{} Project Public License, either version 1.3
of this license or (at your option) any later version.
The latest version of this license is in
  \url{http://www.latex-project.org/lppl.txt}
and version 1.3 or later is part of all distributions of \LaTeX{}
version 2005/12/01 or later.

This work has the LPPL maintenance status `maintained'.

The Current Maintainer of this work is Niklas Beisert.

This work consists of the files |README.txt|, |childdoc.ins| and |childdoc.dtx|
as well as the derived files |childdoc.def|, |cdocsamp.tex|
with |cdocsch1.tex|, |cdocsch2.tex|, |cdocspt3.tex|, |cdocspt4.tex|,
|cdocsdrf.tex|, |cdocsfn1.tex|, |cdocsfn2.tex|
as well as |childdoc.pdf|.

%%%%%%%%%%%%%%%%%%%%%%%%%%%%%%%%%%%%%%%%%%%%%%%%%%%%%%%%%%%%%%%%%%%%%%%%%%%%%%%%
\subsection{Files and Installation}

The package consists of the files:
%
\begin{center}
\begin{tabular}{ll}
    |README.txt|   & readme file \\
    |childdoc.ins| & installation file \\
    |childdoc.dtx| & source file \\
    |childdoc.def| & definition file \\
    |cdocsamp.tex| & sample main file \\
    |cdocsch1.tex| & sample include file \\
    |cdocsch2.tex| & sample include file \\
    |cdocspt3.tex| & sample part file \\
    |cdocspt4.tex| & sample part file \\
    |cdocsdrf.tex| & sample redirection file \\
    |cdocsfn1.tex| & sample redirection file \\
    |cdocsfn2.tex| & sample redirection file \\
    |childdoc.pdf| & manual
\end{tabular}
\end{center}
%
The distribution consists of the files
|README.txt|, |childdoc.ins| and |childdoc.dtx|.
%
\begin{itemize}
\item
Run (pdf)\LaTeX{} on |childdoc.dtx|
to compile the manual |childdoc.pdf| (this file).
\item
Run \LaTeX{} on |childdoc.ins| to create the definitions file |childdoc.def|
and the sample |cdocsamp.tex| with include files
|cdocsch1.tex|, |cdocsch2.tex|, |cdocspt3.tex|, |cdocspt4.tex|,
|cdocsdrf.tex|, |cdocsfn1.tex|, |cdocsfn2.tex|.
Then copy the file |childdoc.def| to an appropriate directory of your \LaTeX{}
distribution, e.g.\ \textit{texmf-root}|/tex/latex/childdoc|.
\end{itemize}

%%%%%%%%%%%%%%%%%%%%%%%%%%%%%%%%%%%%%%%%%%%%%%%%%%%%%%%%%%%%%%%%%%%%%%%%%%%%%%%%
\subsection{Related CTAN Packages}

There are several other packages which offer a similar functionality:
%
\begin{itemize}
\item
The packages
\href{http://ctan.org/pkg/docmute}{\textsf{docmute}},
\href{http://ctan.org/pkg/includex}{\textsf{includex}} and
\href{http://ctan.org/pkg/standalone}{\textsf{standalone}}
provide commands to include only the document body of
a child file thus allowing both files to be compiled individually.
\item
The packages \href{http://ctan.org/pkg/subdocs}{\textsf{subdocs}}
and \href{http://ctan.org/pkg/subfiles}{\textsf{subfiles}}
provide structures in which the main and child documents can be
encapsulated and allowing them to be compiled individually.
The inclusion mechanism is different from the conventional |\include|.
\item
The package \href{http://ctan.org/pkg/combine}{\textsf{combine}}
is an elaborate solution to combine several documents into one.
\end{itemize}
%
See also the CTAN topic \href{http://ctan.org/topic/subdocs}{\textsf{subdocs}}
for further related packages.
The present package differs from the above solutions in that
a document structure constructed with the conventional |\include| mechanism
just needs two extra commands at the top of every file
such that all constituent files can be compiled individually.

%%%%%%%%%%%%%%%%%%%%%%%%%%%%%%%%%%%%%%%%%%%%%%%%%%%%%%%%%%%%%%%%%%%%%%%%%%%%%%%%
%\subsection{Feature Suggestions}
%
%The following is a list of features which may be useful for future
%versions of this package:
%%
%\begin{itemize}
%\item
%\ldots
%\end{itemize}

%%%%%%%%%%%%%%%%%%%%%%%%%%%%%%%%%%%%%%%%%%%%%%%%%%%%%%%%%%%%%%%%%%%%%%%%%%%%%%%%
\subsection{Revision History}

%%%%%%%%%%%%%%%%%%%%%%%%%%%%%%%%%%%%%%%%
\paragraph{v2.0:} 2018/12/30

\begin{itemize}
\item
immediate forward processing
\item
added |\childdocby| mechanism
\item
manual restructured
\end{itemize}

%%%%%%%%%%%%%%%%%%%%%%%%%%%%%%%%%%%%%%%%
\paragraph{v1.6:} 2018/01/17

\begin{itemize}
\item
application for development of include files
\item
corrections to manual
\end{itemize}

%%%%%%%%%%%%%%%%%%%%%%%%%%%%%%%%%%%%%%%%
\paragraph{v1.5:} 2017/05/21

\begin{itemize}
\item
more complete structuring introduced
\item
|\childdocof| introduced
\item
|\childdoc| renamed to |\childdocmain|
\item
|\childredirect| renamed to |\childdocforward| and |\childdocforwardprefix|
and functionality expanded
\end{itemize}

%%%%%%%%%%%%%%%%%%%%%%%%%%%%%%%%%%%%%%%%
\paragraph{v1.0:} 2017/04/27

\begin{itemize}
\item
manual and install package
\item
first version published on CTAN
\end{itemize}

%%%%%%%%%%%%%%%%%%%%%%%%%%%%%%%%%%%%%%%%
\paragraph{v0.6:} 2017/04/26

\begin{itemize}
\item
redirection mechanism added
\end{itemize}

%%%%%%%%%%%%%%%%%%%%%%%%%%%%%%%%%%%%%%%%
\paragraph{v0.5:} 2017/04/26

\begin{itemize}
\item
functionality in definition file
\end{itemize}


%%%%%%%%%%%%%%%%%%%%%%%%%%%%%%%%%%%%%%%%%%%%%%%%%%%%%%%%%%%%%%%%%%%%%%%%%%%%%%%%
%%%%%%%%%%%%%%%%%%%%%%%%%%%%%%%%%%%%%%%%%%%%%%%%%%%%%%%%%%%%%%%%%%%%%%%%%%%%%%%%
%%%%%%%%%%%%%%%%%%%%%%%%%%%%%%%%%%%%%%%%%%%%%%%%%%%%%%%%%%%%%%%%%%%%%%%%%%%%%%%%
\appendix

\settowidth\MacroIndent{\rmfamily\scriptsize 000\ }

 \DocInput{childdoc.dtx}

\end{document}
%</driver>
% \fi
%
% %%%%%%%%%%%%%%%%%%%%%%%%%%%%%%%%%%%%%%%%%%%%%%%%%%%%%%%%%%%%%%%%%%%%%%%%%%%%%%
% %%%%%%%%%%%%%%%%%%%%%%%%%%%%%%%%%%%%%%%%%%%%%%%%%%%%%%%%%%%%%%%%%%%%%%%%%%%%%%
% \section{Sample}
%\iffalse
%<*samplemain>
%\fi
%
% The following presents a sample document
% with two chapters, two parts, a title page,
% a compile flag as well as three forwarding files to set the flag.
% It consists of eight |.tex| files:
% \begin{center}
% \begin{tabular}{ll}
% |cdocsamp.tex|&main file\\
% |cdocsch1.tex|&include file for chapter 1\\
% |cdocsch2.tex|&include file for chapter 2\\
% |cdocspt3.tex|&include file for part 3\\
% |cdocspt4.tex|&include file for part 4\\
% |cdocsdrf.tex|&forwarding file for main file in draft mode\\
% |cdocsfi1.tex|&forwarding file for final version of chapter 1\\
% |cdocsfi2.tex|&forwarding file for final version of chapter 2\\
% \end{tabular}
% \end{center}
% Each of the eight files can be compiled directly by the \LaTeX{} compiler.
%
% %%%%%%%%%%%%%%%%%%%%%%%%%%%%%%%%%%%%%%
% \paragraph{Main File.}
%
% The main file is called |cdocsamp.tex|.
%
% Load the \textsf{childdoc} definitions and
% declare the filename for the main document:
%    \begin{macrocode}
\input{childdoc.def}
\childdocmain{}
%    \end{macrocode}

% Optional override for |\version| flag:
%    \begin{macrocode}
%%\ifchilddoc\else\providecommand{\version}{draft}\fi
%    \end{macrocode}

% Define the default values for the |\version| flag
% (|final| for the main file and |draft| for childs):
%    \begin{macrocode}
\ifchilddoc
\providecommand{\version}{draft}
\else
\providecommand{\version}{final}
\fi
%    \end{macrocode}

% Load the standard document class:
%    \begin{macrocode}
\documentclass[12pt]{article}
%    \end{macrocode}

% Start the document body:
%    \begin{macrocode}
\begin{document}
%    \end{macrocode}

% Declare a title page.
% Print title, part of document being processed and version flag:
%    \begin{macrocode}
\addtocounter{page}{-1}
\begin{center}
{\LARGE\bfseries{}childdoc example\par}
\vspace{1cm}
\ifchilddoc
\ifchilddocmanual part\else chapter\fi:
`\childdocname' of `\childdocjob'\par
\else
main document: `\childdocjob'\par
\fi
version: \version\par
\end{center}
\newpage
%    \end{macrocode}

% Manually include selected file,
% otherwise process as usual:
%    \begin{macrocode}
\ifchilddocmanual
\section*{part `\childdocname'}
\input{\childdocname}
\else
%    \end{macrocode}

% Include the two chapters:
%    \begin{macrocode}
\include{cdocsch1}
\include{cdocsch2}
%    \end{macrocode}

% Include the two parts unless only chapters should be displayed:
%    \begin{macrocode}
\ifchilddoc\else
\section{part three}
\input{cdocspt3}
\section{part four}
\input{cdocspt4}
\fi
%    \end{macrocode}

% Process as usual until here:
%    \begin{macrocode}
\fi
%    \end{macrocode}

% End of document body:
%    \begin{macrocode}
\end{document}
%    \end{macrocode}
%\iffalse
%</samplemain>
%\fi
%
% %%%%%%%%%%%%%%%%%%%%%%%%%%%%%%%%%%%%%%
% \paragraph{Chapter Include Files.}
%
% The include files are called |cdocsch1.tex| and |cdocsch2.tex|.
%
%\iffalse
%<*samplechap1|samplechap2>
%\fi

% Optional override for |\version| flag:
%    \begin{macrocode}
%%\providecommand{\version}{final}
%    \end{macrocode}

% Include the main document:
%    \begin{macrocode}
\input{childdoc.def}
\childdocof{cdocsamp}
%    \end{macrocode}

%\iffalse
%</samplechap1|samplechap2>
%\fi
%
%\iffalse
%<*samplechap1>
%\fi
% Some text for chapter 1:
%    \begin{macrocode}
\section{one}
some text in chapter one
%    \end{macrocode}

%\iffalse
%</samplechap1>
%\fi
% Some text for chapter 2:
%\iffalse
%<*samplechap2>
%\fi
%    \begin{macrocode}
\section{two}
more text in chapter two
%    \end{macrocode}

%\iffalse
%</samplechap2>
%\fi
%
% %%%%%%%%%%%%%%%%%%%%%%%%%%%%%%%%%%%%%%
% \paragraph{Part Include Files.}
%
% The include files are called |cdocspt3.tex| and |cdocspt4.tex|.
%
%\iffalse
%<*samplepart3|samplepart4>
%\fi

% Optional override for |\version| flag:
%    \begin{macrocode}
%%\providecommand{\version}{final}
%    \end{macrocode}

% Include the main document:
%    \begin{macrocode}
\input{childdoc.def}
\childdocby{cdocsamp}
%    \end{macrocode}

%\iffalse
%</samplepart3|samplepart4>
%\fi
%
%\iffalse
%<*samplepart3>
%\fi
% Some text for part 3:
%    \begin{macrocode}
some text in part three
%    \end{macrocode}

%\iffalse
%</samplepart3>
%\fi
% Some text for part 4:
%\iffalse
%<*samplepart4>
%\fi
%    \begin{macrocode}
more text in part four
%    \end{macrocode}

%\iffalse
%</samplepart4>
%\fi
%
% %%%%%%%%%%%%%%%%%%%%%%%%%%%%%%%%%%%%%%
% \paragraph{Forwarding for a Complete Draft.}
%
% The following forwarding file |cdocsdrf.tex|
% compiles the main document in draft mode:
%\iffalse
%<*sampledraft>
%\fi
%    \begin{macrocode}
\def\version{draft}
\input{childdoc.def}
\childdocforward{cdocsamp}
%    \end{macrocode}

%\iffalse
%</sampledraft>
%\fi
%
% %%%%%%%%%%%%%%%%%%%%%%%%%%%%%%%%%%%%%%
% \paragraph{Forwarding for Final Version of the Chapters.}
%
% The following forwarding files |cdocsfn1.tex| and |cdocsfn2.tex|
% (with identical content)
% compile the final versions of the child documents
% |cdocsch1.tex| and |cdocsch2.tex|, respectively:
%\iffalse
%<*samplefinal>
%\fi
%    \begin{macrocode}
\def\version{final}
\input{childdoc.def}
\childdocforwardprefix[cdocsamp]{cdocsfn}{cdocsch}
%    \end{macrocode}

%\iffalse
%</samplefinal>
%\fi
%
% %%%%%%%%%%%%%%%%%%%%%%%%%%%%%%%%%%%%%%
% \paragraph{Command Line Processing.}
%
% The following three command lines generate the output files
% |cdocscld|, |cdocscl1| and |cdocscl2|
% which should be identical to
% |cdocsdrf|, |cdocsch1| and |cdocsfn2|, respectively:
% \begin{center}
% \begin{tabular}{l}
% |latex -jobname cdocscld \|\\
% |  "\def\version{draft}\input{childdoc.def}\childdocforward{cdocsamp}"|\\
% |latex -jobname cdocscl1 \|\\
% |  "\input{childdoc.def}\childdocforward[cdocsamp]{cdocsch1}"|\\
% |latex -jobname cdocscl2 \|\\
% |  "\def\version{final}\input{childdoc.def}\childdocforward{cdocsch2}"|
% \end{tabular}
% \end{center}
% Note that the trailing backslash on each first line
% merely continues the input to the second line
% (for convenient cut ant paste).
% Furthermore, the command |latex| can be replaced by any
% of its alternative versions such as |pdflatex|.
%
% %%%%%%%%%%%%%%%%%%%%%%%%%%%%%%%%%%%%%%%%%%%%%%%%%%%%%%%%%%%%%%%%%%%%%%%%%%%%%%
% %%%%%%%%%%%%%%%%%%%%%%%%%%%%%%%%%%%%%%%%%%%%%%%%%%%%%%%%%%%%%%%%%%%%%%%%%%%%%%
% \section{Implementation}
%\iffalse
%<*package>
%\fi
%
% This section describes the definitions file |childdoc.def|.

% The definitions cannot be loaded using |\usepackage| or |\RequirePackage|
% which has a mechanism to prevent loading a style file more than once.
% When loading the definitions by means of |\input|
% multiple instances have to be prevented manually:
%\iffalse
%This code needs to be before the `\ProvidesFile' directive
%which is defined at the beginning of this file.
%Therefore it is also placed there and commented out here.
%</package>
%<*discard>
%\fi
%    \begin{macrocode}
\ifdefined\childdocmain\endinput\fi
%    \end{macrocode}
%\iffalse
%</discard>
%<*package>
%\fi
%
% \macro{\ifchilddoc}
% \macro{\ifchilddocmanual}
% The conditional |\ifchilddoc| tells whether a
% child (true) or main (false) document is being compiled.
% The conditional |\ifchilddocmanual| tells whether
% the |\includeonly| mechanism is used (false) or
% the selection of child files must be performed manually (true).
% The definitions initialise to false:
%    \begin{macrocode}
\newif\ifchilddoc
\newif\ifchilddocmanual
%    \end{macrocode}

% \macro{\childdocname}
% \macro{\childdocjob}
% The macro |\childdocname| stores the name of the main document
% to be compiled. The macro |\childdocjob| stores the name of
% the document on which the \LaTeX{} compiler was originally invoked.
% The content of |\jobname| cannot be compared
% to filenames specified in the source due to different catcodes.
% The following code rescans |\jobname|, stores the result
% in |\childdocname| and saves a copy in |\childdocjob|:
%    \begin{macrocode}
\edef\childdocname{\scantokens\expandafter{\jobname\noexpand}}
\let\childdocjob\childdocname
%    \end{macrocode}

% \macro{\childdocdisable}
% The macro |\childdocdisable| prevents the main file
% from being processed more than once.
% At this stage, the main document command |\childdocmain|
% is assumed to be called once again where it should do nothing.
% Any subsequent call to it should prevent
% a secondary processing of the main document
% It overwrites the forwarding commands
% |\childdocof| and |\childdocforward|
% with empty macros to prevent further inclusions of the main document:
%    \begin{macrocode}
\newcommand{\childdocdisable}
{
  \renewcommand{\childdocmain}[1]{\renewcommand{\childdocmain}[1]{\endinput}}
  \renewcommand{\childdocof}[1]{}
  \renewcommand{\childdocby}[2][]{}
  \renewcommand{\childdocforward}[2][]{}
  \renewcommand{\childdocdisable}{}
}
%    \end{macrocode}

% \macro{\childdocmain}
% The macro |\childdocmain| is to be called at the top of the main file
% with nothing or the main filename (without extension) as argument.
% First, it breaks loops.
% If the argument is not empty and does not match |\childdocname|
% (which is set by the first inclusion of |childdoc.def|),
% |\ifchilddoc| is set to true, |\includeonly| is applied to the child file
% and |\jobname| is set to the main file
% (for proper handling of |.aux| files):
%    \begin{macrocode}
\newcommand{\childdocmain}[1]
{
  \childdocdisable\childdocmain{}
  \if?#1?\else
    \begingroup
      \def\childdoctmp{#1}
      \ifx\childdoctmp\childdocname
        \def\childdoctmp{}
      \else
        \def\childdoctmp
        {
          \childdoctrue
          \includeonly{\childdocname}
          \def\childdocjob{#1}
          \def\jobname{#1}
        }
      \fi
      \expandafter
    \endgroup
    \childdoctmp
  \fi
}
%    \end{macrocode}

% \macro{\childdocof}
% The command |\childdocof| redirects
% compilation to the main file |#1|.
%    \begin{macrocode}
\newcommand{\childdocof}[1]
{
  \childdocdisable
  \childdoctrue
  \includeonly{\childdocname}
  \def\jobname{#1}
  \def\childdocjob{#1}
  \input{#1}
}
%    \end{macrocode}

% \macro{\childdocby}
% The command |\childdocby| ....
%    \begin{macrocode}
\newcommand{\childdocby}[2][]
{
  \childdocdisable
  \childdoctrue
  \childdocmanualtrue
  \if?#1?\else
    \def\jobname{#2}
  \fi
  \def\childdocjob{#2}
  \input{#2}
  \endinput
}
%    \end{macrocode}

% \macro{\childdocforward}
% The command |\childdocforward| redirects
% compilation to the main file or
% (if the optional argument is given) a child file.
% Parameters are set as if the main file
% or a child file starting with |\childdocof| was compiled.
% Then compilation is handed over to the main file:
%    \begin{macrocode}
\newcommand{\childdocforward}[2][]
{
  \begingroup
    \if?#1?
      \def\childdoctmp
      {
        \def\childdocname{#2}
        \def\childdocjob{#2}
        \def\jobname{#2}
        \input{#2}
        \endinput
      }
    \else
      \def\childdoctmp
      {
        \childdocdisable
        \def\childdocname{#2}
        \childdoctrue
        \includeonly{#2}
        \def\childdocjob{#1}
        \def\jobname{#1}
        \input{#1}
        \endinput
      }
    \fi
    \expandafter
  \endgroup
  \childdoctmp
}
%    \end{macrocode}

% \macro{\childdocforwardprefix}
% The command |\childdocforwardprefix| redirects
% compilation to the main or a child file by means of a pattern.
% The prefix |#1| in the current filename is replaced by |#2|
% and the suffix of the current filename is kept
% (it is assumed that the filename does not contain the substring `|~~~|'
% which is used as a delimiter).
% Compilation is handed over to the new file by |\childdocforward|:
%    \begin{macrocode}
\newcommand{\childdocforwardprefix}[3][]
{
  \begingroup
    \def\childdocextract #2##1~~~{\def\childdoctmp{\childdocforward[#1]{#3##1}}}
    \expandafter\childdocextract\childdocname~~~
    \expandafter
  \endgroup
  \childdoctmp
}
%    \end{macrocode}

% \macro{\childdoc}
% The deprecated macro |\childdoc| is a legacy version of |\childdocmain|:
%    \begin{macrocode}
\newcommand{\childdoc}{\childdocmain}
%    \end{macrocode}

% \macro{\childdocredirect}
% The deprecated macro |\childdocredirect| is a legacy version
% of |\childdocforward| and |\childdocforwardprefix|:
%    \begin{macrocode}
\newcommand{\childdocredirect}[2][]
{
  \begingroup
    \if?#1?
      \def\childdoctmp{\childdocforward{#2}}
    \else
      \def\childdoctmp{\childdocforwardprefix{#1}{#2}}
    \fi
    \expandafter
  \endgroup
  \childdoctmp
}
%    \end{macrocode}

%\iffalse
%</package>
%\fi
%
\endinput
\childdocforward[cdocsamp]{cdocsch1}"|\\
% |latex -jobname cdocscl2 \|\\
% |  "\def\version{final}% \iffalse
%
% childdoc.dtx Copyright (C) 2017-2018 Niklas Beisert
%
% This work may be distributed and/or modified under the
% conditions of the LaTeX Project Public License, either version 1.3
% of this license or (at your option) any later version.
% The latest version of this license is in
%   http://www.latex-project.org/lppl.txt
% and version 1.3 or later is part of all distributions of LaTeX
% version 2005/12/01 or later.
%
% This work has the LPPL maintenance status `maintained'.
%
% The Current Maintainer of this work is Niklas Beisert.
%
% This work consists of the files childdoc.dtx and childdoc.ins
% and the derived files childdoc.def and cdocsamp.tex with
% cdocsch1.tex, cdocsch2.tex, cdocsdrf.tex, cdocsfn1.tex, cdocsfn2.tex.
%
%<package>\ifdefined\childdocmain\endinput\fi
%<package>\ProvidesFile{childdoc.def}[2018/12/30 v2.0 child document driver]
%<samplemain>\ProvidesFile{cdocsamp.tex}[2018/12/30 v2.0 sample for childdoc]
%<*driver>
%\ProvidesFile{childdoc.drv}[2018/12/30 v2.0 childdoc reference manual file]
\PassOptionsToClass{10pt,a4paper}{article}
\documentclass{ltxdoc}

\usepackage[margin=35mm]{geometry}
\usepackage{hyperref}
\usepackage{hyperxmp}
\usepackage[usenames]{color}

\hypersetup{colorlinks=true}
\hypersetup{pdfstartview=FitH}
\hypersetup{pdfpagemode=UseNone}
\hypersetup{pdfsource={}}
\hypersetup{pdflang={en-UK}}
\hypersetup{pdfcopyright={Copyright 2017-2018 Niklas Beisert.
  This work may be distributed and/or modified under the
  conditions of the LaTeX Project Public License, either version 1.3
  of this license or (at your option) any later version.}}
\hypersetup{pdflicenseurl={http://www.latex-project.org/lppl.txt}}
\hypersetup{pdfcontactaddress={ETH Zurich, ITP, HIT K,
  Wolfgang-Pauli-Strasse 27}}
\hypersetup{pdfcontactpostcode={8093}}
\hypersetup{pdfcontactcity={Zurich}}
\hypersetup{pdfcontactcountry={Switzerland}}
\hypersetup{pdfcontactemail={nbeisert@itp.phys.ethz.ch}}
\hypersetup{pdfcontacturl={http://people.phys.ethz.ch/\xmptilde nbeisert/}}

\newcommand{\secref}[1]{\hyperref[#1]{section \ref*{#1}}}

\parskip1ex
\parindent0pt
\let\olditemize\itemize
\def\itemize{\olditemize\parskip0pt}

\begin{document}

\title{The \textsf{childdoc} Package}
\hypersetup{pdftitle={The childdoc Package}}
\author{Niklas Beisert\\[2ex]
  Institut f\"ur Theoretische Physik\\
  Eidgen\"ossische Technische Hochschule Z\"urich\\
  Wolfgang-Pauli-Strasse 27, 8093 Z\"urich, Switzerland\\[1ex]
  \href{mailto:nbeisert@itp.phys.ethz.ch}
  {\texttt{nbeisert@itp.phys.ethz.ch}}}
\hypersetup{pdfauthor={Niklas Beisert}}
\hypersetup{pdfsubject={Manual for the LaTeX2e Package childdoc}}
\date{30 December 2018, \textsf{v2.0}}
\maketitle

\begin{abstract}\noindent
\textsf{childdoc} is a \LaTeXe{} package
that enables the direct compilation
of document sections included by |\include|
to individual files.
\end{abstract}

\begingroup
\parskip0ex
\tableofcontents
\endgroup

%%%%%%%%%%%%%%%%%%%%%%%%%%%%%%%%%%%%%%%%%%%%%%%%%%%%%%%%%%%%%%%%%%%%%%%%%%%%%%%%
%%%%%%%%%%%%%%%%%%%%%%%%%%%%%%%%%%%%%%%%%%%%%%%%%%%%%%%%%%%%%%%%%%%%%%%%%%%%%%%%
\section{Introduction}

\LaTeX{} provides a mechanism to structure a large document (such as a book)
into a main file and several child files (containing the chapters)
using the |\include| command.
This mechanism is beneficial for documents
which span hundreds of pages in order to
make the source file(s) more manageable.
Moreover, compilation can be restricted to
selected child files by means of the |\includeonly| command.
The latter feature can be used to reduce the compilation time while editing
(this was significantly more useful in the earlier days of \LaTeX{})
or to generate a smaller document which is easier to navigate.
Another application of |\includeonly| is to generate
documents consisting of selected parts of the complete document.

However, there are a few drawbacks of the plain |\include| mechanism:
\begin{itemize}
\item
The child files cannot be compiled on their own,
they can only be compiled via the main file.
A naive editing environment
(such as a text editor with an option
to have the current file processed by \LaTeX)
may require one to switch to the main file before compiling;
attempting to compile the child file produces errors.
\item
The main file must be modified (each time)
to adjust the |\includeonly| command
to the present needs. This easily leaves the main file in a messy state.
\item
The generated document will always carry the filename
of the main document. This is inconvenient if
several child files are to be compiled and
to be kept for distribution.
\end{itemize}

The present package provides a simple interface
to make child files individually compilable by \LaTeX{}.
Compiling a child file then has the same effect as compiling
the main file with an |\includeonly| command
to select the appropriate child.
Moreover the generated document will carry the name of the child
rather than the main file.
This resolves all three above issues.

This feature is meant to make the editing of books,
thesis documents and lecture notes somewhat more convenient.
However, the package can also be used efficiently for
composing a series of documents (such as exercise sheets)
which are typically distributed individually.
It then assists the author in generating the individual documents
(potentially in different versions)
as well as a document containing the collected series.
Another application is in developing style files
or other kinds of included material
where compilation of the style file could redirect
to a sample or test file.

%%%%%%%%%%%%%%%%%%%%%%%%%%%%%%%%%%%%%%%%%%%%%%%%%%%%%%%%%%%%%%%%%%%%%%%%%%%%%%%%
%%%%%%%%%%%%%%%%%%%%%%%%%%%%%%%%%%%%%%%%%%%%%%%%%%%%%%%%%%%%%%%%%%%%%%%%%%%%%%%%
\section{Usage}

First of all, the package \textsf{childdoc} is \emph{not} a standard
\LaTeXe{} |.sty| style file! Therefore it needs to be invoked in
a non-standard way.

%%%%%%%%%%%%%%%%%%%%%%%%%%%%%%%%%%%%%%%%%%%%%%%%%%%%%%%%%%%%%%%%%%%%%%%%%%%%%%%%
\subsection{Included Files}
\label{sec:include}

%%%%%%%%%%%%%%%%%%%%%%%%%%%%%%%%%%%%%%%%
\DescribeMacro{\childdocmain}
To use the package, add the commands
\begin{center}
\begin{tabular}{l}
|\input{childdoc.def}|\\
|\childdocmain{}|\\
\end{tabular}
\end{center}
at the very top of the main \LaTeX{} file,
in particular \emph{before} the |\documentclass| statement!
The argument of |\childdocmain| should be left empty
(but it must be present).

%%%%%%%%%%%%%%%%%%%%%%%%%%%%%%%%%%%%%%%%
\DescribeMacro{\childdocof}
Furthermore, add the commands
\begin{center}
\begin{tabular}{l}
|\input{childdoc.def}|\\
|\childdocof{|\textit{main}|}|\\
\end{tabular}
\end{center}
at the top of every child file \textit{child}
which is included by |\include{|\textit{child}|}|
from within the main file
(or at least for those files to be compiled individually).
The argument \textit{main} must be the filename of the main file.

There are a couple of
considerations in setting up the main and child documents:

%%%%%%%%%%%%%%%%%%%%%%%%%%%%%%%%%%%%%%%%
\paragraph{Restrictions.}

Please note the following restrictions:
\begin{itemize}
\item
|\childdocmain| must be called with one argument \textit{main}
to ensure compatibility with earlier version of the package.
It must either be empty (|\childdocmain{}|)
or precisely match the filename of the main file in which it is specified.
See \secref{sec:detection} for further information.
\item
The filename \textit{main} must be specified without the |.tex| extension.
\item
The filename \textit{main} is case sensitive
(even in case-insensitive file systems)
due to internal string comparison.
\item
The argument \textit{main} should be fully expanded, it cannot be a macro.
\item
Subdirectories and special characters should be avoided in filenames.
\item
The command |\childdocmain{|\textit{main}|}| must be followed by a whitespace.
It should not be followed immediately by another command
or by a comment mark `|%|'.
This is because the \TeX{} parser reads the token immediately following
the argument of |\childdocmain| and puts it
at the beginning of every child section;
however, a white\-space is ignored.
\end{itemize}

%%%%%%%%%%%%%%%%%%%%%%%%%%%%%%%%%%%%%%%%
\paragraph{Content of Main File.}

It is advisable to place all content in the child files included by |\include|.
Any output contained in the main file will appear in all child documents
unless suppressed manually;
it cannot be suppressed automatically by the |\includeonly| directive
and thus should normally be avoided.
A method to include some content in the main file
by means of conditional processing is described in \secref{sec:conditional}.

%%%%%%%%%%%%%%%%%%%%%%%%%%%%%%%%%%%%%%%%
\paragraph{Page Numbering.}

When only a part of the document is compiled,
the appropriate numbering of pages
(as well as other status parameters)
is determined from the |.aux| files.
The latter contain information from previous passes.
However this information needs to propagate through
all intermediate child documents.
Therefore the page numbering in child documents may well
be inconsistent until the complete document is compiled at least once.

A useful (if unconventional) way to always ensure a consistent
page numbering is to restart the numbering in each child document
and denote the pages by `\textit{child}|.|\textit{page}'
where \textit{child} represents the chapter/section number of the child file.
This can be achieved by the command
|\numberwithin{page}{|\textit{child}|}|
of the \textsf{amsmath} package
where \textit{child} can be |chapter| or |section|
depending on the chosen structuring.
Alternatively, one can modify the macro |\thepage| appropriately
and reset the counter |page| at the start of each child file.

%%%%%%%%%%%%%%%%%%%%%%%%%%%%%%%%%%%%%%%%%%%%%%%%%%%%%%%%%%%%%%%%%%%%%%%%%%%%%%%%
\subsection{Conditional Processing}
\label{sec:conditional}

The package provides a mechanism to compile different versions
of a document. To customise the versions further some conditional processing
can come in handy to distinguish which version is being compiled.
The package provides two macros to describe the compilation context:

%%%%%%%%%%%%%%%%%%%%%%%%%%%%%%%%%%%%%%%%
\DescribeMacro{\ifchilddoc}
The conditional |\ifchilddoc| distinguishes between the compilation of
child documents and the main document:
%
\begin{center}
|\ifchilddoc |\textit{child-code}| |[|\||else |\textit{main-code}]| \||fi|
\end{center}

%%%%%%%%%%%%%%%%%%%%%%%%%%%%%%%%%%%%%%%%
\DescribeMacro{\childdocname}
\DescribeMacro{\childdocjob}
The macro |\childdocname| contains the filename (without extension)
of the main or child file being processed.
Note that |\childdocjob| will always contain the name of the main file.

%%%%%%%%%%%%%%%%%%%%%%%%%%%%%%%%%%%%%%%%
\paragraph{Title Page.}

Conditional processing can be used to include a title or banner page
in the main document when proper precautions are taken.
Importantly, the code in the main file should ensure that the page counter
(as well as other status parameters which are stored in the |.aux| files)
takes the same value after the conditional processing.
Otherwise the page numbers may take divergent values
depending on which part is compiled.

For example, a title page could be declared by:
%
\begin{center}
\begin{tabular}{l}
|\ifchilddoc\||else|\\
|\addtocounter{page}{-1}|\\
\textit{code for title page}\\
|\newpage|\\
|\||fi|
\end{tabular}
\end{center}
%
A banner page for the child documents can be generated by:
%
\begin{center}
\begin{tabular}{l}
|\ifchilddoc|\\
|\addtocounter{page}{-1}|\\
\textit{code for banner page}\\
|\newpage|\\
|\||fi|
\end{tabular}
\end{center}
%
Here one could write a message such as:
\begin{center}
|This is the part \childdocname{} of \childdocjob{}.|
\end{center}

%%%%%%%%%%%%%%%%%%%%%%%%%%%%%%%%%%%%%%%%%%%%%%%%%%%%%%%%%%%%%%%%%%%%%%%%%%%%%%%%
\subsection{Flags}
\label{sec:flags}

The package makes it easy to generate different versions
of the main or child documents.
To this end compilation flags can be defined
and assigned different default values.
They will be particularly useful in conjunction
with the forwarding mechanism described in \secref{sec:forward}.

For example, it may be useful to have a flag |\version|
which can be set to |draft| or |final|.
The document source will contain some conditional code
depending on the value of |\version|.
Suppose further, the flag should default to |final| for the main file
and to |draft| for child files
which is a natural assignment for editing the document.
This is achieved by placing the following code
in the preamble of the main document
(below the |\childdocmain| directive):
%
\begin{center}
\begin{tabular}{l}
|\ifchilddoc|\\
|\providecommand{\version}{draft}|\\
|\||else|\\
|\providecommand{\version}{final}|\\
|\||fi|
\end{tabular}
\end{center}
%
The definition by |\providecommand| makes sure
that previous definitions are not overwritten.
Further statements |\providecommand{\version}{...}|
can thus be added before the above code to override it.

For the main file, one might add a line
(between |\childdocmain| and the above block)
%
\begin{center}
|%\ifchilddoc\||else\providecommand{\version}{draft}\||fi|
\end{center}
%
which can be uncommented to produce a draft version.
Likewise one can add a line to the very top of a child file
(above the |\childdocof{|\textit{main}|}| directive)
%
\begin{center}
|%\providecommand{\version}{final}|
\end{center}
%
which can be uncommented to produce the final version of this child document.

%%%%%%%%%%%%%%%%%%%%%%%%%%%%%%%%%%%%%%%%%%%%%%%%%%%%%%%%%%%%%%%%%%%%%%%%%%%%%%%%
\subsection{Forwarding}
\label{sec:forward}

Different versions of the main or child documents
using compilation flags as described in \secref{sec:flags}
can be (permanently) stored in different files
for convenient compilation, viewing and distribution.
To this end, the package defines a command
to pass on compilation to a different file:

%%%%%%%%%%%%%%%%%%%%%%%%%%%%%%%%%%%%%%%%
\DescribeMacro{\childdocforward}
The command |\childdocforward| redirects processing to
another source file:
%
\begin{center}
\begin{tabular}{l}
|\input{childdoc.def}|\\
|\childdocforward[|\textit{main}|]{|\textit{dest}|}|\\
\end{tabular}
\end{center}
%
The argument \textit{dest} is the destination file
(without extension).
It should be the main file or one of the child files.
Note that further \textsf{childdoc} directives
such as |\childdocof| and |\childdocforward|
in the indicated file will be processed in this form.
The optional argument \textit{main}
passes on directly to the main file \textit{main}
while pretending to compile the child \textit{dest}.
This form behaves as if \textit{dest}
issues |\childdocof{|\textit{main}|}| right away,
and no further \textsf{childdoc} directives will be processed.

%%%%%%%%%%%%%%%%%%%%%%%%%%%%%%%%%%%%%%%%
\DescribeMacro{\...prefix}
In the alternative form |\childdocforwardprefix|,
%
\begin{center}
\begin{tabular}{l}
|\input{childdoc.def}|\\
|\childdocforwardprefix[|\textit{main}|]{|\textit{prefix}|}{|\textit{dest}|}|
\end{tabular}
\end{center}
%
the destination file is determined by a pattern
depending on the current file:
To make this work, the current file must be called
`{\textit{prefix}\hspace{0.2em}\textit{suffix}}'
with \textit{prefix} matching precisely the argument.
Processing is then passed on to the file
`{\textit{dest}\hspace{0.2em}\textit{suffix}}'.
Surely, the same effect is achieved by
directly specifying the
argument `{\textit{dest}\hspace{0.2em}\textit{suffix}}'
in the first form.
However, that requires to set up a different file
for each child. With the alternative form of the command
all these files can have exactly the same content
which simplifies setting them up and maintaining them.

For example, the following file |draft.tex|
with a compilation flag |\version| as described in \secref{sec:flags}
compiles the main document as a draft:
%
\begin{center}
\begin{tabular}{l}
|\def\version{draft}|\\
|\input{childdoc.def}|\\
|\childdocforward{|\textit{main}|}|
\end{tabular}
\end{center}
%
Likewise, the following files |final|\textit{nn}|.tex|
compile the final version of the child document
|child|\textit{nn}|.tex|:
%
\begin{center}
\begin{tabular}{l}
|\def\version{final}|\\
|\input{childdoc.def}|\\
|\childdocforwardprefix{final}{child}|
\end{tabular}
\end{center}
%

Note that when several versions of a main file and/or of each child file
are to be generated, it may be convenient to set up a |Makefile| or
shell script to automatise the process.

%%%%%%%%%%%%%%%%%%%%%%%%%%%%%%%%%%%%%%%%%%%%%%%%%%%%%%%%%%%%%%%%%%%%%%%%%%%%%%%%
\subsection{Command Line Processing}
\label{sec:commandline}

The effect of redirection files can also be achieved by invoking
the \LaTeX{} compiler with a more elaborate command line.
Most conveniently this should be done as part
of a shell script or a |Makefile|.

When using \textsf{childdoc} in the main file, the following
command lines effectively perform a redirection
(note that depending on the shell being used,
backslashes may have to be doubled: `|\|' $\to$ `|\\|'):
%
\begin{center}
|... -jobname "|\textit{target}|" |\\|"|[\textit{flags}]%
|\input{childdoc.def}\childdocforward[|\textit{main}|]{|\textit{dest}|}"|
\end{center}
%
Here \textit{target} is the name of the output file,
\textit{main} is the name of the main file
and \textit{dest} is the name of the main or child file to be processed
(all filenames without extensions).
The optional argument \textit{main} can be omitted
if \textit{main} matches \textit{dest}.
Optionally, compilation \textit{flags} can be defined via |\def| commands.
This command line makes the \TeX{} engine believe
it is compiling the file \textit{target}
whose content is specified as the latter parameter.
The provided code then forwards the processing to
\textit{main} or \textit{dest} as described in \secref{sec:forward}.

%%%%%%%%%%%%%%%%%%%%%%%%%%%%%%%%%%%%%%%%%%%%%%%%%%%%%%%%%%%%%%%%%%%%%%%%%%%%%%%%
\subsection{Include by Input}
\label{sec:input}

Including child documents by |\include| has some restrictions by design.
Most notably, the content of a child document always occupies
its own set of pages; pages cannot be shared between child documents.
Usually, this behaviour makes perfect sense
because each child document contain an essential part of the document.
However, in some situations it may be desirable to compose
a document from a collection of parts
without having mandatory page breaks between then.
For this case, the package
provides a mechanism to include parts
by |\input| which can also be processed individually.
However, by construction this mechanism
requires manual handling of the content to be output.

%%%%%%%%%%%%%%%%%%%%%%%%%%%%%%%%%%%%%%%%
\DescribeMacro{\ifchilddocmanual}
The main file should be prepared as usual, see \secref{sec:include}.
However, the document body must make a distinction
between processing of an individual part and of the main document, e.g.:
%
\begin{center}
\begin{tabular}{l}
|\ifchilddocmanual|\\
|\input{\childdocname}|\\
|\||else|\\
\textit{document body with }|\input{|\textit{part}|}|\\
|\||fi|
\end{tabular}
\end{center}
%
The conditional |\ifchilddocmanual| is true whenever
a part to be included by |\input| is being compiled,
and the name of the part is stored in |\childdocname|.

%%%%%%%%%%%%%%%%%%%%%%%%%%%%%%%%%%%%%%%%
\DescribeMacro{\childdocby}
Each part to be included by |\input| should start with:
%
\begin{center}
\begin{tabular}{l}
|\input{childdoc.def}|\\
|\childdocby{|\textit{main}|}|\\
\end{tabular}
\end{center}
%
The directive |\childdocby| is similar to |\childdocof|
described in \secref{sec:include},
but the subsequent selection of content must be done manually.
To that end, both |\ifchilddoc| and |\ifchilddocmanual|
will be true upon processing of a part,
and the name of the part is stored in |\childdocname|.
Note that |\jobname| will be set to the filename of the current part
so that each part receives an individual |.aux| file
that does not interfere with the |.aux| file(s) of the main document.
This behaviour can be altered by the alternative form
|\childdocby[*]{|\textit{main}|}| (with a non-empty optional argument)
which uses the |.aux| file of the main document
by setting |\jobname| to \textit{main}.

%%%%%%%%%%%%%%%%%%%%%%%%%%%%%%%%%%%%%%%%%%%%%%%%%%%%%%%%%%%%%%%%%%%%%%%%%%%%%%%%
\subsection{Driver Development}
\label{sec:driver}

The \textsf{childdoc} mechanism can also be use for the development
of definition files such as \LaTeX{} styles or classes.
This case differs from the above setup with multiple parts
included by |\include| in that no |\includeonly| should be invoked.
This can be achieved by starting the include file
(before |\ProvidesPackage|) with:
%
\begin{center}
\begin{tabular}{l}
|\input{childdoc.def}|\\
|\childdocforward{|\textit{main}|}|\\
\end{tabular}
\end{center}
%
or alternatively with:
%
\begin{center}
\begin{tabular}{l}
|\input{childdoc.def}|\\
|\childdocby{|\textit{main}|}|\\
\end{tabular}
\end{center}
%
Both forms have slightly different effects as described above.
The main file is prepared as usual, see \secref{sec:include}.

%%%%%%%%%%%%%%%%%%%%%%%%%%%%%%%%%%%%%%%%%%%%%%%%%%%%%%%%%%%%%%%%%%%%%%%%%%%%%%%%
\subsection{Legacy Detection}
\label{sec:detection}

The directive |\childdocmain| in the main file can detect
whether the complete document or merely a child is to be compiled
even without using the directive |\childdocof|.
This method is deprecated because it is less robust
and there is no compelling reason to use it;
it is merely provided for backward compatibility
and it may be removed in future versions.

If the detection mechanism is to be used,
it is mandatory to correctly specify
the filename of the main file as the argument of |\childdocmain|:
%
\begin{center}
\begin{tabular}{l}
|\input{childdoc.def}|\\
|\childdocmain{|\textit{main}|}|\\
\end{tabular}
\end{center}
%
If |\jobname| does not match the argument \textit{main} of |\childdocmain|,
it is assumed that |\jobname| points to the child file to be compiled.
When using |\childdocmain| with the main file specified as argument,
it suffices to start a child file
with just |\input{|\textit{main}|}|
without loading of the package and using |\childdocof|.
If instead all processing is done
with the appropriate \textsf{childdoc} directives,
the argument of \textit{main} of |\childdocmain| can be empty.

An alternative version of the command line processing described
in \secref{sec:commandline} using the detection mechanism reads:
%
\begin{center}
|... -jobname "|\textit{target}|" "|[\textit{flags}]%
[|\def\jobname{|\textit{dest}|}|]|\input{|\textit{main}|}"|
\end{center}

%%%%%%%%%%%%%%%%%%%%%%%%%%%%%%%%%%%%%%%%%%%%%%%%%%%%%%%%%%%%%%%%%%%%%%%%%%%%%%%%
\subsection{Manual Code}
\label{sec:manual}

In case one cannot be certain whether the definitions file |childdoc.def|
is installed on the target \TeX{} distribution
and one prefers not to ship it,
it is conceivable to paste a few relevant commands into the sources.

To that end, drop all statements |\input{childdoc.def}|
and perform the replacements as outlined below.
Instead of |\childdocmain{|\textit{main}|}| add the following code
to the top of the main file:
%
\begin{center}
\begin{tabular}{l}
|\||ifdefined\childdocname\endinput\||fi\newif\ifchilddoc|\\
|\edef\childdocname{\scantokens\expandafter{\jobname\noexpand}}|\\
|\def\childdocmain{|\textit{main}|}\||ifx\childdocmain\childdocname\||else|\\
|\childdoctrue\includeonly{\childdocname}\let\jobname\childdocmain\||fi|\\
\end{tabular}
\end{center}
%
Instead of |\childdocof{|\textit{main}|}| just include the main file
at the top of each child file:
%
\begin{center}
|\input{|\textit{main}|}|
\end{center}
%
A simple redirection |\childdocforward{|\textit{dest}|}| is achieved by:
%
\begin{center}
|\def\jobname{|\textit{dest}|}\input{\jobname}|
\end{center}
%
The redirection with prefix
|\childdocforwardprefix[|\textit{prefix}|]{|\textit{dest}|}|
is accomplished by:
%
\begin{center}
\begin{tabular}{l}
|{\edef\jobname{\scantokens\expandafter{\jobname\noexpand}}|\\
|\def\redirectjob |\textit{prefix}|#1~~~{\gdef\jobname{|\textit{dest}|#1}}|\\
|\expandafter\redirectjob\jobname~~~}\input{\jobname}|
\end{tabular}
\end{center}

In an alternative approach,
child documents can be compiled by a specific command line
without additional code or specific definitions:
%
\begin{center}
|... -jobname "|\textit{target}|" "|[\textit{flags}]%
|\includeonly{|\textit{dest}|}\input{|\textit{main}|}"|
\end{center}
%

%%%%%%%%%%%%%%%%%%%%%%%%%%%%%%%%%%%%%%%%%%%%%%%%%%%%%%%%%%%%%%%%%%%%%%%%%%%%%%%%
%%%%%%%%%%%%%%%%%%%%%%%%%%%%%%%%%%%%%%%%%%%%%%%%%%%%%%%%%%%%%%%%%%%%%%%%%%%%%%%%
\section{Information}

%%%%%%%%%%%%%%%%%%%%%%%%%%%%%%%%%%%%%%%%%%%%%%%%%%%%%%%%%%%%%%%%%%%%%%%%%%%%%%%%
\subsection{Copyright}

Copyright \copyright{} 2017--2018 Niklas Beisert

This work may be distributed and/or modified under the
conditions of the \LaTeX{} Project Public License, either version 1.3
of this license or (at your option) any later version.
The latest version of this license is in
  \url{http://www.latex-project.org/lppl.txt}
and version 1.3 or later is part of all distributions of \LaTeX{}
version 2005/12/01 or later.

This work has the LPPL maintenance status `maintained'.

The Current Maintainer of this work is Niklas Beisert.

This work consists of the files |README.txt|, |childdoc.ins| and |childdoc.dtx|
as well as the derived files |childdoc.def|, |cdocsamp.tex|
with |cdocsch1.tex|, |cdocsch2.tex|, |cdocspt3.tex|, |cdocspt4.tex|,
|cdocsdrf.tex|, |cdocsfn1.tex|, |cdocsfn2.tex|
as well as |childdoc.pdf|.

%%%%%%%%%%%%%%%%%%%%%%%%%%%%%%%%%%%%%%%%%%%%%%%%%%%%%%%%%%%%%%%%%%%%%%%%%%%%%%%%
\subsection{Files and Installation}

The package consists of the files:
%
\begin{center}
\begin{tabular}{ll}
    |README.txt|   & readme file \\
    |childdoc.ins| & installation file \\
    |childdoc.dtx| & source file \\
    |childdoc.def| & definition file \\
    |cdocsamp.tex| & sample main file \\
    |cdocsch1.tex| & sample include file \\
    |cdocsch2.tex| & sample include file \\
    |cdocspt3.tex| & sample part file \\
    |cdocspt4.tex| & sample part file \\
    |cdocsdrf.tex| & sample redirection file \\
    |cdocsfn1.tex| & sample redirection file \\
    |cdocsfn2.tex| & sample redirection file \\
    |childdoc.pdf| & manual
\end{tabular}
\end{center}
%
The distribution consists of the files
|README.txt|, |childdoc.ins| and |childdoc.dtx|.
%
\begin{itemize}
\item
Run (pdf)\LaTeX{} on |childdoc.dtx|
to compile the manual |childdoc.pdf| (this file).
\item
Run \LaTeX{} on |childdoc.ins| to create the definitions file |childdoc.def|
and the sample |cdocsamp.tex| with include files
|cdocsch1.tex|, |cdocsch2.tex|, |cdocspt3.tex|, |cdocspt4.tex|,
|cdocsdrf.tex|, |cdocsfn1.tex|, |cdocsfn2.tex|.
Then copy the file |childdoc.def| to an appropriate directory of your \LaTeX{}
distribution, e.g.\ \textit{texmf-root}|/tex/latex/childdoc|.
\end{itemize}

%%%%%%%%%%%%%%%%%%%%%%%%%%%%%%%%%%%%%%%%%%%%%%%%%%%%%%%%%%%%%%%%%%%%%%%%%%%%%%%%
\subsection{Related CTAN Packages}

There are several other packages which offer a similar functionality:
%
\begin{itemize}
\item
The packages
\href{http://ctan.org/pkg/docmute}{\textsf{docmute}},
\href{http://ctan.org/pkg/includex}{\textsf{includex}} and
\href{http://ctan.org/pkg/standalone}{\textsf{standalone}}
provide commands to include only the document body of
a child file thus allowing both files to be compiled individually.
\item
The packages \href{http://ctan.org/pkg/subdocs}{\textsf{subdocs}}
and \href{http://ctan.org/pkg/subfiles}{\textsf{subfiles}}
provide structures in which the main and child documents can be
encapsulated and allowing them to be compiled individually.
The inclusion mechanism is different from the conventional |\include|.
\item
The package \href{http://ctan.org/pkg/combine}{\textsf{combine}}
is an elaborate solution to combine several documents into one.
\end{itemize}
%
See also the CTAN topic \href{http://ctan.org/topic/subdocs}{\textsf{subdocs}}
for further related packages.
The present package differs from the above solutions in that
a document structure constructed with the conventional |\include| mechanism
just needs two extra commands at the top of every file
such that all constituent files can be compiled individually.

%%%%%%%%%%%%%%%%%%%%%%%%%%%%%%%%%%%%%%%%%%%%%%%%%%%%%%%%%%%%%%%%%%%%%%%%%%%%%%%%
%\subsection{Feature Suggestions}
%
%The following is a list of features which may be useful for future
%versions of this package:
%%
%\begin{itemize}
%\item
%\ldots
%\end{itemize}

%%%%%%%%%%%%%%%%%%%%%%%%%%%%%%%%%%%%%%%%%%%%%%%%%%%%%%%%%%%%%%%%%%%%%%%%%%%%%%%%
\subsection{Revision History}

%%%%%%%%%%%%%%%%%%%%%%%%%%%%%%%%%%%%%%%%
\paragraph{v2.0:} 2018/12/30

\begin{itemize}
\item
immediate forward processing
\item
added |\childdocby| mechanism
\item
manual restructured
\end{itemize}

%%%%%%%%%%%%%%%%%%%%%%%%%%%%%%%%%%%%%%%%
\paragraph{v1.6:} 2018/01/17

\begin{itemize}
\item
application for development of include files
\item
corrections to manual
\end{itemize}

%%%%%%%%%%%%%%%%%%%%%%%%%%%%%%%%%%%%%%%%
\paragraph{v1.5:} 2017/05/21

\begin{itemize}
\item
more complete structuring introduced
\item
|\childdocof| introduced
\item
|\childdoc| renamed to |\childdocmain|
\item
|\childredirect| renamed to |\childdocforward| and |\childdocforwardprefix|
and functionality expanded
\end{itemize}

%%%%%%%%%%%%%%%%%%%%%%%%%%%%%%%%%%%%%%%%
\paragraph{v1.0:} 2017/04/27

\begin{itemize}
\item
manual and install package
\item
first version published on CTAN
\end{itemize}

%%%%%%%%%%%%%%%%%%%%%%%%%%%%%%%%%%%%%%%%
\paragraph{v0.6:} 2017/04/26

\begin{itemize}
\item
redirection mechanism added
\end{itemize}

%%%%%%%%%%%%%%%%%%%%%%%%%%%%%%%%%%%%%%%%
\paragraph{v0.5:} 2017/04/26

\begin{itemize}
\item
functionality in definition file
\end{itemize}


%%%%%%%%%%%%%%%%%%%%%%%%%%%%%%%%%%%%%%%%%%%%%%%%%%%%%%%%%%%%%%%%%%%%%%%%%%%%%%%%
%%%%%%%%%%%%%%%%%%%%%%%%%%%%%%%%%%%%%%%%%%%%%%%%%%%%%%%%%%%%%%%%%%%%%%%%%%%%%%%%
%%%%%%%%%%%%%%%%%%%%%%%%%%%%%%%%%%%%%%%%%%%%%%%%%%%%%%%%%%%%%%%%%%%%%%%%%%%%%%%%
\appendix

\settowidth\MacroIndent{\rmfamily\scriptsize 000\ }

 \DocInput{childdoc.dtx}

\end{document}
%</driver>
% \fi
%
% %%%%%%%%%%%%%%%%%%%%%%%%%%%%%%%%%%%%%%%%%%%%%%%%%%%%%%%%%%%%%%%%%%%%%%%%%%%%%%
% %%%%%%%%%%%%%%%%%%%%%%%%%%%%%%%%%%%%%%%%%%%%%%%%%%%%%%%%%%%%%%%%%%%%%%%%%%%%%%
% \section{Sample}
%\iffalse
%<*samplemain>
%\fi
%
% The following presents a sample document
% with two chapters, two parts, a title page,
% a compile flag as well as three forwarding files to set the flag.
% It consists of eight |.tex| files:
% \begin{center}
% \begin{tabular}{ll}
% |cdocsamp.tex|&main file\\
% |cdocsch1.tex|&include file for chapter 1\\
% |cdocsch2.tex|&include file for chapter 2\\
% |cdocspt3.tex|&include file for part 3\\
% |cdocspt4.tex|&include file for part 4\\
% |cdocsdrf.tex|&forwarding file for main file in draft mode\\
% |cdocsfi1.tex|&forwarding file for final version of chapter 1\\
% |cdocsfi2.tex|&forwarding file for final version of chapter 2\\
% \end{tabular}
% \end{center}
% Each of the eight files can be compiled directly by the \LaTeX{} compiler.
%
% %%%%%%%%%%%%%%%%%%%%%%%%%%%%%%%%%%%%%%
% \paragraph{Main File.}
%
% The main file is called |cdocsamp.tex|.
%
% Load the \textsf{childdoc} definitions and
% declare the filename for the main document:
%    \begin{macrocode}
\input{childdoc.def}
\childdocmain{}
%    \end{macrocode}

% Optional override for |\version| flag:
%    \begin{macrocode}
%%\ifchilddoc\else\providecommand{\version}{draft}\fi
%    \end{macrocode}

% Define the default values for the |\version| flag
% (|final| for the main file and |draft| for childs):
%    \begin{macrocode}
\ifchilddoc
\providecommand{\version}{draft}
\else
\providecommand{\version}{final}
\fi
%    \end{macrocode}

% Load the standard document class:
%    \begin{macrocode}
\documentclass[12pt]{article}
%    \end{macrocode}

% Start the document body:
%    \begin{macrocode}
\begin{document}
%    \end{macrocode}

% Declare a title page.
% Print title, part of document being processed and version flag:
%    \begin{macrocode}
\addtocounter{page}{-1}
\begin{center}
{\LARGE\bfseries{}childdoc example\par}
\vspace{1cm}
\ifchilddoc
\ifchilddocmanual part\else chapter\fi:
`\childdocname' of `\childdocjob'\par
\else
main document: `\childdocjob'\par
\fi
version: \version\par
\end{center}
\newpage
%    \end{macrocode}

% Manually include selected file,
% otherwise process as usual:
%    \begin{macrocode}
\ifchilddocmanual
\section*{part `\childdocname'}
\input{\childdocname}
\else
%    \end{macrocode}

% Include the two chapters:
%    \begin{macrocode}
\include{cdocsch1}
\include{cdocsch2}
%    \end{macrocode}

% Include the two parts unless only chapters should be displayed:
%    \begin{macrocode}
\ifchilddoc\else
\section{part three}
\input{cdocspt3}
\section{part four}
\input{cdocspt4}
\fi
%    \end{macrocode}

% Process as usual until here:
%    \begin{macrocode}
\fi
%    \end{macrocode}

% End of document body:
%    \begin{macrocode}
\end{document}
%    \end{macrocode}
%\iffalse
%</samplemain>
%\fi
%
% %%%%%%%%%%%%%%%%%%%%%%%%%%%%%%%%%%%%%%
% \paragraph{Chapter Include Files.}
%
% The include files are called |cdocsch1.tex| and |cdocsch2.tex|.
%
%\iffalse
%<*samplechap1|samplechap2>
%\fi

% Optional override for |\version| flag:
%    \begin{macrocode}
%%\providecommand{\version}{final}
%    \end{macrocode}

% Include the main document:
%    \begin{macrocode}
\input{childdoc.def}
\childdocof{cdocsamp}
%    \end{macrocode}

%\iffalse
%</samplechap1|samplechap2>
%\fi
%
%\iffalse
%<*samplechap1>
%\fi
% Some text for chapter 1:
%    \begin{macrocode}
\section{one}
some text in chapter one
%    \end{macrocode}

%\iffalse
%</samplechap1>
%\fi
% Some text for chapter 2:
%\iffalse
%<*samplechap2>
%\fi
%    \begin{macrocode}
\section{two}
more text in chapter two
%    \end{macrocode}

%\iffalse
%</samplechap2>
%\fi
%
% %%%%%%%%%%%%%%%%%%%%%%%%%%%%%%%%%%%%%%
% \paragraph{Part Include Files.}
%
% The include files are called |cdocspt3.tex| and |cdocspt4.tex|.
%
%\iffalse
%<*samplepart3|samplepart4>
%\fi

% Optional override for |\version| flag:
%    \begin{macrocode}
%%\providecommand{\version}{final}
%    \end{macrocode}

% Include the main document:
%    \begin{macrocode}
\input{childdoc.def}
\childdocby{cdocsamp}
%    \end{macrocode}

%\iffalse
%</samplepart3|samplepart4>
%\fi
%
%\iffalse
%<*samplepart3>
%\fi
% Some text for part 3:
%    \begin{macrocode}
some text in part three
%    \end{macrocode}

%\iffalse
%</samplepart3>
%\fi
% Some text for part 4:
%\iffalse
%<*samplepart4>
%\fi
%    \begin{macrocode}
more text in part four
%    \end{macrocode}

%\iffalse
%</samplepart4>
%\fi
%
% %%%%%%%%%%%%%%%%%%%%%%%%%%%%%%%%%%%%%%
% \paragraph{Forwarding for a Complete Draft.}
%
% The following forwarding file |cdocsdrf.tex|
% compiles the main document in draft mode:
%\iffalse
%<*sampledraft>
%\fi
%    \begin{macrocode}
\def\version{draft}
\input{childdoc.def}
\childdocforward{cdocsamp}
%    \end{macrocode}

%\iffalse
%</sampledraft>
%\fi
%
% %%%%%%%%%%%%%%%%%%%%%%%%%%%%%%%%%%%%%%
% \paragraph{Forwarding for Final Version of the Chapters.}
%
% The following forwarding files |cdocsfn1.tex| and |cdocsfn2.tex|
% (with identical content)
% compile the final versions of the child documents
% |cdocsch1.tex| and |cdocsch2.tex|, respectively:
%\iffalse
%<*samplefinal>
%\fi
%    \begin{macrocode}
\def\version{final}
\input{childdoc.def}
\childdocforwardprefix[cdocsamp]{cdocsfn}{cdocsch}
%    \end{macrocode}

%\iffalse
%</samplefinal>
%\fi
%
% %%%%%%%%%%%%%%%%%%%%%%%%%%%%%%%%%%%%%%
% \paragraph{Command Line Processing.}
%
% The following three command lines generate the output files
% |cdocscld|, |cdocscl1| and |cdocscl2|
% which should be identical to
% |cdocsdrf|, |cdocsch1| and |cdocsfn2|, respectively:
% \begin{center}
% \begin{tabular}{l}
% |latex -jobname cdocscld \|\\
% |  "\def\version{draft}\input{childdoc.def}\childdocforward{cdocsamp}"|\\
% |latex -jobname cdocscl1 \|\\
% |  "\input{childdoc.def}\childdocforward[cdocsamp]{cdocsch1}"|\\
% |latex -jobname cdocscl2 \|\\
% |  "\def\version{final}\input{childdoc.def}\childdocforward{cdocsch2}"|
% \end{tabular}
% \end{center}
% Note that the trailing backslash on each first line
% merely continues the input to the second line
% (for convenient cut ant paste).
% Furthermore, the command |latex| can be replaced by any
% of its alternative versions such as |pdflatex|.
%
% %%%%%%%%%%%%%%%%%%%%%%%%%%%%%%%%%%%%%%%%%%%%%%%%%%%%%%%%%%%%%%%%%%%%%%%%%%%%%%
% %%%%%%%%%%%%%%%%%%%%%%%%%%%%%%%%%%%%%%%%%%%%%%%%%%%%%%%%%%%%%%%%%%%%%%%%%%%%%%
% \section{Implementation}
%\iffalse
%<*package>
%\fi
%
% This section describes the definitions file |childdoc.def|.

% The definitions cannot be loaded using |\usepackage| or |\RequirePackage|
% which has a mechanism to prevent loading a style file more than once.
% When loading the definitions by means of |\input|
% multiple instances have to be prevented manually:
%\iffalse
%This code needs to be before the `\ProvidesFile' directive
%which is defined at the beginning of this file.
%Therefore it is also placed there and commented out here.
%</package>
%<*discard>
%\fi
%    \begin{macrocode}
\ifdefined\childdocmain\endinput\fi
%    \end{macrocode}
%\iffalse
%</discard>
%<*package>
%\fi
%
% \macro{\ifchilddoc}
% \macro{\ifchilddocmanual}
% The conditional |\ifchilddoc| tells whether a
% child (true) or main (false) document is being compiled.
% The conditional |\ifchilddocmanual| tells whether
% the |\includeonly| mechanism is used (false) or
% the selection of child files must be performed manually (true).
% The definitions initialise to false:
%    \begin{macrocode}
\newif\ifchilddoc
\newif\ifchilddocmanual
%    \end{macrocode}

% \macro{\childdocname}
% \macro{\childdocjob}
% The macro |\childdocname| stores the name of the main document
% to be compiled. The macro |\childdocjob| stores the name of
% the document on which the \LaTeX{} compiler was originally invoked.
% The content of |\jobname| cannot be compared
% to filenames specified in the source due to different catcodes.
% The following code rescans |\jobname|, stores the result
% in |\childdocname| and saves a copy in |\childdocjob|:
%    \begin{macrocode}
\edef\childdocname{\scantokens\expandafter{\jobname\noexpand}}
\let\childdocjob\childdocname
%    \end{macrocode}

% \macro{\childdocdisable}
% The macro |\childdocdisable| prevents the main file
% from being processed more than once.
% At this stage, the main document command |\childdocmain|
% is assumed to be called once again where it should do nothing.
% Any subsequent call to it should prevent
% a secondary processing of the main document
% It overwrites the forwarding commands
% |\childdocof| and |\childdocforward|
% with empty macros to prevent further inclusions of the main document:
%    \begin{macrocode}
\newcommand{\childdocdisable}
{
  \renewcommand{\childdocmain}[1]{\renewcommand{\childdocmain}[1]{\endinput}}
  \renewcommand{\childdocof}[1]{}
  \renewcommand{\childdocby}[2][]{}
  \renewcommand{\childdocforward}[2][]{}
  \renewcommand{\childdocdisable}{}
}
%    \end{macrocode}

% \macro{\childdocmain}
% The macro |\childdocmain| is to be called at the top of the main file
% with nothing or the main filename (without extension) as argument.
% First, it breaks loops.
% If the argument is not empty and does not match |\childdocname|
% (which is set by the first inclusion of |childdoc.def|),
% |\ifchilddoc| is set to true, |\includeonly| is applied to the child file
% and |\jobname| is set to the main file
% (for proper handling of |.aux| files):
%    \begin{macrocode}
\newcommand{\childdocmain}[1]
{
  \childdocdisable\childdocmain{}
  \if?#1?\else
    \begingroup
      \def\childdoctmp{#1}
      \ifx\childdoctmp\childdocname
        \def\childdoctmp{}
      \else
        \def\childdoctmp
        {
          \childdoctrue
          \includeonly{\childdocname}
          \def\childdocjob{#1}
          \def\jobname{#1}
        }
      \fi
      \expandafter
    \endgroup
    \childdoctmp
  \fi
}
%    \end{macrocode}

% \macro{\childdocof}
% The command |\childdocof| redirects
% compilation to the main file |#1|.
%    \begin{macrocode}
\newcommand{\childdocof}[1]
{
  \childdocdisable
  \childdoctrue
  \includeonly{\childdocname}
  \def\jobname{#1}
  \def\childdocjob{#1}
  \input{#1}
}
%    \end{macrocode}

% \macro{\childdocby}
% The command |\childdocby| ....
%    \begin{macrocode}
\newcommand{\childdocby}[2][]
{
  \childdocdisable
  \childdoctrue
  \childdocmanualtrue
  \if?#1?\else
    \def\jobname{#2}
  \fi
  \def\childdocjob{#2}
  \input{#2}
  \endinput
}
%    \end{macrocode}

% \macro{\childdocforward}
% The command |\childdocforward| redirects
% compilation to the main file or
% (if the optional argument is given) a child file.
% Parameters are set as if the main file
% or a child file starting with |\childdocof| was compiled.
% Then compilation is handed over to the main file:
%    \begin{macrocode}
\newcommand{\childdocforward}[2][]
{
  \begingroup
    \if?#1?
      \def\childdoctmp
      {
        \def\childdocname{#2}
        \def\childdocjob{#2}
        \def\jobname{#2}
        \input{#2}
        \endinput
      }
    \else
      \def\childdoctmp
      {
        \childdocdisable
        \def\childdocname{#2}
        \childdoctrue
        \includeonly{#2}
        \def\childdocjob{#1}
        \def\jobname{#1}
        \input{#1}
        \endinput
      }
    \fi
    \expandafter
  \endgroup
  \childdoctmp
}
%    \end{macrocode}

% \macro{\childdocforwardprefix}
% The command |\childdocforwardprefix| redirects
% compilation to the main or a child file by means of a pattern.
% The prefix |#1| in the current filename is replaced by |#2|
% and the suffix of the current filename is kept
% (it is assumed that the filename does not contain the substring `|~~~|'
% which is used as a delimiter).
% Compilation is handed over to the new file by |\childdocforward|:
%    \begin{macrocode}
\newcommand{\childdocforwardprefix}[3][]
{
  \begingroup
    \def\childdocextract #2##1~~~{\def\childdoctmp{\childdocforward[#1]{#3##1}}}
    \expandafter\childdocextract\childdocname~~~
    \expandafter
  \endgroup
  \childdoctmp
}
%    \end{macrocode}

% \macro{\childdoc}
% The deprecated macro |\childdoc| is a legacy version of |\childdocmain|:
%    \begin{macrocode}
\newcommand{\childdoc}{\childdocmain}
%    \end{macrocode}

% \macro{\childdocredirect}
% The deprecated macro |\childdocredirect| is a legacy version
% of |\childdocforward| and |\childdocforwardprefix|:
%    \begin{macrocode}
\newcommand{\childdocredirect}[2][]
{
  \begingroup
    \if?#1?
      \def\childdoctmp{\childdocforward{#2}}
    \else
      \def\childdoctmp{\childdocforwardprefix{#1}{#2}}
    \fi
    \expandafter
  \endgroup
  \childdoctmp
}
%    \end{macrocode}

%\iffalse
%</package>
%\fi
%
\endinput
\childdocforward{cdocsch2}"|
% \end{tabular}
% \end{center}
% Note that the trailing backslash on each first line
% merely continues the input to the second line
% (for convenient cut ant paste).
% Furthermore, the command |latex| can be replaced by any
% of its alternative versions such as |pdflatex|.
%
% %%%%%%%%%%%%%%%%%%%%%%%%%%%%%%%%%%%%%%%%%%%%%%%%%%%%%%%%%%%%%%%%%%%%%%%%%%%%%%
% %%%%%%%%%%%%%%%%%%%%%%%%%%%%%%%%%%%%%%%%%%%%%%%%%%%%%%%%%%%%%%%%%%%%%%%%%%%%%%
% \section{Implementation}
%\iffalse
%<*package>
%\fi
%
% This section describes the definitions file |childdoc.def|.

% The definitions cannot be loaded using |\usepackage| or |\RequirePackage|
% which has a mechanism to prevent loading a style file more than once.
% When loading the definitions by means of |\input|
% multiple instances have to be prevented manually:
%\iffalse
%This code needs to be before the `\ProvidesFile' directive
%which is defined at the beginning of this file.
%Therefore it is also placed there and commented out here.
%</package>
%<*discard>
%\fi
%    \begin{macrocode}
\ifdefined\childdocmain\endinput\fi
%    \end{macrocode}
%\iffalse
%</discard>
%<*package>
%\fi
%
% \macro{\ifchilddoc}
% \macro{\ifchilddocmanual}
% The conditional |\ifchilddoc| tells whether a
% child (true) or main (false) document is being compiled.
% The conditional |\ifchilddocmanual| tells whether
% the |\includeonly| mechanism is used (false) or
% the selection of child files must be performed manually (true).
% The definitions initialise to false:
%    \begin{macrocode}
\newif\ifchilddoc
\newif\ifchilddocmanual
%    \end{macrocode}

% \macro{\childdocname}
% \macro{\childdocjob}
% The macro |\childdocname| stores the name of the main document
% to be compiled. The macro |\childdocjob| stores the name of
% the document on which the \LaTeX{} compiler was originally invoked.
% The content of |\jobname| cannot be compared
% to filenames specified in the source due to different catcodes.
% The following code rescans |\jobname|, stores the result
% in |\childdocname| and saves a copy in |\childdocjob|:
%    \begin{macrocode}
\edef\childdocname{\scantokens\expandafter{\jobname\noexpand}}
\let\childdocjob\childdocname
%    \end{macrocode}

% \macro{\childdocdisable}
% The macro |\childdocdisable| prevents the main file
% from being processed more than once.
% At this stage, the main document command |\childdocmain|
% is assumed to be called once again where it should do nothing.
% Any subsequent call to it should prevent
% a secondary processing of the main document
% It overwrites the forwarding commands
% |\childdocof| and |\childdocforward|
% with empty macros to prevent further inclusions of the main document:
%    \begin{macrocode}
\newcommand{\childdocdisable}
{
  \renewcommand{\childdocmain}[1]{\renewcommand{\childdocmain}[1]{\endinput}}
  \renewcommand{\childdocof}[1]{}
  \renewcommand{\childdocby}[2][]{}
  \renewcommand{\childdocforward}[2][]{}
  \renewcommand{\childdocdisable}{}
}
%    \end{macrocode}

% \macro{\childdocmain}
% The macro |\childdocmain| is to be called at the top of the main file
% with nothing or the main filename (without extension) as argument.
% First, it breaks loops.
% If the argument is not empty and does not match |\childdocname|
% (which is set by the first inclusion of |childdoc.def|),
% |\ifchilddoc| is set to true, |\includeonly| is applied to the child file
% and |\jobname| is set to the main file
% (for proper handling of |.aux| files):
%    \begin{macrocode}
\newcommand{\childdocmain}[1]
{
  \childdocdisable\childdocmain{}
  \if?#1?\else
    \begingroup
      \def\childdoctmp{#1}
      \ifx\childdoctmp\childdocname
        \def\childdoctmp{}
      \else
        \def\childdoctmp
        {
          \childdoctrue
          \includeonly{\childdocname}
          \def\childdocjob{#1}
          \def\jobname{#1}
        }
      \fi
      \expandafter
    \endgroup
    \childdoctmp
  \fi
}
%    \end{macrocode}

% \macro{\childdocof}
% The command |\childdocof| redirects
% compilation to the main file |#1|.
%    \begin{macrocode}
\newcommand{\childdocof}[1]
{
  \childdocdisable
  \childdoctrue
  \includeonly{\childdocname}
  \def\jobname{#1}
  \def\childdocjob{#1}
  \input{#1}
}
%    \end{macrocode}

% \macro{\childdocby}
% The command |\childdocby| ....
%    \begin{macrocode}
\newcommand{\childdocby}[2][]
{
  \childdocdisable
  \childdoctrue
  \childdocmanualtrue
  \if?#1?\else
    \def\jobname{#2}
  \fi
  \def\childdocjob{#2}
  \input{#2}
  \endinput
}
%    \end{macrocode}

% \macro{\childdocforward}
% The command |\childdocforward| redirects
% compilation to the main file or
% (if the optional argument is given) a child file.
% Parameters are set as if the main file
% or a child file starting with |\childdocof| was compiled.
% Then compilation is handed over to the main file:
%    \begin{macrocode}
\newcommand{\childdocforward}[2][]
{
  \begingroup
    \if?#1?
      \def\childdoctmp
      {
        \def\childdocname{#2}
        \def\childdocjob{#2}
        \def\jobname{#2}
        \input{#2}
        \endinput
      }
    \else
      \def\childdoctmp
      {
        \childdocdisable
        \def\childdocname{#2}
        \childdoctrue
        \includeonly{#2}
        \def\childdocjob{#1}
        \def\jobname{#1}
        \input{#1}
        \endinput
      }
    \fi
    \expandafter
  \endgroup
  \childdoctmp
}
%    \end{macrocode}

% \macro{\childdocforwardprefix}
% The command |\childdocforwardprefix| redirects
% compilation to the main or a child file by means of a pattern.
% The prefix |#1| in the current filename is replaced by |#2|
% and the suffix of the current filename is kept
% (it is assumed that the filename does not contain the substring `|~~~|'
% which is used as a delimiter).
% Compilation is handed over to the new file by |\childdocforward|:
%    \begin{macrocode}
\newcommand{\childdocforwardprefix}[3][]
{
  \begingroup
    \def\childdocextract #2##1~~~{\def\childdoctmp{\childdocforward[#1]{#3##1}}}
    \expandafter\childdocextract\childdocname~~~
    \expandafter
  \endgroup
  \childdoctmp
}
%    \end{macrocode}

% \macro{\childdoc}
% The deprecated macro |\childdoc| is a legacy version of |\childdocmain|:
%    \begin{macrocode}
\newcommand{\childdoc}{\childdocmain}
%    \end{macrocode}

% \macro{\childdocredirect}
% The deprecated macro |\childdocredirect| is a legacy version
% of |\childdocforward| and |\childdocforwardprefix|:
%    \begin{macrocode}
\newcommand{\childdocredirect}[2][]
{
  \begingroup
    \if?#1?
      \def\childdoctmp{\childdocforward{#2}}
    \else
      \def\childdoctmp{\childdocforwardprefix{#1}{#2}}
    \fi
    \expandafter
  \endgroup
  \childdoctmp
}
%    \end{macrocode}

%\iffalse
%</package>
%\fi
%
\endinput
|\\
|\childdocby{|\textit{main}|}|\\
\end{tabular}
\end{center}
%
The directive |\childdocby| is similar to |\childdocof|
described in \secref{sec:include},
but the subsequent selection of content must be done manually.
To that end, both |\ifchilddoc| and |\ifchilddocmanual|
will be true upon processing of a part,
and the name of the part is stored in |\childdocname|.
Note that |\jobname| will be set to the filename of the current part
so that each part receives an individual |.aux| file
that does not interfere with the |.aux| file(s) of the main document.
This behaviour can be altered by the alternative form
|\childdocby[*]{|\textit{main}|}| (with a non-empty optional argument)
which uses the |.aux| file of the main document
by setting |\jobname| to \textit{main}.

%%%%%%%%%%%%%%%%%%%%%%%%%%%%%%%%%%%%%%%%%%%%%%%%%%%%%%%%%%%%%%%%%%%%%%%%%%%%%%%%
\subsection{Driver Development}
\label{sec:driver}

The \textsf{childdoc} mechanism can also be use for the development
of definition files such as \LaTeX{} styles or classes.
This case differs from the above setup with multiple parts
included by |\include| in that no |\includeonly| should be invoked.
This can be achieved by starting the include file
(before |\ProvidesPackage|) with:
%
\begin{center}
\begin{tabular}{l}
|% \iffalse
%
% childdoc.dtx Copyright (C) 2017-2018 Niklas Beisert
%
% This work may be distributed and/or modified under the
% conditions of the LaTeX Project Public License, either version 1.3
% of this license or (at your option) any later version.
% The latest version of this license is in
%   http://www.latex-project.org/lppl.txt
% and version 1.3 or later is part of all distributions of LaTeX
% version 2005/12/01 or later.
%
% This work has the LPPL maintenance status `maintained'.
%
% The Current Maintainer of this work is Niklas Beisert.
%
% This work consists of the files childdoc.dtx and childdoc.ins
% and the derived files childdoc.def and cdocsamp.tex with
% cdocsch1.tex, cdocsch2.tex, cdocsdrf.tex, cdocsfn1.tex, cdocsfn2.tex.
%
%<package>\ifdefined\childdocmain\endinput\fi
%<package>\ProvidesFile{childdoc.def}[2018/12/30 v2.0 child document driver]
%<samplemain>\ProvidesFile{cdocsamp.tex}[2018/12/30 v2.0 sample for childdoc]
%<*driver>
%\ProvidesFile{childdoc.drv}[2018/12/30 v2.0 childdoc reference manual file]
\PassOptionsToClass{10pt,a4paper}{article}
\documentclass{ltxdoc}

\usepackage[margin=35mm]{geometry}
\usepackage{hyperref}
\usepackage{hyperxmp}
\usepackage[usenames]{color}

\hypersetup{colorlinks=true}
\hypersetup{pdfstartview=FitH}
\hypersetup{pdfpagemode=UseNone}
\hypersetup{pdfsource={}}
\hypersetup{pdflang={en-UK}}
\hypersetup{pdfcopyright={Copyright 2017-2018 Niklas Beisert.
  This work may be distributed and/or modified under the
  conditions of the LaTeX Project Public License, either version 1.3
  of this license or (at your option) any later version.}}
\hypersetup{pdflicenseurl={http://www.latex-project.org/lppl.txt}}
\hypersetup{pdfcontactaddress={ETH Zurich, ITP, HIT K,
  Wolfgang-Pauli-Strasse 27}}
\hypersetup{pdfcontactpostcode={8093}}
\hypersetup{pdfcontactcity={Zurich}}
\hypersetup{pdfcontactcountry={Switzerland}}
\hypersetup{pdfcontactemail={nbeisert@itp.phys.ethz.ch}}
\hypersetup{pdfcontacturl={http://people.phys.ethz.ch/\xmptilde nbeisert/}}

\newcommand{\secref}[1]{\hyperref[#1]{section \ref*{#1}}}

\parskip1ex
\parindent0pt
\let\olditemize\itemize
\def\itemize{\olditemize\parskip0pt}

\begin{document}

\title{The \textsf{childdoc} Package}
\hypersetup{pdftitle={The childdoc Package}}
\author{Niklas Beisert\\[2ex]
  Institut f\"ur Theoretische Physik\\
  Eidgen\"ossische Technische Hochschule Z\"urich\\
  Wolfgang-Pauli-Strasse 27, 8093 Z\"urich, Switzerland\\[1ex]
  \href{mailto:nbeisert@itp.phys.ethz.ch}
  {\texttt{nbeisert@itp.phys.ethz.ch}}}
\hypersetup{pdfauthor={Niklas Beisert}}
\hypersetup{pdfsubject={Manual for the LaTeX2e Package childdoc}}
\date{30 December 2018, \textsf{v2.0}}
\maketitle

\begin{abstract}\noindent
\textsf{childdoc} is a \LaTeXe{} package
that enables the direct compilation
of document sections included by |\include|
to individual files.
\end{abstract}

\begingroup
\parskip0ex
\tableofcontents
\endgroup

%%%%%%%%%%%%%%%%%%%%%%%%%%%%%%%%%%%%%%%%%%%%%%%%%%%%%%%%%%%%%%%%%%%%%%%%%%%%%%%%
%%%%%%%%%%%%%%%%%%%%%%%%%%%%%%%%%%%%%%%%%%%%%%%%%%%%%%%%%%%%%%%%%%%%%%%%%%%%%%%%
\section{Introduction}

\LaTeX{} provides a mechanism to structure a large document (such as a book)
into a main file and several child files (containing the chapters)
using the |\include| command.
This mechanism is beneficial for documents
which span hundreds of pages in order to
make the source file(s) more manageable.
Moreover, compilation can be restricted to
selected child files by means of the |\includeonly| command.
The latter feature can be used to reduce the compilation time while editing
(this was significantly more useful in the earlier days of \LaTeX{})
or to generate a smaller document which is easier to navigate.
Another application of |\includeonly| is to generate
documents consisting of selected parts of the complete document.

However, there are a few drawbacks of the plain |\include| mechanism:
\begin{itemize}
\item
The child files cannot be compiled on their own,
they can only be compiled via the main file.
A naive editing environment
(such as a text editor with an option
to have the current file processed by \LaTeX)
may require one to switch to the main file before compiling;
attempting to compile the child file produces errors.
\item
The main file must be modified (each time)
to adjust the |\includeonly| command
to the present needs. This easily leaves the main file in a messy state.
\item
The generated document will always carry the filename
of the main document. This is inconvenient if
several child files are to be compiled and
to be kept for distribution.
\end{itemize}

The present package provides a simple interface
to make child files individually compilable by \LaTeX{}.
Compiling a child file then has the same effect as compiling
the main file with an |\includeonly| command
to select the appropriate child.
Moreover the generated document will carry the name of the child
rather than the main file.
This resolves all three above issues.

This feature is meant to make the editing of books,
thesis documents and lecture notes somewhat more convenient.
However, the package can also be used efficiently for
composing a series of documents (such as exercise sheets)
which are typically distributed individually.
It then assists the author in generating the individual documents
(potentially in different versions)
as well as a document containing the collected series.
Another application is in developing style files
or other kinds of included material
where compilation of the style file could redirect
to a sample or test file.

%%%%%%%%%%%%%%%%%%%%%%%%%%%%%%%%%%%%%%%%%%%%%%%%%%%%%%%%%%%%%%%%%%%%%%%%%%%%%%%%
%%%%%%%%%%%%%%%%%%%%%%%%%%%%%%%%%%%%%%%%%%%%%%%%%%%%%%%%%%%%%%%%%%%%%%%%%%%%%%%%
\section{Usage}

First of all, the package \textsf{childdoc} is \emph{not} a standard
\LaTeXe{} |.sty| style file! Therefore it needs to be invoked in
a non-standard way.

%%%%%%%%%%%%%%%%%%%%%%%%%%%%%%%%%%%%%%%%%%%%%%%%%%%%%%%%%%%%%%%%%%%%%%%%%%%%%%%%
\subsection{Included Files}
\label{sec:include}

%%%%%%%%%%%%%%%%%%%%%%%%%%%%%%%%%%%%%%%%
\DescribeMacro{\childdocmain}
To use the package, add the commands
\begin{center}
\begin{tabular}{l}
|% \iffalse
%
% childdoc.dtx Copyright (C) 2017-2018 Niklas Beisert
%
% This work may be distributed and/or modified under the
% conditions of the LaTeX Project Public License, either version 1.3
% of this license or (at your option) any later version.
% The latest version of this license is in
%   http://www.latex-project.org/lppl.txt
% and version 1.3 or later is part of all distributions of LaTeX
% version 2005/12/01 or later.
%
% This work has the LPPL maintenance status `maintained'.
%
% The Current Maintainer of this work is Niklas Beisert.
%
% This work consists of the files childdoc.dtx and childdoc.ins
% and the derived files childdoc.def and cdocsamp.tex with
% cdocsch1.tex, cdocsch2.tex, cdocsdrf.tex, cdocsfn1.tex, cdocsfn2.tex.
%
%<package>\ifdefined\childdocmain\endinput\fi
%<package>\ProvidesFile{childdoc.def}[2018/12/30 v2.0 child document driver]
%<samplemain>\ProvidesFile{cdocsamp.tex}[2018/12/30 v2.0 sample for childdoc]
%<*driver>
%\ProvidesFile{childdoc.drv}[2018/12/30 v2.0 childdoc reference manual file]
\PassOptionsToClass{10pt,a4paper}{article}
\documentclass{ltxdoc}

\usepackage[margin=35mm]{geometry}
\usepackage{hyperref}
\usepackage{hyperxmp}
\usepackage[usenames]{color}

\hypersetup{colorlinks=true}
\hypersetup{pdfstartview=FitH}
\hypersetup{pdfpagemode=UseNone}
\hypersetup{pdfsource={}}
\hypersetup{pdflang={en-UK}}
\hypersetup{pdfcopyright={Copyright 2017-2018 Niklas Beisert.
  This work may be distributed and/or modified under the
  conditions of the LaTeX Project Public License, either version 1.3
  of this license or (at your option) any later version.}}
\hypersetup{pdflicenseurl={http://www.latex-project.org/lppl.txt}}
\hypersetup{pdfcontactaddress={ETH Zurich, ITP, HIT K,
  Wolfgang-Pauli-Strasse 27}}
\hypersetup{pdfcontactpostcode={8093}}
\hypersetup{pdfcontactcity={Zurich}}
\hypersetup{pdfcontactcountry={Switzerland}}
\hypersetup{pdfcontactemail={nbeisert@itp.phys.ethz.ch}}
\hypersetup{pdfcontacturl={http://people.phys.ethz.ch/\xmptilde nbeisert/}}

\newcommand{\secref}[1]{\hyperref[#1]{section \ref*{#1}}}

\parskip1ex
\parindent0pt
\let\olditemize\itemize
\def\itemize{\olditemize\parskip0pt}

\begin{document}

\title{The \textsf{childdoc} Package}
\hypersetup{pdftitle={The childdoc Package}}
\author{Niklas Beisert\\[2ex]
  Institut f\"ur Theoretische Physik\\
  Eidgen\"ossische Technische Hochschule Z\"urich\\
  Wolfgang-Pauli-Strasse 27, 8093 Z\"urich, Switzerland\\[1ex]
  \href{mailto:nbeisert@itp.phys.ethz.ch}
  {\texttt{nbeisert@itp.phys.ethz.ch}}}
\hypersetup{pdfauthor={Niklas Beisert}}
\hypersetup{pdfsubject={Manual for the LaTeX2e Package childdoc}}
\date{30 December 2018, \textsf{v2.0}}
\maketitle

\begin{abstract}\noindent
\textsf{childdoc} is a \LaTeXe{} package
that enables the direct compilation
of document sections included by |\include|
to individual files.
\end{abstract}

\begingroup
\parskip0ex
\tableofcontents
\endgroup

%%%%%%%%%%%%%%%%%%%%%%%%%%%%%%%%%%%%%%%%%%%%%%%%%%%%%%%%%%%%%%%%%%%%%%%%%%%%%%%%
%%%%%%%%%%%%%%%%%%%%%%%%%%%%%%%%%%%%%%%%%%%%%%%%%%%%%%%%%%%%%%%%%%%%%%%%%%%%%%%%
\section{Introduction}

\LaTeX{} provides a mechanism to structure a large document (such as a book)
into a main file and several child files (containing the chapters)
using the |\include| command.
This mechanism is beneficial for documents
which span hundreds of pages in order to
make the source file(s) more manageable.
Moreover, compilation can be restricted to
selected child files by means of the |\includeonly| command.
The latter feature can be used to reduce the compilation time while editing
(this was significantly more useful in the earlier days of \LaTeX{})
or to generate a smaller document which is easier to navigate.
Another application of |\includeonly| is to generate
documents consisting of selected parts of the complete document.

However, there are a few drawbacks of the plain |\include| mechanism:
\begin{itemize}
\item
The child files cannot be compiled on their own,
they can only be compiled via the main file.
A naive editing environment
(such as a text editor with an option
to have the current file processed by \LaTeX)
may require one to switch to the main file before compiling;
attempting to compile the child file produces errors.
\item
The main file must be modified (each time)
to adjust the |\includeonly| command
to the present needs. This easily leaves the main file in a messy state.
\item
The generated document will always carry the filename
of the main document. This is inconvenient if
several child files are to be compiled and
to be kept for distribution.
\end{itemize}

The present package provides a simple interface
to make child files individually compilable by \LaTeX{}.
Compiling a child file then has the same effect as compiling
the main file with an |\includeonly| command
to select the appropriate child.
Moreover the generated document will carry the name of the child
rather than the main file.
This resolves all three above issues.

This feature is meant to make the editing of books,
thesis documents and lecture notes somewhat more convenient.
However, the package can also be used efficiently for
composing a series of documents (such as exercise sheets)
which are typically distributed individually.
It then assists the author in generating the individual documents
(potentially in different versions)
as well as a document containing the collected series.
Another application is in developing style files
or other kinds of included material
where compilation of the style file could redirect
to a sample or test file.

%%%%%%%%%%%%%%%%%%%%%%%%%%%%%%%%%%%%%%%%%%%%%%%%%%%%%%%%%%%%%%%%%%%%%%%%%%%%%%%%
%%%%%%%%%%%%%%%%%%%%%%%%%%%%%%%%%%%%%%%%%%%%%%%%%%%%%%%%%%%%%%%%%%%%%%%%%%%%%%%%
\section{Usage}

First of all, the package \textsf{childdoc} is \emph{not} a standard
\LaTeXe{} |.sty| style file! Therefore it needs to be invoked in
a non-standard way.

%%%%%%%%%%%%%%%%%%%%%%%%%%%%%%%%%%%%%%%%%%%%%%%%%%%%%%%%%%%%%%%%%%%%%%%%%%%%%%%%
\subsection{Included Files}
\label{sec:include}

%%%%%%%%%%%%%%%%%%%%%%%%%%%%%%%%%%%%%%%%
\DescribeMacro{\childdocmain}
To use the package, add the commands
\begin{center}
\begin{tabular}{l}
|\input{childdoc.def}|\\
|\childdocmain{}|\\
\end{tabular}
\end{center}
at the very top of the main \LaTeX{} file,
in particular \emph{before} the |\documentclass| statement!
The argument of |\childdocmain| should be left empty
(but it must be present).

%%%%%%%%%%%%%%%%%%%%%%%%%%%%%%%%%%%%%%%%
\DescribeMacro{\childdocof}
Furthermore, add the commands
\begin{center}
\begin{tabular}{l}
|\input{childdoc.def}|\\
|\childdocof{|\textit{main}|}|\\
\end{tabular}
\end{center}
at the top of every child file \textit{child}
which is included by |\include{|\textit{child}|}|
from within the main file
(or at least for those files to be compiled individually).
The argument \textit{main} must be the filename of the main file.

There are a couple of
considerations in setting up the main and child documents:

%%%%%%%%%%%%%%%%%%%%%%%%%%%%%%%%%%%%%%%%
\paragraph{Restrictions.}

Please note the following restrictions:
\begin{itemize}
\item
|\childdocmain| must be called with one argument \textit{main}
to ensure compatibility with earlier version of the package.
It must either be empty (|\childdocmain{}|)
or precisely match the filename of the main file in which it is specified.
See \secref{sec:detection} for further information.
\item
The filename \textit{main} must be specified without the |.tex| extension.
\item
The filename \textit{main} is case sensitive
(even in case-insensitive file systems)
due to internal string comparison.
\item
The argument \textit{main} should be fully expanded, it cannot be a macro.
\item
Subdirectories and special characters should be avoided in filenames.
\item
The command |\childdocmain{|\textit{main}|}| must be followed by a whitespace.
It should not be followed immediately by another command
or by a comment mark `|%|'.
This is because the \TeX{} parser reads the token immediately following
the argument of |\childdocmain| and puts it
at the beginning of every child section;
however, a white\-space is ignored.
\end{itemize}

%%%%%%%%%%%%%%%%%%%%%%%%%%%%%%%%%%%%%%%%
\paragraph{Content of Main File.}

It is advisable to place all content in the child files included by |\include|.
Any output contained in the main file will appear in all child documents
unless suppressed manually;
it cannot be suppressed automatically by the |\includeonly| directive
and thus should normally be avoided.
A method to include some content in the main file
by means of conditional processing is described in \secref{sec:conditional}.

%%%%%%%%%%%%%%%%%%%%%%%%%%%%%%%%%%%%%%%%
\paragraph{Page Numbering.}

When only a part of the document is compiled,
the appropriate numbering of pages
(as well as other status parameters)
is determined from the |.aux| files.
The latter contain information from previous passes.
However this information needs to propagate through
all intermediate child documents.
Therefore the page numbering in child documents may well
be inconsistent until the complete document is compiled at least once.

A useful (if unconventional) way to always ensure a consistent
page numbering is to restart the numbering in each child document
and denote the pages by `\textit{child}|.|\textit{page}'
where \textit{child} represents the chapter/section number of the child file.
This can be achieved by the command
|\numberwithin{page}{|\textit{child}|}|
of the \textsf{amsmath} package
where \textit{child} can be |chapter| or |section|
depending on the chosen structuring.
Alternatively, one can modify the macro |\thepage| appropriately
and reset the counter |page| at the start of each child file.

%%%%%%%%%%%%%%%%%%%%%%%%%%%%%%%%%%%%%%%%%%%%%%%%%%%%%%%%%%%%%%%%%%%%%%%%%%%%%%%%
\subsection{Conditional Processing}
\label{sec:conditional}

The package provides a mechanism to compile different versions
of a document. To customise the versions further some conditional processing
can come in handy to distinguish which version is being compiled.
The package provides two macros to describe the compilation context:

%%%%%%%%%%%%%%%%%%%%%%%%%%%%%%%%%%%%%%%%
\DescribeMacro{\ifchilddoc}
The conditional |\ifchilddoc| distinguishes between the compilation of
child documents and the main document:
%
\begin{center}
|\ifchilddoc |\textit{child-code}| |[|\||else |\textit{main-code}]| \||fi|
\end{center}

%%%%%%%%%%%%%%%%%%%%%%%%%%%%%%%%%%%%%%%%
\DescribeMacro{\childdocname}
\DescribeMacro{\childdocjob}
The macro |\childdocname| contains the filename (without extension)
of the main or child file being processed.
Note that |\childdocjob| will always contain the name of the main file.

%%%%%%%%%%%%%%%%%%%%%%%%%%%%%%%%%%%%%%%%
\paragraph{Title Page.}

Conditional processing can be used to include a title or banner page
in the main document when proper precautions are taken.
Importantly, the code in the main file should ensure that the page counter
(as well as other status parameters which are stored in the |.aux| files)
takes the same value after the conditional processing.
Otherwise the page numbers may take divergent values
depending on which part is compiled.

For example, a title page could be declared by:
%
\begin{center}
\begin{tabular}{l}
|\ifchilddoc\||else|\\
|\addtocounter{page}{-1}|\\
\textit{code for title page}\\
|\newpage|\\
|\||fi|
\end{tabular}
\end{center}
%
A banner page for the child documents can be generated by:
%
\begin{center}
\begin{tabular}{l}
|\ifchilddoc|\\
|\addtocounter{page}{-1}|\\
\textit{code for banner page}\\
|\newpage|\\
|\||fi|
\end{tabular}
\end{center}
%
Here one could write a message such as:
\begin{center}
|This is the part \childdocname{} of \childdocjob{}.|
\end{center}

%%%%%%%%%%%%%%%%%%%%%%%%%%%%%%%%%%%%%%%%%%%%%%%%%%%%%%%%%%%%%%%%%%%%%%%%%%%%%%%%
\subsection{Flags}
\label{sec:flags}

The package makes it easy to generate different versions
of the main or child documents.
To this end compilation flags can be defined
and assigned different default values.
They will be particularly useful in conjunction
with the forwarding mechanism described in \secref{sec:forward}.

For example, it may be useful to have a flag |\version|
which can be set to |draft| or |final|.
The document source will contain some conditional code
depending on the value of |\version|.
Suppose further, the flag should default to |final| for the main file
and to |draft| for child files
which is a natural assignment for editing the document.
This is achieved by placing the following code
in the preamble of the main document
(below the |\childdocmain| directive):
%
\begin{center}
\begin{tabular}{l}
|\ifchilddoc|\\
|\providecommand{\version}{draft}|\\
|\||else|\\
|\providecommand{\version}{final}|\\
|\||fi|
\end{tabular}
\end{center}
%
The definition by |\providecommand| makes sure
that previous definitions are not overwritten.
Further statements |\providecommand{\version}{...}|
can thus be added before the above code to override it.

For the main file, one might add a line
(between |\childdocmain| and the above block)
%
\begin{center}
|%\ifchilddoc\||else\providecommand{\version}{draft}\||fi|
\end{center}
%
which can be uncommented to produce a draft version.
Likewise one can add a line to the very top of a child file
(above the |\childdocof{|\textit{main}|}| directive)
%
\begin{center}
|%\providecommand{\version}{final}|
\end{center}
%
which can be uncommented to produce the final version of this child document.

%%%%%%%%%%%%%%%%%%%%%%%%%%%%%%%%%%%%%%%%%%%%%%%%%%%%%%%%%%%%%%%%%%%%%%%%%%%%%%%%
\subsection{Forwarding}
\label{sec:forward}

Different versions of the main or child documents
using compilation flags as described in \secref{sec:flags}
can be (permanently) stored in different files
for convenient compilation, viewing and distribution.
To this end, the package defines a command
to pass on compilation to a different file:

%%%%%%%%%%%%%%%%%%%%%%%%%%%%%%%%%%%%%%%%
\DescribeMacro{\childdocforward}
The command |\childdocforward| redirects processing to
another source file:
%
\begin{center}
\begin{tabular}{l}
|\input{childdoc.def}|\\
|\childdocforward[|\textit{main}|]{|\textit{dest}|}|\\
\end{tabular}
\end{center}
%
The argument \textit{dest} is the destination file
(without extension).
It should be the main file or one of the child files.
Note that further \textsf{childdoc} directives
such as |\childdocof| and |\childdocforward|
in the indicated file will be processed in this form.
The optional argument \textit{main}
passes on directly to the main file \textit{main}
while pretending to compile the child \textit{dest}.
This form behaves as if \textit{dest}
issues |\childdocof{|\textit{main}|}| right away,
and no further \textsf{childdoc} directives will be processed.

%%%%%%%%%%%%%%%%%%%%%%%%%%%%%%%%%%%%%%%%
\DescribeMacro{\...prefix}
In the alternative form |\childdocforwardprefix|,
%
\begin{center}
\begin{tabular}{l}
|\input{childdoc.def}|\\
|\childdocforwardprefix[|\textit{main}|]{|\textit{prefix}|}{|\textit{dest}|}|
\end{tabular}
\end{center}
%
the destination file is determined by a pattern
depending on the current file:
To make this work, the current file must be called
`{\textit{prefix}\hspace{0.2em}\textit{suffix}}'
with \textit{prefix} matching precisely the argument.
Processing is then passed on to the file
`{\textit{dest}\hspace{0.2em}\textit{suffix}}'.
Surely, the same effect is achieved by
directly specifying the
argument `{\textit{dest}\hspace{0.2em}\textit{suffix}}'
in the first form.
However, that requires to set up a different file
for each child. With the alternative form of the command
all these files can have exactly the same content
which simplifies setting them up and maintaining them.

For example, the following file |draft.tex|
with a compilation flag |\version| as described in \secref{sec:flags}
compiles the main document as a draft:
%
\begin{center}
\begin{tabular}{l}
|\def\version{draft}|\\
|\input{childdoc.def}|\\
|\childdocforward{|\textit{main}|}|
\end{tabular}
\end{center}
%
Likewise, the following files |final|\textit{nn}|.tex|
compile the final version of the child document
|child|\textit{nn}|.tex|:
%
\begin{center}
\begin{tabular}{l}
|\def\version{final}|\\
|\input{childdoc.def}|\\
|\childdocforwardprefix{final}{child}|
\end{tabular}
\end{center}
%

Note that when several versions of a main file and/or of each child file
are to be generated, it may be convenient to set up a |Makefile| or
shell script to automatise the process.

%%%%%%%%%%%%%%%%%%%%%%%%%%%%%%%%%%%%%%%%%%%%%%%%%%%%%%%%%%%%%%%%%%%%%%%%%%%%%%%%
\subsection{Command Line Processing}
\label{sec:commandline}

The effect of redirection files can also be achieved by invoking
the \LaTeX{} compiler with a more elaborate command line.
Most conveniently this should be done as part
of a shell script or a |Makefile|.

When using \textsf{childdoc} in the main file, the following
command lines effectively perform a redirection
(note that depending on the shell being used,
backslashes may have to be doubled: `|\|' $\to$ `|\\|'):
%
\begin{center}
|... -jobname "|\textit{target}|" |\\|"|[\textit{flags}]%
|\input{childdoc.def}\childdocforward[|\textit{main}|]{|\textit{dest}|}"|
\end{center}
%
Here \textit{target} is the name of the output file,
\textit{main} is the name of the main file
and \textit{dest} is the name of the main or child file to be processed
(all filenames without extensions).
The optional argument \textit{main} can be omitted
if \textit{main} matches \textit{dest}.
Optionally, compilation \textit{flags} can be defined via |\def| commands.
This command line makes the \TeX{} engine believe
it is compiling the file \textit{target}
whose content is specified as the latter parameter.
The provided code then forwards the processing to
\textit{main} or \textit{dest} as described in \secref{sec:forward}.

%%%%%%%%%%%%%%%%%%%%%%%%%%%%%%%%%%%%%%%%%%%%%%%%%%%%%%%%%%%%%%%%%%%%%%%%%%%%%%%%
\subsection{Include by Input}
\label{sec:input}

Including child documents by |\include| has some restrictions by design.
Most notably, the content of a child document always occupies
its own set of pages; pages cannot be shared between child documents.
Usually, this behaviour makes perfect sense
because each child document contain an essential part of the document.
However, in some situations it may be desirable to compose
a document from a collection of parts
without having mandatory page breaks between then.
For this case, the package
provides a mechanism to include parts
by |\input| which can also be processed individually.
However, by construction this mechanism
requires manual handling of the content to be output.

%%%%%%%%%%%%%%%%%%%%%%%%%%%%%%%%%%%%%%%%
\DescribeMacro{\ifchilddocmanual}
The main file should be prepared as usual, see \secref{sec:include}.
However, the document body must make a distinction
between processing of an individual part and of the main document, e.g.:
%
\begin{center}
\begin{tabular}{l}
|\ifchilddocmanual|\\
|\input{\childdocname}|\\
|\||else|\\
\textit{document body with }|\input{|\textit{part}|}|\\
|\||fi|
\end{tabular}
\end{center}
%
The conditional |\ifchilddocmanual| is true whenever
a part to be included by |\input| is being compiled,
and the name of the part is stored in |\childdocname|.

%%%%%%%%%%%%%%%%%%%%%%%%%%%%%%%%%%%%%%%%
\DescribeMacro{\childdocby}
Each part to be included by |\input| should start with:
%
\begin{center}
\begin{tabular}{l}
|\input{childdoc.def}|\\
|\childdocby{|\textit{main}|}|\\
\end{tabular}
\end{center}
%
The directive |\childdocby| is similar to |\childdocof|
described in \secref{sec:include},
but the subsequent selection of content must be done manually.
To that end, both |\ifchilddoc| and |\ifchilddocmanual|
will be true upon processing of a part,
and the name of the part is stored in |\childdocname|.
Note that |\jobname| will be set to the filename of the current part
so that each part receives an individual |.aux| file
that does not interfere with the |.aux| file(s) of the main document.
This behaviour can be altered by the alternative form
|\childdocby[*]{|\textit{main}|}| (with a non-empty optional argument)
which uses the |.aux| file of the main document
by setting |\jobname| to \textit{main}.

%%%%%%%%%%%%%%%%%%%%%%%%%%%%%%%%%%%%%%%%%%%%%%%%%%%%%%%%%%%%%%%%%%%%%%%%%%%%%%%%
\subsection{Driver Development}
\label{sec:driver}

The \textsf{childdoc} mechanism can also be use for the development
of definition files such as \LaTeX{} styles or classes.
This case differs from the above setup with multiple parts
included by |\include| in that no |\includeonly| should be invoked.
This can be achieved by starting the include file
(before |\ProvidesPackage|) with:
%
\begin{center}
\begin{tabular}{l}
|\input{childdoc.def}|\\
|\childdocforward{|\textit{main}|}|\\
\end{tabular}
\end{center}
%
or alternatively with:
%
\begin{center}
\begin{tabular}{l}
|\input{childdoc.def}|\\
|\childdocby{|\textit{main}|}|\\
\end{tabular}
\end{center}
%
Both forms have slightly different effects as described above.
The main file is prepared as usual, see \secref{sec:include}.

%%%%%%%%%%%%%%%%%%%%%%%%%%%%%%%%%%%%%%%%%%%%%%%%%%%%%%%%%%%%%%%%%%%%%%%%%%%%%%%%
\subsection{Legacy Detection}
\label{sec:detection}

The directive |\childdocmain| in the main file can detect
whether the complete document or merely a child is to be compiled
even without using the directive |\childdocof|.
This method is deprecated because it is less robust
and there is no compelling reason to use it;
it is merely provided for backward compatibility
and it may be removed in future versions.

If the detection mechanism is to be used,
it is mandatory to correctly specify
the filename of the main file as the argument of |\childdocmain|:
%
\begin{center}
\begin{tabular}{l}
|\input{childdoc.def}|\\
|\childdocmain{|\textit{main}|}|\\
\end{tabular}
\end{center}
%
If |\jobname| does not match the argument \textit{main} of |\childdocmain|,
it is assumed that |\jobname| points to the child file to be compiled.
When using |\childdocmain| with the main file specified as argument,
it suffices to start a child file
with just |\input{|\textit{main}|}|
without loading of the package and using |\childdocof|.
If instead all processing is done
with the appropriate \textsf{childdoc} directives,
the argument of \textit{main} of |\childdocmain| can be empty.

An alternative version of the command line processing described
in \secref{sec:commandline} using the detection mechanism reads:
%
\begin{center}
|... -jobname "|\textit{target}|" "|[\textit{flags}]%
[|\def\jobname{|\textit{dest}|}|]|\input{|\textit{main}|}"|
\end{center}

%%%%%%%%%%%%%%%%%%%%%%%%%%%%%%%%%%%%%%%%%%%%%%%%%%%%%%%%%%%%%%%%%%%%%%%%%%%%%%%%
\subsection{Manual Code}
\label{sec:manual}

In case one cannot be certain whether the definitions file |childdoc.def|
is installed on the target \TeX{} distribution
and one prefers not to ship it,
it is conceivable to paste a few relevant commands into the sources.

To that end, drop all statements |\input{childdoc.def}|
and perform the replacements as outlined below.
Instead of |\childdocmain{|\textit{main}|}| add the following code
to the top of the main file:
%
\begin{center}
\begin{tabular}{l}
|\||ifdefined\childdocname\endinput\||fi\newif\ifchilddoc|\\
|\edef\childdocname{\scantokens\expandafter{\jobname\noexpand}}|\\
|\def\childdocmain{|\textit{main}|}\||ifx\childdocmain\childdocname\||else|\\
|\childdoctrue\includeonly{\childdocname}\let\jobname\childdocmain\||fi|\\
\end{tabular}
\end{center}
%
Instead of |\childdocof{|\textit{main}|}| just include the main file
at the top of each child file:
%
\begin{center}
|\input{|\textit{main}|}|
\end{center}
%
A simple redirection |\childdocforward{|\textit{dest}|}| is achieved by:
%
\begin{center}
|\def\jobname{|\textit{dest}|}\input{\jobname}|
\end{center}
%
The redirection with prefix
|\childdocforwardprefix[|\textit{prefix}|]{|\textit{dest}|}|
is accomplished by:
%
\begin{center}
\begin{tabular}{l}
|{\edef\jobname{\scantokens\expandafter{\jobname\noexpand}}|\\
|\def\redirectjob |\textit{prefix}|#1~~~{\gdef\jobname{|\textit{dest}|#1}}|\\
|\expandafter\redirectjob\jobname~~~}\input{\jobname}|
\end{tabular}
\end{center}

In an alternative approach,
child documents can be compiled by a specific command line
without additional code or specific definitions:
%
\begin{center}
|... -jobname "|\textit{target}|" "|[\textit{flags}]%
|\includeonly{|\textit{dest}|}\input{|\textit{main}|}"|
\end{center}
%

%%%%%%%%%%%%%%%%%%%%%%%%%%%%%%%%%%%%%%%%%%%%%%%%%%%%%%%%%%%%%%%%%%%%%%%%%%%%%%%%
%%%%%%%%%%%%%%%%%%%%%%%%%%%%%%%%%%%%%%%%%%%%%%%%%%%%%%%%%%%%%%%%%%%%%%%%%%%%%%%%
\section{Information}

%%%%%%%%%%%%%%%%%%%%%%%%%%%%%%%%%%%%%%%%%%%%%%%%%%%%%%%%%%%%%%%%%%%%%%%%%%%%%%%%
\subsection{Copyright}

Copyright \copyright{} 2017--2018 Niklas Beisert

This work may be distributed and/or modified under the
conditions of the \LaTeX{} Project Public License, either version 1.3
of this license or (at your option) any later version.
The latest version of this license is in
  \url{http://www.latex-project.org/lppl.txt}
and version 1.3 or later is part of all distributions of \LaTeX{}
version 2005/12/01 or later.

This work has the LPPL maintenance status `maintained'.

The Current Maintainer of this work is Niklas Beisert.

This work consists of the files |README.txt|, |childdoc.ins| and |childdoc.dtx|
as well as the derived files |childdoc.def|, |cdocsamp.tex|
with |cdocsch1.tex|, |cdocsch2.tex|, |cdocspt3.tex|, |cdocspt4.tex|,
|cdocsdrf.tex|, |cdocsfn1.tex|, |cdocsfn2.tex|
as well as |childdoc.pdf|.

%%%%%%%%%%%%%%%%%%%%%%%%%%%%%%%%%%%%%%%%%%%%%%%%%%%%%%%%%%%%%%%%%%%%%%%%%%%%%%%%
\subsection{Files and Installation}

The package consists of the files:
%
\begin{center}
\begin{tabular}{ll}
    |README.txt|   & readme file \\
    |childdoc.ins| & installation file \\
    |childdoc.dtx| & source file \\
    |childdoc.def| & definition file \\
    |cdocsamp.tex| & sample main file \\
    |cdocsch1.tex| & sample include file \\
    |cdocsch2.tex| & sample include file \\
    |cdocspt3.tex| & sample part file \\
    |cdocspt4.tex| & sample part file \\
    |cdocsdrf.tex| & sample redirection file \\
    |cdocsfn1.tex| & sample redirection file \\
    |cdocsfn2.tex| & sample redirection file \\
    |childdoc.pdf| & manual
\end{tabular}
\end{center}
%
The distribution consists of the files
|README.txt|, |childdoc.ins| and |childdoc.dtx|.
%
\begin{itemize}
\item
Run (pdf)\LaTeX{} on |childdoc.dtx|
to compile the manual |childdoc.pdf| (this file).
\item
Run \LaTeX{} on |childdoc.ins| to create the definitions file |childdoc.def|
and the sample |cdocsamp.tex| with include files
|cdocsch1.tex|, |cdocsch2.tex|, |cdocspt3.tex|, |cdocspt4.tex|,
|cdocsdrf.tex|, |cdocsfn1.tex|, |cdocsfn2.tex|.
Then copy the file |childdoc.def| to an appropriate directory of your \LaTeX{}
distribution, e.g.\ \textit{texmf-root}|/tex/latex/childdoc|.
\end{itemize}

%%%%%%%%%%%%%%%%%%%%%%%%%%%%%%%%%%%%%%%%%%%%%%%%%%%%%%%%%%%%%%%%%%%%%%%%%%%%%%%%
\subsection{Related CTAN Packages}

There are several other packages which offer a similar functionality:
%
\begin{itemize}
\item
The packages
\href{http://ctan.org/pkg/docmute}{\textsf{docmute}},
\href{http://ctan.org/pkg/includex}{\textsf{includex}} and
\href{http://ctan.org/pkg/standalone}{\textsf{standalone}}
provide commands to include only the document body of
a child file thus allowing both files to be compiled individually.
\item
The packages \href{http://ctan.org/pkg/subdocs}{\textsf{subdocs}}
and \href{http://ctan.org/pkg/subfiles}{\textsf{subfiles}}
provide structures in which the main and child documents can be
encapsulated and allowing them to be compiled individually.
The inclusion mechanism is different from the conventional |\include|.
\item
The package \href{http://ctan.org/pkg/combine}{\textsf{combine}}
is an elaborate solution to combine several documents into one.
\end{itemize}
%
See also the CTAN topic \href{http://ctan.org/topic/subdocs}{\textsf{subdocs}}
for further related packages.
The present package differs from the above solutions in that
a document structure constructed with the conventional |\include| mechanism
just needs two extra commands at the top of every file
such that all constituent files can be compiled individually.

%%%%%%%%%%%%%%%%%%%%%%%%%%%%%%%%%%%%%%%%%%%%%%%%%%%%%%%%%%%%%%%%%%%%%%%%%%%%%%%%
%\subsection{Feature Suggestions}
%
%The following is a list of features which may be useful for future
%versions of this package:
%%
%\begin{itemize}
%\item
%\ldots
%\end{itemize}

%%%%%%%%%%%%%%%%%%%%%%%%%%%%%%%%%%%%%%%%%%%%%%%%%%%%%%%%%%%%%%%%%%%%%%%%%%%%%%%%
\subsection{Revision History}

%%%%%%%%%%%%%%%%%%%%%%%%%%%%%%%%%%%%%%%%
\paragraph{v2.0:} 2018/12/30

\begin{itemize}
\item
immediate forward processing
\item
added |\childdocby| mechanism
\item
manual restructured
\end{itemize}

%%%%%%%%%%%%%%%%%%%%%%%%%%%%%%%%%%%%%%%%
\paragraph{v1.6:} 2018/01/17

\begin{itemize}
\item
application for development of include files
\item
corrections to manual
\end{itemize}

%%%%%%%%%%%%%%%%%%%%%%%%%%%%%%%%%%%%%%%%
\paragraph{v1.5:} 2017/05/21

\begin{itemize}
\item
more complete structuring introduced
\item
|\childdocof| introduced
\item
|\childdoc| renamed to |\childdocmain|
\item
|\childredirect| renamed to |\childdocforward| and |\childdocforwardprefix|
and functionality expanded
\end{itemize}

%%%%%%%%%%%%%%%%%%%%%%%%%%%%%%%%%%%%%%%%
\paragraph{v1.0:} 2017/04/27

\begin{itemize}
\item
manual and install package
\item
first version published on CTAN
\end{itemize}

%%%%%%%%%%%%%%%%%%%%%%%%%%%%%%%%%%%%%%%%
\paragraph{v0.6:} 2017/04/26

\begin{itemize}
\item
redirection mechanism added
\end{itemize}

%%%%%%%%%%%%%%%%%%%%%%%%%%%%%%%%%%%%%%%%
\paragraph{v0.5:} 2017/04/26

\begin{itemize}
\item
functionality in definition file
\end{itemize}


%%%%%%%%%%%%%%%%%%%%%%%%%%%%%%%%%%%%%%%%%%%%%%%%%%%%%%%%%%%%%%%%%%%%%%%%%%%%%%%%
%%%%%%%%%%%%%%%%%%%%%%%%%%%%%%%%%%%%%%%%%%%%%%%%%%%%%%%%%%%%%%%%%%%%%%%%%%%%%%%%
%%%%%%%%%%%%%%%%%%%%%%%%%%%%%%%%%%%%%%%%%%%%%%%%%%%%%%%%%%%%%%%%%%%%%%%%%%%%%%%%
\appendix

\settowidth\MacroIndent{\rmfamily\scriptsize 000\ }

 \DocInput{childdoc.dtx}

\end{document}
%</driver>
% \fi
%
% %%%%%%%%%%%%%%%%%%%%%%%%%%%%%%%%%%%%%%%%%%%%%%%%%%%%%%%%%%%%%%%%%%%%%%%%%%%%%%
% %%%%%%%%%%%%%%%%%%%%%%%%%%%%%%%%%%%%%%%%%%%%%%%%%%%%%%%%%%%%%%%%%%%%%%%%%%%%%%
% \section{Sample}
%\iffalse
%<*samplemain>
%\fi
%
% The following presents a sample document
% with two chapters, two parts, a title page,
% a compile flag as well as three forwarding files to set the flag.
% It consists of eight |.tex| files:
% \begin{center}
% \begin{tabular}{ll}
% |cdocsamp.tex|&main file\\
% |cdocsch1.tex|&include file for chapter 1\\
% |cdocsch2.tex|&include file for chapter 2\\
% |cdocspt3.tex|&include file for part 3\\
% |cdocspt4.tex|&include file for part 4\\
% |cdocsdrf.tex|&forwarding file for main file in draft mode\\
% |cdocsfi1.tex|&forwarding file for final version of chapter 1\\
% |cdocsfi2.tex|&forwarding file for final version of chapter 2\\
% \end{tabular}
% \end{center}
% Each of the eight files can be compiled directly by the \LaTeX{} compiler.
%
% %%%%%%%%%%%%%%%%%%%%%%%%%%%%%%%%%%%%%%
% \paragraph{Main File.}
%
% The main file is called |cdocsamp.tex|.
%
% Load the \textsf{childdoc} definitions and
% declare the filename for the main document:
%    \begin{macrocode}
\input{childdoc.def}
\childdocmain{}
%    \end{macrocode}

% Optional override for |\version| flag:
%    \begin{macrocode}
%%\ifchilddoc\else\providecommand{\version}{draft}\fi
%    \end{macrocode}

% Define the default values for the |\version| flag
% (|final| for the main file and |draft| for childs):
%    \begin{macrocode}
\ifchilddoc
\providecommand{\version}{draft}
\else
\providecommand{\version}{final}
\fi
%    \end{macrocode}

% Load the standard document class:
%    \begin{macrocode}
\documentclass[12pt]{article}
%    \end{macrocode}

% Start the document body:
%    \begin{macrocode}
\begin{document}
%    \end{macrocode}

% Declare a title page.
% Print title, part of document being processed and version flag:
%    \begin{macrocode}
\addtocounter{page}{-1}
\begin{center}
{\LARGE\bfseries{}childdoc example\par}
\vspace{1cm}
\ifchilddoc
\ifchilddocmanual part\else chapter\fi:
`\childdocname' of `\childdocjob'\par
\else
main document: `\childdocjob'\par
\fi
version: \version\par
\end{center}
\newpage
%    \end{macrocode}

% Manually include selected file,
% otherwise process as usual:
%    \begin{macrocode}
\ifchilddocmanual
\section*{part `\childdocname'}
\input{\childdocname}
\else
%    \end{macrocode}

% Include the two chapters:
%    \begin{macrocode}
\include{cdocsch1}
\include{cdocsch2}
%    \end{macrocode}

% Include the two parts unless only chapters should be displayed:
%    \begin{macrocode}
\ifchilddoc\else
\section{part three}
\input{cdocspt3}
\section{part four}
\input{cdocspt4}
\fi
%    \end{macrocode}

% Process as usual until here:
%    \begin{macrocode}
\fi
%    \end{macrocode}

% End of document body:
%    \begin{macrocode}
\end{document}
%    \end{macrocode}
%\iffalse
%</samplemain>
%\fi
%
% %%%%%%%%%%%%%%%%%%%%%%%%%%%%%%%%%%%%%%
% \paragraph{Chapter Include Files.}
%
% The include files are called |cdocsch1.tex| and |cdocsch2.tex|.
%
%\iffalse
%<*samplechap1|samplechap2>
%\fi

% Optional override for |\version| flag:
%    \begin{macrocode}
%%\providecommand{\version}{final}
%    \end{macrocode}

% Include the main document:
%    \begin{macrocode}
\input{childdoc.def}
\childdocof{cdocsamp}
%    \end{macrocode}

%\iffalse
%</samplechap1|samplechap2>
%\fi
%
%\iffalse
%<*samplechap1>
%\fi
% Some text for chapter 1:
%    \begin{macrocode}
\section{one}
some text in chapter one
%    \end{macrocode}

%\iffalse
%</samplechap1>
%\fi
% Some text for chapter 2:
%\iffalse
%<*samplechap2>
%\fi
%    \begin{macrocode}
\section{two}
more text in chapter two
%    \end{macrocode}

%\iffalse
%</samplechap2>
%\fi
%
% %%%%%%%%%%%%%%%%%%%%%%%%%%%%%%%%%%%%%%
% \paragraph{Part Include Files.}
%
% The include files are called |cdocspt3.tex| and |cdocspt4.tex|.
%
%\iffalse
%<*samplepart3|samplepart4>
%\fi

% Optional override for |\version| flag:
%    \begin{macrocode}
%%\providecommand{\version}{final}
%    \end{macrocode}

% Include the main document:
%    \begin{macrocode}
\input{childdoc.def}
\childdocby{cdocsamp}
%    \end{macrocode}

%\iffalse
%</samplepart3|samplepart4>
%\fi
%
%\iffalse
%<*samplepart3>
%\fi
% Some text for part 3:
%    \begin{macrocode}
some text in part three
%    \end{macrocode}

%\iffalse
%</samplepart3>
%\fi
% Some text for part 4:
%\iffalse
%<*samplepart4>
%\fi
%    \begin{macrocode}
more text in part four
%    \end{macrocode}

%\iffalse
%</samplepart4>
%\fi
%
% %%%%%%%%%%%%%%%%%%%%%%%%%%%%%%%%%%%%%%
% \paragraph{Forwarding for a Complete Draft.}
%
% The following forwarding file |cdocsdrf.tex|
% compiles the main document in draft mode:
%\iffalse
%<*sampledraft>
%\fi
%    \begin{macrocode}
\def\version{draft}
\input{childdoc.def}
\childdocforward{cdocsamp}
%    \end{macrocode}

%\iffalse
%</sampledraft>
%\fi
%
% %%%%%%%%%%%%%%%%%%%%%%%%%%%%%%%%%%%%%%
% \paragraph{Forwarding for Final Version of the Chapters.}
%
% The following forwarding files |cdocsfn1.tex| and |cdocsfn2.tex|
% (with identical content)
% compile the final versions of the child documents
% |cdocsch1.tex| and |cdocsch2.tex|, respectively:
%\iffalse
%<*samplefinal>
%\fi
%    \begin{macrocode}
\def\version{final}
\input{childdoc.def}
\childdocforwardprefix[cdocsamp]{cdocsfn}{cdocsch}
%    \end{macrocode}

%\iffalse
%</samplefinal>
%\fi
%
% %%%%%%%%%%%%%%%%%%%%%%%%%%%%%%%%%%%%%%
% \paragraph{Command Line Processing.}
%
% The following three command lines generate the output files
% |cdocscld|, |cdocscl1| and |cdocscl2|
% which should be identical to
% |cdocsdrf|, |cdocsch1| and |cdocsfn2|, respectively:
% \begin{center}
% \begin{tabular}{l}
% |latex -jobname cdocscld \|\\
% |  "\def\version{draft}\input{childdoc.def}\childdocforward{cdocsamp}"|\\
% |latex -jobname cdocscl1 \|\\
% |  "\input{childdoc.def}\childdocforward[cdocsamp]{cdocsch1}"|\\
% |latex -jobname cdocscl2 \|\\
% |  "\def\version{final}\input{childdoc.def}\childdocforward{cdocsch2}"|
% \end{tabular}
% \end{center}
% Note that the trailing backslash on each first line
% merely continues the input to the second line
% (for convenient cut ant paste).
% Furthermore, the command |latex| can be replaced by any
% of its alternative versions such as |pdflatex|.
%
% %%%%%%%%%%%%%%%%%%%%%%%%%%%%%%%%%%%%%%%%%%%%%%%%%%%%%%%%%%%%%%%%%%%%%%%%%%%%%%
% %%%%%%%%%%%%%%%%%%%%%%%%%%%%%%%%%%%%%%%%%%%%%%%%%%%%%%%%%%%%%%%%%%%%%%%%%%%%%%
% \section{Implementation}
%\iffalse
%<*package>
%\fi
%
% This section describes the definitions file |childdoc.def|.

% The definitions cannot be loaded using |\usepackage| or |\RequirePackage|
% which has a mechanism to prevent loading a style file more than once.
% When loading the definitions by means of |\input|
% multiple instances have to be prevented manually:
%\iffalse
%This code needs to be before the `\ProvidesFile' directive
%which is defined at the beginning of this file.
%Therefore it is also placed there and commented out here.
%</package>
%<*discard>
%\fi
%    \begin{macrocode}
\ifdefined\childdocmain\endinput\fi
%    \end{macrocode}
%\iffalse
%</discard>
%<*package>
%\fi
%
% \macro{\ifchilddoc}
% \macro{\ifchilddocmanual}
% The conditional |\ifchilddoc| tells whether a
% child (true) or main (false) document is being compiled.
% The conditional |\ifchilddocmanual| tells whether
% the |\includeonly| mechanism is used (false) or
% the selection of child files must be performed manually (true).
% The definitions initialise to false:
%    \begin{macrocode}
\newif\ifchilddoc
\newif\ifchilddocmanual
%    \end{macrocode}

% \macro{\childdocname}
% \macro{\childdocjob}
% The macro |\childdocname| stores the name of the main document
% to be compiled. The macro |\childdocjob| stores the name of
% the document on which the \LaTeX{} compiler was originally invoked.
% The content of |\jobname| cannot be compared
% to filenames specified in the source due to different catcodes.
% The following code rescans |\jobname|, stores the result
% in |\childdocname| and saves a copy in |\childdocjob|:
%    \begin{macrocode}
\edef\childdocname{\scantokens\expandafter{\jobname\noexpand}}
\let\childdocjob\childdocname
%    \end{macrocode}

% \macro{\childdocdisable}
% The macro |\childdocdisable| prevents the main file
% from being processed more than once.
% At this stage, the main document command |\childdocmain|
% is assumed to be called once again where it should do nothing.
% Any subsequent call to it should prevent
% a secondary processing of the main document
% It overwrites the forwarding commands
% |\childdocof| and |\childdocforward|
% with empty macros to prevent further inclusions of the main document:
%    \begin{macrocode}
\newcommand{\childdocdisable}
{
  \renewcommand{\childdocmain}[1]{\renewcommand{\childdocmain}[1]{\endinput}}
  \renewcommand{\childdocof}[1]{}
  \renewcommand{\childdocby}[2][]{}
  \renewcommand{\childdocforward}[2][]{}
  \renewcommand{\childdocdisable}{}
}
%    \end{macrocode}

% \macro{\childdocmain}
% The macro |\childdocmain| is to be called at the top of the main file
% with nothing or the main filename (without extension) as argument.
% First, it breaks loops.
% If the argument is not empty and does not match |\childdocname|
% (which is set by the first inclusion of |childdoc.def|),
% |\ifchilddoc| is set to true, |\includeonly| is applied to the child file
% and |\jobname| is set to the main file
% (for proper handling of |.aux| files):
%    \begin{macrocode}
\newcommand{\childdocmain}[1]
{
  \childdocdisable\childdocmain{}
  \if?#1?\else
    \begingroup
      \def\childdoctmp{#1}
      \ifx\childdoctmp\childdocname
        \def\childdoctmp{}
      \else
        \def\childdoctmp
        {
          \childdoctrue
          \includeonly{\childdocname}
          \def\childdocjob{#1}
          \def\jobname{#1}
        }
      \fi
      \expandafter
    \endgroup
    \childdoctmp
  \fi
}
%    \end{macrocode}

% \macro{\childdocof}
% The command |\childdocof| redirects
% compilation to the main file |#1|.
%    \begin{macrocode}
\newcommand{\childdocof}[1]
{
  \childdocdisable
  \childdoctrue
  \includeonly{\childdocname}
  \def\jobname{#1}
  \def\childdocjob{#1}
  \input{#1}
}
%    \end{macrocode}

% \macro{\childdocby}
% The command |\childdocby| ....
%    \begin{macrocode}
\newcommand{\childdocby}[2][]
{
  \childdocdisable
  \childdoctrue
  \childdocmanualtrue
  \if?#1?\else
    \def\jobname{#2}
  \fi
  \def\childdocjob{#2}
  \input{#2}
  \endinput
}
%    \end{macrocode}

% \macro{\childdocforward}
% The command |\childdocforward| redirects
% compilation to the main file or
% (if the optional argument is given) a child file.
% Parameters are set as if the main file
% or a child file starting with |\childdocof| was compiled.
% Then compilation is handed over to the main file:
%    \begin{macrocode}
\newcommand{\childdocforward}[2][]
{
  \begingroup
    \if?#1?
      \def\childdoctmp
      {
        \def\childdocname{#2}
        \def\childdocjob{#2}
        \def\jobname{#2}
        \input{#2}
        \endinput
      }
    \else
      \def\childdoctmp
      {
        \childdocdisable
        \def\childdocname{#2}
        \childdoctrue
        \includeonly{#2}
        \def\childdocjob{#1}
        \def\jobname{#1}
        \input{#1}
        \endinput
      }
    \fi
    \expandafter
  \endgroup
  \childdoctmp
}
%    \end{macrocode}

% \macro{\childdocforwardprefix}
% The command |\childdocforwardprefix| redirects
% compilation to the main or a child file by means of a pattern.
% The prefix |#1| in the current filename is replaced by |#2|
% and the suffix of the current filename is kept
% (it is assumed that the filename does not contain the substring `|~~~|'
% which is used as a delimiter).
% Compilation is handed over to the new file by |\childdocforward|:
%    \begin{macrocode}
\newcommand{\childdocforwardprefix}[3][]
{
  \begingroup
    \def\childdocextract #2##1~~~{\def\childdoctmp{\childdocforward[#1]{#3##1}}}
    \expandafter\childdocextract\childdocname~~~
    \expandafter
  \endgroup
  \childdoctmp
}
%    \end{macrocode}

% \macro{\childdoc}
% The deprecated macro |\childdoc| is a legacy version of |\childdocmain|:
%    \begin{macrocode}
\newcommand{\childdoc}{\childdocmain}
%    \end{macrocode}

% \macro{\childdocredirect}
% The deprecated macro |\childdocredirect| is a legacy version
% of |\childdocforward| and |\childdocforwardprefix|:
%    \begin{macrocode}
\newcommand{\childdocredirect}[2][]
{
  \begingroup
    \if?#1?
      \def\childdoctmp{\childdocforward{#2}}
    \else
      \def\childdoctmp{\childdocforwardprefix{#1}{#2}}
    \fi
    \expandafter
  \endgroup
  \childdoctmp
}
%    \end{macrocode}

%\iffalse
%</package>
%\fi
%
\endinput
|\\
|\childdocmain{}|\\
\end{tabular}
\end{center}
at the very top of the main \LaTeX{} file,
in particular \emph{before} the |\documentclass| statement!
The argument of |\childdocmain| should be left empty
(but it must be present).

%%%%%%%%%%%%%%%%%%%%%%%%%%%%%%%%%%%%%%%%
\DescribeMacro{\childdocof}
Furthermore, add the commands
\begin{center}
\begin{tabular}{l}
|% \iffalse
%
% childdoc.dtx Copyright (C) 2017-2018 Niklas Beisert
%
% This work may be distributed and/or modified under the
% conditions of the LaTeX Project Public License, either version 1.3
% of this license or (at your option) any later version.
% The latest version of this license is in
%   http://www.latex-project.org/lppl.txt
% and version 1.3 or later is part of all distributions of LaTeX
% version 2005/12/01 or later.
%
% This work has the LPPL maintenance status `maintained'.
%
% The Current Maintainer of this work is Niklas Beisert.
%
% This work consists of the files childdoc.dtx and childdoc.ins
% and the derived files childdoc.def and cdocsamp.tex with
% cdocsch1.tex, cdocsch2.tex, cdocsdrf.tex, cdocsfn1.tex, cdocsfn2.tex.
%
%<package>\ifdefined\childdocmain\endinput\fi
%<package>\ProvidesFile{childdoc.def}[2018/12/30 v2.0 child document driver]
%<samplemain>\ProvidesFile{cdocsamp.tex}[2018/12/30 v2.0 sample for childdoc]
%<*driver>
%\ProvidesFile{childdoc.drv}[2018/12/30 v2.0 childdoc reference manual file]
\PassOptionsToClass{10pt,a4paper}{article}
\documentclass{ltxdoc}

\usepackage[margin=35mm]{geometry}
\usepackage{hyperref}
\usepackage{hyperxmp}
\usepackage[usenames]{color}

\hypersetup{colorlinks=true}
\hypersetup{pdfstartview=FitH}
\hypersetup{pdfpagemode=UseNone}
\hypersetup{pdfsource={}}
\hypersetup{pdflang={en-UK}}
\hypersetup{pdfcopyright={Copyright 2017-2018 Niklas Beisert.
  This work may be distributed and/or modified under the
  conditions of the LaTeX Project Public License, either version 1.3
  of this license or (at your option) any later version.}}
\hypersetup{pdflicenseurl={http://www.latex-project.org/lppl.txt}}
\hypersetup{pdfcontactaddress={ETH Zurich, ITP, HIT K,
  Wolfgang-Pauli-Strasse 27}}
\hypersetup{pdfcontactpostcode={8093}}
\hypersetup{pdfcontactcity={Zurich}}
\hypersetup{pdfcontactcountry={Switzerland}}
\hypersetup{pdfcontactemail={nbeisert@itp.phys.ethz.ch}}
\hypersetup{pdfcontacturl={http://people.phys.ethz.ch/\xmptilde nbeisert/}}

\newcommand{\secref}[1]{\hyperref[#1]{section \ref*{#1}}}

\parskip1ex
\parindent0pt
\let\olditemize\itemize
\def\itemize{\olditemize\parskip0pt}

\begin{document}

\title{The \textsf{childdoc} Package}
\hypersetup{pdftitle={The childdoc Package}}
\author{Niklas Beisert\\[2ex]
  Institut f\"ur Theoretische Physik\\
  Eidgen\"ossische Technische Hochschule Z\"urich\\
  Wolfgang-Pauli-Strasse 27, 8093 Z\"urich, Switzerland\\[1ex]
  \href{mailto:nbeisert@itp.phys.ethz.ch}
  {\texttt{nbeisert@itp.phys.ethz.ch}}}
\hypersetup{pdfauthor={Niklas Beisert}}
\hypersetup{pdfsubject={Manual for the LaTeX2e Package childdoc}}
\date{30 December 2018, \textsf{v2.0}}
\maketitle

\begin{abstract}\noindent
\textsf{childdoc} is a \LaTeXe{} package
that enables the direct compilation
of document sections included by |\include|
to individual files.
\end{abstract}

\begingroup
\parskip0ex
\tableofcontents
\endgroup

%%%%%%%%%%%%%%%%%%%%%%%%%%%%%%%%%%%%%%%%%%%%%%%%%%%%%%%%%%%%%%%%%%%%%%%%%%%%%%%%
%%%%%%%%%%%%%%%%%%%%%%%%%%%%%%%%%%%%%%%%%%%%%%%%%%%%%%%%%%%%%%%%%%%%%%%%%%%%%%%%
\section{Introduction}

\LaTeX{} provides a mechanism to structure a large document (such as a book)
into a main file and several child files (containing the chapters)
using the |\include| command.
This mechanism is beneficial for documents
which span hundreds of pages in order to
make the source file(s) more manageable.
Moreover, compilation can be restricted to
selected child files by means of the |\includeonly| command.
The latter feature can be used to reduce the compilation time while editing
(this was significantly more useful in the earlier days of \LaTeX{})
or to generate a smaller document which is easier to navigate.
Another application of |\includeonly| is to generate
documents consisting of selected parts of the complete document.

However, there are a few drawbacks of the plain |\include| mechanism:
\begin{itemize}
\item
The child files cannot be compiled on their own,
they can only be compiled via the main file.
A naive editing environment
(such as a text editor with an option
to have the current file processed by \LaTeX)
may require one to switch to the main file before compiling;
attempting to compile the child file produces errors.
\item
The main file must be modified (each time)
to adjust the |\includeonly| command
to the present needs. This easily leaves the main file in a messy state.
\item
The generated document will always carry the filename
of the main document. This is inconvenient if
several child files are to be compiled and
to be kept for distribution.
\end{itemize}

The present package provides a simple interface
to make child files individually compilable by \LaTeX{}.
Compiling a child file then has the same effect as compiling
the main file with an |\includeonly| command
to select the appropriate child.
Moreover the generated document will carry the name of the child
rather than the main file.
This resolves all three above issues.

This feature is meant to make the editing of books,
thesis documents and lecture notes somewhat more convenient.
However, the package can also be used efficiently for
composing a series of documents (such as exercise sheets)
which are typically distributed individually.
It then assists the author in generating the individual documents
(potentially in different versions)
as well as a document containing the collected series.
Another application is in developing style files
or other kinds of included material
where compilation of the style file could redirect
to a sample or test file.

%%%%%%%%%%%%%%%%%%%%%%%%%%%%%%%%%%%%%%%%%%%%%%%%%%%%%%%%%%%%%%%%%%%%%%%%%%%%%%%%
%%%%%%%%%%%%%%%%%%%%%%%%%%%%%%%%%%%%%%%%%%%%%%%%%%%%%%%%%%%%%%%%%%%%%%%%%%%%%%%%
\section{Usage}

First of all, the package \textsf{childdoc} is \emph{not} a standard
\LaTeXe{} |.sty| style file! Therefore it needs to be invoked in
a non-standard way.

%%%%%%%%%%%%%%%%%%%%%%%%%%%%%%%%%%%%%%%%%%%%%%%%%%%%%%%%%%%%%%%%%%%%%%%%%%%%%%%%
\subsection{Included Files}
\label{sec:include}

%%%%%%%%%%%%%%%%%%%%%%%%%%%%%%%%%%%%%%%%
\DescribeMacro{\childdocmain}
To use the package, add the commands
\begin{center}
\begin{tabular}{l}
|\input{childdoc.def}|\\
|\childdocmain{}|\\
\end{tabular}
\end{center}
at the very top of the main \LaTeX{} file,
in particular \emph{before} the |\documentclass| statement!
The argument of |\childdocmain| should be left empty
(but it must be present).

%%%%%%%%%%%%%%%%%%%%%%%%%%%%%%%%%%%%%%%%
\DescribeMacro{\childdocof}
Furthermore, add the commands
\begin{center}
\begin{tabular}{l}
|\input{childdoc.def}|\\
|\childdocof{|\textit{main}|}|\\
\end{tabular}
\end{center}
at the top of every child file \textit{child}
which is included by |\include{|\textit{child}|}|
from within the main file
(or at least for those files to be compiled individually).
The argument \textit{main} must be the filename of the main file.

There are a couple of
considerations in setting up the main and child documents:

%%%%%%%%%%%%%%%%%%%%%%%%%%%%%%%%%%%%%%%%
\paragraph{Restrictions.}

Please note the following restrictions:
\begin{itemize}
\item
|\childdocmain| must be called with one argument \textit{main}
to ensure compatibility with earlier version of the package.
It must either be empty (|\childdocmain{}|)
or precisely match the filename of the main file in which it is specified.
See \secref{sec:detection} for further information.
\item
The filename \textit{main} must be specified without the |.tex| extension.
\item
The filename \textit{main} is case sensitive
(even in case-insensitive file systems)
due to internal string comparison.
\item
The argument \textit{main} should be fully expanded, it cannot be a macro.
\item
Subdirectories and special characters should be avoided in filenames.
\item
The command |\childdocmain{|\textit{main}|}| must be followed by a whitespace.
It should not be followed immediately by another command
or by a comment mark `|%|'.
This is because the \TeX{} parser reads the token immediately following
the argument of |\childdocmain| and puts it
at the beginning of every child section;
however, a white\-space is ignored.
\end{itemize}

%%%%%%%%%%%%%%%%%%%%%%%%%%%%%%%%%%%%%%%%
\paragraph{Content of Main File.}

It is advisable to place all content in the child files included by |\include|.
Any output contained in the main file will appear in all child documents
unless suppressed manually;
it cannot be suppressed automatically by the |\includeonly| directive
and thus should normally be avoided.
A method to include some content in the main file
by means of conditional processing is described in \secref{sec:conditional}.

%%%%%%%%%%%%%%%%%%%%%%%%%%%%%%%%%%%%%%%%
\paragraph{Page Numbering.}

When only a part of the document is compiled,
the appropriate numbering of pages
(as well as other status parameters)
is determined from the |.aux| files.
The latter contain information from previous passes.
However this information needs to propagate through
all intermediate child documents.
Therefore the page numbering in child documents may well
be inconsistent until the complete document is compiled at least once.

A useful (if unconventional) way to always ensure a consistent
page numbering is to restart the numbering in each child document
and denote the pages by `\textit{child}|.|\textit{page}'
where \textit{child} represents the chapter/section number of the child file.
This can be achieved by the command
|\numberwithin{page}{|\textit{child}|}|
of the \textsf{amsmath} package
where \textit{child} can be |chapter| or |section|
depending on the chosen structuring.
Alternatively, one can modify the macro |\thepage| appropriately
and reset the counter |page| at the start of each child file.

%%%%%%%%%%%%%%%%%%%%%%%%%%%%%%%%%%%%%%%%%%%%%%%%%%%%%%%%%%%%%%%%%%%%%%%%%%%%%%%%
\subsection{Conditional Processing}
\label{sec:conditional}

The package provides a mechanism to compile different versions
of a document. To customise the versions further some conditional processing
can come in handy to distinguish which version is being compiled.
The package provides two macros to describe the compilation context:

%%%%%%%%%%%%%%%%%%%%%%%%%%%%%%%%%%%%%%%%
\DescribeMacro{\ifchilddoc}
The conditional |\ifchilddoc| distinguishes between the compilation of
child documents and the main document:
%
\begin{center}
|\ifchilddoc |\textit{child-code}| |[|\||else |\textit{main-code}]| \||fi|
\end{center}

%%%%%%%%%%%%%%%%%%%%%%%%%%%%%%%%%%%%%%%%
\DescribeMacro{\childdocname}
\DescribeMacro{\childdocjob}
The macro |\childdocname| contains the filename (without extension)
of the main or child file being processed.
Note that |\childdocjob| will always contain the name of the main file.

%%%%%%%%%%%%%%%%%%%%%%%%%%%%%%%%%%%%%%%%
\paragraph{Title Page.}

Conditional processing can be used to include a title or banner page
in the main document when proper precautions are taken.
Importantly, the code in the main file should ensure that the page counter
(as well as other status parameters which are stored in the |.aux| files)
takes the same value after the conditional processing.
Otherwise the page numbers may take divergent values
depending on which part is compiled.

For example, a title page could be declared by:
%
\begin{center}
\begin{tabular}{l}
|\ifchilddoc\||else|\\
|\addtocounter{page}{-1}|\\
\textit{code for title page}\\
|\newpage|\\
|\||fi|
\end{tabular}
\end{center}
%
A banner page for the child documents can be generated by:
%
\begin{center}
\begin{tabular}{l}
|\ifchilddoc|\\
|\addtocounter{page}{-1}|\\
\textit{code for banner page}\\
|\newpage|\\
|\||fi|
\end{tabular}
\end{center}
%
Here one could write a message such as:
\begin{center}
|This is the part \childdocname{} of \childdocjob{}.|
\end{center}

%%%%%%%%%%%%%%%%%%%%%%%%%%%%%%%%%%%%%%%%%%%%%%%%%%%%%%%%%%%%%%%%%%%%%%%%%%%%%%%%
\subsection{Flags}
\label{sec:flags}

The package makes it easy to generate different versions
of the main or child documents.
To this end compilation flags can be defined
and assigned different default values.
They will be particularly useful in conjunction
with the forwarding mechanism described in \secref{sec:forward}.

For example, it may be useful to have a flag |\version|
which can be set to |draft| or |final|.
The document source will contain some conditional code
depending on the value of |\version|.
Suppose further, the flag should default to |final| for the main file
and to |draft| for child files
which is a natural assignment for editing the document.
This is achieved by placing the following code
in the preamble of the main document
(below the |\childdocmain| directive):
%
\begin{center}
\begin{tabular}{l}
|\ifchilddoc|\\
|\providecommand{\version}{draft}|\\
|\||else|\\
|\providecommand{\version}{final}|\\
|\||fi|
\end{tabular}
\end{center}
%
The definition by |\providecommand| makes sure
that previous definitions are not overwritten.
Further statements |\providecommand{\version}{...}|
can thus be added before the above code to override it.

For the main file, one might add a line
(between |\childdocmain| and the above block)
%
\begin{center}
|%\ifchilddoc\||else\providecommand{\version}{draft}\||fi|
\end{center}
%
which can be uncommented to produce a draft version.
Likewise one can add a line to the very top of a child file
(above the |\childdocof{|\textit{main}|}| directive)
%
\begin{center}
|%\providecommand{\version}{final}|
\end{center}
%
which can be uncommented to produce the final version of this child document.

%%%%%%%%%%%%%%%%%%%%%%%%%%%%%%%%%%%%%%%%%%%%%%%%%%%%%%%%%%%%%%%%%%%%%%%%%%%%%%%%
\subsection{Forwarding}
\label{sec:forward}

Different versions of the main or child documents
using compilation flags as described in \secref{sec:flags}
can be (permanently) stored in different files
for convenient compilation, viewing and distribution.
To this end, the package defines a command
to pass on compilation to a different file:

%%%%%%%%%%%%%%%%%%%%%%%%%%%%%%%%%%%%%%%%
\DescribeMacro{\childdocforward}
The command |\childdocforward| redirects processing to
another source file:
%
\begin{center}
\begin{tabular}{l}
|\input{childdoc.def}|\\
|\childdocforward[|\textit{main}|]{|\textit{dest}|}|\\
\end{tabular}
\end{center}
%
The argument \textit{dest} is the destination file
(without extension).
It should be the main file or one of the child files.
Note that further \textsf{childdoc} directives
such as |\childdocof| and |\childdocforward|
in the indicated file will be processed in this form.
The optional argument \textit{main}
passes on directly to the main file \textit{main}
while pretending to compile the child \textit{dest}.
This form behaves as if \textit{dest}
issues |\childdocof{|\textit{main}|}| right away,
and no further \textsf{childdoc} directives will be processed.

%%%%%%%%%%%%%%%%%%%%%%%%%%%%%%%%%%%%%%%%
\DescribeMacro{\...prefix}
In the alternative form |\childdocforwardprefix|,
%
\begin{center}
\begin{tabular}{l}
|\input{childdoc.def}|\\
|\childdocforwardprefix[|\textit{main}|]{|\textit{prefix}|}{|\textit{dest}|}|
\end{tabular}
\end{center}
%
the destination file is determined by a pattern
depending on the current file:
To make this work, the current file must be called
`{\textit{prefix}\hspace{0.2em}\textit{suffix}}'
with \textit{prefix} matching precisely the argument.
Processing is then passed on to the file
`{\textit{dest}\hspace{0.2em}\textit{suffix}}'.
Surely, the same effect is achieved by
directly specifying the
argument `{\textit{dest}\hspace{0.2em}\textit{suffix}}'
in the first form.
However, that requires to set up a different file
for each child. With the alternative form of the command
all these files can have exactly the same content
which simplifies setting them up and maintaining them.

For example, the following file |draft.tex|
with a compilation flag |\version| as described in \secref{sec:flags}
compiles the main document as a draft:
%
\begin{center}
\begin{tabular}{l}
|\def\version{draft}|\\
|\input{childdoc.def}|\\
|\childdocforward{|\textit{main}|}|
\end{tabular}
\end{center}
%
Likewise, the following files |final|\textit{nn}|.tex|
compile the final version of the child document
|child|\textit{nn}|.tex|:
%
\begin{center}
\begin{tabular}{l}
|\def\version{final}|\\
|\input{childdoc.def}|\\
|\childdocforwardprefix{final}{child}|
\end{tabular}
\end{center}
%

Note that when several versions of a main file and/or of each child file
are to be generated, it may be convenient to set up a |Makefile| or
shell script to automatise the process.

%%%%%%%%%%%%%%%%%%%%%%%%%%%%%%%%%%%%%%%%%%%%%%%%%%%%%%%%%%%%%%%%%%%%%%%%%%%%%%%%
\subsection{Command Line Processing}
\label{sec:commandline}

The effect of redirection files can also be achieved by invoking
the \LaTeX{} compiler with a more elaborate command line.
Most conveniently this should be done as part
of a shell script or a |Makefile|.

When using \textsf{childdoc} in the main file, the following
command lines effectively perform a redirection
(note that depending on the shell being used,
backslashes may have to be doubled: `|\|' $\to$ `|\\|'):
%
\begin{center}
|... -jobname "|\textit{target}|" |\\|"|[\textit{flags}]%
|\input{childdoc.def}\childdocforward[|\textit{main}|]{|\textit{dest}|}"|
\end{center}
%
Here \textit{target} is the name of the output file,
\textit{main} is the name of the main file
and \textit{dest} is the name of the main or child file to be processed
(all filenames without extensions).
The optional argument \textit{main} can be omitted
if \textit{main} matches \textit{dest}.
Optionally, compilation \textit{flags} can be defined via |\def| commands.
This command line makes the \TeX{} engine believe
it is compiling the file \textit{target}
whose content is specified as the latter parameter.
The provided code then forwards the processing to
\textit{main} or \textit{dest} as described in \secref{sec:forward}.

%%%%%%%%%%%%%%%%%%%%%%%%%%%%%%%%%%%%%%%%%%%%%%%%%%%%%%%%%%%%%%%%%%%%%%%%%%%%%%%%
\subsection{Include by Input}
\label{sec:input}

Including child documents by |\include| has some restrictions by design.
Most notably, the content of a child document always occupies
its own set of pages; pages cannot be shared between child documents.
Usually, this behaviour makes perfect sense
because each child document contain an essential part of the document.
However, in some situations it may be desirable to compose
a document from a collection of parts
without having mandatory page breaks between then.
For this case, the package
provides a mechanism to include parts
by |\input| which can also be processed individually.
However, by construction this mechanism
requires manual handling of the content to be output.

%%%%%%%%%%%%%%%%%%%%%%%%%%%%%%%%%%%%%%%%
\DescribeMacro{\ifchilddocmanual}
The main file should be prepared as usual, see \secref{sec:include}.
However, the document body must make a distinction
between processing of an individual part and of the main document, e.g.:
%
\begin{center}
\begin{tabular}{l}
|\ifchilddocmanual|\\
|\input{\childdocname}|\\
|\||else|\\
\textit{document body with }|\input{|\textit{part}|}|\\
|\||fi|
\end{tabular}
\end{center}
%
The conditional |\ifchilddocmanual| is true whenever
a part to be included by |\input| is being compiled,
and the name of the part is stored in |\childdocname|.

%%%%%%%%%%%%%%%%%%%%%%%%%%%%%%%%%%%%%%%%
\DescribeMacro{\childdocby}
Each part to be included by |\input| should start with:
%
\begin{center}
\begin{tabular}{l}
|\input{childdoc.def}|\\
|\childdocby{|\textit{main}|}|\\
\end{tabular}
\end{center}
%
The directive |\childdocby| is similar to |\childdocof|
described in \secref{sec:include},
but the subsequent selection of content must be done manually.
To that end, both |\ifchilddoc| and |\ifchilddocmanual|
will be true upon processing of a part,
and the name of the part is stored in |\childdocname|.
Note that |\jobname| will be set to the filename of the current part
so that each part receives an individual |.aux| file
that does not interfere with the |.aux| file(s) of the main document.
This behaviour can be altered by the alternative form
|\childdocby[*]{|\textit{main}|}| (with a non-empty optional argument)
which uses the |.aux| file of the main document
by setting |\jobname| to \textit{main}.

%%%%%%%%%%%%%%%%%%%%%%%%%%%%%%%%%%%%%%%%%%%%%%%%%%%%%%%%%%%%%%%%%%%%%%%%%%%%%%%%
\subsection{Driver Development}
\label{sec:driver}

The \textsf{childdoc} mechanism can also be use for the development
of definition files such as \LaTeX{} styles or classes.
This case differs from the above setup with multiple parts
included by |\include| in that no |\includeonly| should be invoked.
This can be achieved by starting the include file
(before |\ProvidesPackage|) with:
%
\begin{center}
\begin{tabular}{l}
|\input{childdoc.def}|\\
|\childdocforward{|\textit{main}|}|\\
\end{tabular}
\end{center}
%
or alternatively with:
%
\begin{center}
\begin{tabular}{l}
|\input{childdoc.def}|\\
|\childdocby{|\textit{main}|}|\\
\end{tabular}
\end{center}
%
Both forms have slightly different effects as described above.
The main file is prepared as usual, see \secref{sec:include}.

%%%%%%%%%%%%%%%%%%%%%%%%%%%%%%%%%%%%%%%%%%%%%%%%%%%%%%%%%%%%%%%%%%%%%%%%%%%%%%%%
\subsection{Legacy Detection}
\label{sec:detection}

The directive |\childdocmain| in the main file can detect
whether the complete document or merely a child is to be compiled
even without using the directive |\childdocof|.
This method is deprecated because it is less robust
and there is no compelling reason to use it;
it is merely provided for backward compatibility
and it may be removed in future versions.

If the detection mechanism is to be used,
it is mandatory to correctly specify
the filename of the main file as the argument of |\childdocmain|:
%
\begin{center}
\begin{tabular}{l}
|\input{childdoc.def}|\\
|\childdocmain{|\textit{main}|}|\\
\end{tabular}
\end{center}
%
If |\jobname| does not match the argument \textit{main} of |\childdocmain|,
it is assumed that |\jobname| points to the child file to be compiled.
When using |\childdocmain| with the main file specified as argument,
it suffices to start a child file
with just |\input{|\textit{main}|}|
without loading of the package and using |\childdocof|.
If instead all processing is done
with the appropriate \textsf{childdoc} directives,
the argument of \textit{main} of |\childdocmain| can be empty.

An alternative version of the command line processing described
in \secref{sec:commandline} using the detection mechanism reads:
%
\begin{center}
|... -jobname "|\textit{target}|" "|[\textit{flags}]%
[|\def\jobname{|\textit{dest}|}|]|\input{|\textit{main}|}"|
\end{center}

%%%%%%%%%%%%%%%%%%%%%%%%%%%%%%%%%%%%%%%%%%%%%%%%%%%%%%%%%%%%%%%%%%%%%%%%%%%%%%%%
\subsection{Manual Code}
\label{sec:manual}

In case one cannot be certain whether the definitions file |childdoc.def|
is installed on the target \TeX{} distribution
and one prefers not to ship it,
it is conceivable to paste a few relevant commands into the sources.

To that end, drop all statements |\input{childdoc.def}|
and perform the replacements as outlined below.
Instead of |\childdocmain{|\textit{main}|}| add the following code
to the top of the main file:
%
\begin{center}
\begin{tabular}{l}
|\||ifdefined\childdocname\endinput\||fi\newif\ifchilddoc|\\
|\edef\childdocname{\scantokens\expandafter{\jobname\noexpand}}|\\
|\def\childdocmain{|\textit{main}|}\||ifx\childdocmain\childdocname\||else|\\
|\childdoctrue\includeonly{\childdocname}\let\jobname\childdocmain\||fi|\\
\end{tabular}
\end{center}
%
Instead of |\childdocof{|\textit{main}|}| just include the main file
at the top of each child file:
%
\begin{center}
|\input{|\textit{main}|}|
\end{center}
%
A simple redirection |\childdocforward{|\textit{dest}|}| is achieved by:
%
\begin{center}
|\def\jobname{|\textit{dest}|}\input{\jobname}|
\end{center}
%
The redirection with prefix
|\childdocforwardprefix[|\textit{prefix}|]{|\textit{dest}|}|
is accomplished by:
%
\begin{center}
\begin{tabular}{l}
|{\edef\jobname{\scantokens\expandafter{\jobname\noexpand}}|\\
|\def\redirectjob |\textit{prefix}|#1~~~{\gdef\jobname{|\textit{dest}|#1}}|\\
|\expandafter\redirectjob\jobname~~~}\input{\jobname}|
\end{tabular}
\end{center}

In an alternative approach,
child documents can be compiled by a specific command line
without additional code or specific definitions:
%
\begin{center}
|... -jobname "|\textit{target}|" "|[\textit{flags}]%
|\includeonly{|\textit{dest}|}\input{|\textit{main}|}"|
\end{center}
%

%%%%%%%%%%%%%%%%%%%%%%%%%%%%%%%%%%%%%%%%%%%%%%%%%%%%%%%%%%%%%%%%%%%%%%%%%%%%%%%%
%%%%%%%%%%%%%%%%%%%%%%%%%%%%%%%%%%%%%%%%%%%%%%%%%%%%%%%%%%%%%%%%%%%%%%%%%%%%%%%%
\section{Information}

%%%%%%%%%%%%%%%%%%%%%%%%%%%%%%%%%%%%%%%%%%%%%%%%%%%%%%%%%%%%%%%%%%%%%%%%%%%%%%%%
\subsection{Copyright}

Copyright \copyright{} 2017--2018 Niklas Beisert

This work may be distributed and/or modified under the
conditions of the \LaTeX{} Project Public License, either version 1.3
of this license or (at your option) any later version.
The latest version of this license is in
  \url{http://www.latex-project.org/lppl.txt}
and version 1.3 or later is part of all distributions of \LaTeX{}
version 2005/12/01 or later.

This work has the LPPL maintenance status `maintained'.

The Current Maintainer of this work is Niklas Beisert.

This work consists of the files |README.txt|, |childdoc.ins| and |childdoc.dtx|
as well as the derived files |childdoc.def|, |cdocsamp.tex|
with |cdocsch1.tex|, |cdocsch2.tex|, |cdocspt3.tex|, |cdocspt4.tex|,
|cdocsdrf.tex|, |cdocsfn1.tex|, |cdocsfn2.tex|
as well as |childdoc.pdf|.

%%%%%%%%%%%%%%%%%%%%%%%%%%%%%%%%%%%%%%%%%%%%%%%%%%%%%%%%%%%%%%%%%%%%%%%%%%%%%%%%
\subsection{Files and Installation}

The package consists of the files:
%
\begin{center}
\begin{tabular}{ll}
    |README.txt|   & readme file \\
    |childdoc.ins| & installation file \\
    |childdoc.dtx| & source file \\
    |childdoc.def| & definition file \\
    |cdocsamp.tex| & sample main file \\
    |cdocsch1.tex| & sample include file \\
    |cdocsch2.tex| & sample include file \\
    |cdocspt3.tex| & sample part file \\
    |cdocspt4.tex| & sample part file \\
    |cdocsdrf.tex| & sample redirection file \\
    |cdocsfn1.tex| & sample redirection file \\
    |cdocsfn2.tex| & sample redirection file \\
    |childdoc.pdf| & manual
\end{tabular}
\end{center}
%
The distribution consists of the files
|README.txt|, |childdoc.ins| and |childdoc.dtx|.
%
\begin{itemize}
\item
Run (pdf)\LaTeX{} on |childdoc.dtx|
to compile the manual |childdoc.pdf| (this file).
\item
Run \LaTeX{} on |childdoc.ins| to create the definitions file |childdoc.def|
and the sample |cdocsamp.tex| with include files
|cdocsch1.tex|, |cdocsch2.tex|, |cdocspt3.tex|, |cdocspt4.tex|,
|cdocsdrf.tex|, |cdocsfn1.tex|, |cdocsfn2.tex|.
Then copy the file |childdoc.def| to an appropriate directory of your \LaTeX{}
distribution, e.g.\ \textit{texmf-root}|/tex/latex/childdoc|.
\end{itemize}

%%%%%%%%%%%%%%%%%%%%%%%%%%%%%%%%%%%%%%%%%%%%%%%%%%%%%%%%%%%%%%%%%%%%%%%%%%%%%%%%
\subsection{Related CTAN Packages}

There are several other packages which offer a similar functionality:
%
\begin{itemize}
\item
The packages
\href{http://ctan.org/pkg/docmute}{\textsf{docmute}},
\href{http://ctan.org/pkg/includex}{\textsf{includex}} and
\href{http://ctan.org/pkg/standalone}{\textsf{standalone}}
provide commands to include only the document body of
a child file thus allowing both files to be compiled individually.
\item
The packages \href{http://ctan.org/pkg/subdocs}{\textsf{subdocs}}
and \href{http://ctan.org/pkg/subfiles}{\textsf{subfiles}}
provide structures in which the main and child documents can be
encapsulated and allowing them to be compiled individually.
The inclusion mechanism is different from the conventional |\include|.
\item
The package \href{http://ctan.org/pkg/combine}{\textsf{combine}}
is an elaborate solution to combine several documents into one.
\end{itemize}
%
See also the CTAN topic \href{http://ctan.org/topic/subdocs}{\textsf{subdocs}}
for further related packages.
The present package differs from the above solutions in that
a document structure constructed with the conventional |\include| mechanism
just needs two extra commands at the top of every file
such that all constituent files can be compiled individually.

%%%%%%%%%%%%%%%%%%%%%%%%%%%%%%%%%%%%%%%%%%%%%%%%%%%%%%%%%%%%%%%%%%%%%%%%%%%%%%%%
%\subsection{Feature Suggestions}
%
%The following is a list of features which may be useful for future
%versions of this package:
%%
%\begin{itemize}
%\item
%\ldots
%\end{itemize}

%%%%%%%%%%%%%%%%%%%%%%%%%%%%%%%%%%%%%%%%%%%%%%%%%%%%%%%%%%%%%%%%%%%%%%%%%%%%%%%%
\subsection{Revision History}

%%%%%%%%%%%%%%%%%%%%%%%%%%%%%%%%%%%%%%%%
\paragraph{v2.0:} 2018/12/30

\begin{itemize}
\item
immediate forward processing
\item
added |\childdocby| mechanism
\item
manual restructured
\end{itemize}

%%%%%%%%%%%%%%%%%%%%%%%%%%%%%%%%%%%%%%%%
\paragraph{v1.6:} 2018/01/17

\begin{itemize}
\item
application for development of include files
\item
corrections to manual
\end{itemize}

%%%%%%%%%%%%%%%%%%%%%%%%%%%%%%%%%%%%%%%%
\paragraph{v1.5:} 2017/05/21

\begin{itemize}
\item
more complete structuring introduced
\item
|\childdocof| introduced
\item
|\childdoc| renamed to |\childdocmain|
\item
|\childredirect| renamed to |\childdocforward| and |\childdocforwardprefix|
and functionality expanded
\end{itemize}

%%%%%%%%%%%%%%%%%%%%%%%%%%%%%%%%%%%%%%%%
\paragraph{v1.0:} 2017/04/27

\begin{itemize}
\item
manual and install package
\item
first version published on CTAN
\end{itemize}

%%%%%%%%%%%%%%%%%%%%%%%%%%%%%%%%%%%%%%%%
\paragraph{v0.6:} 2017/04/26

\begin{itemize}
\item
redirection mechanism added
\end{itemize}

%%%%%%%%%%%%%%%%%%%%%%%%%%%%%%%%%%%%%%%%
\paragraph{v0.5:} 2017/04/26

\begin{itemize}
\item
functionality in definition file
\end{itemize}


%%%%%%%%%%%%%%%%%%%%%%%%%%%%%%%%%%%%%%%%%%%%%%%%%%%%%%%%%%%%%%%%%%%%%%%%%%%%%%%%
%%%%%%%%%%%%%%%%%%%%%%%%%%%%%%%%%%%%%%%%%%%%%%%%%%%%%%%%%%%%%%%%%%%%%%%%%%%%%%%%
%%%%%%%%%%%%%%%%%%%%%%%%%%%%%%%%%%%%%%%%%%%%%%%%%%%%%%%%%%%%%%%%%%%%%%%%%%%%%%%%
\appendix

\settowidth\MacroIndent{\rmfamily\scriptsize 000\ }

 \DocInput{childdoc.dtx}

\end{document}
%</driver>
% \fi
%
% %%%%%%%%%%%%%%%%%%%%%%%%%%%%%%%%%%%%%%%%%%%%%%%%%%%%%%%%%%%%%%%%%%%%%%%%%%%%%%
% %%%%%%%%%%%%%%%%%%%%%%%%%%%%%%%%%%%%%%%%%%%%%%%%%%%%%%%%%%%%%%%%%%%%%%%%%%%%%%
% \section{Sample}
%\iffalse
%<*samplemain>
%\fi
%
% The following presents a sample document
% with two chapters, two parts, a title page,
% a compile flag as well as three forwarding files to set the flag.
% It consists of eight |.tex| files:
% \begin{center}
% \begin{tabular}{ll}
% |cdocsamp.tex|&main file\\
% |cdocsch1.tex|&include file for chapter 1\\
% |cdocsch2.tex|&include file for chapter 2\\
% |cdocspt3.tex|&include file for part 3\\
% |cdocspt4.tex|&include file for part 4\\
% |cdocsdrf.tex|&forwarding file for main file in draft mode\\
% |cdocsfi1.tex|&forwarding file for final version of chapter 1\\
% |cdocsfi2.tex|&forwarding file for final version of chapter 2\\
% \end{tabular}
% \end{center}
% Each of the eight files can be compiled directly by the \LaTeX{} compiler.
%
% %%%%%%%%%%%%%%%%%%%%%%%%%%%%%%%%%%%%%%
% \paragraph{Main File.}
%
% The main file is called |cdocsamp.tex|.
%
% Load the \textsf{childdoc} definitions and
% declare the filename for the main document:
%    \begin{macrocode}
\input{childdoc.def}
\childdocmain{}
%    \end{macrocode}

% Optional override for |\version| flag:
%    \begin{macrocode}
%%\ifchilddoc\else\providecommand{\version}{draft}\fi
%    \end{macrocode}

% Define the default values for the |\version| flag
% (|final| for the main file and |draft| for childs):
%    \begin{macrocode}
\ifchilddoc
\providecommand{\version}{draft}
\else
\providecommand{\version}{final}
\fi
%    \end{macrocode}

% Load the standard document class:
%    \begin{macrocode}
\documentclass[12pt]{article}
%    \end{macrocode}

% Start the document body:
%    \begin{macrocode}
\begin{document}
%    \end{macrocode}

% Declare a title page.
% Print title, part of document being processed and version flag:
%    \begin{macrocode}
\addtocounter{page}{-1}
\begin{center}
{\LARGE\bfseries{}childdoc example\par}
\vspace{1cm}
\ifchilddoc
\ifchilddocmanual part\else chapter\fi:
`\childdocname' of `\childdocjob'\par
\else
main document: `\childdocjob'\par
\fi
version: \version\par
\end{center}
\newpage
%    \end{macrocode}

% Manually include selected file,
% otherwise process as usual:
%    \begin{macrocode}
\ifchilddocmanual
\section*{part `\childdocname'}
\input{\childdocname}
\else
%    \end{macrocode}

% Include the two chapters:
%    \begin{macrocode}
\include{cdocsch1}
\include{cdocsch2}
%    \end{macrocode}

% Include the two parts unless only chapters should be displayed:
%    \begin{macrocode}
\ifchilddoc\else
\section{part three}
\input{cdocspt3}
\section{part four}
\input{cdocspt4}
\fi
%    \end{macrocode}

% Process as usual until here:
%    \begin{macrocode}
\fi
%    \end{macrocode}

% End of document body:
%    \begin{macrocode}
\end{document}
%    \end{macrocode}
%\iffalse
%</samplemain>
%\fi
%
% %%%%%%%%%%%%%%%%%%%%%%%%%%%%%%%%%%%%%%
% \paragraph{Chapter Include Files.}
%
% The include files are called |cdocsch1.tex| and |cdocsch2.tex|.
%
%\iffalse
%<*samplechap1|samplechap2>
%\fi

% Optional override for |\version| flag:
%    \begin{macrocode}
%%\providecommand{\version}{final}
%    \end{macrocode}

% Include the main document:
%    \begin{macrocode}
\input{childdoc.def}
\childdocof{cdocsamp}
%    \end{macrocode}

%\iffalse
%</samplechap1|samplechap2>
%\fi
%
%\iffalse
%<*samplechap1>
%\fi
% Some text for chapter 1:
%    \begin{macrocode}
\section{one}
some text in chapter one
%    \end{macrocode}

%\iffalse
%</samplechap1>
%\fi
% Some text for chapter 2:
%\iffalse
%<*samplechap2>
%\fi
%    \begin{macrocode}
\section{two}
more text in chapter two
%    \end{macrocode}

%\iffalse
%</samplechap2>
%\fi
%
% %%%%%%%%%%%%%%%%%%%%%%%%%%%%%%%%%%%%%%
% \paragraph{Part Include Files.}
%
% The include files are called |cdocspt3.tex| and |cdocspt4.tex|.
%
%\iffalse
%<*samplepart3|samplepart4>
%\fi

% Optional override for |\version| flag:
%    \begin{macrocode}
%%\providecommand{\version}{final}
%    \end{macrocode}

% Include the main document:
%    \begin{macrocode}
\input{childdoc.def}
\childdocby{cdocsamp}
%    \end{macrocode}

%\iffalse
%</samplepart3|samplepart4>
%\fi
%
%\iffalse
%<*samplepart3>
%\fi
% Some text for part 3:
%    \begin{macrocode}
some text in part three
%    \end{macrocode}

%\iffalse
%</samplepart3>
%\fi
% Some text for part 4:
%\iffalse
%<*samplepart4>
%\fi
%    \begin{macrocode}
more text in part four
%    \end{macrocode}

%\iffalse
%</samplepart4>
%\fi
%
% %%%%%%%%%%%%%%%%%%%%%%%%%%%%%%%%%%%%%%
% \paragraph{Forwarding for a Complete Draft.}
%
% The following forwarding file |cdocsdrf.tex|
% compiles the main document in draft mode:
%\iffalse
%<*sampledraft>
%\fi
%    \begin{macrocode}
\def\version{draft}
\input{childdoc.def}
\childdocforward{cdocsamp}
%    \end{macrocode}

%\iffalse
%</sampledraft>
%\fi
%
% %%%%%%%%%%%%%%%%%%%%%%%%%%%%%%%%%%%%%%
% \paragraph{Forwarding for Final Version of the Chapters.}
%
% The following forwarding files |cdocsfn1.tex| and |cdocsfn2.tex|
% (with identical content)
% compile the final versions of the child documents
% |cdocsch1.tex| and |cdocsch2.tex|, respectively:
%\iffalse
%<*samplefinal>
%\fi
%    \begin{macrocode}
\def\version{final}
\input{childdoc.def}
\childdocforwardprefix[cdocsamp]{cdocsfn}{cdocsch}
%    \end{macrocode}

%\iffalse
%</samplefinal>
%\fi
%
% %%%%%%%%%%%%%%%%%%%%%%%%%%%%%%%%%%%%%%
% \paragraph{Command Line Processing.}
%
% The following three command lines generate the output files
% |cdocscld|, |cdocscl1| and |cdocscl2|
% which should be identical to
% |cdocsdrf|, |cdocsch1| and |cdocsfn2|, respectively:
% \begin{center}
% \begin{tabular}{l}
% |latex -jobname cdocscld \|\\
% |  "\def\version{draft}\input{childdoc.def}\childdocforward{cdocsamp}"|\\
% |latex -jobname cdocscl1 \|\\
% |  "\input{childdoc.def}\childdocforward[cdocsamp]{cdocsch1}"|\\
% |latex -jobname cdocscl2 \|\\
% |  "\def\version{final}\input{childdoc.def}\childdocforward{cdocsch2}"|
% \end{tabular}
% \end{center}
% Note that the trailing backslash on each first line
% merely continues the input to the second line
% (for convenient cut ant paste).
% Furthermore, the command |latex| can be replaced by any
% of its alternative versions such as |pdflatex|.
%
% %%%%%%%%%%%%%%%%%%%%%%%%%%%%%%%%%%%%%%%%%%%%%%%%%%%%%%%%%%%%%%%%%%%%%%%%%%%%%%
% %%%%%%%%%%%%%%%%%%%%%%%%%%%%%%%%%%%%%%%%%%%%%%%%%%%%%%%%%%%%%%%%%%%%%%%%%%%%%%
% \section{Implementation}
%\iffalse
%<*package>
%\fi
%
% This section describes the definitions file |childdoc.def|.

% The definitions cannot be loaded using |\usepackage| or |\RequirePackage|
% which has a mechanism to prevent loading a style file more than once.
% When loading the definitions by means of |\input|
% multiple instances have to be prevented manually:
%\iffalse
%This code needs to be before the `\ProvidesFile' directive
%which is defined at the beginning of this file.
%Therefore it is also placed there and commented out here.
%</package>
%<*discard>
%\fi
%    \begin{macrocode}
\ifdefined\childdocmain\endinput\fi
%    \end{macrocode}
%\iffalse
%</discard>
%<*package>
%\fi
%
% \macro{\ifchilddoc}
% \macro{\ifchilddocmanual}
% The conditional |\ifchilddoc| tells whether a
% child (true) or main (false) document is being compiled.
% The conditional |\ifchilddocmanual| tells whether
% the |\includeonly| mechanism is used (false) or
% the selection of child files must be performed manually (true).
% The definitions initialise to false:
%    \begin{macrocode}
\newif\ifchilddoc
\newif\ifchilddocmanual
%    \end{macrocode}

% \macro{\childdocname}
% \macro{\childdocjob}
% The macro |\childdocname| stores the name of the main document
% to be compiled. The macro |\childdocjob| stores the name of
% the document on which the \LaTeX{} compiler was originally invoked.
% The content of |\jobname| cannot be compared
% to filenames specified in the source due to different catcodes.
% The following code rescans |\jobname|, stores the result
% in |\childdocname| and saves a copy in |\childdocjob|:
%    \begin{macrocode}
\edef\childdocname{\scantokens\expandafter{\jobname\noexpand}}
\let\childdocjob\childdocname
%    \end{macrocode}

% \macro{\childdocdisable}
% The macro |\childdocdisable| prevents the main file
% from being processed more than once.
% At this stage, the main document command |\childdocmain|
% is assumed to be called once again where it should do nothing.
% Any subsequent call to it should prevent
% a secondary processing of the main document
% It overwrites the forwarding commands
% |\childdocof| and |\childdocforward|
% with empty macros to prevent further inclusions of the main document:
%    \begin{macrocode}
\newcommand{\childdocdisable}
{
  \renewcommand{\childdocmain}[1]{\renewcommand{\childdocmain}[1]{\endinput}}
  \renewcommand{\childdocof}[1]{}
  \renewcommand{\childdocby}[2][]{}
  \renewcommand{\childdocforward}[2][]{}
  \renewcommand{\childdocdisable}{}
}
%    \end{macrocode}

% \macro{\childdocmain}
% The macro |\childdocmain| is to be called at the top of the main file
% with nothing or the main filename (without extension) as argument.
% First, it breaks loops.
% If the argument is not empty and does not match |\childdocname|
% (which is set by the first inclusion of |childdoc.def|),
% |\ifchilddoc| is set to true, |\includeonly| is applied to the child file
% and |\jobname| is set to the main file
% (for proper handling of |.aux| files):
%    \begin{macrocode}
\newcommand{\childdocmain}[1]
{
  \childdocdisable\childdocmain{}
  \if?#1?\else
    \begingroup
      \def\childdoctmp{#1}
      \ifx\childdoctmp\childdocname
        \def\childdoctmp{}
      \else
        \def\childdoctmp
        {
          \childdoctrue
          \includeonly{\childdocname}
          \def\childdocjob{#1}
          \def\jobname{#1}
        }
      \fi
      \expandafter
    \endgroup
    \childdoctmp
  \fi
}
%    \end{macrocode}

% \macro{\childdocof}
% The command |\childdocof| redirects
% compilation to the main file |#1|.
%    \begin{macrocode}
\newcommand{\childdocof}[1]
{
  \childdocdisable
  \childdoctrue
  \includeonly{\childdocname}
  \def\jobname{#1}
  \def\childdocjob{#1}
  \input{#1}
}
%    \end{macrocode}

% \macro{\childdocby}
% The command |\childdocby| ....
%    \begin{macrocode}
\newcommand{\childdocby}[2][]
{
  \childdocdisable
  \childdoctrue
  \childdocmanualtrue
  \if?#1?\else
    \def\jobname{#2}
  \fi
  \def\childdocjob{#2}
  \input{#2}
  \endinput
}
%    \end{macrocode}

% \macro{\childdocforward}
% The command |\childdocforward| redirects
% compilation to the main file or
% (if the optional argument is given) a child file.
% Parameters are set as if the main file
% or a child file starting with |\childdocof| was compiled.
% Then compilation is handed over to the main file:
%    \begin{macrocode}
\newcommand{\childdocforward}[2][]
{
  \begingroup
    \if?#1?
      \def\childdoctmp
      {
        \def\childdocname{#2}
        \def\childdocjob{#2}
        \def\jobname{#2}
        \input{#2}
        \endinput
      }
    \else
      \def\childdoctmp
      {
        \childdocdisable
        \def\childdocname{#2}
        \childdoctrue
        \includeonly{#2}
        \def\childdocjob{#1}
        \def\jobname{#1}
        \input{#1}
        \endinput
      }
    \fi
    \expandafter
  \endgroup
  \childdoctmp
}
%    \end{macrocode}

% \macro{\childdocforwardprefix}
% The command |\childdocforwardprefix| redirects
% compilation to the main or a child file by means of a pattern.
% The prefix |#1| in the current filename is replaced by |#2|
% and the suffix of the current filename is kept
% (it is assumed that the filename does not contain the substring `|~~~|'
% which is used as a delimiter).
% Compilation is handed over to the new file by |\childdocforward|:
%    \begin{macrocode}
\newcommand{\childdocforwardprefix}[3][]
{
  \begingroup
    \def\childdocextract #2##1~~~{\def\childdoctmp{\childdocforward[#1]{#3##1}}}
    \expandafter\childdocextract\childdocname~~~
    \expandafter
  \endgroup
  \childdoctmp
}
%    \end{macrocode}

% \macro{\childdoc}
% The deprecated macro |\childdoc| is a legacy version of |\childdocmain|:
%    \begin{macrocode}
\newcommand{\childdoc}{\childdocmain}
%    \end{macrocode}

% \macro{\childdocredirect}
% The deprecated macro |\childdocredirect| is a legacy version
% of |\childdocforward| and |\childdocforwardprefix|:
%    \begin{macrocode}
\newcommand{\childdocredirect}[2][]
{
  \begingroup
    \if?#1?
      \def\childdoctmp{\childdocforward{#2}}
    \else
      \def\childdoctmp{\childdocforwardprefix{#1}{#2}}
    \fi
    \expandafter
  \endgroup
  \childdoctmp
}
%    \end{macrocode}

%\iffalse
%</package>
%\fi
%
\endinput
|\\
|\childdocof{|\textit{main}|}|\\
\end{tabular}
\end{center}
at the top of every child file \textit{child}
which is included by |\include{|\textit{child}|}|
from within the main file
(or at least for those files to be compiled individually).
The argument \textit{main} must be the filename of the main file.

There are a couple of
considerations in setting up the main and child documents:

%%%%%%%%%%%%%%%%%%%%%%%%%%%%%%%%%%%%%%%%
\paragraph{Restrictions.}

Please note the following restrictions:
\begin{itemize}
\item
|\childdocmain| must be called with one argument \textit{main}
to ensure compatibility with earlier version of the package.
It must either be empty (|\childdocmain{}|)
or precisely match the filename of the main file in which it is specified.
See \secref{sec:detection} for further information.
\item
The filename \textit{main} must be specified without the |.tex| extension.
\item
The filename \textit{main} is case sensitive
(even in case-insensitive file systems)
due to internal string comparison.
\item
The argument \textit{main} should be fully expanded, it cannot be a macro.
\item
Subdirectories and special characters should be avoided in filenames.
\item
The command |\childdocmain{|\textit{main}|}| must be followed by a whitespace.
It should not be followed immediately by another command
or by a comment mark `|%|'.
This is because the \TeX{} parser reads the token immediately following
the argument of |\childdocmain| and puts it
at the beginning of every child section;
however, a white\-space is ignored.
\end{itemize}

%%%%%%%%%%%%%%%%%%%%%%%%%%%%%%%%%%%%%%%%
\paragraph{Content of Main File.}

It is advisable to place all content in the child files included by |\include|.
Any output contained in the main file will appear in all child documents
unless suppressed manually;
it cannot be suppressed automatically by the |\includeonly| directive
and thus should normally be avoided.
A method to include some content in the main file
by means of conditional processing is described in \secref{sec:conditional}.

%%%%%%%%%%%%%%%%%%%%%%%%%%%%%%%%%%%%%%%%
\paragraph{Page Numbering.}

When only a part of the document is compiled,
the appropriate numbering of pages
(as well as other status parameters)
is determined from the |.aux| files.
The latter contain information from previous passes.
However this information needs to propagate through
all intermediate child documents.
Therefore the page numbering in child documents may well
be inconsistent until the complete document is compiled at least once.

A useful (if unconventional) way to always ensure a consistent
page numbering is to restart the numbering in each child document
and denote the pages by `\textit{child}|.|\textit{page}'
where \textit{child} represents the chapter/section number of the child file.
This can be achieved by the command
|\numberwithin{page}{|\textit{child}|}|
of the \textsf{amsmath} package
where \textit{child} can be |chapter| or |section|
depending on the chosen structuring.
Alternatively, one can modify the macro |\thepage| appropriately
and reset the counter |page| at the start of each child file.

%%%%%%%%%%%%%%%%%%%%%%%%%%%%%%%%%%%%%%%%%%%%%%%%%%%%%%%%%%%%%%%%%%%%%%%%%%%%%%%%
\subsection{Conditional Processing}
\label{sec:conditional}

The package provides a mechanism to compile different versions
of a document. To customise the versions further some conditional processing
can come in handy to distinguish which version is being compiled.
The package provides two macros to describe the compilation context:

%%%%%%%%%%%%%%%%%%%%%%%%%%%%%%%%%%%%%%%%
\DescribeMacro{\ifchilddoc}
The conditional |\ifchilddoc| distinguishes between the compilation of
child documents and the main document:
%
\begin{center}
|\ifchilddoc |\textit{child-code}| |[|\||else |\textit{main-code}]| \||fi|
\end{center}

%%%%%%%%%%%%%%%%%%%%%%%%%%%%%%%%%%%%%%%%
\DescribeMacro{\childdocname}
\DescribeMacro{\childdocjob}
The macro |\childdocname| contains the filename (without extension)
of the main or child file being processed.
Note that |\childdocjob| will always contain the name of the main file.

%%%%%%%%%%%%%%%%%%%%%%%%%%%%%%%%%%%%%%%%
\paragraph{Title Page.}

Conditional processing can be used to include a title or banner page
in the main document when proper precautions are taken.
Importantly, the code in the main file should ensure that the page counter
(as well as other status parameters which are stored in the |.aux| files)
takes the same value after the conditional processing.
Otherwise the page numbers may take divergent values
depending on which part is compiled.

For example, a title page could be declared by:
%
\begin{center}
\begin{tabular}{l}
|\ifchilddoc\||else|\\
|\addtocounter{page}{-1}|\\
\textit{code for title page}\\
|\newpage|\\
|\||fi|
\end{tabular}
\end{center}
%
A banner page for the child documents can be generated by:
%
\begin{center}
\begin{tabular}{l}
|\ifchilddoc|\\
|\addtocounter{page}{-1}|\\
\textit{code for banner page}\\
|\newpage|\\
|\||fi|
\end{tabular}
\end{center}
%
Here one could write a message such as:
\begin{center}
|This is the part \childdocname{} of \childdocjob{}.|
\end{center}

%%%%%%%%%%%%%%%%%%%%%%%%%%%%%%%%%%%%%%%%%%%%%%%%%%%%%%%%%%%%%%%%%%%%%%%%%%%%%%%%
\subsection{Flags}
\label{sec:flags}

The package makes it easy to generate different versions
of the main or child documents.
To this end compilation flags can be defined
and assigned different default values.
They will be particularly useful in conjunction
with the forwarding mechanism described in \secref{sec:forward}.

For example, it may be useful to have a flag |\version|
which can be set to |draft| or |final|.
The document source will contain some conditional code
depending on the value of |\version|.
Suppose further, the flag should default to |final| for the main file
and to |draft| for child files
which is a natural assignment for editing the document.
This is achieved by placing the following code
in the preamble of the main document
(below the |\childdocmain| directive):
%
\begin{center}
\begin{tabular}{l}
|\ifchilddoc|\\
|\providecommand{\version}{draft}|\\
|\||else|\\
|\providecommand{\version}{final}|\\
|\||fi|
\end{tabular}
\end{center}
%
The definition by |\providecommand| makes sure
that previous definitions are not overwritten.
Further statements |\providecommand{\version}{...}|
can thus be added before the above code to override it.

For the main file, one might add a line
(between |\childdocmain| and the above block)
%
\begin{center}
|%\ifchilddoc\||else\providecommand{\version}{draft}\||fi|
\end{center}
%
which can be uncommented to produce a draft version.
Likewise one can add a line to the very top of a child file
(above the |\childdocof{|\textit{main}|}| directive)
%
\begin{center}
|%\providecommand{\version}{final}|
\end{center}
%
which can be uncommented to produce the final version of this child document.

%%%%%%%%%%%%%%%%%%%%%%%%%%%%%%%%%%%%%%%%%%%%%%%%%%%%%%%%%%%%%%%%%%%%%%%%%%%%%%%%
\subsection{Forwarding}
\label{sec:forward}

Different versions of the main or child documents
using compilation flags as described in \secref{sec:flags}
can be (permanently) stored in different files
for convenient compilation, viewing and distribution.
To this end, the package defines a command
to pass on compilation to a different file:

%%%%%%%%%%%%%%%%%%%%%%%%%%%%%%%%%%%%%%%%
\DescribeMacro{\childdocforward}
The command |\childdocforward| redirects processing to
another source file:
%
\begin{center}
\begin{tabular}{l}
|% \iffalse
%
% childdoc.dtx Copyright (C) 2017-2018 Niklas Beisert
%
% This work may be distributed and/or modified under the
% conditions of the LaTeX Project Public License, either version 1.3
% of this license or (at your option) any later version.
% The latest version of this license is in
%   http://www.latex-project.org/lppl.txt
% and version 1.3 or later is part of all distributions of LaTeX
% version 2005/12/01 or later.
%
% This work has the LPPL maintenance status `maintained'.
%
% The Current Maintainer of this work is Niklas Beisert.
%
% This work consists of the files childdoc.dtx and childdoc.ins
% and the derived files childdoc.def and cdocsamp.tex with
% cdocsch1.tex, cdocsch2.tex, cdocsdrf.tex, cdocsfn1.tex, cdocsfn2.tex.
%
%<package>\ifdefined\childdocmain\endinput\fi
%<package>\ProvidesFile{childdoc.def}[2018/12/30 v2.0 child document driver]
%<samplemain>\ProvidesFile{cdocsamp.tex}[2018/12/30 v2.0 sample for childdoc]
%<*driver>
%\ProvidesFile{childdoc.drv}[2018/12/30 v2.0 childdoc reference manual file]
\PassOptionsToClass{10pt,a4paper}{article}
\documentclass{ltxdoc}

\usepackage[margin=35mm]{geometry}
\usepackage{hyperref}
\usepackage{hyperxmp}
\usepackage[usenames]{color}

\hypersetup{colorlinks=true}
\hypersetup{pdfstartview=FitH}
\hypersetup{pdfpagemode=UseNone}
\hypersetup{pdfsource={}}
\hypersetup{pdflang={en-UK}}
\hypersetup{pdfcopyright={Copyright 2017-2018 Niklas Beisert.
  This work may be distributed and/or modified under the
  conditions of the LaTeX Project Public License, either version 1.3
  of this license or (at your option) any later version.}}
\hypersetup{pdflicenseurl={http://www.latex-project.org/lppl.txt}}
\hypersetup{pdfcontactaddress={ETH Zurich, ITP, HIT K,
  Wolfgang-Pauli-Strasse 27}}
\hypersetup{pdfcontactpostcode={8093}}
\hypersetup{pdfcontactcity={Zurich}}
\hypersetup{pdfcontactcountry={Switzerland}}
\hypersetup{pdfcontactemail={nbeisert@itp.phys.ethz.ch}}
\hypersetup{pdfcontacturl={http://people.phys.ethz.ch/\xmptilde nbeisert/}}

\newcommand{\secref}[1]{\hyperref[#1]{section \ref*{#1}}}

\parskip1ex
\parindent0pt
\let\olditemize\itemize
\def\itemize{\olditemize\parskip0pt}

\begin{document}

\title{The \textsf{childdoc} Package}
\hypersetup{pdftitle={The childdoc Package}}
\author{Niklas Beisert\\[2ex]
  Institut f\"ur Theoretische Physik\\
  Eidgen\"ossische Technische Hochschule Z\"urich\\
  Wolfgang-Pauli-Strasse 27, 8093 Z\"urich, Switzerland\\[1ex]
  \href{mailto:nbeisert@itp.phys.ethz.ch}
  {\texttt{nbeisert@itp.phys.ethz.ch}}}
\hypersetup{pdfauthor={Niklas Beisert}}
\hypersetup{pdfsubject={Manual for the LaTeX2e Package childdoc}}
\date{30 December 2018, \textsf{v2.0}}
\maketitle

\begin{abstract}\noindent
\textsf{childdoc} is a \LaTeXe{} package
that enables the direct compilation
of document sections included by |\include|
to individual files.
\end{abstract}

\begingroup
\parskip0ex
\tableofcontents
\endgroup

%%%%%%%%%%%%%%%%%%%%%%%%%%%%%%%%%%%%%%%%%%%%%%%%%%%%%%%%%%%%%%%%%%%%%%%%%%%%%%%%
%%%%%%%%%%%%%%%%%%%%%%%%%%%%%%%%%%%%%%%%%%%%%%%%%%%%%%%%%%%%%%%%%%%%%%%%%%%%%%%%
\section{Introduction}

\LaTeX{} provides a mechanism to structure a large document (such as a book)
into a main file and several child files (containing the chapters)
using the |\include| command.
This mechanism is beneficial for documents
which span hundreds of pages in order to
make the source file(s) more manageable.
Moreover, compilation can be restricted to
selected child files by means of the |\includeonly| command.
The latter feature can be used to reduce the compilation time while editing
(this was significantly more useful in the earlier days of \LaTeX{})
or to generate a smaller document which is easier to navigate.
Another application of |\includeonly| is to generate
documents consisting of selected parts of the complete document.

However, there are a few drawbacks of the plain |\include| mechanism:
\begin{itemize}
\item
The child files cannot be compiled on their own,
they can only be compiled via the main file.
A naive editing environment
(such as a text editor with an option
to have the current file processed by \LaTeX)
may require one to switch to the main file before compiling;
attempting to compile the child file produces errors.
\item
The main file must be modified (each time)
to adjust the |\includeonly| command
to the present needs. This easily leaves the main file in a messy state.
\item
The generated document will always carry the filename
of the main document. This is inconvenient if
several child files are to be compiled and
to be kept for distribution.
\end{itemize}

The present package provides a simple interface
to make child files individually compilable by \LaTeX{}.
Compiling a child file then has the same effect as compiling
the main file with an |\includeonly| command
to select the appropriate child.
Moreover the generated document will carry the name of the child
rather than the main file.
This resolves all three above issues.

This feature is meant to make the editing of books,
thesis documents and lecture notes somewhat more convenient.
However, the package can also be used efficiently for
composing a series of documents (such as exercise sheets)
which are typically distributed individually.
It then assists the author in generating the individual documents
(potentially in different versions)
as well as a document containing the collected series.
Another application is in developing style files
or other kinds of included material
where compilation of the style file could redirect
to a sample or test file.

%%%%%%%%%%%%%%%%%%%%%%%%%%%%%%%%%%%%%%%%%%%%%%%%%%%%%%%%%%%%%%%%%%%%%%%%%%%%%%%%
%%%%%%%%%%%%%%%%%%%%%%%%%%%%%%%%%%%%%%%%%%%%%%%%%%%%%%%%%%%%%%%%%%%%%%%%%%%%%%%%
\section{Usage}

First of all, the package \textsf{childdoc} is \emph{not} a standard
\LaTeXe{} |.sty| style file! Therefore it needs to be invoked in
a non-standard way.

%%%%%%%%%%%%%%%%%%%%%%%%%%%%%%%%%%%%%%%%%%%%%%%%%%%%%%%%%%%%%%%%%%%%%%%%%%%%%%%%
\subsection{Included Files}
\label{sec:include}

%%%%%%%%%%%%%%%%%%%%%%%%%%%%%%%%%%%%%%%%
\DescribeMacro{\childdocmain}
To use the package, add the commands
\begin{center}
\begin{tabular}{l}
|\input{childdoc.def}|\\
|\childdocmain{}|\\
\end{tabular}
\end{center}
at the very top of the main \LaTeX{} file,
in particular \emph{before} the |\documentclass| statement!
The argument of |\childdocmain| should be left empty
(but it must be present).

%%%%%%%%%%%%%%%%%%%%%%%%%%%%%%%%%%%%%%%%
\DescribeMacro{\childdocof}
Furthermore, add the commands
\begin{center}
\begin{tabular}{l}
|\input{childdoc.def}|\\
|\childdocof{|\textit{main}|}|\\
\end{tabular}
\end{center}
at the top of every child file \textit{child}
which is included by |\include{|\textit{child}|}|
from within the main file
(or at least for those files to be compiled individually).
The argument \textit{main} must be the filename of the main file.

There are a couple of
considerations in setting up the main and child documents:

%%%%%%%%%%%%%%%%%%%%%%%%%%%%%%%%%%%%%%%%
\paragraph{Restrictions.}

Please note the following restrictions:
\begin{itemize}
\item
|\childdocmain| must be called with one argument \textit{main}
to ensure compatibility with earlier version of the package.
It must either be empty (|\childdocmain{}|)
or precisely match the filename of the main file in which it is specified.
See \secref{sec:detection} for further information.
\item
The filename \textit{main} must be specified without the |.tex| extension.
\item
The filename \textit{main} is case sensitive
(even in case-insensitive file systems)
due to internal string comparison.
\item
The argument \textit{main} should be fully expanded, it cannot be a macro.
\item
Subdirectories and special characters should be avoided in filenames.
\item
The command |\childdocmain{|\textit{main}|}| must be followed by a whitespace.
It should not be followed immediately by another command
or by a comment mark `|%|'.
This is because the \TeX{} parser reads the token immediately following
the argument of |\childdocmain| and puts it
at the beginning of every child section;
however, a white\-space is ignored.
\end{itemize}

%%%%%%%%%%%%%%%%%%%%%%%%%%%%%%%%%%%%%%%%
\paragraph{Content of Main File.}

It is advisable to place all content in the child files included by |\include|.
Any output contained in the main file will appear in all child documents
unless suppressed manually;
it cannot be suppressed automatically by the |\includeonly| directive
and thus should normally be avoided.
A method to include some content in the main file
by means of conditional processing is described in \secref{sec:conditional}.

%%%%%%%%%%%%%%%%%%%%%%%%%%%%%%%%%%%%%%%%
\paragraph{Page Numbering.}

When only a part of the document is compiled,
the appropriate numbering of pages
(as well as other status parameters)
is determined from the |.aux| files.
The latter contain information from previous passes.
However this information needs to propagate through
all intermediate child documents.
Therefore the page numbering in child documents may well
be inconsistent until the complete document is compiled at least once.

A useful (if unconventional) way to always ensure a consistent
page numbering is to restart the numbering in each child document
and denote the pages by `\textit{child}|.|\textit{page}'
where \textit{child} represents the chapter/section number of the child file.
This can be achieved by the command
|\numberwithin{page}{|\textit{child}|}|
of the \textsf{amsmath} package
where \textit{child} can be |chapter| or |section|
depending on the chosen structuring.
Alternatively, one can modify the macro |\thepage| appropriately
and reset the counter |page| at the start of each child file.

%%%%%%%%%%%%%%%%%%%%%%%%%%%%%%%%%%%%%%%%%%%%%%%%%%%%%%%%%%%%%%%%%%%%%%%%%%%%%%%%
\subsection{Conditional Processing}
\label{sec:conditional}

The package provides a mechanism to compile different versions
of a document. To customise the versions further some conditional processing
can come in handy to distinguish which version is being compiled.
The package provides two macros to describe the compilation context:

%%%%%%%%%%%%%%%%%%%%%%%%%%%%%%%%%%%%%%%%
\DescribeMacro{\ifchilddoc}
The conditional |\ifchilddoc| distinguishes between the compilation of
child documents and the main document:
%
\begin{center}
|\ifchilddoc |\textit{child-code}| |[|\||else |\textit{main-code}]| \||fi|
\end{center}

%%%%%%%%%%%%%%%%%%%%%%%%%%%%%%%%%%%%%%%%
\DescribeMacro{\childdocname}
\DescribeMacro{\childdocjob}
The macro |\childdocname| contains the filename (without extension)
of the main or child file being processed.
Note that |\childdocjob| will always contain the name of the main file.

%%%%%%%%%%%%%%%%%%%%%%%%%%%%%%%%%%%%%%%%
\paragraph{Title Page.}

Conditional processing can be used to include a title or banner page
in the main document when proper precautions are taken.
Importantly, the code in the main file should ensure that the page counter
(as well as other status parameters which are stored in the |.aux| files)
takes the same value after the conditional processing.
Otherwise the page numbers may take divergent values
depending on which part is compiled.

For example, a title page could be declared by:
%
\begin{center}
\begin{tabular}{l}
|\ifchilddoc\||else|\\
|\addtocounter{page}{-1}|\\
\textit{code for title page}\\
|\newpage|\\
|\||fi|
\end{tabular}
\end{center}
%
A banner page for the child documents can be generated by:
%
\begin{center}
\begin{tabular}{l}
|\ifchilddoc|\\
|\addtocounter{page}{-1}|\\
\textit{code for banner page}\\
|\newpage|\\
|\||fi|
\end{tabular}
\end{center}
%
Here one could write a message such as:
\begin{center}
|This is the part \childdocname{} of \childdocjob{}.|
\end{center}

%%%%%%%%%%%%%%%%%%%%%%%%%%%%%%%%%%%%%%%%%%%%%%%%%%%%%%%%%%%%%%%%%%%%%%%%%%%%%%%%
\subsection{Flags}
\label{sec:flags}

The package makes it easy to generate different versions
of the main or child documents.
To this end compilation flags can be defined
and assigned different default values.
They will be particularly useful in conjunction
with the forwarding mechanism described in \secref{sec:forward}.

For example, it may be useful to have a flag |\version|
which can be set to |draft| or |final|.
The document source will contain some conditional code
depending on the value of |\version|.
Suppose further, the flag should default to |final| for the main file
and to |draft| for child files
which is a natural assignment for editing the document.
This is achieved by placing the following code
in the preamble of the main document
(below the |\childdocmain| directive):
%
\begin{center}
\begin{tabular}{l}
|\ifchilddoc|\\
|\providecommand{\version}{draft}|\\
|\||else|\\
|\providecommand{\version}{final}|\\
|\||fi|
\end{tabular}
\end{center}
%
The definition by |\providecommand| makes sure
that previous definitions are not overwritten.
Further statements |\providecommand{\version}{...}|
can thus be added before the above code to override it.

For the main file, one might add a line
(between |\childdocmain| and the above block)
%
\begin{center}
|%\ifchilddoc\||else\providecommand{\version}{draft}\||fi|
\end{center}
%
which can be uncommented to produce a draft version.
Likewise one can add a line to the very top of a child file
(above the |\childdocof{|\textit{main}|}| directive)
%
\begin{center}
|%\providecommand{\version}{final}|
\end{center}
%
which can be uncommented to produce the final version of this child document.

%%%%%%%%%%%%%%%%%%%%%%%%%%%%%%%%%%%%%%%%%%%%%%%%%%%%%%%%%%%%%%%%%%%%%%%%%%%%%%%%
\subsection{Forwarding}
\label{sec:forward}

Different versions of the main or child documents
using compilation flags as described in \secref{sec:flags}
can be (permanently) stored in different files
for convenient compilation, viewing and distribution.
To this end, the package defines a command
to pass on compilation to a different file:

%%%%%%%%%%%%%%%%%%%%%%%%%%%%%%%%%%%%%%%%
\DescribeMacro{\childdocforward}
The command |\childdocforward| redirects processing to
another source file:
%
\begin{center}
\begin{tabular}{l}
|\input{childdoc.def}|\\
|\childdocforward[|\textit{main}|]{|\textit{dest}|}|\\
\end{tabular}
\end{center}
%
The argument \textit{dest} is the destination file
(without extension).
It should be the main file or one of the child files.
Note that further \textsf{childdoc} directives
such as |\childdocof| and |\childdocforward|
in the indicated file will be processed in this form.
The optional argument \textit{main}
passes on directly to the main file \textit{main}
while pretending to compile the child \textit{dest}.
This form behaves as if \textit{dest}
issues |\childdocof{|\textit{main}|}| right away,
and no further \textsf{childdoc} directives will be processed.

%%%%%%%%%%%%%%%%%%%%%%%%%%%%%%%%%%%%%%%%
\DescribeMacro{\...prefix}
In the alternative form |\childdocforwardprefix|,
%
\begin{center}
\begin{tabular}{l}
|\input{childdoc.def}|\\
|\childdocforwardprefix[|\textit{main}|]{|\textit{prefix}|}{|\textit{dest}|}|
\end{tabular}
\end{center}
%
the destination file is determined by a pattern
depending on the current file:
To make this work, the current file must be called
`{\textit{prefix}\hspace{0.2em}\textit{suffix}}'
with \textit{prefix} matching precisely the argument.
Processing is then passed on to the file
`{\textit{dest}\hspace{0.2em}\textit{suffix}}'.
Surely, the same effect is achieved by
directly specifying the
argument `{\textit{dest}\hspace{0.2em}\textit{suffix}}'
in the first form.
However, that requires to set up a different file
for each child. With the alternative form of the command
all these files can have exactly the same content
which simplifies setting them up and maintaining them.

For example, the following file |draft.tex|
with a compilation flag |\version| as described in \secref{sec:flags}
compiles the main document as a draft:
%
\begin{center}
\begin{tabular}{l}
|\def\version{draft}|\\
|\input{childdoc.def}|\\
|\childdocforward{|\textit{main}|}|
\end{tabular}
\end{center}
%
Likewise, the following files |final|\textit{nn}|.tex|
compile the final version of the child document
|child|\textit{nn}|.tex|:
%
\begin{center}
\begin{tabular}{l}
|\def\version{final}|\\
|\input{childdoc.def}|\\
|\childdocforwardprefix{final}{child}|
\end{tabular}
\end{center}
%

Note that when several versions of a main file and/or of each child file
are to be generated, it may be convenient to set up a |Makefile| or
shell script to automatise the process.

%%%%%%%%%%%%%%%%%%%%%%%%%%%%%%%%%%%%%%%%%%%%%%%%%%%%%%%%%%%%%%%%%%%%%%%%%%%%%%%%
\subsection{Command Line Processing}
\label{sec:commandline}

The effect of redirection files can also be achieved by invoking
the \LaTeX{} compiler with a more elaborate command line.
Most conveniently this should be done as part
of a shell script or a |Makefile|.

When using \textsf{childdoc} in the main file, the following
command lines effectively perform a redirection
(note that depending on the shell being used,
backslashes may have to be doubled: `|\|' $\to$ `|\\|'):
%
\begin{center}
|... -jobname "|\textit{target}|" |\\|"|[\textit{flags}]%
|\input{childdoc.def}\childdocforward[|\textit{main}|]{|\textit{dest}|}"|
\end{center}
%
Here \textit{target} is the name of the output file,
\textit{main} is the name of the main file
and \textit{dest} is the name of the main or child file to be processed
(all filenames without extensions).
The optional argument \textit{main} can be omitted
if \textit{main} matches \textit{dest}.
Optionally, compilation \textit{flags} can be defined via |\def| commands.
This command line makes the \TeX{} engine believe
it is compiling the file \textit{target}
whose content is specified as the latter parameter.
The provided code then forwards the processing to
\textit{main} or \textit{dest} as described in \secref{sec:forward}.

%%%%%%%%%%%%%%%%%%%%%%%%%%%%%%%%%%%%%%%%%%%%%%%%%%%%%%%%%%%%%%%%%%%%%%%%%%%%%%%%
\subsection{Include by Input}
\label{sec:input}

Including child documents by |\include| has some restrictions by design.
Most notably, the content of a child document always occupies
its own set of pages; pages cannot be shared between child documents.
Usually, this behaviour makes perfect sense
because each child document contain an essential part of the document.
However, in some situations it may be desirable to compose
a document from a collection of parts
without having mandatory page breaks between then.
For this case, the package
provides a mechanism to include parts
by |\input| which can also be processed individually.
However, by construction this mechanism
requires manual handling of the content to be output.

%%%%%%%%%%%%%%%%%%%%%%%%%%%%%%%%%%%%%%%%
\DescribeMacro{\ifchilddocmanual}
The main file should be prepared as usual, see \secref{sec:include}.
However, the document body must make a distinction
between processing of an individual part and of the main document, e.g.:
%
\begin{center}
\begin{tabular}{l}
|\ifchilddocmanual|\\
|\input{\childdocname}|\\
|\||else|\\
\textit{document body with }|\input{|\textit{part}|}|\\
|\||fi|
\end{tabular}
\end{center}
%
The conditional |\ifchilddocmanual| is true whenever
a part to be included by |\input| is being compiled,
and the name of the part is stored in |\childdocname|.

%%%%%%%%%%%%%%%%%%%%%%%%%%%%%%%%%%%%%%%%
\DescribeMacro{\childdocby}
Each part to be included by |\input| should start with:
%
\begin{center}
\begin{tabular}{l}
|\input{childdoc.def}|\\
|\childdocby{|\textit{main}|}|\\
\end{tabular}
\end{center}
%
The directive |\childdocby| is similar to |\childdocof|
described in \secref{sec:include},
but the subsequent selection of content must be done manually.
To that end, both |\ifchilddoc| and |\ifchilddocmanual|
will be true upon processing of a part,
and the name of the part is stored in |\childdocname|.
Note that |\jobname| will be set to the filename of the current part
so that each part receives an individual |.aux| file
that does not interfere with the |.aux| file(s) of the main document.
This behaviour can be altered by the alternative form
|\childdocby[*]{|\textit{main}|}| (with a non-empty optional argument)
which uses the |.aux| file of the main document
by setting |\jobname| to \textit{main}.

%%%%%%%%%%%%%%%%%%%%%%%%%%%%%%%%%%%%%%%%%%%%%%%%%%%%%%%%%%%%%%%%%%%%%%%%%%%%%%%%
\subsection{Driver Development}
\label{sec:driver}

The \textsf{childdoc} mechanism can also be use for the development
of definition files such as \LaTeX{} styles or classes.
This case differs from the above setup with multiple parts
included by |\include| in that no |\includeonly| should be invoked.
This can be achieved by starting the include file
(before |\ProvidesPackage|) with:
%
\begin{center}
\begin{tabular}{l}
|\input{childdoc.def}|\\
|\childdocforward{|\textit{main}|}|\\
\end{tabular}
\end{center}
%
or alternatively with:
%
\begin{center}
\begin{tabular}{l}
|\input{childdoc.def}|\\
|\childdocby{|\textit{main}|}|\\
\end{tabular}
\end{center}
%
Both forms have slightly different effects as described above.
The main file is prepared as usual, see \secref{sec:include}.

%%%%%%%%%%%%%%%%%%%%%%%%%%%%%%%%%%%%%%%%%%%%%%%%%%%%%%%%%%%%%%%%%%%%%%%%%%%%%%%%
\subsection{Legacy Detection}
\label{sec:detection}

The directive |\childdocmain| in the main file can detect
whether the complete document or merely a child is to be compiled
even without using the directive |\childdocof|.
This method is deprecated because it is less robust
and there is no compelling reason to use it;
it is merely provided for backward compatibility
and it may be removed in future versions.

If the detection mechanism is to be used,
it is mandatory to correctly specify
the filename of the main file as the argument of |\childdocmain|:
%
\begin{center}
\begin{tabular}{l}
|\input{childdoc.def}|\\
|\childdocmain{|\textit{main}|}|\\
\end{tabular}
\end{center}
%
If |\jobname| does not match the argument \textit{main} of |\childdocmain|,
it is assumed that |\jobname| points to the child file to be compiled.
When using |\childdocmain| with the main file specified as argument,
it suffices to start a child file
with just |\input{|\textit{main}|}|
without loading of the package and using |\childdocof|.
If instead all processing is done
with the appropriate \textsf{childdoc} directives,
the argument of \textit{main} of |\childdocmain| can be empty.

An alternative version of the command line processing described
in \secref{sec:commandline} using the detection mechanism reads:
%
\begin{center}
|... -jobname "|\textit{target}|" "|[\textit{flags}]%
[|\def\jobname{|\textit{dest}|}|]|\input{|\textit{main}|}"|
\end{center}

%%%%%%%%%%%%%%%%%%%%%%%%%%%%%%%%%%%%%%%%%%%%%%%%%%%%%%%%%%%%%%%%%%%%%%%%%%%%%%%%
\subsection{Manual Code}
\label{sec:manual}

In case one cannot be certain whether the definitions file |childdoc.def|
is installed on the target \TeX{} distribution
and one prefers not to ship it,
it is conceivable to paste a few relevant commands into the sources.

To that end, drop all statements |\input{childdoc.def}|
and perform the replacements as outlined below.
Instead of |\childdocmain{|\textit{main}|}| add the following code
to the top of the main file:
%
\begin{center}
\begin{tabular}{l}
|\||ifdefined\childdocname\endinput\||fi\newif\ifchilddoc|\\
|\edef\childdocname{\scantokens\expandafter{\jobname\noexpand}}|\\
|\def\childdocmain{|\textit{main}|}\||ifx\childdocmain\childdocname\||else|\\
|\childdoctrue\includeonly{\childdocname}\let\jobname\childdocmain\||fi|\\
\end{tabular}
\end{center}
%
Instead of |\childdocof{|\textit{main}|}| just include the main file
at the top of each child file:
%
\begin{center}
|\input{|\textit{main}|}|
\end{center}
%
A simple redirection |\childdocforward{|\textit{dest}|}| is achieved by:
%
\begin{center}
|\def\jobname{|\textit{dest}|}\input{\jobname}|
\end{center}
%
The redirection with prefix
|\childdocforwardprefix[|\textit{prefix}|]{|\textit{dest}|}|
is accomplished by:
%
\begin{center}
\begin{tabular}{l}
|{\edef\jobname{\scantokens\expandafter{\jobname\noexpand}}|\\
|\def\redirectjob |\textit{prefix}|#1~~~{\gdef\jobname{|\textit{dest}|#1}}|\\
|\expandafter\redirectjob\jobname~~~}\input{\jobname}|
\end{tabular}
\end{center}

In an alternative approach,
child documents can be compiled by a specific command line
without additional code or specific definitions:
%
\begin{center}
|... -jobname "|\textit{target}|" "|[\textit{flags}]%
|\includeonly{|\textit{dest}|}\input{|\textit{main}|}"|
\end{center}
%

%%%%%%%%%%%%%%%%%%%%%%%%%%%%%%%%%%%%%%%%%%%%%%%%%%%%%%%%%%%%%%%%%%%%%%%%%%%%%%%%
%%%%%%%%%%%%%%%%%%%%%%%%%%%%%%%%%%%%%%%%%%%%%%%%%%%%%%%%%%%%%%%%%%%%%%%%%%%%%%%%
\section{Information}

%%%%%%%%%%%%%%%%%%%%%%%%%%%%%%%%%%%%%%%%%%%%%%%%%%%%%%%%%%%%%%%%%%%%%%%%%%%%%%%%
\subsection{Copyright}

Copyright \copyright{} 2017--2018 Niklas Beisert

This work may be distributed and/or modified under the
conditions of the \LaTeX{} Project Public License, either version 1.3
of this license or (at your option) any later version.
The latest version of this license is in
  \url{http://www.latex-project.org/lppl.txt}
and version 1.3 or later is part of all distributions of \LaTeX{}
version 2005/12/01 or later.

This work has the LPPL maintenance status `maintained'.

The Current Maintainer of this work is Niklas Beisert.

This work consists of the files |README.txt|, |childdoc.ins| and |childdoc.dtx|
as well as the derived files |childdoc.def|, |cdocsamp.tex|
with |cdocsch1.tex|, |cdocsch2.tex|, |cdocspt3.tex|, |cdocspt4.tex|,
|cdocsdrf.tex|, |cdocsfn1.tex|, |cdocsfn2.tex|
as well as |childdoc.pdf|.

%%%%%%%%%%%%%%%%%%%%%%%%%%%%%%%%%%%%%%%%%%%%%%%%%%%%%%%%%%%%%%%%%%%%%%%%%%%%%%%%
\subsection{Files and Installation}

The package consists of the files:
%
\begin{center}
\begin{tabular}{ll}
    |README.txt|   & readme file \\
    |childdoc.ins| & installation file \\
    |childdoc.dtx| & source file \\
    |childdoc.def| & definition file \\
    |cdocsamp.tex| & sample main file \\
    |cdocsch1.tex| & sample include file \\
    |cdocsch2.tex| & sample include file \\
    |cdocspt3.tex| & sample part file \\
    |cdocspt4.tex| & sample part file \\
    |cdocsdrf.tex| & sample redirection file \\
    |cdocsfn1.tex| & sample redirection file \\
    |cdocsfn2.tex| & sample redirection file \\
    |childdoc.pdf| & manual
\end{tabular}
\end{center}
%
The distribution consists of the files
|README.txt|, |childdoc.ins| and |childdoc.dtx|.
%
\begin{itemize}
\item
Run (pdf)\LaTeX{} on |childdoc.dtx|
to compile the manual |childdoc.pdf| (this file).
\item
Run \LaTeX{} on |childdoc.ins| to create the definitions file |childdoc.def|
and the sample |cdocsamp.tex| with include files
|cdocsch1.tex|, |cdocsch2.tex|, |cdocspt3.tex|, |cdocspt4.tex|,
|cdocsdrf.tex|, |cdocsfn1.tex|, |cdocsfn2.tex|.
Then copy the file |childdoc.def| to an appropriate directory of your \LaTeX{}
distribution, e.g.\ \textit{texmf-root}|/tex/latex/childdoc|.
\end{itemize}

%%%%%%%%%%%%%%%%%%%%%%%%%%%%%%%%%%%%%%%%%%%%%%%%%%%%%%%%%%%%%%%%%%%%%%%%%%%%%%%%
\subsection{Related CTAN Packages}

There are several other packages which offer a similar functionality:
%
\begin{itemize}
\item
The packages
\href{http://ctan.org/pkg/docmute}{\textsf{docmute}},
\href{http://ctan.org/pkg/includex}{\textsf{includex}} and
\href{http://ctan.org/pkg/standalone}{\textsf{standalone}}
provide commands to include only the document body of
a child file thus allowing both files to be compiled individually.
\item
The packages \href{http://ctan.org/pkg/subdocs}{\textsf{subdocs}}
and \href{http://ctan.org/pkg/subfiles}{\textsf{subfiles}}
provide structures in which the main and child documents can be
encapsulated and allowing them to be compiled individually.
The inclusion mechanism is different from the conventional |\include|.
\item
The package \href{http://ctan.org/pkg/combine}{\textsf{combine}}
is an elaborate solution to combine several documents into one.
\end{itemize}
%
See also the CTAN topic \href{http://ctan.org/topic/subdocs}{\textsf{subdocs}}
for further related packages.
The present package differs from the above solutions in that
a document structure constructed with the conventional |\include| mechanism
just needs two extra commands at the top of every file
such that all constituent files can be compiled individually.

%%%%%%%%%%%%%%%%%%%%%%%%%%%%%%%%%%%%%%%%%%%%%%%%%%%%%%%%%%%%%%%%%%%%%%%%%%%%%%%%
%\subsection{Feature Suggestions}
%
%The following is a list of features which may be useful for future
%versions of this package:
%%
%\begin{itemize}
%\item
%\ldots
%\end{itemize}

%%%%%%%%%%%%%%%%%%%%%%%%%%%%%%%%%%%%%%%%%%%%%%%%%%%%%%%%%%%%%%%%%%%%%%%%%%%%%%%%
\subsection{Revision History}

%%%%%%%%%%%%%%%%%%%%%%%%%%%%%%%%%%%%%%%%
\paragraph{v2.0:} 2018/12/30

\begin{itemize}
\item
immediate forward processing
\item
added |\childdocby| mechanism
\item
manual restructured
\end{itemize}

%%%%%%%%%%%%%%%%%%%%%%%%%%%%%%%%%%%%%%%%
\paragraph{v1.6:} 2018/01/17

\begin{itemize}
\item
application for development of include files
\item
corrections to manual
\end{itemize}

%%%%%%%%%%%%%%%%%%%%%%%%%%%%%%%%%%%%%%%%
\paragraph{v1.5:} 2017/05/21

\begin{itemize}
\item
more complete structuring introduced
\item
|\childdocof| introduced
\item
|\childdoc| renamed to |\childdocmain|
\item
|\childredirect| renamed to |\childdocforward| and |\childdocforwardprefix|
and functionality expanded
\end{itemize}

%%%%%%%%%%%%%%%%%%%%%%%%%%%%%%%%%%%%%%%%
\paragraph{v1.0:} 2017/04/27

\begin{itemize}
\item
manual and install package
\item
first version published on CTAN
\end{itemize}

%%%%%%%%%%%%%%%%%%%%%%%%%%%%%%%%%%%%%%%%
\paragraph{v0.6:} 2017/04/26

\begin{itemize}
\item
redirection mechanism added
\end{itemize}

%%%%%%%%%%%%%%%%%%%%%%%%%%%%%%%%%%%%%%%%
\paragraph{v0.5:} 2017/04/26

\begin{itemize}
\item
functionality in definition file
\end{itemize}


%%%%%%%%%%%%%%%%%%%%%%%%%%%%%%%%%%%%%%%%%%%%%%%%%%%%%%%%%%%%%%%%%%%%%%%%%%%%%%%%
%%%%%%%%%%%%%%%%%%%%%%%%%%%%%%%%%%%%%%%%%%%%%%%%%%%%%%%%%%%%%%%%%%%%%%%%%%%%%%%%
%%%%%%%%%%%%%%%%%%%%%%%%%%%%%%%%%%%%%%%%%%%%%%%%%%%%%%%%%%%%%%%%%%%%%%%%%%%%%%%%
\appendix

\settowidth\MacroIndent{\rmfamily\scriptsize 000\ }

 \DocInput{childdoc.dtx}

\end{document}
%</driver>
% \fi
%
% %%%%%%%%%%%%%%%%%%%%%%%%%%%%%%%%%%%%%%%%%%%%%%%%%%%%%%%%%%%%%%%%%%%%%%%%%%%%%%
% %%%%%%%%%%%%%%%%%%%%%%%%%%%%%%%%%%%%%%%%%%%%%%%%%%%%%%%%%%%%%%%%%%%%%%%%%%%%%%
% \section{Sample}
%\iffalse
%<*samplemain>
%\fi
%
% The following presents a sample document
% with two chapters, two parts, a title page,
% a compile flag as well as three forwarding files to set the flag.
% It consists of eight |.tex| files:
% \begin{center}
% \begin{tabular}{ll}
% |cdocsamp.tex|&main file\\
% |cdocsch1.tex|&include file for chapter 1\\
% |cdocsch2.tex|&include file for chapter 2\\
% |cdocspt3.tex|&include file for part 3\\
% |cdocspt4.tex|&include file for part 4\\
% |cdocsdrf.tex|&forwarding file for main file in draft mode\\
% |cdocsfi1.tex|&forwarding file for final version of chapter 1\\
% |cdocsfi2.tex|&forwarding file for final version of chapter 2\\
% \end{tabular}
% \end{center}
% Each of the eight files can be compiled directly by the \LaTeX{} compiler.
%
% %%%%%%%%%%%%%%%%%%%%%%%%%%%%%%%%%%%%%%
% \paragraph{Main File.}
%
% The main file is called |cdocsamp.tex|.
%
% Load the \textsf{childdoc} definitions and
% declare the filename for the main document:
%    \begin{macrocode}
\input{childdoc.def}
\childdocmain{}
%    \end{macrocode}

% Optional override for |\version| flag:
%    \begin{macrocode}
%%\ifchilddoc\else\providecommand{\version}{draft}\fi
%    \end{macrocode}

% Define the default values for the |\version| flag
% (|final| for the main file and |draft| for childs):
%    \begin{macrocode}
\ifchilddoc
\providecommand{\version}{draft}
\else
\providecommand{\version}{final}
\fi
%    \end{macrocode}

% Load the standard document class:
%    \begin{macrocode}
\documentclass[12pt]{article}
%    \end{macrocode}

% Start the document body:
%    \begin{macrocode}
\begin{document}
%    \end{macrocode}

% Declare a title page.
% Print title, part of document being processed and version flag:
%    \begin{macrocode}
\addtocounter{page}{-1}
\begin{center}
{\LARGE\bfseries{}childdoc example\par}
\vspace{1cm}
\ifchilddoc
\ifchilddocmanual part\else chapter\fi:
`\childdocname' of `\childdocjob'\par
\else
main document: `\childdocjob'\par
\fi
version: \version\par
\end{center}
\newpage
%    \end{macrocode}

% Manually include selected file,
% otherwise process as usual:
%    \begin{macrocode}
\ifchilddocmanual
\section*{part `\childdocname'}
\input{\childdocname}
\else
%    \end{macrocode}

% Include the two chapters:
%    \begin{macrocode}
\include{cdocsch1}
\include{cdocsch2}
%    \end{macrocode}

% Include the two parts unless only chapters should be displayed:
%    \begin{macrocode}
\ifchilddoc\else
\section{part three}
\input{cdocspt3}
\section{part four}
\input{cdocspt4}
\fi
%    \end{macrocode}

% Process as usual until here:
%    \begin{macrocode}
\fi
%    \end{macrocode}

% End of document body:
%    \begin{macrocode}
\end{document}
%    \end{macrocode}
%\iffalse
%</samplemain>
%\fi
%
% %%%%%%%%%%%%%%%%%%%%%%%%%%%%%%%%%%%%%%
% \paragraph{Chapter Include Files.}
%
% The include files are called |cdocsch1.tex| and |cdocsch2.tex|.
%
%\iffalse
%<*samplechap1|samplechap2>
%\fi

% Optional override for |\version| flag:
%    \begin{macrocode}
%%\providecommand{\version}{final}
%    \end{macrocode}

% Include the main document:
%    \begin{macrocode}
\input{childdoc.def}
\childdocof{cdocsamp}
%    \end{macrocode}

%\iffalse
%</samplechap1|samplechap2>
%\fi
%
%\iffalse
%<*samplechap1>
%\fi
% Some text for chapter 1:
%    \begin{macrocode}
\section{one}
some text in chapter one
%    \end{macrocode}

%\iffalse
%</samplechap1>
%\fi
% Some text for chapter 2:
%\iffalse
%<*samplechap2>
%\fi
%    \begin{macrocode}
\section{two}
more text in chapter two
%    \end{macrocode}

%\iffalse
%</samplechap2>
%\fi
%
% %%%%%%%%%%%%%%%%%%%%%%%%%%%%%%%%%%%%%%
% \paragraph{Part Include Files.}
%
% The include files are called |cdocspt3.tex| and |cdocspt4.tex|.
%
%\iffalse
%<*samplepart3|samplepart4>
%\fi

% Optional override for |\version| flag:
%    \begin{macrocode}
%%\providecommand{\version}{final}
%    \end{macrocode}

% Include the main document:
%    \begin{macrocode}
\input{childdoc.def}
\childdocby{cdocsamp}
%    \end{macrocode}

%\iffalse
%</samplepart3|samplepart4>
%\fi
%
%\iffalse
%<*samplepart3>
%\fi
% Some text for part 3:
%    \begin{macrocode}
some text in part three
%    \end{macrocode}

%\iffalse
%</samplepart3>
%\fi
% Some text for part 4:
%\iffalse
%<*samplepart4>
%\fi
%    \begin{macrocode}
more text in part four
%    \end{macrocode}

%\iffalse
%</samplepart4>
%\fi
%
% %%%%%%%%%%%%%%%%%%%%%%%%%%%%%%%%%%%%%%
% \paragraph{Forwarding for a Complete Draft.}
%
% The following forwarding file |cdocsdrf.tex|
% compiles the main document in draft mode:
%\iffalse
%<*sampledraft>
%\fi
%    \begin{macrocode}
\def\version{draft}
\input{childdoc.def}
\childdocforward{cdocsamp}
%    \end{macrocode}

%\iffalse
%</sampledraft>
%\fi
%
% %%%%%%%%%%%%%%%%%%%%%%%%%%%%%%%%%%%%%%
% \paragraph{Forwarding for Final Version of the Chapters.}
%
% The following forwarding files |cdocsfn1.tex| and |cdocsfn2.tex|
% (with identical content)
% compile the final versions of the child documents
% |cdocsch1.tex| and |cdocsch2.tex|, respectively:
%\iffalse
%<*samplefinal>
%\fi
%    \begin{macrocode}
\def\version{final}
\input{childdoc.def}
\childdocforwardprefix[cdocsamp]{cdocsfn}{cdocsch}
%    \end{macrocode}

%\iffalse
%</samplefinal>
%\fi
%
% %%%%%%%%%%%%%%%%%%%%%%%%%%%%%%%%%%%%%%
% \paragraph{Command Line Processing.}
%
% The following three command lines generate the output files
% |cdocscld|, |cdocscl1| and |cdocscl2|
% which should be identical to
% |cdocsdrf|, |cdocsch1| and |cdocsfn2|, respectively:
% \begin{center}
% \begin{tabular}{l}
% |latex -jobname cdocscld \|\\
% |  "\def\version{draft}\input{childdoc.def}\childdocforward{cdocsamp}"|\\
% |latex -jobname cdocscl1 \|\\
% |  "\input{childdoc.def}\childdocforward[cdocsamp]{cdocsch1}"|\\
% |latex -jobname cdocscl2 \|\\
% |  "\def\version{final}\input{childdoc.def}\childdocforward{cdocsch2}"|
% \end{tabular}
% \end{center}
% Note that the trailing backslash on each first line
% merely continues the input to the second line
% (for convenient cut ant paste).
% Furthermore, the command |latex| can be replaced by any
% of its alternative versions such as |pdflatex|.
%
% %%%%%%%%%%%%%%%%%%%%%%%%%%%%%%%%%%%%%%%%%%%%%%%%%%%%%%%%%%%%%%%%%%%%%%%%%%%%%%
% %%%%%%%%%%%%%%%%%%%%%%%%%%%%%%%%%%%%%%%%%%%%%%%%%%%%%%%%%%%%%%%%%%%%%%%%%%%%%%
% \section{Implementation}
%\iffalse
%<*package>
%\fi
%
% This section describes the definitions file |childdoc.def|.

% The definitions cannot be loaded using |\usepackage| or |\RequirePackage|
% which has a mechanism to prevent loading a style file more than once.
% When loading the definitions by means of |\input|
% multiple instances have to be prevented manually:
%\iffalse
%This code needs to be before the `\ProvidesFile' directive
%which is defined at the beginning of this file.
%Therefore it is also placed there and commented out here.
%</package>
%<*discard>
%\fi
%    \begin{macrocode}
\ifdefined\childdocmain\endinput\fi
%    \end{macrocode}
%\iffalse
%</discard>
%<*package>
%\fi
%
% \macro{\ifchilddoc}
% \macro{\ifchilddocmanual}
% The conditional |\ifchilddoc| tells whether a
% child (true) or main (false) document is being compiled.
% The conditional |\ifchilddocmanual| tells whether
% the |\includeonly| mechanism is used (false) or
% the selection of child files must be performed manually (true).
% The definitions initialise to false:
%    \begin{macrocode}
\newif\ifchilddoc
\newif\ifchilddocmanual
%    \end{macrocode}

% \macro{\childdocname}
% \macro{\childdocjob}
% The macro |\childdocname| stores the name of the main document
% to be compiled. The macro |\childdocjob| stores the name of
% the document on which the \LaTeX{} compiler was originally invoked.
% The content of |\jobname| cannot be compared
% to filenames specified in the source due to different catcodes.
% The following code rescans |\jobname|, stores the result
% in |\childdocname| and saves a copy in |\childdocjob|:
%    \begin{macrocode}
\edef\childdocname{\scantokens\expandafter{\jobname\noexpand}}
\let\childdocjob\childdocname
%    \end{macrocode}

% \macro{\childdocdisable}
% The macro |\childdocdisable| prevents the main file
% from being processed more than once.
% At this stage, the main document command |\childdocmain|
% is assumed to be called once again where it should do nothing.
% Any subsequent call to it should prevent
% a secondary processing of the main document
% It overwrites the forwarding commands
% |\childdocof| and |\childdocforward|
% with empty macros to prevent further inclusions of the main document:
%    \begin{macrocode}
\newcommand{\childdocdisable}
{
  \renewcommand{\childdocmain}[1]{\renewcommand{\childdocmain}[1]{\endinput}}
  \renewcommand{\childdocof}[1]{}
  \renewcommand{\childdocby}[2][]{}
  \renewcommand{\childdocforward}[2][]{}
  \renewcommand{\childdocdisable}{}
}
%    \end{macrocode}

% \macro{\childdocmain}
% The macro |\childdocmain| is to be called at the top of the main file
% with nothing or the main filename (without extension) as argument.
% First, it breaks loops.
% If the argument is not empty and does not match |\childdocname|
% (which is set by the first inclusion of |childdoc.def|),
% |\ifchilddoc| is set to true, |\includeonly| is applied to the child file
% and |\jobname| is set to the main file
% (for proper handling of |.aux| files):
%    \begin{macrocode}
\newcommand{\childdocmain}[1]
{
  \childdocdisable\childdocmain{}
  \if?#1?\else
    \begingroup
      \def\childdoctmp{#1}
      \ifx\childdoctmp\childdocname
        \def\childdoctmp{}
      \else
        \def\childdoctmp
        {
          \childdoctrue
          \includeonly{\childdocname}
          \def\childdocjob{#1}
          \def\jobname{#1}
        }
      \fi
      \expandafter
    \endgroup
    \childdoctmp
  \fi
}
%    \end{macrocode}

% \macro{\childdocof}
% The command |\childdocof| redirects
% compilation to the main file |#1|.
%    \begin{macrocode}
\newcommand{\childdocof}[1]
{
  \childdocdisable
  \childdoctrue
  \includeonly{\childdocname}
  \def\jobname{#1}
  \def\childdocjob{#1}
  \input{#1}
}
%    \end{macrocode}

% \macro{\childdocby}
% The command |\childdocby| ....
%    \begin{macrocode}
\newcommand{\childdocby}[2][]
{
  \childdocdisable
  \childdoctrue
  \childdocmanualtrue
  \if?#1?\else
    \def\jobname{#2}
  \fi
  \def\childdocjob{#2}
  \input{#2}
  \endinput
}
%    \end{macrocode}

% \macro{\childdocforward}
% The command |\childdocforward| redirects
% compilation to the main file or
% (if the optional argument is given) a child file.
% Parameters are set as if the main file
% or a child file starting with |\childdocof| was compiled.
% Then compilation is handed over to the main file:
%    \begin{macrocode}
\newcommand{\childdocforward}[2][]
{
  \begingroup
    \if?#1?
      \def\childdoctmp
      {
        \def\childdocname{#2}
        \def\childdocjob{#2}
        \def\jobname{#2}
        \input{#2}
        \endinput
      }
    \else
      \def\childdoctmp
      {
        \childdocdisable
        \def\childdocname{#2}
        \childdoctrue
        \includeonly{#2}
        \def\childdocjob{#1}
        \def\jobname{#1}
        \input{#1}
        \endinput
      }
    \fi
    \expandafter
  \endgroup
  \childdoctmp
}
%    \end{macrocode}

% \macro{\childdocforwardprefix}
% The command |\childdocforwardprefix| redirects
% compilation to the main or a child file by means of a pattern.
% The prefix |#1| in the current filename is replaced by |#2|
% and the suffix of the current filename is kept
% (it is assumed that the filename does not contain the substring `|~~~|'
% which is used as a delimiter).
% Compilation is handed over to the new file by |\childdocforward|:
%    \begin{macrocode}
\newcommand{\childdocforwardprefix}[3][]
{
  \begingroup
    \def\childdocextract #2##1~~~{\def\childdoctmp{\childdocforward[#1]{#3##1}}}
    \expandafter\childdocextract\childdocname~~~
    \expandafter
  \endgroup
  \childdoctmp
}
%    \end{macrocode}

% \macro{\childdoc}
% The deprecated macro |\childdoc| is a legacy version of |\childdocmain|:
%    \begin{macrocode}
\newcommand{\childdoc}{\childdocmain}
%    \end{macrocode}

% \macro{\childdocredirect}
% The deprecated macro |\childdocredirect| is a legacy version
% of |\childdocforward| and |\childdocforwardprefix|:
%    \begin{macrocode}
\newcommand{\childdocredirect}[2][]
{
  \begingroup
    \if?#1?
      \def\childdoctmp{\childdocforward{#2}}
    \else
      \def\childdoctmp{\childdocforwardprefix{#1}{#2}}
    \fi
    \expandafter
  \endgroup
  \childdoctmp
}
%    \end{macrocode}

%\iffalse
%</package>
%\fi
%
\endinput
|\\
|\childdocforward[|\textit{main}|]{|\textit{dest}|}|\\
\end{tabular}
\end{center}
%
The argument \textit{dest} is the destination file
(without extension).
It should be the main file or one of the child files.
Note that further \textsf{childdoc} directives
such as |\childdocof| and |\childdocforward|
in the indicated file will be processed in this form.
The optional argument \textit{main}
passes on directly to the main file \textit{main}
while pretending to compile the child \textit{dest}.
This form behaves as if \textit{dest}
issues |\childdocof{|\textit{main}|}| right away,
and no further \textsf{childdoc} directives will be processed.

%%%%%%%%%%%%%%%%%%%%%%%%%%%%%%%%%%%%%%%%
\DescribeMacro{\...prefix}
In the alternative form |\childdocforwardprefix|,
%
\begin{center}
\begin{tabular}{l}
|% \iffalse
%
% childdoc.dtx Copyright (C) 2017-2018 Niklas Beisert
%
% This work may be distributed and/or modified under the
% conditions of the LaTeX Project Public License, either version 1.3
% of this license or (at your option) any later version.
% The latest version of this license is in
%   http://www.latex-project.org/lppl.txt
% and version 1.3 or later is part of all distributions of LaTeX
% version 2005/12/01 or later.
%
% This work has the LPPL maintenance status `maintained'.
%
% The Current Maintainer of this work is Niklas Beisert.
%
% This work consists of the files childdoc.dtx and childdoc.ins
% and the derived files childdoc.def and cdocsamp.tex with
% cdocsch1.tex, cdocsch2.tex, cdocsdrf.tex, cdocsfn1.tex, cdocsfn2.tex.
%
%<package>\ifdefined\childdocmain\endinput\fi
%<package>\ProvidesFile{childdoc.def}[2018/12/30 v2.0 child document driver]
%<samplemain>\ProvidesFile{cdocsamp.tex}[2018/12/30 v2.0 sample for childdoc]
%<*driver>
%\ProvidesFile{childdoc.drv}[2018/12/30 v2.0 childdoc reference manual file]
\PassOptionsToClass{10pt,a4paper}{article}
\documentclass{ltxdoc}

\usepackage[margin=35mm]{geometry}
\usepackage{hyperref}
\usepackage{hyperxmp}
\usepackage[usenames]{color}

\hypersetup{colorlinks=true}
\hypersetup{pdfstartview=FitH}
\hypersetup{pdfpagemode=UseNone}
\hypersetup{pdfsource={}}
\hypersetup{pdflang={en-UK}}
\hypersetup{pdfcopyright={Copyright 2017-2018 Niklas Beisert.
  This work may be distributed and/or modified under the
  conditions of the LaTeX Project Public License, either version 1.3
  of this license or (at your option) any later version.}}
\hypersetup{pdflicenseurl={http://www.latex-project.org/lppl.txt}}
\hypersetup{pdfcontactaddress={ETH Zurich, ITP, HIT K,
  Wolfgang-Pauli-Strasse 27}}
\hypersetup{pdfcontactpostcode={8093}}
\hypersetup{pdfcontactcity={Zurich}}
\hypersetup{pdfcontactcountry={Switzerland}}
\hypersetup{pdfcontactemail={nbeisert@itp.phys.ethz.ch}}
\hypersetup{pdfcontacturl={http://people.phys.ethz.ch/\xmptilde nbeisert/}}

\newcommand{\secref}[1]{\hyperref[#1]{section \ref*{#1}}}

\parskip1ex
\parindent0pt
\let\olditemize\itemize
\def\itemize{\olditemize\parskip0pt}

\begin{document}

\title{The \textsf{childdoc} Package}
\hypersetup{pdftitle={The childdoc Package}}
\author{Niklas Beisert\\[2ex]
  Institut f\"ur Theoretische Physik\\
  Eidgen\"ossische Technische Hochschule Z\"urich\\
  Wolfgang-Pauli-Strasse 27, 8093 Z\"urich, Switzerland\\[1ex]
  \href{mailto:nbeisert@itp.phys.ethz.ch}
  {\texttt{nbeisert@itp.phys.ethz.ch}}}
\hypersetup{pdfauthor={Niklas Beisert}}
\hypersetup{pdfsubject={Manual for the LaTeX2e Package childdoc}}
\date{30 December 2018, \textsf{v2.0}}
\maketitle

\begin{abstract}\noindent
\textsf{childdoc} is a \LaTeXe{} package
that enables the direct compilation
of document sections included by |\include|
to individual files.
\end{abstract}

\begingroup
\parskip0ex
\tableofcontents
\endgroup

%%%%%%%%%%%%%%%%%%%%%%%%%%%%%%%%%%%%%%%%%%%%%%%%%%%%%%%%%%%%%%%%%%%%%%%%%%%%%%%%
%%%%%%%%%%%%%%%%%%%%%%%%%%%%%%%%%%%%%%%%%%%%%%%%%%%%%%%%%%%%%%%%%%%%%%%%%%%%%%%%
\section{Introduction}

\LaTeX{} provides a mechanism to structure a large document (such as a book)
into a main file and several child files (containing the chapters)
using the |\include| command.
This mechanism is beneficial for documents
which span hundreds of pages in order to
make the source file(s) more manageable.
Moreover, compilation can be restricted to
selected child files by means of the |\includeonly| command.
The latter feature can be used to reduce the compilation time while editing
(this was significantly more useful in the earlier days of \LaTeX{})
or to generate a smaller document which is easier to navigate.
Another application of |\includeonly| is to generate
documents consisting of selected parts of the complete document.

However, there are a few drawbacks of the plain |\include| mechanism:
\begin{itemize}
\item
The child files cannot be compiled on their own,
they can only be compiled via the main file.
A naive editing environment
(such as a text editor with an option
to have the current file processed by \LaTeX)
may require one to switch to the main file before compiling;
attempting to compile the child file produces errors.
\item
The main file must be modified (each time)
to adjust the |\includeonly| command
to the present needs. This easily leaves the main file in a messy state.
\item
The generated document will always carry the filename
of the main document. This is inconvenient if
several child files are to be compiled and
to be kept for distribution.
\end{itemize}

The present package provides a simple interface
to make child files individually compilable by \LaTeX{}.
Compiling a child file then has the same effect as compiling
the main file with an |\includeonly| command
to select the appropriate child.
Moreover the generated document will carry the name of the child
rather than the main file.
This resolves all three above issues.

This feature is meant to make the editing of books,
thesis documents and lecture notes somewhat more convenient.
However, the package can also be used efficiently for
composing a series of documents (such as exercise sheets)
which are typically distributed individually.
It then assists the author in generating the individual documents
(potentially in different versions)
as well as a document containing the collected series.
Another application is in developing style files
or other kinds of included material
where compilation of the style file could redirect
to a sample or test file.

%%%%%%%%%%%%%%%%%%%%%%%%%%%%%%%%%%%%%%%%%%%%%%%%%%%%%%%%%%%%%%%%%%%%%%%%%%%%%%%%
%%%%%%%%%%%%%%%%%%%%%%%%%%%%%%%%%%%%%%%%%%%%%%%%%%%%%%%%%%%%%%%%%%%%%%%%%%%%%%%%
\section{Usage}

First of all, the package \textsf{childdoc} is \emph{not} a standard
\LaTeXe{} |.sty| style file! Therefore it needs to be invoked in
a non-standard way.

%%%%%%%%%%%%%%%%%%%%%%%%%%%%%%%%%%%%%%%%%%%%%%%%%%%%%%%%%%%%%%%%%%%%%%%%%%%%%%%%
\subsection{Included Files}
\label{sec:include}

%%%%%%%%%%%%%%%%%%%%%%%%%%%%%%%%%%%%%%%%
\DescribeMacro{\childdocmain}
To use the package, add the commands
\begin{center}
\begin{tabular}{l}
|\input{childdoc.def}|\\
|\childdocmain{}|\\
\end{tabular}
\end{center}
at the very top of the main \LaTeX{} file,
in particular \emph{before} the |\documentclass| statement!
The argument of |\childdocmain| should be left empty
(but it must be present).

%%%%%%%%%%%%%%%%%%%%%%%%%%%%%%%%%%%%%%%%
\DescribeMacro{\childdocof}
Furthermore, add the commands
\begin{center}
\begin{tabular}{l}
|\input{childdoc.def}|\\
|\childdocof{|\textit{main}|}|\\
\end{tabular}
\end{center}
at the top of every child file \textit{child}
which is included by |\include{|\textit{child}|}|
from within the main file
(or at least for those files to be compiled individually).
The argument \textit{main} must be the filename of the main file.

There are a couple of
considerations in setting up the main and child documents:

%%%%%%%%%%%%%%%%%%%%%%%%%%%%%%%%%%%%%%%%
\paragraph{Restrictions.}

Please note the following restrictions:
\begin{itemize}
\item
|\childdocmain| must be called with one argument \textit{main}
to ensure compatibility with earlier version of the package.
It must either be empty (|\childdocmain{}|)
or precisely match the filename of the main file in which it is specified.
See \secref{sec:detection} for further information.
\item
The filename \textit{main} must be specified without the |.tex| extension.
\item
The filename \textit{main} is case sensitive
(even in case-insensitive file systems)
due to internal string comparison.
\item
The argument \textit{main} should be fully expanded, it cannot be a macro.
\item
Subdirectories and special characters should be avoided in filenames.
\item
The command |\childdocmain{|\textit{main}|}| must be followed by a whitespace.
It should not be followed immediately by another command
or by a comment mark `|%|'.
This is because the \TeX{} parser reads the token immediately following
the argument of |\childdocmain| and puts it
at the beginning of every child section;
however, a white\-space is ignored.
\end{itemize}

%%%%%%%%%%%%%%%%%%%%%%%%%%%%%%%%%%%%%%%%
\paragraph{Content of Main File.}

It is advisable to place all content in the child files included by |\include|.
Any output contained in the main file will appear in all child documents
unless suppressed manually;
it cannot be suppressed automatically by the |\includeonly| directive
and thus should normally be avoided.
A method to include some content in the main file
by means of conditional processing is described in \secref{sec:conditional}.

%%%%%%%%%%%%%%%%%%%%%%%%%%%%%%%%%%%%%%%%
\paragraph{Page Numbering.}

When only a part of the document is compiled,
the appropriate numbering of pages
(as well as other status parameters)
is determined from the |.aux| files.
The latter contain information from previous passes.
However this information needs to propagate through
all intermediate child documents.
Therefore the page numbering in child documents may well
be inconsistent until the complete document is compiled at least once.

A useful (if unconventional) way to always ensure a consistent
page numbering is to restart the numbering in each child document
and denote the pages by `\textit{child}|.|\textit{page}'
where \textit{child} represents the chapter/section number of the child file.
This can be achieved by the command
|\numberwithin{page}{|\textit{child}|}|
of the \textsf{amsmath} package
where \textit{child} can be |chapter| or |section|
depending on the chosen structuring.
Alternatively, one can modify the macro |\thepage| appropriately
and reset the counter |page| at the start of each child file.

%%%%%%%%%%%%%%%%%%%%%%%%%%%%%%%%%%%%%%%%%%%%%%%%%%%%%%%%%%%%%%%%%%%%%%%%%%%%%%%%
\subsection{Conditional Processing}
\label{sec:conditional}

The package provides a mechanism to compile different versions
of a document. To customise the versions further some conditional processing
can come in handy to distinguish which version is being compiled.
The package provides two macros to describe the compilation context:

%%%%%%%%%%%%%%%%%%%%%%%%%%%%%%%%%%%%%%%%
\DescribeMacro{\ifchilddoc}
The conditional |\ifchilddoc| distinguishes between the compilation of
child documents and the main document:
%
\begin{center}
|\ifchilddoc |\textit{child-code}| |[|\||else |\textit{main-code}]| \||fi|
\end{center}

%%%%%%%%%%%%%%%%%%%%%%%%%%%%%%%%%%%%%%%%
\DescribeMacro{\childdocname}
\DescribeMacro{\childdocjob}
The macro |\childdocname| contains the filename (without extension)
of the main or child file being processed.
Note that |\childdocjob| will always contain the name of the main file.

%%%%%%%%%%%%%%%%%%%%%%%%%%%%%%%%%%%%%%%%
\paragraph{Title Page.}

Conditional processing can be used to include a title or banner page
in the main document when proper precautions are taken.
Importantly, the code in the main file should ensure that the page counter
(as well as other status parameters which are stored in the |.aux| files)
takes the same value after the conditional processing.
Otherwise the page numbers may take divergent values
depending on which part is compiled.

For example, a title page could be declared by:
%
\begin{center}
\begin{tabular}{l}
|\ifchilddoc\||else|\\
|\addtocounter{page}{-1}|\\
\textit{code for title page}\\
|\newpage|\\
|\||fi|
\end{tabular}
\end{center}
%
A banner page for the child documents can be generated by:
%
\begin{center}
\begin{tabular}{l}
|\ifchilddoc|\\
|\addtocounter{page}{-1}|\\
\textit{code for banner page}\\
|\newpage|\\
|\||fi|
\end{tabular}
\end{center}
%
Here one could write a message such as:
\begin{center}
|This is the part \childdocname{} of \childdocjob{}.|
\end{center}

%%%%%%%%%%%%%%%%%%%%%%%%%%%%%%%%%%%%%%%%%%%%%%%%%%%%%%%%%%%%%%%%%%%%%%%%%%%%%%%%
\subsection{Flags}
\label{sec:flags}

The package makes it easy to generate different versions
of the main or child documents.
To this end compilation flags can be defined
and assigned different default values.
They will be particularly useful in conjunction
with the forwarding mechanism described in \secref{sec:forward}.

For example, it may be useful to have a flag |\version|
which can be set to |draft| or |final|.
The document source will contain some conditional code
depending on the value of |\version|.
Suppose further, the flag should default to |final| for the main file
and to |draft| for child files
which is a natural assignment for editing the document.
This is achieved by placing the following code
in the preamble of the main document
(below the |\childdocmain| directive):
%
\begin{center}
\begin{tabular}{l}
|\ifchilddoc|\\
|\providecommand{\version}{draft}|\\
|\||else|\\
|\providecommand{\version}{final}|\\
|\||fi|
\end{tabular}
\end{center}
%
The definition by |\providecommand| makes sure
that previous definitions are not overwritten.
Further statements |\providecommand{\version}{...}|
can thus be added before the above code to override it.

For the main file, one might add a line
(between |\childdocmain| and the above block)
%
\begin{center}
|%\ifchilddoc\||else\providecommand{\version}{draft}\||fi|
\end{center}
%
which can be uncommented to produce a draft version.
Likewise one can add a line to the very top of a child file
(above the |\childdocof{|\textit{main}|}| directive)
%
\begin{center}
|%\providecommand{\version}{final}|
\end{center}
%
which can be uncommented to produce the final version of this child document.

%%%%%%%%%%%%%%%%%%%%%%%%%%%%%%%%%%%%%%%%%%%%%%%%%%%%%%%%%%%%%%%%%%%%%%%%%%%%%%%%
\subsection{Forwarding}
\label{sec:forward}

Different versions of the main or child documents
using compilation flags as described in \secref{sec:flags}
can be (permanently) stored in different files
for convenient compilation, viewing and distribution.
To this end, the package defines a command
to pass on compilation to a different file:

%%%%%%%%%%%%%%%%%%%%%%%%%%%%%%%%%%%%%%%%
\DescribeMacro{\childdocforward}
The command |\childdocforward| redirects processing to
another source file:
%
\begin{center}
\begin{tabular}{l}
|\input{childdoc.def}|\\
|\childdocforward[|\textit{main}|]{|\textit{dest}|}|\\
\end{tabular}
\end{center}
%
The argument \textit{dest} is the destination file
(without extension).
It should be the main file or one of the child files.
Note that further \textsf{childdoc} directives
such as |\childdocof| and |\childdocforward|
in the indicated file will be processed in this form.
The optional argument \textit{main}
passes on directly to the main file \textit{main}
while pretending to compile the child \textit{dest}.
This form behaves as if \textit{dest}
issues |\childdocof{|\textit{main}|}| right away,
and no further \textsf{childdoc} directives will be processed.

%%%%%%%%%%%%%%%%%%%%%%%%%%%%%%%%%%%%%%%%
\DescribeMacro{\...prefix}
In the alternative form |\childdocforwardprefix|,
%
\begin{center}
\begin{tabular}{l}
|\input{childdoc.def}|\\
|\childdocforwardprefix[|\textit{main}|]{|\textit{prefix}|}{|\textit{dest}|}|
\end{tabular}
\end{center}
%
the destination file is determined by a pattern
depending on the current file:
To make this work, the current file must be called
`{\textit{prefix}\hspace{0.2em}\textit{suffix}}'
with \textit{prefix} matching precisely the argument.
Processing is then passed on to the file
`{\textit{dest}\hspace{0.2em}\textit{suffix}}'.
Surely, the same effect is achieved by
directly specifying the
argument `{\textit{dest}\hspace{0.2em}\textit{suffix}}'
in the first form.
However, that requires to set up a different file
for each child. With the alternative form of the command
all these files can have exactly the same content
which simplifies setting them up and maintaining them.

For example, the following file |draft.tex|
with a compilation flag |\version| as described in \secref{sec:flags}
compiles the main document as a draft:
%
\begin{center}
\begin{tabular}{l}
|\def\version{draft}|\\
|\input{childdoc.def}|\\
|\childdocforward{|\textit{main}|}|
\end{tabular}
\end{center}
%
Likewise, the following files |final|\textit{nn}|.tex|
compile the final version of the child document
|child|\textit{nn}|.tex|:
%
\begin{center}
\begin{tabular}{l}
|\def\version{final}|\\
|\input{childdoc.def}|\\
|\childdocforwardprefix{final}{child}|
\end{tabular}
\end{center}
%

Note that when several versions of a main file and/or of each child file
are to be generated, it may be convenient to set up a |Makefile| or
shell script to automatise the process.

%%%%%%%%%%%%%%%%%%%%%%%%%%%%%%%%%%%%%%%%%%%%%%%%%%%%%%%%%%%%%%%%%%%%%%%%%%%%%%%%
\subsection{Command Line Processing}
\label{sec:commandline}

The effect of redirection files can also be achieved by invoking
the \LaTeX{} compiler with a more elaborate command line.
Most conveniently this should be done as part
of a shell script or a |Makefile|.

When using \textsf{childdoc} in the main file, the following
command lines effectively perform a redirection
(note that depending on the shell being used,
backslashes may have to be doubled: `|\|' $\to$ `|\\|'):
%
\begin{center}
|... -jobname "|\textit{target}|" |\\|"|[\textit{flags}]%
|\input{childdoc.def}\childdocforward[|\textit{main}|]{|\textit{dest}|}"|
\end{center}
%
Here \textit{target} is the name of the output file,
\textit{main} is the name of the main file
and \textit{dest} is the name of the main or child file to be processed
(all filenames without extensions).
The optional argument \textit{main} can be omitted
if \textit{main} matches \textit{dest}.
Optionally, compilation \textit{flags} can be defined via |\def| commands.
This command line makes the \TeX{} engine believe
it is compiling the file \textit{target}
whose content is specified as the latter parameter.
The provided code then forwards the processing to
\textit{main} or \textit{dest} as described in \secref{sec:forward}.

%%%%%%%%%%%%%%%%%%%%%%%%%%%%%%%%%%%%%%%%%%%%%%%%%%%%%%%%%%%%%%%%%%%%%%%%%%%%%%%%
\subsection{Include by Input}
\label{sec:input}

Including child documents by |\include| has some restrictions by design.
Most notably, the content of a child document always occupies
its own set of pages; pages cannot be shared between child documents.
Usually, this behaviour makes perfect sense
because each child document contain an essential part of the document.
However, in some situations it may be desirable to compose
a document from a collection of parts
without having mandatory page breaks between then.
For this case, the package
provides a mechanism to include parts
by |\input| which can also be processed individually.
However, by construction this mechanism
requires manual handling of the content to be output.

%%%%%%%%%%%%%%%%%%%%%%%%%%%%%%%%%%%%%%%%
\DescribeMacro{\ifchilddocmanual}
The main file should be prepared as usual, see \secref{sec:include}.
However, the document body must make a distinction
between processing of an individual part and of the main document, e.g.:
%
\begin{center}
\begin{tabular}{l}
|\ifchilddocmanual|\\
|\input{\childdocname}|\\
|\||else|\\
\textit{document body with }|\input{|\textit{part}|}|\\
|\||fi|
\end{tabular}
\end{center}
%
The conditional |\ifchilddocmanual| is true whenever
a part to be included by |\input| is being compiled,
and the name of the part is stored in |\childdocname|.

%%%%%%%%%%%%%%%%%%%%%%%%%%%%%%%%%%%%%%%%
\DescribeMacro{\childdocby}
Each part to be included by |\input| should start with:
%
\begin{center}
\begin{tabular}{l}
|\input{childdoc.def}|\\
|\childdocby{|\textit{main}|}|\\
\end{tabular}
\end{center}
%
The directive |\childdocby| is similar to |\childdocof|
described in \secref{sec:include},
but the subsequent selection of content must be done manually.
To that end, both |\ifchilddoc| and |\ifchilddocmanual|
will be true upon processing of a part,
and the name of the part is stored in |\childdocname|.
Note that |\jobname| will be set to the filename of the current part
so that each part receives an individual |.aux| file
that does not interfere with the |.aux| file(s) of the main document.
This behaviour can be altered by the alternative form
|\childdocby[*]{|\textit{main}|}| (with a non-empty optional argument)
which uses the |.aux| file of the main document
by setting |\jobname| to \textit{main}.

%%%%%%%%%%%%%%%%%%%%%%%%%%%%%%%%%%%%%%%%%%%%%%%%%%%%%%%%%%%%%%%%%%%%%%%%%%%%%%%%
\subsection{Driver Development}
\label{sec:driver}

The \textsf{childdoc} mechanism can also be use for the development
of definition files such as \LaTeX{} styles or classes.
This case differs from the above setup with multiple parts
included by |\include| in that no |\includeonly| should be invoked.
This can be achieved by starting the include file
(before |\ProvidesPackage|) with:
%
\begin{center}
\begin{tabular}{l}
|\input{childdoc.def}|\\
|\childdocforward{|\textit{main}|}|\\
\end{tabular}
\end{center}
%
or alternatively with:
%
\begin{center}
\begin{tabular}{l}
|\input{childdoc.def}|\\
|\childdocby{|\textit{main}|}|\\
\end{tabular}
\end{center}
%
Both forms have slightly different effects as described above.
The main file is prepared as usual, see \secref{sec:include}.

%%%%%%%%%%%%%%%%%%%%%%%%%%%%%%%%%%%%%%%%%%%%%%%%%%%%%%%%%%%%%%%%%%%%%%%%%%%%%%%%
\subsection{Legacy Detection}
\label{sec:detection}

The directive |\childdocmain| in the main file can detect
whether the complete document or merely a child is to be compiled
even without using the directive |\childdocof|.
This method is deprecated because it is less robust
and there is no compelling reason to use it;
it is merely provided for backward compatibility
and it may be removed in future versions.

If the detection mechanism is to be used,
it is mandatory to correctly specify
the filename of the main file as the argument of |\childdocmain|:
%
\begin{center}
\begin{tabular}{l}
|\input{childdoc.def}|\\
|\childdocmain{|\textit{main}|}|\\
\end{tabular}
\end{center}
%
If |\jobname| does not match the argument \textit{main} of |\childdocmain|,
it is assumed that |\jobname| points to the child file to be compiled.
When using |\childdocmain| with the main file specified as argument,
it suffices to start a child file
with just |\input{|\textit{main}|}|
without loading of the package and using |\childdocof|.
If instead all processing is done
with the appropriate \textsf{childdoc} directives,
the argument of \textit{main} of |\childdocmain| can be empty.

An alternative version of the command line processing described
in \secref{sec:commandline} using the detection mechanism reads:
%
\begin{center}
|... -jobname "|\textit{target}|" "|[\textit{flags}]%
[|\def\jobname{|\textit{dest}|}|]|\input{|\textit{main}|}"|
\end{center}

%%%%%%%%%%%%%%%%%%%%%%%%%%%%%%%%%%%%%%%%%%%%%%%%%%%%%%%%%%%%%%%%%%%%%%%%%%%%%%%%
\subsection{Manual Code}
\label{sec:manual}

In case one cannot be certain whether the definitions file |childdoc.def|
is installed on the target \TeX{} distribution
and one prefers not to ship it,
it is conceivable to paste a few relevant commands into the sources.

To that end, drop all statements |\input{childdoc.def}|
and perform the replacements as outlined below.
Instead of |\childdocmain{|\textit{main}|}| add the following code
to the top of the main file:
%
\begin{center}
\begin{tabular}{l}
|\||ifdefined\childdocname\endinput\||fi\newif\ifchilddoc|\\
|\edef\childdocname{\scantokens\expandafter{\jobname\noexpand}}|\\
|\def\childdocmain{|\textit{main}|}\||ifx\childdocmain\childdocname\||else|\\
|\childdoctrue\includeonly{\childdocname}\let\jobname\childdocmain\||fi|\\
\end{tabular}
\end{center}
%
Instead of |\childdocof{|\textit{main}|}| just include the main file
at the top of each child file:
%
\begin{center}
|\input{|\textit{main}|}|
\end{center}
%
A simple redirection |\childdocforward{|\textit{dest}|}| is achieved by:
%
\begin{center}
|\def\jobname{|\textit{dest}|}\input{\jobname}|
\end{center}
%
The redirection with prefix
|\childdocforwardprefix[|\textit{prefix}|]{|\textit{dest}|}|
is accomplished by:
%
\begin{center}
\begin{tabular}{l}
|{\edef\jobname{\scantokens\expandafter{\jobname\noexpand}}|\\
|\def\redirectjob |\textit{prefix}|#1~~~{\gdef\jobname{|\textit{dest}|#1}}|\\
|\expandafter\redirectjob\jobname~~~}\input{\jobname}|
\end{tabular}
\end{center}

In an alternative approach,
child documents can be compiled by a specific command line
without additional code or specific definitions:
%
\begin{center}
|... -jobname "|\textit{target}|" "|[\textit{flags}]%
|\includeonly{|\textit{dest}|}\input{|\textit{main}|}"|
\end{center}
%

%%%%%%%%%%%%%%%%%%%%%%%%%%%%%%%%%%%%%%%%%%%%%%%%%%%%%%%%%%%%%%%%%%%%%%%%%%%%%%%%
%%%%%%%%%%%%%%%%%%%%%%%%%%%%%%%%%%%%%%%%%%%%%%%%%%%%%%%%%%%%%%%%%%%%%%%%%%%%%%%%
\section{Information}

%%%%%%%%%%%%%%%%%%%%%%%%%%%%%%%%%%%%%%%%%%%%%%%%%%%%%%%%%%%%%%%%%%%%%%%%%%%%%%%%
\subsection{Copyright}

Copyright \copyright{} 2017--2018 Niklas Beisert

This work may be distributed and/or modified under the
conditions of the \LaTeX{} Project Public License, either version 1.3
of this license or (at your option) any later version.
The latest version of this license is in
  \url{http://www.latex-project.org/lppl.txt}
and version 1.3 or later is part of all distributions of \LaTeX{}
version 2005/12/01 or later.

This work has the LPPL maintenance status `maintained'.

The Current Maintainer of this work is Niklas Beisert.

This work consists of the files |README.txt|, |childdoc.ins| and |childdoc.dtx|
as well as the derived files |childdoc.def|, |cdocsamp.tex|
with |cdocsch1.tex|, |cdocsch2.tex|, |cdocspt3.tex|, |cdocspt4.tex|,
|cdocsdrf.tex|, |cdocsfn1.tex|, |cdocsfn2.tex|
as well as |childdoc.pdf|.

%%%%%%%%%%%%%%%%%%%%%%%%%%%%%%%%%%%%%%%%%%%%%%%%%%%%%%%%%%%%%%%%%%%%%%%%%%%%%%%%
\subsection{Files and Installation}

The package consists of the files:
%
\begin{center}
\begin{tabular}{ll}
    |README.txt|   & readme file \\
    |childdoc.ins| & installation file \\
    |childdoc.dtx| & source file \\
    |childdoc.def| & definition file \\
    |cdocsamp.tex| & sample main file \\
    |cdocsch1.tex| & sample include file \\
    |cdocsch2.tex| & sample include file \\
    |cdocspt3.tex| & sample part file \\
    |cdocspt4.tex| & sample part file \\
    |cdocsdrf.tex| & sample redirection file \\
    |cdocsfn1.tex| & sample redirection file \\
    |cdocsfn2.tex| & sample redirection file \\
    |childdoc.pdf| & manual
\end{tabular}
\end{center}
%
The distribution consists of the files
|README.txt|, |childdoc.ins| and |childdoc.dtx|.
%
\begin{itemize}
\item
Run (pdf)\LaTeX{} on |childdoc.dtx|
to compile the manual |childdoc.pdf| (this file).
\item
Run \LaTeX{} on |childdoc.ins| to create the definitions file |childdoc.def|
and the sample |cdocsamp.tex| with include files
|cdocsch1.tex|, |cdocsch2.tex|, |cdocspt3.tex|, |cdocspt4.tex|,
|cdocsdrf.tex|, |cdocsfn1.tex|, |cdocsfn2.tex|.
Then copy the file |childdoc.def| to an appropriate directory of your \LaTeX{}
distribution, e.g.\ \textit{texmf-root}|/tex/latex/childdoc|.
\end{itemize}

%%%%%%%%%%%%%%%%%%%%%%%%%%%%%%%%%%%%%%%%%%%%%%%%%%%%%%%%%%%%%%%%%%%%%%%%%%%%%%%%
\subsection{Related CTAN Packages}

There are several other packages which offer a similar functionality:
%
\begin{itemize}
\item
The packages
\href{http://ctan.org/pkg/docmute}{\textsf{docmute}},
\href{http://ctan.org/pkg/includex}{\textsf{includex}} and
\href{http://ctan.org/pkg/standalone}{\textsf{standalone}}
provide commands to include only the document body of
a child file thus allowing both files to be compiled individually.
\item
The packages \href{http://ctan.org/pkg/subdocs}{\textsf{subdocs}}
and \href{http://ctan.org/pkg/subfiles}{\textsf{subfiles}}
provide structures in which the main and child documents can be
encapsulated and allowing them to be compiled individually.
The inclusion mechanism is different from the conventional |\include|.
\item
The package \href{http://ctan.org/pkg/combine}{\textsf{combine}}
is an elaborate solution to combine several documents into one.
\end{itemize}
%
See also the CTAN topic \href{http://ctan.org/topic/subdocs}{\textsf{subdocs}}
for further related packages.
The present package differs from the above solutions in that
a document structure constructed with the conventional |\include| mechanism
just needs two extra commands at the top of every file
such that all constituent files can be compiled individually.

%%%%%%%%%%%%%%%%%%%%%%%%%%%%%%%%%%%%%%%%%%%%%%%%%%%%%%%%%%%%%%%%%%%%%%%%%%%%%%%%
%\subsection{Feature Suggestions}
%
%The following is a list of features which may be useful for future
%versions of this package:
%%
%\begin{itemize}
%\item
%\ldots
%\end{itemize}

%%%%%%%%%%%%%%%%%%%%%%%%%%%%%%%%%%%%%%%%%%%%%%%%%%%%%%%%%%%%%%%%%%%%%%%%%%%%%%%%
\subsection{Revision History}

%%%%%%%%%%%%%%%%%%%%%%%%%%%%%%%%%%%%%%%%
\paragraph{v2.0:} 2018/12/30

\begin{itemize}
\item
immediate forward processing
\item
added |\childdocby| mechanism
\item
manual restructured
\end{itemize}

%%%%%%%%%%%%%%%%%%%%%%%%%%%%%%%%%%%%%%%%
\paragraph{v1.6:} 2018/01/17

\begin{itemize}
\item
application for development of include files
\item
corrections to manual
\end{itemize}

%%%%%%%%%%%%%%%%%%%%%%%%%%%%%%%%%%%%%%%%
\paragraph{v1.5:} 2017/05/21

\begin{itemize}
\item
more complete structuring introduced
\item
|\childdocof| introduced
\item
|\childdoc| renamed to |\childdocmain|
\item
|\childredirect| renamed to |\childdocforward| and |\childdocforwardprefix|
and functionality expanded
\end{itemize}

%%%%%%%%%%%%%%%%%%%%%%%%%%%%%%%%%%%%%%%%
\paragraph{v1.0:} 2017/04/27

\begin{itemize}
\item
manual and install package
\item
first version published on CTAN
\end{itemize}

%%%%%%%%%%%%%%%%%%%%%%%%%%%%%%%%%%%%%%%%
\paragraph{v0.6:} 2017/04/26

\begin{itemize}
\item
redirection mechanism added
\end{itemize}

%%%%%%%%%%%%%%%%%%%%%%%%%%%%%%%%%%%%%%%%
\paragraph{v0.5:} 2017/04/26

\begin{itemize}
\item
functionality in definition file
\end{itemize}


%%%%%%%%%%%%%%%%%%%%%%%%%%%%%%%%%%%%%%%%%%%%%%%%%%%%%%%%%%%%%%%%%%%%%%%%%%%%%%%%
%%%%%%%%%%%%%%%%%%%%%%%%%%%%%%%%%%%%%%%%%%%%%%%%%%%%%%%%%%%%%%%%%%%%%%%%%%%%%%%%
%%%%%%%%%%%%%%%%%%%%%%%%%%%%%%%%%%%%%%%%%%%%%%%%%%%%%%%%%%%%%%%%%%%%%%%%%%%%%%%%
\appendix

\settowidth\MacroIndent{\rmfamily\scriptsize 000\ }

 \DocInput{childdoc.dtx}

\end{document}
%</driver>
% \fi
%
% %%%%%%%%%%%%%%%%%%%%%%%%%%%%%%%%%%%%%%%%%%%%%%%%%%%%%%%%%%%%%%%%%%%%%%%%%%%%%%
% %%%%%%%%%%%%%%%%%%%%%%%%%%%%%%%%%%%%%%%%%%%%%%%%%%%%%%%%%%%%%%%%%%%%%%%%%%%%%%
% \section{Sample}
%\iffalse
%<*samplemain>
%\fi
%
% The following presents a sample document
% with two chapters, two parts, a title page,
% a compile flag as well as three forwarding files to set the flag.
% It consists of eight |.tex| files:
% \begin{center}
% \begin{tabular}{ll}
% |cdocsamp.tex|&main file\\
% |cdocsch1.tex|&include file for chapter 1\\
% |cdocsch2.tex|&include file for chapter 2\\
% |cdocspt3.tex|&include file for part 3\\
% |cdocspt4.tex|&include file for part 4\\
% |cdocsdrf.tex|&forwarding file for main file in draft mode\\
% |cdocsfi1.tex|&forwarding file for final version of chapter 1\\
% |cdocsfi2.tex|&forwarding file for final version of chapter 2\\
% \end{tabular}
% \end{center}
% Each of the eight files can be compiled directly by the \LaTeX{} compiler.
%
% %%%%%%%%%%%%%%%%%%%%%%%%%%%%%%%%%%%%%%
% \paragraph{Main File.}
%
% The main file is called |cdocsamp.tex|.
%
% Load the \textsf{childdoc} definitions and
% declare the filename for the main document:
%    \begin{macrocode}
\input{childdoc.def}
\childdocmain{}
%    \end{macrocode}

% Optional override for |\version| flag:
%    \begin{macrocode}
%%\ifchilddoc\else\providecommand{\version}{draft}\fi
%    \end{macrocode}

% Define the default values for the |\version| flag
% (|final| for the main file and |draft| for childs):
%    \begin{macrocode}
\ifchilddoc
\providecommand{\version}{draft}
\else
\providecommand{\version}{final}
\fi
%    \end{macrocode}

% Load the standard document class:
%    \begin{macrocode}
\documentclass[12pt]{article}
%    \end{macrocode}

% Start the document body:
%    \begin{macrocode}
\begin{document}
%    \end{macrocode}

% Declare a title page.
% Print title, part of document being processed and version flag:
%    \begin{macrocode}
\addtocounter{page}{-1}
\begin{center}
{\LARGE\bfseries{}childdoc example\par}
\vspace{1cm}
\ifchilddoc
\ifchilddocmanual part\else chapter\fi:
`\childdocname' of `\childdocjob'\par
\else
main document: `\childdocjob'\par
\fi
version: \version\par
\end{center}
\newpage
%    \end{macrocode}

% Manually include selected file,
% otherwise process as usual:
%    \begin{macrocode}
\ifchilddocmanual
\section*{part `\childdocname'}
\input{\childdocname}
\else
%    \end{macrocode}

% Include the two chapters:
%    \begin{macrocode}
\include{cdocsch1}
\include{cdocsch2}
%    \end{macrocode}

% Include the two parts unless only chapters should be displayed:
%    \begin{macrocode}
\ifchilddoc\else
\section{part three}
\input{cdocspt3}
\section{part four}
\input{cdocspt4}
\fi
%    \end{macrocode}

% Process as usual until here:
%    \begin{macrocode}
\fi
%    \end{macrocode}

% End of document body:
%    \begin{macrocode}
\end{document}
%    \end{macrocode}
%\iffalse
%</samplemain>
%\fi
%
% %%%%%%%%%%%%%%%%%%%%%%%%%%%%%%%%%%%%%%
% \paragraph{Chapter Include Files.}
%
% The include files are called |cdocsch1.tex| and |cdocsch2.tex|.
%
%\iffalse
%<*samplechap1|samplechap2>
%\fi

% Optional override for |\version| flag:
%    \begin{macrocode}
%%\providecommand{\version}{final}
%    \end{macrocode}

% Include the main document:
%    \begin{macrocode}
\input{childdoc.def}
\childdocof{cdocsamp}
%    \end{macrocode}

%\iffalse
%</samplechap1|samplechap2>
%\fi
%
%\iffalse
%<*samplechap1>
%\fi
% Some text for chapter 1:
%    \begin{macrocode}
\section{one}
some text in chapter one
%    \end{macrocode}

%\iffalse
%</samplechap1>
%\fi
% Some text for chapter 2:
%\iffalse
%<*samplechap2>
%\fi
%    \begin{macrocode}
\section{two}
more text in chapter two
%    \end{macrocode}

%\iffalse
%</samplechap2>
%\fi
%
% %%%%%%%%%%%%%%%%%%%%%%%%%%%%%%%%%%%%%%
% \paragraph{Part Include Files.}
%
% The include files are called |cdocspt3.tex| and |cdocspt4.tex|.
%
%\iffalse
%<*samplepart3|samplepart4>
%\fi

% Optional override for |\version| flag:
%    \begin{macrocode}
%%\providecommand{\version}{final}
%    \end{macrocode}

% Include the main document:
%    \begin{macrocode}
\input{childdoc.def}
\childdocby{cdocsamp}
%    \end{macrocode}

%\iffalse
%</samplepart3|samplepart4>
%\fi
%
%\iffalse
%<*samplepart3>
%\fi
% Some text for part 3:
%    \begin{macrocode}
some text in part three
%    \end{macrocode}

%\iffalse
%</samplepart3>
%\fi
% Some text for part 4:
%\iffalse
%<*samplepart4>
%\fi
%    \begin{macrocode}
more text in part four
%    \end{macrocode}

%\iffalse
%</samplepart4>
%\fi
%
% %%%%%%%%%%%%%%%%%%%%%%%%%%%%%%%%%%%%%%
% \paragraph{Forwarding for a Complete Draft.}
%
% The following forwarding file |cdocsdrf.tex|
% compiles the main document in draft mode:
%\iffalse
%<*sampledraft>
%\fi
%    \begin{macrocode}
\def\version{draft}
\input{childdoc.def}
\childdocforward{cdocsamp}
%    \end{macrocode}

%\iffalse
%</sampledraft>
%\fi
%
% %%%%%%%%%%%%%%%%%%%%%%%%%%%%%%%%%%%%%%
% \paragraph{Forwarding for Final Version of the Chapters.}
%
% The following forwarding files |cdocsfn1.tex| and |cdocsfn2.tex|
% (with identical content)
% compile the final versions of the child documents
% |cdocsch1.tex| and |cdocsch2.tex|, respectively:
%\iffalse
%<*samplefinal>
%\fi
%    \begin{macrocode}
\def\version{final}
\input{childdoc.def}
\childdocforwardprefix[cdocsamp]{cdocsfn}{cdocsch}
%    \end{macrocode}

%\iffalse
%</samplefinal>
%\fi
%
% %%%%%%%%%%%%%%%%%%%%%%%%%%%%%%%%%%%%%%
% \paragraph{Command Line Processing.}
%
% The following three command lines generate the output files
% |cdocscld|, |cdocscl1| and |cdocscl2|
% which should be identical to
% |cdocsdrf|, |cdocsch1| and |cdocsfn2|, respectively:
% \begin{center}
% \begin{tabular}{l}
% |latex -jobname cdocscld \|\\
% |  "\def\version{draft}\input{childdoc.def}\childdocforward{cdocsamp}"|\\
% |latex -jobname cdocscl1 \|\\
% |  "\input{childdoc.def}\childdocforward[cdocsamp]{cdocsch1}"|\\
% |latex -jobname cdocscl2 \|\\
% |  "\def\version{final}\input{childdoc.def}\childdocforward{cdocsch2}"|
% \end{tabular}
% \end{center}
% Note that the trailing backslash on each first line
% merely continues the input to the second line
% (for convenient cut ant paste).
% Furthermore, the command |latex| can be replaced by any
% of its alternative versions such as |pdflatex|.
%
% %%%%%%%%%%%%%%%%%%%%%%%%%%%%%%%%%%%%%%%%%%%%%%%%%%%%%%%%%%%%%%%%%%%%%%%%%%%%%%
% %%%%%%%%%%%%%%%%%%%%%%%%%%%%%%%%%%%%%%%%%%%%%%%%%%%%%%%%%%%%%%%%%%%%%%%%%%%%%%
% \section{Implementation}
%\iffalse
%<*package>
%\fi
%
% This section describes the definitions file |childdoc.def|.

% The definitions cannot be loaded using |\usepackage| or |\RequirePackage|
% which has a mechanism to prevent loading a style file more than once.
% When loading the definitions by means of |\input|
% multiple instances have to be prevented manually:
%\iffalse
%This code needs to be before the `\ProvidesFile' directive
%which is defined at the beginning of this file.
%Therefore it is also placed there and commented out here.
%</package>
%<*discard>
%\fi
%    \begin{macrocode}
\ifdefined\childdocmain\endinput\fi
%    \end{macrocode}
%\iffalse
%</discard>
%<*package>
%\fi
%
% \macro{\ifchilddoc}
% \macro{\ifchilddocmanual}
% The conditional |\ifchilddoc| tells whether a
% child (true) or main (false) document is being compiled.
% The conditional |\ifchilddocmanual| tells whether
% the |\includeonly| mechanism is used (false) or
% the selection of child files must be performed manually (true).
% The definitions initialise to false:
%    \begin{macrocode}
\newif\ifchilddoc
\newif\ifchilddocmanual
%    \end{macrocode}

% \macro{\childdocname}
% \macro{\childdocjob}
% The macro |\childdocname| stores the name of the main document
% to be compiled. The macro |\childdocjob| stores the name of
% the document on which the \LaTeX{} compiler was originally invoked.
% The content of |\jobname| cannot be compared
% to filenames specified in the source due to different catcodes.
% The following code rescans |\jobname|, stores the result
% in |\childdocname| and saves a copy in |\childdocjob|:
%    \begin{macrocode}
\edef\childdocname{\scantokens\expandafter{\jobname\noexpand}}
\let\childdocjob\childdocname
%    \end{macrocode}

% \macro{\childdocdisable}
% The macro |\childdocdisable| prevents the main file
% from being processed more than once.
% At this stage, the main document command |\childdocmain|
% is assumed to be called once again where it should do nothing.
% Any subsequent call to it should prevent
% a secondary processing of the main document
% It overwrites the forwarding commands
% |\childdocof| and |\childdocforward|
% with empty macros to prevent further inclusions of the main document:
%    \begin{macrocode}
\newcommand{\childdocdisable}
{
  \renewcommand{\childdocmain}[1]{\renewcommand{\childdocmain}[1]{\endinput}}
  \renewcommand{\childdocof}[1]{}
  \renewcommand{\childdocby}[2][]{}
  \renewcommand{\childdocforward}[2][]{}
  \renewcommand{\childdocdisable}{}
}
%    \end{macrocode}

% \macro{\childdocmain}
% The macro |\childdocmain| is to be called at the top of the main file
% with nothing or the main filename (without extension) as argument.
% First, it breaks loops.
% If the argument is not empty and does not match |\childdocname|
% (which is set by the first inclusion of |childdoc.def|),
% |\ifchilddoc| is set to true, |\includeonly| is applied to the child file
% and |\jobname| is set to the main file
% (for proper handling of |.aux| files):
%    \begin{macrocode}
\newcommand{\childdocmain}[1]
{
  \childdocdisable\childdocmain{}
  \if?#1?\else
    \begingroup
      \def\childdoctmp{#1}
      \ifx\childdoctmp\childdocname
        \def\childdoctmp{}
      \else
        \def\childdoctmp
        {
          \childdoctrue
          \includeonly{\childdocname}
          \def\childdocjob{#1}
          \def\jobname{#1}
        }
      \fi
      \expandafter
    \endgroup
    \childdoctmp
  \fi
}
%    \end{macrocode}

% \macro{\childdocof}
% The command |\childdocof| redirects
% compilation to the main file |#1|.
%    \begin{macrocode}
\newcommand{\childdocof}[1]
{
  \childdocdisable
  \childdoctrue
  \includeonly{\childdocname}
  \def\jobname{#1}
  \def\childdocjob{#1}
  \input{#1}
}
%    \end{macrocode}

% \macro{\childdocby}
% The command |\childdocby| ....
%    \begin{macrocode}
\newcommand{\childdocby}[2][]
{
  \childdocdisable
  \childdoctrue
  \childdocmanualtrue
  \if?#1?\else
    \def\jobname{#2}
  \fi
  \def\childdocjob{#2}
  \input{#2}
  \endinput
}
%    \end{macrocode}

% \macro{\childdocforward}
% The command |\childdocforward| redirects
% compilation to the main file or
% (if the optional argument is given) a child file.
% Parameters are set as if the main file
% or a child file starting with |\childdocof| was compiled.
% Then compilation is handed over to the main file:
%    \begin{macrocode}
\newcommand{\childdocforward}[2][]
{
  \begingroup
    \if?#1?
      \def\childdoctmp
      {
        \def\childdocname{#2}
        \def\childdocjob{#2}
        \def\jobname{#2}
        \input{#2}
        \endinput
      }
    \else
      \def\childdoctmp
      {
        \childdocdisable
        \def\childdocname{#2}
        \childdoctrue
        \includeonly{#2}
        \def\childdocjob{#1}
        \def\jobname{#1}
        \input{#1}
        \endinput
      }
    \fi
    \expandafter
  \endgroup
  \childdoctmp
}
%    \end{macrocode}

% \macro{\childdocforwardprefix}
% The command |\childdocforwardprefix| redirects
% compilation to the main or a child file by means of a pattern.
% The prefix |#1| in the current filename is replaced by |#2|
% and the suffix of the current filename is kept
% (it is assumed that the filename does not contain the substring `|~~~|'
% which is used as a delimiter).
% Compilation is handed over to the new file by |\childdocforward|:
%    \begin{macrocode}
\newcommand{\childdocforwardprefix}[3][]
{
  \begingroup
    \def\childdocextract #2##1~~~{\def\childdoctmp{\childdocforward[#1]{#3##1}}}
    \expandafter\childdocextract\childdocname~~~
    \expandafter
  \endgroup
  \childdoctmp
}
%    \end{macrocode}

% \macro{\childdoc}
% The deprecated macro |\childdoc| is a legacy version of |\childdocmain|:
%    \begin{macrocode}
\newcommand{\childdoc}{\childdocmain}
%    \end{macrocode}

% \macro{\childdocredirect}
% The deprecated macro |\childdocredirect| is a legacy version
% of |\childdocforward| and |\childdocforwardprefix|:
%    \begin{macrocode}
\newcommand{\childdocredirect}[2][]
{
  \begingroup
    \if?#1?
      \def\childdoctmp{\childdocforward{#2}}
    \else
      \def\childdoctmp{\childdocforwardprefix{#1}{#2}}
    \fi
    \expandafter
  \endgroup
  \childdoctmp
}
%    \end{macrocode}

%\iffalse
%</package>
%\fi
%
\endinput
|\\
|\childdocforwardprefix[|\textit{main}|]{|\textit{prefix}|}{|\textit{dest}|}|
\end{tabular}
\end{center}
%
the destination file is determined by a pattern
depending on the current file:
To make this work, the current file must be called
`{\textit{prefix}\hspace{0.2em}\textit{suffix}}'
with \textit{prefix} matching precisely the argument.
Processing is then passed on to the file
`{\textit{dest}\hspace{0.2em}\textit{suffix}}'.
Surely, the same effect is achieved by
directly specifying the
argument `{\textit{dest}\hspace{0.2em}\textit{suffix}}'
in the first form.
However, that requires to set up a different file
for each child. With the alternative form of the command
all these files can have exactly the same content
which simplifies setting them up and maintaining them.

For example, the following file |draft.tex|
with a compilation flag |\version| as described in \secref{sec:flags}
compiles the main document as a draft:
%
\begin{center}
\begin{tabular}{l}
|\def\version{draft}|\\
|% \iffalse
%
% childdoc.dtx Copyright (C) 2017-2018 Niklas Beisert
%
% This work may be distributed and/or modified under the
% conditions of the LaTeX Project Public License, either version 1.3
% of this license or (at your option) any later version.
% The latest version of this license is in
%   http://www.latex-project.org/lppl.txt
% and version 1.3 or later is part of all distributions of LaTeX
% version 2005/12/01 or later.
%
% This work has the LPPL maintenance status `maintained'.
%
% The Current Maintainer of this work is Niklas Beisert.
%
% This work consists of the files childdoc.dtx and childdoc.ins
% and the derived files childdoc.def and cdocsamp.tex with
% cdocsch1.tex, cdocsch2.tex, cdocsdrf.tex, cdocsfn1.tex, cdocsfn2.tex.
%
%<package>\ifdefined\childdocmain\endinput\fi
%<package>\ProvidesFile{childdoc.def}[2018/12/30 v2.0 child document driver]
%<samplemain>\ProvidesFile{cdocsamp.tex}[2018/12/30 v2.0 sample for childdoc]
%<*driver>
%\ProvidesFile{childdoc.drv}[2018/12/30 v2.0 childdoc reference manual file]
\PassOptionsToClass{10pt,a4paper}{article}
\documentclass{ltxdoc}

\usepackage[margin=35mm]{geometry}
\usepackage{hyperref}
\usepackage{hyperxmp}
\usepackage[usenames]{color}

\hypersetup{colorlinks=true}
\hypersetup{pdfstartview=FitH}
\hypersetup{pdfpagemode=UseNone}
\hypersetup{pdfsource={}}
\hypersetup{pdflang={en-UK}}
\hypersetup{pdfcopyright={Copyright 2017-2018 Niklas Beisert.
  This work may be distributed and/or modified under the
  conditions of the LaTeX Project Public License, either version 1.3
  of this license or (at your option) any later version.}}
\hypersetup{pdflicenseurl={http://www.latex-project.org/lppl.txt}}
\hypersetup{pdfcontactaddress={ETH Zurich, ITP, HIT K,
  Wolfgang-Pauli-Strasse 27}}
\hypersetup{pdfcontactpostcode={8093}}
\hypersetup{pdfcontactcity={Zurich}}
\hypersetup{pdfcontactcountry={Switzerland}}
\hypersetup{pdfcontactemail={nbeisert@itp.phys.ethz.ch}}
\hypersetup{pdfcontacturl={http://people.phys.ethz.ch/\xmptilde nbeisert/}}

\newcommand{\secref}[1]{\hyperref[#1]{section \ref*{#1}}}

\parskip1ex
\parindent0pt
\let\olditemize\itemize
\def\itemize{\olditemize\parskip0pt}

\begin{document}

\title{The \textsf{childdoc} Package}
\hypersetup{pdftitle={The childdoc Package}}
\author{Niklas Beisert\\[2ex]
  Institut f\"ur Theoretische Physik\\
  Eidgen\"ossische Technische Hochschule Z\"urich\\
  Wolfgang-Pauli-Strasse 27, 8093 Z\"urich, Switzerland\\[1ex]
  \href{mailto:nbeisert@itp.phys.ethz.ch}
  {\texttt{nbeisert@itp.phys.ethz.ch}}}
\hypersetup{pdfauthor={Niklas Beisert}}
\hypersetup{pdfsubject={Manual for the LaTeX2e Package childdoc}}
\date{30 December 2018, \textsf{v2.0}}
\maketitle

\begin{abstract}\noindent
\textsf{childdoc} is a \LaTeXe{} package
that enables the direct compilation
of document sections included by |\include|
to individual files.
\end{abstract}

\begingroup
\parskip0ex
\tableofcontents
\endgroup

%%%%%%%%%%%%%%%%%%%%%%%%%%%%%%%%%%%%%%%%%%%%%%%%%%%%%%%%%%%%%%%%%%%%%%%%%%%%%%%%
%%%%%%%%%%%%%%%%%%%%%%%%%%%%%%%%%%%%%%%%%%%%%%%%%%%%%%%%%%%%%%%%%%%%%%%%%%%%%%%%
\section{Introduction}

\LaTeX{} provides a mechanism to structure a large document (such as a book)
into a main file and several child files (containing the chapters)
using the |\include| command.
This mechanism is beneficial for documents
which span hundreds of pages in order to
make the source file(s) more manageable.
Moreover, compilation can be restricted to
selected child files by means of the |\includeonly| command.
The latter feature can be used to reduce the compilation time while editing
(this was significantly more useful in the earlier days of \LaTeX{})
or to generate a smaller document which is easier to navigate.
Another application of |\includeonly| is to generate
documents consisting of selected parts of the complete document.

However, there are a few drawbacks of the plain |\include| mechanism:
\begin{itemize}
\item
The child files cannot be compiled on their own,
they can only be compiled via the main file.
A naive editing environment
(such as a text editor with an option
to have the current file processed by \LaTeX)
may require one to switch to the main file before compiling;
attempting to compile the child file produces errors.
\item
The main file must be modified (each time)
to adjust the |\includeonly| command
to the present needs. This easily leaves the main file in a messy state.
\item
The generated document will always carry the filename
of the main document. This is inconvenient if
several child files are to be compiled and
to be kept for distribution.
\end{itemize}

The present package provides a simple interface
to make child files individually compilable by \LaTeX{}.
Compiling a child file then has the same effect as compiling
the main file with an |\includeonly| command
to select the appropriate child.
Moreover the generated document will carry the name of the child
rather than the main file.
This resolves all three above issues.

This feature is meant to make the editing of books,
thesis documents and lecture notes somewhat more convenient.
However, the package can also be used efficiently for
composing a series of documents (such as exercise sheets)
which are typically distributed individually.
It then assists the author in generating the individual documents
(potentially in different versions)
as well as a document containing the collected series.
Another application is in developing style files
or other kinds of included material
where compilation of the style file could redirect
to a sample or test file.

%%%%%%%%%%%%%%%%%%%%%%%%%%%%%%%%%%%%%%%%%%%%%%%%%%%%%%%%%%%%%%%%%%%%%%%%%%%%%%%%
%%%%%%%%%%%%%%%%%%%%%%%%%%%%%%%%%%%%%%%%%%%%%%%%%%%%%%%%%%%%%%%%%%%%%%%%%%%%%%%%
\section{Usage}

First of all, the package \textsf{childdoc} is \emph{not} a standard
\LaTeXe{} |.sty| style file! Therefore it needs to be invoked in
a non-standard way.

%%%%%%%%%%%%%%%%%%%%%%%%%%%%%%%%%%%%%%%%%%%%%%%%%%%%%%%%%%%%%%%%%%%%%%%%%%%%%%%%
\subsection{Included Files}
\label{sec:include}

%%%%%%%%%%%%%%%%%%%%%%%%%%%%%%%%%%%%%%%%
\DescribeMacro{\childdocmain}
To use the package, add the commands
\begin{center}
\begin{tabular}{l}
|\input{childdoc.def}|\\
|\childdocmain{}|\\
\end{tabular}
\end{center}
at the very top of the main \LaTeX{} file,
in particular \emph{before} the |\documentclass| statement!
The argument of |\childdocmain| should be left empty
(but it must be present).

%%%%%%%%%%%%%%%%%%%%%%%%%%%%%%%%%%%%%%%%
\DescribeMacro{\childdocof}
Furthermore, add the commands
\begin{center}
\begin{tabular}{l}
|\input{childdoc.def}|\\
|\childdocof{|\textit{main}|}|\\
\end{tabular}
\end{center}
at the top of every child file \textit{child}
which is included by |\include{|\textit{child}|}|
from within the main file
(or at least for those files to be compiled individually).
The argument \textit{main} must be the filename of the main file.

There are a couple of
considerations in setting up the main and child documents:

%%%%%%%%%%%%%%%%%%%%%%%%%%%%%%%%%%%%%%%%
\paragraph{Restrictions.}

Please note the following restrictions:
\begin{itemize}
\item
|\childdocmain| must be called with one argument \textit{main}
to ensure compatibility with earlier version of the package.
It must either be empty (|\childdocmain{}|)
or precisely match the filename of the main file in which it is specified.
See \secref{sec:detection} for further information.
\item
The filename \textit{main} must be specified without the |.tex| extension.
\item
The filename \textit{main} is case sensitive
(even in case-insensitive file systems)
due to internal string comparison.
\item
The argument \textit{main} should be fully expanded, it cannot be a macro.
\item
Subdirectories and special characters should be avoided in filenames.
\item
The command |\childdocmain{|\textit{main}|}| must be followed by a whitespace.
It should not be followed immediately by another command
or by a comment mark `|%|'.
This is because the \TeX{} parser reads the token immediately following
the argument of |\childdocmain| and puts it
at the beginning of every child section;
however, a white\-space is ignored.
\end{itemize}

%%%%%%%%%%%%%%%%%%%%%%%%%%%%%%%%%%%%%%%%
\paragraph{Content of Main File.}

It is advisable to place all content in the child files included by |\include|.
Any output contained in the main file will appear in all child documents
unless suppressed manually;
it cannot be suppressed automatically by the |\includeonly| directive
and thus should normally be avoided.
A method to include some content in the main file
by means of conditional processing is described in \secref{sec:conditional}.

%%%%%%%%%%%%%%%%%%%%%%%%%%%%%%%%%%%%%%%%
\paragraph{Page Numbering.}

When only a part of the document is compiled,
the appropriate numbering of pages
(as well as other status parameters)
is determined from the |.aux| files.
The latter contain information from previous passes.
However this information needs to propagate through
all intermediate child documents.
Therefore the page numbering in child documents may well
be inconsistent until the complete document is compiled at least once.

A useful (if unconventional) way to always ensure a consistent
page numbering is to restart the numbering in each child document
and denote the pages by `\textit{child}|.|\textit{page}'
where \textit{child} represents the chapter/section number of the child file.
This can be achieved by the command
|\numberwithin{page}{|\textit{child}|}|
of the \textsf{amsmath} package
where \textit{child} can be |chapter| or |section|
depending on the chosen structuring.
Alternatively, one can modify the macro |\thepage| appropriately
and reset the counter |page| at the start of each child file.

%%%%%%%%%%%%%%%%%%%%%%%%%%%%%%%%%%%%%%%%%%%%%%%%%%%%%%%%%%%%%%%%%%%%%%%%%%%%%%%%
\subsection{Conditional Processing}
\label{sec:conditional}

The package provides a mechanism to compile different versions
of a document. To customise the versions further some conditional processing
can come in handy to distinguish which version is being compiled.
The package provides two macros to describe the compilation context:

%%%%%%%%%%%%%%%%%%%%%%%%%%%%%%%%%%%%%%%%
\DescribeMacro{\ifchilddoc}
The conditional |\ifchilddoc| distinguishes between the compilation of
child documents and the main document:
%
\begin{center}
|\ifchilddoc |\textit{child-code}| |[|\||else |\textit{main-code}]| \||fi|
\end{center}

%%%%%%%%%%%%%%%%%%%%%%%%%%%%%%%%%%%%%%%%
\DescribeMacro{\childdocname}
\DescribeMacro{\childdocjob}
The macro |\childdocname| contains the filename (without extension)
of the main or child file being processed.
Note that |\childdocjob| will always contain the name of the main file.

%%%%%%%%%%%%%%%%%%%%%%%%%%%%%%%%%%%%%%%%
\paragraph{Title Page.}

Conditional processing can be used to include a title or banner page
in the main document when proper precautions are taken.
Importantly, the code in the main file should ensure that the page counter
(as well as other status parameters which are stored in the |.aux| files)
takes the same value after the conditional processing.
Otherwise the page numbers may take divergent values
depending on which part is compiled.

For example, a title page could be declared by:
%
\begin{center}
\begin{tabular}{l}
|\ifchilddoc\||else|\\
|\addtocounter{page}{-1}|\\
\textit{code for title page}\\
|\newpage|\\
|\||fi|
\end{tabular}
\end{center}
%
A banner page for the child documents can be generated by:
%
\begin{center}
\begin{tabular}{l}
|\ifchilddoc|\\
|\addtocounter{page}{-1}|\\
\textit{code for banner page}\\
|\newpage|\\
|\||fi|
\end{tabular}
\end{center}
%
Here one could write a message such as:
\begin{center}
|This is the part \childdocname{} of \childdocjob{}.|
\end{center}

%%%%%%%%%%%%%%%%%%%%%%%%%%%%%%%%%%%%%%%%%%%%%%%%%%%%%%%%%%%%%%%%%%%%%%%%%%%%%%%%
\subsection{Flags}
\label{sec:flags}

The package makes it easy to generate different versions
of the main or child documents.
To this end compilation flags can be defined
and assigned different default values.
They will be particularly useful in conjunction
with the forwarding mechanism described in \secref{sec:forward}.

For example, it may be useful to have a flag |\version|
which can be set to |draft| or |final|.
The document source will contain some conditional code
depending on the value of |\version|.
Suppose further, the flag should default to |final| for the main file
and to |draft| for child files
which is a natural assignment for editing the document.
This is achieved by placing the following code
in the preamble of the main document
(below the |\childdocmain| directive):
%
\begin{center}
\begin{tabular}{l}
|\ifchilddoc|\\
|\providecommand{\version}{draft}|\\
|\||else|\\
|\providecommand{\version}{final}|\\
|\||fi|
\end{tabular}
\end{center}
%
The definition by |\providecommand| makes sure
that previous definitions are not overwritten.
Further statements |\providecommand{\version}{...}|
can thus be added before the above code to override it.

For the main file, one might add a line
(between |\childdocmain| and the above block)
%
\begin{center}
|%\ifchilddoc\||else\providecommand{\version}{draft}\||fi|
\end{center}
%
which can be uncommented to produce a draft version.
Likewise one can add a line to the very top of a child file
(above the |\childdocof{|\textit{main}|}| directive)
%
\begin{center}
|%\providecommand{\version}{final}|
\end{center}
%
which can be uncommented to produce the final version of this child document.

%%%%%%%%%%%%%%%%%%%%%%%%%%%%%%%%%%%%%%%%%%%%%%%%%%%%%%%%%%%%%%%%%%%%%%%%%%%%%%%%
\subsection{Forwarding}
\label{sec:forward}

Different versions of the main or child documents
using compilation flags as described in \secref{sec:flags}
can be (permanently) stored in different files
for convenient compilation, viewing and distribution.
To this end, the package defines a command
to pass on compilation to a different file:

%%%%%%%%%%%%%%%%%%%%%%%%%%%%%%%%%%%%%%%%
\DescribeMacro{\childdocforward}
The command |\childdocforward| redirects processing to
another source file:
%
\begin{center}
\begin{tabular}{l}
|\input{childdoc.def}|\\
|\childdocforward[|\textit{main}|]{|\textit{dest}|}|\\
\end{tabular}
\end{center}
%
The argument \textit{dest} is the destination file
(without extension).
It should be the main file or one of the child files.
Note that further \textsf{childdoc} directives
such as |\childdocof| and |\childdocforward|
in the indicated file will be processed in this form.
The optional argument \textit{main}
passes on directly to the main file \textit{main}
while pretending to compile the child \textit{dest}.
This form behaves as if \textit{dest}
issues |\childdocof{|\textit{main}|}| right away,
and no further \textsf{childdoc} directives will be processed.

%%%%%%%%%%%%%%%%%%%%%%%%%%%%%%%%%%%%%%%%
\DescribeMacro{\...prefix}
In the alternative form |\childdocforwardprefix|,
%
\begin{center}
\begin{tabular}{l}
|\input{childdoc.def}|\\
|\childdocforwardprefix[|\textit{main}|]{|\textit{prefix}|}{|\textit{dest}|}|
\end{tabular}
\end{center}
%
the destination file is determined by a pattern
depending on the current file:
To make this work, the current file must be called
`{\textit{prefix}\hspace{0.2em}\textit{suffix}}'
with \textit{prefix} matching precisely the argument.
Processing is then passed on to the file
`{\textit{dest}\hspace{0.2em}\textit{suffix}}'.
Surely, the same effect is achieved by
directly specifying the
argument `{\textit{dest}\hspace{0.2em}\textit{suffix}}'
in the first form.
However, that requires to set up a different file
for each child. With the alternative form of the command
all these files can have exactly the same content
which simplifies setting them up and maintaining them.

For example, the following file |draft.tex|
with a compilation flag |\version| as described in \secref{sec:flags}
compiles the main document as a draft:
%
\begin{center}
\begin{tabular}{l}
|\def\version{draft}|\\
|\input{childdoc.def}|\\
|\childdocforward{|\textit{main}|}|
\end{tabular}
\end{center}
%
Likewise, the following files |final|\textit{nn}|.tex|
compile the final version of the child document
|child|\textit{nn}|.tex|:
%
\begin{center}
\begin{tabular}{l}
|\def\version{final}|\\
|\input{childdoc.def}|\\
|\childdocforwardprefix{final}{child}|
\end{tabular}
\end{center}
%

Note that when several versions of a main file and/or of each child file
are to be generated, it may be convenient to set up a |Makefile| or
shell script to automatise the process.

%%%%%%%%%%%%%%%%%%%%%%%%%%%%%%%%%%%%%%%%%%%%%%%%%%%%%%%%%%%%%%%%%%%%%%%%%%%%%%%%
\subsection{Command Line Processing}
\label{sec:commandline}

The effect of redirection files can also be achieved by invoking
the \LaTeX{} compiler with a more elaborate command line.
Most conveniently this should be done as part
of a shell script or a |Makefile|.

When using \textsf{childdoc} in the main file, the following
command lines effectively perform a redirection
(note that depending on the shell being used,
backslashes may have to be doubled: `|\|' $\to$ `|\\|'):
%
\begin{center}
|... -jobname "|\textit{target}|" |\\|"|[\textit{flags}]%
|\input{childdoc.def}\childdocforward[|\textit{main}|]{|\textit{dest}|}"|
\end{center}
%
Here \textit{target} is the name of the output file,
\textit{main} is the name of the main file
and \textit{dest} is the name of the main or child file to be processed
(all filenames without extensions).
The optional argument \textit{main} can be omitted
if \textit{main} matches \textit{dest}.
Optionally, compilation \textit{flags} can be defined via |\def| commands.
This command line makes the \TeX{} engine believe
it is compiling the file \textit{target}
whose content is specified as the latter parameter.
The provided code then forwards the processing to
\textit{main} or \textit{dest} as described in \secref{sec:forward}.

%%%%%%%%%%%%%%%%%%%%%%%%%%%%%%%%%%%%%%%%%%%%%%%%%%%%%%%%%%%%%%%%%%%%%%%%%%%%%%%%
\subsection{Include by Input}
\label{sec:input}

Including child documents by |\include| has some restrictions by design.
Most notably, the content of a child document always occupies
its own set of pages; pages cannot be shared between child documents.
Usually, this behaviour makes perfect sense
because each child document contain an essential part of the document.
However, in some situations it may be desirable to compose
a document from a collection of parts
without having mandatory page breaks between then.
For this case, the package
provides a mechanism to include parts
by |\input| which can also be processed individually.
However, by construction this mechanism
requires manual handling of the content to be output.

%%%%%%%%%%%%%%%%%%%%%%%%%%%%%%%%%%%%%%%%
\DescribeMacro{\ifchilddocmanual}
The main file should be prepared as usual, see \secref{sec:include}.
However, the document body must make a distinction
between processing of an individual part and of the main document, e.g.:
%
\begin{center}
\begin{tabular}{l}
|\ifchilddocmanual|\\
|\input{\childdocname}|\\
|\||else|\\
\textit{document body with }|\input{|\textit{part}|}|\\
|\||fi|
\end{tabular}
\end{center}
%
The conditional |\ifchilddocmanual| is true whenever
a part to be included by |\input| is being compiled,
and the name of the part is stored in |\childdocname|.

%%%%%%%%%%%%%%%%%%%%%%%%%%%%%%%%%%%%%%%%
\DescribeMacro{\childdocby}
Each part to be included by |\input| should start with:
%
\begin{center}
\begin{tabular}{l}
|\input{childdoc.def}|\\
|\childdocby{|\textit{main}|}|\\
\end{tabular}
\end{center}
%
The directive |\childdocby| is similar to |\childdocof|
described in \secref{sec:include},
but the subsequent selection of content must be done manually.
To that end, both |\ifchilddoc| and |\ifchilddocmanual|
will be true upon processing of a part,
and the name of the part is stored in |\childdocname|.
Note that |\jobname| will be set to the filename of the current part
so that each part receives an individual |.aux| file
that does not interfere with the |.aux| file(s) of the main document.
This behaviour can be altered by the alternative form
|\childdocby[*]{|\textit{main}|}| (with a non-empty optional argument)
which uses the |.aux| file of the main document
by setting |\jobname| to \textit{main}.

%%%%%%%%%%%%%%%%%%%%%%%%%%%%%%%%%%%%%%%%%%%%%%%%%%%%%%%%%%%%%%%%%%%%%%%%%%%%%%%%
\subsection{Driver Development}
\label{sec:driver}

The \textsf{childdoc} mechanism can also be use for the development
of definition files such as \LaTeX{} styles or classes.
This case differs from the above setup with multiple parts
included by |\include| in that no |\includeonly| should be invoked.
This can be achieved by starting the include file
(before |\ProvidesPackage|) with:
%
\begin{center}
\begin{tabular}{l}
|\input{childdoc.def}|\\
|\childdocforward{|\textit{main}|}|\\
\end{tabular}
\end{center}
%
or alternatively with:
%
\begin{center}
\begin{tabular}{l}
|\input{childdoc.def}|\\
|\childdocby{|\textit{main}|}|\\
\end{tabular}
\end{center}
%
Both forms have slightly different effects as described above.
The main file is prepared as usual, see \secref{sec:include}.

%%%%%%%%%%%%%%%%%%%%%%%%%%%%%%%%%%%%%%%%%%%%%%%%%%%%%%%%%%%%%%%%%%%%%%%%%%%%%%%%
\subsection{Legacy Detection}
\label{sec:detection}

The directive |\childdocmain| in the main file can detect
whether the complete document or merely a child is to be compiled
even without using the directive |\childdocof|.
This method is deprecated because it is less robust
and there is no compelling reason to use it;
it is merely provided for backward compatibility
and it may be removed in future versions.

If the detection mechanism is to be used,
it is mandatory to correctly specify
the filename of the main file as the argument of |\childdocmain|:
%
\begin{center}
\begin{tabular}{l}
|\input{childdoc.def}|\\
|\childdocmain{|\textit{main}|}|\\
\end{tabular}
\end{center}
%
If |\jobname| does not match the argument \textit{main} of |\childdocmain|,
it is assumed that |\jobname| points to the child file to be compiled.
When using |\childdocmain| with the main file specified as argument,
it suffices to start a child file
with just |\input{|\textit{main}|}|
without loading of the package and using |\childdocof|.
If instead all processing is done
with the appropriate \textsf{childdoc} directives,
the argument of \textit{main} of |\childdocmain| can be empty.

An alternative version of the command line processing described
in \secref{sec:commandline} using the detection mechanism reads:
%
\begin{center}
|... -jobname "|\textit{target}|" "|[\textit{flags}]%
[|\def\jobname{|\textit{dest}|}|]|\input{|\textit{main}|}"|
\end{center}

%%%%%%%%%%%%%%%%%%%%%%%%%%%%%%%%%%%%%%%%%%%%%%%%%%%%%%%%%%%%%%%%%%%%%%%%%%%%%%%%
\subsection{Manual Code}
\label{sec:manual}

In case one cannot be certain whether the definitions file |childdoc.def|
is installed on the target \TeX{} distribution
and one prefers not to ship it,
it is conceivable to paste a few relevant commands into the sources.

To that end, drop all statements |\input{childdoc.def}|
and perform the replacements as outlined below.
Instead of |\childdocmain{|\textit{main}|}| add the following code
to the top of the main file:
%
\begin{center}
\begin{tabular}{l}
|\||ifdefined\childdocname\endinput\||fi\newif\ifchilddoc|\\
|\edef\childdocname{\scantokens\expandafter{\jobname\noexpand}}|\\
|\def\childdocmain{|\textit{main}|}\||ifx\childdocmain\childdocname\||else|\\
|\childdoctrue\includeonly{\childdocname}\let\jobname\childdocmain\||fi|\\
\end{tabular}
\end{center}
%
Instead of |\childdocof{|\textit{main}|}| just include the main file
at the top of each child file:
%
\begin{center}
|\input{|\textit{main}|}|
\end{center}
%
A simple redirection |\childdocforward{|\textit{dest}|}| is achieved by:
%
\begin{center}
|\def\jobname{|\textit{dest}|}\input{\jobname}|
\end{center}
%
The redirection with prefix
|\childdocforwardprefix[|\textit{prefix}|]{|\textit{dest}|}|
is accomplished by:
%
\begin{center}
\begin{tabular}{l}
|{\edef\jobname{\scantokens\expandafter{\jobname\noexpand}}|\\
|\def\redirectjob |\textit{prefix}|#1~~~{\gdef\jobname{|\textit{dest}|#1}}|\\
|\expandafter\redirectjob\jobname~~~}\input{\jobname}|
\end{tabular}
\end{center}

In an alternative approach,
child documents can be compiled by a specific command line
without additional code or specific definitions:
%
\begin{center}
|... -jobname "|\textit{target}|" "|[\textit{flags}]%
|\includeonly{|\textit{dest}|}\input{|\textit{main}|}"|
\end{center}
%

%%%%%%%%%%%%%%%%%%%%%%%%%%%%%%%%%%%%%%%%%%%%%%%%%%%%%%%%%%%%%%%%%%%%%%%%%%%%%%%%
%%%%%%%%%%%%%%%%%%%%%%%%%%%%%%%%%%%%%%%%%%%%%%%%%%%%%%%%%%%%%%%%%%%%%%%%%%%%%%%%
\section{Information}

%%%%%%%%%%%%%%%%%%%%%%%%%%%%%%%%%%%%%%%%%%%%%%%%%%%%%%%%%%%%%%%%%%%%%%%%%%%%%%%%
\subsection{Copyright}

Copyright \copyright{} 2017--2018 Niklas Beisert

This work may be distributed and/or modified under the
conditions of the \LaTeX{} Project Public License, either version 1.3
of this license or (at your option) any later version.
The latest version of this license is in
  \url{http://www.latex-project.org/lppl.txt}
and version 1.3 or later is part of all distributions of \LaTeX{}
version 2005/12/01 or later.

This work has the LPPL maintenance status `maintained'.

The Current Maintainer of this work is Niklas Beisert.

This work consists of the files |README.txt|, |childdoc.ins| and |childdoc.dtx|
as well as the derived files |childdoc.def|, |cdocsamp.tex|
with |cdocsch1.tex|, |cdocsch2.tex|, |cdocspt3.tex|, |cdocspt4.tex|,
|cdocsdrf.tex|, |cdocsfn1.tex|, |cdocsfn2.tex|
as well as |childdoc.pdf|.

%%%%%%%%%%%%%%%%%%%%%%%%%%%%%%%%%%%%%%%%%%%%%%%%%%%%%%%%%%%%%%%%%%%%%%%%%%%%%%%%
\subsection{Files and Installation}

The package consists of the files:
%
\begin{center}
\begin{tabular}{ll}
    |README.txt|   & readme file \\
    |childdoc.ins| & installation file \\
    |childdoc.dtx| & source file \\
    |childdoc.def| & definition file \\
    |cdocsamp.tex| & sample main file \\
    |cdocsch1.tex| & sample include file \\
    |cdocsch2.tex| & sample include file \\
    |cdocspt3.tex| & sample part file \\
    |cdocspt4.tex| & sample part file \\
    |cdocsdrf.tex| & sample redirection file \\
    |cdocsfn1.tex| & sample redirection file \\
    |cdocsfn2.tex| & sample redirection file \\
    |childdoc.pdf| & manual
\end{tabular}
\end{center}
%
The distribution consists of the files
|README.txt|, |childdoc.ins| and |childdoc.dtx|.
%
\begin{itemize}
\item
Run (pdf)\LaTeX{} on |childdoc.dtx|
to compile the manual |childdoc.pdf| (this file).
\item
Run \LaTeX{} on |childdoc.ins| to create the definitions file |childdoc.def|
and the sample |cdocsamp.tex| with include files
|cdocsch1.tex|, |cdocsch2.tex|, |cdocspt3.tex|, |cdocspt4.tex|,
|cdocsdrf.tex|, |cdocsfn1.tex|, |cdocsfn2.tex|.
Then copy the file |childdoc.def| to an appropriate directory of your \LaTeX{}
distribution, e.g.\ \textit{texmf-root}|/tex/latex/childdoc|.
\end{itemize}

%%%%%%%%%%%%%%%%%%%%%%%%%%%%%%%%%%%%%%%%%%%%%%%%%%%%%%%%%%%%%%%%%%%%%%%%%%%%%%%%
\subsection{Related CTAN Packages}

There are several other packages which offer a similar functionality:
%
\begin{itemize}
\item
The packages
\href{http://ctan.org/pkg/docmute}{\textsf{docmute}},
\href{http://ctan.org/pkg/includex}{\textsf{includex}} and
\href{http://ctan.org/pkg/standalone}{\textsf{standalone}}
provide commands to include only the document body of
a child file thus allowing both files to be compiled individually.
\item
The packages \href{http://ctan.org/pkg/subdocs}{\textsf{subdocs}}
and \href{http://ctan.org/pkg/subfiles}{\textsf{subfiles}}
provide structures in which the main and child documents can be
encapsulated and allowing them to be compiled individually.
The inclusion mechanism is different from the conventional |\include|.
\item
The package \href{http://ctan.org/pkg/combine}{\textsf{combine}}
is an elaborate solution to combine several documents into one.
\end{itemize}
%
See also the CTAN topic \href{http://ctan.org/topic/subdocs}{\textsf{subdocs}}
for further related packages.
The present package differs from the above solutions in that
a document structure constructed with the conventional |\include| mechanism
just needs two extra commands at the top of every file
such that all constituent files can be compiled individually.

%%%%%%%%%%%%%%%%%%%%%%%%%%%%%%%%%%%%%%%%%%%%%%%%%%%%%%%%%%%%%%%%%%%%%%%%%%%%%%%%
%\subsection{Feature Suggestions}
%
%The following is a list of features which may be useful for future
%versions of this package:
%%
%\begin{itemize}
%\item
%\ldots
%\end{itemize}

%%%%%%%%%%%%%%%%%%%%%%%%%%%%%%%%%%%%%%%%%%%%%%%%%%%%%%%%%%%%%%%%%%%%%%%%%%%%%%%%
\subsection{Revision History}

%%%%%%%%%%%%%%%%%%%%%%%%%%%%%%%%%%%%%%%%
\paragraph{v2.0:} 2018/12/30

\begin{itemize}
\item
immediate forward processing
\item
added |\childdocby| mechanism
\item
manual restructured
\end{itemize}

%%%%%%%%%%%%%%%%%%%%%%%%%%%%%%%%%%%%%%%%
\paragraph{v1.6:} 2018/01/17

\begin{itemize}
\item
application for development of include files
\item
corrections to manual
\end{itemize}

%%%%%%%%%%%%%%%%%%%%%%%%%%%%%%%%%%%%%%%%
\paragraph{v1.5:} 2017/05/21

\begin{itemize}
\item
more complete structuring introduced
\item
|\childdocof| introduced
\item
|\childdoc| renamed to |\childdocmain|
\item
|\childredirect| renamed to |\childdocforward| and |\childdocforwardprefix|
and functionality expanded
\end{itemize}

%%%%%%%%%%%%%%%%%%%%%%%%%%%%%%%%%%%%%%%%
\paragraph{v1.0:} 2017/04/27

\begin{itemize}
\item
manual and install package
\item
first version published on CTAN
\end{itemize}

%%%%%%%%%%%%%%%%%%%%%%%%%%%%%%%%%%%%%%%%
\paragraph{v0.6:} 2017/04/26

\begin{itemize}
\item
redirection mechanism added
\end{itemize}

%%%%%%%%%%%%%%%%%%%%%%%%%%%%%%%%%%%%%%%%
\paragraph{v0.5:} 2017/04/26

\begin{itemize}
\item
functionality in definition file
\end{itemize}


%%%%%%%%%%%%%%%%%%%%%%%%%%%%%%%%%%%%%%%%%%%%%%%%%%%%%%%%%%%%%%%%%%%%%%%%%%%%%%%%
%%%%%%%%%%%%%%%%%%%%%%%%%%%%%%%%%%%%%%%%%%%%%%%%%%%%%%%%%%%%%%%%%%%%%%%%%%%%%%%%
%%%%%%%%%%%%%%%%%%%%%%%%%%%%%%%%%%%%%%%%%%%%%%%%%%%%%%%%%%%%%%%%%%%%%%%%%%%%%%%%
\appendix

\settowidth\MacroIndent{\rmfamily\scriptsize 000\ }

 \DocInput{childdoc.dtx}

\end{document}
%</driver>
% \fi
%
% %%%%%%%%%%%%%%%%%%%%%%%%%%%%%%%%%%%%%%%%%%%%%%%%%%%%%%%%%%%%%%%%%%%%%%%%%%%%%%
% %%%%%%%%%%%%%%%%%%%%%%%%%%%%%%%%%%%%%%%%%%%%%%%%%%%%%%%%%%%%%%%%%%%%%%%%%%%%%%
% \section{Sample}
%\iffalse
%<*samplemain>
%\fi
%
% The following presents a sample document
% with two chapters, two parts, a title page,
% a compile flag as well as three forwarding files to set the flag.
% It consists of eight |.tex| files:
% \begin{center}
% \begin{tabular}{ll}
% |cdocsamp.tex|&main file\\
% |cdocsch1.tex|&include file for chapter 1\\
% |cdocsch2.tex|&include file for chapter 2\\
% |cdocspt3.tex|&include file for part 3\\
% |cdocspt4.tex|&include file for part 4\\
% |cdocsdrf.tex|&forwarding file for main file in draft mode\\
% |cdocsfi1.tex|&forwarding file for final version of chapter 1\\
% |cdocsfi2.tex|&forwarding file for final version of chapter 2\\
% \end{tabular}
% \end{center}
% Each of the eight files can be compiled directly by the \LaTeX{} compiler.
%
% %%%%%%%%%%%%%%%%%%%%%%%%%%%%%%%%%%%%%%
% \paragraph{Main File.}
%
% The main file is called |cdocsamp.tex|.
%
% Load the \textsf{childdoc} definitions and
% declare the filename for the main document:
%    \begin{macrocode}
\input{childdoc.def}
\childdocmain{}
%    \end{macrocode}

% Optional override for |\version| flag:
%    \begin{macrocode}
%%\ifchilddoc\else\providecommand{\version}{draft}\fi
%    \end{macrocode}

% Define the default values for the |\version| flag
% (|final| for the main file and |draft| for childs):
%    \begin{macrocode}
\ifchilddoc
\providecommand{\version}{draft}
\else
\providecommand{\version}{final}
\fi
%    \end{macrocode}

% Load the standard document class:
%    \begin{macrocode}
\documentclass[12pt]{article}
%    \end{macrocode}

% Start the document body:
%    \begin{macrocode}
\begin{document}
%    \end{macrocode}

% Declare a title page.
% Print title, part of document being processed and version flag:
%    \begin{macrocode}
\addtocounter{page}{-1}
\begin{center}
{\LARGE\bfseries{}childdoc example\par}
\vspace{1cm}
\ifchilddoc
\ifchilddocmanual part\else chapter\fi:
`\childdocname' of `\childdocjob'\par
\else
main document: `\childdocjob'\par
\fi
version: \version\par
\end{center}
\newpage
%    \end{macrocode}

% Manually include selected file,
% otherwise process as usual:
%    \begin{macrocode}
\ifchilddocmanual
\section*{part `\childdocname'}
\input{\childdocname}
\else
%    \end{macrocode}

% Include the two chapters:
%    \begin{macrocode}
\include{cdocsch1}
\include{cdocsch2}
%    \end{macrocode}

% Include the two parts unless only chapters should be displayed:
%    \begin{macrocode}
\ifchilddoc\else
\section{part three}
\input{cdocspt3}
\section{part four}
\input{cdocspt4}
\fi
%    \end{macrocode}

% Process as usual until here:
%    \begin{macrocode}
\fi
%    \end{macrocode}

% End of document body:
%    \begin{macrocode}
\end{document}
%    \end{macrocode}
%\iffalse
%</samplemain>
%\fi
%
% %%%%%%%%%%%%%%%%%%%%%%%%%%%%%%%%%%%%%%
% \paragraph{Chapter Include Files.}
%
% The include files are called |cdocsch1.tex| and |cdocsch2.tex|.
%
%\iffalse
%<*samplechap1|samplechap2>
%\fi

% Optional override for |\version| flag:
%    \begin{macrocode}
%%\providecommand{\version}{final}
%    \end{macrocode}

% Include the main document:
%    \begin{macrocode}
\input{childdoc.def}
\childdocof{cdocsamp}
%    \end{macrocode}

%\iffalse
%</samplechap1|samplechap2>
%\fi
%
%\iffalse
%<*samplechap1>
%\fi
% Some text for chapter 1:
%    \begin{macrocode}
\section{one}
some text in chapter one
%    \end{macrocode}

%\iffalse
%</samplechap1>
%\fi
% Some text for chapter 2:
%\iffalse
%<*samplechap2>
%\fi
%    \begin{macrocode}
\section{two}
more text in chapter two
%    \end{macrocode}

%\iffalse
%</samplechap2>
%\fi
%
% %%%%%%%%%%%%%%%%%%%%%%%%%%%%%%%%%%%%%%
% \paragraph{Part Include Files.}
%
% The include files are called |cdocspt3.tex| and |cdocspt4.tex|.
%
%\iffalse
%<*samplepart3|samplepart4>
%\fi

% Optional override for |\version| flag:
%    \begin{macrocode}
%%\providecommand{\version}{final}
%    \end{macrocode}

% Include the main document:
%    \begin{macrocode}
\input{childdoc.def}
\childdocby{cdocsamp}
%    \end{macrocode}

%\iffalse
%</samplepart3|samplepart4>
%\fi
%
%\iffalse
%<*samplepart3>
%\fi
% Some text for part 3:
%    \begin{macrocode}
some text in part three
%    \end{macrocode}

%\iffalse
%</samplepart3>
%\fi
% Some text for part 4:
%\iffalse
%<*samplepart4>
%\fi
%    \begin{macrocode}
more text in part four
%    \end{macrocode}

%\iffalse
%</samplepart4>
%\fi
%
% %%%%%%%%%%%%%%%%%%%%%%%%%%%%%%%%%%%%%%
% \paragraph{Forwarding for a Complete Draft.}
%
% The following forwarding file |cdocsdrf.tex|
% compiles the main document in draft mode:
%\iffalse
%<*sampledraft>
%\fi
%    \begin{macrocode}
\def\version{draft}
\input{childdoc.def}
\childdocforward{cdocsamp}
%    \end{macrocode}

%\iffalse
%</sampledraft>
%\fi
%
% %%%%%%%%%%%%%%%%%%%%%%%%%%%%%%%%%%%%%%
% \paragraph{Forwarding for Final Version of the Chapters.}
%
% The following forwarding files |cdocsfn1.tex| and |cdocsfn2.tex|
% (with identical content)
% compile the final versions of the child documents
% |cdocsch1.tex| and |cdocsch2.tex|, respectively:
%\iffalse
%<*samplefinal>
%\fi
%    \begin{macrocode}
\def\version{final}
\input{childdoc.def}
\childdocforwardprefix[cdocsamp]{cdocsfn}{cdocsch}
%    \end{macrocode}

%\iffalse
%</samplefinal>
%\fi
%
% %%%%%%%%%%%%%%%%%%%%%%%%%%%%%%%%%%%%%%
% \paragraph{Command Line Processing.}
%
% The following three command lines generate the output files
% |cdocscld|, |cdocscl1| and |cdocscl2|
% which should be identical to
% |cdocsdrf|, |cdocsch1| and |cdocsfn2|, respectively:
% \begin{center}
% \begin{tabular}{l}
% |latex -jobname cdocscld \|\\
% |  "\def\version{draft}\input{childdoc.def}\childdocforward{cdocsamp}"|\\
% |latex -jobname cdocscl1 \|\\
% |  "\input{childdoc.def}\childdocforward[cdocsamp]{cdocsch1}"|\\
% |latex -jobname cdocscl2 \|\\
% |  "\def\version{final}\input{childdoc.def}\childdocforward{cdocsch2}"|
% \end{tabular}
% \end{center}
% Note that the trailing backslash on each first line
% merely continues the input to the second line
% (for convenient cut ant paste).
% Furthermore, the command |latex| can be replaced by any
% of its alternative versions such as |pdflatex|.
%
% %%%%%%%%%%%%%%%%%%%%%%%%%%%%%%%%%%%%%%%%%%%%%%%%%%%%%%%%%%%%%%%%%%%%%%%%%%%%%%
% %%%%%%%%%%%%%%%%%%%%%%%%%%%%%%%%%%%%%%%%%%%%%%%%%%%%%%%%%%%%%%%%%%%%%%%%%%%%%%
% \section{Implementation}
%\iffalse
%<*package>
%\fi
%
% This section describes the definitions file |childdoc.def|.

% The definitions cannot be loaded using |\usepackage| or |\RequirePackage|
% which has a mechanism to prevent loading a style file more than once.
% When loading the definitions by means of |\input|
% multiple instances have to be prevented manually:
%\iffalse
%This code needs to be before the `\ProvidesFile' directive
%which is defined at the beginning of this file.
%Therefore it is also placed there and commented out here.
%</package>
%<*discard>
%\fi
%    \begin{macrocode}
\ifdefined\childdocmain\endinput\fi
%    \end{macrocode}
%\iffalse
%</discard>
%<*package>
%\fi
%
% \macro{\ifchilddoc}
% \macro{\ifchilddocmanual}
% The conditional |\ifchilddoc| tells whether a
% child (true) or main (false) document is being compiled.
% The conditional |\ifchilddocmanual| tells whether
% the |\includeonly| mechanism is used (false) or
% the selection of child files must be performed manually (true).
% The definitions initialise to false:
%    \begin{macrocode}
\newif\ifchilddoc
\newif\ifchilddocmanual
%    \end{macrocode}

% \macro{\childdocname}
% \macro{\childdocjob}
% The macro |\childdocname| stores the name of the main document
% to be compiled. The macro |\childdocjob| stores the name of
% the document on which the \LaTeX{} compiler was originally invoked.
% The content of |\jobname| cannot be compared
% to filenames specified in the source due to different catcodes.
% The following code rescans |\jobname|, stores the result
% in |\childdocname| and saves a copy in |\childdocjob|:
%    \begin{macrocode}
\edef\childdocname{\scantokens\expandafter{\jobname\noexpand}}
\let\childdocjob\childdocname
%    \end{macrocode}

% \macro{\childdocdisable}
% The macro |\childdocdisable| prevents the main file
% from being processed more than once.
% At this stage, the main document command |\childdocmain|
% is assumed to be called once again where it should do nothing.
% Any subsequent call to it should prevent
% a secondary processing of the main document
% It overwrites the forwarding commands
% |\childdocof| and |\childdocforward|
% with empty macros to prevent further inclusions of the main document:
%    \begin{macrocode}
\newcommand{\childdocdisable}
{
  \renewcommand{\childdocmain}[1]{\renewcommand{\childdocmain}[1]{\endinput}}
  \renewcommand{\childdocof}[1]{}
  \renewcommand{\childdocby}[2][]{}
  \renewcommand{\childdocforward}[2][]{}
  \renewcommand{\childdocdisable}{}
}
%    \end{macrocode}

% \macro{\childdocmain}
% The macro |\childdocmain| is to be called at the top of the main file
% with nothing or the main filename (without extension) as argument.
% First, it breaks loops.
% If the argument is not empty and does not match |\childdocname|
% (which is set by the first inclusion of |childdoc.def|),
% |\ifchilddoc| is set to true, |\includeonly| is applied to the child file
% and |\jobname| is set to the main file
% (for proper handling of |.aux| files):
%    \begin{macrocode}
\newcommand{\childdocmain}[1]
{
  \childdocdisable\childdocmain{}
  \if?#1?\else
    \begingroup
      \def\childdoctmp{#1}
      \ifx\childdoctmp\childdocname
        \def\childdoctmp{}
      \else
        \def\childdoctmp
        {
          \childdoctrue
          \includeonly{\childdocname}
          \def\childdocjob{#1}
          \def\jobname{#1}
        }
      \fi
      \expandafter
    \endgroup
    \childdoctmp
  \fi
}
%    \end{macrocode}

% \macro{\childdocof}
% The command |\childdocof| redirects
% compilation to the main file |#1|.
%    \begin{macrocode}
\newcommand{\childdocof}[1]
{
  \childdocdisable
  \childdoctrue
  \includeonly{\childdocname}
  \def\jobname{#1}
  \def\childdocjob{#1}
  \input{#1}
}
%    \end{macrocode}

% \macro{\childdocby}
% The command |\childdocby| ....
%    \begin{macrocode}
\newcommand{\childdocby}[2][]
{
  \childdocdisable
  \childdoctrue
  \childdocmanualtrue
  \if?#1?\else
    \def\jobname{#2}
  \fi
  \def\childdocjob{#2}
  \input{#2}
  \endinput
}
%    \end{macrocode}

% \macro{\childdocforward}
% The command |\childdocforward| redirects
% compilation to the main file or
% (if the optional argument is given) a child file.
% Parameters are set as if the main file
% or a child file starting with |\childdocof| was compiled.
% Then compilation is handed over to the main file:
%    \begin{macrocode}
\newcommand{\childdocforward}[2][]
{
  \begingroup
    \if?#1?
      \def\childdoctmp
      {
        \def\childdocname{#2}
        \def\childdocjob{#2}
        \def\jobname{#2}
        \input{#2}
        \endinput
      }
    \else
      \def\childdoctmp
      {
        \childdocdisable
        \def\childdocname{#2}
        \childdoctrue
        \includeonly{#2}
        \def\childdocjob{#1}
        \def\jobname{#1}
        \input{#1}
        \endinput
      }
    \fi
    \expandafter
  \endgroup
  \childdoctmp
}
%    \end{macrocode}

% \macro{\childdocforwardprefix}
% The command |\childdocforwardprefix| redirects
% compilation to the main or a child file by means of a pattern.
% The prefix |#1| in the current filename is replaced by |#2|
% and the suffix of the current filename is kept
% (it is assumed that the filename does not contain the substring `|~~~|'
% which is used as a delimiter).
% Compilation is handed over to the new file by |\childdocforward|:
%    \begin{macrocode}
\newcommand{\childdocforwardprefix}[3][]
{
  \begingroup
    \def\childdocextract #2##1~~~{\def\childdoctmp{\childdocforward[#1]{#3##1}}}
    \expandafter\childdocextract\childdocname~~~
    \expandafter
  \endgroup
  \childdoctmp
}
%    \end{macrocode}

% \macro{\childdoc}
% The deprecated macro |\childdoc| is a legacy version of |\childdocmain|:
%    \begin{macrocode}
\newcommand{\childdoc}{\childdocmain}
%    \end{macrocode}

% \macro{\childdocredirect}
% The deprecated macro |\childdocredirect| is a legacy version
% of |\childdocforward| and |\childdocforwardprefix|:
%    \begin{macrocode}
\newcommand{\childdocredirect}[2][]
{
  \begingroup
    \if?#1?
      \def\childdoctmp{\childdocforward{#2}}
    \else
      \def\childdoctmp{\childdocforwardprefix{#1}{#2}}
    \fi
    \expandafter
  \endgroup
  \childdoctmp
}
%    \end{macrocode}

%\iffalse
%</package>
%\fi
%
\endinput
|\\
|\childdocforward{|\textit{main}|}|
\end{tabular}
\end{center}
%
Likewise, the following files |final|\textit{nn}|.tex|
compile the final version of the child document
|child|\textit{nn}|.tex|:
%
\begin{center}
\begin{tabular}{l}
|\def\version{final}|\\
|% \iffalse
%
% childdoc.dtx Copyright (C) 2017-2018 Niklas Beisert
%
% This work may be distributed and/or modified under the
% conditions of the LaTeX Project Public License, either version 1.3
% of this license or (at your option) any later version.
% The latest version of this license is in
%   http://www.latex-project.org/lppl.txt
% and version 1.3 or later is part of all distributions of LaTeX
% version 2005/12/01 or later.
%
% This work has the LPPL maintenance status `maintained'.
%
% The Current Maintainer of this work is Niklas Beisert.
%
% This work consists of the files childdoc.dtx and childdoc.ins
% and the derived files childdoc.def and cdocsamp.tex with
% cdocsch1.tex, cdocsch2.tex, cdocsdrf.tex, cdocsfn1.tex, cdocsfn2.tex.
%
%<package>\ifdefined\childdocmain\endinput\fi
%<package>\ProvidesFile{childdoc.def}[2018/12/30 v2.0 child document driver]
%<samplemain>\ProvidesFile{cdocsamp.tex}[2018/12/30 v2.0 sample for childdoc]
%<*driver>
%\ProvidesFile{childdoc.drv}[2018/12/30 v2.0 childdoc reference manual file]
\PassOptionsToClass{10pt,a4paper}{article}
\documentclass{ltxdoc}

\usepackage[margin=35mm]{geometry}
\usepackage{hyperref}
\usepackage{hyperxmp}
\usepackage[usenames]{color}

\hypersetup{colorlinks=true}
\hypersetup{pdfstartview=FitH}
\hypersetup{pdfpagemode=UseNone}
\hypersetup{pdfsource={}}
\hypersetup{pdflang={en-UK}}
\hypersetup{pdfcopyright={Copyright 2017-2018 Niklas Beisert.
  This work may be distributed and/or modified under the
  conditions of the LaTeX Project Public License, either version 1.3
  of this license or (at your option) any later version.}}
\hypersetup{pdflicenseurl={http://www.latex-project.org/lppl.txt}}
\hypersetup{pdfcontactaddress={ETH Zurich, ITP, HIT K,
  Wolfgang-Pauli-Strasse 27}}
\hypersetup{pdfcontactpostcode={8093}}
\hypersetup{pdfcontactcity={Zurich}}
\hypersetup{pdfcontactcountry={Switzerland}}
\hypersetup{pdfcontactemail={nbeisert@itp.phys.ethz.ch}}
\hypersetup{pdfcontacturl={http://people.phys.ethz.ch/\xmptilde nbeisert/}}

\newcommand{\secref}[1]{\hyperref[#1]{section \ref*{#1}}}

\parskip1ex
\parindent0pt
\let\olditemize\itemize
\def\itemize{\olditemize\parskip0pt}

\begin{document}

\title{The \textsf{childdoc} Package}
\hypersetup{pdftitle={The childdoc Package}}
\author{Niklas Beisert\\[2ex]
  Institut f\"ur Theoretische Physik\\
  Eidgen\"ossische Technische Hochschule Z\"urich\\
  Wolfgang-Pauli-Strasse 27, 8093 Z\"urich, Switzerland\\[1ex]
  \href{mailto:nbeisert@itp.phys.ethz.ch}
  {\texttt{nbeisert@itp.phys.ethz.ch}}}
\hypersetup{pdfauthor={Niklas Beisert}}
\hypersetup{pdfsubject={Manual for the LaTeX2e Package childdoc}}
\date{30 December 2018, \textsf{v2.0}}
\maketitle

\begin{abstract}\noindent
\textsf{childdoc} is a \LaTeXe{} package
that enables the direct compilation
of document sections included by |\include|
to individual files.
\end{abstract}

\begingroup
\parskip0ex
\tableofcontents
\endgroup

%%%%%%%%%%%%%%%%%%%%%%%%%%%%%%%%%%%%%%%%%%%%%%%%%%%%%%%%%%%%%%%%%%%%%%%%%%%%%%%%
%%%%%%%%%%%%%%%%%%%%%%%%%%%%%%%%%%%%%%%%%%%%%%%%%%%%%%%%%%%%%%%%%%%%%%%%%%%%%%%%
\section{Introduction}

\LaTeX{} provides a mechanism to structure a large document (such as a book)
into a main file and several child files (containing the chapters)
using the |\include| command.
This mechanism is beneficial for documents
which span hundreds of pages in order to
make the source file(s) more manageable.
Moreover, compilation can be restricted to
selected child files by means of the |\includeonly| command.
The latter feature can be used to reduce the compilation time while editing
(this was significantly more useful in the earlier days of \LaTeX{})
or to generate a smaller document which is easier to navigate.
Another application of |\includeonly| is to generate
documents consisting of selected parts of the complete document.

However, there are a few drawbacks of the plain |\include| mechanism:
\begin{itemize}
\item
The child files cannot be compiled on their own,
they can only be compiled via the main file.
A naive editing environment
(such as a text editor with an option
to have the current file processed by \LaTeX)
may require one to switch to the main file before compiling;
attempting to compile the child file produces errors.
\item
The main file must be modified (each time)
to adjust the |\includeonly| command
to the present needs. This easily leaves the main file in a messy state.
\item
The generated document will always carry the filename
of the main document. This is inconvenient if
several child files are to be compiled and
to be kept for distribution.
\end{itemize}

The present package provides a simple interface
to make child files individually compilable by \LaTeX{}.
Compiling a child file then has the same effect as compiling
the main file with an |\includeonly| command
to select the appropriate child.
Moreover the generated document will carry the name of the child
rather than the main file.
This resolves all three above issues.

This feature is meant to make the editing of books,
thesis documents and lecture notes somewhat more convenient.
However, the package can also be used efficiently for
composing a series of documents (such as exercise sheets)
which are typically distributed individually.
It then assists the author in generating the individual documents
(potentially in different versions)
as well as a document containing the collected series.
Another application is in developing style files
or other kinds of included material
where compilation of the style file could redirect
to a sample or test file.

%%%%%%%%%%%%%%%%%%%%%%%%%%%%%%%%%%%%%%%%%%%%%%%%%%%%%%%%%%%%%%%%%%%%%%%%%%%%%%%%
%%%%%%%%%%%%%%%%%%%%%%%%%%%%%%%%%%%%%%%%%%%%%%%%%%%%%%%%%%%%%%%%%%%%%%%%%%%%%%%%
\section{Usage}

First of all, the package \textsf{childdoc} is \emph{not} a standard
\LaTeXe{} |.sty| style file! Therefore it needs to be invoked in
a non-standard way.

%%%%%%%%%%%%%%%%%%%%%%%%%%%%%%%%%%%%%%%%%%%%%%%%%%%%%%%%%%%%%%%%%%%%%%%%%%%%%%%%
\subsection{Included Files}
\label{sec:include}

%%%%%%%%%%%%%%%%%%%%%%%%%%%%%%%%%%%%%%%%
\DescribeMacro{\childdocmain}
To use the package, add the commands
\begin{center}
\begin{tabular}{l}
|\input{childdoc.def}|\\
|\childdocmain{}|\\
\end{tabular}
\end{center}
at the very top of the main \LaTeX{} file,
in particular \emph{before} the |\documentclass| statement!
The argument of |\childdocmain| should be left empty
(but it must be present).

%%%%%%%%%%%%%%%%%%%%%%%%%%%%%%%%%%%%%%%%
\DescribeMacro{\childdocof}
Furthermore, add the commands
\begin{center}
\begin{tabular}{l}
|\input{childdoc.def}|\\
|\childdocof{|\textit{main}|}|\\
\end{tabular}
\end{center}
at the top of every child file \textit{child}
which is included by |\include{|\textit{child}|}|
from within the main file
(or at least for those files to be compiled individually).
The argument \textit{main} must be the filename of the main file.

There are a couple of
considerations in setting up the main and child documents:

%%%%%%%%%%%%%%%%%%%%%%%%%%%%%%%%%%%%%%%%
\paragraph{Restrictions.}

Please note the following restrictions:
\begin{itemize}
\item
|\childdocmain| must be called with one argument \textit{main}
to ensure compatibility with earlier version of the package.
It must either be empty (|\childdocmain{}|)
or precisely match the filename of the main file in which it is specified.
See \secref{sec:detection} for further information.
\item
The filename \textit{main} must be specified without the |.tex| extension.
\item
The filename \textit{main} is case sensitive
(even in case-insensitive file systems)
due to internal string comparison.
\item
The argument \textit{main} should be fully expanded, it cannot be a macro.
\item
Subdirectories and special characters should be avoided in filenames.
\item
The command |\childdocmain{|\textit{main}|}| must be followed by a whitespace.
It should not be followed immediately by another command
or by a comment mark `|%|'.
This is because the \TeX{} parser reads the token immediately following
the argument of |\childdocmain| and puts it
at the beginning of every child section;
however, a white\-space is ignored.
\end{itemize}

%%%%%%%%%%%%%%%%%%%%%%%%%%%%%%%%%%%%%%%%
\paragraph{Content of Main File.}

It is advisable to place all content in the child files included by |\include|.
Any output contained in the main file will appear in all child documents
unless suppressed manually;
it cannot be suppressed automatically by the |\includeonly| directive
and thus should normally be avoided.
A method to include some content in the main file
by means of conditional processing is described in \secref{sec:conditional}.

%%%%%%%%%%%%%%%%%%%%%%%%%%%%%%%%%%%%%%%%
\paragraph{Page Numbering.}

When only a part of the document is compiled,
the appropriate numbering of pages
(as well as other status parameters)
is determined from the |.aux| files.
The latter contain information from previous passes.
However this information needs to propagate through
all intermediate child documents.
Therefore the page numbering in child documents may well
be inconsistent until the complete document is compiled at least once.

A useful (if unconventional) way to always ensure a consistent
page numbering is to restart the numbering in each child document
and denote the pages by `\textit{child}|.|\textit{page}'
where \textit{child} represents the chapter/section number of the child file.
This can be achieved by the command
|\numberwithin{page}{|\textit{child}|}|
of the \textsf{amsmath} package
where \textit{child} can be |chapter| or |section|
depending on the chosen structuring.
Alternatively, one can modify the macro |\thepage| appropriately
and reset the counter |page| at the start of each child file.

%%%%%%%%%%%%%%%%%%%%%%%%%%%%%%%%%%%%%%%%%%%%%%%%%%%%%%%%%%%%%%%%%%%%%%%%%%%%%%%%
\subsection{Conditional Processing}
\label{sec:conditional}

The package provides a mechanism to compile different versions
of a document. To customise the versions further some conditional processing
can come in handy to distinguish which version is being compiled.
The package provides two macros to describe the compilation context:

%%%%%%%%%%%%%%%%%%%%%%%%%%%%%%%%%%%%%%%%
\DescribeMacro{\ifchilddoc}
The conditional |\ifchilddoc| distinguishes between the compilation of
child documents and the main document:
%
\begin{center}
|\ifchilddoc |\textit{child-code}| |[|\||else |\textit{main-code}]| \||fi|
\end{center}

%%%%%%%%%%%%%%%%%%%%%%%%%%%%%%%%%%%%%%%%
\DescribeMacro{\childdocname}
\DescribeMacro{\childdocjob}
The macro |\childdocname| contains the filename (without extension)
of the main or child file being processed.
Note that |\childdocjob| will always contain the name of the main file.

%%%%%%%%%%%%%%%%%%%%%%%%%%%%%%%%%%%%%%%%
\paragraph{Title Page.}

Conditional processing can be used to include a title or banner page
in the main document when proper precautions are taken.
Importantly, the code in the main file should ensure that the page counter
(as well as other status parameters which are stored in the |.aux| files)
takes the same value after the conditional processing.
Otherwise the page numbers may take divergent values
depending on which part is compiled.

For example, a title page could be declared by:
%
\begin{center}
\begin{tabular}{l}
|\ifchilddoc\||else|\\
|\addtocounter{page}{-1}|\\
\textit{code for title page}\\
|\newpage|\\
|\||fi|
\end{tabular}
\end{center}
%
A banner page for the child documents can be generated by:
%
\begin{center}
\begin{tabular}{l}
|\ifchilddoc|\\
|\addtocounter{page}{-1}|\\
\textit{code for banner page}\\
|\newpage|\\
|\||fi|
\end{tabular}
\end{center}
%
Here one could write a message such as:
\begin{center}
|This is the part \childdocname{} of \childdocjob{}.|
\end{center}

%%%%%%%%%%%%%%%%%%%%%%%%%%%%%%%%%%%%%%%%%%%%%%%%%%%%%%%%%%%%%%%%%%%%%%%%%%%%%%%%
\subsection{Flags}
\label{sec:flags}

The package makes it easy to generate different versions
of the main or child documents.
To this end compilation flags can be defined
and assigned different default values.
They will be particularly useful in conjunction
with the forwarding mechanism described in \secref{sec:forward}.

For example, it may be useful to have a flag |\version|
which can be set to |draft| or |final|.
The document source will contain some conditional code
depending on the value of |\version|.
Suppose further, the flag should default to |final| for the main file
and to |draft| for child files
which is a natural assignment for editing the document.
This is achieved by placing the following code
in the preamble of the main document
(below the |\childdocmain| directive):
%
\begin{center}
\begin{tabular}{l}
|\ifchilddoc|\\
|\providecommand{\version}{draft}|\\
|\||else|\\
|\providecommand{\version}{final}|\\
|\||fi|
\end{tabular}
\end{center}
%
The definition by |\providecommand| makes sure
that previous definitions are not overwritten.
Further statements |\providecommand{\version}{...}|
can thus be added before the above code to override it.

For the main file, one might add a line
(between |\childdocmain| and the above block)
%
\begin{center}
|%\ifchilddoc\||else\providecommand{\version}{draft}\||fi|
\end{center}
%
which can be uncommented to produce a draft version.
Likewise one can add a line to the very top of a child file
(above the |\childdocof{|\textit{main}|}| directive)
%
\begin{center}
|%\providecommand{\version}{final}|
\end{center}
%
which can be uncommented to produce the final version of this child document.

%%%%%%%%%%%%%%%%%%%%%%%%%%%%%%%%%%%%%%%%%%%%%%%%%%%%%%%%%%%%%%%%%%%%%%%%%%%%%%%%
\subsection{Forwarding}
\label{sec:forward}

Different versions of the main or child documents
using compilation flags as described in \secref{sec:flags}
can be (permanently) stored in different files
for convenient compilation, viewing and distribution.
To this end, the package defines a command
to pass on compilation to a different file:

%%%%%%%%%%%%%%%%%%%%%%%%%%%%%%%%%%%%%%%%
\DescribeMacro{\childdocforward}
The command |\childdocforward| redirects processing to
another source file:
%
\begin{center}
\begin{tabular}{l}
|\input{childdoc.def}|\\
|\childdocforward[|\textit{main}|]{|\textit{dest}|}|\\
\end{tabular}
\end{center}
%
The argument \textit{dest} is the destination file
(without extension).
It should be the main file or one of the child files.
Note that further \textsf{childdoc} directives
such as |\childdocof| and |\childdocforward|
in the indicated file will be processed in this form.
The optional argument \textit{main}
passes on directly to the main file \textit{main}
while pretending to compile the child \textit{dest}.
This form behaves as if \textit{dest}
issues |\childdocof{|\textit{main}|}| right away,
and no further \textsf{childdoc} directives will be processed.

%%%%%%%%%%%%%%%%%%%%%%%%%%%%%%%%%%%%%%%%
\DescribeMacro{\...prefix}
In the alternative form |\childdocforwardprefix|,
%
\begin{center}
\begin{tabular}{l}
|\input{childdoc.def}|\\
|\childdocforwardprefix[|\textit{main}|]{|\textit{prefix}|}{|\textit{dest}|}|
\end{tabular}
\end{center}
%
the destination file is determined by a pattern
depending on the current file:
To make this work, the current file must be called
`{\textit{prefix}\hspace{0.2em}\textit{suffix}}'
with \textit{prefix} matching precisely the argument.
Processing is then passed on to the file
`{\textit{dest}\hspace{0.2em}\textit{suffix}}'.
Surely, the same effect is achieved by
directly specifying the
argument `{\textit{dest}\hspace{0.2em}\textit{suffix}}'
in the first form.
However, that requires to set up a different file
for each child. With the alternative form of the command
all these files can have exactly the same content
which simplifies setting them up and maintaining them.

For example, the following file |draft.tex|
with a compilation flag |\version| as described in \secref{sec:flags}
compiles the main document as a draft:
%
\begin{center}
\begin{tabular}{l}
|\def\version{draft}|\\
|\input{childdoc.def}|\\
|\childdocforward{|\textit{main}|}|
\end{tabular}
\end{center}
%
Likewise, the following files |final|\textit{nn}|.tex|
compile the final version of the child document
|child|\textit{nn}|.tex|:
%
\begin{center}
\begin{tabular}{l}
|\def\version{final}|\\
|\input{childdoc.def}|\\
|\childdocforwardprefix{final}{child}|
\end{tabular}
\end{center}
%

Note that when several versions of a main file and/or of each child file
are to be generated, it may be convenient to set up a |Makefile| or
shell script to automatise the process.

%%%%%%%%%%%%%%%%%%%%%%%%%%%%%%%%%%%%%%%%%%%%%%%%%%%%%%%%%%%%%%%%%%%%%%%%%%%%%%%%
\subsection{Command Line Processing}
\label{sec:commandline}

The effect of redirection files can also be achieved by invoking
the \LaTeX{} compiler with a more elaborate command line.
Most conveniently this should be done as part
of a shell script or a |Makefile|.

When using \textsf{childdoc} in the main file, the following
command lines effectively perform a redirection
(note that depending on the shell being used,
backslashes may have to be doubled: `|\|' $\to$ `|\\|'):
%
\begin{center}
|... -jobname "|\textit{target}|" |\\|"|[\textit{flags}]%
|\input{childdoc.def}\childdocforward[|\textit{main}|]{|\textit{dest}|}"|
\end{center}
%
Here \textit{target} is the name of the output file,
\textit{main} is the name of the main file
and \textit{dest} is the name of the main or child file to be processed
(all filenames without extensions).
The optional argument \textit{main} can be omitted
if \textit{main} matches \textit{dest}.
Optionally, compilation \textit{flags} can be defined via |\def| commands.
This command line makes the \TeX{} engine believe
it is compiling the file \textit{target}
whose content is specified as the latter parameter.
The provided code then forwards the processing to
\textit{main} or \textit{dest} as described in \secref{sec:forward}.

%%%%%%%%%%%%%%%%%%%%%%%%%%%%%%%%%%%%%%%%%%%%%%%%%%%%%%%%%%%%%%%%%%%%%%%%%%%%%%%%
\subsection{Include by Input}
\label{sec:input}

Including child documents by |\include| has some restrictions by design.
Most notably, the content of a child document always occupies
its own set of pages; pages cannot be shared between child documents.
Usually, this behaviour makes perfect sense
because each child document contain an essential part of the document.
However, in some situations it may be desirable to compose
a document from a collection of parts
without having mandatory page breaks between then.
For this case, the package
provides a mechanism to include parts
by |\input| which can also be processed individually.
However, by construction this mechanism
requires manual handling of the content to be output.

%%%%%%%%%%%%%%%%%%%%%%%%%%%%%%%%%%%%%%%%
\DescribeMacro{\ifchilddocmanual}
The main file should be prepared as usual, see \secref{sec:include}.
However, the document body must make a distinction
between processing of an individual part and of the main document, e.g.:
%
\begin{center}
\begin{tabular}{l}
|\ifchilddocmanual|\\
|\input{\childdocname}|\\
|\||else|\\
\textit{document body with }|\input{|\textit{part}|}|\\
|\||fi|
\end{tabular}
\end{center}
%
The conditional |\ifchilddocmanual| is true whenever
a part to be included by |\input| is being compiled,
and the name of the part is stored in |\childdocname|.

%%%%%%%%%%%%%%%%%%%%%%%%%%%%%%%%%%%%%%%%
\DescribeMacro{\childdocby}
Each part to be included by |\input| should start with:
%
\begin{center}
\begin{tabular}{l}
|\input{childdoc.def}|\\
|\childdocby{|\textit{main}|}|\\
\end{tabular}
\end{center}
%
The directive |\childdocby| is similar to |\childdocof|
described in \secref{sec:include},
but the subsequent selection of content must be done manually.
To that end, both |\ifchilddoc| and |\ifchilddocmanual|
will be true upon processing of a part,
and the name of the part is stored in |\childdocname|.
Note that |\jobname| will be set to the filename of the current part
so that each part receives an individual |.aux| file
that does not interfere with the |.aux| file(s) of the main document.
This behaviour can be altered by the alternative form
|\childdocby[*]{|\textit{main}|}| (with a non-empty optional argument)
which uses the |.aux| file of the main document
by setting |\jobname| to \textit{main}.

%%%%%%%%%%%%%%%%%%%%%%%%%%%%%%%%%%%%%%%%%%%%%%%%%%%%%%%%%%%%%%%%%%%%%%%%%%%%%%%%
\subsection{Driver Development}
\label{sec:driver}

The \textsf{childdoc} mechanism can also be use for the development
of definition files such as \LaTeX{} styles or classes.
This case differs from the above setup with multiple parts
included by |\include| in that no |\includeonly| should be invoked.
This can be achieved by starting the include file
(before |\ProvidesPackage|) with:
%
\begin{center}
\begin{tabular}{l}
|\input{childdoc.def}|\\
|\childdocforward{|\textit{main}|}|\\
\end{tabular}
\end{center}
%
or alternatively with:
%
\begin{center}
\begin{tabular}{l}
|\input{childdoc.def}|\\
|\childdocby{|\textit{main}|}|\\
\end{tabular}
\end{center}
%
Both forms have slightly different effects as described above.
The main file is prepared as usual, see \secref{sec:include}.

%%%%%%%%%%%%%%%%%%%%%%%%%%%%%%%%%%%%%%%%%%%%%%%%%%%%%%%%%%%%%%%%%%%%%%%%%%%%%%%%
\subsection{Legacy Detection}
\label{sec:detection}

The directive |\childdocmain| in the main file can detect
whether the complete document or merely a child is to be compiled
even without using the directive |\childdocof|.
This method is deprecated because it is less robust
and there is no compelling reason to use it;
it is merely provided for backward compatibility
and it may be removed in future versions.

If the detection mechanism is to be used,
it is mandatory to correctly specify
the filename of the main file as the argument of |\childdocmain|:
%
\begin{center}
\begin{tabular}{l}
|\input{childdoc.def}|\\
|\childdocmain{|\textit{main}|}|\\
\end{tabular}
\end{center}
%
If |\jobname| does not match the argument \textit{main} of |\childdocmain|,
it is assumed that |\jobname| points to the child file to be compiled.
When using |\childdocmain| with the main file specified as argument,
it suffices to start a child file
with just |\input{|\textit{main}|}|
without loading of the package and using |\childdocof|.
If instead all processing is done
with the appropriate \textsf{childdoc} directives,
the argument of \textit{main} of |\childdocmain| can be empty.

An alternative version of the command line processing described
in \secref{sec:commandline} using the detection mechanism reads:
%
\begin{center}
|... -jobname "|\textit{target}|" "|[\textit{flags}]%
[|\def\jobname{|\textit{dest}|}|]|\input{|\textit{main}|}"|
\end{center}

%%%%%%%%%%%%%%%%%%%%%%%%%%%%%%%%%%%%%%%%%%%%%%%%%%%%%%%%%%%%%%%%%%%%%%%%%%%%%%%%
\subsection{Manual Code}
\label{sec:manual}

In case one cannot be certain whether the definitions file |childdoc.def|
is installed on the target \TeX{} distribution
and one prefers not to ship it,
it is conceivable to paste a few relevant commands into the sources.

To that end, drop all statements |\input{childdoc.def}|
and perform the replacements as outlined below.
Instead of |\childdocmain{|\textit{main}|}| add the following code
to the top of the main file:
%
\begin{center}
\begin{tabular}{l}
|\||ifdefined\childdocname\endinput\||fi\newif\ifchilddoc|\\
|\edef\childdocname{\scantokens\expandafter{\jobname\noexpand}}|\\
|\def\childdocmain{|\textit{main}|}\||ifx\childdocmain\childdocname\||else|\\
|\childdoctrue\includeonly{\childdocname}\let\jobname\childdocmain\||fi|\\
\end{tabular}
\end{center}
%
Instead of |\childdocof{|\textit{main}|}| just include the main file
at the top of each child file:
%
\begin{center}
|\input{|\textit{main}|}|
\end{center}
%
A simple redirection |\childdocforward{|\textit{dest}|}| is achieved by:
%
\begin{center}
|\def\jobname{|\textit{dest}|}\input{\jobname}|
\end{center}
%
The redirection with prefix
|\childdocforwardprefix[|\textit{prefix}|]{|\textit{dest}|}|
is accomplished by:
%
\begin{center}
\begin{tabular}{l}
|{\edef\jobname{\scantokens\expandafter{\jobname\noexpand}}|\\
|\def\redirectjob |\textit{prefix}|#1~~~{\gdef\jobname{|\textit{dest}|#1}}|\\
|\expandafter\redirectjob\jobname~~~}\input{\jobname}|
\end{tabular}
\end{center}

In an alternative approach,
child documents can be compiled by a specific command line
without additional code or specific definitions:
%
\begin{center}
|... -jobname "|\textit{target}|" "|[\textit{flags}]%
|\includeonly{|\textit{dest}|}\input{|\textit{main}|}"|
\end{center}
%

%%%%%%%%%%%%%%%%%%%%%%%%%%%%%%%%%%%%%%%%%%%%%%%%%%%%%%%%%%%%%%%%%%%%%%%%%%%%%%%%
%%%%%%%%%%%%%%%%%%%%%%%%%%%%%%%%%%%%%%%%%%%%%%%%%%%%%%%%%%%%%%%%%%%%%%%%%%%%%%%%
\section{Information}

%%%%%%%%%%%%%%%%%%%%%%%%%%%%%%%%%%%%%%%%%%%%%%%%%%%%%%%%%%%%%%%%%%%%%%%%%%%%%%%%
\subsection{Copyright}

Copyright \copyright{} 2017--2018 Niklas Beisert

This work may be distributed and/or modified under the
conditions of the \LaTeX{} Project Public License, either version 1.3
of this license or (at your option) any later version.
The latest version of this license is in
  \url{http://www.latex-project.org/lppl.txt}
and version 1.3 or later is part of all distributions of \LaTeX{}
version 2005/12/01 or later.

This work has the LPPL maintenance status `maintained'.

The Current Maintainer of this work is Niklas Beisert.

This work consists of the files |README.txt|, |childdoc.ins| and |childdoc.dtx|
as well as the derived files |childdoc.def|, |cdocsamp.tex|
with |cdocsch1.tex|, |cdocsch2.tex|, |cdocspt3.tex|, |cdocspt4.tex|,
|cdocsdrf.tex|, |cdocsfn1.tex|, |cdocsfn2.tex|
as well as |childdoc.pdf|.

%%%%%%%%%%%%%%%%%%%%%%%%%%%%%%%%%%%%%%%%%%%%%%%%%%%%%%%%%%%%%%%%%%%%%%%%%%%%%%%%
\subsection{Files and Installation}

The package consists of the files:
%
\begin{center}
\begin{tabular}{ll}
    |README.txt|   & readme file \\
    |childdoc.ins| & installation file \\
    |childdoc.dtx| & source file \\
    |childdoc.def| & definition file \\
    |cdocsamp.tex| & sample main file \\
    |cdocsch1.tex| & sample include file \\
    |cdocsch2.tex| & sample include file \\
    |cdocspt3.tex| & sample part file \\
    |cdocspt4.tex| & sample part file \\
    |cdocsdrf.tex| & sample redirection file \\
    |cdocsfn1.tex| & sample redirection file \\
    |cdocsfn2.tex| & sample redirection file \\
    |childdoc.pdf| & manual
\end{tabular}
\end{center}
%
The distribution consists of the files
|README.txt|, |childdoc.ins| and |childdoc.dtx|.
%
\begin{itemize}
\item
Run (pdf)\LaTeX{} on |childdoc.dtx|
to compile the manual |childdoc.pdf| (this file).
\item
Run \LaTeX{} on |childdoc.ins| to create the definitions file |childdoc.def|
and the sample |cdocsamp.tex| with include files
|cdocsch1.tex|, |cdocsch2.tex|, |cdocspt3.tex|, |cdocspt4.tex|,
|cdocsdrf.tex|, |cdocsfn1.tex|, |cdocsfn2.tex|.
Then copy the file |childdoc.def| to an appropriate directory of your \LaTeX{}
distribution, e.g.\ \textit{texmf-root}|/tex/latex/childdoc|.
\end{itemize}

%%%%%%%%%%%%%%%%%%%%%%%%%%%%%%%%%%%%%%%%%%%%%%%%%%%%%%%%%%%%%%%%%%%%%%%%%%%%%%%%
\subsection{Related CTAN Packages}

There are several other packages which offer a similar functionality:
%
\begin{itemize}
\item
The packages
\href{http://ctan.org/pkg/docmute}{\textsf{docmute}},
\href{http://ctan.org/pkg/includex}{\textsf{includex}} and
\href{http://ctan.org/pkg/standalone}{\textsf{standalone}}
provide commands to include only the document body of
a child file thus allowing both files to be compiled individually.
\item
The packages \href{http://ctan.org/pkg/subdocs}{\textsf{subdocs}}
and \href{http://ctan.org/pkg/subfiles}{\textsf{subfiles}}
provide structures in which the main and child documents can be
encapsulated and allowing them to be compiled individually.
The inclusion mechanism is different from the conventional |\include|.
\item
The package \href{http://ctan.org/pkg/combine}{\textsf{combine}}
is an elaborate solution to combine several documents into one.
\end{itemize}
%
See also the CTAN topic \href{http://ctan.org/topic/subdocs}{\textsf{subdocs}}
for further related packages.
The present package differs from the above solutions in that
a document structure constructed with the conventional |\include| mechanism
just needs two extra commands at the top of every file
such that all constituent files can be compiled individually.

%%%%%%%%%%%%%%%%%%%%%%%%%%%%%%%%%%%%%%%%%%%%%%%%%%%%%%%%%%%%%%%%%%%%%%%%%%%%%%%%
%\subsection{Feature Suggestions}
%
%The following is a list of features which may be useful for future
%versions of this package:
%%
%\begin{itemize}
%\item
%\ldots
%\end{itemize}

%%%%%%%%%%%%%%%%%%%%%%%%%%%%%%%%%%%%%%%%%%%%%%%%%%%%%%%%%%%%%%%%%%%%%%%%%%%%%%%%
\subsection{Revision History}

%%%%%%%%%%%%%%%%%%%%%%%%%%%%%%%%%%%%%%%%
\paragraph{v2.0:} 2018/12/30

\begin{itemize}
\item
immediate forward processing
\item
added |\childdocby| mechanism
\item
manual restructured
\end{itemize}

%%%%%%%%%%%%%%%%%%%%%%%%%%%%%%%%%%%%%%%%
\paragraph{v1.6:} 2018/01/17

\begin{itemize}
\item
application for development of include files
\item
corrections to manual
\end{itemize}

%%%%%%%%%%%%%%%%%%%%%%%%%%%%%%%%%%%%%%%%
\paragraph{v1.5:} 2017/05/21

\begin{itemize}
\item
more complete structuring introduced
\item
|\childdocof| introduced
\item
|\childdoc| renamed to |\childdocmain|
\item
|\childredirect| renamed to |\childdocforward| and |\childdocforwardprefix|
and functionality expanded
\end{itemize}

%%%%%%%%%%%%%%%%%%%%%%%%%%%%%%%%%%%%%%%%
\paragraph{v1.0:} 2017/04/27

\begin{itemize}
\item
manual and install package
\item
first version published on CTAN
\end{itemize}

%%%%%%%%%%%%%%%%%%%%%%%%%%%%%%%%%%%%%%%%
\paragraph{v0.6:} 2017/04/26

\begin{itemize}
\item
redirection mechanism added
\end{itemize}

%%%%%%%%%%%%%%%%%%%%%%%%%%%%%%%%%%%%%%%%
\paragraph{v0.5:} 2017/04/26

\begin{itemize}
\item
functionality in definition file
\end{itemize}


%%%%%%%%%%%%%%%%%%%%%%%%%%%%%%%%%%%%%%%%%%%%%%%%%%%%%%%%%%%%%%%%%%%%%%%%%%%%%%%%
%%%%%%%%%%%%%%%%%%%%%%%%%%%%%%%%%%%%%%%%%%%%%%%%%%%%%%%%%%%%%%%%%%%%%%%%%%%%%%%%
%%%%%%%%%%%%%%%%%%%%%%%%%%%%%%%%%%%%%%%%%%%%%%%%%%%%%%%%%%%%%%%%%%%%%%%%%%%%%%%%
\appendix

\settowidth\MacroIndent{\rmfamily\scriptsize 000\ }

 \DocInput{childdoc.dtx}

\end{document}
%</driver>
% \fi
%
% %%%%%%%%%%%%%%%%%%%%%%%%%%%%%%%%%%%%%%%%%%%%%%%%%%%%%%%%%%%%%%%%%%%%%%%%%%%%%%
% %%%%%%%%%%%%%%%%%%%%%%%%%%%%%%%%%%%%%%%%%%%%%%%%%%%%%%%%%%%%%%%%%%%%%%%%%%%%%%
% \section{Sample}
%\iffalse
%<*samplemain>
%\fi
%
% The following presents a sample document
% with two chapters, two parts, a title page,
% a compile flag as well as three forwarding files to set the flag.
% It consists of eight |.tex| files:
% \begin{center}
% \begin{tabular}{ll}
% |cdocsamp.tex|&main file\\
% |cdocsch1.tex|&include file for chapter 1\\
% |cdocsch2.tex|&include file for chapter 2\\
% |cdocspt3.tex|&include file for part 3\\
% |cdocspt4.tex|&include file for part 4\\
% |cdocsdrf.tex|&forwarding file for main file in draft mode\\
% |cdocsfi1.tex|&forwarding file for final version of chapter 1\\
% |cdocsfi2.tex|&forwarding file for final version of chapter 2\\
% \end{tabular}
% \end{center}
% Each of the eight files can be compiled directly by the \LaTeX{} compiler.
%
% %%%%%%%%%%%%%%%%%%%%%%%%%%%%%%%%%%%%%%
% \paragraph{Main File.}
%
% The main file is called |cdocsamp.tex|.
%
% Load the \textsf{childdoc} definitions and
% declare the filename for the main document:
%    \begin{macrocode}
\input{childdoc.def}
\childdocmain{}
%    \end{macrocode}

% Optional override for |\version| flag:
%    \begin{macrocode}
%%\ifchilddoc\else\providecommand{\version}{draft}\fi
%    \end{macrocode}

% Define the default values for the |\version| flag
% (|final| for the main file and |draft| for childs):
%    \begin{macrocode}
\ifchilddoc
\providecommand{\version}{draft}
\else
\providecommand{\version}{final}
\fi
%    \end{macrocode}

% Load the standard document class:
%    \begin{macrocode}
\documentclass[12pt]{article}
%    \end{macrocode}

% Start the document body:
%    \begin{macrocode}
\begin{document}
%    \end{macrocode}

% Declare a title page.
% Print title, part of document being processed and version flag:
%    \begin{macrocode}
\addtocounter{page}{-1}
\begin{center}
{\LARGE\bfseries{}childdoc example\par}
\vspace{1cm}
\ifchilddoc
\ifchilddocmanual part\else chapter\fi:
`\childdocname' of `\childdocjob'\par
\else
main document: `\childdocjob'\par
\fi
version: \version\par
\end{center}
\newpage
%    \end{macrocode}

% Manually include selected file,
% otherwise process as usual:
%    \begin{macrocode}
\ifchilddocmanual
\section*{part `\childdocname'}
\input{\childdocname}
\else
%    \end{macrocode}

% Include the two chapters:
%    \begin{macrocode}
\include{cdocsch1}
\include{cdocsch2}
%    \end{macrocode}

% Include the two parts unless only chapters should be displayed:
%    \begin{macrocode}
\ifchilddoc\else
\section{part three}
\input{cdocspt3}
\section{part four}
\input{cdocspt4}
\fi
%    \end{macrocode}

% Process as usual until here:
%    \begin{macrocode}
\fi
%    \end{macrocode}

% End of document body:
%    \begin{macrocode}
\end{document}
%    \end{macrocode}
%\iffalse
%</samplemain>
%\fi
%
% %%%%%%%%%%%%%%%%%%%%%%%%%%%%%%%%%%%%%%
% \paragraph{Chapter Include Files.}
%
% The include files are called |cdocsch1.tex| and |cdocsch2.tex|.
%
%\iffalse
%<*samplechap1|samplechap2>
%\fi

% Optional override for |\version| flag:
%    \begin{macrocode}
%%\providecommand{\version}{final}
%    \end{macrocode}

% Include the main document:
%    \begin{macrocode}
\input{childdoc.def}
\childdocof{cdocsamp}
%    \end{macrocode}

%\iffalse
%</samplechap1|samplechap2>
%\fi
%
%\iffalse
%<*samplechap1>
%\fi
% Some text for chapter 1:
%    \begin{macrocode}
\section{one}
some text in chapter one
%    \end{macrocode}

%\iffalse
%</samplechap1>
%\fi
% Some text for chapter 2:
%\iffalse
%<*samplechap2>
%\fi
%    \begin{macrocode}
\section{two}
more text in chapter two
%    \end{macrocode}

%\iffalse
%</samplechap2>
%\fi
%
% %%%%%%%%%%%%%%%%%%%%%%%%%%%%%%%%%%%%%%
% \paragraph{Part Include Files.}
%
% The include files are called |cdocspt3.tex| and |cdocspt4.tex|.
%
%\iffalse
%<*samplepart3|samplepart4>
%\fi

% Optional override for |\version| flag:
%    \begin{macrocode}
%%\providecommand{\version}{final}
%    \end{macrocode}

% Include the main document:
%    \begin{macrocode}
\input{childdoc.def}
\childdocby{cdocsamp}
%    \end{macrocode}

%\iffalse
%</samplepart3|samplepart4>
%\fi
%
%\iffalse
%<*samplepart3>
%\fi
% Some text for part 3:
%    \begin{macrocode}
some text in part three
%    \end{macrocode}

%\iffalse
%</samplepart3>
%\fi
% Some text for part 4:
%\iffalse
%<*samplepart4>
%\fi
%    \begin{macrocode}
more text in part four
%    \end{macrocode}

%\iffalse
%</samplepart4>
%\fi
%
% %%%%%%%%%%%%%%%%%%%%%%%%%%%%%%%%%%%%%%
% \paragraph{Forwarding for a Complete Draft.}
%
% The following forwarding file |cdocsdrf.tex|
% compiles the main document in draft mode:
%\iffalse
%<*sampledraft>
%\fi
%    \begin{macrocode}
\def\version{draft}
\input{childdoc.def}
\childdocforward{cdocsamp}
%    \end{macrocode}

%\iffalse
%</sampledraft>
%\fi
%
% %%%%%%%%%%%%%%%%%%%%%%%%%%%%%%%%%%%%%%
% \paragraph{Forwarding for Final Version of the Chapters.}
%
% The following forwarding files |cdocsfn1.tex| and |cdocsfn2.tex|
% (with identical content)
% compile the final versions of the child documents
% |cdocsch1.tex| and |cdocsch2.tex|, respectively:
%\iffalse
%<*samplefinal>
%\fi
%    \begin{macrocode}
\def\version{final}
\input{childdoc.def}
\childdocforwardprefix[cdocsamp]{cdocsfn}{cdocsch}
%    \end{macrocode}

%\iffalse
%</samplefinal>
%\fi
%
% %%%%%%%%%%%%%%%%%%%%%%%%%%%%%%%%%%%%%%
% \paragraph{Command Line Processing.}
%
% The following three command lines generate the output files
% |cdocscld|, |cdocscl1| and |cdocscl2|
% which should be identical to
% |cdocsdrf|, |cdocsch1| and |cdocsfn2|, respectively:
% \begin{center}
% \begin{tabular}{l}
% |latex -jobname cdocscld \|\\
% |  "\def\version{draft}\input{childdoc.def}\childdocforward{cdocsamp}"|\\
% |latex -jobname cdocscl1 \|\\
% |  "\input{childdoc.def}\childdocforward[cdocsamp]{cdocsch1}"|\\
% |latex -jobname cdocscl2 \|\\
% |  "\def\version{final}\input{childdoc.def}\childdocforward{cdocsch2}"|
% \end{tabular}
% \end{center}
% Note that the trailing backslash on each first line
% merely continues the input to the second line
% (for convenient cut ant paste).
% Furthermore, the command |latex| can be replaced by any
% of its alternative versions such as |pdflatex|.
%
% %%%%%%%%%%%%%%%%%%%%%%%%%%%%%%%%%%%%%%%%%%%%%%%%%%%%%%%%%%%%%%%%%%%%%%%%%%%%%%
% %%%%%%%%%%%%%%%%%%%%%%%%%%%%%%%%%%%%%%%%%%%%%%%%%%%%%%%%%%%%%%%%%%%%%%%%%%%%%%
% \section{Implementation}
%\iffalse
%<*package>
%\fi
%
% This section describes the definitions file |childdoc.def|.

% The definitions cannot be loaded using |\usepackage| or |\RequirePackage|
% which has a mechanism to prevent loading a style file more than once.
% When loading the definitions by means of |\input|
% multiple instances have to be prevented manually:
%\iffalse
%This code needs to be before the `\ProvidesFile' directive
%which is defined at the beginning of this file.
%Therefore it is also placed there and commented out here.
%</package>
%<*discard>
%\fi
%    \begin{macrocode}
\ifdefined\childdocmain\endinput\fi
%    \end{macrocode}
%\iffalse
%</discard>
%<*package>
%\fi
%
% \macro{\ifchilddoc}
% \macro{\ifchilddocmanual}
% The conditional |\ifchilddoc| tells whether a
% child (true) or main (false) document is being compiled.
% The conditional |\ifchilddocmanual| tells whether
% the |\includeonly| mechanism is used (false) or
% the selection of child files must be performed manually (true).
% The definitions initialise to false:
%    \begin{macrocode}
\newif\ifchilddoc
\newif\ifchilddocmanual
%    \end{macrocode}

% \macro{\childdocname}
% \macro{\childdocjob}
% The macro |\childdocname| stores the name of the main document
% to be compiled. The macro |\childdocjob| stores the name of
% the document on which the \LaTeX{} compiler was originally invoked.
% The content of |\jobname| cannot be compared
% to filenames specified in the source due to different catcodes.
% The following code rescans |\jobname|, stores the result
% in |\childdocname| and saves a copy in |\childdocjob|:
%    \begin{macrocode}
\edef\childdocname{\scantokens\expandafter{\jobname\noexpand}}
\let\childdocjob\childdocname
%    \end{macrocode}

% \macro{\childdocdisable}
% The macro |\childdocdisable| prevents the main file
% from being processed more than once.
% At this stage, the main document command |\childdocmain|
% is assumed to be called once again where it should do nothing.
% Any subsequent call to it should prevent
% a secondary processing of the main document
% It overwrites the forwarding commands
% |\childdocof| and |\childdocforward|
% with empty macros to prevent further inclusions of the main document:
%    \begin{macrocode}
\newcommand{\childdocdisable}
{
  \renewcommand{\childdocmain}[1]{\renewcommand{\childdocmain}[1]{\endinput}}
  \renewcommand{\childdocof}[1]{}
  \renewcommand{\childdocby}[2][]{}
  \renewcommand{\childdocforward}[2][]{}
  \renewcommand{\childdocdisable}{}
}
%    \end{macrocode}

% \macro{\childdocmain}
% The macro |\childdocmain| is to be called at the top of the main file
% with nothing or the main filename (without extension) as argument.
% First, it breaks loops.
% If the argument is not empty and does not match |\childdocname|
% (which is set by the first inclusion of |childdoc.def|),
% |\ifchilddoc| is set to true, |\includeonly| is applied to the child file
% and |\jobname| is set to the main file
% (for proper handling of |.aux| files):
%    \begin{macrocode}
\newcommand{\childdocmain}[1]
{
  \childdocdisable\childdocmain{}
  \if?#1?\else
    \begingroup
      \def\childdoctmp{#1}
      \ifx\childdoctmp\childdocname
        \def\childdoctmp{}
      \else
        \def\childdoctmp
        {
          \childdoctrue
          \includeonly{\childdocname}
          \def\childdocjob{#1}
          \def\jobname{#1}
        }
      \fi
      \expandafter
    \endgroup
    \childdoctmp
  \fi
}
%    \end{macrocode}

% \macro{\childdocof}
% The command |\childdocof| redirects
% compilation to the main file |#1|.
%    \begin{macrocode}
\newcommand{\childdocof}[1]
{
  \childdocdisable
  \childdoctrue
  \includeonly{\childdocname}
  \def\jobname{#1}
  \def\childdocjob{#1}
  \input{#1}
}
%    \end{macrocode}

% \macro{\childdocby}
% The command |\childdocby| ....
%    \begin{macrocode}
\newcommand{\childdocby}[2][]
{
  \childdocdisable
  \childdoctrue
  \childdocmanualtrue
  \if?#1?\else
    \def\jobname{#2}
  \fi
  \def\childdocjob{#2}
  \input{#2}
  \endinput
}
%    \end{macrocode}

% \macro{\childdocforward}
% The command |\childdocforward| redirects
% compilation to the main file or
% (if the optional argument is given) a child file.
% Parameters are set as if the main file
% or a child file starting with |\childdocof| was compiled.
% Then compilation is handed over to the main file:
%    \begin{macrocode}
\newcommand{\childdocforward}[2][]
{
  \begingroup
    \if?#1?
      \def\childdoctmp
      {
        \def\childdocname{#2}
        \def\childdocjob{#2}
        \def\jobname{#2}
        \input{#2}
        \endinput
      }
    \else
      \def\childdoctmp
      {
        \childdocdisable
        \def\childdocname{#2}
        \childdoctrue
        \includeonly{#2}
        \def\childdocjob{#1}
        \def\jobname{#1}
        \input{#1}
        \endinput
      }
    \fi
    \expandafter
  \endgroup
  \childdoctmp
}
%    \end{macrocode}

% \macro{\childdocforwardprefix}
% The command |\childdocforwardprefix| redirects
% compilation to the main or a child file by means of a pattern.
% The prefix |#1| in the current filename is replaced by |#2|
% and the suffix of the current filename is kept
% (it is assumed that the filename does not contain the substring `|~~~|'
% which is used as a delimiter).
% Compilation is handed over to the new file by |\childdocforward|:
%    \begin{macrocode}
\newcommand{\childdocforwardprefix}[3][]
{
  \begingroup
    \def\childdocextract #2##1~~~{\def\childdoctmp{\childdocforward[#1]{#3##1}}}
    \expandafter\childdocextract\childdocname~~~
    \expandafter
  \endgroup
  \childdoctmp
}
%    \end{macrocode}

% \macro{\childdoc}
% The deprecated macro |\childdoc| is a legacy version of |\childdocmain|:
%    \begin{macrocode}
\newcommand{\childdoc}{\childdocmain}
%    \end{macrocode}

% \macro{\childdocredirect}
% The deprecated macro |\childdocredirect| is a legacy version
% of |\childdocforward| and |\childdocforwardprefix|:
%    \begin{macrocode}
\newcommand{\childdocredirect}[2][]
{
  \begingroup
    \if?#1?
      \def\childdoctmp{\childdocforward{#2}}
    \else
      \def\childdoctmp{\childdocforwardprefix{#1}{#2}}
    \fi
    \expandafter
  \endgroup
  \childdoctmp
}
%    \end{macrocode}

%\iffalse
%</package>
%\fi
%
\endinput
|\\
|\childdocforwardprefix{final}{child}|
\end{tabular}
\end{center}
%

Note that when several versions of a main file and/or of each child file
are to be generated, it may be convenient to set up a |Makefile| or
shell script to automatise the process.

%%%%%%%%%%%%%%%%%%%%%%%%%%%%%%%%%%%%%%%%%%%%%%%%%%%%%%%%%%%%%%%%%%%%%%%%%%%%%%%%
\subsection{Command Line Processing}
\label{sec:commandline}

The effect of redirection files can also be achieved by invoking
the \LaTeX{} compiler with a more elaborate command line.
Most conveniently this should be done as part
of a shell script or a |Makefile|.

When using \textsf{childdoc} in the main file, the following
command lines effectively perform a redirection
(note that depending on the shell being used,
backslashes may have to be doubled: `|\|' $\to$ `|\\|'):
%
\begin{center}
|... -jobname "|\textit{target}|" |\\|"|[\textit{flags}]%
|% \iffalse
%
% childdoc.dtx Copyright (C) 2017-2018 Niklas Beisert
%
% This work may be distributed and/or modified under the
% conditions of the LaTeX Project Public License, either version 1.3
% of this license or (at your option) any later version.
% The latest version of this license is in
%   http://www.latex-project.org/lppl.txt
% and version 1.3 or later is part of all distributions of LaTeX
% version 2005/12/01 or later.
%
% This work has the LPPL maintenance status `maintained'.
%
% The Current Maintainer of this work is Niklas Beisert.
%
% This work consists of the files childdoc.dtx and childdoc.ins
% and the derived files childdoc.def and cdocsamp.tex with
% cdocsch1.tex, cdocsch2.tex, cdocsdrf.tex, cdocsfn1.tex, cdocsfn2.tex.
%
%<package>\ifdefined\childdocmain\endinput\fi
%<package>\ProvidesFile{childdoc.def}[2018/12/30 v2.0 child document driver]
%<samplemain>\ProvidesFile{cdocsamp.tex}[2018/12/30 v2.0 sample for childdoc]
%<*driver>
%\ProvidesFile{childdoc.drv}[2018/12/30 v2.0 childdoc reference manual file]
\PassOptionsToClass{10pt,a4paper}{article}
\documentclass{ltxdoc}

\usepackage[margin=35mm]{geometry}
\usepackage{hyperref}
\usepackage{hyperxmp}
\usepackage[usenames]{color}

\hypersetup{colorlinks=true}
\hypersetup{pdfstartview=FitH}
\hypersetup{pdfpagemode=UseNone}
\hypersetup{pdfsource={}}
\hypersetup{pdflang={en-UK}}
\hypersetup{pdfcopyright={Copyright 2017-2018 Niklas Beisert.
  This work may be distributed and/or modified under the
  conditions of the LaTeX Project Public License, either version 1.3
  of this license or (at your option) any later version.}}
\hypersetup{pdflicenseurl={http://www.latex-project.org/lppl.txt}}
\hypersetup{pdfcontactaddress={ETH Zurich, ITP, HIT K,
  Wolfgang-Pauli-Strasse 27}}
\hypersetup{pdfcontactpostcode={8093}}
\hypersetup{pdfcontactcity={Zurich}}
\hypersetup{pdfcontactcountry={Switzerland}}
\hypersetup{pdfcontactemail={nbeisert@itp.phys.ethz.ch}}
\hypersetup{pdfcontacturl={http://people.phys.ethz.ch/\xmptilde nbeisert/}}

\newcommand{\secref}[1]{\hyperref[#1]{section \ref*{#1}}}

\parskip1ex
\parindent0pt
\let\olditemize\itemize
\def\itemize{\olditemize\parskip0pt}

\begin{document}

\title{The \textsf{childdoc} Package}
\hypersetup{pdftitle={The childdoc Package}}
\author{Niklas Beisert\\[2ex]
  Institut f\"ur Theoretische Physik\\
  Eidgen\"ossische Technische Hochschule Z\"urich\\
  Wolfgang-Pauli-Strasse 27, 8093 Z\"urich, Switzerland\\[1ex]
  \href{mailto:nbeisert@itp.phys.ethz.ch}
  {\texttt{nbeisert@itp.phys.ethz.ch}}}
\hypersetup{pdfauthor={Niklas Beisert}}
\hypersetup{pdfsubject={Manual for the LaTeX2e Package childdoc}}
\date{30 December 2018, \textsf{v2.0}}
\maketitle

\begin{abstract}\noindent
\textsf{childdoc} is a \LaTeXe{} package
that enables the direct compilation
of document sections included by |\include|
to individual files.
\end{abstract}

\begingroup
\parskip0ex
\tableofcontents
\endgroup

%%%%%%%%%%%%%%%%%%%%%%%%%%%%%%%%%%%%%%%%%%%%%%%%%%%%%%%%%%%%%%%%%%%%%%%%%%%%%%%%
%%%%%%%%%%%%%%%%%%%%%%%%%%%%%%%%%%%%%%%%%%%%%%%%%%%%%%%%%%%%%%%%%%%%%%%%%%%%%%%%
\section{Introduction}

\LaTeX{} provides a mechanism to structure a large document (such as a book)
into a main file and several child files (containing the chapters)
using the |\include| command.
This mechanism is beneficial for documents
which span hundreds of pages in order to
make the source file(s) more manageable.
Moreover, compilation can be restricted to
selected child files by means of the |\includeonly| command.
The latter feature can be used to reduce the compilation time while editing
(this was significantly more useful in the earlier days of \LaTeX{})
or to generate a smaller document which is easier to navigate.
Another application of |\includeonly| is to generate
documents consisting of selected parts of the complete document.

However, there are a few drawbacks of the plain |\include| mechanism:
\begin{itemize}
\item
The child files cannot be compiled on their own,
they can only be compiled via the main file.
A naive editing environment
(such as a text editor with an option
to have the current file processed by \LaTeX)
may require one to switch to the main file before compiling;
attempting to compile the child file produces errors.
\item
The main file must be modified (each time)
to adjust the |\includeonly| command
to the present needs. This easily leaves the main file in a messy state.
\item
The generated document will always carry the filename
of the main document. This is inconvenient if
several child files are to be compiled and
to be kept for distribution.
\end{itemize}

The present package provides a simple interface
to make child files individually compilable by \LaTeX{}.
Compiling a child file then has the same effect as compiling
the main file with an |\includeonly| command
to select the appropriate child.
Moreover the generated document will carry the name of the child
rather than the main file.
This resolves all three above issues.

This feature is meant to make the editing of books,
thesis documents and lecture notes somewhat more convenient.
However, the package can also be used efficiently for
composing a series of documents (such as exercise sheets)
which are typically distributed individually.
It then assists the author in generating the individual documents
(potentially in different versions)
as well as a document containing the collected series.
Another application is in developing style files
or other kinds of included material
where compilation of the style file could redirect
to a sample or test file.

%%%%%%%%%%%%%%%%%%%%%%%%%%%%%%%%%%%%%%%%%%%%%%%%%%%%%%%%%%%%%%%%%%%%%%%%%%%%%%%%
%%%%%%%%%%%%%%%%%%%%%%%%%%%%%%%%%%%%%%%%%%%%%%%%%%%%%%%%%%%%%%%%%%%%%%%%%%%%%%%%
\section{Usage}

First of all, the package \textsf{childdoc} is \emph{not} a standard
\LaTeXe{} |.sty| style file! Therefore it needs to be invoked in
a non-standard way.

%%%%%%%%%%%%%%%%%%%%%%%%%%%%%%%%%%%%%%%%%%%%%%%%%%%%%%%%%%%%%%%%%%%%%%%%%%%%%%%%
\subsection{Included Files}
\label{sec:include}

%%%%%%%%%%%%%%%%%%%%%%%%%%%%%%%%%%%%%%%%
\DescribeMacro{\childdocmain}
To use the package, add the commands
\begin{center}
\begin{tabular}{l}
|\input{childdoc.def}|\\
|\childdocmain{}|\\
\end{tabular}
\end{center}
at the very top of the main \LaTeX{} file,
in particular \emph{before} the |\documentclass| statement!
The argument of |\childdocmain| should be left empty
(but it must be present).

%%%%%%%%%%%%%%%%%%%%%%%%%%%%%%%%%%%%%%%%
\DescribeMacro{\childdocof}
Furthermore, add the commands
\begin{center}
\begin{tabular}{l}
|\input{childdoc.def}|\\
|\childdocof{|\textit{main}|}|\\
\end{tabular}
\end{center}
at the top of every child file \textit{child}
which is included by |\include{|\textit{child}|}|
from within the main file
(or at least for those files to be compiled individually).
The argument \textit{main} must be the filename of the main file.

There are a couple of
considerations in setting up the main and child documents:

%%%%%%%%%%%%%%%%%%%%%%%%%%%%%%%%%%%%%%%%
\paragraph{Restrictions.}

Please note the following restrictions:
\begin{itemize}
\item
|\childdocmain| must be called with one argument \textit{main}
to ensure compatibility with earlier version of the package.
It must either be empty (|\childdocmain{}|)
or precisely match the filename of the main file in which it is specified.
See \secref{sec:detection} for further information.
\item
The filename \textit{main} must be specified without the |.tex| extension.
\item
The filename \textit{main} is case sensitive
(even in case-insensitive file systems)
due to internal string comparison.
\item
The argument \textit{main} should be fully expanded, it cannot be a macro.
\item
Subdirectories and special characters should be avoided in filenames.
\item
The command |\childdocmain{|\textit{main}|}| must be followed by a whitespace.
It should not be followed immediately by another command
or by a comment mark `|%|'.
This is because the \TeX{} parser reads the token immediately following
the argument of |\childdocmain| and puts it
at the beginning of every child section;
however, a white\-space is ignored.
\end{itemize}

%%%%%%%%%%%%%%%%%%%%%%%%%%%%%%%%%%%%%%%%
\paragraph{Content of Main File.}

It is advisable to place all content in the child files included by |\include|.
Any output contained in the main file will appear in all child documents
unless suppressed manually;
it cannot be suppressed automatically by the |\includeonly| directive
and thus should normally be avoided.
A method to include some content in the main file
by means of conditional processing is described in \secref{sec:conditional}.

%%%%%%%%%%%%%%%%%%%%%%%%%%%%%%%%%%%%%%%%
\paragraph{Page Numbering.}

When only a part of the document is compiled,
the appropriate numbering of pages
(as well as other status parameters)
is determined from the |.aux| files.
The latter contain information from previous passes.
However this information needs to propagate through
all intermediate child documents.
Therefore the page numbering in child documents may well
be inconsistent until the complete document is compiled at least once.

A useful (if unconventional) way to always ensure a consistent
page numbering is to restart the numbering in each child document
and denote the pages by `\textit{child}|.|\textit{page}'
where \textit{child} represents the chapter/section number of the child file.
This can be achieved by the command
|\numberwithin{page}{|\textit{child}|}|
of the \textsf{amsmath} package
where \textit{child} can be |chapter| or |section|
depending on the chosen structuring.
Alternatively, one can modify the macro |\thepage| appropriately
and reset the counter |page| at the start of each child file.

%%%%%%%%%%%%%%%%%%%%%%%%%%%%%%%%%%%%%%%%%%%%%%%%%%%%%%%%%%%%%%%%%%%%%%%%%%%%%%%%
\subsection{Conditional Processing}
\label{sec:conditional}

The package provides a mechanism to compile different versions
of a document. To customise the versions further some conditional processing
can come in handy to distinguish which version is being compiled.
The package provides two macros to describe the compilation context:

%%%%%%%%%%%%%%%%%%%%%%%%%%%%%%%%%%%%%%%%
\DescribeMacro{\ifchilddoc}
The conditional |\ifchilddoc| distinguishes between the compilation of
child documents and the main document:
%
\begin{center}
|\ifchilddoc |\textit{child-code}| |[|\||else |\textit{main-code}]| \||fi|
\end{center}

%%%%%%%%%%%%%%%%%%%%%%%%%%%%%%%%%%%%%%%%
\DescribeMacro{\childdocname}
\DescribeMacro{\childdocjob}
The macro |\childdocname| contains the filename (without extension)
of the main or child file being processed.
Note that |\childdocjob| will always contain the name of the main file.

%%%%%%%%%%%%%%%%%%%%%%%%%%%%%%%%%%%%%%%%
\paragraph{Title Page.}

Conditional processing can be used to include a title or banner page
in the main document when proper precautions are taken.
Importantly, the code in the main file should ensure that the page counter
(as well as other status parameters which are stored in the |.aux| files)
takes the same value after the conditional processing.
Otherwise the page numbers may take divergent values
depending on which part is compiled.

For example, a title page could be declared by:
%
\begin{center}
\begin{tabular}{l}
|\ifchilddoc\||else|\\
|\addtocounter{page}{-1}|\\
\textit{code for title page}\\
|\newpage|\\
|\||fi|
\end{tabular}
\end{center}
%
A banner page for the child documents can be generated by:
%
\begin{center}
\begin{tabular}{l}
|\ifchilddoc|\\
|\addtocounter{page}{-1}|\\
\textit{code for banner page}\\
|\newpage|\\
|\||fi|
\end{tabular}
\end{center}
%
Here one could write a message such as:
\begin{center}
|This is the part \childdocname{} of \childdocjob{}.|
\end{center}

%%%%%%%%%%%%%%%%%%%%%%%%%%%%%%%%%%%%%%%%%%%%%%%%%%%%%%%%%%%%%%%%%%%%%%%%%%%%%%%%
\subsection{Flags}
\label{sec:flags}

The package makes it easy to generate different versions
of the main or child documents.
To this end compilation flags can be defined
and assigned different default values.
They will be particularly useful in conjunction
with the forwarding mechanism described in \secref{sec:forward}.

For example, it may be useful to have a flag |\version|
which can be set to |draft| or |final|.
The document source will contain some conditional code
depending on the value of |\version|.
Suppose further, the flag should default to |final| for the main file
and to |draft| for child files
which is a natural assignment for editing the document.
This is achieved by placing the following code
in the preamble of the main document
(below the |\childdocmain| directive):
%
\begin{center}
\begin{tabular}{l}
|\ifchilddoc|\\
|\providecommand{\version}{draft}|\\
|\||else|\\
|\providecommand{\version}{final}|\\
|\||fi|
\end{tabular}
\end{center}
%
The definition by |\providecommand| makes sure
that previous definitions are not overwritten.
Further statements |\providecommand{\version}{...}|
can thus be added before the above code to override it.

For the main file, one might add a line
(between |\childdocmain| and the above block)
%
\begin{center}
|%\ifchilddoc\||else\providecommand{\version}{draft}\||fi|
\end{center}
%
which can be uncommented to produce a draft version.
Likewise one can add a line to the very top of a child file
(above the |\childdocof{|\textit{main}|}| directive)
%
\begin{center}
|%\providecommand{\version}{final}|
\end{center}
%
which can be uncommented to produce the final version of this child document.

%%%%%%%%%%%%%%%%%%%%%%%%%%%%%%%%%%%%%%%%%%%%%%%%%%%%%%%%%%%%%%%%%%%%%%%%%%%%%%%%
\subsection{Forwarding}
\label{sec:forward}

Different versions of the main or child documents
using compilation flags as described in \secref{sec:flags}
can be (permanently) stored in different files
for convenient compilation, viewing and distribution.
To this end, the package defines a command
to pass on compilation to a different file:

%%%%%%%%%%%%%%%%%%%%%%%%%%%%%%%%%%%%%%%%
\DescribeMacro{\childdocforward}
The command |\childdocforward| redirects processing to
another source file:
%
\begin{center}
\begin{tabular}{l}
|\input{childdoc.def}|\\
|\childdocforward[|\textit{main}|]{|\textit{dest}|}|\\
\end{tabular}
\end{center}
%
The argument \textit{dest} is the destination file
(without extension).
It should be the main file or one of the child files.
Note that further \textsf{childdoc} directives
such as |\childdocof| and |\childdocforward|
in the indicated file will be processed in this form.
The optional argument \textit{main}
passes on directly to the main file \textit{main}
while pretending to compile the child \textit{dest}.
This form behaves as if \textit{dest}
issues |\childdocof{|\textit{main}|}| right away,
and no further \textsf{childdoc} directives will be processed.

%%%%%%%%%%%%%%%%%%%%%%%%%%%%%%%%%%%%%%%%
\DescribeMacro{\...prefix}
In the alternative form |\childdocforwardprefix|,
%
\begin{center}
\begin{tabular}{l}
|\input{childdoc.def}|\\
|\childdocforwardprefix[|\textit{main}|]{|\textit{prefix}|}{|\textit{dest}|}|
\end{tabular}
\end{center}
%
the destination file is determined by a pattern
depending on the current file:
To make this work, the current file must be called
`{\textit{prefix}\hspace{0.2em}\textit{suffix}}'
with \textit{prefix} matching precisely the argument.
Processing is then passed on to the file
`{\textit{dest}\hspace{0.2em}\textit{suffix}}'.
Surely, the same effect is achieved by
directly specifying the
argument `{\textit{dest}\hspace{0.2em}\textit{suffix}}'
in the first form.
However, that requires to set up a different file
for each child. With the alternative form of the command
all these files can have exactly the same content
which simplifies setting them up and maintaining them.

For example, the following file |draft.tex|
with a compilation flag |\version| as described in \secref{sec:flags}
compiles the main document as a draft:
%
\begin{center}
\begin{tabular}{l}
|\def\version{draft}|\\
|\input{childdoc.def}|\\
|\childdocforward{|\textit{main}|}|
\end{tabular}
\end{center}
%
Likewise, the following files |final|\textit{nn}|.tex|
compile the final version of the child document
|child|\textit{nn}|.tex|:
%
\begin{center}
\begin{tabular}{l}
|\def\version{final}|\\
|\input{childdoc.def}|\\
|\childdocforwardprefix{final}{child}|
\end{tabular}
\end{center}
%

Note that when several versions of a main file and/or of each child file
are to be generated, it may be convenient to set up a |Makefile| or
shell script to automatise the process.

%%%%%%%%%%%%%%%%%%%%%%%%%%%%%%%%%%%%%%%%%%%%%%%%%%%%%%%%%%%%%%%%%%%%%%%%%%%%%%%%
\subsection{Command Line Processing}
\label{sec:commandline}

The effect of redirection files can also be achieved by invoking
the \LaTeX{} compiler with a more elaborate command line.
Most conveniently this should be done as part
of a shell script or a |Makefile|.

When using \textsf{childdoc} in the main file, the following
command lines effectively perform a redirection
(note that depending on the shell being used,
backslashes may have to be doubled: `|\|' $\to$ `|\\|'):
%
\begin{center}
|... -jobname "|\textit{target}|" |\\|"|[\textit{flags}]%
|\input{childdoc.def}\childdocforward[|\textit{main}|]{|\textit{dest}|}"|
\end{center}
%
Here \textit{target} is the name of the output file,
\textit{main} is the name of the main file
and \textit{dest} is the name of the main or child file to be processed
(all filenames without extensions).
The optional argument \textit{main} can be omitted
if \textit{main} matches \textit{dest}.
Optionally, compilation \textit{flags} can be defined via |\def| commands.
This command line makes the \TeX{} engine believe
it is compiling the file \textit{target}
whose content is specified as the latter parameter.
The provided code then forwards the processing to
\textit{main} or \textit{dest} as described in \secref{sec:forward}.

%%%%%%%%%%%%%%%%%%%%%%%%%%%%%%%%%%%%%%%%%%%%%%%%%%%%%%%%%%%%%%%%%%%%%%%%%%%%%%%%
\subsection{Include by Input}
\label{sec:input}

Including child documents by |\include| has some restrictions by design.
Most notably, the content of a child document always occupies
its own set of pages; pages cannot be shared between child documents.
Usually, this behaviour makes perfect sense
because each child document contain an essential part of the document.
However, in some situations it may be desirable to compose
a document from a collection of parts
without having mandatory page breaks between then.
For this case, the package
provides a mechanism to include parts
by |\input| which can also be processed individually.
However, by construction this mechanism
requires manual handling of the content to be output.

%%%%%%%%%%%%%%%%%%%%%%%%%%%%%%%%%%%%%%%%
\DescribeMacro{\ifchilddocmanual}
The main file should be prepared as usual, see \secref{sec:include}.
However, the document body must make a distinction
between processing of an individual part and of the main document, e.g.:
%
\begin{center}
\begin{tabular}{l}
|\ifchilddocmanual|\\
|\input{\childdocname}|\\
|\||else|\\
\textit{document body with }|\input{|\textit{part}|}|\\
|\||fi|
\end{tabular}
\end{center}
%
The conditional |\ifchilddocmanual| is true whenever
a part to be included by |\input| is being compiled,
and the name of the part is stored in |\childdocname|.

%%%%%%%%%%%%%%%%%%%%%%%%%%%%%%%%%%%%%%%%
\DescribeMacro{\childdocby}
Each part to be included by |\input| should start with:
%
\begin{center}
\begin{tabular}{l}
|\input{childdoc.def}|\\
|\childdocby{|\textit{main}|}|\\
\end{tabular}
\end{center}
%
The directive |\childdocby| is similar to |\childdocof|
described in \secref{sec:include},
but the subsequent selection of content must be done manually.
To that end, both |\ifchilddoc| and |\ifchilddocmanual|
will be true upon processing of a part,
and the name of the part is stored in |\childdocname|.
Note that |\jobname| will be set to the filename of the current part
so that each part receives an individual |.aux| file
that does not interfere with the |.aux| file(s) of the main document.
This behaviour can be altered by the alternative form
|\childdocby[*]{|\textit{main}|}| (with a non-empty optional argument)
which uses the |.aux| file of the main document
by setting |\jobname| to \textit{main}.

%%%%%%%%%%%%%%%%%%%%%%%%%%%%%%%%%%%%%%%%%%%%%%%%%%%%%%%%%%%%%%%%%%%%%%%%%%%%%%%%
\subsection{Driver Development}
\label{sec:driver}

The \textsf{childdoc} mechanism can also be use for the development
of definition files such as \LaTeX{} styles or classes.
This case differs from the above setup with multiple parts
included by |\include| in that no |\includeonly| should be invoked.
This can be achieved by starting the include file
(before |\ProvidesPackage|) with:
%
\begin{center}
\begin{tabular}{l}
|\input{childdoc.def}|\\
|\childdocforward{|\textit{main}|}|\\
\end{tabular}
\end{center}
%
or alternatively with:
%
\begin{center}
\begin{tabular}{l}
|\input{childdoc.def}|\\
|\childdocby{|\textit{main}|}|\\
\end{tabular}
\end{center}
%
Both forms have slightly different effects as described above.
The main file is prepared as usual, see \secref{sec:include}.

%%%%%%%%%%%%%%%%%%%%%%%%%%%%%%%%%%%%%%%%%%%%%%%%%%%%%%%%%%%%%%%%%%%%%%%%%%%%%%%%
\subsection{Legacy Detection}
\label{sec:detection}

The directive |\childdocmain| in the main file can detect
whether the complete document or merely a child is to be compiled
even without using the directive |\childdocof|.
This method is deprecated because it is less robust
and there is no compelling reason to use it;
it is merely provided for backward compatibility
and it may be removed in future versions.

If the detection mechanism is to be used,
it is mandatory to correctly specify
the filename of the main file as the argument of |\childdocmain|:
%
\begin{center}
\begin{tabular}{l}
|\input{childdoc.def}|\\
|\childdocmain{|\textit{main}|}|\\
\end{tabular}
\end{center}
%
If |\jobname| does not match the argument \textit{main} of |\childdocmain|,
it is assumed that |\jobname| points to the child file to be compiled.
When using |\childdocmain| with the main file specified as argument,
it suffices to start a child file
with just |\input{|\textit{main}|}|
without loading of the package and using |\childdocof|.
If instead all processing is done
with the appropriate \textsf{childdoc} directives,
the argument of \textit{main} of |\childdocmain| can be empty.

An alternative version of the command line processing described
in \secref{sec:commandline} using the detection mechanism reads:
%
\begin{center}
|... -jobname "|\textit{target}|" "|[\textit{flags}]%
[|\def\jobname{|\textit{dest}|}|]|\input{|\textit{main}|}"|
\end{center}

%%%%%%%%%%%%%%%%%%%%%%%%%%%%%%%%%%%%%%%%%%%%%%%%%%%%%%%%%%%%%%%%%%%%%%%%%%%%%%%%
\subsection{Manual Code}
\label{sec:manual}

In case one cannot be certain whether the definitions file |childdoc.def|
is installed on the target \TeX{} distribution
and one prefers not to ship it,
it is conceivable to paste a few relevant commands into the sources.

To that end, drop all statements |\input{childdoc.def}|
and perform the replacements as outlined below.
Instead of |\childdocmain{|\textit{main}|}| add the following code
to the top of the main file:
%
\begin{center}
\begin{tabular}{l}
|\||ifdefined\childdocname\endinput\||fi\newif\ifchilddoc|\\
|\edef\childdocname{\scantokens\expandafter{\jobname\noexpand}}|\\
|\def\childdocmain{|\textit{main}|}\||ifx\childdocmain\childdocname\||else|\\
|\childdoctrue\includeonly{\childdocname}\let\jobname\childdocmain\||fi|\\
\end{tabular}
\end{center}
%
Instead of |\childdocof{|\textit{main}|}| just include the main file
at the top of each child file:
%
\begin{center}
|\input{|\textit{main}|}|
\end{center}
%
A simple redirection |\childdocforward{|\textit{dest}|}| is achieved by:
%
\begin{center}
|\def\jobname{|\textit{dest}|}\input{\jobname}|
\end{center}
%
The redirection with prefix
|\childdocforwardprefix[|\textit{prefix}|]{|\textit{dest}|}|
is accomplished by:
%
\begin{center}
\begin{tabular}{l}
|{\edef\jobname{\scantokens\expandafter{\jobname\noexpand}}|\\
|\def\redirectjob |\textit{prefix}|#1~~~{\gdef\jobname{|\textit{dest}|#1}}|\\
|\expandafter\redirectjob\jobname~~~}\input{\jobname}|
\end{tabular}
\end{center}

In an alternative approach,
child documents can be compiled by a specific command line
without additional code or specific definitions:
%
\begin{center}
|... -jobname "|\textit{target}|" "|[\textit{flags}]%
|\includeonly{|\textit{dest}|}\input{|\textit{main}|}"|
\end{center}
%

%%%%%%%%%%%%%%%%%%%%%%%%%%%%%%%%%%%%%%%%%%%%%%%%%%%%%%%%%%%%%%%%%%%%%%%%%%%%%%%%
%%%%%%%%%%%%%%%%%%%%%%%%%%%%%%%%%%%%%%%%%%%%%%%%%%%%%%%%%%%%%%%%%%%%%%%%%%%%%%%%
\section{Information}

%%%%%%%%%%%%%%%%%%%%%%%%%%%%%%%%%%%%%%%%%%%%%%%%%%%%%%%%%%%%%%%%%%%%%%%%%%%%%%%%
\subsection{Copyright}

Copyright \copyright{} 2017--2018 Niklas Beisert

This work may be distributed and/or modified under the
conditions of the \LaTeX{} Project Public License, either version 1.3
of this license or (at your option) any later version.
The latest version of this license is in
  \url{http://www.latex-project.org/lppl.txt}
and version 1.3 or later is part of all distributions of \LaTeX{}
version 2005/12/01 or later.

This work has the LPPL maintenance status `maintained'.

The Current Maintainer of this work is Niklas Beisert.

This work consists of the files |README.txt|, |childdoc.ins| and |childdoc.dtx|
as well as the derived files |childdoc.def|, |cdocsamp.tex|
with |cdocsch1.tex|, |cdocsch2.tex|, |cdocspt3.tex|, |cdocspt4.tex|,
|cdocsdrf.tex|, |cdocsfn1.tex|, |cdocsfn2.tex|
as well as |childdoc.pdf|.

%%%%%%%%%%%%%%%%%%%%%%%%%%%%%%%%%%%%%%%%%%%%%%%%%%%%%%%%%%%%%%%%%%%%%%%%%%%%%%%%
\subsection{Files and Installation}

The package consists of the files:
%
\begin{center}
\begin{tabular}{ll}
    |README.txt|   & readme file \\
    |childdoc.ins| & installation file \\
    |childdoc.dtx| & source file \\
    |childdoc.def| & definition file \\
    |cdocsamp.tex| & sample main file \\
    |cdocsch1.tex| & sample include file \\
    |cdocsch2.tex| & sample include file \\
    |cdocspt3.tex| & sample part file \\
    |cdocspt4.tex| & sample part file \\
    |cdocsdrf.tex| & sample redirection file \\
    |cdocsfn1.tex| & sample redirection file \\
    |cdocsfn2.tex| & sample redirection file \\
    |childdoc.pdf| & manual
\end{tabular}
\end{center}
%
The distribution consists of the files
|README.txt|, |childdoc.ins| and |childdoc.dtx|.
%
\begin{itemize}
\item
Run (pdf)\LaTeX{} on |childdoc.dtx|
to compile the manual |childdoc.pdf| (this file).
\item
Run \LaTeX{} on |childdoc.ins| to create the definitions file |childdoc.def|
and the sample |cdocsamp.tex| with include files
|cdocsch1.tex|, |cdocsch2.tex|, |cdocspt3.tex|, |cdocspt4.tex|,
|cdocsdrf.tex|, |cdocsfn1.tex|, |cdocsfn2.tex|.
Then copy the file |childdoc.def| to an appropriate directory of your \LaTeX{}
distribution, e.g.\ \textit{texmf-root}|/tex/latex/childdoc|.
\end{itemize}

%%%%%%%%%%%%%%%%%%%%%%%%%%%%%%%%%%%%%%%%%%%%%%%%%%%%%%%%%%%%%%%%%%%%%%%%%%%%%%%%
\subsection{Related CTAN Packages}

There are several other packages which offer a similar functionality:
%
\begin{itemize}
\item
The packages
\href{http://ctan.org/pkg/docmute}{\textsf{docmute}},
\href{http://ctan.org/pkg/includex}{\textsf{includex}} and
\href{http://ctan.org/pkg/standalone}{\textsf{standalone}}
provide commands to include only the document body of
a child file thus allowing both files to be compiled individually.
\item
The packages \href{http://ctan.org/pkg/subdocs}{\textsf{subdocs}}
and \href{http://ctan.org/pkg/subfiles}{\textsf{subfiles}}
provide structures in which the main and child documents can be
encapsulated and allowing them to be compiled individually.
The inclusion mechanism is different from the conventional |\include|.
\item
The package \href{http://ctan.org/pkg/combine}{\textsf{combine}}
is an elaborate solution to combine several documents into one.
\end{itemize}
%
See also the CTAN topic \href{http://ctan.org/topic/subdocs}{\textsf{subdocs}}
for further related packages.
The present package differs from the above solutions in that
a document structure constructed with the conventional |\include| mechanism
just needs two extra commands at the top of every file
such that all constituent files can be compiled individually.

%%%%%%%%%%%%%%%%%%%%%%%%%%%%%%%%%%%%%%%%%%%%%%%%%%%%%%%%%%%%%%%%%%%%%%%%%%%%%%%%
%\subsection{Feature Suggestions}
%
%The following is a list of features which may be useful for future
%versions of this package:
%%
%\begin{itemize}
%\item
%\ldots
%\end{itemize}

%%%%%%%%%%%%%%%%%%%%%%%%%%%%%%%%%%%%%%%%%%%%%%%%%%%%%%%%%%%%%%%%%%%%%%%%%%%%%%%%
\subsection{Revision History}

%%%%%%%%%%%%%%%%%%%%%%%%%%%%%%%%%%%%%%%%
\paragraph{v2.0:} 2018/12/30

\begin{itemize}
\item
immediate forward processing
\item
added |\childdocby| mechanism
\item
manual restructured
\end{itemize}

%%%%%%%%%%%%%%%%%%%%%%%%%%%%%%%%%%%%%%%%
\paragraph{v1.6:} 2018/01/17

\begin{itemize}
\item
application for development of include files
\item
corrections to manual
\end{itemize}

%%%%%%%%%%%%%%%%%%%%%%%%%%%%%%%%%%%%%%%%
\paragraph{v1.5:} 2017/05/21

\begin{itemize}
\item
more complete structuring introduced
\item
|\childdocof| introduced
\item
|\childdoc| renamed to |\childdocmain|
\item
|\childredirect| renamed to |\childdocforward| and |\childdocforwardprefix|
and functionality expanded
\end{itemize}

%%%%%%%%%%%%%%%%%%%%%%%%%%%%%%%%%%%%%%%%
\paragraph{v1.0:} 2017/04/27

\begin{itemize}
\item
manual and install package
\item
first version published on CTAN
\end{itemize}

%%%%%%%%%%%%%%%%%%%%%%%%%%%%%%%%%%%%%%%%
\paragraph{v0.6:} 2017/04/26

\begin{itemize}
\item
redirection mechanism added
\end{itemize}

%%%%%%%%%%%%%%%%%%%%%%%%%%%%%%%%%%%%%%%%
\paragraph{v0.5:} 2017/04/26

\begin{itemize}
\item
functionality in definition file
\end{itemize}


%%%%%%%%%%%%%%%%%%%%%%%%%%%%%%%%%%%%%%%%%%%%%%%%%%%%%%%%%%%%%%%%%%%%%%%%%%%%%%%%
%%%%%%%%%%%%%%%%%%%%%%%%%%%%%%%%%%%%%%%%%%%%%%%%%%%%%%%%%%%%%%%%%%%%%%%%%%%%%%%%
%%%%%%%%%%%%%%%%%%%%%%%%%%%%%%%%%%%%%%%%%%%%%%%%%%%%%%%%%%%%%%%%%%%%%%%%%%%%%%%%
\appendix

\settowidth\MacroIndent{\rmfamily\scriptsize 000\ }

 \DocInput{childdoc.dtx}

\end{document}
%</driver>
% \fi
%
% %%%%%%%%%%%%%%%%%%%%%%%%%%%%%%%%%%%%%%%%%%%%%%%%%%%%%%%%%%%%%%%%%%%%%%%%%%%%%%
% %%%%%%%%%%%%%%%%%%%%%%%%%%%%%%%%%%%%%%%%%%%%%%%%%%%%%%%%%%%%%%%%%%%%%%%%%%%%%%
% \section{Sample}
%\iffalse
%<*samplemain>
%\fi
%
% The following presents a sample document
% with two chapters, two parts, a title page,
% a compile flag as well as three forwarding files to set the flag.
% It consists of eight |.tex| files:
% \begin{center}
% \begin{tabular}{ll}
% |cdocsamp.tex|&main file\\
% |cdocsch1.tex|&include file for chapter 1\\
% |cdocsch2.tex|&include file for chapter 2\\
% |cdocspt3.tex|&include file for part 3\\
% |cdocspt4.tex|&include file for part 4\\
% |cdocsdrf.tex|&forwarding file for main file in draft mode\\
% |cdocsfi1.tex|&forwarding file for final version of chapter 1\\
% |cdocsfi2.tex|&forwarding file for final version of chapter 2\\
% \end{tabular}
% \end{center}
% Each of the eight files can be compiled directly by the \LaTeX{} compiler.
%
% %%%%%%%%%%%%%%%%%%%%%%%%%%%%%%%%%%%%%%
% \paragraph{Main File.}
%
% The main file is called |cdocsamp.tex|.
%
% Load the \textsf{childdoc} definitions and
% declare the filename for the main document:
%    \begin{macrocode}
\input{childdoc.def}
\childdocmain{}
%    \end{macrocode}

% Optional override for |\version| flag:
%    \begin{macrocode}
%%\ifchilddoc\else\providecommand{\version}{draft}\fi
%    \end{macrocode}

% Define the default values for the |\version| flag
% (|final| for the main file and |draft| for childs):
%    \begin{macrocode}
\ifchilddoc
\providecommand{\version}{draft}
\else
\providecommand{\version}{final}
\fi
%    \end{macrocode}

% Load the standard document class:
%    \begin{macrocode}
\documentclass[12pt]{article}
%    \end{macrocode}

% Start the document body:
%    \begin{macrocode}
\begin{document}
%    \end{macrocode}

% Declare a title page.
% Print title, part of document being processed and version flag:
%    \begin{macrocode}
\addtocounter{page}{-1}
\begin{center}
{\LARGE\bfseries{}childdoc example\par}
\vspace{1cm}
\ifchilddoc
\ifchilddocmanual part\else chapter\fi:
`\childdocname' of `\childdocjob'\par
\else
main document: `\childdocjob'\par
\fi
version: \version\par
\end{center}
\newpage
%    \end{macrocode}

% Manually include selected file,
% otherwise process as usual:
%    \begin{macrocode}
\ifchilddocmanual
\section*{part `\childdocname'}
\input{\childdocname}
\else
%    \end{macrocode}

% Include the two chapters:
%    \begin{macrocode}
\include{cdocsch1}
\include{cdocsch2}
%    \end{macrocode}

% Include the two parts unless only chapters should be displayed:
%    \begin{macrocode}
\ifchilddoc\else
\section{part three}
\input{cdocspt3}
\section{part four}
\input{cdocspt4}
\fi
%    \end{macrocode}

% Process as usual until here:
%    \begin{macrocode}
\fi
%    \end{macrocode}

% End of document body:
%    \begin{macrocode}
\end{document}
%    \end{macrocode}
%\iffalse
%</samplemain>
%\fi
%
% %%%%%%%%%%%%%%%%%%%%%%%%%%%%%%%%%%%%%%
% \paragraph{Chapter Include Files.}
%
% The include files are called |cdocsch1.tex| and |cdocsch2.tex|.
%
%\iffalse
%<*samplechap1|samplechap2>
%\fi

% Optional override for |\version| flag:
%    \begin{macrocode}
%%\providecommand{\version}{final}
%    \end{macrocode}

% Include the main document:
%    \begin{macrocode}
\input{childdoc.def}
\childdocof{cdocsamp}
%    \end{macrocode}

%\iffalse
%</samplechap1|samplechap2>
%\fi
%
%\iffalse
%<*samplechap1>
%\fi
% Some text for chapter 1:
%    \begin{macrocode}
\section{one}
some text in chapter one
%    \end{macrocode}

%\iffalse
%</samplechap1>
%\fi
% Some text for chapter 2:
%\iffalse
%<*samplechap2>
%\fi
%    \begin{macrocode}
\section{two}
more text in chapter two
%    \end{macrocode}

%\iffalse
%</samplechap2>
%\fi
%
% %%%%%%%%%%%%%%%%%%%%%%%%%%%%%%%%%%%%%%
% \paragraph{Part Include Files.}
%
% The include files are called |cdocspt3.tex| and |cdocspt4.tex|.
%
%\iffalse
%<*samplepart3|samplepart4>
%\fi

% Optional override for |\version| flag:
%    \begin{macrocode}
%%\providecommand{\version}{final}
%    \end{macrocode}

% Include the main document:
%    \begin{macrocode}
\input{childdoc.def}
\childdocby{cdocsamp}
%    \end{macrocode}

%\iffalse
%</samplepart3|samplepart4>
%\fi
%
%\iffalse
%<*samplepart3>
%\fi
% Some text for part 3:
%    \begin{macrocode}
some text in part three
%    \end{macrocode}

%\iffalse
%</samplepart3>
%\fi
% Some text for part 4:
%\iffalse
%<*samplepart4>
%\fi
%    \begin{macrocode}
more text in part four
%    \end{macrocode}

%\iffalse
%</samplepart4>
%\fi
%
% %%%%%%%%%%%%%%%%%%%%%%%%%%%%%%%%%%%%%%
% \paragraph{Forwarding for a Complete Draft.}
%
% The following forwarding file |cdocsdrf.tex|
% compiles the main document in draft mode:
%\iffalse
%<*sampledraft>
%\fi
%    \begin{macrocode}
\def\version{draft}
\input{childdoc.def}
\childdocforward{cdocsamp}
%    \end{macrocode}

%\iffalse
%</sampledraft>
%\fi
%
% %%%%%%%%%%%%%%%%%%%%%%%%%%%%%%%%%%%%%%
% \paragraph{Forwarding for Final Version of the Chapters.}
%
% The following forwarding files |cdocsfn1.tex| and |cdocsfn2.tex|
% (with identical content)
% compile the final versions of the child documents
% |cdocsch1.tex| and |cdocsch2.tex|, respectively:
%\iffalse
%<*samplefinal>
%\fi
%    \begin{macrocode}
\def\version{final}
\input{childdoc.def}
\childdocforwardprefix[cdocsamp]{cdocsfn}{cdocsch}
%    \end{macrocode}

%\iffalse
%</samplefinal>
%\fi
%
% %%%%%%%%%%%%%%%%%%%%%%%%%%%%%%%%%%%%%%
% \paragraph{Command Line Processing.}
%
% The following three command lines generate the output files
% |cdocscld|, |cdocscl1| and |cdocscl2|
% which should be identical to
% |cdocsdrf|, |cdocsch1| and |cdocsfn2|, respectively:
% \begin{center}
% \begin{tabular}{l}
% |latex -jobname cdocscld \|\\
% |  "\def\version{draft}\input{childdoc.def}\childdocforward{cdocsamp}"|\\
% |latex -jobname cdocscl1 \|\\
% |  "\input{childdoc.def}\childdocforward[cdocsamp]{cdocsch1}"|\\
% |latex -jobname cdocscl2 \|\\
% |  "\def\version{final}\input{childdoc.def}\childdocforward{cdocsch2}"|
% \end{tabular}
% \end{center}
% Note that the trailing backslash on each first line
% merely continues the input to the second line
% (for convenient cut ant paste).
% Furthermore, the command |latex| can be replaced by any
% of its alternative versions such as |pdflatex|.
%
% %%%%%%%%%%%%%%%%%%%%%%%%%%%%%%%%%%%%%%%%%%%%%%%%%%%%%%%%%%%%%%%%%%%%%%%%%%%%%%
% %%%%%%%%%%%%%%%%%%%%%%%%%%%%%%%%%%%%%%%%%%%%%%%%%%%%%%%%%%%%%%%%%%%%%%%%%%%%%%
% \section{Implementation}
%\iffalse
%<*package>
%\fi
%
% This section describes the definitions file |childdoc.def|.

% The definitions cannot be loaded using |\usepackage| or |\RequirePackage|
% which has a mechanism to prevent loading a style file more than once.
% When loading the definitions by means of |\input|
% multiple instances have to be prevented manually:
%\iffalse
%This code needs to be before the `\ProvidesFile' directive
%which is defined at the beginning of this file.
%Therefore it is also placed there and commented out here.
%</package>
%<*discard>
%\fi
%    \begin{macrocode}
\ifdefined\childdocmain\endinput\fi
%    \end{macrocode}
%\iffalse
%</discard>
%<*package>
%\fi
%
% \macro{\ifchilddoc}
% \macro{\ifchilddocmanual}
% The conditional |\ifchilddoc| tells whether a
% child (true) or main (false) document is being compiled.
% The conditional |\ifchilddocmanual| tells whether
% the |\includeonly| mechanism is used (false) or
% the selection of child files must be performed manually (true).
% The definitions initialise to false:
%    \begin{macrocode}
\newif\ifchilddoc
\newif\ifchilddocmanual
%    \end{macrocode}

% \macro{\childdocname}
% \macro{\childdocjob}
% The macro |\childdocname| stores the name of the main document
% to be compiled. The macro |\childdocjob| stores the name of
% the document on which the \LaTeX{} compiler was originally invoked.
% The content of |\jobname| cannot be compared
% to filenames specified in the source due to different catcodes.
% The following code rescans |\jobname|, stores the result
% in |\childdocname| and saves a copy in |\childdocjob|:
%    \begin{macrocode}
\edef\childdocname{\scantokens\expandafter{\jobname\noexpand}}
\let\childdocjob\childdocname
%    \end{macrocode}

% \macro{\childdocdisable}
% The macro |\childdocdisable| prevents the main file
% from being processed more than once.
% At this stage, the main document command |\childdocmain|
% is assumed to be called once again where it should do nothing.
% Any subsequent call to it should prevent
% a secondary processing of the main document
% It overwrites the forwarding commands
% |\childdocof| and |\childdocforward|
% with empty macros to prevent further inclusions of the main document:
%    \begin{macrocode}
\newcommand{\childdocdisable}
{
  \renewcommand{\childdocmain}[1]{\renewcommand{\childdocmain}[1]{\endinput}}
  \renewcommand{\childdocof}[1]{}
  \renewcommand{\childdocby}[2][]{}
  \renewcommand{\childdocforward}[2][]{}
  \renewcommand{\childdocdisable}{}
}
%    \end{macrocode}

% \macro{\childdocmain}
% The macro |\childdocmain| is to be called at the top of the main file
% with nothing or the main filename (without extension) as argument.
% First, it breaks loops.
% If the argument is not empty and does not match |\childdocname|
% (which is set by the first inclusion of |childdoc.def|),
% |\ifchilddoc| is set to true, |\includeonly| is applied to the child file
% and |\jobname| is set to the main file
% (for proper handling of |.aux| files):
%    \begin{macrocode}
\newcommand{\childdocmain}[1]
{
  \childdocdisable\childdocmain{}
  \if?#1?\else
    \begingroup
      \def\childdoctmp{#1}
      \ifx\childdoctmp\childdocname
        \def\childdoctmp{}
      \else
        \def\childdoctmp
        {
          \childdoctrue
          \includeonly{\childdocname}
          \def\childdocjob{#1}
          \def\jobname{#1}
        }
      \fi
      \expandafter
    \endgroup
    \childdoctmp
  \fi
}
%    \end{macrocode}

% \macro{\childdocof}
% The command |\childdocof| redirects
% compilation to the main file |#1|.
%    \begin{macrocode}
\newcommand{\childdocof}[1]
{
  \childdocdisable
  \childdoctrue
  \includeonly{\childdocname}
  \def\jobname{#1}
  \def\childdocjob{#1}
  \input{#1}
}
%    \end{macrocode}

% \macro{\childdocby}
% The command |\childdocby| ....
%    \begin{macrocode}
\newcommand{\childdocby}[2][]
{
  \childdocdisable
  \childdoctrue
  \childdocmanualtrue
  \if?#1?\else
    \def\jobname{#2}
  \fi
  \def\childdocjob{#2}
  \input{#2}
  \endinput
}
%    \end{macrocode}

% \macro{\childdocforward}
% The command |\childdocforward| redirects
% compilation to the main file or
% (if the optional argument is given) a child file.
% Parameters are set as if the main file
% or a child file starting with |\childdocof| was compiled.
% Then compilation is handed over to the main file:
%    \begin{macrocode}
\newcommand{\childdocforward}[2][]
{
  \begingroup
    \if?#1?
      \def\childdoctmp
      {
        \def\childdocname{#2}
        \def\childdocjob{#2}
        \def\jobname{#2}
        \input{#2}
        \endinput
      }
    \else
      \def\childdoctmp
      {
        \childdocdisable
        \def\childdocname{#2}
        \childdoctrue
        \includeonly{#2}
        \def\childdocjob{#1}
        \def\jobname{#1}
        \input{#1}
        \endinput
      }
    \fi
    \expandafter
  \endgroup
  \childdoctmp
}
%    \end{macrocode}

% \macro{\childdocforwardprefix}
% The command |\childdocforwardprefix| redirects
% compilation to the main or a child file by means of a pattern.
% The prefix |#1| in the current filename is replaced by |#2|
% and the suffix of the current filename is kept
% (it is assumed that the filename does not contain the substring `|~~~|'
% which is used as a delimiter).
% Compilation is handed over to the new file by |\childdocforward|:
%    \begin{macrocode}
\newcommand{\childdocforwardprefix}[3][]
{
  \begingroup
    \def\childdocextract #2##1~~~{\def\childdoctmp{\childdocforward[#1]{#3##1}}}
    \expandafter\childdocextract\childdocname~~~
    \expandafter
  \endgroup
  \childdoctmp
}
%    \end{macrocode}

% \macro{\childdoc}
% The deprecated macro |\childdoc| is a legacy version of |\childdocmain|:
%    \begin{macrocode}
\newcommand{\childdoc}{\childdocmain}
%    \end{macrocode}

% \macro{\childdocredirect}
% The deprecated macro |\childdocredirect| is a legacy version
% of |\childdocforward| and |\childdocforwardprefix|:
%    \begin{macrocode}
\newcommand{\childdocredirect}[2][]
{
  \begingroup
    \if?#1?
      \def\childdoctmp{\childdocforward{#2}}
    \else
      \def\childdoctmp{\childdocforwardprefix{#1}{#2}}
    \fi
    \expandafter
  \endgroup
  \childdoctmp
}
%    \end{macrocode}

%\iffalse
%</package>
%\fi
%
\endinput
\childdocforward[|\textit{main}|]{|\textit{dest}|}"|
\end{center}
%
Here \textit{target} is the name of the output file,
\textit{main} is the name of the main file
and \textit{dest} is the name of the main or child file to be processed
(all filenames without extensions).
The optional argument \textit{main} can be omitted
if \textit{main} matches \textit{dest}.
Optionally, compilation \textit{flags} can be defined via |\def| commands.
This command line makes the \TeX{} engine believe
it is compiling the file \textit{target}
whose content is specified as the latter parameter.
The provided code then forwards the processing to
\textit{main} or \textit{dest} as described in \secref{sec:forward}.

%%%%%%%%%%%%%%%%%%%%%%%%%%%%%%%%%%%%%%%%%%%%%%%%%%%%%%%%%%%%%%%%%%%%%%%%%%%%%%%%
\subsection{Include by Input}
\label{sec:input}

Including child documents by |\include| has some restrictions by design.
Most notably, the content of a child document always occupies
its own set of pages; pages cannot be shared between child documents.
Usually, this behaviour makes perfect sense
because each child document contain an essential part of the document.
However, in some situations it may be desirable to compose
a document from a collection of parts
without having mandatory page breaks between then.
For this case, the package
provides a mechanism to include parts
by |\input| which can also be processed individually.
However, by construction this mechanism
requires manual handling of the content to be output.

%%%%%%%%%%%%%%%%%%%%%%%%%%%%%%%%%%%%%%%%
\DescribeMacro{\ifchilddocmanual}
The main file should be prepared as usual, see \secref{sec:include}.
However, the document body must make a distinction
between processing of an individual part and of the main document, e.g.:
%
\begin{center}
\begin{tabular}{l}
|\ifchilddocmanual|\\
|\input{\childdocname}|\\
|\||else|\\
\textit{document body with }|\input{|\textit{part}|}|\\
|\||fi|
\end{tabular}
\end{center}
%
The conditional |\ifchilddocmanual| is true whenever
a part to be included by |\input| is being compiled,
and the name of the part is stored in |\childdocname|.

%%%%%%%%%%%%%%%%%%%%%%%%%%%%%%%%%%%%%%%%
\DescribeMacro{\childdocby}
Each part to be included by |\input| should start with:
%
\begin{center}
\begin{tabular}{l}
|% \iffalse
%
% childdoc.dtx Copyright (C) 2017-2018 Niklas Beisert
%
% This work may be distributed and/or modified under the
% conditions of the LaTeX Project Public License, either version 1.3
% of this license or (at your option) any later version.
% The latest version of this license is in
%   http://www.latex-project.org/lppl.txt
% and version 1.3 or later is part of all distributions of LaTeX
% version 2005/12/01 or later.
%
% This work has the LPPL maintenance status `maintained'.
%
% The Current Maintainer of this work is Niklas Beisert.
%
% This work consists of the files childdoc.dtx and childdoc.ins
% and the derived files childdoc.def and cdocsamp.tex with
% cdocsch1.tex, cdocsch2.tex, cdocsdrf.tex, cdocsfn1.tex, cdocsfn2.tex.
%
%<package>\ifdefined\childdocmain\endinput\fi
%<package>\ProvidesFile{childdoc.def}[2018/12/30 v2.0 child document driver]
%<samplemain>\ProvidesFile{cdocsamp.tex}[2018/12/30 v2.0 sample for childdoc]
%<*driver>
%\ProvidesFile{childdoc.drv}[2018/12/30 v2.0 childdoc reference manual file]
\PassOptionsToClass{10pt,a4paper}{article}
\documentclass{ltxdoc}

\usepackage[margin=35mm]{geometry}
\usepackage{hyperref}
\usepackage{hyperxmp}
\usepackage[usenames]{color}

\hypersetup{colorlinks=true}
\hypersetup{pdfstartview=FitH}
\hypersetup{pdfpagemode=UseNone}
\hypersetup{pdfsource={}}
\hypersetup{pdflang={en-UK}}
\hypersetup{pdfcopyright={Copyright 2017-2018 Niklas Beisert.
  This work may be distributed and/or modified under the
  conditions of the LaTeX Project Public License, either version 1.3
  of this license or (at your option) any later version.}}
\hypersetup{pdflicenseurl={http://www.latex-project.org/lppl.txt}}
\hypersetup{pdfcontactaddress={ETH Zurich, ITP, HIT K,
  Wolfgang-Pauli-Strasse 27}}
\hypersetup{pdfcontactpostcode={8093}}
\hypersetup{pdfcontactcity={Zurich}}
\hypersetup{pdfcontactcountry={Switzerland}}
\hypersetup{pdfcontactemail={nbeisert@itp.phys.ethz.ch}}
\hypersetup{pdfcontacturl={http://people.phys.ethz.ch/\xmptilde nbeisert/}}

\newcommand{\secref}[1]{\hyperref[#1]{section \ref*{#1}}}

\parskip1ex
\parindent0pt
\let\olditemize\itemize
\def\itemize{\olditemize\parskip0pt}

\begin{document}

\title{The \textsf{childdoc} Package}
\hypersetup{pdftitle={The childdoc Package}}
\author{Niklas Beisert\\[2ex]
  Institut f\"ur Theoretische Physik\\
  Eidgen\"ossische Technische Hochschule Z\"urich\\
  Wolfgang-Pauli-Strasse 27, 8093 Z\"urich, Switzerland\\[1ex]
  \href{mailto:nbeisert@itp.phys.ethz.ch}
  {\texttt{nbeisert@itp.phys.ethz.ch}}}
\hypersetup{pdfauthor={Niklas Beisert}}
\hypersetup{pdfsubject={Manual for the LaTeX2e Package childdoc}}
\date{30 December 2018, \textsf{v2.0}}
\maketitle

\begin{abstract}\noindent
\textsf{childdoc} is a \LaTeXe{} package
that enables the direct compilation
of document sections included by |\include|
to individual files.
\end{abstract}

\begingroup
\parskip0ex
\tableofcontents
\endgroup

%%%%%%%%%%%%%%%%%%%%%%%%%%%%%%%%%%%%%%%%%%%%%%%%%%%%%%%%%%%%%%%%%%%%%%%%%%%%%%%%
%%%%%%%%%%%%%%%%%%%%%%%%%%%%%%%%%%%%%%%%%%%%%%%%%%%%%%%%%%%%%%%%%%%%%%%%%%%%%%%%
\section{Introduction}

\LaTeX{} provides a mechanism to structure a large document (such as a book)
into a main file and several child files (containing the chapters)
using the |\include| command.
This mechanism is beneficial for documents
which span hundreds of pages in order to
make the source file(s) more manageable.
Moreover, compilation can be restricted to
selected child files by means of the |\includeonly| command.
The latter feature can be used to reduce the compilation time while editing
(this was significantly more useful in the earlier days of \LaTeX{})
or to generate a smaller document which is easier to navigate.
Another application of |\includeonly| is to generate
documents consisting of selected parts of the complete document.

However, there are a few drawbacks of the plain |\include| mechanism:
\begin{itemize}
\item
The child files cannot be compiled on their own,
they can only be compiled via the main file.
A naive editing environment
(such as a text editor with an option
to have the current file processed by \LaTeX)
may require one to switch to the main file before compiling;
attempting to compile the child file produces errors.
\item
The main file must be modified (each time)
to adjust the |\includeonly| command
to the present needs. This easily leaves the main file in a messy state.
\item
The generated document will always carry the filename
of the main document. This is inconvenient if
several child files are to be compiled and
to be kept for distribution.
\end{itemize}

The present package provides a simple interface
to make child files individually compilable by \LaTeX{}.
Compiling a child file then has the same effect as compiling
the main file with an |\includeonly| command
to select the appropriate child.
Moreover the generated document will carry the name of the child
rather than the main file.
This resolves all three above issues.

This feature is meant to make the editing of books,
thesis documents and lecture notes somewhat more convenient.
However, the package can also be used efficiently for
composing a series of documents (such as exercise sheets)
which are typically distributed individually.
It then assists the author in generating the individual documents
(potentially in different versions)
as well as a document containing the collected series.
Another application is in developing style files
or other kinds of included material
where compilation of the style file could redirect
to a sample or test file.

%%%%%%%%%%%%%%%%%%%%%%%%%%%%%%%%%%%%%%%%%%%%%%%%%%%%%%%%%%%%%%%%%%%%%%%%%%%%%%%%
%%%%%%%%%%%%%%%%%%%%%%%%%%%%%%%%%%%%%%%%%%%%%%%%%%%%%%%%%%%%%%%%%%%%%%%%%%%%%%%%
\section{Usage}

First of all, the package \textsf{childdoc} is \emph{not} a standard
\LaTeXe{} |.sty| style file! Therefore it needs to be invoked in
a non-standard way.

%%%%%%%%%%%%%%%%%%%%%%%%%%%%%%%%%%%%%%%%%%%%%%%%%%%%%%%%%%%%%%%%%%%%%%%%%%%%%%%%
\subsection{Included Files}
\label{sec:include}

%%%%%%%%%%%%%%%%%%%%%%%%%%%%%%%%%%%%%%%%
\DescribeMacro{\childdocmain}
To use the package, add the commands
\begin{center}
\begin{tabular}{l}
|\input{childdoc.def}|\\
|\childdocmain{}|\\
\end{tabular}
\end{center}
at the very top of the main \LaTeX{} file,
in particular \emph{before} the |\documentclass| statement!
The argument of |\childdocmain| should be left empty
(but it must be present).

%%%%%%%%%%%%%%%%%%%%%%%%%%%%%%%%%%%%%%%%
\DescribeMacro{\childdocof}
Furthermore, add the commands
\begin{center}
\begin{tabular}{l}
|\input{childdoc.def}|\\
|\childdocof{|\textit{main}|}|\\
\end{tabular}
\end{center}
at the top of every child file \textit{child}
which is included by |\include{|\textit{child}|}|
from within the main file
(or at least for those files to be compiled individually).
The argument \textit{main} must be the filename of the main file.

There are a couple of
considerations in setting up the main and child documents:

%%%%%%%%%%%%%%%%%%%%%%%%%%%%%%%%%%%%%%%%
\paragraph{Restrictions.}

Please note the following restrictions:
\begin{itemize}
\item
|\childdocmain| must be called with one argument \textit{main}
to ensure compatibility with earlier version of the package.
It must either be empty (|\childdocmain{}|)
or precisely match the filename of the main file in which it is specified.
See \secref{sec:detection} for further information.
\item
The filename \textit{main} must be specified without the |.tex| extension.
\item
The filename \textit{main} is case sensitive
(even in case-insensitive file systems)
due to internal string comparison.
\item
The argument \textit{main} should be fully expanded, it cannot be a macro.
\item
Subdirectories and special characters should be avoided in filenames.
\item
The command |\childdocmain{|\textit{main}|}| must be followed by a whitespace.
It should not be followed immediately by another command
or by a comment mark `|%|'.
This is because the \TeX{} parser reads the token immediately following
the argument of |\childdocmain| and puts it
at the beginning of every child section;
however, a white\-space is ignored.
\end{itemize}

%%%%%%%%%%%%%%%%%%%%%%%%%%%%%%%%%%%%%%%%
\paragraph{Content of Main File.}

It is advisable to place all content in the child files included by |\include|.
Any output contained in the main file will appear in all child documents
unless suppressed manually;
it cannot be suppressed automatically by the |\includeonly| directive
and thus should normally be avoided.
A method to include some content in the main file
by means of conditional processing is described in \secref{sec:conditional}.

%%%%%%%%%%%%%%%%%%%%%%%%%%%%%%%%%%%%%%%%
\paragraph{Page Numbering.}

When only a part of the document is compiled,
the appropriate numbering of pages
(as well as other status parameters)
is determined from the |.aux| files.
The latter contain information from previous passes.
However this information needs to propagate through
all intermediate child documents.
Therefore the page numbering in child documents may well
be inconsistent until the complete document is compiled at least once.

A useful (if unconventional) way to always ensure a consistent
page numbering is to restart the numbering in each child document
and denote the pages by `\textit{child}|.|\textit{page}'
where \textit{child} represents the chapter/section number of the child file.
This can be achieved by the command
|\numberwithin{page}{|\textit{child}|}|
of the \textsf{amsmath} package
where \textit{child} can be |chapter| or |section|
depending on the chosen structuring.
Alternatively, one can modify the macro |\thepage| appropriately
and reset the counter |page| at the start of each child file.

%%%%%%%%%%%%%%%%%%%%%%%%%%%%%%%%%%%%%%%%%%%%%%%%%%%%%%%%%%%%%%%%%%%%%%%%%%%%%%%%
\subsection{Conditional Processing}
\label{sec:conditional}

The package provides a mechanism to compile different versions
of a document. To customise the versions further some conditional processing
can come in handy to distinguish which version is being compiled.
The package provides two macros to describe the compilation context:

%%%%%%%%%%%%%%%%%%%%%%%%%%%%%%%%%%%%%%%%
\DescribeMacro{\ifchilddoc}
The conditional |\ifchilddoc| distinguishes between the compilation of
child documents and the main document:
%
\begin{center}
|\ifchilddoc |\textit{child-code}| |[|\||else |\textit{main-code}]| \||fi|
\end{center}

%%%%%%%%%%%%%%%%%%%%%%%%%%%%%%%%%%%%%%%%
\DescribeMacro{\childdocname}
\DescribeMacro{\childdocjob}
The macro |\childdocname| contains the filename (without extension)
of the main or child file being processed.
Note that |\childdocjob| will always contain the name of the main file.

%%%%%%%%%%%%%%%%%%%%%%%%%%%%%%%%%%%%%%%%
\paragraph{Title Page.}

Conditional processing can be used to include a title or banner page
in the main document when proper precautions are taken.
Importantly, the code in the main file should ensure that the page counter
(as well as other status parameters which are stored in the |.aux| files)
takes the same value after the conditional processing.
Otherwise the page numbers may take divergent values
depending on which part is compiled.

For example, a title page could be declared by:
%
\begin{center}
\begin{tabular}{l}
|\ifchilddoc\||else|\\
|\addtocounter{page}{-1}|\\
\textit{code for title page}\\
|\newpage|\\
|\||fi|
\end{tabular}
\end{center}
%
A banner page for the child documents can be generated by:
%
\begin{center}
\begin{tabular}{l}
|\ifchilddoc|\\
|\addtocounter{page}{-1}|\\
\textit{code for banner page}\\
|\newpage|\\
|\||fi|
\end{tabular}
\end{center}
%
Here one could write a message such as:
\begin{center}
|This is the part \childdocname{} of \childdocjob{}.|
\end{center}

%%%%%%%%%%%%%%%%%%%%%%%%%%%%%%%%%%%%%%%%%%%%%%%%%%%%%%%%%%%%%%%%%%%%%%%%%%%%%%%%
\subsection{Flags}
\label{sec:flags}

The package makes it easy to generate different versions
of the main or child documents.
To this end compilation flags can be defined
and assigned different default values.
They will be particularly useful in conjunction
with the forwarding mechanism described in \secref{sec:forward}.

For example, it may be useful to have a flag |\version|
which can be set to |draft| or |final|.
The document source will contain some conditional code
depending on the value of |\version|.
Suppose further, the flag should default to |final| for the main file
and to |draft| for child files
which is a natural assignment for editing the document.
This is achieved by placing the following code
in the preamble of the main document
(below the |\childdocmain| directive):
%
\begin{center}
\begin{tabular}{l}
|\ifchilddoc|\\
|\providecommand{\version}{draft}|\\
|\||else|\\
|\providecommand{\version}{final}|\\
|\||fi|
\end{tabular}
\end{center}
%
The definition by |\providecommand| makes sure
that previous definitions are not overwritten.
Further statements |\providecommand{\version}{...}|
can thus be added before the above code to override it.

For the main file, one might add a line
(between |\childdocmain| and the above block)
%
\begin{center}
|%\ifchilddoc\||else\providecommand{\version}{draft}\||fi|
\end{center}
%
which can be uncommented to produce a draft version.
Likewise one can add a line to the very top of a child file
(above the |\childdocof{|\textit{main}|}| directive)
%
\begin{center}
|%\providecommand{\version}{final}|
\end{center}
%
which can be uncommented to produce the final version of this child document.

%%%%%%%%%%%%%%%%%%%%%%%%%%%%%%%%%%%%%%%%%%%%%%%%%%%%%%%%%%%%%%%%%%%%%%%%%%%%%%%%
\subsection{Forwarding}
\label{sec:forward}

Different versions of the main or child documents
using compilation flags as described in \secref{sec:flags}
can be (permanently) stored in different files
for convenient compilation, viewing and distribution.
To this end, the package defines a command
to pass on compilation to a different file:

%%%%%%%%%%%%%%%%%%%%%%%%%%%%%%%%%%%%%%%%
\DescribeMacro{\childdocforward}
The command |\childdocforward| redirects processing to
another source file:
%
\begin{center}
\begin{tabular}{l}
|\input{childdoc.def}|\\
|\childdocforward[|\textit{main}|]{|\textit{dest}|}|\\
\end{tabular}
\end{center}
%
The argument \textit{dest} is the destination file
(without extension).
It should be the main file or one of the child files.
Note that further \textsf{childdoc} directives
such as |\childdocof| and |\childdocforward|
in the indicated file will be processed in this form.
The optional argument \textit{main}
passes on directly to the main file \textit{main}
while pretending to compile the child \textit{dest}.
This form behaves as if \textit{dest}
issues |\childdocof{|\textit{main}|}| right away,
and no further \textsf{childdoc} directives will be processed.

%%%%%%%%%%%%%%%%%%%%%%%%%%%%%%%%%%%%%%%%
\DescribeMacro{\...prefix}
In the alternative form |\childdocforwardprefix|,
%
\begin{center}
\begin{tabular}{l}
|\input{childdoc.def}|\\
|\childdocforwardprefix[|\textit{main}|]{|\textit{prefix}|}{|\textit{dest}|}|
\end{tabular}
\end{center}
%
the destination file is determined by a pattern
depending on the current file:
To make this work, the current file must be called
`{\textit{prefix}\hspace{0.2em}\textit{suffix}}'
with \textit{prefix} matching precisely the argument.
Processing is then passed on to the file
`{\textit{dest}\hspace{0.2em}\textit{suffix}}'.
Surely, the same effect is achieved by
directly specifying the
argument `{\textit{dest}\hspace{0.2em}\textit{suffix}}'
in the first form.
However, that requires to set up a different file
for each child. With the alternative form of the command
all these files can have exactly the same content
which simplifies setting them up and maintaining them.

For example, the following file |draft.tex|
with a compilation flag |\version| as described in \secref{sec:flags}
compiles the main document as a draft:
%
\begin{center}
\begin{tabular}{l}
|\def\version{draft}|\\
|\input{childdoc.def}|\\
|\childdocforward{|\textit{main}|}|
\end{tabular}
\end{center}
%
Likewise, the following files |final|\textit{nn}|.tex|
compile the final version of the child document
|child|\textit{nn}|.tex|:
%
\begin{center}
\begin{tabular}{l}
|\def\version{final}|\\
|\input{childdoc.def}|\\
|\childdocforwardprefix{final}{child}|
\end{tabular}
\end{center}
%

Note that when several versions of a main file and/or of each child file
are to be generated, it may be convenient to set up a |Makefile| or
shell script to automatise the process.

%%%%%%%%%%%%%%%%%%%%%%%%%%%%%%%%%%%%%%%%%%%%%%%%%%%%%%%%%%%%%%%%%%%%%%%%%%%%%%%%
\subsection{Command Line Processing}
\label{sec:commandline}

The effect of redirection files can also be achieved by invoking
the \LaTeX{} compiler with a more elaborate command line.
Most conveniently this should be done as part
of a shell script or a |Makefile|.

When using \textsf{childdoc} in the main file, the following
command lines effectively perform a redirection
(note that depending on the shell being used,
backslashes may have to be doubled: `|\|' $\to$ `|\\|'):
%
\begin{center}
|... -jobname "|\textit{target}|" |\\|"|[\textit{flags}]%
|\input{childdoc.def}\childdocforward[|\textit{main}|]{|\textit{dest}|}"|
\end{center}
%
Here \textit{target} is the name of the output file,
\textit{main} is the name of the main file
and \textit{dest} is the name of the main or child file to be processed
(all filenames without extensions).
The optional argument \textit{main} can be omitted
if \textit{main} matches \textit{dest}.
Optionally, compilation \textit{flags} can be defined via |\def| commands.
This command line makes the \TeX{} engine believe
it is compiling the file \textit{target}
whose content is specified as the latter parameter.
The provided code then forwards the processing to
\textit{main} or \textit{dest} as described in \secref{sec:forward}.

%%%%%%%%%%%%%%%%%%%%%%%%%%%%%%%%%%%%%%%%%%%%%%%%%%%%%%%%%%%%%%%%%%%%%%%%%%%%%%%%
\subsection{Include by Input}
\label{sec:input}

Including child documents by |\include| has some restrictions by design.
Most notably, the content of a child document always occupies
its own set of pages; pages cannot be shared between child documents.
Usually, this behaviour makes perfect sense
because each child document contain an essential part of the document.
However, in some situations it may be desirable to compose
a document from a collection of parts
without having mandatory page breaks between then.
For this case, the package
provides a mechanism to include parts
by |\input| which can also be processed individually.
However, by construction this mechanism
requires manual handling of the content to be output.

%%%%%%%%%%%%%%%%%%%%%%%%%%%%%%%%%%%%%%%%
\DescribeMacro{\ifchilddocmanual}
The main file should be prepared as usual, see \secref{sec:include}.
However, the document body must make a distinction
between processing of an individual part and of the main document, e.g.:
%
\begin{center}
\begin{tabular}{l}
|\ifchilddocmanual|\\
|\input{\childdocname}|\\
|\||else|\\
\textit{document body with }|\input{|\textit{part}|}|\\
|\||fi|
\end{tabular}
\end{center}
%
The conditional |\ifchilddocmanual| is true whenever
a part to be included by |\input| is being compiled,
and the name of the part is stored in |\childdocname|.

%%%%%%%%%%%%%%%%%%%%%%%%%%%%%%%%%%%%%%%%
\DescribeMacro{\childdocby}
Each part to be included by |\input| should start with:
%
\begin{center}
\begin{tabular}{l}
|\input{childdoc.def}|\\
|\childdocby{|\textit{main}|}|\\
\end{tabular}
\end{center}
%
The directive |\childdocby| is similar to |\childdocof|
described in \secref{sec:include},
but the subsequent selection of content must be done manually.
To that end, both |\ifchilddoc| and |\ifchilddocmanual|
will be true upon processing of a part,
and the name of the part is stored in |\childdocname|.
Note that |\jobname| will be set to the filename of the current part
so that each part receives an individual |.aux| file
that does not interfere with the |.aux| file(s) of the main document.
This behaviour can be altered by the alternative form
|\childdocby[*]{|\textit{main}|}| (with a non-empty optional argument)
which uses the |.aux| file of the main document
by setting |\jobname| to \textit{main}.

%%%%%%%%%%%%%%%%%%%%%%%%%%%%%%%%%%%%%%%%%%%%%%%%%%%%%%%%%%%%%%%%%%%%%%%%%%%%%%%%
\subsection{Driver Development}
\label{sec:driver}

The \textsf{childdoc} mechanism can also be use for the development
of definition files such as \LaTeX{} styles or classes.
This case differs from the above setup with multiple parts
included by |\include| in that no |\includeonly| should be invoked.
This can be achieved by starting the include file
(before |\ProvidesPackage|) with:
%
\begin{center}
\begin{tabular}{l}
|\input{childdoc.def}|\\
|\childdocforward{|\textit{main}|}|\\
\end{tabular}
\end{center}
%
or alternatively with:
%
\begin{center}
\begin{tabular}{l}
|\input{childdoc.def}|\\
|\childdocby{|\textit{main}|}|\\
\end{tabular}
\end{center}
%
Both forms have slightly different effects as described above.
The main file is prepared as usual, see \secref{sec:include}.

%%%%%%%%%%%%%%%%%%%%%%%%%%%%%%%%%%%%%%%%%%%%%%%%%%%%%%%%%%%%%%%%%%%%%%%%%%%%%%%%
\subsection{Legacy Detection}
\label{sec:detection}

The directive |\childdocmain| in the main file can detect
whether the complete document or merely a child is to be compiled
even without using the directive |\childdocof|.
This method is deprecated because it is less robust
and there is no compelling reason to use it;
it is merely provided for backward compatibility
and it may be removed in future versions.

If the detection mechanism is to be used,
it is mandatory to correctly specify
the filename of the main file as the argument of |\childdocmain|:
%
\begin{center}
\begin{tabular}{l}
|\input{childdoc.def}|\\
|\childdocmain{|\textit{main}|}|\\
\end{tabular}
\end{center}
%
If |\jobname| does not match the argument \textit{main} of |\childdocmain|,
it is assumed that |\jobname| points to the child file to be compiled.
When using |\childdocmain| with the main file specified as argument,
it suffices to start a child file
with just |\input{|\textit{main}|}|
without loading of the package and using |\childdocof|.
If instead all processing is done
with the appropriate \textsf{childdoc} directives,
the argument of \textit{main} of |\childdocmain| can be empty.

An alternative version of the command line processing described
in \secref{sec:commandline} using the detection mechanism reads:
%
\begin{center}
|... -jobname "|\textit{target}|" "|[\textit{flags}]%
[|\def\jobname{|\textit{dest}|}|]|\input{|\textit{main}|}"|
\end{center}

%%%%%%%%%%%%%%%%%%%%%%%%%%%%%%%%%%%%%%%%%%%%%%%%%%%%%%%%%%%%%%%%%%%%%%%%%%%%%%%%
\subsection{Manual Code}
\label{sec:manual}

In case one cannot be certain whether the definitions file |childdoc.def|
is installed on the target \TeX{} distribution
and one prefers not to ship it,
it is conceivable to paste a few relevant commands into the sources.

To that end, drop all statements |\input{childdoc.def}|
and perform the replacements as outlined below.
Instead of |\childdocmain{|\textit{main}|}| add the following code
to the top of the main file:
%
\begin{center}
\begin{tabular}{l}
|\||ifdefined\childdocname\endinput\||fi\newif\ifchilddoc|\\
|\edef\childdocname{\scantokens\expandafter{\jobname\noexpand}}|\\
|\def\childdocmain{|\textit{main}|}\||ifx\childdocmain\childdocname\||else|\\
|\childdoctrue\includeonly{\childdocname}\let\jobname\childdocmain\||fi|\\
\end{tabular}
\end{center}
%
Instead of |\childdocof{|\textit{main}|}| just include the main file
at the top of each child file:
%
\begin{center}
|\input{|\textit{main}|}|
\end{center}
%
A simple redirection |\childdocforward{|\textit{dest}|}| is achieved by:
%
\begin{center}
|\def\jobname{|\textit{dest}|}\input{\jobname}|
\end{center}
%
The redirection with prefix
|\childdocforwardprefix[|\textit{prefix}|]{|\textit{dest}|}|
is accomplished by:
%
\begin{center}
\begin{tabular}{l}
|{\edef\jobname{\scantokens\expandafter{\jobname\noexpand}}|\\
|\def\redirectjob |\textit{prefix}|#1~~~{\gdef\jobname{|\textit{dest}|#1}}|\\
|\expandafter\redirectjob\jobname~~~}\input{\jobname}|
\end{tabular}
\end{center}

In an alternative approach,
child documents can be compiled by a specific command line
without additional code or specific definitions:
%
\begin{center}
|... -jobname "|\textit{target}|" "|[\textit{flags}]%
|\includeonly{|\textit{dest}|}\input{|\textit{main}|}"|
\end{center}
%

%%%%%%%%%%%%%%%%%%%%%%%%%%%%%%%%%%%%%%%%%%%%%%%%%%%%%%%%%%%%%%%%%%%%%%%%%%%%%%%%
%%%%%%%%%%%%%%%%%%%%%%%%%%%%%%%%%%%%%%%%%%%%%%%%%%%%%%%%%%%%%%%%%%%%%%%%%%%%%%%%
\section{Information}

%%%%%%%%%%%%%%%%%%%%%%%%%%%%%%%%%%%%%%%%%%%%%%%%%%%%%%%%%%%%%%%%%%%%%%%%%%%%%%%%
\subsection{Copyright}

Copyright \copyright{} 2017--2018 Niklas Beisert

This work may be distributed and/or modified under the
conditions of the \LaTeX{} Project Public License, either version 1.3
of this license or (at your option) any later version.
The latest version of this license is in
  \url{http://www.latex-project.org/lppl.txt}
and version 1.3 or later is part of all distributions of \LaTeX{}
version 2005/12/01 or later.

This work has the LPPL maintenance status `maintained'.

The Current Maintainer of this work is Niklas Beisert.

This work consists of the files |README.txt|, |childdoc.ins| and |childdoc.dtx|
as well as the derived files |childdoc.def|, |cdocsamp.tex|
with |cdocsch1.tex|, |cdocsch2.tex|, |cdocspt3.tex|, |cdocspt4.tex|,
|cdocsdrf.tex|, |cdocsfn1.tex|, |cdocsfn2.tex|
as well as |childdoc.pdf|.

%%%%%%%%%%%%%%%%%%%%%%%%%%%%%%%%%%%%%%%%%%%%%%%%%%%%%%%%%%%%%%%%%%%%%%%%%%%%%%%%
\subsection{Files and Installation}

The package consists of the files:
%
\begin{center}
\begin{tabular}{ll}
    |README.txt|   & readme file \\
    |childdoc.ins| & installation file \\
    |childdoc.dtx| & source file \\
    |childdoc.def| & definition file \\
    |cdocsamp.tex| & sample main file \\
    |cdocsch1.tex| & sample include file \\
    |cdocsch2.tex| & sample include file \\
    |cdocspt3.tex| & sample part file \\
    |cdocspt4.tex| & sample part file \\
    |cdocsdrf.tex| & sample redirection file \\
    |cdocsfn1.tex| & sample redirection file \\
    |cdocsfn2.tex| & sample redirection file \\
    |childdoc.pdf| & manual
\end{tabular}
\end{center}
%
The distribution consists of the files
|README.txt|, |childdoc.ins| and |childdoc.dtx|.
%
\begin{itemize}
\item
Run (pdf)\LaTeX{} on |childdoc.dtx|
to compile the manual |childdoc.pdf| (this file).
\item
Run \LaTeX{} on |childdoc.ins| to create the definitions file |childdoc.def|
and the sample |cdocsamp.tex| with include files
|cdocsch1.tex|, |cdocsch2.tex|, |cdocspt3.tex|, |cdocspt4.tex|,
|cdocsdrf.tex|, |cdocsfn1.tex|, |cdocsfn2.tex|.
Then copy the file |childdoc.def| to an appropriate directory of your \LaTeX{}
distribution, e.g.\ \textit{texmf-root}|/tex/latex/childdoc|.
\end{itemize}

%%%%%%%%%%%%%%%%%%%%%%%%%%%%%%%%%%%%%%%%%%%%%%%%%%%%%%%%%%%%%%%%%%%%%%%%%%%%%%%%
\subsection{Related CTAN Packages}

There are several other packages which offer a similar functionality:
%
\begin{itemize}
\item
The packages
\href{http://ctan.org/pkg/docmute}{\textsf{docmute}},
\href{http://ctan.org/pkg/includex}{\textsf{includex}} and
\href{http://ctan.org/pkg/standalone}{\textsf{standalone}}
provide commands to include only the document body of
a child file thus allowing both files to be compiled individually.
\item
The packages \href{http://ctan.org/pkg/subdocs}{\textsf{subdocs}}
and \href{http://ctan.org/pkg/subfiles}{\textsf{subfiles}}
provide structures in which the main and child documents can be
encapsulated and allowing them to be compiled individually.
The inclusion mechanism is different from the conventional |\include|.
\item
The package \href{http://ctan.org/pkg/combine}{\textsf{combine}}
is an elaborate solution to combine several documents into one.
\end{itemize}
%
See also the CTAN topic \href{http://ctan.org/topic/subdocs}{\textsf{subdocs}}
for further related packages.
The present package differs from the above solutions in that
a document structure constructed with the conventional |\include| mechanism
just needs two extra commands at the top of every file
such that all constituent files can be compiled individually.

%%%%%%%%%%%%%%%%%%%%%%%%%%%%%%%%%%%%%%%%%%%%%%%%%%%%%%%%%%%%%%%%%%%%%%%%%%%%%%%%
%\subsection{Feature Suggestions}
%
%The following is a list of features which may be useful for future
%versions of this package:
%%
%\begin{itemize}
%\item
%\ldots
%\end{itemize}

%%%%%%%%%%%%%%%%%%%%%%%%%%%%%%%%%%%%%%%%%%%%%%%%%%%%%%%%%%%%%%%%%%%%%%%%%%%%%%%%
\subsection{Revision History}

%%%%%%%%%%%%%%%%%%%%%%%%%%%%%%%%%%%%%%%%
\paragraph{v2.0:} 2018/12/30

\begin{itemize}
\item
immediate forward processing
\item
added |\childdocby| mechanism
\item
manual restructured
\end{itemize}

%%%%%%%%%%%%%%%%%%%%%%%%%%%%%%%%%%%%%%%%
\paragraph{v1.6:} 2018/01/17

\begin{itemize}
\item
application for development of include files
\item
corrections to manual
\end{itemize}

%%%%%%%%%%%%%%%%%%%%%%%%%%%%%%%%%%%%%%%%
\paragraph{v1.5:} 2017/05/21

\begin{itemize}
\item
more complete structuring introduced
\item
|\childdocof| introduced
\item
|\childdoc| renamed to |\childdocmain|
\item
|\childredirect| renamed to |\childdocforward| and |\childdocforwardprefix|
and functionality expanded
\end{itemize}

%%%%%%%%%%%%%%%%%%%%%%%%%%%%%%%%%%%%%%%%
\paragraph{v1.0:} 2017/04/27

\begin{itemize}
\item
manual and install package
\item
first version published on CTAN
\end{itemize}

%%%%%%%%%%%%%%%%%%%%%%%%%%%%%%%%%%%%%%%%
\paragraph{v0.6:} 2017/04/26

\begin{itemize}
\item
redirection mechanism added
\end{itemize}

%%%%%%%%%%%%%%%%%%%%%%%%%%%%%%%%%%%%%%%%
\paragraph{v0.5:} 2017/04/26

\begin{itemize}
\item
functionality in definition file
\end{itemize}


%%%%%%%%%%%%%%%%%%%%%%%%%%%%%%%%%%%%%%%%%%%%%%%%%%%%%%%%%%%%%%%%%%%%%%%%%%%%%%%%
%%%%%%%%%%%%%%%%%%%%%%%%%%%%%%%%%%%%%%%%%%%%%%%%%%%%%%%%%%%%%%%%%%%%%%%%%%%%%%%%
%%%%%%%%%%%%%%%%%%%%%%%%%%%%%%%%%%%%%%%%%%%%%%%%%%%%%%%%%%%%%%%%%%%%%%%%%%%%%%%%
\appendix

\settowidth\MacroIndent{\rmfamily\scriptsize 000\ }

 \DocInput{childdoc.dtx}

\end{document}
%</driver>
% \fi
%
% %%%%%%%%%%%%%%%%%%%%%%%%%%%%%%%%%%%%%%%%%%%%%%%%%%%%%%%%%%%%%%%%%%%%%%%%%%%%%%
% %%%%%%%%%%%%%%%%%%%%%%%%%%%%%%%%%%%%%%%%%%%%%%%%%%%%%%%%%%%%%%%%%%%%%%%%%%%%%%
% \section{Sample}
%\iffalse
%<*samplemain>
%\fi
%
% The following presents a sample document
% with two chapters, two parts, a title page,
% a compile flag as well as three forwarding files to set the flag.
% It consists of eight |.tex| files:
% \begin{center}
% \begin{tabular}{ll}
% |cdocsamp.tex|&main file\\
% |cdocsch1.tex|&include file for chapter 1\\
% |cdocsch2.tex|&include file for chapter 2\\
% |cdocspt3.tex|&include file for part 3\\
% |cdocspt4.tex|&include file for part 4\\
% |cdocsdrf.tex|&forwarding file for main file in draft mode\\
% |cdocsfi1.tex|&forwarding file for final version of chapter 1\\
% |cdocsfi2.tex|&forwarding file for final version of chapter 2\\
% \end{tabular}
% \end{center}
% Each of the eight files can be compiled directly by the \LaTeX{} compiler.
%
% %%%%%%%%%%%%%%%%%%%%%%%%%%%%%%%%%%%%%%
% \paragraph{Main File.}
%
% The main file is called |cdocsamp.tex|.
%
% Load the \textsf{childdoc} definitions and
% declare the filename for the main document:
%    \begin{macrocode}
\input{childdoc.def}
\childdocmain{}
%    \end{macrocode}

% Optional override for |\version| flag:
%    \begin{macrocode}
%%\ifchilddoc\else\providecommand{\version}{draft}\fi
%    \end{macrocode}

% Define the default values for the |\version| flag
% (|final| for the main file and |draft| for childs):
%    \begin{macrocode}
\ifchilddoc
\providecommand{\version}{draft}
\else
\providecommand{\version}{final}
\fi
%    \end{macrocode}

% Load the standard document class:
%    \begin{macrocode}
\documentclass[12pt]{article}
%    \end{macrocode}

% Start the document body:
%    \begin{macrocode}
\begin{document}
%    \end{macrocode}

% Declare a title page.
% Print title, part of document being processed and version flag:
%    \begin{macrocode}
\addtocounter{page}{-1}
\begin{center}
{\LARGE\bfseries{}childdoc example\par}
\vspace{1cm}
\ifchilddoc
\ifchilddocmanual part\else chapter\fi:
`\childdocname' of `\childdocjob'\par
\else
main document: `\childdocjob'\par
\fi
version: \version\par
\end{center}
\newpage
%    \end{macrocode}

% Manually include selected file,
% otherwise process as usual:
%    \begin{macrocode}
\ifchilddocmanual
\section*{part `\childdocname'}
\input{\childdocname}
\else
%    \end{macrocode}

% Include the two chapters:
%    \begin{macrocode}
\include{cdocsch1}
\include{cdocsch2}
%    \end{macrocode}

% Include the two parts unless only chapters should be displayed:
%    \begin{macrocode}
\ifchilddoc\else
\section{part three}
\input{cdocspt3}
\section{part four}
\input{cdocspt4}
\fi
%    \end{macrocode}

% Process as usual until here:
%    \begin{macrocode}
\fi
%    \end{macrocode}

% End of document body:
%    \begin{macrocode}
\end{document}
%    \end{macrocode}
%\iffalse
%</samplemain>
%\fi
%
% %%%%%%%%%%%%%%%%%%%%%%%%%%%%%%%%%%%%%%
% \paragraph{Chapter Include Files.}
%
% The include files are called |cdocsch1.tex| and |cdocsch2.tex|.
%
%\iffalse
%<*samplechap1|samplechap2>
%\fi

% Optional override for |\version| flag:
%    \begin{macrocode}
%%\providecommand{\version}{final}
%    \end{macrocode}

% Include the main document:
%    \begin{macrocode}
\input{childdoc.def}
\childdocof{cdocsamp}
%    \end{macrocode}

%\iffalse
%</samplechap1|samplechap2>
%\fi
%
%\iffalse
%<*samplechap1>
%\fi
% Some text for chapter 1:
%    \begin{macrocode}
\section{one}
some text in chapter one
%    \end{macrocode}

%\iffalse
%</samplechap1>
%\fi
% Some text for chapter 2:
%\iffalse
%<*samplechap2>
%\fi
%    \begin{macrocode}
\section{two}
more text in chapter two
%    \end{macrocode}

%\iffalse
%</samplechap2>
%\fi
%
% %%%%%%%%%%%%%%%%%%%%%%%%%%%%%%%%%%%%%%
% \paragraph{Part Include Files.}
%
% The include files are called |cdocspt3.tex| and |cdocspt4.tex|.
%
%\iffalse
%<*samplepart3|samplepart4>
%\fi

% Optional override for |\version| flag:
%    \begin{macrocode}
%%\providecommand{\version}{final}
%    \end{macrocode}

% Include the main document:
%    \begin{macrocode}
\input{childdoc.def}
\childdocby{cdocsamp}
%    \end{macrocode}

%\iffalse
%</samplepart3|samplepart4>
%\fi
%
%\iffalse
%<*samplepart3>
%\fi
% Some text for part 3:
%    \begin{macrocode}
some text in part three
%    \end{macrocode}

%\iffalse
%</samplepart3>
%\fi
% Some text for part 4:
%\iffalse
%<*samplepart4>
%\fi
%    \begin{macrocode}
more text in part four
%    \end{macrocode}

%\iffalse
%</samplepart4>
%\fi
%
% %%%%%%%%%%%%%%%%%%%%%%%%%%%%%%%%%%%%%%
% \paragraph{Forwarding for a Complete Draft.}
%
% The following forwarding file |cdocsdrf.tex|
% compiles the main document in draft mode:
%\iffalse
%<*sampledraft>
%\fi
%    \begin{macrocode}
\def\version{draft}
\input{childdoc.def}
\childdocforward{cdocsamp}
%    \end{macrocode}

%\iffalse
%</sampledraft>
%\fi
%
% %%%%%%%%%%%%%%%%%%%%%%%%%%%%%%%%%%%%%%
% \paragraph{Forwarding for Final Version of the Chapters.}
%
% The following forwarding files |cdocsfn1.tex| and |cdocsfn2.tex|
% (with identical content)
% compile the final versions of the child documents
% |cdocsch1.tex| and |cdocsch2.tex|, respectively:
%\iffalse
%<*samplefinal>
%\fi
%    \begin{macrocode}
\def\version{final}
\input{childdoc.def}
\childdocforwardprefix[cdocsamp]{cdocsfn}{cdocsch}
%    \end{macrocode}

%\iffalse
%</samplefinal>
%\fi
%
% %%%%%%%%%%%%%%%%%%%%%%%%%%%%%%%%%%%%%%
% \paragraph{Command Line Processing.}
%
% The following three command lines generate the output files
% |cdocscld|, |cdocscl1| and |cdocscl2|
% which should be identical to
% |cdocsdrf|, |cdocsch1| and |cdocsfn2|, respectively:
% \begin{center}
% \begin{tabular}{l}
% |latex -jobname cdocscld \|\\
% |  "\def\version{draft}\input{childdoc.def}\childdocforward{cdocsamp}"|\\
% |latex -jobname cdocscl1 \|\\
% |  "\input{childdoc.def}\childdocforward[cdocsamp]{cdocsch1}"|\\
% |latex -jobname cdocscl2 \|\\
% |  "\def\version{final}\input{childdoc.def}\childdocforward{cdocsch2}"|
% \end{tabular}
% \end{center}
% Note that the trailing backslash on each first line
% merely continues the input to the second line
% (for convenient cut ant paste).
% Furthermore, the command |latex| can be replaced by any
% of its alternative versions such as |pdflatex|.
%
% %%%%%%%%%%%%%%%%%%%%%%%%%%%%%%%%%%%%%%%%%%%%%%%%%%%%%%%%%%%%%%%%%%%%%%%%%%%%%%
% %%%%%%%%%%%%%%%%%%%%%%%%%%%%%%%%%%%%%%%%%%%%%%%%%%%%%%%%%%%%%%%%%%%%%%%%%%%%%%
% \section{Implementation}
%\iffalse
%<*package>
%\fi
%
% This section describes the definitions file |childdoc.def|.

% The definitions cannot be loaded using |\usepackage| or |\RequirePackage|
% which has a mechanism to prevent loading a style file more than once.
% When loading the definitions by means of |\input|
% multiple instances have to be prevented manually:
%\iffalse
%This code needs to be before the `\ProvidesFile' directive
%which is defined at the beginning of this file.
%Therefore it is also placed there and commented out here.
%</package>
%<*discard>
%\fi
%    \begin{macrocode}
\ifdefined\childdocmain\endinput\fi
%    \end{macrocode}
%\iffalse
%</discard>
%<*package>
%\fi
%
% \macro{\ifchilddoc}
% \macro{\ifchilddocmanual}
% The conditional |\ifchilddoc| tells whether a
% child (true) or main (false) document is being compiled.
% The conditional |\ifchilddocmanual| tells whether
% the |\includeonly| mechanism is used (false) or
% the selection of child files must be performed manually (true).
% The definitions initialise to false:
%    \begin{macrocode}
\newif\ifchilddoc
\newif\ifchilddocmanual
%    \end{macrocode}

% \macro{\childdocname}
% \macro{\childdocjob}
% The macro |\childdocname| stores the name of the main document
% to be compiled. The macro |\childdocjob| stores the name of
% the document on which the \LaTeX{} compiler was originally invoked.
% The content of |\jobname| cannot be compared
% to filenames specified in the source due to different catcodes.
% The following code rescans |\jobname|, stores the result
% in |\childdocname| and saves a copy in |\childdocjob|:
%    \begin{macrocode}
\edef\childdocname{\scantokens\expandafter{\jobname\noexpand}}
\let\childdocjob\childdocname
%    \end{macrocode}

% \macro{\childdocdisable}
% The macro |\childdocdisable| prevents the main file
% from being processed more than once.
% At this stage, the main document command |\childdocmain|
% is assumed to be called once again where it should do nothing.
% Any subsequent call to it should prevent
% a secondary processing of the main document
% It overwrites the forwarding commands
% |\childdocof| and |\childdocforward|
% with empty macros to prevent further inclusions of the main document:
%    \begin{macrocode}
\newcommand{\childdocdisable}
{
  \renewcommand{\childdocmain}[1]{\renewcommand{\childdocmain}[1]{\endinput}}
  \renewcommand{\childdocof}[1]{}
  \renewcommand{\childdocby}[2][]{}
  \renewcommand{\childdocforward}[2][]{}
  \renewcommand{\childdocdisable}{}
}
%    \end{macrocode}

% \macro{\childdocmain}
% The macro |\childdocmain| is to be called at the top of the main file
% with nothing or the main filename (without extension) as argument.
% First, it breaks loops.
% If the argument is not empty and does not match |\childdocname|
% (which is set by the first inclusion of |childdoc.def|),
% |\ifchilddoc| is set to true, |\includeonly| is applied to the child file
% and |\jobname| is set to the main file
% (for proper handling of |.aux| files):
%    \begin{macrocode}
\newcommand{\childdocmain}[1]
{
  \childdocdisable\childdocmain{}
  \if?#1?\else
    \begingroup
      \def\childdoctmp{#1}
      \ifx\childdoctmp\childdocname
        \def\childdoctmp{}
      \else
        \def\childdoctmp
        {
          \childdoctrue
          \includeonly{\childdocname}
          \def\childdocjob{#1}
          \def\jobname{#1}
        }
      \fi
      \expandafter
    \endgroup
    \childdoctmp
  \fi
}
%    \end{macrocode}

% \macro{\childdocof}
% The command |\childdocof| redirects
% compilation to the main file |#1|.
%    \begin{macrocode}
\newcommand{\childdocof}[1]
{
  \childdocdisable
  \childdoctrue
  \includeonly{\childdocname}
  \def\jobname{#1}
  \def\childdocjob{#1}
  \input{#1}
}
%    \end{macrocode}

% \macro{\childdocby}
% The command |\childdocby| ....
%    \begin{macrocode}
\newcommand{\childdocby}[2][]
{
  \childdocdisable
  \childdoctrue
  \childdocmanualtrue
  \if?#1?\else
    \def\jobname{#2}
  \fi
  \def\childdocjob{#2}
  \input{#2}
  \endinput
}
%    \end{macrocode}

% \macro{\childdocforward}
% The command |\childdocforward| redirects
% compilation to the main file or
% (if the optional argument is given) a child file.
% Parameters are set as if the main file
% or a child file starting with |\childdocof| was compiled.
% Then compilation is handed over to the main file:
%    \begin{macrocode}
\newcommand{\childdocforward}[2][]
{
  \begingroup
    \if?#1?
      \def\childdoctmp
      {
        \def\childdocname{#2}
        \def\childdocjob{#2}
        \def\jobname{#2}
        \input{#2}
        \endinput
      }
    \else
      \def\childdoctmp
      {
        \childdocdisable
        \def\childdocname{#2}
        \childdoctrue
        \includeonly{#2}
        \def\childdocjob{#1}
        \def\jobname{#1}
        \input{#1}
        \endinput
      }
    \fi
    \expandafter
  \endgroup
  \childdoctmp
}
%    \end{macrocode}

% \macro{\childdocforwardprefix}
% The command |\childdocforwardprefix| redirects
% compilation to the main or a child file by means of a pattern.
% The prefix |#1| in the current filename is replaced by |#2|
% and the suffix of the current filename is kept
% (it is assumed that the filename does not contain the substring `|~~~|'
% which is used as a delimiter).
% Compilation is handed over to the new file by |\childdocforward|:
%    \begin{macrocode}
\newcommand{\childdocforwardprefix}[3][]
{
  \begingroup
    \def\childdocextract #2##1~~~{\def\childdoctmp{\childdocforward[#1]{#3##1}}}
    \expandafter\childdocextract\childdocname~~~
    \expandafter
  \endgroup
  \childdoctmp
}
%    \end{macrocode}

% \macro{\childdoc}
% The deprecated macro |\childdoc| is a legacy version of |\childdocmain|:
%    \begin{macrocode}
\newcommand{\childdoc}{\childdocmain}
%    \end{macrocode}

% \macro{\childdocredirect}
% The deprecated macro |\childdocredirect| is a legacy version
% of |\childdocforward| and |\childdocforwardprefix|:
%    \begin{macrocode}
\newcommand{\childdocredirect}[2][]
{
  \begingroup
    \if?#1?
      \def\childdoctmp{\childdocforward{#2}}
    \else
      \def\childdoctmp{\childdocforwardprefix{#1}{#2}}
    \fi
    \expandafter
  \endgroup
  \childdoctmp
}
%    \end{macrocode}

%\iffalse
%</package>
%\fi
%
\endinput
|\\
|\childdocby{|\textit{main}|}|\\
\end{tabular}
\end{center}
%
The directive |\childdocby| is similar to |\childdocof|
described in \secref{sec:include},
but the subsequent selection of content must be done manually.
To that end, both |\ifchilddoc| and |\ifchilddocmanual|
will be true upon processing of a part,
and the name of the part is stored in |\childdocname|.
Note that |\jobname| will be set to the filename of the current part
so that each part receives an individual |.aux| file
that does not interfere with the |.aux| file(s) of the main document.
This behaviour can be altered by the alternative form
|\childdocby[*]{|\textit{main}|}| (with a non-empty optional argument)
which uses the |.aux| file of the main document
by setting |\jobname| to \textit{main}.

%%%%%%%%%%%%%%%%%%%%%%%%%%%%%%%%%%%%%%%%%%%%%%%%%%%%%%%%%%%%%%%%%%%%%%%%%%%%%%%%
\subsection{Driver Development}
\label{sec:driver}

The \textsf{childdoc} mechanism can also be use for the development
of definition files such as \LaTeX{} styles or classes.
This case differs from the above setup with multiple parts
included by |\include| in that no |\includeonly| should be invoked.
This can be achieved by starting the include file
(before |\ProvidesPackage|) with:
%
\begin{center}
\begin{tabular}{l}
|% \iffalse
%
% childdoc.dtx Copyright (C) 2017-2018 Niklas Beisert
%
% This work may be distributed and/or modified under the
% conditions of the LaTeX Project Public License, either version 1.3
% of this license or (at your option) any later version.
% The latest version of this license is in
%   http://www.latex-project.org/lppl.txt
% and version 1.3 or later is part of all distributions of LaTeX
% version 2005/12/01 or later.
%
% This work has the LPPL maintenance status `maintained'.
%
% The Current Maintainer of this work is Niklas Beisert.
%
% This work consists of the files childdoc.dtx and childdoc.ins
% and the derived files childdoc.def and cdocsamp.tex with
% cdocsch1.tex, cdocsch2.tex, cdocsdrf.tex, cdocsfn1.tex, cdocsfn2.tex.
%
%<package>\ifdefined\childdocmain\endinput\fi
%<package>\ProvidesFile{childdoc.def}[2018/12/30 v2.0 child document driver]
%<samplemain>\ProvidesFile{cdocsamp.tex}[2018/12/30 v2.0 sample for childdoc]
%<*driver>
%\ProvidesFile{childdoc.drv}[2018/12/30 v2.0 childdoc reference manual file]
\PassOptionsToClass{10pt,a4paper}{article}
\documentclass{ltxdoc}

\usepackage[margin=35mm]{geometry}
\usepackage{hyperref}
\usepackage{hyperxmp}
\usepackage[usenames]{color}

\hypersetup{colorlinks=true}
\hypersetup{pdfstartview=FitH}
\hypersetup{pdfpagemode=UseNone}
\hypersetup{pdfsource={}}
\hypersetup{pdflang={en-UK}}
\hypersetup{pdfcopyright={Copyright 2017-2018 Niklas Beisert.
  This work may be distributed and/or modified under the
  conditions of the LaTeX Project Public License, either version 1.3
  of this license or (at your option) any later version.}}
\hypersetup{pdflicenseurl={http://www.latex-project.org/lppl.txt}}
\hypersetup{pdfcontactaddress={ETH Zurich, ITP, HIT K,
  Wolfgang-Pauli-Strasse 27}}
\hypersetup{pdfcontactpostcode={8093}}
\hypersetup{pdfcontactcity={Zurich}}
\hypersetup{pdfcontactcountry={Switzerland}}
\hypersetup{pdfcontactemail={nbeisert@itp.phys.ethz.ch}}
\hypersetup{pdfcontacturl={http://people.phys.ethz.ch/\xmptilde nbeisert/}}

\newcommand{\secref}[1]{\hyperref[#1]{section \ref*{#1}}}

\parskip1ex
\parindent0pt
\let\olditemize\itemize
\def\itemize{\olditemize\parskip0pt}

\begin{document}

\title{The \textsf{childdoc} Package}
\hypersetup{pdftitle={The childdoc Package}}
\author{Niklas Beisert\\[2ex]
  Institut f\"ur Theoretische Physik\\
  Eidgen\"ossische Technische Hochschule Z\"urich\\
  Wolfgang-Pauli-Strasse 27, 8093 Z\"urich, Switzerland\\[1ex]
  \href{mailto:nbeisert@itp.phys.ethz.ch}
  {\texttt{nbeisert@itp.phys.ethz.ch}}}
\hypersetup{pdfauthor={Niklas Beisert}}
\hypersetup{pdfsubject={Manual for the LaTeX2e Package childdoc}}
\date{30 December 2018, \textsf{v2.0}}
\maketitle

\begin{abstract}\noindent
\textsf{childdoc} is a \LaTeXe{} package
that enables the direct compilation
of document sections included by |\include|
to individual files.
\end{abstract}

\begingroup
\parskip0ex
\tableofcontents
\endgroup

%%%%%%%%%%%%%%%%%%%%%%%%%%%%%%%%%%%%%%%%%%%%%%%%%%%%%%%%%%%%%%%%%%%%%%%%%%%%%%%%
%%%%%%%%%%%%%%%%%%%%%%%%%%%%%%%%%%%%%%%%%%%%%%%%%%%%%%%%%%%%%%%%%%%%%%%%%%%%%%%%
\section{Introduction}

\LaTeX{} provides a mechanism to structure a large document (such as a book)
into a main file and several child files (containing the chapters)
using the |\include| command.
This mechanism is beneficial for documents
which span hundreds of pages in order to
make the source file(s) more manageable.
Moreover, compilation can be restricted to
selected child files by means of the |\includeonly| command.
The latter feature can be used to reduce the compilation time while editing
(this was significantly more useful in the earlier days of \LaTeX{})
or to generate a smaller document which is easier to navigate.
Another application of |\includeonly| is to generate
documents consisting of selected parts of the complete document.

However, there are a few drawbacks of the plain |\include| mechanism:
\begin{itemize}
\item
The child files cannot be compiled on their own,
they can only be compiled via the main file.
A naive editing environment
(such as a text editor with an option
to have the current file processed by \LaTeX)
may require one to switch to the main file before compiling;
attempting to compile the child file produces errors.
\item
The main file must be modified (each time)
to adjust the |\includeonly| command
to the present needs. This easily leaves the main file in a messy state.
\item
The generated document will always carry the filename
of the main document. This is inconvenient if
several child files are to be compiled and
to be kept for distribution.
\end{itemize}

The present package provides a simple interface
to make child files individually compilable by \LaTeX{}.
Compiling a child file then has the same effect as compiling
the main file with an |\includeonly| command
to select the appropriate child.
Moreover the generated document will carry the name of the child
rather than the main file.
This resolves all three above issues.

This feature is meant to make the editing of books,
thesis documents and lecture notes somewhat more convenient.
However, the package can also be used efficiently for
composing a series of documents (such as exercise sheets)
which are typically distributed individually.
It then assists the author in generating the individual documents
(potentially in different versions)
as well as a document containing the collected series.
Another application is in developing style files
or other kinds of included material
where compilation of the style file could redirect
to a sample or test file.

%%%%%%%%%%%%%%%%%%%%%%%%%%%%%%%%%%%%%%%%%%%%%%%%%%%%%%%%%%%%%%%%%%%%%%%%%%%%%%%%
%%%%%%%%%%%%%%%%%%%%%%%%%%%%%%%%%%%%%%%%%%%%%%%%%%%%%%%%%%%%%%%%%%%%%%%%%%%%%%%%
\section{Usage}

First of all, the package \textsf{childdoc} is \emph{not} a standard
\LaTeXe{} |.sty| style file! Therefore it needs to be invoked in
a non-standard way.

%%%%%%%%%%%%%%%%%%%%%%%%%%%%%%%%%%%%%%%%%%%%%%%%%%%%%%%%%%%%%%%%%%%%%%%%%%%%%%%%
\subsection{Included Files}
\label{sec:include}

%%%%%%%%%%%%%%%%%%%%%%%%%%%%%%%%%%%%%%%%
\DescribeMacro{\childdocmain}
To use the package, add the commands
\begin{center}
\begin{tabular}{l}
|\input{childdoc.def}|\\
|\childdocmain{}|\\
\end{tabular}
\end{center}
at the very top of the main \LaTeX{} file,
in particular \emph{before} the |\documentclass| statement!
The argument of |\childdocmain| should be left empty
(but it must be present).

%%%%%%%%%%%%%%%%%%%%%%%%%%%%%%%%%%%%%%%%
\DescribeMacro{\childdocof}
Furthermore, add the commands
\begin{center}
\begin{tabular}{l}
|\input{childdoc.def}|\\
|\childdocof{|\textit{main}|}|\\
\end{tabular}
\end{center}
at the top of every child file \textit{child}
which is included by |\include{|\textit{child}|}|
from within the main file
(or at least for those files to be compiled individually).
The argument \textit{main} must be the filename of the main file.

There are a couple of
considerations in setting up the main and child documents:

%%%%%%%%%%%%%%%%%%%%%%%%%%%%%%%%%%%%%%%%
\paragraph{Restrictions.}

Please note the following restrictions:
\begin{itemize}
\item
|\childdocmain| must be called with one argument \textit{main}
to ensure compatibility with earlier version of the package.
It must either be empty (|\childdocmain{}|)
or precisely match the filename of the main file in which it is specified.
See \secref{sec:detection} for further information.
\item
The filename \textit{main} must be specified without the |.tex| extension.
\item
The filename \textit{main} is case sensitive
(even in case-insensitive file systems)
due to internal string comparison.
\item
The argument \textit{main} should be fully expanded, it cannot be a macro.
\item
Subdirectories and special characters should be avoided in filenames.
\item
The command |\childdocmain{|\textit{main}|}| must be followed by a whitespace.
It should not be followed immediately by another command
or by a comment mark `|%|'.
This is because the \TeX{} parser reads the token immediately following
the argument of |\childdocmain| and puts it
at the beginning of every child section;
however, a white\-space is ignored.
\end{itemize}

%%%%%%%%%%%%%%%%%%%%%%%%%%%%%%%%%%%%%%%%
\paragraph{Content of Main File.}

It is advisable to place all content in the child files included by |\include|.
Any output contained in the main file will appear in all child documents
unless suppressed manually;
it cannot be suppressed automatically by the |\includeonly| directive
and thus should normally be avoided.
A method to include some content in the main file
by means of conditional processing is described in \secref{sec:conditional}.

%%%%%%%%%%%%%%%%%%%%%%%%%%%%%%%%%%%%%%%%
\paragraph{Page Numbering.}

When only a part of the document is compiled,
the appropriate numbering of pages
(as well as other status parameters)
is determined from the |.aux| files.
The latter contain information from previous passes.
However this information needs to propagate through
all intermediate child documents.
Therefore the page numbering in child documents may well
be inconsistent until the complete document is compiled at least once.

A useful (if unconventional) way to always ensure a consistent
page numbering is to restart the numbering in each child document
and denote the pages by `\textit{child}|.|\textit{page}'
where \textit{child} represents the chapter/section number of the child file.
This can be achieved by the command
|\numberwithin{page}{|\textit{child}|}|
of the \textsf{amsmath} package
where \textit{child} can be |chapter| or |section|
depending on the chosen structuring.
Alternatively, one can modify the macro |\thepage| appropriately
and reset the counter |page| at the start of each child file.

%%%%%%%%%%%%%%%%%%%%%%%%%%%%%%%%%%%%%%%%%%%%%%%%%%%%%%%%%%%%%%%%%%%%%%%%%%%%%%%%
\subsection{Conditional Processing}
\label{sec:conditional}

The package provides a mechanism to compile different versions
of a document. To customise the versions further some conditional processing
can come in handy to distinguish which version is being compiled.
The package provides two macros to describe the compilation context:

%%%%%%%%%%%%%%%%%%%%%%%%%%%%%%%%%%%%%%%%
\DescribeMacro{\ifchilddoc}
The conditional |\ifchilddoc| distinguishes between the compilation of
child documents and the main document:
%
\begin{center}
|\ifchilddoc |\textit{child-code}| |[|\||else |\textit{main-code}]| \||fi|
\end{center}

%%%%%%%%%%%%%%%%%%%%%%%%%%%%%%%%%%%%%%%%
\DescribeMacro{\childdocname}
\DescribeMacro{\childdocjob}
The macro |\childdocname| contains the filename (without extension)
of the main or child file being processed.
Note that |\childdocjob| will always contain the name of the main file.

%%%%%%%%%%%%%%%%%%%%%%%%%%%%%%%%%%%%%%%%
\paragraph{Title Page.}

Conditional processing can be used to include a title or banner page
in the main document when proper precautions are taken.
Importantly, the code in the main file should ensure that the page counter
(as well as other status parameters which are stored in the |.aux| files)
takes the same value after the conditional processing.
Otherwise the page numbers may take divergent values
depending on which part is compiled.

For example, a title page could be declared by:
%
\begin{center}
\begin{tabular}{l}
|\ifchilddoc\||else|\\
|\addtocounter{page}{-1}|\\
\textit{code for title page}\\
|\newpage|\\
|\||fi|
\end{tabular}
\end{center}
%
A banner page for the child documents can be generated by:
%
\begin{center}
\begin{tabular}{l}
|\ifchilddoc|\\
|\addtocounter{page}{-1}|\\
\textit{code for banner page}\\
|\newpage|\\
|\||fi|
\end{tabular}
\end{center}
%
Here one could write a message such as:
\begin{center}
|This is the part \childdocname{} of \childdocjob{}.|
\end{center}

%%%%%%%%%%%%%%%%%%%%%%%%%%%%%%%%%%%%%%%%%%%%%%%%%%%%%%%%%%%%%%%%%%%%%%%%%%%%%%%%
\subsection{Flags}
\label{sec:flags}

The package makes it easy to generate different versions
of the main or child documents.
To this end compilation flags can be defined
and assigned different default values.
They will be particularly useful in conjunction
with the forwarding mechanism described in \secref{sec:forward}.

For example, it may be useful to have a flag |\version|
which can be set to |draft| or |final|.
The document source will contain some conditional code
depending on the value of |\version|.
Suppose further, the flag should default to |final| for the main file
and to |draft| for child files
which is a natural assignment for editing the document.
This is achieved by placing the following code
in the preamble of the main document
(below the |\childdocmain| directive):
%
\begin{center}
\begin{tabular}{l}
|\ifchilddoc|\\
|\providecommand{\version}{draft}|\\
|\||else|\\
|\providecommand{\version}{final}|\\
|\||fi|
\end{tabular}
\end{center}
%
The definition by |\providecommand| makes sure
that previous definitions are not overwritten.
Further statements |\providecommand{\version}{...}|
can thus be added before the above code to override it.

For the main file, one might add a line
(between |\childdocmain| and the above block)
%
\begin{center}
|%\ifchilddoc\||else\providecommand{\version}{draft}\||fi|
\end{center}
%
which can be uncommented to produce a draft version.
Likewise one can add a line to the very top of a child file
(above the |\childdocof{|\textit{main}|}| directive)
%
\begin{center}
|%\providecommand{\version}{final}|
\end{center}
%
which can be uncommented to produce the final version of this child document.

%%%%%%%%%%%%%%%%%%%%%%%%%%%%%%%%%%%%%%%%%%%%%%%%%%%%%%%%%%%%%%%%%%%%%%%%%%%%%%%%
\subsection{Forwarding}
\label{sec:forward}

Different versions of the main or child documents
using compilation flags as described in \secref{sec:flags}
can be (permanently) stored in different files
for convenient compilation, viewing and distribution.
To this end, the package defines a command
to pass on compilation to a different file:

%%%%%%%%%%%%%%%%%%%%%%%%%%%%%%%%%%%%%%%%
\DescribeMacro{\childdocforward}
The command |\childdocforward| redirects processing to
another source file:
%
\begin{center}
\begin{tabular}{l}
|\input{childdoc.def}|\\
|\childdocforward[|\textit{main}|]{|\textit{dest}|}|\\
\end{tabular}
\end{center}
%
The argument \textit{dest} is the destination file
(without extension).
It should be the main file or one of the child files.
Note that further \textsf{childdoc} directives
such as |\childdocof| and |\childdocforward|
in the indicated file will be processed in this form.
The optional argument \textit{main}
passes on directly to the main file \textit{main}
while pretending to compile the child \textit{dest}.
This form behaves as if \textit{dest}
issues |\childdocof{|\textit{main}|}| right away,
and no further \textsf{childdoc} directives will be processed.

%%%%%%%%%%%%%%%%%%%%%%%%%%%%%%%%%%%%%%%%
\DescribeMacro{\...prefix}
In the alternative form |\childdocforwardprefix|,
%
\begin{center}
\begin{tabular}{l}
|\input{childdoc.def}|\\
|\childdocforwardprefix[|\textit{main}|]{|\textit{prefix}|}{|\textit{dest}|}|
\end{tabular}
\end{center}
%
the destination file is determined by a pattern
depending on the current file:
To make this work, the current file must be called
`{\textit{prefix}\hspace{0.2em}\textit{suffix}}'
with \textit{prefix} matching precisely the argument.
Processing is then passed on to the file
`{\textit{dest}\hspace{0.2em}\textit{suffix}}'.
Surely, the same effect is achieved by
directly specifying the
argument `{\textit{dest}\hspace{0.2em}\textit{suffix}}'
in the first form.
However, that requires to set up a different file
for each child. With the alternative form of the command
all these files can have exactly the same content
which simplifies setting them up and maintaining them.

For example, the following file |draft.tex|
with a compilation flag |\version| as described in \secref{sec:flags}
compiles the main document as a draft:
%
\begin{center}
\begin{tabular}{l}
|\def\version{draft}|\\
|\input{childdoc.def}|\\
|\childdocforward{|\textit{main}|}|
\end{tabular}
\end{center}
%
Likewise, the following files |final|\textit{nn}|.tex|
compile the final version of the child document
|child|\textit{nn}|.tex|:
%
\begin{center}
\begin{tabular}{l}
|\def\version{final}|\\
|\input{childdoc.def}|\\
|\childdocforwardprefix{final}{child}|
\end{tabular}
\end{center}
%

Note that when several versions of a main file and/or of each child file
are to be generated, it may be convenient to set up a |Makefile| or
shell script to automatise the process.

%%%%%%%%%%%%%%%%%%%%%%%%%%%%%%%%%%%%%%%%%%%%%%%%%%%%%%%%%%%%%%%%%%%%%%%%%%%%%%%%
\subsection{Command Line Processing}
\label{sec:commandline}

The effect of redirection files can also be achieved by invoking
the \LaTeX{} compiler with a more elaborate command line.
Most conveniently this should be done as part
of a shell script or a |Makefile|.

When using \textsf{childdoc} in the main file, the following
command lines effectively perform a redirection
(note that depending on the shell being used,
backslashes may have to be doubled: `|\|' $\to$ `|\\|'):
%
\begin{center}
|... -jobname "|\textit{target}|" |\\|"|[\textit{flags}]%
|\input{childdoc.def}\childdocforward[|\textit{main}|]{|\textit{dest}|}"|
\end{center}
%
Here \textit{target} is the name of the output file,
\textit{main} is the name of the main file
and \textit{dest} is the name of the main or child file to be processed
(all filenames without extensions).
The optional argument \textit{main} can be omitted
if \textit{main} matches \textit{dest}.
Optionally, compilation \textit{flags} can be defined via |\def| commands.
This command line makes the \TeX{} engine believe
it is compiling the file \textit{target}
whose content is specified as the latter parameter.
The provided code then forwards the processing to
\textit{main} or \textit{dest} as described in \secref{sec:forward}.

%%%%%%%%%%%%%%%%%%%%%%%%%%%%%%%%%%%%%%%%%%%%%%%%%%%%%%%%%%%%%%%%%%%%%%%%%%%%%%%%
\subsection{Include by Input}
\label{sec:input}

Including child documents by |\include| has some restrictions by design.
Most notably, the content of a child document always occupies
its own set of pages; pages cannot be shared between child documents.
Usually, this behaviour makes perfect sense
because each child document contain an essential part of the document.
However, in some situations it may be desirable to compose
a document from a collection of parts
without having mandatory page breaks between then.
For this case, the package
provides a mechanism to include parts
by |\input| which can also be processed individually.
However, by construction this mechanism
requires manual handling of the content to be output.

%%%%%%%%%%%%%%%%%%%%%%%%%%%%%%%%%%%%%%%%
\DescribeMacro{\ifchilddocmanual}
The main file should be prepared as usual, see \secref{sec:include}.
However, the document body must make a distinction
between processing of an individual part and of the main document, e.g.:
%
\begin{center}
\begin{tabular}{l}
|\ifchilddocmanual|\\
|\input{\childdocname}|\\
|\||else|\\
\textit{document body with }|\input{|\textit{part}|}|\\
|\||fi|
\end{tabular}
\end{center}
%
The conditional |\ifchilddocmanual| is true whenever
a part to be included by |\input| is being compiled,
and the name of the part is stored in |\childdocname|.

%%%%%%%%%%%%%%%%%%%%%%%%%%%%%%%%%%%%%%%%
\DescribeMacro{\childdocby}
Each part to be included by |\input| should start with:
%
\begin{center}
\begin{tabular}{l}
|\input{childdoc.def}|\\
|\childdocby{|\textit{main}|}|\\
\end{tabular}
\end{center}
%
The directive |\childdocby| is similar to |\childdocof|
described in \secref{sec:include},
but the subsequent selection of content must be done manually.
To that end, both |\ifchilddoc| and |\ifchilddocmanual|
will be true upon processing of a part,
and the name of the part is stored in |\childdocname|.
Note that |\jobname| will be set to the filename of the current part
so that each part receives an individual |.aux| file
that does not interfere with the |.aux| file(s) of the main document.
This behaviour can be altered by the alternative form
|\childdocby[*]{|\textit{main}|}| (with a non-empty optional argument)
which uses the |.aux| file of the main document
by setting |\jobname| to \textit{main}.

%%%%%%%%%%%%%%%%%%%%%%%%%%%%%%%%%%%%%%%%%%%%%%%%%%%%%%%%%%%%%%%%%%%%%%%%%%%%%%%%
\subsection{Driver Development}
\label{sec:driver}

The \textsf{childdoc} mechanism can also be use for the development
of definition files such as \LaTeX{} styles or classes.
This case differs from the above setup with multiple parts
included by |\include| in that no |\includeonly| should be invoked.
This can be achieved by starting the include file
(before |\ProvidesPackage|) with:
%
\begin{center}
\begin{tabular}{l}
|\input{childdoc.def}|\\
|\childdocforward{|\textit{main}|}|\\
\end{tabular}
\end{center}
%
or alternatively with:
%
\begin{center}
\begin{tabular}{l}
|\input{childdoc.def}|\\
|\childdocby{|\textit{main}|}|\\
\end{tabular}
\end{center}
%
Both forms have slightly different effects as described above.
The main file is prepared as usual, see \secref{sec:include}.

%%%%%%%%%%%%%%%%%%%%%%%%%%%%%%%%%%%%%%%%%%%%%%%%%%%%%%%%%%%%%%%%%%%%%%%%%%%%%%%%
\subsection{Legacy Detection}
\label{sec:detection}

The directive |\childdocmain| in the main file can detect
whether the complete document or merely a child is to be compiled
even without using the directive |\childdocof|.
This method is deprecated because it is less robust
and there is no compelling reason to use it;
it is merely provided for backward compatibility
and it may be removed in future versions.

If the detection mechanism is to be used,
it is mandatory to correctly specify
the filename of the main file as the argument of |\childdocmain|:
%
\begin{center}
\begin{tabular}{l}
|\input{childdoc.def}|\\
|\childdocmain{|\textit{main}|}|\\
\end{tabular}
\end{center}
%
If |\jobname| does not match the argument \textit{main} of |\childdocmain|,
it is assumed that |\jobname| points to the child file to be compiled.
When using |\childdocmain| with the main file specified as argument,
it suffices to start a child file
with just |\input{|\textit{main}|}|
without loading of the package and using |\childdocof|.
If instead all processing is done
with the appropriate \textsf{childdoc} directives,
the argument of \textit{main} of |\childdocmain| can be empty.

An alternative version of the command line processing described
in \secref{sec:commandline} using the detection mechanism reads:
%
\begin{center}
|... -jobname "|\textit{target}|" "|[\textit{flags}]%
[|\def\jobname{|\textit{dest}|}|]|\input{|\textit{main}|}"|
\end{center}

%%%%%%%%%%%%%%%%%%%%%%%%%%%%%%%%%%%%%%%%%%%%%%%%%%%%%%%%%%%%%%%%%%%%%%%%%%%%%%%%
\subsection{Manual Code}
\label{sec:manual}

In case one cannot be certain whether the definitions file |childdoc.def|
is installed on the target \TeX{} distribution
and one prefers not to ship it,
it is conceivable to paste a few relevant commands into the sources.

To that end, drop all statements |\input{childdoc.def}|
and perform the replacements as outlined below.
Instead of |\childdocmain{|\textit{main}|}| add the following code
to the top of the main file:
%
\begin{center}
\begin{tabular}{l}
|\||ifdefined\childdocname\endinput\||fi\newif\ifchilddoc|\\
|\edef\childdocname{\scantokens\expandafter{\jobname\noexpand}}|\\
|\def\childdocmain{|\textit{main}|}\||ifx\childdocmain\childdocname\||else|\\
|\childdoctrue\includeonly{\childdocname}\let\jobname\childdocmain\||fi|\\
\end{tabular}
\end{center}
%
Instead of |\childdocof{|\textit{main}|}| just include the main file
at the top of each child file:
%
\begin{center}
|\input{|\textit{main}|}|
\end{center}
%
A simple redirection |\childdocforward{|\textit{dest}|}| is achieved by:
%
\begin{center}
|\def\jobname{|\textit{dest}|}\input{\jobname}|
\end{center}
%
The redirection with prefix
|\childdocforwardprefix[|\textit{prefix}|]{|\textit{dest}|}|
is accomplished by:
%
\begin{center}
\begin{tabular}{l}
|{\edef\jobname{\scantokens\expandafter{\jobname\noexpand}}|\\
|\def\redirectjob |\textit{prefix}|#1~~~{\gdef\jobname{|\textit{dest}|#1}}|\\
|\expandafter\redirectjob\jobname~~~}\input{\jobname}|
\end{tabular}
\end{center}

In an alternative approach,
child documents can be compiled by a specific command line
without additional code or specific definitions:
%
\begin{center}
|... -jobname "|\textit{target}|" "|[\textit{flags}]%
|\includeonly{|\textit{dest}|}\input{|\textit{main}|}"|
\end{center}
%

%%%%%%%%%%%%%%%%%%%%%%%%%%%%%%%%%%%%%%%%%%%%%%%%%%%%%%%%%%%%%%%%%%%%%%%%%%%%%%%%
%%%%%%%%%%%%%%%%%%%%%%%%%%%%%%%%%%%%%%%%%%%%%%%%%%%%%%%%%%%%%%%%%%%%%%%%%%%%%%%%
\section{Information}

%%%%%%%%%%%%%%%%%%%%%%%%%%%%%%%%%%%%%%%%%%%%%%%%%%%%%%%%%%%%%%%%%%%%%%%%%%%%%%%%
\subsection{Copyright}

Copyright \copyright{} 2017--2018 Niklas Beisert

This work may be distributed and/or modified under the
conditions of the \LaTeX{} Project Public License, either version 1.3
of this license or (at your option) any later version.
The latest version of this license is in
  \url{http://www.latex-project.org/lppl.txt}
and version 1.3 or later is part of all distributions of \LaTeX{}
version 2005/12/01 or later.

This work has the LPPL maintenance status `maintained'.

The Current Maintainer of this work is Niklas Beisert.

This work consists of the files |README.txt|, |childdoc.ins| and |childdoc.dtx|
as well as the derived files |childdoc.def|, |cdocsamp.tex|
with |cdocsch1.tex|, |cdocsch2.tex|, |cdocspt3.tex|, |cdocspt4.tex|,
|cdocsdrf.tex|, |cdocsfn1.tex|, |cdocsfn2.tex|
as well as |childdoc.pdf|.

%%%%%%%%%%%%%%%%%%%%%%%%%%%%%%%%%%%%%%%%%%%%%%%%%%%%%%%%%%%%%%%%%%%%%%%%%%%%%%%%
\subsection{Files and Installation}

The package consists of the files:
%
\begin{center}
\begin{tabular}{ll}
    |README.txt|   & readme file \\
    |childdoc.ins| & installation file \\
    |childdoc.dtx| & source file \\
    |childdoc.def| & definition file \\
    |cdocsamp.tex| & sample main file \\
    |cdocsch1.tex| & sample include file \\
    |cdocsch2.tex| & sample include file \\
    |cdocspt3.tex| & sample part file \\
    |cdocspt4.tex| & sample part file \\
    |cdocsdrf.tex| & sample redirection file \\
    |cdocsfn1.tex| & sample redirection file \\
    |cdocsfn2.tex| & sample redirection file \\
    |childdoc.pdf| & manual
\end{tabular}
\end{center}
%
The distribution consists of the files
|README.txt|, |childdoc.ins| and |childdoc.dtx|.
%
\begin{itemize}
\item
Run (pdf)\LaTeX{} on |childdoc.dtx|
to compile the manual |childdoc.pdf| (this file).
\item
Run \LaTeX{} on |childdoc.ins| to create the definitions file |childdoc.def|
and the sample |cdocsamp.tex| with include files
|cdocsch1.tex|, |cdocsch2.tex|, |cdocspt3.tex|, |cdocspt4.tex|,
|cdocsdrf.tex|, |cdocsfn1.tex|, |cdocsfn2.tex|.
Then copy the file |childdoc.def| to an appropriate directory of your \LaTeX{}
distribution, e.g.\ \textit{texmf-root}|/tex/latex/childdoc|.
\end{itemize}

%%%%%%%%%%%%%%%%%%%%%%%%%%%%%%%%%%%%%%%%%%%%%%%%%%%%%%%%%%%%%%%%%%%%%%%%%%%%%%%%
\subsection{Related CTAN Packages}

There are several other packages which offer a similar functionality:
%
\begin{itemize}
\item
The packages
\href{http://ctan.org/pkg/docmute}{\textsf{docmute}},
\href{http://ctan.org/pkg/includex}{\textsf{includex}} and
\href{http://ctan.org/pkg/standalone}{\textsf{standalone}}
provide commands to include only the document body of
a child file thus allowing both files to be compiled individually.
\item
The packages \href{http://ctan.org/pkg/subdocs}{\textsf{subdocs}}
and \href{http://ctan.org/pkg/subfiles}{\textsf{subfiles}}
provide structures in which the main and child documents can be
encapsulated and allowing them to be compiled individually.
The inclusion mechanism is different from the conventional |\include|.
\item
The package \href{http://ctan.org/pkg/combine}{\textsf{combine}}
is an elaborate solution to combine several documents into one.
\end{itemize}
%
See also the CTAN topic \href{http://ctan.org/topic/subdocs}{\textsf{subdocs}}
for further related packages.
The present package differs from the above solutions in that
a document structure constructed with the conventional |\include| mechanism
just needs two extra commands at the top of every file
such that all constituent files can be compiled individually.

%%%%%%%%%%%%%%%%%%%%%%%%%%%%%%%%%%%%%%%%%%%%%%%%%%%%%%%%%%%%%%%%%%%%%%%%%%%%%%%%
%\subsection{Feature Suggestions}
%
%The following is a list of features which may be useful for future
%versions of this package:
%%
%\begin{itemize}
%\item
%\ldots
%\end{itemize}

%%%%%%%%%%%%%%%%%%%%%%%%%%%%%%%%%%%%%%%%%%%%%%%%%%%%%%%%%%%%%%%%%%%%%%%%%%%%%%%%
\subsection{Revision History}

%%%%%%%%%%%%%%%%%%%%%%%%%%%%%%%%%%%%%%%%
\paragraph{v2.0:} 2018/12/30

\begin{itemize}
\item
immediate forward processing
\item
added |\childdocby| mechanism
\item
manual restructured
\end{itemize}

%%%%%%%%%%%%%%%%%%%%%%%%%%%%%%%%%%%%%%%%
\paragraph{v1.6:} 2018/01/17

\begin{itemize}
\item
application for development of include files
\item
corrections to manual
\end{itemize}

%%%%%%%%%%%%%%%%%%%%%%%%%%%%%%%%%%%%%%%%
\paragraph{v1.5:} 2017/05/21

\begin{itemize}
\item
more complete structuring introduced
\item
|\childdocof| introduced
\item
|\childdoc| renamed to |\childdocmain|
\item
|\childredirect| renamed to |\childdocforward| and |\childdocforwardprefix|
and functionality expanded
\end{itemize}

%%%%%%%%%%%%%%%%%%%%%%%%%%%%%%%%%%%%%%%%
\paragraph{v1.0:} 2017/04/27

\begin{itemize}
\item
manual and install package
\item
first version published on CTAN
\end{itemize}

%%%%%%%%%%%%%%%%%%%%%%%%%%%%%%%%%%%%%%%%
\paragraph{v0.6:} 2017/04/26

\begin{itemize}
\item
redirection mechanism added
\end{itemize}

%%%%%%%%%%%%%%%%%%%%%%%%%%%%%%%%%%%%%%%%
\paragraph{v0.5:} 2017/04/26

\begin{itemize}
\item
functionality in definition file
\end{itemize}


%%%%%%%%%%%%%%%%%%%%%%%%%%%%%%%%%%%%%%%%%%%%%%%%%%%%%%%%%%%%%%%%%%%%%%%%%%%%%%%%
%%%%%%%%%%%%%%%%%%%%%%%%%%%%%%%%%%%%%%%%%%%%%%%%%%%%%%%%%%%%%%%%%%%%%%%%%%%%%%%%
%%%%%%%%%%%%%%%%%%%%%%%%%%%%%%%%%%%%%%%%%%%%%%%%%%%%%%%%%%%%%%%%%%%%%%%%%%%%%%%%
\appendix

\settowidth\MacroIndent{\rmfamily\scriptsize 000\ }

 \DocInput{childdoc.dtx}

\end{document}
%</driver>
% \fi
%
% %%%%%%%%%%%%%%%%%%%%%%%%%%%%%%%%%%%%%%%%%%%%%%%%%%%%%%%%%%%%%%%%%%%%%%%%%%%%%%
% %%%%%%%%%%%%%%%%%%%%%%%%%%%%%%%%%%%%%%%%%%%%%%%%%%%%%%%%%%%%%%%%%%%%%%%%%%%%%%
% \section{Sample}
%\iffalse
%<*samplemain>
%\fi
%
% The following presents a sample document
% with two chapters, two parts, a title page,
% a compile flag as well as three forwarding files to set the flag.
% It consists of eight |.tex| files:
% \begin{center}
% \begin{tabular}{ll}
% |cdocsamp.tex|&main file\\
% |cdocsch1.tex|&include file for chapter 1\\
% |cdocsch2.tex|&include file for chapter 2\\
% |cdocspt3.tex|&include file for part 3\\
% |cdocspt4.tex|&include file for part 4\\
% |cdocsdrf.tex|&forwarding file for main file in draft mode\\
% |cdocsfi1.tex|&forwarding file for final version of chapter 1\\
% |cdocsfi2.tex|&forwarding file for final version of chapter 2\\
% \end{tabular}
% \end{center}
% Each of the eight files can be compiled directly by the \LaTeX{} compiler.
%
% %%%%%%%%%%%%%%%%%%%%%%%%%%%%%%%%%%%%%%
% \paragraph{Main File.}
%
% The main file is called |cdocsamp.tex|.
%
% Load the \textsf{childdoc} definitions and
% declare the filename for the main document:
%    \begin{macrocode}
\input{childdoc.def}
\childdocmain{}
%    \end{macrocode}

% Optional override for |\version| flag:
%    \begin{macrocode}
%%\ifchilddoc\else\providecommand{\version}{draft}\fi
%    \end{macrocode}

% Define the default values for the |\version| flag
% (|final| for the main file and |draft| for childs):
%    \begin{macrocode}
\ifchilddoc
\providecommand{\version}{draft}
\else
\providecommand{\version}{final}
\fi
%    \end{macrocode}

% Load the standard document class:
%    \begin{macrocode}
\documentclass[12pt]{article}
%    \end{macrocode}

% Start the document body:
%    \begin{macrocode}
\begin{document}
%    \end{macrocode}

% Declare a title page.
% Print title, part of document being processed and version flag:
%    \begin{macrocode}
\addtocounter{page}{-1}
\begin{center}
{\LARGE\bfseries{}childdoc example\par}
\vspace{1cm}
\ifchilddoc
\ifchilddocmanual part\else chapter\fi:
`\childdocname' of `\childdocjob'\par
\else
main document: `\childdocjob'\par
\fi
version: \version\par
\end{center}
\newpage
%    \end{macrocode}

% Manually include selected file,
% otherwise process as usual:
%    \begin{macrocode}
\ifchilddocmanual
\section*{part `\childdocname'}
\input{\childdocname}
\else
%    \end{macrocode}

% Include the two chapters:
%    \begin{macrocode}
\include{cdocsch1}
\include{cdocsch2}
%    \end{macrocode}

% Include the two parts unless only chapters should be displayed:
%    \begin{macrocode}
\ifchilddoc\else
\section{part three}
\input{cdocspt3}
\section{part four}
\input{cdocspt4}
\fi
%    \end{macrocode}

% Process as usual until here:
%    \begin{macrocode}
\fi
%    \end{macrocode}

% End of document body:
%    \begin{macrocode}
\end{document}
%    \end{macrocode}
%\iffalse
%</samplemain>
%\fi
%
% %%%%%%%%%%%%%%%%%%%%%%%%%%%%%%%%%%%%%%
% \paragraph{Chapter Include Files.}
%
% The include files are called |cdocsch1.tex| and |cdocsch2.tex|.
%
%\iffalse
%<*samplechap1|samplechap2>
%\fi

% Optional override for |\version| flag:
%    \begin{macrocode}
%%\providecommand{\version}{final}
%    \end{macrocode}

% Include the main document:
%    \begin{macrocode}
\input{childdoc.def}
\childdocof{cdocsamp}
%    \end{macrocode}

%\iffalse
%</samplechap1|samplechap2>
%\fi
%
%\iffalse
%<*samplechap1>
%\fi
% Some text for chapter 1:
%    \begin{macrocode}
\section{one}
some text in chapter one
%    \end{macrocode}

%\iffalse
%</samplechap1>
%\fi
% Some text for chapter 2:
%\iffalse
%<*samplechap2>
%\fi
%    \begin{macrocode}
\section{two}
more text in chapter two
%    \end{macrocode}

%\iffalse
%</samplechap2>
%\fi
%
% %%%%%%%%%%%%%%%%%%%%%%%%%%%%%%%%%%%%%%
% \paragraph{Part Include Files.}
%
% The include files are called |cdocspt3.tex| and |cdocspt4.tex|.
%
%\iffalse
%<*samplepart3|samplepart4>
%\fi

% Optional override for |\version| flag:
%    \begin{macrocode}
%%\providecommand{\version}{final}
%    \end{macrocode}

% Include the main document:
%    \begin{macrocode}
\input{childdoc.def}
\childdocby{cdocsamp}
%    \end{macrocode}

%\iffalse
%</samplepart3|samplepart4>
%\fi
%
%\iffalse
%<*samplepart3>
%\fi
% Some text for part 3:
%    \begin{macrocode}
some text in part three
%    \end{macrocode}

%\iffalse
%</samplepart3>
%\fi
% Some text for part 4:
%\iffalse
%<*samplepart4>
%\fi
%    \begin{macrocode}
more text in part four
%    \end{macrocode}

%\iffalse
%</samplepart4>
%\fi
%
% %%%%%%%%%%%%%%%%%%%%%%%%%%%%%%%%%%%%%%
% \paragraph{Forwarding for a Complete Draft.}
%
% The following forwarding file |cdocsdrf.tex|
% compiles the main document in draft mode:
%\iffalse
%<*sampledraft>
%\fi
%    \begin{macrocode}
\def\version{draft}
\input{childdoc.def}
\childdocforward{cdocsamp}
%    \end{macrocode}

%\iffalse
%</sampledraft>
%\fi
%
% %%%%%%%%%%%%%%%%%%%%%%%%%%%%%%%%%%%%%%
% \paragraph{Forwarding for Final Version of the Chapters.}
%
% The following forwarding files |cdocsfn1.tex| and |cdocsfn2.tex|
% (with identical content)
% compile the final versions of the child documents
% |cdocsch1.tex| and |cdocsch2.tex|, respectively:
%\iffalse
%<*samplefinal>
%\fi
%    \begin{macrocode}
\def\version{final}
\input{childdoc.def}
\childdocforwardprefix[cdocsamp]{cdocsfn}{cdocsch}
%    \end{macrocode}

%\iffalse
%</samplefinal>
%\fi
%
% %%%%%%%%%%%%%%%%%%%%%%%%%%%%%%%%%%%%%%
% \paragraph{Command Line Processing.}
%
% The following three command lines generate the output files
% |cdocscld|, |cdocscl1| and |cdocscl2|
% which should be identical to
% |cdocsdrf|, |cdocsch1| and |cdocsfn2|, respectively:
% \begin{center}
% \begin{tabular}{l}
% |latex -jobname cdocscld \|\\
% |  "\def\version{draft}\input{childdoc.def}\childdocforward{cdocsamp}"|\\
% |latex -jobname cdocscl1 \|\\
% |  "\input{childdoc.def}\childdocforward[cdocsamp]{cdocsch1}"|\\
% |latex -jobname cdocscl2 \|\\
% |  "\def\version{final}\input{childdoc.def}\childdocforward{cdocsch2}"|
% \end{tabular}
% \end{center}
% Note that the trailing backslash on each first line
% merely continues the input to the second line
% (for convenient cut ant paste).
% Furthermore, the command |latex| can be replaced by any
% of its alternative versions such as |pdflatex|.
%
% %%%%%%%%%%%%%%%%%%%%%%%%%%%%%%%%%%%%%%%%%%%%%%%%%%%%%%%%%%%%%%%%%%%%%%%%%%%%%%
% %%%%%%%%%%%%%%%%%%%%%%%%%%%%%%%%%%%%%%%%%%%%%%%%%%%%%%%%%%%%%%%%%%%%%%%%%%%%%%
% \section{Implementation}
%\iffalse
%<*package>
%\fi
%
% This section describes the definitions file |childdoc.def|.

% The definitions cannot be loaded using |\usepackage| or |\RequirePackage|
% which has a mechanism to prevent loading a style file more than once.
% When loading the definitions by means of |\input|
% multiple instances have to be prevented manually:
%\iffalse
%This code needs to be before the `\ProvidesFile' directive
%which is defined at the beginning of this file.
%Therefore it is also placed there and commented out here.
%</package>
%<*discard>
%\fi
%    \begin{macrocode}
\ifdefined\childdocmain\endinput\fi
%    \end{macrocode}
%\iffalse
%</discard>
%<*package>
%\fi
%
% \macro{\ifchilddoc}
% \macro{\ifchilddocmanual}
% The conditional |\ifchilddoc| tells whether a
% child (true) or main (false) document is being compiled.
% The conditional |\ifchilddocmanual| tells whether
% the |\includeonly| mechanism is used (false) or
% the selection of child files must be performed manually (true).
% The definitions initialise to false:
%    \begin{macrocode}
\newif\ifchilddoc
\newif\ifchilddocmanual
%    \end{macrocode}

% \macro{\childdocname}
% \macro{\childdocjob}
% The macro |\childdocname| stores the name of the main document
% to be compiled. The macro |\childdocjob| stores the name of
% the document on which the \LaTeX{} compiler was originally invoked.
% The content of |\jobname| cannot be compared
% to filenames specified in the source due to different catcodes.
% The following code rescans |\jobname|, stores the result
% in |\childdocname| and saves a copy in |\childdocjob|:
%    \begin{macrocode}
\edef\childdocname{\scantokens\expandafter{\jobname\noexpand}}
\let\childdocjob\childdocname
%    \end{macrocode}

% \macro{\childdocdisable}
% The macro |\childdocdisable| prevents the main file
% from being processed more than once.
% At this stage, the main document command |\childdocmain|
% is assumed to be called once again where it should do nothing.
% Any subsequent call to it should prevent
% a secondary processing of the main document
% It overwrites the forwarding commands
% |\childdocof| and |\childdocforward|
% with empty macros to prevent further inclusions of the main document:
%    \begin{macrocode}
\newcommand{\childdocdisable}
{
  \renewcommand{\childdocmain}[1]{\renewcommand{\childdocmain}[1]{\endinput}}
  \renewcommand{\childdocof}[1]{}
  \renewcommand{\childdocby}[2][]{}
  \renewcommand{\childdocforward}[2][]{}
  \renewcommand{\childdocdisable}{}
}
%    \end{macrocode}

% \macro{\childdocmain}
% The macro |\childdocmain| is to be called at the top of the main file
% with nothing or the main filename (without extension) as argument.
% First, it breaks loops.
% If the argument is not empty and does not match |\childdocname|
% (which is set by the first inclusion of |childdoc.def|),
% |\ifchilddoc| is set to true, |\includeonly| is applied to the child file
% and |\jobname| is set to the main file
% (for proper handling of |.aux| files):
%    \begin{macrocode}
\newcommand{\childdocmain}[1]
{
  \childdocdisable\childdocmain{}
  \if?#1?\else
    \begingroup
      \def\childdoctmp{#1}
      \ifx\childdoctmp\childdocname
        \def\childdoctmp{}
      \else
        \def\childdoctmp
        {
          \childdoctrue
          \includeonly{\childdocname}
          \def\childdocjob{#1}
          \def\jobname{#1}
        }
      \fi
      \expandafter
    \endgroup
    \childdoctmp
  \fi
}
%    \end{macrocode}

% \macro{\childdocof}
% The command |\childdocof| redirects
% compilation to the main file |#1|.
%    \begin{macrocode}
\newcommand{\childdocof}[1]
{
  \childdocdisable
  \childdoctrue
  \includeonly{\childdocname}
  \def\jobname{#1}
  \def\childdocjob{#1}
  \input{#1}
}
%    \end{macrocode}

% \macro{\childdocby}
% The command |\childdocby| ....
%    \begin{macrocode}
\newcommand{\childdocby}[2][]
{
  \childdocdisable
  \childdoctrue
  \childdocmanualtrue
  \if?#1?\else
    \def\jobname{#2}
  \fi
  \def\childdocjob{#2}
  \input{#2}
  \endinput
}
%    \end{macrocode}

% \macro{\childdocforward}
% The command |\childdocforward| redirects
% compilation to the main file or
% (if the optional argument is given) a child file.
% Parameters are set as if the main file
% or a child file starting with |\childdocof| was compiled.
% Then compilation is handed over to the main file:
%    \begin{macrocode}
\newcommand{\childdocforward}[2][]
{
  \begingroup
    \if?#1?
      \def\childdoctmp
      {
        \def\childdocname{#2}
        \def\childdocjob{#2}
        \def\jobname{#2}
        \input{#2}
        \endinput
      }
    \else
      \def\childdoctmp
      {
        \childdocdisable
        \def\childdocname{#2}
        \childdoctrue
        \includeonly{#2}
        \def\childdocjob{#1}
        \def\jobname{#1}
        \input{#1}
        \endinput
      }
    \fi
    \expandafter
  \endgroup
  \childdoctmp
}
%    \end{macrocode}

% \macro{\childdocforwardprefix}
% The command |\childdocforwardprefix| redirects
% compilation to the main or a child file by means of a pattern.
% The prefix |#1| in the current filename is replaced by |#2|
% and the suffix of the current filename is kept
% (it is assumed that the filename does not contain the substring `|~~~|'
% which is used as a delimiter).
% Compilation is handed over to the new file by |\childdocforward|:
%    \begin{macrocode}
\newcommand{\childdocforwardprefix}[3][]
{
  \begingroup
    \def\childdocextract #2##1~~~{\def\childdoctmp{\childdocforward[#1]{#3##1}}}
    \expandafter\childdocextract\childdocname~~~
    \expandafter
  \endgroup
  \childdoctmp
}
%    \end{macrocode}

% \macro{\childdoc}
% The deprecated macro |\childdoc| is a legacy version of |\childdocmain|:
%    \begin{macrocode}
\newcommand{\childdoc}{\childdocmain}
%    \end{macrocode}

% \macro{\childdocredirect}
% The deprecated macro |\childdocredirect| is a legacy version
% of |\childdocforward| and |\childdocforwardprefix|:
%    \begin{macrocode}
\newcommand{\childdocredirect}[2][]
{
  \begingroup
    \if?#1?
      \def\childdoctmp{\childdocforward{#2}}
    \else
      \def\childdoctmp{\childdocforwardprefix{#1}{#2}}
    \fi
    \expandafter
  \endgroup
  \childdoctmp
}
%    \end{macrocode}

%\iffalse
%</package>
%\fi
%
\endinput
|\\
|\childdocforward{|\textit{main}|}|\\
\end{tabular}
\end{center}
%
or alternatively with:
%
\begin{center}
\begin{tabular}{l}
|% \iffalse
%
% childdoc.dtx Copyright (C) 2017-2018 Niklas Beisert
%
% This work may be distributed and/or modified under the
% conditions of the LaTeX Project Public License, either version 1.3
% of this license or (at your option) any later version.
% The latest version of this license is in
%   http://www.latex-project.org/lppl.txt
% and version 1.3 or later is part of all distributions of LaTeX
% version 2005/12/01 or later.
%
% This work has the LPPL maintenance status `maintained'.
%
% The Current Maintainer of this work is Niklas Beisert.
%
% This work consists of the files childdoc.dtx and childdoc.ins
% and the derived files childdoc.def and cdocsamp.tex with
% cdocsch1.tex, cdocsch2.tex, cdocsdrf.tex, cdocsfn1.tex, cdocsfn2.tex.
%
%<package>\ifdefined\childdocmain\endinput\fi
%<package>\ProvidesFile{childdoc.def}[2018/12/30 v2.0 child document driver]
%<samplemain>\ProvidesFile{cdocsamp.tex}[2018/12/30 v2.0 sample for childdoc]
%<*driver>
%\ProvidesFile{childdoc.drv}[2018/12/30 v2.0 childdoc reference manual file]
\PassOptionsToClass{10pt,a4paper}{article}
\documentclass{ltxdoc}

\usepackage[margin=35mm]{geometry}
\usepackage{hyperref}
\usepackage{hyperxmp}
\usepackage[usenames]{color}

\hypersetup{colorlinks=true}
\hypersetup{pdfstartview=FitH}
\hypersetup{pdfpagemode=UseNone}
\hypersetup{pdfsource={}}
\hypersetup{pdflang={en-UK}}
\hypersetup{pdfcopyright={Copyright 2017-2018 Niklas Beisert.
  This work may be distributed and/or modified under the
  conditions of the LaTeX Project Public License, either version 1.3
  of this license or (at your option) any later version.}}
\hypersetup{pdflicenseurl={http://www.latex-project.org/lppl.txt}}
\hypersetup{pdfcontactaddress={ETH Zurich, ITP, HIT K,
  Wolfgang-Pauli-Strasse 27}}
\hypersetup{pdfcontactpostcode={8093}}
\hypersetup{pdfcontactcity={Zurich}}
\hypersetup{pdfcontactcountry={Switzerland}}
\hypersetup{pdfcontactemail={nbeisert@itp.phys.ethz.ch}}
\hypersetup{pdfcontacturl={http://people.phys.ethz.ch/\xmptilde nbeisert/}}

\newcommand{\secref}[1]{\hyperref[#1]{section \ref*{#1}}}

\parskip1ex
\parindent0pt
\let\olditemize\itemize
\def\itemize{\olditemize\parskip0pt}

\begin{document}

\title{The \textsf{childdoc} Package}
\hypersetup{pdftitle={The childdoc Package}}
\author{Niklas Beisert\\[2ex]
  Institut f\"ur Theoretische Physik\\
  Eidgen\"ossische Technische Hochschule Z\"urich\\
  Wolfgang-Pauli-Strasse 27, 8093 Z\"urich, Switzerland\\[1ex]
  \href{mailto:nbeisert@itp.phys.ethz.ch}
  {\texttt{nbeisert@itp.phys.ethz.ch}}}
\hypersetup{pdfauthor={Niklas Beisert}}
\hypersetup{pdfsubject={Manual for the LaTeX2e Package childdoc}}
\date{30 December 2018, \textsf{v2.0}}
\maketitle

\begin{abstract}\noindent
\textsf{childdoc} is a \LaTeXe{} package
that enables the direct compilation
of document sections included by |\include|
to individual files.
\end{abstract}

\begingroup
\parskip0ex
\tableofcontents
\endgroup

%%%%%%%%%%%%%%%%%%%%%%%%%%%%%%%%%%%%%%%%%%%%%%%%%%%%%%%%%%%%%%%%%%%%%%%%%%%%%%%%
%%%%%%%%%%%%%%%%%%%%%%%%%%%%%%%%%%%%%%%%%%%%%%%%%%%%%%%%%%%%%%%%%%%%%%%%%%%%%%%%
\section{Introduction}

\LaTeX{} provides a mechanism to structure a large document (such as a book)
into a main file and several child files (containing the chapters)
using the |\include| command.
This mechanism is beneficial for documents
which span hundreds of pages in order to
make the source file(s) more manageable.
Moreover, compilation can be restricted to
selected child files by means of the |\includeonly| command.
The latter feature can be used to reduce the compilation time while editing
(this was significantly more useful in the earlier days of \LaTeX{})
or to generate a smaller document which is easier to navigate.
Another application of |\includeonly| is to generate
documents consisting of selected parts of the complete document.

However, there are a few drawbacks of the plain |\include| mechanism:
\begin{itemize}
\item
The child files cannot be compiled on their own,
they can only be compiled via the main file.
A naive editing environment
(such as a text editor with an option
to have the current file processed by \LaTeX)
may require one to switch to the main file before compiling;
attempting to compile the child file produces errors.
\item
The main file must be modified (each time)
to adjust the |\includeonly| command
to the present needs. This easily leaves the main file in a messy state.
\item
The generated document will always carry the filename
of the main document. This is inconvenient if
several child files are to be compiled and
to be kept for distribution.
\end{itemize}

The present package provides a simple interface
to make child files individually compilable by \LaTeX{}.
Compiling a child file then has the same effect as compiling
the main file with an |\includeonly| command
to select the appropriate child.
Moreover the generated document will carry the name of the child
rather than the main file.
This resolves all three above issues.

This feature is meant to make the editing of books,
thesis documents and lecture notes somewhat more convenient.
However, the package can also be used efficiently for
composing a series of documents (such as exercise sheets)
which are typically distributed individually.
It then assists the author in generating the individual documents
(potentially in different versions)
as well as a document containing the collected series.
Another application is in developing style files
or other kinds of included material
where compilation of the style file could redirect
to a sample or test file.

%%%%%%%%%%%%%%%%%%%%%%%%%%%%%%%%%%%%%%%%%%%%%%%%%%%%%%%%%%%%%%%%%%%%%%%%%%%%%%%%
%%%%%%%%%%%%%%%%%%%%%%%%%%%%%%%%%%%%%%%%%%%%%%%%%%%%%%%%%%%%%%%%%%%%%%%%%%%%%%%%
\section{Usage}

First of all, the package \textsf{childdoc} is \emph{not} a standard
\LaTeXe{} |.sty| style file! Therefore it needs to be invoked in
a non-standard way.

%%%%%%%%%%%%%%%%%%%%%%%%%%%%%%%%%%%%%%%%%%%%%%%%%%%%%%%%%%%%%%%%%%%%%%%%%%%%%%%%
\subsection{Included Files}
\label{sec:include}

%%%%%%%%%%%%%%%%%%%%%%%%%%%%%%%%%%%%%%%%
\DescribeMacro{\childdocmain}
To use the package, add the commands
\begin{center}
\begin{tabular}{l}
|\input{childdoc.def}|\\
|\childdocmain{}|\\
\end{tabular}
\end{center}
at the very top of the main \LaTeX{} file,
in particular \emph{before} the |\documentclass| statement!
The argument of |\childdocmain| should be left empty
(but it must be present).

%%%%%%%%%%%%%%%%%%%%%%%%%%%%%%%%%%%%%%%%
\DescribeMacro{\childdocof}
Furthermore, add the commands
\begin{center}
\begin{tabular}{l}
|\input{childdoc.def}|\\
|\childdocof{|\textit{main}|}|\\
\end{tabular}
\end{center}
at the top of every child file \textit{child}
which is included by |\include{|\textit{child}|}|
from within the main file
(or at least for those files to be compiled individually).
The argument \textit{main} must be the filename of the main file.

There are a couple of
considerations in setting up the main and child documents:

%%%%%%%%%%%%%%%%%%%%%%%%%%%%%%%%%%%%%%%%
\paragraph{Restrictions.}

Please note the following restrictions:
\begin{itemize}
\item
|\childdocmain| must be called with one argument \textit{main}
to ensure compatibility with earlier version of the package.
It must either be empty (|\childdocmain{}|)
or precisely match the filename of the main file in which it is specified.
See \secref{sec:detection} for further information.
\item
The filename \textit{main} must be specified without the |.tex| extension.
\item
The filename \textit{main} is case sensitive
(even in case-insensitive file systems)
due to internal string comparison.
\item
The argument \textit{main} should be fully expanded, it cannot be a macro.
\item
Subdirectories and special characters should be avoided in filenames.
\item
The command |\childdocmain{|\textit{main}|}| must be followed by a whitespace.
It should not be followed immediately by another command
or by a comment mark `|%|'.
This is because the \TeX{} parser reads the token immediately following
the argument of |\childdocmain| and puts it
at the beginning of every child section;
however, a white\-space is ignored.
\end{itemize}

%%%%%%%%%%%%%%%%%%%%%%%%%%%%%%%%%%%%%%%%
\paragraph{Content of Main File.}

It is advisable to place all content in the child files included by |\include|.
Any output contained in the main file will appear in all child documents
unless suppressed manually;
it cannot be suppressed automatically by the |\includeonly| directive
and thus should normally be avoided.
A method to include some content in the main file
by means of conditional processing is described in \secref{sec:conditional}.

%%%%%%%%%%%%%%%%%%%%%%%%%%%%%%%%%%%%%%%%
\paragraph{Page Numbering.}

When only a part of the document is compiled,
the appropriate numbering of pages
(as well as other status parameters)
is determined from the |.aux| files.
The latter contain information from previous passes.
However this information needs to propagate through
all intermediate child documents.
Therefore the page numbering in child documents may well
be inconsistent until the complete document is compiled at least once.

A useful (if unconventional) way to always ensure a consistent
page numbering is to restart the numbering in each child document
and denote the pages by `\textit{child}|.|\textit{page}'
where \textit{child} represents the chapter/section number of the child file.
This can be achieved by the command
|\numberwithin{page}{|\textit{child}|}|
of the \textsf{amsmath} package
where \textit{child} can be |chapter| or |section|
depending on the chosen structuring.
Alternatively, one can modify the macro |\thepage| appropriately
and reset the counter |page| at the start of each child file.

%%%%%%%%%%%%%%%%%%%%%%%%%%%%%%%%%%%%%%%%%%%%%%%%%%%%%%%%%%%%%%%%%%%%%%%%%%%%%%%%
\subsection{Conditional Processing}
\label{sec:conditional}

The package provides a mechanism to compile different versions
of a document. To customise the versions further some conditional processing
can come in handy to distinguish which version is being compiled.
The package provides two macros to describe the compilation context:

%%%%%%%%%%%%%%%%%%%%%%%%%%%%%%%%%%%%%%%%
\DescribeMacro{\ifchilddoc}
The conditional |\ifchilddoc| distinguishes between the compilation of
child documents and the main document:
%
\begin{center}
|\ifchilddoc |\textit{child-code}| |[|\||else |\textit{main-code}]| \||fi|
\end{center}

%%%%%%%%%%%%%%%%%%%%%%%%%%%%%%%%%%%%%%%%
\DescribeMacro{\childdocname}
\DescribeMacro{\childdocjob}
The macro |\childdocname| contains the filename (without extension)
of the main or child file being processed.
Note that |\childdocjob| will always contain the name of the main file.

%%%%%%%%%%%%%%%%%%%%%%%%%%%%%%%%%%%%%%%%
\paragraph{Title Page.}

Conditional processing can be used to include a title or banner page
in the main document when proper precautions are taken.
Importantly, the code in the main file should ensure that the page counter
(as well as other status parameters which are stored in the |.aux| files)
takes the same value after the conditional processing.
Otherwise the page numbers may take divergent values
depending on which part is compiled.

For example, a title page could be declared by:
%
\begin{center}
\begin{tabular}{l}
|\ifchilddoc\||else|\\
|\addtocounter{page}{-1}|\\
\textit{code for title page}\\
|\newpage|\\
|\||fi|
\end{tabular}
\end{center}
%
A banner page for the child documents can be generated by:
%
\begin{center}
\begin{tabular}{l}
|\ifchilddoc|\\
|\addtocounter{page}{-1}|\\
\textit{code for banner page}\\
|\newpage|\\
|\||fi|
\end{tabular}
\end{center}
%
Here one could write a message such as:
\begin{center}
|This is the part \childdocname{} of \childdocjob{}.|
\end{center}

%%%%%%%%%%%%%%%%%%%%%%%%%%%%%%%%%%%%%%%%%%%%%%%%%%%%%%%%%%%%%%%%%%%%%%%%%%%%%%%%
\subsection{Flags}
\label{sec:flags}

The package makes it easy to generate different versions
of the main or child documents.
To this end compilation flags can be defined
and assigned different default values.
They will be particularly useful in conjunction
with the forwarding mechanism described in \secref{sec:forward}.

For example, it may be useful to have a flag |\version|
which can be set to |draft| or |final|.
The document source will contain some conditional code
depending on the value of |\version|.
Suppose further, the flag should default to |final| for the main file
and to |draft| for child files
which is a natural assignment for editing the document.
This is achieved by placing the following code
in the preamble of the main document
(below the |\childdocmain| directive):
%
\begin{center}
\begin{tabular}{l}
|\ifchilddoc|\\
|\providecommand{\version}{draft}|\\
|\||else|\\
|\providecommand{\version}{final}|\\
|\||fi|
\end{tabular}
\end{center}
%
The definition by |\providecommand| makes sure
that previous definitions are not overwritten.
Further statements |\providecommand{\version}{...}|
can thus be added before the above code to override it.

For the main file, one might add a line
(between |\childdocmain| and the above block)
%
\begin{center}
|%\ifchilddoc\||else\providecommand{\version}{draft}\||fi|
\end{center}
%
which can be uncommented to produce a draft version.
Likewise one can add a line to the very top of a child file
(above the |\childdocof{|\textit{main}|}| directive)
%
\begin{center}
|%\providecommand{\version}{final}|
\end{center}
%
which can be uncommented to produce the final version of this child document.

%%%%%%%%%%%%%%%%%%%%%%%%%%%%%%%%%%%%%%%%%%%%%%%%%%%%%%%%%%%%%%%%%%%%%%%%%%%%%%%%
\subsection{Forwarding}
\label{sec:forward}

Different versions of the main or child documents
using compilation flags as described in \secref{sec:flags}
can be (permanently) stored in different files
for convenient compilation, viewing and distribution.
To this end, the package defines a command
to pass on compilation to a different file:

%%%%%%%%%%%%%%%%%%%%%%%%%%%%%%%%%%%%%%%%
\DescribeMacro{\childdocforward}
The command |\childdocforward| redirects processing to
another source file:
%
\begin{center}
\begin{tabular}{l}
|\input{childdoc.def}|\\
|\childdocforward[|\textit{main}|]{|\textit{dest}|}|\\
\end{tabular}
\end{center}
%
The argument \textit{dest} is the destination file
(without extension).
It should be the main file or one of the child files.
Note that further \textsf{childdoc} directives
such as |\childdocof| and |\childdocforward|
in the indicated file will be processed in this form.
The optional argument \textit{main}
passes on directly to the main file \textit{main}
while pretending to compile the child \textit{dest}.
This form behaves as if \textit{dest}
issues |\childdocof{|\textit{main}|}| right away,
and no further \textsf{childdoc} directives will be processed.

%%%%%%%%%%%%%%%%%%%%%%%%%%%%%%%%%%%%%%%%
\DescribeMacro{\...prefix}
In the alternative form |\childdocforwardprefix|,
%
\begin{center}
\begin{tabular}{l}
|\input{childdoc.def}|\\
|\childdocforwardprefix[|\textit{main}|]{|\textit{prefix}|}{|\textit{dest}|}|
\end{tabular}
\end{center}
%
the destination file is determined by a pattern
depending on the current file:
To make this work, the current file must be called
`{\textit{prefix}\hspace{0.2em}\textit{suffix}}'
with \textit{prefix} matching precisely the argument.
Processing is then passed on to the file
`{\textit{dest}\hspace{0.2em}\textit{suffix}}'.
Surely, the same effect is achieved by
directly specifying the
argument `{\textit{dest}\hspace{0.2em}\textit{suffix}}'
in the first form.
However, that requires to set up a different file
for each child. With the alternative form of the command
all these files can have exactly the same content
which simplifies setting them up and maintaining them.

For example, the following file |draft.tex|
with a compilation flag |\version| as described in \secref{sec:flags}
compiles the main document as a draft:
%
\begin{center}
\begin{tabular}{l}
|\def\version{draft}|\\
|\input{childdoc.def}|\\
|\childdocforward{|\textit{main}|}|
\end{tabular}
\end{center}
%
Likewise, the following files |final|\textit{nn}|.tex|
compile the final version of the child document
|child|\textit{nn}|.tex|:
%
\begin{center}
\begin{tabular}{l}
|\def\version{final}|\\
|\input{childdoc.def}|\\
|\childdocforwardprefix{final}{child}|
\end{tabular}
\end{center}
%

Note that when several versions of a main file and/or of each child file
are to be generated, it may be convenient to set up a |Makefile| or
shell script to automatise the process.

%%%%%%%%%%%%%%%%%%%%%%%%%%%%%%%%%%%%%%%%%%%%%%%%%%%%%%%%%%%%%%%%%%%%%%%%%%%%%%%%
\subsection{Command Line Processing}
\label{sec:commandline}

The effect of redirection files can also be achieved by invoking
the \LaTeX{} compiler with a more elaborate command line.
Most conveniently this should be done as part
of a shell script or a |Makefile|.

When using \textsf{childdoc} in the main file, the following
command lines effectively perform a redirection
(note that depending on the shell being used,
backslashes may have to be doubled: `|\|' $\to$ `|\\|'):
%
\begin{center}
|... -jobname "|\textit{target}|" |\\|"|[\textit{flags}]%
|\input{childdoc.def}\childdocforward[|\textit{main}|]{|\textit{dest}|}"|
\end{center}
%
Here \textit{target} is the name of the output file,
\textit{main} is the name of the main file
and \textit{dest} is the name of the main or child file to be processed
(all filenames without extensions).
The optional argument \textit{main} can be omitted
if \textit{main} matches \textit{dest}.
Optionally, compilation \textit{flags} can be defined via |\def| commands.
This command line makes the \TeX{} engine believe
it is compiling the file \textit{target}
whose content is specified as the latter parameter.
The provided code then forwards the processing to
\textit{main} or \textit{dest} as described in \secref{sec:forward}.

%%%%%%%%%%%%%%%%%%%%%%%%%%%%%%%%%%%%%%%%%%%%%%%%%%%%%%%%%%%%%%%%%%%%%%%%%%%%%%%%
\subsection{Include by Input}
\label{sec:input}

Including child documents by |\include| has some restrictions by design.
Most notably, the content of a child document always occupies
its own set of pages; pages cannot be shared between child documents.
Usually, this behaviour makes perfect sense
because each child document contain an essential part of the document.
However, in some situations it may be desirable to compose
a document from a collection of parts
without having mandatory page breaks between then.
For this case, the package
provides a mechanism to include parts
by |\input| which can also be processed individually.
However, by construction this mechanism
requires manual handling of the content to be output.

%%%%%%%%%%%%%%%%%%%%%%%%%%%%%%%%%%%%%%%%
\DescribeMacro{\ifchilddocmanual}
The main file should be prepared as usual, see \secref{sec:include}.
However, the document body must make a distinction
between processing of an individual part and of the main document, e.g.:
%
\begin{center}
\begin{tabular}{l}
|\ifchilddocmanual|\\
|\input{\childdocname}|\\
|\||else|\\
\textit{document body with }|\input{|\textit{part}|}|\\
|\||fi|
\end{tabular}
\end{center}
%
The conditional |\ifchilddocmanual| is true whenever
a part to be included by |\input| is being compiled,
and the name of the part is stored in |\childdocname|.

%%%%%%%%%%%%%%%%%%%%%%%%%%%%%%%%%%%%%%%%
\DescribeMacro{\childdocby}
Each part to be included by |\input| should start with:
%
\begin{center}
\begin{tabular}{l}
|\input{childdoc.def}|\\
|\childdocby{|\textit{main}|}|\\
\end{tabular}
\end{center}
%
The directive |\childdocby| is similar to |\childdocof|
described in \secref{sec:include},
but the subsequent selection of content must be done manually.
To that end, both |\ifchilddoc| and |\ifchilddocmanual|
will be true upon processing of a part,
and the name of the part is stored in |\childdocname|.
Note that |\jobname| will be set to the filename of the current part
so that each part receives an individual |.aux| file
that does not interfere with the |.aux| file(s) of the main document.
This behaviour can be altered by the alternative form
|\childdocby[*]{|\textit{main}|}| (with a non-empty optional argument)
which uses the |.aux| file of the main document
by setting |\jobname| to \textit{main}.

%%%%%%%%%%%%%%%%%%%%%%%%%%%%%%%%%%%%%%%%%%%%%%%%%%%%%%%%%%%%%%%%%%%%%%%%%%%%%%%%
\subsection{Driver Development}
\label{sec:driver}

The \textsf{childdoc} mechanism can also be use for the development
of definition files such as \LaTeX{} styles or classes.
This case differs from the above setup with multiple parts
included by |\include| in that no |\includeonly| should be invoked.
This can be achieved by starting the include file
(before |\ProvidesPackage|) with:
%
\begin{center}
\begin{tabular}{l}
|\input{childdoc.def}|\\
|\childdocforward{|\textit{main}|}|\\
\end{tabular}
\end{center}
%
or alternatively with:
%
\begin{center}
\begin{tabular}{l}
|\input{childdoc.def}|\\
|\childdocby{|\textit{main}|}|\\
\end{tabular}
\end{center}
%
Both forms have slightly different effects as described above.
The main file is prepared as usual, see \secref{sec:include}.

%%%%%%%%%%%%%%%%%%%%%%%%%%%%%%%%%%%%%%%%%%%%%%%%%%%%%%%%%%%%%%%%%%%%%%%%%%%%%%%%
\subsection{Legacy Detection}
\label{sec:detection}

The directive |\childdocmain| in the main file can detect
whether the complete document or merely a child is to be compiled
even without using the directive |\childdocof|.
This method is deprecated because it is less robust
and there is no compelling reason to use it;
it is merely provided for backward compatibility
and it may be removed in future versions.

If the detection mechanism is to be used,
it is mandatory to correctly specify
the filename of the main file as the argument of |\childdocmain|:
%
\begin{center}
\begin{tabular}{l}
|\input{childdoc.def}|\\
|\childdocmain{|\textit{main}|}|\\
\end{tabular}
\end{center}
%
If |\jobname| does not match the argument \textit{main} of |\childdocmain|,
it is assumed that |\jobname| points to the child file to be compiled.
When using |\childdocmain| with the main file specified as argument,
it suffices to start a child file
with just |\input{|\textit{main}|}|
without loading of the package and using |\childdocof|.
If instead all processing is done
with the appropriate \textsf{childdoc} directives,
the argument of \textit{main} of |\childdocmain| can be empty.

An alternative version of the command line processing described
in \secref{sec:commandline} using the detection mechanism reads:
%
\begin{center}
|... -jobname "|\textit{target}|" "|[\textit{flags}]%
[|\def\jobname{|\textit{dest}|}|]|\input{|\textit{main}|}"|
\end{center}

%%%%%%%%%%%%%%%%%%%%%%%%%%%%%%%%%%%%%%%%%%%%%%%%%%%%%%%%%%%%%%%%%%%%%%%%%%%%%%%%
\subsection{Manual Code}
\label{sec:manual}

In case one cannot be certain whether the definitions file |childdoc.def|
is installed on the target \TeX{} distribution
and one prefers not to ship it,
it is conceivable to paste a few relevant commands into the sources.

To that end, drop all statements |\input{childdoc.def}|
and perform the replacements as outlined below.
Instead of |\childdocmain{|\textit{main}|}| add the following code
to the top of the main file:
%
\begin{center}
\begin{tabular}{l}
|\||ifdefined\childdocname\endinput\||fi\newif\ifchilddoc|\\
|\edef\childdocname{\scantokens\expandafter{\jobname\noexpand}}|\\
|\def\childdocmain{|\textit{main}|}\||ifx\childdocmain\childdocname\||else|\\
|\childdoctrue\includeonly{\childdocname}\let\jobname\childdocmain\||fi|\\
\end{tabular}
\end{center}
%
Instead of |\childdocof{|\textit{main}|}| just include the main file
at the top of each child file:
%
\begin{center}
|\input{|\textit{main}|}|
\end{center}
%
A simple redirection |\childdocforward{|\textit{dest}|}| is achieved by:
%
\begin{center}
|\def\jobname{|\textit{dest}|}\input{\jobname}|
\end{center}
%
The redirection with prefix
|\childdocforwardprefix[|\textit{prefix}|]{|\textit{dest}|}|
is accomplished by:
%
\begin{center}
\begin{tabular}{l}
|{\edef\jobname{\scantokens\expandafter{\jobname\noexpand}}|\\
|\def\redirectjob |\textit{prefix}|#1~~~{\gdef\jobname{|\textit{dest}|#1}}|\\
|\expandafter\redirectjob\jobname~~~}\input{\jobname}|
\end{tabular}
\end{center}

In an alternative approach,
child documents can be compiled by a specific command line
without additional code or specific definitions:
%
\begin{center}
|... -jobname "|\textit{target}|" "|[\textit{flags}]%
|\includeonly{|\textit{dest}|}\input{|\textit{main}|}"|
\end{center}
%

%%%%%%%%%%%%%%%%%%%%%%%%%%%%%%%%%%%%%%%%%%%%%%%%%%%%%%%%%%%%%%%%%%%%%%%%%%%%%%%%
%%%%%%%%%%%%%%%%%%%%%%%%%%%%%%%%%%%%%%%%%%%%%%%%%%%%%%%%%%%%%%%%%%%%%%%%%%%%%%%%
\section{Information}

%%%%%%%%%%%%%%%%%%%%%%%%%%%%%%%%%%%%%%%%%%%%%%%%%%%%%%%%%%%%%%%%%%%%%%%%%%%%%%%%
\subsection{Copyright}

Copyright \copyright{} 2017--2018 Niklas Beisert

This work may be distributed and/or modified under the
conditions of the \LaTeX{} Project Public License, either version 1.3
of this license or (at your option) any later version.
The latest version of this license is in
  \url{http://www.latex-project.org/lppl.txt}
and version 1.3 or later is part of all distributions of \LaTeX{}
version 2005/12/01 or later.

This work has the LPPL maintenance status `maintained'.

The Current Maintainer of this work is Niklas Beisert.

This work consists of the files |README.txt|, |childdoc.ins| and |childdoc.dtx|
as well as the derived files |childdoc.def|, |cdocsamp.tex|
with |cdocsch1.tex|, |cdocsch2.tex|, |cdocspt3.tex|, |cdocspt4.tex|,
|cdocsdrf.tex|, |cdocsfn1.tex|, |cdocsfn2.tex|
as well as |childdoc.pdf|.

%%%%%%%%%%%%%%%%%%%%%%%%%%%%%%%%%%%%%%%%%%%%%%%%%%%%%%%%%%%%%%%%%%%%%%%%%%%%%%%%
\subsection{Files and Installation}

The package consists of the files:
%
\begin{center}
\begin{tabular}{ll}
    |README.txt|   & readme file \\
    |childdoc.ins| & installation file \\
    |childdoc.dtx| & source file \\
    |childdoc.def| & definition file \\
    |cdocsamp.tex| & sample main file \\
    |cdocsch1.tex| & sample include file \\
    |cdocsch2.tex| & sample include file \\
    |cdocspt3.tex| & sample part file \\
    |cdocspt4.tex| & sample part file \\
    |cdocsdrf.tex| & sample redirection file \\
    |cdocsfn1.tex| & sample redirection file \\
    |cdocsfn2.tex| & sample redirection file \\
    |childdoc.pdf| & manual
\end{tabular}
\end{center}
%
The distribution consists of the files
|README.txt|, |childdoc.ins| and |childdoc.dtx|.
%
\begin{itemize}
\item
Run (pdf)\LaTeX{} on |childdoc.dtx|
to compile the manual |childdoc.pdf| (this file).
\item
Run \LaTeX{} on |childdoc.ins| to create the definitions file |childdoc.def|
and the sample |cdocsamp.tex| with include files
|cdocsch1.tex|, |cdocsch2.tex|, |cdocspt3.tex|, |cdocspt4.tex|,
|cdocsdrf.tex|, |cdocsfn1.tex|, |cdocsfn2.tex|.
Then copy the file |childdoc.def| to an appropriate directory of your \LaTeX{}
distribution, e.g.\ \textit{texmf-root}|/tex/latex/childdoc|.
\end{itemize}

%%%%%%%%%%%%%%%%%%%%%%%%%%%%%%%%%%%%%%%%%%%%%%%%%%%%%%%%%%%%%%%%%%%%%%%%%%%%%%%%
\subsection{Related CTAN Packages}

There are several other packages which offer a similar functionality:
%
\begin{itemize}
\item
The packages
\href{http://ctan.org/pkg/docmute}{\textsf{docmute}},
\href{http://ctan.org/pkg/includex}{\textsf{includex}} and
\href{http://ctan.org/pkg/standalone}{\textsf{standalone}}
provide commands to include only the document body of
a child file thus allowing both files to be compiled individually.
\item
The packages \href{http://ctan.org/pkg/subdocs}{\textsf{subdocs}}
and \href{http://ctan.org/pkg/subfiles}{\textsf{subfiles}}
provide structures in which the main and child documents can be
encapsulated and allowing them to be compiled individually.
The inclusion mechanism is different from the conventional |\include|.
\item
The package \href{http://ctan.org/pkg/combine}{\textsf{combine}}
is an elaborate solution to combine several documents into one.
\end{itemize}
%
See also the CTAN topic \href{http://ctan.org/topic/subdocs}{\textsf{subdocs}}
for further related packages.
The present package differs from the above solutions in that
a document structure constructed with the conventional |\include| mechanism
just needs two extra commands at the top of every file
such that all constituent files can be compiled individually.

%%%%%%%%%%%%%%%%%%%%%%%%%%%%%%%%%%%%%%%%%%%%%%%%%%%%%%%%%%%%%%%%%%%%%%%%%%%%%%%%
%\subsection{Feature Suggestions}
%
%The following is a list of features which may be useful for future
%versions of this package:
%%
%\begin{itemize}
%\item
%\ldots
%\end{itemize}

%%%%%%%%%%%%%%%%%%%%%%%%%%%%%%%%%%%%%%%%%%%%%%%%%%%%%%%%%%%%%%%%%%%%%%%%%%%%%%%%
\subsection{Revision History}

%%%%%%%%%%%%%%%%%%%%%%%%%%%%%%%%%%%%%%%%
\paragraph{v2.0:} 2018/12/30

\begin{itemize}
\item
immediate forward processing
\item
added |\childdocby| mechanism
\item
manual restructured
\end{itemize}

%%%%%%%%%%%%%%%%%%%%%%%%%%%%%%%%%%%%%%%%
\paragraph{v1.6:} 2018/01/17

\begin{itemize}
\item
application for development of include files
\item
corrections to manual
\end{itemize}

%%%%%%%%%%%%%%%%%%%%%%%%%%%%%%%%%%%%%%%%
\paragraph{v1.5:} 2017/05/21

\begin{itemize}
\item
more complete structuring introduced
\item
|\childdocof| introduced
\item
|\childdoc| renamed to |\childdocmain|
\item
|\childredirect| renamed to |\childdocforward| and |\childdocforwardprefix|
and functionality expanded
\end{itemize}

%%%%%%%%%%%%%%%%%%%%%%%%%%%%%%%%%%%%%%%%
\paragraph{v1.0:} 2017/04/27

\begin{itemize}
\item
manual and install package
\item
first version published on CTAN
\end{itemize}

%%%%%%%%%%%%%%%%%%%%%%%%%%%%%%%%%%%%%%%%
\paragraph{v0.6:} 2017/04/26

\begin{itemize}
\item
redirection mechanism added
\end{itemize}

%%%%%%%%%%%%%%%%%%%%%%%%%%%%%%%%%%%%%%%%
\paragraph{v0.5:} 2017/04/26

\begin{itemize}
\item
functionality in definition file
\end{itemize}


%%%%%%%%%%%%%%%%%%%%%%%%%%%%%%%%%%%%%%%%%%%%%%%%%%%%%%%%%%%%%%%%%%%%%%%%%%%%%%%%
%%%%%%%%%%%%%%%%%%%%%%%%%%%%%%%%%%%%%%%%%%%%%%%%%%%%%%%%%%%%%%%%%%%%%%%%%%%%%%%%
%%%%%%%%%%%%%%%%%%%%%%%%%%%%%%%%%%%%%%%%%%%%%%%%%%%%%%%%%%%%%%%%%%%%%%%%%%%%%%%%
\appendix

\settowidth\MacroIndent{\rmfamily\scriptsize 000\ }

 \DocInput{childdoc.dtx}

\end{document}
%</driver>
% \fi
%
% %%%%%%%%%%%%%%%%%%%%%%%%%%%%%%%%%%%%%%%%%%%%%%%%%%%%%%%%%%%%%%%%%%%%%%%%%%%%%%
% %%%%%%%%%%%%%%%%%%%%%%%%%%%%%%%%%%%%%%%%%%%%%%%%%%%%%%%%%%%%%%%%%%%%%%%%%%%%%%
% \section{Sample}
%\iffalse
%<*samplemain>
%\fi
%
% The following presents a sample document
% with two chapters, two parts, a title page,
% a compile flag as well as three forwarding files to set the flag.
% It consists of eight |.tex| files:
% \begin{center}
% \begin{tabular}{ll}
% |cdocsamp.tex|&main file\\
% |cdocsch1.tex|&include file for chapter 1\\
% |cdocsch2.tex|&include file for chapter 2\\
% |cdocspt3.tex|&include file for part 3\\
% |cdocspt4.tex|&include file for part 4\\
% |cdocsdrf.tex|&forwarding file for main file in draft mode\\
% |cdocsfi1.tex|&forwarding file for final version of chapter 1\\
% |cdocsfi2.tex|&forwarding file for final version of chapter 2\\
% \end{tabular}
% \end{center}
% Each of the eight files can be compiled directly by the \LaTeX{} compiler.
%
% %%%%%%%%%%%%%%%%%%%%%%%%%%%%%%%%%%%%%%
% \paragraph{Main File.}
%
% The main file is called |cdocsamp.tex|.
%
% Load the \textsf{childdoc} definitions and
% declare the filename for the main document:
%    \begin{macrocode}
\input{childdoc.def}
\childdocmain{}
%    \end{macrocode}

% Optional override for |\version| flag:
%    \begin{macrocode}
%%\ifchilddoc\else\providecommand{\version}{draft}\fi
%    \end{macrocode}

% Define the default values for the |\version| flag
% (|final| for the main file and |draft| for childs):
%    \begin{macrocode}
\ifchilddoc
\providecommand{\version}{draft}
\else
\providecommand{\version}{final}
\fi
%    \end{macrocode}

% Load the standard document class:
%    \begin{macrocode}
\documentclass[12pt]{article}
%    \end{macrocode}

% Start the document body:
%    \begin{macrocode}
\begin{document}
%    \end{macrocode}

% Declare a title page.
% Print title, part of document being processed and version flag:
%    \begin{macrocode}
\addtocounter{page}{-1}
\begin{center}
{\LARGE\bfseries{}childdoc example\par}
\vspace{1cm}
\ifchilddoc
\ifchilddocmanual part\else chapter\fi:
`\childdocname' of `\childdocjob'\par
\else
main document: `\childdocjob'\par
\fi
version: \version\par
\end{center}
\newpage
%    \end{macrocode}

% Manually include selected file,
% otherwise process as usual:
%    \begin{macrocode}
\ifchilddocmanual
\section*{part `\childdocname'}
\input{\childdocname}
\else
%    \end{macrocode}

% Include the two chapters:
%    \begin{macrocode}
\include{cdocsch1}
\include{cdocsch2}
%    \end{macrocode}

% Include the two parts unless only chapters should be displayed:
%    \begin{macrocode}
\ifchilddoc\else
\section{part three}
\input{cdocspt3}
\section{part four}
\input{cdocspt4}
\fi
%    \end{macrocode}

% Process as usual until here:
%    \begin{macrocode}
\fi
%    \end{macrocode}

% End of document body:
%    \begin{macrocode}
\end{document}
%    \end{macrocode}
%\iffalse
%</samplemain>
%\fi
%
% %%%%%%%%%%%%%%%%%%%%%%%%%%%%%%%%%%%%%%
% \paragraph{Chapter Include Files.}
%
% The include files are called |cdocsch1.tex| and |cdocsch2.tex|.
%
%\iffalse
%<*samplechap1|samplechap2>
%\fi

% Optional override for |\version| flag:
%    \begin{macrocode}
%%\providecommand{\version}{final}
%    \end{macrocode}

% Include the main document:
%    \begin{macrocode}
\input{childdoc.def}
\childdocof{cdocsamp}
%    \end{macrocode}

%\iffalse
%</samplechap1|samplechap2>
%\fi
%
%\iffalse
%<*samplechap1>
%\fi
% Some text for chapter 1:
%    \begin{macrocode}
\section{one}
some text in chapter one
%    \end{macrocode}

%\iffalse
%</samplechap1>
%\fi
% Some text for chapter 2:
%\iffalse
%<*samplechap2>
%\fi
%    \begin{macrocode}
\section{two}
more text in chapter two
%    \end{macrocode}

%\iffalse
%</samplechap2>
%\fi
%
% %%%%%%%%%%%%%%%%%%%%%%%%%%%%%%%%%%%%%%
% \paragraph{Part Include Files.}
%
% The include files are called |cdocspt3.tex| and |cdocspt4.tex|.
%
%\iffalse
%<*samplepart3|samplepart4>
%\fi

% Optional override for |\version| flag:
%    \begin{macrocode}
%%\providecommand{\version}{final}
%    \end{macrocode}

% Include the main document:
%    \begin{macrocode}
\input{childdoc.def}
\childdocby{cdocsamp}
%    \end{macrocode}

%\iffalse
%</samplepart3|samplepart4>
%\fi
%
%\iffalse
%<*samplepart3>
%\fi
% Some text for part 3:
%    \begin{macrocode}
some text in part three
%    \end{macrocode}

%\iffalse
%</samplepart3>
%\fi
% Some text for part 4:
%\iffalse
%<*samplepart4>
%\fi
%    \begin{macrocode}
more text in part four
%    \end{macrocode}

%\iffalse
%</samplepart4>
%\fi
%
% %%%%%%%%%%%%%%%%%%%%%%%%%%%%%%%%%%%%%%
% \paragraph{Forwarding for a Complete Draft.}
%
% The following forwarding file |cdocsdrf.tex|
% compiles the main document in draft mode:
%\iffalse
%<*sampledraft>
%\fi
%    \begin{macrocode}
\def\version{draft}
\input{childdoc.def}
\childdocforward{cdocsamp}
%    \end{macrocode}

%\iffalse
%</sampledraft>
%\fi
%
% %%%%%%%%%%%%%%%%%%%%%%%%%%%%%%%%%%%%%%
% \paragraph{Forwarding for Final Version of the Chapters.}
%
% The following forwarding files |cdocsfn1.tex| and |cdocsfn2.tex|
% (with identical content)
% compile the final versions of the child documents
% |cdocsch1.tex| and |cdocsch2.tex|, respectively:
%\iffalse
%<*samplefinal>
%\fi
%    \begin{macrocode}
\def\version{final}
\input{childdoc.def}
\childdocforwardprefix[cdocsamp]{cdocsfn}{cdocsch}
%    \end{macrocode}

%\iffalse
%</samplefinal>
%\fi
%
% %%%%%%%%%%%%%%%%%%%%%%%%%%%%%%%%%%%%%%
% \paragraph{Command Line Processing.}
%
% The following three command lines generate the output files
% |cdocscld|, |cdocscl1| and |cdocscl2|
% which should be identical to
% |cdocsdrf|, |cdocsch1| and |cdocsfn2|, respectively:
% \begin{center}
% \begin{tabular}{l}
% |latex -jobname cdocscld \|\\
% |  "\def\version{draft}\input{childdoc.def}\childdocforward{cdocsamp}"|\\
% |latex -jobname cdocscl1 \|\\
% |  "\input{childdoc.def}\childdocforward[cdocsamp]{cdocsch1}"|\\
% |latex -jobname cdocscl2 \|\\
% |  "\def\version{final}\input{childdoc.def}\childdocforward{cdocsch2}"|
% \end{tabular}
% \end{center}
% Note that the trailing backslash on each first line
% merely continues the input to the second line
% (for convenient cut ant paste).
% Furthermore, the command |latex| can be replaced by any
% of its alternative versions such as |pdflatex|.
%
% %%%%%%%%%%%%%%%%%%%%%%%%%%%%%%%%%%%%%%%%%%%%%%%%%%%%%%%%%%%%%%%%%%%%%%%%%%%%%%
% %%%%%%%%%%%%%%%%%%%%%%%%%%%%%%%%%%%%%%%%%%%%%%%%%%%%%%%%%%%%%%%%%%%%%%%%%%%%%%
% \section{Implementation}
%\iffalse
%<*package>
%\fi
%
% This section describes the definitions file |childdoc.def|.

% The definitions cannot be loaded using |\usepackage| or |\RequirePackage|
% which has a mechanism to prevent loading a style file more than once.
% When loading the definitions by means of |\input|
% multiple instances have to be prevented manually:
%\iffalse
%This code needs to be before the `\ProvidesFile' directive
%which is defined at the beginning of this file.
%Therefore it is also placed there and commented out here.
%</package>
%<*discard>
%\fi
%    \begin{macrocode}
\ifdefined\childdocmain\endinput\fi
%    \end{macrocode}
%\iffalse
%</discard>
%<*package>
%\fi
%
% \macro{\ifchilddoc}
% \macro{\ifchilddocmanual}
% The conditional |\ifchilddoc| tells whether a
% child (true) or main (false) document is being compiled.
% The conditional |\ifchilddocmanual| tells whether
% the |\includeonly| mechanism is used (false) or
% the selection of child files must be performed manually (true).
% The definitions initialise to false:
%    \begin{macrocode}
\newif\ifchilddoc
\newif\ifchilddocmanual
%    \end{macrocode}

% \macro{\childdocname}
% \macro{\childdocjob}
% The macro |\childdocname| stores the name of the main document
% to be compiled. The macro |\childdocjob| stores the name of
% the document on which the \LaTeX{} compiler was originally invoked.
% The content of |\jobname| cannot be compared
% to filenames specified in the source due to different catcodes.
% The following code rescans |\jobname|, stores the result
% in |\childdocname| and saves a copy in |\childdocjob|:
%    \begin{macrocode}
\edef\childdocname{\scantokens\expandafter{\jobname\noexpand}}
\let\childdocjob\childdocname
%    \end{macrocode}

% \macro{\childdocdisable}
% The macro |\childdocdisable| prevents the main file
% from being processed more than once.
% At this stage, the main document command |\childdocmain|
% is assumed to be called once again where it should do nothing.
% Any subsequent call to it should prevent
% a secondary processing of the main document
% It overwrites the forwarding commands
% |\childdocof| and |\childdocforward|
% with empty macros to prevent further inclusions of the main document:
%    \begin{macrocode}
\newcommand{\childdocdisable}
{
  \renewcommand{\childdocmain}[1]{\renewcommand{\childdocmain}[1]{\endinput}}
  \renewcommand{\childdocof}[1]{}
  \renewcommand{\childdocby}[2][]{}
  \renewcommand{\childdocforward}[2][]{}
  \renewcommand{\childdocdisable}{}
}
%    \end{macrocode}

% \macro{\childdocmain}
% The macro |\childdocmain| is to be called at the top of the main file
% with nothing or the main filename (without extension) as argument.
% First, it breaks loops.
% If the argument is not empty and does not match |\childdocname|
% (which is set by the first inclusion of |childdoc.def|),
% |\ifchilddoc| is set to true, |\includeonly| is applied to the child file
% and |\jobname| is set to the main file
% (for proper handling of |.aux| files):
%    \begin{macrocode}
\newcommand{\childdocmain}[1]
{
  \childdocdisable\childdocmain{}
  \if?#1?\else
    \begingroup
      \def\childdoctmp{#1}
      \ifx\childdoctmp\childdocname
        \def\childdoctmp{}
      \else
        \def\childdoctmp
        {
          \childdoctrue
          \includeonly{\childdocname}
          \def\childdocjob{#1}
          \def\jobname{#1}
        }
      \fi
      \expandafter
    \endgroup
    \childdoctmp
  \fi
}
%    \end{macrocode}

% \macro{\childdocof}
% The command |\childdocof| redirects
% compilation to the main file |#1|.
%    \begin{macrocode}
\newcommand{\childdocof}[1]
{
  \childdocdisable
  \childdoctrue
  \includeonly{\childdocname}
  \def\jobname{#1}
  \def\childdocjob{#1}
  \input{#1}
}
%    \end{macrocode}

% \macro{\childdocby}
% The command |\childdocby| ....
%    \begin{macrocode}
\newcommand{\childdocby}[2][]
{
  \childdocdisable
  \childdoctrue
  \childdocmanualtrue
  \if?#1?\else
    \def\jobname{#2}
  \fi
  \def\childdocjob{#2}
  \input{#2}
  \endinput
}
%    \end{macrocode}

% \macro{\childdocforward}
% The command |\childdocforward| redirects
% compilation to the main file or
% (if the optional argument is given) a child file.
% Parameters are set as if the main file
% or a child file starting with |\childdocof| was compiled.
% Then compilation is handed over to the main file:
%    \begin{macrocode}
\newcommand{\childdocforward}[2][]
{
  \begingroup
    \if?#1?
      \def\childdoctmp
      {
        \def\childdocname{#2}
        \def\childdocjob{#2}
        \def\jobname{#2}
        \input{#2}
        \endinput
      }
    \else
      \def\childdoctmp
      {
        \childdocdisable
        \def\childdocname{#2}
        \childdoctrue
        \includeonly{#2}
        \def\childdocjob{#1}
        \def\jobname{#1}
        \input{#1}
        \endinput
      }
    \fi
    \expandafter
  \endgroup
  \childdoctmp
}
%    \end{macrocode}

% \macro{\childdocforwardprefix}
% The command |\childdocforwardprefix| redirects
% compilation to the main or a child file by means of a pattern.
% The prefix |#1| in the current filename is replaced by |#2|
% and the suffix of the current filename is kept
% (it is assumed that the filename does not contain the substring `|~~~|'
% which is used as a delimiter).
% Compilation is handed over to the new file by |\childdocforward|:
%    \begin{macrocode}
\newcommand{\childdocforwardprefix}[3][]
{
  \begingroup
    \def\childdocextract #2##1~~~{\def\childdoctmp{\childdocforward[#1]{#3##1}}}
    \expandafter\childdocextract\childdocname~~~
    \expandafter
  \endgroup
  \childdoctmp
}
%    \end{macrocode}

% \macro{\childdoc}
% The deprecated macro |\childdoc| is a legacy version of |\childdocmain|:
%    \begin{macrocode}
\newcommand{\childdoc}{\childdocmain}
%    \end{macrocode}

% \macro{\childdocredirect}
% The deprecated macro |\childdocredirect| is a legacy version
% of |\childdocforward| and |\childdocforwardprefix|:
%    \begin{macrocode}
\newcommand{\childdocredirect}[2][]
{
  \begingroup
    \if?#1?
      \def\childdoctmp{\childdocforward{#2}}
    \else
      \def\childdoctmp{\childdocforwardprefix{#1}{#2}}
    \fi
    \expandafter
  \endgroup
  \childdoctmp
}
%    \end{macrocode}

%\iffalse
%</package>
%\fi
%
\endinput
|\\
|\childdocby{|\textit{main}|}|\\
\end{tabular}
\end{center}
%
Both forms have slightly different effects as described above.
The main file is prepared as usual, see \secref{sec:include}.

%%%%%%%%%%%%%%%%%%%%%%%%%%%%%%%%%%%%%%%%%%%%%%%%%%%%%%%%%%%%%%%%%%%%%%%%%%%%%%%%
\subsection{Legacy Detection}
\label{sec:detection}

The directive |\childdocmain| in the main file can detect
whether the complete document or merely a child is to be compiled
even without using the directive |\childdocof|.
This method is deprecated because it is less robust
and there is no compelling reason to use it;
it is merely provided for backward compatibility
and it may be removed in future versions.

If the detection mechanism is to be used,
it is mandatory to correctly specify
the filename of the main file as the argument of |\childdocmain|:
%
\begin{center}
\begin{tabular}{l}
|% \iffalse
%
% childdoc.dtx Copyright (C) 2017-2018 Niklas Beisert
%
% This work may be distributed and/or modified under the
% conditions of the LaTeX Project Public License, either version 1.3
% of this license or (at your option) any later version.
% The latest version of this license is in
%   http://www.latex-project.org/lppl.txt
% and version 1.3 or later is part of all distributions of LaTeX
% version 2005/12/01 or later.
%
% This work has the LPPL maintenance status `maintained'.
%
% The Current Maintainer of this work is Niklas Beisert.
%
% This work consists of the files childdoc.dtx and childdoc.ins
% and the derived files childdoc.def and cdocsamp.tex with
% cdocsch1.tex, cdocsch2.tex, cdocsdrf.tex, cdocsfn1.tex, cdocsfn2.tex.
%
%<package>\ifdefined\childdocmain\endinput\fi
%<package>\ProvidesFile{childdoc.def}[2018/12/30 v2.0 child document driver]
%<samplemain>\ProvidesFile{cdocsamp.tex}[2018/12/30 v2.0 sample for childdoc]
%<*driver>
%\ProvidesFile{childdoc.drv}[2018/12/30 v2.0 childdoc reference manual file]
\PassOptionsToClass{10pt,a4paper}{article}
\documentclass{ltxdoc}

\usepackage[margin=35mm]{geometry}
\usepackage{hyperref}
\usepackage{hyperxmp}
\usepackage[usenames]{color}

\hypersetup{colorlinks=true}
\hypersetup{pdfstartview=FitH}
\hypersetup{pdfpagemode=UseNone}
\hypersetup{pdfsource={}}
\hypersetup{pdflang={en-UK}}
\hypersetup{pdfcopyright={Copyright 2017-2018 Niklas Beisert.
  This work may be distributed and/or modified under the
  conditions of the LaTeX Project Public License, either version 1.3
  of this license or (at your option) any later version.}}
\hypersetup{pdflicenseurl={http://www.latex-project.org/lppl.txt}}
\hypersetup{pdfcontactaddress={ETH Zurich, ITP, HIT K,
  Wolfgang-Pauli-Strasse 27}}
\hypersetup{pdfcontactpostcode={8093}}
\hypersetup{pdfcontactcity={Zurich}}
\hypersetup{pdfcontactcountry={Switzerland}}
\hypersetup{pdfcontactemail={nbeisert@itp.phys.ethz.ch}}
\hypersetup{pdfcontacturl={http://people.phys.ethz.ch/\xmptilde nbeisert/}}

\newcommand{\secref}[1]{\hyperref[#1]{section \ref*{#1}}}

\parskip1ex
\parindent0pt
\let\olditemize\itemize
\def\itemize{\olditemize\parskip0pt}

\begin{document}

\title{The \textsf{childdoc} Package}
\hypersetup{pdftitle={The childdoc Package}}
\author{Niklas Beisert\\[2ex]
  Institut f\"ur Theoretische Physik\\
  Eidgen\"ossische Technische Hochschule Z\"urich\\
  Wolfgang-Pauli-Strasse 27, 8093 Z\"urich, Switzerland\\[1ex]
  \href{mailto:nbeisert@itp.phys.ethz.ch}
  {\texttt{nbeisert@itp.phys.ethz.ch}}}
\hypersetup{pdfauthor={Niklas Beisert}}
\hypersetup{pdfsubject={Manual for the LaTeX2e Package childdoc}}
\date{30 December 2018, \textsf{v2.0}}
\maketitle

\begin{abstract}\noindent
\textsf{childdoc} is a \LaTeXe{} package
that enables the direct compilation
of document sections included by |\include|
to individual files.
\end{abstract}

\begingroup
\parskip0ex
\tableofcontents
\endgroup

%%%%%%%%%%%%%%%%%%%%%%%%%%%%%%%%%%%%%%%%%%%%%%%%%%%%%%%%%%%%%%%%%%%%%%%%%%%%%%%%
%%%%%%%%%%%%%%%%%%%%%%%%%%%%%%%%%%%%%%%%%%%%%%%%%%%%%%%%%%%%%%%%%%%%%%%%%%%%%%%%
\section{Introduction}

\LaTeX{} provides a mechanism to structure a large document (such as a book)
into a main file and several child files (containing the chapters)
using the |\include| command.
This mechanism is beneficial for documents
which span hundreds of pages in order to
make the source file(s) more manageable.
Moreover, compilation can be restricted to
selected child files by means of the |\includeonly| command.
The latter feature can be used to reduce the compilation time while editing
(this was significantly more useful in the earlier days of \LaTeX{})
or to generate a smaller document which is easier to navigate.
Another application of |\includeonly| is to generate
documents consisting of selected parts of the complete document.

However, there are a few drawbacks of the plain |\include| mechanism:
\begin{itemize}
\item
The child files cannot be compiled on their own,
they can only be compiled via the main file.
A naive editing environment
(such as a text editor with an option
to have the current file processed by \LaTeX)
may require one to switch to the main file before compiling;
attempting to compile the child file produces errors.
\item
The main file must be modified (each time)
to adjust the |\includeonly| command
to the present needs. This easily leaves the main file in a messy state.
\item
The generated document will always carry the filename
of the main document. This is inconvenient if
several child files are to be compiled and
to be kept for distribution.
\end{itemize}

The present package provides a simple interface
to make child files individually compilable by \LaTeX{}.
Compiling a child file then has the same effect as compiling
the main file with an |\includeonly| command
to select the appropriate child.
Moreover the generated document will carry the name of the child
rather than the main file.
This resolves all three above issues.

This feature is meant to make the editing of books,
thesis documents and lecture notes somewhat more convenient.
However, the package can also be used efficiently for
composing a series of documents (such as exercise sheets)
which are typically distributed individually.
It then assists the author in generating the individual documents
(potentially in different versions)
as well as a document containing the collected series.
Another application is in developing style files
or other kinds of included material
where compilation of the style file could redirect
to a sample or test file.

%%%%%%%%%%%%%%%%%%%%%%%%%%%%%%%%%%%%%%%%%%%%%%%%%%%%%%%%%%%%%%%%%%%%%%%%%%%%%%%%
%%%%%%%%%%%%%%%%%%%%%%%%%%%%%%%%%%%%%%%%%%%%%%%%%%%%%%%%%%%%%%%%%%%%%%%%%%%%%%%%
\section{Usage}

First of all, the package \textsf{childdoc} is \emph{not} a standard
\LaTeXe{} |.sty| style file! Therefore it needs to be invoked in
a non-standard way.

%%%%%%%%%%%%%%%%%%%%%%%%%%%%%%%%%%%%%%%%%%%%%%%%%%%%%%%%%%%%%%%%%%%%%%%%%%%%%%%%
\subsection{Included Files}
\label{sec:include}

%%%%%%%%%%%%%%%%%%%%%%%%%%%%%%%%%%%%%%%%
\DescribeMacro{\childdocmain}
To use the package, add the commands
\begin{center}
\begin{tabular}{l}
|\input{childdoc.def}|\\
|\childdocmain{}|\\
\end{tabular}
\end{center}
at the very top of the main \LaTeX{} file,
in particular \emph{before} the |\documentclass| statement!
The argument of |\childdocmain| should be left empty
(but it must be present).

%%%%%%%%%%%%%%%%%%%%%%%%%%%%%%%%%%%%%%%%
\DescribeMacro{\childdocof}
Furthermore, add the commands
\begin{center}
\begin{tabular}{l}
|\input{childdoc.def}|\\
|\childdocof{|\textit{main}|}|\\
\end{tabular}
\end{center}
at the top of every child file \textit{child}
which is included by |\include{|\textit{child}|}|
from within the main file
(or at least for those files to be compiled individually).
The argument \textit{main} must be the filename of the main file.

There are a couple of
considerations in setting up the main and child documents:

%%%%%%%%%%%%%%%%%%%%%%%%%%%%%%%%%%%%%%%%
\paragraph{Restrictions.}

Please note the following restrictions:
\begin{itemize}
\item
|\childdocmain| must be called with one argument \textit{main}
to ensure compatibility with earlier version of the package.
It must either be empty (|\childdocmain{}|)
or precisely match the filename of the main file in which it is specified.
See \secref{sec:detection} for further information.
\item
The filename \textit{main} must be specified without the |.tex| extension.
\item
The filename \textit{main} is case sensitive
(even in case-insensitive file systems)
due to internal string comparison.
\item
The argument \textit{main} should be fully expanded, it cannot be a macro.
\item
Subdirectories and special characters should be avoided in filenames.
\item
The command |\childdocmain{|\textit{main}|}| must be followed by a whitespace.
It should not be followed immediately by another command
or by a comment mark `|%|'.
This is because the \TeX{} parser reads the token immediately following
the argument of |\childdocmain| and puts it
at the beginning of every child section;
however, a white\-space is ignored.
\end{itemize}

%%%%%%%%%%%%%%%%%%%%%%%%%%%%%%%%%%%%%%%%
\paragraph{Content of Main File.}

It is advisable to place all content in the child files included by |\include|.
Any output contained in the main file will appear in all child documents
unless suppressed manually;
it cannot be suppressed automatically by the |\includeonly| directive
and thus should normally be avoided.
A method to include some content in the main file
by means of conditional processing is described in \secref{sec:conditional}.

%%%%%%%%%%%%%%%%%%%%%%%%%%%%%%%%%%%%%%%%
\paragraph{Page Numbering.}

When only a part of the document is compiled,
the appropriate numbering of pages
(as well as other status parameters)
is determined from the |.aux| files.
The latter contain information from previous passes.
However this information needs to propagate through
all intermediate child documents.
Therefore the page numbering in child documents may well
be inconsistent until the complete document is compiled at least once.

A useful (if unconventional) way to always ensure a consistent
page numbering is to restart the numbering in each child document
and denote the pages by `\textit{child}|.|\textit{page}'
where \textit{child} represents the chapter/section number of the child file.
This can be achieved by the command
|\numberwithin{page}{|\textit{child}|}|
of the \textsf{amsmath} package
where \textit{child} can be |chapter| or |section|
depending on the chosen structuring.
Alternatively, one can modify the macro |\thepage| appropriately
and reset the counter |page| at the start of each child file.

%%%%%%%%%%%%%%%%%%%%%%%%%%%%%%%%%%%%%%%%%%%%%%%%%%%%%%%%%%%%%%%%%%%%%%%%%%%%%%%%
\subsection{Conditional Processing}
\label{sec:conditional}

The package provides a mechanism to compile different versions
of a document. To customise the versions further some conditional processing
can come in handy to distinguish which version is being compiled.
The package provides two macros to describe the compilation context:

%%%%%%%%%%%%%%%%%%%%%%%%%%%%%%%%%%%%%%%%
\DescribeMacro{\ifchilddoc}
The conditional |\ifchilddoc| distinguishes between the compilation of
child documents and the main document:
%
\begin{center}
|\ifchilddoc |\textit{child-code}| |[|\||else |\textit{main-code}]| \||fi|
\end{center}

%%%%%%%%%%%%%%%%%%%%%%%%%%%%%%%%%%%%%%%%
\DescribeMacro{\childdocname}
\DescribeMacro{\childdocjob}
The macro |\childdocname| contains the filename (without extension)
of the main or child file being processed.
Note that |\childdocjob| will always contain the name of the main file.

%%%%%%%%%%%%%%%%%%%%%%%%%%%%%%%%%%%%%%%%
\paragraph{Title Page.}

Conditional processing can be used to include a title or banner page
in the main document when proper precautions are taken.
Importantly, the code in the main file should ensure that the page counter
(as well as other status parameters which are stored in the |.aux| files)
takes the same value after the conditional processing.
Otherwise the page numbers may take divergent values
depending on which part is compiled.

For example, a title page could be declared by:
%
\begin{center}
\begin{tabular}{l}
|\ifchilddoc\||else|\\
|\addtocounter{page}{-1}|\\
\textit{code for title page}\\
|\newpage|\\
|\||fi|
\end{tabular}
\end{center}
%
A banner page for the child documents can be generated by:
%
\begin{center}
\begin{tabular}{l}
|\ifchilddoc|\\
|\addtocounter{page}{-1}|\\
\textit{code for banner page}\\
|\newpage|\\
|\||fi|
\end{tabular}
\end{center}
%
Here one could write a message such as:
\begin{center}
|This is the part \childdocname{} of \childdocjob{}.|
\end{center}

%%%%%%%%%%%%%%%%%%%%%%%%%%%%%%%%%%%%%%%%%%%%%%%%%%%%%%%%%%%%%%%%%%%%%%%%%%%%%%%%
\subsection{Flags}
\label{sec:flags}

The package makes it easy to generate different versions
of the main or child documents.
To this end compilation flags can be defined
and assigned different default values.
They will be particularly useful in conjunction
with the forwarding mechanism described in \secref{sec:forward}.

For example, it may be useful to have a flag |\version|
which can be set to |draft| or |final|.
The document source will contain some conditional code
depending on the value of |\version|.
Suppose further, the flag should default to |final| for the main file
and to |draft| for child files
which is a natural assignment for editing the document.
This is achieved by placing the following code
in the preamble of the main document
(below the |\childdocmain| directive):
%
\begin{center}
\begin{tabular}{l}
|\ifchilddoc|\\
|\providecommand{\version}{draft}|\\
|\||else|\\
|\providecommand{\version}{final}|\\
|\||fi|
\end{tabular}
\end{center}
%
The definition by |\providecommand| makes sure
that previous definitions are not overwritten.
Further statements |\providecommand{\version}{...}|
can thus be added before the above code to override it.

For the main file, one might add a line
(between |\childdocmain| and the above block)
%
\begin{center}
|%\ifchilddoc\||else\providecommand{\version}{draft}\||fi|
\end{center}
%
which can be uncommented to produce a draft version.
Likewise one can add a line to the very top of a child file
(above the |\childdocof{|\textit{main}|}| directive)
%
\begin{center}
|%\providecommand{\version}{final}|
\end{center}
%
which can be uncommented to produce the final version of this child document.

%%%%%%%%%%%%%%%%%%%%%%%%%%%%%%%%%%%%%%%%%%%%%%%%%%%%%%%%%%%%%%%%%%%%%%%%%%%%%%%%
\subsection{Forwarding}
\label{sec:forward}

Different versions of the main or child documents
using compilation flags as described in \secref{sec:flags}
can be (permanently) stored in different files
for convenient compilation, viewing and distribution.
To this end, the package defines a command
to pass on compilation to a different file:

%%%%%%%%%%%%%%%%%%%%%%%%%%%%%%%%%%%%%%%%
\DescribeMacro{\childdocforward}
The command |\childdocforward| redirects processing to
another source file:
%
\begin{center}
\begin{tabular}{l}
|\input{childdoc.def}|\\
|\childdocforward[|\textit{main}|]{|\textit{dest}|}|\\
\end{tabular}
\end{center}
%
The argument \textit{dest} is the destination file
(without extension).
It should be the main file or one of the child files.
Note that further \textsf{childdoc} directives
such as |\childdocof| and |\childdocforward|
in the indicated file will be processed in this form.
The optional argument \textit{main}
passes on directly to the main file \textit{main}
while pretending to compile the child \textit{dest}.
This form behaves as if \textit{dest}
issues |\childdocof{|\textit{main}|}| right away,
and no further \textsf{childdoc} directives will be processed.

%%%%%%%%%%%%%%%%%%%%%%%%%%%%%%%%%%%%%%%%
\DescribeMacro{\...prefix}
In the alternative form |\childdocforwardprefix|,
%
\begin{center}
\begin{tabular}{l}
|\input{childdoc.def}|\\
|\childdocforwardprefix[|\textit{main}|]{|\textit{prefix}|}{|\textit{dest}|}|
\end{tabular}
\end{center}
%
the destination file is determined by a pattern
depending on the current file:
To make this work, the current file must be called
`{\textit{prefix}\hspace{0.2em}\textit{suffix}}'
with \textit{prefix} matching precisely the argument.
Processing is then passed on to the file
`{\textit{dest}\hspace{0.2em}\textit{suffix}}'.
Surely, the same effect is achieved by
directly specifying the
argument `{\textit{dest}\hspace{0.2em}\textit{suffix}}'
in the first form.
However, that requires to set up a different file
for each child. With the alternative form of the command
all these files can have exactly the same content
which simplifies setting them up and maintaining them.

For example, the following file |draft.tex|
with a compilation flag |\version| as described in \secref{sec:flags}
compiles the main document as a draft:
%
\begin{center}
\begin{tabular}{l}
|\def\version{draft}|\\
|\input{childdoc.def}|\\
|\childdocforward{|\textit{main}|}|
\end{tabular}
\end{center}
%
Likewise, the following files |final|\textit{nn}|.tex|
compile the final version of the child document
|child|\textit{nn}|.tex|:
%
\begin{center}
\begin{tabular}{l}
|\def\version{final}|\\
|\input{childdoc.def}|\\
|\childdocforwardprefix{final}{child}|
\end{tabular}
\end{center}
%

Note that when several versions of a main file and/or of each child file
are to be generated, it may be convenient to set up a |Makefile| or
shell script to automatise the process.

%%%%%%%%%%%%%%%%%%%%%%%%%%%%%%%%%%%%%%%%%%%%%%%%%%%%%%%%%%%%%%%%%%%%%%%%%%%%%%%%
\subsection{Command Line Processing}
\label{sec:commandline}

The effect of redirection files can also be achieved by invoking
the \LaTeX{} compiler with a more elaborate command line.
Most conveniently this should be done as part
of a shell script or a |Makefile|.

When using \textsf{childdoc} in the main file, the following
command lines effectively perform a redirection
(note that depending on the shell being used,
backslashes may have to be doubled: `|\|' $\to$ `|\\|'):
%
\begin{center}
|... -jobname "|\textit{target}|" |\\|"|[\textit{flags}]%
|\input{childdoc.def}\childdocforward[|\textit{main}|]{|\textit{dest}|}"|
\end{center}
%
Here \textit{target} is the name of the output file,
\textit{main} is the name of the main file
and \textit{dest} is the name of the main or child file to be processed
(all filenames without extensions).
The optional argument \textit{main} can be omitted
if \textit{main} matches \textit{dest}.
Optionally, compilation \textit{flags} can be defined via |\def| commands.
This command line makes the \TeX{} engine believe
it is compiling the file \textit{target}
whose content is specified as the latter parameter.
The provided code then forwards the processing to
\textit{main} or \textit{dest} as described in \secref{sec:forward}.

%%%%%%%%%%%%%%%%%%%%%%%%%%%%%%%%%%%%%%%%%%%%%%%%%%%%%%%%%%%%%%%%%%%%%%%%%%%%%%%%
\subsection{Include by Input}
\label{sec:input}

Including child documents by |\include| has some restrictions by design.
Most notably, the content of a child document always occupies
its own set of pages; pages cannot be shared between child documents.
Usually, this behaviour makes perfect sense
because each child document contain an essential part of the document.
However, in some situations it may be desirable to compose
a document from a collection of parts
without having mandatory page breaks between then.
For this case, the package
provides a mechanism to include parts
by |\input| which can also be processed individually.
However, by construction this mechanism
requires manual handling of the content to be output.

%%%%%%%%%%%%%%%%%%%%%%%%%%%%%%%%%%%%%%%%
\DescribeMacro{\ifchilddocmanual}
The main file should be prepared as usual, see \secref{sec:include}.
However, the document body must make a distinction
between processing of an individual part and of the main document, e.g.:
%
\begin{center}
\begin{tabular}{l}
|\ifchilddocmanual|\\
|\input{\childdocname}|\\
|\||else|\\
\textit{document body with }|\input{|\textit{part}|}|\\
|\||fi|
\end{tabular}
\end{center}
%
The conditional |\ifchilddocmanual| is true whenever
a part to be included by |\input| is being compiled,
and the name of the part is stored in |\childdocname|.

%%%%%%%%%%%%%%%%%%%%%%%%%%%%%%%%%%%%%%%%
\DescribeMacro{\childdocby}
Each part to be included by |\input| should start with:
%
\begin{center}
\begin{tabular}{l}
|\input{childdoc.def}|\\
|\childdocby{|\textit{main}|}|\\
\end{tabular}
\end{center}
%
The directive |\childdocby| is similar to |\childdocof|
described in \secref{sec:include},
but the subsequent selection of content must be done manually.
To that end, both |\ifchilddoc| and |\ifchilddocmanual|
will be true upon processing of a part,
and the name of the part is stored in |\childdocname|.
Note that |\jobname| will be set to the filename of the current part
so that each part receives an individual |.aux| file
that does not interfere with the |.aux| file(s) of the main document.
This behaviour can be altered by the alternative form
|\childdocby[*]{|\textit{main}|}| (with a non-empty optional argument)
which uses the |.aux| file of the main document
by setting |\jobname| to \textit{main}.

%%%%%%%%%%%%%%%%%%%%%%%%%%%%%%%%%%%%%%%%%%%%%%%%%%%%%%%%%%%%%%%%%%%%%%%%%%%%%%%%
\subsection{Driver Development}
\label{sec:driver}

The \textsf{childdoc} mechanism can also be use for the development
of definition files such as \LaTeX{} styles or classes.
This case differs from the above setup with multiple parts
included by |\include| in that no |\includeonly| should be invoked.
This can be achieved by starting the include file
(before |\ProvidesPackage|) with:
%
\begin{center}
\begin{tabular}{l}
|\input{childdoc.def}|\\
|\childdocforward{|\textit{main}|}|\\
\end{tabular}
\end{center}
%
or alternatively with:
%
\begin{center}
\begin{tabular}{l}
|\input{childdoc.def}|\\
|\childdocby{|\textit{main}|}|\\
\end{tabular}
\end{center}
%
Both forms have slightly different effects as described above.
The main file is prepared as usual, see \secref{sec:include}.

%%%%%%%%%%%%%%%%%%%%%%%%%%%%%%%%%%%%%%%%%%%%%%%%%%%%%%%%%%%%%%%%%%%%%%%%%%%%%%%%
\subsection{Legacy Detection}
\label{sec:detection}

The directive |\childdocmain| in the main file can detect
whether the complete document or merely a child is to be compiled
even without using the directive |\childdocof|.
This method is deprecated because it is less robust
and there is no compelling reason to use it;
it is merely provided for backward compatibility
and it may be removed in future versions.

If the detection mechanism is to be used,
it is mandatory to correctly specify
the filename of the main file as the argument of |\childdocmain|:
%
\begin{center}
\begin{tabular}{l}
|\input{childdoc.def}|\\
|\childdocmain{|\textit{main}|}|\\
\end{tabular}
\end{center}
%
If |\jobname| does not match the argument \textit{main} of |\childdocmain|,
it is assumed that |\jobname| points to the child file to be compiled.
When using |\childdocmain| with the main file specified as argument,
it suffices to start a child file
with just |\input{|\textit{main}|}|
without loading of the package and using |\childdocof|.
If instead all processing is done
with the appropriate \textsf{childdoc} directives,
the argument of \textit{main} of |\childdocmain| can be empty.

An alternative version of the command line processing described
in \secref{sec:commandline} using the detection mechanism reads:
%
\begin{center}
|... -jobname "|\textit{target}|" "|[\textit{flags}]%
[|\def\jobname{|\textit{dest}|}|]|\input{|\textit{main}|}"|
\end{center}

%%%%%%%%%%%%%%%%%%%%%%%%%%%%%%%%%%%%%%%%%%%%%%%%%%%%%%%%%%%%%%%%%%%%%%%%%%%%%%%%
\subsection{Manual Code}
\label{sec:manual}

In case one cannot be certain whether the definitions file |childdoc.def|
is installed on the target \TeX{} distribution
and one prefers not to ship it,
it is conceivable to paste a few relevant commands into the sources.

To that end, drop all statements |\input{childdoc.def}|
and perform the replacements as outlined below.
Instead of |\childdocmain{|\textit{main}|}| add the following code
to the top of the main file:
%
\begin{center}
\begin{tabular}{l}
|\||ifdefined\childdocname\endinput\||fi\newif\ifchilddoc|\\
|\edef\childdocname{\scantokens\expandafter{\jobname\noexpand}}|\\
|\def\childdocmain{|\textit{main}|}\||ifx\childdocmain\childdocname\||else|\\
|\childdoctrue\includeonly{\childdocname}\let\jobname\childdocmain\||fi|\\
\end{tabular}
\end{center}
%
Instead of |\childdocof{|\textit{main}|}| just include the main file
at the top of each child file:
%
\begin{center}
|\input{|\textit{main}|}|
\end{center}
%
A simple redirection |\childdocforward{|\textit{dest}|}| is achieved by:
%
\begin{center}
|\def\jobname{|\textit{dest}|}\input{\jobname}|
\end{center}
%
The redirection with prefix
|\childdocforwardprefix[|\textit{prefix}|]{|\textit{dest}|}|
is accomplished by:
%
\begin{center}
\begin{tabular}{l}
|{\edef\jobname{\scantokens\expandafter{\jobname\noexpand}}|\\
|\def\redirectjob |\textit{prefix}|#1~~~{\gdef\jobname{|\textit{dest}|#1}}|\\
|\expandafter\redirectjob\jobname~~~}\input{\jobname}|
\end{tabular}
\end{center}

In an alternative approach,
child documents can be compiled by a specific command line
without additional code or specific definitions:
%
\begin{center}
|... -jobname "|\textit{target}|" "|[\textit{flags}]%
|\includeonly{|\textit{dest}|}\input{|\textit{main}|}"|
\end{center}
%

%%%%%%%%%%%%%%%%%%%%%%%%%%%%%%%%%%%%%%%%%%%%%%%%%%%%%%%%%%%%%%%%%%%%%%%%%%%%%%%%
%%%%%%%%%%%%%%%%%%%%%%%%%%%%%%%%%%%%%%%%%%%%%%%%%%%%%%%%%%%%%%%%%%%%%%%%%%%%%%%%
\section{Information}

%%%%%%%%%%%%%%%%%%%%%%%%%%%%%%%%%%%%%%%%%%%%%%%%%%%%%%%%%%%%%%%%%%%%%%%%%%%%%%%%
\subsection{Copyright}

Copyright \copyright{} 2017--2018 Niklas Beisert

This work may be distributed and/or modified under the
conditions of the \LaTeX{} Project Public License, either version 1.3
of this license or (at your option) any later version.
The latest version of this license is in
  \url{http://www.latex-project.org/lppl.txt}
and version 1.3 or later is part of all distributions of \LaTeX{}
version 2005/12/01 or later.

This work has the LPPL maintenance status `maintained'.

The Current Maintainer of this work is Niklas Beisert.

This work consists of the files |README.txt|, |childdoc.ins| and |childdoc.dtx|
as well as the derived files |childdoc.def|, |cdocsamp.tex|
with |cdocsch1.tex|, |cdocsch2.tex|, |cdocspt3.tex|, |cdocspt4.tex|,
|cdocsdrf.tex|, |cdocsfn1.tex|, |cdocsfn2.tex|
as well as |childdoc.pdf|.

%%%%%%%%%%%%%%%%%%%%%%%%%%%%%%%%%%%%%%%%%%%%%%%%%%%%%%%%%%%%%%%%%%%%%%%%%%%%%%%%
\subsection{Files and Installation}

The package consists of the files:
%
\begin{center}
\begin{tabular}{ll}
    |README.txt|   & readme file \\
    |childdoc.ins| & installation file \\
    |childdoc.dtx| & source file \\
    |childdoc.def| & definition file \\
    |cdocsamp.tex| & sample main file \\
    |cdocsch1.tex| & sample include file \\
    |cdocsch2.tex| & sample include file \\
    |cdocspt3.tex| & sample part file \\
    |cdocspt4.tex| & sample part file \\
    |cdocsdrf.tex| & sample redirection file \\
    |cdocsfn1.tex| & sample redirection file \\
    |cdocsfn2.tex| & sample redirection file \\
    |childdoc.pdf| & manual
\end{tabular}
\end{center}
%
The distribution consists of the files
|README.txt|, |childdoc.ins| and |childdoc.dtx|.
%
\begin{itemize}
\item
Run (pdf)\LaTeX{} on |childdoc.dtx|
to compile the manual |childdoc.pdf| (this file).
\item
Run \LaTeX{} on |childdoc.ins| to create the definitions file |childdoc.def|
and the sample |cdocsamp.tex| with include files
|cdocsch1.tex|, |cdocsch2.tex|, |cdocspt3.tex|, |cdocspt4.tex|,
|cdocsdrf.tex|, |cdocsfn1.tex|, |cdocsfn2.tex|.
Then copy the file |childdoc.def| to an appropriate directory of your \LaTeX{}
distribution, e.g.\ \textit{texmf-root}|/tex/latex/childdoc|.
\end{itemize}

%%%%%%%%%%%%%%%%%%%%%%%%%%%%%%%%%%%%%%%%%%%%%%%%%%%%%%%%%%%%%%%%%%%%%%%%%%%%%%%%
\subsection{Related CTAN Packages}

There are several other packages which offer a similar functionality:
%
\begin{itemize}
\item
The packages
\href{http://ctan.org/pkg/docmute}{\textsf{docmute}},
\href{http://ctan.org/pkg/includex}{\textsf{includex}} and
\href{http://ctan.org/pkg/standalone}{\textsf{standalone}}
provide commands to include only the document body of
a child file thus allowing both files to be compiled individually.
\item
The packages \href{http://ctan.org/pkg/subdocs}{\textsf{subdocs}}
and \href{http://ctan.org/pkg/subfiles}{\textsf{subfiles}}
provide structures in which the main and child documents can be
encapsulated and allowing them to be compiled individually.
The inclusion mechanism is different from the conventional |\include|.
\item
The package \href{http://ctan.org/pkg/combine}{\textsf{combine}}
is an elaborate solution to combine several documents into one.
\end{itemize}
%
See also the CTAN topic \href{http://ctan.org/topic/subdocs}{\textsf{subdocs}}
for further related packages.
The present package differs from the above solutions in that
a document structure constructed with the conventional |\include| mechanism
just needs two extra commands at the top of every file
such that all constituent files can be compiled individually.

%%%%%%%%%%%%%%%%%%%%%%%%%%%%%%%%%%%%%%%%%%%%%%%%%%%%%%%%%%%%%%%%%%%%%%%%%%%%%%%%
%\subsection{Feature Suggestions}
%
%The following is a list of features which may be useful for future
%versions of this package:
%%
%\begin{itemize}
%\item
%\ldots
%\end{itemize}

%%%%%%%%%%%%%%%%%%%%%%%%%%%%%%%%%%%%%%%%%%%%%%%%%%%%%%%%%%%%%%%%%%%%%%%%%%%%%%%%
\subsection{Revision History}

%%%%%%%%%%%%%%%%%%%%%%%%%%%%%%%%%%%%%%%%
\paragraph{v2.0:} 2018/12/30

\begin{itemize}
\item
immediate forward processing
\item
added |\childdocby| mechanism
\item
manual restructured
\end{itemize}

%%%%%%%%%%%%%%%%%%%%%%%%%%%%%%%%%%%%%%%%
\paragraph{v1.6:} 2018/01/17

\begin{itemize}
\item
application for development of include files
\item
corrections to manual
\end{itemize}

%%%%%%%%%%%%%%%%%%%%%%%%%%%%%%%%%%%%%%%%
\paragraph{v1.5:} 2017/05/21

\begin{itemize}
\item
more complete structuring introduced
\item
|\childdocof| introduced
\item
|\childdoc| renamed to |\childdocmain|
\item
|\childredirect| renamed to |\childdocforward| and |\childdocforwardprefix|
and functionality expanded
\end{itemize}

%%%%%%%%%%%%%%%%%%%%%%%%%%%%%%%%%%%%%%%%
\paragraph{v1.0:} 2017/04/27

\begin{itemize}
\item
manual and install package
\item
first version published on CTAN
\end{itemize}

%%%%%%%%%%%%%%%%%%%%%%%%%%%%%%%%%%%%%%%%
\paragraph{v0.6:} 2017/04/26

\begin{itemize}
\item
redirection mechanism added
\end{itemize}

%%%%%%%%%%%%%%%%%%%%%%%%%%%%%%%%%%%%%%%%
\paragraph{v0.5:} 2017/04/26

\begin{itemize}
\item
functionality in definition file
\end{itemize}


%%%%%%%%%%%%%%%%%%%%%%%%%%%%%%%%%%%%%%%%%%%%%%%%%%%%%%%%%%%%%%%%%%%%%%%%%%%%%%%%
%%%%%%%%%%%%%%%%%%%%%%%%%%%%%%%%%%%%%%%%%%%%%%%%%%%%%%%%%%%%%%%%%%%%%%%%%%%%%%%%
%%%%%%%%%%%%%%%%%%%%%%%%%%%%%%%%%%%%%%%%%%%%%%%%%%%%%%%%%%%%%%%%%%%%%%%%%%%%%%%%
\appendix

\settowidth\MacroIndent{\rmfamily\scriptsize 000\ }

 \DocInput{childdoc.dtx}

\end{document}
%</driver>
% \fi
%
% %%%%%%%%%%%%%%%%%%%%%%%%%%%%%%%%%%%%%%%%%%%%%%%%%%%%%%%%%%%%%%%%%%%%%%%%%%%%%%
% %%%%%%%%%%%%%%%%%%%%%%%%%%%%%%%%%%%%%%%%%%%%%%%%%%%%%%%%%%%%%%%%%%%%%%%%%%%%%%
% \section{Sample}
%\iffalse
%<*samplemain>
%\fi
%
% The following presents a sample document
% with two chapters, two parts, a title page,
% a compile flag as well as three forwarding files to set the flag.
% It consists of eight |.tex| files:
% \begin{center}
% \begin{tabular}{ll}
% |cdocsamp.tex|&main file\\
% |cdocsch1.tex|&include file for chapter 1\\
% |cdocsch2.tex|&include file for chapter 2\\
% |cdocspt3.tex|&include file for part 3\\
% |cdocspt4.tex|&include file for part 4\\
% |cdocsdrf.tex|&forwarding file for main file in draft mode\\
% |cdocsfi1.tex|&forwarding file for final version of chapter 1\\
% |cdocsfi2.tex|&forwarding file for final version of chapter 2\\
% \end{tabular}
% \end{center}
% Each of the eight files can be compiled directly by the \LaTeX{} compiler.
%
% %%%%%%%%%%%%%%%%%%%%%%%%%%%%%%%%%%%%%%
% \paragraph{Main File.}
%
% The main file is called |cdocsamp.tex|.
%
% Load the \textsf{childdoc} definitions and
% declare the filename for the main document:
%    \begin{macrocode}
\input{childdoc.def}
\childdocmain{}
%    \end{macrocode}

% Optional override for |\version| flag:
%    \begin{macrocode}
%%\ifchilddoc\else\providecommand{\version}{draft}\fi
%    \end{macrocode}

% Define the default values for the |\version| flag
% (|final| for the main file and |draft| for childs):
%    \begin{macrocode}
\ifchilddoc
\providecommand{\version}{draft}
\else
\providecommand{\version}{final}
\fi
%    \end{macrocode}

% Load the standard document class:
%    \begin{macrocode}
\documentclass[12pt]{article}
%    \end{macrocode}

% Start the document body:
%    \begin{macrocode}
\begin{document}
%    \end{macrocode}

% Declare a title page.
% Print title, part of document being processed and version flag:
%    \begin{macrocode}
\addtocounter{page}{-1}
\begin{center}
{\LARGE\bfseries{}childdoc example\par}
\vspace{1cm}
\ifchilddoc
\ifchilddocmanual part\else chapter\fi:
`\childdocname' of `\childdocjob'\par
\else
main document: `\childdocjob'\par
\fi
version: \version\par
\end{center}
\newpage
%    \end{macrocode}

% Manually include selected file,
% otherwise process as usual:
%    \begin{macrocode}
\ifchilddocmanual
\section*{part `\childdocname'}
\input{\childdocname}
\else
%    \end{macrocode}

% Include the two chapters:
%    \begin{macrocode}
\include{cdocsch1}
\include{cdocsch2}
%    \end{macrocode}

% Include the two parts unless only chapters should be displayed:
%    \begin{macrocode}
\ifchilddoc\else
\section{part three}
\input{cdocspt3}
\section{part four}
\input{cdocspt4}
\fi
%    \end{macrocode}

% Process as usual until here:
%    \begin{macrocode}
\fi
%    \end{macrocode}

% End of document body:
%    \begin{macrocode}
\end{document}
%    \end{macrocode}
%\iffalse
%</samplemain>
%\fi
%
% %%%%%%%%%%%%%%%%%%%%%%%%%%%%%%%%%%%%%%
% \paragraph{Chapter Include Files.}
%
% The include files are called |cdocsch1.tex| and |cdocsch2.tex|.
%
%\iffalse
%<*samplechap1|samplechap2>
%\fi

% Optional override for |\version| flag:
%    \begin{macrocode}
%%\providecommand{\version}{final}
%    \end{macrocode}

% Include the main document:
%    \begin{macrocode}
\input{childdoc.def}
\childdocof{cdocsamp}
%    \end{macrocode}

%\iffalse
%</samplechap1|samplechap2>
%\fi
%
%\iffalse
%<*samplechap1>
%\fi
% Some text for chapter 1:
%    \begin{macrocode}
\section{one}
some text in chapter one
%    \end{macrocode}

%\iffalse
%</samplechap1>
%\fi
% Some text for chapter 2:
%\iffalse
%<*samplechap2>
%\fi
%    \begin{macrocode}
\section{two}
more text in chapter two
%    \end{macrocode}

%\iffalse
%</samplechap2>
%\fi
%
% %%%%%%%%%%%%%%%%%%%%%%%%%%%%%%%%%%%%%%
% \paragraph{Part Include Files.}
%
% The include files are called |cdocspt3.tex| and |cdocspt4.tex|.
%
%\iffalse
%<*samplepart3|samplepart4>
%\fi

% Optional override for |\version| flag:
%    \begin{macrocode}
%%\providecommand{\version}{final}
%    \end{macrocode}

% Include the main document:
%    \begin{macrocode}
\input{childdoc.def}
\childdocby{cdocsamp}
%    \end{macrocode}

%\iffalse
%</samplepart3|samplepart4>
%\fi
%
%\iffalse
%<*samplepart3>
%\fi
% Some text for part 3:
%    \begin{macrocode}
some text in part three
%    \end{macrocode}

%\iffalse
%</samplepart3>
%\fi
% Some text for part 4:
%\iffalse
%<*samplepart4>
%\fi
%    \begin{macrocode}
more text in part four
%    \end{macrocode}

%\iffalse
%</samplepart4>
%\fi
%
% %%%%%%%%%%%%%%%%%%%%%%%%%%%%%%%%%%%%%%
% \paragraph{Forwarding for a Complete Draft.}
%
% The following forwarding file |cdocsdrf.tex|
% compiles the main document in draft mode:
%\iffalse
%<*sampledraft>
%\fi
%    \begin{macrocode}
\def\version{draft}
\input{childdoc.def}
\childdocforward{cdocsamp}
%    \end{macrocode}

%\iffalse
%</sampledraft>
%\fi
%
% %%%%%%%%%%%%%%%%%%%%%%%%%%%%%%%%%%%%%%
% \paragraph{Forwarding for Final Version of the Chapters.}
%
% The following forwarding files |cdocsfn1.tex| and |cdocsfn2.tex|
% (with identical content)
% compile the final versions of the child documents
% |cdocsch1.tex| and |cdocsch2.tex|, respectively:
%\iffalse
%<*samplefinal>
%\fi
%    \begin{macrocode}
\def\version{final}
\input{childdoc.def}
\childdocforwardprefix[cdocsamp]{cdocsfn}{cdocsch}
%    \end{macrocode}

%\iffalse
%</samplefinal>
%\fi
%
% %%%%%%%%%%%%%%%%%%%%%%%%%%%%%%%%%%%%%%
% \paragraph{Command Line Processing.}
%
% The following three command lines generate the output files
% |cdocscld|, |cdocscl1| and |cdocscl2|
% which should be identical to
% |cdocsdrf|, |cdocsch1| and |cdocsfn2|, respectively:
% \begin{center}
% \begin{tabular}{l}
% |latex -jobname cdocscld \|\\
% |  "\def\version{draft}\input{childdoc.def}\childdocforward{cdocsamp}"|\\
% |latex -jobname cdocscl1 \|\\
% |  "\input{childdoc.def}\childdocforward[cdocsamp]{cdocsch1}"|\\
% |latex -jobname cdocscl2 \|\\
% |  "\def\version{final}\input{childdoc.def}\childdocforward{cdocsch2}"|
% \end{tabular}
% \end{center}
% Note that the trailing backslash on each first line
% merely continues the input to the second line
% (for convenient cut ant paste).
% Furthermore, the command |latex| can be replaced by any
% of its alternative versions such as |pdflatex|.
%
% %%%%%%%%%%%%%%%%%%%%%%%%%%%%%%%%%%%%%%%%%%%%%%%%%%%%%%%%%%%%%%%%%%%%%%%%%%%%%%
% %%%%%%%%%%%%%%%%%%%%%%%%%%%%%%%%%%%%%%%%%%%%%%%%%%%%%%%%%%%%%%%%%%%%%%%%%%%%%%
% \section{Implementation}
%\iffalse
%<*package>
%\fi
%
% This section describes the definitions file |childdoc.def|.

% The definitions cannot be loaded using |\usepackage| or |\RequirePackage|
% which has a mechanism to prevent loading a style file more than once.
% When loading the definitions by means of |\input|
% multiple instances have to be prevented manually:
%\iffalse
%This code needs to be before the `\ProvidesFile' directive
%which is defined at the beginning of this file.
%Therefore it is also placed there and commented out here.
%</package>
%<*discard>
%\fi
%    \begin{macrocode}
\ifdefined\childdocmain\endinput\fi
%    \end{macrocode}
%\iffalse
%</discard>
%<*package>
%\fi
%
% \macro{\ifchilddoc}
% \macro{\ifchilddocmanual}
% The conditional |\ifchilddoc| tells whether a
% child (true) or main (false) document is being compiled.
% The conditional |\ifchilddocmanual| tells whether
% the |\includeonly| mechanism is used (false) or
% the selection of child files must be performed manually (true).
% The definitions initialise to false:
%    \begin{macrocode}
\newif\ifchilddoc
\newif\ifchilddocmanual
%    \end{macrocode}

% \macro{\childdocname}
% \macro{\childdocjob}
% The macro |\childdocname| stores the name of the main document
% to be compiled. The macro |\childdocjob| stores the name of
% the document on which the \LaTeX{} compiler was originally invoked.
% The content of |\jobname| cannot be compared
% to filenames specified in the source due to different catcodes.
% The following code rescans |\jobname|, stores the result
% in |\childdocname| and saves a copy in |\childdocjob|:
%    \begin{macrocode}
\edef\childdocname{\scantokens\expandafter{\jobname\noexpand}}
\let\childdocjob\childdocname
%    \end{macrocode}

% \macro{\childdocdisable}
% The macro |\childdocdisable| prevents the main file
% from being processed more than once.
% At this stage, the main document command |\childdocmain|
% is assumed to be called once again where it should do nothing.
% Any subsequent call to it should prevent
% a secondary processing of the main document
% It overwrites the forwarding commands
% |\childdocof| and |\childdocforward|
% with empty macros to prevent further inclusions of the main document:
%    \begin{macrocode}
\newcommand{\childdocdisable}
{
  \renewcommand{\childdocmain}[1]{\renewcommand{\childdocmain}[1]{\endinput}}
  \renewcommand{\childdocof}[1]{}
  \renewcommand{\childdocby}[2][]{}
  \renewcommand{\childdocforward}[2][]{}
  \renewcommand{\childdocdisable}{}
}
%    \end{macrocode}

% \macro{\childdocmain}
% The macro |\childdocmain| is to be called at the top of the main file
% with nothing or the main filename (without extension) as argument.
% First, it breaks loops.
% If the argument is not empty and does not match |\childdocname|
% (which is set by the first inclusion of |childdoc.def|),
% |\ifchilddoc| is set to true, |\includeonly| is applied to the child file
% and |\jobname| is set to the main file
% (for proper handling of |.aux| files):
%    \begin{macrocode}
\newcommand{\childdocmain}[1]
{
  \childdocdisable\childdocmain{}
  \if?#1?\else
    \begingroup
      \def\childdoctmp{#1}
      \ifx\childdoctmp\childdocname
        \def\childdoctmp{}
      \else
        \def\childdoctmp
        {
          \childdoctrue
          \includeonly{\childdocname}
          \def\childdocjob{#1}
          \def\jobname{#1}
        }
      \fi
      \expandafter
    \endgroup
    \childdoctmp
  \fi
}
%    \end{macrocode}

% \macro{\childdocof}
% The command |\childdocof| redirects
% compilation to the main file |#1|.
%    \begin{macrocode}
\newcommand{\childdocof}[1]
{
  \childdocdisable
  \childdoctrue
  \includeonly{\childdocname}
  \def\jobname{#1}
  \def\childdocjob{#1}
  \input{#1}
}
%    \end{macrocode}

% \macro{\childdocby}
% The command |\childdocby| ....
%    \begin{macrocode}
\newcommand{\childdocby}[2][]
{
  \childdocdisable
  \childdoctrue
  \childdocmanualtrue
  \if?#1?\else
    \def\jobname{#2}
  \fi
  \def\childdocjob{#2}
  \input{#2}
  \endinput
}
%    \end{macrocode}

% \macro{\childdocforward}
% The command |\childdocforward| redirects
% compilation to the main file or
% (if the optional argument is given) a child file.
% Parameters are set as if the main file
% or a child file starting with |\childdocof| was compiled.
% Then compilation is handed over to the main file:
%    \begin{macrocode}
\newcommand{\childdocforward}[2][]
{
  \begingroup
    \if?#1?
      \def\childdoctmp
      {
        \def\childdocname{#2}
        \def\childdocjob{#2}
        \def\jobname{#2}
        \input{#2}
        \endinput
      }
    \else
      \def\childdoctmp
      {
        \childdocdisable
        \def\childdocname{#2}
        \childdoctrue
        \includeonly{#2}
        \def\childdocjob{#1}
        \def\jobname{#1}
        \input{#1}
        \endinput
      }
    \fi
    \expandafter
  \endgroup
  \childdoctmp
}
%    \end{macrocode}

% \macro{\childdocforwardprefix}
% The command |\childdocforwardprefix| redirects
% compilation to the main or a child file by means of a pattern.
% The prefix |#1| in the current filename is replaced by |#2|
% and the suffix of the current filename is kept
% (it is assumed that the filename does not contain the substring `|~~~|'
% which is used as a delimiter).
% Compilation is handed over to the new file by |\childdocforward|:
%    \begin{macrocode}
\newcommand{\childdocforwardprefix}[3][]
{
  \begingroup
    \def\childdocextract #2##1~~~{\def\childdoctmp{\childdocforward[#1]{#3##1}}}
    \expandafter\childdocextract\childdocname~~~
    \expandafter
  \endgroup
  \childdoctmp
}
%    \end{macrocode}

% \macro{\childdoc}
% The deprecated macro |\childdoc| is a legacy version of |\childdocmain|:
%    \begin{macrocode}
\newcommand{\childdoc}{\childdocmain}
%    \end{macrocode}

% \macro{\childdocredirect}
% The deprecated macro |\childdocredirect| is a legacy version
% of |\childdocforward| and |\childdocforwardprefix|:
%    \begin{macrocode}
\newcommand{\childdocredirect}[2][]
{
  \begingroup
    \if?#1?
      \def\childdoctmp{\childdocforward{#2}}
    \else
      \def\childdoctmp{\childdocforwardprefix{#1}{#2}}
    \fi
    \expandafter
  \endgroup
  \childdoctmp
}
%    \end{macrocode}

%\iffalse
%</package>
%\fi
%
\endinput
|\\
|\childdocmain{|\textit{main}|}|\\
\end{tabular}
\end{center}
%
If |\jobname| does not match the argument \textit{main} of |\childdocmain|,
it is assumed that |\jobname| points to the child file to be compiled.
When using |\childdocmain| with the main file specified as argument,
it suffices to start a child file
with just |\input{|\textit{main}|}|
without loading of the package and using |\childdocof|.
If instead all processing is done
with the appropriate \textsf{childdoc} directives,
the argument of \textit{main} of |\childdocmain| can be empty.

An alternative version of the command line processing described
in \secref{sec:commandline} using the detection mechanism reads:
%
\begin{center}
|... -jobname "|\textit{target}|" "|[\textit{flags}]%
[|\def\jobname{|\textit{dest}|}|]|\input{|\textit{main}|}"|
\end{center}

%%%%%%%%%%%%%%%%%%%%%%%%%%%%%%%%%%%%%%%%%%%%%%%%%%%%%%%%%%%%%%%%%%%%%%%%%%%%%%%%
\subsection{Manual Code}
\label{sec:manual}

In case one cannot be certain whether the definitions file |childdoc.def|
is installed on the target \TeX{} distribution
and one prefers not to ship it,
it is conceivable to paste a few relevant commands into the sources.

To that end, drop all statements |% \iffalse
%
% childdoc.dtx Copyright (C) 2017-2018 Niklas Beisert
%
% This work may be distributed and/or modified under the
% conditions of the LaTeX Project Public License, either version 1.3
% of this license or (at your option) any later version.
% The latest version of this license is in
%   http://www.latex-project.org/lppl.txt
% and version 1.3 or later is part of all distributions of LaTeX
% version 2005/12/01 or later.
%
% This work has the LPPL maintenance status `maintained'.
%
% The Current Maintainer of this work is Niklas Beisert.
%
% This work consists of the files childdoc.dtx and childdoc.ins
% and the derived files childdoc.def and cdocsamp.tex with
% cdocsch1.tex, cdocsch2.tex, cdocsdrf.tex, cdocsfn1.tex, cdocsfn2.tex.
%
%<package>\ifdefined\childdocmain\endinput\fi
%<package>\ProvidesFile{childdoc.def}[2018/12/30 v2.0 child document driver]
%<samplemain>\ProvidesFile{cdocsamp.tex}[2018/12/30 v2.0 sample for childdoc]
%<*driver>
%\ProvidesFile{childdoc.drv}[2018/12/30 v2.0 childdoc reference manual file]
\PassOptionsToClass{10pt,a4paper}{article}
\documentclass{ltxdoc}

\usepackage[margin=35mm]{geometry}
\usepackage{hyperref}
\usepackage{hyperxmp}
\usepackage[usenames]{color}

\hypersetup{colorlinks=true}
\hypersetup{pdfstartview=FitH}
\hypersetup{pdfpagemode=UseNone}
\hypersetup{pdfsource={}}
\hypersetup{pdflang={en-UK}}
\hypersetup{pdfcopyright={Copyright 2017-2018 Niklas Beisert.
  This work may be distributed and/or modified under the
  conditions of the LaTeX Project Public License, either version 1.3
  of this license or (at your option) any later version.}}
\hypersetup{pdflicenseurl={http://www.latex-project.org/lppl.txt}}
\hypersetup{pdfcontactaddress={ETH Zurich, ITP, HIT K,
  Wolfgang-Pauli-Strasse 27}}
\hypersetup{pdfcontactpostcode={8093}}
\hypersetup{pdfcontactcity={Zurich}}
\hypersetup{pdfcontactcountry={Switzerland}}
\hypersetup{pdfcontactemail={nbeisert@itp.phys.ethz.ch}}
\hypersetup{pdfcontacturl={http://people.phys.ethz.ch/\xmptilde nbeisert/}}

\newcommand{\secref}[1]{\hyperref[#1]{section \ref*{#1}}}

\parskip1ex
\parindent0pt
\let\olditemize\itemize
\def\itemize{\olditemize\parskip0pt}

\begin{document}

\title{The \textsf{childdoc} Package}
\hypersetup{pdftitle={The childdoc Package}}
\author{Niklas Beisert\\[2ex]
  Institut f\"ur Theoretische Physik\\
  Eidgen\"ossische Technische Hochschule Z\"urich\\
  Wolfgang-Pauli-Strasse 27, 8093 Z\"urich, Switzerland\\[1ex]
  \href{mailto:nbeisert@itp.phys.ethz.ch}
  {\texttt{nbeisert@itp.phys.ethz.ch}}}
\hypersetup{pdfauthor={Niklas Beisert}}
\hypersetup{pdfsubject={Manual for the LaTeX2e Package childdoc}}
\date{30 December 2018, \textsf{v2.0}}
\maketitle

\begin{abstract}\noindent
\textsf{childdoc} is a \LaTeXe{} package
that enables the direct compilation
of document sections included by |\include|
to individual files.
\end{abstract}

\begingroup
\parskip0ex
\tableofcontents
\endgroup

%%%%%%%%%%%%%%%%%%%%%%%%%%%%%%%%%%%%%%%%%%%%%%%%%%%%%%%%%%%%%%%%%%%%%%%%%%%%%%%%
%%%%%%%%%%%%%%%%%%%%%%%%%%%%%%%%%%%%%%%%%%%%%%%%%%%%%%%%%%%%%%%%%%%%%%%%%%%%%%%%
\section{Introduction}

\LaTeX{} provides a mechanism to structure a large document (such as a book)
into a main file and several child files (containing the chapters)
using the |\include| command.
This mechanism is beneficial for documents
which span hundreds of pages in order to
make the source file(s) more manageable.
Moreover, compilation can be restricted to
selected child files by means of the |\includeonly| command.
The latter feature can be used to reduce the compilation time while editing
(this was significantly more useful in the earlier days of \LaTeX{})
or to generate a smaller document which is easier to navigate.
Another application of |\includeonly| is to generate
documents consisting of selected parts of the complete document.

However, there are a few drawbacks of the plain |\include| mechanism:
\begin{itemize}
\item
The child files cannot be compiled on their own,
they can only be compiled via the main file.
A naive editing environment
(such as a text editor with an option
to have the current file processed by \LaTeX)
may require one to switch to the main file before compiling;
attempting to compile the child file produces errors.
\item
The main file must be modified (each time)
to adjust the |\includeonly| command
to the present needs. This easily leaves the main file in a messy state.
\item
The generated document will always carry the filename
of the main document. This is inconvenient if
several child files are to be compiled and
to be kept for distribution.
\end{itemize}

The present package provides a simple interface
to make child files individually compilable by \LaTeX{}.
Compiling a child file then has the same effect as compiling
the main file with an |\includeonly| command
to select the appropriate child.
Moreover the generated document will carry the name of the child
rather than the main file.
This resolves all three above issues.

This feature is meant to make the editing of books,
thesis documents and lecture notes somewhat more convenient.
However, the package can also be used efficiently for
composing a series of documents (such as exercise sheets)
which are typically distributed individually.
It then assists the author in generating the individual documents
(potentially in different versions)
as well as a document containing the collected series.
Another application is in developing style files
or other kinds of included material
where compilation of the style file could redirect
to a sample or test file.

%%%%%%%%%%%%%%%%%%%%%%%%%%%%%%%%%%%%%%%%%%%%%%%%%%%%%%%%%%%%%%%%%%%%%%%%%%%%%%%%
%%%%%%%%%%%%%%%%%%%%%%%%%%%%%%%%%%%%%%%%%%%%%%%%%%%%%%%%%%%%%%%%%%%%%%%%%%%%%%%%
\section{Usage}

First of all, the package \textsf{childdoc} is \emph{not} a standard
\LaTeXe{} |.sty| style file! Therefore it needs to be invoked in
a non-standard way.

%%%%%%%%%%%%%%%%%%%%%%%%%%%%%%%%%%%%%%%%%%%%%%%%%%%%%%%%%%%%%%%%%%%%%%%%%%%%%%%%
\subsection{Included Files}
\label{sec:include}

%%%%%%%%%%%%%%%%%%%%%%%%%%%%%%%%%%%%%%%%
\DescribeMacro{\childdocmain}
To use the package, add the commands
\begin{center}
\begin{tabular}{l}
|\input{childdoc.def}|\\
|\childdocmain{}|\\
\end{tabular}
\end{center}
at the very top of the main \LaTeX{} file,
in particular \emph{before} the |\documentclass| statement!
The argument of |\childdocmain| should be left empty
(but it must be present).

%%%%%%%%%%%%%%%%%%%%%%%%%%%%%%%%%%%%%%%%
\DescribeMacro{\childdocof}
Furthermore, add the commands
\begin{center}
\begin{tabular}{l}
|\input{childdoc.def}|\\
|\childdocof{|\textit{main}|}|\\
\end{tabular}
\end{center}
at the top of every child file \textit{child}
which is included by |\include{|\textit{child}|}|
from within the main file
(or at least for those files to be compiled individually).
The argument \textit{main} must be the filename of the main file.

There are a couple of
considerations in setting up the main and child documents:

%%%%%%%%%%%%%%%%%%%%%%%%%%%%%%%%%%%%%%%%
\paragraph{Restrictions.}

Please note the following restrictions:
\begin{itemize}
\item
|\childdocmain| must be called with one argument \textit{main}
to ensure compatibility with earlier version of the package.
It must either be empty (|\childdocmain{}|)
or precisely match the filename of the main file in which it is specified.
See \secref{sec:detection} for further information.
\item
The filename \textit{main} must be specified without the |.tex| extension.
\item
The filename \textit{main} is case sensitive
(even in case-insensitive file systems)
due to internal string comparison.
\item
The argument \textit{main} should be fully expanded, it cannot be a macro.
\item
Subdirectories and special characters should be avoided in filenames.
\item
The command |\childdocmain{|\textit{main}|}| must be followed by a whitespace.
It should not be followed immediately by another command
or by a comment mark `|%|'.
This is because the \TeX{} parser reads the token immediately following
the argument of |\childdocmain| and puts it
at the beginning of every child section;
however, a white\-space is ignored.
\end{itemize}

%%%%%%%%%%%%%%%%%%%%%%%%%%%%%%%%%%%%%%%%
\paragraph{Content of Main File.}

It is advisable to place all content in the child files included by |\include|.
Any output contained in the main file will appear in all child documents
unless suppressed manually;
it cannot be suppressed automatically by the |\includeonly| directive
and thus should normally be avoided.
A method to include some content in the main file
by means of conditional processing is described in \secref{sec:conditional}.

%%%%%%%%%%%%%%%%%%%%%%%%%%%%%%%%%%%%%%%%
\paragraph{Page Numbering.}

When only a part of the document is compiled,
the appropriate numbering of pages
(as well as other status parameters)
is determined from the |.aux| files.
The latter contain information from previous passes.
However this information needs to propagate through
all intermediate child documents.
Therefore the page numbering in child documents may well
be inconsistent until the complete document is compiled at least once.

A useful (if unconventional) way to always ensure a consistent
page numbering is to restart the numbering in each child document
and denote the pages by `\textit{child}|.|\textit{page}'
where \textit{child} represents the chapter/section number of the child file.
This can be achieved by the command
|\numberwithin{page}{|\textit{child}|}|
of the \textsf{amsmath} package
where \textit{child} can be |chapter| or |section|
depending on the chosen structuring.
Alternatively, one can modify the macro |\thepage| appropriately
and reset the counter |page| at the start of each child file.

%%%%%%%%%%%%%%%%%%%%%%%%%%%%%%%%%%%%%%%%%%%%%%%%%%%%%%%%%%%%%%%%%%%%%%%%%%%%%%%%
\subsection{Conditional Processing}
\label{sec:conditional}

The package provides a mechanism to compile different versions
of a document. To customise the versions further some conditional processing
can come in handy to distinguish which version is being compiled.
The package provides two macros to describe the compilation context:

%%%%%%%%%%%%%%%%%%%%%%%%%%%%%%%%%%%%%%%%
\DescribeMacro{\ifchilddoc}
The conditional |\ifchilddoc| distinguishes between the compilation of
child documents and the main document:
%
\begin{center}
|\ifchilddoc |\textit{child-code}| |[|\||else |\textit{main-code}]| \||fi|
\end{center}

%%%%%%%%%%%%%%%%%%%%%%%%%%%%%%%%%%%%%%%%
\DescribeMacro{\childdocname}
\DescribeMacro{\childdocjob}
The macro |\childdocname| contains the filename (without extension)
of the main or child file being processed.
Note that |\childdocjob| will always contain the name of the main file.

%%%%%%%%%%%%%%%%%%%%%%%%%%%%%%%%%%%%%%%%
\paragraph{Title Page.}

Conditional processing can be used to include a title or banner page
in the main document when proper precautions are taken.
Importantly, the code in the main file should ensure that the page counter
(as well as other status parameters which are stored in the |.aux| files)
takes the same value after the conditional processing.
Otherwise the page numbers may take divergent values
depending on which part is compiled.

For example, a title page could be declared by:
%
\begin{center}
\begin{tabular}{l}
|\ifchilddoc\||else|\\
|\addtocounter{page}{-1}|\\
\textit{code for title page}\\
|\newpage|\\
|\||fi|
\end{tabular}
\end{center}
%
A banner page for the child documents can be generated by:
%
\begin{center}
\begin{tabular}{l}
|\ifchilddoc|\\
|\addtocounter{page}{-1}|\\
\textit{code for banner page}\\
|\newpage|\\
|\||fi|
\end{tabular}
\end{center}
%
Here one could write a message such as:
\begin{center}
|This is the part \childdocname{} of \childdocjob{}.|
\end{center}

%%%%%%%%%%%%%%%%%%%%%%%%%%%%%%%%%%%%%%%%%%%%%%%%%%%%%%%%%%%%%%%%%%%%%%%%%%%%%%%%
\subsection{Flags}
\label{sec:flags}

The package makes it easy to generate different versions
of the main or child documents.
To this end compilation flags can be defined
and assigned different default values.
They will be particularly useful in conjunction
with the forwarding mechanism described in \secref{sec:forward}.

For example, it may be useful to have a flag |\version|
which can be set to |draft| or |final|.
The document source will contain some conditional code
depending on the value of |\version|.
Suppose further, the flag should default to |final| for the main file
and to |draft| for child files
which is a natural assignment for editing the document.
This is achieved by placing the following code
in the preamble of the main document
(below the |\childdocmain| directive):
%
\begin{center}
\begin{tabular}{l}
|\ifchilddoc|\\
|\providecommand{\version}{draft}|\\
|\||else|\\
|\providecommand{\version}{final}|\\
|\||fi|
\end{tabular}
\end{center}
%
The definition by |\providecommand| makes sure
that previous definitions are not overwritten.
Further statements |\providecommand{\version}{...}|
can thus be added before the above code to override it.

For the main file, one might add a line
(between |\childdocmain| and the above block)
%
\begin{center}
|%\ifchilddoc\||else\providecommand{\version}{draft}\||fi|
\end{center}
%
which can be uncommented to produce a draft version.
Likewise one can add a line to the very top of a child file
(above the |\childdocof{|\textit{main}|}| directive)
%
\begin{center}
|%\providecommand{\version}{final}|
\end{center}
%
which can be uncommented to produce the final version of this child document.

%%%%%%%%%%%%%%%%%%%%%%%%%%%%%%%%%%%%%%%%%%%%%%%%%%%%%%%%%%%%%%%%%%%%%%%%%%%%%%%%
\subsection{Forwarding}
\label{sec:forward}

Different versions of the main or child documents
using compilation flags as described in \secref{sec:flags}
can be (permanently) stored in different files
for convenient compilation, viewing and distribution.
To this end, the package defines a command
to pass on compilation to a different file:

%%%%%%%%%%%%%%%%%%%%%%%%%%%%%%%%%%%%%%%%
\DescribeMacro{\childdocforward}
The command |\childdocforward| redirects processing to
another source file:
%
\begin{center}
\begin{tabular}{l}
|\input{childdoc.def}|\\
|\childdocforward[|\textit{main}|]{|\textit{dest}|}|\\
\end{tabular}
\end{center}
%
The argument \textit{dest} is the destination file
(without extension).
It should be the main file or one of the child files.
Note that further \textsf{childdoc} directives
such as |\childdocof| and |\childdocforward|
in the indicated file will be processed in this form.
The optional argument \textit{main}
passes on directly to the main file \textit{main}
while pretending to compile the child \textit{dest}.
This form behaves as if \textit{dest}
issues |\childdocof{|\textit{main}|}| right away,
and no further \textsf{childdoc} directives will be processed.

%%%%%%%%%%%%%%%%%%%%%%%%%%%%%%%%%%%%%%%%
\DescribeMacro{\...prefix}
In the alternative form |\childdocforwardprefix|,
%
\begin{center}
\begin{tabular}{l}
|\input{childdoc.def}|\\
|\childdocforwardprefix[|\textit{main}|]{|\textit{prefix}|}{|\textit{dest}|}|
\end{tabular}
\end{center}
%
the destination file is determined by a pattern
depending on the current file:
To make this work, the current file must be called
`{\textit{prefix}\hspace{0.2em}\textit{suffix}}'
with \textit{prefix} matching precisely the argument.
Processing is then passed on to the file
`{\textit{dest}\hspace{0.2em}\textit{suffix}}'.
Surely, the same effect is achieved by
directly specifying the
argument `{\textit{dest}\hspace{0.2em}\textit{suffix}}'
in the first form.
However, that requires to set up a different file
for each child. With the alternative form of the command
all these files can have exactly the same content
which simplifies setting them up and maintaining them.

For example, the following file |draft.tex|
with a compilation flag |\version| as described in \secref{sec:flags}
compiles the main document as a draft:
%
\begin{center}
\begin{tabular}{l}
|\def\version{draft}|\\
|\input{childdoc.def}|\\
|\childdocforward{|\textit{main}|}|
\end{tabular}
\end{center}
%
Likewise, the following files |final|\textit{nn}|.tex|
compile the final version of the child document
|child|\textit{nn}|.tex|:
%
\begin{center}
\begin{tabular}{l}
|\def\version{final}|\\
|\input{childdoc.def}|\\
|\childdocforwardprefix{final}{child}|
\end{tabular}
\end{center}
%

Note that when several versions of a main file and/or of each child file
are to be generated, it may be convenient to set up a |Makefile| or
shell script to automatise the process.

%%%%%%%%%%%%%%%%%%%%%%%%%%%%%%%%%%%%%%%%%%%%%%%%%%%%%%%%%%%%%%%%%%%%%%%%%%%%%%%%
\subsection{Command Line Processing}
\label{sec:commandline}

The effect of redirection files can also be achieved by invoking
the \LaTeX{} compiler with a more elaborate command line.
Most conveniently this should be done as part
of a shell script or a |Makefile|.

When using \textsf{childdoc} in the main file, the following
command lines effectively perform a redirection
(note that depending on the shell being used,
backslashes may have to be doubled: `|\|' $\to$ `|\\|'):
%
\begin{center}
|... -jobname "|\textit{target}|" |\\|"|[\textit{flags}]%
|\input{childdoc.def}\childdocforward[|\textit{main}|]{|\textit{dest}|}"|
\end{center}
%
Here \textit{target} is the name of the output file,
\textit{main} is the name of the main file
and \textit{dest} is the name of the main or child file to be processed
(all filenames without extensions).
The optional argument \textit{main} can be omitted
if \textit{main} matches \textit{dest}.
Optionally, compilation \textit{flags} can be defined via |\def| commands.
This command line makes the \TeX{} engine believe
it is compiling the file \textit{target}
whose content is specified as the latter parameter.
The provided code then forwards the processing to
\textit{main} or \textit{dest} as described in \secref{sec:forward}.

%%%%%%%%%%%%%%%%%%%%%%%%%%%%%%%%%%%%%%%%%%%%%%%%%%%%%%%%%%%%%%%%%%%%%%%%%%%%%%%%
\subsection{Include by Input}
\label{sec:input}

Including child documents by |\include| has some restrictions by design.
Most notably, the content of a child document always occupies
its own set of pages; pages cannot be shared between child documents.
Usually, this behaviour makes perfect sense
because each child document contain an essential part of the document.
However, in some situations it may be desirable to compose
a document from a collection of parts
without having mandatory page breaks between then.
For this case, the package
provides a mechanism to include parts
by |\input| which can also be processed individually.
However, by construction this mechanism
requires manual handling of the content to be output.

%%%%%%%%%%%%%%%%%%%%%%%%%%%%%%%%%%%%%%%%
\DescribeMacro{\ifchilddocmanual}
The main file should be prepared as usual, see \secref{sec:include}.
However, the document body must make a distinction
between processing of an individual part and of the main document, e.g.:
%
\begin{center}
\begin{tabular}{l}
|\ifchilddocmanual|\\
|\input{\childdocname}|\\
|\||else|\\
\textit{document body with }|\input{|\textit{part}|}|\\
|\||fi|
\end{tabular}
\end{center}
%
The conditional |\ifchilddocmanual| is true whenever
a part to be included by |\input| is being compiled,
and the name of the part is stored in |\childdocname|.

%%%%%%%%%%%%%%%%%%%%%%%%%%%%%%%%%%%%%%%%
\DescribeMacro{\childdocby}
Each part to be included by |\input| should start with:
%
\begin{center}
\begin{tabular}{l}
|\input{childdoc.def}|\\
|\childdocby{|\textit{main}|}|\\
\end{tabular}
\end{center}
%
The directive |\childdocby| is similar to |\childdocof|
described in \secref{sec:include},
but the subsequent selection of content must be done manually.
To that end, both |\ifchilddoc| and |\ifchilddocmanual|
will be true upon processing of a part,
and the name of the part is stored in |\childdocname|.
Note that |\jobname| will be set to the filename of the current part
so that each part receives an individual |.aux| file
that does not interfere with the |.aux| file(s) of the main document.
This behaviour can be altered by the alternative form
|\childdocby[*]{|\textit{main}|}| (with a non-empty optional argument)
which uses the |.aux| file of the main document
by setting |\jobname| to \textit{main}.

%%%%%%%%%%%%%%%%%%%%%%%%%%%%%%%%%%%%%%%%%%%%%%%%%%%%%%%%%%%%%%%%%%%%%%%%%%%%%%%%
\subsection{Driver Development}
\label{sec:driver}

The \textsf{childdoc} mechanism can also be use for the development
of definition files such as \LaTeX{} styles or classes.
This case differs from the above setup with multiple parts
included by |\include| in that no |\includeonly| should be invoked.
This can be achieved by starting the include file
(before |\ProvidesPackage|) with:
%
\begin{center}
\begin{tabular}{l}
|\input{childdoc.def}|\\
|\childdocforward{|\textit{main}|}|\\
\end{tabular}
\end{center}
%
or alternatively with:
%
\begin{center}
\begin{tabular}{l}
|\input{childdoc.def}|\\
|\childdocby{|\textit{main}|}|\\
\end{tabular}
\end{center}
%
Both forms have slightly different effects as described above.
The main file is prepared as usual, see \secref{sec:include}.

%%%%%%%%%%%%%%%%%%%%%%%%%%%%%%%%%%%%%%%%%%%%%%%%%%%%%%%%%%%%%%%%%%%%%%%%%%%%%%%%
\subsection{Legacy Detection}
\label{sec:detection}

The directive |\childdocmain| in the main file can detect
whether the complete document or merely a child is to be compiled
even without using the directive |\childdocof|.
This method is deprecated because it is less robust
and there is no compelling reason to use it;
it is merely provided for backward compatibility
and it may be removed in future versions.

If the detection mechanism is to be used,
it is mandatory to correctly specify
the filename of the main file as the argument of |\childdocmain|:
%
\begin{center}
\begin{tabular}{l}
|\input{childdoc.def}|\\
|\childdocmain{|\textit{main}|}|\\
\end{tabular}
\end{center}
%
If |\jobname| does not match the argument \textit{main} of |\childdocmain|,
it is assumed that |\jobname| points to the child file to be compiled.
When using |\childdocmain| with the main file specified as argument,
it suffices to start a child file
with just |\input{|\textit{main}|}|
without loading of the package and using |\childdocof|.
If instead all processing is done
with the appropriate \textsf{childdoc} directives,
the argument of \textit{main} of |\childdocmain| can be empty.

An alternative version of the command line processing described
in \secref{sec:commandline} using the detection mechanism reads:
%
\begin{center}
|... -jobname "|\textit{target}|" "|[\textit{flags}]%
[|\def\jobname{|\textit{dest}|}|]|\input{|\textit{main}|}"|
\end{center}

%%%%%%%%%%%%%%%%%%%%%%%%%%%%%%%%%%%%%%%%%%%%%%%%%%%%%%%%%%%%%%%%%%%%%%%%%%%%%%%%
\subsection{Manual Code}
\label{sec:manual}

In case one cannot be certain whether the definitions file |childdoc.def|
is installed on the target \TeX{} distribution
and one prefers not to ship it,
it is conceivable to paste a few relevant commands into the sources.

To that end, drop all statements |\input{childdoc.def}|
and perform the replacements as outlined below.
Instead of |\childdocmain{|\textit{main}|}| add the following code
to the top of the main file:
%
\begin{center}
\begin{tabular}{l}
|\||ifdefined\childdocname\endinput\||fi\newif\ifchilddoc|\\
|\edef\childdocname{\scantokens\expandafter{\jobname\noexpand}}|\\
|\def\childdocmain{|\textit{main}|}\||ifx\childdocmain\childdocname\||else|\\
|\childdoctrue\includeonly{\childdocname}\let\jobname\childdocmain\||fi|\\
\end{tabular}
\end{center}
%
Instead of |\childdocof{|\textit{main}|}| just include the main file
at the top of each child file:
%
\begin{center}
|\input{|\textit{main}|}|
\end{center}
%
A simple redirection |\childdocforward{|\textit{dest}|}| is achieved by:
%
\begin{center}
|\def\jobname{|\textit{dest}|}\input{\jobname}|
\end{center}
%
The redirection with prefix
|\childdocforwardprefix[|\textit{prefix}|]{|\textit{dest}|}|
is accomplished by:
%
\begin{center}
\begin{tabular}{l}
|{\edef\jobname{\scantokens\expandafter{\jobname\noexpand}}|\\
|\def\redirectjob |\textit{prefix}|#1~~~{\gdef\jobname{|\textit{dest}|#1}}|\\
|\expandafter\redirectjob\jobname~~~}\input{\jobname}|
\end{tabular}
\end{center}

In an alternative approach,
child documents can be compiled by a specific command line
without additional code or specific definitions:
%
\begin{center}
|... -jobname "|\textit{target}|" "|[\textit{flags}]%
|\includeonly{|\textit{dest}|}\input{|\textit{main}|}"|
\end{center}
%

%%%%%%%%%%%%%%%%%%%%%%%%%%%%%%%%%%%%%%%%%%%%%%%%%%%%%%%%%%%%%%%%%%%%%%%%%%%%%%%%
%%%%%%%%%%%%%%%%%%%%%%%%%%%%%%%%%%%%%%%%%%%%%%%%%%%%%%%%%%%%%%%%%%%%%%%%%%%%%%%%
\section{Information}

%%%%%%%%%%%%%%%%%%%%%%%%%%%%%%%%%%%%%%%%%%%%%%%%%%%%%%%%%%%%%%%%%%%%%%%%%%%%%%%%
\subsection{Copyright}

Copyright \copyright{} 2017--2018 Niklas Beisert

This work may be distributed and/or modified under the
conditions of the \LaTeX{} Project Public License, either version 1.3
of this license or (at your option) any later version.
The latest version of this license is in
  \url{http://www.latex-project.org/lppl.txt}
and version 1.3 or later is part of all distributions of \LaTeX{}
version 2005/12/01 or later.

This work has the LPPL maintenance status `maintained'.

The Current Maintainer of this work is Niklas Beisert.

This work consists of the files |README.txt|, |childdoc.ins| and |childdoc.dtx|
as well as the derived files |childdoc.def|, |cdocsamp.tex|
with |cdocsch1.tex|, |cdocsch2.tex|, |cdocspt3.tex|, |cdocspt4.tex|,
|cdocsdrf.tex|, |cdocsfn1.tex|, |cdocsfn2.tex|
as well as |childdoc.pdf|.

%%%%%%%%%%%%%%%%%%%%%%%%%%%%%%%%%%%%%%%%%%%%%%%%%%%%%%%%%%%%%%%%%%%%%%%%%%%%%%%%
\subsection{Files and Installation}

The package consists of the files:
%
\begin{center}
\begin{tabular}{ll}
    |README.txt|   & readme file \\
    |childdoc.ins| & installation file \\
    |childdoc.dtx| & source file \\
    |childdoc.def| & definition file \\
    |cdocsamp.tex| & sample main file \\
    |cdocsch1.tex| & sample include file \\
    |cdocsch2.tex| & sample include file \\
    |cdocspt3.tex| & sample part file \\
    |cdocspt4.tex| & sample part file \\
    |cdocsdrf.tex| & sample redirection file \\
    |cdocsfn1.tex| & sample redirection file \\
    |cdocsfn2.tex| & sample redirection file \\
    |childdoc.pdf| & manual
\end{tabular}
\end{center}
%
The distribution consists of the files
|README.txt|, |childdoc.ins| and |childdoc.dtx|.
%
\begin{itemize}
\item
Run (pdf)\LaTeX{} on |childdoc.dtx|
to compile the manual |childdoc.pdf| (this file).
\item
Run \LaTeX{} on |childdoc.ins| to create the definitions file |childdoc.def|
and the sample |cdocsamp.tex| with include files
|cdocsch1.tex|, |cdocsch2.tex|, |cdocspt3.tex|, |cdocspt4.tex|,
|cdocsdrf.tex|, |cdocsfn1.tex|, |cdocsfn2.tex|.
Then copy the file |childdoc.def| to an appropriate directory of your \LaTeX{}
distribution, e.g.\ \textit{texmf-root}|/tex/latex/childdoc|.
\end{itemize}

%%%%%%%%%%%%%%%%%%%%%%%%%%%%%%%%%%%%%%%%%%%%%%%%%%%%%%%%%%%%%%%%%%%%%%%%%%%%%%%%
\subsection{Related CTAN Packages}

There are several other packages which offer a similar functionality:
%
\begin{itemize}
\item
The packages
\href{http://ctan.org/pkg/docmute}{\textsf{docmute}},
\href{http://ctan.org/pkg/includex}{\textsf{includex}} and
\href{http://ctan.org/pkg/standalone}{\textsf{standalone}}
provide commands to include only the document body of
a child file thus allowing both files to be compiled individually.
\item
The packages \href{http://ctan.org/pkg/subdocs}{\textsf{subdocs}}
and \href{http://ctan.org/pkg/subfiles}{\textsf{subfiles}}
provide structures in which the main and child documents can be
encapsulated and allowing them to be compiled individually.
The inclusion mechanism is different from the conventional |\include|.
\item
The package \href{http://ctan.org/pkg/combine}{\textsf{combine}}
is an elaborate solution to combine several documents into one.
\end{itemize}
%
See also the CTAN topic \href{http://ctan.org/topic/subdocs}{\textsf{subdocs}}
for further related packages.
The present package differs from the above solutions in that
a document structure constructed with the conventional |\include| mechanism
just needs two extra commands at the top of every file
such that all constituent files can be compiled individually.

%%%%%%%%%%%%%%%%%%%%%%%%%%%%%%%%%%%%%%%%%%%%%%%%%%%%%%%%%%%%%%%%%%%%%%%%%%%%%%%%
%\subsection{Feature Suggestions}
%
%The following is a list of features which may be useful for future
%versions of this package:
%%
%\begin{itemize}
%\item
%\ldots
%\end{itemize}

%%%%%%%%%%%%%%%%%%%%%%%%%%%%%%%%%%%%%%%%%%%%%%%%%%%%%%%%%%%%%%%%%%%%%%%%%%%%%%%%
\subsection{Revision History}

%%%%%%%%%%%%%%%%%%%%%%%%%%%%%%%%%%%%%%%%
\paragraph{v2.0:} 2018/12/30

\begin{itemize}
\item
immediate forward processing
\item
added |\childdocby| mechanism
\item
manual restructured
\end{itemize}

%%%%%%%%%%%%%%%%%%%%%%%%%%%%%%%%%%%%%%%%
\paragraph{v1.6:} 2018/01/17

\begin{itemize}
\item
application for development of include files
\item
corrections to manual
\end{itemize}

%%%%%%%%%%%%%%%%%%%%%%%%%%%%%%%%%%%%%%%%
\paragraph{v1.5:} 2017/05/21

\begin{itemize}
\item
more complete structuring introduced
\item
|\childdocof| introduced
\item
|\childdoc| renamed to |\childdocmain|
\item
|\childredirect| renamed to |\childdocforward| and |\childdocforwardprefix|
and functionality expanded
\end{itemize}

%%%%%%%%%%%%%%%%%%%%%%%%%%%%%%%%%%%%%%%%
\paragraph{v1.0:} 2017/04/27

\begin{itemize}
\item
manual and install package
\item
first version published on CTAN
\end{itemize}

%%%%%%%%%%%%%%%%%%%%%%%%%%%%%%%%%%%%%%%%
\paragraph{v0.6:} 2017/04/26

\begin{itemize}
\item
redirection mechanism added
\end{itemize}

%%%%%%%%%%%%%%%%%%%%%%%%%%%%%%%%%%%%%%%%
\paragraph{v0.5:} 2017/04/26

\begin{itemize}
\item
functionality in definition file
\end{itemize}


%%%%%%%%%%%%%%%%%%%%%%%%%%%%%%%%%%%%%%%%%%%%%%%%%%%%%%%%%%%%%%%%%%%%%%%%%%%%%%%%
%%%%%%%%%%%%%%%%%%%%%%%%%%%%%%%%%%%%%%%%%%%%%%%%%%%%%%%%%%%%%%%%%%%%%%%%%%%%%%%%
%%%%%%%%%%%%%%%%%%%%%%%%%%%%%%%%%%%%%%%%%%%%%%%%%%%%%%%%%%%%%%%%%%%%%%%%%%%%%%%%
\appendix

\settowidth\MacroIndent{\rmfamily\scriptsize 000\ }

 \DocInput{childdoc.dtx}

\end{document}
%</driver>
% \fi
%
% %%%%%%%%%%%%%%%%%%%%%%%%%%%%%%%%%%%%%%%%%%%%%%%%%%%%%%%%%%%%%%%%%%%%%%%%%%%%%%
% %%%%%%%%%%%%%%%%%%%%%%%%%%%%%%%%%%%%%%%%%%%%%%%%%%%%%%%%%%%%%%%%%%%%%%%%%%%%%%
% \section{Sample}
%\iffalse
%<*samplemain>
%\fi
%
% The following presents a sample document
% with two chapters, two parts, a title page,
% a compile flag as well as three forwarding files to set the flag.
% It consists of eight |.tex| files:
% \begin{center}
% \begin{tabular}{ll}
% |cdocsamp.tex|&main file\\
% |cdocsch1.tex|&include file for chapter 1\\
% |cdocsch2.tex|&include file for chapter 2\\
% |cdocspt3.tex|&include file for part 3\\
% |cdocspt4.tex|&include file for part 4\\
% |cdocsdrf.tex|&forwarding file for main file in draft mode\\
% |cdocsfi1.tex|&forwarding file for final version of chapter 1\\
% |cdocsfi2.tex|&forwarding file for final version of chapter 2\\
% \end{tabular}
% \end{center}
% Each of the eight files can be compiled directly by the \LaTeX{} compiler.
%
% %%%%%%%%%%%%%%%%%%%%%%%%%%%%%%%%%%%%%%
% \paragraph{Main File.}
%
% The main file is called |cdocsamp.tex|.
%
% Load the \textsf{childdoc} definitions and
% declare the filename for the main document:
%    \begin{macrocode}
\input{childdoc.def}
\childdocmain{}
%    \end{macrocode}

% Optional override for |\version| flag:
%    \begin{macrocode}
%%\ifchilddoc\else\providecommand{\version}{draft}\fi
%    \end{macrocode}

% Define the default values for the |\version| flag
% (|final| for the main file and |draft| for childs):
%    \begin{macrocode}
\ifchilddoc
\providecommand{\version}{draft}
\else
\providecommand{\version}{final}
\fi
%    \end{macrocode}

% Load the standard document class:
%    \begin{macrocode}
\documentclass[12pt]{article}
%    \end{macrocode}

% Start the document body:
%    \begin{macrocode}
\begin{document}
%    \end{macrocode}

% Declare a title page.
% Print title, part of document being processed and version flag:
%    \begin{macrocode}
\addtocounter{page}{-1}
\begin{center}
{\LARGE\bfseries{}childdoc example\par}
\vspace{1cm}
\ifchilddoc
\ifchilddocmanual part\else chapter\fi:
`\childdocname' of `\childdocjob'\par
\else
main document: `\childdocjob'\par
\fi
version: \version\par
\end{center}
\newpage
%    \end{macrocode}

% Manually include selected file,
% otherwise process as usual:
%    \begin{macrocode}
\ifchilddocmanual
\section*{part `\childdocname'}
\input{\childdocname}
\else
%    \end{macrocode}

% Include the two chapters:
%    \begin{macrocode}
\include{cdocsch1}
\include{cdocsch2}
%    \end{macrocode}

% Include the two parts unless only chapters should be displayed:
%    \begin{macrocode}
\ifchilddoc\else
\section{part three}
\input{cdocspt3}
\section{part four}
\input{cdocspt4}
\fi
%    \end{macrocode}

% Process as usual until here:
%    \begin{macrocode}
\fi
%    \end{macrocode}

% End of document body:
%    \begin{macrocode}
\end{document}
%    \end{macrocode}
%\iffalse
%</samplemain>
%\fi
%
% %%%%%%%%%%%%%%%%%%%%%%%%%%%%%%%%%%%%%%
% \paragraph{Chapter Include Files.}
%
% The include files are called |cdocsch1.tex| and |cdocsch2.tex|.
%
%\iffalse
%<*samplechap1|samplechap2>
%\fi

% Optional override for |\version| flag:
%    \begin{macrocode}
%%\providecommand{\version}{final}
%    \end{macrocode}

% Include the main document:
%    \begin{macrocode}
\input{childdoc.def}
\childdocof{cdocsamp}
%    \end{macrocode}

%\iffalse
%</samplechap1|samplechap2>
%\fi
%
%\iffalse
%<*samplechap1>
%\fi
% Some text for chapter 1:
%    \begin{macrocode}
\section{one}
some text in chapter one
%    \end{macrocode}

%\iffalse
%</samplechap1>
%\fi
% Some text for chapter 2:
%\iffalse
%<*samplechap2>
%\fi
%    \begin{macrocode}
\section{two}
more text in chapter two
%    \end{macrocode}

%\iffalse
%</samplechap2>
%\fi
%
% %%%%%%%%%%%%%%%%%%%%%%%%%%%%%%%%%%%%%%
% \paragraph{Part Include Files.}
%
% The include files are called |cdocspt3.tex| and |cdocspt4.tex|.
%
%\iffalse
%<*samplepart3|samplepart4>
%\fi

% Optional override for |\version| flag:
%    \begin{macrocode}
%%\providecommand{\version}{final}
%    \end{macrocode}

% Include the main document:
%    \begin{macrocode}
\input{childdoc.def}
\childdocby{cdocsamp}
%    \end{macrocode}

%\iffalse
%</samplepart3|samplepart4>
%\fi
%
%\iffalse
%<*samplepart3>
%\fi
% Some text for part 3:
%    \begin{macrocode}
some text in part three
%    \end{macrocode}

%\iffalse
%</samplepart3>
%\fi
% Some text for part 4:
%\iffalse
%<*samplepart4>
%\fi
%    \begin{macrocode}
more text in part four
%    \end{macrocode}

%\iffalse
%</samplepart4>
%\fi
%
% %%%%%%%%%%%%%%%%%%%%%%%%%%%%%%%%%%%%%%
% \paragraph{Forwarding for a Complete Draft.}
%
% The following forwarding file |cdocsdrf.tex|
% compiles the main document in draft mode:
%\iffalse
%<*sampledraft>
%\fi
%    \begin{macrocode}
\def\version{draft}
\input{childdoc.def}
\childdocforward{cdocsamp}
%    \end{macrocode}

%\iffalse
%</sampledraft>
%\fi
%
% %%%%%%%%%%%%%%%%%%%%%%%%%%%%%%%%%%%%%%
% \paragraph{Forwarding for Final Version of the Chapters.}
%
% The following forwarding files |cdocsfn1.tex| and |cdocsfn2.tex|
% (with identical content)
% compile the final versions of the child documents
% |cdocsch1.tex| and |cdocsch2.tex|, respectively:
%\iffalse
%<*samplefinal>
%\fi
%    \begin{macrocode}
\def\version{final}
\input{childdoc.def}
\childdocforwardprefix[cdocsamp]{cdocsfn}{cdocsch}
%    \end{macrocode}

%\iffalse
%</samplefinal>
%\fi
%
% %%%%%%%%%%%%%%%%%%%%%%%%%%%%%%%%%%%%%%
% \paragraph{Command Line Processing.}
%
% The following three command lines generate the output files
% |cdocscld|, |cdocscl1| and |cdocscl2|
% which should be identical to
% |cdocsdrf|, |cdocsch1| and |cdocsfn2|, respectively:
% \begin{center}
% \begin{tabular}{l}
% |latex -jobname cdocscld \|\\
% |  "\def\version{draft}\input{childdoc.def}\childdocforward{cdocsamp}"|\\
% |latex -jobname cdocscl1 \|\\
% |  "\input{childdoc.def}\childdocforward[cdocsamp]{cdocsch1}"|\\
% |latex -jobname cdocscl2 \|\\
% |  "\def\version{final}\input{childdoc.def}\childdocforward{cdocsch2}"|
% \end{tabular}
% \end{center}
% Note that the trailing backslash on each first line
% merely continues the input to the second line
% (for convenient cut ant paste).
% Furthermore, the command |latex| can be replaced by any
% of its alternative versions such as |pdflatex|.
%
% %%%%%%%%%%%%%%%%%%%%%%%%%%%%%%%%%%%%%%%%%%%%%%%%%%%%%%%%%%%%%%%%%%%%%%%%%%%%%%
% %%%%%%%%%%%%%%%%%%%%%%%%%%%%%%%%%%%%%%%%%%%%%%%%%%%%%%%%%%%%%%%%%%%%%%%%%%%%%%
% \section{Implementation}
%\iffalse
%<*package>
%\fi
%
% This section describes the definitions file |childdoc.def|.

% The definitions cannot be loaded using |\usepackage| or |\RequirePackage|
% which has a mechanism to prevent loading a style file more than once.
% When loading the definitions by means of |\input|
% multiple instances have to be prevented manually:
%\iffalse
%This code needs to be before the `\ProvidesFile' directive
%which is defined at the beginning of this file.
%Therefore it is also placed there and commented out here.
%</package>
%<*discard>
%\fi
%    \begin{macrocode}
\ifdefined\childdocmain\endinput\fi
%    \end{macrocode}
%\iffalse
%</discard>
%<*package>
%\fi
%
% \macro{\ifchilddoc}
% \macro{\ifchilddocmanual}
% The conditional |\ifchilddoc| tells whether a
% child (true) or main (false) document is being compiled.
% The conditional |\ifchilddocmanual| tells whether
% the |\includeonly| mechanism is used (false) or
% the selection of child files must be performed manually (true).
% The definitions initialise to false:
%    \begin{macrocode}
\newif\ifchilddoc
\newif\ifchilddocmanual
%    \end{macrocode}

% \macro{\childdocname}
% \macro{\childdocjob}
% The macro |\childdocname| stores the name of the main document
% to be compiled. The macro |\childdocjob| stores the name of
% the document on which the \LaTeX{} compiler was originally invoked.
% The content of |\jobname| cannot be compared
% to filenames specified in the source due to different catcodes.
% The following code rescans |\jobname|, stores the result
% in |\childdocname| and saves a copy in |\childdocjob|:
%    \begin{macrocode}
\edef\childdocname{\scantokens\expandafter{\jobname\noexpand}}
\let\childdocjob\childdocname
%    \end{macrocode}

% \macro{\childdocdisable}
% The macro |\childdocdisable| prevents the main file
% from being processed more than once.
% At this stage, the main document command |\childdocmain|
% is assumed to be called once again where it should do nothing.
% Any subsequent call to it should prevent
% a secondary processing of the main document
% It overwrites the forwarding commands
% |\childdocof| and |\childdocforward|
% with empty macros to prevent further inclusions of the main document:
%    \begin{macrocode}
\newcommand{\childdocdisable}
{
  \renewcommand{\childdocmain}[1]{\renewcommand{\childdocmain}[1]{\endinput}}
  \renewcommand{\childdocof}[1]{}
  \renewcommand{\childdocby}[2][]{}
  \renewcommand{\childdocforward}[2][]{}
  \renewcommand{\childdocdisable}{}
}
%    \end{macrocode}

% \macro{\childdocmain}
% The macro |\childdocmain| is to be called at the top of the main file
% with nothing or the main filename (without extension) as argument.
% First, it breaks loops.
% If the argument is not empty and does not match |\childdocname|
% (which is set by the first inclusion of |childdoc.def|),
% |\ifchilddoc| is set to true, |\includeonly| is applied to the child file
% and |\jobname| is set to the main file
% (for proper handling of |.aux| files):
%    \begin{macrocode}
\newcommand{\childdocmain}[1]
{
  \childdocdisable\childdocmain{}
  \if?#1?\else
    \begingroup
      \def\childdoctmp{#1}
      \ifx\childdoctmp\childdocname
        \def\childdoctmp{}
      \else
        \def\childdoctmp
        {
          \childdoctrue
          \includeonly{\childdocname}
          \def\childdocjob{#1}
          \def\jobname{#1}
        }
      \fi
      \expandafter
    \endgroup
    \childdoctmp
  \fi
}
%    \end{macrocode}

% \macro{\childdocof}
% The command |\childdocof| redirects
% compilation to the main file |#1|.
%    \begin{macrocode}
\newcommand{\childdocof}[1]
{
  \childdocdisable
  \childdoctrue
  \includeonly{\childdocname}
  \def\jobname{#1}
  \def\childdocjob{#1}
  \input{#1}
}
%    \end{macrocode}

% \macro{\childdocby}
% The command |\childdocby| ....
%    \begin{macrocode}
\newcommand{\childdocby}[2][]
{
  \childdocdisable
  \childdoctrue
  \childdocmanualtrue
  \if?#1?\else
    \def\jobname{#2}
  \fi
  \def\childdocjob{#2}
  \input{#2}
  \endinput
}
%    \end{macrocode}

% \macro{\childdocforward}
% The command |\childdocforward| redirects
% compilation to the main file or
% (if the optional argument is given) a child file.
% Parameters are set as if the main file
% or a child file starting with |\childdocof| was compiled.
% Then compilation is handed over to the main file:
%    \begin{macrocode}
\newcommand{\childdocforward}[2][]
{
  \begingroup
    \if?#1?
      \def\childdoctmp
      {
        \def\childdocname{#2}
        \def\childdocjob{#2}
        \def\jobname{#2}
        \input{#2}
        \endinput
      }
    \else
      \def\childdoctmp
      {
        \childdocdisable
        \def\childdocname{#2}
        \childdoctrue
        \includeonly{#2}
        \def\childdocjob{#1}
        \def\jobname{#1}
        \input{#1}
        \endinput
      }
    \fi
    \expandafter
  \endgroup
  \childdoctmp
}
%    \end{macrocode}

% \macro{\childdocforwardprefix}
% The command |\childdocforwardprefix| redirects
% compilation to the main or a child file by means of a pattern.
% The prefix |#1| in the current filename is replaced by |#2|
% and the suffix of the current filename is kept
% (it is assumed that the filename does not contain the substring `|~~~|'
% which is used as a delimiter).
% Compilation is handed over to the new file by |\childdocforward|:
%    \begin{macrocode}
\newcommand{\childdocforwardprefix}[3][]
{
  \begingroup
    \def\childdocextract #2##1~~~{\def\childdoctmp{\childdocforward[#1]{#3##1}}}
    \expandafter\childdocextract\childdocname~~~
    \expandafter
  \endgroup
  \childdoctmp
}
%    \end{macrocode}

% \macro{\childdoc}
% The deprecated macro |\childdoc| is a legacy version of |\childdocmain|:
%    \begin{macrocode}
\newcommand{\childdoc}{\childdocmain}
%    \end{macrocode}

% \macro{\childdocredirect}
% The deprecated macro |\childdocredirect| is a legacy version
% of |\childdocforward| and |\childdocforwardprefix|:
%    \begin{macrocode}
\newcommand{\childdocredirect}[2][]
{
  \begingroup
    \if?#1?
      \def\childdoctmp{\childdocforward{#2}}
    \else
      \def\childdoctmp{\childdocforwardprefix{#1}{#2}}
    \fi
    \expandafter
  \endgroup
  \childdoctmp
}
%    \end{macrocode}

%\iffalse
%</package>
%\fi
%
\endinput
|
and perform the replacements as outlined below.
Instead of |\childdocmain{|\textit{main}|}| add the following code
to the top of the main file:
%
\begin{center}
\begin{tabular}{l}
|\||ifdefined\childdocname\endinput\||fi\newif\ifchilddoc|\\
|\edef\childdocname{\scantokens\expandafter{\jobname\noexpand}}|\\
|\def\childdocmain{|\textit{main}|}\||ifx\childdocmain\childdocname\||else|\\
|\childdoctrue\includeonly{\childdocname}\let\jobname\childdocmain\||fi|\\
\end{tabular}
\end{center}
%
Instead of |\childdocof{|\textit{main}|}| just include the main file
at the top of each child file:
%
\begin{center}
|\input{|\textit{main}|}|
\end{center}
%
A simple redirection |\childdocforward{|\textit{dest}|}| is achieved by:
%
\begin{center}
|\def\jobname{|\textit{dest}|}\input{\jobname}|
\end{center}
%
The redirection with prefix
|\childdocforwardprefix[|\textit{prefix}|]{|\textit{dest}|}|
is accomplished by:
%
\begin{center}
\begin{tabular}{l}
|{\edef\jobname{\scantokens\expandafter{\jobname\noexpand}}|\\
|\def\redirectjob |\textit{prefix}|#1~~~{\gdef\jobname{|\textit{dest}|#1}}|\\
|\expandafter\redirectjob\jobname~~~}\input{\jobname}|
\end{tabular}
\end{center}

In an alternative approach,
child documents can be compiled by a specific command line
without additional code or specific definitions:
%
\begin{center}
|... -jobname "|\textit{target}|" "|[\textit{flags}]%
|\includeonly{|\textit{dest}|}\input{|\textit{main}|}"|
\end{center}
%

%%%%%%%%%%%%%%%%%%%%%%%%%%%%%%%%%%%%%%%%%%%%%%%%%%%%%%%%%%%%%%%%%%%%%%%%%%%%%%%%
%%%%%%%%%%%%%%%%%%%%%%%%%%%%%%%%%%%%%%%%%%%%%%%%%%%%%%%%%%%%%%%%%%%%%%%%%%%%%%%%
\section{Information}

%%%%%%%%%%%%%%%%%%%%%%%%%%%%%%%%%%%%%%%%%%%%%%%%%%%%%%%%%%%%%%%%%%%%%%%%%%%%%%%%
\subsection{Copyright}

Copyright \copyright{} 2017--2018 Niklas Beisert

This work may be distributed and/or modified under the
conditions of the \LaTeX{} Project Public License, either version 1.3
of this license or (at your option) any later version.
The latest version of this license is in
  \url{http://www.latex-project.org/lppl.txt}
and version 1.3 or later is part of all distributions of \LaTeX{}
version 2005/12/01 or later.

This work has the LPPL maintenance status `maintained'.

The Current Maintainer of this work is Niklas Beisert.

This work consists of the files |README.txt|, |childdoc.ins| and |childdoc.dtx|
as well as the derived files |childdoc.def|, |cdocsamp.tex|
with |cdocsch1.tex|, |cdocsch2.tex|, |cdocspt3.tex|, |cdocspt4.tex|,
|cdocsdrf.tex|, |cdocsfn1.tex|, |cdocsfn2.tex|
as well as |childdoc.pdf|.

%%%%%%%%%%%%%%%%%%%%%%%%%%%%%%%%%%%%%%%%%%%%%%%%%%%%%%%%%%%%%%%%%%%%%%%%%%%%%%%%
\subsection{Files and Installation}

The package consists of the files:
%
\begin{center}
\begin{tabular}{ll}
    |README.txt|   & readme file \\
    |childdoc.ins| & installation file \\
    |childdoc.dtx| & source file \\
    |childdoc.def| & definition file \\
    |cdocsamp.tex| & sample main file \\
    |cdocsch1.tex| & sample include file \\
    |cdocsch2.tex| & sample include file \\
    |cdocspt3.tex| & sample part file \\
    |cdocspt4.tex| & sample part file \\
    |cdocsdrf.tex| & sample redirection file \\
    |cdocsfn1.tex| & sample redirection file \\
    |cdocsfn2.tex| & sample redirection file \\
    |childdoc.pdf| & manual
\end{tabular}
\end{center}
%
The distribution consists of the files
|README.txt|, |childdoc.ins| and |childdoc.dtx|.
%
\begin{itemize}
\item
Run (pdf)\LaTeX{} on |childdoc.dtx|
to compile the manual |childdoc.pdf| (this file).
\item
Run \LaTeX{} on |childdoc.ins| to create the definitions file |childdoc.def|
and the sample |cdocsamp.tex| with include files
|cdocsch1.tex|, |cdocsch2.tex|, |cdocspt3.tex|, |cdocspt4.tex|,
|cdocsdrf.tex|, |cdocsfn1.tex|, |cdocsfn2.tex|.
Then copy the file |childdoc.def| to an appropriate directory of your \LaTeX{}
distribution, e.g.\ \textit{texmf-root}|/tex/latex/childdoc|.
\end{itemize}

%%%%%%%%%%%%%%%%%%%%%%%%%%%%%%%%%%%%%%%%%%%%%%%%%%%%%%%%%%%%%%%%%%%%%%%%%%%%%%%%
\subsection{Related CTAN Packages}

There are several other packages which offer a similar functionality:
%
\begin{itemize}
\item
The packages
\href{http://ctan.org/pkg/docmute}{\textsf{docmute}},
\href{http://ctan.org/pkg/includex}{\textsf{includex}} and
\href{http://ctan.org/pkg/standalone}{\textsf{standalone}}
provide commands to include only the document body of
a child file thus allowing both files to be compiled individually.
\item
The packages \href{http://ctan.org/pkg/subdocs}{\textsf{subdocs}}
and \href{http://ctan.org/pkg/subfiles}{\textsf{subfiles}}
provide structures in which the main and child documents can be
encapsulated and allowing them to be compiled individually.
The inclusion mechanism is different from the conventional |\include|.
\item
The package \href{http://ctan.org/pkg/combine}{\textsf{combine}}
is an elaborate solution to combine several documents into one.
\end{itemize}
%
See also the CTAN topic \href{http://ctan.org/topic/subdocs}{\textsf{subdocs}}
for further related packages.
The present package differs from the above solutions in that
a document structure constructed with the conventional |\include| mechanism
just needs two extra commands at the top of every file
such that all constituent files can be compiled individually.

%%%%%%%%%%%%%%%%%%%%%%%%%%%%%%%%%%%%%%%%%%%%%%%%%%%%%%%%%%%%%%%%%%%%%%%%%%%%%%%%
%\subsection{Feature Suggestions}
%
%The following is a list of features which may be useful for future
%versions of this package:
%%
%\begin{itemize}
%\item
%\ldots
%\end{itemize}

%%%%%%%%%%%%%%%%%%%%%%%%%%%%%%%%%%%%%%%%%%%%%%%%%%%%%%%%%%%%%%%%%%%%%%%%%%%%%%%%
\subsection{Revision History}

%%%%%%%%%%%%%%%%%%%%%%%%%%%%%%%%%%%%%%%%
\paragraph{v2.0:} 2018/12/30

\begin{itemize}
\item
immediate forward processing
\item
added |\childdocby| mechanism
\item
manual restructured
\end{itemize}

%%%%%%%%%%%%%%%%%%%%%%%%%%%%%%%%%%%%%%%%
\paragraph{v1.6:} 2018/01/17

\begin{itemize}
\item
application for development of include files
\item
corrections to manual
\end{itemize}

%%%%%%%%%%%%%%%%%%%%%%%%%%%%%%%%%%%%%%%%
\paragraph{v1.5:} 2017/05/21

\begin{itemize}
\item
more complete structuring introduced
\item
|\childdocof| introduced
\item
|\childdoc| renamed to |\childdocmain|
\item
|\childredirect| renamed to |\childdocforward| and |\childdocforwardprefix|
and functionality expanded
\end{itemize}

%%%%%%%%%%%%%%%%%%%%%%%%%%%%%%%%%%%%%%%%
\paragraph{v1.0:} 2017/04/27

\begin{itemize}
\item
manual and install package
\item
first version published on CTAN
\end{itemize}

%%%%%%%%%%%%%%%%%%%%%%%%%%%%%%%%%%%%%%%%
\paragraph{v0.6:} 2017/04/26

\begin{itemize}
\item
redirection mechanism added
\end{itemize}

%%%%%%%%%%%%%%%%%%%%%%%%%%%%%%%%%%%%%%%%
\paragraph{v0.5:} 2017/04/26

\begin{itemize}
\item
functionality in definition file
\end{itemize}


%%%%%%%%%%%%%%%%%%%%%%%%%%%%%%%%%%%%%%%%%%%%%%%%%%%%%%%%%%%%%%%%%%%%%%%%%%%%%%%%
%%%%%%%%%%%%%%%%%%%%%%%%%%%%%%%%%%%%%%%%%%%%%%%%%%%%%%%%%%%%%%%%%%%%%%%%%%%%%%%%
%%%%%%%%%%%%%%%%%%%%%%%%%%%%%%%%%%%%%%%%%%%%%%%%%%%%%%%%%%%%%%%%%%%%%%%%%%%%%%%%
\appendix

\settowidth\MacroIndent{\rmfamily\scriptsize 000\ }

 \DocInput{childdoc.dtx}

\end{document}
%</driver>
% \fi
%
% %%%%%%%%%%%%%%%%%%%%%%%%%%%%%%%%%%%%%%%%%%%%%%%%%%%%%%%%%%%%%%%%%%%%%%%%%%%%%%
% %%%%%%%%%%%%%%%%%%%%%%%%%%%%%%%%%%%%%%%%%%%%%%%%%%%%%%%%%%%%%%%%%%%%%%%%%%%%%%
% \section{Sample}
%\iffalse
%<*samplemain>
%\fi
%
% The following presents a sample document
% with two chapters, two parts, a title page,
% a compile flag as well as three forwarding files to set the flag.
% It consists of eight |.tex| files:
% \begin{center}
% \begin{tabular}{ll}
% |cdocsamp.tex|&main file\\
% |cdocsch1.tex|&include file for chapter 1\\
% |cdocsch2.tex|&include file for chapter 2\\
% |cdocspt3.tex|&include file for part 3\\
% |cdocspt4.tex|&include file for part 4\\
% |cdocsdrf.tex|&forwarding file for main file in draft mode\\
% |cdocsfi1.tex|&forwarding file for final version of chapter 1\\
% |cdocsfi2.tex|&forwarding file for final version of chapter 2\\
% \end{tabular}
% \end{center}
% Each of the eight files can be compiled directly by the \LaTeX{} compiler.
%
% %%%%%%%%%%%%%%%%%%%%%%%%%%%%%%%%%%%%%%
% \paragraph{Main File.}
%
% The main file is called |cdocsamp.tex|.
%
% Load the \textsf{childdoc} definitions and
% declare the filename for the main document:
%    \begin{macrocode}
% \iffalse
%
% childdoc.dtx Copyright (C) 2017-2018 Niklas Beisert
%
% This work may be distributed and/or modified under the
% conditions of the LaTeX Project Public License, either version 1.3
% of this license or (at your option) any later version.
% The latest version of this license is in
%   http://www.latex-project.org/lppl.txt
% and version 1.3 or later is part of all distributions of LaTeX
% version 2005/12/01 or later.
%
% This work has the LPPL maintenance status `maintained'.
%
% The Current Maintainer of this work is Niklas Beisert.
%
% This work consists of the files childdoc.dtx and childdoc.ins
% and the derived files childdoc.def and cdocsamp.tex with
% cdocsch1.tex, cdocsch2.tex, cdocsdrf.tex, cdocsfn1.tex, cdocsfn2.tex.
%
%<package>\ifdefined\childdocmain\endinput\fi
%<package>\ProvidesFile{childdoc.def}[2018/12/30 v2.0 child document driver]
%<samplemain>\ProvidesFile{cdocsamp.tex}[2018/12/30 v2.0 sample for childdoc]
%<*driver>
%\ProvidesFile{childdoc.drv}[2018/12/30 v2.0 childdoc reference manual file]
\PassOptionsToClass{10pt,a4paper}{article}
\documentclass{ltxdoc}

\usepackage[margin=35mm]{geometry}
\usepackage{hyperref}
\usepackage{hyperxmp}
\usepackage[usenames]{color}

\hypersetup{colorlinks=true}
\hypersetup{pdfstartview=FitH}
\hypersetup{pdfpagemode=UseNone}
\hypersetup{pdfsource={}}
\hypersetup{pdflang={en-UK}}
\hypersetup{pdfcopyright={Copyright 2017-2018 Niklas Beisert.
  This work may be distributed and/or modified under the
  conditions of the LaTeX Project Public License, either version 1.3
  of this license or (at your option) any later version.}}
\hypersetup{pdflicenseurl={http://www.latex-project.org/lppl.txt}}
\hypersetup{pdfcontactaddress={ETH Zurich, ITP, HIT K,
  Wolfgang-Pauli-Strasse 27}}
\hypersetup{pdfcontactpostcode={8093}}
\hypersetup{pdfcontactcity={Zurich}}
\hypersetup{pdfcontactcountry={Switzerland}}
\hypersetup{pdfcontactemail={nbeisert@itp.phys.ethz.ch}}
\hypersetup{pdfcontacturl={http://people.phys.ethz.ch/\xmptilde nbeisert/}}

\newcommand{\secref}[1]{\hyperref[#1]{section \ref*{#1}}}

\parskip1ex
\parindent0pt
\let\olditemize\itemize
\def\itemize{\olditemize\parskip0pt}

\begin{document}

\title{The \textsf{childdoc} Package}
\hypersetup{pdftitle={The childdoc Package}}
\author{Niklas Beisert\\[2ex]
  Institut f\"ur Theoretische Physik\\
  Eidgen\"ossische Technische Hochschule Z\"urich\\
  Wolfgang-Pauli-Strasse 27, 8093 Z\"urich, Switzerland\\[1ex]
  \href{mailto:nbeisert@itp.phys.ethz.ch}
  {\texttt{nbeisert@itp.phys.ethz.ch}}}
\hypersetup{pdfauthor={Niklas Beisert}}
\hypersetup{pdfsubject={Manual for the LaTeX2e Package childdoc}}
\date{30 December 2018, \textsf{v2.0}}
\maketitle

\begin{abstract}\noindent
\textsf{childdoc} is a \LaTeXe{} package
that enables the direct compilation
of document sections included by |\include|
to individual files.
\end{abstract}

\begingroup
\parskip0ex
\tableofcontents
\endgroup

%%%%%%%%%%%%%%%%%%%%%%%%%%%%%%%%%%%%%%%%%%%%%%%%%%%%%%%%%%%%%%%%%%%%%%%%%%%%%%%%
%%%%%%%%%%%%%%%%%%%%%%%%%%%%%%%%%%%%%%%%%%%%%%%%%%%%%%%%%%%%%%%%%%%%%%%%%%%%%%%%
\section{Introduction}

\LaTeX{} provides a mechanism to structure a large document (such as a book)
into a main file and several child files (containing the chapters)
using the |\include| command.
This mechanism is beneficial for documents
which span hundreds of pages in order to
make the source file(s) more manageable.
Moreover, compilation can be restricted to
selected child files by means of the |\includeonly| command.
The latter feature can be used to reduce the compilation time while editing
(this was significantly more useful in the earlier days of \LaTeX{})
or to generate a smaller document which is easier to navigate.
Another application of |\includeonly| is to generate
documents consisting of selected parts of the complete document.

However, there are a few drawbacks of the plain |\include| mechanism:
\begin{itemize}
\item
The child files cannot be compiled on their own,
they can only be compiled via the main file.
A naive editing environment
(such as a text editor with an option
to have the current file processed by \LaTeX)
may require one to switch to the main file before compiling;
attempting to compile the child file produces errors.
\item
The main file must be modified (each time)
to adjust the |\includeonly| command
to the present needs. This easily leaves the main file in a messy state.
\item
The generated document will always carry the filename
of the main document. This is inconvenient if
several child files are to be compiled and
to be kept for distribution.
\end{itemize}

The present package provides a simple interface
to make child files individually compilable by \LaTeX{}.
Compiling a child file then has the same effect as compiling
the main file with an |\includeonly| command
to select the appropriate child.
Moreover the generated document will carry the name of the child
rather than the main file.
This resolves all three above issues.

This feature is meant to make the editing of books,
thesis documents and lecture notes somewhat more convenient.
However, the package can also be used efficiently for
composing a series of documents (such as exercise sheets)
which are typically distributed individually.
It then assists the author in generating the individual documents
(potentially in different versions)
as well as a document containing the collected series.
Another application is in developing style files
or other kinds of included material
where compilation of the style file could redirect
to a sample or test file.

%%%%%%%%%%%%%%%%%%%%%%%%%%%%%%%%%%%%%%%%%%%%%%%%%%%%%%%%%%%%%%%%%%%%%%%%%%%%%%%%
%%%%%%%%%%%%%%%%%%%%%%%%%%%%%%%%%%%%%%%%%%%%%%%%%%%%%%%%%%%%%%%%%%%%%%%%%%%%%%%%
\section{Usage}

First of all, the package \textsf{childdoc} is \emph{not} a standard
\LaTeXe{} |.sty| style file! Therefore it needs to be invoked in
a non-standard way.

%%%%%%%%%%%%%%%%%%%%%%%%%%%%%%%%%%%%%%%%%%%%%%%%%%%%%%%%%%%%%%%%%%%%%%%%%%%%%%%%
\subsection{Included Files}
\label{sec:include}

%%%%%%%%%%%%%%%%%%%%%%%%%%%%%%%%%%%%%%%%
\DescribeMacro{\childdocmain}
To use the package, add the commands
\begin{center}
\begin{tabular}{l}
|\input{childdoc.def}|\\
|\childdocmain{}|\\
\end{tabular}
\end{center}
at the very top of the main \LaTeX{} file,
in particular \emph{before} the |\documentclass| statement!
The argument of |\childdocmain| should be left empty
(but it must be present).

%%%%%%%%%%%%%%%%%%%%%%%%%%%%%%%%%%%%%%%%
\DescribeMacro{\childdocof}
Furthermore, add the commands
\begin{center}
\begin{tabular}{l}
|\input{childdoc.def}|\\
|\childdocof{|\textit{main}|}|\\
\end{tabular}
\end{center}
at the top of every child file \textit{child}
which is included by |\include{|\textit{child}|}|
from within the main file
(or at least for those files to be compiled individually).
The argument \textit{main} must be the filename of the main file.

There are a couple of
considerations in setting up the main and child documents:

%%%%%%%%%%%%%%%%%%%%%%%%%%%%%%%%%%%%%%%%
\paragraph{Restrictions.}

Please note the following restrictions:
\begin{itemize}
\item
|\childdocmain| must be called with one argument \textit{main}
to ensure compatibility with earlier version of the package.
It must either be empty (|\childdocmain{}|)
or precisely match the filename of the main file in which it is specified.
See \secref{sec:detection} for further information.
\item
The filename \textit{main} must be specified without the |.tex| extension.
\item
The filename \textit{main} is case sensitive
(even in case-insensitive file systems)
due to internal string comparison.
\item
The argument \textit{main} should be fully expanded, it cannot be a macro.
\item
Subdirectories and special characters should be avoided in filenames.
\item
The command |\childdocmain{|\textit{main}|}| must be followed by a whitespace.
It should not be followed immediately by another command
or by a comment mark `|%|'.
This is because the \TeX{} parser reads the token immediately following
the argument of |\childdocmain| and puts it
at the beginning of every child section;
however, a white\-space is ignored.
\end{itemize}

%%%%%%%%%%%%%%%%%%%%%%%%%%%%%%%%%%%%%%%%
\paragraph{Content of Main File.}

It is advisable to place all content in the child files included by |\include|.
Any output contained in the main file will appear in all child documents
unless suppressed manually;
it cannot be suppressed automatically by the |\includeonly| directive
and thus should normally be avoided.
A method to include some content in the main file
by means of conditional processing is described in \secref{sec:conditional}.

%%%%%%%%%%%%%%%%%%%%%%%%%%%%%%%%%%%%%%%%
\paragraph{Page Numbering.}

When only a part of the document is compiled,
the appropriate numbering of pages
(as well as other status parameters)
is determined from the |.aux| files.
The latter contain information from previous passes.
However this information needs to propagate through
all intermediate child documents.
Therefore the page numbering in child documents may well
be inconsistent until the complete document is compiled at least once.

A useful (if unconventional) way to always ensure a consistent
page numbering is to restart the numbering in each child document
and denote the pages by `\textit{child}|.|\textit{page}'
where \textit{child} represents the chapter/section number of the child file.
This can be achieved by the command
|\numberwithin{page}{|\textit{child}|}|
of the \textsf{amsmath} package
where \textit{child} can be |chapter| or |section|
depending on the chosen structuring.
Alternatively, one can modify the macro |\thepage| appropriately
and reset the counter |page| at the start of each child file.

%%%%%%%%%%%%%%%%%%%%%%%%%%%%%%%%%%%%%%%%%%%%%%%%%%%%%%%%%%%%%%%%%%%%%%%%%%%%%%%%
\subsection{Conditional Processing}
\label{sec:conditional}

The package provides a mechanism to compile different versions
of a document. To customise the versions further some conditional processing
can come in handy to distinguish which version is being compiled.
The package provides two macros to describe the compilation context:

%%%%%%%%%%%%%%%%%%%%%%%%%%%%%%%%%%%%%%%%
\DescribeMacro{\ifchilddoc}
The conditional |\ifchilddoc| distinguishes between the compilation of
child documents and the main document:
%
\begin{center}
|\ifchilddoc |\textit{child-code}| |[|\||else |\textit{main-code}]| \||fi|
\end{center}

%%%%%%%%%%%%%%%%%%%%%%%%%%%%%%%%%%%%%%%%
\DescribeMacro{\childdocname}
\DescribeMacro{\childdocjob}
The macro |\childdocname| contains the filename (without extension)
of the main or child file being processed.
Note that |\childdocjob| will always contain the name of the main file.

%%%%%%%%%%%%%%%%%%%%%%%%%%%%%%%%%%%%%%%%
\paragraph{Title Page.}

Conditional processing can be used to include a title or banner page
in the main document when proper precautions are taken.
Importantly, the code in the main file should ensure that the page counter
(as well as other status parameters which are stored in the |.aux| files)
takes the same value after the conditional processing.
Otherwise the page numbers may take divergent values
depending on which part is compiled.

For example, a title page could be declared by:
%
\begin{center}
\begin{tabular}{l}
|\ifchilddoc\||else|\\
|\addtocounter{page}{-1}|\\
\textit{code for title page}\\
|\newpage|\\
|\||fi|
\end{tabular}
\end{center}
%
A banner page for the child documents can be generated by:
%
\begin{center}
\begin{tabular}{l}
|\ifchilddoc|\\
|\addtocounter{page}{-1}|\\
\textit{code for banner page}\\
|\newpage|\\
|\||fi|
\end{tabular}
\end{center}
%
Here one could write a message such as:
\begin{center}
|This is the part \childdocname{} of \childdocjob{}.|
\end{center}

%%%%%%%%%%%%%%%%%%%%%%%%%%%%%%%%%%%%%%%%%%%%%%%%%%%%%%%%%%%%%%%%%%%%%%%%%%%%%%%%
\subsection{Flags}
\label{sec:flags}

The package makes it easy to generate different versions
of the main or child documents.
To this end compilation flags can be defined
and assigned different default values.
They will be particularly useful in conjunction
with the forwarding mechanism described in \secref{sec:forward}.

For example, it may be useful to have a flag |\version|
which can be set to |draft| or |final|.
The document source will contain some conditional code
depending on the value of |\version|.
Suppose further, the flag should default to |final| for the main file
and to |draft| for child files
which is a natural assignment for editing the document.
This is achieved by placing the following code
in the preamble of the main document
(below the |\childdocmain| directive):
%
\begin{center}
\begin{tabular}{l}
|\ifchilddoc|\\
|\providecommand{\version}{draft}|\\
|\||else|\\
|\providecommand{\version}{final}|\\
|\||fi|
\end{tabular}
\end{center}
%
The definition by |\providecommand| makes sure
that previous definitions are not overwritten.
Further statements |\providecommand{\version}{...}|
can thus be added before the above code to override it.

For the main file, one might add a line
(between |\childdocmain| and the above block)
%
\begin{center}
|%\ifchilddoc\||else\providecommand{\version}{draft}\||fi|
\end{center}
%
which can be uncommented to produce a draft version.
Likewise one can add a line to the very top of a child file
(above the |\childdocof{|\textit{main}|}| directive)
%
\begin{center}
|%\providecommand{\version}{final}|
\end{center}
%
which can be uncommented to produce the final version of this child document.

%%%%%%%%%%%%%%%%%%%%%%%%%%%%%%%%%%%%%%%%%%%%%%%%%%%%%%%%%%%%%%%%%%%%%%%%%%%%%%%%
\subsection{Forwarding}
\label{sec:forward}

Different versions of the main or child documents
using compilation flags as described in \secref{sec:flags}
can be (permanently) stored in different files
for convenient compilation, viewing and distribution.
To this end, the package defines a command
to pass on compilation to a different file:

%%%%%%%%%%%%%%%%%%%%%%%%%%%%%%%%%%%%%%%%
\DescribeMacro{\childdocforward}
The command |\childdocforward| redirects processing to
another source file:
%
\begin{center}
\begin{tabular}{l}
|\input{childdoc.def}|\\
|\childdocforward[|\textit{main}|]{|\textit{dest}|}|\\
\end{tabular}
\end{center}
%
The argument \textit{dest} is the destination file
(without extension).
It should be the main file or one of the child files.
Note that further \textsf{childdoc} directives
such as |\childdocof| and |\childdocforward|
in the indicated file will be processed in this form.
The optional argument \textit{main}
passes on directly to the main file \textit{main}
while pretending to compile the child \textit{dest}.
This form behaves as if \textit{dest}
issues |\childdocof{|\textit{main}|}| right away,
and no further \textsf{childdoc} directives will be processed.

%%%%%%%%%%%%%%%%%%%%%%%%%%%%%%%%%%%%%%%%
\DescribeMacro{\...prefix}
In the alternative form |\childdocforwardprefix|,
%
\begin{center}
\begin{tabular}{l}
|\input{childdoc.def}|\\
|\childdocforwardprefix[|\textit{main}|]{|\textit{prefix}|}{|\textit{dest}|}|
\end{tabular}
\end{center}
%
the destination file is determined by a pattern
depending on the current file:
To make this work, the current file must be called
`{\textit{prefix}\hspace{0.2em}\textit{suffix}}'
with \textit{prefix} matching precisely the argument.
Processing is then passed on to the file
`{\textit{dest}\hspace{0.2em}\textit{suffix}}'.
Surely, the same effect is achieved by
directly specifying the
argument `{\textit{dest}\hspace{0.2em}\textit{suffix}}'
in the first form.
However, that requires to set up a different file
for each child. With the alternative form of the command
all these files can have exactly the same content
which simplifies setting them up and maintaining them.

For example, the following file |draft.tex|
with a compilation flag |\version| as described in \secref{sec:flags}
compiles the main document as a draft:
%
\begin{center}
\begin{tabular}{l}
|\def\version{draft}|\\
|\input{childdoc.def}|\\
|\childdocforward{|\textit{main}|}|
\end{tabular}
\end{center}
%
Likewise, the following files |final|\textit{nn}|.tex|
compile the final version of the child document
|child|\textit{nn}|.tex|:
%
\begin{center}
\begin{tabular}{l}
|\def\version{final}|\\
|\input{childdoc.def}|\\
|\childdocforwardprefix{final}{child}|
\end{tabular}
\end{center}
%

Note that when several versions of a main file and/or of each child file
are to be generated, it may be convenient to set up a |Makefile| or
shell script to automatise the process.

%%%%%%%%%%%%%%%%%%%%%%%%%%%%%%%%%%%%%%%%%%%%%%%%%%%%%%%%%%%%%%%%%%%%%%%%%%%%%%%%
\subsection{Command Line Processing}
\label{sec:commandline}

The effect of redirection files can also be achieved by invoking
the \LaTeX{} compiler with a more elaborate command line.
Most conveniently this should be done as part
of a shell script or a |Makefile|.

When using \textsf{childdoc} in the main file, the following
command lines effectively perform a redirection
(note that depending on the shell being used,
backslashes may have to be doubled: `|\|' $\to$ `|\\|'):
%
\begin{center}
|... -jobname "|\textit{target}|" |\\|"|[\textit{flags}]%
|\input{childdoc.def}\childdocforward[|\textit{main}|]{|\textit{dest}|}"|
\end{center}
%
Here \textit{target} is the name of the output file,
\textit{main} is the name of the main file
and \textit{dest} is the name of the main or child file to be processed
(all filenames without extensions).
The optional argument \textit{main} can be omitted
if \textit{main} matches \textit{dest}.
Optionally, compilation \textit{flags} can be defined via |\def| commands.
This command line makes the \TeX{} engine believe
it is compiling the file \textit{target}
whose content is specified as the latter parameter.
The provided code then forwards the processing to
\textit{main} or \textit{dest} as described in \secref{sec:forward}.

%%%%%%%%%%%%%%%%%%%%%%%%%%%%%%%%%%%%%%%%%%%%%%%%%%%%%%%%%%%%%%%%%%%%%%%%%%%%%%%%
\subsection{Include by Input}
\label{sec:input}

Including child documents by |\include| has some restrictions by design.
Most notably, the content of a child document always occupies
its own set of pages; pages cannot be shared between child documents.
Usually, this behaviour makes perfect sense
because each child document contain an essential part of the document.
However, in some situations it may be desirable to compose
a document from a collection of parts
without having mandatory page breaks between then.
For this case, the package
provides a mechanism to include parts
by |\input| which can also be processed individually.
However, by construction this mechanism
requires manual handling of the content to be output.

%%%%%%%%%%%%%%%%%%%%%%%%%%%%%%%%%%%%%%%%
\DescribeMacro{\ifchilddocmanual}
The main file should be prepared as usual, see \secref{sec:include}.
However, the document body must make a distinction
between processing of an individual part and of the main document, e.g.:
%
\begin{center}
\begin{tabular}{l}
|\ifchilddocmanual|\\
|\input{\childdocname}|\\
|\||else|\\
\textit{document body with }|\input{|\textit{part}|}|\\
|\||fi|
\end{tabular}
\end{center}
%
The conditional |\ifchilddocmanual| is true whenever
a part to be included by |\input| is being compiled,
and the name of the part is stored in |\childdocname|.

%%%%%%%%%%%%%%%%%%%%%%%%%%%%%%%%%%%%%%%%
\DescribeMacro{\childdocby}
Each part to be included by |\input| should start with:
%
\begin{center}
\begin{tabular}{l}
|\input{childdoc.def}|\\
|\childdocby{|\textit{main}|}|\\
\end{tabular}
\end{center}
%
The directive |\childdocby| is similar to |\childdocof|
described in \secref{sec:include},
but the subsequent selection of content must be done manually.
To that end, both |\ifchilddoc| and |\ifchilddocmanual|
will be true upon processing of a part,
and the name of the part is stored in |\childdocname|.
Note that |\jobname| will be set to the filename of the current part
so that each part receives an individual |.aux| file
that does not interfere with the |.aux| file(s) of the main document.
This behaviour can be altered by the alternative form
|\childdocby[*]{|\textit{main}|}| (with a non-empty optional argument)
which uses the |.aux| file of the main document
by setting |\jobname| to \textit{main}.

%%%%%%%%%%%%%%%%%%%%%%%%%%%%%%%%%%%%%%%%%%%%%%%%%%%%%%%%%%%%%%%%%%%%%%%%%%%%%%%%
\subsection{Driver Development}
\label{sec:driver}

The \textsf{childdoc} mechanism can also be use for the development
of definition files such as \LaTeX{} styles or classes.
This case differs from the above setup with multiple parts
included by |\include| in that no |\includeonly| should be invoked.
This can be achieved by starting the include file
(before |\ProvidesPackage|) with:
%
\begin{center}
\begin{tabular}{l}
|\input{childdoc.def}|\\
|\childdocforward{|\textit{main}|}|\\
\end{tabular}
\end{center}
%
or alternatively with:
%
\begin{center}
\begin{tabular}{l}
|\input{childdoc.def}|\\
|\childdocby{|\textit{main}|}|\\
\end{tabular}
\end{center}
%
Both forms have slightly different effects as described above.
The main file is prepared as usual, see \secref{sec:include}.

%%%%%%%%%%%%%%%%%%%%%%%%%%%%%%%%%%%%%%%%%%%%%%%%%%%%%%%%%%%%%%%%%%%%%%%%%%%%%%%%
\subsection{Legacy Detection}
\label{sec:detection}

The directive |\childdocmain| in the main file can detect
whether the complete document or merely a child is to be compiled
even without using the directive |\childdocof|.
This method is deprecated because it is less robust
and there is no compelling reason to use it;
it is merely provided for backward compatibility
and it may be removed in future versions.

If the detection mechanism is to be used,
it is mandatory to correctly specify
the filename of the main file as the argument of |\childdocmain|:
%
\begin{center}
\begin{tabular}{l}
|\input{childdoc.def}|\\
|\childdocmain{|\textit{main}|}|\\
\end{tabular}
\end{center}
%
If |\jobname| does not match the argument \textit{main} of |\childdocmain|,
it is assumed that |\jobname| points to the child file to be compiled.
When using |\childdocmain| with the main file specified as argument,
it suffices to start a child file
with just |\input{|\textit{main}|}|
without loading of the package and using |\childdocof|.
If instead all processing is done
with the appropriate \textsf{childdoc} directives,
the argument of \textit{main} of |\childdocmain| can be empty.

An alternative version of the command line processing described
in \secref{sec:commandline} using the detection mechanism reads:
%
\begin{center}
|... -jobname "|\textit{target}|" "|[\textit{flags}]%
[|\def\jobname{|\textit{dest}|}|]|\input{|\textit{main}|}"|
\end{center}

%%%%%%%%%%%%%%%%%%%%%%%%%%%%%%%%%%%%%%%%%%%%%%%%%%%%%%%%%%%%%%%%%%%%%%%%%%%%%%%%
\subsection{Manual Code}
\label{sec:manual}

In case one cannot be certain whether the definitions file |childdoc.def|
is installed on the target \TeX{} distribution
and one prefers not to ship it,
it is conceivable to paste a few relevant commands into the sources.

To that end, drop all statements |\input{childdoc.def}|
and perform the replacements as outlined below.
Instead of |\childdocmain{|\textit{main}|}| add the following code
to the top of the main file:
%
\begin{center}
\begin{tabular}{l}
|\||ifdefined\childdocname\endinput\||fi\newif\ifchilddoc|\\
|\edef\childdocname{\scantokens\expandafter{\jobname\noexpand}}|\\
|\def\childdocmain{|\textit{main}|}\||ifx\childdocmain\childdocname\||else|\\
|\childdoctrue\includeonly{\childdocname}\let\jobname\childdocmain\||fi|\\
\end{tabular}
\end{center}
%
Instead of |\childdocof{|\textit{main}|}| just include the main file
at the top of each child file:
%
\begin{center}
|\input{|\textit{main}|}|
\end{center}
%
A simple redirection |\childdocforward{|\textit{dest}|}| is achieved by:
%
\begin{center}
|\def\jobname{|\textit{dest}|}\input{\jobname}|
\end{center}
%
The redirection with prefix
|\childdocforwardprefix[|\textit{prefix}|]{|\textit{dest}|}|
is accomplished by:
%
\begin{center}
\begin{tabular}{l}
|{\edef\jobname{\scantokens\expandafter{\jobname\noexpand}}|\\
|\def\redirectjob |\textit{prefix}|#1~~~{\gdef\jobname{|\textit{dest}|#1}}|\\
|\expandafter\redirectjob\jobname~~~}\input{\jobname}|
\end{tabular}
\end{center}

In an alternative approach,
child documents can be compiled by a specific command line
without additional code or specific definitions:
%
\begin{center}
|... -jobname "|\textit{target}|" "|[\textit{flags}]%
|\includeonly{|\textit{dest}|}\input{|\textit{main}|}"|
\end{center}
%

%%%%%%%%%%%%%%%%%%%%%%%%%%%%%%%%%%%%%%%%%%%%%%%%%%%%%%%%%%%%%%%%%%%%%%%%%%%%%%%%
%%%%%%%%%%%%%%%%%%%%%%%%%%%%%%%%%%%%%%%%%%%%%%%%%%%%%%%%%%%%%%%%%%%%%%%%%%%%%%%%
\section{Information}

%%%%%%%%%%%%%%%%%%%%%%%%%%%%%%%%%%%%%%%%%%%%%%%%%%%%%%%%%%%%%%%%%%%%%%%%%%%%%%%%
\subsection{Copyright}

Copyright \copyright{} 2017--2018 Niklas Beisert

This work may be distributed and/or modified under the
conditions of the \LaTeX{} Project Public License, either version 1.3
of this license or (at your option) any later version.
The latest version of this license is in
  \url{http://www.latex-project.org/lppl.txt}
and version 1.3 or later is part of all distributions of \LaTeX{}
version 2005/12/01 or later.

This work has the LPPL maintenance status `maintained'.

The Current Maintainer of this work is Niklas Beisert.

This work consists of the files |README.txt|, |childdoc.ins| and |childdoc.dtx|
as well as the derived files |childdoc.def|, |cdocsamp.tex|
with |cdocsch1.tex|, |cdocsch2.tex|, |cdocspt3.tex|, |cdocspt4.tex|,
|cdocsdrf.tex|, |cdocsfn1.tex|, |cdocsfn2.tex|
as well as |childdoc.pdf|.

%%%%%%%%%%%%%%%%%%%%%%%%%%%%%%%%%%%%%%%%%%%%%%%%%%%%%%%%%%%%%%%%%%%%%%%%%%%%%%%%
\subsection{Files and Installation}

The package consists of the files:
%
\begin{center}
\begin{tabular}{ll}
    |README.txt|   & readme file \\
    |childdoc.ins| & installation file \\
    |childdoc.dtx| & source file \\
    |childdoc.def| & definition file \\
    |cdocsamp.tex| & sample main file \\
    |cdocsch1.tex| & sample include file \\
    |cdocsch2.tex| & sample include file \\
    |cdocspt3.tex| & sample part file \\
    |cdocspt4.tex| & sample part file \\
    |cdocsdrf.tex| & sample redirection file \\
    |cdocsfn1.tex| & sample redirection file \\
    |cdocsfn2.tex| & sample redirection file \\
    |childdoc.pdf| & manual
\end{tabular}
\end{center}
%
The distribution consists of the files
|README.txt|, |childdoc.ins| and |childdoc.dtx|.
%
\begin{itemize}
\item
Run (pdf)\LaTeX{} on |childdoc.dtx|
to compile the manual |childdoc.pdf| (this file).
\item
Run \LaTeX{} on |childdoc.ins| to create the definitions file |childdoc.def|
and the sample |cdocsamp.tex| with include files
|cdocsch1.tex|, |cdocsch2.tex|, |cdocspt3.tex|, |cdocspt4.tex|,
|cdocsdrf.tex|, |cdocsfn1.tex|, |cdocsfn2.tex|.
Then copy the file |childdoc.def| to an appropriate directory of your \LaTeX{}
distribution, e.g.\ \textit{texmf-root}|/tex/latex/childdoc|.
\end{itemize}

%%%%%%%%%%%%%%%%%%%%%%%%%%%%%%%%%%%%%%%%%%%%%%%%%%%%%%%%%%%%%%%%%%%%%%%%%%%%%%%%
\subsection{Related CTAN Packages}

There are several other packages which offer a similar functionality:
%
\begin{itemize}
\item
The packages
\href{http://ctan.org/pkg/docmute}{\textsf{docmute}},
\href{http://ctan.org/pkg/includex}{\textsf{includex}} and
\href{http://ctan.org/pkg/standalone}{\textsf{standalone}}
provide commands to include only the document body of
a child file thus allowing both files to be compiled individually.
\item
The packages \href{http://ctan.org/pkg/subdocs}{\textsf{subdocs}}
and \href{http://ctan.org/pkg/subfiles}{\textsf{subfiles}}
provide structures in which the main and child documents can be
encapsulated and allowing them to be compiled individually.
The inclusion mechanism is different from the conventional |\include|.
\item
The package \href{http://ctan.org/pkg/combine}{\textsf{combine}}
is an elaborate solution to combine several documents into one.
\end{itemize}
%
See also the CTAN topic \href{http://ctan.org/topic/subdocs}{\textsf{subdocs}}
for further related packages.
The present package differs from the above solutions in that
a document structure constructed with the conventional |\include| mechanism
just needs two extra commands at the top of every file
such that all constituent files can be compiled individually.

%%%%%%%%%%%%%%%%%%%%%%%%%%%%%%%%%%%%%%%%%%%%%%%%%%%%%%%%%%%%%%%%%%%%%%%%%%%%%%%%
%\subsection{Feature Suggestions}
%
%The following is a list of features which may be useful for future
%versions of this package:
%%
%\begin{itemize}
%\item
%\ldots
%\end{itemize}

%%%%%%%%%%%%%%%%%%%%%%%%%%%%%%%%%%%%%%%%%%%%%%%%%%%%%%%%%%%%%%%%%%%%%%%%%%%%%%%%
\subsection{Revision History}

%%%%%%%%%%%%%%%%%%%%%%%%%%%%%%%%%%%%%%%%
\paragraph{v2.0:} 2018/12/30

\begin{itemize}
\item
immediate forward processing
\item
added |\childdocby| mechanism
\item
manual restructured
\end{itemize}

%%%%%%%%%%%%%%%%%%%%%%%%%%%%%%%%%%%%%%%%
\paragraph{v1.6:} 2018/01/17

\begin{itemize}
\item
application for development of include files
\item
corrections to manual
\end{itemize}

%%%%%%%%%%%%%%%%%%%%%%%%%%%%%%%%%%%%%%%%
\paragraph{v1.5:} 2017/05/21

\begin{itemize}
\item
more complete structuring introduced
\item
|\childdocof| introduced
\item
|\childdoc| renamed to |\childdocmain|
\item
|\childredirect| renamed to |\childdocforward| and |\childdocforwardprefix|
and functionality expanded
\end{itemize}

%%%%%%%%%%%%%%%%%%%%%%%%%%%%%%%%%%%%%%%%
\paragraph{v1.0:} 2017/04/27

\begin{itemize}
\item
manual and install package
\item
first version published on CTAN
\end{itemize}

%%%%%%%%%%%%%%%%%%%%%%%%%%%%%%%%%%%%%%%%
\paragraph{v0.6:} 2017/04/26

\begin{itemize}
\item
redirection mechanism added
\end{itemize}

%%%%%%%%%%%%%%%%%%%%%%%%%%%%%%%%%%%%%%%%
\paragraph{v0.5:} 2017/04/26

\begin{itemize}
\item
functionality in definition file
\end{itemize}


%%%%%%%%%%%%%%%%%%%%%%%%%%%%%%%%%%%%%%%%%%%%%%%%%%%%%%%%%%%%%%%%%%%%%%%%%%%%%%%%
%%%%%%%%%%%%%%%%%%%%%%%%%%%%%%%%%%%%%%%%%%%%%%%%%%%%%%%%%%%%%%%%%%%%%%%%%%%%%%%%
%%%%%%%%%%%%%%%%%%%%%%%%%%%%%%%%%%%%%%%%%%%%%%%%%%%%%%%%%%%%%%%%%%%%%%%%%%%%%%%%
\appendix

\settowidth\MacroIndent{\rmfamily\scriptsize 000\ }

 \DocInput{childdoc.dtx}

\end{document}
%</driver>
% \fi
%
% %%%%%%%%%%%%%%%%%%%%%%%%%%%%%%%%%%%%%%%%%%%%%%%%%%%%%%%%%%%%%%%%%%%%%%%%%%%%%%
% %%%%%%%%%%%%%%%%%%%%%%%%%%%%%%%%%%%%%%%%%%%%%%%%%%%%%%%%%%%%%%%%%%%%%%%%%%%%%%
% \section{Sample}
%\iffalse
%<*samplemain>
%\fi
%
% The following presents a sample document
% with two chapters, two parts, a title page,
% a compile flag as well as three forwarding files to set the flag.
% It consists of eight |.tex| files:
% \begin{center}
% \begin{tabular}{ll}
% |cdocsamp.tex|&main file\\
% |cdocsch1.tex|&include file for chapter 1\\
% |cdocsch2.tex|&include file for chapter 2\\
% |cdocspt3.tex|&include file for part 3\\
% |cdocspt4.tex|&include file for part 4\\
% |cdocsdrf.tex|&forwarding file for main file in draft mode\\
% |cdocsfi1.tex|&forwarding file for final version of chapter 1\\
% |cdocsfi2.tex|&forwarding file for final version of chapter 2\\
% \end{tabular}
% \end{center}
% Each of the eight files can be compiled directly by the \LaTeX{} compiler.
%
% %%%%%%%%%%%%%%%%%%%%%%%%%%%%%%%%%%%%%%
% \paragraph{Main File.}
%
% The main file is called |cdocsamp.tex|.
%
% Load the \textsf{childdoc} definitions and
% declare the filename for the main document:
%    \begin{macrocode}
\input{childdoc.def}
\childdocmain{}
%    \end{macrocode}

% Optional override for |\version| flag:
%    \begin{macrocode}
%%\ifchilddoc\else\providecommand{\version}{draft}\fi
%    \end{macrocode}

% Define the default values for the |\version| flag
% (|final| for the main file and |draft| for childs):
%    \begin{macrocode}
\ifchilddoc
\providecommand{\version}{draft}
\else
\providecommand{\version}{final}
\fi
%    \end{macrocode}

% Load the standard document class:
%    \begin{macrocode}
\documentclass[12pt]{article}
%    \end{macrocode}

% Start the document body:
%    \begin{macrocode}
\begin{document}
%    \end{macrocode}

% Declare a title page.
% Print title, part of document being processed and version flag:
%    \begin{macrocode}
\addtocounter{page}{-1}
\begin{center}
{\LARGE\bfseries{}childdoc example\par}
\vspace{1cm}
\ifchilddoc
\ifchilddocmanual part\else chapter\fi:
`\childdocname' of `\childdocjob'\par
\else
main document: `\childdocjob'\par
\fi
version: \version\par
\end{center}
\newpage
%    \end{macrocode}

% Manually include selected file,
% otherwise process as usual:
%    \begin{macrocode}
\ifchilddocmanual
\section*{part `\childdocname'}
\input{\childdocname}
\else
%    \end{macrocode}

% Include the two chapters:
%    \begin{macrocode}
\include{cdocsch1}
\include{cdocsch2}
%    \end{macrocode}

% Include the two parts unless only chapters should be displayed:
%    \begin{macrocode}
\ifchilddoc\else
\section{part three}
\input{cdocspt3}
\section{part four}
\input{cdocspt4}
\fi
%    \end{macrocode}

% Process as usual until here:
%    \begin{macrocode}
\fi
%    \end{macrocode}

% End of document body:
%    \begin{macrocode}
\end{document}
%    \end{macrocode}
%\iffalse
%</samplemain>
%\fi
%
% %%%%%%%%%%%%%%%%%%%%%%%%%%%%%%%%%%%%%%
% \paragraph{Chapter Include Files.}
%
% The include files are called |cdocsch1.tex| and |cdocsch2.tex|.
%
%\iffalse
%<*samplechap1|samplechap2>
%\fi

% Optional override for |\version| flag:
%    \begin{macrocode}
%%\providecommand{\version}{final}
%    \end{macrocode}

% Include the main document:
%    \begin{macrocode}
\input{childdoc.def}
\childdocof{cdocsamp}
%    \end{macrocode}

%\iffalse
%</samplechap1|samplechap2>
%\fi
%
%\iffalse
%<*samplechap1>
%\fi
% Some text for chapter 1:
%    \begin{macrocode}
\section{one}
some text in chapter one
%    \end{macrocode}

%\iffalse
%</samplechap1>
%\fi
% Some text for chapter 2:
%\iffalse
%<*samplechap2>
%\fi
%    \begin{macrocode}
\section{two}
more text in chapter two
%    \end{macrocode}

%\iffalse
%</samplechap2>
%\fi
%
% %%%%%%%%%%%%%%%%%%%%%%%%%%%%%%%%%%%%%%
% \paragraph{Part Include Files.}
%
% The include files are called |cdocspt3.tex| and |cdocspt4.tex|.
%
%\iffalse
%<*samplepart3|samplepart4>
%\fi

% Optional override for |\version| flag:
%    \begin{macrocode}
%%\providecommand{\version}{final}
%    \end{macrocode}

% Include the main document:
%    \begin{macrocode}
\input{childdoc.def}
\childdocby{cdocsamp}
%    \end{macrocode}

%\iffalse
%</samplepart3|samplepart4>
%\fi
%
%\iffalse
%<*samplepart3>
%\fi
% Some text for part 3:
%    \begin{macrocode}
some text in part three
%    \end{macrocode}

%\iffalse
%</samplepart3>
%\fi
% Some text for part 4:
%\iffalse
%<*samplepart4>
%\fi
%    \begin{macrocode}
more text in part four
%    \end{macrocode}

%\iffalse
%</samplepart4>
%\fi
%
% %%%%%%%%%%%%%%%%%%%%%%%%%%%%%%%%%%%%%%
% \paragraph{Forwarding for a Complete Draft.}
%
% The following forwarding file |cdocsdrf.tex|
% compiles the main document in draft mode:
%\iffalse
%<*sampledraft>
%\fi
%    \begin{macrocode}
\def\version{draft}
\input{childdoc.def}
\childdocforward{cdocsamp}
%    \end{macrocode}

%\iffalse
%</sampledraft>
%\fi
%
% %%%%%%%%%%%%%%%%%%%%%%%%%%%%%%%%%%%%%%
% \paragraph{Forwarding for Final Version of the Chapters.}
%
% The following forwarding files |cdocsfn1.tex| and |cdocsfn2.tex|
% (with identical content)
% compile the final versions of the child documents
% |cdocsch1.tex| and |cdocsch2.tex|, respectively:
%\iffalse
%<*samplefinal>
%\fi
%    \begin{macrocode}
\def\version{final}
\input{childdoc.def}
\childdocforwardprefix[cdocsamp]{cdocsfn}{cdocsch}
%    \end{macrocode}

%\iffalse
%</samplefinal>
%\fi
%
% %%%%%%%%%%%%%%%%%%%%%%%%%%%%%%%%%%%%%%
% \paragraph{Command Line Processing.}
%
% The following three command lines generate the output files
% |cdocscld|, |cdocscl1| and |cdocscl2|
% which should be identical to
% |cdocsdrf|, |cdocsch1| and |cdocsfn2|, respectively:
% \begin{center}
% \begin{tabular}{l}
% |latex -jobname cdocscld \|\\
% |  "\def\version{draft}\input{childdoc.def}\childdocforward{cdocsamp}"|\\
% |latex -jobname cdocscl1 \|\\
% |  "\input{childdoc.def}\childdocforward[cdocsamp]{cdocsch1}"|\\
% |latex -jobname cdocscl2 \|\\
% |  "\def\version{final}\input{childdoc.def}\childdocforward{cdocsch2}"|
% \end{tabular}
% \end{center}
% Note that the trailing backslash on each first line
% merely continues the input to the second line
% (for convenient cut ant paste).
% Furthermore, the command |latex| can be replaced by any
% of its alternative versions such as |pdflatex|.
%
% %%%%%%%%%%%%%%%%%%%%%%%%%%%%%%%%%%%%%%%%%%%%%%%%%%%%%%%%%%%%%%%%%%%%%%%%%%%%%%
% %%%%%%%%%%%%%%%%%%%%%%%%%%%%%%%%%%%%%%%%%%%%%%%%%%%%%%%%%%%%%%%%%%%%%%%%%%%%%%
% \section{Implementation}
%\iffalse
%<*package>
%\fi
%
% This section describes the definitions file |childdoc.def|.

% The definitions cannot be loaded using |\usepackage| or |\RequirePackage|
% which has a mechanism to prevent loading a style file more than once.
% When loading the definitions by means of |\input|
% multiple instances have to be prevented manually:
%\iffalse
%This code needs to be before the `\ProvidesFile' directive
%which is defined at the beginning of this file.
%Therefore it is also placed there and commented out here.
%</package>
%<*discard>
%\fi
%    \begin{macrocode}
\ifdefined\childdocmain\endinput\fi
%    \end{macrocode}
%\iffalse
%</discard>
%<*package>
%\fi
%
% \macro{\ifchilddoc}
% \macro{\ifchilddocmanual}
% The conditional |\ifchilddoc| tells whether a
% child (true) or main (false) document is being compiled.
% The conditional |\ifchilddocmanual| tells whether
% the |\includeonly| mechanism is used (false) or
% the selection of child files must be performed manually (true).
% The definitions initialise to false:
%    \begin{macrocode}
\newif\ifchilddoc
\newif\ifchilddocmanual
%    \end{macrocode}

% \macro{\childdocname}
% \macro{\childdocjob}
% The macro |\childdocname| stores the name of the main document
% to be compiled. The macro |\childdocjob| stores the name of
% the document on which the \LaTeX{} compiler was originally invoked.
% The content of |\jobname| cannot be compared
% to filenames specified in the source due to different catcodes.
% The following code rescans |\jobname|, stores the result
% in |\childdocname| and saves a copy in |\childdocjob|:
%    \begin{macrocode}
\edef\childdocname{\scantokens\expandafter{\jobname\noexpand}}
\let\childdocjob\childdocname
%    \end{macrocode}

% \macro{\childdocdisable}
% The macro |\childdocdisable| prevents the main file
% from being processed more than once.
% At this stage, the main document command |\childdocmain|
% is assumed to be called once again where it should do nothing.
% Any subsequent call to it should prevent
% a secondary processing of the main document
% It overwrites the forwarding commands
% |\childdocof| and |\childdocforward|
% with empty macros to prevent further inclusions of the main document:
%    \begin{macrocode}
\newcommand{\childdocdisable}
{
  \renewcommand{\childdocmain}[1]{\renewcommand{\childdocmain}[1]{\endinput}}
  \renewcommand{\childdocof}[1]{}
  \renewcommand{\childdocby}[2][]{}
  \renewcommand{\childdocforward}[2][]{}
  \renewcommand{\childdocdisable}{}
}
%    \end{macrocode}

% \macro{\childdocmain}
% The macro |\childdocmain| is to be called at the top of the main file
% with nothing or the main filename (without extension) as argument.
% First, it breaks loops.
% If the argument is not empty and does not match |\childdocname|
% (which is set by the first inclusion of |childdoc.def|),
% |\ifchilddoc| is set to true, |\includeonly| is applied to the child file
% and |\jobname| is set to the main file
% (for proper handling of |.aux| files):
%    \begin{macrocode}
\newcommand{\childdocmain}[1]
{
  \childdocdisable\childdocmain{}
  \if?#1?\else
    \begingroup
      \def\childdoctmp{#1}
      \ifx\childdoctmp\childdocname
        \def\childdoctmp{}
      \else
        \def\childdoctmp
        {
          \childdoctrue
          \includeonly{\childdocname}
          \def\childdocjob{#1}
          \def\jobname{#1}
        }
      \fi
      \expandafter
    \endgroup
    \childdoctmp
  \fi
}
%    \end{macrocode}

% \macro{\childdocof}
% The command |\childdocof| redirects
% compilation to the main file |#1|.
%    \begin{macrocode}
\newcommand{\childdocof}[1]
{
  \childdocdisable
  \childdoctrue
  \includeonly{\childdocname}
  \def\jobname{#1}
  \def\childdocjob{#1}
  \input{#1}
}
%    \end{macrocode}

% \macro{\childdocby}
% The command |\childdocby| ....
%    \begin{macrocode}
\newcommand{\childdocby}[2][]
{
  \childdocdisable
  \childdoctrue
  \childdocmanualtrue
  \if?#1?\else
    \def\jobname{#2}
  \fi
  \def\childdocjob{#2}
  \input{#2}
  \endinput
}
%    \end{macrocode}

% \macro{\childdocforward}
% The command |\childdocforward| redirects
% compilation to the main file or
% (if the optional argument is given) a child file.
% Parameters are set as if the main file
% or a child file starting with |\childdocof| was compiled.
% Then compilation is handed over to the main file:
%    \begin{macrocode}
\newcommand{\childdocforward}[2][]
{
  \begingroup
    \if?#1?
      \def\childdoctmp
      {
        \def\childdocname{#2}
        \def\childdocjob{#2}
        \def\jobname{#2}
        \input{#2}
        \endinput
      }
    \else
      \def\childdoctmp
      {
        \childdocdisable
        \def\childdocname{#2}
        \childdoctrue
        \includeonly{#2}
        \def\childdocjob{#1}
        \def\jobname{#1}
        \input{#1}
        \endinput
      }
    \fi
    \expandafter
  \endgroup
  \childdoctmp
}
%    \end{macrocode}

% \macro{\childdocforwardprefix}
% The command |\childdocforwardprefix| redirects
% compilation to the main or a child file by means of a pattern.
% The prefix |#1| in the current filename is replaced by |#2|
% and the suffix of the current filename is kept
% (it is assumed that the filename does not contain the substring `|~~~|'
% which is used as a delimiter).
% Compilation is handed over to the new file by |\childdocforward|:
%    \begin{macrocode}
\newcommand{\childdocforwardprefix}[3][]
{
  \begingroup
    \def\childdocextract #2##1~~~{\def\childdoctmp{\childdocforward[#1]{#3##1}}}
    \expandafter\childdocextract\childdocname~~~
    \expandafter
  \endgroup
  \childdoctmp
}
%    \end{macrocode}

% \macro{\childdoc}
% The deprecated macro |\childdoc| is a legacy version of |\childdocmain|:
%    \begin{macrocode}
\newcommand{\childdoc}{\childdocmain}
%    \end{macrocode}

% \macro{\childdocredirect}
% The deprecated macro |\childdocredirect| is a legacy version
% of |\childdocforward| and |\childdocforwardprefix|:
%    \begin{macrocode}
\newcommand{\childdocredirect}[2][]
{
  \begingroup
    \if?#1?
      \def\childdoctmp{\childdocforward{#2}}
    \else
      \def\childdoctmp{\childdocforwardprefix{#1}{#2}}
    \fi
    \expandafter
  \endgroup
  \childdoctmp
}
%    \end{macrocode}

%\iffalse
%</package>
%\fi
%
\endinput

\childdocmain{}
%    \end{macrocode}

% Optional override for |\version| flag:
%    \begin{macrocode}
%%\ifchilddoc\else\providecommand{\version}{draft}\fi
%    \end{macrocode}

% Define the default values for the |\version| flag
% (|final| for the main file and |draft| for childs):
%    \begin{macrocode}
\ifchilddoc
\providecommand{\version}{draft}
\else
\providecommand{\version}{final}
\fi
%    \end{macrocode}

% Load the standard document class:
%    \begin{macrocode}
\documentclass[12pt]{article}
%    \end{macrocode}

% Start the document body:
%    \begin{macrocode}
\begin{document}
%    \end{macrocode}

% Declare a title page.
% Print title, part of document being processed and version flag:
%    \begin{macrocode}
\addtocounter{page}{-1}
\begin{center}
{\LARGE\bfseries{}childdoc example\par}
\vspace{1cm}
\ifchilddoc
\ifchilddocmanual part\else chapter\fi:
`\childdocname' of `\childdocjob'\par
\else
main document: `\childdocjob'\par
\fi
version: \version\par
\end{center}
\newpage
%    \end{macrocode}

% Manually include selected file,
% otherwise process as usual:
%    \begin{macrocode}
\ifchilddocmanual
\section*{part `\childdocname'}
\input{\childdocname}
\else
%    \end{macrocode}

% Include the two chapters:
%    \begin{macrocode}
\include{cdocsch1}
\include{cdocsch2}
%    \end{macrocode}

% Include the two parts unless only chapters should be displayed:
%    \begin{macrocode}
\ifchilddoc\else
\section{part three}
\input{cdocspt3}
\section{part four}
\input{cdocspt4}
\fi
%    \end{macrocode}

% Process as usual until here:
%    \begin{macrocode}
\fi
%    \end{macrocode}

% End of document body:
%    \begin{macrocode}
\end{document}
%    \end{macrocode}
%\iffalse
%</samplemain>
%\fi
%
% %%%%%%%%%%%%%%%%%%%%%%%%%%%%%%%%%%%%%%
% \paragraph{Chapter Include Files.}
%
% The include files are called |cdocsch1.tex| and |cdocsch2.tex|.
%
%\iffalse
%<*samplechap1|samplechap2>
%\fi

% Optional override for |\version| flag:
%    \begin{macrocode}
%%\providecommand{\version}{final}
%    \end{macrocode}

% Include the main document:
%    \begin{macrocode}
% \iffalse
%
% childdoc.dtx Copyright (C) 2017-2018 Niklas Beisert
%
% This work may be distributed and/or modified under the
% conditions of the LaTeX Project Public License, either version 1.3
% of this license or (at your option) any later version.
% The latest version of this license is in
%   http://www.latex-project.org/lppl.txt
% and version 1.3 or later is part of all distributions of LaTeX
% version 2005/12/01 or later.
%
% This work has the LPPL maintenance status `maintained'.
%
% The Current Maintainer of this work is Niklas Beisert.
%
% This work consists of the files childdoc.dtx and childdoc.ins
% and the derived files childdoc.def and cdocsamp.tex with
% cdocsch1.tex, cdocsch2.tex, cdocsdrf.tex, cdocsfn1.tex, cdocsfn2.tex.
%
%<package>\ifdefined\childdocmain\endinput\fi
%<package>\ProvidesFile{childdoc.def}[2018/12/30 v2.0 child document driver]
%<samplemain>\ProvidesFile{cdocsamp.tex}[2018/12/30 v2.0 sample for childdoc]
%<*driver>
%\ProvidesFile{childdoc.drv}[2018/12/30 v2.0 childdoc reference manual file]
\PassOptionsToClass{10pt,a4paper}{article}
\documentclass{ltxdoc}

\usepackage[margin=35mm]{geometry}
\usepackage{hyperref}
\usepackage{hyperxmp}
\usepackage[usenames]{color}

\hypersetup{colorlinks=true}
\hypersetup{pdfstartview=FitH}
\hypersetup{pdfpagemode=UseNone}
\hypersetup{pdfsource={}}
\hypersetup{pdflang={en-UK}}
\hypersetup{pdfcopyright={Copyright 2017-2018 Niklas Beisert.
  This work may be distributed and/or modified under the
  conditions of the LaTeX Project Public License, either version 1.3
  of this license or (at your option) any later version.}}
\hypersetup{pdflicenseurl={http://www.latex-project.org/lppl.txt}}
\hypersetup{pdfcontactaddress={ETH Zurich, ITP, HIT K,
  Wolfgang-Pauli-Strasse 27}}
\hypersetup{pdfcontactpostcode={8093}}
\hypersetup{pdfcontactcity={Zurich}}
\hypersetup{pdfcontactcountry={Switzerland}}
\hypersetup{pdfcontactemail={nbeisert@itp.phys.ethz.ch}}
\hypersetup{pdfcontacturl={http://people.phys.ethz.ch/\xmptilde nbeisert/}}

\newcommand{\secref}[1]{\hyperref[#1]{section \ref*{#1}}}

\parskip1ex
\parindent0pt
\let\olditemize\itemize
\def\itemize{\olditemize\parskip0pt}

\begin{document}

\title{The \textsf{childdoc} Package}
\hypersetup{pdftitle={The childdoc Package}}
\author{Niklas Beisert\\[2ex]
  Institut f\"ur Theoretische Physik\\
  Eidgen\"ossische Technische Hochschule Z\"urich\\
  Wolfgang-Pauli-Strasse 27, 8093 Z\"urich, Switzerland\\[1ex]
  \href{mailto:nbeisert@itp.phys.ethz.ch}
  {\texttt{nbeisert@itp.phys.ethz.ch}}}
\hypersetup{pdfauthor={Niklas Beisert}}
\hypersetup{pdfsubject={Manual for the LaTeX2e Package childdoc}}
\date{30 December 2018, \textsf{v2.0}}
\maketitle

\begin{abstract}\noindent
\textsf{childdoc} is a \LaTeXe{} package
that enables the direct compilation
of document sections included by |\include|
to individual files.
\end{abstract}

\begingroup
\parskip0ex
\tableofcontents
\endgroup

%%%%%%%%%%%%%%%%%%%%%%%%%%%%%%%%%%%%%%%%%%%%%%%%%%%%%%%%%%%%%%%%%%%%%%%%%%%%%%%%
%%%%%%%%%%%%%%%%%%%%%%%%%%%%%%%%%%%%%%%%%%%%%%%%%%%%%%%%%%%%%%%%%%%%%%%%%%%%%%%%
\section{Introduction}

\LaTeX{} provides a mechanism to structure a large document (such as a book)
into a main file and several child files (containing the chapters)
using the |\include| command.
This mechanism is beneficial for documents
which span hundreds of pages in order to
make the source file(s) more manageable.
Moreover, compilation can be restricted to
selected child files by means of the |\includeonly| command.
The latter feature can be used to reduce the compilation time while editing
(this was significantly more useful in the earlier days of \LaTeX{})
or to generate a smaller document which is easier to navigate.
Another application of |\includeonly| is to generate
documents consisting of selected parts of the complete document.

However, there are a few drawbacks of the plain |\include| mechanism:
\begin{itemize}
\item
The child files cannot be compiled on their own,
they can only be compiled via the main file.
A naive editing environment
(such as a text editor with an option
to have the current file processed by \LaTeX)
may require one to switch to the main file before compiling;
attempting to compile the child file produces errors.
\item
The main file must be modified (each time)
to adjust the |\includeonly| command
to the present needs. This easily leaves the main file in a messy state.
\item
The generated document will always carry the filename
of the main document. This is inconvenient if
several child files are to be compiled and
to be kept for distribution.
\end{itemize}

The present package provides a simple interface
to make child files individually compilable by \LaTeX{}.
Compiling a child file then has the same effect as compiling
the main file with an |\includeonly| command
to select the appropriate child.
Moreover the generated document will carry the name of the child
rather than the main file.
This resolves all three above issues.

This feature is meant to make the editing of books,
thesis documents and lecture notes somewhat more convenient.
However, the package can also be used efficiently for
composing a series of documents (such as exercise sheets)
which are typically distributed individually.
It then assists the author in generating the individual documents
(potentially in different versions)
as well as a document containing the collected series.
Another application is in developing style files
or other kinds of included material
where compilation of the style file could redirect
to a sample or test file.

%%%%%%%%%%%%%%%%%%%%%%%%%%%%%%%%%%%%%%%%%%%%%%%%%%%%%%%%%%%%%%%%%%%%%%%%%%%%%%%%
%%%%%%%%%%%%%%%%%%%%%%%%%%%%%%%%%%%%%%%%%%%%%%%%%%%%%%%%%%%%%%%%%%%%%%%%%%%%%%%%
\section{Usage}

First of all, the package \textsf{childdoc} is \emph{not} a standard
\LaTeXe{} |.sty| style file! Therefore it needs to be invoked in
a non-standard way.

%%%%%%%%%%%%%%%%%%%%%%%%%%%%%%%%%%%%%%%%%%%%%%%%%%%%%%%%%%%%%%%%%%%%%%%%%%%%%%%%
\subsection{Included Files}
\label{sec:include}

%%%%%%%%%%%%%%%%%%%%%%%%%%%%%%%%%%%%%%%%
\DescribeMacro{\childdocmain}
To use the package, add the commands
\begin{center}
\begin{tabular}{l}
|\input{childdoc.def}|\\
|\childdocmain{}|\\
\end{tabular}
\end{center}
at the very top of the main \LaTeX{} file,
in particular \emph{before} the |\documentclass| statement!
The argument of |\childdocmain| should be left empty
(but it must be present).

%%%%%%%%%%%%%%%%%%%%%%%%%%%%%%%%%%%%%%%%
\DescribeMacro{\childdocof}
Furthermore, add the commands
\begin{center}
\begin{tabular}{l}
|\input{childdoc.def}|\\
|\childdocof{|\textit{main}|}|\\
\end{tabular}
\end{center}
at the top of every child file \textit{child}
which is included by |\include{|\textit{child}|}|
from within the main file
(or at least for those files to be compiled individually).
The argument \textit{main} must be the filename of the main file.

There are a couple of
considerations in setting up the main and child documents:

%%%%%%%%%%%%%%%%%%%%%%%%%%%%%%%%%%%%%%%%
\paragraph{Restrictions.}

Please note the following restrictions:
\begin{itemize}
\item
|\childdocmain| must be called with one argument \textit{main}
to ensure compatibility with earlier version of the package.
It must either be empty (|\childdocmain{}|)
or precisely match the filename of the main file in which it is specified.
See \secref{sec:detection} for further information.
\item
The filename \textit{main} must be specified without the |.tex| extension.
\item
The filename \textit{main} is case sensitive
(even in case-insensitive file systems)
due to internal string comparison.
\item
The argument \textit{main} should be fully expanded, it cannot be a macro.
\item
Subdirectories and special characters should be avoided in filenames.
\item
The command |\childdocmain{|\textit{main}|}| must be followed by a whitespace.
It should not be followed immediately by another command
or by a comment mark `|%|'.
This is because the \TeX{} parser reads the token immediately following
the argument of |\childdocmain| and puts it
at the beginning of every child section;
however, a white\-space is ignored.
\end{itemize}

%%%%%%%%%%%%%%%%%%%%%%%%%%%%%%%%%%%%%%%%
\paragraph{Content of Main File.}

It is advisable to place all content in the child files included by |\include|.
Any output contained in the main file will appear in all child documents
unless suppressed manually;
it cannot be suppressed automatically by the |\includeonly| directive
and thus should normally be avoided.
A method to include some content in the main file
by means of conditional processing is described in \secref{sec:conditional}.

%%%%%%%%%%%%%%%%%%%%%%%%%%%%%%%%%%%%%%%%
\paragraph{Page Numbering.}

When only a part of the document is compiled,
the appropriate numbering of pages
(as well as other status parameters)
is determined from the |.aux| files.
The latter contain information from previous passes.
However this information needs to propagate through
all intermediate child documents.
Therefore the page numbering in child documents may well
be inconsistent until the complete document is compiled at least once.

A useful (if unconventional) way to always ensure a consistent
page numbering is to restart the numbering in each child document
and denote the pages by `\textit{child}|.|\textit{page}'
where \textit{child} represents the chapter/section number of the child file.
This can be achieved by the command
|\numberwithin{page}{|\textit{child}|}|
of the \textsf{amsmath} package
where \textit{child} can be |chapter| or |section|
depending on the chosen structuring.
Alternatively, one can modify the macro |\thepage| appropriately
and reset the counter |page| at the start of each child file.

%%%%%%%%%%%%%%%%%%%%%%%%%%%%%%%%%%%%%%%%%%%%%%%%%%%%%%%%%%%%%%%%%%%%%%%%%%%%%%%%
\subsection{Conditional Processing}
\label{sec:conditional}

The package provides a mechanism to compile different versions
of a document. To customise the versions further some conditional processing
can come in handy to distinguish which version is being compiled.
The package provides two macros to describe the compilation context:

%%%%%%%%%%%%%%%%%%%%%%%%%%%%%%%%%%%%%%%%
\DescribeMacro{\ifchilddoc}
The conditional |\ifchilddoc| distinguishes between the compilation of
child documents and the main document:
%
\begin{center}
|\ifchilddoc |\textit{child-code}| |[|\||else |\textit{main-code}]| \||fi|
\end{center}

%%%%%%%%%%%%%%%%%%%%%%%%%%%%%%%%%%%%%%%%
\DescribeMacro{\childdocname}
\DescribeMacro{\childdocjob}
The macro |\childdocname| contains the filename (without extension)
of the main or child file being processed.
Note that |\childdocjob| will always contain the name of the main file.

%%%%%%%%%%%%%%%%%%%%%%%%%%%%%%%%%%%%%%%%
\paragraph{Title Page.}

Conditional processing can be used to include a title or banner page
in the main document when proper precautions are taken.
Importantly, the code in the main file should ensure that the page counter
(as well as other status parameters which are stored in the |.aux| files)
takes the same value after the conditional processing.
Otherwise the page numbers may take divergent values
depending on which part is compiled.

For example, a title page could be declared by:
%
\begin{center}
\begin{tabular}{l}
|\ifchilddoc\||else|\\
|\addtocounter{page}{-1}|\\
\textit{code for title page}\\
|\newpage|\\
|\||fi|
\end{tabular}
\end{center}
%
A banner page for the child documents can be generated by:
%
\begin{center}
\begin{tabular}{l}
|\ifchilddoc|\\
|\addtocounter{page}{-1}|\\
\textit{code for banner page}\\
|\newpage|\\
|\||fi|
\end{tabular}
\end{center}
%
Here one could write a message such as:
\begin{center}
|This is the part \childdocname{} of \childdocjob{}.|
\end{center}

%%%%%%%%%%%%%%%%%%%%%%%%%%%%%%%%%%%%%%%%%%%%%%%%%%%%%%%%%%%%%%%%%%%%%%%%%%%%%%%%
\subsection{Flags}
\label{sec:flags}

The package makes it easy to generate different versions
of the main or child documents.
To this end compilation flags can be defined
and assigned different default values.
They will be particularly useful in conjunction
with the forwarding mechanism described in \secref{sec:forward}.

For example, it may be useful to have a flag |\version|
which can be set to |draft| or |final|.
The document source will contain some conditional code
depending on the value of |\version|.
Suppose further, the flag should default to |final| for the main file
and to |draft| for child files
which is a natural assignment for editing the document.
This is achieved by placing the following code
in the preamble of the main document
(below the |\childdocmain| directive):
%
\begin{center}
\begin{tabular}{l}
|\ifchilddoc|\\
|\providecommand{\version}{draft}|\\
|\||else|\\
|\providecommand{\version}{final}|\\
|\||fi|
\end{tabular}
\end{center}
%
The definition by |\providecommand| makes sure
that previous definitions are not overwritten.
Further statements |\providecommand{\version}{...}|
can thus be added before the above code to override it.

For the main file, one might add a line
(between |\childdocmain| and the above block)
%
\begin{center}
|%\ifchilddoc\||else\providecommand{\version}{draft}\||fi|
\end{center}
%
which can be uncommented to produce a draft version.
Likewise one can add a line to the very top of a child file
(above the |\childdocof{|\textit{main}|}| directive)
%
\begin{center}
|%\providecommand{\version}{final}|
\end{center}
%
which can be uncommented to produce the final version of this child document.

%%%%%%%%%%%%%%%%%%%%%%%%%%%%%%%%%%%%%%%%%%%%%%%%%%%%%%%%%%%%%%%%%%%%%%%%%%%%%%%%
\subsection{Forwarding}
\label{sec:forward}

Different versions of the main or child documents
using compilation flags as described in \secref{sec:flags}
can be (permanently) stored in different files
for convenient compilation, viewing and distribution.
To this end, the package defines a command
to pass on compilation to a different file:

%%%%%%%%%%%%%%%%%%%%%%%%%%%%%%%%%%%%%%%%
\DescribeMacro{\childdocforward}
The command |\childdocforward| redirects processing to
another source file:
%
\begin{center}
\begin{tabular}{l}
|\input{childdoc.def}|\\
|\childdocforward[|\textit{main}|]{|\textit{dest}|}|\\
\end{tabular}
\end{center}
%
The argument \textit{dest} is the destination file
(without extension).
It should be the main file or one of the child files.
Note that further \textsf{childdoc} directives
such as |\childdocof| and |\childdocforward|
in the indicated file will be processed in this form.
The optional argument \textit{main}
passes on directly to the main file \textit{main}
while pretending to compile the child \textit{dest}.
This form behaves as if \textit{dest}
issues |\childdocof{|\textit{main}|}| right away,
and no further \textsf{childdoc} directives will be processed.

%%%%%%%%%%%%%%%%%%%%%%%%%%%%%%%%%%%%%%%%
\DescribeMacro{\...prefix}
In the alternative form |\childdocforwardprefix|,
%
\begin{center}
\begin{tabular}{l}
|\input{childdoc.def}|\\
|\childdocforwardprefix[|\textit{main}|]{|\textit{prefix}|}{|\textit{dest}|}|
\end{tabular}
\end{center}
%
the destination file is determined by a pattern
depending on the current file:
To make this work, the current file must be called
`{\textit{prefix}\hspace{0.2em}\textit{suffix}}'
with \textit{prefix} matching precisely the argument.
Processing is then passed on to the file
`{\textit{dest}\hspace{0.2em}\textit{suffix}}'.
Surely, the same effect is achieved by
directly specifying the
argument `{\textit{dest}\hspace{0.2em}\textit{suffix}}'
in the first form.
However, that requires to set up a different file
for each child. With the alternative form of the command
all these files can have exactly the same content
which simplifies setting them up and maintaining them.

For example, the following file |draft.tex|
with a compilation flag |\version| as described in \secref{sec:flags}
compiles the main document as a draft:
%
\begin{center}
\begin{tabular}{l}
|\def\version{draft}|\\
|\input{childdoc.def}|\\
|\childdocforward{|\textit{main}|}|
\end{tabular}
\end{center}
%
Likewise, the following files |final|\textit{nn}|.tex|
compile the final version of the child document
|child|\textit{nn}|.tex|:
%
\begin{center}
\begin{tabular}{l}
|\def\version{final}|\\
|\input{childdoc.def}|\\
|\childdocforwardprefix{final}{child}|
\end{tabular}
\end{center}
%

Note that when several versions of a main file and/or of each child file
are to be generated, it may be convenient to set up a |Makefile| or
shell script to automatise the process.

%%%%%%%%%%%%%%%%%%%%%%%%%%%%%%%%%%%%%%%%%%%%%%%%%%%%%%%%%%%%%%%%%%%%%%%%%%%%%%%%
\subsection{Command Line Processing}
\label{sec:commandline}

The effect of redirection files can also be achieved by invoking
the \LaTeX{} compiler with a more elaborate command line.
Most conveniently this should be done as part
of a shell script or a |Makefile|.

When using \textsf{childdoc} in the main file, the following
command lines effectively perform a redirection
(note that depending on the shell being used,
backslashes may have to be doubled: `|\|' $\to$ `|\\|'):
%
\begin{center}
|... -jobname "|\textit{target}|" |\\|"|[\textit{flags}]%
|\input{childdoc.def}\childdocforward[|\textit{main}|]{|\textit{dest}|}"|
\end{center}
%
Here \textit{target} is the name of the output file,
\textit{main} is the name of the main file
and \textit{dest} is the name of the main or child file to be processed
(all filenames without extensions).
The optional argument \textit{main} can be omitted
if \textit{main} matches \textit{dest}.
Optionally, compilation \textit{flags} can be defined via |\def| commands.
This command line makes the \TeX{} engine believe
it is compiling the file \textit{target}
whose content is specified as the latter parameter.
The provided code then forwards the processing to
\textit{main} or \textit{dest} as described in \secref{sec:forward}.

%%%%%%%%%%%%%%%%%%%%%%%%%%%%%%%%%%%%%%%%%%%%%%%%%%%%%%%%%%%%%%%%%%%%%%%%%%%%%%%%
\subsection{Include by Input}
\label{sec:input}

Including child documents by |\include| has some restrictions by design.
Most notably, the content of a child document always occupies
its own set of pages; pages cannot be shared between child documents.
Usually, this behaviour makes perfect sense
because each child document contain an essential part of the document.
However, in some situations it may be desirable to compose
a document from a collection of parts
without having mandatory page breaks between then.
For this case, the package
provides a mechanism to include parts
by |\input| which can also be processed individually.
However, by construction this mechanism
requires manual handling of the content to be output.

%%%%%%%%%%%%%%%%%%%%%%%%%%%%%%%%%%%%%%%%
\DescribeMacro{\ifchilddocmanual}
The main file should be prepared as usual, see \secref{sec:include}.
However, the document body must make a distinction
between processing of an individual part and of the main document, e.g.:
%
\begin{center}
\begin{tabular}{l}
|\ifchilddocmanual|\\
|\input{\childdocname}|\\
|\||else|\\
\textit{document body with }|\input{|\textit{part}|}|\\
|\||fi|
\end{tabular}
\end{center}
%
The conditional |\ifchilddocmanual| is true whenever
a part to be included by |\input| is being compiled,
and the name of the part is stored in |\childdocname|.

%%%%%%%%%%%%%%%%%%%%%%%%%%%%%%%%%%%%%%%%
\DescribeMacro{\childdocby}
Each part to be included by |\input| should start with:
%
\begin{center}
\begin{tabular}{l}
|\input{childdoc.def}|\\
|\childdocby{|\textit{main}|}|\\
\end{tabular}
\end{center}
%
The directive |\childdocby| is similar to |\childdocof|
described in \secref{sec:include},
but the subsequent selection of content must be done manually.
To that end, both |\ifchilddoc| and |\ifchilddocmanual|
will be true upon processing of a part,
and the name of the part is stored in |\childdocname|.
Note that |\jobname| will be set to the filename of the current part
so that each part receives an individual |.aux| file
that does not interfere with the |.aux| file(s) of the main document.
This behaviour can be altered by the alternative form
|\childdocby[*]{|\textit{main}|}| (with a non-empty optional argument)
which uses the |.aux| file of the main document
by setting |\jobname| to \textit{main}.

%%%%%%%%%%%%%%%%%%%%%%%%%%%%%%%%%%%%%%%%%%%%%%%%%%%%%%%%%%%%%%%%%%%%%%%%%%%%%%%%
\subsection{Driver Development}
\label{sec:driver}

The \textsf{childdoc} mechanism can also be use for the development
of definition files such as \LaTeX{} styles or classes.
This case differs from the above setup with multiple parts
included by |\include| in that no |\includeonly| should be invoked.
This can be achieved by starting the include file
(before |\ProvidesPackage|) with:
%
\begin{center}
\begin{tabular}{l}
|\input{childdoc.def}|\\
|\childdocforward{|\textit{main}|}|\\
\end{tabular}
\end{center}
%
or alternatively with:
%
\begin{center}
\begin{tabular}{l}
|\input{childdoc.def}|\\
|\childdocby{|\textit{main}|}|\\
\end{tabular}
\end{center}
%
Both forms have slightly different effects as described above.
The main file is prepared as usual, see \secref{sec:include}.

%%%%%%%%%%%%%%%%%%%%%%%%%%%%%%%%%%%%%%%%%%%%%%%%%%%%%%%%%%%%%%%%%%%%%%%%%%%%%%%%
\subsection{Legacy Detection}
\label{sec:detection}

The directive |\childdocmain| in the main file can detect
whether the complete document or merely a child is to be compiled
even without using the directive |\childdocof|.
This method is deprecated because it is less robust
and there is no compelling reason to use it;
it is merely provided for backward compatibility
and it may be removed in future versions.

If the detection mechanism is to be used,
it is mandatory to correctly specify
the filename of the main file as the argument of |\childdocmain|:
%
\begin{center}
\begin{tabular}{l}
|\input{childdoc.def}|\\
|\childdocmain{|\textit{main}|}|\\
\end{tabular}
\end{center}
%
If |\jobname| does not match the argument \textit{main} of |\childdocmain|,
it is assumed that |\jobname| points to the child file to be compiled.
When using |\childdocmain| with the main file specified as argument,
it suffices to start a child file
with just |\input{|\textit{main}|}|
without loading of the package and using |\childdocof|.
If instead all processing is done
with the appropriate \textsf{childdoc} directives,
the argument of \textit{main} of |\childdocmain| can be empty.

An alternative version of the command line processing described
in \secref{sec:commandline} using the detection mechanism reads:
%
\begin{center}
|... -jobname "|\textit{target}|" "|[\textit{flags}]%
[|\def\jobname{|\textit{dest}|}|]|\input{|\textit{main}|}"|
\end{center}

%%%%%%%%%%%%%%%%%%%%%%%%%%%%%%%%%%%%%%%%%%%%%%%%%%%%%%%%%%%%%%%%%%%%%%%%%%%%%%%%
\subsection{Manual Code}
\label{sec:manual}

In case one cannot be certain whether the definitions file |childdoc.def|
is installed on the target \TeX{} distribution
and one prefers not to ship it,
it is conceivable to paste a few relevant commands into the sources.

To that end, drop all statements |\input{childdoc.def}|
and perform the replacements as outlined below.
Instead of |\childdocmain{|\textit{main}|}| add the following code
to the top of the main file:
%
\begin{center}
\begin{tabular}{l}
|\||ifdefined\childdocname\endinput\||fi\newif\ifchilddoc|\\
|\edef\childdocname{\scantokens\expandafter{\jobname\noexpand}}|\\
|\def\childdocmain{|\textit{main}|}\||ifx\childdocmain\childdocname\||else|\\
|\childdoctrue\includeonly{\childdocname}\let\jobname\childdocmain\||fi|\\
\end{tabular}
\end{center}
%
Instead of |\childdocof{|\textit{main}|}| just include the main file
at the top of each child file:
%
\begin{center}
|\input{|\textit{main}|}|
\end{center}
%
A simple redirection |\childdocforward{|\textit{dest}|}| is achieved by:
%
\begin{center}
|\def\jobname{|\textit{dest}|}\input{\jobname}|
\end{center}
%
The redirection with prefix
|\childdocforwardprefix[|\textit{prefix}|]{|\textit{dest}|}|
is accomplished by:
%
\begin{center}
\begin{tabular}{l}
|{\edef\jobname{\scantokens\expandafter{\jobname\noexpand}}|\\
|\def\redirectjob |\textit{prefix}|#1~~~{\gdef\jobname{|\textit{dest}|#1}}|\\
|\expandafter\redirectjob\jobname~~~}\input{\jobname}|
\end{tabular}
\end{center}

In an alternative approach,
child documents can be compiled by a specific command line
without additional code or specific definitions:
%
\begin{center}
|... -jobname "|\textit{target}|" "|[\textit{flags}]%
|\includeonly{|\textit{dest}|}\input{|\textit{main}|}"|
\end{center}
%

%%%%%%%%%%%%%%%%%%%%%%%%%%%%%%%%%%%%%%%%%%%%%%%%%%%%%%%%%%%%%%%%%%%%%%%%%%%%%%%%
%%%%%%%%%%%%%%%%%%%%%%%%%%%%%%%%%%%%%%%%%%%%%%%%%%%%%%%%%%%%%%%%%%%%%%%%%%%%%%%%
\section{Information}

%%%%%%%%%%%%%%%%%%%%%%%%%%%%%%%%%%%%%%%%%%%%%%%%%%%%%%%%%%%%%%%%%%%%%%%%%%%%%%%%
\subsection{Copyright}

Copyright \copyright{} 2017--2018 Niklas Beisert

This work may be distributed and/or modified under the
conditions of the \LaTeX{} Project Public License, either version 1.3
of this license or (at your option) any later version.
The latest version of this license is in
  \url{http://www.latex-project.org/lppl.txt}
and version 1.3 or later is part of all distributions of \LaTeX{}
version 2005/12/01 or later.

This work has the LPPL maintenance status `maintained'.

The Current Maintainer of this work is Niklas Beisert.

This work consists of the files |README.txt|, |childdoc.ins| and |childdoc.dtx|
as well as the derived files |childdoc.def|, |cdocsamp.tex|
with |cdocsch1.tex|, |cdocsch2.tex|, |cdocspt3.tex|, |cdocspt4.tex|,
|cdocsdrf.tex|, |cdocsfn1.tex|, |cdocsfn2.tex|
as well as |childdoc.pdf|.

%%%%%%%%%%%%%%%%%%%%%%%%%%%%%%%%%%%%%%%%%%%%%%%%%%%%%%%%%%%%%%%%%%%%%%%%%%%%%%%%
\subsection{Files and Installation}

The package consists of the files:
%
\begin{center}
\begin{tabular}{ll}
    |README.txt|   & readme file \\
    |childdoc.ins| & installation file \\
    |childdoc.dtx| & source file \\
    |childdoc.def| & definition file \\
    |cdocsamp.tex| & sample main file \\
    |cdocsch1.tex| & sample include file \\
    |cdocsch2.tex| & sample include file \\
    |cdocspt3.tex| & sample part file \\
    |cdocspt4.tex| & sample part file \\
    |cdocsdrf.tex| & sample redirection file \\
    |cdocsfn1.tex| & sample redirection file \\
    |cdocsfn2.tex| & sample redirection file \\
    |childdoc.pdf| & manual
\end{tabular}
\end{center}
%
The distribution consists of the files
|README.txt|, |childdoc.ins| and |childdoc.dtx|.
%
\begin{itemize}
\item
Run (pdf)\LaTeX{} on |childdoc.dtx|
to compile the manual |childdoc.pdf| (this file).
\item
Run \LaTeX{} on |childdoc.ins| to create the definitions file |childdoc.def|
and the sample |cdocsamp.tex| with include files
|cdocsch1.tex|, |cdocsch2.tex|, |cdocspt3.tex|, |cdocspt4.tex|,
|cdocsdrf.tex|, |cdocsfn1.tex|, |cdocsfn2.tex|.
Then copy the file |childdoc.def| to an appropriate directory of your \LaTeX{}
distribution, e.g.\ \textit{texmf-root}|/tex/latex/childdoc|.
\end{itemize}

%%%%%%%%%%%%%%%%%%%%%%%%%%%%%%%%%%%%%%%%%%%%%%%%%%%%%%%%%%%%%%%%%%%%%%%%%%%%%%%%
\subsection{Related CTAN Packages}

There are several other packages which offer a similar functionality:
%
\begin{itemize}
\item
The packages
\href{http://ctan.org/pkg/docmute}{\textsf{docmute}},
\href{http://ctan.org/pkg/includex}{\textsf{includex}} and
\href{http://ctan.org/pkg/standalone}{\textsf{standalone}}
provide commands to include only the document body of
a child file thus allowing both files to be compiled individually.
\item
The packages \href{http://ctan.org/pkg/subdocs}{\textsf{subdocs}}
and \href{http://ctan.org/pkg/subfiles}{\textsf{subfiles}}
provide structures in which the main and child documents can be
encapsulated and allowing them to be compiled individually.
The inclusion mechanism is different from the conventional |\include|.
\item
The package \href{http://ctan.org/pkg/combine}{\textsf{combine}}
is an elaborate solution to combine several documents into one.
\end{itemize}
%
See also the CTAN topic \href{http://ctan.org/topic/subdocs}{\textsf{subdocs}}
for further related packages.
The present package differs from the above solutions in that
a document structure constructed with the conventional |\include| mechanism
just needs two extra commands at the top of every file
such that all constituent files can be compiled individually.

%%%%%%%%%%%%%%%%%%%%%%%%%%%%%%%%%%%%%%%%%%%%%%%%%%%%%%%%%%%%%%%%%%%%%%%%%%%%%%%%
%\subsection{Feature Suggestions}
%
%The following is a list of features which may be useful for future
%versions of this package:
%%
%\begin{itemize}
%\item
%\ldots
%\end{itemize}

%%%%%%%%%%%%%%%%%%%%%%%%%%%%%%%%%%%%%%%%%%%%%%%%%%%%%%%%%%%%%%%%%%%%%%%%%%%%%%%%
\subsection{Revision History}

%%%%%%%%%%%%%%%%%%%%%%%%%%%%%%%%%%%%%%%%
\paragraph{v2.0:} 2018/12/30

\begin{itemize}
\item
immediate forward processing
\item
added |\childdocby| mechanism
\item
manual restructured
\end{itemize}

%%%%%%%%%%%%%%%%%%%%%%%%%%%%%%%%%%%%%%%%
\paragraph{v1.6:} 2018/01/17

\begin{itemize}
\item
application for development of include files
\item
corrections to manual
\end{itemize}

%%%%%%%%%%%%%%%%%%%%%%%%%%%%%%%%%%%%%%%%
\paragraph{v1.5:} 2017/05/21

\begin{itemize}
\item
more complete structuring introduced
\item
|\childdocof| introduced
\item
|\childdoc| renamed to |\childdocmain|
\item
|\childredirect| renamed to |\childdocforward| and |\childdocforwardprefix|
and functionality expanded
\end{itemize}

%%%%%%%%%%%%%%%%%%%%%%%%%%%%%%%%%%%%%%%%
\paragraph{v1.0:} 2017/04/27

\begin{itemize}
\item
manual and install package
\item
first version published on CTAN
\end{itemize}

%%%%%%%%%%%%%%%%%%%%%%%%%%%%%%%%%%%%%%%%
\paragraph{v0.6:} 2017/04/26

\begin{itemize}
\item
redirection mechanism added
\end{itemize}

%%%%%%%%%%%%%%%%%%%%%%%%%%%%%%%%%%%%%%%%
\paragraph{v0.5:} 2017/04/26

\begin{itemize}
\item
functionality in definition file
\end{itemize}


%%%%%%%%%%%%%%%%%%%%%%%%%%%%%%%%%%%%%%%%%%%%%%%%%%%%%%%%%%%%%%%%%%%%%%%%%%%%%%%%
%%%%%%%%%%%%%%%%%%%%%%%%%%%%%%%%%%%%%%%%%%%%%%%%%%%%%%%%%%%%%%%%%%%%%%%%%%%%%%%%
%%%%%%%%%%%%%%%%%%%%%%%%%%%%%%%%%%%%%%%%%%%%%%%%%%%%%%%%%%%%%%%%%%%%%%%%%%%%%%%%
\appendix

\settowidth\MacroIndent{\rmfamily\scriptsize 000\ }

 \DocInput{childdoc.dtx}

\end{document}
%</driver>
% \fi
%
% %%%%%%%%%%%%%%%%%%%%%%%%%%%%%%%%%%%%%%%%%%%%%%%%%%%%%%%%%%%%%%%%%%%%%%%%%%%%%%
% %%%%%%%%%%%%%%%%%%%%%%%%%%%%%%%%%%%%%%%%%%%%%%%%%%%%%%%%%%%%%%%%%%%%%%%%%%%%%%
% \section{Sample}
%\iffalse
%<*samplemain>
%\fi
%
% The following presents a sample document
% with two chapters, two parts, a title page,
% a compile flag as well as three forwarding files to set the flag.
% It consists of eight |.tex| files:
% \begin{center}
% \begin{tabular}{ll}
% |cdocsamp.tex|&main file\\
% |cdocsch1.tex|&include file for chapter 1\\
% |cdocsch2.tex|&include file for chapter 2\\
% |cdocspt3.tex|&include file for part 3\\
% |cdocspt4.tex|&include file for part 4\\
% |cdocsdrf.tex|&forwarding file for main file in draft mode\\
% |cdocsfi1.tex|&forwarding file for final version of chapter 1\\
% |cdocsfi2.tex|&forwarding file for final version of chapter 2\\
% \end{tabular}
% \end{center}
% Each of the eight files can be compiled directly by the \LaTeX{} compiler.
%
% %%%%%%%%%%%%%%%%%%%%%%%%%%%%%%%%%%%%%%
% \paragraph{Main File.}
%
% The main file is called |cdocsamp.tex|.
%
% Load the \textsf{childdoc} definitions and
% declare the filename for the main document:
%    \begin{macrocode}
\input{childdoc.def}
\childdocmain{}
%    \end{macrocode}

% Optional override for |\version| flag:
%    \begin{macrocode}
%%\ifchilddoc\else\providecommand{\version}{draft}\fi
%    \end{macrocode}

% Define the default values for the |\version| flag
% (|final| for the main file and |draft| for childs):
%    \begin{macrocode}
\ifchilddoc
\providecommand{\version}{draft}
\else
\providecommand{\version}{final}
\fi
%    \end{macrocode}

% Load the standard document class:
%    \begin{macrocode}
\documentclass[12pt]{article}
%    \end{macrocode}

% Start the document body:
%    \begin{macrocode}
\begin{document}
%    \end{macrocode}

% Declare a title page.
% Print title, part of document being processed and version flag:
%    \begin{macrocode}
\addtocounter{page}{-1}
\begin{center}
{\LARGE\bfseries{}childdoc example\par}
\vspace{1cm}
\ifchilddoc
\ifchilddocmanual part\else chapter\fi:
`\childdocname' of `\childdocjob'\par
\else
main document: `\childdocjob'\par
\fi
version: \version\par
\end{center}
\newpage
%    \end{macrocode}

% Manually include selected file,
% otherwise process as usual:
%    \begin{macrocode}
\ifchilddocmanual
\section*{part `\childdocname'}
\input{\childdocname}
\else
%    \end{macrocode}

% Include the two chapters:
%    \begin{macrocode}
\include{cdocsch1}
\include{cdocsch2}
%    \end{macrocode}

% Include the two parts unless only chapters should be displayed:
%    \begin{macrocode}
\ifchilddoc\else
\section{part three}
\input{cdocspt3}
\section{part four}
\input{cdocspt4}
\fi
%    \end{macrocode}

% Process as usual until here:
%    \begin{macrocode}
\fi
%    \end{macrocode}

% End of document body:
%    \begin{macrocode}
\end{document}
%    \end{macrocode}
%\iffalse
%</samplemain>
%\fi
%
% %%%%%%%%%%%%%%%%%%%%%%%%%%%%%%%%%%%%%%
% \paragraph{Chapter Include Files.}
%
% The include files are called |cdocsch1.tex| and |cdocsch2.tex|.
%
%\iffalse
%<*samplechap1|samplechap2>
%\fi

% Optional override for |\version| flag:
%    \begin{macrocode}
%%\providecommand{\version}{final}
%    \end{macrocode}

% Include the main document:
%    \begin{macrocode}
\input{childdoc.def}
\childdocof{cdocsamp}
%    \end{macrocode}

%\iffalse
%</samplechap1|samplechap2>
%\fi
%
%\iffalse
%<*samplechap1>
%\fi
% Some text for chapter 1:
%    \begin{macrocode}
\section{one}
some text in chapter one
%    \end{macrocode}

%\iffalse
%</samplechap1>
%\fi
% Some text for chapter 2:
%\iffalse
%<*samplechap2>
%\fi
%    \begin{macrocode}
\section{two}
more text in chapter two
%    \end{macrocode}

%\iffalse
%</samplechap2>
%\fi
%
% %%%%%%%%%%%%%%%%%%%%%%%%%%%%%%%%%%%%%%
% \paragraph{Part Include Files.}
%
% The include files are called |cdocspt3.tex| and |cdocspt4.tex|.
%
%\iffalse
%<*samplepart3|samplepart4>
%\fi

% Optional override for |\version| flag:
%    \begin{macrocode}
%%\providecommand{\version}{final}
%    \end{macrocode}

% Include the main document:
%    \begin{macrocode}
\input{childdoc.def}
\childdocby{cdocsamp}
%    \end{macrocode}

%\iffalse
%</samplepart3|samplepart4>
%\fi
%
%\iffalse
%<*samplepart3>
%\fi
% Some text for part 3:
%    \begin{macrocode}
some text in part three
%    \end{macrocode}

%\iffalse
%</samplepart3>
%\fi
% Some text for part 4:
%\iffalse
%<*samplepart4>
%\fi
%    \begin{macrocode}
more text in part four
%    \end{macrocode}

%\iffalse
%</samplepart4>
%\fi
%
% %%%%%%%%%%%%%%%%%%%%%%%%%%%%%%%%%%%%%%
% \paragraph{Forwarding for a Complete Draft.}
%
% The following forwarding file |cdocsdrf.tex|
% compiles the main document in draft mode:
%\iffalse
%<*sampledraft>
%\fi
%    \begin{macrocode}
\def\version{draft}
\input{childdoc.def}
\childdocforward{cdocsamp}
%    \end{macrocode}

%\iffalse
%</sampledraft>
%\fi
%
% %%%%%%%%%%%%%%%%%%%%%%%%%%%%%%%%%%%%%%
% \paragraph{Forwarding for Final Version of the Chapters.}
%
% The following forwarding files |cdocsfn1.tex| and |cdocsfn2.tex|
% (with identical content)
% compile the final versions of the child documents
% |cdocsch1.tex| and |cdocsch2.tex|, respectively:
%\iffalse
%<*samplefinal>
%\fi
%    \begin{macrocode}
\def\version{final}
\input{childdoc.def}
\childdocforwardprefix[cdocsamp]{cdocsfn}{cdocsch}
%    \end{macrocode}

%\iffalse
%</samplefinal>
%\fi
%
% %%%%%%%%%%%%%%%%%%%%%%%%%%%%%%%%%%%%%%
% \paragraph{Command Line Processing.}
%
% The following three command lines generate the output files
% |cdocscld|, |cdocscl1| and |cdocscl2|
% which should be identical to
% |cdocsdrf|, |cdocsch1| and |cdocsfn2|, respectively:
% \begin{center}
% \begin{tabular}{l}
% |latex -jobname cdocscld \|\\
% |  "\def\version{draft}\input{childdoc.def}\childdocforward{cdocsamp}"|\\
% |latex -jobname cdocscl1 \|\\
% |  "\input{childdoc.def}\childdocforward[cdocsamp]{cdocsch1}"|\\
% |latex -jobname cdocscl2 \|\\
% |  "\def\version{final}\input{childdoc.def}\childdocforward{cdocsch2}"|
% \end{tabular}
% \end{center}
% Note that the trailing backslash on each first line
% merely continues the input to the second line
% (for convenient cut ant paste).
% Furthermore, the command |latex| can be replaced by any
% of its alternative versions such as |pdflatex|.
%
% %%%%%%%%%%%%%%%%%%%%%%%%%%%%%%%%%%%%%%%%%%%%%%%%%%%%%%%%%%%%%%%%%%%%%%%%%%%%%%
% %%%%%%%%%%%%%%%%%%%%%%%%%%%%%%%%%%%%%%%%%%%%%%%%%%%%%%%%%%%%%%%%%%%%%%%%%%%%%%
% \section{Implementation}
%\iffalse
%<*package>
%\fi
%
% This section describes the definitions file |childdoc.def|.

% The definitions cannot be loaded using |\usepackage| or |\RequirePackage|
% which has a mechanism to prevent loading a style file more than once.
% When loading the definitions by means of |\input|
% multiple instances have to be prevented manually:
%\iffalse
%This code needs to be before the `\ProvidesFile' directive
%which is defined at the beginning of this file.
%Therefore it is also placed there and commented out here.
%</package>
%<*discard>
%\fi
%    \begin{macrocode}
\ifdefined\childdocmain\endinput\fi
%    \end{macrocode}
%\iffalse
%</discard>
%<*package>
%\fi
%
% \macro{\ifchilddoc}
% \macro{\ifchilddocmanual}
% The conditional |\ifchilddoc| tells whether a
% child (true) or main (false) document is being compiled.
% The conditional |\ifchilddocmanual| tells whether
% the |\includeonly| mechanism is used (false) or
% the selection of child files must be performed manually (true).
% The definitions initialise to false:
%    \begin{macrocode}
\newif\ifchilddoc
\newif\ifchilddocmanual
%    \end{macrocode}

% \macro{\childdocname}
% \macro{\childdocjob}
% The macro |\childdocname| stores the name of the main document
% to be compiled. The macro |\childdocjob| stores the name of
% the document on which the \LaTeX{} compiler was originally invoked.
% The content of |\jobname| cannot be compared
% to filenames specified in the source due to different catcodes.
% The following code rescans |\jobname|, stores the result
% in |\childdocname| and saves a copy in |\childdocjob|:
%    \begin{macrocode}
\edef\childdocname{\scantokens\expandafter{\jobname\noexpand}}
\let\childdocjob\childdocname
%    \end{macrocode}

% \macro{\childdocdisable}
% The macro |\childdocdisable| prevents the main file
% from being processed more than once.
% At this stage, the main document command |\childdocmain|
% is assumed to be called once again where it should do nothing.
% Any subsequent call to it should prevent
% a secondary processing of the main document
% It overwrites the forwarding commands
% |\childdocof| and |\childdocforward|
% with empty macros to prevent further inclusions of the main document:
%    \begin{macrocode}
\newcommand{\childdocdisable}
{
  \renewcommand{\childdocmain}[1]{\renewcommand{\childdocmain}[1]{\endinput}}
  \renewcommand{\childdocof}[1]{}
  \renewcommand{\childdocby}[2][]{}
  \renewcommand{\childdocforward}[2][]{}
  \renewcommand{\childdocdisable}{}
}
%    \end{macrocode}

% \macro{\childdocmain}
% The macro |\childdocmain| is to be called at the top of the main file
% with nothing or the main filename (without extension) as argument.
% First, it breaks loops.
% If the argument is not empty and does not match |\childdocname|
% (which is set by the first inclusion of |childdoc.def|),
% |\ifchilddoc| is set to true, |\includeonly| is applied to the child file
% and |\jobname| is set to the main file
% (for proper handling of |.aux| files):
%    \begin{macrocode}
\newcommand{\childdocmain}[1]
{
  \childdocdisable\childdocmain{}
  \if?#1?\else
    \begingroup
      \def\childdoctmp{#1}
      \ifx\childdoctmp\childdocname
        \def\childdoctmp{}
      \else
        \def\childdoctmp
        {
          \childdoctrue
          \includeonly{\childdocname}
          \def\childdocjob{#1}
          \def\jobname{#1}
        }
      \fi
      \expandafter
    \endgroup
    \childdoctmp
  \fi
}
%    \end{macrocode}

% \macro{\childdocof}
% The command |\childdocof| redirects
% compilation to the main file |#1|.
%    \begin{macrocode}
\newcommand{\childdocof}[1]
{
  \childdocdisable
  \childdoctrue
  \includeonly{\childdocname}
  \def\jobname{#1}
  \def\childdocjob{#1}
  \input{#1}
}
%    \end{macrocode}

% \macro{\childdocby}
% The command |\childdocby| ....
%    \begin{macrocode}
\newcommand{\childdocby}[2][]
{
  \childdocdisable
  \childdoctrue
  \childdocmanualtrue
  \if?#1?\else
    \def\jobname{#2}
  \fi
  \def\childdocjob{#2}
  \input{#2}
  \endinput
}
%    \end{macrocode}

% \macro{\childdocforward}
% The command |\childdocforward| redirects
% compilation to the main file or
% (if the optional argument is given) a child file.
% Parameters are set as if the main file
% or a child file starting with |\childdocof| was compiled.
% Then compilation is handed over to the main file:
%    \begin{macrocode}
\newcommand{\childdocforward}[2][]
{
  \begingroup
    \if?#1?
      \def\childdoctmp
      {
        \def\childdocname{#2}
        \def\childdocjob{#2}
        \def\jobname{#2}
        \input{#2}
        \endinput
      }
    \else
      \def\childdoctmp
      {
        \childdocdisable
        \def\childdocname{#2}
        \childdoctrue
        \includeonly{#2}
        \def\childdocjob{#1}
        \def\jobname{#1}
        \input{#1}
        \endinput
      }
    \fi
    \expandafter
  \endgroup
  \childdoctmp
}
%    \end{macrocode}

% \macro{\childdocforwardprefix}
% The command |\childdocforwardprefix| redirects
% compilation to the main or a child file by means of a pattern.
% The prefix |#1| in the current filename is replaced by |#2|
% and the suffix of the current filename is kept
% (it is assumed that the filename does not contain the substring `|~~~|'
% which is used as a delimiter).
% Compilation is handed over to the new file by |\childdocforward|:
%    \begin{macrocode}
\newcommand{\childdocforwardprefix}[3][]
{
  \begingroup
    \def\childdocextract #2##1~~~{\def\childdoctmp{\childdocforward[#1]{#3##1}}}
    \expandafter\childdocextract\childdocname~~~
    \expandafter
  \endgroup
  \childdoctmp
}
%    \end{macrocode}

% \macro{\childdoc}
% The deprecated macro |\childdoc| is a legacy version of |\childdocmain|:
%    \begin{macrocode}
\newcommand{\childdoc}{\childdocmain}
%    \end{macrocode}

% \macro{\childdocredirect}
% The deprecated macro |\childdocredirect| is a legacy version
% of |\childdocforward| and |\childdocforwardprefix|:
%    \begin{macrocode}
\newcommand{\childdocredirect}[2][]
{
  \begingroup
    \if?#1?
      \def\childdoctmp{\childdocforward{#2}}
    \else
      \def\childdoctmp{\childdocforwardprefix{#1}{#2}}
    \fi
    \expandafter
  \endgroup
  \childdoctmp
}
%    \end{macrocode}

%\iffalse
%</package>
%\fi
%
\endinput

\childdocof{cdocsamp}
%    \end{macrocode}

%\iffalse
%</samplechap1|samplechap2>
%\fi
%
%\iffalse
%<*samplechap1>
%\fi
% Some text for chapter 1:
%    \begin{macrocode}
\section{one}
some text in chapter one
%    \end{macrocode}

%\iffalse
%</samplechap1>
%\fi
% Some text for chapter 2:
%\iffalse
%<*samplechap2>
%\fi
%    \begin{macrocode}
\section{two}
more text in chapter two
%    \end{macrocode}

%\iffalse
%</samplechap2>
%\fi
%
% %%%%%%%%%%%%%%%%%%%%%%%%%%%%%%%%%%%%%%
% \paragraph{Part Include Files.}
%
% The include files are called |cdocspt3.tex| and |cdocspt4.tex|.
%
%\iffalse
%<*samplepart3|samplepart4>
%\fi

% Optional override for |\version| flag:
%    \begin{macrocode}
%%\providecommand{\version}{final}
%    \end{macrocode}

% Include the main document:
%    \begin{macrocode}
% \iffalse
%
% childdoc.dtx Copyright (C) 2017-2018 Niklas Beisert
%
% This work may be distributed and/or modified under the
% conditions of the LaTeX Project Public License, either version 1.3
% of this license or (at your option) any later version.
% The latest version of this license is in
%   http://www.latex-project.org/lppl.txt
% and version 1.3 or later is part of all distributions of LaTeX
% version 2005/12/01 or later.
%
% This work has the LPPL maintenance status `maintained'.
%
% The Current Maintainer of this work is Niklas Beisert.
%
% This work consists of the files childdoc.dtx and childdoc.ins
% and the derived files childdoc.def and cdocsamp.tex with
% cdocsch1.tex, cdocsch2.tex, cdocsdrf.tex, cdocsfn1.tex, cdocsfn2.tex.
%
%<package>\ifdefined\childdocmain\endinput\fi
%<package>\ProvidesFile{childdoc.def}[2018/12/30 v2.0 child document driver]
%<samplemain>\ProvidesFile{cdocsamp.tex}[2018/12/30 v2.0 sample for childdoc]
%<*driver>
%\ProvidesFile{childdoc.drv}[2018/12/30 v2.0 childdoc reference manual file]
\PassOptionsToClass{10pt,a4paper}{article}
\documentclass{ltxdoc}

\usepackage[margin=35mm]{geometry}
\usepackage{hyperref}
\usepackage{hyperxmp}
\usepackage[usenames]{color}

\hypersetup{colorlinks=true}
\hypersetup{pdfstartview=FitH}
\hypersetup{pdfpagemode=UseNone}
\hypersetup{pdfsource={}}
\hypersetup{pdflang={en-UK}}
\hypersetup{pdfcopyright={Copyright 2017-2018 Niklas Beisert.
  This work may be distributed and/or modified under the
  conditions of the LaTeX Project Public License, either version 1.3
  of this license or (at your option) any later version.}}
\hypersetup{pdflicenseurl={http://www.latex-project.org/lppl.txt}}
\hypersetup{pdfcontactaddress={ETH Zurich, ITP, HIT K,
  Wolfgang-Pauli-Strasse 27}}
\hypersetup{pdfcontactpostcode={8093}}
\hypersetup{pdfcontactcity={Zurich}}
\hypersetup{pdfcontactcountry={Switzerland}}
\hypersetup{pdfcontactemail={nbeisert@itp.phys.ethz.ch}}
\hypersetup{pdfcontacturl={http://people.phys.ethz.ch/\xmptilde nbeisert/}}

\newcommand{\secref}[1]{\hyperref[#1]{section \ref*{#1}}}

\parskip1ex
\parindent0pt
\let\olditemize\itemize
\def\itemize{\olditemize\parskip0pt}

\begin{document}

\title{The \textsf{childdoc} Package}
\hypersetup{pdftitle={The childdoc Package}}
\author{Niklas Beisert\\[2ex]
  Institut f\"ur Theoretische Physik\\
  Eidgen\"ossische Technische Hochschule Z\"urich\\
  Wolfgang-Pauli-Strasse 27, 8093 Z\"urich, Switzerland\\[1ex]
  \href{mailto:nbeisert@itp.phys.ethz.ch}
  {\texttt{nbeisert@itp.phys.ethz.ch}}}
\hypersetup{pdfauthor={Niklas Beisert}}
\hypersetup{pdfsubject={Manual for the LaTeX2e Package childdoc}}
\date{30 December 2018, \textsf{v2.0}}
\maketitle

\begin{abstract}\noindent
\textsf{childdoc} is a \LaTeXe{} package
that enables the direct compilation
of document sections included by |\include|
to individual files.
\end{abstract}

\begingroup
\parskip0ex
\tableofcontents
\endgroup

%%%%%%%%%%%%%%%%%%%%%%%%%%%%%%%%%%%%%%%%%%%%%%%%%%%%%%%%%%%%%%%%%%%%%%%%%%%%%%%%
%%%%%%%%%%%%%%%%%%%%%%%%%%%%%%%%%%%%%%%%%%%%%%%%%%%%%%%%%%%%%%%%%%%%%%%%%%%%%%%%
\section{Introduction}

\LaTeX{} provides a mechanism to structure a large document (such as a book)
into a main file and several child files (containing the chapters)
using the |\include| command.
This mechanism is beneficial for documents
which span hundreds of pages in order to
make the source file(s) more manageable.
Moreover, compilation can be restricted to
selected child files by means of the |\includeonly| command.
The latter feature can be used to reduce the compilation time while editing
(this was significantly more useful in the earlier days of \LaTeX{})
or to generate a smaller document which is easier to navigate.
Another application of |\includeonly| is to generate
documents consisting of selected parts of the complete document.

However, there are a few drawbacks of the plain |\include| mechanism:
\begin{itemize}
\item
The child files cannot be compiled on their own,
they can only be compiled via the main file.
A naive editing environment
(such as a text editor with an option
to have the current file processed by \LaTeX)
may require one to switch to the main file before compiling;
attempting to compile the child file produces errors.
\item
The main file must be modified (each time)
to adjust the |\includeonly| command
to the present needs. This easily leaves the main file in a messy state.
\item
The generated document will always carry the filename
of the main document. This is inconvenient if
several child files are to be compiled and
to be kept for distribution.
\end{itemize}

The present package provides a simple interface
to make child files individually compilable by \LaTeX{}.
Compiling a child file then has the same effect as compiling
the main file with an |\includeonly| command
to select the appropriate child.
Moreover the generated document will carry the name of the child
rather than the main file.
This resolves all three above issues.

This feature is meant to make the editing of books,
thesis documents and lecture notes somewhat more convenient.
However, the package can also be used efficiently for
composing a series of documents (such as exercise sheets)
which are typically distributed individually.
It then assists the author in generating the individual documents
(potentially in different versions)
as well as a document containing the collected series.
Another application is in developing style files
or other kinds of included material
where compilation of the style file could redirect
to a sample or test file.

%%%%%%%%%%%%%%%%%%%%%%%%%%%%%%%%%%%%%%%%%%%%%%%%%%%%%%%%%%%%%%%%%%%%%%%%%%%%%%%%
%%%%%%%%%%%%%%%%%%%%%%%%%%%%%%%%%%%%%%%%%%%%%%%%%%%%%%%%%%%%%%%%%%%%%%%%%%%%%%%%
\section{Usage}

First of all, the package \textsf{childdoc} is \emph{not} a standard
\LaTeXe{} |.sty| style file! Therefore it needs to be invoked in
a non-standard way.

%%%%%%%%%%%%%%%%%%%%%%%%%%%%%%%%%%%%%%%%%%%%%%%%%%%%%%%%%%%%%%%%%%%%%%%%%%%%%%%%
\subsection{Included Files}
\label{sec:include}

%%%%%%%%%%%%%%%%%%%%%%%%%%%%%%%%%%%%%%%%
\DescribeMacro{\childdocmain}
To use the package, add the commands
\begin{center}
\begin{tabular}{l}
|\input{childdoc.def}|\\
|\childdocmain{}|\\
\end{tabular}
\end{center}
at the very top of the main \LaTeX{} file,
in particular \emph{before} the |\documentclass| statement!
The argument of |\childdocmain| should be left empty
(but it must be present).

%%%%%%%%%%%%%%%%%%%%%%%%%%%%%%%%%%%%%%%%
\DescribeMacro{\childdocof}
Furthermore, add the commands
\begin{center}
\begin{tabular}{l}
|\input{childdoc.def}|\\
|\childdocof{|\textit{main}|}|\\
\end{tabular}
\end{center}
at the top of every child file \textit{child}
which is included by |\include{|\textit{child}|}|
from within the main file
(or at least for those files to be compiled individually).
The argument \textit{main} must be the filename of the main file.

There are a couple of
considerations in setting up the main and child documents:

%%%%%%%%%%%%%%%%%%%%%%%%%%%%%%%%%%%%%%%%
\paragraph{Restrictions.}

Please note the following restrictions:
\begin{itemize}
\item
|\childdocmain| must be called with one argument \textit{main}
to ensure compatibility with earlier version of the package.
It must either be empty (|\childdocmain{}|)
or precisely match the filename of the main file in which it is specified.
See \secref{sec:detection} for further information.
\item
The filename \textit{main} must be specified without the |.tex| extension.
\item
The filename \textit{main} is case sensitive
(even in case-insensitive file systems)
due to internal string comparison.
\item
The argument \textit{main} should be fully expanded, it cannot be a macro.
\item
Subdirectories and special characters should be avoided in filenames.
\item
The command |\childdocmain{|\textit{main}|}| must be followed by a whitespace.
It should not be followed immediately by another command
or by a comment mark `|%|'.
This is because the \TeX{} parser reads the token immediately following
the argument of |\childdocmain| and puts it
at the beginning of every child section;
however, a white\-space is ignored.
\end{itemize}

%%%%%%%%%%%%%%%%%%%%%%%%%%%%%%%%%%%%%%%%
\paragraph{Content of Main File.}

It is advisable to place all content in the child files included by |\include|.
Any output contained in the main file will appear in all child documents
unless suppressed manually;
it cannot be suppressed automatically by the |\includeonly| directive
and thus should normally be avoided.
A method to include some content in the main file
by means of conditional processing is described in \secref{sec:conditional}.

%%%%%%%%%%%%%%%%%%%%%%%%%%%%%%%%%%%%%%%%
\paragraph{Page Numbering.}

When only a part of the document is compiled,
the appropriate numbering of pages
(as well as other status parameters)
is determined from the |.aux| files.
The latter contain information from previous passes.
However this information needs to propagate through
all intermediate child documents.
Therefore the page numbering in child documents may well
be inconsistent until the complete document is compiled at least once.

A useful (if unconventional) way to always ensure a consistent
page numbering is to restart the numbering in each child document
and denote the pages by `\textit{child}|.|\textit{page}'
where \textit{child} represents the chapter/section number of the child file.
This can be achieved by the command
|\numberwithin{page}{|\textit{child}|}|
of the \textsf{amsmath} package
where \textit{child} can be |chapter| or |section|
depending on the chosen structuring.
Alternatively, one can modify the macro |\thepage| appropriately
and reset the counter |page| at the start of each child file.

%%%%%%%%%%%%%%%%%%%%%%%%%%%%%%%%%%%%%%%%%%%%%%%%%%%%%%%%%%%%%%%%%%%%%%%%%%%%%%%%
\subsection{Conditional Processing}
\label{sec:conditional}

The package provides a mechanism to compile different versions
of a document. To customise the versions further some conditional processing
can come in handy to distinguish which version is being compiled.
The package provides two macros to describe the compilation context:

%%%%%%%%%%%%%%%%%%%%%%%%%%%%%%%%%%%%%%%%
\DescribeMacro{\ifchilddoc}
The conditional |\ifchilddoc| distinguishes between the compilation of
child documents and the main document:
%
\begin{center}
|\ifchilddoc |\textit{child-code}| |[|\||else |\textit{main-code}]| \||fi|
\end{center}

%%%%%%%%%%%%%%%%%%%%%%%%%%%%%%%%%%%%%%%%
\DescribeMacro{\childdocname}
\DescribeMacro{\childdocjob}
The macro |\childdocname| contains the filename (without extension)
of the main or child file being processed.
Note that |\childdocjob| will always contain the name of the main file.

%%%%%%%%%%%%%%%%%%%%%%%%%%%%%%%%%%%%%%%%
\paragraph{Title Page.}

Conditional processing can be used to include a title or banner page
in the main document when proper precautions are taken.
Importantly, the code in the main file should ensure that the page counter
(as well as other status parameters which are stored in the |.aux| files)
takes the same value after the conditional processing.
Otherwise the page numbers may take divergent values
depending on which part is compiled.

For example, a title page could be declared by:
%
\begin{center}
\begin{tabular}{l}
|\ifchilddoc\||else|\\
|\addtocounter{page}{-1}|\\
\textit{code for title page}\\
|\newpage|\\
|\||fi|
\end{tabular}
\end{center}
%
A banner page for the child documents can be generated by:
%
\begin{center}
\begin{tabular}{l}
|\ifchilddoc|\\
|\addtocounter{page}{-1}|\\
\textit{code for banner page}\\
|\newpage|\\
|\||fi|
\end{tabular}
\end{center}
%
Here one could write a message such as:
\begin{center}
|This is the part \childdocname{} of \childdocjob{}.|
\end{center}

%%%%%%%%%%%%%%%%%%%%%%%%%%%%%%%%%%%%%%%%%%%%%%%%%%%%%%%%%%%%%%%%%%%%%%%%%%%%%%%%
\subsection{Flags}
\label{sec:flags}

The package makes it easy to generate different versions
of the main or child documents.
To this end compilation flags can be defined
and assigned different default values.
They will be particularly useful in conjunction
with the forwarding mechanism described in \secref{sec:forward}.

For example, it may be useful to have a flag |\version|
which can be set to |draft| or |final|.
The document source will contain some conditional code
depending on the value of |\version|.
Suppose further, the flag should default to |final| for the main file
and to |draft| for child files
which is a natural assignment for editing the document.
This is achieved by placing the following code
in the preamble of the main document
(below the |\childdocmain| directive):
%
\begin{center}
\begin{tabular}{l}
|\ifchilddoc|\\
|\providecommand{\version}{draft}|\\
|\||else|\\
|\providecommand{\version}{final}|\\
|\||fi|
\end{tabular}
\end{center}
%
The definition by |\providecommand| makes sure
that previous definitions are not overwritten.
Further statements |\providecommand{\version}{...}|
can thus be added before the above code to override it.

For the main file, one might add a line
(between |\childdocmain| and the above block)
%
\begin{center}
|%\ifchilddoc\||else\providecommand{\version}{draft}\||fi|
\end{center}
%
which can be uncommented to produce a draft version.
Likewise one can add a line to the very top of a child file
(above the |\childdocof{|\textit{main}|}| directive)
%
\begin{center}
|%\providecommand{\version}{final}|
\end{center}
%
which can be uncommented to produce the final version of this child document.

%%%%%%%%%%%%%%%%%%%%%%%%%%%%%%%%%%%%%%%%%%%%%%%%%%%%%%%%%%%%%%%%%%%%%%%%%%%%%%%%
\subsection{Forwarding}
\label{sec:forward}

Different versions of the main or child documents
using compilation flags as described in \secref{sec:flags}
can be (permanently) stored in different files
for convenient compilation, viewing and distribution.
To this end, the package defines a command
to pass on compilation to a different file:

%%%%%%%%%%%%%%%%%%%%%%%%%%%%%%%%%%%%%%%%
\DescribeMacro{\childdocforward}
The command |\childdocforward| redirects processing to
another source file:
%
\begin{center}
\begin{tabular}{l}
|\input{childdoc.def}|\\
|\childdocforward[|\textit{main}|]{|\textit{dest}|}|\\
\end{tabular}
\end{center}
%
The argument \textit{dest} is the destination file
(without extension).
It should be the main file or one of the child files.
Note that further \textsf{childdoc} directives
such as |\childdocof| and |\childdocforward|
in the indicated file will be processed in this form.
The optional argument \textit{main}
passes on directly to the main file \textit{main}
while pretending to compile the child \textit{dest}.
This form behaves as if \textit{dest}
issues |\childdocof{|\textit{main}|}| right away,
and no further \textsf{childdoc} directives will be processed.

%%%%%%%%%%%%%%%%%%%%%%%%%%%%%%%%%%%%%%%%
\DescribeMacro{\...prefix}
In the alternative form |\childdocforwardprefix|,
%
\begin{center}
\begin{tabular}{l}
|\input{childdoc.def}|\\
|\childdocforwardprefix[|\textit{main}|]{|\textit{prefix}|}{|\textit{dest}|}|
\end{tabular}
\end{center}
%
the destination file is determined by a pattern
depending on the current file:
To make this work, the current file must be called
`{\textit{prefix}\hspace{0.2em}\textit{suffix}}'
with \textit{prefix} matching precisely the argument.
Processing is then passed on to the file
`{\textit{dest}\hspace{0.2em}\textit{suffix}}'.
Surely, the same effect is achieved by
directly specifying the
argument `{\textit{dest}\hspace{0.2em}\textit{suffix}}'
in the first form.
However, that requires to set up a different file
for each child. With the alternative form of the command
all these files can have exactly the same content
which simplifies setting them up and maintaining them.

For example, the following file |draft.tex|
with a compilation flag |\version| as described in \secref{sec:flags}
compiles the main document as a draft:
%
\begin{center}
\begin{tabular}{l}
|\def\version{draft}|\\
|\input{childdoc.def}|\\
|\childdocforward{|\textit{main}|}|
\end{tabular}
\end{center}
%
Likewise, the following files |final|\textit{nn}|.tex|
compile the final version of the child document
|child|\textit{nn}|.tex|:
%
\begin{center}
\begin{tabular}{l}
|\def\version{final}|\\
|\input{childdoc.def}|\\
|\childdocforwardprefix{final}{child}|
\end{tabular}
\end{center}
%

Note that when several versions of a main file and/or of each child file
are to be generated, it may be convenient to set up a |Makefile| or
shell script to automatise the process.

%%%%%%%%%%%%%%%%%%%%%%%%%%%%%%%%%%%%%%%%%%%%%%%%%%%%%%%%%%%%%%%%%%%%%%%%%%%%%%%%
\subsection{Command Line Processing}
\label{sec:commandline}

The effect of redirection files can also be achieved by invoking
the \LaTeX{} compiler with a more elaborate command line.
Most conveniently this should be done as part
of a shell script or a |Makefile|.

When using \textsf{childdoc} in the main file, the following
command lines effectively perform a redirection
(note that depending on the shell being used,
backslashes may have to be doubled: `|\|' $\to$ `|\\|'):
%
\begin{center}
|... -jobname "|\textit{target}|" |\\|"|[\textit{flags}]%
|\input{childdoc.def}\childdocforward[|\textit{main}|]{|\textit{dest}|}"|
\end{center}
%
Here \textit{target} is the name of the output file,
\textit{main} is the name of the main file
and \textit{dest} is the name of the main or child file to be processed
(all filenames without extensions).
The optional argument \textit{main} can be omitted
if \textit{main} matches \textit{dest}.
Optionally, compilation \textit{flags} can be defined via |\def| commands.
This command line makes the \TeX{} engine believe
it is compiling the file \textit{target}
whose content is specified as the latter parameter.
The provided code then forwards the processing to
\textit{main} or \textit{dest} as described in \secref{sec:forward}.

%%%%%%%%%%%%%%%%%%%%%%%%%%%%%%%%%%%%%%%%%%%%%%%%%%%%%%%%%%%%%%%%%%%%%%%%%%%%%%%%
\subsection{Include by Input}
\label{sec:input}

Including child documents by |\include| has some restrictions by design.
Most notably, the content of a child document always occupies
its own set of pages; pages cannot be shared between child documents.
Usually, this behaviour makes perfect sense
because each child document contain an essential part of the document.
However, in some situations it may be desirable to compose
a document from a collection of parts
without having mandatory page breaks between then.
For this case, the package
provides a mechanism to include parts
by |\input| which can also be processed individually.
However, by construction this mechanism
requires manual handling of the content to be output.

%%%%%%%%%%%%%%%%%%%%%%%%%%%%%%%%%%%%%%%%
\DescribeMacro{\ifchilddocmanual}
The main file should be prepared as usual, see \secref{sec:include}.
However, the document body must make a distinction
between processing of an individual part and of the main document, e.g.:
%
\begin{center}
\begin{tabular}{l}
|\ifchilddocmanual|\\
|\input{\childdocname}|\\
|\||else|\\
\textit{document body with }|\input{|\textit{part}|}|\\
|\||fi|
\end{tabular}
\end{center}
%
The conditional |\ifchilddocmanual| is true whenever
a part to be included by |\input| is being compiled,
and the name of the part is stored in |\childdocname|.

%%%%%%%%%%%%%%%%%%%%%%%%%%%%%%%%%%%%%%%%
\DescribeMacro{\childdocby}
Each part to be included by |\input| should start with:
%
\begin{center}
\begin{tabular}{l}
|\input{childdoc.def}|\\
|\childdocby{|\textit{main}|}|\\
\end{tabular}
\end{center}
%
The directive |\childdocby| is similar to |\childdocof|
described in \secref{sec:include},
but the subsequent selection of content must be done manually.
To that end, both |\ifchilddoc| and |\ifchilddocmanual|
will be true upon processing of a part,
and the name of the part is stored in |\childdocname|.
Note that |\jobname| will be set to the filename of the current part
so that each part receives an individual |.aux| file
that does not interfere with the |.aux| file(s) of the main document.
This behaviour can be altered by the alternative form
|\childdocby[*]{|\textit{main}|}| (with a non-empty optional argument)
which uses the |.aux| file of the main document
by setting |\jobname| to \textit{main}.

%%%%%%%%%%%%%%%%%%%%%%%%%%%%%%%%%%%%%%%%%%%%%%%%%%%%%%%%%%%%%%%%%%%%%%%%%%%%%%%%
\subsection{Driver Development}
\label{sec:driver}

The \textsf{childdoc} mechanism can also be use for the development
of definition files such as \LaTeX{} styles or classes.
This case differs from the above setup with multiple parts
included by |\include| in that no |\includeonly| should be invoked.
This can be achieved by starting the include file
(before |\ProvidesPackage|) with:
%
\begin{center}
\begin{tabular}{l}
|\input{childdoc.def}|\\
|\childdocforward{|\textit{main}|}|\\
\end{tabular}
\end{center}
%
or alternatively with:
%
\begin{center}
\begin{tabular}{l}
|\input{childdoc.def}|\\
|\childdocby{|\textit{main}|}|\\
\end{tabular}
\end{center}
%
Both forms have slightly different effects as described above.
The main file is prepared as usual, see \secref{sec:include}.

%%%%%%%%%%%%%%%%%%%%%%%%%%%%%%%%%%%%%%%%%%%%%%%%%%%%%%%%%%%%%%%%%%%%%%%%%%%%%%%%
\subsection{Legacy Detection}
\label{sec:detection}

The directive |\childdocmain| in the main file can detect
whether the complete document or merely a child is to be compiled
even without using the directive |\childdocof|.
This method is deprecated because it is less robust
and there is no compelling reason to use it;
it is merely provided for backward compatibility
and it may be removed in future versions.

If the detection mechanism is to be used,
it is mandatory to correctly specify
the filename of the main file as the argument of |\childdocmain|:
%
\begin{center}
\begin{tabular}{l}
|\input{childdoc.def}|\\
|\childdocmain{|\textit{main}|}|\\
\end{tabular}
\end{center}
%
If |\jobname| does not match the argument \textit{main} of |\childdocmain|,
it is assumed that |\jobname| points to the child file to be compiled.
When using |\childdocmain| with the main file specified as argument,
it suffices to start a child file
with just |\input{|\textit{main}|}|
without loading of the package and using |\childdocof|.
If instead all processing is done
with the appropriate \textsf{childdoc} directives,
the argument of \textit{main} of |\childdocmain| can be empty.

An alternative version of the command line processing described
in \secref{sec:commandline} using the detection mechanism reads:
%
\begin{center}
|... -jobname "|\textit{target}|" "|[\textit{flags}]%
[|\def\jobname{|\textit{dest}|}|]|\input{|\textit{main}|}"|
\end{center}

%%%%%%%%%%%%%%%%%%%%%%%%%%%%%%%%%%%%%%%%%%%%%%%%%%%%%%%%%%%%%%%%%%%%%%%%%%%%%%%%
\subsection{Manual Code}
\label{sec:manual}

In case one cannot be certain whether the definitions file |childdoc.def|
is installed on the target \TeX{} distribution
and one prefers not to ship it,
it is conceivable to paste a few relevant commands into the sources.

To that end, drop all statements |\input{childdoc.def}|
and perform the replacements as outlined below.
Instead of |\childdocmain{|\textit{main}|}| add the following code
to the top of the main file:
%
\begin{center}
\begin{tabular}{l}
|\||ifdefined\childdocname\endinput\||fi\newif\ifchilddoc|\\
|\edef\childdocname{\scantokens\expandafter{\jobname\noexpand}}|\\
|\def\childdocmain{|\textit{main}|}\||ifx\childdocmain\childdocname\||else|\\
|\childdoctrue\includeonly{\childdocname}\let\jobname\childdocmain\||fi|\\
\end{tabular}
\end{center}
%
Instead of |\childdocof{|\textit{main}|}| just include the main file
at the top of each child file:
%
\begin{center}
|\input{|\textit{main}|}|
\end{center}
%
A simple redirection |\childdocforward{|\textit{dest}|}| is achieved by:
%
\begin{center}
|\def\jobname{|\textit{dest}|}\input{\jobname}|
\end{center}
%
The redirection with prefix
|\childdocforwardprefix[|\textit{prefix}|]{|\textit{dest}|}|
is accomplished by:
%
\begin{center}
\begin{tabular}{l}
|{\edef\jobname{\scantokens\expandafter{\jobname\noexpand}}|\\
|\def\redirectjob |\textit{prefix}|#1~~~{\gdef\jobname{|\textit{dest}|#1}}|\\
|\expandafter\redirectjob\jobname~~~}\input{\jobname}|
\end{tabular}
\end{center}

In an alternative approach,
child documents can be compiled by a specific command line
without additional code or specific definitions:
%
\begin{center}
|... -jobname "|\textit{target}|" "|[\textit{flags}]%
|\includeonly{|\textit{dest}|}\input{|\textit{main}|}"|
\end{center}
%

%%%%%%%%%%%%%%%%%%%%%%%%%%%%%%%%%%%%%%%%%%%%%%%%%%%%%%%%%%%%%%%%%%%%%%%%%%%%%%%%
%%%%%%%%%%%%%%%%%%%%%%%%%%%%%%%%%%%%%%%%%%%%%%%%%%%%%%%%%%%%%%%%%%%%%%%%%%%%%%%%
\section{Information}

%%%%%%%%%%%%%%%%%%%%%%%%%%%%%%%%%%%%%%%%%%%%%%%%%%%%%%%%%%%%%%%%%%%%%%%%%%%%%%%%
\subsection{Copyright}

Copyright \copyright{} 2017--2018 Niklas Beisert

This work may be distributed and/or modified under the
conditions of the \LaTeX{} Project Public License, either version 1.3
of this license or (at your option) any later version.
The latest version of this license is in
  \url{http://www.latex-project.org/lppl.txt}
and version 1.3 or later is part of all distributions of \LaTeX{}
version 2005/12/01 or later.

This work has the LPPL maintenance status `maintained'.

The Current Maintainer of this work is Niklas Beisert.

This work consists of the files |README.txt|, |childdoc.ins| and |childdoc.dtx|
as well as the derived files |childdoc.def|, |cdocsamp.tex|
with |cdocsch1.tex|, |cdocsch2.tex|, |cdocspt3.tex|, |cdocspt4.tex|,
|cdocsdrf.tex|, |cdocsfn1.tex|, |cdocsfn2.tex|
as well as |childdoc.pdf|.

%%%%%%%%%%%%%%%%%%%%%%%%%%%%%%%%%%%%%%%%%%%%%%%%%%%%%%%%%%%%%%%%%%%%%%%%%%%%%%%%
\subsection{Files and Installation}

The package consists of the files:
%
\begin{center}
\begin{tabular}{ll}
    |README.txt|   & readme file \\
    |childdoc.ins| & installation file \\
    |childdoc.dtx| & source file \\
    |childdoc.def| & definition file \\
    |cdocsamp.tex| & sample main file \\
    |cdocsch1.tex| & sample include file \\
    |cdocsch2.tex| & sample include file \\
    |cdocspt3.tex| & sample part file \\
    |cdocspt4.tex| & sample part file \\
    |cdocsdrf.tex| & sample redirection file \\
    |cdocsfn1.tex| & sample redirection file \\
    |cdocsfn2.tex| & sample redirection file \\
    |childdoc.pdf| & manual
\end{tabular}
\end{center}
%
The distribution consists of the files
|README.txt|, |childdoc.ins| and |childdoc.dtx|.
%
\begin{itemize}
\item
Run (pdf)\LaTeX{} on |childdoc.dtx|
to compile the manual |childdoc.pdf| (this file).
\item
Run \LaTeX{} on |childdoc.ins| to create the definitions file |childdoc.def|
and the sample |cdocsamp.tex| with include files
|cdocsch1.tex|, |cdocsch2.tex|, |cdocspt3.tex|, |cdocspt4.tex|,
|cdocsdrf.tex|, |cdocsfn1.tex|, |cdocsfn2.tex|.
Then copy the file |childdoc.def| to an appropriate directory of your \LaTeX{}
distribution, e.g.\ \textit{texmf-root}|/tex/latex/childdoc|.
\end{itemize}

%%%%%%%%%%%%%%%%%%%%%%%%%%%%%%%%%%%%%%%%%%%%%%%%%%%%%%%%%%%%%%%%%%%%%%%%%%%%%%%%
\subsection{Related CTAN Packages}

There are several other packages which offer a similar functionality:
%
\begin{itemize}
\item
The packages
\href{http://ctan.org/pkg/docmute}{\textsf{docmute}},
\href{http://ctan.org/pkg/includex}{\textsf{includex}} and
\href{http://ctan.org/pkg/standalone}{\textsf{standalone}}
provide commands to include only the document body of
a child file thus allowing both files to be compiled individually.
\item
The packages \href{http://ctan.org/pkg/subdocs}{\textsf{subdocs}}
and \href{http://ctan.org/pkg/subfiles}{\textsf{subfiles}}
provide structures in which the main and child documents can be
encapsulated and allowing them to be compiled individually.
The inclusion mechanism is different from the conventional |\include|.
\item
The package \href{http://ctan.org/pkg/combine}{\textsf{combine}}
is an elaborate solution to combine several documents into one.
\end{itemize}
%
See also the CTAN topic \href{http://ctan.org/topic/subdocs}{\textsf{subdocs}}
for further related packages.
The present package differs from the above solutions in that
a document structure constructed with the conventional |\include| mechanism
just needs two extra commands at the top of every file
such that all constituent files can be compiled individually.

%%%%%%%%%%%%%%%%%%%%%%%%%%%%%%%%%%%%%%%%%%%%%%%%%%%%%%%%%%%%%%%%%%%%%%%%%%%%%%%%
%\subsection{Feature Suggestions}
%
%The following is a list of features which may be useful for future
%versions of this package:
%%
%\begin{itemize}
%\item
%\ldots
%\end{itemize}

%%%%%%%%%%%%%%%%%%%%%%%%%%%%%%%%%%%%%%%%%%%%%%%%%%%%%%%%%%%%%%%%%%%%%%%%%%%%%%%%
\subsection{Revision History}

%%%%%%%%%%%%%%%%%%%%%%%%%%%%%%%%%%%%%%%%
\paragraph{v2.0:} 2018/12/30

\begin{itemize}
\item
immediate forward processing
\item
added |\childdocby| mechanism
\item
manual restructured
\end{itemize}

%%%%%%%%%%%%%%%%%%%%%%%%%%%%%%%%%%%%%%%%
\paragraph{v1.6:} 2018/01/17

\begin{itemize}
\item
application for development of include files
\item
corrections to manual
\end{itemize}

%%%%%%%%%%%%%%%%%%%%%%%%%%%%%%%%%%%%%%%%
\paragraph{v1.5:} 2017/05/21

\begin{itemize}
\item
more complete structuring introduced
\item
|\childdocof| introduced
\item
|\childdoc| renamed to |\childdocmain|
\item
|\childredirect| renamed to |\childdocforward| and |\childdocforwardprefix|
and functionality expanded
\end{itemize}

%%%%%%%%%%%%%%%%%%%%%%%%%%%%%%%%%%%%%%%%
\paragraph{v1.0:} 2017/04/27

\begin{itemize}
\item
manual and install package
\item
first version published on CTAN
\end{itemize}

%%%%%%%%%%%%%%%%%%%%%%%%%%%%%%%%%%%%%%%%
\paragraph{v0.6:} 2017/04/26

\begin{itemize}
\item
redirection mechanism added
\end{itemize}

%%%%%%%%%%%%%%%%%%%%%%%%%%%%%%%%%%%%%%%%
\paragraph{v0.5:} 2017/04/26

\begin{itemize}
\item
functionality in definition file
\end{itemize}


%%%%%%%%%%%%%%%%%%%%%%%%%%%%%%%%%%%%%%%%%%%%%%%%%%%%%%%%%%%%%%%%%%%%%%%%%%%%%%%%
%%%%%%%%%%%%%%%%%%%%%%%%%%%%%%%%%%%%%%%%%%%%%%%%%%%%%%%%%%%%%%%%%%%%%%%%%%%%%%%%
%%%%%%%%%%%%%%%%%%%%%%%%%%%%%%%%%%%%%%%%%%%%%%%%%%%%%%%%%%%%%%%%%%%%%%%%%%%%%%%%
\appendix

\settowidth\MacroIndent{\rmfamily\scriptsize 000\ }

 \DocInput{childdoc.dtx}

\end{document}
%</driver>
% \fi
%
% %%%%%%%%%%%%%%%%%%%%%%%%%%%%%%%%%%%%%%%%%%%%%%%%%%%%%%%%%%%%%%%%%%%%%%%%%%%%%%
% %%%%%%%%%%%%%%%%%%%%%%%%%%%%%%%%%%%%%%%%%%%%%%%%%%%%%%%%%%%%%%%%%%%%%%%%%%%%%%
% \section{Sample}
%\iffalse
%<*samplemain>
%\fi
%
% The following presents a sample document
% with two chapters, two parts, a title page,
% a compile flag as well as three forwarding files to set the flag.
% It consists of eight |.tex| files:
% \begin{center}
% \begin{tabular}{ll}
% |cdocsamp.tex|&main file\\
% |cdocsch1.tex|&include file for chapter 1\\
% |cdocsch2.tex|&include file for chapter 2\\
% |cdocspt3.tex|&include file for part 3\\
% |cdocspt4.tex|&include file for part 4\\
% |cdocsdrf.tex|&forwarding file for main file in draft mode\\
% |cdocsfi1.tex|&forwarding file for final version of chapter 1\\
% |cdocsfi2.tex|&forwarding file for final version of chapter 2\\
% \end{tabular}
% \end{center}
% Each of the eight files can be compiled directly by the \LaTeX{} compiler.
%
% %%%%%%%%%%%%%%%%%%%%%%%%%%%%%%%%%%%%%%
% \paragraph{Main File.}
%
% The main file is called |cdocsamp.tex|.
%
% Load the \textsf{childdoc} definitions and
% declare the filename for the main document:
%    \begin{macrocode}
\input{childdoc.def}
\childdocmain{}
%    \end{macrocode}

% Optional override for |\version| flag:
%    \begin{macrocode}
%%\ifchilddoc\else\providecommand{\version}{draft}\fi
%    \end{macrocode}

% Define the default values for the |\version| flag
% (|final| for the main file and |draft| for childs):
%    \begin{macrocode}
\ifchilddoc
\providecommand{\version}{draft}
\else
\providecommand{\version}{final}
\fi
%    \end{macrocode}

% Load the standard document class:
%    \begin{macrocode}
\documentclass[12pt]{article}
%    \end{macrocode}

% Start the document body:
%    \begin{macrocode}
\begin{document}
%    \end{macrocode}

% Declare a title page.
% Print title, part of document being processed and version flag:
%    \begin{macrocode}
\addtocounter{page}{-1}
\begin{center}
{\LARGE\bfseries{}childdoc example\par}
\vspace{1cm}
\ifchilddoc
\ifchilddocmanual part\else chapter\fi:
`\childdocname' of `\childdocjob'\par
\else
main document: `\childdocjob'\par
\fi
version: \version\par
\end{center}
\newpage
%    \end{macrocode}

% Manually include selected file,
% otherwise process as usual:
%    \begin{macrocode}
\ifchilddocmanual
\section*{part `\childdocname'}
\input{\childdocname}
\else
%    \end{macrocode}

% Include the two chapters:
%    \begin{macrocode}
\include{cdocsch1}
\include{cdocsch2}
%    \end{macrocode}

% Include the two parts unless only chapters should be displayed:
%    \begin{macrocode}
\ifchilddoc\else
\section{part three}
\input{cdocspt3}
\section{part four}
\input{cdocspt4}
\fi
%    \end{macrocode}

% Process as usual until here:
%    \begin{macrocode}
\fi
%    \end{macrocode}

% End of document body:
%    \begin{macrocode}
\end{document}
%    \end{macrocode}
%\iffalse
%</samplemain>
%\fi
%
% %%%%%%%%%%%%%%%%%%%%%%%%%%%%%%%%%%%%%%
% \paragraph{Chapter Include Files.}
%
% The include files are called |cdocsch1.tex| and |cdocsch2.tex|.
%
%\iffalse
%<*samplechap1|samplechap2>
%\fi

% Optional override for |\version| flag:
%    \begin{macrocode}
%%\providecommand{\version}{final}
%    \end{macrocode}

% Include the main document:
%    \begin{macrocode}
\input{childdoc.def}
\childdocof{cdocsamp}
%    \end{macrocode}

%\iffalse
%</samplechap1|samplechap2>
%\fi
%
%\iffalse
%<*samplechap1>
%\fi
% Some text for chapter 1:
%    \begin{macrocode}
\section{one}
some text in chapter one
%    \end{macrocode}

%\iffalse
%</samplechap1>
%\fi
% Some text for chapter 2:
%\iffalse
%<*samplechap2>
%\fi
%    \begin{macrocode}
\section{two}
more text in chapter two
%    \end{macrocode}

%\iffalse
%</samplechap2>
%\fi
%
% %%%%%%%%%%%%%%%%%%%%%%%%%%%%%%%%%%%%%%
% \paragraph{Part Include Files.}
%
% The include files are called |cdocspt3.tex| and |cdocspt4.tex|.
%
%\iffalse
%<*samplepart3|samplepart4>
%\fi

% Optional override for |\version| flag:
%    \begin{macrocode}
%%\providecommand{\version}{final}
%    \end{macrocode}

% Include the main document:
%    \begin{macrocode}
\input{childdoc.def}
\childdocby{cdocsamp}
%    \end{macrocode}

%\iffalse
%</samplepart3|samplepart4>
%\fi
%
%\iffalse
%<*samplepart3>
%\fi
% Some text for part 3:
%    \begin{macrocode}
some text in part three
%    \end{macrocode}

%\iffalse
%</samplepart3>
%\fi
% Some text for part 4:
%\iffalse
%<*samplepart4>
%\fi
%    \begin{macrocode}
more text in part four
%    \end{macrocode}

%\iffalse
%</samplepart4>
%\fi
%
% %%%%%%%%%%%%%%%%%%%%%%%%%%%%%%%%%%%%%%
% \paragraph{Forwarding for a Complete Draft.}
%
% The following forwarding file |cdocsdrf.tex|
% compiles the main document in draft mode:
%\iffalse
%<*sampledraft>
%\fi
%    \begin{macrocode}
\def\version{draft}
\input{childdoc.def}
\childdocforward{cdocsamp}
%    \end{macrocode}

%\iffalse
%</sampledraft>
%\fi
%
% %%%%%%%%%%%%%%%%%%%%%%%%%%%%%%%%%%%%%%
% \paragraph{Forwarding for Final Version of the Chapters.}
%
% The following forwarding files |cdocsfn1.tex| and |cdocsfn2.tex|
% (with identical content)
% compile the final versions of the child documents
% |cdocsch1.tex| and |cdocsch2.tex|, respectively:
%\iffalse
%<*samplefinal>
%\fi
%    \begin{macrocode}
\def\version{final}
\input{childdoc.def}
\childdocforwardprefix[cdocsamp]{cdocsfn}{cdocsch}
%    \end{macrocode}

%\iffalse
%</samplefinal>
%\fi
%
% %%%%%%%%%%%%%%%%%%%%%%%%%%%%%%%%%%%%%%
% \paragraph{Command Line Processing.}
%
% The following three command lines generate the output files
% |cdocscld|, |cdocscl1| and |cdocscl2|
% which should be identical to
% |cdocsdrf|, |cdocsch1| and |cdocsfn2|, respectively:
% \begin{center}
% \begin{tabular}{l}
% |latex -jobname cdocscld \|\\
% |  "\def\version{draft}\input{childdoc.def}\childdocforward{cdocsamp}"|\\
% |latex -jobname cdocscl1 \|\\
% |  "\input{childdoc.def}\childdocforward[cdocsamp]{cdocsch1}"|\\
% |latex -jobname cdocscl2 \|\\
% |  "\def\version{final}\input{childdoc.def}\childdocforward{cdocsch2}"|
% \end{tabular}
% \end{center}
% Note that the trailing backslash on each first line
% merely continues the input to the second line
% (for convenient cut ant paste).
% Furthermore, the command |latex| can be replaced by any
% of its alternative versions such as |pdflatex|.
%
% %%%%%%%%%%%%%%%%%%%%%%%%%%%%%%%%%%%%%%%%%%%%%%%%%%%%%%%%%%%%%%%%%%%%%%%%%%%%%%
% %%%%%%%%%%%%%%%%%%%%%%%%%%%%%%%%%%%%%%%%%%%%%%%%%%%%%%%%%%%%%%%%%%%%%%%%%%%%%%
% \section{Implementation}
%\iffalse
%<*package>
%\fi
%
% This section describes the definitions file |childdoc.def|.

% The definitions cannot be loaded using |\usepackage| or |\RequirePackage|
% which has a mechanism to prevent loading a style file more than once.
% When loading the definitions by means of |\input|
% multiple instances have to be prevented manually:
%\iffalse
%This code needs to be before the `\ProvidesFile' directive
%which is defined at the beginning of this file.
%Therefore it is also placed there and commented out here.
%</package>
%<*discard>
%\fi
%    \begin{macrocode}
\ifdefined\childdocmain\endinput\fi
%    \end{macrocode}
%\iffalse
%</discard>
%<*package>
%\fi
%
% \macro{\ifchilddoc}
% \macro{\ifchilddocmanual}
% The conditional |\ifchilddoc| tells whether a
% child (true) or main (false) document is being compiled.
% The conditional |\ifchilddocmanual| tells whether
% the |\includeonly| mechanism is used (false) or
% the selection of child files must be performed manually (true).
% The definitions initialise to false:
%    \begin{macrocode}
\newif\ifchilddoc
\newif\ifchilddocmanual
%    \end{macrocode}

% \macro{\childdocname}
% \macro{\childdocjob}
% The macro |\childdocname| stores the name of the main document
% to be compiled. The macro |\childdocjob| stores the name of
% the document on which the \LaTeX{} compiler was originally invoked.
% The content of |\jobname| cannot be compared
% to filenames specified in the source due to different catcodes.
% The following code rescans |\jobname|, stores the result
% in |\childdocname| and saves a copy in |\childdocjob|:
%    \begin{macrocode}
\edef\childdocname{\scantokens\expandafter{\jobname\noexpand}}
\let\childdocjob\childdocname
%    \end{macrocode}

% \macro{\childdocdisable}
% The macro |\childdocdisable| prevents the main file
% from being processed more than once.
% At this stage, the main document command |\childdocmain|
% is assumed to be called once again where it should do nothing.
% Any subsequent call to it should prevent
% a secondary processing of the main document
% It overwrites the forwarding commands
% |\childdocof| and |\childdocforward|
% with empty macros to prevent further inclusions of the main document:
%    \begin{macrocode}
\newcommand{\childdocdisable}
{
  \renewcommand{\childdocmain}[1]{\renewcommand{\childdocmain}[1]{\endinput}}
  \renewcommand{\childdocof}[1]{}
  \renewcommand{\childdocby}[2][]{}
  \renewcommand{\childdocforward}[2][]{}
  \renewcommand{\childdocdisable}{}
}
%    \end{macrocode}

% \macro{\childdocmain}
% The macro |\childdocmain| is to be called at the top of the main file
% with nothing or the main filename (without extension) as argument.
% First, it breaks loops.
% If the argument is not empty and does not match |\childdocname|
% (which is set by the first inclusion of |childdoc.def|),
% |\ifchilddoc| is set to true, |\includeonly| is applied to the child file
% and |\jobname| is set to the main file
% (for proper handling of |.aux| files):
%    \begin{macrocode}
\newcommand{\childdocmain}[1]
{
  \childdocdisable\childdocmain{}
  \if?#1?\else
    \begingroup
      \def\childdoctmp{#1}
      \ifx\childdoctmp\childdocname
        \def\childdoctmp{}
      \else
        \def\childdoctmp
        {
          \childdoctrue
          \includeonly{\childdocname}
          \def\childdocjob{#1}
          \def\jobname{#1}
        }
      \fi
      \expandafter
    \endgroup
    \childdoctmp
  \fi
}
%    \end{macrocode}

% \macro{\childdocof}
% The command |\childdocof| redirects
% compilation to the main file |#1|.
%    \begin{macrocode}
\newcommand{\childdocof}[1]
{
  \childdocdisable
  \childdoctrue
  \includeonly{\childdocname}
  \def\jobname{#1}
  \def\childdocjob{#1}
  \input{#1}
}
%    \end{macrocode}

% \macro{\childdocby}
% The command |\childdocby| ....
%    \begin{macrocode}
\newcommand{\childdocby}[2][]
{
  \childdocdisable
  \childdoctrue
  \childdocmanualtrue
  \if?#1?\else
    \def\jobname{#2}
  \fi
  \def\childdocjob{#2}
  \input{#2}
  \endinput
}
%    \end{macrocode}

% \macro{\childdocforward}
% The command |\childdocforward| redirects
% compilation to the main file or
% (if the optional argument is given) a child file.
% Parameters are set as if the main file
% or a child file starting with |\childdocof| was compiled.
% Then compilation is handed over to the main file:
%    \begin{macrocode}
\newcommand{\childdocforward}[2][]
{
  \begingroup
    \if?#1?
      \def\childdoctmp
      {
        \def\childdocname{#2}
        \def\childdocjob{#2}
        \def\jobname{#2}
        \input{#2}
        \endinput
      }
    \else
      \def\childdoctmp
      {
        \childdocdisable
        \def\childdocname{#2}
        \childdoctrue
        \includeonly{#2}
        \def\childdocjob{#1}
        \def\jobname{#1}
        \input{#1}
        \endinput
      }
    \fi
    \expandafter
  \endgroup
  \childdoctmp
}
%    \end{macrocode}

% \macro{\childdocforwardprefix}
% The command |\childdocforwardprefix| redirects
% compilation to the main or a child file by means of a pattern.
% The prefix |#1| in the current filename is replaced by |#2|
% and the suffix of the current filename is kept
% (it is assumed that the filename does not contain the substring `|~~~|'
% which is used as a delimiter).
% Compilation is handed over to the new file by |\childdocforward|:
%    \begin{macrocode}
\newcommand{\childdocforwardprefix}[3][]
{
  \begingroup
    \def\childdocextract #2##1~~~{\def\childdoctmp{\childdocforward[#1]{#3##1}}}
    \expandafter\childdocextract\childdocname~~~
    \expandafter
  \endgroup
  \childdoctmp
}
%    \end{macrocode}

% \macro{\childdoc}
% The deprecated macro |\childdoc| is a legacy version of |\childdocmain|:
%    \begin{macrocode}
\newcommand{\childdoc}{\childdocmain}
%    \end{macrocode}

% \macro{\childdocredirect}
% The deprecated macro |\childdocredirect| is a legacy version
% of |\childdocforward| and |\childdocforwardprefix|:
%    \begin{macrocode}
\newcommand{\childdocredirect}[2][]
{
  \begingroup
    \if?#1?
      \def\childdoctmp{\childdocforward{#2}}
    \else
      \def\childdoctmp{\childdocforwardprefix{#1}{#2}}
    \fi
    \expandafter
  \endgroup
  \childdoctmp
}
%    \end{macrocode}

%\iffalse
%</package>
%\fi
%
\endinput

\childdocby{cdocsamp}
%    \end{macrocode}

%\iffalse
%</samplepart3|samplepart4>
%\fi
%
%\iffalse
%<*samplepart3>
%\fi
% Some text for part 3:
%    \begin{macrocode}
some text in part three
%    \end{macrocode}

%\iffalse
%</samplepart3>
%\fi
% Some text for part 4:
%\iffalse
%<*samplepart4>
%\fi
%    \begin{macrocode}
more text in part four
%    \end{macrocode}

%\iffalse
%</samplepart4>
%\fi
%
% %%%%%%%%%%%%%%%%%%%%%%%%%%%%%%%%%%%%%%
% \paragraph{Forwarding for a Complete Draft.}
%
% The following forwarding file |cdocsdrf.tex|
% compiles the main document in draft mode:
%\iffalse
%<*sampledraft>
%\fi
%    \begin{macrocode}
\def\version{draft}
% \iffalse
%
% childdoc.dtx Copyright (C) 2017-2018 Niklas Beisert
%
% This work may be distributed and/or modified under the
% conditions of the LaTeX Project Public License, either version 1.3
% of this license or (at your option) any later version.
% The latest version of this license is in
%   http://www.latex-project.org/lppl.txt
% and version 1.3 or later is part of all distributions of LaTeX
% version 2005/12/01 or later.
%
% This work has the LPPL maintenance status `maintained'.
%
% The Current Maintainer of this work is Niklas Beisert.
%
% This work consists of the files childdoc.dtx and childdoc.ins
% and the derived files childdoc.def and cdocsamp.tex with
% cdocsch1.tex, cdocsch2.tex, cdocsdrf.tex, cdocsfn1.tex, cdocsfn2.tex.
%
%<package>\ifdefined\childdocmain\endinput\fi
%<package>\ProvidesFile{childdoc.def}[2018/12/30 v2.0 child document driver]
%<samplemain>\ProvidesFile{cdocsamp.tex}[2018/12/30 v2.0 sample for childdoc]
%<*driver>
%\ProvidesFile{childdoc.drv}[2018/12/30 v2.0 childdoc reference manual file]
\PassOptionsToClass{10pt,a4paper}{article}
\documentclass{ltxdoc}

\usepackage[margin=35mm]{geometry}
\usepackage{hyperref}
\usepackage{hyperxmp}
\usepackage[usenames]{color}

\hypersetup{colorlinks=true}
\hypersetup{pdfstartview=FitH}
\hypersetup{pdfpagemode=UseNone}
\hypersetup{pdfsource={}}
\hypersetup{pdflang={en-UK}}
\hypersetup{pdfcopyright={Copyright 2017-2018 Niklas Beisert.
  This work may be distributed and/or modified under the
  conditions of the LaTeX Project Public License, either version 1.3
  of this license or (at your option) any later version.}}
\hypersetup{pdflicenseurl={http://www.latex-project.org/lppl.txt}}
\hypersetup{pdfcontactaddress={ETH Zurich, ITP, HIT K,
  Wolfgang-Pauli-Strasse 27}}
\hypersetup{pdfcontactpostcode={8093}}
\hypersetup{pdfcontactcity={Zurich}}
\hypersetup{pdfcontactcountry={Switzerland}}
\hypersetup{pdfcontactemail={nbeisert@itp.phys.ethz.ch}}
\hypersetup{pdfcontacturl={http://people.phys.ethz.ch/\xmptilde nbeisert/}}

\newcommand{\secref}[1]{\hyperref[#1]{section \ref*{#1}}}

\parskip1ex
\parindent0pt
\let\olditemize\itemize
\def\itemize{\olditemize\parskip0pt}

\begin{document}

\title{The \textsf{childdoc} Package}
\hypersetup{pdftitle={The childdoc Package}}
\author{Niklas Beisert\\[2ex]
  Institut f\"ur Theoretische Physik\\
  Eidgen\"ossische Technische Hochschule Z\"urich\\
  Wolfgang-Pauli-Strasse 27, 8093 Z\"urich, Switzerland\\[1ex]
  \href{mailto:nbeisert@itp.phys.ethz.ch}
  {\texttt{nbeisert@itp.phys.ethz.ch}}}
\hypersetup{pdfauthor={Niklas Beisert}}
\hypersetup{pdfsubject={Manual for the LaTeX2e Package childdoc}}
\date{30 December 2018, \textsf{v2.0}}
\maketitle

\begin{abstract}\noindent
\textsf{childdoc} is a \LaTeXe{} package
that enables the direct compilation
of document sections included by |\include|
to individual files.
\end{abstract}

\begingroup
\parskip0ex
\tableofcontents
\endgroup

%%%%%%%%%%%%%%%%%%%%%%%%%%%%%%%%%%%%%%%%%%%%%%%%%%%%%%%%%%%%%%%%%%%%%%%%%%%%%%%%
%%%%%%%%%%%%%%%%%%%%%%%%%%%%%%%%%%%%%%%%%%%%%%%%%%%%%%%%%%%%%%%%%%%%%%%%%%%%%%%%
\section{Introduction}

\LaTeX{} provides a mechanism to structure a large document (such as a book)
into a main file and several child files (containing the chapters)
using the |\include| command.
This mechanism is beneficial for documents
which span hundreds of pages in order to
make the source file(s) more manageable.
Moreover, compilation can be restricted to
selected child files by means of the |\includeonly| command.
The latter feature can be used to reduce the compilation time while editing
(this was significantly more useful in the earlier days of \LaTeX{})
or to generate a smaller document which is easier to navigate.
Another application of |\includeonly| is to generate
documents consisting of selected parts of the complete document.

However, there are a few drawbacks of the plain |\include| mechanism:
\begin{itemize}
\item
The child files cannot be compiled on their own,
they can only be compiled via the main file.
A naive editing environment
(such as a text editor with an option
to have the current file processed by \LaTeX)
may require one to switch to the main file before compiling;
attempting to compile the child file produces errors.
\item
The main file must be modified (each time)
to adjust the |\includeonly| command
to the present needs. This easily leaves the main file in a messy state.
\item
The generated document will always carry the filename
of the main document. This is inconvenient if
several child files are to be compiled and
to be kept for distribution.
\end{itemize}

The present package provides a simple interface
to make child files individually compilable by \LaTeX{}.
Compiling a child file then has the same effect as compiling
the main file with an |\includeonly| command
to select the appropriate child.
Moreover the generated document will carry the name of the child
rather than the main file.
This resolves all three above issues.

This feature is meant to make the editing of books,
thesis documents and lecture notes somewhat more convenient.
However, the package can also be used efficiently for
composing a series of documents (such as exercise sheets)
which are typically distributed individually.
It then assists the author in generating the individual documents
(potentially in different versions)
as well as a document containing the collected series.
Another application is in developing style files
or other kinds of included material
where compilation of the style file could redirect
to a sample or test file.

%%%%%%%%%%%%%%%%%%%%%%%%%%%%%%%%%%%%%%%%%%%%%%%%%%%%%%%%%%%%%%%%%%%%%%%%%%%%%%%%
%%%%%%%%%%%%%%%%%%%%%%%%%%%%%%%%%%%%%%%%%%%%%%%%%%%%%%%%%%%%%%%%%%%%%%%%%%%%%%%%
\section{Usage}

First of all, the package \textsf{childdoc} is \emph{not} a standard
\LaTeXe{} |.sty| style file! Therefore it needs to be invoked in
a non-standard way.

%%%%%%%%%%%%%%%%%%%%%%%%%%%%%%%%%%%%%%%%%%%%%%%%%%%%%%%%%%%%%%%%%%%%%%%%%%%%%%%%
\subsection{Included Files}
\label{sec:include}

%%%%%%%%%%%%%%%%%%%%%%%%%%%%%%%%%%%%%%%%
\DescribeMacro{\childdocmain}
To use the package, add the commands
\begin{center}
\begin{tabular}{l}
|\input{childdoc.def}|\\
|\childdocmain{}|\\
\end{tabular}
\end{center}
at the very top of the main \LaTeX{} file,
in particular \emph{before} the |\documentclass| statement!
The argument of |\childdocmain| should be left empty
(but it must be present).

%%%%%%%%%%%%%%%%%%%%%%%%%%%%%%%%%%%%%%%%
\DescribeMacro{\childdocof}
Furthermore, add the commands
\begin{center}
\begin{tabular}{l}
|\input{childdoc.def}|\\
|\childdocof{|\textit{main}|}|\\
\end{tabular}
\end{center}
at the top of every child file \textit{child}
which is included by |\include{|\textit{child}|}|
from within the main file
(or at least for those files to be compiled individually).
The argument \textit{main} must be the filename of the main file.

There are a couple of
considerations in setting up the main and child documents:

%%%%%%%%%%%%%%%%%%%%%%%%%%%%%%%%%%%%%%%%
\paragraph{Restrictions.}

Please note the following restrictions:
\begin{itemize}
\item
|\childdocmain| must be called with one argument \textit{main}
to ensure compatibility with earlier version of the package.
It must either be empty (|\childdocmain{}|)
or precisely match the filename of the main file in which it is specified.
See \secref{sec:detection} for further information.
\item
The filename \textit{main} must be specified without the |.tex| extension.
\item
The filename \textit{main} is case sensitive
(even in case-insensitive file systems)
due to internal string comparison.
\item
The argument \textit{main} should be fully expanded, it cannot be a macro.
\item
Subdirectories and special characters should be avoided in filenames.
\item
The command |\childdocmain{|\textit{main}|}| must be followed by a whitespace.
It should not be followed immediately by another command
or by a comment mark `|%|'.
This is because the \TeX{} parser reads the token immediately following
the argument of |\childdocmain| and puts it
at the beginning of every child section;
however, a white\-space is ignored.
\end{itemize}

%%%%%%%%%%%%%%%%%%%%%%%%%%%%%%%%%%%%%%%%
\paragraph{Content of Main File.}

It is advisable to place all content in the child files included by |\include|.
Any output contained in the main file will appear in all child documents
unless suppressed manually;
it cannot be suppressed automatically by the |\includeonly| directive
and thus should normally be avoided.
A method to include some content in the main file
by means of conditional processing is described in \secref{sec:conditional}.

%%%%%%%%%%%%%%%%%%%%%%%%%%%%%%%%%%%%%%%%
\paragraph{Page Numbering.}

When only a part of the document is compiled,
the appropriate numbering of pages
(as well as other status parameters)
is determined from the |.aux| files.
The latter contain information from previous passes.
However this information needs to propagate through
all intermediate child documents.
Therefore the page numbering in child documents may well
be inconsistent until the complete document is compiled at least once.

A useful (if unconventional) way to always ensure a consistent
page numbering is to restart the numbering in each child document
and denote the pages by `\textit{child}|.|\textit{page}'
where \textit{child} represents the chapter/section number of the child file.
This can be achieved by the command
|\numberwithin{page}{|\textit{child}|}|
of the \textsf{amsmath} package
where \textit{child} can be |chapter| or |section|
depending on the chosen structuring.
Alternatively, one can modify the macro |\thepage| appropriately
and reset the counter |page| at the start of each child file.

%%%%%%%%%%%%%%%%%%%%%%%%%%%%%%%%%%%%%%%%%%%%%%%%%%%%%%%%%%%%%%%%%%%%%%%%%%%%%%%%
\subsection{Conditional Processing}
\label{sec:conditional}

The package provides a mechanism to compile different versions
of a document. To customise the versions further some conditional processing
can come in handy to distinguish which version is being compiled.
The package provides two macros to describe the compilation context:

%%%%%%%%%%%%%%%%%%%%%%%%%%%%%%%%%%%%%%%%
\DescribeMacro{\ifchilddoc}
The conditional |\ifchilddoc| distinguishes between the compilation of
child documents and the main document:
%
\begin{center}
|\ifchilddoc |\textit{child-code}| |[|\||else |\textit{main-code}]| \||fi|
\end{center}

%%%%%%%%%%%%%%%%%%%%%%%%%%%%%%%%%%%%%%%%
\DescribeMacro{\childdocname}
\DescribeMacro{\childdocjob}
The macro |\childdocname| contains the filename (without extension)
of the main or child file being processed.
Note that |\childdocjob| will always contain the name of the main file.

%%%%%%%%%%%%%%%%%%%%%%%%%%%%%%%%%%%%%%%%
\paragraph{Title Page.}

Conditional processing can be used to include a title or banner page
in the main document when proper precautions are taken.
Importantly, the code in the main file should ensure that the page counter
(as well as other status parameters which are stored in the |.aux| files)
takes the same value after the conditional processing.
Otherwise the page numbers may take divergent values
depending on which part is compiled.

For example, a title page could be declared by:
%
\begin{center}
\begin{tabular}{l}
|\ifchilddoc\||else|\\
|\addtocounter{page}{-1}|\\
\textit{code for title page}\\
|\newpage|\\
|\||fi|
\end{tabular}
\end{center}
%
A banner page for the child documents can be generated by:
%
\begin{center}
\begin{tabular}{l}
|\ifchilddoc|\\
|\addtocounter{page}{-1}|\\
\textit{code for banner page}\\
|\newpage|\\
|\||fi|
\end{tabular}
\end{center}
%
Here one could write a message such as:
\begin{center}
|This is the part \childdocname{} of \childdocjob{}.|
\end{center}

%%%%%%%%%%%%%%%%%%%%%%%%%%%%%%%%%%%%%%%%%%%%%%%%%%%%%%%%%%%%%%%%%%%%%%%%%%%%%%%%
\subsection{Flags}
\label{sec:flags}

The package makes it easy to generate different versions
of the main or child documents.
To this end compilation flags can be defined
and assigned different default values.
They will be particularly useful in conjunction
with the forwarding mechanism described in \secref{sec:forward}.

For example, it may be useful to have a flag |\version|
which can be set to |draft| or |final|.
The document source will contain some conditional code
depending on the value of |\version|.
Suppose further, the flag should default to |final| for the main file
and to |draft| for child files
which is a natural assignment for editing the document.
This is achieved by placing the following code
in the preamble of the main document
(below the |\childdocmain| directive):
%
\begin{center}
\begin{tabular}{l}
|\ifchilddoc|\\
|\providecommand{\version}{draft}|\\
|\||else|\\
|\providecommand{\version}{final}|\\
|\||fi|
\end{tabular}
\end{center}
%
The definition by |\providecommand| makes sure
that previous definitions are not overwritten.
Further statements |\providecommand{\version}{...}|
can thus be added before the above code to override it.

For the main file, one might add a line
(between |\childdocmain| and the above block)
%
\begin{center}
|%\ifchilddoc\||else\providecommand{\version}{draft}\||fi|
\end{center}
%
which can be uncommented to produce a draft version.
Likewise one can add a line to the very top of a child file
(above the |\childdocof{|\textit{main}|}| directive)
%
\begin{center}
|%\providecommand{\version}{final}|
\end{center}
%
which can be uncommented to produce the final version of this child document.

%%%%%%%%%%%%%%%%%%%%%%%%%%%%%%%%%%%%%%%%%%%%%%%%%%%%%%%%%%%%%%%%%%%%%%%%%%%%%%%%
\subsection{Forwarding}
\label{sec:forward}

Different versions of the main or child documents
using compilation flags as described in \secref{sec:flags}
can be (permanently) stored in different files
for convenient compilation, viewing and distribution.
To this end, the package defines a command
to pass on compilation to a different file:

%%%%%%%%%%%%%%%%%%%%%%%%%%%%%%%%%%%%%%%%
\DescribeMacro{\childdocforward}
The command |\childdocforward| redirects processing to
another source file:
%
\begin{center}
\begin{tabular}{l}
|\input{childdoc.def}|\\
|\childdocforward[|\textit{main}|]{|\textit{dest}|}|\\
\end{tabular}
\end{center}
%
The argument \textit{dest} is the destination file
(without extension).
It should be the main file or one of the child files.
Note that further \textsf{childdoc} directives
such as |\childdocof| and |\childdocforward|
in the indicated file will be processed in this form.
The optional argument \textit{main}
passes on directly to the main file \textit{main}
while pretending to compile the child \textit{dest}.
This form behaves as if \textit{dest}
issues |\childdocof{|\textit{main}|}| right away,
and no further \textsf{childdoc} directives will be processed.

%%%%%%%%%%%%%%%%%%%%%%%%%%%%%%%%%%%%%%%%
\DescribeMacro{\...prefix}
In the alternative form |\childdocforwardprefix|,
%
\begin{center}
\begin{tabular}{l}
|\input{childdoc.def}|\\
|\childdocforwardprefix[|\textit{main}|]{|\textit{prefix}|}{|\textit{dest}|}|
\end{tabular}
\end{center}
%
the destination file is determined by a pattern
depending on the current file:
To make this work, the current file must be called
`{\textit{prefix}\hspace{0.2em}\textit{suffix}}'
with \textit{prefix} matching precisely the argument.
Processing is then passed on to the file
`{\textit{dest}\hspace{0.2em}\textit{suffix}}'.
Surely, the same effect is achieved by
directly specifying the
argument `{\textit{dest}\hspace{0.2em}\textit{suffix}}'
in the first form.
However, that requires to set up a different file
for each child. With the alternative form of the command
all these files can have exactly the same content
which simplifies setting them up and maintaining them.

For example, the following file |draft.tex|
with a compilation flag |\version| as described in \secref{sec:flags}
compiles the main document as a draft:
%
\begin{center}
\begin{tabular}{l}
|\def\version{draft}|\\
|\input{childdoc.def}|\\
|\childdocforward{|\textit{main}|}|
\end{tabular}
\end{center}
%
Likewise, the following files |final|\textit{nn}|.tex|
compile the final version of the child document
|child|\textit{nn}|.tex|:
%
\begin{center}
\begin{tabular}{l}
|\def\version{final}|\\
|\input{childdoc.def}|\\
|\childdocforwardprefix{final}{child}|
\end{tabular}
\end{center}
%

Note that when several versions of a main file and/or of each child file
are to be generated, it may be convenient to set up a |Makefile| or
shell script to automatise the process.

%%%%%%%%%%%%%%%%%%%%%%%%%%%%%%%%%%%%%%%%%%%%%%%%%%%%%%%%%%%%%%%%%%%%%%%%%%%%%%%%
\subsection{Command Line Processing}
\label{sec:commandline}

The effect of redirection files can also be achieved by invoking
the \LaTeX{} compiler with a more elaborate command line.
Most conveniently this should be done as part
of a shell script or a |Makefile|.

When using \textsf{childdoc} in the main file, the following
command lines effectively perform a redirection
(note that depending on the shell being used,
backslashes may have to be doubled: `|\|' $\to$ `|\\|'):
%
\begin{center}
|... -jobname "|\textit{target}|" |\\|"|[\textit{flags}]%
|\input{childdoc.def}\childdocforward[|\textit{main}|]{|\textit{dest}|}"|
\end{center}
%
Here \textit{target} is the name of the output file,
\textit{main} is the name of the main file
and \textit{dest} is the name of the main or child file to be processed
(all filenames without extensions).
The optional argument \textit{main} can be omitted
if \textit{main} matches \textit{dest}.
Optionally, compilation \textit{flags} can be defined via |\def| commands.
This command line makes the \TeX{} engine believe
it is compiling the file \textit{target}
whose content is specified as the latter parameter.
The provided code then forwards the processing to
\textit{main} or \textit{dest} as described in \secref{sec:forward}.

%%%%%%%%%%%%%%%%%%%%%%%%%%%%%%%%%%%%%%%%%%%%%%%%%%%%%%%%%%%%%%%%%%%%%%%%%%%%%%%%
\subsection{Include by Input}
\label{sec:input}

Including child documents by |\include| has some restrictions by design.
Most notably, the content of a child document always occupies
its own set of pages; pages cannot be shared between child documents.
Usually, this behaviour makes perfect sense
because each child document contain an essential part of the document.
However, in some situations it may be desirable to compose
a document from a collection of parts
without having mandatory page breaks between then.
For this case, the package
provides a mechanism to include parts
by |\input| which can also be processed individually.
However, by construction this mechanism
requires manual handling of the content to be output.

%%%%%%%%%%%%%%%%%%%%%%%%%%%%%%%%%%%%%%%%
\DescribeMacro{\ifchilddocmanual}
The main file should be prepared as usual, see \secref{sec:include}.
However, the document body must make a distinction
between processing of an individual part and of the main document, e.g.:
%
\begin{center}
\begin{tabular}{l}
|\ifchilddocmanual|\\
|\input{\childdocname}|\\
|\||else|\\
\textit{document body with }|\input{|\textit{part}|}|\\
|\||fi|
\end{tabular}
\end{center}
%
The conditional |\ifchilddocmanual| is true whenever
a part to be included by |\input| is being compiled,
and the name of the part is stored in |\childdocname|.

%%%%%%%%%%%%%%%%%%%%%%%%%%%%%%%%%%%%%%%%
\DescribeMacro{\childdocby}
Each part to be included by |\input| should start with:
%
\begin{center}
\begin{tabular}{l}
|\input{childdoc.def}|\\
|\childdocby{|\textit{main}|}|\\
\end{tabular}
\end{center}
%
The directive |\childdocby| is similar to |\childdocof|
described in \secref{sec:include},
but the subsequent selection of content must be done manually.
To that end, both |\ifchilddoc| and |\ifchilddocmanual|
will be true upon processing of a part,
and the name of the part is stored in |\childdocname|.
Note that |\jobname| will be set to the filename of the current part
so that each part receives an individual |.aux| file
that does not interfere with the |.aux| file(s) of the main document.
This behaviour can be altered by the alternative form
|\childdocby[*]{|\textit{main}|}| (with a non-empty optional argument)
which uses the |.aux| file of the main document
by setting |\jobname| to \textit{main}.

%%%%%%%%%%%%%%%%%%%%%%%%%%%%%%%%%%%%%%%%%%%%%%%%%%%%%%%%%%%%%%%%%%%%%%%%%%%%%%%%
\subsection{Driver Development}
\label{sec:driver}

The \textsf{childdoc} mechanism can also be use for the development
of definition files such as \LaTeX{} styles or classes.
This case differs from the above setup with multiple parts
included by |\include| in that no |\includeonly| should be invoked.
This can be achieved by starting the include file
(before |\ProvidesPackage|) with:
%
\begin{center}
\begin{tabular}{l}
|\input{childdoc.def}|\\
|\childdocforward{|\textit{main}|}|\\
\end{tabular}
\end{center}
%
or alternatively with:
%
\begin{center}
\begin{tabular}{l}
|\input{childdoc.def}|\\
|\childdocby{|\textit{main}|}|\\
\end{tabular}
\end{center}
%
Both forms have slightly different effects as described above.
The main file is prepared as usual, see \secref{sec:include}.

%%%%%%%%%%%%%%%%%%%%%%%%%%%%%%%%%%%%%%%%%%%%%%%%%%%%%%%%%%%%%%%%%%%%%%%%%%%%%%%%
\subsection{Legacy Detection}
\label{sec:detection}

The directive |\childdocmain| in the main file can detect
whether the complete document or merely a child is to be compiled
even without using the directive |\childdocof|.
This method is deprecated because it is less robust
and there is no compelling reason to use it;
it is merely provided for backward compatibility
and it may be removed in future versions.

If the detection mechanism is to be used,
it is mandatory to correctly specify
the filename of the main file as the argument of |\childdocmain|:
%
\begin{center}
\begin{tabular}{l}
|\input{childdoc.def}|\\
|\childdocmain{|\textit{main}|}|\\
\end{tabular}
\end{center}
%
If |\jobname| does not match the argument \textit{main} of |\childdocmain|,
it is assumed that |\jobname| points to the child file to be compiled.
When using |\childdocmain| with the main file specified as argument,
it suffices to start a child file
with just |\input{|\textit{main}|}|
without loading of the package and using |\childdocof|.
If instead all processing is done
with the appropriate \textsf{childdoc} directives,
the argument of \textit{main} of |\childdocmain| can be empty.

An alternative version of the command line processing described
in \secref{sec:commandline} using the detection mechanism reads:
%
\begin{center}
|... -jobname "|\textit{target}|" "|[\textit{flags}]%
[|\def\jobname{|\textit{dest}|}|]|\input{|\textit{main}|}"|
\end{center}

%%%%%%%%%%%%%%%%%%%%%%%%%%%%%%%%%%%%%%%%%%%%%%%%%%%%%%%%%%%%%%%%%%%%%%%%%%%%%%%%
\subsection{Manual Code}
\label{sec:manual}

In case one cannot be certain whether the definitions file |childdoc.def|
is installed on the target \TeX{} distribution
and one prefers not to ship it,
it is conceivable to paste a few relevant commands into the sources.

To that end, drop all statements |\input{childdoc.def}|
and perform the replacements as outlined below.
Instead of |\childdocmain{|\textit{main}|}| add the following code
to the top of the main file:
%
\begin{center}
\begin{tabular}{l}
|\||ifdefined\childdocname\endinput\||fi\newif\ifchilddoc|\\
|\edef\childdocname{\scantokens\expandafter{\jobname\noexpand}}|\\
|\def\childdocmain{|\textit{main}|}\||ifx\childdocmain\childdocname\||else|\\
|\childdoctrue\includeonly{\childdocname}\let\jobname\childdocmain\||fi|\\
\end{tabular}
\end{center}
%
Instead of |\childdocof{|\textit{main}|}| just include the main file
at the top of each child file:
%
\begin{center}
|\input{|\textit{main}|}|
\end{center}
%
A simple redirection |\childdocforward{|\textit{dest}|}| is achieved by:
%
\begin{center}
|\def\jobname{|\textit{dest}|}\input{\jobname}|
\end{center}
%
The redirection with prefix
|\childdocforwardprefix[|\textit{prefix}|]{|\textit{dest}|}|
is accomplished by:
%
\begin{center}
\begin{tabular}{l}
|{\edef\jobname{\scantokens\expandafter{\jobname\noexpand}}|\\
|\def\redirectjob |\textit{prefix}|#1~~~{\gdef\jobname{|\textit{dest}|#1}}|\\
|\expandafter\redirectjob\jobname~~~}\input{\jobname}|
\end{tabular}
\end{center}

In an alternative approach,
child documents can be compiled by a specific command line
without additional code or specific definitions:
%
\begin{center}
|... -jobname "|\textit{target}|" "|[\textit{flags}]%
|\includeonly{|\textit{dest}|}\input{|\textit{main}|}"|
\end{center}
%

%%%%%%%%%%%%%%%%%%%%%%%%%%%%%%%%%%%%%%%%%%%%%%%%%%%%%%%%%%%%%%%%%%%%%%%%%%%%%%%%
%%%%%%%%%%%%%%%%%%%%%%%%%%%%%%%%%%%%%%%%%%%%%%%%%%%%%%%%%%%%%%%%%%%%%%%%%%%%%%%%
\section{Information}

%%%%%%%%%%%%%%%%%%%%%%%%%%%%%%%%%%%%%%%%%%%%%%%%%%%%%%%%%%%%%%%%%%%%%%%%%%%%%%%%
\subsection{Copyright}

Copyright \copyright{} 2017--2018 Niklas Beisert

This work may be distributed and/or modified under the
conditions of the \LaTeX{} Project Public License, either version 1.3
of this license or (at your option) any later version.
The latest version of this license is in
  \url{http://www.latex-project.org/lppl.txt}
and version 1.3 or later is part of all distributions of \LaTeX{}
version 2005/12/01 or later.

This work has the LPPL maintenance status `maintained'.

The Current Maintainer of this work is Niklas Beisert.

This work consists of the files |README.txt|, |childdoc.ins| and |childdoc.dtx|
as well as the derived files |childdoc.def|, |cdocsamp.tex|
with |cdocsch1.tex|, |cdocsch2.tex|, |cdocspt3.tex|, |cdocspt4.tex|,
|cdocsdrf.tex|, |cdocsfn1.tex|, |cdocsfn2.tex|
as well as |childdoc.pdf|.

%%%%%%%%%%%%%%%%%%%%%%%%%%%%%%%%%%%%%%%%%%%%%%%%%%%%%%%%%%%%%%%%%%%%%%%%%%%%%%%%
\subsection{Files and Installation}

The package consists of the files:
%
\begin{center}
\begin{tabular}{ll}
    |README.txt|   & readme file \\
    |childdoc.ins| & installation file \\
    |childdoc.dtx| & source file \\
    |childdoc.def| & definition file \\
    |cdocsamp.tex| & sample main file \\
    |cdocsch1.tex| & sample include file \\
    |cdocsch2.tex| & sample include file \\
    |cdocspt3.tex| & sample part file \\
    |cdocspt4.tex| & sample part file \\
    |cdocsdrf.tex| & sample redirection file \\
    |cdocsfn1.tex| & sample redirection file \\
    |cdocsfn2.tex| & sample redirection file \\
    |childdoc.pdf| & manual
\end{tabular}
\end{center}
%
The distribution consists of the files
|README.txt|, |childdoc.ins| and |childdoc.dtx|.
%
\begin{itemize}
\item
Run (pdf)\LaTeX{} on |childdoc.dtx|
to compile the manual |childdoc.pdf| (this file).
\item
Run \LaTeX{} on |childdoc.ins| to create the definitions file |childdoc.def|
and the sample |cdocsamp.tex| with include files
|cdocsch1.tex|, |cdocsch2.tex|, |cdocspt3.tex|, |cdocspt4.tex|,
|cdocsdrf.tex|, |cdocsfn1.tex|, |cdocsfn2.tex|.
Then copy the file |childdoc.def| to an appropriate directory of your \LaTeX{}
distribution, e.g.\ \textit{texmf-root}|/tex/latex/childdoc|.
\end{itemize}

%%%%%%%%%%%%%%%%%%%%%%%%%%%%%%%%%%%%%%%%%%%%%%%%%%%%%%%%%%%%%%%%%%%%%%%%%%%%%%%%
\subsection{Related CTAN Packages}

There are several other packages which offer a similar functionality:
%
\begin{itemize}
\item
The packages
\href{http://ctan.org/pkg/docmute}{\textsf{docmute}},
\href{http://ctan.org/pkg/includex}{\textsf{includex}} and
\href{http://ctan.org/pkg/standalone}{\textsf{standalone}}
provide commands to include only the document body of
a child file thus allowing both files to be compiled individually.
\item
The packages \href{http://ctan.org/pkg/subdocs}{\textsf{subdocs}}
and \href{http://ctan.org/pkg/subfiles}{\textsf{subfiles}}
provide structures in which the main and child documents can be
encapsulated and allowing them to be compiled individually.
The inclusion mechanism is different from the conventional |\include|.
\item
The package \href{http://ctan.org/pkg/combine}{\textsf{combine}}
is an elaborate solution to combine several documents into one.
\end{itemize}
%
See also the CTAN topic \href{http://ctan.org/topic/subdocs}{\textsf{subdocs}}
for further related packages.
The present package differs from the above solutions in that
a document structure constructed with the conventional |\include| mechanism
just needs two extra commands at the top of every file
such that all constituent files can be compiled individually.

%%%%%%%%%%%%%%%%%%%%%%%%%%%%%%%%%%%%%%%%%%%%%%%%%%%%%%%%%%%%%%%%%%%%%%%%%%%%%%%%
%\subsection{Feature Suggestions}
%
%The following is a list of features which may be useful for future
%versions of this package:
%%
%\begin{itemize}
%\item
%\ldots
%\end{itemize}

%%%%%%%%%%%%%%%%%%%%%%%%%%%%%%%%%%%%%%%%%%%%%%%%%%%%%%%%%%%%%%%%%%%%%%%%%%%%%%%%
\subsection{Revision History}

%%%%%%%%%%%%%%%%%%%%%%%%%%%%%%%%%%%%%%%%
\paragraph{v2.0:} 2018/12/30

\begin{itemize}
\item
immediate forward processing
\item
added |\childdocby| mechanism
\item
manual restructured
\end{itemize}

%%%%%%%%%%%%%%%%%%%%%%%%%%%%%%%%%%%%%%%%
\paragraph{v1.6:} 2018/01/17

\begin{itemize}
\item
application for development of include files
\item
corrections to manual
\end{itemize}

%%%%%%%%%%%%%%%%%%%%%%%%%%%%%%%%%%%%%%%%
\paragraph{v1.5:} 2017/05/21

\begin{itemize}
\item
more complete structuring introduced
\item
|\childdocof| introduced
\item
|\childdoc| renamed to |\childdocmain|
\item
|\childredirect| renamed to |\childdocforward| and |\childdocforwardprefix|
and functionality expanded
\end{itemize}

%%%%%%%%%%%%%%%%%%%%%%%%%%%%%%%%%%%%%%%%
\paragraph{v1.0:} 2017/04/27

\begin{itemize}
\item
manual and install package
\item
first version published on CTAN
\end{itemize}

%%%%%%%%%%%%%%%%%%%%%%%%%%%%%%%%%%%%%%%%
\paragraph{v0.6:} 2017/04/26

\begin{itemize}
\item
redirection mechanism added
\end{itemize}

%%%%%%%%%%%%%%%%%%%%%%%%%%%%%%%%%%%%%%%%
\paragraph{v0.5:} 2017/04/26

\begin{itemize}
\item
functionality in definition file
\end{itemize}


%%%%%%%%%%%%%%%%%%%%%%%%%%%%%%%%%%%%%%%%%%%%%%%%%%%%%%%%%%%%%%%%%%%%%%%%%%%%%%%%
%%%%%%%%%%%%%%%%%%%%%%%%%%%%%%%%%%%%%%%%%%%%%%%%%%%%%%%%%%%%%%%%%%%%%%%%%%%%%%%%
%%%%%%%%%%%%%%%%%%%%%%%%%%%%%%%%%%%%%%%%%%%%%%%%%%%%%%%%%%%%%%%%%%%%%%%%%%%%%%%%
\appendix

\settowidth\MacroIndent{\rmfamily\scriptsize 000\ }

 \DocInput{childdoc.dtx}

\end{document}
%</driver>
% \fi
%
% %%%%%%%%%%%%%%%%%%%%%%%%%%%%%%%%%%%%%%%%%%%%%%%%%%%%%%%%%%%%%%%%%%%%%%%%%%%%%%
% %%%%%%%%%%%%%%%%%%%%%%%%%%%%%%%%%%%%%%%%%%%%%%%%%%%%%%%%%%%%%%%%%%%%%%%%%%%%%%
% \section{Sample}
%\iffalse
%<*samplemain>
%\fi
%
% The following presents a sample document
% with two chapters, two parts, a title page,
% a compile flag as well as three forwarding files to set the flag.
% It consists of eight |.tex| files:
% \begin{center}
% \begin{tabular}{ll}
% |cdocsamp.tex|&main file\\
% |cdocsch1.tex|&include file for chapter 1\\
% |cdocsch2.tex|&include file for chapter 2\\
% |cdocspt3.tex|&include file for part 3\\
% |cdocspt4.tex|&include file for part 4\\
% |cdocsdrf.tex|&forwarding file for main file in draft mode\\
% |cdocsfi1.tex|&forwarding file for final version of chapter 1\\
% |cdocsfi2.tex|&forwarding file for final version of chapter 2\\
% \end{tabular}
% \end{center}
% Each of the eight files can be compiled directly by the \LaTeX{} compiler.
%
% %%%%%%%%%%%%%%%%%%%%%%%%%%%%%%%%%%%%%%
% \paragraph{Main File.}
%
% The main file is called |cdocsamp.tex|.
%
% Load the \textsf{childdoc} definitions and
% declare the filename for the main document:
%    \begin{macrocode}
\input{childdoc.def}
\childdocmain{}
%    \end{macrocode}

% Optional override for |\version| flag:
%    \begin{macrocode}
%%\ifchilddoc\else\providecommand{\version}{draft}\fi
%    \end{macrocode}

% Define the default values for the |\version| flag
% (|final| for the main file and |draft| for childs):
%    \begin{macrocode}
\ifchilddoc
\providecommand{\version}{draft}
\else
\providecommand{\version}{final}
\fi
%    \end{macrocode}

% Load the standard document class:
%    \begin{macrocode}
\documentclass[12pt]{article}
%    \end{macrocode}

% Start the document body:
%    \begin{macrocode}
\begin{document}
%    \end{macrocode}

% Declare a title page.
% Print title, part of document being processed and version flag:
%    \begin{macrocode}
\addtocounter{page}{-1}
\begin{center}
{\LARGE\bfseries{}childdoc example\par}
\vspace{1cm}
\ifchilddoc
\ifchilddocmanual part\else chapter\fi:
`\childdocname' of `\childdocjob'\par
\else
main document: `\childdocjob'\par
\fi
version: \version\par
\end{center}
\newpage
%    \end{macrocode}

% Manually include selected file,
% otherwise process as usual:
%    \begin{macrocode}
\ifchilddocmanual
\section*{part `\childdocname'}
\input{\childdocname}
\else
%    \end{macrocode}

% Include the two chapters:
%    \begin{macrocode}
\include{cdocsch1}
\include{cdocsch2}
%    \end{macrocode}

% Include the two parts unless only chapters should be displayed:
%    \begin{macrocode}
\ifchilddoc\else
\section{part three}
\input{cdocspt3}
\section{part four}
\input{cdocspt4}
\fi
%    \end{macrocode}

% Process as usual until here:
%    \begin{macrocode}
\fi
%    \end{macrocode}

% End of document body:
%    \begin{macrocode}
\end{document}
%    \end{macrocode}
%\iffalse
%</samplemain>
%\fi
%
% %%%%%%%%%%%%%%%%%%%%%%%%%%%%%%%%%%%%%%
% \paragraph{Chapter Include Files.}
%
% The include files are called |cdocsch1.tex| and |cdocsch2.tex|.
%
%\iffalse
%<*samplechap1|samplechap2>
%\fi

% Optional override for |\version| flag:
%    \begin{macrocode}
%%\providecommand{\version}{final}
%    \end{macrocode}

% Include the main document:
%    \begin{macrocode}
\input{childdoc.def}
\childdocof{cdocsamp}
%    \end{macrocode}

%\iffalse
%</samplechap1|samplechap2>
%\fi
%
%\iffalse
%<*samplechap1>
%\fi
% Some text for chapter 1:
%    \begin{macrocode}
\section{one}
some text in chapter one
%    \end{macrocode}

%\iffalse
%</samplechap1>
%\fi
% Some text for chapter 2:
%\iffalse
%<*samplechap2>
%\fi
%    \begin{macrocode}
\section{two}
more text in chapter two
%    \end{macrocode}

%\iffalse
%</samplechap2>
%\fi
%
% %%%%%%%%%%%%%%%%%%%%%%%%%%%%%%%%%%%%%%
% \paragraph{Part Include Files.}
%
% The include files are called |cdocspt3.tex| and |cdocspt4.tex|.
%
%\iffalse
%<*samplepart3|samplepart4>
%\fi

% Optional override for |\version| flag:
%    \begin{macrocode}
%%\providecommand{\version}{final}
%    \end{macrocode}

% Include the main document:
%    \begin{macrocode}
\input{childdoc.def}
\childdocby{cdocsamp}
%    \end{macrocode}

%\iffalse
%</samplepart3|samplepart4>
%\fi
%
%\iffalse
%<*samplepart3>
%\fi
% Some text for part 3:
%    \begin{macrocode}
some text in part three
%    \end{macrocode}

%\iffalse
%</samplepart3>
%\fi
% Some text for part 4:
%\iffalse
%<*samplepart4>
%\fi
%    \begin{macrocode}
more text in part four
%    \end{macrocode}

%\iffalse
%</samplepart4>
%\fi
%
% %%%%%%%%%%%%%%%%%%%%%%%%%%%%%%%%%%%%%%
% \paragraph{Forwarding for a Complete Draft.}
%
% The following forwarding file |cdocsdrf.tex|
% compiles the main document in draft mode:
%\iffalse
%<*sampledraft>
%\fi
%    \begin{macrocode}
\def\version{draft}
\input{childdoc.def}
\childdocforward{cdocsamp}
%    \end{macrocode}

%\iffalse
%</sampledraft>
%\fi
%
% %%%%%%%%%%%%%%%%%%%%%%%%%%%%%%%%%%%%%%
% \paragraph{Forwarding for Final Version of the Chapters.}
%
% The following forwarding files |cdocsfn1.tex| and |cdocsfn2.tex|
% (with identical content)
% compile the final versions of the child documents
% |cdocsch1.tex| and |cdocsch2.tex|, respectively:
%\iffalse
%<*samplefinal>
%\fi
%    \begin{macrocode}
\def\version{final}
\input{childdoc.def}
\childdocforwardprefix[cdocsamp]{cdocsfn}{cdocsch}
%    \end{macrocode}

%\iffalse
%</samplefinal>
%\fi
%
% %%%%%%%%%%%%%%%%%%%%%%%%%%%%%%%%%%%%%%
% \paragraph{Command Line Processing.}
%
% The following three command lines generate the output files
% |cdocscld|, |cdocscl1| and |cdocscl2|
% which should be identical to
% |cdocsdrf|, |cdocsch1| and |cdocsfn2|, respectively:
% \begin{center}
% \begin{tabular}{l}
% |latex -jobname cdocscld \|\\
% |  "\def\version{draft}\input{childdoc.def}\childdocforward{cdocsamp}"|\\
% |latex -jobname cdocscl1 \|\\
% |  "\input{childdoc.def}\childdocforward[cdocsamp]{cdocsch1}"|\\
% |latex -jobname cdocscl2 \|\\
% |  "\def\version{final}\input{childdoc.def}\childdocforward{cdocsch2}"|
% \end{tabular}
% \end{center}
% Note that the trailing backslash on each first line
% merely continues the input to the second line
% (for convenient cut ant paste).
% Furthermore, the command |latex| can be replaced by any
% of its alternative versions such as |pdflatex|.
%
% %%%%%%%%%%%%%%%%%%%%%%%%%%%%%%%%%%%%%%%%%%%%%%%%%%%%%%%%%%%%%%%%%%%%%%%%%%%%%%
% %%%%%%%%%%%%%%%%%%%%%%%%%%%%%%%%%%%%%%%%%%%%%%%%%%%%%%%%%%%%%%%%%%%%%%%%%%%%%%
% \section{Implementation}
%\iffalse
%<*package>
%\fi
%
% This section describes the definitions file |childdoc.def|.

% The definitions cannot be loaded using |\usepackage| or |\RequirePackage|
% which has a mechanism to prevent loading a style file more than once.
% When loading the definitions by means of |\input|
% multiple instances have to be prevented manually:
%\iffalse
%This code needs to be before the `\ProvidesFile' directive
%which is defined at the beginning of this file.
%Therefore it is also placed there and commented out here.
%</package>
%<*discard>
%\fi
%    \begin{macrocode}
\ifdefined\childdocmain\endinput\fi
%    \end{macrocode}
%\iffalse
%</discard>
%<*package>
%\fi
%
% \macro{\ifchilddoc}
% \macro{\ifchilddocmanual}
% The conditional |\ifchilddoc| tells whether a
% child (true) or main (false) document is being compiled.
% The conditional |\ifchilddocmanual| tells whether
% the |\includeonly| mechanism is used (false) or
% the selection of child files must be performed manually (true).
% The definitions initialise to false:
%    \begin{macrocode}
\newif\ifchilddoc
\newif\ifchilddocmanual
%    \end{macrocode}

% \macro{\childdocname}
% \macro{\childdocjob}
% The macro |\childdocname| stores the name of the main document
% to be compiled. The macro |\childdocjob| stores the name of
% the document on which the \LaTeX{} compiler was originally invoked.
% The content of |\jobname| cannot be compared
% to filenames specified in the source due to different catcodes.
% The following code rescans |\jobname|, stores the result
% in |\childdocname| and saves a copy in |\childdocjob|:
%    \begin{macrocode}
\edef\childdocname{\scantokens\expandafter{\jobname\noexpand}}
\let\childdocjob\childdocname
%    \end{macrocode}

% \macro{\childdocdisable}
% The macro |\childdocdisable| prevents the main file
% from being processed more than once.
% At this stage, the main document command |\childdocmain|
% is assumed to be called once again where it should do nothing.
% Any subsequent call to it should prevent
% a secondary processing of the main document
% It overwrites the forwarding commands
% |\childdocof| and |\childdocforward|
% with empty macros to prevent further inclusions of the main document:
%    \begin{macrocode}
\newcommand{\childdocdisable}
{
  \renewcommand{\childdocmain}[1]{\renewcommand{\childdocmain}[1]{\endinput}}
  \renewcommand{\childdocof}[1]{}
  \renewcommand{\childdocby}[2][]{}
  \renewcommand{\childdocforward}[2][]{}
  \renewcommand{\childdocdisable}{}
}
%    \end{macrocode}

% \macro{\childdocmain}
% The macro |\childdocmain| is to be called at the top of the main file
% with nothing or the main filename (without extension) as argument.
% First, it breaks loops.
% If the argument is not empty and does not match |\childdocname|
% (which is set by the first inclusion of |childdoc.def|),
% |\ifchilddoc| is set to true, |\includeonly| is applied to the child file
% and |\jobname| is set to the main file
% (for proper handling of |.aux| files):
%    \begin{macrocode}
\newcommand{\childdocmain}[1]
{
  \childdocdisable\childdocmain{}
  \if?#1?\else
    \begingroup
      \def\childdoctmp{#1}
      \ifx\childdoctmp\childdocname
        \def\childdoctmp{}
      \else
        \def\childdoctmp
        {
          \childdoctrue
          \includeonly{\childdocname}
          \def\childdocjob{#1}
          \def\jobname{#1}
        }
      \fi
      \expandafter
    \endgroup
    \childdoctmp
  \fi
}
%    \end{macrocode}

% \macro{\childdocof}
% The command |\childdocof| redirects
% compilation to the main file |#1|.
%    \begin{macrocode}
\newcommand{\childdocof}[1]
{
  \childdocdisable
  \childdoctrue
  \includeonly{\childdocname}
  \def\jobname{#1}
  \def\childdocjob{#1}
  \input{#1}
}
%    \end{macrocode}

% \macro{\childdocby}
% The command |\childdocby| ....
%    \begin{macrocode}
\newcommand{\childdocby}[2][]
{
  \childdocdisable
  \childdoctrue
  \childdocmanualtrue
  \if?#1?\else
    \def\jobname{#2}
  \fi
  \def\childdocjob{#2}
  \input{#2}
  \endinput
}
%    \end{macrocode}

% \macro{\childdocforward}
% The command |\childdocforward| redirects
% compilation to the main file or
% (if the optional argument is given) a child file.
% Parameters are set as if the main file
% or a child file starting with |\childdocof| was compiled.
% Then compilation is handed over to the main file:
%    \begin{macrocode}
\newcommand{\childdocforward}[2][]
{
  \begingroup
    \if?#1?
      \def\childdoctmp
      {
        \def\childdocname{#2}
        \def\childdocjob{#2}
        \def\jobname{#2}
        \input{#2}
        \endinput
      }
    \else
      \def\childdoctmp
      {
        \childdocdisable
        \def\childdocname{#2}
        \childdoctrue
        \includeonly{#2}
        \def\childdocjob{#1}
        \def\jobname{#1}
        \input{#1}
        \endinput
      }
    \fi
    \expandafter
  \endgroup
  \childdoctmp
}
%    \end{macrocode}

% \macro{\childdocforwardprefix}
% The command |\childdocforwardprefix| redirects
% compilation to the main or a child file by means of a pattern.
% The prefix |#1| in the current filename is replaced by |#2|
% and the suffix of the current filename is kept
% (it is assumed that the filename does not contain the substring `|~~~|'
% which is used as a delimiter).
% Compilation is handed over to the new file by |\childdocforward|:
%    \begin{macrocode}
\newcommand{\childdocforwardprefix}[3][]
{
  \begingroup
    \def\childdocextract #2##1~~~{\def\childdoctmp{\childdocforward[#1]{#3##1}}}
    \expandafter\childdocextract\childdocname~~~
    \expandafter
  \endgroup
  \childdoctmp
}
%    \end{macrocode}

% \macro{\childdoc}
% The deprecated macro |\childdoc| is a legacy version of |\childdocmain|:
%    \begin{macrocode}
\newcommand{\childdoc}{\childdocmain}
%    \end{macrocode}

% \macro{\childdocredirect}
% The deprecated macro |\childdocredirect| is a legacy version
% of |\childdocforward| and |\childdocforwardprefix|:
%    \begin{macrocode}
\newcommand{\childdocredirect}[2][]
{
  \begingroup
    \if?#1?
      \def\childdoctmp{\childdocforward{#2}}
    \else
      \def\childdoctmp{\childdocforwardprefix{#1}{#2}}
    \fi
    \expandafter
  \endgroup
  \childdoctmp
}
%    \end{macrocode}

%\iffalse
%</package>
%\fi
%
\endinput

\childdocforward{cdocsamp}
%    \end{macrocode}

%\iffalse
%</sampledraft>
%\fi
%
% %%%%%%%%%%%%%%%%%%%%%%%%%%%%%%%%%%%%%%
% \paragraph{Forwarding for Final Version of the Chapters.}
%
% The following forwarding files |cdocsfn1.tex| and |cdocsfn2.tex|
% (with identical content)
% compile the final versions of the child documents
% |cdocsch1.tex| and |cdocsch2.tex|, respectively:
%\iffalse
%<*samplefinal>
%\fi
%    \begin{macrocode}
\def\version{final}
% \iffalse
%
% childdoc.dtx Copyright (C) 2017-2018 Niklas Beisert
%
% This work may be distributed and/or modified under the
% conditions of the LaTeX Project Public License, either version 1.3
% of this license or (at your option) any later version.
% The latest version of this license is in
%   http://www.latex-project.org/lppl.txt
% and version 1.3 or later is part of all distributions of LaTeX
% version 2005/12/01 or later.
%
% This work has the LPPL maintenance status `maintained'.
%
% The Current Maintainer of this work is Niklas Beisert.
%
% This work consists of the files childdoc.dtx and childdoc.ins
% and the derived files childdoc.def and cdocsamp.tex with
% cdocsch1.tex, cdocsch2.tex, cdocsdrf.tex, cdocsfn1.tex, cdocsfn2.tex.
%
%<package>\ifdefined\childdocmain\endinput\fi
%<package>\ProvidesFile{childdoc.def}[2018/12/30 v2.0 child document driver]
%<samplemain>\ProvidesFile{cdocsamp.tex}[2018/12/30 v2.0 sample for childdoc]
%<*driver>
%\ProvidesFile{childdoc.drv}[2018/12/30 v2.0 childdoc reference manual file]
\PassOptionsToClass{10pt,a4paper}{article}
\documentclass{ltxdoc}

\usepackage[margin=35mm]{geometry}
\usepackage{hyperref}
\usepackage{hyperxmp}
\usepackage[usenames]{color}

\hypersetup{colorlinks=true}
\hypersetup{pdfstartview=FitH}
\hypersetup{pdfpagemode=UseNone}
\hypersetup{pdfsource={}}
\hypersetup{pdflang={en-UK}}
\hypersetup{pdfcopyright={Copyright 2017-2018 Niklas Beisert.
  This work may be distributed and/or modified under the
  conditions of the LaTeX Project Public License, either version 1.3
  of this license or (at your option) any later version.}}
\hypersetup{pdflicenseurl={http://www.latex-project.org/lppl.txt}}
\hypersetup{pdfcontactaddress={ETH Zurich, ITP, HIT K,
  Wolfgang-Pauli-Strasse 27}}
\hypersetup{pdfcontactpostcode={8093}}
\hypersetup{pdfcontactcity={Zurich}}
\hypersetup{pdfcontactcountry={Switzerland}}
\hypersetup{pdfcontactemail={nbeisert@itp.phys.ethz.ch}}
\hypersetup{pdfcontacturl={http://people.phys.ethz.ch/\xmptilde nbeisert/}}

\newcommand{\secref}[1]{\hyperref[#1]{section \ref*{#1}}}

\parskip1ex
\parindent0pt
\let\olditemize\itemize
\def\itemize{\olditemize\parskip0pt}

\begin{document}

\title{The \textsf{childdoc} Package}
\hypersetup{pdftitle={The childdoc Package}}
\author{Niklas Beisert\\[2ex]
  Institut f\"ur Theoretische Physik\\
  Eidgen\"ossische Technische Hochschule Z\"urich\\
  Wolfgang-Pauli-Strasse 27, 8093 Z\"urich, Switzerland\\[1ex]
  \href{mailto:nbeisert@itp.phys.ethz.ch}
  {\texttt{nbeisert@itp.phys.ethz.ch}}}
\hypersetup{pdfauthor={Niklas Beisert}}
\hypersetup{pdfsubject={Manual for the LaTeX2e Package childdoc}}
\date{30 December 2018, \textsf{v2.0}}
\maketitle

\begin{abstract}\noindent
\textsf{childdoc} is a \LaTeXe{} package
that enables the direct compilation
of document sections included by |\include|
to individual files.
\end{abstract}

\begingroup
\parskip0ex
\tableofcontents
\endgroup

%%%%%%%%%%%%%%%%%%%%%%%%%%%%%%%%%%%%%%%%%%%%%%%%%%%%%%%%%%%%%%%%%%%%%%%%%%%%%%%%
%%%%%%%%%%%%%%%%%%%%%%%%%%%%%%%%%%%%%%%%%%%%%%%%%%%%%%%%%%%%%%%%%%%%%%%%%%%%%%%%
\section{Introduction}

\LaTeX{} provides a mechanism to structure a large document (such as a book)
into a main file and several child files (containing the chapters)
using the |\include| command.
This mechanism is beneficial for documents
which span hundreds of pages in order to
make the source file(s) more manageable.
Moreover, compilation can be restricted to
selected child files by means of the |\includeonly| command.
The latter feature can be used to reduce the compilation time while editing
(this was significantly more useful in the earlier days of \LaTeX{})
or to generate a smaller document which is easier to navigate.
Another application of |\includeonly| is to generate
documents consisting of selected parts of the complete document.

However, there are a few drawbacks of the plain |\include| mechanism:
\begin{itemize}
\item
The child files cannot be compiled on their own,
they can only be compiled via the main file.
A naive editing environment
(such as a text editor with an option
to have the current file processed by \LaTeX)
may require one to switch to the main file before compiling;
attempting to compile the child file produces errors.
\item
The main file must be modified (each time)
to adjust the |\includeonly| command
to the present needs. This easily leaves the main file in a messy state.
\item
The generated document will always carry the filename
of the main document. This is inconvenient if
several child files are to be compiled and
to be kept for distribution.
\end{itemize}

The present package provides a simple interface
to make child files individually compilable by \LaTeX{}.
Compiling a child file then has the same effect as compiling
the main file with an |\includeonly| command
to select the appropriate child.
Moreover the generated document will carry the name of the child
rather than the main file.
This resolves all three above issues.

This feature is meant to make the editing of books,
thesis documents and lecture notes somewhat more convenient.
However, the package can also be used efficiently for
composing a series of documents (such as exercise sheets)
which are typically distributed individually.
It then assists the author in generating the individual documents
(potentially in different versions)
as well as a document containing the collected series.
Another application is in developing style files
or other kinds of included material
where compilation of the style file could redirect
to a sample or test file.

%%%%%%%%%%%%%%%%%%%%%%%%%%%%%%%%%%%%%%%%%%%%%%%%%%%%%%%%%%%%%%%%%%%%%%%%%%%%%%%%
%%%%%%%%%%%%%%%%%%%%%%%%%%%%%%%%%%%%%%%%%%%%%%%%%%%%%%%%%%%%%%%%%%%%%%%%%%%%%%%%
\section{Usage}

First of all, the package \textsf{childdoc} is \emph{not} a standard
\LaTeXe{} |.sty| style file! Therefore it needs to be invoked in
a non-standard way.

%%%%%%%%%%%%%%%%%%%%%%%%%%%%%%%%%%%%%%%%%%%%%%%%%%%%%%%%%%%%%%%%%%%%%%%%%%%%%%%%
\subsection{Included Files}
\label{sec:include}

%%%%%%%%%%%%%%%%%%%%%%%%%%%%%%%%%%%%%%%%
\DescribeMacro{\childdocmain}
To use the package, add the commands
\begin{center}
\begin{tabular}{l}
|\input{childdoc.def}|\\
|\childdocmain{}|\\
\end{tabular}
\end{center}
at the very top of the main \LaTeX{} file,
in particular \emph{before} the |\documentclass| statement!
The argument of |\childdocmain| should be left empty
(but it must be present).

%%%%%%%%%%%%%%%%%%%%%%%%%%%%%%%%%%%%%%%%
\DescribeMacro{\childdocof}
Furthermore, add the commands
\begin{center}
\begin{tabular}{l}
|\input{childdoc.def}|\\
|\childdocof{|\textit{main}|}|\\
\end{tabular}
\end{center}
at the top of every child file \textit{child}
which is included by |\include{|\textit{child}|}|
from within the main file
(or at least for those files to be compiled individually).
The argument \textit{main} must be the filename of the main file.

There are a couple of
considerations in setting up the main and child documents:

%%%%%%%%%%%%%%%%%%%%%%%%%%%%%%%%%%%%%%%%
\paragraph{Restrictions.}

Please note the following restrictions:
\begin{itemize}
\item
|\childdocmain| must be called with one argument \textit{main}
to ensure compatibility with earlier version of the package.
It must either be empty (|\childdocmain{}|)
or precisely match the filename of the main file in which it is specified.
See \secref{sec:detection} for further information.
\item
The filename \textit{main} must be specified without the |.tex| extension.
\item
The filename \textit{main} is case sensitive
(even in case-insensitive file systems)
due to internal string comparison.
\item
The argument \textit{main} should be fully expanded, it cannot be a macro.
\item
Subdirectories and special characters should be avoided in filenames.
\item
The command |\childdocmain{|\textit{main}|}| must be followed by a whitespace.
It should not be followed immediately by another command
or by a comment mark `|%|'.
This is because the \TeX{} parser reads the token immediately following
the argument of |\childdocmain| and puts it
at the beginning of every child section;
however, a white\-space is ignored.
\end{itemize}

%%%%%%%%%%%%%%%%%%%%%%%%%%%%%%%%%%%%%%%%
\paragraph{Content of Main File.}

It is advisable to place all content in the child files included by |\include|.
Any output contained in the main file will appear in all child documents
unless suppressed manually;
it cannot be suppressed automatically by the |\includeonly| directive
and thus should normally be avoided.
A method to include some content in the main file
by means of conditional processing is described in \secref{sec:conditional}.

%%%%%%%%%%%%%%%%%%%%%%%%%%%%%%%%%%%%%%%%
\paragraph{Page Numbering.}

When only a part of the document is compiled,
the appropriate numbering of pages
(as well as other status parameters)
is determined from the |.aux| files.
The latter contain information from previous passes.
However this information needs to propagate through
all intermediate child documents.
Therefore the page numbering in child documents may well
be inconsistent until the complete document is compiled at least once.

A useful (if unconventional) way to always ensure a consistent
page numbering is to restart the numbering in each child document
and denote the pages by `\textit{child}|.|\textit{page}'
where \textit{child} represents the chapter/section number of the child file.
This can be achieved by the command
|\numberwithin{page}{|\textit{child}|}|
of the \textsf{amsmath} package
where \textit{child} can be |chapter| or |section|
depending on the chosen structuring.
Alternatively, one can modify the macro |\thepage| appropriately
and reset the counter |page| at the start of each child file.

%%%%%%%%%%%%%%%%%%%%%%%%%%%%%%%%%%%%%%%%%%%%%%%%%%%%%%%%%%%%%%%%%%%%%%%%%%%%%%%%
\subsection{Conditional Processing}
\label{sec:conditional}

The package provides a mechanism to compile different versions
of a document. To customise the versions further some conditional processing
can come in handy to distinguish which version is being compiled.
The package provides two macros to describe the compilation context:

%%%%%%%%%%%%%%%%%%%%%%%%%%%%%%%%%%%%%%%%
\DescribeMacro{\ifchilddoc}
The conditional |\ifchilddoc| distinguishes between the compilation of
child documents and the main document:
%
\begin{center}
|\ifchilddoc |\textit{child-code}| |[|\||else |\textit{main-code}]| \||fi|
\end{center}

%%%%%%%%%%%%%%%%%%%%%%%%%%%%%%%%%%%%%%%%
\DescribeMacro{\childdocname}
\DescribeMacro{\childdocjob}
The macro |\childdocname| contains the filename (without extension)
of the main or child file being processed.
Note that |\childdocjob| will always contain the name of the main file.

%%%%%%%%%%%%%%%%%%%%%%%%%%%%%%%%%%%%%%%%
\paragraph{Title Page.}

Conditional processing can be used to include a title or banner page
in the main document when proper precautions are taken.
Importantly, the code in the main file should ensure that the page counter
(as well as other status parameters which are stored in the |.aux| files)
takes the same value after the conditional processing.
Otherwise the page numbers may take divergent values
depending on which part is compiled.

For example, a title page could be declared by:
%
\begin{center}
\begin{tabular}{l}
|\ifchilddoc\||else|\\
|\addtocounter{page}{-1}|\\
\textit{code for title page}\\
|\newpage|\\
|\||fi|
\end{tabular}
\end{center}
%
A banner page for the child documents can be generated by:
%
\begin{center}
\begin{tabular}{l}
|\ifchilddoc|\\
|\addtocounter{page}{-1}|\\
\textit{code for banner page}\\
|\newpage|\\
|\||fi|
\end{tabular}
\end{center}
%
Here one could write a message such as:
\begin{center}
|This is the part \childdocname{} of \childdocjob{}.|
\end{center}

%%%%%%%%%%%%%%%%%%%%%%%%%%%%%%%%%%%%%%%%%%%%%%%%%%%%%%%%%%%%%%%%%%%%%%%%%%%%%%%%
\subsection{Flags}
\label{sec:flags}

The package makes it easy to generate different versions
of the main or child documents.
To this end compilation flags can be defined
and assigned different default values.
They will be particularly useful in conjunction
with the forwarding mechanism described in \secref{sec:forward}.

For example, it may be useful to have a flag |\version|
which can be set to |draft| or |final|.
The document source will contain some conditional code
depending on the value of |\version|.
Suppose further, the flag should default to |final| for the main file
and to |draft| for child files
which is a natural assignment for editing the document.
This is achieved by placing the following code
in the preamble of the main document
(below the |\childdocmain| directive):
%
\begin{center}
\begin{tabular}{l}
|\ifchilddoc|\\
|\providecommand{\version}{draft}|\\
|\||else|\\
|\providecommand{\version}{final}|\\
|\||fi|
\end{tabular}
\end{center}
%
The definition by |\providecommand| makes sure
that previous definitions are not overwritten.
Further statements |\providecommand{\version}{...}|
can thus be added before the above code to override it.

For the main file, one might add a line
(between |\childdocmain| and the above block)
%
\begin{center}
|%\ifchilddoc\||else\providecommand{\version}{draft}\||fi|
\end{center}
%
which can be uncommented to produce a draft version.
Likewise one can add a line to the very top of a child file
(above the |\childdocof{|\textit{main}|}| directive)
%
\begin{center}
|%\providecommand{\version}{final}|
\end{center}
%
which can be uncommented to produce the final version of this child document.

%%%%%%%%%%%%%%%%%%%%%%%%%%%%%%%%%%%%%%%%%%%%%%%%%%%%%%%%%%%%%%%%%%%%%%%%%%%%%%%%
\subsection{Forwarding}
\label{sec:forward}

Different versions of the main or child documents
using compilation flags as described in \secref{sec:flags}
can be (permanently) stored in different files
for convenient compilation, viewing and distribution.
To this end, the package defines a command
to pass on compilation to a different file:

%%%%%%%%%%%%%%%%%%%%%%%%%%%%%%%%%%%%%%%%
\DescribeMacro{\childdocforward}
The command |\childdocforward| redirects processing to
another source file:
%
\begin{center}
\begin{tabular}{l}
|\input{childdoc.def}|\\
|\childdocforward[|\textit{main}|]{|\textit{dest}|}|\\
\end{tabular}
\end{center}
%
The argument \textit{dest} is the destination file
(without extension).
It should be the main file or one of the child files.
Note that further \textsf{childdoc} directives
such as |\childdocof| and |\childdocforward|
in the indicated file will be processed in this form.
The optional argument \textit{main}
passes on directly to the main file \textit{main}
while pretending to compile the child \textit{dest}.
This form behaves as if \textit{dest}
issues |\childdocof{|\textit{main}|}| right away,
and no further \textsf{childdoc} directives will be processed.

%%%%%%%%%%%%%%%%%%%%%%%%%%%%%%%%%%%%%%%%
\DescribeMacro{\...prefix}
In the alternative form |\childdocforwardprefix|,
%
\begin{center}
\begin{tabular}{l}
|\input{childdoc.def}|\\
|\childdocforwardprefix[|\textit{main}|]{|\textit{prefix}|}{|\textit{dest}|}|
\end{tabular}
\end{center}
%
the destination file is determined by a pattern
depending on the current file:
To make this work, the current file must be called
`{\textit{prefix}\hspace{0.2em}\textit{suffix}}'
with \textit{prefix} matching precisely the argument.
Processing is then passed on to the file
`{\textit{dest}\hspace{0.2em}\textit{suffix}}'.
Surely, the same effect is achieved by
directly specifying the
argument `{\textit{dest}\hspace{0.2em}\textit{suffix}}'
in the first form.
However, that requires to set up a different file
for each child. With the alternative form of the command
all these files can have exactly the same content
which simplifies setting them up and maintaining them.

For example, the following file |draft.tex|
with a compilation flag |\version| as described in \secref{sec:flags}
compiles the main document as a draft:
%
\begin{center}
\begin{tabular}{l}
|\def\version{draft}|\\
|\input{childdoc.def}|\\
|\childdocforward{|\textit{main}|}|
\end{tabular}
\end{center}
%
Likewise, the following files |final|\textit{nn}|.tex|
compile the final version of the child document
|child|\textit{nn}|.tex|:
%
\begin{center}
\begin{tabular}{l}
|\def\version{final}|\\
|\input{childdoc.def}|\\
|\childdocforwardprefix{final}{child}|
\end{tabular}
\end{center}
%

Note that when several versions of a main file and/or of each child file
are to be generated, it may be convenient to set up a |Makefile| or
shell script to automatise the process.

%%%%%%%%%%%%%%%%%%%%%%%%%%%%%%%%%%%%%%%%%%%%%%%%%%%%%%%%%%%%%%%%%%%%%%%%%%%%%%%%
\subsection{Command Line Processing}
\label{sec:commandline}

The effect of redirection files can also be achieved by invoking
the \LaTeX{} compiler with a more elaborate command line.
Most conveniently this should be done as part
of a shell script or a |Makefile|.

When using \textsf{childdoc} in the main file, the following
command lines effectively perform a redirection
(note that depending on the shell being used,
backslashes may have to be doubled: `|\|' $\to$ `|\\|'):
%
\begin{center}
|... -jobname "|\textit{target}|" |\\|"|[\textit{flags}]%
|\input{childdoc.def}\childdocforward[|\textit{main}|]{|\textit{dest}|}"|
\end{center}
%
Here \textit{target} is the name of the output file,
\textit{main} is the name of the main file
and \textit{dest} is the name of the main or child file to be processed
(all filenames without extensions).
The optional argument \textit{main} can be omitted
if \textit{main} matches \textit{dest}.
Optionally, compilation \textit{flags} can be defined via |\def| commands.
This command line makes the \TeX{} engine believe
it is compiling the file \textit{target}
whose content is specified as the latter parameter.
The provided code then forwards the processing to
\textit{main} or \textit{dest} as described in \secref{sec:forward}.

%%%%%%%%%%%%%%%%%%%%%%%%%%%%%%%%%%%%%%%%%%%%%%%%%%%%%%%%%%%%%%%%%%%%%%%%%%%%%%%%
\subsection{Include by Input}
\label{sec:input}

Including child documents by |\include| has some restrictions by design.
Most notably, the content of a child document always occupies
its own set of pages; pages cannot be shared between child documents.
Usually, this behaviour makes perfect sense
because each child document contain an essential part of the document.
However, in some situations it may be desirable to compose
a document from a collection of parts
without having mandatory page breaks between then.
For this case, the package
provides a mechanism to include parts
by |\input| which can also be processed individually.
However, by construction this mechanism
requires manual handling of the content to be output.

%%%%%%%%%%%%%%%%%%%%%%%%%%%%%%%%%%%%%%%%
\DescribeMacro{\ifchilddocmanual}
The main file should be prepared as usual, see \secref{sec:include}.
However, the document body must make a distinction
between processing of an individual part and of the main document, e.g.:
%
\begin{center}
\begin{tabular}{l}
|\ifchilddocmanual|\\
|\input{\childdocname}|\\
|\||else|\\
\textit{document body with }|\input{|\textit{part}|}|\\
|\||fi|
\end{tabular}
\end{center}
%
The conditional |\ifchilddocmanual| is true whenever
a part to be included by |\input| is being compiled,
and the name of the part is stored in |\childdocname|.

%%%%%%%%%%%%%%%%%%%%%%%%%%%%%%%%%%%%%%%%
\DescribeMacro{\childdocby}
Each part to be included by |\input| should start with:
%
\begin{center}
\begin{tabular}{l}
|\input{childdoc.def}|\\
|\childdocby{|\textit{main}|}|\\
\end{tabular}
\end{center}
%
The directive |\childdocby| is similar to |\childdocof|
described in \secref{sec:include},
but the subsequent selection of content must be done manually.
To that end, both |\ifchilddoc| and |\ifchilddocmanual|
will be true upon processing of a part,
and the name of the part is stored in |\childdocname|.
Note that |\jobname| will be set to the filename of the current part
so that each part receives an individual |.aux| file
that does not interfere with the |.aux| file(s) of the main document.
This behaviour can be altered by the alternative form
|\childdocby[*]{|\textit{main}|}| (with a non-empty optional argument)
which uses the |.aux| file of the main document
by setting |\jobname| to \textit{main}.

%%%%%%%%%%%%%%%%%%%%%%%%%%%%%%%%%%%%%%%%%%%%%%%%%%%%%%%%%%%%%%%%%%%%%%%%%%%%%%%%
\subsection{Driver Development}
\label{sec:driver}

The \textsf{childdoc} mechanism can also be use for the development
of definition files such as \LaTeX{} styles or classes.
This case differs from the above setup with multiple parts
included by |\include| in that no |\includeonly| should be invoked.
This can be achieved by starting the include file
(before |\ProvidesPackage|) with:
%
\begin{center}
\begin{tabular}{l}
|\input{childdoc.def}|\\
|\childdocforward{|\textit{main}|}|\\
\end{tabular}
\end{center}
%
or alternatively with:
%
\begin{center}
\begin{tabular}{l}
|\input{childdoc.def}|\\
|\childdocby{|\textit{main}|}|\\
\end{tabular}
\end{center}
%
Both forms have slightly different effects as described above.
The main file is prepared as usual, see \secref{sec:include}.

%%%%%%%%%%%%%%%%%%%%%%%%%%%%%%%%%%%%%%%%%%%%%%%%%%%%%%%%%%%%%%%%%%%%%%%%%%%%%%%%
\subsection{Legacy Detection}
\label{sec:detection}

The directive |\childdocmain| in the main file can detect
whether the complete document or merely a child is to be compiled
even without using the directive |\childdocof|.
This method is deprecated because it is less robust
and there is no compelling reason to use it;
it is merely provided for backward compatibility
and it may be removed in future versions.

If the detection mechanism is to be used,
it is mandatory to correctly specify
the filename of the main file as the argument of |\childdocmain|:
%
\begin{center}
\begin{tabular}{l}
|\input{childdoc.def}|\\
|\childdocmain{|\textit{main}|}|\\
\end{tabular}
\end{center}
%
If |\jobname| does not match the argument \textit{main} of |\childdocmain|,
it is assumed that |\jobname| points to the child file to be compiled.
When using |\childdocmain| with the main file specified as argument,
it suffices to start a child file
with just |\input{|\textit{main}|}|
without loading of the package and using |\childdocof|.
If instead all processing is done
with the appropriate \textsf{childdoc} directives,
the argument of \textit{main} of |\childdocmain| can be empty.

An alternative version of the command line processing described
in \secref{sec:commandline} using the detection mechanism reads:
%
\begin{center}
|... -jobname "|\textit{target}|" "|[\textit{flags}]%
[|\def\jobname{|\textit{dest}|}|]|\input{|\textit{main}|}"|
\end{center}

%%%%%%%%%%%%%%%%%%%%%%%%%%%%%%%%%%%%%%%%%%%%%%%%%%%%%%%%%%%%%%%%%%%%%%%%%%%%%%%%
\subsection{Manual Code}
\label{sec:manual}

In case one cannot be certain whether the definitions file |childdoc.def|
is installed on the target \TeX{} distribution
and one prefers not to ship it,
it is conceivable to paste a few relevant commands into the sources.

To that end, drop all statements |\input{childdoc.def}|
and perform the replacements as outlined below.
Instead of |\childdocmain{|\textit{main}|}| add the following code
to the top of the main file:
%
\begin{center}
\begin{tabular}{l}
|\||ifdefined\childdocname\endinput\||fi\newif\ifchilddoc|\\
|\edef\childdocname{\scantokens\expandafter{\jobname\noexpand}}|\\
|\def\childdocmain{|\textit{main}|}\||ifx\childdocmain\childdocname\||else|\\
|\childdoctrue\includeonly{\childdocname}\let\jobname\childdocmain\||fi|\\
\end{tabular}
\end{center}
%
Instead of |\childdocof{|\textit{main}|}| just include the main file
at the top of each child file:
%
\begin{center}
|\input{|\textit{main}|}|
\end{center}
%
A simple redirection |\childdocforward{|\textit{dest}|}| is achieved by:
%
\begin{center}
|\def\jobname{|\textit{dest}|}\input{\jobname}|
\end{center}
%
The redirection with prefix
|\childdocforwardprefix[|\textit{prefix}|]{|\textit{dest}|}|
is accomplished by:
%
\begin{center}
\begin{tabular}{l}
|{\edef\jobname{\scantokens\expandafter{\jobname\noexpand}}|\\
|\def\redirectjob |\textit{prefix}|#1~~~{\gdef\jobname{|\textit{dest}|#1}}|\\
|\expandafter\redirectjob\jobname~~~}\input{\jobname}|
\end{tabular}
\end{center}

In an alternative approach,
child documents can be compiled by a specific command line
without additional code or specific definitions:
%
\begin{center}
|... -jobname "|\textit{target}|" "|[\textit{flags}]%
|\includeonly{|\textit{dest}|}\input{|\textit{main}|}"|
\end{center}
%

%%%%%%%%%%%%%%%%%%%%%%%%%%%%%%%%%%%%%%%%%%%%%%%%%%%%%%%%%%%%%%%%%%%%%%%%%%%%%%%%
%%%%%%%%%%%%%%%%%%%%%%%%%%%%%%%%%%%%%%%%%%%%%%%%%%%%%%%%%%%%%%%%%%%%%%%%%%%%%%%%
\section{Information}

%%%%%%%%%%%%%%%%%%%%%%%%%%%%%%%%%%%%%%%%%%%%%%%%%%%%%%%%%%%%%%%%%%%%%%%%%%%%%%%%
\subsection{Copyright}

Copyright \copyright{} 2017--2018 Niklas Beisert

This work may be distributed and/or modified under the
conditions of the \LaTeX{} Project Public License, either version 1.3
of this license or (at your option) any later version.
The latest version of this license is in
  \url{http://www.latex-project.org/lppl.txt}
and version 1.3 or later is part of all distributions of \LaTeX{}
version 2005/12/01 or later.

This work has the LPPL maintenance status `maintained'.

The Current Maintainer of this work is Niklas Beisert.

This work consists of the files |README.txt|, |childdoc.ins| and |childdoc.dtx|
as well as the derived files |childdoc.def|, |cdocsamp.tex|
with |cdocsch1.tex|, |cdocsch2.tex|, |cdocspt3.tex|, |cdocspt4.tex|,
|cdocsdrf.tex|, |cdocsfn1.tex|, |cdocsfn2.tex|
as well as |childdoc.pdf|.

%%%%%%%%%%%%%%%%%%%%%%%%%%%%%%%%%%%%%%%%%%%%%%%%%%%%%%%%%%%%%%%%%%%%%%%%%%%%%%%%
\subsection{Files and Installation}

The package consists of the files:
%
\begin{center}
\begin{tabular}{ll}
    |README.txt|   & readme file \\
    |childdoc.ins| & installation file \\
    |childdoc.dtx| & source file \\
    |childdoc.def| & definition file \\
    |cdocsamp.tex| & sample main file \\
    |cdocsch1.tex| & sample include file \\
    |cdocsch2.tex| & sample include file \\
    |cdocspt3.tex| & sample part file \\
    |cdocspt4.tex| & sample part file \\
    |cdocsdrf.tex| & sample redirection file \\
    |cdocsfn1.tex| & sample redirection file \\
    |cdocsfn2.tex| & sample redirection file \\
    |childdoc.pdf| & manual
\end{tabular}
\end{center}
%
The distribution consists of the files
|README.txt|, |childdoc.ins| and |childdoc.dtx|.
%
\begin{itemize}
\item
Run (pdf)\LaTeX{} on |childdoc.dtx|
to compile the manual |childdoc.pdf| (this file).
\item
Run \LaTeX{} on |childdoc.ins| to create the definitions file |childdoc.def|
and the sample |cdocsamp.tex| with include files
|cdocsch1.tex|, |cdocsch2.tex|, |cdocspt3.tex|, |cdocspt4.tex|,
|cdocsdrf.tex|, |cdocsfn1.tex|, |cdocsfn2.tex|.
Then copy the file |childdoc.def| to an appropriate directory of your \LaTeX{}
distribution, e.g.\ \textit{texmf-root}|/tex/latex/childdoc|.
\end{itemize}

%%%%%%%%%%%%%%%%%%%%%%%%%%%%%%%%%%%%%%%%%%%%%%%%%%%%%%%%%%%%%%%%%%%%%%%%%%%%%%%%
\subsection{Related CTAN Packages}

There are several other packages which offer a similar functionality:
%
\begin{itemize}
\item
The packages
\href{http://ctan.org/pkg/docmute}{\textsf{docmute}},
\href{http://ctan.org/pkg/includex}{\textsf{includex}} and
\href{http://ctan.org/pkg/standalone}{\textsf{standalone}}
provide commands to include only the document body of
a child file thus allowing both files to be compiled individually.
\item
The packages \href{http://ctan.org/pkg/subdocs}{\textsf{subdocs}}
and \href{http://ctan.org/pkg/subfiles}{\textsf{subfiles}}
provide structures in which the main and child documents can be
encapsulated and allowing them to be compiled individually.
The inclusion mechanism is different from the conventional |\include|.
\item
The package \href{http://ctan.org/pkg/combine}{\textsf{combine}}
is an elaborate solution to combine several documents into one.
\end{itemize}
%
See also the CTAN topic \href{http://ctan.org/topic/subdocs}{\textsf{subdocs}}
for further related packages.
The present package differs from the above solutions in that
a document structure constructed with the conventional |\include| mechanism
just needs two extra commands at the top of every file
such that all constituent files can be compiled individually.

%%%%%%%%%%%%%%%%%%%%%%%%%%%%%%%%%%%%%%%%%%%%%%%%%%%%%%%%%%%%%%%%%%%%%%%%%%%%%%%%
%\subsection{Feature Suggestions}
%
%The following is a list of features which may be useful for future
%versions of this package:
%%
%\begin{itemize}
%\item
%\ldots
%\end{itemize}

%%%%%%%%%%%%%%%%%%%%%%%%%%%%%%%%%%%%%%%%%%%%%%%%%%%%%%%%%%%%%%%%%%%%%%%%%%%%%%%%
\subsection{Revision History}

%%%%%%%%%%%%%%%%%%%%%%%%%%%%%%%%%%%%%%%%
\paragraph{v2.0:} 2018/12/30

\begin{itemize}
\item
immediate forward processing
\item
added |\childdocby| mechanism
\item
manual restructured
\end{itemize}

%%%%%%%%%%%%%%%%%%%%%%%%%%%%%%%%%%%%%%%%
\paragraph{v1.6:} 2018/01/17

\begin{itemize}
\item
application for development of include files
\item
corrections to manual
\end{itemize}

%%%%%%%%%%%%%%%%%%%%%%%%%%%%%%%%%%%%%%%%
\paragraph{v1.5:} 2017/05/21

\begin{itemize}
\item
more complete structuring introduced
\item
|\childdocof| introduced
\item
|\childdoc| renamed to |\childdocmain|
\item
|\childredirect| renamed to |\childdocforward| and |\childdocforwardprefix|
and functionality expanded
\end{itemize}

%%%%%%%%%%%%%%%%%%%%%%%%%%%%%%%%%%%%%%%%
\paragraph{v1.0:} 2017/04/27

\begin{itemize}
\item
manual and install package
\item
first version published on CTAN
\end{itemize}

%%%%%%%%%%%%%%%%%%%%%%%%%%%%%%%%%%%%%%%%
\paragraph{v0.6:} 2017/04/26

\begin{itemize}
\item
redirection mechanism added
\end{itemize}

%%%%%%%%%%%%%%%%%%%%%%%%%%%%%%%%%%%%%%%%
\paragraph{v0.5:} 2017/04/26

\begin{itemize}
\item
functionality in definition file
\end{itemize}


%%%%%%%%%%%%%%%%%%%%%%%%%%%%%%%%%%%%%%%%%%%%%%%%%%%%%%%%%%%%%%%%%%%%%%%%%%%%%%%%
%%%%%%%%%%%%%%%%%%%%%%%%%%%%%%%%%%%%%%%%%%%%%%%%%%%%%%%%%%%%%%%%%%%%%%%%%%%%%%%%
%%%%%%%%%%%%%%%%%%%%%%%%%%%%%%%%%%%%%%%%%%%%%%%%%%%%%%%%%%%%%%%%%%%%%%%%%%%%%%%%
\appendix

\settowidth\MacroIndent{\rmfamily\scriptsize 000\ }

 \DocInput{childdoc.dtx}

\end{document}
%</driver>
% \fi
%
% %%%%%%%%%%%%%%%%%%%%%%%%%%%%%%%%%%%%%%%%%%%%%%%%%%%%%%%%%%%%%%%%%%%%%%%%%%%%%%
% %%%%%%%%%%%%%%%%%%%%%%%%%%%%%%%%%%%%%%%%%%%%%%%%%%%%%%%%%%%%%%%%%%%%%%%%%%%%%%
% \section{Sample}
%\iffalse
%<*samplemain>
%\fi
%
% The following presents a sample document
% with two chapters, two parts, a title page,
% a compile flag as well as three forwarding files to set the flag.
% It consists of eight |.tex| files:
% \begin{center}
% \begin{tabular}{ll}
% |cdocsamp.tex|&main file\\
% |cdocsch1.tex|&include file for chapter 1\\
% |cdocsch2.tex|&include file for chapter 2\\
% |cdocspt3.tex|&include file for part 3\\
% |cdocspt4.tex|&include file for part 4\\
% |cdocsdrf.tex|&forwarding file for main file in draft mode\\
% |cdocsfi1.tex|&forwarding file for final version of chapter 1\\
% |cdocsfi2.tex|&forwarding file for final version of chapter 2\\
% \end{tabular}
% \end{center}
% Each of the eight files can be compiled directly by the \LaTeX{} compiler.
%
% %%%%%%%%%%%%%%%%%%%%%%%%%%%%%%%%%%%%%%
% \paragraph{Main File.}
%
% The main file is called |cdocsamp.tex|.
%
% Load the \textsf{childdoc} definitions and
% declare the filename for the main document:
%    \begin{macrocode}
\input{childdoc.def}
\childdocmain{}
%    \end{macrocode}

% Optional override for |\version| flag:
%    \begin{macrocode}
%%\ifchilddoc\else\providecommand{\version}{draft}\fi
%    \end{macrocode}

% Define the default values for the |\version| flag
% (|final| for the main file and |draft| for childs):
%    \begin{macrocode}
\ifchilddoc
\providecommand{\version}{draft}
\else
\providecommand{\version}{final}
\fi
%    \end{macrocode}

% Load the standard document class:
%    \begin{macrocode}
\documentclass[12pt]{article}
%    \end{macrocode}

% Start the document body:
%    \begin{macrocode}
\begin{document}
%    \end{macrocode}

% Declare a title page.
% Print title, part of document being processed and version flag:
%    \begin{macrocode}
\addtocounter{page}{-1}
\begin{center}
{\LARGE\bfseries{}childdoc example\par}
\vspace{1cm}
\ifchilddoc
\ifchilddocmanual part\else chapter\fi:
`\childdocname' of `\childdocjob'\par
\else
main document: `\childdocjob'\par
\fi
version: \version\par
\end{center}
\newpage
%    \end{macrocode}

% Manually include selected file,
% otherwise process as usual:
%    \begin{macrocode}
\ifchilddocmanual
\section*{part `\childdocname'}
\input{\childdocname}
\else
%    \end{macrocode}

% Include the two chapters:
%    \begin{macrocode}
\include{cdocsch1}
\include{cdocsch2}
%    \end{macrocode}

% Include the two parts unless only chapters should be displayed:
%    \begin{macrocode}
\ifchilddoc\else
\section{part three}
\input{cdocspt3}
\section{part four}
\input{cdocspt4}
\fi
%    \end{macrocode}

% Process as usual until here:
%    \begin{macrocode}
\fi
%    \end{macrocode}

% End of document body:
%    \begin{macrocode}
\end{document}
%    \end{macrocode}
%\iffalse
%</samplemain>
%\fi
%
% %%%%%%%%%%%%%%%%%%%%%%%%%%%%%%%%%%%%%%
% \paragraph{Chapter Include Files.}
%
% The include files are called |cdocsch1.tex| and |cdocsch2.tex|.
%
%\iffalse
%<*samplechap1|samplechap2>
%\fi

% Optional override for |\version| flag:
%    \begin{macrocode}
%%\providecommand{\version}{final}
%    \end{macrocode}

% Include the main document:
%    \begin{macrocode}
\input{childdoc.def}
\childdocof{cdocsamp}
%    \end{macrocode}

%\iffalse
%</samplechap1|samplechap2>
%\fi
%
%\iffalse
%<*samplechap1>
%\fi
% Some text for chapter 1:
%    \begin{macrocode}
\section{one}
some text in chapter one
%    \end{macrocode}

%\iffalse
%</samplechap1>
%\fi
% Some text for chapter 2:
%\iffalse
%<*samplechap2>
%\fi
%    \begin{macrocode}
\section{two}
more text in chapter two
%    \end{macrocode}

%\iffalse
%</samplechap2>
%\fi
%
% %%%%%%%%%%%%%%%%%%%%%%%%%%%%%%%%%%%%%%
% \paragraph{Part Include Files.}
%
% The include files are called |cdocspt3.tex| and |cdocspt4.tex|.
%
%\iffalse
%<*samplepart3|samplepart4>
%\fi

% Optional override for |\version| flag:
%    \begin{macrocode}
%%\providecommand{\version}{final}
%    \end{macrocode}

% Include the main document:
%    \begin{macrocode}
\input{childdoc.def}
\childdocby{cdocsamp}
%    \end{macrocode}

%\iffalse
%</samplepart3|samplepart4>
%\fi
%
%\iffalse
%<*samplepart3>
%\fi
% Some text for part 3:
%    \begin{macrocode}
some text in part three
%    \end{macrocode}

%\iffalse
%</samplepart3>
%\fi
% Some text for part 4:
%\iffalse
%<*samplepart4>
%\fi
%    \begin{macrocode}
more text in part four
%    \end{macrocode}

%\iffalse
%</samplepart4>
%\fi
%
% %%%%%%%%%%%%%%%%%%%%%%%%%%%%%%%%%%%%%%
% \paragraph{Forwarding for a Complete Draft.}
%
% The following forwarding file |cdocsdrf.tex|
% compiles the main document in draft mode:
%\iffalse
%<*sampledraft>
%\fi
%    \begin{macrocode}
\def\version{draft}
\input{childdoc.def}
\childdocforward{cdocsamp}
%    \end{macrocode}

%\iffalse
%</sampledraft>
%\fi
%
% %%%%%%%%%%%%%%%%%%%%%%%%%%%%%%%%%%%%%%
% \paragraph{Forwarding for Final Version of the Chapters.}
%
% The following forwarding files |cdocsfn1.tex| and |cdocsfn2.tex|
% (with identical content)
% compile the final versions of the child documents
% |cdocsch1.tex| and |cdocsch2.tex|, respectively:
%\iffalse
%<*samplefinal>
%\fi
%    \begin{macrocode}
\def\version{final}
\input{childdoc.def}
\childdocforwardprefix[cdocsamp]{cdocsfn}{cdocsch}
%    \end{macrocode}

%\iffalse
%</samplefinal>
%\fi
%
% %%%%%%%%%%%%%%%%%%%%%%%%%%%%%%%%%%%%%%
% \paragraph{Command Line Processing.}
%
% The following three command lines generate the output files
% |cdocscld|, |cdocscl1| and |cdocscl2|
% which should be identical to
% |cdocsdrf|, |cdocsch1| and |cdocsfn2|, respectively:
% \begin{center}
% \begin{tabular}{l}
% |latex -jobname cdocscld \|\\
% |  "\def\version{draft}\input{childdoc.def}\childdocforward{cdocsamp}"|\\
% |latex -jobname cdocscl1 \|\\
% |  "\input{childdoc.def}\childdocforward[cdocsamp]{cdocsch1}"|\\
% |latex -jobname cdocscl2 \|\\
% |  "\def\version{final}\input{childdoc.def}\childdocforward{cdocsch2}"|
% \end{tabular}
% \end{center}
% Note that the trailing backslash on each first line
% merely continues the input to the second line
% (for convenient cut ant paste).
% Furthermore, the command |latex| can be replaced by any
% of its alternative versions such as |pdflatex|.
%
% %%%%%%%%%%%%%%%%%%%%%%%%%%%%%%%%%%%%%%%%%%%%%%%%%%%%%%%%%%%%%%%%%%%%%%%%%%%%%%
% %%%%%%%%%%%%%%%%%%%%%%%%%%%%%%%%%%%%%%%%%%%%%%%%%%%%%%%%%%%%%%%%%%%%%%%%%%%%%%
% \section{Implementation}
%\iffalse
%<*package>
%\fi
%
% This section describes the definitions file |childdoc.def|.

% The definitions cannot be loaded using |\usepackage| or |\RequirePackage|
% which has a mechanism to prevent loading a style file more than once.
% When loading the definitions by means of |\input|
% multiple instances have to be prevented manually:
%\iffalse
%This code needs to be before the `\ProvidesFile' directive
%which is defined at the beginning of this file.
%Therefore it is also placed there and commented out here.
%</package>
%<*discard>
%\fi
%    \begin{macrocode}
\ifdefined\childdocmain\endinput\fi
%    \end{macrocode}
%\iffalse
%</discard>
%<*package>
%\fi
%
% \macro{\ifchilddoc}
% \macro{\ifchilddocmanual}
% The conditional |\ifchilddoc| tells whether a
% child (true) or main (false) document is being compiled.
% The conditional |\ifchilddocmanual| tells whether
% the |\includeonly| mechanism is used (false) or
% the selection of child files must be performed manually (true).
% The definitions initialise to false:
%    \begin{macrocode}
\newif\ifchilddoc
\newif\ifchilddocmanual
%    \end{macrocode}

% \macro{\childdocname}
% \macro{\childdocjob}
% The macro |\childdocname| stores the name of the main document
% to be compiled. The macro |\childdocjob| stores the name of
% the document on which the \LaTeX{} compiler was originally invoked.
% The content of |\jobname| cannot be compared
% to filenames specified in the source due to different catcodes.
% The following code rescans |\jobname|, stores the result
% in |\childdocname| and saves a copy in |\childdocjob|:
%    \begin{macrocode}
\edef\childdocname{\scantokens\expandafter{\jobname\noexpand}}
\let\childdocjob\childdocname
%    \end{macrocode}

% \macro{\childdocdisable}
% The macro |\childdocdisable| prevents the main file
% from being processed more than once.
% At this stage, the main document command |\childdocmain|
% is assumed to be called once again where it should do nothing.
% Any subsequent call to it should prevent
% a secondary processing of the main document
% It overwrites the forwarding commands
% |\childdocof| and |\childdocforward|
% with empty macros to prevent further inclusions of the main document:
%    \begin{macrocode}
\newcommand{\childdocdisable}
{
  \renewcommand{\childdocmain}[1]{\renewcommand{\childdocmain}[1]{\endinput}}
  \renewcommand{\childdocof}[1]{}
  \renewcommand{\childdocby}[2][]{}
  \renewcommand{\childdocforward}[2][]{}
  \renewcommand{\childdocdisable}{}
}
%    \end{macrocode}

% \macro{\childdocmain}
% The macro |\childdocmain| is to be called at the top of the main file
% with nothing or the main filename (without extension) as argument.
% First, it breaks loops.
% If the argument is not empty and does not match |\childdocname|
% (which is set by the first inclusion of |childdoc.def|),
% |\ifchilddoc| is set to true, |\includeonly| is applied to the child file
% and |\jobname| is set to the main file
% (for proper handling of |.aux| files):
%    \begin{macrocode}
\newcommand{\childdocmain}[1]
{
  \childdocdisable\childdocmain{}
  \if?#1?\else
    \begingroup
      \def\childdoctmp{#1}
      \ifx\childdoctmp\childdocname
        \def\childdoctmp{}
      \else
        \def\childdoctmp
        {
          \childdoctrue
          \includeonly{\childdocname}
          \def\childdocjob{#1}
          \def\jobname{#1}
        }
      \fi
      \expandafter
    \endgroup
    \childdoctmp
  \fi
}
%    \end{macrocode}

% \macro{\childdocof}
% The command |\childdocof| redirects
% compilation to the main file |#1|.
%    \begin{macrocode}
\newcommand{\childdocof}[1]
{
  \childdocdisable
  \childdoctrue
  \includeonly{\childdocname}
  \def\jobname{#1}
  \def\childdocjob{#1}
  \input{#1}
}
%    \end{macrocode}

% \macro{\childdocby}
% The command |\childdocby| ....
%    \begin{macrocode}
\newcommand{\childdocby}[2][]
{
  \childdocdisable
  \childdoctrue
  \childdocmanualtrue
  \if?#1?\else
    \def\jobname{#2}
  \fi
  \def\childdocjob{#2}
  \input{#2}
  \endinput
}
%    \end{macrocode}

% \macro{\childdocforward}
% The command |\childdocforward| redirects
% compilation to the main file or
% (if the optional argument is given) a child file.
% Parameters are set as if the main file
% or a child file starting with |\childdocof| was compiled.
% Then compilation is handed over to the main file:
%    \begin{macrocode}
\newcommand{\childdocforward}[2][]
{
  \begingroup
    \if?#1?
      \def\childdoctmp
      {
        \def\childdocname{#2}
        \def\childdocjob{#2}
        \def\jobname{#2}
        \input{#2}
        \endinput
      }
    \else
      \def\childdoctmp
      {
        \childdocdisable
        \def\childdocname{#2}
        \childdoctrue
        \includeonly{#2}
        \def\childdocjob{#1}
        \def\jobname{#1}
        \input{#1}
        \endinput
      }
    \fi
    \expandafter
  \endgroup
  \childdoctmp
}
%    \end{macrocode}

% \macro{\childdocforwardprefix}
% The command |\childdocforwardprefix| redirects
% compilation to the main or a child file by means of a pattern.
% The prefix |#1| in the current filename is replaced by |#2|
% and the suffix of the current filename is kept
% (it is assumed that the filename does not contain the substring `|~~~|'
% which is used as a delimiter).
% Compilation is handed over to the new file by |\childdocforward|:
%    \begin{macrocode}
\newcommand{\childdocforwardprefix}[3][]
{
  \begingroup
    \def\childdocextract #2##1~~~{\def\childdoctmp{\childdocforward[#1]{#3##1}}}
    \expandafter\childdocextract\childdocname~~~
    \expandafter
  \endgroup
  \childdoctmp
}
%    \end{macrocode}

% \macro{\childdoc}
% The deprecated macro |\childdoc| is a legacy version of |\childdocmain|:
%    \begin{macrocode}
\newcommand{\childdoc}{\childdocmain}
%    \end{macrocode}

% \macro{\childdocredirect}
% The deprecated macro |\childdocredirect| is a legacy version
% of |\childdocforward| and |\childdocforwardprefix|:
%    \begin{macrocode}
\newcommand{\childdocredirect}[2][]
{
  \begingroup
    \if?#1?
      \def\childdoctmp{\childdocforward{#2}}
    \else
      \def\childdoctmp{\childdocforwardprefix{#1}{#2}}
    \fi
    \expandafter
  \endgroup
  \childdoctmp
}
%    \end{macrocode}

%\iffalse
%</package>
%\fi
%
\endinput

\childdocforwardprefix[cdocsamp]{cdocsfn}{cdocsch}
%    \end{macrocode}

%\iffalse
%</samplefinal>
%\fi
%
% %%%%%%%%%%%%%%%%%%%%%%%%%%%%%%%%%%%%%%
% \paragraph{Command Line Processing.}
%
% The following three command lines generate the output files
% |cdocscld|, |cdocscl1| and |cdocscl2|
% which should be identical to
% |cdocsdrf|, |cdocsch1| and |cdocsfn2|, respectively:
% \begin{center}
% \begin{tabular}{l}
% |latex -jobname cdocscld \|\\
% |  "\def\version{draft}% \iffalse
%
% childdoc.dtx Copyright (C) 2017-2018 Niklas Beisert
%
% This work may be distributed and/or modified under the
% conditions of the LaTeX Project Public License, either version 1.3
% of this license or (at your option) any later version.
% The latest version of this license is in
%   http://www.latex-project.org/lppl.txt
% and version 1.3 or later is part of all distributions of LaTeX
% version 2005/12/01 or later.
%
% This work has the LPPL maintenance status `maintained'.
%
% The Current Maintainer of this work is Niklas Beisert.
%
% This work consists of the files childdoc.dtx and childdoc.ins
% and the derived files childdoc.def and cdocsamp.tex with
% cdocsch1.tex, cdocsch2.tex, cdocsdrf.tex, cdocsfn1.tex, cdocsfn2.tex.
%
%<package>\ifdefined\childdocmain\endinput\fi
%<package>\ProvidesFile{childdoc.def}[2018/12/30 v2.0 child document driver]
%<samplemain>\ProvidesFile{cdocsamp.tex}[2018/12/30 v2.0 sample for childdoc]
%<*driver>
%\ProvidesFile{childdoc.drv}[2018/12/30 v2.0 childdoc reference manual file]
\PassOptionsToClass{10pt,a4paper}{article}
\documentclass{ltxdoc}

\usepackage[margin=35mm]{geometry}
\usepackage{hyperref}
\usepackage{hyperxmp}
\usepackage[usenames]{color}

\hypersetup{colorlinks=true}
\hypersetup{pdfstartview=FitH}
\hypersetup{pdfpagemode=UseNone}
\hypersetup{pdfsource={}}
\hypersetup{pdflang={en-UK}}
\hypersetup{pdfcopyright={Copyright 2017-2018 Niklas Beisert.
  This work may be distributed and/or modified under the
  conditions of the LaTeX Project Public License, either version 1.3
  of this license or (at your option) any later version.}}
\hypersetup{pdflicenseurl={http://www.latex-project.org/lppl.txt}}
\hypersetup{pdfcontactaddress={ETH Zurich, ITP, HIT K,
  Wolfgang-Pauli-Strasse 27}}
\hypersetup{pdfcontactpostcode={8093}}
\hypersetup{pdfcontactcity={Zurich}}
\hypersetup{pdfcontactcountry={Switzerland}}
\hypersetup{pdfcontactemail={nbeisert@itp.phys.ethz.ch}}
\hypersetup{pdfcontacturl={http://people.phys.ethz.ch/\xmptilde nbeisert/}}

\newcommand{\secref}[1]{\hyperref[#1]{section \ref*{#1}}}

\parskip1ex
\parindent0pt
\let\olditemize\itemize
\def\itemize{\olditemize\parskip0pt}

\begin{document}

\title{The \textsf{childdoc} Package}
\hypersetup{pdftitle={The childdoc Package}}
\author{Niklas Beisert\\[2ex]
  Institut f\"ur Theoretische Physik\\
  Eidgen\"ossische Technische Hochschule Z\"urich\\
  Wolfgang-Pauli-Strasse 27, 8093 Z\"urich, Switzerland\\[1ex]
  \href{mailto:nbeisert@itp.phys.ethz.ch}
  {\texttt{nbeisert@itp.phys.ethz.ch}}}
\hypersetup{pdfauthor={Niklas Beisert}}
\hypersetup{pdfsubject={Manual for the LaTeX2e Package childdoc}}
\date{30 December 2018, \textsf{v2.0}}
\maketitle

\begin{abstract}\noindent
\textsf{childdoc} is a \LaTeXe{} package
that enables the direct compilation
of document sections included by |\include|
to individual files.
\end{abstract}

\begingroup
\parskip0ex
\tableofcontents
\endgroup

%%%%%%%%%%%%%%%%%%%%%%%%%%%%%%%%%%%%%%%%%%%%%%%%%%%%%%%%%%%%%%%%%%%%%%%%%%%%%%%%
%%%%%%%%%%%%%%%%%%%%%%%%%%%%%%%%%%%%%%%%%%%%%%%%%%%%%%%%%%%%%%%%%%%%%%%%%%%%%%%%
\section{Introduction}

\LaTeX{} provides a mechanism to structure a large document (such as a book)
into a main file and several child files (containing the chapters)
using the |\include| command.
This mechanism is beneficial for documents
which span hundreds of pages in order to
make the source file(s) more manageable.
Moreover, compilation can be restricted to
selected child files by means of the |\includeonly| command.
The latter feature can be used to reduce the compilation time while editing
(this was significantly more useful in the earlier days of \LaTeX{})
or to generate a smaller document which is easier to navigate.
Another application of |\includeonly| is to generate
documents consisting of selected parts of the complete document.

However, there are a few drawbacks of the plain |\include| mechanism:
\begin{itemize}
\item
The child files cannot be compiled on their own,
they can only be compiled via the main file.
A naive editing environment
(such as a text editor with an option
to have the current file processed by \LaTeX)
may require one to switch to the main file before compiling;
attempting to compile the child file produces errors.
\item
The main file must be modified (each time)
to adjust the |\includeonly| command
to the present needs. This easily leaves the main file in a messy state.
\item
The generated document will always carry the filename
of the main document. This is inconvenient if
several child files are to be compiled and
to be kept for distribution.
\end{itemize}

The present package provides a simple interface
to make child files individually compilable by \LaTeX{}.
Compiling a child file then has the same effect as compiling
the main file with an |\includeonly| command
to select the appropriate child.
Moreover the generated document will carry the name of the child
rather than the main file.
This resolves all three above issues.

This feature is meant to make the editing of books,
thesis documents and lecture notes somewhat more convenient.
However, the package can also be used efficiently for
composing a series of documents (such as exercise sheets)
which are typically distributed individually.
It then assists the author in generating the individual documents
(potentially in different versions)
as well as a document containing the collected series.
Another application is in developing style files
or other kinds of included material
where compilation of the style file could redirect
to a sample or test file.

%%%%%%%%%%%%%%%%%%%%%%%%%%%%%%%%%%%%%%%%%%%%%%%%%%%%%%%%%%%%%%%%%%%%%%%%%%%%%%%%
%%%%%%%%%%%%%%%%%%%%%%%%%%%%%%%%%%%%%%%%%%%%%%%%%%%%%%%%%%%%%%%%%%%%%%%%%%%%%%%%
\section{Usage}

First of all, the package \textsf{childdoc} is \emph{not} a standard
\LaTeXe{} |.sty| style file! Therefore it needs to be invoked in
a non-standard way.

%%%%%%%%%%%%%%%%%%%%%%%%%%%%%%%%%%%%%%%%%%%%%%%%%%%%%%%%%%%%%%%%%%%%%%%%%%%%%%%%
\subsection{Included Files}
\label{sec:include}

%%%%%%%%%%%%%%%%%%%%%%%%%%%%%%%%%%%%%%%%
\DescribeMacro{\childdocmain}
To use the package, add the commands
\begin{center}
\begin{tabular}{l}
|\input{childdoc.def}|\\
|\childdocmain{}|\\
\end{tabular}
\end{center}
at the very top of the main \LaTeX{} file,
in particular \emph{before} the |\documentclass| statement!
The argument of |\childdocmain| should be left empty
(but it must be present).

%%%%%%%%%%%%%%%%%%%%%%%%%%%%%%%%%%%%%%%%
\DescribeMacro{\childdocof}
Furthermore, add the commands
\begin{center}
\begin{tabular}{l}
|\input{childdoc.def}|\\
|\childdocof{|\textit{main}|}|\\
\end{tabular}
\end{center}
at the top of every child file \textit{child}
which is included by |\include{|\textit{child}|}|
from within the main file
(or at least for those files to be compiled individually).
The argument \textit{main} must be the filename of the main file.

There are a couple of
considerations in setting up the main and child documents:

%%%%%%%%%%%%%%%%%%%%%%%%%%%%%%%%%%%%%%%%
\paragraph{Restrictions.}

Please note the following restrictions:
\begin{itemize}
\item
|\childdocmain| must be called with one argument \textit{main}
to ensure compatibility with earlier version of the package.
It must either be empty (|\childdocmain{}|)
or precisely match the filename of the main file in which it is specified.
See \secref{sec:detection} for further information.
\item
The filename \textit{main} must be specified without the |.tex| extension.
\item
The filename \textit{main} is case sensitive
(even in case-insensitive file systems)
due to internal string comparison.
\item
The argument \textit{main} should be fully expanded, it cannot be a macro.
\item
Subdirectories and special characters should be avoided in filenames.
\item
The command |\childdocmain{|\textit{main}|}| must be followed by a whitespace.
It should not be followed immediately by another command
or by a comment mark `|%|'.
This is because the \TeX{} parser reads the token immediately following
the argument of |\childdocmain| and puts it
at the beginning of every child section;
however, a white\-space is ignored.
\end{itemize}

%%%%%%%%%%%%%%%%%%%%%%%%%%%%%%%%%%%%%%%%
\paragraph{Content of Main File.}

It is advisable to place all content in the child files included by |\include|.
Any output contained in the main file will appear in all child documents
unless suppressed manually;
it cannot be suppressed automatically by the |\includeonly| directive
and thus should normally be avoided.
A method to include some content in the main file
by means of conditional processing is described in \secref{sec:conditional}.

%%%%%%%%%%%%%%%%%%%%%%%%%%%%%%%%%%%%%%%%
\paragraph{Page Numbering.}

When only a part of the document is compiled,
the appropriate numbering of pages
(as well as other status parameters)
is determined from the |.aux| files.
The latter contain information from previous passes.
However this information needs to propagate through
all intermediate child documents.
Therefore the page numbering in child documents may well
be inconsistent until the complete document is compiled at least once.

A useful (if unconventional) way to always ensure a consistent
page numbering is to restart the numbering in each child document
and denote the pages by `\textit{child}|.|\textit{page}'
where \textit{child} represents the chapter/section number of the child file.
This can be achieved by the command
|\numberwithin{page}{|\textit{child}|}|
of the \textsf{amsmath} package
where \textit{child} can be |chapter| or |section|
depending on the chosen structuring.
Alternatively, one can modify the macro |\thepage| appropriately
and reset the counter |page| at the start of each child file.

%%%%%%%%%%%%%%%%%%%%%%%%%%%%%%%%%%%%%%%%%%%%%%%%%%%%%%%%%%%%%%%%%%%%%%%%%%%%%%%%
\subsection{Conditional Processing}
\label{sec:conditional}

The package provides a mechanism to compile different versions
of a document. To customise the versions further some conditional processing
can come in handy to distinguish which version is being compiled.
The package provides two macros to describe the compilation context:

%%%%%%%%%%%%%%%%%%%%%%%%%%%%%%%%%%%%%%%%
\DescribeMacro{\ifchilddoc}
The conditional |\ifchilddoc| distinguishes between the compilation of
child documents and the main document:
%
\begin{center}
|\ifchilddoc |\textit{child-code}| |[|\||else |\textit{main-code}]| \||fi|
\end{center}

%%%%%%%%%%%%%%%%%%%%%%%%%%%%%%%%%%%%%%%%
\DescribeMacro{\childdocname}
\DescribeMacro{\childdocjob}
The macro |\childdocname| contains the filename (without extension)
of the main or child file being processed.
Note that |\childdocjob| will always contain the name of the main file.

%%%%%%%%%%%%%%%%%%%%%%%%%%%%%%%%%%%%%%%%
\paragraph{Title Page.}

Conditional processing can be used to include a title or banner page
in the main document when proper precautions are taken.
Importantly, the code in the main file should ensure that the page counter
(as well as other status parameters which are stored in the |.aux| files)
takes the same value after the conditional processing.
Otherwise the page numbers may take divergent values
depending on which part is compiled.

For example, a title page could be declared by:
%
\begin{center}
\begin{tabular}{l}
|\ifchilddoc\||else|\\
|\addtocounter{page}{-1}|\\
\textit{code for title page}\\
|\newpage|\\
|\||fi|
\end{tabular}
\end{center}
%
A banner page for the child documents can be generated by:
%
\begin{center}
\begin{tabular}{l}
|\ifchilddoc|\\
|\addtocounter{page}{-1}|\\
\textit{code for banner page}\\
|\newpage|\\
|\||fi|
\end{tabular}
\end{center}
%
Here one could write a message such as:
\begin{center}
|This is the part \childdocname{} of \childdocjob{}.|
\end{center}

%%%%%%%%%%%%%%%%%%%%%%%%%%%%%%%%%%%%%%%%%%%%%%%%%%%%%%%%%%%%%%%%%%%%%%%%%%%%%%%%
\subsection{Flags}
\label{sec:flags}

The package makes it easy to generate different versions
of the main or child documents.
To this end compilation flags can be defined
and assigned different default values.
They will be particularly useful in conjunction
with the forwarding mechanism described in \secref{sec:forward}.

For example, it may be useful to have a flag |\version|
which can be set to |draft| or |final|.
The document source will contain some conditional code
depending on the value of |\version|.
Suppose further, the flag should default to |final| for the main file
and to |draft| for child files
which is a natural assignment for editing the document.
This is achieved by placing the following code
in the preamble of the main document
(below the |\childdocmain| directive):
%
\begin{center}
\begin{tabular}{l}
|\ifchilddoc|\\
|\providecommand{\version}{draft}|\\
|\||else|\\
|\providecommand{\version}{final}|\\
|\||fi|
\end{tabular}
\end{center}
%
The definition by |\providecommand| makes sure
that previous definitions are not overwritten.
Further statements |\providecommand{\version}{...}|
can thus be added before the above code to override it.

For the main file, one might add a line
(between |\childdocmain| and the above block)
%
\begin{center}
|%\ifchilddoc\||else\providecommand{\version}{draft}\||fi|
\end{center}
%
which can be uncommented to produce a draft version.
Likewise one can add a line to the very top of a child file
(above the |\childdocof{|\textit{main}|}| directive)
%
\begin{center}
|%\providecommand{\version}{final}|
\end{center}
%
which can be uncommented to produce the final version of this child document.

%%%%%%%%%%%%%%%%%%%%%%%%%%%%%%%%%%%%%%%%%%%%%%%%%%%%%%%%%%%%%%%%%%%%%%%%%%%%%%%%
\subsection{Forwarding}
\label{sec:forward}

Different versions of the main or child documents
using compilation flags as described in \secref{sec:flags}
can be (permanently) stored in different files
for convenient compilation, viewing and distribution.
To this end, the package defines a command
to pass on compilation to a different file:

%%%%%%%%%%%%%%%%%%%%%%%%%%%%%%%%%%%%%%%%
\DescribeMacro{\childdocforward}
The command |\childdocforward| redirects processing to
another source file:
%
\begin{center}
\begin{tabular}{l}
|\input{childdoc.def}|\\
|\childdocforward[|\textit{main}|]{|\textit{dest}|}|\\
\end{tabular}
\end{center}
%
The argument \textit{dest} is the destination file
(without extension).
It should be the main file or one of the child files.
Note that further \textsf{childdoc} directives
such as |\childdocof| and |\childdocforward|
in the indicated file will be processed in this form.
The optional argument \textit{main}
passes on directly to the main file \textit{main}
while pretending to compile the child \textit{dest}.
This form behaves as if \textit{dest}
issues |\childdocof{|\textit{main}|}| right away,
and no further \textsf{childdoc} directives will be processed.

%%%%%%%%%%%%%%%%%%%%%%%%%%%%%%%%%%%%%%%%
\DescribeMacro{\...prefix}
In the alternative form |\childdocforwardprefix|,
%
\begin{center}
\begin{tabular}{l}
|\input{childdoc.def}|\\
|\childdocforwardprefix[|\textit{main}|]{|\textit{prefix}|}{|\textit{dest}|}|
\end{tabular}
\end{center}
%
the destination file is determined by a pattern
depending on the current file:
To make this work, the current file must be called
`{\textit{prefix}\hspace{0.2em}\textit{suffix}}'
with \textit{prefix} matching precisely the argument.
Processing is then passed on to the file
`{\textit{dest}\hspace{0.2em}\textit{suffix}}'.
Surely, the same effect is achieved by
directly specifying the
argument `{\textit{dest}\hspace{0.2em}\textit{suffix}}'
in the first form.
However, that requires to set up a different file
for each child. With the alternative form of the command
all these files can have exactly the same content
which simplifies setting them up and maintaining them.

For example, the following file |draft.tex|
with a compilation flag |\version| as described in \secref{sec:flags}
compiles the main document as a draft:
%
\begin{center}
\begin{tabular}{l}
|\def\version{draft}|\\
|\input{childdoc.def}|\\
|\childdocforward{|\textit{main}|}|
\end{tabular}
\end{center}
%
Likewise, the following files |final|\textit{nn}|.tex|
compile the final version of the child document
|child|\textit{nn}|.tex|:
%
\begin{center}
\begin{tabular}{l}
|\def\version{final}|\\
|\input{childdoc.def}|\\
|\childdocforwardprefix{final}{child}|
\end{tabular}
\end{center}
%

Note that when several versions of a main file and/or of each child file
are to be generated, it may be convenient to set up a |Makefile| or
shell script to automatise the process.

%%%%%%%%%%%%%%%%%%%%%%%%%%%%%%%%%%%%%%%%%%%%%%%%%%%%%%%%%%%%%%%%%%%%%%%%%%%%%%%%
\subsection{Command Line Processing}
\label{sec:commandline}

The effect of redirection files can also be achieved by invoking
the \LaTeX{} compiler with a more elaborate command line.
Most conveniently this should be done as part
of a shell script or a |Makefile|.

When using \textsf{childdoc} in the main file, the following
command lines effectively perform a redirection
(note that depending on the shell being used,
backslashes may have to be doubled: `|\|' $\to$ `|\\|'):
%
\begin{center}
|... -jobname "|\textit{target}|" |\\|"|[\textit{flags}]%
|\input{childdoc.def}\childdocforward[|\textit{main}|]{|\textit{dest}|}"|
\end{center}
%
Here \textit{target} is the name of the output file,
\textit{main} is the name of the main file
and \textit{dest} is the name of the main or child file to be processed
(all filenames without extensions).
The optional argument \textit{main} can be omitted
if \textit{main} matches \textit{dest}.
Optionally, compilation \textit{flags} can be defined via |\def| commands.
This command line makes the \TeX{} engine believe
it is compiling the file \textit{target}
whose content is specified as the latter parameter.
The provided code then forwards the processing to
\textit{main} or \textit{dest} as described in \secref{sec:forward}.

%%%%%%%%%%%%%%%%%%%%%%%%%%%%%%%%%%%%%%%%%%%%%%%%%%%%%%%%%%%%%%%%%%%%%%%%%%%%%%%%
\subsection{Include by Input}
\label{sec:input}

Including child documents by |\include| has some restrictions by design.
Most notably, the content of a child document always occupies
its own set of pages; pages cannot be shared between child documents.
Usually, this behaviour makes perfect sense
because each child document contain an essential part of the document.
However, in some situations it may be desirable to compose
a document from a collection of parts
without having mandatory page breaks between then.
For this case, the package
provides a mechanism to include parts
by |\input| which can also be processed individually.
However, by construction this mechanism
requires manual handling of the content to be output.

%%%%%%%%%%%%%%%%%%%%%%%%%%%%%%%%%%%%%%%%
\DescribeMacro{\ifchilddocmanual}
The main file should be prepared as usual, see \secref{sec:include}.
However, the document body must make a distinction
between processing of an individual part and of the main document, e.g.:
%
\begin{center}
\begin{tabular}{l}
|\ifchilddocmanual|\\
|\input{\childdocname}|\\
|\||else|\\
\textit{document body with }|\input{|\textit{part}|}|\\
|\||fi|
\end{tabular}
\end{center}
%
The conditional |\ifchilddocmanual| is true whenever
a part to be included by |\input| is being compiled,
and the name of the part is stored in |\childdocname|.

%%%%%%%%%%%%%%%%%%%%%%%%%%%%%%%%%%%%%%%%
\DescribeMacro{\childdocby}
Each part to be included by |\input| should start with:
%
\begin{center}
\begin{tabular}{l}
|\input{childdoc.def}|\\
|\childdocby{|\textit{main}|}|\\
\end{tabular}
\end{center}
%
The directive |\childdocby| is similar to |\childdocof|
described in \secref{sec:include},
but the subsequent selection of content must be done manually.
To that end, both |\ifchilddoc| and |\ifchilddocmanual|
will be true upon processing of a part,
and the name of the part is stored in |\childdocname|.
Note that |\jobname| will be set to the filename of the current part
so that each part receives an individual |.aux| file
that does not interfere with the |.aux| file(s) of the main document.
This behaviour can be altered by the alternative form
|\childdocby[*]{|\textit{main}|}| (with a non-empty optional argument)
which uses the |.aux| file of the main document
by setting |\jobname| to \textit{main}.

%%%%%%%%%%%%%%%%%%%%%%%%%%%%%%%%%%%%%%%%%%%%%%%%%%%%%%%%%%%%%%%%%%%%%%%%%%%%%%%%
\subsection{Driver Development}
\label{sec:driver}

The \textsf{childdoc} mechanism can also be use for the development
of definition files such as \LaTeX{} styles or classes.
This case differs from the above setup with multiple parts
included by |\include| in that no |\includeonly| should be invoked.
This can be achieved by starting the include file
(before |\ProvidesPackage|) with:
%
\begin{center}
\begin{tabular}{l}
|\input{childdoc.def}|\\
|\childdocforward{|\textit{main}|}|\\
\end{tabular}
\end{center}
%
or alternatively with:
%
\begin{center}
\begin{tabular}{l}
|\input{childdoc.def}|\\
|\childdocby{|\textit{main}|}|\\
\end{tabular}
\end{center}
%
Both forms have slightly different effects as described above.
The main file is prepared as usual, see \secref{sec:include}.

%%%%%%%%%%%%%%%%%%%%%%%%%%%%%%%%%%%%%%%%%%%%%%%%%%%%%%%%%%%%%%%%%%%%%%%%%%%%%%%%
\subsection{Legacy Detection}
\label{sec:detection}

The directive |\childdocmain| in the main file can detect
whether the complete document or merely a child is to be compiled
even without using the directive |\childdocof|.
This method is deprecated because it is less robust
and there is no compelling reason to use it;
it is merely provided for backward compatibility
and it may be removed in future versions.

If the detection mechanism is to be used,
it is mandatory to correctly specify
the filename of the main file as the argument of |\childdocmain|:
%
\begin{center}
\begin{tabular}{l}
|\input{childdoc.def}|\\
|\childdocmain{|\textit{main}|}|\\
\end{tabular}
\end{center}
%
If |\jobname| does not match the argument \textit{main} of |\childdocmain|,
it is assumed that |\jobname| points to the child file to be compiled.
When using |\childdocmain| with the main file specified as argument,
it suffices to start a child file
with just |\input{|\textit{main}|}|
without loading of the package and using |\childdocof|.
If instead all processing is done
with the appropriate \textsf{childdoc} directives,
the argument of \textit{main} of |\childdocmain| can be empty.

An alternative version of the command line processing described
in \secref{sec:commandline} using the detection mechanism reads:
%
\begin{center}
|... -jobname "|\textit{target}|" "|[\textit{flags}]%
[|\def\jobname{|\textit{dest}|}|]|\input{|\textit{main}|}"|
\end{center}

%%%%%%%%%%%%%%%%%%%%%%%%%%%%%%%%%%%%%%%%%%%%%%%%%%%%%%%%%%%%%%%%%%%%%%%%%%%%%%%%
\subsection{Manual Code}
\label{sec:manual}

In case one cannot be certain whether the definitions file |childdoc.def|
is installed on the target \TeX{} distribution
and one prefers not to ship it,
it is conceivable to paste a few relevant commands into the sources.

To that end, drop all statements |\input{childdoc.def}|
and perform the replacements as outlined below.
Instead of |\childdocmain{|\textit{main}|}| add the following code
to the top of the main file:
%
\begin{center}
\begin{tabular}{l}
|\||ifdefined\childdocname\endinput\||fi\newif\ifchilddoc|\\
|\edef\childdocname{\scantokens\expandafter{\jobname\noexpand}}|\\
|\def\childdocmain{|\textit{main}|}\||ifx\childdocmain\childdocname\||else|\\
|\childdoctrue\includeonly{\childdocname}\let\jobname\childdocmain\||fi|\\
\end{tabular}
\end{center}
%
Instead of |\childdocof{|\textit{main}|}| just include the main file
at the top of each child file:
%
\begin{center}
|\input{|\textit{main}|}|
\end{center}
%
A simple redirection |\childdocforward{|\textit{dest}|}| is achieved by:
%
\begin{center}
|\def\jobname{|\textit{dest}|}\input{\jobname}|
\end{center}
%
The redirection with prefix
|\childdocforwardprefix[|\textit{prefix}|]{|\textit{dest}|}|
is accomplished by:
%
\begin{center}
\begin{tabular}{l}
|{\edef\jobname{\scantokens\expandafter{\jobname\noexpand}}|\\
|\def\redirectjob |\textit{prefix}|#1~~~{\gdef\jobname{|\textit{dest}|#1}}|\\
|\expandafter\redirectjob\jobname~~~}\input{\jobname}|
\end{tabular}
\end{center}

In an alternative approach,
child documents can be compiled by a specific command line
without additional code or specific definitions:
%
\begin{center}
|... -jobname "|\textit{target}|" "|[\textit{flags}]%
|\includeonly{|\textit{dest}|}\input{|\textit{main}|}"|
\end{center}
%

%%%%%%%%%%%%%%%%%%%%%%%%%%%%%%%%%%%%%%%%%%%%%%%%%%%%%%%%%%%%%%%%%%%%%%%%%%%%%%%%
%%%%%%%%%%%%%%%%%%%%%%%%%%%%%%%%%%%%%%%%%%%%%%%%%%%%%%%%%%%%%%%%%%%%%%%%%%%%%%%%
\section{Information}

%%%%%%%%%%%%%%%%%%%%%%%%%%%%%%%%%%%%%%%%%%%%%%%%%%%%%%%%%%%%%%%%%%%%%%%%%%%%%%%%
\subsection{Copyright}

Copyright \copyright{} 2017--2018 Niklas Beisert

This work may be distributed and/or modified under the
conditions of the \LaTeX{} Project Public License, either version 1.3
of this license or (at your option) any later version.
The latest version of this license is in
  \url{http://www.latex-project.org/lppl.txt}
and version 1.3 or later is part of all distributions of \LaTeX{}
version 2005/12/01 or later.

This work has the LPPL maintenance status `maintained'.

The Current Maintainer of this work is Niklas Beisert.

This work consists of the files |README.txt|, |childdoc.ins| and |childdoc.dtx|
as well as the derived files |childdoc.def|, |cdocsamp.tex|
with |cdocsch1.tex|, |cdocsch2.tex|, |cdocspt3.tex|, |cdocspt4.tex|,
|cdocsdrf.tex|, |cdocsfn1.tex|, |cdocsfn2.tex|
as well as |childdoc.pdf|.

%%%%%%%%%%%%%%%%%%%%%%%%%%%%%%%%%%%%%%%%%%%%%%%%%%%%%%%%%%%%%%%%%%%%%%%%%%%%%%%%
\subsection{Files and Installation}

The package consists of the files:
%
\begin{center}
\begin{tabular}{ll}
    |README.txt|   & readme file \\
    |childdoc.ins| & installation file \\
    |childdoc.dtx| & source file \\
    |childdoc.def| & definition file \\
    |cdocsamp.tex| & sample main file \\
    |cdocsch1.tex| & sample include file \\
    |cdocsch2.tex| & sample include file \\
    |cdocspt3.tex| & sample part file \\
    |cdocspt4.tex| & sample part file \\
    |cdocsdrf.tex| & sample redirection file \\
    |cdocsfn1.tex| & sample redirection file \\
    |cdocsfn2.tex| & sample redirection file \\
    |childdoc.pdf| & manual
\end{tabular}
\end{center}
%
The distribution consists of the files
|README.txt|, |childdoc.ins| and |childdoc.dtx|.
%
\begin{itemize}
\item
Run (pdf)\LaTeX{} on |childdoc.dtx|
to compile the manual |childdoc.pdf| (this file).
\item
Run \LaTeX{} on |childdoc.ins| to create the definitions file |childdoc.def|
and the sample |cdocsamp.tex| with include files
|cdocsch1.tex|, |cdocsch2.tex|, |cdocspt3.tex|, |cdocspt4.tex|,
|cdocsdrf.tex|, |cdocsfn1.tex|, |cdocsfn2.tex|.
Then copy the file |childdoc.def| to an appropriate directory of your \LaTeX{}
distribution, e.g.\ \textit{texmf-root}|/tex/latex/childdoc|.
\end{itemize}

%%%%%%%%%%%%%%%%%%%%%%%%%%%%%%%%%%%%%%%%%%%%%%%%%%%%%%%%%%%%%%%%%%%%%%%%%%%%%%%%
\subsection{Related CTAN Packages}

There are several other packages which offer a similar functionality:
%
\begin{itemize}
\item
The packages
\href{http://ctan.org/pkg/docmute}{\textsf{docmute}},
\href{http://ctan.org/pkg/includex}{\textsf{includex}} and
\href{http://ctan.org/pkg/standalone}{\textsf{standalone}}
provide commands to include only the document body of
a child file thus allowing both files to be compiled individually.
\item
The packages \href{http://ctan.org/pkg/subdocs}{\textsf{subdocs}}
and \href{http://ctan.org/pkg/subfiles}{\textsf{subfiles}}
provide structures in which the main and child documents can be
encapsulated and allowing them to be compiled individually.
The inclusion mechanism is different from the conventional |\include|.
\item
The package \href{http://ctan.org/pkg/combine}{\textsf{combine}}
is an elaborate solution to combine several documents into one.
\end{itemize}
%
See also the CTAN topic \href{http://ctan.org/topic/subdocs}{\textsf{subdocs}}
for further related packages.
The present package differs from the above solutions in that
a document structure constructed with the conventional |\include| mechanism
just needs two extra commands at the top of every file
such that all constituent files can be compiled individually.

%%%%%%%%%%%%%%%%%%%%%%%%%%%%%%%%%%%%%%%%%%%%%%%%%%%%%%%%%%%%%%%%%%%%%%%%%%%%%%%%
%\subsection{Feature Suggestions}
%
%The following is a list of features which may be useful for future
%versions of this package:
%%
%\begin{itemize}
%\item
%\ldots
%\end{itemize}

%%%%%%%%%%%%%%%%%%%%%%%%%%%%%%%%%%%%%%%%%%%%%%%%%%%%%%%%%%%%%%%%%%%%%%%%%%%%%%%%
\subsection{Revision History}

%%%%%%%%%%%%%%%%%%%%%%%%%%%%%%%%%%%%%%%%
\paragraph{v2.0:} 2018/12/30

\begin{itemize}
\item
immediate forward processing
\item
added |\childdocby| mechanism
\item
manual restructured
\end{itemize}

%%%%%%%%%%%%%%%%%%%%%%%%%%%%%%%%%%%%%%%%
\paragraph{v1.6:} 2018/01/17

\begin{itemize}
\item
application for development of include files
\item
corrections to manual
\end{itemize}

%%%%%%%%%%%%%%%%%%%%%%%%%%%%%%%%%%%%%%%%
\paragraph{v1.5:} 2017/05/21

\begin{itemize}
\item
more complete structuring introduced
\item
|\childdocof| introduced
\item
|\childdoc| renamed to |\childdocmain|
\item
|\childredirect| renamed to |\childdocforward| and |\childdocforwardprefix|
and functionality expanded
\end{itemize}

%%%%%%%%%%%%%%%%%%%%%%%%%%%%%%%%%%%%%%%%
\paragraph{v1.0:} 2017/04/27

\begin{itemize}
\item
manual and install package
\item
first version published on CTAN
\end{itemize}

%%%%%%%%%%%%%%%%%%%%%%%%%%%%%%%%%%%%%%%%
\paragraph{v0.6:} 2017/04/26

\begin{itemize}
\item
redirection mechanism added
\end{itemize}

%%%%%%%%%%%%%%%%%%%%%%%%%%%%%%%%%%%%%%%%
\paragraph{v0.5:} 2017/04/26

\begin{itemize}
\item
functionality in definition file
\end{itemize}


%%%%%%%%%%%%%%%%%%%%%%%%%%%%%%%%%%%%%%%%%%%%%%%%%%%%%%%%%%%%%%%%%%%%%%%%%%%%%%%%
%%%%%%%%%%%%%%%%%%%%%%%%%%%%%%%%%%%%%%%%%%%%%%%%%%%%%%%%%%%%%%%%%%%%%%%%%%%%%%%%
%%%%%%%%%%%%%%%%%%%%%%%%%%%%%%%%%%%%%%%%%%%%%%%%%%%%%%%%%%%%%%%%%%%%%%%%%%%%%%%%
\appendix

\settowidth\MacroIndent{\rmfamily\scriptsize 000\ }

 \DocInput{childdoc.dtx}

\end{document}
%</driver>
% \fi
%
% %%%%%%%%%%%%%%%%%%%%%%%%%%%%%%%%%%%%%%%%%%%%%%%%%%%%%%%%%%%%%%%%%%%%%%%%%%%%%%
% %%%%%%%%%%%%%%%%%%%%%%%%%%%%%%%%%%%%%%%%%%%%%%%%%%%%%%%%%%%%%%%%%%%%%%%%%%%%%%
% \section{Sample}
%\iffalse
%<*samplemain>
%\fi
%
% The following presents a sample document
% with two chapters, two parts, a title page,
% a compile flag as well as three forwarding files to set the flag.
% It consists of eight |.tex| files:
% \begin{center}
% \begin{tabular}{ll}
% |cdocsamp.tex|&main file\\
% |cdocsch1.tex|&include file for chapter 1\\
% |cdocsch2.tex|&include file for chapter 2\\
% |cdocspt3.tex|&include file for part 3\\
% |cdocspt4.tex|&include file for part 4\\
% |cdocsdrf.tex|&forwarding file for main file in draft mode\\
% |cdocsfi1.tex|&forwarding file for final version of chapter 1\\
% |cdocsfi2.tex|&forwarding file for final version of chapter 2\\
% \end{tabular}
% \end{center}
% Each of the eight files can be compiled directly by the \LaTeX{} compiler.
%
% %%%%%%%%%%%%%%%%%%%%%%%%%%%%%%%%%%%%%%
% \paragraph{Main File.}
%
% The main file is called |cdocsamp.tex|.
%
% Load the \textsf{childdoc} definitions and
% declare the filename for the main document:
%    \begin{macrocode}
\input{childdoc.def}
\childdocmain{}
%    \end{macrocode}

% Optional override for |\version| flag:
%    \begin{macrocode}
%%\ifchilddoc\else\providecommand{\version}{draft}\fi
%    \end{macrocode}

% Define the default values for the |\version| flag
% (|final| for the main file and |draft| for childs):
%    \begin{macrocode}
\ifchilddoc
\providecommand{\version}{draft}
\else
\providecommand{\version}{final}
\fi
%    \end{macrocode}

% Load the standard document class:
%    \begin{macrocode}
\documentclass[12pt]{article}
%    \end{macrocode}

% Start the document body:
%    \begin{macrocode}
\begin{document}
%    \end{macrocode}

% Declare a title page.
% Print title, part of document being processed and version flag:
%    \begin{macrocode}
\addtocounter{page}{-1}
\begin{center}
{\LARGE\bfseries{}childdoc example\par}
\vspace{1cm}
\ifchilddoc
\ifchilddocmanual part\else chapter\fi:
`\childdocname' of `\childdocjob'\par
\else
main document: `\childdocjob'\par
\fi
version: \version\par
\end{center}
\newpage
%    \end{macrocode}

% Manually include selected file,
% otherwise process as usual:
%    \begin{macrocode}
\ifchilddocmanual
\section*{part `\childdocname'}
\input{\childdocname}
\else
%    \end{macrocode}

% Include the two chapters:
%    \begin{macrocode}
\include{cdocsch1}
\include{cdocsch2}
%    \end{macrocode}

% Include the two parts unless only chapters should be displayed:
%    \begin{macrocode}
\ifchilddoc\else
\section{part three}
\input{cdocspt3}
\section{part four}
\input{cdocspt4}
\fi
%    \end{macrocode}

% Process as usual until here:
%    \begin{macrocode}
\fi
%    \end{macrocode}

% End of document body:
%    \begin{macrocode}
\end{document}
%    \end{macrocode}
%\iffalse
%</samplemain>
%\fi
%
% %%%%%%%%%%%%%%%%%%%%%%%%%%%%%%%%%%%%%%
% \paragraph{Chapter Include Files.}
%
% The include files are called |cdocsch1.tex| and |cdocsch2.tex|.
%
%\iffalse
%<*samplechap1|samplechap2>
%\fi

% Optional override for |\version| flag:
%    \begin{macrocode}
%%\providecommand{\version}{final}
%    \end{macrocode}

% Include the main document:
%    \begin{macrocode}
\input{childdoc.def}
\childdocof{cdocsamp}
%    \end{macrocode}

%\iffalse
%</samplechap1|samplechap2>
%\fi
%
%\iffalse
%<*samplechap1>
%\fi
% Some text for chapter 1:
%    \begin{macrocode}
\section{one}
some text in chapter one
%    \end{macrocode}

%\iffalse
%</samplechap1>
%\fi
% Some text for chapter 2:
%\iffalse
%<*samplechap2>
%\fi
%    \begin{macrocode}
\section{two}
more text in chapter two
%    \end{macrocode}

%\iffalse
%</samplechap2>
%\fi
%
% %%%%%%%%%%%%%%%%%%%%%%%%%%%%%%%%%%%%%%
% \paragraph{Part Include Files.}
%
% The include files are called |cdocspt3.tex| and |cdocspt4.tex|.
%
%\iffalse
%<*samplepart3|samplepart4>
%\fi

% Optional override for |\version| flag:
%    \begin{macrocode}
%%\providecommand{\version}{final}
%    \end{macrocode}

% Include the main document:
%    \begin{macrocode}
\input{childdoc.def}
\childdocby{cdocsamp}
%    \end{macrocode}

%\iffalse
%</samplepart3|samplepart4>
%\fi
%
%\iffalse
%<*samplepart3>
%\fi
% Some text for part 3:
%    \begin{macrocode}
some text in part three
%    \end{macrocode}

%\iffalse
%</samplepart3>
%\fi
% Some text for part 4:
%\iffalse
%<*samplepart4>
%\fi
%    \begin{macrocode}
more text in part four
%    \end{macrocode}

%\iffalse
%</samplepart4>
%\fi
%
% %%%%%%%%%%%%%%%%%%%%%%%%%%%%%%%%%%%%%%
% \paragraph{Forwarding for a Complete Draft.}
%
% The following forwarding file |cdocsdrf.tex|
% compiles the main document in draft mode:
%\iffalse
%<*sampledraft>
%\fi
%    \begin{macrocode}
\def\version{draft}
\input{childdoc.def}
\childdocforward{cdocsamp}
%    \end{macrocode}

%\iffalse
%</sampledraft>
%\fi
%
% %%%%%%%%%%%%%%%%%%%%%%%%%%%%%%%%%%%%%%
% \paragraph{Forwarding for Final Version of the Chapters.}
%
% The following forwarding files |cdocsfn1.tex| and |cdocsfn2.tex|
% (with identical content)
% compile the final versions of the child documents
% |cdocsch1.tex| and |cdocsch2.tex|, respectively:
%\iffalse
%<*samplefinal>
%\fi
%    \begin{macrocode}
\def\version{final}
\input{childdoc.def}
\childdocforwardprefix[cdocsamp]{cdocsfn}{cdocsch}
%    \end{macrocode}

%\iffalse
%</samplefinal>
%\fi
%
% %%%%%%%%%%%%%%%%%%%%%%%%%%%%%%%%%%%%%%
% \paragraph{Command Line Processing.}
%
% The following three command lines generate the output files
% |cdocscld|, |cdocscl1| and |cdocscl2|
% which should be identical to
% |cdocsdrf|, |cdocsch1| and |cdocsfn2|, respectively:
% \begin{center}
% \begin{tabular}{l}
% |latex -jobname cdocscld \|\\
% |  "\def\version{draft}\input{childdoc.def}\childdocforward{cdocsamp}"|\\
% |latex -jobname cdocscl1 \|\\
% |  "\input{childdoc.def}\childdocforward[cdocsamp]{cdocsch1}"|\\
% |latex -jobname cdocscl2 \|\\
% |  "\def\version{final}\input{childdoc.def}\childdocforward{cdocsch2}"|
% \end{tabular}
% \end{center}
% Note that the trailing backslash on each first line
% merely continues the input to the second line
% (for convenient cut ant paste).
% Furthermore, the command |latex| can be replaced by any
% of its alternative versions such as |pdflatex|.
%
% %%%%%%%%%%%%%%%%%%%%%%%%%%%%%%%%%%%%%%%%%%%%%%%%%%%%%%%%%%%%%%%%%%%%%%%%%%%%%%
% %%%%%%%%%%%%%%%%%%%%%%%%%%%%%%%%%%%%%%%%%%%%%%%%%%%%%%%%%%%%%%%%%%%%%%%%%%%%%%
% \section{Implementation}
%\iffalse
%<*package>
%\fi
%
% This section describes the definitions file |childdoc.def|.

% The definitions cannot be loaded using |\usepackage| or |\RequirePackage|
% which has a mechanism to prevent loading a style file more than once.
% When loading the definitions by means of |\input|
% multiple instances have to be prevented manually:
%\iffalse
%This code needs to be before the `\ProvidesFile' directive
%which is defined at the beginning of this file.
%Therefore it is also placed there and commented out here.
%</package>
%<*discard>
%\fi
%    \begin{macrocode}
\ifdefined\childdocmain\endinput\fi
%    \end{macrocode}
%\iffalse
%</discard>
%<*package>
%\fi
%
% \macro{\ifchilddoc}
% \macro{\ifchilddocmanual}
% The conditional |\ifchilddoc| tells whether a
% child (true) or main (false) document is being compiled.
% The conditional |\ifchilddocmanual| tells whether
% the |\includeonly| mechanism is used (false) or
% the selection of child files must be performed manually (true).
% The definitions initialise to false:
%    \begin{macrocode}
\newif\ifchilddoc
\newif\ifchilddocmanual
%    \end{macrocode}

% \macro{\childdocname}
% \macro{\childdocjob}
% The macro |\childdocname| stores the name of the main document
% to be compiled. The macro |\childdocjob| stores the name of
% the document on which the \LaTeX{} compiler was originally invoked.
% The content of |\jobname| cannot be compared
% to filenames specified in the source due to different catcodes.
% The following code rescans |\jobname|, stores the result
% in |\childdocname| and saves a copy in |\childdocjob|:
%    \begin{macrocode}
\edef\childdocname{\scantokens\expandafter{\jobname\noexpand}}
\let\childdocjob\childdocname
%    \end{macrocode}

% \macro{\childdocdisable}
% The macro |\childdocdisable| prevents the main file
% from being processed more than once.
% At this stage, the main document command |\childdocmain|
% is assumed to be called once again where it should do nothing.
% Any subsequent call to it should prevent
% a secondary processing of the main document
% It overwrites the forwarding commands
% |\childdocof| and |\childdocforward|
% with empty macros to prevent further inclusions of the main document:
%    \begin{macrocode}
\newcommand{\childdocdisable}
{
  \renewcommand{\childdocmain}[1]{\renewcommand{\childdocmain}[1]{\endinput}}
  \renewcommand{\childdocof}[1]{}
  \renewcommand{\childdocby}[2][]{}
  \renewcommand{\childdocforward}[2][]{}
  \renewcommand{\childdocdisable}{}
}
%    \end{macrocode}

% \macro{\childdocmain}
% The macro |\childdocmain| is to be called at the top of the main file
% with nothing or the main filename (without extension) as argument.
% First, it breaks loops.
% If the argument is not empty and does not match |\childdocname|
% (which is set by the first inclusion of |childdoc.def|),
% |\ifchilddoc| is set to true, |\includeonly| is applied to the child file
% and |\jobname| is set to the main file
% (for proper handling of |.aux| files):
%    \begin{macrocode}
\newcommand{\childdocmain}[1]
{
  \childdocdisable\childdocmain{}
  \if?#1?\else
    \begingroup
      \def\childdoctmp{#1}
      \ifx\childdoctmp\childdocname
        \def\childdoctmp{}
      \else
        \def\childdoctmp
        {
          \childdoctrue
          \includeonly{\childdocname}
          \def\childdocjob{#1}
          \def\jobname{#1}
        }
      \fi
      \expandafter
    \endgroup
    \childdoctmp
  \fi
}
%    \end{macrocode}

% \macro{\childdocof}
% The command |\childdocof| redirects
% compilation to the main file |#1|.
%    \begin{macrocode}
\newcommand{\childdocof}[1]
{
  \childdocdisable
  \childdoctrue
  \includeonly{\childdocname}
  \def\jobname{#1}
  \def\childdocjob{#1}
  \input{#1}
}
%    \end{macrocode}

% \macro{\childdocby}
% The command |\childdocby| ....
%    \begin{macrocode}
\newcommand{\childdocby}[2][]
{
  \childdocdisable
  \childdoctrue
  \childdocmanualtrue
  \if?#1?\else
    \def\jobname{#2}
  \fi
  \def\childdocjob{#2}
  \input{#2}
  \endinput
}
%    \end{macrocode}

% \macro{\childdocforward}
% The command |\childdocforward| redirects
% compilation to the main file or
% (if the optional argument is given) a child file.
% Parameters are set as if the main file
% or a child file starting with |\childdocof| was compiled.
% Then compilation is handed over to the main file:
%    \begin{macrocode}
\newcommand{\childdocforward}[2][]
{
  \begingroup
    \if?#1?
      \def\childdoctmp
      {
        \def\childdocname{#2}
        \def\childdocjob{#2}
        \def\jobname{#2}
        \input{#2}
        \endinput
      }
    \else
      \def\childdoctmp
      {
        \childdocdisable
        \def\childdocname{#2}
        \childdoctrue
        \includeonly{#2}
        \def\childdocjob{#1}
        \def\jobname{#1}
        \input{#1}
        \endinput
      }
    \fi
    \expandafter
  \endgroup
  \childdoctmp
}
%    \end{macrocode}

% \macro{\childdocforwardprefix}
% The command |\childdocforwardprefix| redirects
% compilation to the main or a child file by means of a pattern.
% The prefix |#1| in the current filename is replaced by |#2|
% and the suffix of the current filename is kept
% (it is assumed that the filename does not contain the substring `|~~~|'
% which is used as a delimiter).
% Compilation is handed over to the new file by |\childdocforward|:
%    \begin{macrocode}
\newcommand{\childdocforwardprefix}[3][]
{
  \begingroup
    \def\childdocextract #2##1~~~{\def\childdoctmp{\childdocforward[#1]{#3##1}}}
    \expandafter\childdocextract\childdocname~~~
    \expandafter
  \endgroup
  \childdoctmp
}
%    \end{macrocode}

% \macro{\childdoc}
% The deprecated macro |\childdoc| is a legacy version of |\childdocmain|:
%    \begin{macrocode}
\newcommand{\childdoc}{\childdocmain}
%    \end{macrocode}

% \macro{\childdocredirect}
% The deprecated macro |\childdocredirect| is a legacy version
% of |\childdocforward| and |\childdocforwardprefix|:
%    \begin{macrocode}
\newcommand{\childdocredirect}[2][]
{
  \begingroup
    \if?#1?
      \def\childdoctmp{\childdocforward{#2}}
    \else
      \def\childdoctmp{\childdocforwardprefix{#1}{#2}}
    \fi
    \expandafter
  \endgroup
  \childdoctmp
}
%    \end{macrocode}

%\iffalse
%</package>
%\fi
%
\endinput
\childdocforward{cdocsamp}"|\\
% |latex -jobname cdocscl1 \|\\
% |  "% \iffalse
%
% childdoc.dtx Copyright (C) 2017-2018 Niklas Beisert
%
% This work may be distributed and/or modified under the
% conditions of the LaTeX Project Public License, either version 1.3
% of this license or (at your option) any later version.
% The latest version of this license is in
%   http://www.latex-project.org/lppl.txt
% and version 1.3 or later is part of all distributions of LaTeX
% version 2005/12/01 or later.
%
% This work has the LPPL maintenance status `maintained'.
%
% The Current Maintainer of this work is Niklas Beisert.
%
% This work consists of the files childdoc.dtx and childdoc.ins
% and the derived files childdoc.def and cdocsamp.tex with
% cdocsch1.tex, cdocsch2.tex, cdocsdrf.tex, cdocsfn1.tex, cdocsfn2.tex.
%
%<package>\ifdefined\childdocmain\endinput\fi
%<package>\ProvidesFile{childdoc.def}[2018/12/30 v2.0 child document driver]
%<samplemain>\ProvidesFile{cdocsamp.tex}[2018/12/30 v2.0 sample for childdoc]
%<*driver>
%\ProvidesFile{childdoc.drv}[2018/12/30 v2.0 childdoc reference manual file]
\PassOptionsToClass{10pt,a4paper}{article}
\documentclass{ltxdoc}

\usepackage[margin=35mm]{geometry}
\usepackage{hyperref}
\usepackage{hyperxmp}
\usepackage[usenames]{color}

\hypersetup{colorlinks=true}
\hypersetup{pdfstartview=FitH}
\hypersetup{pdfpagemode=UseNone}
\hypersetup{pdfsource={}}
\hypersetup{pdflang={en-UK}}
\hypersetup{pdfcopyright={Copyright 2017-2018 Niklas Beisert.
  This work may be distributed and/or modified under the
  conditions of the LaTeX Project Public License, either version 1.3
  of this license or (at your option) any later version.}}
\hypersetup{pdflicenseurl={http://www.latex-project.org/lppl.txt}}
\hypersetup{pdfcontactaddress={ETH Zurich, ITP, HIT K,
  Wolfgang-Pauli-Strasse 27}}
\hypersetup{pdfcontactpostcode={8093}}
\hypersetup{pdfcontactcity={Zurich}}
\hypersetup{pdfcontactcountry={Switzerland}}
\hypersetup{pdfcontactemail={nbeisert@itp.phys.ethz.ch}}
\hypersetup{pdfcontacturl={http://people.phys.ethz.ch/\xmptilde nbeisert/}}

\newcommand{\secref}[1]{\hyperref[#1]{section \ref*{#1}}}

\parskip1ex
\parindent0pt
\let\olditemize\itemize
\def\itemize{\olditemize\parskip0pt}

\begin{document}

\title{The \textsf{childdoc} Package}
\hypersetup{pdftitle={The childdoc Package}}
\author{Niklas Beisert\\[2ex]
  Institut f\"ur Theoretische Physik\\
  Eidgen\"ossische Technische Hochschule Z\"urich\\
  Wolfgang-Pauli-Strasse 27, 8093 Z\"urich, Switzerland\\[1ex]
  \href{mailto:nbeisert@itp.phys.ethz.ch}
  {\texttt{nbeisert@itp.phys.ethz.ch}}}
\hypersetup{pdfauthor={Niklas Beisert}}
\hypersetup{pdfsubject={Manual for the LaTeX2e Package childdoc}}
\date{30 December 2018, \textsf{v2.0}}
\maketitle

\begin{abstract}\noindent
\textsf{childdoc} is a \LaTeXe{} package
that enables the direct compilation
of document sections included by |\include|
to individual files.
\end{abstract}

\begingroup
\parskip0ex
\tableofcontents
\endgroup

%%%%%%%%%%%%%%%%%%%%%%%%%%%%%%%%%%%%%%%%%%%%%%%%%%%%%%%%%%%%%%%%%%%%%%%%%%%%%%%%
%%%%%%%%%%%%%%%%%%%%%%%%%%%%%%%%%%%%%%%%%%%%%%%%%%%%%%%%%%%%%%%%%%%%%%%%%%%%%%%%
\section{Introduction}

\LaTeX{} provides a mechanism to structure a large document (such as a book)
into a main file and several child files (containing the chapters)
using the |\include| command.
This mechanism is beneficial for documents
which span hundreds of pages in order to
make the source file(s) more manageable.
Moreover, compilation can be restricted to
selected child files by means of the |\includeonly| command.
The latter feature can be used to reduce the compilation time while editing
(this was significantly more useful in the earlier days of \LaTeX{})
or to generate a smaller document which is easier to navigate.
Another application of |\includeonly| is to generate
documents consisting of selected parts of the complete document.

However, there are a few drawbacks of the plain |\include| mechanism:
\begin{itemize}
\item
The child files cannot be compiled on their own,
they can only be compiled via the main file.
A naive editing environment
(such as a text editor with an option
to have the current file processed by \LaTeX)
may require one to switch to the main file before compiling;
attempting to compile the child file produces errors.
\item
The main file must be modified (each time)
to adjust the |\includeonly| command
to the present needs. This easily leaves the main file in a messy state.
\item
The generated document will always carry the filename
of the main document. This is inconvenient if
several child files are to be compiled and
to be kept for distribution.
\end{itemize}

The present package provides a simple interface
to make child files individually compilable by \LaTeX{}.
Compiling a child file then has the same effect as compiling
the main file with an |\includeonly| command
to select the appropriate child.
Moreover the generated document will carry the name of the child
rather than the main file.
This resolves all three above issues.

This feature is meant to make the editing of books,
thesis documents and lecture notes somewhat more convenient.
However, the package can also be used efficiently for
composing a series of documents (such as exercise sheets)
which are typically distributed individually.
It then assists the author in generating the individual documents
(potentially in different versions)
as well as a document containing the collected series.
Another application is in developing style files
or other kinds of included material
where compilation of the style file could redirect
to a sample or test file.

%%%%%%%%%%%%%%%%%%%%%%%%%%%%%%%%%%%%%%%%%%%%%%%%%%%%%%%%%%%%%%%%%%%%%%%%%%%%%%%%
%%%%%%%%%%%%%%%%%%%%%%%%%%%%%%%%%%%%%%%%%%%%%%%%%%%%%%%%%%%%%%%%%%%%%%%%%%%%%%%%
\section{Usage}

First of all, the package \textsf{childdoc} is \emph{not} a standard
\LaTeXe{} |.sty| style file! Therefore it needs to be invoked in
a non-standard way.

%%%%%%%%%%%%%%%%%%%%%%%%%%%%%%%%%%%%%%%%%%%%%%%%%%%%%%%%%%%%%%%%%%%%%%%%%%%%%%%%
\subsection{Included Files}
\label{sec:include}

%%%%%%%%%%%%%%%%%%%%%%%%%%%%%%%%%%%%%%%%
\DescribeMacro{\childdocmain}
To use the package, add the commands
\begin{center}
\begin{tabular}{l}
|\input{childdoc.def}|\\
|\childdocmain{}|\\
\end{tabular}
\end{center}
at the very top of the main \LaTeX{} file,
in particular \emph{before} the |\documentclass| statement!
The argument of |\childdocmain| should be left empty
(but it must be present).

%%%%%%%%%%%%%%%%%%%%%%%%%%%%%%%%%%%%%%%%
\DescribeMacro{\childdocof}
Furthermore, add the commands
\begin{center}
\begin{tabular}{l}
|\input{childdoc.def}|\\
|\childdocof{|\textit{main}|}|\\
\end{tabular}
\end{center}
at the top of every child file \textit{child}
which is included by |\include{|\textit{child}|}|
from within the main file
(or at least for those files to be compiled individually).
The argument \textit{main} must be the filename of the main file.

There are a couple of
considerations in setting up the main and child documents:

%%%%%%%%%%%%%%%%%%%%%%%%%%%%%%%%%%%%%%%%
\paragraph{Restrictions.}

Please note the following restrictions:
\begin{itemize}
\item
|\childdocmain| must be called with one argument \textit{main}
to ensure compatibility with earlier version of the package.
It must either be empty (|\childdocmain{}|)
or precisely match the filename of the main file in which it is specified.
See \secref{sec:detection} for further information.
\item
The filename \textit{main} must be specified without the |.tex| extension.
\item
The filename \textit{main} is case sensitive
(even in case-insensitive file systems)
due to internal string comparison.
\item
The argument \textit{main} should be fully expanded, it cannot be a macro.
\item
Subdirectories and special characters should be avoided in filenames.
\item
The command |\childdocmain{|\textit{main}|}| must be followed by a whitespace.
It should not be followed immediately by another command
or by a comment mark `|%|'.
This is because the \TeX{} parser reads the token immediately following
the argument of |\childdocmain| and puts it
at the beginning of every child section;
however, a white\-space is ignored.
\end{itemize}

%%%%%%%%%%%%%%%%%%%%%%%%%%%%%%%%%%%%%%%%
\paragraph{Content of Main File.}

It is advisable to place all content in the child files included by |\include|.
Any output contained in the main file will appear in all child documents
unless suppressed manually;
it cannot be suppressed automatically by the |\includeonly| directive
and thus should normally be avoided.
A method to include some content in the main file
by means of conditional processing is described in \secref{sec:conditional}.

%%%%%%%%%%%%%%%%%%%%%%%%%%%%%%%%%%%%%%%%
\paragraph{Page Numbering.}

When only a part of the document is compiled,
the appropriate numbering of pages
(as well as other status parameters)
is determined from the |.aux| files.
The latter contain information from previous passes.
However this information needs to propagate through
all intermediate child documents.
Therefore the page numbering in child documents may well
be inconsistent until the complete document is compiled at least once.

A useful (if unconventional) way to always ensure a consistent
page numbering is to restart the numbering in each child document
and denote the pages by `\textit{child}|.|\textit{page}'
where \textit{child} represents the chapter/section number of the child file.
This can be achieved by the command
|\numberwithin{page}{|\textit{child}|}|
of the \textsf{amsmath} package
where \textit{child} can be |chapter| or |section|
depending on the chosen structuring.
Alternatively, one can modify the macro |\thepage| appropriately
and reset the counter |page| at the start of each child file.

%%%%%%%%%%%%%%%%%%%%%%%%%%%%%%%%%%%%%%%%%%%%%%%%%%%%%%%%%%%%%%%%%%%%%%%%%%%%%%%%
\subsection{Conditional Processing}
\label{sec:conditional}

The package provides a mechanism to compile different versions
of a document. To customise the versions further some conditional processing
can come in handy to distinguish which version is being compiled.
The package provides two macros to describe the compilation context:

%%%%%%%%%%%%%%%%%%%%%%%%%%%%%%%%%%%%%%%%
\DescribeMacro{\ifchilddoc}
The conditional |\ifchilddoc| distinguishes between the compilation of
child documents and the main document:
%
\begin{center}
|\ifchilddoc |\textit{child-code}| |[|\||else |\textit{main-code}]| \||fi|
\end{center}

%%%%%%%%%%%%%%%%%%%%%%%%%%%%%%%%%%%%%%%%
\DescribeMacro{\childdocname}
\DescribeMacro{\childdocjob}
The macro |\childdocname| contains the filename (without extension)
of the main or child file being processed.
Note that |\childdocjob| will always contain the name of the main file.

%%%%%%%%%%%%%%%%%%%%%%%%%%%%%%%%%%%%%%%%
\paragraph{Title Page.}

Conditional processing can be used to include a title or banner page
in the main document when proper precautions are taken.
Importantly, the code in the main file should ensure that the page counter
(as well as other status parameters which are stored in the |.aux| files)
takes the same value after the conditional processing.
Otherwise the page numbers may take divergent values
depending on which part is compiled.

For example, a title page could be declared by:
%
\begin{center}
\begin{tabular}{l}
|\ifchilddoc\||else|\\
|\addtocounter{page}{-1}|\\
\textit{code for title page}\\
|\newpage|\\
|\||fi|
\end{tabular}
\end{center}
%
A banner page for the child documents can be generated by:
%
\begin{center}
\begin{tabular}{l}
|\ifchilddoc|\\
|\addtocounter{page}{-1}|\\
\textit{code for banner page}\\
|\newpage|\\
|\||fi|
\end{tabular}
\end{center}
%
Here one could write a message such as:
\begin{center}
|This is the part \childdocname{} of \childdocjob{}.|
\end{center}

%%%%%%%%%%%%%%%%%%%%%%%%%%%%%%%%%%%%%%%%%%%%%%%%%%%%%%%%%%%%%%%%%%%%%%%%%%%%%%%%
\subsection{Flags}
\label{sec:flags}

The package makes it easy to generate different versions
of the main or child documents.
To this end compilation flags can be defined
and assigned different default values.
They will be particularly useful in conjunction
with the forwarding mechanism described in \secref{sec:forward}.

For example, it may be useful to have a flag |\version|
which can be set to |draft| or |final|.
The document source will contain some conditional code
depending on the value of |\version|.
Suppose further, the flag should default to |final| for the main file
and to |draft| for child files
which is a natural assignment for editing the document.
This is achieved by placing the following code
in the preamble of the main document
(below the |\childdocmain| directive):
%
\begin{center}
\begin{tabular}{l}
|\ifchilddoc|\\
|\providecommand{\version}{draft}|\\
|\||else|\\
|\providecommand{\version}{final}|\\
|\||fi|
\end{tabular}
\end{center}
%
The definition by |\providecommand| makes sure
that previous definitions are not overwritten.
Further statements |\providecommand{\version}{...}|
can thus be added before the above code to override it.

For the main file, one might add a line
(between |\childdocmain| and the above block)
%
\begin{center}
|%\ifchilddoc\||else\providecommand{\version}{draft}\||fi|
\end{center}
%
which can be uncommented to produce a draft version.
Likewise one can add a line to the very top of a child file
(above the |\childdocof{|\textit{main}|}| directive)
%
\begin{center}
|%\providecommand{\version}{final}|
\end{center}
%
which can be uncommented to produce the final version of this child document.

%%%%%%%%%%%%%%%%%%%%%%%%%%%%%%%%%%%%%%%%%%%%%%%%%%%%%%%%%%%%%%%%%%%%%%%%%%%%%%%%
\subsection{Forwarding}
\label{sec:forward}

Different versions of the main or child documents
using compilation flags as described in \secref{sec:flags}
can be (permanently) stored in different files
for convenient compilation, viewing and distribution.
To this end, the package defines a command
to pass on compilation to a different file:

%%%%%%%%%%%%%%%%%%%%%%%%%%%%%%%%%%%%%%%%
\DescribeMacro{\childdocforward}
The command |\childdocforward| redirects processing to
another source file:
%
\begin{center}
\begin{tabular}{l}
|\input{childdoc.def}|\\
|\childdocforward[|\textit{main}|]{|\textit{dest}|}|\\
\end{tabular}
\end{center}
%
The argument \textit{dest} is the destination file
(without extension).
It should be the main file or one of the child files.
Note that further \textsf{childdoc} directives
such as |\childdocof| and |\childdocforward|
in the indicated file will be processed in this form.
The optional argument \textit{main}
passes on directly to the main file \textit{main}
while pretending to compile the child \textit{dest}.
This form behaves as if \textit{dest}
issues |\childdocof{|\textit{main}|}| right away,
and no further \textsf{childdoc} directives will be processed.

%%%%%%%%%%%%%%%%%%%%%%%%%%%%%%%%%%%%%%%%
\DescribeMacro{\...prefix}
In the alternative form |\childdocforwardprefix|,
%
\begin{center}
\begin{tabular}{l}
|\input{childdoc.def}|\\
|\childdocforwardprefix[|\textit{main}|]{|\textit{prefix}|}{|\textit{dest}|}|
\end{tabular}
\end{center}
%
the destination file is determined by a pattern
depending on the current file:
To make this work, the current file must be called
`{\textit{prefix}\hspace{0.2em}\textit{suffix}}'
with \textit{prefix} matching precisely the argument.
Processing is then passed on to the file
`{\textit{dest}\hspace{0.2em}\textit{suffix}}'.
Surely, the same effect is achieved by
directly specifying the
argument `{\textit{dest}\hspace{0.2em}\textit{suffix}}'
in the first form.
However, that requires to set up a different file
for each child. With the alternative form of the command
all these files can have exactly the same content
which simplifies setting them up and maintaining them.

For example, the following file |draft.tex|
with a compilation flag |\version| as described in \secref{sec:flags}
compiles the main document as a draft:
%
\begin{center}
\begin{tabular}{l}
|\def\version{draft}|\\
|\input{childdoc.def}|\\
|\childdocforward{|\textit{main}|}|
\end{tabular}
\end{center}
%
Likewise, the following files |final|\textit{nn}|.tex|
compile the final version of the child document
|child|\textit{nn}|.tex|:
%
\begin{center}
\begin{tabular}{l}
|\def\version{final}|\\
|\input{childdoc.def}|\\
|\childdocforwardprefix{final}{child}|
\end{tabular}
\end{center}
%

Note that when several versions of a main file and/or of each child file
are to be generated, it may be convenient to set up a |Makefile| or
shell script to automatise the process.

%%%%%%%%%%%%%%%%%%%%%%%%%%%%%%%%%%%%%%%%%%%%%%%%%%%%%%%%%%%%%%%%%%%%%%%%%%%%%%%%
\subsection{Command Line Processing}
\label{sec:commandline}

The effect of redirection files can also be achieved by invoking
the \LaTeX{} compiler with a more elaborate command line.
Most conveniently this should be done as part
of a shell script or a |Makefile|.

When using \textsf{childdoc} in the main file, the following
command lines effectively perform a redirection
(note that depending on the shell being used,
backslashes may have to be doubled: `|\|' $\to$ `|\\|'):
%
\begin{center}
|... -jobname "|\textit{target}|" |\\|"|[\textit{flags}]%
|\input{childdoc.def}\childdocforward[|\textit{main}|]{|\textit{dest}|}"|
\end{center}
%
Here \textit{target} is the name of the output file,
\textit{main} is the name of the main file
and \textit{dest} is the name of the main or child file to be processed
(all filenames without extensions).
The optional argument \textit{main} can be omitted
if \textit{main} matches \textit{dest}.
Optionally, compilation \textit{flags} can be defined via |\def| commands.
This command line makes the \TeX{} engine believe
it is compiling the file \textit{target}
whose content is specified as the latter parameter.
The provided code then forwards the processing to
\textit{main} or \textit{dest} as described in \secref{sec:forward}.

%%%%%%%%%%%%%%%%%%%%%%%%%%%%%%%%%%%%%%%%%%%%%%%%%%%%%%%%%%%%%%%%%%%%%%%%%%%%%%%%
\subsection{Include by Input}
\label{sec:input}

Including child documents by |\include| has some restrictions by design.
Most notably, the content of a child document always occupies
its own set of pages; pages cannot be shared between child documents.
Usually, this behaviour makes perfect sense
because each child document contain an essential part of the document.
However, in some situations it may be desirable to compose
a document from a collection of parts
without having mandatory page breaks between then.
For this case, the package
provides a mechanism to include parts
by |\input| which can also be processed individually.
However, by construction this mechanism
requires manual handling of the content to be output.

%%%%%%%%%%%%%%%%%%%%%%%%%%%%%%%%%%%%%%%%
\DescribeMacro{\ifchilddocmanual}
The main file should be prepared as usual, see \secref{sec:include}.
However, the document body must make a distinction
between processing of an individual part and of the main document, e.g.:
%
\begin{center}
\begin{tabular}{l}
|\ifchilddocmanual|\\
|\input{\childdocname}|\\
|\||else|\\
\textit{document body with }|\input{|\textit{part}|}|\\
|\||fi|
\end{tabular}
\end{center}
%
The conditional |\ifchilddocmanual| is true whenever
a part to be included by |\input| is being compiled,
and the name of the part is stored in |\childdocname|.

%%%%%%%%%%%%%%%%%%%%%%%%%%%%%%%%%%%%%%%%
\DescribeMacro{\childdocby}
Each part to be included by |\input| should start with:
%
\begin{center}
\begin{tabular}{l}
|\input{childdoc.def}|\\
|\childdocby{|\textit{main}|}|\\
\end{tabular}
\end{center}
%
The directive |\childdocby| is similar to |\childdocof|
described in \secref{sec:include},
but the subsequent selection of content must be done manually.
To that end, both |\ifchilddoc| and |\ifchilddocmanual|
will be true upon processing of a part,
and the name of the part is stored in |\childdocname|.
Note that |\jobname| will be set to the filename of the current part
so that each part receives an individual |.aux| file
that does not interfere with the |.aux| file(s) of the main document.
This behaviour can be altered by the alternative form
|\childdocby[*]{|\textit{main}|}| (with a non-empty optional argument)
which uses the |.aux| file of the main document
by setting |\jobname| to \textit{main}.

%%%%%%%%%%%%%%%%%%%%%%%%%%%%%%%%%%%%%%%%%%%%%%%%%%%%%%%%%%%%%%%%%%%%%%%%%%%%%%%%
\subsection{Driver Development}
\label{sec:driver}

The \textsf{childdoc} mechanism can also be use for the development
of definition files such as \LaTeX{} styles or classes.
This case differs from the above setup with multiple parts
included by |\include| in that no |\includeonly| should be invoked.
This can be achieved by starting the include file
(before |\ProvidesPackage|) with:
%
\begin{center}
\begin{tabular}{l}
|\input{childdoc.def}|\\
|\childdocforward{|\textit{main}|}|\\
\end{tabular}
\end{center}
%
or alternatively with:
%
\begin{center}
\begin{tabular}{l}
|\input{childdoc.def}|\\
|\childdocby{|\textit{main}|}|\\
\end{tabular}
\end{center}
%
Both forms have slightly different effects as described above.
The main file is prepared as usual, see \secref{sec:include}.

%%%%%%%%%%%%%%%%%%%%%%%%%%%%%%%%%%%%%%%%%%%%%%%%%%%%%%%%%%%%%%%%%%%%%%%%%%%%%%%%
\subsection{Legacy Detection}
\label{sec:detection}

The directive |\childdocmain| in the main file can detect
whether the complete document or merely a child is to be compiled
even without using the directive |\childdocof|.
This method is deprecated because it is less robust
and there is no compelling reason to use it;
it is merely provided for backward compatibility
and it may be removed in future versions.

If the detection mechanism is to be used,
it is mandatory to correctly specify
the filename of the main file as the argument of |\childdocmain|:
%
\begin{center}
\begin{tabular}{l}
|\input{childdoc.def}|\\
|\childdocmain{|\textit{main}|}|\\
\end{tabular}
\end{center}
%
If |\jobname| does not match the argument \textit{main} of |\childdocmain|,
it is assumed that |\jobname| points to the child file to be compiled.
When using |\childdocmain| with the main file specified as argument,
it suffices to start a child file
with just |\input{|\textit{main}|}|
without loading of the package and using |\childdocof|.
If instead all processing is done
with the appropriate \textsf{childdoc} directives,
the argument of \textit{main} of |\childdocmain| can be empty.

An alternative version of the command line processing described
in \secref{sec:commandline} using the detection mechanism reads:
%
\begin{center}
|... -jobname "|\textit{target}|" "|[\textit{flags}]%
[|\def\jobname{|\textit{dest}|}|]|\input{|\textit{main}|}"|
\end{center}

%%%%%%%%%%%%%%%%%%%%%%%%%%%%%%%%%%%%%%%%%%%%%%%%%%%%%%%%%%%%%%%%%%%%%%%%%%%%%%%%
\subsection{Manual Code}
\label{sec:manual}

In case one cannot be certain whether the definitions file |childdoc.def|
is installed on the target \TeX{} distribution
and one prefers not to ship it,
it is conceivable to paste a few relevant commands into the sources.

To that end, drop all statements |\input{childdoc.def}|
and perform the replacements as outlined below.
Instead of |\childdocmain{|\textit{main}|}| add the following code
to the top of the main file:
%
\begin{center}
\begin{tabular}{l}
|\||ifdefined\childdocname\endinput\||fi\newif\ifchilddoc|\\
|\edef\childdocname{\scantokens\expandafter{\jobname\noexpand}}|\\
|\def\childdocmain{|\textit{main}|}\||ifx\childdocmain\childdocname\||else|\\
|\childdoctrue\includeonly{\childdocname}\let\jobname\childdocmain\||fi|\\
\end{tabular}
\end{center}
%
Instead of |\childdocof{|\textit{main}|}| just include the main file
at the top of each child file:
%
\begin{center}
|\input{|\textit{main}|}|
\end{center}
%
A simple redirection |\childdocforward{|\textit{dest}|}| is achieved by:
%
\begin{center}
|\def\jobname{|\textit{dest}|}\input{\jobname}|
\end{center}
%
The redirection with prefix
|\childdocforwardprefix[|\textit{prefix}|]{|\textit{dest}|}|
is accomplished by:
%
\begin{center}
\begin{tabular}{l}
|{\edef\jobname{\scantokens\expandafter{\jobname\noexpand}}|\\
|\def\redirectjob |\textit{prefix}|#1~~~{\gdef\jobname{|\textit{dest}|#1}}|\\
|\expandafter\redirectjob\jobname~~~}\input{\jobname}|
\end{tabular}
\end{center}

In an alternative approach,
child documents can be compiled by a specific command line
without additional code or specific definitions:
%
\begin{center}
|... -jobname "|\textit{target}|" "|[\textit{flags}]%
|\includeonly{|\textit{dest}|}\input{|\textit{main}|}"|
\end{center}
%

%%%%%%%%%%%%%%%%%%%%%%%%%%%%%%%%%%%%%%%%%%%%%%%%%%%%%%%%%%%%%%%%%%%%%%%%%%%%%%%%
%%%%%%%%%%%%%%%%%%%%%%%%%%%%%%%%%%%%%%%%%%%%%%%%%%%%%%%%%%%%%%%%%%%%%%%%%%%%%%%%
\section{Information}

%%%%%%%%%%%%%%%%%%%%%%%%%%%%%%%%%%%%%%%%%%%%%%%%%%%%%%%%%%%%%%%%%%%%%%%%%%%%%%%%
\subsection{Copyright}

Copyright \copyright{} 2017--2018 Niklas Beisert

This work may be distributed and/or modified under the
conditions of the \LaTeX{} Project Public License, either version 1.3
of this license or (at your option) any later version.
The latest version of this license is in
  \url{http://www.latex-project.org/lppl.txt}
and version 1.3 or later is part of all distributions of \LaTeX{}
version 2005/12/01 or later.

This work has the LPPL maintenance status `maintained'.

The Current Maintainer of this work is Niklas Beisert.

This work consists of the files |README.txt|, |childdoc.ins| and |childdoc.dtx|
as well as the derived files |childdoc.def|, |cdocsamp.tex|
with |cdocsch1.tex|, |cdocsch2.tex|, |cdocspt3.tex|, |cdocspt4.tex|,
|cdocsdrf.tex|, |cdocsfn1.tex|, |cdocsfn2.tex|
as well as |childdoc.pdf|.

%%%%%%%%%%%%%%%%%%%%%%%%%%%%%%%%%%%%%%%%%%%%%%%%%%%%%%%%%%%%%%%%%%%%%%%%%%%%%%%%
\subsection{Files and Installation}

The package consists of the files:
%
\begin{center}
\begin{tabular}{ll}
    |README.txt|   & readme file \\
    |childdoc.ins| & installation file \\
    |childdoc.dtx| & source file \\
    |childdoc.def| & definition file \\
    |cdocsamp.tex| & sample main file \\
    |cdocsch1.tex| & sample include file \\
    |cdocsch2.tex| & sample include file \\
    |cdocspt3.tex| & sample part file \\
    |cdocspt4.tex| & sample part file \\
    |cdocsdrf.tex| & sample redirection file \\
    |cdocsfn1.tex| & sample redirection file \\
    |cdocsfn2.tex| & sample redirection file \\
    |childdoc.pdf| & manual
\end{tabular}
\end{center}
%
The distribution consists of the files
|README.txt|, |childdoc.ins| and |childdoc.dtx|.
%
\begin{itemize}
\item
Run (pdf)\LaTeX{} on |childdoc.dtx|
to compile the manual |childdoc.pdf| (this file).
\item
Run \LaTeX{} on |childdoc.ins| to create the definitions file |childdoc.def|
and the sample |cdocsamp.tex| with include files
|cdocsch1.tex|, |cdocsch2.tex|, |cdocspt3.tex|, |cdocspt4.tex|,
|cdocsdrf.tex|, |cdocsfn1.tex|, |cdocsfn2.tex|.
Then copy the file |childdoc.def| to an appropriate directory of your \LaTeX{}
distribution, e.g.\ \textit{texmf-root}|/tex/latex/childdoc|.
\end{itemize}

%%%%%%%%%%%%%%%%%%%%%%%%%%%%%%%%%%%%%%%%%%%%%%%%%%%%%%%%%%%%%%%%%%%%%%%%%%%%%%%%
\subsection{Related CTAN Packages}

There are several other packages which offer a similar functionality:
%
\begin{itemize}
\item
The packages
\href{http://ctan.org/pkg/docmute}{\textsf{docmute}},
\href{http://ctan.org/pkg/includex}{\textsf{includex}} and
\href{http://ctan.org/pkg/standalone}{\textsf{standalone}}
provide commands to include only the document body of
a child file thus allowing both files to be compiled individually.
\item
The packages \href{http://ctan.org/pkg/subdocs}{\textsf{subdocs}}
and \href{http://ctan.org/pkg/subfiles}{\textsf{subfiles}}
provide structures in which the main and child documents can be
encapsulated and allowing them to be compiled individually.
The inclusion mechanism is different from the conventional |\include|.
\item
The package \href{http://ctan.org/pkg/combine}{\textsf{combine}}
is an elaborate solution to combine several documents into one.
\end{itemize}
%
See also the CTAN topic \href{http://ctan.org/topic/subdocs}{\textsf{subdocs}}
for further related packages.
The present package differs from the above solutions in that
a document structure constructed with the conventional |\include| mechanism
just needs two extra commands at the top of every file
such that all constituent files can be compiled individually.

%%%%%%%%%%%%%%%%%%%%%%%%%%%%%%%%%%%%%%%%%%%%%%%%%%%%%%%%%%%%%%%%%%%%%%%%%%%%%%%%
%\subsection{Feature Suggestions}
%
%The following is a list of features which may be useful for future
%versions of this package:
%%
%\begin{itemize}
%\item
%\ldots
%\end{itemize}

%%%%%%%%%%%%%%%%%%%%%%%%%%%%%%%%%%%%%%%%%%%%%%%%%%%%%%%%%%%%%%%%%%%%%%%%%%%%%%%%
\subsection{Revision History}

%%%%%%%%%%%%%%%%%%%%%%%%%%%%%%%%%%%%%%%%
\paragraph{v2.0:} 2018/12/30

\begin{itemize}
\item
immediate forward processing
\item
added |\childdocby| mechanism
\item
manual restructured
\end{itemize}

%%%%%%%%%%%%%%%%%%%%%%%%%%%%%%%%%%%%%%%%
\paragraph{v1.6:} 2018/01/17

\begin{itemize}
\item
application for development of include files
\item
corrections to manual
\end{itemize}

%%%%%%%%%%%%%%%%%%%%%%%%%%%%%%%%%%%%%%%%
\paragraph{v1.5:} 2017/05/21

\begin{itemize}
\item
more complete structuring introduced
\item
|\childdocof| introduced
\item
|\childdoc| renamed to |\childdocmain|
\item
|\childredirect| renamed to |\childdocforward| and |\childdocforwardprefix|
and functionality expanded
\end{itemize}

%%%%%%%%%%%%%%%%%%%%%%%%%%%%%%%%%%%%%%%%
\paragraph{v1.0:} 2017/04/27

\begin{itemize}
\item
manual and install package
\item
first version published on CTAN
\end{itemize}

%%%%%%%%%%%%%%%%%%%%%%%%%%%%%%%%%%%%%%%%
\paragraph{v0.6:} 2017/04/26

\begin{itemize}
\item
redirection mechanism added
\end{itemize}

%%%%%%%%%%%%%%%%%%%%%%%%%%%%%%%%%%%%%%%%
\paragraph{v0.5:} 2017/04/26

\begin{itemize}
\item
functionality in definition file
\end{itemize}


%%%%%%%%%%%%%%%%%%%%%%%%%%%%%%%%%%%%%%%%%%%%%%%%%%%%%%%%%%%%%%%%%%%%%%%%%%%%%%%%
%%%%%%%%%%%%%%%%%%%%%%%%%%%%%%%%%%%%%%%%%%%%%%%%%%%%%%%%%%%%%%%%%%%%%%%%%%%%%%%%
%%%%%%%%%%%%%%%%%%%%%%%%%%%%%%%%%%%%%%%%%%%%%%%%%%%%%%%%%%%%%%%%%%%%%%%%%%%%%%%%
\appendix

\settowidth\MacroIndent{\rmfamily\scriptsize 000\ }

 \DocInput{childdoc.dtx}

\end{document}
%</driver>
% \fi
%
% %%%%%%%%%%%%%%%%%%%%%%%%%%%%%%%%%%%%%%%%%%%%%%%%%%%%%%%%%%%%%%%%%%%%%%%%%%%%%%
% %%%%%%%%%%%%%%%%%%%%%%%%%%%%%%%%%%%%%%%%%%%%%%%%%%%%%%%%%%%%%%%%%%%%%%%%%%%%%%
% \section{Sample}
%\iffalse
%<*samplemain>
%\fi
%
% The following presents a sample document
% with two chapters, two parts, a title page,
% a compile flag as well as three forwarding files to set the flag.
% It consists of eight |.tex| files:
% \begin{center}
% \begin{tabular}{ll}
% |cdocsamp.tex|&main file\\
% |cdocsch1.tex|&include file for chapter 1\\
% |cdocsch2.tex|&include file for chapter 2\\
% |cdocspt3.tex|&include file for part 3\\
% |cdocspt4.tex|&include file for part 4\\
% |cdocsdrf.tex|&forwarding file for main file in draft mode\\
% |cdocsfi1.tex|&forwarding file for final version of chapter 1\\
% |cdocsfi2.tex|&forwarding file for final version of chapter 2\\
% \end{tabular}
% \end{center}
% Each of the eight files can be compiled directly by the \LaTeX{} compiler.
%
% %%%%%%%%%%%%%%%%%%%%%%%%%%%%%%%%%%%%%%
% \paragraph{Main File.}
%
% The main file is called |cdocsamp.tex|.
%
% Load the \textsf{childdoc} definitions and
% declare the filename for the main document:
%    \begin{macrocode}
\input{childdoc.def}
\childdocmain{}
%    \end{macrocode}

% Optional override for |\version| flag:
%    \begin{macrocode}
%%\ifchilddoc\else\providecommand{\version}{draft}\fi
%    \end{macrocode}

% Define the default values for the |\version| flag
% (|final| for the main file and |draft| for childs):
%    \begin{macrocode}
\ifchilddoc
\providecommand{\version}{draft}
\else
\providecommand{\version}{final}
\fi
%    \end{macrocode}

% Load the standard document class:
%    \begin{macrocode}
\documentclass[12pt]{article}
%    \end{macrocode}

% Start the document body:
%    \begin{macrocode}
\begin{document}
%    \end{macrocode}

% Declare a title page.
% Print title, part of document being processed and version flag:
%    \begin{macrocode}
\addtocounter{page}{-1}
\begin{center}
{\LARGE\bfseries{}childdoc example\par}
\vspace{1cm}
\ifchilddoc
\ifchilddocmanual part\else chapter\fi:
`\childdocname' of `\childdocjob'\par
\else
main document: `\childdocjob'\par
\fi
version: \version\par
\end{center}
\newpage
%    \end{macrocode}

% Manually include selected file,
% otherwise process as usual:
%    \begin{macrocode}
\ifchilddocmanual
\section*{part `\childdocname'}
\input{\childdocname}
\else
%    \end{macrocode}

% Include the two chapters:
%    \begin{macrocode}
\include{cdocsch1}
\include{cdocsch2}
%    \end{macrocode}

% Include the two parts unless only chapters should be displayed:
%    \begin{macrocode}
\ifchilddoc\else
\section{part three}
\input{cdocspt3}
\section{part four}
\input{cdocspt4}
\fi
%    \end{macrocode}

% Process as usual until here:
%    \begin{macrocode}
\fi
%    \end{macrocode}

% End of document body:
%    \begin{macrocode}
\end{document}
%    \end{macrocode}
%\iffalse
%</samplemain>
%\fi
%
% %%%%%%%%%%%%%%%%%%%%%%%%%%%%%%%%%%%%%%
% \paragraph{Chapter Include Files.}
%
% The include files are called |cdocsch1.tex| and |cdocsch2.tex|.
%
%\iffalse
%<*samplechap1|samplechap2>
%\fi

% Optional override for |\version| flag:
%    \begin{macrocode}
%%\providecommand{\version}{final}
%    \end{macrocode}

% Include the main document:
%    \begin{macrocode}
\input{childdoc.def}
\childdocof{cdocsamp}
%    \end{macrocode}

%\iffalse
%</samplechap1|samplechap2>
%\fi
%
%\iffalse
%<*samplechap1>
%\fi
% Some text for chapter 1:
%    \begin{macrocode}
\section{one}
some text in chapter one
%    \end{macrocode}

%\iffalse
%</samplechap1>
%\fi
% Some text for chapter 2:
%\iffalse
%<*samplechap2>
%\fi
%    \begin{macrocode}
\section{two}
more text in chapter two
%    \end{macrocode}

%\iffalse
%</samplechap2>
%\fi
%
% %%%%%%%%%%%%%%%%%%%%%%%%%%%%%%%%%%%%%%
% \paragraph{Part Include Files.}
%
% The include files are called |cdocspt3.tex| and |cdocspt4.tex|.
%
%\iffalse
%<*samplepart3|samplepart4>
%\fi

% Optional override for |\version| flag:
%    \begin{macrocode}
%%\providecommand{\version}{final}
%    \end{macrocode}

% Include the main document:
%    \begin{macrocode}
\input{childdoc.def}
\childdocby{cdocsamp}
%    \end{macrocode}

%\iffalse
%</samplepart3|samplepart4>
%\fi
%
%\iffalse
%<*samplepart3>
%\fi
% Some text for part 3:
%    \begin{macrocode}
some text in part three
%    \end{macrocode}

%\iffalse
%</samplepart3>
%\fi
% Some text for part 4:
%\iffalse
%<*samplepart4>
%\fi
%    \begin{macrocode}
more text in part four
%    \end{macrocode}

%\iffalse
%</samplepart4>
%\fi
%
% %%%%%%%%%%%%%%%%%%%%%%%%%%%%%%%%%%%%%%
% \paragraph{Forwarding for a Complete Draft.}
%
% The following forwarding file |cdocsdrf.tex|
% compiles the main document in draft mode:
%\iffalse
%<*sampledraft>
%\fi
%    \begin{macrocode}
\def\version{draft}
\input{childdoc.def}
\childdocforward{cdocsamp}
%    \end{macrocode}

%\iffalse
%</sampledraft>
%\fi
%
% %%%%%%%%%%%%%%%%%%%%%%%%%%%%%%%%%%%%%%
% \paragraph{Forwarding for Final Version of the Chapters.}
%
% The following forwarding files |cdocsfn1.tex| and |cdocsfn2.tex|
% (with identical content)
% compile the final versions of the child documents
% |cdocsch1.tex| and |cdocsch2.tex|, respectively:
%\iffalse
%<*samplefinal>
%\fi
%    \begin{macrocode}
\def\version{final}
\input{childdoc.def}
\childdocforwardprefix[cdocsamp]{cdocsfn}{cdocsch}
%    \end{macrocode}

%\iffalse
%</samplefinal>
%\fi
%
% %%%%%%%%%%%%%%%%%%%%%%%%%%%%%%%%%%%%%%
% \paragraph{Command Line Processing.}
%
% The following three command lines generate the output files
% |cdocscld|, |cdocscl1| and |cdocscl2|
% which should be identical to
% |cdocsdrf|, |cdocsch1| and |cdocsfn2|, respectively:
% \begin{center}
% \begin{tabular}{l}
% |latex -jobname cdocscld \|\\
% |  "\def\version{draft}\input{childdoc.def}\childdocforward{cdocsamp}"|\\
% |latex -jobname cdocscl1 \|\\
% |  "\input{childdoc.def}\childdocforward[cdocsamp]{cdocsch1}"|\\
% |latex -jobname cdocscl2 \|\\
% |  "\def\version{final}\input{childdoc.def}\childdocforward{cdocsch2}"|
% \end{tabular}
% \end{center}
% Note that the trailing backslash on each first line
% merely continues the input to the second line
% (for convenient cut ant paste).
% Furthermore, the command |latex| can be replaced by any
% of its alternative versions such as |pdflatex|.
%
% %%%%%%%%%%%%%%%%%%%%%%%%%%%%%%%%%%%%%%%%%%%%%%%%%%%%%%%%%%%%%%%%%%%%%%%%%%%%%%
% %%%%%%%%%%%%%%%%%%%%%%%%%%%%%%%%%%%%%%%%%%%%%%%%%%%%%%%%%%%%%%%%%%%%%%%%%%%%%%
% \section{Implementation}
%\iffalse
%<*package>
%\fi
%
% This section describes the definitions file |childdoc.def|.

% The definitions cannot be loaded using |\usepackage| or |\RequirePackage|
% which has a mechanism to prevent loading a style file more than once.
% When loading the definitions by means of |\input|
% multiple instances have to be prevented manually:
%\iffalse
%This code needs to be before the `\ProvidesFile' directive
%which is defined at the beginning of this file.
%Therefore it is also placed there and commented out here.
%</package>
%<*discard>
%\fi
%    \begin{macrocode}
\ifdefined\childdocmain\endinput\fi
%    \end{macrocode}
%\iffalse
%</discard>
%<*package>
%\fi
%
% \macro{\ifchilddoc}
% \macro{\ifchilddocmanual}
% The conditional |\ifchilddoc| tells whether a
% child (true) or main (false) document is being compiled.
% The conditional |\ifchilddocmanual| tells whether
% the |\includeonly| mechanism is used (false) or
% the selection of child files must be performed manually (true).
% The definitions initialise to false:
%    \begin{macrocode}
\newif\ifchilddoc
\newif\ifchilddocmanual
%    \end{macrocode}

% \macro{\childdocname}
% \macro{\childdocjob}
% The macro |\childdocname| stores the name of the main document
% to be compiled. The macro |\childdocjob| stores the name of
% the document on which the \LaTeX{} compiler was originally invoked.
% The content of |\jobname| cannot be compared
% to filenames specified in the source due to different catcodes.
% The following code rescans |\jobname|, stores the result
% in |\childdocname| and saves a copy in |\childdocjob|:
%    \begin{macrocode}
\edef\childdocname{\scantokens\expandafter{\jobname\noexpand}}
\let\childdocjob\childdocname
%    \end{macrocode}

% \macro{\childdocdisable}
% The macro |\childdocdisable| prevents the main file
% from being processed more than once.
% At this stage, the main document command |\childdocmain|
% is assumed to be called once again where it should do nothing.
% Any subsequent call to it should prevent
% a secondary processing of the main document
% It overwrites the forwarding commands
% |\childdocof| and |\childdocforward|
% with empty macros to prevent further inclusions of the main document:
%    \begin{macrocode}
\newcommand{\childdocdisable}
{
  \renewcommand{\childdocmain}[1]{\renewcommand{\childdocmain}[1]{\endinput}}
  \renewcommand{\childdocof}[1]{}
  \renewcommand{\childdocby}[2][]{}
  \renewcommand{\childdocforward}[2][]{}
  \renewcommand{\childdocdisable}{}
}
%    \end{macrocode}

% \macro{\childdocmain}
% The macro |\childdocmain| is to be called at the top of the main file
% with nothing or the main filename (without extension) as argument.
% First, it breaks loops.
% If the argument is not empty and does not match |\childdocname|
% (which is set by the first inclusion of |childdoc.def|),
% |\ifchilddoc| is set to true, |\includeonly| is applied to the child file
% and |\jobname| is set to the main file
% (for proper handling of |.aux| files):
%    \begin{macrocode}
\newcommand{\childdocmain}[1]
{
  \childdocdisable\childdocmain{}
  \if?#1?\else
    \begingroup
      \def\childdoctmp{#1}
      \ifx\childdoctmp\childdocname
        \def\childdoctmp{}
      \else
        \def\childdoctmp
        {
          \childdoctrue
          \includeonly{\childdocname}
          \def\childdocjob{#1}
          \def\jobname{#1}
        }
      \fi
      \expandafter
    \endgroup
    \childdoctmp
  \fi
}
%    \end{macrocode}

% \macro{\childdocof}
% The command |\childdocof| redirects
% compilation to the main file |#1|.
%    \begin{macrocode}
\newcommand{\childdocof}[1]
{
  \childdocdisable
  \childdoctrue
  \includeonly{\childdocname}
  \def\jobname{#1}
  \def\childdocjob{#1}
  \input{#1}
}
%    \end{macrocode}

% \macro{\childdocby}
% The command |\childdocby| ....
%    \begin{macrocode}
\newcommand{\childdocby}[2][]
{
  \childdocdisable
  \childdoctrue
  \childdocmanualtrue
  \if?#1?\else
    \def\jobname{#2}
  \fi
  \def\childdocjob{#2}
  \input{#2}
  \endinput
}
%    \end{macrocode}

% \macro{\childdocforward}
% The command |\childdocforward| redirects
% compilation to the main file or
% (if the optional argument is given) a child file.
% Parameters are set as if the main file
% or a child file starting with |\childdocof| was compiled.
% Then compilation is handed over to the main file:
%    \begin{macrocode}
\newcommand{\childdocforward}[2][]
{
  \begingroup
    \if?#1?
      \def\childdoctmp
      {
        \def\childdocname{#2}
        \def\childdocjob{#2}
        \def\jobname{#2}
        \input{#2}
        \endinput
      }
    \else
      \def\childdoctmp
      {
        \childdocdisable
        \def\childdocname{#2}
        \childdoctrue
        \includeonly{#2}
        \def\childdocjob{#1}
        \def\jobname{#1}
        \input{#1}
        \endinput
      }
    \fi
    \expandafter
  \endgroup
  \childdoctmp
}
%    \end{macrocode}

% \macro{\childdocforwardprefix}
% The command |\childdocforwardprefix| redirects
% compilation to the main or a child file by means of a pattern.
% The prefix |#1| in the current filename is replaced by |#2|
% and the suffix of the current filename is kept
% (it is assumed that the filename does not contain the substring `|~~~|'
% which is used as a delimiter).
% Compilation is handed over to the new file by |\childdocforward|:
%    \begin{macrocode}
\newcommand{\childdocforwardprefix}[3][]
{
  \begingroup
    \def\childdocextract #2##1~~~{\def\childdoctmp{\childdocforward[#1]{#3##1}}}
    \expandafter\childdocextract\childdocname~~~
    \expandafter
  \endgroup
  \childdoctmp
}
%    \end{macrocode}

% \macro{\childdoc}
% The deprecated macro |\childdoc| is a legacy version of |\childdocmain|:
%    \begin{macrocode}
\newcommand{\childdoc}{\childdocmain}
%    \end{macrocode}

% \macro{\childdocredirect}
% The deprecated macro |\childdocredirect| is a legacy version
% of |\childdocforward| and |\childdocforwardprefix|:
%    \begin{macrocode}
\newcommand{\childdocredirect}[2][]
{
  \begingroup
    \if?#1?
      \def\childdoctmp{\childdocforward{#2}}
    \else
      \def\childdoctmp{\childdocforwardprefix{#1}{#2}}
    \fi
    \expandafter
  \endgroup
  \childdoctmp
}
%    \end{macrocode}

%\iffalse
%</package>
%\fi
%
\endinput
\childdocforward[cdocsamp]{cdocsch1}"|\\
% |latex -jobname cdocscl2 \|\\
% |  "\def\version{final}% \iffalse
%
% childdoc.dtx Copyright (C) 2017-2018 Niklas Beisert
%
% This work may be distributed and/or modified under the
% conditions of the LaTeX Project Public License, either version 1.3
% of this license or (at your option) any later version.
% The latest version of this license is in
%   http://www.latex-project.org/lppl.txt
% and version 1.3 or later is part of all distributions of LaTeX
% version 2005/12/01 or later.
%
% This work has the LPPL maintenance status `maintained'.
%
% The Current Maintainer of this work is Niklas Beisert.
%
% This work consists of the files childdoc.dtx and childdoc.ins
% and the derived files childdoc.def and cdocsamp.tex with
% cdocsch1.tex, cdocsch2.tex, cdocsdrf.tex, cdocsfn1.tex, cdocsfn2.tex.
%
%<package>\ifdefined\childdocmain\endinput\fi
%<package>\ProvidesFile{childdoc.def}[2018/12/30 v2.0 child document driver]
%<samplemain>\ProvidesFile{cdocsamp.tex}[2018/12/30 v2.0 sample for childdoc]
%<*driver>
%\ProvidesFile{childdoc.drv}[2018/12/30 v2.0 childdoc reference manual file]
\PassOptionsToClass{10pt,a4paper}{article}
\documentclass{ltxdoc}

\usepackage[margin=35mm]{geometry}
\usepackage{hyperref}
\usepackage{hyperxmp}
\usepackage[usenames]{color}

\hypersetup{colorlinks=true}
\hypersetup{pdfstartview=FitH}
\hypersetup{pdfpagemode=UseNone}
\hypersetup{pdfsource={}}
\hypersetup{pdflang={en-UK}}
\hypersetup{pdfcopyright={Copyright 2017-2018 Niklas Beisert.
  This work may be distributed and/or modified under the
  conditions of the LaTeX Project Public License, either version 1.3
  of this license or (at your option) any later version.}}
\hypersetup{pdflicenseurl={http://www.latex-project.org/lppl.txt}}
\hypersetup{pdfcontactaddress={ETH Zurich, ITP, HIT K,
  Wolfgang-Pauli-Strasse 27}}
\hypersetup{pdfcontactpostcode={8093}}
\hypersetup{pdfcontactcity={Zurich}}
\hypersetup{pdfcontactcountry={Switzerland}}
\hypersetup{pdfcontactemail={nbeisert@itp.phys.ethz.ch}}
\hypersetup{pdfcontacturl={http://people.phys.ethz.ch/\xmptilde nbeisert/}}

\newcommand{\secref}[1]{\hyperref[#1]{section \ref*{#1}}}

\parskip1ex
\parindent0pt
\let\olditemize\itemize
\def\itemize{\olditemize\parskip0pt}

\begin{document}

\title{The \textsf{childdoc} Package}
\hypersetup{pdftitle={The childdoc Package}}
\author{Niklas Beisert\\[2ex]
  Institut f\"ur Theoretische Physik\\
  Eidgen\"ossische Technische Hochschule Z\"urich\\
  Wolfgang-Pauli-Strasse 27, 8093 Z\"urich, Switzerland\\[1ex]
  \href{mailto:nbeisert@itp.phys.ethz.ch}
  {\texttt{nbeisert@itp.phys.ethz.ch}}}
\hypersetup{pdfauthor={Niklas Beisert}}
\hypersetup{pdfsubject={Manual for the LaTeX2e Package childdoc}}
\date{30 December 2018, \textsf{v2.0}}
\maketitle

\begin{abstract}\noindent
\textsf{childdoc} is a \LaTeXe{} package
that enables the direct compilation
of document sections included by |\include|
to individual files.
\end{abstract}

\begingroup
\parskip0ex
\tableofcontents
\endgroup

%%%%%%%%%%%%%%%%%%%%%%%%%%%%%%%%%%%%%%%%%%%%%%%%%%%%%%%%%%%%%%%%%%%%%%%%%%%%%%%%
%%%%%%%%%%%%%%%%%%%%%%%%%%%%%%%%%%%%%%%%%%%%%%%%%%%%%%%%%%%%%%%%%%%%%%%%%%%%%%%%
\section{Introduction}

\LaTeX{} provides a mechanism to structure a large document (such as a book)
into a main file and several child files (containing the chapters)
using the |\include| command.
This mechanism is beneficial for documents
which span hundreds of pages in order to
make the source file(s) more manageable.
Moreover, compilation can be restricted to
selected child files by means of the |\includeonly| command.
The latter feature can be used to reduce the compilation time while editing
(this was significantly more useful in the earlier days of \LaTeX{})
or to generate a smaller document which is easier to navigate.
Another application of |\includeonly| is to generate
documents consisting of selected parts of the complete document.

However, there are a few drawbacks of the plain |\include| mechanism:
\begin{itemize}
\item
The child files cannot be compiled on their own,
they can only be compiled via the main file.
A naive editing environment
(such as a text editor with an option
to have the current file processed by \LaTeX)
may require one to switch to the main file before compiling;
attempting to compile the child file produces errors.
\item
The main file must be modified (each time)
to adjust the |\includeonly| command
to the present needs. This easily leaves the main file in a messy state.
\item
The generated document will always carry the filename
of the main document. This is inconvenient if
several child files are to be compiled and
to be kept for distribution.
\end{itemize}

The present package provides a simple interface
to make child files individually compilable by \LaTeX{}.
Compiling a child file then has the same effect as compiling
the main file with an |\includeonly| command
to select the appropriate child.
Moreover the generated document will carry the name of the child
rather than the main file.
This resolves all three above issues.

This feature is meant to make the editing of books,
thesis documents and lecture notes somewhat more convenient.
However, the package can also be used efficiently for
composing a series of documents (such as exercise sheets)
which are typically distributed individually.
It then assists the author in generating the individual documents
(potentially in different versions)
as well as a document containing the collected series.
Another application is in developing style files
or other kinds of included material
where compilation of the style file could redirect
to a sample or test file.

%%%%%%%%%%%%%%%%%%%%%%%%%%%%%%%%%%%%%%%%%%%%%%%%%%%%%%%%%%%%%%%%%%%%%%%%%%%%%%%%
%%%%%%%%%%%%%%%%%%%%%%%%%%%%%%%%%%%%%%%%%%%%%%%%%%%%%%%%%%%%%%%%%%%%%%%%%%%%%%%%
\section{Usage}

First of all, the package \textsf{childdoc} is \emph{not} a standard
\LaTeXe{} |.sty| style file! Therefore it needs to be invoked in
a non-standard way.

%%%%%%%%%%%%%%%%%%%%%%%%%%%%%%%%%%%%%%%%%%%%%%%%%%%%%%%%%%%%%%%%%%%%%%%%%%%%%%%%
\subsection{Included Files}
\label{sec:include}

%%%%%%%%%%%%%%%%%%%%%%%%%%%%%%%%%%%%%%%%
\DescribeMacro{\childdocmain}
To use the package, add the commands
\begin{center}
\begin{tabular}{l}
|\input{childdoc.def}|\\
|\childdocmain{}|\\
\end{tabular}
\end{center}
at the very top of the main \LaTeX{} file,
in particular \emph{before} the |\documentclass| statement!
The argument of |\childdocmain| should be left empty
(but it must be present).

%%%%%%%%%%%%%%%%%%%%%%%%%%%%%%%%%%%%%%%%
\DescribeMacro{\childdocof}
Furthermore, add the commands
\begin{center}
\begin{tabular}{l}
|\input{childdoc.def}|\\
|\childdocof{|\textit{main}|}|\\
\end{tabular}
\end{center}
at the top of every child file \textit{child}
which is included by |\include{|\textit{child}|}|
from within the main file
(or at least for those files to be compiled individually).
The argument \textit{main} must be the filename of the main file.

There are a couple of
considerations in setting up the main and child documents:

%%%%%%%%%%%%%%%%%%%%%%%%%%%%%%%%%%%%%%%%
\paragraph{Restrictions.}

Please note the following restrictions:
\begin{itemize}
\item
|\childdocmain| must be called with one argument \textit{main}
to ensure compatibility with earlier version of the package.
It must either be empty (|\childdocmain{}|)
or precisely match the filename of the main file in which it is specified.
See \secref{sec:detection} for further information.
\item
The filename \textit{main} must be specified without the |.tex| extension.
\item
The filename \textit{main} is case sensitive
(even in case-insensitive file systems)
due to internal string comparison.
\item
The argument \textit{main} should be fully expanded, it cannot be a macro.
\item
Subdirectories and special characters should be avoided in filenames.
\item
The command |\childdocmain{|\textit{main}|}| must be followed by a whitespace.
It should not be followed immediately by another command
or by a comment mark `|%|'.
This is because the \TeX{} parser reads the token immediately following
the argument of |\childdocmain| and puts it
at the beginning of every child section;
however, a white\-space is ignored.
\end{itemize}

%%%%%%%%%%%%%%%%%%%%%%%%%%%%%%%%%%%%%%%%
\paragraph{Content of Main File.}

It is advisable to place all content in the child files included by |\include|.
Any output contained in the main file will appear in all child documents
unless suppressed manually;
it cannot be suppressed automatically by the |\includeonly| directive
and thus should normally be avoided.
A method to include some content in the main file
by means of conditional processing is described in \secref{sec:conditional}.

%%%%%%%%%%%%%%%%%%%%%%%%%%%%%%%%%%%%%%%%
\paragraph{Page Numbering.}

When only a part of the document is compiled,
the appropriate numbering of pages
(as well as other status parameters)
is determined from the |.aux| files.
The latter contain information from previous passes.
However this information needs to propagate through
all intermediate child documents.
Therefore the page numbering in child documents may well
be inconsistent until the complete document is compiled at least once.

A useful (if unconventional) way to always ensure a consistent
page numbering is to restart the numbering in each child document
and denote the pages by `\textit{child}|.|\textit{page}'
where \textit{child} represents the chapter/section number of the child file.
This can be achieved by the command
|\numberwithin{page}{|\textit{child}|}|
of the \textsf{amsmath} package
where \textit{child} can be |chapter| or |section|
depending on the chosen structuring.
Alternatively, one can modify the macro |\thepage| appropriately
and reset the counter |page| at the start of each child file.

%%%%%%%%%%%%%%%%%%%%%%%%%%%%%%%%%%%%%%%%%%%%%%%%%%%%%%%%%%%%%%%%%%%%%%%%%%%%%%%%
\subsection{Conditional Processing}
\label{sec:conditional}

The package provides a mechanism to compile different versions
of a document. To customise the versions further some conditional processing
can come in handy to distinguish which version is being compiled.
The package provides two macros to describe the compilation context:

%%%%%%%%%%%%%%%%%%%%%%%%%%%%%%%%%%%%%%%%
\DescribeMacro{\ifchilddoc}
The conditional |\ifchilddoc| distinguishes between the compilation of
child documents and the main document:
%
\begin{center}
|\ifchilddoc |\textit{child-code}| |[|\||else |\textit{main-code}]| \||fi|
\end{center}

%%%%%%%%%%%%%%%%%%%%%%%%%%%%%%%%%%%%%%%%
\DescribeMacro{\childdocname}
\DescribeMacro{\childdocjob}
The macro |\childdocname| contains the filename (without extension)
of the main or child file being processed.
Note that |\childdocjob| will always contain the name of the main file.

%%%%%%%%%%%%%%%%%%%%%%%%%%%%%%%%%%%%%%%%
\paragraph{Title Page.}

Conditional processing can be used to include a title or banner page
in the main document when proper precautions are taken.
Importantly, the code in the main file should ensure that the page counter
(as well as other status parameters which are stored in the |.aux| files)
takes the same value after the conditional processing.
Otherwise the page numbers may take divergent values
depending on which part is compiled.

For example, a title page could be declared by:
%
\begin{center}
\begin{tabular}{l}
|\ifchilddoc\||else|\\
|\addtocounter{page}{-1}|\\
\textit{code for title page}\\
|\newpage|\\
|\||fi|
\end{tabular}
\end{center}
%
A banner page for the child documents can be generated by:
%
\begin{center}
\begin{tabular}{l}
|\ifchilddoc|\\
|\addtocounter{page}{-1}|\\
\textit{code for banner page}\\
|\newpage|\\
|\||fi|
\end{tabular}
\end{center}
%
Here one could write a message such as:
\begin{center}
|This is the part \childdocname{} of \childdocjob{}.|
\end{center}

%%%%%%%%%%%%%%%%%%%%%%%%%%%%%%%%%%%%%%%%%%%%%%%%%%%%%%%%%%%%%%%%%%%%%%%%%%%%%%%%
\subsection{Flags}
\label{sec:flags}

The package makes it easy to generate different versions
of the main or child documents.
To this end compilation flags can be defined
and assigned different default values.
They will be particularly useful in conjunction
with the forwarding mechanism described in \secref{sec:forward}.

For example, it may be useful to have a flag |\version|
which can be set to |draft| or |final|.
The document source will contain some conditional code
depending on the value of |\version|.
Suppose further, the flag should default to |final| for the main file
and to |draft| for child files
which is a natural assignment for editing the document.
This is achieved by placing the following code
in the preamble of the main document
(below the |\childdocmain| directive):
%
\begin{center}
\begin{tabular}{l}
|\ifchilddoc|\\
|\providecommand{\version}{draft}|\\
|\||else|\\
|\providecommand{\version}{final}|\\
|\||fi|
\end{tabular}
\end{center}
%
The definition by |\providecommand| makes sure
that previous definitions are not overwritten.
Further statements |\providecommand{\version}{...}|
can thus be added before the above code to override it.

For the main file, one might add a line
(between |\childdocmain| and the above block)
%
\begin{center}
|%\ifchilddoc\||else\providecommand{\version}{draft}\||fi|
\end{center}
%
which can be uncommented to produce a draft version.
Likewise one can add a line to the very top of a child file
(above the |\childdocof{|\textit{main}|}| directive)
%
\begin{center}
|%\providecommand{\version}{final}|
\end{center}
%
which can be uncommented to produce the final version of this child document.

%%%%%%%%%%%%%%%%%%%%%%%%%%%%%%%%%%%%%%%%%%%%%%%%%%%%%%%%%%%%%%%%%%%%%%%%%%%%%%%%
\subsection{Forwarding}
\label{sec:forward}

Different versions of the main or child documents
using compilation flags as described in \secref{sec:flags}
can be (permanently) stored in different files
for convenient compilation, viewing and distribution.
To this end, the package defines a command
to pass on compilation to a different file:

%%%%%%%%%%%%%%%%%%%%%%%%%%%%%%%%%%%%%%%%
\DescribeMacro{\childdocforward}
The command |\childdocforward| redirects processing to
another source file:
%
\begin{center}
\begin{tabular}{l}
|\input{childdoc.def}|\\
|\childdocforward[|\textit{main}|]{|\textit{dest}|}|\\
\end{tabular}
\end{center}
%
The argument \textit{dest} is the destination file
(without extension).
It should be the main file or one of the child files.
Note that further \textsf{childdoc} directives
such as |\childdocof| and |\childdocforward|
in the indicated file will be processed in this form.
The optional argument \textit{main}
passes on directly to the main file \textit{main}
while pretending to compile the child \textit{dest}.
This form behaves as if \textit{dest}
issues |\childdocof{|\textit{main}|}| right away,
and no further \textsf{childdoc} directives will be processed.

%%%%%%%%%%%%%%%%%%%%%%%%%%%%%%%%%%%%%%%%
\DescribeMacro{\...prefix}
In the alternative form |\childdocforwardprefix|,
%
\begin{center}
\begin{tabular}{l}
|\input{childdoc.def}|\\
|\childdocforwardprefix[|\textit{main}|]{|\textit{prefix}|}{|\textit{dest}|}|
\end{tabular}
\end{center}
%
the destination file is determined by a pattern
depending on the current file:
To make this work, the current file must be called
`{\textit{prefix}\hspace{0.2em}\textit{suffix}}'
with \textit{prefix} matching precisely the argument.
Processing is then passed on to the file
`{\textit{dest}\hspace{0.2em}\textit{suffix}}'.
Surely, the same effect is achieved by
directly specifying the
argument `{\textit{dest}\hspace{0.2em}\textit{suffix}}'
in the first form.
However, that requires to set up a different file
for each child. With the alternative form of the command
all these files can have exactly the same content
which simplifies setting them up and maintaining them.

For example, the following file |draft.tex|
with a compilation flag |\version| as described in \secref{sec:flags}
compiles the main document as a draft:
%
\begin{center}
\begin{tabular}{l}
|\def\version{draft}|\\
|\input{childdoc.def}|\\
|\childdocforward{|\textit{main}|}|
\end{tabular}
\end{center}
%
Likewise, the following files |final|\textit{nn}|.tex|
compile the final version of the child document
|child|\textit{nn}|.tex|:
%
\begin{center}
\begin{tabular}{l}
|\def\version{final}|\\
|\input{childdoc.def}|\\
|\childdocforwardprefix{final}{child}|
\end{tabular}
\end{center}
%

Note that when several versions of a main file and/or of each child file
are to be generated, it may be convenient to set up a |Makefile| or
shell script to automatise the process.

%%%%%%%%%%%%%%%%%%%%%%%%%%%%%%%%%%%%%%%%%%%%%%%%%%%%%%%%%%%%%%%%%%%%%%%%%%%%%%%%
\subsection{Command Line Processing}
\label{sec:commandline}

The effect of redirection files can also be achieved by invoking
the \LaTeX{} compiler with a more elaborate command line.
Most conveniently this should be done as part
of a shell script or a |Makefile|.

When using \textsf{childdoc} in the main file, the following
command lines effectively perform a redirection
(note that depending on the shell being used,
backslashes may have to be doubled: `|\|' $\to$ `|\\|'):
%
\begin{center}
|... -jobname "|\textit{target}|" |\\|"|[\textit{flags}]%
|\input{childdoc.def}\childdocforward[|\textit{main}|]{|\textit{dest}|}"|
\end{center}
%
Here \textit{target} is the name of the output file,
\textit{main} is the name of the main file
and \textit{dest} is the name of the main or child file to be processed
(all filenames without extensions).
The optional argument \textit{main} can be omitted
if \textit{main} matches \textit{dest}.
Optionally, compilation \textit{flags} can be defined via |\def| commands.
This command line makes the \TeX{} engine believe
it is compiling the file \textit{target}
whose content is specified as the latter parameter.
The provided code then forwards the processing to
\textit{main} or \textit{dest} as described in \secref{sec:forward}.

%%%%%%%%%%%%%%%%%%%%%%%%%%%%%%%%%%%%%%%%%%%%%%%%%%%%%%%%%%%%%%%%%%%%%%%%%%%%%%%%
\subsection{Include by Input}
\label{sec:input}

Including child documents by |\include| has some restrictions by design.
Most notably, the content of a child document always occupies
its own set of pages; pages cannot be shared between child documents.
Usually, this behaviour makes perfect sense
because each child document contain an essential part of the document.
However, in some situations it may be desirable to compose
a document from a collection of parts
without having mandatory page breaks between then.
For this case, the package
provides a mechanism to include parts
by |\input| which can also be processed individually.
However, by construction this mechanism
requires manual handling of the content to be output.

%%%%%%%%%%%%%%%%%%%%%%%%%%%%%%%%%%%%%%%%
\DescribeMacro{\ifchilddocmanual}
The main file should be prepared as usual, see \secref{sec:include}.
However, the document body must make a distinction
between processing of an individual part and of the main document, e.g.:
%
\begin{center}
\begin{tabular}{l}
|\ifchilddocmanual|\\
|\input{\childdocname}|\\
|\||else|\\
\textit{document body with }|\input{|\textit{part}|}|\\
|\||fi|
\end{tabular}
\end{center}
%
The conditional |\ifchilddocmanual| is true whenever
a part to be included by |\input| is being compiled,
and the name of the part is stored in |\childdocname|.

%%%%%%%%%%%%%%%%%%%%%%%%%%%%%%%%%%%%%%%%
\DescribeMacro{\childdocby}
Each part to be included by |\input| should start with:
%
\begin{center}
\begin{tabular}{l}
|\input{childdoc.def}|\\
|\childdocby{|\textit{main}|}|\\
\end{tabular}
\end{center}
%
The directive |\childdocby| is similar to |\childdocof|
described in \secref{sec:include},
but the subsequent selection of content must be done manually.
To that end, both |\ifchilddoc| and |\ifchilddocmanual|
will be true upon processing of a part,
and the name of the part is stored in |\childdocname|.
Note that |\jobname| will be set to the filename of the current part
so that each part receives an individual |.aux| file
that does not interfere with the |.aux| file(s) of the main document.
This behaviour can be altered by the alternative form
|\childdocby[*]{|\textit{main}|}| (with a non-empty optional argument)
which uses the |.aux| file of the main document
by setting |\jobname| to \textit{main}.

%%%%%%%%%%%%%%%%%%%%%%%%%%%%%%%%%%%%%%%%%%%%%%%%%%%%%%%%%%%%%%%%%%%%%%%%%%%%%%%%
\subsection{Driver Development}
\label{sec:driver}

The \textsf{childdoc} mechanism can also be use for the development
of definition files such as \LaTeX{} styles or classes.
This case differs from the above setup with multiple parts
included by |\include| in that no |\includeonly| should be invoked.
This can be achieved by starting the include file
(before |\ProvidesPackage|) with:
%
\begin{center}
\begin{tabular}{l}
|\input{childdoc.def}|\\
|\childdocforward{|\textit{main}|}|\\
\end{tabular}
\end{center}
%
or alternatively with:
%
\begin{center}
\begin{tabular}{l}
|\input{childdoc.def}|\\
|\childdocby{|\textit{main}|}|\\
\end{tabular}
\end{center}
%
Both forms have slightly different effects as described above.
The main file is prepared as usual, see \secref{sec:include}.

%%%%%%%%%%%%%%%%%%%%%%%%%%%%%%%%%%%%%%%%%%%%%%%%%%%%%%%%%%%%%%%%%%%%%%%%%%%%%%%%
\subsection{Legacy Detection}
\label{sec:detection}

The directive |\childdocmain| in the main file can detect
whether the complete document or merely a child is to be compiled
even without using the directive |\childdocof|.
This method is deprecated because it is less robust
and there is no compelling reason to use it;
it is merely provided for backward compatibility
and it may be removed in future versions.

If the detection mechanism is to be used,
it is mandatory to correctly specify
the filename of the main file as the argument of |\childdocmain|:
%
\begin{center}
\begin{tabular}{l}
|\input{childdoc.def}|\\
|\childdocmain{|\textit{main}|}|\\
\end{tabular}
\end{center}
%
If |\jobname| does not match the argument \textit{main} of |\childdocmain|,
it is assumed that |\jobname| points to the child file to be compiled.
When using |\childdocmain| with the main file specified as argument,
it suffices to start a child file
with just |\input{|\textit{main}|}|
without loading of the package and using |\childdocof|.
If instead all processing is done
with the appropriate \textsf{childdoc} directives,
the argument of \textit{main} of |\childdocmain| can be empty.

An alternative version of the command line processing described
in \secref{sec:commandline} using the detection mechanism reads:
%
\begin{center}
|... -jobname "|\textit{target}|" "|[\textit{flags}]%
[|\def\jobname{|\textit{dest}|}|]|\input{|\textit{main}|}"|
\end{center}

%%%%%%%%%%%%%%%%%%%%%%%%%%%%%%%%%%%%%%%%%%%%%%%%%%%%%%%%%%%%%%%%%%%%%%%%%%%%%%%%
\subsection{Manual Code}
\label{sec:manual}

In case one cannot be certain whether the definitions file |childdoc.def|
is installed on the target \TeX{} distribution
and one prefers not to ship it,
it is conceivable to paste a few relevant commands into the sources.

To that end, drop all statements |\input{childdoc.def}|
and perform the replacements as outlined below.
Instead of |\childdocmain{|\textit{main}|}| add the following code
to the top of the main file:
%
\begin{center}
\begin{tabular}{l}
|\||ifdefined\childdocname\endinput\||fi\newif\ifchilddoc|\\
|\edef\childdocname{\scantokens\expandafter{\jobname\noexpand}}|\\
|\def\childdocmain{|\textit{main}|}\||ifx\childdocmain\childdocname\||else|\\
|\childdoctrue\includeonly{\childdocname}\let\jobname\childdocmain\||fi|\\
\end{tabular}
\end{center}
%
Instead of |\childdocof{|\textit{main}|}| just include the main file
at the top of each child file:
%
\begin{center}
|\input{|\textit{main}|}|
\end{center}
%
A simple redirection |\childdocforward{|\textit{dest}|}| is achieved by:
%
\begin{center}
|\def\jobname{|\textit{dest}|}\input{\jobname}|
\end{center}
%
The redirection with prefix
|\childdocforwardprefix[|\textit{prefix}|]{|\textit{dest}|}|
is accomplished by:
%
\begin{center}
\begin{tabular}{l}
|{\edef\jobname{\scantokens\expandafter{\jobname\noexpand}}|\\
|\def\redirectjob |\textit{prefix}|#1~~~{\gdef\jobname{|\textit{dest}|#1}}|\\
|\expandafter\redirectjob\jobname~~~}\input{\jobname}|
\end{tabular}
\end{center}

In an alternative approach,
child documents can be compiled by a specific command line
without additional code or specific definitions:
%
\begin{center}
|... -jobname "|\textit{target}|" "|[\textit{flags}]%
|\includeonly{|\textit{dest}|}\input{|\textit{main}|}"|
\end{center}
%

%%%%%%%%%%%%%%%%%%%%%%%%%%%%%%%%%%%%%%%%%%%%%%%%%%%%%%%%%%%%%%%%%%%%%%%%%%%%%%%%
%%%%%%%%%%%%%%%%%%%%%%%%%%%%%%%%%%%%%%%%%%%%%%%%%%%%%%%%%%%%%%%%%%%%%%%%%%%%%%%%
\section{Information}

%%%%%%%%%%%%%%%%%%%%%%%%%%%%%%%%%%%%%%%%%%%%%%%%%%%%%%%%%%%%%%%%%%%%%%%%%%%%%%%%
\subsection{Copyright}

Copyright \copyright{} 2017--2018 Niklas Beisert

This work may be distributed and/or modified under the
conditions of the \LaTeX{} Project Public License, either version 1.3
of this license or (at your option) any later version.
The latest version of this license is in
  \url{http://www.latex-project.org/lppl.txt}
and version 1.3 or later is part of all distributions of \LaTeX{}
version 2005/12/01 or later.

This work has the LPPL maintenance status `maintained'.

The Current Maintainer of this work is Niklas Beisert.

This work consists of the files |README.txt|, |childdoc.ins| and |childdoc.dtx|
as well as the derived files |childdoc.def|, |cdocsamp.tex|
with |cdocsch1.tex|, |cdocsch2.tex|, |cdocspt3.tex|, |cdocspt4.tex|,
|cdocsdrf.tex|, |cdocsfn1.tex|, |cdocsfn2.tex|
as well as |childdoc.pdf|.

%%%%%%%%%%%%%%%%%%%%%%%%%%%%%%%%%%%%%%%%%%%%%%%%%%%%%%%%%%%%%%%%%%%%%%%%%%%%%%%%
\subsection{Files and Installation}

The package consists of the files:
%
\begin{center}
\begin{tabular}{ll}
    |README.txt|   & readme file \\
    |childdoc.ins| & installation file \\
    |childdoc.dtx| & source file \\
    |childdoc.def| & definition file \\
    |cdocsamp.tex| & sample main file \\
    |cdocsch1.tex| & sample include file \\
    |cdocsch2.tex| & sample include file \\
    |cdocspt3.tex| & sample part file \\
    |cdocspt4.tex| & sample part file \\
    |cdocsdrf.tex| & sample redirection file \\
    |cdocsfn1.tex| & sample redirection file \\
    |cdocsfn2.tex| & sample redirection file \\
    |childdoc.pdf| & manual
\end{tabular}
\end{center}
%
The distribution consists of the files
|README.txt|, |childdoc.ins| and |childdoc.dtx|.
%
\begin{itemize}
\item
Run (pdf)\LaTeX{} on |childdoc.dtx|
to compile the manual |childdoc.pdf| (this file).
\item
Run \LaTeX{} on |childdoc.ins| to create the definitions file |childdoc.def|
and the sample |cdocsamp.tex| with include files
|cdocsch1.tex|, |cdocsch2.tex|, |cdocspt3.tex|, |cdocspt4.tex|,
|cdocsdrf.tex|, |cdocsfn1.tex|, |cdocsfn2.tex|.
Then copy the file |childdoc.def| to an appropriate directory of your \LaTeX{}
distribution, e.g.\ \textit{texmf-root}|/tex/latex/childdoc|.
\end{itemize}

%%%%%%%%%%%%%%%%%%%%%%%%%%%%%%%%%%%%%%%%%%%%%%%%%%%%%%%%%%%%%%%%%%%%%%%%%%%%%%%%
\subsection{Related CTAN Packages}

There are several other packages which offer a similar functionality:
%
\begin{itemize}
\item
The packages
\href{http://ctan.org/pkg/docmute}{\textsf{docmute}},
\href{http://ctan.org/pkg/includex}{\textsf{includex}} and
\href{http://ctan.org/pkg/standalone}{\textsf{standalone}}
provide commands to include only the document body of
a child file thus allowing both files to be compiled individually.
\item
The packages \href{http://ctan.org/pkg/subdocs}{\textsf{subdocs}}
and \href{http://ctan.org/pkg/subfiles}{\textsf{subfiles}}
provide structures in which the main and child documents can be
encapsulated and allowing them to be compiled individually.
The inclusion mechanism is different from the conventional |\include|.
\item
The package \href{http://ctan.org/pkg/combine}{\textsf{combine}}
is an elaborate solution to combine several documents into one.
\end{itemize}
%
See also the CTAN topic \href{http://ctan.org/topic/subdocs}{\textsf{subdocs}}
for further related packages.
The present package differs from the above solutions in that
a document structure constructed with the conventional |\include| mechanism
just needs two extra commands at the top of every file
such that all constituent files can be compiled individually.

%%%%%%%%%%%%%%%%%%%%%%%%%%%%%%%%%%%%%%%%%%%%%%%%%%%%%%%%%%%%%%%%%%%%%%%%%%%%%%%%
%\subsection{Feature Suggestions}
%
%The following is a list of features which may be useful for future
%versions of this package:
%%
%\begin{itemize}
%\item
%\ldots
%\end{itemize}

%%%%%%%%%%%%%%%%%%%%%%%%%%%%%%%%%%%%%%%%%%%%%%%%%%%%%%%%%%%%%%%%%%%%%%%%%%%%%%%%
\subsection{Revision History}

%%%%%%%%%%%%%%%%%%%%%%%%%%%%%%%%%%%%%%%%
\paragraph{v2.0:} 2018/12/30

\begin{itemize}
\item
immediate forward processing
\item
added |\childdocby| mechanism
\item
manual restructured
\end{itemize}

%%%%%%%%%%%%%%%%%%%%%%%%%%%%%%%%%%%%%%%%
\paragraph{v1.6:} 2018/01/17

\begin{itemize}
\item
application for development of include files
\item
corrections to manual
\end{itemize}

%%%%%%%%%%%%%%%%%%%%%%%%%%%%%%%%%%%%%%%%
\paragraph{v1.5:} 2017/05/21

\begin{itemize}
\item
more complete structuring introduced
\item
|\childdocof| introduced
\item
|\childdoc| renamed to |\childdocmain|
\item
|\childredirect| renamed to |\childdocforward| and |\childdocforwardprefix|
and functionality expanded
\end{itemize}

%%%%%%%%%%%%%%%%%%%%%%%%%%%%%%%%%%%%%%%%
\paragraph{v1.0:} 2017/04/27

\begin{itemize}
\item
manual and install package
\item
first version published on CTAN
\end{itemize}

%%%%%%%%%%%%%%%%%%%%%%%%%%%%%%%%%%%%%%%%
\paragraph{v0.6:} 2017/04/26

\begin{itemize}
\item
redirection mechanism added
\end{itemize}

%%%%%%%%%%%%%%%%%%%%%%%%%%%%%%%%%%%%%%%%
\paragraph{v0.5:} 2017/04/26

\begin{itemize}
\item
functionality in definition file
\end{itemize}


%%%%%%%%%%%%%%%%%%%%%%%%%%%%%%%%%%%%%%%%%%%%%%%%%%%%%%%%%%%%%%%%%%%%%%%%%%%%%%%%
%%%%%%%%%%%%%%%%%%%%%%%%%%%%%%%%%%%%%%%%%%%%%%%%%%%%%%%%%%%%%%%%%%%%%%%%%%%%%%%%
%%%%%%%%%%%%%%%%%%%%%%%%%%%%%%%%%%%%%%%%%%%%%%%%%%%%%%%%%%%%%%%%%%%%%%%%%%%%%%%%
\appendix

\settowidth\MacroIndent{\rmfamily\scriptsize 000\ }

 \DocInput{childdoc.dtx}

\end{document}
%</driver>
% \fi
%
% %%%%%%%%%%%%%%%%%%%%%%%%%%%%%%%%%%%%%%%%%%%%%%%%%%%%%%%%%%%%%%%%%%%%%%%%%%%%%%
% %%%%%%%%%%%%%%%%%%%%%%%%%%%%%%%%%%%%%%%%%%%%%%%%%%%%%%%%%%%%%%%%%%%%%%%%%%%%%%
% \section{Sample}
%\iffalse
%<*samplemain>
%\fi
%
% The following presents a sample document
% with two chapters, two parts, a title page,
% a compile flag as well as three forwarding files to set the flag.
% It consists of eight |.tex| files:
% \begin{center}
% \begin{tabular}{ll}
% |cdocsamp.tex|&main file\\
% |cdocsch1.tex|&include file for chapter 1\\
% |cdocsch2.tex|&include file for chapter 2\\
% |cdocspt3.tex|&include file for part 3\\
% |cdocspt4.tex|&include file for part 4\\
% |cdocsdrf.tex|&forwarding file for main file in draft mode\\
% |cdocsfi1.tex|&forwarding file for final version of chapter 1\\
% |cdocsfi2.tex|&forwarding file for final version of chapter 2\\
% \end{tabular}
% \end{center}
% Each of the eight files can be compiled directly by the \LaTeX{} compiler.
%
% %%%%%%%%%%%%%%%%%%%%%%%%%%%%%%%%%%%%%%
% \paragraph{Main File.}
%
% The main file is called |cdocsamp.tex|.
%
% Load the \textsf{childdoc} definitions and
% declare the filename for the main document:
%    \begin{macrocode}
\input{childdoc.def}
\childdocmain{}
%    \end{macrocode}

% Optional override for |\version| flag:
%    \begin{macrocode}
%%\ifchilddoc\else\providecommand{\version}{draft}\fi
%    \end{macrocode}

% Define the default values for the |\version| flag
% (|final| for the main file and |draft| for childs):
%    \begin{macrocode}
\ifchilddoc
\providecommand{\version}{draft}
\else
\providecommand{\version}{final}
\fi
%    \end{macrocode}

% Load the standard document class:
%    \begin{macrocode}
\documentclass[12pt]{article}
%    \end{macrocode}

% Start the document body:
%    \begin{macrocode}
\begin{document}
%    \end{macrocode}

% Declare a title page.
% Print title, part of document being processed and version flag:
%    \begin{macrocode}
\addtocounter{page}{-1}
\begin{center}
{\LARGE\bfseries{}childdoc example\par}
\vspace{1cm}
\ifchilddoc
\ifchilddocmanual part\else chapter\fi:
`\childdocname' of `\childdocjob'\par
\else
main document: `\childdocjob'\par
\fi
version: \version\par
\end{center}
\newpage
%    \end{macrocode}

% Manually include selected file,
% otherwise process as usual:
%    \begin{macrocode}
\ifchilddocmanual
\section*{part `\childdocname'}
\input{\childdocname}
\else
%    \end{macrocode}

% Include the two chapters:
%    \begin{macrocode}
\include{cdocsch1}
\include{cdocsch2}
%    \end{macrocode}

% Include the two parts unless only chapters should be displayed:
%    \begin{macrocode}
\ifchilddoc\else
\section{part three}
\input{cdocspt3}
\section{part four}
\input{cdocspt4}
\fi
%    \end{macrocode}

% Process as usual until here:
%    \begin{macrocode}
\fi
%    \end{macrocode}

% End of document body:
%    \begin{macrocode}
\end{document}
%    \end{macrocode}
%\iffalse
%</samplemain>
%\fi
%
% %%%%%%%%%%%%%%%%%%%%%%%%%%%%%%%%%%%%%%
% \paragraph{Chapter Include Files.}
%
% The include files are called |cdocsch1.tex| and |cdocsch2.tex|.
%
%\iffalse
%<*samplechap1|samplechap2>
%\fi

% Optional override for |\version| flag:
%    \begin{macrocode}
%%\providecommand{\version}{final}
%    \end{macrocode}

% Include the main document:
%    \begin{macrocode}
\input{childdoc.def}
\childdocof{cdocsamp}
%    \end{macrocode}

%\iffalse
%</samplechap1|samplechap2>
%\fi
%
%\iffalse
%<*samplechap1>
%\fi
% Some text for chapter 1:
%    \begin{macrocode}
\section{one}
some text in chapter one
%    \end{macrocode}

%\iffalse
%</samplechap1>
%\fi
% Some text for chapter 2:
%\iffalse
%<*samplechap2>
%\fi
%    \begin{macrocode}
\section{two}
more text in chapter two
%    \end{macrocode}

%\iffalse
%</samplechap2>
%\fi
%
% %%%%%%%%%%%%%%%%%%%%%%%%%%%%%%%%%%%%%%
% \paragraph{Part Include Files.}
%
% The include files are called |cdocspt3.tex| and |cdocspt4.tex|.
%
%\iffalse
%<*samplepart3|samplepart4>
%\fi

% Optional override for |\version| flag:
%    \begin{macrocode}
%%\providecommand{\version}{final}
%    \end{macrocode}

% Include the main document:
%    \begin{macrocode}
\input{childdoc.def}
\childdocby{cdocsamp}
%    \end{macrocode}

%\iffalse
%</samplepart3|samplepart4>
%\fi
%
%\iffalse
%<*samplepart3>
%\fi
% Some text for part 3:
%    \begin{macrocode}
some text in part three
%    \end{macrocode}

%\iffalse
%</samplepart3>
%\fi
% Some text for part 4:
%\iffalse
%<*samplepart4>
%\fi
%    \begin{macrocode}
more text in part four
%    \end{macrocode}

%\iffalse
%</samplepart4>
%\fi
%
% %%%%%%%%%%%%%%%%%%%%%%%%%%%%%%%%%%%%%%
% \paragraph{Forwarding for a Complete Draft.}
%
% The following forwarding file |cdocsdrf.tex|
% compiles the main document in draft mode:
%\iffalse
%<*sampledraft>
%\fi
%    \begin{macrocode}
\def\version{draft}
\input{childdoc.def}
\childdocforward{cdocsamp}
%    \end{macrocode}

%\iffalse
%</sampledraft>
%\fi
%
% %%%%%%%%%%%%%%%%%%%%%%%%%%%%%%%%%%%%%%
% \paragraph{Forwarding for Final Version of the Chapters.}
%
% The following forwarding files |cdocsfn1.tex| and |cdocsfn2.tex|
% (with identical content)
% compile the final versions of the child documents
% |cdocsch1.tex| and |cdocsch2.tex|, respectively:
%\iffalse
%<*samplefinal>
%\fi
%    \begin{macrocode}
\def\version{final}
\input{childdoc.def}
\childdocforwardprefix[cdocsamp]{cdocsfn}{cdocsch}
%    \end{macrocode}

%\iffalse
%</samplefinal>
%\fi
%
% %%%%%%%%%%%%%%%%%%%%%%%%%%%%%%%%%%%%%%
% \paragraph{Command Line Processing.}
%
% The following three command lines generate the output files
% |cdocscld|, |cdocscl1| and |cdocscl2|
% which should be identical to
% |cdocsdrf|, |cdocsch1| and |cdocsfn2|, respectively:
% \begin{center}
% \begin{tabular}{l}
% |latex -jobname cdocscld \|\\
% |  "\def\version{draft}\input{childdoc.def}\childdocforward{cdocsamp}"|\\
% |latex -jobname cdocscl1 \|\\
% |  "\input{childdoc.def}\childdocforward[cdocsamp]{cdocsch1}"|\\
% |latex -jobname cdocscl2 \|\\
% |  "\def\version{final}\input{childdoc.def}\childdocforward{cdocsch2}"|
% \end{tabular}
% \end{center}
% Note that the trailing backslash on each first line
% merely continues the input to the second line
% (for convenient cut ant paste).
% Furthermore, the command |latex| can be replaced by any
% of its alternative versions such as |pdflatex|.
%
% %%%%%%%%%%%%%%%%%%%%%%%%%%%%%%%%%%%%%%%%%%%%%%%%%%%%%%%%%%%%%%%%%%%%%%%%%%%%%%
% %%%%%%%%%%%%%%%%%%%%%%%%%%%%%%%%%%%%%%%%%%%%%%%%%%%%%%%%%%%%%%%%%%%%%%%%%%%%%%
% \section{Implementation}
%\iffalse
%<*package>
%\fi
%
% This section describes the definitions file |childdoc.def|.

% The definitions cannot be loaded using |\usepackage| or |\RequirePackage|
% which has a mechanism to prevent loading a style file more than once.
% When loading the definitions by means of |\input|
% multiple instances have to be prevented manually:
%\iffalse
%This code needs to be before the `\ProvidesFile' directive
%which is defined at the beginning of this file.
%Therefore it is also placed there and commented out here.
%</package>
%<*discard>
%\fi
%    \begin{macrocode}
\ifdefined\childdocmain\endinput\fi
%    \end{macrocode}
%\iffalse
%</discard>
%<*package>
%\fi
%
% \macro{\ifchilddoc}
% \macro{\ifchilddocmanual}
% The conditional |\ifchilddoc| tells whether a
% child (true) or main (false) document is being compiled.
% The conditional |\ifchilddocmanual| tells whether
% the |\includeonly| mechanism is used (false) or
% the selection of child files must be performed manually (true).
% The definitions initialise to false:
%    \begin{macrocode}
\newif\ifchilddoc
\newif\ifchilddocmanual
%    \end{macrocode}

% \macro{\childdocname}
% \macro{\childdocjob}
% The macro |\childdocname| stores the name of the main document
% to be compiled. The macro |\childdocjob| stores the name of
% the document on which the \LaTeX{} compiler was originally invoked.
% The content of |\jobname| cannot be compared
% to filenames specified in the source due to different catcodes.
% The following code rescans |\jobname|, stores the result
% in |\childdocname| and saves a copy in |\childdocjob|:
%    \begin{macrocode}
\edef\childdocname{\scantokens\expandafter{\jobname\noexpand}}
\let\childdocjob\childdocname
%    \end{macrocode}

% \macro{\childdocdisable}
% The macro |\childdocdisable| prevents the main file
% from being processed more than once.
% At this stage, the main document command |\childdocmain|
% is assumed to be called once again where it should do nothing.
% Any subsequent call to it should prevent
% a secondary processing of the main document
% It overwrites the forwarding commands
% |\childdocof| and |\childdocforward|
% with empty macros to prevent further inclusions of the main document:
%    \begin{macrocode}
\newcommand{\childdocdisable}
{
  \renewcommand{\childdocmain}[1]{\renewcommand{\childdocmain}[1]{\endinput}}
  \renewcommand{\childdocof}[1]{}
  \renewcommand{\childdocby}[2][]{}
  \renewcommand{\childdocforward}[2][]{}
  \renewcommand{\childdocdisable}{}
}
%    \end{macrocode}

% \macro{\childdocmain}
% The macro |\childdocmain| is to be called at the top of the main file
% with nothing or the main filename (without extension) as argument.
% First, it breaks loops.
% If the argument is not empty and does not match |\childdocname|
% (which is set by the first inclusion of |childdoc.def|),
% |\ifchilddoc| is set to true, |\includeonly| is applied to the child file
% and |\jobname| is set to the main file
% (for proper handling of |.aux| files):
%    \begin{macrocode}
\newcommand{\childdocmain}[1]
{
  \childdocdisable\childdocmain{}
  \if?#1?\else
    \begingroup
      \def\childdoctmp{#1}
      \ifx\childdoctmp\childdocname
        \def\childdoctmp{}
      \else
        \def\childdoctmp
        {
          \childdoctrue
          \includeonly{\childdocname}
          \def\childdocjob{#1}
          \def\jobname{#1}
        }
      \fi
      \expandafter
    \endgroup
    \childdoctmp
  \fi
}
%    \end{macrocode}

% \macro{\childdocof}
% The command |\childdocof| redirects
% compilation to the main file |#1|.
%    \begin{macrocode}
\newcommand{\childdocof}[1]
{
  \childdocdisable
  \childdoctrue
  \includeonly{\childdocname}
  \def\jobname{#1}
  \def\childdocjob{#1}
  \input{#1}
}
%    \end{macrocode}

% \macro{\childdocby}
% The command |\childdocby| ....
%    \begin{macrocode}
\newcommand{\childdocby}[2][]
{
  \childdocdisable
  \childdoctrue
  \childdocmanualtrue
  \if?#1?\else
    \def\jobname{#2}
  \fi
  \def\childdocjob{#2}
  \input{#2}
  \endinput
}
%    \end{macrocode}

% \macro{\childdocforward}
% The command |\childdocforward| redirects
% compilation to the main file or
% (if the optional argument is given) a child file.
% Parameters are set as if the main file
% or a child file starting with |\childdocof| was compiled.
% Then compilation is handed over to the main file:
%    \begin{macrocode}
\newcommand{\childdocforward}[2][]
{
  \begingroup
    \if?#1?
      \def\childdoctmp
      {
        \def\childdocname{#2}
        \def\childdocjob{#2}
        \def\jobname{#2}
        \input{#2}
        \endinput
      }
    \else
      \def\childdoctmp
      {
        \childdocdisable
        \def\childdocname{#2}
        \childdoctrue
        \includeonly{#2}
        \def\childdocjob{#1}
        \def\jobname{#1}
        \input{#1}
        \endinput
      }
    \fi
    \expandafter
  \endgroup
  \childdoctmp
}
%    \end{macrocode}

% \macro{\childdocforwardprefix}
% The command |\childdocforwardprefix| redirects
% compilation to the main or a child file by means of a pattern.
% The prefix |#1| in the current filename is replaced by |#2|
% and the suffix of the current filename is kept
% (it is assumed that the filename does not contain the substring `|~~~|'
% which is used as a delimiter).
% Compilation is handed over to the new file by |\childdocforward|:
%    \begin{macrocode}
\newcommand{\childdocforwardprefix}[3][]
{
  \begingroup
    \def\childdocextract #2##1~~~{\def\childdoctmp{\childdocforward[#1]{#3##1}}}
    \expandafter\childdocextract\childdocname~~~
    \expandafter
  \endgroup
  \childdoctmp
}
%    \end{macrocode}

% \macro{\childdoc}
% The deprecated macro |\childdoc| is a legacy version of |\childdocmain|:
%    \begin{macrocode}
\newcommand{\childdoc}{\childdocmain}
%    \end{macrocode}

% \macro{\childdocredirect}
% The deprecated macro |\childdocredirect| is a legacy version
% of |\childdocforward| and |\childdocforwardprefix|:
%    \begin{macrocode}
\newcommand{\childdocredirect}[2][]
{
  \begingroup
    \if?#1?
      \def\childdoctmp{\childdocforward{#2}}
    \else
      \def\childdoctmp{\childdocforwardprefix{#1}{#2}}
    \fi
    \expandafter
  \endgroup
  \childdoctmp
}
%    \end{macrocode}

%\iffalse
%</package>
%\fi
%
\endinput
\childdocforward{cdocsch2}"|
% \end{tabular}
% \end{center}
% Note that the trailing backslash on each first line
% merely continues the input to the second line
% (for convenient cut ant paste).
% Furthermore, the command |latex| can be replaced by any
% of its alternative versions such as |pdflatex|.
%
% %%%%%%%%%%%%%%%%%%%%%%%%%%%%%%%%%%%%%%%%%%%%%%%%%%%%%%%%%%%%%%%%%%%%%%%%%%%%%%
% %%%%%%%%%%%%%%%%%%%%%%%%%%%%%%%%%%%%%%%%%%%%%%%%%%%%%%%%%%%%%%%%%%%%%%%%%%%%%%
% \section{Implementation}
%\iffalse
%<*package>
%\fi
%
% This section describes the definitions file |childdoc.def|.

% The definitions cannot be loaded using |\usepackage| or |\RequirePackage|
% which has a mechanism to prevent loading a style file more than once.
% When loading the definitions by means of |\input|
% multiple instances have to be prevented manually:
%\iffalse
%This code needs to be before the `\ProvidesFile' directive
%which is defined at the beginning of this file.
%Therefore it is also placed there and commented out here.
%</package>
%<*discard>
%\fi
%    \begin{macrocode}
\ifdefined\childdocmain\endinput\fi
%    \end{macrocode}
%\iffalse
%</discard>
%<*package>
%\fi
%
% \macro{\ifchilddoc}
% \macro{\ifchilddocmanual}
% The conditional |\ifchilddoc| tells whether a
% child (true) or main (false) document is being compiled.
% The conditional |\ifchilddocmanual| tells whether
% the |\includeonly| mechanism is used (false) or
% the selection of child files must be performed manually (true).
% The definitions initialise to false:
%    \begin{macrocode}
\newif\ifchilddoc
\newif\ifchilddocmanual
%    \end{macrocode}

% \macro{\childdocname}
% \macro{\childdocjob}
% The macro |\childdocname| stores the name of the main document
% to be compiled. The macro |\childdocjob| stores the name of
% the document on which the \LaTeX{} compiler was originally invoked.
% The content of |\jobname| cannot be compared
% to filenames specified in the source due to different catcodes.
% The following code rescans |\jobname|, stores the result
% in |\childdocname| and saves a copy in |\childdocjob|:
%    \begin{macrocode}
\edef\childdocname{\scantokens\expandafter{\jobname\noexpand}}
\let\childdocjob\childdocname
%    \end{macrocode}

% \macro{\childdocdisable}
% The macro |\childdocdisable| prevents the main file
% from being processed more than once.
% At this stage, the main document command |\childdocmain|
% is assumed to be called once again where it should do nothing.
% Any subsequent call to it should prevent
% a secondary processing of the main document
% It overwrites the forwarding commands
% |\childdocof| and |\childdocforward|
% with empty macros to prevent further inclusions of the main document:
%    \begin{macrocode}
\newcommand{\childdocdisable}
{
  \renewcommand{\childdocmain}[1]{\renewcommand{\childdocmain}[1]{\endinput}}
  \renewcommand{\childdocof}[1]{}
  \renewcommand{\childdocby}[2][]{}
  \renewcommand{\childdocforward}[2][]{}
  \renewcommand{\childdocdisable}{}
}
%    \end{macrocode}

% \macro{\childdocmain}
% The macro |\childdocmain| is to be called at the top of the main file
% with nothing or the main filename (without extension) as argument.
% First, it breaks loops.
% If the argument is not empty and does not match |\childdocname|
% (which is set by the first inclusion of |childdoc.def|),
% |\ifchilddoc| is set to true, |\includeonly| is applied to the child file
% and |\jobname| is set to the main file
% (for proper handling of |.aux| files):
%    \begin{macrocode}
\newcommand{\childdocmain}[1]
{
  \childdocdisable\childdocmain{}
  \if?#1?\else
    \begingroup
      \def\childdoctmp{#1}
      \ifx\childdoctmp\childdocname
        \def\childdoctmp{}
      \else
        \def\childdoctmp
        {
          \childdoctrue
          \includeonly{\childdocname}
          \def\childdocjob{#1}
          \def\jobname{#1}
        }
      \fi
      \expandafter
    \endgroup
    \childdoctmp
  \fi
}
%    \end{macrocode}

% \macro{\childdocof}
% The command |\childdocof| redirects
% compilation to the main file |#1|.
%    \begin{macrocode}
\newcommand{\childdocof}[1]
{
  \childdocdisable
  \childdoctrue
  \includeonly{\childdocname}
  \def\jobname{#1}
  \def\childdocjob{#1}
  \input{#1}
}
%    \end{macrocode}

% \macro{\childdocby}
% The command |\childdocby| ....
%    \begin{macrocode}
\newcommand{\childdocby}[2][]
{
  \childdocdisable
  \childdoctrue
  \childdocmanualtrue
  \if?#1?\else
    \def\jobname{#2}
  \fi
  \def\childdocjob{#2}
  \input{#2}
  \endinput
}
%    \end{macrocode}

% \macro{\childdocforward}
% The command |\childdocforward| redirects
% compilation to the main file or
% (if the optional argument is given) a child file.
% Parameters are set as if the main file
% or a child file starting with |\childdocof| was compiled.
% Then compilation is handed over to the main file:
%    \begin{macrocode}
\newcommand{\childdocforward}[2][]
{
  \begingroup
    \if?#1?
      \def\childdoctmp
      {
        \def\childdocname{#2}
        \def\childdocjob{#2}
        \def\jobname{#2}
        \input{#2}
        \endinput
      }
    \else
      \def\childdoctmp
      {
        \childdocdisable
        \def\childdocname{#2}
        \childdoctrue
        \includeonly{#2}
        \def\childdocjob{#1}
        \def\jobname{#1}
        \input{#1}
        \endinput
      }
    \fi
    \expandafter
  \endgroup
  \childdoctmp
}
%    \end{macrocode}

% \macro{\childdocforwardprefix}
% The command |\childdocforwardprefix| redirects
% compilation to the main or a child file by means of a pattern.
% The prefix |#1| in the current filename is replaced by |#2|
% and the suffix of the current filename is kept
% (it is assumed that the filename does not contain the substring `|~~~|'
% which is used as a delimiter).
% Compilation is handed over to the new file by |\childdocforward|:
%    \begin{macrocode}
\newcommand{\childdocforwardprefix}[3][]
{
  \begingroup
    \def\childdocextract #2##1~~~{\def\childdoctmp{\childdocforward[#1]{#3##1}}}
    \expandafter\childdocextract\childdocname~~~
    \expandafter
  \endgroup
  \childdoctmp
}
%    \end{macrocode}

% \macro{\childdoc}
% The deprecated macro |\childdoc| is a legacy version of |\childdocmain|:
%    \begin{macrocode}
\newcommand{\childdoc}{\childdocmain}
%    \end{macrocode}

% \macro{\childdocredirect}
% The deprecated macro |\childdocredirect| is a legacy version
% of |\childdocforward| and |\childdocforwardprefix|:
%    \begin{macrocode}
\newcommand{\childdocredirect}[2][]
{
  \begingroup
    \if?#1?
      \def\childdoctmp{\childdocforward{#2}}
    \else
      \def\childdoctmp{\childdocforwardprefix{#1}{#2}}
    \fi
    \expandafter
  \endgroup
  \childdoctmp
}
%    \end{macrocode}

%\iffalse
%</package>
%\fi
%
\endinput
|\\
|\childdocforward{|\textit{main}|}|\\
\end{tabular}
\end{center}
%
or alternatively with:
%
\begin{center}
\begin{tabular}{l}
|% \iffalse
%
% childdoc.dtx Copyright (C) 2017-2018 Niklas Beisert
%
% This work may be distributed and/or modified under the
% conditions of the LaTeX Project Public License, either version 1.3
% of this license or (at your option) any later version.
% The latest version of this license is in
%   http://www.latex-project.org/lppl.txt
% and version 1.3 or later is part of all distributions of LaTeX
% version 2005/12/01 or later.
%
% This work has the LPPL maintenance status `maintained'.
%
% The Current Maintainer of this work is Niklas Beisert.
%
% This work consists of the files childdoc.dtx and childdoc.ins
% and the derived files childdoc.def and cdocsamp.tex with
% cdocsch1.tex, cdocsch2.tex, cdocsdrf.tex, cdocsfn1.tex, cdocsfn2.tex.
%
%<package>\ifdefined\childdocmain\endinput\fi
%<package>\ProvidesFile{childdoc.def}[2018/12/30 v2.0 child document driver]
%<samplemain>\ProvidesFile{cdocsamp.tex}[2018/12/30 v2.0 sample for childdoc]
%<*driver>
%\ProvidesFile{childdoc.drv}[2018/12/30 v2.0 childdoc reference manual file]
\PassOptionsToClass{10pt,a4paper}{article}
\documentclass{ltxdoc}

\usepackage[margin=35mm]{geometry}
\usepackage{hyperref}
\usepackage{hyperxmp}
\usepackage[usenames]{color}

\hypersetup{colorlinks=true}
\hypersetup{pdfstartview=FitH}
\hypersetup{pdfpagemode=UseNone}
\hypersetup{pdfsource={}}
\hypersetup{pdflang={en-UK}}
\hypersetup{pdfcopyright={Copyright 2017-2018 Niklas Beisert.
  This work may be distributed and/or modified under the
  conditions of the LaTeX Project Public License, either version 1.3
  of this license or (at your option) any later version.}}
\hypersetup{pdflicenseurl={http://www.latex-project.org/lppl.txt}}
\hypersetup{pdfcontactaddress={ETH Zurich, ITP, HIT K,
  Wolfgang-Pauli-Strasse 27}}
\hypersetup{pdfcontactpostcode={8093}}
\hypersetup{pdfcontactcity={Zurich}}
\hypersetup{pdfcontactcountry={Switzerland}}
\hypersetup{pdfcontactemail={nbeisert@itp.phys.ethz.ch}}
\hypersetup{pdfcontacturl={http://people.phys.ethz.ch/\xmptilde nbeisert/}}

\newcommand{\secref}[1]{\hyperref[#1]{section \ref*{#1}}}

\parskip1ex
\parindent0pt
\let\olditemize\itemize
\def\itemize{\olditemize\parskip0pt}

\begin{document}

\title{The \textsf{childdoc} Package}
\hypersetup{pdftitle={The childdoc Package}}
\author{Niklas Beisert\\[2ex]
  Institut f\"ur Theoretische Physik\\
  Eidgen\"ossische Technische Hochschule Z\"urich\\
  Wolfgang-Pauli-Strasse 27, 8093 Z\"urich, Switzerland\\[1ex]
  \href{mailto:nbeisert@itp.phys.ethz.ch}
  {\texttt{nbeisert@itp.phys.ethz.ch}}}
\hypersetup{pdfauthor={Niklas Beisert}}
\hypersetup{pdfsubject={Manual for the LaTeX2e Package childdoc}}
\date{30 December 2018, \textsf{v2.0}}
\maketitle

\begin{abstract}\noindent
\textsf{childdoc} is a \LaTeXe{} package
that enables the direct compilation
of document sections included by |\include|
to individual files.
\end{abstract}

\begingroup
\parskip0ex
\tableofcontents
\endgroup

%%%%%%%%%%%%%%%%%%%%%%%%%%%%%%%%%%%%%%%%%%%%%%%%%%%%%%%%%%%%%%%%%%%%%%%%%%%%%%%%
%%%%%%%%%%%%%%%%%%%%%%%%%%%%%%%%%%%%%%%%%%%%%%%%%%%%%%%%%%%%%%%%%%%%%%%%%%%%%%%%
\section{Introduction}

\LaTeX{} provides a mechanism to structure a large document (such as a book)
into a main file and several child files (containing the chapters)
using the |\include| command.
This mechanism is beneficial for documents
which span hundreds of pages in order to
make the source file(s) more manageable.
Moreover, compilation can be restricted to
selected child files by means of the |\includeonly| command.
The latter feature can be used to reduce the compilation time while editing
(this was significantly more useful in the earlier days of \LaTeX{})
or to generate a smaller document which is easier to navigate.
Another application of |\includeonly| is to generate
documents consisting of selected parts of the complete document.

However, there are a few drawbacks of the plain |\include| mechanism:
\begin{itemize}
\item
The child files cannot be compiled on their own,
they can only be compiled via the main file.
A naive editing environment
(such as a text editor with an option
to have the current file processed by \LaTeX)
may require one to switch to the main file before compiling;
attempting to compile the child file produces errors.
\item
The main file must be modified (each time)
to adjust the |\includeonly| command
to the present needs. This easily leaves the main file in a messy state.
\item
The generated document will always carry the filename
of the main document. This is inconvenient if
several child files are to be compiled and
to be kept for distribution.
\end{itemize}

The present package provides a simple interface
to make child files individually compilable by \LaTeX{}.
Compiling a child file then has the same effect as compiling
the main file with an |\includeonly| command
to select the appropriate child.
Moreover the generated document will carry the name of the child
rather than the main file.
This resolves all three above issues.

This feature is meant to make the editing of books,
thesis documents and lecture notes somewhat more convenient.
However, the package can also be used efficiently for
composing a series of documents (such as exercise sheets)
which are typically distributed individually.
It then assists the author in generating the individual documents
(potentially in different versions)
as well as a document containing the collected series.
Another application is in developing style files
or other kinds of included material
where compilation of the style file could redirect
to a sample or test file.

%%%%%%%%%%%%%%%%%%%%%%%%%%%%%%%%%%%%%%%%%%%%%%%%%%%%%%%%%%%%%%%%%%%%%%%%%%%%%%%%
%%%%%%%%%%%%%%%%%%%%%%%%%%%%%%%%%%%%%%%%%%%%%%%%%%%%%%%%%%%%%%%%%%%%%%%%%%%%%%%%
\section{Usage}

First of all, the package \textsf{childdoc} is \emph{not} a standard
\LaTeXe{} |.sty| style file! Therefore it needs to be invoked in
a non-standard way.

%%%%%%%%%%%%%%%%%%%%%%%%%%%%%%%%%%%%%%%%%%%%%%%%%%%%%%%%%%%%%%%%%%%%%%%%%%%%%%%%
\subsection{Included Files}
\label{sec:include}

%%%%%%%%%%%%%%%%%%%%%%%%%%%%%%%%%%%%%%%%
\DescribeMacro{\childdocmain}
To use the package, add the commands
\begin{center}
\begin{tabular}{l}
|% \iffalse
%
% childdoc.dtx Copyright (C) 2017-2018 Niklas Beisert
%
% This work may be distributed and/or modified under the
% conditions of the LaTeX Project Public License, either version 1.3
% of this license or (at your option) any later version.
% The latest version of this license is in
%   http://www.latex-project.org/lppl.txt
% and version 1.3 or later is part of all distributions of LaTeX
% version 2005/12/01 or later.
%
% This work has the LPPL maintenance status `maintained'.
%
% The Current Maintainer of this work is Niklas Beisert.
%
% This work consists of the files childdoc.dtx and childdoc.ins
% and the derived files childdoc.def and cdocsamp.tex with
% cdocsch1.tex, cdocsch2.tex, cdocsdrf.tex, cdocsfn1.tex, cdocsfn2.tex.
%
%<package>\ifdefined\childdocmain\endinput\fi
%<package>\ProvidesFile{childdoc.def}[2018/12/30 v2.0 child document driver]
%<samplemain>\ProvidesFile{cdocsamp.tex}[2018/12/30 v2.0 sample for childdoc]
%<*driver>
%\ProvidesFile{childdoc.drv}[2018/12/30 v2.0 childdoc reference manual file]
\PassOptionsToClass{10pt,a4paper}{article}
\documentclass{ltxdoc}

\usepackage[margin=35mm]{geometry}
\usepackage{hyperref}
\usepackage{hyperxmp}
\usepackage[usenames]{color}

\hypersetup{colorlinks=true}
\hypersetup{pdfstartview=FitH}
\hypersetup{pdfpagemode=UseNone}
\hypersetup{pdfsource={}}
\hypersetup{pdflang={en-UK}}
\hypersetup{pdfcopyright={Copyright 2017-2018 Niklas Beisert.
  This work may be distributed and/or modified under the
  conditions of the LaTeX Project Public License, either version 1.3
  of this license or (at your option) any later version.}}
\hypersetup{pdflicenseurl={http://www.latex-project.org/lppl.txt}}
\hypersetup{pdfcontactaddress={ETH Zurich, ITP, HIT K,
  Wolfgang-Pauli-Strasse 27}}
\hypersetup{pdfcontactpostcode={8093}}
\hypersetup{pdfcontactcity={Zurich}}
\hypersetup{pdfcontactcountry={Switzerland}}
\hypersetup{pdfcontactemail={nbeisert@itp.phys.ethz.ch}}
\hypersetup{pdfcontacturl={http://people.phys.ethz.ch/\xmptilde nbeisert/}}

\newcommand{\secref}[1]{\hyperref[#1]{section \ref*{#1}}}

\parskip1ex
\parindent0pt
\let\olditemize\itemize
\def\itemize{\olditemize\parskip0pt}

\begin{document}

\title{The \textsf{childdoc} Package}
\hypersetup{pdftitle={The childdoc Package}}
\author{Niklas Beisert\\[2ex]
  Institut f\"ur Theoretische Physik\\
  Eidgen\"ossische Technische Hochschule Z\"urich\\
  Wolfgang-Pauli-Strasse 27, 8093 Z\"urich, Switzerland\\[1ex]
  \href{mailto:nbeisert@itp.phys.ethz.ch}
  {\texttt{nbeisert@itp.phys.ethz.ch}}}
\hypersetup{pdfauthor={Niklas Beisert}}
\hypersetup{pdfsubject={Manual for the LaTeX2e Package childdoc}}
\date{30 December 2018, \textsf{v2.0}}
\maketitle

\begin{abstract}\noindent
\textsf{childdoc} is a \LaTeXe{} package
that enables the direct compilation
of document sections included by |\include|
to individual files.
\end{abstract}

\begingroup
\parskip0ex
\tableofcontents
\endgroup

%%%%%%%%%%%%%%%%%%%%%%%%%%%%%%%%%%%%%%%%%%%%%%%%%%%%%%%%%%%%%%%%%%%%%%%%%%%%%%%%
%%%%%%%%%%%%%%%%%%%%%%%%%%%%%%%%%%%%%%%%%%%%%%%%%%%%%%%%%%%%%%%%%%%%%%%%%%%%%%%%
\section{Introduction}

\LaTeX{} provides a mechanism to structure a large document (such as a book)
into a main file and several child files (containing the chapters)
using the |\include| command.
This mechanism is beneficial for documents
which span hundreds of pages in order to
make the source file(s) more manageable.
Moreover, compilation can be restricted to
selected child files by means of the |\includeonly| command.
The latter feature can be used to reduce the compilation time while editing
(this was significantly more useful in the earlier days of \LaTeX{})
or to generate a smaller document which is easier to navigate.
Another application of |\includeonly| is to generate
documents consisting of selected parts of the complete document.

However, there are a few drawbacks of the plain |\include| mechanism:
\begin{itemize}
\item
The child files cannot be compiled on their own,
they can only be compiled via the main file.
A naive editing environment
(such as a text editor with an option
to have the current file processed by \LaTeX)
may require one to switch to the main file before compiling;
attempting to compile the child file produces errors.
\item
The main file must be modified (each time)
to adjust the |\includeonly| command
to the present needs. This easily leaves the main file in a messy state.
\item
The generated document will always carry the filename
of the main document. This is inconvenient if
several child files are to be compiled and
to be kept for distribution.
\end{itemize}

The present package provides a simple interface
to make child files individually compilable by \LaTeX{}.
Compiling a child file then has the same effect as compiling
the main file with an |\includeonly| command
to select the appropriate child.
Moreover the generated document will carry the name of the child
rather than the main file.
This resolves all three above issues.

This feature is meant to make the editing of books,
thesis documents and lecture notes somewhat more convenient.
However, the package can also be used efficiently for
composing a series of documents (such as exercise sheets)
which are typically distributed individually.
It then assists the author in generating the individual documents
(potentially in different versions)
as well as a document containing the collected series.
Another application is in developing style files
or other kinds of included material
where compilation of the style file could redirect
to a sample or test file.

%%%%%%%%%%%%%%%%%%%%%%%%%%%%%%%%%%%%%%%%%%%%%%%%%%%%%%%%%%%%%%%%%%%%%%%%%%%%%%%%
%%%%%%%%%%%%%%%%%%%%%%%%%%%%%%%%%%%%%%%%%%%%%%%%%%%%%%%%%%%%%%%%%%%%%%%%%%%%%%%%
\section{Usage}

First of all, the package \textsf{childdoc} is \emph{not} a standard
\LaTeXe{} |.sty| style file! Therefore it needs to be invoked in
a non-standard way.

%%%%%%%%%%%%%%%%%%%%%%%%%%%%%%%%%%%%%%%%%%%%%%%%%%%%%%%%%%%%%%%%%%%%%%%%%%%%%%%%
\subsection{Included Files}
\label{sec:include}

%%%%%%%%%%%%%%%%%%%%%%%%%%%%%%%%%%%%%%%%
\DescribeMacro{\childdocmain}
To use the package, add the commands
\begin{center}
\begin{tabular}{l}
|\input{childdoc.def}|\\
|\childdocmain{}|\\
\end{tabular}
\end{center}
at the very top of the main \LaTeX{} file,
in particular \emph{before} the |\documentclass| statement!
The argument of |\childdocmain| should be left empty
(but it must be present).

%%%%%%%%%%%%%%%%%%%%%%%%%%%%%%%%%%%%%%%%
\DescribeMacro{\childdocof}
Furthermore, add the commands
\begin{center}
\begin{tabular}{l}
|\input{childdoc.def}|\\
|\childdocof{|\textit{main}|}|\\
\end{tabular}
\end{center}
at the top of every child file \textit{child}
which is included by |\include{|\textit{child}|}|
from within the main file
(or at least for those files to be compiled individually).
The argument \textit{main} must be the filename of the main file.

There are a couple of
considerations in setting up the main and child documents:

%%%%%%%%%%%%%%%%%%%%%%%%%%%%%%%%%%%%%%%%
\paragraph{Restrictions.}

Please note the following restrictions:
\begin{itemize}
\item
|\childdocmain| must be called with one argument \textit{main}
to ensure compatibility with earlier version of the package.
It must either be empty (|\childdocmain{}|)
or precisely match the filename of the main file in which it is specified.
See \secref{sec:detection} for further information.
\item
The filename \textit{main} must be specified without the |.tex| extension.
\item
The filename \textit{main} is case sensitive
(even in case-insensitive file systems)
due to internal string comparison.
\item
The argument \textit{main} should be fully expanded, it cannot be a macro.
\item
Subdirectories and special characters should be avoided in filenames.
\item
The command |\childdocmain{|\textit{main}|}| must be followed by a whitespace.
It should not be followed immediately by another command
or by a comment mark `|%|'.
This is because the \TeX{} parser reads the token immediately following
the argument of |\childdocmain| and puts it
at the beginning of every child section;
however, a white\-space is ignored.
\end{itemize}

%%%%%%%%%%%%%%%%%%%%%%%%%%%%%%%%%%%%%%%%
\paragraph{Content of Main File.}

It is advisable to place all content in the child files included by |\include|.
Any output contained in the main file will appear in all child documents
unless suppressed manually;
it cannot be suppressed automatically by the |\includeonly| directive
and thus should normally be avoided.
A method to include some content in the main file
by means of conditional processing is described in \secref{sec:conditional}.

%%%%%%%%%%%%%%%%%%%%%%%%%%%%%%%%%%%%%%%%
\paragraph{Page Numbering.}

When only a part of the document is compiled,
the appropriate numbering of pages
(as well as other status parameters)
is determined from the |.aux| files.
The latter contain information from previous passes.
However this information needs to propagate through
all intermediate child documents.
Therefore the page numbering in child documents may well
be inconsistent until the complete document is compiled at least once.

A useful (if unconventional) way to always ensure a consistent
page numbering is to restart the numbering in each child document
and denote the pages by `\textit{child}|.|\textit{page}'
where \textit{child} represents the chapter/section number of the child file.
This can be achieved by the command
|\numberwithin{page}{|\textit{child}|}|
of the \textsf{amsmath} package
where \textit{child} can be |chapter| or |section|
depending on the chosen structuring.
Alternatively, one can modify the macro |\thepage| appropriately
and reset the counter |page| at the start of each child file.

%%%%%%%%%%%%%%%%%%%%%%%%%%%%%%%%%%%%%%%%%%%%%%%%%%%%%%%%%%%%%%%%%%%%%%%%%%%%%%%%
\subsection{Conditional Processing}
\label{sec:conditional}

The package provides a mechanism to compile different versions
of a document. To customise the versions further some conditional processing
can come in handy to distinguish which version is being compiled.
The package provides two macros to describe the compilation context:

%%%%%%%%%%%%%%%%%%%%%%%%%%%%%%%%%%%%%%%%
\DescribeMacro{\ifchilddoc}
The conditional |\ifchilddoc| distinguishes between the compilation of
child documents and the main document:
%
\begin{center}
|\ifchilddoc |\textit{child-code}| |[|\||else |\textit{main-code}]| \||fi|
\end{center}

%%%%%%%%%%%%%%%%%%%%%%%%%%%%%%%%%%%%%%%%
\DescribeMacro{\childdocname}
\DescribeMacro{\childdocjob}
The macro |\childdocname| contains the filename (without extension)
of the main or child file being processed.
Note that |\childdocjob| will always contain the name of the main file.

%%%%%%%%%%%%%%%%%%%%%%%%%%%%%%%%%%%%%%%%
\paragraph{Title Page.}

Conditional processing can be used to include a title or banner page
in the main document when proper precautions are taken.
Importantly, the code in the main file should ensure that the page counter
(as well as other status parameters which are stored in the |.aux| files)
takes the same value after the conditional processing.
Otherwise the page numbers may take divergent values
depending on which part is compiled.

For example, a title page could be declared by:
%
\begin{center}
\begin{tabular}{l}
|\ifchilddoc\||else|\\
|\addtocounter{page}{-1}|\\
\textit{code for title page}\\
|\newpage|\\
|\||fi|
\end{tabular}
\end{center}
%
A banner page for the child documents can be generated by:
%
\begin{center}
\begin{tabular}{l}
|\ifchilddoc|\\
|\addtocounter{page}{-1}|\\
\textit{code for banner page}\\
|\newpage|\\
|\||fi|
\end{tabular}
\end{center}
%
Here one could write a message such as:
\begin{center}
|This is the part \childdocname{} of \childdocjob{}.|
\end{center}

%%%%%%%%%%%%%%%%%%%%%%%%%%%%%%%%%%%%%%%%%%%%%%%%%%%%%%%%%%%%%%%%%%%%%%%%%%%%%%%%
\subsection{Flags}
\label{sec:flags}

The package makes it easy to generate different versions
of the main or child documents.
To this end compilation flags can be defined
and assigned different default values.
They will be particularly useful in conjunction
with the forwarding mechanism described in \secref{sec:forward}.

For example, it may be useful to have a flag |\version|
which can be set to |draft| or |final|.
The document source will contain some conditional code
depending on the value of |\version|.
Suppose further, the flag should default to |final| for the main file
and to |draft| for child files
which is a natural assignment for editing the document.
This is achieved by placing the following code
in the preamble of the main document
(below the |\childdocmain| directive):
%
\begin{center}
\begin{tabular}{l}
|\ifchilddoc|\\
|\providecommand{\version}{draft}|\\
|\||else|\\
|\providecommand{\version}{final}|\\
|\||fi|
\end{tabular}
\end{center}
%
The definition by |\providecommand| makes sure
that previous definitions are not overwritten.
Further statements |\providecommand{\version}{...}|
can thus be added before the above code to override it.

For the main file, one might add a line
(between |\childdocmain| and the above block)
%
\begin{center}
|%\ifchilddoc\||else\providecommand{\version}{draft}\||fi|
\end{center}
%
which can be uncommented to produce a draft version.
Likewise one can add a line to the very top of a child file
(above the |\childdocof{|\textit{main}|}| directive)
%
\begin{center}
|%\providecommand{\version}{final}|
\end{center}
%
which can be uncommented to produce the final version of this child document.

%%%%%%%%%%%%%%%%%%%%%%%%%%%%%%%%%%%%%%%%%%%%%%%%%%%%%%%%%%%%%%%%%%%%%%%%%%%%%%%%
\subsection{Forwarding}
\label{sec:forward}

Different versions of the main or child documents
using compilation flags as described in \secref{sec:flags}
can be (permanently) stored in different files
for convenient compilation, viewing and distribution.
To this end, the package defines a command
to pass on compilation to a different file:

%%%%%%%%%%%%%%%%%%%%%%%%%%%%%%%%%%%%%%%%
\DescribeMacro{\childdocforward}
The command |\childdocforward| redirects processing to
another source file:
%
\begin{center}
\begin{tabular}{l}
|\input{childdoc.def}|\\
|\childdocforward[|\textit{main}|]{|\textit{dest}|}|\\
\end{tabular}
\end{center}
%
The argument \textit{dest} is the destination file
(without extension).
It should be the main file or one of the child files.
Note that further \textsf{childdoc} directives
such as |\childdocof| and |\childdocforward|
in the indicated file will be processed in this form.
The optional argument \textit{main}
passes on directly to the main file \textit{main}
while pretending to compile the child \textit{dest}.
This form behaves as if \textit{dest}
issues |\childdocof{|\textit{main}|}| right away,
and no further \textsf{childdoc} directives will be processed.

%%%%%%%%%%%%%%%%%%%%%%%%%%%%%%%%%%%%%%%%
\DescribeMacro{\...prefix}
In the alternative form |\childdocforwardprefix|,
%
\begin{center}
\begin{tabular}{l}
|\input{childdoc.def}|\\
|\childdocforwardprefix[|\textit{main}|]{|\textit{prefix}|}{|\textit{dest}|}|
\end{tabular}
\end{center}
%
the destination file is determined by a pattern
depending on the current file:
To make this work, the current file must be called
`{\textit{prefix}\hspace{0.2em}\textit{suffix}}'
with \textit{prefix} matching precisely the argument.
Processing is then passed on to the file
`{\textit{dest}\hspace{0.2em}\textit{suffix}}'.
Surely, the same effect is achieved by
directly specifying the
argument `{\textit{dest}\hspace{0.2em}\textit{suffix}}'
in the first form.
However, that requires to set up a different file
for each child. With the alternative form of the command
all these files can have exactly the same content
which simplifies setting them up and maintaining them.

For example, the following file |draft.tex|
with a compilation flag |\version| as described in \secref{sec:flags}
compiles the main document as a draft:
%
\begin{center}
\begin{tabular}{l}
|\def\version{draft}|\\
|\input{childdoc.def}|\\
|\childdocforward{|\textit{main}|}|
\end{tabular}
\end{center}
%
Likewise, the following files |final|\textit{nn}|.tex|
compile the final version of the child document
|child|\textit{nn}|.tex|:
%
\begin{center}
\begin{tabular}{l}
|\def\version{final}|\\
|\input{childdoc.def}|\\
|\childdocforwardprefix{final}{child}|
\end{tabular}
\end{center}
%

Note that when several versions of a main file and/or of each child file
are to be generated, it may be convenient to set up a |Makefile| or
shell script to automatise the process.

%%%%%%%%%%%%%%%%%%%%%%%%%%%%%%%%%%%%%%%%%%%%%%%%%%%%%%%%%%%%%%%%%%%%%%%%%%%%%%%%
\subsection{Command Line Processing}
\label{sec:commandline}

The effect of redirection files can also be achieved by invoking
the \LaTeX{} compiler with a more elaborate command line.
Most conveniently this should be done as part
of a shell script or a |Makefile|.

When using \textsf{childdoc} in the main file, the following
command lines effectively perform a redirection
(note that depending on the shell being used,
backslashes may have to be doubled: `|\|' $\to$ `|\\|'):
%
\begin{center}
|... -jobname "|\textit{target}|" |\\|"|[\textit{flags}]%
|\input{childdoc.def}\childdocforward[|\textit{main}|]{|\textit{dest}|}"|
\end{center}
%
Here \textit{target} is the name of the output file,
\textit{main} is the name of the main file
and \textit{dest} is the name of the main or child file to be processed
(all filenames without extensions).
The optional argument \textit{main} can be omitted
if \textit{main} matches \textit{dest}.
Optionally, compilation \textit{flags} can be defined via |\def| commands.
This command line makes the \TeX{} engine believe
it is compiling the file \textit{target}
whose content is specified as the latter parameter.
The provided code then forwards the processing to
\textit{main} or \textit{dest} as described in \secref{sec:forward}.

%%%%%%%%%%%%%%%%%%%%%%%%%%%%%%%%%%%%%%%%%%%%%%%%%%%%%%%%%%%%%%%%%%%%%%%%%%%%%%%%
\subsection{Include by Input}
\label{sec:input}

Including child documents by |\include| has some restrictions by design.
Most notably, the content of a child document always occupies
its own set of pages; pages cannot be shared between child documents.
Usually, this behaviour makes perfect sense
because each child document contain an essential part of the document.
However, in some situations it may be desirable to compose
a document from a collection of parts
without having mandatory page breaks between then.
For this case, the package
provides a mechanism to include parts
by |\input| which can also be processed individually.
However, by construction this mechanism
requires manual handling of the content to be output.

%%%%%%%%%%%%%%%%%%%%%%%%%%%%%%%%%%%%%%%%
\DescribeMacro{\ifchilddocmanual}
The main file should be prepared as usual, see \secref{sec:include}.
However, the document body must make a distinction
between processing of an individual part and of the main document, e.g.:
%
\begin{center}
\begin{tabular}{l}
|\ifchilddocmanual|\\
|\input{\childdocname}|\\
|\||else|\\
\textit{document body with }|\input{|\textit{part}|}|\\
|\||fi|
\end{tabular}
\end{center}
%
The conditional |\ifchilddocmanual| is true whenever
a part to be included by |\input| is being compiled,
and the name of the part is stored in |\childdocname|.

%%%%%%%%%%%%%%%%%%%%%%%%%%%%%%%%%%%%%%%%
\DescribeMacro{\childdocby}
Each part to be included by |\input| should start with:
%
\begin{center}
\begin{tabular}{l}
|\input{childdoc.def}|\\
|\childdocby{|\textit{main}|}|\\
\end{tabular}
\end{center}
%
The directive |\childdocby| is similar to |\childdocof|
described in \secref{sec:include},
but the subsequent selection of content must be done manually.
To that end, both |\ifchilddoc| and |\ifchilddocmanual|
will be true upon processing of a part,
and the name of the part is stored in |\childdocname|.
Note that |\jobname| will be set to the filename of the current part
so that each part receives an individual |.aux| file
that does not interfere with the |.aux| file(s) of the main document.
This behaviour can be altered by the alternative form
|\childdocby[*]{|\textit{main}|}| (with a non-empty optional argument)
which uses the |.aux| file of the main document
by setting |\jobname| to \textit{main}.

%%%%%%%%%%%%%%%%%%%%%%%%%%%%%%%%%%%%%%%%%%%%%%%%%%%%%%%%%%%%%%%%%%%%%%%%%%%%%%%%
\subsection{Driver Development}
\label{sec:driver}

The \textsf{childdoc} mechanism can also be use for the development
of definition files such as \LaTeX{} styles or classes.
This case differs from the above setup with multiple parts
included by |\include| in that no |\includeonly| should be invoked.
This can be achieved by starting the include file
(before |\ProvidesPackage|) with:
%
\begin{center}
\begin{tabular}{l}
|\input{childdoc.def}|\\
|\childdocforward{|\textit{main}|}|\\
\end{tabular}
\end{center}
%
or alternatively with:
%
\begin{center}
\begin{tabular}{l}
|\input{childdoc.def}|\\
|\childdocby{|\textit{main}|}|\\
\end{tabular}
\end{center}
%
Both forms have slightly different effects as described above.
The main file is prepared as usual, see \secref{sec:include}.

%%%%%%%%%%%%%%%%%%%%%%%%%%%%%%%%%%%%%%%%%%%%%%%%%%%%%%%%%%%%%%%%%%%%%%%%%%%%%%%%
\subsection{Legacy Detection}
\label{sec:detection}

The directive |\childdocmain| in the main file can detect
whether the complete document or merely a child is to be compiled
even without using the directive |\childdocof|.
This method is deprecated because it is less robust
and there is no compelling reason to use it;
it is merely provided for backward compatibility
and it may be removed in future versions.

If the detection mechanism is to be used,
it is mandatory to correctly specify
the filename of the main file as the argument of |\childdocmain|:
%
\begin{center}
\begin{tabular}{l}
|\input{childdoc.def}|\\
|\childdocmain{|\textit{main}|}|\\
\end{tabular}
\end{center}
%
If |\jobname| does not match the argument \textit{main} of |\childdocmain|,
it is assumed that |\jobname| points to the child file to be compiled.
When using |\childdocmain| with the main file specified as argument,
it suffices to start a child file
with just |\input{|\textit{main}|}|
without loading of the package and using |\childdocof|.
If instead all processing is done
with the appropriate \textsf{childdoc} directives,
the argument of \textit{main} of |\childdocmain| can be empty.

An alternative version of the command line processing described
in \secref{sec:commandline} using the detection mechanism reads:
%
\begin{center}
|... -jobname "|\textit{target}|" "|[\textit{flags}]%
[|\def\jobname{|\textit{dest}|}|]|\input{|\textit{main}|}"|
\end{center}

%%%%%%%%%%%%%%%%%%%%%%%%%%%%%%%%%%%%%%%%%%%%%%%%%%%%%%%%%%%%%%%%%%%%%%%%%%%%%%%%
\subsection{Manual Code}
\label{sec:manual}

In case one cannot be certain whether the definitions file |childdoc.def|
is installed on the target \TeX{} distribution
and one prefers not to ship it,
it is conceivable to paste a few relevant commands into the sources.

To that end, drop all statements |\input{childdoc.def}|
and perform the replacements as outlined below.
Instead of |\childdocmain{|\textit{main}|}| add the following code
to the top of the main file:
%
\begin{center}
\begin{tabular}{l}
|\||ifdefined\childdocname\endinput\||fi\newif\ifchilddoc|\\
|\edef\childdocname{\scantokens\expandafter{\jobname\noexpand}}|\\
|\def\childdocmain{|\textit{main}|}\||ifx\childdocmain\childdocname\||else|\\
|\childdoctrue\includeonly{\childdocname}\let\jobname\childdocmain\||fi|\\
\end{tabular}
\end{center}
%
Instead of |\childdocof{|\textit{main}|}| just include the main file
at the top of each child file:
%
\begin{center}
|\input{|\textit{main}|}|
\end{center}
%
A simple redirection |\childdocforward{|\textit{dest}|}| is achieved by:
%
\begin{center}
|\def\jobname{|\textit{dest}|}\input{\jobname}|
\end{center}
%
The redirection with prefix
|\childdocforwardprefix[|\textit{prefix}|]{|\textit{dest}|}|
is accomplished by:
%
\begin{center}
\begin{tabular}{l}
|{\edef\jobname{\scantokens\expandafter{\jobname\noexpand}}|\\
|\def\redirectjob |\textit{prefix}|#1~~~{\gdef\jobname{|\textit{dest}|#1}}|\\
|\expandafter\redirectjob\jobname~~~}\input{\jobname}|
\end{tabular}
\end{center}

In an alternative approach,
child documents can be compiled by a specific command line
without additional code or specific definitions:
%
\begin{center}
|... -jobname "|\textit{target}|" "|[\textit{flags}]%
|\includeonly{|\textit{dest}|}\input{|\textit{main}|}"|
\end{center}
%

%%%%%%%%%%%%%%%%%%%%%%%%%%%%%%%%%%%%%%%%%%%%%%%%%%%%%%%%%%%%%%%%%%%%%%%%%%%%%%%%
%%%%%%%%%%%%%%%%%%%%%%%%%%%%%%%%%%%%%%%%%%%%%%%%%%%%%%%%%%%%%%%%%%%%%%%%%%%%%%%%
\section{Information}

%%%%%%%%%%%%%%%%%%%%%%%%%%%%%%%%%%%%%%%%%%%%%%%%%%%%%%%%%%%%%%%%%%%%%%%%%%%%%%%%
\subsection{Copyright}

Copyright \copyright{} 2017--2018 Niklas Beisert

This work may be distributed and/or modified under the
conditions of the \LaTeX{} Project Public License, either version 1.3
of this license or (at your option) any later version.
The latest version of this license is in
  \url{http://www.latex-project.org/lppl.txt}
and version 1.3 or later is part of all distributions of \LaTeX{}
version 2005/12/01 or later.

This work has the LPPL maintenance status `maintained'.

The Current Maintainer of this work is Niklas Beisert.

This work consists of the files |README.txt|, |childdoc.ins| and |childdoc.dtx|
as well as the derived files |childdoc.def|, |cdocsamp.tex|
with |cdocsch1.tex|, |cdocsch2.tex|, |cdocspt3.tex|, |cdocspt4.tex|,
|cdocsdrf.tex|, |cdocsfn1.tex|, |cdocsfn2.tex|
as well as |childdoc.pdf|.

%%%%%%%%%%%%%%%%%%%%%%%%%%%%%%%%%%%%%%%%%%%%%%%%%%%%%%%%%%%%%%%%%%%%%%%%%%%%%%%%
\subsection{Files and Installation}

The package consists of the files:
%
\begin{center}
\begin{tabular}{ll}
    |README.txt|   & readme file \\
    |childdoc.ins| & installation file \\
    |childdoc.dtx| & source file \\
    |childdoc.def| & definition file \\
    |cdocsamp.tex| & sample main file \\
    |cdocsch1.tex| & sample include file \\
    |cdocsch2.tex| & sample include file \\
    |cdocspt3.tex| & sample part file \\
    |cdocspt4.tex| & sample part file \\
    |cdocsdrf.tex| & sample redirection file \\
    |cdocsfn1.tex| & sample redirection file \\
    |cdocsfn2.tex| & sample redirection file \\
    |childdoc.pdf| & manual
\end{tabular}
\end{center}
%
The distribution consists of the files
|README.txt|, |childdoc.ins| and |childdoc.dtx|.
%
\begin{itemize}
\item
Run (pdf)\LaTeX{} on |childdoc.dtx|
to compile the manual |childdoc.pdf| (this file).
\item
Run \LaTeX{} on |childdoc.ins| to create the definitions file |childdoc.def|
and the sample |cdocsamp.tex| with include files
|cdocsch1.tex|, |cdocsch2.tex|, |cdocspt3.tex|, |cdocspt4.tex|,
|cdocsdrf.tex|, |cdocsfn1.tex|, |cdocsfn2.tex|.
Then copy the file |childdoc.def| to an appropriate directory of your \LaTeX{}
distribution, e.g.\ \textit{texmf-root}|/tex/latex/childdoc|.
\end{itemize}

%%%%%%%%%%%%%%%%%%%%%%%%%%%%%%%%%%%%%%%%%%%%%%%%%%%%%%%%%%%%%%%%%%%%%%%%%%%%%%%%
\subsection{Related CTAN Packages}

There are several other packages which offer a similar functionality:
%
\begin{itemize}
\item
The packages
\href{http://ctan.org/pkg/docmute}{\textsf{docmute}},
\href{http://ctan.org/pkg/includex}{\textsf{includex}} and
\href{http://ctan.org/pkg/standalone}{\textsf{standalone}}
provide commands to include only the document body of
a child file thus allowing both files to be compiled individually.
\item
The packages \href{http://ctan.org/pkg/subdocs}{\textsf{subdocs}}
and \href{http://ctan.org/pkg/subfiles}{\textsf{subfiles}}
provide structures in which the main and child documents can be
encapsulated and allowing them to be compiled individually.
The inclusion mechanism is different from the conventional |\include|.
\item
The package \href{http://ctan.org/pkg/combine}{\textsf{combine}}
is an elaborate solution to combine several documents into one.
\end{itemize}
%
See also the CTAN topic \href{http://ctan.org/topic/subdocs}{\textsf{subdocs}}
for further related packages.
The present package differs from the above solutions in that
a document structure constructed with the conventional |\include| mechanism
just needs two extra commands at the top of every file
such that all constituent files can be compiled individually.

%%%%%%%%%%%%%%%%%%%%%%%%%%%%%%%%%%%%%%%%%%%%%%%%%%%%%%%%%%%%%%%%%%%%%%%%%%%%%%%%
%\subsection{Feature Suggestions}
%
%The following is a list of features which may be useful for future
%versions of this package:
%%
%\begin{itemize}
%\item
%\ldots
%\end{itemize}

%%%%%%%%%%%%%%%%%%%%%%%%%%%%%%%%%%%%%%%%%%%%%%%%%%%%%%%%%%%%%%%%%%%%%%%%%%%%%%%%
\subsection{Revision History}

%%%%%%%%%%%%%%%%%%%%%%%%%%%%%%%%%%%%%%%%
\paragraph{v2.0:} 2018/12/30

\begin{itemize}
\item
immediate forward processing
\item
added |\childdocby| mechanism
\item
manual restructured
\end{itemize}

%%%%%%%%%%%%%%%%%%%%%%%%%%%%%%%%%%%%%%%%
\paragraph{v1.6:} 2018/01/17

\begin{itemize}
\item
application for development of include files
\item
corrections to manual
\end{itemize}

%%%%%%%%%%%%%%%%%%%%%%%%%%%%%%%%%%%%%%%%
\paragraph{v1.5:} 2017/05/21

\begin{itemize}
\item
more complete structuring introduced
\item
|\childdocof| introduced
\item
|\childdoc| renamed to |\childdocmain|
\item
|\childredirect| renamed to |\childdocforward| and |\childdocforwardprefix|
and functionality expanded
\end{itemize}

%%%%%%%%%%%%%%%%%%%%%%%%%%%%%%%%%%%%%%%%
\paragraph{v1.0:} 2017/04/27

\begin{itemize}
\item
manual and install package
\item
first version published on CTAN
\end{itemize}

%%%%%%%%%%%%%%%%%%%%%%%%%%%%%%%%%%%%%%%%
\paragraph{v0.6:} 2017/04/26

\begin{itemize}
\item
redirection mechanism added
\end{itemize}

%%%%%%%%%%%%%%%%%%%%%%%%%%%%%%%%%%%%%%%%
\paragraph{v0.5:} 2017/04/26

\begin{itemize}
\item
functionality in definition file
\end{itemize}


%%%%%%%%%%%%%%%%%%%%%%%%%%%%%%%%%%%%%%%%%%%%%%%%%%%%%%%%%%%%%%%%%%%%%%%%%%%%%%%%
%%%%%%%%%%%%%%%%%%%%%%%%%%%%%%%%%%%%%%%%%%%%%%%%%%%%%%%%%%%%%%%%%%%%%%%%%%%%%%%%
%%%%%%%%%%%%%%%%%%%%%%%%%%%%%%%%%%%%%%%%%%%%%%%%%%%%%%%%%%%%%%%%%%%%%%%%%%%%%%%%
\appendix

\settowidth\MacroIndent{\rmfamily\scriptsize 000\ }

 \DocInput{childdoc.dtx}

\end{document}
%</driver>
% \fi
%
% %%%%%%%%%%%%%%%%%%%%%%%%%%%%%%%%%%%%%%%%%%%%%%%%%%%%%%%%%%%%%%%%%%%%%%%%%%%%%%
% %%%%%%%%%%%%%%%%%%%%%%%%%%%%%%%%%%%%%%%%%%%%%%%%%%%%%%%%%%%%%%%%%%%%%%%%%%%%%%
% \section{Sample}
%\iffalse
%<*samplemain>
%\fi
%
% The following presents a sample document
% with two chapters, two parts, a title page,
% a compile flag as well as three forwarding files to set the flag.
% It consists of eight |.tex| files:
% \begin{center}
% \begin{tabular}{ll}
% |cdocsamp.tex|&main file\\
% |cdocsch1.tex|&include file for chapter 1\\
% |cdocsch2.tex|&include file for chapter 2\\
% |cdocspt3.tex|&include file for part 3\\
% |cdocspt4.tex|&include file for part 4\\
% |cdocsdrf.tex|&forwarding file for main file in draft mode\\
% |cdocsfi1.tex|&forwarding file for final version of chapter 1\\
% |cdocsfi2.tex|&forwarding file for final version of chapter 2\\
% \end{tabular}
% \end{center}
% Each of the eight files can be compiled directly by the \LaTeX{} compiler.
%
% %%%%%%%%%%%%%%%%%%%%%%%%%%%%%%%%%%%%%%
% \paragraph{Main File.}
%
% The main file is called |cdocsamp.tex|.
%
% Load the \textsf{childdoc} definitions and
% declare the filename for the main document:
%    \begin{macrocode}
\input{childdoc.def}
\childdocmain{}
%    \end{macrocode}

% Optional override for |\version| flag:
%    \begin{macrocode}
%%\ifchilddoc\else\providecommand{\version}{draft}\fi
%    \end{macrocode}

% Define the default values for the |\version| flag
% (|final| for the main file and |draft| for childs):
%    \begin{macrocode}
\ifchilddoc
\providecommand{\version}{draft}
\else
\providecommand{\version}{final}
\fi
%    \end{macrocode}

% Load the standard document class:
%    \begin{macrocode}
\documentclass[12pt]{article}
%    \end{macrocode}

% Start the document body:
%    \begin{macrocode}
\begin{document}
%    \end{macrocode}

% Declare a title page.
% Print title, part of document being processed and version flag:
%    \begin{macrocode}
\addtocounter{page}{-1}
\begin{center}
{\LARGE\bfseries{}childdoc example\par}
\vspace{1cm}
\ifchilddoc
\ifchilddocmanual part\else chapter\fi:
`\childdocname' of `\childdocjob'\par
\else
main document: `\childdocjob'\par
\fi
version: \version\par
\end{center}
\newpage
%    \end{macrocode}

% Manually include selected file,
% otherwise process as usual:
%    \begin{macrocode}
\ifchilddocmanual
\section*{part `\childdocname'}
\input{\childdocname}
\else
%    \end{macrocode}

% Include the two chapters:
%    \begin{macrocode}
\include{cdocsch1}
\include{cdocsch2}
%    \end{macrocode}

% Include the two parts unless only chapters should be displayed:
%    \begin{macrocode}
\ifchilddoc\else
\section{part three}
\input{cdocspt3}
\section{part four}
\input{cdocspt4}
\fi
%    \end{macrocode}

% Process as usual until here:
%    \begin{macrocode}
\fi
%    \end{macrocode}

% End of document body:
%    \begin{macrocode}
\end{document}
%    \end{macrocode}
%\iffalse
%</samplemain>
%\fi
%
% %%%%%%%%%%%%%%%%%%%%%%%%%%%%%%%%%%%%%%
% \paragraph{Chapter Include Files.}
%
% The include files are called |cdocsch1.tex| and |cdocsch2.tex|.
%
%\iffalse
%<*samplechap1|samplechap2>
%\fi

% Optional override for |\version| flag:
%    \begin{macrocode}
%%\providecommand{\version}{final}
%    \end{macrocode}

% Include the main document:
%    \begin{macrocode}
\input{childdoc.def}
\childdocof{cdocsamp}
%    \end{macrocode}

%\iffalse
%</samplechap1|samplechap2>
%\fi
%
%\iffalse
%<*samplechap1>
%\fi
% Some text for chapter 1:
%    \begin{macrocode}
\section{one}
some text in chapter one
%    \end{macrocode}

%\iffalse
%</samplechap1>
%\fi
% Some text for chapter 2:
%\iffalse
%<*samplechap2>
%\fi
%    \begin{macrocode}
\section{two}
more text in chapter two
%    \end{macrocode}

%\iffalse
%</samplechap2>
%\fi
%
% %%%%%%%%%%%%%%%%%%%%%%%%%%%%%%%%%%%%%%
% \paragraph{Part Include Files.}
%
% The include files are called |cdocspt3.tex| and |cdocspt4.tex|.
%
%\iffalse
%<*samplepart3|samplepart4>
%\fi

% Optional override for |\version| flag:
%    \begin{macrocode}
%%\providecommand{\version}{final}
%    \end{macrocode}

% Include the main document:
%    \begin{macrocode}
\input{childdoc.def}
\childdocby{cdocsamp}
%    \end{macrocode}

%\iffalse
%</samplepart3|samplepart4>
%\fi
%
%\iffalse
%<*samplepart3>
%\fi
% Some text for part 3:
%    \begin{macrocode}
some text in part three
%    \end{macrocode}

%\iffalse
%</samplepart3>
%\fi
% Some text for part 4:
%\iffalse
%<*samplepart4>
%\fi
%    \begin{macrocode}
more text in part four
%    \end{macrocode}

%\iffalse
%</samplepart4>
%\fi
%
% %%%%%%%%%%%%%%%%%%%%%%%%%%%%%%%%%%%%%%
% \paragraph{Forwarding for a Complete Draft.}
%
% The following forwarding file |cdocsdrf.tex|
% compiles the main document in draft mode:
%\iffalse
%<*sampledraft>
%\fi
%    \begin{macrocode}
\def\version{draft}
\input{childdoc.def}
\childdocforward{cdocsamp}
%    \end{macrocode}

%\iffalse
%</sampledraft>
%\fi
%
% %%%%%%%%%%%%%%%%%%%%%%%%%%%%%%%%%%%%%%
% \paragraph{Forwarding for Final Version of the Chapters.}
%
% The following forwarding files |cdocsfn1.tex| and |cdocsfn2.tex|
% (with identical content)
% compile the final versions of the child documents
% |cdocsch1.tex| and |cdocsch2.tex|, respectively:
%\iffalse
%<*samplefinal>
%\fi
%    \begin{macrocode}
\def\version{final}
\input{childdoc.def}
\childdocforwardprefix[cdocsamp]{cdocsfn}{cdocsch}
%    \end{macrocode}

%\iffalse
%</samplefinal>
%\fi
%
% %%%%%%%%%%%%%%%%%%%%%%%%%%%%%%%%%%%%%%
% \paragraph{Command Line Processing.}
%
% The following three command lines generate the output files
% |cdocscld|, |cdocscl1| and |cdocscl2|
% which should be identical to
% |cdocsdrf|, |cdocsch1| and |cdocsfn2|, respectively:
% \begin{center}
% \begin{tabular}{l}
% |latex -jobname cdocscld \|\\
% |  "\def\version{draft}\input{childdoc.def}\childdocforward{cdocsamp}"|\\
% |latex -jobname cdocscl1 \|\\
% |  "\input{childdoc.def}\childdocforward[cdocsamp]{cdocsch1}"|\\
% |latex -jobname cdocscl2 \|\\
% |  "\def\version{final}\input{childdoc.def}\childdocforward{cdocsch2}"|
% \end{tabular}
% \end{center}
% Note that the trailing backslash on each first line
% merely continues the input to the second line
% (for convenient cut ant paste).
% Furthermore, the command |latex| can be replaced by any
% of its alternative versions such as |pdflatex|.
%
% %%%%%%%%%%%%%%%%%%%%%%%%%%%%%%%%%%%%%%%%%%%%%%%%%%%%%%%%%%%%%%%%%%%%%%%%%%%%%%
% %%%%%%%%%%%%%%%%%%%%%%%%%%%%%%%%%%%%%%%%%%%%%%%%%%%%%%%%%%%%%%%%%%%%%%%%%%%%%%
% \section{Implementation}
%\iffalse
%<*package>
%\fi
%
% This section describes the definitions file |childdoc.def|.

% The definitions cannot be loaded using |\usepackage| or |\RequirePackage|
% which has a mechanism to prevent loading a style file more than once.
% When loading the definitions by means of |\input|
% multiple instances have to be prevented manually:
%\iffalse
%This code needs to be before the `\ProvidesFile' directive
%which is defined at the beginning of this file.
%Therefore it is also placed there and commented out here.
%</package>
%<*discard>
%\fi
%    \begin{macrocode}
\ifdefined\childdocmain\endinput\fi
%    \end{macrocode}
%\iffalse
%</discard>
%<*package>
%\fi
%
% \macro{\ifchilddoc}
% \macro{\ifchilddocmanual}
% The conditional |\ifchilddoc| tells whether a
% child (true) or main (false) document is being compiled.
% The conditional |\ifchilddocmanual| tells whether
% the |\includeonly| mechanism is used (false) or
% the selection of child files must be performed manually (true).
% The definitions initialise to false:
%    \begin{macrocode}
\newif\ifchilddoc
\newif\ifchilddocmanual
%    \end{macrocode}

% \macro{\childdocname}
% \macro{\childdocjob}
% The macro |\childdocname| stores the name of the main document
% to be compiled. The macro |\childdocjob| stores the name of
% the document on which the \LaTeX{} compiler was originally invoked.
% The content of |\jobname| cannot be compared
% to filenames specified in the source due to different catcodes.
% The following code rescans |\jobname|, stores the result
% in |\childdocname| and saves a copy in |\childdocjob|:
%    \begin{macrocode}
\edef\childdocname{\scantokens\expandafter{\jobname\noexpand}}
\let\childdocjob\childdocname
%    \end{macrocode}

% \macro{\childdocdisable}
% The macro |\childdocdisable| prevents the main file
% from being processed more than once.
% At this stage, the main document command |\childdocmain|
% is assumed to be called once again where it should do nothing.
% Any subsequent call to it should prevent
% a secondary processing of the main document
% It overwrites the forwarding commands
% |\childdocof| and |\childdocforward|
% with empty macros to prevent further inclusions of the main document:
%    \begin{macrocode}
\newcommand{\childdocdisable}
{
  \renewcommand{\childdocmain}[1]{\renewcommand{\childdocmain}[1]{\endinput}}
  \renewcommand{\childdocof}[1]{}
  \renewcommand{\childdocby}[2][]{}
  \renewcommand{\childdocforward}[2][]{}
  \renewcommand{\childdocdisable}{}
}
%    \end{macrocode}

% \macro{\childdocmain}
% The macro |\childdocmain| is to be called at the top of the main file
% with nothing or the main filename (without extension) as argument.
% First, it breaks loops.
% If the argument is not empty and does not match |\childdocname|
% (which is set by the first inclusion of |childdoc.def|),
% |\ifchilddoc| is set to true, |\includeonly| is applied to the child file
% and |\jobname| is set to the main file
% (for proper handling of |.aux| files):
%    \begin{macrocode}
\newcommand{\childdocmain}[1]
{
  \childdocdisable\childdocmain{}
  \if?#1?\else
    \begingroup
      \def\childdoctmp{#1}
      \ifx\childdoctmp\childdocname
        \def\childdoctmp{}
      \else
        \def\childdoctmp
        {
          \childdoctrue
          \includeonly{\childdocname}
          \def\childdocjob{#1}
          \def\jobname{#1}
        }
      \fi
      \expandafter
    \endgroup
    \childdoctmp
  \fi
}
%    \end{macrocode}

% \macro{\childdocof}
% The command |\childdocof| redirects
% compilation to the main file |#1|.
%    \begin{macrocode}
\newcommand{\childdocof}[1]
{
  \childdocdisable
  \childdoctrue
  \includeonly{\childdocname}
  \def\jobname{#1}
  \def\childdocjob{#1}
  \input{#1}
}
%    \end{macrocode}

% \macro{\childdocby}
% The command |\childdocby| ....
%    \begin{macrocode}
\newcommand{\childdocby}[2][]
{
  \childdocdisable
  \childdoctrue
  \childdocmanualtrue
  \if?#1?\else
    \def\jobname{#2}
  \fi
  \def\childdocjob{#2}
  \input{#2}
  \endinput
}
%    \end{macrocode}

% \macro{\childdocforward}
% The command |\childdocforward| redirects
% compilation to the main file or
% (if the optional argument is given) a child file.
% Parameters are set as if the main file
% or a child file starting with |\childdocof| was compiled.
% Then compilation is handed over to the main file:
%    \begin{macrocode}
\newcommand{\childdocforward}[2][]
{
  \begingroup
    \if?#1?
      \def\childdoctmp
      {
        \def\childdocname{#2}
        \def\childdocjob{#2}
        \def\jobname{#2}
        \input{#2}
        \endinput
      }
    \else
      \def\childdoctmp
      {
        \childdocdisable
        \def\childdocname{#2}
        \childdoctrue
        \includeonly{#2}
        \def\childdocjob{#1}
        \def\jobname{#1}
        \input{#1}
        \endinput
      }
    \fi
    \expandafter
  \endgroup
  \childdoctmp
}
%    \end{macrocode}

% \macro{\childdocforwardprefix}
% The command |\childdocforwardprefix| redirects
% compilation to the main or a child file by means of a pattern.
% The prefix |#1| in the current filename is replaced by |#2|
% and the suffix of the current filename is kept
% (it is assumed that the filename does not contain the substring `|~~~|'
% which is used as a delimiter).
% Compilation is handed over to the new file by |\childdocforward|:
%    \begin{macrocode}
\newcommand{\childdocforwardprefix}[3][]
{
  \begingroup
    \def\childdocextract #2##1~~~{\def\childdoctmp{\childdocforward[#1]{#3##1}}}
    \expandafter\childdocextract\childdocname~~~
    \expandafter
  \endgroup
  \childdoctmp
}
%    \end{macrocode}

% \macro{\childdoc}
% The deprecated macro |\childdoc| is a legacy version of |\childdocmain|:
%    \begin{macrocode}
\newcommand{\childdoc}{\childdocmain}
%    \end{macrocode}

% \macro{\childdocredirect}
% The deprecated macro |\childdocredirect| is a legacy version
% of |\childdocforward| and |\childdocforwardprefix|:
%    \begin{macrocode}
\newcommand{\childdocredirect}[2][]
{
  \begingroup
    \if?#1?
      \def\childdoctmp{\childdocforward{#2}}
    \else
      \def\childdoctmp{\childdocforwardprefix{#1}{#2}}
    \fi
    \expandafter
  \endgroup
  \childdoctmp
}
%    \end{macrocode}

%\iffalse
%</package>
%\fi
%
\endinput
|\\
|\childdocmain{}|\\
\end{tabular}
\end{center}
at the very top of the main \LaTeX{} file,
in particular \emph{before} the |\documentclass| statement!
The argument of |\childdocmain| should be left empty
(but it must be present).

%%%%%%%%%%%%%%%%%%%%%%%%%%%%%%%%%%%%%%%%
\DescribeMacro{\childdocof}
Furthermore, add the commands
\begin{center}
\begin{tabular}{l}
|% \iffalse
%
% childdoc.dtx Copyright (C) 2017-2018 Niklas Beisert
%
% This work may be distributed and/or modified under the
% conditions of the LaTeX Project Public License, either version 1.3
% of this license or (at your option) any later version.
% The latest version of this license is in
%   http://www.latex-project.org/lppl.txt
% and version 1.3 or later is part of all distributions of LaTeX
% version 2005/12/01 or later.
%
% This work has the LPPL maintenance status `maintained'.
%
% The Current Maintainer of this work is Niklas Beisert.
%
% This work consists of the files childdoc.dtx and childdoc.ins
% and the derived files childdoc.def and cdocsamp.tex with
% cdocsch1.tex, cdocsch2.tex, cdocsdrf.tex, cdocsfn1.tex, cdocsfn2.tex.
%
%<package>\ifdefined\childdocmain\endinput\fi
%<package>\ProvidesFile{childdoc.def}[2018/12/30 v2.0 child document driver]
%<samplemain>\ProvidesFile{cdocsamp.tex}[2018/12/30 v2.0 sample for childdoc]
%<*driver>
%\ProvidesFile{childdoc.drv}[2018/12/30 v2.0 childdoc reference manual file]
\PassOptionsToClass{10pt,a4paper}{article}
\documentclass{ltxdoc}

\usepackage[margin=35mm]{geometry}
\usepackage{hyperref}
\usepackage{hyperxmp}
\usepackage[usenames]{color}

\hypersetup{colorlinks=true}
\hypersetup{pdfstartview=FitH}
\hypersetup{pdfpagemode=UseNone}
\hypersetup{pdfsource={}}
\hypersetup{pdflang={en-UK}}
\hypersetup{pdfcopyright={Copyright 2017-2018 Niklas Beisert.
  This work may be distributed and/or modified under the
  conditions of the LaTeX Project Public License, either version 1.3
  of this license or (at your option) any later version.}}
\hypersetup{pdflicenseurl={http://www.latex-project.org/lppl.txt}}
\hypersetup{pdfcontactaddress={ETH Zurich, ITP, HIT K,
  Wolfgang-Pauli-Strasse 27}}
\hypersetup{pdfcontactpostcode={8093}}
\hypersetup{pdfcontactcity={Zurich}}
\hypersetup{pdfcontactcountry={Switzerland}}
\hypersetup{pdfcontactemail={nbeisert@itp.phys.ethz.ch}}
\hypersetup{pdfcontacturl={http://people.phys.ethz.ch/\xmptilde nbeisert/}}

\newcommand{\secref}[1]{\hyperref[#1]{section \ref*{#1}}}

\parskip1ex
\parindent0pt
\let\olditemize\itemize
\def\itemize{\olditemize\parskip0pt}

\begin{document}

\title{The \textsf{childdoc} Package}
\hypersetup{pdftitle={The childdoc Package}}
\author{Niklas Beisert\\[2ex]
  Institut f\"ur Theoretische Physik\\
  Eidgen\"ossische Technische Hochschule Z\"urich\\
  Wolfgang-Pauli-Strasse 27, 8093 Z\"urich, Switzerland\\[1ex]
  \href{mailto:nbeisert@itp.phys.ethz.ch}
  {\texttt{nbeisert@itp.phys.ethz.ch}}}
\hypersetup{pdfauthor={Niklas Beisert}}
\hypersetup{pdfsubject={Manual for the LaTeX2e Package childdoc}}
\date{30 December 2018, \textsf{v2.0}}
\maketitle

\begin{abstract}\noindent
\textsf{childdoc} is a \LaTeXe{} package
that enables the direct compilation
of document sections included by |\include|
to individual files.
\end{abstract}

\begingroup
\parskip0ex
\tableofcontents
\endgroup

%%%%%%%%%%%%%%%%%%%%%%%%%%%%%%%%%%%%%%%%%%%%%%%%%%%%%%%%%%%%%%%%%%%%%%%%%%%%%%%%
%%%%%%%%%%%%%%%%%%%%%%%%%%%%%%%%%%%%%%%%%%%%%%%%%%%%%%%%%%%%%%%%%%%%%%%%%%%%%%%%
\section{Introduction}

\LaTeX{} provides a mechanism to structure a large document (such as a book)
into a main file and several child files (containing the chapters)
using the |\include| command.
This mechanism is beneficial for documents
which span hundreds of pages in order to
make the source file(s) more manageable.
Moreover, compilation can be restricted to
selected child files by means of the |\includeonly| command.
The latter feature can be used to reduce the compilation time while editing
(this was significantly more useful in the earlier days of \LaTeX{})
or to generate a smaller document which is easier to navigate.
Another application of |\includeonly| is to generate
documents consisting of selected parts of the complete document.

However, there are a few drawbacks of the plain |\include| mechanism:
\begin{itemize}
\item
The child files cannot be compiled on their own,
they can only be compiled via the main file.
A naive editing environment
(such as a text editor with an option
to have the current file processed by \LaTeX)
may require one to switch to the main file before compiling;
attempting to compile the child file produces errors.
\item
The main file must be modified (each time)
to adjust the |\includeonly| command
to the present needs. This easily leaves the main file in a messy state.
\item
The generated document will always carry the filename
of the main document. This is inconvenient if
several child files are to be compiled and
to be kept for distribution.
\end{itemize}

The present package provides a simple interface
to make child files individually compilable by \LaTeX{}.
Compiling a child file then has the same effect as compiling
the main file with an |\includeonly| command
to select the appropriate child.
Moreover the generated document will carry the name of the child
rather than the main file.
This resolves all three above issues.

This feature is meant to make the editing of books,
thesis documents and lecture notes somewhat more convenient.
However, the package can also be used efficiently for
composing a series of documents (such as exercise sheets)
which are typically distributed individually.
It then assists the author in generating the individual documents
(potentially in different versions)
as well as a document containing the collected series.
Another application is in developing style files
or other kinds of included material
where compilation of the style file could redirect
to a sample or test file.

%%%%%%%%%%%%%%%%%%%%%%%%%%%%%%%%%%%%%%%%%%%%%%%%%%%%%%%%%%%%%%%%%%%%%%%%%%%%%%%%
%%%%%%%%%%%%%%%%%%%%%%%%%%%%%%%%%%%%%%%%%%%%%%%%%%%%%%%%%%%%%%%%%%%%%%%%%%%%%%%%
\section{Usage}

First of all, the package \textsf{childdoc} is \emph{not} a standard
\LaTeXe{} |.sty| style file! Therefore it needs to be invoked in
a non-standard way.

%%%%%%%%%%%%%%%%%%%%%%%%%%%%%%%%%%%%%%%%%%%%%%%%%%%%%%%%%%%%%%%%%%%%%%%%%%%%%%%%
\subsection{Included Files}
\label{sec:include}

%%%%%%%%%%%%%%%%%%%%%%%%%%%%%%%%%%%%%%%%
\DescribeMacro{\childdocmain}
To use the package, add the commands
\begin{center}
\begin{tabular}{l}
|\input{childdoc.def}|\\
|\childdocmain{}|\\
\end{tabular}
\end{center}
at the very top of the main \LaTeX{} file,
in particular \emph{before} the |\documentclass| statement!
The argument of |\childdocmain| should be left empty
(but it must be present).

%%%%%%%%%%%%%%%%%%%%%%%%%%%%%%%%%%%%%%%%
\DescribeMacro{\childdocof}
Furthermore, add the commands
\begin{center}
\begin{tabular}{l}
|\input{childdoc.def}|\\
|\childdocof{|\textit{main}|}|\\
\end{tabular}
\end{center}
at the top of every child file \textit{child}
which is included by |\include{|\textit{child}|}|
from within the main file
(or at least for those files to be compiled individually).
The argument \textit{main} must be the filename of the main file.

There are a couple of
considerations in setting up the main and child documents:

%%%%%%%%%%%%%%%%%%%%%%%%%%%%%%%%%%%%%%%%
\paragraph{Restrictions.}

Please note the following restrictions:
\begin{itemize}
\item
|\childdocmain| must be called with one argument \textit{main}
to ensure compatibility with earlier version of the package.
It must either be empty (|\childdocmain{}|)
or precisely match the filename of the main file in which it is specified.
See \secref{sec:detection} for further information.
\item
The filename \textit{main} must be specified without the |.tex| extension.
\item
The filename \textit{main} is case sensitive
(even in case-insensitive file systems)
due to internal string comparison.
\item
The argument \textit{main} should be fully expanded, it cannot be a macro.
\item
Subdirectories and special characters should be avoided in filenames.
\item
The command |\childdocmain{|\textit{main}|}| must be followed by a whitespace.
It should not be followed immediately by another command
or by a comment mark `|%|'.
This is because the \TeX{} parser reads the token immediately following
the argument of |\childdocmain| and puts it
at the beginning of every child section;
however, a white\-space is ignored.
\end{itemize}

%%%%%%%%%%%%%%%%%%%%%%%%%%%%%%%%%%%%%%%%
\paragraph{Content of Main File.}

It is advisable to place all content in the child files included by |\include|.
Any output contained in the main file will appear in all child documents
unless suppressed manually;
it cannot be suppressed automatically by the |\includeonly| directive
and thus should normally be avoided.
A method to include some content in the main file
by means of conditional processing is described in \secref{sec:conditional}.

%%%%%%%%%%%%%%%%%%%%%%%%%%%%%%%%%%%%%%%%
\paragraph{Page Numbering.}

When only a part of the document is compiled,
the appropriate numbering of pages
(as well as other status parameters)
is determined from the |.aux| files.
The latter contain information from previous passes.
However this information needs to propagate through
all intermediate child documents.
Therefore the page numbering in child documents may well
be inconsistent until the complete document is compiled at least once.

A useful (if unconventional) way to always ensure a consistent
page numbering is to restart the numbering in each child document
and denote the pages by `\textit{child}|.|\textit{page}'
where \textit{child} represents the chapter/section number of the child file.
This can be achieved by the command
|\numberwithin{page}{|\textit{child}|}|
of the \textsf{amsmath} package
where \textit{child} can be |chapter| or |section|
depending on the chosen structuring.
Alternatively, one can modify the macro |\thepage| appropriately
and reset the counter |page| at the start of each child file.

%%%%%%%%%%%%%%%%%%%%%%%%%%%%%%%%%%%%%%%%%%%%%%%%%%%%%%%%%%%%%%%%%%%%%%%%%%%%%%%%
\subsection{Conditional Processing}
\label{sec:conditional}

The package provides a mechanism to compile different versions
of a document. To customise the versions further some conditional processing
can come in handy to distinguish which version is being compiled.
The package provides two macros to describe the compilation context:

%%%%%%%%%%%%%%%%%%%%%%%%%%%%%%%%%%%%%%%%
\DescribeMacro{\ifchilddoc}
The conditional |\ifchilddoc| distinguishes between the compilation of
child documents and the main document:
%
\begin{center}
|\ifchilddoc |\textit{child-code}| |[|\||else |\textit{main-code}]| \||fi|
\end{center}

%%%%%%%%%%%%%%%%%%%%%%%%%%%%%%%%%%%%%%%%
\DescribeMacro{\childdocname}
\DescribeMacro{\childdocjob}
The macro |\childdocname| contains the filename (without extension)
of the main or child file being processed.
Note that |\childdocjob| will always contain the name of the main file.

%%%%%%%%%%%%%%%%%%%%%%%%%%%%%%%%%%%%%%%%
\paragraph{Title Page.}

Conditional processing can be used to include a title or banner page
in the main document when proper precautions are taken.
Importantly, the code in the main file should ensure that the page counter
(as well as other status parameters which are stored in the |.aux| files)
takes the same value after the conditional processing.
Otherwise the page numbers may take divergent values
depending on which part is compiled.

For example, a title page could be declared by:
%
\begin{center}
\begin{tabular}{l}
|\ifchilddoc\||else|\\
|\addtocounter{page}{-1}|\\
\textit{code for title page}\\
|\newpage|\\
|\||fi|
\end{tabular}
\end{center}
%
A banner page for the child documents can be generated by:
%
\begin{center}
\begin{tabular}{l}
|\ifchilddoc|\\
|\addtocounter{page}{-1}|\\
\textit{code for banner page}\\
|\newpage|\\
|\||fi|
\end{tabular}
\end{center}
%
Here one could write a message such as:
\begin{center}
|This is the part \childdocname{} of \childdocjob{}.|
\end{center}

%%%%%%%%%%%%%%%%%%%%%%%%%%%%%%%%%%%%%%%%%%%%%%%%%%%%%%%%%%%%%%%%%%%%%%%%%%%%%%%%
\subsection{Flags}
\label{sec:flags}

The package makes it easy to generate different versions
of the main or child documents.
To this end compilation flags can be defined
and assigned different default values.
They will be particularly useful in conjunction
with the forwarding mechanism described in \secref{sec:forward}.

For example, it may be useful to have a flag |\version|
which can be set to |draft| or |final|.
The document source will contain some conditional code
depending on the value of |\version|.
Suppose further, the flag should default to |final| for the main file
and to |draft| for child files
which is a natural assignment for editing the document.
This is achieved by placing the following code
in the preamble of the main document
(below the |\childdocmain| directive):
%
\begin{center}
\begin{tabular}{l}
|\ifchilddoc|\\
|\providecommand{\version}{draft}|\\
|\||else|\\
|\providecommand{\version}{final}|\\
|\||fi|
\end{tabular}
\end{center}
%
The definition by |\providecommand| makes sure
that previous definitions are not overwritten.
Further statements |\providecommand{\version}{...}|
can thus be added before the above code to override it.

For the main file, one might add a line
(between |\childdocmain| and the above block)
%
\begin{center}
|%\ifchilddoc\||else\providecommand{\version}{draft}\||fi|
\end{center}
%
which can be uncommented to produce a draft version.
Likewise one can add a line to the very top of a child file
(above the |\childdocof{|\textit{main}|}| directive)
%
\begin{center}
|%\providecommand{\version}{final}|
\end{center}
%
which can be uncommented to produce the final version of this child document.

%%%%%%%%%%%%%%%%%%%%%%%%%%%%%%%%%%%%%%%%%%%%%%%%%%%%%%%%%%%%%%%%%%%%%%%%%%%%%%%%
\subsection{Forwarding}
\label{sec:forward}

Different versions of the main or child documents
using compilation flags as described in \secref{sec:flags}
can be (permanently) stored in different files
for convenient compilation, viewing and distribution.
To this end, the package defines a command
to pass on compilation to a different file:

%%%%%%%%%%%%%%%%%%%%%%%%%%%%%%%%%%%%%%%%
\DescribeMacro{\childdocforward}
The command |\childdocforward| redirects processing to
another source file:
%
\begin{center}
\begin{tabular}{l}
|\input{childdoc.def}|\\
|\childdocforward[|\textit{main}|]{|\textit{dest}|}|\\
\end{tabular}
\end{center}
%
The argument \textit{dest} is the destination file
(without extension).
It should be the main file or one of the child files.
Note that further \textsf{childdoc} directives
such as |\childdocof| and |\childdocforward|
in the indicated file will be processed in this form.
The optional argument \textit{main}
passes on directly to the main file \textit{main}
while pretending to compile the child \textit{dest}.
This form behaves as if \textit{dest}
issues |\childdocof{|\textit{main}|}| right away,
and no further \textsf{childdoc} directives will be processed.

%%%%%%%%%%%%%%%%%%%%%%%%%%%%%%%%%%%%%%%%
\DescribeMacro{\...prefix}
In the alternative form |\childdocforwardprefix|,
%
\begin{center}
\begin{tabular}{l}
|\input{childdoc.def}|\\
|\childdocforwardprefix[|\textit{main}|]{|\textit{prefix}|}{|\textit{dest}|}|
\end{tabular}
\end{center}
%
the destination file is determined by a pattern
depending on the current file:
To make this work, the current file must be called
`{\textit{prefix}\hspace{0.2em}\textit{suffix}}'
with \textit{prefix} matching precisely the argument.
Processing is then passed on to the file
`{\textit{dest}\hspace{0.2em}\textit{suffix}}'.
Surely, the same effect is achieved by
directly specifying the
argument `{\textit{dest}\hspace{0.2em}\textit{suffix}}'
in the first form.
However, that requires to set up a different file
for each child. With the alternative form of the command
all these files can have exactly the same content
which simplifies setting them up and maintaining them.

For example, the following file |draft.tex|
with a compilation flag |\version| as described in \secref{sec:flags}
compiles the main document as a draft:
%
\begin{center}
\begin{tabular}{l}
|\def\version{draft}|\\
|\input{childdoc.def}|\\
|\childdocforward{|\textit{main}|}|
\end{tabular}
\end{center}
%
Likewise, the following files |final|\textit{nn}|.tex|
compile the final version of the child document
|child|\textit{nn}|.tex|:
%
\begin{center}
\begin{tabular}{l}
|\def\version{final}|\\
|\input{childdoc.def}|\\
|\childdocforwardprefix{final}{child}|
\end{tabular}
\end{center}
%

Note that when several versions of a main file and/or of each child file
are to be generated, it may be convenient to set up a |Makefile| or
shell script to automatise the process.

%%%%%%%%%%%%%%%%%%%%%%%%%%%%%%%%%%%%%%%%%%%%%%%%%%%%%%%%%%%%%%%%%%%%%%%%%%%%%%%%
\subsection{Command Line Processing}
\label{sec:commandline}

The effect of redirection files can also be achieved by invoking
the \LaTeX{} compiler with a more elaborate command line.
Most conveniently this should be done as part
of a shell script or a |Makefile|.

When using \textsf{childdoc} in the main file, the following
command lines effectively perform a redirection
(note that depending on the shell being used,
backslashes may have to be doubled: `|\|' $\to$ `|\\|'):
%
\begin{center}
|... -jobname "|\textit{target}|" |\\|"|[\textit{flags}]%
|\input{childdoc.def}\childdocforward[|\textit{main}|]{|\textit{dest}|}"|
\end{center}
%
Here \textit{target} is the name of the output file,
\textit{main} is the name of the main file
and \textit{dest} is the name of the main or child file to be processed
(all filenames without extensions).
The optional argument \textit{main} can be omitted
if \textit{main} matches \textit{dest}.
Optionally, compilation \textit{flags} can be defined via |\def| commands.
This command line makes the \TeX{} engine believe
it is compiling the file \textit{target}
whose content is specified as the latter parameter.
The provided code then forwards the processing to
\textit{main} or \textit{dest} as described in \secref{sec:forward}.

%%%%%%%%%%%%%%%%%%%%%%%%%%%%%%%%%%%%%%%%%%%%%%%%%%%%%%%%%%%%%%%%%%%%%%%%%%%%%%%%
\subsection{Include by Input}
\label{sec:input}

Including child documents by |\include| has some restrictions by design.
Most notably, the content of a child document always occupies
its own set of pages; pages cannot be shared between child documents.
Usually, this behaviour makes perfect sense
because each child document contain an essential part of the document.
However, in some situations it may be desirable to compose
a document from a collection of parts
without having mandatory page breaks between then.
For this case, the package
provides a mechanism to include parts
by |\input| which can also be processed individually.
However, by construction this mechanism
requires manual handling of the content to be output.

%%%%%%%%%%%%%%%%%%%%%%%%%%%%%%%%%%%%%%%%
\DescribeMacro{\ifchilddocmanual}
The main file should be prepared as usual, see \secref{sec:include}.
However, the document body must make a distinction
between processing of an individual part and of the main document, e.g.:
%
\begin{center}
\begin{tabular}{l}
|\ifchilddocmanual|\\
|\input{\childdocname}|\\
|\||else|\\
\textit{document body with }|\input{|\textit{part}|}|\\
|\||fi|
\end{tabular}
\end{center}
%
The conditional |\ifchilddocmanual| is true whenever
a part to be included by |\input| is being compiled,
and the name of the part is stored in |\childdocname|.

%%%%%%%%%%%%%%%%%%%%%%%%%%%%%%%%%%%%%%%%
\DescribeMacro{\childdocby}
Each part to be included by |\input| should start with:
%
\begin{center}
\begin{tabular}{l}
|\input{childdoc.def}|\\
|\childdocby{|\textit{main}|}|\\
\end{tabular}
\end{center}
%
The directive |\childdocby| is similar to |\childdocof|
described in \secref{sec:include},
but the subsequent selection of content must be done manually.
To that end, both |\ifchilddoc| and |\ifchilddocmanual|
will be true upon processing of a part,
and the name of the part is stored in |\childdocname|.
Note that |\jobname| will be set to the filename of the current part
so that each part receives an individual |.aux| file
that does not interfere with the |.aux| file(s) of the main document.
This behaviour can be altered by the alternative form
|\childdocby[*]{|\textit{main}|}| (with a non-empty optional argument)
which uses the |.aux| file of the main document
by setting |\jobname| to \textit{main}.

%%%%%%%%%%%%%%%%%%%%%%%%%%%%%%%%%%%%%%%%%%%%%%%%%%%%%%%%%%%%%%%%%%%%%%%%%%%%%%%%
\subsection{Driver Development}
\label{sec:driver}

The \textsf{childdoc} mechanism can also be use for the development
of definition files such as \LaTeX{} styles or classes.
This case differs from the above setup with multiple parts
included by |\include| in that no |\includeonly| should be invoked.
This can be achieved by starting the include file
(before |\ProvidesPackage|) with:
%
\begin{center}
\begin{tabular}{l}
|\input{childdoc.def}|\\
|\childdocforward{|\textit{main}|}|\\
\end{tabular}
\end{center}
%
or alternatively with:
%
\begin{center}
\begin{tabular}{l}
|\input{childdoc.def}|\\
|\childdocby{|\textit{main}|}|\\
\end{tabular}
\end{center}
%
Both forms have slightly different effects as described above.
The main file is prepared as usual, see \secref{sec:include}.

%%%%%%%%%%%%%%%%%%%%%%%%%%%%%%%%%%%%%%%%%%%%%%%%%%%%%%%%%%%%%%%%%%%%%%%%%%%%%%%%
\subsection{Legacy Detection}
\label{sec:detection}

The directive |\childdocmain| in the main file can detect
whether the complete document or merely a child is to be compiled
even without using the directive |\childdocof|.
This method is deprecated because it is less robust
and there is no compelling reason to use it;
it is merely provided for backward compatibility
and it may be removed in future versions.

If the detection mechanism is to be used,
it is mandatory to correctly specify
the filename of the main file as the argument of |\childdocmain|:
%
\begin{center}
\begin{tabular}{l}
|\input{childdoc.def}|\\
|\childdocmain{|\textit{main}|}|\\
\end{tabular}
\end{center}
%
If |\jobname| does not match the argument \textit{main} of |\childdocmain|,
it is assumed that |\jobname| points to the child file to be compiled.
When using |\childdocmain| with the main file specified as argument,
it suffices to start a child file
with just |\input{|\textit{main}|}|
without loading of the package and using |\childdocof|.
If instead all processing is done
with the appropriate \textsf{childdoc} directives,
the argument of \textit{main} of |\childdocmain| can be empty.

An alternative version of the command line processing described
in \secref{sec:commandline} using the detection mechanism reads:
%
\begin{center}
|... -jobname "|\textit{target}|" "|[\textit{flags}]%
[|\def\jobname{|\textit{dest}|}|]|\input{|\textit{main}|}"|
\end{center}

%%%%%%%%%%%%%%%%%%%%%%%%%%%%%%%%%%%%%%%%%%%%%%%%%%%%%%%%%%%%%%%%%%%%%%%%%%%%%%%%
\subsection{Manual Code}
\label{sec:manual}

In case one cannot be certain whether the definitions file |childdoc.def|
is installed on the target \TeX{} distribution
and one prefers not to ship it,
it is conceivable to paste a few relevant commands into the sources.

To that end, drop all statements |\input{childdoc.def}|
and perform the replacements as outlined below.
Instead of |\childdocmain{|\textit{main}|}| add the following code
to the top of the main file:
%
\begin{center}
\begin{tabular}{l}
|\||ifdefined\childdocname\endinput\||fi\newif\ifchilddoc|\\
|\edef\childdocname{\scantokens\expandafter{\jobname\noexpand}}|\\
|\def\childdocmain{|\textit{main}|}\||ifx\childdocmain\childdocname\||else|\\
|\childdoctrue\includeonly{\childdocname}\let\jobname\childdocmain\||fi|\\
\end{tabular}
\end{center}
%
Instead of |\childdocof{|\textit{main}|}| just include the main file
at the top of each child file:
%
\begin{center}
|\input{|\textit{main}|}|
\end{center}
%
A simple redirection |\childdocforward{|\textit{dest}|}| is achieved by:
%
\begin{center}
|\def\jobname{|\textit{dest}|}\input{\jobname}|
\end{center}
%
The redirection with prefix
|\childdocforwardprefix[|\textit{prefix}|]{|\textit{dest}|}|
is accomplished by:
%
\begin{center}
\begin{tabular}{l}
|{\edef\jobname{\scantokens\expandafter{\jobname\noexpand}}|\\
|\def\redirectjob |\textit{prefix}|#1~~~{\gdef\jobname{|\textit{dest}|#1}}|\\
|\expandafter\redirectjob\jobname~~~}\input{\jobname}|
\end{tabular}
\end{center}

In an alternative approach,
child documents can be compiled by a specific command line
without additional code or specific definitions:
%
\begin{center}
|... -jobname "|\textit{target}|" "|[\textit{flags}]%
|\includeonly{|\textit{dest}|}\input{|\textit{main}|}"|
\end{center}
%

%%%%%%%%%%%%%%%%%%%%%%%%%%%%%%%%%%%%%%%%%%%%%%%%%%%%%%%%%%%%%%%%%%%%%%%%%%%%%%%%
%%%%%%%%%%%%%%%%%%%%%%%%%%%%%%%%%%%%%%%%%%%%%%%%%%%%%%%%%%%%%%%%%%%%%%%%%%%%%%%%
\section{Information}

%%%%%%%%%%%%%%%%%%%%%%%%%%%%%%%%%%%%%%%%%%%%%%%%%%%%%%%%%%%%%%%%%%%%%%%%%%%%%%%%
\subsection{Copyright}

Copyright \copyright{} 2017--2018 Niklas Beisert

This work may be distributed and/or modified under the
conditions of the \LaTeX{} Project Public License, either version 1.3
of this license or (at your option) any later version.
The latest version of this license is in
  \url{http://www.latex-project.org/lppl.txt}
and version 1.3 or later is part of all distributions of \LaTeX{}
version 2005/12/01 or later.

This work has the LPPL maintenance status `maintained'.

The Current Maintainer of this work is Niklas Beisert.

This work consists of the files |README.txt|, |childdoc.ins| and |childdoc.dtx|
as well as the derived files |childdoc.def|, |cdocsamp.tex|
with |cdocsch1.tex|, |cdocsch2.tex|, |cdocspt3.tex|, |cdocspt4.tex|,
|cdocsdrf.tex|, |cdocsfn1.tex|, |cdocsfn2.tex|
as well as |childdoc.pdf|.

%%%%%%%%%%%%%%%%%%%%%%%%%%%%%%%%%%%%%%%%%%%%%%%%%%%%%%%%%%%%%%%%%%%%%%%%%%%%%%%%
\subsection{Files and Installation}

The package consists of the files:
%
\begin{center}
\begin{tabular}{ll}
    |README.txt|   & readme file \\
    |childdoc.ins| & installation file \\
    |childdoc.dtx| & source file \\
    |childdoc.def| & definition file \\
    |cdocsamp.tex| & sample main file \\
    |cdocsch1.tex| & sample include file \\
    |cdocsch2.tex| & sample include file \\
    |cdocspt3.tex| & sample part file \\
    |cdocspt4.tex| & sample part file \\
    |cdocsdrf.tex| & sample redirection file \\
    |cdocsfn1.tex| & sample redirection file \\
    |cdocsfn2.tex| & sample redirection file \\
    |childdoc.pdf| & manual
\end{tabular}
\end{center}
%
The distribution consists of the files
|README.txt|, |childdoc.ins| and |childdoc.dtx|.
%
\begin{itemize}
\item
Run (pdf)\LaTeX{} on |childdoc.dtx|
to compile the manual |childdoc.pdf| (this file).
\item
Run \LaTeX{} on |childdoc.ins| to create the definitions file |childdoc.def|
and the sample |cdocsamp.tex| with include files
|cdocsch1.tex|, |cdocsch2.tex|, |cdocspt3.tex|, |cdocspt4.tex|,
|cdocsdrf.tex|, |cdocsfn1.tex|, |cdocsfn2.tex|.
Then copy the file |childdoc.def| to an appropriate directory of your \LaTeX{}
distribution, e.g.\ \textit{texmf-root}|/tex/latex/childdoc|.
\end{itemize}

%%%%%%%%%%%%%%%%%%%%%%%%%%%%%%%%%%%%%%%%%%%%%%%%%%%%%%%%%%%%%%%%%%%%%%%%%%%%%%%%
\subsection{Related CTAN Packages}

There are several other packages which offer a similar functionality:
%
\begin{itemize}
\item
The packages
\href{http://ctan.org/pkg/docmute}{\textsf{docmute}},
\href{http://ctan.org/pkg/includex}{\textsf{includex}} and
\href{http://ctan.org/pkg/standalone}{\textsf{standalone}}
provide commands to include only the document body of
a child file thus allowing both files to be compiled individually.
\item
The packages \href{http://ctan.org/pkg/subdocs}{\textsf{subdocs}}
and \href{http://ctan.org/pkg/subfiles}{\textsf{subfiles}}
provide structures in which the main and child documents can be
encapsulated and allowing them to be compiled individually.
The inclusion mechanism is different from the conventional |\include|.
\item
The package \href{http://ctan.org/pkg/combine}{\textsf{combine}}
is an elaborate solution to combine several documents into one.
\end{itemize}
%
See also the CTAN topic \href{http://ctan.org/topic/subdocs}{\textsf{subdocs}}
for further related packages.
The present package differs from the above solutions in that
a document structure constructed with the conventional |\include| mechanism
just needs two extra commands at the top of every file
such that all constituent files can be compiled individually.

%%%%%%%%%%%%%%%%%%%%%%%%%%%%%%%%%%%%%%%%%%%%%%%%%%%%%%%%%%%%%%%%%%%%%%%%%%%%%%%%
%\subsection{Feature Suggestions}
%
%The following is a list of features which may be useful for future
%versions of this package:
%%
%\begin{itemize}
%\item
%\ldots
%\end{itemize}

%%%%%%%%%%%%%%%%%%%%%%%%%%%%%%%%%%%%%%%%%%%%%%%%%%%%%%%%%%%%%%%%%%%%%%%%%%%%%%%%
\subsection{Revision History}

%%%%%%%%%%%%%%%%%%%%%%%%%%%%%%%%%%%%%%%%
\paragraph{v2.0:} 2018/12/30

\begin{itemize}
\item
immediate forward processing
\item
added |\childdocby| mechanism
\item
manual restructured
\end{itemize}

%%%%%%%%%%%%%%%%%%%%%%%%%%%%%%%%%%%%%%%%
\paragraph{v1.6:} 2018/01/17

\begin{itemize}
\item
application for development of include files
\item
corrections to manual
\end{itemize}

%%%%%%%%%%%%%%%%%%%%%%%%%%%%%%%%%%%%%%%%
\paragraph{v1.5:} 2017/05/21

\begin{itemize}
\item
more complete structuring introduced
\item
|\childdocof| introduced
\item
|\childdoc| renamed to |\childdocmain|
\item
|\childredirect| renamed to |\childdocforward| and |\childdocforwardprefix|
and functionality expanded
\end{itemize}

%%%%%%%%%%%%%%%%%%%%%%%%%%%%%%%%%%%%%%%%
\paragraph{v1.0:} 2017/04/27

\begin{itemize}
\item
manual and install package
\item
first version published on CTAN
\end{itemize}

%%%%%%%%%%%%%%%%%%%%%%%%%%%%%%%%%%%%%%%%
\paragraph{v0.6:} 2017/04/26

\begin{itemize}
\item
redirection mechanism added
\end{itemize}

%%%%%%%%%%%%%%%%%%%%%%%%%%%%%%%%%%%%%%%%
\paragraph{v0.5:} 2017/04/26

\begin{itemize}
\item
functionality in definition file
\end{itemize}


%%%%%%%%%%%%%%%%%%%%%%%%%%%%%%%%%%%%%%%%%%%%%%%%%%%%%%%%%%%%%%%%%%%%%%%%%%%%%%%%
%%%%%%%%%%%%%%%%%%%%%%%%%%%%%%%%%%%%%%%%%%%%%%%%%%%%%%%%%%%%%%%%%%%%%%%%%%%%%%%%
%%%%%%%%%%%%%%%%%%%%%%%%%%%%%%%%%%%%%%%%%%%%%%%%%%%%%%%%%%%%%%%%%%%%%%%%%%%%%%%%
\appendix

\settowidth\MacroIndent{\rmfamily\scriptsize 000\ }

 \DocInput{childdoc.dtx}

\end{document}
%</driver>
% \fi
%
% %%%%%%%%%%%%%%%%%%%%%%%%%%%%%%%%%%%%%%%%%%%%%%%%%%%%%%%%%%%%%%%%%%%%%%%%%%%%%%
% %%%%%%%%%%%%%%%%%%%%%%%%%%%%%%%%%%%%%%%%%%%%%%%%%%%%%%%%%%%%%%%%%%%%%%%%%%%%%%
% \section{Sample}
%\iffalse
%<*samplemain>
%\fi
%
% The following presents a sample document
% with two chapters, two parts, a title page,
% a compile flag as well as three forwarding files to set the flag.
% It consists of eight |.tex| files:
% \begin{center}
% \begin{tabular}{ll}
% |cdocsamp.tex|&main file\\
% |cdocsch1.tex|&include file for chapter 1\\
% |cdocsch2.tex|&include file for chapter 2\\
% |cdocspt3.tex|&include file for part 3\\
% |cdocspt4.tex|&include file for part 4\\
% |cdocsdrf.tex|&forwarding file for main file in draft mode\\
% |cdocsfi1.tex|&forwarding file for final version of chapter 1\\
% |cdocsfi2.tex|&forwarding file for final version of chapter 2\\
% \end{tabular}
% \end{center}
% Each of the eight files can be compiled directly by the \LaTeX{} compiler.
%
% %%%%%%%%%%%%%%%%%%%%%%%%%%%%%%%%%%%%%%
% \paragraph{Main File.}
%
% The main file is called |cdocsamp.tex|.
%
% Load the \textsf{childdoc} definitions and
% declare the filename for the main document:
%    \begin{macrocode}
\input{childdoc.def}
\childdocmain{}
%    \end{macrocode}

% Optional override for |\version| flag:
%    \begin{macrocode}
%%\ifchilddoc\else\providecommand{\version}{draft}\fi
%    \end{macrocode}

% Define the default values for the |\version| flag
% (|final| for the main file and |draft| for childs):
%    \begin{macrocode}
\ifchilddoc
\providecommand{\version}{draft}
\else
\providecommand{\version}{final}
\fi
%    \end{macrocode}

% Load the standard document class:
%    \begin{macrocode}
\documentclass[12pt]{article}
%    \end{macrocode}

% Start the document body:
%    \begin{macrocode}
\begin{document}
%    \end{macrocode}

% Declare a title page.
% Print title, part of document being processed and version flag:
%    \begin{macrocode}
\addtocounter{page}{-1}
\begin{center}
{\LARGE\bfseries{}childdoc example\par}
\vspace{1cm}
\ifchilddoc
\ifchilddocmanual part\else chapter\fi:
`\childdocname' of `\childdocjob'\par
\else
main document: `\childdocjob'\par
\fi
version: \version\par
\end{center}
\newpage
%    \end{macrocode}

% Manually include selected file,
% otherwise process as usual:
%    \begin{macrocode}
\ifchilddocmanual
\section*{part `\childdocname'}
\input{\childdocname}
\else
%    \end{macrocode}

% Include the two chapters:
%    \begin{macrocode}
\include{cdocsch1}
\include{cdocsch2}
%    \end{macrocode}

% Include the two parts unless only chapters should be displayed:
%    \begin{macrocode}
\ifchilddoc\else
\section{part three}
\input{cdocspt3}
\section{part four}
\input{cdocspt4}
\fi
%    \end{macrocode}

% Process as usual until here:
%    \begin{macrocode}
\fi
%    \end{macrocode}

% End of document body:
%    \begin{macrocode}
\end{document}
%    \end{macrocode}
%\iffalse
%</samplemain>
%\fi
%
% %%%%%%%%%%%%%%%%%%%%%%%%%%%%%%%%%%%%%%
% \paragraph{Chapter Include Files.}
%
% The include files are called |cdocsch1.tex| and |cdocsch2.tex|.
%
%\iffalse
%<*samplechap1|samplechap2>
%\fi

% Optional override for |\version| flag:
%    \begin{macrocode}
%%\providecommand{\version}{final}
%    \end{macrocode}

% Include the main document:
%    \begin{macrocode}
\input{childdoc.def}
\childdocof{cdocsamp}
%    \end{macrocode}

%\iffalse
%</samplechap1|samplechap2>
%\fi
%
%\iffalse
%<*samplechap1>
%\fi
% Some text for chapter 1:
%    \begin{macrocode}
\section{one}
some text in chapter one
%    \end{macrocode}

%\iffalse
%</samplechap1>
%\fi
% Some text for chapter 2:
%\iffalse
%<*samplechap2>
%\fi
%    \begin{macrocode}
\section{two}
more text in chapter two
%    \end{macrocode}

%\iffalse
%</samplechap2>
%\fi
%
% %%%%%%%%%%%%%%%%%%%%%%%%%%%%%%%%%%%%%%
% \paragraph{Part Include Files.}
%
% The include files are called |cdocspt3.tex| and |cdocspt4.tex|.
%
%\iffalse
%<*samplepart3|samplepart4>
%\fi

% Optional override for |\version| flag:
%    \begin{macrocode}
%%\providecommand{\version}{final}
%    \end{macrocode}

% Include the main document:
%    \begin{macrocode}
\input{childdoc.def}
\childdocby{cdocsamp}
%    \end{macrocode}

%\iffalse
%</samplepart3|samplepart4>
%\fi
%
%\iffalse
%<*samplepart3>
%\fi
% Some text for part 3:
%    \begin{macrocode}
some text in part three
%    \end{macrocode}

%\iffalse
%</samplepart3>
%\fi
% Some text for part 4:
%\iffalse
%<*samplepart4>
%\fi
%    \begin{macrocode}
more text in part four
%    \end{macrocode}

%\iffalse
%</samplepart4>
%\fi
%
% %%%%%%%%%%%%%%%%%%%%%%%%%%%%%%%%%%%%%%
% \paragraph{Forwarding for a Complete Draft.}
%
% The following forwarding file |cdocsdrf.tex|
% compiles the main document in draft mode:
%\iffalse
%<*sampledraft>
%\fi
%    \begin{macrocode}
\def\version{draft}
\input{childdoc.def}
\childdocforward{cdocsamp}
%    \end{macrocode}

%\iffalse
%</sampledraft>
%\fi
%
% %%%%%%%%%%%%%%%%%%%%%%%%%%%%%%%%%%%%%%
% \paragraph{Forwarding for Final Version of the Chapters.}
%
% The following forwarding files |cdocsfn1.tex| and |cdocsfn2.tex|
% (with identical content)
% compile the final versions of the child documents
% |cdocsch1.tex| and |cdocsch2.tex|, respectively:
%\iffalse
%<*samplefinal>
%\fi
%    \begin{macrocode}
\def\version{final}
\input{childdoc.def}
\childdocforwardprefix[cdocsamp]{cdocsfn}{cdocsch}
%    \end{macrocode}

%\iffalse
%</samplefinal>
%\fi
%
% %%%%%%%%%%%%%%%%%%%%%%%%%%%%%%%%%%%%%%
% \paragraph{Command Line Processing.}
%
% The following three command lines generate the output files
% |cdocscld|, |cdocscl1| and |cdocscl2|
% which should be identical to
% |cdocsdrf|, |cdocsch1| and |cdocsfn2|, respectively:
% \begin{center}
% \begin{tabular}{l}
% |latex -jobname cdocscld \|\\
% |  "\def\version{draft}\input{childdoc.def}\childdocforward{cdocsamp}"|\\
% |latex -jobname cdocscl1 \|\\
% |  "\input{childdoc.def}\childdocforward[cdocsamp]{cdocsch1}"|\\
% |latex -jobname cdocscl2 \|\\
% |  "\def\version{final}\input{childdoc.def}\childdocforward{cdocsch2}"|
% \end{tabular}
% \end{center}
% Note that the trailing backslash on each first line
% merely continues the input to the second line
% (for convenient cut ant paste).
% Furthermore, the command |latex| can be replaced by any
% of its alternative versions such as |pdflatex|.
%
% %%%%%%%%%%%%%%%%%%%%%%%%%%%%%%%%%%%%%%%%%%%%%%%%%%%%%%%%%%%%%%%%%%%%%%%%%%%%%%
% %%%%%%%%%%%%%%%%%%%%%%%%%%%%%%%%%%%%%%%%%%%%%%%%%%%%%%%%%%%%%%%%%%%%%%%%%%%%%%
% \section{Implementation}
%\iffalse
%<*package>
%\fi
%
% This section describes the definitions file |childdoc.def|.

% The definitions cannot be loaded using |\usepackage| or |\RequirePackage|
% which has a mechanism to prevent loading a style file more than once.
% When loading the definitions by means of |\input|
% multiple instances have to be prevented manually:
%\iffalse
%This code needs to be before the `\ProvidesFile' directive
%which is defined at the beginning of this file.
%Therefore it is also placed there and commented out here.
%</package>
%<*discard>
%\fi
%    \begin{macrocode}
\ifdefined\childdocmain\endinput\fi
%    \end{macrocode}
%\iffalse
%</discard>
%<*package>
%\fi
%
% \macro{\ifchilddoc}
% \macro{\ifchilddocmanual}
% The conditional |\ifchilddoc| tells whether a
% child (true) or main (false) document is being compiled.
% The conditional |\ifchilddocmanual| tells whether
% the |\includeonly| mechanism is used (false) or
% the selection of child files must be performed manually (true).
% The definitions initialise to false:
%    \begin{macrocode}
\newif\ifchilddoc
\newif\ifchilddocmanual
%    \end{macrocode}

% \macro{\childdocname}
% \macro{\childdocjob}
% The macro |\childdocname| stores the name of the main document
% to be compiled. The macro |\childdocjob| stores the name of
% the document on which the \LaTeX{} compiler was originally invoked.
% The content of |\jobname| cannot be compared
% to filenames specified in the source due to different catcodes.
% The following code rescans |\jobname|, stores the result
% in |\childdocname| and saves a copy in |\childdocjob|:
%    \begin{macrocode}
\edef\childdocname{\scantokens\expandafter{\jobname\noexpand}}
\let\childdocjob\childdocname
%    \end{macrocode}

% \macro{\childdocdisable}
% The macro |\childdocdisable| prevents the main file
% from being processed more than once.
% At this stage, the main document command |\childdocmain|
% is assumed to be called once again where it should do nothing.
% Any subsequent call to it should prevent
% a secondary processing of the main document
% It overwrites the forwarding commands
% |\childdocof| and |\childdocforward|
% with empty macros to prevent further inclusions of the main document:
%    \begin{macrocode}
\newcommand{\childdocdisable}
{
  \renewcommand{\childdocmain}[1]{\renewcommand{\childdocmain}[1]{\endinput}}
  \renewcommand{\childdocof}[1]{}
  \renewcommand{\childdocby}[2][]{}
  \renewcommand{\childdocforward}[2][]{}
  \renewcommand{\childdocdisable}{}
}
%    \end{macrocode}

% \macro{\childdocmain}
% The macro |\childdocmain| is to be called at the top of the main file
% with nothing or the main filename (without extension) as argument.
% First, it breaks loops.
% If the argument is not empty and does not match |\childdocname|
% (which is set by the first inclusion of |childdoc.def|),
% |\ifchilddoc| is set to true, |\includeonly| is applied to the child file
% and |\jobname| is set to the main file
% (for proper handling of |.aux| files):
%    \begin{macrocode}
\newcommand{\childdocmain}[1]
{
  \childdocdisable\childdocmain{}
  \if?#1?\else
    \begingroup
      \def\childdoctmp{#1}
      \ifx\childdoctmp\childdocname
        \def\childdoctmp{}
      \else
        \def\childdoctmp
        {
          \childdoctrue
          \includeonly{\childdocname}
          \def\childdocjob{#1}
          \def\jobname{#1}
        }
      \fi
      \expandafter
    \endgroup
    \childdoctmp
  \fi
}
%    \end{macrocode}

% \macro{\childdocof}
% The command |\childdocof| redirects
% compilation to the main file |#1|.
%    \begin{macrocode}
\newcommand{\childdocof}[1]
{
  \childdocdisable
  \childdoctrue
  \includeonly{\childdocname}
  \def\jobname{#1}
  \def\childdocjob{#1}
  \input{#1}
}
%    \end{macrocode}

% \macro{\childdocby}
% The command |\childdocby| ....
%    \begin{macrocode}
\newcommand{\childdocby}[2][]
{
  \childdocdisable
  \childdoctrue
  \childdocmanualtrue
  \if?#1?\else
    \def\jobname{#2}
  \fi
  \def\childdocjob{#2}
  \input{#2}
  \endinput
}
%    \end{macrocode}

% \macro{\childdocforward}
% The command |\childdocforward| redirects
% compilation to the main file or
% (if the optional argument is given) a child file.
% Parameters are set as if the main file
% or a child file starting with |\childdocof| was compiled.
% Then compilation is handed over to the main file:
%    \begin{macrocode}
\newcommand{\childdocforward}[2][]
{
  \begingroup
    \if?#1?
      \def\childdoctmp
      {
        \def\childdocname{#2}
        \def\childdocjob{#2}
        \def\jobname{#2}
        \input{#2}
        \endinput
      }
    \else
      \def\childdoctmp
      {
        \childdocdisable
        \def\childdocname{#2}
        \childdoctrue
        \includeonly{#2}
        \def\childdocjob{#1}
        \def\jobname{#1}
        \input{#1}
        \endinput
      }
    \fi
    \expandafter
  \endgroup
  \childdoctmp
}
%    \end{macrocode}

% \macro{\childdocforwardprefix}
% The command |\childdocforwardprefix| redirects
% compilation to the main or a child file by means of a pattern.
% The prefix |#1| in the current filename is replaced by |#2|
% and the suffix of the current filename is kept
% (it is assumed that the filename does not contain the substring `|~~~|'
% which is used as a delimiter).
% Compilation is handed over to the new file by |\childdocforward|:
%    \begin{macrocode}
\newcommand{\childdocforwardprefix}[3][]
{
  \begingroup
    \def\childdocextract #2##1~~~{\def\childdoctmp{\childdocforward[#1]{#3##1}}}
    \expandafter\childdocextract\childdocname~~~
    \expandafter
  \endgroup
  \childdoctmp
}
%    \end{macrocode}

% \macro{\childdoc}
% The deprecated macro |\childdoc| is a legacy version of |\childdocmain|:
%    \begin{macrocode}
\newcommand{\childdoc}{\childdocmain}
%    \end{macrocode}

% \macro{\childdocredirect}
% The deprecated macro |\childdocredirect| is a legacy version
% of |\childdocforward| and |\childdocforwardprefix|:
%    \begin{macrocode}
\newcommand{\childdocredirect}[2][]
{
  \begingroup
    \if?#1?
      \def\childdoctmp{\childdocforward{#2}}
    \else
      \def\childdoctmp{\childdocforwardprefix{#1}{#2}}
    \fi
    \expandafter
  \endgroup
  \childdoctmp
}
%    \end{macrocode}

%\iffalse
%</package>
%\fi
%
\endinput
|\\
|\childdocof{|\textit{main}|}|\\
\end{tabular}
\end{center}
at the top of every child file \textit{child}
which is included by |\include{|\textit{child}|}|
from within the main file
(or at least for those files to be compiled individually).
The argument \textit{main} must be the filename of the main file.

There are a couple of
considerations in setting up the main and child documents:

%%%%%%%%%%%%%%%%%%%%%%%%%%%%%%%%%%%%%%%%
\paragraph{Restrictions.}

Please note the following restrictions:
\begin{itemize}
\item
|\childdocmain| must be called with one argument \textit{main}
to ensure compatibility with earlier version of the package.
It must either be empty (|\childdocmain{}|)
or precisely match the filename of the main file in which it is specified.
See \secref{sec:detection} for further information.
\item
The filename \textit{main} must be specified without the |.tex| extension.
\item
The filename \textit{main} is case sensitive
(even in case-insensitive file systems)
due to internal string comparison.
\item
The argument \textit{main} should be fully expanded, it cannot be a macro.
\item
Subdirectories and special characters should be avoided in filenames.
\item
The command |\childdocmain{|\textit{main}|}| must be followed by a whitespace.
It should not be followed immediately by another command
or by a comment mark `|%|'.
This is because the \TeX{} parser reads the token immediately following
the argument of |\childdocmain| and puts it
at the beginning of every child section;
however, a white\-space is ignored.
\end{itemize}

%%%%%%%%%%%%%%%%%%%%%%%%%%%%%%%%%%%%%%%%
\paragraph{Content of Main File.}

It is advisable to place all content in the child files included by |\include|.
Any output contained in the main file will appear in all child documents
unless suppressed manually;
it cannot be suppressed automatically by the |\includeonly| directive
and thus should normally be avoided.
A method to include some content in the main file
by means of conditional processing is described in \secref{sec:conditional}.

%%%%%%%%%%%%%%%%%%%%%%%%%%%%%%%%%%%%%%%%
\paragraph{Page Numbering.}

When only a part of the document is compiled,
the appropriate numbering of pages
(as well as other status parameters)
is determined from the |.aux| files.
The latter contain information from previous passes.
However this information needs to propagate through
all intermediate child documents.
Therefore the page numbering in child documents may well
be inconsistent until the complete document is compiled at least once.

A useful (if unconventional) way to always ensure a consistent
page numbering is to restart the numbering in each child document
and denote the pages by `\textit{child}|.|\textit{page}'
where \textit{child} represents the chapter/section number of the child file.
This can be achieved by the command
|\numberwithin{page}{|\textit{child}|}|
of the \textsf{amsmath} package
where \textit{child} can be |chapter| or |section|
depending on the chosen structuring.
Alternatively, one can modify the macro |\thepage| appropriately
and reset the counter |page| at the start of each child file.

%%%%%%%%%%%%%%%%%%%%%%%%%%%%%%%%%%%%%%%%%%%%%%%%%%%%%%%%%%%%%%%%%%%%%%%%%%%%%%%%
\subsection{Conditional Processing}
\label{sec:conditional}

The package provides a mechanism to compile different versions
of a document. To customise the versions further some conditional processing
can come in handy to distinguish which version is being compiled.
The package provides two macros to describe the compilation context:

%%%%%%%%%%%%%%%%%%%%%%%%%%%%%%%%%%%%%%%%
\DescribeMacro{\ifchilddoc}
The conditional |\ifchilddoc| distinguishes between the compilation of
child documents and the main document:
%
\begin{center}
|\ifchilddoc |\textit{child-code}| |[|\||else |\textit{main-code}]| \||fi|
\end{center}

%%%%%%%%%%%%%%%%%%%%%%%%%%%%%%%%%%%%%%%%
\DescribeMacro{\childdocname}
\DescribeMacro{\childdocjob}
The macro |\childdocname| contains the filename (without extension)
of the main or child file being processed.
Note that |\childdocjob| will always contain the name of the main file.

%%%%%%%%%%%%%%%%%%%%%%%%%%%%%%%%%%%%%%%%
\paragraph{Title Page.}

Conditional processing can be used to include a title or banner page
in the main document when proper precautions are taken.
Importantly, the code in the main file should ensure that the page counter
(as well as other status parameters which are stored in the |.aux| files)
takes the same value after the conditional processing.
Otherwise the page numbers may take divergent values
depending on which part is compiled.

For example, a title page could be declared by:
%
\begin{center}
\begin{tabular}{l}
|\ifchilddoc\||else|\\
|\addtocounter{page}{-1}|\\
\textit{code for title page}\\
|\newpage|\\
|\||fi|
\end{tabular}
\end{center}
%
A banner page for the child documents can be generated by:
%
\begin{center}
\begin{tabular}{l}
|\ifchilddoc|\\
|\addtocounter{page}{-1}|\\
\textit{code for banner page}\\
|\newpage|\\
|\||fi|
\end{tabular}
\end{center}
%
Here one could write a message such as:
\begin{center}
|This is the part \childdocname{} of \childdocjob{}.|
\end{center}

%%%%%%%%%%%%%%%%%%%%%%%%%%%%%%%%%%%%%%%%%%%%%%%%%%%%%%%%%%%%%%%%%%%%%%%%%%%%%%%%
\subsection{Flags}
\label{sec:flags}

The package makes it easy to generate different versions
of the main or child documents.
To this end compilation flags can be defined
and assigned different default values.
They will be particularly useful in conjunction
with the forwarding mechanism described in \secref{sec:forward}.

For example, it may be useful to have a flag |\version|
which can be set to |draft| or |final|.
The document source will contain some conditional code
depending on the value of |\version|.
Suppose further, the flag should default to |final| for the main file
and to |draft| for child files
which is a natural assignment for editing the document.
This is achieved by placing the following code
in the preamble of the main document
(below the |\childdocmain| directive):
%
\begin{center}
\begin{tabular}{l}
|\ifchilddoc|\\
|\providecommand{\version}{draft}|\\
|\||else|\\
|\providecommand{\version}{final}|\\
|\||fi|
\end{tabular}
\end{center}
%
The definition by |\providecommand| makes sure
that previous definitions are not overwritten.
Further statements |\providecommand{\version}{...}|
can thus be added before the above code to override it.

For the main file, one might add a line
(between |\childdocmain| and the above block)
%
\begin{center}
|%\ifchilddoc\||else\providecommand{\version}{draft}\||fi|
\end{center}
%
which can be uncommented to produce a draft version.
Likewise one can add a line to the very top of a child file
(above the |\childdocof{|\textit{main}|}| directive)
%
\begin{center}
|%\providecommand{\version}{final}|
\end{center}
%
which can be uncommented to produce the final version of this child document.

%%%%%%%%%%%%%%%%%%%%%%%%%%%%%%%%%%%%%%%%%%%%%%%%%%%%%%%%%%%%%%%%%%%%%%%%%%%%%%%%
\subsection{Forwarding}
\label{sec:forward}

Different versions of the main or child documents
using compilation flags as described in \secref{sec:flags}
can be (permanently) stored in different files
for convenient compilation, viewing and distribution.
To this end, the package defines a command
to pass on compilation to a different file:

%%%%%%%%%%%%%%%%%%%%%%%%%%%%%%%%%%%%%%%%
\DescribeMacro{\childdocforward}
The command |\childdocforward| redirects processing to
another source file:
%
\begin{center}
\begin{tabular}{l}
|% \iffalse
%
% childdoc.dtx Copyright (C) 2017-2018 Niklas Beisert
%
% This work may be distributed and/or modified under the
% conditions of the LaTeX Project Public License, either version 1.3
% of this license or (at your option) any later version.
% The latest version of this license is in
%   http://www.latex-project.org/lppl.txt
% and version 1.3 or later is part of all distributions of LaTeX
% version 2005/12/01 or later.
%
% This work has the LPPL maintenance status `maintained'.
%
% The Current Maintainer of this work is Niklas Beisert.
%
% This work consists of the files childdoc.dtx and childdoc.ins
% and the derived files childdoc.def and cdocsamp.tex with
% cdocsch1.tex, cdocsch2.tex, cdocsdrf.tex, cdocsfn1.tex, cdocsfn2.tex.
%
%<package>\ifdefined\childdocmain\endinput\fi
%<package>\ProvidesFile{childdoc.def}[2018/12/30 v2.0 child document driver]
%<samplemain>\ProvidesFile{cdocsamp.tex}[2018/12/30 v2.0 sample for childdoc]
%<*driver>
%\ProvidesFile{childdoc.drv}[2018/12/30 v2.0 childdoc reference manual file]
\PassOptionsToClass{10pt,a4paper}{article}
\documentclass{ltxdoc}

\usepackage[margin=35mm]{geometry}
\usepackage{hyperref}
\usepackage{hyperxmp}
\usepackage[usenames]{color}

\hypersetup{colorlinks=true}
\hypersetup{pdfstartview=FitH}
\hypersetup{pdfpagemode=UseNone}
\hypersetup{pdfsource={}}
\hypersetup{pdflang={en-UK}}
\hypersetup{pdfcopyright={Copyright 2017-2018 Niklas Beisert.
  This work may be distributed and/or modified under the
  conditions of the LaTeX Project Public License, either version 1.3
  of this license or (at your option) any later version.}}
\hypersetup{pdflicenseurl={http://www.latex-project.org/lppl.txt}}
\hypersetup{pdfcontactaddress={ETH Zurich, ITP, HIT K,
  Wolfgang-Pauli-Strasse 27}}
\hypersetup{pdfcontactpostcode={8093}}
\hypersetup{pdfcontactcity={Zurich}}
\hypersetup{pdfcontactcountry={Switzerland}}
\hypersetup{pdfcontactemail={nbeisert@itp.phys.ethz.ch}}
\hypersetup{pdfcontacturl={http://people.phys.ethz.ch/\xmptilde nbeisert/}}

\newcommand{\secref}[1]{\hyperref[#1]{section \ref*{#1}}}

\parskip1ex
\parindent0pt
\let\olditemize\itemize
\def\itemize{\olditemize\parskip0pt}

\begin{document}

\title{The \textsf{childdoc} Package}
\hypersetup{pdftitle={The childdoc Package}}
\author{Niklas Beisert\\[2ex]
  Institut f\"ur Theoretische Physik\\
  Eidgen\"ossische Technische Hochschule Z\"urich\\
  Wolfgang-Pauli-Strasse 27, 8093 Z\"urich, Switzerland\\[1ex]
  \href{mailto:nbeisert@itp.phys.ethz.ch}
  {\texttt{nbeisert@itp.phys.ethz.ch}}}
\hypersetup{pdfauthor={Niklas Beisert}}
\hypersetup{pdfsubject={Manual for the LaTeX2e Package childdoc}}
\date{30 December 2018, \textsf{v2.0}}
\maketitle

\begin{abstract}\noindent
\textsf{childdoc} is a \LaTeXe{} package
that enables the direct compilation
of document sections included by |\include|
to individual files.
\end{abstract}

\begingroup
\parskip0ex
\tableofcontents
\endgroup

%%%%%%%%%%%%%%%%%%%%%%%%%%%%%%%%%%%%%%%%%%%%%%%%%%%%%%%%%%%%%%%%%%%%%%%%%%%%%%%%
%%%%%%%%%%%%%%%%%%%%%%%%%%%%%%%%%%%%%%%%%%%%%%%%%%%%%%%%%%%%%%%%%%%%%%%%%%%%%%%%
\section{Introduction}

\LaTeX{} provides a mechanism to structure a large document (such as a book)
into a main file and several child files (containing the chapters)
using the |\include| command.
This mechanism is beneficial for documents
which span hundreds of pages in order to
make the source file(s) more manageable.
Moreover, compilation can be restricted to
selected child files by means of the |\includeonly| command.
The latter feature can be used to reduce the compilation time while editing
(this was significantly more useful in the earlier days of \LaTeX{})
or to generate a smaller document which is easier to navigate.
Another application of |\includeonly| is to generate
documents consisting of selected parts of the complete document.

However, there are a few drawbacks of the plain |\include| mechanism:
\begin{itemize}
\item
The child files cannot be compiled on their own,
they can only be compiled via the main file.
A naive editing environment
(such as a text editor with an option
to have the current file processed by \LaTeX)
may require one to switch to the main file before compiling;
attempting to compile the child file produces errors.
\item
The main file must be modified (each time)
to adjust the |\includeonly| command
to the present needs. This easily leaves the main file in a messy state.
\item
The generated document will always carry the filename
of the main document. This is inconvenient if
several child files are to be compiled and
to be kept for distribution.
\end{itemize}

The present package provides a simple interface
to make child files individually compilable by \LaTeX{}.
Compiling a child file then has the same effect as compiling
the main file with an |\includeonly| command
to select the appropriate child.
Moreover the generated document will carry the name of the child
rather than the main file.
This resolves all three above issues.

This feature is meant to make the editing of books,
thesis documents and lecture notes somewhat more convenient.
However, the package can also be used efficiently for
composing a series of documents (such as exercise sheets)
which are typically distributed individually.
It then assists the author in generating the individual documents
(potentially in different versions)
as well as a document containing the collected series.
Another application is in developing style files
or other kinds of included material
where compilation of the style file could redirect
to a sample or test file.

%%%%%%%%%%%%%%%%%%%%%%%%%%%%%%%%%%%%%%%%%%%%%%%%%%%%%%%%%%%%%%%%%%%%%%%%%%%%%%%%
%%%%%%%%%%%%%%%%%%%%%%%%%%%%%%%%%%%%%%%%%%%%%%%%%%%%%%%%%%%%%%%%%%%%%%%%%%%%%%%%
\section{Usage}

First of all, the package \textsf{childdoc} is \emph{not} a standard
\LaTeXe{} |.sty| style file! Therefore it needs to be invoked in
a non-standard way.

%%%%%%%%%%%%%%%%%%%%%%%%%%%%%%%%%%%%%%%%%%%%%%%%%%%%%%%%%%%%%%%%%%%%%%%%%%%%%%%%
\subsection{Included Files}
\label{sec:include}

%%%%%%%%%%%%%%%%%%%%%%%%%%%%%%%%%%%%%%%%
\DescribeMacro{\childdocmain}
To use the package, add the commands
\begin{center}
\begin{tabular}{l}
|\input{childdoc.def}|\\
|\childdocmain{}|\\
\end{tabular}
\end{center}
at the very top of the main \LaTeX{} file,
in particular \emph{before} the |\documentclass| statement!
The argument of |\childdocmain| should be left empty
(but it must be present).

%%%%%%%%%%%%%%%%%%%%%%%%%%%%%%%%%%%%%%%%
\DescribeMacro{\childdocof}
Furthermore, add the commands
\begin{center}
\begin{tabular}{l}
|\input{childdoc.def}|\\
|\childdocof{|\textit{main}|}|\\
\end{tabular}
\end{center}
at the top of every child file \textit{child}
which is included by |\include{|\textit{child}|}|
from within the main file
(or at least for those files to be compiled individually).
The argument \textit{main} must be the filename of the main file.

There are a couple of
considerations in setting up the main and child documents:

%%%%%%%%%%%%%%%%%%%%%%%%%%%%%%%%%%%%%%%%
\paragraph{Restrictions.}

Please note the following restrictions:
\begin{itemize}
\item
|\childdocmain| must be called with one argument \textit{main}
to ensure compatibility with earlier version of the package.
It must either be empty (|\childdocmain{}|)
or precisely match the filename of the main file in which it is specified.
See \secref{sec:detection} for further information.
\item
The filename \textit{main} must be specified without the |.tex| extension.
\item
The filename \textit{main} is case sensitive
(even in case-insensitive file systems)
due to internal string comparison.
\item
The argument \textit{main} should be fully expanded, it cannot be a macro.
\item
Subdirectories and special characters should be avoided in filenames.
\item
The command |\childdocmain{|\textit{main}|}| must be followed by a whitespace.
It should not be followed immediately by another command
or by a comment mark `|%|'.
This is because the \TeX{} parser reads the token immediately following
the argument of |\childdocmain| and puts it
at the beginning of every child section;
however, a white\-space is ignored.
\end{itemize}

%%%%%%%%%%%%%%%%%%%%%%%%%%%%%%%%%%%%%%%%
\paragraph{Content of Main File.}

It is advisable to place all content in the child files included by |\include|.
Any output contained in the main file will appear in all child documents
unless suppressed manually;
it cannot be suppressed automatically by the |\includeonly| directive
and thus should normally be avoided.
A method to include some content in the main file
by means of conditional processing is described in \secref{sec:conditional}.

%%%%%%%%%%%%%%%%%%%%%%%%%%%%%%%%%%%%%%%%
\paragraph{Page Numbering.}

When only a part of the document is compiled,
the appropriate numbering of pages
(as well as other status parameters)
is determined from the |.aux| files.
The latter contain information from previous passes.
However this information needs to propagate through
all intermediate child documents.
Therefore the page numbering in child documents may well
be inconsistent until the complete document is compiled at least once.

A useful (if unconventional) way to always ensure a consistent
page numbering is to restart the numbering in each child document
and denote the pages by `\textit{child}|.|\textit{page}'
where \textit{child} represents the chapter/section number of the child file.
This can be achieved by the command
|\numberwithin{page}{|\textit{child}|}|
of the \textsf{amsmath} package
where \textit{child} can be |chapter| or |section|
depending on the chosen structuring.
Alternatively, one can modify the macro |\thepage| appropriately
and reset the counter |page| at the start of each child file.

%%%%%%%%%%%%%%%%%%%%%%%%%%%%%%%%%%%%%%%%%%%%%%%%%%%%%%%%%%%%%%%%%%%%%%%%%%%%%%%%
\subsection{Conditional Processing}
\label{sec:conditional}

The package provides a mechanism to compile different versions
of a document. To customise the versions further some conditional processing
can come in handy to distinguish which version is being compiled.
The package provides two macros to describe the compilation context:

%%%%%%%%%%%%%%%%%%%%%%%%%%%%%%%%%%%%%%%%
\DescribeMacro{\ifchilddoc}
The conditional |\ifchilddoc| distinguishes between the compilation of
child documents and the main document:
%
\begin{center}
|\ifchilddoc |\textit{child-code}| |[|\||else |\textit{main-code}]| \||fi|
\end{center}

%%%%%%%%%%%%%%%%%%%%%%%%%%%%%%%%%%%%%%%%
\DescribeMacro{\childdocname}
\DescribeMacro{\childdocjob}
The macro |\childdocname| contains the filename (without extension)
of the main or child file being processed.
Note that |\childdocjob| will always contain the name of the main file.

%%%%%%%%%%%%%%%%%%%%%%%%%%%%%%%%%%%%%%%%
\paragraph{Title Page.}

Conditional processing can be used to include a title or banner page
in the main document when proper precautions are taken.
Importantly, the code in the main file should ensure that the page counter
(as well as other status parameters which are stored in the |.aux| files)
takes the same value after the conditional processing.
Otherwise the page numbers may take divergent values
depending on which part is compiled.

For example, a title page could be declared by:
%
\begin{center}
\begin{tabular}{l}
|\ifchilddoc\||else|\\
|\addtocounter{page}{-1}|\\
\textit{code for title page}\\
|\newpage|\\
|\||fi|
\end{tabular}
\end{center}
%
A banner page for the child documents can be generated by:
%
\begin{center}
\begin{tabular}{l}
|\ifchilddoc|\\
|\addtocounter{page}{-1}|\\
\textit{code for banner page}\\
|\newpage|\\
|\||fi|
\end{tabular}
\end{center}
%
Here one could write a message such as:
\begin{center}
|This is the part \childdocname{} of \childdocjob{}.|
\end{center}

%%%%%%%%%%%%%%%%%%%%%%%%%%%%%%%%%%%%%%%%%%%%%%%%%%%%%%%%%%%%%%%%%%%%%%%%%%%%%%%%
\subsection{Flags}
\label{sec:flags}

The package makes it easy to generate different versions
of the main or child documents.
To this end compilation flags can be defined
and assigned different default values.
They will be particularly useful in conjunction
with the forwarding mechanism described in \secref{sec:forward}.

For example, it may be useful to have a flag |\version|
which can be set to |draft| or |final|.
The document source will contain some conditional code
depending on the value of |\version|.
Suppose further, the flag should default to |final| for the main file
and to |draft| for child files
which is a natural assignment for editing the document.
This is achieved by placing the following code
in the preamble of the main document
(below the |\childdocmain| directive):
%
\begin{center}
\begin{tabular}{l}
|\ifchilddoc|\\
|\providecommand{\version}{draft}|\\
|\||else|\\
|\providecommand{\version}{final}|\\
|\||fi|
\end{tabular}
\end{center}
%
The definition by |\providecommand| makes sure
that previous definitions are not overwritten.
Further statements |\providecommand{\version}{...}|
can thus be added before the above code to override it.

For the main file, one might add a line
(between |\childdocmain| and the above block)
%
\begin{center}
|%\ifchilddoc\||else\providecommand{\version}{draft}\||fi|
\end{center}
%
which can be uncommented to produce a draft version.
Likewise one can add a line to the very top of a child file
(above the |\childdocof{|\textit{main}|}| directive)
%
\begin{center}
|%\providecommand{\version}{final}|
\end{center}
%
which can be uncommented to produce the final version of this child document.

%%%%%%%%%%%%%%%%%%%%%%%%%%%%%%%%%%%%%%%%%%%%%%%%%%%%%%%%%%%%%%%%%%%%%%%%%%%%%%%%
\subsection{Forwarding}
\label{sec:forward}

Different versions of the main or child documents
using compilation flags as described in \secref{sec:flags}
can be (permanently) stored in different files
for convenient compilation, viewing and distribution.
To this end, the package defines a command
to pass on compilation to a different file:

%%%%%%%%%%%%%%%%%%%%%%%%%%%%%%%%%%%%%%%%
\DescribeMacro{\childdocforward}
The command |\childdocforward| redirects processing to
another source file:
%
\begin{center}
\begin{tabular}{l}
|\input{childdoc.def}|\\
|\childdocforward[|\textit{main}|]{|\textit{dest}|}|\\
\end{tabular}
\end{center}
%
The argument \textit{dest} is the destination file
(without extension).
It should be the main file or one of the child files.
Note that further \textsf{childdoc} directives
such as |\childdocof| and |\childdocforward|
in the indicated file will be processed in this form.
The optional argument \textit{main}
passes on directly to the main file \textit{main}
while pretending to compile the child \textit{dest}.
This form behaves as if \textit{dest}
issues |\childdocof{|\textit{main}|}| right away,
and no further \textsf{childdoc} directives will be processed.

%%%%%%%%%%%%%%%%%%%%%%%%%%%%%%%%%%%%%%%%
\DescribeMacro{\...prefix}
In the alternative form |\childdocforwardprefix|,
%
\begin{center}
\begin{tabular}{l}
|\input{childdoc.def}|\\
|\childdocforwardprefix[|\textit{main}|]{|\textit{prefix}|}{|\textit{dest}|}|
\end{tabular}
\end{center}
%
the destination file is determined by a pattern
depending on the current file:
To make this work, the current file must be called
`{\textit{prefix}\hspace{0.2em}\textit{suffix}}'
with \textit{prefix} matching precisely the argument.
Processing is then passed on to the file
`{\textit{dest}\hspace{0.2em}\textit{suffix}}'.
Surely, the same effect is achieved by
directly specifying the
argument `{\textit{dest}\hspace{0.2em}\textit{suffix}}'
in the first form.
However, that requires to set up a different file
for each child. With the alternative form of the command
all these files can have exactly the same content
which simplifies setting them up and maintaining them.

For example, the following file |draft.tex|
with a compilation flag |\version| as described in \secref{sec:flags}
compiles the main document as a draft:
%
\begin{center}
\begin{tabular}{l}
|\def\version{draft}|\\
|\input{childdoc.def}|\\
|\childdocforward{|\textit{main}|}|
\end{tabular}
\end{center}
%
Likewise, the following files |final|\textit{nn}|.tex|
compile the final version of the child document
|child|\textit{nn}|.tex|:
%
\begin{center}
\begin{tabular}{l}
|\def\version{final}|\\
|\input{childdoc.def}|\\
|\childdocforwardprefix{final}{child}|
\end{tabular}
\end{center}
%

Note that when several versions of a main file and/or of each child file
are to be generated, it may be convenient to set up a |Makefile| or
shell script to automatise the process.

%%%%%%%%%%%%%%%%%%%%%%%%%%%%%%%%%%%%%%%%%%%%%%%%%%%%%%%%%%%%%%%%%%%%%%%%%%%%%%%%
\subsection{Command Line Processing}
\label{sec:commandline}

The effect of redirection files can also be achieved by invoking
the \LaTeX{} compiler with a more elaborate command line.
Most conveniently this should be done as part
of a shell script or a |Makefile|.

When using \textsf{childdoc} in the main file, the following
command lines effectively perform a redirection
(note that depending on the shell being used,
backslashes may have to be doubled: `|\|' $\to$ `|\\|'):
%
\begin{center}
|... -jobname "|\textit{target}|" |\\|"|[\textit{flags}]%
|\input{childdoc.def}\childdocforward[|\textit{main}|]{|\textit{dest}|}"|
\end{center}
%
Here \textit{target} is the name of the output file,
\textit{main} is the name of the main file
and \textit{dest} is the name of the main or child file to be processed
(all filenames without extensions).
The optional argument \textit{main} can be omitted
if \textit{main} matches \textit{dest}.
Optionally, compilation \textit{flags} can be defined via |\def| commands.
This command line makes the \TeX{} engine believe
it is compiling the file \textit{target}
whose content is specified as the latter parameter.
The provided code then forwards the processing to
\textit{main} or \textit{dest} as described in \secref{sec:forward}.

%%%%%%%%%%%%%%%%%%%%%%%%%%%%%%%%%%%%%%%%%%%%%%%%%%%%%%%%%%%%%%%%%%%%%%%%%%%%%%%%
\subsection{Include by Input}
\label{sec:input}

Including child documents by |\include| has some restrictions by design.
Most notably, the content of a child document always occupies
its own set of pages; pages cannot be shared between child documents.
Usually, this behaviour makes perfect sense
because each child document contain an essential part of the document.
However, in some situations it may be desirable to compose
a document from a collection of parts
without having mandatory page breaks between then.
For this case, the package
provides a mechanism to include parts
by |\input| which can also be processed individually.
However, by construction this mechanism
requires manual handling of the content to be output.

%%%%%%%%%%%%%%%%%%%%%%%%%%%%%%%%%%%%%%%%
\DescribeMacro{\ifchilddocmanual}
The main file should be prepared as usual, see \secref{sec:include}.
However, the document body must make a distinction
between processing of an individual part and of the main document, e.g.:
%
\begin{center}
\begin{tabular}{l}
|\ifchilddocmanual|\\
|\input{\childdocname}|\\
|\||else|\\
\textit{document body with }|\input{|\textit{part}|}|\\
|\||fi|
\end{tabular}
\end{center}
%
The conditional |\ifchilddocmanual| is true whenever
a part to be included by |\input| is being compiled,
and the name of the part is stored in |\childdocname|.

%%%%%%%%%%%%%%%%%%%%%%%%%%%%%%%%%%%%%%%%
\DescribeMacro{\childdocby}
Each part to be included by |\input| should start with:
%
\begin{center}
\begin{tabular}{l}
|\input{childdoc.def}|\\
|\childdocby{|\textit{main}|}|\\
\end{tabular}
\end{center}
%
The directive |\childdocby| is similar to |\childdocof|
described in \secref{sec:include},
but the subsequent selection of content must be done manually.
To that end, both |\ifchilddoc| and |\ifchilddocmanual|
will be true upon processing of a part,
and the name of the part is stored in |\childdocname|.
Note that |\jobname| will be set to the filename of the current part
so that each part receives an individual |.aux| file
that does not interfere with the |.aux| file(s) of the main document.
This behaviour can be altered by the alternative form
|\childdocby[*]{|\textit{main}|}| (with a non-empty optional argument)
which uses the |.aux| file of the main document
by setting |\jobname| to \textit{main}.

%%%%%%%%%%%%%%%%%%%%%%%%%%%%%%%%%%%%%%%%%%%%%%%%%%%%%%%%%%%%%%%%%%%%%%%%%%%%%%%%
\subsection{Driver Development}
\label{sec:driver}

The \textsf{childdoc} mechanism can also be use for the development
of definition files such as \LaTeX{} styles or classes.
This case differs from the above setup with multiple parts
included by |\include| in that no |\includeonly| should be invoked.
This can be achieved by starting the include file
(before |\ProvidesPackage|) with:
%
\begin{center}
\begin{tabular}{l}
|\input{childdoc.def}|\\
|\childdocforward{|\textit{main}|}|\\
\end{tabular}
\end{center}
%
or alternatively with:
%
\begin{center}
\begin{tabular}{l}
|\input{childdoc.def}|\\
|\childdocby{|\textit{main}|}|\\
\end{tabular}
\end{center}
%
Both forms have slightly different effects as described above.
The main file is prepared as usual, see \secref{sec:include}.

%%%%%%%%%%%%%%%%%%%%%%%%%%%%%%%%%%%%%%%%%%%%%%%%%%%%%%%%%%%%%%%%%%%%%%%%%%%%%%%%
\subsection{Legacy Detection}
\label{sec:detection}

The directive |\childdocmain| in the main file can detect
whether the complete document or merely a child is to be compiled
even without using the directive |\childdocof|.
This method is deprecated because it is less robust
and there is no compelling reason to use it;
it is merely provided for backward compatibility
and it may be removed in future versions.

If the detection mechanism is to be used,
it is mandatory to correctly specify
the filename of the main file as the argument of |\childdocmain|:
%
\begin{center}
\begin{tabular}{l}
|\input{childdoc.def}|\\
|\childdocmain{|\textit{main}|}|\\
\end{tabular}
\end{center}
%
If |\jobname| does not match the argument \textit{main} of |\childdocmain|,
it is assumed that |\jobname| points to the child file to be compiled.
When using |\childdocmain| with the main file specified as argument,
it suffices to start a child file
with just |\input{|\textit{main}|}|
without loading of the package and using |\childdocof|.
If instead all processing is done
with the appropriate \textsf{childdoc} directives,
the argument of \textit{main} of |\childdocmain| can be empty.

An alternative version of the command line processing described
in \secref{sec:commandline} using the detection mechanism reads:
%
\begin{center}
|... -jobname "|\textit{target}|" "|[\textit{flags}]%
[|\def\jobname{|\textit{dest}|}|]|\input{|\textit{main}|}"|
\end{center}

%%%%%%%%%%%%%%%%%%%%%%%%%%%%%%%%%%%%%%%%%%%%%%%%%%%%%%%%%%%%%%%%%%%%%%%%%%%%%%%%
\subsection{Manual Code}
\label{sec:manual}

In case one cannot be certain whether the definitions file |childdoc.def|
is installed on the target \TeX{} distribution
and one prefers not to ship it,
it is conceivable to paste a few relevant commands into the sources.

To that end, drop all statements |\input{childdoc.def}|
and perform the replacements as outlined below.
Instead of |\childdocmain{|\textit{main}|}| add the following code
to the top of the main file:
%
\begin{center}
\begin{tabular}{l}
|\||ifdefined\childdocname\endinput\||fi\newif\ifchilddoc|\\
|\edef\childdocname{\scantokens\expandafter{\jobname\noexpand}}|\\
|\def\childdocmain{|\textit{main}|}\||ifx\childdocmain\childdocname\||else|\\
|\childdoctrue\includeonly{\childdocname}\let\jobname\childdocmain\||fi|\\
\end{tabular}
\end{center}
%
Instead of |\childdocof{|\textit{main}|}| just include the main file
at the top of each child file:
%
\begin{center}
|\input{|\textit{main}|}|
\end{center}
%
A simple redirection |\childdocforward{|\textit{dest}|}| is achieved by:
%
\begin{center}
|\def\jobname{|\textit{dest}|}\input{\jobname}|
\end{center}
%
The redirection with prefix
|\childdocforwardprefix[|\textit{prefix}|]{|\textit{dest}|}|
is accomplished by:
%
\begin{center}
\begin{tabular}{l}
|{\edef\jobname{\scantokens\expandafter{\jobname\noexpand}}|\\
|\def\redirectjob |\textit{prefix}|#1~~~{\gdef\jobname{|\textit{dest}|#1}}|\\
|\expandafter\redirectjob\jobname~~~}\input{\jobname}|
\end{tabular}
\end{center}

In an alternative approach,
child documents can be compiled by a specific command line
without additional code or specific definitions:
%
\begin{center}
|... -jobname "|\textit{target}|" "|[\textit{flags}]%
|\includeonly{|\textit{dest}|}\input{|\textit{main}|}"|
\end{center}
%

%%%%%%%%%%%%%%%%%%%%%%%%%%%%%%%%%%%%%%%%%%%%%%%%%%%%%%%%%%%%%%%%%%%%%%%%%%%%%%%%
%%%%%%%%%%%%%%%%%%%%%%%%%%%%%%%%%%%%%%%%%%%%%%%%%%%%%%%%%%%%%%%%%%%%%%%%%%%%%%%%
\section{Information}

%%%%%%%%%%%%%%%%%%%%%%%%%%%%%%%%%%%%%%%%%%%%%%%%%%%%%%%%%%%%%%%%%%%%%%%%%%%%%%%%
\subsection{Copyright}

Copyright \copyright{} 2017--2018 Niklas Beisert

This work may be distributed and/or modified under the
conditions of the \LaTeX{} Project Public License, either version 1.3
of this license or (at your option) any later version.
The latest version of this license is in
  \url{http://www.latex-project.org/lppl.txt}
and version 1.3 or later is part of all distributions of \LaTeX{}
version 2005/12/01 or later.

This work has the LPPL maintenance status `maintained'.

The Current Maintainer of this work is Niklas Beisert.

This work consists of the files |README.txt|, |childdoc.ins| and |childdoc.dtx|
as well as the derived files |childdoc.def|, |cdocsamp.tex|
with |cdocsch1.tex|, |cdocsch2.tex|, |cdocspt3.tex|, |cdocspt4.tex|,
|cdocsdrf.tex|, |cdocsfn1.tex|, |cdocsfn2.tex|
as well as |childdoc.pdf|.

%%%%%%%%%%%%%%%%%%%%%%%%%%%%%%%%%%%%%%%%%%%%%%%%%%%%%%%%%%%%%%%%%%%%%%%%%%%%%%%%
\subsection{Files and Installation}

The package consists of the files:
%
\begin{center}
\begin{tabular}{ll}
    |README.txt|   & readme file \\
    |childdoc.ins| & installation file \\
    |childdoc.dtx| & source file \\
    |childdoc.def| & definition file \\
    |cdocsamp.tex| & sample main file \\
    |cdocsch1.tex| & sample include file \\
    |cdocsch2.tex| & sample include file \\
    |cdocspt3.tex| & sample part file \\
    |cdocspt4.tex| & sample part file \\
    |cdocsdrf.tex| & sample redirection file \\
    |cdocsfn1.tex| & sample redirection file \\
    |cdocsfn2.tex| & sample redirection file \\
    |childdoc.pdf| & manual
\end{tabular}
\end{center}
%
The distribution consists of the files
|README.txt|, |childdoc.ins| and |childdoc.dtx|.
%
\begin{itemize}
\item
Run (pdf)\LaTeX{} on |childdoc.dtx|
to compile the manual |childdoc.pdf| (this file).
\item
Run \LaTeX{} on |childdoc.ins| to create the definitions file |childdoc.def|
and the sample |cdocsamp.tex| with include files
|cdocsch1.tex|, |cdocsch2.tex|, |cdocspt3.tex|, |cdocspt4.tex|,
|cdocsdrf.tex|, |cdocsfn1.tex|, |cdocsfn2.tex|.
Then copy the file |childdoc.def| to an appropriate directory of your \LaTeX{}
distribution, e.g.\ \textit{texmf-root}|/tex/latex/childdoc|.
\end{itemize}

%%%%%%%%%%%%%%%%%%%%%%%%%%%%%%%%%%%%%%%%%%%%%%%%%%%%%%%%%%%%%%%%%%%%%%%%%%%%%%%%
\subsection{Related CTAN Packages}

There are several other packages which offer a similar functionality:
%
\begin{itemize}
\item
The packages
\href{http://ctan.org/pkg/docmute}{\textsf{docmute}},
\href{http://ctan.org/pkg/includex}{\textsf{includex}} and
\href{http://ctan.org/pkg/standalone}{\textsf{standalone}}
provide commands to include only the document body of
a child file thus allowing both files to be compiled individually.
\item
The packages \href{http://ctan.org/pkg/subdocs}{\textsf{subdocs}}
and \href{http://ctan.org/pkg/subfiles}{\textsf{subfiles}}
provide structures in which the main and child documents can be
encapsulated and allowing them to be compiled individually.
The inclusion mechanism is different from the conventional |\include|.
\item
The package \href{http://ctan.org/pkg/combine}{\textsf{combine}}
is an elaborate solution to combine several documents into one.
\end{itemize}
%
See also the CTAN topic \href{http://ctan.org/topic/subdocs}{\textsf{subdocs}}
for further related packages.
The present package differs from the above solutions in that
a document structure constructed with the conventional |\include| mechanism
just needs two extra commands at the top of every file
such that all constituent files can be compiled individually.

%%%%%%%%%%%%%%%%%%%%%%%%%%%%%%%%%%%%%%%%%%%%%%%%%%%%%%%%%%%%%%%%%%%%%%%%%%%%%%%%
%\subsection{Feature Suggestions}
%
%The following is a list of features which may be useful for future
%versions of this package:
%%
%\begin{itemize}
%\item
%\ldots
%\end{itemize}

%%%%%%%%%%%%%%%%%%%%%%%%%%%%%%%%%%%%%%%%%%%%%%%%%%%%%%%%%%%%%%%%%%%%%%%%%%%%%%%%
\subsection{Revision History}

%%%%%%%%%%%%%%%%%%%%%%%%%%%%%%%%%%%%%%%%
\paragraph{v2.0:} 2018/12/30

\begin{itemize}
\item
immediate forward processing
\item
added |\childdocby| mechanism
\item
manual restructured
\end{itemize}

%%%%%%%%%%%%%%%%%%%%%%%%%%%%%%%%%%%%%%%%
\paragraph{v1.6:} 2018/01/17

\begin{itemize}
\item
application for development of include files
\item
corrections to manual
\end{itemize}

%%%%%%%%%%%%%%%%%%%%%%%%%%%%%%%%%%%%%%%%
\paragraph{v1.5:} 2017/05/21

\begin{itemize}
\item
more complete structuring introduced
\item
|\childdocof| introduced
\item
|\childdoc| renamed to |\childdocmain|
\item
|\childredirect| renamed to |\childdocforward| and |\childdocforwardprefix|
and functionality expanded
\end{itemize}

%%%%%%%%%%%%%%%%%%%%%%%%%%%%%%%%%%%%%%%%
\paragraph{v1.0:} 2017/04/27

\begin{itemize}
\item
manual and install package
\item
first version published on CTAN
\end{itemize}

%%%%%%%%%%%%%%%%%%%%%%%%%%%%%%%%%%%%%%%%
\paragraph{v0.6:} 2017/04/26

\begin{itemize}
\item
redirection mechanism added
\end{itemize}

%%%%%%%%%%%%%%%%%%%%%%%%%%%%%%%%%%%%%%%%
\paragraph{v0.5:} 2017/04/26

\begin{itemize}
\item
functionality in definition file
\end{itemize}


%%%%%%%%%%%%%%%%%%%%%%%%%%%%%%%%%%%%%%%%%%%%%%%%%%%%%%%%%%%%%%%%%%%%%%%%%%%%%%%%
%%%%%%%%%%%%%%%%%%%%%%%%%%%%%%%%%%%%%%%%%%%%%%%%%%%%%%%%%%%%%%%%%%%%%%%%%%%%%%%%
%%%%%%%%%%%%%%%%%%%%%%%%%%%%%%%%%%%%%%%%%%%%%%%%%%%%%%%%%%%%%%%%%%%%%%%%%%%%%%%%
\appendix

\settowidth\MacroIndent{\rmfamily\scriptsize 000\ }

 \DocInput{childdoc.dtx}

\end{document}
%</driver>
% \fi
%
% %%%%%%%%%%%%%%%%%%%%%%%%%%%%%%%%%%%%%%%%%%%%%%%%%%%%%%%%%%%%%%%%%%%%%%%%%%%%%%
% %%%%%%%%%%%%%%%%%%%%%%%%%%%%%%%%%%%%%%%%%%%%%%%%%%%%%%%%%%%%%%%%%%%%%%%%%%%%%%
% \section{Sample}
%\iffalse
%<*samplemain>
%\fi
%
% The following presents a sample document
% with two chapters, two parts, a title page,
% a compile flag as well as three forwarding files to set the flag.
% It consists of eight |.tex| files:
% \begin{center}
% \begin{tabular}{ll}
% |cdocsamp.tex|&main file\\
% |cdocsch1.tex|&include file for chapter 1\\
% |cdocsch2.tex|&include file for chapter 2\\
% |cdocspt3.tex|&include file for part 3\\
% |cdocspt4.tex|&include file for part 4\\
% |cdocsdrf.tex|&forwarding file for main file in draft mode\\
% |cdocsfi1.tex|&forwarding file for final version of chapter 1\\
% |cdocsfi2.tex|&forwarding file for final version of chapter 2\\
% \end{tabular}
% \end{center}
% Each of the eight files can be compiled directly by the \LaTeX{} compiler.
%
% %%%%%%%%%%%%%%%%%%%%%%%%%%%%%%%%%%%%%%
% \paragraph{Main File.}
%
% The main file is called |cdocsamp.tex|.
%
% Load the \textsf{childdoc} definitions and
% declare the filename for the main document:
%    \begin{macrocode}
\input{childdoc.def}
\childdocmain{}
%    \end{macrocode}

% Optional override for |\version| flag:
%    \begin{macrocode}
%%\ifchilddoc\else\providecommand{\version}{draft}\fi
%    \end{macrocode}

% Define the default values for the |\version| flag
% (|final| for the main file and |draft| for childs):
%    \begin{macrocode}
\ifchilddoc
\providecommand{\version}{draft}
\else
\providecommand{\version}{final}
\fi
%    \end{macrocode}

% Load the standard document class:
%    \begin{macrocode}
\documentclass[12pt]{article}
%    \end{macrocode}

% Start the document body:
%    \begin{macrocode}
\begin{document}
%    \end{macrocode}

% Declare a title page.
% Print title, part of document being processed and version flag:
%    \begin{macrocode}
\addtocounter{page}{-1}
\begin{center}
{\LARGE\bfseries{}childdoc example\par}
\vspace{1cm}
\ifchilddoc
\ifchilddocmanual part\else chapter\fi:
`\childdocname' of `\childdocjob'\par
\else
main document: `\childdocjob'\par
\fi
version: \version\par
\end{center}
\newpage
%    \end{macrocode}

% Manually include selected file,
% otherwise process as usual:
%    \begin{macrocode}
\ifchilddocmanual
\section*{part `\childdocname'}
\input{\childdocname}
\else
%    \end{macrocode}

% Include the two chapters:
%    \begin{macrocode}
\include{cdocsch1}
\include{cdocsch2}
%    \end{macrocode}

% Include the two parts unless only chapters should be displayed:
%    \begin{macrocode}
\ifchilddoc\else
\section{part three}
\input{cdocspt3}
\section{part four}
\input{cdocspt4}
\fi
%    \end{macrocode}

% Process as usual until here:
%    \begin{macrocode}
\fi
%    \end{macrocode}

% End of document body:
%    \begin{macrocode}
\end{document}
%    \end{macrocode}
%\iffalse
%</samplemain>
%\fi
%
% %%%%%%%%%%%%%%%%%%%%%%%%%%%%%%%%%%%%%%
% \paragraph{Chapter Include Files.}
%
% The include files are called |cdocsch1.tex| and |cdocsch2.tex|.
%
%\iffalse
%<*samplechap1|samplechap2>
%\fi

% Optional override for |\version| flag:
%    \begin{macrocode}
%%\providecommand{\version}{final}
%    \end{macrocode}

% Include the main document:
%    \begin{macrocode}
\input{childdoc.def}
\childdocof{cdocsamp}
%    \end{macrocode}

%\iffalse
%</samplechap1|samplechap2>
%\fi
%
%\iffalse
%<*samplechap1>
%\fi
% Some text for chapter 1:
%    \begin{macrocode}
\section{one}
some text in chapter one
%    \end{macrocode}

%\iffalse
%</samplechap1>
%\fi
% Some text for chapter 2:
%\iffalse
%<*samplechap2>
%\fi
%    \begin{macrocode}
\section{two}
more text in chapter two
%    \end{macrocode}

%\iffalse
%</samplechap2>
%\fi
%
% %%%%%%%%%%%%%%%%%%%%%%%%%%%%%%%%%%%%%%
% \paragraph{Part Include Files.}
%
% The include files are called |cdocspt3.tex| and |cdocspt4.tex|.
%
%\iffalse
%<*samplepart3|samplepart4>
%\fi

% Optional override for |\version| flag:
%    \begin{macrocode}
%%\providecommand{\version}{final}
%    \end{macrocode}

% Include the main document:
%    \begin{macrocode}
\input{childdoc.def}
\childdocby{cdocsamp}
%    \end{macrocode}

%\iffalse
%</samplepart3|samplepart4>
%\fi
%
%\iffalse
%<*samplepart3>
%\fi
% Some text for part 3:
%    \begin{macrocode}
some text in part three
%    \end{macrocode}

%\iffalse
%</samplepart3>
%\fi
% Some text for part 4:
%\iffalse
%<*samplepart4>
%\fi
%    \begin{macrocode}
more text in part four
%    \end{macrocode}

%\iffalse
%</samplepart4>
%\fi
%
% %%%%%%%%%%%%%%%%%%%%%%%%%%%%%%%%%%%%%%
% \paragraph{Forwarding for a Complete Draft.}
%
% The following forwarding file |cdocsdrf.tex|
% compiles the main document in draft mode:
%\iffalse
%<*sampledraft>
%\fi
%    \begin{macrocode}
\def\version{draft}
\input{childdoc.def}
\childdocforward{cdocsamp}
%    \end{macrocode}

%\iffalse
%</sampledraft>
%\fi
%
% %%%%%%%%%%%%%%%%%%%%%%%%%%%%%%%%%%%%%%
% \paragraph{Forwarding for Final Version of the Chapters.}
%
% The following forwarding files |cdocsfn1.tex| and |cdocsfn2.tex|
% (with identical content)
% compile the final versions of the child documents
% |cdocsch1.tex| and |cdocsch2.tex|, respectively:
%\iffalse
%<*samplefinal>
%\fi
%    \begin{macrocode}
\def\version{final}
\input{childdoc.def}
\childdocforwardprefix[cdocsamp]{cdocsfn}{cdocsch}
%    \end{macrocode}

%\iffalse
%</samplefinal>
%\fi
%
% %%%%%%%%%%%%%%%%%%%%%%%%%%%%%%%%%%%%%%
% \paragraph{Command Line Processing.}
%
% The following three command lines generate the output files
% |cdocscld|, |cdocscl1| and |cdocscl2|
% which should be identical to
% |cdocsdrf|, |cdocsch1| and |cdocsfn2|, respectively:
% \begin{center}
% \begin{tabular}{l}
% |latex -jobname cdocscld \|\\
% |  "\def\version{draft}\input{childdoc.def}\childdocforward{cdocsamp}"|\\
% |latex -jobname cdocscl1 \|\\
% |  "\input{childdoc.def}\childdocforward[cdocsamp]{cdocsch1}"|\\
% |latex -jobname cdocscl2 \|\\
% |  "\def\version{final}\input{childdoc.def}\childdocforward{cdocsch2}"|
% \end{tabular}
% \end{center}
% Note that the trailing backslash on each first line
% merely continues the input to the second line
% (for convenient cut ant paste).
% Furthermore, the command |latex| can be replaced by any
% of its alternative versions such as |pdflatex|.
%
% %%%%%%%%%%%%%%%%%%%%%%%%%%%%%%%%%%%%%%%%%%%%%%%%%%%%%%%%%%%%%%%%%%%%%%%%%%%%%%
% %%%%%%%%%%%%%%%%%%%%%%%%%%%%%%%%%%%%%%%%%%%%%%%%%%%%%%%%%%%%%%%%%%%%%%%%%%%%%%
% \section{Implementation}
%\iffalse
%<*package>
%\fi
%
% This section describes the definitions file |childdoc.def|.

% The definitions cannot be loaded using |\usepackage| or |\RequirePackage|
% which has a mechanism to prevent loading a style file more than once.
% When loading the definitions by means of |\input|
% multiple instances have to be prevented manually:
%\iffalse
%This code needs to be before the `\ProvidesFile' directive
%which is defined at the beginning of this file.
%Therefore it is also placed there and commented out here.
%</package>
%<*discard>
%\fi
%    \begin{macrocode}
\ifdefined\childdocmain\endinput\fi
%    \end{macrocode}
%\iffalse
%</discard>
%<*package>
%\fi
%
% \macro{\ifchilddoc}
% \macro{\ifchilddocmanual}
% The conditional |\ifchilddoc| tells whether a
% child (true) or main (false) document is being compiled.
% The conditional |\ifchilddocmanual| tells whether
% the |\includeonly| mechanism is used (false) or
% the selection of child files must be performed manually (true).
% The definitions initialise to false:
%    \begin{macrocode}
\newif\ifchilddoc
\newif\ifchilddocmanual
%    \end{macrocode}

% \macro{\childdocname}
% \macro{\childdocjob}
% The macro |\childdocname| stores the name of the main document
% to be compiled. The macro |\childdocjob| stores the name of
% the document on which the \LaTeX{} compiler was originally invoked.
% The content of |\jobname| cannot be compared
% to filenames specified in the source due to different catcodes.
% The following code rescans |\jobname|, stores the result
% in |\childdocname| and saves a copy in |\childdocjob|:
%    \begin{macrocode}
\edef\childdocname{\scantokens\expandafter{\jobname\noexpand}}
\let\childdocjob\childdocname
%    \end{macrocode}

% \macro{\childdocdisable}
% The macro |\childdocdisable| prevents the main file
% from being processed more than once.
% At this stage, the main document command |\childdocmain|
% is assumed to be called once again where it should do nothing.
% Any subsequent call to it should prevent
% a secondary processing of the main document
% It overwrites the forwarding commands
% |\childdocof| and |\childdocforward|
% with empty macros to prevent further inclusions of the main document:
%    \begin{macrocode}
\newcommand{\childdocdisable}
{
  \renewcommand{\childdocmain}[1]{\renewcommand{\childdocmain}[1]{\endinput}}
  \renewcommand{\childdocof}[1]{}
  \renewcommand{\childdocby}[2][]{}
  \renewcommand{\childdocforward}[2][]{}
  \renewcommand{\childdocdisable}{}
}
%    \end{macrocode}

% \macro{\childdocmain}
% The macro |\childdocmain| is to be called at the top of the main file
% with nothing or the main filename (without extension) as argument.
% First, it breaks loops.
% If the argument is not empty and does not match |\childdocname|
% (which is set by the first inclusion of |childdoc.def|),
% |\ifchilddoc| is set to true, |\includeonly| is applied to the child file
% and |\jobname| is set to the main file
% (for proper handling of |.aux| files):
%    \begin{macrocode}
\newcommand{\childdocmain}[1]
{
  \childdocdisable\childdocmain{}
  \if?#1?\else
    \begingroup
      \def\childdoctmp{#1}
      \ifx\childdoctmp\childdocname
        \def\childdoctmp{}
      \else
        \def\childdoctmp
        {
          \childdoctrue
          \includeonly{\childdocname}
          \def\childdocjob{#1}
          \def\jobname{#1}
        }
      \fi
      \expandafter
    \endgroup
    \childdoctmp
  \fi
}
%    \end{macrocode}

% \macro{\childdocof}
% The command |\childdocof| redirects
% compilation to the main file |#1|.
%    \begin{macrocode}
\newcommand{\childdocof}[1]
{
  \childdocdisable
  \childdoctrue
  \includeonly{\childdocname}
  \def\jobname{#1}
  \def\childdocjob{#1}
  \input{#1}
}
%    \end{macrocode}

% \macro{\childdocby}
% The command |\childdocby| ....
%    \begin{macrocode}
\newcommand{\childdocby}[2][]
{
  \childdocdisable
  \childdoctrue
  \childdocmanualtrue
  \if?#1?\else
    \def\jobname{#2}
  \fi
  \def\childdocjob{#2}
  \input{#2}
  \endinput
}
%    \end{macrocode}

% \macro{\childdocforward}
% The command |\childdocforward| redirects
% compilation to the main file or
% (if the optional argument is given) a child file.
% Parameters are set as if the main file
% or a child file starting with |\childdocof| was compiled.
% Then compilation is handed over to the main file:
%    \begin{macrocode}
\newcommand{\childdocforward}[2][]
{
  \begingroup
    \if?#1?
      \def\childdoctmp
      {
        \def\childdocname{#2}
        \def\childdocjob{#2}
        \def\jobname{#2}
        \input{#2}
        \endinput
      }
    \else
      \def\childdoctmp
      {
        \childdocdisable
        \def\childdocname{#2}
        \childdoctrue
        \includeonly{#2}
        \def\childdocjob{#1}
        \def\jobname{#1}
        \input{#1}
        \endinput
      }
    \fi
    \expandafter
  \endgroup
  \childdoctmp
}
%    \end{macrocode}

% \macro{\childdocforwardprefix}
% The command |\childdocforwardprefix| redirects
% compilation to the main or a child file by means of a pattern.
% The prefix |#1| in the current filename is replaced by |#2|
% and the suffix of the current filename is kept
% (it is assumed that the filename does not contain the substring `|~~~|'
% which is used as a delimiter).
% Compilation is handed over to the new file by |\childdocforward|:
%    \begin{macrocode}
\newcommand{\childdocforwardprefix}[3][]
{
  \begingroup
    \def\childdocextract #2##1~~~{\def\childdoctmp{\childdocforward[#1]{#3##1}}}
    \expandafter\childdocextract\childdocname~~~
    \expandafter
  \endgroup
  \childdoctmp
}
%    \end{macrocode}

% \macro{\childdoc}
% The deprecated macro |\childdoc| is a legacy version of |\childdocmain|:
%    \begin{macrocode}
\newcommand{\childdoc}{\childdocmain}
%    \end{macrocode}

% \macro{\childdocredirect}
% The deprecated macro |\childdocredirect| is a legacy version
% of |\childdocforward| and |\childdocforwardprefix|:
%    \begin{macrocode}
\newcommand{\childdocredirect}[2][]
{
  \begingroup
    \if?#1?
      \def\childdoctmp{\childdocforward{#2}}
    \else
      \def\childdoctmp{\childdocforwardprefix{#1}{#2}}
    \fi
    \expandafter
  \endgroup
  \childdoctmp
}
%    \end{macrocode}

%\iffalse
%</package>
%\fi
%
\endinput
|\\
|\childdocforward[|\textit{main}|]{|\textit{dest}|}|\\
\end{tabular}
\end{center}
%
The argument \textit{dest} is the destination file
(without extension).
It should be the main file or one of the child files.
Note that further \textsf{childdoc} directives
such as |\childdocof| and |\childdocforward|
in the indicated file will be processed in this form.
The optional argument \textit{main}
passes on directly to the main file \textit{main}
while pretending to compile the child \textit{dest}.
This form behaves as if \textit{dest}
issues |\childdocof{|\textit{main}|}| right away,
and no further \textsf{childdoc} directives will be processed.

%%%%%%%%%%%%%%%%%%%%%%%%%%%%%%%%%%%%%%%%
\DescribeMacro{\...prefix}
In the alternative form |\childdocforwardprefix|,
%
\begin{center}
\begin{tabular}{l}
|% \iffalse
%
% childdoc.dtx Copyright (C) 2017-2018 Niklas Beisert
%
% This work may be distributed and/or modified under the
% conditions of the LaTeX Project Public License, either version 1.3
% of this license or (at your option) any later version.
% The latest version of this license is in
%   http://www.latex-project.org/lppl.txt
% and version 1.3 or later is part of all distributions of LaTeX
% version 2005/12/01 or later.
%
% This work has the LPPL maintenance status `maintained'.
%
% The Current Maintainer of this work is Niklas Beisert.
%
% This work consists of the files childdoc.dtx and childdoc.ins
% and the derived files childdoc.def and cdocsamp.tex with
% cdocsch1.tex, cdocsch2.tex, cdocsdrf.tex, cdocsfn1.tex, cdocsfn2.tex.
%
%<package>\ifdefined\childdocmain\endinput\fi
%<package>\ProvidesFile{childdoc.def}[2018/12/30 v2.0 child document driver]
%<samplemain>\ProvidesFile{cdocsamp.tex}[2018/12/30 v2.0 sample for childdoc]
%<*driver>
%\ProvidesFile{childdoc.drv}[2018/12/30 v2.0 childdoc reference manual file]
\PassOptionsToClass{10pt,a4paper}{article}
\documentclass{ltxdoc}

\usepackage[margin=35mm]{geometry}
\usepackage{hyperref}
\usepackage{hyperxmp}
\usepackage[usenames]{color}

\hypersetup{colorlinks=true}
\hypersetup{pdfstartview=FitH}
\hypersetup{pdfpagemode=UseNone}
\hypersetup{pdfsource={}}
\hypersetup{pdflang={en-UK}}
\hypersetup{pdfcopyright={Copyright 2017-2018 Niklas Beisert.
  This work may be distributed and/or modified under the
  conditions of the LaTeX Project Public License, either version 1.3
  of this license or (at your option) any later version.}}
\hypersetup{pdflicenseurl={http://www.latex-project.org/lppl.txt}}
\hypersetup{pdfcontactaddress={ETH Zurich, ITP, HIT K,
  Wolfgang-Pauli-Strasse 27}}
\hypersetup{pdfcontactpostcode={8093}}
\hypersetup{pdfcontactcity={Zurich}}
\hypersetup{pdfcontactcountry={Switzerland}}
\hypersetup{pdfcontactemail={nbeisert@itp.phys.ethz.ch}}
\hypersetup{pdfcontacturl={http://people.phys.ethz.ch/\xmptilde nbeisert/}}

\newcommand{\secref}[1]{\hyperref[#1]{section \ref*{#1}}}

\parskip1ex
\parindent0pt
\let\olditemize\itemize
\def\itemize{\olditemize\parskip0pt}

\begin{document}

\title{The \textsf{childdoc} Package}
\hypersetup{pdftitle={The childdoc Package}}
\author{Niklas Beisert\\[2ex]
  Institut f\"ur Theoretische Physik\\
  Eidgen\"ossische Technische Hochschule Z\"urich\\
  Wolfgang-Pauli-Strasse 27, 8093 Z\"urich, Switzerland\\[1ex]
  \href{mailto:nbeisert@itp.phys.ethz.ch}
  {\texttt{nbeisert@itp.phys.ethz.ch}}}
\hypersetup{pdfauthor={Niklas Beisert}}
\hypersetup{pdfsubject={Manual for the LaTeX2e Package childdoc}}
\date{30 December 2018, \textsf{v2.0}}
\maketitle

\begin{abstract}\noindent
\textsf{childdoc} is a \LaTeXe{} package
that enables the direct compilation
of document sections included by |\include|
to individual files.
\end{abstract}

\begingroup
\parskip0ex
\tableofcontents
\endgroup

%%%%%%%%%%%%%%%%%%%%%%%%%%%%%%%%%%%%%%%%%%%%%%%%%%%%%%%%%%%%%%%%%%%%%%%%%%%%%%%%
%%%%%%%%%%%%%%%%%%%%%%%%%%%%%%%%%%%%%%%%%%%%%%%%%%%%%%%%%%%%%%%%%%%%%%%%%%%%%%%%
\section{Introduction}

\LaTeX{} provides a mechanism to structure a large document (such as a book)
into a main file and several child files (containing the chapters)
using the |\include| command.
This mechanism is beneficial for documents
which span hundreds of pages in order to
make the source file(s) more manageable.
Moreover, compilation can be restricted to
selected child files by means of the |\includeonly| command.
The latter feature can be used to reduce the compilation time while editing
(this was significantly more useful in the earlier days of \LaTeX{})
or to generate a smaller document which is easier to navigate.
Another application of |\includeonly| is to generate
documents consisting of selected parts of the complete document.

However, there are a few drawbacks of the plain |\include| mechanism:
\begin{itemize}
\item
The child files cannot be compiled on their own,
they can only be compiled via the main file.
A naive editing environment
(such as a text editor with an option
to have the current file processed by \LaTeX)
may require one to switch to the main file before compiling;
attempting to compile the child file produces errors.
\item
The main file must be modified (each time)
to adjust the |\includeonly| command
to the present needs. This easily leaves the main file in a messy state.
\item
The generated document will always carry the filename
of the main document. This is inconvenient if
several child files are to be compiled and
to be kept for distribution.
\end{itemize}

The present package provides a simple interface
to make child files individually compilable by \LaTeX{}.
Compiling a child file then has the same effect as compiling
the main file with an |\includeonly| command
to select the appropriate child.
Moreover the generated document will carry the name of the child
rather than the main file.
This resolves all three above issues.

This feature is meant to make the editing of books,
thesis documents and lecture notes somewhat more convenient.
However, the package can also be used efficiently for
composing a series of documents (such as exercise sheets)
which are typically distributed individually.
It then assists the author in generating the individual documents
(potentially in different versions)
as well as a document containing the collected series.
Another application is in developing style files
or other kinds of included material
where compilation of the style file could redirect
to a sample or test file.

%%%%%%%%%%%%%%%%%%%%%%%%%%%%%%%%%%%%%%%%%%%%%%%%%%%%%%%%%%%%%%%%%%%%%%%%%%%%%%%%
%%%%%%%%%%%%%%%%%%%%%%%%%%%%%%%%%%%%%%%%%%%%%%%%%%%%%%%%%%%%%%%%%%%%%%%%%%%%%%%%
\section{Usage}

First of all, the package \textsf{childdoc} is \emph{not} a standard
\LaTeXe{} |.sty| style file! Therefore it needs to be invoked in
a non-standard way.

%%%%%%%%%%%%%%%%%%%%%%%%%%%%%%%%%%%%%%%%%%%%%%%%%%%%%%%%%%%%%%%%%%%%%%%%%%%%%%%%
\subsection{Included Files}
\label{sec:include}

%%%%%%%%%%%%%%%%%%%%%%%%%%%%%%%%%%%%%%%%
\DescribeMacro{\childdocmain}
To use the package, add the commands
\begin{center}
\begin{tabular}{l}
|\input{childdoc.def}|\\
|\childdocmain{}|\\
\end{tabular}
\end{center}
at the very top of the main \LaTeX{} file,
in particular \emph{before} the |\documentclass| statement!
The argument of |\childdocmain| should be left empty
(but it must be present).

%%%%%%%%%%%%%%%%%%%%%%%%%%%%%%%%%%%%%%%%
\DescribeMacro{\childdocof}
Furthermore, add the commands
\begin{center}
\begin{tabular}{l}
|\input{childdoc.def}|\\
|\childdocof{|\textit{main}|}|\\
\end{tabular}
\end{center}
at the top of every child file \textit{child}
which is included by |\include{|\textit{child}|}|
from within the main file
(or at least for those files to be compiled individually).
The argument \textit{main} must be the filename of the main file.

There are a couple of
considerations in setting up the main and child documents:

%%%%%%%%%%%%%%%%%%%%%%%%%%%%%%%%%%%%%%%%
\paragraph{Restrictions.}

Please note the following restrictions:
\begin{itemize}
\item
|\childdocmain| must be called with one argument \textit{main}
to ensure compatibility with earlier version of the package.
It must either be empty (|\childdocmain{}|)
or precisely match the filename of the main file in which it is specified.
See \secref{sec:detection} for further information.
\item
The filename \textit{main} must be specified without the |.tex| extension.
\item
The filename \textit{main} is case sensitive
(even in case-insensitive file systems)
due to internal string comparison.
\item
The argument \textit{main} should be fully expanded, it cannot be a macro.
\item
Subdirectories and special characters should be avoided in filenames.
\item
The command |\childdocmain{|\textit{main}|}| must be followed by a whitespace.
It should not be followed immediately by another command
or by a comment mark `|%|'.
This is because the \TeX{} parser reads the token immediately following
the argument of |\childdocmain| and puts it
at the beginning of every child section;
however, a white\-space is ignored.
\end{itemize}

%%%%%%%%%%%%%%%%%%%%%%%%%%%%%%%%%%%%%%%%
\paragraph{Content of Main File.}

It is advisable to place all content in the child files included by |\include|.
Any output contained in the main file will appear in all child documents
unless suppressed manually;
it cannot be suppressed automatically by the |\includeonly| directive
and thus should normally be avoided.
A method to include some content in the main file
by means of conditional processing is described in \secref{sec:conditional}.

%%%%%%%%%%%%%%%%%%%%%%%%%%%%%%%%%%%%%%%%
\paragraph{Page Numbering.}

When only a part of the document is compiled,
the appropriate numbering of pages
(as well as other status parameters)
is determined from the |.aux| files.
The latter contain information from previous passes.
However this information needs to propagate through
all intermediate child documents.
Therefore the page numbering in child documents may well
be inconsistent until the complete document is compiled at least once.

A useful (if unconventional) way to always ensure a consistent
page numbering is to restart the numbering in each child document
and denote the pages by `\textit{child}|.|\textit{page}'
where \textit{child} represents the chapter/section number of the child file.
This can be achieved by the command
|\numberwithin{page}{|\textit{child}|}|
of the \textsf{amsmath} package
where \textit{child} can be |chapter| or |section|
depending on the chosen structuring.
Alternatively, one can modify the macro |\thepage| appropriately
and reset the counter |page| at the start of each child file.

%%%%%%%%%%%%%%%%%%%%%%%%%%%%%%%%%%%%%%%%%%%%%%%%%%%%%%%%%%%%%%%%%%%%%%%%%%%%%%%%
\subsection{Conditional Processing}
\label{sec:conditional}

The package provides a mechanism to compile different versions
of a document. To customise the versions further some conditional processing
can come in handy to distinguish which version is being compiled.
The package provides two macros to describe the compilation context:

%%%%%%%%%%%%%%%%%%%%%%%%%%%%%%%%%%%%%%%%
\DescribeMacro{\ifchilddoc}
The conditional |\ifchilddoc| distinguishes between the compilation of
child documents and the main document:
%
\begin{center}
|\ifchilddoc |\textit{child-code}| |[|\||else |\textit{main-code}]| \||fi|
\end{center}

%%%%%%%%%%%%%%%%%%%%%%%%%%%%%%%%%%%%%%%%
\DescribeMacro{\childdocname}
\DescribeMacro{\childdocjob}
The macro |\childdocname| contains the filename (without extension)
of the main or child file being processed.
Note that |\childdocjob| will always contain the name of the main file.

%%%%%%%%%%%%%%%%%%%%%%%%%%%%%%%%%%%%%%%%
\paragraph{Title Page.}

Conditional processing can be used to include a title or banner page
in the main document when proper precautions are taken.
Importantly, the code in the main file should ensure that the page counter
(as well as other status parameters which are stored in the |.aux| files)
takes the same value after the conditional processing.
Otherwise the page numbers may take divergent values
depending on which part is compiled.

For example, a title page could be declared by:
%
\begin{center}
\begin{tabular}{l}
|\ifchilddoc\||else|\\
|\addtocounter{page}{-1}|\\
\textit{code for title page}\\
|\newpage|\\
|\||fi|
\end{tabular}
\end{center}
%
A banner page for the child documents can be generated by:
%
\begin{center}
\begin{tabular}{l}
|\ifchilddoc|\\
|\addtocounter{page}{-1}|\\
\textit{code for banner page}\\
|\newpage|\\
|\||fi|
\end{tabular}
\end{center}
%
Here one could write a message such as:
\begin{center}
|This is the part \childdocname{} of \childdocjob{}.|
\end{center}

%%%%%%%%%%%%%%%%%%%%%%%%%%%%%%%%%%%%%%%%%%%%%%%%%%%%%%%%%%%%%%%%%%%%%%%%%%%%%%%%
\subsection{Flags}
\label{sec:flags}

The package makes it easy to generate different versions
of the main or child documents.
To this end compilation flags can be defined
and assigned different default values.
They will be particularly useful in conjunction
with the forwarding mechanism described in \secref{sec:forward}.

For example, it may be useful to have a flag |\version|
which can be set to |draft| or |final|.
The document source will contain some conditional code
depending on the value of |\version|.
Suppose further, the flag should default to |final| for the main file
and to |draft| for child files
which is a natural assignment for editing the document.
This is achieved by placing the following code
in the preamble of the main document
(below the |\childdocmain| directive):
%
\begin{center}
\begin{tabular}{l}
|\ifchilddoc|\\
|\providecommand{\version}{draft}|\\
|\||else|\\
|\providecommand{\version}{final}|\\
|\||fi|
\end{tabular}
\end{center}
%
The definition by |\providecommand| makes sure
that previous definitions are not overwritten.
Further statements |\providecommand{\version}{...}|
can thus be added before the above code to override it.

For the main file, one might add a line
(between |\childdocmain| and the above block)
%
\begin{center}
|%\ifchilddoc\||else\providecommand{\version}{draft}\||fi|
\end{center}
%
which can be uncommented to produce a draft version.
Likewise one can add a line to the very top of a child file
(above the |\childdocof{|\textit{main}|}| directive)
%
\begin{center}
|%\providecommand{\version}{final}|
\end{center}
%
which can be uncommented to produce the final version of this child document.

%%%%%%%%%%%%%%%%%%%%%%%%%%%%%%%%%%%%%%%%%%%%%%%%%%%%%%%%%%%%%%%%%%%%%%%%%%%%%%%%
\subsection{Forwarding}
\label{sec:forward}

Different versions of the main or child documents
using compilation flags as described in \secref{sec:flags}
can be (permanently) stored in different files
for convenient compilation, viewing and distribution.
To this end, the package defines a command
to pass on compilation to a different file:

%%%%%%%%%%%%%%%%%%%%%%%%%%%%%%%%%%%%%%%%
\DescribeMacro{\childdocforward}
The command |\childdocforward| redirects processing to
another source file:
%
\begin{center}
\begin{tabular}{l}
|\input{childdoc.def}|\\
|\childdocforward[|\textit{main}|]{|\textit{dest}|}|\\
\end{tabular}
\end{center}
%
The argument \textit{dest} is the destination file
(without extension).
It should be the main file or one of the child files.
Note that further \textsf{childdoc} directives
such as |\childdocof| and |\childdocforward|
in the indicated file will be processed in this form.
The optional argument \textit{main}
passes on directly to the main file \textit{main}
while pretending to compile the child \textit{dest}.
This form behaves as if \textit{dest}
issues |\childdocof{|\textit{main}|}| right away,
and no further \textsf{childdoc} directives will be processed.

%%%%%%%%%%%%%%%%%%%%%%%%%%%%%%%%%%%%%%%%
\DescribeMacro{\...prefix}
In the alternative form |\childdocforwardprefix|,
%
\begin{center}
\begin{tabular}{l}
|\input{childdoc.def}|\\
|\childdocforwardprefix[|\textit{main}|]{|\textit{prefix}|}{|\textit{dest}|}|
\end{tabular}
\end{center}
%
the destination file is determined by a pattern
depending on the current file:
To make this work, the current file must be called
`{\textit{prefix}\hspace{0.2em}\textit{suffix}}'
with \textit{prefix} matching precisely the argument.
Processing is then passed on to the file
`{\textit{dest}\hspace{0.2em}\textit{suffix}}'.
Surely, the same effect is achieved by
directly specifying the
argument `{\textit{dest}\hspace{0.2em}\textit{suffix}}'
in the first form.
However, that requires to set up a different file
for each child. With the alternative form of the command
all these files can have exactly the same content
which simplifies setting them up and maintaining them.

For example, the following file |draft.tex|
with a compilation flag |\version| as described in \secref{sec:flags}
compiles the main document as a draft:
%
\begin{center}
\begin{tabular}{l}
|\def\version{draft}|\\
|\input{childdoc.def}|\\
|\childdocforward{|\textit{main}|}|
\end{tabular}
\end{center}
%
Likewise, the following files |final|\textit{nn}|.tex|
compile the final version of the child document
|child|\textit{nn}|.tex|:
%
\begin{center}
\begin{tabular}{l}
|\def\version{final}|\\
|\input{childdoc.def}|\\
|\childdocforwardprefix{final}{child}|
\end{tabular}
\end{center}
%

Note that when several versions of a main file and/or of each child file
are to be generated, it may be convenient to set up a |Makefile| or
shell script to automatise the process.

%%%%%%%%%%%%%%%%%%%%%%%%%%%%%%%%%%%%%%%%%%%%%%%%%%%%%%%%%%%%%%%%%%%%%%%%%%%%%%%%
\subsection{Command Line Processing}
\label{sec:commandline}

The effect of redirection files can also be achieved by invoking
the \LaTeX{} compiler with a more elaborate command line.
Most conveniently this should be done as part
of a shell script or a |Makefile|.

When using \textsf{childdoc} in the main file, the following
command lines effectively perform a redirection
(note that depending on the shell being used,
backslashes may have to be doubled: `|\|' $\to$ `|\\|'):
%
\begin{center}
|... -jobname "|\textit{target}|" |\\|"|[\textit{flags}]%
|\input{childdoc.def}\childdocforward[|\textit{main}|]{|\textit{dest}|}"|
\end{center}
%
Here \textit{target} is the name of the output file,
\textit{main} is the name of the main file
and \textit{dest} is the name of the main or child file to be processed
(all filenames without extensions).
The optional argument \textit{main} can be omitted
if \textit{main} matches \textit{dest}.
Optionally, compilation \textit{flags} can be defined via |\def| commands.
This command line makes the \TeX{} engine believe
it is compiling the file \textit{target}
whose content is specified as the latter parameter.
The provided code then forwards the processing to
\textit{main} or \textit{dest} as described in \secref{sec:forward}.

%%%%%%%%%%%%%%%%%%%%%%%%%%%%%%%%%%%%%%%%%%%%%%%%%%%%%%%%%%%%%%%%%%%%%%%%%%%%%%%%
\subsection{Include by Input}
\label{sec:input}

Including child documents by |\include| has some restrictions by design.
Most notably, the content of a child document always occupies
its own set of pages; pages cannot be shared between child documents.
Usually, this behaviour makes perfect sense
because each child document contain an essential part of the document.
However, in some situations it may be desirable to compose
a document from a collection of parts
without having mandatory page breaks between then.
For this case, the package
provides a mechanism to include parts
by |\input| which can also be processed individually.
However, by construction this mechanism
requires manual handling of the content to be output.

%%%%%%%%%%%%%%%%%%%%%%%%%%%%%%%%%%%%%%%%
\DescribeMacro{\ifchilddocmanual}
The main file should be prepared as usual, see \secref{sec:include}.
However, the document body must make a distinction
between processing of an individual part and of the main document, e.g.:
%
\begin{center}
\begin{tabular}{l}
|\ifchilddocmanual|\\
|\input{\childdocname}|\\
|\||else|\\
\textit{document body with }|\input{|\textit{part}|}|\\
|\||fi|
\end{tabular}
\end{center}
%
The conditional |\ifchilddocmanual| is true whenever
a part to be included by |\input| is being compiled,
and the name of the part is stored in |\childdocname|.

%%%%%%%%%%%%%%%%%%%%%%%%%%%%%%%%%%%%%%%%
\DescribeMacro{\childdocby}
Each part to be included by |\input| should start with:
%
\begin{center}
\begin{tabular}{l}
|\input{childdoc.def}|\\
|\childdocby{|\textit{main}|}|\\
\end{tabular}
\end{center}
%
The directive |\childdocby| is similar to |\childdocof|
described in \secref{sec:include},
but the subsequent selection of content must be done manually.
To that end, both |\ifchilddoc| and |\ifchilddocmanual|
will be true upon processing of a part,
and the name of the part is stored in |\childdocname|.
Note that |\jobname| will be set to the filename of the current part
so that each part receives an individual |.aux| file
that does not interfere with the |.aux| file(s) of the main document.
This behaviour can be altered by the alternative form
|\childdocby[*]{|\textit{main}|}| (with a non-empty optional argument)
which uses the |.aux| file of the main document
by setting |\jobname| to \textit{main}.

%%%%%%%%%%%%%%%%%%%%%%%%%%%%%%%%%%%%%%%%%%%%%%%%%%%%%%%%%%%%%%%%%%%%%%%%%%%%%%%%
\subsection{Driver Development}
\label{sec:driver}

The \textsf{childdoc} mechanism can also be use for the development
of definition files such as \LaTeX{} styles or classes.
This case differs from the above setup with multiple parts
included by |\include| in that no |\includeonly| should be invoked.
This can be achieved by starting the include file
(before |\ProvidesPackage|) with:
%
\begin{center}
\begin{tabular}{l}
|\input{childdoc.def}|\\
|\childdocforward{|\textit{main}|}|\\
\end{tabular}
\end{center}
%
or alternatively with:
%
\begin{center}
\begin{tabular}{l}
|\input{childdoc.def}|\\
|\childdocby{|\textit{main}|}|\\
\end{tabular}
\end{center}
%
Both forms have slightly different effects as described above.
The main file is prepared as usual, see \secref{sec:include}.

%%%%%%%%%%%%%%%%%%%%%%%%%%%%%%%%%%%%%%%%%%%%%%%%%%%%%%%%%%%%%%%%%%%%%%%%%%%%%%%%
\subsection{Legacy Detection}
\label{sec:detection}

The directive |\childdocmain| in the main file can detect
whether the complete document or merely a child is to be compiled
even without using the directive |\childdocof|.
This method is deprecated because it is less robust
and there is no compelling reason to use it;
it is merely provided for backward compatibility
and it may be removed in future versions.

If the detection mechanism is to be used,
it is mandatory to correctly specify
the filename of the main file as the argument of |\childdocmain|:
%
\begin{center}
\begin{tabular}{l}
|\input{childdoc.def}|\\
|\childdocmain{|\textit{main}|}|\\
\end{tabular}
\end{center}
%
If |\jobname| does not match the argument \textit{main} of |\childdocmain|,
it is assumed that |\jobname| points to the child file to be compiled.
When using |\childdocmain| with the main file specified as argument,
it suffices to start a child file
with just |\input{|\textit{main}|}|
without loading of the package and using |\childdocof|.
If instead all processing is done
with the appropriate \textsf{childdoc} directives,
the argument of \textit{main} of |\childdocmain| can be empty.

An alternative version of the command line processing described
in \secref{sec:commandline} using the detection mechanism reads:
%
\begin{center}
|... -jobname "|\textit{target}|" "|[\textit{flags}]%
[|\def\jobname{|\textit{dest}|}|]|\input{|\textit{main}|}"|
\end{center}

%%%%%%%%%%%%%%%%%%%%%%%%%%%%%%%%%%%%%%%%%%%%%%%%%%%%%%%%%%%%%%%%%%%%%%%%%%%%%%%%
\subsection{Manual Code}
\label{sec:manual}

In case one cannot be certain whether the definitions file |childdoc.def|
is installed on the target \TeX{} distribution
and one prefers not to ship it,
it is conceivable to paste a few relevant commands into the sources.

To that end, drop all statements |\input{childdoc.def}|
and perform the replacements as outlined below.
Instead of |\childdocmain{|\textit{main}|}| add the following code
to the top of the main file:
%
\begin{center}
\begin{tabular}{l}
|\||ifdefined\childdocname\endinput\||fi\newif\ifchilddoc|\\
|\edef\childdocname{\scantokens\expandafter{\jobname\noexpand}}|\\
|\def\childdocmain{|\textit{main}|}\||ifx\childdocmain\childdocname\||else|\\
|\childdoctrue\includeonly{\childdocname}\let\jobname\childdocmain\||fi|\\
\end{tabular}
\end{center}
%
Instead of |\childdocof{|\textit{main}|}| just include the main file
at the top of each child file:
%
\begin{center}
|\input{|\textit{main}|}|
\end{center}
%
A simple redirection |\childdocforward{|\textit{dest}|}| is achieved by:
%
\begin{center}
|\def\jobname{|\textit{dest}|}\input{\jobname}|
\end{center}
%
The redirection with prefix
|\childdocforwardprefix[|\textit{prefix}|]{|\textit{dest}|}|
is accomplished by:
%
\begin{center}
\begin{tabular}{l}
|{\edef\jobname{\scantokens\expandafter{\jobname\noexpand}}|\\
|\def\redirectjob |\textit{prefix}|#1~~~{\gdef\jobname{|\textit{dest}|#1}}|\\
|\expandafter\redirectjob\jobname~~~}\input{\jobname}|
\end{tabular}
\end{center}

In an alternative approach,
child documents can be compiled by a specific command line
without additional code or specific definitions:
%
\begin{center}
|... -jobname "|\textit{target}|" "|[\textit{flags}]%
|\includeonly{|\textit{dest}|}\input{|\textit{main}|}"|
\end{center}
%

%%%%%%%%%%%%%%%%%%%%%%%%%%%%%%%%%%%%%%%%%%%%%%%%%%%%%%%%%%%%%%%%%%%%%%%%%%%%%%%%
%%%%%%%%%%%%%%%%%%%%%%%%%%%%%%%%%%%%%%%%%%%%%%%%%%%%%%%%%%%%%%%%%%%%%%%%%%%%%%%%
\section{Information}

%%%%%%%%%%%%%%%%%%%%%%%%%%%%%%%%%%%%%%%%%%%%%%%%%%%%%%%%%%%%%%%%%%%%%%%%%%%%%%%%
\subsection{Copyright}

Copyright \copyright{} 2017--2018 Niklas Beisert

This work may be distributed and/or modified under the
conditions of the \LaTeX{} Project Public License, either version 1.3
of this license or (at your option) any later version.
The latest version of this license is in
  \url{http://www.latex-project.org/lppl.txt}
and version 1.3 or later is part of all distributions of \LaTeX{}
version 2005/12/01 or later.

This work has the LPPL maintenance status `maintained'.

The Current Maintainer of this work is Niklas Beisert.

This work consists of the files |README.txt|, |childdoc.ins| and |childdoc.dtx|
as well as the derived files |childdoc.def|, |cdocsamp.tex|
with |cdocsch1.tex|, |cdocsch2.tex|, |cdocspt3.tex|, |cdocspt4.tex|,
|cdocsdrf.tex|, |cdocsfn1.tex|, |cdocsfn2.tex|
as well as |childdoc.pdf|.

%%%%%%%%%%%%%%%%%%%%%%%%%%%%%%%%%%%%%%%%%%%%%%%%%%%%%%%%%%%%%%%%%%%%%%%%%%%%%%%%
\subsection{Files and Installation}

The package consists of the files:
%
\begin{center}
\begin{tabular}{ll}
    |README.txt|   & readme file \\
    |childdoc.ins| & installation file \\
    |childdoc.dtx| & source file \\
    |childdoc.def| & definition file \\
    |cdocsamp.tex| & sample main file \\
    |cdocsch1.tex| & sample include file \\
    |cdocsch2.tex| & sample include file \\
    |cdocspt3.tex| & sample part file \\
    |cdocspt4.tex| & sample part file \\
    |cdocsdrf.tex| & sample redirection file \\
    |cdocsfn1.tex| & sample redirection file \\
    |cdocsfn2.tex| & sample redirection file \\
    |childdoc.pdf| & manual
\end{tabular}
\end{center}
%
The distribution consists of the files
|README.txt|, |childdoc.ins| and |childdoc.dtx|.
%
\begin{itemize}
\item
Run (pdf)\LaTeX{} on |childdoc.dtx|
to compile the manual |childdoc.pdf| (this file).
\item
Run \LaTeX{} on |childdoc.ins| to create the definitions file |childdoc.def|
and the sample |cdocsamp.tex| with include files
|cdocsch1.tex|, |cdocsch2.tex|, |cdocspt3.tex|, |cdocspt4.tex|,
|cdocsdrf.tex|, |cdocsfn1.tex|, |cdocsfn2.tex|.
Then copy the file |childdoc.def| to an appropriate directory of your \LaTeX{}
distribution, e.g.\ \textit{texmf-root}|/tex/latex/childdoc|.
\end{itemize}

%%%%%%%%%%%%%%%%%%%%%%%%%%%%%%%%%%%%%%%%%%%%%%%%%%%%%%%%%%%%%%%%%%%%%%%%%%%%%%%%
\subsection{Related CTAN Packages}

There are several other packages which offer a similar functionality:
%
\begin{itemize}
\item
The packages
\href{http://ctan.org/pkg/docmute}{\textsf{docmute}},
\href{http://ctan.org/pkg/includex}{\textsf{includex}} and
\href{http://ctan.org/pkg/standalone}{\textsf{standalone}}
provide commands to include only the document body of
a child file thus allowing both files to be compiled individually.
\item
The packages \href{http://ctan.org/pkg/subdocs}{\textsf{subdocs}}
and \href{http://ctan.org/pkg/subfiles}{\textsf{subfiles}}
provide structures in which the main and child documents can be
encapsulated and allowing them to be compiled individually.
The inclusion mechanism is different from the conventional |\include|.
\item
The package \href{http://ctan.org/pkg/combine}{\textsf{combine}}
is an elaborate solution to combine several documents into one.
\end{itemize}
%
See also the CTAN topic \href{http://ctan.org/topic/subdocs}{\textsf{subdocs}}
for further related packages.
The present package differs from the above solutions in that
a document structure constructed with the conventional |\include| mechanism
just needs two extra commands at the top of every file
such that all constituent files can be compiled individually.

%%%%%%%%%%%%%%%%%%%%%%%%%%%%%%%%%%%%%%%%%%%%%%%%%%%%%%%%%%%%%%%%%%%%%%%%%%%%%%%%
%\subsection{Feature Suggestions}
%
%The following is a list of features which may be useful for future
%versions of this package:
%%
%\begin{itemize}
%\item
%\ldots
%\end{itemize}

%%%%%%%%%%%%%%%%%%%%%%%%%%%%%%%%%%%%%%%%%%%%%%%%%%%%%%%%%%%%%%%%%%%%%%%%%%%%%%%%
\subsection{Revision History}

%%%%%%%%%%%%%%%%%%%%%%%%%%%%%%%%%%%%%%%%
\paragraph{v2.0:} 2018/12/30

\begin{itemize}
\item
immediate forward processing
\item
added |\childdocby| mechanism
\item
manual restructured
\end{itemize}

%%%%%%%%%%%%%%%%%%%%%%%%%%%%%%%%%%%%%%%%
\paragraph{v1.6:} 2018/01/17

\begin{itemize}
\item
application for development of include files
\item
corrections to manual
\end{itemize}

%%%%%%%%%%%%%%%%%%%%%%%%%%%%%%%%%%%%%%%%
\paragraph{v1.5:} 2017/05/21

\begin{itemize}
\item
more complete structuring introduced
\item
|\childdocof| introduced
\item
|\childdoc| renamed to |\childdocmain|
\item
|\childredirect| renamed to |\childdocforward| and |\childdocforwardprefix|
and functionality expanded
\end{itemize}

%%%%%%%%%%%%%%%%%%%%%%%%%%%%%%%%%%%%%%%%
\paragraph{v1.0:} 2017/04/27

\begin{itemize}
\item
manual and install package
\item
first version published on CTAN
\end{itemize}

%%%%%%%%%%%%%%%%%%%%%%%%%%%%%%%%%%%%%%%%
\paragraph{v0.6:} 2017/04/26

\begin{itemize}
\item
redirection mechanism added
\end{itemize}

%%%%%%%%%%%%%%%%%%%%%%%%%%%%%%%%%%%%%%%%
\paragraph{v0.5:} 2017/04/26

\begin{itemize}
\item
functionality in definition file
\end{itemize}


%%%%%%%%%%%%%%%%%%%%%%%%%%%%%%%%%%%%%%%%%%%%%%%%%%%%%%%%%%%%%%%%%%%%%%%%%%%%%%%%
%%%%%%%%%%%%%%%%%%%%%%%%%%%%%%%%%%%%%%%%%%%%%%%%%%%%%%%%%%%%%%%%%%%%%%%%%%%%%%%%
%%%%%%%%%%%%%%%%%%%%%%%%%%%%%%%%%%%%%%%%%%%%%%%%%%%%%%%%%%%%%%%%%%%%%%%%%%%%%%%%
\appendix

\settowidth\MacroIndent{\rmfamily\scriptsize 000\ }

 \DocInput{childdoc.dtx}

\end{document}
%</driver>
% \fi
%
% %%%%%%%%%%%%%%%%%%%%%%%%%%%%%%%%%%%%%%%%%%%%%%%%%%%%%%%%%%%%%%%%%%%%%%%%%%%%%%
% %%%%%%%%%%%%%%%%%%%%%%%%%%%%%%%%%%%%%%%%%%%%%%%%%%%%%%%%%%%%%%%%%%%%%%%%%%%%%%
% \section{Sample}
%\iffalse
%<*samplemain>
%\fi
%
% The following presents a sample document
% with two chapters, two parts, a title page,
% a compile flag as well as three forwarding files to set the flag.
% It consists of eight |.tex| files:
% \begin{center}
% \begin{tabular}{ll}
% |cdocsamp.tex|&main file\\
% |cdocsch1.tex|&include file for chapter 1\\
% |cdocsch2.tex|&include file for chapter 2\\
% |cdocspt3.tex|&include file for part 3\\
% |cdocspt4.tex|&include file for part 4\\
% |cdocsdrf.tex|&forwarding file for main file in draft mode\\
% |cdocsfi1.tex|&forwarding file for final version of chapter 1\\
% |cdocsfi2.tex|&forwarding file for final version of chapter 2\\
% \end{tabular}
% \end{center}
% Each of the eight files can be compiled directly by the \LaTeX{} compiler.
%
% %%%%%%%%%%%%%%%%%%%%%%%%%%%%%%%%%%%%%%
% \paragraph{Main File.}
%
% The main file is called |cdocsamp.tex|.
%
% Load the \textsf{childdoc} definitions and
% declare the filename for the main document:
%    \begin{macrocode}
\input{childdoc.def}
\childdocmain{}
%    \end{macrocode}

% Optional override for |\version| flag:
%    \begin{macrocode}
%%\ifchilddoc\else\providecommand{\version}{draft}\fi
%    \end{macrocode}

% Define the default values for the |\version| flag
% (|final| for the main file and |draft| for childs):
%    \begin{macrocode}
\ifchilddoc
\providecommand{\version}{draft}
\else
\providecommand{\version}{final}
\fi
%    \end{macrocode}

% Load the standard document class:
%    \begin{macrocode}
\documentclass[12pt]{article}
%    \end{macrocode}

% Start the document body:
%    \begin{macrocode}
\begin{document}
%    \end{macrocode}

% Declare a title page.
% Print title, part of document being processed and version flag:
%    \begin{macrocode}
\addtocounter{page}{-1}
\begin{center}
{\LARGE\bfseries{}childdoc example\par}
\vspace{1cm}
\ifchilddoc
\ifchilddocmanual part\else chapter\fi:
`\childdocname' of `\childdocjob'\par
\else
main document: `\childdocjob'\par
\fi
version: \version\par
\end{center}
\newpage
%    \end{macrocode}

% Manually include selected file,
% otherwise process as usual:
%    \begin{macrocode}
\ifchilddocmanual
\section*{part `\childdocname'}
\input{\childdocname}
\else
%    \end{macrocode}

% Include the two chapters:
%    \begin{macrocode}
\include{cdocsch1}
\include{cdocsch2}
%    \end{macrocode}

% Include the two parts unless only chapters should be displayed:
%    \begin{macrocode}
\ifchilddoc\else
\section{part three}
\input{cdocspt3}
\section{part four}
\input{cdocspt4}
\fi
%    \end{macrocode}

% Process as usual until here:
%    \begin{macrocode}
\fi
%    \end{macrocode}

% End of document body:
%    \begin{macrocode}
\end{document}
%    \end{macrocode}
%\iffalse
%</samplemain>
%\fi
%
% %%%%%%%%%%%%%%%%%%%%%%%%%%%%%%%%%%%%%%
% \paragraph{Chapter Include Files.}
%
% The include files are called |cdocsch1.tex| and |cdocsch2.tex|.
%
%\iffalse
%<*samplechap1|samplechap2>
%\fi

% Optional override for |\version| flag:
%    \begin{macrocode}
%%\providecommand{\version}{final}
%    \end{macrocode}

% Include the main document:
%    \begin{macrocode}
\input{childdoc.def}
\childdocof{cdocsamp}
%    \end{macrocode}

%\iffalse
%</samplechap1|samplechap2>
%\fi
%
%\iffalse
%<*samplechap1>
%\fi
% Some text for chapter 1:
%    \begin{macrocode}
\section{one}
some text in chapter one
%    \end{macrocode}

%\iffalse
%</samplechap1>
%\fi
% Some text for chapter 2:
%\iffalse
%<*samplechap2>
%\fi
%    \begin{macrocode}
\section{two}
more text in chapter two
%    \end{macrocode}

%\iffalse
%</samplechap2>
%\fi
%
% %%%%%%%%%%%%%%%%%%%%%%%%%%%%%%%%%%%%%%
% \paragraph{Part Include Files.}
%
% The include files are called |cdocspt3.tex| and |cdocspt4.tex|.
%
%\iffalse
%<*samplepart3|samplepart4>
%\fi

% Optional override for |\version| flag:
%    \begin{macrocode}
%%\providecommand{\version}{final}
%    \end{macrocode}

% Include the main document:
%    \begin{macrocode}
\input{childdoc.def}
\childdocby{cdocsamp}
%    \end{macrocode}

%\iffalse
%</samplepart3|samplepart4>
%\fi
%
%\iffalse
%<*samplepart3>
%\fi
% Some text for part 3:
%    \begin{macrocode}
some text in part three
%    \end{macrocode}

%\iffalse
%</samplepart3>
%\fi
% Some text for part 4:
%\iffalse
%<*samplepart4>
%\fi
%    \begin{macrocode}
more text in part four
%    \end{macrocode}

%\iffalse
%</samplepart4>
%\fi
%
% %%%%%%%%%%%%%%%%%%%%%%%%%%%%%%%%%%%%%%
% \paragraph{Forwarding for a Complete Draft.}
%
% The following forwarding file |cdocsdrf.tex|
% compiles the main document in draft mode:
%\iffalse
%<*sampledraft>
%\fi
%    \begin{macrocode}
\def\version{draft}
\input{childdoc.def}
\childdocforward{cdocsamp}
%    \end{macrocode}

%\iffalse
%</sampledraft>
%\fi
%
% %%%%%%%%%%%%%%%%%%%%%%%%%%%%%%%%%%%%%%
% \paragraph{Forwarding for Final Version of the Chapters.}
%
% The following forwarding files |cdocsfn1.tex| and |cdocsfn2.tex|
% (with identical content)
% compile the final versions of the child documents
% |cdocsch1.tex| and |cdocsch2.tex|, respectively:
%\iffalse
%<*samplefinal>
%\fi
%    \begin{macrocode}
\def\version{final}
\input{childdoc.def}
\childdocforwardprefix[cdocsamp]{cdocsfn}{cdocsch}
%    \end{macrocode}

%\iffalse
%</samplefinal>
%\fi
%
% %%%%%%%%%%%%%%%%%%%%%%%%%%%%%%%%%%%%%%
% \paragraph{Command Line Processing.}
%
% The following three command lines generate the output files
% |cdocscld|, |cdocscl1| and |cdocscl2|
% which should be identical to
% |cdocsdrf|, |cdocsch1| and |cdocsfn2|, respectively:
% \begin{center}
% \begin{tabular}{l}
% |latex -jobname cdocscld \|\\
% |  "\def\version{draft}\input{childdoc.def}\childdocforward{cdocsamp}"|\\
% |latex -jobname cdocscl1 \|\\
% |  "\input{childdoc.def}\childdocforward[cdocsamp]{cdocsch1}"|\\
% |latex -jobname cdocscl2 \|\\
% |  "\def\version{final}\input{childdoc.def}\childdocforward{cdocsch2}"|
% \end{tabular}
% \end{center}
% Note that the trailing backslash on each first line
% merely continues the input to the second line
% (for convenient cut ant paste).
% Furthermore, the command |latex| can be replaced by any
% of its alternative versions such as |pdflatex|.
%
% %%%%%%%%%%%%%%%%%%%%%%%%%%%%%%%%%%%%%%%%%%%%%%%%%%%%%%%%%%%%%%%%%%%%%%%%%%%%%%
% %%%%%%%%%%%%%%%%%%%%%%%%%%%%%%%%%%%%%%%%%%%%%%%%%%%%%%%%%%%%%%%%%%%%%%%%%%%%%%
% \section{Implementation}
%\iffalse
%<*package>
%\fi
%
% This section describes the definitions file |childdoc.def|.

% The definitions cannot be loaded using |\usepackage| or |\RequirePackage|
% which has a mechanism to prevent loading a style file more than once.
% When loading the definitions by means of |\input|
% multiple instances have to be prevented manually:
%\iffalse
%This code needs to be before the `\ProvidesFile' directive
%which is defined at the beginning of this file.
%Therefore it is also placed there and commented out here.
%</package>
%<*discard>
%\fi
%    \begin{macrocode}
\ifdefined\childdocmain\endinput\fi
%    \end{macrocode}
%\iffalse
%</discard>
%<*package>
%\fi
%
% \macro{\ifchilddoc}
% \macro{\ifchilddocmanual}
% The conditional |\ifchilddoc| tells whether a
% child (true) or main (false) document is being compiled.
% The conditional |\ifchilddocmanual| tells whether
% the |\includeonly| mechanism is used (false) or
% the selection of child files must be performed manually (true).
% The definitions initialise to false:
%    \begin{macrocode}
\newif\ifchilddoc
\newif\ifchilddocmanual
%    \end{macrocode}

% \macro{\childdocname}
% \macro{\childdocjob}
% The macro |\childdocname| stores the name of the main document
% to be compiled. The macro |\childdocjob| stores the name of
% the document on which the \LaTeX{} compiler was originally invoked.
% The content of |\jobname| cannot be compared
% to filenames specified in the source due to different catcodes.
% The following code rescans |\jobname|, stores the result
% in |\childdocname| and saves a copy in |\childdocjob|:
%    \begin{macrocode}
\edef\childdocname{\scantokens\expandafter{\jobname\noexpand}}
\let\childdocjob\childdocname
%    \end{macrocode}

% \macro{\childdocdisable}
% The macro |\childdocdisable| prevents the main file
% from being processed more than once.
% At this stage, the main document command |\childdocmain|
% is assumed to be called once again where it should do nothing.
% Any subsequent call to it should prevent
% a secondary processing of the main document
% It overwrites the forwarding commands
% |\childdocof| and |\childdocforward|
% with empty macros to prevent further inclusions of the main document:
%    \begin{macrocode}
\newcommand{\childdocdisable}
{
  \renewcommand{\childdocmain}[1]{\renewcommand{\childdocmain}[1]{\endinput}}
  \renewcommand{\childdocof}[1]{}
  \renewcommand{\childdocby}[2][]{}
  \renewcommand{\childdocforward}[2][]{}
  \renewcommand{\childdocdisable}{}
}
%    \end{macrocode}

% \macro{\childdocmain}
% The macro |\childdocmain| is to be called at the top of the main file
% with nothing or the main filename (without extension) as argument.
% First, it breaks loops.
% If the argument is not empty and does not match |\childdocname|
% (which is set by the first inclusion of |childdoc.def|),
% |\ifchilddoc| is set to true, |\includeonly| is applied to the child file
% and |\jobname| is set to the main file
% (for proper handling of |.aux| files):
%    \begin{macrocode}
\newcommand{\childdocmain}[1]
{
  \childdocdisable\childdocmain{}
  \if?#1?\else
    \begingroup
      \def\childdoctmp{#1}
      \ifx\childdoctmp\childdocname
        \def\childdoctmp{}
      \else
        \def\childdoctmp
        {
          \childdoctrue
          \includeonly{\childdocname}
          \def\childdocjob{#1}
          \def\jobname{#1}
        }
      \fi
      \expandafter
    \endgroup
    \childdoctmp
  \fi
}
%    \end{macrocode}

% \macro{\childdocof}
% The command |\childdocof| redirects
% compilation to the main file |#1|.
%    \begin{macrocode}
\newcommand{\childdocof}[1]
{
  \childdocdisable
  \childdoctrue
  \includeonly{\childdocname}
  \def\jobname{#1}
  \def\childdocjob{#1}
  \input{#1}
}
%    \end{macrocode}

% \macro{\childdocby}
% The command |\childdocby| ....
%    \begin{macrocode}
\newcommand{\childdocby}[2][]
{
  \childdocdisable
  \childdoctrue
  \childdocmanualtrue
  \if?#1?\else
    \def\jobname{#2}
  \fi
  \def\childdocjob{#2}
  \input{#2}
  \endinput
}
%    \end{macrocode}

% \macro{\childdocforward}
% The command |\childdocforward| redirects
% compilation to the main file or
% (if the optional argument is given) a child file.
% Parameters are set as if the main file
% or a child file starting with |\childdocof| was compiled.
% Then compilation is handed over to the main file:
%    \begin{macrocode}
\newcommand{\childdocforward}[2][]
{
  \begingroup
    \if?#1?
      \def\childdoctmp
      {
        \def\childdocname{#2}
        \def\childdocjob{#2}
        \def\jobname{#2}
        \input{#2}
        \endinput
      }
    \else
      \def\childdoctmp
      {
        \childdocdisable
        \def\childdocname{#2}
        \childdoctrue
        \includeonly{#2}
        \def\childdocjob{#1}
        \def\jobname{#1}
        \input{#1}
        \endinput
      }
    \fi
    \expandafter
  \endgroup
  \childdoctmp
}
%    \end{macrocode}

% \macro{\childdocforwardprefix}
% The command |\childdocforwardprefix| redirects
% compilation to the main or a child file by means of a pattern.
% The prefix |#1| in the current filename is replaced by |#2|
% and the suffix of the current filename is kept
% (it is assumed that the filename does not contain the substring `|~~~|'
% which is used as a delimiter).
% Compilation is handed over to the new file by |\childdocforward|:
%    \begin{macrocode}
\newcommand{\childdocforwardprefix}[3][]
{
  \begingroup
    \def\childdocextract #2##1~~~{\def\childdoctmp{\childdocforward[#1]{#3##1}}}
    \expandafter\childdocextract\childdocname~~~
    \expandafter
  \endgroup
  \childdoctmp
}
%    \end{macrocode}

% \macro{\childdoc}
% The deprecated macro |\childdoc| is a legacy version of |\childdocmain|:
%    \begin{macrocode}
\newcommand{\childdoc}{\childdocmain}
%    \end{macrocode}

% \macro{\childdocredirect}
% The deprecated macro |\childdocredirect| is a legacy version
% of |\childdocforward| and |\childdocforwardprefix|:
%    \begin{macrocode}
\newcommand{\childdocredirect}[2][]
{
  \begingroup
    \if?#1?
      \def\childdoctmp{\childdocforward{#2}}
    \else
      \def\childdoctmp{\childdocforwardprefix{#1}{#2}}
    \fi
    \expandafter
  \endgroup
  \childdoctmp
}
%    \end{macrocode}

%\iffalse
%</package>
%\fi
%
\endinput
|\\
|\childdocforwardprefix[|\textit{main}|]{|\textit{prefix}|}{|\textit{dest}|}|
\end{tabular}
\end{center}
%
the destination file is determined by a pattern
depending on the current file:
To make this work, the current file must be called
`{\textit{prefix}\hspace{0.2em}\textit{suffix}}'
with \textit{prefix} matching precisely the argument.
Processing is then passed on to the file
`{\textit{dest}\hspace{0.2em}\textit{suffix}}'.
Surely, the same effect is achieved by
directly specifying the
argument `{\textit{dest}\hspace{0.2em}\textit{suffix}}'
in the first form.
However, that requires to set up a different file
for each child. With the alternative form of the command
all these files can have exactly the same content
which simplifies setting them up and maintaining them.

For example, the following file |draft.tex|
with a compilation flag |\version| as described in \secref{sec:flags}
compiles the main document as a draft:
%
\begin{center}
\begin{tabular}{l}
|\def\version{draft}|\\
|% \iffalse
%
% childdoc.dtx Copyright (C) 2017-2018 Niklas Beisert
%
% This work may be distributed and/or modified under the
% conditions of the LaTeX Project Public License, either version 1.3
% of this license or (at your option) any later version.
% The latest version of this license is in
%   http://www.latex-project.org/lppl.txt
% and version 1.3 or later is part of all distributions of LaTeX
% version 2005/12/01 or later.
%
% This work has the LPPL maintenance status `maintained'.
%
% The Current Maintainer of this work is Niklas Beisert.
%
% This work consists of the files childdoc.dtx and childdoc.ins
% and the derived files childdoc.def and cdocsamp.tex with
% cdocsch1.tex, cdocsch2.tex, cdocsdrf.tex, cdocsfn1.tex, cdocsfn2.tex.
%
%<package>\ifdefined\childdocmain\endinput\fi
%<package>\ProvidesFile{childdoc.def}[2018/12/30 v2.0 child document driver]
%<samplemain>\ProvidesFile{cdocsamp.tex}[2018/12/30 v2.0 sample for childdoc]
%<*driver>
%\ProvidesFile{childdoc.drv}[2018/12/30 v2.0 childdoc reference manual file]
\PassOptionsToClass{10pt,a4paper}{article}
\documentclass{ltxdoc}

\usepackage[margin=35mm]{geometry}
\usepackage{hyperref}
\usepackage{hyperxmp}
\usepackage[usenames]{color}

\hypersetup{colorlinks=true}
\hypersetup{pdfstartview=FitH}
\hypersetup{pdfpagemode=UseNone}
\hypersetup{pdfsource={}}
\hypersetup{pdflang={en-UK}}
\hypersetup{pdfcopyright={Copyright 2017-2018 Niklas Beisert.
  This work may be distributed and/or modified under the
  conditions of the LaTeX Project Public License, either version 1.3
  of this license or (at your option) any later version.}}
\hypersetup{pdflicenseurl={http://www.latex-project.org/lppl.txt}}
\hypersetup{pdfcontactaddress={ETH Zurich, ITP, HIT K,
  Wolfgang-Pauli-Strasse 27}}
\hypersetup{pdfcontactpostcode={8093}}
\hypersetup{pdfcontactcity={Zurich}}
\hypersetup{pdfcontactcountry={Switzerland}}
\hypersetup{pdfcontactemail={nbeisert@itp.phys.ethz.ch}}
\hypersetup{pdfcontacturl={http://people.phys.ethz.ch/\xmptilde nbeisert/}}

\newcommand{\secref}[1]{\hyperref[#1]{section \ref*{#1}}}

\parskip1ex
\parindent0pt
\let\olditemize\itemize
\def\itemize{\olditemize\parskip0pt}

\begin{document}

\title{The \textsf{childdoc} Package}
\hypersetup{pdftitle={The childdoc Package}}
\author{Niklas Beisert\\[2ex]
  Institut f\"ur Theoretische Physik\\
  Eidgen\"ossische Technische Hochschule Z\"urich\\
  Wolfgang-Pauli-Strasse 27, 8093 Z\"urich, Switzerland\\[1ex]
  \href{mailto:nbeisert@itp.phys.ethz.ch}
  {\texttt{nbeisert@itp.phys.ethz.ch}}}
\hypersetup{pdfauthor={Niklas Beisert}}
\hypersetup{pdfsubject={Manual for the LaTeX2e Package childdoc}}
\date{30 December 2018, \textsf{v2.0}}
\maketitle

\begin{abstract}\noindent
\textsf{childdoc} is a \LaTeXe{} package
that enables the direct compilation
of document sections included by |\include|
to individual files.
\end{abstract}

\begingroup
\parskip0ex
\tableofcontents
\endgroup

%%%%%%%%%%%%%%%%%%%%%%%%%%%%%%%%%%%%%%%%%%%%%%%%%%%%%%%%%%%%%%%%%%%%%%%%%%%%%%%%
%%%%%%%%%%%%%%%%%%%%%%%%%%%%%%%%%%%%%%%%%%%%%%%%%%%%%%%%%%%%%%%%%%%%%%%%%%%%%%%%
\section{Introduction}

\LaTeX{} provides a mechanism to structure a large document (such as a book)
into a main file and several child files (containing the chapters)
using the |\include| command.
This mechanism is beneficial for documents
which span hundreds of pages in order to
make the source file(s) more manageable.
Moreover, compilation can be restricted to
selected child files by means of the |\includeonly| command.
The latter feature can be used to reduce the compilation time while editing
(this was significantly more useful in the earlier days of \LaTeX{})
or to generate a smaller document which is easier to navigate.
Another application of |\includeonly| is to generate
documents consisting of selected parts of the complete document.

However, there are a few drawbacks of the plain |\include| mechanism:
\begin{itemize}
\item
The child files cannot be compiled on their own,
they can only be compiled via the main file.
A naive editing environment
(such as a text editor with an option
to have the current file processed by \LaTeX)
may require one to switch to the main file before compiling;
attempting to compile the child file produces errors.
\item
The main file must be modified (each time)
to adjust the |\includeonly| command
to the present needs. This easily leaves the main file in a messy state.
\item
The generated document will always carry the filename
of the main document. This is inconvenient if
several child files are to be compiled and
to be kept for distribution.
\end{itemize}

The present package provides a simple interface
to make child files individually compilable by \LaTeX{}.
Compiling a child file then has the same effect as compiling
the main file with an |\includeonly| command
to select the appropriate child.
Moreover the generated document will carry the name of the child
rather than the main file.
This resolves all three above issues.

This feature is meant to make the editing of books,
thesis documents and lecture notes somewhat more convenient.
However, the package can also be used efficiently for
composing a series of documents (such as exercise sheets)
which are typically distributed individually.
It then assists the author in generating the individual documents
(potentially in different versions)
as well as a document containing the collected series.
Another application is in developing style files
or other kinds of included material
where compilation of the style file could redirect
to a sample or test file.

%%%%%%%%%%%%%%%%%%%%%%%%%%%%%%%%%%%%%%%%%%%%%%%%%%%%%%%%%%%%%%%%%%%%%%%%%%%%%%%%
%%%%%%%%%%%%%%%%%%%%%%%%%%%%%%%%%%%%%%%%%%%%%%%%%%%%%%%%%%%%%%%%%%%%%%%%%%%%%%%%
\section{Usage}

First of all, the package \textsf{childdoc} is \emph{not} a standard
\LaTeXe{} |.sty| style file! Therefore it needs to be invoked in
a non-standard way.

%%%%%%%%%%%%%%%%%%%%%%%%%%%%%%%%%%%%%%%%%%%%%%%%%%%%%%%%%%%%%%%%%%%%%%%%%%%%%%%%
\subsection{Included Files}
\label{sec:include}

%%%%%%%%%%%%%%%%%%%%%%%%%%%%%%%%%%%%%%%%
\DescribeMacro{\childdocmain}
To use the package, add the commands
\begin{center}
\begin{tabular}{l}
|\input{childdoc.def}|\\
|\childdocmain{}|\\
\end{tabular}
\end{center}
at the very top of the main \LaTeX{} file,
in particular \emph{before} the |\documentclass| statement!
The argument of |\childdocmain| should be left empty
(but it must be present).

%%%%%%%%%%%%%%%%%%%%%%%%%%%%%%%%%%%%%%%%
\DescribeMacro{\childdocof}
Furthermore, add the commands
\begin{center}
\begin{tabular}{l}
|\input{childdoc.def}|\\
|\childdocof{|\textit{main}|}|\\
\end{tabular}
\end{center}
at the top of every child file \textit{child}
which is included by |\include{|\textit{child}|}|
from within the main file
(or at least for those files to be compiled individually).
The argument \textit{main} must be the filename of the main file.

There are a couple of
considerations in setting up the main and child documents:

%%%%%%%%%%%%%%%%%%%%%%%%%%%%%%%%%%%%%%%%
\paragraph{Restrictions.}

Please note the following restrictions:
\begin{itemize}
\item
|\childdocmain| must be called with one argument \textit{main}
to ensure compatibility with earlier version of the package.
It must either be empty (|\childdocmain{}|)
or precisely match the filename of the main file in which it is specified.
See \secref{sec:detection} for further information.
\item
The filename \textit{main} must be specified without the |.tex| extension.
\item
The filename \textit{main} is case sensitive
(even in case-insensitive file systems)
due to internal string comparison.
\item
The argument \textit{main} should be fully expanded, it cannot be a macro.
\item
Subdirectories and special characters should be avoided in filenames.
\item
The command |\childdocmain{|\textit{main}|}| must be followed by a whitespace.
It should not be followed immediately by another command
or by a comment mark `|%|'.
This is because the \TeX{} parser reads the token immediately following
the argument of |\childdocmain| and puts it
at the beginning of every child section;
however, a white\-space is ignored.
\end{itemize}

%%%%%%%%%%%%%%%%%%%%%%%%%%%%%%%%%%%%%%%%
\paragraph{Content of Main File.}

It is advisable to place all content in the child files included by |\include|.
Any output contained in the main file will appear in all child documents
unless suppressed manually;
it cannot be suppressed automatically by the |\includeonly| directive
and thus should normally be avoided.
A method to include some content in the main file
by means of conditional processing is described in \secref{sec:conditional}.

%%%%%%%%%%%%%%%%%%%%%%%%%%%%%%%%%%%%%%%%
\paragraph{Page Numbering.}

When only a part of the document is compiled,
the appropriate numbering of pages
(as well as other status parameters)
is determined from the |.aux| files.
The latter contain information from previous passes.
However this information needs to propagate through
all intermediate child documents.
Therefore the page numbering in child documents may well
be inconsistent until the complete document is compiled at least once.

A useful (if unconventional) way to always ensure a consistent
page numbering is to restart the numbering in each child document
and denote the pages by `\textit{child}|.|\textit{page}'
where \textit{child} represents the chapter/section number of the child file.
This can be achieved by the command
|\numberwithin{page}{|\textit{child}|}|
of the \textsf{amsmath} package
where \textit{child} can be |chapter| or |section|
depending on the chosen structuring.
Alternatively, one can modify the macro |\thepage| appropriately
and reset the counter |page| at the start of each child file.

%%%%%%%%%%%%%%%%%%%%%%%%%%%%%%%%%%%%%%%%%%%%%%%%%%%%%%%%%%%%%%%%%%%%%%%%%%%%%%%%
\subsection{Conditional Processing}
\label{sec:conditional}

The package provides a mechanism to compile different versions
of a document. To customise the versions further some conditional processing
can come in handy to distinguish which version is being compiled.
The package provides two macros to describe the compilation context:

%%%%%%%%%%%%%%%%%%%%%%%%%%%%%%%%%%%%%%%%
\DescribeMacro{\ifchilddoc}
The conditional |\ifchilddoc| distinguishes between the compilation of
child documents and the main document:
%
\begin{center}
|\ifchilddoc |\textit{child-code}| |[|\||else |\textit{main-code}]| \||fi|
\end{center}

%%%%%%%%%%%%%%%%%%%%%%%%%%%%%%%%%%%%%%%%
\DescribeMacro{\childdocname}
\DescribeMacro{\childdocjob}
The macro |\childdocname| contains the filename (without extension)
of the main or child file being processed.
Note that |\childdocjob| will always contain the name of the main file.

%%%%%%%%%%%%%%%%%%%%%%%%%%%%%%%%%%%%%%%%
\paragraph{Title Page.}

Conditional processing can be used to include a title or banner page
in the main document when proper precautions are taken.
Importantly, the code in the main file should ensure that the page counter
(as well as other status parameters which are stored in the |.aux| files)
takes the same value after the conditional processing.
Otherwise the page numbers may take divergent values
depending on which part is compiled.

For example, a title page could be declared by:
%
\begin{center}
\begin{tabular}{l}
|\ifchilddoc\||else|\\
|\addtocounter{page}{-1}|\\
\textit{code for title page}\\
|\newpage|\\
|\||fi|
\end{tabular}
\end{center}
%
A banner page for the child documents can be generated by:
%
\begin{center}
\begin{tabular}{l}
|\ifchilddoc|\\
|\addtocounter{page}{-1}|\\
\textit{code for banner page}\\
|\newpage|\\
|\||fi|
\end{tabular}
\end{center}
%
Here one could write a message such as:
\begin{center}
|This is the part \childdocname{} of \childdocjob{}.|
\end{center}

%%%%%%%%%%%%%%%%%%%%%%%%%%%%%%%%%%%%%%%%%%%%%%%%%%%%%%%%%%%%%%%%%%%%%%%%%%%%%%%%
\subsection{Flags}
\label{sec:flags}

The package makes it easy to generate different versions
of the main or child documents.
To this end compilation flags can be defined
and assigned different default values.
They will be particularly useful in conjunction
with the forwarding mechanism described in \secref{sec:forward}.

For example, it may be useful to have a flag |\version|
which can be set to |draft| or |final|.
The document source will contain some conditional code
depending on the value of |\version|.
Suppose further, the flag should default to |final| for the main file
and to |draft| for child files
which is a natural assignment for editing the document.
This is achieved by placing the following code
in the preamble of the main document
(below the |\childdocmain| directive):
%
\begin{center}
\begin{tabular}{l}
|\ifchilddoc|\\
|\providecommand{\version}{draft}|\\
|\||else|\\
|\providecommand{\version}{final}|\\
|\||fi|
\end{tabular}
\end{center}
%
The definition by |\providecommand| makes sure
that previous definitions are not overwritten.
Further statements |\providecommand{\version}{...}|
can thus be added before the above code to override it.

For the main file, one might add a line
(between |\childdocmain| and the above block)
%
\begin{center}
|%\ifchilddoc\||else\providecommand{\version}{draft}\||fi|
\end{center}
%
which can be uncommented to produce a draft version.
Likewise one can add a line to the very top of a child file
(above the |\childdocof{|\textit{main}|}| directive)
%
\begin{center}
|%\providecommand{\version}{final}|
\end{center}
%
which can be uncommented to produce the final version of this child document.

%%%%%%%%%%%%%%%%%%%%%%%%%%%%%%%%%%%%%%%%%%%%%%%%%%%%%%%%%%%%%%%%%%%%%%%%%%%%%%%%
\subsection{Forwarding}
\label{sec:forward}

Different versions of the main or child documents
using compilation flags as described in \secref{sec:flags}
can be (permanently) stored in different files
for convenient compilation, viewing and distribution.
To this end, the package defines a command
to pass on compilation to a different file:

%%%%%%%%%%%%%%%%%%%%%%%%%%%%%%%%%%%%%%%%
\DescribeMacro{\childdocforward}
The command |\childdocforward| redirects processing to
another source file:
%
\begin{center}
\begin{tabular}{l}
|\input{childdoc.def}|\\
|\childdocforward[|\textit{main}|]{|\textit{dest}|}|\\
\end{tabular}
\end{center}
%
The argument \textit{dest} is the destination file
(without extension).
It should be the main file or one of the child files.
Note that further \textsf{childdoc} directives
such as |\childdocof| and |\childdocforward|
in the indicated file will be processed in this form.
The optional argument \textit{main}
passes on directly to the main file \textit{main}
while pretending to compile the child \textit{dest}.
This form behaves as if \textit{dest}
issues |\childdocof{|\textit{main}|}| right away,
and no further \textsf{childdoc} directives will be processed.

%%%%%%%%%%%%%%%%%%%%%%%%%%%%%%%%%%%%%%%%
\DescribeMacro{\...prefix}
In the alternative form |\childdocforwardprefix|,
%
\begin{center}
\begin{tabular}{l}
|\input{childdoc.def}|\\
|\childdocforwardprefix[|\textit{main}|]{|\textit{prefix}|}{|\textit{dest}|}|
\end{tabular}
\end{center}
%
the destination file is determined by a pattern
depending on the current file:
To make this work, the current file must be called
`{\textit{prefix}\hspace{0.2em}\textit{suffix}}'
with \textit{prefix} matching precisely the argument.
Processing is then passed on to the file
`{\textit{dest}\hspace{0.2em}\textit{suffix}}'.
Surely, the same effect is achieved by
directly specifying the
argument `{\textit{dest}\hspace{0.2em}\textit{suffix}}'
in the first form.
However, that requires to set up a different file
for each child. With the alternative form of the command
all these files can have exactly the same content
which simplifies setting them up and maintaining them.

For example, the following file |draft.tex|
with a compilation flag |\version| as described in \secref{sec:flags}
compiles the main document as a draft:
%
\begin{center}
\begin{tabular}{l}
|\def\version{draft}|\\
|\input{childdoc.def}|\\
|\childdocforward{|\textit{main}|}|
\end{tabular}
\end{center}
%
Likewise, the following files |final|\textit{nn}|.tex|
compile the final version of the child document
|child|\textit{nn}|.tex|:
%
\begin{center}
\begin{tabular}{l}
|\def\version{final}|\\
|\input{childdoc.def}|\\
|\childdocforwardprefix{final}{child}|
\end{tabular}
\end{center}
%

Note that when several versions of a main file and/or of each child file
are to be generated, it may be convenient to set up a |Makefile| or
shell script to automatise the process.

%%%%%%%%%%%%%%%%%%%%%%%%%%%%%%%%%%%%%%%%%%%%%%%%%%%%%%%%%%%%%%%%%%%%%%%%%%%%%%%%
\subsection{Command Line Processing}
\label{sec:commandline}

The effect of redirection files can also be achieved by invoking
the \LaTeX{} compiler with a more elaborate command line.
Most conveniently this should be done as part
of a shell script or a |Makefile|.

When using \textsf{childdoc} in the main file, the following
command lines effectively perform a redirection
(note that depending on the shell being used,
backslashes may have to be doubled: `|\|' $\to$ `|\\|'):
%
\begin{center}
|... -jobname "|\textit{target}|" |\\|"|[\textit{flags}]%
|\input{childdoc.def}\childdocforward[|\textit{main}|]{|\textit{dest}|}"|
\end{center}
%
Here \textit{target} is the name of the output file,
\textit{main} is the name of the main file
and \textit{dest} is the name of the main or child file to be processed
(all filenames without extensions).
The optional argument \textit{main} can be omitted
if \textit{main} matches \textit{dest}.
Optionally, compilation \textit{flags} can be defined via |\def| commands.
This command line makes the \TeX{} engine believe
it is compiling the file \textit{target}
whose content is specified as the latter parameter.
The provided code then forwards the processing to
\textit{main} or \textit{dest} as described in \secref{sec:forward}.

%%%%%%%%%%%%%%%%%%%%%%%%%%%%%%%%%%%%%%%%%%%%%%%%%%%%%%%%%%%%%%%%%%%%%%%%%%%%%%%%
\subsection{Include by Input}
\label{sec:input}

Including child documents by |\include| has some restrictions by design.
Most notably, the content of a child document always occupies
its own set of pages; pages cannot be shared between child documents.
Usually, this behaviour makes perfect sense
because each child document contain an essential part of the document.
However, in some situations it may be desirable to compose
a document from a collection of parts
without having mandatory page breaks between then.
For this case, the package
provides a mechanism to include parts
by |\input| which can also be processed individually.
However, by construction this mechanism
requires manual handling of the content to be output.

%%%%%%%%%%%%%%%%%%%%%%%%%%%%%%%%%%%%%%%%
\DescribeMacro{\ifchilddocmanual}
The main file should be prepared as usual, see \secref{sec:include}.
However, the document body must make a distinction
between processing of an individual part and of the main document, e.g.:
%
\begin{center}
\begin{tabular}{l}
|\ifchilddocmanual|\\
|\input{\childdocname}|\\
|\||else|\\
\textit{document body with }|\input{|\textit{part}|}|\\
|\||fi|
\end{tabular}
\end{center}
%
The conditional |\ifchilddocmanual| is true whenever
a part to be included by |\input| is being compiled,
and the name of the part is stored in |\childdocname|.

%%%%%%%%%%%%%%%%%%%%%%%%%%%%%%%%%%%%%%%%
\DescribeMacro{\childdocby}
Each part to be included by |\input| should start with:
%
\begin{center}
\begin{tabular}{l}
|\input{childdoc.def}|\\
|\childdocby{|\textit{main}|}|\\
\end{tabular}
\end{center}
%
The directive |\childdocby| is similar to |\childdocof|
described in \secref{sec:include},
but the subsequent selection of content must be done manually.
To that end, both |\ifchilddoc| and |\ifchilddocmanual|
will be true upon processing of a part,
and the name of the part is stored in |\childdocname|.
Note that |\jobname| will be set to the filename of the current part
so that each part receives an individual |.aux| file
that does not interfere with the |.aux| file(s) of the main document.
This behaviour can be altered by the alternative form
|\childdocby[*]{|\textit{main}|}| (with a non-empty optional argument)
which uses the |.aux| file of the main document
by setting |\jobname| to \textit{main}.

%%%%%%%%%%%%%%%%%%%%%%%%%%%%%%%%%%%%%%%%%%%%%%%%%%%%%%%%%%%%%%%%%%%%%%%%%%%%%%%%
\subsection{Driver Development}
\label{sec:driver}

The \textsf{childdoc} mechanism can also be use for the development
of definition files such as \LaTeX{} styles or classes.
This case differs from the above setup with multiple parts
included by |\include| in that no |\includeonly| should be invoked.
This can be achieved by starting the include file
(before |\ProvidesPackage|) with:
%
\begin{center}
\begin{tabular}{l}
|\input{childdoc.def}|\\
|\childdocforward{|\textit{main}|}|\\
\end{tabular}
\end{center}
%
or alternatively with:
%
\begin{center}
\begin{tabular}{l}
|\input{childdoc.def}|\\
|\childdocby{|\textit{main}|}|\\
\end{tabular}
\end{center}
%
Both forms have slightly different effects as described above.
The main file is prepared as usual, see \secref{sec:include}.

%%%%%%%%%%%%%%%%%%%%%%%%%%%%%%%%%%%%%%%%%%%%%%%%%%%%%%%%%%%%%%%%%%%%%%%%%%%%%%%%
\subsection{Legacy Detection}
\label{sec:detection}

The directive |\childdocmain| in the main file can detect
whether the complete document or merely a child is to be compiled
even without using the directive |\childdocof|.
This method is deprecated because it is less robust
and there is no compelling reason to use it;
it is merely provided for backward compatibility
and it may be removed in future versions.

If the detection mechanism is to be used,
it is mandatory to correctly specify
the filename of the main file as the argument of |\childdocmain|:
%
\begin{center}
\begin{tabular}{l}
|\input{childdoc.def}|\\
|\childdocmain{|\textit{main}|}|\\
\end{tabular}
\end{center}
%
If |\jobname| does not match the argument \textit{main} of |\childdocmain|,
it is assumed that |\jobname| points to the child file to be compiled.
When using |\childdocmain| with the main file specified as argument,
it suffices to start a child file
with just |\input{|\textit{main}|}|
without loading of the package and using |\childdocof|.
If instead all processing is done
with the appropriate \textsf{childdoc} directives,
the argument of \textit{main} of |\childdocmain| can be empty.

An alternative version of the command line processing described
in \secref{sec:commandline} using the detection mechanism reads:
%
\begin{center}
|... -jobname "|\textit{target}|" "|[\textit{flags}]%
[|\def\jobname{|\textit{dest}|}|]|\input{|\textit{main}|}"|
\end{center}

%%%%%%%%%%%%%%%%%%%%%%%%%%%%%%%%%%%%%%%%%%%%%%%%%%%%%%%%%%%%%%%%%%%%%%%%%%%%%%%%
\subsection{Manual Code}
\label{sec:manual}

In case one cannot be certain whether the definitions file |childdoc.def|
is installed on the target \TeX{} distribution
and one prefers not to ship it,
it is conceivable to paste a few relevant commands into the sources.

To that end, drop all statements |\input{childdoc.def}|
and perform the replacements as outlined below.
Instead of |\childdocmain{|\textit{main}|}| add the following code
to the top of the main file:
%
\begin{center}
\begin{tabular}{l}
|\||ifdefined\childdocname\endinput\||fi\newif\ifchilddoc|\\
|\edef\childdocname{\scantokens\expandafter{\jobname\noexpand}}|\\
|\def\childdocmain{|\textit{main}|}\||ifx\childdocmain\childdocname\||else|\\
|\childdoctrue\includeonly{\childdocname}\let\jobname\childdocmain\||fi|\\
\end{tabular}
\end{center}
%
Instead of |\childdocof{|\textit{main}|}| just include the main file
at the top of each child file:
%
\begin{center}
|\input{|\textit{main}|}|
\end{center}
%
A simple redirection |\childdocforward{|\textit{dest}|}| is achieved by:
%
\begin{center}
|\def\jobname{|\textit{dest}|}\input{\jobname}|
\end{center}
%
The redirection with prefix
|\childdocforwardprefix[|\textit{prefix}|]{|\textit{dest}|}|
is accomplished by:
%
\begin{center}
\begin{tabular}{l}
|{\edef\jobname{\scantokens\expandafter{\jobname\noexpand}}|\\
|\def\redirectjob |\textit{prefix}|#1~~~{\gdef\jobname{|\textit{dest}|#1}}|\\
|\expandafter\redirectjob\jobname~~~}\input{\jobname}|
\end{tabular}
\end{center}

In an alternative approach,
child documents can be compiled by a specific command line
without additional code or specific definitions:
%
\begin{center}
|... -jobname "|\textit{target}|" "|[\textit{flags}]%
|\includeonly{|\textit{dest}|}\input{|\textit{main}|}"|
\end{center}
%

%%%%%%%%%%%%%%%%%%%%%%%%%%%%%%%%%%%%%%%%%%%%%%%%%%%%%%%%%%%%%%%%%%%%%%%%%%%%%%%%
%%%%%%%%%%%%%%%%%%%%%%%%%%%%%%%%%%%%%%%%%%%%%%%%%%%%%%%%%%%%%%%%%%%%%%%%%%%%%%%%
\section{Information}

%%%%%%%%%%%%%%%%%%%%%%%%%%%%%%%%%%%%%%%%%%%%%%%%%%%%%%%%%%%%%%%%%%%%%%%%%%%%%%%%
\subsection{Copyright}

Copyright \copyright{} 2017--2018 Niklas Beisert

This work may be distributed and/or modified under the
conditions of the \LaTeX{} Project Public License, either version 1.3
of this license or (at your option) any later version.
The latest version of this license is in
  \url{http://www.latex-project.org/lppl.txt}
and version 1.3 or later is part of all distributions of \LaTeX{}
version 2005/12/01 or later.

This work has the LPPL maintenance status `maintained'.

The Current Maintainer of this work is Niklas Beisert.

This work consists of the files |README.txt|, |childdoc.ins| and |childdoc.dtx|
as well as the derived files |childdoc.def|, |cdocsamp.tex|
with |cdocsch1.tex|, |cdocsch2.tex|, |cdocspt3.tex|, |cdocspt4.tex|,
|cdocsdrf.tex|, |cdocsfn1.tex|, |cdocsfn2.tex|
as well as |childdoc.pdf|.

%%%%%%%%%%%%%%%%%%%%%%%%%%%%%%%%%%%%%%%%%%%%%%%%%%%%%%%%%%%%%%%%%%%%%%%%%%%%%%%%
\subsection{Files and Installation}

The package consists of the files:
%
\begin{center}
\begin{tabular}{ll}
    |README.txt|   & readme file \\
    |childdoc.ins| & installation file \\
    |childdoc.dtx| & source file \\
    |childdoc.def| & definition file \\
    |cdocsamp.tex| & sample main file \\
    |cdocsch1.tex| & sample include file \\
    |cdocsch2.tex| & sample include file \\
    |cdocspt3.tex| & sample part file \\
    |cdocspt4.tex| & sample part file \\
    |cdocsdrf.tex| & sample redirection file \\
    |cdocsfn1.tex| & sample redirection file \\
    |cdocsfn2.tex| & sample redirection file \\
    |childdoc.pdf| & manual
\end{tabular}
\end{center}
%
The distribution consists of the files
|README.txt|, |childdoc.ins| and |childdoc.dtx|.
%
\begin{itemize}
\item
Run (pdf)\LaTeX{} on |childdoc.dtx|
to compile the manual |childdoc.pdf| (this file).
\item
Run \LaTeX{} on |childdoc.ins| to create the definitions file |childdoc.def|
and the sample |cdocsamp.tex| with include files
|cdocsch1.tex|, |cdocsch2.tex|, |cdocspt3.tex|, |cdocspt4.tex|,
|cdocsdrf.tex|, |cdocsfn1.tex|, |cdocsfn2.tex|.
Then copy the file |childdoc.def| to an appropriate directory of your \LaTeX{}
distribution, e.g.\ \textit{texmf-root}|/tex/latex/childdoc|.
\end{itemize}

%%%%%%%%%%%%%%%%%%%%%%%%%%%%%%%%%%%%%%%%%%%%%%%%%%%%%%%%%%%%%%%%%%%%%%%%%%%%%%%%
\subsection{Related CTAN Packages}

There are several other packages which offer a similar functionality:
%
\begin{itemize}
\item
The packages
\href{http://ctan.org/pkg/docmute}{\textsf{docmute}},
\href{http://ctan.org/pkg/includex}{\textsf{includex}} and
\href{http://ctan.org/pkg/standalone}{\textsf{standalone}}
provide commands to include only the document body of
a child file thus allowing both files to be compiled individually.
\item
The packages \href{http://ctan.org/pkg/subdocs}{\textsf{subdocs}}
and \href{http://ctan.org/pkg/subfiles}{\textsf{subfiles}}
provide structures in which the main and child documents can be
encapsulated and allowing them to be compiled individually.
The inclusion mechanism is different from the conventional |\include|.
\item
The package \href{http://ctan.org/pkg/combine}{\textsf{combine}}
is an elaborate solution to combine several documents into one.
\end{itemize}
%
See also the CTAN topic \href{http://ctan.org/topic/subdocs}{\textsf{subdocs}}
for further related packages.
The present package differs from the above solutions in that
a document structure constructed with the conventional |\include| mechanism
just needs two extra commands at the top of every file
such that all constituent files can be compiled individually.

%%%%%%%%%%%%%%%%%%%%%%%%%%%%%%%%%%%%%%%%%%%%%%%%%%%%%%%%%%%%%%%%%%%%%%%%%%%%%%%%
%\subsection{Feature Suggestions}
%
%The following is a list of features which may be useful for future
%versions of this package:
%%
%\begin{itemize}
%\item
%\ldots
%\end{itemize}

%%%%%%%%%%%%%%%%%%%%%%%%%%%%%%%%%%%%%%%%%%%%%%%%%%%%%%%%%%%%%%%%%%%%%%%%%%%%%%%%
\subsection{Revision History}

%%%%%%%%%%%%%%%%%%%%%%%%%%%%%%%%%%%%%%%%
\paragraph{v2.0:} 2018/12/30

\begin{itemize}
\item
immediate forward processing
\item
added |\childdocby| mechanism
\item
manual restructured
\end{itemize}

%%%%%%%%%%%%%%%%%%%%%%%%%%%%%%%%%%%%%%%%
\paragraph{v1.6:} 2018/01/17

\begin{itemize}
\item
application for development of include files
\item
corrections to manual
\end{itemize}

%%%%%%%%%%%%%%%%%%%%%%%%%%%%%%%%%%%%%%%%
\paragraph{v1.5:} 2017/05/21

\begin{itemize}
\item
more complete structuring introduced
\item
|\childdocof| introduced
\item
|\childdoc| renamed to |\childdocmain|
\item
|\childredirect| renamed to |\childdocforward| and |\childdocforwardprefix|
and functionality expanded
\end{itemize}

%%%%%%%%%%%%%%%%%%%%%%%%%%%%%%%%%%%%%%%%
\paragraph{v1.0:} 2017/04/27

\begin{itemize}
\item
manual and install package
\item
first version published on CTAN
\end{itemize}

%%%%%%%%%%%%%%%%%%%%%%%%%%%%%%%%%%%%%%%%
\paragraph{v0.6:} 2017/04/26

\begin{itemize}
\item
redirection mechanism added
\end{itemize}

%%%%%%%%%%%%%%%%%%%%%%%%%%%%%%%%%%%%%%%%
\paragraph{v0.5:} 2017/04/26

\begin{itemize}
\item
functionality in definition file
\end{itemize}


%%%%%%%%%%%%%%%%%%%%%%%%%%%%%%%%%%%%%%%%%%%%%%%%%%%%%%%%%%%%%%%%%%%%%%%%%%%%%%%%
%%%%%%%%%%%%%%%%%%%%%%%%%%%%%%%%%%%%%%%%%%%%%%%%%%%%%%%%%%%%%%%%%%%%%%%%%%%%%%%%
%%%%%%%%%%%%%%%%%%%%%%%%%%%%%%%%%%%%%%%%%%%%%%%%%%%%%%%%%%%%%%%%%%%%%%%%%%%%%%%%
\appendix

\settowidth\MacroIndent{\rmfamily\scriptsize 000\ }

 \DocInput{childdoc.dtx}

\end{document}
%</driver>
% \fi
%
% %%%%%%%%%%%%%%%%%%%%%%%%%%%%%%%%%%%%%%%%%%%%%%%%%%%%%%%%%%%%%%%%%%%%%%%%%%%%%%
% %%%%%%%%%%%%%%%%%%%%%%%%%%%%%%%%%%%%%%%%%%%%%%%%%%%%%%%%%%%%%%%%%%%%%%%%%%%%%%
% \section{Sample}
%\iffalse
%<*samplemain>
%\fi
%
% The following presents a sample document
% with two chapters, two parts, a title page,
% a compile flag as well as three forwarding files to set the flag.
% It consists of eight |.tex| files:
% \begin{center}
% \begin{tabular}{ll}
% |cdocsamp.tex|&main file\\
% |cdocsch1.tex|&include file for chapter 1\\
% |cdocsch2.tex|&include file for chapter 2\\
% |cdocspt3.tex|&include file for part 3\\
% |cdocspt4.tex|&include file for part 4\\
% |cdocsdrf.tex|&forwarding file for main file in draft mode\\
% |cdocsfi1.tex|&forwarding file for final version of chapter 1\\
% |cdocsfi2.tex|&forwarding file for final version of chapter 2\\
% \end{tabular}
% \end{center}
% Each of the eight files can be compiled directly by the \LaTeX{} compiler.
%
% %%%%%%%%%%%%%%%%%%%%%%%%%%%%%%%%%%%%%%
% \paragraph{Main File.}
%
% The main file is called |cdocsamp.tex|.
%
% Load the \textsf{childdoc} definitions and
% declare the filename for the main document:
%    \begin{macrocode}
\input{childdoc.def}
\childdocmain{}
%    \end{macrocode}

% Optional override for |\version| flag:
%    \begin{macrocode}
%%\ifchilddoc\else\providecommand{\version}{draft}\fi
%    \end{macrocode}

% Define the default values for the |\version| flag
% (|final| for the main file and |draft| for childs):
%    \begin{macrocode}
\ifchilddoc
\providecommand{\version}{draft}
\else
\providecommand{\version}{final}
\fi
%    \end{macrocode}

% Load the standard document class:
%    \begin{macrocode}
\documentclass[12pt]{article}
%    \end{macrocode}

% Start the document body:
%    \begin{macrocode}
\begin{document}
%    \end{macrocode}

% Declare a title page.
% Print title, part of document being processed and version flag:
%    \begin{macrocode}
\addtocounter{page}{-1}
\begin{center}
{\LARGE\bfseries{}childdoc example\par}
\vspace{1cm}
\ifchilddoc
\ifchilddocmanual part\else chapter\fi:
`\childdocname' of `\childdocjob'\par
\else
main document: `\childdocjob'\par
\fi
version: \version\par
\end{center}
\newpage
%    \end{macrocode}

% Manually include selected file,
% otherwise process as usual:
%    \begin{macrocode}
\ifchilddocmanual
\section*{part `\childdocname'}
\input{\childdocname}
\else
%    \end{macrocode}

% Include the two chapters:
%    \begin{macrocode}
\include{cdocsch1}
\include{cdocsch2}
%    \end{macrocode}

% Include the two parts unless only chapters should be displayed:
%    \begin{macrocode}
\ifchilddoc\else
\section{part three}
\input{cdocspt3}
\section{part four}
\input{cdocspt4}
\fi
%    \end{macrocode}

% Process as usual until here:
%    \begin{macrocode}
\fi
%    \end{macrocode}

% End of document body:
%    \begin{macrocode}
\end{document}
%    \end{macrocode}
%\iffalse
%</samplemain>
%\fi
%
% %%%%%%%%%%%%%%%%%%%%%%%%%%%%%%%%%%%%%%
% \paragraph{Chapter Include Files.}
%
% The include files are called |cdocsch1.tex| and |cdocsch2.tex|.
%
%\iffalse
%<*samplechap1|samplechap2>
%\fi

% Optional override for |\version| flag:
%    \begin{macrocode}
%%\providecommand{\version}{final}
%    \end{macrocode}

% Include the main document:
%    \begin{macrocode}
\input{childdoc.def}
\childdocof{cdocsamp}
%    \end{macrocode}

%\iffalse
%</samplechap1|samplechap2>
%\fi
%
%\iffalse
%<*samplechap1>
%\fi
% Some text for chapter 1:
%    \begin{macrocode}
\section{one}
some text in chapter one
%    \end{macrocode}

%\iffalse
%</samplechap1>
%\fi
% Some text for chapter 2:
%\iffalse
%<*samplechap2>
%\fi
%    \begin{macrocode}
\section{two}
more text in chapter two
%    \end{macrocode}

%\iffalse
%</samplechap2>
%\fi
%
% %%%%%%%%%%%%%%%%%%%%%%%%%%%%%%%%%%%%%%
% \paragraph{Part Include Files.}
%
% The include files are called |cdocspt3.tex| and |cdocspt4.tex|.
%
%\iffalse
%<*samplepart3|samplepart4>
%\fi

% Optional override for |\version| flag:
%    \begin{macrocode}
%%\providecommand{\version}{final}
%    \end{macrocode}

% Include the main document:
%    \begin{macrocode}
\input{childdoc.def}
\childdocby{cdocsamp}
%    \end{macrocode}

%\iffalse
%</samplepart3|samplepart4>
%\fi
%
%\iffalse
%<*samplepart3>
%\fi
% Some text for part 3:
%    \begin{macrocode}
some text in part three
%    \end{macrocode}

%\iffalse
%</samplepart3>
%\fi
% Some text for part 4:
%\iffalse
%<*samplepart4>
%\fi
%    \begin{macrocode}
more text in part four
%    \end{macrocode}

%\iffalse
%</samplepart4>
%\fi
%
% %%%%%%%%%%%%%%%%%%%%%%%%%%%%%%%%%%%%%%
% \paragraph{Forwarding for a Complete Draft.}
%
% The following forwarding file |cdocsdrf.tex|
% compiles the main document in draft mode:
%\iffalse
%<*sampledraft>
%\fi
%    \begin{macrocode}
\def\version{draft}
\input{childdoc.def}
\childdocforward{cdocsamp}
%    \end{macrocode}

%\iffalse
%</sampledraft>
%\fi
%
% %%%%%%%%%%%%%%%%%%%%%%%%%%%%%%%%%%%%%%
% \paragraph{Forwarding for Final Version of the Chapters.}
%
% The following forwarding files |cdocsfn1.tex| and |cdocsfn2.tex|
% (with identical content)
% compile the final versions of the child documents
% |cdocsch1.tex| and |cdocsch2.tex|, respectively:
%\iffalse
%<*samplefinal>
%\fi
%    \begin{macrocode}
\def\version{final}
\input{childdoc.def}
\childdocforwardprefix[cdocsamp]{cdocsfn}{cdocsch}
%    \end{macrocode}

%\iffalse
%</samplefinal>
%\fi
%
% %%%%%%%%%%%%%%%%%%%%%%%%%%%%%%%%%%%%%%
% \paragraph{Command Line Processing.}
%
% The following three command lines generate the output files
% |cdocscld|, |cdocscl1| and |cdocscl2|
% which should be identical to
% |cdocsdrf|, |cdocsch1| and |cdocsfn2|, respectively:
% \begin{center}
% \begin{tabular}{l}
% |latex -jobname cdocscld \|\\
% |  "\def\version{draft}\input{childdoc.def}\childdocforward{cdocsamp}"|\\
% |latex -jobname cdocscl1 \|\\
% |  "\input{childdoc.def}\childdocforward[cdocsamp]{cdocsch1}"|\\
% |latex -jobname cdocscl2 \|\\
% |  "\def\version{final}\input{childdoc.def}\childdocforward{cdocsch2}"|
% \end{tabular}
% \end{center}
% Note that the trailing backslash on each first line
% merely continues the input to the second line
% (for convenient cut ant paste).
% Furthermore, the command |latex| can be replaced by any
% of its alternative versions such as |pdflatex|.
%
% %%%%%%%%%%%%%%%%%%%%%%%%%%%%%%%%%%%%%%%%%%%%%%%%%%%%%%%%%%%%%%%%%%%%%%%%%%%%%%
% %%%%%%%%%%%%%%%%%%%%%%%%%%%%%%%%%%%%%%%%%%%%%%%%%%%%%%%%%%%%%%%%%%%%%%%%%%%%%%
% \section{Implementation}
%\iffalse
%<*package>
%\fi
%
% This section describes the definitions file |childdoc.def|.

% The definitions cannot be loaded using |\usepackage| or |\RequirePackage|
% which has a mechanism to prevent loading a style file more than once.
% When loading the definitions by means of |\input|
% multiple instances have to be prevented manually:
%\iffalse
%This code needs to be before the `\ProvidesFile' directive
%which is defined at the beginning of this file.
%Therefore it is also placed there and commented out here.
%</package>
%<*discard>
%\fi
%    \begin{macrocode}
\ifdefined\childdocmain\endinput\fi
%    \end{macrocode}
%\iffalse
%</discard>
%<*package>
%\fi
%
% \macro{\ifchilddoc}
% \macro{\ifchilddocmanual}
% The conditional |\ifchilddoc| tells whether a
% child (true) or main (false) document is being compiled.
% The conditional |\ifchilddocmanual| tells whether
% the |\includeonly| mechanism is used (false) or
% the selection of child files must be performed manually (true).
% The definitions initialise to false:
%    \begin{macrocode}
\newif\ifchilddoc
\newif\ifchilddocmanual
%    \end{macrocode}

% \macro{\childdocname}
% \macro{\childdocjob}
% The macro |\childdocname| stores the name of the main document
% to be compiled. The macro |\childdocjob| stores the name of
% the document on which the \LaTeX{} compiler was originally invoked.
% The content of |\jobname| cannot be compared
% to filenames specified in the source due to different catcodes.
% The following code rescans |\jobname|, stores the result
% in |\childdocname| and saves a copy in |\childdocjob|:
%    \begin{macrocode}
\edef\childdocname{\scantokens\expandafter{\jobname\noexpand}}
\let\childdocjob\childdocname
%    \end{macrocode}

% \macro{\childdocdisable}
% The macro |\childdocdisable| prevents the main file
% from being processed more than once.
% At this stage, the main document command |\childdocmain|
% is assumed to be called once again where it should do nothing.
% Any subsequent call to it should prevent
% a secondary processing of the main document
% It overwrites the forwarding commands
% |\childdocof| and |\childdocforward|
% with empty macros to prevent further inclusions of the main document:
%    \begin{macrocode}
\newcommand{\childdocdisable}
{
  \renewcommand{\childdocmain}[1]{\renewcommand{\childdocmain}[1]{\endinput}}
  \renewcommand{\childdocof}[1]{}
  \renewcommand{\childdocby}[2][]{}
  \renewcommand{\childdocforward}[2][]{}
  \renewcommand{\childdocdisable}{}
}
%    \end{macrocode}

% \macro{\childdocmain}
% The macro |\childdocmain| is to be called at the top of the main file
% with nothing or the main filename (without extension) as argument.
% First, it breaks loops.
% If the argument is not empty and does not match |\childdocname|
% (which is set by the first inclusion of |childdoc.def|),
% |\ifchilddoc| is set to true, |\includeonly| is applied to the child file
% and |\jobname| is set to the main file
% (for proper handling of |.aux| files):
%    \begin{macrocode}
\newcommand{\childdocmain}[1]
{
  \childdocdisable\childdocmain{}
  \if?#1?\else
    \begingroup
      \def\childdoctmp{#1}
      \ifx\childdoctmp\childdocname
        \def\childdoctmp{}
      \else
        \def\childdoctmp
        {
          \childdoctrue
          \includeonly{\childdocname}
          \def\childdocjob{#1}
          \def\jobname{#1}
        }
      \fi
      \expandafter
    \endgroup
    \childdoctmp
  \fi
}
%    \end{macrocode}

% \macro{\childdocof}
% The command |\childdocof| redirects
% compilation to the main file |#1|.
%    \begin{macrocode}
\newcommand{\childdocof}[1]
{
  \childdocdisable
  \childdoctrue
  \includeonly{\childdocname}
  \def\jobname{#1}
  \def\childdocjob{#1}
  \input{#1}
}
%    \end{macrocode}

% \macro{\childdocby}
% The command |\childdocby| ....
%    \begin{macrocode}
\newcommand{\childdocby}[2][]
{
  \childdocdisable
  \childdoctrue
  \childdocmanualtrue
  \if?#1?\else
    \def\jobname{#2}
  \fi
  \def\childdocjob{#2}
  \input{#2}
  \endinput
}
%    \end{macrocode}

% \macro{\childdocforward}
% The command |\childdocforward| redirects
% compilation to the main file or
% (if the optional argument is given) a child file.
% Parameters are set as if the main file
% or a child file starting with |\childdocof| was compiled.
% Then compilation is handed over to the main file:
%    \begin{macrocode}
\newcommand{\childdocforward}[2][]
{
  \begingroup
    \if?#1?
      \def\childdoctmp
      {
        \def\childdocname{#2}
        \def\childdocjob{#2}
        \def\jobname{#2}
        \input{#2}
        \endinput
      }
    \else
      \def\childdoctmp
      {
        \childdocdisable
        \def\childdocname{#2}
        \childdoctrue
        \includeonly{#2}
        \def\childdocjob{#1}
        \def\jobname{#1}
        \input{#1}
        \endinput
      }
    \fi
    \expandafter
  \endgroup
  \childdoctmp
}
%    \end{macrocode}

% \macro{\childdocforwardprefix}
% The command |\childdocforwardprefix| redirects
% compilation to the main or a child file by means of a pattern.
% The prefix |#1| in the current filename is replaced by |#2|
% and the suffix of the current filename is kept
% (it is assumed that the filename does not contain the substring `|~~~|'
% which is used as a delimiter).
% Compilation is handed over to the new file by |\childdocforward|:
%    \begin{macrocode}
\newcommand{\childdocforwardprefix}[3][]
{
  \begingroup
    \def\childdocextract #2##1~~~{\def\childdoctmp{\childdocforward[#1]{#3##1}}}
    \expandafter\childdocextract\childdocname~~~
    \expandafter
  \endgroup
  \childdoctmp
}
%    \end{macrocode}

% \macro{\childdoc}
% The deprecated macro |\childdoc| is a legacy version of |\childdocmain|:
%    \begin{macrocode}
\newcommand{\childdoc}{\childdocmain}
%    \end{macrocode}

% \macro{\childdocredirect}
% The deprecated macro |\childdocredirect| is a legacy version
% of |\childdocforward| and |\childdocforwardprefix|:
%    \begin{macrocode}
\newcommand{\childdocredirect}[2][]
{
  \begingroup
    \if?#1?
      \def\childdoctmp{\childdocforward{#2}}
    \else
      \def\childdoctmp{\childdocforwardprefix{#1}{#2}}
    \fi
    \expandafter
  \endgroup
  \childdoctmp
}
%    \end{macrocode}

%\iffalse
%</package>
%\fi
%
\endinput
|\\
|\childdocforward{|\textit{main}|}|
\end{tabular}
\end{center}
%
Likewise, the following files |final|\textit{nn}|.tex|
compile the final version of the child document
|child|\textit{nn}|.tex|:
%
\begin{center}
\begin{tabular}{l}
|\def\version{final}|\\
|% \iffalse
%
% childdoc.dtx Copyright (C) 2017-2018 Niklas Beisert
%
% This work may be distributed and/or modified under the
% conditions of the LaTeX Project Public License, either version 1.3
% of this license or (at your option) any later version.
% The latest version of this license is in
%   http://www.latex-project.org/lppl.txt
% and version 1.3 or later is part of all distributions of LaTeX
% version 2005/12/01 or later.
%
% This work has the LPPL maintenance status `maintained'.
%
% The Current Maintainer of this work is Niklas Beisert.
%
% This work consists of the files childdoc.dtx and childdoc.ins
% and the derived files childdoc.def and cdocsamp.tex with
% cdocsch1.tex, cdocsch2.tex, cdocsdrf.tex, cdocsfn1.tex, cdocsfn2.tex.
%
%<package>\ifdefined\childdocmain\endinput\fi
%<package>\ProvidesFile{childdoc.def}[2018/12/30 v2.0 child document driver]
%<samplemain>\ProvidesFile{cdocsamp.tex}[2018/12/30 v2.0 sample for childdoc]
%<*driver>
%\ProvidesFile{childdoc.drv}[2018/12/30 v2.0 childdoc reference manual file]
\PassOptionsToClass{10pt,a4paper}{article}
\documentclass{ltxdoc}

\usepackage[margin=35mm]{geometry}
\usepackage{hyperref}
\usepackage{hyperxmp}
\usepackage[usenames]{color}

\hypersetup{colorlinks=true}
\hypersetup{pdfstartview=FitH}
\hypersetup{pdfpagemode=UseNone}
\hypersetup{pdfsource={}}
\hypersetup{pdflang={en-UK}}
\hypersetup{pdfcopyright={Copyright 2017-2018 Niklas Beisert.
  This work may be distributed and/or modified under the
  conditions of the LaTeX Project Public License, either version 1.3
  of this license or (at your option) any later version.}}
\hypersetup{pdflicenseurl={http://www.latex-project.org/lppl.txt}}
\hypersetup{pdfcontactaddress={ETH Zurich, ITP, HIT K,
  Wolfgang-Pauli-Strasse 27}}
\hypersetup{pdfcontactpostcode={8093}}
\hypersetup{pdfcontactcity={Zurich}}
\hypersetup{pdfcontactcountry={Switzerland}}
\hypersetup{pdfcontactemail={nbeisert@itp.phys.ethz.ch}}
\hypersetup{pdfcontacturl={http://people.phys.ethz.ch/\xmptilde nbeisert/}}

\newcommand{\secref}[1]{\hyperref[#1]{section \ref*{#1}}}

\parskip1ex
\parindent0pt
\let\olditemize\itemize
\def\itemize{\olditemize\parskip0pt}

\begin{document}

\title{The \textsf{childdoc} Package}
\hypersetup{pdftitle={The childdoc Package}}
\author{Niklas Beisert\\[2ex]
  Institut f\"ur Theoretische Physik\\
  Eidgen\"ossische Technische Hochschule Z\"urich\\
  Wolfgang-Pauli-Strasse 27, 8093 Z\"urich, Switzerland\\[1ex]
  \href{mailto:nbeisert@itp.phys.ethz.ch}
  {\texttt{nbeisert@itp.phys.ethz.ch}}}
\hypersetup{pdfauthor={Niklas Beisert}}
\hypersetup{pdfsubject={Manual for the LaTeX2e Package childdoc}}
\date{30 December 2018, \textsf{v2.0}}
\maketitle

\begin{abstract}\noindent
\textsf{childdoc} is a \LaTeXe{} package
that enables the direct compilation
of document sections included by |\include|
to individual files.
\end{abstract}

\begingroup
\parskip0ex
\tableofcontents
\endgroup

%%%%%%%%%%%%%%%%%%%%%%%%%%%%%%%%%%%%%%%%%%%%%%%%%%%%%%%%%%%%%%%%%%%%%%%%%%%%%%%%
%%%%%%%%%%%%%%%%%%%%%%%%%%%%%%%%%%%%%%%%%%%%%%%%%%%%%%%%%%%%%%%%%%%%%%%%%%%%%%%%
\section{Introduction}

\LaTeX{} provides a mechanism to structure a large document (such as a book)
into a main file and several child files (containing the chapters)
using the |\include| command.
This mechanism is beneficial for documents
which span hundreds of pages in order to
make the source file(s) more manageable.
Moreover, compilation can be restricted to
selected child files by means of the |\includeonly| command.
The latter feature can be used to reduce the compilation time while editing
(this was significantly more useful in the earlier days of \LaTeX{})
or to generate a smaller document which is easier to navigate.
Another application of |\includeonly| is to generate
documents consisting of selected parts of the complete document.

However, there are a few drawbacks of the plain |\include| mechanism:
\begin{itemize}
\item
The child files cannot be compiled on their own,
they can only be compiled via the main file.
A naive editing environment
(such as a text editor with an option
to have the current file processed by \LaTeX)
may require one to switch to the main file before compiling;
attempting to compile the child file produces errors.
\item
The main file must be modified (each time)
to adjust the |\includeonly| command
to the present needs. This easily leaves the main file in a messy state.
\item
The generated document will always carry the filename
of the main document. This is inconvenient if
several child files are to be compiled and
to be kept for distribution.
\end{itemize}

The present package provides a simple interface
to make child files individually compilable by \LaTeX{}.
Compiling a child file then has the same effect as compiling
the main file with an |\includeonly| command
to select the appropriate child.
Moreover the generated document will carry the name of the child
rather than the main file.
This resolves all three above issues.

This feature is meant to make the editing of books,
thesis documents and lecture notes somewhat more convenient.
However, the package can also be used efficiently for
composing a series of documents (such as exercise sheets)
which are typically distributed individually.
It then assists the author in generating the individual documents
(potentially in different versions)
as well as a document containing the collected series.
Another application is in developing style files
or other kinds of included material
where compilation of the style file could redirect
to a sample or test file.

%%%%%%%%%%%%%%%%%%%%%%%%%%%%%%%%%%%%%%%%%%%%%%%%%%%%%%%%%%%%%%%%%%%%%%%%%%%%%%%%
%%%%%%%%%%%%%%%%%%%%%%%%%%%%%%%%%%%%%%%%%%%%%%%%%%%%%%%%%%%%%%%%%%%%%%%%%%%%%%%%
\section{Usage}

First of all, the package \textsf{childdoc} is \emph{not} a standard
\LaTeXe{} |.sty| style file! Therefore it needs to be invoked in
a non-standard way.

%%%%%%%%%%%%%%%%%%%%%%%%%%%%%%%%%%%%%%%%%%%%%%%%%%%%%%%%%%%%%%%%%%%%%%%%%%%%%%%%
\subsection{Included Files}
\label{sec:include}

%%%%%%%%%%%%%%%%%%%%%%%%%%%%%%%%%%%%%%%%
\DescribeMacro{\childdocmain}
To use the package, add the commands
\begin{center}
\begin{tabular}{l}
|\input{childdoc.def}|\\
|\childdocmain{}|\\
\end{tabular}
\end{center}
at the very top of the main \LaTeX{} file,
in particular \emph{before} the |\documentclass| statement!
The argument of |\childdocmain| should be left empty
(but it must be present).

%%%%%%%%%%%%%%%%%%%%%%%%%%%%%%%%%%%%%%%%
\DescribeMacro{\childdocof}
Furthermore, add the commands
\begin{center}
\begin{tabular}{l}
|\input{childdoc.def}|\\
|\childdocof{|\textit{main}|}|\\
\end{tabular}
\end{center}
at the top of every child file \textit{child}
which is included by |\include{|\textit{child}|}|
from within the main file
(or at least for those files to be compiled individually).
The argument \textit{main} must be the filename of the main file.

There are a couple of
considerations in setting up the main and child documents:

%%%%%%%%%%%%%%%%%%%%%%%%%%%%%%%%%%%%%%%%
\paragraph{Restrictions.}

Please note the following restrictions:
\begin{itemize}
\item
|\childdocmain| must be called with one argument \textit{main}
to ensure compatibility with earlier version of the package.
It must either be empty (|\childdocmain{}|)
or precisely match the filename of the main file in which it is specified.
See \secref{sec:detection} for further information.
\item
The filename \textit{main} must be specified without the |.tex| extension.
\item
The filename \textit{main} is case sensitive
(even in case-insensitive file systems)
due to internal string comparison.
\item
The argument \textit{main} should be fully expanded, it cannot be a macro.
\item
Subdirectories and special characters should be avoided in filenames.
\item
The command |\childdocmain{|\textit{main}|}| must be followed by a whitespace.
It should not be followed immediately by another command
or by a comment mark `|%|'.
This is because the \TeX{} parser reads the token immediately following
the argument of |\childdocmain| and puts it
at the beginning of every child section;
however, a white\-space is ignored.
\end{itemize}

%%%%%%%%%%%%%%%%%%%%%%%%%%%%%%%%%%%%%%%%
\paragraph{Content of Main File.}

It is advisable to place all content in the child files included by |\include|.
Any output contained in the main file will appear in all child documents
unless suppressed manually;
it cannot be suppressed automatically by the |\includeonly| directive
and thus should normally be avoided.
A method to include some content in the main file
by means of conditional processing is described in \secref{sec:conditional}.

%%%%%%%%%%%%%%%%%%%%%%%%%%%%%%%%%%%%%%%%
\paragraph{Page Numbering.}

When only a part of the document is compiled,
the appropriate numbering of pages
(as well as other status parameters)
is determined from the |.aux| files.
The latter contain information from previous passes.
However this information needs to propagate through
all intermediate child documents.
Therefore the page numbering in child documents may well
be inconsistent until the complete document is compiled at least once.

A useful (if unconventional) way to always ensure a consistent
page numbering is to restart the numbering in each child document
and denote the pages by `\textit{child}|.|\textit{page}'
where \textit{child} represents the chapter/section number of the child file.
This can be achieved by the command
|\numberwithin{page}{|\textit{child}|}|
of the \textsf{amsmath} package
where \textit{child} can be |chapter| or |section|
depending on the chosen structuring.
Alternatively, one can modify the macro |\thepage| appropriately
and reset the counter |page| at the start of each child file.

%%%%%%%%%%%%%%%%%%%%%%%%%%%%%%%%%%%%%%%%%%%%%%%%%%%%%%%%%%%%%%%%%%%%%%%%%%%%%%%%
\subsection{Conditional Processing}
\label{sec:conditional}

The package provides a mechanism to compile different versions
of a document. To customise the versions further some conditional processing
can come in handy to distinguish which version is being compiled.
The package provides two macros to describe the compilation context:

%%%%%%%%%%%%%%%%%%%%%%%%%%%%%%%%%%%%%%%%
\DescribeMacro{\ifchilddoc}
The conditional |\ifchilddoc| distinguishes between the compilation of
child documents and the main document:
%
\begin{center}
|\ifchilddoc |\textit{child-code}| |[|\||else |\textit{main-code}]| \||fi|
\end{center}

%%%%%%%%%%%%%%%%%%%%%%%%%%%%%%%%%%%%%%%%
\DescribeMacro{\childdocname}
\DescribeMacro{\childdocjob}
The macro |\childdocname| contains the filename (without extension)
of the main or child file being processed.
Note that |\childdocjob| will always contain the name of the main file.

%%%%%%%%%%%%%%%%%%%%%%%%%%%%%%%%%%%%%%%%
\paragraph{Title Page.}

Conditional processing can be used to include a title or banner page
in the main document when proper precautions are taken.
Importantly, the code in the main file should ensure that the page counter
(as well as other status parameters which are stored in the |.aux| files)
takes the same value after the conditional processing.
Otherwise the page numbers may take divergent values
depending on which part is compiled.

For example, a title page could be declared by:
%
\begin{center}
\begin{tabular}{l}
|\ifchilddoc\||else|\\
|\addtocounter{page}{-1}|\\
\textit{code for title page}\\
|\newpage|\\
|\||fi|
\end{tabular}
\end{center}
%
A banner page for the child documents can be generated by:
%
\begin{center}
\begin{tabular}{l}
|\ifchilddoc|\\
|\addtocounter{page}{-1}|\\
\textit{code for banner page}\\
|\newpage|\\
|\||fi|
\end{tabular}
\end{center}
%
Here one could write a message such as:
\begin{center}
|This is the part \childdocname{} of \childdocjob{}.|
\end{center}

%%%%%%%%%%%%%%%%%%%%%%%%%%%%%%%%%%%%%%%%%%%%%%%%%%%%%%%%%%%%%%%%%%%%%%%%%%%%%%%%
\subsection{Flags}
\label{sec:flags}

The package makes it easy to generate different versions
of the main or child documents.
To this end compilation flags can be defined
and assigned different default values.
They will be particularly useful in conjunction
with the forwarding mechanism described in \secref{sec:forward}.

For example, it may be useful to have a flag |\version|
which can be set to |draft| or |final|.
The document source will contain some conditional code
depending on the value of |\version|.
Suppose further, the flag should default to |final| for the main file
and to |draft| for child files
which is a natural assignment for editing the document.
This is achieved by placing the following code
in the preamble of the main document
(below the |\childdocmain| directive):
%
\begin{center}
\begin{tabular}{l}
|\ifchilddoc|\\
|\providecommand{\version}{draft}|\\
|\||else|\\
|\providecommand{\version}{final}|\\
|\||fi|
\end{tabular}
\end{center}
%
The definition by |\providecommand| makes sure
that previous definitions are not overwritten.
Further statements |\providecommand{\version}{...}|
can thus be added before the above code to override it.

For the main file, one might add a line
(between |\childdocmain| and the above block)
%
\begin{center}
|%\ifchilddoc\||else\providecommand{\version}{draft}\||fi|
\end{center}
%
which can be uncommented to produce a draft version.
Likewise one can add a line to the very top of a child file
(above the |\childdocof{|\textit{main}|}| directive)
%
\begin{center}
|%\providecommand{\version}{final}|
\end{center}
%
which can be uncommented to produce the final version of this child document.

%%%%%%%%%%%%%%%%%%%%%%%%%%%%%%%%%%%%%%%%%%%%%%%%%%%%%%%%%%%%%%%%%%%%%%%%%%%%%%%%
\subsection{Forwarding}
\label{sec:forward}

Different versions of the main or child documents
using compilation flags as described in \secref{sec:flags}
can be (permanently) stored in different files
for convenient compilation, viewing and distribution.
To this end, the package defines a command
to pass on compilation to a different file:

%%%%%%%%%%%%%%%%%%%%%%%%%%%%%%%%%%%%%%%%
\DescribeMacro{\childdocforward}
The command |\childdocforward| redirects processing to
another source file:
%
\begin{center}
\begin{tabular}{l}
|\input{childdoc.def}|\\
|\childdocforward[|\textit{main}|]{|\textit{dest}|}|\\
\end{tabular}
\end{center}
%
The argument \textit{dest} is the destination file
(without extension).
It should be the main file or one of the child files.
Note that further \textsf{childdoc} directives
such as |\childdocof| and |\childdocforward|
in the indicated file will be processed in this form.
The optional argument \textit{main}
passes on directly to the main file \textit{main}
while pretending to compile the child \textit{dest}.
This form behaves as if \textit{dest}
issues |\childdocof{|\textit{main}|}| right away,
and no further \textsf{childdoc} directives will be processed.

%%%%%%%%%%%%%%%%%%%%%%%%%%%%%%%%%%%%%%%%
\DescribeMacro{\...prefix}
In the alternative form |\childdocforwardprefix|,
%
\begin{center}
\begin{tabular}{l}
|\input{childdoc.def}|\\
|\childdocforwardprefix[|\textit{main}|]{|\textit{prefix}|}{|\textit{dest}|}|
\end{tabular}
\end{center}
%
the destination file is determined by a pattern
depending on the current file:
To make this work, the current file must be called
`{\textit{prefix}\hspace{0.2em}\textit{suffix}}'
with \textit{prefix} matching precisely the argument.
Processing is then passed on to the file
`{\textit{dest}\hspace{0.2em}\textit{suffix}}'.
Surely, the same effect is achieved by
directly specifying the
argument `{\textit{dest}\hspace{0.2em}\textit{suffix}}'
in the first form.
However, that requires to set up a different file
for each child. With the alternative form of the command
all these files can have exactly the same content
which simplifies setting them up and maintaining them.

For example, the following file |draft.tex|
with a compilation flag |\version| as described in \secref{sec:flags}
compiles the main document as a draft:
%
\begin{center}
\begin{tabular}{l}
|\def\version{draft}|\\
|\input{childdoc.def}|\\
|\childdocforward{|\textit{main}|}|
\end{tabular}
\end{center}
%
Likewise, the following files |final|\textit{nn}|.tex|
compile the final version of the child document
|child|\textit{nn}|.tex|:
%
\begin{center}
\begin{tabular}{l}
|\def\version{final}|\\
|\input{childdoc.def}|\\
|\childdocforwardprefix{final}{child}|
\end{tabular}
\end{center}
%

Note that when several versions of a main file and/or of each child file
are to be generated, it may be convenient to set up a |Makefile| or
shell script to automatise the process.

%%%%%%%%%%%%%%%%%%%%%%%%%%%%%%%%%%%%%%%%%%%%%%%%%%%%%%%%%%%%%%%%%%%%%%%%%%%%%%%%
\subsection{Command Line Processing}
\label{sec:commandline}

The effect of redirection files can also be achieved by invoking
the \LaTeX{} compiler with a more elaborate command line.
Most conveniently this should be done as part
of a shell script or a |Makefile|.

When using \textsf{childdoc} in the main file, the following
command lines effectively perform a redirection
(note that depending on the shell being used,
backslashes may have to be doubled: `|\|' $\to$ `|\\|'):
%
\begin{center}
|... -jobname "|\textit{target}|" |\\|"|[\textit{flags}]%
|\input{childdoc.def}\childdocforward[|\textit{main}|]{|\textit{dest}|}"|
\end{center}
%
Here \textit{target} is the name of the output file,
\textit{main} is the name of the main file
and \textit{dest} is the name of the main or child file to be processed
(all filenames without extensions).
The optional argument \textit{main} can be omitted
if \textit{main} matches \textit{dest}.
Optionally, compilation \textit{flags} can be defined via |\def| commands.
This command line makes the \TeX{} engine believe
it is compiling the file \textit{target}
whose content is specified as the latter parameter.
The provided code then forwards the processing to
\textit{main} or \textit{dest} as described in \secref{sec:forward}.

%%%%%%%%%%%%%%%%%%%%%%%%%%%%%%%%%%%%%%%%%%%%%%%%%%%%%%%%%%%%%%%%%%%%%%%%%%%%%%%%
\subsection{Include by Input}
\label{sec:input}

Including child documents by |\include| has some restrictions by design.
Most notably, the content of a child document always occupies
its own set of pages; pages cannot be shared between child documents.
Usually, this behaviour makes perfect sense
because each child document contain an essential part of the document.
However, in some situations it may be desirable to compose
a document from a collection of parts
without having mandatory page breaks between then.
For this case, the package
provides a mechanism to include parts
by |\input| which can also be processed individually.
However, by construction this mechanism
requires manual handling of the content to be output.

%%%%%%%%%%%%%%%%%%%%%%%%%%%%%%%%%%%%%%%%
\DescribeMacro{\ifchilddocmanual}
The main file should be prepared as usual, see \secref{sec:include}.
However, the document body must make a distinction
between processing of an individual part and of the main document, e.g.:
%
\begin{center}
\begin{tabular}{l}
|\ifchilddocmanual|\\
|\input{\childdocname}|\\
|\||else|\\
\textit{document body with }|\input{|\textit{part}|}|\\
|\||fi|
\end{tabular}
\end{center}
%
The conditional |\ifchilddocmanual| is true whenever
a part to be included by |\input| is being compiled,
and the name of the part is stored in |\childdocname|.

%%%%%%%%%%%%%%%%%%%%%%%%%%%%%%%%%%%%%%%%
\DescribeMacro{\childdocby}
Each part to be included by |\input| should start with:
%
\begin{center}
\begin{tabular}{l}
|\input{childdoc.def}|\\
|\childdocby{|\textit{main}|}|\\
\end{tabular}
\end{center}
%
The directive |\childdocby| is similar to |\childdocof|
described in \secref{sec:include},
but the subsequent selection of content must be done manually.
To that end, both |\ifchilddoc| and |\ifchilddocmanual|
will be true upon processing of a part,
and the name of the part is stored in |\childdocname|.
Note that |\jobname| will be set to the filename of the current part
so that each part receives an individual |.aux| file
that does not interfere with the |.aux| file(s) of the main document.
This behaviour can be altered by the alternative form
|\childdocby[*]{|\textit{main}|}| (with a non-empty optional argument)
which uses the |.aux| file of the main document
by setting |\jobname| to \textit{main}.

%%%%%%%%%%%%%%%%%%%%%%%%%%%%%%%%%%%%%%%%%%%%%%%%%%%%%%%%%%%%%%%%%%%%%%%%%%%%%%%%
\subsection{Driver Development}
\label{sec:driver}

The \textsf{childdoc} mechanism can also be use for the development
of definition files such as \LaTeX{} styles or classes.
This case differs from the above setup with multiple parts
included by |\include| in that no |\includeonly| should be invoked.
This can be achieved by starting the include file
(before |\ProvidesPackage|) with:
%
\begin{center}
\begin{tabular}{l}
|\input{childdoc.def}|\\
|\childdocforward{|\textit{main}|}|\\
\end{tabular}
\end{center}
%
or alternatively with:
%
\begin{center}
\begin{tabular}{l}
|\input{childdoc.def}|\\
|\childdocby{|\textit{main}|}|\\
\end{tabular}
\end{center}
%
Both forms have slightly different effects as described above.
The main file is prepared as usual, see \secref{sec:include}.

%%%%%%%%%%%%%%%%%%%%%%%%%%%%%%%%%%%%%%%%%%%%%%%%%%%%%%%%%%%%%%%%%%%%%%%%%%%%%%%%
\subsection{Legacy Detection}
\label{sec:detection}

The directive |\childdocmain| in the main file can detect
whether the complete document or merely a child is to be compiled
even without using the directive |\childdocof|.
This method is deprecated because it is less robust
and there is no compelling reason to use it;
it is merely provided for backward compatibility
and it may be removed in future versions.

If the detection mechanism is to be used,
it is mandatory to correctly specify
the filename of the main file as the argument of |\childdocmain|:
%
\begin{center}
\begin{tabular}{l}
|\input{childdoc.def}|\\
|\childdocmain{|\textit{main}|}|\\
\end{tabular}
\end{center}
%
If |\jobname| does not match the argument \textit{main} of |\childdocmain|,
it is assumed that |\jobname| points to the child file to be compiled.
When using |\childdocmain| with the main file specified as argument,
it suffices to start a child file
with just |\input{|\textit{main}|}|
without loading of the package and using |\childdocof|.
If instead all processing is done
with the appropriate \textsf{childdoc} directives,
the argument of \textit{main} of |\childdocmain| can be empty.

An alternative version of the command line processing described
in \secref{sec:commandline} using the detection mechanism reads:
%
\begin{center}
|... -jobname "|\textit{target}|" "|[\textit{flags}]%
[|\def\jobname{|\textit{dest}|}|]|\input{|\textit{main}|}"|
\end{center}

%%%%%%%%%%%%%%%%%%%%%%%%%%%%%%%%%%%%%%%%%%%%%%%%%%%%%%%%%%%%%%%%%%%%%%%%%%%%%%%%
\subsection{Manual Code}
\label{sec:manual}

In case one cannot be certain whether the definitions file |childdoc.def|
is installed on the target \TeX{} distribution
and one prefers not to ship it,
it is conceivable to paste a few relevant commands into the sources.

To that end, drop all statements |\input{childdoc.def}|
and perform the replacements as outlined below.
Instead of |\childdocmain{|\textit{main}|}| add the following code
to the top of the main file:
%
\begin{center}
\begin{tabular}{l}
|\||ifdefined\childdocname\endinput\||fi\newif\ifchilddoc|\\
|\edef\childdocname{\scantokens\expandafter{\jobname\noexpand}}|\\
|\def\childdocmain{|\textit{main}|}\||ifx\childdocmain\childdocname\||else|\\
|\childdoctrue\includeonly{\childdocname}\let\jobname\childdocmain\||fi|\\
\end{tabular}
\end{center}
%
Instead of |\childdocof{|\textit{main}|}| just include the main file
at the top of each child file:
%
\begin{center}
|\input{|\textit{main}|}|
\end{center}
%
A simple redirection |\childdocforward{|\textit{dest}|}| is achieved by:
%
\begin{center}
|\def\jobname{|\textit{dest}|}\input{\jobname}|
\end{center}
%
The redirection with prefix
|\childdocforwardprefix[|\textit{prefix}|]{|\textit{dest}|}|
is accomplished by:
%
\begin{center}
\begin{tabular}{l}
|{\edef\jobname{\scantokens\expandafter{\jobname\noexpand}}|\\
|\def\redirectjob |\textit{prefix}|#1~~~{\gdef\jobname{|\textit{dest}|#1}}|\\
|\expandafter\redirectjob\jobname~~~}\input{\jobname}|
\end{tabular}
\end{center}

In an alternative approach,
child documents can be compiled by a specific command line
without additional code or specific definitions:
%
\begin{center}
|... -jobname "|\textit{target}|" "|[\textit{flags}]%
|\includeonly{|\textit{dest}|}\input{|\textit{main}|}"|
\end{center}
%

%%%%%%%%%%%%%%%%%%%%%%%%%%%%%%%%%%%%%%%%%%%%%%%%%%%%%%%%%%%%%%%%%%%%%%%%%%%%%%%%
%%%%%%%%%%%%%%%%%%%%%%%%%%%%%%%%%%%%%%%%%%%%%%%%%%%%%%%%%%%%%%%%%%%%%%%%%%%%%%%%
\section{Information}

%%%%%%%%%%%%%%%%%%%%%%%%%%%%%%%%%%%%%%%%%%%%%%%%%%%%%%%%%%%%%%%%%%%%%%%%%%%%%%%%
\subsection{Copyright}

Copyright \copyright{} 2017--2018 Niklas Beisert

This work may be distributed and/or modified under the
conditions of the \LaTeX{} Project Public License, either version 1.3
of this license or (at your option) any later version.
The latest version of this license is in
  \url{http://www.latex-project.org/lppl.txt}
and version 1.3 or later is part of all distributions of \LaTeX{}
version 2005/12/01 or later.

This work has the LPPL maintenance status `maintained'.

The Current Maintainer of this work is Niklas Beisert.

This work consists of the files |README.txt|, |childdoc.ins| and |childdoc.dtx|
as well as the derived files |childdoc.def|, |cdocsamp.tex|
with |cdocsch1.tex|, |cdocsch2.tex|, |cdocspt3.tex|, |cdocspt4.tex|,
|cdocsdrf.tex|, |cdocsfn1.tex|, |cdocsfn2.tex|
as well as |childdoc.pdf|.

%%%%%%%%%%%%%%%%%%%%%%%%%%%%%%%%%%%%%%%%%%%%%%%%%%%%%%%%%%%%%%%%%%%%%%%%%%%%%%%%
\subsection{Files and Installation}

The package consists of the files:
%
\begin{center}
\begin{tabular}{ll}
    |README.txt|   & readme file \\
    |childdoc.ins| & installation file \\
    |childdoc.dtx| & source file \\
    |childdoc.def| & definition file \\
    |cdocsamp.tex| & sample main file \\
    |cdocsch1.tex| & sample include file \\
    |cdocsch2.tex| & sample include file \\
    |cdocspt3.tex| & sample part file \\
    |cdocspt4.tex| & sample part file \\
    |cdocsdrf.tex| & sample redirection file \\
    |cdocsfn1.tex| & sample redirection file \\
    |cdocsfn2.tex| & sample redirection file \\
    |childdoc.pdf| & manual
\end{tabular}
\end{center}
%
The distribution consists of the files
|README.txt|, |childdoc.ins| and |childdoc.dtx|.
%
\begin{itemize}
\item
Run (pdf)\LaTeX{} on |childdoc.dtx|
to compile the manual |childdoc.pdf| (this file).
\item
Run \LaTeX{} on |childdoc.ins| to create the definitions file |childdoc.def|
and the sample |cdocsamp.tex| with include files
|cdocsch1.tex|, |cdocsch2.tex|, |cdocspt3.tex|, |cdocspt4.tex|,
|cdocsdrf.tex|, |cdocsfn1.tex|, |cdocsfn2.tex|.
Then copy the file |childdoc.def| to an appropriate directory of your \LaTeX{}
distribution, e.g.\ \textit{texmf-root}|/tex/latex/childdoc|.
\end{itemize}

%%%%%%%%%%%%%%%%%%%%%%%%%%%%%%%%%%%%%%%%%%%%%%%%%%%%%%%%%%%%%%%%%%%%%%%%%%%%%%%%
\subsection{Related CTAN Packages}

There are several other packages which offer a similar functionality:
%
\begin{itemize}
\item
The packages
\href{http://ctan.org/pkg/docmute}{\textsf{docmute}},
\href{http://ctan.org/pkg/includex}{\textsf{includex}} and
\href{http://ctan.org/pkg/standalone}{\textsf{standalone}}
provide commands to include only the document body of
a child file thus allowing both files to be compiled individually.
\item
The packages \href{http://ctan.org/pkg/subdocs}{\textsf{subdocs}}
and \href{http://ctan.org/pkg/subfiles}{\textsf{subfiles}}
provide structures in which the main and child documents can be
encapsulated and allowing them to be compiled individually.
The inclusion mechanism is different from the conventional |\include|.
\item
The package \href{http://ctan.org/pkg/combine}{\textsf{combine}}
is an elaborate solution to combine several documents into one.
\end{itemize}
%
See also the CTAN topic \href{http://ctan.org/topic/subdocs}{\textsf{subdocs}}
for further related packages.
The present package differs from the above solutions in that
a document structure constructed with the conventional |\include| mechanism
just needs two extra commands at the top of every file
such that all constituent files can be compiled individually.

%%%%%%%%%%%%%%%%%%%%%%%%%%%%%%%%%%%%%%%%%%%%%%%%%%%%%%%%%%%%%%%%%%%%%%%%%%%%%%%%
%\subsection{Feature Suggestions}
%
%The following is a list of features which may be useful for future
%versions of this package:
%%
%\begin{itemize}
%\item
%\ldots
%\end{itemize}

%%%%%%%%%%%%%%%%%%%%%%%%%%%%%%%%%%%%%%%%%%%%%%%%%%%%%%%%%%%%%%%%%%%%%%%%%%%%%%%%
\subsection{Revision History}

%%%%%%%%%%%%%%%%%%%%%%%%%%%%%%%%%%%%%%%%
\paragraph{v2.0:} 2018/12/30

\begin{itemize}
\item
immediate forward processing
\item
added |\childdocby| mechanism
\item
manual restructured
\end{itemize}

%%%%%%%%%%%%%%%%%%%%%%%%%%%%%%%%%%%%%%%%
\paragraph{v1.6:} 2018/01/17

\begin{itemize}
\item
application for development of include files
\item
corrections to manual
\end{itemize}

%%%%%%%%%%%%%%%%%%%%%%%%%%%%%%%%%%%%%%%%
\paragraph{v1.5:} 2017/05/21

\begin{itemize}
\item
more complete structuring introduced
\item
|\childdocof| introduced
\item
|\childdoc| renamed to |\childdocmain|
\item
|\childredirect| renamed to |\childdocforward| and |\childdocforwardprefix|
and functionality expanded
\end{itemize}

%%%%%%%%%%%%%%%%%%%%%%%%%%%%%%%%%%%%%%%%
\paragraph{v1.0:} 2017/04/27

\begin{itemize}
\item
manual and install package
\item
first version published on CTAN
\end{itemize}

%%%%%%%%%%%%%%%%%%%%%%%%%%%%%%%%%%%%%%%%
\paragraph{v0.6:} 2017/04/26

\begin{itemize}
\item
redirection mechanism added
\end{itemize}

%%%%%%%%%%%%%%%%%%%%%%%%%%%%%%%%%%%%%%%%
\paragraph{v0.5:} 2017/04/26

\begin{itemize}
\item
functionality in definition file
\end{itemize}


%%%%%%%%%%%%%%%%%%%%%%%%%%%%%%%%%%%%%%%%%%%%%%%%%%%%%%%%%%%%%%%%%%%%%%%%%%%%%%%%
%%%%%%%%%%%%%%%%%%%%%%%%%%%%%%%%%%%%%%%%%%%%%%%%%%%%%%%%%%%%%%%%%%%%%%%%%%%%%%%%
%%%%%%%%%%%%%%%%%%%%%%%%%%%%%%%%%%%%%%%%%%%%%%%%%%%%%%%%%%%%%%%%%%%%%%%%%%%%%%%%
\appendix

\settowidth\MacroIndent{\rmfamily\scriptsize 000\ }

 \DocInput{childdoc.dtx}

\end{document}
%</driver>
% \fi
%
% %%%%%%%%%%%%%%%%%%%%%%%%%%%%%%%%%%%%%%%%%%%%%%%%%%%%%%%%%%%%%%%%%%%%%%%%%%%%%%
% %%%%%%%%%%%%%%%%%%%%%%%%%%%%%%%%%%%%%%%%%%%%%%%%%%%%%%%%%%%%%%%%%%%%%%%%%%%%%%
% \section{Sample}
%\iffalse
%<*samplemain>
%\fi
%
% The following presents a sample document
% with two chapters, two parts, a title page,
% a compile flag as well as three forwarding files to set the flag.
% It consists of eight |.tex| files:
% \begin{center}
% \begin{tabular}{ll}
% |cdocsamp.tex|&main file\\
% |cdocsch1.tex|&include file for chapter 1\\
% |cdocsch2.tex|&include file for chapter 2\\
% |cdocspt3.tex|&include file for part 3\\
% |cdocspt4.tex|&include file for part 4\\
% |cdocsdrf.tex|&forwarding file for main file in draft mode\\
% |cdocsfi1.tex|&forwarding file for final version of chapter 1\\
% |cdocsfi2.tex|&forwarding file for final version of chapter 2\\
% \end{tabular}
% \end{center}
% Each of the eight files can be compiled directly by the \LaTeX{} compiler.
%
% %%%%%%%%%%%%%%%%%%%%%%%%%%%%%%%%%%%%%%
% \paragraph{Main File.}
%
% The main file is called |cdocsamp.tex|.
%
% Load the \textsf{childdoc} definitions and
% declare the filename for the main document:
%    \begin{macrocode}
\input{childdoc.def}
\childdocmain{}
%    \end{macrocode}

% Optional override for |\version| flag:
%    \begin{macrocode}
%%\ifchilddoc\else\providecommand{\version}{draft}\fi
%    \end{macrocode}

% Define the default values for the |\version| flag
% (|final| for the main file and |draft| for childs):
%    \begin{macrocode}
\ifchilddoc
\providecommand{\version}{draft}
\else
\providecommand{\version}{final}
\fi
%    \end{macrocode}

% Load the standard document class:
%    \begin{macrocode}
\documentclass[12pt]{article}
%    \end{macrocode}

% Start the document body:
%    \begin{macrocode}
\begin{document}
%    \end{macrocode}

% Declare a title page.
% Print title, part of document being processed and version flag:
%    \begin{macrocode}
\addtocounter{page}{-1}
\begin{center}
{\LARGE\bfseries{}childdoc example\par}
\vspace{1cm}
\ifchilddoc
\ifchilddocmanual part\else chapter\fi:
`\childdocname' of `\childdocjob'\par
\else
main document: `\childdocjob'\par
\fi
version: \version\par
\end{center}
\newpage
%    \end{macrocode}

% Manually include selected file,
% otherwise process as usual:
%    \begin{macrocode}
\ifchilddocmanual
\section*{part `\childdocname'}
\input{\childdocname}
\else
%    \end{macrocode}

% Include the two chapters:
%    \begin{macrocode}
\include{cdocsch1}
\include{cdocsch2}
%    \end{macrocode}

% Include the two parts unless only chapters should be displayed:
%    \begin{macrocode}
\ifchilddoc\else
\section{part three}
\input{cdocspt3}
\section{part four}
\input{cdocspt4}
\fi
%    \end{macrocode}

% Process as usual until here:
%    \begin{macrocode}
\fi
%    \end{macrocode}

% End of document body:
%    \begin{macrocode}
\end{document}
%    \end{macrocode}
%\iffalse
%</samplemain>
%\fi
%
% %%%%%%%%%%%%%%%%%%%%%%%%%%%%%%%%%%%%%%
% \paragraph{Chapter Include Files.}
%
% The include files are called |cdocsch1.tex| and |cdocsch2.tex|.
%
%\iffalse
%<*samplechap1|samplechap2>
%\fi

% Optional override for |\version| flag:
%    \begin{macrocode}
%%\providecommand{\version}{final}
%    \end{macrocode}

% Include the main document:
%    \begin{macrocode}
\input{childdoc.def}
\childdocof{cdocsamp}
%    \end{macrocode}

%\iffalse
%</samplechap1|samplechap2>
%\fi
%
%\iffalse
%<*samplechap1>
%\fi
% Some text for chapter 1:
%    \begin{macrocode}
\section{one}
some text in chapter one
%    \end{macrocode}

%\iffalse
%</samplechap1>
%\fi
% Some text for chapter 2:
%\iffalse
%<*samplechap2>
%\fi
%    \begin{macrocode}
\section{two}
more text in chapter two
%    \end{macrocode}

%\iffalse
%</samplechap2>
%\fi
%
% %%%%%%%%%%%%%%%%%%%%%%%%%%%%%%%%%%%%%%
% \paragraph{Part Include Files.}
%
% The include files are called |cdocspt3.tex| and |cdocspt4.tex|.
%
%\iffalse
%<*samplepart3|samplepart4>
%\fi

% Optional override for |\version| flag:
%    \begin{macrocode}
%%\providecommand{\version}{final}
%    \end{macrocode}

% Include the main document:
%    \begin{macrocode}
\input{childdoc.def}
\childdocby{cdocsamp}
%    \end{macrocode}

%\iffalse
%</samplepart3|samplepart4>
%\fi
%
%\iffalse
%<*samplepart3>
%\fi
% Some text for part 3:
%    \begin{macrocode}
some text in part three
%    \end{macrocode}

%\iffalse
%</samplepart3>
%\fi
% Some text for part 4:
%\iffalse
%<*samplepart4>
%\fi
%    \begin{macrocode}
more text in part four
%    \end{macrocode}

%\iffalse
%</samplepart4>
%\fi
%
% %%%%%%%%%%%%%%%%%%%%%%%%%%%%%%%%%%%%%%
% \paragraph{Forwarding for a Complete Draft.}
%
% The following forwarding file |cdocsdrf.tex|
% compiles the main document in draft mode:
%\iffalse
%<*sampledraft>
%\fi
%    \begin{macrocode}
\def\version{draft}
\input{childdoc.def}
\childdocforward{cdocsamp}
%    \end{macrocode}

%\iffalse
%</sampledraft>
%\fi
%
% %%%%%%%%%%%%%%%%%%%%%%%%%%%%%%%%%%%%%%
% \paragraph{Forwarding for Final Version of the Chapters.}
%
% The following forwarding files |cdocsfn1.tex| and |cdocsfn2.tex|
% (with identical content)
% compile the final versions of the child documents
% |cdocsch1.tex| and |cdocsch2.tex|, respectively:
%\iffalse
%<*samplefinal>
%\fi
%    \begin{macrocode}
\def\version{final}
\input{childdoc.def}
\childdocforwardprefix[cdocsamp]{cdocsfn}{cdocsch}
%    \end{macrocode}

%\iffalse
%</samplefinal>
%\fi
%
% %%%%%%%%%%%%%%%%%%%%%%%%%%%%%%%%%%%%%%
% \paragraph{Command Line Processing.}
%
% The following three command lines generate the output files
% |cdocscld|, |cdocscl1| and |cdocscl2|
% which should be identical to
% |cdocsdrf|, |cdocsch1| and |cdocsfn2|, respectively:
% \begin{center}
% \begin{tabular}{l}
% |latex -jobname cdocscld \|\\
% |  "\def\version{draft}\input{childdoc.def}\childdocforward{cdocsamp}"|\\
% |latex -jobname cdocscl1 \|\\
% |  "\input{childdoc.def}\childdocforward[cdocsamp]{cdocsch1}"|\\
% |latex -jobname cdocscl2 \|\\
% |  "\def\version{final}\input{childdoc.def}\childdocforward{cdocsch2}"|
% \end{tabular}
% \end{center}
% Note that the trailing backslash on each first line
% merely continues the input to the second line
% (for convenient cut ant paste).
% Furthermore, the command |latex| can be replaced by any
% of its alternative versions such as |pdflatex|.
%
% %%%%%%%%%%%%%%%%%%%%%%%%%%%%%%%%%%%%%%%%%%%%%%%%%%%%%%%%%%%%%%%%%%%%%%%%%%%%%%
% %%%%%%%%%%%%%%%%%%%%%%%%%%%%%%%%%%%%%%%%%%%%%%%%%%%%%%%%%%%%%%%%%%%%%%%%%%%%%%
% \section{Implementation}
%\iffalse
%<*package>
%\fi
%
% This section describes the definitions file |childdoc.def|.

% The definitions cannot be loaded using |\usepackage| or |\RequirePackage|
% which has a mechanism to prevent loading a style file more than once.
% When loading the definitions by means of |\input|
% multiple instances have to be prevented manually:
%\iffalse
%This code needs to be before the `\ProvidesFile' directive
%which is defined at the beginning of this file.
%Therefore it is also placed there and commented out here.
%</package>
%<*discard>
%\fi
%    \begin{macrocode}
\ifdefined\childdocmain\endinput\fi
%    \end{macrocode}
%\iffalse
%</discard>
%<*package>
%\fi
%
% \macro{\ifchilddoc}
% \macro{\ifchilddocmanual}
% The conditional |\ifchilddoc| tells whether a
% child (true) or main (false) document is being compiled.
% The conditional |\ifchilddocmanual| tells whether
% the |\includeonly| mechanism is used (false) or
% the selection of child files must be performed manually (true).
% The definitions initialise to false:
%    \begin{macrocode}
\newif\ifchilddoc
\newif\ifchilddocmanual
%    \end{macrocode}

% \macro{\childdocname}
% \macro{\childdocjob}
% The macro |\childdocname| stores the name of the main document
% to be compiled. The macro |\childdocjob| stores the name of
% the document on which the \LaTeX{} compiler was originally invoked.
% The content of |\jobname| cannot be compared
% to filenames specified in the source due to different catcodes.
% The following code rescans |\jobname|, stores the result
% in |\childdocname| and saves a copy in |\childdocjob|:
%    \begin{macrocode}
\edef\childdocname{\scantokens\expandafter{\jobname\noexpand}}
\let\childdocjob\childdocname
%    \end{macrocode}

% \macro{\childdocdisable}
% The macro |\childdocdisable| prevents the main file
% from being processed more than once.
% At this stage, the main document command |\childdocmain|
% is assumed to be called once again where it should do nothing.
% Any subsequent call to it should prevent
% a secondary processing of the main document
% It overwrites the forwarding commands
% |\childdocof| and |\childdocforward|
% with empty macros to prevent further inclusions of the main document:
%    \begin{macrocode}
\newcommand{\childdocdisable}
{
  \renewcommand{\childdocmain}[1]{\renewcommand{\childdocmain}[1]{\endinput}}
  \renewcommand{\childdocof}[1]{}
  \renewcommand{\childdocby}[2][]{}
  \renewcommand{\childdocforward}[2][]{}
  \renewcommand{\childdocdisable}{}
}
%    \end{macrocode}

% \macro{\childdocmain}
% The macro |\childdocmain| is to be called at the top of the main file
% with nothing or the main filename (without extension) as argument.
% First, it breaks loops.
% If the argument is not empty and does not match |\childdocname|
% (which is set by the first inclusion of |childdoc.def|),
% |\ifchilddoc| is set to true, |\includeonly| is applied to the child file
% and |\jobname| is set to the main file
% (for proper handling of |.aux| files):
%    \begin{macrocode}
\newcommand{\childdocmain}[1]
{
  \childdocdisable\childdocmain{}
  \if?#1?\else
    \begingroup
      \def\childdoctmp{#1}
      \ifx\childdoctmp\childdocname
        \def\childdoctmp{}
      \else
        \def\childdoctmp
        {
          \childdoctrue
          \includeonly{\childdocname}
          \def\childdocjob{#1}
          \def\jobname{#1}
        }
      \fi
      \expandafter
    \endgroup
    \childdoctmp
  \fi
}
%    \end{macrocode}

% \macro{\childdocof}
% The command |\childdocof| redirects
% compilation to the main file |#1|.
%    \begin{macrocode}
\newcommand{\childdocof}[1]
{
  \childdocdisable
  \childdoctrue
  \includeonly{\childdocname}
  \def\jobname{#1}
  \def\childdocjob{#1}
  \input{#1}
}
%    \end{macrocode}

% \macro{\childdocby}
% The command |\childdocby| ....
%    \begin{macrocode}
\newcommand{\childdocby}[2][]
{
  \childdocdisable
  \childdoctrue
  \childdocmanualtrue
  \if?#1?\else
    \def\jobname{#2}
  \fi
  \def\childdocjob{#2}
  \input{#2}
  \endinput
}
%    \end{macrocode}

% \macro{\childdocforward}
% The command |\childdocforward| redirects
% compilation to the main file or
% (if the optional argument is given) a child file.
% Parameters are set as if the main file
% or a child file starting with |\childdocof| was compiled.
% Then compilation is handed over to the main file:
%    \begin{macrocode}
\newcommand{\childdocforward}[2][]
{
  \begingroup
    \if?#1?
      \def\childdoctmp
      {
        \def\childdocname{#2}
        \def\childdocjob{#2}
        \def\jobname{#2}
        \input{#2}
        \endinput
      }
    \else
      \def\childdoctmp
      {
        \childdocdisable
        \def\childdocname{#2}
        \childdoctrue
        \includeonly{#2}
        \def\childdocjob{#1}
        \def\jobname{#1}
        \input{#1}
        \endinput
      }
    \fi
    \expandafter
  \endgroup
  \childdoctmp
}
%    \end{macrocode}

% \macro{\childdocforwardprefix}
% The command |\childdocforwardprefix| redirects
% compilation to the main or a child file by means of a pattern.
% The prefix |#1| in the current filename is replaced by |#2|
% and the suffix of the current filename is kept
% (it is assumed that the filename does not contain the substring `|~~~|'
% which is used as a delimiter).
% Compilation is handed over to the new file by |\childdocforward|:
%    \begin{macrocode}
\newcommand{\childdocforwardprefix}[3][]
{
  \begingroup
    \def\childdocextract #2##1~~~{\def\childdoctmp{\childdocforward[#1]{#3##1}}}
    \expandafter\childdocextract\childdocname~~~
    \expandafter
  \endgroup
  \childdoctmp
}
%    \end{macrocode}

% \macro{\childdoc}
% The deprecated macro |\childdoc| is a legacy version of |\childdocmain|:
%    \begin{macrocode}
\newcommand{\childdoc}{\childdocmain}
%    \end{macrocode}

% \macro{\childdocredirect}
% The deprecated macro |\childdocredirect| is a legacy version
% of |\childdocforward| and |\childdocforwardprefix|:
%    \begin{macrocode}
\newcommand{\childdocredirect}[2][]
{
  \begingroup
    \if?#1?
      \def\childdoctmp{\childdocforward{#2}}
    \else
      \def\childdoctmp{\childdocforwardprefix{#1}{#2}}
    \fi
    \expandafter
  \endgroup
  \childdoctmp
}
%    \end{macrocode}

%\iffalse
%</package>
%\fi
%
\endinput
|\\
|\childdocforwardprefix{final}{child}|
\end{tabular}
\end{center}
%

Note that when several versions of a main file and/or of each child file
are to be generated, it may be convenient to set up a |Makefile| or
shell script to automatise the process.

%%%%%%%%%%%%%%%%%%%%%%%%%%%%%%%%%%%%%%%%%%%%%%%%%%%%%%%%%%%%%%%%%%%%%%%%%%%%%%%%
\subsection{Command Line Processing}
\label{sec:commandline}

The effect of redirection files can also be achieved by invoking
the \LaTeX{} compiler with a more elaborate command line.
Most conveniently this should be done as part
of a shell script or a |Makefile|.

When using \textsf{childdoc} in the main file, the following
command lines effectively perform a redirection
(note that depending on the shell being used,
backslashes may have to be doubled: `|\|' $\to$ `|\\|'):
%
\begin{center}
|... -jobname "|\textit{target}|" |\\|"|[\textit{flags}]%
|% \iffalse
%
% childdoc.dtx Copyright (C) 2017-2018 Niklas Beisert
%
% This work may be distributed and/or modified under the
% conditions of the LaTeX Project Public License, either version 1.3
% of this license or (at your option) any later version.
% The latest version of this license is in
%   http://www.latex-project.org/lppl.txt
% and version 1.3 or later is part of all distributions of LaTeX
% version 2005/12/01 or later.
%
% This work has the LPPL maintenance status `maintained'.
%
% The Current Maintainer of this work is Niklas Beisert.
%
% This work consists of the files childdoc.dtx and childdoc.ins
% and the derived files childdoc.def and cdocsamp.tex with
% cdocsch1.tex, cdocsch2.tex, cdocsdrf.tex, cdocsfn1.tex, cdocsfn2.tex.
%
%<package>\ifdefined\childdocmain\endinput\fi
%<package>\ProvidesFile{childdoc.def}[2018/12/30 v2.0 child document driver]
%<samplemain>\ProvidesFile{cdocsamp.tex}[2018/12/30 v2.0 sample for childdoc]
%<*driver>
%\ProvidesFile{childdoc.drv}[2018/12/30 v2.0 childdoc reference manual file]
\PassOptionsToClass{10pt,a4paper}{article}
\documentclass{ltxdoc}

\usepackage[margin=35mm]{geometry}
\usepackage{hyperref}
\usepackage{hyperxmp}
\usepackage[usenames]{color}

\hypersetup{colorlinks=true}
\hypersetup{pdfstartview=FitH}
\hypersetup{pdfpagemode=UseNone}
\hypersetup{pdfsource={}}
\hypersetup{pdflang={en-UK}}
\hypersetup{pdfcopyright={Copyright 2017-2018 Niklas Beisert.
  This work may be distributed and/or modified under the
  conditions of the LaTeX Project Public License, either version 1.3
  of this license or (at your option) any later version.}}
\hypersetup{pdflicenseurl={http://www.latex-project.org/lppl.txt}}
\hypersetup{pdfcontactaddress={ETH Zurich, ITP, HIT K,
  Wolfgang-Pauli-Strasse 27}}
\hypersetup{pdfcontactpostcode={8093}}
\hypersetup{pdfcontactcity={Zurich}}
\hypersetup{pdfcontactcountry={Switzerland}}
\hypersetup{pdfcontactemail={nbeisert@itp.phys.ethz.ch}}
\hypersetup{pdfcontacturl={http://people.phys.ethz.ch/\xmptilde nbeisert/}}

\newcommand{\secref}[1]{\hyperref[#1]{section \ref*{#1}}}

\parskip1ex
\parindent0pt
\let\olditemize\itemize
\def\itemize{\olditemize\parskip0pt}

\begin{document}

\title{The \textsf{childdoc} Package}
\hypersetup{pdftitle={The childdoc Package}}
\author{Niklas Beisert\\[2ex]
  Institut f\"ur Theoretische Physik\\
  Eidgen\"ossische Technische Hochschule Z\"urich\\
  Wolfgang-Pauli-Strasse 27, 8093 Z\"urich, Switzerland\\[1ex]
  \href{mailto:nbeisert@itp.phys.ethz.ch}
  {\texttt{nbeisert@itp.phys.ethz.ch}}}
\hypersetup{pdfauthor={Niklas Beisert}}
\hypersetup{pdfsubject={Manual for the LaTeX2e Package childdoc}}
\date{30 December 2018, \textsf{v2.0}}
\maketitle

\begin{abstract}\noindent
\textsf{childdoc} is a \LaTeXe{} package
that enables the direct compilation
of document sections included by |\include|
to individual files.
\end{abstract}

\begingroup
\parskip0ex
\tableofcontents
\endgroup

%%%%%%%%%%%%%%%%%%%%%%%%%%%%%%%%%%%%%%%%%%%%%%%%%%%%%%%%%%%%%%%%%%%%%%%%%%%%%%%%
%%%%%%%%%%%%%%%%%%%%%%%%%%%%%%%%%%%%%%%%%%%%%%%%%%%%%%%%%%%%%%%%%%%%%%%%%%%%%%%%
\section{Introduction}

\LaTeX{} provides a mechanism to structure a large document (such as a book)
into a main file and several child files (containing the chapters)
using the |\include| command.
This mechanism is beneficial for documents
which span hundreds of pages in order to
make the source file(s) more manageable.
Moreover, compilation can be restricted to
selected child files by means of the |\includeonly| command.
The latter feature can be used to reduce the compilation time while editing
(this was significantly more useful in the earlier days of \LaTeX{})
or to generate a smaller document which is easier to navigate.
Another application of |\includeonly| is to generate
documents consisting of selected parts of the complete document.

However, there are a few drawbacks of the plain |\include| mechanism:
\begin{itemize}
\item
The child files cannot be compiled on their own,
they can only be compiled via the main file.
A naive editing environment
(such as a text editor with an option
to have the current file processed by \LaTeX)
may require one to switch to the main file before compiling;
attempting to compile the child file produces errors.
\item
The main file must be modified (each time)
to adjust the |\includeonly| command
to the present needs. This easily leaves the main file in a messy state.
\item
The generated document will always carry the filename
of the main document. This is inconvenient if
several child files are to be compiled and
to be kept for distribution.
\end{itemize}

The present package provides a simple interface
to make child files individually compilable by \LaTeX{}.
Compiling a child file then has the same effect as compiling
the main file with an |\includeonly| command
to select the appropriate child.
Moreover the generated document will carry the name of the child
rather than the main file.
This resolves all three above issues.

This feature is meant to make the editing of books,
thesis documents and lecture notes somewhat more convenient.
However, the package can also be used efficiently for
composing a series of documents (such as exercise sheets)
which are typically distributed individually.
It then assists the author in generating the individual documents
(potentially in different versions)
as well as a document containing the collected series.
Another application is in developing style files
or other kinds of included material
where compilation of the style file could redirect
to a sample or test file.

%%%%%%%%%%%%%%%%%%%%%%%%%%%%%%%%%%%%%%%%%%%%%%%%%%%%%%%%%%%%%%%%%%%%%%%%%%%%%%%%
%%%%%%%%%%%%%%%%%%%%%%%%%%%%%%%%%%%%%%%%%%%%%%%%%%%%%%%%%%%%%%%%%%%%%%%%%%%%%%%%
\section{Usage}

First of all, the package \textsf{childdoc} is \emph{not} a standard
\LaTeXe{} |.sty| style file! Therefore it needs to be invoked in
a non-standard way.

%%%%%%%%%%%%%%%%%%%%%%%%%%%%%%%%%%%%%%%%%%%%%%%%%%%%%%%%%%%%%%%%%%%%%%%%%%%%%%%%
\subsection{Included Files}
\label{sec:include}

%%%%%%%%%%%%%%%%%%%%%%%%%%%%%%%%%%%%%%%%
\DescribeMacro{\childdocmain}
To use the package, add the commands
\begin{center}
\begin{tabular}{l}
|\input{childdoc.def}|\\
|\childdocmain{}|\\
\end{tabular}
\end{center}
at the very top of the main \LaTeX{} file,
in particular \emph{before} the |\documentclass| statement!
The argument of |\childdocmain| should be left empty
(but it must be present).

%%%%%%%%%%%%%%%%%%%%%%%%%%%%%%%%%%%%%%%%
\DescribeMacro{\childdocof}
Furthermore, add the commands
\begin{center}
\begin{tabular}{l}
|\input{childdoc.def}|\\
|\childdocof{|\textit{main}|}|\\
\end{tabular}
\end{center}
at the top of every child file \textit{child}
which is included by |\include{|\textit{child}|}|
from within the main file
(or at least for those files to be compiled individually).
The argument \textit{main} must be the filename of the main file.

There are a couple of
considerations in setting up the main and child documents:

%%%%%%%%%%%%%%%%%%%%%%%%%%%%%%%%%%%%%%%%
\paragraph{Restrictions.}

Please note the following restrictions:
\begin{itemize}
\item
|\childdocmain| must be called with one argument \textit{main}
to ensure compatibility with earlier version of the package.
It must either be empty (|\childdocmain{}|)
or precisely match the filename of the main file in which it is specified.
See \secref{sec:detection} for further information.
\item
The filename \textit{main} must be specified without the |.tex| extension.
\item
The filename \textit{main} is case sensitive
(even in case-insensitive file systems)
due to internal string comparison.
\item
The argument \textit{main} should be fully expanded, it cannot be a macro.
\item
Subdirectories and special characters should be avoided in filenames.
\item
The command |\childdocmain{|\textit{main}|}| must be followed by a whitespace.
It should not be followed immediately by another command
or by a comment mark `|%|'.
This is because the \TeX{} parser reads the token immediately following
the argument of |\childdocmain| and puts it
at the beginning of every child section;
however, a white\-space is ignored.
\end{itemize}

%%%%%%%%%%%%%%%%%%%%%%%%%%%%%%%%%%%%%%%%
\paragraph{Content of Main File.}

It is advisable to place all content in the child files included by |\include|.
Any output contained in the main file will appear in all child documents
unless suppressed manually;
it cannot be suppressed automatically by the |\includeonly| directive
and thus should normally be avoided.
A method to include some content in the main file
by means of conditional processing is described in \secref{sec:conditional}.

%%%%%%%%%%%%%%%%%%%%%%%%%%%%%%%%%%%%%%%%
\paragraph{Page Numbering.}

When only a part of the document is compiled,
the appropriate numbering of pages
(as well as other status parameters)
is determined from the |.aux| files.
The latter contain information from previous passes.
However this information needs to propagate through
all intermediate child documents.
Therefore the page numbering in child documents may well
be inconsistent until the complete document is compiled at least once.

A useful (if unconventional) way to always ensure a consistent
page numbering is to restart the numbering in each child document
and denote the pages by `\textit{child}|.|\textit{page}'
where \textit{child} represents the chapter/section number of the child file.
This can be achieved by the command
|\numberwithin{page}{|\textit{child}|}|
of the \textsf{amsmath} package
where \textit{child} can be |chapter| or |section|
depending on the chosen structuring.
Alternatively, one can modify the macro |\thepage| appropriately
and reset the counter |page| at the start of each child file.

%%%%%%%%%%%%%%%%%%%%%%%%%%%%%%%%%%%%%%%%%%%%%%%%%%%%%%%%%%%%%%%%%%%%%%%%%%%%%%%%
\subsection{Conditional Processing}
\label{sec:conditional}

The package provides a mechanism to compile different versions
of a document. To customise the versions further some conditional processing
can come in handy to distinguish which version is being compiled.
The package provides two macros to describe the compilation context:

%%%%%%%%%%%%%%%%%%%%%%%%%%%%%%%%%%%%%%%%
\DescribeMacro{\ifchilddoc}
The conditional |\ifchilddoc| distinguishes between the compilation of
child documents and the main document:
%
\begin{center}
|\ifchilddoc |\textit{child-code}| |[|\||else |\textit{main-code}]| \||fi|
\end{center}

%%%%%%%%%%%%%%%%%%%%%%%%%%%%%%%%%%%%%%%%
\DescribeMacro{\childdocname}
\DescribeMacro{\childdocjob}
The macro |\childdocname| contains the filename (without extension)
of the main or child file being processed.
Note that |\childdocjob| will always contain the name of the main file.

%%%%%%%%%%%%%%%%%%%%%%%%%%%%%%%%%%%%%%%%
\paragraph{Title Page.}

Conditional processing can be used to include a title or banner page
in the main document when proper precautions are taken.
Importantly, the code in the main file should ensure that the page counter
(as well as other status parameters which are stored in the |.aux| files)
takes the same value after the conditional processing.
Otherwise the page numbers may take divergent values
depending on which part is compiled.

For example, a title page could be declared by:
%
\begin{center}
\begin{tabular}{l}
|\ifchilddoc\||else|\\
|\addtocounter{page}{-1}|\\
\textit{code for title page}\\
|\newpage|\\
|\||fi|
\end{tabular}
\end{center}
%
A banner page for the child documents can be generated by:
%
\begin{center}
\begin{tabular}{l}
|\ifchilddoc|\\
|\addtocounter{page}{-1}|\\
\textit{code for banner page}\\
|\newpage|\\
|\||fi|
\end{tabular}
\end{center}
%
Here one could write a message such as:
\begin{center}
|This is the part \childdocname{} of \childdocjob{}.|
\end{center}

%%%%%%%%%%%%%%%%%%%%%%%%%%%%%%%%%%%%%%%%%%%%%%%%%%%%%%%%%%%%%%%%%%%%%%%%%%%%%%%%
\subsection{Flags}
\label{sec:flags}

The package makes it easy to generate different versions
of the main or child documents.
To this end compilation flags can be defined
and assigned different default values.
They will be particularly useful in conjunction
with the forwarding mechanism described in \secref{sec:forward}.

For example, it may be useful to have a flag |\version|
which can be set to |draft| or |final|.
The document source will contain some conditional code
depending on the value of |\version|.
Suppose further, the flag should default to |final| for the main file
and to |draft| for child files
which is a natural assignment for editing the document.
This is achieved by placing the following code
in the preamble of the main document
(below the |\childdocmain| directive):
%
\begin{center}
\begin{tabular}{l}
|\ifchilddoc|\\
|\providecommand{\version}{draft}|\\
|\||else|\\
|\providecommand{\version}{final}|\\
|\||fi|
\end{tabular}
\end{center}
%
The definition by |\providecommand| makes sure
that previous definitions are not overwritten.
Further statements |\providecommand{\version}{...}|
can thus be added before the above code to override it.

For the main file, one might add a line
(between |\childdocmain| and the above block)
%
\begin{center}
|%\ifchilddoc\||else\providecommand{\version}{draft}\||fi|
\end{center}
%
which can be uncommented to produce a draft version.
Likewise one can add a line to the very top of a child file
(above the |\childdocof{|\textit{main}|}| directive)
%
\begin{center}
|%\providecommand{\version}{final}|
\end{center}
%
which can be uncommented to produce the final version of this child document.

%%%%%%%%%%%%%%%%%%%%%%%%%%%%%%%%%%%%%%%%%%%%%%%%%%%%%%%%%%%%%%%%%%%%%%%%%%%%%%%%
\subsection{Forwarding}
\label{sec:forward}

Different versions of the main or child documents
using compilation flags as described in \secref{sec:flags}
can be (permanently) stored in different files
for convenient compilation, viewing and distribution.
To this end, the package defines a command
to pass on compilation to a different file:

%%%%%%%%%%%%%%%%%%%%%%%%%%%%%%%%%%%%%%%%
\DescribeMacro{\childdocforward}
The command |\childdocforward| redirects processing to
another source file:
%
\begin{center}
\begin{tabular}{l}
|\input{childdoc.def}|\\
|\childdocforward[|\textit{main}|]{|\textit{dest}|}|\\
\end{tabular}
\end{center}
%
The argument \textit{dest} is the destination file
(without extension).
It should be the main file or one of the child files.
Note that further \textsf{childdoc} directives
such as |\childdocof| and |\childdocforward|
in the indicated file will be processed in this form.
The optional argument \textit{main}
passes on directly to the main file \textit{main}
while pretending to compile the child \textit{dest}.
This form behaves as if \textit{dest}
issues |\childdocof{|\textit{main}|}| right away,
and no further \textsf{childdoc} directives will be processed.

%%%%%%%%%%%%%%%%%%%%%%%%%%%%%%%%%%%%%%%%
\DescribeMacro{\...prefix}
In the alternative form |\childdocforwardprefix|,
%
\begin{center}
\begin{tabular}{l}
|\input{childdoc.def}|\\
|\childdocforwardprefix[|\textit{main}|]{|\textit{prefix}|}{|\textit{dest}|}|
\end{tabular}
\end{center}
%
the destination file is determined by a pattern
depending on the current file:
To make this work, the current file must be called
`{\textit{prefix}\hspace{0.2em}\textit{suffix}}'
with \textit{prefix} matching precisely the argument.
Processing is then passed on to the file
`{\textit{dest}\hspace{0.2em}\textit{suffix}}'.
Surely, the same effect is achieved by
directly specifying the
argument `{\textit{dest}\hspace{0.2em}\textit{suffix}}'
in the first form.
However, that requires to set up a different file
for each child. With the alternative form of the command
all these files can have exactly the same content
which simplifies setting them up and maintaining them.

For example, the following file |draft.tex|
with a compilation flag |\version| as described in \secref{sec:flags}
compiles the main document as a draft:
%
\begin{center}
\begin{tabular}{l}
|\def\version{draft}|\\
|\input{childdoc.def}|\\
|\childdocforward{|\textit{main}|}|
\end{tabular}
\end{center}
%
Likewise, the following files |final|\textit{nn}|.tex|
compile the final version of the child document
|child|\textit{nn}|.tex|:
%
\begin{center}
\begin{tabular}{l}
|\def\version{final}|\\
|\input{childdoc.def}|\\
|\childdocforwardprefix{final}{child}|
\end{tabular}
\end{center}
%

Note that when several versions of a main file and/or of each child file
are to be generated, it may be convenient to set up a |Makefile| or
shell script to automatise the process.

%%%%%%%%%%%%%%%%%%%%%%%%%%%%%%%%%%%%%%%%%%%%%%%%%%%%%%%%%%%%%%%%%%%%%%%%%%%%%%%%
\subsection{Command Line Processing}
\label{sec:commandline}

The effect of redirection files can also be achieved by invoking
the \LaTeX{} compiler with a more elaborate command line.
Most conveniently this should be done as part
of a shell script or a |Makefile|.

When using \textsf{childdoc} in the main file, the following
command lines effectively perform a redirection
(note that depending on the shell being used,
backslashes may have to be doubled: `|\|' $\to$ `|\\|'):
%
\begin{center}
|... -jobname "|\textit{target}|" |\\|"|[\textit{flags}]%
|\input{childdoc.def}\childdocforward[|\textit{main}|]{|\textit{dest}|}"|
\end{center}
%
Here \textit{target} is the name of the output file,
\textit{main} is the name of the main file
and \textit{dest} is the name of the main or child file to be processed
(all filenames without extensions).
The optional argument \textit{main} can be omitted
if \textit{main} matches \textit{dest}.
Optionally, compilation \textit{flags} can be defined via |\def| commands.
This command line makes the \TeX{} engine believe
it is compiling the file \textit{target}
whose content is specified as the latter parameter.
The provided code then forwards the processing to
\textit{main} or \textit{dest} as described in \secref{sec:forward}.

%%%%%%%%%%%%%%%%%%%%%%%%%%%%%%%%%%%%%%%%%%%%%%%%%%%%%%%%%%%%%%%%%%%%%%%%%%%%%%%%
\subsection{Include by Input}
\label{sec:input}

Including child documents by |\include| has some restrictions by design.
Most notably, the content of a child document always occupies
its own set of pages; pages cannot be shared between child documents.
Usually, this behaviour makes perfect sense
because each child document contain an essential part of the document.
However, in some situations it may be desirable to compose
a document from a collection of parts
without having mandatory page breaks between then.
For this case, the package
provides a mechanism to include parts
by |\input| which can also be processed individually.
However, by construction this mechanism
requires manual handling of the content to be output.

%%%%%%%%%%%%%%%%%%%%%%%%%%%%%%%%%%%%%%%%
\DescribeMacro{\ifchilddocmanual}
The main file should be prepared as usual, see \secref{sec:include}.
However, the document body must make a distinction
between processing of an individual part and of the main document, e.g.:
%
\begin{center}
\begin{tabular}{l}
|\ifchilddocmanual|\\
|\input{\childdocname}|\\
|\||else|\\
\textit{document body with }|\input{|\textit{part}|}|\\
|\||fi|
\end{tabular}
\end{center}
%
The conditional |\ifchilddocmanual| is true whenever
a part to be included by |\input| is being compiled,
and the name of the part is stored in |\childdocname|.

%%%%%%%%%%%%%%%%%%%%%%%%%%%%%%%%%%%%%%%%
\DescribeMacro{\childdocby}
Each part to be included by |\input| should start with:
%
\begin{center}
\begin{tabular}{l}
|\input{childdoc.def}|\\
|\childdocby{|\textit{main}|}|\\
\end{tabular}
\end{center}
%
The directive |\childdocby| is similar to |\childdocof|
described in \secref{sec:include},
but the subsequent selection of content must be done manually.
To that end, both |\ifchilddoc| and |\ifchilddocmanual|
will be true upon processing of a part,
and the name of the part is stored in |\childdocname|.
Note that |\jobname| will be set to the filename of the current part
so that each part receives an individual |.aux| file
that does not interfere with the |.aux| file(s) of the main document.
This behaviour can be altered by the alternative form
|\childdocby[*]{|\textit{main}|}| (with a non-empty optional argument)
which uses the |.aux| file of the main document
by setting |\jobname| to \textit{main}.

%%%%%%%%%%%%%%%%%%%%%%%%%%%%%%%%%%%%%%%%%%%%%%%%%%%%%%%%%%%%%%%%%%%%%%%%%%%%%%%%
\subsection{Driver Development}
\label{sec:driver}

The \textsf{childdoc} mechanism can also be use for the development
of definition files such as \LaTeX{} styles or classes.
This case differs from the above setup with multiple parts
included by |\include| in that no |\includeonly| should be invoked.
This can be achieved by starting the include file
(before |\ProvidesPackage|) with:
%
\begin{center}
\begin{tabular}{l}
|\input{childdoc.def}|\\
|\childdocforward{|\textit{main}|}|\\
\end{tabular}
\end{center}
%
or alternatively with:
%
\begin{center}
\begin{tabular}{l}
|\input{childdoc.def}|\\
|\childdocby{|\textit{main}|}|\\
\end{tabular}
\end{center}
%
Both forms have slightly different effects as described above.
The main file is prepared as usual, see \secref{sec:include}.

%%%%%%%%%%%%%%%%%%%%%%%%%%%%%%%%%%%%%%%%%%%%%%%%%%%%%%%%%%%%%%%%%%%%%%%%%%%%%%%%
\subsection{Legacy Detection}
\label{sec:detection}

The directive |\childdocmain| in the main file can detect
whether the complete document or merely a child is to be compiled
even without using the directive |\childdocof|.
This method is deprecated because it is less robust
and there is no compelling reason to use it;
it is merely provided for backward compatibility
and it may be removed in future versions.

If the detection mechanism is to be used,
it is mandatory to correctly specify
the filename of the main file as the argument of |\childdocmain|:
%
\begin{center}
\begin{tabular}{l}
|\input{childdoc.def}|\\
|\childdocmain{|\textit{main}|}|\\
\end{tabular}
\end{center}
%
If |\jobname| does not match the argument \textit{main} of |\childdocmain|,
it is assumed that |\jobname| points to the child file to be compiled.
When using |\childdocmain| with the main file specified as argument,
it suffices to start a child file
with just |\input{|\textit{main}|}|
without loading of the package and using |\childdocof|.
If instead all processing is done
with the appropriate \textsf{childdoc} directives,
the argument of \textit{main} of |\childdocmain| can be empty.

An alternative version of the command line processing described
in \secref{sec:commandline} using the detection mechanism reads:
%
\begin{center}
|... -jobname "|\textit{target}|" "|[\textit{flags}]%
[|\def\jobname{|\textit{dest}|}|]|\input{|\textit{main}|}"|
\end{center}

%%%%%%%%%%%%%%%%%%%%%%%%%%%%%%%%%%%%%%%%%%%%%%%%%%%%%%%%%%%%%%%%%%%%%%%%%%%%%%%%
\subsection{Manual Code}
\label{sec:manual}

In case one cannot be certain whether the definitions file |childdoc.def|
is installed on the target \TeX{} distribution
and one prefers not to ship it,
it is conceivable to paste a few relevant commands into the sources.

To that end, drop all statements |\input{childdoc.def}|
and perform the replacements as outlined below.
Instead of |\childdocmain{|\textit{main}|}| add the following code
to the top of the main file:
%
\begin{center}
\begin{tabular}{l}
|\||ifdefined\childdocname\endinput\||fi\newif\ifchilddoc|\\
|\edef\childdocname{\scantokens\expandafter{\jobname\noexpand}}|\\
|\def\childdocmain{|\textit{main}|}\||ifx\childdocmain\childdocname\||else|\\
|\childdoctrue\includeonly{\childdocname}\let\jobname\childdocmain\||fi|\\
\end{tabular}
\end{center}
%
Instead of |\childdocof{|\textit{main}|}| just include the main file
at the top of each child file:
%
\begin{center}
|\input{|\textit{main}|}|
\end{center}
%
A simple redirection |\childdocforward{|\textit{dest}|}| is achieved by:
%
\begin{center}
|\def\jobname{|\textit{dest}|}\input{\jobname}|
\end{center}
%
The redirection with prefix
|\childdocforwardprefix[|\textit{prefix}|]{|\textit{dest}|}|
is accomplished by:
%
\begin{center}
\begin{tabular}{l}
|{\edef\jobname{\scantokens\expandafter{\jobname\noexpand}}|\\
|\def\redirectjob |\textit{prefix}|#1~~~{\gdef\jobname{|\textit{dest}|#1}}|\\
|\expandafter\redirectjob\jobname~~~}\input{\jobname}|
\end{tabular}
\end{center}

In an alternative approach,
child documents can be compiled by a specific command line
without additional code or specific definitions:
%
\begin{center}
|... -jobname "|\textit{target}|" "|[\textit{flags}]%
|\includeonly{|\textit{dest}|}\input{|\textit{main}|}"|
\end{center}
%

%%%%%%%%%%%%%%%%%%%%%%%%%%%%%%%%%%%%%%%%%%%%%%%%%%%%%%%%%%%%%%%%%%%%%%%%%%%%%%%%
%%%%%%%%%%%%%%%%%%%%%%%%%%%%%%%%%%%%%%%%%%%%%%%%%%%%%%%%%%%%%%%%%%%%%%%%%%%%%%%%
\section{Information}

%%%%%%%%%%%%%%%%%%%%%%%%%%%%%%%%%%%%%%%%%%%%%%%%%%%%%%%%%%%%%%%%%%%%%%%%%%%%%%%%
\subsection{Copyright}

Copyright \copyright{} 2017--2018 Niklas Beisert

This work may be distributed and/or modified under the
conditions of the \LaTeX{} Project Public License, either version 1.3
of this license or (at your option) any later version.
The latest version of this license is in
  \url{http://www.latex-project.org/lppl.txt}
and version 1.3 or later is part of all distributions of \LaTeX{}
version 2005/12/01 or later.

This work has the LPPL maintenance status `maintained'.

The Current Maintainer of this work is Niklas Beisert.

This work consists of the files |README.txt|, |childdoc.ins| and |childdoc.dtx|
as well as the derived files |childdoc.def|, |cdocsamp.tex|
with |cdocsch1.tex|, |cdocsch2.tex|, |cdocspt3.tex|, |cdocspt4.tex|,
|cdocsdrf.tex|, |cdocsfn1.tex|, |cdocsfn2.tex|
as well as |childdoc.pdf|.

%%%%%%%%%%%%%%%%%%%%%%%%%%%%%%%%%%%%%%%%%%%%%%%%%%%%%%%%%%%%%%%%%%%%%%%%%%%%%%%%
\subsection{Files and Installation}

The package consists of the files:
%
\begin{center}
\begin{tabular}{ll}
    |README.txt|   & readme file \\
    |childdoc.ins| & installation file \\
    |childdoc.dtx| & source file \\
    |childdoc.def| & definition file \\
    |cdocsamp.tex| & sample main file \\
    |cdocsch1.tex| & sample include file \\
    |cdocsch2.tex| & sample include file \\
    |cdocspt3.tex| & sample part file \\
    |cdocspt4.tex| & sample part file \\
    |cdocsdrf.tex| & sample redirection file \\
    |cdocsfn1.tex| & sample redirection file \\
    |cdocsfn2.tex| & sample redirection file \\
    |childdoc.pdf| & manual
\end{tabular}
\end{center}
%
The distribution consists of the files
|README.txt|, |childdoc.ins| and |childdoc.dtx|.
%
\begin{itemize}
\item
Run (pdf)\LaTeX{} on |childdoc.dtx|
to compile the manual |childdoc.pdf| (this file).
\item
Run \LaTeX{} on |childdoc.ins| to create the definitions file |childdoc.def|
and the sample |cdocsamp.tex| with include files
|cdocsch1.tex|, |cdocsch2.tex|, |cdocspt3.tex|, |cdocspt4.tex|,
|cdocsdrf.tex|, |cdocsfn1.tex|, |cdocsfn2.tex|.
Then copy the file |childdoc.def| to an appropriate directory of your \LaTeX{}
distribution, e.g.\ \textit{texmf-root}|/tex/latex/childdoc|.
\end{itemize}

%%%%%%%%%%%%%%%%%%%%%%%%%%%%%%%%%%%%%%%%%%%%%%%%%%%%%%%%%%%%%%%%%%%%%%%%%%%%%%%%
\subsection{Related CTAN Packages}

There are several other packages which offer a similar functionality:
%
\begin{itemize}
\item
The packages
\href{http://ctan.org/pkg/docmute}{\textsf{docmute}},
\href{http://ctan.org/pkg/includex}{\textsf{includex}} and
\href{http://ctan.org/pkg/standalone}{\textsf{standalone}}
provide commands to include only the document body of
a child file thus allowing both files to be compiled individually.
\item
The packages \href{http://ctan.org/pkg/subdocs}{\textsf{subdocs}}
and \href{http://ctan.org/pkg/subfiles}{\textsf{subfiles}}
provide structures in which the main and child documents can be
encapsulated and allowing them to be compiled individually.
The inclusion mechanism is different from the conventional |\include|.
\item
The package \href{http://ctan.org/pkg/combine}{\textsf{combine}}
is an elaborate solution to combine several documents into one.
\end{itemize}
%
See also the CTAN topic \href{http://ctan.org/topic/subdocs}{\textsf{subdocs}}
for further related packages.
The present package differs from the above solutions in that
a document structure constructed with the conventional |\include| mechanism
just needs two extra commands at the top of every file
such that all constituent files can be compiled individually.

%%%%%%%%%%%%%%%%%%%%%%%%%%%%%%%%%%%%%%%%%%%%%%%%%%%%%%%%%%%%%%%%%%%%%%%%%%%%%%%%
%\subsection{Feature Suggestions}
%
%The following is a list of features which may be useful for future
%versions of this package:
%%
%\begin{itemize}
%\item
%\ldots
%\end{itemize}

%%%%%%%%%%%%%%%%%%%%%%%%%%%%%%%%%%%%%%%%%%%%%%%%%%%%%%%%%%%%%%%%%%%%%%%%%%%%%%%%
\subsection{Revision History}

%%%%%%%%%%%%%%%%%%%%%%%%%%%%%%%%%%%%%%%%
\paragraph{v2.0:} 2018/12/30

\begin{itemize}
\item
immediate forward processing
\item
added |\childdocby| mechanism
\item
manual restructured
\end{itemize}

%%%%%%%%%%%%%%%%%%%%%%%%%%%%%%%%%%%%%%%%
\paragraph{v1.6:} 2018/01/17

\begin{itemize}
\item
application for development of include files
\item
corrections to manual
\end{itemize}

%%%%%%%%%%%%%%%%%%%%%%%%%%%%%%%%%%%%%%%%
\paragraph{v1.5:} 2017/05/21

\begin{itemize}
\item
more complete structuring introduced
\item
|\childdocof| introduced
\item
|\childdoc| renamed to |\childdocmain|
\item
|\childredirect| renamed to |\childdocforward| and |\childdocforwardprefix|
and functionality expanded
\end{itemize}

%%%%%%%%%%%%%%%%%%%%%%%%%%%%%%%%%%%%%%%%
\paragraph{v1.0:} 2017/04/27

\begin{itemize}
\item
manual and install package
\item
first version published on CTAN
\end{itemize}

%%%%%%%%%%%%%%%%%%%%%%%%%%%%%%%%%%%%%%%%
\paragraph{v0.6:} 2017/04/26

\begin{itemize}
\item
redirection mechanism added
\end{itemize}

%%%%%%%%%%%%%%%%%%%%%%%%%%%%%%%%%%%%%%%%
\paragraph{v0.5:} 2017/04/26

\begin{itemize}
\item
functionality in definition file
\end{itemize}


%%%%%%%%%%%%%%%%%%%%%%%%%%%%%%%%%%%%%%%%%%%%%%%%%%%%%%%%%%%%%%%%%%%%%%%%%%%%%%%%
%%%%%%%%%%%%%%%%%%%%%%%%%%%%%%%%%%%%%%%%%%%%%%%%%%%%%%%%%%%%%%%%%%%%%%%%%%%%%%%%
%%%%%%%%%%%%%%%%%%%%%%%%%%%%%%%%%%%%%%%%%%%%%%%%%%%%%%%%%%%%%%%%%%%%%%%%%%%%%%%%
\appendix

\settowidth\MacroIndent{\rmfamily\scriptsize 000\ }

 \DocInput{childdoc.dtx}

\end{document}
%</driver>
% \fi
%
% %%%%%%%%%%%%%%%%%%%%%%%%%%%%%%%%%%%%%%%%%%%%%%%%%%%%%%%%%%%%%%%%%%%%%%%%%%%%%%
% %%%%%%%%%%%%%%%%%%%%%%%%%%%%%%%%%%%%%%%%%%%%%%%%%%%%%%%%%%%%%%%%%%%%%%%%%%%%%%
% \section{Sample}
%\iffalse
%<*samplemain>
%\fi
%
% The following presents a sample document
% with two chapters, two parts, a title page,
% a compile flag as well as three forwarding files to set the flag.
% It consists of eight |.tex| files:
% \begin{center}
% \begin{tabular}{ll}
% |cdocsamp.tex|&main file\\
% |cdocsch1.tex|&include file for chapter 1\\
% |cdocsch2.tex|&include file for chapter 2\\
% |cdocspt3.tex|&include file for part 3\\
% |cdocspt4.tex|&include file for part 4\\
% |cdocsdrf.tex|&forwarding file for main file in draft mode\\
% |cdocsfi1.tex|&forwarding file for final version of chapter 1\\
% |cdocsfi2.tex|&forwarding file for final version of chapter 2\\
% \end{tabular}
% \end{center}
% Each of the eight files can be compiled directly by the \LaTeX{} compiler.
%
% %%%%%%%%%%%%%%%%%%%%%%%%%%%%%%%%%%%%%%
% \paragraph{Main File.}
%
% The main file is called |cdocsamp.tex|.
%
% Load the \textsf{childdoc} definitions and
% declare the filename for the main document:
%    \begin{macrocode}
\input{childdoc.def}
\childdocmain{}
%    \end{macrocode}

% Optional override for |\version| flag:
%    \begin{macrocode}
%%\ifchilddoc\else\providecommand{\version}{draft}\fi
%    \end{macrocode}

% Define the default values for the |\version| flag
% (|final| for the main file and |draft| for childs):
%    \begin{macrocode}
\ifchilddoc
\providecommand{\version}{draft}
\else
\providecommand{\version}{final}
\fi
%    \end{macrocode}

% Load the standard document class:
%    \begin{macrocode}
\documentclass[12pt]{article}
%    \end{macrocode}

% Start the document body:
%    \begin{macrocode}
\begin{document}
%    \end{macrocode}

% Declare a title page.
% Print title, part of document being processed and version flag:
%    \begin{macrocode}
\addtocounter{page}{-1}
\begin{center}
{\LARGE\bfseries{}childdoc example\par}
\vspace{1cm}
\ifchilddoc
\ifchilddocmanual part\else chapter\fi:
`\childdocname' of `\childdocjob'\par
\else
main document: `\childdocjob'\par
\fi
version: \version\par
\end{center}
\newpage
%    \end{macrocode}

% Manually include selected file,
% otherwise process as usual:
%    \begin{macrocode}
\ifchilddocmanual
\section*{part `\childdocname'}
\input{\childdocname}
\else
%    \end{macrocode}

% Include the two chapters:
%    \begin{macrocode}
\include{cdocsch1}
\include{cdocsch2}
%    \end{macrocode}

% Include the two parts unless only chapters should be displayed:
%    \begin{macrocode}
\ifchilddoc\else
\section{part three}
\input{cdocspt3}
\section{part four}
\input{cdocspt4}
\fi
%    \end{macrocode}

% Process as usual until here:
%    \begin{macrocode}
\fi
%    \end{macrocode}

% End of document body:
%    \begin{macrocode}
\end{document}
%    \end{macrocode}
%\iffalse
%</samplemain>
%\fi
%
% %%%%%%%%%%%%%%%%%%%%%%%%%%%%%%%%%%%%%%
% \paragraph{Chapter Include Files.}
%
% The include files are called |cdocsch1.tex| and |cdocsch2.tex|.
%
%\iffalse
%<*samplechap1|samplechap2>
%\fi

% Optional override for |\version| flag:
%    \begin{macrocode}
%%\providecommand{\version}{final}
%    \end{macrocode}

% Include the main document:
%    \begin{macrocode}
\input{childdoc.def}
\childdocof{cdocsamp}
%    \end{macrocode}

%\iffalse
%</samplechap1|samplechap2>
%\fi
%
%\iffalse
%<*samplechap1>
%\fi
% Some text for chapter 1:
%    \begin{macrocode}
\section{one}
some text in chapter one
%    \end{macrocode}

%\iffalse
%</samplechap1>
%\fi
% Some text for chapter 2:
%\iffalse
%<*samplechap2>
%\fi
%    \begin{macrocode}
\section{two}
more text in chapter two
%    \end{macrocode}

%\iffalse
%</samplechap2>
%\fi
%
% %%%%%%%%%%%%%%%%%%%%%%%%%%%%%%%%%%%%%%
% \paragraph{Part Include Files.}
%
% The include files are called |cdocspt3.tex| and |cdocspt4.tex|.
%
%\iffalse
%<*samplepart3|samplepart4>
%\fi

% Optional override for |\version| flag:
%    \begin{macrocode}
%%\providecommand{\version}{final}
%    \end{macrocode}

% Include the main document:
%    \begin{macrocode}
\input{childdoc.def}
\childdocby{cdocsamp}
%    \end{macrocode}

%\iffalse
%</samplepart3|samplepart4>
%\fi
%
%\iffalse
%<*samplepart3>
%\fi
% Some text for part 3:
%    \begin{macrocode}
some text in part three
%    \end{macrocode}

%\iffalse
%</samplepart3>
%\fi
% Some text for part 4:
%\iffalse
%<*samplepart4>
%\fi
%    \begin{macrocode}
more text in part four
%    \end{macrocode}

%\iffalse
%</samplepart4>
%\fi
%
% %%%%%%%%%%%%%%%%%%%%%%%%%%%%%%%%%%%%%%
% \paragraph{Forwarding for a Complete Draft.}
%
% The following forwarding file |cdocsdrf.tex|
% compiles the main document in draft mode:
%\iffalse
%<*sampledraft>
%\fi
%    \begin{macrocode}
\def\version{draft}
\input{childdoc.def}
\childdocforward{cdocsamp}
%    \end{macrocode}

%\iffalse
%</sampledraft>
%\fi
%
% %%%%%%%%%%%%%%%%%%%%%%%%%%%%%%%%%%%%%%
% \paragraph{Forwarding for Final Version of the Chapters.}
%
% The following forwarding files |cdocsfn1.tex| and |cdocsfn2.tex|
% (with identical content)
% compile the final versions of the child documents
% |cdocsch1.tex| and |cdocsch2.tex|, respectively:
%\iffalse
%<*samplefinal>
%\fi
%    \begin{macrocode}
\def\version{final}
\input{childdoc.def}
\childdocforwardprefix[cdocsamp]{cdocsfn}{cdocsch}
%    \end{macrocode}

%\iffalse
%</samplefinal>
%\fi
%
% %%%%%%%%%%%%%%%%%%%%%%%%%%%%%%%%%%%%%%
% \paragraph{Command Line Processing.}
%
% The following three command lines generate the output files
% |cdocscld|, |cdocscl1| and |cdocscl2|
% which should be identical to
% |cdocsdrf|, |cdocsch1| and |cdocsfn2|, respectively:
% \begin{center}
% \begin{tabular}{l}
% |latex -jobname cdocscld \|\\
% |  "\def\version{draft}\input{childdoc.def}\childdocforward{cdocsamp}"|\\
% |latex -jobname cdocscl1 \|\\
% |  "\input{childdoc.def}\childdocforward[cdocsamp]{cdocsch1}"|\\
% |latex -jobname cdocscl2 \|\\
% |  "\def\version{final}\input{childdoc.def}\childdocforward{cdocsch2}"|
% \end{tabular}
% \end{center}
% Note that the trailing backslash on each first line
% merely continues the input to the second line
% (for convenient cut ant paste).
% Furthermore, the command |latex| can be replaced by any
% of its alternative versions such as |pdflatex|.
%
% %%%%%%%%%%%%%%%%%%%%%%%%%%%%%%%%%%%%%%%%%%%%%%%%%%%%%%%%%%%%%%%%%%%%%%%%%%%%%%
% %%%%%%%%%%%%%%%%%%%%%%%%%%%%%%%%%%%%%%%%%%%%%%%%%%%%%%%%%%%%%%%%%%%%%%%%%%%%%%
% \section{Implementation}
%\iffalse
%<*package>
%\fi
%
% This section describes the definitions file |childdoc.def|.

% The definitions cannot be loaded using |\usepackage| or |\RequirePackage|
% which has a mechanism to prevent loading a style file more than once.
% When loading the definitions by means of |\input|
% multiple instances have to be prevented manually:
%\iffalse
%This code needs to be before the `\ProvidesFile' directive
%which is defined at the beginning of this file.
%Therefore it is also placed there and commented out here.
%</package>
%<*discard>
%\fi
%    \begin{macrocode}
\ifdefined\childdocmain\endinput\fi
%    \end{macrocode}
%\iffalse
%</discard>
%<*package>
%\fi
%
% \macro{\ifchilddoc}
% \macro{\ifchilddocmanual}
% The conditional |\ifchilddoc| tells whether a
% child (true) or main (false) document is being compiled.
% The conditional |\ifchilddocmanual| tells whether
% the |\includeonly| mechanism is used (false) or
% the selection of child files must be performed manually (true).
% The definitions initialise to false:
%    \begin{macrocode}
\newif\ifchilddoc
\newif\ifchilddocmanual
%    \end{macrocode}

% \macro{\childdocname}
% \macro{\childdocjob}
% The macro |\childdocname| stores the name of the main document
% to be compiled. The macro |\childdocjob| stores the name of
% the document on which the \LaTeX{} compiler was originally invoked.
% The content of |\jobname| cannot be compared
% to filenames specified in the source due to different catcodes.
% The following code rescans |\jobname|, stores the result
% in |\childdocname| and saves a copy in |\childdocjob|:
%    \begin{macrocode}
\edef\childdocname{\scantokens\expandafter{\jobname\noexpand}}
\let\childdocjob\childdocname
%    \end{macrocode}

% \macro{\childdocdisable}
% The macro |\childdocdisable| prevents the main file
% from being processed more than once.
% At this stage, the main document command |\childdocmain|
% is assumed to be called once again where it should do nothing.
% Any subsequent call to it should prevent
% a secondary processing of the main document
% It overwrites the forwarding commands
% |\childdocof| and |\childdocforward|
% with empty macros to prevent further inclusions of the main document:
%    \begin{macrocode}
\newcommand{\childdocdisable}
{
  \renewcommand{\childdocmain}[1]{\renewcommand{\childdocmain}[1]{\endinput}}
  \renewcommand{\childdocof}[1]{}
  \renewcommand{\childdocby}[2][]{}
  \renewcommand{\childdocforward}[2][]{}
  \renewcommand{\childdocdisable}{}
}
%    \end{macrocode}

% \macro{\childdocmain}
% The macro |\childdocmain| is to be called at the top of the main file
% with nothing or the main filename (without extension) as argument.
% First, it breaks loops.
% If the argument is not empty and does not match |\childdocname|
% (which is set by the first inclusion of |childdoc.def|),
% |\ifchilddoc| is set to true, |\includeonly| is applied to the child file
% and |\jobname| is set to the main file
% (for proper handling of |.aux| files):
%    \begin{macrocode}
\newcommand{\childdocmain}[1]
{
  \childdocdisable\childdocmain{}
  \if?#1?\else
    \begingroup
      \def\childdoctmp{#1}
      \ifx\childdoctmp\childdocname
        \def\childdoctmp{}
      \else
        \def\childdoctmp
        {
          \childdoctrue
          \includeonly{\childdocname}
          \def\childdocjob{#1}
          \def\jobname{#1}
        }
      \fi
      \expandafter
    \endgroup
    \childdoctmp
  \fi
}
%    \end{macrocode}

% \macro{\childdocof}
% The command |\childdocof| redirects
% compilation to the main file |#1|.
%    \begin{macrocode}
\newcommand{\childdocof}[1]
{
  \childdocdisable
  \childdoctrue
  \includeonly{\childdocname}
  \def\jobname{#1}
  \def\childdocjob{#1}
  \input{#1}
}
%    \end{macrocode}

% \macro{\childdocby}
% The command |\childdocby| ....
%    \begin{macrocode}
\newcommand{\childdocby}[2][]
{
  \childdocdisable
  \childdoctrue
  \childdocmanualtrue
  \if?#1?\else
    \def\jobname{#2}
  \fi
  \def\childdocjob{#2}
  \input{#2}
  \endinput
}
%    \end{macrocode}

% \macro{\childdocforward}
% The command |\childdocforward| redirects
% compilation to the main file or
% (if the optional argument is given) a child file.
% Parameters are set as if the main file
% or a child file starting with |\childdocof| was compiled.
% Then compilation is handed over to the main file:
%    \begin{macrocode}
\newcommand{\childdocforward}[2][]
{
  \begingroup
    \if?#1?
      \def\childdoctmp
      {
        \def\childdocname{#2}
        \def\childdocjob{#2}
        \def\jobname{#2}
        \input{#2}
        \endinput
      }
    \else
      \def\childdoctmp
      {
        \childdocdisable
        \def\childdocname{#2}
        \childdoctrue
        \includeonly{#2}
        \def\childdocjob{#1}
        \def\jobname{#1}
        \input{#1}
        \endinput
      }
    \fi
    \expandafter
  \endgroup
  \childdoctmp
}
%    \end{macrocode}

% \macro{\childdocforwardprefix}
% The command |\childdocforwardprefix| redirects
% compilation to the main or a child file by means of a pattern.
% The prefix |#1| in the current filename is replaced by |#2|
% and the suffix of the current filename is kept
% (it is assumed that the filename does not contain the substring `|~~~|'
% which is used as a delimiter).
% Compilation is handed over to the new file by |\childdocforward|:
%    \begin{macrocode}
\newcommand{\childdocforwardprefix}[3][]
{
  \begingroup
    \def\childdocextract #2##1~~~{\def\childdoctmp{\childdocforward[#1]{#3##1}}}
    \expandafter\childdocextract\childdocname~~~
    \expandafter
  \endgroup
  \childdoctmp
}
%    \end{macrocode}

% \macro{\childdoc}
% The deprecated macro |\childdoc| is a legacy version of |\childdocmain|:
%    \begin{macrocode}
\newcommand{\childdoc}{\childdocmain}
%    \end{macrocode}

% \macro{\childdocredirect}
% The deprecated macro |\childdocredirect| is a legacy version
% of |\childdocforward| and |\childdocforwardprefix|:
%    \begin{macrocode}
\newcommand{\childdocredirect}[2][]
{
  \begingroup
    \if?#1?
      \def\childdoctmp{\childdocforward{#2}}
    \else
      \def\childdoctmp{\childdocforwardprefix{#1}{#2}}
    \fi
    \expandafter
  \endgroup
  \childdoctmp
}
%    \end{macrocode}

%\iffalse
%</package>
%\fi
%
\endinput
\childdocforward[|\textit{main}|]{|\textit{dest}|}"|
\end{center}
%
Here \textit{target} is the name of the output file,
\textit{main} is the name of the main file
and \textit{dest} is the name of the main or child file to be processed
(all filenames without extensions).
The optional argument \textit{main} can be omitted
if \textit{main} matches \textit{dest}.
Optionally, compilation \textit{flags} can be defined via |\def| commands.
This command line makes the \TeX{} engine believe
it is compiling the file \textit{target}
whose content is specified as the latter parameter.
The provided code then forwards the processing to
\textit{main} or \textit{dest} as described in \secref{sec:forward}.

%%%%%%%%%%%%%%%%%%%%%%%%%%%%%%%%%%%%%%%%%%%%%%%%%%%%%%%%%%%%%%%%%%%%%%%%%%%%%%%%
\subsection{Include by Input}
\label{sec:input}

Including child documents by |\include| has some restrictions by design.
Most notably, the content of a child document always occupies
its own set of pages; pages cannot be shared between child documents.
Usually, this behaviour makes perfect sense
because each child document contain an essential part of the document.
However, in some situations it may be desirable to compose
a document from a collection of parts
without having mandatory page breaks between then.
For this case, the package
provides a mechanism to include parts
by |\input| which can also be processed individually.
However, by construction this mechanism
requires manual handling of the content to be output.

%%%%%%%%%%%%%%%%%%%%%%%%%%%%%%%%%%%%%%%%
\DescribeMacro{\ifchilddocmanual}
The main file should be prepared as usual, see \secref{sec:include}.
However, the document body must make a distinction
between processing of an individual part and of the main document, e.g.:
%
\begin{center}
\begin{tabular}{l}
|\ifchilddocmanual|\\
|\input{\childdocname}|\\
|\||else|\\
\textit{document body with }|\input{|\textit{part}|}|\\
|\||fi|
\end{tabular}
\end{center}
%
The conditional |\ifchilddocmanual| is true whenever
a part to be included by |\input| is being compiled,
and the name of the part is stored in |\childdocname|.

%%%%%%%%%%%%%%%%%%%%%%%%%%%%%%%%%%%%%%%%
\DescribeMacro{\childdocby}
Each part to be included by |\input| should start with:
%
\begin{center}
\begin{tabular}{l}
|% \iffalse
%
% childdoc.dtx Copyright (C) 2017-2018 Niklas Beisert
%
% This work may be distributed and/or modified under the
% conditions of the LaTeX Project Public License, either version 1.3
% of this license or (at your option) any later version.
% The latest version of this license is in
%   http://www.latex-project.org/lppl.txt
% and version 1.3 or later is part of all distributions of LaTeX
% version 2005/12/01 or later.
%
% This work has the LPPL maintenance status `maintained'.
%
% The Current Maintainer of this work is Niklas Beisert.
%
% This work consists of the files childdoc.dtx and childdoc.ins
% and the derived files childdoc.def and cdocsamp.tex with
% cdocsch1.tex, cdocsch2.tex, cdocsdrf.tex, cdocsfn1.tex, cdocsfn2.tex.
%
%<package>\ifdefined\childdocmain\endinput\fi
%<package>\ProvidesFile{childdoc.def}[2018/12/30 v2.0 child document driver]
%<samplemain>\ProvidesFile{cdocsamp.tex}[2018/12/30 v2.0 sample for childdoc]
%<*driver>
%\ProvidesFile{childdoc.drv}[2018/12/30 v2.0 childdoc reference manual file]
\PassOptionsToClass{10pt,a4paper}{article}
\documentclass{ltxdoc}

\usepackage[margin=35mm]{geometry}
\usepackage{hyperref}
\usepackage{hyperxmp}
\usepackage[usenames]{color}

\hypersetup{colorlinks=true}
\hypersetup{pdfstartview=FitH}
\hypersetup{pdfpagemode=UseNone}
\hypersetup{pdfsource={}}
\hypersetup{pdflang={en-UK}}
\hypersetup{pdfcopyright={Copyright 2017-2018 Niklas Beisert.
  This work may be distributed and/or modified under the
  conditions of the LaTeX Project Public License, either version 1.3
  of this license or (at your option) any later version.}}
\hypersetup{pdflicenseurl={http://www.latex-project.org/lppl.txt}}
\hypersetup{pdfcontactaddress={ETH Zurich, ITP, HIT K,
  Wolfgang-Pauli-Strasse 27}}
\hypersetup{pdfcontactpostcode={8093}}
\hypersetup{pdfcontactcity={Zurich}}
\hypersetup{pdfcontactcountry={Switzerland}}
\hypersetup{pdfcontactemail={nbeisert@itp.phys.ethz.ch}}
\hypersetup{pdfcontacturl={http://people.phys.ethz.ch/\xmptilde nbeisert/}}

\newcommand{\secref}[1]{\hyperref[#1]{section \ref*{#1}}}

\parskip1ex
\parindent0pt
\let\olditemize\itemize
\def\itemize{\olditemize\parskip0pt}

\begin{document}

\title{The \textsf{childdoc} Package}
\hypersetup{pdftitle={The childdoc Package}}
\author{Niklas Beisert\\[2ex]
  Institut f\"ur Theoretische Physik\\
  Eidgen\"ossische Technische Hochschule Z\"urich\\
  Wolfgang-Pauli-Strasse 27, 8093 Z\"urich, Switzerland\\[1ex]
  \href{mailto:nbeisert@itp.phys.ethz.ch}
  {\texttt{nbeisert@itp.phys.ethz.ch}}}
\hypersetup{pdfauthor={Niklas Beisert}}
\hypersetup{pdfsubject={Manual for the LaTeX2e Package childdoc}}
\date{30 December 2018, \textsf{v2.0}}
\maketitle

\begin{abstract}\noindent
\textsf{childdoc} is a \LaTeXe{} package
that enables the direct compilation
of document sections included by |\include|
to individual files.
\end{abstract}

\begingroup
\parskip0ex
\tableofcontents
\endgroup

%%%%%%%%%%%%%%%%%%%%%%%%%%%%%%%%%%%%%%%%%%%%%%%%%%%%%%%%%%%%%%%%%%%%%%%%%%%%%%%%
%%%%%%%%%%%%%%%%%%%%%%%%%%%%%%%%%%%%%%%%%%%%%%%%%%%%%%%%%%%%%%%%%%%%%%%%%%%%%%%%
\section{Introduction}

\LaTeX{} provides a mechanism to structure a large document (such as a book)
into a main file and several child files (containing the chapters)
using the |\include| command.
This mechanism is beneficial for documents
which span hundreds of pages in order to
make the source file(s) more manageable.
Moreover, compilation can be restricted to
selected child files by means of the |\includeonly| command.
The latter feature can be used to reduce the compilation time while editing
(this was significantly more useful in the earlier days of \LaTeX{})
or to generate a smaller document which is easier to navigate.
Another application of |\includeonly| is to generate
documents consisting of selected parts of the complete document.

However, there are a few drawbacks of the plain |\include| mechanism:
\begin{itemize}
\item
The child files cannot be compiled on their own,
they can only be compiled via the main file.
A naive editing environment
(such as a text editor with an option
to have the current file processed by \LaTeX)
may require one to switch to the main file before compiling;
attempting to compile the child file produces errors.
\item
The main file must be modified (each time)
to adjust the |\includeonly| command
to the present needs. This easily leaves the main file in a messy state.
\item
The generated document will always carry the filename
of the main document. This is inconvenient if
several child files are to be compiled and
to be kept for distribution.
\end{itemize}

The present package provides a simple interface
to make child files individually compilable by \LaTeX{}.
Compiling a child file then has the same effect as compiling
the main file with an |\includeonly| command
to select the appropriate child.
Moreover the generated document will carry the name of the child
rather than the main file.
This resolves all three above issues.

This feature is meant to make the editing of books,
thesis documents and lecture notes somewhat more convenient.
However, the package can also be used efficiently for
composing a series of documents (such as exercise sheets)
which are typically distributed individually.
It then assists the author in generating the individual documents
(potentially in different versions)
as well as a document containing the collected series.
Another application is in developing style files
or other kinds of included material
where compilation of the style file could redirect
to a sample or test file.

%%%%%%%%%%%%%%%%%%%%%%%%%%%%%%%%%%%%%%%%%%%%%%%%%%%%%%%%%%%%%%%%%%%%%%%%%%%%%%%%
%%%%%%%%%%%%%%%%%%%%%%%%%%%%%%%%%%%%%%%%%%%%%%%%%%%%%%%%%%%%%%%%%%%%%%%%%%%%%%%%
\section{Usage}

First of all, the package \textsf{childdoc} is \emph{not} a standard
\LaTeXe{} |.sty| style file! Therefore it needs to be invoked in
a non-standard way.

%%%%%%%%%%%%%%%%%%%%%%%%%%%%%%%%%%%%%%%%%%%%%%%%%%%%%%%%%%%%%%%%%%%%%%%%%%%%%%%%
\subsection{Included Files}
\label{sec:include}

%%%%%%%%%%%%%%%%%%%%%%%%%%%%%%%%%%%%%%%%
\DescribeMacro{\childdocmain}
To use the package, add the commands
\begin{center}
\begin{tabular}{l}
|\input{childdoc.def}|\\
|\childdocmain{}|\\
\end{tabular}
\end{center}
at the very top of the main \LaTeX{} file,
in particular \emph{before} the |\documentclass| statement!
The argument of |\childdocmain| should be left empty
(but it must be present).

%%%%%%%%%%%%%%%%%%%%%%%%%%%%%%%%%%%%%%%%
\DescribeMacro{\childdocof}
Furthermore, add the commands
\begin{center}
\begin{tabular}{l}
|\input{childdoc.def}|\\
|\childdocof{|\textit{main}|}|\\
\end{tabular}
\end{center}
at the top of every child file \textit{child}
which is included by |\include{|\textit{child}|}|
from within the main file
(or at least for those files to be compiled individually).
The argument \textit{main} must be the filename of the main file.

There are a couple of
considerations in setting up the main and child documents:

%%%%%%%%%%%%%%%%%%%%%%%%%%%%%%%%%%%%%%%%
\paragraph{Restrictions.}

Please note the following restrictions:
\begin{itemize}
\item
|\childdocmain| must be called with one argument \textit{main}
to ensure compatibility with earlier version of the package.
It must either be empty (|\childdocmain{}|)
or precisely match the filename of the main file in which it is specified.
See \secref{sec:detection} for further information.
\item
The filename \textit{main} must be specified without the |.tex| extension.
\item
The filename \textit{main} is case sensitive
(even in case-insensitive file systems)
due to internal string comparison.
\item
The argument \textit{main} should be fully expanded, it cannot be a macro.
\item
Subdirectories and special characters should be avoided in filenames.
\item
The command |\childdocmain{|\textit{main}|}| must be followed by a whitespace.
It should not be followed immediately by another command
or by a comment mark `|%|'.
This is because the \TeX{} parser reads the token immediately following
the argument of |\childdocmain| and puts it
at the beginning of every child section;
however, a white\-space is ignored.
\end{itemize}

%%%%%%%%%%%%%%%%%%%%%%%%%%%%%%%%%%%%%%%%
\paragraph{Content of Main File.}

It is advisable to place all content in the child files included by |\include|.
Any output contained in the main file will appear in all child documents
unless suppressed manually;
it cannot be suppressed automatically by the |\includeonly| directive
and thus should normally be avoided.
A method to include some content in the main file
by means of conditional processing is described in \secref{sec:conditional}.

%%%%%%%%%%%%%%%%%%%%%%%%%%%%%%%%%%%%%%%%
\paragraph{Page Numbering.}

When only a part of the document is compiled,
the appropriate numbering of pages
(as well as other status parameters)
is determined from the |.aux| files.
The latter contain information from previous passes.
However this information needs to propagate through
all intermediate child documents.
Therefore the page numbering in child documents may well
be inconsistent until the complete document is compiled at least once.

A useful (if unconventional) way to always ensure a consistent
page numbering is to restart the numbering in each child document
and denote the pages by `\textit{child}|.|\textit{page}'
where \textit{child} represents the chapter/section number of the child file.
This can be achieved by the command
|\numberwithin{page}{|\textit{child}|}|
of the \textsf{amsmath} package
where \textit{child} can be |chapter| or |section|
depending on the chosen structuring.
Alternatively, one can modify the macro |\thepage| appropriately
and reset the counter |page| at the start of each child file.

%%%%%%%%%%%%%%%%%%%%%%%%%%%%%%%%%%%%%%%%%%%%%%%%%%%%%%%%%%%%%%%%%%%%%%%%%%%%%%%%
\subsection{Conditional Processing}
\label{sec:conditional}

The package provides a mechanism to compile different versions
of a document. To customise the versions further some conditional processing
can come in handy to distinguish which version is being compiled.
The package provides two macros to describe the compilation context:

%%%%%%%%%%%%%%%%%%%%%%%%%%%%%%%%%%%%%%%%
\DescribeMacro{\ifchilddoc}
The conditional |\ifchilddoc| distinguishes between the compilation of
child documents and the main document:
%
\begin{center}
|\ifchilddoc |\textit{child-code}| |[|\||else |\textit{main-code}]| \||fi|
\end{center}

%%%%%%%%%%%%%%%%%%%%%%%%%%%%%%%%%%%%%%%%
\DescribeMacro{\childdocname}
\DescribeMacro{\childdocjob}
The macro |\childdocname| contains the filename (without extension)
of the main or child file being processed.
Note that |\childdocjob| will always contain the name of the main file.

%%%%%%%%%%%%%%%%%%%%%%%%%%%%%%%%%%%%%%%%
\paragraph{Title Page.}

Conditional processing can be used to include a title or banner page
in the main document when proper precautions are taken.
Importantly, the code in the main file should ensure that the page counter
(as well as other status parameters which are stored in the |.aux| files)
takes the same value after the conditional processing.
Otherwise the page numbers may take divergent values
depending on which part is compiled.

For example, a title page could be declared by:
%
\begin{center}
\begin{tabular}{l}
|\ifchilddoc\||else|\\
|\addtocounter{page}{-1}|\\
\textit{code for title page}\\
|\newpage|\\
|\||fi|
\end{tabular}
\end{center}
%
A banner page for the child documents can be generated by:
%
\begin{center}
\begin{tabular}{l}
|\ifchilddoc|\\
|\addtocounter{page}{-1}|\\
\textit{code for banner page}\\
|\newpage|\\
|\||fi|
\end{tabular}
\end{center}
%
Here one could write a message such as:
\begin{center}
|This is the part \childdocname{} of \childdocjob{}.|
\end{center}

%%%%%%%%%%%%%%%%%%%%%%%%%%%%%%%%%%%%%%%%%%%%%%%%%%%%%%%%%%%%%%%%%%%%%%%%%%%%%%%%
\subsection{Flags}
\label{sec:flags}

The package makes it easy to generate different versions
of the main or child documents.
To this end compilation flags can be defined
and assigned different default values.
They will be particularly useful in conjunction
with the forwarding mechanism described in \secref{sec:forward}.

For example, it may be useful to have a flag |\version|
which can be set to |draft| or |final|.
The document source will contain some conditional code
depending on the value of |\version|.
Suppose further, the flag should default to |final| for the main file
and to |draft| for child files
which is a natural assignment for editing the document.
This is achieved by placing the following code
in the preamble of the main document
(below the |\childdocmain| directive):
%
\begin{center}
\begin{tabular}{l}
|\ifchilddoc|\\
|\providecommand{\version}{draft}|\\
|\||else|\\
|\providecommand{\version}{final}|\\
|\||fi|
\end{tabular}
\end{center}
%
The definition by |\providecommand| makes sure
that previous definitions are not overwritten.
Further statements |\providecommand{\version}{...}|
can thus be added before the above code to override it.

For the main file, one might add a line
(between |\childdocmain| and the above block)
%
\begin{center}
|%\ifchilddoc\||else\providecommand{\version}{draft}\||fi|
\end{center}
%
which can be uncommented to produce a draft version.
Likewise one can add a line to the very top of a child file
(above the |\childdocof{|\textit{main}|}| directive)
%
\begin{center}
|%\providecommand{\version}{final}|
\end{center}
%
which can be uncommented to produce the final version of this child document.

%%%%%%%%%%%%%%%%%%%%%%%%%%%%%%%%%%%%%%%%%%%%%%%%%%%%%%%%%%%%%%%%%%%%%%%%%%%%%%%%
\subsection{Forwarding}
\label{sec:forward}

Different versions of the main or child documents
using compilation flags as described in \secref{sec:flags}
can be (permanently) stored in different files
for convenient compilation, viewing and distribution.
To this end, the package defines a command
to pass on compilation to a different file:

%%%%%%%%%%%%%%%%%%%%%%%%%%%%%%%%%%%%%%%%
\DescribeMacro{\childdocforward}
The command |\childdocforward| redirects processing to
another source file:
%
\begin{center}
\begin{tabular}{l}
|\input{childdoc.def}|\\
|\childdocforward[|\textit{main}|]{|\textit{dest}|}|\\
\end{tabular}
\end{center}
%
The argument \textit{dest} is the destination file
(without extension).
It should be the main file or one of the child files.
Note that further \textsf{childdoc} directives
such as |\childdocof| and |\childdocforward|
in the indicated file will be processed in this form.
The optional argument \textit{main}
passes on directly to the main file \textit{main}
while pretending to compile the child \textit{dest}.
This form behaves as if \textit{dest}
issues |\childdocof{|\textit{main}|}| right away,
and no further \textsf{childdoc} directives will be processed.

%%%%%%%%%%%%%%%%%%%%%%%%%%%%%%%%%%%%%%%%
\DescribeMacro{\...prefix}
In the alternative form |\childdocforwardprefix|,
%
\begin{center}
\begin{tabular}{l}
|\input{childdoc.def}|\\
|\childdocforwardprefix[|\textit{main}|]{|\textit{prefix}|}{|\textit{dest}|}|
\end{tabular}
\end{center}
%
the destination file is determined by a pattern
depending on the current file:
To make this work, the current file must be called
`{\textit{prefix}\hspace{0.2em}\textit{suffix}}'
with \textit{prefix} matching precisely the argument.
Processing is then passed on to the file
`{\textit{dest}\hspace{0.2em}\textit{suffix}}'.
Surely, the same effect is achieved by
directly specifying the
argument `{\textit{dest}\hspace{0.2em}\textit{suffix}}'
in the first form.
However, that requires to set up a different file
for each child. With the alternative form of the command
all these files can have exactly the same content
which simplifies setting them up and maintaining them.

For example, the following file |draft.tex|
with a compilation flag |\version| as described in \secref{sec:flags}
compiles the main document as a draft:
%
\begin{center}
\begin{tabular}{l}
|\def\version{draft}|\\
|\input{childdoc.def}|\\
|\childdocforward{|\textit{main}|}|
\end{tabular}
\end{center}
%
Likewise, the following files |final|\textit{nn}|.tex|
compile the final version of the child document
|child|\textit{nn}|.tex|:
%
\begin{center}
\begin{tabular}{l}
|\def\version{final}|\\
|\input{childdoc.def}|\\
|\childdocforwardprefix{final}{child}|
\end{tabular}
\end{center}
%

Note that when several versions of a main file and/or of each child file
are to be generated, it may be convenient to set up a |Makefile| or
shell script to automatise the process.

%%%%%%%%%%%%%%%%%%%%%%%%%%%%%%%%%%%%%%%%%%%%%%%%%%%%%%%%%%%%%%%%%%%%%%%%%%%%%%%%
\subsection{Command Line Processing}
\label{sec:commandline}

The effect of redirection files can also be achieved by invoking
the \LaTeX{} compiler with a more elaborate command line.
Most conveniently this should be done as part
of a shell script or a |Makefile|.

When using \textsf{childdoc} in the main file, the following
command lines effectively perform a redirection
(note that depending on the shell being used,
backslashes may have to be doubled: `|\|' $\to$ `|\\|'):
%
\begin{center}
|... -jobname "|\textit{target}|" |\\|"|[\textit{flags}]%
|\input{childdoc.def}\childdocforward[|\textit{main}|]{|\textit{dest}|}"|
\end{center}
%
Here \textit{target} is the name of the output file,
\textit{main} is the name of the main file
and \textit{dest} is the name of the main or child file to be processed
(all filenames without extensions).
The optional argument \textit{main} can be omitted
if \textit{main} matches \textit{dest}.
Optionally, compilation \textit{flags} can be defined via |\def| commands.
This command line makes the \TeX{} engine believe
it is compiling the file \textit{target}
whose content is specified as the latter parameter.
The provided code then forwards the processing to
\textit{main} or \textit{dest} as described in \secref{sec:forward}.

%%%%%%%%%%%%%%%%%%%%%%%%%%%%%%%%%%%%%%%%%%%%%%%%%%%%%%%%%%%%%%%%%%%%%%%%%%%%%%%%
\subsection{Include by Input}
\label{sec:input}

Including child documents by |\include| has some restrictions by design.
Most notably, the content of a child document always occupies
its own set of pages; pages cannot be shared between child documents.
Usually, this behaviour makes perfect sense
because each child document contain an essential part of the document.
However, in some situations it may be desirable to compose
a document from a collection of parts
without having mandatory page breaks between then.
For this case, the package
provides a mechanism to include parts
by |\input| which can also be processed individually.
However, by construction this mechanism
requires manual handling of the content to be output.

%%%%%%%%%%%%%%%%%%%%%%%%%%%%%%%%%%%%%%%%
\DescribeMacro{\ifchilddocmanual}
The main file should be prepared as usual, see \secref{sec:include}.
However, the document body must make a distinction
between processing of an individual part and of the main document, e.g.:
%
\begin{center}
\begin{tabular}{l}
|\ifchilddocmanual|\\
|\input{\childdocname}|\\
|\||else|\\
\textit{document body with }|\input{|\textit{part}|}|\\
|\||fi|
\end{tabular}
\end{center}
%
The conditional |\ifchilddocmanual| is true whenever
a part to be included by |\input| is being compiled,
and the name of the part is stored in |\childdocname|.

%%%%%%%%%%%%%%%%%%%%%%%%%%%%%%%%%%%%%%%%
\DescribeMacro{\childdocby}
Each part to be included by |\input| should start with:
%
\begin{center}
\begin{tabular}{l}
|\input{childdoc.def}|\\
|\childdocby{|\textit{main}|}|\\
\end{tabular}
\end{center}
%
The directive |\childdocby| is similar to |\childdocof|
described in \secref{sec:include},
but the subsequent selection of content must be done manually.
To that end, both |\ifchilddoc| and |\ifchilddocmanual|
will be true upon processing of a part,
and the name of the part is stored in |\childdocname|.
Note that |\jobname| will be set to the filename of the current part
so that each part receives an individual |.aux| file
that does not interfere with the |.aux| file(s) of the main document.
This behaviour can be altered by the alternative form
|\childdocby[*]{|\textit{main}|}| (with a non-empty optional argument)
which uses the |.aux| file of the main document
by setting |\jobname| to \textit{main}.

%%%%%%%%%%%%%%%%%%%%%%%%%%%%%%%%%%%%%%%%%%%%%%%%%%%%%%%%%%%%%%%%%%%%%%%%%%%%%%%%
\subsection{Driver Development}
\label{sec:driver}

The \textsf{childdoc} mechanism can also be use for the development
of definition files such as \LaTeX{} styles or classes.
This case differs from the above setup with multiple parts
included by |\include| in that no |\includeonly| should be invoked.
This can be achieved by starting the include file
(before |\ProvidesPackage|) with:
%
\begin{center}
\begin{tabular}{l}
|\input{childdoc.def}|\\
|\childdocforward{|\textit{main}|}|\\
\end{tabular}
\end{center}
%
or alternatively with:
%
\begin{center}
\begin{tabular}{l}
|\input{childdoc.def}|\\
|\childdocby{|\textit{main}|}|\\
\end{tabular}
\end{center}
%
Both forms have slightly different effects as described above.
The main file is prepared as usual, see \secref{sec:include}.

%%%%%%%%%%%%%%%%%%%%%%%%%%%%%%%%%%%%%%%%%%%%%%%%%%%%%%%%%%%%%%%%%%%%%%%%%%%%%%%%
\subsection{Legacy Detection}
\label{sec:detection}

The directive |\childdocmain| in the main file can detect
whether the complete document or merely a child is to be compiled
even without using the directive |\childdocof|.
This method is deprecated because it is less robust
and there is no compelling reason to use it;
it is merely provided for backward compatibility
and it may be removed in future versions.

If the detection mechanism is to be used,
it is mandatory to correctly specify
the filename of the main file as the argument of |\childdocmain|:
%
\begin{center}
\begin{tabular}{l}
|\input{childdoc.def}|\\
|\childdocmain{|\textit{main}|}|\\
\end{tabular}
\end{center}
%
If |\jobname| does not match the argument \textit{main} of |\childdocmain|,
it is assumed that |\jobname| points to the child file to be compiled.
When using |\childdocmain| with the main file specified as argument,
it suffices to start a child file
with just |\input{|\textit{main}|}|
without loading of the package and using |\childdocof|.
If instead all processing is done
with the appropriate \textsf{childdoc} directives,
the argument of \textit{main} of |\childdocmain| can be empty.

An alternative version of the command line processing described
in \secref{sec:commandline} using the detection mechanism reads:
%
\begin{center}
|... -jobname "|\textit{target}|" "|[\textit{flags}]%
[|\def\jobname{|\textit{dest}|}|]|\input{|\textit{main}|}"|
\end{center}

%%%%%%%%%%%%%%%%%%%%%%%%%%%%%%%%%%%%%%%%%%%%%%%%%%%%%%%%%%%%%%%%%%%%%%%%%%%%%%%%
\subsection{Manual Code}
\label{sec:manual}

In case one cannot be certain whether the definitions file |childdoc.def|
is installed on the target \TeX{} distribution
and one prefers not to ship it,
it is conceivable to paste a few relevant commands into the sources.

To that end, drop all statements |\input{childdoc.def}|
and perform the replacements as outlined below.
Instead of |\childdocmain{|\textit{main}|}| add the following code
to the top of the main file:
%
\begin{center}
\begin{tabular}{l}
|\||ifdefined\childdocname\endinput\||fi\newif\ifchilddoc|\\
|\edef\childdocname{\scantokens\expandafter{\jobname\noexpand}}|\\
|\def\childdocmain{|\textit{main}|}\||ifx\childdocmain\childdocname\||else|\\
|\childdoctrue\includeonly{\childdocname}\let\jobname\childdocmain\||fi|\\
\end{tabular}
\end{center}
%
Instead of |\childdocof{|\textit{main}|}| just include the main file
at the top of each child file:
%
\begin{center}
|\input{|\textit{main}|}|
\end{center}
%
A simple redirection |\childdocforward{|\textit{dest}|}| is achieved by:
%
\begin{center}
|\def\jobname{|\textit{dest}|}\input{\jobname}|
\end{center}
%
The redirection with prefix
|\childdocforwardprefix[|\textit{prefix}|]{|\textit{dest}|}|
is accomplished by:
%
\begin{center}
\begin{tabular}{l}
|{\edef\jobname{\scantokens\expandafter{\jobname\noexpand}}|\\
|\def\redirectjob |\textit{prefix}|#1~~~{\gdef\jobname{|\textit{dest}|#1}}|\\
|\expandafter\redirectjob\jobname~~~}\input{\jobname}|
\end{tabular}
\end{center}

In an alternative approach,
child documents can be compiled by a specific command line
without additional code or specific definitions:
%
\begin{center}
|... -jobname "|\textit{target}|" "|[\textit{flags}]%
|\includeonly{|\textit{dest}|}\input{|\textit{main}|}"|
\end{center}
%

%%%%%%%%%%%%%%%%%%%%%%%%%%%%%%%%%%%%%%%%%%%%%%%%%%%%%%%%%%%%%%%%%%%%%%%%%%%%%%%%
%%%%%%%%%%%%%%%%%%%%%%%%%%%%%%%%%%%%%%%%%%%%%%%%%%%%%%%%%%%%%%%%%%%%%%%%%%%%%%%%
\section{Information}

%%%%%%%%%%%%%%%%%%%%%%%%%%%%%%%%%%%%%%%%%%%%%%%%%%%%%%%%%%%%%%%%%%%%%%%%%%%%%%%%
\subsection{Copyright}

Copyright \copyright{} 2017--2018 Niklas Beisert

This work may be distributed and/or modified under the
conditions of the \LaTeX{} Project Public License, either version 1.3
of this license or (at your option) any later version.
The latest version of this license is in
  \url{http://www.latex-project.org/lppl.txt}
and version 1.3 or later is part of all distributions of \LaTeX{}
version 2005/12/01 or later.

This work has the LPPL maintenance status `maintained'.

The Current Maintainer of this work is Niklas Beisert.

This work consists of the files |README.txt|, |childdoc.ins| and |childdoc.dtx|
as well as the derived files |childdoc.def|, |cdocsamp.tex|
with |cdocsch1.tex|, |cdocsch2.tex|, |cdocspt3.tex|, |cdocspt4.tex|,
|cdocsdrf.tex|, |cdocsfn1.tex|, |cdocsfn2.tex|
as well as |childdoc.pdf|.

%%%%%%%%%%%%%%%%%%%%%%%%%%%%%%%%%%%%%%%%%%%%%%%%%%%%%%%%%%%%%%%%%%%%%%%%%%%%%%%%
\subsection{Files and Installation}

The package consists of the files:
%
\begin{center}
\begin{tabular}{ll}
    |README.txt|   & readme file \\
    |childdoc.ins| & installation file \\
    |childdoc.dtx| & source file \\
    |childdoc.def| & definition file \\
    |cdocsamp.tex| & sample main file \\
    |cdocsch1.tex| & sample include file \\
    |cdocsch2.tex| & sample include file \\
    |cdocspt3.tex| & sample part file \\
    |cdocspt4.tex| & sample part file \\
    |cdocsdrf.tex| & sample redirection file \\
    |cdocsfn1.tex| & sample redirection file \\
    |cdocsfn2.tex| & sample redirection file \\
    |childdoc.pdf| & manual
\end{tabular}
\end{center}
%
The distribution consists of the files
|README.txt|, |childdoc.ins| and |childdoc.dtx|.
%
\begin{itemize}
\item
Run (pdf)\LaTeX{} on |childdoc.dtx|
to compile the manual |childdoc.pdf| (this file).
\item
Run \LaTeX{} on |childdoc.ins| to create the definitions file |childdoc.def|
and the sample |cdocsamp.tex| with include files
|cdocsch1.tex|, |cdocsch2.tex|, |cdocspt3.tex|, |cdocspt4.tex|,
|cdocsdrf.tex|, |cdocsfn1.tex|, |cdocsfn2.tex|.
Then copy the file |childdoc.def| to an appropriate directory of your \LaTeX{}
distribution, e.g.\ \textit{texmf-root}|/tex/latex/childdoc|.
\end{itemize}

%%%%%%%%%%%%%%%%%%%%%%%%%%%%%%%%%%%%%%%%%%%%%%%%%%%%%%%%%%%%%%%%%%%%%%%%%%%%%%%%
\subsection{Related CTAN Packages}

There are several other packages which offer a similar functionality:
%
\begin{itemize}
\item
The packages
\href{http://ctan.org/pkg/docmute}{\textsf{docmute}},
\href{http://ctan.org/pkg/includex}{\textsf{includex}} and
\href{http://ctan.org/pkg/standalone}{\textsf{standalone}}
provide commands to include only the document body of
a child file thus allowing both files to be compiled individually.
\item
The packages \href{http://ctan.org/pkg/subdocs}{\textsf{subdocs}}
and \href{http://ctan.org/pkg/subfiles}{\textsf{subfiles}}
provide structures in which the main and child documents can be
encapsulated and allowing them to be compiled individually.
The inclusion mechanism is different from the conventional |\include|.
\item
The package \href{http://ctan.org/pkg/combine}{\textsf{combine}}
is an elaborate solution to combine several documents into one.
\end{itemize}
%
See also the CTAN topic \href{http://ctan.org/topic/subdocs}{\textsf{subdocs}}
for further related packages.
The present package differs from the above solutions in that
a document structure constructed with the conventional |\include| mechanism
just needs two extra commands at the top of every file
such that all constituent files can be compiled individually.

%%%%%%%%%%%%%%%%%%%%%%%%%%%%%%%%%%%%%%%%%%%%%%%%%%%%%%%%%%%%%%%%%%%%%%%%%%%%%%%%
%\subsection{Feature Suggestions}
%
%The following is a list of features which may be useful for future
%versions of this package:
%%
%\begin{itemize}
%\item
%\ldots
%\end{itemize}

%%%%%%%%%%%%%%%%%%%%%%%%%%%%%%%%%%%%%%%%%%%%%%%%%%%%%%%%%%%%%%%%%%%%%%%%%%%%%%%%
\subsection{Revision History}

%%%%%%%%%%%%%%%%%%%%%%%%%%%%%%%%%%%%%%%%
\paragraph{v2.0:} 2018/12/30

\begin{itemize}
\item
immediate forward processing
\item
added |\childdocby| mechanism
\item
manual restructured
\end{itemize}

%%%%%%%%%%%%%%%%%%%%%%%%%%%%%%%%%%%%%%%%
\paragraph{v1.6:} 2018/01/17

\begin{itemize}
\item
application for development of include files
\item
corrections to manual
\end{itemize}

%%%%%%%%%%%%%%%%%%%%%%%%%%%%%%%%%%%%%%%%
\paragraph{v1.5:} 2017/05/21

\begin{itemize}
\item
more complete structuring introduced
\item
|\childdocof| introduced
\item
|\childdoc| renamed to |\childdocmain|
\item
|\childredirect| renamed to |\childdocforward| and |\childdocforwardprefix|
and functionality expanded
\end{itemize}

%%%%%%%%%%%%%%%%%%%%%%%%%%%%%%%%%%%%%%%%
\paragraph{v1.0:} 2017/04/27

\begin{itemize}
\item
manual and install package
\item
first version published on CTAN
\end{itemize}

%%%%%%%%%%%%%%%%%%%%%%%%%%%%%%%%%%%%%%%%
\paragraph{v0.6:} 2017/04/26

\begin{itemize}
\item
redirection mechanism added
\end{itemize}

%%%%%%%%%%%%%%%%%%%%%%%%%%%%%%%%%%%%%%%%
\paragraph{v0.5:} 2017/04/26

\begin{itemize}
\item
functionality in definition file
\end{itemize}


%%%%%%%%%%%%%%%%%%%%%%%%%%%%%%%%%%%%%%%%%%%%%%%%%%%%%%%%%%%%%%%%%%%%%%%%%%%%%%%%
%%%%%%%%%%%%%%%%%%%%%%%%%%%%%%%%%%%%%%%%%%%%%%%%%%%%%%%%%%%%%%%%%%%%%%%%%%%%%%%%
%%%%%%%%%%%%%%%%%%%%%%%%%%%%%%%%%%%%%%%%%%%%%%%%%%%%%%%%%%%%%%%%%%%%%%%%%%%%%%%%
\appendix

\settowidth\MacroIndent{\rmfamily\scriptsize 000\ }

 \DocInput{childdoc.dtx}

\end{document}
%</driver>
% \fi
%
% %%%%%%%%%%%%%%%%%%%%%%%%%%%%%%%%%%%%%%%%%%%%%%%%%%%%%%%%%%%%%%%%%%%%%%%%%%%%%%
% %%%%%%%%%%%%%%%%%%%%%%%%%%%%%%%%%%%%%%%%%%%%%%%%%%%%%%%%%%%%%%%%%%%%%%%%%%%%%%
% \section{Sample}
%\iffalse
%<*samplemain>
%\fi
%
% The following presents a sample document
% with two chapters, two parts, a title page,
% a compile flag as well as three forwarding files to set the flag.
% It consists of eight |.tex| files:
% \begin{center}
% \begin{tabular}{ll}
% |cdocsamp.tex|&main file\\
% |cdocsch1.tex|&include file for chapter 1\\
% |cdocsch2.tex|&include file for chapter 2\\
% |cdocspt3.tex|&include file for part 3\\
% |cdocspt4.tex|&include file for part 4\\
% |cdocsdrf.tex|&forwarding file for main file in draft mode\\
% |cdocsfi1.tex|&forwarding file for final version of chapter 1\\
% |cdocsfi2.tex|&forwarding file for final version of chapter 2\\
% \end{tabular}
% \end{center}
% Each of the eight files can be compiled directly by the \LaTeX{} compiler.
%
% %%%%%%%%%%%%%%%%%%%%%%%%%%%%%%%%%%%%%%
% \paragraph{Main File.}
%
% The main file is called |cdocsamp.tex|.
%
% Load the \textsf{childdoc} definitions and
% declare the filename for the main document:
%    \begin{macrocode}
\input{childdoc.def}
\childdocmain{}
%    \end{macrocode}

% Optional override for |\version| flag:
%    \begin{macrocode}
%%\ifchilddoc\else\providecommand{\version}{draft}\fi
%    \end{macrocode}

% Define the default values for the |\version| flag
% (|final| for the main file and |draft| for childs):
%    \begin{macrocode}
\ifchilddoc
\providecommand{\version}{draft}
\else
\providecommand{\version}{final}
\fi
%    \end{macrocode}

% Load the standard document class:
%    \begin{macrocode}
\documentclass[12pt]{article}
%    \end{macrocode}

% Start the document body:
%    \begin{macrocode}
\begin{document}
%    \end{macrocode}

% Declare a title page.
% Print title, part of document being processed and version flag:
%    \begin{macrocode}
\addtocounter{page}{-1}
\begin{center}
{\LARGE\bfseries{}childdoc example\par}
\vspace{1cm}
\ifchilddoc
\ifchilddocmanual part\else chapter\fi:
`\childdocname' of `\childdocjob'\par
\else
main document: `\childdocjob'\par
\fi
version: \version\par
\end{center}
\newpage
%    \end{macrocode}

% Manually include selected file,
% otherwise process as usual:
%    \begin{macrocode}
\ifchilddocmanual
\section*{part `\childdocname'}
\input{\childdocname}
\else
%    \end{macrocode}

% Include the two chapters:
%    \begin{macrocode}
\include{cdocsch1}
\include{cdocsch2}
%    \end{macrocode}

% Include the two parts unless only chapters should be displayed:
%    \begin{macrocode}
\ifchilddoc\else
\section{part three}
\input{cdocspt3}
\section{part four}
\input{cdocspt4}
\fi
%    \end{macrocode}

% Process as usual until here:
%    \begin{macrocode}
\fi
%    \end{macrocode}

% End of document body:
%    \begin{macrocode}
\end{document}
%    \end{macrocode}
%\iffalse
%</samplemain>
%\fi
%
% %%%%%%%%%%%%%%%%%%%%%%%%%%%%%%%%%%%%%%
% \paragraph{Chapter Include Files.}
%
% The include files are called |cdocsch1.tex| and |cdocsch2.tex|.
%
%\iffalse
%<*samplechap1|samplechap2>
%\fi

% Optional override for |\version| flag:
%    \begin{macrocode}
%%\providecommand{\version}{final}
%    \end{macrocode}

% Include the main document:
%    \begin{macrocode}
\input{childdoc.def}
\childdocof{cdocsamp}
%    \end{macrocode}

%\iffalse
%</samplechap1|samplechap2>
%\fi
%
%\iffalse
%<*samplechap1>
%\fi
% Some text for chapter 1:
%    \begin{macrocode}
\section{one}
some text in chapter one
%    \end{macrocode}

%\iffalse
%</samplechap1>
%\fi
% Some text for chapter 2:
%\iffalse
%<*samplechap2>
%\fi
%    \begin{macrocode}
\section{two}
more text in chapter two
%    \end{macrocode}

%\iffalse
%</samplechap2>
%\fi
%
% %%%%%%%%%%%%%%%%%%%%%%%%%%%%%%%%%%%%%%
% \paragraph{Part Include Files.}
%
% The include files are called |cdocspt3.tex| and |cdocspt4.tex|.
%
%\iffalse
%<*samplepart3|samplepart4>
%\fi

% Optional override for |\version| flag:
%    \begin{macrocode}
%%\providecommand{\version}{final}
%    \end{macrocode}

% Include the main document:
%    \begin{macrocode}
\input{childdoc.def}
\childdocby{cdocsamp}
%    \end{macrocode}

%\iffalse
%</samplepart3|samplepart4>
%\fi
%
%\iffalse
%<*samplepart3>
%\fi
% Some text for part 3:
%    \begin{macrocode}
some text in part three
%    \end{macrocode}

%\iffalse
%</samplepart3>
%\fi
% Some text for part 4:
%\iffalse
%<*samplepart4>
%\fi
%    \begin{macrocode}
more text in part four
%    \end{macrocode}

%\iffalse
%</samplepart4>
%\fi
%
% %%%%%%%%%%%%%%%%%%%%%%%%%%%%%%%%%%%%%%
% \paragraph{Forwarding for a Complete Draft.}
%
% The following forwarding file |cdocsdrf.tex|
% compiles the main document in draft mode:
%\iffalse
%<*sampledraft>
%\fi
%    \begin{macrocode}
\def\version{draft}
\input{childdoc.def}
\childdocforward{cdocsamp}
%    \end{macrocode}

%\iffalse
%</sampledraft>
%\fi
%
% %%%%%%%%%%%%%%%%%%%%%%%%%%%%%%%%%%%%%%
% \paragraph{Forwarding for Final Version of the Chapters.}
%
% The following forwarding files |cdocsfn1.tex| and |cdocsfn2.tex|
% (with identical content)
% compile the final versions of the child documents
% |cdocsch1.tex| and |cdocsch2.tex|, respectively:
%\iffalse
%<*samplefinal>
%\fi
%    \begin{macrocode}
\def\version{final}
\input{childdoc.def}
\childdocforwardprefix[cdocsamp]{cdocsfn}{cdocsch}
%    \end{macrocode}

%\iffalse
%</samplefinal>
%\fi
%
% %%%%%%%%%%%%%%%%%%%%%%%%%%%%%%%%%%%%%%
% \paragraph{Command Line Processing.}
%
% The following three command lines generate the output files
% |cdocscld|, |cdocscl1| and |cdocscl2|
% which should be identical to
% |cdocsdrf|, |cdocsch1| and |cdocsfn2|, respectively:
% \begin{center}
% \begin{tabular}{l}
% |latex -jobname cdocscld \|\\
% |  "\def\version{draft}\input{childdoc.def}\childdocforward{cdocsamp}"|\\
% |latex -jobname cdocscl1 \|\\
% |  "\input{childdoc.def}\childdocforward[cdocsamp]{cdocsch1}"|\\
% |latex -jobname cdocscl2 \|\\
% |  "\def\version{final}\input{childdoc.def}\childdocforward{cdocsch2}"|
% \end{tabular}
% \end{center}
% Note that the trailing backslash on each first line
% merely continues the input to the second line
% (for convenient cut ant paste).
% Furthermore, the command |latex| can be replaced by any
% of its alternative versions such as |pdflatex|.
%
% %%%%%%%%%%%%%%%%%%%%%%%%%%%%%%%%%%%%%%%%%%%%%%%%%%%%%%%%%%%%%%%%%%%%%%%%%%%%%%
% %%%%%%%%%%%%%%%%%%%%%%%%%%%%%%%%%%%%%%%%%%%%%%%%%%%%%%%%%%%%%%%%%%%%%%%%%%%%%%
% \section{Implementation}
%\iffalse
%<*package>
%\fi
%
% This section describes the definitions file |childdoc.def|.

% The definitions cannot be loaded using |\usepackage| or |\RequirePackage|
% which has a mechanism to prevent loading a style file more than once.
% When loading the definitions by means of |\input|
% multiple instances have to be prevented manually:
%\iffalse
%This code needs to be before the `\ProvidesFile' directive
%which is defined at the beginning of this file.
%Therefore it is also placed there and commented out here.
%</package>
%<*discard>
%\fi
%    \begin{macrocode}
\ifdefined\childdocmain\endinput\fi
%    \end{macrocode}
%\iffalse
%</discard>
%<*package>
%\fi
%
% \macro{\ifchilddoc}
% \macro{\ifchilddocmanual}
% The conditional |\ifchilddoc| tells whether a
% child (true) or main (false) document is being compiled.
% The conditional |\ifchilddocmanual| tells whether
% the |\includeonly| mechanism is used (false) or
% the selection of child files must be performed manually (true).
% The definitions initialise to false:
%    \begin{macrocode}
\newif\ifchilddoc
\newif\ifchilddocmanual
%    \end{macrocode}

% \macro{\childdocname}
% \macro{\childdocjob}
% The macro |\childdocname| stores the name of the main document
% to be compiled. The macro |\childdocjob| stores the name of
% the document on which the \LaTeX{} compiler was originally invoked.
% The content of |\jobname| cannot be compared
% to filenames specified in the source due to different catcodes.
% The following code rescans |\jobname|, stores the result
% in |\childdocname| and saves a copy in |\childdocjob|:
%    \begin{macrocode}
\edef\childdocname{\scantokens\expandafter{\jobname\noexpand}}
\let\childdocjob\childdocname
%    \end{macrocode}

% \macro{\childdocdisable}
% The macro |\childdocdisable| prevents the main file
% from being processed more than once.
% At this stage, the main document command |\childdocmain|
% is assumed to be called once again where it should do nothing.
% Any subsequent call to it should prevent
% a secondary processing of the main document
% It overwrites the forwarding commands
% |\childdocof| and |\childdocforward|
% with empty macros to prevent further inclusions of the main document:
%    \begin{macrocode}
\newcommand{\childdocdisable}
{
  \renewcommand{\childdocmain}[1]{\renewcommand{\childdocmain}[1]{\endinput}}
  \renewcommand{\childdocof}[1]{}
  \renewcommand{\childdocby}[2][]{}
  \renewcommand{\childdocforward}[2][]{}
  \renewcommand{\childdocdisable}{}
}
%    \end{macrocode}

% \macro{\childdocmain}
% The macro |\childdocmain| is to be called at the top of the main file
% with nothing or the main filename (without extension) as argument.
% First, it breaks loops.
% If the argument is not empty and does not match |\childdocname|
% (which is set by the first inclusion of |childdoc.def|),
% |\ifchilddoc| is set to true, |\includeonly| is applied to the child file
% and |\jobname| is set to the main file
% (for proper handling of |.aux| files):
%    \begin{macrocode}
\newcommand{\childdocmain}[1]
{
  \childdocdisable\childdocmain{}
  \if?#1?\else
    \begingroup
      \def\childdoctmp{#1}
      \ifx\childdoctmp\childdocname
        \def\childdoctmp{}
      \else
        \def\childdoctmp
        {
          \childdoctrue
          \includeonly{\childdocname}
          \def\childdocjob{#1}
          \def\jobname{#1}
        }
      \fi
      \expandafter
    \endgroup
    \childdoctmp
  \fi
}
%    \end{macrocode}

% \macro{\childdocof}
% The command |\childdocof| redirects
% compilation to the main file |#1|.
%    \begin{macrocode}
\newcommand{\childdocof}[1]
{
  \childdocdisable
  \childdoctrue
  \includeonly{\childdocname}
  \def\jobname{#1}
  \def\childdocjob{#1}
  \input{#1}
}
%    \end{macrocode}

% \macro{\childdocby}
% The command |\childdocby| ....
%    \begin{macrocode}
\newcommand{\childdocby}[2][]
{
  \childdocdisable
  \childdoctrue
  \childdocmanualtrue
  \if?#1?\else
    \def\jobname{#2}
  \fi
  \def\childdocjob{#2}
  \input{#2}
  \endinput
}
%    \end{macrocode}

% \macro{\childdocforward}
% The command |\childdocforward| redirects
% compilation to the main file or
% (if the optional argument is given) a child file.
% Parameters are set as if the main file
% or a child file starting with |\childdocof| was compiled.
% Then compilation is handed over to the main file:
%    \begin{macrocode}
\newcommand{\childdocforward}[2][]
{
  \begingroup
    \if?#1?
      \def\childdoctmp
      {
        \def\childdocname{#2}
        \def\childdocjob{#2}
        \def\jobname{#2}
        \input{#2}
        \endinput
      }
    \else
      \def\childdoctmp
      {
        \childdocdisable
        \def\childdocname{#2}
        \childdoctrue
        \includeonly{#2}
        \def\childdocjob{#1}
        \def\jobname{#1}
        \input{#1}
        \endinput
      }
    \fi
    \expandafter
  \endgroup
  \childdoctmp
}
%    \end{macrocode}

% \macro{\childdocforwardprefix}
% The command |\childdocforwardprefix| redirects
% compilation to the main or a child file by means of a pattern.
% The prefix |#1| in the current filename is replaced by |#2|
% and the suffix of the current filename is kept
% (it is assumed that the filename does not contain the substring `|~~~|'
% which is used as a delimiter).
% Compilation is handed over to the new file by |\childdocforward|:
%    \begin{macrocode}
\newcommand{\childdocforwardprefix}[3][]
{
  \begingroup
    \def\childdocextract #2##1~~~{\def\childdoctmp{\childdocforward[#1]{#3##1}}}
    \expandafter\childdocextract\childdocname~~~
    \expandafter
  \endgroup
  \childdoctmp
}
%    \end{macrocode}

% \macro{\childdoc}
% The deprecated macro |\childdoc| is a legacy version of |\childdocmain|:
%    \begin{macrocode}
\newcommand{\childdoc}{\childdocmain}
%    \end{macrocode}

% \macro{\childdocredirect}
% The deprecated macro |\childdocredirect| is a legacy version
% of |\childdocforward| and |\childdocforwardprefix|:
%    \begin{macrocode}
\newcommand{\childdocredirect}[2][]
{
  \begingroup
    \if?#1?
      \def\childdoctmp{\childdocforward{#2}}
    \else
      \def\childdoctmp{\childdocforwardprefix{#1}{#2}}
    \fi
    \expandafter
  \endgroup
  \childdoctmp
}
%    \end{macrocode}

%\iffalse
%</package>
%\fi
%
\endinput
|\\
|\childdocby{|\textit{main}|}|\\
\end{tabular}
\end{center}
%
The directive |\childdocby| is similar to |\childdocof|
described in \secref{sec:include},
but the subsequent selection of content must be done manually.
To that end, both |\ifchilddoc| and |\ifchilddocmanual|
will be true upon processing of a part,
and the name of the part is stored in |\childdocname|.
Note that |\jobname| will be set to the filename of the current part
so that each part receives an individual |.aux| file
that does not interfere with the |.aux| file(s) of the main document.
This behaviour can be altered by the alternative form
|\childdocby[*]{|\textit{main}|}| (with a non-empty optional argument)
which uses the |.aux| file of the main document
by setting |\jobname| to \textit{main}.

%%%%%%%%%%%%%%%%%%%%%%%%%%%%%%%%%%%%%%%%%%%%%%%%%%%%%%%%%%%%%%%%%%%%%%%%%%%%%%%%
\subsection{Driver Development}
\label{sec:driver}

The \textsf{childdoc} mechanism can also be use for the development
of definition files such as \LaTeX{} styles or classes.
This case differs from the above setup with multiple parts
included by |\include| in that no |\includeonly| should be invoked.
This can be achieved by starting the include file
(before |\ProvidesPackage|) with:
%
\begin{center}
\begin{tabular}{l}
|% \iffalse
%
% childdoc.dtx Copyright (C) 2017-2018 Niklas Beisert
%
% This work may be distributed and/or modified under the
% conditions of the LaTeX Project Public License, either version 1.3
% of this license or (at your option) any later version.
% The latest version of this license is in
%   http://www.latex-project.org/lppl.txt
% and version 1.3 or later is part of all distributions of LaTeX
% version 2005/12/01 or later.
%
% This work has the LPPL maintenance status `maintained'.
%
% The Current Maintainer of this work is Niklas Beisert.
%
% This work consists of the files childdoc.dtx and childdoc.ins
% and the derived files childdoc.def and cdocsamp.tex with
% cdocsch1.tex, cdocsch2.tex, cdocsdrf.tex, cdocsfn1.tex, cdocsfn2.tex.
%
%<package>\ifdefined\childdocmain\endinput\fi
%<package>\ProvidesFile{childdoc.def}[2018/12/30 v2.0 child document driver]
%<samplemain>\ProvidesFile{cdocsamp.tex}[2018/12/30 v2.0 sample for childdoc]
%<*driver>
%\ProvidesFile{childdoc.drv}[2018/12/30 v2.0 childdoc reference manual file]
\PassOptionsToClass{10pt,a4paper}{article}
\documentclass{ltxdoc}

\usepackage[margin=35mm]{geometry}
\usepackage{hyperref}
\usepackage{hyperxmp}
\usepackage[usenames]{color}

\hypersetup{colorlinks=true}
\hypersetup{pdfstartview=FitH}
\hypersetup{pdfpagemode=UseNone}
\hypersetup{pdfsource={}}
\hypersetup{pdflang={en-UK}}
\hypersetup{pdfcopyright={Copyright 2017-2018 Niklas Beisert.
  This work may be distributed and/or modified under the
  conditions of the LaTeX Project Public License, either version 1.3
  of this license or (at your option) any later version.}}
\hypersetup{pdflicenseurl={http://www.latex-project.org/lppl.txt}}
\hypersetup{pdfcontactaddress={ETH Zurich, ITP, HIT K,
  Wolfgang-Pauli-Strasse 27}}
\hypersetup{pdfcontactpostcode={8093}}
\hypersetup{pdfcontactcity={Zurich}}
\hypersetup{pdfcontactcountry={Switzerland}}
\hypersetup{pdfcontactemail={nbeisert@itp.phys.ethz.ch}}
\hypersetup{pdfcontacturl={http://people.phys.ethz.ch/\xmptilde nbeisert/}}

\newcommand{\secref}[1]{\hyperref[#1]{section \ref*{#1}}}

\parskip1ex
\parindent0pt
\let\olditemize\itemize
\def\itemize{\olditemize\parskip0pt}

\begin{document}

\title{The \textsf{childdoc} Package}
\hypersetup{pdftitle={The childdoc Package}}
\author{Niklas Beisert\\[2ex]
  Institut f\"ur Theoretische Physik\\
  Eidgen\"ossische Technische Hochschule Z\"urich\\
  Wolfgang-Pauli-Strasse 27, 8093 Z\"urich, Switzerland\\[1ex]
  \href{mailto:nbeisert@itp.phys.ethz.ch}
  {\texttt{nbeisert@itp.phys.ethz.ch}}}
\hypersetup{pdfauthor={Niklas Beisert}}
\hypersetup{pdfsubject={Manual for the LaTeX2e Package childdoc}}
\date{30 December 2018, \textsf{v2.0}}
\maketitle

\begin{abstract}\noindent
\textsf{childdoc} is a \LaTeXe{} package
that enables the direct compilation
of document sections included by |\include|
to individual files.
\end{abstract}

\begingroup
\parskip0ex
\tableofcontents
\endgroup

%%%%%%%%%%%%%%%%%%%%%%%%%%%%%%%%%%%%%%%%%%%%%%%%%%%%%%%%%%%%%%%%%%%%%%%%%%%%%%%%
%%%%%%%%%%%%%%%%%%%%%%%%%%%%%%%%%%%%%%%%%%%%%%%%%%%%%%%%%%%%%%%%%%%%%%%%%%%%%%%%
\section{Introduction}

\LaTeX{} provides a mechanism to structure a large document (such as a book)
into a main file and several child files (containing the chapters)
using the |\include| command.
This mechanism is beneficial for documents
which span hundreds of pages in order to
make the source file(s) more manageable.
Moreover, compilation can be restricted to
selected child files by means of the |\includeonly| command.
The latter feature can be used to reduce the compilation time while editing
(this was significantly more useful in the earlier days of \LaTeX{})
or to generate a smaller document which is easier to navigate.
Another application of |\includeonly| is to generate
documents consisting of selected parts of the complete document.

However, there are a few drawbacks of the plain |\include| mechanism:
\begin{itemize}
\item
The child files cannot be compiled on their own,
they can only be compiled via the main file.
A naive editing environment
(such as a text editor with an option
to have the current file processed by \LaTeX)
may require one to switch to the main file before compiling;
attempting to compile the child file produces errors.
\item
The main file must be modified (each time)
to adjust the |\includeonly| command
to the present needs. This easily leaves the main file in a messy state.
\item
The generated document will always carry the filename
of the main document. This is inconvenient if
several child files are to be compiled and
to be kept for distribution.
\end{itemize}

The present package provides a simple interface
to make child files individually compilable by \LaTeX{}.
Compiling a child file then has the same effect as compiling
the main file with an |\includeonly| command
to select the appropriate child.
Moreover the generated document will carry the name of the child
rather than the main file.
This resolves all three above issues.

This feature is meant to make the editing of books,
thesis documents and lecture notes somewhat more convenient.
However, the package can also be used efficiently for
composing a series of documents (such as exercise sheets)
which are typically distributed individually.
It then assists the author in generating the individual documents
(potentially in different versions)
as well as a document containing the collected series.
Another application is in developing style files
or other kinds of included material
where compilation of the style file could redirect
to a sample or test file.

%%%%%%%%%%%%%%%%%%%%%%%%%%%%%%%%%%%%%%%%%%%%%%%%%%%%%%%%%%%%%%%%%%%%%%%%%%%%%%%%
%%%%%%%%%%%%%%%%%%%%%%%%%%%%%%%%%%%%%%%%%%%%%%%%%%%%%%%%%%%%%%%%%%%%%%%%%%%%%%%%
\section{Usage}

First of all, the package \textsf{childdoc} is \emph{not} a standard
\LaTeXe{} |.sty| style file! Therefore it needs to be invoked in
a non-standard way.

%%%%%%%%%%%%%%%%%%%%%%%%%%%%%%%%%%%%%%%%%%%%%%%%%%%%%%%%%%%%%%%%%%%%%%%%%%%%%%%%
\subsection{Included Files}
\label{sec:include}

%%%%%%%%%%%%%%%%%%%%%%%%%%%%%%%%%%%%%%%%
\DescribeMacro{\childdocmain}
To use the package, add the commands
\begin{center}
\begin{tabular}{l}
|\input{childdoc.def}|\\
|\childdocmain{}|\\
\end{tabular}
\end{center}
at the very top of the main \LaTeX{} file,
in particular \emph{before} the |\documentclass| statement!
The argument of |\childdocmain| should be left empty
(but it must be present).

%%%%%%%%%%%%%%%%%%%%%%%%%%%%%%%%%%%%%%%%
\DescribeMacro{\childdocof}
Furthermore, add the commands
\begin{center}
\begin{tabular}{l}
|\input{childdoc.def}|\\
|\childdocof{|\textit{main}|}|\\
\end{tabular}
\end{center}
at the top of every child file \textit{child}
which is included by |\include{|\textit{child}|}|
from within the main file
(or at least for those files to be compiled individually).
The argument \textit{main} must be the filename of the main file.

There are a couple of
considerations in setting up the main and child documents:

%%%%%%%%%%%%%%%%%%%%%%%%%%%%%%%%%%%%%%%%
\paragraph{Restrictions.}

Please note the following restrictions:
\begin{itemize}
\item
|\childdocmain| must be called with one argument \textit{main}
to ensure compatibility with earlier version of the package.
It must either be empty (|\childdocmain{}|)
or precisely match the filename of the main file in which it is specified.
See \secref{sec:detection} for further information.
\item
The filename \textit{main} must be specified without the |.tex| extension.
\item
The filename \textit{main} is case sensitive
(even in case-insensitive file systems)
due to internal string comparison.
\item
The argument \textit{main} should be fully expanded, it cannot be a macro.
\item
Subdirectories and special characters should be avoided in filenames.
\item
The command |\childdocmain{|\textit{main}|}| must be followed by a whitespace.
It should not be followed immediately by another command
or by a comment mark `|%|'.
This is because the \TeX{} parser reads the token immediately following
the argument of |\childdocmain| and puts it
at the beginning of every child section;
however, a white\-space is ignored.
\end{itemize}

%%%%%%%%%%%%%%%%%%%%%%%%%%%%%%%%%%%%%%%%
\paragraph{Content of Main File.}

It is advisable to place all content in the child files included by |\include|.
Any output contained in the main file will appear in all child documents
unless suppressed manually;
it cannot be suppressed automatically by the |\includeonly| directive
and thus should normally be avoided.
A method to include some content in the main file
by means of conditional processing is described in \secref{sec:conditional}.

%%%%%%%%%%%%%%%%%%%%%%%%%%%%%%%%%%%%%%%%
\paragraph{Page Numbering.}

When only a part of the document is compiled,
the appropriate numbering of pages
(as well as other status parameters)
is determined from the |.aux| files.
The latter contain information from previous passes.
However this information needs to propagate through
all intermediate child documents.
Therefore the page numbering in child documents may well
be inconsistent until the complete document is compiled at least once.

A useful (if unconventional) way to always ensure a consistent
page numbering is to restart the numbering in each child document
and denote the pages by `\textit{child}|.|\textit{page}'
where \textit{child} represents the chapter/section number of the child file.
This can be achieved by the command
|\numberwithin{page}{|\textit{child}|}|
of the \textsf{amsmath} package
where \textit{child} can be |chapter| or |section|
depending on the chosen structuring.
Alternatively, one can modify the macro |\thepage| appropriately
and reset the counter |page| at the start of each child file.

%%%%%%%%%%%%%%%%%%%%%%%%%%%%%%%%%%%%%%%%%%%%%%%%%%%%%%%%%%%%%%%%%%%%%%%%%%%%%%%%
\subsection{Conditional Processing}
\label{sec:conditional}

The package provides a mechanism to compile different versions
of a document. To customise the versions further some conditional processing
can come in handy to distinguish which version is being compiled.
The package provides two macros to describe the compilation context:

%%%%%%%%%%%%%%%%%%%%%%%%%%%%%%%%%%%%%%%%
\DescribeMacro{\ifchilddoc}
The conditional |\ifchilddoc| distinguishes between the compilation of
child documents and the main document:
%
\begin{center}
|\ifchilddoc |\textit{child-code}| |[|\||else |\textit{main-code}]| \||fi|
\end{center}

%%%%%%%%%%%%%%%%%%%%%%%%%%%%%%%%%%%%%%%%
\DescribeMacro{\childdocname}
\DescribeMacro{\childdocjob}
The macro |\childdocname| contains the filename (without extension)
of the main or child file being processed.
Note that |\childdocjob| will always contain the name of the main file.

%%%%%%%%%%%%%%%%%%%%%%%%%%%%%%%%%%%%%%%%
\paragraph{Title Page.}

Conditional processing can be used to include a title or banner page
in the main document when proper precautions are taken.
Importantly, the code in the main file should ensure that the page counter
(as well as other status parameters which are stored in the |.aux| files)
takes the same value after the conditional processing.
Otherwise the page numbers may take divergent values
depending on which part is compiled.

For example, a title page could be declared by:
%
\begin{center}
\begin{tabular}{l}
|\ifchilddoc\||else|\\
|\addtocounter{page}{-1}|\\
\textit{code for title page}\\
|\newpage|\\
|\||fi|
\end{tabular}
\end{center}
%
A banner page for the child documents can be generated by:
%
\begin{center}
\begin{tabular}{l}
|\ifchilddoc|\\
|\addtocounter{page}{-1}|\\
\textit{code for banner page}\\
|\newpage|\\
|\||fi|
\end{tabular}
\end{center}
%
Here one could write a message such as:
\begin{center}
|This is the part \childdocname{} of \childdocjob{}.|
\end{center}

%%%%%%%%%%%%%%%%%%%%%%%%%%%%%%%%%%%%%%%%%%%%%%%%%%%%%%%%%%%%%%%%%%%%%%%%%%%%%%%%
\subsection{Flags}
\label{sec:flags}

The package makes it easy to generate different versions
of the main or child documents.
To this end compilation flags can be defined
and assigned different default values.
They will be particularly useful in conjunction
with the forwarding mechanism described in \secref{sec:forward}.

For example, it may be useful to have a flag |\version|
which can be set to |draft| or |final|.
The document source will contain some conditional code
depending on the value of |\version|.
Suppose further, the flag should default to |final| for the main file
and to |draft| for child files
which is a natural assignment for editing the document.
This is achieved by placing the following code
in the preamble of the main document
(below the |\childdocmain| directive):
%
\begin{center}
\begin{tabular}{l}
|\ifchilddoc|\\
|\providecommand{\version}{draft}|\\
|\||else|\\
|\providecommand{\version}{final}|\\
|\||fi|
\end{tabular}
\end{center}
%
The definition by |\providecommand| makes sure
that previous definitions are not overwritten.
Further statements |\providecommand{\version}{...}|
can thus be added before the above code to override it.

For the main file, one might add a line
(between |\childdocmain| and the above block)
%
\begin{center}
|%\ifchilddoc\||else\providecommand{\version}{draft}\||fi|
\end{center}
%
which can be uncommented to produce a draft version.
Likewise one can add a line to the very top of a child file
(above the |\childdocof{|\textit{main}|}| directive)
%
\begin{center}
|%\providecommand{\version}{final}|
\end{center}
%
which can be uncommented to produce the final version of this child document.

%%%%%%%%%%%%%%%%%%%%%%%%%%%%%%%%%%%%%%%%%%%%%%%%%%%%%%%%%%%%%%%%%%%%%%%%%%%%%%%%
\subsection{Forwarding}
\label{sec:forward}

Different versions of the main or child documents
using compilation flags as described in \secref{sec:flags}
can be (permanently) stored in different files
for convenient compilation, viewing and distribution.
To this end, the package defines a command
to pass on compilation to a different file:

%%%%%%%%%%%%%%%%%%%%%%%%%%%%%%%%%%%%%%%%
\DescribeMacro{\childdocforward}
The command |\childdocforward| redirects processing to
another source file:
%
\begin{center}
\begin{tabular}{l}
|\input{childdoc.def}|\\
|\childdocforward[|\textit{main}|]{|\textit{dest}|}|\\
\end{tabular}
\end{center}
%
The argument \textit{dest} is the destination file
(without extension).
It should be the main file or one of the child files.
Note that further \textsf{childdoc} directives
such as |\childdocof| and |\childdocforward|
in the indicated file will be processed in this form.
The optional argument \textit{main}
passes on directly to the main file \textit{main}
while pretending to compile the child \textit{dest}.
This form behaves as if \textit{dest}
issues |\childdocof{|\textit{main}|}| right away,
and no further \textsf{childdoc} directives will be processed.

%%%%%%%%%%%%%%%%%%%%%%%%%%%%%%%%%%%%%%%%
\DescribeMacro{\...prefix}
In the alternative form |\childdocforwardprefix|,
%
\begin{center}
\begin{tabular}{l}
|\input{childdoc.def}|\\
|\childdocforwardprefix[|\textit{main}|]{|\textit{prefix}|}{|\textit{dest}|}|
\end{tabular}
\end{center}
%
the destination file is determined by a pattern
depending on the current file:
To make this work, the current file must be called
`{\textit{prefix}\hspace{0.2em}\textit{suffix}}'
with \textit{prefix} matching precisely the argument.
Processing is then passed on to the file
`{\textit{dest}\hspace{0.2em}\textit{suffix}}'.
Surely, the same effect is achieved by
directly specifying the
argument `{\textit{dest}\hspace{0.2em}\textit{suffix}}'
in the first form.
However, that requires to set up a different file
for each child. With the alternative form of the command
all these files can have exactly the same content
which simplifies setting them up and maintaining them.

For example, the following file |draft.tex|
with a compilation flag |\version| as described in \secref{sec:flags}
compiles the main document as a draft:
%
\begin{center}
\begin{tabular}{l}
|\def\version{draft}|\\
|\input{childdoc.def}|\\
|\childdocforward{|\textit{main}|}|
\end{tabular}
\end{center}
%
Likewise, the following files |final|\textit{nn}|.tex|
compile the final version of the child document
|child|\textit{nn}|.tex|:
%
\begin{center}
\begin{tabular}{l}
|\def\version{final}|\\
|\input{childdoc.def}|\\
|\childdocforwardprefix{final}{child}|
\end{tabular}
\end{center}
%

Note that when several versions of a main file and/or of each child file
are to be generated, it may be convenient to set up a |Makefile| or
shell script to automatise the process.

%%%%%%%%%%%%%%%%%%%%%%%%%%%%%%%%%%%%%%%%%%%%%%%%%%%%%%%%%%%%%%%%%%%%%%%%%%%%%%%%
\subsection{Command Line Processing}
\label{sec:commandline}

The effect of redirection files can also be achieved by invoking
the \LaTeX{} compiler with a more elaborate command line.
Most conveniently this should be done as part
of a shell script or a |Makefile|.

When using \textsf{childdoc} in the main file, the following
command lines effectively perform a redirection
(note that depending on the shell being used,
backslashes may have to be doubled: `|\|' $\to$ `|\\|'):
%
\begin{center}
|... -jobname "|\textit{target}|" |\\|"|[\textit{flags}]%
|\input{childdoc.def}\childdocforward[|\textit{main}|]{|\textit{dest}|}"|
\end{center}
%
Here \textit{target} is the name of the output file,
\textit{main} is the name of the main file
and \textit{dest} is the name of the main or child file to be processed
(all filenames without extensions).
The optional argument \textit{main} can be omitted
if \textit{main} matches \textit{dest}.
Optionally, compilation \textit{flags} can be defined via |\def| commands.
This command line makes the \TeX{} engine believe
it is compiling the file \textit{target}
whose content is specified as the latter parameter.
The provided code then forwards the processing to
\textit{main} or \textit{dest} as described in \secref{sec:forward}.

%%%%%%%%%%%%%%%%%%%%%%%%%%%%%%%%%%%%%%%%%%%%%%%%%%%%%%%%%%%%%%%%%%%%%%%%%%%%%%%%
\subsection{Include by Input}
\label{sec:input}

Including child documents by |\include| has some restrictions by design.
Most notably, the content of a child document always occupies
its own set of pages; pages cannot be shared between child documents.
Usually, this behaviour makes perfect sense
because each child document contain an essential part of the document.
However, in some situations it may be desirable to compose
a document from a collection of parts
without having mandatory page breaks between then.
For this case, the package
provides a mechanism to include parts
by |\input| which can also be processed individually.
However, by construction this mechanism
requires manual handling of the content to be output.

%%%%%%%%%%%%%%%%%%%%%%%%%%%%%%%%%%%%%%%%
\DescribeMacro{\ifchilddocmanual}
The main file should be prepared as usual, see \secref{sec:include}.
However, the document body must make a distinction
between processing of an individual part and of the main document, e.g.:
%
\begin{center}
\begin{tabular}{l}
|\ifchilddocmanual|\\
|\input{\childdocname}|\\
|\||else|\\
\textit{document body with }|\input{|\textit{part}|}|\\
|\||fi|
\end{tabular}
\end{center}
%
The conditional |\ifchilddocmanual| is true whenever
a part to be included by |\input| is being compiled,
and the name of the part is stored in |\childdocname|.

%%%%%%%%%%%%%%%%%%%%%%%%%%%%%%%%%%%%%%%%
\DescribeMacro{\childdocby}
Each part to be included by |\input| should start with:
%
\begin{center}
\begin{tabular}{l}
|\input{childdoc.def}|\\
|\childdocby{|\textit{main}|}|\\
\end{tabular}
\end{center}
%
The directive |\childdocby| is similar to |\childdocof|
described in \secref{sec:include},
but the subsequent selection of content must be done manually.
To that end, both |\ifchilddoc| and |\ifchilddocmanual|
will be true upon processing of a part,
and the name of the part is stored in |\childdocname|.
Note that |\jobname| will be set to the filename of the current part
so that each part receives an individual |.aux| file
that does not interfere with the |.aux| file(s) of the main document.
This behaviour can be altered by the alternative form
|\childdocby[*]{|\textit{main}|}| (with a non-empty optional argument)
which uses the |.aux| file of the main document
by setting |\jobname| to \textit{main}.

%%%%%%%%%%%%%%%%%%%%%%%%%%%%%%%%%%%%%%%%%%%%%%%%%%%%%%%%%%%%%%%%%%%%%%%%%%%%%%%%
\subsection{Driver Development}
\label{sec:driver}

The \textsf{childdoc} mechanism can also be use for the development
of definition files such as \LaTeX{} styles or classes.
This case differs from the above setup with multiple parts
included by |\include| in that no |\includeonly| should be invoked.
This can be achieved by starting the include file
(before |\ProvidesPackage|) with:
%
\begin{center}
\begin{tabular}{l}
|\input{childdoc.def}|\\
|\childdocforward{|\textit{main}|}|\\
\end{tabular}
\end{center}
%
or alternatively with:
%
\begin{center}
\begin{tabular}{l}
|\input{childdoc.def}|\\
|\childdocby{|\textit{main}|}|\\
\end{tabular}
\end{center}
%
Both forms have slightly different effects as described above.
The main file is prepared as usual, see \secref{sec:include}.

%%%%%%%%%%%%%%%%%%%%%%%%%%%%%%%%%%%%%%%%%%%%%%%%%%%%%%%%%%%%%%%%%%%%%%%%%%%%%%%%
\subsection{Legacy Detection}
\label{sec:detection}

The directive |\childdocmain| in the main file can detect
whether the complete document or merely a child is to be compiled
even without using the directive |\childdocof|.
This method is deprecated because it is less robust
and there is no compelling reason to use it;
it is merely provided for backward compatibility
and it may be removed in future versions.

If the detection mechanism is to be used,
it is mandatory to correctly specify
the filename of the main file as the argument of |\childdocmain|:
%
\begin{center}
\begin{tabular}{l}
|\input{childdoc.def}|\\
|\childdocmain{|\textit{main}|}|\\
\end{tabular}
\end{center}
%
If |\jobname| does not match the argument \textit{main} of |\childdocmain|,
it is assumed that |\jobname| points to the child file to be compiled.
When using |\childdocmain| with the main file specified as argument,
it suffices to start a child file
with just |\input{|\textit{main}|}|
without loading of the package and using |\childdocof|.
If instead all processing is done
with the appropriate \textsf{childdoc} directives,
the argument of \textit{main} of |\childdocmain| can be empty.

An alternative version of the command line processing described
in \secref{sec:commandline} using the detection mechanism reads:
%
\begin{center}
|... -jobname "|\textit{target}|" "|[\textit{flags}]%
[|\def\jobname{|\textit{dest}|}|]|\input{|\textit{main}|}"|
\end{center}

%%%%%%%%%%%%%%%%%%%%%%%%%%%%%%%%%%%%%%%%%%%%%%%%%%%%%%%%%%%%%%%%%%%%%%%%%%%%%%%%
\subsection{Manual Code}
\label{sec:manual}

In case one cannot be certain whether the definitions file |childdoc.def|
is installed on the target \TeX{} distribution
and one prefers not to ship it,
it is conceivable to paste a few relevant commands into the sources.

To that end, drop all statements |\input{childdoc.def}|
and perform the replacements as outlined below.
Instead of |\childdocmain{|\textit{main}|}| add the following code
to the top of the main file:
%
\begin{center}
\begin{tabular}{l}
|\||ifdefined\childdocname\endinput\||fi\newif\ifchilddoc|\\
|\edef\childdocname{\scantokens\expandafter{\jobname\noexpand}}|\\
|\def\childdocmain{|\textit{main}|}\||ifx\childdocmain\childdocname\||else|\\
|\childdoctrue\includeonly{\childdocname}\let\jobname\childdocmain\||fi|\\
\end{tabular}
\end{center}
%
Instead of |\childdocof{|\textit{main}|}| just include the main file
at the top of each child file:
%
\begin{center}
|\input{|\textit{main}|}|
\end{center}
%
A simple redirection |\childdocforward{|\textit{dest}|}| is achieved by:
%
\begin{center}
|\def\jobname{|\textit{dest}|}\input{\jobname}|
\end{center}
%
The redirection with prefix
|\childdocforwardprefix[|\textit{prefix}|]{|\textit{dest}|}|
is accomplished by:
%
\begin{center}
\begin{tabular}{l}
|{\edef\jobname{\scantokens\expandafter{\jobname\noexpand}}|\\
|\def\redirectjob |\textit{prefix}|#1~~~{\gdef\jobname{|\textit{dest}|#1}}|\\
|\expandafter\redirectjob\jobname~~~}\input{\jobname}|
\end{tabular}
\end{center}

In an alternative approach,
child documents can be compiled by a specific command line
without additional code or specific definitions:
%
\begin{center}
|... -jobname "|\textit{target}|" "|[\textit{flags}]%
|\includeonly{|\textit{dest}|}\input{|\textit{main}|}"|
\end{center}
%

%%%%%%%%%%%%%%%%%%%%%%%%%%%%%%%%%%%%%%%%%%%%%%%%%%%%%%%%%%%%%%%%%%%%%%%%%%%%%%%%
%%%%%%%%%%%%%%%%%%%%%%%%%%%%%%%%%%%%%%%%%%%%%%%%%%%%%%%%%%%%%%%%%%%%%%%%%%%%%%%%
\section{Information}

%%%%%%%%%%%%%%%%%%%%%%%%%%%%%%%%%%%%%%%%%%%%%%%%%%%%%%%%%%%%%%%%%%%%%%%%%%%%%%%%
\subsection{Copyright}

Copyright \copyright{} 2017--2018 Niklas Beisert

This work may be distributed and/or modified under the
conditions of the \LaTeX{} Project Public License, either version 1.3
of this license or (at your option) any later version.
The latest version of this license is in
  \url{http://www.latex-project.org/lppl.txt}
and version 1.3 or later is part of all distributions of \LaTeX{}
version 2005/12/01 or later.

This work has the LPPL maintenance status `maintained'.

The Current Maintainer of this work is Niklas Beisert.

This work consists of the files |README.txt|, |childdoc.ins| and |childdoc.dtx|
as well as the derived files |childdoc.def|, |cdocsamp.tex|
with |cdocsch1.tex|, |cdocsch2.tex|, |cdocspt3.tex|, |cdocspt4.tex|,
|cdocsdrf.tex|, |cdocsfn1.tex|, |cdocsfn2.tex|
as well as |childdoc.pdf|.

%%%%%%%%%%%%%%%%%%%%%%%%%%%%%%%%%%%%%%%%%%%%%%%%%%%%%%%%%%%%%%%%%%%%%%%%%%%%%%%%
\subsection{Files and Installation}

The package consists of the files:
%
\begin{center}
\begin{tabular}{ll}
    |README.txt|   & readme file \\
    |childdoc.ins| & installation file \\
    |childdoc.dtx| & source file \\
    |childdoc.def| & definition file \\
    |cdocsamp.tex| & sample main file \\
    |cdocsch1.tex| & sample include file \\
    |cdocsch2.tex| & sample include file \\
    |cdocspt3.tex| & sample part file \\
    |cdocspt4.tex| & sample part file \\
    |cdocsdrf.tex| & sample redirection file \\
    |cdocsfn1.tex| & sample redirection file \\
    |cdocsfn2.tex| & sample redirection file \\
    |childdoc.pdf| & manual
\end{tabular}
\end{center}
%
The distribution consists of the files
|README.txt|, |childdoc.ins| and |childdoc.dtx|.
%
\begin{itemize}
\item
Run (pdf)\LaTeX{} on |childdoc.dtx|
to compile the manual |childdoc.pdf| (this file).
\item
Run \LaTeX{} on |childdoc.ins| to create the definitions file |childdoc.def|
and the sample |cdocsamp.tex| with include files
|cdocsch1.tex|, |cdocsch2.tex|, |cdocspt3.tex|, |cdocspt4.tex|,
|cdocsdrf.tex|, |cdocsfn1.tex|, |cdocsfn2.tex|.
Then copy the file |childdoc.def| to an appropriate directory of your \LaTeX{}
distribution, e.g.\ \textit{texmf-root}|/tex/latex/childdoc|.
\end{itemize}

%%%%%%%%%%%%%%%%%%%%%%%%%%%%%%%%%%%%%%%%%%%%%%%%%%%%%%%%%%%%%%%%%%%%%%%%%%%%%%%%
\subsection{Related CTAN Packages}

There are several other packages which offer a similar functionality:
%
\begin{itemize}
\item
The packages
\href{http://ctan.org/pkg/docmute}{\textsf{docmute}},
\href{http://ctan.org/pkg/includex}{\textsf{includex}} and
\href{http://ctan.org/pkg/standalone}{\textsf{standalone}}
provide commands to include only the document body of
a child file thus allowing both files to be compiled individually.
\item
The packages \href{http://ctan.org/pkg/subdocs}{\textsf{subdocs}}
and \href{http://ctan.org/pkg/subfiles}{\textsf{subfiles}}
provide structures in which the main and child documents can be
encapsulated and allowing them to be compiled individually.
The inclusion mechanism is different from the conventional |\include|.
\item
The package \href{http://ctan.org/pkg/combine}{\textsf{combine}}
is an elaborate solution to combine several documents into one.
\end{itemize}
%
See also the CTAN topic \href{http://ctan.org/topic/subdocs}{\textsf{subdocs}}
for further related packages.
The present package differs from the above solutions in that
a document structure constructed with the conventional |\include| mechanism
just needs two extra commands at the top of every file
such that all constituent files can be compiled individually.

%%%%%%%%%%%%%%%%%%%%%%%%%%%%%%%%%%%%%%%%%%%%%%%%%%%%%%%%%%%%%%%%%%%%%%%%%%%%%%%%
%\subsection{Feature Suggestions}
%
%The following is a list of features which may be useful for future
%versions of this package:
%%
%\begin{itemize}
%\item
%\ldots
%\end{itemize}

%%%%%%%%%%%%%%%%%%%%%%%%%%%%%%%%%%%%%%%%%%%%%%%%%%%%%%%%%%%%%%%%%%%%%%%%%%%%%%%%
\subsection{Revision History}

%%%%%%%%%%%%%%%%%%%%%%%%%%%%%%%%%%%%%%%%
\paragraph{v2.0:} 2018/12/30

\begin{itemize}
\item
immediate forward processing
\item
added |\childdocby| mechanism
\item
manual restructured
\end{itemize}

%%%%%%%%%%%%%%%%%%%%%%%%%%%%%%%%%%%%%%%%
\paragraph{v1.6:} 2018/01/17

\begin{itemize}
\item
application for development of include files
\item
corrections to manual
\end{itemize}

%%%%%%%%%%%%%%%%%%%%%%%%%%%%%%%%%%%%%%%%
\paragraph{v1.5:} 2017/05/21

\begin{itemize}
\item
more complete structuring introduced
\item
|\childdocof| introduced
\item
|\childdoc| renamed to |\childdocmain|
\item
|\childredirect| renamed to |\childdocforward| and |\childdocforwardprefix|
and functionality expanded
\end{itemize}

%%%%%%%%%%%%%%%%%%%%%%%%%%%%%%%%%%%%%%%%
\paragraph{v1.0:} 2017/04/27

\begin{itemize}
\item
manual and install package
\item
first version published on CTAN
\end{itemize}

%%%%%%%%%%%%%%%%%%%%%%%%%%%%%%%%%%%%%%%%
\paragraph{v0.6:} 2017/04/26

\begin{itemize}
\item
redirection mechanism added
\end{itemize}

%%%%%%%%%%%%%%%%%%%%%%%%%%%%%%%%%%%%%%%%
\paragraph{v0.5:} 2017/04/26

\begin{itemize}
\item
functionality in definition file
\end{itemize}


%%%%%%%%%%%%%%%%%%%%%%%%%%%%%%%%%%%%%%%%%%%%%%%%%%%%%%%%%%%%%%%%%%%%%%%%%%%%%%%%
%%%%%%%%%%%%%%%%%%%%%%%%%%%%%%%%%%%%%%%%%%%%%%%%%%%%%%%%%%%%%%%%%%%%%%%%%%%%%%%%
%%%%%%%%%%%%%%%%%%%%%%%%%%%%%%%%%%%%%%%%%%%%%%%%%%%%%%%%%%%%%%%%%%%%%%%%%%%%%%%%
\appendix

\settowidth\MacroIndent{\rmfamily\scriptsize 000\ }

 \DocInput{childdoc.dtx}

\end{document}
%</driver>
% \fi
%
% %%%%%%%%%%%%%%%%%%%%%%%%%%%%%%%%%%%%%%%%%%%%%%%%%%%%%%%%%%%%%%%%%%%%%%%%%%%%%%
% %%%%%%%%%%%%%%%%%%%%%%%%%%%%%%%%%%%%%%%%%%%%%%%%%%%%%%%%%%%%%%%%%%%%%%%%%%%%%%
% \section{Sample}
%\iffalse
%<*samplemain>
%\fi
%
% The following presents a sample document
% with two chapters, two parts, a title page,
% a compile flag as well as three forwarding files to set the flag.
% It consists of eight |.tex| files:
% \begin{center}
% \begin{tabular}{ll}
% |cdocsamp.tex|&main file\\
% |cdocsch1.tex|&include file for chapter 1\\
% |cdocsch2.tex|&include file for chapter 2\\
% |cdocspt3.tex|&include file for part 3\\
% |cdocspt4.tex|&include file for part 4\\
% |cdocsdrf.tex|&forwarding file for main file in draft mode\\
% |cdocsfi1.tex|&forwarding file for final version of chapter 1\\
% |cdocsfi2.tex|&forwarding file for final version of chapter 2\\
% \end{tabular}
% \end{center}
% Each of the eight files can be compiled directly by the \LaTeX{} compiler.
%
% %%%%%%%%%%%%%%%%%%%%%%%%%%%%%%%%%%%%%%
% \paragraph{Main File.}
%
% The main file is called |cdocsamp.tex|.
%
% Load the \textsf{childdoc} definitions and
% declare the filename for the main document:
%    \begin{macrocode}
\input{childdoc.def}
\childdocmain{}
%    \end{macrocode}

% Optional override for |\version| flag:
%    \begin{macrocode}
%%\ifchilddoc\else\providecommand{\version}{draft}\fi
%    \end{macrocode}

% Define the default values for the |\version| flag
% (|final| for the main file and |draft| for childs):
%    \begin{macrocode}
\ifchilddoc
\providecommand{\version}{draft}
\else
\providecommand{\version}{final}
\fi
%    \end{macrocode}

% Load the standard document class:
%    \begin{macrocode}
\documentclass[12pt]{article}
%    \end{macrocode}

% Start the document body:
%    \begin{macrocode}
\begin{document}
%    \end{macrocode}

% Declare a title page.
% Print title, part of document being processed and version flag:
%    \begin{macrocode}
\addtocounter{page}{-1}
\begin{center}
{\LARGE\bfseries{}childdoc example\par}
\vspace{1cm}
\ifchilddoc
\ifchilddocmanual part\else chapter\fi:
`\childdocname' of `\childdocjob'\par
\else
main document: `\childdocjob'\par
\fi
version: \version\par
\end{center}
\newpage
%    \end{macrocode}

% Manually include selected file,
% otherwise process as usual:
%    \begin{macrocode}
\ifchilddocmanual
\section*{part `\childdocname'}
\input{\childdocname}
\else
%    \end{macrocode}

% Include the two chapters:
%    \begin{macrocode}
\include{cdocsch1}
\include{cdocsch2}
%    \end{macrocode}

% Include the two parts unless only chapters should be displayed:
%    \begin{macrocode}
\ifchilddoc\else
\section{part three}
\input{cdocspt3}
\section{part four}
\input{cdocspt4}
\fi
%    \end{macrocode}

% Process as usual until here:
%    \begin{macrocode}
\fi
%    \end{macrocode}

% End of document body:
%    \begin{macrocode}
\end{document}
%    \end{macrocode}
%\iffalse
%</samplemain>
%\fi
%
% %%%%%%%%%%%%%%%%%%%%%%%%%%%%%%%%%%%%%%
% \paragraph{Chapter Include Files.}
%
% The include files are called |cdocsch1.tex| and |cdocsch2.tex|.
%
%\iffalse
%<*samplechap1|samplechap2>
%\fi

% Optional override for |\version| flag:
%    \begin{macrocode}
%%\providecommand{\version}{final}
%    \end{macrocode}

% Include the main document:
%    \begin{macrocode}
\input{childdoc.def}
\childdocof{cdocsamp}
%    \end{macrocode}

%\iffalse
%</samplechap1|samplechap2>
%\fi
%
%\iffalse
%<*samplechap1>
%\fi
% Some text for chapter 1:
%    \begin{macrocode}
\section{one}
some text in chapter one
%    \end{macrocode}

%\iffalse
%</samplechap1>
%\fi
% Some text for chapter 2:
%\iffalse
%<*samplechap2>
%\fi
%    \begin{macrocode}
\section{two}
more text in chapter two
%    \end{macrocode}

%\iffalse
%</samplechap2>
%\fi
%
% %%%%%%%%%%%%%%%%%%%%%%%%%%%%%%%%%%%%%%
% \paragraph{Part Include Files.}
%
% The include files are called |cdocspt3.tex| and |cdocspt4.tex|.
%
%\iffalse
%<*samplepart3|samplepart4>
%\fi

% Optional override for |\version| flag:
%    \begin{macrocode}
%%\providecommand{\version}{final}
%    \end{macrocode}

% Include the main document:
%    \begin{macrocode}
\input{childdoc.def}
\childdocby{cdocsamp}
%    \end{macrocode}

%\iffalse
%</samplepart3|samplepart4>
%\fi
%
%\iffalse
%<*samplepart3>
%\fi
% Some text for part 3:
%    \begin{macrocode}
some text in part three
%    \end{macrocode}

%\iffalse
%</samplepart3>
%\fi
% Some text for part 4:
%\iffalse
%<*samplepart4>
%\fi
%    \begin{macrocode}
more text in part four
%    \end{macrocode}

%\iffalse
%</samplepart4>
%\fi
%
% %%%%%%%%%%%%%%%%%%%%%%%%%%%%%%%%%%%%%%
% \paragraph{Forwarding for a Complete Draft.}
%
% The following forwarding file |cdocsdrf.tex|
% compiles the main document in draft mode:
%\iffalse
%<*sampledraft>
%\fi
%    \begin{macrocode}
\def\version{draft}
\input{childdoc.def}
\childdocforward{cdocsamp}
%    \end{macrocode}

%\iffalse
%</sampledraft>
%\fi
%
% %%%%%%%%%%%%%%%%%%%%%%%%%%%%%%%%%%%%%%
% \paragraph{Forwarding for Final Version of the Chapters.}
%
% The following forwarding files |cdocsfn1.tex| and |cdocsfn2.tex|
% (with identical content)
% compile the final versions of the child documents
% |cdocsch1.tex| and |cdocsch2.tex|, respectively:
%\iffalse
%<*samplefinal>
%\fi
%    \begin{macrocode}
\def\version{final}
\input{childdoc.def}
\childdocforwardprefix[cdocsamp]{cdocsfn}{cdocsch}
%    \end{macrocode}

%\iffalse
%</samplefinal>
%\fi
%
% %%%%%%%%%%%%%%%%%%%%%%%%%%%%%%%%%%%%%%
% \paragraph{Command Line Processing.}
%
% The following three command lines generate the output files
% |cdocscld|, |cdocscl1| and |cdocscl2|
% which should be identical to
% |cdocsdrf|, |cdocsch1| and |cdocsfn2|, respectively:
% \begin{center}
% \begin{tabular}{l}
% |latex -jobname cdocscld \|\\
% |  "\def\version{draft}\input{childdoc.def}\childdocforward{cdocsamp}"|\\
% |latex -jobname cdocscl1 \|\\
% |  "\input{childdoc.def}\childdocforward[cdocsamp]{cdocsch1}"|\\
% |latex -jobname cdocscl2 \|\\
% |  "\def\version{final}\input{childdoc.def}\childdocforward{cdocsch2}"|
% \end{tabular}
% \end{center}
% Note that the trailing backslash on each first line
% merely continues the input to the second line
% (for convenient cut ant paste).
% Furthermore, the command |latex| can be replaced by any
% of its alternative versions such as |pdflatex|.
%
% %%%%%%%%%%%%%%%%%%%%%%%%%%%%%%%%%%%%%%%%%%%%%%%%%%%%%%%%%%%%%%%%%%%%%%%%%%%%%%
% %%%%%%%%%%%%%%%%%%%%%%%%%%%%%%%%%%%%%%%%%%%%%%%%%%%%%%%%%%%%%%%%%%%%%%%%%%%%%%
% \section{Implementation}
%\iffalse
%<*package>
%\fi
%
% This section describes the definitions file |childdoc.def|.

% The definitions cannot be loaded using |\usepackage| or |\RequirePackage|
% which has a mechanism to prevent loading a style file more than once.
% When loading the definitions by means of |\input|
% multiple instances have to be prevented manually:
%\iffalse
%This code needs to be before the `\ProvidesFile' directive
%which is defined at the beginning of this file.
%Therefore it is also placed there and commented out here.
%</package>
%<*discard>
%\fi
%    \begin{macrocode}
\ifdefined\childdocmain\endinput\fi
%    \end{macrocode}
%\iffalse
%</discard>
%<*package>
%\fi
%
% \macro{\ifchilddoc}
% \macro{\ifchilddocmanual}
% The conditional |\ifchilddoc| tells whether a
% child (true) or main (false) document is being compiled.
% The conditional |\ifchilddocmanual| tells whether
% the |\includeonly| mechanism is used (false) or
% the selection of child files must be performed manually (true).
% The definitions initialise to false:
%    \begin{macrocode}
\newif\ifchilddoc
\newif\ifchilddocmanual
%    \end{macrocode}

% \macro{\childdocname}
% \macro{\childdocjob}
% The macro |\childdocname| stores the name of the main document
% to be compiled. The macro |\childdocjob| stores the name of
% the document on which the \LaTeX{} compiler was originally invoked.
% The content of |\jobname| cannot be compared
% to filenames specified in the source due to different catcodes.
% The following code rescans |\jobname|, stores the result
% in |\childdocname| and saves a copy in |\childdocjob|:
%    \begin{macrocode}
\edef\childdocname{\scantokens\expandafter{\jobname\noexpand}}
\let\childdocjob\childdocname
%    \end{macrocode}

% \macro{\childdocdisable}
% The macro |\childdocdisable| prevents the main file
% from being processed more than once.
% At this stage, the main document command |\childdocmain|
% is assumed to be called once again where it should do nothing.
% Any subsequent call to it should prevent
% a secondary processing of the main document
% It overwrites the forwarding commands
% |\childdocof| and |\childdocforward|
% with empty macros to prevent further inclusions of the main document:
%    \begin{macrocode}
\newcommand{\childdocdisable}
{
  \renewcommand{\childdocmain}[1]{\renewcommand{\childdocmain}[1]{\endinput}}
  \renewcommand{\childdocof}[1]{}
  \renewcommand{\childdocby}[2][]{}
  \renewcommand{\childdocforward}[2][]{}
  \renewcommand{\childdocdisable}{}
}
%    \end{macrocode}

% \macro{\childdocmain}
% The macro |\childdocmain| is to be called at the top of the main file
% with nothing or the main filename (without extension) as argument.
% First, it breaks loops.
% If the argument is not empty and does not match |\childdocname|
% (which is set by the first inclusion of |childdoc.def|),
% |\ifchilddoc| is set to true, |\includeonly| is applied to the child file
% and |\jobname| is set to the main file
% (for proper handling of |.aux| files):
%    \begin{macrocode}
\newcommand{\childdocmain}[1]
{
  \childdocdisable\childdocmain{}
  \if?#1?\else
    \begingroup
      \def\childdoctmp{#1}
      \ifx\childdoctmp\childdocname
        \def\childdoctmp{}
      \else
        \def\childdoctmp
        {
          \childdoctrue
          \includeonly{\childdocname}
          \def\childdocjob{#1}
          \def\jobname{#1}
        }
      \fi
      \expandafter
    \endgroup
    \childdoctmp
  \fi
}
%    \end{macrocode}

% \macro{\childdocof}
% The command |\childdocof| redirects
% compilation to the main file |#1|.
%    \begin{macrocode}
\newcommand{\childdocof}[1]
{
  \childdocdisable
  \childdoctrue
  \includeonly{\childdocname}
  \def\jobname{#1}
  \def\childdocjob{#1}
  \input{#1}
}
%    \end{macrocode}

% \macro{\childdocby}
% The command |\childdocby| ....
%    \begin{macrocode}
\newcommand{\childdocby}[2][]
{
  \childdocdisable
  \childdoctrue
  \childdocmanualtrue
  \if?#1?\else
    \def\jobname{#2}
  \fi
  \def\childdocjob{#2}
  \input{#2}
  \endinput
}
%    \end{macrocode}

% \macro{\childdocforward}
% The command |\childdocforward| redirects
% compilation to the main file or
% (if the optional argument is given) a child file.
% Parameters are set as if the main file
% or a child file starting with |\childdocof| was compiled.
% Then compilation is handed over to the main file:
%    \begin{macrocode}
\newcommand{\childdocforward}[2][]
{
  \begingroup
    \if?#1?
      \def\childdoctmp
      {
        \def\childdocname{#2}
        \def\childdocjob{#2}
        \def\jobname{#2}
        \input{#2}
        \endinput
      }
    \else
      \def\childdoctmp
      {
        \childdocdisable
        \def\childdocname{#2}
        \childdoctrue
        \includeonly{#2}
        \def\childdocjob{#1}
        \def\jobname{#1}
        \input{#1}
        \endinput
      }
    \fi
    \expandafter
  \endgroup
  \childdoctmp
}
%    \end{macrocode}

% \macro{\childdocforwardprefix}
% The command |\childdocforwardprefix| redirects
% compilation to the main or a child file by means of a pattern.
% The prefix |#1| in the current filename is replaced by |#2|
% and the suffix of the current filename is kept
% (it is assumed that the filename does not contain the substring `|~~~|'
% which is used as a delimiter).
% Compilation is handed over to the new file by |\childdocforward|:
%    \begin{macrocode}
\newcommand{\childdocforwardprefix}[3][]
{
  \begingroup
    \def\childdocextract #2##1~~~{\def\childdoctmp{\childdocforward[#1]{#3##1}}}
    \expandafter\childdocextract\childdocname~~~
    \expandafter
  \endgroup
  \childdoctmp
}
%    \end{macrocode}

% \macro{\childdoc}
% The deprecated macro |\childdoc| is a legacy version of |\childdocmain|:
%    \begin{macrocode}
\newcommand{\childdoc}{\childdocmain}
%    \end{macrocode}

% \macro{\childdocredirect}
% The deprecated macro |\childdocredirect| is a legacy version
% of |\childdocforward| and |\childdocforwardprefix|:
%    \begin{macrocode}
\newcommand{\childdocredirect}[2][]
{
  \begingroup
    \if?#1?
      \def\childdoctmp{\childdocforward{#2}}
    \else
      \def\childdoctmp{\childdocforwardprefix{#1}{#2}}
    \fi
    \expandafter
  \endgroup
  \childdoctmp
}
%    \end{macrocode}

%\iffalse
%</package>
%\fi
%
\endinput
|\\
|\childdocforward{|\textit{main}|}|\\
\end{tabular}
\end{center}
%
or alternatively with:
%
\begin{center}
\begin{tabular}{l}
|% \iffalse
%
% childdoc.dtx Copyright (C) 2017-2018 Niklas Beisert
%
% This work may be distributed and/or modified under the
% conditions of the LaTeX Project Public License, either version 1.3
% of this license or (at your option) any later version.
% The latest version of this license is in
%   http://www.latex-project.org/lppl.txt
% and version 1.3 or later is part of all distributions of LaTeX
% version 2005/12/01 or later.
%
% This work has the LPPL maintenance status `maintained'.
%
% The Current Maintainer of this work is Niklas Beisert.
%
% This work consists of the files childdoc.dtx and childdoc.ins
% and the derived files childdoc.def and cdocsamp.tex with
% cdocsch1.tex, cdocsch2.tex, cdocsdrf.tex, cdocsfn1.tex, cdocsfn2.tex.
%
%<package>\ifdefined\childdocmain\endinput\fi
%<package>\ProvidesFile{childdoc.def}[2018/12/30 v2.0 child document driver]
%<samplemain>\ProvidesFile{cdocsamp.tex}[2018/12/30 v2.0 sample for childdoc]
%<*driver>
%\ProvidesFile{childdoc.drv}[2018/12/30 v2.0 childdoc reference manual file]
\PassOptionsToClass{10pt,a4paper}{article}
\documentclass{ltxdoc}

\usepackage[margin=35mm]{geometry}
\usepackage{hyperref}
\usepackage{hyperxmp}
\usepackage[usenames]{color}

\hypersetup{colorlinks=true}
\hypersetup{pdfstartview=FitH}
\hypersetup{pdfpagemode=UseNone}
\hypersetup{pdfsource={}}
\hypersetup{pdflang={en-UK}}
\hypersetup{pdfcopyright={Copyright 2017-2018 Niklas Beisert.
  This work may be distributed and/or modified under the
  conditions of the LaTeX Project Public License, either version 1.3
  of this license or (at your option) any later version.}}
\hypersetup{pdflicenseurl={http://www.latex-project.org/lppl.txt}}
\hypersetup{pdfcontactaddress={ETH Zurich, ITP, HIT K,
  Wolfgang-Pauli-Strasse 27}}
\hypersetup{pdfcontactpostcode={8093}}
\hypersetup{pdfcontactcity={Zurich}}
\hypersetup{pdfcontactcountry={Switzerland}}
\hypersetup{pdfcontactemail={nbeisert@itp.phys.ethz.ch}}
\hypersetup{pdfcontacturl={http://people.phys.ethz.ch/\xmptilde nbeisert/}}

\newcommand{\secref}[1]{\hyperref[#1]{section \ref*{#1}}}

\parskip1ex
\parindent0pt
\let\olditemize\itemize
\def\itemize{\olditemize\parskip0pt}

\begin{document}

\title{The \textsf{childdoc} Package}
\hypersetup{pdftitle={The childdoc Package}}
\author{Niklas Beisert\\[2ex]
  Institut f\"ur Theoretische Physik\\
  Eidgen\"ossische Technische Hochschule Z\"urich\\
  Wolfgang-Pauli-Strasse 27, 8093 Z\"urich, Switzerland\\[1ex]
  \href{mailto:nbeisert@itp.phys.ethz.ch}
  {\texttt{nbeisert@itp.phys.ethz.ch}}}
\hypersetup{pdfauthor={Niklas Beisert}}
\hypersetup{pdfsubject={Manual for the LaTeX2e Package childdoc}}
\date{30 December 2018, \textsf{v2.0}}
\maketitle

\begin{abstract}\noindent
\textsf{childdoc} is a \LaTeXe{} package
that enables the direct compilation
of document sections included by |\include|
to individual files.
\end{abstract}

\begingroup
\parskip0ex
\tableofcontents
\endgroup

%%%%%%%%%%%%%%%%%%%%%%%%%%%%%%%%%%%%%%%%%%%%%%%%%%%%%%%%%%%%%%%%%%%%%%%%%%%%%%%%
%%%%%%%%%%%%%%%%%%%%%%%%%%%%%%%%%%%%%%%%%%%%%%%%%%%%%%%%%%%%%%%%%%%%%%%%%%%%%%%%
\section{Introduction}

\LaTeX{} provides a mechanism to structure a large document (such as a book)
into a main file and several child files (containing the chapters)
using the |\include| command.
This mechanism is beneficial for documents
which span hundreds of pages in order to
make the source file(s) more manageable.
Moreover, compilation can be restricted to
selected child files by means of the |\includeonly| command.
The latter feature can be used to reduce the compilation time while editing
(this was significantly more useful in the earlier days of \LaTeX{})
or to generate a smaller document which is easier to navigate.
Another application of |\includeonly| is to generate
documents consisting of selected parts of the complete document.

However, there are a few drawbacks of the plain |\include| mechanism:
\begin{itemize}
\item
The child files cannot be compiled on their own,
they can only be compiled via the main file.
A naive editing environment
(such as a text editor with an option
to have the current file processed by \LaTeX)
may require one to switch to the main file before compiling;
attempting to compile the child file produces errors.
\item
The main file must be modified (each time)
to adjust the |\includeonly| command
to the present needs. This easily leaves the main file in a messy state.
\item
The generated document will always carry the filename
of the main document. This is inconvenient if
several child files are to be compiled and
to be kept for distribution.
\end{itemize}

The present package provides a simple interface
to make child files individually compilable by \LaTeX{}.
Compiling a child file then has the same effect as compiling
the main file with an |\includeonly| command
to select the appropriate child.
Moreover the generated document will carry the name of the child
rather than the main file.
This resolves all three above issues.

This feature is meant to make the editing of books,
thesis documents and lecture notes somewhat more convenient.
However, the package can also be used efficiently for
composing a series of documents (such as exercise sheets)
which are typically distributed individually.
It then assists the author in generating the individual documents
(potentially in different versions)
as well as a document containing the collected series.
Another application is in developing style files
or other kinds of included material
where compilation of the style file could redirect
to a sample or test file.

%%%%%%%%%%%%%%%%%%%%%%%%%%%%%%%%%%%%%%%%%%%%%%%%%%%%%%%%%%%%%%%%%%%%%%%%%%%%%%%%
%%%%%%%%%%%%%%%%%%%%%%%%%%%%%%%%%%%%%%%%%%%%%%%%%%%%%%%%%%%%%%%%%%%%%%%%%%%%%%%%
\section{Usage}

First of all, the package \textsf{childdoc} is \emph{not} a standard
\LaTeXe{} |.sty| style file! Therefore it needs to be invoked in
a non-standard way.

%%%%%%%%%%%%%%%%%%%%%%%%%%%%%%%%%%%%%%%%%%%%%%%%%%%%%%%%%%%%%%%%%%%%%%%%%%%%%%%%
\subsection{Included Files}
\label{sec:include}

%%%%%%%%%%%%%%%%%%%%%%%%%%%%%%%%%%%%%%%%
\DescribeMacro{\childdocmain}
To use the package, add the commands
\begin{center}
\begin{tabular}{l}
|\input{childdoc.def}|\\
|\childdocmain{}|\\
\end{tabular}
\end{center}
at the very top of the main \LaTeX{} file,
in particular \emph{before} the |\documentclass| statement!
The argument of |\childdocmain| should be left empty
(but it must be present).

%%%%%%%%%%%%%%%%%%%%%%%%%%%%%%%%%%%%%%%%
\DescribeMacro{\childdocof}
Furthermore, add the commands
\begin{center}
\begin{tabular}{l}
|\input{childdoc.def}|\\
|\childdocof{|\textit{main}|}|\\
\end{tabular}
\end{center}
at the top of every child file \textit{child}
which is included by |\include{|\textit{child}|}|
from within the main file
(or at least for those files to be compiled individually).
The argument \textit{main} must be the filename of the main file.

There are a couple of
considerations in setting up the main and child documents:

%%%%%%%%%%%%%%%%%%%%%%%%%%%%%%%%%%%%%%%%
\paragraph{Restrictions.}

Please note the following restrictions:
\begin{itemize}
\item
|\childdocmain| must be called with one argument \textit{main}
to ensure compatibility with earlier version of the package.
It must either be empty (|\childdocmain{}|)
or precisely match the filename of the main file in which it is specified.
See \secref{sec:detection} for further information.
\item
The filename \textit{main} must be specified without the |.tex| extension.
\item
The filename \textit{main} is case sensitive
(even in case-insensitive file systems)
due to internal string comparison.
\item
The argument \textit{main} should be fully expanded, it cannot be a macro.
\item
Subdirectories and special characters should be avoided in filenames.
\item
The command |\childdocmain{|\textit{main}|}| must be followed by a whitespace.
It should not be followed immediately by another command
or by a comment mark `|%|'.
This is because the \TeX{} parser reads the token immediately following
the argument of |\childdocmain| and puts it
at the beginning of every child section;
however, a white\-space is ignored.
\end{itemize}

%%%%%%%%%%%%%%%%%%%%%%%%%%%%%%%%%%%%%%%%
\paragraph{Content of Main File.}

It is advisable to place all content in the child files included by |\include|.
Any output contained in the main file will appear in all child documents
unless suppressed manually;
it cannot be suppressed automatically by the |\includeonly| directive
and thus should normally be avoided.
A method to include some content in the main file
by means of conditional processing is described in \secref{sec:conditional}.

%%%%%%%%%%%%%%%%%%%%%%%%%%%%%%%%%%%%%%%%
\paragraph{Page Numbering.}

When only a part of the document is compiled,
the appropriate numbering of pages
(as well as other status parameters)
is determined from the |.aux| files.
The latter contain information from previous passes.
However this information needs to propagate through
all intermediate child documents.
Therefore the page numbering in child documents may well
be inconsistent until the complete document is compiled at least once.

A useful (if unconventional) way to always ensure a consistent
page numbering is to restart the numbering in each child document
and denote the pages by `\textit{child}|.|\textit{page}'
where \textit{child} represents the chapter/section number of the child file.
This can be achieved by the command
|\numberwithin{page}{|\textit{child}|}|
of the \textsf{amsmath} package
where \textit{child} can be |chapter| or |section|
depending on the chosen structuring.
Alternatively, one can modify the macro |\thepage| appropriately
and reset the counter |page| at the start of each child file.

%%%%%%%%%%%%%%%%%%%%%%%%%%%%%%%%%%%%%%%%%%%%%%%%%%%%%%%%%%%%%%%%%%%%%%%%%%%%%%%%
\subsection{Conditional Processing}
\label{sec:conditional}

The package provides a mechanism to compile different versions
of a document. To customise the versions further some conditional processing
can come in handy to distinguish which version is being compiled.
The package provides two macros to describe the compilation context:

%%%%%%%%%%%%%%%%%%%%%%%%%%%%%%%%%%%%%%%%
\DescribeMacro{\ifchilddoc}
The conditional |\ifchilddoc| distinguishes between the compilation of
child documents and the main document:
%
\begin{center}
|\ifchilddoc |\textit{child-code}| |[|\||else |\textit{main-code}]| \||fi|
\end{center}

%%%%%%%%%%%%%%%%%%%%%%%%%%%%%%%%%%%%%%%%
\DescribeMacro{\childdocname}
\DescribeMacro{\childdocjob}
The macro |\childdocname| contains the filename (without extension)
of the main or child file being processed.
Note that |\childdocjob| will always contain the name of the main file.

%%%%%%%%%%%%%%%%%%%%%%%%%%%%%%%%%%%%%%%%
\paragraph{Title Page.}

Conditional processing can be used to include a title or banner page
in the main document when proper precautions are taken.
Importantly, the code in the main file should ensure that the page counter
(as well as other status parameters which are stored in the |.aux| files)
takes the same value after the conditional processing.
Otherwise the page numbers may take divergent values
depending on which part is compiled.

For example, a title page could be declared by:
%
\begin{center}
\begin{tabular}{l}
|\ifchilddoc\||else|\\
|\addtocounter{page}{-1}|\\
\textit{code for title page}\\
|\newpage|\\
|\||fi|
\end{tabular}
\end{center}
%
A banner page for the child documents can be generated by:
%
\begin{center}
\begin{tabular}{l}
|\ifchilddoc|\\
|\addtocounter{page}{-1}|\\
\textit{code for banner page}\\
|\newpage|\\
|\||fi|
\end{tabular}
\end{center}
%
Here one could write a message such as:
\begin{center}
|This is the part \childdocname{} of \childdocjob{}.|
\end{center}

%%%%%%%%%%%%%%%%%%%%%%%%%%%%%%%%%%%%%%%%%%%%%%%%%%%%%%%%%%%%%%%%%%%%%%%%%%%%%%%%
\subsection{Flags}
\label{sec:flags}

The package makes it easy to generate different versions
of the main or child documents.
To this end compilation flags can be defined
and assigned different default values.
They will be particularly useful in conjunction
with the forwarding mechanism described in \secref{sec:forward}.

For example, it may be useful to have a flag |\version|
which can be set to |draft| or |final|.
The document source will contain some conditional code
depending on the value of |\version|.
Suppose further, the flag should default to |final| for the main file
and to |draft| for child files
which is a natural assignment for editing the document.
This is achieved by placing the following code
in the preamble of the main document
(below the |\childdocmain| directive):
%
\begin{center}
\begin{tabular}{l}
|\ifchilddoc|\\
|\providecommand{\version}{draft}|\\
|\||else|\\
|\providecommand{\version}{final}|\\
|\||fi|
\end{tabular}
\end{center}
%
The definition by |\providecommand| makes sure
that previous definitions are not overwritten.
Further statements |\providecommand{\version}{...}|
can thus be added before the above code to override it.

For the main file, one might add a line
(between |\childdocmain| and the above block)
%
\begin{center}
|%\ifchilddoc\||else\providecommand{\version}{draft}\||fi|
\end{center}
%
which can be uncommented to produce a draft version.
Likewise one can add a line to the very top of a child file
(above the |\childdocof{|\textit{main}|}| directive)
%
\begin{center}
|%\providecommand{\version}{final}|
\end{center}
%
which can be uncommented to produce the final version of this child document.

%%%%%%%%%%%%%%%%%%%%%%%%%%%%%%%%%%%%%%%%%%%%%%%%%%%%%%%%%%%%%%%%%%%%%%%%%%%%%%%%
\subsection{Forwarding}
\label{sec:forward}

Different versions of the main or child documents
using compilation flags as described in \secref{sec:flags}
can be (permanently) stored in different files
for convenient compilation, viewing and distribution.
To this end, the package defines a command
to pass on compilation to a different file:

%%%%%%%%%%%%%%%%%%%%%%%%%%%%%%%%%%%%%%%%
\DescribeMacro{\childdocforward}
The command |\childdocforward| redirects processing to
another source file:
%
\begin{center}
\begin{tabular}{l}
|\input{childdoc.def}|\\
|\childdocforward[|\textit{main}|]{|\textit{dest}|}|\\
\end{tabular}
\end{center}
%
The argument \textit{dest} is the destination file
(without extension).
It should be the main file or one of the child files.
Note that further \textsf{childdoc} directives
such as |\childdocof| and |\childdocforward|
in the indicated file will be processed in this form.
The optional argument \textit{main}
passes on directly to the main file \textit{main}
while pretending to compile the child \textit{dest}.
This form behaves as if \textit{dest}
issues |\childdocof{|\textit{main}|}| right away,
and no further \textsf{childdoc} directives will be processed.

%%%%%%%%%%%%%%%%%%%%%%%%%%%%%%%%%%%%%%%%
\DescribeMacro{\...prefix}
In the alternative form |\childdocforwardprefix|,
%
\begin{center}
\begin{tabular}{l}
|\input{childdoc.def}|\\
|\childdocforwardprefix[|\textit{main}|]{|\textit{prefix}|}{|\textit{dest}|}|
\end{tabular}
\end{center}
%
the destination file is determined by a pattern
depending on the current file:
To make this work, the current file must be called
`{\textit{prefix}\hspace{0.2em}\textit{suffix}}'
with \textit{prefix} matching precisely the argument.
Processing is then passed on to the file
`{\textit{dest}\hspace{0.2em}\textit{suffix}}'.
Surely, the same effect is achieved by
directly specifying the
argument `{\textit{dest}\hspace{0.2em}\textit{suffix}}'
in the first form.
However, that requires to set up a different file
for each child. With the alternative form of the command
all these files can have exactly the same content
which simplifies setting them up and maintaining them.

For example, the following file |draft.tex|
with a compilation flag |\version| as described in \secref{sec:flags}
compiles the main document as a draft:
%
\begin{center}
\begin{tabular}{l}
|\def\version{draft}|\\
|\input{childdoc.def}|\\
|\childdocforward{|\textit{main}|}|
\end{tabular}
\end{center}
%
Likewise, the following files |final|\textit{nn}|.tex|
compile the final version of the child document
|child|\textit{nn}|.tex|:
%
\begin{center}
\begin{tabular}{l}
|\def\version{final}|\\
|\input{childdoc.def}|\\
|\childdocforwardprefix{final}{child}|
\end{tabular}
\end{center}
%

Note that when several versions of a main file and/or of each child file
are to be generated, it may be convenient to set up a |Makefile| or
shell script to automatise the process.

%%%%%%%%%%%%%%%%%%%%%%%%%%%%%%%%%%%%%%%%%%%%%%%%%%%%%%%%%%%%%%%%%%%%%%%%%%%%%%%%
\subsection{Command Line Processing}
\label{sec:commandline}

The effect of redirection files can also be achieved by invoking
the \LaTeX{} compiler with a more elaborate command line.
Most conveniently this should be done as part
of a shell script or a |Makefile|.

When using \textsf{childdoc} in the main file, the following
command lines effectively perform a redirection
(note that depending on the shell being used,
backslashes may have to be doubled: `|\|' $\to$ `|\\|'):
%
\begin{center}
|... -jobname "|\textit{target}|" |\\|"|[\textit{flags}]%
|\input{childdoc.def}\childdocforward[|\textit{main}|]{|\textit{dest}|}"|
\end{center}
%
Here \textit{target} is the name of the output file,
\textit{main} is the name of the main file
and \textit{dest} is the name of the main or child file to be processed
(all filenames without extensions).
The optional argument \textit{main} can be omitted
if \textit{main} matches \textit{dest}.
Optionally, compilation \textit{flags} can be defined via |\def| commands.
This command line makes the \TeX{} engine believe
it is compiling the file \textit{target}
whose content is specified as the latter parameter.
The provided code then forwards the processing to
\textit{main} or \textit{dest} as described in \secref{sec:forward}.

%%%%%%%%%%%%%%%%%%%%%%%%%%%%%%%%%%%%%%%%%%%%%%%%%%%%%%%%%%%%%%%%%%%%%%%%%%%%%%%%
\subsection{Include by Input}
\label{sec:input}

Including child documents by |\include| has some restrictions by design.
Most notably, the content of a child document always occupies
its own set of pages; pages cannot be shared between child documents.
Usually, this behaviour makes perfect sense
because each child document contain an essential part of the document.
However, in some situations it may be desirable to compose
a document from a collection of parts
without having mandatory page breaks between then.
For this case, the package
provides a mechanism to include parts
by |\input| which can also be processed individually.
However, by construction this mechanism
requires manual handling of the content to be output.

%%%%%%%%%%%%%%%%%%%%%%%%%%%%%%%%%%%%%%%%
\DescribeMacro{\ifchilddocmanual}
The main file should be prepared as usual, see \secref{sec:include}.
However, the document body must make a distinction
between processing of an individual part and of the main document, e.g.:
%
\begin{center}
\begin{tabular}{l}
|\ifchilddocmanual|\\
|\input{\childdocname}|\\
|\||else|\\
\textit{document body with }|\input{|\textit{part}|}|\\
|\||fi|
\end{tabular}
\end{center}
%
The conditional |\ifchilddocmanual| is true whenever
a part to be included by |\input| is being compiled,
and the name of the part is stored in |\childdocname|.

%%%%%%%%%%%%%%%%%%%%%%%%%%%%%%%%%%%%%%%%
\DescribeMacro{\childdocby}
Each part to be included by |\input| should start with:
%
\begin{center}
\begin{tabular}{l}
|\input{childdoc.def}|\\
|\childdocby{|\textit{main}|}|\\
\end{tabular}
\end{center}
%
The directive |\childdocby| is similar to |\childdocof|
described in \secref{sec:include},
but the subsequent selection of content must be done manually.
To that end, both |\ifchilddoc| and |\ifchilddocmanual|
will be true upon processing of a part,
and the name of the part is stored in |\childdocname|.
Note that |\jobname| will be set to the filename of the current part
so that each part receives an individual |.aux| file
that does not interfere with the |.aux| file(s) of the main document.
This behaviour can be altered by the alternative form
|\childdocby[*]{|\textit{main}|}| (with a non-empty optional argument)
which uses the |.aux| file of the main document
by setting |\jobname| to \textit{main}.

%%%%%%%%%%%%%%%%%%%%%%%%%%%%%%%%%%%%%%%%%%%%%%%%%%%%%%%%%%%%%%%%%%%%%%%%%%%%%%%%
\subsection{Driver Development}
\label{sec:driver}

The \textsf{childdoc} mechanism can also be use for the development
of definition files such as \LaTeX{} styles or classes.
This case differs from the above setup with multiple parts
included by |\include| in that no |\includeonly| should be invoked.
This can be achieved by starting the include file
(before |\ProvidesPackage|) with:
%
\begin{center}
\begin{tabular}{l}
|\input{childdoc.def}|\\
|\childdocforward{|\textit{main}|}|\\
\end{tabular}
\end{center}
%
or alternatively with:
%
\begin{center}
\begin{tabular}{l}
|\input{childdoc.def}|\\
|\childdocby{|\textit{main}|}|\\
\end{tabular}
\end{center}
%
Both forms have slightly different effects as described above.
The main file is prepared as usual, see \secref{sec:include}.

%%%%%%%%%%%%%%%%%%%%%%%%%%%%%%%%%%%%%%%%%%%%%%%%%%%%%%%%%%%%%%%%%%%%%%%%%%%%%%%%
\subsection{Legacy Detection}
\label{sec:detection}

The directive |\childdocmain| in the main file can detect
whether the complete document or merely a child is to be compiled
even without using the directive |\childdocof|.
This method is deprecated because it is less robust
and there is no compelling reason to use it;
it is merely provided for backward compatibility
and it may be removed in future versions.

If the detection mechanism is to be used,
it is mandatory to correctly specify
the filename of the main file as the argument of |\childdocmain|:
%
\begin{center}
\begin{tabular}{l}
|\input{childdoc.def}|\\
|\childdocmain{|\textit{main}|}|\\
\end{tabular}
\end{center}
%
If |\jobname| does not match the argument \textit{main} of |\childdocmain|,
it is assumed that |\jobname| points to the child file to be compiled.
When using |\childdocmain| with the main file specified as argument,
it suffices to start a child file
with just |\input{|\textit{main}|}|
without loading of the package and using |\childdocof|.
If instead all processing is done
with the appropriate \textsf{childdoc} directives,
the argument of \textit{main} of |\childdocmain| can be empty.

An alternative version of the command line processing described
in \secref{sec:commandline} using the detection mechanism reads:
%
\begin{center}
|... -jobname "|\textit{target}|" "|[\textit{flags}]%
[|\def\jobname{|\textit{dest}|}|]|\input{|\textit{main}|}"|
\end{center}

%%%%%%%%%%%%%%%%%%%%%%%%%%%%%%%%%%%%%%%%%%%%%%%%%%%%%%%%%%%%%%%%%%%%%%%%%%%%%%%%
\subsection{Manual Code}
\label{sec:manual}

In case one cannot be certain whether the definitions file |childdoc.def|
is installed on the target \TeX{} distribution
and one prefers not to ship it,
it is conceivable to paste a few relevant commands into the sources.

To that end, drop all statements |\input{childdoc.def}|
and perform the replacements as outlined below.
Instead of |\childdocmain{|\textit{main}|}| add the following code
to the top of the main file:
%
\begin{center}
\begin{tabular}{l}
|\||ifdefined\childdocname\endinput\||fi\newif\ifchilddoc|\\
|\edef\childdocname{\scantokens\expandafter{\jobname\noexpand}}|\\
|\def\childdocmain{|\textit{main}|}\||ifx\childdocmain\childdocname\||else|\\
|\childdoctrue\includeonly{\childdocname}\let\jobname\childdocmain\||fi|\\
\end{tabular}
\end{center}
%
Instead of |\childdocof{|\textit{main}|}| just include the main file
at the top of each child file:
%
\begin{center}
|\input{|\textit{main}|}|
\end{center}
%
A simple redirection |\childdocforward{|\textit{dest}|}| is achieved by:
%
\begin{center}
|\def\jobname{|\textit{dest}|}\input{\jobname}|
\end{center}
%
The redirection with prefix
|\childdocforwardprefix[|\textit{prefix}|]{|\textit{dest}|}|
is accomplished by:
%
\begin{center}
\begin{tabular}{l}
|{\edef\jobname{\scantokens\expandafter{\jobname\noexpand}}|\\
|\def\redirectjob |\textit{prefix}|#1~~~{\gdef\jobname{|\textit{dest}|#1}}|\\
|\expandafter\redirectjob\jobname~~~}\input{\jobname}|
\end{tabular}
\end{center}

In an alternative approach,
child documents can be compiled by a specific command line
without additional code or specific definitions:
%
\begin{center}
|... -jobname "|\textit{target}|" "|[\textit{flags}]%
|\includeonly{|\textit{dest}|}\input{|\textit{main}|}"|
\end{center}
%

%%%%%%%%%%%%%%%%%%%%%%%%%%%%%%%%%%%%%%%%%%%%%%%%%%%%%%%%%%%%%%%%%%%%%%%%%%%%%%%%
%%%%%%%%%%%%%%%%%%%%%%%%%%%%%%%%%%%%%%%%%%%%%%%%%%%%%%%%%%%%%%%%%%%%%%%%%%%%%%%%
\section{Information}

%%%%%%%%%%%%%%%%%%%%%%%%%%%%%%%%%%%%%%%%%%%%%%%%%%%%%%%%%%%%%%%%%%%%%%%%%%%%%%%%
\subsection{Copyright}

Copyright \copyright{} 2017--2018 Niklas Beisert

This work may be distributed and/or modified under the
conditions of the \LaTeX{} Project Public License, either version 1.3
of this license or (at your option) any later version.
The latest version of this license is in
  \url{http://www.latex-project.org/lppl.txt}
and version 1.3 or later is part of all distributions of \LaTeX{}
version 2005/12/01 or later.

This work has the LPPL maintenance status `maintained'.

The Current Maintainer of this work is Niklas Beisert.

This work consists of the files |README.txt|, |childdoc.ins| and |childdoc.dtx|
as well as the derived files |childdoc.def|, |cdocsamp.tex|
with |cdocsch1.tex|, |cdocsch2.tex|, |cdocspt3.tex|, |cdocspt4.tex|,
|cdocsdrf.tex|, |cdocsfn1.tex|, |cdocsfn2.tex|
as well as |childdoc.pdf|.

%%%%%%%%%%%%%%%%%%%%%%%%%%%%%%%%%%%%%%%%%%%%%%%%%%%%%%%%%%%%%%%%%%%%%%%%%%%%%%%%
\subsection{Files and Installation}

The package consists of the files:
%
\begin{center}
\begin{tabular}{ll}
    |README.txt|   & readme file \\
    |childdoc.ins| & installation file \\
    |childdoc.dtx| & source file \\
    |childdoc.def| & definition file \\
    |cdocsamp.tex| & sample main file \\
    |cdocsch1.tex| & sample include file \\
    |cdocsch2.tex| & sample include file \\
    |cdocspt3.tex| & sample part file \\
    |cdocspt4.tex| & sample part file \\
    |cdocsdrf.tex| & sample redirection file \\
    |cdocsfn1.tex| & sample redirection file \\
    |cdocsfn2.tex| & sample redirection file \\
    |childdoc.pdf| & manual
\end{tabular}
\end{center}
%
The distribution consists of the files
|README.txt|, |childdoc.ins| and |childdoc.dtx|.
%
\begin{itemize}
\item
Run (pdf)\LaTeX{} on |childdoc.dtx|
to compile the manual |childdoc.pdf| (this file).
\item
Run \LaTeX{} on |childdoc.ins| to create the definitions file |childdoc.def|
and the sample |cdocsamp.tex| with include files
|cdocsch1.tex|, |cdocsch2.tex|, |cdocspt3.tex|, |cdocspt4.tex|,
|cdocsdrf.tex|, |cdocsfn1.tex|, |cdocsfn2.tex|.
Then copy the file |childdoc.def| to an appropriate directory of your \LaTeX{}
distribution, e.g.\ \textit{texmf-root}|/tex/latex/childdoc|.
\end{itemize}

%%%%%%%%%%%%%%%%%%%%%%%%%%%%%%%%%%%%%%%%%%%%%%%%%%%%%%%%%%%%%%%%%%%%%%%%%%%%%%%%
\subsection{Related CTAN Packages}

There are several other packages which offer a similar functionality:
%
\begin{itemize}
\item
The packages
\href{http://ctan.org/pkg/docmute}{\textsf{docmute}},
\href{http://ctan.org/pkg/includex}{\textsf{includex}} and
\href{http://ctan.org/pkg/standalone}{\textsf{standalone}}
provide commands to include only the document body of
a child file thus allowing both files to be compiled individually.
\item
The packages \href{http://ctan.org/pkg/subdocs}{\textsf{subdocs}}
and \href{http://ctan.org/pkg/subfiles}{\textsf{subfiles}}
provide structures in which the main and child documents can be
encapsulated and allowing them to be compiled individually.
The inclusion mechanism is different from the conventional |\include|.
\item
The package \href{http://ctan.org/pkg/combine}{\textsf{combine}}
is an elaborate solution to combine several documents into one.
\end{itemize}
%
See also the CTAN topic \href{http://ctan.org/topic/subdocs}{\textsf{subdocs}}
for further related packages.
The present package differs from the above solutions in that
a document structure constructed with the conventional |\include| mechanism
just needs two extra commands at the top of every file
such that all constituent files can be compiled individually.

%%%%%%%%%%%%%%%%%%%%%%%%%%%%%%%%%%%%%%%%%%%%%%%%%%%%%%%%%%%%%%%%%%%%%%%%%%%%%%%%
%\subsection{Feature Suggestions}
%
%The following is a list of features which may be useful for future
%versions of this package:
%%
%\begin{itemize}
%\item
%\ldots
%\end{itemize}

%%%%%%%%%%%%%%%%%%%%%%%%%%%%%%%%%%%%%%%%%%%%%%%%%%%%%%%%%%%%%%%%%%%%%%%%%%%%%%%%
\subsection{Revision History}

%%%%%%%%%%%%%%%%%%%%%%%%%%%%%%%%%%%%%%%%
\paragraph{v2.0:} 2018/12/30

\begin{itemize}
\item
immediate forward processing
\item
added |\childdocby| mechanism
\item
manual restructured
\end{itemize}

%%%%%%%%%%%%%%%%%%%%%%%%%%%%%%%%%%%%%%%%
\paragraph{v1.6:} 2018/01/17

\begin{itemize}
\item
application for development of include files
\item
corrections to manual
\end{itemize}

%%%%%%%%%%%%%%%%%%%%%%%%%%%%%%%%%%%%%%%%
\paragraph{v1.5:} 2017/05/21

\begin{itemize}
\item
more complete structuring introduced
\item
|\childdocof| introduced
\item
|\childdoc| renamed to |\childdocmain|
\item
|\childredirect| renamed to |\childdocforward| and |\childdocforwardprefix|
and functionality expanded
\end{itemize}

%%%%%%%%%%%%%%%%%%%%%%%%%%%%%%%%%%%%%%%%
\paragraph{v1.0:} 2017/04/27

\begin{itemize}
\item
manual and install package
\item
first version published on CTAN
\end{itemize}

%%%%%%%%%%%%%%%%%%%%%%%%%%%%%%%%%%%%%%%%
\paragraph{v0.6:} 2017/04/26

\begin{itemize}
\item
redirection mechanism added
\end{itemize}

%%%%%%%%%%%%%%%%%%%%%%%%%%%%%%%%%%%%%%%%
\paragraph{v0.5:} 2017/04/26

\begin{itemize}
\item
functionality in definition file
\end{itemize}


%%%%%%%%%%%%%%%%%%%%%%%%%%%%%%%%%%%%%%%%%%%%%%%%%%%%%%%%%%%%%%%%%%%%%%%%%%%%%%%%
%%%%%%%%%%%%%%%%%%%%%%%%%%%%%%%%%%%%%%%%%%%%%%%%%%%%%%%%%%%%%%%%%%%%%%%%%%%%%%%%
%%%%%%%%%%%%%%%%%%%%%%%%%%%%%%%%%%%%%%%%%%%%%%%%%%%%%%%%%%%%%%%%%%%%%%%%%%%%%%%%
\appendix

\settowidth\MacroIndent{\rmfamily\scriptsize 000\ }

 \DocInput{childdoc.dtx}

\end{document}
%</driver>
% \fi
%
% %%%%%%%%%%%%%%%%%%%%%%%%%%%%%%%%%%%%%%%%%%%%%%%%%%%%%%%%%%%%%%%%%%%%%%%%%%%%%%
% %%%%%%%%%%%%%%%%%%%%%%%%%%%%%%%%%%%%%%%%%%%%%%%%%%%%%%%%%%%%%%%%%%%%%%%%%%%%%%
% \section{Sample}
%\iffalse
%<*samplemain>
%\fi
%
% The following presents a sample document
% with two chapters, two parts, a title page,
% a compile flag as well as three forwarding files to set the flag.
% It consists of eight |.tex| files:
% \begin{center}
% \begin{tabular}{ll}
% |cdocsamp.tex|&main file\\
% |cdocsch1.tex|&include file for chapter 1\\
% |cdocsch2.tex|&include file for chapter 2\\
% |cdocspt3.tex|&include file for part 3\\
% |cdocspt4.tex|&include file for part 4\\
% |cdocsdrf.tex|&forwarding file for main file in draft mode\\
% |cdocsfi1.tex|&forwarding file for final version of chapter 1\\
% |cdocsfi2.tex|&forwarding file for final version of chapter 2\\
% \end{tabular}
% \end{center}
% Each of the eight files can be compiled directly by the \LaTeX{} compiler.
%
% %%%%%%%%%%%%%%%%%%%%%%%%%%%%%%%%%%%%%%
% \paragraph{Main File.}
%
% The main file is called |cdocsamp.tex|.
%
% Load the \textsf{childdoc} definitions and
% declare the filename for the main document:
%    \begin{macrocode}
\input{childdoc.def}
\childdocmain{}
%    \end{macrocode}

% Optional override for |\version| flag:
%    \begin{macrocode}
%%\ifchilddoc\else\providecommand{\version}{draft}\fi
%    \end{macrocode}

% Define the default values for the |\version| flag
% (|final| for the main file and |draft| for childs):
%    \begin{macrocode}
\ifchilddoc
\providecommand{\version}{draft}
\else
\providecommand{\version}{final}
\fi
%    \end{macrocode}

% Load the standard document class:
%    \begin{macrocode}
\documentclass[12pt]{article}
%    \end{macrocode}

% Start the document body:
%    \begin{macrocode}
\begin{document}
%    \end{macrocode}

% Declare a title page.
% Print title, part of document being processed and version flag:
%    \begin{macrocode}
\addtocounter{page}{-1}
\begin{center}
{\LARGE\bfseries{}childdoc example\par}
\vspace{1cm}
\ifchilddoc
\ifchilddocmanual part\else chapter\fi:
`\childdocname' of `\childdocjob'\par
\else
main document: `\childdocjob'\par
\fi
version: \version\par
\end{center}
\newpage
%    \end{macrocode}

% Manually include selected file,
% otherwise process as usual:
%    \begin{macrocode}
\ifchilddocmanual
\section*{part `\childdocname'}
\input{\childdocname}
\else
%    \end{macrocode}

% Include the two chapters:
%    \begin{macrocode}
\include{cdocsch1}
\include{cdocsch2}
%    \end{macrocode}

% Include the two parts unless only chapters should be displayed:
%    \begin{macrocode}
\ifchilddoc\else
\section{part three}
\input{cdocspt3}
\section{part four}
\input{cdocspt4}
\fi
%    \end{macrocode}

% Process as usual until here:
%    \begin{macrocode}
\fi
%    \end{macrocode}

% End of document body:
%    \begin{macrocode}
\end{document}
%    \end{macrocode}
%\iffalse
%</samplemain>
%\fi
%
% %%%%%%%%%%%%%%%%%%%%%%%%%%%%%%%%%%%%%%
% \paragraph{Chapter Include Files.}
%
% The include files are called |cdocsch1.tex| and |cdocsch2.tex|.
%
%\iffalse
%<*samplechap1|samplechap2>
%\fi

% Optional override for |\version| flag:
%    \begin{macrocode}
%%\providecommand{\version}{final}
%    \end{macrocode}

% Include the main document:
%    \begin{macrocode}
\input{childdoc.def}
\childdocof{cdocsamp}
%    \end{macrocode}

%\iffalse
%</samplechap1|samplechap2>
%\fi
%
%\iffalse
%<*samplechap1>
%\fi
% Some text for chapter 1:
%    \begin{macrocode}
\section{one}
some text in chapter one
%    \end{macrocode}

%\iffalse
%</samplechap1>
%\fi
% Some text for chapter 2:
%\iffalse
%<*samplechap2>
%\fi
%    \begin{macrocode}
\section{two}
more text in chapter two
%    \end{macrocode}

%\iffalse
%</samplechap2>
%\fi
%
% %%%%%%%%%%%%%%%%%%%%%%%%%%%%%%%%%%%%%%
% \paragraph{Part Include Files.}
%
% The include files are called |cdocspt3.tex| and |cdocspt4.tex|.
%
%\iffalse
%<*samplepart3|samplepart4>
%\fi

% Optional override for |\version| flag:
%    \begin{macrocode}
%%\providecommand{\version}{final}
%    \end{macrocode}

% Include the main document:
%    \begin{macrocode}
\input{childdoc.def}
\childdocby{cdocsamp}
%    \end{macrocode}

%\iffalse
%</samplepart3|samplepart4>
%\fi
%
%\iffalse
%<*samplepart3>
%\fi
% Some text for part 3:
%    \begin{macrocode}
some text in part three
%    \end{macrocode}

%\iffalse
%</samplepart3>
%\fi
% Some text for part 4:
%\iffalse
%<*samplepart4>
%\fi
%    \begin{macrocode}
more text in part four
%    \end{macrocode}

%\iffalse
%</samplepart4>
%\fi
%
% %%%%%%%%%%%%%%%%%%%%%%%%%%%%%%%%%%%%%%
% \paragraph{Forwarding for a Complete Draft.}
%
% The following forwarding file |cdocsdrf.tex|
% compiles the main document in draft mode:
%\iffalse
%<*sampledraft>
%\fi
%    \begin{macrocode}
\def\version{draft}
\input{childdoc.def}
\childdocforward{cdocsamp}
%    \end{macrocode}

%\iffalse
%</sampledraft>
%\fi
%
% %%%%%%%%%%%%%%%%%%%%%%%%%%%%%%%%%%%%%%
% \paragraph{Forwarding for Final Version of the Chapters.}
%
% The following forwarding files |cdocsfn1.tex| and |cdocsfn2.tex|
% (with identical content)
% compile the final versions of the child documents
% |cdocsch1.tex| and |cdocsch2.tex|, respectively:
%\iffalse
%<*samplefinal>
%\fi
%    \begin{macrocode}
\def\version{final}
\input{childdoc.def}
\childdocforwardprefix[cdocsamp]{cdocsfn}{cdocsch}
%    \end{macrocode}

%\iffalse
%</samplefinal>
%\fi
%
% %%%%%%%%%%%%%%%%%%%%%%%%%%%%%%%%%%%%%%
% \paragraph{Command Line Processing.}
%
% The following three command lines generate the output files
% |cdocscld|, |cdocscl1| and |cdocscl2|
% which should be identical to
% |cdocsdrf|, |cdocsch1| and |cdocsfn2|, respectively:
% \begin{center}
% \begin{tabular}{l}
% |latex -jobname cdocscld \|\\
% |  "\def\version{draft}\input{childdoc.def}\childdocforward{cdocsamp}"|\\
% |latex -jobname cdocscl1 \|\\
% |  "\input{childdoc.def}\childdocforward[cdocsamp]{cdocsch1}"|\\
% |latex -jobname cdocscl2 \|\\
% |  "\def\version{final}\input{childdoc.def}\childdocforward{cdocsch2}"|
% \end{tabular}
% \end{center}
% Note that the trailing backslash on each first line
% merely continues the input to the second line
% (for convenient cut ant paste).
% Furthermore, the command |latex| can be replaced by any
% of its alternative versions such as |pdflatex|.
%
% %%%%%%%%%%%%%%%%%%%%%%%%%%%%%%%%%%%%%%%%%%%%%%%%%%%%%%%%%%%%%%%%%%%%%%%%%%%%%%
% %%%%%%%%%%%%%%%%%%%%%%%%%%%%%%%%%%%%%%%%%%%%%%%%%%%%%%%%%%%%%%%%%%%%%%%%%%%%%%
% \section{Implementation}
%\iffalse
%<*package>
%\fi
%
% This section describes the definitions file |childdoc.def|.

% The definitions cannot be loaded using |\usepackage| or |\RequirePackage|
% which has a mechanism to prevent loading a style file more than once.
% When loading the definitions by means of |\input|
% multiple instances have to be prevented manually:
%\iffalse
%This code needs to be before the `\ProvidesFile' directive
%which is defined at the beginning of this file.
%Therefore it is also placed there and commented out here.
%</package>
%<*discard>
%\fi
%    \begin{macrocode}
\ifdefined\childdocmain\endinput\fi
%    \end{macrocode}
%\iffalse
%</discard>
%<*package>
%\fi
%
% \macro{\ifchilddoc}
% \macro{\ifchilddocmanual}
% The conditional |\ifchilddoc| tells whether a
% child (true) or main (false) document is being compiled.
% The conditional |\ifchilddocmanual| tells whether
% the |\includeonly| mechanism is used (false) or
% the selection of child files must be performed manually (true).
% The definitions initialise to false:
%    \begin{macrocode}
\newif\ifchilddoc
\newif\ifchilddocmanual
%    \end{macrocode}

% \macro{\childdocname}
% \macro{\childdocjob}
% The macro |\childdocname| stores the name of the main document
% to be compiled. The macro |\childdocjob| stores the name of
% the document on which the \LaTeX{} compiler was originally invoked.
% The content of |\jobname| cannot be compared
% to filenames specified in the source due to different catcodes.
% The following code rescans |\jobname|, stores the result
% in |\childdocname| and saves a copy in |\childdocjob|:
%    \begin{macrocode}
\edef\childdocname{\scantokens\expandafter{\jobname\noexpand}}
\let\childdocjob\childdocname
%    \end{macrocode}

% \macro{\childdocdisable}
% The macro |\childdocdisable| prevents the main file
% from being processed more than once.
% At this stage, the main document command |\childdocmain|
% is assumed to be called once again where it should do nothing.
% Any subsequent call to it should prevent
% a secondary processing of the main document
% It overwrites the forwarding commands
% |\childdocof| and |\childdocforward|
% with empty macros to prevent further inclusions of the main document:
%    \begin{macrocode}
\newcommand{\childdocdisable}
{
  \renewcommand{\childdocmain}[1]{\renewcommand{\childdocmain}[1]{\endinput}}
  \renewcommand{\childdocof}[1]{}
  \renewcommand{\childdocby}[2][]{}
  \renewcommand{\childdocforward}[2][]{}
  \renewcommand{\childdocdisable}{}
}
%    \end{macrocode}

% \macro{\childdocmain}
% The macro |\childdocmain| is to be called at the top of the main file
% with nothing or the main filename (without extension) as argument.
% First, it breaks loops.
% If the argument is not empty and does not match |\childdocname|
% (which is set by the first inclusion of |childdoc.def|),
% |\ifchilddoc| is set to true, |\includeonly| is applied to the child file
% and |\jobname| is set to the main file
% (for proper handling of |.aux| files):
%    \begin{macrocode}
\newcommand{\childdocmain}[1]
{
  \childdocdisable\childdocmain{}
  \if?#1?\else
    \begingroup
      \def\childdoctmp{#1}
      \ifx\childdoctmp\childdocname
        \def\childdoctmp{}
      \else
        \def\childdoctmp
        {
          \childdoctrue
          \includeonly{\childdocname}
          \def\childdocjob{#1}
          \def\jobname{#1}
        }
      \fi
      \expandafter
    \endgroup
    \childdoctmp
  \fi
}
%    \end{macrocode}

% \macro{\childdocof}
% The command |\childdocof| redirects
% compilation to the main file |#1|.
%    \begin{macrocode}
\newcommand{\childdocof}[1]
{
  \childdocdisable
  \childdoctrue
  \includeonly{\childdocname}
  \def\jobname{#1}
  \def\childdocjob{#1}
  \input{#1}
}
%    \end{macrocode}

% \macro{\childdocby}
% The command |\childdocby| ....
%    \begin{macrocode}
\newcommand{\childdocby}[2][]
{
  \childdocdisable
  \childdoctrue
  \childdocmanualtrue
  \if?#1?\else
    \def\jobname{#2}
  \fi
  \def\childdocjob{#2}
  \input{#2}
  \endinput
}
%    \end{macrocode}

% \macro{\childdocforward}
% The command |\childdocforward| redirects
% compilation to the main file or
% (if the optional argument is given) a child file.
% Parameters are set as if the main file
% or a child file starting with |\childdocof| was compiled.
% Then compilation is handed over to the main file:
%    \begin{macrocode}
\newcommand{\childdocforward}[2][]
{
  \begingroup
    \if?#1?
      \def\childdoctmp
      {
        \def\childdocname{#2}
        \def\childdocjob{#2}
        \def\jobname{#2}
        \input{#2}
        \endinput
      }
    \else
      \def\childdoctmp
      {
        \childdocdisable
        \def\childdocname{#2}
        \childdoctrue
        \includeonly{#2}
        \def\childdocjob{#1}
        \def\jobname{#1}
        \input{#1}
        \endinput
      }
    \fi
    \expandafter
  \endgroup
  \childdoctmp
}
%    \end{macrocode}

% \macro{\childdocforwardprefix}
% The command |\childdocforwardprefix| redirects
% compilation to the main or a child file by means of a pattern.
% The prefix |#1| in the current filename is replaced by |#2|
% and the suffix of the current filename is kept
% (it is assumed that the filename does not contain the substring `|~~~|'
% which is used as a delimiter).
% Compilation is handed over to the new file by |\childdocforward|:
%    \begin{macrocode}
\newcommand{\childdocforwardprefix}[3][]
{
  \begingroup
    \def\childdocextract #2##1~~~{\def\childdoctmp{\childdocforward[#1]{#3##1}}}
    \expandafter\childdocextract\childdocname~~~
    \expandafter
  \endgroup
  \childdoctmp
}
%    \end{macrocode}

% \macro{\childdoc}
% The deprecated macro |\childdoc| is a legacy version of |\childdocmain|:
%    \begin{macrocode}
\newcommand{\childdoc}{\childdocmain}
%    \end{macrocode}

% \macro{\childdocredirect}
% The deprecated macro |\childdocredirect| is a legacy version
% of |\childdocforward| and |\childdocforwardprefix|:
%    \begin{macrocode}
\newcommand{\childdocredirect}[2][]
{
  \begingroup
    \if?#1?
      \def\childdoctmp{\childdocforward{#2}}
    \else
      \def\childdoctmp{\childdocforwardprefix{#1}{#2}}
    \fi
    \expandafter
  \endgroup
  \childdoctmp
}
%    \end{macrocode}

%\iffalse
%</package>
%\fi
%
\endinput
|\\
|\childdocby{|\textit{main}|}|\\
\end{tabular}
\end{center}
%
Both forms have slightly different effects as described above.
The main file is prepared as usual, see \secref{sec:include}.

%%%%%%%%%%%%%%%%%%%%%%%%%%%%%%%%%%%%%%%%%%%%%%%%%%%%%%%%%%%%%%%%%%%%%%%%%%%%%%%%
\subsection{Legacy Detection}
\label{sec:detection}

The directive |\childdocmain| in the main file can detect
whether the complete document or merely a child is to be compiled
even without using the directive |\childdocof|.
This method is deprecated because it is less robust
and there is no compelling reason to use it;
it is merely provided for backward compatibility
and it may be removed in future versions.

If the detection mechanism is to be used,
it is mandatory to correctly specify
the filename of the main file as the argument of |\childdocmain|:
%
\begin{center}
\begin{tabular}{l}
|% \iffalse
%
% childdoc.dtx Copyright (C) 2017-2018 Niklas Beisert
%
% This work may be distributed and/or modified under the
% conditions of the LaTeX Project Public License, either version 1.3
% of this license or (at your option) any later version.
% The latest version of this license is in
%   http://www.latex-project.org/lppl.txt
% and version 1.3 or later is part of all distributions of LaTeX
% version 2005/12/01 or later.
%
% This work has the LPPL maintenance status `maintained'.
%
% The Current Maintainer of this work is Niklas Beisert.
%
% This work consists of the files childdoc.dtx and childdoc.ins
% and the derived files childdoc.def and cdocsamp.tex with
% cdocsch1.tex, cdocsch2.tex, cdocsdrf.tex, cdocsfn1.tex, cdocsfn2.tex.
%
%<package>\ifdefined\childdocmain\endinput\fi
%<package>\ProvidesFile{childdoc.def}[2018/12/30 v2.0 child document driver]
%<samplemain>\ProvidesFile{cdocsamp.tex}[2018/12/30 v2.0 sample for childdoc]
%<*driver>
%\ProvidesFile{childdoc.drv}[2018/12/30 v2.0 childdoc reference manual file]
\PassOptionsToClass{10pt,a4paper}{article}
\documentclass{ltxdoc}

\usepackage[margin=35mm]{geometry}
\usepackage{hyperref}
\usepackage{hyperxmp}
\usepackage[usenames]{color}

\hypersetup{colorlinks=true}
\hypersetup{pdfstartview=FitH}
\hypersetup{pdfpagemode=UseNone}
\hypersetup{pdfsource={}}
\hypersetup{pdflang={en-UK}}
\hypersetup{pdfcopyright={Copyright 2017-2018 Niklas Beisert.
  This work may be distributed and/or modified under the
  conditions of the LaTeX Project Public License, either version 1.3
  of this license or (at your option) any later version.}}
\hypersetup{pdflicenseurl={http://www.latex-project.org/lppl.txt}}
\hypersetup{pdfcontactaddress={ETH Zurich, ITP, HIT K,
  Wolfgang-Pauli-Strasse 27}}
\hypersetup{pdfcontactpostcode={8093}}
\hypersetup{pdfcontactcity={Zurich}}
\hypersetup{pdfcontactcountry={Switzerland}}
\hypersetup{pdfcontactemail={nbeisert@itp.phys.ethz.ch}}
\hypersetup{pdfcontacturl={http://people.phys.ethz.ch/\xmptilde nbeisert/}}

\newcommand{\secref}[1]{\hyperref[#1]{section \ref*{#1}}}

\parskip1ex
\parindent0pt
\let\olditemize\itemize
\def\itemize{\olditemize\parskip0pt}

\begin{document}

\title{The \textsf{childdoc} Package}
\hypersetup{pdftitle={The childdoc Package}}
\author{Niklas Beisert\\[2ex]
  Institut f\"ur Theoretische Physik\\
  Eidgen\"ossische Technische Hochschule Z\"urich\\
  Wolfgang-Pauli-Strasse 27, 8093 Z\"urich, Switzerland\\[1ex]
  \href{mailto:nbeisert@itp.phys.ethz.ch}
  {\texttt{nbeisert@itp.phys.ethz.ch}}}
\hypersetup{pdfauthor={Niklas Beisert}}
\hypersetup{pdfsubject={Manual for the LaTeX2e Package childdoc}}
\date{30 December 2018, \textsf{v2.0}}
\maketitle

\begin{abstract}\noindent
\textsf{childdoc} is a \LaTeXe{} package
that enables the direct compilation
of document sections included by |\include|
to individual files.
\end{abstract}

\begingroup
\parskip0ex
\tableofcontents
\endgroup

%%%%%%%%%%%%%%%%%%%%%%%%%%%%%%%%%%%%%%%%%%%%%%%%%%%%%%%%%%%%%%%%%%%%%%%%%%%%%%%%
%%%%%%%%%%%%%%%%%%%%%%%%%%%%%%%%%%%%%%%%%%%%%%%%%%%%%%%%%%%%%%%%%%%%%%%%%%%%%%%%
\section{Introduction}

\LaTeX{} provides a mechanism to structure a large document (such as a book)
into a main file and several child files (containing the chapters)
using the |\include| command.
This mechanism is beneficial for documents
which span hundreds of pages in order to
make the source file(s) more manageable.
Moreover, compilation can be restricted to
selected child files by means of the |\includeonly| command.
The latter feature can be used to reduce the compilation time while editing
(this was significantly more useful in the earlier days of \LaTeX{})
or to generate a smaller document which is easier to navigate.
Another application of |\includeonly| is to generate
documents consisting of selected parts of the complete document.

However, there are a few drawbacks of the plain |\include| mechanism:
\begin{itemize}
\item
The child files cannot be compiled on their own,
they can only be compiled via the main file.
A naive editing environment
(such as a text editor with an option
to have the current file processed by \LaTeX)
may require one to switch to the main file before compiling;
attempting to compile the child file produces errors.
\item
The main file must be modified (each time)
to adjust the |\includeonly| command
to the present needs. This easily leaves the main file in a messy state.
\item
The generated document will always carry the filename
of the main document. This is inconvenient if
several child files are to be compiled and
to be kept for distribution.
\end{itemize}

The present package provides a simple interface
to make child files individually compilable by \LaTeX{}.
Compiling a child file then has the same effect as compiling
the main file with an |\includeonly| command
to select the appropriate child.
Moreover the generated document will carry the name of the child
rather than the main file.
This resolves all three above issues.

This feature is meant to make the editing of books,
thesis documents and lecture notes somewhat more convenient.
However, the package can also be used efficiently for
composing a series of documents (such as exercise sheets)
which are typically distributed individually.
It then assists the author in generating the individual documents
(potentially in different versions)
as well as a document containing the collected series.
Another application is in developing style files
or other kinds of included material
where compilation of the style file could redirect
to a sample or test file.

%%%%%%%%%%%%%%%%%%%%%%%%%%%%%%%%%%%%%%%%%%%%%%%%%%%%%%%%%%%%%%%%%%%%%%%%%%%%%%%%
%%%%%%%%%%%%%%%%%%%%%%%%%%%%%%%%%%%%%%%%%%%%%%%%%%%%%%%%%%%%%%%%%%%%%%%%%%%%%%%%
\section{Usage}

First of all, the package \textsf{childdoc} is \emph{not} a standard
\LaTeXe{} |.sty| style file! Therefore it needs to be invoked in
a non-standard way.

%%%%%%%%%%%%%%%%%%%%%%%%%%%%%%%%%%%%%%%%%%%%%%%%%%%%%%%%%%%%%%%%%%%%%%%%%%%%%%%%
\subsection{Included Files}
\label{sec:include}

%%%%%%%%%%%%%%%%%%%%%%%%%%%%%%%%%%%%%%%%
\DescribeMacro{\childdocmain}
To use the package, add the commands
\begin{center}
\begin{tabular}{l}
|\input{childdoc.def}|\\
|\childdocmain{}|\\
\end{tabular}
\end{center}
at the very top of the main \LaTeX{} file,
in particular \emph{before} the |\documentclass| statement!
The argument of |\childdocmain| should be left empty
(but it must be present).

%%%%%%%%%%%%%%%%%%%%%%%%%%%%%%%%%%%%%%%%
\DescribeMacro{\childdocof}
Furthermore, add the commands
\begin{center}
\begin{tabular}{l}
|\input{childdoc.def}|\\
|\childdocof{|\textit{main}|}|\\
\end{tabular}
\end{center}
at the top of every child file \textit{child}
which is included by |\include{|\textit{child}|}|
from within the main file
(or at least for those files to be compiled individually).
The argument \textit{main} must be the filename of the main file.

There are a couple of
considerations in setting up the main and child documents:

%%%%%%%%%%%%%%%%%%%%%%%%%%%%%%%%%%%%%%%%
\paragraph{Restrictions.}

Please note the following restrictions:
\begin{itemize}
\item
|\childdocmain| must be called with one argument \textit{main}
to ensure compatibility with earlier version of the package.
It must either be empty (|\childdocmain{}|)
or precisely match the filename of the main file in which it is specified.
See \secref{sec:detection} for further information.
\item
The filename \textit{main} must be specified without the |.tex| extension.
\item
The filename \textit{main} is case sensitive
(even in case-insensitive file systems)
due to internal string comparison.
\item
The argument \textit{main} should be fully expanded, it cannot be a macro.
\item
Subdirectories and special characters should be avoided in filenames.
\item
The command |\childdocmain{|\textit{main}|}| must be followed by a whitespace.
It should not be followed immediately by another command
or by a comment mark `|%|'.
This is because the \TeX{} parser reads the token immediately following
the argument of |\childdocmain| and puts it
at the beginning of every child section;
however, a white\-space is ignored.
\end{itemize}

%%%%%%%%%%%%%%%%%%%%%%%%%%%%%%%%%%%%%%%%
\paragraph{Content of Main File.}

It is advisable to place all content in the child files included by |\include|.
Any output contained in the main file will appear in all child documents
unless suppressed manually;
it cannot be suppressed automatically by the |\includeonly| directive
and thus should normally be avoided.
A method to include some content in the main file
by means of conditional processing is described in \secref{sec:conditional}.

%%%%%%%%%%%%%%%%%%%%%%%%%%%%%%%%%%%%%%%%
\paragraph{Page Numbering.}

When only a part of the document is compiled,
the appropriate numbering of pages
(as well as other status parameters)
is determined from the |.aux| files.
The latter contain information from previous passes.
However this information needs to propagate through
all intermediate child documents.
Therefore the page numbering in child documents may well
be inconsistent until the complete document is compiled at least once.

A useful (if unconventional) way to always ensure a consistent
page numbering is to restart the numbering in each child document
and denote the pages by `\textit{child}|.|\textit{page}'
where \textit{child} represents the chapter/section number of the child file.
This can be achieved by the command
|\numberwithin{page}{|\textit{child}|}|
of the \textsf{amsmath} package
where \textit{child} can be |chapter| or |section|
depending on the chosen structuring.
Alternatively, one can modify the macro |\thepage| appropriately
and reset the counter |page| at the start of each child file.

%%%%%%%%%%%%%%%%%%%%%%%%%%%%%%%%%%%%%%%%%%%%%%%%%%%%%%%%%%%%%%%%%%%%%%%%%%%%%%%%
\subsection{Conditional Processing}
\label{sec:conditional}

The package provides a mechanism to compile different versions
of a document. To customise the versions further some conditional processing
can come in handy to distinguish which version is being compiled.
The package provides two macros to describe the compilation context:

%%%%%%%%%%%%%%%%%%%%%%%%%%%%%%%%%%%%%%%%
\DescribeMacro{\ifchilddoc}
The conditional |\ifchilddoc| distinguishes between the compilation of
child documents and the main document:
%
\begin{center}
|\ifchilddoc |\textit{child-code}| |[|\||else |\textit{main-code}]| \||fi|
\end{center}

%%%%%%%%%%%%%%%%%%%%%%%%%%%%%%%%%%%%%%%%
\DescribeMacro{\childdocname}
\DescribeMacro{\childdocjob}
The macro |\childdocname| contains the filename (without extension)
of the main or child file being processed.
Note that |\childdocjob| will always contain the name of the main file.

%%%%%%%%%%%%%%%%%%%%%%%%%%%%%%%%%%%%%%%%
\paragraph{Title Page.}

Conditional processing can be used to include a title or banner page
in the main document when proper precautions are taken.
Importantly, the code in the main file should ensure that the page counter
(as well as other status parameters which are stored in the |.aux| files)
takes the same value after the conditional processing.
Otherwise the page numbers may take divergent values
depending on which part is compiled.

For example, a title page could be declared by:
%
\begin{center}
\begin{tabular}{l}
|\ifchilddoc\||else|\\
|\addtocounter{page}{-1}|\\
\textit{code for title page}\\
|\newpage|\\
|\||fi|
\end{tabular}
\end{center}
%
A banner page for the child documents can be generated by:
%
\begin{center}
\begin{tabular}{l}
|\ifchilddoc|\\
|\addtocounter{page}{-1}|\\
\textit{code for banner page}\\
|\newpage|\\
|\||fi|
\end{tabular}
\end{center}
%
Here one could write a message such as:
\begin{center}
|This is the part \childdocname{} of \childdocjob{}.|
\end{center}

%%%%%%%%%%%%%%%%%%%%%%%%%%%%%%%%%%%%%%%%%%%%%%%%%%%%%%%%%%%%%%%%%%%%%%%%%%%%%%%%
\subsection{Flags}
\label{sec:flags}

The package makes it easy to generate different versions
of the main or child documents.
To this end compilation flags can be defined
and assigned different default values.
They will be particularly useful in conjunction
with the forwarding mechanism described in \secref{sec:forward}.

For example, it may be useful to have a flag |\version|
which can be set to |draft| or |final|.
The document source will contain some conditional code
depending on the value of |\version|.
Suppose further, the flag should default to |final| for the main file
and to |draft| for child files
which is a natural assignment for editing the document.
This is achieved by placing the following code
in the preamble of the main document
(below the |\childdocmain| directive):
%
\begin{center}
\begin{tabular}{l}
|\ifchilddoc|\\
|\providecommand{\version}{draft}|\\
|\||else|\\
|\providecommand{\version}{final}|\\
|\||fi|
\end{tabular}
\end{center}
%
The definition by |\providecommand| makes sure
that previous definitions are not overwritten.
Further statements |\providecommand{\version}{...}|
can thus be added before the above code to override it.

For the main file, one might add a line
(between |\childdocmain| and the above block)
%
\begin{center}
|%\ifchilddoc\||else\providecommand{\version}{draft}\||fi|
\end{center}
%
which can be uncommented to produce a draft version.
Likewise one can add a line to the very top of a child file
(above the |\childdocof{|\textit{main}|}| directive)
%
\begin{center}
|%\providecommand{\version}{final}|
\end{center}
%
which can be uncommented to produce the final version of this child document.

%%%%%%%%%%%%%%%%%%%%%%%%%%%%%%%%%%%%%%%%%%%%%%%%%%%%%%%%%%%%%%%%%%%%%%%%%%%%%%%%
\subsection{Forwarding}
\label{sec:forward}

Different versions of the main or child documents
using compilation flags as described in \secref{sec:flags}
can be (permanently) stored in different files
for convenient compilation, viewing and distribution.
To this end, the package defines a command
to pass on compilation to a different file:

%%%%%%%%%%%%%%%%%%%%%%%%%%%%%%%%%%%%%%%%
\DescribeMacro{\childdocforward}
The command |\childdocforward| redirects processing to
another source file:
%
\begin{center}
\begin{tabular}{l}
|\input{childdoc.def}|\\
|\childdocforward[|\textit{main}|]{|\textit{dest}|}|\\
\end{tabular}
\end{center}
%
The argument \textit{dest} is the destination file
(without extension).
It should be the main file or one of the child files.
Note that further \textsf{childdoc} directives
such as |\childdocof| and |\childdocforward|
in the indicated file will be processed in this form.
The optional argument \textit{main}
passes on directly to the main file \textit{main}
while pretending to compile the child \textit{dest}.
This form behaves as if \textit{dest}
issues |\childdocof{|\textit{main}|}| right away,
and no further \textsf{childdoc} directives will be processed.

%%%%%%%%%%%%%%%%%%%%%%%%%%%%%%%%%%%%%%%%
\DescribeMacro{\...prefix}
In the alternative form |\childdocforwardprefix|,
%
\begin{center}
\begin{tabular}{l}
|\input{childdoc.def}|\\
|\childdocforwardprefix[|\textit{main}|]{|\textit{prefix}|}{|\textit{dest}|}|
\end{tabular}
\end{center}
%
the destination file is determined by a pattern
depending on the current file:
To make this work, the current file must be called
`{\textit{prefix}\hspace{0.2em}\textit{suffix}}'
with \textit{prefix} matching precisely the argument.
Processing is then passed on to the file
`{\textit{dest}\hspace{0.2em}\textit{suffix}}'.
Surely, the same effect is achieved by
directly specifying the
argument `{\textit{dest}\hspace{0.2em}\textit{suffix}}'
in the first form.
However, that requires to set up a different file
for each child. With the alternative form of the command
all these files can have exactly the same content
which simplifies setting them up and maintaining them.

For example, the following file |draft.tex|
with a compilation flag |\version| as described in \secref{sec:flags}
compiles the main document as a draft:
%
\begin{center}
\begin{tabular}{l}
|\def\version{draft}|\\
|\input{childdoc.def}|\\
|\childdocforward{|\textit{main}|}|
\end{tabular}
\end{center}
%
Likewise, the following files |final|\textit{nn}|.tex|
compile the final version of the child document
|child|\textit{nn}|.tex|:
%
\begin{center}
\begin{tabular}{l}
|\def\version{final}|\\
|\input{childdoc.def}|\\
|\childdocforwardprefix{final}{child}|
\end{tabular}
\end{center}
%

Note that when several versions of a main file and/or of each child file
are to be generated, it may be convenient to set up a |Makefile| or
shell script to automatise the process.

%%%%%%%%%%%%%%%%%%%%%%%%%%%%%%%%%%%%%%%%%%%%%%%%%%%%%%%%%%%%%%%%%%%%%%%%%%%%%%%%
\subsection{Command Line Processing}
\label{sec:commandline}

The effect of redirection files can also be achieved by invoking
the \LaTeX{} compiler with a more elaborate command line.
Most conveniently this should be done as part
of a shell script or a |Makefile|.

When using \textsf{childdoc} in the main file, the following
command lines effectively perform a redirection
(note that depending on the shell being used,
backslashes may have to be doubled: `|\|' $\to$ `|\\|'):
%
\begin{center}
|... -jobname "|\textit{target}|" |\\|"|[\textit{flags}]%
|\input{childdoc.def}\childdocforward[|\textit{main}|]{|\textit{dest}|}"|
\end{center}
%
Here \textit{target} is the name of the output file,
\textit{main} is the name of the main file
and \textit{dest} is the name of the main or child file to be processed
(all filenames without extensions).
The optional argument \textit{main} can be omitted
if \textit{main} matches \textit{dest}.
Optionally, compilation \textit{flags} can be defined via |\def| commands.
This command line makes the \TeX{} engine believe
it is compiling the file \textit{target}
whose content is specified as the latter parameter.
The provided code then forwards the processing to
\textit{main} or \textit{dest} as described in \secref{sec:forward}.

%%%%%%%%%%%%%%%%%%%%%%%%%%%%%%%%%%%%%%%%%%%%%%%%%%%%%%%%%%%%%%%%%%%%%%%%%%%%%%%%
\subsection{Include by Input}
\label{sec:input}

Including child documents by |\include| has some restrictions by design.
Most notably, the content of a child document always occupies
its own set of pages; pages cannot be shared between child documents.
Usually, this behaviour makes perfect sense
because each child document contain an essential part of the document.
However, in some situations it may be desirable to compose
a document from a collection of parts
without having mandatory page breaks between then.
For this case, the package
provides a mechanism to include parts
by |\input| which can also be processed individually.
However, by construction this mechanism
requires manual handling of the content to be output.

%%%%%%%%%%%%%%%%%%%%%%%%%%%%%%%%%%%%%%%%
\DescribeMacro{\ifchilddocmanual}
The main file should be prepared as usual, see \secref{sec:include}.
However, the document body must make a distinction
between processing of an individual part and of the main document, e.g.:
%
\begin{center}
\begin{tabular}{l}
|\ifchilddocmanual|\\
|\input{\childdocname}|\\
|\||else|\\
\textit{document body with }|\input{|\textit{part}|}|\\
|\||fi|
\end{tabular}
\end{center}
%
The conditional |\ifchilddocmanual| is true whenever
a part to be included by |\input| is being compiled,
and the name of the part is stored in |\childdocname|.

%%%%%%%%%%%%%%%%%%%%%%%%%%%%%%%%%%%%%%%%
\DescribeMacro{\childdocby}
Each part to be included by |\input| should start with:
%
\begin{center}
\begin{tabular}{l}
|\input{childdoc.def}|\\
|\childdocby{|\textit{main}|}|\\
\end{tabular}
\end{center}
%
The directive |\childdocby| is similar to |\childdocof|
described in \secref{sec:include},
but the subsequent selection of content must be done manually.
To that end, both |\ifchilddoc| and |\ifchilddocmanual|
will be true upon processing of a part,
and the name of the part is stored in |\childdocname|.
Note that |\jobname| will be set to the filename of the current part
so that each part receives an individual |.aux| file
that does not interfere with the |.aux| file(s) of the main document.
This behaviour can be altered by the alternative form
|\childdocby[*]{|\textit{main}|}| (with a non-empty optional argument)
which uses the |.aux| file of the main document
by setting |\jobname| to \textit{main}.

%%%%%%%%%%%%%%%%%%%%%%%%%%%%%%%%%%%%%%%%%%%%%%%%%%%%%%%%%%%%%%%%%%%%%%%%%%%%%%%%
\subsection{Driver Development}
\label{sec:driver}

The \textsf{childdoc} mechanism can also be use for the development
of definition files such as \LaTeX{} styles or classes.
This case differs from the above setup with multiple parts
included by |\include| in that no |\includeonly| should be invoked.
This can be achieved by starting the include file
(before |\ProvidesPackage|) with:
%
\begin{center}
\begin{tabular}{l}
|\input{childdoc.def}|\\
|\childdocforward{|\textit{main}|}|\\
\end{tabular}
\end{center}
%
or alternatively with:
%
\begin{center}
\begin{tabular}{l}
|\input{childdoc.def}|\\
|\childdocby{|\textit{main}|}|\\
\end{tabular}
\end{center}
%
Both forms have slightly different effects as described above.
The main file is prepared as usual, see \secref{sec:include}.

%%%%%%%%%%%%%%%%%%%%%%%%%%%%%%%%%%%%%%%%%%%%%%%%%%%%%%%%%%%%%%%%%%%%%%%%%%%%%%%%
\subsection{Legacy Detection}
\label{sec:detection}

The directive |\childdocmain| in the main file can detect
whether the complete document or merely a child is to be compiled
even without using the directive |\childdocof|.
This method is deprecated because it is less robust
and there is no compelling reason to use it;
it is merely provided for backward compatibility
and it may be removed in future versions.

If the detection mechanism is to be used,
it is mandatory to correctly specify
the filename of the main file as the argument of |\childdocmain|:
%
\begin{center}
\begin{tabular}{l}
|\input{childdoc.def}|\\
|\childdocmain{|\textit{main}|}|\\
\end{tabular}
\end{center}
%
If |\jobname| does not match the argument \textit{main} of |\childdocmain|,
it is assumed that |\jobname| points to the child file to be compiled.
When using |\childdocmain| with the main file specified as argument,
it suffices to start a child file
with just |\input{|\textit{main}|}|
without loading of the package and using |\childdocof|.
If instead all processing is done
with the appropriate \textsf{childdoc} directives,
the argument of \textit{main} of |\childdocmain| can be empty.

An alternative version of the command line processing described
in \secref{sec:commandline} using the detection mechanism reads:
%
\begin{center}
|... -jobname "|\textit{target}|" "|[\textit{flags}]%
[|\def\jobname{|\textit{dest}|}|]|\input{|\textit{main}|}"|
\end{center}

%%%%%%%%%%%%%%%%%%%%%%%%%%%%%%%%%%%%%%%%%%%%%%%%%%%%%%%%%%%%%%%%%%%%%%%%%%%%%%%%
\subsection{Manual Code}
\label{sec:manual}

In case one cannot be certain whether the definitions file |childdoc.def|
is installed on the target \TeX{} distribution
and one prefers not to ship it,
it is conceivable to paste a few relevant commands into the sources.

To that end, drop all statements |\input{childdoc.def}|
and perform the replacements as outlined below.
Instead of |\childdocmain{|\textit{main}|}| add the following code
to the top of the main file:
%
\begin{center}
\begin{tabular}{l}
|\||ifdefined\childdocname\endinput\||fi\newif\ifchilddoc|\\
|\edef\childdocname{\scantokens\expandafter{\jobname\noexpand}}|\\
|\def\childdocmain{|\textit{main}|}\||ifx\childdocmain\childdocname\||else|\\
|\childdoctrue\includeonly{\childdocname}\let\jobname\childdocmain\||fi|\\
\end{tabular}
\end{center}
%
Instead of |\childdocof{|\textit{main}|}| just include the main file
at the top of each child file:
%
\begin{center}
|\input{|\textit{main}|}|
\end{center}
%
A simple redirection |\childdocforward{|\textit{dest}|}| is achieved by:
%
\begin{center}
|\def\jobname{|\textit{dest}|}\input{\jobname}|
\end{center}
%
The redirection with prefix
|\childdocforwardprefix[|\textit{prefix}|]{|\textit{dest}|}|
is accomplished by:
%
\begin{center}
\begin{tabular}{l}
|{\edef\jobname{\scantokens\expandafter{\jobname\noexpand}}|\\
|\def\redirectjob |\textit{prefix}|#1~~~{\gdef\jobname{|\textit{dest}|#1}}|\\
|\expandafter\redirectjob\jobname~~~}\input{\jobname}|
\end{tabular}
\end{center}

In an alternative approach,
child documents can be compiled by a specific command line
without additional code or specific definitions:
%
\begin{center}
|... -jobname "|\textit{target}|" "|[\textit{flags}]%
|\includeonly{|\textit{dest}|}\input{|\textit{main}|}"|
\end{center}
%

%%%%%%%%%%%%%%%%%%%%%%%%%%%%%%%%%%%%%%%%%%%%%%%%%%%%%%%%%%%%%%%%%%%%%%%%%%%%%%%%
%%%%%%%%%%%%%%%%%%%%%%%%%%%%%%%%%%%%%%%%%%%%%%%%%%%%%%%%%%%%%%%%%%%%%%%%%%%%%%%%
\section{Information}

%%%%%%%%%%%%%%%%%%%%%%%%%%%%%%%%%%%%%%%%%%%%%%%%%%%%%%%%%%%%%%%%%%%%%%%%%%%%%%%%
\subsection{Copyright}

Copyright \copyright{} 2017--2018 Niklas Beisert

This work may be distributed and/or modified under the
conditions of the \LaTeX{} Project Public License, either version 1.3
of this license or (at your option) any later version.
The latest version of this license is in
  \url{http://www.latex-project.org/lppl.txt}
and version 1.3 or later is part of all distributions of \LaTeX{}
version 2005/12/01 or later.

This work has the LPPL maintenance status `maintained'.

The Current Maintainer of this work is Niklas Beisert.

This work consists of the files |README.txt|, |childdoc.ins| and |childdoc.dtx|
as well as the derived files |childdoc.def|, |cdocsamp.tex|
with |cdocsch1.tex|, |cdocsch2.tex|, |cdocspt3.tex|, |cdocspt4.tex|,
|cdocsdrf.tex|, |cdocsfn1.tex|, |cdocsfn2.tex|
as well as |childdoc.pdf|.

%%%%%%%%%%%%%%%%%%%%%%%%%%%%%%%%%%%%%%%%%%%%%%%%%%%%%%%%%%%%%%%%%%%%%%%%%%%%%%%%
\subsection{Files and Installation}

The package consists of the files:
%
\begin{center}
\begin{tabular}{ll}
    |README.txt|   & readme file \\
    |childdoc.ins| & installation file \\
    |childdoc.dtx| & source file \\
    |childdoc.def| & definition file \\
    |cdocsamp.tex| & sample main file \\
    |cdocsch1.tex| & sample include file \\
    |cdocsch2.tex| & sample include file \\
    |cdocspt3.tex| & sample part file \\
    |cdocspt4.tex| & sample part file \\
    |cdocsdrf.tex| & sample redirection file \\
    |cdocsfn1.tex| & sample redirection file \\
    |cdocsfn2.tex| & sample redirection file \\
    |childdoc.pdf| & manual
\end{tabular}
\end{center}
%
The distribution consists of the files
|README.txt|, |childdoc.ins| and |childdoc.dtx|.
%
\begin{itemize}
\item
Run (pdf)\LaTeX{} on |childdoc.dtx|
to compile the manual |childdoc.pdf| (this file).
\item
Run \LaTeX{} on |childdoc.ins| to create the definitions file |childdoc.def|
and the sample |cdocsamp.tex| with include files
|cdocsch1.tex|, |cdocsch2.tex|, |cdocspt3.tex|, |cdocspt4.tex|,
|cdocsdrf.tex|, |cdocsfn1.tex|, |cdocsfn2.tex|.
Then copy the file |childdoc.def| to an appropriate directory of your \LaTeX{}
distribution, e.g.\ \textit{texmf-root}|/tex/latex/childdoc|.
\end{itemize}

%%%%%%%%%%%%%%%%%%%%%%%%%%%%%%%%%%%%%%%%%%%%%%%%%%%%%%%%%%%%%%%%%%%%%%%%%%%%%%%%
\subsection{Related CTAN Packages}

There are several other packages which offer a similar functionality:
%
\begin{itemize}
\item
The packages
\href{http://ctan.org/pkg/docmute}{\textsf{docmute}},
\href{http://ctan.org/pkg/includex}{\textsf{includex}} and
\href{http://ctan.org/pkg/standalone}{\textsf{standalone}}
provide commands to include only the document body of
a child file thus allowing both files to be compiled individually.
\item
The packages \href{http://ctan.org/pkg/subdocs}{\textsf{subdocs}}
and \href{http://ctan.org/pkg/subfiles}{\textsf{subfiles}}
provide structures in which the main and child documents can be
encapsulated and allowing them to be compiled individually.
The inclusion mechanism is different from the conventional |\include|.
\item
The package \href{http://ctan.org/pkg/combine}{\textsf{combine}}
is an elaborate solution to combine several documents into one.
\end{itemize}
%
See also the CTAN topic \href{http://ctan.org/topic/subdocs}{\textsf{subdocs}}
for further related packages.
The present package differs from the above solutions in that
a document structure constructed with the conventional |\include| mechanism
just needs two extra commands at the top of every file
such that all constituent files can be compiled individually.

%%%%%%%%%%%%%%%%%%%%%%%%%%%%%%%%%%%%%%%%%%%%%%%%%%%%%%%%%%%%%%%%%%%%%%%%%%%%%%%%
%\subsection{Feature Suggestions}
%
%The following is a list of features which may be useful for future
%versions of this package:
%%
%\begin{itemize}
%\item
%\ldots
%\end{itemize}

%%%%%%%%%%%%%%%%%%%%%%%%%%%%%%%%%%%%%%%%%%%%%%%%%%%%%%%%%%%%%%%%%%%%%%%%%%%%%%%%
\subsection{Revision History}

%%%%%%%%%%%%%%%%%%%%%%%%%%%%%%%%%%%%%%%%
\paragraph{v2.0:} 2018/12/30

\begin{itemize}
\item
immediate forward processing
\item
added |\childdocby| mechanism
\item
manual restructured
\end{itemize}

%%%%%%%%%%%%%%%%%%%%%%%%%%%%%%%%%%%%%%%%
\paragraph{v1.6:} 2018/01/17

\begin{itemize}
\item
application for development of include files
\item
corrections to manual
\end{itemize}

%%%%%%%%%%%%%%%%%%%%%%%%%%%%%%%%%%%%%%%%
\paragraph{v1.5:} 2017/05/21

\begin{itemize}
\item
more complete structuring introduced
\item
|\childdocof| introduced
\item
|\childdoc| renamed to |\childdocmain|
\item
|\childredirect| renamed to |\childdocforward| and |\childdocforwardprefix|
and functionality expanded
\end{itemize}

%%%%%%%%%%%%%%%%%%%%%%%%%%%%%%%%%%%%%%%%
\paragraph{v1.0:} 2017/04/27

\begin{itemize}
\item
manual and install package
\item
first version published on CTAN
\end{itemize}

%%%%%%%%%%%%%%%%%%%%%%%%%%%%%%%%%%%%%%%%
\paragraph{v0.6:} 2017/04/26

\begin{itemize}
\item
redirection mechanism added
\end{itemize}

%%%%%%%%%%%%%%%%%%%%%%%%%%%%%%%%%%%%%%%%
\paragraph{v0.5:} 2017/04/26

\begin{itemize}
\item
functionality in definition file
\end{itemize}


%%%%%%%%%%%%%%%%%%%%%%%%%%%%%%%%%%%%%%%%%%%%%%%%%%%%%%%%%%%%%%%%%%%%%%%%%%%%%%%%
%%%%%%%%%%%%%%%%%%%%%%%%%%%%%%%%%%%%%%%%%%%%%%%%%%%%%%%%%%%%%%%%%%%%%%%%%%%%%%%%
%%%%%%%%%%%%%%%%%%%%%%%%%%%%%%%%%%%%%%%%%%%%%%%%%%%%%%%%%%%%%%%%%%%%%%%%%%%%%%%%
\appendix

\settowidth\MacroIndent{\rmfamily\scriptsize 000\ }

 \DocInput{childdoc.dtx}

\end{document}
%</driver>
% \fi
%
% %%%%%%%%%%%%%%%%%%%%%%%%%%%%%%%%%%%%%%%%%%%%%%%%%%%%%%%%%%%%%%%%%%%%%%%%%%%%%%
% %%%%%%%%%%%%%%%%%%%%%%%%%%%%%%%%%%%%%%%%%%%%%%%%%%%%%%%%%%%%%%%%%%%%%%%%%%%%%%
% \section{Sample}
%\iffalse
%<*samplemain>
%\fi
%
% The following presents a sample document
% with two chapters, two parts, a title page,
% a compile flag as well as three forwarding files to set the flag.
% It consists of eight |.tex| files:
% \begin{center}
% \begin{tabular}{ll}
% |cdocsamp.tex|&main file\\
% |cdocsch1.tex|&include file for chapter 1\\
% |cdocsch2.tex|&include file for chapter 2\\
% |cdocspt3.tex|&include file for part 3\\
% |cdocspt4.tex|&include file for part 4\\
% |cdocsdrf.tex|&forwarding file for main file in draft mode\\
% |cdocsfi1.tex|&forwarding file for final version of chapter 1\\
% |cdocsfi2.tex|&forwarding file for final version of chapter 2\\
% \end{tabular}
% \end{center}
% Each of the eight files can be compiled directly by the \LaTeX{} compiler.
%
% %%%%%%%%%%%%%%%%%%%%%%%%%%%%%%%%%%%%%%
% \paragraph{Main File.}
%
% The main file is called |cdocsamp.tex|.
%
% Load the \textsf{childdoc} definitions and
% declare the filename for the main document:
%    \begin{macrocode}
\input{childdoc.def}
\childdocmain{}
%    \end{macrocode}

% Optional override for |\version| flag:
%    \begin{macrocode}
%%\ifchilddoc\else\providecommand{\version}{draft}\fi
%    \end{macrocode}

% Define the default values for the |\version| flag
% (|final| for the main file and |draft| for childs):
%    \begin{macrocode}
\ifchilddoc
\providecommand{\version}{draft}
\else
\providecommand{\version}{final}
\fi
%    \end{macrocode}

% Load the standard document class:
%    \begin{macrocode}
\documentclass[12pt]{article}
%    \end{macrocode}

% Start the document body:
%    \begin{macrocode}
\begin{document}
%    \end{macrocode}

% Declare a title page.
% Print title, part of document being processed and version flag:
%    \begin{macrocode}
\addtocounter{page}{-1}
\begin{center}
{\LARGE\bfseries{}childdoc example\par}
\vspace{1cm}
\ifchilddoc
\ifchilddocmanual part\else chapter\fi:
`\childdocname' of `\childdocjob'\par
\else
main document: `\childdocjob'\par
\fi
version: \version\par
\end{center}
\newpage
%    \end{macrocode}

% Manually include selected file,
% otherwise process as usual:
%    \begin{macrocode}
\ifchilddocmanual
\section*{part `\childdocname'}
\input{\childdocname}
\else
%    \end{macrocode}

% Include the two chapters:
%    \begin{macrocode}
\include{cdocsch1}
\include{cdocsch2}
%    \end{macrocode}

% Include the two parts unless only chapters should be displayed:
%    \begin{macrocode}
\ifchilddoc\else
\section{part three}
\input{cdocspt3}
\section{part four}
\input{cdocspt4}
\fi
%    \end{macrocode}

% Process as usual until here:
%    \begin{macrocode}
\fi
%    \end{macrocode}

% End of document body:
%    \begin{macrocode}
\end{document}
%    \end{macrocode}
%\iffalse
%</samplemain>
%\fi
%
% %%%%%%%%%%%%%%%%%%%%%%%%%%%%%%%%%%%%%%
% \paragraph{Chapter Include Files.}
%
% The include files are called |cdocsch1.tex| and |cdocsch2.tex|.
%
%\iffalse
%<*samplechap1|samplechap2>
%\fi

% Optional override for |\version| flag:
%    \begin{macrocode}
%%\providecommand{\version}{final}
%    \end{macrocode}

% Include the main document:
%    \begin{macrocode}
\input{childdoc.def}
\childdocof{cdocsamp}
%    \end{macrocode}

%\iffalse
%</samplechap1|samplechap2>
%\fi
%
%\iffalse
%<*samplechap1>
%\fi
% Some text for chapter 1:
%    \begin{macrocode}
\section{one}
some text in chapter one
%    \end{macrocode}

%\iffalse
%</samplechap1>
%\fi
% Some text for chapter 2:
%\iffalse
%<*samplechap2>
%\fi
%    \begin{macrocode}
\section{two}
more text in chapter two
%    \end{macrocode}

%\iffalse
%</samplechap2>
%\fi
%
% %%%%%%%%%%%%%%%%%%%%%%%%%%%%%%%%%%%%%%
% \paragraph{Part Include Files.}
%
% The include files are called |cdocspt3.tex| and |cdocspt4.tex|.
%
%\iffalse
%<*samplepart3|samplepart4>
%\fi

% Optional override for |\version| flag:
%    \begin{macrocode}
%%\providecommand{\version}{final}
%    \end{macrocode}

% Include the main document:
%    \begin{macrocode}
\input{childdoc.def}
\childdocby{cdocsamp}
%    \end{macrocode}

%\iffalse
%</samplepart3|samplepart4>
%\fi
%
%\iffalse
%<*samplepart3>
%\fi
% Some text for part 3:
%    \begin{macrocode}
some text in part three
%    \end{macrocode}

%\iffalse
%</samplepart3>
%\fi
% Some text for part 4:
%\iffalse
%<*samplepart4>
%\fi
%    \begin{macrocode}
more text in part four
%    \end{macrocode}

%\iffalse
%</samplepart4>
%\fi
%
% %%%%%%%%%%%%%%%%%%%%%%%%%%%%%%%%%%%%%%
% \paragraph{Forwarding for a Complete Draft.}
%
% The following forwarding file |cdocsdrf.tex|
% compiles the main document in draft mode:
%\iffalse
%<*sampledraft>
%\fi
%    \begin{macrocode}
\def\version{draft}
\input{childdoc.def}
\childdocforward{cdocsamp}
%    \end{macrocode}

%\iffalse
%</sampledraft>
%\fi
%
% %%%%%%%%%%%%%%%%%%%%%%%%%%%%%%%%%%%%%%
% \paragraph{Forwarding for Final Version of the Chapters.}
%
% The following forwarding files |cdocsfn1.tex| and |cdocsfn2.tex|
% (with identical content)
% compile the final versions of the child documents
% |cdocsch1.tex| and |cdocsch2.tex|, respectively:
%\iffalse
%<*samplefinal>
%\fi
%    \begin{macrocode}
\def\version{final}
\input{childdoc.def}
\childdocforwardprefix[cdocsamp]{cdocsfn}{cdocsch}
%    \end{macrocode}

%\iffalse
%</samplefinal>
%\fi
%
% %%%%%%%%%%%%%%%%%%%%%%%%%%%%%%%%%%%%%%
% \paragraph{Command Line Processing.}
%
% The following three command lines generate the output files
% |cdocscld|, |cdocscl1| and |cdocscl2|
% which should be identical to
% |cdocsdrf|, |cdocsch1| and |cdocsfn2|, respectively:
% \begin{center}
% \begin{tabular}{l}
% |latex -jobname cdocscld \|\\
% |  "\def\version{draft}\input{childdoc.def}\childdocforward{cdocsamp}"|\\
% |latex -jobname cdocscl1 \|\\
% |  "\input{childdoc.def}\childdocforward[cdocsamp]{cdocsch1}"|\\
% |latex -jobname cdocscl2 \|\\
% |  "\def\version{final}\input{childdoc.def}\childdocforward{cdocsch2}"|
% \end{tabular}
% \end{center}
% Note that the trailing backslash on each first line
% merely continues the input to the second line
% (for convenient cut ant paste).
% Furthermore, the command |latex| can be replaced by any
% of its alternative versions such as |pdflatex|.
%
% %%%%%%%%%%%%%%%%%%%%%%%%%%%%%%%%%%%%%%%%%%%%%%%%%%%%%%%%%%%%%%%%%%%%%%%%%%%%%%
% %%%%%%%%%%%%%%%%%%%%%%%%%%%%%%%%%%%%%%%%%%%%%%%%%%%%%%%%%%%%%%%%%%%%%%%%%%%%%%
% \section{Implementation}
%\iffalse
%<*package>
%\fi
%
% This section describes the definitions file |childdoc.def|.

% The definitions cannot be loaded using |\usepackage| or |\RequirePackage|
% which has a mechanism to prevent loading a style file more than once.
% When loading the definitions by means of |\input|
% multiple instances have to be prevented manually:
%\iffalse
%This code needs to be before the `\ProvidesFile' directive
%which is defined at the beginning of this file.
%Therefore it is also placed there and commented out here.
%</package>
%<*discard>
%\fi
%    \begin{macrocode}
\ifdefined\childdocmain\endinput\fi
%    \end{macrocode}
%\iffalse
%</discard>
%<*package>
%\fi
%
% \macro{\ifchilddoc}
% \macro{\ifchilddocmanual}
% The conditional |\ifchilddoc| tells whether a
% child (true) or main (false) document is being compiled.
% The conditional |\ifchilddocmanual| tells whether
% the |\includeonly| mechanism is used (false) or
% the selection of child files must be performed manually (true).
% The definitions initialise to false:
%    \begin{macrocode}
\newif\ifchilddoc
\newif\ifchilddocmanual
%    \end{macrocode}

% \macro{\childdocname}
% \macro{\childdocjob}
% The macro |\childdocname| stores the name of the main document
% to be compiled. The macro |\childdocjob| stores the name of
% the document on which the \LaTeX{} compiler was originally invoked.
% The content of |\jobname| cannot be compared
% to filenames specified in the source due to different catcodes.
% The following code rescans |\jobname|, stores the result
% in |\childdocname| and saves a copy in |\childdocjob|:
%    \begin{macrocode}
\edef\childdocname{\scantokens\expandafter{\jobname\noexpand}}
\let\childdocjob\childdocname
%    \end{macrocode}

% \macro{\childdocdisable}
% The macro |\childdocdisable| prevents the main file
% from being processed more than once.
% At this stage, the main document command |\childdocmain|
% is assumed to be called once again where it should do nothing.
% Any subsequent call to it should prevent
% a secondary processing of the main document
% It overwrites the forwarding commands
% |\childdocof| and |\childdocforward|
% with empty macros to prevent further inclusions of the main document:
%    \begin{macrocode}
\newcommand{\childdocdisable}
{
  \renewcommand{\childdocmain}[1]{\renewcommand{\childdocmain}[1]{\endinput}}
  \renewcommand{\childdocof}[1]{}
  \renewcommand{\childdocby}[2][]{}
  \renewcommand{\childdocforward}[2][]{}
  \renewcommand{\childdocdisable}{}
}
%    \end{macrocode}

% \macro{\childdocmain}
% The macro |\childdocmain| is to be called at the top of the main file
% with nothing or the main filename (without extension) as argument.
% First, it breaks loops.
% If the argument is not empty and does not match |\childdocname|
% (which is set by the first inclusion of |childdoc.def|),
% |\ifchilddoc| is set to true, |\includeonly| is applied to the child file
% and |\jobname| is set to the main file
% (for proper handling of |.aux| files):
%    \begin{macrocode}
\newcommand{\childdocmain}[1]
{
  \childdocdisable\childdocmain{}
  \if?#1?\else
    \begingroup
      \def\childdoctmp{#1}
      \ifx\childdoctmp\childdocname
        \def\childdoctmp{}
      \else
        \def\childdoctmp
        {
          \childdoctrue
          \includeonly{\childdocname}
          \def\childdocjob{#1}
          \def\jobname{#1}
        }
      \fi
      \expandafter
    \endgroup
    \childdoctmp
  \fi
}
%    \end{macrocode}

% \macro{\childdocof}
% The command |\childdocof| redirects
% compilation to the main file |#1|.
%    \begin{macrocode}
\newcommand{\childdocof}[1]
{
  \childdocdisable
  \childdoctrue
  \includeonly{\childdocname}
  \def\jobname{#1}
  \def\childdocjob{#1}
  \input{#1}
}
%    \end{macrocode}

% \macro{\childdocby}
% The command |\childdocby| ....
%    \begin{macrocode}
\newcommand{\childdocby}[2][]
{
  \childdocdisable
  \childdoctrue
  \childdocmanualtrue
  \if?#1?\else
    \def\jobname{#2}
  \fi
  \def\childdocjob{#2}
  \input{#2}
  \endinput
}
%    \end{macrocode}

% \macro{\childdocforward}
% The command |\childdocforward| redirects
% compilation to the main file or
% (if the optional argument is given) a child file.
% Parameters are set as if the main file
% or a child file starting with |\childdocof| was compiled.
% Then compilation is handed over to the main file:
%    \begin{macrocode}
\newcommand{\childdocforward}[2][]
{
  \begingroup
    \if?#1?
      \def\childdoctmp
      {
        \def\childdocname{#2}
        \def\childdocjob{#2}
        \def\jobname{#2}
        \input{#2}
        \endinput
      }
    \else
      \def\childdoctmp
      {
        \childdocdisable
        \def\childdocname{#2}
        \childdoctrue
        \includeonly{#2}
        \def\childdocjob{#1}
        \def\jobname{#1}
        \input{#1}
        \endinput
      }
    \fi
    \expandafter
  \endgroup
  \childdoctmp
}
%    \end{macrocode}

% \macro{\childdocforwardprefix}
% The command |\childdocforwardprefix| redirects
% compilation to the main or a child file by means of a pattern.
% The prefix |#1| in the current filename is replaced by |#2|
% and the suffix of the current filename is kept
% (it is assumed that the filename does not contain the substring `|~~~|'
% which is used as a delimiter).
% Compilation is handed over to the new file by |\childdocforward|:
%    \begin{macrocode}
\newcommand{\childdocforwardprefix}[3][]
{
  \begingroup
    \def\childdocextract #2##1~~~{\def\childdoctmp{\childdocforward[#1]{#3##1}}}
    \expandafter\childdocextract\childdocname~~~
    \expandafter
  \endgroup
  \childdoctmp
}
%    \end{macrocode}

% \macro{\childdoc}
% The deprecated macro |\childdoc| is a legacy version of |\childdocmain|:
%    \begin{macrocode}
\newcommand{\childdoc}{\childdocmain}
%    \end{macrocode}

% \macro{\childdocredirect}
% The deprecated macro |\childdocredirect| is a legacy version
% of |\childdocforward| and |\childdocforwardprefix|:
%    \begin{macrocode}
\newcommand{\childdocredirect}[2][]
{
  \begingroup
    \if?#1?
      \def\childdoctmp{\childdocforward{#2}}
    \else
      \def\childdoctmp{\childdocforwardprefix{#1}{#2}}
    \fi
    \expandafter
  \endgroup
  \childdoctmp
}
%    \end{macrocode}

%\iffalse
%</package>
%\fi
%
\endinput
|\\
|\childdocmain{|\textit{main}|}|\\
\end{tabular}
\end{center}
%
If |\jobname| does not match the argument \textit{main} of |\childdocmain|,
it is assumed that |\jobname| points to the child file to be compiled.
When using |\childdocmain| with the main file specified as argument,
it suffices to start a child file
with just |\input{|\textit{main}|}|
without loading of the package and using |\childdocof|.
If instead all processing is done
with the appropriate \textsf{childdoc} directives,
the argument of \textit{main} of |\childdocmain| can be empty.

An alternative version of the command line processing described
in \secref{sec:commandline} using the detection mechanism reads:
%
\begin{center}
|... -jobname "|\textit{target}|" "|[\textit{flags}]%
[|\def\jobname{|\textit{dest}|}|]|\input{|\textit{main}|}"|
\end{center}

%%%%%%%%%%%%%%%%%%%%%%%%%%%%%%%%%%%%%%%%%%%%%%%%%%%%%%%%%%%%%%%%%%%%%%%%%%%%%%%%
\subsection{Manual Code}
\label{sec:manual}

In case one cannot be certain whether the definitions file |childdoc.def|
is installed on the target \TeX{} distribution
and one prefers not to ship it,
it is conceivable to paste a few relevant commands into the sources.

To that end, drop all statements |% \iffalse
%
% childdoc.dtx Copyright (C) 2017-2018 Niklas Beisert
%
% This work may be distributed and/or modified under the
% conditions of the LaTeX Project Public License, either version 1.3
% of this license or (at your option) any later version.
% The latest version of this license is in
%   http://www.latex-project.org/lppl.txt
% and version 1.3 or later is part of all distributions of LaTeX
% version 2005/12/01 or later.
%
% This work has the LPPL maintenance status `maintained'.
%
% The Current Maintainer of this work is Niklas Beisert.
%
% This work consists of the files childdoc.dtx and childdoc.ins
% and the derived files childdoc.def and cdocsamp.tex with
% cdocsch1.tex, cdocsch2.tex, cdocsdrf.tex, cdocsfn1.tex, cdocsfn2.tex.
%
%<package>\ifdefined\childdocmain\endinput\fi
%<package>\ProvidesFile{childdoc.def}[2018/12/30 v2.0 child document driver]
%<samplemain>\ProvidesFile{cdocsamp.tex}[2018/12/30 v2.0 sample for childdoc]
%<*driver>
%\ProvidesFile{childdoc.drv}[2018/12/30 v2.0 childdoc reference manual file]
\PassOptionsToClass{10pt,a4paper}{article}
\documentclass{ltxdoc}

\usepackage[margin=35mm]{geometry}
\usepackage{hyperref}
\usepackage{hyperxmp}
\usepackage[usenames]{color}

\hypersetup{colorlinks=true}
\hypersetup{pdfstartview=FitH}
\hypersetup{pdfpagemode=UseNone}
\hypersetup{pdfsource={}}
\hypersetup{pdflang={en-UK}}
\hypersetup{pdfcopyright={Copyright 2017-2018 Niklas Beisert.
  This work may be distributed and/or modified under the
  conditions of the LaTeX Project Public License, either version 1.3
  of this license or (at your option) any later version.}}
\hypersetup{pdflicenseurl={http://www.latex-project.org/lppl.txt}}
\hypersetup{pdfcontactaddress={ETH Zurich, ITP, HIT K,
  Wolfgang-Pauli-Strasse 27}}
\hypersetup{pdfcontactpostcode={8093}}
\hypersetup{pdfcontactcity={Zurich}}
\hypersetup{pdfcontactcountry={Switzerland}}
\hypersetup{pdfcontactemail={nbeisert@itp.phys.ethz.ch}}
\hypersetup{pdfcontacturl={http://people.phys.ethz.ch/\xmptilde nbeisert/}}

\newcommand{\secref}[1]{\hyperref[#1]{section \ref*{#1}}}

\parskip1ex
\parindent0pt
\let\olditemize\itemize
\def\itemize{\olditemize\parskip0pt}

\begin{document}

\title{The \textsf{childdoc} Package}
\hypersetup{pdftitle={The childdoc Package}}
\author{Niklas Beisert\\[2ex]
  Institut f\"ur Theoretische Physik\\
  Eidgen\"ossische Technische Hochschule Z\"urich\\
  Wolfgang-Pauli-Strasse 27, 8093 Z\"urich, Switzerland\\[1ex]
  \href{mailto:nbeisert@itp.phys.ethz.ch}
  {\texttt{nbeisert@itp.phys.ethz.ch}}}
\hypersetup{pdfauthor={Niklas Beisert}}
\hypersetup{pdfsubject={Manual for the LaTeX2e Package childdoc}}
\date{30 December 2018, \textsf{v2.0}}
\maketitle

\begin{abstract}\noindent
\textsf{childdoc} is a \LaTeXe{} package
that enables the direct compilation
of document sections included by |\include|
to individual files.
\end{abstract}

\begingroup
\parskip0ex
\tableofcontents
\endgroup

%%%%%%%%%%%%%%%%%%%%%%%%%%%%%%%%%%%%%%%%%%%%%%%%%%%%%%%%%%%%%%%%%%%%%%%%%%%%%%%%
%%%%%%%%%%%%%%%%%%%%%%%%%%%%%%%%%%%%%%%%%%%%%%%%%%%%%%%%%%%%%%%%%%%%%%%%%%%%%%%%
\section{Introduction}

\LaTeX{} provides a mechanism to structure a large document (such as a book)
into a main file and several child files (containing the chapters)
using the |\include| command.
This mechanism is beneficial for documents
which span hundreds of pages in order to
make the source file(s) more manageable.
Moreover, compilation can be restricted to
selected child files by means of the |\includeonly| command.
The latter feature can be used to reduce the compilation time while editing
(this was significantly more useful in the earlier days of \LaTeX{})
or to generate a smaller document which is easier to navigate.
Another application of |\includeonly| is to generate
documents consisting of selected parts of the complete document.

However, there are a few drawbacks of the plain |\include| mechanism:
\begin{itemize}
\item
The child files cannot be compiled on their own,
they can only be compiled via the main file.
A naive editing environment
(such as a text editor with an option
to have the current file processed by \LaTeX)
may require one to switch to the main file before compiling;
attempting to compile the child file produces errors.
\item
The main file must be modified (each time)
to adjust the |\includeonly| command
to the present needs. This easily leaves the main file in a messy state.
\item
The generated document will always carry the filename
of the main document. This is inconvenient if
several child files are to be compiled and
to be kept for distribution.
\end{itemize}

The present package provides a simple interface
to make child files individually compilable by \LaTeX{}.
Compiling a child file then has the same effect as compiling
the main file with an |\includeonly| command
to select the appropriate child.
Moreover the generated document will carry the name of the child
rather than the main file.
This resolves all three above issues.

This feature is meant to make the editing of books,
thesis documents and lecture notes somewhat more convenient.
However, the package can also be used efficiently for
composing a series of documents (such as exercise sheets)
which are typically distributed individually.
It then assists the author in generating the individual documents
(potentially in different versions)
as well as a document containing the collected series.
Another application is in developing style files
or other kinds of included material
where compilation of the style file could redirect
to a sample or test file.

%%%%%%%%%%%%%%%%%%%%%%%%%%%%%%%%%%%%%%%%%%%%%%%%%%%%%%%%%%%%%%%%%%%%%%%%%%%%%%%%
%%%%%%%%%%%%%%%%%%%%%%%%%%%%%%%%%%%%%%%%%%%%%%%%%%%%%%%%%%%%%%%%%%%%%%%%%%%%%%%%
\section{Usage}

First of all, the package \textsf{childdoc} is \emph{not} a standard
\LaTeXe{} |.sty| style file! Therefore it needs to be invoked in
a non-standard way.

%%%%%%%%%%%%%%%%%%%%%%%%%%%%%%%%%%%%%%%%%%%%%%%%%%%%%%%%%%%%%%%%%%%%%%%%%%%%%%%%
\subsection{Included Files}
\label{sec:include}

%%%%%%%%%%%%%%%%%%%%%%%%%%%%%%%%%%%%%%%%
\DescribeMacro{\childdocmain}
To use the package, add the commands
\begin{center}
\begin{tabular}{l}
|\input{childdoc.def}|\\
|\childdocmain{}|\\
\end{tabular}
\end{center}
at the very top of the main \LaTeX{} file,
in particular \emph{before} the |\documentclass| statement!
The argument of |\childdocmain| should be left empty
(but it must be present).

%%%%%%%%%%%%%%%%%%%%%%%%%%%%%%%%%%%%%%%%
\DescribeMacro{\childdocof}
Furthermore, add the commands
\begin{center}
\begin{tabular}{l}
|\input{childdoc.def}|\\
|\childdocof{|\textit{main}|}|\\
\end{tabular}
\end{center}
at the top of every child file \textit{child}
which is included by |\include{|\textit{child}|}|
from within the main file
(or at least for those files to be compiled individually).
The argument \textit{main} must be the filename of the main file.

There are a couple of
considerations in setting up the main and child documents:

%%%%%%%%%%%%%%%%%%%%%%%%%%%%%%%%%%%%%%%%
\paragraph{Restrictions.}

Please note the following restrictions:
\begin{itemize}
\item
|\childdocmain| must be called with one argument \textit{main}
to ensure compatibility with earlier version of the package.
It must either be empty (|\childdocmain{}|)
or precisely match the filename of the main file in which it is specified.
See \secref{sec:detection} for further information.
\item
The filename \textit{main} must be specified without the |.tex| extension.
\item
The filename \textit{main} is case sensitive
(even in case-insensitive file systems)
due to internal string comparison.
\item
The argument \textit{main} should be fully expanded, it cannot be a macro.
\item
Subdirectories and special characters should be avoided in filenames.
\item
The command |\childdocmain{|\textit{main}|}| must be followed by a whitespace.
It should not be followed immediately by another command
or by a comment mark `|%|'.
This is because the \TeX{} parser reads the token immediately following
the argument of |\childdocmain| and puts it
at the beginning of every child section;
however, a white\-space is ignored.
\end{itemize}

%%%%%%%%%%%%%%%%%%%%%%%%%%%%%%%%%%%%%%%%
\paragraph{Content of Main File.}

It is advisable to place all content in the child files included by |\include|.
Any output contained in the main file will appear in all child documents
unless suppressed manually;
it cannot be suppressed automatically by the |\includeonly| directive
and thus should normally be avoided.
A method to include some content in the main file
by means of conditional processing is described in \secref{sec:conditional}.

%%%%%%%%%%%%%%%%%%%%%%%%%%%%%%%%%%%%%%%%
\paragraph{Page Numbering.}

When only a part of the document is compiled,
the appropriate numbering of pages
(as well as other status parameters)
is determined from the |.aux| files.
The latter contain information from previous passes.
However this information needs to propagate through
all intermediate child documents.
Therefore the page numbering in child documents may well
be inconsistent until the complete document is compiled at least once.

A useful (if unconventional) way to always ensure a consistent
page numbering is to restart the numbering in each child document
and denote the pages by `\textit{child}|.|\textit{page}'
where \textit{child} represents the chapter/section number of the child file.
This can be achieved by the command
|\numberwithin{page}{|\textit{child}|}|
of the \textsf{amsmath} package
where \textit{child} can be |chapter| or |section|
depending on the chosen structuring.
Alternatively, one can modify the macro |\thepage| appropriately
and reset the counter |page| at the start of each child file.

%%%%%%%%%%%%%%%%%%%%%%%%%%%%%%%%%%%%%%%%%%%%%%%%%%%%%%%%%%%%%%%%%%%%%%%%%%%%%%%%
\subsection{Conditional Processing}
\label{sec:conditional}

The package provides a mechanism to compile different versions
of a document. To customise the versions further some conditional processing
can come in handy to distinguish which version is being compiled.
The package provides two macros to describe the compilation context:

%%%%%%%%%%%%%%%%%%%%%%%%%%%%%%%%%%%%%%%%
\DescribeMacro{\ifchilddoc}
The conditional |\ifchilddoc| distinguishes between the compilation of
child documents and the main document:
%
\begin{center}
|\ifchilddoc |\textit{child-code}| |[|\||else |\textit{main-code}]| \||fi|
\end{center}

%%%%%%%%%%%%%%%%%%%%%%%%%%%%%%%%%%%%%%%%
\DescribeMacro{\childdocname}
\DescribeMacro{\childdocjob}
The macro |\childdocname| contains the filename (without extension)
of the main or child file being processed.
Note that |\childdocjob| will always contain the name of the main file.

%%%%%%%%%%%%%%%%%%%%%%%%%%%%%%%%%%%%%%%%
\paragraph{Title Page.}

Conditional processing can be used to include a title or banner page
in the main document when proper precautions are taken.
Importantly, the code in the main file should ensure that the page counter
(as well as other status parameters which are stored in the |.aux| files)
takes the same value after the conditional processing.
Otherwise the page numbers may take divergent values
depending on which part is compiled.

For example, a title page could be declared by:
%
\begin{center}
\begin{tabular}{l}
|\ifchilddoc\||else|\\
|\addtocounter{page}{-1}|\\
\textit{code for title page}\\
|\newpage|\\
|\||fi|
\end{tabular}
\end{center}
%
A banner page for the child documents can be generated by:
%
\begin{center}
\begin{tabular}{l}
|\ifchilddoc|\\
|\addtocounter{page}{-1}|\\
\textit{code for banner page}\\
|\newpage|\\
|\||fi|
\end{tabular}
\end{center}
%
Here one could write a message such as:
\begin{center}
|This is the part \childdocname{} of \childdocjob{}.|
\end{center}

%%%%%%%%%%%%%%%%%%%%%%%%%%%%%%%%%%%%%%%%%%%%%%%%%%%%%%%%%%%%%%%%%%%%%%%%%%%%%%%%
\subsection{Flags}
\label{sec:flags}

The package makes it easy to generate different versions
of the main or child documents.
To this end compilation flags can be defined
and assigned different default values.
They will be particularly useful in conjunction
with the forwarding mechanism described in \secref{sec:forward}.

For example, it may be useful to have a flag |\version|
which can be set to |draft| or |final|.
The document source will contain some conditional code
depending on the value of |\version|.
Suppose further, the flag should default to |final| for the main file
and to |draft| for child files
which is a natural assignment for editing the document.
This is achieved by placing the following code
in the preamble of the main document
(below the |\childdocmain| directive):
%
\begin{center}
\begin{tabular}{l}
|\ifchilddoc|\\
|\providecommand{\version}{draft}|\\
|\||else|\\
|\providecommand{\version}{final}|\\
|\||fi|
\end{tabular}
\end{center}
%
The definition by |\providecommand| makes sure
that previous definitions are not overwritten.
Further statements |\providecommand{\version}{...}|
can thus be added before the above code to override it.

For the main file, one might add a line
(between |\childdocmain| and the above block)
%
\begin{center}
|%\ifchilddoc\||else\providecommand{\version}{draft}\||fi|
\end{center}
%
which can be uncommented to produce a draft version.
Likewise one can add a line to the very top of a child file
(above the |\childdocof{|\textit{main}|}| directive)
%
\begin{center}
|%\providecommand{\version}{final}|
\end{center}
%
which can be uncommented to produce the final version of this child document.

%%%%%%%%%%%%%%%%%%%%%%%%%%%%%%%%%%%%%%%%%%%%%%%%%%%%%%%%%%%%%%%%%%%%%%%%%%%%%%%%
\subsection{Forwarding}
\label{sec:forward}

Different versions of the main or child documents
using compilation flags as described in \secref{sec:flags}
can be (permanently) stored in different files
for convenient compilation, viewing and distribution.
To this end, the package defines a command
to pass on compilation to a different file:

%%%%%%%%%%%%%%%%%%%%%%%%%%%%%%%%%%%%%%%%
\DescribeMacro{\childdocforward}
The command |\childdocforward| redirects processing to
another source file:
%
\begin{center}
\begin{tabular}{l}
|\input{childdoc.def}|\\
|\childdocforward[|\textit{main}|]{|\textit{dest}|}|\\
\end{tabular}
\end{center}
%
The argument \textit{dest} is the destination file
(without extension).
It should be the main file or one of the child files.
Note that further \textsf{childdoc} directives
such as |\childdocof| and |\childdocforward|
in the indicated file will be processed in this form.
The optional argument \textit{main}
passes on directly to the main file \textit{main}
while pretending to compile the child \textit{dest}.
This form behaves as if \textit{dest}
issues |\childdocof{|\textit{main}|}| right away,
and no further \textsf{childdoc} directives will be processed.

%%%%%%%%%%%%%%%%%%%%%%%%%%%%%%%%%%%%%%%%
\DescribeMacro{\...prefix}
In the alternative form |\childdocforwardprefix|,
%
\begin{center}
\begin{tabular}{l}
|\input{childdoc.def}|\\
|\childdocforwardprefix[|\textit{main}|]{|\textit{prefix}|}{|\textit{dest}|}|
\end{tabular}
\end{center}
%
the destination file is determined by a pattern
depending on the current file:
To make this work, the current file must be called
`{\textit{prefix}\hspace{0.2em}\textit{suffix}}'
with \textit{prefix} matching precisely the argument.
Processing is then passed on to the file
`{\textit{dest}\hspace{0.2em}\textit{suffix}}'.
Surely, the same effect is achieved by
directly specifying the
argument `{\textit{dest}\hspace{0.2em}\textit{suffix}}'
in the first form.
However, that requires to set up a different file
for each child. With the alternative form of the command
all these files can have exactly the same content
which simplifies setting them up and maintaining them.

For example, the following file |draft.tex|
with a compilation flag |\version| as described in \secref{sec:flags}
compiles the main document as a draft:
%
\begin{center}
\begin{tabular}{l}
|\def\version{draft}|\\
|\input{childdoc.def}|\\
|\childdocforward{|\textit{main}|}|
\end{tabular}
\end{center}
%
Likewise, the following files |final|\textit{nn}|.tex|
compile the final version of the child document
|child|\textit{nn}|.tex|:
%
\begin{center}
\begin{tabular}{l}
|\def\version{final}|\\
|\input{childdoc.def}|\\
|\childdocforwardprefix{final}{child}|
\end{tabular}
\end{center}
%

Note that when several versions of a main file and/or of each child file
are to be generated, it may be convenient to set up a |Makefile| or
shell script to automatise the process.

%%%%%%%%%%%%%%%%%%%%%%%%%%%%%%%%%%%%%%%%%%%%%%%%%%%%%%%%%%%%%%%%%%%%%%%%%%%%%%%%
\subsection{Command Line Processing}
\label{sec:commandline}

The effect of redirection files can also be achieved by invoking
the \LaTeX{} compiler with a more elaborate command line.
Most conveniently this should be done as part
of a shell script or a |Makefile|.

When using \textsf{childdoc} in the main file, the following
command lines effectively perform a redirection
(note that depending on the shell being used,
backslashes may have to be doubled: `|\|' $\to$ `|\\|'):
%
\begin{center}
|... -jobname "|\textit{target}|" |\\|"|[\textit{flags}]%
|\input{childdoc.def}\childdocforward[|\textit{main}|]{|\textit{dest}|}"|
\end{center}
%
Here \textit{target} is the name of the output file,
\textit{main} is the name of the main file
and \textit{dest} is the name of the main or child file to be processed
(all filenames without extensions).
The optional argument \textit{main} can be omitted
if \textit{main} matches \textit{dest}.
Optionally, compilation \textit{flags} can be defined via |\def| commands.
This command line makes the \TeX{} engine believe
it is compiling the file \textit{target}
whose content is specified as the latter parameter.
The provided code then forwards the processing to
\textit{main} or \textit{dest} as described in \secref{sec:forward}.

%%%%%%%%%%%%%%%%%%%%%%%%%%%%%%%%%%%%%%%%%%%%%%%%%%%%%%%%%%%%%%%%%%%%%%%%%%%%%%%%
\subsection{Include by Input}
\label{sec:input}

Including child documents by |\include| has some restrictions by design.
Most notably, the content of a child document always occupies
its own set of pages; pages cannot be shared between child documents.
Usually, this behaviour makes perfect sense
because each child document contain an essential part of the document.
However, in some situations it may be desirable to compose
a document from a collection of parts
without having mandatory page breaks between then.
For this case, the package
provides a mechanism to include parts
by |\input| which can also be processed individually.
However, by construction this mechanism
requires manual handling of the content to be output.

%%%%%%%%%%%%%%%%%%%%%%%%%%%%%%%%%%%%%%%%
\DescribeMacro{\ifchilddocmanual}
The main file should be prepared as usual, see \secref{sec:include}.
However, the document body must make a distinction
between processing of an individual part and of the main document, e.g.:
%
\begin{center}
\begin{tabular}{l}
|\ifchilddocmanual|\\
|\input{\childdocname}|\\
|\||else|\\
\textit{document body with }|\input{|\textit{part}|}|\\
|\||fi|
\end{tabular}
\end{center}
%
The conditional |\ifchilddocmanual| is true whenever
a part to be included by |\input| is being compiled,
and the name of the part is stored in |\childdocname|.

%%%%%%%%%%%%%%%%%%%%%%%%%%%%%%%%%%%%%%%%
\DescribeMacro{\childdocby}
Each part to be included by |\input| should start with:
%
\begin{center}
\begin{tabular}{l}
|\input{childdoc.def}|\\
|\childdocby{|\textit{main}|}|\\
\end{tabular}
\end{center}
%
The directive |\childdocby| is similar to |\childdocof|
described in \secref{sec:include},
but the subsequent selection of content must be done manually.
To that end, both |\ifchilddoc| and |\ifchilddocmanual|
will be true upon processing of a part,
and the name of the part is stored in |\childdocname|.
Note that |\jobname| will be set to the filename of the current part
so that each part receives an individual |.aux| file
that does not interfere with the |.aux| file(s) of the main document.
This behaviour can be altered by the alternative form
|\childdocby[*]{|\textit{main}|}| (with a non-empty optional argument)
which uses the |.aux| file of the main document
by setting |\jobname| to \textit{main}.

%%%%%%%%%%%%%%%%%%%%%%%%%%%%%%%%%%%%%%%%%%%%%%%%%%%%%%%%%%%%%%%%%%%%%%%%%%%%%%%%
\subsection{Driver Development}
\label{sec:driver}

The \textsf{childdoc} mechanism can also be use for the development
of definition files such as \LaTeX{} styles or classes.
This case differs from the above setup with multiple parts
included by |\include| in that no |\includeonly| should be invoked.
This can be achieved by starting the include file
(before |\ProvidesPackage|) with:
%
\begin{center}
\begin{tabular}{l}
|\input{childdoc.def}|\\
|\childdocforward{|\textit{main}|}|\\
\end{tabular}
\end{center}
%
or alternatively with:
%
\begin{center}
\begin{tabular}{l}
|\input{childdoc.def}|\\
|\childdocby{|\textit{main}|}|\\
\end{tabular}
\end{center}
%
Both forms have slightly different effects as described above.
The main file is prepared as usual, see \secref{sec:include}.

%%%%%%%%%%%%%%%%%%%%%%%%%%%%%%%%%%%%%%%%%%%%%%%%%%%%%%%%%%%%%%%%%%%%%%%%%%%%%%%%
\subsection{Legacy Detection}
\label{sec:detection}

The directive |\childdocmain| in the main file can detect
whether the complete document or merely a child is to be compiled
even without using the directive |\childdocof|.
This method is deprecated because it is less robust
and there is no compelling reason to use it;
it is merely provided for backward compatibility
and it may be removed in future versions.

If the detection mechanism is to be used,
it is mandatory to correctly specify
the filename of the main file as the argument of |\childdocmain|:
%
\begin{center}
\begin{tabular}{l}
|\input{childdoc.def}|\\
|\childdocmain{|\textit{main}|}|\\
\end{tabular}
\end{center}
%
If |\jobname| does not match the argument \textit{main} of |\childdocmain|,
it is assumed that |\jobname| points to the child file to be compiled.
When using |\childdocmain| with the main file specified as argument,
it suffices to start a child file
with just |\input{|\textit{main}|}|
without loading of the package and using |\childdocof|.
If instead all processing is done
with the appropriate \textsf{childdoc} directives,
the argument of \textit{main} of |\childdocmain| can be empty.

An alternative version of the command line processing described
in \secref{sec:commandline} using the detection mechanism reads:
%
\begin{center}
|... -jobname "|\textit{target}|" "|[\textit{flags}]%
[|\def\jobname{|\textit{dest}|}|]|\input{|\textit{main}|}"|
\end{center}

%%%%%%%%%%%%%%%%%%%%%%%%%%%%%%%%%%%%%%%%%%%%%%%%%%%%%%%%%%%%%%%%%%%%%%%%%%%%%%%%
\subsection{Manual Code}
\label{sec:manual}

In case one cannot be certain whether the definitions file |childdoc.def|
is installed on the target \TeX{} distribution
and one prefers not to ship it,
it is conceivable to paste a few relevant commands into the sources.

To that end, drop all statements |\input{childdoc.def}|
and perform the replacements as outlined below.
Instead of |\childdocmain{|\textit{main}|}| add the following code
to the top of the main file:
%
\begin{center}
\begin{tabular}{l}
|\||ifdefined\childdocname\endinput\||fi\newif\ifchilddoc|\\
|\edef\childdocname{\scantokens\expandafter{\jobname\noexpand}}|\\
|\def\childdocmain{|\textit{main}|}\||ifx\childdocmain\childdocname\||else|\\
|\childdoctrue\includeonly{\childdocname}\let\jobname\childdocmain\||fi|\\
\end{tabular}
\end{center}
%
Instead of |\childdocof{|\textit{main}|}| just include the main file
at the top of each child file:
%
\begin{center}
|\input{|\textit{main}|}|
\end{center}
%
A simple redirection |\childdocforward{|\textit{dest}|}| is achieved by:
%
\begin{center}
|\def\jobname{|\textit{dest}|}\input{\jobname}|
\end{center}
%
The redirection with prefix
|\childdocforwardprefix[|\textit{prefix}|]{|\textit{dest}|}|
is accomplished by:
%
\begin{center}
\begin{tabular}{l}
|{\edef\jobname{\scantokens\expandafter{\jobname\noexpand}}|\\
|\def\redirectjob |\textit{prefix}|#1~~~{\gdef\jobname{|\textit{dest}|#1}}|\\
|\expandafter\redirectjob\jobname~~~}\input{\jobname}|
\end{tabular}
\end{center}

In an alternative approach,
child documents can be compiled by a specific command line
without additional code or specific definitions:
%
\begin{center}
|... -jobname "|\textit{target}|" "|[\textit{flags}]%
|\includeonly{|\textit{dest}|}\input{|\textit{main}|}"|
\end{center}
%

%%%%%%%%%%%%%%%%%%%%%%%%%%%%%%%%%%%%%%%%%%%%%%%%%%%%%%%%%%%%%%%%%%%%%%%%%%%%%%%%
%%%%%%%%%%%%%%%%%%%%%%%%%%%%%%%%%%%%%%%%%%%%%%%%%%%%%%%%%%%%%%%%%%%%%%%%%%%%%%%%
\section{Information}

%%%%%%%%%%%%%%%%%%%%%%%%%%%%%%%%%%%%%%%%%%%%%%%%%%%%%%%%%%%%%%%%%%%%%%%%%%%%%%%%
\subsection{Copyright}

Copyright \copyright{} 2017--2018 Niklas Beisert

This work may be distributed and/or modified under the
conditions of the \LaTeX{} Project Public License, either version 1.3
of this license or (at your option) any later version.
The latest version of this license is in
  \url{http://www.latex-project.org/lppl.txt}
and version 1.3 or later is part of all distributions of \LaTeX{}
version 2005/12/01 or later.

This work has the LPPL maintenance status `maintained'.

The Current Maintainer of this work is Niklas Beisert.

This work consists of the files |README.txt|, |childdoc.ins| and |childdoc.dtx|
as well as the derived files |childdoc.def|, |cdocsamp.tex|
with |cdocsch1.tex|, |cdocsch2.tex|, |cdocspt3.tex|, |cdocspt4.tex|,
|cdocsdrf.tex|, |cdocsfn1.tex|, |cdocsfn2.tex|
as well as |childdoc.pdf|.

%%%%%%%%%%%%%%%%%%%%%%%%%%%%%%%%%%%%%%%%%%%%%%%%%%%%%%%%%%%%%%%%%%%%%%%%%%%%%%%%
\subsection{Files and Installation}

The package consists of the files:
%
\begin{center}
\begin{tabular}{ll}
    |README.txt|   & readme file \\
    |childdoc.ins| & installation file \\
    |childdoc.dtx| & source file \\
    |childdoc.def| & definition file \\
    |cdocsamp.tex| & sample main file \\
    |cdocsch1.tex| & sample include file \\
    |cdocsch2.tex| & sample include file \\
    |cdocspt3.tex| & sample part file \\
    |cdocspt4.tex| & sample part file \\
    |cdocsdrf.tex| & sample redirection file \\
    |cdocsfn1.tex| & sample redirection file \\
    |cdocsfn2.tex| & sample redirection file \\
    |childdoc.pdf| & manual
\end{tabular}
\end{center}
%
The distribution consists of the files
|README.txt|, |childdoc.ins| and |childdoc.dtx|.
%
\begin{itemize}
\item
Run (pdf)\LaTeX{} on |childdoc.dtx|
to compile the manual |childdoc.pdf| (this file).
\item
Run \LaTeX{} on |childdoc.ins| to create the definitions file |childdoc.def|
and the sample |cdocsamp.tex| with include files
|cdocsch1.tex|, |cdocsch2.tex|, |cdocspt3.tex|, |cdocspt4.tex|,
|cdocsdrf.tex|, |cdocsfn1.tex|, |cdocsfn2.tex|.
Then copy the file |childdoc.def| to an appropriate directory of your \LaTeX{}
distribution, e.g.\ \textit{texmf-root}|/tex/latex/childdoc|.
\end{itemize}

%%%%%%%%%%%%%%%%%%%%%%%%%%%%%%%%%%%%%%%%%%%%%%%%%%%%%%%%%%%%%%%%%%%%%%%%%%%%%%%%
\subsection{Related CTAN Packages}

There are several other packages which offer a similar functionality:
%
\begin{itemize}
\item
The packages
\href{http://ctan.org/pkg/docmute}{\textsf{docmute}},
\href{http://ctan.org/pkg/includex}{\textsf{includex}} and
\href{http://ctan.org/pkg/standalone}{\textsf{standalone}}
provide commands to include only the document body of
a child file thus allowing both files to be compiled individually.
\item
The packages \href{http://ctan.org/pkg/subdocs}{\textsf{subdocs}}
and \href{http://ctan.org/pkg/subfiles}{\textsf{subfiles}}
provide structures in which the main and child documents can be
encapsulated and allowing them to be compiled individually.
The inclusion mechanism is different from the conventional |\include|.
\item
The package \href{http://ctan.org/pkg/combine}{\textsf{combine}}
is an elaborate solution to combine several documents into one.
\end{itemize}
%
See also the CTAN topic \href{http://ctan.org/topic/subdocs}{\textsf{subdocs}}
for further related packages.
The present package differs from the above solutions in that
a document structure constructed with the conventional |\include| mechanism
just needs two extra commands at the top of every file
such that all constituent files can be compiled individually.

%%%%%%%%%%%%%%%%%%%%%%%%%%%%%%%%%%%%%%%%%%%%%%%%%%%%%%%%%%%%%%%%%%%%%%%%%%%%%%%%
%\subsection{Feature Suggestions}
%
%The following is a list of features which may be useful for future
%versions of this package:
%%
%\begin{itemize}
%\item
%\ldots
%\end{itemize}

%%%%%%%%%%%%%%%%%%%%%%%%%%%%%%%%%%%%%%%%%%%%%%%%%%%%%%%%%%%%%%%%%%%%%%%%%%%%%%%%
\subsection{Revision History}

%%%%%%%%%%%%%%%%%%%%%%%%%%%%%%%%%%%%%%%%
\paragraph{v2.0:} 2018/12/30

\begin{itemize}
\item
immediate forward processing
\item
added |\childdocby| mechanism
\item
manual restructured
\end{itemize}

%%%%%%%%%%%%%%%%%%%%%%%%%%%%%%%%%%%%%%%%
\paragraph{v1.6:} 2018/01/17

\begin{itemize}
\item
application for development of include files
\item
corrections to manual
\end{itemize}

%%%%%%%%%%%%%%%%%%%%%%%%%%%%%%%%%%%%%%%%
\paragraph{v1.5:} 2017/05/21

\begin{itemize}
\item
more complete structuring introduced
\item
|\childdocof| introduced
\item
|\childdoc| renamed to |\childdocmain|
\item
|\childredirect| renamed to |\childdocforward| and |\childdocforwardprefix|
and functionality expanded
\end{itemize}

%%%%%%%%%%%%%%%%%%%%%%%%%%%%%%%%%%%%%%%%
\paragraph{v1.0:} 2017/04/27

\begin{itemize}
\item
manual and install package
\item
first version published on CTAN
\end{itemize}

%%%%%%%%%%%%%%%%%%%%%%%%%%%%%%%%%%%%%%%%
\paragraph{v0.6:} 2017/04/26

\begin{itemize}
\item
redirection mechanism added
\end{itemize}

%%%%%%%%%%%%%%%%%%%%%%%%%%%%%%%%%%%%%%%%
\paragraph{v0.5:} 2017/04/26

\begin{itemize}
\item
functionality in definition file
\end{itemize}


%%%%%%%%%%%%%%%%%%%%%%%%%%%%%%%%%%%%%%%%%%%%%%%%%%%%%%%%%%%%%%%%%%%%%%%%%%%%%%%%
%%%%%%%%%%%%%%%%%%%%%%%%%%%%%%%%%%%%%%%%%%%%%%%%%%%%%%%%%%%%%%%%%%%%%%%%%%%%%%%%
%%%%%%%%%%%%%%%%%%%%%%%%%%%%%%%%%%%%%%%%%%%%%%%%%%%%%%%%%%%%%%%%%%%%%%%%%%%%%%%%
\appendix

\settowidth\MacroIndent{\rmfamily\scriptsize 000\ }

 \DocInput{childdoc.dtx}

\end{document}
%</driver>
% \fi
%
% %%%%%%%%%%%%%%%%%%%%%%%%%%%%%%%%%%%%%%%%%%%%%%%%%%%%%%%%%%%%%%%%%%%%%%%%%%%%%%
% %%%%%%%%%%%%%%%%%%%%%%%%%%%%%%%%%%%%%%%%%%%%%%%%%%%%%%%%%%%%%%%%%%%%%%%%%%%%%%
% \section{Sample}
%\iffalse
%<*samplemain>
%\fi
%
% The following presents a sample document
% with two chapters, two parts, a title page,
% a compile flag as well as three forwarding files to set the flag.
% It consists of eight |.tex| files:
% \begin{center}
% \begin{tabular}{ll}
% |cdocsamp.tex|&main file\\
% |cdocsch1.tex|&include file for chapter 1\\
% |cdocsch2.tex|&include file for chapter 2\\
% |cdocspt3.tex|&include file for part 3\\
% |cdocspt4.tex|&include file for part 4\\
% |cdocsdrf.tex|&forwarding file for main file in draft mode\\
% |cdocsfi1.tex|&forwarding file for final version of chapter 1\\
% |cdocsfi2.tex|&forwarding file for final version of chapter 2\\
% \end{tabular}
% \end{center}
% Each of the eight files can be compiled directly by the \LaTeX{} compiler.
%
% %%%%%%%%%%%%%%%%%%%%%%%%%%%%%%%%%%%%%%
% \paragraph{Main File.}
%
% The main file is called |cdocsamp.tex|.
%
% Load the \textsf{childdoc} definitions and
% declare the filename for the main document:
%    \begin{macrocode}
\input{childdoc.def}
\childdocmain{}
%    \end{macrocode}

% Optional override for |\version| flag:
%    \begin{macrocode}
%%\ifchilddoc\else\providecommand{\version}{draft}\fi
%    \end{macrocode}

% Define the default values for the |\version| flag
% (|final| for the main file and |draft| for childs):
%    \begin{macrocode}
\ifchilddoc
\providecommand{\version}{draft}
\else
\providecommand{\version}{final}
\fi
%    \end{macrocode}

% Load the standard document class:
%    \begin{macrocode}
\documentclass[12pt]{article}
%    \end{macrocode}

% Start the document body:
%    \begin{macrocode}
\begin{document}
%    \end{macrocode}

% Declare a title page.
% Print title, part of document being processed and version flag:
%    \begin{macrocode}
\addtocounter{page}{-1}
\begin{center}
{\LARGE\bfseries{}childdoc example\par}
\vspace{1cm}
\ifchilddoc
\ifchilddocmanual part\else chapter\fi:
`\childdocname' of `\childdocjob'\par
\else
main document: `\childdocjob'\par
\fi
version: \version\par
\end{center}
\newpage
%    \end{macrocode}

% Manually include selected file,
% otherwise process as usual:
%    \begin{macrocode}
\ifchilddocmanual
\section*{part `\childdocname'}
\input{\childdocname}
\else
%    \end{macrocode}

% Include the two chapters:
%    \begin{macrocode}
\include{cdocsch1}
\include{cdocsch2}
%    \end{macrocode}

% Include the two parts unless only chapters should be displayed:
%    \begin{macrocode}
\ifchilddoc\else
\section{part three}
\input{cdocspt3}
\section{part four}
\input{cdocspt4}
\fi
%    \end{macrocode}

% Process as usual until here:
%    \begin{macrocode}
\fi
%    \end{macrocode}

% End of document body:
%    \begin{macrocode}
\end{document}
%    \end{macrocode}
%\iffalse
%</samplemain>
%\fi
%
% %%%%%%%%%%%%%%%%%%%%%%%%%%%%%%%%%%%%%%
% \paragraph{Chapter Include Files.}
%
% The include files are called |cdocsch1.tex| and |cdocsch2.tex|.
%
%\iffalse
%<*samplechap1|samplechap2>
%\fi

% Optional override for |\version| flag:
%    \begin{macrocode}
%%\providecommand{\version}{final}
%    \end{macrocode}

% Include the main document:
%    \begin{macrocode}
\input{childdoc.def}
\childdocof{cdocsamp}
%    \end{macrocode}

%\iffalse
%</samplechap1|samplechap2>
%\fi
%
%\iffalse
%<*samplechap1>
%\fi
% Some text for chapter 1:
%    \begin{macrocode}
\section{one}
some text in chapter one
%    \end{macrocode}

%\iffalse
%</samplechap1>
%\fi
% Some text for chapter 2:
%\iffalse
%<*samplechap2>
%\fi
%    \begin{macrocode}
\section{two}
more text in chapter two
%    \end{macrocode}

%\iffalse
%</samplechap2>
%\fi
%
% %%%%%%%%%%%%%%%%%%%%%%%%%%%%%%%%%%%%%%
% \paragraph{Part Include Files.}
%
% The include files are called |cdocspt3.tex| and |cdocspt4.tex|.
%
%\iffalse
%<*samplepart3|samplepart4>
%\fi

% Optional override for |\version| flag:
%    \begin{macrocode}
%%\providecommand{\version}{final}
%    \end{macrocode}

% Include the main document:
%    \begin{macrocode}
\input{childdoc.def}
\childdocby{cdocsamp}
%    \end{macrocode}

%\iffalse
%</samplepart3|samplepart4>
%\fi
%
%\iffalse
%<*samplepart3>
%\fi
% Some text for part 3:
%    \begin{macrocode}
some text in part three
%    \end{macrocode}

%\iffalse
%</samplepart3>
%\fi
% Some text for part 4:
%\iffalse
%<*samplepart4>
%\fi
%    \begin{macrocode}
more text in part four
%    \end{macrocode}

%\iffalse
%</samplepart4>
%\fi
%
% %%%%%%%%%%%%%%%%%%%%%%%%%%%%%%%%%%%%%%
% \paragraph{Forwarding for a Complete Draft.}
%
% The following forwarding file |cdocsdrf.tex|
% compiles the main document in draft mode:
%\iffalse
%<*sampledraft>
%\fi
%    \begin{macrocode}
\def\version{draft}
\input{childdoc.def}
\childdocforward{cdocsamp}
%    \end{macrocode}

%\iffalse
%</sampledraft>
%\fi
%
% %%%%%%%%%%%%%%%%%%%%%%%%%%%%%%%%%%%%%%
% \paragraph{Forwarding for Final Version of the Chapters.}
%
% The following forwarding files |cdocsfn1.tex| and |cdocsfn2.tex|
% (with identical content)
% compile the final versions of the child documents
% |cdocsch1.tex| and |cdocsch2.tex|, respectively:
%\iffalse
%<*samplefinal>
%\fi
%    \begin{macrocode}
\def\version{final}
\input{childdoc.def}
\childdocforwardprefix[cdocsamp]{cdocsfn}{cdocsch}
%    \end{macrocode}

%\iffalse
%</samplefinal>
%\fi
%
% %%%%%%%%%%%%%%%%%%%%%%%%%%%%%%%%%%%%%%
% \paragraph{Command Line Processing.}
%
% The following three command lines generate the output files
% |cdocscld|, |cdocscl1| and |cdocscl2|
% which should be identical to
% |cdocsdrf|, |cdocsch1| and |cdocsfn2|, respectively:
% \begin{center}
% \begin{tabular}{l}
% |latex -jobname cdocscld \|\\
% |  "\def\version{draft}\input{childdoc.def}\childdocforward{cdocsamp}"|\\
% |latex -jobname cdocscl1 \|\\
% |  "\input{childdoc.def}\childdocforward[cdocsamp]{cdocsch1}"|\\
% |latex -jobname cdocscl2 \|\\
% |  "\def\version{final}\input{childdoc.def}\childdocforward{cdocsch2}"|
% \end{tabular}
% \end{center}
% Note that the trailing backslash on each first line
% merely continues the input to the second line
% (for convenient cut ant paste).
% Furthermore, the command |latex| can be replaced by any
% of its alternative versions such as |pdflatex|.
%
% %%%%%%%%%%%%%%%%%%%%%%%%%%%%%%%%%%%%%%%%%%%%%%%%%%%%%%%%%%%%%%%%%%%%%%%%%%%%%%
% %%%%%%%%%%%%%%%%%%%%%%%%%%%%%%%%%%%%%%%%%%%%%%%%%%%%%%%%%%%%%%%%%%%%%%%%%%%%%%
% \section{Implementation}
%\iffalse
%<*package>
%\fi
%
% This section describes the definitions file |childdoc.def|.

% The definitions cannot be loaded using |\usepackage| or |\RequirePackage|
% which has a mechanism to prevent loading a style file more than once.
% When loading the definitions by means of |\input|
% multiple instances have to be prevented manually:
%\iffalse
%This code needs to be before the `\ProvidesFile' directive
%which is defined at the beginning of this file.
%Therefore it is also placed there and commented out here.
%</package>
%<*discard>
%\fi
%    \begin{macrocode}
\ifdefined\childdocmain\endinput\fi
%    \end{macrocode}
%\iffalse
%</discard>
%<*package>
%\fi
%
% \macro{\ifchilddoc}
% \macro{\ifchilddocmanual}
% The conditional |\ifchilddoc| tells whether a
% child (true) or main (false) document is being compiled.
% The conditional |\ifchilddocmanual| tells whether
% the |\includeonly| mechanism is used (false) or
% the selection of child files must be performed manually (true).
% The definitions initialise to false:
%    \begin{macrocode}
\newif\ifchilddoc
\newif\ifchilddocmanual
%    \end{macrocode}

% \macro{\childdocname}
% \macro{\childdocjob}
% The macro |\childdocname| stores the name of the main document
% to be compiled. The macro |\childdocjob| stores the name of
% the document on which the \LaTeX{} compiler was originally invoked.
% The content of |\jobname| cannot be compared
% to filenames specified in the source due to different catcodes.
% The following code rescans |\jobname|, stores the result
% in |\childdocname| and saves a copy in |\childdocjob|:
%    \begin{macrocode}
\edef\childdocname{\scantokens\expandafter{\jobname\noexpand}}
\let\childdocjob\childdocname
%    \end{macrocode}

% \macro{\childdocdisable}
% The macro |\childdocdisable| prevents the main file
% from being processed more than once.
% At this stage, the main document command |\childdocmain|
% is assumed to be called once again where it should do nothing.
% Any subsequent call to it should prevent
% a secondary processing of the main document
% It overwrites the forwarding commands
% |\childdocof| and |\childdocforward|
% with empty macros to prevent further inclusions of the main document:
%    \begin{macrocode}
\newcommand{\childdocdisable}
{
  \renewcommand{\childdocmain}[1]{\renewcommand{\childdocmain}[1]{\endinput}}
  \renewcommand{\childdocof}[1]{}
  \renewcommand{\childdocby}[2][]{}
  \renewcommand{\childdocforward}[2][]{}
  \renewcommand{\childdocdisable}{}
}
%    \end{macrocode}

% \macro{\childdocmain}
% The macro |\childdocmain| is to be called at the top of the main file
% with nothing or the main filename (without extension) as argument.
% First, it breaks loops.
% If the argument is not empty and does not match |\childdocname|
% (which is set by the first inclusion of |childdoc.def|),
% |\ifchilddoc| is set to true, |\includeonly| is applied to the child file
% and |\jobname| is set to the main file
% (for proper handling of |.aux| files):
%    \begin{macrocode}
\newcommand{\childdocmain}[1]
{
  \childdocdisable\childdocmain{}
  \if?#1?\else
    \begingroup
      \def\childdoctmp{#1}
      \ifx\childdoctmp\childdocname
        \def\childdoctmp{}
      \else
        \def\childdoctmp
        {
          \childdoctrue
          \includeonly{\childdocname}
          \def\childdocjob{#1}
          \def\jobname{#1}
        }
      \fi
      \expandafter
    \endgroup
    \childdoctmp
  \fi
}
%    \end{macrocode}

% \macro{\childdocof}
% The command |\childdocof| redirects
% compilation to the main file |#1|.
%    \begin{macrocode}
\newcommand{\childdocof}[1]
{
  \childdocdisable
  \childdoctrue
  \includeonly{\childdocname}
  \def\jobname{#1}
  \def\childdocjob{#1}
  \input{#1}
}
%    \end{macrocode}

% \macro{\childdocby}
% The command |\childdocby| ....
%    \begin{macrocode}
\newcommand{\childdocby}[2][]
{
  \childdocdisable
  \childdoctrue
  \childdocmanualtrue
  \if?#1?\else
    \def\jobname{#2}
  \fi
  \def\childdocjob{#2}
  \input{#2}
  \endinput
}
%    \end{macrocode}

% \macro{\childdocforward}
% The command |\childdocforward| redirects
% compilation to the main file or
% (if the optional argument is given) a child file.
% Parameters are set as if the main file
% or a child file starting with |\childdocof| was compiled.
% Then compilation is handed over to the main file:
%    \begin{macrocode}
\newcommand{\childdocforward}[2][]
{
  \begingroup
    \if?#1?
      \def\childdoctmp
      {
        \def\childdocname{#2}
        \def\childdocjob{#2}
        \def\jobname{#2}
        \input{#2}
        \endinput
      }
    \else
      \def\childdoctmp
      {
        \childdocdisable
        \def\childdocname{#2}
        \childdoctrue
        \includeonly{#2}
        \def\childdocjob{#1}
        \def\jobname{#1}
        \input{#1}
        \endinput
      }
    \fi
    \expandafter
  \endgroup
  \childdoctmp
}
%    \end{macrocode}

% \macro{\childdocforwardprefix}
% The command |\childdocforwardprefix| redirects
% compilation to the main or a child file by means of a pattern.
% The prefix |#1| in the current filename is replaced by |#2|
% and the suffix of the current filename is kept
% (it is assumed that the filename does not contain the substring `|~~~|'
% which is used as a delimiter).
% Compilation is handed over to the new file by |\childdocforward|:
%    \begin{macrocode}
\newcommand{\childdocforwardprefix}[3][]
{
  \begingroup
    \def\childdocextract #2##1~~~{\def\childdoctmp{\childdocforward[#1]{#3##1}}}
    \expandafter\childdocextract\childdocname~~~
    \expandafter
  \endgroup
  \childdoctmp
}
%    \end{macrocode}

% \macro{\childdoc}
% The deprecated macro |\childdoc| is a legacy version of |\childdocmain|:
%    \begin{macrocode}
\newcommand{\childdoc}{\childdocmain}
%    \end{macrocode}

% \macro{\childdocredirect}
% The deprecated macro |\childdocredirect| is a legacy version
% of |\childdocforward| and |\childdocforwardprefix|:
%    \begin{macrocode}
\newcommand{\childdocredirect}[2][]
{
  \begingroup
    \if?#1?
      \def\childdoctmp{\childdocforward{#2}}
    \else
      \def\childdoctmp{\childdocforwardprefix{#1}{#2}}
    \fi
    \expandafter
  \endgroup
  \childdoctmp
}
%    \end{macrocode}

%\iffalse
%</package>
%\fi
%
\endinput
|
and perform the replacements as outlined below.
Instead of |\childdocmain{|\textit{main}|}| add the following code
to the top of the main file:
%
\begin{center}
\begin{tabular}{l}
|\||ifdefined\childdocname\endinput\||fi\newif\ifchilddoc|\\
|\edef\childdocname{\scantokens\expandafter{\jobname\noexpand}}|\\
|\def\childdocmain{|\textit{main}|}\||ifx\childdocmain\childdocname\||else|\\
|\childdoctrue\includeonly{\childdocname}\let\jobname\childdocmain\||fi|\\
\end{tabular}
\end{center}
%
Instead of |\childdocof{|\textit{main}|}| just include the main file
at the top of each child file:
%
\begin{center}
|\input{|\textit{main}|}|
\end{center}
%
A simple redirection |\childdocforward{|\textit{dest}|}| is achieved by:
%
\begin{center}
|\def\jobname{|\textit{dest}|}\input{\jobname}|
\end{center}
%
The redirection with prefix
|\childdocforwardprefix[|\textit{prefix}|]{|\textit{dest}|}|
is accomplished by:
%
\begin{center}
\begin{tabular}{l}
|{\edef\jobname{\scantokens\expandafter{\jobname\noexpand}}|\\
|\def\redirectjob |\textit{prefix}|#1~~~{\gdef\jobname{|\textit{dest}|#1}}|\\
|\expandafter\redirectjob\jobname~~~}\input{\jobname}|
\end{tabular}
\end{center}

In an alternative approach,
child documents can be compiled by a specific command line
without additional code or specific definitions:
%
\begin{center}
|... -jobname "|\textit{target}|" "|[\textit{flags}]%
|\includeonly{|\textit{dest}|}\input{|\textit{main}|}"|
\end{center}
%

%%%%%%%%%%%%%%%%%%%%%%%%%%%%%%%%%%%%%%%%%%%%%%%%%%%%%%%%%%%%%%%%%%%%%%%%%%%%%%%%
%%%%%%%%%%%%%%%%%%%%%%%%%%%%%%%%%%%%%%%%%%%%%%%%%%%%%%%%%%%%%%%%%%%%%%%%%%%%%%%%
\section{Information}

%%%%%%%%%%%%%%%%%%%%%%%%%%%%%%%%%%%%%%%%%%%%%%%%%%%%%%%%%%%%%%%%%%%%%%%%%%%%%%%%
\subsection{Copyright}

Copyright \copyright{} 2017--2018 Niklas Beisert

This work may be distributed and/or modified under the
conditions of the \LaTeX{} Project Public License, either version 1.3
of this license or (at your option) any later version.
The latest version of this license is in
  \url{http://www.latex-project.org/lppl.txt}
and version 1.3 or later is part of all distributions of \LaTeX{}
version 2005/12/01 or later.

This work has the LPPL maintenance status `maintained'.

The Current Maintainer of this work is Niklas Beisert.

This work consists of the files |README.txt|, |childdoc.ins| and |childdoc.dtx|
as well as the derived files |childdoc.def|, |cdocsamp.tex|
with |cdocsch1.tex|, |cdocsch2.tex|, |cdocspt3.tex|, |cdocspt4.tex|,
|cdocsdrf.tex|, |cdocsfn1.tex|, |cdocsfn2.tex|
as well as |childdoc.pdf|.

%%%%%%%%%%%%%%%%%%%%%%%%%%%%%%%%%%%%%%%%%%%%%%%%%%%%%%%%%%%%%%%%%%%%%%%%%%%%%%%%
\subsection{Files and Installation}

The package consists of the files:
%
\begin{center}
\begin{tabular}{ll}
    |README.txt|   & readme file \\
    |childdoc.ins| & installation file \\
    |childdoc.dtx| & source file \\
    |childdoc.def| & definition file \\
    |cdocsamp.tex| & sample main file \\
    |cdocsch1.tex| & sample include file \\
    |cdocsch2.tex| & sample include file \\
    |cdocspt3.tex| & sample part file \\
    |cdocspt4.tex| & sample part file \\
    |cdocsdrf.tex| & sample redirection file \\
    |cdocsfn1.tex| & sample redirection file \\
    |cdocsfn2.tex| & sample redirection file \\
    |childdoc.pdf| & manual
\end{tabular}
\end{center}
%
The distribution consists of the files
|README.txt|, |childdoc.ins| and |childdoc.dtx|.
%
\begin{itemize}
\item
Run (pdf)\LaTeX{} on |childdoc.dtx|
to compile the manual |childdoc.pdf| (this file).
\item
Run \LaTeX{} on |childdoc.ins| to create the definitions file |childdoc.def|
and the sample |cdocsamp.tex| with include files
|cdocsch1.tex|, |cdocsch2.tex|, |cdocspt3.tex|, |cdocspt4.tex|,
|cdocsdrf.tex|, |cdocsfn1.tex|, |cdocsfn2.tex|.
Then copy the file |childdoc.def| to an appropriate directory of your \LaTeX{}
distribution, e.g.\ \textit{texmf-root}|/tex/latex/childdoc|.
\end{itemize}

%%%%%%%%%%%%%%%%%%%%%%%%%%%%%%%%%%%%%%%%%%%%%%%%%%%%%%%%%%%%%%%%%%%%%%%%%%%%%%%%
\subsection{Related CTAN Packages}

There are several other packages which offer a similar functionality:
%
\begin{itemize}
\item
The packages
\href{http://ctan.org/pkg/docmute}{\textsf{docmute}},
\href{http://ctan.org/pkg/includex}{\textsf{includex}} and
\href{http://ctan.org/pkg/standalone}{\textsf{standalone}}
provide commands to include only the document body of
a child file thus allowing both files to be compiled individually.
\item
The packages \href{http://ctan.org/pkg/subdocs}{\textsf{subdocs}}
and \href{http://ctan.org/pkg/subfiles}{\textsf{subfiles}}
provide structures in which the main and child documents can be
encapsulated and allowing them to be compiled individually.
The inclusion mechanism is different from the conventional |\include|.
\item
The package \href{http://ctan.org/pkg/combine}{\textsf{combine}}
is an elaborate solution to combine several documents into one.
\end{itemize}
%
See also the CTAN topic \href{http://ctan.org/topic/subdocs}{\textsf{subdocs}}
for further related packages.
The present package differs from the above solutions in that
a document structure constructed with the conventional |\include| mechanism
just needs two extra commands at the top of every file
such that all constituent files can be compiled individually.

%%%%%%%%%%%%%%%%%%%%%%%%%%%%%%%%%%%%%%%%%%%%%%%%%%%%%%%%%%%%%%%%%%%%%%%%%%%%%%%%
%\subsection{Feature Suggestions}
%
%The following is a list of features which may be useful for future
%versions of this package:
%%
%\begin{itemize}
%\item
%\ldots
%\end{itemize}

%%%%%%%%%%%%%%%%%%%%%%%%%%%%%%%%%%%%%%%%%%%%%%%%%%%%%%%%%%%%%%%%%%%%%%%%%%%%%%%%
\subsection{Revision History}

%%%%%%%%%%%%%%%%%%%%%%%%%%%%%%%%%%%%%%%%
\paragraph{v2.0:} 2018/12/30

\begin{itemize}
\item
immediate forward processing
\item
added |\childdocby| mechanism
\item
manual restructured
\end{itemize}

%%%%%%%%%%%%%%%%%%%%%%%%%%%%%%%%%%%%%%%%
\paragraph{v1.6:} 2018/01/17

\begin{itemize}
\item
application for development of include files
\item
corrections to manual
\end{itemize}

%%%%%%%%%%%%%%%%%%%%%%%%%%%%%%%%%%%%%%%%
\paragraph{v1.5:} 2017/05/21

\begin{itemize}
\item
more complete structuring introduced
\item
|\childdocof| introduced
\item
|\childdoc| renamed to |\childdocmain|
\item
|\childredirect| renamed to |\childdocforward| and |\childdocforwardprefix|
and functionality expanded
\end{itemize}

%%%%%%%%%%%%%%%%%%%%%%%%%%%%%%%%%%%%%%%%
\paragraph{v1.0:} 2017/04/27

\begin{itemize}
\item
manual and install package
\item
first version published on CTAN
\end{itemize}

%%%%%%%%%%%%%%%%%%%%%%%%%%%%%%%%%%%%%%%%
\paragraph{v0.6:} 2017/04/26

\begin{itemize}
\item
redirection mechanism added
\end{itemize}

%%%%%%%%%%%%%%%%%%%%%%%%%%%%%%%%%%%%%%%%
\paragraph{v0.5:} 2017/04/26

\begin{itemize}
\item
functionality in definition file
\end{itemize}


%%%%%%%%%%%%%%%%%%%%%%%%%%%%%%%%%%%%%%%%%%%%%%%%%%%%%%%%%%%%%%%%%%%%%%%%%%%%%%%%
%%%%%%%%%%%%%%%%%%%%%%%%%%%%%%%%%%%%%%%%%%%%%%%%%%%%%%%%%%%%%%%%%%%%%%%%%%%%%%%%
%%%%%%%%%%%%%%%%%%%%%%%%%%%%%%%%%%%%%%%%%%%%%%%%%%%%%%%%%%%%%%%%%%%%%%%%%%%%%%%%
\appendix

\settowidth\MacroIndent{\rmfamily\scriptsize 000\ }

 \DocInput{childdoc.dtx}

\end{document}
%</driver>
% \fi
%
% %%%%%%%%%%%%%%%%%%%%%%%%%%%%%%%%%%%%%%%%%%%%%%%%%%%%%%%%%%%%%%%%%%%%%%%%%%%%%%
% %%%%%%%%%%%%%%%%%%%%%%%%%%%%%%%%%%%%%%%%%%%%%%%%%%%%%%%%%%%%%%%%%%%%%%%%%%%%%%
% \section{Sample}
%\iffalse
%<*samplemain>
%\fi
%
% The following presents a sample document
% with two chapters, two parts, a title page,
% a compile flag as well as three forwarding files to set the flag.
% It consists of eight |.tex| files:
% \begin{center}
% \begin{tabular}{ll}
% |cdocsamp.tex|&main file\\
% |cdocsch1.tex|&include file for chapter 1\\
% |cdocsch2.tex|&include file for chapter 2\\
% |cdocspt3.tex|&include file for part 3\\
% |cdocspt4.tex|&include file for part 4\\
% |cdocsdrf.tex|&forwarding file for main file in draft mode\\
% |cdocsfi1.tex|&forwarding file for final version of chapter 1\\
% |cdocsfi2.tex|&forwarding file for final version of chapter 2\\
% \end{tabular}
% \end{center}
% Each of the eight files can be compiled directly by the \LaTeX{} compiler.
%
% %%%%%%%%%%%%%%%%%%%%%%%%%%%%%%%%%%%%%%
% \paragraph{Main File.}
%
% The main file is called |cdocsamp.tex|.
%
% Load the \textsf{childdoc} definitions and
% declare the filename for the main document:
%    \begin{macrocode}
% \iffalse
%
% childdoc.dtx Copyright (C) 2017-2018 Niklas Beisert
%
% This work may be distributed and/or modified under the
% conditions of the LaTeX Project Public License, either version 1.3
% of this license or (at your option) any later version.
% The latest version of this license is in
%   http://www.latex-project.org/lppl.txt
% and version 1.3 or later is part of all distributions of LaTeX
% version 2005/12/01 or later.
%
% This work has the LPPL maintenance status `maintained'.
%
% The Current Maintainer of this work is Niklas Beisert.
%
% This work consists of the files childdoc.dtx and childdoc.ins
% and the derived files childdoc.def and cdocsamp.tex with
% cdocsch1.tex, cdocsch2.tex, cdocsdrf.tex, cdocsfn1.tex, cdocsfn2.tex.
%
%<package>\ifdefined\childdocmain\endinput\fi
%<package>\ProvidesFile{childdoc.def}[2018/12/30 v2.0 child document driver]
%<samplemain>\ProvidesFile{cdocsamp.tex}[2018/12/30 v2.0 sample for childdoc]
%<*driver>
%\ProvidesFile{childdoc.drv}[2018/12/30 v2.0 childdoc reference manual file]
\PassOptionsToClass{10pt,a4paper}{article}
\documentclass{ltxdoc}

\usepackage[margin=35mm]{geometry}
\usepackage{hyperref}
\usepackage{hyperxmp}
\usepackage[usenames]{color}

\hypersetup{colorlinks=true}
\hypersetup{pdfstartview=FitH}
\hypersetup{pdfpagemode=UseNone}
\hypersetup{pdfsource={}}
\hypersetup{pdflang={en-UK}}
\hypersetup{pdfcopyright={Copyright 2017-2018 Niklas Beisert.
  This work may be distributed and/or modified under the
  conditions of the LaTeX Project Public License, either version 1.3
  of this license or (at your option) any later version.}}
\hypersetup{pdflicenseurl={http://www.latex-project.org/lppl.txt}}
\hypersetup{pdfcontactaddress={ETH Zurich, ITP, HIT K,
  Wolfgang-Pauli-Strasse 27}}
\hypersetup{pdfcontactpostcode={8093}}
\hypersetup{pdfcontactcity={Zurich}}
\hypersetup{pdfcontactcountry={Switzerland}}
\hypersetup{pdfcontactemail={nbeisert@itp.phys.ethz.ch}}
\hypersetup{pdfcontacturl={http://people.phys.ethz.ch/\xmptilde nbeisert/}}

\newcommand{\secref}[1]{\hyperref[#1]{section \ref*{#1}}}

\parskip1ex
\parindent0pt
\let\olditemize\itemize
\def\itemize{\olditemize\parskip0pt}

\begin{document}

\title{The \textsf{childdoc} Package}
\hypersetup{pdftitle={The childdoc Package}}
\author{Niklas Beisert\\[2ex]
  Institut f\"ur Theoretische Physik\\
  Eidgen\"ossische Technische Hochschule Z\"urich\\
  Wolfgang-Pauli-Strasse 27, 8093 Z\"urich, Switzerland\\[1ex]
  \href{mailto:nbeisert@itp.phys.ethz.ch}
  {\texttt{nbeisert@itp.phys.ethz.ch}}}
\hypersetup{pdfauthor={Niklas Beisert}}
\hypersetup{pdfsubject={Manual for the LaTeX2e Package childdoc}}
\date{30 December 2018, \textsf{v2.0}}
\maketitle

\begin{abstract}\noindent
\textsf{childdoc} is a \LaTeXe{} package
that enables the direct compilation
of document sections included by |\include|
to individual files.
\end{abstract}

\begingroup
\parskip0ex
\tableofcontents
\endgroup

%%%%%%%%%%%%%%%%%%%%%%%%%%%%%%%%%%%%%%%%%%%%%%%%%%%%%%%%%%%%%%%%%%%%%%%%%%%%%%%%
%%%%%%%%%%%%%%%%%%%%%%%%%%%%%%%%%%%%%%%%%%%%%%%%%%%%%%%%%%%%%%%%%%%%%%%%%%%%%%%%
\section{Introduction}

\LaTeX{} provides a mechanism to structure a large document (such as a book)
into a main file and several child files (containing the chapters)
using the |\include| command.
This mechanism is beneficial for documents
which span hundreds of pages in order to
make the source file(s) more manageable.
Moreover, compilation can be restricted to
selected child files by means of the |\includeonly| command.
The latter feature can be used to reduce the compilation time while editing
(this was significantly more useful in the earlier days of \LaTeX{})
or to generate a smaller document which is easier to navigate.
Another application of |\includeonly| is to generate
documents consisting of selected parts of the complete document.

However, there are a few drawbacks of the plain |\include| mechanism:
\begin{itemize}
\item
The child files cannot be compiled on their own,
they can only be compiled via the main file.
A naive editing environment
(such as a text editor with an option
to have the current file processed by \LaTeX)
may require one to switch to the main file before compiling;
attempting to compile the child file produces errors.
\item
The main file must be modified (each time)
to adjust the |\includeonly| command
to the present needs. This easily leaves the main file in a messy state.
\item
The generated document will always carry the filename
of the main document. This is inconvenient if
several child files are to be compiled and
to be kept for distribution.
\end{itemize}

The present package provides a simple interface
to make child files individually compilable by \LaTeX{}.
Compiling a child file then has the same effect as compiling
the main file with an |\includeonly| command
to select the appropriate child.
Moreover the generated document will carry the name of the child
rather than the main file.
This resolves all three above issues.

This feature is meant to make the editing of books,
thesis documents and lecture notes somewhat more convenient.
However, the package can also be used efficiently for
composing a series of documents (such as exercise sheets)
which are typically distributed individually.
It then assists the author in generating the individual documents
(potentially in different versions)
as well as a document containing the collected series.
Another application is in developing style files
or other kinds of included material
where compilation of the style file could redirect
to a sample or test file.

%%%%%%%%%%%%%%%%%%%%%%%%%%%%%%%%%%%%%%%%%%%%%%%%%%%%%%%%%%%%%%%%%%%%%%%%%%%%%%%%
%%%%%%%%%%%%%%%%%%%%%%%%%%%%%%%%%%%%%%%%%%%%%%%%%%%%%%%%%%%%%%%%%%%%%%%%%%%%%%%%
\section{Usage}

First of all, the package \textsf{childdoc} is \emph{not} a standard
\LaTeXe{} |.sty| style file! Therefore it needs to be invoked in
a non-standard way.

%%%%%%%%%%%%%%%%%%%%%%%%%%%%%%%%%%%%%%%%%%%%%%%%%%%%%%%%%%%%%%%%%%%%%%%%%%%%%%%%
\subsection{Included Files}
\label{sec:include}

%%%%%%%%%%%%%%%%%%%%%%%%%%%%%%%%%%%%%%%%
\DescribeMacro{\childdocmain}
To use the package, add the commands
\begin{center}
\begin{tabular}{l}
|\input{childdoc.def}|\\
|\childdocmain{}|\\
\end{tabular}
\end{center}
at the very top of the main \LaTeX{} file,
in particular \emph{before} the |\documentclass| statement!
The argument of |\childdocmain| should be left empty
(but it must be present).

%%%%%%%%%%%%%%%%%%%%%%%%%%%%%%%%%%%%%%%%
\DescribeMacro{\childdocof}
Furthermore, add the commands
\begin{center}
\begin{tabular}{l}
|\input{childdoc.def}|\\
|\childdocof{|\textit{main}|}|\\
\end{tabular}
\end{center}
at the top of every child file \textit{child}
which is included by |\include{|\textit{child}|}|
from within the main file
(or at least for those files to be compiled individually).
The argument \textit{main} must be the filename of the main file.

There are a couple of
considerations in setting up the main and child documents:

%%%%%%%%%%%%%%%%%%%%%%%%%%%%%%%%%%%%%%%%
\paragraph{Restrictions.}

Please note the following restrictions:
\begin{itemize}
\item
|\childdocmain| must be called with one argument \textit{main}
to ensure compatibility with earlier version of the package.
It must either be empty (|\childdocmain{}|)
or precisely match the filename of the main file in which it is specified.
See \secref{sec:detection} for further information.
\item
The filename \textit{main} must be specified without the |.tex| extension.
\item
The filename \textit{main} is case sensitive
(even in case-insensitive file systems)
due to internal string comparison.
\item
The argument \textit{main} should be fully expanded, it cannot be a macro.
\item
Subdirectories and special characters should be avoided in filenames.
\item
The command |\childdocmain{|\textit{main}|}| must be followed by a whitespace.
It should not be followed immediately by another command
or by a comment mark `|%|'.
This is because the \TeX{} parser reads the token immediately following
the argument of |\childdocmain| and puts it
at the beginning of every child section;
however, a white\-space is ignored.
\end{itemize}

%%%%%%%%%%%%%%%%%%%%%%%%%%%%%%%%%%%%%%%%
\paragraph{Content of Main File.}

It is advisable to place all content in the child files included by |\include|.
Any output contained in the main file will appear in all child documents
unless suppressed manually;
it cannot be suppressed automatically by the |\includeonly| directive
and thus should normally be avoided.
A method to include some content in the main file
by means of conditional processing is described in \secref{sec:conditional}.

%%%%%%%%%%%%%%%%%%%%%%%%%%%%%%%%%%%%%%%%
\paragraph{Page Numbering.}

When only a part of the document is compiled,
the appropriate numbering of pages
(as well as other status parameters)
is determined from the |.aux| files.
The latter contain information from previous passes.
However this information needs to propagate through
all intermediate child documents.
Therefore the page numbering in child documents may well
be inconsistent until the complete document is compiled at least once.

A useful (if unconventional) way to always ensure a consistent
page numbering is to restart the numbering in each child document
and denote the pages by `\textit{child}|.|\textit{page}'
where \textit{child} represents the chapter/section number of the child file.
This can be achieved by the command
|\numberwithin{page}{|\textit{child}|}|
of the \textsf{amsmath} package
where \textit{child} can be |chapter| or |section|
depending on the chosen structuring.
Alternatively, one can modify the macro |\thepage| appropriately
and reset the counter |page| at the start of each child file.

%%%%%%%%%%%%%%%%%%%%%%%%%%%%%%%%%%%%%%%%%%%%%%%%%%%%%%%%%%%%%%%%%%%%%%%%%%%%%%%%
\subsection{Conditional Processing}
\label{sec:conditional}

The package provides a mechanism to compile different versions
of a document. To customise the versions further some conditional processing
can come in handy to distinguish which version is being compiled.
The package provides two macros to describe the compilation context:

%%%%%%%%%%%%%%%%%%%%%%%%%%%%%%%%%%%%%%%%
\DescribeMacro{\ifchilddoc}
The conditional |\ifchilddoc| distinguishes between the compilation of
child documents and the main document:
%
\begin{center}
|\ifchilddoc |\textit{child-code}| |[|\||else |\textit{main-code}]| \||fi|
\end{center}

%%%%%%%%%%%%%%%%%%%%%%%%%%%%%%%%%%%%%%%%
\DescribeMacro{\childdocname}
\DescribeMacro{\childdocjob}
The macro |\childdocname| contains the filename (without extension)
of the main or child file being processed.
Note that |\childdocjob| will always contain the name of the main file.

%%%%%%%%%%%%%%%%%%%%%%%%%%%%%%%%%%%%%%%%
\paragraph{Title Page.}

Conditional processing can be used to include a title or banner page
in the main document when proper precautions are taken.
Importantly, the code in the main file should ensure that the page counter
(as well as other status parameters which are stored in the |.aux| files)
takes the same value after the conditional processing.
Otherwise the page numbers may take divergent values
depending on which part is compiled.

For example, a title page could be declared by:
%
\begin{center}
\begin{tabular}{l}
|\ifchilddoc\||else|\\
|\addtocounter{page}{-1}|\\
\textit{code for title page}\\
|\newpage|\\
|\||fi|
\end{tabular}
\end{center}
%
A banner page for the child documents can be generated by:
%
\begin{center}
\begin{tabular}{l}
|\ifchilddoc|\\
|\addtocounter{page}{-1}|\\
\textit{code for banner page}\\
|\newpage|\\
|\||fi|
\end{tabular}
\end{center}
%
Here one could write a message such as:
\begin{center}
|This is the part \childdocname{} of \childdocjob{}.|
\end{center}

%%%%%%%%%%%%%%%%%%%%%%%%%%%%%%%%%%%%%%%%%%%%%%%%%%%%%%%%%%%%%%%%%%%%%%%%%%%%%%%%
\subsection{Flags}
\label{sec:flags}

The package makes it easy to generate different versions
of the main or child documents.
To this end compilation flags can be defined
and assigned different default values.
They will be particularly useful in conjunction
with the forwarding mechanism described in \secref{sec:forward}.

For example, it may be useful to have a flag |\version|
which can be set to |draft| or |final|.
The document source will contain some conditional code
depending on the value of |\version|.
Suppose further, the flag should default to |final| for the main file
and to |draft| for child files
which is a natural assignment for editing the document.
This is achieved by placing the following code
in the preamble of the main document
(below the |\childdocmain| directive):
%
\begin{center}
\begin{tabular}{l}
|\ifchilddoc|\\
|\providecommand{\version}{draft}|\\
|\||else|\\
|\providecommand{\version}{final}|\\
|\||fi|
\end{tabular}
\end{center}
%
The definition by |\providecommand| makes sure
that previous definitions are not overwritten.
Further statements |\providecommand{\version}{...}|
can thus be added before the above code to override it.

For the main file, one might add a line
(between |\childdocmain| and the above block)
%
\begin{center}
|%\ifchilddoc\||else\providecommand{\version}{draft}\||fi|
\end{center}
%
which can be uncommented to produce a draft version.
Likewise one can add a line to the very top of a child file
(above the |\childdocof{|\textit{main}|}| directive)
%
\begin{center}
|%\providecommand{\version}{final}|
\end{center}
%
which can be uncommented to produce the final version of this child document.

%%%%%%%%%%%%%%%%%%%%%%%%%%%%%%%%%%%%%%%%%%%%%%%%%%%%%%%%%%%%%%%%%%%%%%%%%%%%%%%%
\subsection{Forwarding}
\label{sec:forward}

Different versions of the main or child documents
using compilation flags as described in \secref{sec:flags}
can be (permanently) stored in different files
for convenient compilation, viewing and distribution.
To this end, the package defines a command
to pass on compilation to a different file:

%%%%%%%%%%%%%%%%%%%%%%%%%%%%%%%%%%%%%%%%
\DescribeMacro{\childdocforward}
The command |\childdocforward| redirects processing to
another source file:
%
\begin{center}
\begin{tabular}{l}
|\input{childdoc.def}|\\
|\childdocforward[|\textit{main}|]{|\textit{dest}|}|\\
\end{tabular}
\end{center}
%
The argument \textit{dest} is the destination file
(without extension).
It should be the main file or one of the child files.
Note that further \textsf{childdoc} directives
such as |\childdocof| and |\childdocforward|
in the indicated file will be processed in this form.
The optional argument \textit{main}
passes on directly to the main file \textit{main}
while pretending to compile the child \textit{dest}.
This form behaves as if \textit{dest}
issues |\childdocof{|\textit{main}|}| right away,
and no further \textsf{childdoc} directives will be processed.

%%%%%%%%%%%%%%%%%%%%%%%%%%%%%%%%%%%%%%%%
\DescribeMacro{\...prefix}
In the alternative form |\childdocforwardprefix|,
%
\begin{center}
\begin{tabular}{l}
|\input{childdoc.def}|\\
|\childdocforwardprefix[|\textit{main}|]{|\textit{prefix}|}{|\textit{dest}|}|
\end{tabular}
\end{center}
%
the destination file is determined by a pattern
depending on the current file:
To make this work, the current file must be called
`{\textit{prefix}\hspace{0.2em}\textit{suffix}}'
with \textit{prefix} matching precisely the argument.
Processing is then passed on to the file
`{\textit{dest}\hspace{0.2em}\textit{suffix}}'.
Surely, the same effect is achieved by
directly specifying the
argument `{\textit{dest}\hspace{0.2em}\textit{suffix}}'
in the first form.
However, that requires to set up a different file
for each child. With the alternative form of the command
all these files can have exactly the same content
which simplifies setting them up and maintaining them.

For example, the following file |draft.tex|
with a compilation flag |\version| as described in \secref{sec:flags}
compiles the main document as a draft:
%
\begin{center}
\begin{tabular}{l}
|\def\version{draft}|\\
|\input{childdoc.def}|\\
|\childdocforward{|\textit{main}|}|
\end{tabular}
\end{center}
%
Likewise, the following files |final|\textit{nn}|.tex|
compile the final version of the child document
|child|\textit{nn}|.tex|:
%
\begin{center}
\begin{tabular}{l}
|\def\version{final}|\\
|\input{childdoc.def}|\\
|\childdocforwardprefix{final}{child}|
\end{tabular}
\end{center}
%

Note that when several versions of a main file and/or of each child file
are to be generated, it may be convenient to set up a |Makefile| or
shell script to automatise the process.

%%%%%%%%%%%%%%%%%%%%%%%%%%%%%%%%%%%%%%%%%%%%%%%%%%%%%%%%%%%%%%%%%%%%%%%%%%%%%%%%
\subsection{Command Line Processing}
\label{sec:commandline}

The effect of redirection files can also be achieved by invoking
the \LaTeX{} compiler with a more elaborate command line.
Most conveniently this should be done as part
of a shell script or a |Makefile|.

When using \textsf{childdoc} in the main file, the following
command lines effectively perform a redirection
(note that depending on the shell being used,
backslashes may have to be doubled: `|\|' $\to$ `|\\|'):
%
\begin{center}
|... -jobname "|\textit{target}|" |\\|"|[\textit{flags}]%
|\input{childdoc.def}\childdocforward[|\textit{main}|]{|\textit{dest}|}"|
\end{center}
%
Here \textit{target} is the name of the output file,
\textit{main} is the name of the main file
and \textit{dest} is the name of the main or child file to be processed
(all filenames without extensions).
The optional argument \textit{main} can be omitted
if \textit{main} matches \textit{dest}.
Optionally, compilation \textit{flags} can be defined via |\def| commands.
This command line makes the \TeX{} engine believe
it is compiling the file \textit{target}
whose content is specified as the latter parameter.
The provided code then forwards the processing to
\textit{main} or \textit{dest} as described in \secref{sec:forward}.

%%%%%%%%%%%%%%%%%%%%%%%%%%%%%%%%%%%%%%%%%%%%%%%%%%%%%%%%%%%%%%%%%%%%%%%%%%%%%%%%
\subsection{Include by Input}
\label{sec:input}

Including child documents by |\include| has some restrictions by design.
Most notably, the content of a child document always occupies
its own set of pages; pages cannot be shared between child documents.
Usually, this behaviour makes perfect sense
because each child document contain an essential part of the document.
However, in some situations it may be desirable to compose
a document from a collection of parts
without having mandatory page breaks between then.
For this case, the package
provides a mechanism to include parts
by |\input| which can also be processed individually.
However, by construction this mechanism
requires manual handling of the content to be output.

%%%%%%%%%%%%%%%%%%%%%%%%%%%%%%%%%%%%%%%%
\DescribeMacro{\ifchilddocmanual}
The main file should be prepared as usual, see \secref{sec:include}.
However, the document body must make a distinction
between processing of an individual part and of the main document, e.g.:
%
\begin{center}
\begin{tabular}{l}
|\ifchilddocmanual|\\
|\input{\childdocname}|\\
|\||else|\\
\textit{document body with }|\input{|\textit{part}|}|\\
|\||fi|
\end{tabular}
\end{center}
%
The conditional |\ifchilddocmanual| is true whenever
a part to be included by |\input| is being compiled,
and the name of the part is stored in |\childdocname|.

%%%%%%%%%%%%%%%%%%%%%%%%%%%%%%%%%%%%%%%%
\DescribeMacro{\childdocby}
Each part to be included by |\input| should start with:
%
\begin{center}
\begin{tabular}{l}
|\input{childdoc.def}|\\
|\childdocby{|\textit{main}|}|\\
\end{tabular}
\end{center}
%
The directive |\childdocby| is similar to |\childdocof|
described in \secref{sec:include},
but the subsequent selection of content must be done manually.
To that end, both |\ifchilddoc| and |\ifchilddocmanual|
will be true upon processing of a part,
and the name of the part is stored in |\childdocname|.
Note that |\jobname| will be set to the filename of the current part
so that each part receives an individual |.aux| file
that does not interfere with the |.aux| file(s) of the main document.
This behaviour can be altered by the alternative form
|\childdocby[*]{|\textit{main}|}| (with a non-empty optional argument)
which uses the |.aux| file of the main document
by setting |\jobname| to \textit{main}.

%%%%%%%%%%%%%%%%%%%%%%%%%%%%%%%%%%%%%%%%%%%%%%%%%%%%%%%%%%%%%%%%%%%%%%%%%%%%%%%%
\subsection{Driver Development}
\label{sec:driver}

The \textsf{childdoc} mechanism can also be use for the development
of definition files such as \LaTeX{} styles or classes.
This case differs from the above setup with multiple parts
included by |\include| in that no |\includeonly| should be invoked.
This can be achieved by starting the include file
(before |\ProvidesPackage|) with:
%
\begin{center}
\begin{tabular}{l}
|\input{childdoc.def}|\\
|\childdocforward{|\textit{main}|}|\\
\end{tabular}
\end{center}
%
or alternatively with:
%
\begin{center}
\begin{tabular}{l}
|\input{childdoc.def}|\\
|\childdocby{|\textit{main}|}|\\
\end{tabular}
\end{center}
%
Both forms have slightly different effects as described above.
The main file is prepared as usual, see \secref{sec:include}.

%%%%%%%%%%%%%%%%%%%%%%%%%%%%%%%%%%%%%%%%%%%%%%%%%%%%%%%%%%%%%%%%%%%%%%%%%%%%%%%%
\subsection{Legacy Detection}
\label{sec:detection}

The directive |\childdocmain| in the main file can detect
whether the complete document or merely a child is to be compiled
even without using the directive |\childdocof|.
This method is deprecated because it is less robust
and there is no compelling reason to use it;
it is merely provided for backward compatibility
and it may be removed in future versions.

If the detection mechanism is to be used,
it is mandatory to correctly specify
the filename of the main file as the argument of |\childdocmain|:
%
\begin{center}
\begin{tabular}{l}
|\input{childdoc.def}|\\
|\childdocmain{|\textit{main}|}|\\
\end{tabular}
\end{center}
%
If |\jobname| does not match the argument \textit{main} of |\childdocmain|,
it is assumed that |\jobname| points to the child file to be compiled.
When using |\childdocmain| with the main file specified as argument,
it suffices to start a child file
with just |\input{|\textit{main}|}|
without loading of the package and using |\childdocof|.
If instead all processing is done
with the appropriate \textsf{childdoc} directives,
the argument of \textit{main} of |\childdocmain| can be empty.

An alternative version of the command line processing described
in \secref{sec:commandline} using the detection mechanism reads:
%
\begin{center}
|... -jobname "|\textit{target}|" "|[\textit{flags}]%
[|\def\jobname{|\textit{dest}|}|]|\input{|\textit{main}|}"|
\end{center}

%%%%%%%%%%%%%%%%%%%%%%%%%%%%%%%%%%%%%%%%%%%%%%%%%%%%%%%%%%%%%%%%%%%%%%%%%%%%%%%%
\subsection{Manual Code}
\label{sec:manual}

In case one cannot be certain whether the definitions file |childdoc.def|
is installed on the target \TeX{} distribution
and one prefers not to ship it,
it is conceivable to paste a few relevant commands into the sources.

To that end, drop all statements |\input{childdoc.def}|
and perform the replacements as outlined below.
Instead of |\childdocmain{|\textit{main}|}| add the following code
to the top of the main file:
%
\begin{center}
\begin{tabular}{l}
|\||ifdefined\childdocname\endinput\||fi\newif\ifchilddoc|\\
|\edef\childdocname{\scantokens\expandafter{\jobname\noexpand}}|\\
|\def\childdocmain{|\textit{main}|}\||ifx\childdocmain\childdocname\||else|\\
|\childdoctrue\includeonly{\childdocname}\let\jobname\childdocmain\||fi|\\
\end{tabular}
\end{center}
%
Instead of |\childdocof{|\textit{main}|}| just include the main file
at the top of each child file:
%
\begin{center}
|\input{|\textit{main}|}|
\end{center}
%
A simple redirection |\childdocforward{|\textit{dest}|}| is achieved by:
%
\begin{center}
|\def\jobname{|\textit{dest}|}\input{\jobname}|
\end{center}
%
The redirection with prefix
|\childdocforwardprefix[|\textit{prefix}|]{|\textit{dest}|}|
is accomplished by:
%
\begin{center}
\begin{tabular}{l}
|{\edef\jobname{\scantokens\expandafter{\jobname\noexpand}}|\\
|\def\redirectjob |\textit{prefix}|#1~~~{\gdef\jobname{|\textit{dest}|#1}}|\\
|\expandafter\redirectjob\jobname~~~}\input{\jobname}|
\end{tabular}
\end{center}

In an alternative approach,
child documents can be compiled by a specific command line
without additional code or specific definitions:
%
\begin{center}
|... -jobname "|\textit{target}|" "|[\textit{flags}]%
|\includeonly{|\textit{dest}|}\input{|\textit{main}|}"|
\end{center}
%

%%%%%%%%%%%%%%%%%%%%%%%%%%%%%%%%%%%%%%%%%%%%%%%%%%%%%%%%%%%%%%%%%%%%%%%%%%%%%%%%
%%%%%%%%%%%%%%%%%%%%%%%%%%%%%%%%%%%%%%%%%%%%%%%%%%%%%%%%%%%%%%%%%%%%%%%%%%%%%%%%
\section{Information}

%%%%%%%%%%%%%%%%%%%%%%%%%%%%%%%%%%%%%%%%%%%%%%%%%%%%%%%%%%%%%%%%%%%%%%%%%%%%%%%%
\subsection{Copyright}

Copyright \copyright{} 2017--2018 Niklas Beisert

This work may be distributed and/or modified under the
conditions of the \LaTeX{} Project Public License, either version 1.3
of this license or (at your option) any later version.
The latest version of this license is in
  \url{http://www.latex-project.org/lppl.txt}
and version 1.3 or later is part of all distributions of \LaTeX{}
version 2005/12/01 or later.

This work has the LPPL maintenance status `maintained'.

The Current Maintainer of this work is Niklas Beisert.

This work consists of the files |README.txt|, |childdoc.ins| and |childdoc.dtx|
as well as the derived files |childdoc.def|, |cdocsamp.tex|
with |cdocsch1.tex|, |cdocsch2.tex|, |cdocspt3.tex|, |cdocspt4.tex|,
|cdocsdrf.tex|, |cdocsfn1.tex|, |cdocsfn2.tex|
as well as |childdoc.pdf|.

%%%%%%%%%%%%%%%%%%%%%%%%%%%%%%%%%%%%%%%%%%%%%%%%%%%%%%%%%%%%%%%%%%%%%%%%%%%%%%%%
\subsection{Files and Installation}

The package consists of the files:
%
\begin{center}
\begin{tabular}{ll}
    |README.txt|   & readme file \\
    |childdoc.ins| & installation file \\
    |childdoc.dtx| & source file \\
    |childdoc.def| & definition file \\
    |cdocsamp.tex| & sample main file \\
    |cdocsch1.tex| & sample include file \\
    |cdocsch2.tex| & sample include file \\
    |cdocspt3.tex| & sample part file \\
    |cdocspt4.tex| & sample part file \\
    |cdocsdrf.tex| & sample redirection file \\
    |cdocsfn1.tex| & sample redirection file \\
    |cdocsfn2.tex| & sample redirection file \\
    |childdoc.pdf| & manual
\end{tabular}
\end{center}
%
The distribution consists of the files
|README.txt|, |childdoc.ins| and |childdoc.dtx|.
%
\begin{itemize}
\item
Run (pdf)\LaTeX{} on |childdoc.dtx|
to compile the manual |childdoc.pdf| (this file).
\item
Run \LaTeX{} on |childdoc.ins| to create the definitions file |childdoc.def|
and the sample |cdocsamp.tex| with include files
|cdocsch1.tex|, |cdocsch2.tex|, |cdocspt3.tex|, |cdocspt4.tex|,
|cdocsdrf.tex|, |cdocsfn1.tex|, |cdocsfn2.tex|.
Then copy the file |childdoc.def| to an appropriate directory of your \LaTeX{}
distribution, e.g.\ \textit{texmf-root}|/tex/latex/childdoc|.
\end{itemize}

%%%%%%%%%%%%%%%%%%%%%%%%%%%%%%%%%%%%%%%%%%%%%%%%%%%%%%%%%%%%%%%%%%%%%%%%%%%%%%%%
\subsection{Related CTAN Packages}

There are several other packages which offer a similar functionality:
%
\begin{itemize}
\item
The packages
\href{http://ctan.org/pkg/docmute}{\textsf{docmute}},
\href{http://ctan.org/pkg/includex}{\textsf{includex}} and
\href{http://ctan.org/pkg/standalone}{\textsf{standalone}}
provide commands to include only the document body of
a child file thus allowing both files to be compiled individually.
\item
The packages \href{http://ctan.org/pkg/subdocs}{\textsf{subdocs}}
and \href{http://ctan.org/pkg/subfiles}{\textsf{subfiles}}
provide structures in which the main and child documents can be
encapsulated and allowing them to be compiled individually.
The inclusion mechanism is different from the conventional |\include|.
\item
The package \href{http://ctan.org/pkg/combine}{\textsf{combine}}
is an elaborate solution to combine several documents into one.
\end{itemize}
%
See also the CTAN topic \href{http://ctan.org/topic/subdocs}{\textsf{subdocs}}
for further related packages.
The present package differs from the above solutions in that
a document structure constructed with the conventional |\include| mechanism
just needs two extra commands at the top of every file
such that all constituent files can be compiled individually.

%%%%%%%%%%%%%%%%%%%%%%%%%%%%%%%%%%%%%%%%%%%%%%%%%%%%%%%%%%%%%%%%%%%%%%%%%%%%%%%%
%\subsection{Feature Suggestions}
%
%The following is a list of features which may be useful for future
%versions of this package:
%%
%\begin{itemize}
%\item
%\ldots
%\end{itemize}

%%%%%%%%%%%%%%%%%%%%%%%%%%%%%%%%%%%%%%%%%%%%%%%%%%%%%%%%%%%%%%%%%%%%%%%%%%%%%%%%
\subsection{Revision History}

%%%%%%%%%%%%%%%%%%%%%%%%%%%%%%%%%%%%%%%%
\paragraph{v2.0:} 2018/12/30

\begin{itemize}
\item
immediate forward processing
\item
added |\childdocby| mechanism
\item
manual restructured
\end{itemize}

%%%%%%%%%%%%%%%%%%%%%%%%%%%%%%%%%%%%%%%%
\paragraph{v1.6:} 2018/01/17

\begin{itemize}
\item
application for development of include files
\item
corrections to manual
\end{itemize}

%%%%%%%%%%%%%%%%%%%%%%%%%%%%%%%%%%%%%%%%
\paragraph{v1.5:} 2017/05/21

\begin{itemize}
\item
more complete structuring introduced
\item
|\childdocof| introduced
\item
|\childdoc| renamed to |\childdocmain|
\item
|\childredirect| renamed to |\childdocforward| and |\childdocforwardprefix|
and functionality expanded
\end{itemize}

%%%%%%%%%%%%%%%%%%%%%%%%%%%%%%%%%%%%%%%%
\paragraph{v1.0:} 2017/04/27

\begin{itemize}
\item
manual and install package
\item
first version published on CTAN
\end{itemize}

%%%%%%%%%%%%%%%%%%%%%%%%%%%%%%%%%%%%%%%%
\paragraph{v0.6:} 2017/04/26

\begin{itemize}
\item
redirection mechanism added
\end{itemize}

%%%%%%%%%%%%%%%%%%%%%%%%%%%%%%%%%%%%%%%%
\paragraph{v0.5:} 2017/04/26

\begin{itemize}
\item
functionality in definition file
\end{itemize}


%%%%%%%%%%%%%%%%%%%%%%%%%%%%%%%%%%%%%%%%%%%%%%%%%%%%%%%%%%%%%%%%%%%%%%%%%%%%%%%%
%%%%%%%%%%%%%%%%%%%%%%%%%%%%%%%%%%%%%%%%%%%%%%%%%%%%%%%%%%%%%%%%%%%%%%%%%%%%%%%%
%%%%%%%%%%%%%%%%%%%%%%%%%%%%%%%%%%%%%%%%%%%%%%%%%%%%%%%%%%%%%%%%%%%%%%%%%%%%%%%%
\appendix

\settowidth\MacroIndent{\rmfamily\scriptsize 000\ }

 \DocInput{childdoc.dtx}

\end{document}
%</driver>
% \fi
%
% %%%%%%%%%%%%%%%%%%%%%%%%%%%%%%%%%%%%%%%%%%%%%%%%%%%%%%%%%%%%%%%%%%%%%%%%%%%%%%
% %%%%%%%%%%%%%%%%%%%%%%%%%%%%%%%%%%%%%%%%%%%%%%%%%%%%%%%%%%%%%%%%%%%%%%%%%%%%%%
% \section{Sample}
%\iffalse
%<*samplemain>
%\fi
%
% The following presents a sample document
% with two chapters, two parts, a title page,
% a compile flag as well as three forwarding files to set the flag.
% It consists of eight |.tex| files:
% \begin{center}
% \begin{tabular}{ll}
% |cdocsamp.tex|&main file\\
% |cdocsch1.tex|&include file for chapter 1\\
% |cdocsch2.tex|&include file for chapter 2\\
% |cdocspt3.tex|&include file for part 3\\
% |cdocspt4.tex|&include file for part 4\\
% |cdocsdrf.tex|&forwarding file for main file in draft mode\\
% |cdocsfi1.tex|&forwarding file for final version of chapter 1\\
% |cdocsfi2.tex|&forwarding file for final version of chapter 2\\
% \end{tabular}
% \end{center}
% Each of the eight files can be compiled directly by the \LaTeX{} compiler.
%
% %%%%%%%%%%%%%%%%%%%%%%%%%%%%%%%%%%%%%%
% \paragraph{Main File.}
%
% The main file is called |cdocsamp.tex|.
%
% Load the \textsf{childdoc} definitions and
% declare the filename for the main document:
%    \begin{macrocode}
\input{childdoc.def}
\childdocmain{}
%    \end{macrocode}

% Optional override for |\version| flag:
%    \begin{macrocode}
%%\ifchilddoc\else\providecommand{\version}{draft}\fi
%    \end{macrocode}

% Define the default values for the |\version| flag
% (|final| for the main file and |draft| for childs):
%    \begin{macrocode}
\ifchilddoc
\providecommand{\version}{draft}
\else
\providecommand{\version}{final}
\fi
%    \end{macrocode}

% Load the standard document class:
%    \begin{macrocode}
\documentclass[12pt]{article}
%    \end{macrocode}

% Start the document body:
%    \begin{macrocode}
\begin{document}
%    \end{macrocode}

% Declare a title page.
% Print title, part of document being processed and version flag:
%    \begin{macrocode}
\addtocounter{page}{-1}
\begin{center}
{\LARGE\bfseries{}childdoc example\par}
\vspace{1cm}
\ifchilddoc
\ifchilddocmanual part\else chapter\fi:
`\childdocname' of `\childdocjob'\par
\else
main document: `\childdocjob'\par
\fi
version: \version\par
\end{center}
\newpage
%    \end{macrocode}

% Manually include selected file,
% otherwise process as usual:
%    \begin{macrocode}
\ifchilddocmanual
\section*{part `\childdocname'}
\input{\childdocname}
\else
%    \end{macrocode}

% Include the two chapters:
%    \begin{macrocode}
\include{cdocsch1}
\include{cdocsch2}
%    \end{macrocode}

% Include the two parts unless only chapters should be displayed:
%    \begin{macrocode}
\ifchilddoc\else
\section{part three}
\input{cdocspt3}
\section{part four}
\input{cdocspt4}
\fi
%    \end{macrocode}

% Process as usual until here:
%    \begin{macrocode}
\fi
%    \end{macrocode}

% End of document body:
%    \begin{macrocode}
\end{document}
%    \end{macrocode}
%\iffalse
%</samplemain>
%\fi
%
% %%%%%%%%%%%%%%%%%%%%%%%%%%%%%%%%%%%%%%
% \paragraph{Chapter Include Files.}
%
% The include files are called |cdocsch1.tex| and |cdocsch2.tex|.
%
%\iffalse
%<*samplechap1|samplechap2>
%\fi

% Optional override for |\version| flag:
%    \begin{macrocode}
%%\providecommand{\version}{final}
%    \end{macrocode}

% Include the main document:
%    \begin{macrocode}
\input{childdoc.def}
\childdocof{cdocsamp}
%    \end{macrocode}

%\iffalse
%</samplechap1|samplechap2>
%\fi
%
%\iffalse
%<*samplechap1>
%\fi
% Some text for chapter 1:
%    \begin{macrocode}
\section{one}
some text in chapter one
%    \end{macrocode}

%\iffalse
%</samplechap1>
%\fi
% Some text for chapter 2:
%\iffalse
%<*samplechap2>
%\fi
%    \begin{macrocode}
\section{two}
more text in chapter two
%    \end{macrocode}

%\iffalse
%</samplechap2>
%\fi
%
% %%%%%%%%%%%%%%%%%%%%%%%%%%%%%%%%%%%%%%
% \paragraph{Part Include Files.}
%
% The include files are called |cdocspt3.tex| and |cdocspt4.tex|.
%
%\iffalse
%<*samplepart3|samplepart4>
%\fi

% Optional override for |\version| flag:
%    \begin{macrocode}
%%\providecommand{\version}{final}
%    \end{macrocode}

% Include the main document:
%    \begin{macrocode}
\input{childdoc.def}
\childdocby{cdocsamp}
%    \end{macrocode}

%\iffalse
%</samplepart3|samplepart4>
%\fi
%
%\iffalse
%<*samplepart3>
%\fi
% Some text for part 3:
%    \begin{macrocode}
some text in part three
%    \end{macrocode}

%\iffalse
%</samplepart3>
%\fi
% Some text for part 4:
%\iffalse
%<*samplepart4>
%\fi
%    \begin{macrocode}
more text in part four
%    \end{macrocode}

%\iffalse
%</samplepart4>
%\fi
%
% %%%%%%%%%%%%%%%%%%%%%%%%%%%%%%%%%%%%%%
% \paragraph{Forwarding for a Complete Draft.}
%
% The following forwarding file |cdocsdrf.tex|
% compiles the main document in draft mode:
%\iffalse
%<*sampledraft>
%\fi
%    \begin{macrocode}
\def\version{draft}
\input{childdoc.def}
\childdocforward{cdocsamp}
%    \end{macrocode}

%\iffalse
%</sampledraft>
%\fi
%
% %%%%%%%%%%%%%%%%%%%%%%%%%%%%%%%%%%%%%%
% \paragraph{Forwarding for Final Version of the Chapters.}
%
% The following forwarding files |cdocsfn1.tex| and |cdocsfn2.tex|
% (with identical content)
% compile the final versions of the child documents
% |cdocsch1.tex| and |cdocsch2.tex|, respectively:
%\iffalse
%<*samplefinal>
%\fi
%    \begin{macrocode}
\def\version{final}
\input{childdoc.def}
\childdocforwardprefix[cdocsamp]{cdocsfn}{cdocsch}
%    \end{macrocode}

%\iffalse
%</samplefinal>
%\fi
%
% %%%%%%%%%%%%%%%%%%%%%%%%%%%%%%%%%%%%%%
% \paragraph{Command Line Processing.}
%
% The following three command lines generate the output files
% |cdocscld|, |cdocscl1| and |cdocscl2|
% which should be identical to
% |cdocsdrf|, |cdocsch1| and |cdocsfn2|, respectively:
% \begin{center}
% \begin{tabular}{l}
% |latex -jobname cdocscld \|\\
% |  "\def\version{draft}\input{childdoc.def}\childdocforward{cdocsamp}"|\\
% |latex -jobname cdocscl1 \|\\
% |  "\input{childdoc.def}\childdocforward[cdocsamp]{cdocsch1}"|\\
% |latex -jobname cdocscl2 \|\\
% |  "\def\version{final}\input{childdoc.def}\childdocforward{cdocsch2}"|
% \end{tabular}
% \end{center}
% Note that the trailing backslash on each first line
% merely continues the input to the second line
% (for convenient cut ant paste).
% Furthermore, the command |latex| can be replaced by any
% of its alternative versions such as |pdflatex|.
%
% %%%%%%%%%%%%%%%%%%%%%%%%%%%%%%%%%%%%%%%%%%%%%%%%%%%%%%%%%%%%%%%%%%%%%%%%%%%%%%
% %%%%%%%%%%%%%%%%%%%%%%%%%%%%%%%%%%%%%%%%%%%%%%%%%%%%%%%%%%%%%%%%%%%%%%%%%%%%%%
% \section{Implementation}
%\iffalse
%<*package>
%\fi
%
% This section describes the definitions file |childdoc.def|.

% The definitions cannot be loaded using |\usepackage| or |\RequirePackage|
% which has a mechanism to prevent loading a style file more than once.
% When loading the definitions by means of |\input|
% multiple instances have to be prevented manually:
%\iffalse
%This code needs to be before the `\ProvidesFile' directive
%which is defined at the beginning of this file.
%Therefore it is also placed there and commented out here.
%</package>
%<*discard>
%\fi
%    \begin{macrocode}
\ifdefined\childdocmain\endinput\fi
%    \end{macrocode}
%\iffalse
%</discard>
%<*package>
%\fi
%
% \macro{\ifchilddoc}
% \macro{\ifchilddocmanual}
% The conditional |\ifchilddoc| tells whether a
% child (true) or main (false) document is being compiled.
% The conditional |\ifchilddocmanual| tells whether
% the |\includeonly| mechanism is used (false) or
% the selection of child files must be performed manually (true).
% The definitions initialise to false:
%    \begin{macrocode}
\newif\ifchilddoc
\newif\ifchilddocmanual
%    \end{macrocode}

% \macro{\childdocname}
% \macro{\childdocjob}
% The macro |\childdocname| stores the name of the main document
% to be compiled. The macro |\childdocjob| stores the name of
% the document on which the \LaTeX{} compiler was originally invoked.
% The content of |\jobname| cannot be compared
% to filenames specified in the source due to different catcodes.
% The following code rescans |\jobname|, stores the result
% in |\childdocname| and saves a copy in |\childdocjob|:
%    \begin{macrocode}
\edef\childdocname{\scantokens\expandafter{\jobname\noexpand}}
\let\childdocjob\childdocname
%    \end{macrocode}

% \macro{\childdocdisable}
% The macro |\childdocdisable| prevents the main file
% from being processed more than once.
% At this stage, the main document command |\childdocmain|
% is assumed to be called once again where it should do nothing.
% Any subsequent call to it should prevent
% a secondary processing of the main document
% It overwrites the forwarding commands
% |\childdocof| and |\childdocforward|
% with empty macros to prevent further inclusions of the main document:
%    \begin{macrocode}
\newcommand{\childdocdisable}
{
  \renewcommand{\childdocmain}[1]{\renewcommand{\childdocmain}[1]{\endinput}}
  \renewcommand{\childdocof}[1]{}
  \renewcommand{\childdocby}[2][]{}
  \renewcommand{\childdocforward}[2][]{}
  \renewcommand{\childdocdisable}{}
}
%    \end{macrocode}

% \macro{\childdocmain}
% The macro |\childdocmain| is to be called at the top of the main file
% with nothing or the main filename (without extension) as argument.
% First, it breaks loops.
% If the argument is not empty and does not match |\childdocname|
% (which is set by the first inclusion of |childdoc.def|),
% |\ifchilddoc| is set to true, |\includeonly| is applied to the child file
% and |\jobname| is set to the main file
% (for proper handling of |.aux| files):
%    \begin{macrocode}
\newcommand{\childdocmain}[1]
{
  \childdocdisable\childdocmain{}
  \if?#1?\else
    \begingroup
      \def\childdoctmp{#1}
      \ifx\childdoctmp\childdocname
        \def\childdoctmp{}
      \else
        \def\childdoctmp
        {
          \childdoctrue
          \includeonly{\childdocname}
          \def\childdocjob{#1}
          \def\jobname{#1}
        }
      \fi
      \expandafter
    \endgroup
    \childdoctmp
  \fi
}
%    \end{macrocode}

% \macro{\childdocof}
% The command |\childdocof| redirects
% compilation to the main file |#1|.
%    \begin{macrocode}
\newcommand{\childdocof}[1]
{
  \childdocdisable
  \childdoctrue
  \includeonly{\childdocname}
  \def\jobname{#1}
  \def\childdocjob{#1}
  \input{#1}
}
%    \end{macrocode}

% \macro{\childdocby}
% The command |\childdocby| ....
%    \begin{macrocode}
\newcommand{\childdocby}[2][]
{
  \childdocdisable
  \childdoctrue
  \childdocmanualtrue
  \if?#1?\else
    \def\jobname{#2}
  \fi
  \def\childdocjob{#2}
  \input{#2}
  \endinput
}
%    \end{macrocode}

% \macro{\childdocforward}
% The command |\childdocforward| redirects
% compilation to the main file or
% (if the optional argument is given) a child file.
% Parameters are set as if the main file
% or a child file starting with |\childdocof| was compiled.
% Then compilation is handed over to the main file:
%    \begin{macrocode}
\newcommand{\childdocforward}[2][]
{
  \begingroup
    \if?#1?
      \def\childdoctmp
      {
        \def\childdocname{#2}
        \def\childdocjob{#2}
        \def\jobname{#2}
        \input{#2}
        \endinput
      }
    \else
      \def\childdoctmp
      {
        \childdocdisable
        \def\childdocname{#2}
        \childdoctrue
        \includeonly{#2}
        \def\childdocjob{#1}
        \def\jobname{#1}
        \input{#1}
        \endinput
      }
    \fi
    \expandafter
  \endgroup
  \childdoctmp
}
%    \end{macrocode}

% \macro{\childdocforwardprefix}
% The command |\childdocforwardprefix| redirects
% compilation to the main or a child file by means of a pattern.
% The prefix |#1| in the current filename is replaced by |#2|
% and the suffix of the current filename is kept
% (it is assumed that the filename does not contain the substring `|~~~|'
% which is used as a delimiter).
% Compilation is handed over to the new file by |\childdocforward|:
%    \begin{macrocode}
\newcommand{\childdocforwardprefix}[3][]
{
  \begingroup
    \def\childdocextract #2##1~~~{\def\childdoctmp{\childdocforward[#1]{#3##1}}}
    \expandafter\childdocextract\childdocname~~~
    \expandafter
  \endgroup
  \childdoctmp
}
%    \end{macrocode}

% \macro{\childdoc}
% The deprecated macro |\childdoc| is a legacy version of |\childdocmain|:
%    \begin{macrocode}
\newcommand{\childdoc}{\childdocmain}
%    \end{macrocode}

% \macro{\childdocredirect}
% The deprecated macro |\childdocredirect| is a legacy version
% of |\childdocforward| and |\childdocforwardprefix|:
%    \begin{macrocode}
\newcommand{\childdocredirect}[2][]
{
  \begingroup
    \if?#1?
      \def\childdoctmp{\childdocforward{#2}}
    \else
      \def\childdoctmp{\childdocforwardprefix{#1}{#2}}
    \fi
    \expandafter
  \endgroup
  \childdoctmp
}
%    \end{macrocode}

%\iffalse
%</package>
%\fi
%
\endinput

\childdocmain{}
%    \end{macrocode}

% Optional override for |\version| flag:
%    \begin{macrocode}
%%\ifchilddoc\else\providecommand{\version}{draft}\fi
%    \end{macrocode}

% Define the default values for the |\version| flag
% (|final| for the main file and |draft| for childs):
%    \begin{macrocode}
\ifchilddoc
\providecommand{\version}{draft}
\else
\providecommand{\version}{final}
\fi
%    \end{macrocode}

% Load the standard document class:
%    \begin{macrocode}
\documentclass[12pt]{article}
%    \end{macrocode}

% Start the document body:
%    \begin{macrocode}
\begin{document}
%    \end{macrocode}

% Declare a title page.
% Print title, part of document being processed and version flag:
%    \begin{macrocode}
\addtocounter{page}{-1}
\begin{center}
{\LARGE\bfseries{}childdoc example\par}
\vspace{1cm}
\ifchilddoc
\ifchilddocmanual part\else chapter\fi:
`\childdocname' of `\childdocjob'\par
\else
main document: `\childdocjob'\par
\fi
version: \version\par
\end{center}
\newpage
%    \end{macrocode}

% Manually include selected file,
% otherwise process as usual:
%    \begin{macrocode}
\ifchilddocmanual
\section*{part `\childdocname'}
\input{\childdocname}
\else
%    \end{macrocode}

% Include the two chapters:
%    \begin{macrocode}
\include{cdocsch1}
\include{cdocsch2}
%    \end{macrocode}

% Include the two parts unless only chapters should be displayed:
%    \begin{macrocode}
\ifchilddoc\else
\section{part three}
\input{cdocspt3}
\section{part four}
\input{cdocspt4}
\fi
%    \end{macrocode}

% Process as usual until here:
%    \begin{macrocode}
\fi
%    \end{macrocode}

% End of document body:
%    \begin{macrocode}
\end{document}
%    \end{macrocode}
%\iffalse
%</samplemain>
%\fi
%
% %%%%%%%%%%%%%%%%%%%%%%%%%%%%%%%%%%%%%%
% \paragraph{Chapter Include Files.}
%
% The include files are called |cdocsch1.tex| and |cdocsch2.tex|.
%
%\iffalse
%<*samplechap1|samplechap2>
%\fi

% Optional override for |\version| flag:
%    \begin{macrocode}
%%\providecommand{\version}{final}
%    \end{macrocode}

% Include the main document:
%    \begin{macrocode}
% \iffalse
%
% childdoc.dtx Copyright (C) 2017-2018 Niklas Beisert
%
% This work may be distributed and/or modified under the
% conditions of the LaTeX Project Public License, either version 1.3
% of this license or (at your option) any later version.
% The latest version of this license is in
%   http://www.latex-project.org/lppl.txt
% and version 1.3 or later is part of all distributions of LaTeX
% version 2005/12/01 or later.
%
% This work has the LPPL maintenance status `maintained'.
%
% The Current Maintainer of this work is Niklas Beisert.
%
% This work consists of the files childdoc.dtx and childdoc.ins
% and the derived files childdoc.def and cdocsamp.tex with
% cdocsch1.tex, cdocsch2.tex, cdocsdrf.tex, cdocsfn1.tex, cdocsfn2.tex.
%
%<package>\ifdefined\childdocmain\endinput\fi
%<package>\ProvidesFile{childdoc.def}[2018/12/30 v2.0 child document driver]
%<samplemain>\ProvidesFile{cdocsamp.tex}[2018/12/30 v2.0 sample for childdoc]
%<*driver>
%\ProvidesFile{childdoc.drv}[2018/12/30 v2.0 childdoc reference manual file]
\PassOptionsToClass{10pt,a4paper}{article}
\documentclass{ltxdoc}

\usepackage[margin=35mm]{geometry}
\usepackage{hyperref}
\usepackage{hyperxmp}
\usepackage[usenames]{color}

\hypersetup{colorlinks=true}
\hypersetup{pdfstartview=FitH}
\hypersetup{pdfpagemode=UseNone}
\hypersetup{pdfsource={}}
\hypersetup{pdflang={en-UK}}
\hypersetup{pdfcopyright={Copyright 2017-2018 Niklas Beisert.
  This work may be distributed and/or modified under the
  conditions of the LaTeX Project Public License, either version 1.3
  of this license or (at your option) any later version.}}
\hypersetup{pdflicenseurl={http://www.latex-project.org/lppl.txt}}
\hypersetup{pdfcontactaddress={ETH Zurich, ITP, HIT K,
  Wolfgang-Pauli-Strasse 27}}
\hypersetup{pdfcontactpostcode={8093}}
\hypersetup{pdfcontactcity={Zurich}}
\hypersetup{pdfcontactcountry={Switzerland}}
\hypersetup{pdfcontactemail={nbeisert@itp.phys.ethz.ch}}
\hypersetup{pdfcontacturl={http://people.phys.ethz.ch/\xmptilde nbeisert/}}

\newcommand{\secref}[1]{\hyperref[#1]{section \ref*{#1}}}

\parskip1ex
\parindent0pt
\let\olditemize\itemize
\def\itemize{\olditemize\parskip0pt}

\begin{document}

\title{The \textsf{childdoc} Package}
\hypersetup{pdftitle={The childdoc Package}}
\author{Niklas Beisert\\[2ex]
  Institut f\"ur Theoretische Physik\\
  Eidgen\"ossische Technische Hochschule Z\"urich\\
  Wolfgang-Pauli-Strasse 27, 8093 Z\"urich, Switzerland\\[1ex]
  \href{mailto:nbeisert@itp.phys.ethz.ch}
  {\texttt{nbeisert@itp.phys.ethz.ch}}}
\hypersetup{pdfauthor={Niklas Beisert}}
\hypersetup{pdfsubject={Manual for the LaTeX2e Package childdoc}}
\date{30 December 2018, \textsf{v2.0}}
\maketitle

\begin{abstract}\noindent
\textsf{childdoc} is a \LaTeXe{} package
that enables the direct compilation
of document sections included by |\include|
to individual files.
\end{abstract}

\begingroup
\parskip0ex
\tableofcontents
\endgroup

%%%%%%%%%%%%%%%%%%%%%%%%%%%%%%%%%%%%%%%%%%%%%%%%%%%%%%%%%%%%%%%%%%%%%%%%%%%%%%%%
%%%%%%%%%%%%%%%%%%%%%%%%%%%%%%%%%%%%%%%%%%%%%%%%%%%%%%%%%%%%%%%%%%%%%%%%%%%%%%%%
\section{Introduction}

\LaTeX{} provides a mechanism to structure a large document (such as a book)
into a main file and several child files (containing the chapters)
using the |\include| command.
This mechanism is beneficial for documents
which span hundreds of pages in order to
make the source file(s) more manageable.
Moreover, compilation can be restricted to
selected child files by means of the |\includeonly| command.
The latter feature can be used to reduce the compilation time while editing
(this was significantly more useful in the earlier days of \LaTeX{})
or to generate a smaller document which is easier to navigate.
Another application of |\includeonly| is to generate
documents consisting of selected parts of the complete document.

However, there are a few drawbacks of the plain |\include| mechanism:
\begin{itemize}
\item
The child files cannot be compiled on their own,
they can only be compiled via the main file.
A naive editing environment
(such as a text editor with an option
to have the current file processed by \LaTeX)
may require one to switch to the main file before compiling;
attempting to compile the child file produces errors.
\item
The main file must be modified (each time)
to adjust the |\includeonly| command
to the present needs. This easily leaves the main file in a messy state.
\item
The generated document will always carry the filename
of the main document. This is inconvenient if
several child files are to be compiled and
to be kept for distribution.
\end{itemize}

The present package provides a simple interface
to make child files individually compilable by \LaTeX{}.
Compiling a child file then has the same effect as compiling
the main file with an |\includeonly| command
to select the appropriate child.
Moreover the generated document will carry the name of the child
rather than the main file.
This resolves all three above issues.

This feature is meant to make the editing of books,
thesis documents and lecture notes somewhat more convenient.
However, the package can also be used efficiently for
composing a series of documents (such as exercise sheets)
which are typically distributed individually.
It then assists the author in generating the individual documents
(potentially in different versions)
as well as a document containing the collected series.
Another application is in developing style files
or other kinds of included material
where compilation of the style file could redirect
to a sample or test file.

%%%%%%%%%%%%%%%%%%%%%%%%%%%%%%%%%%%%%%%%%%%%%%%%%%%%%%%%%%%%%%%%%%%%%%%%%%%%%%%%
%%%%%%%%%%%%%%%%%%%%%%%%%%%%%%%%%%%%%%%%%%%%%%%%%%%%%%%%%%%%%%%%%%%%%%%%%%%%%%%%
\section{Usage}

First of all, the package \textsf{childdoc} is \emph{not} a standard
\LaTeXe{} |.sty| style file! Therefore it needs to be invoked in
a non-standard way.

%%%%%%%%%%%%%%%%%%%%%%%%%%%%%%%%%%%%%%%%%%%%%%%%%%%%%%%%%%%%%%%%%%%%%%%%%%%%%%%%
\subsection{Included Files}
\label{sec:include}

%%%%%%%%%%%%%%%%%%%%%%%%%%%%%%%%%%%%%%%%
\DescribeMacro{\childdocmain}
To use the package, add the commands
\begin{center}
\begin{tabular}{l}
|\input{childdoc.def}|\\
|\childdocmain{}|\\
\end{tabular}
\end{center}
at the very top of the main \LaTeX{} file,
in particular \emph{before} the |\documentclass| statement!
The argument of |\childdocmain| should be left empty
(but it must be present).

%%%%%%%%%%%%%%%%%%%%%%%%%%%%%%%%%%%%%%%%
\DescribeMacro{\childdocof}
Furthermore, add the commands
\begin{center}
\begin{tabular}{l}
|\input{childdoc.def}|\\
|\childdocof{|\textit{main}|}|\\
\end{tabular}
\end{center}
at the top of every child file \textit{child}
which is included by |\include{|\textit{child}|}|
from within the main file
(or at least for those files to be compiled individually).
The argument \textit{main} must be the filename of the main file.

There are a couple of
considerations in setting up the main and child documents:

%%%%%%%%%%%%%%%%%%%%%%%%%%%%%%%%%%%%%%%%
\paragraph{Restrictions.}

Please note the following restrictions:
\begin{itemize}
\item
|\childdocmain| must be called with one argument \textit{main}
to ensure compatibility with earlier version of the package.
It must either be empty (|\childdocmain{}|)
or precisely match the filename of the main file in which it is specified.
See \secref{sec:detection} for further information.
\item
The filename \textit{main} must be specified without the |.tex| extension.
\item
The filename \textit{main} is case sensitive
(even in case-insensitive file systems)
due to internal string comparison.
\item
The argument \textit{main} should be fully expanded, it cannot be a macro.
\item
Subdirectories and special characters should be avoided in filenames.
\item
The command |\childdocmain{|\textit{main}|}| must be followed by a whitespace.
It should not be followed immediately by another command
or by a comment mark `|%|'.
This is because the \TeX{} parser reads the token immediately following
the argument of |\childdocmain| and puts it
at the beginning of every child section;
however, a white\-space is ignored.
\end{itemize}

%%%%%%%%%%%%%%%%%%%%%%%%%%%%%%%%%%%%%%%%
\paragraph{Content of Main File.}

It is advisable to place all content in the child files included by |\include|.
Any output contained in the main file will appear in all child documents
unless suppressed manually;
it cannot be suppressed automatically by the |\includeonly| directive
and thus should normally be avoided.
A method to include some content in the main file
by means of conditional processing is described in \secref{sec:conditional}.

%%%%%%%%%%%%%%%%%%%%%%%%%%%%%%%%%%%%%%%%
\paragraph{Page Numbering.}

When only a part of the document is compiled,
the appropriate numbering of pages
(as well as other status parameters)
is determined from the |.aux| files.
The latter contain information from previous passes.
However this information needs to propagate through
all intermediate child documents.
Therefore the page numbering in child documents may well
be inconsistent until the complete document is compiled at least once.

A useful (if unconventional) way to always ensure a consistent
page numbering is to restart the numbering in each child document
and denote the pages by `\textit{child}|.|\textit{page}'
where \textit{child} represents the chapter/section number of the child file.
This can be achieved by the command
|\numberwithin{page}{|\textit{child}|}|
of the \textsf{amsmath} package
where \textit{child} can be |chapter| or |section|
depending on the chosen structuring.
Alternatively, one can modify the macro |\thepage| appropriately
and reset the counter |page| at the start of each child file.

%%%%%%%%%%%%%%%%%%%%%%%%%%%%%%%%%%%%%%%%%%%%%%%%%%%%%%%%%%%%%%%%%%%%%%%%%%%%%%%%
\subsection{Conditional Processing}
\label{sec:conditional}

The package provides a mechanism to compile different versions
of a document. To customise the versions further some conditional processing
can come in handy to distinguish which version is being compiled.
The package provides two macros to describe the compilation context:

%%%%%%%%%%%%%%%%%%%%%%%%%%%%%%%%%%%%%%%%
\DescribeMacro{\ifchilddoc}
The conditional |\ifchilddoc| distinguishes between the compilation of
child documents and the main document:
%
\begin{center}
|\ifchilddoc |\textit{child-code}| |[|\||else |\textit{main-code}]| \||fi|
\end{center}

%%%%%%%%%%%%%%%%%%%%%%%%%%%%%%%%%%%%%%%%
\DescribeMacro{\childdocname}
\DescribeMacro{\childdocjob}
The macro |\childdocname| contains the filename (without extension)
of the main or child file being processed.
Note that |\childdocjob| will always contain the name of the main file.

%%%%%%%%%%%%%%%%%%%%%%%%%%%%%%%%%%%%%%%%
\paragraph{Title Page.}

Conditional processing can be used to include a title or banner page
in the main document when proper precautions are taken.
Importantly, the code in the main file should ensure that the page counter
(as well as other status parameters which are stored in the |.aux| files)
takes the same value after the conditional processing.
Otherwise the page numbers may take divergent values
depending on which part is compiled.

For example, a title page could be declared by:
%
\begin{center}
\begin{tabular}{l}
|\ifchilddoc\||else|\\
|\addtocounter{page}{-1}|\\
\textit{code for title page}\\
|\newpage|\\
|\||fi|
\end{tabular}
\end{center}
%
A banner page for the child documents can be generated by:
%
\begin{center}
\begin{tabular}{l}
|\ifchilddoc|\\
|\addtocounter{page}{-1}|\\
\textit{code for banner page}\\
|\newpage|\\
|\||fi|
\end{tabular}
\end{center}
%
Here one could write a message such as:
\begin{center}
|This is the part \childdocname{} of \childdocjob{}.|
\end{center}

%%%%%%%%%%%%%%%%%%%%%%%%%%%%%%%%%%%%%%%%%%%%%%%%%%%%%%%%%%%%%%%%%%%%%%%%%%%%%%%%
\subsection{Flags}
\label{sec:flags}

The package makes it easy to generate different versions
of the main or child documents.
To this end compilation flags can be defined
and assigned different default values.
They will be particularly useful in conjunction
with the forwarding mechanism described in \secref{sec:forward}.

For example, it may be useful to have a flag |\version|
which can be set to |draft| or |final|.
The document source will contain some conditional code
depending on the value of |\version|.
Suppose further, the flag should default to |final| for the main file
and to |draft| for child files
which is a natural assignment for editing the document.
This is achieved by placing the following code
in the preamble of the main document
(below the |\childdocmain| directive):
%
\begin{center}
\begin{tabular}{l}
|\ifchilddoc|\\
|\providecommand{\version}{draft}|\\
|\||else|\\
|\providecommand{\version}{final}|\\
|\||fi|
\end{tabular}
\end{center}
%
The definition by |\providecommand| makes sure
that previous definitions are not overwritten.
Further statements |\providecommand{\version}{...}|
can thus be added before the above code to override it.

For the main file, one might add a line
(between |\childdocmain| and the above block)
%
\begin{center}
|%\ifchilddoc\||else\providecommand{\version}{draft}\||fi|
\end{center}
%
which can be uncommented to produce a draft version.
Likewise one can add a line to the very top of a child file
(above the |\childdocof{|\textit{main}|}| directive)
%
\begin{center}
|%\providecommand{\version}{final}|
\end{center}
%
which can be uncommented to produce the final version of this child document.

%%%%%%%%%%%%%%%%%%%%%%%%%%%%%%%%%%%%%%%%%%%%%%%%%%%%%%%%%%%%%%%%%%%%%%%%%%%%%%%%
\subsection{Forwarding}
\label{sec:forward}

Different versions of the main or child documents
using compilation flags as described in \secref{sec:flags}
can be (permanently) stored in different files
for convenient compilation, viewing and distribution.
To this end, the package defines a command
to pass on compilation to a different file:

%%%%%%%%%%%%%%%%%%%%%%%%%%%%%%%%%%%%%%%%
\DescribeMacro{\childdocforward}
The command |\childdocforward| redirects processing to
another source file:
%
\begin{center}
\begin{tabular}{l}
|\input{childdoc.def}|\\
|\childdocforward[|\textit{main}|]{|\textit{dest}|}|\\
\end{tabular}
\end{center}
%
The argument \textit{dest} is the destination file
(without extension).
It should be the main file or one of the child files.
Note that further \textsf{childdoc} directives
such as |\childdocof| and |\childdocforward|
in the indicated file will be processed in this form.
The optional argument \textit{main}
passes on directly to the main file \textit{main}
while pretending to compile the child \textit{dest}.
This form behaves as if \textit{dest}
issues |\childdocof{|\textit{main}|}| right away,
and no further \textsf{childdoc} directives will be processed.

%%%%%%%%%%%%%%%%%%%%%%%%%%%%%%%%%%%%%%%%
\DescribeMacro{\...prefix}
In the alternative form |\childdocforwardprefix|,
%
\begin{center}
\begin{tabular}{l}
|\input{childdoc.def}|\\
|\childdocforwardprefix[|\textit{main}|]{|\textit{prefix}|}{|\textit{dest}|}|
\end{tabular}
\end{center}
%
the destination file is determined by a pattern
depending on the current file:
To make this work, the current file must be called
`{\textit{prefix}\hspace{0.2em}\textit{suffix}}'
with \textit{prefix} matching precisely the argument.
Processing is then passed on to the file
`{\textit{dest}\hspace{0.2em}\textit{suffix}}'.
Surely, the same effect is achieved by
directly specifying the
argument `{\textit{dest}\hspace{0.2em}\textit{suffix}}'
in the first form.
However, that requires to set up a different file
for each child. With the alternative form of the command
all these files can have exactly the same content
which simplifies setting them up and maintaining them.

For example, the following file |draft.tex|
with a compilation flag |\version| as described in \secref{sec:flags}
compiles the main document as a draft:
%
\begin{center}
\begin{tabular}{l}
|\def\version{draft}|\\
|\input{childdoc.def}|\\
|\childdocforward{|\textit{main}|}|
\end{tabular}
\end{center}
%
Likewise, the following files |final|\textit{nn}|.tex|
compile the final version of the child document
|child|\textit{nn}|.tex|:
%
\begin{center}
\begin{tabular}{l}
|\def\version{final}|\\
|\input{childdoc.def}|\\
|\childdocforwardprefix{final}{child}|
\end{tabular}
\end{center}
%

Note that when several versions of a main file and/or of each child file
are to be generated, it may be convenient to set up a |Makefile| or
shell script to automatise the process.

%%%%%%%%%%%%%%%%%%%%%%%%%%%%%%%%%%%%%%%%%%%%%%%%%%%%%%%%%%%%%%%%%%%%%%%%%%%%%%%%
\subsection{Command Line Processing}
\label{sec:commandline}

The effect of redirection files can also be achieved by invoking
the \LaTeX{} compiler with a more elaborate command line.
Most conveniently this should be done as part
of a shell script or a |Makefile|.

When using \textsf{childdoc} in the main file, the following
command lines effectively perform a redirection
(note that depending on the shell being used,
backslashes may have to be doubled: `|\|' $\to$ `|\\|'):
%
\begin{center}
|... -jobname "|\textit{target}|" |\\|"|[\textit{flags}]%
|\input{childdoc.def}\childdocforward[|\textit{main}|]{|\textit{dest}|}"|
\end{center}
%
Here \textit{target} is the name of the output file,
\textit{main} is the name of the main file
and \textit{dest} is the name of the main or child file to be processed
(all filenames without extensions).
The optional argument \textit{main} can be omitted
if \textit{main} matches \textit{dest}.
Optionally, compilation \textit{flags} can be defined via |\def| commands.
This command line makes the \TeX{} engine believe
it is compiling the file \textit{target}
whose content is specified as the latter parameter.
The provided code then forwards the processing to
\textit{main} or \textit{dest} as described in \secref{sec:forward}.

%%%%%%%%%%%%%%%%%%%%%%%%%%%%%%%%%%%%%%%%%%%%%%%%%%%%%%%%%%%%%%%%%%%%%%%%%%%%%%%%
\subsection{Include by Input}
\label{sec:input}

Including child documents by |\include| has some restrictions by design.
Most notably, the content of a child document always occupies
its own set of pages; pages cannot be shared between child documents.
Usually, this behaviour makes perfect sense
because each child document contain an essential part of the document.
However, in some situations it may be desirable to compose
a document from a collection of parts
without having mandatory page breaks between then.
For this case, the package
provides a mechanism to include parts
by |\input| which can also be processed individually.
However, by construction this mechanism
requires manual handling of the content to be output.

%%%%%%%%%%%%%%%%%%%%%%%%%%%%%%%%%%%%%%%%
\DescribeMacro{\ifchilddocmanual}
The main file should be prepared as usual, see \secref{sec:include}.
However, the document body must make a distinction
between processing of an individual part and of the main document, e.g.:
%
\begin{center}
\begin{tabular}{l}
|\ifchilddocmanual|\\
|\input{\childdocname}|\\
|\||else|\\
\textit{document body with }|\input{|\textit{part}|}|\\
|\||fi|
\end{tabular}
\end{center}
%
The conditional |\ifchilddocmanual| is true whenever
a part to be included by |\input| is being compiled,
and the name of the part is stored in |\childdocname|.

%%%%%%%%%%%%%%%%%%%%%%%%%%%%%%%%%%%%%%%%
\DescribeMacro{\childdocby}
Each part to be included by |\input| should start with:
%
\begin{center}
\begin{tabular}{l}
|\input{childdoc.def}|\\
|\childdocby{|\textit{main}|}|\\
\end{tabular}
\end{center}
%
The directive |\childdocby| is similar to |\childdocof|
described in \secref{sec:include},
but the subsequent selection of content must be done manually.
To that end, both |\ifchilddoc| and |\ifchilddocmanual|
will be true upon processing of a part,
and the name of the part is stored in |\childdocname|.
Note that |\jobname| will be set to the filename of the current part
so that each part receives an individual |.aux| file
that does not interfere with the |.aux| file(s) of the main document.
This behaviour can be altered by the alternative form
|\childdocby[*]{|\textit{main}|}| (with a non-empty optional argument)
which uses the |.aux| file of the main document
by setting |\jobname| to \textit{main}.

%%%%%%%%%%%%%%%%%%%%%%%%%%%%%%%%%%%%%%%%%%%%%%%%%%%%%%%%%%%%%%%%%%%%%%%%%%%%%%%%
\subsection{Driver Development}
\label{sec:driver}

The \textsf{childdoc} mechanism can also be use for the development
of definition files such as \LaTeX{} styles or classes.
This case differs from the above setup with multiple parts
included by |\include| in that no |\includeonly| should be invoked.
This can be achieved by starting the include file
(before |\ProvidesPackage|) with:
%
\begin{center}
\begin{tabular}{l}
|\input{childdoc.def}|\\
|\childdocforward{|\textit{main}|}|\\
\end{tabular}
\end{center}
%
or alternatively with:
%
\begin{center}
\begin{tabular}{l}
|\input{childdoc.def}|\\
|\childdocby{|\textit{main}|}|\\
\end{tabular}
\end{center}
%
Both forms have slightly different effects as described above.
The main file is prepared as usual, see \secref{sec:include}.

%%%%%%%%%%%%%%%%%%%%%%%%%%%%%%%%%%%%%%%%%%%%%%%%%%%%%%%%%%%%%%%%%%%%%%%%%%%%%%%%
\subsection{Legacy Detection}
\label{sec:detection}

The directive |\childdocmain| in the main file can detect
whether the complete document or merely a child is to be compiled
even without using the directive |\childdocof|.
This method is deprecated because it is less robust
and there is no compelling reason to use it;
it is merely provided for backward compatibility
and it may be removed in future versions.

If the detection mechanism is to be used,
it is mandatory to correctly specify
the filename of the main file as the argument of |\childdocmain|:
%
\begin{center}
\begin{tabular}{l}
|\input{childdoc.def}|\\
|\childdocmain{|\textit{main}|}|\\
\end{tabular}
\end{center}
%
If |\jobname| does not match the argument \textit{main} of |\childdocmain|,
it is assumed that |\jobname| points to the child file to be compiled.
When using |\childdocmain| with the main file specified as argument,
it suffices to start a child file
with just |\input{|\textit{main}|}|
without loading of the package and using |\childdocof|.
If instead all processing is done
with the appropriate \textsf{childdoc} directives,
the argument of \textit{main} of |\childdocmain| can be empty.

An alternative version of the command line processing described
in \secref{sec:commandline} using the detection mechanism reads:
%
\begin{center}
|... -jobname "|\textit{target}|" "|[\textit{flags}]%
[|\def\jobname{|\textit{dest}|}|]|\input{|\textit{main}|}"|
\end{center}

%%%%%%%%%%%%%%%%%%%%%%%%%%%%%%%%%%%%%%%%%%%%%%%%%%%%%%%%%%%%%%%%%%%%%%%%%%%%%%%%
\subsection{Manual Code}
\label{sec:manual}

In case one cannot be certain whether the definitions file |childdoc.def|
is installed on the target \TeX{} distribution
and one prefers not to ship it,
it is conceivable to paste a few relevant commands into the sources.

To that end, drop all statements |\input{childdoc.def}|
and perform the replacements as outlined below.
Instead of |\childdocmain{|\textit{main}|}| add the following code
to the top of the main file:
%
\begin{center}
\begin{tabular}{l}
|\||ifdefined\childdocname\endinput\||fi\newif\ifchilddoc|\\
|\edef\childdocname{\scantokens\expandafter{\jobname\noexpand}}|\\
|\def\childdocmain{|\textit{main}|}\||ifx\childdocmain\childdocname\||else|\\
|\childdoctrue\includeonly{\childdocname}\let\jobname\childdocmain\||fi|\\
\end{tabular}
\end{center}
%
Instead of |\childdocof{|\textit{main}|}| just include the main file
at the top of each child file:
%
\begin{center}
|\input{|\textit{main}|}|
\end{center}
%
A simple redirection |\childdocforward{|\textit{dest}|}| is achieved by:
%
\begin{center}
|\def\jobname{|\textit{dest}|}\input{\jobname}|
\end{center}
%
The redirection with prefix
|\childdocforwardprefix[|\textit{prefix}|]{|\textit{dest}|}|
is accomplished by:
%
\begin{center}
\begin{tabular}{l}
|{\edef\jobname{\scantokens\expandafter{\jobname\noexpand}}|\\
|\def\redirectjob |\textit{prefix}|#1~~~{\gdef\jobname{|\textit{dest}|#1}}|\\
|\expandafter\redirectjob\jobname~~~}\input{\jobname}|
\end{tabular}
\end{center}

In an alternative approach,
child documents can be compiled by a specific command line
without additional code or specific definitions:
%
\begin{center}
|... -jobname "|\textit{target}|" "|[\textit{flags}]%
|\includeonly{|\textit{dest}|}\input{|\textit{main}|}"|
\end{center}
%

%%%%%%%%%%%%%%%%%%%%%%%%%%%%%%%%%%%%%%%%%%%%%%%%%%%%%%%%%%%%%%%%%%%%%%%%%%%%%%%%
%%%%%%%%%%%%%%%%%%%%%%%%%%%%%%%%%%%%%%%%%%%%%%%%%%%%%%%%%%%%%%%%%%%%%%%%%%%%%%%%
\section{Information}

%%%%%%%%%%%%%%%%%%%%%%%%%%%%%%%%%%%%%%%%%%%%%%%%%%%%%%%%%%%%%%%%%%%%%%%%%%%%%%%%
\subsection{Copyright}

Copyright \copyright{} 2017--2018 Niklas Beisert

This work may be distributed and/or modified under the
conditions of the \LaTeX{} Project Public License, either version 1.3
of this license or (at your option) any later version.
The latest version of this license is in
  \url{http://www.latex-project.org/lppl.txt}
and version 1.3 or later is part of all distributions of \LaTeX{}
version 2005/12/01 or later.

This work has the LPPL maintenance status `maintained'.

The Current Maintainer of this work is Niklas Beisert.

This work consists of the files |README.txt|, |childdoc.ins| and |childdoc.dtx|
as well as the derived files |childdoc.def|, |cdocsamp.tex|
with |cdocsch1.tex|, |cdocsch2.tex|, |cdocspt3.tex|, |cdocspt4.tex|,
|cdocsdrf.tex|, |cdocsfn1.tex|, |cdocsfn2.tex|
as well as |childdoc.pdf|.

%%%%%%%%%%%%%%%%%%%%%%%%%%%%%%%%%%%%%%%%%%%%%%%%%%%%%%%%%%%%%%%%%%%%%%%%%%%%%%%%
\subsection{Files and Installation}

The package consists of the files:
%
\begin{center}
\begin{tabular}{ll}
    |README.txt|   & readme file \\
    |childdoc.ins| & installation file \\
    |childdoc.dtx| & source file \\
    |childdoc.def| & definition file \\
    |cdocsamp.tex| & sample main file \\
    |cdocsch1.tex| & sample include file \\
    |cdocsch2.tex| & sample include file \\
    |cdocspt3.tex| & sample part file \\
    |cdocspt4.tex| & sample part file \\
    |cdocsdrf.tex| & sample redirection file \\
    |cdocsfn1.tex| & sample redirection file \\
    |cdocsfn2.tex| & sample redirection file \\
    |childdoc.pdf| & manual
\end{tabular}
\end{center}
%
The distribution consists of the files
|README.txt|, |childdoc.ins| and |childdoc.dtx|.
%
\begin{itemize}
\item
Run (pdf)\LaTeX{} on |childdoc.dtx|
to compile the manual |childdoc.pdf| (this file).
\item
Run \LaTeX{} on |childdoc.ins| to create the definitions file |childdoc.def|
and the sample |cdocsamp.tex| with include files
|cdocsch1.tex|, |cdocsch2.tex|, |cdocspt3.tex|, |cdocspt4.tex|,
|cdocsdrf.tex|, |cdocsfn1.tex|, |cdocsfn2.tex|.
Then copy the file |childdoc.def| to an appropriate directory of your \LaTeX{}
distribution, e.g.\ \textit{texmf-root}|/tex/latex/childdoc|.
\end{itemize}

%%%%%%%%%%%%%%%%%%%%%%%%%%%%%%%%%%%%%%%%%%%%%%%%%%%%%%%%%%%%%%%%%%%%%%%%%%%%%%%%
\subsection{Related CTAN Packages}

There are several other packages which offer a similar functionality:
%
\begin{itemize}
\item
The packages
\href{http://ctan.org/pkg/docmute}{\textsf{docmute}},
\href{http://ctan.org/pkg/includex}{\textsf{includex}} and
\href{http://ctan.org/pkg/standalone}{\textsf{standalone}}
provide commands to include only the document body of
a child file thus allowing both files to be compiled individually.
\item
The packages \href{http://ctan.org/pkg/subdocs}{\textsf{subdocs}}
and \href{http://ctan.org/pkg/subfiles}{\textsf{subfiles}}
provide structures in which the main and child documents can be
encapsulated and allowing them to be compiled individually.
The inclusion mechanism is different from the conventional |\include|.
\item
The package \href{http://ctan.org/pkg/combine}{\textsf{combine}}
is an elaborate solution to combine several documents into one.
\end{itemize}
%
See also the CTAN topic \href{http://ctan.org/topic/subdocs}{\textsf{subdocs}}
for further related packages.
The present package differs from the above solutions in that
a document structure constructed with the conventional |\include| mechanism
just needs two extra commands at the top of every file
such that all constituent files can be compiled individually.

%%%%%%%%%%%%%%%%%%%%%%%%%%%%%%%%%%%%%%%%%%%%%%%%%%%%%%%%%%%%%%%%%%%%%%%%%%%%%%%%
%\subsection{Feature Suggestions}
%
%The following is a list of features which may be useful for future
%versions of this package:
%%
%\begin{itemize}
%\item
%\ldots
%\end{itemize}

%%%%%%%%%%%%%%%%%%%%%%%%%%%%%%%%%%%%%%%%%%%%%%%%%%%%%%%%%%%%%%%%%%%%%%%%%%%%%%%%
\subsection{Revision History}

%%%%%%%%%%%%%%%%%%%%%%%%%%%%%%%%%%%%%%%%
\paragraph{v2.0:} 2018/12/30

\begin{itemize}
\item
immediate forward processing
\item
added |\childdocby| mechanism
\item
manual restructured
\end{itemize}

%%%%%%%%%%%%%%%%%%%%%%%%%%%%%%%%%%%%%%%%
\paragraph{v1.6:} 2018/01/17

\begin{itemize}
\item
application for development of include files
\item
corrections to manual
\end{itemize}

%%%%%%%%%%%%%%%%%%%%%%%%%%%%%%%%%%%%%%%%
\paragraph{v1.5:} 2017/05/21

\begin{itemize}
\item
more complete structuring introduced
\item
|\childdocof| introduced
\item
|\childdoc| renamed to |\childdocmain|
\item
|\childredirect| renamed to |\childdocforward| and |\childdocforwardprefix|
and functionality expanded
\end{itemize}

%%%%%%%%%%%%%%%%%%%%%%%%%%%%%%%%%%%%%%%%
\paragraph{v1.0:} 2017/04/27

\begin{itemize}
\item
manual and install package
\item
first version published on CTAN
\end{itemize}

%%%%%%%%%%%%%%%%%%%%%%%%%%%%%%%%%%%%%%%%
\paragraph{v0.6:} 2017/04/26

\begin{itemize}
\item
redirection mechanism added
\end{itemize}

%%%%%%%%%%%%%%%%%%%%%%%%%%%%%%%%%%%%%%%%
\paragraph{v0.5:} 2017/04/26

\begin{itemize}
\item
functionality in definition file
\end{itemize}


%%%%%%%%%%%%%%%%%%%%%%%%%%%%%%%%%%%%%%%%%%%%%%%%%%%%%%%%%%%%%%%%%%%%%%%%%%%%%%%%
%%%%%%%%%%%%%%%%%%%%%%%%%%%%%%%%%%%%%%%%%%%%%%%%%%%%%%%%%%%%%%%%%%%%%%%%%%%%%%%%
%%%%%%%%%%%%%%%%%%%%%%%%%%%%%%%%%%%%%%%%%%%%%%%%%%%%%%%%%%%%%%%%%%%%%%%%%%%%%%%%
\appendix

\settowidth\MacroIndent{\rmfamily\scriptsize 000\ }

 \DocInput{childdoc.dtx}

\end{document}
%</driver>
% \fi
%
% %%%%%%%%%%%%%%%%%%%%%%%%%%%%%%%%%%%%%%%%%%%%%%%%%%%%%%%%%%%%%%%%%%%%%%%%%%%%%%
% %%%%%%%%%%%%%%%%%%%%%%%%%%%%%%%%%%%%%%%%%%%%%%%%%%%%%%%%%%%%%%%%%%%%%%%%%%%%%%
% \section{Sample}
%\iffalse
%<*samplemain>
%\fi
%
% The following presents a sample document
% with two chapters, two parts, a title page,
% a compile flag as well as three forwarding files to set the flag.
% It consists of eight |.tex| files:
% \begin{center}
% \begin{tabular}{ll}
% |cdocsamp.tex|&main file\\
% |cdocsch1.tex|&include file for chapter 1\\
% |cdocsch2.tex|&include file for chapter 2\\
% |cdocspt3.tex|&include file for part 3\\
% |cdocspt4.tex|&include file for part 4\\
% |cdocsdrf.tex|&forwarding file for main file in draft mode\\
% |cdocsfi1.tex|&forwarding file for final version of chapter 1\\
% |cdocsfi2.tex|&forwarding file for final version of chapter 2\\
% \end{tabular}
% \end{center}
% Each of the eight files can be compiled directly by the \LaTeX{} compiler.
%
% %%%%%%%%%%%%%%%%%%%%%%%%%%%%%%%%%%%%%%
% \paragraph{Main File.}
%
% The main file is called |cdocsamp.tex|.
%
% Load the \textsf{childdoc} definitions and
% declare the filename for the main document:
%    \begin{macrocode}
\input{childdoc.def}
\childdocmain{}
%    \end{macrocode}

% Optional override for |\version| flag:
%    \begin{macrocode}
%%\ifchilddoc\else\providecommand{\version}{draft}\fi
%    \end{macrocode}

% Define the default values for the |\version| flag
% (|final| for the main file and |draft| for childs):
%    \begin{macrocode}
\ifchilddoc
\providecommand{\version}{draft}
\else
\providecommand{\version}{final}
\fi
%    \end{macrocode}

% Load the standard document class:
%    \begin{macrocode}
\documentclass[12pt]{article}
%    \end{macrocode}

% Start the document body:
%    \begin{macrocode}
\begin{document}
%    \end{macrocode}

% Declare a title page.
% Print title, part of document being processed and version flag:
%    \begin{macrocode}
\addtocounter{page}{-1}
\begin{center}
{\LARGE\bfseries{}childdoc example\par}
\vspace{1cm}
\ifchilddoc
\ifchilddocmanual part\else chapter\fi:
`\childdocname' of `\childdocjob'\par
\else
main document: `\childdocjob'\par
\fi
version: \version\par
\end{center}
\newpage
%    \end{macrocode}

% Manually include selected file,
% otherwise process as usual:
%    \begin{macrocode}
\ifchilddocmanual
\section*{part `\childdocname'}
\input{\childdocname}
\else
%    \end{macrocode}

% Include the two chapters:
%    \begin{macrocode}
\include{cdocsch1}
\include{cdocsch2}
%    \end{macrocode}

% Include the two parts unless only chapters should be displayed:
%    \begin{macrocode}
\ifchilddoc\else
\section{part three}
\input{cdocspt3}
\section{part four}
\input{cdocspt4}
\fi
%    \end{macrocode}

% Process as usual until here:
%    \begin{macrocode}
\fi
%    \end{macrocode}

% End of document body:
%    \begin{macrocode}
\end{document}
%    \end{macrocode}
%\iffalse
%</samplemain>
%\fi
%
% %%%%%%%%%%%%%%%%%%%%%%%%%%%%%%%%%%%%%%
% \paragraph{Chapter Include Files.}
%
% The include files are called |cdocsch1.tex| and |cdocsch2.tex|.
%
%\iffalse
%<*samplechap1|samplechap2>
%\fi

% Optional override for |\version| flag:
%    \begin{macrocode}
%%\providecommand{\version}{final}
%    \end{macrocode}

% Include the main document:
%    \begin{macrocode}
\input{childdoc.def}
\childdocof{cdocsamp}
%    \end{macrocode}

%\iffalse
%</samplechap1|samplechap2>
%\fi
%
%\iffalse
%<*samplechap1>
%\fi
% Some text for chapter 1:
%    \begin{macrocode}
\section{one}
some text in chapter one
%    \end{macrocode}

%\iffalse
%</samplechap1>
%\fi
% Some text for chapter 2:
%\iffalse
%<*samplechap2>
%\fi
%    \begin{macrocode}
\section{two}
more text in chapter two
%    \end{macrocode}

%\iffalse
%</samplechap2>
%\fi
%
% %%%%%%%%%%%%%%%%%%%%%%%%%%%%%%%%%%%%%%
% \paragraph{Part Include Files.}
%
% The include files are called |cdocspt3.tex| and |cdocspt4.tex|.
%
%\iffalse
%<*samplepart3|samplepart4>
%\fi

% Optional override for |\version| flag:
%    \begin{macrocode}
%%\providecommand{\version}{final}
%    \end{macrocode}

% Include the main document:
%    \begin{macrocode}
\input{childdoc.def}
\childdocby{cdocsamp}
%    \end{macrocode}

%\iffalse
%</samplepart3|samplepart4>
%\fi
%
%\iffalse
%<*samplepart3>
%\fi
% Some text for part 3:
%    \begin{macrocode}
some text in part three
%    \end{macrocode}

%\iffalse
%</samplepart3>
%\fi
% Some text for part 4:
%\iffalse
%<*samplepart4>
%\fi
%    \begin{macrocode}
more text in part four
%    \end{macrocode}

%\iffalse
%</samplepart4>
%\fi
%
% %%%%%%%%%%%%%%%%%%%%%%%%%%%%%%%%%%%%%%
% \paragraph{Forwarding for a Complete Draft.}
%
% The following forwarding file |cdocsdrf.tex|
% compiles the main document in draft mode:
%\iffalse
%<*sampledraft>
%\fi
%    \begin{macrocode}
\def\version{draft}
\input{childdoc.def}
\childdocforward{cdocsamp}
%    \end{macrocode}

%\iffalse
%</sampledraft>
%\fi
%
% %%%%%%%%%%%%%%%%%%%%%%%%%%%%%%%%%%%%%%
% \paragraph{Forwarding for Final Version of the Chapters.}
%
% The following forwarding files |cdocsfn1.tex| and |cdocsfn2.tex|
% (with identical content)
% compile the final versions of the child documents
% |cdocsch1.tex| and |cdocsch2.tex|, respectively:
%\iffalse
%<*samplefinal>
%\fi
%    \begin{macrocode}
\def\version{final}
\input{childdoc.def}
\childdocforwardprefix[cdocsamp]{cdocsfn}{cdocsch}
%    \end{macrocode}

%\iffalse
%</samplefinal>
%\fi
%
% %%%%%%%%%%%%%%%%%%%%%%%%%%%%%%%%%%%%%%
% \paragraph{Command Line Processing.}
%
% The following three command lines generate the output files
% |cdocscld|, |cdocscl1| and |cdocscl2|
% which should be identical to
% |cdocsdrf|, |cdocsch1| and |cdocsfn2|, respectively:
% \begin{center}
% \begin{tabular}{l}
% |latex -jobname cdocscld \|\\
% |  "\def\version{draft}\input{childdoc.def}\childdocforward{cdocsamp}"|\\
% |latex -jobname cdocscl1 \|\\
% |  "\input{childdoc.def}\childdocforward[cdocsamp]{cdocsch1}"|\\
% |latex -jobname cdocscl2 \|\\
% |  "\def\version{final}\input{childdoc.def}\childdocforward{cdocsch2}"|
% \end{tabular}
% \end{center}
% Note that the trailing backslash on each first line
% merely continues the input to the second line
% (for convenient cut ant paste).
% Furthermore, the command |latex| can be replaced by any
% of its alternative versions such as |pdflatex|.
%
% %%%%%%%%%%%%%%%%%%%%%%%%%%%%%%%%%%%%%%%%%%%%%%%%%%%%%%%%%%%%%%%%%%%%%%%%%%%%%%
% %%%%%%%%%%%%%%%%%%%%%%%%%%%%%%%%%%%%%%%%%%%%%%%%%%%%%%%%%%%%%%%%%%%%%%%%%%%%%%
% \section{Implementation}
%\iffalse
%<*package>
%\fi
%
% This section describes the definitions file |childdoc.def|.

% The definitions cannot be loaded using |\usepackage| or |\RequirePackage|
% which has a mechanism to prevent loading a style file more than once.
% When loading the definitions by means of |\input|
% multiple instances have to be prevented manually:
%\iffalse
%This code needs to be before the `\ProvidesFile' directive
%which is defined at the beginning of this file.
%Therefore it is also placed there and commented out here.
%</package>
%<*discard>
%\fi
%    \begin{macrocode}
\ifdefined\childdocmain\endinput\fi
%    \end{macrocode}
%\iffalse
%</discard>
%<*package>
%\fi
%
% \macro{\ifchilddoc}
% \macro{\ifchilddocmanual}
% The conditional |\ifchilddoc| tells whether a
% child (true) or main (false) document is being compiled.
% The conditional |\ifchilddocmanual| tells whether
% the |\includeonly| mechanism is used (false) or
% the selection of child files must be performed manually (true).
% The definitions initialise to false:
%    \begin{macrocode}
\newif\ifchilddoc
\newif\ifchilddocmanual
%    \end{macrocode}

% \macro{\childdocname}
% \macro{\childdocjob}
% The macro |\childdocname| stores the name of the main document
% to be compiled. The macro |\childdocjob| stores the name of
% the document on which the \LaTeX{} compiler was originally invoked.
% The content of |\jobname| cannot be compared
% to filenames specified in the source due to different catcodes.
% The following code rescans |\jobname|, stores the result
% in |\childdocname| and saves a copy in |\childdocjob|:
%    \begin{macrocode}
\edef\childdocname{\scantokens\expandafter{\jobname\noexpand}}
\let\childdocjob\childdocname
%    \end{macrocode}

% \macro{\childdocdisable}
% The macro |\childdocdisable| prevents the main file
% from being processed more than once.
% At this stage, the main document command |\childdocmain|
% is assumed to be called once again where it should do nothing.
% Any subsequent call to it should prevent
% a secondary processing of the main document
% It overwrites the forwarding commands
% |\childdocof| and |\childdocforward|
% with empty macros to prevent further inclusions of the main document:
%    \begin{macrocode}
\newcommand{\childdocdisable}
{
  \renewcommand{\childdocmain}[1]{\renewcommand{\childdocmain}[1]{\endinput}}
  \renewcommand{\childdocof}[1]{}
  \renewcommand{\childdocby}[2][]{}
  \renewcommand{\childdocforward}[2][]{}
  \renewcommand{\childdocdisable}{}
}
%    \end{macrocode}

% \macro{\childdocmain}
% The macro |\childdocmain| is to be called at the top of the main file
% with nothing or the main filename (without extension) as argument.
% First, it breaks loops.
% If the argument is not empty and does not match |\childdocname|
% (which is set by the first inclusion of |childdoc.def|),
% |\ifchilddoc| is set to true, |\includeonly| is applied to the child file
% and |\jobname| is set to the main file
% (for proper handling of |.aux| files):
%    \begin{macrocode}
\newcommand{\childdocmain}[1]
{
  \childdocdisable\childdocmain{}
  \if?#1?\else
    \begingroup
      \def\childdoctmp{#1}
      \ifx\childdoctmp\childdocname
        \def\childdoctmp{}
      \else
        \def\childdoctmp
        {
          \childdoctrue
          \includeonly{\childdocname}
          \def\childdocjob{#1}
          \def\jobname{#1}
        }
      \fi
      \expandafter
    \endgroup
    \childdoctmp
  \fi
}
%    \end{macrocode}

% \macro{\childdocof}
% The command |\childdocof| redirects
% compilation to the main file |#1|.
%    \begin{macrocode}
\newcommand{\childdocof}[1]
{
  \childdocdisable
  \childdoctrue
  \includeonly{\childdocname}
  \def\jobname{#1}
  \def\childdocjob{#1}
  \input{#1}
}
%    \end{macrocode}

% \macro{\childdocby}
% The command |\childdocby| ....
%    \begin{macrocode}
\newcommand{\childdocby}[2][]
{
  \childdocdisable
  \childdoctrue
  \childdocmanualtrue
  \if?#1?\else
    \def\jobname{#2}
  \fi
  \def\childdocjob{#2}
  \input{#2}
  \endinput
}
%    \end{macrocode}

% \macro{\childdocforward}
% The command |\childdocforward| redirects
% compilation to the main file or
% (if the optional argument is given) a child file.
% Parameters are set as if the main file
% or a child file starting with |\childdocof| was compiled.
% Then compilation is handed over to the main file:
%    \begin{macrocode}
\newcommand{\childdocforward}[2][]
{
  \begingroup
    \if?#1?
      \def\childdoctmp
      {
        \def\childdocname{#2}
        \def\childdocjob{#2}
        \def\jobname{#2}
        \input{#2}
        \endinput
      }
    \else
      \def\childdoctmp
      {
        \childdocdisable
        \def\childdocname{#2}
        \childdoctrue
        \includeonly{#2}
        \def\childdocjob{#1}
        \def\jobname{#1}
        \input{#1}
        \endinput
      }
    \fi
    \expandafter
  \endgroup
  \childdoctmp
}
%    \end{macrocode}

% \macro{\childdocforwardprefix}
% The command |\childdocforwardprefix| redirects
% compilation to the main or a child file by means of a pattern.
% The prefix |#1| in the current filename is replaced by |#2|
% and the suffix of the current filename is kept
% (it is assumed that the filename does not contain the substring `|~~~|'
% which is used as a delimiter).
% Compilation is handed over to the new file by |\childdocforward|:
%    \begin{macrocode}
\newcommand{\childdocforwardprefix}[3][]
{
  \begingroup
    \def\childdocextract #2##1~~~{\def\childdoctmp{\childdocforward[#1]{#3##1}}}
    \expandafter\childdocextract\childdocname~~~
    \expandafter
  \endgroup
  \childdoctmp
}
%    \end{macrocode}

% \macro{\childdoc}
% The deprecated macro |\childdoc| is a legacy version of |\childdocmain|:
%    \begin{macrocode}
\newcommand{\childdoc}{\childdocmain}
%    \end{macrocode}

% \macro{\childdocredirect}
% The deprecated macro |\childdocredirect| is a legacy version
% of |\childdocforward| and |\childdocforwardprefix|:
%    \begin{macrocode}
\newcommand{\childdocredirect}[2][]
{
  \begingroup
    \if?#1?
      \def\childdoctmp{\childdocforward{#2}}
    \else
      \def\childdoctmp{\childdocforwardprefix{#1}{#2}}
    \fi
    \expandafter
  \endgroup
  \childdoctmp
}
%    \end{macrocode}

%\iffalse
%</package>
%\fi
%
\endinput

\childdocof{cdocsamp}
%    \end{macrocode}

%\iffalse
%</samplechap1|samplechap2>
%\fi
%
%\iffalse
%<*samplechap1>
%\fi
% Some text for chapter 1:
%    \begin{macrocode}
\section{one}
some text in chapter one
%    \end{macrocode}

%\iffalse
%</samplechap1>
%\fi
% Some text for chapter 2:
%\iffalse
%<*samplechap2>
%\fi
%    \begin{macrocode}
\section{two}
more text in chapter two
%    \end{macrocode}

%\iffalse
%</samplechap2>
%\fi
%
% %%%%%%%%%%%%%%%%%%%%%%%%%%%%%%%%%%%%%%
% \paragraph{Part Include Files.}
%
% The include files are called |cdocspt3.tex| and |cdocspt4.tex|.
%
%\iffalse
%<*samplepart3|samplepart4>
%\fi

% Optional override for |\version| flag:
%    \begin{macrocode}
%%\providecommand{\version}{final}
%    \end{macrocode}

% Include the main document:
%    \begin{macrocode}
% \iffalse
%
% childdoc.dtx Copyright (C) 2017-2018 Niklas Beisert
%
% This work may be distributed and/or modified under the
% conditions of the LaTeX Project Public License, either version 1.3
% of this license or (at your option) any later version.
% The latest version of this license is in
%   http://www.latex-project.org/lppl.txt
% and version 1.3 or later is part of all distributions of LaTeX
% version 2005/12/01 or later.
%
% This work has the LPPL maintenance status `maintained'.
%
% The Current Maintainer of this work is Niklas Beisert.
%
% This work consists of the files childdoc.dtx and childdoc.ins
% and the derived files childdoc.def and cdocsamp.tex with
% cdocsch1.tex, cdocsch2.tex, cdocsdrf.tex, cdocsfn1.tex, cdocsfn2.tex.
%
%<package>\ifdefined\childdocmain\endinput\fi
%<package>\ProvidesFile{childdoc.def}[2018/12/30 v2.0 child document driver]
%<samplemain>\ProvidesFile{cdocsamp.tex}[2018/12/30 v2.0 sample for childdoc]
%<*driver>
%\ProvidesFile{childdoc.drv}[2018/12/30 v2.0 childdoc reference manual file]
\PassOptionsToClass{10pt,a4paper}{article}
\documentclass{ltxdoc}

\usepackage[margin=35mm]{geometry}
\usepackage{hyperref}
\usepackage{hyperxmp}
\usepackage[usenames]{color}

\hypersetup{colorlinks=true}
\hypersetup{pdfstartview=FitH}
\hypersetup{pdfpagemode=UseNone}
\hypersetup{pdfsource={}}
\hypersetup{pdflang={en-UK}}
\hypersetup{pdfcopyright={Copyright 2017-2018 Niklas Beisert.
  This work may be distributed and/or modified under the
  conditions of the LaTeX Project Public License, either version 1.3
  of this license or (at your option) any later version.}}
\hypersetup{pdflicenseurl={http://www.latex-project.org/lppl.txt}}
\hypersetup{pdfcontactaddress={ETH Zurich, ITP, HIT K,
  Wolfgang-Pauli-Strasse 27}}
\hypersetup{pdfcontactpostcode={8093}}
\hypersetup{pdfcontactcity={Zurich}}
\hypersetup{pdfcontactcountry={Switzerland}}
\hypersetup{pdfcontactemail={nbeisert@itp.phys.ethz.ch}}
\hypersetup{pdfcontacturl={http://people.phys.ethz.ch/\xmptilde nbeisert/}}

\newcommand{\secref}[1]{\hyperref[#1]{section \ref*{#1}}}

\parskip1ex
\parindent0pt
\let\olditemize\itemize
\def\itemize{\olditemize\parskip0pt}

\begin{document}

\title{The \textsf{childdoc} Package}
\hypersetup{pdftitle={The childdoc Package}}
\author{Niklas Beisert\\[2ex]
  Institut f\"ur Theoretische Physik\\
  Eidgen\"ossische Technische Hochschule Z\"urich\\
  Wolfgang-Pauli-Strasse 27, 8093 Z\"urich, Switzerland\\[1ex]
  \href{mailto:nbeisert@itp.phys.ethz.ch}
  {\texttt{nbeisert@itp.phys.ethz.ch}}}
\hypersetup{pdfauthor={Niklas Beisert}}
\hypersetup{pdfsubject={Manual for the LaTeX2e Package childdoc}}
\date{30 December 2018, \textsf{v2.0}}
\maketitle

\begin{abstract}\noindent
\textsf{childdoc} is a \LaTeXe{} package
that enables the direct compilation
of document sections included by |\include|
to individual files.
\end{abstract}

\begingroup
\parskip0ex
\tableofcontents
\endgroup

%%%%%%%%%%%%%%%%%%%%%%%%%%%%%%%%%%%%%%%%%%%%%%%%%%%%%%%%%%%%%%%%%%%%%%%%%%%%%%%%
%%%%%%%%%%%%%%%%%%%%%%%%%%%%%%%%%%%%%%%%%%%%%%%%%%%%%%%%%%%%%%%%%%%%%%%%%%%%%%%%
\section{Introduction}

\LaTeX{} provides a mechanism to structure a large document (such as a book)
into a main file and several child files (containing the chapters)
using the |\include| command.
This mechanism is beneficial for documents
which span hundreds of pages in order to
make the source file(s) more manageable.
Moreover, compilation can be restricted to
selected child files by means of the |\includeonly| command.
The latter feature can be used to reduce the compilation time while editing
(this was significantly more useful in the earlier days of \LaTeX{})
or to generate a smaller document which is easier to navigate.
Another application of |\includeonly| is to generate
documents consisting of selected parts of the complete document.

However, there are a few drawbacks of the plain |\include| mechanism:
\begin{itemize}
\item
The child files cannot be compiled on their own,
they can only be compiled via the main file.
A naive editing environment
(such as a text editor with an option
to have the current file processed by \LaTeX)
may require one to switch to the main file before compiling;
attempting to compile the child file produces errors.
\item
The main file must be modified (each time)
to adjust the |\includeonly| command
to the present needs. This easily leaves the main file in a messy state.
\item
The generated document will always carry the filename
of the main document. This is inconvenient if
several child files are to be compiled and
to be kept for distribution.
\end{itemize}

The present package provides a simple interface
to make child files individually compilable by \LaTeX{}.
Compiling a child file then has the same effect as compiling
the main file with an |\includeonly| command
to select the appropriate child.
Moreover the generated document will carry the name of the child
rather than the main file.
This resolves all three above issues.

This feature is meant to make the editing of books,
thesis documents and lecture notes somewhat more convenient.
However, the package can also be used efficiently for
composing a series of documents (such as exercise sheets)
which are typically distributed individually.
It then assists the author in generating the individual documents
(potentially in different versions)
as well as a document containing the collected series.
Another application is in developing style files
or other kinds of included material
where compilation of the style file could redirect
to a sample or test file.

%%%%%%%%%%%%%%%%%%%%%%%%%%%%%%%%%%%%%%%%%%%%%%%%%%%%%%%%%%%%%%%%%%%%%%%%%%%%%%%%
%%%%%%%%%%%%%%%%%%%%%%%%%%%%%%%%%%%%%%%%%%%%%%%%%%%%%%%%%%%%%%%%%%%%%%%%%%%%%%%%
\section{Usage}

First of all, the package \textsf{childdoc} is \emph{not} a standard
\LaTeXe{} |.sty| style file! Therefore it needs to be invoked in
a non-standard way.

%%%%%%%%%%%%%%%%%%%%%%%%%%%%%%%%%%%%%%%%%%%%%%%%%%%%%%%%%%%%%%%%%%%%%%%%%%%%%%%%
\subsection{Included Files}
\label{sec:include}

%%%%%%%%%%%%%%%%%%%%%%%%%%%%%%%%%%%%%%%%
\DescribeMacro{\childdocmain}
To use the package, add the commands
\begin{center}
\begin{tabular}{l}
|\input{childdoc.def}|\\
|\childdocmain{}|\\
\end{tabular}
\end{center}
at the very top of the main \LaTeX{} file,
in particular \emph{before} the |\documentclass| statement!
The argument of |\childdocmain| should be left empty
(but it must be present).

%%%%%%%%%%%%%%%%%%%%%%%%%%%%%%%%%%%%%%%%
\DescribeMacro{\childdocof}
Furthermore, add the commands
\begin{center}
\begin{tabular}{l}
|\input{childdoc.def}|\\
|\childdocof{|\textit{main}|}|\\
\end{tabular}
\end{center}
at the top of every child file \textit{child}
which is included by |\include{|\textit{child}|}|
from within the main file
(or at least for those files to be compiled individually).
The argument \textit{main} must be the filename of the main file.

There are a couple of
considerations in setting up the main and child documents:

%%%%%%%%%%%%%%%%%%%%%%%%%%%%%%%%%%%%%%%%
\paragraph{Restrictions.}

Please note the following restrictions:
\begin{itemize}
\item
|\childdocmain| must be called with one argument \textit{main}
to ensure compatibility with earlier version of the package.
It must either be empty (|\childdocmain{}|)
or precisely match the filename of the main file in which it is specified.
See \secref{sec:detection} for further information.
\item
The filename \textit{main} must be specified without the |.tex| extension.
\item
The filename \textit{main} is case sensitive
(even in case-insensitive file systems)
due to internal string comparison.
\item
The argument \textit{main} should be fully expanded, it cannot be a macro.
\item
Subdirectories and special characters should be avoided in filenames.
\item
The command |\childdocmain{|\textit{main}|}| must be followed by a whitespace.
It should not be followed immediately by another command
or by a comment mark `|%|'.
This is because the \TeX{} parser reads the token immediately following
the argument of |\childdocmain| and puts it
at the beginning of every child section;
however, a white\-space is ignored.
\end{itemize}

%%%%%%%%%%%%%%%%%%%%%%%%%%%%%%%%%%%%%%%%
\paragraph{Content of Main File.}

It is advisable to place all content in the child files included by |\include|.
Any output contained in the main file will appear in all child documents
unless suppressed manually;
it cannot be suppressed automatically by the |\includeonly| directive
and thus should normally be avoided.
A method to include some content in the main file
by means of conditional processing is described in \secref{sec:conditional}.

%%%%%%%%%%%%%%%%%%%%%%%%%%%%%%%%%%%%%%%%
\paragraph{Page Numbering.}

When only a part of the document is compiled,
the appropriate numbering of pages
(as well as other status parameters)
is determined from the |.aux| files.
The latter contain information from previous passes.
However this information needs to propagate through
all intermediate child documents.
Therefore the page numbering in child documents may well
be inconsistent until the complete document is compiled at least once.

A useful (if unconventional) way to always ensure a consistent
page numbering is to restart the numbering in each child document
and denote the pages by `\textit{child}|.|\textit{page}'
where \textit{child} represents the chapter/section number of the child file.
This can be achieved by the command
|\numberwithin{page}{|\textit{child}|}|
of the \textsf{amsmath} package
where \textit{child} can be |chapter| or |section|
depending on the chosen structuring.
Alternatively, one can modify the macro |\thepage| appropriately
and reset the counter |page| at the start of each child file.

%%%%%%%%%%%%%%%%%%%%%%%%%%%%%%%%%%%%%%%%%%%%%%%%%%%%%%%%%%%%%%%%%%%%%%%%%%%%%%%%
\subsection{Conditional Processing}
\label{sec:conditional}

The package provides a mechanism to compile different versions
of a document. To customise the versions further some conditional processing
can come in handy to distinguish which version is being compiled.
The package provides two macros to describe the compilation context:

%%%%%%%%%%%%%%%%%%%%%%%%%%%%%%%%%%%%%%%%
\DescribeMacro{\ifchilddoc}
The conditional |\ifchilddoc| distinguishes between the compilation of
child documents and the main document:
%
\begin{center}
|\ifchilddoc |\textit{child-code}| |[|\||else |\textit{main-code}]| \||fi|
\end{center}

%%%%%%%%%%%%%%%%%%%%%%%%%%%%%%%%%%%%%%%%
\DescribeMacro{\childdocname}
\DescribeMacro{\childdocjob}
The macro |\childdocname| contains the filename (without extension)
of the main or child file being processed.
Note that |\childdocjob| will always contain the name of the main file.

%%%%%%%%%%%%%%%%%%%%%%%%%%%%%%%%%%%%%%%%
\paragraph{Title Page.}

Conditional processing can be used to include a title or banner page
in the main document when proper precautions are taken.
Importantly, the code in the main file should ensure that the page counter
(as well as other status parameters which are stored in the |.aux| files)
takes the same value after the conditional processing.
Otherwise the page numbers may take divergent values
depending on which part is compiled.

For example, a title page could be declared by:
%
\begin{center}
\begin{tabular}{l}
|\ifchilddoc\||else|\\
|\addtocounter{page}{-1}|\\
\textit{code for title page}\\
|\newpage|\\
|\||fi|
\end{tabular}
\end{center}
%
A banner page for the child documents can be generated by:
%
\begin{center}
\begin{tabular}{l}
|\ifchilddoc|\\
|\addtocounter{page}{-1}|\\
\textit{code for banner page}\\
|\newpage|\\
|\||fi|
\end{tabular}
\end{center}
%
Here one could write a message such as:
\begin{center}
|This is the part \childdocname{} of \childdocjob{}.|
\end{center}

%%%%%%%%%%%%%%%%%%%%%%%%%%%%%%%%%%%%%%%%%%%%%%%%%%%%%%%%%%%%%%%%%%%%%%%%%%%%%%%%
\subsection{Flags}
\label{sec:flags}

The package makes it easy to generate different versions
of the main or child documents.
To this end compilation flags can be defined
and assigned different default values.
They will be particularly useful in conjunction
with the forwarding mechanism described in \secref{sec:forward}.

For example, it may be useful to have a flag |\version|
which can be set to |draft| or |final|.
The document source will contain some conditional code
depending on the value of |\version|.
Suppose further, the flag should default to |final| for the main file
and to |draft| for child files
which is a natural assignment for editing the document.
This is achieved by placing the following code
in the preamble of the main document
(below the |\childdocmain| directive):
%
\begin{center}
\begin{tabular}{l}
|\ifchilddoc|\\
|\providecommand{\version}{draft}|\\
|\||else|\\
|\providecommand{\version}{final}|\\
|\||fi|
\end{tabular}
\end{center}
%
The definition by |\providecommand| makes sure
that previous definitions are not overwritten.
Further statements |\providecommand{\version}{...}|
can thus be added before the above code to override it.

For the main file, one might add a line
(between |\childdocmain| and the above block)
%
\begin{center}
|%\ifchilddoc\||else\providecommand{\version}{draft}\||fi|
\end{center}
%
which can be uncommented to produce a draft version.
Likewise one can add a line to the very top of a child file
(above the |\childdocof{|\textit{main}|}| directive)
%
\begin{center}
|%\providecommand{\version}{final}|
\end{center}
%
which can be uncommented to produce the final version of this child document.

%%%%%%%%%%%%%%%%%%%%%%%%%%%%%%%%%%%%%%%%%%%%%%%%%%%%%%%%%%%%%%%%%%%%%%%%%%%%%%%%
\subsection{Forwarding}
\label{sec:forward}

Different versions of the main or child documents
using compilation flags as described in \secref{sec:flags}
can be (permanently) stored in different files
for convenient compilation, viewing and distribution.
To this end, the package defines a command
to pass on compilation to a different file:

%%%%%%%%%%%%%%%%%%%%%%%%%%%%%%%%%%%%%%%%
\DescribeMacro{\childdocforward}
The command |\childdocforward| redirects processing to
another source file:
%
\begin{center}
\begin{tabular}{l}
|\input{childdoc.def}|\\
|\childdocforward[|\textit{main}|]{|\textit{dest}|}|\\
\end{tabular}
\end{center}
%
The argument \textit{dest} is the destination file
(without extension).
It should be the main file or one of the child files.
Note that further \textsf{childdoc} directives
such as |\childdocof| and |\childdocforward|
in the indicated file will be processed in this form.
The optional argument \textit{main}
passes on directly to the main file \textit{main}
while pretending to compile the child \textit{dest}.
This form behaves as if \textit{dest}
issues |\childdocof{|\textit{main}|}| right away,
and no further \textsf{childdoc} directives will be processed.

%%%%%%%%%%%%%%%%%%%%%%%%%%%%%%%%%%%%%%%%
\DescribeMacro{\...prefix}
In the alternative form |\childdocforwardprefix|,
%
\begin{center}
\begin{tabular}{l}
|\input{childdoc.def}|\\
|\childdocforwardprefix[|\textit{main}|]{|\textit{prefix}|}{|\textit{dest}|}|
\end{tabular}
\end{center}
%
the destination file is determined by a pattern
depending on the current file:
To make this work, the current file must be called
`{\textit{prefix}\hspace{0.2em}\textit{suffix}}'
with \textit{prefix} matching precisely the argument.
Processing is then passed on to the file
`{\textit{dest}\hspace{0.2em}\textit{suffix}}'.
Surely, the same effect is achieved by
directly specifying the
argument `{\textit{dest}\hspace{0.2em}\textit{suffix}}'
in the first form.
However, that requires to set up a different file
for each child. With the alternative form of the command
all these files can have exactly the same content
which simplifies setting them up and maintaining them.

For example, the following file |draft.tex|
with a compilation flag |\version| as described in \secref{sec:flags}
compiles the main document as a draft:
%
\begin{center}
\begin{tabular}{l}
|\def\version{draft}|\\
|\input{childdoc.def}|\\
|\childdocforward{|\textit{main}|}|
\end{tabular}
\end{center}
%
Likewise, the following files |final|\textit{nn}|.tex|
compile the final version of the child document
|child|\textit{nn}|.tex|:
%
\begin{center}
\begin{tabular}{l}
|\def\version{final}|\\
|\input{childdoc.def}|\\
|\childdocforwardprefix{final}{child}|
\end{tabular}
\end{center}
%

Note that when several versions of a main file and/or of each child file
are to be generated, it may be convenient to set up a |Makefile| or
shell script to automatise the process.

%%%%%%%%%%%%%%%%%%%%%%%%%%%%%%%%%%%%%%%%%%%%%%%%%%%%%%%%%%%%%%%%%%%%%%%%%%%%%%%%
\subsection{Command Line Processing}
\label{sec:commandline}

The effect of redirection files can also be achieved by invoking
the \LaTeX{} compiler with a more elaborate command line.
Most conveniently this should be done as part
of a shell script or a |Makefile|.

When using \textsf{childdoc} in the main file, the following
command lines effectively perform a redirection
(note that depending on the shell being used,
backslashes may have to be doubled: `|\|' $\to$ `|\\|'):
%
\begin{center}
|... -jobname "|\textit{target}|" |\\|"|[\textit{flags}]%
|\input{childdoc.def}\childdocforward[|\textit{main}|]{|\textit{dest}|}"|
\end{center}
%
Here \textit{target} is the name of the output file,
\textit{main} is the name of the main file
and \textit{dest} is the name of the main or child file to be processed
(all filenames without extensions).
The optional argument \textit{main} can be omitted
if \textit{main} matches \textit{dest}.
Optionally, compilation \textit{flags} can be defined via |\def| commands.
This command line makes the \TeX{} engine believe
it is compiling the file \textit{target}
whose content is specified as the latter parameter.
The provided code then forwards the processing to
\textit{main} or \textit{dest} as described in \secref{sec:forward}.

%%%%%%%%%%%%%%%%%%%%%%%%%%%%%%%%%%%%%%%%%%%%%%%%%%%%%%%%%%%%%%%%%%%%%%%%%%%%%%%%
\subsection{Include by Input}
\label{sec:input}

Including child documents by |\include| has some restrictions by design.
Most notably, the content of a child document always occupies
its own set of pages; pages cannot be shared between child documents.
Usually, this behaviour makes perfect sense
because each child document contain an essential part of the document.
However, in some situations it may be desirable to compose
a document from a collection of parts
without having mandatory page breaks between then.
For this case, the package
provides a mechanism to include parts
by |\input| which can also be processed individually.
However, by construction this mechanism
requires manual handling of the content to be output.

%%%%%%%%%%%%%%%%%%%%%%%%%%%%%%%%%%%%%%%%
\DescribeMacro{\ifchilddocmanual}
The main file should be prepared as usual, see \secref{sec:include}.
However, the document body must make a distinction
between processing of an individual part and of the main document, e.g.:
%
\begin{center}
\begin{tabular}{l}
|\ifchilddocmanual|\\
|\input{\childdocname}|\\
|\||else|\\
\textit{document body with }|\input{|\textit{part}|}|\\
|\||fi|
\end{tabular}
\end{center}
%
The conditional |\ifchilddocmanual| is true whenever
a part to be included by |\input| is being compiled,
and the name of the part is stored in |\childdocname|.

%%%%%%%%%%%%%%%%%%%%%%%%%%%%%%%%%%%%%%%%
\DescribeMacro{\childdocby}
Each part to be included by |\input| should start with:
%
\begin{center}
\begin{tabular}{l}
|\input{childdoc.def}|\\
|\childdocby{|\textit{main}|}|\\
\end{tabular}
\end{center}
%
The directive |\childdocby| is similar to |\childdocof|
described in \secref{sec:include},
but the subsequent selection of content must be done manually.
To that end, both |\ifchilddoc| and |\ifchilddocmanual|
will be true upon processing of a part,
and the name of the part is stored in |\childdocname|.
Note that |\jobname| will be set to the filename of the current part
so that each part receives an individual |.aux| file
that does not interfere with the |.aux| file(s) of the main document.
This behaviour can be altered by the alternative form
|\childdocby[*]{|\textit{main}|}| (with a non-empty optional argument)
which uses the |.aux| file of the main document
by setting |\jobname| to \textit{main}.

%%%%%%%%%%%%%%%%%%%%%%%%%%%%%%%%%%%%%%%%%%%%%%%%%%%%%%%%%%%%%%%%%%%%%%%%%%%%%%%%
\subsection{Driver Development}
\label{sec:driver}

The \textsf{childdoc} mechanism can also be use for the development
of definition files such as \LaTeX{} styles or classes.
This case differs from the above setup with multiple parts
included by |\include| in that no |\includeonly| should be invoked.
This can be achieved by starting the include file
(before |\ProvidesPackage|) with:
%
\begin{center}
\begin{tabular}{l}
|\input{childdoc.def}|\\
|\childdocforward{|\textit{main}|}|\\
\end{tabular}
\end{center}
%
or alternatively with:
%
\begin{center}
\begin{tabular}{l}
|\input{childdoc.def}|\\
|\childdocby{|\textit{main}|}|\\
\end{tabular}
\end{center}
%
Both forms have slightly different effects as described above.
The main file is prepared as usual, see \secref{sec:include}.

%%%%%%%%%%%%%%%%%%%%%%%%%%%%%%%%%%%%%%%%%%%%%%%%%%%%%%%%%%%%%%%%%%%%%%%%%%%%%%%%
\subsection{Legacy Detection}
\label{sec:detection}

The directive |\childdocmain| in the main file can detect
whether the complete document or merely a child is to be compiled
even without using the directive |\childdocof|.
This method is deprecated because it is less robust
and there is no compelling reason to use it;
it is merely provided for backward compatibility
and it may be removed in future versions.

If the detection mechanism is to be used,
it is mandatory to correctly specify
the filename of the main file as the argument of |\childdocmain|:
%
\begin{center}
\begin{tabular}{l}
|\input{childdoc.def}|\\
|\childdocmain{|\textit{main}|}|\\
\end{tabular}
\end{center}
%
If |\jobname| does not match the argument \textit{main} of |\childdocmain|,
it is assumed that |\jobname| points to the child file to be compiled.
When using |\childdocmain| with the main file specified as argument,
it suffices to start a child file
with just |\input{|\textit{main}|}|
without loading of the package and using |\childdocof|.
If instead all processing is done
with the appropriate \textsf{childdoc} directives,
the argument of \textit{main} of |\childdocmain| can be empty.

An alternative version of the command line processing described
in \secref{sec:commandline} using the detection mechanism reads:
%
\begin{center}
|... -jobname "|\textit{target}|" "|[\textit{flags}]%
[|\def\jobname{|\textit{dest}|}|]|\input{|\textit{main}|}"|
\end{center}

%%%%%%%%%%%%%%%%%%%%%%%%%%%%%%%%%%%%%%%%%%%%%%%%%%%%%%%%%%%%%%%%%%%%%%%%%%%%%%%%
\subsection{Manual Code}
\label{sec:manual}

In case one cannot be certain whether the definitions file |childdoc.def|
is installed on the target \TeX{} distribution
and one prefers not to ship it,
it is conceivable to paste a few relevant commands into the sources.

To that end, drop all statements |\input{childdoc.def}|
and perform the replacements as outlined below.
Instead of |\childdocmain{|\textit{main}|}| add the following code
to the top of the main file:
%
\begin{center}
\begin{tabular}{l}
|\||ifdefined\childdocname\endinput\||fi\newif\ifchilddoc|\\
|\edef\childdocname{\scantokens\expandafter{\jobname\noexpand}}|\\
|\def\childdocmain{|\textit{main}|}\||ifx\childdocmain\childdocname\||else|\\
|\childdoctrue\includeonly{\childdocname}\let\jobname\childdocmain\||fi|\\
\end{tabular}
\end{center}
%
Instead of |\childdocof{|\textit{main}|}| just include the main file
at the top of each child file:
%
\begin{center}
|\input{|\textit{main}|}|
\end{center}
%
A simple redirection |\childdocforward{|\textit{dest}|}| is achieved by:
%
\begin{center}
|\def\jobname{|\textit{dest}|}\input{\jobname}|
\end{center}
%
The redirection with prefix
|\childdocforwardprefix[|\textit{prefix}|]{|\textit{dest}|}|
is accomplished by:
%
\begin{center}
\begin{tabular}{l}
|{\edef\jobname{\scantokens\expandafter{\jobname\noexpand}}|\\
|\def\redirectjob |\textit{prefix}|#1~~~{\gdef\jobname{|\textit{dest}|#1}}|\\
|\expandafter\redirectjob\jobname~~~}\input{\jobname}|
\end{tabular}
\end{center}

In an alternative approach,
child documents can be compiled by a specific command line
without additional code or specific definitions:
%
\begin{center}
|... -jobname "|\textit{target}|" "|[\textit{flags}]%
|\includeonly{|\textit{dest}|}\input{|\textit{main}|}"|
\end{center}
%

%%%%%%%%%%%%%%%%%%%%%%%%%%%%%%%%%%%%%%%%%%%%%%%%%%%%%%%%%%%%%%%%%%%%%%%%%%%%%%%%
%%%%%%%%%%%%%%%%%%%%%%%%%%%%%%%%%%%%%%%%%%%%%%%%%%%%%%%%%%%%%%%%%%%%%%%%%%%%%%%%
\section{Information}

%%%%%%%%%%%%%%%%%%%%%%%%%%%%%%%%%%%%%%%%%%%%%%%%%%%%%%%%%%%%%%%%%%%%%%%%%%%%%%%%
\subsection{Copyright}

Copyright \copyright{} 2017--2018 Niklas Beisert

This work may be distributed and/or modified under the
conditions of the \LaTeX{} Project Public License, either version 1.3
of this license or (at your option) any later version.
The latest version of this license is in
  \url{http://www.latex-project.org/lppl.txt}
and version 1.3 or later is part of all distributions of \LaTeX{}
version 2005/12/01 or later.

This work has the LPPL maintenance status `maintained'.

The Current Maintainer of this work is Niklas Beisert.

This work consists of the files |README.txt|, |childdoc.ins| and |childdoc.dtx|
as well as the derived files |childdoc.def|, |cdocsamp.tex|
with |cdocsch1.tex|, |cdocsch2.tex|, |cdocspt3.tex|, |cdocspt4.tex|,
|cdocsdrf.tex|, |cdocsfn1.tex|, |cdocsfn2.tex|
as well as |childdoc.pdf|.

%%%%%%%%%%%%%%%%%%%%%%%%%%%%%%%%%%%%%%%%%%%%%%%%%%%%%%%%%%%%%%%%%%%%%%%%%%%%%%%%
\subsection{Files and Installation}

The package consists of the files:
%
\begin{center}
\begin{tabular}{ll}
    |README.txt|   & readme file \\
    |childdoc.ins| & installation file \\
    |childdoc.dtx| & source file \\
    |childdoc.def| & definition file \\
    |cdocsamp.tex| & sample main file \\
    |cdocsch1.tex| & sample include file \\
    |cdocsch2.tex| & sample include file \\
    |cdocspt3.tex| & sample part file \\
    |cdocspt4.tex| & sample part file \\
    |cdocsdrf.tex| & sample redirection file \\
    |cdocsfn1.tex| & sample redirection file \\
    |cdocsfn2.tex| & sample redirection file \\
    |childdoc.pdf| & manual
\end{tabular}
\end{center}
%
The distribution consists of the files
|README.txt|, |childdoc.ins| and |childdoc.dtx|.
%
\begin{itemize}
\item
Run (pdf)\LaTeX{} on |childdoc.dtx|
to compile the manual |childdoc.pdf| (this file).
\item
Run \LaTeX{} on |childdoc.ins| to create the definitions file |childdoc.def|
and the sample |cdocsamp.tex| with include files
|cdocsch1.tex|, |cdocsch2.tex|, |cdocspt3.tex|, |cdocspt4.tex|,
|cdocsdrf.tex|, |cdocsfn1.tex|, |cdocsfn2.tex|.
Then copy the file |childdoc.def| to an appropriate directory of your \LaTeX{}
distribution, e.g.\ \textit{texmf-root}|/tex/latex/childdoc|.
\end{itemize}

%%%%%%%%%%%%%%%%%%%%%%%%%%%%%%%%%%%%%%%%%%%%%%%%%%%%%%%%%%%%%%%%%%%%%%%%%%%%%%%%
\subsection{Related CTAN Packages}

There are several other packages which offer a similar functionality:
%
\begin{itemize}
\item
The packages
\href{http://ctan.org/pkg/docmute}{\textsf{docmute}},
\href{http://ctan.org/pkg/includex}{\textsf{includex}} and
\href{http://ctan.org/pkg/standalone}{\textsf{standalone}}
provide commands to include only the document body of
a child file thus allowing both files to be compiled individually.
\item
The packages \href{http://ctan.org/pkg/subdocs}{\textsf{subdocs}}
and \href{http://ctan.org/pkg/subfiles}{\textsf{subfiles}}
provide structures in which the main and child documents can be
encapsulated and allowing them to be compiled individually.
The inclusion mechanism is different from the conventional |\include|.
\item
The package \href{http://ctan.org/pkg/combine}{\textsf{combine}}
is an elaborate solution to combine several documents into one.
\end{itemize}
%
See also the CTAN topic \href{http://ctan.org/topic/subdocs}{\textsf{subdocs}}
for further related packages.
The present package differs from the above solutions in that
a document structure constructed with the conventional |\include| mechanism
just needs two extra commands at the top of every file
such that all constituent files can be compiled individually.

%%%%%%%%%%%%%%%%%%%%%%%%%%%%%%%%%%%%%%%%%%%%%%%%%%%%%%%%%%%%%%%%%%%%%%%%%%%%%%%%
%\subsection{Feature Suggestions}
%
%The following is a list of features which may be useful for future
%versions of this package:
%%
%\begin{itemize}
%\item
%\ldots
%\end{itemize}

%%%%%%%%%%%%%%%%%%%%%%%%%%%%%%%%%%%%%%%%%%%%%%%%%%%%%%%%%%%%%%%%%%%%%%%%%%%%%%%%
\subsection{Revision History}

%%%%%%%%%%%%%%%%%%%%%%%%%%%%%%%%%%%%%%%%
\paragraph{v2.0:} 2018/12/30

\begin{itemize}
\item
immediate forward processing
\item
added |\childdocby| mechanism
\item
manual restructured
\end{itemize}

%%%%%%%%%%%%%%%%%%%%%%%%%%%%%%%%%%%%%%%%
\paragraph{v1.6:} 2018/01/17

\begin{itemize}
\item
application for development of include files
\item
corrections to manual
\end{itemize}

%%%%%%%%%%%%%%%%%%%%%%%%%%%%%%%%%%%%%%%%
\paragraph{v1.5:} 2017/05/21

\begin{itemize}
\item
more complete structuring introduced
\item
|\childdocof| introduced
\item
|\childdoc| renamed to |\childdocmain|
\item
|\childredirect| renamed to |\childdocforward| and |\childdocforwardprefix|
and functionality expanded
\end{itemize}

%%%%%%%%%%%%%%%%%%%%%%%%%%%%%%%%%%%%%%%%
\paragraph{v1.0:} 2017/04/27

\begin{itemize}
\item
manual and install package
\item
first version published on CTAN
\end{itemize}

%%%%%%%%%%%%%%%%%%%%%%%%%%%%%%%%%%%%%%%%
\paragraph{v0.6:} 2017/04/26

\begin{itemize}
\item
redirection mechanism added
\end{itemize}

%%%%%%%%%%%%%%%%%%%%%%%%%%%%%%%%%%%%%%%%
\paragraph{v0.5:} 2017/04/26

\begin{itemize}
\item
functionality in definition file
\end{itemize}


%%%%%%%%%%%%%%%%%%%%%%%%%%%%%%%%%%%%%%%%%%%%%%%%%%%%%%%%%%%%%%%%%%%%%%%%%%%%%%%%
%%%%%%%%%%%%%%%%%%%%%%%%%%%%%%%%%%%%%%%%%%%%%%%%%%%%%%%%%%%%%%%%%%%%%%%%%%%%%%%%
%%%%%%%%%%%%%%%%%%%%%%%%%%%%%%%%%%%%%%%%%%%%%%%%%%%%%%%%%%%%%%%%%%%%%%%%%%%%%%%%
\appendix

\settowidth\MacroIndent{\rmfamily\scriptsize 000\ }

 \DocInput{childdoc.dtx}

\end{document}
%</driver>
% \fi
%
% %%%%%%%%%%%%%%%%%%%%%%%%%%%%%%%%%%%%%%%%%%%%%%%%%%%%%%%%%%%%%%%%%%%%%%%%%%%%%%
% %%%%%%%%%%%%%%%%%%%%%%%%%%%%%%%%%%%%%%%%%%%%%%%%%%%%%%%%%%%%%%%%%%%%%%%%%%%%%%
% \section{Sample}
%\iffalse
%<*samplemain>
%\fi
%
% The following presents a sample document
% with two chapters, two parts, a title page,
% a compile flag as well as three forwarding files to set the flag.
% It consists of eight |.tex| files:
% \begin{center}
% \begin{tabular}{ll}
% |cdocsamp.tex|&main file\\
% |cdocsch1.tex|&include file for chapter 1\\
% |cdocsch2.tex|&include file for chapter 2\\
% |cdocspt3.tex|&include file for part 3\\
% |cdocspt4.tex|&include file for part 4\\
% |cdocsdrf.tex|&forwarding file for main file in draft mode\\
% |cdocsfi1.tex|&forwarding file for final version of chapter 1\\
% |cdocsfi2.tex|&forwarding file for final version of chapter 2\\
% \end{tabular}
% \end{center}
% Each of the eight files can be compiled directly by the \LaTeX{} compiler.
%
% %%%%%%%%%%%%%%%%%%%%%%%%%%%%%%%%%%%%%%
% \paragraph{Main File.}
%
% The main file is called |cdocsamp.tex|.
%
% Load the \textsf{childdoc} definitions and
% declare the filename for the main document:
%    \begin{macrocode}
\input{childdoc.def}
\childdocmain{}
%    \end{macrocode}

% Optional override for |\version| flag:
%    \begin{macrocode}
%%\ifchilddoc\else\providecommand{\version}{draft}\fi
%    \end{macrocode}

% Define the default values for the |\version| flag
% (|final| for the main file and |draft| for childs):
%    \begin{macrocode}
\ifchilddoc
\providecommand{\version}{draft}
\else
\providecommand{\version}{final}
\fi
%    \end{macrocode}

% Load the standard document class:
%    \begin{macrocode}
\documentclass[12pt]{article}
%    \end{macrocode}

% Start the document body:
%    \begin{macrocode}
\begin{document}
%    \end{macrocode}

% Declare a title page.
% Print title, part of document being processed and version flag:
%    \begin{macrocode}
\addtocounter{page}{-1}
\begin{center}
{\LARGE\bfseries{}childdoc example\par}
\vspace{1cm}
\ifchilddoc
\ifchilddocmanual part\else chapter\fi:
`\childdocname' of `\childdocjob'\par
\else
main document: `\childdocjob'\par
\fi
version: \version\par
\end{center}
\newpage
%    \end{macrocode}

% Manually include selected file,
% otherwise process as usual:
%    \begin{macrocode}
\ifchilddocmanual
\section*{part `\childdocname'}
\input{\childdocname}
\else
%    \end{macrocode}

% Include the two chapters:
%    \begin{macrocode}
\include{cdocsch1}
\include{cdocsch2}
%    \end{macrocode}

% Include the two parts unless only chapters should be displayed:
%    \begin{macrocode}
\ifchilddoc\else
\section{part three}
\input{cdocspt3}
\section{part four}
\input{cdocspt4}
\fi
%    \end{macrocode}

% Process as usual until here:
%    \begin{macrocode}
\fi
%    \end{macrocode}

% End of document body:
%    \begin{macrocode}
\end{document}
%    \end{macrocode}
%\iffalse
%</samplemain>
%\fi
%
% %%%%%%%%%%%%%%%%%%%%%%%%%%%%%%%%%%%%%%
% \paragraph{Chapter Include Files.}
%
% The include files are called |cdocsch1.tex| and |cdocsch2.tex|.
%
%\iffalse
%<*samplechap1|samplechap2>
%\fi

% Optional override for |\version| flag:
%    \begin{macrocode}
%%\providecommand{\version}{final}
%    \end{macrocode}

% Include the main document:
%    \begin{macrocode}
\input{childdoc.def}
\childdocof{cdocsamp}
%    \end{macrocode}

%\iffalse
%</samplechap1|samplechap2>
%\fi
%
%\iffalse
%<*samplechap1>
%\fi
% Some text for chapter 1:
%    \begin{macrocode}
\section{one}
some text in chapter one
%    \end{macrocode}

%\iffalse
%</samplechap1>
%\fi
% Some text for chapter 2:
%\iffalse
%<*samplechap2>
%\fi
%    \begin{macrocode}
\section{two}
more text in chapter two
%    \end{macrocode}

%\iffalse
%</samplechap2>
%\fi
%
% %%%%%%%%%%%%%%%%%%%%%%%%%%%%%%%%%%%%%%
% \paragraph{Part Include Files.}
%
% The include files are called |cdocspt3.tex| and |cdocspt4.tex|.
%
%\iffalse
%<*samplepart3|samplepart4>
%\fi

% Optional override for |\version| flag:
%    \begin{macrocode}
%%\providecommand{\version}{final}
%    \end{macrocode}

% Include the main document:
%    \begin{macrocode}
\input{childdoc.def}
\childdocby{cdocsamp}
%    \end{macrocode}

%\iffalse
%</samplepart3|samplepart4>
%\fi
%
%\iffalse
%<*samplepart3>
%\fi
% Some text for part 3:
%    \begin{macrocode}
some text in part three
%    \end{macrocode}

%\iffalse
%</samplepart3>
%\fi
% Some text for part 4:
%\iffalse
%<*samplepart4>
%\fi
%    \begin{macrocode}
more text in part four
%    \end{macrocode}

%\iffalse
%</samplepart4>
%\fi
%
% %%%%%%%%%%%%%%%%%%%%%%%%%%%%%%%%%%%%%%
% \paragraph{Forwarding for a Complete Draft.}
%
% The following forwarding file |cdocsdrf.tex|
% compiles the main document in draft mode:
%\iffalse
%<*sampledraft>
%\fi
%    \begin{macrocode}
\def\version{draft}
\input{childdoc.def}
\childdocforward{cdocsamp}
%    \end{macrocode}

%\iffalse
%</sampledraft>
%\fi
%
% %%%%%%%%%%%%%%%%%%%%%%%%%%%%%%%%%%%%%%
% \paragraph{Forwarding for Final Version of the Chapters.}
%
% The following forwarding files |cdocsfn1.tex| and |cdocsfn2.tex|
% (with identical content)
% compile the final versions of the child documents
% |cdocsch1.tex| and |cdocsch2.tex|, respectively:
%\iffalse
%<*samplefinal>
%\fi
%    \begin{macrocode}
\def\version{final}
\input{childdoc.def}
\childdocforwardprefix[cdocsamp]{cdocsfn}{cdocsch}
%    \end{macrocode}

%\iffalse
%</samplefinal>
%\fi
%
% %%%%%%%%%%%%%%%%%%%%%%%%%%%%%%%%%%%%%%
% \paragraph{Command Line Processing.}
%
% The following three command lines generate the output files
% |cdocscld|, |cdocscl1| and |cdocscl2|
% which should be identical to
% |cdocsdrf|, |cdocsch1| and |cdocsfn2|, respectively:
% \begin{center}
% \begin{tabular}{l}
% |latex -jobname cdocscld \|\\
% |  "\def\version{draft}\input{childdoc.def}\childdocforward{cdocsamp}"|\\
% |latex -jobname cdocscl1 \|\\
% |  "\input{childdoc.def}\childdocforward[cdocsamp]{cdocsch1}"|\\
% |latex -jobname cdocscl2 \|\\
% |  "\def\version{final}\input{childdoc.def}\childdocforward{cdocsch2}"|
% \end{tabular}
% \end{center}
% Note that the trailing backslash on each first line
% merely continues the input to the second line
% (for convenient cut ant paste).
% Furthermore, the command |latex| can be replaced by any
% of its alternative versions such as |pdflatex|.
%
% %%%%%%%%%%%%%%%%%%%%%%%%%%%%%%%%%%%%%%%%%%%%%%%%%%%%%%%%%%%%%%%%%%%%%%%%%%%%%%
% %%%%%%%%%%%%%%%%%%%%%%%%%%%%%%%%%%%%%%%%%%%%%%%%%%%%%%%%%%%%%%%%%%%%%%%%%%%%%%
% \section{Implementation}
%\iffalse
%<*package>
%\fi
%
% This section describes the definitions file |childdoc.def|.

% The definitions cannot be loaded using |\usepackage| or |\RequirePackage|
% which has a mechanism to prevent loading a style file more than once.
% When loading the definitions by means of |\input|
% multiple instances have to be prevented manually:
%\iffalse
%This code needs to be before the `\ProvidesFile' directive
%which is defined at the beginning of this file.
%Therefore it is also placed there and commented out here.
%</package>
%<*discard>
%\fi
%    \begin{macrocode}
\ifdefined\childdocmain\endinput\fi
%    \end{macrocode}
%\iffalse
%</discard>
%<*package>
%\fi
%
% \macro{\ifchilddoc}
% \macro{\ifchilddocmanual}
% The conditional |\ifchilddoc| tells whether a
% child (true) or main (false) document is being compiled.
% The conditional |\ifchilddocmanual| tells whether
% the |\includeonly| mechanism is used (false) or
% the selection of child files must be performed manually (true).
% The definitions initialise to false:
%    \begin{macrocode}
\newif\ifchilddoc
\newif\ifchilddocmanual
%    \end{macrocode}

% \macro{\childdocname}
% \macro{\childdocjob}
% The macro |\childdocname| stores the name of the main document
% to be compiled. The macro |\childdocjob| stores the name of
% the document on which the \LaTeX{} compiler was originally invoked.
% The content of |\jobname| cannot be compared
% to filenames specified in the source due to different catcodes.
% The following code rescans |\jobname|, stores the result
% in |\childdocname| and saves a copy in |\childdocjob|:
%    \begin{macrocode}
\edef\childdocname{\scantokens\expandafter{\jobname\noexpand}}
\let\childdocjob\childdocname
%    \end{macrocode}

% \macro{\childdocdisable}
% The macro |\childdocdisable| prevents the main file
% from being processed more than once.
% At this stage, the main document command |\childdocmain|
% is assumed to be called once again where it should do nothing.
% Any subsequent call to it should prevent
% a secondary processing of the main document
% It overwrites the forwarding commands
% |\childdocof| and |\childdocforward|
% with empty macros to prevent further inclusions of the main document:
%    \begin{macrocode}
\newcommand{\childdocdisable}
{
  \renewcommand{\childdocmain}[1]{\renewcommand{\childdocmain}[1]{\endinput}}
  \renewcommand{\childdocof}[1]{}
  \renewcommand{\childdocby}[2][]{}
  \renewcommand{\childdocforward}[2][]{}
  \renewcommand{\childdocdisable}{}
}
%    \end{macrocode}

% \macro{\childdocmain}
% The macro |\childdocmain| is to be called at the top of the main file
% with nothing or the main filename (without extension) as argument.
% First, it breaks loops.
% If the argument is not empty and does not match |\childdocname|
% (which is set by the first inclusion of |childdoc.def|),
% |\ifchilddoc| is set to true, |\includeonly| is applied to the child file
% and |\jobname| is set to the main file
% (for proper handling of |.aux| files):
%    \begin{macrocode}
\newcommand{\childdocmain}[1]
{
  \childdocdisable\childdocmain{}
  \if?#1?\else
    \begingroup
      \def\childdoctmp{#1}
      \ifx\childdoctmp\childdocname
        \def\childdoctmp{}
      \else
        \def\childdoctmp
        {
          \childdoctrue
          \includeonly{\childdocname}
          \def\childdocjob{#1}
          \def\jobname{#1}
        }
      \fi
      \expandafter
    \endgroup
    \childdoctmp
  \fi
}
%    \end{macrocode}

% \macro{\childdocof}
% The command |\childdocof| redirects
% compilation to the main file |#1|.
%    \begin{macrocode}
\newcommand{\childdocof}[1]
{
  \childdocdisable
  \childdoctrue
  \includeonly{\childdocname}
  \def\jobname{#1}
  \def\childdocjob{#1}
  \input{#1}
}
%    \end{macrocode}

% \macro{\childdocby}
% The command |\childdocby| ....
%    \begin{macrocode}
\newcommand{\childdocby}[2][]
{
  \childdocdisable
  \childdoctrue
  \childdocmanualtrue
  \if?#1?\else
    \def\jobname{#2}
  \fi
  \def\childdocjob{#2}
  \input{#2}
  \endinput
}
%    \end{macrocode}

% \macro{\childdocforward}
% The command |\childdocforward| redirects
% compilation to the main file or
% (if the optional argument is given) a child file.
% Parameters are set as if the main file
% or a child file starting with |\childdocof| was compiled.
% Then compilation is handed over to the main file:
%    \begin{macrocode}
\newcommand{\childdocforward}[2][]
{
  \begingroup
    \if?#1?
      \def\childdoctmp
      {
        \def\childdocname{#2}
        \def\childdocjob{#2}
        \def\jobname{#2}
        \input{#2}
        \endinput
      }
    \else
      \def\childdoctmp
      {
        \childdocdisable
        \def\childdocname{#2}
        \childdoctrue
        \includeonly{#2}
        \def\childdocjob{#1}
        \def\jobname{#1}
        \input{#1}
        \endinput
      }
    \fi
    \expandafter
  \endgroup
  \childdoctmp
}
%    \end{macrocode}

% \macro{\childdocforwardprefix}
% The command |\childdocforwardprefix| redirects
% compilation to the main or a child file by means of a pattern.
% The prefix |#1| in the current filename is replaced by |#2|
% and the suffix of the current filename is kept
% (it is assumed that the filename does not contain the substring `|~~~|'
% which is used as a delimiter).
% Compilation is handed over to the new file by |\childdocforward|:
%    \begin{macrocode}
\newcommand{\childdocforwardprefix}[3][]
{
  \begingroup
    \def\childdocextract #2##1~~~{\def\childdoctmp{\childdocforward[#1]{#3##1}}}
    \expandafter\childdocextract\childdocname~~~
    \expandafter
  \endgroup
  \childdoctmp
}
%    \end{macrocode}

% \macro{\childdoc}
% The deprecated macro |\childdoc| is a legacy version of |\childdocmain|:
%    \begin{macrocode}
\newcommand{\childdoc}{\childdocmain}
%    \end{macrocode}

% \macro{\childdocredirect}
% The deprecated macro |\childdocredirect| is a legacy version
% of |\childdocforward| and |\childdocforwardprefix|:
%    \begin{macrocode}
\newcommand{\childdocredirect}[2][]
{
  \begingroup
    \if?#1?
      \def\childdoctmp{\childdocforward{#2}}
    \else
      \def\childdoctmp{\childdocforwardprefix{#1}{#2}}
    \fi
    \expandafter
  \endgroup
  \childdoctmp
}
%    \end{macrocode}

%\iffalse
%</package>
%\fi
%
\endinput

\childdocby{cdocsamp}
%    \end{macrocode}

%\iffalse
%</samplepart3|samplepart4>
%\fi
%
%\iffalse
%<*samplepart3>
%\fi
% Some text for part 3:
%    \begin{macrocode}
some text in part three
%    \end{macrocode}

%\iffalse
%</samplepart3>
%\fi
% Some text for part 4:
%\iffalse
%<*samplepart4>
%\fi
%    \begin{macrocode}
more text in part four
%    \end{macrocode}

%\iffalse
%</samplepart4>
%\fi
%
% %%%%%%%%%%%%%%%%%%%%%%%%%%%%%%%%%%%%%%
% \paragraph{Forwarding for a Complete Draft.}
%
% The following forwarding file |cdocsdrf.tex|
% compiles the main document in draft mode:
%\iffalse
%<*sampledraft>
%\fi
%    \begin{macrocode}
\def\version{draft}
% \iffalse
%
% childdoc.dtx Copyright (C) 2017-2018 Niklas Beisert
%
% This work may be distributed and/or modified under the
% conditions of the LaTeX Project Public License, either version 1.3
% of this license or (at your option) any later version.
% The latest version of this license is in
%   http://www.latex-project.org/lppl.txt
% and version 1.3 or later is part of all distributions of LaTeX
% version 2005/12/01 or later.
%
% This work has the LPPL maintenance status `maintained'.
%
% The Current Maintainer of this work is Niklas Beisert.
%
% This work consists of the files childdoc.dtx and childdoc.ins
% and the derived files childdoc.def and cdocsamp.tex with
% cdocsch1.tex, cdocsch2.tex, cdocsdrf.tex, cdocsfn1.tex, cdocsfn2.tex.
%
%<package>\ifdefined\childdocmain\endinput\fi
%<package>\ProvidesFile{childdoc.def}[2018/12/30 v2.0 child document driver]
%<samplemain>\ProvidesFile{cdocsamp.tex}[2018/12/30 v2.0 sample for childdoc]
%<*driver>
%\ProvidesFile{childdoc.drv}[2018/12/30 v2.0 childdoc reference manual file]
\PassOptionsToClass{10pt,a4paper}{article}
\documentclass{ltxdoc}

\usepackage[margin=35mm]{geometry}
\usepackage{hyperref}
\usepackage{hyperxmp}
\usepackage[usenames]{color}

\hypersetup{colorlinks=true}
\hypersetup{pdfstartview=FitH}
\hypersetup{pdfpagemode=UseNone}
\hypersetup{pdfsource={}}
\hypersetup{pdflang={en-UK}}
\hypersetup{pdfcopyright={Copyright 2017-2018 Niklas Beisert.
  This work may be distributed and/or modified under the
  conditions of the LaTeX Project Public License, either version 1.3
  of this license or (at your option) any later version.}}
\hypersetup{pdflicenseurl={http://www.latex-project.org/lppl.txt}}
\hypersetup{pdfcontactaddress={ETH Zurich, ITP, HIT K,
  Wolfgang-Pauli-Strasse 27}}
\hypersetup{pdfcontactpostcode={8093}}
\hypersetup{pdfcontactcity={Zurich}}
\hypersetup{pdfcontactcountry={Switzerland}}
\hypersetup{pdfcontactemail={nbeisert@itp.phys.ethz.ch}}
\hypersetup{pdfcontacturl={http://people.phys.ethz.ch/\xmptilde nbeisert/}}

\newcommand{\secref}[1]{\hyperref[#1]{section \ref*{#1}}}

\parskip1ex
\parindent0pt
\let\olditemize\itemize
\def\itemize{\olditemize\parskip0pt}

\begin{document}

\title{The \textsf{childdoc} Package}
\hypersetup{pdftitle={The childdoc Package}}
\author{Niklas Beisert\\[2ex]
  Institut f\"ur Theoretische Physik\\
  Eidgen\"ossische Technische Hochschule Z\"urich\\
  Wolfgang-Pauli-Strasse 27, 8093 Z\"urich, Switzerland\\[1ex]
  \href{mailto:nbeisert@itp.phys.ethz.ch}
  {\texttt{nbeisert@itp.phys.ethz.ch}}}
\hypersetup{pdfauthor={Niklas Beisert}}
\hypersetup{pdfsubject={Manual for the LaTeX2e Package childdoc}}
\date{30 December 2018, \textsf{v2.0}}
\maketitle

\begin{abstract}\noindent
\textsf{childdoc} is a \LaTeXe{} package
that enables the direct compilation
of document sections included by |\include|
to individual files.
\end{abstract}

\begingroup
\parskip0ex
\tableofcontents
\endgroup

%%%%%%%%%%%%%%%%%%%%%%%%%%%%%%%%%%%%%%%%%%%%%%%%%%%%%%%%%%%%%%%%%%%%%%%%%%%%%%%%
%%%%%%%%%%%%%%%%%%%%%%%%%%%%%%%%%%%%%%%%%%%%%%%%%%%%%%%%%%%%%%%%%%%%%%%%%%%%%%%%
\section{Introduction}

\LaTeX{} provides a mechanism to structure a large document (such as a book)
into a main file and several child files (containing the chapters)
using the |\include| command.
This mechanism is beneficial for documents
which span hundreds of pages in order to
make the source file(s) more manageable.
Moreover, compilation can be restricted to
selected child files by means of the |\includeonly| command.
The latter feature can be used to reduce the compilation time while editing
(this was significantly more useful in the earlier days of \LaTeX{})
or to generate a smaller document which is easier to navigate.
Another application of |\includeonly| is to generate
documents consisting of selected parts of the complete document.

However, there are a few drawbacks of the plain |\include| mechanism:
\begin{itemize}
\item
The child files cannot be compiled on their own,
they can only be compiled via the main file.
A naive editing environment
(such as a text editor with an option
to have the current file processed by \LaTeX)
may require one to switch to the main file before compiling;
attempting to compile the child file produces errors.
\item
The main file must be modified (each time)
to adjust the |\includeonly| command
to the present needs. This easily leaves the main file in a messy state.
\item
The generated document will always carry the filename
of the main document. This is inconvenient if
several child files are to be compiled and
to be kept for distribution.
\end{itemize}

The present package provides a simple interface
to make child files individually compilable by \LaTeX{}.
Compiling a child file then has the same effect as compiling
the main file with an |\includeonly| command
to select the appropriate child.
Moreover the generated document will carry the name of the child
rather than the main file.
This resolves all three above issues.

This feature is meant to make the editing of books,
thesis documents and lecture notes somewhat more convenient.
However, the package can also be used efficiently for
composing a series of documents (such as exercise sheets)
which are typically distributed individually.
It then assists the author in generating the individual documents
(potentially in different versions)
as well as a document containing the collected series.
Another application is in developing style files
or other kinds of included material
where compilation of the style file could redirect
to a sample or test file.

%%%%%%%%%%%%%%%%%%%%%%%%%%%%%%%%%%%%%%%%%%%%%%%%%%%%%%%%%%%%%%%%%%%%%%%%%%%%%%%%
%%%%%%%%%%%%%%%%%%%%%%%%%%%%%%%%%%%%%%%%%%%%%%%%%%%%%%%%%%%%%%%%%%%%%%%%%%%%%%%%
\section{Usage}

First of all, the package \textsf{childdoc} is \emph{not} a standard
\LaTeXe{} |.sty| style file! Therefore it needs to be invoked in
a non-standard way.

%%%%%%%%%%%%%%%%%%%%%%%%%%%%%%%%%%%%%%%%%%%%%%%%%%%%%%%%%%%%%%%%%%%%%%%%%%%%%%%%
\subsection{Included Files}
\label{sec:include}

%%%%%%%%%%%%%%%%%%%%%%%%%%%%%%%%%%%%%%%%
\DescribeMacro{\childdocmain}
To use the package, add the commands
\begin{center}
\begin{tabular}{l}
|\input{childdoc.def}|\\
|\childdocmain{}|\\
\end{tabular}
\end{center}
at the very top of the main \LaTeX{} file,
in particular \emph{before} the |\documentclass| statement!
The argument of |\childdocmain| should be left empty
(but it must be present).

%%%%%%%%%%%%%%%%%%%%%%%%%%%%%%%%%%%%%%%%
\DescribeMacro{\childdocof}
Furthermore, add the commands
\begin{center}
\begin{tabular}{l}
|\input{childdoc.def}|\\
|\childdocof{|\textit{main}|}|\\
\end{tabular}
\end{center}
at the top of every child file \textit{child}
which is included by |\include{|\textit{child}|}|
from within the main file
(or at least for those files to be compiled individually).
The argument \textit{main} must be the filename of the main file.

There are a couple of
considerations in setting up the main and child documents:

%%%%%%%%%%%%%%%%%%%%%%%%%%%%%%%%%%%%%%%%
\paragraph{Restrictions.}

Please note the following restrictions:
\begin{itemize}
\item
|\childdocmain| must be called with one argument \textit{main}
to ensure compatibility with earlier version of the package.
It must either be empty (|\childdocmain{}|)
or precisely match the filename of the main file in which it is specified.
See \secref{sec:detection} for further information.
\item
The filename \textit{main} must be specified without the |.tex| extension.
\item
The filename \textit{main} is case sensitive
(even in case-insensitive file systems)
due to internal string comparison.
\item
The argument \textit{main} should be fully expanded, it cannot be a macro.
\item
Subdirectories and special characters should be avoided in filenames.
\item
The command |\childdocmain{|\textit{main}|}| must be followed by a whitespace.
It should not be followed immediately by another command
or by a comment mark `|%|'.
This is because the \TeX{} parser reads the token immediately following
the argument of |\childdocmain| and puts it
at the beginning of every child section;
however, a white\-space is ignored.
\end{itemize}

%%%%%%%%%%%%%%%%%%%%%%%%%%%%%%%%%%%%%%%%
\paragraph{Content of Main File.}

It is advisable to place all content in the child files included by |\include|.
Any output contained in the main file will appear in all child documents
unless suppressed manually;
it cannot be suppressed automatically by the |\includeonly| directive
and thus should normally be avoided.
A method to include some content in the main file
by means of conditional processing is described in \secref{sec:conditional}.

%%%%%%%%%%%%%%%%%%%%%%%%%%%%%%%%%%%%%%%%
\paragraph{Page Numbering.}

When only a part of the document is compiled,
the appropriate numbering of pages
(as well as other status parameters)
is determined from the |.aux| files.
The latter contain information from previous passes.
However this information needs to propagate through
all intermediate child documents.
Therefore the page numbering in child documents may well
be inconsistent until the complete document is compiled at least once.

A useful (if unconventional) way to always ensure a consistent
page numbering is to restart the numbering in each child document
and denote the pages by `\textit{child}|.|\textit{page}'
where \textit{child} represents the chapter/section number of the child file.
This can be achieved by the command
|\numberwithin{page}{|\textit{child}|}|
of the \textsf{amsmath} package
where \textit{child} can be |chapter| or |section|
depending on the chosen structuring.
Alternatively, one can modify the macro |\thepage| appropriately
and reset the counter |page| at the start of each child file.

%%%%%%%%%%%%%%%%%%%%%%%%%%%%%%%%%%%%%%%%%%%%%%%%%%%%%%%%%%%%%%%%%%%%%%%%%%%%%%%%
\subsection{Conditional Processing}
\label{sec:conditional}

The package provides a mechanism to compile different versions
of a document. To customise the versions further some conditional processing
can come in handy to distinguish which version is being compiled.
The package provides two macros to describe the compilation context:

%%%%%%%%%%%%%%%%%%%%%%%%%%%%%%%%%%%%%%%%
\DescribeMacro{\ifchilddoc}
The conditional |\ifchilddoc| distinguishes between the compilation of
child documents and the main document:
%
\begin{center}
|\ifchilddoc |\textit{child-code}| |[|\||else |\textit{main-code}]| \||fi|
\end{center}

%%%%%%%%%%%%%%%%%%%%%%%%%%%%%%%%%%%%%%%%
\DescribeMacro{\childdocname}
\DescribeMacro{\childdocjob}
The macro |\childdocname| contains the filename (without extension)
of the main or child file being processed.
Note that |\childdocjob| will always contain the name of the main file.

%%%%%%%%%%%%%%%%%%%%%%%%%%%%%%%%%%%%%%%%
\paragraph{Title Page.}

Conditional processing can be used to include a title or banner page
in the main document when proper precautions are taken.
Importantly, the code in the main file should ensure that the page counter
(as well as other status parameters which are stored in the |.aux| files)
takes the same value after the conditional processing.
Otherwise the page numbers may take divergent values
depending on which part is compiled.

For example, a title page could be declared by:
%
\begin{center}
\begin{tabular}{l}
|\ifchilddoc\||else|\\
|\addtocounter{page}{-1}|\\
\textit{code for title page}\\
|\newpage|\\
|\||fi|
\end{tabular}
\end{center}
%
A banner page for the child documents can be generated by:
%
\begin{center}
\begin{tabular}{l}
|\ifchilddoc|\\
|\addtocounter{page}{-1}|\\
\textit{code for banner page}\\
|\newpage|\\
|\||fi|
\end{tabular}
\end{center}
%
Here one could write a message such as:
\begin{center}
|This is the part \childdocname{} of \childdocjob{}.|
\end{center}

%%%%%%%%%%%%%%%%%%%%%%%%%%%%%%%%%%%%%%%%%%%%%%%%%%%%%%%%%%%%%%%%%%%%%%%%%%%%%%%%
\subsection{Flags}
\label{sec:flags}

The package makes it easy to generate different versions
of the main or child documents.
To this end compilation flags can be defined
and assigned different default values.
They will be particularly useful in conjunction
with the forwarding mechanism described in \secref{sec:forward}.

For example, it may be useful to have a flag |\version|
which can be set to |draft| or |final|.
The document source will contain some conditional code
depending on the value of |\version|.
Suppose further, the flag should default to |final| for the main file
and to |draft| for child files
which is a natural assignment for editing the document.
This is achieved by placing the following code
in the preamble of the main document
(below the |\childdocmain| directive):
%
\begin{center}
\begin{tabular}{l}
|\ifchilddoc|\\
|\providecommand{\version}{draft}|\\
|\||else|\\
|\providecommand{\version}{final}|\\
|\||fi|
\end{tabular}
\end{center}
%
The definition by |\providecommand| makes sure
that previous definitions are not overwritten.
Further statements |\providecommand{\version}{...}|
can thus be added before the above code to override it.

For the main file, one might add a line
(between |\childdocmain| and the above block)
%
\begin{center}
|%\ifchilddoc\||else\providecommand{\version}{draft}\||fi|
\end{center}
%
which can be uncommented to produce a draft version.
Likewise one can add a line to the very top of a child file
(above the |\childdocof{|\textit{main}|}| directive)
%
\begin{center}
|%\providecommand{\version}{final}|
\end{center}
%
which can be uncommented to produce the final version of this child document.

%%%%%%%%%%%%%%%%%%%%%%%%%%%%%%%%%%%%%%%%%%%%%%%%%%%%%%%%%%%%%%%%%%%%%%%%%%%%%%%%
\subsection{Forwarding}
\label{sec:forward}

Different versions of the main or child documents
using compilation flags as described in \secref{sec:flags}
can be (permanently) stored in different files
for convenient compilation, viewing and distribution.
To this end, the package defines a command
to pass on compilation to a different file:

%%%%%%%%%%%%%%%%%%%%%%%%%%%%%%%%%%%%%%%%
\DescribeMacro{\childdocforward}
The command |\childdocforward| redirects processing to
another source file:
%
\begin{center}
\begin{tabular}{l}
|\input{childdoc.def}|\\
|\childdocforward[|\textit{main}|]{|\textit{dest}|}|\\
\end{tabular}
\end{center}
%
The argument \textit{dest} is the destination file
(without extension).
It should be the main file or one of the child files.
Note that further \textsf{childdoc} directives
such as |\childdocof| and |\childdocforward|
in the indicated file will be processed in this form.
The optional argument \textit{main}
passes on directly to the main file \textit{main}
while pretending to compile the child \textit{dest}.
This form behaves as if \textit{dest}
issues |\childdocof{|\textit{main}|}| right away,
and no further \textsf{childdoc} directives will be processed.

%%%%%%%%%%%%%%%%%%%%%%%%%%%%%%%%%%%%%%%%
\DescribeMacro{\...prefix}
In the alternative form |\childdocforwardprefix|,
%
\begin{center}
\begin{tabular}{l}
|\input{childdoc.def}|\\
|\childdocforwardprefix[|\textit{main}|]{|\textit{prefix}|}{|\textit{dest}|}|
\end{tabular}
\end{center}
%
the destination file is determined by a pattern
depending on the current file:
To make this work, the current file must be called
`{\textit{prefix}\hspace{0.2em}\textit{suffix}}'
with \textit{prefix} matching precisely the argument.
Processing is then passed on to the file
`{\textit{dest}\hspace{0.2em}\textit{suffix}}'.
Surely, the same effect is achieved by
directly specifying the
argument `{\textit{dest}\hspace{0.2em}\textit{suffix}}'
in the first form.
However, that requires to set up a different file
for each child. With the alternative form of the command
all these files can have exactly the same content
which simplifies setting them up and maintaining them.

For example, the following file |draft.tex|
with a compilation flag |\version| as described in \secref{sec:flags}
compiles the main document as a draft:
%
\begin{center}
\begin{tabular}{l}
|\def\version{draft}|\\
|\input{childdoc.def}|\\
|\childdocforward{|\textit{main}|}|
\end{tabular}
\end{center}
%
Likewise, the following files |final|\textit{nn}|.tex|
compile the final version of the child document
|child|\textit{nn}|.tex|:
%
\begin{center}
\begin{tabular}{l}
|\def\version{final}|\\
|\input{childdoc.def}|\\
|\childdocforwardprefix{final}{child}|
\end{tabular}
\end{center}
%

Note that when several versions of a main file and/or of each child file
are to be generated, it may be convenient to set up a |Makefile| or
shell script to automatise the process.

%%%%%%%%%%%%%%%%%%%%%%%%%%%%%%%%%%%%%%%%%%%%%%%%%%%%%%%%%%%%%%%%%%%%%%%%%%%%%%%%
\subsection{Command Line Processing}
\label{sec:commandline}

The effect of redirection files can also be achieved by invoking
the \LaTeX{} compiler with a more elaborate command line.
Most conveniently this should be done as part
of a shell script or a |Makefile|.

When using \textsf{childdoc} in the main file, the following
command lines effectively perform a redirection
(note that depending on the shell being used,
backslashes may have to be doubled: `|\|' $\to$ `|\\|'):
%
\begin{center}
|... -jobname "|\textit{target}|" |\\|"|[\textit{flags}]%
|\input{childdoc.def}\childdocforward[|\textit{main}|]{|\textit{dest}|}"|
\end{center}
%
Here \textit{target} is the name of the output file,
\textit{main} is the name of the main file
and \textit{dest} is the name of the main or child file to be processed
(all filenames without extensions).
The optional argument \textit{main} can be omitted
if \textit{main} matches \textit{dest}.
Optionally, compilation \textit{flags} can be defined via |\def| commands.
This command line makes the \TeX{} engine believe
it is compiling the file \textit{target}
whose content is specified as the latter parameter.
The provided code then forwards the processing to
\textit{main} or \textit{dest} as described in \secref{sec:forward}.

%%%%%%%%%%%%%%%%%%%%%%%%%%%%%%%%%%%%%%%%%%%%%%%%%%%%%%%%%%%%%%%%%%%%%%%%%%%%%%%%
\subsection{Include by Input}
\label{sec:input}

Including child documents by |\include| has some restrictions by design.
Most notably, the content of a child document always occupies
its own set of pages; pages cannot be shared between child documents.
Usually, this behaviour makes perfect sense
because each child document contain an essential part of the document.
However, in some situations it may be desirable to compose
a document from a collection of parts
without having mandatory page breaks between then.
For this case, the package
provides a mechanism to include parts
by |\input| which can also be processed individually.
However, by construction this mechanism
requires manual handling of the content to be output.

%%%%%%%%%%%%%%%%%%%%%%%%%%%%%%%%%%%%%%%%
\DescribeMacro{\ifchilddocmanual}
The main file should be prepared as usual, see \secref{sec:include}.
However, the document body must make a distinction
between processing of an individual part and of the main document, e.g.:
%
\begin{center}
\begin{tabular}{l}
|\ifchilddocmanual|\\
|\input{\childdocname}|\\
|\||else|\\
\textit{document body with }|\input{|\textit{part}|}|\\
|\||fi|
\end{tabular}
\end{center}
%
The conditional |\ifchilddocmanual| is true whenever
a part to be included by |\input| is being compiled,
and the name of the part is stored in |\childdocname|.

%%%%%%%%%%%%%%%%%%%%%%%%%%%%%%%%%%%%%%%%
\DescribeMacro{\childdocby}
Each part to be included by |\input| should start with:
%
\begin{center}
\begin{tabular}{l}
|\input{childdoc.def}|\\
|\childdocby{|\textit{main}|}|\\
\end{tabular}
\end{center}
%
The directive |\childdocby| is similar to |\childdocof|
described in \secref{sec:include},
but the subsequent selection of content must be done manually.
To that end, both |\ifchilddoc| and |\ifchilddocmanual|
will be true upon processing of a part,
and the name of the part is stored in |\childdocname|.
Note that |\jobname| will be set to the filename of the current part
so that each part receives an individual |.aux| file
that does not interfere with the |.aux| file(s) of the main document.
This behaviour can be altered by the alternative form
|\childdocby[*]{|\textit{main}|}| (with a non-empty optional argument)
which uses the |.aux| file of the main document
by setting |\jobname| to \textit{main}.

%%%%%%%%%%%%%%%%%%%%%%%%%%%%%%%%%%%%%%%%%%%%%%%%%%%%%%%%%%%%%%%%%%%%%%%%%%%%%%%%
\subsection{Driver Development}
\label{sec:driver}

The \textsf{childdoc} mechanism can also be use for the development
of definition files such as \LaTeX{} styles or classes.
This case differs from the above setup with multiple parts
included by |\include| in that no |\includeonly| should be invoked.
This can be achieved by starting the include file
(before |\ProvidesPackage|) with:
%
\begin{center}
\begin{tabular}{l}
|\input{childdoc.def}|\\
|\childdocforward{|\textit{main}|}|\\
\end{tabular}
\end{center}
%
or alternatively with:
%
\begin{center}
\begin{tabular}{l}
|\input{childdoc.def}|\\
|\childdocby{|\textit{main}|}|\\
\end{tabular}
\end{center}
%
Both forms have slightly different effects as described above.
The main file is prepared as usual, see \secref{sec:include}.

%%%%%%%%%%%%%%%%%%%%%%%%%%%%%%%%%%%%%%%%%%%%%%%%%%%%%%%%%%%%%%%%%%%%%%%%%%%%%%%%
\subsection{Legacy Detection}
\label{sec:detection}

The directive |\childdocmain| in the main file can detect
whether the complete document or merely a child is to be compiled
even without using the directive |\childdocof|.
This method is deprecated because it is less robust
and there is no compelling reason to use it;
it is merely provided for backward compatibility
and it may be removed in future versions.

If the detection mechanism is to be used,
it is mandatory to correctly specify
the filename of the main file as the argument of |\childdocmain|:
%
\begin{center}
\begin{tabular}{l}
|\input{childdoc.def}|\\
|\childdocmain{|\textit{main}|}|\\
\end{tabular}
\end{center}
%
If |\jobname| does not match the argument \textit{main} of |\childdocmain|,
it is assumed that |\jobname| points to the child file to be compiled.
When using |\childdocmain| with the main file specified as argument,
it suffices to start a child file
with just |\input{|\textit{main}|}|
without loading of the package and using |\childdocof|.
If instead all processing is done
with the appropriate \textsf{childdoc} directives,
the argument of \textit{main} of |\childdocmain| can be empty.

An alternative version of the command line processing described
in \secref{sec:commandline} using the detection mechanism reads:
%
\begin{center}
|... -jobname "|\textit{target}|" "|[\textit{flags}]%
[|\def\jobname{|\textit{dest}|}|]|\input{|\textit{main}|}"|
\end{center}

%%%%%%%%%%%%%%%%%%%%%%%%%%%%%%%%%%%%%%%%%%%%%%%%%%%%%%%%%%%%%%%%%%%%%%%%%%%%%%%%
\subsection{Manual Code}
\label{sec:manual}

In case one cannot be certain whether the definitions file |childdoc.def|
is installed on the target \TeX{} distribution
and one prefers not to ship it,
it is conceivable to paste a few relevant commands into the sources.

To that end, drop all statements |\input{childdoc.def}|
and perform the replacements as outlined below.
Instead of |\childdocmain{|\textit{main}|}| add the following code
to the top of the main file:
%
\begin{center}
\begin{tabular}{l}
|\||ifdefined\childdocname\endinput\||fi\newif\ifchilddoc|\\
|\edef\childdocname{\scantokens\expandafter{\jobname\noexpand}}|\\
|\def\childdocmain{|\textit{main}|}\||ifx\childdocmain\childdocname\||else|\\
|\childdoctrue\includeonly{\childdocname}\let\jobname\childdocmain\||fi|\\
\end{tabular}
\end{center}
%
Instead of |\childdocof{|\textit{main}|}| just include the main file
at the top of each child file:
%
\begin{center}
|\input{|\textit{main}|}|
\end{center}
%
A simple redirection |\childdocforward{|\textit{dest}|}| is achieved by:
%
\begin{center}
|\def\jobname{|\textit{dest}|}\input{\jobname}|
\end{center}
%
The redirection with prefix
|\childdocforwardprefix[|\textit{prefix}|]{|\textit{dest}|}|
is accomplished by:
%
\begin{center}
\begin{tabular}{l}
|{\edef\jobname{\scantokens\expandafter{\jobname\noexpand}}|\\
|\def\redirectjob |\textit{prefix}|#1~~~{\gdef\jobname{|\textit{dest}|#1}}|\\
|\expandafter\redirectjob\jobname~~~}\input{\jobname}|
\end{tabular}
\end{center}

In an alternative approach,
child documents can be compiled by a specific command line
without additional code or specific definitions:
%
\begin{center}
|... -jobname "|\textit{target}|" "|[\textit{flags}]%
|\includeonly{|\textit{dest}|}\input{|\textit{main}|}"|
\end{center}
%

%%%%%%%%%%%%%%%%%%%%%%%%%%%%%%%%%%%%%%%%%%%%%%%%%%%%%%%%%%%%%%%%%%%%%%%%%%%%%%%%
%%%%%%%%%%%%%%%%%%%%%%%%%%%%%%%%%%%%%%%%%%%%%%%%%%%%%%%%%%%%%%%%%%%%%%%%%%%%%%%%
\section{Information}

%%%%%%%%%%%%%%%%%%%%%%%%%%%%%%%%%%%%%%%%%%%%%%%%%%%%%%%%%%%%%%%%%%%%%%%%%%%%%%%%
\subsection{Copyright}

Copyright \copyright{} 2017--2018 Niklas Beisert

This work may be distributed and/or modified under the
conditions of the \LaTeX{} Project Public License, either version 1.3
of this license or (at your option) any later version.
The latest version of this license is in
  \url{http://www.latex-project.org/lppl.txt}
and version 1.3 or later is part of all distributions of \LaTeX{}
version 2005/12/01 or later.

This work has the LPPL maintenance status `maintained'.

The Current Maintainer of this work is Niklas Beisert.

This work consists of the files |README.txt|, |childdoc.ins| and |childdoc.dtx|
as well as the derived files |childdoc.def|, |cdocsamp.tex|
with |cdocsch1.tex|, |cdocsch2.tex|, |cdocspt3.tex|, |cdocspt4.tex|,
|cdocsdrf.tex|, |cdocsfn1.tex|, |cdocsfn2.tex|
as well as |childdoc.pdf|.

%%%%%%%%%%%%%%%%%%%%%%%%%%%%%%%%%%%%%%%%%%%%%%%%%%%%%%%%%%%%%%%%%%%%%%%%%%%%%%%%
\subsection{Files and Installation}

The package consists of the files:
%
\begin{center}
\begin{tabular}{ll}
    |README.txt|   & readme file \\
    |childdoc.ins| & installation file \\
    |childdoc.dtx| & source file \\
    |childdoc.def| & definition file \\
    |cdocsamp.tex| & sample main file \\
    |cdocsch1.tex| & sample include file \\
    |cdocsch2.tex| & sample include file \\
    |cdocspt3.tex| & sample part file \\
    |cdocspt4.tex| & sample part file \\
    |cdocsdrf.tex| & sample redirection file \\
    |cdocsfn1.tex| & sample redirection file \\
    |cdocsfn2.tex| & sample redirection file \\
    |childdoc.pdf| & manual
\end{tabular}
\end{center}
%
The distribution consists of the files
|README.txt|, |childdoc.ins| and |childdoc.dtx|.
%
\begin{itemize}
\item
Run (pdf)\LaTeX{} on |childdoc.dtx|
to compile the manual |childdoc.pdf| (this file).
\item
Run \LaTeX{} on |childdoc.ins| to create the definitions file |childdoc.def|
and the sample |cdocsamp.tex| with include files
|cdocsch1.tex|, |cdocsch2.tex|, |cdocspt3.tex|, |cdocspt4.tex|,
|cdocsdrf.tex|, |cdocsfn1.tex|, |cdocsfn2.tex|.
Then copy the file |childdoc.def| to an appropriate directory of your \LaTeX{}
distribution, e.g.\ \textit{texmf-root}|/tex/latex/childdoc|.
\end{itemize}

%%%%%%%%%%%%%%%%%%%%%%%%%%%%%%%%%%%%%%%%%%%%%%%%%%%%%%%%%%%%%%%%%%%%%%%%%%%%%%%%
\subsection{Related CTAN Packages}

There are several other packages which offer a similar functionality:
%
\begin{itemize}
\item
The packages
\href{http://ctan.org/pkg/docmute}{\textsf{docmute}},
\href{http://ctan.org/pkg/includex}{\textsf{includex}} and
\href{http://ctan.org/pkg/standalone}{\textsf{standalone}}
provide commands to include only the document body of
a child file thus allowing both files to be compiled individually.
\item
The packages \href{http://ctan.org/pkg/subdocs}{\textsf{subdocs}}
and \href{http://ctan.org/pkg/subfiles}{\textsf{subfiles}}
provide structures in which the main and child documents can be
encapsulated and allowing them to be compiled individually.
The inclusion mechanism is different from the conventional |\include|.
\item
The package \href{http://ctan.org/pkg/combine}{\textsf{combine}}
is an elaborate solution to combine several documents into one.
\end{itemize}
%
See also the CTAN topic \href{http://ctan.org/topic/subdocs}{\textsf{subdocs}}
for further related packages.
The present package differs from the above solutions in that
a document structure constructed with the conventional |\include| mechanism
just needs two extra commands at the top of every file
such that all constituent files can be compiled individually.

%%%%%%%%%%%%%%%%%%%%%%%%%%%%%%%%%%%%%%%%%%%%%%%%%%%%%%%%%%%%%%%%%%%%%%%%%%%%%%%%
%\subsection{Feature Suggestions}
%
%The following is a list of features which may be useful for future
%versions of this package:
%%
%\begin{itemize}
%\item
%\ldots
%\end{itemize}

%%%%%%%%%%%%%%%%%%%%%%%%%%%%%%%%%%%%%%%%%%%%%%%%%%%%%%%%%%%%%%%%%%%%%%%%%%%%%%%%
\subsection{Revision History}

%%%%%%%%%%%%%%%%%%%%%%%%%%%%%%%%%%%%%%%%
\paragraph{v2.0:} 2018/12/30

\begin{itemize}
\item
immediate forward processing
\item
added |\childdocby| mechanism
\item
manual restructured
\end{itemize}

%%%%%%%%%%%%%%%%%%%%%%%%%%%%%%%%%%%%%%%%
\paragraph{v1.6:} 2018/01/17

\begin{itemize}
\item
application for development of include files
\item
corrections to manual
\end{itemize}

%%%%%%%%%%%%%%%%%%%%%%%%%%%%%%%%%%%%%%%%
\paragraph{v1.5:} 2017/05/21

\begin{itemize}
\item
more complete structuring introduced
\item
|\childdocof| introduced
\item
|\childdoc| renamed to |\childdocmain|
\item
|\childredirect| renamed to |\childdocforward| and |\childdocforwardprefix|
and functionality expanded
\end{itemize}

%%%%%%%%%%%%%%%%%%%%%%%%%%%%%%%%%%%%%%%%
\paragraph{v1.0:} 2017/04/27

\begin{itemize}
\item
manual and install package
\item
first version published on CTAN
\end{itemize}

%%%%%%%%%%%%%%%%%%%%%%%%%%%%%%%%%%%%%%%%
\paragraph{v0.6:} 2017/04/26

\begin{itemize}
\item
redirection mechanism added
\end{itemize}

%%%%%%%%%%%%%%%%%%%%%%%%%%%%%%%%%%%%%%%%
\paragraph{v0.5:} 2017/04/26

\begin{itemize}
\item
functionality in definition file
\end{itemize}


%%%%%%%%%%%%%%%%%%%%%%%%%%%%%%%%%%%%%%%%%%%%%%%%%%%%%%%%%%%%%%%%%%%%%%%%%%%%%%%%
%%%%%%%%%%%%%%%%%%%%%%%%%%%%%%%%%%%%%%%%%%%%%%%%%%%%%%%%%%%%%%%%%%%%%%%%%%%%%%%%
%%%%%%%%%%%%%%%%%%%%%%%%%%%%%%%%%%%%%%%%%%%%%%%%%%%%%%%%%%%%%%%%%%%%%%%%%%%%%%%%
\appendix

\settowidth\MacroIndent{\rmfamily\scriptsize 000\ }

 \DocInput{childdoc.dtx}

\end{document}
%</driver>
% \fi
%
% %%%%%%%%%%%%%%%%%%%%%%%%%%%%%%%%%%%%%%%%%%%%%%%%%%%%%%%%%%%%%%%%%%%%%%%%%%%%%%
% %%%%%%%%%%%%%%%%%%%%%%%%%%%%%%%%%%%%%%%%%%%%%%%%%%%%%%%%%%%%%%%%%%%%%%%%%%%%%%
% \section{Sample}
%\iffalse
%<*samplemain>
%\fi
%
% The following presents a sample document
% with two chapters, two parts, a title page,
% a compile flag as well as three forwarding files to set the flag.
% It consists of eight |.tex| files:
% \begin{center}
% \begin{tabular}{ll}
% |cdocsamp.tex|&main file\\
% |cdocsch1.tex|&include file for chapter 1\\
% |cdocsch2.tex|&include file for chapter 2\\
% |cdocspt3.tex|&include file for part 3\\
% |cdocspt4.tex|&include file for part 4\\
% |cdocsdrf.tex|&forwarding file for main file in draft mode\\
% |cdocsfi1.tex|&forwarding file for final version of chapter 1\\
% |cdocsfi2.tex|&forwarding file for final version of chapter 2\\
% \end{tabular}
% \end{center}
% Each of the eight files can be compiled directly by the \LaTeX{} compiler.
%
% %%%%%%%%%%%%%%%%%%%%%%%%%%%%%%%%%%%%%%
% \paragraph{Main File.}
%
% The main file is called |cdocsamp.tex|.
%
% Load the \textsf{childdoc} definitions and
% declare the filename for the main document:
%    \begin{macrocode}
\input{childdoc.def}
\childdocmain{}
%    \end{macrocode}

% Optional override for |\version| flag:
%    \begin{macrocode}
%%\ifchilddoc\else\providecommand{\version}{draft}\fi
%    \end{macrocode}

% Define the default values for the |\version| flag
% (|final| for the main file and |draft| for childs):
%    \begin{macrocode}
\ifchilddoc
\providecommand{\version}{draft}
\else
\providecommand{\version}{final}
\fi
%    \end{macrocode}

% Load the standard document class:
%    \begin{macrocode}
\documentclass[12pt]{article}
%    \end{macrocode}

% Start the document body:
%    \begin{macrocode}
\begin{document}
%    \end{macrocode}

% Declare a title page.
% Print title, part of document being processed and version flag:
%    \begin{macrocode}
\addtocounter{page}{-1}
\begin{center}
{\LARGE\bfseries{}childdoc example\par}
\vspace{1cm}
\ifchilddoc
\ifchilddocmanual part\else chapter\fi:
`\childdocname' of `\childdocjob'\par
\else
main document: `\childdocjob'\par
\fi
version: \version\par
\end{center}
\newpage
%    \end{macrocode}

% Manually include selected file,
% otherwise process as usual:
%    \begin{macrocode}
\ifchilddocmanual
\section*{part `\childdocname'}
\input{\childdocname}
\else
%    \end{macrocode}

% Include the two chapters:
%    \begin{macrocode}
\include{cdocsch1}
\include{cdocsch2}
%    \end{macrocode}

% Include the two parts unless only chapters should be displayed:
%    \begin{macrocode}
\ifchilddoc\else
\section{part three}
\input{cdocspt3}
\section{part four}
\input{cdocspt4}
\fi
%    \end{macrocode}

% Process as usual until here:
%    \begin{macrocode}
\fi
%    \end{macrocode}

% End of document body:
%    \begin{macrocode}
\end{document}
%    \end{macrocode}
%\iffalse
%</samplemain>
%\fi
%
% %%%%%%%%%%%%%%%%%%%%%%%%%%%%%%%%%%%%%%
% \paragraph{Chapter Include Files.}
%
% The include files are called |cdocsch1.tex| and |cdocsch2.tex|.
%
%\iffalse
%<*samplechap1|samplechap2>
%\fi

% Optional override for |\version| flag:
%    \begin{macrocode}
%%\providecommand{\version}{final}
%    \end{macrocode}

% Include the main document:
%    \begin{macrocode}
\input{childdoc.def}
\childdocof{cdocsamp}
%    \end{macrocode}

%\iffalse
%</samplechap1|samplechap2>
%\fi
%
%\iffalse
%<*samplechap1>
%\fi
% Some text for chapter 1:
%    \begin{macrocode}
\section{one}
some text in chapter one
%    \end{macrocode}

%\iffalse
%</samplechap1>
%\fi
% Some text for chapter 2:
%\iffalse
%<*samplechap2>
%\fi
%    \begin{macrocode}
\section{two}
more text in chapter two
%    \end{macrocode}

%\iffalse
%</samplechap2>
%\fi
%
% %%%%%%%%%%%%%%%%%%%%%%%%%%%%%%%%%%%%%%
% \paragraph{Part Include Files.}
%
% The include files are called |cdocspt3.tex| and |cdocspt4.tex|.
%
%\iffalse
%<*samplepart3|samplepart4>
%\fi

% Optional override for |\version| flag:
%    \begin{macrocode}
%%\providecommand{\version}{final}
%    \end{macrocode}

% Include the main document:
%    \begin{macrocode}
\input{childdoc.def}
\childdocby{cdocsamp}
%    \end{macrocode}

%\iffalse
%</samplepart3|samplepart4>
%\fi
%
%\iffalse
%<*samplepart3>
%\fi
% Some text for part 3:
%    \begin{macrocode}
some text in part three
%    \end{macrocode}

%\iffalse
%</samplepart3>
%\fi
% Some text for part 4:
%\iffalse
%<*samplepart4>
%\fi
%    \begin{macrocode}
more text in part four
%    \end{macrocode}

%\iffalse
%</samplepart4>
%\fi
%
% %%%%%%%%%%%%%%%%%%%%%%%%%%%%%%%%%%%%%%
% \paragraph{Forwarding for a Complete Draft.}
%
% The following forwarding file |cdocsdrf.tex|
% compiles the main document in draft mode:
%\iffalse
%<*sampledraft>
%\fi
%    \begin{macrocode}
\def\version{draft}
\input{childdoc.def}
\childdocforward{cdocsamp}
%    \end{macrocode}

%\iffalse
%</sampledraft>
%\fi
%
% %%%%%%%%%%%%%%%%%%%%%%%%%%%%%%%%%%%%%%
% \paragraph{Forwarding for Final Version of the Chapters.}
%
% The following forwarding files |cdocsfn1.tex| and |cdocsfn2.tex|
% (with identical content)
% compile the final versions of the child documents
% |cdocsch1.tex| and |cdocsch2.tex|, respectively:
%\iffalse
%<*samplefinal>
%\fi
%    \begin{macrocode}
\def\version{final}
\input{childdoc.def}
\childdocforwardprefix[cdocsamp]{cdocsfn}{cdocsch}
%    \end{macrocode}

%\iffalse
%</samplefinal>
%\fi
%
% %%%%%%%%%%%%%%%%%%%%%%%%%%%%%%%%%%%%%%
% \paragraph{Command Line Processing.}
%
% The following three command lines generate the output files
% |cdocscld|, |cdocscl1| and |cdocscl2|
% which should be identical to
% |cdocsdrf|, |cdocsch1| and |cdocsfn2|, respectively:
% \begin{center}
% \begin{tabular}{l}
% |latex -jobname cdocscld \|\\
% |  "\def\version{draft}\input{childdoc.def}\childdocforward{cdocsamp}"|\\
% |latex -jobname cdocscl1 \|\\
% |  "\input{childdoc.def}\childdocforward[cdocsamp]{cdocsch1}"|\\
% |latex -jobname cdocscl2 \|\\
% |  "\def\version{final}\input{childdoc.def}\childdocforward{cdocsch2}"|
% \end{tabular}
% \end{center}
% Note that the trailing backslash on each first line
% merely continues the input to the second line
% (for convenient cut ant paste).
% Furthermore, the command |latex| can be replaced by any
% of its alternative versions such as |pdflatex|.
%
% %%%%%%%%%%%%%%%%%%%%%%%%%%%%%%%%%%%%%%%%%%%%%%%%%%%%%%%%%%%%%%%%%%%%%%%%%%%%%%
% %%%%%%%%%%%%%%%%%%%%%%%%%%%%%%%%%%%%%%%%%%%%%%%%%%%%%%%%%%%%%%%%%%%%%%%%%%%%%%
% \section{Implementation}
%\iffalse
%<*package>
%\fi
%
% This section describes the definitions file |childdoc.def|.

% The definitions cannot be loaded using |\usepackage| or |\RequirePackage|
% which has a mechanism to prevent loading a style file more than once.
% When loading the definitions by means of |\input|
% multiple instances have to be prevented manually:
%\iffalse
%This code needs to be before the `\ProvidesFile' directive
%which is defined at the beginning of this file.
%Therefore it is also placed there and commented out here.
%</package>
%<*discard>
%\fi
%    \begin{macrocode}
\ifdefined\childdocmain\endinput\fi
%    \end{macrocode}
%\iffalse
%</discard>
%<*package>
%\fi
%
% \macro{\ifchilddoc}
% \macro{\ifchilddocmanual}
% The conditional |\ifchilddoc| tells whether a
% child (true) or main (false) document is being compiled.
% The conditional |\ifchilddocmanual| tells whether
% the |\includeonly| mechanism is used (false) or
% the selection of child files must be performed manually (true).
% The definitions initialise to false:
%    \begin{macrocode}
\newif\ifchilddoc
\newif\ifchilddocmanual
%    \end{macrocode}

% \macro{\childdocname}
% \macro{\childdocjob}
% The macro |\childdocname| stores the name of the main document
% to be compiled. The macro |\childdocjob| stores the name of
% the document on which the \LaTeX{} compiler was originally invoked.
% The content of |\jobname| cannot be compared
% to filenames specified in the source due to different catcodes.
% The following code rescans |\jobname|, stores the result
% in |\childdocname| and saves a copy in |\childdocjob|:
%    \begin{macrocode}
\edef\childdocname{\scantokens\expandafter{\jobname\noexpand}}
\let\childdocjob\childdocname
%    \end{macrocode}

% \macro{\childdocdisable}
% The macro |\childdocdisable| prevents the main file
% from being processed more than once.
% At this stage, the main document command |\childdocmain|
% is assumed to be called once again where it should do nothing.
% Any subsequent call to it should prevent
% a secondary processing of the main document
% It overwrites the forwarding commands
% |\childdocof| and |\childdocforward|
% with empty macros to prevent further inclusions of the main document:
%    \begin{macrocode}
\newcommand{\childdocdisable}
{
  \renewcommand{\childdocmain}[1]{\renewcommand{\childdocmain}[1]{\endinput}}
  \renewcommand{\childdocof}[1]{}
  \renewcommand{\childdocby}[2][]{}
  \renewcommand{\childdocforward}[2][]{}
  \renewcommand{\childdocdisable}{}
}
%    \end{macrocode}

% \macro{\childdocmain}
% The macro |\childdocmain| is to be called at the top of the main file
% with nothing or the main filename (without extension) as argument.
% First, it breaks loops.
% If the argument is not empty and does not match |\childdocname|
% (which is set by the first inclusion of |childdoc.def|),
% |\ifchilddoc| is set to true, |\includeonly| is applied to the child file
% and |\jobname| is set to the main file
% (for proper handling of |.aux| files):
%    \begin{macrocode}
\newcommand{\childdocmain}[1]
{
  \childdocdisable\childdocmain{}
  \if?#1?\else
    \begingroup
      \def\childdoctmp{#1}
      \ifx\childdoctmp\childdocname
        \def\childdoctmp{}
      \else
        \def\childdoctmp
        {
          \childdoctrue
          \includeonly{\childdocname}
          \def\childdocjob{#1}
          \def\jobname{#1}
        }
      \fi
      \expandafter
    \endgroup
    \childdoctmp
  \fi
}
%    \end{macrocode}

% \macro{\childdocof}
% The command |\childdocof| redirects
% compilation to the main file |#1|.
%    \begin{macrocode}
\newcommand{\childdocof}[1]
{
  \childdocdisable
  \childdoctrue
  \includeonly{\childdocname}
  \def\jobname{#1}
  \def\childdocjob{#1}
  \input{#1}
}
%    \end{macrocode}

% \macro{\childdocby}
% The command |\childdocby| ....
%    \begin{macrocode}
\newcommand{\childdocby}[2][]
{
  \childdocdisable
  \childdoctrue
  \childdocmanualtrue
  \if?#1?\else
    \def\jobname{#2}
  \fi
  \def\childdocjob{#2}
  \input{#2}
  \endinput
}
%    \end{macrocode}

% \macro{\childdocforward}
% The command |\childdocforward| redirects
% compilation to the main file or
% (if the optional argument is given) a child file.
% Parameters are set as if the main file
% or a child file starting with |\childdocof| was compiled.
% Then compilation is handed over to the main file:
%    \begin{macrocode}
\newcommand{\childdocforward}[2][]
{
  \begingroup
    \if?#1?
      \def\childdoctmp
      {
        \def\childdocname{#2}
        \def\childdocjob{#2}
        \def\jobname{#2}
        \input{#2}
        \endinput
      }
    \else
      \def\childdoctmp
      {
        \childdocdisable
        \def\childdocname{#2}
        \childdoctrue
        \includeonly{#2}
        \def\childdocjob{#1}
        \def\jobname{#1}
        \input{#1}
        \endinput
      }
    \fi
    \expandafter
  \endgroup
  \childdoctmp
}
%    \end{macrocode}

% \macro{\childdocforwardprefix}
% The command |\childdocforwardprefix| redirects
% compilation to the main or a child file by means of a pattern.
% The prefix |#1| in the current filename is replaced by |#2|
% and the suffix of the current filename is kept
% (it is assumed that the filename does not contain the substring `|~~~|'
% which is used as a delimiter).
% Compilation is handed over to the new file by |\childdocforward|:
%    \begin{macrocode}
\newcommand{\childdocforwardprefix}[3][]
{
  \begingroup
    \def\childdocextract #2##1~~~{\def\childdoctmp{\childdocforward[#1]{#3##1}}}
    \expandafter\childdocextract\childdocname~~~
    \expandafter
  \endgroup
  \childdoctmp
}
%    \end{macrocode}

% \macro{\childdoc}
% The deprecated macro |\childdoc| is a legacy version of |\childdocmain|:
%    \begin{macrocode}
\newcommand{\childdoc}{\childdocmain}
%    \end{macrocode}

% \macro{\childdocredirect}
% The deprecated macro |\childdocredirect| is a legacy version
% of |\childdocforward| and |\childdocforwardprefix|:
%    \begin{macrocode}
\newcommand{\childdocredirect}[2][]
{
  \begingroup
    \if?#1?
      \def\childdoctmp{\childdocforward{#2}}
    \else
      \def\childdoctmp{\childdocforwardprefix{#1}{#2}}
    \fi
    \expandafter
  \endgroup
  \childdoctmp
}
%    \end{macrocode}

%\iffalse
%</package>
%\fi
%
\endinput

\childdocforward{cdocsamp}
%    \end{macrocode}

%\iffalse
%</sampledraft>
%\fi
%
% %%%%%%%%%%%%%%%%%%%%%%%%%%%%%%%%%%%%%%
% \paragraph{Forwarding for Final Version of the Chapters.}
%
% The following forwarding files |cdocsfn1.tex| and |cdocsfn2.tex|
% (with identical content)
% compile the final versions of the child documents
% |cdocsch1.tex| and |cdocsch2.tex|, respectively:
%\iffalse
%<*samplefinal>
%\fi
%    \begin{macrocode}
\def\version{final}
% \iffalse
%
% childdoc.dtx Copyright (C) 2017-2018 Niklas Beisert
%
% This work may be distributed and/or modified under the
% conditions of the LaTeX Project Public License, either version 1.3
% of this license or (at your option) any later version.
% The latest version of this license is in
%   http://www.latex-project.org/lppl.txt
% and version 1.3 or later is part of all distributions of LaTeX
% version 2005/12/01 or later.
%
% This work has the LPPL maintenance status `maintained'.
%
% The Current Maintainer of this work is Niklas Beisert.
%
% This work consists of the files childdoc.dtx and childdoc.ins
% and the derived files childdoc.def and cdocsamp.tex with
% cdocsch1.tex, cdocsch2.tex, cdocsdrf.tex, cdocsfn1.tex, cdocsfn2.tex.
%
%<package>\ifdefined\childdocmain\endinput\fi
%<package>\ProvidesFile{childdoc.def}[2018/12/30 v2.0 child document driver]
%<samplemain>\ProvidesFile{cdocsamp.tex}[2018/12/30 v2.0 sample for childdoc]
%<*driver>
%\ProvidesFile{childdoc.drv}[2018/12/30 v2.0 childdoc reference manual file]
\PassOptionsToClass{10pt,a4paper}{article}
\documentclass{ltxdoc}

\usepackage[margin=35mm]{geometry}
\usepackage{hyperref}
\usepackage{hyperxmp}
\usepackage[usenames]{color}

\hypersetup{colorlinks=true}
\hypersetup{pdfstartview=FitH}
\hypersetup{pdfpagemode=UseNone}
\hypersetup{pdfsource={}}
\hypersetup{pdflang={en-UK}}
\hypersetup{pdfcopyright={Copyright 2017-2018 Niklas Beisert.
  This work may be distributed and/or modified under the
  conditions of the LaTeX Project Public License, either version 1.3
  of this license or (at your option) any later version.}}
\hypersetup{pdflicenseurl={http://www.latex-project.org/lppl.txt}}
\hypersetup{pdfcontactaddress={ETH Zurich, ITP, HIT K,
  Wolfgang-Pauli-Strasse 27}}
\hypersetup{pdfcontactpostcode={8093}}
\hypersetup{pdfcontactcity={Zurich}}
\hypersetup{pdfcontactcountry={Switzerland}}
\hypersetup{pdfcontactemail={nbeisert@itp.phys.ethz.ch}}
\hypersetup{pdfcontacturl={http://people.phys.ethz.ch/\xmptilde nbeisert/}}

\newcommand{\secref}[1]{\hyperref[#1]{section \ref*{#1}}}

\parskip1ex
\parindent0pt
\let\olditemize\itemize
\def\itemize{\olditemize\parskip0pt}

\begin{document}

\title{The \textsf{childdoc} Package}
\hypersetup{pdftitle={The childdoc Package}}
\author{Niklas Beisert\\[2ex]
  Institut f\"ur Theoretische Physik\\
  Eidgen\"ossische Technische Hochschule Z\"urich\\
  Wolfgang-Pauli-Strasse 27, 8093 Z\"urich, Switzerland\\[1ex]
  \href{mailto:nbeisert@itp.phys.ethz.ch}
  {\texttt{nbeisert@itp.phys.ethz.ch}}}
\hypersetup{pdfauthor={Niklas Beisert}}
\hypersetup{pdfsubject={Manual for the LaTeX2e Package childdoc}}
\date{30 December 2018, \textsf{v2.0}}
\maketitle

\begin{abstract}\noindent
\textsf{childdoc} is a \LaTeXe{} package
that enables the direct compilation
of document sections included by |\include|
to individual files.
\end{abstract}

\begingroup
\parskip0ex
\tableofcontents
\endgroup

%%%%%%%%%%%%%%%%%%%%%%%%%%%%%%%%%%%%%%%%%%%%%%%%%%%%%%%%%%%%%%%%%%%%%%%%%%%%%%%%
%%%%%%%%%%%%%%%%%%%%%%%%%%%%%%%%%%%%%%%%%%%%%%%%%%%%%%%%%%%%%%%%%%%%%%%%%%%%%%%%
\section{Introduction}

\LaTeX{} provides a mechanism to structure a large document (such as a book)
into a main file and several child files (containing the chapters)
using the |\include| command.
This mechanism is beneficial for documents
which span hundreds of pages in order to
make the source file(s) more manageable.
Moreover, compilation can be restricted to
selected child files by means of the |\includeonly| command.
The latter feature can be used to reduce the compilation time while editing
(this was significantly more useful in the earlier days of \LaTeX{})
or to generate a smaller document which is easier to navigate.
Another application of |\includeonly| is to generate
documents consisting of selected parts of the complete document.

However, there are a few drawbacks of the plain |\include| mechanism:
\begin{itemize}
\item
The child files cannot be compiled on their own,
they can only be compiled via the main file.
A naive editing environment
(such as a text editor with an option
to have the current file processed by \LaTeX)
may require one to switch to the main file before compiling;
attempting to compile the child file produces errors.
\item
The main file must be modified (each time)
to adjust the |\includeonly| command
to the present needs. This easily leaves the main file in a messy state.
\item
The generated document will always carry the filename
of the main document. This is inconvenient if
several child files are to be compiled and
to be kept for distribution.
\end{itemize}

The present package provides a simple interface
to make child files individually compilable by \LaTeX{}.
Compiling a child file then has the same effect as compiling
the main file with an |\includeonly| command
to select the appropriate child.
Moreover the generated document will carry the name of the child
rather than the main file.
This resolves all three above issues.

This feature is meant to make the editing of books,
thesis documents and lecture notes somewhat more convenient.
However, the package can also be used efficiently for
composing a series of documents (such as exercise sheets)
which are typically distributed individually.
It then assists the author in generating the individual documents
(potentially in different versions)
as well as a document containing the collected series.
Another application is in developing style files
or other kinds of included material
where compilation of the style file could redirect
to a sample or test file.

%%%%%%%%%%%%%%%%%%%%%%%%%%%%%%%%%%%%%%%%%%%%%%%%%%%%%%%%%%%%%%%%%%%%%%%%%%%%%%%%
%%%%%%%%%%%%%%%%%%%%%%%%%%%%%%%%%%%%%%%%%%%%%%%%%%%%%%%%%%%%%%%%%%%%%%%%%%%%%%%%
\section{Usage}

First of all, the package \textsf{childdoc} is \emph{not} a standard
\LaTeXe{} |.sty| style file! Therefore it needs to be invoked in
a non-standard way.

%%%%%%%%%%%%%%%%%%%%%%%%%%%%%%%%%%%%%%%%%%%%%%%%%%%%%%%%%%%%%%%%%%%%%%%%%%%%%%%%
\subsection{Included Files}
\label{sec:include}

%%%%%%%%%%%%%%%%%%%%%%%%%%%%%%%%%%%%%%%%
\DescribeMacro{\childdocmain}
To use the package, add the commands
\begin{center}
\begin{tabular}{l}
|\input{childdoc.def}|\\
|\childdocmain{}|\\
\end{tabular}
\end{center}
at the very top of the main \LaTeX{} file,
in particular \emph{before} the |\documentclass| statement!
The argument of |\childdocmain| should be left empty
(but it must be present).

%%%%%%%%%%%%%%%%%%%%%%%%%%%%%%%%%%%%%%%%
\DescribeMacro{\childdocof}
Furthermore, add the commands
\begin{center}
\begin{tabular}{l}
|\input{childdoc.def}|\\
|\childdocof{|\textit{main}|}|\\
\end{tabular}
\end{center}
at the top of every child file \textit{child}
which is included by |\include{|\textit{child}|}|
from within the main file
(or at least for those files to be compiled individually).
The argument \textit{main} must be the filename of the main file.

There are a couple of
considerations in setting up the main and child documents:

%%%%%%%%%%%%%%%%%%%%%%%%%%%%%%%%%%%%%%%%
\paragraph{Restrictions.}

Please note the following restrictions:
\begin{itemize}
\item
|\childdocmain| must be called with one argument \textit{main}
to ensure compatibility with earlier version of the package.
It must either be empty (|\childdocmain{}|)
or precisely match the filename of the main file in which it is specified.
See \secref{sec:detection} for further information.
\item
The filename \textit{main} must be specified without the |.tex| extension.
\item
The filename \textit{main} is case sensitive
(even in case-insensitive file systems)
due to internal string comparison.
\item
The argument \textit{main} should be fully expanded, it cannot be a macro.
\item
Subdirectories and special characters should be avoided in filenames.
\item
The command |\childdocmain{|\textit{main}|}| must be followed by a whitespace.
It should not be followed immediately by another command
or by a comment mark `|%|'.
This is because the \TeX{} parser reads the token immediately following
the argument of |\childdocmain| and puts it
at the beginning of every child section;
however, a white\-space is ignored.
\end{itemize}

%%%%%%%%%%%%%%%%%%%%%%%%%%%%%%%%%%%%%%%%
\paragraph{Content of Main File.}

It is advisable to place all content in the child files included by |\include|.
Any output contained in the main file will appear in all child documents
unless suppressed manually;
it cannot be suppressed automatically by the |\includeonly| directive
and thus should normally be avoided.
A method to include some content in the main file
by means of conditional processing is described in \secref{sec:conditional}.

%%%%%%%%%%%%%%%%%%%%%%%%%%%%%%%%%%%%%%%%
\paragraph{Page Numbering.}

When only a part of the document is compiled,
the appropriate numbering of pages
(as well as other status parameters)
is determined from the |.aux| files.
The latter contain information from previous passes.
However this information needs to propagate through
all intermediate child documents.
Therefore the page numbering in child documents may well
be inconsistent until the complete document is compiled at least once.

A useful (if unconventional) way to always ensure a consistent
page numbering is to restart the numbering in each child document
and denote the pages by `\textit{child}|.|\textit{page}'
where \textit{child} represents the chapter/section number of the child file.
This can be achieved by the command
|\numberwithin{page}{|\textit{child}|}|
of the \textsf{amsmath} package
where \textit{child} can be |chapter| or |section|
depending on the chosen structuring.
Alternatively, one can modify the macro |\thepage| appropriately
and reset the counter |page| at the start of each child file.

%%%%%%%%%%%%%%%%%%%%%%%%%%%%%%%%%%%%%%%%%%%%%%%%%%%%%%%%%%%%%%%%%%%%%%%%%%%%%%%%
\subsection{Conditional Processing}
\label{sec:conditional}

The package provides a mechanism to compile different versions
of a document. To customise the versions further some conditional processing
can come in handy to distinguish which version is being compiled.
The package provides two macros to describe the compilation context:

%%%%%%%%%%%%%%%%%%%%%%%%%%%%%%%%%%%%%%%%
\DescribeMacro{\ifchilddoc}
The conditional |\ifchilddoc| distinguishes between the compilation of
child documents and the main document:
%
\begin{center}
|\ifchilddoc |\textit{child-code}| |[|\||else |\textit{main-code}]| \||fi|
\end{center}

%%%%%%%%%%%%%%%%%%%%%%%%%%%%%%%%%%%%%%%%
\DescribeMacro{\childdocname}
\DescribeMacro{\childdocjob}
The macro |\childdocname| contains the filename (without extension)
of the main or child file being processed.
Note that |\childdocjob| will always contain the name of the main file.

%%%%%%%%%%%%%%%%%%%%%%%%%%%%%%%%%%%%%%%%
\paragraph{Title Page.}

Conditional processing can be used to include a title or banner page
in the main document when proper precautions are taken.
Importantly, the code in the main file should ensure that the page counter
(as well as other status parameters which are stored in the |.aux| files)
takes the same value after the conditional processing.
Otherwise the page numbers may take divergent values
depending on which part is compiled.

For example, a title page could be declared by:
%
\begin{center}
\begin{tabular}{l}
|\ifchilddoc\||else|\\
|\addtocounter{page}{-1}|\\
\textit{code for title page}\\
|\newpage|\\
|\||fi|
\end{tabular}
\end{center}
%
A banner page for the child documents can be generated by:
%
\begin{center}
\begin{tabular}{l}
|\ifchilddoc|\\
|\addtocounter{page}{-1}|\\
\textit{code for banner page}\\
|\newpage|\\
|\||fi|
\end{tabular}
\end{center}
%
Here one could write a message such as:
\begin{center}
|This is the part \childdocname{} of \childdocjob{}.|
\end{center}

%%%%%%%%%%%%%%%%%%%%%%%%%%%%%%%%%%%%%%%%%%%%%%%%%%%%%%%%%%%%%%%%%%%%%%%%%%%%%%%%
\subsection{Flags}
\label{sec:flags}

The package makes it easy to generate different versions
of the main or child documents.
To this end compilation flags can be defined
and assigned different default values.
They will be particularly useful in conjunction
with the forwarding mechanism described in \secref{sec:forward}.

For example, it may be useful to have a flag |\version|
which can be set to |draft| or |final|.
The document source will contain some conditional code
depending on the value of |\version|.
Suppose further, the flag should default to |final| for the main file
and to |draft| for child files
which is a natural assignment for editing the document.
This is achieved by placing the following code
in the preamble of the main document
(below the |\childdocmain| directive):
%
\begin{center}
\begin{tabular}{l}
|\ifchilddoc|\\
|\providecommand{\version}{draft}|\\
|\||else|\\
|\providecommand{\version}{final}|\\
|\||fi|
\end{tabular}
\end{center}
%
The definition by |\providecommand| makes sure
that previous definitions are not overwritten.
Further statements |\providecommand{\version}{...}|
can thus be added before the above code to override it.

For the main file, one might add a line
(between |\childdocmain| and the above block)
%
\begin{center}
|%\ifchilddoc\||else\providecommand{\version}{draft}\||fi|
\end{center}
%
which can be uncommented to produce a draft version.
Likewise one can add a line to the very top of a child file
(above the |\childdocof{|\textit{main}|}| directive)
%
\begin{center}
|%\providecommand{\version}{final}|
\end{center}
%
which can be uncommented to produce the final version of this child document.

%%%%%%%%%%%%%%%%%%%%%%%%%%%%%%%%%%%%%%%%%%%%%%%%%%%%%%%%%%%%%%%%%%%%%%%%%%%%%%%%
\subsection{Forwarding}
\label{sec:forward}

Different versions of the main or child documents
using compilation flags as described in \secref{sec:flags}
can be (permanently) stored in different files
for convenient compilation, viewing and distribution.
To this end, the package defines a command
to pass on compilation to a different file:

%%%%%%%%%%%%%%%%%%%%%%%%%%%%%%%%%%%%%%%%
\DescribeMacro{\childdocforward}
The command |\childdocforward| redirects processing to
another source file:
%
\begin{center}
\begin{tabular}{l}
|\input{childdoc.def}|\\
|\childdocforward[|\textit{main}|]{|\textit{dest}|}|\\
\end{tabular}
\end{center}
%
The argument \textit{dest} is the destination file
(without extension).
It should be the main file or one of the child files.
Note that further \textsf{childdoc} directives
such as |\childdocof| and |\childdocforward|
in the indicated file will be processed in this form.
The optional argument \textit{main}
passes on directly to the main file \textit{main}
while pretending to compile the child \textit{dest}.
This form behaves as if \textit{dest}
issues |\childdocof{|\textit{main}|}| right away,
and no further \textsf{childdoc} directives will be processed.

%%%%%%%%%%%%%%%%%%%%%%%%%%%%%%%%%%%%%%%%
\DescribeMacro{\...prefix}
In the alternative form |\childdocforwardprefix|,
%
\begin{center}
\begin{tabular}{l}
|\input{childdoc.def}|\\
|\childdocforwardprefix[|\textit{main}|]{|\textit{prefix}|}{|\textit{dest}|}|
\end{tabular}
\end{center}
%
the destination file is determined by a pattern
depending on the current file:
To make this work, the current file must be called
`{\textit{prefix}\hspace{0.2em}\textit{suffix}}'
with \textit{prefix} matching precisely the argument.
Processing is then passed on to the file
`{\textit{dest}\hspace{0.2em}\textit{suffix}}'.
Surely, the same effect is achieved by
directly specifying the
argument `{\textit{dest}\hspace{0.2em}\textit{suffix}}'
in the first form.
However, that requires to set up a different file
for each child. With the alternative form of the command
all these files can have exactly the same content
which simplifies setting them up and maintaining them.

For example, the following file |draft.tex|
with a compilation flag |\version| as described in \secref{sec:flags}
compiles the main document as a draft:
%
\begin{center}
\begin{tabular}{l}
|\def\version{draft}|\\
|\input{childdoc.def}|\\
|\childdocforward{|\textit{main}|}|
\end{tabular}
\end{center}
%
Likewise, the following files |final|\textit{nn}|.tex|
compile the final version of the child document
|child|\textit{nn}|.tex|:
%
\begin{center}
\begin{tabular}{l}
|\def\version{final}|\\
|\input{childdoc.def}|\\
|\childdocforwardprefix{final}{child}|
\end{tabular}
\end{center}
%

Note that when several versions of a main file and/or of each child file
are to be generated, it may be convenient to set up a |Makefile| or
shell script to automatise the process.

%%%%%%%%%%%%%%%%%%%%%%%%%%%%%%%%%%%%%%%%%%%%%%%%%%%%%%%%%%%%%%%%%%%%%%%%%%%%%%%%
\subsection{Command Line Processing}
\label{sec:commandline}

The effect of redirection files can also be achieved by invoking
the \LaTeX{} compiler with a more elaborate command line.
Most conveniently this should be done as part
of a shell script or a |Makefile|.

When using \textsf{childdoc} in the main file, the following
command lines effectively perform a redirection
(note that depending on the shell being used,
backslashes may have to be doubled: `|\|' $\to$ `|\\|'):
%
\begin{center}
|... -jobname "|\textit{target}|" |\\|"|[\textit{flags}]%
|\input{childdoc.def}\childdocforward[|\textit{main}|]{|\textit{dest}|}"|
\end{center}
%
Here \textit{target} is the name of the output file,
\textit{main} is the name of the main file
and \textit{dest} is the name of the main or child file to be processed
(all filenames without extensions).
The optional argument \textit{main} can be omitted
if \textit{main} matches \textit{dest}.
Optionally, compilation \textit{flags} can be defined via |\def| commands.
This command line makes the \TeX{} engine believe
it is compiling the file \textit{target}
whose content is specified as the latter parameter.
The provided code then forwards the processing to
\textit{main} or \textit{dest} as described in \secref{sec:forward}.

%%%%%%%%%%%%%%%%%%%%%%%%%%%%%%%%%%%%%%%%%%%%%%%%%%%%%%%%%%%%%%%%%%%%%%%%%%%%%%%%
\subsection{Include by Input}
\label{sec:input}

Including child documents by |\include| has some restrictions by design.
Most notably, the content of a child document always occupies
its own set of pages; pages cannot be shared between child documents.
Usually, this behaviour makes perfect sense
because each child document contain an essential part of the document.
However, in some situations it may be desirable to compose
a document from a collection of parts
without having mandatory page breaks between then.
For this case, the package
provides a mechanism to include parts
by |\input| which can also be processed individually.
However, by construction this mechanism
requires manual handling of the content to be output.

%%%%%%%%%%%%%%%%%%%%%%%%%%%%%%%%%%%%%%%%
\DescribeMacro{\ifchilddocmanual}
The main file should be prepared as usual, see \secref{sec:include}.
However, the document body must make a distinction
between processing of an individual part and of the main document, e.g.:
%
\begin{center}
\begin{tabular}{l}
|\ifchilddocmanual|\\
|\input{\childdocname}|\\
|\||else|\\
\textit{document body with }|\input{|\textit{part}|}|\\
|\||fi|
\end{tabular}
\end{center}
%
The conditional |\ifchilddocmanual| is true whenever
a part to be included by |\input| is being compiled,
and the name of the part is stored in |\childdocname|.

%%%%%%%%%%%%%%%%%%%%%%%%%%%%%%%%%%%%%%%%
\DescribeMacro{\childdocby}
Each part to be included by |\input| should start with:
%
\begin{center}
\begin{tabular}{l}
|\input{childdoc.def}|\\
|\childdocby{|\textit{main}|}|\\
\end{tabular}
\end{center}
%
The directive |\childdocby| is similar to |\childdocof|
described in \secref{sec:include},
but the subsequent selection of content must be done manually.
To that end, both |\ifchilddoc| and |\ifchilddocmanual|
will be true upon processing of a part,
and the name of the part is stored in |\childdocname|.
Note that |\jobname| will be set to the filename of the current part
so that each part receives an individual |.aux| file
that does not interfere with the |.aux| file(s) of the main document.
This behaviour can be altered by the alternative form
|\childdocby[*]{|\textit{main}|}| (with a non-empty optional argument)
which uses the |.aux| file of the main document
by setting |\jobname| to \textit{main}.

%%%%%%%%%%%%%%%%%%%%%%%%%%%%%%%%%%%%%%%%%%%%%%%%%%%%%%%%%%%%%%%%%%%%%%%%%%%%%%%%
\subsection{Driver Development}
\label{sec:driver}

The \textsf{childdoc} mechanism can also be use for the development
of definition files such as \LaTeX{} styles or classes.
This case differs from the above setup with multiple parts
included by |\include| in that no |\includeonly| should be invoked.
This can be achieved by starting the include file
(before |\ProvidesPackage|) with:
%
\begin{center}
\begin{tabular}{l}
|\input{childdoc.def}|\\
|\childdocforward{|\textit{main}|}|\\
\end{tabular}
\end{center}
%
or alternatively with:
%
\begin{center}
\begin{tabular}{l}
|\input{childdoc.def}|\\
|\childdocby{|\textit{main}|}|\\
\end{tabular}
\end{center}
%
Both forms have slightly different effects as described above.
The main file is prepared as usual, see \secref{sec:include}.

%%%%%%%%%%%%%%%%%%%%%%%%%%%%%%%%%%%%%%%%%%%%%%%%%%%%%%%%%%%%%%%%%%%%%%%%%%%%%%%%
\subsection{Legacy Detection}
\label{sec:detection}

The directive |\childdocmain| in the main file can detect
whether the complete document or merely a child is to be compiled
even without using the directive |\childdocof|.
This method is deprecated because it is less robust
and there is no compelling reason to use it;
it is merely provided for backward compatibility
and it may be removed in future versions.

If the detection mechanism is to be used,
it is mandatory to correctly specify
the filename of the main file as the argument of |\childdocmain|:
%
\begin{center}
\begin{tabular}{l}
|\input{childdoc.def}|\\
|\childdocmain{|\textit{main}|}|\\
\end{tabular}
\end{center}
%
If |\jobname| does not match the argument \textit{main} of |\childdocmain|,
it is assumed that |\jobname| points to the child file to be compiled.
When using |\childdocmain| with the main file specified as argument,
it suffices to start a child file
with just |\input{|\textit{main}|}|
without loading of the package and using |\childdocof|.
If instead all processing is done
with the appropriate \textsf{childdoc} directives,
the argument of \textit{main} of |\childdocmain| can be empty.

An alternative version of the command line processing described
in \secref{sec:commandline} using the detection mechanism reads:
%
\begin{center}
|... -jobname "|\textit{target}|" "|[\textit{flags}]%
[|\def\jobname{|\textit{dest}|}|]|\input{|\textit{main}|}"|
\end{center}

%%%%%%%%%%%%%%%%%%%%%%%%%%%%%%%%%%%%%%%%%%%%%%%%%%%%%%%%%%%%%%%%%%%%%%%%%%%%%%%%
\subsection{Manual Code}
\label{sec:manual}

In case one cannot be certain whether the definitions file |childdoc.def|
is installed on the target \TeX{} distribution
and one prefers not to ship it,
it is conceivable to paste a few relevant commands into the sources.

To that end, drop all statements |\input{childdoc.def}|
and perform the replacements as outlined below.
Instead of |\childdocmain{|\textit{main}|}| add the following code
to the top of the main file:
%
\begin{center}
\begin{tabular}{l}
|\||ifdefined\childdocname\endinput\||fi\newif\ifchilddoc|\\
|\edef\childdocname{\scantokens\expandafter{\jobname\noexpand}}|\\
|\def\childdocmain{|\textit{main}|}\||ifx\childdocmain\childdocname\||else|\\
|\childdoctrue\includeonly{\childdocname}\let\jobname\childdocmain\||fi|\\
\end{tabular}
\end{center}
%
Instead of |\childdocof{|\textit{main}|}| just include the main file
at the top of each child file:
%
\begin{center}
|\input{|\textit{main}|}|
\end{center}
%
A simple redirection |\childdocforward{|\textit{dest}|}| is achieved by:
%
\begin{center}
|\def\jobname{|\textit{dest}|}\input{\jobname}|
\end{center}
%
The redirection with prefix
|\childdocforwardprefix[|\textit{prefix}|]{|\textit{dest}|}|
is accomplished by:
%
\begin{center}
\begin{tabular}{l}
|{\edef\jobname{\scantokens\expandafter{\jobname\noexpand}}|\\
|\def\redirectjob |\textit{prefix}|#1~~~{\gdef\jobname{|\textit{dest}|#1}}|\\
|\expandafter\redirectjob\jobname~~~}\input{\jobname}|
\end{tabular}
\end{center}

In an alternative approach,
child documents can be compiled by a specific command line
without additional code or specific definitions:
%
\begin{center}
|... -jobname "|\textit{target}|" "|[\textit{flags}]%
|\includeonly{|\textit{dest}|}\input{|\textit{main}|}"|
\end{center}
%

%%%%%%%%%%%%%%%%%%%%%%%%%%%%%%%%%%%%%%%%%%%%%%%%%%%%%%%%%%%%%%%%%%%%%%%%%%%%%%%%
%%%%%%%%%%%%%%%%%%%%%%%%%%%%%%%%%%%%%%%%%%%%%%%%%%%%%%%%%%%%%%%%%%%%%%%%%%%%%%%%
\section{Information}

%%%%%%%%%%%%%%%%%%%%%%%%%%%%%%%%%%%%%%%%%%%%%%%%%%%%%%%%%%%%%%%%%%%%%%%%%%%%%%%%
\subsection{Copyright}

Copyright \copyright{} 2017--2018 Niklas Beisert

This work may be distributed and/or modified under the
conditions of the \LaTeX{} Project Public License, either version 1.3
of this license or (at your option) any later version.
The latest version of this license is in
  \url{http://www.latex-project.org/lppl.txt}
and version 1.3 or later is part of all distributions of \LaTeX{}
version 2005/12/01 or later.

This work has the LPPL maintenance status `maintained'.

The Current Maintainer of this work is Niklas Beisert.

This work consists of the files |README.txt|, |childdoc.ins| and |childdoc.dtx|
as well as the derived files |childdoc.def|, |cdocsamp.tex|
with |cdocsch1.tex|, |cdocsch2.tex|, |cdocspt3.tex|, |cdocspt4.tex|,
|cdocsdrf.tex|, |cdocsfn1.tex|, |cdocsfn2.tex|
as well as |childdoc.pdf|.

%%%%%%%%%%%%%%%%%%%%%%%%%%%%%%%%%%%%%%%%%%%%%%%%%%%%%%%%%%%%%%%%%%%%%%%%%%%%%%%%
\subsection{Files and Installation}

The package consists of the files:
%
\begin{center}
\begin{tabular}{ll}
    |README.txt|   & readme file \\
    |childdoc.ins| & installation file \\
    |childdoc.dtx| & source file \\
    |childdoc.def| & definition file \\
    |cdocsamp.tex| & sample main file \\
    |cdocsch1.tex| & sample include file \\
    |cdocsch2.tex| & sample include file \\
    |cdocspt3.tex| & sample part file \\
    |cdocspt4.tex| & sample part file \\
    |cdocsdrf.tex| & sample redirection file \\
    |cdocsfn1.tex| & sample redirection file \\
    |cdocsfn2.tex| & sample redirection file \\
    |childdoc.pdf| & manual
\end{tabular}
\end{center}
%
The distribution consists of the files
|README.txt|, |childdoc.ins| and |childdoc.dtx|.
%
\begin{itemize}
\item
Run (pdf)\LaTeX{} on |childdoc.dtx|
to compile the manual |childdoc.pdf| (this file).
\item
Run \LaTeX{} on |childdoc.ins| to create the definitions file |childdoc.def|
and the sample |cdocsamp.tex| with include files
|cdocsch1.tex|, |cdocsch2.tex|, |cdocspt3.tex|, |cdocspt4.tex|,
|cdocsdrf.tex|, |cdocsfn1.tex|, |cdocsfn2.tex|.
Then copy the file |childdoc.def| to an appropriate directory of your \LaTeX{}
distribution, e.g.\ \textit{texmf-root}|/tex/latex/childdoc|.
\end{itemize}

%%%%%%%%%%%%%%%%%%%%%%%%%%%%%%%%%%%%%%%%%%%%%%%%%%%%%%%%%%%%%%%%%%%%%%%%%%%%%%%%
\subsection{Related CTAN Packages}

There are several other packages which offer a similar functionality:
%
\begin{itemize}
\item
The packages
\href{http://ctan.org/pkg/docmute}{\textsf{docmute}},
\href{http://ctan.org/pkg/includex}{\textsf{includex}} and
\href{http://ctan.org/pkg/standalone}{\textsf{standalone}}
provide commands to include only the document body of
a child file thus allowing both files to be compiled individually.
\item
The packages \href{http://ctan.org/pkg/subdocs}{\textsf{subdocs}}
and \href{http://ctan.org/pkg/subfiles}{\textsf{subfiles}}
provide structures in which the main and child documents can be
encapsulated and allowing them to be compiled individually.
The inclusion mechanism is different from the conventional |\include|.
\item
The package \href{http://ctan.org/pkg/combine}{\textsf{combine}}
is an elaborate solution to combine several documents into one.
\end{itemize}
%
See also the CTAN topic \href{http://ctan.org/topic/subdocs}{\textsf{subdocs}}
for further related packages.
The present package differs from the above solutions in that
a document structure constructed with the conventional |\include| mechanism
just needs two extra commands at the top of every file
such that all constituent files can be compiled individually.

%%%%%%%%%%%%%%%%%%%%%%%%%%%%%%%%%%%%%%%%%%%%%%%%%%%%%%%%%%%%%%%%%%%%%%%%%%%%%%%%
%\subsection{Feature Suggestions}
%
%The following is a list of features which may be useful for future
%versions of this package:
%%
%\begin{itemize}
%\item
%\ldots
%\end{itemize}

%%%%%%%%%%%%%%%%%%%%%%%%%%%%%%%%%%%%%%%%%%%%%%%%%%%%%%%%%%%%%%%%%%%%%%%%%%%%%%%%
\subsection{Revision History}

%%%%%%%%%%%%%%%%%%%%%%%%%%%%%%%%%%%%%%%%
\paragraph{v2.0:} 2018/12/30

\begin{itemize}
\item
immediate forward processing
\item
added |\childdocby| mechanism
\item
manual restructured
\end{itemize}

%%%%%%%%%%%%%%%%%%%%%%%%%%%%%%%%%%%%%%%%
\paragraph{v1.6:} 2018/01/17

\begin{itemize}
\item
application for development of include files
\item
corrections to manual
\end{itemize}

%%%%%%%%%%%%%%%%%%%%%%%%%%%%%%%%%%%%%%%%
\paragraph{v1.5:} 2017/05/21

\begin{itemize}
\item
more complete structuring introduced
\item
|\childdocof| introduced
\item
|\childdoc| renamed to |\childdocmain|
\item
|\childredirect| renamed to |\childdocforward| and |\childdocforwardprefix|
and functionality expanded
\end{itemize}

%%%%%%%%%%%%%%%%%%%%%%%%%%%%%%%%%%%%%%%%
\paragraph{v1.0:} 2017/04/27

\begin{itemize}
\item
manual and install package
\item
first version published on CTAN
\end{itemize}

%%%%%%%%%%%%%%%%%%%%%%%%%%%%%%%%%%%%%%%%
\paragraph{v0.6:} 2017/04/26

\begin{itemize}
\item
redirection mechanism added
\end{itemize}

%%%%%%%%%%%%%%%%%%%%%%%%%%%%%%%%%%%%%%%%
\paragraph{v0.5:} 2017/04/26

\begin{itemize}
\item
functionality in definition file
\end{itemize}


%%%%%%%%%%%%%%%%%%%%%%%%%%%%%%%%%%%%%%%%%%%%%%%%%%%%%%%%%%%%%%%%%%%%%%%%%%%%%%%%
%%%%%%%%%%%%%%%%%%%%%%%%%%%%%%%%%%%%%%%%%%%%%%%%%%%%%%%%%%%%%%%%%%%%%%%%%%%%%%%%
%%%%%%%%%%%%%%%%%%%%%%%%%%%%%%%%%%%%%%%%%%%%%%%%%%%%%%%%%%%%%%%%%%%%%%%%%%%%%%%%
\appendix

\settowidth\MacroIndent{\rmfamily\scriptsize 000\ }

 \DocInput{childdoc.dtx}

\end{document}
%</driver>
% \fi
%
% %%%%%%%%%%%%%%%%%%%%%%%%%%%%%%%%%%%%%%%%%%%%%%%%%%%%%%%%%%%%%%%%%%%%%%%%%%%%%%
% %%%%%%%%%%%%%%%%%%%%%%%%%%%%%%%%%%%%%%%%%%%%%%%%%%%%%%%%%%%%%%%%%%%%%%%%%%%%%%
% \section{Sample}
%\iffalse
%<*samplemain>
%\fi
%
% The following presents a sample document
% with two chapters, two parts, a title page,
% a compile flag as well as three forwarding files to set the flag.
% It consists of eight |.tex| files:
% \begin{center}
% \begin{tabular}{ll}
% |cdocsamp.tex|&main file\\
% |cdocsch1.tex|&include file for chapter 1\\
% |cdocsch2.tex|&include file for chapter 2\\
% |cdocspt3.tex|&include file for part 3\\
% |cdocspt4.tex|&include file for part 4\\
% |cdocsdrf.tex|&forwarding file for main file in draft mode\\
% |cdocsfi1.tex|&forwarding file for final version of chapter 1\\
% |cdocsfi2.tex|&forwarding file for final version of chapter 2\\
% \end{tabular}
% \end{center}
% Each of the eight files can be compiled directly by the \LaTeX{} compiler.
%
% %%%%%%%%%%%%%%%%%%%%%%%%%%%%%%%%%%%%%%
% \paragraph{Main File.}
%
% The main file is called |cdocsamp.tex|.
%
% Load the \textsf{childdoc} definitions and
% declare the filename for the main document:
%    \begin{macrocode}
\input{childdoc.def}
\childdocmain{}
%    \end{macrocode}

% Optional override for |\version| flag:
%    \begin{macrocode}
%%\ifchilddoc\else\providecommand{\version}{draft}\fi
%    \end{macrocode}

% Define the default values for the |\version| flag
% (|final| for the main file and |draft| for childs):
%    \begin{macrocode}
\ifchilddoc
\providecommand{\version}{draft}
\else
\providecommand{\version}{final}
\fi
%    \end{macrocode}

% Load the standard document class:
%    \begin{macrocode}
\documentclass[12pt]{article}
%    \end{macrocode}

% Start the document body:
%    \begin{macrocode}
\begin{document}
%    \end{macrocode}

% Declare a title page.
% Print title, part of document being processed and version flag:
%    \begin{macrocode}
\addtocounter{page}{-1}
\begin{center}
{\LARGE\bfseries{}childdoc example\par}
\vspace{1cm}
\ifchilddoc
\ifchilddocmanual part\else chapter\fi:
`\childdocname' of `\childdocjob'\par
\else
main document: `\childdocjob'\par
\fi
version: \version\par
\end{center}
\newpage
%    \end{macrocode}

% Manually include selected file,
% otherwise process as usual:
%    \begin{macrocode}
\ifchilddocmanual
\section*{part `\childdocname'}
\input{\childdocname}
\else
%    \end{macrocode}

% Include the two chapters:
%    \begin{macrocode}
\include{cdocsch1}
\include{cdocsch2}
%    \end{macrocode}

% Include the two parts unless only chapters should be displayed:
%    \begin{macrocode}
\ifchilddoc\else
\section{part three}
\input{cdocspt3}
\section{part four}
\input{cdocspt4}
\fi
%    \end{macrocode}

% Process as usual until here:
%    \begin{macrocode}
\fi
%    \end{macrocode}

% End of document body:
%    \begin{macrocode}
\end{document}
%    \end{macrocode}
%\iffalse
%</samplemain>
%\fi
%
% %%%%%%%%%%%%%%%%%%%%%%%%%%%%%%%%%%%%%%
% \paragraph{Chapter Include Files.}
%
% The include files are called |cdocsch1.tex| and |cdocsch2.tex|.
%
%\iffalse
%<*samplechap1|samplechap2>
%\fi

% Optional override for |\version| flag:
%    \begin{macrocode}
%%\providecommand{\version}{final}
%    \end{macrocode}

% Include the main document:
%    \begin{macrocode}
\input{childdoc.def}
\childdocof{cdocsamp}
%    \end{macrocode}

%\iffalse
%</samplechap1|samplechap2>
%\fi
%
%\iffalse
%<*samplechap1>
%\fi
% Some text for chapter 1:
%    \begin{macrocode}
\section{one}
some text in chapter one
%    \end{macrocode}

%\iffalse
%</samplechap1>
%\fi
% Some text for chapter 2:
%\iffalse
%<*samplechap2>
%\fi
%    \begin{macrocode}
\section{two}
more text in chapter two
%    \end{macrocode}

%\iffalse
%</samplechap2>
%\fi
%
% %%%%%%%%%%%%%%%%%%%%%%%%%%%%%%%%%%%%%%
% \paragraph{Part Include Files.}
%
% The include files are called |cdocspt3.tex| and |cdocspt4.tex|.
%
%\iffalse
%<*samplepart3|samplepart4>
%\fi

% Optional override for |\version| flag:
%    \begin{macrocode}
%%\providecommand{\version}{final}
%    \end{macrocode}

% Include the main document:
%    \begin{macrocode}
\input{childdoc.def}
\childdocby{cdocsamp}
%    \end{macrocode}

%\iffalse
%</samplepart3|samplepart4>
%\fi
%
%\iffalse
%<*samplepart3>
%\fi
% Some text for part 3:
%    \begin{macrocode}
some text in part three
%    \end{macrocode}

%\iffalse
%</samplepart3>
%\fi
% Some text for part 4:
%\iffalse
%<*samplepart4>
%\fi
%    \begin{macrocode}
more text in part four
%    \end{macrocode}

%\iffalse
%</samplepart4>
%\fi
%
% %%%%%%%%%%%%%%%%%%%%%%%%%%%%%%%%%%%%%%
% \paragraph{Forwarding for a Complete Draft.}
%
% The following forwarding file |cdocsdrf.tex|
% compiles the main document in draft mode:
%\iffalse
%<*sampledraft>
%\fi
%    \begin{macrocode}
\def\version{draft}
\input{childdoc.def}
\childdocforward{cdocsamp}
%    \end{macrocode}

%\iffalse
%</sampledraft>
%\fi
%
% %%%%%%%%%%%%%%%%%%%%%%%%%%%%%%%%%%%%%%
% \paragraph{Forwarding for Final Version of the Chapters.}
%
% The following forwarding files |cdocsfn1.tex| and |cdocsfn2.tex|
% (with identical content)
% compile the final versions of the child documents
% |cdocsch1.tex| and |cdocsch2.tex|, respectively:
%\iffalse
%<*samplefinal>
%\fi
%    \begin{macrocode}
\def\version{final}
\input{childdoc.def}
\childdocforwardprefix[cdocsamp]{cdocsfn}{cdocsch}
%    \end{macrocode}

%\iffalse
%</samplefinal>
%\fi
%
% %%%%%%%%%%%%%%%%%%%%%%%%%%%%%%%%%%%%%%
% \paragraph{Command Line Processing.}
%
% The following three command lines generate the output files
% |cdocscld|, |cdocscl1| and |cdocscl2|
% which should be identical to
% |cdocsdrf|, |cdocsch1| and |cdocsfn2|, respectively:
% \begin{center}
% \begin{tabular}{l}
% |latex -jobname cdocscld \|\\
% |  "\def\version{draft}\input{childdoc.def}\childdocforward{cdocsamp}"|\\
% |latex -jobname cdocscl1 \|\\
% |  "\input{childdoc.def}\childdocforward[cdocsamp]{cdocsch1}"|\\
% |latex -jobname cdocscl2 \|\\
% |  "\def\version{final}\input{childdoc.def}\childdocforward{cdocsch2}"|
% \end{tabular}
% \end{center}
% Note that the trailing backslash on each first line
% merely continues the input to the second line
% (for convenient cut ant paste).
% Furthermore, the command |latex| can be replaced by any
% of its alternative versions such as |pdflatex|.
%
% %%%%%%%%%%%%%%%%%%%%%%%%%%%%%%%%%%%%%%%%%%%%%%%%%%%%%%%%%%%%%%%%%%%%%%%%%%%%%%
% %%%%%%%%%%%%%%%%%%%%%%%%%%%%%%%%%%%%%%%%%%%%%%%%%%%%%%%%%%%%%%%%%%%%%%%%%%%%%%
% \section{Implementation}
%\iffalse
%<*package>
%\fi
%
% This section describes the definitions file |childdoc.def|.

% The definitions cannot be loaded using |\usepackage| or |\RequirePackage|
% which has a mechanism to prevent loading a style file more than once.
% When loading the definitions by means of |\input|
% multiple instances have to be prevented manually:
%\iffalse
%This code needs to be before the `\ProvidesFile' directive
%which is defined at the beginning of this file.
%Therefore it is also placed there and commented out here.
%</package>
%<*discard>
%\fi
%    \begin{macrocode}
\ifdefined\childdocmain\endinput\fi
%    \end{macrocode}
%\iffalse
%</discard>
%<*package>
%\fi
%
% \macro{\ifchilddoc}
% \macro{\ifchilddocmanual}
% The conditional |\ifchilddoc| tells whether a
% child (true) or main (false) document is being compiled.
% The conditional |\ifchilddocmanual| tells whether
% the |\includeonly| mechanism is used (false) or
% the selection of child files must be performed manually (true).
% The definitions initialise to false:
%    \begin{macrocode}
\newif\ifchilddoc
\newif\ifchilddocmanual
%    \end{macrocode}

% \macro{\childdocname}
% \macro{\childdocjob}
% The macro |\childdocname| stores the name of the main document
% to be compiled. The macro |\childdocjob| stores the name of
% the document on which the \LaTeX{} compiler was originally invoked.
% The content of |\jobname| cannot be compared
% to filenames specified in the source due to different catcodes.
% The following code rescans |\jobname|, stores the result
% in |\childdocname| and saves a copy in |\childdocjob|:
%    \begin{macrocode}
\edef\childdocname{\scantokens\expandafter{\jobname\noexpand}}
\let\childdocjob\childdocname
%    \end{macrocode}

% \macro{\childdocdisable}
% The macro |\childdocdisable| prevents the main file
% from being processed more than once.
% At this stage, the main document command |\childdocmain|
% is assumed to be called once again where it should do nothing.
% Any subsequent call to it should prevent
% a secondary processing of the main document
% It overwrites the forwarding commands
% |\childdocof| and |\childdocforward|
% with empty macros to prevent further inclusions of the main document:
%    \begin{macrocode}
\newcommand{\childdocdisable}
{
  \renewcommand{\childdocmain}[1]{\renewcommand{\childdocmain}[1]{\endinput}}
  \renewcommand{\childdocof}[1]{}
  \renewcommand{\childdocby}[2][]{}
  \renewcommand{\childdocforward}[2][]{}
  \renewcommand{\childdocdisable}{}
}
%    \end{macrocode}

% \macro{\childdocmain}
% The macro |\childdocmain| is to be called at the top of the main file
% with nothing or the main filename (without extension) as argument.
% First, it breaks loops.
% If the argument is not empty and does not match |\childdocname|
% (which is set by the first inclusion of |childdoc.def|),
% |\ifchilddoc| is set to true, |\includeonly| is applied to the child file
% and |\jobname| is set to the main file
% (for proper handling of |.aux| files):
%    \begin{macrocode}
\newcommand{\childdocmain}[1]
{
  \childdocdisable\childdocmain{}
  \if?#1?\else
    \begingroup
      \def\childdoctmp{#1}
      \ifx\childdoctmp\childdocname
        \def\childdoctmp{}
      \else
        \def\childdoctmp
        {
          \childdoctrue
          \includeonly{\childdocname}
          \def\childdocjob{#1}
          \def\jobname{#1}
        }
      \fi
      \expandafter
    \endgroup
    \childdoctmp
  \fi
}
%    \end{macrocode}

% \macro{\childdocof}
% The command |\childdocof| redirects
% compilation to the main file |#1|.
%    \begin{macrocode}
\newcommand{\childdocof}[1]
{
  \childdocdisable
  \childdoctrue
  \includeonly{\childdocname}
  \def\jobname{#1}
  \def\childdocjob{#1}
  \input{#1}
}
%    \end{macrocode}

% \macro{\childdocby}
% The command |\childdocby| ....
%    \begin{macrocode}
\newcommand{\childdocby}[2][]
{
  \childdocdisable
  \childdoctrue
  \childdocmanualtrue
  \if?#1?\else
    \def\jobname{#2}
  \fi
  \def\childdocjob{#2}
  \input{#2}
  \endinput
}
%    \end{macrocode}

% \macro{\childdocforward}
% The command |\childdocforward| redirects
% compilation to the main file or
% (if the optional argument is given) a child file.
% Parameters are set as if the main file
% or a child file starting with |\childdocof| was compiled.
% Then compilation is handed over to the main file:
%    \begin{macrocode}
\newcommand{\childdocforward}[2][]
{
  \begingroup
    \if?#1?
      \def\childdoctmp
      {
        \def\childdocname{#2}
        \def\childdocjob{#2}
        \def\jobname{#2}
        \input{#2}
        \endinput
      }
    \else
      \def\childdoctmp
      {
        \childdocdisable
        \def\childdocname{#2}
        \childdoctrue
        \includeonly{#2}
        \def\childdocjob{#1}
        \def\jobname{#1}
        \input{#1}
        \endinput
      }
    \fi
    \expandafter
  \endgroup
  \childdoctmp
}
%    \end{macrocode}

% \macro{\childdocforwardprefix}
% The command |\childdocforwardprefix| redirects
% compilation to the main or a child file by means of a pattern.
% The prefix |#1| in the current filename is replaced by |#2|
% and the suffix of the current filename is kept
% (it is assumed that the filename does not contain the substring `|~~~|'
% which is used as a delimiter).
% Compilation is handed over to the new file by |\childdocforward|:
%    \begin{macrocode}
\newcommand{\childdocforwardprefix}[3][]
{
  \begingroup
    \def\childdocextract #2##1~~~{\def\childdoctmp{\childdocforward[#1]{#3##1}}}
    \expandafter\childdocextract\childdocname~~~
    \expandafter
  \endgroup
  \childdoctmp
}
%    \end{macrocode}

% \macro{\childdoc}
% The deprecated macro |\childdoc| is a legacy version of |\childdocmain|:
%    \begin{macrocode}
\newcommand{\childdoc}{\childdocmain}
%    \end{macrocode}

% \macro{\childdocredirect}
% The deprecated macro |\childdocredirect| is a legacy version
% of |\childdocforward| and |\childdocforwardprefix|:
%    \begin{macrocode}
\newcommand{\childdocredirect}[2][]
{
  \begingroup
    \if?#1?
      \def\childdoctmp{\childdocforward{#2}}
    \else
      \def\childdoctmp{\childdocforwardprefix{#1}{#2}}
    \fi
    \expandafter
  \endgroup
  \childdoctmp
}
%    \end{macrocode}

%\iffalse
%</package>
%\fi
%
\endinput

\childdocforwardprefix[cdocsamp]{cdocsfn}{cdocsch}
%    \end{macrocode}

%\iffalse
%</samplefinal>
%\fi
%
% %%%%%%%%%%%%%%%%%%%%%%%%%%%%%%%%%%%%%%
% \paragraph{Command Line Processing.}
%
% The following three command lines generate the output files
% |cdocscld|, |cdocscl1| and |cdocscl2|
% which should be identical to
% |cdocsdrf|, |cdocsch1| and |cdocsfn2|, respectively:
% \begin{center}
% \begin{tabular}{l}
% |latex -jobname cdocscld \|\\
% |  "\def\version{draft}% \iffalse
%
% childdoc.dtx Copyright (C) 2017-2018 Niklas Beisert
%
% This work may be distributed and/or modified under the
% conditions of the LaTeX Project Public License, either version 1.3
% of this license or (at your option) any later version.
% The latest version of this license is in
%   http://www.latex-project.org/lppl.txt
% and version 1.3 or later is part of all distributions of LaTeX
% version 2005/12/01 or later.
%
% This work has the LPPL maintenance status `maintained'.
%
% The Current Maintainer of this work is Niklas Beisert.
%
% This work consists of the files childdoc.dtx and childdoc.ins
% and the derived files childdoc.def and cdocsamp.tex with
% cdocsch1.tex, cdocsch2.tex, cdocsdrf.tex, cdocsfn1.tex, cdocsfn2.tex.
%
%<package>\ifdefined\childdocmain\endinput\fi
%<package>\ProvidesFile{childdoc.def}[2018/12/30 v2.0 child document driver]
%<samplemain>\ProvidesFile{cdocsamp.tex}[2018/12/30 v2.0 sample for childdoc]
%<*driver>
%\ProvidesFile{childdoc.drv}[2018/12/30 v2.0 childdoc reference manual file]
\PassOptionsToClass{10pt,a4paper}{article}
\documentclass{ltxdoc}

\usepackage[margin=35mm]{geometry}
\usepackage{hyperref}
\usepackage{hyperxmp}
\usepackage[usenames]{color}

\hypersetup{colorlinks=true}
\hypersetup{pdfstartview=FitH}
\hypersetup{pdfpagemode=UseNone}
\hypersetup{pdfsource={}}
\hypersetup{pdflang={en-UK}}
\hypersetup{pdfcopyright={Copyright 2017-2018 Niklas Beisert.
  This work may be distributed and/or modified under the
  conditions of the LaTeX Project Public License, either version 1.3
  of this license or (at your option) any later version.}}
\hypersetup{pdflicenseurl={http://www.latex-project.org/lppl.txt}}
\hypersetup{pdfcontactaddress={ETH Zurich, ITP, HIT K,
  Wolfgang-Pauli-Strasse 27}}
\hypersetup{pdfcontactpostcode={8093}}
\hypersetup{pdfcontactcity={Zurich}}
\hypersetup{pdfcontactcountry={Switzerland}}
\hypersetup{pdfcontactemail={nbeisert@itp.phys.ethz.ch}}
\hypersetup{pdfcontacturl={http://people.phys.ethz.ch/\xmptilde nbeisert/}}

\newcommand{\secref}[1]{\hyperref[#1]{section \ref*{#1}}}

\parskip1ex
\parindent0pt
\let\olditemize\itemize
\def\itemize{\olditemize\parskip0pt}

\begin{document}

\title{The \textsf{childdoc} Package}
\hypersetup{pdftitle={The childdoc Package}}
\author{Niklas Beisert\\[2ex]
  Institut f\"ur Theoretische Physik\\
  Eidgen\"ossische Technische Hochschule Z\"urich\\
  Wolfgang-Pauli-Strasse 27, 8093 Z\"urich, Switzerland\\[1ex]
  \href{mailto:nbeisert@itp.phys.ethz.ch}
  {\texttt{nbeisert@itp.phys.ethz.ch}}}
\hypersetup{pdfauthor={Niklas Beisert}}
\hypersetup{pdfsubject={Manual for the LaTeX2e Package childdoc}}
\date{30 December 2018, \textsf{v2.0}}
\maketitle

\begin{abstract}\noindent
\textsf{childdoc} is a \LaTeXe{} package
that enables the direct compilation
of document sections included by |\include|
to individual files.
\end{abstract}

\begingroup
\parskip0ex
\tableofcontents
\endgroup

%%%%%%%%%%%%%%%%%%%%%%%%%%%%%%%%%%%%%%%%%%%%%%%%%%%%%%%%%%%%%%%%%%%%%%%%%%%%%%%%
%%%%%%%%%%%%%%%%%%%%%%%%%%%%%%%%%%%%%%%%%%%%%%%%%%%%%%%%%%%%%%%%%%%%%%%%%%%%%%%%
\section{Introduction}

\LaTeX{} provides a mechanism to structure a large document (such as a book)
into a main file and several child files (containing the chapters)
using the |\include| command.
This mechanism is beneficial for documents
which span hundreds of pages in order to
make the source file(s) more manageable.
Moreover, compilation can be restricted to
selected child files by means of the |\includeonly| command.
The latter feature can be used to reduce the compilation time while editing
(this was significantly more useful in the earlier days of \LaTeX{})
or to generate a smaller document which is easier to navigate.
Another application of |\includeonly| is to generate
documents consisting of selected parts of the complete document.

However, there are a few drawbacks of the plain |\include| mechanism:
\begin{itemize}
\item
The child files cannot be compiled on their own,
they can only be compiled via the main file.
A naive editing environment
(such as a text editor with an option
to have the current file processed by \LaTeX)
may require one to switch to the main file before compiling;
attempting to compile the child file produces errors.
\item
The main file must be modified (each time)
to adjust the |\includeonly| command
to the present needs. This easily leaves the main file in a messy state.
\item
The generated document will always carry the filename
of the main document. This is inconvenient if
several child files are to be compiled and
to be kept for distribution.
\end{itemize}

The present package provides a simple interface
to make child files individually compilable by \LaTeX{}.
Compiling a child file then has the same effect as compiling
the main file with an |\includeonly| command
to select the appropriate child.
Moreover the generated document will carry the name of the child
rather than the main file.
This resolves all three above issues.

This feature is meant to make the editing of books,
thesis documents and lecture notes somewhat more convenient.
However, the package can also be used efficiently for
composing a series of documents (such as exercise sheets)
which are typically distributed individually.
It then assists the author in generating the individual documents
(potentially in different versions)
as well as a document containing the collected series.
Another application is in developing style files
or other kinds of included material
where compilation of the style file could redirect
to a sample or test file.

%%%%%%%%%%%%%%%%%%%%%%%%%%%%%%%%%%%%%%%%%%%%%%%%%%%%%%%%%%%%%%%%%%%%%%%%%%%%%%%%
%%%%%%%%%%%%%%%%%%%%%%%%%%%%%%%%%%%%%%%%%%%%%%%%%%%%%%%%%%%%%%%%%%%%%%%%%%%%%%%%
\section{Usage}

First of all, the package \textsf{childdoc} is \emph{not} a standard
\LaTeXe{} |.sty| style file! Therefore it needs to be invoked in
a non-standard way.

%%%%%%%%%%%%%%%%%%%%%%%%%%%%%%%%%%%%%%%%%%%%%%%%%%%%%%%%%%%%%%%%%%%%%%%%%%%%%%%%
\subsection{Included Files}
\label{sec:include}

%%%%%%%%%%%%%%%%%%%%%%%%%%%%%%%%%%%%%%%%
\DescribeMacro{\childdocmain}
To use the package, add the commands
\begin{center}
\begin{tabular}{l}
|\input{childdoc.def}|\\
|\childdocmain{}|\\
\end{tabular}
\end{center}
at the very top of the main \LaTeX{} file,
in particular \emph{before} the |\documentclass| statement!
The argument of |\childdocmain| should be left empty
(but it must be present).

%%%%%%%%%%%%%%%%%%%%%%%%%%%%%%%%%%%%%%%%
\DescribeMacro{\childdocof}
Furthermore, add the commands
\begin{center}
\begin{tabular}{l}
|\input{childdoc.def}|\\
|\childdocof{|\textit{main}|}|\\
\end{tabular}
\end{center}
at the top of every child file \textit{child}
which is included by |\include{|\textit{child}|}|
from within the main file
(or at least for those files to be compiled individually).
The argument \textit{main} must be the filename of the main file.

There are a couple of
considerations in setting up the main and child documents:

%%%%%%%%%%%%%%%%%%%%%%%%%%%%%%%%%%%%%%%%
\paragraph{Restrictions.}

Please note the following restrictions:
\begin{itemize}
\item
|\childdocmain| must be called with one argument \textit{main}
to ensure compatibility with earlier version of the package.
It must either be empty (|\childdocmain{}|)
or precisely match the filename of the main file in which it is specified.
See \secref{sec:detection} for further information.
\item
The filename \textit{main} must be specified without the |.tex| extension.
\item
The filename \textit{main} is case sensitive
(even in case-insensitive file systems)
due to internal string comparison.
\item
The argument \textit{main} should be fully expanded, it cannot be a macro.
\item
Subdirectories and special characters should be avoided in filenames.
\item
The command |\childdocmain{|\textit{main}|}| must be followed by a whitespace.
It should not be followed immediately by another command
or by a comment mark `|%|'.
This is because the \TeX{} parser reads the token immediately following
the argument of |\childdocmain| and puts it
at the beginning of every child section;
however, a white\-space is ignored.
\end{itemize}

%%%%%%%%%%%%%%%%%%%%%%%%%%%%%%%%%%%%%%%%
\paragraph{Content of Main File.}

It is advisable to place all content in the child files included by |\include|.
Any output contained in the main file will appear in all child documents
unless suppressed manually;
it cannot be suppressed automatically by the |\includeonly| directive
and thus should normally be avoided.
A method to include some content in the main file
by means of conditional processing is described in \secref{sec:conditional}.

%%%%%%%%%%%%%%%%%%%%%%%%%%%%%%%%%%%%%%%%
\paragraph{Page Numbering.}

When only a part of the document is compiled,
the appropriate numbering of pages
(as well as other status parameters)
is determined from the |.aux| files.
The latter contain information from previous passes.
However this information needs to propagate through
all intermediate child documents.
Therefore the page numbering in child documents may well
be inconsistent until the complete document is compiled at least once.

A useful (if unconventional) way to always ensure a consistent
page numbering is to restart the numbering in each child document
and denote the pages by `\textit{child}|.|\textit{page}'
where \textit{child} represents the chapter/section number of the child file.
This can be achieved by the command
|\numberwithin{page}{|\textit{child}|}|
of the \textsf{amsmath} package
where \textit{child} can be |chapter| or |section|
depending on the chosen structuring.
Alternatively, one can modify the macro |\thepage| appropriately
and reset the counter |page| at the start of each child file.

%%%%%%%%%%%%%%%%%%%%%%%%%%%%%%%%%%%%%%%%%%%%%%%%%%%%%%%%%%%%%%%%%%%%%%%%%%%%%%%%
\subsection{Conditional Processing}
\label{sec:conditional}

The package provides a mechanism to compile different versions
of a document. To customise the versions further some conditional processing
can come in handy to distinguish which version is being compiled.
The package provides two macros to describe the compilation context:

%%%%%%%%%%%%%%%%%%%%%%%%%%%%%%%%%%%%%%%%
\DescribeMacro{\ifchilddoc}
The conditional |\ifchilddoc| distinguishes between the compilation of
child documents and the main document:
%
\begin{center}
|\ifchilddoc |\textit{child-code}| |[|\||else |\textit{main-code}]| \||fi|
\end{center}

%%%%%%%%%%%%%%%%%%%%%%%%%%%%%%%%%%%%%%%%
\DescribeMacro{\childdocname}
\DescribeMacro{\childdocjob}
The macro |\childdocname| contains the filename (without extension)
of the main or child file being processed.
Note that |\childdocjob| will always contain the name of the main file.

%%%%%%%%%%%%%%%%%%%%%%%%%%%%%%%%%%%%%%%%
\paragraph{Title Page.}

Conditional processing can be used to include a title or banner page
in the main document when proper precautions are taken.
Importantly, the code in the main file should ensure that the page counter
(as well as other status parameters which are stored in the |.aux| files)
takes the same value after the conditional processing.
Otherwise the page numbers may take divergent values
depending on which part is compiled.

For example, a title page could be declared by:
%
\begin{center}
\begin{tabular}{l}
|\ifchilddoc\||else|\\
|\addtocounter{page}{-1}|\\
\textit{code for title page}\\
|\newpage|\\
|\||fi|
\end{tabular}
\end{center}
%
A banner page for the child documents can be generated by:
%
\begin{center}
\begin{tabular}{l}
|\ifchilddoc|\\
|\addtocounter{page}{-1}|\\
\textit{code for banner page}\\
|\newpage|\\
|\||fi|
\end{tabular}
\end{center}
%
Here one could write a message such as:
\begin{center}
|This is the part \childdocname{} of \childdocjob{}.|
\end{center}

%%%%%%%%%%%%%%%%%%%%%%%%%%%%%%%%%%%%%%%%%%%%%%%%%%%%%%%%%%%%%%%%%%%%%%%%%%%%%%%%
\subsection{Flags}
\label{sec:flags}

The package makes it easy to generate different versions
of the main or child documents.
To this end compilation flags can be defined
and assigned different default values.
They will be particularly useful in conjunction
with the forwarding mechanism described in \secref{sec:forward}.

For example, it may be useful to have a flag |\version|
which can be set to |draft| or |final|.
The document source will contain some conditional code
depending on the value of |\version|.
Suppose further, the flag should default to |final| for the main file
and to |draft| for child files
which is a natural assignment for editing the document.
This is achieved by placing the following code
in the preamble of the main document
(below the |\childdocmain| directive):
%
\begin{center}
\begin{tabular}{l}
|\ifchilddoc|\\
|\providecommand{\version}{draft}|\\
|\||else|\\
|\providecommand{\version}{final}|\\
|\||fi|
\end{tabular}
\end{center}
%
The definition by |\providecommand| makes sure
that previous definitions are not overwritten.
Further statements |\providecommand{\version}{...}|
can thus be added before the above code to override it.

For the main file, one might add a line
(between |\childdocmain| and the above block)
%
\begin{center}
|%\ifchilddoc\||else\providecommand{\version}{draft}\||fi|
\end{center}
%
which can be uncommented to produce a draft version.
Likewise one can add a line to the very top of a child file
(above the |\childdocof{|\textit{main}|}| directive)
%
\begin{center}
|%\providecommand{\version}{final}|
\end{center}
%
which can be uncommented to produce the final version of this child document.

%%%%%%%%%%%%%%%%%%%%%%%%%%%%%%%%%%%%%%%%%%%%%%%%%%%%%%%%%%%%%%%%%%%%%%%%%%%%%%%%
\subsection{Forwarding}
\label{sec:forward}

Different versions of the main or child documents
using compilation flags as described in \secref{sec:flags}
can be (permanently) stored in different files
for convenient compilation, viewing and distribution.
To this end, the package defines a command
to pass on compilation to a different file:

%%%%%%%%%%%%%%%%%%%%%%%%%%%%%%%%%%%%%%%%
\DescribeMacro{\childdocforward}
The command |\childdocforward| redirects processing to
another source file:
%
\begin{center}
\begin{tabular}{l}
|\input{childdoc.def}|\\
|\childdocforward[|\textit{main}|]{|\textit{dest}|}|\\
\end{tabular}
\end{center}
%
The argument \textit{dest} is the destination file
(without extension).
It should be the main file or one of the child files.
Note that further \textsf{childdoc} directives
such as |\childdocof| and |\childdocforward|
in the indicated file will be processed in this form.
The optional argument \textit{main}
passes on directly to the main file \textit{main}
while pretending to compile the child \textit{dest}.
This form behaves as if \textit{dest}
issues |\childdocof{|\textit{main}|}| right away,
and no further \textsf{childdoc} directives will be processed.

%%%%%%%%%%%%%%%%%%%%%%%%%%%%%%%%%%%%%%%%
\DescribeMacro{\...prefix}
In the alternative form |\childdocforwardprefix|,
%
\begin{center}
\begin{tabular}{l}
|\input{childdoc.def}|\\
|\childdocforwardprefix[|\textit{main}|]{|\textit{prefix}|}{|\textit{dest}|}|
\end{tabular}
\end{center}
%
the destination file is determined by a pattern
depending on the current file:
To make this work, the current file must be called
`{\textit{prefix}\hspace{0.2em}\textit{suffix}}'
with \textit{prefix} matching precisely the argument.
Processing is then passed on to the file
`{\textit{dest}\hspace{0.2em}\textit{suffix}}'.
Surely, the same effect is achieved by
directly specifying the
argument `{\textit{dest}\hspace{0.2em}\textit{suffix}}'
in the first form.
However, that requires to set up a different file
for each child. With the alternative form of the command
all these files can have exactly the same content
which simplifies setting them up and maintaining them.

For example, the following file |draft.tex|
with a compilation flag |\version| as described in \secref{sec:flags}
compiles the main document as a draft:
%
\begin{center}
\begin{tabular}{l}
|\def\version{draft}|\\
|\input{childdoc.def}|\\
|\childdocforward{|\textit{main}|}|
\end{tabular}
\end{center}
%
Likewise, the following files |final|\textit{nn}|.tex|
compile the final version of the child document
|child|\textit{nn}|.tex|:
%
\begin{center}
\begin{tabular}{l}
|\def\version{final}|\\
|\input{childdoc.def}|\\
|\childdocforwardprefix{final}{child}|
\end{tabular}
\end{center}
%

Note that when several versions of a main file and/or of each child file
are to be generated, it may be convenient to set up a |Makefile| or
shell script to automatise the process.

%%%%%%%%%%%%%%%%%%%%%%%%%%%%%%%%%%%%%%%%%%%%%%%%%%%%%%%%%%%%%%%%%%%%%%%%%%%%%%%%
\subsection{Command Line Processing}
\label{sec:commandline}

The effect of redirection files can also be achieved by invoking
the \LaTeX{} compiler with a more elaborate command line.
Most conveniently this should be done as part
of a shell script or a |Makefile|.

When using \textsf{childdoc} in the main file, the following
command lines effectively perform a redirection
(note that depending on the shell being used,
backslashes may have to be doubled: `|\|' $\to$ `|\\|'):
%
\begin{center}
|... -jobname "|\textit{target}|" |\\|"|[\textit{flags}]%
|\input{childdoc.def}\childdocforward[|\textit{main}|]{|\textit{dest}|}"|
\end{center}
%
Here \textit{target} is the name of the output file,
\textit{main} is the name of the main file
and \textit{dest} is the name of the main or child file to be processed
(all filenames without extensions).
The optional argument \textit{main} can be omitted
if \textit{main} matches \textit{dest}.
Optionally, compilation \textit{flags} can be defined via |\def| commands.
This command line makes the \TeX{} engine believe
it is compiling the file \textit{target}
whose content is specified as the latter parameter.
The provided code then forwards the processing to
\textit{main} or \textit{dest} as described in \secref{sec:forward}.

%%%%%%%%%%%%%%%%%%%%%%%%%%%%%%%%%%%%%%%%%%%%%%%%%%%%%%%%%%%%%%%%%%%%%%%%%%%%%%%%
\subsection{Include by Input}
\label{sec:input}

Including child documents by |\include| has some restrictions by design.
Most notably, the content of a child document always occupies
its own set of pages; pages cannot be shared between child documents.
Usually, this behaviour makes perfect sense
because each child document contain an essential part of the document.
However, in some situations it may be desirable to compose
a document from a collection of parts
without having mandatory page breaks between then.
For this case, the package
provides a mechanism to include parts
by |\input| which can also be processed individually.
However, by construction this mechanism
requires manual handling of the content to be output.

%%%%%%%%%%%%%%%%%%%%%%%%%%%%%%%%%%%%%%%%
\DescribeMacro{\ifchilddocmanual}
The main file should be prepared as usual, see \secref{sec:include}.
However, the document body must make a distinction
between processing of an individual part and of the main document, e.g.:
%
\begin{center}
\begin{tabular}{l}
|\ifchilddocmanual|\\
|\input{\childdocname}|\\
|\||else|\\
\textit{document body with }|\input{|\textit{part}|}|\\
|\||fi|
\end{tabular}
\end{center}
%
The conditional |\ifchilddocmanual| is true whenever
a part to be included by |\input| is being compiled,
and the name of the part is stored in |\childdocname|.

%%%%%%%%%%%%%%%%%%%%%%%%%%%%%%%%%%%%%%%%
\DescribeMacro{\childdocby}
Each part to be included by |\input| should start with:
%
\begin{center}
\begin{tabular}{l}
|\input{childdoc.def}|\\
|\childdocby{|\textit{main}|}|\\
\end{tabular}
\end{center}
%
The directive |\childdocby| is similar to |\childdocof|
described in \secref{sec:include},
but the subsequent selection of content must be done manually.
To that end, both |\ifchilddoc| and |\ifchilddocmanual|
will be true upon processing of a part,
and the name of the part is stored in |\childdocname|.
Note that |\jobname| will be set to the filename of the current part
so that each part receives an individual |.aux| file
that does not interfere with the |.aux| file(s) of the main document.
This behaviour can be altered by the alternative form
|\childdocby[*]{|\textit{main}|}| (with a non-empty optional argument)
which uses the |.aux| file of the main document
by setting |\jobname| to \textit{main}.

%%%%%%%%%%%%%%%%%%%%%%%%%%%%%%%%%%%%%%%%%%%%%%%%%%%%%%%%%%%%%%%%%%%%%%%%%%%%%%%%
\subsection{Driver Development}
\label{sec:driver}

The \textsf{childdoc} mechanism can also be use for the development
of definition files such as \LaTeX{} styles or classes.
This case differs from the above setup with multiple parts
included by |\include| in that no |\includeonly| should be invoked.
This can be achieved by starting the include file
(before |\ProvidesPackage|) with:
%
\begin{center}
\begin{tabular}{l}
|\input{childdoc.def}|\\
|\childdocforward{|\textit{main}|}|\\
\end{tabular}
\end{center}
%
or alternatively with:
%
\begin{center}
\begin{tabular}{l}
|\input{childdoc.def}|\\
|\childdocby{|\textit{main}|}|\\
\end{tabular}
\end{center}
%
Both forms have slightly different effects as described above.
The main file is prepared as usual, see \secref{sec:include}.

%%%%%%%%%%%%%%%%%%%%%%%%%%%%%%%%%%%%%%%%%%%%%%%%%%%%%%%%%%%%%%%%%%%%%%%%%%%%%%%%
\subsection{Legacy Detection}
\label{sec:detection}

The directive |\childdocmain| in the main file can detect
whether the complete document or merely a child is to be compiled
even without using the directive |\childdocof|.
This method is deprecated because it is less robust
and there is no compelling reason to use it;
it is merely provided for backward compatibility
and it may be removed in future versions.

If the detection mechanism is to be used,
it is mandatory to correctly specify
the filename of the main file as the argument of |\childdocmain|:
%
\begin{center}
\begin{tabular}{l}
|\input{childdoc.def}|\\
|\childdocmain{|\textit{main}|}|\\
\end{tabular}
\end{center}
%
If |\jobname| does not match the argument \textit{main} of |\childdocmain|,
it is assumed that |\jobname| points to the child file to be compiled.
When using |\childdocmain| with the main file specified as argument,
it suffices to start a child file
with just |\input{|\textit{main}|}|
without loading of the package and using |\childdocof|.
If instead all processing is done
with the appropriate \textsf{childdoc} directives,
the argument of \textit{main} of |\childdocmain| can be empty.

An alternative version of the command line processing described
in \secref{sec:commandline} using the detection mechanism reads:
%
\begin{center}
|... -jobname "|\textit{target}|" "|[\textit{flags}]%
[|\def\jobname{|\textit{dest}|}|]|\input{|\textit{main}|}"|
\end{center}

%%%%%%%%%%%%%%%%%%%%%%%%%%%%%%%%%%%%%%%%%%%%%%%%%%%%%%%%%%%%%%%%%%%%%%%%%%%%%%%%
\subsection{Manual Code}
\label{sec:manual}

In case one cannot be certain whether the definitions file |childdoc.def|
is installed on the target \TeX{} distribution
and one prefers not to ship it,
it is conceivable to paste a few relevant commands into the sources.

To that end, drop all statements |\input{childdoc.def}|
and perform the replacements as outlined below.
Instead of |\childdocmain{|\textit{main}|}| add the following code
to the top of the main file:
%
\begin{center}
\begin{tabular}{l}
|\||ifdefined\childdocname\endinput\||fi\newif\ifchilddoc|\\
|\edef\childdocname{\scantokens\expandafter{\jobname\noexpand}}|\\
|\def\childdocmain{|\textit{main}|}\||ifx\childdocmain\childdocname\||else|\\
|\childdoctrue\includeonly{\childdocname}\let\jobname\childdocmain\||fi|\\
\end{tabular}
\end{center}
%
Instead of |\childdocof{|\textit{main}|}| just include the main file
at the top of each child file:
%
\begin{center}
|\input{|\textit{main}|}|
\end{center}
%
A simple redirection |\childdocforward{|\textit{dest}|}| is achieved by:
%
\begin{center}
|\def\jobname{|\textit{dest}|}\input{\jobname}|
\end{center}
%
The redirection with prefix
|\childdocforwardprefix[|\textit{prefix}|]{|\textit{dest}|}|
is accomplished by:
%
\begin{center}
\begin{tabular}{l}
|{\edef\jobname{\scantokens\expandafter{\jobname\noexpand}}|\\
|\def\redirectjob |\textit{prefix}|#1~~~{\gdef\jobname{|\textit{dest}|#1}}|\\
|\expandafter\redirectjob\jobname~~~}\input{\jobname}|
\end{tabular}
\end{center}

In an alternative approach,
child documents can be compiled by a specific command line
without additional code or specific definitions:
%
\begin{center}
|... -jobname "|\textit{target}|" "|[\textit{flags}]%
|\includeonly{|\textit{dest}|}\input{|\textit{main}|}"|
\end{center}
%

%%%%%%%%%%%%%%%%%%%%%%%%%%%%%%%%%%%%%%%%%%%%%%%%%%%%%%%%%%%%%%%%%%%%%%%%%%%%%%%%
%%%%%%%%%%%%%%%%%%%%%%%%%%%%%%%%%%%%%%%%%%%%%%%%%%%%%%%%%%%%%%%%%%%%%%%%%%%%%%%%
\section{Information}

%%%%%%%%%%%%%%%%%%%%%%%%%%%%%%%%%%%%%%%%%%%%%%%%%%%%%%%%%%%%%%%%%%%%%%%%%%%%%%%%
\subsection{Copyright}

Copyright \copyright{} 2017--2018 Niklas Beisert

This work may be distributed and/or modified under the
conditions of the \LaTeX{} Project Public License, either version 1.3
of this license or (at your option) any later version.
The latest version of this license is in
  \url{http://www.latex-project.org/lppl.txt}
and version 1.3 or later is part of all distributions of \LaTeX{}
version 2005/12/01 or later.

This work has the LPPL maintenance status `maintained'.

The Current Maintainer of this work is Niklas Beisert.

This work consists of the files |README.txt|, |childdoc.ins| and |childdoc.dtx|
as well as the derived files |childdoc.def|, |cdocsamp.tex|
with |cdocsch1.tex|, |cdocsch2.tex|, |cdocspt3.tex|, |cdocspt4.tex|,
|cdocsdrf.tex|, |cdocsfn1.tex|, |cdocsfn2.tex|
as well as |childdoc.pdf|.

%%%%%%%%%%%%%%%%%%%%%%%%%%%%%%%%%%%%%%%%%%%%%%%%%%%%%%%%%%%%%%%%%%%%%%%%%%%%%%%%
\subsection{Files and Installation}

The package consists of the files:
%
\begin{center}
\begin{tabular}{ll}
    |README.txt|   & readme file \\
    |childdoc.ins| & installation file \\
    |childdoc.dtx| & source file \\
    |childdoc.def| & definition file \\
    |cdocsamp.tex| & sample main file \\
    |cdocsch1.tex| & sample include file \\
    |cdocsch2.tex| & sample include file \\
    |cdocspt3.tex| & sample part file \\
    |cdocspt4.tex| & sample part file \\
    |cdocsdrf.tex| & sample redirection file \\
    |cdocsfn1.tex| & sample redirection file \\
    |cdocsfn2.tex| & sample redirection file \\
    |childdoc.pdf| & manual
\end{tabular}
\end{center}
%
The distribution consists of the files
|README.txt|, |childdoc.ins| and |childdoc.dtx|.
%
\begin{itemize}
\item
Run (pdf)\LaTeX{} on |childdoc.dtx|
to compile the manual |childdoc.pdf| (this file).
\item
Run \LaTeX{} on |childdoc.ins| to create the definitions file |childdoc.def|
and the sample |cdocsamp.tex| with include files
|cdocsch1.tex|, |cdocsch2.tex|, |cdocspt3.tex|, |cdocspt4.tex|,
|cdocsdrf.tex|, |cdocsfn1.tex|, |cdocsfn2.tex|.
Then copy the file |childdoc.def| to an appropriate directory of your \LaTeX{}
distribution, e.g.\ \textit{texmf-root}|/tex/latex/childdoc|.
\end{itemize}

%%%%%%%%%%%%%%%%%%%%%%%%%%%%%%%%%%%%%%%%%%%%%%%%%%%%%%%%%%%%%%%%%%%%%%%%%%%%%%%%
\subsection{Related CTAN Packages}

There are several other packages which offer a similar functionality:
%
\begin{itemize}
\item
The packages
\href{http://ctan.org/pkg/docmute}{\textsf{docmute}},
\href{http://ctan.org/pkg/includex}{\textsf{includex}} and
\href{http://ctan.org/pkg/standalone}{\textsf{standalone}}
provide commands to include only the document body of
a child file thus allowing both files to be compiled individually.
\item
The packages \href{http://ctan.org/pkg/subdocs}{\textsf{subdocs}}
and \href{http://ctan.org/pkg/subfiles}{\textsf{subfiles}}
provide structures in which the main and child documents can be
encapsulated and allowing them to be compiled individually.
The inclusion mechanism is different from the conventional |\include|.
\item
The package \href{http://ctan.org/pkg/combine}{\textsf{combine}}
is an elaborate solution to combine several documents into one.
\end{itemize}
%
See also the CTAN topic \href{http://ctan.org/topic/subdocs}{\textsf{subdocs}}
for further related packages.
The present package differs from the above solutions in that
a document structure constructed with the conventional |\include| mechanism
just needs two extra commands at the top of every file
such that all constituent files can be compiled individually.

%%%%%%%%%%%%%%%%%%%%%%%%%%%%%%%%%%%%%%%%%%%%%%%%%%%%%%%%%%%%%%%%%%%%%%%%%%%%%%%%
%\subsection{Feature Suggestions}
%
%The following is a list of features which may be useful for future
%versions of this package:
%%
%\begin{itemize}
%\item
%\ldots
%\end{itemize}

%%%%%%%%%%%%%%%%%%%%%%%%%%%%%%%%%%%%%%%%%%%%%%%%%%%%%%%%%%%%%%%%%%%%%%%%%%%%%%%%
\subsection{Revision History}

%%%%%%%%%%%%%%%%%%%%%%%%%%%%%%%%%%%%%%%%
\paragraph{v2.0:} 2018/12/30

\begin{itemize}
\item
immediate forward processing
\item
added |\childdocby| mechanism
\item
manual restructured
\end{itemize}

%%%%%%%%%%%%%%%%%%%%%%%%%%%%%%%%%%%%%%%%
\paragraph{v1.6:} 2018/01/17

\begin{itemize}
\item
application for development of include files
\item
corrections to manual
\end{itemize}

%%%%%%%%%%%%%%%%%%%%%%%%%%%%%%%%%%%%%%%%
\paragraph{v1.5:} 2017/05/21

\begin{itemize}
\item
more complete structuring introduced
\item
|\childdocof| introduced
\item
|\childdoc| renamed to |\childdocmain|
\item
|\childredirect| renamed to |\childdocforward| and |\childdocforwardprefix|
and functionality expanded
\end{itemize}

%%%%%%%%%%%%%%%%%%%%%%%%%%%%%%%%%%%%%%%%
\paragraph{v1.0:} 2017/04/27

\begin{itemize}
\item
manual and install package
\item
first version published on CTAN
\end{itemize}

%%%%%%%%%%%%%%%%%%%%%%%%%%%%%%%%%%%%%%%%
\paragraph{v0.6:} 2017/04/26

\begin{itemize}
\item
redirection mechanism added
\end{itemize}

%%%%%%%%%%%%%%%%%%%%%%%%%%%%%%%%%%%%%%%%
\paragraph{v0.5:} 2017/04/26

\begin{itemize}
\item
functionality in definition file
\end{itemize}


%%%%%%%%%%%%%%%%%%%%%%%%%%%%%%%%%%%%%%%%%%%%%%%%%%%%%%%%%%%%%%%%%%%%%%%%%%%%%%%%
%%%%%%%%%%%%%%%%%%%%%%%%%%%%%%%%%%%%%%%%%%%%%%%%%%%%%%%%%%%%%%%%%%%%%%%%%%%%%%%%
%%%%%%%%%%%%%%%%%%%%%%%%%%%%%%%%%%%%%%%%%%%%%%%%%%%%%%%%%%%%%%%%%%%%%%%%%%%%%%%%
\appendix

\settowidth\MacroIndent{\rmfamily\scriptsize 000\ }

 \DocInput{childdoc.dtx}

\end{document}
%</driver>
% \fi
%
% %%%%%%%%%%%%%%%%%%%%%%%%%%%%%%%%%%%%%%%%%%%%%%%%%%%%%%%%%%%%%%%%%%%%%%%%%%%%%%
% %%%%%%%%%%%%%%%%%%%%%%%%%%%%%%%%%%%%%%%%%%%%%%%%%%%%%%%%%%%%%%%%%%%%%%%%%%%%%%
% \section{Sample}
%\iffalse
%<*samplemain>
%\fi
%
% The following presents a sample document
% with two chapters, two parts, a title page,
% a compile flag as well as three forwarding files to set the flag.
% It consists of eight |.tex| files:
% \begin{center}
% \begin{tabular}{ll}
% |cdocsamp.tex|&main file\\
% |cdocsch1.tex|&include file for chapter 1\\
% |cdocsch2.tex|&include file for chapter 2\\
% |cdocspt3.tex|&include file for part 3\\
% |cdocspt4.tex|&include file for part 4\\
% |cdocsdrf.tex|&forwarding file for main file in draft mode\\
% |cdocsfi1.tex|&forwarding file for final version of chapter 1\\
% |cdocsfi2.tex|&forwarding file for final version of chapter 2\\
% \end{tabular}
% \end{center}
% Each of the eight files can be compiled directly by the \LaTeX{} compiler.
%
% %%%%%%%%%%%%%%%%%%%%%%%%%%%%%%%%%%%%%%
% \paragraph{Main File.}
%
% The main file is called |cdocsamp.tex|.
%
% Load the \textsf{childdoc} definitions and
% declare the filename for the main document:
%    \begin{macrocode}
\input{childdoc.def}
\childdocmain{}
%    \end{macrocode}

% Optional override for |\version| flag:
%    \begin{macrocode}
%%\ifchilddoc\else\providecommand{\version}{draft}\fi
%    \end{macrocode}

% Define the default values for the |\version| flag
% (|final| for the main file and |draft| for childs):
%    \begin{macrocode}
\ifchilddoc
\providecommand{\version}{draft}
\else
\providecommand{\version}{final}
\fi
%    \end{macrocode}

% Load the standard document class:
%    \begin{macrocode}
\documentclass[12pt]{article}
%    \end{macrocode}

% Start the document body:
%    \begin{macrocode}
\begin{document}
%    \end{macrocode}

% Declare a title page.
% Print title, part of document being processed and version flag:
%    \begin{macrocode}
\addtocounter{page}{-1}
\begin{center}
{\LARGE\bfseries{}childdoc example\par}
\vspace{1cm}
\ifchilddoc
\ifchilddocmanual part\else chapter\fi:
`\childdocname' of `\childdocjob'\par
\else
main document: `\childdocjob'\par
\fi
version: \version\par
\end{center}
\newpage
%    \end{macrocode}

% Manually include selected file,
% otherwise process as usual:
%    \begin{macrocode}
\ifchilddocmanual
\section*{part `\childdocname'}
\input{\childdocname}
\else
%    \end{macrocode}

% Include the two chapters:
%    \begin{macrocode}
\include{cdocsch1}
\include{cdocsch2}
%    \end{macrocode}

% Include the two parts unless only chapters should be displayed:
%    \begin{macrocode}
\ifchilddoc\else
\section{part three}
\input{cdocspt3}
\section{part four}
\input{cdocspt4}
\fi
%    \end{macrocode}

% Process as usual until here:
%    \begin{macrocode}
\fi
%    \end{macrocode}

% End of document body:
%    \begin{macrocode}
\end{document}
%    \end{macrocode}
%\iffalse
%</samplemain>
%\fi
%
% %%%%%%%%%%%%%%%%%%%%%%%%%%%%%%%%%%%%%%
% \paragraph{Chapter Include Files.}
%
% The include files are called |cdocsch1.tex| and |cdocsch2.tex|.
%
%\iffalse
%<*samplechap1|samplechap2>
%\fi

% Optional override for |\version| flag:
%    \begin{macrocode}
%%\providecommand{\version}{final}
%    \end{macrocode}

% Include the main document:
%    \begin{macrocode}
\input{childdoc.def}
\childdocof{cdocsamp}
%    \end{macrocode}

%\iffalse
%</samplechap1|samplechap2>
%\fi
%
%\iffalse
%<*samplechap1>
%\fi
% Some text for chapter 1:
%    \begin{macrocode}
\section{one}
some text in chapter one
%    \end{macrocode}

%\iffalse
%</samplechap1>
%\fi
% Some text for chapter 2:
%\iffalse
%<*samplechap2>
%\fi
%    \begin{macrocode}
\section{two}
more text in chapter two
%    \end{macrocode}

%\iffalse
%</samplechap2>
%\fi
%
% %%%%%%%%%%%%%%%%%%%%%%%%%%%%%%%%%%%%%%
% \paragraph{Part Include Files.}
%
% The include files are called |cdocspt3.tex| and |cdocspt4.tex|.
%
%\iffalse
%<*samplepart3|samplepart4>
%\fi

% Optional override for |\version| flag:
%    \begin{macrocode}
%%\providecommand{\version}{final}
%    \end{macrocode}

% Include the main document:
%    \begin{macrocode}
\input{childdoc.def}
\childdocby{cdocsamp}
%    \end{macrocode}

%\iffalse
%</samplepart3|samplepart4>
%\fi
%
%\iffalse
%<*samplepart3>
%\fi
% Some text for part 3:
%    \begin{macrocode}
some text in part three
%    \end{macrocode}

%\iffalse
%</samplepart3>
%\fi
% Some text for part 4:
%\iffalse
%<*samplepart4>
%\fi
%    \begin{macrocode}
more text in part four
%    \end{macrocode}

%\iffalse
%</samplepart4>
%\fi
%
% %%%%%%%%%%%%%%%%%%%%%%%%%%%%%%%%%%%%%%
% \paragraph{Forwarding for a Complete Draft.}
%
% The following forwarding file |cdocsdrf.tex|
% compiles the main document in draft mode:
%\iffalse
%<*sampledraft>
%\fi
%    \begin{macrocode}
\def\version{draft}
\input{childdoc.def}
\childdocforward{cdocsamp}
%    \end{macrocode}

%\iffalse
%</sampledraft>
%\fi
%
% %%%%%%%%%%%%%%%%%%%%%%%%%%%%%%%%%%%%%%
% \paragraph{Forwarding for Final Version of the Chapters.}
%
% The following forwarding files |cdocsfn1.tex| and |cdocsfn2.tex|
% (with identical content)
% compile the final versions of the child documents
% |cdocsch1.tex| and |cdocsch2.tex|, respectively:
%\iffalse
%<*samplefinal>
%\fi
%    \begin{macrocode}
\def\version{final}
\input{childdoc.def}
\childdocforwardprefix[cdocsamp]{cdocsfn}{cdocsch}
%    \end{macrocode}

%\iffalse
%</samplefinal>
%\fi
%
% %%%%%%%%%%%%%%%%%%%%%%%%%%%%%%%%%%%%%%
% \paragraph{Command Line Processing.}
%
% The following three command lines generate the output files
% |cdocscld|, |cdocscl1| and |cdocscl2|
% which should be identical to
% |cdocsdrf|, |cdocsch1| and |cdocsfn2|, respectively:
% \begin{center}
% \begin{tabular}{l}
% |latex -jobname cdocscld \|\\
% |  "\def\version{draft}\input{childdoc.def}\childdocforward{cdocsamp}"|\\
% |latex -jobname cdocscl1 \|\\
% |  "\input{childdoc.def}\childdocforward[cdocsamp]{cdocsch1}"|\\
% |latex -jobname cdocscl2 \|\\
% |  "\def\version{final}\input{childdoc.def}\childdocforward{cdocsch2}"|
% \end{tabular}
% \end{center}
% Note that the trailing backslash on each first line
% merely continues the input to the second line
% (for convenient cut ant paste).
% Furthermore, the command |latex| can be replaced by any
% of its alternative versions such as |pdflatex|.
%
% %%%%%%%%%%%%%%%%%%%%%%%%%%%%%%%%%%%%%%%%%%%%%%%%%%%%%%%%%%%%%%%%%%%%%%%%%%%%%%
% %%%%%%%%%%%%%%%%%%%%%%%%%%%%%%%%%%%%%%%%%%%%%%%%%%%%%%%%%%%%%%%%%%%%%%%%%%%%%%
% \section{Implementation}
%\iffalse
%<*package>
%\fi
%
% This section describes the definitions file |childdoc.def|.

% The definitions cannot be loaded using |\usepackage| or |\RequirePackage|
% which has a mechanism to prevent loading a style file more than once.
% When loading the definitions by means of |\input|
% multiple instances have to be prevented manually:
%\iffalse
%This code needs to be before the `\ProvidesFile' directive
%which is defined at the beginning of this file.
%Therefore it is also placed there and commented out here.
%</package>
%<*discard>
%\fi
%    \begin{macrocode}
\ifdefined\childdocmain\endinput\fi
%    \end{macrocode}
%\iffalse
%</discard>
%<*package>
%\fi
%
% \macro{\ifchilddoc}
% \macro{\ifchilddocmanual}
% The conditional |\ifchilddoc| tells whether a
% child (true) or main (false) document is being compiled.
% The conditional |\ifchilddocmanual| tells whether
% the |\includeonly| mechanism is used (false) or
% the selection of child files must be performed manually (true).
% The definitions initialise to false:
%    \begin{macrocode}
\newif\ifchilddoc
\newif\ifchilddocmanual
%    \end{macrocode}

% \macro{\childdocname}
% \macro{\childdocjob}
% The macro |\childdocname| stores the name of the main document
% to be compiled. The macro |\childdocjob| stores the name of
% the document on which the \LaTeX{} compiler was originally invoked.
% The content of |\jobname| cannot be compared
% to filenames specified in the source due to different catcodes.
% The following code rescans |\jobname|, stores the result
% in |\childdocname| and saves a copy in |\childdocjob|:
%    \begin{macrocode}
\edef\childdocname{\scantokens\expandafter{\jobname\noexpand}}
\let\childdocjob\childdocname
%    \end{macrocode}

% \macro{\childdocdisable}
% The macro |\childdocdisable| prevents the main file
% from being processed more than once.
% At this stage, the main document command |\childdocmain|
% is assumed to be called once again where it should do nothing.
% Any subsequent call to it should prevent
% a secondary processing of the main document
% It overwrites the forwarding commands
% |\childdocof| and |\childdocforward|
% with empty macros to prevent further inclusions of the main document:
%    \begin{macrocode}
\newcommand{\childdocdisable}
{
  \renewcommand{\childdocmain}[1]{\renewcommand{\childdocmain}[1]{\endinput}}
  \renewcommand{\childdocof}[1]{}
  \renewcommand{\childdocby}[2][]{}
  \renewcommand{\childdocforward}[2][]{}
  \renewcommand{\childdocdisable}{}
}
%    \end{macrocode}

% \macro{\childdocmain}
% The macro |\childdocmain| is to be called at the top of the main file
% with nothing or the main filename (without extension) as argument.
% First, it breaks loops.
% If the argument is not empty and does not match |\childdocname|
% (which is set by the first inclusion of |childdoc.def|),
% |\ifchilddoc| is set to true, |\includeonly| is applied to the child file
% and |\jobname| is set to the main file
% (for proper handling of |.aux| files):
%    \begin{macrocode}
\newcommand{\childdocmain}[1]
{
  \childdocdisable\childdocmain{}
  \if?#1?\else
    \begingroup
      \def\childdoctmp{#1}
      \ifx\childdoctmp\childdocname
        \def\childdoctmp{}
      \else
        \def\childdoctmp
        {
          \childdoctrue
          \includeonly{\childdocname}
          \def\childdocjob{#1}
          \def\jobname{#1}
        }
      \fi
      \expandafter
    \endgroup
    \childdoctmp
  \fi
}
%    \end{macrocode}

% \macro{\childdocof}
% The command |\childdocof| redirects
% compilation to the main file |#1|.
%    \begin{macrocode}
\newcommand{\childdocof}[1]
{
  \childdocdisable
  \childdoctrue
  \includeonly{\childdocname}
  \def\jobname{#1}
  \def\childdocjob{#1}
  \input{#1}
}
%    \end{macrocode}

% \macro{\childdocby}
% The command |\childdocby| ....
%    \begin{macrocode}
\newcommand{\childdocby}[2][]
{
  \childdocdisable
  \childdoctrue
  \childdocmanualtrue
  \if?#1?\else
    \def\jobname{#2}
  \fi
  \def\childdocjob{#2}
  \input{#2}
  \endinput
}
%    \end{macrocode}

% \macro{\childdocforward}
% The command |\childdocforward| redirects
% compilation to the main file or
% (if the optional argument is given) a child file.
% Parameters are set as if the main file
% or a child file starting with |\childdocof| was compiled.
% Then compilation is handed over to the main file:
%    \begin{macrocode}
\newcommand{\childdocforward}[2][]
{
  \begingroup
    \if?#1?
      \def\childdoctmp
      {
        \def\childdocname{#2}
        \def\childdocjob{#2}
        \def\jobname{#2}
        \input{#2}
        \endinput
      }
    \else
      \def\childdoctmp
      {
        \childdocdisable
        \def\childdocname{#2}
        \childdoctrue
        \includeonly{#2}
        \def\childdocjob{#1}
        \def\jobname{#1}
        \input{#1}
        \endinput
      }
    \fi
    \expandafter
  \endgroup
  \childdoctmp
}
%    \end{macrocode}

% \macro{\childdocforwardprefix}
% The command |\childdocforwardprefix| redirects
% compilation to the main or a child file by means of a pattern.
% The prefix |#1| in the current filename is replaced by |#2|
% and the suffix of the current filename is kept
% (it is assumed that the filename does not contain the substring `|~~~|'
% which is used as a delimiter).
% Compilation is handed over to the new file by |\childdocforward|:
%    \begin{macrocode}
\newcommand{\childdocforwardprefix}[3][]
{
  \begingroup
    \def\childdocextract #2##1~~~{\def\childdoctmp{\childdocforward[#1]{#3##1}}}
    \expandafter\childdocextract\childdocname~~~
    \expandafter
  \endgroup
  \childdoctmp
}
%    \end{macrocode}

% \macro{\childdoc}
% The deprecated macro |\childdoc| is a legacy version of |\childdocmain|:
%    \begin{macrocode}
\newcommand{\childdoc}{\childdocmain}
%    \end{macrocode}

% \macro{\childdocredirect}
% The deprecated macro |\childdocredirect| is a legacy version
% of |\childdocforward| and |\childdocforwardprefix|:
%    \begin{macrocode}
\newcommand{\childdocredirect}[2][]
{
  \begingroup
    \if?#1?
      \def\childdoctmp{\childdocforward{#2}}
    \else
      \def\childdoctmp{\childdocforwardprefix{#1}{#2}}
    \fi
    \expandafter
  \endgroup
  \childdoctmp
}
%    \end{macrocode}

%\iffalse
%</package>
%\fi
%
\endinput
\childdocforward{cdocsamp}"|\\
% |latex -jobname cdocscl1 \|\\
% |  "% \iffalse
%
% childdoc.dtx Copyright (C) 2017-2018 Niklas Beisert
%
% This work may be distributed and/or modified under the
% conditions of the LaTeX Project Public License, either version 1.3
% of this license or (at your option) any later version.
% The latest version of this license is in
%   http://www.latex-project.org/lppl.txt
% and version 1.3 or later is part of all distributions of LaTeX
% version 2005/12/01 or later.
%
% This work has the LPPL maintenance status `maintained'.
%
% The Current Maintainer of this work is Niklas Beisert.
%
% This work consists of the files childdoc.dtx and childdoc.ins
% and the derived files childdoc.def and cdocsamp.tex with
% cdocsch1.tex, cdocsch2.tex, cdocsdrf.tex, cdocsfn1.tex, cdocsfn2.tex.
%
%<package>\ifdefined\childdocmain\endinput\fi
%<package>\ProvidesFile{childdoc.def}[2018/12/30 v2.0 child document driver]
%<samplemain>\ProvidesFile{cdocsamp.tex}[2018/12/30 v2.0 sample for childdoc]
%<*driver>
%\ProvidesFile{childdoc.drv}[2018/12/30 v2.0 childdoc reference manual file]
\PassOptionsToClass{10pt,a4paper}{article}
\documentclass{ltxdoc}

\usepackage[margin=35mm]{geometry}
\usepackage{hyperref}
\usepackage{hyperxmp}
\usepackage[usenames]{color}

\hypersetup{colorlinks=true}
\hypersetup{pdfstartview=FitH}
\hypersetup{pdfpagemode=UseNone}
\hypersetup{pdfsource={}}
\hypersetup{pdflang={en-UK}}
\hypersetup{pdfcopyright={Copyright 2017-2018 Niklas Beisert.
  This work may be distributed and/or modified under the
  conditions of the LaTeX Project Public License, either version 1.3
  of this license or (at your option) any later version.}}
\hypersetup{pdflicenseurl={http://www.latex-project.org/lppl.txt}}
\hypersetup{pdfcontactaddress={ETH Zurich, ITP, HIT K,
  Wolfgang-Pauli-Strasse 27}}
\hypersetup{pdfcontactpostcode={8093}}
\hypersetup{pdfcontactcity={Zurich}}
\hypersetup{pdfcontactcountry={Switzerland}}
\hypersetup{pdfcontactemail={nbeisert@itp.phys.ethz.ch}}
\hypersetup{pdfcontacturl={http://people.phys.ethz.ch/\xmptilde nbeisert/}}

\newcommand{\secref}[1]{\hyperref[#1]{section \ref*{#1}}}

\parskip1ex
\parindent0pt
\let\olditemize\itemize
\def\itemize{\olditemize\parskip0pt}

\begin{document}

\title{The \textsf{childdoc} Package}
\hypersetup{pdftitle={The childdoc Package}}
\author{Niklas Beisert\\[2ex]
  Institut f\"ur Theoretische Physik\\
  Eidgen\"ossische Technische Hochschule Z\"urich\\
  Wolfgang-Pauli-Strasse 27, 8093 Z\"urich, Switzerland\\[1ex]
  \href{mailto:nbeisert@itp.phys.ethz.ch}
  {\texttt{nbeisert@itp.phys.ethz.ch}}}
\hypersetup{pdfauthor={Niklas Beisert}}
\hypersetup{pdfsubject={Manual for the LaTeX2e Package childdoc}}
\date{30 December 2018, \textsf{v2.0}}
\maketitle

\begin{abstract}\noindent
\textsf{childdoc} is a \LaTeXe{} package
that enables the direct compilation
of document sections included by |\include|
to individual files.
\end{abstract}

\begingroup
\parskip0ex
\tableofcontents
\endgroup

%%%%%%%%%%%%%%%%%%%%%%%%%%%%%%%%%%%%%%%%%%%%%%%%%%%%%%%%%%%%%%%%%%%%%%%%%%%%%%%%
%%%%%%%%%%%%%%%%%%%%%%%%%%%%%%%%%%%%%%%%%%%%%%%%%%%%%%%%%%%%%%%%%%%%%%%%%%%%%%%%
\section{Introduction}

\LaTeX{} provides a mechanism to structure a large document (such as a book)
into a main file and several child files (containing the chapters)
using the |\include| command.
This mechanism is beneficial for documents
which span hundreds of pages in order to
make the source file(s) more manageable.
Moreover, compilation can be restricted to
selected child files by means of the |\includeonly| command.
The latter feature can be used to reduce the compilation time while editing
(this was significantly more useful in the earlier days of \LaTeX{})
or to generate a smaller document which is easier to navigate.
Another application of |\includeonly| is to generate
documents consisting of selected parts of the complete document.

However, there are a few drawbacks of the plain |\include| mechanism:
\begin{itemize}
\item
The child files cannot be compiled on their own,
they can only be compiled via the main file.
A naive editing environment
(such as a text editor with an option
to have the current file processed by \LaTeX)
may require one to switch to the main file before compiling;
attempting to compile the child file produces errors.
\item
The main file must be modified (each time)
to adjust the |\includeonly| command
to the present needs. This easily leaves the main file in a messy state.
\item
The generated document will always carry the filename
of the main document. This is inconvenient if
several child files are to be compiled and
to be kept for distribution.
\end{itemize}

The present package provides a simple interface
to make child files individually compilable by \LaTeX{}.
Compiling a child file then has the same effect as compiling
the main file with an |\includeonly| command
to select the appropriate child.
Moreover the generated document will carry the name of the child
rather than the main file.
This resolves all three above issues.

This feature is meant to make the editing of books,
thesis documents and lecture notes somewhat more convenient.
However, the package can also be used efficiently for
composing a series of documents (such as exercise sheets)
which are typically distributed individually.
It then assists the author in generating the individual documents
(potentially in different versions)
as well as a document containing the collected series.
Another application is in developing style files
or other kinds of included material
where compilation of the style file could redirect
to a sample or test file.

%%%%%%%%%%%%%%%%%%%%%%%%%%%%%%%%%%%%%%%%%%%%%%%%%%%%%%%%%%%%%%%%%%%%%%%%%%%%%%%%
%%%%%%%%%%%%%%%%%%%%%%%%%%%%%%%%%%%%%%%%%%%%%%%%%%%%%%%%%%%%%%%%%%%%%%%%%%%%%%%%
\section{Usage}

First of all, the package \textsf{childdoc} is \emph{not} a standard
\LaTeXe{} |.sty| style file! Therefore it needs to be invoked in
a non-standard way.

%%%%%%%%%%%%%%%%%%%%%%%%%%%%%%%%%%%%%%%%%%%%%%%%%%%%%%%%%%%%%%%%%%%%%%%%%%%%%%%%
\subsection{Included Files}
\label{sec:include}

%%%%%%%%%%%%%%%%%%%%%%%%%%%%%%%%%%%%%%%%
\DescribeMacro{\childdocmain}
To use the package, add the commands
\begin{center}
\begin{tabular}{l}
|\input{childdoc.def}|\\
|\childdocmain{}|\\
\end{tabular}
\end{center}
at the very top of the main \LaTeX{} file,
in particular \emph{before} the |\documentclass| statement!
The argument of |\childdocmain| should be left empty
(but it must be present).

%%%%%%%%%%%%%%%%%%%%%%%%%%%%%%%%%%%%%%%%
\DescribeMacro{\childdocof}
Furthermore, add the commands
\begin{center}
\begin{tabular}{l}
|\input{childdoc.def}|\\
|\childdocof{|\textit{main}|}|\\
\end{tabular}
\end{center}
at the top of every child file \textit{child}
which is included by |\include{|\textit{child}|}|
from within the main file
(or at least for those files to be compiled individually).
The argument \textit{main} must be the filename of the main file.

There are a couple of
considerations in setting up the main and child documents:

%%%%%%%%%%%%%%%%%%%%%%%%%%%%%%%%%%%%%%%%
\paragraph{Restrictions.}

Please note the following restrictions:
\begin{itemize}
\item
|\childdocmain| must be called with one argument \textit{main}
to ensure compatibility with earlier version of the package.
It must either be empty (|\childdocmain{}|)
or precisely match the filename of the main file in which it is specified.
See \secref{sec:detection} for further information.
\item
The filename \textit{main} must be specified without the |.tex| extension.
\item
The filename \textit{main} is case sensitive
(even in case-insensitive file systems)
due to internal string comparison.
\item
The argument \textit{main} should be fully expanded, it cannot be a macro.
\item
Subdirectories and special characters should be avoided in filenames.
\item
The command |\childdocmain{|\textit{main}|}| must be followed by a whitespace.
It should not be followed immediately by another command
or by a comment mark `|%|'.
This is because the \TeX{} parser reads the token immediately following
the argument of |\childdocmain| and puts it
at the beginning of every child section;
however, a white\-space is ignored.
\end{itemize}

%%%%%%%%%%%%%%%%%%%%%%%%%%%%%%%%%%%%%%%%
\paragraph{Content of Main File.}

It is advisable to place all content in the child files included by |\include|.
Any output contained in the main file will appear in all child documents
unless suppressed manually;
it cannot be suppressed automatically by the |\includeonly| directive
and thus should normally be avoided.
A method to include some content in the main file
by means of conditional processing is described in \secref{sec:conditional}.

%%%%%%%%%%%%%%%%%%%%%%%%%%%%%%%%%%%%%%%%
\paragraph{Page Numbering.}

When only a part of the document is compiled,
the appropriate numbering of pages
(as well as other status parameters)
is determined from the |.aux| files.
The latter contain information from previous passes.
However this information needs to propagate through
all intermediate child documents.
Therefore the page numbering in child documents may well
be inconsistent until the complete document is compiled at least once.

A useful (if unconventional) way to always ensure a consistent
page numbering is to restart the numbering in each child document
and denote the pages by `\textit{child}|.|\textit{page}'
where \textit{child} represents the chapter/section number of the child file.
This can be achieved by the command
|\numberwithin{page}{|\textit{child}|}|
of the \textsf{amsmath} package
where \textit{child} can be |chapter| or |section|
depending on the chosen structuring.
Alternatively, one can modify the macro |\thepage| appropriately
and reset the counter |page| at the start of each child file.

%%%%%%%%%%%%%%%%%%%%%%%%%%%%%%%%%%%%%%%%%%%%%%%%%%%%%%%%%%%%%%%%%%%%%%%%%%%%%%%%
\subsection{Conditional Processing}
\label{sec:conditional}

The package provides a mechanism to compile different versions
of a document. To customise the versions further some conditional processing
can come in handy to distinguish which version is being compiled.
The package provides two macros to describe the compilation context:

%%%%%%%%%%%%%%%%%%%%%%%%%%%%%%%%%%%%%%%%
\DescribeMacro{\ifchilddoc}
The conditional |\ifchilddoc| distinguishes between the compilation of
child documents and the main document:
%
\begin{center}
|\ifchilddoc |\textit{child-code}| |[|\||else |\textit{main-code}]| \||fi|
\end{center}

%%%%%%%%%%%%%%%%%%%%%%%%%%%%%%%%%%%%%%%%
\DescribeMacro{\childdocname}
\DescribeMacro{\childdocjob}
The macro |\childdocname| contains the filename (without extension)
of the main or child file being processed.
Note that |\childdocjob| will always contain the name of the main file.

%%%%%%%%%%%%%%%%%%%%%%%%%%%%%%%%%%%%%%%%
\paragraph{Title Page.}

Conditional processing can be used to include a title or banner page
in the main document when proper precautions are taken.
Importantly, the code in the main file should ensure that the page counter
(as well as other status parameters which are stored in the |.aux| files)
takes the same value after the conditional processing.
Otherwise the page numbers may take divergent values
depending on which part is compiled.

For example, a title page could be declared by:
%
\begin{center}
\begin{tabular}{l}
|\ifchilddoc\||else|\\
|\addtocounter{page}{-1}|\\
\textit{code for title page}\\
|\newpage|\\
|\||fi|
\end{tabular}
\end{center}
%
A banner page for the child documents can be generated by:
%
\begin{center}
\begin{tabular}{l}
|\ifchilddoc|\\
|\addtocounter{page}{-1}|\\
\textit{code for banner page}\\
|\newpage|\\
|\||fi|
\end{tabular}
\end{center}
%
Here one could write a message such as:
\begin{center}
|This is the part \childdocname{} of \childdocjob{}.|
\end{center}

%%%%%%%%%%%%%%%%%%%%%%%%%%%%%%%%%%%%%%%%%%%%%%%%%%%%%%%%%%%%%%%%%%%%%%%%%%%%%%%%
\subsection{Flags}
\label{sec:flags}

The package makes it easy to generate different versions
of the main or child documents.
To this end compilation flags can be defined
and assigned different default values.
They will be particularly useful in conjunction
with the forwarding mechanism described in \secref{sec:forward}.

For example, it may be useful to have a flag |\version|
which can be set to |draft| or |final|.
The document source will contain some conditional code
depending on the value of |\version|.
Suppose further, the flag should default to |final| for the main file
and to |draft| for child files
which is a natural assignment for editing the document.
This is achieved by placing the following code
in the preamble of the main document
(below the |\childdocmain| directive):
%
\begin{center}
\begin{tabular}{l}
|\ifchilddoc|\\
|\providecommand{\version}{draft}|\\
|\||else|\\
|\providecommand{\version}{final}|\\
|\||fi|
\end{tabular}
\end{center}
%
The definition by |\providecommand| makes sure
that previous definitions are not overwritten.
Further statements |\providecommand{\version}{...}|
can thus be added before the above code to override it.

For the main file, one might add a line
(between |\childdocmain| and the above block)
%
\begin{center}
|%\ifchilddoc\||else\providecommand{\version}{draft}\||fi|
\end{center}
%
which can be uncommented to produce a draft version.
Likewise one can add a line to the very top of a child file
(above the |\childdocof{|\textit{main}|}| directive)
%
\begin{center}
|%\providecommand{\version}{final}|
\end{center}
%
which can be uncommented to produce the final version of this child document.

%%%%%%%%%%%%%%%%%%%%%%%%%%%%%%%%%%%%%%%%%%%%%%%%%%%%%%%%%%%%%%%%%%%%%%%%%%%%%%%%
\subsection{Forwarding}
\label{sec:forward}

Different versions of the main or child documents
using compilation flags as described in \secref{sec:flags}
can be (permanently) stored in different files
for convenient compilation, viewing and distribution.
To this end, the package defines a command
to pass on compilation to a different file:

%%%%%%%%%%%%%%%%%%%%%%%%%%%%%%%%%%%%%%%%
\DescribeMacro{\childdocforward}
The command |\childdocforward| redirects processing to
another source file:
%
\begin{center}
\begin{tabular}{l}
|\input{childdoc.def}|\\
|\childdocforward[|\textit{main}|]{|\textit{dest}|}|\\
\end{tabular}
\end{center}
%
The argument \textit{dest} is the destination file
(without extension).
It should be the main file or one of the child files.
Note that further \textsf{childdoc} directives
such as |\childdocof| and |\childdocforward|
in the indicated file will be processed in this form.
The optional argument \textit{main}
passes on directly to the main file \textit{main}
while pretending to compile the child \textit{dest}.
This form behaves as if \textit{dest}
issues |\childdocof{|\textit{main}|}| right away,
and no further \textsf{childdoc} directives will be processed.

%%%%%%%%%%%%%%%%%%%%%%%%%%%%%%%%%%%%%%%%
\DescribeMacro{\...prefix}
In the alternative form |\childdocforwardprefix|,
%
\begin{center}
\begin{tabular}{l}
|\input{childdoc.def}|\\
|\childdocforwardprefix[|\textit{main}|]{|\textit{prefix}|}{|\textit{dest}|}|
\end{tabular}
\end{center}
%
the destination file is determined by a pattern
depending on the current file:
To make this work, the current file must be called
`{\textit{prefix}\hspace{0.2em}\textit{suffix}}'
with \textit{prefix} matching precisely the argument.
Processing is then passed on to the file
`{\textit{dest}\hspace{0.2em}\textit{suffix}}'.
Surely, the same effect is achieved by
directly specifying the
argument `{\textit{dest}\hspace{0.2em}\textit{suffix}}'
in the first form.
However, that requires to set up a different file
for each child. With the alternative form of the command
all these files can have exactly the same content
which simplifies setting them up and maintaining them.

For example, the following file |draft.tex|
with a compilation flag |\version| as described in \secref{sec:flags}
compiles the main document as a draft:
%
\begin{center}
\begin{tabular}{l}
|\def\version{draft}|\\
|\input{childdoc.def}|\\
|\childdocforward{|\textit{main}|}|
\end{tabular}
\end{center}
%
Likewise, the following files |final|\textit{nn}|.tex|
compile the final version of the child document
|child|\textit{nn}|.tex|:
%
\begin{center}
\begin{tabular}{l}
|\def\version{final}|\\
|\input{childdoc.def}|\\
|\childdocforwardprefix{final}{child}|
\end{tabular}
\end{center}
%

Note that when several versions of a main file and/or of each child file
are to be generated, it may be convenient to set up a |Makefile| or
shell script to automatise the process.

%%%%%%%%%%%%%%%%%%%%%%%%%%%%%%%%%%%%%%%%%%%%%%%%%%%%%%%%%%%%%%%%%%%%%%%%%%%%%%%%
\subsection{Command Line Processing}
\label{sec:commandline}

The effect of redirection files can also be achieved by invoking
the \LaTeX{} compiler with a more elaborate command line.
Most conveniently this should be done as part
of a shell script or a |Makefile|.

When using \textsf{childdoc} in the main file, the following
command lines effectively perform a redirection
(note that depending on the shell being used,
backslashes may have to be doubled: `|\|' $\to$ `|\\|'):
%
\begin{center}
|... -jobname "|\textit{target}|" |\\|"|[\textit{flags}]%
|\input{childdoc.def}\childdocforward[|\textit{main}|]{|\textit{dest}|}"|
\end{center}
%
Here \textit{target} is the name of the output file,
\textit{main} is the name of the main file
and \textit{dest} is the name of the main or child file to be processed
(all filenames without extensions).
The optional argument \textit{main} can be omitted
if \textit{main} matches \textit{dest}.
Optionally, compilation \textit{flags} can be defined via |\def| commands.
This command line makes the \TeX{} engine believe
it is compiling the file \textit{target}
whose content is specified as the latter parameter.
The provided code then forwards the processing to
\textit{main} or \textit{dest} as described in \secref{sec:forward}.

%%%%%%%%%%%%%%%%%%%%%%%%%%%%%%%%%%%%%%%%%%%%%%%%%%%%%%%%%%%%%%%%%%%%%%%%%%%%%%%%
\subsection{Include by Input}
\label{sec:input}

Including child documents by |\include| has some restrictions by design.
Most notably, the content of a child document always occupies
its own set of pages; pages cannot be shared between child documents.
Usually, this behaviour makes perfect sense
because each child document contain an essential part of the document.
However, in some situations it may be desirable to compose
a document from a collection of parts
without having mandatory page breaks between then.
For this case, the package
provides a mechanism to include parts
by |\input| which can also be processed individually.
However, by construction this mechanism
requires manual handling of the content to be output.

%%%%%%%%%%%%%%%%%%%%%%%%%%%%%%%%%%%%%%%%
\DescribeMacro{\ifchilddocmanual}
The main file should be prepared as usual, see \secref{sec:include}.
However, the document body must make a distinction
between processing of an individual part and of the main document, e.g.:
%
\begin{center}
\begin{tabular}{l}
|\ifchilddocmanual|\\
|\input{\childdocname}|\\
|\||else|\\
\textit{document body with }|\input{|\textit{part}|}|\\
|\||fi|
\end{tabular}
\end{center}
%
The conditional |\ifchilddocmanual| is true whenever
a part to be included by |\input| is being compiled,
and the name of the part is stored in |\childdocname|.

%%%%%%%%%%%%%%%%%%%%%%%%%%%%%%%%%%%%%%%%
\DescribeMacro{\childdocby}
Each part to be included by |\input| should start with:
%
\begin{center}
\begin{tabular}{l}
|\input{childdoc.def}|\\
|\childdocby{|\textit{main}|}|\\
\end{tabular}
\end{center}
%
The directive |\childdocby| is similar to |\childdocof|
described in \secref{sec:include},
but the subsequent selection of content must be done manually.
To that end, both |\ifchilddoc| and |\ifchilddocmanual|
will be true upon processing of a part,
and the name of the part is stored in |\childdocname|.
Note that |\jobname| will be set to the filename of the current part
so that each part receives an individual |.aux| file
that does not interfere with the |.aux| file(s) of the main document.
This behaviour can be altered by the alternative form
|\childdocby[*]{|\textit{main}|}| (with a non-empty optional argument)
which uses the |.aux| file of the main document
by setting |\jobname| to \textit{main}.

%%%%%%%%%%%%%%%%%%%%%%%%%%%%%%%%%%%%%%%%%%%%%%%%%%%%%%%%%%%%%%%%%%%%%%%%%%%%%%%%
\subsection{Driver Development}
\label{sec:driver}

The \textsf{childdoc} mechanism can also be use for the development
of definition files such as \LaTeX{} styles or classes.
This case differs from the above setup with multiple parts
included by |\include| in that no |\includeonly| should be invoked.
This can be achieved by starting the include file
(before |\ProvidesPackage|) with:
%
\begin{center}
\begin{tabular}{l}
|\input{childdoc.def}|\\
|\childdocforward{|\textit{main}|}|\\
\end{tabular}
\end{center}
%
or alternatively with:
%
\begin{center}
\begin{tabular}{l}
|\input{childdoc.def}|\\
|\childdocby{|\textit{main}|}|\\
\end{tabular}
\end{center}
%
Both forms have slightly different effects as described above.
The main file is prepared as usual, see \secref{sec:include}.

%%%%%%%%%%%%%%%%%%%%%%%%%%%%%%%%%%%%%%%%%%%%%%%%%%%%%%%%%%%%%%%%%%%%%%%%%%%%%%%%
\subsection{Legacy Detection}
\label{sec:detection}

The directive |\childdocmain| in the main file can detect
whether the complete document or merely a child is to be compiled
even without using the directive |\childdocof|.
This method is deprecated because it is less robust
and there is no compelling reason to use it;
it is merely provided for backward compatibility
and it may be removed in future versions.

If the detection mechanism is to be used,
it is mandatory to correctly specify
the filename of the main file as the argument of |\childdocmain|:
%
\begin{center}
\begin{tabular}{l}
|\input{childdoc.def}|\\
|\childdocmain{|\textit{main}|}|\\
\end{tabular}
\end{center}
%
If |\jobname| does not match the argument \textit{main} of |\childdocmain|,
it is assumed that |\jobname| points to the child file to be compiled.
When using |\childdocmain| with the main file specified as argument,
it suffices to start a child file
with just |\input{|\textit{main}|}|
without loading of the package and using |\childdocof|.
If instead all processing is done
with the appropriate \textsf{childdoc} directives,
the argument of \textit{main} of |\childdocmain| can be empty.

An alternative version of the command line processing described
in \secref{sec:commandline} using the detection mechanism reads:
%
\begin{center}
|... -jobname "|\textit{target}|" "|[\textit{flags}]%
[|\def\jobname{|\textit{dest}|}|]|\input{|\textit{main}|}"|
\end{center}

%%%%%%%%%%%%%%%%%%%%%%%%%%%%%%%%%%%%%%%%%%%%%%%%%%%%%%%%%%%%%%%%%%%%%%%%%%%%%%%%
\subsection{Manual Code}
\label{sec:manual}

In case one cannot be certain whether the definitions file |childdoc.def|
is installed on the target \TeX{} distribution
and one prefers not to ship it,
it is conceivable to paste a few relevant commands into the sources.

To that end, drop all statements |\input{childdoc.def}|
and perform the replacements as outlined below.
Instead of |\childdocmain{|\textit{main}|}| add the following code
to the top of the main file:
%
\begin{center}
\begin{tabular}{l}
|\||ifdefined\childdocname\endinput\||fi\newif\ifchilddoc|\\
|\edef\childdocname{\scantokens\expandafter{\jobname\noexpand}}|\\
|\def\childdocmain{|\textit{main}|}\||ifx\childdocmain\childdocname\||else|\\
|\childdoctrue\includeonly{\childdocname}\let\jobname\childdocmain\||fi|\\
\end{tabular}
\end{center}
%
Instead of |\childdocof{|\textit{main}|}| just include the main file
at the top of each child file:
%
\begin{center}
|\input{|\textit{main}|}|
\end{center}
%
A simple redirection |\childdocforward{|\textit{dest}|}| is achieved by:
%
\begin{center}
|\def\jobname{|\textit{dest}|}\input{\jobname}|
\end{center}
%
The redirection with prefix
|\childdocforwardprefix[|\textit{prefix}|]{|\textit{dest}|}|
is accomplished by:
%
\begin{center}
\begin{tabular}{l}
|{\edef\jobname{\scantokens\expandafter{\jobname\noexpand}}|\\
|\def\redirectjob |\textit{prefix}|#1~~~{\gdef\jobname{|\textit{dest}|#1}}|\\
|\expandafter\redirectjob\jobname~~~}\input{\jobname}|
\end{tabular}
\end{center}

In an alternative approach,
child documents can be compiled by a specific command line
without additional code or specific definitions:
%
\begin{center}
|... -jobname "|\textit{target}|" "|[\textit{flags}]%
|\includeonly{|\textit{dest}|}\input{|\textit{main}|}"|
\end{center}
%

%%%%%%%%%%%%%%%%%%%%%%%%%%%%%%%%%%%%%%%%%%%%%%%%%%%%%%%%%%%%%%%%%%%%%%%%%%%%%%%%
%%%%%%%%%%%%%%%%%%%%%%%%%%%%%%%%%%%%%%%%%%%%%%%%%%%%%%%%%%%%%%%%%%%%%%%%%%%%%%%%
\section{Information}

%%%%%%%%%%%%%%%%%%%%%%%%%%%%%%%%%%%%%%%%%%%%%%%%%%%%%%%%%%%%%%%%%%%%%%%%%%%%%%%%
\subsection{Copyright}

Copyright \copyright{} 2017--2018 Niklas Beisert

This work may be distributed and/or modified under the
conditions of the \LaTeX{} Project Public License, either version 1.3
of this license or (at your option) any later version.
The latest version of this license is in
  \url{http://www.latex-project.org/lppl.txt}
and version 1.3 or later is part of all distributions of \LaTeX{}
version 2005/12/01 or later.

This work has the LPPL maintenance status `maintained'.

The Current Maintainer of this work is Niklas Beisert.

This work consists of the files |README.txt|, |childdoc.ins| and |childdoc.dtx|
as well as the derived files |childdoc.def|, |cdocsamp.tex|
with |cdocsch1.tex|, |cdocsch2.tex|, |cdocspt3.tex|, |cdocspt4.tex|,
|cdocsdrf.tex|, |cdocsfn1.tex|, |cdocsfn2.tex|
as well as |childdoc.pdf|.

%%%%%%%%%%%%%%%%%%%%%%%%%%%%%%%%%%%%%%%%%%%%%%%%%%%%%%%%%%%%%%%%%%%%%%%%%%%%%%%%
\subsection{Files and Installation}

The package consists of the files:
%
\begin{center}
\begin{tabular}{ll}
    |README.txt|   & readme file \\
    |childdoc.ins| & installation file \\
    |childdoc.dtx| & source file \\
    |childdoc.def| & definition file \\
    |cdocsamp.tex| & sample main file \\
    |cdocsch1.tex| & sample include file \\
    |cdocsch2.tex| & sample include file \\
    |cdocspt3.tex| & sample part file \\
    |cdocspt4.tex| & sample part file \\
    |cdocsdrf.tex| & sample redirection file \\
    |cdocsfn1.tex| & sample redirection file \\
    |cdocsfn2.tex| & sample redirection file \\
    |childdoc.pdf| & manual
\end{tabular}
\end{center}
%
The distribution consists of the files
|README.txt|, |childdoc.ins| and |childdoc.dtx|.
%
\begin{itemize}
\item
Run (pdf)\LaTeX{} on |childdoc.dtx|
to compile the manual |childdoc.pdf| (this file).
\item
Run \LaTeX{} on |childdoc.ins| to create the definitions file |childdoc.def|
and the sample |cdocsamp.tex| with include files
|cdocsch1.tex|, |cdocsch2.tex|, |cdocspt3.tex|, |cdocspt4.tex|,
|cdocsdrf.tex|, |cdocsfn1.tex|, |cdocsfn2.tex|.
Then copy the file |childdoc.def| to an appropriate directory of your \LaTeX{}
distribution, e.g.\ \textit{texmf-root}|/tex/latex/childdoc|.
\end{itemize}

%%%%%%%%%%%%%%%%%%%%%%%%%%%%%%%%%%%%%%%%%%%%%%%%%%%%%%%%%%%%%%%%%%%%%%%%%%%%%%%%
\subsection{Related CTAN Packages}

There are several other packages which offer a similar functionality:
%
\begin{itemize}
\item
The packages
\href{http://ctan.org/pkg/docmute}{\textsf{docmute}},
\href{http://ctan.org/pkg/includex}{\textsf{includex}} and
\href{http://ctan.org/pkg/standalone}{\textsf{standalone}}
provide commands to include only the document body of
a child file thus allowing both files to be compiled individually.
\item
The packages \href{http://ctan.org/pkg/subdocs}{\textsf{subdocs}}
and \href{http://ctan.org/pkg/subfiles}{\textsf{subfiles}}
provide structures in which the main and child documents can be
encapsulated and allowing them to be compiled individually.
The inclusion mechanism is different from the conventional |\include|.
\item
The package \href{http://ctan.org/pkg/combine}{\textsf{combine}}
is an elaborate solution to combine several documents into one.
\end{itemize}
%
See also the CTAN topic \href{http://ctan.org/topic/subdocs}{\textsf{subdocs}}
for further related packages.
The present package differs from the above solutions in that
a document structure constructed with the conventional |\include| mechanism
just needs two extra commands at the top of every file
such that all constituent files can be compiled individually.

%%%%%%%%%%%%%%%%%%%%%%%%%%%%%%%%%%%%%%%%%%%%%%%%%%%%%%%%%%%%%%%%%%%%%%%%%%%%%%%%
%\subsection{Feature Suggestions}
%
%The following is a list of features which may be useful for future
%versions of this package:
%%
%\begin{itemize}
%\item
%\ldots
%\end{itemize}

%%%%%%%%%%%%%%%%%%%%%%%%%%%%%%%%%%%%%%%%%%%%%%%%%%%%%%%%%%%%%%%%%%%%%%%%%%%%%%%%
\subsection{Revision History}

%%%%%%%%%%%%%%%%%%%%%%%%%%%%%%%%%%%%%%%%
\paragraph{v2.0:} 2018/12/30

\begin{itemize}
\item
immediate forward processing
\item
added |\childdocby| mechanism
\item
manual restructured
\end{itemize}

%%%%%%%%%%%%%%%%%%%%%%%%%%%%%%%%%%%%%%%%
\paragraph{v1.6:} 2018/01/17

\begin{itemize}
\item
application for development of include files
\item
corrections to manual
\end{itemize}

%%%%%%%%%%%%%%%%%%%%%%%%%%%%%%%%%%%%%%%%
\paragraph{v1.5:} 2017/05/21

\begin{itemize}
\item
more complete structuring introduced
\item
|\childdocof| introduced
\item
|\childdoc| renamed to |\childdocmain|
\item
|\childredirect| renamed to |\childdocforward| and |\childdocforwardprefix|
and functionality expanded
\end{itemize}

%%%%%%%%%%%%%%%%%%%%%%%%%%%%%%%%%%%%%%%%
\paragraph{v1.0:} 2017/04/27

\begin{itemize}
\item
manual and install package
\item
first version published on CTAN
\end{itemize}

%%%%%%%%%%%%%%%%%%%%%%%%%%%%%%%%%%%%%%%%
\paragraph{v0.6:} 2017/04/26

\begin{itemize}
\item
redirection mechanism added
\end{itemize}

%%%%%%%%%%%%%%%%%%%%%%%%%%%%%%%%%%%%%%%%
\paragraph{v0.5:} 2017/04/26

\begin{itemize}
\item
functionality in definition file
\end{itemize}


%%%%%%%%%%%%%%%%%%%%%%%%%%%%%%%%%%%%%%%%%%%%%%%%%%%%%%%%%%%%%%%%%%%%%%%%%%%%%%%%
%%%%%%%%%%%%%%%%%%%%%%%%%%%%%%%%%%%%%%%%%%%%%%%%%%%%%%%%%%%%%%%%%%%%%%%%%%%%%%%%
%%%%%%%%%%%%%%%%%%%%%%%%%%%%%%%%%%%%%%%%%%%%%%%%%%%%%%%%%%%%%%%%%%%%%%%%%%%%%%%%
\appendix

\settowidth\MacroIndent{\rmfamily\scriptsize 000\ }

 \DocInput{childdoc.dtx}

\end{document}
%</driver>
% \fi
%
% %%%%%%%%%%%%%%%%%%%%%%%%%%%%%%%%%%%%%%%%%%%%%%%%%%%%%%%%%%%%%%%%%%%%%%%%%%%%%%
% %%%%%%%%%%%%%%%%%%%%%%%%%%%%%%%%%%%%%%%%%%%%%%%%%%%%%%%%%%%%%%%%%%%%%%%%%%%%%%
% \section{Sample}
%\iffalse
%<*samplemain>
%\fi
%
% The following presents a sample document
% with two chapters, two parts, a title page,
% a compile flag as well as three forwarding files to set the flag.
% It consists of eight |.tex| files:
% \begin{center}
% \begin{tabular}{ll}
% |cdocsamp.tex|&main file\\
% |cdocsch1.tex|&include file for chapter 1\\
% |cdocsch2.tex|&include file for chapter 2\\
% |cdocspt3.tex|&include file for part 3\\
% |cdocspt4.tex|&include file for part 4\\
% |cdocsdrf.tex|&forwarding file for main file in draft mode\\
% |cdocsfi1.tex|&forwarding file for final version of chapter 1\\
% |cdocsfi2.tex|&forwarding file for final version of chapter 2\\
% \end{tabular}
% \end{center}
% Each of the eight files can be compiled directly by the \LaTeX{} compiler.
%
% %%%%%%%%%%%%%%%%%%%%%%%%%%%%%%%%%%%%%%
% \paragraph{Main File.}
%
% The main file is called |cdocsamp.tex|.
%
% Load the \textsf{childdoc} definitions and
% declare the filename for the main document:
%    \begin{macrocode}
\input{childdoc.def}
\childdocmain{}
%    \end{macrocode}

% Optional override for |\version| flag:
%    \begin{macrocode}
%%\ifchilddoc\else\providecommand{\version}{draft}\fi
%    \end{macrocode}

% Define the default values for the |\version| flag
% (|final| for the main file and |draft| for childs):
%    \begin{macrocode}
\ifchilddoc
\providecommand{\version}{draft}
\else
\providecommand{\version}{final}
\fi
%    \end{macrocode}

% Load the standard document class:
%    \begin{macrocode}
\documentclass[12pt]{article}
%    \end{macrocode}

% Start the document body:
%    \begin{macrocode}
\begin{document}
%    \end{macrocode}

% Declare a title page.
% Print title, part of document being processed and version flag:
%    \begin{macrocode}
\addtocounter{page}{-1}
\begin{center}
{\LARGE\bfseries{}childdoc example\par}
\vspace{1cm}
\ifchilddoc
\ifchilddocmanual part\else chapter\fi:
`\childdocname' of `\childdocjob'\par
\else
main document: `\childdocjob'\par
\fi
version: \version\par
\end{center}
\newpage
%    \end{macrocode}

% Manually include selected file,
% otherwise process as usual:
%    \begin{macrocode}
\ifchilddocmanual
\section*{part `\childdocname'}
\input{\childdocname}
\else
%    \end{macrocode}

% Include the two chapters:
%    \begin{macrocode}
\include{cdocsch1}
\include{cdocsch2}
%    \end{macrocode}

% Include the two parts unless only chapters should be displayed:
%    \begin{macrocode}
\ifchilddoc\else
\section{part three}
\input{cdocspt3}
\section{part four}
\input{cdocspt4}
\fi
%    \end{macrocode}

% Process as usual until here:
%    \begin{macrocode}
\fi
%    \end{macrocode}

% End of document body:
%    \begin{macrocode}
\end{document}
%    \end{macrocode}
%\iffalse
%</samplemain>
%\fi
%
% %%%%%%%%%%%%%%%%%%%%%%%%%%%%%%%%%%%%%%
% \paragraph{Chapter Include Files.}
%
% The include files are called |cdocsch1.tex| and |cdocsch2.tex|.
%
%\iffalse
%<*samplechap1|samplechap2>
%\fi

% Optional override for |\version| flag:
%    \begin{macrocode}
%%\providecommand{\version}{final}
%    \end{macrocode}

% Include the main document:
%    \begin{macrocode}
\input{childdoc.def}
\childdocof{cdocsamp}
%    \end{macrocode}

%\iffalse
%</samplechap1|samplechap2>
%\fi
%
%\iffalse
%<*samplechap1>
%\fi
% Some text for chapter 1:
%    \begin{macrocode}
\section{one}
some text in chapter one
%    \end{macrocode}

%\iffalse
%</samplechap1>
%\fi
% Some text for chapter 2:
%\iffalse
%<*samplechap2>
%\fi
%    \begin{macrocode}
\section{two}
more text in chapter two
%    \end{macrocode}

%\iffalse
%</samplechap2>
%\fi
%
% %%%%%%%%%%%%%%%%%%%%%%%%%%%%%%%%%%%%%%
% \paragraph{Part Include Files.}
%
% The include files are called |cdocspt3.tex| and |cdocspt4.tex|.
%
%\iffalse
%<*samplepart3|samplepart4>
%\fi

% Optional override for |\version| flag:
%    \begin{macrocode}
%%\providecommand{\version}{final}
%    \end{macrocode}

% Include the main document:
%    \begin{macrocode}
\input{childdoc.def}
\childdocby{cdocsamp}
%    \end{macrocode}

%\iffalse
%</samplepart3|samplepart4>
%\fi
%
%\iffalse
%<*samplepart3>
%\fi
% Some text for part 3:
%    \begin{macrocode}
some text in part three
%    \end{macrocode}

%\iffalse
%</samplepart3>
%\fi
% Some text for part 4:
%\iffalse
%<*samplepart4>
%\fi
%    \begin{macrocode}
more text in part four
%    \end{macrocode}

%\iffalse
%</samplepart4>
%\fi
%
% %%%%%%%%%%%%%%%%%%%%%%%%%%%%%%%%%%%%%%
% \paragraph{Forwarding for a Complete Draft.}
%
% The following forwarding file |cdocsdrf.tex|
% compiles the main document in draft mode:
%\iffalse
%<*sampledraft>
%\fi
%    \begin{macrocode}
\def\version{draft}
\input{childdoc.def}
\childdocforward{cdocsamp}
%    \end{macrocode}

%\iffalse
%</sampledraft>
%\fi
%
% %%%%%%%%%%%%%%%%%%%%%%%%%%%%%%%%%%%%%%
% \paragraph{Forwarding for Final Version of the Chapters.}
%
% The following forwarding files |cdocsfn1.tex| and |cdocsfn2.tex|
% (with identical content)
% compile the final versions of the child documents
% |cdocsch1.tex| and |cdocsch2.tex|, respectively:
%\iffalse
%<*samplefinal>
%\fi
%    \begin{macrocode}
\def\version{final}
\input{childdoc.def}
\childdocforwardprefix[cdocsamp]{cdocsfn}{cdocsch}
%    \end{macrocode}

%\iffalse
%</samplefinal>
%\fi
%
% %%%%%%%%%%%%%%%%%%%%%%%%%%%%%%%%%%%%%%
% \paragraph{Command Line Processing.}
%
% The following three command lines generate the output files
% |cdocscld|, |cdocscl1| and |cdocscl2|
% which should be identical to
% |cdocsdrf|, |cdocsch1| and |cdocsfn2|, respectively:
% \begin{center}
% \begin{tabular}{l}
% |latex -jobname cdocscld \|\\
% |  "\def\version{draft}\input{childdoc.def}\childdocforward{cdocsamp}"|\\
% |latex -jobname cdocscl1 \|\\
% |  "\input{childdoc.def}\childdocforward[cdocsamp]{cdocsch1}"|\\
% |latex -jobname cdocscl2 \|\\
% |  "\def\version{final}\input{childdoc.def}\childdocforward{cdocsch2}"|
% \end{tabular}
% \end{center}
% Note that the trailing backslash on each first line
% merely continues the input to the second line
% (for convenient cut ant paste).
% Furthermore, the command |latex| can be replaced by any
% of its alternative versions such as |pdflatex|.
%
% %%%%%%%%%%%%%%%%%%%%%%%%%%%%%%%%%%%%%%%%%%%%%%%%%%%%%%%%%%%%%%%%%%%%%%%%%%%%%%
% %%%%%%%%%%%%%%%%%%%%%%%%%%%%%%%%%%%%%%%%%%%%%%%%%%%%%%%%%%%%%%%%%%%%%%%%%%%%%%
% \section{Implementation}
%\iffalse
%<*package>
%\fi
%
% This section describes the definitions file |childdoc.def|.

% The definitions cannot be loaded using |\usepackage| or |\RequirePackage|
% which has a mechanism to prevent loading a style file more than once.
% When loading the definitions by means of |\input|
% multiple instances have to be prevented manually:
%\iffalse
%This code needs to be before the `\ProvidesFile' directive
%which is defined at the beginning of this file.
%Therefore it is also placed there and commented out here.
%</package>
%<*discard>
%\fi
%    \begin{macrocode}
\ifdefined\childdocmain\endinput\fi
%    \end{macrocode}
%\iffalse
%</discard>
%<*package>
%\fi
%
% \macro{\ifchilddoc}
% \macro{\ifchilddocmanual}
% The conditional |\ifchilddoc| tells whether a
% child (true) or main (false) document is being compiled.
% The conditional |\ifchilddocmanual| tells whether
% the |\includeonly| mechanism is used (false) or
% the selection of child files must be performed manually (true).
% The definitions initialise to false:
%    \begin{macrocode}
\newif\ifchilddoc
\newif\ifchilddocmanual
%    \end{macrocode}

% \macro{\childdocname}
% \macro{\childdocjob}
% The macro |\childdocname| stores the name of the main document
% to be compiled. The macro |\childdocjob| stores the name of
% the document on which the \LaTeX{} compiler was originally invoked.
% The content of |\jobname| cannot be compared
% to filenames specified in the source due to different catcodes.
% The following code rescans |\jobname|, stores the result
% in |\childdocname| and saves a copy in |\childdocjob|:
%    \begin{macrocode}
\edef\childdocname{\scantokens\expandafter{\jobname\noexpand}}
\let\childdocjob\childdocname
%    \end{macrocode}

% \macro{\childdocdisable}
% The macro |\childdocdisable| prevents the main file
% from being processed more than once.
% At this stage, the main document command |\childdocmain|
% is assumed to be called once again where it should do nothing.
% Any subsequent call to it should prevent
% a secondary processing of the main document
% It overwrites the forwarding commands
% |\childdocof| and |\childdocforward|
% with empty macros to prevent further inclusions of the main document:
%    \begin{macrocode}
\newcommand{\childdocdisable}
{
  \renewcommand{\childdocmain}[1]{\renewcommand{\childdocmain}[1]{\endinput}}
  \renewcommand{\childdocof}[1]{}
  \renewcommand{\childdocby}[2][]{}
  \renewcommand{\childdocforward}[2][]{}
  \renewcommand{\childdocdisable}{}
}
%    \end{macrocode}

% \macro{\childdocmain}
% The macro |\childdocmain| is to be called at the top of the main file
% with nothing or the main filename (without extension) as argument.
% First, it breaks loops.
% If the argument is not empty and does not match |\childdocname|
% (which is set by the first inclusion of |childdoc.def|),
% |\ifchilddoc| is set to true, |\includeonly| is applied to the child file
% and |\jobname| is set to the main file
% (for proper handling of |.aux| files):
%    \begin{macrocode}
\newcommand{\childdocmain}[1]
{
  \childdocdisable\childdocmain{}
  \if?#1?\else
    \begingroup
      \def\childdoctmp{#1}
      \ifx\childdoctmp\childdocname
        \def\childdoctmp{}
      \else
        \def\childdoctmp
        {
          \childdoctrue
          \includeonly{\childdocname}
          \def\childdocjob{#1}
          \def\jobname{#1}
        }
      \fi
      \expandafter
    \endgroup
    \childdoctmp
  \fi
}
%    \end{macrocode}

% \macro{\childdocof}
% The command |\childdocof| redirects
% compilation to the main file |#1|.
%    \begin{macrocode}
\newcommand{\childdocof}[1]
{
  \childdocdisable
  \childdoctrue
  \includeonly{\childdocname}
  \def\jobname{#1}
  \def\childdocjob{#1}
  \input{#1}
}
%    \end{macrocode}

% \macro{\childdocby}
% The command |\childdocby| ....
%    \begin{macrocode}
\newcommand{\childdocby}[2][]
{
  \childdocdisable
  \childdoctrue
  \childdocmanualtrue
  \if?#1?\else
    \def\jobname{#2}
  \fi
  \def\childdocjob{#2}
  \input{#2}
  \endinput
}
%    \end{macrocode}

% \macro{\childdocforward}
% The command |\childdocforward| redirects
% compilation to the main file or
% (if the optional argument is given) a child file.
% Parameters are set as if the main file
% or a child file starting with |\childdocof| was compiled.
% Then compilation is handed over to the main file:
%    \begin{macrocode}
\newcommand{\childdocforward}[2][]
{
  \begingroup
    \if?#1?
      \def\childdoctmp
      {
        \def\childdocname{#2}
        \def\childdocjob{#2}
        \def\jobname{#2}
        \input{#2}
        \endinput
      }
    \else
      \def\childdoctmp
      {
        \childdocdisable
        \def\childdocname{#2}
        \childdoctrue
        \includeonly{#2}
        \def\childdocjob{#1}
        \def\jobname{#1}
        \input{#1}
        \endinput
      }
    \fi
    \expandafter
  \endgroup
  \childdoctmp
}
%    \end{macrocode}

% \macro{\childdocforwardprefix}
% The command |\childdocforwardprefix| redirects
% compilation to the main or a child file by means of a pattern.
% The prefix |#1| in the current filename is replaced by |#2|
% and the suffix of the current filename is kept
% (it is assumed that the filename does not contain the substring `|~~~|'
% which is used as a delimiter).
% Compilation is handed over to the new file by |\childdocforward|:
%    \begin{macrocode}
\newcommand{\childdocforwardprefix}[3][]
{
  \begingroup
    \def\childdocextract #2##1~~~{\def\childdoctmp{\childdocforward[#1]{#3##1}}}
    \expandafter\childdocextract\childdocname~~~
    \expandafter
  \endgroup
  \childdoctmp
}
%    \end{macrocode}

% \macro{\childdoc}
% The deprecated macro |\childdoc| is a legacy version of |\childdocmain|:
%    \begin{macrocode}
\newcommand{\childdoc}{\childdocmain}
%    \end{macrocode}

% \macro{\childdocredirect}
% The deprecated macro |\childdocredirect| is a legacy version
% of |\childdocforward| and |\childdocforwardprefix|:
%    \begin{macrocode}
\newcommand{\childdocredirect}[2][]
{
  \begingroup
    \if?#1?
      \def\childdoctmp{\childdocforward{#2}}
    \else
      \def\childdoctmp{\childdocforwardprefix{#1}{#2}}
    \fi
    \expandafter
  \endgroup
  \childdoctmp
}
%    \end{macrocode}

%\iffalse
%</package>
%\fi
%
\endinput
\childdocforward[cdocsamp]{cdocsch1}"|\\
% |latex -jobname cdocscl2 \|\\
% |  "\def\version{final}% \iffalse
%
% childdoc.dtx Copyright (C) 2017-2018 Niklas Beisert
%
% This work may be distributed and/or modified under the
% conditions of the LaTeX Project Public License, either version 1.3
% of this license or (at your option) any later version.
% The latest version of this license is in
%   http://www.latex-project.org/lppl.txt
% and version 1.3 or later is part of all distributions of LaTeX
% version 2005/12/01 or later.
%
% This work has the LPPL maintenance status `maintained'.
%
% The Current Maintainer of this work is Niklas Beisert.
%
% This work consists of the files childdoc.dtx and childdoc.ins
% and the derived files childdoc.def and cdocsamp.tex with
% cdocsch1.tex, cdocsch2.tex, cdocsdrf.tex, cdocsfn1.tex, cdocsfn2.tex.
%
%<package>\ifdefined\childdocmain\endinput\fi
%<package>\ProvidesFile{childdoc.def}[2018/12/30 v2.0 child document driver]
%<samplemain>\ProvidesFile{cdocsamp.tex}[2018/12/30 v2.0 sample for childdoc]
%<*driver>
%\ProvidesFile{childdoc.drv}[2018/12/30 v2.0 childdoc reference manual file]
\PassOptionsToClass{10pt,a4paper}{article}
\documentclass{ltxdoc}

\usepackage[margin=35mm]{geometry}
\usepackage{hyperref}
\usepackage{hyperxmp}
\usepackage[usenames]{color}

\hypersetup{colorlinks=true}
\hypersetup{pdfstartview=FitH}
\hypersetup{pdfpagemode=UseNone}
\hypersetup{pdfsource={}}
\hypersetup{pdflang={en-UK}}
\hypersetup{pdfcopyright={Copyright 2017-2018 Niklas Beisert.
  This work may be distributed and/or modified under the
  conditions of the LaTeX Project Public License, either version 1.3
  of this license or (at your option) any later version.}}
\hypersetup{pdflicenseurl={http://www.latex-project.org/lppl.txt}}
\hypersetup{pdfcontactaddress={ETH Zurich, ITP, HIT K,
  Wolfgang-Pauli-Strasse 27}}
\hypersetup{pdfcontactpostcode={8093}}
\hypersetup{pdfcontactcity={Zurich}}
\hypersetup{pdfcontactcountry={Switzerland}}
\hypersetup{pdfcontactemail={nbeisert@itp.phys.ethz.ch}}
\hypersetup{pdfcontacturl={http://people.phys.ethz.ch/\xmptilde nbeisert/}}

\newcommand{\secref}[1]{\hyperref[#1]{section \ref*{#1}}}

\parskip1ex
\parindent0pt
\let\olditemize\itemize
\def\itemize{\olditemize\parskip0pt}

\begin{document}

\title{The \textsf{childdoc} Package}
\hypersetup{pdftitle={The childdoc Package}}
\author{Niklas Beisert\\[2ex]
  Institut f\"ur Theoretische Physik\\
  Eidgen\"ossische Technische Hochschule Z\"urich\\
  Wolfgang-Pauli-Strasse 27, 8093 Z\"urich, Switzerland\\[1ex]
  \href{mailto:nbeisert@itp.phys.ethz.ch}
  {\texttt{nbeisert@itp.phys.ethz.ch}}}
\hypersetup{pdfauthor={Niklas Beisert}}
\hypersetup{pdfsubject={Manual for the LaTeX2e Package childdoc}}
\date{30 December 2018, \textsf{v2.0}}
\maketitle

\begin{abstract}\noindent
\textsf{childdoc} is a \LaTeXe{} package
that enables the direct compilation
of document sections included by |\include|
to individual files.
\end{abstract}

\begingroup
\parskip0ex
\tableofcontents
\endgroup

%%%%%%%%%%%%%%%%%%%%%%%%%%%%%%%%%%%%%%%%%%%%%%%%%%%%%%%%%%%%%%%%%%%%%%%%%%%%%%%%
%%%%%%%%%%%%%%%%%%%%%%%%%%%%%%%%%%%%%%%%%%%%%%%%%%%%%%%%%%%%%%%%%%%%%%%%%%%%%%%%
\section{Introduction}

\LaTeX{} provides a mechanism to structure a large document (such as a book)
into a main file and several child files (containing the chapters)
using the |\include| command.
This mechanism is beneficial for documents
which span hundreds of pages in order to
make the source file(s) more manageable.
Moreover, compilation can be restricted to
selected child files by means of the |\includeonly| command.
The latter feature can be used to reduce the compilation time while editing
(this was significantly more useful in the earlier days of \LaTeX{})
or to generate a smaller document which is easier to navigate.
Another application of |\includeonly| is to generate
documents consisting of selected parts of the complete document.

However, there are a few drawbacks of the plain |\include| mechanism:
\begin{itemize}
\item
The child files cannot be compiled on their own,
they can only be compiled via the main file.
A naive editing environment
(such as a text editor with an option
to have the current file processed by \LaTeX)
may require one to switch to the main file before compiling;
attempting to compile the child file produces errors.
\item
The main file must be modified (each time)
to adjust the |\includeonly| command
to the present needs. This easily leaves the main file in a messy state.
\item
The generated document will always carry the filename
of the main document. This is inconvenient if
several child files are to be compiled and
to be kept for distribution.
\end{itemize}

The present package provides a simple interface
to make child files individually compilable by \LaTeX{}.
Compiling a child file then has the same effect as compiling
the main file with an |\includeonly| command
to select the appropriate child.
Moreover the generated document will carry the name of the child
rather than the main file.
This resolves all three above issues.

This feature is meant to make the editing of books,
thesis documents and lecture notes somewhat more convenient.
However, the package can also be used efficiently for
composing a series of documents (such as exercise sheets)
which are typically distributed individually.
It then assists the author in generating the individual documents
(potentially in different versions)
as well as a document containing the collected series.
Another application is in developing style files
or other kinds of included material
where compilation of the style file could redirect
to a sample or test file.

%%%%%%%%%%%%%%%%%%%%%%%%%%%%%%%%%%%%%%%%%%%%%%%%%%%%%%%%%%%%%%%%%%%%%%%%%%%%%%%%
%%%%%%%%%%%%%%%%%%%%%%%%%%%%%%%%%%%%%%%%%%%%%%%%%%%%%%%%%%%%%%%%%%%%%%%%%%%%%%%%
\section{Usage}

First of all, the package \textsf{childdoc} is \emph{not} a standard
\LaTeXe{} |.sty| style file! Therefore it needs to be invoked in
a non-standard way.

%%%%%%%%%%%%%%%%%%%%%%%%%%%%%%%%%%%%%%%%%%%%%%%%%%%%%%%%%%%%%%%%%%%%%%%%%%%%%%%%
\subsection{Included Files}
\label{sec:include}

%%%%%%%%%%%%%%%%%%%%%%%%%%%%%%%%%%%%%%%%
\DescribeMacro{\childdocmain}
To use the package, add the commands
\begin{center}
\begin{tabular}{l}
|\input{childdoc.def}|\\
|\childdocmain{}|\\
\end{tabular}
\end{center}
at the very top of the main \LaTeX{} file,
in particular \emph{before} the |\documentclass| statement!
The argument of |\childdocmain| should be left empty
(but it must be present).

%%%%%%%%%%%%%%%%%%%%%%%%%%%%%%%%%%%%%%%%
\DescribeMacro{\childdocof}
Furthermore, add the commands
\begin{center}
\begin{tabular}{l}
|\input{childdoc.def}|\\
|\childdocof{|\textit{main}|}|\\
\end{tabular}
\end{center}
at the top of every child file \textit{child}
which is included by |\include{|\textit{child}|}|
from within the main file
(or at least for those files to be compiled individually).
The argument \textit{main} must be the filename of the main file.

There are a couple of
considerations in setting up the main and child documents:

%%%%%%%%%%%%%%%%%%%%%%%%%%%%%%%%%%%%%%%%
\paragraph{Restrictions.}

Please note the following restrictions:
\begin{itemize}
\item
|\childdocmain| must be called with one argument \textit{main}
to ensure compatibility with earlier version of the package.
It must either be empty (|\childdocmain{}|)
or precisely match the filename of the main file in which it is specified.
See \secref{sec:detection} for further information.
\item
The filename \textit{main} must be specified without the |.tex| extension.
\item
The filename \textit{main} is case sensitive
(even in case-insensitive file systems)
due to internal string comparison.
\item
The argument \textit{main} should be fully expanded, it cannot be a macro.
\item
Subdirectories and special characters should be avoided in filenames.
\item
The command |\childdocmain{|\textit{main}|}| must be followed by a whitespace.
It should not be followed immediately by another command
or by a comment mark `|%|'.
This is because the \TeX{} parser reads the token immediately following
the argument of |\childdocmain| and puts it
at the beginning of every child section;
however, a white\-space is ignored.
\end{itemize}

%%%%%%%%%%%%%%%%%%%%%%%%%%%%%%%%%%%%%%%%
\paragraph{Content of Main File.}

It is advisable to place all content in the child files included by |\include|.
Any output contained in the main file will appear in all child documents
unless suppressed manually;
it cannot be suppressed automatically by the |\includeonly| directive
and thus should normally be avoided.
A method to include some content in the main file
by means of conditional processing is described in \secref{sec:conditional}.

%%%%%%%%%%%%%%%%%%%%%%%%%%%%%%%%%%%%%%%%
\paragraph{Page Numbering.}

When only a part of the document is compiled,
the appropriate numbering of pages
(as well as other status parameters)
is determined from the |.aux| files.
The latter contain information from previous passes.
However this information needs to propagate through
all intermediate child documents.
Therefore the page numbering in child documents may well
be inconsistent until the complete document is compiled at least once.

A useful (if unconventional) way to always ensure a consistent
page numbering is to restart the numbering in each child document
and denote the pages by `\textit{child}|.|\textit{page}'
where \textit{child} represents the chapter/section number of the child file.
This can be achieved by the command
|\numberwithin{page}{|\textit{child}|}|
of the \textsf{amsmath} package
where \textit{child} can be |chapter| or |section|
depending on the chosen structuring.
Alternatively, one can modify the macro |\thepage| appropriately
and reset the counter |page| at the start of each child file.

%%%%%%%%%%%%%%%%%%%%%%%%%%%%%%%%%%%%%%%%%%%%%%%%%%%%%%%%%%%%%%%%%%%%%%%%%%%%%%%%
\subsection{Conditional Processing}
\label{sec:conditional}

The package provides a mechanism to compile different versions
of a document. To customise the versions further some conditional processing
can come in handy to distinguish which version is being compiled.
The package provides two macros to describe the compilation context:

%%%%%%%%%%%%%%%%%%%%%%%%%%%%%%%%%%%%%%%%
\DescribeMacro{\ifchilddoc}
The conditional |\ifchilddoc| distinguishes between the compilation of
child documents and the main document:
%
\begin{center}
|\ifchilddoc |\textit{child-code}| |[|\||else |\textit{main-code}]| \||fi|
\end{center}

%%%%%%%%%%%%%%%%%%%%%%%%%%%%%%%%%%%%%%%%
\DescribeMacro{\childdocname}
\DescribeMacro{\childdocjob}
The macro |\childdocname| contains the filename (without extension)
of the main or child file being processed.
Note that |\childdocjob| will always contain the name of the main file.

%%%%%%%%%%%%%%%%%%%%%%%%%%%%%%%%%%%%%%%%
\paragraph{Title Page.}

Conditional processing can be used to include a title or banner page
in the main document when proper precautions are taken.
Importantly, the code in the main file should ensure that the page counter
(as well as other status parameters which are stored in the |.aux| files)
takes the same value after the conditional processing.
Otherwise the page numbers may take divergent values
depending on which part is compiled.

For example, a title page could be declared by:
%
\begin{center}
\begin{tabular}{l}
|\ifchilddoc\||else|\\
|\addtocounter{page}{-1}|\\
\textit{code for title page}\\
|\newpage|\\
|\||fi|
\end{tabular}
\end{center}
%
A banner page for the child documents can be generated by:
%
\begin{center}
\begin{tabular}{l}
|\ifchilddoc|\\
|\addtocounter{page}{-1}|\\
\textit{code for banner page}\\
|\newpage|\\
|\||fi|
\end{tabular}
\end{center}
%
Here one could write a message such as:
\begin{center}
|This is the part \childdocname{} of \childdocjob{}.|
\end{center}

%%%%%%%%%%%%%%%%%%%%%%%%%%%%%%%%%%%%%%%%%%%%%%%%%%%%%%%%%%%%%%%%%%%%%%%%%%%%%%%%
\subsection{Flags}
\label{sec:flags}

The package makes it easy to generate different versions
of the main or child documents.
To this end compilation flags can be defined
and assigned different default values.
They will be particularly useful in conjunction
with the forwarding mechanism described in \secref{sec:forward}.

For example, it may be useful to have a flag |\version|
which can be set to |draft| or |final|.
The document source will contain some conditional code
depending on the value of |\version|.
Suppose further, the flag should default to |final| for the main file
and to |draft| for child files
which is a natural assignment for editing the document.
This is achieved by placing the following code
in the preamble of the main document
(below the |\childdocmain| directive):
%
\begin{center}
\begin{tabular}{l}
|\ifchilddoc|\\
|\providecommand{\version}{draft}|\\
|\||else|\\
|\providecommand{\version}{final}|\\
|\||fi|
\end{tabular}
\end{center}
%
The definition by |\providecommand| makes sure
that previous definitions are not overwritten.
Further statements |\providecommand{\version}{...}|
can thus be added before the above code to override it.

For the main file, one might add a line
(between |\childdocmain| and the above block)
%
\begin{center}
|%\ifchilddoc\||else\providecommand{\version}{draft}\||fi|
\end{center}
%
which can be uncommented to produce a draft version.
Likewise one can add a line to the very top of a child file
(above the |\childdocof{|\textit{main}|}| directive)
%
\begin{center}
|%\providecommand{\version}{final}|
\end{center}
%
which can be uncommented to produce the final version of this child document.

%%%%%%%%%%%%%%%%%%%%%%%%%%%%%%%%%%%%%%%%%%%%%%%%%%%%%%%%%%%%%%%%%%%%%%%%%%%%%%%%
\subsection{Forwarding}
\label{sec:forward}

Different versions of the main or child documents
using compilation flags as described in \secref{sec:flags}
can be (permanently) stored in different files
for convenient compilation, viewing and distribution.
To this end, the package defines a command
to pass on compilation to a different file:

%%%%%%%%%%%%%%%%%%%%%%%%%%%%%%%%%%%%%%%%
\DescribeMacro{\childdocforward}
The command |\childdocforward| redirects processing to
another source file:
%
\begin{center}
\begin{tabular}{l}
|\input{childdoc.def}|\\
|\childdocforward[|\textit{main}|]{|\textit{dest}|}|\\
\end{tabular}
\end{center}
%
The argument \textit{dest} is the destination file
(without extension).
It should be the main file or one of the child files.
Note that further \textsf{childdoc} directives
such as |\childdocof| and |\childdocforward|
in the indicated file will be processed in this form.
The optional argument \textit{main}
passes on directly to the main file \textit{main}
while pretending to compile the child \textit{dest}.
This form behaves as if \textit{dest}
issues |\childdocof{|\textit{main}|}| right away,
and no further \textsf{childdoc} directives will be processed.

%%%%%%%%%%%%%%%%%%%%%%%%%%%%%%%%%%%%%%%%
\DescribeMacro{\...prefix}
In the alternative form |\childdocforwardprefix|,
%
\begin{center}
\begin{tabular}{l}
|\input{childdoc.def}|\\
|\childdocforwardprefix[|\textit{main}|]{|\textit{prefix}|}{|\textit{dest}|}|
\end{tabular}
\end{center}
%
the destination file is determined by a pattern
depending on the current file:
To make this work, the current file must be called
`{\textit{prefix}\hspace{0.2em}\textit{suffix}}'
with \textit{prefix} matching precisely the argument.
Processing is then passed on to the file
`{\textit{dest}\hspace{0.2em}\textit{suffix}}'.
Surely, the same effect is achieved by
directly specifying the
argument `{\textit{dest}\hspace{0.2em}\textit{suffix}}'
in the first form.
However, that requires to set up a different file
for each child. With the alternative form of the command
all these files can have exactly the same content
which simplifies setting them up and maintaining them.

For example, the following file |draft.tex|
with a compilation flag |\version| as described in \secref{sec:flags}
compiles the main document as a draft:
%
\begin{center}
\begin{tabular}{l}
|\def\version{draft}|\\
|\input{childdoc.def}|\\
|\childdocforward{|\textit{main}|}|
\end{tabular}
\end{center}
%
Likewise, the following files |final|\textit{nn}|.tex|
compile the final version of the child document
|child|\textit{nn}|.tex|:
%
\begin{center}
\begin{tabular}{l}
|\def\version{final}|\\
|\input{childdoc.def}|\\
|\childdocforwardprefix{final}{child}|
\end{tabular}
\end{center}
%

Note that when several versions of a main file and/or of each child file
are to be generated, it may be convenient to set up a |Makefile| or
shell script to automatise the process.

%%%%%%%%%%%%%%%%%%%%%%%%%%%%%%%%%%%%%%%%%%%%%%%%%%%%%%%%%%%%%%%%%%%%%%%%%%%%%%%%
\subsection{Command Line Processing}
\label{sec:commandline}

The effect of redirection files can also be achieved by invoking
the \LaTeX{} compiler with a more elaborate command line.
Most conveniently this should be done as part
of a shell script or a |Makefile|.

When using \textsf{childdoc} in the main file, the following
command lines effectively perform a redirection
(note that depending on the shell being used,
backslashes may have to be doubled: `|\|' $\to$ `|\\|'):
%
\begin{center}
|... -jobname "|\textit{target}|" |\\|"|[\textit{flags}]%
|\input{childdoc.def}\childdocforward[|\textit{main}|]{|\textit{dest}|}"|
\end{center}
%
Here \textit{target} is the name of the output file,
\textit{main} is the name of the main file
and \textit{dest} is the name of the main or child file to be processed
(all filenames without extensions).
The optional argument \textit{main} can be omitted
if \textit{main} matches \textit{dest}.
Optionally, compilation \textit{flags} can be defined via |\def| commands.
This command line makes the \TeX{} engine believe
it is compiling the file \textit{target}
whose content is specified as the latter parameter.
The provided code then forwards the processing to
\textit{main} or \textit{dest} as described in \secref{sec:forward}.

%%%%%%%%%%%%%%%%%%%%%%%%%%%%%%%%%%%%%%%%%%%%%%%%%%%%%%%%%%%%%%%%%%%%%%%%%%%%%%%%
\subsection{Include by Input}
\label{sec:input}

Including child documents by |\include| has some restrictions by design.
Most notably, the content of a child document always occupies
its own set of pages; pages cannot be shared between child documents.
Usually, this behaviour makes perfect sense
because each child document contain an essential part of the document.
However, in some situations it may be desirable to compose
a document from a collection of parts
without having mandatory page breaks between then.
For this case, the package
provides a mechanism to include parts
by |\input| which can also be processed individually.
However, by construction this mechanism
requires manual handling of the content to be output.

%%%%%%%%%%%%%%%%%%%%%%%%%%%%%%%%%%%%%%%%
\DescribeMacro{\ifchilddocmanual}
The main file should be prepared as usual, see \secref{sec:include}.
However, the document body must make a distinction
between processing of an individual part and of the main document, e.g.:
%
\begin{center}
\begin{tabular}{l}
|\ifchilddocmanual|\\
|\input{\childdocname}|\\
|\||else|\\
\textit{document body with }|\input{|\textit{part}|}|\\
|\||fi|
\end{tabular}
\end{center}
%
The conditional |\ifchilddocmanual| is true whenever
a part to be included by |\input| is being compiled,
and the name of the part is stored in |\childdocname|.

%%%%%%%%%%%%%%%%%%%%%%%%%%%%%%%%%%%%%%%%
\DescribeMacro{\childdocby}
Each part to be included by |\input| should start with:
%
\begin{center}
\begin{tabular}{l}
|\input{childdoc.def}|\\
|\childdocby{|\textit{main}|}|\\
\end{tabular}
\end{center}
%
The directive |\childdocby| is similar to |\childdocof|
described in \secref{sec:include},
but the subsequent selection of content must be done manually.
To that end, both |\ifchilddoc| and |\ifchilddocmanual|
will be true upon processing of a part,
and the name of the part is stored in |\childdocname|.
Note that |\jobname| will be set to the filename of the current part
so that each part receives an individual |.aux| file
that does not interfere with the |.aux| file(s) of the main document.
This behaviour can be altered by the alternative form
|\childdocby[*]{|\textit{main}|}| (with a non-empty optional argument)
which uses the |.aux| file of the main document
by setting |\jobname| to \textit{main}.

%%%%%%%%%%%%%%%%%%%%%%%%%%%%%%%%%%%%%%%%%%%%%%%%%%%%%%%%%%%%%%%%%%%%%%%%%%%%%%%%
\subsection{Driver Development}
\label{sec:driver}

The \textsf{childdoc} mechanism can also be use for the development
of definition files such as \LaTeX{} styles or classes.
This case differs from the above setup with multiple parts
included by |\include| in that no |\includeonly| should be invoked.
This can be achieved by starting the include file
(before |\ProvidesPackage|) with:
%
\begin{center}
\begin{tabular}{l}
|\input{childdoc.def}|\\
|\childdocforward{|\textit{main}|}|\\
\end{tabular}
\end{center}
%
or alternatively with:
%
\begin{center}
\begin{tabular}{l}
|\input{childdoc.def}|\\
|\childdocby{|\textit{main}|}|\\
\end{tabular}
\end{center}
%
Both forms have slightly different effects as described above.
The main file is prepared as usual, see \secref{sec:include}.

%%%%%%%%%%%%%%%%%%%%%%%%%%%%%%%%%%%%%%%%%%%%%%%%%%%%%%%%%%%%%%%%%%%%%%%%%%%%%%%%
\subsection{Legacy Detection}
\label{sec:detection}

The directive |\childdocmain| in the main file can detect
whether the complete document or merely a child is to be compiled
even without using the directive |\childdocof|.
This method is deprecated because it is less robust
and there is no compelling reason to use it;
it is merely provided for backward compatibility
and it may be removed in future versions.

If the detection mechanism is to be used,
it is mandatory to correctly specify
the filename of the main file as the argument of |\childdocmain|:
%
\begin{center}
\begin{tabular}{l}
|\input{childdoc.def}|\\
|\childdocmain{|\textit{main}|}|\\
\end{tabular}
\end{center}
%
If |\jobname| does not match the argument \textit{main} of |\childdocmain|,
it is assumed that |\jobname| points to the child file to be compiled.
When using |\childdocmain| with the main file specified as argument,
it suffices to start a child file
with just |\input{|\textit{main}|}|
without loading of the package and using |\childdocof|.
If instead all processing is done
with the appropriate \textsf{childdoc} directives,
the argument of \textit{main} of |\childdocmain| can be empty.

An alternative version of the command line processing described
in \secref{sec:commandline} using the detection mechanism reads:
%
\begin{center}
|... -jobname "|\textit{target}|" "|[\textit{flags}]%
[|\def\jobname{|\textit{dest}|}|]|\input{|\textit{main}|}"|
\end{center}

%%%%%%%%%%%%%%%%%%%%%%%%%%%%%%%%%%%%%%%%%%%%%%%%%%%%%%%%%%%%%%%%%%%%%%%%%%%%%%%%
\subsection{Manual Code}
\label{sec:manual}

In case one cannot be certain whether the definitions file |childdoc.def|
is installed on the target \TeX{} distribution
and one prefers not to ship it,
it is conceivable to paste a few relevant commands into the sources.

To that end, drop all statements |\input{childdoc.def}|
and perform the replacements as outlined below.
Instead of |\childdocmain{|\textit{main}|}| add the following code
to the top of the main file:
%
\begin{center}
\begin{tabular}{l}
|\||ifdefined\childdocname\endinput\||fi\newif\ifchilddoc|\\
|\edef\childdocname{\scantokens\expandafter{\jobname\noexpand}}|\\
|\def\childdocmain{|\textit{main}|}\||ifx\childdocmain\childdocname\||else|\\
|\childdoctrue\includeonly{\childdocname}\let\jobname\childdocmain\||fi|\\
\end{tabular}
\end{center}
%
Instead of |\childdocof{|\textit{main}|}| just include the main file
at the top of each child file:
%
\begin{center}
|\input{|\textit{main}|}|
\end{center}
%
A simple redirection |\childdocforward{|\textit{dest}|}| is achieved by:
%
\begin{center}
|\def\jobname{|\textit{dest}|}\input{\jobname}|
\end{center}
%
The redirection with prefix
|\childdocforwardprefix[|\textit{prefix}|]{|\textit{dest}|}|
is accomplished by:
%
\begin{center}
\begin{tabular}{l}
|{\edef\jobname{\scantokens\expandafter{\jobname\noexpand}}|\\
|\def\redirectjob |\textit{prefix}|#1~~~{\gdef\jobname{|\textit{dest}|#1}}|\\
|\expandafter\redirectjob\jobname~~~}\input{\jobname}|
\end{tabular}
\end{center}

In an alternative approach,
child documents can be compiled by a specific command line
without additional code or specific definitions:
%
\begin{center}
|... -jobname "|\textit{target}|" "|[\textit{flags}]%
|\includeonly{|\textit{dest}|}\input{|\textit{main}|}"|
\end{center}
%

%%%%%%%%%%%%%%%%%%%%%%%%%%%%%%%%%%%%%%%%%%%%%%%%%%%%%%%%%%%%%%%%%%%%%%%%%%%%%%%%
%%%%%%%%%%%%%%%%%%%%%%%%%%%%%%%%%%%%%%%%%%%%%%%%%%%%%%%%%%%%%%%%%%%%%%%%%%%%%%%%
\section{Information}

%%%%%%%%%%%%%%%%%%%%%%%%%%%%%%%%%%%%%%%%%%%%%%%%%%%%%%%%%%%%%%%%%%%%%%%%%%%%%%%%
\subsection{Copyright}

Copyright \copyright{} 2017--2018 Niklas Beisert

This work may be distributed and/or modified under the
conditions of the \LaTeX{} Project Public License, either version 1.3
of this license or (at your option) any later version.
The latest version of this license is in
  \url{http://www.latex-project.org/lppl.txt}
and version 1.3 or later is part of all distributions of \LaTeX{}
version 2005/12/01 or later.

This work has the LPPL maintenance status `maintained'.

The Current Maintainer of this work is Niklas Beisert.

This work consists of the files |README.txt|, |childdoc.ins| and |childdoc.dtx|
as well as the derived files |childdoc.def|, |cdocsamp.tex|
with |cdocsch1.tex|, |cdocsch2.tex|, |cdocspt3.tex|, |cdocspt4.tex|,
|cdocsdrf.tex|, |cdocsfn1.tex|, |cdocsfn2.tex|
as well as |childdoc.pdf|.

%%%%%%%%%%%%%%%%%%%%%%%%%%%%%%%%%%%%%%%%%%%%%%%%%%%%%%%%%%%%%%%%%%%%%%%%%%%%%%%%
\subsection{Files and Installation}

The package consists of the files:
%
\begin{center}
\begin{tabular}{ll}
    |README.txt|   & readme file \\
    |childdoc.ins| & installation file \\
    |childdoc.dtx| & source file \\
    |childdoc.def| & definition file \\
    |cdocsamp.tex| & sample main file \\
    |cdocsch1.tex| & sample include file \\
    |cdocsch2.tex| & sample include file \\
    |cdocspt3.tex| & sample part file \\
    |cdocspt4.tex| & sample part file \\
    |cdocsdrf.tex| & sample redirection file \\
    |cdocsfn1.tex| & sample redirection file \\
    |cdocsfn2.tex| & sample redirection file \\
    |childdoc.pdf| & manual
\end{tabular}
\end{center}
%
The distribution consists of the files
|README.txt|, |childdoc.ins| and |childdoc.dtx|.
%
\begin{itemize}
\item
Run (pdf)\LaTeX{} on |childdoc.dtx|
to compile the manual |childdoc.pdf| (this file).
\item
Run \LaTeX{} on |childdoc.ins| to create the definitions file |childdoc.def|
and the sample |cdocsamp.tex| with include files
|cdocsch1.tex|, |cdocsch2.tex|, |cdocspt3.tex|, |cdocspt4.tex|,
|cdocsdrf.tex|, |cdocsfn1.tex|, |cdocsfn2.tex|.
Then copy the file |childdoc.def| to an appropriate directory of your \LaTeX{}
distribution, e.g.\ \textit{texmf-root}|/tex/latex/childdoc|.
\end{itemize}

%%%%%%%%%%%%%%%%%%%%%%%%%%%%%%%%%%%%%%%%%%%%%%%%%%%%%%%%%%%%%%%%%%%%%%%%%%%%%%%%
\subsection{Related CTAN Packages}

There are several other packages which offer a similar functionality:
%
\begin{itemize}
\item
The packages
\href{http://ctan.org/pkg/docmute}{\textsf{docmute}},
\href{http://ctan.org/pkg/includex}{\textsf{includex}} and
\href{http://ctan.org/pkg/standalone}{\textsf{standalone}}
provide commands to include only the document body of
a child file thus allowing both files to be compiled individually.
\item
The packages \href{http://ctan.org/pkg/subdocs}{\textsf{subdocs}}
and \href{http://ctan.org/pkg/subfiles}{\textsf{subfiles}}
provide structures in which the main and child documents can be
encapsulated and allowing them to be compiled individually.
The inclusion mechanism is different from the conventional |\include|.
\item
The package \href{http://ctan.org/pkg/combine}{\textsf{combine}}
is an elaborate solution to combine several documents into one.
\end{itemize}
%
See also the CTAN topic \href{http://ctan.org/topic/subdocs}{\textsf{subdocs}}
for further related packages.
The present package differs from the above solutions in that
a document structure constructed with the conventional |\include| mechanism
just needs two extra commands at the top of every file
such that all constituent files can be compiled individually.

%%%%%%%%%%%%%%%%%%%%%%%%%%%%%%%%%%%%%%%%%%%%%%%%%%%%%%%%%%%%%%%%%%%%%%%%%%%%%%%%
%\subsection{Feature Suggestions}
%
%The following is a list of features which may be useful for future
%versions of this package:
%%
%\begin{itemize}
%\item
%\ldots
%\end{itemize}

%%%%%%%%%%%%%%%%%%%%%%%%%%%%%%%%%%%%%%%%%%%%%%%%%%%%%%%%%%%%%%%%%%%%%%%%%%%%%%%%
\subsection{Revision History}

%%%%%%%%%%%%%%%%%%%%%%%%%%%%%%%%%%%%%%%%
\paragraph{v2.0:} 2018/12/30

\begin{itemize}
\item
immediate forward processing
\item
added |\childdocby| mechanism
\item
manual restructured
\end{itemize}

%%%%%%%%%%%%%%%%%%%%%%%%%%%%%%%%%%%%%%%%
\paragraph{v1.6:} 2018/01/17

\begin{itemize}
\item
application for development of include files
\item
corrections to manual
\end{itemize}

%%%%%%%%%%%%%%%%%%%%%%%%%%%%%%%%%%%%%%%%
\paragraph{v1.5:} 2017/05/21

\begin{itemize}
\item
more complete structuring introduced
\item
|\childdocof| introduced
\item
|\childdoc| renamed to |\childdocmain|
\item
|\childredirect| renamed to |\childdocforward| and |\childdocforwardprefix|
and functionality expanded
\end{itemize}

%%%%%%%%%%%%%%%%%%%%%%%%%%%%%%%%%%%%%%%%
\paragraph{v1.0:} 2017/04/27

\begin{itemize}
\item
manual and install package
\item
first version published on CTAN
\end{itemize}

%%%%%%%%%%%%%%%%%%%%%%%%%%%%%%%%%%%%%%%%
\paragraph{v0.6:} 2017/04/26

\begin{itemize}
\item
redirection mechanism added
\end{itemize}

%%%%%%%%%%%%%%%%%%%%%%%%%%%%%%%%%%%%%%%%
\paragraph{v0.5:} 2017/04/26

\begin{itemize}
\item
functionality in definition file
\end{itemize}


%%%%%%%%%%%%%%%%%%%%%%%%%%%%%%%%%%%%%%%%%%%%%%%%%%%%%%%%%%%%%%%%%%%%%%%%%%%%%%%%
%%%%%%%%%%%%%%%%%%%%%%%%%%%%%%%%%%%%%%%%%%%%%%%%%%%%%%%%%%%%%%%%%%%%%%%%%%%%%%%%
%%%%%%%%%%%%%%%%%%%%%%%%%%%%%%%%%%%%%%%%%%%%%%%%%%%%%%%%%%%%%%%%%%%%%%%%%%%%%%%%
\appendix

\settowidth\MacroIndent{\rmfamily\scriptsize 000\ }

 \DocInput{childdoc.dtx}

\end{document}
%</driver>
% \fi
%
% %%%%%%%%%%%%%%%%%%%%%%%%%%%%%%%%%%%%%%%%%%%%%%%%%%%%%%%%%%%%%%%%%%%%%%%%%%%%%%
% %%%%%%%%%%%%%%%%%%%%%%%%%%%%%%%%%%%%%%%%%%%%%%%%%%%%%%%%%%%%%%%%%%%%%%%%%%%%%%
% \section{Sample}
%\iffalse
%<*samplemain>
%\fi
%
% The following presents a sample document
% with two chapters, two parts, a title page,
% a compile flag as well as three forwarding files to set the flag.
% It consists of eight |.tex| files:
% \begin{center}
% \begin{tabular}{ll}
% |cdocsamp.tex|&main file\\
% |cdocsch1.tex|&include file for chapter 1\\
% |cdocsch2.tex|&include file for chapter 2\\
% |cdocspt3.tex|&include file for part 3\\
% |cdocspt4.tex|&include file for part 4\\
% |cdocsdrf.tex|&forwarding file for main file in draft mode\\
% |cdocsfi1.tex|&forwarding file for final version of chapter 1\\
% |cdocsfi2.tex|&forwarding file for final version of chapter 2\\
% \end{tabular}
% \end{center}
% Each of the eight files can be compiled directly by the \LaTeX{} compiler.
%
% %%%%%%%%%%%%%%%%%%%%%%%%%%%%%%%%%%%%%%
% \paragraph{Main File.}
%
% The main file is called |cdocsamp.tex|.
%
% Load the \textsf{childdoc} definitions and
% declare the filename for the main document:
%    \begin{macrocode}
\input{childdoc.def}
\childdocmain{}
%    \end{macrocode}

% Optional override for |\version| flag:
%    \begin{macrocode}
%%\ifchilddoc\else\providecommand{\version}{draft}\fi
%    \end{macrocode}

% Define the default values for the |\version| flag
% (|final| for the main file and |draft| for childs):
%    \begin{macrocode}
\ifchilddoc
\providecommand{\version}{draft}
\else
\providecommand{\version}{final}
\fi
%    \end{macrocode}

% Load the standard document class:
%    \begin{macrocode}
\documentclass[12pt]{article}
%    \end{macrocode}

% Start the document body:
%    \begin{macrocode}
\begin{document}
%    \end{macrocode}

% Declare a title page.
% Print title, part of document being processed and version flag:
%    \begin{macrocode}
\addtocounter{page}{-1}
\begin{center}
{\LARGE\bfseries{}childdoc example\par}
\vspace{1cm}
\ifchilddoc
\ifchilddocmanual part\else chapter\fi:
`\childdocname' of `\childdocjob'\par
\else
main document: `\childdocjob'\par
\fi
version: \version\par
\end{center}
\newpage
%    \end{macrocode}

% Manually include selected file,
% otherwise process as usual:
%    \begin{macrocode}
\ifchilddocmanual
\section*{part `\childdocname'}
\input{\childdocname}
\else
%    \end{macrocode}

% Include the two chapters:
%    \begin{macrocode}
\include{cdocsch1}
\include{cdocsch2}
%    \end{macrocode}

% Include the two parts unless only chapters should be displayed:
%    \begin{macrocode}
\ifchilddoc\else
\section{part three}
\input{cdocspt3}
\section{part four}
\input{cdocspt4}
\fi
%    \end{macrocode}

% Process as usual until here:
%    \begin{macrocode}
\fi
%    \end{macrocode}

% End of document body:
%    \begin{macrocode}
\end{document}
%    \end{macrocode}
%\iffalse
%</samplemain>
%\fi
%
% %%%%%%%%%%%%%%%%%%%%%%%%%%%%%%%%%%%%%%
% \paragraph{Chapter Include Files.}
%
% The include files are called |cdocsch1.tex| and |cdocsch2.tex|.
%
%\iffalse
%<*samplechap1|samplechap2>
%\fi

% Optional override for |\version| flag:
%    \begin{macrocode}
%%\providecommand{\version}{final}
%    \end{macrocode}

% Include the main document:
%    \begin{macrocode}
\input{childdoc.def}
\childdocof{cdocsamp}
%    \end{macrocode}

%\iffalse
%</samplechap1|samplechap2>
%\fi
%
%\iffalse
%<*samplechap1>
%\fi
% Some text for chapter 1:
%    \begin{macrocode}
\section{one}
some text in chapter one
%    \end{macrocode}

%\iffalse
%</samplechap1>
%\fi
% Some text for chapter 2:
%\iffalse
%<*samplechap2>
%\fi
%    \begin{macrocode}
\section{two}
more text in chapter two
%    \end{macrocode}

%\iffalse
%</samplechap2>
%\fi
%
% %%%%%%%%%%%%%%%%%%%%%%%%%%%%%%%%%%%%%%
% \paragraph{Part Include Files.}
%
% The include files are called |cdocspt3.tex| and |cdocspt4.tex|.
%
%\iffalse
%<*samplepart3|samplepart4>
%\fi

% Optional override for |\version| flag:
%    \begin{macrocode}
%%\providecommand{\version}{final}
%    \end{macrocode}

% Include the main document:
%    \begin{macrocode}
\input{childdoc.def}
\childdocby{cdocsamp}
%    \end{macrocode}

%\iffalse
%</samplepart3|samplepart4>
%\fi
%
%\iffalse
%<*samplepart3>
%\fi
% Some text for part 3:
%    \begin{macrocode}
some text in part three
%    \end{macrocode}

%\iffalse
%</samplepart3>
%\fi
% Some text for part 4:
%\iffalse
%<*samplepart4>
%\fi
%    \begin{macrocode}
more text in part four
%    \end{macrocode}

%\iffalse
%</samplepart4>
%\fi
%
% %%%%%%%%%%%%%%%%%%%%%%%%%%%%%%%%%%%%%%
% \paragraph{Forwarding for a Complete Draft.}
%
% The following forwarding file |cdocsdrf.tex|
% compiles the main document in draft mode:
%\iffalse
%<*sampledraft>
%\fi
%    \begin{macrocode}
\def\version{draft}
\input{childdoc.def}
\childdocforward{cdocsamp}
%    \end{macrocode}

%\iffalse
%</sampledraft>
%\fi
%
% %%%%%%%%%%%%%%%%%%%%%%%%%%%%%%%%%%%%%%
% \paragraph{Forwarding for Final Version of the Chapters.}
%
% The following forwarding files |cdocsfn1.tex| and |cdocsfn2.tex|
% (with identical content)
% compile the final versions of the child documents
% |cdocsch1.tex| and |cdocsch2.tex|, respectively:
%\iffalse
%<*samplefinal>
%\fi
%    \begin{macrocode}
\def\version{final}
\input{childdoc.def}
\childdocforwardprefix[cdocsamp]{cdocsfn}{cdocsch}
%    \end{macrocode}

%\iffalse
%</samplefinal>
%\fi
%
% %%%%%%%%%%%%%%%%%%%%%%%%%%%%%%%%%%%%%%
% \paragraph{Command Line Processing.}
%
% The following three command lines generate the output files
% |cdocscld|, |cdocscl1| and |cdocscl2|
% which should be identical to
% |cdocsdrf|, |cdocsch1| and |cdocsfn2|, respectively:
% \begin{center}
% \begin{tabular}{l}
% |latex -jobname cdocscld \|\\
% |  "\def\version{draft}\input{childdoc.def}\childdocforward{cdocsamp}"|\\
% |latex -jobname cdocscl1 \|\\
% |  "\input{childdoc.def}\childdocforward[cdocsamp]{cdocsch1}"|\\
% |latex -jobname cdocscl2 \|\\
% |  "\def\version{final}\input{childdoc.def}\childdocforward{cdocsch2}"|
% \end{tabular}
% \end{center}
% Note that the trailing backslash on each first line
% merely continues the input to the second line
% (for convenient cut ant paste).
% Furthermore, the command |latex| can be replaced by any
% of its alternative versions such as |pdflatex|.
%
% %%%%%%%%%%%%%%%%%%%%%%%%%%%%%%%%%%%%%%%%%%%%%%%%%%%%%%%%%%%%%%%%%%%%%%%%%%%%%%
% %%%%%%%%%%%%%%%%%%%%%%%%%%%%%%%%%%%%%%%%%%%%%%%%%%%%%%%%%%%%%%%%%%%%%%%%%%%%%%
% \section{Implementation}
%\iffalse
%<*package>
%\fi
%
% This section describes the definitions file |childdoc.def|.

% The definitions cannot be loaded using |\usepackage| or |\RequirePackage|
% which has a mechanism to prevent loading a style file more than once.
% When loading the definitions by means of |\input|
% multiple instances have to be prevented manually:
%\iffalse
%This code needs to be before the `\ProvidesFile' directive
%which is defined at the beginning of this file.
%Therefore it is also placed there and commented out here.
%</package>
%<*discard>
%\fi
%    \begin{macrocode}
\ifdefined\childdocmain\endinput\fi
%    \end{macrocode}
%\iffalse
%</discard>
%<*package>
%\fi
%
% \macro{\ifchilddoc}
% \macro{\ifchilddocmanual}
% The conditional |\ifchilddoc| tells whether a
% child (true) or main (false) document is being compiled.
% The conditional |\ifchilddocmanual| tells whether
% the |\includeonly| mechanism is used (false) or
% the selection of child files must be performed manually (true).
% The definitions initialise to false:
%    \begin{macrocode}
\newif\ifchilddoc
\newif\ifchilddocmanual
%    \end{macrocode}

% \macro{\childdocname}
% \macro{\childdocjob}
% The macro |\childdocname| stores the name of the main document
% to be compiled. The macro |\childdocjob| stores the name of
% the document on which the \LaTeX{} compiler was originally invoked.
% The content of |\jobname| cannot be compared
% to filenames specified in the source due to different catcodes.
% The following code rescans |\jobname|, stores the result
% in |\childdocname| and saves a copy in |\childdocjob|:
%    \begin{macrocode}
\edef\childdocname{\scantokens\expandafter{\jobname\noexpand}}
\let\childdocjob\childdocname
%    \end{macrocode}

% \macro{\childdocdisable}
% The macro |\childdocdisable| prevents the main file
% from being processed more than once.
% At this stage, the main document command |\childdocmain|
% is assumed to be called once again where it should do nothing.
% Any subsequent call to it should prevent
% a secondary processing of the main document
% It overwrites the forwarding commands
% |\childdocof| and |\childdocforward|
% with empty macros to prevent further inclusions of the main document:
%    \begin{macrocode}
\newcommand{\childdocdisable}
{
  \renewcommand{\childdocmain}[1]{\renewcommand{\childdocmain}[1]{\endinput}}
  \renewcommand{\childdocof}[1]{}
  \renewcommand{\childdocby}[2][]{}
  \renewcommand{\childdocforward}[2][]{}
  \renewcommand{\childdocdisable}{}
}
%    \end{macrocode}

% \macro{\childdocmain}
% The macro |\childdocmain| is to be called at the top of the main file
% with nothing or the main filename (without extension) as argument.
% First, it breaks loops.
% If the argument is not empty and does not match |\childdocname|
% (which is set by the first inclusion of |childdoc.def|),
% |\ifchilddoc| is set to true, |\includeonly| is applied to the child file
% and |\jobname| is set to the main file
% (for proper handling of |.aux| files):
%    \begin{macrocode}
\newcommand{\childdocmain}[1]
{
  \childdocdisable\childdocmain{}
  \if?#1?\else
    \begingroup
      \def\childdoctmp{#1}
      \ifx\childdoctmp\childdocname
        \def\childdoctmp{}
      \else
        \def\childdoctmp
        {
          \childdoctrue
          \includeonly{\childdocname}
          \def\childdocjob{#1}
          \def\jobname{#1}
        }
      \fi
      \expandafter
    \endgroup
    \childdoctmp
  \fi
}
%    \end{macrocode}

% \macro{\childdocof}
% The command |\childdocof| redirects
% compilation to the main file |#1|.
%    \begin{macrocode}
\newcommand{\childdocof}[1]
{
  \childdocdisable
  \childdoctrue
  \includeonly{\childdocname}
  \def\jobname{#1}
  \def\childdocjob{#1}
  \input{#1}
}
%    \end{macrocode}

% \macro{\childdocby}
% The command |\childdocby| ....
%    \begin{macrocode}
\newcommand{\childdocby}[2][]
{
  \childdocdisable
  \childdoctrue
  \childdocmanualtrue
  \if?#1?\else
    \def\jobname{#2}
  \fi
  \def\childdocjob{#2}
  \input{#2}
  \endinput
}
%    \end{macrocode}

% \macro{\childdocforward}
% The command |\childdocforward| redirects
% compilation to the main file or
% (if the optional argument is given) a child file.
% Parameters are set as if the main file
% or a child file starting with |\childdocof| was compiled.
% Then compilation is handed over to the main file:
%    \begin{macrocode}
\newcommand{\childdocforward}[2][]
{
  \begingroup
    \if?#1?
      \def\childdoctmp
      {
        \def\childdocname{#2}
        \def\childdocjob{#2}
        \def\jobname{#2}
        \input{#2}
        \endinput
      }
    \else
      \def\childdoctmp
      {
        \childdocdisable
        \def\childdocname{#2}
        \childdoctrue
        \includeonly{#2}
        \def\childdocjob{#1}
        \def\jobname{#1}
        \input{#1}
        \endinput
      }
    \fi
    \expandafter
  \endgroup
  \childdoctmp
}
%    \end{macrocode}

% \macro{\childdocforwardprefix}
% The command |\childdocforwardprefix| redirects
% compilation to the main or a child file by means of a pattern.
% The prefix |#1| in the current filename is replaced by |#2|
% and the suffix of the current filename is kept
% (it is assumed that the filename does not contain the substring `|~~~|'
% which is used as a delimiter).
% Compilation is handed over to the new file by |\childdocforward|:
%    \begin{macrocode}
\newcommand{\childdocforwardprefix}[3][]
{
  \begingroup
    \def\childdocextract #2##1~~~{\def\childdoctmp{\childdocforward[#1]{#3##1}}}
    \expandafter\childdocextract\childdocname~~~
    \expandafter
  \endgroup
  \childdoctmp
}
%    \end{macrocode}

% \macro{\childdoc}
% The deprecated macro |\childdoc| is a legacy version of |\childdocmain|:
%    \begin{macrocode}
\newcommand{\childdoc}{\childdocmain}
%    \end{macrocode}

% \macro{\childdocredirect}
% The deprecated macro |\childdocredirect| is a legacy version
% of |\childdocforward| and |\childdocforwardprefix|:
%    \begin{macrocode}
\newcommand{\childdocredirect}[2][]
{
  \begingroup
    \if?#1?
      \def\childdoctmp{\childdocforward{#2}}
    \else
      \def\childdoctmp{\childdocforwardprefix{#1}{#2}}
    \fi
    \expandafter
  \endgroup
  \childdoctmp
}
%    \end{macrocode}

%\iffalse
%</package>
%\fi
%
\endinput
\childdocforward{cdocsch2}"|
% \end{tabular}
% \end{center}
% Note that the trailing backslash on each first line
% merely continues the input to the second line
% (for convenient cut ant paste).
% Furthermore, the command |latex| can be replaced by any
% of its alternative versions such as |pdflatex|.
%
% %%%%%%%%%%%%%%%%%%%%%%%%%%%%%%%%%%%%%%%%%%%%%%%%%%%%%%%%%%%%%%%%%%%%%%%%%%%%%%
% %%%%%%%%%%%%%%%%%%%%%%%%%%%%%%%%%%%%%%%%%%%%%%%%%%%%%%%%%%%%%%%%%%%%%%%%%%%%%%
% \section{Implementation}
%\iffalse
%<*package>
%\fi
%
% This section describes the definitions file |childdoc.def|.

% The definitions cannot be loaded using |\usepackage| or |\RequirePackage|
% which has a mechanism to prevent loading a style file more than once.
% When loading the definitions by means of |\input|
% multiple instances have to be prevented manually:
%\iffalse
%This code needs to be before the `\ProvidesFile' directive
%which is defined at the beginning of this file.
%Therefore it is also placed there and commented out here.
%</package>
%<*discard>
%\fi
%    \begin{macrocode}
\ifdefined\childdocmain\endinput\fi
%    \end{macrocode}
%\iffalse
%</discard>
%<*package>
%\fi
%
% \macro{\ifchilddoc}
% \macro{\ifchilddocmanual}
% The conditional |\ifchilddoc| tells whether a
% child (true) or main (false) document is being compiled.
% The conditional |\ifchilddocmanual| tells whether
% the |\includeonly| mechanism is used (false) or
% the selection of child files must be performed manually (true).
% The definitions initialise to false:
%    \begin{macrocode}
\newif\ifchilddoc
\newif\ifchilddocmanual
%    \end{macrocode}

% \macro{\childdocname}
% \macro{\childdocjob}
% The macro |\childdocname| stores the name of the main document
% to be compiled. The macro |\childdocjob| stores the name of
% the document on which the \LaTeX{} compiler was originally invoked.
% The content of |\jobname| cannot be compared
% to filenames specified in the source due to different catcodes.
% The following code rescans |\jobname|, stores the result
% in |\childdocname| and saves a copy in |\childdocjob|:
%    \begin{macrocode}
\edef\childdocname{\scantokens\expandafter{\jobname\noexpand}}
\let\childdocjob\childdocname
%    \end{macrocode}

% \macro{\childdocdisable}
% The macro |\childdocdisable| prevents the main file
% from being processed more than once.
% At this stage, the main document command |\childdocmain|
% is assumed to be called once again where it should do nothing.
% Any subsequent call to it should prevent
% a secondary processing of the main document
% It overwrites the forwarding commands
% |\childdocof| and |\childdocforward|
% with empty macros to prevent further inclusions of the main document:
%    \begin{macrocode}
\newcommand{\childdocdisable}
{
  \renewcommand{\childdocmain}[1]{\renewcommand{\childdocmain}[1]{\endinput}}
  \renewcommand{\childdocof}[1]{}
  \renewcommand{\childdocby}[2][]{}
  \renewcommand{\childdocforward}[2][]{}
  \renewcommand{\childdocdisable}{}
}
%    \end{macrocode}

% \macro{\childdocmain}
% The macro |\childdocmain| is to be called at the top of the main file
% with nothing or the main filename (without extension) as argument.
% First, it breaks loops.
% If the argument is not empty and does not match |\childdocname|
% (which is set by the first inclusion of |childdoc.def|),
% |\ifchilddoc| is set to true, |\includeonly| is applied to the child file
% and |\jobname| is set to the main file
% (for proper handling of |.aux| files):
%    \begin{macrocode}
\newcommand{\childdocmain}[1]
{
  \childdocdisable\childdocmain{}
  \if?#1?\else
    \begingroup
      \def\childdoctmp{#1}
      \ifx\childdoctmp\childdocname
        \def\childdoctmp{}
      \else
        \def\childdoctmp
        {
          \childdoctrue
          \includeonly{\childdocname}
          \def\childdocjob{#1}
          \def\jobname{#1}
        }
      \fi
      \expandafter
    \endgroup
    \childdoctmp
  \fi
}
%    \end{macrocode}

% \macro{\childdocof}
% The command |\childdocof| redirects
% compilation to the main file |#1|.
%    \begin{macrocode}
\newcommand{\childdocof}[1]
{
  \childdocdisable
  \childdoctrue
  \includeonly{\childdocname}
  \def\jobname{#1}
  \def\childdocjob{#1}
  \input{#1}
}
%    \end{macrocode}

% \macro{\childdocby}
% The command |\childdocby| ....
%    \begin{macrocode}
\newcommand{\childdocby}[2][]
{
  \childdocdisable
  \childdoctrue
  \childdocmanualtrue
  \if?#1?\else
    \def\jobname{#2}
  \fi
  \def\childdocjob{#2}
  \input{#2}
  \endinput
}
%    \end{macrocode}

% \macro{\childdocforward}
% The command |\childdocforward| redirects
% compilation to the main file or
% (if the optional argument is given) a child file.
% Parameters are set as if the main file
% or a child file starting with |\childdocof| was compiled.
% Then compilation is handed over to the main file:
%    \begin{macrocode}
\newcommand{\childdocforward}[2][]
{
  \begingroup
    \if?#1?
      \def\childdoctmp
      {
        \def\childdocname{#2}
        \def\childdocjob{#2}
        \def\jobname{#2}
        \input{#2}
        \endinput
      }
    \else
      \def\childdoctmp
      {
        \childdocdisable
        \def\childdocname{#2}
        \childdoctrue
        \includeonly{#2}
        \def\childdocjob{#1}
        \def\jobname{#1}
        \input{#1}
        \endinput
      }
    \fi
    \expandafter
  \endgroup
  \childdoctmp
}
%    \end{macrocode}

% \macro{\childdocforwardprefix}
% The command |\childdocforwardprefix| redirects
% compilation to the main or a child file by means of a pattern.
% The prefix |#1| in the current filename is replaced by |#2|
% and the suffix of the current filename is kept
% (it is assumed that the filename does not contain the substring `|~~~|'
% which is used as a delimiter).
% Compilation is handed over to the new file by |\childdocforward|:
%    \begin{macrocode}
\newcommand{\childdocforwardprefix}[3][]
{
  \begingroup
    \def\childdocextract #2##1~~~{\def\childdoctmp{\childdocforward[#1]{#3##1}}}
    \expandafter\childdocextract\childdocname~~~
    \expandafter
  \endgroup
  \childdoctmp
}
%    \end{macrocode}

% \macro{\childdoc}
% The deprecated macro |\childdoc| is a legacy version of |\childdocmain|:
%    \begin{macrocode}
\newcommand{\childdoc}{\childdocmain}
%    \end{macrocode}

% \macro{\childdocredirect}
% The deprecated macro |\childdocredirect| is a legacy version
% of |\childdocforward| and |\childdocforwardprefix|:
%    \begin{macrocode}
\newcommand{\childdocredirect}[2][]
{
  \begingroup
    \if?#1?
      \def\childdoctmp{\childdocforward{#2}}
    \else
      \def\childdoctmp{\childdocforwardprefix{#1}{#2}}
    \fi
    \expandafter
  \endgroup
  \childdoctmp
}
%    \end{macrocode}

%\iffalse
%</package>
%\fi
%
\endinput
|\\
|\childdocby{|\textit{main}|}|\\
\end{tabular}
\end{center}
%
Both forms have slightly different effects as described above.
The main file is prepared as usual, see \secref{sec:include}.

%%%%%%%%%%%%%%%%%%%%%%%%%%%%%%%%%%%%%%%%%%%%%%%%%%%%%%%%%%%%%%%%%%%%%%%%%%%%%%%%
\subsection{Legacy Detection}
\label{sec:detection}

The directive |\childdocmain| in the main file can detect
whether the complete document or merely a child is to be compiled
even without using the directive |\childdocof|.
This method is deprecated because it is less robust
and there is no compelling reason to use it;
it is merely provided for backward compatibility
and it may be removed in future versions.

If the detection mechanism is to be used,
it is mandatory to correctly specify
the filename of the main file as the argument of |\childdocmain|:
%
\begin{center}
\begin{tabular}{l}
|% \iffalse
%
% childdoc.dtx Copyright (C) 2017-2018 Niklas Beisert
%
% This work may be distributed and/or modified under the
% conditions of the LaTeX Project Public License, either version 1.3
% of this license or (at your option) any later version.
% The latest version of this license is in
%   http://www.latex-project.org/lppl.txt
% and version 1.3 or later is part of all distributions of LaTeX
% version 2005/12/01 or later.
%
% This work has the LPPL maintenance status `maintained'.
%
% The Current Maintainer of this work is Niklas Beisert.
%
% This work consists of the files childdoc.dtx and childdoc.ins
% and the derived files childdoc.def and cdocsamp.tex with
% cdocsch1.tex, cdocsch2.tex, cdocsdrf.tex, cdocsfn1.tex, cdocsfn2.tex.
%
%<package>\ifdefined\childdocmain\endinput\fi
%<package>\ProvidesFile{childdoc.def}[2018/12/30 v2.0 child document driver]
%<samplemain>\ProvidesFile{cdocsamp.tex}[2018/12/30 v2.0 sample for childdoc]
%<*driver>
%\ProvidesFile{childdoc.drv}[2018/12/30 v2.0 childdoc reference manual file]
\PassOptionsToClass{10pt,a4paper}{article}
\documentclass{ltxdoc}

\usepackage[margin=35mm]{geometry}
\usepackage{hyperref}
\usepackage{hyperxmp}
\usepackage[usenames]{color}

\hypersetup{colorlinks=true}
\hypersetup{pdfstartview=FitH}
\hypersetup{pdfpagemode=UseNone}
\hypersetup{pdfsource={}}
\hypersetup{pdflang={en-UK}}
\hypersetup{pdfcopyright={Copyright 2017-2018 Niklas Beisert.
  This work may be distributed and/or modified under the
  conditions of the LaTeX Project Public License, either version 1.3
  of this license or (at your option) any later version.}}
\hypersetup{pdflicenseurl={http://www.latex-project.org/lppl.txt}}
\hypersetup{pdfcontactaddress={ETH Zurich, ITP, HIT K,
  Wolfgang-Pauli-Strasse 27}}
\hypersetup{pdfcontactpostcode={8093}}
\hypersetup{pdfcontactcity={Zurich}}
\hypersetup{pdfcontactcountry={Switzerland}}
\hypersetup{pdfcontactemail={nbeisert@itp.phys.ethz.ch}}
\hypersetup{pdfcontacturl={http://people.phys.ethz.ch/\xmptilde nbeisert/}}

\newcommand{\secref}[1]{\hyperref[#1]{section \ref*{#1}}}

\parskip1ex
\parindent0pt
\let\olditemize\itemize
\def\itemize{\olditemize\parskip0pt}

\begin{document}

\title{The \textsf{childdoc} Package}
\hypersetup{pdftitle={The childdoc Package}}
\author{Niklas Beisert\\[2ex]
  Institut f\"ur Theoretische Physik\\
  Eidgen\"ossische Technische Hochschule Z\"urich\\
  Wolfgang-Pauli-Strasse 27, 8093 Z\"urich, Switzerland\\[1ex]
  \href{mailto:nbeisert@itp.phys.ethz.ch}
  {\texttt{nbeisert@itp.phys.ethz.ch}}}
\hypersetup{pdfauthor={Niklas Beisert}}
\hypersetup{pdfsubject={Manual for the LaTeX2e Package childdoc}}
\date{30 December 2018, \textsf{v2.0}}
\maketitle

\begin{abstract}\noindent
\textsf{childdoc} is a \LaTeXe{} package
that enables the direct compilation
of document sections included by |\include|
to individual files.
\end{abstract}

\begingroup
\parskip0ex
\tableofcontents
\endgroup

%%%%%%%%%%%%%%%%%%%%%%%%%%%%%%%%%%%%%%%%%%%%%%%%%%%%%%%%%%%%%%%%%%%%%%%%%%%%%%%%
%%%%%%%%%%%%%%%%%%%%%%%%%%%%%%%%%%%%%%%%%%%%%%%%%%%%%%%%%%%%%%%%%%%%%%%%%%%%%%%%
\section{Introduction}

\LaTeX{} provides a mechanism to structure a large document (such as a book)
into a main file and several child files (containing the chapters)
using the |\include| command.
This mechanism is beneficial for documents
which span hundreds of pages in order to
make the source file(s) more manageable.
Moreover, compilation can be restricted to
selected child files by means of the |\includeonly| command.
The latter feature can be used to reduce the compilation time while editing
(this was significantly more useful in the earlier days of \LaTeX{})
or to generate a smaller document which is easier to navigate.
Another application of |\includeonly| is to generate
documents consisting of selected parts of the complete document.

However, there are a few drawbacks of the plain |\include| mechanism:
\begin{itemize}
\item
The child files cannot be compiled on their own,
they can only be compiled via the main file.
A naive editing environment
(such as a text editor with an option
to have the current file processed by \LaTeX)
may require one to switch to the main file before compiling;
attempting to compile the child file produces errors.
\item
The main file must be modified (each time)
to adjust the |\includeonly| command
to the present needs. This easily leaves the main file in a messy state.
\item
The generated document will always carry the filename
of the main document. This is inconvenient if
several child files are to be compiled and
to be kept for distribution.
\end{itemize}

The present package provides a simple interface
to make child files individually compilable by \LaTeX{}.
Compiling a child file then has the same effect as compiling
the main file with an |\includeonly| command
to select the appropriate child.
Moreover the generated document will carry the name of the child
rather than the main file.
This resolves all three above issues.

This feature is meant to make the editing of books,
thesis documents and lecture notes somewhat more convenient.
However, the package can also be used efficiently for
composing a series of documents (such as exercise sheets)
which are typically distributed individually.
It then assists the author in generating the individual documents
(potentially in different versions)
as well as a document containing the collected series.
Another application is in developing style files
or other kinds of included material
where compilation of the style file could redirect
to a sample or test file.

%%%%%%%%%%%%%%%%%%%%%%%%%%%%%%%%%%%%%%%%%%%%%%%%%%%%%%%%%%%%%%%%%%%%%%%%%%%%%%%%
%%%%%%%%%%%%%%%%%%%%%%%%%%%%%%%%%%%%%%%%%%%%%%%%%%%%%%%%%%%%%%%%%%%%%%%%%%%%%%%%
\section{Usage}

First of all, the package \textsf{childdoc} is \emph{not} a standard
\LaTeXe{} |.sty| style file! Therefore it needs to be invoked in
a non-standard way.

%%%%%%%%%%%%%%%%%%%%%%%%%%%%%%%%%%%%%%%%%%%%%%%%%%%%%%%%%%%%%%%%%%%%%%%%%%%%%%%%
\subsection{Included Files}
\label{sec:include}

%%%%%%%%%%%%%%%%%%%%%%%%%%%%%%%%%%%%%%%%
\DescribeMacro{\childdocmain}
To use the package, add the commands
\begin{center}
\begin{tabular}{l}
|% \iffalse
%
% childdoc.dtx Copyright (C) 2017-2018 Niklas Beisert
%
% This work may be distributed and/or modified under the
% conditions of the LaTeX Project Public License, either version 1.3
% of this license or (at your option) any later version.
% The latest version of this license is in
%   http://www.latex-project.org/lppl.txt
% and version 1.3 or later is part of all distributions of LaTeX
% version 2005/12/01 or later.
%
% This work has the LPPL maintenance status `maintained'.
%
% The Current Maintainer of this work is Niklas Beisert.
%
% This work consists of the files childdoc.dtx and childdoc.ins
% and the derived files childdoc.def and cdocsamp.tex with
% cdocsch1.tex, cdocsch2.tex, cdocsdrf.tex, cdocsfn1.tex, cdocsfn2.tex.
%
%<package>\ifdefined\childdocmain\endinput\fi
%<package>\ProvidesFile{childdoc.def}[2018/12/30 v2.0 child document driver]
%<samplemain>\ProvidesFile{cdocsamp.tex}[2018/12/30 v2.0 sample for childdoc]
%<*driver>
%\ProvidesFile{childdoc.drv}[2018/12/30 v2.0 childdoc reference manual file]
\PassOptionsToClass{10pt,a4paper}{article}
\documentclass{ltxdoc}

\usepackage[margin=35mm]{geometry}
\usepackage{hyperref}
\usepackage{hyperxmp}
\usepackage[usenames]{color}

\hypersetup{colorlinks=true}
\hypersetup{pdfstartview=FitH}
\hypersetup{pdfpagemode=UseNone}
\hypersetup{pdfsource={}}
\hypersetup{pdflang={en-UK}}
\hypersetup{pdfcopyright={Copyright 2017-2018 Niklas Beisert.
  This work may be distributed and/or modified under the
  conditions of the LaTeX Project Public License, either version 1.3
  of this license or (at your option) any later version.}}
\hypersetup{pdflicenseurl={http://www.latex-project.org/lppl.txt}}
\hypersetup{pdfcontactaddress={ETH Zurich, ITP, HIT K,
  Wolfgang-Pauli-Strasse 27}}
\hypersetup{pdfcontactpostcode={8093}}
\hypersetup{pdfcontactcity={Zurich}}
\hypersetup{pdfcontactcountry={Switzerland}}
\hypersetup{pdfcontactemail={nbeisert@itp.phys.ethz.ch}}
\hypersetup{pdfcontacturl={http://people.phys.ethz.ch/\xmptilde nbeisert/}}

\newcommand{\secref}[1]{\hyperref[#1]{section \ref*{#1}}}

\parskip1ex
\parindent0pt
\let\olditemize\itemize
\def\itemize{\olditemize\parskip0pt}

\begin{document}

\title{The \textsf{childdoc} Package}
\hypersetup{pdftitle={The childdoc Package}}
\author{Niklas Beisert\\[2ex]
  Institut f\"ur Theoretische Physik\\
  Eidgen\"ossische Technische Hochschule Z\"urich\\
  Wolfgang-Pauli-Strasse 27, 8093 Z\"urich, Switzerland\\[1ex]
  \href{mailto:nbeisert@itp.phys.ethz.ch}
  {\texttt{nbeisert@itp.phys.ethz.ch}}}
\hypersetup{pdfauthor={Niklas Beisert}}
\hypersetup{pdfsubject={Manual for the LaTeX2e Package childdoc}}
\date{30 December 2018, \textsf{v2.0}}
\maketitle

\begin{abstract}\noindent
\textsf{childdoc} is a \LaTeXe{} package
that enables the direct compilation
of document sections included by |\include|
to individual files.
\end{abstract}

\begingroup
\parskip0ex
\tableofcontents
\endgroup

%%%%%%%%%%%%%%%%%%%%%%%%%%%%%%%%%%%%%%%%%%%%%%%%%%%%%%%%%%%%%%%%%%%%%%%%%%%%%%%%
%%%%%%%%%%%%%%%%%%%%%%%%%%%%%%%%%%%%%%%%%%%%%%%%%%%%%%%%%%%%%%%%%%%%%%%%%%%%%%%%
\section{Introduction}

\LaTeX{} provides a mechanism to structure a large document (such as a book)
into a main file and several child files (containing the chapters)
using the |\include| command.
This mechanism is beneficial for documents
which span hundreds of pages in order to
make the source file(s) more manageable.
Moreover, compilation can be restricted to
selected child files by means of the |\includeonly| command.
The latter feature can be used to reduce the compilation time while editing
(this was significantly more useful in the earlier days of \LaTeX{})
or to generate a smaller document which is easier to navigate.
Another application of |\includeonly| is to generate
documents consisting of selected parts of the complete document.

However, there are a few drawbacks of the plain |\include| mechanism:
\begin{itemize}
\item
The child files cannot be compiled on their own,
they can only be compiled via the main file.
A naive editing environment
(such as a text editor with an option
to have the current file processed by \LaTeX)
may require one to switch to the main file before compiling;
attempting to compile the child file produces errors.
\item
The main file must be modified (each time)
to adjust the |\includeonly| command
to the present needs. This easily leaves the main file in a messy state.
\item
The generated document will always carry the filename
of the main document. This is inconvenient if
several child files are to be compiled and
to be kept for distribution.
\end{itemize}

The present package provides a simple interface
to make child files individually compilable by \LaTeX{}.
Compiling a child file then has the same effect as compiling
the main file with an |\includeonly| command
to select the appropriate child.
Moreover the generated document will carry the name of the child
rather than the main file.
This resolves all three above issues.

This feature is meant to make the editing of books,
thesis documents and lecture notes somewhat more convenient.
However, the package can also be used efficiently for
composing a series of documents (such as exercise sheets)
which are typically distributed individually.
It then assists the author in generating the individual documents
(potentially in different versions)
as well as a document containing the collected series.
Another application is in developing style files
or other kinds of included material
where compilation of the style file could redirect
to a sample or test file.

%%%%%%%%%%%%%%%%%%%%%%%%%%%%%%%%%%%%%%%%%%%%%%%%%%%%%%%%%%%%%%%%%%%%%%%%%%%%%%%%
%%%%%%%%%%%%%%%%%%%%%%%%%%%%%%%%%%%%%%%%%%%%%%%%%%%%%%%%%%%%%%%%%%%%%%%%%%%%%%%%
\section{Usage}

First of all, the package \textsf{childdoc} is \emph{not} a standard
\LaTeXe{} |.sty| style file! Therefore it needs to be invoked in
a non-standard way.

%%%%%%%%%%%%%%%%%%%%%%%%%%%%%%%%%%%%%%%%%%%%%%%%%%%%%%%%%%%%%%%%%%%%%%%%%%%%%%%%
\subsection{Included Files}
\label{sec:include}

%%%%%%%%%%%%%%%%%%%%%%%%%%%%%%%%%%%%%%%%
\DescribeMacro{\childdocmain}
To use the package, add the commands
\begin{center}
\begin{tabular}{l}
|\input{childdoc.def}|\\
|\childdocmain{}|\\
\end{tabular}
\end{center}
at the very top of the main \LaTeX{} file,
in particular \emph{before} the |\documentclass| statement!
The argument of |\childdocmain| should be left empty
(but it must be present).

%%%%%%%%%%%%%%%%%%%%%%%%%%%%%%%%%%%%%%%%
\DescribeMacro{\childdocof}
Furthermore, add the commands
\begin{center}
\begin{tabular}{l}
|\input{childdoc.def}|\\
|\childdocof{|\textit{main}|}|\\
\end{tabular}
\end{center}
at the top of every child file \textit{child}
which is included by |\include{|\textit{child}|}|
from within the main file
(or at least for those files to be compiled individually).
The argument \textit{main} must be the filename of the main file.

There are a couple of
considerations in setting up the main and child documents:

%%%%%%%%%%%%%%%%%%%%%%%%%%%%%%%%%%%%%%%%
\paragraph{Restrictions.}

Please note the following restrictions:
\begin{itemize}
\item
|\childdocmain| must be called with one argument \textit{main}
to ensure compatibility with earlier version of the package.
It must either be empty (|\childdocmain{}|)
or precisely match the filename of the main file in which it is specified.
See \secref{sec:detection} for further information.
\item
The filename \textit{main} must be specified without the |.tex| extension.
\item
The filename \textit{main} is case sensitive
(even in case-insensitive file systems)
due to internal string comparison.
\item
The argument \textit{main} should be fully expanded, it cannot be a macro.
\item
Subdirectories and special characters should be avoided in filenames.
\item
The command |\childdocmain{|\textit{main}|}| must be followed by a whitespace.
It should not be followed immediately by another command
or by a comment mark `|%|'.
This is because the \TeX{} parser reads the token immediately following
the argument of |\childdocmain| and puts it
at the beginning of every child section;
however, a white\-space is ignored.
\end{itemize}

%%%%%%%%%%%%%%%%%%%%%%%%%%%%%%%%%%%%%%%%
\paragraph{Content of Main File.}

It is advisable to place all content in the child files included by |\include|.
Any output contained in the main file will appear in all child documents
unless suppressed manually;
it cannot be suppressed automatically by the |\includeonly| directive
and thus should normally be avoided.
A method to include some content in the main file
by means of conditional processing is described in \secref{sec:conditional}.

%%%%%%%%%%%%%%%%%%%%%%%%%%%%%%%%%%%%%%%%
\paragraph{Page Numbering.}

When only a part of the document is compiled,
the appropriate numbering of pages
(as well as other status parameters)
is determined from the |.aux| files.
The latter contain information from previous passes.
However this information needs to propagate through
all intermediate child documents.
Therefore the page numbering in child documents may well
be inconsistent until the complete document is compiled at least once.

A useful (if unconventional) way to always ensure a consistent
page numbering is to restart the numbering in each child document
and denote the pages by `\textit{child}|.|\textit{page}'
where \textit{child} represents the chapter/section number of the child file.
This can be achieved by the command
|\numberwithin{page}{|\textit{child}|}|
of the \textsf{amsmath} package
where \textit{child} can be |chapter| or |section|
depending on the chosen structuring.
Alternatively, one can modify the macro |\thepage| appropriately
and reset the counter |page| at the start of each child file.

%%%%%%%%%%%%%%%%%%%%%%%%%%%%%%%%%%%%%%%%%%%%%%%%%%%%%%%%%%%%%%%%%%%%%%%%%%%%%%%%
\subsection{Conditional Processing}
\label{sec:conditional}

The package provides a mechanism to compile different versions
of a document. To customise the versions further some conditional processing
can come in handy to distinguish which version is being compiled.
The package provides two macros to describe the compilation context:

%%%%%%%%%%%%%%%%%%%%%%%%%%%%%%%%%%%%%%%%
\DescribeMacro{\ifchilddoc}
The conditional |\ifchilddoc| distinguishes between the compilation of
child documents and the main document:
%
\begin{center}
|\ifchilddoc |\textit{child-code}| |[|\||else |\textit{main-code}]| \||fi|
\end{center}

%%%%%%%%%%%%%%%%%%%%%%%%%%%%%%%%%%%%%%%%
\DescribeMacro{\childdocname}
\DescribeMacro{\childdocjob}
The macro |\childdocname| contains the filename (without extension)
of the main or child file being processed.
Note that |\childdocjob| will always contain the name of the main file.

%%%%%%%%%%%%%%%%%%%%%%%%%%%%%%%%%%%%%%%%
\paragraph{Title Page.}

Conditional processing can be used to include a title or banner page
in the main document when proper precautions are taken.
Importantly, the code in the main file should ensure that the page counter
(as well as other status parameters which are stored in the |.aux| files)
takes the same value after the conditional processing.
Otherwise the page numbers may take divergent values
depending on which part is compiled.

For example, a title page could be declared by:
%
\begin{center}
\begin{tabular}{l}
|\ifchilddoc\||else|\\
|\addtocounter{page}{-1}|\\
\textit{code for title page}\\
|\newpage|\\
|\||fi|
\end{tabular}
\end{center}
%
A banner page for the child documents can be generated by:
%
\begin{center}
\begin{tabular}{l}
|\ifchilddoc|\\
|\addtocounter{page}{-1}|\\
\textit{code for banner page}\\
|\newpage|\\
|\||fi|
\end{tabular}
\end{center}
%
Here one could write a message such as:
\begin{center}
|This is the part \childdocname{} of \childdocjob{}.|
\end{center}

%%%%%%%%%%%%%%%%%%%%%%%%%%%%%%%%%%%%%%%%%%%%%%%%%%%%%%%%%%%%%%%%%%%%%%%%%%%%%%%%
\subsection{Flags}
\label{sec:flags}

The package makes it easy to generate different versions
of the main or child documents.
To this end compilation flags can be defined
and assigned different default values.
They will be particularly useful in conjunction
with the forwarding mechanism described in \secref{sec:forward}.

For example, it may be useful to have a flag |\version|
which can be set to |draft| or |final|.
The document source will contain some conditional code
depending on the value of |\version|.
Suppose further, the flag should default to |final| for the main file
and to |draft| for child files
which is a natural assignment for editing the document.
This is achieved by placing the following code
in the preamble of the main document
(below the |\childdocmain| directive):
%
\begin{center}
\begin{tabular}{l}
|\ifchilddoc|\\
|\providecommand{\version}{draft}|\\
|\||else|\\
|\providecommand{\version}{final}|\\
|\||fi|
\end{tabular}
\end{center}
%
The definition by |\providecommand| makes sure
that previous definitions are not overwritten.
Further statements |\providecommand{\version}{...}|
can thus be added before the above code to override it.

For the main file, one might add a line
(between |\childdocmain| and the above block)
%
\begin{center}
|%\ifchilddoc\||else\providecommand{\version}{draft}\||fi|
\end{center}
%
which can be uncommented to produce a draft version.
Likewise one can add a line to the very top of a child file
(above the |\childdocof{|\textit{main}|}| directive)
%
\begin{center}
|%\providecommand{\version}{final}|
\end{center}
%
which can be uncommented to produce the final version of this child document.

%%%%%%%%%%%%%%%%%%%%%%%%%%%%%%%%%%%%%%%%%%%%%%%%%%%%%%%%%%%%%%%%%%%%%%%%%%%%%%%%
\subsection{Forwarding}
\label{sec:forward}

Different versions of the main or child documents
using compilation flags as described in \secref{sec:flags}
can be (permanently) stored in different files
for convenient compilation, viewing and distribution.
To this end, the package defines a command
to pass on compilation to a different file:

%%%%%%%%%%%%%%%%%%%%%%%%%%%%%%%%%%%%%%%%
\DescribeMacro{\childdocforward}
The command |\childdocforward| redirects processing to
another source file:
%
\begin{center}
\begin{tabular}{l}
|\input{childdoc.def}|\\
|\childdocforward[|\textit{main}|]{|\textit{dest}|}|\\
\end{tabular}
\end{center}
%
The argument \textit{dest} is the destination file
(without extension).
It should be the main file or one of the child files.
Note that further \textsf{childdoc} directives
such as |\childdocof| and |\childdocforward|
in the indicated file will be processed in this form.
The optional argument \textit{main}
passes on directly to the main file \textit{main}
while pretending to compile the child \textit{dest}.
This form behaves as if \textit{dest}
issues |\childdocof{|\textit{main}|}| right away,
and no further \textsf{childdoc} directives will be processed.

%%%%%%%%%%%%%%%%%%%%%%%%%%%%%%%%%%%%%%%%
\DescribeMacro{\...prefix}
In the alternative form |\childdocforwardprefix|,
%
\begin{center}
\begin{tabular}{l}
|\input{childdoc.def}|\\
|\childdocforwardprefix[|\textit{main}|]{|\textit{prefix}|}{|\textit{dest}|}|
\end{tabular}
\end{center}
%
the destination file is determined by a pattern
depending on the current file:
To make this work, the current file must be called
`{\textit{prefix}\hspace{0.2em}\textit{suffix}}'
with \textit{prefix} matching precisely the argument.
Processing is then passed on to the file
`{\textit{dest}\hspace{0.2em}\textit{suffix}}'.
Surely, the same effect is achieved by
directly specifying the
argument `{\textit{dest}\hspace{0.2em}\textit{suffix}}'
in the first form.
However, that requires to set up a different file
for each child. With the alternative form of the command
all these files can have exactly the same content
which simplifies setting them up and maintaining them.

For example, the following file |draft.tex|
with a compilation flag |\version| as described in \secref{sec:flags}
compiles the main document as a draft:
%
\begin{center}
\begin{tabular}{l}
|\def\version{draft}|\\
|\input{childdoc.def}|\\
|\childdocforward{|\textit{main}|}|
\end{tabular}
\end{center}
%
Likewise, the following files |final|\textit{nn}|.tex|
compile the final version of the child document
|child|\textit{nn}|.tex|:
%
\begin{center}
\begin{tabular}{l}
|\def\version{final}|\\
|\input{childdoc.def}|\\
|\childdocforwardprefix{final}{child}|
\end{tabular}
\end{center}
%

Note that when several versions of a main file and/or of each child file
are to be generated, it may be convenient to set up a |Makefile| or
shell script to automatise the process.

%%%%%%%%%%%%%%%%%%%%%%%%%%%%%%%%%%%%%%%%%%%%%%%%%%%%%%%%%%%%%%%%%%%%%%%%%%%%%%%%
\subsection{Command Line Processing}
\label{sec:commandline}

The effect of redirection files can also be achieved by invoking
the \LaTeX{} compiler with a more elaborate command line.
Most conveniently this should be done as part
of a shell script or a |Makefile|.

When using \textsf{childdoc} in the main file, the following
command lines effectively perform a redirection
(note that depending on the shell being used,
backslashes may have to be doubled: `|\|' $\to$ `|\\|'):
%
\begin{center}
|... -jobname "|\textit{target}|" |\\|"|[\textit{flags}]%
|\input{childdoc.def}\childdocforward[|\textit{main}|]{|\textit{dest}|}"|
\end{center}
%
Here \textit{target} is the name of the output file,
\textit{main} is the name of the main file
and \textit{dest} is the name of the main or child file to be processed
(all filenames without extensions).
The optional argument \textit{main} can be omitted
if \textit{main} matches \textit{dest}.
Optionally, compilation \textit{flags} can be defined via |\def| commands.
This command line makes the \TeX{} engine believe
it is compiling the file \textit{target}
whose content is specified as the latter parameter.
The provided code then forwards the processing to
\textit{main} or \textit{dest} as described in \secref{sec:forward}.

%%%%%%%%%%%%%%%%%%%%%%%%%%%%%%%%%%%%%%%%%%%%%%%%%%%%%%%%%%%%%%%%%%%%%%%%%%%%%%%%
\subsection{Include by Input}
\label{sec:input}

Including child documents by |\include| has some restrictions by design.
Most notably, the content of a child document always occupies
its own set of pages; pages cannot be shared between child documents.
Usually, this behaviour makes perfect sense
because each child document contain an essential part of the document.
However, in some situations it may be desirable to compose
a document from a collection of parts
without having mandatory page breaks between then.
For this case, the package
provides a mechanism to include parts
by |\input| which can also be processed individually.
However, by construction this mechanism
requires manual handling of the content to be output.

%%%%%%%%%%%%%%%%%%%%%%%%%%%%%%%%%%%%%%%%
\DescribeMacro{\ifchilddocmanual}
The main file should be prepared as usual, see \secref{sec:include}.
However, the document body must make a distinction
between processing of an individual part and of the main document, e.g.:
%
\begin{center}
\begin{tabular}{l}
|\ifchilddocmanual|\\
|\input{\childdocname}|\\
|\||else|\\
\textit{document body with }|\input{|\textit{part}|}|\\
|\||fi|
\end{tabular}
\end{center}
%
The conditional |\ifchilddocmanual| is true whenever
a part to be included by |\input| is being compiled,
and the name of the part is stored in |\childdocname|.

%%%%%%%%%%%%%%%%%%%%%%%%%%%%%%%%%%%%%%%%
\DescribeMacro{\childdocby}
Each part to be included by |\input| should start with:
%
\begin{center}
\begin{tabular}{l}
|\input{childdoc.def}|\\
|\childdocby{|\textit{main}|}|\\
\end{tabular}
\end{center}
%
The directive |\childdocby| is similar to |\childdocof|
described in \secref{sec:include},
but the subsequent selection of content must be done manually.
To that end, both |\ifchilddoc| and |\ifchilddocmanual|
will be true upon processing of a part,
and the name of the part is stored in |\childdocname|.
Note that |\jobname| will be set to the filename of the current part
so that each part receives an individual |.aux| file
that does not interfere with the |.aux| file(s) of the main document.
This behaviour can be altered by the alternative form
|\childdocby[*]{|\textit{main}|}| (with a non-empty optional argument)
which uses the |.aux| file of the main document
by setting |\jobname| to \textit{main}.

%%%%%%%%%%%%%%%%%%%%%%%%%%%%%%%%%%%%%%%%%%%%%%%%%%%%%%%%%%%%%%%%%%%%%%%%%%%%%%%%
\subsection{Driver Development}
\label{sec:driver}

The \textsf{childdoc} mechanism can also be use for the development
of definition files such as \LaTeX{} styles or classes.
This case differs from the above setup with multiple parts
included by |\include| in that no |\includeonly| should be invoked.
This can be achieved by starting the include file
(before |\ProvidesPackage|) with:
%
\begin{center}
\begin{tabular}{l}
|\input{childdoc.def}|\\
|\childdocforward{|\textit{main}|}|\\
\end{tabular}
\end{center}
%
or alternatively with:
%
\begin{center}
\begin{tabular}{l}
|\input{childdoc.def}|\\
|\childdocby{|\textit{main}|}|\\
\end{tabular}
\end{center}
%
Both forms have slightly different effects as described above.
The main file is prepared as usual, see \secref{sec:include}.

%%%%%%%%%%%%%%%%%%%%%%%%%%%%%%%%%%%%%%%%%%%%%%%%%%%%%%%%%%%%%%%%%%%%%%%%%%%%%%%%
\subsection{Legacy Detection}
\label{sec:detection}

The directive |\childdocmain| in the main file can detect
whether the complete document or merely a child is to be compiled
even without using the directive |\childdocof|.
This method is deprecated because it is less robust
and there is no compelling reason to use it;
it is merely provided for backward compatibility
and it may be removed in future versions.

If the detection mechanism is to be used,
it is mandatory to correctly specify
the filename of the main file as the argument of |\childdocmain|:
%
\begin{center}
\begin{tabular}{l}
|\input{childdoc.def}|\\
|\childdocmain{|\textit{main}|}|\\
\end{tabular}
\end{center}
%
If |\jobname| does not match the argument \textit{main} of |\childdocmain|,
it is assumed that |\jobname| points to the child file to be compiled.
When using |\childdocmain| with the main file specified as argument,
it suffices to start a child file
with just |\input{|\textit{main}|}|
without loading of the package and using |\childdocof|.
If instead all processing is done
with the appropriate \textsf{childdoc} directives,
the argument of \textit{main} of |\childdocmain| can be empty.

An alternative version of the command line processing described
in \secref{sec:commandline} using the detection mechanism reads:
%
\begin{center}
|... -jobname "|\textit{target}|" "|[\textit{flags}]%
[|\def\jobname{|\textit{dest}|}|]|\input{|\textit{main}|}"|
\end{center}

%%%%%%%%%%%%%%%%%%%%%%%%%%%%%%%%%%%%%%%%%%%%%%%%%%%%%%%%%%%%%%%%%%%%%%%%%%%%%%%%
\subsection{Manual Code}
\label{sec:manual}

In case one cannot be certain whether the definitions file |childdoc.def|
is installed on the target \TeX{} distribution
and one prefers not to ship it,
it is conceivable to paste a few relevant commands into the sources.

To that end, drop all statements |\input{childdoc.def}|
and perform the replacements as outlined below.
Instead of |\childdocmain{|\textit{main}|}| add the following code
to the top of the main file:
%
\begin{center}
\begin{tabular}{l}
|\||ifdefined\childdocname\endinput\||fi\newif\ifchilddoc|\\
|\edef\childdocname{\scantokens\expandafter{\jobname\noexpand}}|\\
|\def\childdocmain{|\textit{main}|}\||ifx\childdocmain\childdocname\||else|\\
|\childdoctrue\includeonly{\childdocname}\let\jobname\childdocmain\||fi|\\
\end{tabular}
\end{center}
%
Instead of |\childdocof{|\textit{main}|}| just include the main file
at the top of each child file:
%
\begin{center}
|\input{|\textit{main}|}|
\end{center}
%
A simple redirection |\childdocforward{|\textit{dest}|}| is achieved by:
%
\begin{center}
|\def\jobname{|\textit{dest}|}\input{\jobname}|
\end{center}
%
The redirection with prefix
|\childdocforwardprefix[|\textit{prefix}|]{|\textit{dest}|}|
is accomplished by:
%
\begin{center}
\begin{tabular}{l}
|{\edef\jobname{\scantokens\expandafter{\jobname\noexpand}}|\\
|\def\redirectjob |\textit{prefix}|#1~~~{\gdef\jobname{|\textit{dest}|#1}}|\\
|\expandafter\redirectjob\jobname~~~}\input{\jobname}|
\end{tabular}
\end{center}

In an alternative approach,
child documents can be compiled by a specific command line
without additional code or specific definitions:
%
\begin{center}
|... -jobname "|\textit{target}|" "|[\textit{flags}]%
|\includeonly{|\textit{dest}|}\input{|\textit{main}|}"|
\end{center}
%

%%%%%%%%%%%%%%%%%%%%%%%%%%%%%%%%%%%%%%%%%%%%%%%%%%%%%%%%%%%%%%%%%%%%%%%%%%%%%%%%
%%%%%%%%%%%%%%%%%%%%%%%%%%%%%%%%%%%%%%%%%%%%%%%%%%%%%%%%%%%%%%%%%%%%%%%%%%%%%%%%
\section{Information}

%%%%%%%%%%%%%%%%%%%%%%%%%%%%%%%%%%%%%%%%%%%%%%%%%%%%%%%%%%%%%%%%%%%%%%%%%%%%%%%%
\subsection{Copyright}

Copyright \copyright{} 2017--2018 Niklas Beisert

This work may be distributed and/or modified under the
conditions of the \LaTeX{} Project Public License, either version 1.3
of this license or (at your option) any later version.
The latest version of this license is in
  \url{http://www.latex-project.org/lppl.txt}
and version 1.3 or later is part of all distributions of \LaTeX{}
version 2005/12/01 or later.

This work has the LPPL maintenance status `maintained'.

The Current Maintainer of this work is Niklas Beisert.

This work consists of the files |README.txt|, |childdoc.ins| and |childdoc.dtx|
as well as the derived files |childdoc.def|, |cdocsamp.tex|
with |cdocsch1.tex|, |cdocsch2.tex|, |cdocspt3.tex|, |cdocspt4.tex|,
|cdocsdrf.tex|, |cdocsfn1.tex|, |cdocsfn2.tex|
as well as |childdoc.pdf|.

%%%%%%%%%%%%%%%%%%%%%%%%%%%%%%%%%%%%%%%%%%%%%%%%%%%%%%%%%%%%%%%%%%%%%%%%%%%%%%%%
\subsection{Files and Installation}

The package consists of the files:
%
\begin{center}
\begin{tabular}{ll}
    |README.txt|   & readme file \\
    |childdoc.ins| & installation file \\
    |childdoc.dtx| & source file \\
    |childdoc.def| & definition file \\
    |cdocsamp.tex| & sample main file \\
    |cdocsch1.tex| & sample include file \\
    |cdocsch2.tex| & sample include file \\
    |cdocspt3.tex| & sample part file \\
    |cdocspt4.tex| & sample part file \\
    |cdocsdrf.tex| & sample redirection file \\
    |cdocsfn1.tex| & sample redirection file \\
    |cdocsfn2.tex| & sample redirection file \\
    |childdoc.pdf| & manual
\end{tabular}
\end{center}
%
The distribution consists of the files
|README.txt|, |childdoc.ins| and |childdoc.dtx|.
%
\begin{itemize}
\item
Run (pdf)\LaTeX{} on |childdoc.dtx|
to compile the manual |childdoc.pdf| (this file).
\item
Run \LaTeX{} on |childdoc.ins| to create the definitions file |childdoc.def|
and the sample |cdocsamp.tex| with include files
|cdocsch1.tex|, |cdocsch2.tex|, |cdocspt3.tex|, |cdocspt4.tex|,
|cdocsdrf.tex|, |cdocsfn1.tex|, |cdocsfn2.tex|.
Then copy the file |childdoc.def| to an appropriate directory of your \LaTeX{}
distribution, e.g.\ \textit{texmf-root}|/tex/latex/childdoc|.
\end{itemize}

%%%%%%%%%%%%%%%%%%%%%%%%%%%%%%%%%%%%%%%%%%%%%%%%%%%%%%%%%%%%%%%%%%%%%%%%%%%%%%%%
\subsection{Related CTAN Packages}

There are several other packages which offer a similar functionality:
%
\begin{itemize}
\item
The packages
\href{http://ctan.org/pkg/docmute}{\textsf{docmute}},
\href{http://ctan.org/pkg/includex}{\textsf{includex}} and
\href{http://ctan.org/pkg/standalone}{\textsf{standalone}}
provide commands to include only the document body of
a child file thus allowing both files to be compiled individually.
\item
The packages \href{http://ctan.org/pkg/subdocs}{\textsf{subdocs}}
and \href{http://ctan.org/pkg/subfiles}{\textsf{subfiles}}
provide structures in which the main and child documents can be
encapsulated and allowing them to be compiled individually.
The inclusion mechanism is different from the conventional |\include|.
\item
The package \href{http://ctan.org/pkg/combine}{\textsf{combine}}
is an elaborate solution to combine several documents into one.
\end{itemize}
%
See also the CTAN topic \href{http://ctan.org/topic/subdocs}{\textsf{subdocs}}
for further related packages.
The present package differs from the above solutions in that
a document structure constructed with the conventional |\include| mechanism
just needs two extra commands at the top of every file
such that all constituent files can be compiled individually.

%%%%%%%%%%%%%%%%%%%%%%%%%%%%%%%%%%%%%%%%%%%%%%%%%%%%%%%%%%%%%%%%%%%%%%%%%%%%%%%%
%\subsection{Feature Suggestions}
%
%The following is a list of features which may be useful for future
%versions of this package:
%%
%\begin{itemize}
%\item
%\ldots
%\end{itemize}

%%%%%%%%%%%%%%%%%%%%%%%%%%%%%%%%%%%%%%%%%%%%%%%%%%%%%%%%%%%%%%%%%%%%%%%%%%%%%%%%
\subsection{Revision History}

%%%%%%%%%%%%%%%%%%%%%%%%%%%%%%%%%%%%%%%%
\paragraph{v2.0:} 2018/12/30

\begin{itemize}
\item
immediate forward processing
\item
added |\childdocby| mechanism
\item
manual restructured
\end{itemize}

%%%%%%%%%%%%%%%%%%%%%%%%%%%%%%%%%%%%%%%%
\paragraph{v1.6:} 2018/01/17

\begin{itemize}
\item
application for development of include files
\item
corrections to manual
\end{itemize}

%%%%%%%%%%%%%%%%%%%%%%%%%%%%%%%%%%%%%%%%
\paragraph{v1.5:} 2017/05/21

\begin{itemize}
\item
more complete structuring introduced
\item
|\childdocof| introduced
\item
|\childdoc| renamed to |\childdocmain|
\item
|\childredirect| renamed to |\childdocforward| and |\childdocforwardprefix|
and functionality expanded
\end{itemize}

%%%%%%%%%%%%%%%%%%%%%%%%%%%%%%%%%%%%%%%%
\paragraph{v1.0:} 2017/04/27

\begin{itemize}
\item
manual and install package
\item
first version published on CTAN
\end{itemize}

%%%%%%%%%%%%%%%%%%%%%%%%%%%%%%%%%%%%%%%%
\paragraph{v0.6:} 2017/04/26

\begin{itemize}
\item
redirection mechanism added
\end{itemize}

%%%%%%%%%%%%%%%%%%%%%%%%%%%%%%%%%%%%%%%%
\paragraph{v0.5:} 2017/04/26

\begin{itemize}
\item
functionality in definition file
\end{itemize}


%%%%%%%%%%%%%%%%%%%%%%%%%%%%%%%%%%%%%%%%%%%%%%%%%%%%%%%%%%%%%%%%%%%%%%%%%%%%%%%%
%%%%%%%%%%%%%%%%%%%%%%%%%%%%%%%%%%%%%%%%%%%%%%%%%%%%%%%%%%%%%%%%%%%%%%%%%%%%%%%%
%%%%%%%%%%%%%%%%%%%%%%%%%%%%%%%%%%%%%%%%%%%%%%%%%%%%%%%%%%%%%%%%%%%%%%%%%%%%%%%%
\appendix

\settowidth\MacroIndent{\rmfamily\scriptsize 000\ }

 \DocInput{childdoc.dtx}

\end{document}
%</driver>
% \fi
%
% %%%%%%%%%%%%%%%%%%%%%%%%%%%%%%%%%%%%%%%%%%%%%%%%%%%%%%%%%%%%%%%%%%%%%%%%%%%%%%
% %%%%%%%%%%%%%%%%%%%%%%%%%%%%%%%%%%%%%%%%%%%%%%%%%%%%%%%%%%%%%%%%%%%%%%%%%%%%%%
% \section{Sample}
%\iffalse
%<*samplemain>
%\fi
%
% The following presents a sample document
% with two chapters, two parts, a title page,
% a compile flag as well as three forwarding files to set the flag.
% It consists of eight |.tex| files:
% \begin{center}
% \begin{tabular}{ll}
% |cdocsamp.tex|&main file\\
% |cdocsch1.tex|&include file for chapter 1\\
% |cdocsch2.tex|&include file for chapter 2\\
% |cdocspt3.tex|&include file for part 3\\
% |cdocspt4.tex|&include file for part 4\\
% |cdocsdrf.tex|&forwarding file for main file in draft mode\\
% |cdocsfi1.tex|&forwarding file for final version of chapter 1\\
% |cdocsfi2.tex|&forwarding file for final version of chapter 2\\
% \end{tabular}
% \end{center}
% Each of the eight files can be compiled directly by the \LaTeX{} compiler.
%
% %%%%%%%%%%%%%%%%%%%%%%%%%%%%%%%%%%%%%%
% \paragraph{Main File.}
%
% The main file is called |cdocsamp.tex|.
%
% Load the \textsf{childdoc} definitions and
% declare the filename for the main document:
%    \begin{macrocode}
\input{childdoc.def}
\childdocmain{}
%    \end{macrocode}

% Optional override for |\version| flag:
%    \begin{macrocode}
%%\ifchilddoc\else\providecommand{\version}{draft}\fi
%    \end{macrocode}

% Define the default values for the |\version| flag
% (|final| for the main file and |draft| for childs):
%    \begin{macrocode}
\ifchilddoc
\providecommand{\version}{draft}
\else
\providecommand{\version}{final}
\fi
%    \end{macrocode}

% Load the standard document class:
%    \begin{macrocode}
\documentclass[12pt]{article}
%    \end{macrocode}

% Start the document body:
%    \begin{macrocode}
\begin{document}
%    \end{macrocode}

% Declare a title page.
% Print title, part of document being processed and version flag:
%    \begin{macrocode}
\addtocounter{page}{-1}
\begin{center}
{\LARGE\bfseries{}childdoc example\par}
\vspace{1cm}
\ifchilddoc
\ifchilddocmanual part\else chapter\fi:
`\childdocname' of `\childdocjob'\par
\else
main document: `\childdocjob'\par
\fi
version: \version\par
\end{center}
\newpage
%    \end{macrocode}

% Manually include selected file,
% otherwise process as usual:
%    \begin{macrocode}
\ifchilddocmanual
\section*{part `\childdocname'}
\input{\childdocname}
\else
%    \end{macrocode}

% Include the two chapters:
%    \begin{macrocode}
\include{cdocsch1}
\include{cdocsch2}
%    \end{macrocode}

% Include the two parts unless only chapters should be displayed:
%    \begin{macrocode}
\ifchilddoc\else
\section{part three}
\input{cdocspt3}
\section{part four}
\input{cdocspt4}
\fi
%    \end{macrocode}

% Process as usual until here:
%    \begin{macrocode}
\fi
%    \end{macrocode}

% End of document body:
%    \begin{macrocode}
\end{document}
%    \end{macrocode}
%\iffalse
%</samplemain>
%\fi
%
% %%%%%%%%%%%%%%%%%%%%%%%%%%%%%%%%%%%%%%
% \paragraph{Chapter Include Files.}
%
% The include files are called |cdocsch1.tex| and |cdocsch2.tex|.
%
%\iffalse
%<*samplechap1|samplechap2>
%\fi

% Optional override for |\version| flag:
%    \begin{macrocode}
%%\providecommand{\version}{final}
%    \end{macrocode}

% Include the main document:
%    \begin{macrocode}
\input{childdoc.def}
\childdocof{cdocsamp}
%    \end{macrocode}

%\iffalse
%</samplechap1|samplechap2>
%\fi
%
%\iffalse
%<*samplechap1>
%\fi
% Some text for chapter 1:
%    \begin{macrocode}
\section{one}
some text in chapter one
%    \end{macrocode}

%\iffalse
%</samplechap1>
%\fi
% Some text for chapter 2:
%\iffalse
%<*samplechap2>
%\fi
%    \begin{macrocode}
\section{two}
more text in chapter two
%    \end{macrocode}

%\iffalse
%</samplechap2>
%\fi
%
% %%%%%%%%%%%%%%%%%%%%%%%%%%%%%%%%%%%%%%
% \paragraph{Part Include Files.}
%
% The include files are called |cdocspt3.tex| and |cdocspt4.tex|.
%
%\iffalse
%<*samplepart3|samplepart4>
%\fi

% Optional override for |\version| flag:
%    \begin{macrocode}
%%\providecommand{\version}{final}
%    \end{macrocode}

% Include the main document:
%    \begin{macrocode}
\input{childdoc.def}
\childdocby{cdocsamp}
%    \end{macrocode}

%\iffalse
%</samplepart3|samplepart4>
%\fi
%
%\iffalse
%<*samplepart3>
%\fi
% Some text for part 3:
%    \begin{macrocode}
some text in part three
%    \end{macrocode}

%\iffalse
%</samplepart3>
%\fi
% Some text for part 4:
%\iffalse
%<*samplepart4>
%\fi
%    \begin{macrocode}
more text in part four
%    \end{macrocode}

%\iffalse
%</samplepart4>
%\fi
%
% %%%%%%%%%%%%%%%%%%%%%%%%%%%%%%%%%%%%%%
% \paragraph{Forwarding for a Complete Draft.}
%
% The following forwarding file |cdocsdrf.tex|
% compiles the main document in draft mode:
%\iffalse
%<*sampledraft>
%\fi
%    \begin{macrocode}
\def\version{draft}
\input{childdoc.def}
\childdocforward{cdocsamp}
%    \end{macrocode}

%\iffalse
%</sampledraft>
%\fi
%
% %%%%%%%%%%%%%%%%%%%%%%%%%%%%%%%%%%%%%%
% \paragraph{Forwarding for Final Version of the Chapters.}
%
% The following forwarding files |cdocsfn1.tex| and |cdocsfn2.tex|
% (with identical content)
% compile the final versions of the child documents
% |cdocsch1.tex| and |cdocsch2.tex|, respectively:
%\iffalse
%<*samplefinal>
%\fi
%    \begin{macrocode}
\def\version{final}
\input{childdoc.def}
\childdocforwardprefix[cdocsamp]{cdocsfn}{cdocsch}
%    \end{macrocode}

%\iffalse
%</samplefinal>
%\fi
%
% %%%%%%%%%%%%%%%%%%%%%%%%%%%%%%%%%%%%%%
% \paragraph{Command Line Processing.}
%
% The following three command lines generate the output files
% |cdocscld|, |cdocscl1| and |cdocscl2|
% which should be identical to
% |cdocsdrf|, |cdocsch1| and |cdocsfn2|, respectively:
% \begin{center}
% \begin{tabular}{l}
% |latex -jobname cdocscld \|\\
% |  "\def\version{draft}\input{childdoc.def}\childdocforward{cdocsamp}"|\\
% |latex -jobname cdocscl1 \|\\
% |  "\input{childdoc.def}\childdocforward[cdocsamp]{cdocsch1}"|\\
% |latex -jobname cdocscl2 \|\\
% |  "\def\version{final}\input{childdoc.def}\childdocforward{cdocsch2}"|
% \end{tabular}
% \end{center}
% Note that the trailing backslash on each first line
% merely continues the input to the second line
% (for convenient cut ant paste).
% Furthermore, the command |latex| can be replaced by any
% of its alternative versions such as |pdflatex|.
%
% %%%%%%%%%%%%%%%%%%%%%%%%%%%%%%%%%%%%%%%%%%%%%%%%%%%%%%%%%%%%%%%%%%%%%%%%%%%%%%
% %%%%%%%%%%%%%%%%%%%%%%%%%%%%%%%%%%%%%%%%%%%%%%%%%%%%%%%%%%%%%%%%%%%%%%%%%%%%%%
% \section{Implementation}
%\iffalse
%<*package>
%\fi
%
% This section describes the definitions file |childdoc.def|.

% The definitions cannot be loaded using |\usepackage| or |\RequirePackage|
% which has a mechanism to prevent loading a style file more than once.
% When loading the definitions by means of |\input|
% multiple instances have to be prevented manually:
%\iffalse
%This code needs to be before the `\ProvidesFile' directive
%which is defined at the beginning of this file.
%Therefore it is also placed there and commented out here.
%</package>
%<*discard>
%\fi
%    \begin{macrocode}
\ifdefined\childdocmain\endinput\fi
%    \end{macrocode}
%\iffalse
%</discard>
%<*package>
%\fi
%
% \macro{\ifchilddoc}
% \macro{\ifchilddocmanual}
% The conditional |\ifchilddoc| tells whether a
% child (true) or main (false) document is being compiled.
% The conditional |\ifchilddocmanual| tells whether
% the |\includeonly| mechanism is used (false) or
% the selection of child files must be performed manually (true).
% The definitions initialise to false:
%    \begin{macrocode}
\newif\ifchilddoc
\newif\ifchilddocmanual
%    \end{macrocode}

% \macro{\childdocname}
% \macro{\childdocjob}
% The macro |\childdocname| stores the name of the main document
% to be compiled. The macro |\childdocjob| stores the name of
% the document on which the \LaTeX{} compiler was originally invoked.
% The content of |\jobname| cannot be compared
% to filenames specified in the source due to different catcodes.
% The following code rescans |\jobname|, stores the result
% in |\childdocname| and saves a copy in |\childdocjob|:
%    \begin{macrocode}
\edef\childdocname{\scantokens\expandafter{\jobname\noexpand}}
\let\childdocjob\childdocname
%    \end{macrocode}

% \macro{\childdocdisable}
% The macro |\childdocdisable| prevents the main file
% from being processed more than once.
% At this stage, the main document command |\childdocmain|
% is assumed to be called once again where it should do nothing.
% Any subsequent call to it should prevent
% a secondary processing of the main document
% It overwrites the forwarding commands
% |\childdocof| and |\childdocforward|
% with empty macros to prevent further inclusions of the main document:
%    \begin{macrocode}
\newcommand{\childdocdisable}
{
  \renewcommand{\childdocmain}[1]{\renewcommand{\childdocmain}[1]{\endinput}}
  \renewcommand{\childdocof}[1]{}
  \renewcommand{\childdocby}[2][]{}
  \renewcommand{\childdocforward}[2][]{}
  \renewcommand{\childdocdisable}{}
}
%    \end{macrocode}

% \macro{\childdocmain}
% The macro |\childdocmain| is to be called at the top of the main file
% with nothing or the main filename (without extension) as argument.
% First, it breaks loops.
% If the argument is not empty and does not match |\childdocname|
% (which is set by the first inclusion of |childdoc.def|),
% |\ifchilddoc| is set to true, |\includeonly| is applied to the child file
% and |\jobname| is set to the main file
% (for proper handling of |.aux| files):
%    \begin{macrocode}
\newcommand{\childdocmain}[1]
{
  \childdocdisable\childdocmain{}
  \if?#1?\else
    \begingroup
      \def\childdoctmp{#1}
      \ifx\childdoctmp\childdocname
        \def\childdoctmp{}
      \else
        \def\childdoctmp
        {
          \childdoctrue
          \includeonly{\childdocname}
          \def\childdocjob{#1}
          \def\jobname{#1}
        }
      \fi
      \expandafter
    \endgroup
    \childdoctmp
  \fi
}
%    \end{macrocode}

% \macro{\childdocof}
% The command |\childdocof| redirects
% compilation to the main file |#1|.
%    \begin{macrocode}
\newcommand{\childdocof}[1]
{
  \childdocdisable
  \childdoctrue
  \includeonly{\childdocname}
  \def\jobname{#1}
  \def\childdocjob{#1}
  \input{#1}
}
%    \end{macrocode}

% \macro{\childdocby}
% The command |\childdocby| ....
%    \begin{macrocode}
\newcommand{\childdocby}[2][]
{
  \childdocdisable
  \childdoctrue
  \childdocmanualtrue
  \if?#1?\else
    \def\jobname{#2}
  \fi
  \def\childdocjob{#2}
  \input{#2}
  \endinput
}
%    \end{macrocode}

% \macro{\childdocforward}
% The command |\childdocforward| redirects
% compilation to the main file or
% (if the optional argument is given) a child file.
% Parameters are set as if the main file
% or a child file starting with |\childdocof| was compiled.
% Then compilation is handed over to the main file:
%    \begin{macrocode}
\newcommand{\childdocforward}[2][]
{
  \begingroup
    \if?#1?
      \def\childdoctmp
      {
        \def\childdocname{#2}
        \def\childdocjob{#2}
        \def\jobname{#2}
        \input{#2}
        \endinput
      }
    \else
      \def\childdoctmp
      {
        \childdocdisable
        \def\childdocname{#2}
        \childdoctrue
        \includeonly{#2}
        \def\childdocjob{#1}
        \def\jobname{#1}
        \input{#1}
        \endinput
      }
    \fi
    \expandafter
  \endgroup
  \childdoctmp
}
%    \end{macrocode}

% \macro{\childdocforwardprefix}
% The command |\childdocforwardprefix| redirects
% compilation to the main or a child file by means of a pattern.
% The prefix |#1| in the current filename is replaced by |#2|
% and the suffix of the current filename is kept
% (it is assumed that the filename does not contain the substring `|~~~|'
% which is used as a delimiter).
% Compilation is handed over to the new file by |\childdocforward|:
%    \begin{macrocode}
\newcommand{\childdocforwardprefix}[3][]
{
  \begingroup
    \def\childdocextract #2##1~~~{\def\childdoctmp{\childdocforward[#1]{#3##1}}}
    \expandafter\childdocextract\childdocname~~~
    \expandafter
  \endgroup
  \childdoctmp
}
%    \end{macrocode}

% \macro{\childdoc}
% The deprecated macro |\childdoc| is a legacy version of |\childdocmain|:
%    \begin{macrocode}
\newcommand{\childdoc}{\childdocmain}
%    \end{macrocode}

% \macro{\childdocredirect}
% The deprecated macro |\childdocredirect| is a legacy version
% of |\childdocforward| and |\childdocforwardprefix|:
%    \begin{macrocode}
\newcommand{\childdocredirect}[2][]
{
  \begingroup
    \if?#1?
      \def\childdoctmp{\childdocforward{#2}}
    \else
      \def\childdoctmp{\childdocforwardprefix{#1}{#2}}
    \fi
    \expandafter
  \endgroup
  \childdoctmp
}
%    \end{macrocode}

%\iffalse
%</package>
%\fi
%
\endinput
|\\
|\childdocmain{}|\\
\end{tabular}
\end{center}
at the very top of the main \LaTeX{} file,
in particular \emph{before} the |\documentclass| statement!
The argument of |\childdocmain| should be left empty
(but it must be present).

%%%%%%%%%%%%%%%%%%%%%%%%%%%%%%%%%%%%%%%%
\DescribeMacro{\childdocof}
Furthermore, add the commands
\begin{center}
\begin{tabular}{l}
|% \iffalse
%
% childdoc.dtx Copyright (C) 2017-2018 Niklas Beisert
%
% This work may be distributed and/or modified under the
% conditions of the LaTeX Project Public License, either version 1.3
% of this license or (at your option) any later version.
% The latest version of this license is in
%   http://www.latex-project.org/lppl.txt
% and version 1.3 or later is part of all distributions of LaTeX
% version 2005/12/01 or later.
%
% This work has the LPPL maintenance status `maintained'.
%
% The Current Maintainer of this work is Niklas Beisert.
%
% This work consists of the files childdoc.dtx and childdoc.ins
% and the derived files childdoc.def and cdocsamp.tex with
% cdocsch1.tex, cdocsch2.tex, cdocsdrf.tex, cdocsfn1.tex, cdocsfn2.tex.
%
%<package>\ifdefined\childdocmain\endinput\fi
%<package>\ProvidesFile{childdoc.def}[2018/12/30 v2.0 child document driver]
%<samplemain>\ProvidesFile{cdocsamp.tex}[2018/12/30 v2.0 sample for childdoc]
%<*driver>
%\ProvidesFile{childdoc.drv}[2018/12/30 v2.0 childdoc reference manual file]
\PassOptionsToClass{10pt,a4paper}{article}
\documentclass{ltxdoc}

\usepackage[margin=35mm]{geometry}
\usepackage{hyperref}
\usepackage{hyperxmp}
\usepackage[usenames]{color}

\hypersetup{colorlinks=true}
\hypersetup{pdfstartview=FitH}
\hypersetup{pdfpagemode=UseNone}
\hypersetup{pdfsource={}}
\hypersetup{pdflang={en-UK}}
\hypersetup{pdfcopyright={Copyright 2017-2018 Niklas Beisert.
  This work may be distributed and/or modified under the
  conditions of the LaTeX Project Public License, either version 1.3
  of this license or (at your option) any later version.}}
\hypersetup{pdflicenseurl={http://www.latex-project.org/lppl.txt}}
\hypersetup{pdfcontactaddress={ETH Zurich, ITP, HIT K,
  Wolfgang-Pauli-Strasse 27}}
\hypersetup{pdfcontactpostcode={8093}}
\hypersetup{pdfcontactcity={Zurich}}
\hypersetup{pdfcontactcountry={Switzerland}}
\hypersetup{pdfcontactemail={nbeisert@itp.phys.ethz.ch}}
\hypersetup{pdfcontacturl={http://people.phys.ethz.ch/\xmptilde nbeisert/}}

\newcommand{\secref}[1]{\hyperref[#1]{section \ref*{#1}}}

\parskip1ex
\parindent0pt
\let\olditemize\itemize
\def\itemize{\olditemize\parskip0pt}

\begin{document}

\title{The \textsf{childdoc} Package}
\hypersetup{pdftitle={The childdoc Package}}
\author{Niklas Beisert\\[2ex]
  Institut f\"ur Theoretische Physik\\
  Eidgen\"ossische Technische Hochschule Z\"urich\\
  Wolfgang-Pauli-Strasse 27, 8093 Z\"urich, Switzerland\\[1ex]
  \href{mailto:nbeisert@itp.phys.ethz.ch}
  {\texttt{nbeisert@itp.phys.ethz.ch}}}
\hypersetup{pdfauthor={Niklas Beisert}}
\hypersetup{pdfsubject={Manual for the LaTeX2e Package childdoc}}
\date{30 December 2018, \textsf{v2.0}}
\maketitle

\begin{abstract}\noindent
\textsf{childdoc} is a \LaTeXe{} package
that enables the direct compilation
of document sections included by |\include|
to individual files.
\end{abstract}

\begingroup
\parskip0ex
\tableofcontents
\endgroup

%%%%%%%%%%%%%%%%%%%%%%%%%%%%%%%%%%%%%%%%%%%%%%%%%%%%%%%%%%%%%%%%%%%%%%%%%%%%%%%%
%%%%%%%%%%%%%%%%%%%%%%%%%%%%%%%%%%%%%%%%%%%%%%%%%%%%%%%%%%%%%%%%%%%%%%%%%%%%%%%%
\section{Introduction}

\LaTeX{} provides a mechanism to structure a large document (such as a book)
into a main file and several child files (containing the chapters)
using the |\include| command.
This mechanism is beneficial for documents
which span hundreds of pages in order to
make the source file(s) more manageable.
Moreover, compilation can be restricted to
selected child files by means of the |\includeonly| command.
The latter feature can be used to reduce the compilation time while editing
(this was significantly more useful in the earlier days of \LaTeX{})
or to generate a smaller document which is easier to navigate.
Another application of |\includeonly| is to generate
documents consisting of selected parts of the complete document.

However, there are a few drawbacks of the plain |\include| mechanism:
\begin{itemize}
\item
The child files cannot be compiled on their own,
they can only be compiled via the main file.
A naive editing environment
(such as a text editor with an option
to have the current file processed by \LaTeX)
may require one to switch to the main file before compiling;
attempting to compile the child file produces errors.
\item
The main file must be modified (each time)
to adjust the |\includeonly| command
to the present needs. This easily leaves the main file in a messy state.
\item
The generated document will always carry the filename
of the main document. This is inconvenient if
several child files are to be compiled and
to be kept for distribution.
\end{itemize}

The present package provides a simple interface
to make child files individually compilable by \LaTeX{}.
Compiling a child file then has the same effect as compiling
the main file with an |\includeonly| command
to select the appropriate child.
Moreover the generated document will carry the name of the child
rather than the main file.
This resolves all three above issues.

This feature is meant to make the editing of books,
thesis documents and lecture notes somewhat more convenient.
However, the package can also be used efficiently for
composing a series of documents (such as exercise sheets)
which are typically distributed individually.
It then assists the author in generating the individual documents
(potentially in different versions)
as well as a document containing the collected series.
Another application is in developing style files
or other kinds of included material
where compilation of the style file could redirect
to a sample or test file.

%%%%%%%%%%%%%%%%%%%%%%%%%%%%%%%%%%%%%%%%%%%%%%%%%%%%%%%%%%%%%%%%%%%%%%%%%%%%%%%%
%%%%%%%%%%%%%%%%%%%%%%%%%%%%%%%%%%%%%%%%%%%%%%%%%%%%%%%%%%%%%%%%%%%%%%%%%%%%%%%%
\section{Usage}

First of all, the package \textsf{childdoc} is \emph{not} a standard
\LaTeXe{} |.sty| style file! Therefore it needs to be invoked in
a non-standard way.

%%%%%%%%%%%%%%%%%%%%%%%%%%%%%%%%%%%%%%%%%%%%%%%%%%%%%%%%%%%%%%%%%%%%%%%%%%%%%%%%
\subsection{Included Files}
\label{sec:include}

%%%%%%%%%%%%%%%%%%%%%%%%%%%%%%%%%%%%%%%%
\DescribeMacro{\childdocmain}
To use the package, add the commands
\begin{center}
\begin{tabular}{l}
|\input{childdoc.def}|\\
|\childdocmain{}|\\
\end{tabular}
\end{center}
at the very top of the main \LaTeX{} file,
in particular \emph{before} the |\documentclass| statement!
The argument of |\childdocmain| should be left empty
(but it must be present).

%%%%%%%%%%%%%%%%%%%%%%%%%%%%%%%%%%%%%%%%
\DescribeMacro{\childdocof}
Furthermore, add the commands
\begin{center}
\begin{tabular}{l}
|\input{childdoc.def}|\\
|\childdocof{|\textit{main}|}|\\
\end{tabular}
\end{center}
at the top of every child file \textit{child}
which is included by |\include{|\textit{child}|}|
from within the main file
(or at least for those files to be compiled individually).
The argument \textit{main} must be the filename of the main file.

There are a couple of
considerations in setting up the main and child documents:

%%%%%%%%%%%%%%%%%%%%%%%%%%%%%%%%%%%%%%%%
\paragraph{Restrictions.}

Please note the following restrictions:
\begin{itemize}
\item
|\childdocmain| must be called with one argument \textit{main}
to ensure compatibility with earlier version of the package.
It must either be empty (|\childdocmain{}|)
or precisely match the filename of the main file in which it is specified.
See \secref{sec:detection} for further information.
\item
The filename \textit{main} must be specified without the |.tex| extension.
\item
The filename \textit{main} is case sensitive
(even in case-insensitive file systems)
due to internal string comparison.
\item
The argument \textit{main} should be fully expanded, it cannot be a macro.
\item
Subdirectories and special characters should be avoided in filenames.
\item
The command |\childdocmain{|\textit{main}|}| must be followed by a whitespace.
It should not be followed immediately by another command
or by a comment mark `|%|'.
This is because the \TeX{} parser reads the token immediately following
the argument of |\childdocmain| and puts it
at the beginning of every child section;
however, a white\-space is ignored.
\end{itemize}

%%%%%%%%%%%%%%%%%%%%%%%%%%%%%%%%%%%%%%%%
\paragraph{Content of Main File.}

It is advisable to place all content in the child files included by |\include|.
Any output contained in the main file will appear in all child documents
unless suppressed manually;
it cannot be suppressed automatically by the |\includeonly| directive
and thus should normally be avoided.
A method to include some content in the main file
by means of conditional processing is described in \secref{sec:conditional}.

%%%%%%%%%%%%%%%%%%%%%%%%%%%%%%%%%%%%%%%%
\paragraph{Page Numbering.}

When only a part of the document is compiled,
the appropriate numbering of pages
(as well as other status parameters)
is determined from the |.aux| files.
The latter contain information from previous passes.
However this information needs to propagate through
all intermediate child documents.
Therefore the page numbering in child documents may well
be inconsistent until the complete document is compiled at least once.

A useful (if unconventional) way to always ensure a consistent
page numbering is to restart the numbering in each child document
and denote the pages by `\textit{child}|.|\textit{page}'
where \textit{child} represents the chapter/section number of the child file.
This can be achieved by the command
|\numberwithin{page}{|\textit{child}|}|
of the \textsf{amsmath} package
where \textit{child} can be |chapter| or |section|
depending on the chosen structuring.
Alternatively, one can modify the macro |\thepage| appropriately
and reset the counter |page| at the start of each child file.

%%%%%%%%%%%%%%%%%%%%%%%%%%%%%%%%%%%%%%%%%%%%%%%%%%%%%%%%%%%%%%%%%%%%%%%%%%%%%%%%
\subsection{Conditional Processing}
\label{sec:conditional}

The package provides a mechanism to compile different versions
of a document. To customise the versions further some conditional processing
can come in handy to distinguish which version is being compiled.
The package provides two macros to describe the compilation context:

%%%%%%%%%%%%%%%%%%%%%%%%%%%%%%%%%%%%%%%%
\DescribeMacro{\ifchilddoc}
The conditional |\ifchilddoc| distinguishes between the compilation of
child documents and the main document:
%
\begin{center}
|\ifchilddoc |\textit{child-code}| |[|\||else |\textit{main-code}]| \||fi|
\end{center}

%%%%%%%%%%%%%%%%%%%%%%%%%%%%%%%%%%%%%%%%
\DescribeMacro{\childdocname}
\DescribeMacro{\childdocjob}
The macro |\childdocname| contains the filename (without extension)
of the main or child file being processed.
Note that |\childdocjob| will always contain the name of the main file.

%%%%%%%%%%%%%%%%%%%%%%%%%%%%%%%%%%%%%%%%
\paragraph{Title Page.}

Conditional processing can be used to include a title or banner page
in the main document when proper precautions are taken.
Importantly, the code in the main file should ensure that the page counter
(as well as other status parameters which are stored in the |.aux| files)
takes the same value after the conditional processing.
Otherwise the page numbers may take divergent values
depending on which part is compiled.

For example, a title page could be declared by:
%
\begin{center}
\begin{tabular}{l}
|\ifchilddoc\||else|\\
|\addtocounter{page}{-1}|\\
\textit{code for title page}\\
|\newpage|\\
|\||fi|
\end{tabular}
\end{center}
%
A banner page for the child documents can be generated by:
%
\begin{center}
\begin{tabular}{l}
|\ifchilddoc|\\
|\addtocounter{page}{-1}|\\
\textit{code for banner page}\\
|\newpage|\\
|\||fi|
\end{tabular}
\end{center}
%
Here one could write a message such as:
\begin{center}
|This is the part \childdocname{} of \childdocjob{}.|
\end{center}

%%%%%%%%%%%%%%%%%%%%%%%%%%%%%%%%%%%%%%%%%%%%%%%%%%%%%%%%%%%%%%%%%%%%%%%%%%%%%%%%
\subsection{Flags}
\label{sec:flags}

The package makes it easy to generate different versions
of the main or child documents.
To this end compilation flags can be defined
and assigned different default values.
They will be particularly useful in conjunction
with the forwarding mechanism described in \secref{sec:forward}.

For example, it may be useful to have a flag |\version|
which can be set to |draft| or |final|.
The document source will contain some conditional code
depending on the value of |\version|.
Suppose further, the flag should default to |final| for the main file
and to |draft| for child files
which is a natural assignment for editing the document.
This is achieved by placing the following code
in the preamble of the main document
(below the |\childdocmain| directive):
%
\begin{center}
\begin{tabular}{l}
|\ifchilddoc|\\
|\providecommand{\version}{draft}|\\
|\||else|\\
|\providecommand{\version}{final}|\\
|\||fi|
\end{tabular}
\end{center}
%
The definition by |\providecommand| makes sure
that previous definitions are not overwritten.
Further statements |\providecommand{\version}{...}|
can thus be added before the above code to override it.

For the main file, one might add a line
(between |\childdocmain| and the above block)
%
\begin{center}
|%\ifchilddoc\||else\providecommand{\version}{draft}\||fi|
\end{center}
%
which can be uncommented to produce a draft version.
Likewise one can add a line to the very top of a child file
(above the |\childdocof{|\textit{main}|}| directive)
%
\begin{center}
|%\providecommand{\version}{final}|
\end{center}
%
which can be uncommented to produce the final version of this child document.

%%%%%%%%%%%%%%%%%%%%%%%%%%%%%%%%%%%%%%%%%%%%%%%%%%%%%%%%%%%%%%%%%%%%%%%%%%%%%%%%
\subsection{Forwarding}
\label{sec:forward}

Different versions of the main or child documents
using compilation flags as described in \secref{sec:flags}
can be (permanently) stored in different files
for convenient compilation, viewing and distribution.
To this end, the package defines a command
to pass on compilation to a different file:

%%%%%%%%%%%%%%%%%%%%%%%%%%%%%%%%%%%%%%%%
\DescribeMacro{\childdocforward}
The command |\childdocforward| redirects processing to
another source file:
%
\begin{center}
\begin{tabular}{l}
|\input{childdoc.def}|\\
|\childdocforward[|\textit{main}|]{|\textit{dest}|}|\\
\end{tabular}
\end{center}
%
The argument \textit{dest} is the destination file
(without extension).
It should be the main file or one of the child files.
Note that further \textsf{childdoc} directives
such as |\childdocof| and |\childdocforward|
in the indicated file will be processed in this form.
The optional argument \textit{main}
passes on directly to the main file \textit{main}
while pretending to compile the child \textit{dest}.
This form behaves as if \textit{dest}
issues |\childdocof{|\textit{main}|}| right away,
and no further \textsf{childdoc} directives will be processed.

%%%%%%%%%%%%%%%%%%%%%%%%%%%%%%%%%%%%%%%%
\DescribeMacro{\...prefix}
In the alternative form |\childdocforwardprefix|,
%
\begin{center}
\begin{tabular}{l}
|\input{childdoc.def}|\\
|\childdocforwardprefix[|\textit{main}|]{|\textit{prefix}|}{|\textit{dest}|}|
\end{tabular}
\end{center}
%
the destination file is determined by a pattern
depending on the current file:
To make this work, the current file must be called
`{\textit{prefix}\hspace{0.2em}\textit{suffix}}'
with \textit{prefix} matching precisely the argument.
Processing is then passed on to the file
`{\textit{dest}\hspace{0.2em}\textit{suffix}}'.
Surely, the same effect is achieved by
directly specifying the
argument `{\textit{dest}\hspace{0.2em}\textit{suffix}}'
in the first form.
However, that requires to set up a different file
for each child. With the alternative form of the command
all these files can have exactly the same content
which simplifies setting them up and maintaining them.

For example, the following file |draft.tex|
with a compilation flag |\version| as described in \secref{sec:flags}
compiles the main document as a draft:
%
\begin{center}
\begin{tabular}{l}
|\def\version{draft}|\\
|\input{childdoc.def}|\\
|\childdocforward{|\textit{main}|}|
\end{tabular}
\end{center}
%
Likewise, the following files |final|\textit{nn}|.tex|
compile the final version of the child document
|child|\textit{nn}|.tex|:
%
\begin{center}
\begin{tabular}{l}
|\def\version{final}|\\
|\input{childdoc.def}|\\
|\childdocforwardprefix{final}{child}|
\end{tabular}
\end{center}
%

Note that when several versions of a main file and/or of each child file
are to be generated, it may be convenient to set up a |Makefile| or
shell script to automatise the process.

%%%%%%%%%%%%%%%%%%%%%%%%%%%%%%%%%%%%%%%%%%%%%%%%%%%%%%%%%%%%%%%%%%%%%%%%%%%%%%%%
\subsection{Command Line Processing}
\label{sec:commandline}

The effect of redirection files can also be achieved by invoking
the \LaTeX{} compiler with a more elaborate command line.
Most conveniently this should be done as part
of a shell script or a |Makefile|.

When using \textsf{childdoc} in the main file, the following
command lines effectively perform a redirection
(note that depending on the shell being used,
backslashes may have to be doubled: `|\|' $\to$ `|\\|'):
%
\begin{center}
|... -jobname "|\textit{target}|" |\\|"|[\textit{flags}]%
|\input{childdoc.def}\childdocforward[|\textit{main}|]{|\textit{dest}|}"|
\end{center}
%
Here \textit{target} is the name of the output file,
\textit{main} is the name of the main file
and \textit{dest} is the name of the main or child file to be processed
(all filenames without extensions).
The optional argument \textit{main} can be omitted
if \textit{main} matches \textit{dest}.
Optionally, compilation \textit{flags} can be defined via |\def| commands.
This command line makes the \TeX{} engine believe
it is compiling the file \textit{target}
whose content is specified as the latter parameter.
The provided code then forwards the processing to
\textit{main} or \textit{dest} as described in \secref{sec:forward}.

%%%%%%%%%%%%%%%%%%%%%%%%%%%%%%%%%%%%%%%%%%%%%%%%%%%%%%%%%%%%%%%%%%%%%%%%%%%%%%%%
\subsection{Include by Input}
\label{sec:input}

Including child documents by |\include| has some restrictions by design.
Most notably, the content of a child document always occupies
its own set of pages; pages cannot be shared between child documents.
Usually, this behaviour makes perfect sense
because each child document contain an essential part of the document.
However, in some situations it may be desirable to compose
a document from a collection of parts
without having mandatory page breaks between then.
For this case, the package
provides a mechanism to include parts
by |\input| which can also be processed individually.
However, by construction this mechanism
requires manual handling of the content to be output.

%%%%%%%%%%%%%%%%%%%%%%%%%%%%%%%%%%%%%%%%
\DescribeMacro{\ifchilddocmanual}
The main file should be prepared as usual, see \secref{sec:include}.
However, the document body must make a distinction
between processing of an individual part and of the main document, e.g.:
%
\begin{center}
\begin{tabular}{l}
|\ifchilddocmanual|\\
|\input{\childdocname}|\\
|\||else|\\
\textit{document body with }|\input{|\textit{part}|}|\\
|\||fi|
\end{tabular}
\end{center}
%
The conditional |\ifchilddocmanual| is true whenever
a part to be included by |\input| is being compiled,
and the name of the part is stored in |\childdocname|.

%%%%%%%%%%%%%%%%%%%%%%%%%%%%%%%%%%%%%%%%
\DescribeMacro{\childdocby}
Each part to be included by |\input| should start with:
%
\begin{center}
\begin{tabular}{l}
|\input{childdoc.def}|\\
|\childdocby{|\textit{main}|}|\\
\end{tabular}
\end{center}
%
The directive |\childdocby| is similar to |\childdocof|
described in \secref{sec:include},
but the subsequent selection of content must be done manually.
To that end, both |\ifchilddoc| and |\ifchilddocmanual|
will be true upon processing of a part,
and the name of the part is stored in |\childdocname|.
Note that |\jobname| will be set to the filename of the current part
so that each part receives an individual |.aux| file
that does not interfere with the |.aux| file(s) of the main document.
This behaviour can be altered by the alternative form
|\childdocby[*]{|\textit{main}|}| (with a non-empty optional argument)
which uses the |.aux| file of the main document
by setting |\jobname| to \textit{main}.

%%%%%%%%%%%%%%%%%%%%%%%%%%%%%%%%%%%%%%%%%%%%%%%%%%%%%%%%%%%%%%%%%%%%%%%%%%%%%%%%
\subsection{Driver Development}
\label{sec:driver}

The \textsf{childdoc} mechanism can also be use for the development
of definition files such as \LaTeX{} styles or classes.
This case differs from the above setup with multiple parts
included by |\include| in that no |\includeonly| should be invoked.
This can be achieved by starting the include file
(before |\ProvidesPackage|) with:
%
\begin{center}
\begin{tabular}{l}
|\input{childdoc.def}|\\
|\childdocforward{|\textit{main}|}|\\
\end{tabular}
\end{center}
%
or alternatively with:
%
\begin{center}
\begin{tabular}{l}
|\input{childdoc.def}|\\
|\childdocby{|\textit{main}|}|\\
\end{tabular}
\end{center}
%
Both forms have slightly different effects as described above.
The main file is prepared as usual, see \secref{sec:include}.

%%%%%%%%%%%%%%%%%%%%%%%%%%%%%%%%%%%%%%%%%%%%%%%%%%%%%%%%%%%%%%%%%%%%%%%%%%%%%%%%
\subsection{Legacy Detection}
\label{sec:detection}

The directive |\childdocmain| in the main file can detect
whether the complete document or merely a child is to be compiled
even without using the directive |\childdocof|.
This method is deprecated because it is less robust
and there is no compelling reason to use it;
it is merely provided for backward compatibility
and it may be removed in future versions.

If the detection mechanism is to be used,
it is mandatory to correctly specify
the filename of the main file as the argument of |\childdocmain|:
%
\begin{center}
\begin{tabular}{l}
|\input{childdoc.def}|\\
|\childdocmain{|\textit{main}|}|\\
\end{tabular}
\end{center}
%
If |\jobname| does not match the argument \textit{main} of |\childdocmain|,
it is assumed that |\jobname| points to the child file to be compiled.
When using |\childdocmain| with the main file specified as argument,
it suffices to start a child file
with just |\input{|\textit{main}|}|
without loading of the package and using |\childdocof|.
If instead all processing is done
with the appropriate \textsf{childdoc} directives,
the argument of \textit{main} of |\childdocmain| can be empty.

An alternative version of the command line processing described
in \secref{sec:commandline} using the detection mechanism reads:
%
\begin{center}
|... -jobname "|\textit{target}|" "|[\textit{flags}]%
[|\def\jobname{|\textit{dest}|}|]|\input{|\textit{main}|}"|
\end{center}

%%%%%%%%%%%%%%%%%%%%%%%%%%%%%%%%%%%%%%%%%%%%%%%%%%%%%%%%%%%%%%%%%%%%%%%%%%%%%%%%
\subsection{Manual Code}
\label{sec:manual}

In case one cannot be certain whether the definitions file |childdoc.def|
is installed on the target \TeX{} distribution
and one prefers not to ship it,
it is conceivable to paste a few relevant commands into the sources.

To that end, drop all statements |\input{childdoc.def}|
and perform the replacements as outlined below.
Instead of |\childdocmain{|\textit{main}|}| add the following code
to the top of the main file:
%
\begin{center}
\begin{tabular}{l}
|\||ifdefined\childdocname\endinput\||fi\newif\ifchilddoc|\\
|\edef\childdocname{\scantokens\expandafter{\jobname\noexpand}}|\\
|\def\childdocmain{|\textit{main}|}\||ifx\childdocmain\childdocname\||else|\\
|\childdoctrue\includeonly{\childdocname}\let\jobname\childdocmain\||fi|\\
\end{tabular}
\end{center}
%
Instead of |\childdocof{|\textit{main}|}| just include the main file
at the top of each child file:
%
\begin{center}
|\input{|\textit{main}|}|
\end{center}
%
A simple redirection |\childdocforward{|\textit{dest}|}| is achieved by:
%
\begin{center}
|\def\jobname{|\textit{dest}|}\input{\jobname}|
\end{center}
%
The redirection with prefix
|\childdocforwardprefix[|\textit{prefix}|]{|\textit{dest}|}|
is accomplished by:
%
\begin{center}
\begin{tabular}{l}
|{\edef\jobname{\scantokens\expandafter{\jobname\noexpand}}|\\
|\def\redirectjob |\textit{prefix}|#1~~~{\gdef\jobname{|\textit{dest}|#1}}|\\
|\expandafter\redirectjob\jobname~~~}\input{\jobname}|
\end{tabular}
\end{center}

In an alternative approach,
child documents can be compiled by a specific command line
without additional code or specific definitions:
%
\begin{center}
|... -jobname "|\textit{target}|" "|[\textit{flags}]%
|\includeonly{|\textit{dest}|}\input{|\textit{main}|}"|
\end{center}
%

%%%%%%%%%%%%%%%%%%%%%%%%%%%%%%%%%%%%%%%%%%%%%%%%%%%%%%%%%%%%%%%%%%%%%%%%%%%%%%%%
%%%%%%%%%%%%%%%%%%%%%%%%%%%%%%%%%%%%%%%%%%%%%%%%%%%%%%%%%%%%%%%%%%%%%%%%%%%%%%%%
\section{Information}

%%%%%%%%%%%%%%%%%%%%%%%%%%%%%%%%%%%%%%%%%%%%%%%%%%%%%%%%%%%%%%%%%%%%%%%%%%%%%%%%
\subsection{Copyright}

Copyright \copyright{} 2017--2018 Niklas Beisert

This work may be distributed and/or modified under the
conditions of the \LaTeX{} Project Public License, either version 1.3
of this license or (at your option) any later version.
The latest version of this license is in
  \url{http://www.latex-project.org/lppl.txt}
and version 1.3 or later is part of all distributions of \LaTeX{}
version 2005/12/01 or later.

This work has the LPPL maintenance status `maintained'.

The Current Maintainer of this work is Niklas Beisert.

This work consists of the files |README.txt|, |childdoc.ins| and |childdoc.dtx|
as well as the derived files |childdoc.def|, |cdocsamp.tex|
with |cdocsch1.tex|, |cdocsch2.tex|, |cdocspt3.tex|, |cdocspt4.tex|,
|cdocsdrf.tex|, |cdocsfn1.tex|, |cdocsfn2.tex|
as well as |childdoc.pdf|.

%%%%%%%%%%%%%%%%%%%%%%%%%%%%%%%%%%%%%%%%%%%%%%%%%%%%%%%%%%%%%%%%%%%%%%%%%%%%%%%%
\subsection{Files and Installation}

The package consists of the files:
%
\begin{center}
\begin{tabular}{ll}
    |README.txt|   & readme file \\
    |childdoc.ins| & installation file \\
    |childdoc.dtx| & source file \\
    |childdoc.def| & definition file \\
    |cdocsamp.tex| & sample main file \\
    |cdocsch1.tex| & sample include file \\
    |cdocsch2.tex| & sample include file \\
    |cdocspt3.tex| & sample part file \\
    |cdocspt4.tex| & sample part file \\
    |cdocsdrf.tex| & sample redirection file \\
    |cdocsfn1.tex| & sample redirection file \\
    |cdocsfn2.tex| & sample redirection file \\
    |childdoc.pdf| & manual
\end{tabular}
\end{center}
%
The distribution consists of the files
|README.txt|, |childdoc.ins| and |childdoc.dtx|.
%
\begin{itemize}
\item
Run (pdf)\LaTeX{} on |childdoc.dtx|
to compile the manual |childdoc.pdf| (this file).
\item
Run \LaTeX{} on |childdoc.ins| to create the definitions file |childdoc.def|
and the sample |cdocsamp.tex| with include files
|cdocsch1.tex|, |cdocsch2.tex|, |cdocspt3.tex|, |cdocspt4.tex|,
|cdocsdrf.tex|, |cdocsfn1.tex|, |cdocsfn2.tex|.
Then copy the file |childdoc.def| to an appropriate directory of your \LaTeX{}
distribution, e.g.\ \textit{texmf-root}|/tex/latex/childdoc|.
\end{itemize}

%%%%%%%%%%%%%%%%%%%%%%%%%%%%%%%%%%%%%%%%%%%%%%%%%%%%%%%%%%%%%%%%%%%%%%%%%%%%%%%%
\subsection{Related CTAN Packages}

There are several other packages which offer a similar functionality:
%
\begin{itemize}
\item
The packages
\href{http://ctan.org/pkg/docmute}{\textsf{docmute}},
\href{http://ctan.org/pkg/includex}{\textsf{includex}} and
\href{http://ctan.org/pkg/standalone}{\textsf{standalone}}
provide commands to include only the document body of
a child file thus allowing both files to be compiled individually.
\item
The packages \href{http://ctan.org/pkg/subdocs}{\textsf{subdocs}}
and \href{http://ctan.org/pkg/subfiles}{\textsf{subfiles}}
provide structures in which the main and child documents can be
encapsulated and allowing them to be compiled individually.
The inclusion mechanism is different from the conventional |\include|.
\item
The package \href{http://ctan.org/pkg/combine}{\textsf{combine}}
is an elaborate solution to combine several documents into one.
\end{itemize}
%
See also the CTAN topic \href{http://ctan.org/topic/subdocs}{\textsf{subdocs}}
for further related packages.
The present package differs from the above solutions in that
a document structure constructed with the conventional |\include| mechanism
just needs two extra commands at the top of every file
such that all constituent files can be compiled individually.

%%%%%%%%%%%%%%%%%%%%%%%%%%%%%%%%%%%%%%%%%%%%%%%%%%%%%%%%%%%%%%%%%%%%%%%%%%%%%%%%
%\subsection{Feature Suggestions}
%
%The following is a list of features which may be useful for future
%versions of this package:
%%
%\begin{itemize}
%\item
%\ldots
%\end{itemize}

%%%%%%%%%%%%%%%%%%%%%%%%%%%%%%%%%%%%%%%%%%%%%%%%%%%%%%%%%%%%%%%%%%%%%%%%%%%%%%%%
\subsection{Revision History}

%%%%%%%%%%%%%%%%%%%%%%%%%%%%%%%%%%%%%%%%
\paragraph{v2.0:} 2018/12/30

\begin{itemize}
\item
immediate forward processing
\item
added |\childdocby| mechanism
\item
manual restructured
\end{itemize}

%%%%%%%%%%%%%%%%%%%%%%%%%%%%%%%%%%%%%%%%
\paragraph{v1.6:} 2018/01/17

\begin{itemize}
\item
application for development of include files
\item
corrections to manual
\end{itemize}

%%%%%%%%%%%%%%%%%%%%%%%%%%%%%%%%%%%%%%%%
\paragraph{v1.5:} 2017/05/21

\begin{itemize}
\item
more complete structuring introduced
\item
|\childdocof| introduced
\item
|\childdoc| renamed to |\childdocmain|
\item
|\childredirect| renamed to |\childdocforward| and |\childdocforwardprefix|
and functionality expanded
\end{itemize}

%%%%%%%%%%%%%%%%%%%%%%%%%%%%%%%%%%%%%%%%
\paragraph{v1.0:} 2017/04/27

\begin{itemize}
\item
manual and install package
\item
first version published on CTAN
\end{itemize}

%%%%%%%%%%%%%%%%%%%%%%%%%%%%%%%%%%%%%%%%
\paragraph{v0.6:} 2017/04/26

\begin{itemize}
\item
redirection mechanism added
\end{itemize}

%%%%%%%%%%%%%%%%%%%%%%%%%%%%%%%%%%%%%%%%
\paragraph{v0.5:} 2017/04/26

\begin{itemize}
\item
functionality in definition file
\end{itemize}


%%%%%%%%%%%%%%%%%%%%%%%%%%%%%%%%%%%%%%%%%%%%%%%%%%%%%%%%%%%%%%%%%%%%%%%%%%%%%%%%
%%%%%%%%%%%%%%%%%%%%%%%%%%%%%%%%%%%%%%%%%%%%%%%%%%%%%%%%%%%%%%%%%%%%%%%%%%%%%%%%
%%%%%%%%%%%%%%%%%%%%%%%%%%%%%%%%%%%%%%%%%%%%%%%%%%%%%%%%%%%%%%%%%%%%%%%%%%%%%%%%
\appendix

\settowidth\MacroIndent{\rmfamily\scriptsize 000\ }

 \DocInput{childdoc.dtx}

\end{document}
%</driver>
% \fi
%
% %%%%%%%%%%%%%%%%%%%%%%%%%%%%%%%%%%%%%%%%%%%%%%%%%%%%%%%%%%%%%%%%%%%%%%%%%%%%%%
% %%%%%%%%%%%%%%%%%%%%%%%%%%%%%%%%%%%%%%%%%%%%%%%%%%%%%%%%%%%%%%%%%%%%%%%%%%%%%%
% \section{Sample}
%\iffalse
%<*samplemain>
%\fi
%
% The following presents a sample document
% with two chapters, two parts, a title page,
% a compile flag as well as three forwarding files to set the flag.
% It consists of eight |.tex| files:
% \begin{center}
% \begin{tabular}{ll}
% |cdocsamp.tex|&main file\\
% |cdocsch1.tex|&include file for chapter 1\\
% |cdocsch2.tex|&include file for chapter 2\\
% |cdocspt3.tex|&include file for part 3\\
% |cdocspt4.tex|&include file for part 4\\
% |cdocsdrf.tex|&forwarding file for main file in draft mode\\
% |cdocsfi1.tex|&forwarding file for final version of chapter 1\\
% |cdocsfi2.tex|&forwarding file for final version of chapter 2\\
% \end{tabular}
% \end{center}
% Each of the eight files can be compiled directly by the \LaTeX{} compiler.
%
% %%%%%%%%%%%%%%%%%%%%%%%%%%%%%%%%%%%%%%
% \paragraph{Main File.}
%
% The main file is called |cdocsamp.tex|.
%
% Load the \textsf{childdoc} definitions and
% declare the filename for the main document:
%    \begin{macrocode}
\input{childdoc.def}
\childdocmain{}
%    \end{macrocode}

% Optional override for |\version| flag:
%    \begin{macrocode}
%%\ifchilddoc\else\providecommand{\version}{draft}\fi
%    \end{macrocode}

% Define the default values for the |\version| flag
% (|final| for the main file and |draft| for childs):
%    \begin{macrocode}
\ifchilddoc
\providecommand{\version}{draft}
\else
\providecommand{\version}{final}
\fi
%    \end{macrocode}

% Load the standard document class:
%    \begin{macrocode}
\documentclass[12pt]{article}
%    \end{macrocode}

% Start the document body:
%    \begin{macrocode}
\begin{document}
%    \end{macrocode}

% Declare a title page.
% Print title, part of document being processed and version flag:
%    \begin{macrocode}
\addtocounter{page}{-1}
\begin{center}
{\LARGE\bfseries{}childdoc example\par}
\vspace{1cm}
\ifchilddoc
\ifchilddocmanual part\else chapter\fi:
`\childdocname' of `\childdocjob'\par
\else
main document: `\childdocjob'\par
\fi
version: \version\par
\end{center}
\newpage
%    \end{macrocode}

% Manually include selected file,
% otherwise process as usual:
%    \begin{macrocode}
\ifchilddocmanual
\section*{part `\childdocname'}
\input{\childdocname}
\else
%    \end{macrocode}

% Include the two chapters:
%    \begin{macrocode}
\include{cdocsch1}
\include{cdocsch2}
%    \end{macrocode}

% Include the two parts unless only chapters should be displayed:
%    \begin{macrocode}
\ifchilddoc\else
\section{part three}
\input{cdocspt3}
\section{part four}
\input{cdocspt4}
\fi
%    \end{macrocode}

% Process as usual until here:
%    \begin{macrocode}
\fi
%    \end{macrocode}

% End of document body:
%    \begin{macrocode}
\end{document}
%    \end{macrocode}
%\iffalse
%</samplemain>
%\fi
%
% %%%%%%%%%%%%%%%%%%%%%%%%%%%%%%%%%%%%%%
% \paragraph{Chapter Include Files.}
%
% The include files are called |cdocsch1.tex| and |cdocsch2.tex|.
%
%\iffalse
%<*samplechap1|samplechap2>
%\fi

% Optional override for |\version| flag:
%    \begin{macrocode}
%%\providecommand{\version}{final}
%    \end{macrocode}

% Include the main document:
%    \begin{macrocode}
\input{childdoc.def}
\childdocof{cdocsamp}
%    \end{macrocode}

%\iffalse
%</samplechap1|samplechap2>
%\fi
%
%\iffalse
%<*samplechap1>
%\fi
% Some text for chapter 1:
%    \begin{macrocode}
\section{one}
some text in chapter one
%    \end{macrocode}

%\iffalse
%</samplechap1>
%\fi
% Some text for chapter 2:
%\iffalse
%<*samplechap2>
%\fi
%    \begin{macrocode}
\section{two}
more text in chapter two
%    \end{macrocode}

%\iffalse
%</samplechap2>
%\fi
%
% %%%%%%%%%%%%%%%%%%%%%%%%%%%%%%%%%%%%%%
% \paragraph{Part Include Files.}
%
% The include files are called |cdocspt3.tex| and |cdocspt4.tex|.
%
%\iffalse
%<*samplepart3|samplepart4>
%\fi

% Optional override for |\version| flag:
%    \begin{macrocode}
%%\providecommand{\version}{final}
%    \end{macrocode}

% Include the main document:
%    \begin{macrocode}
\input{childdoc.def}
\childdocby{cdocsamp}
%    \end{macrocode}

%\iffalse
%</samplepart3|samplepart4>
%\fi
%
%\iffalse
%<*samplepart3>
%\fi
% Some text for part 3:
%    \begin{macrocode}
some text in part three
%    \end{macrocode}

%\iffalse
%</samplepart3>
%\fi
% Some text for part 4:
%\iffalse
%<*samplepart4>
%\fi
%    \begin{macrocode}
more text in part four
%    \end{macrocode}

%\iffalse
%</samplepart4>
%\fi
%
% %%%%%%%%%%%%%%%%%%%%%%%%%%%%%%%%%%%%%%
% \paragraph{Forwarding for a Complete Draft.}
%
% The following forwarding file |cdocsdrf.tex|
% compiles the main document in draft mode:
%\iffalse
%<*sampledraft>
%\fi
%    \begin{macrocode}
\def\version{draft}
\input{childdoc.def}
\childdocforward{cdocsamp}
%    \end{macrocode}

%\iffalse
%</sampledraft>
%\fi
%
% %%%%%%%%%%%%%%%%%%%%%%%%%%%%%%%%%%%%%%
% \paragraph{Forwarding for Final Version of the Chapters.}
%
% The following forwarding files |cdocsfn1.tex| and |cdocsfn2.tex|
% (with identical content)
% compile the final versions of the child documents
% |cdocsch1.tex| and |cdocsch2.tex|, respectively:
%\iffalse
%<*samplefinal>
%\fi
%    \begin{macrocode}
\def\version{final}
\input{childdoc.def}
\childdocforwardprefix[cdocsamp]{cdocsfn}{cdocsch}
%    \end{macrocode}

%\iffalse
%</samplefinal>
%\fi
%
% %%%%%%%%%%%%%%%%%%%%%%%%%%%%%%%%%%%%%%
% \paragraph{Command Line Processing.}
%
% The following three command lines generate the output files
% |cdocscld|, |cdocscl1| and |cdocscl2|
% which should be identical to
% |cdocsdrf|, |cdocsch1| and |cdocsfn2|, respectively:
% \begin{center}
% \begin{tabular}{l}
% |latex -jobname cdocscld \|\\
% |  "\def\version{draft}\input{childdoc.def}\childdocforward{cdocsamp}"|\\
% |latex -jobname cdocscl1 \|\\
% |  "\input{childdoc.def}\childdocforward[cdocsamp]{cdocsch1}"|\\
% |latex -jobname cdocscl2 \|\\
% |  "\def\version{final}\input{childdoc.def}\childdocforward{cdocsch2}"|
% \end{tabular}
% \end{center}
% Note that the trailing backslash on each first line
% merely continues the input to the second line
% (for convenient cut ant paste).
% Furthermore, the command |latex| can be replaced by any
% of its alternative versions such as |pdflatex|.
%
% %%%%%%%%%%%%%%%%%%%%%%%%%%%%%%%%%%%%%%%%%%%%%%%%%%%%%%%%%%%%%%%%%%%%%%%%%%%%%%
% %%%%%%%%%%%%%%%%%%%%%%%%%%%%%%%%%%%%%%%%%%%%%%%%%%%%%%%%%%%%%%%%%%%%%%%%%%%%%%
% \section{Implementation}
%\iffalse
%<*package>
%\fi
%
% This section describes the definitions file |childdoc.def|.

% The definitions cannot be loaded using |\usepackage| or |\RequirePackage|
% which has a mechanism to prevent loading a style file more than once.
% When loading the definitions by means of |\input|
% multiple instances have to be prevented manually:
%\iffalse
%This code needs to be before the `\ProvidesFile' directive
%which is defined at the beginning of this file.
%Therefore it is also placed there and commented out here.
%</package>
%<*discard>
%\fi
%    \begin{macrocode}
\ifdefined\childdocmain\endinput\fi
%    \end{macrocode}
%\iffalse
%</discard>
%<*package>
%\fi
%
% \macro{\ifchilddoc}
% \macro{\ifchilddocmanual}
% The conditional |\ifchilddoc| tells whether a
% child (true) or main (false) document is being compiled.
% The conditional |\ifchilddocmanual| tells whether
% the |\includeonly| mechanism is used (false) or
% the selection of child files must be performed manually (true).
% The definitions initialise to false:
%    \begin{macrocode}
\newif\ifchilddoc
\newif\ifchilddocmanual
%    \end{macrocode}

% \macro{\childdocname}
% \macro{\childdocjob}
% The macro |\childdocname| stores the name of the main document
% to be compiled. The macro |\childdocjob| stores the name of
% the document on which the \LaTeX{} compiler was originally invoked.
% The content of |\jobname| cannot be compared
% to filenames specified in the source due to different catcodes.
% The following code rescans |\jobname|, stores the result
% in |\childdocname| and saves a copy in |\childdocjob|:
%    \begin{macrocode}
\edef\childdocname{\scantokens\expandafter{\jobname\noexpand}}
\let\childdocjob\childdocname
%    \end{macrocode}

% \macro{\childdocdisable}
% The macro |\childdocdisable| prevents the main file
% from being processed more than once.
% At this stage, the main document command |\childdocmain|
% is assumed to be called once again where it should do nothing.
% Any subsequent call to it should prevent
% a secondary processing of the main document
% It overwrites the forwarding commands
% |\childdocof| and |\childdocforward|
% with empty macros to prevent further inclusions of the main document:
%    \begin{macrocode}
\newcommand{\childdocdisable}
{
  \renewcommand{\childdocmain}[1]{\renewcommand{\childdocmain}[1]{\endinput}}
  \renewcommand{\childdocof}[1]{}
  \renewcommand{\childdocby}[2][]{}
  \renewcommand{\childdocforward}[2][]{}
  \renewcommand{\childdocdisable}{}
}
%    \end{macrocode}

% \macro{\childdocmain}
% The macro |\childdocmain| is to be called at the top of the main file
% with nothing or the main filename (without extension) as argument.
% First, it breaks loops.
% If the argument is not empty and does not match |\childdocname|
% (which is set by the first inclusion of |childdoc.def|),
% |\ifchilddoc| is set to true, |\includeonly| is applied to the child file
% and |\jobname| is set to the main file
% (for proper handling of |.aux| files):
%    \begin{macrocode}
\newcommand{\childdocmain}[1]
{
  \childdocdisable\childdocmain{}
  \if?#1?\else
    \begingroup
      \def\childdoctmp{#1}
      \ifx\childdoctmp\childdocname
        \def\childdoctmp{}
      \else
        \def\childdoctmp
        {
          \childdoctrue
          \includeonly{\childdocname}
          \def\childdocjob{#1}
          \def\jobname{#1}
        }
      \fi
      \expandafter
    \endgroup
    \childdoctmp
  \fi
}
%    \end{macrocode}

% \macro{\childdocof}
% The command |\childdocof| redirects
% compilation to the main file |#1|.
%    \begin{macrocode}
\newcommand{\childdocof}[1]
{
  \childdocdisable
  \childdoctrue
  \includeonly{\childdocname}
  \def\jobname{#1}
  \def\childdocjob{#1}
  \input{#1}
}
%    \end{macrocode}

% \macro{\childdocby}
% The command |\childdocby| ....
%    \begin{macrocode}
\newcommand{\childdocby}[2][]
{
  \childdocdisable
  \childdoctrue
  \childdocmanualtrue
  \if?#1?\else
    \def\jobname{#2}
  \fi
  \def\childdocjob{#2}
  \input{#2}
  \endinput
}
%    \end{macrocode}

% \macro{\childdocforward}
% The command |\childdocforward| redirects
% compilation to the main file or
% (if the optional argument is given) a child file.
% Parameters are set as if the main file
% or a child file starting with |\childdocof| was compiled.
% Then compilation is handed over to the main file:
%    \begin{macrocode}
\newcommand{\childdocforward}[2][]
{
  \begingroup
    \if?#1?
      \def\childdoctmp
      {
        \def\childdocname{#2}
        \def\childdocjob{#2}
        \def\jobname{#2}
        \input{#2}
        \endinput
      }
    \else
      \def\childdoctmp
      {
        \childdocdisable
        \def\childdocname{#2}
        \childdoctrue
        \includeonly{#2}
        \def\childdocjob{#1}
        \def\jobname{#1}
        \input{#1}
        \endinput
      }
    \fi
    \expandafter
  \endgroup
  \childdoctmp
}
%    \end{macrocode}

% \macro{\childdocforwardprefix}
% The command |\childdocforwardprefix| redirects
% compilation to the main or a child file by means of a pattern.
% The prefix |#1| in the current filename is replaced by |#2|
% and the suffix of the current filename is kept
% (it is assumed that the filename does not contain the substring `|~~~|'
% which is used as a delimiter).
% Compilation is handed over to the new file by |\childdocforward|:
%    \begin{macrocode}
\newcommand{\childdocforwardprefix}[3][]
{
  \begingroup
    \def\childdocextract #2##1~~~{\def\childdoctmp{\childdocforward[#1]{#3##1}}}
    \expandafter\childdocextract\childdocname~~~
    \expandafter
  \endgroup
  \childdoctmp
}
%    \end{macrocode}

% \macro{\childdoc}
% The deprecated macro |\childdoc| is a legacy version of |\childdocmain|:
%    \begin{macrocode}
\newcommand{\childdoc}{\childdocmain}
%    \end{macrocode}

% \macro{\childdocredirect}
% The deprecated macro |\childdocredirect| is a legacy version
% of |\childdocforward| and |\childdocforwardprefix|:
%    \begin{macrocode}
\newcommand{\childdocredirect}[2][]
{
  \begingroup
    \if?#1?
      \def\childdoctmp{\childdocforward{#2}}
    \else
      \def\childdoctmp{\childdocforwardprefix{#1}{#2}}
    \fi
    \expandafter
  \endgroup
  \childdoctmp
}
%    \end{macrocode}

%\iffalse
%</package>
%\fi
%
\endinput
|\\
|\childdocof{|\textit{main}|}|\\
\end{tabular}
\end{center}
at the top of every child file \textit{child}
which is included by |\include{|\textit{child}|}|
from within the main file
(or at least for those files to be compiled individually).
The argument \textit{main} must be the filename of the main file.

There are a couple of
considerations in setting up the main and child documents:

%%%%%%%%%%%%%%%%%%%%%%%%%%%%%%%%%%%%%%%%
\paragraph{Restrictions.}

Please note the following restrictions:
\begin{itemize}
\item
|\childdocmain| must be called with one argument \textit{main}
to ensure compatibility with earlier version of the package.
It must either be empty (|\childdocmain{}|)
or precisely match the filename of the main file in which it is specified.
See \secref{sec:detection} for further information.
\item
The filename \textit{main} must be specified without the |.tex| extension.
\item
The filename \textit{main} is case sensitive
(even in case-insensitive file systems)
due to internal string comparison.
\item
The argument \textit{main} should be fully expanded, it cannot be a macro.
\item
Subdirectories and special characters should be avoided in filenames.
\item
The command |\childdocmain{|\textit{main}|}| must be followed by a whitespace.
It should not be followed immediately by another command
or by a comment mark `|%|'.
This is because the \TeX{} parser reads the token immediately following
the argument of |\childdocmain| and puts it
at the beginning of every child section;
however, a white\-space is ignored.
\end{itemize}

%%%%%%%%%%%%%%%%%%%%%%%%%%%%%%%%%%%%%%%%
\paragraph{Content of Main File.}

It is advisable to place all content in the child files included by |\include|.
Any output contained in the main file will appear in all child documents
unless suppressed manually;
it cannot be suppressed automatically by the |\includeonly| directive
and thus should normally be avoided.
A method to include some content in the main file
by means of conditional processing is described in \secref{sec:conditional}.

%%%%%%%%%%%%%%%%%%%%%%%%%%%%%%%%%%%%%%%%
\paragraph{Page Numbering.}

When only a part of the document is compiled,
the appropriate numbering of pages
(as well as other status parameters)
is determined from the |.aux| files.
The latter contain information from previous passes.
However this information needs to propagate through
all intermediate child documents.
Therefore the page numbering in child documents may well
be inconsistent until the complete document is compiled at least once.

A useful (if unconventional) way to always ensure a consistent
page numbering is to restart the numbering in each child document
and denote the pages by `\textit{child}|.|\textit{page}'
where \textit{child} represents the chapter/section number of the child file.
This can be achieved by the command
|\numberwithin{page}{|\textit{child}|}|
of the \textsf{amsmath} package
where \textit{child} can be |chapter| or |section|
depending on the chosen structuring.
Alternatively, one can modify the macro |\thepage| appropriately
and reset the counter |page| at the start of each child file.

%%%%%%%%%%%%%%%%%%%%%%%%%%%%%%%%%%%%%%%%%%%%%%%%%%%%%%%%%%%%%%%%%%%%%%%%%%%%%%%%
\subsection{Conditional Processing}
\label{sec:conditional}

The package provides a mechanism to compile different versions
of a document. To customise the versions further some conditional processing
can come in handy to distinguish which version is being compiled.
The package provides two macros to describe the compilation context:

%%%%%%%%%%%%%%%%%%%%%%%%%%%%%%%%%%%%%%%%
\DescribeMacro{\ifchilddoc}
The conditional |\ifchilddoc| distinguishes between the compilation of
child documents and the main document:
%
\begin{center}
|\ifchilddoc |\textit{child-code}| |[|\||else |\textit{main-code}]| \||fi|
\end{center}

%%%%%%%%%%%%%%%%%%%%%%%%%%%%%%%%%%%%%%%%
\DescribeMacro{\childdocname}
\DescribeMacro{\childdocjob}
The macro |\childdocname| contains the filename (without extension)
of the main or child file being processed.
Note that |\childdocjob| will always contain the name of the main file.

%%%%%%%%%%%%%%%%%%%%%%%%%%%%%%%%%%%%%%%%
\paragraph{Title Page.}

Conditional processing can be used to include a title or banner page
in the main document when proper precautions are taken.
Importantly, the code in the main file should ensure that the page counter
(as well as other status parameters which are stored in the |.aux| files)
takes the same value after the conditional processing.
Otherwise the page numbers may take divergent values
depending on which part is compiled.

For example, a title page could be declared by:
%
\begin{center}
\begin{tabular}{l}
|\ifchilddoc\||else|\\
|\addtocounter{page}{-1}|\\
\textit{code for title page}\\
|\newpage|\\
|\||fi|
\end{tabular}
\end{center}
%
A banner page for the child documents can be generated by:
%
\begin{center}
\begin{tabular}{l}
|\ifchilddoc|\\
|\addtocounter{page}{-1}|\\
\textit{code for banner page}\\
|\newpage|\\
|\||fi|
\end{tabular}
\end{center}
%
Here one could write a message such as:
\begin{center}
|This is the part \childdocname{} of \childdocjob{}.|
\end{center}

%%%%%%%%%%%%%%%%%%%%%%%%%%%%%%%%%%%%%%%%%%%%%%%%%%%%%%%%%%%%%%%%%%%%%%%%%%%%%%%%
\subsection{Flags}
\label{sec:flags}

The package makes it easy to generate different versions
of the main or child documents.
To this end compilation flags can be defined
and assigned different default values.
They will be particularly useful in conjunction
with the forwarding mechanism described in \secref{sec:forward}.

For example, it may be useful to have a flag |\version|
which can be set to |draft| or |final|.
The document source will contain some conditional code
depending on the value of |\version|.
Suppose further, the flag should default to |final| for the main file
and to |draft| for child files
which is a natural assignment for editing the document.
This is achieved by placing the following code
in the preamble of the main document
(below the |\childdocmain| directive):
%
\begin{center}
\begin{tabular}{l}
|\ifchilddoc|\\
|\providecommand{\version}{draft}|\\
|\||else|\\
|\providecommand{\version}{final}|\\
|\||fi|
\end{tabular}
\end{center}
%
The definition by |\providecommand| makes sure
that previous definitions are not overwritten.
Further statements |\providecommand{\version}{...}|
can thus be added before the above code to override it.

For the main file, one might add a line
(between |\childdocmain| and the above block)
%
\begin{center}
|%\ifchilddoc\||else\providecommand{\version}{draft}\||fi|
\end{center}
%
which can be uncommented to produce a draft version.
Likewise one can add a line to the very top of a child file
(above the |\childdocof{|\textit{main}|}| directive)
%
\begin{center}
|%\providecommand{\version}{final}|
\end{center}
%
which can be uncommented to produce the final version of this child document.

%%%%%%%%%%%%%%%%%%%%%%%%%%%%%%%%%%%%%%%%%%%%%%%%%%%%%%%%%%%%%%%%%%%%%%%%%%%%%%%%
\subsection{Forwarding}
\label{sec:forward}

Different versions of the main or child documents
using compilation flags as described in \secref{sec:flags}
can be (permanently) stored in different files
for convenient compilation, viewing and distribution.
To this end, the package defines a command
to pass on compilation to a different file:

%%%%%%%%%%%%%%%%%%%%%%%%%%%%%%%%%%%%%%%%
\DescribeMacro{\childdocforward}
The command |\childdocforward| redirects processing to
another source file:
%
\begin{center}
\begin{tabular}{l}
|% \iffalse
%
% childdoc.dtx Copyright (C) 2017-2018 Niklas Beisert
%
% This work may be distributed and/or modified under the
% conditions of the LaTeX Project Public License, either version 1.3
% of this license or (at your option) any later version.
% The latest version of this license is in
%   http://www.latex-project.org/lppl.txt
% and version 1.3 or later is part of all distributions of LaTeX
% version 2005/12/01 or later.
%
% This work has the LPPL maintenance status `maintained'.
%
% The Current Maintainer of this work is Niklas Beisert.
%
% This work consists of the files childdoc.dtx and childdoc.ins
% and the derived files childdoc.def and cdocsamp.tex with
% cdocsch1.tex, cdocsch2.tex, cdocsdrf.tex, cdocsfn1.tex, cdocsfn2.tex.
%
%<package>\ifdefined\childdocmain\endinput\fi
%<package>\ProvidesFile{childdoc.def}[2018/12/30 v2.0 child document driver]
%<samplemain>\ProvidesFile{cdocsamp.tex}[2018/12/30 v2.0 sample for childdoc]
%<*driver>
%\ProvidesFile{childdoc.drv}[2018/12/30 v2.0 childdoc reference manual file]
\PassOptionsToClass{10pt,a4paper}{article}
\documentclass{ltxdoc}

\usepackage[margin=35mm]{geometry}
\usepackage{hyperref}
\usepackage{hyperxmp}
\usepackage[usenames]{color}

\hypersetup{colorlinks=true}
\hypersetup{pdfstartview=FitH}
\hypersetup{pdfpagemode=UseNone}
\hypersetup{pdfsource={}}
\hypersetup{pdflang={en-UK}}
\hypersetup{pdfcopyright={Copyright 2017-2018 Niklas Beisert.
  This work may be distributed and/or modified under the
  conditions of the LaTeX Project Public License, either version 1.3
  of this license or (at your option) any later version.}}
\hypersetup{pdflicenseurl={http://www.latex-project.org/lppl.txt}}
\hypersetup{pdfcontactaddress={ETH Zurich, ITP, HIT K,
  Wolfgang-Pauli-Strasse 27}}
\hypersetup{pdfcontactpostcode={8093}}
\hypersetup{pdfcontactcity={Zurich}}
\hypersetup{pdfcontactcountry={Switzerland}}
\hypersetup{pdfcontactemail={nbeisert@itp.phys.ethz.ch}}
\hypersetup{pdfcontacturl={http://people.phys.ethz.ch/\xmptilde nbeisert/}}

\newcommand{\secref}[1]{\hyperref[#1]{section \ref*{#1}}}

\parskip1ex
\parindent0pt
\let\olditemize\itemize
\def\itemize{\olditemize\parskip0pt}

\begin{document}

\title{The \textsf{childdoc} Package}
\hypersetup{pdftitle={The childdoc Package}}
\author{Niklas Beisert\\[2ex]
  Institut f\"ur Theoretische Physik\\
  Eidgen\"ossische Technische Hochschule Z\"urich\\
  Wolfgang-Pauli-Strasse 27, 8093 Z\"urich, Switzerland\\[1ex]
  \href{mailto:nbeisert@itp.phys.ethz.ch}
  {\texttt{nbeisert@itp.phys.ethz.ch}}}
\hypersetup{pdfauthor={Niklas Beisert}}
\hypersetup{pdfsubject={Manual for the LaTeX2e Package childdoc}}
\date{30 December 2018, \textsf{v2.0}}
\maketitle

\begin{abstract}\noindent
\textsf{childdoc} is a \LaTeXe{} package
that enables the direct compilation
of document sections included by |\include|
to individual files.
\end{abstract}

\begingroup
\parskip0ex
\tableofcontents
\endgroup

%%%%%%%%%%%%%%%%%%%%%%%%%%%%%%%%%%%%%%%%%%%%%%%%%%%%%%%%%%%%%%%%%%%%%%%%%%%%%%%%
%%%%%%%%%%%%%%%%%%%%%%%%%%%%%%%%%%%%%%%%%%%%%%%%%%%%%%%%%%%%%%%%%%%%%%%%%%%%%%%%
\section{Introduction}

\LaTeX{} provides a mechanism to structure a large document (such as a book)
into a main file and several child files (containing the chapters)
using the |\include| command.
This mechanism is beneficial for documents
which span hundreds of pages in order to
make the source file(s) more manageable.
Moreover, compilation can be restricted to
selected child files by means of the |\includeonly| command.
The latter feature can be used to reduce the compilation time while editing
(this was significantly more useful in the earlier days of \LaTeX{})
or to generate a smaller document which is easier to navigate.
Another application of |\includeonly| is to generate
documents consisting of selected parts of the complete document.

However, there are a few drawbacks of the plain |\include| mechanism:
\begin{itemize}
\item
The child files cannot be compiled on their own,
they can only be compiled via the main file.
A naive editing environment
(such as a text editor with an option
to have the current file processed by \LaTeX)
may require one to switch to the main file before compiling;
attempting to compile the child file produces errors.
\item
The main file must be modified (each time)
to adjust the |\includeonly| command
to the present needs. This easily leaves the main file in a messy state.
\item
The generated document will always carry the filename
of the main document. This is inconvenient if
several child files are to be compiled and
to be kept for distribution.
\end{itemize}

The present package provides a simple interface
to make child files individually compilable by \LaTeX{}.
Compiling a child file then has the same effect as compiling
the main file with an |\includeonly| command
to select the appropriate child.
Moreover the generated document will carry the name of the child
rather than the main file.
This resolves all three above issues.

This feature is meant to make the editing of books,
thesis documents and lecture notes somewhat more convenient.
However, the package can also be used efficiently for
composing a series of documents (such as exercise sheets)
which are typically distributed individually.
It then assists the author in generating the individual documents
(potentially in different versions)
as well as a document containing the collected series.
Another application is in developing style files
or other kinds of included material
where compilation of the style file could redirect
to a sample or test file.

%%%%%%%%%%%%%%%%%%%%%%%%%%%%%%%%%%%%%%%%%%%%%%%%%%%%%%%%%%%%%%%%%%%%%%%%%%%%%%%%
%%%%%%%%%%%%%%%%%%%%%%%%%%%%%%%%%%%%%%%%%%%%%%%%%%%%%%%%%%%%%%%%%%%%%%%%%%%%%%%%
\section{Usage}

First of all, the package \textsf{childdoc} is \emph{not} a standard
\LaTeXe{} |.sty| style file! Therefore it needs to be invoked in
a non-standard way.

%%%%%%%%%%%%%%%%%%%%%%%%%%%%%%%%%%%%%%%%%%%%%%%%%%%%%%%%%%%%%%%%%%%%%%%%%%%%%%%%
\subsection{Included Files}
\label{sec:include}

%%%%%%%%%%%%%%%%%%%%%%%%%%%%%%%%%%%%%%%%
\DescribeMacro{\childdocmain}
To use the package, add the commands
\begin{center}
\begin{tabular}{l}
|\input{childdoc.def}|\\
|\childdocmain{}|\\
\end{tabular}
\end{center}
at the very top of the main \LaTeX{} file,
in particular \emph{before} the |\documentclass| statement!
The argument of |\childdocmain| should be left empty
(but it must be present).

%%%%%%%%%%%%%%%%%%%%%%%%%%%%%%%%%%%%%%%%
\DescribeMacro{\childdocof}
Furthermore, add the commands
\begin{center}
\begin{tabular}{l}
|\input{childdoc.def}|\\
|\childdocof{|\textit{main}|}|\\
\end{tabular}
\end{center}
at the top of every child file \textit{child}
which is included by |\include{|\textit{child}|}|
from within the main file
(or at least for those files to be compiled individually).
The argument \textit{main} must be the filename of the main file.

There are a couple of
considerations in setting up the main and child documents:

%%%%%%%%%%%%%%%%%%%%%%%%%%%%%%%%%%%%%%%%
\paragraph{Restrictions.}

Please note the following restrictions:
\begin{itemize}
\item
|\childdocmain| must be called with one argument \textit{main}
to ensure compatibility with earlier version of the package.
It must either be empty (|\childdocmain{}|)
or precisely match the filename of the main file in which it is specified.
See \secref{sec:detection} for further information.
\item
The filename \textit{main} must be specified without the |.tex| extension.
\item
The filename \textit{main} is case sensitive
(even in case-insensitive file systems)
due to internal string comparison.
\item
The argument \textit{main} should be fully expanded, it cannot be a macro.
\item
Subdirectories and special characters should be avoided in filenames.
\item
The command |\childdocmain{|\textit{main}|}| must be followed by a whitespace.
It should not be followed immediately by another command
or by a comment mark `|%|'.
This is because the \TeX{} parser reads the token immediately following
the argument of |\childdocmain| and puts it
at the beginning of every child section;
however, a white\-space is ignored.
\end{itemize}

%%%%%%%%%%%%%%%%%%%%%%%%%%%%%%%%%%%%%%%%
\paragraph{Content of Main File.}

It is advisable to place all content in the child files included by |\include|.
Any output contained in the main file will appear in all child documents
unless suppressed manually;
it cannot be suppressed automatically by the |\includeonly| directive
and thus should normally be avoided.
A method to include some content in the main file
by means of conditional processing is described in \secref{sec:conditional}.

%%%%%%%%%%%%%%%%%%%%%%%%%%%%%%%%%%%%%%%%
\paragraph{Page Numbering.}

When only a part of the document is compiled,
the appropriate numbering of pages
(as well as other status parameters)
is determined from the |.aux| files.
The latter contain information from previous passes.
However this information needs to propagate through
all intermediate child documents.
Therefore the page numbering in child documents may well
be inconsistent until the complete document is compiled at least once.

A useful (if unconventional) way to always ensure a consistent
page numbering is to restart the numbering in each child document
and denote the pages by `\textit{child}|.|\textit{page}'
where \textit{child} represents the chapter/section number of the child file.
This can be achieved by the command
|\numberwithin{page}{|\textit{child}|}|
of the \textsf{amsmath} package
where \textit{child} can be |chapter| or |section|
depending on the chosen structuring.
Alternatively, one can modify the macro |\thepage| appropriately
and reset the counter |page| at the start of each child file.

%%%%%%%%%%%%%%%%%%%%%%%%%%%%%%%%%%%%%%%%%%%%%%%%%%%%%%%%%%%%%%%%%%%%%%%%%%%%%%%%
\subsection{Conditional Processing}
\label{sec:conditional}

The package provides a mechanism to compile different versions
of a document. To customise the versions further some conditional processing
can come in handy to distinguish which version is being compiled.
The package provides two macros to describe the compilation context:

%%%%%%%%%%%%%%%%%%%%%%%%%%%%%%%%%%%%%%%%
\DescribeMacro{\ifchilddoc}
The conditional |\ifchilddoc| distinguishes between the compilation of
child documents and the main document:
%
\begin{center}
|\ifchilddoc |\textit{child-code}| |[|\||else |\textit{main-code}]| \||fi|
\end{center}

%%%%%%%%%%%%%%%%%%%%%%%%%%%%%%%%%%%%%%%%
\DescribeMacro{\childdocname}
\DescribeMacro{\childdocjob}
The macro |\childdocname| contains the filename (without extension)
of the main or child file being processed.
Note that |\childdocjob| will always contain the name of the main file.

%%%%%%%%%%%%%%%%%%%%%%%%%%%%%%%%%%%%%%%%
\paragraph{Title Page.}

Conditional processing can be used to include a title or banner page
in the main document when proper precautions are taken.
Importantly, the code in the main file should ensure that the page counter
(as well as other status parameters which are stored in the |.aux| files)
takes the same value after the conditional processing.
Otherwise the page numbers may take divergent values
depending on which part is compiled.

For example, a title page could be declared by:
%
\begin{center}
\begin{tabular}{l}
|\ifchilddoc\||else|\\
|\addtocounter{page}{-1}|\\
\textit{code for title page}\\
|\newpage|\\
|\||fi|
\end{tabular}
\end{center}
%
A banner page for the child documents can be generated by:
%
\begin{center}
\begin{tabular}{l}
|\ifchilddoc|\\
|\addtocounter{page}{-1}|\\
\textit{code for banner page}\\
|\newpage|\\
|\||fi|
\end{tabular}
\end{center}
%
Here one could write a message such as:
\begin{center}
|This is the part \childdocname{} of \childdocjob{}.|
\end{center}

%%%%%%%%%%%%%%%%%%%%%%%%%%%%%%%%%%%%%%%%%%%%%%%%%%%%%%%%%%%%%%%%%%%%%%%%%%%%%%%%
\subsection{Flags}
\label{sec:flags}

The package makes it easy to generate different versions
of the main or child documents.
To this end compilation flags can be defined
and assigned different default values.
They will be particularly useful in conjunction
with the forwarding mechanism described in \secref{sec:forward}.

For example, it may be useful to have a flag |\version|
which can be set to |draft| or |final|.
The document source will contain some conditional code
depending on the value of |\version|.
Suppose further, the flag should default to |final| for the main file
and to |draft| for child files
which is a natural assignment for editing the document.
This is achieved by placing the following code
in the preamble of the main document
(below the |\childdocmain| directive):
%
\begin{center}
\begin{tabular}{l}
|\ifchilddoc|\\
|\providecommand{\version}{draft}|\\
|\||else|\\
|\providecommand{\version}{final}|\\
|\||fi|
\end{tabular}
\end{center}
%
The definition by |\providecommand| makes sure
that previous definitions are not overwritten.
Further statements |\providecommand{\version}{...}|
can thus be added before the above code to override it.

For the main file, one might add a line
(between |\childdocmain| and the above block)
%
\begin{center}
|%\ifchilddoc\||else\providecommand{\version}{draft}\||fi|
\end{center}
%
which can be uncommented to produce a draft version.
Likewise one can add a line to the very top of a child file
(above the |\childdocof{|\textit{main}|}| directive)
%
\begin{center}
|%\providecommand{\version}{final}|
\end{center}
%
which can be uncommented to produce the final version of this child document.

%%%%%%%%%%%%%%%%%%%%%%%%%%%%%%%%%%%%%%%%%%%%%%%%%%%%%%%%%%%%%%%%%%%%%%%%%%%%%%%%
\subsection{Forwarding}
\label{sec:forward}

Different versions of the main or child documents
using compilation flags as described in \secref{sec:flags}
can be (permanently) stored in different files
for convenient compilation, viewing and distribution.
To this end, the package defines a command
to pass on compilation to a different file:

%%%%%%%%%%%%%%%%%%%%%%%%%%%%%%%%%%%%%%%%
\DescribeMacro{\childdocforward}
The command |\childdocforward| redirects processing to
another source file:
%
\begin{center}
\begin{tabular}{l}
|\input{childdoc.def}|\\
|\childdocforward[|\textit{main}|]{|\textit{dest}|}|\\
\end{tabular}
\end{center}
%
The argument \textit{dest} is the destination file
(without extension).
It should be the main file or one of the child files.
Note that further \textsf{childdoc} directives
such as |\childdocof| and |\childdocforward|
in the indicated file will be processed in this form.
The optional argument \textit{main}
passes on directly to the main file \textit{main}
while pretending to compile the child \textit{dest}.
This form behaves as if \textit{dest}
issues |\childdocof{|\textit{main}|}| right away,
and no further \textsf{childdoc} directives will be processed.

%%%%%%%%%%%%%%%%%%%%%%%%%%%%%%%%%%%%%%%%
\DescribeMacro{\...prefix}
In the alternative form |\childdocforwardprefix|,
%
\begin{center}
\begin{tabular}{l}
|\input{childdoc.def}|\\
|\childdocforwardprefix[|\textit{main}|]{|\textit{prefix}|}{|\textit{dest}|}|
\end{tabular}
\end{center}
%
the destination file is determined by a pattern
depending on the current file:
To make this work, the current file must be called
`{\textit{prefix}\hspace{0.2em}\textit{suffix}}'
with \textit{prefix} matching precisely the argument.
Processing is then passed on to the file
`{\textit{dest}\hspace{0.2em}\textit{suffix}}'.
Surely, the same effect is achieved by
directly specifying the
argument `{\textit{dest}\hspace{0.2em}\textit{suffix}}'
in the first form.
However, that requires to set up a different file
for each child. With the alternative form of the command
all these files can have exactly the same content
which simplifies setting them up and maintaining them.

For example, the following file |draft.tex|
with a compilation flag |\version| as described in \secref{sec:flags}
compiles the main document as a draft:
%
\begin{center}
\begin{tabular}{l}
|\def\version{draft}|\\
|\input{childdoc.def}|\\
|\childdocforward{|\textit{main}|}|
\end{tabular}
\end{center}
%
Likewise, the following files |final|\textit{nn}|.tex|
compile the final version of the child document
|child|\textit{nn}|.tex|:
%
\begin{center}
\begin{tabular}{l}
|\def\version{final}|\\
|\input{childdoc.def}|\\
|\childdocforwardprefix{final}{child}|
\end{tabular}
\end{center}
%

Note that when several versions of a main file and/or of each child file
are to be generated, it may be convenient to set up a |Makefile| or
shell script to automatise the process.

%%%%%%%%%%%%%%%%%%%%%%%%%%%%%%%%%%%%%%%%%%%%%%%%%%%%%%%%%%%%%%%%%%%%%%%%%%%%%%%%
\subsection{Command Line Processing}
\label{sec:commandline}

The effect of redirection files can also be achieved by invoking
the \LaTeX{} compiler with a more elaborate command line.
Most conveniently this should be done as part
of a shell script or a |Makefile|.

When using \textsf{childdoc} in the main file, the following
command lines effectively perform a redirection
(note that depending on the shell being used,
backslashes may have to be doubled: `|\|' $\to$ `|\\|'):
%
\begin{center}
|... -jobname "|\textit{target}|" |\\|"|[\textit{flags}]%
|\input{childdoc.def}\childdocforward[|\textit{main}|]{|\textit{dest}|}"|
\end{center}
%
Here \textit{target} is the name of the output file,
\textit{main} is the name of the main file
and \textit{dest} is the name of the main or child file to be processed
(all filenames without extensions).
The optional argument \textit{main} can be omitted
if \textit{main} matches \textit{dest}.
Optionally, compilation \textit{flags} can be defined via |\def| commands.
This command line makes the \TeX{} engine believe
it is compiling the file \textit{target}
whose content is specified as the latter parameter.
The provided code then forwards the processing to
\textit{main} or \textit{dest} as described in \secref{sec:forward}.

%%%%%%%%%%%%%%%%%%%%%%%%%%%%%%%%%%%%%%%%%%%%%%%%%%%%%%%%%%%%%%%%%%%%%%%%%%%%%%%%
\subsection{Include by Input}
\label{sec:input}

Including child documents by |\include| has some restrictions by design.
Most notably, the content of a child document always occupies
its own set of pages; pages cannot be shared between child documents.
Usually, this behaviour makes perfect sense
because each child document contain an essential part of the document.
However, in some situations it may be desirable to compose
a document from a collection of parts
without having mandatory page breaks between then.
For this case, the package
provides a mechanism to include parts
by |\input| which can also be processed individually.
However, by construction this mechanism
requires manual handling of the content to be output.

%%%%%%%%%%%%%%%%%%%%%%%%%%%%%%%%%%%%%%%%
\DescribeMacro{\ifchilddocmanual}
The main file should be prepared as usual, see \secref{sec:include}.
However, the document body must make a distinction
between processing of an individual part and of the main document, e.g.:
%
\begin{center}
\begin{tabular}{l}
|\ifchilddocmanual|\\
|\input{\childdocname}|\\
|\||else|\\
\textit{document body with }|\input{|\textit{part}|}|\\
|\||fi|
\end{tabular}
\end{center}
%
The conditional |\ifchilddocmanual| is true whenever
a part to be included by |\input| is being compiled,
and the name of the part is stored in |\childdocname|.

%%%%%%%%%%%%%%%%%%%%%%%%%%%%%%%%%%%%%%%%
\DescribeMacro{\childdocby}
Each part to be included by |\input| should start with:
%
\begin{center}
\begin{tabular}{l}
|\input{childdoc.def}|\\
|\childdocby{|\textit{main}|}|\\
\end{tabular}
\end{center}
%
The directive |\childdocby| is similar to |\childdocof|
described in \secref{sec:include},
but the subsequent selection of content must be done manually.
To that end, both |\ifchilddoc| and |\ifchilddocmanual|
will be true upon processing of a part,
and the name of the part is stored in |\childdocname|.
Note that |\jobname| will be set to the filename of the current part
so that each part receives an individual |.aux| file
that does not interfere with the |.aux| file(s) of the main document.
This behaviour can be altered by the alternative form
|\childdocby[*]{|\textit{main}|}| (with a non-empty optional argument)
which uses the |.aux| file of the main document
by setting |\jobname| to \textit{main}.

%%%%%%%%%%%%%%%%%%%%%%%%%%%%%%%%%%%%%%%%%%%%%%%%%%%%%%%%%%%%%%%%%%%%%%%%%%%%%%%%
\subsection{Driver Development}
\label{sec:driver}

The \textsf{childdoc} mechanism can also be use for the development
of definition files such as \LaTeX{} styles or classes.
This case differs from the above setup with multiple parts
included by |\include| in that no |\includeonly| should be invoked.
This can be achieved by starting the include file
(before |\ProvidesPackage|) with:
%
\begin{center}
\begin{tabular}{l}
|\input{childdoc.def}|\\
|\childdocforward{|\textit{main}|}|\\
\end{tabular}
\end{center}
%
or alternatively with:
%
\begin{center}
\begin{tabular}{l}
|\input{childdoc.def}|\\
|\childdocby{|\textit{main}|}|\\
\end{tabular}
\end{center}
%
Both forms have slightly different effects as described above.
The main file is prepared as usual, see \secref{sec:include}.

%%%%%%%%%%%%%%%%%%%%%%%%%%%%%%%%%%%%%%%%%%%%%%%%%%%%%%%%%%%%%%%%%%%%%%%%%%%%%%%%
\subsection{Legacy Detection}
\label{sec:detection}

The directive |\childdocmain| in the main file can detect
whether the complete document or merely a child is to be compiled
even without using the directive |\childdocof|.
This method is deprecated because it is less robust
and there is no compelling reason to use it;
it is merely provided for backward compatibility
and it may be removed in future versions.

If the detection mechanism is to be used,
it is mandatory to correctly specify
the filename of the main file as the argument of |\childdocmain|:
%
\begin{center}
\begin{tabular}{l}
|\input{childdoc.def}|\\
|\childdocmain{|\textit{main}|}|\\
\end{tabular}
\end{center}
%
If |\jobname| does not match the argument \textit{main} of |\childdocmain|,
it is assumed that |\jobname| points to the child file to be compiled.
When using |\childdocmain| with the main file specified as argument,
it suffices to start a child file
with just |\input{|\textit{main}|}|
without loading of the package and using |\childdocof|.
If instead all processing is done
with the appropriate \textsf{childdoc} directives,
the argument of \textit{main} of |\childdocmain| can be empty.

An alternative version of the command line processing described
in \secref{sec:commandline} using the detection mechanism reads:
%
\begin{center}
|... -jobname "|\textit{target}|" "|[\textit{flags}]%
[|\def\jobname{|\textit{dest}|}|]|\input{|\textit{main}|}"|
\end{center}

%%%%%%%%%%%%%%%%%%%%%%%%%%%%%%%%%%%%%%%%%%%%%%%%%%%%%%%%%%%%%%%%%%%%%%%%%%%%%%%%
\subsection{Manual Code}
\label{sec:manual}

In case one cannot be certain whether the definitions file |childdoc.def|
is installed on the target \TeX{} distribution
and one prefers not to ship it,
it is conceivable to paste a few relevant commands into the sources.

To that end, drop all statements |\input{childdoc.def}|
and perform the replacements as outlined below.
Instead of |\childdocmain{|\textit{main}|}| add the following code
to the top of the main file:
%
\begin{center}
\begin{tabular}{l}
|\||ifdefined\childdocname\endinput\||fi\newif\ifchilddoc|\\
|\edef\childdocname{\scantokens\expandafter{\jobname\noexpand}}|\\
|\def\childdocmain{|\textit{main}|}\||ifx\childdocmain\childdocname\||else|\\
|\childdoctrue\includeonly{\childdocname}\let\jobname\childdocmain\||fi|\\
\end{tabular}
\end{center}
%
Instead of |\childdocof{|\textit{main}|}| just include the main file
at the top of each child file:
%
\begin{center}
|\input{|\textit{main}|}|
\end{center}
%
A simple redirection |\childdocforward{|\textit{dest}|}| is achieved by:
%
\begin{center}
|\def\jobname{|\textit{dest}|}\input{\jobname}|
\end{center}
%
The redirection with prefix
|\childdocforwardprefix[|\textit{prefix}|]{|\textit{dest}|}|
is accomplished by:
%
\begin{center}
\begin{tabular}{l}
|{\edef\jobname{\scantokens\expandafter{\jobname\noexpand}}|\\
|\def\redirectjob |\textit{prefix}|#1~~~{\gdef\jobname{|\textit{dest}|#1}}|\\
|\expandafter\redirectjob\jobname~~~}\input{\jobname}|
\end{tabular}
\end{center}

In an alternative approach,
child documents can be compiled by a specific command line
without additional code or specific definitions:
%
\begin{center}
|... -jobname "|\textit{target}|" "|[\textit{flags}]%
|\includeonly{|\textit{dest}|}\input{|\textit{main}|}"|
\end{center}
%

%%%%%%%%%%%%%%%%%%%%%%%%%%%%%%%%%%%%%%%%%%%%%%%%%%%%%%%%%%%%%%%%%%%%%%%%%%%%%%%%
%%%%%%%%%%%%%%%%%%%%%%%%%%%%%%%%%%%%%%%%%%%%%%%%%%%%%%%%%%%%%%%%%%%%%%%%%%%%%%%%
\section{Information}

%%%%%%%%%%%%%%%%%%%%%%%%%%%%%%%%%%%%%%%%%%%%%%%%%%%%%%%%%%%%%%%%%%%%%%%%%%%%%%%%
\subsection{Copyright}

Copyright \copyright{} 2017--2018 Niklas Beisert

This work may be distributed and/or modified under the
conditions of the \LaTeX{} Project Public License, either version 1.3
of this license or (at your option) any later version.
The latest version of this license is in
  \url{http://www.latex-project.org/lppl.txt}
and version 1.3 or later is part of all distributions of \LaTeX{}
version 2005/12/01 or later.

This work has the LPPL maintenance status `maintained'.

The Current Maintainer of this work is Niklas Beisert.

This work consists of the files |README.txt|, |childdoc.ins| and |childdoc.dtx|
as well as the derived files |childdoc.def|, |cdocsamp.tex|
with |cdocsch1.tex|, |cdocsch2.tex|, |cdocspt3.tex|, |cdocspt4.tex|,
|cdocsdrf.tex|, |cdocsfn1.tex|, |cdocsfn2.tex|
as well as |childdoc.pdf|.

%%%%%%%%%%%%%%%%%%%%%%%%%%%%%%%%%%%%%%%%%%%%%%%%%%%%%%%%%%%%%%%%%%%%%%%%%%%%%%%%
\subsection{Files and Installation}

The package consists of the files:
%
\begin{center}
\begin{tabular}{ll}
    |README.txt|   & readme file \\
    |childdoc.ins| & installation file \\
    |childdoc.dtx| & source file \\
    |childdoc.def| & definition file \\
    |cdocsamp.tex| & sample main file \\
    |cdocsch1.tex| & sample include file \\
    |cdocsch2.tex| & sample include file \\
    |cdocspt3.tex| & sample part file \\
    |cdocspt4.tex| & sample part file \\
    |cdocsdrf.tex| & sample redirection file \\
    |cdocsfn1.tex| & sample redirection file \\
    |cdocsfn2.tex| & sample redirection file \\
    |childdoc.pdf| & manual
\end{tabular}
\end{center}
%
The distribution consists of the files
|README.txt|, |childdoc.ins| and |childdoc.dtx|.
%
\begin{itemize}
\item
Run (pdf)\LaTeX{} on |childdoc.dtx|
to compile the manual |childdoc.pdf| (this file).
\item
Run \LaTeX{} on |childdoc.ins| to create the definitions file |childdoc.def|
and the sample |cdocsamp.tex| with include files
|cdocsch1.tex|, |cdocsch2.tex|, |cdocspt3.tex|, |cdocspt4.tex|,
|cdocsdrf.tex|, |cdocsfn1.tex|, |cdocsfn2.tex|.
Then copy the file |childdoc.def| to an appropriate directory of your \LaTeX{}
distribution, e.g.\ \textit{texmf-root}|/tex/latex/childdoc|.
\end{itemize}

%%%%%%%%%%%%%%%%%%%%%%%%%%%%%%%%%%%%%%%%%%%%%%%%%%%%%%%%%%%%%%%%%%%%%%%%%%%%%%%%
\subsection{Related CTAN Packages}

There are several other packages which offer a similar functionality:
%
\begin{itemize}
\item
The packages
\href{http://ctan.org/pkg/docmute}{\textsf{docmute}},
\href{http://ctan.org/pkg/includex}{\textsf{includex}} and
\href{http://ctan.org/pkg/standalone}{\textsf{standalone}}
provide commands to include only the document body of
a child file thus allowing both files to be compiled individually.
\item
The packages \href{http://ctan.org/pkg/subdocs}{\textsf{subdocs}}
and \href{http://ctan.org/pkg/subfiles}{\textsf{subfiles}}
provide structures in which the main and child documents can be
encapsulated and allowing them to be compiled individually.
The inclusion mechanism is different from the conventional |\include|.
\item
The package \href{http://ctan.org/pkg/combine}{\textsf{combine}}
is an elaborate solution to combine several documents into one.
\end{itemize}
%
See also the CTAN topic \href{http://ctan.org/topic/subdocs}{\textsf{subdocs}}
for further related packages.
The present package differs from the above solutions in that
a document structure constructed with the conventional |\include| mechanism
just needs two extra commands at the top of every file
such that all constituent files can be compiled individually.

%%%%%%%%%%%%%%%%%%%%%%%%%%%%%%%%%%%%%%%%%%%%%%%%%%%%%%%%%%%%%%%%%%%%%%%%%%%%%%%%
%\subsection{Feature Suggestions}
%
%The following is a list of features which may be useful for future
%versions of this package:
%%
%\begin{itemize}
%\item
%\ldots
%\end{itemize}

%%%%%%%%%%%%%%%%%%%%%%%%%%%%%%%%%%%%%%%%%%%%%%%%%%%%%%%%%%%%%%%%%%%%%%%%%%%%%%%%
\subsection{Revision History}

%%%%%%%%%%%%%%%%%%%%%%%%%%%%%%%%%%%%%%%%
\paragraph{v2.0:} 2018/12/30

\begin{itemize}
\item
immediate forward processing
\item
added |\childdocby| mechanism
\item
manual restructured
\end{itemize}

%%%%%%%%%%%%%%%%%%%%%%%%%%%%%%%%%%%%%%%%
\paragraph{v1.6:} 2018/01/17

\begin{itemize}
\item
application for development of include files
\item
corrections to manual
\end{itemize}

%%%%%%%%%%%%%%%%%%%%%%%%%%%%%%%%%%%%%%%%
\paragraph{v1.5:} 2017/05/21

\begin{itemize}
\item
more complete structuring introduced
\item
|\childdocof| introduced
\item
|\childdoc| renamed to |\childdocmain|
\item
|\childredirect| renamed to |\childdocforward| and |\childdocforwardprefix|
and functionality expanded
\end{itemize}

%%%%%%%%%%%%%%%%%%%%%%%%%%%%%%%%%%%%%%%%
\paragraph{v1.0:} 2017/04/27

\begin{itemize}
\item
manual and install package
\item
first version published on CTAN
\end{itemize}

%%%%%%%%%%%%%%%%%%%%%%%%%%%%%%%%%%%%%%%%
\paragraph{v0.6:} 2017/04/26

\begin{itemize}
\item
redirection mechanism added
\end{itemize}

%%%%%%%%%%%%%%%%%%%%%%%%%%%%%%%%%%%%%%%%
\paragraph{v0.5:} 2017/04/26

\begin{itemize}
\item
functionality in definition file
\end{itemize}


%%%%%%%%%%%%%%%%%%%%%%%%%%%%%%%%%%%%%%%%%%%%%%%%%%%%%%%%%%%%%%%%%%%%%%%%%%%%%%%%
%%%%%%%%%%%%%%%%%%%%%%%%%%%%%%%%%%%%%%%%%%%%%%%%%%%%%%%%%%%%%%%%%%%%%%%%%%%%%%%%
%%%%%%%%%%%%%%%%%%%%%%%%%%%%%%%%%%%%%%%%%%%%%%%%%%%%%%%%%%%%%%%%%%%%%%%%%%%%%%%%
\appendix

\settowidth\MacroIndent{\rmfamily\scriptsize 000\ }

 \DocInput{childdoc.dtx}

\end{document}
%</driver>
% \fi
%
% %%%%%%%%%%%%%%%%%%%%%%%%%%%%%%%%%%%%%%%%%%%%%%%%%%%%%%%%%%%%%%%%%%%%%%%%%%%%%%
% %%%%%%%%%%%%%%%%%%%%%%%%%%%%%%%%%%%%%%%%%%%%%%%%%%%%%%%%%%%%%%%%%%%%%%%%%%%%%%
% \section{Sample}
%\iffalse
%<*samplemain>
%\fi
%
% The following presents a sample document
% with two chapters, two parts, a title page,
% a compile flag as well as three forwarding files to set the flag.
% It consists of eight |.tex| files:
% \begin{center}
% \begin{tabular}{ll}
% |cdocsamp.tex|&main file\\
% |cdocsch1.tex|&include file for chapter 1\\
% |cdocsch2.tex|&include file for chapter 2\\
% |cdocspt3.tex|&include file for part 3\\
% |cdocspt4.tex|&include file for part 4\\
% |cdocsdrf.tex|&forwarding file for main file in draft mode\\
% |cdocsfi1.tex|&forwarding file for final version of chapter 1\\
% |cdocsfi2.tex|&forwarding file for final version of chapter 2\\
% \end{tabular}
% \end{center}
% Each of the eight files can be compiled directly by the \LaTeX{} compiler.
%
% %%%%%%%%%%%%%%%%%%%%%%%%%%%%%%%%%%%%%%
% \paragraph{Main File.}
%
% The main file is called |cdocsamp.tex|.
%
% Load the \textsf{childdoc} definitions and
% declare the filename for the main document:
%    \begin{macrocode}
\input{childdoc.def}
\childdocmain{}
%    \end{macrocode}

% Optional override for |\version| flag:
%    \begin{macrocode}
%%\ifchilddoc\else\providecommand{\version}{draft}\fi
%    \end{macrocode}

% Define the default values for the |\version| flag
% (|final| for the main file and |draft| for childs):
%    \begin{macrocode}
\ifchilddoc
\providecommand{\version}{draft}
\else
\providecommand{\version}{final}
\fi
%    \end{macrocode}

% Load the standard document class:
%    \begin{macrocode}
\documentclass[12pt]{article}
%    \end{macrocode}

% Start the document body:
%    \begin{macrocode}
\begin{document}
%    \end{macrocode}

% Declare a title page.
% Print title, part of document being processed and version flag:
%    \begin{macrocode}
\addtocounter{page}{-1}
\begin{center}
{\LARGE\bfseries{}childdoc example\par}
\vspace{1cm}
\ifchilddoc
\ifchilddocmanual part\else chapter\fi:
`\childdocname' of `\childdocjob'\par
\else
main document: `\childdocjob'\par
\fi
version: \version\par
\end{center}
\newpage
%    \end{macrocode}

% Manually include selected file,
% otherwise process as usual:
%    \begin{macrocode}
\ifchilddocmanual
\section*{part `\childdocname'}
\input{\childdocname}
\else
%    \end{macrocode}

% Include the two chapters:
%    \begin{macrocode}
\include{cdocsch1}
\include{cdocsch2}
%    \end{macrocode}

% Include the two parts unless only chapters should be displayed:
%    \begin{macrocode}
\ifchilddoc\else
\section{part three}
\input{cdocspt3}
\section{part four}
\input{cdocspt4}
\fi
%    \end{macrocode}

% Process as usual until here:
%    \begin{macrocode}
\fi
%    \end{macrocode}

% End of document body:
%    \begin{macrocode}
\end{document}
%    \end{macrocode}
%\iffalse
%</samplemain>
%\fi
%
% %%%%%%%%%%%%%%%%%%%%%%%%%%%%%%%%%%%%%%
% \paragraph{Chapter Include Files.}
%
% The include files are called |cdocsch1.tex| and |cdocsch2.tex|.
%
%\iffalse
%<*samplechap1|samplechap2>
%\fi

% Optional override for |\version| flag:
%    \begin{macrocode}
%%\providecommand{\version}{final}
%    \end{macrocode}

% Include the main document:
%    \begin{macrocode}
\input{childdoc.def}
\childdocof{cdocsamp}
%    \end{macrocode}

%\iffalse
%</samplechap1|samplechap2>
%\fi
%
%\iffalse
%<*samplechap1>
%\fi
% Some text for chapter 1:
%    \begin{macrocode}
\section{one}
some text in chapter one
%    \end{macrocode}

%\iffalse
%</samplechap1>
%\fi
% Some text for chapter 2:
%\iffalse
%<*samplechap2>
%\fi
%    \begin{macrocode}
\section{two}
more text in chapter two
%    \end{macrocode}

%\iffalse
%</samplechap2>
%\fi
%
% %%%%%%%%%%%%%%%%%%%%%%%%%%%%%%%%%%%%%%
% \paragraph{Part Include Files.}
%
% The include files are called |cdocspt3.tex| and |cdocspt4.tex|.
%
%\iffalse
%<*samplepart3|samplepart4>
%\fi

% Optional override for |\version| flag:
%    \begin{macrocode}
%%\providecommand{\version}{final}
%    \end{macrocode}

% Include the main document:
%    \begin{macrocode}
\input{childdoc.def}
\childdocby{cdocsamp}
%    \end{macrocode}

%\iffalse
%</samplepart3|samplepart4>
%\fi
%
%\iffalse
%<*samplepart3>
%\fi
% Some text for part 3:
%    \begin{macrocode}
some text in part three
%    \end{macrocode}

%\iffalse
%</samplepart3>
%\fi
% Some text for part 4:
%\iffalse
%<*samplepart4>
%\fi
%    \begin{macrocode}
more text in part four
%    \end{macrocode}

%\iffalse
%</samplepart4>
%\fi
%
% %%%%%%%%%%%%%%%%%%%%%%%%%%%%%%%%%%%%%%
% \paragraph{Forwarding for a Complete Draft.}
%
% The following forwarding file |cdocsdrf.tex|
% compiles the main document in draft mode:
%\iffalse
%<*sampledraft>
%\fi
%    \begin{macrocode}
\def\version{draft}
\input{childdoc.def}
\childdocforward{cdocsamp}
%    \end{macrocode}

%\iffalse
%</sampledraft>
%\fi
%
% %%%%%%%%%%%%%%%%%%%%%%%%%%%%%%%%%%%%%%
% \paragraph{Forwarding for Final Version of the Chapters.}
%
% The following forwarding files |cdocsfn1.tex| and |cdocsfn2.tex|
% (with identical content)
% compile the final versions of the child documents
% |cdocsch1.tex| and |cdocsch2.tex|, respectively:
%\iffalse
%<*samplefinal>
%\fi
%    \begin{macrocode}
\def\version{final}
\input{childdoc.def}
\childdocforwardprefix[cdocsamp]{cdocsfn}{cdocsch}
%    \end{macrocode}

%\iffalse
%</samplefinal>
%\fi
%
% %%%%%%%%%%%%%%%%%%%%%%%%%%%%%%%%%%%%%%
% \paragraph{Command Line Processing.}
%
% The following three command lines generate the output files
% |cdocscld|, |cdocscl1| and |cdocscl2|
% which should be identical to
% |cdocsdrf|, |cdocsch1| and |cdocsfn2|, respectively:
% \begin{center}
% \begin{tabular}{l}
% |latex -jobname cdocscld \|\\
% |  "\def\version{draft}\input{childdoc.def}\childdocforward{cdocsamp}"|\\
% |latex -jobname cdocscl1 \|\\
% |  "\input{childdoc.def}\childdocforward[cdocsamp]{cdocsch1}"|\\
% |latex -jobname cdocscl2 \|\\
% |  "\def\version{final}\input{childdoc.def}\childdocforward{cdocsch2}"|
% \end{tabular}
% \end{center}
% Note that the trailing backslash on each first line
% merely continues the input to the second line
% (for convenient cut ant paste).
% Furthermore, the command |latex| can be replaced by any
% of its alternative versions such as |pdflatex|.
%
% %%%%%%%%%%%%%%%%%%%%%%%%%%%%%%%%%%%%%%%%%%%%%%%%%%%%%%%%%%%%%%%%%%%%%%%%%%%%%%
% %%%%%%%%%%%%%%%%%%%%%%%%%%%%%%%%%%%%%%%%%%%%%%%%%%%%%%%%%%%%%%%%%%%%%%%%%%%%%%
% \section{Implementation}
%\iffalse
%<*package>
%\fi
%
% This section describes the definitions file |childdoc.def|.

% The definitions cannot be loaded using |\usepackage| or |\RequirePackage|
% which has a mechanism to prevent loading a style file more than once.
% When loading the definitions by means of |\input|
% multiple instances have to be prevented manually:
%\iffalse
%This code needs to be before the `\ProvidesFile' directive
%which is defined at the beginning of this file.
%Therefore it is also placed there and commented out here.
%</package>
%<*discard>
%\fi
%    \begin{macrocode}
\ifdefined\childdocmain\endinput\fi
%    \end{macrocode}
%\iffalse
%</discard>
%<*package>
%\fi
%
% \macro{\ifchilddoc}
% \macro{\ifchilddocmanual}
% The conditional |\ifchilddoc| tells whether a
% child (true) or main (false) document is being compiled.
% The conditional |\ifchilddocmanual| tells whether
% the |\includeonly| mechanism is used (false) or
% the selection of child files must be performed manually (true).
% The definitions initialise to false:
%    \begin{macrocode}
\newif\ifchilddoc
\newif\ifchilddocmanual
%    \end{macrocode}

% \macro{\childdocname}
% \macro{\childdocjob}
% The macro |\childdocname| stores the name of the main document
% to be compiled. The macro |\childdocjob| stores the name of
% the document on which the \LaTeX{} compiler was originally invoked.
% The content of |\jobname| cannot be compared
% to filenames specified in the source due to different catcodes.
% The following code rescans |\jobname|, stores the result
% in |\childdocname| and saves a copy in |\childdocjob|:
%    \begin{macrocode}
\edef\childdocname{\scantokens\expandafter{\jobname\noexpand}}
\let\childdocjob\childdocname
%    \end{macrocode}

% \macro{\childdocdisable}
% The macro |\childdocdisable| prevents the main file
% from being processed more than once.
% At this stage, the main document command |\childdocmain|
% is assumed to be called once again where it should do nothing.
% Any subsequent call to it should prevent
% a secondary processing of the main document
% It overwrites the forwarding commands
% |\childdocof| and |\childdocforward|
% with empty macros to prevent further inclusions of the main document:
%    \begin{macrocode}
\newcommand{\childdocdisable}
{
  \renewcommand{\childdocmain}[1]{\renewcommand{\childdocmain}[1]{\endinput}}
  \renewcommand{\childdocof}[1]{}
  \renewcommand{\childdocby}[2][]{}
  \renewcommand{\childdocforward}[2][]{}
  \renewcommand{\childdocdisable}{}
}
%    \end{macrocode}

% \macro{\childdocmain}
% The macro |\childdocmain| is to be called at the top of the main file
% with nothing or the main filename (without extension) as argument.
% First, it breaks loops.
% If the argument is not empty and does not match |\childdocname|
% (which is set by the first inclusion of |childdoc.def|),
% |\ifchilddoc| is set to true, |\includeonly| is applied to the child file
% and |\jobname| is set to the main file
% (for proper handling of |.aux| files):
%    \begin{macrocode}
\newcommand{\childdocmain}[1]
{
  \childdocdisable\childdocmain{}
  \if?#1?\else
    \begingroup
      \def\childdoctmp{#1}
      \ifx\childdoctmp\childdocname
        \def\childdoctmp{}
      \else
        \def\childdoctmp
        {
          \childdoctrue
          \includeonly{\childdocname}
          \def\childdocjob{#1}
          \def\jobname{#1}
        }
      \fi
      \expandafter
    \endgroup
    \childdoctmp
  \fi
}
%    \end{macrocode}

% \macro{\childdocof}
% The command |\childdocof| redirects
% compilation to the main file |#1|.
%    \begin{macrocode}
\newcommand{\childdocof}[1]
{
  \childdocdisable
  \childdoctrue
  \includeonly{\childdocname}
  \def\jobname{#1}
  \def\childdocjob{#1}
  \input{#1}
}
%    \end{macrocode}

% \macro{\childdocby}
% The command |\childdocby| ....
%    \begin{macrocode}
\newcommand{\childdocby}[2][]
{
  \childdocdisable
  \childdoctrue
  \childdocmanualtrue
  \if?#1?\else
    \def\jobname{#2}
  \fi
  \def\childdocjob{#2}
  \input{#2}
  \endinput
}
%    \end{macrocode}

% \macro{\childdocforward}
% The command |\childdocforward| redirects
% compilation to the main file or
% (if the optional argument is given) a child file.
% Parameters are set as if the main file
% or a child file starting with |\childdocof| was compiled.
% Then compilation is handed over to the main file:
%    \begin{macrocode}
\newcommand{\childdocforward}[2][]
{
  \begingroup
    \if?#1?
      \def\childdoctmp
      {
        \def\childdocname{#2}
        \def\childdocjob{#2}
        \def\jobname{#2}
        \input{#2}
        \endinput
      }
    \else
      \def\childdoctmp
      {
        \childdocdisable
        \def\childdocname{#2}
        \childdoctrue
        \includeonly{#2}
        \def\childdocjob{#1}
        \def\jobname{#1}
        \input{#1}
        \endinput
      }
    \fi
    \expandafter
  \endgroup
  \childdoctmp
}
%    \end{macrocode}

% \macro{\childdocforwardprefix}
% The command |\childdocforwardprefix| redirects
% compilation to the main or a child file by means of a pattern.
% The prefix |#1| in the current filename is replaced by |#2|
% and the suffix of the current filename is kept
% (it is assumed that the filename does not contain the substring `|~~~|'
% which is used as a delimiter).
% Compilation is handed over to the new file by |\childdocforward|:
%    \begin{macrocode}
\newcommand{\childdocforwardprefix}[3][]
{
  \begingroup
    \def\childdocextract #2##1~~~{\def\childdoctmp{\childdocforward[#1]{#3##1}}}
    \expandafter\childdocextract\childdocname~~~
    \expandafter
  \endgroup
  \childdoctmp
}
%    \end{macrocode}

% \macro{\childdoc}
% The deprecated macro |\childdoc| is a legacy version of |\childdocmain|:
%    \begin{macrocode}
\newcommand{\childdoc}{\childdocmain}
%    \end{macrocode}

% \macro{\childdocredirect}
% The deprecated macro |\childdocredirect| is a legacy version
% of |\childdocforward| and |\childdocforwardprefix|:
%    \begin{macrocode}
\newcommand{\childdocredirect}[2][]
{
  \begingroup
    \if?#1?
      \def\childdoctmp{\childdocforward{#2}}
    \else
      \def\childdoctmp{\childdocforwardprefix{#1}{#2}}
    \fi
    \expandafter
  \endgroup
  \childdoctmp
}
%    \end{macrocode}

%\iffalse
%</package>
%\fi
%
\endinput
|\\
|\childdocforward[|\textit{main}|]{|\textit{dest}|}|\\
\end{tabular}
\end{center}
%
The argument \textit{dest} is the destination file
(without extension).
It should be the main file or one of the child files.
Note that further \textsf{childdoc} directives
such as |\childdocof| and |\childdocforward|
in the indicated file will be processed in this form.
The optional argument \textit{main}
passes on directly to the main file \textit{main}
while pretending to compile the child \textit{dest}.
This form behaves as if \textit{dest}
issues |\childdocof{|\textit{main}|}| right away,
and no further \textsf{childdoc} directives will be processed.

%%%%%%%%%%%%%%%%%%%%%%%%%%%%%%%%%%%%%%%%
\DescribeMacro{\...prefix}
In the alternative form |\childdocforwardprefix|,
%
\begin{center}
\begin{tabular}{l}
|% \iffalse
%
% childdoc.dtx Copyright (C) 2017-2018 Niklas Beisert
%
% This work may be distributed and/or modified under the
% conditions of the LaTeX Project Public License, either version 1.3
% of this license or (at your option) any later version.
% The latest version of this license is in
%   http://www.latex-project.org/lppl.txt
% and version 1.3 or later is part of all distributions of LaTeX
% version 2005/12/01 or later.
%
% This work has the LPPL maintenance status `maintained'.
%
% The Current Maintainer of this work is Niklas Beisert.
%
% This work consists of the files childdoc.dtx and childdoc.ins
% and the derived files childdoc.def and cdocsamp.tex with
% cdocsch1.tex, cdocsch2.tex, cdocsdrf.tex, cdocsfn1.tex, cdocsfn2.tex.
%
%<package>\ifdefined\childdocmain\endinput\fi
%<package>\ProvidesFile{childdoc.def}[2018/12/30 v2.0 child document driver]
%<samplemain>\ProvidesFile{cdocsamp.tex}[2018/12/30 v2.0 sample for childdoc]
%<*driver>
%\ProvidesFile{childdoc.drv}[2018/12/30 v2.0 childdoc reference manual file]
\PassOptionsToClass{10pt,a4paper}{article}
\documentclass{ltxdoc}

\usepackage[margin=35mm]{geometry}
\usepackage{hyperref}
\usepackage{hyperxmp}
\usepackage[usenames]{color}

\hypersetup{colorlinks=true}
\hypersetup{pdfstartview=FitH}
\hypersetup{pdfpagemode=UseNone}
\hypersetup{pdfsource={}}
\hypersetup{pdflang={en-UK}}
\hypersetup{pdfcopyright={Copyright 2017-2018 Niklas Beisert.
  This work may be distributed and/or modified under the
  conditions of the LaTeX Project Public License, either version 1.3
  of this license or (at your option) any later version.}}
\hypersetup{pdflicenseurl={http://www.latex-project.org/lppl.txt}}
\hypersetup{pdfcontactaddress={ETH Zurich, ITP, HIT K,
  Wolfgang-Pauli-Strasse 27}}
\hypersetup{pdfcontactpostcode={8093}}
\hypersetup{pdfcontactcity={Zurich}}
\hypersetup{pdfcontactcountry={Switzerland}}
\hypersetup{pdfcontactemail={nbeisert@itp.phys.ethz.ch}}
\hypersetup{pdfcontacturl={http://people.phys.ethz.ch/\xmptilde nbeisert/}}

\newcommand{\secref}[1]{\hyperref[#1]{section \ref*{#1}}}

\parskip1ex
\parindent0pt
\let\olditemize\itemize
\def\itemize{\olditemize\parskip0pt}

\begin{document}

\title{The \textsf{childdoc} Package}
\hypersetup{pdftitle={The childdoc Package}}
\author{Niklas Beisert\\[2ex]
  Institut f\"ur Theoretische Physik\\
  Eidgen\"ossische Technische Hochschule Z\"urich\\
  Wolfgang-Pauli-Strasse 27, 8093 Z\"urich, Switzerland\\[1ex]
  \href{mailto:nbeisert@itp.phys.ethz.ch}
  {\texttt{nbeisert@itp.phys.ethz.ch}}}
\hypersetup{pdfauthor={Niklas Beisert}}
\hypersetup{pdfsubject={Manual for the LaTeX2e Package childdoc}}
\date{30 December 2018, \textsf{v2.0}}
\maketitle

\begin{abstract}\noindent
\textsf{childdoc} is a \LaTeXe{} package
that enables the direct compilation
of document sections included by |\include|
to individual files.
\end{abstract}

\begingroup
\parskip0ex
\tableofcontents
\endgroup

%%%%%%%%%%%%%%%%%%%%%%%%%%%%%%%%%%%%%%%%%%%%%%%%%%%%%%%%%%%%%%%%%%%%%%%%%%%%%%%%
%%%%%%%%%%%%%%%%%%%%%%%%%%%%%%%%%%%%%%%%%%%%%%%%%%%%%%%%%%%%%%%%%%%%%%%%%%%%%%%%
\section{Introduction}

\LaTeX{} provides a mechanism to structure a large document (such as a book)
into a main file and several child files (containing the chapters)
using the |\include| command.
This mechanism is beneficial for documents
which span hundreds of pages in order to
make the source file(s) more manageable.
Moreover, compilation can be restricted to
selected child files by means of the |\includeonly| command.
The latter feature can be used to reduce the compilation time while editing
(this was significantly more useful in the earlier days of \LaTeX{})
or to generate a smaller document which is easier to navigate.
Another application of |\includeonly| is to generate
documents consisting of selected parts of the complete document.

However, there are a few drawbacks of the plain |\include| mechanism:
\begin{itemize}
\item
The child files cannot be compiled on their own,
they can only be compiled via the main file.
A naive editing environment
(such as a text editor with an option
to have the current file processed by \LaTeX)
may require one to switch to the main file before compiling;
attempting to compile the child file produces errors.
\item
The main file must be modified (each time)
to adjust the |\includeonly| command
to the present needs. This easily leaves the main file in a messy state.
\item
The generated document will always carry the filename
of the main document. This is inconvenient if
several child files are to be compiled and
to be kept for distribution.
\end{itemize}

The present package provides a simple interface
to make child files individually compilable by \LaTeX{}.
Compiling a child file then has the same effect as compiling
the main file with an |\includeonly| command
to select the appropriate child.
Moreover the generated document will carry the name of the child
rather than the main file.
This resolves all three above issues.

This feature is meant to make the editing of books,
thesis documents and lecture notes somewhat more convenient.
However, the package can also be used efficiently for
composing a series of documents (such as exercise sheets)
which are typically distributed individually.
It then assists the author in generating the individual documents
(potentially in different versions)
as well as a document containing the collected series.
Another application is in developing style files
or other kinds of included material
where compilation of the style file could redirect
to a sample or test file.

%%%%%%%%%%%%%%%%%%%%%%%%%%%%%%%%%%%%%%%%%%%%%%%%%%%%%%%%%%%%%%%%%%%%%%%%%%%%%%%%
%%%%%%%%%%%%%%%%%%%%%%%%%%%%%%%%%%%%%%%%%%%%%%%%%%%%%%%%%%%%%%%%%%%%%%%%%%%%%%%%
\section{Usage}

First of all, the package \textsf{childdoc} is \emph{not} a standard
\LaTeXe{} |.sty| style file! Therefore it needs to be invoked in
a non-standard way.

%%%%%%%%%%%%%%%%%%%%%%%%%%%%%%%%%%%%%%%%%%%%%%%%%%%%%%%%%%%%%%%%%%%%%%%%%%%%%%%%
\subsection{Included Files}
\label{sec:include}

%%%%%%%%%%%%%%%%%%%%%%%%%%%%%%%%%%%%%%%%
\DescribeMacro{\childdocmain}
To use the package, add the commands
\begin{center}
\begin{tabular}{l}
|\input{childdoc.def}|\\
|\childdocmain{}|\\
\end{tabular}
\end{center}
at the very top of the main \LaTeX{} file,
in particular \emph{before} the |\documentclass| statement!
The argument of |\childdocmain| should be left empty
(but it must be present).

%%%%%%%%%%%%%%%%%%%%%%%%%%%%%%%%%%%%%%%%
\DescribeMacro{\childdocof}
Furthermore, add the commands
\begin{center}
\begin{tabular}{l}
|\input{childdoc.def}|\\
|\childdocof{|\textit{main}|}|\\
\end{tabular}
\end{center}
at the top of every child file \textit{child}
which is included by |\include{|\textit{child}|}|
from within the main file
(or at least for those files to be compiled individually).
The argument \textit{main} must be the filename of the main file.

There are a couple of
considerations in setting up the main and child documents:

%%%%%%%%%%%%%%%%%%%%%%%%%%%%%%%%%%%%%%%%
\paragraph{Restrictions.}

Please note the following restrictions:
\begin{itemize}
\item
|\childdocmain| must be called with one argument \textit{main}
to ensure compatibility with earlier version of the package.
It must either be empty (|\childdocmain{}|)
or precisely match the filename of the main file in which it is specified.
See \secref{sec:detection} for further information.
\item
The filename \textit{main} must be specified without the |.tex| extension.
\item
The filename \textit{main} is case sensitive
(even in case-insensitive file systems)
due to internal string comparison.
\item
The argument \textit{main} should be fully expanded, it cannot be a macro.
\item
Subdirectories and special characters should be avoided in filenames.
\item
The command |\childdocmain{|\textit{main}|}| must be followed by a whitespace.
It should not be followed immediately by another command
or by a comment mark `|%|'.
This is because the \TeX{} parser reads the token immediately following
the argument of |\childdocmain| and puts it
at the beginning of every child section;
however, a white\-space is ignored.
\end{itemize}

%%%%%%%%%%%%%%%%%%%%%%%%%%%%%%%%%%%%%%%%
\paragraph{Content of Main File.}

It is advisable to place all content in the child files included by |\include|.
Any output contained in the main file will appear in all child documents
unless suppressed manually;
it cannot be suppressed automatically by the |\includeonly| directive
and thus should normally be avoided.
A method to include some content in the main file
by means of conditional processing is described in \secref{sec:conditional}.

%%%%%%%%%%%%%%%%%%%%%%%%%%%%%%%%%%%%%%%%
\paragraph{Page Numbering.}

When only a part of the document is compiled,
the appropriate numbering of pages
(as well as other status parameters)
is determined from the |.aux| files.
The latter contain information from previous passes.
However this information needs to propagate through
all intermediate child documents.
Therefore the page numbering in child documents may well
be inconsistent until the complete document is compiled at least once.

A useful (if unconventional) way to always ensure a consistent
page numbering is to restart the numbering in each child document
and denote the pages by `\textit{child}|.|\textit{page}'
where \textit{child} represents the chapter/section number of the child file.
This can be achieved by the command
|\numberwithin{page}{|\textit{child}|}|
of the \textsf{amsmath} package
where \textit{child} can be |chapter| or |section|
depending on the chosen structuring.
Alternatively, one can modify the macro |\thepage| appropriately
and reset the counter |page| at the start of each child file.

%%%%%%%%%%%%%%%%%%%%%%%%%%%%%%%%%%%%%%%%%%%%%%%%%%%%%%%%%%%%%%%%%%%%%%%%%%%%%%%%
\subsection{Conditional Processing}
\label{sec:conditional}

The package provides a mechanism to compile different versions
of a document. To customise the versions further some conditional processing
can come in handy to distinguish which version is being compiled.
The package provides two macros to describe the compilation context:

%%%%%%%%%%%%%%%%%%%%%%%%%%%%%%%%%%%%%%%%
\DescribeMacro{\ifchilddoc}
The conditional |\ifchilddoc| distinguishes between the compilation of
child documents and the main document:
%
\begin{center}
|\ifchilddoc |\textit{child-code}| |[|\||else |\textit{main-code}]| \||fi|
\end{center}

%%%%%%%%%%%%%%%%%%%%%%%%%%%%%%%%%%%%%%%%
\DescribeMacro{\childdocname}
\DescribeMacro{\childdocjob}
The macro |\childdocname| contains the filename (without extension)
of the main or child file being processed.
Note that |\childdocjob| will always contain the name of the main file.

%%%%%%%%%%%%%%%%%%%%%%%%%%%%%%%%%%%%%%%%
\paragraph{Title Page.}

Conditional processing can be used to include a title or banner page
in the main document when proper precautions are taken.
Importantly, the code in the main file should ensure that the page counter
(as well as other status parameters which are stored in the |.aux| files)
takes the same value after the conditional processing.
Otherwise the page numbers may take divergent values
depending on which part is compiled.

For example, a title page could be declared by:
%
\begin{center}
\begin{tabular}{l}
|\ifchilddoc\||else|\\
|\addtocounter{page}{-1}|\\
\textit{code for title page}\\
|\newpage|\\
|\||fi|
\end{tabular}
\end{center}
%
A banner page for the child documents can be generated by:
%
\begin{center}
\begin{tabular}{l}
|\ifchilddoc|\\
|\addtocounter{page}{-1}|\\
\textit{code for banner page}\\
|\newpage|\\
|\||fi|
\end{tabular}
\end{center}
%
Here one could write a message such as:
\begin{center}
|This is the part \childdocname{} of \childdocjob{}.|
\end{center}

%%%%%%%%%%%%%%%%%%%%%%%%%%%%%%%%%%%%%%%%%%%%%%%%%%%%%%%%%%%%%%%%%%%%%%%%%%%%%%%%
\subsection{Flags}
\label{sec:flags}

The package makes it easy to generate different versions
of the main or child documents.
To this end compilation flags can be defined
and assigned different default values.
They will be particularly useful in conjunction
with the forwarding mechanism described in \secref{sec:forward}.

For example, it may be useful to have a flag |\version|
which can be set to |draft| or |final|.
The document source will contain some conditional code
depending on the value of |\version|.
Suppose further, the flag should default to |final| for the main file
and to |draft| for child files
which is a natural assignment for editing the document.
This is achieved by placing the following code
in the preamble of the main document
(below the |\childdocmain| directive):
%
\begin{center}
\begin{tabular}{l}
|\ifchilddoc|\\
|\providecommand{\version}{draft}|\\
|\||else|\\
|\providecommand{\version}{final}|\\
|\||fi|
\end{tabular}
\end{center}
%
The definition by |\providecommand| makes sure
that previous definitions are not overwritten.
Further statements |\providecommand{\version}{...}|
can thus be added before the above code to override it.

For the main file, one might add a line
(between |\childdocmain| and the above block)
%
\begin{center}
|%\ifchilddoc\||else\providecommand{\version}{draft}\||fi|
\end{center}
%
which can be uncommented to produce a draft version.
Likewise one can add a line to the very top of a child file
(above the |\childdocof{|\textit{main}|}| directive)
%
\begin{center}
|%\providecommand{\version}{final}|
\end{center}
%
which can be uncommented to produce the final version of this child document.

%%%%%%%%%%%%%%%%%%%%%%%%%%%%%%%%%%%%%%%%%%%%%%%%%%%%%%%%%%%%%%%%%%%%%%%%%%%%%%%%
\subsection{Forwarding}
\label{sec:forward}

Different versions of the main or child documents
using compilation flags as described in \secref{sec:flags}
can be (permanently) stored in different files
for convenient compilation, viewing and distribution.
To this end, the package defines a command
to pass on compilation to a different file:

%%%%%%%%%%%%%%%%%%%%%%%%%%%%%%%%%%%%%%%%
\DescribeMacro{\childdocforward}
The command |\childdocforward| redirects processing to
another source file:
%
\begin{center}
\begin{tabular}{l}
|\input{childdoc.def}|\\
|\childdocforward[|\textit{main}|]{|\textit{dest}|}|\\
\end{tabular}
\end{center}
%
The argument \textit{dest} is the destination file
(without extension).
It should be the main file or one of the child files.
Note that further \textsf{childdoc} directives
such as |\childdocof| and |\childdocforward|
in the indicated file will be processed in this form.
The optional argument \textit{main}
passes on directly to the main file \textit{main}
while pretending to compile the child \textit{dest}.
This form behaves as if \textit{dest}
issues |\childdocof{|\textit{main}|}| right away,
and no further \textsf{childdoc} directives will be processed.

%%%%%%%%%%%%%%%%%%%%%%%%%%%%%%%%%%%%%%%%
\DescribeMacro{\...prefix}
In the alternative form |\childdocforwardprefix|,
%
\begin{center}
\begin{tabular}{l}
|\input{childdoc.def}|\\
|\childdocforwardprefix[|\textit{main}|]{|\textit{prefix}|}{|\textit{dest}|}|
\end{tabular}
\end{center}
%
the destination file is determined by a pattern
depending on the current file:
To make this work, the current file must be called
`{\textit{prefix}\hspace{0.2em}\textit{suffix}}'
with \textit{prefix} matching precisely the argument.
Processing is then passed on to the file
`{\textit{dest}\hspace{0.2em}\textit{suffix}}'.
Surely, the same effect is achieved by
directly specifying the
argument `{\textit{dest}\hspace{0.2em}\textit{suffix}}'
in the first form.
However, that requires to set up a different file
for each child. With the alternative form of the command
all these files can have exactly the same content
which simplifies setting them up and maintaining them.

For example, the following file |draft.tex|
with a compilation flag |\version| as described in \secref{sec:flags}
compiles the main document as a draft:
%
\begin{center}
\begin{tabular}{l}
|\def\version{draft}|\\
|\input{childdoc.def}|\\
|\childdocforward{|\textit{main}|}|
\end{tabular}
\end{center}
%
Likewise, the following files |final|\textit{nn}|.tex|
compile the final version of the child document
|child|\textit{nn}|.tex|:
%
\begin{center}
\begin{tabular}{l}
|\def\version{final}|\\
|\input{childdoc.def}|\\
|\childdocforwardprefix{final}{child}|
\end{tabular}
\end{center}
%

Note that when several versions of a main file and/or of each child file
are to be generated, it may be convenient to set up a |Makefile| or
shell script to automatise the process.

%%%%%%%%%%%%%%%%%%%%%%%%%%%%%%%%%%%%%%%%%%%%%%%%%%%%%%%%%%%%%%%%%%%%%%%%%%%%%%%%
\subsection{Command Line Processing}
\label{sec:commandline}

The effect of redirection files can also be achieved by invoking
the \LaTeX{} compiler with a more elaborate command line.
Most conveniently this should be done as part
of a shell script or a |Makefile|.

When using \textsf{childdoc} in the main file, the following
command lines effectively perform a redirection
(note that depending on the shell being used,
backslashes may have to be doubled: `|\|' $\to$ `|\\|'):
%
\begin{center}
|... -jobname "|\textit{target}|" |\\|"|[\textit{flags}]%
|\input{childdoc.def}\childdocforward[|\textit{main}|]{|\textit{dest}|}"|
\end{center}
%
Here \textit{target} is the name of the output file,
\textit{main} is the name of the main file
and \textit{dest} is the name of the main or child file to be processed
(all filenames without extensions).
The optional argument \textit{main} can be omitted
if \textit{main} matches \textit{dest}.
Optionally, compilation \textit{flags} can be defined via |\def| commands.
This command line makes the \TeX{} engine believe
it is compiling the file \textit{target}
whose content is specified as the latter parameter.
The provided code then forwards the processing to
\textit{main} or \textit{dest} as described in \secref{sec:forward}.

%%%%%%%%%%%%%%%%%%%%%%%%%%%%%%%%%%%%%%%%%%%%%%%%%%%%%%%%%%%%%%%%%%%%%%%%%%%%%%%%
\subsection{Include by Input}
\label{sec:input}

Including child documents by |\include| has some restrictions by design.
Most notably, the content of a child document always occupies
its own set of pages; pages cannot be shared between child documents.
Usually, this behaviour makes perfect sense
because each child document contain an essential part of the document.
However, in some situations it may be desirable to compose
a document from a collection of parts
without having mandatory page breaks between then.
For this case, the package
provides a mechanism to include parts
by |\input| which can also be processed individually.
However, by construction this mechanism
requires manual handling of the content to be output.

%%%%%%%%%%%%%%%%%%%%%%%%%%%%%%%%%%%%%%%%
\DescribeMacro{\ifchilddocmanual}
The main file should be prepared as usual, see \secref{sec:include}.
However, the document body must make a distinction
between processing of an individual part and of the main document, e.g.:
%
\begin{center}
\begin{tabular}{l}
|\ifchilddocmanual|\\
|\input{\childdocname}|\\
|\||else|\\
\textit{document body with }|\input{|\textit{part}|}|\\
|\||fi|
\end{tabular}
\end{center}
%
The conditional |\ifchilddocmanual| is true whenever
a part to be included by |\input| is being compiled,
and the name of the part is stored in |\childdocname|.

%%%%%%%%%%%%%%%%%%%%%%%%%%%%%%%%%%%%%%%%
\DescribeMacro{\childdocby}
Each part to be included by |\input| should start with:
%
\begin{center}
\begin{tabular}{l}
|\input{childdoc.def}|\\
|\childdocby{|\textit{main}|}|\\
\end{tabular}
\end{center}
%
The directive |\childdocby| is similar to |\childdocof|
described in \secref{sec:include},
but the subsequent selection of content must be done manually.
To that end, both |\ifchilddoc| and |\ifchilddocmanual|
will be true upon processing of a part,
and the name of the part is stored in |\childdocname|.
Note that |\jobname| will be set to the filename of the current part
so that each part receives an individual |.aux| file
that does not interfere with the |.aux| file(s) of the main document.
This behaviour can be altered by the alternative form
|\childdocby[*]{|\textit{main}|}| (with a non-empty optional argument)
which uses the |.aux| file of the main document
by setting |\jobname| to \textit{main}.

%%%%%%%%%%%%%%%%%%%%%%%%%%%%%%%%%%%%%%%%%%%%%%%%%%%%%%%%%%%%%%%%%%%%%%%%%%%%%%%%
\subsection{Driver Development}
\label{sec:driver}

The \textsf{childdoc} mechanism can also be use for the development
of definition files such as \LaTeX{} styles or classes.
This case differs from the above setup with multiple parts
included by |\include| in that no |\includeonly| should be invoked.
This can be achieved by starting the include file
(before |\ProvidesPackage|) with:
%
\begin{center}
\begin{tabular}{l}
|\input{childdoc.def}|\\
|\childdocforward{|\textit{main}|}|\\
\end{tabular}
\end{center}
%
or alternatively with:
%
\begin{center}
\begin{tabular}{l}
|\input{childdoc.def}|\\
|\childdocby{|\textit{main}|}|\\
\end{tabular}
\end{center}
%
Both forms have slightly different effects as described above.
The main file is prepared as usual, see \secref{sec:include}.

%%%%%%%%%%%%%%%%%%%%%%%%%%%%%%%%%%%%%%%%%%%%%%%%%%%%%%%%%%%%%%%%%%%%%%%%%%%%%%%%
\subsection{Legacy Detection}
\label{sec:detection}

The directive |\childdocmain| in the main file can detect
whether the complete document or merely a child is to be compiled
even without using the directive |\childdocof|.
This method is deprecated because it is less robust
and there is no compelling reason to use it;
it is merely provided for backward compatibility
and it may be removed in future versions.

If the detection mechanism is to be used,
it is mandatory to correctly specify
the filename of the main file as the argument of |\childdocmain|:
%
\begin{center}
\begin{tabular}{l}
|\input{childdoc.def}|\\
|\childdocmain{|\textit{main}|}|\\
\end{tabular}
\end{center}
%
If |\jobname| does not match the argument \textit{main} of |\childdocmain|,
it is assumed that |\jobname| points to the child file to be compiled.
When using |\childdocmain| with the main file specified as argument,
it suffices to start a child file
with just |\input{|\textit{main}|}|
without loading of the package and using |\childdocof|.
If instead all processing is done
with the appropriate \textsf{childdoc} directives,
the argument of \textit{main} of |\childdocmain| can be empty.

An alternative version of the command line processing described
in \secref{sec:commandline} using the detection mechanism reads:
%
\begin{center}
|... -jobname "|\textit{target}|" "|[\textit{flags}]%
[|\def\jobname{|\textit{dest}|}|]|\input{|\textit{main}|}"|
\end{center}

%%%%%%%%%%%%%%%%%%%%%%%%%%%%%%%%%%%%%%%%%%%%%%%%%%%%%%%%%%%%%%%%%%%%%%%%%%%%%%%%
\subsection{Manual Code}
\label{sec:manual}

In case one cannot be certain whether the definitions file |childdoc.def|
is installed on the target \TeX{} distribution
and one prefers not to ship it,
it is conceivable to paste a few relevant commands into the sources.

To that end, drop all statements |\input{childdoc.def}|
and perform the replacements as outlined below.
Instead of |\childdocmain{|\textit{main}|}| add the following code
to the top of the main file:
%
\begin{center}
\begin{tabular}{l}
|\||ifdefined\childdocname\endinput\||fi\newif\ifchilddoc|\\
|\edef\childdocname{\scantokens\expandafter{\jobname\noexpand}}|\\
|\def\childdocmain{|\textit{main}|}\||ifx\childdocmain\childdocname\||else|\\
|\childdoctrue\includeonly{\childdocname}\let\jobname\childdocmain\||fi|\\
\end{tabular}
\end{center}
%
Instead of |\childdocof{|\textit{main}|}| just include the main file
at the top of each child file:
%
\begin{center}
|\input{|\textit{main}|}|
\end{center}
%
A simple redirection |\childdocforward{|\textit{dest}|}| is achieved by:
%
\begin{center}
|\def\jobname{|\textit{dest}|}\input{\jobname}|
\end{center}
%
The redirection with prefix
|\childdocforwardprefix[|\textit{prefix}|]{|\textit{dest}|}|
is accomplished by:
%
\begin{center}
\begin{tabular}{l}
|{\edef\jobname{\scantokens\expandafter{\jobname\noexpand}}|\\
|\def\redirectjob |\textit{prefix}|#1~~~{\gdef\jobname{|\textit{dest}|#1}}|\\
|\expandafter\redirectjob\jobname~~~}\input{\jobname}|
\end{tabular}
\end{center}

In an alternative approach,
child documents can be compiled by a specific command line
without additional code or specific definitions:
%
\begin{center}
|... -jobname "|\textit{target}|" "|[\textit{flags}]%
|\includeonly{|\textit{dest}|}\input{|\textit{main}|}"|
\end{center}
%

%%%%%%%%%%%%%%%%%%%%%%%%%%%%%%%%%%%%%%%%%%%%%%%%%%%%%%%%%%%%%%%%%%%%%%%%%%%%%%%%
%%%%%%%%%%%%%%%%%%%%%%%%%%%%%%%%%%%%%%%%%%%%%%%%%%%%%%%%%%%%%%%%%%%%%%%%%%%%%%%%
\section{Information}

%%%%%%%%%%%%%%%%%%%%%%%%%%%%%%%%%%%%%%%%%%%%%%%%%%%%%%%%%%%%%%%%%%%%%%%%%%%%%%%%
\subsection{Copyright}

Copyright \copyright{} 2017--2018 Niklas Beisert

This work may be distributed and/or modified under the
conditions of the \LaTeX{} Project Public License, either version 1.3
of this license or (at your option) any later version.
The latest version of this license is in
  \url{http://www.latex-project.org/lppl.txt}
and version 1.3 or later is part of all distributions of \LaTeX{}
version 2005/12/01 or later.

This work has the LPPL maintenance status `maintained'.

The Current Maintainer of this work is Niklas Beisert.

This work consists of the files |README.txt|, |childdoc.ins| and |childdoc.dtx|
as well as the derived files |childdoc.def|, |cdocsamp.tex|
with |cdocsch1.tex|, |cdocsch2.tex|, |cdocspt3.tex|, |cdocspt4.tex|,
|cdocsdrf.tex|, |cdocsfn1.tex|, |cdocsfn2.tex|
as well as |childdoc.pdf|.

%%%%%%%%%%%%%%%%%%%%%%%%%%%%%%%%%%%%%%%%%%%%%%%%%%%%%%%%%%%%%%%%%%%%%%%%%%%%%%%%
\subsection{Files and Installation}

The package consists of the files:
%
\begin{center}
\begin{tabular}{ll}
    |README.txt|   & readme file \\
    |childdoc.ins| & installation file \\
    |childdoc.dtx| & source file \\
    |childdoc.def| & definition file \\
    |cdocsamp.tex| & sample main file \\
    |cdocsch1.tex| & sample include file \\
    |cdocsch2.tex| & sample include file \\
    |cdocspt3.tex| & sample part file \\
    |cdocspt4.tex| & sample part file \\
    |cdocsdrf.tex| & sample redirection file \\
    |cdocsfn1.tex| & sample redirection file \\
    |cdocsfn2.tex| & sample redirection file \\
    |childdoc.pdf| & manual
\end{tabular}
\end{center}
%
The distribution consists of the files
|README.txt|, |childdoc.ins| and |childdoc.dtx|.
%
\begin{itemize}
\item
Run (pdf)\LaTeX{} on |childdoc.dtx|
to compile the manual |childdoc.pdf| (this file).
\item
Run \LaTeX{} on |childdoc.ins| to create the definitions file |childdoc.def|
and the sample |cdocsamp.tex| with include files
|cdocsch1.tex|, |cdocsch2.tex|, |cdocspt3.tex|, |cdocspt4.tex|,
|cdocsdrf.tex|, |cdocsfn1.tex|, |cdocsfn2.tex|.
Then copy the file |childdoc.def| to an appropriate directory of your \LaTeX{}
distribution, e.g.\ \textit{texmf-root}|/tex/latex/childdoc|.
\end{itemize}

%%%%%%%%%%%%%%%%%%%%%%%%%%%%%%%%%%%%%%%%%%%%%%%%%%%%%%%%%%%%%%%%%%%%%%%%%%%%%%%%
\subsection{Related CTAN Packages}

There are several other packages which offer a similar functionality:
%
\begin{itemize}
\item
The packages
\href{http://ctan.org/pkg/docmute}{\textsf{docmute}},
\href{http://ctan.org/pkg/includex}{\textsf{includex}} and
\href{http://ctan.org/pkg/standalone}{\textsf{standalone}}
provide commands to include only the document body of
a child file thus allowing both files to be compiled individually.
\item
The packages \href{http://ctan.org/pkg/subdocs}{\textsf{subdocs}}
and \href{http://ctan.org/pkg/subfiles}{\textsf{subfiles}}
provide structures in which the main and child documents can be
encapsulated and allowing them to be compiled individually.
The inclusion mechanism is different from the conventional |\include|.
\item
The package \href{http://ctan.org/pkg/combine}{\textsf{combine}}
is an elaborate solution to combine several documents into one.
\end{itemize}
%
See also the CTAN topic \href{http://ctan.org/topic/subdocs}{\textsf{subdocs}}
for further related packages.
The present package differs from the above solutions in that
a document structure constructed with the conventional |\include| mechanism
just needs two extra commands at the top of every file
such that all constituent files can be compiled individually.

%%%%%%%%%%%%%%%%%%%%%%%%%%%%%%%%%%%%%%%%%%%%%%%%%%%%%%%%%%%%%%%%%%%%%%%%%%%%%%%%
%\subsection{Feature Suggestions}
%
%The following is a list of features which may be useful for future
%versions of this package:
%%
%\begin{itemize}
%\item
%\ldots
%\end{itemize}

%%%%%%%%%%%%%%%%%%%%%%%%%%%%%%%%%%%%%%%%%%%%%%%%%%%%%%%%%%%%%%%%%%%%%%%%%%%%%%%%
\subsection{Revision History}

%%%%%%%%%%%%%%%%%%%%%%%%%%%%%%%%%%%%%%%%
\paragraph{v2.0:} 2018/12/30

\begin{itemize}
\item
immediate forward processing
\item
added |\childdocby| mechanism
\item
manual restructured
\end{itemize}

%%%%%%%%%%%%%%%%%%%%%%%%%%%%%%%%%%%%%%%%
\paragraph{v1.6:} 2018/01/17

\begin{itemize}
\item
application for development of include files
\item
corrections to manual
\end{itemize}

%%%%%%%%%%%%%%%%%%%%%%%%%%%%%%%%%%%%%%%%
\paragraph{v1.5:} 2017/05/21

\begin{itemize}
\item
more complete structuring introduced
\item
|\childdocof| introduced
\item
|\childdoc| renamed to |\childdocmain|
\item
|\childredirect| renamed to |\childdocforward| and |\childdocforwardprefix|
and functionality expanded
\end{itemize}

%%%%%%%%%%%%%%%%%%%%%%%%%%%%%%%%%%%%%%%%
\paragraph{v1.0:} 2017/04/27

\begin{itemize}
\item
manual and install package
\item
first version published on CTAN
\end{itemize}

%%%%%%%%%%%%%%%%%%%%%%%%%%%%%%%%%%%%%%%%
\paragraph{v0.6:} 2017/04/26

\begin{itemize}
\item
redirection mechanism added
\end{itemize}

%%%%%%%%%%%%%%%%%%%%%%%%%%%%%%%%%%%%%%%%
\paragraph{v0.5:} 2017/04/26

\begin{itemize}
\item
functionality in definition file
\end{itemize}


%%%%%%%%%%%%%%%%%%%%%%%%%%%%%%%%%%%%%%%%%%%%%%%%%%%%%%%%%%%%%%%%%%%%%%%%%%%%%%%%
%%%%%%%%%%%%%%%%%%%%%%%%%%%%%%%%%%%%%%%%%%%%%%%%%%%%%%%%%%%%%%%%%%%%%%%%%%%%%%%%
%%%%%%%%%%%%%%%%%%%%%%%%%%%%%%%%%%%%%%%%%%%%%%%%%%%%%%%%%%%%%%%%%%%%%%%%%%%%%%%%
\appendix

\settowidth\MacroIndent{\rmfamily\scriptsize 000\ }

 \DocInput{childdoc.dtx}

\end{document}
%</driver>
% \fi
%
% %%%%%%%%%%%%%%%%%%%%%%%%%%%%%%%%%%%%%%%%%%%%%%%%%%%%%%%%%%%%%%%%%%%%%%%%%%%%%%
% %%%%%%%%%%%%%%%%%%%%%%%%%%%%%%%%%%%%%%%%%%%%%%%%%%%%%%%%%%%%%%%%%%%%%%%%%%%%%%
% \section{Sample}
%\iffalse
%<*samplemain>
%\fi
%
% The following presents a sample document
% with two chapters, two parts, a title page,
% a compile flag as well as three forwarding files to set the flag.
% It consists of eight |.tex| files:
% \begin{center}
% \begin{tabular}{ll}
% |cdocsamp.tex|&main file\\
% |cdocsch1.tex|&include file for chapter 1\\
% |cdocsch2.tex|&include file for chapter 2\\
% |cdocspt3.tex|&include file for part 3\\
% |cdocspt4.tex|&include file for part 4\\
% |cdocsdrf.tex|&forwarding file for main file in draft mode\\
% |cdocsfi1.tex|&forwarding file for final version of chapter 1\\
% |cdocsfi2.tex|&forwarding file for final version of chapter 2\\
% \end{tabular}
% \end{center}
% Each of the eight files can be compiled directly by the \LaTeX{} compiler.
%
% %%%%%%%%%%%%%%%%%%%%%%%%%%%%%%%%%%%%%%
% \paragraph{Main File.}
%
% The main file is called |cdocsamp.tex|.
%
% Load the \textsf{childdoc} definitions and
% declare the filename for the main document:
%    \begin{macrocode}
\input{childdoc.def}
\childdocmain{}
%    \end{macrocode}

% Optional override for |\version| flag:
%    \begin{macrocode}
%%\ifchilddoc\else\providecommand{\version}{draft}\fi
%    \end{macrocode}

% Define the default values for the |\version| flag
% (|final| for the main file and |draft| for childs):
%    \begin{macrocode}
\ifchilddoc
\providecommand{\version}{draft}
\else
\providecommand{\version}{final}
\fi
%    \end{macrocode}

% Load the standard document class:
%    \begin{macrocode}
\documentclass[12pt]{article}
%    \end{macrocode}

% Start the document body:
%    \begin{macrocode}
\begin{document}
%    \end{macrocode}

% Declare a title page.
% Print title, part of document being processed and version flag:
%    \begin{macrocode}
\addtocounter{page}{-1}
\begin{center}
{\LARGE\bfseries{}childdoc example\par}
\vspace{1cm}
\ifchilddoc
\ifchilddocmanual part\else chapter\fi:
`\childdocname' of `\childdocjob'\par
\else
main document: `\childdocjob'\par
\fi
version: \version\par
\end{center}
\newpage
%    \end{macrocode}

% Manually include selected file,
% otherwise process as usual:
%    \begin{macrocode}
\ifchilddocmanual
\section*{part `\childdocname'}
\input{\childdocname}
\else
%    \end{macrocode}

% Include the two chapters:
%    \begin{macrocode}
\include{cdocsch1}
\include{cdocsch2}
%    \end{macrocode}

% Include the two parts unless only chapters should be displayed:
%    \begin{macrocode}
\ifchilddoc\else
\section{part three}
\input{cdocspt3}
\section{part four}
\input{cdocspt4}
\fi
%    \end{macrocode}

% Process as usual until here:
%    \begin{macrocode}
\fi
%    \end{macrocode}

% End of document body:
%    \begin{macrocode}
\end{document}
%    \end{macrocode}
%\iffalse
%</samplemain>
%\fi
%
% %%%%%%%%%%%%%%%%%%%%%%%%%%%%%%%%%%%%%%
% \paragraph{Chapter Include Files.}
%
% The include files are called |cdocsch1.tex| and |cdocsch2.tex|.
%
%\iffalse
%<*samplechap1|samplechap2>
%\fi

% Optional override for |\version| flag:
%    \begin{macrocode}
%%\providecommand{\version}{final}
%    \end{macrocode}

% Include the main document:
%    \begin{macrocode}
\input{childdoc.def}
\childdocof{cdocsamp}
%    \end{macrocode}

%\iffalse
%</samplechap1|samplechap2>
%\fi
%
%\iffalse
%<*samplechap1>
%\fi
% Some text for chapter 1:
%    \begin{macrocode}
\section{one}
some text in chapter one
%    \end{macrocode}

%\iffalse
%</samplechap1>
%\fi
% Some text for chapter 2:
%\iffalse
%<*samplechap2>
%\fi
%    \begin{macrocode}
\section{two}
more text in chapter two
%    \end{macrocode}

%\iffalse
%</samplechap2>
%\fi
%
% %%%%%%%%%%%%%%%%%%%%%%%%%%%%%%%%%%%%%%
% \paragraph{Part Include Files.}
%
% The include files are called |cdocspt3.tex| and |cdocspt4.tex|.
%
%\iffalse
%<*samplepart3|samplepart4>
%\fi

% Optional override for |\version| flag:
%    \begin{macrocode}
%%\providecommand{\version}{final}
%    \end{macrocode}

% Include the main document:
%    \begin{macrocode}
\input{childdoc.def}
\childdocby{cdocsamp}
%    \end{macrocode}

%\iffalse
%</samplepart3|samplepart4>
%\fi
%
%\iffalse
%<*samplepart3>
%\fi
% Some text for part 3:
%    \begin{macrocode}
some text in part three
%    \end{macrocode}

%\iffalse
%</samplepart3>
%\fi
% Some text for part 4:
%\iffalse
%<*samplepart4>
%\fi
%    \begin{macrocode}
more text in part four
%    \end{macrocode}

%\iffalse
%</samplepart4>
%\fi
%
% %%%%%%%%%%%%%%%%%%%%%%%%%%%%%%%%%%%%%%
% \paragraph{Forwarding for a Complete Draft.}
%
% The following forwarding file |cdocsdrf.tex|
% compiles the main document in draft mode:
%\iffalse
%<*sampledraft>
%\fi
%    \begin{macrocode}
\def\version{draft}
\input{childdoc.def}
\childdocforward{cdocsamp}
%    \end{macrocode}

%\iffalse
%</sampledraft>
%\fi
%
% %%%%%%%%%%%%%%%%%%%%%%%%%%%%%%%%%%%%%%
% \paragraph{Forwarding for Final Version of the Chapters.}
%
% The following forwarding files |cdocsfn1.tex| and |cdocsfn2.tex|
% (with identical content)
% compile the final versions of the child documents
% |cdocsch1.tex| and |cdocsch2.tex|, respectively:
%\iffalse
%<*samplefinal>
%\fi
%    \begin{macrocode}
\def\version{final}
\input{childdoc.def}
\childdocforwardprefix[cdocsamp]{cdocsfn}{cdocsch}
%    \end{macrocode}

%\iffalse
%</samplefinal>
%\fi
%
% %%%%%%%%%%%%%%%%%%%%%%%%%%%%%%%%%%%%%%
% \paragraph{Command Line Processing.}
%
% The following three command lines generate the output files
% |cdocscld|, |cdocscl1| and |cdocscl2|
% which should be identical to
% |cdocsdrf|, |cdocsch1| and |cdocsfn2|, respectively:
% \begin{center}
% \begin{tabular}{l}
% |latex -jobname cdocscld \|\\
% |  "\def\version{draft}\input{childdoc.def}\childdocforward{cdocsamp}"|\\
% |latex -jobname cdocscl1 \|\\
% |  "\input{childdoc.def}\childdocforward[cdocsamp]{cdocsch1}"|\\
% |latex -jobname cdocscl2 \|\\
% |  "\def\version{final}\input{childdoc.def}\childdocforward{cdocsch2}"|
% \end{tabular}
% \end{center}
% Note that the trailing backslash on each first line
% merely continues the input to the second line
% (for convenient cut ant paste).
% Furthermore, the command |latex| can be replaced by any
% of its alternative versions such as |pdflatex|.
%
% %%%%%%%%%%%%%%%%%%%%%%%%%%%%%%%%%%%%%%%%%%%%%%%%%%%%%%%%%%%%%%%%%%%%%%%%%%%%%%
% %%%%%%%%%%%%%%%%%%%%%%%%%%%%%%%%%%%%%%%%%%%%%%%%%%%%%%%%%%%%%%%%%%%%%%%%%%%%%%
% \section{Implementation}
%\iffalse
%<*package>
%\fi
%
% This section describes the definitions file |childdoc.def|.

% The definitions cannot be loaded using |\usepackage| or |\RequirePackage|
% which has a mechanism to prevent loading a style file more than once.
% When loading the definitions by means of |\input|
% multiple instances have to be prevented manually:
%\iffalse
%This code needs to be before the `\ProvidesFile' directive
%which is defined at the beginning of this file.
%Therefore it is also placed there and commented out here.
%</package>
%<*discard>
%\fi
%    \begin{macrocode}
\ifdefined\childdocmain\endinput\fi
%    \end{macrocode}
%\iffalse
%</discard>
%<*package>
%\fi
%
% \macro{\ifchilddoc}
% \macro{\ifchilddocmanual}
% The conditional |\ifchilddoc| tells whether a
% child (true) or main (false) document is being compiled.
% The conditional |\ifchilddocmanual| tells whether
% the |\includeonly| mechanism is used (false) or
% the selection of child files must be performed manually (true).
% The definitions initialise to false:
%    \begin{macrocode}
\newif\ifchilddoc
\newif\ifchilddocmanual
%    \end{macrocode}

% \macro{\childdocname}
% \macro{\childdocjob}
% The macro |\childdocname| stores the name of the main document
% to be compiled. The macro |\childdocjob| stores the name of
% the document on which the \LaTeX{} compiler was originally invoked.
% The content of |\jobname| cannot be compared
% to filenames specified in the source due to different catcodes.
% The following code rescans |\jobname|, stores the result
% in |\childdocname| and saves a copy in |\childdocjob|:
%    \begin{macrocode}
\edef\childdocname{\scantokens\expandafter{\jobname\noexpand}}
\let\childdocjob\childdocname
%    \end{macrocode}

% \macro{\childdocdisable}
% The macro |\childdocdisable| prevents the main file
% from being processed more than once.
% At this stage, the main document command |\childdocmain|
% is assumed to be called once again where it should do nothing.
% Any subsequent call to it should prevent
% a secondary processing of the main document
% It overwrites the forwarding commands
% |\childdocof| and |\childdocforward|
% with empty macros to prevent further inclusions of the main document:
%    \begin{macrocode}
\newcommand{\childdocdisable}
{
  \renewcommand{\childdocmain}[1]{\renewcommand{\childdocmain}[1]{\endinput}}
  \renewcommand{\childdocof}[1]{}
  \renewcommand{\childdocby}[2][]{}
  \renewcommand{\childdocforward}[2][]{}
  \renewcommand{\childdocdisable}{}
}
%    \end{macrocode}

% \macro{\childdocmain}
% The macro |\childdocmain| is to be called at the top of the main file
% with nothing or the main filename (without extension) as argument.
% First, it breaks loops.
% If the argument is not empty and does not match |\childdocname|
% (which is set by the first inclusion of |childdoc.def|),
% |\ifchilddoc| is set to true, |\includeonly| is applied to the child file
% and |\jobname| is set to the main file
% (for proper handling of |.aux| files):
%    \begin{macrocode}
\newcommand{\childdocmain}[1]
{
  \childdocdisable\childdocmain{}
  \if?#1?\else
    \begingroup
      \def\childdoctmp{#1}
      \ifx\childdoctmp\childdocname
        \def\childdoctmp{}
      \else
        \def\childdoctmp
        {
          \childdoctrue
          \includeonly{\childdocname}
          \def\childdocjob{#1}
          \def\jobname{#1}
        }
      \fi
      \expandafter
    \endgroup
    \childdoctmp
  \fi
}
%    \end{macrocode}

% \macro{\childdocof}
% The command |\childdocof| redirects
% compilation to the main file |#1|.
%    \begin{macrocode}
\newcommand{\childdocof}[1]
{
  \childdocdisable
  \childdoctrue
  \includeonly{\childdocname}
  \def\jobname{#1}
  \def\childdocjob{#1}
  \input{#1}
}
%    \end{macrocode}

% \macro{\childdocby}
% The command |\childdocby| ....
%    \begin{macrocode}
\newcommand{\childdocby}[2][]
{
  \childdocdisable
  \childdoctrue
  \childdocmanualtrue
  \if?#1?\else
    \def\jobname{#2}
  \fi
  \def\childdocjob{#2}
  \input{#2}
  \endinput
}
%    \end{macrocode}

% \macro{\childdocforward}
% The command |\childdocforward| redirects
% compilation to the main file or
% (if the optional argument is given) a child file.
% Parameters are set as if the main file
% or a child file starting with |\childdocof| was compiled.
% Then compilation is handed over to the main file:
%    \begin{macrocode}
\newcommand{\childdocforward}[2][]
{
  \begingroup
    \if?#1?
      \def\childdoctmp
      {
        \def\childdocname{#2}
        \def\childdocjob{#2}
        \def\jobname{#2}
        \input{#2}
        \endinput
      }
    \else
      \def\childdoctmp
      {
        \childdocdisable
        \def\childdocname{#2}
        \childdoctrue
        \includeonly{#2}
        \def\childdocjob{#1}
        \def\jobname{#1}
        \input{#1}
        \endinput
      }
    \fi
    \expandafter
  \endgroup
  \childdoctmp
}
%    \end{macrocode}

% \macro{\childdocforwardprefix}
% The command |\childdocforwardprefix| redirects
% compilation to the main or a child file by means of a pattern.
% The prefix |#1| in the current filename is replaced by |#2|
% and the suffix of the current filename is kept
% (it is assumed that the filename does not contain the substring `|~~~|'
% which is used as a delimiter).
% Compilation is handed over to the new file by |\childdocforward|:
%    \begin{macrocode}
\newcommand{\childdocforwardprefix}[3][]
{
  \begingroup
    \def\childdocextract #2##1~~~{\def\childdoctmp{\childdocforward[#1]{#3##1}}}
    \expandafter\childdocextract\childdocname~~~
    \expandafter
  \endgroup
  \childdoctmp
}
%    \end{macrocode}

% \macro{\childdoc}
% The deprecated macro |\childdoc| is a legacy version of |\childdocmain|:
%    \begin{macrocode}
\newcommand{\childdoc}{\childdocmain}
%    \end{macrocode}

% \macro{\childdocredirect}
% The deprecated macro |\childdocredirect| is a legacy version
% of |\childdocforward| and |\childdocforwardprefix|:
%    \begin{macrocode}
\newcommand{\childdocredirect}[2][]
{
  \begingroup
    \if?#1?
      \def\childdoctmp{\childdocforward{#2}}
    \else
      \def\childdoctmp{\childdocforwardprefix{#1}{#2}}
    \fi
    \expandafter
  \endgroup
  \childdoctmp
}
%    \end{macrocode}

%\iffalse
%</package>
%\fi
%
\endinput
|\\
|\childdocforwardprefix[|\textit{main}|]{|\textit{prefix}|}{|\textit{dest}|}|
\end{tabular}
\end{center}
%
the destination file is determined by a pattern
depending on the current file:
To make this work, the current file must be called
`{\textit{prefix}\hspace{0.2em}\textit{suffix}}'
with \textit{prefix} matching precisely the argument.
Processing is then passed on to the file
`{\textit{dest}\hspace{0.2em}\textit{suffix}}'.
Surely, the same effect is achieved by
directly specifying the
argument `{\textit{dest}\hspace{0.2em}\textit{suffix}}'
in the first form.
However, that requires to set up a different file
for each child. With the alternative form of the command
all these files can have exactly the same content
which simplifies setting them up and maintaining them.

For example, the following file |draft.tex|
with a compilation flag |\version| as described in \secref{sec:flags}
compiles the main document as a draft:
%
\begin{center}
\begin{tabular}{l}
|\def\version{draft}|\\
|% \iffalse
%
% childdoc.dtx Copyright (C) 2017-2018 Niklas Beisert
%
% This work may be distributed and/or modified under the
% conditions of the LaTeX Project Public License, either version 1.3
% of this license or (at your option) any later version.
% The latest version of this license is in
%   http://www.latex-project.org/lppl.txt
% and version 1.3 or later is part of all distributions of LaTeX
% version 2005/12/01 or later.
%
% This work has the LPPL maintenance status `maintained'.
%
% The Current Maintainer of this work is Niklas Beisert.
%
% This work consists of the files childdoc.dtx and childdoc.ins
% and the derived files childdoc.def and cdocsamp.tex with
% cdocsch1.tex, cdocsch2.tex, cdocsdrf.tex, cdocsfn1.tex, cdocsfn2.tex.
%
%<package>\ifdefined\childdocmain\endinput\fi
%<package>\ProvidesFile{childdoc.def}[2018/12/30 v2.0 child document driver]
%<samplemain>\ProvidesFile{cdocsamp.tex}[2018/12/30 v2.0 sample for childdoc]
%<*driver>
%\ProvidesFile{childdoc.drv}[2018/12/30 v2.0 childdoc reference manual file]
\PassOptionsToClass{10pt,a4paper}{article}
\documentclass{ltxdoc}

\usepackage[margin=35mm]{geometry}
\usepackage{hyperref}
\usepackage{hyperxmp}
\usepackage[usenames]{color}

\hypersetup{colorlinks=true}
\hypersetup{pdfstartview=FitH}
\hypersetup{pdfpagemode=UseNone}
\hypersetup{pdfsource={}}
\hypersetup{pdflang={en-UK}}
\hypersetup{pdfcopyright={Copyright 2017-2018 Niklas Beisert.
  This work may be distributed and/or modified under the
  conditions of the LaTeX Project Public License, either version 1.3
  of this license or (at your option) any later version.}}
\hypersetup{pdflicenseurl={http://www.latex-project.org/lppl.txt}}
\hypersetup{pdfcontactaddress={ETH Zurich, ITP, HIT K,
  Wolfgang-Pauli-Strasse 27}}
\hypersetup{pdfcontactpostcode={8093}}
\hypersetup{pdfcontactcity={Zurich}}
\hypersetup{pdfcontactcountry={Switzerland}}
\hypersetup{pdfcontactemail={nbeisert@itp.phys.ethz.ch}}
\hypersetup{pdfcontacturl={http://people.phys.ethz.ch/\xmptilde nbeisert/}}

\newcommand{\secref}[1]{\hyperref[#1]{section \ref*{#1}}}

\parskip1ex
\parindent0pt
\let\olditemize\itemize
\def\itemize{\olditemize\parskip0pt}

\begin{document}

\title{The \textsf{childdoc} Package}
\hypersetup{pdftitle={The childdoc Package}}
\author{Niklas Beisert\\[2ex]
  Institut f\"ur Theoretische Physik\\
  Eidgen\"ossische Technische Hochschule Z\"urich\\
  Wolfgang-Pauli-Strasse 27, 8093 Z\"urich, Switzerland\\[1ex]
  \href{mailto:nbeisert@itp.phys.ethz.ch}
  {\texttt{nbeisert@itp.phys.ethz.ch}}}
\hypersetup{pdfauthor={Niklas Beisert}}
\hypersetup{pdfsubject={Manual for the LaTeX2e Package childdoc}}
\date{30 December 2018, \textsf{v2.0}}
\maketitle

\begin{abstract}\noindent
\textsf{childdoc} is a \LaTeXe{} package
that enables the direct compilation
of document sections included by |\include|
to individual files.
\end{abstract}

\begingroup
\parskip0ex
\tableofcontents
\endgroup

%%%%%%%%%%%%%%%%%%%%%%%%%%%%%%%%%%%%%%%%%%%%%%%%%%%%%%%%%%%%%%%%%%%%%%%%%%%%%%%%
%%%%%%%%%%%%%%%%%%%%%%%%%%%%%%%%%%%%%%%%%%%%%%%%%%%%%%%%%%%%%%%%%%%%%%%%%%%%%%%%
\section{Introduction}

\LaTeX{} provides a mechanism to structure a large document (such as a book)
into a main file and several child files (containing the chapters)
using the |\include| command.
This mechanism is beneficial for documents
which span hundreds of pages in order to
make the source file(s) more manageable.
Moreover, compilation can be restricted to
selected child files by means of the |\includeonly| command.
The latter feature can be used to reduce the compilation time while editing
(this was significantly more useful in the earlier days of \LaTeX{})
or to generate a smaller document which is easier to navigate.
Another application of |\includeonly| is to generate
documents consisting of selected parts of the complete document.

However, there are a few drawbacks of the plain |\include| mechanism:
\begin{itemize}
\item
The child files cannot be compiled on their own,
they can only be compiled via the main file.
A naive editing environment
(such as a text editor with an option
to have the current file processed by \LaTeX)
may require one to switch to the main file before compiling;
attempting to compile the child file produces errors.
\item
The main file must be modified (each time)
to adjust the |\includeonly| command
to the present needs. This easily leaves the main file in a messy state.
\item
The generated document will always carry the filename
of the main document. This is inconvenient if
several child files are to be compiled and
to be kept for distribution.
\end{itemize}

The present package provides a simple interface
to make child files individually compilable by \LaTeX{}.
Compiling a child file then has the same effect as compiling
the main file with an |\includeonly| command
to select the appropriate child.
Moreover the generated document will carry the name of the child
rather than the main file.
This resolves all three above issues.

This feature is meant to make the editing of books,
thesis documents and lecture notes somewhat more convenient.
However, the package can also be used efficiently for
composing a series of documents (such as exercise sheets)
which are typically distributed individually.
It then assists the author in generating the individual documents
(potentially in different versions)
as well as a document containing the collected series.
Another application is in developing style files
or other kinds of included material
where compilation of the style file could redirect
to a sample or test file.

%%%%%%%%%%%%%%%%%%%%%%%%%%%%%%%%%%%%%%%%%%%%%%%%%%%%%%%%%%%%%%%%%%%%%%%%%%%%%%%%
%%%%%%%%%%%%%%%%%%%%%%%%%%%%%%%%%%%%%%%%%%%%%%%%%%%%%%%%%%%%%%%%%%%%%%%%%%%%%%%%
\section{Usage}

First of all, the package \textsf{childdoc} is \emph{not} a standard
\LaTeXe{} |.sty| style file! Therefore it needs to be invoked in
a non-standard way.

%%%%%%%%%%%%%%%%%%%%%%%%%%%%%%%%%%%%%%%%%%%%%%%%%%%%%%%%%%%%%%%%%%%%%%%%%%%%%%%%
\subsection{Included Files}
\label{sec:include}

%%%%%%%%%%%%%%%%%%%%%%%%%%%%%%%%%%%%%%%%
\DescribeMacro{\childdocmain}
To use the package, add the commands
\begin{center}
\begin{tabular}{l}
|\input{childdoc.def}|\\
|\childdocmain{}|\\
\end{tabular}
\end{center}
at the very top of the main \LaTeX{} file,
in particular \emph{before} the |\documentclass| statement!
The argument of |\childdocmain| should be left empty
(but it must be present).

%%%%%%%%%%%%%%%%%%%%%%%%%%%%%%%%%%%%%%%%
\DescribeMacro{\childdocof}
Furthermore, add the commands
\begin{center}
\begin{tabular}{l}
|\input{childdoc.def}|\\
|\childdocof{|\textit{main}|}|\\
\end{tabular}
\end{center}
at the top of every child file \textit{child}
which is included by |\include{|\textit{child}|}|
from within the main file
(or at least for those files to be compiled individually).
The argument \textit{main} must be the filename of the main file.

There are a couple of
considerations in setting up the main and child documents:

%%%%%%%%%%%%%%%%%%%%%%%%%%%%%%%%%%%%%%%%
\paragraph{Restrictions.}

Please note the following restrictions:
\begin{itemize}
\item
|\childdocmain| must be called with one argument \textit{main}
to ensure compatibility with earlier version of the package.
It must either be empty (|\childdocmain{}|)
or precisely match the filename of the main file in which it is specified.
See \secref{sec:detection} for further information.
\item
The filename \textit{main} must be specified without the |.tex| extension.
\item
The filename \textit{main} is case sensitive
(even in case-insensitive file systems)
due to internal string comparison.
\item
The argument \textit{main} should be fully expanded, it cannot be a macro.
\item
Subdirectories and special characters should be avoided in filenames.
\item
The command |\childdocmain{|\textit{main}|}| must be followed by a whitespace.
It should not be followed immediately by another command
or by a comment mark `|%|'.
This is because the \TeX{} parser reads the token immediately following
the argument of |\childdocmain| and puts it
at the beginning of every child section;
however, a white\-space is ignored.
\end{itemize}

%%%%%%%%%%%%%%%%%%%%%%%%%%%%%%%%%%%%%%%%
\paragraph{Content of Main File.}

It is advisable to place all content in the child files included by |\include|.
Any output contained in the main file will appear in all child documents
unless suppressed manually;
it cannot be suppressed automatically by the |\includeonly| directive
and thus should normally be avoided.
A method to include some content in the main file
by means of conditional processing is described in \secref{sec:conditional}.

%%%%%%%%%%%%%%%%%%%%%%%%%%%%%%%%%%%%%%%%
\paragraph{Page Numbering.}

When only a part of the document is compiled,
the appropriate numbering of pages
(as well as other status parameters)
is determined from the |.aux| files.
The latter contain information from previous passes.
However this information needs to propagate through
all intermediate child documents.
Therefore the page numbering in child documents may well
be inconsistent until the complete document is compiled at least once.

A useful (if unconventional) way to always ensure a consistent
page numbering is to restart the numbering in each child document
and denote the pages by `\textit{child}|.|\textit{page}'
where \textit{child} represents the chapter/section number of the child file.
This can be achieved by the command
|\numberwithin{page}{|\textit{child}|}|
of the \textsf{amsmath} package
where \textit{child} can be |chapter| or |section|
depending on the chosen structuring.
Alternatively, one can modify the macro |\thepage| appropriately
and reset the counter |page| at the start of each child file.

%%%%%%%%%%%%%%%%%%%%%%%%%%%%%%%%%%%%%%%%%%%%%%%%%%%%%%%%%%%%%%%%%%%%%%%%%%%%%%%%
\subsection{Conditional Processing}
\label{sec:conditional}

The package provides a mechanism to compile different versions
of a document. To customise the versions further some conditional processing
can come in handy to distinguish which version is being compiled.
The package provides two macros to describe the compilation context:

%%%%%%%%%%%%%%%%%%%%%%%%%%%%%%%%%%%%%%%%
\DescribeMacro{\ifchilddoc}
The conditional |\ifchilddoc| distinguishes between the compilation of
child documents and the main document:
%
\begin{center}
|\ifchilddoc |\textit{child-code}| |[|\||else |\textit{main-code}]| \||fi|
\end{center}

%%%%%%%%%%%%%%%%%%%%%%%%%%%%%%%%%%%%%%%%
\DescribeMacro{\childdocname}
\DescribeMacro{\childdocjob}
The macro |\childdocname| contains the filename (without extension)
of the main or child file being processed.
Note that |\childdocjob| will always contain the name of the main file.

%%%%%%%%%%%%%%%%%%%%%%%%%%%%%%%%%%%%%%%%
\paragraph{Title Page.}

Conditional processing can be used to include a title or banner page
in the main document when proper precautions are taken.
Importantly, the code in the main file should ensure that the page counter
(as well as other status parameters which are stored in the |.aux| files)
takes the same value after the conditional processing.
Otherwise the page numbers may take divergent values
depending on which part is compiled.

For example, a title page could be declared by:
%
\begin{center}
\begin{tabular}{l}
|\ifchilddoc\||else|\\
|\addtocounter{page}{-1}|\\
\textit{code for title page}\\
|\newpage|\\
|\||fi|
\end{tabular}
\end{center}
%
A banner page for the child documents can be generated by:
%
\begin{center}
\begin{tabular}{l}
|\ifchilddoc|\\
|\addtocounter{page}{-1}|\\
\textit{code for banner page}\\
|\newpage|\\
|\||fi|
\end{tabular}
\end{center}
%
Here one could write a message such as:
\begin{center}
|This is the part \childdocname{} of \childdocjob{}.|
\end{center}

%%%%%%%%%%%%%%%%%%%%%%%%%%%%%%%%%%%%%%%%%%%%%%%%%%%%%%%%%%%%%%%%%%%%%%%%%%%%%%%%
\subsection{Flags}
\label{sec:flags}

The package makes it easy to generate different versions
of the main or child documents.
To this end compilation flags can be defined
and assigned different default values.
They will be particularly useful in conjunction
with the forwarding mechanism described in \secref{sec:forward}.

For example, it may be useful to have a flag |\version|
which can be set to |draft| or |final|.
The document source will contain some conditional code
depending on the value of |\version|.
Suppose further, the flag should default to |final| for the main file
and to |draft| for child files
which is a natural assignment for editing the document.
This is achieved by placing the following code
in the preamble of the main document
(below the |\childdocmain| directive):
%
\begin{center}
\begin{tabular}{l}
|\ifchilddoc|\\
|\providecommand{\version}{draft}|\\
|\||else|\\
|\providecommand{\version}{final}|\\
|\||fi|
\end{tabular}
\end{center}
%
The definition by |\providecommand| makes sure
that previous definitions are not overwritten.
Further statements |\providecommand{\version}{...}|
can thus be added before the above code to override it.

For the main file, one might add a line
(between |\childdocmain| and the above block)
%
\begin{center}
|%\ifchilddoc\||else\providecommand{\version}{draft}\||fi|
\end{center}
%
which can be uncommented to produce a draft version.
Likewise one can add a line to the very top of a child file
(above the |\childdocof{|\textit{main}|}| directive)
%
\begin{center}
|%\providecommand{\version}{final}|
\end{center}
%
which can be uncommented to produce the final version of this child document.

%%%%%%%%%%%%%%%%%%%%%%%%%%%%%%%%%%%%%%%%%%%%%%%%%%%%%%%%%%%%%%%%%%%%%%%%%%%%%%%%
\subsection{Forwarding}
\label{sec:forward}

Different versions of the main or child documents
using compilation flags as described in \secref{sec:flags}
can be (permanently) stored in different files
for convenient compilation, viewing and distribution.
To this end, the package defines a command
to pass on compilation to a different file:

%%%%%%%%%%%%%%%%%%%%%%%%%%%%%%%%%%%%%%%%
\DescribeMacro{\childdocforward}
The command |\childdocforward| redirects processing to
another source file:
%
\begin{center}
\begin{tabular}{l}
|\input{childdoc.def}|\\
|\childdocforward[|\textit{main}|]{|\textit{dest}|}|\\
\end{tabular}
\end{center}
%
The argument \textit{dest} is the destination file
(without extension).
It should be the main file or one of the child files.
Note that further \textsf{childdoc} directives
such as |\childdocof| and |\childdocforward|
in the indicated file will be processed in this form.
The optional argument \textit{main}
passes on directly to the main file \textit{main}
while pretending to compile the child \textit{dest}.
This form behaves as if \textit{dest}
issues |\childdocof{|\textit{main}|}| right away,
and no further \textsf{childdoc} directives will be processed.

%%%%%%%%%%%%%%%%%%%%%%%%%%%%%%%%%%%%%%%%
\DescribeMacro{\...prefix}
In the alternative form |\childdocforwardprefix|,
%
\begin{center}
\begin{tabular}{l}
|\input{childdoc.def}|\\
|\childdocforwardprefix[|\textit{main}|]{|\textit{prefix}|}{|\textit{dest}|}|
\end{tabular}
\end{center}
%
the destination file is determined by a pattern
depending on the current file:
To make this work, the current file must be called
`{\textit{prefix}\hspace{0.2em}\textit{suffix}}'
with \textit{prefix} matching precisely the argument.
Processing is then passed on to the file
`{\textit{dest}\hspace{0.2em}\textit{suffix}}'.
Surely, the same effect is achieved by
directly specifying the
argument `{\textit{dest}\hspace{0.2em}\textit{suffix}}'
in the first form.
However, that requires to set up a different file
for each child. With the alternative form of the command
all these files can have exactly the same content
which simplifies setting them up and maintaining them.

For example, the following file |draft.tex|
with a compilation flag |\version| as described in \secref{sec:flags}
compiles the main document as a draft:
%
\begin{center}
\begin{tabular}{l}
|\def\version{draft}|\\
|\input{childdoc.def}|\\
|\childdocforward{|\textit{main}|}|
\end{tabular}
\end{center}
%
Likewise, the following files |final|\textit{nn}|.tex|
compile the final version of the child document
|child|\textit{nn}|.tex|:
%
\begin{center}
\begin{tabular}{l}
|\def\version{final}|\\
|\input{childdoc.def}|\\
|\childdocforwardprefix{final}{child}|
\end{tabular}
\end{center}
%

Note that when several versions of a main file and/or of each child file
are to be generated, it may be convenient to set up a |Makefile| or
shell script to automatise the process.

%%%%%%%%%%%%%%%%%%%%%%%%%%%%%%%%%%%%%%%%%%%%%%%%%%%%%%%%%%%%%%%%%%%%%%%%%%%%%%%%
\subsection{Command Line Processing}
\label{sec:commandline}

The effect of redirection files can also be achieved by invoking
the \LaTeX{} compiler with a more elaborate command line.
Most conveniently this should be done as part
of a shell script or a |Makefile|.

When using \textsf{childdoc} in the main file, the following
command lines effectively perform a redirection
(note that depending on the shell being used,
backslashes may have to be doubled: `|\|' $\to$ `|\\|'):
%
\begin{center}
|... -jobname "|\textit{target}|" |\\|"|[\textit{flags}]%
|\input{childdoc.def}\childdocforward[|\textit{main}|]{|\textit{dest}|}"|
\end{center}
%
Here \textit{target} is the name of the output file,
\textit{main} is the name of the main file
and \textit{dest} is the name of the main or child file to be processed
(all filenames without extensions).
The optional argument \textit{main} can be omitted
if \textit{main} matches \textit{dest}.
Optionally, compilation \textit{flags} can be defined via |\def| commands.
This command line makes the \TeX{} engine believe
it is compiling the file \textit{target}
whose content is specified as the latter parameter.
The provided code then forwards the processing to
\textit{main} or \textit{dest} as described in \secref{sec:forward}.

%%%%%%%%%%%%%%%%%%%%%%%%%%%%%%%%%%%%%%%%%%%%%%%%%%%%%%%%%%%%%%%%%%%%%%%%%%%%%%%%
\subsection{Include by Input}
\label{sec:input}

Including child documents by |\include| has some restrictions by design.
Most notably, the content of a child document always occupies
its own set of pages; pages cannot be shared between child documents.
Usually, this behaviour makes perfect sense
because each child document contain an essential part of the document.
However, in some situations it may be desirable to compose
a document from a collection of parts
without having mandatory page breaks between then.
For this case, the package
provides a mechanism to include parts
by |\input| which can also be processed individually.
However, by construction this mechanism
requires manual handling of the content to be output.

%%%%%%%%%%%%%%%%%%%%%%%%%%%%%%%%%%%%%%%%
\DescribeMacro{\ifchilddocmanual}
The main file should be prepared as usual, see \secref{sec:include}.
However, the document body must make a distinction
between processing of an individual part and of the main document, e.g.:
%
\begin{center}
\begin{tabular}{l}
|\ifchilddocmanual|\\
|\input{\childdocname}|\\
|\||else|\\
\textit{document body with }|\input{|\textit{part}|}|\\
|\||fi|
\end{tabular}
\end{center}
%
The conditional |\ifchilddocmanual| is true whenever
a part to be included by |\input| is being compiled,
and the name of the part is stored in |\childdocname|.

%%%%%%%%%%%%%%%%%%%%%%%%%%%%%%%%%%%%%%%%
\DescribeMacro{\childdocby}
Each part to be included by |\input| should start with:
%
\begin{center}
\begin{tabular}{l}
|\input{childdoc.def}|\\
|\childdocby{|\textit{main}|}|\\
\end{tabular}
\end{center}
%
The directive |\childdocby| is similar to |\childdocof|
described in \secref{sec:include},
but the subsequent selection of content must be done manually.
To that end, both |\ifchilddoc| and |\ifchilddocmanual|
will be true upon processing of a part,
and the name of the part is stored in |\childdocname|.
Note that |\jobname| will be set to the filename of the current part
so that each part receives an individual |.aux| file
that does not interfere with the |.aux| file(s) of the main document.
This behaviour can be altered by the alternative form
|\childdocby[*]{|\textit{main}|}| (with a non-empty optional argument)
which uses the |.aux| file of the main document
by setting |\jobname| to \textit{main}.

%%%%%%%%%%%%%%%%%%%%%%%%%%%%%%%%%%%%%%%%%%%%%%%%%%%%%%%%%%%%%%%%%%%%%%%%%%%%%%%%
\subsection{Driver Development}
\label{sec:driver}

The \textsf{childdoc} mechanism can also be use for the development
of definition files such as \LaTeX{} styles or classes.
This case differs from the above setup with multiple parts
included by |\include| in that no |\includeonly| should be invoked.
This can be achieved by starting the include file
(before |\ProvidesPackage|) with:
%
\begin{center}
\begin{tabular}{l}
|\input{childdoc.def}|\\
|\childdocforward{|\textit{main}|}|\\
\end{tabular}
\end{center}
%
or alternatively with:
%
\begin{center}
\begin{tabular}{l}
|\input{childdoc.def}|\\
|\childdocby{|\textit{main}|}|\\
\end{tabular}
\end{center}
%
Both forms have slightly different effects as described above.
The main file is prepared as usual, see \secref{sec:include}.

%%%%%%%%%%%%%%%%%%%%%%%%%%%%%%%%%%%%%%%%%%%%%%%%%%%%%%%%%%%%%%%%%%%%%%%%%%%%%%%%
\subsection{Legacy Detection}
\label{sec:detection}

The directive |\childdocmain| in the main file can detect
whether the complete document or merely a child is to be compiled
even without using the directive |\childdocof|.
This method is deprecated because it is less robust
and there is no compelling reason to use it;
it is merely provided for backward compatibility
and it may be removed in future versions.

If the detection mechanism is to be used,
it is mandatory to correctly specify
the filename of the main file as the argument of |\childdocmain|:
%
\begin{center}
\begin{tabular}{l}
|\input{childdoc.def}|\\
|\childdocmain{|\textit{main}|}|\\
\end{tabular}
\end{center}
%
If |\jobname| does not match the argument \textit{main} of |\childdocmain|,
it is assumed that |\jobname| points to the child file to be compiled.
When using |\childdocmain| with the main file specified as argument,
it suffices to start a child file
with just |\input{|\textit{main}|}|
without loading of the package and using |\childdocof|.
If instead all processing is done
with the appropriate \textsf{childdoc} directives,
the argument of \textit{main} of |\childdocmain| can be empty.

An alternative version of the command line processing described
in \secref{sec:commandline} using the detection mechanism reads:
%
\begin{center}
|... -jobname "|\textit{target}|" "|[\textit{flags}]%
[|\def\jobname{|\textit{dest}|}|]|\input{|\textit{main}|}"|
\end{center}

%%%%%%%%%%%%%%%%%%%%%%%%%%%%%%%%%%%%%%%%%%%%%%%%%%%%%%%%%%%%%%%%%%%%%%%%%%%%%%%%
\subsection{Manual Code}
\label{sec:manual}

In case one cannot be certain whether the definitions file |childdoc.def|
is installed on the target \TeX{} distribution
and one prefers not to ship it,
it is conceivable to paste a few relevant commands into the sources.

To that end, drop all statements |\input{childdoc.def}|
and perform the replacements as outlined below.
Instead of |\childdocmain{|\textit{main}|}| add the following code
to the top of the main file:
%
\begin{center}
\begin{tabular}{l}
|\||ifdefined\childdocname\endinput\||fi\newif\ifchilddoc|\\
|\edef\childdocname{\scantokens\expandafter{\jobname\noexpand}}|\\
|\def\childdocmain{|\textit{main}|}\||ifx\childdocmain\childdocname\||else|\\
|\childdoctrue\includeonly{\childdocname}\let\jobname\childdocmain\||fi|\\
\end{tabular}
\end{center}
%
Instead of |\childdocof{|\textit{main}|}| just include the main file
at the top of each child file:
%
\begin{center}
|\input{|\textit{main}|}|
\end{center}
%
A simple redirection |\childdocforward{|\textit{dest}|}| is achieved by:
%
\begin{center}
|\def\jobname{|\textit{dest}|}\input{\jobname}|
\end{center}
%
The redirection with prefix
|\childdocforwardprefix[|\textit{prefix}|]{|\textit{dest}|}|
is accomplished by:
%
\begin{center}
\begin{tabular}{l}
|{\edef\jobname{\scantokens\expandafter{\jobname\noexpand}}|\\
|\def\redirectjob |\textit{prefix}|#1~~~{\gdef\jobname{|\textit{dest}|#1}}|\\
|\expandafter\redirectjob\jobname~~~}\input{\jobname}|
\end{tabular}
\end{center}

In an alternative approach,
child documents can be compiled by a specific command line
without additional code or specific definitions:
%
\begin{center}
|... -jobname "|\textit{target}|" "|[\textit{flags}]%
|\includeonly{|\textit{dest}|}\input{|\textit{main}|}"|
\end{center}
%

%%%%%%%%%%%%%%%%%%%%%%%%%%%%%%%%%%%%%%%%%%%%%%%%%%%%%%%%%%%%%%%%%%%%%%%%%%%%%%%%
%%%%%%%%%%%%%%%%%%%%%%%%%%%%%%%%%%%%%%%%%%%%%%%%%%%%%%%%%%%%%%%%%%%%%%%%%%%%%%%%
\section{Information}

%%%%%%%%%%%%%%%%%%%%%%%%%%%%%%%%%%%%%%%%%%%%%%%%%%%%%%%%%%%%%%%%%%%%%%%%%%%%%%%%
\subsection{Copyright}

Copyright \copyright{} 2017--2018 Niklas Beisert

This work may be distributed and/or modified under the
conditions of the \LaTeX{} Project Public License, either version 1.3
of this license or (at your option) any later version.
The latest version of this license is in
  \url{http://www.latex-project.org/lppl.txt}
and version 1.3 or later is part of all distributions of \LaTeX{}
version 2005/12/01 or later.

This work has the LPPL maintenance status `maintained'.

The Current Maintainer of this work is Niklas Beisert.

This work consists of the files |README.txt|, |childdoc.ins| and |childdoc.dtx|
as well as the derived files |childdoc.def|, |cdocsamp.tex|
with |cdocsch1.tex|, |cdocsch2.tex|, |cdocspt3.tex|, |cdocspt4.tex|,
|cdocsdrf.tex|, |cdocsfn1.tex|, |cdocsfn2.tex|
as well as |childdoc.pdf|.

%%%%%%%%%%%%%%%%%%%%%%%%%%%%%%%%%%%%%%%%%%%%%%%%%%%%%%%%%%%%%%%%%%%%%%%%%%%%%%%%
\subsection{Files and Installation}

The package consists of the files:
%
\begin{center}
\begin{tabular}{ll}
    |README.txt|   & readme file \\
    |childdoc.ins| & installation file \\
    |childdoc.dtx| & source file \\
    |childdoc.def| & definition file \\
    |cdocsamp.tex| & sample main file \\
    |cdocsch1.tex| & sample include file \\
    |cdocsch2.tex| & sample include file \\
    |cdocspt3.tex| & sample part file \\
    |cdocspt4.tex| & sample part file \\
    |cdocsdrf.tex| & sample redirection file \\
    |cdocsfn1.tex| & sample redirection file \\
    |cdocsfn2.tex| & sample redirection file \\
    |childdoc.pdf| & manual
\end{tabular}
\end{center}
%
The distribution consists of the files
|README.txt|, |childdoc.ins| and |childdoc.dtx|.
%
\begin{itemize}
\item
Run (pdf)\LaTeX{} on |childdoc.dtx|
to compile the manual |childdoc.pdf| (this file).
\item
Run \LaTeX{} on |childdoc.ins| to create the definitions file |childdoc.def|
and the sample |cdocsamp.tex| with include files
|cdocsch1.tex|, |cdocsch2.tex|, |cdocspt3.tex|, |cdocspt4.tex|,
|cdocsdrf.tex|, |cdocsfn1.tex|, |cdocsfn2.tex|.
Then copy the file |childdoc.def| to an appropriate directory of your \LaTeX{}
distribution, e.g.\ \textit{texmf-root}|/tex/latex/childdoc|.
\end{itemize}

%%%%%%%%%%%%%%%%%%%%%%%%%%%%%%%%%%%%%%%%%%%%%%%%%%%%%%%%%%%%%%%%%%%%%%%%%%%%%%%%
\subsection{Related CTAN Packages}

There are several other packages which offer a similar functionality:
%
\begin{itemize}
\item
The packages
\href{http://ctan.org/pkg/docmute}{\textsf{docmute}},
\href{http://ctan.org/pkg/includex}{\textsf{includex}} and
\href{http://ctan.org/pkg/standalone}{\textsf{standalone}}
provide commands to include only the document body of
a child file thus allowing both files to be compiled individually.
\item
The packages \href{http://ctan.org/pkg/subdocs}{\textsf{subdocs}}
and \href{http://ctan.org/pkg/subfiles}{\textsf{subfiles}}
provide structures in which the main and child documents can be
encapsulated and allowing them to be compiled individually.
The inclusion mechanism is different from the conventional |\include|.
\item
The package \href{http://ctan.org/pkg/combine}{\textsf{combine}}
is an elaborate solution to combine several documents into one.
\end{itemize}
%
See also the CTAN topic \href{http://ctan.org/topic/subdocs}{\textsf{subdocs}}
for further related packages.
The present package differs from the above solutions in that
a document structure constructed with the conventional |\include| mechanism
just needs two extra commands at the top of every file
such that all constituent files can be compiled individually.

%%%%%%%%%%%%%%%%%%%%%%%%%%%%%%%%%%%%%%%%%%%%%%%%%%%%%%%%%%%%%%%%%%%%%%%%%%%%%%%%
%\subsection{Feature Suggestions}
%
%The following is a list of features which may be useful for future
%versions of this package:
%%
%\begin{itemize}
%\item
%\ldots
%\end{itemize}

%%%%%%%%%%%%%%%%%%%%%%%%%%%%%%%%%%%%%%%%%%%%%%%%%%%%%%%%%%%%%%%%%%%%%%%%%%%%%%%%
\subsection{Revision History}

%%%%%%%%%%%%%%%%%%%%%%%%%%%%%%%%%%%%%%%%
\paragraph{v2.0:} 2018/12/30

\begin{itemize}
\item
immediate forward processing
\item
added |\childdocby| mechanism
\item
manual restructured
\end{itemize}

%%%%%%%%%%%%%%%%%%%%%%%%%%%%%%%%%%%%%%%%
\paragraph{v1.6:} 2018/01/17

\begin{itemize}
\item
application for development of include files
\item
corrections to manual
\end{itemize}

%%%%%%%%%%%%%%%%%%%%%%%%%%%%%%%%%%%%%%%%
\paragraph{v1.5:} 2017/05/21

\begin{itemize}
\item
more complete structuring introduced
\item
|\childdocof| introduced
\item
|\childdoc| renamed to |\childdocmain|
\item
|\childredirect| renamed to |\childdocforward| and |\childdocforwardprefix|
and functionality expanded
\end{itemize}

%%%%%%%%%%%%%%%%%%%%%%%%%%%%%%%%%%%%%%%%
\paragraph{v1.0:} 2017/04/27

\begin{itemize}
\item
manual and install package
\item
first version published on CTAN
\end{itemize}

%%%%%%%%%%%%%%%%%%%%%%%%%%%%%%%%%%%%%%%%
\paragraph{v0.6:} 2017/04/26

\begin{itemize}
\item
redirection mechanism added
\end{itemize}

%%%%%%%%%%%%%%%%%%%%%%%%%%%%%%%%%%%%%%%%
\paragraph{v0.5:} 2017/04/26

\begin{itemize}
\item
functionality in definition file
\end{itemize}


%%%%%%%%%%%%%%%%%%%%%%%%%%%%%%%%%%%%%%%%%%%%%%%%%%%%%%%%%%%%%%%%%%%%%%%%%%%%%%%%
%%%%%%%%%%%%%%%%%%%%%%%%%%%%%%%%%%%%%%%%%%%%%%%%%%%%%%%%%%%%%%%%%%%%%%%%%%%%%%%%
%%%%%%%%%%%%%%%%%%%%%%%%%%%%%%%%%%%%%%%%%%%%%%%%%%%%%%%%%%%%%%%%%%%%%%%%%%%%%%%%
\appendix

\settowidth\MacroIndent{\rmfamily\scriptsize 000\ }

 \DocInput{childdoc.dtx}

\end{document}
%</driver>
% \fi
%
% %%%%%%%%%%%%%%%%%%%%%%%%%%%%%%%%%%%%%%%%%%%%%%%%%%%%%%%%%%%%%%%%%%%%%%%%%%%%%%
% %%%%%%%%%%%%%%%%%%%%%%%%%%%%%%%%%%%%%%%%%%%%%%%%%%%%%%%%%%%%%%%%%%%%%%%%%%%%%%
% \section{Sample}
%\iffalse
%<*samplemain>
%\fi
%
% The following presents a sample document
% with two chapters, two parts, a title page,
% a compile flag as well as three forwarding files to set the flag.
% It consists of eight |.tex| files:
% \begin{center}
% \begin{tabular}{ll}
% |cdocsamp.tex|&main file\\
% |cdocsch1.tex|&include file for chapter 1\\
% |cdocsch2.tex|&include file for chapter 2\\
% |cdocspt3.tex|&include file for part 3\\
% |cdocspt4.tex|&include file for part 4\\
% |cdocsdrf.tex|&forwarding file for main file in draft mode\\
% |cdocsfi1.tex|&forwarding file for final version of chapter 1\\
% |cdocsfi2.tex|&forwarding file for final version of chapter 2\\
% \end{tabular}
% \end{center}
% Each of the eight files can be compiled directly by the \LaTeX{} compiler.
%
% %%%%%%%%%%%%%%%%%%%%%%%%%%%%%%%%%%%%%%
% \paragraph{Main File.}
%
% The main file is called |cdocsamp.tex|.
%
% Load the \textsf{childdoc} definitions and
% declare the filename for the main document:
%    \begin{macrocode}
\input{childdoc.def}
\childdocmain{}
%    \end{macrocode}

% Optional override for |\version| flag:
%    \begin{macrocode}
%%\ifchilddoc\else\providecommand{\version}{draft}\fi
%    \end{macrocode}

% Define the default values for the |\version| flag
% (|final| for the main file and |draft| for childs):
%    \begin{macrocode}
\ifchilddoc
\providecommand{\version}{draft}
\else
\providecommand{\version}{final}
\fi
%    \end{macrocode}

% Load the standard document class:
%    \begin{macrocode}
\documentclass[12pt]{article}
%    \end{macrocode}

% Start the document body:
%    \begin{macrocode}
\begin{document}
%    \end{macrocode}

% Declare a title page.
% Print title, part of document being processed and version flag:
%    \begin{macrocode}
\addtocounter{page}{-1}
\begin{center}
{\LARGE\bfseries{}childdoc example\par}
\vspace{1cm}
\ifchilddoc
\ifchilddocmanual part\else chapter\fi:
`\childdocname' of `\childdocjob'\par
\else
main document: `\childdocjob'\par
\fi
version: \version\par
\end{center}
\newpage
%    \end{macrocode}

% Manually include selected file,
% otherwise process as usual:
%    \begin{macrocode}
\ifchilddocmanual
\section*{part `\childdocname'}
\input{\childdocname}
\else
%    \end{macrocode}

% Include the two chapters:
%    \begin{macrocode}
\include{cdocsch1}
\include{cdocsch2}
%    \end{macrocode}

% Include the two parts unless only chapters should be displayed:
%    \begin{macrocode}
\ifchilddoc\else
\section{part three}
\input{cdocspt3}
\section{part four}
\input{cdocspt4}
\fi
%    \end{macrocode}

% Process as usual until here:
%    \begin{macrocode}
\fi
%    \end{macrocode}

% End of document body:
%    \begin{macrocode}
\end{document}
%    \end{macrocode}
%\iffalse
%</samplemain>
%\fi
%
% %%%%%%%%%%%%%%%%%%%%%%%%%%%%%%%%%%%%%%
% \paragraph{Chapter Include Files.}
%
% The include files are called |cdocsch1.tex| and |cdocsch2.tex|.
%
%\iffalse
%<*samplechap1|samplechap2>
%\fi

% Optional override for |\version| flag:
%    \begin{macrocode}
%%\providecommand{\version}{final}
%    \end{macrocode}

% Include the main document:
%    \begin{macrocode}
\input{childdoc.def}
\childdocof{cdocsamp}
%    \end{macrocode}

%\iffalse
%</samplechap1|samplechap2>
%\fi
%
%\iffalse
%<*samplechap1>
%\fi
% Some text for chapter 1:
%    \begin{macrocode}
\section{one}
some text in chapter one
%    \end{macrocode}

%\iffalse
%</samplechap1>
%\fi
% Some text for chapter 2:
%\iffalse
%<*samplechap2>
%\fi
%    \begin{macrocode}
\section{two}
more text in chapter two
%    \end{macrocode}

%\iffalse
%</samplechap2>
%\fi
%
% %%%%%%%%%%%%%%%%%%%%%%%%%%%%%%%%%%%%%%
% \paragraph{Part Include Files.}
%
% The include files are called |cdocspt3.tex| and |cdocspt4.tex|.
%
%\iffalse
%<*samplepart3|samplepart4>
%\fi

% Optional override for |\version| flag:
%    \begin{macrocode}
%%\providecommand{\version}{final}
%    \end{macrocode}

% Include the main document:
%    \begin{macrocode}
\input{childdoc.def}
\childdocby{cdocsamp}
%    \end{macrocode}

%\iffalse
%</samplepart3|samplepart4>
%\fi
%
%\iffalse
%<*samplepart3>
%\fi
% Some text for part 3:
%    \begin{macrocode}
some text in part three
%    \end{macrocode}

%\iffalse
%</samplepart3>
%\fi
% Some text for part 4:
%\iffalse
%<*samplepart4>
%\fi
%    \begin{macrocode}
more text in part four
%    \end{macrocode}

%\iffalse
%</samplepart4>
%\fi
%
% %%%%%%%%%%%%%%%%%%%%%%%%%%%%%%%%%%%%%%
% \paragraph{Forwarding for a Complete Draft.}
%
% The following forwarding file |cdocsdrf.tex|
% compiles the main document in draft mode:
%\iffalse
%<*sampledraft>
%\fi
%    \begin{macrocode}
\def\version{draft}
\input{childdoc.def}
\childdocforward{cdocsamp}
%    \end{macrocode}

%\iffalse
%</sampledraft>
%\fi
%
% %%%%%%%%%%%%%%%%%%%%%%%%%%%%%%%%%%%%%%
% \paragraph{Forwarding for Final Version of the Chapters.}
%
% The following forwarding files |cdocsfn1.tex| and |cdocsfn2.tex|
% (with identical content)
% compile the final versions of the child documents
% |cdocsch1.tex| and |cdocsch2.tex|, respectively:
%\iffalse
%<*samplefinal>
%\fi
%    \begin{macrocode}
\def\version{final}
\input{childdoc.def}
\childdocforwardprefix[cdocsamp]{cdocsfn}{cdocsch}
%    \end{macrocode}

%\iffalse
%</samplefinal>
%\fi
%
% %%%%%%%%%%%%%%%%%%%%%%%%%%%%%%%%%%%%%%
% \paragraph{Command Line Processing.}
%
% The following three command lines generate the output files
% |cdocscld|, |cdocscl1| and |cdocscl2|
% which should be identical to
% |cdocsdrf|, |cdocsch1| and |cdocsfn2|, respectively:
% \begin{center}
% \begin{tabular}{l}
% |latex -jobname cdocscld \|\\
% |  "\def\version{draft}\input{childdoc.def}\childdocforward{cdocsamp}"|\\
% |latex -jobname cdocscl1 \|\\
% |  "\input{childdoc.def}\childdocforward[cdocsamp]{cdocsch1}"|\\
% |latex -jobname cdocscl2 \|\\
% |  "\def\version{final}\input{childdoc.def}\childdocforward{cdocsch2}"|
% \end{tabular}
% \end{center}
% Note that the trailing backslash on each first line
% merely continues the input to the second line
% (for convenient cut ant paste).
% Furthermore, the command |latex| can be replaced by any
% of its alternative versions such as |pdflatex|.
%
% %%%%%%%%%%%%%%%%%%%%%%%%%%%%%%%%%%%%%%%%%%%%%%%%%%%%%%%%%%%%%%%%%%%%%%%%%%%%%%
% %%%%%%%%%%%%%%%%%%%%%%%%%%%%%%%%%%%%%%%%%%%%%%%%%%%%%%%%%%%%%%%%%%%%%%%%%%%%%%
% \section{Implementation}
%\iffalse
%<*package>
%\fi
%
% This section describes the definitions file |childdoc.def|.

% The definitions cannot be loaded using |\usepackage| or |\RequirePackage|
% which has a mechanism to prevent loading a style file more than once.
% When loading the definitions by means of |\input|
% multiple instances have to be prevented manually:
%\iffalse
%This code needs to be before the `\ProvidesFile' directive
%which is defined at the beginning of this file.
%Therefore it is also placed there and commented out here.
%</package>
%<*discard>
%\fi
%    \begin{macrocode}
\ifdefined\childdocmain\endinput\fi
%    \end{macrocode}
%\iffalse
%</discard>
%<*package>
%\fi
%
% \macro{\ifchilddoc}
% \macro{\ifchilddocmanual}
% The conditional |\ifchilddoc| tells whether a
% child (true) or main (false) document is being compiled.
% The conditional |\ifchilddocmanual| tells whether
% the |\includeonly| mechanism is used (false) or
% the selection of child files must be performed manually (true).
% The definitions initialise to false:
%    \begin{macrocode}
\newif\ifchilddoc
\newif\ifchilddocmanual
%    \end{macrocode}

% \macro{\childdocname}
% \macro{\childdocjob}
% The macro |\childdocname| stores the name of the main document
% to be compiled. The macro |\childdocjob| stores the name of
% the document on which the \LaTeX{} compiler was originally invoked.
% The content of |\jobname| cannot be compared
% to filenames specified in the source due to different catcodes.
% The following code rescans |\jobname|, stores the result
% in |\childdocname| and saves a copy in |\childdocjob|:
%    \begin{macrocode}
\edef\childdocname{\scantokens\expandafter{\jobname\noexpand}}
\let\childdocjob\childdocname
%    \end{macrocode}

% \macro{\childdocdisable}
% The macro |\childdocdisable| prevents the main file
% from being processed more than once.
% At this stage, the main document command |\childdocmain|
% is assumed to be called once again where it should do nothing.
% Any subsequent call to it should prevent
% a secondary processing of the main document
% It overwrites the forwarding commands
% |\childdocof| and |\childdocforward|
% with empty macros to prevent further inclusions of the main document:
%    \begin{macrocode}
\newcommand{\childdocdisable}
{
  \renewcommand{\childdocmain}[1]{\renewcommand{\childdocmain}[1]{\endinput}}
  \renewcommand{\childdocof}[1]{}
  \renewcommand{\childdocby}[2][]{}
  \renewcommand{\childdocforward}[2][]{}
  \renewcommand{\childdocdisable}{}
}
%    \end{macrocode}

% \macro{\childdocmain}
% The macro |\childdocmain| is to be called at the top of the main file
% with nothing or the main filename (without extension) as argument.
% First, it breaks loops.
% If the argument is not empty and does not match |\childdocname|
% (which is set by the first inclusion of |childdoc.def|),
% |\ifchilddoc| is set to true, |\includeonly| is applied to the child file
% and |\jobname| is set to the main file
% (for proper handling of |.aux| files):
%    \begin{macrocode}
\newcommand{\childdocmain}[1]
{
  \childdocdisable\childdocmain{}
  \if?#1?\else
    \begingroup
      \def\childdoctmp{#1}
      \ifx\childdoctmp\childdocname
        \def\childdoctmp{}
      \else
        \def\childdoctmp
        {
          \childdoctrue
          \includeonly{\childdocname}
          \def\childdocjob{#1}
          \def\jobname{#1}
        }
      \fi
      \expandafter
    \endgroup
    \childdoctmp
  \fi
}
%    \end{macrocode}

% \macro{\childdocof}
% The command |\childdocof| redirects
% compilation to the main file |#1|.
%    \begin{macrocode}
\newcommand{\childdocof}[1]
{
  \childdocdisable
  \childdoctrue
  \includeonly{\childdocname}
  \def\jobname{#1}
  \def\childdocjob{#1}
  \input{#1}
}
%    \end{macrocode}

% \macro{\childdocby}
% The command |\childdocby| ....
%    \begin{macrocode}
\newcommand{\childdocby}[2][]
{
  \childdocdisable
  \childdoctrue
  \childdocmanualtrue
  \if?#1?\else
    \def\jobname{#2}
  \fi
  \def\childdocjob{#2}
  \input{#2}
  \endinput
}
%    \end{macrocode}

% \macro{\childdocforward}
% The command |\childdocforward| redirects
% compilation to the main file or
% (if the optional argument is given) a child file.
% Parameters are set as if the main file
% or a child file starting with |\childdocof| was compiled.
% Then compilation is handed over to the main file:
%    \begin{macrocode}
\newcommand{\childdocforward}[2][]
{
  \begingroup
    \if?#1?
      \def\childdoctmp
      {
        \def\childdocname{#2}
        \def\childdocjob{#2}
        \def\jobname{#2}
        \input{#2}
        \endinput
      }
    \else
      \def\childdoctmp
      {
        \childdocdisable
        \def\childdocname{#2}
        \childdoctrue
        \includeonly{#2}
        \def\childdocjob{#1}
        \def\jobname{#1}
        \input{#1}
        \endinput
      }
    \fi
    \expandafter
  \endgroup
  \childdoctmp
}
%    \end{macrocode}

% \macro{\childdocforwardprefix}
% The command |\childdocforwardprefix| redirects
% compilation to the main or a child file by means of a pattern.
% The prefix |#1| in the current filename is replaced by |#2|
% and the suffix of the current filename is kept
% (it is assumed that the filename does not contain the substring `|~~~|'
% which is used as a delimiter).
% Compilation is handed over to the new file by |\childdocforward|:
%    \begin{macrocode}
\newcommand{\childdocforwardprefix}[3][]
{
  \begingroup
    \def\childdocextract #2##1~~~{\def\childdoctmp{\childdocforward[#1]{#3##1}}}
    \expandafter\childdocextract\childdocname~~~
    \expandafter
  \endgroup
  \childdoctmp
}
%    \end{macrocode}

% \macro{\childdoc}
% The deprecated macro |\childdoc| is a legacy version of |\childdocmain|:
%    \begin{macrocode}
\newcommand{\childdoc}{\childdocmain}
%    \end{macrocode}

% \macro{\childdocredirect}
% The deprecated macro |\childdocredirect| is a legacy version
% of |\childdocforward| and |\childdocforwardprefix|:
%    \begin{macrocode}
\newcommand{\childdocredirect}[2][]
{
  \begingroup
    \if?#1?
      \def\childdoctmp{\childdocforward{#2}}
    \else
      \def\childdoctmp{\childdocforwardprefix{#1}{#2}}
    \fi
    \expandafter
  \endgroup
  \childdoctmp
}
%    \end{macrocode}

%\iffalse
%</package>
%\fi
%
\endinput
|\\
|\childdocforward{|\textit{main}|}|
\end{tabular}
\end{center}
%
Likewise, the following files |final|\textit{nn}|.tex|
compile the final version of the child document
|child|\textit{nn}|.tex|:
%
\begin{center}
\begin{tabular}{l}
|\def\version{final}|\\
|% \iffalse
%
% childdoc.dtx Copyright (C) 2017-2018 Niklas Beisert
%
% This work may be distributed and/or modified under the
% conditions of the LaTeX Project Public License, either version 1.3
% of this license or (at your option) any later version.
% The latest version of this license is in
%   http://www.latex-project.org/lppl.txt
% and version 1.3 or later is part of all distributions of LaTeX
% version 2005/12/01 or later.
%
% This work has the LPPL maintenance status `maintained'.
%
% The Current Maintainer of this work is Niklas Beisert.
%
% This work consists of the files childdoc.dtx and childdoc.ins
% and the derived files childdoc.def and cdocsamp.tex with
% cdocsch1.tex, cdocsch2.tex, cdocsdrf.tex, cdocsfn1.tex, cdocsfn2.tex.
%
%<package>\ifdefined\childdocmain\endinput\fi
%<package>\ProvidesFile{childdoc.def}[2018/12/30 v2.0 child document driver]
%<samplemain>\ProvidesFile{cdocsamp.tex}[2018/12/30 v2.0 sample for childdoc]
%<*driver>
%\ProvidesFile{childdoc.drv}[2018/12/30 v2.0 childdoc reference manual file]
\PassOptionsToClass{10pt,a4paper}{article}
\documentclass{ltxdoc}

\usepackage[margin=35mm]{geometry}
\usepackage{hyperref}
\usepackage{hyperxmp}
\usepackage[usenames]{color}

\hypersetup{colorlinks=true}
\hypersetup{pdfstartview=FitH}
\hypersetup{pdfpagemode=UseNone}
\hypersetup{pdfsource={}}
\hypersetup{pdflang={en-UK}}
\hypersetup{pdfcopyright={Copyright 2017-2018 Niklas Beisert.
  This work may be distributed and/or modified under the
  conditions of the LaTeX Project Public License, either version 1.3
  of this license or (at your option) any later version.}}
\hypersetup{pdflicenseurl={http://www.latex-project.org/lppl.txt}}
\hypersetup{pdfcontactaddress={ETH Zurich, ITP, HIT K,
  Wolfgang-Pauli-Strasse 27}}
\hypersetup{pdfcontactpostcode={8093}}
\hypersetup{pdfcontactcity={Zurich}}
\hypersetup{pdfcontactcountry={Switzerland}}
\hypersetup{pdfcontactemail={nbeisert@itp.phys.ethz.ch}}
\hypersetup{pdfcontacturl={http://people.phys.ethz.ch/\xmptilde nbeisert/}}

\newcommand{\secref}[1]{\hyperref[#1]{section \ref*{#1}}}

\parskip1ex
\parindent0pt
\let\olditemize\itemize
\def\itemize{\olditemize\parskip0pt}

\begin{document}

\title{The \textsf{childdoc} Package}
\hypersetup{pdftitle={The childdoc Package}}
\author{Niklas Beisert\\[2ex]
  Institut f\"ur Theoretische Physik\\
  Eidgen\"ossische Technische Hochschule Z\"urich\\
  Wolfgang-Pauli-Strasse 27, 8093 Z\"urich, Switzerland\\[1ex]
  \href{mailto:nbeisert@itp.phys.ethz.ch}
  {\texttt{nbeisert@itp.phys.ethz.ch}}}
\hypersetup{pdfauthor={Niklas Beisert}}
\hypersetup{pdfsubject={Manual for the LaTeX2e Package childdoc}}
\date{30 December 2018, \textsf{v2.0}}
\maketitle

\begin{abstract}\noindent
\textsf{childdoc} is a \LaTeXe{} package
that enables the direct compilation
of document sections included by |\include|
to individual files.
\end{abstract}

\begingroup
\parskip0ex
\tableofcontents
\endgroup

%%%%%%%%%%%%%%%%%%%%%%%%%%%%%%%%%%%%%%%%%%%%%%%%%%%%%%%%%%%%%%%%%%%%%%%%%%%%%%%%
%%%%%%%%%%%%%%%%%%%%%%%%%%%%%%%%%%%%%%%%%%%%%%%%%%%%%%%%%%%%%%%%%%%%%%%%%%%%%%%%
\section{Introduction}

\LaTeX{} provides a mechanism to structure a large document (such as a book)
into a main file and several child files (containing the chapters)
using the |\include| command.
This mechanism is beneficial for documents
which span hundreds of pages in order to
make the source file(s) more manageable.
Moreover, compilation can be restricted to
selected child files by means of the |\includeonly| command.
The latter feature can be used to reduce the compilation time while editing
(this was significantly more useful in the earlier days of \LaTeX{})
or to generate a smaller document which is easier to navigate.
Another application of |\includeonly| is to generate
documents consisting of selected parts of the complete document.

However, there are a few drawbacks of the plain |\include| mechanism:
\begin{itemize}
\item
The child files cannot be compiled on their own,
they can only be compiled via the main file.
A naive editing environment
(such as a text editor with an option
to have the current file processed by \LaTeX)
may require one to switch to the main file before compiling;
attempting to compile the child file produces errors.
\item
The main file must be modified (each time)
to adjust the |\includeonly| command
to the present needs. This easily leaves the main file in a messy state.
\item
The generated document will always carry the filename
of the main document. This is inconvenient if
several child files are to be compiled and
to be kept for distribution.
\end{itemize}

The present package provides a simple interface
to make child files individually compilable by \LaTeX{}.
Compiling a child file then has the same effect as compiling
the main file with an |\includeonly| command
to select the appropriate child.
Moreover the generated document will carry the name of the child
rather than the main file.
This resolves all three above issues.

This feature is meant to make the editing of books,
thesis documents and lecture notes somewhat more convenient.
However, the package can also be used efficiently for
composing a series of documents (such as exercise sheets)
which are typically distributed individually.
It then assists the author in generating the individual documents
(potentially in different versions)
as well as a document containing the collected series.
Another application is in developing style files
or other kinds of included material
where compilation of the style file could redirect
to a sample or test file.

%%%%%%%%%%%%%%%%%%%%%%%%%%%%%%%%%%%%%%%%%%%%%%%%%%%%%%%%%%%%%%%%%%%%%%%%%%%%%%%%
%%%%%%%%%%%%%%%%%%%%%%%%%%%%%%%%%%%%%%%%%%%%%%%%%%%%%%%%%%%%%%%%%%%%%%%%%%%%%%%%
\section{Usage}

First of all, the package \textsf{childdoc} is \emph{not} a standard
\LaTeXe{} |.sty| style file! Therefore it needs to be invoked in
a non-standard way.

%%%%%%%%%%%%%%%%%%%%%%%%%%%%%%%%%%%%%%%%%%%%%%%%%%%%%%%%%%%%%%%%%%%%%%%%%%%%%%%%
\subsection{Included Files}
\label{sec:include}

%%%%%%%%%%%%%%%%%%%%%%%%%%%%%%%%%%%%%%%%
\DescribeMacro{\childdocmain}
To use the package, add the commands
\begin{center}
\begin{tabular}{l}
|\input{childdoc.def}|\\
|\childdocmain{}|\\
\end{tabular}
\end{center}
at the very top of the main \LaTeX{} file,
in particular \emph{before} the |\documentclass| statement!
The argument of |\childdocmain| should be left empty
(but it must be present).

%%%%%%%%%%%%%%%%%%%%%%%%%%%%%%%%%%%%%%%%
\DescribeMacro{\childdocof}
Furthermore, add the commands
\begin{center}
\begin{tabular}{l}
|\input{childdoc.def}|\\
|\childdocof{|\textit{main}|}|\\
\end{tabular}
\end{center}
at the top of every child file \textit{child}
which is included by |\include{|\textit{child}|}|
from within the main file
(or at least for those files to be compiled individually).
The argument \textit{main} must be the filename of the main file.

There are a couple of
considerations in setting up the main and child documents:

%%%%%%%%%%%%%%%%%%%%%%%%%%%%%%%%%%%%%%%%
\paragraph{Restrictions.}

Please note the following restrictions:
\begin{itemize}
\item
|\childdocmain| must be called with one argument \textit{main}
to ensure compatibility with earlier version of the package.
It must either be empty (|\childdocmain{}|)
or precisely match the filename of the main file in which it is specified.
See \secref{sec:detection} for further information.
\item
The filename \textit{main} must be specified without the |.tex| extension.
\item
The filename \textit{main} is case sensitive
(even in case-insensitive file systems)
due to internal string comparison.
\item
The argument \textit{main} should be fully expanded, it cannot be a macro.
\item
Subdirectories and special characters should be avoided in filenames.
\item
The command |\childdocmain{|\textit{main}|}| must be followed by a whitespace.
It should not be followed immediately by another command
or by a comment mark `|%|'.
This is because the \TeX{} parser reads the token immediately following
the argument of |\childdocmain| and puts it
at the beginning of every child section;
however, a white\-space is ignored.
\end{itemize}

%%%%%%%%%%%%%%%%%%%%%%%%%%%%%%%%%%%%%%%%
\paragraph{Content of Main File.}

It is advisable to place all content in the child files included by |\include|.
Any output contained in the main file will appear in all child documents
unless suppressed manually;
it cannot be suppressed automatically by the |\includeonly| directive
and thus should normally be avoided.
A method to include some content in the main file
by means of conditional processing is described in \secref{sec:conditional}.

%%%%%%%%%%%%%%%%%%%%%%%%%%%%%%%%%%%%%%%%
\paragraph{Page Numbering.}

When only a part of the document is compiled,
the appropriate numbering of pages
(as well as other status parameters)
is determined from the |.aux| files.
The latter contain information from previous passes.
However this information needs to propagate through
all intermediate child documents.
Therefore the page numbering in child documents may well
be inconsistent until the complete document is compiled at least once.

A useful (if unconventional) way to always ensure a consistent
page numbering is to restart the numbering in each child document
and denote the pages by `\textit{child}|.|\textit{page}'
where \textit{child} represents the chapter/section number of the child file.
This can be achieved by the command
|\numberwithin{page}{|\textit{child}|}|
of the \textsf{amsmath} package
where \textit{child} can be |chapter| or |section|
depending on the chosen structuring.
Alternatively, one can modify the macro |\thepage| appropriately
and reset the counter |page| at the start of each child file.

%%%%%%%%%%%%%%%%%%%%%%%%%%%%%%%%%%%%%%%%%%%%%%%%%%%%%%%%%%%%%%%%%%%%%%%%%%%%%%%%
\subsection{Conditional Processing}
\label{sec:conditional}

The package provides a mechanism to compile different versions
of a document. To customise the versions further some conditional processing
can come in handy to distinguish which version is being compiled.
The package provides two macros to describe the compilation context:

%%%%%%%%%%%%%%%%%%%%%%%%%%%%%%%%%%%%%%%%
\DescribeMacro{\ifchilddoc}
The conditional |\ifchilddoc| distinguishes between the compilation of
child documents and the main document:
%
\begin{center}
|\ifchilddoc |\textit{child-code}| |[|\||else |\textit{main-code}]| \||fi|
\end{center}

%%%%%%%%%%%%%%%%%%%%%%%%%%%%%%%%%%%%%%%%
\DescribeMacro{\childdocname}
\DescribeMacro{\childdocjob}
The macro |\childdocname| contains the filename (without extension)
of the main or child file being processed.
Note that |\childdocjob| will always contain the name of the main file.

%%%%%%%%%%%%%%%%%%%%%%%%%%%%%%%%%%%%%%%%
\paragraph{Title Page.}

Conditional processing can be used to include a title or banner page
in the main document when proper precautions are taken.
Importantly, the code in the main file should ensure that the page counter
(as well as other status parameters which are stored in the |.aux| files)
takes the same value after the conditional processing.
Otherwise the page numbers may take divergent values
depending on which part is compiled.

For example, a title page could be declared by:
%
\begin{center}
\begin{tabular}{l}
|\ifchilddoc\||else|\\
|\addtocounter{page}{-1}|\\
\textit{code for title page}\\
|\newpage|\\
|\||fi|
\end{tabular}
\end{center}
%
A banner page for the child documents can be generated by:
%
\begin{center}
\begin{tabular}{l}
|\ifchilddoc|\\
|\addtocounter{page}{-1}|\\
\textit{code for banner page}\\
|\newpage|\\
|\||fi|
\end{tabular}
\end{center}
%
Here one could write a message such as:
\begin{center}
|This is the part \childdocname{} of \childdocjob{}.|
\end{center}

%%%%%%%%%%%%%%%%%%%%%%%%%%%%%%%%%%%%%%%%%%%%%%%%%%%%%%%%%%%%%%%%%%%%%%%%%%%%%%%%
\subsection{Flags}
\label{sec:flags}

The package makes it easy to generate different versions
of the main or child documents.
To this end compilation flags can be defined
and assigned different default values.
They will be particularly useful in conjunction
with the forwarding mechanism described in \secref{sec:forward}.

For example, it may be useful to have a flag |\version|
which can be set to |draft| or |final|.
The document source will contain some conditional code
depending on the value of |\version|.
Suppose further, the flag should default to |final| for the main file
and to |draft| for child files
which is a natural assignment for editing the document.
This is achieved by placing the following code
in the preamble of the main document
(below the |\childdocmain| directive):
%
\begin{center}
\begin{tabular}{l}
|\ifchilddoc|\\
|\providecommand{\version}{draft}|\\
|\||else|\\
|\providecommand{\version}{final}|\\
|\||fi|
\end{tabular}
\end{center}
%
The definition by |\providecommand| makes sure
that previous definitions are not overwritten.
Further statements |\providecommand{\version}{...}|
can thus be added before the above code to override it.

For the main file, one might add a line
(between |\childdocmain| and the above block)
%
\begin{center}
|%\ifchilddoc\||else\providecommand{\version}{draft}\||fi|
\end{center}
%
which can be uncommented to produce a draft version.
Likewise one can add a line to the very top of a child file
(above the |\childdocof{|\textit{main}|}| directive)
%
\begin{center}
|%\providecommand{\version}{final}|
\end{center}
%
which can be uncommented to produce the final version of this child document.

%%%%%%%%%%%%%%%%%%%%%%%%%%%%%%%%%%%%%%%%%%%%%%%%%%%%%%%%%%%%%%%%%%%%%%%%%%%%%%%%
\subsection{Forwarding}
\label{sec:forward}

Different versions of the main or child documents
using compilation flags as described in \secref{sec:flags}
can be (permanently) stored in different files
for convenient compilation, viewing and distribution.
To this end, the package defines a command
to pass on compilation to a different file:

%%%%%%%%%%%%%%%%%%%%%%%%%%%%%%%%%%%%%%%%
\DescribeMacro{\childdocforward}
The command |\childdocforward| redirects processing to
another source file:
%
\begin{center}
\begin{tabular}{l}
|\input{childdoc.def}|\\
|\childdocforward[|\textit{main}|]{|\textit{dest}|}|\\
\end{tabular}
\end{center}
%
The argument \textit{dest} is the destination file
(without extension).
It should be the main file or one of the child files.
Note that further \textsf{childdoc} directives
such as |\childdocof| and |\childdocforward|
in the indicated file will be processed in this form.
The optional argument \textit{main}
passes on directly to the main file \textit{main}
while pretending to compile the child \textit{dest}.
This form behaves as if \textit{dest}
issues |\childdocof{|\textit{main}|}| right away,
and no further \textsf{childdoc} directives will be processed.

%%%%%%%%%%%%%%%%%%%%%%%%%%%%%%%%%%%%%%%%
\DescribeMacro{\...prefix}
In the alternative form |\childdocforwardprefix|,
%
\begin{center}
\begin{tabular}{l}
|\input{childdoc.def}|\\
|\childdocforwardprefix[|\textit{main}|]{|\textit{prefix}|}{|\textit{dest}|}|
\end{tabular}
\end{center}
%
the destination file is determined by a pattern
depending on the current file:
To make this work, the current file must be called
`{\textit{prefix}\hspace{0.2em}\textit{suffix}}'
with \textit{prefix} matching precisely the argument.
Processing is then passed on to the file
`{\textit{dest}\hspace{0.2em}\textit{suffix}}'.
Surely, the same effect is achieved by
directly specifying the
argument `{\textit{dest}\hspace{0.2em}\textit{suffix}}'
in the first form.
However, that requires to set up a different file
for each child. With the alternative form of the command
all these files can have exactly the same content
which simplifies setting them up and maintaining them.

For example, the following file |draft.tex|
with a compilation flag |\version| as described in \secref{sec:flags}
compiles the main document as a draft:
%
\begin{center}
\begin{tabular}{l}
|\def\version{draft}|\\
|\input{childdoc.def}|\\
|\childdocforward{|\textit{main}|}|
\end{tabular}
\end{center}
%
Likewise, the following files |final|\textit{nn}|.tex|
compile the final version of the child document
|child|\textit{nn}|.tex|:
%
\begin{center}
\begin{tabular}{l}
|\def\version{final}|\\
|\input{childdoc.def}|\\
|\childdocforwardprefix{final}{child}|
\end{tabular}
\end{center}
%

Note that when several versions of a main file and/or of each child file
are to be generated, it may be convenient to set up a |Makefile| or
shell script to automatise the process.

%%%%%%%%%%%%%%%%%%%%%%%%%%%%%%%%%%%%%%%%%%%%%%%%%%%%%%%%%%%%%%%%%%%%%%%%%%%%%%%%
\subsection{Command Line Processing}
\label{sec:commandline}

The effect of redirection files can also be achieved by invoking
the \LaTeX{} compiler with a more elaborate command line.
Most conveniently this should be done as part
of a shell script or a |Makefile|.

When using \textsf{childdoc} in the main file, the following
command lines effectively perform a redirection
(note that depending on the shell being used,
backslashes may have to be doubled: `|\|' $\to$ `|\\|'):
%
\begin{center}
|... -jobname "|\textit{target}|" |\\|"|[\textit{flags}]%
|\input{childdoc.def}\childdocforward[|\textit{main}|]{|\textit{dest}|}"|
\end{center}
%
Here \textit{target} is the name of the output file,
\textit{main} is the name of the main file
and \textit{dest} is the name of the main or child file to be processed
(all filenames without extensions).
The optional argument \textit{main} can be omitted
if \textit{main} matches \textit{dest}.
Optionally, compilation \textit{flags} can be defined via |\def| commands.
This command line makes the \TeX{} engine believe
it is compiling the file \textit{target}
whose content is specified as the latter parameter.
The provided code then forwards the processing to
\textit{main} or \textit{dest} as described in \secref{sec:forward}.

%%%%%%%%%%%%%%%%%%%%%%%%%%%%%%%%%%%%%%%%%%%%%%%%%%%%%%%%%%%%%%%%%%%%%%%%%%%%%%%%
\subsection{Include by Input}
\label{sec:input}

Including child documents by |\include| has some restrictions by design.
Most notably, the content of a child document always occupies
its own set of pages; pages cannot be shared between child documents.
Usually, this behaviour makes perfect sense
because each child document contain an essential part of the document.
However, in some situations it may be desirable to compose
a document from a collection of parts
without having mandatory page breaks between then.
For this case, the package
provides a mechanism to include parts
by |\input| which can also be processed individually.
However, by construction this mechanism
requires manual handling of the content to be output.

%%%%%%%%%%%%%%%%%%%%%%%%%%%%%%%%%%%%%%%%
\DescribeMacro{\ifchilddocmanual}
The main file should be prepared as usual, see \secref{sec:include}.
However, the document body must make a distinction
between processing of an individual part and of the main document, e.g.:
%
\begin{center}
\begin{tabular}{l}
|\ifchilddocmanual|\\
|\input{\childdocname}|\\
|\||else|\\
\textit{document body with }|\input{|\textit{part}|}|\\
|\||fi|
\end{tabular}
\end{center}
%
The conditional |\ifchilddocmanual| is true whenever
a part to be included by |\input| is being compiled,
and the name of the part is stored in |\childdocname|.

%%%%%%%%%%%%%%%%%%%%%%%%%%%%%%%%%%%%%%%%
\DescribeMacro{\childdocby}
Each part to be included by |\input| should start with:
%
\begin{center}
\begin{tabular}{l}
|\input{childdoc.def}|\\
|\childdocby{|\textit{main}|}|\\
\end{tabular}
\end{center}
%
The directive |\childdocby| is similar to |\childdocof|
described in \secref{sec:include},
but the subsequent selection of content must be done manually.
To that end, both |\ifchilddoc| and |\ifchilddocmanual|
will be true upon processing of a part,
and the name of the part is stored in |\childdocname|.
Note that |\jobname| will be set to the filename of the current part
so that each part receives an individual |.aux| file
that does not interfere with the |.aux| file(s) of the main document.
This behaviour can be altered by the alternative form
|\childdocby[*]{|\textit{main}|}| (with a non-empty optional argument)
which uses the |.aux| file of the main document
by setting |\jobname| to \textit{main}.

%%%%%%%%%%%%%%%%%%%%%%%%%%%%%%%%%%%%%%%%%%%%%%%%%%%%%%%%%%%%%%%%%%%%%%%%%%%%%%%%
\subsection{Driver Development}
\label{sec:driver}

The \textsf{childdoc} mechanism can also be use for the development
of definition files such as \LaTeX{} styles or classes.
This case differs from the above setup with multiple parts
included by |\include| in that no |\includeonly| should be invoked.
This can be achieved by starting the include file
(before |\ProvidesPackage|) with:
%
\begin{center}
\begin{tabular}{l}
|\input{childdoc.def}|\\
|\childdocforward{|\textit{main}|}|\\
\end{tabular}
\end{center}
%
or alternatively with:
%
\begin{center}
\begin{tabular}{l}
|\input{childdoc.def}|\\
|\childdocby{|\textit{main}|}|\\
\end{tabular}
\end{center}
%
Both forms have slightly different effects as described above.
The main file is prepared as usual, see \secref{sec:include}.

%%%%%%%%%%%%%%%%%%%%%%%%%%%%%%%%%%%%%%%%%%%%%%%%%%%%%%%%%%%%%%%%%%%%%%%%%%%%%%%%
\subsection{Legacy Detection}
\label{sec:detection}

The directive |\childdocmain| in the main file can detect
whether the complete document or merely a child is to be compiled
even without using the directive |\childdocof|.
This method is deprecated because it is less robust
and there is no compelling reason to use it;
it is merely provided for backward compatibility
and it may be removed in future versions.

If the detection mechanism is to be used,
it is mandatory to correctly specify
the filename of the main file as the argument of |\childdocmain|:
%
\begin{center}
\begin{tabular}{l}
|\input{childdoc.def}|\\
|\childdocmain{|\textit{main}|}|\\
\end{tabular}
\end{center}
%
If |\jobname| does not match the argument \textit{main} of |\childdocmain|,
it is assumed that |\jobname| points to the child file to be compiled.
When using |\childdocmain| with the main file specified as argument,
it suffices to start a child file
with just |\input{|\textit{main}|}|
without loading of the package and using |\childdocof|.
If instead all processing is done
with the appropriate \textsf{childdoc} directives,
the argument of \textit{main} of |\childdocmain| can be empty.

An alternative version of the command line processing described
in \secref{sec:commandline} using the detection mechanism reads:
%
\begin{center}
|... -jobname "|\textit{target}|" "|[\textit{flags}]%
[|\def\jobname{|\textit{dest}|}|]|\input{|\textit{main}|}"|
\end{center}

%%%%%%%%%%%%%%%%%%%%%%%%%%%%%%%%%%%%%%%%%%%%%%%%%%%%%%%%%%%%%%%%%%%%%%%%%%%%%%%%
\subsection{Manual Code}
\label{sec:manual}

In case one cannot be certain whether the definitions file |childdoc.def|
is installed on the target \TeX{} distribution
and one prefers not to ship it,
it is conceivable to paste a few relevant commands into the sources.

To that end, drop all statements |\input{childdoc.def}|
and perform the replacements as outlined below.
Instead of |\childdocmain{|\textit{main}|}| add the following code
to the top of the main file:
%
\begin{center}
\begin{tabular}{l}
|\||ifdefined\childdocname\endinput\||fi\newif\ifchilddoc|\\
|\edef\childdocname{\scantokens\expandafter{\jobname\noexpand}}|\\
|\def\childdocmain{|\textit{main}|}\||ifx\childdocmain\childdocname\||else|\\
|\childdoctrue\includeonly{\childdocname}\let\jobname\childdocmain\||fi|\\
\end{tabular}
\end{center}
%
Instead of |\childdocof{|\textit{main}|}| just include the main file
at the top of each child file:
%
\begin{center}
|\input{|\textit{main}|}|
\end{center}
%
A simple redirection |\childdocforward{|\textit{dest}|}| is achieved by:
%
\begin{center}
|\def\jobname{|\textit{dest}|}\input{\jobname}|
\end{center}
%
The redirection with prefix
|\childdocforwardprefix[|\textit{prefix}|]{|\textit{dest}|}|
is accomplished by:
%
\begin{center}
\begin{tabular}{l}
|{\edef\jobname{\scantokens\expandafter{\jobname\noexpand}}|\\
|\def\redirectjob |\textit{prefix}|#1~~~{\gdef\jobname{|\textit{dest}|#1}}|\\
|\expandafter\redirectjob\jobname~~~}\input{\jobname}|
\end{tabular}
\end{center}

In an alternative approach,
child documents can be compiled by a specific command line
without additional code or specific definitions:
%
\begin{center}
|... -jobname "|\textit{target}|" "|[\textit{flags}]%
|\includeonly{|\textit{dest}|}\input{|\textit{main}|}"|
\end{center}
%

%%%%%%%%%%%%%%%%%%%%%%%%%%%%%%%%%%%%%%%%%%%%%%%%%%%%%%%%%%%%%%%%%%%%%%%%%%%%%%%%
%%%%%%%%%%%%%%%%%%%%%%%%%%%%%%%%%%%%%%%%%%%%%%%%%%%%%%%%%%%%%%%%%%%%%%%%%%%%%%%%
\section{Information}

%%%%%%%%%%%%%%%%%%%%%%%%%%%%%%%%%%%%%%%%%%%%%%%%%%%%%%%%%%%%%%%%%%%%%%%%%%%%%%%%
\subsection{Copyright}

Copyright \copyright{} 2017--2018 Niklas Beisert

This work may be distributed and/or modified under the
conditions of the \LaTeX{} Project Public License, either version 1.3
of this license or (at your option) any later version.
The latest version of this license is in
  \url{http://www.latex-project.org/lppl.txt}
and version 1.3 or later is part of all distributions of \LaTeX{}
version 2005/12/01 or later.

This work has the LPPL maintenance status `maintained'.

The Current Maintainer of this work is Niklas Beisert.

This work consists of the files |README.txt|, |childdoc.ins| and |childdoc.dtx|
as well as the derived files |childdoc.def|, |cdocsamp.tex|
with |cdocsch1.tex|, |cdocsch2.tex|, |cdocspt3.tex|, |cdocspt4.tex|,
|cdocsdrf.tex|, |cdocsfn1.tex|, |cdocsfn2.tex|
as well as |childdoc.pdf|.

%%%%%%%%%%%%%%%%%%%%%%%%%%%%%%%%%%%%%%%%%%%%%%%%%%%%%%%%%%%%%%%%%%%%%%%%%%%%%%%%
\subsection{Files and Installation}

The package consists of the files:
%
\begin{center}
\begin{tabular}{ll}
    |README.txt|   & readme file \\
    |childdoc.ins| & installation file \\
    |childdoc.dtx| & source file \\
    |childdoc.def| & definition file \\
    |cdocsamp.tex| & sample main file \\
    |cdocsch1.tex| & sample include file \\
    |cdocsch2.tex| & sample include file \\
    |cdocspt3.tex| & sample part file \\
    |cdocspt4.tex| & sample part file \\
    |cdocsdrf.tex| & sample redirection file \\
    |cdocsfn1.tex| & sample redirection file \\
    |cdocsfn2.tex| & sample redirection file \\
    |childdoc.pdf| & manual
\end{tabular}
\end{center}
%
The distribution consists of the files
|README.txt|, |childdoc.ins| and |childdoc.dtx|.
%
\begin{itemize}
\item
Run (pdf)\LaTeX{} on |childdoc.dtx|
to compile the manual |childdoc.pdf| (this file).
\item
Run \LaTeX{} on |childdoc.ins| to create the definitions file |childdoc.def|
and the sample |cdocsamp.tex| with include files
|cdocsch1.tex|, |cdocsch2.tex|, |cdocspt3.tex|, |cdocspt4.tex|,
|cdocsdrf.tex|, |cdocsfn1.tex|, |cdocsfn2.tex|.
Then copy the file |childdoc.def| to an appropriate directory of your \LaTeX{}
distribution, e.g.\ \textit{texmf-root}|/tex/latex/childdoc|.
\end{itemize}

%%%%%%%%%%%%%%%%%%%%%%%%%%%%%%%%%%%%%%%%%%%%%%%%%%%%%%%%%%%%%%%%%%%%%%%%%%%%%%%%
\subsection{Related CTAN Packages}

There are several other packages which offer a similar functionality:
%
\begin{itemize}
\item
The packages
\href{http://ctan.org/pkg/docmute}{\textsf{docmute}},
\href{http://ctan.org/pkg/includex}{\textsf{includex}} and
\href{http://ctan.org/pkg/standalone}{\textsf{standalone}}
provide commands to include only the document body of
a child file thus allowing both files to be compiled individually.
\item
The packages \href{http://ctan.org/pkg/subdocs}{\textsf{subdocs}}
and \href{http://ctan.org/pkg/subfiles}{\textsf{subfiles}}
provide structures in which the main and child documents can be
encapsulated and allowing them to be compiled individually.
The inclusion mechanism is different from the conventional |\include|.
\item
The package \href{http://ctan.org/pkg/combine}{\textsf{combine}}
is an elaborate solution to combine several documents into one.
\end{itemize}
%
See also the CTAN topic \href{http://ctan.org/topic/subdocs}{\textsf{subdocs}}
for further related packages.
The present package differs from the above solutions in that
a document structure constructed with the conventional |\include| mechanism
just needs two extra commands at the top of every file
such that all constituent files can be compiled individually.

%%%%%%%%%%%%%%%%%%%%%%%%%%%%%%%%%%%%%%%%%%%%%%%%%%%%%%%%%%%%%%%%%%%%%%%%%%%%%%%%
%\subsection{Feature Suggestions}
%
%The following is a list of features which may be useful for future
%versions of this package:
%%
%\begin{itemize}
%\item
%\ldots
%\end{itemize}

%%%%%%%%%%%%%%%%%%%%%%%%%%%%%%%%%%%%%%%%%%%%%%%%%%%%%%%%%%%%%%%%%%%%%%%%%%%%%%%%
\subsection{Revision History}

%%%%%%%%%%%%%%%%%%%%%%%%%%%%%%%%%%%%%%%%
\paragraph{v2.0:} 2018/12/30

\begin{itemize}
\item
immediate forward processing
\item
added |\childdocby| mechanism
\item
manual restructured
\end{itemize}

%%%%%%%%%%%%%%%%%%%%%%%%%%%%%%%%%%%%%%%%
\paragraph{v1.6:} 2018/01/17

\begin{itemize}
\item
application for development of include files
\item
corrections to manual
\end{itemize}

%%%%%%%%%%%%%%%%%%%%%%%%%%%%%%%%%%%%%%%%
\paragraph{v1.5:} 2017/05/21

\begin{itemize}
\item
more complete structuring introduced
\item
|\childdocof| introduced
\item
|\childdoc| renamed to |\childdocmain|
\item
|\childredirect| renamed to |\childdocforward| and |\childdocforwardprefix|
and functionality expanded
\end{itemize}

%%%%%%%%%%%%%%%%%%%%%%%%%%%%%%%%%%%%%%%%
\paragraph{v1.0:} 2017/04/27

\begin{itemize}
\item
manual and install package
\item
first version published on CTAN
\end{itemize}

%%%%%%%%%%%%%%%%%%%%%%%%%%%%%%%%%%%%%%%%
\paragraph{v0.6:} 2017/04/26

\begin{itemize}
\item
redirection mechanism added
\end{itemize}

%%%%%%%%%%%%%%%%%%%%%%%%%%%%%%%%%%%%%%%%
\paragraph{v0.5:} 2017/04/26

\begin{itemize}
\item
functionality in definition file
\end{itemize}


%%%%%%%%%%%%%%%%%%%%%%%%%%%%%%%%%%%%%%%%%%%%%%%%%%%%%%%%%%%%%%%%%%%%%%%%%%%%%%%%
%%%%%%%%%%%%%%%%%%%%%%%%%%%%%%%%%%%%%%%%%%%%%%%%%%%%%%%%%%%%%%%%%%%%%%%%%%%%%%%%
%%%%%%%%%%%%%%%%%%%%%%%%%%%%%%%%%%%%%%%%%%%%%%%%%%%%%%%%%%%%%%%%%%%%%%%%%%%%%%%%
\appendix

\settowidth\MacroIndent{\rmfamily\scriptsize 000\ }

 \DocInput{childdoc.dtx}

\end{document}
%</driver>
% \fi
%
% %%%%%%%%%%%%%%%%%%%%%%%%%%%%%%%%%%%%%%%%%%%%%%%%%%%%%%%%%%%%%%%%%%%%%%%%%%%%%%
% %%%%%%%%%%%%%%%%%%%%%%%%%%%%%%%%%%%%%%%%%%%%%%%%%%%%%%%%%%%%%%%%%%%%%%%%%%%%%%
% \section{Sample}
%\iffalse
%<*samplemain>
%\fi
%
% The following presents a sample document
% with two chapters, two parts, a title page,
% a compile flag as well as three forwarding files to set the flag.
% It consists of eight |.tex| files:
% \begin{center}
% \begin{tabular}{ll}
% |cdocsamp.tex|&main file\\
% |cdocsch1.tex|&include file for chapter 1\\
% |cdocsch2.tex|&include file for chapter 2\\
% |cdocspt3.tex|&include file for part 3\\
% |cdocspt4.tex|&include file for part 4\\
% |cdocsdrf.tex|&forwarding file for main file in draft mode\\
% |cdocsfi1.tex|&forwarding file for final version of chapter 1\\
% |cdocsfi2.tex|&forwarding file for final version of chapter 2\\
% \end{tabular}
% \end{center}
% Each of the eight files can be compiled directly by the \LaTeX{} compiler.
%
% %%%%%%%%%%%%%%%%%%%%%%%%%%%%%%%%%%%%%%
% \paragraph{Main File.}
%
% The main file is called |cdocsamp.tex|.
%
% Load the \textsf{childdoc} definitions and
% declare the filename for the main document:
%    \begin{macrocode}
\input{childdoc.def}
\childdocmain{}
%    \end{macrocode}

% Optional override for |\version| flag:
%    \begin{macrocode}
%%\ifchilddoc\else\providecommand{\version}{draft}\fi
%    \end{macrocode}

% Define the default values for the |\version| flag
% (|final| for the main file and |draft| for childs):
%    \begin{macrocode}
\ifchilddoc
\providecommand{\version}{draft}
\else
\providecommand{\version}{final}
\fi
%    \end{macrocode}

% Load the standard document class:
%    \begin{macrocode}
\documentclass[12pt]{article}
%    \end{macrocode}

% Start the document body:
%    \begin{macrocode}
\begin{document}
%    \end{macrocode}

% Declare a title page.
% Print title, part of document being processed and version flag:
%    \begin{macrocode}
\addtocounter{page}{-1}
\begin{center}
{\LARGE\bfseries{}childdoc example\par}
\vspace{1cm}
\ifchilddoc
\ifchilddocmanual part\else chapter\fi:
`\childdocname' of `\childdocjob'\par
\else
main document: `\childdocjob'\par
\fi
version: \version\par
\end{center}
\newpage
%    \end{macrocode}

% Manually include selected file,
% otherwise process as usual:
%    \begin{macrocode}
\ifchilddocmanual
\section*{part `\childdocname'}
\input{\childdocname}
\else
%    \end{macrocode}

% Include the two chapters:
%    \begin{macrocode}
\include{cdocsch1}
\include{cdocsch2}
%    \end{macrocode}

% Include the two parts unless only chapters should be displayed:
%    \begin{macrocode}
\ifchilddoc\else
\section{part three}
\input{cdocspt3}
\section{part four}
\input{cdocspt4}
\fi
%    \end{macrocode}

% Process as usual until here:
%    \begin{macrocode}
\fi
%    \end{macrocode}

% End of document body:
%    \begin{macrocode}
\end{document}
%    \end{macrocode}
%\iffalse
%</samplemain>
%\fi
%
% %%%%%%%%%%%%%%%%%%%%%%%%%%%%%%%%%%%%%%
% \paragraph{Chapter Include Files.}
%
% The include files are called |cdocsch1.tex| and |cdocsch2.tex|.
%
%\iffalse
%<*samplechap1|samplechap2>
%\fi

% Optional override for |\version| flag:
%    \begin{macrocode}
%%\providecommand{\version}{final}
%    \end{macrocode}

% Include the main document:
%    \begin{macrocode}
\input{childdoc.def}
\childdocof{cdocsamp}
%    \end{macrocode}

%\iffalse
%</samplechap1|samplechap2>
%\fi
%
%\iffalse
%<*samplechap1>
%\fi
% Some text for chapter 1:
%    \begin{macrocode}
\section{one}
some text in chapter one
%    \end{macrocode}

%\iffalse
%</samplechap1>
%\fi
% Some text for chapter 2:
%\iffalse
%<*samplechap2>
%\fi
%    \begin{macrocode}
\section{two}
more text in chapter two
%    \end{macrocode}

%\iffalse
%</samplechap2>
%\fi
%
% %%%%%%%%%%%%%%%%%%%%%%%%%%%%%%%%%%%%%%
% \paragraph{Part Include Files.}
%
% The include files are called |cdocspt3.tex| and |cdocspt4.tex|.
%
%\iffalse
%<*samplepart3|samplepart4>
%\fi

% Optional override for |\version| flag:
%    \begin{macrocode}
%%\providecommand{\version}{final}
%    \end{macrocode}

% Include the main document:
%    \begin{macrocode}
\input{childdoc.def}
\childdocby{cdocsamp}
%    \end{macrocode}

%\iffalse
%</samplepart3|samplepart4>
%\fi
%
%\iffalse
%<*samplepart3>
%\fi
% Some text for part 3:
%    \begin{macrocode}
some text in part three
%    \end{macrocode}

%\iffalse
%</samplepart3>
%\fi
% Some text for part 4:
%\iffalse
%<*samplepart4>
%\fi
%    \begin{macrocode}
more text in part four
%    \end{macrocode}

%\iffalse
%</samplepart4>
%\fi
%
% %%%%%%%%%%%%%%%%%%%%%%%%%%%%%%%%%%%%%%
% \paragraph{Forwarding for a Complete Draft.}
%
% The following forwarding file |cdocsdrf.tex|
% compiles the main document in draft mode:
%\iffalse
%<*sampledraft>
%\fi
%    \begin{macrocode}
\def\version{draft}
\input{childdoc.def}
\childdocforward{cdocsamp}
%    \end{macrocode}

%\iffalse
%</sampledraft>
%\fi
%
% %%%%%%%%%%%%%%%%%%%%%%%%%%%%%%%%%%%%%%
% \paragraph{Forwarding for Final Version of the Chapters.}
%
% The following forwarding files |cdocsfn1.tex| and |cdocsfn2.tex|
% (with identical content)
% compile the final versions of the child documents
% |cdocsch1.tex| and |cdocsch2.tex|, respectively:
%\iffalse
%<*samplefinal>
%\fi
%    \begin{macrocode}
\def\version{final}
\input{childdoc.def}
\childdocforwardprefix[cdocsamp]{cdocsfn}{cdocsch}
%    \end{macrocode}

%\iffalse
%</samplefinal>
%\fi
%
% %%%%%%%%%%%%%%%%%%%%%%%%%%%%%%%%%%%%%%
% \paragraph{Command Line Processing.}
%
% The following three command lines generate the output files
% |cdocscld|, |cdocscl1| and |cdocscl2|
% which should be identical to
% |cdocsdrf|, |cdocsch1| and |cdocsfn2|, respectively:
% \begin{center}
% \begin{tabular}{l}
% |latex -jobname cdocscld \|\\
% |  "\def\version{draft}\input{childdoc.def}\childdocforward{cdocsamp}"|\\
% |latex -jobname cdocscl1 \|\\
% |  "\input{childdoc.def}\childdocforward[cdocsamp]{cdocsch1}"|\\
% |latex -jobname cdocscl2 \|\\
% |  "\def\version{final}\input{childdoc.def}\childdocforward{cdocsch2}"|
% \end{tabular}
% \end{center}
% Note that the trailing backslash on each first line
% merely continues the input to the second line
% (for convenient cut ant paste).
% Furthermore, the command |latex| can be replaced by any
% of its alternative versions such as |pdflatex|.
%
% %%%%%%%%%%%%%%%%%%%%%%%%%%%%%%%%%%%%%%%%%%%%%%%%%%%%%%%%%%%%%%%%%%%%%%%%%%%%%%
% %%%%%%%%%%%%%%%%%%%%%%%%%%%%%%%%%%%%%%%%%%%%%%%%%%%%%%%%%%%%%%%%%%%%%%%%%%%%%%
% \section{Implementation}
%\iffalse
%<*package>
%\fi
%
% This section describes the definitions file |childdoc.def|.

% The definitions cannot be loaded using |\usepackage| or |\RequirePackage|
% which has a mechanism to prevent loading a style file more than once.
% When loading the definitions by means of |\input|
% multiple instances have to be prevented manually:
%\iffalse
%This code needs to be before the `\ProvidesFile' directive
%which is defined at the beginning of this file.
%Therefore it is also placed there and commented out here.
%</package>
%<*discard>
%\fi
%    \begin{macrocode}
\ifdefined\childdocmain\endinput\fi
%    \end{macrocode}
%\iffalse
%</discard>
%<*package>
%\fi
%
% \macro{\ifchilddoc}
% \macro{\ifchilddocmanual}
% The conditional |\ifchilddoc| tells whether a
% child (true) or main (false) document is being compiled.
% The conditional |\ifchilddocmanual| tells whether
% the |\includeonly| mechanism is used (false) or
% the selection of child files must be performed manually (true).
% The definitions initialise to false:
%    \begin{macrocode}
\newif\ifchilddoc
\newif\ifchilddocmanual
%    \end{macrocode}

% \macro{\childdocname}
% \macro{\childdocjob}
% The macro |\childdocname| stores the name of the main document
% to be compiled. The macro |\childdocjob| stores the name of
% the document on which the \LaTeX{} compiler was originally invoked.
% The content of |\jobname| cannot be compared
% to filenames specified in the source due to different catcodes.
% The following code rescans |\jobname|, stores the result
% in |\childdocname| and saves a copy in |\childdocjob|:
%    \begin{macrocode}
\edef\childdocname{\scantokens\expandafter{\jobname\noexpand}}
\let\childdocjob\childdocname
%    \end{macrocode}

% \macro{\childdocdisable}
% The macro |\childdocdisable| prevents the main file
% from being processed more than once.
% At this stage, the main document command |\childdocmain|
% is assumed to be called once again where it should do nothing.
% Any subsequent call to it should prevent
% a secondary processing of the main document
% It overwrites the forwarding commands
% |\childdocof| and |\childdocforward|
% with empty macros to prevent further inclusions of the main document:
%    \begin{macrocode}
\newcommand{\childdocdisable}
{
  \renewcommand{\childdocmain}[1]{\renewcommand{\childdocmain}[1]{\endinput}}
  \renewcommand{\childdocof}[1]{}
  \renewcommand{\childdocby}[2][]{}
  \renewcommand{\childdocforward}[2][]{}
  \renewcommand{\childdocdisable}{}
}
%    \end{macrocode}

% \macro{\childdocmain}
% The macro |\childdocmain| is to be called at the top of the main file
% with nothing or the main filename (without extension) as argument.
% First, it breaks loops.
% If the argument is not empty and does not match |\childdocname|
% (which is set by the first inclusion of |childdoc.def|),
% |\ifchilddoc| is set to true, |\includeonly| is applied to the child file
% and |\jobname| is set to the main file
% (for proper handling of |.aux| files):
%    \begin{macrocode}
\newcommand{\childdocmain}[1]
{
  \childdocdisable\childdocmain{}
  \if?#1?\else
    \begingroup
      \def\childdoctmp{#1}
      \ifx\childdoctmp\childdocname
        \def\childdoctmp{}
      \else
        \def\childdoctmp
        {
          \childdoctrue
          \includeonly{\childdocname}
          \def\childdocjob{#1}
          \def\jobname{#1}
        }
      \fi
      \expandafter
    \endgroup
    \childdoctmp
  \fi
}
%    \end{macrocode}

% \macro{\childdocof}
% The command |\childdocof| redirects
% compilation to the main file |#1|.
%    \begin{macrocode}
\newcommand{\childdocof}[1]
{
  \childdocdisable
  \childdoctrue
  \includeonly{\childdocname}
  \def\jobname{#1}
  \def\childdocjob{#1}
  \input{#1}
}
%    \end{macrocode}

% \macro{\childdocby}
% The command |\childdocby| ....
%    \begin{macrocode}
\newcommand{\childdocby}[2][]
{
  \childdocdisable
  \childdoctrue
  \childdocmanualtrue
  \if?#1?\else
    \def\jobname{#2}
  \fi
  \def\childdocjob{#2}
  \input{#2}
  \endinput
}
%    \end{macrocode}

% \macro{\childdocforward}
% The command |\childdocforward| redirects
% compilation to the main file or
% (if the optional argument is given) a child file.
% Parameters are set as if the main file
% or a child file starting with |\childdocof| was compiled.
% Then compilation is handed over to the main file:
%    \begin{macrocode}
\newcommand{\childdocforward}[2][]
{
  \begingroup
    \if?#1?
      \def\childdoctmp
      {
        \def\childdocname{#2}
        \def\childdocjob{#2}
        \def\jobname{#2}
        \input{#2}
        \endinput
      }
    \else
      \def\childdoctmp
      {
        \childdocdisable
        \def\childdocname{#2}
        \childdoctrue
        \includeonly{#2}
        \def\childdocjob{#1}
        \def\jobname{#1}
        \input{#1}
        \endinput
      }
    \fi
    \expandafter
  \endgroup
  \childdoctmp
}
%    \end{macrocode}

% \macro{\childdocforwardprefix}
% The command |\childdocforwardprefix| redirects
% compilation to the main or a child file by means of a pattern.
% The prefix |#1| in the current filename is replaced by |#2|
% and the suffix of the current filename is kept
% (it is assumed that the filename does not contain the substring `|~~~|'
% which is used as a delimiter).
% Compilation is handed over to the new file by |\childdocforward|:
%    \begin{macrocode}
\newcommand{\childdocforwardprefix}[3][]
{
  \begingroup
    \def\childdocextract #2##1~~~{\def\childdoctmp{\childdocforward[#1]{#3##1}}}
    \expandafter\childdocextract\childdocname~~~
    \expandafter
  \endgroup
  \childdoctmp
}
%    \end{macrocode}

% \macro{\childdoc}
% The deprecated macro |\childdoc| is a legacy version of |\childdocmain|:
%    \begin{macrocode}
\newcommand{\childdoc}{\childdocmain}
%    \end{macrocode}

% \macro{\childdocredirect}
% The deprecated macro |\childdocredirect| is a legacy version
% of |\childdocforward| and |\childdocforwardprefix|:
%    \begin{macrocode}
\newcommand{\childdocredirect}[2][]
{
  \begingroup
    \if?#1?
      \def\childdoctmp{\childdocforward{#2}}
    \else
      \def\childdoctmp{\childdocforwardprefix{#1}{#2}}
    \fi
    \expandafter
  \endgroup
  \childdoctmp
}
%    \end{macrocode}

%\iffalse
%</package>
%\fi
%
\endinput
|\\
|\childdocforwardprefix{final}{child}|
\end{tabular}
\end{center}
%

Note that when several versions of a main file and/or of each child file
are to be generated, it may be convenient to set up a |Makefile| or
shell script to automatise the process.

%%%%%%%%%%%%%%%%%%%%%%%%%%%%%%%%%%%%%%%%%%%%%%%%%%%%%%%%%%%%%%%%%%%%%%%%%%%%%%%%
\subsection{Command Line Processing}
\label{sec:commandline}

The effect of redirection files can also be achieved by invoking
the \LaTeX{} compiler with a more elaborate command line.
Most conveniently this should be done as part
of a shell script or a |Makefile|.

When using \textsf{childdoc} in the main file, the following
command lines effectively perform a redirection
(note that depending on the shell being used,
backslashes may have to be doubled: `|\|' $\to$ `|\\|'):
%
\begin{center}
|... -jobname "|\textit{target}|" |\\|"|[\textit{flags}]%
|% \iffalse
%
% childdoc.dtx Copyright (C) 2017-2018 Niklas Beisert
%
% This work may be distributed and/or modified under the
% conditions of the LaTeX Project Public License, either version 1.3
% of this license or (at your option) any later version.
% The latest version of this license is in
%   http://www.latex-project.org/lppl.txt
% and version 1.3 or later is part of all distributions of LaTeX
% version 2005/12/01 or later.
%
% This work has the LPPL maintenance status `maintained'.
%
% The Current Maintainer of this work is Niklas Beisert.
%
% This work consists of the files childdoc.dtx and childdoc.ins
% and the derived files childdoc.def and cdocsamp.tex with
% cdocsch1.tex, cdocsch2.tex, cdocsdrf.tex, cdocsfn1.tex, cdocsfn2.tex.
%
%<package>\ifdefined\childdocmain\endinput\fi
%<package>\ProvidesFile{childdoc.def}[2018/12/30 v2.0 child document driver]
%<samplemain>\ProvidesFile{cdocsamp.tex}[2018/12/30 v2.0 sample for childdoc]
%<*driver>
%\ProvidesFile{childdoc.drv}[2018/12/30 v2.0 childdoc reference manual file]
\PassOptionsToClass{10pt,a4paper}{article}
\documentclass{ltxdoc}

\usepackage[margin=35mm]{geometry}
\usepackage{hyperref}
\usepackage{hyperxmp}
\usepackage[usenames]{color}

\hypersetup{colorlinks=true}
\hypersetup{pdfstartview=FitH}
\hypersetup{pdfpagemode=UseNone}
\hypersetup{pdfsource={}}
\hypersetup{pdflang={en-UK}}
\hypersetup{pdfcopyright={Copyright 2017-2018 Niklas Beisert.
  This work may be distributed and/or modified under the
  conditions of the LaTeX Project Public License, either version 1.3
  of this license or (at your option) any later version.}}
\hypersetup{pdflicenseurl={http://www.latex-project.org/lppl.txt}}
\hypersetup{pdfcontactaddress={ETH Zurich, ITP, HIT K,
  Wolfgang-Pauli-Strasse 27}}
\hypersetup{pdfcontactpostcode={8093}}
\hypersetup{pdfcontactcity={Zurich}}
\hypersetup{pdfcontactcountry={Switzerland}}
\hypersetup{pdfcontactemail={nbeisert@itp.phys.ethz.ch}}
\hypersetup{pdfcontacturl={http://people.phys.ethz.ch/\xmptilde nbeisert/}}

\newcommand{\secref}[1]{\hyperref[#1]{section \ref*{#1}}}

\parskip1ex
\parindent0pt
\let\olditemize\itemize
\def\itemize{\olditemize\parskip0pt}

\begin{document}

\title{The \textsf{childdoc} Package}
\hypersetup{pdftitle={The childdoc Package}}
\author{Niklas Beisert\\[2ex]
  Institut f\"ur Theoretische Physik\\
  Eidgen\"ossische Technische Hochschule Z\"urich\\
  Wolfgang-Pauli-Strasse 27, 8093 Z\"urich, Switzerland\\[1ex]
  \href{mailto:nbeisert@itp.phys.ethz.ch}
  {\texttt{nbeisert@itp.phys.ethz.ch}}}
\hypersetup{pdfauthor={Niklas Beisert}}
\hypersetup{pdfsubject={Manual for the LaTeX2e Package childdoc}}
\date{30 December 2018, \textsf{v2.0}}
\maketitle

\begin{abstract}\noindent
\textsf{childdoc} is a \LaTeXe{} package
that enables the direct compilation
of document sections included by |\include|
to individual files.
\end{abstract}

\begingroup
\parskip0ex
\tableofcontents
\endgroup

%%%%%%%%%%%%%%%%%%%%%%%%%%%%%%%%%%%%%%%%%%%%%%%%%%%%%%%%%%%%%%%%%%%%%%%%%%%%%%%%
%%%%%%%%%%%%%%%%%%%%%%%%%%%%%%%%%%%%%%%%%%%%%%%%%%%%%%%%%%%%%%%%%%%%%%%%%%%%%%%%
\section{Introduction}

\LaTeX{} provides a mechanism to structure a large document (such as a book)
into a main file and several child files (containing the chapters)
using the |\include| command.
This mechanism is beneficial for documents
which span hundreds of pages in order to
make the source file(s) more manageable.
Moreover, compilation can be restricted to
selected child files by means of the |\includeonly| command.
The latter feature can be used to reduce the compilation time while editing
(this was significantly more useful in the earlier days of \LaTeX{})
or to generate a smaller document which is easier to navigate.
Another application of |\includeonly| is to generate
documents consisting of selected parts of the complete document.

However, there are a few drawbacks of the plain |\include| mechanism:
\begin{itemize}
\item
The child files cannot be compiled on their own,
they can only be compiled via the main file.
A naive editing environment
(such as a text editor with an option
to have the current file processed by \LaTeX)
may require one to switch to the main file before compiling;
attempting to compile the child file produces errors.
\item
The main file must be modified (each time)
to adjust the |\includeonly| command
to the present needs. This easily leaves the main file in a messy state.
\item
The generated document will always carry the filename
of the main document. This is inconvenient if
several child files are to be compiled and
to be kept for distribution.
\end{itemize}

The present package provides a simple interface
to make child files individually compilable by \LaTeX{}.
Compiling a child file then has the same effect as compiling
the main file with an |\includeonly| command
to select the appropriate child.
Moreover the generated document will carry the name of the child
rather than the main file.
This resolves all three above issues.

This feature is meant to make the editing of books,
thesis documents and lecture notes somewhat more convenient.
However, the package can also be used efficiently for
composing a series of documents (such as exercise sheets)
which are typically distributed individually.
It then assists the author in generating the individual documents
(potentially in different versions)
as well as a document containing the collected series.
Another application is in developing style files
or other kinds of included material
where compilation of the style file could redirect
to a sample or test file.

%%%%%%%%%%%%%%%%%%%%%%%%%%%%%%%%%%%%%%%%%%%%%%%%%%%%%%%%%%%%%%%%%%%%%%%%%%%%%%%%
%%%%%%%%%%%%%%%%%%%%%%%%%%%%%%%%%%%%%%%%%%%%%%%%%%%%%%%%%%%%%%%%%%%%%%%%%%%%%%%%
\section{Usage}

First of all, the package \textsf{childdoc} is \emph{not} a standard
\LaTeXe{} |.sty| style file! Therefore it needs to be invoked in
a non-standard way.

%%%%%%%%%%%%%%%%%%%%%%%%%%%%%%%%%%%%%%%%%%%%%%%%%%%%%%%%%%%%%%%%%%%%%%%%%%%%%%%%
\subsection{Included Files}
\label{sec:include}

%%%%%%%%%%%%%%%%%%%%%%%%%%%%%%%%%%%%%%%%
\DescribeMacro{\childdocmain}
To use the package, add the commands
\begin{center}
\begin{tabular}{l}
|\input{childdoc.def}|\\
|\childdocmain{}|\\
\end{tabular}
\end{center}
at the very top of the main \LaTeX{} file,
in particular \emph{before} the |\documentclass| statement!
The argument of |\childdocmain| should be left empty
(but it must be present).

%%%%%%%%%%%%%%%%%%%%%%%%%%%%%%%%%%%%%%%%
\DescribeMacro{\childdocof}
Furthermore, add the commands
\begin{center}
\begin{tabular}{l}
|\input{childdoc.def}|\\
|\childdocof{|\textit{main}|}|\\
\end{tabular}
\end{center}
at the top of every child file \textit{child}
which is included by |\include{|\textit{child}|}|
from within the main file
(or at least for those files to be compiled individually).
The argument \textit{main} must be the filename of the main file.

There are a couple of
considerations in setting up the main and child documents:

%%%%%%%%%%%%%%%%%%%%%%%%%%%%%%%%%%%%%%%%
\paragraph{Restrictions.}

Please note the following restrictions:
\begin{itemize}
\item
|\childdocmain| must be called with one argument \textit{main}
to ensure compatibility with earlier version of the package.
It must either be empty (|\childdocmain{}|)
or precisely match the filename of the main file in which it is specified.
See \secref{sec:detection} for further information.
\item
The filename \textit{main} must be specified without the |.tex| extension.
\item
The filename \textit{main} is case sensitive
(even in case-insensitive file systems)
due to internal string comparison.
\item
The argument \textit{main} should be fully expanded, it cannot be a macro.
\item
Subdirectories and special characters should be avoided in filenames.
\item
The command |\childdocmain{|\textit{main}|}| must be followed by a whitespace.
It should not be followed immediately by another command
or by a comment mark `|%|'.
This is because the \TeX{} parser reads the token immediately following
the argument of |\childdocmain| and puts it
at the beginning of every child section;
however, a white\-space is ignored.
\end{itemize}

%%%%%%%%%%%%%%%%%%%%%%%%%%%%%%%%%%%%%%%%
\paragraph{Content of Main File.}

It is advisable to place all content in the child files included by |\include|.
Any output contained in the main file will appear in all child documents
unless suppressed manually;
it cannot be suppressed automatically by the |\includeonly| directive
and thus should normally be avoided.
A method to include some content in the main file
by means of conditional processing is described in \secref{sec:conditional}.

%%%%%%%%%%%%%%%%%%%%%%%%%%%%%%%%%%%%%%%%
\paragraph{Page Numbering.}

When only a part of the document is compiled,
the appropriate numbering of pages
(as well as other status parameters)
is determined from the |.aux| files.
The latter contain information from previous passes.
However this information needs to propagate through
all intermediate child documents.
Therefore the page numbering in child documents may well
be inconsistent until the complete document is compiled at least once.

A useful (if unconventional) way to always ensure a consistent
page numbering is to restart the numbering in each child document
and denote the pages by `\textit{child}|.|\textit{page}'
where \textit{child} represents the chapter/section number of the child file.
This can be achieved by the command
|\numberwithin{page}{|\textit{child}|}|
of the \textsf{amsmath} package
where \textit{child} can be |chapter| or |section|
depending on the chosen structuring.
Alternatively, one can modify the macro |\thepage| appropriately
and reset the counter |page| at the start of each child file.

%%%%%%%%%%%%%%%%%%%%%%%%%%%%%%%%%%%%%%%%%%%%%%%%%%%%%%%%%%%%%%%%%%%%%%%%%%%%%%%%
\subsection{Conditional Processing}
\label{sec:conditional}

The package provides a mechanism to compile different versions
of a document. To customise the versions further some conditional processing
can come in handy to distinguish which version is being compiled.
The package provides two macros to describe the compilation context:

%%%%%%%%%%%%%%%%%%%%%%%%%%%%%%%%%%%%%%%%
\DescribeMacro{\ifchilddoc}
The conditional |\ifchilddoc| distinguishes between the compilation of
child documents and the main document:
%
\begin{center}
|\ifchilddoc |\textit{child-code}| |[|\||else |\textit{main-code}]| \||fi|
\end{center}

%%%%%%%%%%%%%%%%%%%%%%%%%%%%%%%%%%%%%%%%
\DescribeMacro{\childdocname}
\DescribeMacro{\childdocjob}
The macro |\childdocname| contains the filename (without extension)
of the main or child file being processed.
Note that |\childdocjob| will always contain the name of the main file.

%%%%%%%%%%%%%%%%%%%%%%%%%%%%%%%%%%%%%%%%
\paragraph{Title Page.}

Conditional processing can be used to include a title or banner page
in the main document when proper precautions are taken.
Importantly, the code in the main file should ensure that the page counter
(as well as other status parameters which are stored in the |.aux| files)
takes the same value after the conditional processing.
Otherwise the page numbers may take divergent values
depending on which part is compiled.

For example, a title page could be declared by:
%
\begin{center}
\begin{tabular}{l}
|\ifchilddoc\||else|\\
|\addtocounter{page}{-1}|\\
\textit{code for title page}\\
|\newpage|\\
|\||fi|
\end{tabular}
\end{center}
%
A banner page for the child documents can be generated by:
%
\begin{center}
\begin{tabular}{l}
|\ifchilddoc|\\
|\addtocounter{page}{-1}|\\
\textit{code for banner page}\\
|\newpage|\\
|\||fi|
\end{tabular}
\end{center}
%
Here one could write a message such as:
\begin{center}
|This is the part \childdocname{} of \childdocjob{}.|
\end{center}

%%%%%%%%%%%%%%%%%%%%%%%%%%%%%%%%%%%%%%%%%%%%%%%%%%%%%%%%%%%%%%%%%%%%%%%%%%%%%%%%
\subsection{Flags}
\label{sec:flags}

The package makes it easy to generate different versions
of the main or child documents.
To this end compilation flags can be defined
and assigned different default values.
They will be particularly useful in conjunction
with the forwarding mechanism described in \secref{sec:forward}.

For example, it may be useful to have a flag |\version|
which can be set to |draft| or |final|.
The document source will contain some conditional code
depending on the value of |\version|.
Suppose further, the flag should default to |final| for the main file
and to |draft| for child files
which is a natural assignment for editing the document.
This is achieved by placing the following code
in the preamble of the main document
(below the |\childdocmain| directive):
%
\begin{center}
\begin{tabular}{l}
|\ifchilddoc|\\
|\providecommand{\version}{draft}|\\
|\||else|\\
|\providecommand{\version}{final}|\\
|\||fi|
\end{tabular}
\end{center}
%
The definition by |\providecommand| makes sure
that previous definitions are not overwritten.
Further statements |\providecommand{\version}{...}|
can thus be added before the above code to override it.

For the main file, one might add a line
(between |\childdocmain| and the above block)
%
\begin{center}
|%\ifchilddoc\||else\providecommand{\version}{draft}\||fi|
\end{center}
%
which can be uncommented to produce a draft version.
Likewise one can add a line to the very top of a child file
(above the |\childdocof{|\textit{main}|}| directive)
%
\begin{center}
|%\providecommand{\version}{final}|
\end{center}
%
which can be uncommented to produce the final version of this child document.

%%%%%%%%%%%%%%%%%%%%%%%%%%%%%%%%%%%%%%%%%%%%%%%%%%%%%%%%%%%%%%%%%%%%%%%%%%%%%%%%
\subsection{Forwarding}
\label{sec:forward}

Different versions of the main or child documents
using compilation flags as described in \secref{sec:flags}
can be (permanently) stored in different files
for convenient compilation, viewing and distribution.
To this end, the package defines a command
to pass on compilation to a different file:

%%%%%%%%%%%%%%%%%%%%%%%%%%%%%%%%%%%%%%%%
\DescribeMacro{\childdocforward}
The command |\childdocforward| redirects processing to
another source file:
%
\begin{center}
\begin{tabular}{l}
|\input{childdoc.def}|\\
|\childdocforward[|\textit{main}|]{|\textit{dest}|}|\\
\end{tabular}
\end{center}
%
The argument \textit{dest} is the destination file
(without extension).
It should be the main file or one of the child files.
Note that further \textsf{childdoc} directives
such as |\childdocof| and |\childdocforward|
in the indicated file will be processed in this form.
The optional argument \textit{main}
passes on directly to the main file \textit{main}
while pretending to compile the child \textit{dest}.
This form behaves as if \textit{dest}
issues |\childdocof{|\textit{main}|}| right away,
and no further \textsf{childdoc} directives will be processed.

%%%%%%%%%%%%%%%%%%%%%%%%%%%%%%%%%%%%%%%%
\DescribeMacro{\...prefix}
In the alternative form |\childdocforwardprefix|,
%
\begin{center}
\begin{tabular}{l}
|\input{childdoc.def}|\\
|\childdocforwardprefix[|\textit{main}|]{|\textit{prefix}|}{|\textit{dest}|}|
\end{tabular}
\end{center}
%
the destination file is determined by a pattern
depending on the current file:
To make this work, the current file must be called
`{\textit{prefix}\hspace{0.2em}\textit{suffix}}'
with \textit{prefix} matching precisely the argument.
Processing is then passed on to the file
`{\textit{dest}\hspace{0.2em}\textit{suffix}}'.
Surely, the same effect is achieved by
directly specifying the
argument `{\textit{dest}\hspace{0.2em}\textit{suffix}}'
in the first form.
However, that requires to set up a different file
for each child. With the alternative form of the command
all these files can have exactly the same content
which simplifies setting them up and maintaining them.

For example, the following file |draft.tex|
with a compilation flag |\version| as described in \secref{sec:flags}
compiles the main document as a draft:
%
\begin{center}
\begin{tabular}{l}
|\def\version{draft}|\\
|\input{childdoc.def}|\\
|\childdocforward{|\textit{main}|}|
\end{tabular}
\end{center}
%
Likewise, the following files |final|\textit{nn}|.tex|
compile the final version of the child document
|child|\textit{nn}|.tex|:
%
\begin{center}
\begin{tabular}{l}
|\def\version{final}|\\
|\input{childdoc.def}|\\
|\childdocforwardprefix{final}{child}|
\end{tabular}
\end{center}
%

Note that when several versions of a main file and/or of each child file
are to be generated, it may be convenient to set up a |Makefile| or
shell script to automatise the process.

%%%%%%%%%%%%%%%%%%%%%%%%%%%%%%%%%%%%%%%%%%%%%%%%%%%%%%%%%%%%%%%%%%%%%%%%%%%%%%%%
\subsection{Command Line Processing}
\label{sec:commandline}

The effect of redirection files can also be achieved by invoking
the \LaTeX{} compiler with a more elaborate command line.
Most conveniently this should be done as part
of a shell script or a |Makefile|.

When using \textsf{childdoc} in the main file, the following
command lines effectively perform a redirection
(note that depending on the shell being used,
backslashes may have to be doubled: `|\|' $\to$ `|\\|'):
%
\begin{center}
|... -jobname "|\textit{target}|" |\\|"|[\textit{flags}]%
|\input{childdoc.def}\childdocforward[|\textit{main}|]{|\textit{dest}|}"|
\end{center}
%
Here \textit{target} is the name of the output file,
\textit{main} is the name of the main file
and \textit{dest} is the name of the main or child file to be processed
(all filenames without extensions).
The optional argument \textit{main} can be omitted
if \textit{main} matches \textit{dest}.
Optionally, compilation \textit{flags} can be defined via |\def| commands.
This command line makes the \TeX{} engine believe
it is compiling the file \textit{target}
whose content is specified as the latter parameter.
The provided code then forwards the processing to
\textit{main} or \textit{dest} as described in \secref{sec:forward}.

%%%%%%%%%%%%%%%%%%%%%%%%%%%%%%%%%%%%%%%%%%%%%%%%%%%%%%%%%%%%%%%%%%%%%%%%%%%%%%%%
\subsection{Include by Input}
\label{sec:input}

Including child documents by |\include| has some restrictions by design.
Most notably, the content of a child document always occupies
its own set of pages; pages cannot be shared between child documents.
Usually, this behaviour makes perfect sense
because each child document contain an essential part of the document.
However, in some situations it may be desirable to compose
a document from a collection of parts
without having mandatory page breaks between then.
For this case, the package
provides a mechanism to include parts
by |\input| which can also be processed individually.
However, by construction this mechanism
requires manual handling of the content to be output.

%%%%%%%%%%%%%%%%%%%%%%%%%%%%%%%%%%%%%%%%
\DescribeMacro{\ifchilddocmanual}
The main file should be prepared as usual, see \secref{sec:include}.
However, the document body must make a distinction
between processing of an individual part and of the main document, e.g.:
%
\begin{center}
\begin{tabular}{l}
|\ifchilddocmanual|\\
|\input{\childdocname}|\\
|\||else|\\
\textit{document body with }|\input{|\textit{part}|}|\\
|\||fi|
\end{tabular}
\end{center}
%
The conditional |\ifchilddocmanual| is true whenever
a part to be included by |\input| is being compiled,
and the name of the part is stored in |\childdocname|.

%%%%%%%%%%%%%%%%%%%%%%%%%%%%%%%%%%%%%%%%
\DescribeMacro{\childdocby}
Each part to be included by |\input| should start with:
%
\begin{center}
\begin{tabular}{l}
|\input{childdoc.def}|\\
|\childdocby{|\textit{main}|}|\\
\end{tabular}
\end{center}
%
The directive |\childdocby| is similar to |\childdocof|
described in \secref{sec:include},
but the subsequent selection of content must be done manually.
To that end, both |\ifchilddoc| and |\ifchilddocmanual|
will be true upon processing of a part,
and the name of the part is stored in |\childdocname|.
Note that |\jobname| will be set to the filename of the current part
so that each part receives an individual |.aux| file
that does not interfere with the |.aux| file(s) of the main document.
This behaviour can be altered by the alternative form
|\childdocby[*]{|\textit{main}|}| (with a non-empty optional argument)
which uses the |.aux| file of the main document
by setting |\jobname| to \textit{main}.

%%%%%%%%%%%%%%%%%%%%%%%%%%%%%%%%%%%%%%%%%%%%%%%%%%%%%%%%%%%%%%%%%%%%%%%%%%%%%%%%
\subsection{Driver Development}
\label{sec:driver}

The \textsf{childdoc} mechanism can also be use for the development
of definition files such as \LaTeX{} styles or classes.
This case differs from the above setup with multiple parts
included by |\include| in that no |\includeonly| should be invoked.
This can be achieved by starting the include file
(before |\ProvidesPackage|) with:
%
\begin{center}
\begin{tabular}{l}
|\input{childdoc.def}|\\
|\childdocforward{|\textit{main}|}|\\
\end{tabular}
\end{center}
%
or alternatively with:
%
\begin{center}
\begin{tabular}{l}
|\input{childdoc.def}|\\
|\childdocby{|\textit{main}|}|\\
\end{tabular}
\end{center}
%
Both forms have slightly different effects as described above.
The main file is prepared as usual, see \secref{sec:include}.

%%%%%%%%%%%%%%%%%%%%%%%%%%%%%%%%%%%%%%%%%%%%%%%%%%%%%%%%%%%%%%%%%%%%%%%%%%%%%%%%
\subsection{Legacy Detection}
\label{sec:detection}

The directive |\childdocmain| in the main file can detect
whether the complete document or merely a child is to be compiled
even without using the directive |\childdocof|.
This method is deprecated because it is less robust
and there is no compelling reason to use it;
it is merely provided for backward compatibility
and it may be removed in future versions.

If the detection mechanism is to be used,
it is mandatory to correctly specify
the filename of the main file as the argument of |\childdocmain|:
%
\begin{center}
\begin{tabular}{l}
|\input{childdoc.def}|\\
|\childdocmain{|\textit{main}|}|\\
\end{tabular}
\end{center}
%
If |\jobname| does not match the argument \textit{main} of |\childdocmain|,
it is assumed that |\jobname| points to the child file to be compiled.
When using |\childdocmain| with the main file specified as argument,
it suffices to start a child file
with just |\input{|\textit{main}|}|
without loading of the package and using |\childdocof|.
If instead all processing is done
with the appropriate \textsf{childdoc} directives,
the argument of \textit{main} of |\childdocmain| can be empty.

An alternative version of the command line processing described
in \secref{sec:commandline} using the detection mechanism reads:
%
\begin{center}
|... -jobname "|\textit{target}|" "|[\textit{flags}]%
[|\def\jobname{|\textit{dest}|}|]|\input{|\textit{main}|}"|
\end{center}

%%%%%%%%%%%%%%%%%%%%%%%%%%%%%%%%%%%%%%%%%%%%%%%%%%%%%%%%%%%%%%%%%%%%%%%%%%%%%%%%
\subsection{Manual Code}
\label{sec:manual}

In case one cannot be certain whether the definitions file |childdoc.def|
is installed on the target \TeX{} distribution
and one prefers not to ship it,
it is conceivable to paste a few relevant commands into the sources.

To that end, drop all statements |\input{childdoc.def}|
and perform the replacements as outlined below.
Instead of |\childdocmain{|\textit{main}|}| add the following code
to the top of the main file:
%
\begin{center}
\begin{tabular}{l}
|\||ifdefined\childdocname\endinput\||fi\newif\ifchilddoc|\\
|\edef\childdocname{\scantokens\expandafter{\jobname\noexpand}}|\\
|\def\childdocmain{|\textit{main}|}\||ifx\childdocmain\childdocname\||else|\\
|\childdoctrue\includeonly{\childdocname}\let\jobname\childdocmain\||fi|\\
\end{tabular}
\end{center}
%
Instead of |\childdocof{|\textit{main}|}| just include the main file
at the top of each child file:
%
\begin{center}
|\input{|\textit{main}|}|
\end{center}
%
A simple redirection |\childdocforward{|\textit{dest}|}| is achieved by:
%
\begin{center}
|\def\jobname{|\textit{dest}|}\input{\jobname}|
\end{center}
%
The redirection with prefix
|\childdocforwardprefix[|\textit{prefix}|]{|\textit{dest}|}|
is accomplished by:
%
\begin{center}
\begin{tabular}{l}
|{\edef\jobname{\scantokens\expandafter{\jobname\noexpand}}|\\
|\def\redirectjob |\textit{prefix}|#1~~~{\gdef\jobname{|\textit{dest}|#1}}|\\
|\expandafter\redirectjob\jobname~~~}\input{\jobname}|
\end{tabular}
\end{center}

In an alternative approach,
child documents can be compiled by a specific command line
without additional code or specific definitions:
%
\begin{center}
|... -jobname "|\textit{target}|" "|[\textit{flags}]%
|\includeonly{|\textit{dest}|}\input{|\textit{main}|}"|
\end{center}
%

%%%%%%%%%%%%%%%%%%%%%%%%%%%%%%%%%%%%%%%%%%%%%%%%%%%%%%%%%%%%%%%%%%%%%%%%%%%%%%%%
%%%%%%%%%%%%%%%%%%%%%%%%%%%%%%%%%%%%%%%%%%%%%%%%%%%%%%%%%%%%%%%%%%%%%%%%%%%%%%%%
\section{Information}

%%%%%%%%%%%%%%%%%%%%%%%%%%%%%%%%%%%%%%%%%%%%%%%%%%%%%%%%%%%%%%%%%%%%%%%%%%%%%%%%
\subsection{Copyright}

Copyright \copyright{} 2017--2018 Niklas Beisert

This work may be distributed and/or modified under the
conditions of the \LaTeX{} Project Public License, either version 1.3
of this license or (at your option) any later version.
The latest version of this license is in
  \url{http://www.latex-project.org/lppl.txt}
and version 1.3 or later is part of all distributions of \LaTeX{}
version 2005/12/01 or later.

This work has the LPPL maintenance status `maintained'.

The Current Maintainer of this work is Niklas Beisert.

This work consists of the files |README.txt|, |childdoc.ins| and |childdoc.dtx|
as well as the derived files |childdoc.def|, |cdocsamp.tex|
with |cdocsch1.tex|, |cdocsch2.tex|, |cdocspt3.tex|, |cdocspt4.tex|,
|cdocsdrf.tex|, |cdocsfn1.tex|, |cdocsfn2.tex|
as well as |childdoc.pdf|.

%%%%%%%%%%%%%%%%%%%%%%%%%%%%%%%%%%%%%%%%%%%%%%%%%%%%%%%%%%%%%%%%%%%%%%%%%%%%%%%%
\subsection{Files and Installation}

The package consists of the files:
%
\begin{center}
\begin{tabular}{ll}
    |README.txt|   & readme file \\
    |childdoc.ins| & installation file \\
    |childdoc.dtx| & source file \\
    |childdoc.def| & definition file \\
    |cdocsamp.tex| & sample main file \\
    |cdocsch1.tex| & sample include file \\
    |cdocsch2.tex| & sample include file \\
    |cdocspt3.tex| & sample part file \\
    |cdocspt4.tex| & sample part file \\
    |cdocsdrf.tex| & sample redirection file \\
    |cdocsfn1.tex| & sample redirection file \\
    |cdocsfn2.tex| & sample redirection file \\
    |childdoc.pdf| & manual
\end{tabular}
\end{center}
%
The distribution consists of the files
|README.txt|, |childdoc.ins| and |childdoc.dtx|.
%
\begin{itemize}
\item
Run (pdf)\LaTeX{} on |childdoc.dtx|
to compile the manual |childdoc.pdf| (this file).
\item
Run \LaTeX{} on |childdoc.ins| to create the definitions file |childdoc.def|
and the sample |cdocsamp.tex| with include files
|cdocsch1.tex|, |cdocsch2.tex|, |cdocspt3.tex|, |cdocspt4.tex|,
|cdocsdrf.tex|, |cdocsfn1.tex|, |cdocsfn2.tex|.
Then copy the file |childdoc.def| to an appropriate directory of your \LaTeX{}
distribution, e.g.\ \textit{texmf-root}|/tex/latex/childdoc|.
\end{itemize}

%%%%%%%%%%%%%%%%%%%%%%%%%%%%%%%%%%%%%%%%%%%%%%%%%%%%%%%%%%%%%%%%%%%%%%%%%%%%%%%%
\subsection{Related CTAN Packages}

There are several other packages which offer a similar functionality:
%
\begin{itemize}
\item
The packages
\href{http://ctan.org/pkg/docmute}{\textsf{docmute}},
\href{http://ctan.org/pkg/includex}{\textsf{includex}} and
\href{http://ctan.org/pkg/standalone}{\textsf{standalone}}
provide commands to include only the document body of
a child file thus allowing both files to be compiled individually.
\item
The packages \href{http://ctan.org/pkg/subdocs}{\textsf{subdocs}}
and \href{http://ctan.org/pkg/subfiles}{\textsf{subfiles}}
provide structures in which the main and child documents can be
encapsulated and allowing them to be compiled individually.
The inclusion mechanism is different from the conventional |\include|.
\item
The package \href{http://ctan.org/pkg/combine}{\textsf{combine}}
is an elaborate solution to combine several documents into one.
\end{itemize}
%
See also the CTAN topic \href{http://ctan.org/topic/subdocs}{\textsf{subdocs}}
for further related packages.
The present package differs from the above solutions in that
a document structure constructed with the conventional |\include| mechanism
just needs two extra commands at the top of every file
such that all constituent files can be compiled individually.

%%%%%%%%%%%%%%%%%%%%%%%%%%%%%%%%%%%%%%%%%%%%%%%%%%%%%%%%%%%%%%%%%%%%%%%%%%%%%%%%
%\subsection{Feature Suggestions}
%
%The following is a list of features which may be useful for future
%versions of this package:
%%
%\begin{itemize}
%\item
%\ldots
%\end{itemize}

%%%%%%%%%%%%%%%%%%%%%%%%%%%%%%%%%%%%%%%%%%%%%%%%%%%%%%%%%%%%%%%%%%%%%%%%%%%%%%%%
\subsection{Revision History}

%%%%%%%%%%%%%%%%%%%%%%%%%%%%%%%%%%%%%%%%
\paragraph{v2.0:} 2018/12/30

\begin{itemize}
\item
immediate forward processing
\item
added |\childdocby| mechanism
\item
manual restructured
\end{itemize}

%%%%%%%%%%%%%%%%%%%%%%%%%%%%%%%%%%%%%%%%
\paragraph{v1.6:} 2018/01/17

\begin{itemize}
\item
application for development of include files
\item
corrections to manual
\end{itemize}

%%%%%%%%%%%%%%%%%%%%%%%%%%%%%%%%%%%%%%%%
\paragraph{v1.5:} 2017/05/21

\begin{itemize}
\item
more complete structuring introduced
\item
|\childdocof| introduced
\item
|\childdoc| renamed to |\childdocmain|
\item
|\childredirect| renamed to |\childdocforward| and |\childdocforwardprefix|
and functionality expanded
\end{itemize}

%%%%%%%%%%%%%%%%%%%%%%%%%%%%%%%%%%%%%%%%
\paragraph{v1.0:} 2017/04/27

\begin{itemize}
\item
manual and install package
\item
first version published on CTAN
\end{itemize}

%%%%%%%%%%%%%%%%%%%%%%%%%%%%%%%%%%%%%%%%
\paragraph{v0.6:} 2017/04/26

\begin{itemize}
\item
redirection mechanism added
\end{itemize}

%%%%%%%%%%%%%%%%%%%%%%%%%%%%%%%%%%%%%%%%
\paragraph{v0.5:} 2017/04/26

\begin{itemize}
\item
functionality in definition file
\end{itemize}


%%%%%%%%%%%%%%%%%%%%%%%%%%%%%%%%%%%%%%%%%%%%%%%%%%%%%%%%%%%%%%%%%%%%%%%%%%%%%%%%
%%%%%%%%%%%%%%%%%%%%%%%%%%%%%%%%%%%%%%%%%%%%%%%%%%%%%%%%%%%%%%%%%%%%%%%%%%%%%%%%
%%%%%%%%%%%%%%%%%%%%%%%%%%%%%%%%%%%%%%%%%%%%%%%%%%%%%%%%%%%%%%%%%%%%%%%%%%%%%%%%
\appendix

\settowidth\MacroIndent{\rmfamily\scriptsize 000\ }

 \DocInput{childdoc.dtx}

\end{document}
%</driver>
% \fi
%
% %%%%%%%%%%%%%%%%%%%%%%%%%%%%%%%%%%%%%%%%%%%%%%%%%%%%%%%%%%%%%%%%%%%%%%%%%%%%%%
% %%%%%%%%%%%%%%%%%%%%%%%%%%%%%%%%%%%%%%%%%%%%%%%%%%%%%%%%%%%%%%%%%%%%%%%%%%%%%%
% \section{Sample}
%\iffalse
%<*samplemain>
%\fi
%
% The following presents a sample document
% with two chapters, two parts, a title page,
% a compile flag as well as three forwarding files to set the flag.
% It consists of eight |.tex| files:
% \begin{center}
% \begin{tabular}{ll}
% |cdocsamp.tex|&main file\\
% |cdocsch1.tex|&include file for chapter 1\\
% |cdocsch2.tex|&include file for chapter 2\\
% |cdocspt3.tex|&include file for part 3\\
% |cdocspt4.tex|&include file for part 4\\
% |cdocsdrf.tex|&forwarding file for main file in draft mode\\
% |cdocsfi1.tex|&forwarding file for final version of chapter 1\\
% |cdocsfi2.tex|&forwarding file for final version of chapter 2\\
% \end{tabular}
% \end{center}
% Each of the eight files can be compiled directly by the \LaTeX{} compiler.
%
% %%%%%%%%%%%%%%%%%%%%%%%%%%%%%%%%%%%%%%
% \paragraph{Main File.}
%
% The main file is called |cdocsamp.tex|.
%
% Load the \textsf{childdoc} definitions and
% declare the filename for the main document:
%    \begin{macrocode}
\input{childdoc.def}
\childdocmain{}
%    \end{macrocode}

% Optional override for |\version| flag:
%    \begin{macrocode}
%%\ifchilddoc\else\providecommand{\version}{draft}\fi
%    \end{macrocode}

% Define the default values for the |\version| flag
% (|final| for the main file and |draft| for childs):
%    \begin{macrocode}
\ifchilddoc
\providecommand{\version}{draft}
\else
\providecommand{\version}{final}
\fi
%    \end{macrocode}

% Load the standard document class:
%    \begin{macrocode}
\documentclass[12pt]{article}
%    \end{macrocode}

% Start the document body:
%    \begin{macrocode}
\begin{document}
%    \end{macrocode}

% Declare a title page.
% Print title, part of document being processed and version flag:
%    \begin{macrocode}
\addtocounter{page}{-1}
\begin{center}
{\LARGE\bfseries{}childdoc example\par}
\vspace{1cm}
\ifchilddoc
\ifchilddocmanual part\else chapter\fi:
`\childdocname' of `\childdocjob'\par
\else
main document: `\childdocjob'\par
\fi
version: \version\par
\end{center}
\newpage
%    \end{macrocode}

% Manually include selected file,
% otherwise process as usual:
%    \begin{macrocode}
\ifchilddocmanual
\section*{part `\childdocname'}
\input{\childdocname}
\else
%    \end{macrocode}

% Include the two chapters:
%    \begin{macrocode}
\include{cdocsch1}
\include{cdocsch2}
%    \end{macrocode}

% Include the two parts unless only chapters should be displayed:
%    \begin{macrocode}
\ifchilddoc\else
\section{part three}
\input{cdocspt3}
\section{part four}
\input{cdocspt4}
\fi
%    \end{macrocode}

% Process as usual until here:
%    \begin{macrocode}
\fi
%    \end{macrocode}

% End of document body:
%    \begin{macrocode}
\end{document}
%    \end{macrocode}
%\iffalse
%</samplemain>
%\fi
%
% %%%%%%%%%%%%%%%%%%%%%%%%%%%%%%%%%%%%%%
% \paragraph{Chapter Include Files.}
%
% The include files are called |cdocsch1.tex| and |cdocsch2.tex|.
%
%\iffalse
%<*samplechap1|samplechap2>
%\fi

% Optional override for |\version| flag:
%    \begin{macrocode}
%%\providecommand{\version}{final}
%    \end{macrocode}

% Include the main document:
%    \begin{macrocode}
\input{childdoc.def}
\childdocof{cdocsamp}
%    \end{macrocode}

%\iffalse
%</samplechap1|samplechap2>
%\fi
%
%\iffalse
%<*samplechap1>
%\fi
% Some text for chapter 1:
%    \begin{macrocode}
\section{one}
some text in chapter one
%    \end{macrocode}

%\iffalse
%</samplechap1>
%\fi
% Some text for chapter 2:
%\iffalse
%<*samplechap2>
%\fi
%    \begin{macrocode}
\section{two}
more text in chapter two
%    \end{macrocode}

%\iffalse
%</samplechap2>
%\fi
%
% %%%%%%%%%%%%%%%%%%%%%%%%%%%%%%%%%%%%%%
% \paragraph{Part Include Files.}
%
% The include files are called |cdocspt3.tex| and |cdocspt4.tex|.
%
%\iffalse
%<*samplepart3|samplepart4>
%\fi

% Optional override for |\version| flag:
%    \begin{macrocode}
%%\providecommand{\version}{final}
%    \end{macrocode}

% Include the main document:
%    \begin{macrocode}
\input{childdoc.def}
\childdocby{cdocsamp}
%    \end{macrocode}

%\iffalse
%</samplepart3|samplepart4>
%\fi
%
%\iffalse
%<*samplepart3>
%\fi
% Some text for part 3:
%    \begin{macrocode}
some text in part three
%    \end{macrocode}

%\iffalse
%</samplepart3>
%\fi
% Some text for part 4:
%\iffalse
%<*samplepart4>
%\fi
%    \begin{macrocode}
more text in part four
%    \end{macrocode}

%\iffalse
%</samplepart4>
%\fi
%
% %%%%%%%%%%%%%%%%%%%%%%%%%%%%%%%%%%%%%%
% \paragraph{Forwarding for a Complete Draft.}
%
% The following forwarding file |cdocsdrf.tex|
% compiles the main document in draft mode:
%\iffalse
%<*sampledraft>
%\fi
%    \begin{macrocode}
\def\version{draft}
\input{childdoc.def}
\childdocforward{cdocsamp}
%    \end{macrocode}

%\iffalse
%</sampledraft>
%\fi
%
% %%%%%%%%%%%%%%%%%%%%%%%%%%%%%%%%%%%%%%
% \paragraph{Forwarding for Final Version of the Chapters.}
%
% The following forwarding files |cdocsfn1.tex| and |cdocsfn2.tex|
% (with identical content)
% compile the final versions of the child documents
% |cdocsch1.tex| and |cdocsch2.tex|, respectively:
%\iffalse
%<*samplefinal>
%\fi
%    \begin{macrocode}
\def\version{final}
\input{childdoc.def}
\childdocforwardprefix[cdocsamp]{cdocsfn}{cdocsch}
%    \end{macrocode}

%\iffalse
%</samplefinal>
%\fi
%
% %%%%%%%%%%%%%%%%%%%%%%%%%%%%%%%%%%%%%%
% \paragraph{Command Line Processing.}
%
% The following three command lines generate the output files
% |cdocscld|, |cdocscl1| and |cdocscl2|
% which should be identical to
% |cdocsdrf|, |cdocsch1| and |cdocsfn2|, respectively:
% \begin{center}
% \begin{tabular}{l}
% |latex -jobname cdocscld \|\\
% |  "\def\version{draft}\input{childdoc.def}\childdocforward{cdocsamp}"|\\
% |latex -jobname cdocscl1 \|\\
% |  "\input{childdoc.def}\childdocforward[cdocsamp]{cdocsch1}"|\\
% |latex -jobname cdocscl2 \|\\
% |  "\def\version{final}\input{childdoc.def}\childdocforward{cdocsch2}"|
% \end{tabular}
% \end{center}
% Note that the trailing backslash on each first line
% merely continues the input to the second line
% (for convenient cut ant paste).
% Furthermore, the command |latex| can be replaced by any
% of its alternative versions such as |pdflatex|.
%
% %%%%%%%%%%%%%%%%%%%%%%%%%%%%%%%%%%%%%%%%%%%%%%%%%%%%%%%%%%%%%%%%%%%%%%%%%%%%%%
% %%%%%%%%%%%%%%%%%%%%%%%%%%%%%%%%%%%%%%%%%%%%%%%%%%%%%%%%%%%%%%%%%%%%%%%%%%%%%%
% \section{Implementation}
%\iffalse
%<*package>
%\fi
%
% This section describes the definitions file |childdoc.def|.

% The definitions cannot be loaded using |\usepackage| or |\RequirePackage|
% which has a mechanism to prevent loading a style file more than once.
% When loading the definitions by means of |\input|
% multiple instances have to be prevented manually:
%\iffalse
%This code needs to be before the `\ProvidesFile' directive
%which is defined at the beginning of this file.
%Therefore it is also placed there and commented out here.
%</package>
%<*discard>
%\fi
%    \begin{macrocode}
\ifdefined\childdocmain\endinput\fi
%    \end{macrocode}
%\iffalse
%</discard>
%<*package>
%\fi
%
% \macro{\ifchilddoc}
% \macro{\ifchilddocmanual}
% The conditional |\ifchilddoc| tells whether a
% child (true) or main (false) document is being compiled.
% The conditional |\ifchilddocmanual| tells whether
% the |\includeonly| mechanism is used (false) or
% the selection of child files must be performed manually (true).
% The definitions initialise to false:
%    \begin{macrocode}
\newif\ifchilddoc
\newif\ifchilddocmanual
%    \end{macrocode}

% \macro{\childdocname}
% \macro{\childdocjob}
% The macro |\childdocname| stores the name of the main document
% to be compiled. The macro |\childdocjob| stores the name of
% the document on which the \LaTeX{} compiler was originally invoked.
% The content of |\jobname| cannot be compared
% to filenames specified in the source due to different catcodes.
% The following code rescans |\jobname|, stores the result
% in |\childdocname| and saves a copy in |\childdocjob|:
%    \begin{macrocode}
\edef\childdocname{\scantokens\expandafter{\jobname\noexpand}}
\let\childdocjob\childdocname
%    \end{macrocode}

% \macro{\childdocdisable}
% The macro |\childdocdisable| prevents the main file
% from being processed more than once.
% At this stage, the main document command |\childdocmain|
% is assumed to be called once again where it should do nothing.
% Any subsequent call to it should prevent
% a secondary processing of the main document
% It overwrites the forwarding commands
% |\childdocof| and |\childdocforward|
% with empty macros to prevent further inclusions of the main document:
%    \begin{macrocode}
\newcommand{\childdocdisable}
{
  \renewcommand{\childdocmain}[1]{\renewcommand{\childdocmain}[1]{\endinput}}
  \renewcommand{\childdocof}[1]{}
  \renewcommand{\childdocby}[2][]{}
  \renewcommand{\childdocforward}[2][]{}
  \renewcommand{\childdocdisable}{}
}
%    \end{macrocode}

% \macro{\childdocmain}
% The macro |\childdocmain| is to be called at the top of the main file
% with nothing or the main filename (without extension) as argument.
% First, it breaks loops.
% If the argument is not empty and does not match |\childdocname|
% (which is set by the first inclusion of |childdoc.def|),
% |\ifchilddoc| is set to true, |\includeonly| is applied to the child file
% and |\jobname| is set to the main file
% (for proper handling of |.aux| files):
%    \begin{macrocode}
\newcommand{\childdocmain}[1]
{
  \childdocdisable\childdocmain{}
  \if?#1?\else
    \begingroup
      \def\childdoctmp{#1}
      \ifx\childdoctmp\childdocname
        \def\childdoctmp{}
      \else
        \def\childdoctmp
        {
          \childdoctrue
          \includeonly{\childdocname}
          \def\childdocjob{#1}
          \def\jobname{#1}
        }
      \fi
      \expandafter
    \endgroup
    \childdoctmp
  \fi
}
%    \end{macrocode}

% \macro{\childdocof}
% The command |\childdocof| redirects
% compilation to the main file |#1|.
%    \begin{macrocode}
\newcommand{\childdocof}[1]
{
  \childdocdisable
  \childdoctrue
  \includeonly{\childdocname}
  \def\jobname{#1}
  \def\childdocjob{#1}
  \input{#1}
}
%    \end{macrocode}

% \macro{\childdocby}
% The command |\childdocby| ....
%    \begin{macrocode}
\newcommand{\childdocby}[2][]
{
  \childdocdisable
  \childdoctrue
  \childdocmanualtrue
  \if?#1?\else
    \def\jobname{#2}
  \fi
  \def\childdocjob{#2}
  \input{#2}
  \endinput
}
%    \end{macrocode}

% \macro{\childdocforward}
% The command |\childdocforward| redirects
% compilation to the main file or
% (if the optional argument is given) a child file.
% Parameters are set as if the main file
% or a child file starting with |\childdocof| was compiled.
% Then compilation is handed over to the main file:
%    \begin{macrocode}
\newcommand{\childdocforward}[2][]
{
  \begingroup
    \if?#1?
      \def\childdoctmp
      {
        \def\childdocname{#2}
        \def\childdocjob{#2}
        \def\jobname{#2}
        \input{#2}
        \endinput
      }
    \else
      \def\childdoctmp
      {
        \childdocdisable
        \def\childdocname{#2}
        \childdoctrue
        \includeonly{#2}
        \def\childdocjob{#1}
        \def\jobname{#1}
        \input{#1}
        \endinput
      }
    \fi
    \expandafter
  \endgroup
  \childdoctmp
}
%    \end{macrocode}

% \macro{\childdocforwardprefix}
% The command |\childdocforwardprefix| redirects
% compilation to the main or a child file by means of a pattern.
% The prefix |#1| in the current filename is replaced by |#2|
% and the suffix of the current filename is kept
% (it is assumed that the filename does not contain the substring `|~~~|'
% which is used as a delimiter).
% Compilation is handed over to the new file by |\childdocforward|:
%    \begin{macrocode}
\newcommand{\childdocforwardprefix}[3][]
{
  \begingroup
    \def\childdocextract #2##1~~~{\def\childdoctmp{\childdocforward[#1]{#3##1}}}
    \expandafter\childdocextract\childdocname~~~
    \expandafter
  \endgroup
  \childdoctmp
}
%    \end{macrocode}

% \macro{\childdoc}
% The deprecated macro |\childdoc| is a legacy version of |\childdocmain|:
%    \begin{macrocode}
\newcommand{\childdoc}{\childdocmain}
%    \end{macrocode}

% \macro{\childdocredirect}
% The deprecated macro |\childdocredirect| is a legacy version
% of |\childdocforward| and |\childdocforwardprefix|:
%    \begin{macrocode}
\newcommand{\childdocredirect}[2][]
{
  \begingroup
    \if?#1?
      \def\childdoctmp{\childdocforward{#2}}
    \else
      \def\childdoctmp{\childdocforwardprefix{#1}{#2}}
    \fi
    \expandafter
  \endgroup
  \childdoctmp
}
%    \end{macrocode}

%\iffalse
%</package>
%\fi
%
\endinput
\childdocforward[|\textit{main}|]{|\textit{dest}|}"|
\end{center}
%
Here \textit{target} is the name of the output file,
\textit{main} is the name of the main file
and \textit{dest} is the name of the main or child file to be processed
(all filenames without extensions).
The optional argument \textit{main} can be omitted
if \textit{main} matches \textit{dest}.
Optionally, compilation \textit{flags} can be defined via |\def| commands.
This command line makes the \TeX{} engine believe
it is compiling the file \textit{target}
whose content is specified as the latter parameter.
The provided code then forwards the processing to
\textit{main} or \textit{dest} as described in \secref{sec:forward}.

%%%%%%%%%%%%%%%%%%%%%%%%%%%%%%%%%%%%%%%%%%%%%%%%%%%%%%%%%%%%%%%%%%%%%%%%%%%%%%%%
\subsection{Include by Input}
\label{sec:input}

Including child documents by |\include| has some restrictions by design.
Most notably, the content of a child document always occupies
its own set of pages; pages cannot be shared between child documents.
Usually, this behaviour makes perfect sense
because each child document contain an essential part of the document.
However, in some situations it may be desirable to compose
a document from a collection of parts
without having mandatory page breaks between then.
For this case, the package
provides a mechanism to include parts
by |\input| which can also be processed individually.
However, by construction this mechanism
requires manual handling of the content to be output.

%%%%%%%%%%%%%%%%%%%%%%%%%%%%%%%%%%%%%%%%
\DescribeMacro{\ifchilddocmanual}
The main file should be prepared as usual, see \secref{sec:include}.
However, the document body must make a distinction
between processing of an individual part and of the main document, e.g.:
%
\begin{center}
\begin{tabular}{l}
|\ifchilddocmanual|\\
|\input{\childdocname}|\\
|\||else|\\
\textit{document body with }|\input{|\textit{part}|}|\\
|\||fi|
\end{tabular}
\end{center}
%
The conditional |\ifchilddocmanual| is true whenever
a part to be included by |\input| is being compiled,
and the name of the part is stored in |\childdocname|.

%%%%%%%%%%%%%%%%%%%%%%%%%%%%%%%%%%%%%%%%
\DescribeMacro{\childdocby}
Each part to be included by |\input| should start with:
%
\begin{center}
\begin{tabular}{l}
|% \iffalse
%
% childdoc.dtx Copyright (C) 2017-2018 Niklas Beisert
%
% This work may be distributed and/or modified under the
% conditions of the LaTeX Project Public License, either version 1.3
% of this license or (at your option) any later version.
% The latest version of this license is in
%   http://www.latex-project.org/lppl.txt
% and version 1.3 or later is part of all distributions of LaTeX
% version 2005/12/01 or later.
%
% This work has the LPPL maintenance status `maintained'.
%
% The Current Maintainer of this work is Niklas Beisert.
%
% This work consists of the files childdoc.dtx and childdoc.ins
% and the derived files childdoc.def and cdocsamp.tex with
% cdocsch1.tex, cdocsch2.tex, cdocsdrf.tex, cdocsfn1.tex, cdocsfn2.tex.
%
%<package>\ifdefined\childdocmain\endinput\fi
%<package>\ProvidesFile{childdoc.def}[2018/12/30 v2.0 child document driver]
%<samplemain>\ProvidesFile{cdocsamp.tex}[2018/12/30 v2.0 sample for childdoc]
%<*driver>
%\ProvidesFile{childdoc.drv}[2018/12/30 v2.0 childdoc reference manual file]
\PassOptionsToClass{10pt,a4paper}{article}
\documentclass{ltxdoc}

\usepackage[margin=35mm]{geometry}
\usepackage{hyperref}
\usepackage{hyperxmp}
\usepackage[usenames]{color}

\hypersetup{colorlinks=true}
\hypersetup{pdfstartview=FitH}
\hypersetup{pdfpagemode=UseNone}
\hypersetup{pdfsource={}}
\hypersetup{pdflang={en-UK}}
\hypersetup{pdfcopyright={Copyright 2017-2018 Niklas Beisert.
  This work may be distributed and/or modified under the
  conditions of the LaTeX Project Public License, either version 1.3
  of this license or (at your option) any later version.}}
\hypersetup{pdflicenseurl={http://www.latex-project.org/lppl.txt}}
\hypersetup{pdfcontactaddress={ETH Zurich, ITP, HIT K,
  Wolfgang-Pauli-Strasse 27}}
\hypersetup{pdfcontactpostcode={8093}}
\hypersetup{pdfcontactcity={Zurich}}
\hypersetup{pdfcontactcountry={Switzerland}}
\hypersetup{pdfcontactemail={nbeisert@itp.phys.ethz.ch}}
\hypersetup{pdfcontacturl={http://people.phys.ethz.ch/\xmptilde nbeisert/}}

\newcommand{\secref}[1]{\hyperref[#1]{section \ref*{#1}}}

\parskip1ex
\parindent0pt
\let\olditemize\itemize
\def\itemize{\olditemize\parskip0pt}

\begin{document}

\title{The \textsf{childdoc} Package}
\hypersetup{pdftitle={The childdoc Package}}
\author{Niklas Beisert\\[2ex]
  Institut f\"ur Theoretische Physik\\
  Eidgen\"ossische Technische Hochschule Z\"urich\\
  Wolfgang-Pauli-Strasse 27, 8093 Z\"urich, Switzerland\\[1ex]
  \href{mailto:nbeisert@itp.phys.ethz.ch}
  {\texttt{nbeisert@itp.phys.ethz.ch}}}
\hypersetup{pdfauthor={Niklas Beisert}}
\hypersetup{pdfsubject={Manual for the LaTeX2e Package childdoc}}
\date{30 December 2018, \textsf{v2.0}}
\maketitle

\begin{abstract}\noindent
\textsf{childdoc} is a \LaTeXe{} package
that enables the direct compilation
of document sections included by |\include|
to individual files.
\end{abstract}

\begingroup
\parskip0ex
\tableofcontents
\endgroup

%%%%%%%%%%%%%%%%%%%%%%%%%%%%%%%%%%%%%%%%%%%%%%%%%%%%%%%%%%%%%%%%%%%%%%%%%%%%%%%%
%%%%%%%%%%%%%%%%%%%%%%%%%%%%%%%%%%%%%%%%%%%%%%%%%%%%%%%%%%%%%%%%%%%%%%%%%%%%%%%%
\section{Introduction}

\LaTeX{} provides a mechanism to structure a large document (such as a book)
into a main file and several child files (containing the chapters)
using the |\include| command.
This mechanism is beneficial for documents
which span hundreds of pages in order to
make the source file(s) more manageable.
Moreover, compilation can be restricted to
selected child files by means of the |\includeonly| command.
The latter feature can be used to reduce the compilation time while editing
(this was significantly more useful in the earlier days of \LaTeX{})
or to generate a smaller document which is easier to navigate.
Another application of |\includeonly| is to generate
documents consisting of selected parts of the complete document.

However, there are a few drawbacks of the plain |\include| mechanism:
\begin{itemize}
\item
The child files cannot be compiled on their own,
they can only be compiled via the main file.
A naive editing environment
(such as a text editor with an option
to have the current file processed by \LaTeX)
may require one to switch to the main file before compiling;
attempting to compile the child file produces errors.
\item
The main file must be modified (each time)
to adjust the |\includeonly| command
to the present needs. This easily leaves the main file in a messy state.
\item
The generated document will always carry the filename
of the main document. This is inconvenient if
several child files are to be compiled and
to be kept for distribution.
\end{itemize}

The present package provides a simple interface
to make child files individually compilable by \LaTeX{}.
Compiling a child file then has the same effect as compiling
the main file with an |\includeonly| command
to select the appropriate child.
Moreover the generated document will carry the name of the child
rather than the main file.
This resolves all three above issues.

This feature is meant to make the editing of books,
thesis documents and lecture notes somewhat more convenient.
However, the package can also be used efficiently for
composing a series of documents (such as exercise sheets)
which are typically distributed individually.
It then assists the author in generating the individual documents
(potentially in different versions)
as well as a document containing the collected series.
Another application is in developing style files
or other kinds of included material
where compilation of the style file could redirect
to a sample or test file.

%%%%%%%%%%%%%%%%%%%%%%%%%%%%%%%%%%%%%%%%%%%%%%%%%%%%%%%%%%%%%%%%%%%%%%%%%%%%%%%%
%%%%%%%%%%%%%%%%%%%%%%%%%%%%%%%%%%%%%%%%%%%%%%%%%%%%%%%%%%%%%%%%%%%%%%%%%%%%%%%%
\section{Usage}

First of all, the package \textsf{childdoc} is \emph{not} a standard
\LaTeXe{} |.sty| style file! Therefore it needs to be invoked in
a non-standard way.

%%%%%%%%%%%%%%%%%%%%%%%%%%%%%%%%%%%%%%%%%%%%%%%%%%%%%%%%%%%%%%%%%%%%%%%%%%%%%%%%
\subsection{Included Files}
\label{sec:include}

%%%%%%%%%%%%%%%%%%%%%%%%%%%%%%%%%%%%%%%%
\DescribeMacro{\childdocmain}
To use the package, add the commands
\begin{center}
\begin{tabular}{l}
|\input{childdoc.def}|\\
|\childdocmain{}|\\
\end{tabular}
\end{center}
at the very top of the main \LaTeX{} file,
in particular \emph{before} the |\documentclass| statement!
The argument of |\childdocmain| should be left empty
(but it must be present).

%%%%%%%%%%%%%%%%%%%%%%%%%%%%%%%%%%%%%%%%
\DescribeMacro{\childdocof}
Furthermore, add the commands
\begin{center}
\begin{tabular}{l}
|\input{childdoc.def}|\\
|\childdocof{|\textit{main}|}|\\
\end{tabular}
\end{center}
at the top of every child file \textit{child}
which is included by |\include{|\textit{child}|}|
from within the main file
(or at least for those files to be compiled individually).
The argument \textit{main} must be the filename of the main file.

There are a couple of
considerations in setting up the main and child documents:

%%%%%%%%%%%%%%%%%%%%%%%%%%%%%%%%%%%%%%%%
\paragraph{Restrictions.}

Please note the following restrictions:
\begin{itemize}
\item
|\childdocmain| must be called with one argument \textit{main}
to ensure compatibility with earlier version of the package.
It must either be empty (|\childdocmain{}|)
or precisely match the filename of the main file in which it is specified.
See \secref{sec:detection} for further information.
\item
The filename \textit{main} must be specified without the |.tex| extension.
\item
The filename \textit{main} is case sensitive
(even in case-insensitive file systems)
due to internal string comparison.
\item
The argument \textit{main} should be fully expanded, it cannot be a macro.
\item
Subdirectories and special characters should be avoided in filenames.
\item
The command |\childdocmain{|\textit{main}|}| must be followed by a whitespace.
It should not be followed immediately by another command
or by a comment mark `|%|'.
This is because the \TeX{} parser reads the token immediately following
the argument of |\childdocmain| and puts it
at the beginning of every child section;
however, a white\-space is ignored.
\end{itemize}

%%%%%%%%%%%%%%%%%%%%%%%%%%%%%%%%%%%%%%%%
\paragraph{Content of Main File.}

It is advisable to place all content in the child files included by |\include|.
Any output contained in the main file will appear in all child documents
unless suppressed manually;
it cannot be suppressed automatically by the |\includeonly| directive
and thus should normally be avoided.
A method to include some content in the main file
by means of conditional processing is described in \secref{sec:conditional}.

%%%%%%%%%%%%%%%%%%%%%%%%%%%%%%%%%%%%%%%%
\paragraph{Page Numbering.}

When only a part of the document is compiled,
the appropriate numbering of pages
(as well as other status parameters)
is determined from the |.aux| files.
The latter contain information from previous passes.
However this information needs to propagate through
all intermediate child documents.
Therefore the page numbering in child documents may well
be inconsistent until the complete document is compiled at least once.

A useful (if unconventional) way to always ensure a consistent
page numbering is to restart the numbering in each child document
and denote the pages by `\textit{child}|.|\textit{page}'
where \textit{child} represents the chapter/section number of the child file.
This can be achieved by the command
|\numberwithin{page}{|\textit{child}|}|
of the \textsf{amsmath} package
where \textit{child} can be |chapter| or |section|
depending on the chosen structuring.
Alternatively, one can modify the macro |\thepage| appropriately
and reset the counter |page| at the start of each child file.

%%%%%%%%%%%%%%%%%%%%%%%%%%%%%%%%%%%%%%%%%%%%%%%%%%%%%%%%%%%%%%%%%%%%%%%%%%%%%%%%
\subsection{Conditional Processing}
\label{sec:conditional}

The package provides a mechanism to compile different versions
of a document. To customise the versions further some conditional processing
can come in handy to distinguish which version is being compiled.
The package provides two macros to describe the compilation context:

%%%%%%%%%%%%%%%%%%%%%%%%%%%%%%%%%%%%%%%%
\DescribeMacro{\ifchilddoc}
The conditional |\ifchilddoc| distinguishes between the compilation of
child documents and the main document:
%
\begin{center}
|\ifchilddoc |\textit{child-code}| |[|\||else |\textit{main-code}]| \||fi|
\end{center}

%%%%%%%%%%%%%%%%%%%%%%%%%%%%%%%%%%%%%%%%
\DescribeMacro{\childdocname}
\DescribeMacro{\childdocjob}
The macro |\childdocname| contains the filename (without extension)
of the main or child file being processed.
Note that |\childdocjob| will always contain the name of the main file.

%%%%%%%%%%%%%%%%%%%%%%%%%%%%%%%%%%%%%%%%
\paragraph{Title Page.}

Conditional processing can be used to include a title or banner page
in the main document when proper precautions are taken.
Importantly, the code in the main file should ensure that the page counter
(as well as other status parameters which are stored in the |.aux| files)
takes the same value after the conditional processing.
Otherwise the page numbers may take divergent values
depending on which part is compiled.

For example, a title page could be declared by:
%
\begin{center}
\begin{tabular}{l}
|\ifchilddoc\||else|\\
|\addtocounter{page}{-1}|\\
\textit{code for title page}\\
|\newpage|\\
|\||fi|
\end{tabular}
\end{center}
%
A banner page for the child documents can be generated by:
%
\begin{center}
\begin{tabular}{l}
|\ifchilddoc|\\
|\addtocounter{page}{-1}|\\
\textit{code for banner page}\\
|\newpage|\\
|\||fi|
\end{tabular}
\end{center}
%
Here one could write a message such as:
\begin{center}
|This is the part \childdocname{} of \childdocjob{}.|
\end{center}

%%%%%%%%%%%%%%%%%%%%%%%%%%%%%%%%%%%%%%%%%%%%%%%%%%%%%%%%%%%%%%%%%%%%%%%%%%%%%%%%
\subsection{Flags}
\label{sec:flags}

The package makes it easy to generate different versions
of the main or child documents.
To this end compilation flags can be defined
and assigned different default values.
They will be particularly useful in conjunction
with the forwarding mechanism described in \secref{sec:forward}.

For example, it may be useful to have a flag |\version|
which can be set to |draft| or |final|.
The document source will contain some conditional code
depending on the value of |\version|.
Suppose further, the flag should default to |final| for the main file
and to |draft| for child files
which is a natural assignment for editing the document.
This is achieved by placing the following code
in the preamble of the main document
(below the |\childdocmain| directive):
%
\begin{center}
\begin{tabular}{l}
|\ifchilddoc|\\
|\providecommand{\version}{draft}|\\
|\||else|\\
|\providecommand{\version}{final}|\\
|\||fi|
\end{tabular}
\end{center}
%
The definition by |\providecommand| makes sure
that previous definitions are not overwritten.
Further statements |\providecommand{\version}{...}|
can thus be added before the above code to override it.

For the main file, one might add a line
(between |\childdocmain| and the above block)
%
\begin{center}
|%\ifchilddoc\||else\providecommand{\version}{draft}\||fi|
\end{center}
%
which can be uncommented to produce a draft version.
Likewise one can add a line to the very top of a child file
(above the |\childdocof{|\textit{main}|}| directive)
%
\begin{center}
|%\providecommand{\version}{final}|
\end{center}
%
which can be uncommented to produce the final version of this child document.

%%%%%%%%%%%%%%%%%%%%%%%%%%%%%%%%%%%%%%%%%%%%%%%%%%%%%%%%%%%%%%%%%%%%%%%%%%%%%%%%
\subsection{Forwarding}
\label{sec:forward}

Different versions of the main or child documents
using compilation flags as described in \secref{sec:flags}
can be (permanently) stored in different files
for convenient compilation, viewing and distribution.
To this end, the package defines a command
to pass on compilation to a different file:

%%%%%%%%%%%%%%%%%%%%%%%%%%%%%%%%%%%%%%%%
\DescribeMacro{\childdocforward}
The command |\childdocforward| redirects processing to
another source file:
%
\begin{center}
\begin{tabular}{l}
|\input{childdoc.def}|\\
|\childdocforward[|\textit{main}|]{|\textit{dest}|}|\\
\end{tabular}
\end{center}
%
The argument \textit{dest} is the destination file
(without extension).
It should be the main file or one of the child files.
Note that further \textsf{childdoc} directives
such as |\childdocof| and |\childdocforward|
in the indicated file will be processed in this form.
The optional argument \textit{main}
passes on directly to the main file \textit{main}
while pretending to compile the child \textit{dest}.
This form behaves as if \textit{dest}
issues |\childdocof{|\textit{main}|}| right away,
and no further \textsf{childdoc} directives will be processed.

%%%%%%%%%%%%%%%%%%%%%%%%%%%%%%%%%%%%%%%%
\DescribeMacro{\...prefix}
In the alternative form |\childdocforwardprefix|,
%
\begin{center}
\begin{tabular}{l}
|\input{childdoc.def}|\\
|\childdocforwardprefix[|\textit{main}|]{|\textit{prefix}|}{|\textit{dest}|}|
\end{tabular}
\end{center}
%
the destination file is determined by a pattern
depending on the current file:
To make this work, the current file must be called
`{\textit{prefix}\hspace{0.2em}\textit{suffix}}'
with \textit{prefix} matching precisely the argument.
Processing is then passed on to the file
`{\textit{dest}\hspace{0.2em}\textit{suffix}}'.
Surely, the same effect is achieved by
directly specifying the
argument `{\textit{dest}\hspace{0.2em}\textit{suffix}}'
in the first form.
However, that requires to set up a different file
for each child. With the alternative form of the command
all these files can have exactly the same content
which simplifies setting them up and maintaining them.

For example, the following file |draft.tex|
with a compilation flag |\version| as described in \secref{sec:flags}
compiles the main document as a draft:
%
\begin{center}
\begin{tabular}{l}
|\def\version{draft}|\\
|\input{childdoc.def}|\\
|\childdocforward{|\textit{main}|}|
\end{tabular}
\end{center}
%
Likewise, the following files |final|\textit{nn}|.tex|
compile the final version of the child document
|child|\textit{nn}|.tex|:
%
\begin{center}
\begin{tabular}{l}
|\def\version{final}|\\
|\input{childdoc.def}|\\
|\childdocforwardprefix{final}{child}|
\end{tabular}
\end{center}
%

Note that when several versions of a main file and/or of each child file
are to be generated, it may be convenient to set up a |Makefile| or
shell script to automatise the process.

%%%%%%%%%%%%%%%%%%%%%%%%%%%%%%%%%%%%%%%%%%%%%%%%%%%%%%%%%%%%%%%%%%%%%%%%%%%%%%%%
\subsection{Command Line Processing}
\label{sec:commandline}

The effect of redirection files can also be achieved by invoking
the \LaTeX{} compiler with a more elaborate command line.
Most conveniently this should be done as part
of a shell script or a |Makefile|.

When using \textsf{childdoc} in the main file, the following
command lines effectively perform a redirection
(note that depending on the shell being used,
backslashes may have to be doubled: `|\|' $\to$ `|\\|'):
%
\begin{center}
|... -jobname "|\textit{target}|" |\\|"|[\textit{flags}]%
|\input{childdoc.def}\childdocforward[|\textit{main}|]{|\textit{dest}|}"|
\end{center}
%
Here \textit{target} is the name of the output file,
\textit{main} is the name of the main file
and \textit{dest} is the name of the main or child file to be processed
(all filenames without extensions).
The optional argument \textit{main} can be omitted
if \textit{main} matches \textit{dest}.
Optionally, compilation \textit{flags} can be defined via |\def| commands.
This command line makes the \TeX{} engine believe
it is compiling the file \textit{target}
whose content is specified as the latter parameter.
The provided code then forwards the processing to
\textit{main} or \textit{dest} as described in \secref{sec:forward}.

%%%%%%%%%%%%%%%%%%%%%%%%%%%%%%%%%%%%%%%%%%%%%%%%%%%%%%%%%%%%%%%%%%%%%%%%%%%%%%%%
\subsection{Include by Input}
\label{sec:input}

Including child documents by |\include| has some restrictions by design.
Most notably, the content of a child document always occupies
its own set of pages; pages cannot be shared between child documents.
Usually, this behaviour makes perfect sense
because each child document contain an essential part of the document.
However, in some situations it may be desirable to compose
a document from a collection of parts
without having mandatory page breaks between then.
For this case, the package
provides a mechanism to include parts
by |\input| which can also be processed individually.
However, by construction this mechanism
requires manual handling of the content to be output.

%%%%%%%%%%%%%%%%%%%%%%%%%%%%%%%%%%%%%%%%
\DescribeMacro{\ifchilddocmanual}
The main file should be prepared as usual, see \secref{sec:include}.
However, the document body must make a distinction
between processing of an individual part and of the main document, e.g.:
%
\begin{center}
\begin{tabular}{l}
|\ifchilddocmanual|\\
|\input{\childdocname}|\\
|\||else|\\
\textit{document body with }|\input{|\textit{part}|}|\\
|\||fi|
\end{tabular}
\end{center}
%
The conditional |\ifchilddocmanual| is true whenever
a part to be included by |\input| is being compiled,
and the name of the part is stored in |\childdocname|.

%%%%%%%%%%%%%%%%%%%%%%%%%%%%%%%%%%%%%%%%
\DescribeMacro{\childdocby}
Each part to be included by |\input| should start with:
%
\begin{center}
\begin{tabular}{l}
|\input{childdoc.def}|\\
|\childdocby{|\textit{main}|}|\\
\end{tabular}
\end{center}
%
The directive |\childdocby| is similar to |\childdocof|
described in \secref{sec:include},
but the subsequent selection of content must be done manually.
To that end, both |\ifchilddoc| and |\ifchilddocmanual|
will be true upon processing of a part,
and the name of the part is stored in |\childdocname|.
Note that |\jobname| will be set to the filename of the current part
so that each part receives an individual |.aux| file
that does not interfere with the |.aux| file(s) of the main document.
This behaviour can be altered by the alternative form
|\childdocby[*]{|\textit{main}|}| (with a non-empty optional argument)
which uses the |.aux| file of the main document
by setting |\jobname| to \textit{main}.

%%%%%%%%%%%%%%%%%%%%%%%%%%%%%%%%%%%%%%%%%%%%%%%%%%%%%%%%%%%%%%%%%%%%%%%%%%%%%%%%
\subsection{Driver Development}
\label{sec:driver}

The \textsf{childdoc} mechanism can also be use for the development
of definition files such as \LaTeX{} styles or classes.
This case differs from the above setup with multiple parts
included by |\include| in that no |\includeonly| should be invoked.
This can be achieved by starting the include file
(before |\ProvidesPackage|) with:
%
\begin{center}
\begin{tabular}{l}
|\input{childdoc.def}|\\
|\childdocforward{|\textit{main}|}|\\
\end{tabular}
\end{center}
%
or alternatively with:
%
\begin{center}
\begin{tabular}{l}
|\input{childdoc.def}|\\
|\childdocby{|\textit{main}|}|\\
\end{tabular}
\end{center}
%
Both forms have slightly different effects as described above.
The main file is prepared as usual, see \secref{sec:include}.

%%%%%%%%%%%%%%%%%%%%%%%%%%%%%%%%%%%%%%%%%%%%%%%%%%%%%%%%%%%%%%%%%%%%%%%%%%%%%%%%
\subsection{Legacy Detection}
\label{sec:detection}

The directive |\childdocmain| in the main file can detect
whether the complete document or merely a child is to be compiled
even without using the directive |\childdocof|.
This method is deprecated because it is less robust
and there is no compelling reason to use it;
it is merely provided for backward compatibility
and it may be removed in future versions.

If the detection mechanism is to be used,
it is mandatory to correctly specify
the filename of the main file as the argument of |\childdocmain|:
%
\begin{center}
\begin{tabular}{l}
|\input{childdoc.def}|\\
|\childdocmain{|\textit{main}|}|\\
\end{tabular}
\end{center}
%
If |\jobname| does not match the argument \textit{main} of |\childdocmain|,
it is assumed that |\jobname| points to the child file to be compiled.
When using |\childdocmain| with the main file specified as argument,
it suffices to start a child file
with just |\input{|\textit{main}|}|
without loading of the package and using |\childdocof|.
If instead all processing is done
with the appropriate \textsf{childdoc} directives,
the argument of \textit{main} of |\childdocmain| can be empty.

An alternative version of the command line processing described
in \secref{sec:commandline} using the detection mechanism reads:
%
\begin{center}
|... -jobname "|\textit{target}|" "|[\textit{flags}]%
[|\def\jobname{|\textit{dest}|}|]|\input{|\textit{main}|}"|
\end{center}

%%%%%%%%%%%%%%%%%%%%%%%%%%%%%%%%%%%%%%%%%%%%%%%%%%%%%%%%%%%%%%%%%%%%%%%%%%%%%%%%
\subsection{Manual Code}
\label{sec:manual}

In case one cannot be certain whether the definitions file |childdoc.def|
is installed on the target \TeX{} distribution
and one prefers not to ship it,
it is conceivable to paste a few relevant commands into the sources.

To that end, drop all statements |\input{childdoc.def}|
and perform the replacements as outlined below.
Instead of |\childdocmain{|\textit{main}|}| add the following code
to the top of the main file:
%
\begin{center}
\begin{tabular}{l}
|\||ifdefined\childdocname\endinput\||fi\newif\ifchilddoc|\\
|\edef\childdocname{\scantokens\expandafter{\jobname\noexpand}}|\\
|\def\childdocmain{|\textit{main}|}\||ifx\childdocmain\childdocname\||else|\\
|\childdoctrue\includeonly{\childdocname}\let\jobname\childdocmain\||fi|\\
\end{tabular}
\end{center}
%
Instead of |\childdocof{|\textit{main}|}| just include the main file
at the top of each child file:
%
\begin{center}
|\input{|\textit{main}|}|
\end{center}
%
A simple redirection |\childdocforward{|\textit{dest}|}| is achieved by:
%
\begin{center}
|\def\jobname{|\textit{dest}|}\input{\jobname}|
\end{center}
%
The redirection with prefix
|\childdocforwardprefix[|\textit{prefix}|]{|\textit{dest}|}|
is accomplished by:
%
\begin{center}
\begin{tabular}{l}
|{\edef\jobname{\scantokens\expandafter{\jobname\noexpand}}|\\
|\def\redirectjob |\textit{prefix}|#1~~~{\gdef\jobname{|\textit{dest}|#1}}|\\
|\expandafter\redirectjob\jobname~~~}\input{\jobname}|
\end{tabular}
\end{center}

In an alternative approach,
child documents can be compiled by a specific command line
without additional code or specific definitions:
%
\begin{center}
|... -jobname "|\textit{target}|" "|[\textit{flags}]%
|\includeonly{|\textit{dest}|}\input{|\textit{main}|}"|
\end{center}
%

%%%%%%%%%%%%%%%%%%%%%%%%%%%%%%%%%%%%%%%%%%%%%%%%%%%%%%%%%%%%%%%%%%%%%%%%%%%%%%%%
%%%%%%%%%%%%%%%%%%%%%%%%%%%%%%%%%%%%%%%%%%%%%%%%%%%%%%%%%%%%%%%%%%%%%%%%%%%%%%%%
\section{Information}

%%%%%%%%%%%%%%%%%%%%%%%%%%%%%%%%%%%%%%%%%%%%%%%%%%%%%%%%%%%%%%%%%%%%%%%%%%%%%%%%
\subsection{Copyright}

Copyright \copyright{} 2017--2018 Niklas Beisert

This work may be distributed and/or modified under the
conditions of the \LaTeX{} Project Public License, either version 1.3
of this license or (at your option) any later version.
The latest version of this license is in
  \url{http://www.latex-project.org/lppl.txt}
and version 1.3 or later is part of all distributions of \LaTeX{}
version 2005/12/01 or later.

This work has the LPPL maintenance status `maintained'.

The Current Maintainer of this work is Niklas Beisert.

This work consists of the files |README.txt|, |childdoc.ins| and |childdoc.dtx|
as well as the derived files |childdoc.def|, |cdocsamp.tex|
with |cdocsch1.tex|, |cdocsch2.tex|, |cdocspt3.tex|, |cdocspt4.tex|,
|cdocsdrf.tex|, |cdocsfn1.tex|, |cdocsfn2.tex|
as well as |childdoc.pdf|.

%%%%%%%%%%%%%%%%%%%%%%%%%%%%%%%%%%%%%%%%%%%%%%%%%%%%%%%%%%%%%%%%%%%%%%%%%%%%%%%%
\subsection{Files and Installation}

The package consists of the files:
%
\begin{center}
\begin{tabular}{ll}
    |README.txt|   & readme file \\
    |childdoc.ins| & installation file \\
    |childdoc.dtx| & source file \\
    |childdoc.def| & definition file \\
    |cdocsamp.tex| & sample main file \\
    |cdocsch1.tex| & sample include file \\
    |cdocsch2.tex| & sample include file \\
    |cdocspt3.tex| & sample part file \\
    |cdocspt4.tex| & sample part file \\
    |cdocsdrf.tex| & sample redirection file \\
    |cdocsfn1.tex| & sample redirection file \\
    |cdocsfn2.tex| & sample redirection file \\
    |childdoc.pdf| & manual
\end{tabular}
\end{center}
%
The distribution consists of the files
|README.txt|, |childdoc.ins| and |childdoc.dtx|.
%
\begin{itemize}
\item
Run (pdf)\LaTeX{} on |childdoc.dtx|
to compile the manual |childdoc.pdf| (this file).
\item
Run \LaTeX{} on |childdoc.ins| to create the definitions file |childdoc.def|
and the sample |cdocsamp.tex| with include files
|cdocsch1.tex|, |cdocsch2.tex|, |cdocspt3.tex|, |cdocspt4.tex|,
|cdocsdrf.tex|, |cdocsfn1.tex|, |cdocsfn2.tex|.
Then copy the file |childdoc.def| to an appropriate directory of your \LaTeX{}
distribution, e.g.\ \textit{texmf-root}|/tex/latex/childdoc|.
\end{itemize}

%%%%%%%%%%%%%%%%%%%%%%%%%%%%%%%%%%%%%%%%%%%%%%%%%%%%%%%%%%%%%%%%%%%%%%%%%%%%%%%%
\subsection{Related CTAN Packages}

There are several other packages which offer a similar functionality:
%
\begin{itemize}
\item
The packages
\href{http://ctan.org/pkg/docmute}{\textsf{docmute}},
\href{http://ctan.org/pkg/includex}{\textsf{includex}} and
\href{http://ctan.org/pkg/standalone}{\textsf{standalone}}
provide commands to include only the document body of
a child file thus allowing both files to be compiled individually.
\item
The packages \href{http://ctan.org/pkg/subdocs}{\textsf{subdocs}}
and \href{http://ctan.org/pkg/subfiles}{\textsf{subfiles}}
provide structures in which the main and child documents can be
encapsulated and allowing them to be compiled individually.
The inclusion mechanism is different from the conventional |\include|.
\item
The package \href{http://ctan.org/pkg/combine}{\textsf{combine}}
is an elaborate solution to combine several documents into one.
\end{itemize}
%
See also the CTAN topic \href{http://ctan.org/topic/subdocs}{\textsf{subdocs}}
for further related packages.
The present package differs from the above solutions in that
a document structure constructed with the conventional |\include| mechanism
just needs two extra commands at the top of every file
such that all constituent files can be compiled individually.

%%%%%%%%%%%%%%%%%%%%%%%%%%%%%%%%%%%%%%%%%%%%%%%%%%%%%%%%%%%%%%%%%%%%%%%%%%%%%%%%
%\subsection{Feature Suggestions}
%
%The following is a list of features which may be useful for future
%versions of this package:
%%
%\begin{itemize}
%\item
%\ldots
%\end{itemize}

%%%%%%%%%%%%%%%%%%%%%%%%%%%%%%%%%%%%%%%%%%%%%%%%%%%%%%%%%%%%%%%%%%%%%%%%%%%%%%%%
\subsection{Revision History}

%%%%%%%%%%%%%%%%%%%%%%%%%%%%%%%%%%%%%%%%
\paragraph{v2.0:} 2018/12/30

\begin{itemize}
\item
immediate forward processing
\item
added |\childdocby| mechanism
\item
manual restructured
\end{itemize}

%%%%%%%%%%%%%%%%%%%%%%%%%%%%%%%%%%%%%%%%
\paragraph{v1.6:} 2018/01/17

\begin{itemize}
\item
application for development of include files
\item
corrections to manual
\end{itemize}

%%%%%%%%%%%%%%%%%%%%%%%%%%%%%%%%%%%%%%%%
\paragraph{v1.5:} 2017/05/21

\begin{itemize}
\item
more complete structuring introduced
\item
|\childdocof| introduced
\item
|\childdoc| renamed to |\childdocmain|
\item
|\childredirect| renamed to |\childdocforward| and |\childdocforwardprefix|
and functionality expanded
\end{itemize}

%%%%%%%%%%%%%%%%%%%%%%%%%%%%%%%%%%%%%%%%
\paragraph{v1.0:} 2017/04/27

\begin{itemize}
\item
manual and install package
\item
first version published on CTAN
\end{itemize}

%%%%%%%%%%%%%%%%%%%%%%%%%%%%%%%%%%%%%%%%
\paragraph{v0.6:} 2017/04/26

\begin{itemize}
\item
redirection mechanism added
\end{itemize}

%%%%%%%%%%%%%%%%%%%%%%%%%%%%%%%%%%%%%%%%
\paragraph{v0.5:} 2017/04/26

\begin{itemize}
\item
functionality in definition file
\end{itemize}


%%%%%%%%%%%%%%%%%%%%%%%%%%%%%%%%%%%%%%%%%%%%%%%%%%%%%%%%%%%%%%%%%%%%%%%%%%%%%%%%
%%%%%%%%%%%%%%%%%%%%%%%%%%%%%%%%%%%%%%%%%%%%%%%%%%%%%%%%%%%%%%%%%%%%%%%%%%%%%%%%
%%%%%%%%%%%%%%%%%%%%%%%%%%%%%%%%%%%%%%%%%%%%%%%%%%%%%%%%%%%%%%%%%%%%%%%%%%%%%%%%
\appendix

\settowidth\MacroIndent{\rmfamily\scriptsize 000\ }

 \DocInput{childdoc.dtx}

\end{document}
%</driver>
% \fi
%
% %%%%%%%%%%%%%%%%%%%%%%%%%%%%%%%%%%%%%%%%%%%%%%%%%%%%%%%%%%%%%%%%%%%%%%%%%%%%%%
% %%%%%%%%%%%%%%%%%%%%%%%%%%%%%%%%%%%%%%%%%%%%%%%%%%%%%%%%%%%%%%%%%%%%%%%%%%%%%%
% \section{Sample}
%\iffalse
%<*samplemain>
%\fi
%
% The following presents a sample document
% with two chapters, two parts, a title page,
% a compile flag as well as three forwarding files to set the flag.
% It consists of eight |.tex| files:
% \begin{center}
% \begin{tabular}{ll}
% |cdocsamp.tex|&main file\\
% |cdocsch1.tex|&include file for chapter 1\\
% |cdocsch2.tex|&include file for chapter 2\\
% |cdocspt3.tex|&include file for part 3\\
% |cdocspt4.tex|&include file for part 4\\
% |cdocsdrf.tex|&forwarding file for main file in draft mode\\
% |cdocsfi1.tex|&forwarding file for final version of chapter 1\\
% |cdocsfi2.tex|&forwarding file for final version of chapter 2\\
% \end{tabular}
% \end{center}
% Each of the eight files can be compiled directly by the \LaTeX{} compiler.
%
% %%%%%%%%%%%%%%%%%%%%%%%%%%%%%%%%%%%%%%
% \paragraph{Main File.}
%
% The main file is called |cdocsamp.tex|.
%
% Load the \textsf{childdoc} definitions and
% declare the filename for the main document:
%    \begin{macrocode}
\input{childdoc.def}
\childdocmain{}
%    \end{macrocode}

% Optional override for |\version| flag:
%    \begin{macrocode}
%%\ifchilddoc\else\providecommand{\version}{draft}\fi
%    \end{macrocode}

% Define the default values for the |\version| flag
% (|final| for the main file and |draft| for childs):
%    \begin{macrocode}
\ifchilddoc
\providecommand{\version}{draft}
\else
\providecommand{\version}{final}
\fi
%    \end{macrocode}

% Load the standard document class:
%    \begin{macrocode}
\documentclass[12pt]{article}
%    \end{macrocode}

% Start the document body:
%    \begin{macrocode}
\begin{document}
%    \end{macrocode}

% Declare a title page.
% Print title, part of document being processed and version flag:
%    \begin{macrocode}
\addtocounter{page}{-1}
\begin{center}
{\LARGE\bfseries{}childdoc example\par}
\vspace{1cm}
\ifchilddoc
\ifchilddocmanual part\else chapter\fi:
`\childdocname' of `\childdocjob'\par
\else
main document: `\childdocjob'\par
\fi
version: \version\par
\end{center}
\newpage
%    \end{macrocode}

% Manually include selected file,
% otherwise process as usual:
%    \begin{macrocode}
\ifchilddocmanual
\section*{part `\childdocname'}
\input{\childdocname}
\else
%    \end{macrocode}

% Include the two chapters:
%    \begin{macrocode}
\include{cdocsch1}
\include{cdocsch2}
%    \end{macrocode}

% Include the two parts unless only chapters should be displayed:
%    \begin{macrocode}
\ifchilddoc\else
\section{part three}
\input{cdocspt3}
\section{part four}
\input{cdocspt4}
\fi
%    \end{macrocode}

% Process as usual until here:
%    \begin{macrocode}
\fi
%    \end{macrocode}

% End of document body:
%    \begin{macrocode}
\end{document}
%    \end{macrocode}
%\iffalse
%</samplemain>
%\fi
%
% %%%%%%%%%%%%%%%%%%%%%%%%%%%%%%%%%%%%%%
% \paragraph{Chapter Include Files.}
%
% The include files are called |cdocsch1.tex| and |cdocsch2.tex|.
%
%\iffalse
%<*samplechap1|samplechap2>
%\fi

% Optional override for |\version| flag:
%    \begin{macrocode}
%%\providecommand{\version}{final}
%    \end{macrocode}

% Include the main document:
%    \begin{macrocode}
\input{childdoc.def}
\childdocof{cdocsamp}
%    \end{macrocode}

%\iffalse
%</samplechap1|samplechap2>
%\fi
%
%\iffalse
%<*samplechap1>
%\fi
% Some text for chapter 1:
%    \begin{macrocode}
\section{one}
some text in chapter one
%    \end{macrocode}

%\iffalse
%</samplechap1>
%\fi
% Some text for chapter 2:
%\iffalse
%<*samplechap2>
%\fi
%    \begin{macrocode}
\section{two}
more text in chapter two
%    \end{macrocode}

%\iffalse
%</samplechap2>
%\fi
%
% %%%%%%%%%%%%%%%%%%%%%%%%%%%%%%%%%%%%%%
% \paragraph{Part Include Files.}
%
% The include files are called |cdocspt3.tex| and |cdocspt4.tex|.
%
%\iffalse
%<*samplepart3|samplepart4>
%\fi

% Optional override for |\version| flag:
%    \begin{macrocode}
%%\providecommand{\version}{final}
%    \end{macrocode}

% Include the main document:
%    \begin{macrocode}
\input{childdoc.def}
\childdocby{cdocsamp}
%    \end{macrocode}

%\iffalse
%</samplepart3|samplepart4>
%\fi
%
%\iffalse
%<*samplepart3>
%\fi
% Some text for part 3:
%    \begin{macrocode}
some text in part three
%    \end{macrocode}

%\iffalse
%</samplepart3>
%\fi
% Some text for part 4:
%\iffalse
%<*samplepart4>
%\fi
%    \begin{macrocode}
more text in part four
%    \end{macrocode}

%\iffalse
%</samplepart4>
%\fi
%
% %%%%%%%%%%%%%%%%%%%%%%%%%%%%%%%%%%%%%%
% \paragraph{Forwarding for a Complete Draft.}
%
% The following forwarding file |cdocsdrf.tex|
% compiles the main document in draft mode:
%\iffalse
%<*sampledraft>
%\fi
%    \begin{macrocode}
\def\version{draft}
\input{childdoc.def}
\childdocforward{cdocsamp}
%    \end{macrocode}

%\iffalse
%</sampledraft>
%\fi
%
% %%%%%%%%%%%%%%%%%%%%%%%%%%%%%%%%%%%%%%
% \paragraph{Forwarding for Final Version of the Chapters.}
%
% The following forwarding files |cdocsfn1.tex| and |cdocsfn2.tex|
% (with identical content)
% compile the final versions of the child documents
% |cdocsch1.tex| and |cdocsch2.tex|, respectively:
%\iffalse
%<*samplefinal>
%\fi
%    \begin{macrocode}
\def\version{final}
\input{childdoc.def}
\childdocforwardprefix[cdocsamp]{cdocsfn}{cdocsch}
%    \end{macrocode}

%\iffalse
%</samplefinal>
%\fi
%
% %%%%%%%%%%%%%%%%%%%%%%%%%%%%%%%%%%%%%%
% \paragraph{Command Line Processing.}
%
% The following three command lines generate the output files
% |cdocscld|, |cdocscl1| and |cdocscl2|
% which should be identical to
% |cdocsdrf|, |cdocsch1| and |cdocsfn2|, respectively:
% \begin{center}
% \begin{tabular}{l}
% |latex -jobname cdocscld \|\\
% |  "\def\version{draft}\input{childdoc.def}\childdocforward{cdocsamp}"|\\
% |latex -jobname cdocscl1 \|\\
% |  "\input{childdoc.def}\childdocforward[cdocsamp]{cdocsch1}"|\\
% |latex -jobname cdocscl2 \|\\
% |  "\def\version{final}\input{childdoc.def}\childdocforward{cdocsch2}"|
% \end{tabular}
% \end{center}
% Note that the trailing backslash on each first line
% merely continues the input to the second line
% (for convenient cut ant paste).
% Furthermore, the command |latex| can be replaced by any
% of its alternative versions such as |pdflatex|.
%
% %%%%%%%%%%%%%%%%%%%%%%%%%%%%%%%%%%%%%%%%%%%%%%%%%%%%%%%%%%%%%%%%%%%%%%%%%%%%%%
% %%%%%%%%%%%%%%%%%%%%%%%%%%%%%%%%%%%%%%%%%%%%%%%%%%%%%%%%%%%%%%%%%%%%%%%%%%%%%%
% \section{Implementation}
%\iffalse
%<*package>
%\fi
%
% This section describes the definitions file |childdoc.def|.

% The definitions cannot be loaded using |\usepackage| or |\RequirePackage|
% which has a mechanism to prevent loading a style file more than once.
% When loading the definitions by means of |\input|
% multiple instances have to be prevented manually:
%\iffalse
%This code needs to be before the `\ProvidesFile' directive
%which is defined at the beginning of this file.
%Therefore it is also placed there and commented out here.
%</package>
%<*discard>
%\fi
%    \begin{macrocode}
\ifdefined\childdocmain\endinput\fi
%    \end{macrocode}
%\iffalse
%</discard>
%<*package>
%\fi
%
% \macro{\ifchilddoc}
% \macro{\ifchilddocmanual}
% The conditional |\ifchilddoc| tells whether a
% child (true) or main (false) document is being compiled.
% The conditional |\ifchilddocmanual| tells whether
% the |\includeonly| mechanism is used (false) or
% the selection of child files must be performed manually (true).
% The definitions initialise to false:
%    \begin{macrocode}
\newif\ifchilddoc
\newif\ifchilddocmanual
%    \end{macrocode}

% \macro{\childdocname}
% \macro{\childdocjob}
% The macro |\childdocname| stores the name of the main document
% to be compiled. The macro |\childdocjob| stores the name of
% the document on which the \LaTeX{} compiler was originally invoked.
% The content of |\jobname| cannot be compared
% to filenames specified in the source due to different catcodes.
% The following code rescans |\jobname|, stores the result
% in |\childdocname| and saves a copy in |\childdocjob|:
%    \begin{macrocode}
\edef\childdocname{\scantokens\expandafter{\jobname\noexpand}}
\let\childdocjob\childdocname
%    \end{macrocode}

% \macro{\childdocdisable}
% The macro |\childdocdisable| prevents the main file
% from being processed more than once.
% At this stage, the main document command |\childdocmain|
% is assumed to be called once again where it should do nothing.
% Any subsequent call to it should prevent
% a secondary processing of the main document
% It overwrites the forwarding commands
% |\childdocof| and |\childdocforward|
% with empty macros to prevent further inclusions of the main document:
%    \begin{macrocode}
\newcommand{\childdocdisable}
{
  \renewcommand{\childdocmain}[1]{\renewcommand{\childdocmain}[1]{\endinput}}
  \renewcommand{\childdocof}[1]{}
  \renewcommand{\childdocby}[2][]{}
  \renewcommand{\childdocforward}[2][]{}
  \renewcommand{\childdocdisable}{}
}
%    \end{macrocode}

% \macro{\childdocmain}
% The macro |\childdocmain| is to be called at the top of the main file
% with nothing or the main filename (without extension) as argument.
% First, it breaks loops.
% If the argument is not empty and does not match |\childdocname|
% (which is set by the first inclusion of |childdoc.def|),
% |\ifchilddoc| is set to true, |\includeonly| is applied to the child file
% and |\jobname| is set to the main file
% (for proper handling of |.aux| files):
%    \begin{macrocode}
\newcommand{\childdocmain}[1]
{
  \childdocdisable\childdocmain{}
  \if?#1?\else
    \begingroup
      \def\childdoctmp{#1}
      \ifx\childdoctmp\childdocname
        \def\childdoctmp{}
      \else
        \def\childdoctmp
        {
          \childdoctrue
          \includeonly{\childdocname}
          \def\childdocjob{#1}
          \def\jobname{#1}
        }
      \fi
      \expandafter
    \endgroup
    \childdoctmp
  \fi
}
%    \end{macrocode}

% \macro{\childdocof}
% The command |\childdocof| redirects
% compilation to the main file |#1|.
%    \begin{macrocode}
\newcommand{\childdocof}[1]
{
  \childdocdisable
  \childdoctrue
  \includeonly{\childdocname}
  \def\jobname{#1}
  \def\childdocjob{#1}
  \input{#1}
}
%    \end{macrocode}

% \macro{\childdocby}
% The command |\childdocby| ....
%    \begin{macrocode}
\newcommand{\childdocby}[2][]
{
  \childdocdisable
  \childdoctrue
  \childdocmanualtrue
  \if?#1?\else
    \def\jobname{#2}
  \fi
  \def\childdocjob{#2}
  \input{#2}
  \endinput
}
%    \end{macrocode}

% \macro{\childdocforward}
% The command |\childdocforward| redirects
% compilation to the main file or
% (if the optional argument is given) a child file.
% Parameters are set as if the main file
% or a child file starting with |\childdocof| was compiled.
% Then compilation is handed over to the main file:
%    \begin{macrocode}
\newcommand{\childdocforward}[2][]
{
  \begingroup
    \if?#1?
      \def\childdoctmp
      {
        \def\childdocname{#2}
        \def\childdocjob{#2}
        \def\jobname{#2}
        \input{#2}
        \endinput
      }
    \else
      \def\childdoctmp
      {
        \childdocdisable
        \def\childdocname{#2}
        \childdoctrue
        \includeonly{#2}
        \def\childdocjob{#1}
        \def\jobname{#1}
        \input{#1}
        \endinput
      }
    \fi
    \expandafter
  \endgroup
  \childdoctmp
}
%    \end{macrocode}

% \macro{\childdocforwardprefix}
% The command |\childdocforwardprefix| redirects
% compilation to the main or a child file by means of a pattern.
% The prefix |#1| in the current filename is replaced by |#2|
% and the suffix of the current filename is kept
% (it is assumed that the filename does not contain the substring `|~~~|'
% which is used as a delimiter).
% Compilation is handed over to the new file by |\childdocforward|:
%    \begin{macrocode}
\newcommand{\childdocforwardprefix}[3][]
{
  \begingroup
    \def\childdocextract #2##1~~~{\def\childdoctmp{\childdocforward[#1]{#3##1}}}
    \expandafter\childdocextract\childdocname~~~
    \expandafter
  \endgroup
  \childdoctmp
}
%    \end{macrocode}

% \macro{\childdoc}
% The deprecated macro |\childdoc| is a legacy version of |\childdocmain|:
%    \begin{macrocode}
\newcommand{\childdoc}{\childdocmain}
%    \end{macrocode}

% \macro{\childdocredirect}
% The deprecated macro |\childdocredirect| is a legacy version
% of |\childdocforward| and |\childdocforwardprefix|:
%    \begin{macrocode}
\newcommand{\childdocredirect}[2][]
{
  \begingroup
    \if?#1?
      \def\childdoctmp{\childdocforward{#2}}
    \else
      \def\childdoctmp{\childdocforwardprefix{#1}{#2}}
    \fi
    \expandafter
  \endgroup
  \childdoctmp
}
%    \end{macrocode}

%\iffalse
%</package>
%\fi
%
\endinput
|\\
|\childdocby{|\textit{main}|}|\\
\end{tabular}
\end{center}
%
The directive |\childdocby| is similar to |\childdocof|
described in \secref{sec:include},
but the subsequent selection of content must be done manually.
To that end, both |\ifchilddoc| and |\ifchilddocmanual|
will be true upon processing of a part,
and the name of the part is stored in |\childdocname|.
Note that |\jobname| will be set to the filename of the current part
so that each part receives an individual |.aux| file
that does not interfere with the |.aux| file(s) of the main document.
This behaviour can be altered by the alternative form
|\childdocby[*]{|\textit{main}|}| (with a non-empty optional argument)
which uses the |.aux| file of the main document
by setting |\jobname| to \textit{main}.

%%%%%%%%%%%%%%%%%%%%%%%%%%%%%%%%%%%%%%%%%%%%%%%%%%%%%%%%%%%%%%%%%%%%%%%%%%%%%%%%
\subsection{Driver Development}
\label{sec:driver}

The \textsf{childdoc} mechanism can also be use for the development
of definition files such as \LaTeX{} styles or classes.
This case differs from the above setup with multiple parts
included by |\include| in that no |\includeonly| should be invoked.
This can be achieved by starting the include file
(before |\ProvidesPackage|) with:
%
\begin{center}
\begin{tabular}{l}
|% \iffalse
%
% childdoc.dtx Copyright (C) 2017-2018 Niklas Beisert
%
% This work may be distributed and/or modified under the
% conditions of the LaTeX Project Public License, either version 1.3
% of this license or (at your option) any later version.
% The latest version of this license is in
%   http://www.latex-project.org/lppl.txt
% and version 1.3 or later is part of all distributions of LaTeX
% version 2005/12/01 or later.
%
% This work has the LPPL maintenance status `maintained'.
%
% The Current Maintainer of this work is Niklas Beisert.
%
% This work consists of the files childdoc.dtx and childdoc.ins
% and the derived files childdoc.def and cdocsamp.tex with
% cdocsch1.tex, cdocsch2.tex, cdocsdrf.tex, cdocsfn1.tex, cdocsfn2.tex.
%
%<package>\ifdefined\childdocmain\endinput\fi
%<package>\ProvidesFile{childdoc.def}[2018/12/30 v2.0 child document driver]
%<samplemain>\ProvidesFile{cdocsamp.tex}[2018/12/30 v2.0 sample for childdoc]
%<*driver>
%\ProvidesFile{childdoc.drv}[2018/12/30 v2.0 childdoc reference manual file]
\PassOptionsToClass{10pt,a4paper}{article}
\documentclass{ltxdoc}

\usepackage[margin=35mm]{geometry}
\usepackage{hyperref}
\usepackage{hyperxmp}
\usepackage[usenames]{color}

\hypersetup{colorlinks=true}
\hypersetup{pdfstartview=FitH}
\hypersetup{pdfpagemode=UseNone}
\hypersetup{pdfsource={}}
\hypersetup{pdflang={en-UK}}
\hypersetup{pdfcopyright={Copyright 2017-2018 Niklas Beisert.
  This work may be distributed and/or modified under the
  conditions of the LaTeX Project Public License, either version 1.3
  of this license or (at your option) any later version.}}
\hypersetup{pdflicenseurl={http://www.latex-project.org/lppl.txt}}
\hypersetup{pdfcontactaddress={ETH Zurich, ITP, HIT K,
  Wolfgang-Pauli-Strasse 27}}
\hypersetup{pdfcontactpostcode={8093}}
\hypersetup{pdfcontactcity={Zurich}}
\hypersetup{pdfcontactcountry={Switzerland}}
\hypersetup{pdfcontactemail={nbeisert@itp.phys.ethz.ch}}
\hypersetup{pdfcontacturl={http://people.phys.ethz.ch/\xmptilde nbeisert/}}

\newcommand{\secref}[1]{\hyperref[#1]{section \ref*{#1}}}

\parskip1ex
\parindent0pt
\let\olditemize\itemize
\def\itemize{\olditemize\parskip0pt}

\begin{document}

\title{The \textsf{childdoc} Package}
\hypersetup{pdftitle={The childdoc Package}}
\author{Niklas Beisert\\[2ex]
  Institut f\"ur Theoretische Physik\\
  Eidgen\"ossische Technische Hochschule Z\"urich\\
  Wolfgang-Pauli-Strasse 27, 8093 Z\"urich, Switzerland\\[1ex]
  \href{mailto:nbeisert@itp.phys.ethz.ch}
  {\texttt{nbeisert@itp.phys.ethz.ch}}}
\hypersetup{pdfauthor={Niklas Beisert}}
\hypersetup{pdfsubject={Manual for the LaTeX2e Package childdoc}}
\date{30 December 2018, \textsf{v2.0}}
\maketitle

\begin{abstract}\noindent
\textsf{childdoc} is a \LaTeXe{} package
that enables the direct compilation
of document sections included by |\include|
to individual files.
\end{abstract}

\begingroup
\parskip0ex
\tableofcontents
\endgroup

%%%%%%%%%%%%%%%%%%%%%%%%%%%%%%%%%%%%%%%%%%%%%%%%%%%%%%%%%%%%%%%%%%%%%%%%%%%%%%%%
%%%%%%%%%%%%%%%%%%%%%%%%%%%%%%%%%%%%%%%%%%%%%%%%%%%%%%%%%%%%%%%%%%%%%%%%%%%%%%%%
\section{Introduction}

\LaTeX{} provides a mechanism to structure a large document (such as a book)
into a main file and several child files (containing the chapters)
using the |\include| command.
This mechanism is beneficial for documents
which span hundreds of pages in order to
make the source file(s) more manageable.
Moreover, compilation can be restricted to
selected child files by means of the |\includeonly| command.
The latter feature can be used to reduce the compilation time while editing
(this was significantly more useful in the earlier days of \LaTeX{})
or to generate a smaller document which is easier to navigate.
Another application of |\includeonly| is to generate
documents consisting of selected parts of the complete document.

However, there are a few drawbacks of the plain |\include| mechanism:
\begin{itemize}
\item
The child files cannot be compiled on their own,
they can only be compiled via the main file.
A naive editing environment
(such as a text editor with an option
to have the current file processed by \LaTeX)
may require one to switch to the main file before compiling;
attempting to compile the child file produces errors.
\item
The main file must be modified (each time)
to adjust the |\includeonly| command
to the present needs. This easily leaves the main file in a messy state.
\item
The generated document will always carry the filename
of the main document. This is inconvenient if
several child files are to be compiled and
to be kept for distribution.
\end{itemize}

The present package provides a simple interface
to make child files individually compilable by \LaTeX{}.
Compiling a child file then has the same effect as compiling
the main file with an |\includeonly| command
to select the appropriate child.
Moreover the generated document will carry the name of the child
rather than the main file.
This resolves all three above issues.

This feature is meant to make the editing of books,
thesis documents and lecture notes somewhat more convenient.
However, the package can also be used efficiently for
composing a series of documents (such as exercise sheets)
which are typically distributed individually.
It then assists the author in generating the individual documents
(potentially in different versions)
as well as a document containing the collected series.
Another application is in developing style files
or other kinds of included material
where compilation of the style file could redirect
to a sample or test file.

%%%%%%%%%%%%%%%%%%%%%%%%%%%%%%%%%%%%%%%%%%%%%%%%%%%%%%%%%%%%%%%%%%%%%%%%%%%%%%%%
%%%%%%%%%%%%%%%%%%%%%%%%%%%%%%%%%%%%%%%%%%%%%%%%%%%%%%%%%%%%%%%%%%%%%%%%%%%%%%%%
\section{Usage}

First of all, the package \textsf{childdoc} is \emph{not} a standard
\LaTeXe{} |.sty| style file! Therefore it needs to be invoked in
a non-standard way.

%%%%%%%%%%%%%%%%%%%%%%%%%%%%%%%%%%%%%%%%%%%%%%%%%%%%%%%%%%%%%%%%%%%%%%%%%%%%%%%%
\subsection{Included Files}
\label{sec:include}

%%%%%%%%%%%%%%%%%%%%%%%%%%%%%%%%%%%%%%%%
\DescribeMacro{\childdocmain}
To use the package, add the commands
\begin{center}
\begin{tabular}{l}
|\input{childdoc.def}|\\
|\childdocmain{}|\\
\end{tabular}
\end{center}
at the very top of the main \LaTeX{} file,
in particular \emph{before} the |\documentclass| statement!
The argument of |\childdocmain| should be left empty
(but it must be present).

%%%%%%%%%%%%%%%%%%%%%%%%%%%%%%%%%%%%%%%%
\DescribeMacro{\childdocof}
Furthermore, add the commands
\begin{center}
\begin{tabular}{l}
|\input{childdoc.def}|\\
|\childdocof{|\textit{main}|}|\\
\end{tabular}
\end{center}
at the top of every child file \textit{child}
which is included by |\include{|\textit{child}|}|
from within the main file
(or at least for those files to be compiled individually).
The argument \textit{main} must be the filename of the main file.

There are a couple of
considerations in setting up the main and child documents:

%%%%%%%%%%%%%%%%%%%%%%%%%%%%%%%%%%%%%%%%
\paragraph{Restrictions.}

Please note the following restrictions:
\begin{itemize}
\item
|\childdocmain| must be called with one argument \textit{main}
to ensure compatibility with earlier version of the package.
It must either be empty (|\childdocmain{}|)
or precisely match the filename of the main file in which it is specified.
See \secref{sec:detection} for further information.
\item
The filename \textit{main} must be specified without the |.tex| extension.
\item
The filename \textit{main} is case sensitive
(even in case-insensitive file systems)
due to internal string comparison.
\item
The argument \textit{main} should be fully expanded, it cannot be a macro.
\item
Subdirectories and special characters should be avoided in filenames.
\item
The command |\childdocmain{|\textit{main}|}| must be followed by a whitespace.
It should not be followed immediately by another command
or by a comment mark `|%|'.
This is because the \TeX{} parser reads the token immediately following
the argument of |\childdocmain| and puts it
at the beginning of every child section;
however, a white\-space is ignored.
\end{itemize}

%%%%%%%%%%%%%%%%%%%%%%%%%%%%%%%%%%%%%%%%
\paragraph{Content of Main File.}

It is advisable to place all content in the child files included by |\include|.
Any output contained in the main file will appear in all child documents
unless suppressed manually;
it cannot be suppressed automatically by the |\includeonly| directive
and thus should normally be avoided.
A method to include some content in the main file
by means of conditional processing is described in \secref{sec:conditional}.

%%%%%%%%%%%%%%%%%%%%%%%%%%%%%%%%%%%%%%%%
\paragraph{Page Numbering.}

When only a part of the document is compiled,
the appropriate numbering of pages
(as well as other status parameters)
is determined from the |.aux| files.
The latter contain information from previous passes.
However this information needs to propagate through
all intermediate child documents.
Therefore the page numbering in child documents may well
be inconsistent until the complete document is compiled at least once.

A useful (if unconventional) way to always ensure a consistent
page numbering is to restart the numbering in each child document
and denote the pages by `\textit{child}|.|\textit{page}'
where \textit{child} represents the chapter/section number of the child file.
This can be achieved by the command
|\numberwithin{page}{|\textit{child}|}|
of the \textsf{amsmath} package
where \textit{child} can be |chapter| or |section|
depending on the chosen structuring.
Alternatively, one can modify the macro |\thepage| appropriately
and reset the counter |page| at the start of each child file.

%%%%%%%%%%%%%%%%%%%%%%%%%%%%%%%%%%%%%%%%%%%%%%%%%%%%%%%%%%%%%%%%%%%%%%%%%%%%%%%%
\subsection{Conditional Processing}
\label{sec:conditional}

The package provides a mechanism to compile different versions
of a document. To customise the versions further some conditional processing
can come in handy to distinguish which version is being compiled.
The package provides two macros to describe the compilation context:

%%%%%%%%%%%%%%%%%%%%%%%%%%%%%%%%%%%%%%%%
\DescribeMacro{\ifchilddoc}
The conditional |\ifchilddoc| distinguishes between the compilation of
child documents and the main document:
%
\begin{center}
|\ifchilddoc |\textit{child-code}| |[|\||else |\textit{main-code}]| \||fi|
\end{center}

%%%%%%%%%%%%%%%%%%%%%%%%%%%%%%%%%%%%%%%%
\DescribeMacro{\childdocname}
\DescribeMacro{\childdocjob}
The macro |\childdocname| contains the filename (without extension)
of the main or child file being processed.
Note that |\childdocjob| will always contain the name of the main file.

%%%%%%%%%%%%%%%%%%%%%%%%%%%%%%%%%%%%%%%%
\paragraph{Title Page.}

Conditional processing can be used to include a title or banner page
in the main document when proper precautions are taken.
Importantly, the code in the main file should ensure that the page counter
(as well as other status parameters which are stored in the |.aux| files)
takes the same value after the conditional processing.
Otherwise the page numbers may take divergent values
depending on which part is compiled.

For example, a title page could be declared by:
%
\begin{center}
\begin{tabular}{l}
|\ifchilddoc\||else|\\
|\addtocounter{page}{-1}|\\
\textit{code for title page}\\
|\newpage|\\
|\||fi|
\end{tabular}
\end{center}
%
A banner page for the child documents can be generated by:
%
\begin{center}
\begin{tabular}{l}
|\ifchilddoc|\\
|\addtocounter{page}{-1}|\\
\textit{code for banner page}\\
|\newpage|\\
|\||fi|
\end{tabular}
\end{center}
%
Here one could write a message such as:
\begin{center}
|This is the part \childdocname{} of \childdocjob{}.|
\end{center}

%%%%%%%%%%%%%%%%%%%%%%%%%%%%%%%%%%%%%%%%%%%%%%%%%%%%%%%%%%%%%%%%%%%%%%%%%%%%%%%%
\subsection{Flags}
\label{sec:flags}

The package makes it easy to generate different versions
of the main or child documents.
To this end compilation flags can be defined
and assigned different default values.
They will be particularly useful in conjunction
with the forwarding mechanism described in \secref{sec:forward}.

For example, it may be useful to have a flag |\version|
which can be set to |draft| or |final|.
The document source will contain some conditional code
depending on the value of |\version|.
Suppose further, the flag should default to |final| for the main file
and to |draft| for child files
which is a natural assignment for editing the document.
This is achieved by placing the following code
in the preamble of the main document
(below the |\childdocmain| directive):
%
\begin{center}
\begin{tabular}{l}
|\ifchilddoc|\\
|\providecommand{\version}{draft}|\\
|\||else|\\
|\providecommand{\version}{final}|\\
|\||fi|
\end{tabular}
\end{center}
%
The definition by |\providecommand| makes sure
that previous definitions are not overwritten.
Further statements |\providecommand{\version}{...}|
can thus be added before the above code to override it.

For the main file, one might add a line
(between |\childdocmain| and the above block)
%
\begin{center}
|%\ifchilddoc\||else\providecommand{\version}{draft}\||fi|
\end{center}
%
which can be uncommented to produce a draft version.
Likewise one can add a line to the very top of a child file
(above the |\childdocof{|\textit{main}|}| directive)
%
\begin{center}
|%\providecommand{\version}{final}|
\end{center}
%
which can be uncommented to produce the final version of this child document.

%%%%%%%%%%%%%%%%%%%%%%%%%%%%%%%%%%%%%%%%%%%%%%%%%%%%%%%%%%%%%%%%%%%%%%%%%%%%%%%%
\subsection{Forwarding}
\label{sec:forward}

Different versions of the main or child documents
using compilation flags as described in \secref{sec:flags}
can be (permanently) stored in different files
for convenient compilation, viewing and distribution.
To this end, the package defines a command
to pass on compilation to a different file:

%%%%%%%%%%%%%%%%%%%%%%%%%%%%%%%%%%%%%%%%
\DescribeMacro{\childdocforward}
The command |\childdocforward| redirects processing to
another source file:
%
\begin{center}
\begin{tabular}{l}
|\input{childdoc.def}|\\
|\childdocforward[|\textit{main}|]{|\textit{dest}|}|\\
\end{tabular}
\end{center}
%
The argument \textit{dest} is the destination file
(without extension).
It should be the main file or one of the child files.
Note that further \textsf{childdoc} directives
such as |\childdocof| and |\childdocforward|
in the indicated file will be processed in this form.
The optional argument \textit{main}
passes on directly to the main file \textit{main}
while pretending to compile the child \textit{dest}.
This form behaves as if \textit{dest}
issues |\childdocof{|\textit{main}|}| right away,
and no further \textsf{childdoc} directives will be processed.

%%%%%%%%%%%%%%%%%%%%%%%%%%%%%%%%%%%%%%%%
\DescribeMacro{\...prefix}
In the alternative form |\childdocforwardprefix|,
%
\begin{center}
\begin{tabular}{l}
|\input{childdoc.def}|\\
|\childdocforwardprefix[|\textit{main}|]{|\textit{prefix}|}{|\textit{dest}|}|
\end{tabular}
\end{center}
%
the destination file is determined by a pattern
depending on the current file:
To make this work, the current file must be called
`{\textit{prefix}\hspace{0.2em}\textit{suffix}}'
with \textit{prefix} matching precisely the argument.
Processing is then passed on to the file
`{\textit{dest}\hspace{0.2em}\textit{suffix}}'.
Surely, the same effect is achieved by
directly specifying the
argument `{\textit{dest}\hspace{0.2em}\textit{suffix}}'
in the first form.
However, that requires to set up a different file
for each child. With the alternative form of the command
all these files can have exactly the same content
which simplifies setting them up and maintaining them.

For example, the following file |draft.tex|
with a compilation flag |\version| as described in \secref{sec:flags}
compiles the main document as a draft:
%
\begin{center}
\begin{tabular}{l}
|\def\version{draft}|\\
|\input{childdoc.def}|\\
|\childdocforward{|\textit{main}|}|
\end{tabular}
\end{center}
%
Likewise, the following files |final|\textit{nn}|.tex|
compile the final version of the child document
|child|\textit{nn}|.tex|:
%
\begin{center}
\begin{tabular}{l}
|\def\version{final}|\\
|\input{childdoc.def}|\\
|\childdocforwardprefix{final}{child}|
\end{tabular}
\end{center}
%

Note that when several versions of a main file and/or of each child file
are to be generated, it may be convenient to set up a |Makefile| or
shell script to automatise the process.

%%%%%%%%%%%%%%%%%%%%%%%%%%%%%%%%%%%%%%%%%%%%%%%%%%%%%%%%%%%%%%%%%%%%%%%%%%%%%%%%
\subsection{Command Line Processing}
\label{sec:commandline}

The effect of redirection files can also be achieved by invoking
the \LaTeX{} compiler with a more elaborate command line.
Most conveniently this should be done as part
of a shell script or a |Makefile|.

When using \textsf{childdoc} in the main file, the following
command lines effectively perform a redirection
(note that depending on the shell being used,
backslashes may have to be doubled: `|\|' $\to$ `|\\|'):
%
\begin{center}
|... -jobname "|\textit{target}|" |\\|"|[\textit{flags}]%
|\input{childdoc.def}\childdocforward[|\textit{main}|]{|\textit{dest}|}"|
\end{center}
%
Here \textit{target} is the name of the output file,
\textit{main} is the name of the main file
and \textit{dest} is the name of the main or child file to be processed
(all filenames without extensions).
The optional argument \textit{main} can be omitted
if \textit{main} matches \textit{dest}.
Optionally, compilation \textit{flags} can be defined via |\def| commands.
This command line makes the \TeX{} engine believe
it is compiling the file \textit{target}
whose content is specified as the latter parameter.
The provided code then forwards the processing to
\textit{main} or \textit{dest} as described in \secref{sec:forward}.

%%%%%%%%%%%%%%%%%%%%%%%%%%%%%%%%%%%%%%%%%%%%%%%%%%%%%%%%%%%%%%%%%%%%%%%%%%%%%%%%
\subsection{Include by Input}
\label{sec:input}

Including child documents by |\include| has some restrictions by design.
Most notably, the content of a child document always occupies
its own set of pages; pages cannot be shared between child documents.
Usually, this behaviour makes perfect sense
because each child document contain an essential part of the document.
However, in some situations it may be desirable to compose
a document from a collection of parts
without having mandatory page breaks between then.
For this case, the package
provides a mechanism to include parts
by |\input| which can also be processed individually.
However, by construction this mechanism
requires manual handling of the content to be output.

%%%%%%%%%%%%%%%%%%%%%%%%%%%%%%%%%%%%%%%%
\DescribeMacro{\ifchilddocmanual}
The main file should be prepared as usual, see \secref{sec:include}.
However, the document body must make a distinction
between processing of an individual part and of the main document, e.g.:
%
\begin{center}
\begin{tabular}{l}
|\ifchilddocmanual|\\
|\input{\childdocname}|\\
|\||else|\\
\textit{document body with }|\input{|\textit{part}|}|\\
|\||fi|
\end{tabular}
\end{center}
%
The conditional |\ifchilddocmanual| is true whenever
a part to be included by |\input| is being compiled,
and the name of the part is stored in |\childdocname|.

%%%%%%%%%%%%%%%%%%%%%%%%%%%%%%%%%%%%%%%%
\DescribeMacro{\childdocby}
Each part to be included by |\input| should start with:
%
\begin{center}
\begin{tabular}{l}
|\input{childdoc.def}|\\
|\childdocby{|\textit{main}|}|\\
\end{tabular}
\end{center}
%
The directive |\childdocby| is similar to |\childdocof|
described in \secref{sec:include},
but the subsequent selection of content must be done manually.
To that end, both |\ifchilddoc| and |\ifchilddocmanual|
will be true upon processing of a part,
and the name of the part is stored in |\childdocname|.
Note that |\jobname| will be set to the filename of the current part
so that each part receives an individual |.aux| file
that does not interfere with the |.aux| file(s) of the main document.
This behaviour can be altered by the alternative form
|\childdocby[*]{|\textit{main}|}| (with a non-empty optional argument)
which uses the |.aux| file of the main document
by setting |\jobname| to \textit{main}.

%%%%%%%%%%%%%%%%%%%%%%%%%%%%%%%%%%%%%%%%%%%%%%%%%%%%%%%%%%%%%%%%%%%%%%%%%%%%%%%%
\subsection{Driver Development}
\label{sec:driver}

The \textsf{childdoc} mechanism can also be use for the development
of definition files such as \LaTeX{} styles or classes.
This case differs from the above setup with multiple parts
included by |\include| in that no |\includeonly| should be invoked.
This can be achieved by starting the include file
(before |\ProvidesPackage|) with:
%
\begin{center}
\begin{tabular}{l}
|\input{childdoc.def}|\\
|\childdocforward{|\textit{main}|}|\\
\end{tabular}
\end{center}
%
or alternatively with:
%
\begin{center}
\begin{tabular}{l}
|\input{childdoc.def}|\\
|\childdocby{|\textit{main}|}|\\
\end{tabular}
\end{center}
%
Both forms have slightly different effects as described above.
The main file is prepared as usual, see \secref{sec:include}.

%%%%%%%%%%%%%%%%%%%%%%%%%%%%%%%%%%%%%%%%%%%%%%%%%%%%%%%%%%%%%%%%%%%%%%%%%%%%%%%%
\subsection{Legacy Detection}
\label{sec:detection}

The directive |\childdocmain| in the main file can detect
whether the complete document or merely a child is to be compiled
even without using the directive |\childdocof|.
This method is deprecated because it is less robust
and there is no compelling reason to use it;
it is merely provided for backward compatibility
and it may be removed in future versions.

If the detection mechanism is to be used,
it is mandatory to correctly specify
the filename of the main file as the argument of |\childdocmain|:
%
\begin{center}
\begin{tabular}{l}
|\input{childdoc.def}|\\
|\childdocmain{|\textit{main}|}|\\
\end{tabular}
\end{center}
%
If |\jobname| does not match the argument \textit{main} of |\childdocmain|,
it is assumed that |\jobname| points to the child file to be compiled.
When using |\childdocmain| with the main file specified as argument,
it suffices to start a child file
with just |\input{|\textit{main}|}|
without loading of the package and using |\childdocof|.
If instead all processing is done
with the appropriate \textsf{childdoc} directives,
the argument of \textit{main} of |\childdocmain| can be empty.

An alternative version of the command line processing described
in \secref{sec:commandline} using the detection mechanism reads:
%
\begin{center}
|... -jobname "|\textit{target}|" "|[\textit{flags}]%
[|\def\jobname{|\textit{dest}|}|]|\input{|\textit{main}|}"|
\end{center}

%%%%%%%%%%%%%%%%%%%%%%%%%%%%%%%%%%%%%%%%%%%%%%%%%%%%%%%%%%%%%%%%%%%%%%%%%%%%%%%%
\subsection{Manual Code}
\label{sec:manual}

In case one cannot be certain whether the definitions file |childdoc.def|
is installed on the target \TeX{} distribution
and one prefers not to ship it,
it is conceivable to paste a few relevant commands into the sources.

To that end, drop all statements |\input{childdoc.def}|
and perform the replacements as outlined below.
Instead of |\childdocmain{|\textit{main}|}| add the following code
to the top of the main file:
%
\begin{center}
\begin{tabular}{l}
|\||ifdefined\childdocname\endinput\||fi\newif\ifchilddoc|\\
|\edef\childdocname{\scantokens\expandafter{\jobname\noexpand}}|\\
|\def\childdocmain{|\textit{main}|}\||ifx\childdocmain\childdocname\||else|\\
|\childdoctrue\includeonly{\childdocname}\let\jobname\childdocmain\||fi|\\
\end{tabular}
\end{center}
%
Instead of |\childdocof{|\textit{main}|}| just include the main file
at the top of each child file:
%
\begin{center}
|\input{|\textit{main}|}|
\end{center}
%
A simple redirection |\childdocforward{|\textit{dest}|}| is achieved by:
%
\begin{center}
|\def\jobname{|\textit{dest}|}\input{\jobname}|
\end{center}
%
The redirection with prefix
|\childdocforwardprefix[|\textit{prefix}|]{|\textit{dest}|}|
is accomplished by:
%
\begin{center}
\begin{tabular}{l}
|{\edef\jobname{\scantokens\expandafter{\jobname\noexpand}}|\\
|\def\redirectjob |\textit{prefix}|#1~~~{\gdef\jobname{|\textit{dest}|#1}}|\\
|\expandafter\redirectjob\jobname~~~}\input{\jobname}|
\end{tabular}
\end{center}

In an alternative approach,
child documents can be compiled by a specific command line
without additional code or specific definitions:
%
\begin{center}
|... -jobname "|\textit{target}|" "|[\textit{flags}]%
|\includeonly{|\textit{dest}|}\input{|\textit{main}|}"|
\end{center}
%

%%%%%%%%%%%%%%%%%%%%%%%%%%%%%%%%%%%%%%%%%%%%%%%%%%%%%%%%%%%%%%%%%%%%%%%%%%%%%%%%
%%%%%%%%%%%%%%%%%%%%%%%%%%%%%%%%%%%%%%%%%%%%%%%%%%%%%%%%%%%%%%%%%%%%%%%%%%%%%%%%
\section{Information}

%%%%%%%%%%%%%%%%%%%%%%%%%%%%%%%%%%%%%%%%%%%%%%%%%%%%%%%%%%%%%%%%%%%%%%%%%%%%%%%%
\subsection{Copyright}

Copyright \copyright{} 2017--2018 Niklas Beisert

This work may be distributed and/or modified under the
conditions of the \LaTeX{} Project Public License, either version 1.3
of this license or (at your option) any later version.
The latest version of this license is in
  \url{http://www.latex-project.org/lppl.txt}
and version 1.3 or later is part of all distributions of \LaTeX{}
version 2005/12/01 or later.

This work has the LPPL maintenance status `maintained'.

The Current Maintainer of this work is Niklas Beisert.

This work consists of the files |README.txt|, |childdoc.ins| and |childdoc.dtx|
as well as the derived files |childdoc.def|, |cdocsamp.tex|
with |cdocsch1.tex|, |cdocsch2.tex|, |cdocspt3.tex|, |cdocspt4.tex|,
|cdocsdrf.tex|, |cdocsfn1.tex|, |cdocsfn2.tex|
as well as |childdoc.pdf|.

%%%%%%%%%%%%%%%%%%%%%%%%%%%%%%%%%%%%%%%%%%%%%%%%%%%%%%%%%%%%%%%%%%%%%%%%%%%%%%%%
\subsection{Files and Installation}

The package consists of the files:
%
\begin{center}
\begin{tabular}{ll}
    |README.txt|   & readme file \\
    |childdoc.ins| & installation file \\
    |childdoc.dtx| & source file \\
    |childdoc.def| & definition file \\
    |cdocsamp.tex| & sample main file \\
    |cdocsch1.tex| & sample include file \\
    |cdocsch2.tex| & sample include file \\
    |cdocspt3.tex| & sample part file \\
    |cdocspt4.tex| & sample part file \\
    |cdocsdrf.tex| & sample redirection file \\
    |cdocsfn1.tex| & sample redirection file \\
    |cdocsfn2.tex| & sample redirection file \\
    |childdoc.pdf| & manual
\end{tabular}
\end{center}
%
The distribution consists of the files
|README.txt|, |childdoc.ins| and |childdoc.dtx|.
%
\begin{itemize}
\item
Run (pdf)\LaTeX{} on |childdoc.dtx|
to compile the manual |childdoc.pdf| (this file).
\item
Run \LaTeX{} on |childdoc.ins| to create the definitions file |childdoc.def|
and the sample |cdocsamp.tex| with include files
|cdocsch1.tex|, |cdocsch2.tex|, |cdocspt3.tex|, |cdocspt4.tex|,
|cdocsdrf.tex|, |cdocsfn1.tex|, |cdocsfn2.tex|.
Then copy the file |childdoc.def| to an appropriate directory of your \LaTeX{}
distribution, e.g.\ \textit{texmf-root}|/tex/latex/childdoc|.
\end{itemize}

%%%%%%%%%%%%%%%%%%%%%%%%%%%%%%%%%%%%%%%%%%%%%%%%%%%%%%%%%%%%%%%%%%%%%%%%%%%%%%%%
\subsection{Related CTAN Packages}

There are several other packages which offer a similar functionality:
%
\begin{itemize}
\item
The packages
\href{http://ctan.org/pkg/docmute}{\textsf{docmute}},
\href{http://ctan.org/pkg/includex}{\textsf{includex}} and
\href{http://ctan.org/pkg/standalone}{\textsf{standalone}}
provide commands to include only the document body of
a child file thus allowing both files to be compiled individually.
\item
The packages \href{http://ctan.org/pkg/subdocs}{\textsf{subdocs}}
and \href{http://ctan.org/pkg/subfiles}{\textsf{subfiles}}
provide structures in which the main and child documents can be
encapsulated and allowing them to be compiled individually.
The inclusion mechanism is different from the conventional |\include|.
\item
The package \href{http://ctan.org/pkg/combine}{\textsf{combine}}
is an elaborate solution to combine several documents into one.
\end{itemize}
%
See also the CTAN topic \href{http://ctan.org/topic/subdocs}{\textsf{subdocs}}
for further related packages.
The present package differs from the above solutions in that
a document structure constructed with the conventional |\include| mechanism
just needs two extra commands at the top of every file
such that all constituent files can be compiled individually.

%%%%%%%%%%%%%%%%%%%%%%%%%%%%%%%%%%%%%%%%%%%%%%%%%%%%%%%%%%%%%%%%%%%%%%%%%%%%%%%%
%\subsection{Feature Suggestions}
%
%The following is a list of features which may be useful for future
%versions of this package:
%%
%\begin{itemize}
%\item
%\ldots
%\end{itemize}

%%%%%%%%%%%%%%%%%%%%%%%%%%%%%%%%%%%%%%%%%%%%%%%%%%%%%%%%%%%%%%%%%%%%%%%%%%%%%%%%
\subsection{Revision History}

%%%%%%%%%%%%%%%%%%%%%%%%%%%%%%%%%%%%%%%%
\paragraph{v2.0:} 2018/12/30

\begin{itemize}
\item
immediate forward processing
\item
added |\childdocby| mechanism
\item
manual restructured
\end{itemize}

%%%%%%%%%%%%%%%%%%%%%%%%%%%%%%%%%%%%%%%%
\paragraph{v1.6:} 2018/01/17

\begin{itemize}
\item
application for development of include files
\item
corrections to manual
\end{itemize}

%%%%%%%%%%%%%%%%%%%%%%%%%%%%%%%%%%%%%%%%
\paragraph{v1.5:} 2017/05/21

\begin{itemize}
\item
more complete structuring introduced
\item
|\childdocof| introduced
\item
|\childdoc| renamed to |\childdocmain|
\item
|\childredirect| renamed to |\childdocforward| and |\childdocforwardprefix|
and functionality expanded
\end{itemize}

%%%%%%%%%%%%%%%%%%%%%%%%%%%%%%%%%%%%%%%%
\paragraph{v1.0:} 2017/04/27

\begin{itemize}
\item
manual and install package
\item
first version published on CTAN
\end{itemize}

%%%%%%%%%%%%%%%%%%%%%%%%%%%%%%%%%%%%%%%%
\paragraph{v0.6:} 2017/04/26

\begin{itemize}
\item
redirection mechanism added
\end{itemize}

%%%%%%%%%%%%%%%%%%%%%%%%%%%%%%%%%%%%%%%%
\paragraph{v0.5:} 2017/04/26

\begin{itemize}
\item
functionality in definition file
\end{itemize}


%%%%%%%%%%%%%%%%%%%%%%%%%%%%%%%%%%%%%%%%%%%%%%%%%%%%%%%%%%%%%%%%%%%%%%%%%%%%%%%%
%%%%%%%%%%%%%%%%%%%%%%%%%%%%%%%%%%%%%%%%%%%%%%%%%%%%%%%%%%%%%%%%%%%%%%%%%%%%%%%%
%%%%%%%%%%%%%%%%%%%%%%%%%%%%%%%%%%%%%%%%%%%%%%%%%%%%%%%%%%%%%%%%%%%%%%%%%%%%%%%%
\appendix

\settowidth\MacroIndent{\rmfamily\scriptsize 000\ }

 \DocInput{childdoc.dtx}

\end{document}
%</driver>
% \fi
%
% %%%%%%%%%%%%%%%%%%%%%%%%%%%%%%%%%%%%%%%%%%%%%%%%%%%%%%%%%%%%%%%%%%%%%%%%%%%%%%
% %%%%%%%%%%%%%%%%%%%%%%%%%%%%%%%%%%%%%%%%%%%%%%%%%%%%%%%%%%%%%%%%%%%%%%%%%%%%%%
% \section{Sample}
%\iffalse
%<*samplemain>
%\fi
%
% The following presents a sample document
% with two chapters, two parts, a title page,
% a compile flag as well as three forwarding files to set the flag.
% It consists of eight |.tex| files:
% \begin{center}
% \begin{tabular}{ll}
% |cdocsamp.tex|&main file\\
% |cdocsch1.tex|&include file for chapter 1\\
% |cdocsch2.tex|&include file for chapter 2\\
% |cdocspt3.tex|&include file for part 3\\
% |cdocspt4.tex|&include file for part 4\\
% |cdocsdrf.tex|&forwarding file for main file in draft mode\\
% |cdocsfi1.tex|&forwarding file for final version of chapter 1\\
% |cdocsfi2.tex|&forwarding file for final version of chapter 2\\
% \end{tabular}
% \end{center}
% Each of the eight files can be compiled directly by the \LaTeX{} compiler.
%
% %%%%%%%%%%%%%%%%%%%%%%%%%%%%%%%%%%%%%%
% \paragraph{Main File.}
%
% The main file is called |cdocsamp.tex|.
%
% Load the \textsf{childdoc} definitions and
% declare the filename for the main document:
%    \begin{macrocode}
\input{childdoc.def}
\childdocmain{}
%    \end{macrocode}

% Optional override for |\version| flag:
%    \begin{macrocode}
%%\ifchilddoc\else\providecommand{\version}{draft}\fi
%    \end{macrocode}

% Define the default values for the |\version| flag
% (|final| for the main file and |draft| for childs):
%    \begin{macrocode}
\ifchilddoc
\providecommand{\version}{draft}
\else
\providecommand{\version}{final}
\fi
%    \end{macrocode}

% Load the standard document class:
%    \begin{macrocode}
\documentclass[12pt]{article}
%    \end{macrocode}

% Start the document body:
%    \begin{macrocode}
\begin{document}
%    \end{macrocode}

% Declare a title page.
% Print title, part of document being processed and version flag:
%    \begin{macrocode}
\addtocounter{page}{-1}
\begin{center}
{\LARGE\bfseries{}childdoc example\par}
\vspace{1cm}
\ifchilddoc
\ifchilddocmanual part\else chapter\fi:
`\childdocname' of `\childdocjob'\par
\else
main document: `\childdocjob'\par
\fi
version: \version\par
\end{center}
\newpage
%    \end{macrocode}

% Manually include selected file,
% otherwise process as usual:
%    \begin{macrocode}
\ifchilddocmanual
\section*{part `\childdocname'}
\input{\childdocname}
\else
%    \end{macrocode}

% Include the two chapters:
%    \begin{macrocode}
\include{cdocsch1}
\include{cdocsch2}
%    \end{macrocode}

% Include the two parts unless only chapters should be displayed:
%    \begin{macrocode}
\ifchilddoc\else
\section{part three}
\input{cdocspt3}
\section{part four}
\input{cdocspt4}
\fi
%    \end{macrocode}

% Process as usual until here:
%    \begin{macrocode}
\fi
%    \end{macrocode}

% End of document body:
%    \begin{macrocode}
\end{document}
%    \end{macrocode}
%\iffalse
%</samplemain>
%\fi
%
% %%%%%%%%%%%%%%%%%%%%%%%%%%%%%%%%%%%%%%
% \paragraph{Chapter Include Files.}
%
% The include files are called |cdocsch1.tex| and |cdocsch2.tex|.
%
%\iffalse
%<*samplechap1|samplechap2>
%\fi

% Optional override for |\version| flag:
%    \begin{macrocode}
%%\providecommand{\version}{final}
%    \end{macrocode}

% Include the main document:
%    \begin{macrocode}
\input{childdoc.def}
\childdocof{cdocsamp}
%    \end{macrocode}

%\iffalse
%</samplechap1|samplechap2>
%\fi
%
%\iffalse
%<*samplechap1>
%\fi
% Some text for chapter 1:
%    \begin{macrocode}
\section{one}
some text in chapter one
%    \end{macrocode}

%\iffalse
%</samplechap1>
%\fi
% Some text for chapter 2:
%\iffalse
%<*samplechap2>
%\fi
%    \begin{macrocode}
\section{two}
more text in chapter two
%    \end{macrocode}

%\iffalse
%</samplechap2>
%\fi
%
% %%%%%%%%%%%%%%%%%%%%%%%%%%%%%%%%%%%%%%
% \paragraph{Part Include Files.}
%
% The include files are called |cdocspt3.tex| and |cdocspt4.tex|.
%
%\iffalse
%<*samplepart3|samplepart4>
%\fi

% Optional override for |\version| flag:
%    \begin{macrocode}
%%\providecommand{\version}{final}
%    \end{macrocode}

% Include the main document:
%    \begin{macrocode}
\input{childdoc.def}
\childdocby{cdocsamp}
%    \end{macrocode}

%\iffalse
%</samplepart3|samplepart4>
%\fi
%
%\iffalse
%<*samplepart3>
%\fi
% Some text for part 3:
%    \begin{macrocode}
some text in part three
%    \end{macrocode}

%\iffalse
%</samplepart3>
%\fi
% Some text for part 4:
%\iffalse
%<*samplepart4>
%\fi
%    \begin{macrocode}
more text in part four
%    \end{macrocode}

%\iffalse
%</samplepart4>
%\fi
%
% %%%%%%%%%%%%%%%%%%%%%%%%%%%%%%%%%%%%%%
% \paragraph{Forwarding for a Complete Draft.}
%
% The following forwarding file |cdocsdrf.tex|
% compiles the main document in draft mode:
%\iffalse
%<*sampledraft>
%\fi
%    \begin{macrocode}
\def\version{draft}
\input{childdoc.def}
\childdocforward{cdocsamp}
%    \end{macrocode}

%\iffalse
%</sampledraft>
%\fi
%
% %%%%%%%%%%%%%%%%%%%%%%%%%%%%%%%%%%%%%%
% \paragraph{Forwarding for Final Version of the Chapters.}
%
% The following forwarding files |cdocsfn1.tex| and |cdocsfn2.tex|
% (with identical content)
% compile the final versions of the child documents
% |cdocsch1.tex| and |cdocsch2.tex|, respectively:
%\iffalse
%<*samplefinal>
%\fi
%    \begin{macrocode}
\def\version{final}
\input{childdoc.def}
\childdocforwardprefix[cdocsamp]{cdocsfn}{cdocsch}
%    \end{macrocode}

%\iffalse
%</samplefinal>
%\fi
%
% %%%%%%%%%%%%%%%%%%%%%%%%%%%%%%%%%%%%%%
% \paragraph{Command Line Processing.}
%
% The following three command lines generate the output files
% |cdocscld|, |cdocscl1| and |cdocscl2|
% which should be identical to
% |cdocsdrf|, |cdocsch1| and |cdocsfn2|, respectively:
% \begin{center}
% \begin{tabular}{l}
% |latex -jobname cdocscld \|\\
% |  "\def\version{draft}\input{childdoc.def}\childdocforward{cdocsamp}"|\\
% |latex -jobname cdocscl1 \|\\
% |  "\input{childdoc.def}\childdocforward[cdocsamp]{cdocsch1}"|\\
% |latex -jobname cdocscl2 \|\\
% |  "\def\version{final}\input{childdoc.def}\childdocforward{cdocsch2}"|
% \end{tabular}
% \end{center}
% Note that the trailing backslash on each first line
% merely continues the input to the second line
% (for convenient cut ant paste).
% Furthermore, the command |latex| can be replaced by any
% of its alternative versions such as |pdflatex|.
%
% %%%%%%%%%%%%%%%%%%%%%%%%%%%%%%%%%%%%%%%%%%%%%%%%%%%%%%%%%%%%%%%%%%%%%%%%%%%%%%
% %%%%%%%%%%%%%%%%%%%%%%%%%%%%%%%%%%%%%%%%%%%%%%%%%%%%%%%%%%%%%%%%%%%%%%%%%%%%%%
% \section{Implementation}
%\iffalse
%<*package>
%\fi
%
% This section describes the definitions file |childdoc.def|.

% The definitions cannot be loaded using |\usepackage| or |\RequirePackage|
% which has a mechanism to prevent loading a style file more than once.
% When loading the definitions by means of |\input|
% multiple instances have to be prevented manually:
%\iffalse
%This code needs to be before the `\ProvidesFile' directive
%which is defined at the beginning of this file.
%Therefore it is also placed there and commented out here.
%</package>
%<*discard>
%\fi
%    \begin{macrocode}
\ifdefined\childdocmain\endinput\fi
%    \end{macrocode}
%\iffalse
%</discard>
%<*package>
%\fi
%
% \macro{\ifchilddoc}
% \macro{\ifchilddocmanual}
% The conditional |\ifchilddoc| tells whether a
% child (true) or main (false) document is being compiled.
% The conditional |\ifchilddocmanual| tells whether
% the |\includeonly| mechanism is used (false) or
% the selection of child files must be performed manually (true).
% The definitions initialise to false:
%    \begin{macrocode}
\newif\ifchilddoc
\newif\ifchilddocmanual
%    \end{macrocode}

% \macro{\childdocname}
% \macro{\childdocjob}
% The macro |\childdocname| stores the name of the main document
% to be compiled. The macro |\childdocjob| stores the name of
% the document on which the \LaTeX{} compiler was originally invoked.
% The content of |\jobname| cannot be compared
% to filenames specified in the source due to different catcodes.
% The following code rescans |\jobname|, stores the result
% in |\childdocname| and saves a copy in |\childdocjob|:
%    \begin{macrocode}
\edef\childdocname{\scantokens\expandafter{\jobname\noexpand}}
\let\childdocjob\childdocname
%    \end{macrocode}

% \macro{\childdocdisable}
% The macro |\childdocdisable| prevents the main file
% from being processed more than once.
% At this stage, the main document command |\childdocmain|
% is assumed to be called once again where it should do nothing.
% Any subsequent call to it should prevent
% a secondary processing of the main document
% It overwrites the forwarding commands
% |\childdocof| and |\childdocforward|
% with empty macros to prevent further inclusions of the main document:
%    \begin{macrocode}
\newcommand{\childdocdisable}
{
  \renewcommand{\childdocmain}[1]{\renewcommand{\childdocmain}[1]{\endinput}}
  \renewcommand{\childdocof}[1]{}
  \renewcommand{\childdocby}[2][]{}
  \renewcommand{\childdocforward}[2][]{}
  \renewcommand{\childdocdisable}{}
}
%    \end{macrocode}

% \macro{\childdocmain}
% The macro |\childdocmain| is to be called at the top of the main file
% with nothing or the main filename (without extension) as argument.
% First, it breaks loops.
% If the argument is not empty and does not match |\childdocname|
% (which is set by the first inclusion of |childdoc.def|),
% |\ifchilddoc| is set to true, |\includeonly| is applied to the child file
% and |\jobname| is set to the main file
% (for proper handling of |.aux| files):
%    \begin{macrocode}
\newcommand{\childdocmain}[1]
{
  \childdocdisable\childdocmain{}
  \if?#1?\else
    \begingroup
      \def\childdoctmp{#1}
      \ifx\childdoctmp\childdocname
        \def\childdoctmp{}
      \else
        \def\childdoctmp
        {
          \childdoctrue
          \includeonly{\childdocname}
          \def\childdocjob{#1}
          \def\jobname{#1}
        }
      \fi
      \expandafter
    \endgroup
    \childdoctmp
  \fi
}
%    \end{macrocode}

% \macro{\childdocof}
% The command |\childdocof| redirects
% compilation to the main file |#1|.
%    \begin{macrocode}
\newcommand{\childdocof}[1]
{
  \childdocdisable
  \childdoctrue
  \includeonly{\childdocname}
  \def\jobname{#1}
  \def\childdocjob{#1}
  \input{#1}
}
%    \end{macrocode}

% \macro{\childdocby}
% The command |\childdocby| ....
%    \begin{macrocode}
\newcommand{\childdocby}[2][]
{
  \childdocdisable
  \childdoctrue
  \childdocmanualtrue
  \if?#1?\else
    \def\jobname{#2}
  \fi
  \def\childdocjob{#2}
  \input{#2}
  \endinput
}
%    \end{macrocode}

% \macro{\childdocforward}
% The command |\childdocforward| redirects
% compilation to the main file or
% (if the optional argument is given) a child file.
% Parameters are set as if the main file
% or a child file starting with |\childdocof| was compiled.
% Then compilation is handed over to the main file:
%    \begin{macrocode}
\newcommand{\childdocforward}[2][]
{
  \begingroup
    \if?#1?
      \def\childdoctmp
      {
        \def\childdocname{#2}
        \def\childdocjob{#2}
        \def\jobname{#2}
        \input{#2}
        \endinput
      }
    \else
      \def\childdoctmp
      {
        \childdocdisable
        \def\childdocname{#2}
        \childdoctrue
        \includeonly{#2}
        \def\childdocjob{#1}
        \def\jobname{#1}
        \input{#1}
        \endinput
      }
    \fi
    \expandafter
  \endgroup
  \childdoctmp
}
%    \end{macrocode}

% \macro{\childdocforwardprefix}
% The command |\childdocforwardprefix| redirects
% compilation to the main or a child file by means of a pattern.
% The prefix |#1| in the current filename is replaced by |#2|
% and the suffix of the current filename is kept
% (it is assumed that the filename does not contain the substring `|~~~|'
% which is used as a delimiter).
% Compilation is handed over to the new file by |\childdocforward|:
%    \begin{macrocode}
\newcommand{\childdocforwardprefix}[3][]
{
  \begingroup
    \def\childdocextract #2##1~~~{\def\childdoctmp{\childdocforward[#1]{#3##1}}}
    \expandafter\childdocextract\childdocname~~~
    \expandafter
  \endgroup
  \childdoctmp
}
%    \end{macrocode}

% \macro{\childdoc}
% The deprecated macro |\childdoc| is a legacy version of |\childdocmain|:
%    \begin{macrocode}
\newcommand{\childdoc}{\childdocmain}
%    \end{macrocode}

% \macro{\childdocredirect}
% The deprecated macro |\childdocredirect| is a legacy version
% of |\childdocforward| and |\childdocforwardprefix|:
%    \begin{macrocode}
\newcommand{\childdocredirect}[2][]
{
  \begingroup
    \if?#1?
      \def\childdoctmp{\childdocforward{#2}}
    \else
      \def\childdoctmp{\childdocforwardprefix{#1}{#2}}
    \fi
    \expandafter
  \endgroup
  \childdoctmp
}
%    \end{macrocode}

%\iffalse
%</package>
%\fi
%
\endinput
|\\
|\childdocforward{|\textit{main}|}|\\
\end{tabular}
\end{center}
%
or alternatively with:
%
\begin{center}
\begin{tabular}{l}
|% \iffalse
%
% childdoc.dtx Copyright (C) 2017-2018 Niklas Beisert
%
% This work may be distributed and/or modified under the
% conditions of the LaTeX Project Public License, either version 1.3
% of this license or (at your option) any later version.
% The latest version of this license is in
%   http://www.latex-project.org/lppl.txt
% and version 1.3 or later is part of all distributions of LaTeX
% version 2005/12/01 or later.
%
% This work has the LPPL maintenance status `maintained'.
%
% The Current Maintainer of this work is Niklas Beisert.
%
% This work consists of the files childdoc.dtx and childdoc.ins
% and the derived files childdoc.def and cdocsamp.tex with
% cdocsch1.tex, cdocsch2.tex, cdocsdrf.tex, cdocsfn1.tex, cdocsfn2.tex.
%
%<package>\ifdefined\childdocmain\endinput\fi
%<package>\ProvidesFile{childdoc.def}[2018/12/30 v2.0 child document driver]
%<samplemain>\ProvidesFile{cdocsamp.tex}[2018/12/30 v2.0 sample for childdoc]
%<*driver>
%\ProvidesFile{childdoc.drv}[2018/12/30 v2.0 childdoc reference manual file]
\PassOptionsToClass{10pt,a4paper}{article}
\documentclass{ltxdoc}

\usepackage[margin=35mm]{geometry}
\usepackage{hyperref}
\usepackage{hyperxmp}
\usepackage[usenames]{color}

\hypersetup{colorlinks=true}
\hypersetup{pdfstartview=FitH}
\hypersetup{pdfpagemode=UseNone}
\hypersetup{pdfsource={}}
\hypersetup{pdflang={en-UK}}
\hypersetup{pdfcopyright={Copyright 2017-2018 Niklas Beisert.
  This work may be distributed and/or modified under the
  conditions of the LaTeX Project Public License, either version 1.3
  of this license or (at your option) any later version.}}
\hypersetup{pdflicenseurl={http://www.latex-project.org/lppl.txt}}
\hypersetup{pdfcontactaddress={ETH Zurich, ITP, HIT K,
  Wolfgang-Pauli-Strasse 27}}
\hypersetup{pdfcontactpostcode={8093}}
\hypersetup{pdfcontactcity={Zurich}}
\hypersetup{pdfcontactcountry={Switzerland}}
\hypersetup{pdfcontactemail={nbeisert@itp.phys.ethz.ch}}
\hypersetup{pdfcontacturl={http://people.phys.ethz.ch/\xmptilde nbeisert/}}

\newcommand{\secref}[1]{\hyperref[#1]{section \ref*{#1}}}

\parskip1ex
\parindent0pt
\let\olditemize\itemize
\def\itemize{\olditemize\parskip0pt}

\begin{document}

\title{The \textsf{childdoc} Package}
\hypersetup{pdftitle={The childdoc Package}}
\author{Niklas Beisert\\[2ex]
  Institut f\"ur Theoretische Physik\\
  Eidgen\"ossische Technische Hochschule Z\"urich\\
  Wolfgang-Pauli-Strasse 27, 8093 Z\"urich, Switzerland\\[1ex]
  \href{mailto:nbeisert@itp.phys.ethz.ch}
  {\texttt{nbeisert@itp.phys.ethz.ch}}}
\hypersetup{pdfauthor={Niklas Beisert}}
\hypersetup{pdfsubject={Manual for the LaTeX2e Package childdoc}}
\date{30 December 2018, \textsf{v2.0}}
\maketitle

\begin{abstract}\noindent
\textsf{childdoc} is a \LaTeXe{} package
that enables the direct compilation
of document sections included by |\include|
to individual files.
\end{abstract}

\begingroup
\parskip0ex
\tableofcontents
\endgroup

%%%%%%%%%%%%%%%%%%%%%%%%%%%%%%%%%%%%%%%%%%%%%%%%%%%%%%%%%%%%%%%%%%%%%%%%%%%%%%%%
%%%%%%%%%%%%%%%%%%%%%%%%%%%%%%%%%%%%%%%%%%%%%%%%%%%%%%%%%%%%%%%%%%%%%%%%%%%%%%%%
\section{Introduction}

\LaTeX{} provides a mechanism to structure a large document (such as a book)
into a main file and several child files (containing the chapters)
using the |\include| command.
This mechanism is beneficial for documents
which span hundreds of pages in order to
make the source file(s) more manageable.
Moreover, compilation can be restricted to
selected child files by means of the |\includeonly| command.
The latter feature can be used to reduce the compilation time while editing
(this was significantly more useful in the earlier days of \LaTeX{})
or to generate a smaller document which is easier to navigate.
Another application of |\includeonly| is to generate
documents consisting of selected parts of the complete document.

However, there are a few drawbacks of the plain |\include| mechanism:
\begin{itemize}
\item
The child files cannot be compiled on their own,
they can only be compiled via the main file.
A naive editing environment
(such as a text editor with an option
to have the current file processed by \LaTeX)
may require one to switch to the main file before compiling;
attempting to compile the child file produces errors.
\item
The main file must be modified (each time)
to adjust the |\includeonly| command
to the present needs. This easily leaves the main file in a messy state.
\item
The generated document will always carry the filename
of the main document. This is inconvenient if
several child files are to be compiled and
to be kept for distribution.
\end{itemize}

The present package provides a simple interface
to make child files individually compilable by \LaTeX{}.
Compiling a child file then has the same effect as compiling
the main file with an |\includeonly| command
to select the appropriate child.
Moreover the generated document will carry the name of the child
rather than the main file.
This resolves all three above issues.

This feature is meant to make the editing of books,
thesis documents and lecture notes somewhat more convenient.
However, the package can also be used efficiently for
composing a series of documents (such as exercise sheets)
which are typically distributed individually.
It then assists the author in generating the individual documents
(potentially in different versions)
as well as a document containing the collected series.
Another application is in developing style files
or other kinds of included material
where compilation of the style file could redirect
to a sample or test file.

%%%%%%%%%%%%%%%%%%%%%%%%%%%%%%%%%%%%%%%%%%%%%%%%%%%%%%%%%%%%%%%%%%%%%%%%%%%%%%%%
%%%%%%%%%%%%%%%%%%%%%%%%%%%%%%%%%%%%%%%%%%%%%%%%%%%%%%%%%%%%%%%%%%%%%%%%%%%%%%%%
\section{Usage}

First of all, the package \textsf{childdoc} is \emph{not} a standard
\LaTeXe{} |.sty| style file! Therefore it needs to be invoked in
a non-standard way.

%%%%%%%%%%%%%%%%%%%%%%%%%%%%%%%%%%%%%%%%%%%%%%%%%%%%%%%%%%%%%%%%%%%%%%%%%%%%%%%%
\subsection{Included Files}
\label{sec:include}

%%%%%%%%%%%%%%%%%%%%%%%%%%%%%%%%%%%%%%%%
\DescribeMacro{\childdocmain}
To use the package, add the commands
\begin{center}
\begin{tabular}{l}
|\input{childdoc.def}|\\
|\childdocmain{}|\\
\end{tabular}
\end{center}
at the very top of the main \LaTeX{} file,
in particular \emph{before} the |\documentclass| statement!
The argument of |\childdocmain| should be left empty
(but it must be present).

%%%%%%%%%%%%%%%%%%%%%%%%%%%%%%%%%%%%%%%%
\DescribeMacro{\childdocof}
Furthermore, add the commands
\begin{center}
\begin{tabular}{l}
|\input{childdoc.def}|\\
|\childdocof{|\textit{main}|}|\\
\end{tabular}
\end{center}
at the top of every child file \textit{child}
which is included by |\include{|\textit{child}|}|
from within the main file
(or at least for those files to be compiled individually).
The argument \textit{main} must be the filename of the main file.

There are a couple of
considerations in setting up the main and child documents:

%%%%%%%%%%%%%%%%%%%%%%%%%%%%%%%%%%%%%%%%
\paragraph{Restrictions.}

Please note the following restrictions:
\begin{itemize}
\item
|\childdocmain| must be called with one argument \textit{main}
to ensure compatibility with earlier version of the package.
It must either be empty (|\childdocmain{}|)
or precisely match the filename of the main file in which it is specified.
See \secref{sec:detection} for further information.
\item
The filename \textit{main} must be specified without the |.tex| extension.
\item
The filename \textit{main} is case sensitive
(even in case-insensitive file systems)
due to internal string comparison.
\item
The argument \textit{main} should be fully expanded, it cannot be a macro.
\item
Subdirectories and special characters should be avoided in filenames.
\item
The command |\childdocmain{|\textit{main}|}| must be followed by a whitespace.
It should not be followed immediately by another command
or by a comment mark `|%|'.
This is because the \TeX{} parser reads the token immediately following
the argument of |\childdocmain| and puts it
at the beginning of every child section;
however, a white\-space is ignored.
\end{itemize}

%%%%%%%%%%%%%%%%%%%%%%%%%%%%%%%%%%%%%%%%
\paragraph{Content of Main File.}

It is advisable to place all content in the child files included by |\include|.
Any output contained in the main file will appear in all child documents
unless suppressed manually;
it cannot be suppressed automatically by the |\includeonly| directive
and thus should normally be avoided.
A method to include some content in the main file
by means of conditional processing is described in \secref{sec:conditional}.

%%%%%%%%%%%%%%%%%%%%%%%%%%%%%%%%%%%%%%%%
\paragraph{Page Numbering.}

When only a part of the document is compiled,
the appropriate numbering of pages
(as well as other status parameters)
is determined from the |.aux| files.
The latter contain information from previous passes.
However this information needs to propagate through
all intermediate child documents.
Therefore the page numbering in child documents may well
be inconsistent until the complete document is compiled at least once.

A useful (if unconventional) way to always ensure a consistent
page numbering is to restart the numbering in each child document
and denote the pages by `\textit{child}|.|\textit{page}'
where \textit{child} represents the chapter/section number of the child file.
This can be achieved by the command
|\numberwithin{page}{|\textit{child}|}|
of the \textsf{amsmath} package
where \textit{child} can be |chapter| or |section|
depending on the chosen structuring.
Alternatively, one can modify the macro |\thepage| appropriately
and reset the counter |page| at the start of each child file.

%%%%%%%%%%%%%%%%%%%%%%%%%%%%%%%%%%%%%%%%%%%%%%%%%%%%%%%%%%%%%%%%%%%%%%%%%%%%%%%%
\subsection{Conditional Processing}
\label{sec:conditional}

The package provides a mechanism to compile different versions
of a document. To customise the versions further some conditional processing
can come in handy to distinguish which version is being compiled.
The package provides two macros to describe the compilation context:

%%%%%%%%%%%%%%%%%%%%%%%%%%%%%%%%%%%%%%%%
\DescribeMacro{\ifchilddoc}
The conditional |\ifchilddoc| distinguishes between the compilation of
child documents and the main document:
%
\begin{center}
|\ifchilddoc |\textit{child-code}| |[|\||else |\textit{main-code}]| \||fi|
\end{center}

%%%%%%%%%%%%%%%%%%%%%%%%%%%%%%%%%%%%%%%%
\DescribeMacro{\childdocname}
\DescribeMacro{\childdocjob}
The macro |\childdocname| contains the filename (without extension)
of the main or child file being processed.
Note that |\childdocjob| will always contain the name of the main file.

%%%%%%%%%%%%%%%%%%%%%%%%%%%%%%%%%%%%%%%%
\paragraph{Title Page.}

Conditional processing can be used to include a title or banner page
in the main document when proper precautions are taken.
Importantly, the code in the main file should ensure that the page counter
(as well as other status parameters which are stored in the |.aux| files)
takes the same value after the conditional processing.
Otherwise the page numbers may take divergent values
depending on which part is compiled.

For example, a title page could be declared by:
%
\begin{center}
\begin{tabular}{l}
|\ifchilddoc\||else|\\
|\addtocounter{page}{-1}|\\
\textit{code for title page}\\
|\newpage|\\
|\||fi|
\end{tabular}
\end{center}
%
A banner page for the child documents can be generated by:
%
\begin{center}
\begin{tabular}{l}
|\ifchilddoc|\\
|\addtocounter{page}{-1}|\\
\textit{code for banner page}\\
|\newpage|\\
|\||fi|
\end{tabular}
\end{center}
%
Here one could write a message such as:
\begin{center}
|This is the part \childdocname{} of \childdocjob{}.|
\end{center}

%%%%%%%%%%%%%%%%%%%%%%%%%%%%%%%%%%%%%%%%%%%%%%%%%%%%%%%%%%%%%%%%%%%%%%%%%%%%%%%%
\subsection{Flags}
\label{sec:flags}

The package makes it easy to generate different versions
of the main or child documents.
To this end compilation flags can be defined
and assigned different default values.
They will be particularly useful in conjunction
with the forwarding mechanism described in \secref{sec:forward}.

For example, it may be useful to have a flag |\version|
which can be set to |draft| or |final|.
The document source will contain some conditional code
depending on the value of |\version|.
Suppose further, the flag should default to |final| for the main file
and to |draft| for child files
which is a natural assignment for editing the document.
This is achieved by placing the following code
in the preamble of the main document
(below the |\childdocmain| directive):
%
\begin{center}
\begin{tabular}{l}
|\ifchilddoc|\\
|\providecommand{\version}{draft}|\\
|\||else|\\
|\providecommand{\version}{final}|\\
|\||fi|
\end{tabular}
\end{center}
%
The definition by |\providecommand| makes sure
that previous definitions are not overwritten.
Further statements |\providecommand{\version}{...}|
can thus be added before the above code to override it.

For the main file, one might add a line
(between |\childdocmain| and the above block)
%
\begin{center}
|%\ifchilddoc\||else\providecommand{\version}{draft}\||fi|
\end{center}
%
which can be uncommented to produce a draft version.
Likewise one can add a line to the very top of a child file
(above the |\childdocof{|\textit{main}|}| directive)
%
\begin{center}
|%\providecommand{\version}{final}|
\end{center}
%
which can be uncommented to produce the final version of this child document.

%%%%%%%%%%%%%%%%%%%%%%%%%%%%%%%%%%%%%%%%%%%%%%%%%%%%%%%%%%%%%%%%%%%%%%%%%%%%%%%%
\subsection{Forwarding}
\label{sec:forward}

Different versions of the main or child documents
using compilation flags as described in \secref{sec:flags}
can be (permanently) stored in different files
for convenient compilation, viewing and distribution.
To this end, the package defines a command
to pass on compilation to a different file:

%%%%%%%%%%%%%%%%%%%%%%%%%%%%%%%%%%%%%%%%
\DescribeMacro{\childdocforward}
The command |\childdocforward| redirects processing to
another source file:
%
\begin{center}
\begin{tabular}{l}
|\input{childdoc.def}|\\
|\childdocforward[|\textit{main}|]{|\textit{dest}|}|\\
\end{tabular}
\end{center}
%
The argument \textit{dest} is the destination file
(without extension).
It should be the main file or one of the child files.
Note that further \textsf{childdoc} directives
such as |\childdocof| and |\childdocforward|
in the indicated file will be processed in this form.
The optional argument \textit{main}
passes on directly to the main file \textit{main}
while pretending to compile the child \textit{dest}.
This form behaves as if \textit{dest}
issues |\childdocof{|\textit{main}|}| right away,
and no further \textsf{childdoc} directives will be processed.

%%%%%%%%%%%%%%%%%%%%%%%%%%%%%%%%%%%%%%%%
\DescribeMacro{\...prefix}
In the alternative form |\childdocforwardprefix|,
%
\begin{center}
\begin{tabular}{l}
|\input{childdoc.def}|\\
|\childdocforwardprefix[|\textit{main}|]{|\textit{prefix}|}{|\textit{dest}|}|
\end{tabular}
\end{center}
%
the destination file is determined by a pattern
depending on the current file:
To make this work, the current file must be called
`{\textit{prefix}\hspace{0.2em}\textit{suffix}}'
with \textit{prefix} matching precisely the argument.
Processing is then passed on to the file
`{\textit{dest}\hspace{0.2em}\textit{suffix}}'.
Surely, the same effect is achieved by
directly specifying the
argument `{\textit{dest}\hspace{0.2em}\textit{suffix}}'
in the first form.
However, that requires to set up a different file
for each child. With the alternative form of the command
all these files can have exactly the same content
which simplifies setting them up and maintaining them.

For example, the following file |draft.tex|
with a compilation flag |\version| as described in \secref{sec:flags}
compiles the main document as a draft:
%
\begin{center}
\begin{tabular}{l}
|\def\version{draft}|\\
|\input{childdoc.def}|\\
|\childdocforward{|\textit{main}|}|
\end{tabular}
\end{center}
%
Likewise, the following files |final|\textit{nn}|.tex|
compile the final version of the child document
|child|\textit{nn}|.tex|:
%
\begin{center}
\begin{tabular}{l}
|\def\version{final}|\\
|\input{childdoc.def}|\\
|\childdocforwardprefix{final}{child}|
\end{tabular}
\end{center}
%

Note that when several versions of a main file and/or of each child file
are to be generated, it may be convenient to set up a |Makefile| or
shell script to automatise the process.

%%%%%%%%%%%%%%%%%%%%%%%%%%%%%%%%%%%%%%%%%%%%%%%%%%%%%%%%%%%%%%%%%%%%%%%%%%%%%%%%
\subsection{Command Line Processing}
\label{sec:commandline}

The effect of redirection files can also be achieved by invoking
the \LaTeX{} compiler with a more elaborate command line.
Most conveniently this should be done as part
of a shell script or a |Makefile|.

When using \textsf{childdoc} in the main file, the following
command lines effectively perform a redirection
(note that depending on the shell being used,
backslashes may have to be doubled: `|\|' $\to$ `|\\|'):
%
\begin{center}
|... -jobname "|\textit{target}|" |\\|"|[\textit{flags}]%
|\input{childdoc.def}\childdocforward[|\textit{main}|]{|\textit{dest}|}"|
\end{center}
%
Here \textit{target} is the name of the output file,
\textit{main} is the name of the main file
and \textit{dest} is the name of the main or child file to be processed
(all filenames without extensions).
The optional argument \textit{main} can be omitted
if \textit{main} matches \textit{dest}.
Optionally, compilation \textit{flags} can be defined via |\def| commands.
This command line makes the \TeX{} engine believe
it is compiling the file \textit{target}
whose content is specified as the latter parameter.
The provided code then forwards the processing to
\textit{main} or \textit{dest} as described in \secref{sec:forward}.

%%%%%%%%%%%%%%%%%%%%%%%%%%%%%%%%%%%%%%%%%%%%%%%%%%%%%%%%%%%%%%%%%%%%%%%%%%%%%%%%
\subsection{Include by Input}
\label{sec:input}

Including child documents by |\include| has some restrictions by design.
Most notably, the content of a child document always occupies
its own set of pages; pages cannot be shared between child documents.
Usually, this behaviour makes perfect sense
because each child document contain an essential part of the document.
However, in some situations it may be desirable to compose
a document from a collection of parts
without having mandatory page breaks between then.
For this case, the package
provides a mechanism to include parts
by |\input| which can also be processed individually.
However, by construction this mechanism
requires manual handling of the content to be output.

%%%%%%%%%%%%%%%%%%%%%%%%%%%%%%%%%%%%%%%%
\DescribeMacro{\ifchilddocmanual}
The main file should be prepared as usual, see \secref{sec:include}.
However, the document body must make a distinction
between processing of an individual part and of the main document, e.g.:
%
\begin{center}
\begin{tabular}{l}
|\ifchilddocmanual|\\
|\input{\childdocname}|\\
|\||else|\\
\textit{document body with }|\input{|\textit{part}|}|\\
|\||fi|
\end{tabular}
\end{center}
%
The conditional |\ifchilddocmanual| is true whenever
a part to be included by |\input| is being compiled,
and the name of the part is stored in |\childdocname|.

%%%%%%%%%%%%%%%%%%%%%%%%%%%%%%%%%%%%%%%%
\DescribeMacro{\childdocby}
Each part to be included by |\input| should start with:
%
\begin{center}
\begin{tabular}{l}
|\input{childdoc.def}|\\
|\childdocby{|\textit{main}|}|\\
\end{tabular}
\end{center}
%
The directive |\childdocby| is similar to |\childdocof|
described in \secref{sec:include},
but the subsequent selection of content must be done manually.
To that end, both |\ifchilddoc| and |\ifchilddocmanual|
will be true upon processing of a part,
and the name of the part is stored in |\childdocname|.
Note that |\jobname| will be set to the filename of the current part
so that each part receives an individual |.aux| file
that does not interfere with the |.aux| file(s) of the main document.
This behaviour can be altered by the alternative form
|\childdocby[*]{|\textit{main}|}| (with a non-empty optional argument)
which uses the |.aux| file of the main document
by setting |\jobname| to \textit{main}.

%%%%%%%%%%%%%%%%%%%%%%%%%%%%%%%%%%%%%%%%%%%%%%%%%%%%%%%%%%%%%%%%%%%%%%%%%%%%%%%%
\subsection{Driver Development}
\label{sec:driver}

The \textsf{childdoc} mechanism can also be use for the development
of definition files such as \LaTeX{} styles or classes.
This case differs from the above setup with multiple parts
included by |\include| in that no |\includeonly| should be invoked.
This can be achieved by starting the include file
(before |\ProvidesPackage|) with:
%
\begin{center}
\begin{tabular}{l}
|\input{childdoc.def}|\\
|\childdocforward{|\textit{main}|}|\\
\end{tabular}
\end{center}
%
or alternatively with:
%
\begin{center}
\begin{tabular}{l}
|\input{childdoc.def}|\\
|\childdocby{|\textit{main}|}|\\
\end{tabular}
\end{center}
%
Both forms have slightly different effects as described above.
The main file is prepared as usual, see \secref{sec:include}.

%%%%%%%%%%%%%%%%%%%%%%%%%%%%%%%%%%%%%%%%%%%%%%%%%%%%%%%%%%%%%%%%%%%%%%%%%%%%%%%%
\subsection{Legacy Detection}
\label{sec:detection}

The directive |\childdocmain| in the main file can detect
whether the complete document or merely a child is to be compiled
even without using the directive |\childdocof|.
This method is deprecated because it is less robust
and there is no compelling reason to use it;
it is merely provided for backward compatibility
and it may be removed in future versions.

If the detection mechanism is to be used,
it is mandatory to correctly specify
the filename of the main file as the argument of |\childdocmain|:
%
\begin{center}
\begin{tabular}{l}
|\input{childdoc.def}|\\
|\childdocmain{|\textit{main}|}|\\
\end{tabular}
\end{center}
%
If |\jobname| does not match the argument \textit{main} of |\childdocmain|,
it is assumed that |\jobname| points to the child file to be compiled.
When using |\childdocmain| with the main file specified as argument,
it suffices to start a child file
with just |\input{|\textit{main}|}|
without loading of the package and using |\childdocof|.
If instead all processing is done
with the appropriate \textsf{childdoc} directives,
the argument of \textit{main} of |\childdocmain| can be empty.

An alternative version of the command line processing described
in \secref{sec:commandline} using the detection mechanism reads:
%
\begin{center}
|... -jobname "|\textit{target}|" "|[\textit{flags}]%
[|\def\jobname{|\textit{dest}|}|]|\input{|\textit{main}|}"|
\end{center}

%%%%%%%%%%%%%%%%%%%%%%%%%%%%%%%%%%%%%%%%%%%%%%%%%%%%%%%%%%%%%%%%%%%%%%%%%%%%%%%%
\subsection{Manual Code}
\label{sec:manual}

In case one cannot be certain whether the definitions file |childdoc.def|
is installed on the target \TeX{} distribution
and one prefers not to ship it,
it is conceivable to paste a few relevant commands into the sources.

To that end, drop all statements |\input{childdoc.def}|
and perform the replacements as outlined below.
Instead of |\childdocmain{|\textit{main}|}| add the following code
to the top of the main file:
%
\begin{center}
\begin{tabular}{l}
|\||ifdefined\childdocname\endinput\||fi\newif\ifchilddoc|\\
|\edef\childdocname{\scantokens\expandafter{\jobname\noexpand}}|\\
|\def\childdocmain{|\textit{main}|}\||ifx\childdocmain\childdocname\||else|\\
|\childdoctrue\includeonly{\childdocname}\let\jobname\childdocmain\||fi|\\
\end{tabular}
\end{center}
%
Instead of |\childdocof{|\textit{main}|}| just include the main file
at the top of each child file:
%
\begin{center}
|\input{|\textit{main}|}|
\end{center}
%
A simple redirection |\childdocforward{|\textit{dest}|}| is achieved by:
%
\begin{center}
|\def\jobname{|\textit{dest}|}\input{\jobname}|
\end{center}
%
The redirection with prefix
|\childdocforwardprefix[|\textit{prefix}|]{|\textit{dest}|}|
is accomplished by:
%
\begin{center}
\begin{tabular}{l}
|{\edef\jobname{\scantokens\expandafter{\jobname\noexpand}}|\\
|\def\redirectjob |\textit{prefix}|#1~~~{\gdef\jobname{|\textit{dest}|#1}}|\\
|\expandafter\redirectjob\jobname~~~}\input{\jobname}|
\end{tabular}
\end{center}

In an alternative approach,
child documents can be compiled by a specific command line
without additional code or specific definitions:
%
\begin{center}
|... -jobname "|\textit{target}|" "|[\textit{flags}]%
|\includeonly{|\textit{dest}|}\input{|\textit{main}|}"|
\end{center}
%

%%%%%%%%%%%%%%%%%%%%%%%%%%%%%%%%%%%%%%%%%%%%%%%%%%%%%%%%%%%%%%%%%%%%%%%%%%%%%%%%
%%%%%%%%%%%%%%%%%%%%%%%%%%%%%%%%%%%%%%%%%%%%%%%%%%%%%%%%%%%%%%%%%%%%%%%%%%%%%%%%
\section{Information}

%%%%%%%%%%%%%%%%%%%%%%%%%%%%%%%%%%%%%%%%%%%%%%%%%%%%%%%%%%%%%%%%%%%%%%%%%%%%%%%%
\subsection{Copyright}

Copyright \copyright{} 2017--2018 Niklas Beisert

This work may be distributed and/or modified under the
conditions of the \LaTeX{} Project Public License, either version 1.3
of this license or (at your option) any later version.
The latest version of this license is in
  \url{http://www.latex-project.org/lppl.txt}
and version 1.3 or later is part of all distributions of \LaTeX{}
version 2005/12/01 or later.

This work has the LPPL maintenance status `maintained'.

The Current Maintainer of this work is Niklas Beisert.

This work consists of the files |README.txt|, |childdoc.ins| and |childdoc.dtx|
as well as the derived files |childdoc.def|, |cdocsamp.tex|
with |cdocsch1.tex|, |cdocsch2.tex|, |cdocspt3.tex|, |cdocspt4.tex|,
|cdocsdrf.tex|, |cdocsfn1.tex|, |cdocsfn2.tex|
as well as |childdoc.pdf|.

%%%%%%%%%%%%%%%%%%%%%%%%%%%%%%%%%%%%%%%%%%%%%%%%%%%%%%%%%%%%%%%%%%%%%%%%%%%%%%%%
\subsection{Files and Installation}

The package consists of the files:
%
\begin{center}
\begin{tabular}{ll}
    |README.txt|   & readme file \\
    |childdoc.ins| & installation file \\
    |childdoc.dtx| & source file \\
    |childdoc.def| & definition file \\
    |cdocsamp.tex| & sample main file \\
    |cdocsch1.tex| & sample include file \\
    |cdocsch2.tex| & sample include file \\
    |cdocspt3.tex| & sample part file \\
    |cdocspt4.tex| & sample part file \\
    |cdocsdrf.tex| & sample redirection file \\
    |cdocsfn1.tex| & sample redirection file \\
    |cdocsfn2.tex| & sample redirection file \\
    |childdoc.pdf| & manual
\end{tabular}
\end{center}
%
The distribution consists of the files
|README.txt|, |childdoc.ins| and |childdoc.dtx|.
%
\begin{itemize}
\item
Run (pdf)\LaTeX{} on |childdoc.dtx|
to compile the manual |childdoc.pdf| (this file).
\item
Run \LaTeX{} on |childdoc.ins| to create the definitions file |childdoc.def|
and the sample |cdocsamp.tex| with include files
|cdocsch1.tex|, |cdocsch2.tex|, |cdocspt3.tex|, |cdocspt4.tex|,
|cdocsdrf.tex|, |cdocsfn1.tex|, |cdocsfn2.tex|.
Then copy the file |childdoc.def| to an appropriate directory of your \LaTeX{}
distribution, e.g.\ \textit{texmf-root}|/tex/latex/childdoc|.
\end{itemize}

%%%%%%%%%%%%%%%%%%%%%%%%%%%%%%%%%%%%%%%%%%%%%%%%%%%%%%%%%%%%%%%%%%%%%%%%%%%%%%%%
\subsection{Related CTAN Packages}

There are several other packages which offer a similar functionality:
%
\begin{itemize}
\item
The packages
\href{http://ctan.org/pkg/docmute}{\textsf{docmute}},
\href{http://ctan.org/pkg/includex}{\textsf{includex}} and
\href{http://ctan.org/pkg/standalone}{\textsf{standalone}}
provide commands to include only the document body of
a child file thus allowing both files to be compiled individually.
\item
The packages \href{http://ctan.org/pkg/subdocs}{\textsf{subdocs}}
and \href{http://ctan.org/pkg/subfiles}{\textsf{subfiles}}
provide structures in which the main and child documents can be
encapsulated and allowing them to be compiled individually.
The inclusion mechanism is different from the conventional |\include|.
\item
The package \href{http://ctan.org/pkg/combine}{\textsf{combine}}
is an elaborate solution to combine several documents into one.
\end{itemize}
%
See also the CTAN topic \href{http://ctan.org/topic/subdocs}{\textsf{subdocs}}
for further related packages.
The present package differs from the above solutions in that
a document structure constructed with the conventional |\include| mechanism
just needs two extra commands at the top of every file
such that all constituent files can be compiled individually.

%%%%%%%%%%%%%%%%%%%%%%%%%%%%%%%%%%%%%%%%%%%%%%%%%%%%%%%%%%%%%%%%%%%%%%%%%%%%%%%%
%\subsection{Feature Suggestions}
%
%The following is a list of features which may be useful for future
%versions of this package:
%%
%\begin{itemize}
%\item
%\ldots
%\end{itemize}

%%%%%%%%%%%%%%%%%%%%%%%%%%%%%%%%%%%%%%%%%%%%%%%%%%%%%%%%%%%%%%%%%%%%%%%%%%%%%%%%
\subsection{Revision History}

%%%%%%%%%%%%%%%%%%%%%%%%%%%%%%%%%%%%%%%%
\paragraph{v2.0:} 2018/12/30

\begin{itemize}
\item
immediate forward processing
\item
added |\childdocby| mechanism
\item
manual restructured
\end{itemize}

%%%%%%%%%%%%%%%%%%%%%%%%%%%%%%%%%%%%%%%%
\paragraph{v1.6:} 2018/01/17

\begin{itemize}
\item
application for development of include files
\item
corrections to manual
\end{itemize}

%%%%%%%%%%%%%%%%%%%%%%%%%%%%%%%%%%%%%%%%
\paragraph{v1.5:} 2017/05/21

\begin{itemize}
\item
more complete structuring introduced
\item
|\childdocof| introduced
\item
|\childdoc| renamed to |\childdocmain|
\item
|\childredirect| renamed to |\childdocforward| and |\childdocforwardprefix|
and functionality expanded
\end{itemize}

%%%%%%%%%%%%%%%%%%%%%%%%%%%%%%%%%%%%%%%%
\paragraph{v1.0:} 2017/04/27

\begin{itemize}
\item
manual and install package
\item
first version published on CTAN
\end{itemize}

%%%%%%%%%%%%%%%%%%%%%%%%%%%%%%%%%%%%%%%%
\paragraph{v0.6:} 2017/04/26

\begin{itemize}
\item
redirection mechanism added
\end{itemize}

%%%%%%%%%%%%%%%%%%%%%%%%%%%%%%%%%%%%%%%%
\paragraph{v0.5:} 2017/04/26

\begin{itemize}
\item
functionality in definition file
\end{itemize}


%%%%%%%%%%%%%%%%%%%%%%%%%%%%%%%%%%%%%%%%%%%%%%%%%%%%%%%%%%%%%%%%%%%%%%%%%%%%%%%%
%%%%%%%%%%%%%%%%%%%%%%%%%%%%%%%%%%%%%%%%%%%%%%%%%%%%%%%%%%%%%%%%%%%%%%%%%%%%%%%%
%%%%%%%%%%%%%%%%%%%%%%%%%%%%%%%%%%%%%%%%%%%%%%%%%%%%%%%%%%%%%%%%%%%%%%%%%%%%%%%%
\appendix

\settowidth\MacroIndent{\rmfamily\scriptsize 000\ }

 \DocInput{childdoc.dtx}

\end{document}
%</driver>
% \fi
%
% %%%%%%%%%%%%%%%%%%%%%%%%%%%%%%%%%%%%%%%%%%%%%%%%%%%%%%%%%%%%%%%%%%%%%%%%%%%%%%
% %%%%%%%%%%%%%%%%%%%%%%%%%%%%%%%%%%%%%%%%%%%%%%%%%%%%%%%%%%%%%%%%%%%%%%%%%%%%%%
% \section{Sample}
%\iffalse
%<*samplemain>
%\fi
%
% The following presents a sample document
% with two chapters, two parts, a title page,
% a compile flag as well as three forwarding files to set the flag.
% It consists of eight |.tex| files:
% \begin{center}
% \begin{tabular}{ll}
% |cdocsamp.tex|&main file\\
% |cdocsch1.tex|&include file for chapter 1\\
% |cdocsch2.tex|&include file for chapter 2\\
% |cdocspt3.tex|&include file for part 3\\
% |cdocspt4.tex|&include file for part 4\\
% |cdocsdrf.tex|&forwarding file for main file in draft mode\\
% |cdocsfi1.tex|&forwarding file for final version of chapter 1\\
% |cdocsfi2.tex|&forwarding file for final version of chapter 2\\
% \end{tabular}
% \end{center}
% Each of the eight files can be compiled directly by the \LaTeX{} compiler.
%
% %%%%%%%%%%%%%%%%%%%%%%%%%%%%%%%%%%%%%%
% \paragraph{Main File.}
%
% The main file is called |cdocsamp.tex|.
%
% Load the \textsf{childdoc} definitions and
% declare the filename for the main document:
%    \begin{macrocode}
\input{childdoc.def}
\childdocmain{}
%    \end{macrocode}

% Optional override for |\version| flag:
%    \begin{macrocode}
%%\ifchilddoc\else\providecommand{\version}{draft}\fi
%    \end{macrocode}

% Define the default values for the |\version| flag
% (|final| for the main file and |draft| for childs):
%    \begin{macrocode}
\ifchilddoc
\providecommand{\version}{draft}
\else
\providecommand{\version}{final}
\fi
%    \end{macrocode}

% Load the standard document class:
%    \begin{macrocode}
\documentclass[12pt]{article}
%    \end{macrocode}

% Start the document body:
%    \begin{macrocode}
\begin{document}
%    \end{macrocode}

% Declare a title page.
% Print title, part of document being processed and version flag:
%    \begin{macrocode}
\addtocounter{page}{-1}
\begin{center}
{\LARGE\bfseries{}childdoc example\par}
\vspace{1cm}
\ifchilddoc
\ifchilddocmanual part\else chapter\fi:
`\childdocname' of `\childdocjob'\par
\else
main document: `\childdocjob'\par
\fi
version: \version\par
\end{center}
\newpage
%    \end{macrocode}

% Manually include selected file,
% otherwise process as usual:
%    \begin{macrocode}
\ifchilddocmanual
\section*{part `\childdocname'}
\input{\childdocname}
\else
%    \end{macrocode}

% Include the two chapters:
%    \begin{macrocode}
\include{cdocsch1}
\include{cdocsch2}
%    \end{macrocode}

% Include the two parts unless only chapters should be displayed:
%    \begin{macrocode}
\ifchilddoc\else
\section{part three}
\input{cdocspt3}
\section{part four}
\input{cdocspt4}
\fi
%    \end{macrocode}

% Process as usual until here:
%    \begin{macrocode}
\fi
%    \end{macrocode}

% End of document body:
%    \begin{macrocode}
\end{document}
%    \end{macrocode}
%\iffalse
%</samplemain>
%\fi
%
% %%%%%%%%%%%%%%%%%%%%%%%%%%%%%%%%%%%%%%
% \paragraph{Chapter Include Files.}
%
% The include files are called |cdocsch1.tex| and |cdocsch2.tex|.
%
%\iffalse
%<*samplechap1|samplechap2>
%\fi

% Optional override for |\version| flag:
%    \begin{macrocode}
%%\providecommand{\version}{final}
%    \end{macrocode}

% Include the main document:
%    \begin{macrocode}
\input{childdoc.def}
\childdocof{cdocsamp}
%    \end{macrocode}

%\iffalse
%</samplechap1|samplechap2>
%\fi
%
%\iffalse
%<*samplechap1>
%\fi
% Some text for chapter 1:
%    \begin{macrocode}
\section{one}
some text in chapter one
%    \end{macrocode}

%\iffalse
%</samplechap1>
%\fi
% Some text for chapter 2:
%\iffalse
%<*samplechap2>
%\fi
%    \begin{macrocode}
\section{two}
more text in chapter two
%    \end{macrocode}

%\iffalse
%</samplechap2>
%\fi
%
% %%%%%%%%%%%%%%%%%%%%%%%%%%%%%%%%%%%%%%
% \paragraph{Part Include Files.}
%
% The include files are called |cdocspt3.tex| and |cdocspt4.tex|.
%
%\iffalse
%<*samplepart3|samplepart4>
%\fi

% Optional override for |\version| flag:
%    \begin{macrocode}
%%\providecommand{\version}{final}
%    \end{macrocode}

% Include the main document:
%    \begin{macrocode}
\input{childdoc.def}
\childdocby{cdocsamp}
%    \end{macrocode}

%\iffalse
%</samplepart3|samplepart4>
%\fi
%
%\iffalse
%<*samplepart3>
%\fi
% Some text for part 3:
%    \begin{macrocode}
some text in part three
%    \end{macrocode}

%\iffalse
%</samplepart3>
%\fi
% Some text for part 4:
%\iffalse
%<*samplepart4>
%\fi
%    \begin{macrocode}
more text in part four
%    \end{macrocode}

%\iffalse
%</samplepart4>
%\fi
%
% %%%%%%%%%%%%%%%%%%%%%%%%%%%%%%%%%%%%%%
% \paragraph{Forwarding for a Complete Draft.}
%
% The following forwarding file |cdocsdrf.tex|
% compiles the main document in draft mode:
%\iffalse
%<*sampledraft>
%\fi
%    \begin{macrocode}
\def\version{draft}
\input{childdoc.def}
\childdocforward{cdocsamp}
%    \end{macrocode}

%\iffalse
%</sampledraft>
%\fi
%
% %%%%%%%%%%%%%%%%%%%%%%%%%%%%%%%%%%%%%%
% \paragraph{Forwarding for Final Version of the Chapters.}
%
% The following forwarding files |cdocsfn1.tex| and |cdocsfn2.tex|
% (with identical content)
% compile the final versions of the child documents
% |cdocsch1.tex| and |cdocsch2.tex|, respectively:
%\iffalse
%<*samplefinal>
%\fi
%    \begin{macrocode}
\def\version{final}
\input{childdoc.def}
\childdocforwardprefix[cdocsamp]{cdocsfn}{cdocsch}
%    \end{macrocode}

%\iffalse
%</samplefinal>
%\fi
%
% %%%%%%%%%%%%%%%%%%%%%%%%%%%%%%%%%%%%%%
% \paragraph{Command Line Processing.}
%
% The following three command lines generate the output files
% |cdocscld|, |cdocscl1| and |cdocscl2|
% which should be identical to
% |cdocsdrf|, |cdocsch1| and |cdocsfn2|, respectively:
% \begin{center}
% \begin{tabular}{l}
% |latex -jobname cdocscld \|\\
% |  "\def\version{draft}\input{childdoc.def}\childdocforward{cdocsamp}"|\\
% |latex -jobname cdocscl1 \|\\
% |  "\input{childdoc.def}\childdocforward[cdocsamp]{cdocsch1}"|\\
% |latex -jobname cdocscl2 \|\\
% |  "\def\version{final}\input{childdoc.def}\childdocforward{cdocsch2}"|
% \end{tabular}
% \end{center}
% Note that the trailing backslash on each first line
% merely continues the input to the second line
% (for convenient cut ant paste).
% Furthermore, the command |latex| can be replaced by any
% of its alternative versions such as |pdflatex|.
%
% %%%%%%%%%%%%%%%%%%%%%%%%%%%%%%%%%%%%%%%%%%%%%%%%%%%%%%%%%%%%%%%%%%%%%%%%%%%%%%
% %%%%%%%%%%%%%%%%%%%%%%%%%%%%%%%%%%%%%%%%%%%%%%%%%%%%%%%%%%%%%%%%%%%%%%%%%%%%%%
% \section{Implementation}
%\iffalse
%<*package>
%\fi
%
% This section describes the definitions file |childdoc.def|.

% The definitions cannot be loaded using |\usepackage| or |\RequirePackage|
% which has a mechanism to prevent loading a style file more than once.
% When loading the definitions by means of |\input|
% multiple instances have to be prevented manually:
%\iffalse
%This code needs to be before the `\ProvidesFile' directive
%which is defined at the beginning of this file.
%Therefore it is also placed there and commented out here.
%</package>
%<*discard>
%\fi
%    \begin{macrocode}
\ifdefined\childdocmain\endinput\fi
%    \end{macrocode}
%\iffalse
%</discard>
%<*package>
%\fi
%
% \macro{\ifchilddoc}
% \macro{\ifchilddocmanual}
% The conditional |\ifchilddoc| tells whether a
% child (true) or main (false) document is being compiled.
% The conditional |\ifchilddocmanual| tells whether
% the |\includeonly| mechanism is used (false) or
% the selection of child files must be performed manually (true).
% The definitions initialise to false:
%    \begin{macrocode}
\newif\ifchilddoc
\newif\ifchilddocmanual
%    \end{macrocode}

% \macro{\childdocname}
% \macro{\childdocjob}
% The macro |\childdocname| stores the name of the main document
% to be compiled. The macro |\childdocjob| stores the name of
% the document on which the \LaTeX{} compiler was originally invoked.
% The content of |\jobname| cannot be compared
% to filenames specified in the source due to different catcodes.
% The following code rescans |\jobname|, stores the result
% in |\childdocname| and saves a copy in |\childdocjob|:
%    \begin{macrocode}
\edef\childdocname{\scantokens\expandafter{\jobname\noexpand}}
\let\childdocjob\childdocname
%    \end{macrocode}

% \macro{\childdocdisable}
% The macro |\childdocdisable| prevents the main file
% from being processed more than once.
% At this stage, the main document command |\childdocmain|
% is assumed to be called once again where it should do nothing.
% Any subsequent call to it should prevent
% a secondary processing of the main document
% It overwrites the forwarding commands
% |\childdocof| and |\childdocforward|
% with empty macros to prevent further inclusions of the main document:
%    \begin{macrocode}
\newcommand{\childdocdisable}
{
  \renewcommand{\childdocmain}[1]{\renewcommand{\childdocmain}[1]{\endinput}}
  \renewcommand{\childdocof}[1]{}
  \renewcommand{\childdocby}[2][]{}
  \renewcommand{\childdocforward}[2][]{}
  \renewcommand{\childdocdisable}{}
}
%    \end{macrocode}

% \macro{\childdocmain}
% The macro |\childdocmain| is to be called at the top of the main file
% with nothing or the main filename (without extension) as argument.
% First, it breaks loops.
% If the argument is not empty and does not match |\childdocname|
% (which is set by the first inclusion of |childdoc.def|),
% |\ifchilddoc| is set to true, |\includeonly| is applied to the child file
% and |\jobname| is set to the main file
% (for proper handling of |.aux| files):
%    \begin{macrocode}
\newcommand{\childdocmain}[1]
{
  \childdocdisable\childdocmain{}
  \if?#1?\else
    \begingroup
      \def\childdoctmp{#1}
      \ifx\childdoctmp\childdocname
        \def\childdoctmp{}
      \else
        \def\childdoctmp
        {
          \childdoctrue
          \includeonly{\childdocname}
          \def\childdocjob{#1}
          \def\jobname{#1}
        }
      \fi
      \expandafter
    \endgroup
    \childdoctmp
  \fi
}
%    \end{macrocode}

% \macro{\childdocof}
% The command |\childdocof| redirects
% compilation to the main file |#1|.
%    \begin{macrocode}
\newcommand{\childdocof}[1]
{
  \childdocdisable
  \childdoctrue
  \includeonly{\childdocname}
  \def\jobname{#1}
  \def\childdocjob{#1}
  \input{#1}
}
%    \end{macrocode}

% \macro{\childdocby}
% The command |\childdocby| ....
%    \begin{macrocode}
\newcommand{\childdocby}[2][]
{
  \childdocdisable
  \childdoctrue
  \childdocmanualtrue
  \if?#1?\else
    \def\jobname{#2}
  \fi
  \def\childdocjob{#2}
  \input{#2}
  \endinput
}
%    \end{macrocode}

% \macro{\childdocforward}
% The command |\childdocforward| redirects
% compilation to the main file or
% (if the optional argument is given) a child file.
% Parameters are set as if the main file
% or a child file starting with |\childdocof| was compiled.
% Then compilation is handed over to the main file:
%    \begin{macrocode}
\newcommand{\childdocforward}[2][]
{
  \begingroup
    \if?#1?
      \def\childdoctmp
      {
        \def\childdocname{#2}
        \def\childdocjob{#2}
        \def\jobname{#2}
        \input{#2}
        \endinput
      }
    \else
      \def\childdoctmp
      {
        \childdocdisable
        \def\childdocname{#2}
        \childdoctrue
        \includeonly{#2}
        \def\childdocjob{#1}
        \def\jobname{#1}
        \input{#1}
        \endinput
      }
    \fi
    \expandafter
  \endgroup
  \childdoctmp
}
%    \end{macrocode}

% \macro{\childdocforwardprefix}
% The command |\childdocforwardprefix| redirects
% compilation to the main or a child file by means of a pattern.
% The prefix |#1| in the current filename is replaced by |#2|
% and the suffix of the current filename is kept
% (it is assumed that the filename does not contain the substring `|~~~|'
% which is used as a delimiter).
% Compilation is handed over to the new file by |\childdocforward|:
%    \begin{macrocode}
\newcommand{\childdocforwardprefix}[3][]
{
  \begingroup
    \def\childdocextract #2##1~~~{\def\childdoctmp{\childdocforward[#1]{#3##1}}}
    \expandafter\childdocextract\childdocname~~~
    \expandafter
  \endgroup
  \childdoctmp
}
%    \end{macrocode}

% \macro{\childdoc}
% The deprecated macro |\childdoc| is a legacy version of |\childdocmain|:
%    \begin{macrocode}
\newcommand{\childdoc}{\childdocmain}
%    \end{macrocode}

% \macro{\childdocredirect}
% The deprecated macro |\childdocredirect| is a legacy version
% of |\childdocforward| and |\childdocforwardprefix|:
%    \begin{macrocode}
\newcommand{\childdocredirect}[2][]
{
  \begingroup
    \if?#1?
      \def\childdoctmp{\childdocforward{#2}}
    \else
      \def\childdoctmp{\childdocforwardprefix{#1}{#2}}
    \fi
    \expandafter
  \endgroup
  \childdoctmp
}
%    \end{macrocode}

%\iffalse
%</package>
%\fi
%
\endinput
|\\
|\childdocby{|\textit{main}|}|\\
\end{tabular}
\end{center}
%
Both forms have slightly different effects as described above.
The main file is prepared as usual, see \secref{sec:include}.

%%%%%%%%%%%%%%%%%%%%%%%%%%%%%%%%%%%%%%%%%%%%%%%%%%%%%%%%%%%%%%%%%%%%%%%%%%%%%%%%
\subsection{Legacy Detection}
\label{sec:detection}

The directive |\childdocmain| in the main file can detect
whether the complete document or merely a child is to be compiled
even without using the directive |\childdocof|.
This method is deprecated because it is less robust
and there is no compelling reason to use it;
it is merely provided for backward compatibility
and it may be removed in future versions.

If the detection mechanism is to be used,
it is mandatory to correctly specify
the filename of the main file as the argument of |\childdocmain|:
%
\begin{center}
\begin{tabular}{l}
|% \iffalse
%
% childdoc.dtx Copyright (C) 2017-2018 Niklas Beisert
%
% This work may be distributed and/or modified under the
% conditions of the LaTeX Project Public License, either version 1.3
% of this license or (at your option) any later version.
% The latest version of this license is in
%   http://www.latex-project.org/lppl.txt
% and version 1.3 or later is part of all distributions of LaTeX
% version 2005/12/01 or later.
%
% This work has the LPPL maintenance status `maintained'.
%
% The Current Maintainer of this work is Niklas Beisert.
%
% This work consists of the files childdoc.dtx and childdoc.ins
% and the derived files childdoc.def and cdocsamp.tex with
% cdocsch1.tex, cdocsch2.tex, cdocsdrf.tex, cdocsfn1.tex, cdocsfn2.tex.
%
%<package>\ifdefined\childdocmain\endinput\fi
%<package>\ProvidesFile{childdoc.def}[2018/12/30 v2.0 child document driver]
%<samplemain>\ProvidesFile{cdocsamp.tex}[2018/12/30 v2.0 sample for childdoc]
%<*driver>
%\ProvidesFile{childdoc.drv}[2018/12/30 v2.0 childdoc reference manual file]
\PassOptionsToClass{10pt,a4paper}{article}
\documentclass{ltxdoc}

\usepackage[margin=35mm]{geometry}
\usepackage{hyperref}
\usepackage{hyperxmp}
\usepackage[usenames]{color}

\hypersetup{colorlinks=true}
\hypersetup{pdfstartview=FitH}
\hypersetup{pdfpagemode=UseNone}
\hypersetup{pdfsource={}}
\hypersetup{pdflang={en-UK}}
\hypersetup{pdfcopyright={Copyright 2017-2018 Niklas Beisert.
  This work may be distributed and/or modified under the
  conditions of the LaTeX Project Public License, either version 1.3
  of this license or (at your option) any later version.}}
\hypersetup{pdflicenseurl={http://www.latex-project.org/lppl.txt}}
\hypersetup{pdfcontactaddress={ETH Zurich, ITP, HIT K,
  Wolfgang-Pauli-Strasse 27}}
\hypersetup{pdfcontactpostcode={8093}}
\hypersetup{pdfcontactcity={Zurich}}
\hypersetup{pdfcontactcountry={Switzerland}}
\hypersetup{pdfcontactemail={nbeisert@itp.phys.ethz.ch}}
\hypersetup{pdfcontacturl={http://people.phys.ethz.ch/\xmptilde nbeisert/}}

\newcommand{\secref}[1]{\hyperref[#1]{section \ref*{#1}}}

\parskip1ex
\parindent0pt
\let\olditemize\itemize
\def\itemize{\olditemize\parskip0pt}

\begin{document}

\title{The \textsf{childdoc} Package}
\hypersetup{pdftitle={The childdoc Package}}
\author{Niklas Beisert\\[2ex]
  Institut f\"ur Theoretische Physik\\
  Eidgen\"ossische Technische Hochschule Z\"urich\\
  Wolfgang-Pauli-Strasse 27, 8093 Z\"urich, Switzerland\\[1ex]
  \href{mailto:nbeisert@itp.phys.ethz.ch}
  {\texttt{nbeisert@itp.phys.ethz.ch}}}
\hypersetup{pdfauthor={Niklas Beisert}}
\hypersetup{pdfsubject={Manual for the LaTeX2e Package childdoc}}
\date{30 December 2018, \textsf{v2.0}}
\maketitle

\begin{abstract}\noindent
\textsf{childdoc} is a \LaTeXe{} package
that enables the direct compilation
of document sections included by |\include|
to individual files.
\end{abstract}

\begingroup
\parskip0ex
\tableofcontents
\endgroup

%%%%%%%%%%%%%%%%%%%%%%%%%%%%%%%%%%%%%%%%%%%%%%%%%%%%%%%%%%%%%%%%%%%%%%%%%%%%%%%%
%%%%%%%%%%%%%%%%%%%%%%%%%%%%%%%%%%%%%%%%%%%%%%%%%%%%%%%%%%%%%%%%%%%%%%%%%%%%%%%%
\section{Introduction}

\LaTeX{} provides a mechanism to structure a large document (such as a book)
into a main file and several child files (containing the chapters)
using the |\include| command.
This mechanism is beneficial for documents
which span hundreds of pages in order to
make the source file(s) more manageable.
Moreover, compilation can be restricted to
selected child files by means of the |\includeonly| command.
The latter feature can be used to reduce the compilation time while editing
(this was significantly more useful in the earlier days of \LaTeX{})
or to generate a smaller document which is easier to navigate.
Another application of |\includeonly| is to generate
documents consisting of selected parts of the complete document.

However, there are a few drawbacks of the plain |\include| mechanism:
\begin{itemize}
\item
The child files cannot be compiled on their own,
they can only be compiled via the main file.
A naive editing environment
(such as a text editor with an option
to have the current file processed by \LaTeX)
may require one to switch to the main file before compiling;
attempting to compile the child file produces errors.
\item
The main file must be modified (each time)
to adjust the |\includeonly| command
to the present needs. This easily leaves the main file in a messy state.
\item
The generated document will always carry the filename
of the main document. This is inconvenient if
several child files are to be compiled and
to be kept for distribution.
\end{itemize}

The present package provides a simple interface
to make child files individually compilable by \LaTeX{}.
Compiling a child file then has the same effect as compiling
the main file with an |\includeonly| command
to select the appropriate child.
Moreover the generated document will carry the name of the child
rather than the main file.
This resolves all three above issues.

This feature is meant to make the editing of books,
thesis documents and lecture notes somewhat more convenient.
However, the package can also be used efficiently for
composing a series of documents (such as exercise sheets)
which are typically distributed individually.
It then assists the author in generating the individual documents
(potentially in different versions)
as well as a document containing the collected series.
Another application is in developing style files
or other kinds of included material
where compilation of the style file could redirect
to a sample or test file.

%%%%%%%%%%%%%%%%%%%%%%%%%%%%%%%%%%%%%%%%%%%%%%%%%%%%%%%%%%%%%%%%%%%%%%%%%%%%%%%%
%%%%%%%%%%%%%%%%%%%%%%%%%%%%%%%%%%%%%%%%%%%%%%%%%%%%%%%%%%%%%%%%%%%%%%%%%%%%%%%%
\section{Usage}

First of all, the package \textsf{childdoc} is \emph{not} a standard
\LaTeXe{} |.sty| style file! Therefore it needs to be invoked in
a non-standard way.

%%%%%%%%%%%%%%%%%%%%%%%%%%%%%%%%%%%%%%%%%%%%%%%%%%%%%%%%%%%%%%%%%%%%%%%%%%%%%%%%
\subsection{Included Files}
\label{sec:include}

%%%%%%%%%%%%%%%%%%%%%%%%%%%%%%%%%%%%%%%%
\DescribeMacro{\childdocmain}
To use the package, add the commands
\begin{center}
\begin{tabular}{l}
|\input{childdoc.def}|\\
|\childdocmain{}|\\
\end{tabular}
\end{center}
at the very top of the main \LaTeX{} file,
in particular \emph{before} the |\documentclass| statement!
The argument of |\childdocmain| should be left empty
(but it must be present).

%%%%%%%%%%%%%%%%%%%%%%%%%%%%%%%%%%%%%%%%
\DescribeMacro{\childdocof}
Furthermore, add the commands
\begin{center}
\begin{tabular}{l}
|\input{childdoc.def}|\\
|\childdocof{|\textit{main}|}|\\
\end{tabular}
\end{center}
at the top of every child file \textit{child}
which is included by |\include{|\textit{child}|}|
from within the main file
(or at least for those files to be compiled individually).
The argument \textit{main} must be the filename of the main file.

There are a couple of
considerations in setting up the main and child documents:

%%%%%%%%%%%%%%%%%%%%%%%%%%%%%%%%%%%%%%%%
\paragraph{Restrictions.}

Please note the following restrictions:
\begin{itemize}
\item
|\childdocmain| must be called with one argument \textit{main}
to ensure compatibility with earlier 